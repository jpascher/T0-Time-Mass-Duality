\documentclass{article}
\usepackage[utf8]{inputenc}
\usepackage[german]{babel}
\usepackage{hyperref}
\usepackage{enumitem}
\begin{document}
	
	\section*{Physik-Dokumente im Verzeichnis}
\vspace{1.5cm}
\textit{Diese Dokumente wurden am 25.03.2025 erstellt und befinden sich im Verzeichnis: /quantencomputer/1/2/pdf/Deutsch kurzgefasst}

\vspace{1cm}

\begin{itemize}[leftmargin=*]
	\item \textbf{Kurzgefasst - Komplementärer Dualismus in der Physik - Von Welle-Teilchen zum Zeit-Raum} (25.03.2025)\\
	Diese kurze Übersichtsarbeit stellt Parallelen zwischen dem bekannten Welle-Teilchen-Dualismus und dem neu entwickelten Zeit-Masse-Dualismus her und zeigt, wie beide Konzepte zu einem umfassenderen Verständnis der Natur beitragen können.
	\item \textbf{Zusammenfassung - Fundamentale Konstanten} (25.03.2025)\\
	Diese Arbeit fasst die wichtigsten Erkenntnisse zur Neuinterpretation fundamentaler Konstanten im Rahmen des Zeit-Masse-Dualismus zusammen und bietet einen übersichtlichen Einstieg in die Thematik.
\end{itemize}

\vspace{1cm}
\textit{Diese Dokumente wurden am 25.03.2025 erstellt und befinden sich im Verzeichnis: /quantencomputer/1/2/pdf/Deutsch}

\vspace{1.5 cm}	
	\begin{itemize}[leftmargin=*]
		\item \textbf{Fundamentale Konstanten und deren Herleitung aus natürlichen Einheiten} (25.03.2025, 12:19)\\
		Diese Arbeit zeigt, wie die fundamentalen Konstanten der Physik im Rahmen des Zeit-Masse-Dualismus neu interpretiert werden können und wie sie mit dem Konzept der intrinsischen Zeit zusammenhängen.
		
		\item \textbf{Zeit als emergente Eigenschaft in der Quantenmechanik} (25.03.2025, 12:21)\\
		Diese Arbeit untersucht, wie Zeit als emergente Eigenschaft aus fundamentaleren Konzepten hervorgehen kann und stellt Verbindungen zwischen dem Zeit-Masse-Dualismus und anderen Ansätzen zur Quantengravitation her.
				
		\item \textbf{Komplementäre Erweiterungen der Physik} (25.03.2025)\\
		Diese grundlegende Arbeit führt das $T_0$-Modell mit absoluter Zeit und variabler Masse ein und stellt die theoretischen Grundlagen des Zeit-Masse-Dualismus vor. Sie enthält die vollständige mathematische Herleitung der Transformation zwischen dem Standardmodell (Zeitdilatation) und dem komplementären Modell (Massenvariation).
		
		\item \textbf{Dynamische Masse von Photonen und ihre Implikationen für Nichtlokalität} (25.03.2025)\\
		Diese Arbeit untersucht die Anwendung des Zeit-Masse-Dualismus auf masselose Teilchen wie Photonen und entwickelt das Konzept der frequenzabhängigen intrinsischen Zeit für Photonen. Sie enthält Vorhersagen für Experimente zur Messung von Verzögerungen in Quantenkorrelationen.
		
				
		\item \textbf{Vereinfachte Beschreibung der vier fundamentalen Kräfte Lagrange-Dichte} (25.03.2025)\\
		Diese Arbeit präsentiert eine vereinfachte Lagrange-Dichte-Formulierung der vier fundamentalen Kräfte unter Berücksichtigung des Zeit-Masse-Dualismus und zeigt, wie sich die vereinheitlichte Beschreibung auf die Standardformulierungen reduzieren lässt.
		
		\item \textbf{Jenseits der Planck-Skala} (25.03.2025)\\
		Diese Arbeit untersucht die Konsequenzen des Zeit-Masse-Dualismus für Phänomene jenseits der Planck-Skala und zeigt, wie er potenzielle Probleme der Quantengravitation lösen könnte.


	\end{itemize}
	

	
	
	
\end{document}