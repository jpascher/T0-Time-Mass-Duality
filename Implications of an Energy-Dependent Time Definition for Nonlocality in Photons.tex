\documentclass[a4paper,12pt]{article}
\usepackage[utf8]{inputenc}
\usepackage[english]{babel} % For English language and hyphenation
\usepackage{amsmath, amssymb}

\begin{document}
	
	\title{Implications of an Energy-Dependent Time Definition for Nonlocality in Photons}
	\author{Johann Pascher}
	\date{March 24, 2025}
	\maketitle
	
	\tableofcontents % Table of contents
	\newpage % Optional: Starts the main content on a new page
	
	\section{Implications of an Energy-Dependent Time Definition for Nonlocality in Photons}
	This appendix examines the consequences of extending quantum mechanics with an energy-dependent time definition \( T = \frac{1}{E} \) for massless particles such as photons, particularly with respect to nonlocality. This extension was developed to improve upon the original models with absolute time (\( T_0 \)) and intrinsic time (\( T = \frac{\hbar}{m c^2} \)), which fail for photons with \( m = 0 \). Planck units (\( \hbar = c_0 = G = 1 \)) are used here to simplify the representation.
	
	\subsection{Original Model and Its Limitations}
	In the standard model of quantum mechanics, nonlocality in entangled photons is described as an "instantaneous" correlation, with the light cone structure, governed by the speed of light (\( c_0 = 1 \)), ensuring causality. Time is relativistically variable, and photons have no rest mass (\( m = 0 \)). The \( T_0 \)-model with absolute time postulates a constant time, but the mass variability (\( m = \gamma m_0 \)) remains undefined for photons since \( m_0 = 0 \). Similarly, the intrinsic time \( T = \frac{\hbar}{m c^2} \) leads to \( T \to \infty \) when \( m = 0 \), halting time evolution and conflicting with the dynamics of photons.
	
	\subsection{Extension of Quantum Mechanics}
	To overcome these limitations, the time definition for massless particles is extended to \( T = \frac{1}{E} \), where \( E = p \) for photons (since \( E = p c_0 \) and \( c_0 = 1 \) in Planck units). For massive particles, \( T = \frac{1}{m} \) remains, and a hybrid definition \( T = \frac{1}{\max(m, E)} \) unifies the treatment. This leads to a modified Schrödinger equation:
	\[
	i \frac{\partial \psi}{\partial (t/T)} = H \psi,
	\]
	with \( H = p \) for photons and \( H = -\frac{1}{2m} \nabla^2 + V \) for massive particles. For a photon with energy \( E = 2\pi \nu \) (where \( \nu \) is the frequency), this yields \( T = \frac{1}{2\pi \nu} \), corresponding to the period and enabling finite time evolution.
	
	\subsection{Implications for Nonlocality in Photons}
	The introduction of \( T = \frac{1}{E} \) has the following effects on nonlocality:
	
	\begin{enumerate}
		\item \textbf{Finite Time Evolution}: Photons acquire an energy-dependent time scale \( T = \frac{1}{E} \), rather than an infinite \( T \). This allows for dynamic evolution dependent on their frequency.
		
		\item \textbf{Delays in Entanglement}: For entangled photons with energies \( E_1 \) and \( E_2 \), the time scales are \( T_1 = \frac{1}{E_1} \) and \( T_2 = \frac{1}{E_2} \). Different energies (e.g., \( E_1 \neq E_2 \)) result in differing evolution rates, implying a delay \( |T_1 - T_2| = \left| \frac{1}{E_1} - \frac{1}{E_2} \right| \) in correlation. This contradicts the instantaneous correlation of the standard model.
		
		\item \textbf{Photon with Massive Particle}: In a hybrid system (e.g., photon and electron), the correlation depends on \( T_\text{photon} = \frac{1}{E} \) and \( T_e = \frac{1}{m_e} \). Since \( E \) is typically smaller than \( m_e \) (e.g., \( E = 1 \, \text{eV} \) vs. \( m_e \approx 5.11 \times 10^5 \, \text{eV} \)), \( T_\text{photon} \gg T_e \), implying a significant delay in the photon’s state.
		
		\item \textbf{Energy-Dependent Nonlocality}: The correlation becomes energy-dependent, making nonlocality appear as an emergent property of the energy-time relationship, analogous to the mass-time relationship in massive particles.
		
		\item \textbf{Causality}: The light cone structure remains preserved by \( c_0 = 1 \), but the time evolution within the cone is scaled for photons by \( T = \frac{1}{E} \), altering the dynamics of state changes.
	\end{enumerate}
	
	\subsection{Experimental Verification}
	These implications could be tested through Bell tests with photons of different frequencies to detect delays proportional to \( \frac{1}{E} \). Similarly, measurements in hybrid systems (photon-electron) could confirm the differing time scales \( T = \frac{1}{E} \) vs. \( T = \frac{1}{m} \).
	
	\subsection{Conclusion}
	The extension of quantum mechanics with \( T = \frac{1}{E} \) for photons consistently integrates massless particles into the model and shifts nonlocality from an instantaneous to an energy-dependent dynamic. This necessitates a reconsideration of correlations in entangled systems and provides new experimental approaches for validation.
	
\end{document}