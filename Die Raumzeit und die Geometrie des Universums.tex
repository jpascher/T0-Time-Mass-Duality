\documentclass[a4paper,12pt]{article}

% Pakete
\usepackage[utf8]{inputenc}
\usepackage[ngerman]{babel}
\usepackage{amsmath,amssymb}
\usepackage{graphicx}
\usepackage{hyperref}
\usepackage{csquotes}
\usepackage[margin=2.5cm]{geometry}
\usepackage{booktabs}
\usepackage{xcolor}
\usepackage{physics}

% Dokumentinformationen
\title{Die Raumzeit und die Geometrie des Universums}
\author{Johann Pascher}
\date{20. März 2025}

\begin{document}
	
	\maketitle
	
	\begin{abstract}
		Diese Dokumentation behandelt die grundlegenden Konzepte der Raumzeit und der Geometrie des Universums. Besonderes Augenmerk wird auf die Unterscheidung zwischen Raum und Raumzeit, verschiedene Krümmungskonzepte, die Expansion des Universums und die Friedmann-Gleichungen gelegt. Das Dokument richtet sich an Leser mit mittlerem Verständnisniveau und bietet sowohl intuitive Erklärungen als auch die relevanten mathematischen Formeln.
	\end{abstract}
	
	\tableofcontents
	
	\section{Grundlegendes Verständnis}
	
	\subsection{Raum und Zeit vs. Raumzeit}
	
	Ein entscheidender Punkt zum Verständnis des Universums: \textbf{Raumzeit ist etwas grundlegend anderes als Raum und Zeit getrennt betrachtet.}
	
	\begin{itemize}
		\item \textbf{Klassische Physik}: Raum (3D) und Zeit sind getrennte, unabhängige Konzepte
		\item \textbf{Relativitätstheorie}: Raum und Zeit bilden zusammen die 4-dimensionale Raumzeit
	\end{itemize}
	
	Diese Unterscheidung ist wichtig, weil:
	\begin{enumerate}
		\item Das Universum kann \textbf{nicht korrekt} als ``flach'' oder ``hyperbolisch'' beschrieben werden, wenn man nur den Raum betrachtet
		\item Die Geometrie des Universums muss immer im Kontext der 4-dimensionalen Raumzeit verstanden werden
	\end{enumerate}
	
	\section{Die Geometrie des Universums}
	
	\subsection{Krümmungskonzepte}
	
	Das Universum kann drei grundlegende Geometrien haben, beschrieben durch den Krümmungsparameter $k$:
	
	\begin{table}[h]
		\centering
		\begin{tabular}{lcll}
			\toprule
			\textbf{Krümmung} & \textbf{Wert} & \textbf{Geometrie} & \textbf{Eigenschaften} \\
			\midrule
			Positiv & $k = +1$ & Geschlossen (sphärisch) & Endlich, unbegrenzt \\
			Null & $k = 0$ & Flach (euklidisch) & Unendlich, Parallelen bleiben parallel \\
			Negativ & $k = -1$ & Offen (hyperbolisch) & Unendlich, Parallelen divergieren \\
			\bottomrule
		\end{tabular}
		\caption{Grundlegende Geometrien des Universums}
	\end{table}
	
	\subsection{Intrinsische Krümmung}
	
	Die Krümmung des Universums ist eine \textbf{intrinsische Eigenschaft}, die von ``innen'' messbar ist, ohne eine höherdimensionale Einbettung zu benötigen:
	
	\begin{itemize}
		\item \textbf{Flaches Universum}: Winkelsumme eines Dreiecks = 180°
		\item \textbf{Sphärisches Universum}: Winkelsumme eines Dreiecks > 180°
		\item \textbf{Hyperbolisches Universum}: Winkelsumme eines Dreiecks < 180°
	\end{itemize}
	
	\subsection{Die ``Form'' des Universums von außen betrachtet}
	
	Wenn ein endliches Universum ($k = +1$) theoretisch von ``außen'' betrachtet werden könnte (was physikalisch unmöglich ist), würde es möglicherweise eine kugelähnliche Form haben. Jedoch:
	
	\begin{enumerate}
		\item \textbf{Idealisierte vs. reale Form}: 
		\begin{itemize}
			\item Ähnlich wie die Erde idealisiert als perfekte Kugel betrachtet wird, aber tatsächlich Unebenheiten und Verformungen aufweist, wäre auch ein endliches Universum nicht perfekt sphärisch
			\item Lokale Massekonzentrationen, Gravitationswellen und andere Effekte würden die ``Form'' verzerren
		\end{itemize}
		
		\item \textbf{Wahrscheinlichkeit}:
		\begin{itemize}
			\item Aktuelle Beobachtungen deuten auf ein nahezu flaches Universum hin ($k \approx 0$)
			\item Die Vorstellung eines endlichen, kugelförmigen Universums ($k = +1$) ist heute weitaus unwahrscheinlicher
		\end{itemize}
		
		\item \textbf{Konzeptuelle Probleme}:
		\begin{itemize}
			\item Die Idee eines ``Außen'' setzt einen höherdimensionalen Raum voraus, in den unser Universum eingebettet wäre
			\item Es gibt keinen empirischen Zugang zu diesem hypothetischen ``Außen''
			\item Die Physik innerhalb unseres Universums wird von dessen intrinsischer Geometrie bestimmt, nicht von einer hypothetischen äußeren Form
		\end{itemize}
	\end{enumerate}
	
	\section{Die Expansion des Universums}
	
	\subsection{Grundkonzept der Expansion}
	
	Die Expansion des Universums bedeutet, dass sich der \textbf{Raum selbst} ausdehnt. Dies ist ein fundamentaler Unterschied zu gewöhnlicher Bewegung:
	
	\begin{itemize}
		\item \textbf{Gewöhnliche Bewegung}: Objekte bewegen sich durch einen festen Raum
		\item \textbf{Kosmische Expansion}: Der Raum selbst dehnt sich aus, wodurch sich Objekte voneinander entfernen
	\end{itemize}
	
	\subsection{Skalenfaktor a(t)}
	
	Der Skalenfaktor $a(t)$ beschreibt, wie stark das Universum zu einem bestimmten Zeitpunkt $t$ ausgedehnt ist:
	
	\begin{itemize}
		\item Bei $a = 1$ (heute, per Definition): Aktuelle Ausdehnung
		\item Bei $a = 0,5$: Das Universum war halb so groß wie heute
		\item Bei $a = 2$: Das Universum wird doppelt so groß sein wie heute
	\end{itemize}
	
	\subsection{Expansion und Geometrie}
	
	Die Expansion kann in allen drei Geometrien (flach, sphärisch, hyperbolisch) stattfinden:
	
	\begin{enumerate}
		\item \textbf{Flaches Universum ($k = 0$)}: 
		\begin{itemize}
			\item Wie ein unendliches Gummiblatt, das in alle Richtungen gedehnt wird
			\item Expansion kann ewig weitergehen
			\item Momentan am besten mit Beobachtungen vereinbar
		\end{itemize}
		
		\item \textbf{Sphärisches Universum ($k = +1$)}:
		\begin{itemize}
			\item Wie eine aufblasbare Kugel, deren Oberfläche sich ausdehnt
			\item Ohne dunkle Energie würde die Expansion irgendwann stoppen und umkehren (Big Crunch)
			\item Mit dunkler Energie kann die Expansion trotz positiver Krümmung ewig weitergehen
		\end{itemize}
		
		\item \textbf{Hyperbolisches Universum ($k = -1$)}:
		\begin{itemize}
			\item Wie eine Sattelfläche, die sich in alle Richtungen ausdehnt
			\item Expansion verläuft schneller als im flachen Fall
		\end{itemize}
	\end{enumerate}
	

	\subsection{Hubble-Gesetz und Rotverschiebung}
	
	Die Expansion führt zu beobachtbaren Effekten:
	
	\begin{itemize}
		\item \textbf{Hubble-Gesetz}: Die Geschwindigkeit, mit der sich Galaxien entfernen, ist proportional zu ihrer Entfernung:
		
		\begin{equation}
			v = H_0 \cdot d \label{eq:hubble}
		\end{equation}
		
		Wobei $H_0$ die Hubble-Konstante ist (ca. 70 km/s/Mpc, wobei Werte zwischen 67 und 73 km/s/Mpc je nach Messmethode variieren). Ein \textbf{Megaparsec (Mpc)} entspricht etwa 3,26 Millionen Lichtjahren. Ein \textbf{Lichtjahr} ist die Entfernung, die Licht in einem Jahr zurücklegt (ca. 9,46 Billionen km). Ein Parsec (pc) ist die Entfernung mit einer Parallaxe von einer Bogensekunde, gemessen von der Erdumlaufbahn. Somit ist 1 Mpc = 1.000.000 pc.
		
		\item \textbf{Rotverschiebung}: Licht von entfernten Galaxien wird rötlicher, weil:
		\begin{itemize}
			\item Lichtwellen durch die Expansion gedehnt werden
			\item Die Rotverschiebung $z$ hängt mit dem Skalenfaktor zusammen: $z = \frac{1}{a} - 1$
		\end{itemize}
	\end{itemize}
	
	\subsection{Wichtige Folgerungen}
	
	\begin{enumerate}
		\item \textbf{Kein Zentrum der Expansion}:
		\begin{itemize}
			\item Jeder Beobachter in jeder Galaxie sieht alle anderen Galaxien von sich wegbewegen
			\item Es gibt keinen bevorzugten Ort im Universum
		\end{itemize}
		
		\item \textbf{Überlichtschnelle Expansion}:
		\begin{itemize}
			\item Genügend weit entfernte Galaxien können sich scheinbar schneller als Licht von uns entfernen
			\item Dies verletzt \textbf{NICHT} die Relativitätstheorie, da es sich um eine Expansion des Raumes handelt, nicht um Bewegung durch den Raum
		\end{itemize}

	1. Kein Zentrum der Expansion:
	\begin{itemize}
		\item \textbf{Kein Ausgangspunkt}: Der Urknall fand nicht an einem bestimmten Ort statt, sondern überall gleichzeitig
		\item \textbf{Fundamentale Eigenschaft}: Es handelt sich nicht um eine Einschränkung unserer Messgenauigkeit oder unseres Wissens, sondern um eine grundlegende Eigenschaft des Universums
		\item \textbf{Analogie}: Im Gegensatz zu einer Explosion in einem vorhandenen Raum war der Urknall der Beginn des Raums selbst
		\item Jeder Beobachter in jeder Galaxie sieht alle anderen Galaxien von sich wegbewegen
		\item Jeder Punkt im Universum erscheint als Zentrum der Expansion, was beweist, dass es kein echtes Zentrum geben kann
		\item Es gibt keinen bevorzugten Ort im Universum
	\end{itemize}
	\textbf{Wichtig zu verstehen}: Der Urknall war kein Ereignis an einem bestimmten Ort in einem bereits existierenden Raum. Er markiert vielmehr den Beginn von Raum und Zeit selbst. Die Frage 'Wo hat der Urknall stattgefunden?' ist ähnlich irreführend wie die Frage 'Was liegt nördlich des Nordpols?' - sie basiert auf falschen Prämissen über die Natur von Raum und Zeit.	
		\item \textbf{Beobachtungshorizont}:
		\begin{itemize}
			\item Es gibt einen Horizont, jenseits dessen wir keine Galaxien mehr sehen können, weil ihr Licht uns nie erreichen wird
			\item Dieser Horizont liegt bei etwa 46 Milliarden Lichtjahren
		\end{itemize}
	\end{enumerate}
	
	\section{Die Friedmann-Gleichungen}
	
	Die Friedmann-Gleichungen beschreiben die Expansion des Universums und verknüpfen die Krümmung mit der Materiedichte und Energie.
	
	\subsection{Erste Friedmann-Gleichung}
	
	\begin{equation}
		H^2 = \left(\frac{\dot{a}}{a}\right)^2 = \frac{8\pi G}{3}\rho - \frac{kc^2}{a^2} + \frac{\Lambda c^2}{3}
	\end{equation}
	
	Wobei:
	\begin{itemize}
		\item $H$ = Hubble-Parameter (Expansionsrate)
		\item $a$ = Skalenfaktor (relative Größe des Universums)
		\item $\dot{a}$ = zeitliche Änderung des Skalenfaktors
		\item $G$ = Gravitationskonstante
		\item $\rho$ = Materiedichte
		\item $k$ = Krümmungsparameter
		\item $c$ = Lichtgeschwindigkeit
		\item $\Lambda$ = kosmologische Konstante
	\end{itemize}
	
	\subsection{Zweite Friedmann-Gleichung (Beschleunigungsgleichung)}
	
	\begin{equation}
		\frac{\ddot{a}}{a} = -\frac{4\pi G}{3}\left(\rho + \frac{3p}{c^2}\right) + \frac{\Lambda c^2}{3}
	\end{equation}
	
	Wobei:
	\begin{itemize}
		\item $\ddot{a}$ = Beschleunigung des Skalenfaktors
		\item $p$ = Druck
	\end{itemize}
	
	\section{Wichtige Klarstellungen}
	
	\subsection{Ein flaches Universum bedeutet NICHT:}
	
	\begin{itemize}
		\item \textbf{NICHT} die Form einer flachen Scheibe
		\item \textbf{NICHT} unbedingt endlich (kann unendlich sein)
		\item \textbf{KEIN} Widerspruch zur Expansion
	\end{itemize}
	
	\subsection{Ein expandierendes Universum verstehen}
	
	Man kann sich ein unendliches, flaches, expandierendes Universum wie ein unendlich großes Gummiblatt vorstellen, das in alle Richtungen gedehnt wird:
	
	\begin{itemize}
		\item \textbf{Kein Mittelpunkt}: Jeder Punkt entfernt sich von jedem anderen
		\item \textbf{Galaxien als Punkte}: Die Galaxien bewegen sich nicht durch den Raum, sondern der Raum selbst dehnt sich aus
		\item \textbf{Rotverschiebung}: Licht wird durch die Expansion "gedehnt" (größere Wellenlänge)
	\end{itemize}
	
\end{document}