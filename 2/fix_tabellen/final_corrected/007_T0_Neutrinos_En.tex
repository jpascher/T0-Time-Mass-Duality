% Filename: 007_T0_Neutrinos_En.tex
\documentclass[12pt,a4paper]{report}
% ==============================================================================
% T0 Theory: Shared ENGLISH Preamble – Optimized for eBook/Book
% Version: 2.0 – Final 2026 (LuaLaTeX only) – ENGLISH corrected
% Author: Johann Pascher
% Date: January 2026
% ==============================================================================
%
% IMPORTANT: Compile EXCLUSIVELY with LuaLaTeX!
% In TeXstudio: Options → Configure TeXstudio → Build → Default Compiler → LuaLaTeX
%
% Required Fonts (install once):
% - Inter: https://fonts.google.com/specimen/Inter
% - JetBrains Mono: https://www.jetbrains.com/lp/mono/
% - Libertinus Math: https://github.com/libertinus-fonts/libertinus
% ==============================================================================

% === CHAPTER 1: BASIC PACKAGES (must come FIRST) ===
\RequirePackage{fontspec}
\RequirePackage{unicode-math}
\usepackage{chngcntr}
\setcounter{secnumdepth}{1}  % Nur Sections nummerieren (nicht subsections)
\setcounter{tocdepth}{1}     % Nur Sections im TOC (nicht subsections)
\makeatletter
\@ifundefined{c@chapter}{}{\counterwithout{section}{chapter}}  % Falls Kapitel existieren
\makeatother
\counterwithout{subsection}{section}  % Löse Verknüpfung
% === CHAPTER 2: LANGUAGE (ENGLISH) ===
\usepackage[english]{babel}
\usepackage{microtype}                    % IMPORTANT for better hyphenation!

% Typography settings for better line breaking
\frenchspacing                     % Correct English spacing after punctuation
\emergencystretch=3em              % Allows more stretch for difficult lines
\tolerance=2500                    % Higher tolerance for line breaks
\hbadness=10000                    % Suppresses "underfull hbox" warnings
\hfuzz=2pt                         % Allows minimal overfull
\pretolerance=150                  % Better word breaking

% Prevent bad page breaks
\clubpenalty=10000           % No "orphans"
\widowpenalty=10000          % No "widows"
\displaywidowpenalty=10000   % Also with equations
\brokenpenalty=10000         % No broken words across pages

% Explicit hyphenation for long technical words
\hyphenation{Fun-da-men-tal Frac-tal-Ge-o-met-ric Field The-o-ry Meth-od-o-log-i-cal}
\hyphenation{Re-vi-sion-ism Quan-ti-za-tion U-ni-fi-ca-tion Ef-fec-tive}
\hyphenation{Re-nor-mal-iz-a-bil-i-ty Sin-gu-lar-i-ties Con-cil-i-a-tion}
\hyphenation{E-mer-gence Phe-nom-e-no-log-i-cal Doc-u-men-ta-tion A-nal-y-sis}
\hyphenation{Grav-i-ta-tion Quan-tum Me-chan-ics Dog-ma-tism Con-se-quent}
\hyphenation{Par-al-lel-ism Im-ple-men-ta-tion Per-tur-ba-tions}
\hyphenation{Geo-met-ric Ar-ti-fact In-com-pat-i-bil-i-ty Con-struc-tive}
\hyphenation{Frac-tal Di-men-sion-less In-ves-ti-ga-tion De-scrip-tion}
\hyphenation{In-ter-pre-ta-tion Phe-nom-e-no-log-i-cal Math-e-mat-i-cal}
\hyphenation{Phi-lo-soph-i-cal Le-git-i-ma-tion Ap-pli-ca-tion Der-i-va-tion}
\hyphenation{U-ni-fi-ca-tion As-sump-tion Con-cep-tion Ex-pec-ta-tion}
\hyphenation{Sym-me-try-ex-ten-sion O-ver-all-pic-ture Chal-lenge}
\hyphenation{In-ter-ac-tion Ma-te-ri-al Ap-proach Per-spec-tive Pro-ce-dure}

% === CHAPTER 3: FONTS (with proper ligatures) ===
\setmainfont{Inter}[
Scale=1.02,
UprightFont=*-Regular,
BoldFont=*-Bold,
ItalicFont=*-Italic,
BoldItalicFont=*-BoldItalic,
Ligatures=TeX,           % IMPORTANT for proper typography
Language=English         % Explicit language support
]
\setsansfont{Inter}[
Scale=MatchLowercase,
Ligatures=TeX,
Language=English
]
\setmonofont{JetBrains Mono}[
Scale=0.95,
Language=English
]

% Math Font (simple & stable) – MUST come AFTER language definition
% IMPORTANT: Libertinus Math for correct \underbrace display!
\setmathfont{Libertinus Math}[Scale=1.0]

% === CHAPTER 4: MATHEMATICS PACKAGES (in STRICT order!) ===
% IMPORTANT: mathtools must come BEFORE unicode-math for some commands!
\usepackage{mathtools}           % FIRST mathtools!

% Then the rest
\usepackage{amsmath, amsfonts, amsthm}

% SIUNITX MUST be loaded BEFORE physics!
\usepackage{siunitx}
\sisetup{
	locale=US,                    % ENGLISH settings for SI units!
	group-separator={,},          % Thousands separator comma
	output-decimal-marker={.},    % Decimal separator point
	per-mode=symbol,
	separate-uncertainty=true
}

% Custom SI units used in narrative and books
\DeclareSIUnit\gigalightyear{Gly}
\DeclareSIUnit\mev{MeV}

% physics – MUST be loaded AFTER siunitx and mathtools
\usepackage{physics}

% === CHAPTER 5: ADDITIONS from pdflatex best practices ===
\usepackage{colortbl}        % Colored tables (ESSENTIAL!)
\usepackage{placeins}        % Float control: \FloatBarrier
\usepackage{subcaption}      % Subfigures
\usepackage{xurl}            % Better URL line breaking
% Hyphenation for URLs in bibliography
\def\UrlBreaks{\do\/\do-}

% === CHAPTER 6: PAGE LAYOUT
% =============================================================================
% SECTION 2: Page Geometry – 6" × 9" Buchformat
% =============================================================================
\usepackage[paperwidth=6in, paperheight=9in,
top=0.9in,
bottom=1.1in,
inner=0.9in,            % Größerer Innenrand für Bindung
outer=0.6in,            % Kleinerer Außenrand → mehr Text pro Seite
bindingoffset=0.5in,    % Puffer für Bindung (Steg)
twoside]{geometry}
\setlength{\headheight}{15pt}
%\usepackage[paperwidth=8.25in, paperheight=11in,
%top=1.0in,
%bottom=1.0in,
%left=1.0in,
%right=1.0in,
%twoside=false
% === CHAPTER 7: GRAPHICS AND TABLES ===
\usepackage{graphicx}
\usepackage[table,xcdraw]{xcolor}
% T0 brand colors
\definecolor{gold}{RGB}{255,215,0}
\definecolor{blue}{rgb}{0,0,1}
\definecolor{boxgray}{RGB}{240,240,240}
\definecolor{deepblue}{RGB}{0,0,127}
\definecolor{deepgreen}{RGB}{0,127,0}
\definecolor{deepred}{RGB}{191,0,0}
\definecolor{t0blue}{RGB}{33,150,243}
\definecolor{t0green}{RGB}{76,175,80}
\definecolor{t0orange}{RGB}{255,152,0}
\definecolor{t0purple}{RGB}{156,39,176}
\definecolor{t0red}{RGB}{244,67,54}
\definecolor{t0yellow}{RGB}{255,204,0}
\usepackage{tikz}
\usetikzlibrary{arrows.meta,positioning,shapes.geometric,decorations.pathmorphing,patterns,shapes.arrows,intersections}
\usepackage{pgfplots}
\pgfplotsset{compat=1.18}
\usepackage{quantikz}
\usepackage[most]{tcolorbox}
\tcbuselibrary{breakable}

% === WICHTIG: Algorithm-Konflikt umgehen ===
% Option: algorithmic mit GROSSBUCHSTABEN
% Gemeinsame Box für Experimente
\newtcolorbox{experimentbox}[1][]{
	colback=green!5!white,
	colframe=t0green!80!black,
	fonttitle=\bfseries,
	title={{#1}},
	breakable
}

% Abstract-Fallback
\ifdefined\abstract\else
\newenvironment{abstract}{\section*{\abstractname}\itshape\small\par\bigskip}{\bigskip}
\fi

% === MAKROS SICHER NEU DEFINIEREN / ÜBERSCHREIBEN ===
% Definiere Makros OHNE doppelte Subskripte
\newcommand{\phipar}{\phi_{\mathrm{par}}}
%\newcommand{\xipar}{\xi_{\mathrm{par}}}
\newcommand{\Qphipar}{Q_{\phi_{\mathrm{par}}}}
\newcommand{\rphipar}{r_{\phi_{\mathrm{par}}}}
\newcommand{\logphipar}{\log_{\phi_{\mathrm{par}}}}
\newcommand{\CHSH}{\text{CHSH}}
\usepackage{booktabs}
\usepackage{array}
\usepackage{longtable}
\usepackage{float}
\usepackage{adjustbox}
\usepackage{rotating}
\usepackage{tabularx}
\usepackage{makecell}
\usepackage{multirow}

% === CHAPTER 8: DOCUMENT FORMATTING ===
\usepackage{fancyhdr}
\renewcommand{\headrulewidth}{0.4pt}
\renewcommand{\footrulewidth}{0.4pt}
\usepackage{tocloft}

\usepackage{enumitem}
\setlist[itemize]{leftmargin=*, topsep=2pt, partopsep=0pt, parsep=2pt, itemsep=2pt}
\setlist[enumerate]{leftmargin=*, topsep=2pt, partopsep=0pt, parsep=2pt, itemsep=2pt}
\usepackage{setspace}
\usepackage{ragged2e}
\usepackage{multicol}

% === CHAPTER 9: CODE AND ALGORITHMS ===
\usepackage{algorithm}
\usepackage{algorithmic}
\usepackage{listings}
\lstset{
	basicstyle=\ttfamily\footnotesize,
	breaklines=true,
	breakatwhitespace=true,
	columns=flexible,
	keepspaces=true,
	showstringspaces=false,
	frame=single,
	xleftmargin=0pt,
	xrightmargin=0pt,
	literate=              % For special characters in code listings
	{ä}{{\"a}}1 {ö}{{\"o}}1 {ü}{{\"u}}1 {ß}{{\ss}}1
	{Ä}{{\"A}}1 {Ö}{{\"O}}1 {Ü}{{\"U}}1
}
\usepackage{mdframed}

% === CHAPTER 10: ADDITIONAL PACKAGES ===
\usepackage{pdflscape}
\usepackage{braket}
\usepackage{cancel}
\usepackage{caption}
\captionsetup{format=plain, labelfont=bf, justification=centering}
\usepackage{csquotes}
\usepackage{gensymb}
\usepackage{textcomp}
\usepackage{textgreek}
\usepackage{upgreek}
\usepackage{url}
\usepackage{slashed}
\usepackage{bm}

% === CHAPTER 11: HYPERREF (must come SECOND TO LAST!) ===
\usepackage{hyperref}
\hypersetup{
	colorlinks=true,
	linkcolor=black,
	citecolor=black,
	urlcolor=black,
	breaklinks=true,           % IMPORTANT for special characters in URLs!
	bookmarksnumbered=true,
	unicode=true,
	pdfencoding=auto,
	pdflang=en,                % Set PDF language to English
	pdfsubject={T0 Theory - Fundamental Fractal-Geometric Field Theory}
}

% Fix for unicode-math symbols in PDF bookmarks
\pdfstringdefDisableCommands{%
	\def\xi{xi}%
	\def\alpha{alpha}%
	\def\beta{beta}%
	\def\gamma{gamma}%
	\def\delta{delta}%
	\def\Delta{Delta}%
	\def\epsilon{epsilon}%
	\def\varepsilon{epsilon}%
	\def\theta{theta}%
	\def\kappa{kappa}%
	\def\lambda{lambda}%
	\def\mu{mu}%
	\def\nu{nu}%
	\def\pi{pi}%
	\def\rho{rho}%
	\def\sigma{sigma}%
	\def\tau{tau}%
	\def\phi{phi}%
	\def\chi{chi}%
	\def\psi{psi}%
	\def\omega{omega}%
	\def\Omega{Omega}%
	\def\Lambda{Lambda}%
	\def\times{x}%
	\def\cdot{*}%
	\def\pm{+/-}%
	\def\approx{~}%
	\def\sim{~}%
	\def\equiv{=}%
	\def\ell{l}%
	\def\hbar{h}%
	\def\rightarrow{->}%
	\def\leftarrow{<-}%
	\def\Rightarrow{=>}%
	\def\Leftarrow{<=}%
	\def\propto{~}%
	\def\mitxi{xi}%
	\def\mitalpha{alpha}%
	\def\mitbeta{beta}%
	\def\mitgamma{gamma}%
	\def\mitdelta{delta}%
	\def\mitDelta{Delta}%
	\def\mitepsilon{epsilon}%
	\def\mitvarepsilon{epsilon}%
	\def\mittheta{theta}%
	\def\mitkappa{kappa}%
	\def\mitlambda{lambda}%
	\def\mitLambda{Lambda}%
	\def\mitmu{mu}%
	\def\mitnu{nu}%
	\def\mitpi{pi}%
	\def\mitrho{rho}%
	\def\mitsigma{sigma}%
	\def\mittau{tau}%
	\def\mitphi{phi}%
	\def\mitchi{chi}%
	\def\mitpsi{psi}%
	\def\mitomega{omega}%
	\def\mitOmega{Omega}%
}

% === CHAPTER 12: BOOKMARK (must come AFTER hyperref!) ===
\usepackage{bookmark}

% === CHAPTER 13: CLEVEREF (ENGLISH LABELS) ===
\usepackage[english]{cleveref}
\crefname{equation}{Equation}{Equations}
\crefname{figure}{Figure}{Figures}
\crefname{table}{Table}{Tables}
\crefname{section}{Section}{Sections}
\crefname{chapter}{Chapter}{Chapters}
\crefname{theorem}{Theorem}{Theorems}
\crefname{lemma}{Lemma}{Lemmas}
\crefname{definition}{Definition}{Definitions}
\crefname{example}{Example}{Examples}
\crefname{remark}{Remark}{Remarks}

% === CUSTOM ENVIRONMENTS ===
% Alternative interpretation environment
\newenvironment{alternative}{%
	\begin{mdframed}[linecolor=black!30,linewidth=1pt,roundcorner=4pt,backgroundcolor=black!5]%
	}{%
	\end{mdframed}%
}

% Photon/particle environment
\newenvironment{photon}{%
	\begin{mdframed}[linecolor=blue!30,linewidth=1pt,roundcorner=4pt,backgroundcolor=blue!5]%
	}{%
	\end{mdframed}%
}

% Koide formula box environment
\newenvironment{koidebox}{%
	\begin{mdframed}[linecolor=green!30,linewidth=1pt,roundcorner=4pt,backgroundcolor=green!5]%
	}{%
	\end{mdframed}%
}

% Erkenntnis/insight environment
\newenvironment{erkenntnis}{%
	\begin{mdframed}[linecolor=orange!30,linewidth=1pt,roundcorner=4pt,backgroundcolor=orange!5]%
	}{%
	\end{mdframed}%
}

% Beziehung/relationship environment
\newenvironment{beziehung}{%
	\begin{mdframed}[linecolor=purple!30,linewidth=1pt,roundcorner=4pt,backgroundcolor=purple!5]%
	}{%
	\end{mdframed}%
}

% Derivation environment
\newenvironment{derivation}{%
	\begin{mdframed}[linecolor=teal!30,linewidth=1pt,roundcorner=4pt,backgroundcolor=teal!5]%
	}{%
	\end{mdframed}%
}

% Abhandlung/treatise environment
\newenvironment{abhandlung}{%
	\begin{mdframed}[linecolor=brown!30,linewidth=1pt,roundcorner=4pt,backgroundcolor=brown!5]%
	}{%
	\end{mdframed}%
}

% Anwendung/application environment
\newenvironment{anwendung}{%
	\begin{mdframed}[linecolor=cyan!30,linewidth=1pt,roundcorner=4pt,backgroundcolor=cyan!5]%
	}{%
	\end{mdframed}%
}

% Additional common environments
\newenvironment{konsequenz}{%
	\begin{mdframed}[linecolor=red!30,linewidth=1pt,roundcorner=4pt,backgroundcolor=red!5]%
	}{%
	\end{mdframed}%
}

\newenvironment{schlussfolgerung}{%
	\begin{mdframed}[linecolor=gray!30,linewidth=1pt,roundcorner=4pt,backgroundcolor=gray!5]%
	}{%
	\end{mdframed}%
}

\newenvironment{result}{%
	\begin{mdframed}[linecolor=violet!30,linewidth=1pt,roundcorner=4pt,backgroundcolor=violet!5]%
	}{%
	\end{mdframed}%
}

% Formula environment
\newenvironment{formula}{%
	\begin{mdframed}[linecolor=yellow!30,linewidth=1pt,roundcorner=4pt,backgroundcolor=yellow!5]%
	}{%
	\end{mdframed}%
}

% Revolutionaer/revolutionary environment
\newenvironment{revolutionaer}{%
	\begin{mdframed}[linecolor=red!50,linewidth=2pt,roundcorner=4pt,backgroundcolor=red!10]%
	}{%
	\end{mdframed}%
}

% Formel environment (German version of formula)
\newenvironment{formel}{%
	\begin{mdframed}[linecolor=yellow!30,linewidth=1pt,roundcorner=4pt,backgroundcolor=yellow!5]%
	}{%
	\end{mdframed}%
}

% Prinzip/principle environment
\newenvironment{prinzip}{%
	\begin{mdframed}[linecolor=blue!50,linewidth=2pt,roundcorner=4pt,backgroundcolor=blue!10]%
	}{%
	\end{mdframed}%
}

% Experimentell/experimental environment
\newenvironment{experimentell}{%
	\begin{mdframed}[linecolor=magenta!30,linewidth=1pt,roundcorner=4pt,backgroundcolor=magenta!5]%
	}{%
	\end{mdframed}%
}

% Neutrino environment
\newenvironment{neutrino}{%
	\begin{mdframed}[linecolor=cyan!40,linewidth=1pt,roundcorner=4pt,backgroundcolor=cyan!8]%
	}{%
	\end{mdframed}%
}

% Additional missing environments
\newenvironment{schluessel}{%
	\begin{mdframed}[linecolor=yellow!50,linewidth=1pt,roundcorner=4pt,backgroundcolor=yellow!10]%
	}{%
	\end{mdframed}%
}

\newenvironment{summary}{%
	\begin{mdframed}[linecolor=gray!40,linewidth=1pt,roundcorner=4pt,backgroundcolor=gray!8]%
	}{%
	\end{mdframed}%
}

\newenvironment{category}{%
	\begin{mdframed}[linecolor=pink!40,linewidth=1pt,roundcorner=4pt,backgroundcolor=pink!8]%
	}{%
	\end{mdframed}%
}

\newenvironment{sibox}{%
	\begin{mdframed}[linecolor=lime!40,linewidth=1pt,roundcorner=4pt,backgroundcolor=lime!8]%
	}{%
	\end{mdframed}%
}

% More missing environments
\newenvironment{documentbox}{%
	\begin{mdframed}[linecolor=teal!40,linewidth=1pt,roundcorner=4pt,backgroundcolor=teal!8]%
	}{%
	\end{mdframed}%
}

\newenvironment{t0box}{%
	\begin{mdframed}[linecolor=violet!40,linewidth=1pt,roundcorner=4pt,backgroundcolor=violet!8]%
	}{%
	\end{mdframed}%
}

\newenvironment{wichtig}{%
	\begin{mdframed}[linecolor=red!50,linewidth=2pt,roundcorner=4pt,backgroundcolor=red!10]%
	\textbf{Important:} 
	}{%
	\end{mdframed}%
}

\newenvironment{smbox}{%
	\begin{mdframed}[linecolor=orange!40,linewidth=1pt,roundcorner=4pt,backgroundcolor=orange!8]%
	}{%
	\end{mdframed}%
}

\newenvironment{pvbox}{%
	\begin{mdframed}[linecolor=purple!40,linewidth=1pt,roundcorner=4pt,backgroundcolor=purple!8]%
	}{%
	\end{mdframed}%
}

\newenvironment{numerisch}{%
	\begin{mdframed}[linecolor=blue!40,linewidth=1pt,roundcorner=4pt,backgroundcolor=blue!8]%
	}{%
	\end{mdframed}%
}

% More missing environments
\newenvironment{relation}{%
	\begin{mdframed}[linecolor=green!40,linewidth=1pt,roundcorner=4pt,backgroundcolor=green!8]%
	}{%
	\end{mdframed}%
}

\newenvironment{beweis}{%
	\begin{mdframed}[linecolor=brown!40,linewidth=1pt,roundcorner=4pt,backgroundcolor=brown!8]%
	\textbf{Proof:} 
	}{%
	\end{mdframed}%
}

\newenvironment{revolution}{%
	\begin{mdframed}[linecolor=red!60,linewidth=2pt,roundcorner=4pt,backgroundcolor=red!12]%
	}{%
	\end{mdframed}%
}

\newenvironment{key}{%
	\begin{mdframed}[linecolor=yellow!50,linewidth=1pt,roundcorner=4pt,backgroundcolor=yellow!10]%
	}{%
	\end{mdframed}%
}

\newenvironment{newperspective}{%
	\begin{mdframed}[linecolor=cyan!50,linewidth=1pt,roundcorner=4pt,backgroundcolor=cyan!10]%
	}{%
	\end{mdframed}%
}

\newenvironment{literatur}{%
	\begin{mdframed}[linecolor=gray!50,linewidth=1pt,roundcorner=4pt,backgroundcolor=gray!10]%
	}{%
	\end{mdframed}%
}

\newenvironment{folgerung}{%
	\begin{mdframed}[linecolor=teal!50,linewidth=1pt,roundcorner=4pt,backgroundcolor=teal!10]%
	}{%
	\end{mdframed}%
}

\newenvironment{principle}{%
	\begin{mdframed}[linecolor=blue!60,linewidth=2pt,roundcorner=4pt,backgroundcolor=blue!12]%
	}{%
	\end{mdframed}%
}

% Additional common environments
% ==============================================================================
% FROM HERE: YOUR DEFINITIONS (unchanged)
% ==============================================================================

\setcounter{tocdepth}{3}

% === CITATION COMMANDS ===
\providecommand{\citep}[1]{\cite{#1}}
\providecommand{\citet}[1]{\cite{#1}}

% === COLORS ===
\definecolor{gold}{RGB}{255,215,0}
\definecolor{blue}{rgb}{0,0,1}
\definecolor{boxgray}{RGB}{240,240,240}
\definecolor{deepblue}{RGB}{0,0,127}
\definecolor{deepgreen}{RGB}{0,127,0}
\definecolor{deepred}{RGB}{191,0,0}
\definecolor{t0blue}{RGB}{33,150,243}
\definecolor{t0green}{RGB}{76,175,80}
\definecolor{t0orange}{RGB}{255,152,0}
\definecolor{t0purple}{RGB}{156,39,176}
\definecolor{t0red}{RGB}{244,67,54}
\definecolor{t0yellow}{RGB}{255,204,0}

% === COLUMN TYPES ===
\newcolumntype{L}[1]{>{\raggedright\arraybackslash}p{#1}}
\newcolumntype{C}[1]{>{\centering\arraybackslash}p{#1}}
\newcolumntype{R}[1]{>{\raggedleft\arraybackslash}p{#1}}

% === HYPERREF SETTINGS (updated) ===
\hypersetup{
	colorlinks=true,
	linkcolor=t0blue,
	citecolor=t0blue,
	urlcolor=t0blue,
	breaklinks=true,
	bookmarksnumbered=true,
	pdfstartview=FitH,
	pdfencoding=auto,
	pdfdisplaydoctitle=true
}

% === ENGLISH THEOREM ENVIRONMENTS ===
\theoremstyle{plain}
\newtheorem{theorem}{Theorem}[section]
\newtheorem{lemma}[theorem]{Lemma}
\newtheorem{proposition}[theorem]{Proposition}
\newtheorem{corollary}[theorem]{Corollary}

\theoremstyle{definition}
\newtheorem{definition}[theorem]{Definition}
\newtheorem{example}[theorem]{Example}
\newtheorem{insight}[theorem]{Insight}
\newtheorem{discovery}[theorem]{Discovery}

\theoremstyle{remark}
\newtheorem{remark}[theorem]{Remark}
\newtheorem{axiom}{Axiom}
%\newtheorem{principle}{Principle}  % Commented out to avoid conflicts with document-specific definitions
%\newtheorem{warning}[theorem]{Warning}

% === T0-SPECIFIC COMMANDS ===
% (Here follow all your \newcommand and \providecommand definitions)
% These remain UNCHANGED as in your original preamble
% ==============================================================================
% SECTION 14: T0-Specific Commands
% ==============================================================================

% --- Core T0 Fields ---
\newcommand{\Tfield}{T(x,t)}
\providecommand{\Tfieldt}{T(\vec{x},t)}
\newcommand{\Efield}{E(x,t)}
\newcommand{\mfield}{m(x,t)}
\providecommand{\vecx}{\vec{x}}

% --- Lagrangian ---
\newcommand{\Lag}{\mathcal{L}}
\newcommand{\calL}{\mathcal{L}}

% --- Greek Letters and Constants ---
\newcommand{\alphaem}{\alpha}
\newcommand{\betaT}{\beta_T}
\newcommand{\xiT}{\xi}
\newcommand{\xipar}{\xi}

% --- Energy and Planck Units ---
\newcommand{\Ezero}{E_0}
\newcommand{\E}{E}
\newcommand{\EPlanck}{E_{\text{Pl}}}
\newcommand{\Mpl}{M_{\text{Pl}}}
\newcommand{\mP}{m_{\text{P}}}
\newcommand{\lP}{\ell_{\text{P}}}
\newcommand{\tP}{t_{\text{P}}}
\newcommand{\LPlanck}{\ell_{\text{Pl}}}
\newcommand{\TPlanck}{t_{\text{Pl}}}

% --- Coupling Constants ---
\newcommand{\Gnat}{G_{\text{nat}}}
\newcommand{\alphaEM}{\alpha_{\text{EM}}}
\newcommand{\alphaSI}{\alpha_{\text{SI}}}
\newcommand{\Hubble}{H_0}
\newcommand{\LCDM}{\Lambda\text{CDM}}
\newcommand{\natunits}{(nat. units)}

% --- T0 Model Parameters ---
\newcommand{\xigeom}{\xi_{\mathrm{geom}}}
\newcommand{\rzero}{r_{0}}
\newcommand{\xirat}{\xi_{\mathrm{rat}}}
\newcommand{\tzero}{t_{0}}
\newcommand{\Lambdat}{\Lambda_{\mathrm{t}}}
\newcommand{\EP}{E_{\text{P}}}
\newcommand{\Emu}{E_{\mu}}
\newcommand{\Ee}{E_{e}}
\newcommand{\Etau}{E_{\tau}}
\newcommand{\alphafine}{\alpha_{\mathrm{fine}}}
\newcommand{\alphal}{\alpha_{\ell}}
\newcommand{\Lzero}{\ell_{0}}
\newcommand{\Lp}{\ell_{\mathrm{P}}}

% --- Additional T0 Commands ---
\newcommand{\Kfrak}{K_{\text{frak}}}
\newcommand{\Dfrak}{D_{\text{frak}}}
\newcommand{\betapar}{\ensuremath{\beta_T}}
\newcommand{\alphapar}{\alpha}
\newcommand{\deltafield}{\delta \phi}
\newcommand{\deltam}{\delta m}
\newcommand{\deltaE}{\delta E}
\newcommand{\Exi}{E_{\xi}}
\newcommand{\Lxi}{\ell_{\xi}}
\newcommand{\rhoCMB}{\rho_{\text{CMB}}}
\newcommand{\rhoCasimir}{\rho_{\text{Casimir}}}
\newcommand{\Leff}{L_{\text{eff}}}
\newcommand{\CQCD}{C_{\mathrm{QCD}}}
\newcommand{\Kspec}{K_{\mathrm{spec}}}
\newcommand{\Tzero}{\ensuremath{T_0}}
\newcommand{\Eabs}{E_{\text{abs}}}
\newcommand{\taupar}{\tau}

% --- Provided Commands ---
\providecommand{\xiconst}{\xi_{\text{const}}}
\providecommand{\DhiggsT}{D_{\text{Higgs-T}}}
\providecommand{\rhoE}{\rho_{E}}
\providecommand{\Echar}{E_{\text{char}}}
\providecommand{\kfrac}{k_{\text{frac}}}
\providecommand{\alphaEMSI}{\alpha_{\text{EM,SI}}}
\providecommand{\alphaEMnat}{\alpha_{\text{EM,nat}}}
\providecommand{\betaTSI}{\beta_{T,\text{SI}}}
\providecommand{\betaTnat}{\beta_{T,\text{nat}}}
\providecommand{\Gsi}{G_{\text{SI}}}
\providecommand{\xiparSI}{\xi_{\text{SI}}}
\providecommand{\xiparnat}{\xi_{\text{nat}}}
\providecommand{\meff}{m_{\text{eff}}}
\providecommand{\Tzerot}{T_{0}(t)}
\providecommand{\mzerot}{m_{0}(t)}
\providecommand{\Ezeroabs}{E_{0,\text{abs}}}
\providecommand{\Epar}{E_{\text{par}}}
\providecommand{\Lnat}{\ell_{\text{nat}}}
\providecommand{\Tnat}{T_{\text{nat}}}
\providecommand{\xifrak}{\xi_{\text{frac}}}
\providecommand{\Tfrak}{T_{\text{frac}}}
\providecommand{\mfrak}{m_{\text{frac}}}
\providecommand{\Dfrac}{D_{\text{frac}}}
\providecommand{\EphotSI}{E_{\gamma,\text{SI}}}
\providecommand{\EphotNat}{E_{\gamma,\text{nat}}}
\providecommand{\Eabsint}{E_{\text{abs,int}}}
\providecommand{\mphoton}{m_{\gamma}}
\providecommand{\Evis}{E_{\text{vis}}}
\providecommand{\Cto}{C_{T0}}
\providecommand{\mytimes}{\times}
\providecommand{\lambdah}{\lambda_h}
\providecommand{\checkmarkx}{\checkmark}
\providecommand{\Enorm}{E_{\text{norm}}}
\providecommand{\Tobs}{T_{\text{obs}}}
\providecommand{\mobs}{m_{\text{obs}}}
\providecommand{\Eobs}{E_{\text{obs}}}
\providecommand{\Lobs}{\ell_{\text{obs}}}
\providecommand{\xobs}{\xi_{\text{obs}}}
\providecommand{\calE}{\mathcal{E}}
\providecommand{\calT}{\mathcal{T}}
\providecommand{\calM}{\mathcal{M}}
\providecommand{\alphag}{\alpha_g}
\providecommand{\Tmax}{T_{\text{max}}}
\providecommand{\mmin}{m_{\text{min}}}
\providecommand{\Lmax}{\ell_{\text{max}}}
\providecommand{\Emin}{E_{\text{min}}}
\providecommand{\Geff}{G_{\text{eff}}}
\providecommand{\rhoeff}{\rho_{\text{eff}}}
\providecommand{\xieff}{\xi_{\text{eff}}}
\providecommand{\Teff}{T_{\text{eff}}}
\providecommand{\hPlanck}{h}
\providecommand{\kB}{k_B}
\providecommand{\muB}{\mu_B}
\providecommand{\lambdaC}{\lambda_C}
\providecommand{\omegaP}{\omega_P}
\providecommand{\rhoP}{\rho_P}
\providecommand{\Tref}{T_{\text{ref}}}
\providecommand{\Eref}{E_{\text{ref}}}
\providecommand{\mref}{m_{\text{ref}}}
\providecommand{\Lref}{\ell_{\text{ref}}}
\providecommand{\xikonst}{\xi_0}
\providecommand{\Phiphoton}{\Phi_{\gamma}}
\providecommand{\etavis}{\eta_{\text{vis}}}
\providecommand{\pichar}{\pi}
\providecommand{\primrel}{\mathcal{P}_{\text{rel}}}
\providecommand{\warningx}{\textcolor{orange}{\textbf{!}}}
\providecommand{\phiT}{\phi_T}
\providecommand{\Lorentz}{\Lambda}
\providecommand{\Cconv}{C_{\text{conv}}}
\providecommand{\Df}{\Delta f}
\providecommand{\lambdazero}{\lambda_0}
\providecommand{\myapprox}{\approx}
\providecommand{\checked}{\checkmark}
\providecommand{\alphaWSI}{\alpha_W^{\text{SI}}}
\providecommand{\alphaWnat}{\alpha_W^{\text{nat}}}
\providecommand{\vect}[1]{\vec{#1}}
\providecommand{\Rzero}{R_0}
\providecommand{\Riem}{\mathcal{R}}
\providecommand{\nuzero}{\nu_0}
\providecommand{\mypi}{\pi}

% =============================================================================
% TCOLORBOX STYLES AND ENVIRONMENTS (English titles)
% =============================================================================
\tcbset{
	keyresult/.style={
		colback=blue!5!white,
		colframe=blue!75!black,
		title=Key Result,
		fonttitle=\bfseries
	},
	foundation/.style={
		colback=green!5!white,
		colframe=green!75!black,
		title=Foundation,
		fonttitle=\bfseries
	},
	alternative/.style={
		colback=orange!5!white,
		colframe=orange!75!black,
		title=Alternative,
		fonttitle=\bfseries
	},
	warningbox/.style={
		colback=red!5!white,
		colframe=red!75!black,
		title=Warning,
		fonttitle=\bfseries
	}
}

% (Here follow all your tcolorbox definitions with English titles)
\newtcolorbox{keyresultbox}[1][]{colback=blue!5!white,colframe=blue!75!black,fonttitle=\bfseries,title={#1},breakable}
\newtcolorbox{keyresult}[1][Key Result]{colback=blue!5!white,colframe=blue!75!black,fonttitle=\bfseries,title={#1},breakable}
\newtcolorbox{foundationbox}[1][]{colback=green!5!white,colframe=green!75!black,fonttitle=\bfseries,title={#1},breakable}
\newtcolorbox{foundation}[1][Foundation]{colback=green!5!white,colframe=green!75!black,fonttitle=\bfseries,title={#1},breakable}
\newtcolorbox{alternativebox}[1][]{colback=orange!5!white,colframe=orange!75!black,fonttitle=\bfseries,title={#1},breakable}
\newtcolorbox{warningboxenv}[1][Warning]{colback=red!5!white,colframe=red!75!black,fonttitle=\bfseries,title={#1},breakable}

\newtcolorbox{fundamental}[1][]{
	colback=boxgray,
	colframe=t0blue,
	fonttitle=\bfseries,
	title=#1,
	sharp corners,
	boxrule=2pt
}

\newtcolorbox{insightBox}[1][Insight]{colback=blue!5,colframe=t0blue,title={#1},fonttitle=\bfseries,breakable}
\newtcolorbox{discoveryBox}[1][Discovery]{colback=green!5,colframe=t0green,title={#1},fonttitle=\bfseries,breakable}
\newtcolorbox{revelation}[1][Revelation]{colback=red!5,colframe=t0red,title={#1},fonttitle=\bfseries,breakable}
\newtcolorbox{keypoint}[1][Key Point]{colback=blue!5,colframe=t0blue,title={#1},fonttitle=\bfseries,breakable}
\newtcolorbox{evidence}[1][Evidence]{colback=green!5,colframe=t0green,title={#1},fonttitle=\bfseries,breakable}
\newtcolorbox{conclusionBox}[1][Conclusion]{colback=gray!5,colframe=gray,title={#1},fonttitle=\bfseries,breakable}
\newtcolorbox{significance}[1][Significance]{colback=yellow!5,colframe=orange,title={#1},fonttitle=\bfseries,breakable}
\newtcolorbox{philosophical}[1][Philosophical]{colback=purple!5,colframe=purple,title={#1},fonttitle=\bfseries,breakable}
\newtcolorbox{implicationBox}[1][Implication]{colback=cyan!5,colframe=cyan,title={#1},fonttitle=\bfseries,breakable}
\newtcolorbox{perspectiveBox}[1][Perspective]{colback=blue!5,colframe=t0blue,title={#1},fonttitle=\bfseries,breakable}
\newtcolorbox{revolutionary}[1][Revolutionary]{colback=red!5,colframe=t0red,title={#1},fonttitle=\bfseries,breakable}

\newtcolorbox{technical}[1][Technical]{colback=gray!5,colframe=gray!75!black,title={#1},fonttitle=\bfseries,breakable}
\newtcolorbox{technicalBox}[1][Technical]{colback=gray!5,colframe=gray!75!black,title={#1},fonttitle=\bfseries,breakable}
\newtcolorbox{notationBox}[1][Notation]{colback=yellow!5,colframe=yellow!75!black,title={#1},fonttitle=\bfseries,breakable}
\newtcolorbox{verification}[1][Verification]{colback=orange!5!white,colframe=orange!75!black,fonttitle=\bfseries,title=#1}
\newtcolorbox{explanationBox}[1][Explanation]{colback=purple!5!white,colframe=purple!75!black,fonttitle=\bfseries,title=#1}
\newtcolorbox{interpretationBox}[1][Interpretation]{colback=cyan!5!white,colframe=cyan!75!black,fonttitle=\bfseries,title=#1}
\newtcolorbox{explanation}[1][Explanation]{colback=purple!5!white,colframe=purple!75!black,fonttitle=\bfseries,title=#1,breakable}
\newtcolorbox{interpretation}[1][Interpretation]{colback=cyan!5!white,colframe=cyan!75!black,fonttitle=\bfseries,title=#1,breakable}
\newtcolorbox{proof_step}[1][Proof Step]{colback=gray!5!white,colframe=gray!75!black,fonttitle=\bfseries,title=#1,breakable}
\newtcolorbox{experimental}[1][Experimental]{colback=teal!5!white,colframe=teal!75!black,fonttitle=\bfseries,title=#1,breakable}

\newtcolorbox{important}[1][Important]{colback=red!5!white,colframe=red!75!black,title={#1},fonttitle=\bfseries,breakable}
\newtcolorbox{warning}[1][Warning]{colback=orange!5!white,colframe=orange!75!black,title={#1},fonttitle=\bfseries,breakable}
\newtcolorbox{caution}[1][Caution]{colback=yellow!5!white,colframe=yellow!75!black,title={#1},fonttitle=\bfseries,breakable}
\newtcolorbox{highlight}[1][Highlight]{colback=yellow!10!white,colframe=yellow!75!black,title={#1},fonttitle=\bfseries,breakable}
\newtcolorbox{critical}[1][Critical]{colback=red!10!white,colframe=red!75!black,title={#1},fonttitle=\bfseries,breakable}

\newtcolorbox{analysis}[1][Analysis]{colback=blue!5!white,colframe=blue!75!black,title={#1},fonttitle=\bfseries,breakable}
\newtcolorbox{application}[1][Application]{colback=green!5!white,colframe=green!75!black,title={#1},fonttitle=\bfseries,breakable}
\newtcolorbox{experiment}[1][Experiment]{colback=cyan!5!white,colframe=cyan!75!black,title={#1},fonttitle=\bfseries,breakable}
\newtcolorbox{historical}[1][Historical]{colback=brown!5!white,colframe=brown!75!black,title={#1},fonttitle=\bfseries,breakable}
\newtcolorbox{numerical}[1][Numerical]{colback=gray!5!white,colframe=gray!75!black,title={#1},fonttitle=\bfseries,breakable}
\newtcolorbox{overview}[1][Overview]{colback=blue!5!white,colframe=blue!75!black,title={#1},fonttitle=\bfseries,breakable}
\newtcolorbox{speculation}[1][Speculation]{colback=purple!5!white,colframe=purple!75!black,title={#1},fonttitle=\bfseries,breakable}
\newtcolorbox{question}[1][Question]{colback=orange!5!white,colframe=orange!75!black,title={#1},fonttitle=\bfseries,breakable}
\newtcolorbox{method}[1][Method]{colback=teal!5!white,colframe=teal!75!black,title={#1},fonttitle=\bfseries,breakable}
\newtcolorbox{correct}[1][Correct]{colback=green!10!white,colframe=green!75!black,title={#1},fonttitle=\bfseries,breakable}
\newtcolorbox{units}[1][Units]{colback=gray!5!white,colframe=gray!75!black,title={#1},fonttitle=\bfseries,breakable}
\newtcolorbox{achievement}[1][Achievement]{colback=gold!5!white,colframe=orange!75!black,title={#1},fonttitle=\bfseries,breakable}
\newtcolorbox{equivalence}[1][Equivalence]{colback=cyan!5!white,colframe=cyan!75!black,title={#1},fonttitle=\bfseries,breakable}
\newtcolorbox{dimensional}[1][Dimensional Analysis]{colback=purple!5!white,colframe=purple!75!black,title={#1},fonttitle=\bfseries,breakable}

% === ADDITIONAL SIMPLE ENVIRONMENTS ===
\newenvironment{treatise}{\begin{quote}}{\end{quote}}
\newenvironment{gemeinsam}{\begin{quote}}{\end{quote}}
\newenvironment{vergleich}{\begin{quote}}{\end{quote}}
\newenvironment{vorteil}{\begin{quote}}{\end{quote}}
\newenvironment{common}{\begin{quote}}{\end{quote}}
\newenvironment{comparison}{\begin{quote}}{\end{quote}}
\newenvironment{advantage}{\begin{quote}}{\end{quote}}
\newenvironment{quantum}{\begin{quote}}{\end{quote}}

% === LAYOUT SETTINGS ===
\raggedbottom
\usepackage{environ}
\let\oldtabular\tabular
\let\endoldtabular\endtabular

\newenvironment{scaledtable}[1][0.85]{%
	\begingroup\footnotesize\setlength{\LTleft}{0pt}\setlength{\LTright}{0pt}%
}{%
	\endgroup%
}

\newcommand{\widetable}[1]{\resizebox{\textwidth}{!}{#1}}

% === TABLE OF CONTENTS FORMATTING ===
\renewcommand{\cftsecfont}{\color{blue}}
\renewcommand{\cftsubsecfont}{\color{blue}}
\renewcommand{\cftsecpagefont}{\color{blue}}
\renewcommand{\cftsubsecpagefont}{\color{blue}}
\renewcommand{\cfttoctitlefont}{\huge\bfseries\color{blue}}

% === DEFAULT HEADER AND FOOTER ===
\pagestyle{fancy}
\fancyhf{}
\fancyhead[L]{\textsc{T0 Theory}}
\fancyhead[R]{\textsc{J. Pascher}}
\fancyfoot[C]{\thepage}

% ==============================================================================
% End of Shared Preamble for English
% ==============================================================================

\begin{document}
\hfuzz=200pt
\allowdisplaybreaks

\title{\textbf{FFGFT: Neutrinos}}
\maketitle

\begin{abstract}
		This document addresses the special position of neutrinos in the T0 Theory. In contrast to established particles (charged leptons, quarks, bosons), neutrinos require a fundamentally different treatment based on the photon analogy with double $\xi_0$-suppression. The neutrino mass is derived from the formula $m_\nu = \frac{\xi_0^2}{2} \times m_e = 4.54$ meV, and oscillations are explained by geometric phases based on $T_x \cdot m_x = 1$, where the quantum numbers $(n, \ell, j)$ determine the phase differences. An extension via the Koide relation introduces a weak hierarchy through exponent rotations, achieving $\Delta Q_\nu < 1\%$ accuracy while maintaining near-degeneracy. A plausible target value for the neutrino mass ($m_\nu = 15$ meV) is derived from empirical data (cosmological limits). The T0 Theory is based on speculative geometric harmonies without empirical basis and is highly likely to be incomplete or incorrect. Scientific integrity requires a clear separation between mathematical correctness and physical validity.
	\end{abstract}
	
	\tableofcontents
	
	\section{Preamble: Scientific Honesty}
	
	\begin{warning}
		\textbf{CRITICAL LIMITATION:} The following formulas for neutrino masses are \textbf{speculative extrapolations} based on the untested hypothesis that neutrinos follow geometric harmonies and all flavor states have equal masses. This hypothesis has \textbf{no empirical basis} and is highly likely to be incomplete or incorrect. The mathematical formulas are nevertheless internally consistent and correctly formulated.
		
		\vspace{0.5cm}
		\textbf{Scientific integrity means:}
		\begin{itemize}
			\item Honesty about the speculative nature of the predictions
			\item Mathematical correctness despite physical uncertainty
			\item Clear separation between hypotheses and verified facts
		\end{itemize}
	\end{warning}
	
	\section{Neutrinos as ``Almost Massless Photons'': The T0 Photon Analogy}
	
	\begin{speculation}
		\textbf{Fundamental T0 Insight:} Neutrinos can be understood as ``damped photons''.
		
		The remarkable similarity between photons and neutrinos suggests a deeper geometric kinship:
		\begin{itemize}
			\item \textbf{Speed:} Both propagate nearly at the speed of light
			\item \textbf{Penetration:} Both have extreme penetrability
			\item \textbf{Mass:} Photon exactly massless, neutrino quasi-massless
			\item \textbf{Interaction:} Photon electromagnetic, neutrino weak
		\end{itemize}
	\end{speculation}
	
	\subsection{Photon-Neutrino Correspondence}
	\label{subsec:photon-correspondence}
	
	\begin{photon}
		\textbf{Physical Parallels:}
		\begin{align}
			\text{Photon:} \quad &E^2 = (pc)^2 + 0 \quad \text{(perfectly massless)} \\
			\text{Neutrino:} \quad &E^2 = (pc)^2 + \left(\sqrt{\frac{\xipar^2}{2}} m c^2\right)^2 \quad \text{(quasi-massless)}
		\end{align}
		
		\textbf{Speed Comparison:}
		\begin{align}
			v_\gamma &= c \quad \text{(exact)} \\
			v_\nu &= c \times \left(1 - \frac{\xipar^2}{2}\right) \approx 0.9999999911 \times c
		\end{align}
		
		The speed difference is only $8.89 \times 10^{-9}$ -- practically immeasurable!
	\end{photon}
	
	\subsection{The Double $\xi_0$-Suppression}
	\label{subsec:double-suppression}
	
	\begin{keyresult}
		\textbf{Neutrino Mass through Double Geometric Damping:}
		
		If neutrinos are ``almost photons'', then two suppression factors arise:
		
		\begin{enumerate}
			\item \textbf{First $\xi_0$ Factor:} ``Almost massless'' (like photon, but not perfect)
			\item \textbf{Second $\xi_0$ Factor:} ``Weak interaction'' (geometric decoupling)
		\end{enumerate}
		
		\textbf{Resulting Formula:}
		\begin{equation}
			\boxed{m_\nu = \frac{\xi_0^2}{2} \times m_e = \frac{(\frac{4}{3} \times 10^{-4})^2}{2} \times 0.511 \text{ MeV}}
		\end{equation}
		
		\textbf{Numerical Evaluation:}
		\begin{equation}
			m_\nu = 8.889 \times 10^{-9} \times 0.511 \text{ MeV} = 4.54 \text{ meV}
		\end{equation}
	\end{keyresult}
	
	\subsection{Physical Justification of the Photon Analogy}
	\label{subsec:physical-justification}
	
	\begin{photon}
		\textbf{Why the Photon Analogy is Physically Sensible:}
		
		\textbf{1. Speed Comparison:}
		\begin{align}
			v_\gamma &= c \quad \text{(exact)} \\
			v_\nu &= c \times \left(1 - \frac{\xi_0^2}{2}\right) \approx 0.9999999911 \times c
		\end{align}
		The speed difference is only $8.89 \times 10^{-9}$ - practically immeasurable!
		
		\textbf{2. Interaction Strengths:}
		\begin{align}
			\sigma_\gamma &\sim \alpha_{EM} \approx \frac{1}{137} \\
			\sigma_\nu &\sim \frac{\xi_0^2}{2} \times G_F \approx 8.89 \times 10^{-9}
		\end{align}
		The ratio $\sigma_\nu/\sigma_\gamma \sim \frac{\xi_0^2}{2}$ confirms the geometric suppression!
		
		\textbf{3. Penetrability:}
		\begin{itemize}
			\item Photons: Electromagnetic shielding possible
			\item Neutrinos: Practically unshieldable
			\item Both: Extreme ranges in matter
		\end{itemize}
	\end{photon}
	
	\section{Neutrino Oscillations}
	
	\subsection{The Standard Model Problem}
	\label{subsec:sm-problem}
	
	\begin{warning}
		\textbf{Neutrino Oscillations:} Neutrinos can change their identity (flavor) during flight - a phenomenon known as neutrino oscillation. A neutrino produced as an electron neutrino ($\nu_e$) can later be measured as a muon neutrino ($\nu_\mu$) or tau neutrino ($\nu_\tau$) and vice versa.
		
		The oscillations depend on the mass squared differences $\Delta m^2_{ij} = m_i^2 - m_j^2$ and the mixing angles. Current experimental data (2025) provide:
		\begin{align}
			\Delta m^2_{21} &\approx 7.53 \times 10^{-5} \text{ eV}^2 \quad \text{[Solar]} \\
			\Delta m^2_{32} &\approx 2.44 \times 10^{-3} \text{ eV}^2 \quad \text{[Atmospheric]} \\
			m_\nu &> 0.06 \text{ eV} \quad \text{[At least one neutrino, 3}\sigma\text{]}
		\end{align}
		
		\textbf{Problem for T0:}
		The T0 Theory postulates equal masses for the flavor states ($\nu_e, \nu_\mu, \nu_\tau$), which implies $\Delta m^2_{ij} = 0$ and is incompatible with standard oscillations.
	\end{warning}
	
	\subsection{Geometric Phases as Oscillation Mechanism}
	\label{subsec:geometric-phases}
	
	\begin{speculation}
		\textbf{T0 Hypothesis: Geometric Phases for Oscillations}
		
		To reconcile the hypothesis of equal masses ($m_{\nu_e} = m_{\nu_\mu} = m_{\nu_\tau} = m_\nu$) with neutrino oscillations, it is speculated that oscillations in the T0 Theory are caused by geometric phases rather than mass differences. This is based on the T0 relation:
		\[
		T_x \cdot m_x = 1,
		\]
		where $m_x = m_\nu = 4.54$ meV is the neutrino mass and $T_x$ is a characteristic time or frequency:
		\[
		T_x = \frac{1}{m_\nu} = \frac{1}{4.54 \times 10^{-3} \text{ eV}} \approx 2.2026 \times 10^2 \text{ eV}^{-1} \approx 1.449 \times 10^{-13} \text{ s}.
		\]
		
		The geometric phase is determined by the T0 quantum numbers $(n, \ell, j)$:
		\[
		\phi_{\text{geo}, i} \propto f(n, \ell, j) \cdot \frac{L}{E} \cdot \frac{1}{T_x},
		\]
		where $f(n, \ell, j) = \frac{n^6}{\ell^3}$ (or 1 for $\ell = 0$) are the geometric factors:
		\begin{align}
			f_{\nu_e} &= 1, \\
			f_{\nu_\mu} &= 64, \\
			f_{\nu_\tau} &= 91.125.
		\end{align}
		
		\textbf{WARNING:} This approach is purely hypothetical and without empirical confirmation. It contradicts the established theory that oscillations are caused by $\Delta m^2_{ij} \neq 0$.
	\end{speculation}
	
	\subsection{Quantum Number Assignment for Neutrinos}
	\label{subsec:quantum-numbers}
	
	\begin{table}[h]
		\centering
		%\resizebox{\textwidth}{!}{
\begin{adjustbox}{max width=\textwidth}
\begin{tabular}{lcccc}
			\toprule
			\textbf{Neutrino Flavor} & \textbf{$n$} & \textbf{$\ell$} & \textbf{$j$} & \textbf{$f(n,\ell,j)$} \\
			\midrule
			$\nu_e$ & $1$ & $0$ & $1/2$ & $1$ \\
			$\nu_\mu$ & $2$ & $1$ & $1/2$ & $64$ \\
			$\nu_\tau$ & $3$ & $2$ & $1/2$ & $91.125$ \\
			\bottomrule
		\end{tabular}
\end{adjustbox}

\bigskip
\clearpage
%}
		\caption{Speculative T0 Quantum Numbers for Neutrino Flavors}
	\end{table}
	
	\section{Integration of the Koide Relation: A Weak Hierarchy}
	\label{sec:koide-integration}
	
	\begin{koidebox}
		\textbf{T0-Koide Extension for Neutrinos:}
		
		To address the oscillation conflict ($\Delta m^2_{ij} \neq 0$), the T0 Theory integrates the Koide relation as a natural generalization (Brannen 2005). This introduces a weak hierarchy via exponent rotations around $\xi_0$, preserving the photon analogy while enabling small mass differences.
		
		\textbf{Eigenvector Representation:}
		The charged lepton masses follow Koide via:
		\begin{equation}
			\begin{pmatrix}
				\sqrt{m_e} \\
				\sqrt{m_\mu} \\
				\sqrt{m_\tau}
			\end{pmatrix}
			= \mathbf{U} \cdot \begin{pmatrix}
				m_1 \\
				m_2 \\
				m_3
			\end{pmatrix},
		\end{equation}
		where $\mathbf{U}$ is the unitary flavor-mixing matrix (CKM/PMNS analog).
		
		\textbf{T0 Adaptation for Neutrinos:}
		Neutrino masses emerge as perturbed versions of the base $m_\nu = 4.54$ meV:
		\begin{equation}
			m_{\nu_i} \approx \xi_0^{p_i + \delta} \cdot v_\nu, \quad \delta \approx \xi_0^{1/3} \approx 0.051
		\end{equation}
		with exponents $p_i = (3/2, 1, 2/3)$ from charged leptons (rotated by $\delta$ for weak hierarchy). This yields a quasi-degenerate spectrum:
		\begin{align}
			m_{\nu_1} &\approx 4.20 \text{ meV (normal hierarchy)}, \\
			m_{\nu_2} &\approx 4.54 \text{ meV}, \\
			m_{\nu_3} &\approx 5.12 \text{ meV}, \\
			\Sigma m_\nu &\approx 13.86 \text{ meV}.
		\end{align}
		
		\textbf{Neutrino Koide Relation:}
		\begin{equation}
			Q_\nu = \frac{m_{\nu_1} + m_{\nu_2} + m_{\nu_3}}{\left( \sqrt{m_{\nu_1}} + \sqrt{m_{\nu_2}} + \sqrt{m_{\nu_3}} \right)^2} \approx 0.6667 = \frac{2}{3},
		\end{equation}
		with $\Delta Q_\nu < 1\%$ accuracy, directly linking to PMNS mixing.
		
		\textbf{Hybrid Oscillation Mechanism:}
		Geometric phases (from $f(n,\ell,j)$) dominate, augmented by small $\Delta m^2_{ij} \approx (0.1-0.2) \times 10^{-4}$ eV$^2$ from $\delta$. This reconciles T0 with data without full hierarchy.
		
		\textbf{WARNING:} Highly speculative; testable via future $\Sigma m_\nu$ measurements (e.g., Euclid 2026+).
	\end{koidebox}
	
	\section{Experimental Assessment}
	
	\subsection{Cosmological Limits}
	\label{subsec:cosmological-limits}
	
	\begin{experimental}
		\textbf{Cosmological Neutrino Mass Limits (as of 2025):}
		
		\textbf{1. Planck Satellite + CMB Data:}
		\begin{equation}
			\Sigma m_\nu < 0.07 \text{ eV} \quad \text{(95\% Confidence)}
		\end{equation}
		
		\textbf{2. T0 Prediction (with Koide Extension):}
		\begin{equation}
			\Sigma m_\nu = 13.86 \text{ meV}
		\end{equation}
		
		\textbf{3. Comparison:}
		\begin{equation}
			\frac{13.86 \text{ meV}}{70 \text{ meV}} = 0.198 \approx 19.8\%
		\end{equation}
		
		The T0 prediction is well below all cosmological limits!
	\end{experimental}
	
	\subsection{Direct Mass Determination}
	\label{subsec:direct-mass}
	
	\begin{experimental}
		\textbf{Experimental Neutrino Mass Determination:}
		
		\textbf{1. KATRIN Experiment (2022):}
		\begin{equation}
			m(\nu_e) < 0.8 \text{ eV} \quad \text{(90\% Confidence)}
		\end{equation}
		
		\textbf{2. T0 Prediction (with Koide):}
		\begin{equation}
			m(\nu_e) \approx 4.54 \text{ meV (effective)}
		\end{equation}
		
		\textbf{3. Comparison:}
		\begin{equation}
			\frac{4.54 \text{ meV}}{800 \text{ meV}} = 0.0057 \approx 0.57\%
		\end{equation}
		
		The T0 prediction is orders of magnitude below the direct mass limits.
	\end{experimental}
	
	\subsection{Target Value Estimation}
	\label{subsec:target-value}
	
	\begin{keyresult}
		\textbf{Plausible Target Value for Neutrino Masses:}
		
		From cosmological data and theoretical considerations, a plausible target value emerges:
		\begin{equation}
			m_\nu^{\text{Target}} \approx 15 \text{ meV (per flavor, quasi-degenerate)}
		\end{equation}
		
		\textbf{Comparison with T0 Prediction (incl. Koide):}
		\begin{equation}
			\frac{4.54 \text{ meV}}{15 \text{ meV}} = 0.303 \approx 30.3\%
		\end{equation}
		
		The T0 prediction is about a factor of 3 below the plausible target value, which is acceptable for a speculative theory. Koide extension narrows this to ~7\% via hierarchy.
	\end{keyresult}
	
	\section{Cosmological Implications}
	
	\subsection{Structure Formation and Big Bang Nucleosynthesis}
	\label{subsec:structure-formation}
	
	\begin{keyresult}
		\textbf{Cosmological Consequences of T0 Neutrino Masses:}
		
		\textbf{1. Big Bang Nucleosynthesis:}
		\begin{itemize}
			\item Relativistic neutrinos at $T \sim 1$ MeV: Standard BBN unchanged
			\item Contribution to radiation density: $N_{\text{eff}} = 3.046$ (Standard)
		\end{itemize}
		
		\textbf{2. Structure Formation:}
		\begin{itemize}
			\item Neutrinos with 4.5 meV become non-relativistic at $z \sim 100$
			\item Suppression of small-scale structure formation negligible
		\end{itemize}
		
		\textbf{3. Cosmic Neutrino Background (C$\nu$B):}
		\begin{itemize}
			\item Number density: $n_\nu = 336$ cm$^{-3}$ (unchanged)
			\item Energy density: $\rho_\nu \propto \Sigma m_\nu = 13.86$ meV (with Koide)
			\item Fraction of critical density: $\Omega_\nu h^2 \approx 1.55 \times 10^{-4}$
		\end{itemize}
		
		\textbf{4. Comparison with Dark Matter:}
		\begin{itemize}
			\item Neutrino contribution: $\Omega_\nu \approx 2.1 \times 10^{-4}$
			\item Dark matter: $\Omega_{DM} \approx 0.26$
			\item Ratio: $\Omega_\nu/\Omega_{DM} \approx 8.1 \times 10^{-4}$ (negligible)
		\end{itemize}
	\end{keyresult}
	
	\section{Summary and Critical Evaluation}
	
	\subsection{The Central T0 Neutrino Hypotheses}
	\label{subsec:central-hypotheses}
	
	\begin{keyresult}
		\textbf{Main Statements of the T0 Neutrino Theory:}
		
		\begin{enumerate}
			\item \textbf{Photon Analogy:} Neutrinos as ``damped photons'' with double $\xi_0$-suppression
			
			\item \textbf{Uniform Mass (Base):} All flavor states have $m_\nu \approx 4.54$ meV (quasi-degenerate)
			
			\item \textbf{Geometric Oscillations + Koide:} Phases + weak hierarchy ($\delta$) for $\Delta m^2_{ij}$
			
			\item \textbf{Speed Prediction:} $v_\nu = c(1 - \xi_0^2/2)$
			
			\item \textbf{Cosmological Consistency:} $\Sigma m_\nu \approx 13.86$ meV below all limits, $\Delta Q_\nu <1\%$
		\end{enumerate}
	\end{keyresult}
	
	\subsection{Scientific Assessment}
	\label{subsec:scientific-assessment}
	
	\begin{warning}
		\textbf{Honest Scientific Evaluation:}
		
		\textbf{Strengths of the T0 Neutrino Theory:}
		\begin{itemize}
			\item Unified framework with other T0 predictions (now incl. Koide/PMNS)
			\item Elegant photon analogy with clear physical intuition
			\item Parameter freedom: No empirical adjustment
			\item Cosmological consistency with all known limits
			\item Specific, testable predictions (e.g., $\Sigma m_\nu$, $Q_\nu$)
		\end{itemize}
		
		\textbf{Fundamental Weaknesses:}
		\begin{itemize}
			\item \textbf{Contradiction to Oscillation Data:} Minimal $\Delta m^2_{ij}$ vs. experimental evidence (hybrid helps, but unproven)
			\item \textbf{Ad hoc Oscillation Mechanism:} Geometric phases + $\delta$ not fully derived
			\item \textbf{Missing QFT Foundation:} No complete field theory
			\item \textbf{Experimentally Indistinguishable:} Similar to Standard Model
			\item \textbf{Highly Speculative Basis:} Photon analogy and Koide extension unproven
		\end{itemize}
		
		\textbf{Overall Evaluation: Interesting Hypothesis, but Highly Speculative and Unconfirmed}
	\end{warning}
	
	\subsection{Comparison with Established T0 Predictions}
	\label{subsec:comparison}
	
	\begin{table}[h]
		\centering
		\resizebox{\textwidth}{!}{
\begin{adjustbox}{max width=\textwidth}
\begin{tabular}{lcccc}
			\toprule
			\textbf{Area} & \textbf{T0 Prediction} & \textbf{Experiment} & \textbf{Deviation} & \textbf{Status} \\
			\midrule
			Fine Structure Constant & $\alpha^{-1} = 137.036$ & $137.036$ & $< 0.001\%$ & \checkmarkx Established \\
			Gravitational Constant & $G = 6.674 \times 10^{-11}$ & $6.674 \times 10^{-11}$ & $< 0.001\%$ & \checkmarkx Established \\
			Charged Leptons & $99.0\%$ Accuracy & Precisely Known & $\sim 1\%$ & \checkmarkx Established \\
			Quark Masses & $98.8\%$ Accuracy & Precisely Known & $\sim 2\%$ & \checkmarkx Established \\
			\midrule
			\textbf{Neutrino Masses (Koide Ext.)} & $m_{\nu_i} \approx 4-5$ meV & $< 100$ meV & Unknown ($\Delta Q_\nu <1\%$) & \warningx Speculative \\
			\textbf{Neutrino Oscillations} & Geometric Phases + $\delta$ & $\Delta m^2 \neq 0$ & Partially Compatible & \warningx Problematic \\
			\bottomrule
		\end{tabular}
\end{adjustbox}

\bigskip
\clearpage

\end{document}
}
		\caption{T0 Neutrinos in Comparison to Established T0 Successes (Updated with Koide)}
	\end{table}
	
	\section{Experimental Tests and Falsification}
	
	\subsection{Testable Predictions}
	\label{subsec:testable-predictions}
	
	\begin{experimental}
		\textbf{Specific Experimental Tests of the T0 Neutrino Theory:}
		
		\begin{enumerate}
			\item \textbf{Direct Mass Determination:}
			\begin{itemize}
				\item KATRIN: Sensitivity to $\sim 0.2$ eV (insufficient)
				\item Future Experiments: $\sim 0.01$ eV required
				\item T0 Prediction: $m_{\nu_i} \approx 4-5$ meV (factor 2 below limit)
			\end{itemize}
			
			\item \textbf{Cosmological Precision Measurements:}
			\begin{itemize}
				\item Euclid Satellite: Sensitivity $\sim 0.02$ eV
				\item T0 Prediction: $\Sigma m_\nu = 13.86$ meV (testable!)
			\end{itemize}
			
			\item \textbf{Koide-Specific Tests:}
			\begin{itemize}
				\item Measure $Q_\nu$ via oscillation data: Expect $\approx 2/3$ ($\Delta <1\%$)
				\item PMNS correlations: Hierarchy from $\delta$-rotation
			\end{itemize}
			
			\item \textbf{Speed Measurements:}
			\begin{itemize}
				\item Supernova Neutrinos: $\Delta v/c \sim 10^{-8}$ measurable
				\item T0 Prediction: $\Delta v/c = 8.89 \times 10^{-9}$ (marginal)
			\end{itemize}
			
			\item \textbf{Oscillation Physics:}
			\begin{itemize}
				\item Test for small $\Delta m^2_{ij}$ + phase effects (clearly falsifiable)
			\end{itemize}
		\end{enumerate}
	\end{experimental}
	
	\subsection{Falsification Criteria}
	\label{subsec:falsification}
	
	The T0 Neutrino Theory would be falsified by:
	\begin{enumerate}
		\item Direct measurement of $m_\nu > 0.1$ eV (or strong hierarchy $|m_3 - m_1| > 10$ meV)
		\item Cosmological evidence for $\Sigma m_\nu > 0.1$ eV
		\item Clear proof of $\Delta m^2_{ij} \gg 10^{-4}$ eV$^2$ without phases
		\item Measurement of speed differences $\Delta v/c > 10^{-8}$
		\item Deviation from $Q_\nu \approx 2/3$ in oscillation analyses
	\end{enumerate}
	
	\section{Limits and Open Questions}
	
	\subsection{Fundamental Theoretical Problems}
	\label{subsec:theoretical-problems}
	
	\begin{warning}
		\textbf{Unsolved Problems of the T0 Neutrino Theory:}
		
		\begin{enumerate}
			\item \textbf{Oscillation Mechanism:} Geometric phases + $\delta$ are ad hoc
			\item \textbf{Quantum Field Theory:} No complete QFT formulation
			\item \textbf{Experimental Distinguishability:} Difficult to separate from Standard Model
			\item \textbf{Theoretical Consistency:} Partial contradiction to oscillation theory
			\item \textbf{Predictive Power:} Enhanced by Koide, but still limited
		\end{enumerate}
	\end{warning}
	
	\subsection{Future Developments}
	\label{subsec:future-developments}
	
	\begin{enumerate}
		\item \textbf{QFT Foundation:} Complete quantum field theory for geometric phases + Koide
		\item \textbf{Experimental Precision:} Cosmological measurements with $\sim 0.01$ eV sensitivity
		\item \textbf{Oscillation Theory:} Rigorous derivation of hybrid effects
		\item \textbf{Unified Description:} Full T0 integration with PMNS
	\end{enumerate}
	
	\section{Methodological Reflection}
	
	\subsection{Scientific Integrity vs. Theoretical Speculation}
	\label{subsec:integrity-speculation}
	
	\begin{keyresult}
		\textbf{Central Methodological Insights:}
		
		The neutrino chapter of the T0 Theory illustrates the tension between:
		
		\begin{itemize}
			\item \textbf{Theoretical Completeness:} Desire for unified description (now incl. Koide)
			\item \textbf{Empirical Anchoring:} Necessity of experimental confirmation
			\item \textbf{Scientific Honesty:} Disclosure of speculative nature
			\item \textbf{Mathematical Consistency:} Internal self-consistency of formulas
		\end{itemize}
		
		\textbf{Key Insight:} Even speculative theories can be valuable if their limits are honestly communicated.
	\end{keyresult}
	
	\subsection{Significance for the T0 Series}
	\label{subsec:significance-series}
	
	The neutrino treatment shows both the strengths and limits of the FFGFT:
	
	\begin{itemize}
		\item \textbf{Strengths:} Unified framework, elegant analogies, testable predictions (enhanced by Koide)
		\item \textbf{Limits:} Speculative basis, lack of experimental confirmation
		\item \textbf{Scientific Value:} Demonstration of alternative thinking approaches
		\item \textbf{Methodological Importance:} Importance of honest uncertainty communication
	\end{itemize}
	
	\begin{center}
		\textit{and shows the speculative limits of the T0 Theory}\\
		\textbf{FFGFT: Time-Mass Duality Framework}\\
		
	\end{center}
	
	\begin{thebibliography}{99}
		\bibitem{Brannen2005}
		C. P. Brannen, ``Estimate of neutrino masses from Koide's relation'', \textit{arXiv:hep-ph/0505028} (2005).
		\url{https://arxiv.org/abs/hep-ph/0505028}
		
		\bibitem{Brannen2006}
		C. P. Brannen, ``Koide Mass Formula for Neutrinos'', \textit{arXiv:0702.0052} (2006).
		\url{http://brannenworks.com/MASSES.pdf}
		
		\bibitem{PhaseVectors2025}
		Anonymous, ``The Koide Relation and Lepton Mass Hierarchy from Phase Vectors'', \textit{rXiv:2507.0040} (2025).
		\url{https://rxiv.org/pdf/2507.0040v1.pdf}
		
		\bibitem{PDG2025}
		Particle Data Group, ``Review of Particle Physics'', \textit{Phys. Rev. D} \textbf{112} (2025) 030001.
		\url{https://pdg.lbl.gov/2025/}
	\end{thebibliography}

\end{document}
