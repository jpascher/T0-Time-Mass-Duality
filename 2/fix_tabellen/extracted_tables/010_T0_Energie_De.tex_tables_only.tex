\documentclass[12pt,a4paper]{book}
% ==============================================================================
% T0 Theory: Shared English Preamble
% Version: 1.0
% Author: Johann Pascher
% Date: 2025
% ==============================================================================
%
% This is the standardized shared preamble for all English T0 Theory documents.
% Place this file in your document's directory or use a path like:
%   % ==============================================================================
% T0 Theory: Shared ENGLISH Preamble – Optimized for eBook/Book
% Version: 2.0 – Final 2026 (LuaLaTeX only) – ENGLISH corrected
% Author: Johann Pascher
% Date: January 2026
% ==============================================================================
%
% IMPORTANT: Compile EXCLUSIVELY with LuaLaTeX!
% In TeXstudio: Options → Configure TeXstudio → Build → Default Compiler → LuaLaTeX
%
% Required Fonts (install once):
% - Inter: https://fonts.google.com/specimen/Inter
% - JetBrains Mono: https://www.jetbrains.com/lp/mono/
% - Libertinus Math: https://github.com/libertinus-fonts/libertinus
% ==============================================================================

% === CHAPTER 1: BASIC PACKAGES (must come FIRST) ===
\RequirePackage{fontspec}
\RequirePackage{unicode-math}
\usepackage{chngcntr}
\setcounter{secnumdepth}{1}  % Nur Sections nummerieren (nicht subsections)
\setcounter{tocdepth}{1}     % Nur Sections im TOC (nicht subsections)
\makeatletter
\@ifundefined{c@chapter}{}{\counterwithout{section}{chapter}}  % Falls Kapitel existieren
\makeatother
\counterwithout{subsection}{section}  % Löse Verknüpfung
% === CHAPTER 2: LANGUAGE (ENGLISH) ===
\usepackage[english]{babel}
\usepackage{microtype}                    % IMPORTANT for better hyphenation!

% Typography settings for better line breaking
\frenchspacing                     % Correct English spacing after punctuation
\emergencystretch=3em              % Allows more stretch for difficult lines
\tolerance=2500                    % Higher tolerance for line breaks
\hbadness=10000                    % Suppresses "underfull hbox" warnings
\hfuzz=2pt                         % Allows minimal overfull
\pretolerance=150                  % Better word breaking

% Prevent bad page breaks
\clubpenalty=10000           % No "orphans"
\widowpenalty=10000          % No "widows"
\displaywidowpenalty=10000   % Also with equations
\brokenpenalty=10000         % No broken words across pages

% Explicit hyphenation for long technical words
\hyphenation{Fun-da-men-tal Frac-tal-Ge-o-met-ric Field The-o-ry Meth-od-o-log-i-cal}
\hyphenation{Re-vi-sion-ism Quan-ti-za-tion U-ni-fi-ca-tion Ef-fec-tive}
\hyphenation{Re-nor-mal-iz-a-bil-i-ty Sin-gu-lar-i-ties Con-cil-i-a-tion}
\hyphenation{E-mer-gence Phe-nom-e-no-log-i-cal Doc-u-men-ta-tion A-nal-y-sis}
\hyphenation{Grav-i-ta-tion Quan-tum Me-chan-ics Dog-ma-tism Con-se-quent}
\hyphenation{Par-al-lel-ism Im-ple-men-ta-tion Per-tur-ba-tions}
\hyphenation{Geo-met-ric Ar-ti-fact In-com-pat-i-bil-i-ty Con-struc-tive}
\hyphenation{Frac-tal Di-men-sion-less In-ves-ti-ga-tion De-scrip-tion}
\hyphenation{In-ter-pre-ta-tion Phe-nom-e-no-log-i-cal Math-e-mat-i-cal}
\hyphenation{Phi-lo-soph-i-cal Le-git-i-ma-tion Ap-pli-ca-tion Der-i-va-tion}
\hyphenation{U-ni-fi-ca-tion As-sump-tion Con-cep-tion Ex-pec-ta-tion}
\hyphenation{Sym-me-try-ex-ten-sion O-ver-all-pic-ture Chal-lenge}
\hyphenation{In-ter-ac-tion Ma-te-ri-al Ap-proach Per-spec-tive Pro-ce-dure}

% === CHAPTER 3: FONTS (with proper ligatures) ===
\setmainfont{Inter}[
Scale=1.02,
UprightFont=*-Regular,
BoldFont=*-Bold,
ItalicFont=*-Italic,
BoldItalicFont=*-BoldItalic,
Ligatures=TeX,           % IMPORTANT for proper typography
Language=English         % Explicit language support
]
\setsansfont{Inter}[
Scale=MatchLowercase,
Ligatures=TeX,
Language=English
]
\setmonofont{JetBrains Mono}[
Scale=0.95,
Language=English
]

% Math Font (simple & stable) – MUST come AFTER language definition
% IMPORTANT: Libertinus Math for correct \underbrace display!
\setmathfont{Libertinus Math}[Scale=1.0]

% === CHAPTER 4: MATHEMATICS PACKAGES (in STRICT order!) ===
% IMPORTANT: mathtools must come BEFORE unicode-math for some commands!
\usepackage{mathtools}           % FIRST mathtools!

% Then the rest
\usepackage{amsmath, amsfonts, amsthm}

% SIUNITX MUST be loaded BEFORE physics!
\usepackage{siunitx}
\sisetup{
	locale=US,                    % ENGLISH settings for SI units!
	group-separator={,},          % Thousands separator comma
	output-decimal-marker={.},    % Decimal separator point
	per-mode=symbol,
	separate-uncertainty=true
}

% Custom SI units used in narrative and books
\DeclareSIUnit\gigalightyear{Gly}
\DeclareSIUnit\mev{MeV}

% physics – MUST be loaded AFTER siunitx and mathtools
\usepackage{physics}

% === CHAPTER 5: ADDITIONS from pdflatex best practices ===
\usepackage{colortbl}        % Colored tables (ESSENTIAL!)
\usepackage{placeins}        % Float control: \FloatBarrier
\usepackage{subcaption}      % Subfigures
\usepackage{xurl}            % Better URL line breaking
% Hyphenation for URLs in bibliography
\def\UrlBreaks{\do\/\do-}

% === CHAPTER 6: PAGE LAYOUT
% =============================================================================
% SECTION 2: Page Geometry – 6" × 9" Buchformat
% =============================================================================
\usepackage[paperwidth=6in, paperheight=9in,
top=0.9in,
bottom=1.1in,
inner=0.9in,            % Größerer Innenrand für Bindung
outer=0.6in,            % Kleinerer Außenrand → mehr Text pro Seite
bindingoffset=0.5in,    % Puffer für Bindung (Steg)
twoside]{geometry}
\setlength{\headheight}{15pt}
%\usepackage[paperwidth=8.25in, paperheight=11in,
%top=1.0in,
%bottom=1.0in,
%left=1.0in,
%right=1.0in,
%twoside=false
% === CHAPTER 7: GRAPHICS AND TABLES ===
\usepackage{graphicx}
\usepackage[table,xcdraw]{xcolor}
% T0 brand colors
\definecolor{gold}{RGB}{255,215,0}
\definecolor{blue}{rgb}{0,0,1}
\definecolor{boxgray}{RGB}{240,240,240}
\definecolor{deepblue}{RGB}{0,0,127}
\definecolor{deepgreen}{RGB}{0,127,0}
\definecolor{deepred}{RGB}{191,0,0}
\definecolor{t0blue}{RGB}{33,150,243}
\definecolor{t0green}{RGB}{76,175,80}
\definecolor{t0orange}{RGB}{255,152,0}
\definecolor{t0purple}{RGB}{156,39,176}
\definecolor{t0red}{RGB}{244,67,54}
\definecolor{t0yellow}{RGB}{255,204,0}
\usepackage{tikz}
\usetikzlibrary{arrows.meta,positioning,shapes.geometric,decorations.pathmorphing,patterns,shapes.arrows,intersections}
\usepackage{pgfplots}
\pgfplotsset{compat=1.18}
\usepackage{quantikz}
\usepackage[most]{tcolorbox}
\tcbuselibrary{breakable}

% === WICHTIG: Algorithm-Konflikt umgehen ===
% Option: algorithmic mit GROSSBUCHSTABEN
% Gemeinsame Box für Experimente
\newtcolorbox{experimentbox}[1][]{
	colback=green!5!white,
	colframe=t0green!80!black,
	fonttitle=\bfseries,
	title={{#1}},
	breakable
}

% Abstract-Fallback
\ifdefined\abstract\else
\newenvironment{abstract}{\section*{\abstractname}\itshape\small\par\bigskip}{\bigskip}
\fi

% === MAKROS SICHER NEU DEFINIEREN / ÜBERSCHREIBEN ===
% Definiere Makros OHNE doppelte Subskripte
\newcommand{\phipar}{\phi_{\mathrm{par}}}
%\newcommand{\xipar}{\xi_{\mathrm{par}}}
\newcommand{\Qphipar}{Q_{\phi_{\mathrm{par}}}}
\newcommand{\rphipar}{r_{\phi_{\mathrm{par}}}}
\newcommand{\logphipar}{\log_{\phi_{\mathrm{par}}}}
\newcommand{\CHSH}{\text{CHSH}}
\usepackage{booktabs}
\usepackage{array}
\usepackage{longtable}
\usepackage{float}
\usepackage{adjustbox}
\usepackage{rotating}
\usepackage{tabularx}
\usepackage{makecell}
\usepackage{multirow}

% === CHAPTER 8: DOCUMENT FORMATTING ===
\usepackage{fancyhdr}
\renewcommand{\headrulewidth}{0.4pt}
\renewcommand{\footrulewidth}{0.4pt}
\usepackage{tocloft}

\usepackage{enumitem}
\setlist[itemize]{leftmargin=*, topsep=2pt, partopsep=0pt, parsep=2pt, itemsep=2pt}
\setlist[enumerate]{leftmargin=*, topsep=2pt, partopsep=0pt, parsep=2pt, itemsep=2pt}
\usepackage{setspace}
\usepackage{ragged2e}
\usepackage{multicol}

% === CHAPTER 9: CODE AND ALGORITHMS ===
\usepackage{algorithm}
\usepackage{algorithmic}
\usepackage{listings}
\lstset{
	basicstyle=\ttfamily\footnotesize,
	breaklines=true,
	breakatwhitespace=true,
	columns=flexible,
	keepspaces=true,
	showstringspaces=false,
	frame=single,
	xleftmargin=0pt,
	xrightmargin=0pt,
	literate=              % For special characters in code listings
	{ä}{{\"a}}1 {ö}{{\"o}}1 {ü}{{\"u}}1 {ß}{{\ss}}1
	{Ä}{{\"A}}1 {Ö}{{\"O}}1 {Ü}{{\"U}}1
}
\usepackage{mdframed}

% === CHAPTER 10: ADDITIONAL PACKAGES ===
\usepackage{pdflscape}
\usepackage{braket}
\usepackage{cancel}
\usepackage{caption}
\captionsetup{format=plain, labelfont=bf, justification=centering}
\usepackage{csquotes}
\usepackage{gensymb}
\usepackage{textcomp}
\usepackage{textgreek}
\usepackage{upgreek}
\usepackage{url}
\usepackage{slashed}
\usepackage{bm}

% === CHAPTER 11: HYPERREF (must come SECOND TO LAST!) ===
\usepackage{hyperref}
\hypersetup{
	colorlinks=true,
	linkcolor=black,
	citecolor=black,
	urlcolor=black,
	breaklinks=true,           % IMPORTANT for special characters in URLs!
	bookmarksnumbered=true,
	unicode=true,
	pdfencoding=auto,
	pdflang=en,                % Set PDF language to English
	pdfsubject={T0 Theory - Fundamental Fractal-Geometric Field Theory}
}

% Fix for unicode-math symbols in PDF bookmarks
\pdfstringdefDisableCommands{%
	\def\xi{xi}%
	\def\alpha{alpha}%
	\def\beta{beta}%
	\def\gamma{gamma}%
	\def\delta{delta}%
	\def\Delta{Delta}%
	\def\epsilon{epsilon}%
	\def\varepsilon{epsilon}%
	\def\theta{theta}%
	\def\kappa{kappa}%
	\def\lambda{lambda}%
	\def\mu{mu}%
	\def\nu{nu}%
	\def\pi{pi}%
	\def\rho{rho}%
	\def\sigma{sigma}%
	\def\tau{tau}%
	\def\phi{phi}%
	\def\chi{chi}%
	\def\psi{psi}%
	\def\omega{omega}%
	\def\Omega{Omega}%
	\def\Lambda{Lambda}%
	\def\times{x}%
	\def\cdot{*}%
	\def\pm{+/-}%
	\def\approx{~}%
	\def\sim{~}%
	\def\equiv{=}%
	\def\ell{l}%
	\def\hbar{h}%
	\def\rightarrow{->}%
	\def\leftarrow{<-}%
	\def\Rightarrow{=>}%
	\def\Leftarrow{<=}%
	\def\propto{~}%
	\def\mitxi{xi}%
	\def\mitalpha{alpha}%
	\def\mitbeta{beta}%
	\def\mitgamma{gamma}%
	\def\mitdelta{delta}%
	\def\mitDelta{Delta}%
	\def\mitepsilon{epsilon}%
	\def\mitvarepsilon{epsilon}%
	\def\mittheta{theta}%
	\def\mitkappa{kappa}%
	\def\mitlambda{lambda}%
	\def\mitLambda{Lambda}%
	\def\mitmu{mu}%
	\def\mitnu{nu}%
	\def\mitpi{pi}%
	\def\mitrho{rho}%
	\def\mitsigma{sigma}%
	\def\mittau{tau}%
	\def\mitphi{phi}%
	\def\mitchi{chi}%
	\def\mitpsi{psi}%
	\def\mitomega{omega}%
	\def\mitOmega{Omega}%
}

% === CHAPTER 12: BOOKMARK (must come AFTER hyperref!) ===
\usepackage{bookmark}

% === CHAPTER 13: CLEVEREF (ENGLISH LABELS) ===
\usepackage[english]{cleveref}
\crefname{equation}{Equation}{Equations}
\crefname{figure}{Figure}{Figures}
\crefname{table}{Table}{Tables}
\crefname{section}{Section}{Sections}
\crefname{chapter}{Chapter}{Chapters}
\crefname{theorem}{Theorem}{Theorems}
\crefname{lemma}{Lemma}{Lemmas}
\crefname{definition}{Definition}{Definitions}
\crefname{example}{Example}{Examples}
\crefname{remark}{Remark}{Remarks}

% === CUSTOM ENVIRONMENTS ===
% Alternative interpretation environment
\newenvironment{alternative}{%
	\begin{mdframed}[linecolor=black!30,linewidth=1pt,roundcorner=4pt,backgroundcolor=black!5]%
	}{%
	\end{mdframed}%
}

% Photon/particle environment
\newenvironment{photon}{%
	\begin{mdframed}[linecolor=blue!30,linewidth=1pt,roundcorner=4pt,backgroundcolor=blue!5]%
	}{%
	\end{mdframed}%
}

% Koide formula box environment
\newenvironment{koidebox}{%
	\begin{mdframed}[linecolor=green!30,linewidth=1pt,roundcorner=4pt,backgroundcolor=green!5]%
	}{%
	\end{mdframed}%
}

% Erkenntnis/insight environment
\newenvironment{erkenntnis}{%
	\begin{mdframed}[linecolor=orange!30,linewidth=1pt,roundcorner=4pt,backgroundcolor=orange!5]%
	}{%
	\end{mdframed}%
}

% Beziehung/relationship environment
\newenvironment{beziehung}{%
	\begin{mdframed}[linecolor=purple!30,linewidth=1pt,roundcorner=4pt,backgroundcolor=purple!5]%
	}{%
	\end{mdframed}%
}

% Derivation environment
\newenvironment{derivation}{%
	\begin{mdframed}[linecolor=teal!30,linewidth=1pt,roundcorner=4pt,backgroundcolor=teal!5]%
	}{%
	\end{mdframed}%
}

% Abhandlung/treatise environment
\newenvironment{abhandlung}{%
	\begin{mdframed}[linecolor=brown!30,linewidth=1pt,roundcorner=4pt,backgroundcolor=brown!5]%
	}{%
	\end{mdframed}%
}

% Anwendung/application environment
\newenvironment{anwendung}{%
	\begin{mdframed}[linecolor=cyan!30,linewidth=1pt,roundcorner=4pt,backgroundcolor=cyan!5]%
	}{%
	\end{mdframed}%
}

% Additional common environments
\newenvironment{konsequenz}{%
	\begin{mdframed}[linecolor=red!30,linewidth=1pt,roundcorner=4pt,backgroundcolor=red!5]%
	}{%
	\end{mdframed}%
}

\newenvironment{schlussfolgerung}{%
	\begin{mdframed}[linecolor=gray!30,linewidth=1pt,roundcorner=4pt,backgroundcolor=gray!5]%
	}{%
	\end{mdframed}%
}

\newenvironment{result}{%
	\begin{mdframed}[linecolor=violet!30,linewidth=1pt,roundcorner=4pt,backgroundcolor=violet!5]%
	}{%
	\end{mdframed}%
}

% Formula environment
\newenvironment{formula}{%
	\begin{mdframed}[linecolor=yellow!30,linewidth=1pt,roundcorner=4pt,backgroundcolor=yellow!5]%
	}{%
	\end{mdframed}%
}

% Revolutionaer/revolutionary environment
\newenvironment{revolutionaer}{%
	\begin{mdframed}[linecolor=red!50,linewidth=2pt,roundcorner=4pt,backgroundcolor=red!10]%
	}{%
	\end{mdframed}%
}

% Formel environment (German version of formula)
\newenvironment{formel}{%
	\begin{mdframed}[linecolor=yellow!30,linewidth=1pt,roundcorner=4pt,backgroundcolor=yellow!5]%
	}{%
	\end{mdframed}%
}

% Prinzip/principle environment
\newenvironment{prinzip}{%
	\begin{mdframed}[linecolor=blue!50,linewidth=2pt,roundcorner=4pt,backgroundcolor=blue!10]%
	}{%
	\end{mdframed}%
}

% Experimentell/experimental environment
\newenvironment{experimentell}{%
	\begin{mdframed}[linecolor=magenta!30,linewidth=1pt,roundcorner=4pt,backgroundcolor=magenta!5]%
	}{%
	\end{mdframed}%
}

% Neutrino environment
\newenvironment{neutrino}{%
	\begin{mdframed}[linecolor=cyan!40,linewidth=1pt,roundcorner=4pt,backgroundcolor=cyan!8]%
	}{%
	\end{mdframed}%
}

% Additional missing environments
\newenvironment{schluessel}{%
	\begin{mdframed}[linecolor=yellow!50,linewidth=1pt,roundcorner=4pt,backgroundcolor=yellow!10]%
	}{%
	\end{mdframed}%
}

\newenvironment{summary}{%
	\begin{mdframed}[linecolor=gray!40,linewidth=1pt,roundcorner=4pt,backgroundcolor=gray!8]%
	}{%
	\end{mdframed}%
}

\newenvironment{category}{%
	\begin{mdframed}[linecolor=pink!40,linewidth=1pt,roundcorner=4pt,backgroundcolor=pink!8]%
	}{%
	\end{mdframed}%
}

\newenvironment{sibox}{%
	\begin{mdframed}[linecolor=lime!40,linewidth=1pt,roundcorner=4pt,backgroundcolor=lime!8]%
	}{%
	\end{mdframed}%
}

% More missing environments
\newenvironment{documentbox}{%
	\begin{mdframed}[linecolor=teal!40,linewidth=1pt,roundcorner=4pt,backgroundcolor=teal!8]%
	}{%
	\end{mdframed}%
}

\newenvironment{t0box}{%
	\begin{mdframed}[linecolor=violet!40,linewidth=1pt,roundcorner=4pt,backgroundcolor=violet!8]%
	}{%
	\end{mdframed}%
}

\newenvironment{wichtig}{%
	\begin{mdframed}[linecolor=red!50,linewidth=2pt,roundcorner=4pt,backgroundcolor=red!10]%
	\textbf{Important:} 
	}{%
	\end{mdframed}%
}

\newenvironment{smbox}{%
	\begin{mdframed}[linecolor=orange!40,linewidth=1pt,roundcorner=4pt,backgroundcolor=orange!8]%
	}{%
	\end{mdframed}%
}

\newenvironment{pvbox}{%
	\begin{mdframed}[linecolor=purple!40,linewidth=1pt,roundcorner=4pt,backgroundcolor=purple!8]%
	}{%
	\end{mdframed}%
}

\newenvironment{numerisch}{%
	\begin{mdframed}[linecolor=blue!40,linewidth=1pt,roundcorner=4pt,backgroundcolor=blue!8]%
	}{%
	\end{mdframed}%
}

% More missing environments
\newenvironment{relation}{%
	\begin{mdframed}[linecolor=green!40,linewidth=1pt,roundcorner=4pt,backgroundcolor=green!8]%
	}{%
	\end{mdframed}%
}

\newenvironment{beweis}{%
	\begin{mdframed}[linecolor=brown!40,linewidth=1pt,roundcorner=4pt,backgroundcolor=brown!8]%
	\textbf{Proof:} 
	}{%
	\end{mdframed}%
}

\newenvironment{revolution}{%
	\begin{mdframed}[linecolor=red!60,linewidth=2pt,roundcorner=4pt,backgroundcolor=red!12]%
	}{%
	\end{mdframed}%
}

\newenvironment{key}{%
	\begin{mdframed}[linecolor=yellow!50,linewidth=1pt,roundcorner=4pt,backgroundcolor=yellow!10]%
	}{%
	\end{mdframed}%
}

\newenvironment{newperspective}{%
	\begin{mdframed}[linecolor=cyan!50,linewidth=1pt,roundcorner=4pt,backgroundcolor=cyan!10]%
	}{%
	\end{mdframed}%
}

\newenvironment{literatur}{%
	\begin{mdframed}[linecolor=gray!50,linewidth=1pt,roundcorner=4pt,backgroundcolor=gray!10]%
	}{%
	\end{mdframed}%
}

\newenvironment{folgerung}{%
	\begin{mdframed}[linecolor=teal!50,linewidth=1pt,roundcorner=4pt,backgroundcolor=teal!10]%
	}{%
	\end{mdframed}%
}

\newenvironment{principle}{%
	\begin{mdframed}[linecolor=blue!60,linewidth=2pt,roundcorner=4pt,backgroundcolor=blue!12]%
	}{%
	\end{mdframed}%
}

% Additional common environments
% ==============================================================================
% FROM HERE: YOUR DEFINITIONS (unchanged)
% ==============================================================================

\setcounter{tocdepth}{3}

% === CITATION COMMANDS ===
\providecommand{\citep}[1]{\cite{#1}}
\providecommand{\citet}[1]{\cite{#1}}

% === COLORS ===
\definecolor{gold}{RGB}{255,215,0}
\definecolor{blue}{rgb}{0,0,1}
\definecolor{boxgray}{RGB}{240,240,240}
\definecolor{deepblue}{RGB}{0,0,127}
\definecolor{deepgreen}{RGB}{0,127,0}
\definecolor{deepred}{RGB}{191,0,0}
\definecolor{t0blue}{RGB}{33,150,243}
\definecolor{t0green}{RGB}{76,175,80}
\definecolor{t0orange}{RGB}{255,152,0}
\definecolor{t0purple}{RGB}{156,39,176}
\definecolor{t0red}{RGB}{244,67,54}
\definecolor{t0yellow}{RGB}{255,204,0}

% === COLUMN TYPES ===
\newcolumntype{L}[1]{>{\raggedright\arraybackslash}p{#1}}
\newcolumntype{C}[1]{>{\centering\arraybackslash}p{#1}}
\newcolumntype{R}[1]{>{\raggedleft\arraybackslash}p{#1}}

% === HYPERREF SETTINGS (updated) ===
\hypersetup{
	colorlinks=true,
	linkcolor=t0blue,
	citecolor=t0blue,
	urlcolor=t0blue,
	breaklinks=true,
	bookmarksnumbered=true,
	pdfstartview=FitH,
	pdfencoding=auto,
	pdfdisplaydoctitle=true
}

% === ENGLISH THEOREM ENVIRONMENTS ===
\theoremstyle{plain}
\newtheorem{theorem}{Theorem}[section]
\newtheorem{lemma}[theorem]{Lemma}
\newtheorem{proposition}[theorem]{Proposition}
\newtheorem{corollary}[theorem]{Corollary}

\theoremstyle{definition}
\newtheorem{definition}[theorem]{Definition}
\newtheorem{example}[theorem]{Example}
\newtheorem{insight}[theorem]{Insight}
\newtheorem{discovery}[theorem]{Discovery}

\theoremstyle{remark}
\newtheorem{remark}[theorem]{Remark}
\newtheorem{axiom}{Axiom}
%\newtheorem{principle}{Principle}  % Commented out to avoid conflicts with document-specific definitions
%\newtheorem{warning}[theorem]{Warning}

% === T0-SPECIFIC COMMANDS ===
% (Here follow all your \newcommand and \providecommand definitions)
% These remain UNCHANGED as in your original preamble
% ==============================================================================
% SECTION 14: T0-Specific Commands
% ==============================================================================

% --- Core T0 Fields ---
\newcommand{\Tfield}{T(x,t)}
\providecommand{\Tfieldt}{T(\vec{x},t)}
\newcommand{\Efield}{E(x,t)}
\newcommand{\mfield}{m(x,t)}
\providecommand{\vecx}{\vec{x}}

% --- Lagrangian ---
\newcommand{\Lag}{\mathcal{L}}
\newcommand{\calL}{\mathcal{L}}

% --- Greek Letters and Constants ---
\newcommand{\alphaem}{\alpha}
\newcommand{\betaT}{\beta_T}
\newcommand{\xiT}{\xi}
\newcommand{\xipar}{\xi}

% --- Energy and Planck Units ---
\newcommand{\Ezero}{E_0}
\newcommand{\E}{E}
\newcommand{\EPlanck}{E_{\text{Pl}}}
\newcommand{\Mpl}{M_{\text{Pl}}}
\newcommand{\mP}{m_{\text{P}}}
\newcommand{\lP}{\ell_{\text{P}}}
\newcommand{\tP}{t_{\text{P}}}
\newcommand{\LPlanck}{\ell_{\text{Pl}}}
\newcommand{\TPlanck}{t_{\text{Pl}}}

% --- Coupling Constants ---
\newcommand{\Gnat}{G_{\text{nat}}}
\newcommand{\alphaEM}{\alpha_{\text{EM}}}
\newcommand{\alphaSI}{\alpha_{\text{SI}}}
\newcommand{\Hubble}{H_0}
\newcommand{\LCDM}{\Lambda\text{CDM}}
\newcommand{\natunits}{(nat. units)}

% --- T0 Model Parameters ---
\newcommand{\xigeom}{\xi_{\mathrm{geom}}}
\newcommand{\rzero}{r_{0}}
\newcommand{\xirat}{\xi_{\mathrm{rat}}}
\newcommand{\tzero}{t_{0}}
\newcommand{\Lambdat}{\Lambda_{\mathrm{t}}}
\newcommand{\EP}{E_{\text{P}}}
\newcommand{\Emu}{E_{\mu}}
\newcommand{\Ee}{E_{e}}
\newcommand{\Etau}{E_{\tau}}
\newcommand{\alphafine}{\alpha_{\mathrm{fine}}}
\newcommand{\alphal}{\alpha_{\ell}}
\newcommand{\Lzero}{\ell_{0}}
\newcommand{\Lp}{\ell_{\mathrm{P}}}

% --- Additional T0 Commands ---
\newcommand{\Kfrak}{K_{\text{frak}}}
\newcommand{\Dfrak}{D_{\text{frak}}}
\newcommand{\betapar}{\ensuremath{\beta_T}}
\newcommand{\alphapar}{\alpha}
\newcommand{\deltafield}{\delta \phi}
\newcommand{\deltam}{\delta m}
\newcommand{\deltaE}{\delta E}
\newcommand{\Exi}{E_{\xi}}
\newcommand{\Lxi}{\ell_{\xi}}
\newcommand{\rhoCMB}{\rho_{\text{CMB}}}
\newcommand{\rhoCasimir}{\rho_{\text{Casimir}}}
\newcommand{\Leff}{L_{\text{eff}}}
\newcommand{\CQCD}{C_{\mathrm{QCD}}}
\newcommand{\Kspec}{K_{\mathrm{spec}}}
\newcommand{\Tzero}{\ensuremath{T_0}}
\newcommand{\Eabs}{E_{\text{abs}}}
\newcommand{\taupar}{\tau}

% --- Provided Commands ---
\providecommand{\xiconst}{\xi_{\text{const}}}
\providecommand{\DhiggsT}{D_{\text{Higgs-T}}}
\providecommand{\rhoE}{\rho_{E}}
\providecommand{\Echar}{E_{\text{char}}}
\providecommand{\kfrac}{k_{\text{frac}}}
\providecommand{\alphaEMSI}{\alpha_{\text{EM,SI}}}
\providecommand{\alphaEMnat}{\alpha_{\text{EM,nat}}}
\providecommand{\betaTSI}{\beta_{T,\text{SI}}}
\providecommand{\betaTnat}{\beta_{T,\text{nat}}}
\providecommand{\Gsi}{G_{\text{SI}}}
\providecommand{\xiparSI}{\xi_{\text{SI}}}
\providecommand{\xiparnat}{\xi_{\text{nat}}}
\providecommand{\meff}{m_{\text{eff}}}
\providecommand{\Tzerot}{T_{0}(t)}
\providecommand{\mzerot}{m_{0}(t)}
\providecommand{\Ezeroabs}{E_{0,\text{abs}}}
\providecommand{\Epar}{E_{\text{par}}}
\providecommand{\Lnat}{\ell_{\text{nat}}}
\providecommand{\Tnat}{T_{\text{nat}}}
\providecommand{\xifrak}{\xi_{\text{frac}}}
\providecommand{\Tfrak}{T_{\text{frac}}}
\providecommand{\mfrak}{m_{\text{frac}}}
\providecommand{\Dfrac}{D_{\text{frac}}}
\providecommand{\EphotSI}{E_{\gamma,\text{SI}}}
\providecommand{\EphotNat}{E_{\gamma,\text{nat}}}
\providecommand{\Eabsint}{E_{\text{abs,int}}}
\providecommand{\mphoton}{m_{\gamma}}
\providecommand{\Evis}{E_{\text{vis}}}
\providecommand{\Cto}{C_{T0}}
\providecommand{\mytimes}{\times}
\providecommand{\lambdah}{\lambda_h}
\providecommand{\checkmarkx}{\checkmark}
\providecommand{\Enorm}{E_{\text{norm}}}
\providecommand{\Tobs}{T_{\text{obs}}}
\providecommand{\mobs}{m_{\text{obs}}}
\providecommand{\Eobs}{E_{\text{obs}}}
\providecommand{\Lobs}{\ell_{\text{obs}}}
\providecommand{\xobs}{\xi_{\text{obs}}}
\providecommand{\calE}{\mathcal{E}}
\providecommand{\calT}{\mathcal{T}}
\providecommand{\calM}{\mathcal{M}}
\providecommand{\alphag}{\alpha_g}
\providecommand{\Tmax}{T_{\text{max}}}
\providecommand{\mmin}{m_{\text{min}}}
\providecommand{\Lmax}{\ell_{\text{max}}}
\providecommand{\Emin}{E_{\text{min}}}
\providecommand{\Geff}{G_{\text{eff}}}
\providecommand{\rhoeff}{\rho_{\text{eff}}}
\providecommand{\xieff}{\xi_{\text{eff}}}
\providecommand{\Teff}{T_{\text{eff}}}
\providecommand{\hPlanck}{h}
\providecommand{\kB}{k_B}
\providecommand{\muB}{\mu_B}
\providecommand{\lambdaC}{\lambda_C}
\providecommand{\omegaP}{\omega_P}
\providecommand{\rhoP}{\rho_P}
\providecommand{\Tref}{T_{\text{ref}}}
\providecommand{\Eref}{E_{\text{ref}}}
\providecommand{\mref}{m_{\text{ref}}}
\providecommand{\Lref}{\ell_{\text{ref}}}
\providecommand{\xikonst}{\xi_0}
\providecommand{\Phiphoton}{\Phi_{\gamma}}
\providecommand{\etavis}{\eta_{\text{vis}}}
\providecommand{\pichar}{\pi}
\providecommand{\primrel}{\mathcal{P}_{\text{rel}}}
\providecommand{\warningx}{\textcolor{orange}{\textbf{!}}}
\providecommand{\phiT}{\phi_T}
\providecommand{\Lorentz}{\Lambda}
\providecommand{\Cconv}{C_{\text{conv}}}
\providecommand{\Df}{\Delta f}
\providecommand{\lambdazero}{\lambda_0}
\providecommand{\myapprox}{\approx}
\providecommand{\checked}{\checkmark}
\providecommand{\alphaWSI}{\alpha_W^{\text{SI}}}
\providecommand{\alphaWnat}{\alpha_W^{\text{nat}}}
\providecommand{\vect}[1]{\vec{#1}}
\providecommand{\Rzero}{R_0}
\providecommand{\Riem}{\mathcal{R}}
\providecommand{\nuzero}{\nu_0}
\providecommand{\mypi}{\pi}

% =============================================================================
% TCOLORBOX STYLES AND ENVIRONMENTS (English titles)
% =============================================================================
\tcbset{
	keyresult/.style={
		colback=blue!5!white,
		colframe=blue!75!black,
		title=Key Result,
		fonttitle=\bfseries
	},
	foundation/.style={
		colback=green!5!white,
		colframe=green!75!black,
		title=Foundation,
		fonttitle=\bfseries
	},
	alternative/.style={
		colback=orange!5!white,
		colframe=orange!75!black,
		title=Alternative,
		fonttitle=\bfseries
	},
	warningbox/.style={
		colback=red!5!white,
		colframe=red!75!black,
		title=Warning,
		fonttitle=\bfseries
	}
}

% (Here follow all your tcolorbox definitions with English titles)
\newtcolorbox{keyresultbox}[1][]{colback=blue!5!white,colframe=blue!75!black,fonttitle=\bfseries,title={#1},breakable}
\newtcolorbox{keyresult}[1][Key Result]{colback=blue!5!white,colframe=blue!75!black,fonttitle=\bfseries,title={#1},breakable}
\newtcolorbox{foundationbox}[1][]{colback=green!5!white,colframe=green!75!black,fonttitle=\bfseries,title={#1},breakable}
\newtcolorbox{foundation}[1][Foundation]{colback=green!5!white,colframe=green!75!black,fonttitle=\bfseries,title={#1},breakable}
\newtcolorbox{alternativebox}[1][]{colback=orange!5!white,colframe=orange!75!black,fonttitle=\bfseries,title={#1},breakable}
\newtcolorbox{warningboxenv}[1][Warning]{colback=red!5!white,colframe=red!75!black,fonttitle=\bfseries,title={#1},breakable}

\newtcolorbox{fundamental}[1][]{
	colback=boxgray,
	colframe=t0blue,
	fonttitle=\bfseries,
	title=#1,
	sharp corners,
	boxrule=2pt
}

\newtcolorbox{insightBox}[1][Insight]{colback=blue!5,colframe=t0blue,title={#1},fonttitle=\bfseries,breakable}
\newtcolorbox{discoveryBox}[1][Discovery]{colback=green!5,colframe=t0green,title={#1},fonttitle=\bfseries,breakable}
\newtcolorbox{revelation}[1][Revelation]{colback=red!5,colframe=t0red,title={#1},fonttitle=\bfseries,breakable}
\newtcolorbox{keypoint}[1][Key Point]{colback=blue!5,colframe=t0blue,title={#1},fonttitle=\bfseries,breakable}
\newtcolorbox{evidence}[1][Evidence]{colback=green!5,colframe=t0green,title={#1},fonttitle=\bfseries,breakable}
\newtcolorbox{conclusionBox}[1][Conclusion]{colback=gray!5,colframe=gray,title={#1},fonttitle=\bfseries,breakable}
\newtcolorbox{significance}[1][Significance]{colback=yellow!5,colframe=orange,title={#1},fonttitle=\bfseries,breakable}
\newtcolorbox{philosophical}[1][Philosophical]{colback=purple!5,colframe=purple,title={#1},fonttitle=\bfseries,breakable}
\newtcolorbox{implicationBox}[1][Implication]{colback=cyan!5,colframe=cyan,title={#1},fonttitle=\bfseries,breakable}
\newtcolorbox{perspectiveBox}[1][Perspective]{colback=blue!5,colframe=t0blue,title={#1},fonttitle=\bfseries,breakable}
\newtcolorbox{revolutionary}[1][Revolutionary]{colback=red!5,colframe=t0red,title={#1},fonttitle=\bfseries,breakable}

\newtcolorbox{technical}[1][Technical]{colback=gray!5,colframe=gray!75!black,title={#1},fonttitle=\bfseries,breakable}
\newtcolorbox{technicalBox}[1][Technical]{colback=gray!5,colframe=gray!75!black,title={#1},fonttitle=\bfseries,breakable}
\newtcolorbox{notationBox}[1][Notation]{colback=yellow!5,colframe=yellow!75!black,title={#1},fonttitle=\bfseries,breakable}
\newtcolorbox{verification}[1][Verification]{colback=orange!5!white,colframe=orange!75!black,fonttitle=\bfseries,title=#1}
\newtcolorbox{explanationBox}[1][Explanation]{colback=purple!5!white,colframe=purple!75!black,fonttitle=\bfseries,title=#1}
\newtcolorbox{interpretationBox}[1][Interpretation]{colback=cyan!5!white,colframe=cyan!75!black,fonttitle=\bfseries,title=#1}
\newtcolorbox{explanation}[1][Explanation]{colback=purple!5!white,colframe=purple!75!black,fonttitle=\bfseries,title=#1,breakable}
\newtcolorbox{interpretation}[1][Interpretation]{colback=cyan!5!white,colframe=cyan!75!black,fonttitle=\bfseries,title=#1,breakable}
\newtcolorbox{proof_step}[1][Proof Step]{colback=gray!5!white,colframe=gray!75!black,fonttitle=\bfseries,title=#1,breakable}
\newtcolorbox{experimental}[1][Experimental]{colback=teal!5!white,colframe=teal!75!black,fonttitle=\bfseries,title=#1,breakable}

\newtcolorbox{important}[1][Important]{colback=red!5!white,colframe=red!75!black,title={#1},fonttitle=\bfseries,breakable}
\newtcolorbox{warning}[1][Warning]{colback=orange!5!white,colframe=orange!75!black,title={#1},fonttitle=\bfseries,breakable}
\newtcolorbox{caution}[1][Caution]{colback=yellow!5!white,colframe=yellow!75!black,title={#1},fonttitle=\bfseries,breakable}
\newtcolorbox{highlight}[1][Highlight]{colback=yellow!10!white,colframe=yellow!75!black,title={#1},fonttitle=\bfseries,breakable}
\newtcolorbox{critical}[1][Critical]{colback=red!10!white,colframe=red!75!black,title={#1},fonttitle=\bfseries,breakable}

\newtcolorbox{analysis}[1][Analysis]{colback=blue!5!white,colframe=blue!75!black,title={#1},fonttitle=\bfseries,breakable}
\newtcolorbox{application}[1][Application]{colback=green!5!white,colframe=green!75!black,title={#1},fonttitle=\bfseries,breakable}
\newtcolorbox{experiment}[1][Experiment]{colback=cyan!5!white,colframe=cyan!75!black,title={#1},fonttitle=\bfseries,breakable}
\newtcolorbox{historical}[1][Historical]{colback=brown!5!white,colframe=brown!75!black,title={#1},fonttitle=\bfseries,breakable}
\newtcolorbox{numerical}[1][Numerical]{colback=gray!5!white,colframe=gray!75!black,title={#1},fonttitle=\bfseries,breakable}
\newtcolorbox{overview}[1][Overview]{colback=blue!5!white,colframe=blue!75!black,title={#1},fonttitle=\bfseries,breakable}
\newtcolorbox{speculation}[1][Speculation]{colback=purple!5!white,colframe=purple!75!black,title={#1},fonttitle=\bfseries,breakable}
\newtcolorbox{question}[1][Question]{colback=orange!5!white,colframe=orange!75!black,title={#1},fonttitle=\bfseries,breakable}
\newtcolorbox{method}[1][Method]{colback=teal!5!white,colframe=teal!75!black,title={#1},fonttitle=\bfseries,breakable}
\newtcolorbox{correct}[1][Correct]{colback=green!10!white,colframe=green!75!black,title={#1},fonttitle=\bfseries,breakable}
\newtcolorbox{units}[1][Units]{colback=gray!5!white,colframe=gray!75!black,title={#1},fonttitle=\bfseries,breakable}
\newtcolorbox{achievement}[1][Achievement]{colback=gold!5!white,colframe=orange!75!black,title={#1},fonttitle=\bfseries,breakable}
\newtcolorbox{equivalence}[1][Equivalence]{colback=cyan!5!white,colframe=cyan!75!black,title={#1},fonttitle=\bfseries,breakable}
\newtcolorbox{dimensional}[1][Dimensional Analysis]{colback=purple!5!white,colframe=purple!75!black,title={#1},fonttitle=\bfseries,breakable}

% === ADDITIONAL SIMPLE ENVIRONMENTS ===
\newenvironment{treatise}{\begin{quote}}{\end{quote}}
\newenvironment{gemeinsam}{\begin{quote}}{\end{quote}}
\newenvironment{vergleich}{\begin{quote}}{\end{quote}}
\newenvironment{vorteil}{\begin{quote}}{\end{quote}}
\newenvironment{common}{\begin{quote}}{\end{quote}}
\newenvironment{comparison}{\begin{quote}}{\end{quote}}
\newenvironment{advantage}{\begin{quote}}{\end{quote}}
\newenvironment{quantum}{\begin{quote}}{\end{quote}}

% === LAYOUT SETTINGS ===
\raggedbottom
\usepackage{environ}
\let\oldtabular\tabular
\let\endoldtabular\endtabular

\newenvironment{scaledtable}[1][0.85]{%
	\begingroup\footnotesize\setlength{\LTleft}{0pt}\setlength{\LTright}{0pt}%
}{%
	\endgroup%
}

\newcommand{\widetable}[1]{\resizebox{\textwidth}{!}{#1}}

% === TABLE OF CONTENTS FORMATTING ===
\renewcommand{\cftsecfont}{\color{blue}}
\renewcommand{\cftsubsecfont}{\color{blue}}
\renewcommand{\cftsecpagefont}{\color{blue}}
\renewcommand{\cftsubsecpagefont}{\color{blue}}
\renewcommand{\cfttoctitlefont}{\huge\bfseries\color{blue}}

% === DEFAULT HEADER AND FOOTER ===
\pagestyle{fancy}
\fancyhf{}
\fancyhead[L]{\textsc{T0 Theory}}
\fancyhead[R]{\textsc{J. Pascher}}
\fancyfoot[C]{\thepage}

% ==============================================================================
% End of Shared Preamble for English
% ==============================================================================
%
% Usage:
%   \documentclass[12pt,a4paper]{article}  % or book, report, etc.
%   % ==============================================================================
% T0 Theory: Shared ENGLISH Preamble – Optimized for eBook/Book
% Version: 2.0 – Final 2026 (LuaLaTeX only) – ENGLISH corrected
% Author: Johann Pascher
% Date: January 2026
% ==============================================================================
%
% IMPORTANT: Compile EXCLUSIVELY with LuaLaTeX!
% In TeXstudio: Options → Configure TeXstudio → Build → Default Compiler → LuaLaTeX
%
% Required Fonts (install once):
% - Inter: https://fonts.google.com/specimen/Inter
% - JetBrains Mono: https://www.jetbrains.com/lp/mono/
% - Libertinus Math: https://github.com/libertinus-fonts/libertinus
% ==============================================================================

% === CHAPTER 1: BASIC PACKAGES (must come FIRST) ===
\RequirePackage{fontspec}
\RequirePackage{unicode-math}
\usepackage{chngcntr}
\setcounter{secnumdepth}{1}  % Nur Sections nummerieren (nicht subsections)
\setcounter{tocdepth}{1}     % Nur Sections im TOC (nicht subsections)
\makeatletter
\@ifundefined{c@chapter}{}{\counterwithout{section}{chapter}}  % Falls Kapitel existieren
\makeatother
\counterwithout{subsection}{section}  % Löse Verknüpfung
% === CHAPTER 2: LANGUAGE (ENGLISH) ===
\usepackage[english]{babel}
\usepackage{microtype}                    % IMPORTANT for better hyphenation!

% Typography settings for better line breaking
\frenchspacing                     % Correct English spacing after punctuation
\emergencystretch=3em              % Allows more stretch for difficult lines
\tolerance=2500                    % Higher tolerance for line breaks
\hbadness=10000                    % Suppresses "underfull hbox" warnings
\hfuzz=2pt                         % Allows minimal overfull
\pretolerance=150                  % Better word breaking

% Prevent bad page breaks
\clubpenalty=10000           % No "orphans"
\widowpenalty=10000          % No "widows"
\displaywidowpenalty=10000   % Also with equations
\brokenpenalty=10000         % No broken words across pages

% Explicit hyphenation for long technical words
\hyphenation{Fun-da-men-tal Frac-tal-Ge-o-met-ric Field The-o-ry Meth-od-o-log-i-cal}
\hyphenation{Re-vi-sion-ism Quan-ti-za-tion U-ni-fi-ca-tion Ef-fec-tive}
\hyphenation{Re-nor-mal-iz-a-bil-i-ty Sin-gu-lar-i-ties Con-cil-i-a-tion}
\hyphenation{E-mer-gence Phe-nom-e-no-log-i-cal Doc-u-men-ta-tion A-nal-y-sis}
\hyphenation{Grav-i-ta-tion Quan-tum Me-chan-ics Dog-ma-tism Con-se-quent}
\hyphenation{Par-al-lel-ism Im-ple-men-ta-tion Per-tur-ba-tions}
\hyphenation{Geo-met-ric Ar-ti-fact In-com-pat-i-bil-i-ty Con-struc-tive}
\hyphenation{Frac-tal Di-men-sion-less In-ves-ti-ga-tion De-scrip-tion}
\hyphenation{In-ter-pre-ta-tion Phe-nom-e-no-log-i-cal Math-e-mat-i-cal}
\hyphenation{Phi-lo-soph-i-cal Le-git-i-ma-tion Ap-pli-ca-tion Der-i-va-tion}
\hyphenation{U-ni-fi-ca-tion As-sump-tion Con-cep-tion Ex-pec-ta-tion}
\hyphenation{Sym-me-try-ex-ten-sion O-ver-all-pic-ture Chal-lenge}
\hyphenation{In-ter-ac-tion Ma-te-ri-al Ap-proach Per-spec-tive Pro-ce-dure}

% === CHAPTER 3: FONTS (with proper ligatures) ===
\setmainfont{Inter}[
Scale=1.02,
UprightFont=*-Regular,
BoldFont=*-Bold,
ItalicFont=*-Italic,
BoldItalicFont=*-BoldItalic,
Ligatures=TeX,           % IMPORTANT for proper typography
Language=English         % Explicit language support
]
\setsansfont{Inter}[
Scale=MatchLowercase,
Ligatures=TeX,
Language=English
]
\setmonofont{JetBrains Mono}[
Scale=0.95,
Language=English
]

% Math Font (simple & stable) – MUST come AFTER language definition
% IMPORTANT: Libertinus Math for correct \underbrace display!
\setmathfont{Libertinus Math}[Scale=1.0]

% === CHAPTER 4: MATHEMATICS PACKAGES (in STRICT order!) ===
% IMPORTANT: mathtools must come BEFORE unicode-math for some commands!
\usepackage{mathtools}           % FIRST mathtools!

% Then the rest
\usepackage{amsmath, amsfonts, amsthm}

% SIUNITX MUST be loaded BEFORE physics!
\usepackage{siunitx}
\sisetup{
	locale=US,                    % ENGLISH settings for SI units!
	group-separator={,},          % Thousands separator comma
	output-decimal-marker={.},    % Decimal separator point
	per-mode=symbol,
	separate-uncertainty=true
}

% Custom SI units used in narrative and books
\DeclareSIUnit\gigalightyear{Gly}
\DeclareSIUnit\mev{MeV}

% physics – MUST be loaded AFTER siunitx and mathtools
\usepackage{physics}

% === CHAPTER 5: ADDITIONS from pdflatex best practices ===
\usepackage{colortbl}        % Colored tables (ESSENTIAL!)
\usepackage{placeins}        % Float control: \FloatBarrier
\usepackage{subcaption}      % Subfigures
\usepackage{xurl}            % Better URL line breaking
% Hyphenation for URLs in bibliography
\def\UrlBreaks{\do\/\do-}

% === CHAPTER 6: PAGE LAYOUT
% =============================================================================
% SECTION 2: Page Geometry – 6" × 9" Buchformat
% =============================================================================
\usepackage[paperwidth=6in, paperheight=9in,
top=0.9in,
bottom=1.1in,
inner=0.9in,            % Größerer Innenrand für Bindung
outer=0.6in,            % Kleinerer Außenrand → mehr Text pro Seite
bindingoffset=0.5in,    % Puffer für Bindung (Steg)
twoside]{geometry}
\setlength{\headheight}{15pt}
%\usepackage[paperwidth=8.25in, paperheight=11in,
%top=1.0in,
%bottom=1.0in,
%left=1.0in,
%right=1.0in,
%twoside=false
% === CHAPTER 7: GRAPHICS AND TABLES ===
\usepackage{graphicx}
\usepackage[table,xcdraw]{xcolor}
% T0 brand colors
\definecolor{gold}{RGB}{255,215,0}
\definecolor{blue}{rgb}{0,0,1}
\definecolor{boxgray}{RGB}{240,240,240}
\definecolor{deepblue}{RGB}{0,0,127}
\definecolor{deepgreen}{RGB}{0,127,0}
\definecolor{deepred}{RGB}{191,0,0}
\definecolor{t0blue}{RGB}{33,150,243}
\definecolor{t0green}{RGB}{76,175,80}
\definecolor{t0orange}{RGB}{255,152,0}
\definecolor{t0purple}{RGB}{156,39,176}
\definecolor{t0red}{RGB}{244,67,54}
\definecolor{t0yellow}{RGB}{255,204,0}
\usepackage{tikz}
\usetikzlibrary{arrows.meta,positioning,shapes.geometric,decorations.pathmorphing,patterns,shapes.arrows,intersections}
\usepackage{pgfplots}
\pgfplotsset{compat=1.18}
\usepackage{quantikz}
\usepackage[most]{tcolorbox}
\tcbuselibrary{breakable}

% === WICHTIG: Algorithm-Konflikt umgehen ===
% Option: algorithmic mit GROSSBUCHSTABEN
% Gemeinsame Box für Experimente
\newtcolorbox{experimentbox}[1][]{
	colback=green!5!white,
	colframe=t0green!80!black,
	fonttitle=\bfseries,
	title={{#1}},
	breakable
}

% Abstract-Fallback
\ifdefined\abstract\else
\newenvironment{abstract}{\section*{\abstractname}\itshape\small\par\bigskip}{\bigskip}
\fi

% === MAKROS SICHER NEU DEFINIEREN / ÜBERSCHREIBEN ===
% Definiere Makros OHNE doppelte Subskripte
\newcommand{\phipar}{\phi_{\mathrm{par}}}
%\newcommand{\xipar}{\xi_{\mathrm{par}}}
\newcommand{\Qphipar}{Q_{\phi_{\mathrm{par}}}}
\newcommand{\rphipar}{r_{\phi_{\mathrm{par}}}}
\newcommand{\logphipar}{\log_{\phi_{\mathrm{par}}}}
\newcommand{\CHSH}{\text{CHSH}}
\usepackage{booktabs}
\usepackage{array}
\usepackage{longtable}
\usepackage{float}
\usepackage{adjustbox}
\usepackage{rotating}
\usepackage{tabularx}
\usepackage{makecell}
\usepackage{multirow}

% === CHAPTER 8: DOCUMENT FORMATTING ===
\usepackage{fancyhdr}
\renewcommand{\headrulewidth}{0.4pt}
\renewcommand{\footrulewidth}{0.4pt}
\usepackage{tocloft}

\usepackage{enumitem}
\setlist[itemize]{leftmargin=*, topsep=2pt, partopsep=0pt, parsep=2pt, itemsep=2pt}
\setlist[enumerate]{leftmargin=*, topsep=2pt, partopsep=0pt, parsep=2pt, itemsep=2pt}
\usepackage{setspace}
\usepackage{ragged2e}
\usepackage{multicol}

% === CHAPTER 9: CODE AND ALGORITHMS ===
\usepackage{algorithm}
\usepackage{algorithmic}
\usepackage{listings}
\lstset{
	basicstyle=\ttfamily\footnotesize,
	breaklines=true,
	breakatwhitespace=true,
	columns=flexible,
	keepspaces=true,
	showstringspaces=false,
	frame=single,
	xleftmargin=0pt,
	xrightmargin=0pt,
	literate=              % For special characters in code listings
	{ä}{{\"a}}1 {ö}{{\"o}}1 {ü}{{\"u}}1 {ß}{{\ss}}1
	{Ä}{{\"A}}1 {Ö}{{\"O}}1 {Ü}{{\"U}}1
}
\usepackage{mdframed}

% === CHAPTER 10: ADDITIONAL PACKAGES ===
\usepackage{pdflscape}
\usepackage{braket}
\usepackage{cancel}
\usepackage{caption}
\captionsetup{format=plain, labelfont=bf, justification=centering}
\usepackage{csquotes}
\usepackage{gensymb}
\usepackage{textcomp}
\usepackage{textgreek}
\usepackage{upgreek}
\usepackage{url}
\usepackage{slashed}
\usepackage{bm}

% === CHAPTER 11: HYPERREF (must come SECOND TO LAST!) ===
\usepackage{hyperref}
\hypersetup{
	colorlinks=true,
	linkcolor=black,
	citecolor=black,
	urlcolor=black,
	breaklinks=true,           % IMPORTANT for special characters in URLs!
	bookmarksnumbered=true,
	unicode=true,
	pdfencoding=auto,
	pdflang=en,                % Set PDF language to English
	pdfsubject={T0 Theory - Fundamental Fractal-Geometric Field Theory}
}

% Fix for unicode-math symbols in PDF bookmarks
\pdfstringdefDisableCommands{%
	\def\xi{xi}%
	\def\alpha{alpha}%
	\def\beta{beta}%
	\def\gamma{gamma}%
	\def\delta{delta}%
	\def\Delta{Delta}%
	\def\epsilon{epsilon}%
	\def\varepsilon{epsilon}%
	\def\theta{theta}%
	\def\kappa{kappa}%
	\def\lambda{lambda}%
	\def\mu{mu}%
	\def\nu{nu}%
	\def\pi{pi}%
	\def\rho{rho}%
	\def\sigma{sigma}%
	\def\tau{tau}%
	\def\phi{phi}%
	\def\chi{chi}%
	\def\psi{psi}%
	\def\omega{omega}%
	\def\Omega{Omega}%
	\def\Lambda{Lambda}%
	\def\times{x}%
	\def\cdot{*}%
	\def\pm{+/-}%
	\def\approx{~}%
	\def\sim{~}%
	\def\equiv{=}%
	\def\ell{l}%
	\def\hbar{h}%
	\def\rightarrow{->}%
	\def\leftarrow{<-}%
	\def\Rightarrow{=>}%
	\def\Leftarrow{<=}%
	\def\propto{~}%
	\def\mitxi{xi}%
	\def\mitalpha{alpha}%
	\def\mitbeta{beta}%
	\def\mitgamma{gamma}%
	\def\mitdelta{delta}%
	\def\mitDelta{Delta}%
	\def\mitepsilon{epsilon}%
	\def\mitvarepsilon{epsilon}%
	\def\mittheta{theta}%
	\def\mitkappa{kappa}%
	\def\mitlambda{lambda}%
	\def\mitLambda{Lambda}%
	\def\mitmu{mu}%
	\def\mitnu{nu}%
	\def\mitpi{pi}%
	\def\mitrho{rho}%
	\def\mitsigma{sigma}%
	\def\mittau{tau}%
	\def\mitphi{phi}%
	\def\mitchi{chi}%
	\def\mitpsi{psi}%
	\def\mitomega{omega}%
	\def\mitOmega{Omega}%
}

% === CHAPTER 12: BOOKMARK (must come AFTER hyperref!) ===
\usepackage{bookmark}

% === CHAPTER 13: CLEVEREF (ENGLISH LABELS) ===
\usepackage[english]{cleveref}
\crefname{equation}{Equation}{Equations}
\crefname{figure}{Figure}{Figures}
\crefname{table}{Table}{Tables}
\crefname{section}{Section}{Sections}
\crefname{chapter}{Chapter}{Chapters}
\crefname{theorem}{Theorem}{Theorems}
\crefname{lemma}{Lemma}{Lemmas}
\crefname{definition}{Definition}{Definitions}
\crefname{example}{Example}{Examples}
\crefname{remark}{Remark}{Remarks}

% === CUSTOM ENVIRONMENTS ===
% Alternative interpretation environment
\newenvironment{alternative}{%
	\begin{mdframed}[linecolor=black!30,linewidth=1pt,roundcorner=4pt,backgroundcolor=black!5]%
	}{%
	\end{mdframed}%
}

% Photon/particle environment
\newenvironment{photon}{%
	\begin{mdframed}[linecolor=blue!30,linewidth=1pt,roundcorner=4pt,backgroundcolor=blue!5]%
	}{%
	\end{mdframed}%
}

% Koide formula box environment
\newenvironment{koidebox}{%
	\begin{mdframed}[linecolor=green!30,linewidth=1pt,roundcorner=4pt,backgroundcolor=green!5]%
	}{%
	\end{mdframed}%
}

% Erkenntnis/insight environment
\newenvironment{erkenntnis}{%
	\begin{mdframed}[linecolor=orange!30,linewidth=1pt,roundcorner=4pt,backgroundcolor=orange!5]%
	}{%
	\end{mdframed}%
}

% Beziehung/relationship environment
\newenvironment{beziehung}{%
	\begin{mdframed}[linecolor=purple!30,linewidth=1pt,roundcorner=4pt,backgroundcolor=purple!5]%
	}{%
	\end{mdframed}%
}

% Derivation environment
\newenvironment{derivation}{%
	\begin{mdframed}[linecolor=teal!30,linewidth=1pt,roundcorner=4pt,backgroundcolor=teal!5]%
	}{%
	\end{mdframed}%
}

% Abhandlung/treatise environment
\newenvironment{abhandlung}{%
	\begin{mdframed}[linecolor=brown!30,linewidth=1pt,roundcorner=4pt,backgroundcolor=brown!5]%
	}{%
	\end{mdframed}%
}

% Anwendung/application environment
\newenvironment{anwendung}{%
	\begin{mdframed}[linecolor=cyan!30,linewidth=1pt,roundcorner=4pt,backgroundcolor=cyan!5]%
	}{%
	\end{mdframed}%
}

% Additional common environments
\newenvironment{konsequenz}{%
	\begin{mdframed}[linecolor=red!30,linewidth=1pt,roundcorner=4pt,backgroundcolor=red!5]%
	}{%
	\end{mdframed}%
}

\newenvironment{schlussfolgerung}{%
	\begin{mdframed}[linecolor=gray!30,linewidth=1pt,roundcorner=4pt,backgroundcolor=gray!5]%
	}{%
	\end{mdframed}%
}

\newenvironment{result}{%
	\begin{mdframed}[linecolor=violet!30,linewidth=1pt,roundcorner=4pt,backgroundcolor=violet!5]%
	}{%
	\end{mdframed}%
}

% Formula environment
\newenvironment{formula}{%
	\begin{mdframed}[linecolor=yellow!30,linewidth=1pt,roundcorner=4pt,backgroundcolor=yellow!5]%
	}{%
	\end{mdframed}%
}

% Revolutionaer/revolutionary environment
\newenvironment{revolutionaer}{%
	\begin{mdframed}[linecolor=red!50,linewidth=2pt,roundcorner=4pt,backgroundcolor=red!10]%
	}{%
	\end{mdframed}%
}

% Formel environment (German version of formula)
\newenvironment{formel}{%
	\begin{mdframed}[linecolor=yellow!30,linewidth=1pt,roundcorner=4pt,backgroundcolor=yellow!5]%
	}{%
	\end{mdframed}%
}

% Prinzip/principle environment
\newenvironment{prinzip}{%
	\begin{mdframed}[linecolor=blue!50,linewidth=2pt,roundcorner=4pt,backgroundcolor=blue!10]%
	}{%
	\end{mdframed}%
}

% Experimentell/experimental environment
\newenvironment{experimentell}{%
	\begin{mdframed}[linecolor=magenta!30,linewidth=1pt,roundcorner=4pt,backgroundcolor=magenta!5]%
	}{%
	\end{mdframed}%
}

% Neutrino environment
\newenvironment{neutrino}{%
	\begin{mdframed}[linecolor=cyan!40,linewidth=1pt,roundcorner=4pt,backgroundcolor=cyan!8]%
	}{%
	\end{mdframed}%
}

% Additional missing environments
\newenvironment{schluessel}{%
	\begin{mdframed}[linecolor=yellow!50,linewidth=1pt,roundcorner=4pt,backgroundcolor=yellow!10]%
	}{%
	\end{mdframed}%
}

\newenvironment{summary}{%
	\begin{mdframed}[linecolor=gray!40,linewidth=1pt,roundcorner=4pt,backgroundcolor=gray!8]%
	}{%
	\end{mdframed}%
}

\newenvironment{category}{%
	\begin{mdframed}[linecolor=pink!40,linewidth=1pt,roundcorner=4pt,backgroundcolor=pink!8]%
	}{%
	\end{mdframed}%
}

\newenvironment{sibox}{%
	\begin{mdframed}[linecolor=lime!40,linewidth=1pt,roundcorner=4pt,backgroundcolor=lime!8]%
	}{%
	\end{mdframed}%
}

% More missing environments
\newenvironment{documentbox}{%
	\begin{mdframed}[linecolor=teal!40,linewidth=1pt,roundcorner=4pt,backgroundcolor=teal!8]%
	}{%
	\end{mdframed}%
}

\newenvironment{t0box}{%
	\begin{mdframed}[linecolor=violet!40,linewidth=1pt,roundcorner=4pt,backgroundcolor=violet!8]%
	}{%
	\end{mdframed}%
}

\newenvironment{wichtig}{%
	\begin{mdframed}[linecolor=red!50,linewidth=2pt,roundcorner=4pt,backgroundcolor=red!10]%
	\textbf{Important:} 
	}{%
	\end{mdframed}%
}

\newenvironment{smbox}{%
	\begin{mdframed}[linecolor=orange!40,linewidth=1pt,roundcorner=4pt,backgroundcolor=orange!8]%
	}{%
	\end{mdframed}%
}

\newenvironment{pvbox}{%
	\begin{mdframed}[linecolor=purple!40,linewidth=1pt,roundcorner=4pt,backgroundcolor=purple!8]%
	}{%
	\end{mdframed}%
}

\newenvironment{numerisch}{%
	\begin{mdframed}[linecolor=blue!40,linewidth=1pt,roundcorner=4pt,backgroundcolor=blue!8]%
	}{%
	\end{mdframed}%
}

% More missing environments
\newenvironment{relation}{%
	\begin{mdframed}[linecolor=green!40,linewidth=1pt,roundcorner=4pt,backgroundcolor=green!8]%
	}{%
	\end{mdframed}%
}

\newenvironment{beweis}{%
	\begin{mdframed}[linecolor=brown!40,linewidth=1pt,roundcorner=4pt,backgroundcolor=brown!8]%
	\textbf{Proof:} 
	}{%
	\end{mdframed}%
}

\newenvironment{revolution}{%
	\begin{mdframed}[linecolor=red!60,linewidth=2pt,roundcorner=4pt,backgroundcolor=red!12]%
	}{%
	\end{mdframed}%
}

\newenvironment{key}{%
	\begin{mdframed}[linecolor=yellow!50,linewidth=1pt,roundcorner=4pt,backgroundcolor=yellow!10]%
	}{%
	\end{mdframed}%
}

\newenvironment{newperspective}{%
	\begin{mdframed}[linecolor=cyan!50,linewidth=1pt,roundcorner=4pt,backgroundcolor=cyan!10]%
	}{%
	\end{mdframed}%
}

\newenvironment{literatur}{%
	\begin{mdframed}[linecolor=gray!50,linewidth=1pt,roundcorner=4pt,backgroundcolor=gray!10]%
	}{%
	\end{mdframed}%
}

\newenvironment{folgerung}{%
	\begin{mdframed}[linecolor=teal!50,linewidth=1pt,roundcorner=4pt,backgroundcolor=teal!10]%
	}{%
	\end{mdframed}%
}

\newenvironment{principle}{%
	\begin{mdframed}[linecolor=blue!60,linewidth=2pt,roundcorner=4pt,backgroundcolor=blue!12]%
	}{%
	\end{mdframed}%
}

% Additional common environments
% ==============================================================================
% FROM HERE: YOUR DEFINITIONS (unchanged)
% ==============================================================================

\setcounter{tocdepth}{3}

% === CITATION COMMANDS ===
\providecommand{\citep}[1]{\cite{#1}}
\providecommand{\citet}[1]{\cite{#1}}

% === COLORS ===
\definecolor{gold}{RGB}{255,215,0}
\definecolor{blue}{rgb}{0,0,1}
\definecolor{boxgray}{RGB}{240,240,240}
\definecolor{deepblue}{RGB}{0,0,127}
\definecolor{deepgreen}{RGB}{0,127,0}
\definecolor{deepred}{RGB}{191,0,0}
\definecolor{t0blue}{RGB}{33,150,243}
\definecolor{t0green}{RGB}{76,175,80}
\definecolor{t0orange}{RGB}{255,152,0}
\definecolor{t0purple}{RGB}{156,39,176}
\definecolor{t0red}{RGB}{244,67,54}
\definecolor{t0yellow}{RGB}{255,204,0}

% === COLUMN TYPES ===
\newcolumntype{L}[1]{>{\raggedright\arraybackslash}p{#1}}
\newcolumntype{C}[1]{>{\centering\arraybackslash}p{#1}}
\newcolumntype{R}[1]{>{\raggedleft\arraybackslash}p{#1}}

% === HYPERREF SETTINGS (updated) ===
\hypersetup{
	colorlinks=true,
	linkcolor=t0blue,
	citecolor=t0blue,
	urlcolor=t0blue,
	breaklinks=true,
	bookmarksnumbered=true,
	pdfstartview=FitH,
	pdfencoding=auto,
	pdfdisplaydoctitle=true
}

% === ENGLISH THEOREM ENVIRONMENTS ===
\theoremstyle{plain}
\newtheorem{theorem}{Theorem}[section]
\newtheorem{lemma}[theorem]{Lemma}
\newtheorem{proposition}[theorem]{Proposition}
\newtheorem{corollary}[theorem]{Corollary}

\theoremstyle{definition}
\newtheorem{definition}[theorem]{Definition}
\newtheorem{example}[theorem]{Example}
\newtheorem{insight}[theorem]{Insight}
\newtheorem{discovery}[theorem]{Discovery}

\theoremstyle{remark}
\newtheorem{remark}[theorem]{Remark}
\newtheorem{axiom}{Axiom}
%\newtheorem{principle}{Principle}  % Commented out to avoid conflicts with document-specific definitions
%\newtheorem{warning}[theorem]{Warning}

% === T0-SPECIFIC COMMANDS ===
% (Here follow all your \newcommand and \providecommand definitions)
% These remain UNCHANGED as in your original preamble
% ==============================================================================
% SECTION 14: T0-Specific Commands
% ==============================================================================

% --- Core T0 Fields ---
\newcommand{\Tfield}{T(x,t)}
\providecommand{\Tfieldt}{T(\vec{x},t)}
\newcommand{\Efield}{E(x,t)}
\newcommand{\mfield}{m(x,t)}
\providecommand{\vecx}{\vec{x}}

% --- Lagrangian ---
\newcommand{\Lag}{\mathcal{L}}
\newcommand{\calL}{\mathcal{L}}

% --- Greek Letters and Constants ---
\newcommand{\alphaem}{\alpha}
\newcommand{\betaT}{\beta_T}
\newcommand{\xiT}{\xi}
\newcommand{\xipar}{\xi}

% --- Energy and Planck Units ---
\newcommand{\Ezero}{E_0}
\newcommand{\E}{E}
\newcommand{\EPlanck}{E_{\text{Pl}}}
\newcommand{\Mpl}{M_{\text{Pl}}}
\newcommand{\mP}{m_{\text{P}}}
\newcommand{\lP}{\ell_{\text{P}}}
\newcommand{\tP}{t_{\text{P}}}
\newcommand{\LPlanck}{\ell_{\text{Pl}}}
\newcommand{\TPlanck}{t_{\text{Pl}}}

% --- Coupling Constants ---
\newcommand{\Gnat}{G_{\text{nat}}}
\newcommand{\alphaEM}{\alpha_{\text{EM}}}
\newcommand{\alphaSI}{\alpha_{\text{SI}}}
\newcommand{\Hubble}{H_0}
\newcommand{\LCDM}{\Lambda\text{CDM}}
\newcommand{\natunits}{(nat. units)}

% --- T0 Model Parameters ---
\newcommand{\xigeom}{\xi_{\mathrm{geom}}}
\newcommand{\rzero}{r_{0}}
\newcommand{\xirat}{\xi_{\mathrm{rat}}}
\newcommand{\tzero}{t_{0}}
\newcommand{\Lambdat}{\Lambda_{\mathrm{t}}}
\newcommand{\EP}{E_{\text{P}}}
\newcommand{\Emu}{E_{\mu}}
\newcommand{\Ee}{E_{e}}
\newcommand{\Etau}{E_{\tau}}
\newcommand{\alphafine}{\alpha_{\mathrm{fine}}}
\newcommand{\alphal}{\alpha_{\ell}}
\newcommand{\Lzero}{\ell_{0}}
\newcommand{\Lp}{\ell_{\mathrm{P}}}

% --- Additional T0 Commands ---
\newcommand{\Kfrak}{K_{\text{frak}}}
\newcommand{\Dfrak}{D_{\text{frak}}}
\newcommand{\betapar}{\ensuremath{\beta_T}}
\newcommand{\alphapar}{\alpha}
\newcommand{\deltafield}{\delta \phi}
\newcommand{\deltam}{\delta m}
\newcommand{\deltaE}{\delta E}
\newcommand{\Exi}{E_{\xi}}
\newcommand{\Lxi}{\ell_{\xi}}
\newcommand{\rhoCMB}{\rho_{\text{CMB}}}
\newcommand{\rhoCasimir}{\rho_{\text{Casimir}}}
\newcommand{\Leff}{L_{\text{eff}}}
\newcommand{\CQCD}{C_{\mathrm{QCD}}}
\newcommand{\Kspec}{K_{\mathrm{spec}}}
\newcommand{\Tzero}{\ensuremath{T_0}}
\newcommand{\Eabs}{E_{\text{abs}}}
\newcommand{\taupar}{\tau}

% --- Provided Commands ---
\providecommand{\xiconst}{\xi_{\text{const}}}
\providecommand{\DhiggsT}{D_{\text{Higgs-T}}}
\providecommand{\rhoE}{\rho_{E}}
\providecommand{\Echar}{E_{\text{char}}}
\providecommand{\kfrac}{k_{\text{frac}}}
\providecommand{\alphaEMSI}{\alpha_{\text{EM,SI}}}
\providecommand{\alphaEMnat}{\alpha_{\text{EM,nat}}}
\providecommand{\betaTSI}{\beta_{T,\text{SI}}}
\providecommand{\betaTnat}{\beta_{T,\text{nat}}}
\providecommand{\Gsi}{G_{\text{SI}}}
\providecommand{\xiparSI}{\xi_{\text{SI}}}
\providecommand{\xiparnat}{\xi_{\text{nat}}}
\providecommand{\meff}{m_{\text{eff}}}
\providecommand{\Tzerot}{T_{0}(t)}
\providecommand{\mzerot}{m_{0}(t)}
\providecommand{\Ezeroabs}{E_{0,\text{abs}}}
\providecommand{\Epar}{E_{\text{par}}}
\providecommand{\Lnat}{\ell_{\text{nat}}}
\providecommand{\Tnat}{T_{\text{nat}}}
\providecommand{\xifrak}{\xi_{\text{frac}}}
\providecommand{\Tfrak}{T_{\text{frac}}}
\providecommand{\mfrak}{m_{\text{frac}}}
\providecommand{\Dfrac}{D_{\text{frac}}}
\providecommand{\EphotSI}{E_{\gamma,\text{SI}}}
\providecommand{\EphotNat}{E_{\gamma,\text{nat}}}
\providecommand{\Eabsint}{E_{\text{abs,int}}}
\providecommand{\mphoton}{m_{\gamma}}
\providecommand{\Evis}{E_{\text{vis}}}
\providecommand{\Cto}{C_{T0}}
\providecommand{\mytimes}{\times}
\providecommand{\lambdah}{\lambda_h}
\providecommand{\checkmarkx}{\checkmark}
\providecommand{\Enorm}{E_{\text{norm}}}
\providecommand{\Tobs}{T_{\text{obs}}}
\providecommand{\mobs}{m_{\text{obs}}}
\providecommand{\Eobs}{E_{\text{obs}}}
\providecommand{\Lobs}{\ell_{\text{obs}}}
\providecommand{\xobs}{\xi_{\text{obs}}}
\providecommand{\calE}{\mathcal{E}}
\providecommand{\calT}{\mathcal{T}}
\providecommand{\calM}{\mathcal{M}}
\providecommand{\alphag}{\alpha_g}
\providecommand{\Tmax}{T_{\text{max}}}
\providecommand{\mmin}{m_{\text{min}}}
\providecommand{\Lmax}{\ell_{\text{max}}}
\providecommand{\Emin}{E_{\text{min}}}
\providecommand{\Geff}{G_{\text{eff}}}
\providecommand{\rhoeff}{\rho_{\text{eff}}}
\providecommand{\xieff}{\xi_{\text{eff}}}
\providecommand{\Teff}{T_{\text{eff}}}
\providecommand{\hPlanck}{h}
\providecommand{\kB}{k_B}
\providecommand{\muB}{\mu_B}
\providecommand{\lambdaC}{\lambda_C}
\providecommand{\omegaP}{\omega_P}
\providecommand{\rhoP}{\rho_P}
\providecommand{\Tref}{T_{\text{ref}}}
\providecommand{\Eref}{E_{\text{ref}}}
\providecommand{\mref}{m_{\text{ref}}}
\providecommand{\Lref}{\ell_{\text{ref}}}
\providecommand{\xikonst}{\xi_0}
\providecommand{\Phiphoton}{\Phi_{\gamma}}
\providecommand{\etavis}{\eta_{\text{vis}}}
\providecommand{\pichar}{\pi}
\providecommand{\primrel}{\mathcal{P}_{\text{rel}}}
\providecommand{\warningx}{\textcolor{orange}{\textbf{!}}}
\providecommand{\phiT}{\phi_T}
\providecommand{\Lorentz}{\Lambda}
\providecommand{\Cconv}{C_{\text{conv}}}
\providecommand{\Df}{\Delta f}
\providecommand{\lambdazero}{\lambda_0}
\providecommand{\myapprox}{\approx}
\providecommand{\checked}{\checkmark}
\providecommand{\alphaWSI}{\alpha_W^{\text{SI}}}
\providecommand{\alphaWnat}{\alpha_W^{\text{nat}}}
\providecommand{\vect}[1]{\vec{#1}}
\providecommand{\Rzero}{R_0}
\providecommand{\Riem}{\mathcal{R}}
\providecommand{\nuzero}{\nu_0}
\providecommand{\mypi}{\pi}

% =============================================================================
% TCOLORBOX STYLES AND ENVIRONMENTS (English titles)
% =============================================================================
\tcbset{
	keyresult/.style={
		colback=blue!5!white,
		colframe=blue!75!black,
		title=Key Result,
		fonttitle=\bfseries
	},
	foundation/.style={
		colback=green!5!white,
		colframe=green!75!black,
		title=Foundation,
		fonttitle=\bfseries
	},
	alternative/.style={
		colback=orange!5!white,
		colframe=orange!75!black,
		title=Alternative,
		fonttitle=\bfseries
	},
	warningbox/.style={
		colback=red!5!white,
		colframe=red!75!black,
		title=Warning,
		fonttitle=\bfseries
	}
}

% (Here follow all your tcolorbox definitions with English titles)
\newtcolorbox{keyresultbox}[1][]{colback=blue!5!white,colframe=blue!75!black,fonttitle=\bfseries,title={#1},breakable}
\newtcolorbox{keyresult}[1][Key Result]{colback=blue!5!white,colframe=blue!75!black,fonttitle=\bfseries,title={#1},breakable}
\newtcolorbox{foundationbox}[1][]{colback=green!5!white,colframe=green!75!black,fonttitle=\bfseries,title={#1},breakable}
\newtcolorbox{foundation}[1][Foundation]{colback=green!5!white,colframe=green!75!black,fonttitle=\bfseries,title={#1},breakable}
\newtcolorbox{alternativebox}[1][]{colback=orange!5!white,colframe=orange!75!black,fonttitle=\bfseries,title={#1},breakable}
\newtcolorbox{warningboxenv}[1][Warning]{colback=red!5!white,colframe=red!75!black,fonttitle=\bfseries,title={#1},breakable}

\newtcolorbox{fundamental}[1][]{
	colback=boxgray,
	colframe=t0blue,
	fonttitle=\bfseries,
	title=#1,
	sharp corners,
	boxrule=2pt
}

\newtcolorbox{insightBox}[1][Insight]{colback=blue!5,colframe=t0blue,title={#1},fonttitle=\bfseries,breakable}
\newtcolorbox{discoveryBox}[1][Discovery]{colback=green!5,colframe=t0green,title={#1},fonttitle=\bfseries,breakable}
\newtcolorbox{revelation}[1][Revelation]{colback=red!5,colframe=t0red,title={#1},fonttitle=\bfseries,breakable}
\newtcolorbox{keypoint}[1][Key Point]{colback=blue!5,colframe=t0blue,title={#1},fonttitle=\bfseries,breakable}
\newtcolorbox{evidence}[1][Evidence]{colback=green!5,colframe=t0green,title={#1},fonttitle=\bfseries,breakable}
\newtcolorbox{conclusionBox}[1][Conclusion]{colback=gray!5,colframe=gray,title={#1},fonttitle=\bfseries,breakable}
\newtcolorbox{significance}[1][Significance]{colback=yellow!5,colframe=orange,title={#1},fonttitle=\bfseries,breakable}
\newtcolorbox{philosophical}[1][Philosophical]{colback=purple!5,colframe=purple,title={#1},fonttitle=\bfseries,breakable}
\newtcolorbox{implicationBox}[1][Implication]{colback=cyan!5,colframe=cyan,title={#1},fonttitle=\bfseries,breakable}
\newtcolorbox{perspectiveBox}[1][Perspective]{colback=blue!5,colframe=t0blue,title={#1},fonttitle=\bfseries,breakable}
\newtcolorbox{revolutionary}[1][Revolutionary]{colback=red!5,colframe=t0red,title={#1},fonttitle=\bfseries,breakable}

\newtcolorbox{technical}[1][Technical]{colback=gray!5,colframe=gray!75!black,title={#1},fonttitle=\bfseries,breakable}
\newtcolorbox{technicalBox}[1][Technical]{colback=gray!5,colframe=gray!75!black,title={#1},fonttitle=\bfseries,breakable}
\newtcolorbox{notationBox}[1][Notation]{colback=yellow!5,colframe=yellow!75!black,title={#1},fonttitle=\bfseries,breakable}
\newtcolorbox{verification}[1][Verification]{colback=orange!5!white,colframe=orange!75!black,fonttitle=\bfseries,title=#1}
\newtcolorbox{explanationBox}[1][Explanation]{colback=purple!5!white,colframe=purple!75!black,fonttitle=\bfseries,title=#1}
\newtcolorbox{interpretationBox}[1][Interpretation]{colback=cyan!5!white,colframe=cyan!75!black,fonttitle=\bfseries,title=#1}
\newtcolorbox{explanation}[1][Explanation]{colback=purple!5!white,colframe=purple!75!black,fonttitle=\bfseries,title=#1,breakable}
\newtcolorbox{interpretation}[1][Interpretation]{colback=cyan!5!white,colframe=cyan!75!black,fonttitle=\bfseries,title=#1,breakable}
\newtcolorbox{proof_step}[1][Proof Step]{colback=gray!5!white,colframe=gray!75!black,fonttitle=\bfseries,title=#1,breakable}
\newtcolorbox{experimental}[1][Experimental]{colback=teal!5!white,colframe=teal!75!black,fonttitle=\bfseries,title=#1,breakable}

\newtcolorbox{important}[1][Important]{colback=red!5!white,colframe=red!75!black,title={#1},fonttitle=\bfseries,breakable}
\newtcolorbox{warning}[1][Warning]{colback=orange!5!white,colframe=orange!75!black,title={#1},fonttitle=\bfseries,breakable}
\newtcolorbox{caution}[1][Caution]{colback=yellow!5!white,colframe=yellow!75!black,title={#1},fonttitle=\bfseries,breakable}
\newtcolorbox{highlight}[1][Highlight]{colback=yellow!10!white,colframe=yellow!75!black,title={#1},fonttitle=\bfseries,breakable}
\newtcolorbox{critical}[1][Critical]{colback=red!10!white,colframe=red!75!black,title={#1},fonttitle=\bfseries,breakable}

\newtcolorbox{analysis}[1][Analysis]{colback=blue!5!white,colframe=blue!75!black,title={#1},fonttitle=\bfseries,breakable}
\newtcolorbox{application}[1][Application]{colback=green!5!white,colframe=green!75!black,title={#1},fonttitle=\bfseries,breakable}
\newtcolorbox{experiment}[1][Experiment]{colback=cyan!5!white,colframe=cyan!75!black,title={#1},fonttitle=\bfseries,breakable}
\newtcolorbox{historical}[1][Historical]{colback=brown!5!white,colframe=brown!75!black,title={#1},fonttitle=\bfseries,breakable}
\newtcolorbox{numerical}[1][Numerical]{colback=gray!5!white,colframe=gray!75!black,title={#1},fonttitle=\bfseries,breakable}
\newtcolorbox{overview}[1][Overview]{colback=blue!5!white,colframe=blue!75!black,title={#1},fonttitle=\bfseries,breakable}
\newtcolorbox{speculation}[1][Speculation]{colback=purple!5!white,colframe=purple!75!black,title={#1},fonttitle=\bfseries,breakable}
\newtcolorbox{question}[1][Question]{colback=orange!5!white,colframe=orange!75!black,title={#1},fonttitle=\bfseries,breakable}
\newtcolorbox{method}[1][Method]{colback=teal!5!white,colframe=teal!75!black,title={#1},fonttitle=\bfseries,breakable}
\newtcolorbox{correct}[1][Correct]{colback=green!10!white,colframe=green!75!black,title={#1},fonttitle=\bfseries,breakable}
\newtcolorbox{units}[1][Units]{colback=gray!5!white,colframe=gray!75!black,title={#1},fonttitle=\bfseries,breakable}
\newtcolorbox{achievement}[1][Achievement]{colback=gold!5!white,colframe=orange!75!black,title={#1},fonttitle=\bfseries,breakable}
\newtcolorbox{equivalence}[1][Equivalence]{colback=cyan!5!white,colframe=cyan!75!black,title={#1},fonttitle=\bfseries,breakable}
\newtcolorbox{dimensional}[1][Dimensional Analysis]{colback=purple!5!white,colframe=purple!75!black,title={#1},fonttitle=\bfseries,breakable}

% === ADDITIONAL SIMPLE ENVIRONMENTS ===
\newenvironment{treatise}{\begin{quote}}{\end{quote}}
\newenvironment{gemeinsam}{\begin{quote}}{\end{quote}}
\newenvironment{vergleich}{\begin{quote}}{\end{quote}}
\newenvironment{vorteil}{\begin{quote}}{\end{quote}}
\newenvironment{common}{\begin{quote}}{\end{quote}}
\newenvironment{comparison}{\begin{quote}}{\end{quote}}
\newenvironment{advantage}{\begin{quote}}{\end{quote}}
\newenvironment{quantum}{\begin{quote}}{\end{quote}}

% === LAYOUT SETTINGS ===
\raggedbottom
\usepackage{environ}
\let\oldtabular\tabular
\let\endoldtabular\endtabular

\newenvironment{scaledtable}[1][0.85]{%
	\begingroup\footnotesize\setlength{\LTleft}{0pt}\setlength{\LTright}{0pt}%
}{%
	\endgroup%
}

\newcommand{\widetable}[1]{\resizebox{\textwidth}{!}{#1}}

% === TABLE OF CONTENTS FORMATTING ===
\renewcommand{\cftsecfont}{\color{blue}}
\renewcommand{\cftsubsecfont}{\color{blue}}
\renewcommand{\cftsecpagefont}{\color{blue}}
\renewcommand{\cftsubsecpagefont}{\color{blue}}
\renewcommand{\cfttoctitlefont}{\huge\bfseries\color{blue}}

% === DEFAULT HEADER AND FOOTER ===
\pagestyle{fancy}
\fancyhf{}
\fancyhead[L]{\textsc{T0 Theory}}
\fancyhead[R]{\textsc{J. Pascher}}
\fancyfoot[C]{\thepage}

% ==============================================================================
% End of Shared Preamble for English
% ==============================================================================
%   \begin{document}
%   ...
%   \end{document}
%
% ==============================================================================

% =============================================================================
% SECTION 1: Encoding and Language
% =============================================================================
\usepackage[utf8]{inputenc}
\usepackage[T1]{fontenc}
\usepackage[ngerman]{babel}
\usepackage{lmodern}

% =============================================================================
% SECTION 2: Page Geometry
% =============================================================================
\usepackage[a4paper, left=2.5cm, right=2.5cm, top=2.5cm, bottom=3.5cm]{geometry}
\setlength{\headheight}{15pt}

% =============================================================================
% SECTION 3: Mathematics and Physics
% =============================================================================
\usepackage{amsmath,amssymb,amsfonts,amsthm}
\usepackage{mathtools}
\usepackage{physics}
\usepackage{siunitx}
\sisetup{
    locale=US,
    group-separator={,},
    output-decimal-marker={.},
    per-mode=symbol
}

% =============================================================================
% SECTION 4: Graphics and Tables
% =============================================================================
\usepackage{graphicx}
\usepackage[table,xcdraw]{xcolor}
\usepackage{tikz}
\usetikzlibrary{arrows.meta,positioning,shapes.geometric,decorations.pathmorphing,patterns,shapes.arrows,intersections}
\usepackage{pgfplots}
\pgfplotsset{compat=1.18}
\usepackage[most]{tcolorbox}
\tcbuselibrary{breakable}
\usepackage{booktabs}
\usepackage{array}
\usepackage{longtable}
\usepackage{float}
\usepackage{adjustbox}
\usepackage{rotating}
\usepackage{tabularx}
\usepackage{makecell}
\usepackage{multirow}

% =============================================================================
% SECTION 5: Document Formatting
% =============================================================================
\usepackage{fancyhdr}
\renewcommand{\headrulewidth}{0.4pt}
\renewcommand{\footrulewidth}{0.4pt}
\usepackage{tocloft}
\usepackage{hyperref}
\hypersetup{
  colorlinks=true,
  linkcolor=black,
  citecolor=black,
  urlcolor=black,
  breaklinks=true,
  bookmarksnumbered=true,
  unicode=true
}
\usepackage{bookmark}
\usepackage{cleveref}

% Table of contents: only show chapters (not sections/subsections)
\setcounter{tocdepth}{3}  % Show sections, subsections, and subsubsections
\usepackage{microtype}
\usepackage{enumitem}
\usepackage{setspace}
\usepackage{ragged2e}
\usepackage{multicol}

% =============================================================================
% SECTION 6: Code and Algorithms
% =============================================================================
\usepackage{algorithm}
\usepackage{algorithmic}
\usepackage{listings}
\lstset{
  basicstyle=\ttfamily\footnotesize,
  breaklines=true,
  breakatwhitespace=true,
  columns=flexible,
  keepspaces=true,
  showstringspaces=false,
  frame=single,
  xleftmargin=0pt,
  xrightmargin=0pt
}
\usepackage{mdframed}

% =============================================================================
% SECTION 7: Additional Packages
% =============================================================================
\usepackage{pdflscape}
\usepackage{braket}
\usepackage{cancel}
\usepackage{caption}
\usepackage{csquotes}
\usepackage{gensymb}
\usepackage{hyphenat}
\usepackage{textcomp}
\usepackage{textgreek}
\usepackage{upgreek}
\usepackage{url}
\usepackage{slashed}
\usepackage{bm}
\usepackage{newunicodechar}

% =============================================================================
% SECTION 8: Citation Commands (Compatibility)
% =============================================================================
\providecommand{\citep}[1]{\cite{#1}}
\providecommand{\citet}[1]{\cite{#1}}

% =============================================================================
% SECTION 9: Colors
% =============================================================================
\definecolor{gold}{RGB}{255,215,0}
\definecolor{blue}{rgb}{0,0,1}
\definecolor{boxgray}{RGB}{240,240,240}
\definecolor{deepblue}{RGB}{0,0,127}
\definecolor{deepgreen}{RGB}{0,127,0}
\definecolor{deepred}{RGB}{191,0,0}
\definecolor{t0blue}{RGB}{33,150,243}
\definecolor{t0green}{RGB}{76,175,80}
\definecolor{t0orange}{RGB}{255,152,0}
\definecolor{t0purple}{RGB}{156,39,176}
\definecolor{t0red}{RGB}{244,67,54}
\definecolor{t0yellow}{RGB}{255,204,0}

% =============================================================================
% SECTION 10: Column Types
% =============================================================================
\newcolumntype{L}[1]{>{\raggedright\arraybackslash}p{#1}}
\newcolumntype{C}[1]{>{\centering\arraybackslash}p{#1}}

% =============================================================================
% SECTION 11: Unicode Character Mappings
% =============================================================================
\newunicodechar{ħ}{$\hbar$}
\newunicodechar{↔}{$\leftrightarrow$}
\newunicodechar{⇐}{$\Leftarrow$}
\newunicodechar{⇒}{$\Rightarrow$}
\newunicodechar{⇔}{$\Leftrightarrow$}
\newunicodechar{∂}{$\partial$}
\newunicodechar{∅}{$\emptyset$}
\newunicodechar{∇}{$\nabla$}
\newunicodechar{∈}{$\in$}
\newunicodechar{∉}{$\notin$}
\newunicodechar{∏}{$\prod$}
\newunicodechar{∑}{$\sum$}
% Note: √ is mapped to an empty sqrt; use \sqrt{x} for proper usage
\newunicodechar{√}{\ensuremath{\sqrt{}}}
\newunicodechar{∝}{$\propto$}
\newunicodechar{∞}{$\infty$}
\newunicodechar{∩}{$\cap$}
\newunicodechar{∪}{$\cup$}
\newunicodechar{∫}{$\int$}
\newunicodechar{≈}{$\approx$}
\newunicodechar{≠}{$\neq$}
\newunicodechar{≤}{$\leq$}
\newunicodechar{≥}{$\geq$}
\newunicodechar{ξ}{\ensuremath{\xi}}
\newunicodechar{μ}{\ensuremath{\mu}}
\newunicodechar{ψ}{\ensuremath{\psi}}
\newunicodechar{φ}{\ensuremath{\phi}}
\newunicodechar{π}{\ensuremath{\pi}}
\newunicodechar{λ}{\ensuremath{\lambda}}
\newunicodechar{Δ}{\ensuremath{\Delta}}

% =============================================================================
% SECTION 12: Hyperref Settings
% =============================================================================
\hypersetup{
    colorlinks=true,
    linkcolor=blue,
    citecolor=blue,
    urlcolor=blue,
    breaklinks=true,
    bookmarksnumbered=true,
    pdfstartview=FitH
}

% =============================================================================
% SECTION 13: Theorem Environments (English)
% =============================================================================
\theoremstyle{plain}
\newtheorem{theorem}{Theorem}[section]
\newtheorem{lemma}[theorem]{Lemma}
\newtheorem{proposition}[theorem]{Proposition}
\newtheorem{corollary}[theorem]{Corollary}

\theoremstyle{definition}
\newtheorem{definition}[theorem]{Definition}
\newtheorem{example}[theorem]{Example}
\newtheorem{insight}[theorem]{Insight}
\newtheorem{discovery}[theorem]{Discovery}
% \newtheorem{erkenntnis}[theorem]{Insight}  % Commented out - conflicts with tcolorbox environment below

\theoremstyle{remark}
\newtheorem{remark}[theorem]{Remark}
\newtheorem{axiom}{Axiom}
\newtheorem{principle}{Principle}
\newtheorem{bemerkung}[theorem]{Remark}
\newtheorem{warnung}[theorem]{Warning}

% =============================================================================
% SECTION 14: T0-Specific Commands
% =============================================================================

% --- Core T0 Fields ---
\newcommand{\Tfield}{T(x,t)}
\providecommand{\Tfieldt}{T(\vec{x},t)}
\newcommand{\Efield}{E(x,t)}
\newcommand{\mfield}{m(x,t)}
\providecommand{\vecx}{\vec{x}}

% --- Lagrangian ---
\newcommand{\Lag}{\mathcal{L}}
\newcommand{\calL}{\mathcal{L}}

% --- Greek Letters and Constants ---
\newcommand{\alphaem}{\alpha}
\newcommand{\betaT}{\beta_T}
\newcommand{\xiT}{\xi}
\newcommand{\xipar}{\xi}

% --- Energy and Planck Units ---
\newcommand{\Ezero}{E_0}
\newcommand{\EPlanck}{E_{\text{Pl}}}
\newcommand{\Mpl}{M_{\text{Pl}}}
\newcommand{\mP}{m_{\text{P}}}
\newcommand{\lP}{\ell_{\text{P}}}
\newcommand{\tP}{t_{\text{P}}}
\newcommand{\LPlanck}{\ell_{\text{Pl}}}
\newcommand{\TPlanck}{t_{\text{Pl}}}

% --- Coupling Constants ---
\newcommand{\Gnat}{G_{\text{nat}}}
\newcommand{\alphaEM}{\alpha_{\text{EM}}}
\newcommand{\alphaSI}{\alpha_{\text{SI}}}
\newcommand{\Hubble}{H_0}
\newcommand{\LCDM}{\Lambda\text{CDM}}
\newcommand{\natunits}{(nat. units)}

% --- T0 Model Parameters ---
\newcommand{\xigeom}{\xi_{\mathrm{geom}}}
\newcommand{\rzero}{r_{0}}
\newcommand{\xirat}{\xi_{\mathrm{rat}}}
\newcommand{\tzero}{t_{0}}
\newcommand{\Lambdat}{\Lambda_{\mathrm{t}}}
\newcommand{\EP}{E_{\mathrm{P}}}
\newcommand{\Emu}{E_{\mu}}
\newcommand{\Ee}{E_{e}}
\newcommand{\Etau}{E_{\tau}}
\newcommand{\alphafine}{\alpha_{\mathrm{fine}}}
\newcommand{\alphal}{\alpha_{\ell}}
\newcommand{\Lzero}{\ell_{0}}
\newcommand{\Lp}{\ell_{\mathrm{P}}}

% --- Additional T0 Commands ---
\newcommand{\Kfrak}{K_{\text{frak}}}
\newcommand{\Dfrak}{D_{\text{frak}}}
\newcommand{\betapar}{\beta_T}
\newcommand{\alphapar}{\alpha}
\newcommand{\deltafield}{\delta \phi}
\newcommand{\deltam}{\delta m}
\newcommand{\deltaE}{\delta E}
\newcommand{\Exi}{E_{\xi}}
\newcommand{\Lxi}{\ell_{\xi}}
\newcommand{\rhoCMB}{\rho_{\text{CMB}}}
\newcommand{\rhoCasimir}{\rho_{\text{Casimir}}}
\newcommand{\Leff}{L_{\text{eff}}}
\newcommand{\CQCD}{C_{\mathrm{QCD}}}
\newcommand{\Kspec}{K_{\mathrm{spec}}}
\newcommand{\Tzero}{\ensuremath{T_0}}
\newcommand{\Eabs}{E_{\text{abs}}}
\newcommand{\taupar}{\tau}

% --- Provided Commands (may be redefined elsewhere) ---
\providecommand{\xiconst}{\xi_{\text{const}}}
\providecommand{\DhiggsT}{D_{\text{Higgs-T}}}
\providecommand{\rhoE}{\rho_{E}}
\providecommand{\Echar}{E_{\text{char}}}
\providecommand{\kfrac}{k_{\text{frac}}}
\providecommand{\alphaEMSI}{\alpha_{\text{EM,SI}}}
\providecommand{\alphaEMnat}{\alpha_{\text{EM,nat}}}
\providecommand{\betaTSI}{\beta_{T,\text{SI}}}
\providecommand{\betaTnat}{\beta_{T,\text{nat}}}
\providecommand{\Gsi}{G_{\text{SI}}}
\providecommand{\xiparSI}{\xi_{\text{SI}}}
\providecommand{\xiparnat}{\xi_{\text{nat}}}
\providecommand{\meff}{m_{\text{eff}}}
\providecommand{\Tzerot}{T_{0}(t)}
\providecommand{\mzerot}{m_{0}(t)}
\providecommand{\Ezeroabs}{E_{0,\text{abs}}}
\providecommand{\Epar}{E_{\text{par}}}
\providecommand{\Lnat}{\ell_{\text{nat}}}
\providecommand{\Tnat}{T_{\text{nat}}}
\providecommand{\xifrak}{\xi_{\text{frac}}}
\providecommand{\Tfrak}{T_{\text{frac}}}
\providecommand{\mfrak}{m_{\text{frac}}}
\providecommand{\Dfrac}{D_{\text{frac}}}
\providecommand{\EphotSI}{E_{\gamma,\text{SI}}}
\providecommand{\EphotNat}{E_{\gamma,\text{nat}}}
\providecommand{\Eabsint}{E_{\text{abs,int}}}
\providecommand{\mphoton}{m_{\gamma}}
\providecommand{\Evis}{E_{\text{vis}}}
\providecommand{\Cto}{C_{T0}}
\providecommand{\mytimes}{\times}
\providecommand{\lambdah}{\lambda_h}
\providecommand{\checkmarkx}{\checkmark}
\providecommand{\Enorm}{E_{\text{norm}}}
\providecommand{\Tobs}{T_{\text{obs}}}
\providecommand{\mobs}{m_{\text{obs}}}
\providecommand{\Eobs}{E_{\text{obs}}}
\providecommand{\Lobs}{\ell_{\text{obs}}}
\providecommand{\xobs}{\xi_{\text{obs}}}
\providecommand{\calE}{\mathcal{E}}
\providecommand{\calT}{\mathcal{T}}
\providecommand{\calM}{\mathcal{M}}
\providecommand{\alphag}{\alpha_g}
\providecommand{\Tmax}{T_{\text{max}}}
\providecommand{\mmin}{m_{\text{min}}}
\providecommand{\Lmax}{\ell_{\text{max}}}
\providecommand{\Emin}{E_{\text{min}}}
\providecommand{\Geff}{G_{\text{eff}}}
\providecommand{\rhoeff}{\rho_{\text{eff}}}
\providecommand{\xieff}{\xi_{\text{eff}}}
\providecommand{\Teff}{T_{\text{eff}}}
\providecommand{\hPlanck}{h}
\providecommand{\kB}{k_B}
\providecommand{\muB}{\mu_B}
\providecommand{\lambdaC}{\lambda_C}
\providecommand{\omegaP}{\omega_P}
\providecommand{\rhoP}{\rho_P}
\providecommand{\Tref}{T_{\text{ref}}}
\providecommand{\Eref}{E_{\text{ref}}}
\providecommand{\mref}{m_{\text{ref}}}
\providecommand{\Lref}{\ell_{\text{ref}}}
\providecommand{\xikonst}{\xi_0}
\providecommand{\Phiphoton}{\Phi_{\gamma}}
\providecommand{\etavis}{\eta_{\text{vis}}}
\providecommand{\pichar}{\pi}
\providecommand{\primrel}{\mathcal{P}_{\text{rel}}}
\providecommand{\warningx}{\textcolor{orange}{\textbf{!}}}
\providecommand{\phiT}{\phi_T}
\providecommand{\Lorentz}{\Lambda}
\providecommand{\Cconv}{C_{\text{conv}}}
\providecommand{\Df}{\Delta f}
\providecommand{\lambdazero}{\lambda_0}
\providecommand{\myapprox}{\approx}
\providecommand{\checked}{\checkmark}
\providecommand{\alphaWSI}{\alpha_W^{\text{SI}}}
\providecommand{\alphaWnat}{\alpha_W^{\text{nat}}}
\providecommand{\vect}[1]{\vec{#1}}
\providecommand{\Rzero}{R_0}
\providecommand{\Riem}{\mathcal{R}}
\providecommand{\nuzero}{\nu_0}
\providecommand{\mypi}{\pi}

% =============================================================================
% SECTION 15: tcolorbox Styles and Environments
% =============================================================================

% --- Predefined Styles ---
\tcbset{
    keyresult/.style={
        colback=blue!5!white,
        colframe=blue!75!black,
        title=Key Result,
        fonttitle=\bfseries
    },
    foundation/.style={
        colback=green!5!white,
        colframe=green!75!black,
        title=Foundation,
        fonttitle=\bfseries
    },
    alternative/.style={
        colback=orange!5!white,
        colframe=orange!75!black,
        title=Alternative,
        fonttitle=\bfseries
    },
    warningbox/.style={
        colback=red!5!white,
        colframe=red!75!black,
        title=Warning,
        fonttitle=\bfseries
    }
}

% --- Core Environments ---
\newtcolorbox{keyresultbox}[1][]{colback=blue!5!white,colframe=blue!75!black,fonttitle=\bfseries,title={#1},breakable}
\newtcolorbox{keyresult}[1][Key Result]{colback=blue!5!white,colframe=blue!75!black,fonttitle=\bfseries,title={#1},breakable}
\newtcolorbox{foundationbox}[1][]{colback=green!5!white,colframe=green!75!black,fonttitle=\bfseries,title={#1},breakable}
\newtcolorbox{foundation}[1][Foundation]{colback=green!5!white,colframe=green!75!black,fonttitle=\bfseries,title={#1},breakable}
\newtcolorbox{alternativebox}[1][]{colback=orange!5!white,colframe=orange!75!black,fonttitle=\bfseries,title={#1},breakable}
\newtcolorbox{warningboxenv}[1][]{colback=red!5!white,colframe=red!75!black,fonttitle=\bfseries,title={#1},breakable}

% --- Formula Environments ---
\newtcolorbox{fundamental}[1][]{
    colback=boxgray,
    colframe=t0blue,
    fonttitle=\bfseries,
    title=#1,
    sharp corners,
    boxrule=2pt
}

\newtcolorbox{newperspective}[1][]{
    colback=red!5!white,
    colframe=t0red,
    fonttitle=\bfseries,
    title=#1,
    sharp corners,
    boxrule=2pt
}

\newtcolorbox{formula}[1][]{
    colback=blue!5!white,
    colframe=blue!75!black,
    fonttitle=\bfseries,
    title=#1
}

\newtcolorbox{result}[1][]{
    colback=green!5!white,
    colframe=green!75!black,
    fonttitle=\bfseries,
    title=#1
}

\newtcolorbox{derivation}[1][]{
    colback=green!5!white,
    colframe=green!75!black,
    title=#1,
    fonttitle=\bfseries,
    breakable
}

\newtcolorbox{summary}[1][]{
    colback=gray!10!white,
    colframe=gray!75!black,
    title=#1,
    fonttitle=\bfseries,
    breakable
}

\newtcolorbox{comparison}[1][]{
    colback=purple!5!white,
    colframe=purple!75!black,
    title=#1,
    fonttitle=\bfseries,
    breakable
}

\newtcolorbox{relation}[1][]{
    colback=cyan!5!white,
    colframe=cyan!75!black,
    title=#1,
    fonttitle=\bfseries,
    breakable
}

\newtcolorbox{principleBox}[1][]{
    colback=yellow!5!white,
    colframe=yellow!75!black,
    title=#1,
    fonttitle=\bfseries,
    breakable
}

% --- Insight and Discovery Environments ---
\newtcolorbox{insightBox}[1][]{colback=blue!5,colframe=t0blue,title={#1},fonttitle=\bfseries,breakable}
\newtcolorbox{discoveryBox}[1][]{colback=green!5,colframe=t0green,title={#1},fonttitle=\bfseries,breakable}
\newtcolorbox{revelation}[1][]{colback=red!5,colframe=t0red,title={#1},fonttitle=\bfseries,breakable}
\newtcolorbox{keypoint}[1][]{colback=blue!5,colframe=t0blue,title={#1},fonttitle=\bfseries,breakable}
\newtcolorbox{evidence}[1][]{colback=green!5,colframe=t0green,title={#1},fonttitle=\bfseries,breakable}
\newtcolorbox{conclusionBox}[1][]{colback=gray!5,colframe=gray,title={#1},fonttitle=\bfseries,breakable}
\newtcolorbox{significance}[1][]{colback=yellow!5,colframe=orange,title={#1},fonttitle=\bfseries,breakable}
\newtcolorbox{philosophical}[1][]{colback=purple!5,colframe=purple,title={#1},fonttitle=\bfseries,breakable}
\newtcolorbox{implicationBox}[1][]{colback=cyan!5,colframe=cyan,title={#1},fonttitle=\bfseries,breakable}
\newtcolorbox{perspectiveBox}[1][]{colback=blue!5,colframe=t0blue,title={#1},fonttitle=\bfseries,breakable}
\newtcolorbox{revolutionary}[1][]{colback=red!5,colframe=t0red,title={#1},fonttitle=\bfseries,breakable}

% --- Technical Environments ---
\newtcolorbox{technical}[1][]{colback=gray!5,colframe=gray!75!black,title={#1},fonttitle=\bfseries,breakable}
\newtcolorbox{technicalBox}[1][]{colback=gray!5,colframe=gray!75!black,title={#1},fonttitle=\bfseries,breakable}
\newtcolorbox{notationBox}[1][]{colback=yellow!5,colframe=yellow!75!black,title={#1},fonttitle=\bfseries,breakable}
\newtcolorbox{verification}[1][]{colback=orange!5!white,colframe=orange!75!black,fonttitle=\bfseries,title=#1}
\newtcolorbox{explanationBox}[1][]{colback=purple!5!white,colframe=purple!75!black,fonttitle=\bfseries,title=#1}
\newtcolorbox{interpretationBox}[1][]{colback=cyan!5!white,colframe=cyan!75!black,fonttitle=\bfseries,title=#1}
\newtcolorbox{explanation}[1][]{colback=purple!5!white,colframe=purple!75!black,fonttitle=\bfseries,title=#1,breakable}
\newtcolorbox{interpretation}[1][]{colback=cyan!5!white,colframe=cyan!75!black,fonttitle=\bfseries,title=#1,breakable}
\newtcolorbox{proof_step}[1][]{colback=gray!5!white,colframe=gray!75!black,fonttitle=\bfseries,title=#1,breakable}
\newtcolorbox{experimental}[1][]{colback=teal!5!white,colframe=teal!75!black,fonttitle=\bfseries,title=#1,breakable}

% --- Warning and Alert Environments ---
\newtcolorbox{important}[1][]{colback=red!5!white,colframe=red!75!black,title={#1},fonttitle=\bfseries,breakable}
\newtcolorbox{warning}[1][]{colback=orange!5!white,colframe=orange!75!black,title={#1},fonttitle=\bfseries,breakable}
\newtcolorbox{caution}[1][]{colback=yellow!5!white,colframe=yellow!75!black,title={#1},fonttitle=\bfseries,breakable}
\newtcolorbox{highlight}[1][]{colback=yellow!10!white,colframe=yellow!75!black,title={#1},fonttitle=\bfseries,breakable}

% --- Additional German-specific Environments for Matsas documents ---
\newtcolorbox{literatur}[1][Literatur]{colback=blue!5!white,colframe=blue!75!black,title={#1},fonttitle=\bfseries,breakable}
\newtcolorbox{zusammenfassung}[1][Zusammenfassung]{colback=green!5!white,colframe=green!75!black,title={#1},fonttitle=\bfseries,breakable}
\newtcolorbox{frage}[1][Frage]{colback=orange!5!white,colframe=orange!75!black,title={#1},fonttitle=\bfseries,breakable}
\newtcolorbox{erkenntnis}[1][Erkenntnis]{colback=purple!5!white,colframe=purple!75!black,title={#1},fonttitle=\bfseries,breakable}
\newtcolorbox{critical}[1][]{colback=red!10!white,colframe=red!75!black,title={#1},fonttitle=\bfseries,breakable}

% --- Analysis and Application Environments ---
\newtcolorbox{analysis}[1][]{colback=blue!5!white,colframe=blue!75!black,title={#1},fonttitle=\bfseries,breakable}
\newtcolorbox{application}[1][]{colback=green!5!white,colframe=green!75!black,title={#1},fonttitle=\bfseries,breakable}
\newtcolorbox{experiment}[1][]{colback=cyan!5!white,colframe=cyan!75!black,title={#1},fonttitle=\bfseries,breakable}
\newtcolorbox{historical}[1][]{colback=brown!5!white,colframe=brown!75!black,title={#1},fonttitle=\bfseries,breakable}
\newtcolorbox{numerical}[1][]{colback=gray!5!white,colframe=gray!75!black,title={#1},fonttitle=\bfseries,breakable}
\newtcolorbox{overview}[1][]{colback=blue!5!white,colframe=blue!75!black,title={#1},fonttitle=\bfseries,breakable}
\newtcolorbox{speculation}[1][]{colback=purple!5!white,colframe=purple!75!black,title={#1},fonttitle=\bfseries,breakable}
\newtcolorbox{question}[1][]{colback=orange!5!white,colframe=orange!75!black,title={#1},fonttitle=\bfseries,breakable}
\newtcolorbox{method}[1][]{colback=teal!5!white,colframe=teal!75!black,title={#1},fonttitle=\bfseries,breakable}
\newtcolorbox{correct}[1][]{colback=green!10!white,colframe=green!75!black,title={#1},fonttitle=\bfseries,breakable}
\newtcolorbox{units}[1][]{colback=gray!5!white,colframe=gray!75!black,title={#1},fonttitle=\bfseries,breakable}
\newtcolorbox{achievement}[1][]{colback=gold!5!white,colframe=orange!75!black,title={#1},fonttitle=\bfseries,breakable}
\newtcolorbox{equivalence}[1][]{colback=cyan!5!white,colframe=cyan!75!black,title={#1},fonttitle=\bfseries,breakable}
\newtcolorbox{dimensional}[1][]{colback=purple!5!white,colframe=purple!75!black,title={#1},fonttitle=\bfseries,breakable}

% --- Physics-specific Environments ---
\newtcolorbox{photon}[1][]{colback=yellow!5!white,colframe=yellow!75!black,title={#1},fonttitle=\bfseries,breakable}
\newtcolorbox{neutrino}[1][]{colback=blue!5!white,colframe=blue!75!black,title={#1},fonttitle=\bfseries,breakable}
\newtcolorbox{revolution}[1][]{colback=red!5!white,colframe=red!75!black,title={#1},fonttitle=\bfseries,breakable}
\newtcolorbox{t0box}[1][]{colback=blue!5!white,colframe=t0blue,title={#1},fonttitle=\bfseries,breakable}
\newtcolorbox{documentbox}[1][]{colback=gray!5!white,colframe=gray!75!black,title={#1},fonttitle=\bfseries,breakable}
\newtcolorbox{sibox}[1][]{colback=green!5!white,colframe=green!75!black,title={#1},fonttitle=\bfseries,breakable}
\newtcolorbox{smbox}[1][]{colback=blue!5!white,colframe=blue!75!black,title={#1},fonttitle=\bfseries,breakable}
\newtcolorbox{pvbox}[1][]{colback=purple!5!white,colframe=purple!75!black,title={#1},fonttitle=\bfseries,breakable}
\newtcolorbox{koidebox}[1][]{colback=orange!5!white,colframe=orange!75!black,title={#1},fonttitle=\bfseries,breakable}

% --- German Compatibility Environments ---
\newtcolorbox{formel}[1][]{colback=blue!5!white,colframe=blue!75!black,title={#1},fonttitle=\bfseries,breakable}
\newtcolorbox{schluessel}[1][]{colback=blue!5!white,colframe=blue!75!black,title={#1},fonttitle=\bfseries,breakable}
\newtcolorbox{wichtig}[1][]{colback=red!5!white,colframe=red!75!black,title={#1},fonttitle=\bfseries,breakable}
\newtcolorbox{vorsicht}[1][]{colback=orange!5!white,colframe=orange!75!black,title={#1},fonttitle=\bfseries,breakable}
\newtcolorbox{revolutionaer}[1][]{colback=red!5!white,colframe=red!75!black,title={#1},fonttitle=\bfseries,breakable}
\newtcolorbox{numerisch}[1][]{colback=gray!5!white,colframe=gray!75!black,title={#1},fonttitle=\bfseries,breakable}
\newtcolorbox{experimentell}[1][]{colback=cyan!5!white,colframe=cyan!75!black,title={#1},fonttitle=\bfseries,breakable}
\newtcolorbox{anwendung}[1][]{colback=green!5!white,colframe=green!75!black,title={#1},fonttitle=\bfseries,breakable}
\newtcolorbox{alternative}[1][]{colback=orange!5!white,colframe=orange!75!black,title={#1},fonttitle=\bfseries,breakable}
\newtcolorbox{beziehung}[1][]{colback=cyan!5!white,colframe=cyan!75!black,title={#1},fonttitle=\bfseries,breakable}
\newtcolorbox{folgerung}[1][]{colback=green!5!white,colframe=green!75!black,title={#1},fonttitle=\bfseries,breakable}
\newtcolorbox{abhandlung}[1][]{colback=gray!5!white,colframe=gray!75!black,title={#1},fonttitle=\bfseries,breakable}
\newtcolorbox{prinzipBox}[1][]{colback=blue!5!white,colframe=blue!75!black,title={#1},fonttitle=\bfseries,breakable}
\newtcolorbox{prinzip}[1][]{colback=blue!5!white,colframe=blue!75!black,title={#1},fonttitle=\bfseries,breakable}
\newtcolorbox{beweis}[1][]{colback=gray!5!white,colframe=gray!75!black,title={#1},fonttitle=\bfseries,breakable}
\newtcolorbox{key}[2][]{colback=blue!5!white,colframe=blue!75!black,title={#2},fonttitle=\bfseries,breakable}
\newtcolorbox{category}[1][]{colback=purple!5!white,colframe=purple!75!black,title={#1},fonttitle=\bfseries,breakable}

% =============================================================================
% SECTION 16: Additional Simple Environments
% =============================================================================
\newenvironment{treatise}{\begin{quote}}{\end{quote}}
\newenvironment{gemeinsam}{\begin{quote}}{\end{quote}}
\newenvironment{vergleich}{\begin{quote}}{\end{quote}}
\newenvironment{vorteil}{\begin{quote}}{\end{quote}}
\newenvironment{quantum}{\begin{quote}}{\end{quote}}

% =============================================================================
% SECTION 17: Layout Settings (Kindle-compatible)
% =============================================================================
\sloppy  % Allow more flexible line breaking
\hfuzz=65pt  % Suppress overfull warnings up to 65pt (Kindle compatibility)
\vfuzz=65pt  
\tolerance=9999  % High tolerance for bad line breaks
\emergencystretch=3em  % Extra stretch to avoid overfull boxes
\hbadness=10000  % Suppress underfull box warnings
\raggedbottom

% Environment for wide tables/longtables that need scaling
\newenvironment{scaledtable}[1][0.85]{%
  \begingroup\footnotesize\setlength{\LTleft}{0pt}\setlength{\LTright}{0pt}%
}{%
  \endgroup%
}

% Command for inline table scaling
\newcommand{\widetable}[1]{\resizebox{\textwidth}{!}{#1}}

% =============================================================================
% SECTION 18: Table of Contents Formatting
% =============================================================================
\renewcommand{\cftsecfont}{\color{blue}}
\renewcommand{\cftsubsecfont}{\color{blue}}
\renewcommand{\cftsecpagefont}{\color{blue}}
\renewcommand{\cftsubsecpagefont}{\color{blue}}
\renewcommand{\cfttoctitlefont}{\huge\bfseries\color{blue}}

% =============================================================================
% SECTION 19: Default Header and Footer
% =============================================================================
\pagestyle{fancy}
\fancyhf{}
\fancyhead[L]{\textsc{T0 Theory}}
\fancyhead[R]{\textsc{J. Pascher}}
\fancyfoot[C]{\thepage}

% ==============================================================================
% End of Shared Preamble
% ==============================================================================


\begin{document}

\title{Tabellen aus 010\_T0\_Energie\_De.tex}
\maketitle

\tableofcontents
\newpage

\section*{Tabelle 1 aus 010\_T0\_Energie\_De.tex}
\begin{adjustbox}{max width=\textwidth}
\begin{tabular}{lccc}
			\toprule
			\textbf{Teilchen} & \textbf{Energie} & \textbf{$\rzero$ (in $\lP$-Einheiten)} & \textbf{$\xi = \lP/\rzero$} \\
			\midrule
			Elektron & $E_e = 0,511$ MeV & $r_{0,e} = 1,02 \times 10^{-3} \lP$ & $9,8 \times 10^{2}$ \\
			Myon & $E_\mu = 105,658$ MeV & $r_{0,\mu} = 2,1 \times 10^{-1} \lP$ & $4,7$ \\
			Proton & $E_p = 938$ MeV & $r_{0,p} = 1,9 \lP$ & $0,53$ \\
			Planck & $E_P = 1,22 \times 10^{19}$ GeV & $r_{0,P} = 2\lP$ & $0,5$ \\
			\bottomrule
		\end{tabular}
\end{adjustbox}

\bigskip
\clearpage

\section*{Tabelle 2 aus 010\_T0\_Energie\_De.tex}
\begin{adjustbox}{max width=\textwidth}
\begin{tabular}{lccc}
			\toprule
			\textbf{Teilchen} & \textbf{Energie (GeV)} & \textbf{$\rzero/\lP$} & \textbf{$\xi = \lP/\rzero$} \\
			\midrule
			Elektron & $E_e = 0,511 \times 10^{-3}$ & $1,02 \times 10^{-3}$ & $9,8 \times 10^{2}$ \\
			Myon & $E_\mu = 0,106$ & $2,12 \times 10^{-1}$ & $4,7 \times 10^{0}$ \\
			Proton & $E_p = 0,938$ & $1,88 \times 10^{0}$ & $5,3 \times 10^{-1}$ \\
			Higgs & $E_h = 125$ & $2,50 \times 10^{2}$ & $4,0 \times 10^{-3}$ \\
			Top-Quark & $E_t = 173$ & $3,46 \times 10^{2}$ & $2,9 \times 10^{-3}$ \\
			\bottomrule
		\end{tabular}
\end{adjustbox}

\bigskip
\clearpage

\section*{Tabelle 3 aus 010\_T0\_Energie\_De.tex}
\begin{adjustbox}{max width=\textwidth}
\begin{tabular}{lccc}
		\toprule
		\textbf{Teilchen} & \textbf{T0-Vorhersage} & \textbf{Experiment} & \textbf{Genauigkeit} \\
		\midrule
		Elektron & 0,512 MeV & 0,511 MeV & 99,95\% \\
		Myon & 105,7 MeV & 105,658 MeV & 99,97\% \\
		Tau & 1778 MeV & 1776,86 MeV & 99,96\% \\
		Up-Quark & 2,5 MeV & 2,2 MeV & 88\%\textsuperscript{*} \\
		Down-Quark & 4,7 MeV & 4,7 MeV & 100\% \\
		Charm-Quark & 1,28 GeV & 1,27 GeV & 99,9\% \\
		\midrule
		\textbf{Durchschnitt} & & & \textbf{97,9\%} \\
		\bottomrule
	\end{tabular}
\end{adjustbox}

\bigskip
\clearpage

\section*{Tabelle 4 aus 010\_T0\_Energie\_De.tex}
\begin{adjustbox}{max width=\textwidth}
\begin{tabular}{@{}lccc@{}}
		\toprule
		\textbf{Theorie} & \textbf{Vorhersage} & \textbf{Abweichung} & \textbf{Signifikanz} \\
		\midrule
		Experiment & $251(59) \times 10^{-11}$ & - & Referenz \\
		Standardmodell & $0(43) \times 10^{-11}$ & $251 \times 10^{-11}$ & $4,2\sigma$ \\
		T0-Modell & $245(12) \times 10^{-11}$ & $6 \times 10^{-11}$ & $0,10\sigma$ \\
		\bottomrule
	\end{tabular}
\end{adjustbox}

\bigskip
\clearpage

\section*{Tabelle 5 aus 010\_T0\_Energie\_De.tex}
\begin{adjustbox}{max width=\textwidth}
\begin{tabular}{lcc}
		\toprule
		\textbf{Skala} & \textbf{Energie (GeV)} & \textbf{Physik} \\
		\midrule
		Planck-Energie & $1,22 \times 10^{19}$ & Quantengravitation \\
		Elektroschwache Skala & $246$ & Higgs-VEV \\
		QCD-Skala & $0,2$ & Confinement \\
		T0-Skala & $10^{-4}$ & Feldkopplung \\
		Atomare Skala & $10^{-5}$ & Bindungsenergien \\
		\bottomrule
	\end{tabular}
\end{adjustbox}

\bigskip
\clearpage

\section*{Tabelle 6 aus 010\_T0\_Energie\_De.tex}
\begin{adjustbox}{max width=\textwidth}
\begin{tabular}{lcc}
			\toprule
			\textbf{Aspekt} & \textbf{Standardmodell} & \textbf{T0-Modell} \\
			\midrule
			Fundamentale Felder & 20+ verschiedene & 1 universelles Energiefeld \\
			Freie Parameter & 19+ empirische & 0 freie \\
			Kopplungskonstanten & Multiple unabhängige & 1 geometrische Konstante \\
			Teilchenmassen & Individuelle Werte & Energieskalenverhältnisse \\
			Kraftstärken & Separate Kopplungen & Vereinheitlicht durch $\xi$ \\
			Empirische Eingaben & Erforderlich für jede & Keine erforderlich \\
			Vorhersagekraft & Begrenzt & Universell \\
			\bottomrule
		\end{tabular}
\end{adjustbox}

\bigskip
\clearpage

\section*{Tabelle 7 aus 010\_T0\_Energie\_De.tex}
\begin{adjustbox}{max width=\textwidth}
\begin{tabular}{lccc}
		\toprule
		\textbf{Observable} & \textbf{T0-Vorhersage} & \textbf{Status} & \textbf{Präzision} \\
		\midrule
		Myon g-2 & $245 \times 10^{-11}$ & Bestätigt & $0.10\sigma$ \\
		Elektron g-2 & $1.15 \times 10^{-12}$ & Testbar & $10^{-13}$ \\
		Tau g-2 & $257 \times 10^{-7}$ & Zukunft & $10^{-9}$ \\
		Feinstrukturkonstante & $\alpha = 1$ (natürl. Einheiten) & Bestätigt & $10^{-10}$ \\
		Schwache Kopplung & $g_W^2/4\pi = \sqrt{\xi}$ & Testbar & $10^{-3}$ \\
		Starke Kopplung & $\alpha_s = \xi^{-1/3}$ & Testbar & $10^{-2}$ \\
		\bottomrule
	\end{tabular}
\end{adjustbox}

\bigskip
\clearpage

\section*{Tabelle 8 aus 010\_T0\_Energie\_De.tex}
\begin{adjustbox}{max width=\textwidth}
\begin{tabular}{lccc}
			\toprule
			\textbf{Skala} & \textbf{Energie (GeV)} & \textbf{T0-Verhältnis} & \textbf{Physik-Domäne} \\
			\midrule
			Planck & $10^{19}$ & $1$ & Quantengravitation \\
			T0-Teilchen & $10^{15}$ & $10^{-4}$ & Labor-zugänglich \\
			Elektroschwach & $10^{2}$ & $10^{-17}$ & Eichvereinigung \\
			QCD & $10^{-1}$ & $10^{-20}$ & Starke Wechselwirkungen \\
			Atomar & $10^{-9}$ & $10^{-28}$ & Elektromagnetische Bindung \\
			\bottomrule
		\end{tabular}
\end{adjustbox}

\bigskip
\clearpage

\section*{Tabelle 9 aus 010\_T0\_Energie\_De.tex}
\begin{adjustbox}{max width=\textwidth}
\begin{tabular}{|l|c|c|c|c|}
			\hline
			\textbf{Gleichung} & \textbf{Skala} & \textbf{Linke Seite} & \textbf{Rechte Seite} & \textbf{Status} \\
			\hline
			Teilchen g-2 & $\xi$ & $[a_\mu] = [1]$ & $[\xi/2\pi] = [1]$ & \checkmark \\
			Feldgleichung & Alle Skalen & $[\nabla^2 E] = [E^3]$ & $[G\rho E] = [E^3]$ & \checkmark \\
			Lagrange-Funktion & Alle Skalen & $[\mathcal{L}] = [E^4]$ & $[\xi(\partial E)^2] = [E^4]$ & \checkmark \\
			\hline
		\end{tabular}
\end{adjustbox}

\bigskip
\clearpage

\section*{Tabelle 10 aus 010\_T0\_Energie\_De.tex}
\begin{adjustbox}{max width=\textwidth}
\begin{tabular}{lcc}
			\toprule
			\textbf{Theorie} & \textbf{Freie Parameter} & \textbf{Vorhersagekraft} \\
			\midrule
			Standardmodell & 19+ empirische & Begrenzt \\
			Standardmodell + ART & 25+ empirische & Fragmentiert \\
			String-Theorie & $\sim 10^{500}$ Vakua & Unbestimmt \\
			T0-Modell & 0 freie & Universell \\
			\bottomrule
		\end{tabular}
\end{adjustbox}

\bigskip
\clearpage

\section*{Tabelle 11 aus 010\_T0\_Energie\_De.tex}
\begin{longtable}{|c|l|l|}
		\hline
		\textbf{Symbol} & \textbf{Bedeutung} & \textbf{Dimension} \\
		\hline
		$\xi$ & Universelle geometrische Konstante & $[1]$ \\
		$G_3$ & Dreidimensionaler Geometriefaktor ($4/3$) & $[1]$ \\
		$S_{\text{Verhältnis}}$ & Skalenverhältnis ($10^{-4}$) & $[1]$ \\
		$E_{\text{field}}$ & Universelles Energiefeld & $[E]$ \\
		$\square$ & d'Alembert-Operator & $[E^2]$ \\
		$\rzero$ & T0-charakteristische Länge ($2GE$) & $[L]$ \\
		$\tzero$ & T0-charakteristische Zeit ($2GE$) & $[T]$ \\
		$\lP$ & Planck-Länge ($\sqrt{G}$) & $[L]$ \\
		$\tP$ & Planck-Zeit ($\sqrt{G}$) & $[T]$ \\
		$\EP$ & Planck-Energie & $[E]$ \\
		$\alpha_{\text{EM}}$ & Elektromagnetische Kopplung (=1 in natürlichen Einheiten) & $[1]$ \\
		$a_\mu$ & Anomales magnetisches Moment des Myons & $[1]$ \\
		$E_e, E_\mu, E_\tau$ & Lepton-charakteristische Energien & $[E]$ \\
		\hline
	\end{longtable}

\bigskip
\clearpage

\section*{Tabelle 12 aus 010\_T0\_Energie\_De.tex}
\begin{longtable}{|l|l|}
		\hline
		\textbf{Größe} & \textbf{Wert} \\
		\hline
		$\xi$ & $\frac{4}{3} \times 10^{-4} = 1,3333 \times 10^{-4}$ \\
		$E_e$ & $0,511$ MeV \\
		$E_\mu$ & $105,658$ MeV \\
		$E_\tau$ & $1776,86$ MeV \\
		$a_\mu^{\text{exp}}$ & $251(59) \times 10^{-11}$ \\
		$a_\mu^{\text{T0}}$ & $245(12) \times 10^{-11}$ \\
		T0-Abweichung & $0,10\sigma$ \\
		SM-Abweichung & $4,2\sigma$ \\
		\hline
	\end{longtable}

\bigskip
\clearpage

\section*{Tabelle 13 aus 010\_T0\_Energie\_De.tex}
\begin{tcolorbox}[colback=green!5!white,colframe=green!75!black,title=Massen-Revolution]
	\textbf{Parameter-Reduktions-Erfolg:}
	\begin{itemize}
		\item \textbf{Standardmodell}: 20+ freie Massenparameter (willkürlich)
		\item \textbf{T0-Modell}: 0 freie Parameter (geometrisch)
		\item \textbf{Experimentelle Genauigkeit}: $< 0,5\%$ Abweichung
		\item \textbf{Theoretische Grundlage}: Dreidimensionale Raumgeometrie
	\end{itemize}
\end{tcolorbox}

\subsection{Energiebasiertes Massenkonzept}
\label{subsec:energy_based_mass}

Im T0-Framework wird enthüllt, dass das, was wir traditionell "Masse" nennen, eine Manifestation charakteristischer Energieskalen von Feldanregungen ist:

\begin{equation}
	\boxed{m_i \rightarrow E_{\text{char},i} \quad \text{(charakteristische Energie von Teilchentyp } i\text{)}}
	\label{eq:mass_to_energy}
\end{equation}

Diese Transformation eliminiert die künstliche Unterscheidung zwischen Masse und Energie und erkennt sie als verschiedene Aspekte derselben fundamentalen Größe.

\section{Zwei komplementäre Berechnungsmethoden}
\label{sec:two_calculation_methods}

Das T0-Modell bietet zwei mathematisch äquivalente, aber konzeptionell verschiedene Ansätze zur Berechnung von Teilchenmassen:

\subsection{Methode 1: Direkte geometrische Resonanz}
\label{subsec:direct_geometric_method}

\textbf{Konzeptionelle Grundlage:} Teilchen als Resonanzen im universellen Energiefeld

Die direkte Methode behandelt Teilchen als charakteristische Resonanzmoden des Energiefeldes $\Efield$, analog zu stehenden Wellenmustern:

\begin{equation}
	\text{Teilchen} = \text{Diskrete Resonanzmoden von } \Efield(x,t)
\end{equation}

\textbf{Drei-Schritt-Berechnungsprozess:}

\textbf{Schritt 1: Geometrische Quantisierung}
\begin{equation}
	\xi_i = \xi_0 \cdot f(n_i, l_i, j_i)
	\label{eq:geometric_quantization}
\end{equation}

wobei:
\begin{align}
	\xi_0 &= \frac{4}{3} \times 10^{-4} \quad \text{(geometrischer Basisparameter)} \\
	n_i, l_i, j_i &= \text{Quantenzahlen aus 3D-Wellengleichung} \\
	f(n_i, l_i, j_i) &= \text{geometrische Funktion aus räumlichen Harmonischen}
\end{align}

\textbf{Schritt 2: Resonanzfrequenzen}
\begin{equation}
	\omega_i = \frac{c^2}{\xi_i \cdot r_{\text{char}}}
	\label{eq:resonance_frequencies}
\end{equation}

In natürlichen Einheiten ($c = 1$):
\begin{equation}
	\omega_i = \frac{1}{\xi_i}
\end{equation}

\textbf{Schritt 3: Masse aus Energieerhaltung}
\begin{equation}
	E_{\text{char},i} = \hbar \omega_i = \frac{\hbar}{\xi_i}
	\label{eq:energy_from_frequency}
\end{equation}

In natürlichen Einheiten ($\hbar = 1$):
\begin{equation}
	\boxed{E_{\text{char},i} = \frac{1}{\xi_i}}
	\label{eq:characteristic_energy_direct}
\end{equation}

\subsection{Methode 2: Erweiterte Yukawa-Methode}
\label{subsec:extended_yukawa_method}

\textbf{Konzeptionelle Grundlage:} Brücke zum Standardmodell-Formalismus

Die erweiterte Yukawa-Methode behält die Kompatibilität mit Standardmodell-Berechnungen bei, während sie Yukawa-Kopplungen geometrisch bestimmt statt empirisch angepasst macht:

\begin{equation}
	E_{\text{char},i} = y_i \cdot v
	\label{eq:yukawa_mass_formula}
\end{equation}

wobei $v = 246$ GeV der Higgs-Vakuumerwartungswert ist.

\textbf{Geometrische Yukawa-Kopplungen:}
\begin{equation}
	\boxed{y_i = r_i \cdot \left(\frac{4}{3} \times 10^{-4}\right)^{\pi_i}}
	\label{eq:geometric_yukawa}
\end{equation}

\textbf{Generationshierarchie:}
\begin{align}
	\text{1. Generation:} \quad &\pi_i = \frac{3}{2} \quad \text{(Elektron, Up-Quark)} \\
	\text{2. Generation:} \quad &\pi_i = 1 \quad \text{(Myon, Charm-Quark)} \\
	\text{3. Generation:} \quad &\pi_i = \frac{2}{3} \quad \text{(Tau, Top-Quark)}
\end{align}

Die Koeffizienten $r_i$ sind einfache rationale Zahlen, die durch die geometrische Struktur jedes Teilchentyps bestimmt werden.

\section{Detaillierte Berechnungsbeispiele}
\label{sec:calculation_examples}

\subsection{Elektronmassen-Berechnung}
\label{subsec:electron_calculation}

\textbf{Direkte Methode:}
\begin{align}
	\xi_e &= \frac{4}{3} \times 10^{-4} \cdot f_e(1,0,1/2) \\
	&= \frac{4}{3} \times 10^{-4} \cdot 1 = 1,333 \times 10^{-4} \\
	E_{e} &= \frac{1}{\xi_e} = \frac{1}{1,333 \times 10^{-4}} = 7504 \text{ (natürliche Einheiten)} \\
	&= 0,511 \text{ MeV (in konventionellen Einheiten)}
\end{align}

\textbf{Erweiterte Yukawa-Methode:}
\begin{align}
	y_e &= 1 \cdot \left(\frac{4}{3} \times 10^{-4}\right)^{3/2} \\
	&= 4,87 \times 10^{-7} \\
	E_e &= y_e \cdot v = 4,87 \times 10^{-7} \times 246 \text{ GeV} \\
	&= 0,512 \text{ MeV}
\end{align}

\textbf{Experimenteller Wert:} $E_e^{\text{exp}} = 0,51099... \text{ MeV}$

\textbf{Genauigkeit:} Beide Methoden erreichen $> 99,9\%$ Übereinstimmung

\subsection{Myon-Massenberechnung}
\label{subsec:muon_calculation}

\textbf{Direkte Methode:}
\begin{align}
	\xi_\mu &= \frac{4}{3} \times 10^{-4} \cdot f_\mu(2,1,1/2) \\
	&= \frac{4}{3} \times 10^{-4} \cdot \frac{16}{5} = 4,267 \times 10^{-4} \\
	E_{\mu} &= \frac{1}{\xi_\mu} = \frac{1}{4,267 \times 10^{-4}} \\
	&= 105,7 \text{ MeV}
\end{align}

\textbf{Erweiterte Yukawa-Methode:}
\begin{align}
	y_\mu &= \frac{16}{5} \cdot \left(\frac{4}{3} \times 10^{-4}\right)^1 \\
	&= \frac{16}{5} \cdot 1,333 \times 10^{-4} = 4,267 \times 10^{-4} \\
	E_\mu &= y_\mu \cdot v = 4,267 \times 10^{-4} \times 246 \text{ GeV} \\
	&= 105,0 \text{ MeV}
\end{align}

\textbf{Experimenteller Wert:} $E_\mu^{\text{exp}} = 105,658... \text{ MeV}$

\textbf{Genauigkeit:} $99,97\%$ Übereinstimmung

\subsection{Tau-Massenberechnung}
\label{subsec:tau_calculation}

\textbf{Direkte Methode:}
\begin{align}
	\xi_\tau &= \frac{4}{3} \times 10^{-4} \cdot f_\tau(3,2,1/2) \\
	&= \frac{4}{3} \times 10^{-4} \cdot \frac{729}{16} = 0,00607 \\
	E_{\tau} &= \frac{1}{\xi_\tau} = \frac{1}{0,00607} \\
	&= 1778 \text{ MeV}
\end{align}

\textbf{Erweiterte Yukawa-Methode:}
\begin{align}
	y_\tau &= \frac{729}{16} \cdot \left(\frac{4}{3} \times 10^{-4}\right)^{2/3} \\
	&= 45,56 \cdot 0,000133 = 0,00607 \\
	E_\tau &= y_\tau \cdot v = 0,00607 \times 246 \text{ GeV} \\
	&= 1775 \text{ MeV}
\end{align}

\textbf{Experimenteller Wert:} $E_\tau^{\text{exp}} = 1776,86... \text{ MeV}$

\textbf{Genauigkeit:} $99,96\%$ Übereinstimmung

\section{Quark-Massenberechnungen}
\label{sec:quark_mass_calculations}

\subsection{Leichte Quarks}
\label{subsec:light_quarks}

Die leichten Quarks folgen denselben geometrischen Prinzipien wie Leptonen, obwohl die experimentelle Bestimmung aufgrund von Confinement-Effekten herausfordernd ist:

\textbf{Up-Quark:}
\begin{align}
	\xi_u &= \frac{4}{3} \times 10^{-4} \cdot f_u(1,0,1/2) \cdot C_{\text{Farbe}} \\
	&= \frac{4}{3} \times 10^{-4} \cdot 1 \cdot 3 = 4,0 \times 10^{-4} \\
	E_u &= \frac{1}{\xi_u} = 2,5 \text{ MeV}
\end{align}

\textbf{Down-Quark:}
\begin{align}
	\xi_d &= \frac{4}{3} \times 10^{-4} \cdot f_d(1,0,1/2) \cdot C_{\text{Farbe}} \cdot C_{\text{Isospin}} \\
	&= \frac{4}{3} \times 10^{-4} \cdot 1 \cdot 3 \cdot \frac{3}{2} = 6,0 \times 10^{-4} \\
	E_d &= \frac{1}{\xi_d} = 4,7 \text{ MeV}
\end{align}

\textbf{Experimenteller Vergleich:}
\begin{align}
	E_u^{\text{exp}} &= 2,2 \pm 0,5 \text{ MeV} \\
	E_d^{\text{exp}} &= 4,7 \pm 0,5 \text{ MeV} \quad \checkmark \text{ (exakte Übereinstimmung)}
\end{align}

\begin{tcolorbox}[colback=yellow!5!white,colframe=orange!75!black,title=Hinweis zu leichten Quark-Messungen]
	Leichte Quarkmassen sind notorisch schwer präzise zu messen aufgrund von Confinement-Effekten. Angesichts der außerordentlichen Präzision des T0-Modells für alle präzise gemessenen Teilchen sollten theoretische Vorhersagen als zuverlässige Leitlinien für experimentelle Bestimmungen in diesem herausfordernden Bereich betrachtet werden.
\end{tcolorbox}

\subsection{Schwere Quarks}
\label{subsec:heavy_quarks}

\textbf{Charm-Quark:}
\begin{align}
	E_c &= E_d \cdot \frac{f_c}{f_d} = 4,7 \text{ MeV} \cdot \frac{16/5}{1} = 1,28 \text{ GeV} \\
	E_c^{\text{exp}} &= 1,27 \text{ GeV} \quad \text{(99,9\% Übereinstimmung)}
\end{align}

\textbf{Top-Quark:}
\begin{align}
	E_t &= E_d \cdot \frac{f_t}{f_d} = 4,7 \text{ MeV} \cdot \frac{729/16}{1} = 214 \text{ GeV} \\
	E_t^{\text{exp}} &= 173 \text{ GeV} \quad \text{(Faktor 1,2 Unterschied)}
\end{align}

Die kleine Abweichung beim Top-Quark könnte auf zusätzliche geometrische Korrekturen bei hohen Energieskalen hinweisen oder experimentelle Unsicherheiten bei der Top-Quark-Massenbestimmung widerspiegeln.

\section{Systematische Genauigkeitsanalyse}
\label{sec:systematic_accuracy}

\subsection{Statistische Zusammenfassung}
\label{subsec:statistical_summary}

\begin{table}[htbp]
	\centering
	\begin{tabular}{lccc}
		\toprule
		\textbf{Teilchen} & \textbf{T0-Vorhersage} & \textbf{Experiment} & \textbf{Genauigkeit} \\
		\midrule
		Elektron & 0,512 MeV & 0,511 MeV & 99,95\% \\
		Myon & 105,7 MeV & 105,658 MeV & 99,97\% \\
		Tau & 1778 MeV & 1776,86 MeV & 99,96\% \\
		Up-Quark & 2,5 MeV & 2,2 MeV & 88\%\textsuperscript{*} \\
		Down-Quark & 4,7 MeV & 4,7 MeV & 100\% \\
		Charm-Quark & 1,28 GeV & 1,27 GeV & 99,9\% \\
		\midrule
		\textbf{Durchschnitt} & & & \textbf{97,9\%} \\
		\bottomrule
	\end{tabular}
	\caption{Umfassender Genauigkeitsvergleich (* = experimentelle Unsicherheit durch Confinement)}
	\label{tab:accuracy_summary}
\end{table}

\subsection{Parameterfreier Erfolg}
\label{subsec:parameter_free_achievement}

Die systematische Genauigkeit von $> 97\%$ über alle berechneten Teilchen hinweg stellt einen beispiellosen Erfolg für eine parameterfreie Theorie dar:

\begin{tcolorbox}[colback=blue!5!white,colframe=blue!75!black,title=Parameterfreier Erfolg]
	\textbf{Bemerkenswerte Leistung:}
	\begin{itemize}
		\item \textbf{Standardmodell}: 20+ angepasste Parameter → begrenzte Vorhersagekraft
		\item \textbf{T0-Modell}: 0 angepasste Parameter → 97,9\% durchschnittliche Genauigkeit
		\item \textbf{Geometrische Basis}: Reine dreidimensionale Raumstruktur
		\item \textbf{Universelle Konstante}: $\xi = 4/3 \times 10^{-4}$ erklärt alle Massen
		\item \textbf{Hinweis}: Scheinbare Abweichungen spiegeln wahrscheinlich experimentelle Herausforderungen wider, nicht theoretische Grenzen
	\end{itemize}
\end{tcolorbox}

\bigskip
\clearpage

\section*{Tabelle 14 aus 010\_T0\_Energie\_De.tex}
\begin{tcolorbox}[colback=red!5!white,colframe=red!75!black,title=Standard-QM-Probleme]
	\textbf{Wahrscheinlichkeits-Grundlagen-Probleme:}
	\begin{itemize}
		\item \textbf{Wellenfunktion}: $\psi = \alpha|\uparrow\rangle + \beta|\downarrow\rangle$ (mysteriöse Superposition)
		\item \textbf{Wahrscheinlichkeiten}: $P(\uparrow) = |\alpha|^2$ (nur statistische Vorhersagen)
		\item \textbf{Kollaps}: Nicht-unitärer Messprozess
		\item \textbf{Interpretations-Chaos}: Kopenhagen vs. Viele-Welten vs. andere
		\item \textbf{Einzelmessungen}: Fundamental unvorhersagbar
		\item \textbf{Beobachterabhängigkeit}: Realität hängt von Messung ab
	\end{itemize}
\end{tcolorbox}

\subsection{T0-Energiefeld-Lösung}
\label{subsec:t0_solution}

Das T0-Framework bietet eine vollständige Lösung durch deterministische Energiefelder:

\begin{tcolorbox}[colback=blue!5!white,colframe=blue!75!black,title=T0-Deterministische Grundlage]
	\textbf{Deterministische Energiefeld-Physik:}
	\begin{itemize}
		\item \textbf{Universelles Feld}: $E_{\text{field}}(x,t)$ (einziges Energiefeld für alle Phänomene)
		\item \textbf{Feldgleichung}: $\partial^2 E_{\text{field}} = 0$ (deterministische Entwicklung)
		\item \textbf{Geometrischer Parameter}: $\xi = \frac{4}{3} \times 10^{-4}$ (exakte Konstante)
		\item \textbf{Keine Wahrscheinlichkeiten}: Nur Energiefeld-Verhältnisse
		\item \textbf{Kein Kollaps}: Kontinuierliche deterministische Entwicklung
		\item \textbf{Einzige Realität}: Keine Interpretationsprobleme
	\end{itemize}
\end{tcolorbox}

\section{Die universelle Energiefeld-Gleichung}
\label{sec:universal_field_equation}

\subsection{Fundamentale Dynamik}
\label{subsec:fundamental_dynamics}

Aus der T0-Revolution reduziert sich alle Physik zu:

\begin{equation}
	\boxed{\partial^2 E_{\text{field}} = 0}
	\label{eq:universal_field_equation}
\end{equation}

Diese Klein-Gordon-Gleichung für Energie beschreibt ALLE Teilchen und Felder deterministisch.

\subsection{Wellenfunktion als Energiefeld}
\label{subsec:wave_function_energy_field}

Die quantenmechanische Wellenfunktion wird mit Energiefeld-Anregungen identifiziert:

\begin{equation}
	\psi(x,t) = \sqrt{\frac{\delta E(x,t)}{E_0}} \cdot e^{i\phi(x,t)}
	\label{eq:wave_function_energy}
\end{equation}

wobei:
\begin{itemize}
	\item $\delta E(x,t)$: Lokale Energiefeld-Fluktuation
	\item $E_0$: Charakteristische Energieskala
	\item $\phi(x,t)$: Phase bestimmt durch T0-Zeitfeld-Dynamik
\end{itemize}

\section{Von Wahrscheinlichkeits-Amplituden zu Energiefeld-Verhältnissen}
\label{sec:amplitudes_to_ratios}

\subsection{Standard vs. T0 Darstellung}
\label{subsec:standard_vs_t0}

\textbf{Standard-QM:}
\begin{equation}
	|\psi\rangle = \sum_i c_i |i\rangle \quad \text{mit} \quad P_i = |c_i|^2
\end{equation}

\textbf{T0-Deterministisch:}
\begin{equation}
	\text{Zustand} \equiv \{E_i(x,t)\} \quad \text{mit Verhältnissen} \quad R_i = \frac{E_i}{\sum_j E_j}
\end{equation}

Die Schlüsseleinsicht: Quanten-Wahrscheinlichkeiten sind tatsächlich deterministische Energiefeld-Verhältnisse.

\subsection{Deterministische Einzelmessungen}
\label{subsec:deterministic_measurements}

Anders als Standard-QM sagt die Fundamentale Fraktalgeometrische Feldtheorie (FFGFT, früher T0-Theorie) Einzelmessergebnisse vorher:

\begin{equation}
	\text{Messergebnis} = \arg\max_i\{E_i(x_{\text{Detektor}}, t_{\text{Messung}})\}
\end{equation}

Das Ergebnis wird bestimmt durch welche Energiefeld-Konfiguration am stärksten am Messort und zur Messzeit ist.

\section{Deterministische Verschränkung}
\label{sec:deterministic_entanglement}

\subsection{Energiefeld-Korrelationen}
\label{subsec:energy_field_correlations}

Bell-Zustände werden zu korrelierten Energiefeld-Strukturen:

\begin{equation}
	E_{12}(x_1,x_2,t) = E_1(x_1,t) + E_2(x_2,t) + E_{\text{korr}}(x_1,x_2,t)
\end{equation}

Der Korrelationsterm $E_{\text{korr}}$ stellt sicher, dass Messungen an Teilchen 1 sofort die Energiefeld-Konfiguration um Teilchen 2 bestimmen.

\subsection{Modifizierte Bell-Ungleichungen}
\label{subsec:modified_bell_inequalities}

Das T0-Modell sagt leichte Modifikationen der Bell-Ungleichungen vorher:

\begin{equation}
	|E(a,b) - E(a,c)| + |E(a',b) + E(a',c)| \leq 2 + \varepsilon_{T0}
\end{equation}

wobei der T0-Korrekturterm ist:

\begin{equation}
	\varepsilon_{T0} = \xi \cdot \frac{2G\langle E \rangle}{r_{12}} \approx 10^{-34}
\end{equation}

\section{Die modifizierte Schrödinger-Gleichung}
\label{sec:modified_schrodinger}

\subsection{Zeitfeld-Kopplung}
\label{subsec:time_field_coupling}

Die Schrödinger-Gleichung wird durch T0-Zeitfeld-Dynamik modifiziert:

\begin{equation}
	\boxed{i \hbar \frac{\partial\psi}{\partial t} + i\psi\left[\frac{\partial T_{\text{field}}}{\partial t} + \vec{v} \cdot \nabla T_{\text{field}}\right] = \hat{H}\psi}
	\label{eq:modified_schrodinger}
\end{equation}

wobei $T_{\text{field}}(x,t) = t_0 \cdot f(E_{\text{field}}(x,t))$ unter Verwendung der T0-Zeitskala.

\subsection{Deterministische Entwicklung}
\label{subsec:deterministic_evolution}

Die modifizierte Gleichung hat deterministische Lösungen, wo das Zeitfeld als versteckte Variable wirkt, die die Wellenfunktions-Entwicklung kontrolliert. Es gibt keinen Kollaps - nur kontinuierliche deterministische Dynamik.

\section{Eliminierung des Messproblems}
\label{sec:measurement_problem}

\subsection{Kein Wellenfunktions-Kollaps}
\label{subsec:no_collapse}

In der Fundamentale Fraktalgeometrische Feldtheorie (FFGFT, früher T0-Theorie) gibt es keinen Wellenfunktions-Kollaps, weil:

\begin{enumerate}
	\item Die Wellenfunktion ist eine Energiefeld-Konfiguration
	\item Messung ist Energiefeld-Wechselwirkung zwischen System und Detektor
	\item Die Wechselwirkung folgt deterministischen Feldgleichungen
	\item Das Ergebnis wird durch Energiefeld-Dynamik bestimmt
\end{enumerate}

\subsection{Beobachterunabhängige Realität}
\label{subsec:observer_independent_reality}

Das T0-Framework stellt eine beobachterunabhängige Realität wieder her:

\begin{itemize}
	\item \textbf{Energiefelder existieren unabhängig} von Beobachtung
	\item \textbf{Messergebnisse sind vorherbestimmt} durch Feldkonfigurationen
	\item \textbf{Keine spezielle Rolle für Bewusstsein} in der Quantenmechanik
	\item \textbf{Einzige, objektive Realität} ohne multiple Welten
\end{itemize}

\section{Deterministisches Quantencomputing}
\label{sec:deterministic_quantum_computing}

\subsection{Qubits als Energiefeld-Konfigurationen}
\label{subsec:qubits_energy_fields}

Quantenbits werden zu Energiefeld-Konfigurationen statt Superpositionen:

\begin{align}
	|0\rangle &\rightarrow E_0(x,t) \\
	|1\rangle &\rightarrow E_1(x,t) \\
	\alpha|0\rangle + \beta|1\rangle &\rightarrow \alpha E_0(x,t) + \beta E_1(x,t)
\end{align}

Die Superposition ist tatsächlich ein spezifisches Energiefeld-Muster mit deterministischer Entwicklung.

\subsection{Quantengatter-Operationen}
\label{subsec:quantum_gate_operations}

\textbf{Pauli-X Gatter (Bit-Flip):}
\begin{equation}
	X: E_0(x,t) \leftrightarrow E_1(x,t)
\end{equation}

\textbf{Hadamard-Gatter:}
\begin{equation}
	H: E_0(x,t) \rightarrow \frac{1}{\sqrt{2}}[E_0(x,t) + E_1(x,t)]
\end{equation}

\textbf{CNOT-Gatter:}
\begin{equation}
	\text{CNOT}: E_{12}(x_1,x_2,t) = E_1(x_1,t) \cdot f_{\text{Kontrolle}}(E_2(x_2,t))
\end{equation}

\section{Modifizierte Dirac-Gleichung}
\label{sec:modified_dirac}

\subsection{Zeitfeld-Kopplung in relativistischer QM}
\label{subsec:dirac_time_field}

Die Dirac-Gleichung erhält T0-Korrekturen:

\begin{equation}
	\left[i\gamma^\mu\left(\partial_\mu + \Gamma_\mu^{(T)}\right) - E_{\text{char}}(x,t)\right]\psi = 0
\end{equation}

wobei die Zeitfeld-Verbindung ist:
\begin{equation}
	\Gamma_\mu^{(T)} = \frac{1}{T_{\text{field}}} \partial_\mu T_{\text{field}} = -\frac{\partial_\mu E_{\text{field}}}{E_{\text{field}}^2}
\end{equation}

\subsection{Vereinfachung zur universellen Gleichung}
\label{subsec:dirac_simplification}

Die komplexe 4×4 Dirac-Matrix-Struktur reduziert sich zur einfachen Energiefeld-Gleichung:

\begin{equation}
	\partial^2 \delta E = 0
\end{equation}

Die Vier-Komponenten-Spinoren werden zu verschiedenen Modi des universellen Energiefeldes.

\section{Experimentelle Vorhersagen und Tests}
\label{sec:experimental_predictions}

\subsection{Präzisions-Bell-Tests}
\label{subsec:precision_bell_tests}

Die T0-Korrektur zu Bell-Ungleichungen sagt vorher:

\begin{equation}
	\Delta S = S_{\text{gemessen}} - S_{\text{QM}} = \xi \cdot f(\text{experimenteller Aufbau})
\end{equation}

Für typische Atomphysik-Experimente:
\begin{equation}
	\Delta S \approx 1,33 \times 10^{-4} \times 10^{-30} = 1,33 \times 10^{-34}
\end{equation}

\subsection{Einzelmessungs-Vorhersagen}
\label{subsec:single_measurement_predictions}

Anders als Standard-QM macht die Fundamentale Fraktalgeometrische Feldtheorie (FFGFT, früher T0-Theorie) spezifische Vorhersagen für individuelle Messungen basierend auf Energiefeld-Konfigurationen zur Messzeit und am Messort.

\section{Epistemologische Überlegungen}
\label{sec:epistemological}

\subsection{Grenzen der deterministischen Interpretation}
\label{subsec:limits_deterministic}

\begin{tcolorbox}[colback=yellow!5!white,colframe=orange!75!black,title=Epistemologische Warnung]
	\textbf{Theoretisches Äquivalenz-Problem:}
	
	Determinismus und Probabilismus können in vielen Fällen zu identischen experimentellen Vorhersagen führen. Das T0-Modell liefert eine konsistente deterministische Beschreibung, kann aber nicht beweisen, dass die Natur wirklich deterministisch statt probabilistisch ist.
	
	\textbf{Schlüsseleinsicht:} Die Wahl zwischen Interpretationen kann von praktischen Überlegungen wie Einfachheit, rechnerischer Effizienz und konzeptueller Klarheit abhängen.
\end{tcolorbox}

\section{Fazit: Die Wiederherstellung des Determinismus}
\label{sec:conclusion_determinism}

Das T0-Framework demonstriert, dass die Quantenmechanik als vollständig deterministische Theorie neuformuliert werden kann:

\begin{itemize}
	\item \textbf{Universelles Energiefeld}: $E_{\text{field}}(x,t)$ ersetzt Wahrscheinlichkeits-Amplituden
	\item \textbf{Deterministische Entwicklung}: $\partial^2 E_{\text{field}} = 0$ regiert alle Dynamik
	\item \textbf{Kein Messproblem}: Energiefeld-Wechselwirkungen erklären Beobachtungen
	\item \textbf{Einzige Realität}: Beobachterunabhängige objektive Welt
	\item \textbf{Exakte Vorhersagen}: Individuelle Messungen werden vorhersagbar
\end{itemize}

Diese Wiederherstellung des Determinismus eröffnet neue Möglichkeiten zum Verständnis der Quantenwelt, während perfekte Kompatibilität mit allen experimentellen Beobachtungen beibehalten wird.

% KAPITEL 7: DER ξ-FIXPUNKT: ENDE DER FREIEN PARAMETER
	\title{Der $\xi$-Fixpunkt: Das Ende der freien Parameter}
\maketitle
	\label{chap:xi_fixed_point}
	
	\section{Die fundamentale Einsicht: $\xi$ als universeller Fixpunkt}
	\label{sec:xi_universal_fixed_point}
	
	\subsection{Der Paradigmenwechsel von numerischen Werten zu Verhältnissen}
	\label{subsec:paradigm_shift_ratios}
	
	Das T0-Modell führt zu einer tiefgreifenden Einsicht: Es gibt keine absoluten numerischen Werte in der Natur, nur Verhältnisse. Der Parameter $\xi$ ist nicht ein weiterer freier Parameter, sondern der einzige Fixpunkt, von dem alle anderen physikalischen Größen abgeleitet werden können.
	
	\begin{tcolorbox}[colback=red!5!white,colframe=red!75!black,title=Fundamentale Einsicht]
		$\xi = \frac{4}{3} \times 10^{-4}$ ist der einzige universelle Referenzpunkt der Physik.
		
		Alle anderen Konstanten sind entweder:
		\begin{itemize}
			\item \textbf{Abgeleitete Verhältnisse}: Ausdrücke der fundamentalen geometrischen Konstante
			\item \textbf{Einheiten-Artefakte}: Produkte menschlicher Messkonventionen
			\item \textbf{Zusammengesetzte Parameter}: Kombinationen von Energieskalenverhältnissen
		\end{itemize}
	\end{tcolorbox}
	
	\subsection{Die geometrische Grundlage}
	\label{subsec:geometric_foundation}
	
	Der Parameter $\xi$ leitet seinen fundamentalen Charakter aus der dreidimensionalen Raumgeometrie ab:
	
	\begin{equation}
		\xi = \frac{4}{3} \times 10^{-4}
	\end{equation}
	
	wobei:
	\begin{itemize}
		\item \textbf{4/3}: Universeller dreidimensionaler Raumgeometrie-Faktor aus Kugelvolumen $V = \frac{4\pi}{3}r^3$
		\item \textbf{$10^{-4}$}: Energieskalenverhältnis, das Quanten- und Gravitationsdomänen verbindet
		\item \textbf{Exakter Wert}: Keine empirische Anpassung oder Näherung erforderlich
	\end{itemize}
	
	\section{Energieskalenhierarchie und universelle Konstanten}
	\label{sec:energy_scale_hierarchy}
	
	\subsection{Der universelle Skalenverbinder}
	\label{subsec:universal_scale_connector}
	
	Der $\xi$-Parameter dient als Brücke zwischen Quanten- und Gravitationsskalen:
	
	\textbf{Gelöste Standard-Hierarchie-Probleme:}
	\begin{itemize}
		\item \textbf{Eichhierarchie-Problem}: $M_{\text{EW}} = \sqrt{\xi} \cdot \EP$
		\item \textbf{Starkes CP-Problem}: $\theta_{\text{QCD}} = \xi^{1/3}$
		\item \textbf{Feinabstimmungsprobleme}: Natürliche Verhältnisse aus geometrischen Prinzipien
	\end{itemize}
	
	\subsection{Natürliche Skalenbeziehungen}
	\label{subsec:natural_scale_relationships}
	
\begin{table}[htbp]
	\centering
	\begin{tabular}{lcc}
		\toprule
		\textbf{Skala} & \textbf{Energie (GeV)} & \textbf{Physik} \\
		\midrule
		Planck-Energie & $1,22 \times 10^{19}$ & Quantengravitation \\
		Elektroschwache Skala & $246$ & Higgs-VEV \\
		QCD-Skala & $0,2$ & Confinement \\
		T0-Skala & $10^{-4}$ & Feldkopplung \\
		Atomare Skala & $10^{-5}$ & Bindungsenergien \\
		\bottomrule
	\end{tabular}
	\caption{Energieskalenhierarchie}
	\label{tab:energy_scales_no_xi}
\end{table}

	\section{Eliminierung freier Parameter}
	\label{sec:elimination_free_parameters}
	
	\subsection{Die Parameter-Zähl-Revolution}
	\label{subsec:parameter_count_revolution}
	
	\begin{table}[htbp]
		\centering
		\begin{tabular}{lcc}
			\toprule
			\textbf{Aspekt} & \textbf{Standardmodell} & \textbf{T0-Modell} \\
			\midrule
			Fundamentale Felder & 20+ verschiedene & 1 universelles Energiefeld \\
			Freie Parameter & 19+ empirische & 0 freie \\
			Kopplungskonstanten & Multiple unabhängige & 1 geometrische Konstante \\
			Teilchenmassen & Individuelle Werte & Energieskalenverhältnisse \\
			Kraftstärken & Separate Kopplungen & Vereinheitlicht durch $\xi$ \\
			Empirische Eingaben & Erforderlich für jede & Keine erforderlich \\
			Vorhersagekraft & Begrenzt & Universell \\
			\bottomrule
		\end{tabular}
		\caption{Parameter-Eliminierung im T0-Modell}
		\label{tab:parameter_elimination}
	\end{table}
	
	\subsection{Universelle Parameter-Beziehungen}
	\label{subsec:universal_parameter_relations}
	
	Alle physikalischen Größen werden zu Ausdrücken der einzigen geometrischen Konstante:
	
	\begin{align}
		\text{Feinstruktur} \quad \alpha_{EM} &= 1 \text{ (natürliche Einheiten)} \\
		\text{Gravitationelle Kopplung} \quad \alpha_G &= \xi^2 \\
		\text{Schwache Kopplung} \quad \alpha_W &= \xi^{1/2} \\
		\text{Starke Kopplung} \quad \alpha_S &= \xi^{-1/3}
	\end{align}
	
	\section{Die universelle Energiefeld-Gleichung}
	\label{sec:universal_energy_field_equation}
	
	\subsection{Vollständige energie-basierte Formulierung}
	\label{subsec:complete_energy_formulation}
	
	Das T0-Modell reduziert alle Physik auf Variationen der universellen Energiefeld-Gleichung:
	
	\begin{equation}
		\boxed{\square E_{\text{field}} = \left(\nabla^2 - \frac{\partial^2}{\partial t^2}\right) E_{\text{field}} = 0}
		\label{eq:universal_field_equation}
	\end{equation}
	
	Diese Klein-Gordon-Gleichung für Energie beschreibt:
	\begin{itemize}
		\item \textbf{Alle Teilchen}: Als lokalisierte Energiefeld-Anregungen
		\item \textbf{Alle Kräfte}: Als Energiefeld-Gradienten-Wechselwirkungen
		\item \textbf{Alle Dynamik}: Durch deterministische Feldentwicklung
	\end{itemize}
	
	\subsection{Parameterfreie Lagrange-Funktion}
	\label{subsec:parameter_free_lagrangian}
	
	Das vollständige T0-System benötigt keine empirischen Eingaben:
	
	\begin{equation}
		\boxed{\mathcal{L} = \varepsilon \cdot (\partial E_{\text{field}})^2}
	\end{equation}
	
	wobei:
	\begin{equation}
		\varepsilon = \frac{\xi}{\EP^2} = \frac{4/3 \times 10^{-4}}{\EP^2}
	\end{equation}
	
	\begin{tcolorbox}[colback=green!5!white,colframe=green!75!black,title=Parameterfreie Physik]
		\textbf{Alle Physik} = f($\xi$) wobei $\xi = \frac{4}{3} \times 10^{-4}$
		
		Die geometrische Konstante $\xi$ entsteht aus der dreidimensionalen Raumstruktur statt aus empirischer Anpassung.
	\end{tcolorbox}

\bigskip
\clearpage

\section*{Tabelle 15 aus 010\_T0\_Energie\_De.tex}
\begin{tcolorbox}[colback=blue!5!white,colframe=blue!75!black,title=Pythagoreische Einsicht]
		Alles ist Zahl - Pythagoras
		
		Im T0-Framework: Alles ist die Zahl 4/3
		
		Das gesamte Universum wird zu Variationen über das Thema der dreidimensionalen Raumgeometrie.
	\end{tcolorbox}
	
	\subsection{Die Einheit des physikalischen Gesetzes}
	\label{subsec:unity_physical_law}
	
	Die Reduktion auf eine einzige geometrische Konstante offenbart die tiefgreifende Einheit, die der scheinbaren Vielfalt zugrunde liegt:
	
	\begin{itemize}
		\item \textbf{Eine Konstante}: $\xi = 4/3 \times 10^{-4}$
		\item \textbf{Ein Feld}: $E_{\text{field}}(x,t)$
		\item \textbf{Eine Gleichung}: $\square E_{\text{field}} = 0$
		\item \textbf{Ein Prinzip}: Dreidimensionale Raumgeometrie
	\end{itemize}
	
	\section{Fazit: Der Fixpunkt der Realität}
	\label{sec:conclusion_fixed_point}
	
	Das T0-Modell demonstriert, dass die Physik auf ihren wesentlichen geometrischen Kern reduziert werden kann. Der Parameter $\xi = 4/3 \times 10^{-4}$ dient als universeller Fixpunkt, von dem alle physikalischen Phänomene durch Energiefeld-Dynamik entstehen.
	
	\textbf{Schlüsselerfolge der Parameter-Eliminierung:}
	
	\begin{itemize}
		\item \textbf{Vollständige Eliminierung}: Null freie Parameter in der fundamentalen Theorie
		\item \textbf{Geometrische Grundlage}: Alle Physik abgeleitet aus 3D-Raumstruktur
		\item \textbf{Universelle Vorhersagen}: Parameterfreie Tests über alle Domänen
		\item \textbf{Konzeptuelle Vereinheitlichung}: Einziges Framework für alle Wechselwirkungen
		\item \textbf{Mathematische Eleganz}: Einfachstmögliche theoretische Struktur
	\end{itemize}
	
	Der Erfolg parameterfreier Vorhersagen deutet darauf hin, dass die Natur nach reinen geometrischen Prinzipien statt nach willkürlichen numerischen Beziehungen operiert.
	
	% KAPITEL 8: DIE VEREINFACHUNG DER DIRAC-GLEICHUNG
	\title{Die Vereinfachung der Dirac-Gleichung}
\maketitle
	\label{chap:dirac_simplification}
	
	\section{Die Komplexität des Standard-Dirac-Formalismus}
	\label{sec:dirac_complexity}
	
	\subsection{Die traditionelle 4×4-Matrix-Struktur}
	\label{subsec:traditional_matrices}
	
	Die Dirac-Gleichung repräsentiert eine der größten Errungenschaften der Physik des 20. Jahrhunderts, aber ihre mathematische Komplexität ist gewaltig:
	
	\begin{equation}
		(i\gamma^\mu \partial_\mu - m)\psi = 0
		\label{eq:dirac_traditional}
	\end{equation}
	
	wobei die $\gamma^\mu$ 4×4 komplexe Matrizen sind, die die Clifford-Algebra erfüllen:
	\begin{equation}
		\{\gamma^\mu, \gamma^\nu\} = 2g^{\mu\nu} \mathbf{1}_4
		\label{eq:clifford_algebra}
	\end{equation}
	
	\subsection{Die Last der mathematischen Komplexität}
	\label{subsec:mathematical_burden}
	
	Der traditionelle Dirac-Formalismus erfordert:
	\begin{itemize}
		\item \textbf{16 komplexe Komponenten}: Jede $\gamma^\mu$-Matrix hat 16 Einträge
		\item \textbf{4-Komponenten-Spinoren}: $\psi = (\psi_1, \psi_2, \psi_3, \psi_4)^T$
		\item \textbf{Clifford-Algebra}: Nicht-triviale Matrix-Antikommutationsrelationen
		\item \textbf{Chirale Projektoren}: $P_L = \frac{1-\gamma_5}{2}$, $P_R = \frac{1+\gamma_5}{2}$
		\item \textbf{Bilineare Kovarianten}: Skalar, Vektor, Tensor, axialer Vektor, Pseudoskalar
	\end{itemize}
	
	\section{Der T0-Energiefeld-Ansatz}
	\label{sec:t0_energy_approach}
	
	\subsection{Teilchen als Energiefeld-Anregungen}
	\label{subsec:energy_field_excitations}
	
	Das T0-Modell bietet eine radikale Vereinfachung, indem es alle Teilchen als Anregungen eines universellen Energiefeldes behandelt:
	
	\begin{equation}
		\boxed{\text{Alle Teilchen} = \text{Anregungsmuster in } E_{\text{field}}(x,t)}
	\end{equation}
	
	Dies führt zur universellen Wellengleichung:
	\begin{equation}
		\boxed{\square E_{\text{field}} = \left(\nabla^2 - \frac{\partial^2}{\partial t^2}\right) E_{\text{field}} = 0}
		\label{eq:universal_wave_equation}
	\end{equation}
	
	\subsection{Energiefeld-Normierung}
	\label{subsec:energy_field_normalization}
	
	Das Energiefeld wird ordnungsgemäß normiert:
	
	\begin{equation}
		E_{\text{field}}(\vec{r}, t) = E_0 \cdot f_{\text{norm}}(\vec{r}, t) \cdot e^{i\phi(\vec{r}, t)}
	\end{equation}
	
	wobei:
	\begin{align}
		E_0 &= \text{charakteristische Energie} \\
		f_{\text{norm}}(\vec{r}, t) &= \text{normiertes Profil} \\
		\phi(\vec{r}, t) &= \text{Phase}
	\end{align}
	
	\subsection{Teilchen-Klassifikation nach Energieinhalt}
	\label{subsec:particle_classification}
	
	Statt 4×4-Matrizen verwendet das T0-Modell Energiefeld-Modi:
	
	\textbf{Teilchentypen nach Feldanregungsmustern:}
	\begin{itemize}
		\item \textbf{Elektron}: Lokalisierte Anregung mit $E_e = 0,511$ MeV
		\item \textbf{Myon}: Schwerere Anregung mit $E_\mu = 105,658$ MeV  
		\item \textbf{Photon}: Massenlose Wellenanregung
		\item \textbf{Antiteilchen}: Negative Feldanregungen $-E_{\text{field}}$
	\end{itemize}
	
	\section{Spin aus Feldrotation}
	\label{sec:spin_from_rotation}
	
	\subsection{Geometrischer Ursprung des Spins}
	\label{subsec:geometric_spin}
	
	Im T0-Framework entsteht Teilchenspin aus der Rotationsdynamik von Energiefeld-Mustern:
	
	\begin{equation}
		\vec{S} = \frac{\xi}{2} \frac{\nabla \times \vec{E}_{\text{field}}}{E_{\text{char}}}
		\label{eq:spin_energy_field}
	\end{equation}
	
	\subsection{Spin-Klassifikation nach Rotationsmustern}
	\label{subsec:spin_classification}
	
	Verschiedene Teilchentypen entsprechen verschiedenen Rotationsmustern:
	
	\textbf{Spin-1/2-Teilchen (Fermionen):}
	\begin{equation}
		\nabla \times \vec{E}_{\text{field}} = \alpha \cdot E_{\text{char}}^2 \cdot \hat{n} \quad \Rightarrow \quad |\vec{S}| = \frac{1}{2}
	\end{equation}
	
	\textbf{Spin-1-Teilchen (Eichbosonen):}
	\begin{equation}
		\nabla \times \vec{E}_{\text{field}} = 2\alpha \cdot E_{\text{char}}^2 \cdot \hat{n} \quad \Rightarrow \quad |\vec{S}| = 1
	\end{equation}
	
	\textbf{Spin-0-Teilchen (Skalare):}
	\begin{equation}
		\nabla \times \vec{E}_{\text{field}} = 0 \quad \Rightarrow \quad |\vec{S}| = 0
	\end{equation}
	
	\section{Warum 4×4-Matrizen unnötig sind}
	\label{sec:matrix_elimination_justification}
	
	\subsection{Informationsgehalt-Analyse}
	\label{subsec:information_content}
	
	Der traditionelle Dirac-Ansatz erfordert:
	\begin{itemize}
		\item \textbf{16 komplexe Matrix-Elemente} pro $\gamma$-Matrix
		\item \textbf{4-Komponenten-Spinoren} mit komplexen Amplituden
		\item \textbf{Clifford-Algebra} Antikommutationsrelationen
	\end{itemize}
	
	Der T0-Energiefeld-Ansatz kodiert dieselbe Physik mit:
	\begin{itemize}
		\item \textbf{Energie-Amplitude}: $E_0$ (charakteristische Energieskala)
		\item \textbf{Räumliches Profil}: $f_{\text{norm}}(\vec{r}, t)$ (Lokalisierungsmuster)
		\item \textbf{Phasenstruktur}: $\phi(\vec{r}, t)$ (Quantenzahlen und Dynamik)
		\item \textbf{Universeller Parameter}: $\xi = 4/3 \times 10^{-4}$
	\end{itemize}
	
	\section{Universelle Feldgleichungen}
	\label{sec:universal_equations}
	
	\subsection{Einzige Gleichung für alle Teilchen}
	\label{subsec:single_equation}
	
	Statt separater Gleichungen für jeden Teilchentyp verwendet das T0-Modell eine universelle Gleichung:
	
	\begin{equation}
		\boxed{\mathcal{L} = \xi \cdot (\partial E_{\text{field}})^2}
		\label{eq:universal_lagrangian}
	\end{equation}
	
	\subsection{Antiteilchen-Vereinheitlichung}
	\label{subsec:antiparticle_unification}
	
	Die mysteriösen negativen Energie-Lösungen der Dirac-Gleichung werden zu einfachen negativen Feldanregungen:
	
	\begin{align}
		\text{Teilchen:} \quad &E_{\text{field}}(x,t) > 0 \\
		\text{Antiteilchen:} \quad &E_{\text{field}}(x,t) < 0
	\end{align}
	
	Dies eliminiert die Notwendigkeit der Loch-Theorie und liefert eine natürliche Erklärung für Teilchen-Antiteilchen-Symmetrie.
	
	\section{Experimentelle Vorhersagen}
	\label{sec:experimental_predictions}
	
	\subsection{Magnetisches Moment-Vorhersagen}
	\label{subsec:magnetic_moment_predictions}
	
	Der vereinfachte Ansatz liefert präzise experimentelle Vorhersagen:
	
	\textbf{Anomales magnetisches Moment des Myons:}
	\begin{equation}
		a_\mu^{\text{T0}} = \frac{\xi}{2\pi} \left(\frac{E_\mu}{E_e}\right)^2 = 245(12) \times 10^{-11}
	\end{equation}
	\textbf{Experimenteller Wert:} $251(59) \times 10^{-11}$ \\
	\textbf{Übereinstimmung:} $0,10\sigma$-Abweichung
	
	\subsection{Wirkungsquerschnitt-Modifikationen}
	\label{subsec:cross_section_modifications}
	
	Das T0-Framework sagt kleine aber messbare Modifikationen von Streuquerschnitten vorher:
	
	\begin{equation}
		\sigma_{\text{T0}} = \sigma_{\text{SM}} \left(1 + \xi \frac{s}{E_{\text{char}}^2}\right)
	\end{equation}
	
	wobei $s$ die Schwerpunktsenergie zum Quadrat ist.
	
	\section{Fazit: Geometrische Vereinfachung}
	\label{sec:conclusion}
	
	Das T0-Modell erreicht eine dramatische Vereinfachung durch:
	
	\begin{itemize}
		\item \textbf{Eliminierung 4×4-Matrix-Komplexität}: Einziges Energiefeld beschreibt alle Teilchen
		\item \textbf{Vereinheitlichung Teilchen und Antiteilchen}: Vorzeichen der Energiefeld-Anregung
		\item \textbf{Geometrische Grundlage}: Spin aus Feldrotation, Masse aus Energieskala
		\item \textbf{Parameterfreie Vorhersagen}: Universelle geometrische Konstante $\xi = 4/3 \times 10^{-4}$
		\item \textbf{Dimensionskonsistenz}: Ordnungsgemäße Energiefeld-Normierung durchgängig
	\end{itemize}
	
	Dies repräsentiert eine Rückkehr zur geometrischen Einfachheit bei Beibehaltung voller Kompatibilität mit experimentellen Beobachtungen.
	
	% KAPITEL 9: GEOMETRISCHE GRUNDLAGEN UND 3D-RAUM-VERBINDUNGEN
	\title{Geometrische Grundlagen und 3D-Raum-Verbindungen}
\maketitle
	\label{chap:geometric_foundations}
	
	\section{Die fundamentale geometrische Konstante}
	\label{sec:fundamental_geometric_constant}
	
	\subsection{Der exakte Wert: $\xi = 4/3 \times 10^{-4}$}
	\label{subsec:exact_value}
	
	Das T0-Modell ist durch den fundamentalen geometrischen Parameter charakterisiert:
	
	\begin{equation}
		\boxed{\xi = \frac{4}{3} \times 10^{-4} = 1,333333... \times 10^{-4}}
		\label{eq:xi_exact}
	\end{equation}
	
	Dieser Parameter repräsentiert die Verbindung zwischen physikalischen Phänomenen und dreidimensionaler Raumgeometrie.
	
	\subsection{Zerlegung der geometrischen Konstante}
	\label{subsec:decomposition}
	
	Der Parameter zerlegt sich in universelle geometrische und skalenspezifische Komponenten:
	
	\begin{align}
		\xi &= \frac{4}{3} \times 10^{-4} = G_3 \times S_{\text{Verhältnis}}
	\end{align}
	
	wobei:
	\begin{align}
		G_3 &= \frac{4}{3} \quad \text{(universeller dreidimensionaler Geometriefaktor)} \\
		S_{\text{Verhältnis}} &= 10^{-4} \quad \text{(Energieskalenverhältnis)}
	\end{align}
	
	\section{Dreidimensionale Raumgeometrie}
	\label{sec:3d_space_geometry}
	
	\subsection{Der universelle Kugelvolumenfaktor}
	\label{subsec:sphere_volume_factor}
	
	Der Faktor 4/3 entsteht aus dem Volumen einer Kugel im dreidimensionalen Raum:
	
	\begin{equation}
		V_{\text{Kugel}} = \frac{4\pi}{3} r^3
	\end{equation}
	
	\textbf{Geometrische Herleitung:}
	Der Koeffizient 4/3 erscheint als fundamentales Verhältnis, das Kugelvolumen zu kubischer Skalierung verbindet:
	
	\begin{equation}
		\frac{V_{\text{Kugel}}}{r^3} = \frac{4\pi}{3} \quad \Rightarrow \quad G_3 = \frac{4}{3}
	\end{equation}
	
	\section{Energieskalengrundlagen und Anwendungen}
	\label{sec:energy_foundations}
	
	\subsection{Labor-Skalen-Anwendungen}
	\label{subsec:laboratory_applications}
	
	\textbf{Direkt messbare Effekte} unter Verwendung von $\xi = 4/3 \times 10^{-4}$:
	
	\begin{itemize}
		\item \textbf{Anomales magnetisches Moment des Myons:}
		\begin{equation}
			a_\mu = \frac{\xi}{2\pi} \left(\frac{E_\mu}{E_e}\right)^2 = \frac{4/3 \times 10^{-4}}{2\pi} \times 42753
		\end{equation}
		
		\item \textbf{Elektromagnetische Kopplungsmodifikationen:}
		\begin{equation}
			\alpha_{\text{eff}}(E) = \alpha_0 \left(1 + \xi \ln\frac{E}{E_0}\right)
		\end{equation}
		
		\item \textbf{Wirkungsquerschnitt-Korrekturen:}
		\begin{equation}
			\sigma_{\text{T0}} = \sigma_{\text{SM}} \left(1 + G_3 \cdot S_{\text{Verhältnis}} \cdot \frac{s}{E_{\text{char}}^2}\right)
		\end{equation}
	\end{itemize}
	
	\section{Experimentelle Verifikation und Validierung}
	\label{sec:experimental_verification}
	
	\subsection{Direkt verifiziert: Laborskala}
	\label{subsec:directly_verified}
	
	\textbf{Bestätigte Messungen} unter Verwendung von $\xi = 4/3 \times 10^{-4}$:
	\begin{itemize}
		\item Myon g-2: $\xi_{\text{gemessen}} = (1,333 \pm 0,006) \times 10^{-4}$ \checkmark
		\item Labor-elektromagnetische Kopplungen \checkmark
		\item Atomare Übergangsfrequenzen \checkmark
	\end{itemize}
	
	\textbf{Präzisionsmess-Möglichkeiten:}
	\begin{itemize}
		\item Tau g-2 Messungen: $\Delta\xi/\xi \sim 10^{-3}$
		\item Ultra-präzises Elektron g-2: $\Delta\xi/\xi \sim 10^{-6}$
		\item Hochenergie-Streuung: $\Delta\xi/\xi \sim 10^{-4}$
	\end{itemize}
	
	\section{Skalenabhängige Parameter-Beziehungen}
	\label{sec:scale_dependent}
	
	\subsection{Hierarchie physikalischer Skalen}
	\label{subsec:hierarchy_scales}
	
	Der Skalenfaktor etabliert natürliche Hierarchien:
	
	\begin{table}[htbp]
		\centering
		\begin{tabular}{lccc}
			\toprule
			\textbf{Skala} & \textbf{Energie (GeV)} & \textbf{T0-Verhältnis} & \textbf{Physik-Domäne} \\
			\midrule
			Planck & $10^{19}$ & $1$ & Quantengravitation \\
			T0-Teilchen & $10^{15}$ & $10^{-4}$ & Labor-zugänglich \\
			Elektroschwach & $10^{2}$ & $10^{-17}$ & Eichvereinigung \\
			QCD & $10^{-1}$ & $10^{-20}$ & Starke Wechselwirkungen \\
			Atomar & $10^{-9}$ & $10^{-28}$ & Elektromagnetische Bindung \\
			\bottomrule
		\end{tabular}
		\caption{Energieskalenhierarchie mit T0-Verhältnissen}
		\label{tab:energy_hierarchy}
	\end{table}
	
	\subsection{Vereinheitlichtes geometrisches Prinzip}
	\label{subsec:unified_geometric_principle}
	
	Alle Skalen folgen demselben geometrischen Kopplungsprinzip:
	
	\begin{equation}
		\text{Physikalischer Effekt} = G_3 \times S_{\text{Verhältnis}} \times \text{Energiefunktion}
	\end{equation}
	
	\textbf{Skalenspezifische Anwendungen:}
	\begin{align}
		\text{Teilchen-Effekte:} \quad &E_{\text{Effekt}} = \frac{4}{3} \times 10^{-4} \times f_{\text{Teilchen}}(E) \\
		\text{Kern-Effekte:} \quad &E_{\text{Effekt}} = \frac{4}{3} \times 10^{-4} \times f_{\text{Kern}}(E)
	\end{align}
	
	\section{Mathematische Konsistenz und Verifikation}
	\label{sec:consistency_verification}
	
	\subsection{Vollständige Dimensionsanalyse}
	\label{subsec:dimensional_analysis}
	
	\begin{table}[htbp]
		\centering
		\begin{tabular}{|l|c|c|c|c|}
			\hline
			\textbf{Gleichung} & \textbf{Skala} & \textbf{Linke Seite} & \textbf{Rechte Seite} & \textbf{Status} \\
			\hline
			Teilchen g-2 & $\xi$ & $[a_\mu] = [1]$ & $[\xi/2\pi] = [1]$ & \checkmark \\
			Feldgleichung & Alle Skalen & $[\nabla^2 E] = [E^3]$ & $[G\rho E] = [E^3]$ & \checkmark \\
			Lagrange-Funktion & Alle Skalen & $[\mathcal{L}] = [E^4]$ & $[\xi(\partial E)^2] = [E^4]$ & \checkmark \\
			\hline
		\end{tabular}
		\caption{Dimensionskonsistenz-Verifikation}
		\label{tab:dim_analysis}
	\end{table}
	
	\section{Fazit und zukünftige Richtungen}
	\label{sec:conclusions_geometric}
	
	\subsection{Geometrisches Framework}
	\label{subsec:geometric_framework}
	
	Das T0-Modell etabliert:
	
	\begin{enumerate}
		\item \textbf{Laborskala}: $\xi = 4/3 \times 10^{-4}$ - experimentell verifiziert durch Myon g-2 und Präzisionsmessungen
		
		\item \textbf{Universeller geometrischer Faktor}: $G_3 = 4/3$ aus dreidimensionaler Raumgeometrie gilt auf allen Skalen
		
		\item \textbf{Klare Methodologie}: Fokus auf direkt messbare Laboreffekte
		
		\item \textbf{Parameterfreie Vorhersagen}: Alle aus einziger geometrischer Konstante
	\end{enumerate}
	
	\subsection{Experimentelle Zugänglichkeit}
	\label{subsec:experimental_accessibility}
	
	\textbf{Direkt testbar:}
	\begin{itemize}
		\item Hochpräzisions-g-2-Messungen über Teilchenarten
		\item Elektromagnetische Kopplungsevolution mit Energie
		\item Wirkungsquerschnitt-Modifikationen in Hochenergie-Streuung
		\item Atom- und Kernphysik-Korrekturen
	\end{itemize}
	
	\textbf{Fundamentalgleichung der geometrischen Physik:}
	\begin{equation}
		\boxed{\text{Physik} = f\left(\frac{4}{3}, 10^{-4}, \text{3D-Geometrie}, \text{Energieskala}\right)}
	\end{equation}
	
	Die geometrische Grundlage liefert ein mathematisch konsistentes Framework, wo Teilchenphysik-Vorhersagen direkt in Laborumgebungen getestet werden können, wobei wissenschaftliche Strenge beibehalten wird, während die fundamentale geometrische Basis der physikalischen Realität erforscht wird.
	
	% KAPITEL 10: FAZIT: EIN NEUES PHYSIK-PARADIGMA
	\title{Fazit: Ein neues Physik-Paradigma}
\maketitle
	\label{chap:conclusion}
	
	\section{Die Transformation}
	\label{sec:revolutionary_transformation}
	
	\subsection{Von Komplexität zu fundamentaler Einfachheit}
	\label{subsec:complexity_to_simplicity}
	
	Diese Arbeit hat eine Transformation in unserem Verständnis der physikalischen Realität demonstriert. Was als Untersuchung der Zeit-Energie-Dualität begann, hat sich zu einer vollständigen Neukonzeption der Physik selbst entwickelt und die gesamte Komplexität des Standardmodells auf ein einziges geometrisches Prinzip reduziert.
	
	\textbf{Die fundamentale Gleichung der Realität:}
	\begin{equation}
		\boxed{\text{Alle Physik} = f\left(\xi = \frac{4}{3} \times 10^{-4}, \text{3D-Raumgeometrie}\right)}
	\end{equation}
	
	Dies repräsentiert die tiefstmögliche Vereinfachung: die Reduktion aller physikalischen Phänomene auf Konsequenzen des Lebens in einem dreidimensionalen Universum mit sphärischer Geometrie, charakterisiert durch den exakten geometrischen Parameter $\xi = 4/3 \times 10^{-4}$.
	
	\subsection{Die Parameter-Eliminierungs-Revolution}
	\label{subsec:parameter_elimination}
	
	Der auffälligste Erfolg des T0-Modells ist die vollständige Eliminierung freier Parameter aus der fundamentalen Physik:
	
	\begin{table}[htbp]
		\centering
		\begin{tabular}{lcc}
			\toprule
			\textbf{Theorie} & \textbf{Freie Parameter} & \textbf{Vorhersagekraft} \\
			\midrule
			Standardmodell & 19+ empirische & Begrenzt \\
			Standardmodell + ART & 25+ empirische & Fragmentiert \\
			String-Theorie & $\sim 10^{500}$ Vakua & Unbestimmt \\
			T0-Modell & 0 freie & Universell \\
			\bottomrule
		\end{tabular}
		\caption{Parameter-Zähl-Vergleich über theoretische Frameworks}
		\label{tab:parameter_comparison}
	\end{table}
	
	\textbf{Parameter-Reduktions-Erfolg:}
	\begin{equation}
		\text{25+ SM+ART-Parameter} \quad \Rightarrow \quad \xi = \frac{4}{3} \times 10^{-4} \text{ (geometrisch)}
	\end{equation}
	
	Dies repräsentiert eine Faktor-25+-Reduktion in theoretischer Komplexität bei Beibehaltung oder Verbesserung experimenteller Genauigkeit.
	
	\section{Experimentelle Validierung}
	\label{sec:experimental_validation}
	
	\subsection{Der Triumph des anomalen magnetischen Moments des Myons}
	\label{subsec:muon_triumph}
	
	Der spektakulärste Erfolg des T0-Modells ist seine parameterfreie Vorhersage des anomalen magnetischen Moments des Myons:
	
	\textbf{Theoretische Vorhersage:}
	\begin{equation}
		a_\mu^{\text{T0}} = \frac{\xi}{2\pi} \left(\frac{E_\mu}{E_e}\right)^2 = 245(12) \times 10^{-11}
	\end{equation}
	
	\textbf{Experimenteller Vergleich:}
	\begin{itemize}
		\item \textbf{Experiment}: $251(59) \times 10^{-11}$
		\item \textbf{T0-Vorhersage}: $245(12) \times 10^{-11}$
		\item \textbf{Übereinstimmung}: $0,10\sigma$-Abweichung (exzellent)
		\item \textbf{Standardmodell}: $4,2\sigma$-Abweichung (problematisch)
	\end{itemize}
	
	\textbf{Verbesserungsfaktor:}
	\begin{equation}
		\text{Verbesserung} = \frac{4,2\sigma}{0,10\sigma} = 42
	\end{equation}
	
	Das T0-Modell erreicht eine 42-fache Verbesserung in theoretischer Präzision ohne empirische Parameter-Anpassung.
	
	\subsection{Universelle Lepton-Vorhersagen}
	\label{subsec:universal_lepton_predictions}
	
	Das T0-Modell macht präzise parameterfreie Vorhersagen für alle Leptonen:
	
	\textbf{Anomales magnetisches Moment des Elektrons:}
	\begin{equation}
		a_e^{\text{T0}} = \frac{\xi}{2\pi} = 2,12 \times 10^{-5}
	\end{equation}
	
	\textbf{Anomales magnetisches Moment des Taus:}
	\begin{equation}
		a_\tau^{\text{T0}} = \frac{\xi}{2\pi} \left(\frac{E_\tau}{E_e}\right)^2 = 257(13) \times 10^{-11}
	\end{equation}
	
	Diese Vorhersagen etablieren das universelle Skalierungsgesetz:
	\begin{equation}
		a_\ell^{\text{T0}} = \frac{\xi}{2\pi} \left(\frac{E_\ell}{E_e}\right)^2
	\end{equation}
	
	\section{Theoretische Errungenschaften}
	\label{sec:theoretical_achievements}
	
	\subsection{Universelle Feld-Vereinheitlichung}
	\label{subsec:universal_field_unification}
	
	Das T0-Modell erreicht vollständige Feld-Vereinheitlichung durch das universelle Energiefeld:
	
	\textbf{Feld-Reduktion:}
	\begin{equation}
		\begin{array}{c}
			\text{20+ SM-Felder} \\
			\text{4D-Raumzeit-Metrik} \\
			\text{Multiple Lagrange-Funktionen}
		\end{array} \quad \Rightarrow \quad
		\begin{array}{c}
			E_{\text{field}}(x,t) \\
			\square E_{\text{field}} = 0 \\
			\mathcal{L} = \xi \cdot (\partial E_{\text{field}})^2
		\end{array}
	\end{equation}
	
	\subsection{Geometrische Grundlage}
	\label{subsec:geometric_foundation}
	
	Alle physikalischen Wechselwirkungen entstehen aus dreidimensionaler Raumgeometrie:
	
	\textbf{Elektromagnetische Wechselwirkung:}
	\begin{equation}
		\alpha_{\text{EM}} = G_3 \times S_{\text{Verhältnis}} \times f_{\text{EM}} = \frac{4}{3} \times 10^{-4} \times f_{\text{EM}}
	\end{equation}
	
	\textbf{Schwache Wechselwirkung:}
	\begin{equation}
		\alpha_W = G_3^{1/2} \times S_{\text{Verhältnis}}^{1/2} \times f_W = \left(\frac{4}{3}\right)^{1/2} \times (10^{-4})^{1/2} \times f_W
	\end{equation}
	
	\textbf{Starke Wechselwirkung:}
	\begin{equation}
		\alpha_S = G_3^{-1/3} \times S_{\text{Verhältnis}}^{-1/3} \times f_S = \left(\frac{4}{3}\right)^{-1/3} \times (10^{-4})^{-1/3} \times f_S
	\end{equation}
	
	\subsection{Quantenmechanik-Vereinfachung}
	\label{subsec:quantum_mechanics_simplification}
	
	Das T0-Modell eliminiert die Komplexität der Standard-Quantenmechanik:
	
	\textbf{Traditionelle Quantenmechanik:}
	\begin{itemize}
		\item Wahrscheinlichkeits-Amplituden und Born-Regel
		\item Wellenfunktions-Kollaps und Messproblem
		\item Multiple Interpretationen (Kopenhagen, Viele-Welten, etc.)
		\item Komplexe 4×4-Dirac-Matrizen für relativistische Teilchen
	\end{itemize}
	
	\textbf{T0-Quantenmechanik:}
	\begin{itemize}
		\item Deterministische Energiefeld-Entwicklung: $\square E_{\text{field}} = 0$
		\item Kein Kollaps: kontinuierliche Feld-Dynamik
		\item Einzige Interpretation: Energiefeld-Anregungen
		\item Einfaches skalares Feld ersetzt Matrix-Formalismus
	\end{itemize}
	
	\textbf{Wellenfunktions-Identifikation:}
	\begin{equation}
		\psi(x,t) = \sqrt{\frac{\delta E(x,t)}{E_0 V_0}} \cdot e^{i\phi(x,t)}
	\end{equation}
	
	\section{Philosophische Implikationen}
	\label{sec:philosophical_implications}
	
	\subsection{Die Rückkehr zur pythagoreischen Physik}
	\label{subsec:pythagorean_physics}
	
	Das T0-Modell repräsentiert die ultimative Realisierung der pythagoreischen Philosophie:
	
	\begin{tcolorbox}[colback=blue!5!white,colframe=blue!75!black,title=Realisierte pythagoreische Einsicht]
		Alles ist Zahl - Pythagoras
		
		Alles ist die Zahl 4/3 - T0-Modell
		
		Jedes physikalische Phänomen reduziert sich auf Manifestationen des geometrischen Verhältnisses 4/3 aus dreidimensionaler Raumstruktur.
	\end{tcolorbox}

\bigskip
\clearpage

\end{document}