\documentclass[12pt,a4paper]{book}
% ==============================================================================
% T0 Theory: Shared English Preamble
% Version: 1.0
% Author: Johann Pascher
% Date: 2025
% ==============================================================================
%
% This is the standardized shared preamble for all English T0 Theory documents.
% Place this file in your document's directory or use a path like:
%   % ==============================================================================
% T0 Theory: Shared ENGLISH Preamble – Optimized for eBook/Book
% Version: 2.0 – Final 2026 (LuaLaTeX only) – ENGLISH corrected
% Author: Johann Pascher
% Date: January 2026
% ==============================================================================
%
% IMPORTANT: Compile EXCLUSIVELY with LuaLaTeX!
% In TeXstudio: Options → Configure TeXstudio → Build → Default Compiler → LuaLaTeX
%
% Required Fonts (install once):
% - Inter: https://fonts.google.com/specimen/Inter
% - JetBrains Mono: https://www.jetbrains.com/lp/mono/
% - Libertinus Math: https://github.com/libertinus-fonts/libertinus
% ==============================================================================

% === CHAPTER 1: BASIC PACKAGES (must come FIRST) ===
\RequirePackage{fontspec}
\RequirePackage{unicode-math}
\usepackage{chngcntr}
\setcounter{secnumdepth}{1}  % Nur Sections nummerieren (nicht subsections)
\setcounter{tocdepth}{1}     % Nur Sections im TOC (nicht subsections)
\makeatletter
\@ifundefined{c@chapter}{}{\counterwithout{section}{chapter}}  % Falls Kapitel existieren
\makeatother
\counterwithout{subsection}{section}  % Löse Verknüpfung
% === CHAPTER 2: LANGUAGE (ENGLISH) ===
\usepackage[english]{babel}
\usepackage{microtype}                    % IMPORTANT for better hyphenation!

% Typography settings for better line breaking
\frenchspacing                     % Correct English spacing after punctuation
\emergencystretch=3em              % Allows more stretch for difficult lines
\tolerance=2500                    % Higher tolerance for line breaks
\hbadness=10000                    % Suppresses "underfull hbox" warnings
\hfuzz=2pt                         % Allows minimal overfull
\pretolerance=150                  % Better word breaking

% Prevent bad page breaks
\clubpenalty=10000           % No "orphans"
\widowpenalty=10000          % No "widows"
\displaywidowpenalty=10000   % Also with equations
\brokenpenalty=10000         % No broken words across pages

% Explicit hyphenation for long technical words
\hyphenation{Fun-da-men-tal Frac-tal-Ge-o-met-ric Field The-o-ry Meth-od-o-log-i-cal}
\hyphenation{Re-vi-sion-ism Quan-ti-za-tion U-ni-fi-ca-tion Ef-fec-tive}
\hyphenation{Re-nor-mal-iz-a-bil-i-ty Sin-gu-lar-i-ties Con-cil-i-a-tion}
\hyphenation{E-mer-gence Phe-nom-e-no-log-i-cal Doc-u-men-ta-tion A-nal-y-sis}
\hyphenation{Grav-i-ta-tion Quan-tum Me-chan-ics Dog-ma-tism Con-se-quent}
\hyphenation{Par-al-lel-ism Im-ple-men-ta-tion Per-tur-ba-tions}
\hyphenation{Geo-met-ric Ar-ti-fact In-com-pat-i-bil-i-ty Con-struc-tive}
\hyphenation{Frac-tal Di-men-sion-less In-ves-ti-ga-tion De-scrip-tion}
\hyphenation{In-ter-pre-ta-tion Phe-nom-e-no-log-i-cal Math-e-mat-i-cal}
\hyphenation{Phi-lo-soph-i-cal Le-git-i-ma-tion Ap-pli-ca-tion Der-i-va-tion}
\hyphenation{U-ni-fi-ca-tion As-sump-tion Con-cep-tion Ex-pec-ta-tion}
\hyphenation{Sym-me-try-ex-ten-sion O-ver-all-pic-ture Chal-lenge}
\hyphenation{In-ter-ac-tion Ma-te-ri-al Ap-proach Per-spec-tive Pro-ce-dure}

% === CHAPTER 3: FONTS (with proper ligatures) ===
\setmainfont{Inter}[
Scale=1.02,
UprightFont=*-Regular,
BoldFont=*-Bold,
ItalicFont=*-Italic,
BoldItalicFont=*-BoldItalic,
Ligatures=TeX,           % IMPORTANT for proper typography
Language=English         % Explicit language support
]
\setsansfont{Inter}[
Scale=MatchLowercase,
Ligatures=TeX,
Language=English
]
\setmonofont{JetBrains Mono}[
Scale=0.95,
Language=English
]

% Math Font (simple & stable) – MUST come AFTER language definition
% IMPORTANT: Libertinus Math for correct \underbrace display!
\setmathfont{Libertinus Math}[Scale=1.0]

% === CHAPTER 4: MATHEMATICS PACKAGES (in STRICT order!) ===
% IMPORTANT: mathtools must come BEFORE unicode-math for some commands!
\usepackage{mathtools}           % FIRST mathtools!

% Then the rest
\usepackage{amsmath, amsfonts, amsthm}

% SIUNITX MUST be loaded BEFORE physics!
\usepackage{siunitx}
\sisetup{
	locale=US,                    % ENGLISH settings for SI units!
	group-separator={,},          % Thousands separator comma
	output-decimal-marker={.},    % Decimal separator point
	per-mode=symbol,
	separate-uncertainty=true
}

% Custom SI units used in narrative and books
\DeclareSIUnit\gigalightyear{Gly}
\DeclareSIUnit\mev{MeV}

% physics – MUST be loaded AFTER siunitx and mathtools
\usepackage{physics}

% === CHAPTER 5: ADDITIONS from pdflatex best practices ===
\usepackage{colortbl}        % Colored tables (ESSENTIAL!)
\usepackage{placeins}        % Float control: \FloatBarrier
\usepackage{subcaption}      % Subfigures
\usepackage{xurl}            % Better URL line breaking
% Hyphenation for URLs in bibliography
\def\UrlBreaks{\do\/\do-}

% === CHAPTER 6: PAGE LAYOUT
% =============================================================================
% SECTION 2: Page Geometry – 6" × 9" Buchformat
% =============================================================================
\usepackage[paperwidth=6in, paperheight=9in,
top=0.9in,
bottom=1.1in,
inner=0.9in,            % Größerer Innenrand für Bindung
outer=0.6in,            % Kleinerer Außenrand → mehr Text pro Seite
bindingoffset=0.5in,    % Puffer für Bindung (Steg)
twoside]{geometry}
\setlength{\headheight}{15pt}
%\usepackage[paperwidth=8.25in, paperheight=11in,
%top=1.0in,
%bottom=1.0in,
%left=1.0in,
%right=1.0in,
%twoside=false
% === CHAPTER 7: GRAPHICS AND TABLES ===
\usepackage{graphicx}
\usepackage[table,xcdraw]{xcolor}
% T0 brand colors
\definecolor{gold}{RGB}{255,215,0}
\definecolor{blue}{rgb}{0,0,1}
\definecolor{boxgray}{RGB}{240,240,240}
\definecolor{deepblue}{RGB}{0,0,127}
\definecolor{deepgreen}{RGB}{0,127,0}
\definecolor{deepred}{RGB}{191,0,0}
\definecolor{t0blue}{RGB}{33,150,243}
\definecolor{t0green}{RGB}{76,175,80}
\definecolor{t0orange}{RGB}{255,152,0}
\definecolor{t0purple}{RGB}{156,39,176}
\definecolor{t0red}{RGB}{244,67,54}
\definecolor{t0yellow}{RGB}{255,204,0}
\usepackage{tikz}
\usetikzlibrary{arrows.meta,positioning,shapes.geometric,decorations.pathmorphing,patterns,shapes.arrows,intersections}
\usepackage{pgfplots}
\pgfplotsset{compat=1.18}
\usepackage{quantikz}
\usepackage[most]{tcolorbox}
\tcbuselibrary{breakable}

% === WICHTIG: Algorithm-Konflikt umgehen ===
% Option: algorithmic mit GROSSBUCHSTABEN
% Gemeinsame Box für Experimente
\newtcolorbox{experimentbox}[1][]{
	colback=green!5!white,
	colframe=t0green!80!black,
	fonttitle=\bfseries,
	title={{#1}},
	breakable
}

% Abstract-Fallback
\ifdefined\abstract\else
\newenvironment{abstract}{\section*{\abstractname}\itshape\small\par\bigskip}{\bigskip}
\fi

% === MAKROS SICHER NEU DEFINIEREN / ÜBERSCHREIBEN ===
% Definiere Makros OHNE doppelte Subskripte
\newcommand{\phipar}{\phi_{\mathrm{par}}}
%\newcommand{\xipar}{\xi_{\mathrm{par}}}
\newcommand{\Qphipar}{Q_{\phi_{\mathrm{par}}}}
\newcommand{\rphipar}{r_{\phi_{\mathrm{par}}}}
\newcommand{\logphipar}{\log_{\phi_{\mathrm{par}}}}
\newcommand{\CHSH}{\text{CHSH}}
\usepackage{booktabs}
\usepackage{array}
\usepackage{longtable}
\usepackage{float}
\usepackage{adjustbox}
\usepackage{rotating}
\usepackage{tabularx}
\usepackage{makecell}
\usepackage{multirow}

% === CHAPTER 8: DOCUMENT FORMATTING ===
\usepackage{fancyhdr}
\renewcommand{\headrulewidth}{0.4pt}
\renewcommand{\footrulewidth}{0.4pt}
\usepackage{tocloft}

\usepackage{enumitem}
\setlist[itemize]{leftmargin=*, topsep=2pt, partopsep=0pt, parsep=2pt, itemsep=2pt}
\setlist[enumerate]{leftmargin=*, topsep=2pt, partopsep=0pt, parsep=2pt, itemsep=2pt}
\usepackage{setspace}
\usepackage{ragged2e}
\usepackage{multicol}

% === CHAPTER 9: CODE AND ALGORITHMS ===
\usepackage{algorithm}
\usepackage{algorithmic}
\usepackage{listings}
\lstset{
	basicstyle=\ttfamily\footnotesize,
	breaklines=true,
	breakatwhitespace=true,
	columns=flexible,
	keepspaces=true,
	showstringspaces=false,
	frame=single,
	xleftmargin=0pt,
	xrightmargin=0pt,
	literate=              % For special characters in code listings
	{ä}{{\"a}}1 {ö}{{\"o}}1 {ü}{{\"u}}1 {ß}{{\ss}}1
	{Ä}{{\"A}}1 {Ö}{{\"O}}1 {Ü}{{\"U}}1
}
\usepackage{mdframed}

% === CHAPTER 10: ADDITIONAL PACKAGES ===
\usepackage{pdflscape}
\usepackage{braket}
\usepackage{cancel}
\usepackage{caption}
\captionsetup{format=plain, labelfont=bf, justification=centering}
\usepackage{csquotes}
\usepackage{gensymb}
\usepackage{textcomp}
\usepackage{textgreek}
\usepackage{upgreek}
\usepackage{url}
\usepackage{slashed}
\usepackage{bm}

% === CHAPTER 11: HYPERREF (must come SECOND TO LAST!) ===
\usepackage{hyperref}
\hypersetup{
	colorlinks=true,
	linkcolor=black,
	citecolor=black,
	urlcolor=black,
	breaklinks=true,           % IMPORTANT for special characters in URLs!
	bookmarksnumbered=true,
	unicode=true,
	pdfencoding=auto,
	pdflang=en,                % Set PDF language to English
	pdfsubject={T0 Theory - Fundamental Fractal-Geometric Field Theory}
}

% Fix for unicode-math symbols in PDF bookmarks
\pdfstringdefDisableCommands{%
	\def\xi{xi}%
	\def\alpha{alpha}%
	\def\beta{beta}%
	\def\gamma{gamma}%
	\def\delta{delta}%
	\def\Delta{Delta}%
	\def\epsilon{epsilon}%
	\def\varepsilon{epsilon}%
	\def\theta{theta}%
	\def\kappa{kappa}%
	\def\lambda{lambda}%
	\def\mu{mu}%
	\def\nu{nu}%
	\def\pi{pi}%
	\def\rho{rho}%
	\def\sigma{sigma}%
	\def\tau{tau}%
	\def\phi{phi}%
	\def\chi{chi}%
	\def\psi{psi}%
	\def\omega{omega}%
	\def\Omega{Omega}%
	\def\Lambda{Lambda}%
	\def\times{x}%
	\def\cdot{*}%
	\def\pm{+/-}%
	\def\approx{~}%
	\def\sim{~}%
	\def\equiv{=}%
	\def\ell{l}%
	\def\hbar{h}%
	\def\rightarrow{->}%
	\def\leftarrow{<-}%
	\def\Rightarrow{=>}%
	\def\Leftarrow{<=}%
	\def\propto{~}%
	\def\mitxi{xi}%
	\def\mitalpha{alpha}%
	\def\mitbeta{beta}%
	\def\mitgamma{gamma}%
	\def\mitdelta{delta}%
	\def\mitDelta{Delta}%
	\def\mitepsilon{epsilon}%
	\def\mitvarepsilon{epsilon}%
	\def\mittheta{theta}%
	\def\mitkappa{kappa}%
	\def\mitlambda{lambda}%
	\def\mitLambda{Lambda}%
	\def\mitmu{mu}%
	\def\mitnu{nu}%
	\def\mitpi{pi}%
	\def\mitrho{rho}%
	\def\mitsigma{sigma}%
	\def\mittau{tau}%
	\def\mitphi{phi}%
	\def\mitchi{chi}%
	\def\mitpsi{psi}%
	\def\mitomega{omega}%
	\def\mitOmega{Omega}%
}

% === CHAPTER 12: BOOKMARK (must come AFTER hyperref!) ===
\usepackage{bookmark}

% === CHAPTER 13: CLEVEREF (ENGLISH LABELS) ===
\usepackage[english]{cleveref}
\crefname{equation}{Equation}{Equations}
\crefname{figure}{Figure}{Figures}
\crefname{table}{Table}{Tables}
\crefname{section}{Section}{Sections}
\crefname{chapter}{Chapter}{Chapters}
\crefname{theorem}{Theorem}{Theorems}
\crefname{lemma}{Lemma}{Lemmas}
\crefname{definition}{Definition}{Definitions}
\crefname{example}{Example}{Examples}
\crefname{remark}{Remark}{Remarks}

% === CUSTOM ENVIRONMENTS ===
% Alternative interpretation environment
\newenvironment{alternative}{%
	\begin{mdframed}[linecolor=black!30,linewidth=1pt,roundcorner=4pt,backgroundcolor=black!5]%
	}{%
	\end{mdframed}%
}

% Photon/particle environment
\newenvironment{photon}{%
	\begin{mdframed}[linecolor=blue!30,linewidth=1pt,roundcorner=4pt,backgroundcolor=blue!5]%
	}{%
	\end{mdframed}%
}

% Koide formula box environment
\newenvironment{koidebox}{%
	\begin{mdframed}[linecolor=green!30,linewidth=1pt,roundcorner=4pt,backgroundcolor=green!5]%
	}{%
	\end{mdframed}%
}

% Erkenntnis/insight environment
\newenvironment{erkenntnis}{%
	\begin{mdframed}[linecolor=orange!30,linewidth=1pt,roundcorner=4pt,backgroundcolor=orange!5]%
	}{%
	\end{mdframed}%
}

% Beziehung/relationship environment
\newenvironment{beziehung}{%
	\begin{mdframed}[linecolor=purple!30,linewidth=1pt,roundcorner=4pt,backgroundcolor=purple!5]%
	}{%
	\end{mdframed}%
}

% Derivation environment
\newenvironment{derivation}{%
	\begin{mdframed}[linecolor=teal!30,linewidth=1pt,roundcorner=4pt,backgroundcolor=teal!5]%
	}{%
	\end{mdframed}%
}

% Abhandlung/treatise environment
\newenvironment{abhandlung}{%
	\begin{mdframed}[linecolor=brown!30,linewidth=1pt,roundcorner=4pt,backgroundcolor=brown!5]%
	}{%
	\end{mdframed}%
}

% Anwendung/application environment
\newenvironment{anwendung}{%
	\begin{mdframed}[linecolor=cyan!30,linewidth=1pt,roundcorner=4pt,backgroundcolor=cyan!5]%
	}{%
	\end{mdframed}%
}

% Additional common environments
\newenvironment{konsequenz}{%
	\begin{mdframed}[linecolor=red!30,linewidth=1pt,roundcorner=4pt,backgroundcolor=red!5]%
	}{%
	\end{mdframed}%
}

\newenvironment{schlussfolgerung}{%
	\begin{mdframed}[linecolor=gray!30,linewidth=1pt,roundcorner=4pt,backgroundcolor=gray!5]%
	}{%
	\end{mdframed}%
}

\newenvironment{result}{%
	\begin{mdframed}[linecolor=violet!30,linewidth=1pt,roundcorner=4pt,backgroundcolor=violet!5]%
	}{%
	\end{mdframed}%
}

% Formula environment
\newenvironment{formula}{%
	\begin{mdframed}[linecolor=yellow!30,linewidth=1pt,roundcorner=4pt,backgroundcolor=yellow!5]%
	}{%
	\end{mdframed}%
}

% Revolutionaer/revolutionary environment
\newenvironment{revolutionaer}{%
	\begin{mdframed}[linecolor=red!50,linewidth=2pt,roundcorner=4pt,backgroundcolor=red!10]%
	}{%
	\end{mdframed}%
}

% Formel environment (German version of formula)
\newenvironment{formel}{%
	\begin{mdframed}[linecolor=yellow!30,linewidth=1pt,roundcorner=4pt,backgroundcolor=yellow!5]%
	}{%
	\end{mdframed}%
}

% Prinzip/principle environment
\newenvironment{prinzip}{%
	\begin{mdframed}[linecolor=blue!50,linewidth=2pt,roundcorner=4pt,backgroundcolor=blue!10]%
	}{%
	\end{mdframed}%
}

% Experimentell/experimental environment
\newenvironment{experimentell}{%
	\begin{mdframed}[linecolor=magenta!30,linewidth=1pt,roundcorner=4pt,backgroundcolor=magenta!5]%
	}{%
	\end{mdframed}%
}

% Neutrino environment
\newenvironment{neutrino}{%
	\begin{mdframed}[linecolor=cyan!40,linewidth=1pt,roundcorner=4pt,backgroundcolor=cyan!8]%
	}{%
	\end{mdframed}%
}

% Additional missing environments
\newenvironment{schluessel}{%
	\begin{mdframed}[linecolor=yellow!50,linewidth=1pt,roundcorner=4pt,backgroundcolor=yellow!10]%
	}{%
	\end{mdframed}%
}

\newenvironment{summary}{%
	\begin{mdframed}[linecolor=gray!40,linewidth=1pt,roundcorner=4pt,backgroundcolor=gray!8]%
	}{%
	\end{mdframed}%
}

\newenvironment{category}{%
	\begin{mdframed}[linecolor=pink!40,linewidth=1pt,roundcorner=4pt,backgroundcolor=pink!8]%
	}{%
	\end{mdframed}%
}

\newenvironment{sibox}{%
	\begin{mdframed}[linecolor=lime!40,linewidth=1pt,roundcorner=4pt,backgroundcolor=lime!8]%
	}{%
	\end{mdframed}%
}

% More missing environments
\newenvironment{documentbox}{%
	\begin{mdframed}[linecolor=teal!40,linewidth=1pt,roundcorner=4pt,backgroundcolor=teal!8]%
	}{%
	\end{mdframed}%
}

\newenvironment{t0box}{%
	\begin{mdframed}[linecolor=violet!40,linewidth=1pt,roundcorner=4pt,backgroundcolor=violet!8]%
	}{%
	\end{mdframed}%
}

\newenvironment{wichtig}{%
	\begin{mdframed}[linecolor=red!50,linewidth=2pt,roundcorner=4pt,backgroundcolor=red!10]%
	\textbf{Important:} 
	}{%
	\end{mdframed}%
}

\newenvironment{smbox}{%
	\begin{mdframed}[linecolor=orange!40,linewidth=1pt,roundcorner=4pt,backgroundcolor=orange!8]%
	}{%
	\end{mdframed}%
}

\newenvironment{pvbox}{%
	\begin{mdframed}[linecolor=purple!40,linewidth=1pt,roundcorner=4pt,backgroundcolor=purple!8]%
	}{%
	\end{mdframed}%
}

\newenvironment{numerisch}{%
	\begin{mdframed}[linecolor=blue!40,linewidth=1pt,roundcorner=4pt,backgroundcolor=blue!8]%
	}{%
	\end{mdframed}%
}

% More missing environments
\newenvironment{relation}{%
	\begin{mdframed}[linecolor=green!40,linewidth=1pt,roundcorner=4pt,backgroundcolor=green!8]%
	}{%
	\end{mdframed}%
}

\newenvironment{beweis}{%
	\begin{mdframed}[linecolor=brown!40,linewidth=1pt,roundcorner=4pt,backgroundcolor=brown!8]%
	\textbf{Proof:} 
	}{%
	\end{mdframed}%
}

\newenvironment{revolution}{%
	\begin{mdframed}[linecolor=red!60,linewidth=2pt,roundcorner=4pt,backgroundcolor=red!12]%
	}{%
	\end{mdframed}%
}

\newenvironment{key}{%
	\begin{mdframed}[linecolor=yellow!50,linewidth=1pt,roundcorner=4pt,backgroundcolor=yellow!10]%
	}{%
	\end{mdframed}%
}

\newenvironment{newperspective}{%
	\begin{mdframed}[linecolor=cyan!50,linewidth=1pt,roundcorner=4pt,backgroundcolor=cyan!10]%
	}{%
	\end{mdframed}%
}

\newenvironment{literatur}{%
	\begin{mdframed}[linecolor=gray!50,linewidth=1pt,roundcorner=4pt,backgroundcolor=gray!10]%
	}{%
	\end{mdframed}%
}

\newenvironment{folgerung}{%
	\begin{mdframed}[linecolor=teal!50,linewidth=1pt,roundcorner=4pt,backgroundcolor=teal!10]%
	}{%
	\end{mdframed}%
}

\newenvironment{principle}{%
	\begin{mdframed}[linecolor=blue!60,linewidth=2pt,roundcorner=4pt,backgroundcolor=blue!12]%
	}{%
	\end{mdframed}%
}

% Additional common environments
% ==============================================================================
% FROM HERE: YOUR DEFINITIONS (unchanged)
% ==============================================================================

\setcounter{tocdepth}{3}

% === CITATION COMMANDS ===
\providecommand{\citep}[1]{\cite{#1}}
\providecommand{\citet}[1]{\cite{#1}}

% === COLORS ===
\definecolor{gold}{RGB}{255,215,0}
\definecolor{blue}{rgb}{0,0,1}
\definecolor{boxgray}{RGB}{240,240,240}
\definecolor{deepblue}{RGB}{0,0,127}
\definecolor{deepgreen}{RGB}{0,127,0}
\definecolor{deepred}{RGB}{191,0,0}
\definecolor{t0blue}{RGB}{33,150,243}
\definecolor{t0green}{RGB}{76,175,80}
\definecolor{t0orange}{RGB}{255,152,0}
\definecolor{t0purple}{RGB}{156,39,176}
\definecolor{t0red}{RGB}{244,67,54}
\definecolor{t0yellow}{RGB}{255,204,0}

% === COLUMN TYPES ===
\newcolumntype{L}[1]{>{\raggedright\arraybackslash}p{#1}}
\newcolumntype{C}[1]{>{\centering\arraybackslash}p{#1}}
\newcolumntype{R}[1]{>{\raggedleft\arraybackslash}p{#1}}

% === HYPERREF SETTINGS (updated) ===
\hypersetup{
	colorlinks=true,
	linkcolor=t0blue,
	citecolor=t0blue,
	urlcolor=t0blue,
	breaklinks=true,
	bookmarksnumbered=true,
	pdfstartview=FitH,
	pdfencoding=auto,
	pdfdisplaydoctitle=true
}

% === ENGLISH THEOREM ENVIRONMENTS ===
\theoremstyle{plain}
\newtheorem{theorem}{Theorem}[section]
\newtheorem{lemma}[theorem]{Lemma}
\newtheorem{proposition}[theorem]{Proposition}
\newtheorem{corollary}[theorem]{Corollary}

\theoremstyle{definition}
\newtheorem{definition}[theorem]{Definition}
\newtheorem{example}[theorem]{Example}
\newtheorem{insight}[theorem]{Insight}
\newtheorem{discovery}[theorem]{Discovery}

\theoremstyle{remark}
\newtheorem{remark}[theorem]{Remark}
\newtheorem{axiom}{Axiom}
%\newtheorem{principle}{Principle}  % Commented out to avoid conflicts with document-specific definitions
%\newtheorem{warning}[theorem]{Warning}

% === T0-SPECIFIC COMMANDS ===
% (Here follow all your \newcommand and \providecommand definitions)
% These remain UNCHANGED as in your original preamble
% ==============================================================================
% SECTION 14: T0-Specific Commands
% ==============================================================================

% --- Core T0 Fields ---
\newcommand{\Tfield}{T(x,t)}
\providecommand{\Tfieldt}{T(\vec{x},t)}
\newcommand{\Efield}{E(x,t)}
\newcommand{\mfield}{m(x,t)}
\providecommand{\vecx}{\vec{x}}

% --- Lagrangian ---
\newcommand{\Lag}{\mathcal{L}}
\newcommand{\calL}{\mathcal{L}}

% --- Greek Letters and Constants ---
\newcommand{\alphaem}{\alpha}
\newcommand{\betaT}{\beta_T}
\newcommand{\xiT}{\xi}
\newcommand{\xipar}{\xi}

% --- Energy and Planck Units ---
\newcommand{\Ezero}{E_0}
\newcommand{\E}{E}
\newcommand{\EPlanck}{E_{\text{Pl}}}
\newcommand{\Mpl}{M_{\text{Pl}}}
\newcommand{\mP}{m_{\text{P}}}
\newcommand{\lP}{\ell_{\text{P}}}
\newcommand{\tP}{t_{\text{P}}}
\newcommand{\LPlanck}{\ell_{\text{Pl}}}
\newcommand{\TPlanck}{t_{\text{Pl}}}

% --- Coupling Constants ---
\newcommand{\Gnat}{G_{\text{nat}}}
\newcommand{\alphaEM}{\alpha_{\text{EM}}}
\newcommand{\alphaSI}{\alpha_{\text{SI}}}
\newcommand{\Hubble}{H_0}
\newcommand{\LCDM}{\Lambda\text{CDM}}
\newcommand{\natunits}{(nat. units)}

% --- T0 Model Parameters ---
\newcommand{\xigeom}{\xi_{\mathrm{geom}}}
\newcommand{\rzero}{r_{0}}
\newcommand{\xirat}{\xi_{\mathrm{rat}}}
\newcommand{\tzero}{t_{0}}
\newcommand{\Lambdat}{\Lambda_{\mathrm{t}}}
\newcommand{\EP}{E_{\text{P}}}
\newcommand{\Emu}{E_{\mu}}
\newcommand{\Ee}{E_{e}}
\newcommand{\Etau}{E_{\tau}}
\newcommand{\alphafine}{\alpha_{\mathrm{fine}}}
\newcommand{\alphal}{\alpha_{\ell}}
\newcommand{\Lzero}{\ell_{0}}
\newcommand{\Lp}{\ell_{\mathrm{P}}}

% --- Additional T0 Commands ---
\newcommand{\Kfrak}{K_{\text{frak}}}
\newcommand{\Dfrak}{D_{\text{frak}}}
\newcommand{\betapar}{\ensuremath{\beta_T}}
\newcommand{\alphapar}{\alpha}
\newcommand{\deltafield}{\delta \phi}
\newcommand{\deltam}{\delta m}
\newcommand{\deltaE}{\delta E}
\newcommand{\Exi}{E_{\xi}}
\newcommand{\Lxi}{\ell_{\xi}}
\newcommand{\rhoCMB}{\rho_{\text{CMB}}}
\newcommand{\rhoCasimir}{\rho_{\text{Casimir}}}
\newcommand{\Leff}{L_{\text{eff}}}
\newcommand{\CQCD}{C_{\mathrm{QCD}}}
\newcommand{\Kspec}{K_{\mathrm{spec}}}
\newcommand{\Tzero}{\ensuremath{T_0}}
\newcommand{\Eabs}{E_{\text{abs}}}
\newcommand{\taupar}{\tau}

% --- Provided Commands ---
\providecommand{\xiconst}{\xi_{\text{const}}}
\providecommand{\DhiggsT}{D_{\text{Higgs-T}}}
\providecommand{\rhoE}{\rho_{E}}
\providecommand{\Echar}{E_{\text{char}}}
\providecommand{\kfrac}{k_{\text{frac}}}
\providecommand{\alphaEMSI}{\alpha_{\text{EM,SI}}}
\providecommand{\alphaEMnat}{\alpha_{\text{EM,nat}}}
\providecommand{\betaTSI}{\beta_{T,\text{SI}}}
\providecommand{\betaTnat}{\beta_{T,\text{nat}}}
\providecommand{\Gsi}{G_{\text{SI}}}
\providecommand{\xiparSI}{\xi_{\text{SI}}}
\providecommand{\xiparnat}{\xi_{\text{nat}}}
\providecommand{\meff}{m_{\text{eff}}}
\providecommand{\Tzerot}{T_{0}(t)}
\providecommand{\mzerot}{m_{0}(t)}
\providecommand{\Ezeroabs}{E_{0,\text{abs}}}
\providecommand{\Epar}{E_{\text{par}}}
\providecommand{\Lnat}{\ell_{\text{nat}}}
\providecommand{\Tnat}{T_{\text{nat}}}
\providecommand{\xifrak}{\xi_{\text{frac}}}
\providecommand{\Tfrak}{T_{\text{frac}}}
\providecommand{\mfrak}{m_{\text{frac}}}
\providecommand{\Dfrac}{D_{\text{frac}}}
\providecommand{\EphotSI}{E_{\gamma,\text{SI}}}
\providecommand{\EphotNat}{E_{\gamma,\text{nat}}}
\providecommand{\Eabsint}{E_{\text{abs,int}}}
\providecommand{\mphoton}{m_{\gamma}}
\providecommand{\Evis}{E_{\text{vis}}}
\providecommand{\Cto}{C_{T0}}
\providecommand{\mytimes}{\times}
\providecommand{\lambdah}{\lambda_h}
\providecommand{\checkmarkx}{\checkmark}
\providecommand{\Enorm}{E_{\text{norm}}}
\providecommand{\Tobs}{T_{\text{obs}}}
\providecommand{\mobs}{m_{\text{obs}}}
\providecommand{\Eobs}{E_{\text{obs}}}
\providecommand{\Lobs}{\ell_{\text{obs}}}
\providecommand{\xobs}{\xi_{\text{obs}}}
\providecommand{\calE}{\mathcal{E}}
\providecommand{\calT}{\mathcal{T}}
\providecommand{\calM}{\mathcal{M}}
\providecommand{\alphag}{\alpha_g}
\providecommand{\Tmax}{T_{\text{max}}}
\providecommand{\mmin}{m_{\text{min}}}
\providecommand{\Lmax}{\ell_{\text{max}}}
\providecommand{\Emin}{E_{\text{min}}}
\providecommand{\Geff}{G_{\text{eff}}}
\providecommand{\rhoeff}{\rho_{\text{eff}}}
\providecommand{\xieff}{\xi_{\text{eff}}}
\providecommand{\Teff}{T_{\text{eff}}}
\providecommand{\hPlanck}{h}
\providecommand{\kB}{k_B}
\providecommand{\muB}{\mu_B}
\providecommand{\lambdaC}{\lambda_C}
\providecommand{\omegaP}{\omega_P}
\providecommand{\rhoP}{\rho_P}
\providecommand{\Tref}{T_{\text{ref}}}
\providecommand{\Eref}{E_{\text{ref}}}
\providecommand{\mref}{m_{\text{ref}}}
\providecommand{\Lref}{\ell_{\text{ref}}}
\providecommand{\xikonst}{\xi_0}
\providecommand{\Phiphoton}{\Phi_{\gamma}}
\providecommand{\etavis}{\eta_{\text{vis}}}
\providecommand{\pichar}{\pi}
\providecommand{\primrel}{\mathcal{P}_{\text{rel}}}
\providecommand{\warningx}{\textcolor{orange}{\textbf{!}}}
\providecommand{\phiT}{\phi_T}
\providecommand{\Lorentz}{\Lambda}
\providecommand{\Cconv}{C_{\text{conv}}}
\providecommand{\Df}{\Delta f}
\providecommand{\lambdazero}{\lambda_0}
\providecommand{\myapprox}{\approx}
\providecommand{\checked}{\checkmark}
\providecommand{\alphaWSI}{\alpha_W^{\text{SI}}}
\providecommand{\alphaWnat}{\alpha_W^{\text{nat}}}
\providecommand{\vect}[1]{\vec{#1}}
\providecommand{\Rzero}{R_0}
\providecommand{\Riem}{\mathcal{R}}
\providecommand{\nuzero}{\nu_0}
\providecommand{\mypi}{\pi}

% =============================================================================
% TCOLORBOX STYLES AND ENVIRONMENTS (English titles)
% =============================================================================
\tcbset{
	keyresult/.style={
		colback=blue!5!white,
		colframe=blue!75!black,
		title=Key Result,
		fonttitle=\bfseries
	},
	foundation/.style={
		colback=green!5!white,
		colframe=green!75!black,
		title=Foundation,
		fonttitle=\bfseries
	},
	alternative/.style={
		colback=orange!5!white,
		colframe=orange!75!black,
		title=Alternative,
		fonttitle=\bfseries
	},
	warningbox/.style={
		colback=red!5!white,
		colframe=red!75!black,
		title=Warning,
		fonttitle=\bfseries
	}
}

% (Here follow all your tcolorbox definitions with English titles)
\newtcolorbox{keyresultbox}[1][]{colback=blue!5!white,colframe=blue!75!black,fonttitle=\bfseries,title={#1},breakable}
\newtcolorbox{keyresult}[1][Key Result]{colback=blue!5!white,colframe=blue!75!black,fonttitle=\bfseries,title={#1},breakable}
\newtcolorbox{foundationbox}[1][]{colback=green!5!white,colframe=green!75!black,fonttitle=\bfseries,title={#1},breakable}
\newtcolorbox{foundation}[1][Foundation]{colback=green!5!white,colframe=green!75!black,fonttitle=\bfseries,title={#1},breakable}
\newtcolorbox{alternativebox}[1][]{colback=orange!5!white,colframe=orange!75!black,fonttitle=\bfseries,title={#1},breakable}
\newtcolorbox{warningboxenv}[1][Warning]{colback=red!5!white,colframe=red!75!black,fonttitle=\bfseries,title={#1},breakable}

\newtcolorbox{fundamental}[1][]{
	colback=boxgray,
	colframe=t0blue,
	fonttitle=\bfseries,
	title=#1,
	sharp corners,
	boxrule=2pt
}

\newtcolorbox{insightBox}[1][Insight]{colback=blue!5,colframe=t0blue,title={#1},fonttitle=\bfseries,breakable}
\newtcolorbox{discoveryBox}[1][Discovery]{colback=green!5,colframe=t0green,title={#1},fonttitle=\bfseries,breakable}
\newtcolorbox{revelation}[1][Revelation]{colback=red!5,colframe=t0red,title={#1},fonttitle=\bfseries,breakable}
\newtcolorbox{keypoint}[1][Key Point]{colback=blue!5,colframe=t0blue,title={#1},fonttitle=\bfseries,breakable}
\newtcolorbox{evidence}[1][Evidence]{colback=green!5,colframe=t0green,title={#1},fonttitle=\bfseries,breakable}
\newtcolorbox{conclusionBox}[1][Conclusion]{colback=gray!5,colframe=gray,title={#1},fonttitle=\bfseries,breakable}
\newtcolorbox{significance}[1][Significance]{colback=yellow!5,colframe=orange,title={#1},fonttitle=\bfseries,breakable}
\newtcolorbox{philosophical}[1][Philosophical]{colback=purple!5,colframe=purple,title={#1},fonttitle=\bfseries,breakable}
\newtcolorbox{implicationBox}[1][Implication]{colback=cyan!5,colframe=cyan,title={#1},fonttitle=\bfseries,breakable}
\newtcolorbox{perspectiveBox}[1][Perspective]{colback=blue!5,colframe=t0blue,title={#1},fonttitle=\bfseries,breakable}
\newtcolorbox{revolutionary}[1][Revolutionary]{colback=red!5,colframe=t0red,title={#1},fonttitle=\bfseries,breakable}

\newtcolorbox{technical}[1][Technical]{colback=gray!5,colframe=gray!75!black,title={#1},fonttitle=\bfseries,breakable}
\newtcolorbox{technicalBox}[1][Technical]{colback=gray!5,colframe=gray!75!black,title={#1},fonttitle=\bfseries,breakable}
\newtcolorbox{notationBox}[1][Notation]{colback=yellow!5,colframe=yellow!75!black,title={#1},fonttitle=\bfseries,breakable}
\newtcolorbox{verification}[1][Verification]{colback=orange!5!white,colframe=orange!75!black,fonttitle=\bfseries,title=#1}
\newtcolorbox{explanationBox}[1][Explanation]{colback=purple!5!white,colframe=purple!75!black,fonttitle=\bfseries,title=#1}
\newtcolorbox{interpretationBox}[1][Interpretation]{colback=cyan!5!white,colframe=cyan!75!black,fonttitle=\bfseries,title=#1}
\newtcolorbox{explanation}[1][Explanation]{colback=purple!5!white,colframe=purple!75!black,fonttitle=\bfseries,title=#1,breakable}
\newtcolorbox{interpretation}[1][Interpretation]{colback=cyan!5!white,colframe=cyan!75!black,fonttitle=\bfseries,title=#1,breakable}
\newtcolorbox{proof_step}[1][Proof Step]{colback=gray!5!white,colframe=gray!75!black,fonttitle=\bfseries,title=#1,breakable}
\newtcolorbox{experimental}[1][Experimental]{colback=teal!5!white,colframe=teal!75!black,fonttitle=\bfseries,title=#1,breakable}

\newtcolorbox{important}[1][Important]{colback=red!5!white,colframe=red!75!black,title={#1},fonttitle=\bfseries,breakable}
\newtcolorbox{warning}[1][Warning]{colback=orange!5!white,colframe=orange!75!black,title={#1},fonttitle=\bfseries,breakable}
\newtcolorbox{caution}[1][Caution]{colback=yellow!5!white,colframe=yellow!75!black,title={#1},fonttitle=\bfseries,breakable}
\newtcolorbox{highlight}[1][Highlight]{colback=yellow!10!white,colframe=yellow!75!black,title={#1},fonttitle=\bfseries,breakable}
\newtcolorbox{critical}[1][Critical]{colback=red!10!white,colframe=red!75!black,title={#1},fonttitle=\bfseries,breakable}

\newtcolorbox{analysis}[1][Analysis]{colback=blue!5!white,colframe=blue!75!black,title={#1},fonttitle=\bfseries,breakable}
\newtcolorbox{application}[1][Application]{colback=green!5!white,colframe=green!75!black,title={#1},fonttitle=\bfseries,breakable}
\newtcolorbox{experiment}[1][Experiment]{colback=cyan!5!white,colframe=cyan!75!black,title={#1},fonttitle=\bfseries,breakable}
\newtcolorbox{historical}[1][Historical]{colback=brown!5!white,colframe=brown!75!black,title={#1},fonttitle=\bfseries,breakable}
\newtcolorbox{numerical}[1][Numerical]{colback=gray!5!white,colframe=gray!75!black,title={#1},fonttitle=\bfseries,breakable}
\newtcolorbox{overview}[1][Overview]{colback=blue!5!white,colframe=blue!75!black,title={#1},fonttitle=\bfseries,breakable}
\newtcolorbox{speculation}[1][Speculation]{colback=purple!5!white,colframe=purple!75!black,title={#1},fonttitle=\bfseries,breakable}
\newtcolorbox{question}[1][Question]{colback=orange!5!white,colframe=orange!75!black,title={#1},fonttitle=\bfseries,breakable}
\newtcolorbox{method}[1][Method]{colback=teal!5!white,colframe=teal!75!black,title={#1},fonttitle=\bfseries,breakable}
\newtcolorbox{correct}[1][Correct]{colback=green!10!white,colframe=green!75!black,title={#1},fonttitle=\bfseries,breakable}
\newtcolorbox{units}[1][Units]{colback=gray!5!white,colframe=gray!75!black,title={#1},fonttitle=\bfseries,breakable}
\newtcolorbox{achievement}[1][Achievement]{colback=gold!5!white,colframe=orange!75!black,title={#1},fonttitle=\bfseries,breakable}
\newtcolorbox{equivalence}[1][Equivalence]{colback=cyan!5!white,colframe=cyan!75!black,title={#1},fonttitle=\bfseries,breakable}
\newtcolorbox{dimensional}[1][Dimensional Analysis]{colback=purple!5!white,colframe=purple!75!black,title={#1},fonttitle=\bfseries,breakable}

% === ADDITIONAL SIMPLE ENVIRONMENTS ===
\newenvironment{treatise}{\begin{quote}}{\end{quote}}
\newenvironment{gemeinsam}{\begin{quote}}{\end{quote}}
\newenvironment{vergleich}{\begin{quote}}{\end{quote}}
\newenvironment{vorteil}{\begin{quote}}{\end{quote}}
\newenvironment{common}{\begin{quote}}{\end{quote}}
\newenvironment{comparison}{\begin{quote}}{\end{quote}}
\newenvironment{advantage}{\begin{quote}}{\end{quote}}
\newenvironment{quantum}{\begin{quote}}{\end{quote}}

% === LAYOUT SETTINGS ===
\raggedbottom
\usepackage{environ}
\let\oldtabular\tabular
\let\endoldtabular\endtabular

\newenvironment{scaledtable}[1][0.85]{%
	\begingroup\footnotesize\setlength{\LTleft}{0pt}\setlength{\LTright}{0pt}%
}{%
	\endgroup%
}

\newcommand{\widetable}[1]{\resizebox{\textwidth}{!}{#1}}

% === TABLE OF CONTENTS FORMATTING ===
\renewcommand{\cftsecfont}{\color{blue}}
\renewcommand{\cftsubsecfont}{\color{blue}}
\renewcommand{\cftsecpagefont}{\color{blue}}
\renewcommand{\cftsubsecpagefont}{\color{blue}}
\renewcommand{\cfttoctitlefont}{\huge\bfseries\color{blue}}

% === DEFAULT HEADER AND FOOTER ===
\pagestyle{fancy}
\fancyhf{}
\fancyhead[L]{\textsc{T0 Theory}}
\fancyhead[R]{\textsc{J. Pascher}}
\fancyfoot[C]{\thepage}

% ==============================================================================
% End of Shared Preamble for English
% ==============================================================================
%
% Usage:
%   \documentclass[12pt,a4paper]{article}  % or book, report, etc.
%   % ==============================================================================
% T0 Theory: Shared ENGLISH Preamble – Optimized for eBook/Book
% Version: 2.0 – Final 2026 (LuaLaTeX only) – ENGLISH corrected
% Author: Johann Pascher
% Date: January 2026
% ==============================================================================
%
% IMPORTANT: Compile EXCLUSIVELY with LuaLaTeX!
% In TeXstudio: Options → Configure TeXstudio → Build → Default Compiler → LuaLaTeX
%
% Required Fonts (install once):
% - Inter: https://fonts.google.com/specimen/Inter
% - JetBrains Mono: https://www.jetbrains.com/lp/mono/
% - Libertinus Math: https://github.com/libertinus-fonts/libertinus
% ==============================================================================

% === CHAPTER 1: BASIC PACKAGES (must come FIRST) ===
\RequirePackage{fontspec}
\RequirePackage{unicode-math}
\usepackage{chngcntr}
\setcounter{secnumdepth}{1}  % Nur Sections nummerieren (nicht subsections)
\setcounter{tocdepth}{1}     % Nur Sections im TOC (nicht subsections)
\makeatletter
\@ifundefined{c@chapter}{}{\counterwithout{section}{chapter}}  % Falls Kapitel existieren
\makeatother
\counterwithout{subsection}{section}  % Löse Verknüpfung
% === CHAPTER 2: LANGUAGE (ENGLISH) ===
\usepackage[english]{babel}
\usepackage{microtype}                    % IMPORTANT for better hyphenation!

% Typography settings for better line breaking
\frenchspacing                     % Correct English spacing after punctuation
\emergencystretch=3em              % Allows more stretch for difficult lines
\tolerance=2500                    % Higher tolerance for line breaks
\hbadness=10000                    % Suppresses "underfull hbox" warnings
\hfuzz=2pt                         % Allows minimal overfull
\pretolerance=150                  % Better word breaking

% Prevent bad page breaks
\clubpenalty=10000           % No "orphans"
\widowpenalty=10000          % No "widows"
\displaywidowpenalty=10000   % Also with equations
\brokenpenalty=10000         % No broken words across pages

% Explicit hyphenation for long technical words
\hyphenation{Fun-da-men-tal Frac-tal-Ge-o-met-ric Field The-o-ry Meth-od-o-log-i-cal}
\hyphenation{Re-vi-sion-ism Quan-ti-za-tion U-ni-fi-ca-tion Ef-fec-tive}
\hyphenation{Re-nor-mal-iz-a-bil-i-ty Sin-gu-lar-i-ties Con-cil-i-a-tion}
\hyphenation{E-mer-gence Phe-nom-e-no-log-i-cal Doc-u-men-ta-tion A-nal-y-sis}
\hyphenation{Grav-i-ta-tion Quan-tum Me-chan-ics Dog-ma-tism Con-se-quent}
\hyphenation{Par-al-lel-ism Im-ple-men-ta-tion Per-tur-ba-tions}
\hyphenation{Geo-met-ric Ar-ti-fact In-com-pat-i-bil-i-ty Con-struc-tive}
\hyphenation{Frac-tal Di-men-sion-less In-ves-ti-ga-tion De-scrip-tion}
\hyphenation{In-ter-pre-ta-tion Phe-nom-e-no-log-i-cal Math-e-mat-i-cal}
\hyphenation{Phi-lo-soph-i-cal Le-git-i-ma-tion Ap-pli-ca-tion Der-i-va-tion}
\hyphenation{U-ni-fi-ca-tion As-sump-tion Con-cep-tion Ex-pec-ta-tion}
\hyphenation{Sym-me-try-ex-ten-sion O-ver-all-pic-ture Chal-lenge}
\hyphenation{In-ter-ac-tion Ma-te-ri-al Ap-proach Per-spec-tive Pro-ce-dure}

% === CHAPTER 3: FONTS (with proper ligatures) ===
\setmainfont{Inter}[
Scale=1.02,
UprightFont=*-Regular,
BoldFont=*-Bold,
ItalicFont=*-Italic,
BoldItalicFont=*-BoldItalic,
Ligatures=TeX,           % IMPORTANT for proper typography
Language=English         % Explicit language support
]
\setsansfont{Inter}[
Scale=MatchLowercase,
Ligatures=TeX,
Language=English
]
\setmonofont{JetBrains Mono}[
Scale=0.95,
Language=English
]

% Math Font (simple & stable) – MUST come AFTER language definition
% IMPORTANT: Libertinus Math for correct \underbrace display!
\setmathfont{Libertinus Math}[Scale=1.0]

% === CHAPTER 4: MATHEMATICS PACKAGES (in STRICT order!) ===
% IMPORTANT: mathtools must come BEFORE unicode-math for some commands!
\usepackage{mathtools}           % FIRST mathtools!

% Then the rest
\usepackage{amsmath, amsfonts, amsthm}

% SIUNITX MUST be loaded BEFORE physics!
\usepackage{siunitx}
\sisetup{
	locale=US,                    % ENGLISH settings for SI units!
	group-separator={,},          % Thousands separator comma
	output-decimal-marker={.},    % Decimal separator point
	per-mode=symbol,
	separate-uncertainty=true
}

% Custom SI units used in narrative and books
\DeclareSIUnit\gigalightyear{Gly}
\DeclareSIUnit\mev{MeV}

% physics – MUST be loaded AFTER siunitx and mathtools
\usepackage{physics}

% === CHAPTER 5: ADDITIONS from pdflatex best practices ===
\usepackage{colortbl}        % Colored tables (ESSENTIAL!)
\usepackage{placeins}        % Float control: \FloatBarrier
\usepackage{subcaption}      % Subfigures
\usepackage{xurl}            % Better URL line breaking
% Hyphenation for URLs in bibliography
\def\UrlBreaks{\do\/\do-}

% === CHAPTER 6: PAGE LAYOUT
% =============================================================================
% SECTION 2: Page Geometry – 6" × 9" Buchformat
% =============================================================================
\usepackage[paperwidth=6in, paperheight=9in,
top=0.9in,
bottom=1.1in,
inner=0.9in,            % Größerer Innenrand für Bindung
outer=0.6in,            % Kleinerer Außenrand → mehr Text pro Seite
bindingoffset=0.5in,    % Puffer für Bindung (Steg)
twoside]{geometry}
\setlength{\headheight}{15pt}
%\usepackage[paperwidth=8.25in, paperheight=11in,
%top=1.0in,
%bottom=1.0in,
%left=1.0in,
%right=1.0in,
%twoside=false
% === CHAPTER 7: GRAPHICS AND TABLES ===
\usepackage{graphicx}
\usepackage[table,xcdraw]{xcolor}
% T0 brand colors
\definecolor{gold}{RGB}{255,215,0}
\definecolor{blue}{rgb}{0,0,1}
\definecolor{boxgray}{RGB}{240,240,240}
\definecolor{deepblue}{RGB}{0,0,127}
\definecolor{deepgreen}{RGB}{0,127,0}
\definecolor{deepred}{RGB}{191,0,0}
\definecolor{t0blue}{RGB}{33,150,243}
\definecolor{t0green}{RGB}{76,175,80}
\definecolor{t0orange}{RGB}{255,152,0}
\definecolor{t0purple}{RGB}{156,39,176}
\definecolor{t0red}{RGB}{244,67,54}
\definecolor{t0yellow}{RGB}{255,204,0}
\usepackage{tikz}
\usetikzlibrary{arrows.meta,positioning,shapes.geometric,decorations.pathmorphing,patterns,shapes.arrows,intersections}
\usepackage{pgfplots}
\pgfplotsset{compat=1.18}
\usepackage{quantikz}
\usepackage[most]{tcolorbox}
\tcbuselibrary{breakable}

% === WICHTIG: Algorithm-Konflikt umgehen ===
% Option: algorithmic mit GROSSBUCHSTABEN
% Gemeinsame Box für Experimente
\newtcolorbox{experimentbox}[1][]{
	colback=green!5!white,
	colframe=t0green!80!black,
	fonttitle=\bfseries,
	title={{#1}},
	breakable
}

% Abstract-Fallback
\ifdefined\abstract\else
\newenvironment{abstract}{\section*{\abstractname}\itshape\small\par\bigskip}{\bigskip}
\fi

% === MAKROS SICHER NEU DEFINIEREN / ÜBERSCHREIBEN ===
% Definiere Makros OHNE doppelte Subskripte
\newcommand{\phipar}{\phi_{\mathrm{par}}}
%\newcommand{\xipar}{\xi_{\mathrm{par}}}
\newcommand{\Qphipar}{Q_{\phi_{\mathrm{par}}}}
\newcommand{\rphipar}{r_{\phi_{\mathrm{par}}}}
\newcommand{\logphipar}{\log_{\phi_{\mathrm{par}}}}
\newcommand{\CHSH}{\text{CHSH}}
\usepackage{booktabs}
\usepackage{array}
\usepackage{longtable}
\usepackage{float}
\usepackage{adjustbox}
\usepackage{rotating}
\usepackage{tabularx}
\usepackage{makecell}
\usepackage{multirow}

% === CHAPTER 8: DOCUMENT FORMATTING ===
\usepackage{fancyhdr}
\renewcommand{\headrulewidth}{0.4pt}
\renewcommand{\footrulewidth}{0.4pt}
\usepackage{tocloft}

\usepackage{enumitem}
\setlist[itemize]{leftmargin=*, topsep=2pt, partopsep=0pt, parsep=2pt, itemsep=2pt}
\setlist[enumerate]{leftmargin=*, topsep=2pt, partopsep=0pt, parsep=2pt, itemsep=2pt}
\usepackage{setspace}
\usepackage{ragged2e}
\usepackage{multicol}

% === CHAPTER 9: CODE AND ALGORITHMS ===
\usepackage{algorithm}
\usepackage{algorithmic}
\usepackage{listings}
\lstset{
	basicstyle=\ttfamily\footnotesize,
	breaklines=true,
	breakatwhitespace=true,
	columns=flexible,
	keepspaces=true,
	showstringspaces=false,
	frame=single,
	xleftmargin=0pt,
	xrightmargin=0pt,
	literate=              % For special characters in code listings
	{ä}{{\"a}}1 {ö}{{\"o}}1 {ü}{{\"u}}1 {ß}{{\ss}}1
	{Ä}{{\"A}}1 {Ö}{{\"O}}1 {Ü}{{\"U}}1
}
\usepackage{mdframed}

% === CHAPTER 10: ADDITIONAL PACKAGES ===
\usepackage{pdflscape}
\usepackage{braket}
\usepackage{cancel}
\usepackage{caption}
\captionsetup{format=plain, labelfont=bf, justification=centering}
\usepackage{csquotes}
\usepackage{gensymb}
\usepackage{textcomp}
\usepackage{textgreek}
\usepackage{upgreek}
\usepackage{url}
\usepackage{slashed}
\usepackage{bm}

% === CHAPTER 11: HYPERREF (must come SECOND TO LAST!) ===
\usepackage{hyperref}
\hypersetup{
	colorlinks=true,
	linkcolor=black,
	citecolor=black,
	urlcolor=black,
	breaklinks=true,           % IMPORTANT for special characters in URLs!
	bookmarksnumbered=true,
	unicode=true,
	pdfencoding=auto,
	pdflang=en,                % Set PDF language to English
	pdfsubject={T0 Theory - Fundamental Fractal-Geometric Field Theory}
}

% Fix for unicode-math symbols in PDF bookmarks
\pdfstringdefDisableCommands{%
	\def\xi{xi}%
	\def\alpha{alpha}%
	\def\beta{beta}%
	\def\gamma{gamma}%
	\def\delta{delta}%
	\def\Delta{Delta}%
	\def\epsilon{epsilon}%
	\def\varepsilon{epsilon}%
	\def\theta{theta}%
	\def\kappa{kappa}%
	\def\lambda{lambda}%
	\def\mu{mu}%
	\def\nu{nu}%
	\def\pi{pi}%
	\def\rho{rho}%
	\def\sigma{sigma}%
	\def\tau{tau}%
	\def\phi{phi}%
	\def\chi{chi}%
	\def\psi{psi}%
	\def\omega{omega}%
	\def\Omega{Omega}%
	\def\Lambda{Lambda}%
	\def\times{x}%
	\def\cdot{*}%
	\def\pm{+/-}%
	\def\approx{~}%
	\def\sim{~}%
	\def\equiv{=}%
	\def\ell{l}%
	\def\hbar{h}%
	\def\rightarrow{->}%
	\def\leftarrow{<-}%
	\def\Rightarrow{=>}%
	\def\Leftarrow{<=}%
	\def\propto{~}%
	\def\mitxi{xi}%
	\def\mitalpha{alpha}%
	\def\mitbeta{beta}%
	\def\mitgamma{gamma}%
	\def\mitdelta{delta}%
	\def\mitDelta{Delta}%
	\def\mitepsilon{epsilon}%
	\def\mitvarepsilon{epsilon}%
	\def\mittheta{theta}%
	\def\mitkappa{kappa}%
	\def\mitlambda{lambda}%
	\def\mitLambda{Lambda}%
	\def\mitmu{mu}%
	\def\mitnu{nu}%
	\def\mitpi{pi}%
	\def\mitrho{rho}%
	\def\mitsigma{sigma}%
	\def\mittau{tau}%
	\def\mitphi{phi}%
	\def\mitchi{chi}%
	\def\mitpsi{psi}%
	\def\mitomega{omega}%
	\def\mitOmega{Omega}%
}

% === CHAPTER 12: BOOKMARK (must come AFTER hyperref!) ===
\usepackage{bookmark}

% === CHAPTER 13: CLEVEREF (ENGLISH LABELS) ===
\usepackage[english]{cleveref}
\crefname{equation}{Equation}{Equations}
\crefname{figure}{Figure}{Figures}
\crefname{table}{Table}{Tables}
\crefname{section}{Section}{Sections}
\crefname{chapter}{Chapter}{Chapters}
\crefname{theorem}{Theorem}{Theorems}
\crefname{lemma}{Lemma}{Lemmas}
\crefname{definition}{Definition}{Definitions}
\crefname{example}{Example}{Examples}
\crefname{remark}{Remark}{Remarks}

% === CUSTOM ENVIRONMENTS ===
% Alternative interpretation environment
\newenvironment{alternative}{%
	\begin{mdframed}[linecolor=black!30,linewidth=1pt,roundcorner=4pt,backgroundcolor=black!5]%
	}{%
	\end{mdframed}%
}

% Photon/particle environment
\newenvironment{photon}{%
	\begin{mdframed}[linecolor=blue!30,linewidth=1pt,roundcorner=4pt,backgroundcolor=blue!5]%
	}{%
	\end{mdframed}%
}

% Koide formula box environment
\newenvironment{koidebox}{%
	\begin{mdframed}[linecolor=green!30,linewidth=1pt,roundcorner=4pt,backgroundcolor=green!5]%
	}{%
	\end{mdframed}%
}

% Erkenntnis/insight environment
\newenvironment{erkenntnis}{%
	\begin{mdframed}[linecolor=orange!30,linewidth=1pt,roundcorner=4pt,backgroundcolor=orange!5]%
	}{%
	\end{mdframed}%
}

% Beziehung/relationship environment
\newenvironment{beziehung}{%
	\begin{mdframed}[linecolor=purple!30,linewidth=1pt,roundcorner=4pt,backgroundcolor=purple!5]%
	}{%
	\end{mdframed}%
}

% Derivation environment
\newenvironment{derivation}{%
	\begin{mdframed}[linecolor=teal!30,linewidth=1pt,roundcorner=4pt,backgroundcolor=teal!5]%
	}{%
	\end{mdframed}%
}

% Abhandlung/treatise environment
\newenvironment{abhandlung}{%
	\begin{mdframed}[linecolor=brown!30,linewidth=1pt,roundcorner=4pt,backgroundcolor=brown!5]%
	}{%
	\end{mdframed}%
}

% Anwendung/application environment
\newenvironment{anwendung}{%
	\begin{mdframed}[linecolor=cyan!30,linewidth=1pt,roundcorner=4pt,backgroundcolor=cyan!5]%
	}{%
	\end{mdframed}%
}

% Additional common environments
\newenvironment{konsequenz}{%
	\begin{mdframed}[linecolor=red!30,linewidth=1pt,roundcorner=4pt,backgroundcolor=red!5]%
	}{%
	\end{mdframed}%
}

\newenvironment{schlussfolgerung}{%
	\begin{mdframed}[linecolor=gray!30,linewidth=1pt,roundcorner=4pt,backgroundcolor=gray!5]%
	}{%
	\end{mdframed}%
}

\newenvironment{result}{%
	\begin{mdframed}[linecolor=violet!30,linewidth=1pt,roundcorner=4pt,backgroundcolor=violet!5]%
	}{%
	\end{mdframed}%
}

% Formula environment
\newenvironment{formula}{%
	\begin{mdframed}[linecolor=yellow!30,linewidth=1pt,roundcorner=4pt,backgroundcolor=yellow!5]%
	}{%
	\end{mdframed}%
}

% Revolutionaer/revolutionary environment
\newenvironment{revolutionaer}{%
	\begin{mdframed}[linecolor=red!50,linewidth=2pt,roundcorner=4pt,backgroundcolor=red!10]%
	}{%
	\end{mdframed}%
}

% Formel environment (German version of formula)
\newenvironment{formel}{%
	\begin{mdframed}[linecolor=yellow!30,linewidth=1pt,roundcorner=4pt,backgroundcolor=yellow!5]%
	}{%
	\end{mdframed}%
}

% Prinzip/principle environment
\newenvironment{prinzip}{%
	\begin{mdframed}[linecolor=blue!50,linewidth=2pt,roundcorner=4pt,backgroundcolor=blue!10]%
	}{%
	\end{mdframed}%
}

% Experimentell/experimental environment
\newenvironment{experimentell}{%
	\begin{mdframed}[linecolor=magenta!30,linewidth=1pt,roundcorner=4pt,backgroundcolor=magenta!5]%
	}{%
	\end{mdframed}%
}

% Neutrino environment
\newenvironment{neutrino}{%
	\begin{mdframed}[linecolor=cyan!40,linewidth=1pt,roundcorner=4pt,backgroundcolor=cyan!8]%
	}{%
	\end{mdframed}%
}

% Additional missing environments
\newenvironment{schluessel}{%
	\begin{mdframed}[linecolor=yellow!50,linewidth=1pt,roundcorner=4pt,backgroundcolor=yellow!10]%
	}{%
	\end{mdframed}%
}

\newenvironment{summary}{%
	\begin{mdframed}[linecolor=gray!40,linewidth=1pt,roundcorner=4pt,backgroundcolor=gray!8]%
	}{%
	\end{mdframed}%
}

\newenvironment{category}{%
	\begin{mdframed}[linecolor=pink!40,linewidth=1pt,roundcorner=4pt,backgroundcolor=pink!8]%
	}{%
	\end{mdframed}%
}

\newenvironment{sibox}{%
	\begin{mdframed}[linecolor=lime!40,linewidth=1pt,roundcorner=4pt,backgroundcolor=lime!8]%
	}{%
	\end{mdframed}%
}

% More missing environments
\newenvironment{documentbox}{%
	\begin{mdframed}[linecolor=teal!40,linewidth=1pt,roundcorner=4pt,backgroundcolor=teal!8]%
	}{%
	\end{mdframed}%
}

\newenvironment{t0box}{%
	\begin{mdframed}[linecolor=violet!40,linewidth=1pt,roundcorner=4pt,backgroundcolor=violet!8]%
	}{%
	\end{mdframed}%
}

\newenvironment{wichtig}{%
	\begin{mdframed}[linecolor=red!50,linewidth=2pt,roundcorner=4pt,backgroundcolor=red!10]%
	\textbf{Important:} 
	}{%
	\end{mdframed}%
}

\newenvironment{smbox}{%
	\begin{mdframed}[linecolor=orange!40,linewidth=1pt,roundcorner=4pt,backgroundcolor=orange!8]%
	}{%
	\end{mdframed}%
}

\newenvironment{pvbox}{%
	\begin{mdframed}[linecolor=purple!40,linewidth=1pt,roundcorner=4pt,backgroundcolor=purple!8]%
	}{%
	\end{mdframed}%
}

\newenvironment{numerisch}{%
	\begin{mdframed}[linecolor=blue!40,linewidth=1pt,roundcorner=4pt,backgroundcolor=blue!8]%
	}{%
	\end{mdframed}%
}

% More missing environments
\newenvironment{relation}{%
	\begin{mdframed}[linecolor=green!40,linewidth=1pt,roundcorner=4pt,backgroundcolor=green!8]%
	}{%
	\end{mdframed}%
}

\newenvironment{beweis}{%
	\begin{mdframed}[linecolor=brown!40,linewidth=1pt,roundcorner=4pt,backgroundcolor=brown!8]%
	\textbf{Proof:} 
	}{%
	\end{mdframed}%
}

\newenvironment{revolution}{%
	\begin{mdframed}[linecolor=red!60,linewidth=2pt,roundcorner=4pt,backgroundcolor=red!12]%
	}{%
	\end{mdframed}%
}

\newenvironment{key}{%
	\begin{mdframed}[linecolor=yellow!50,linewidth=1pt,roundcorner=4pt,backgroundcolor=yellow!10]%
	}{%
	\end{mdframed}%
}

\newenvironment{newperspective}{%
	\begin{mdframed}[linecolor=cyan!50,linewidth=1pt,roundcorner=4pt,backgroundcolor=cyan!10]%
	}{%
	\end{mdframed}%
}

\newenvironment{literatur}{%
	\begin{mdframed}[linecolor=gray!50,linewidth=1pt,roundcorner=4pt,backgroundcolor=gray!10]%
	}{%
	\end{mdframed}%
}

\newenvironment{folgerung}{%
	\begin{mdframed}[linecolor=teal!50,linewidth=1pt,roundcorner=4pt,backgroundcolor=teal!10]%
	}{%
	\end{mdframed}%
}

\newenvironment{principle}{%
	\begin{mdframed}[linecolor=blue!60,linewidth=2pt,roundcorner=4pt,backgroundcolor=blue!12]%
	}{%
	\end{mdframed}%
}

% Additional common environments
% ==============================================================================
% FROM HERE: YOUR DEFINITIONS (unchanged)
% ==============================================================================

\setcounter{tocdepth}{3}

% === CITATION COMMANDS ===
\providecommand{\citep}[1]{\cite{#1}}
\providecommand{\citet}[1]{\cite{#1}}

% === COLORS ===
\definecolor{gold}{RGB}{255,215,0}
\definecolor{blue}{rgb}{0,0,1}
\definecolor{boxgray}{RGB}{240,240,240}
\definecolor{deepblue}{RGB}{0,0,127}
\definecolor{deepgreen}{RGB}{0,127,0}
\definecolor{deepred}{RGB}{191,0,0}
\definecolor{t0blue}{RGB}{33,150,243}
\definecolor{t0green}{RGB}{76,175,80}
\definecolor{t0orange}{RGB}{255,152,0}
\definecolor{t0purple}{RGB}{156,39,176}
\definecolor{t0red}{RGB}{244,67,54}
\definecolor{t0yellow}{RGB}{255,204,0}

% === COLUMN TYPES ===
\newcolumntype{L}[1]{>{\raggedright\arraybackslash}p{#1}}
\newcolumntype{C}[1]{>{\centering\arraybackslash}p{#1}}
\newcolumntype{R}[1]{>{\raggedleft\arraybackslash}p{#1}}

% === HYPERREF SETTINGS (updated) ===
\hypersetup{
	colorlinks=true,
	linkcolor=t0blue,
	citecolor=t0blue,
	urlcolor=t0blue,
	breaklinks=true,
	bookmarksnumbered=true,
	pdfstartview=FitH,
	pdfencoding=auto,
	pdfdisplaydoctitle=true
}

% === ENGLISH THEOREM ENVIRONMENTS ===
\theoremstyle{plain}
\newtheorem{theorem}{Theorem}[section]
\newtheorem{lemma}[theorem]{Lemma}
\newtheorem{proposition}[theorem]{Proposition}
\newtheorem{corollary}[theorem]{Corollary}

\theoremstyle{definition}
\newtheorem{definition}[theorem]{Definition}
\newtheorem{example}[theorem]{Example}
\newtheorem{insight}[theorem]{Insight}
\newtheorem{discovery}[theorem]{Discovery}

\theoremstyle{remark}
\newtheorem{remark}[theorem]{Remark}
\newtheorem{axiom}{Axiom}
%\newtheorem{principle}{Principle}  % Commented out to avoid conflicts with document-specific definitions
%\newtheorem{warning}[theorem]{Warning}

% === T0-SPECIFIC COMMANDS ===
% (Here follow all your \newcommand and \providecommand definitions)
% These remain UNCHANGED as in your original preamble
% ==============================================================================
% SECTION 14: T0-Specific Commands
% ==============================================================================

% --- Core T0 Fields ---
\newcommand{\Tfield}{T(x,t)}
\providecommand{\Tfieldt}{T(\vec{x},t)}
\newcommand{\Efield}{E(x,t)}
\newcommand{\mfield}{m(x,t)}
\providecommand{\vecx}{\vec{x}}

% --- Lagrangian ---
\newcommand{\Lag}{\mathcal{L}}
\newcommand{\calL}{\mathcal{L}}

% --- Greek Letters and Constants ---
\newcommand{\alphaem}{\alpha}
\newcommand{\betaT}{\beta_T}
\newcommand{\xiT}{\xi}
\newcommand{\xipar}{\xi}

% --- Energy and Planck Units ---
\newcommand{\Ezero}{E_0}
\newcommand{\E}{E}
\newcommand{\EPlanck}{E_{\text{Pl}}}
\newcommand{\Mpl}{M_{\text{Pl}}}
\newcommand{\mP}{m_{\text{P}}}
\newcommand{\lP}{\ell_{\text{P}}}
\newcommand{\tP}{t_{\text{P}}}
\newcommand{\LPlanck}{\ell_{\text{Pl}}}
\newcommand{\TPlanck}{t_{\text{Pl}}}

% --- Coupling Constants ---
\newcommand{\Gnat}{G_{\text{nat}}}
\newcommand{\alphaEM}{\alpha_{\text{EM}}}
\newcommand{\alphaSI}{\alpha_{\text{SI}}}
\newcommand{\Hubble}{H_0}
\newcommand{\LCDM}{\Lambda\text{CDM}}
\newcommand{\natunits}{(nat. units)}

% --- T0 Model Parameters ---
\newcommand{\xigeom}{\xi_{\mathrm{geom}}}
\newcommand{\rzero}{r_{0}}
\newcommand{\xirat}{\xi_{\mathrm{rat}}}
\newcommand{\tzero}{t_{0}}
\newcommand{\Lambdat}{\Lambda_{\mathrm{t}}}
\newcommand{\EP}{E_{\text{P}}}
\newcommand{\Emu}{E_{\mu}}
\newcommand{\Ee}{E_{e}}
\newcommand{\Etau}{E_{\tau}}
\newcommand{\alphafine}{\alpha_{\mathrm{fine}}}
\newcommand{\alphal}{\alpha_{\ell}}
\newcommand{\Lzero}{\ell_{0}}
\newcommand{\Lp}{\ell_{\mathrm{P}}}

% --- Additional T0 Commands ---
\newcommand{\Kfrak}{K_{\text{frak}}}
\newcommand{\Dfrak}{D_{\text{frak}}}
\newcommand{\betapar}{\ensuremath{\beta_T}}
\newcommand{\alphapar}{\alpha}
\newcommand{\deltafield}{\delta \phi}
\newcommand{\deltam}{\delta m}
\newcommand{\deltaE}{\delta E}
\newcommand{\Exi}{E_{\xi}}
\newcommand{\Lxi}{\ell_{\xi}}
\newcommand{\rhoCMB}{\rho_{\text{CMB}}}
\newcommand{\rhoCasimir}{\rho_{\text{Casimir}}}
\newcommand{\Leff}{L_{\text{eff}}}
\newcommand{\CQCD}{C_{\mathrm{QCD}}}
\newcommand{\Kspec}{K_{\mathrm{spec}}}
\newcommand{\Tzero}{\ensuremath{T_0}}
\newcommand{\Eabs}{E_{\text{abs}}}
\newcommand{\taupar}{\tau}

% --- Provided Commands ---
\providecommand{\xiconst}{\xi_{\text{const}}}
\providecommand{\DhiggsT}{D_{\text{Higgs-T}}}
\providecommand{\rhoE}{\rho_{E}}
\providecommand{\Echar}{E_{\text{char}}}
\providecommand{\kfrac}{k_{\text{frac}}}
\providecommand{\alphaEMSI}{\alpha_{\text{EM,SI}}}
\providecommand{\alphaEMnat}{\alpha_{\text{EM,nat}}}
\providecommand{\betaTSI}{\beta_{T,\text{SI}}}
\providecommand{\betaTnat}{\beta_{T,\text{nat}}}
\providecommand{\Gsi}{G_{\text{SI}}}
\providecommand{\xiparSI}{\xi_{\text{SI}}}
\providecommand{\xiparnat}{\xi_{\text{nat}}}
\providecommand{\meff}{m_{\text{eff}}}
\providecommand{\Tzerot}{T_{0}(t)}
\providecommand{\mzerot}{m_{0}(t)}
\providecommand{\Ezeroabs}{E_{0,\text{abs}}}
\providecommand{\Epar}{E_{\text{par}}}
\providecommand{\Lnat}{\ell_{\text{nat}}}
\providecommand{\Tnat}{T_{\text{nat}}}
\providecommand{\xifrak}{\xi_{\text{frac}}}
\providecommand{\Tfrak}{T_{\text{frac}}}
\providecommand{\mfrak}{m_{\text{frac}}}
\providecommand{\Dfrac}{D_{\text{frac}}}
\providecommand{\EphotSI}{E_{\gamma,\text{SI}}}
\providecommand{\EphotNat}{E_{\gamma,\text{nat}}}
\providecommand{\Eabsint}{E_{\text{abs,int}}}
\providecommand{\mphoton}{m_{\gamma}}
\providecommand{\Evis}{E_{\text{vis}}}
\providecommand{\Cto}{C_{T0}}
\providecommand{\mytimes}{\times}
\providecommand{\lambdah}{\lambda_h}
\providecommand{\checkmarkx}{\checkmark}
\providecommand{\Enorm}{E_{\text{norm}}}
\providecommand{\Tobs}{T_{\text{obs}}}
\providecommand{\mobs}{m_{\text{obs}}}
\providecommand{\Eobs}{E_{\text{obs}}}
\providecommand{\Lobs}{\ell_{\text{obs}}}
\providecommand{\xobs}{\xi_{\text{obs}}}
\providecommand{\calE}{\mathcal{E}}
\providecommand{\calT}{\mathcal{T}}
\providecommand{\calM}{\mathcal{M}}
\providecommand{\alphag}{\alpha_g}
\providecommand{\Tmax}{T_{\text{max}}}
\providecommand{\mmin}{m_{\text{min}}}
\providecommand{\Lmax}{\ell_{\text{max}}}
\providecommand{\Emin}{E_{\text{min}}}
\providecommand{\Geff}{G_{\text{eff}}}
\providecommand{\rhoeff}{\rho_{\text{eff}}}
\providecommand{\xieff}{\xi_{\text{eff}}}
\providecommand{\Teff}{T_{\text{eff}}}
\providecommand{\hPlanck}{h}
\providecommand{\kB}{k_B}
\providecommand{\muB}{\mu_B}
\providecommand{\lambdaC}{\lambda_C}
\providecommand{\omegaP}{\omega_P}
\providecommand{\rhoP}{\rho_P}
\providecommand{\Tref}{T_{\text{ref}}}
\providecommand{\Eref}{E_{\text{ref}}}
\providecommand{\mref}{m_{\text{ref}}}
\providecommand{\Lref}{\ell_{\text{ref}}}
\providecommand{\xikonst}{\xi_0}
\providecommand{\Phiphoton}{\Phi_{\gamma}}
\providecommand{\etavis}{\eta_{\text{vis}}}
\providecommand{\pichar}{\pi}
\providecommand{\primrel}{\mathcal{P}_{\text{rel}}}
\providecommand{\warningx}{\textcolor{orange}{\textbf{!}}}
\providecommand{\phiT}{\phi_T}
\providecommand{\Lorentz}{\Lambda}
\providecommand{\Cconv}{C_{\text{conv}}}
\providecommand{\Df}{\Delta f}
\providecommand{\lambdazero}{\lambda_0}
\providecommand{\myapprox}{\approx}
\providecommand{\checked}{\checkmark}
\providecommand{\alphaWSI}{\alpha_W^{\text{SI}}}
\providecommand{\alphaWnat}{\alpha_W^{\text{nat}}}
\providecommand{\vect}[1]{\vec{#1}}
\providecommand{\Rzero}{R_0}
\providecommand{\Riem}{\mathcal{R}}
\providecommand{\nuzero}{\nu_0}
\providecommand{\mypi}{\pi}

% =============================================================================
% TCOLORBOX STYLES AND ENVIRONMENTS (English titles)
% =============================================================================
\tcbset{
	keyresult/.style={
		colback=blue!5!white,
		colframe=blue!75!black,
		title=Key Result,
		fonttitle=\bfseries
	},
	foundation/.style={
		colback=green!5!white,
		colframe=green!75!black,
		title=Foundation,
		fonttitle=\bfseries
	},
	alternative/.style={
		colback=orange!5!white,
		colframe=orange!75!black,
		title=Alternative,
		fonttitle=\bfseries
	},
	warningbox/.style={
		colback=red!5!white,
		colframe=red!75!black,
		title=Warning,
		fonttitle=\bfseries
	}
}

% (Here follow all your tcolorbox definitions with English titles)
\newtcolorbox{keyresultbox}[1][]{colback=blue!5!white,colframe=blue!75!black,fonttitle=\bfseries,title={#1},breakable}
\newtcolorbox{keyresult}[1][Key Result]{colback=blue!5!white,colframe=blue!75!black,fonttitle=\bfseries,title={#1},breakable}
\newtcolorbox{foundationbox}[1][]{colback=green!5!white,colframe=green!75!black,fonttitle=\bfseries,title={#1},breakable}
\newtcolorbox{foundation}[1][Foundation]{colback=green!5!white,colframe=green!75!black,fonttitle=\bfseries,title={#1},breakable}
\newtcolorbox{alternativebox}[1][]{colback=orange!5!white,colframe=orange!75!black,fonttitle=\bfseries,title={#1},breakable}
\newtcolorbox{warningboxenv}[1][Warning]{colback=red!5!white,colframe=red!75!black,fonttitle=\bfseries,title={#1},breakable}

\newtcolorbox{fundamental}[1][]{
	colback=boxgray,
	colframe=t0blue,
	fonttitle=\bfseries,
	title=#1,
	sharp corners,
	boxrule=2pt
}

\newtcolorbox{insightBox}[1][Insight]{colback=blue!5,colframe=t0blue,title={#1},fonttitle=\bfseries,breakable}
\newtcolorbox{discoveryBox}[1][Discovery]{colback=green!5,colframe=t0green,title={#1},fonttitle=\bfseries,breakable}
\newtcolorbox{revelation}[1][Revelation]{colback=red!5,colframe=t0red,title={#1},fonttitle=\bfseries,breakable}
\newtcolorbox{keypoint}[1][Key Point]{colback=blue!5,colframe=t0blue,title={#1},fonttitle=\bfseries,breakable}
\newtcolorbox{evidence}[1][Evidence]{colback=green!5,colframe=t0green,title={#1},fonttitle=\bfseries,breakable}
\newtcolorbox{conclusionBox}[1][Conclusion]{colback=gray!5,colframe=gray,title={#1},fonttitle=\bfseries,breakable}
\newtcolorbox{significance}[1][Significance]{colback=yellow!5,colframe=orange,title={#1},fonttitle=\bfseries,breakable}
\newtcolorbox{philosophical}[1][Philosophical]{colback=purple!5,colframe=purple,title={#1},fonttitle=\bfseries,breakable}
\newtcolorbox{implicationBox}[1][Implication]{colback=cyan!5,colframe=cyan,title={#1},fonttitle=\bfseries,breakable}
\newtcolorbox{perspectiveBox}[1][Perspective]{colback=blue!5,colframe=t0blue,title={#1},fonttitle=\bfseries,breakable}
\newtcolorbox{revolutionary}[1][Revolutionary]{colback=red!5,colframe=t0red,title={#1},fonttitle=\bfseries,breakable}

\newtcolorbox{technical}[1][Technical]{colback=gray!5,colframe=gray!75!black,title={#1},fonttitle=\bfseries,breakable}
\newtcolorbox{technicalBox}[1][Technical]{colback=gray!5,colframe=gray!75!black,title={#1},fonttitle=\bfseries,breakable}
\newtcolorbox{notationBox}[1][Notation]{colback=yellow!5,colframe=yellow!75!black,title={#1},fonttitle=\bfseries,breakable}
\newtcolorbox{verification}[1][Verification]{colback=orange!5!white,colframe=orange!75!black,fonttitle=\bfseries,title=#1}
\newtcolorbox{explanationBox}[1][Explanation]{colback=purple!5!white,colframe=purple!75!black,fonttitle=\bfseries,title=#1}
\newtcolorbox{interpretationBox}[1][Interpretation]{colback=cyan!5!white,colframe=cyan!75!black,fonttitle=\bfseries,title=#1}
\newtcolorbox{explanation}[1][Explanation]{colback=purple!5!white,colframe=purple!75!black,fonttitle=\bfseries,title=#1,breakable}
\newtcolorbox{interpretation}[1][Interpretation]{colback=cyan!5!white,colframe=cyan!75!black,fonttitle=\bfseries,title=#1,breakable}
\newtcolorbox{proof_step}[1][Proof Step]{colback=gray!5!white,colframe=gray!75!black,fonttitle=\bfseries,title=#1,breakable}
\newtcolorbox{experimental}[1][Experimental]{colback=teal!5!white,colframe=teal!75!black,fonttitle=\bfseries,title=#1,breakable}

\newtcolorbox{important}[1][Important]{colback=red!5!white,colframe=red!75!black,title={#1},fonttitle=\bfseries,breakable}
\newtcolorbox{warning}[1][Warning]{colback=orange!5!white,colframe=orange!75!black,title={#1},fonttitle=\bfseries,breakable}
\newtcolorbox{caution}[1][Caution]{colback=yellow!5!white,colframe=yellow!75!black,title={#1},fonttitle=\bfseries,breakable}
\newtcolorbox{highlight}[1][Highlight]{colback=yellow!10!white,colframe=yellow!75!black,title={#1},fonttitle=\bfseries,breakable}
\newtcolorbox{critical}[1][Critical]{colback=red!10!white,colframe=red!75!black,title={#1},fonttitle=\bfseries,breakable}

\newtcolorbox{analysis}[1][Analysis]{colback=blue!5!white,colframe=blue!75!black,title={#1},fonttitle=\bfseries,breakable}
\newtcolorbox{application}[1][Application]{colback=green!5!white,colframe=green!75!black,title={#1},fonttitle=\bfseries,breakable}
\newtcolorbox{experiment}[1][Experiment]{colback=cyan!5!white,colframe=cyan!75!black,title={#1},fonttitle=\bfseries,breakable}
\newtcolorbox{historical}[1][Historical]{colback=brown!5!white,colframe=brown!75!black,title={#1},fonttitle=\bfseries,breakable}
\newtcolorbox{numerical}[1][Numerical]{colback=gray!5!white,colframe=gray!75!black,title={#1},fonttitle=\bfseries,breakable}
\newtcolorbox{overview}[1][Overview]{colback=blue!5!white,colframe=blue!75!black,title={#1},fonttitle=\bfseries,breakable}
\newtcolorbox{speculation}[1][Speculation]{colback=purple!5!white,colframe=purple!75!black,title={#1},fonttitle=\bfseries,breakable}
\newtcolorbox{question}[1][Question]{colback=orange!5!white,colframe=orange!75!black,title={#1},fonttitle=\bfseries,breakable}
\newtcolorbox{method}[1][Method]{colback=teal!5!white,colframe=teal!75!black,title={#1},fonttitle=\bfseries,breakable}
\newtcolorbox{correct}[1][Correct]{colback=green!10!white,colframe=green!75!black,title={#1},fonttitle=\bfseries,breakable}
\newtcolorbox{units}[1][Units]{colback=gray!5!white,colframe=gray!75!black,title={#1},fonttitle=\bfseries,breakable}
\newtcolorbox{achievement}[1][Achievement]{colback=gold!5!white,colframe=orange!75!black,title={#1},fonttitle=\bfseries,breakable}
\newtcolorbox{equivalence}[1][Equivalence]{colback=cyan!5!white,colframe=cyan!75!black,title={#1},fonttitle=\bfseries,breakable}
\newtcolorbox{dimensional}[1][Dimensional Analysis]{colback=purple!5!white,colframe=purple!75!black,title={#1},fonttitle=\bfseries,breakable}

% === ADDITIONAL SIMPLE ENVIRONMENTS ===
\newenvironment{treatise}{\begin{quote}}{\end{quote}}
\newenvironment{gemeinsam}{\begin{quote}}{\end{quote}}
\newenvironment{vergleich}{\begin{quote}}{\end{quote}}
\newenvironment{vorteil}{\begin{quote}}{\end{quote}}
\newenvironment{common}{\begin{quote}}{\end{quote}}
\newenvironment{comparison}{\begin{quote}}{\end{quote}}
\newenvironment{advantage}{\begin{quote}}{\end{quote}}
\newenvironment{quantum}{\begin{quote}}{\end{quote}}

% === LAYOUT SETTINGS ===
\raggedbottom
\usepackage{environ}
\let\oldtabular\tabular
\let\endoldtabular\endtabular

\newenvironment{scaledtable}[1][0.85]{%
	\begingroup\footnotesize\setlength{\LTleft}{0pt}\setlength{\LTright}{0pt}%
}{%
	\endgroup%
}

\newcommand{\widetable}[1]{\resizebox{\textwidth}{!}{#1}}

% === TABLE OF CONTENTS FORMATTING ===
\renewcommand{\cftsecfont}{\color{blue}}
\renewcommand{\cftsubsecfont}{\color{blue}}
\renewcommand{\cftsecpagefont}{\color{blue}}
\renewcommand{\cftsubsecpagefont}{\color{blue}}
\renewcommand{\cfttoctitlefont}{\huge\bfseries\color{blue}}

% === DEFAULT HEADER AND FOOTER ===
\pagestyle{fancy}
\fancyhf{}
\fancyhead[L]{\textsc{T0 Theory}}
\fancyhead[R]{\textsc{J. Pascher}}
\fancyfoot[C]{\thepage}

% ==============================================================================
% End of Shared Preamble for English
% ==============================================================================
%   \begin{document}
%   ...
%   \end{document}
%
% ==============================================================================

% =============================================================================
% SECTION 1: Encoding and Language
% =============================================================================
\usepackage[utf8]{inputenc}
\usepackage[T1]{fontenc}
\usepackage[ngerman]{babel}
\usepackage{lmodern}

% =============================================================================
% SECTION 2: Page Geometry
% =============================================================================
\usepackage[a4paper, left=2.5cm, right=2.5cm, top=2.5cm, bottom=3.5cm]{geometry}
\setlength{\headheight}{15pt}

% =============================================================================
% SECTION 3: Mathematics and Physics
% =============================================================================
\usepackage{amsmath,amssymb,amsfonts,amsthm}
\usepackage{mathtools}
\usepackage{physics}
\usepackage{siunitx}
\sisetup{
    locale=US,
    group-separator={,},
    output-decimal-marker={.},
    per-mode=symbol
}

% =============================================================================
% SECTION 4: Graphics and Tables
% =============================================================================
\usepackage{graphicx}
\usepackage[table,xcdraw]{xcolor}
\usepackage{tikz}
\usetikzlibrary{arrows.meta,positioning,shapes.geometric,decorations.pathmorphing,patterns,shapes.arrows,intersections}
\usepackage{pgfplots}
\pgfplotsset{compat=1.18}
\usepackage[most]{tcolorbox}
\tcbuselibrary{breakable}
\usepackage{booktabs}
\usepackage{array}
\usepackage{longtable}
\usepackage{float}
\usepackage{adjustbox}
\usepackage{rotating}
\usepackage{tabularx}
\usepackage{makecell}
\usepackage{multirow}

% =============================================================================
% SECTION 5: Document Formatting
% =============================================================================
\usepackage{fancyhdr}
\renewcommand{\headrulewidth}{0.4pt}
\renewcommand{\footrulewidth}{0.4pt}
\usepackage{tocloft}
\usepackage{hyperref}
\hypersetup{
  colorlinks=true,
  linkcolor=black,
  citecolor=black,
  urlcolor=black,
  breaklinks=true,
  bookmarksnumbered=true,
  unicode=true
}
\usepackage{bookmark}
\usepackage{cleveref}

% Table of contents: only show chapters (not sections/subsections)
\setcounter{tocdepth}{3}  % Show sections, subsections, and subsubsections
\usepackage{microtype}
\usepackage{enumitem}
\usepackage{setspace}
\usepackage{ragged2e}
\usepackage{multicol}

% =============================================================================
% SECTION 6: Code and Algorithms
% =============================================================================
\usepackage{algorithm}
\usepackage{algorithmic}
\usepackage{listings}
\lstset{
  basicstyle=\ttfamily\footnotesize,
  breaklines=true,
  breakatwhitespace=true,
  columns=flexible,
  keepspaces=true,
  showstringspaces=false,
  frame=single,
  xleftmargin=0pt,
  xrightmargin=0pt
}
\usepackage{mdframed}

% =============================================================================
% SECTION 7: Additional Packages
% =============================================================================
\usepackage{pdflscape}
\usepackage{braket}
\usepackage{cancel}
\usepackage{caption}
\usepackage{csquotes}
\usepackage{gensymb}
\usepackage{hyphenat}
\usepackage{textcomp}
\usepackage{textgreek}
\usepackage{upgreek}
\usepackage{url}
\usepackage{slashed}
\usepackage{bm}
\usepackage{newunicodechar}

% =============================================================================
% SECTION 8: Citation Commands (Compatibility)
% =============================================================================
\providecommand{\citep}[1]{\cite{#1}}
\providecommand{\citet}[1]{\cite{#1}}

% =============================================================================
% SECTION 9: Colors
% =============================================================================
\definecolor{gold}{RGB}{255,215,0}
\definecolor{blue}{rgb}{0,0,1}
\definecolor{boxgray}{RGB}{240,240,240}
\definecolor{deepblue}{RGB}{0,0,127}
\definecolor{deepgreen}{RGB}{0,127,0}
\definecolor{deepred}{RGB}{191,0,0}
\definecolor{t0blue}{RGB}{33,150,243}
\definecolor{t0green}{RGB}{76,175,80}
\definecolor{t0orange}{RGB}{255,152,0}
\definecolor{t0purple}{RGB}{156,39,176}
\definecolor{t0red}{RGB}{244,67,54}
\definecolor{t0yellow}{RGB}{255,204,0}

% =============================================================================
% SECTION 10: Column Types
% =============================================================================
\newcolumntype{L}[1]{>{\raggedright\arraybackslash}p{#1}}
\newcolumntype{C}[1]{>{\centering\arraybackslash}p{#1}}

% =============================================================================
% SECTION 11: Unicode Character Mappings
% =============================================================================
\newunicodechar{ħ}{$\hbar$}
\newunicodechar{↔}{$\leftrightarrow$}
\newunicodechar{⇐}{$\Leftarrow$}
\newunicodechar{⇒}{$\Rightarrow$}
\newunicodechar{⇔}{$\Leftrightarrow$}
\newunicodechar{∂}{$\partial$}
\newunicodechar{∅}{$\emptyset$}
\newunicodechar{∇}{$\nabla$}
\newunicodechar{∈}{$\in$}
\newunicodechar{∉}{$\notin$}
\newunicodechar{∏}{$\prod$}
\newunicodechar{∑}{$\sum$}
% Note: √ is mapped to an empty sqrt; use \sqrt{x} for proper usage
\newunicodechar{√}{\ensuremath{\sqrt{}}}
\newunicodechar{∝}{$\propto$}
\newunicodechar{∞}{$\infty$}
\newunicodechar{∩}{$\cap$}
\newunicodechar{∪}{$\cup$}
\newunicodechar{∫}{$\int$}
\newunicodechar{≈}{$\approx$}
\newunicodechar{≠}{$\neq$}
\newunicodechar{≤}{$\leq$}
\newunicodechar{≥}{$\geq$}
\newunicodechar{ξ}{\ensuremath{\xi}}
\newunicodechar{μ}{\ensuremath{\mu}}
\newunicodechar{ψ}{\ensuremath{\psi}}
\newunicodechar{φ}{\ensuremath{\phi}}
\newunicodechar{π}{\ensuremath{\pi}}
\newunicodechar{λ}{\ensuremath{\lambda}}
\newunicodechar{Δ}{\ensuremath{\Delta}}

% =============================================================================
% SECTION 12: Hyperref Settings
% =============================================================================
\hypersetup{
    colorlinks=true,
    linkcolor=blue,
    citecolor=blue,
    urlcolor=blue,
    breaklinks=true,
    bookmarksnumbered=true,
    pdfstartview=FitH
}

% =============================================================================
% SECTION 13: Theorem Environments (English)
% =============================================================================
\theoremstyle{plain}
\newtheorem{theorem}{Theorem}[section]
\newtheorem{lemma}[theorem]{Lemma}
\newtheorem{proposition}[theorem]{Proposition}
\newtheorem{corollary}[theorem]{Corollary}

\theoremstyle{definition}
\newtheorem{definition}[theorem]{Definition}
\newtheorem{example}[theorem]{Example}
\newtheorem{insight}[theorem]{Insight}
\newtheorem{discovery}[theorem]{Discovery}
% \newtheorem{erkenntnis}[theorem]{Insight}  % Commented out - conflicts with tcolorbox environment below

\theoremstyle{remark}
\newtheorem{remark}[theorem]{Remark}
\newtheorem{axiom}{Axiom}
\newtheorem{principle}{Principle}
\newtheorem{bemerkung}[theorem]{Remark}
\newtheorem{warnung}[theorem]{Warning}

% =============================================================================
% SECTION 14: T0-Specific Commands
% =============================================================================

% --- Core T0 Fields ---
\newcommand{\Tfield}{T(x,t)}
\providecommand{\Tfieldt}{T(\vec{x},t)}
\newcommand{\Efield}{E(x,t)}
\newcommand{\mfield}{m(x,t)}
\providecommand{\vecx}{\vec{x}}

% --- Lagrangian ---
\newcommand{\Lag}{\mathcal{L}}
\newcommand{\calL}{\mathcal{L}}

% --- Greek Letters and Constants ---
\newcommand{\alphaem}{\alpha}
\newcommand{\betaT}{\beta_T}
\newcommand{\xiT}{\xi}
\newcommand{\xipar}{\xi}

% --- Energy and Planck Units ---
\newcommand{\Ezero}{E_0}
\newcommand{\EPlanck}{E_{\text{Pl}}}
\newcommand{\Mpl}{M_{\text{Pl}}}
\newcommand{\mP}{m_{\text{P}}}
\newcommand{\lP}{\ell_{\text{P}}}
\newcommand{\tP}{t_{\text{P}}}
\newcommand{\LPlanck}{\ell_{\text{Pl}}}
\newcommand{\TPlanck}{t_{\text{Pl}}}

% --- Coupling Constants ---
\newcommand{\Gnat}{G_{\text{nat}}}
\newcommand{\alphaEM}{\alpha_{\text{EM}}}
\newcommand{\alphaSI}{\alpha_{\text{SI}}}
\newcommand{\Hubble}{H_0}
\newcommand{\LCDM}{\Lambda\text{CDM}}
\newcommand{\natunits}{(nat. units)}

% --- T0 Model Parameters ---
\newcommand{\xigeom}{\xi_{\mathrm{geom}}}
\newcommand{\rzero}{r_{0}}
\newcommand{\xirat}{\xi_{\mathrm{rat}}}
\newcommand{\tzero}{t_{0}}
\newcommand{\Lambdat}{\Lambda_{\mathrm{t}}}
\newcommand{\EP}{E_{\mathrm{P}}}
\newcommand{\Emu}{E_{\mu}}
\newcommand{\Ee}{E_{e}}
\newcommand{\Etau}{E_{\tau}}
\newcommand{\alphafine}{\alpha_{\mathrm{fine}}}
\newcommand{\alphal}{\alpha_{\ell}}
\newcommand{\Lzero}{\ell_{0}}
\newcommand{\Lp}{\ell_{\mathrm{P}}}

% --- Additional T0 Commands ---
\newcommand{\Kfrak}{K_{\text{frak}}}
\newcommand{\Dfrak}{D_{\text{frak}}}
\newcommand{\betapar}{\beta_T}
\newcommand{\alphapar}{\alpha}
\newcommand{\deltafield}{\delta \phi}
\newcommand{\deltam}{\delta m}
\newcommand{\deltaE}{\delta E}
\newcommand{\Exi}{E_{\xi}}
\newcommand{\Lxi}{\ell_{\xi}}
\newcommand{\rhoCMB}{\rho_{\text{CMB}}}
\newcommand{\rhoCasimir}{\rho_{\text{Casimir}}}
\newcommand{\Leff}{L_{\text{eff}}}
\newcommand{\CQCD}{C_{\mathrm{QCD}}}
\newcommand{\Kspec}{K_{\mathrm{spec}}}
\newcommand{\Tzero}{\ensuremath{T_0}}
\newcommand{\Eabs}{E_{\text{abs}}}
\newcommand{\taupar}{\tau}

% --- Provided Commands (may be redefined elsewhere) ---
\providecommand{\xiconst}{\xi_{\text{const}}}
\providecommand{\DhiggsT}{D_{\text{Higgs-T}}}
\providecommand{\rhoE}{\rho_{E}}
\providecommand{\Echar}{E_{\text{char}}}
\providecommand{\kfrac}{k_{\text{frac}}}
\providecommand{\alphaEMSI}{\alpha_{\text{EM,SI}}}
\providecommand{\alphaEMnat}{\alpha_{\text{EM,nat}}}
\providecommand{\betaTSI}{\beta_{T,\text{SI}}}
\providecommand{\betaTnat}{\beta_{T,\text{nat}}}
\providecommand{\Gsi}{G_{\text{SI}}}
\providecommand{\xiparSI}{\xi_{\text{SI}}}
\providecommand{\xiparnat}{\xi_{\text{nat}}}
\providecommand{\meff}{m_{\text{eff}}}
\providecommand{\Tzerot}{T_{0}(t)}
\providecommand{\mzerot}{m_{0}(t)}
\providecommand{\Ezeroabs}{E_{0,\text{abs}}}
\providecommand{\Epar}{E_{\text{par}}}
\providecommand{\Lnat}{\ell_{\text{nat}}}
\providecommand{\Tnat}{T_{\text{nat}}}
\providecommand{\xifrak}{\xi_{\text{frac}}}
\providecommand{\Tfrak}{T_{\text{frac}}}
\providecommand{\mfrak}{m_{\text{frac}}}
\providecommand{\Dfrac}{D_{\text{frac}}}
\providecommand{\EphotSI}{E_{\gamma,\text{SI}}}
\providecommand{\EphotNat}{E_{\gamma,\text{nat}}}
\providecommand{\Eabsint}{E_{\text{abs,int}}}
\providecommand{\mphoton}{m_{\gamma}}
\providecommand{\Evis}{E_{\text{vis}}}
\providecommand{\Cto}{C_{T0}}
\providecommand{\mytimes}{\times}
\providecommand{\lambdah}{\lambda_h}
\providecommand{\checkmarkx}{\checkmark}
\providecommand{\Enorm}{E_{\text{norm}}}
\providecommand{\Tobs}{T_{\text{obs}}}
\providecommand{\mobs}{m_{\text{obs}}}
\providecommand{\Eobs}{E_{\text{obs}}}
\providecommand{\Lobs}{\ell_{\text{obs}}}
\providecommand{\xobs}{\xi_{\text{obs}}}
\providecommand{\calE}{\mathcal{E}}
\providecommand{\calT}{\mathcal{T}}
\providecommand{\calM}{\mathcal{M}}
\providecommand{\alphag}{\alpha_g}
\providecommand{\Tmax}{T_{\text{max}}}
\providecommand{\mmin}{m_{\text{min}}}
\providecommand{\Lmax}{\ell_{\text{max}}}
\providecommand{\Emin}{E_{\text{min}}}
\providecommand{\Geff}{G_{\text{eff}}}
\providecommand{\rhoeff}{\rho_{\text{eff}}}
\providecommand{\xieff}{\xi_{\text{eff}}}
\providecommand{\Teff}{T_{\text{eff}}}
\providecommand{\hPlanck}{h}
\providecommand{\kB}{k_B}
\providecommand{\muB}{\mu_B}
\providecommand{\lambdaC}{\lambda_C}
\providecommand{\omegaP}{\omega_P}
\providecommand{\rhoP}{\rho_P}
\providecommand{\Tref}{T_{\text{ref}}}
\providecommand{\Eref}{E_{\text{ref}}}
\providecommand{\mref}{m_{\text{ref}}}
\providecommand{\Lref}{\ell_{\text{ref}}}
\providecommand{\xikonst}{\xi_0}
\providecommand{\Phiphoton}{\Phi_{\gamma}}
\providecommand{\etavis}{\eta_{\text{vis}}}
\providecommand{\pichar}{\pi}
\providecommand{\primrel}{\mathcal{P}_{\text{rel}}}
\providecommand{\warningx}{\textcolor{orange}{\textbf{!}}}
\providecommand{\phiT}{\phi_T}
\providecommand{\Lorentz}{\Lambda}
\providecommand{\Cconv}{C_{\text{conv}}}
\providecommand{\Df}{\Delta f}
\providecommand{\lambdazero}{\lambda_0}
\providecommand{\myapprox}{\approx}
\providecommand{\checked}{\checkmark}
\providecommand{\alphaWSI}{\alpha_W^{\text{SI}}}
\providecommand{\alphaWnat}{\alpha_W^{\text{nat}}}
\providecommand{\vect}[1]{\vec{#1}}
\providecommand{\Rzero}{R_0}
\providecommand{\Riem}{\mathcal{R}}
\providecommand{\nuzero}{\nu_0}
\providecommand{\mypi}{\pi}

% =============================================================================
% SECTION 15: tcolorbox Styles and Environments
% =============================================================================

% --- Predefined Styles ---
\tcbset{
    keyresult/.style={
        colback=blue!5!white,
        colframe=blue!75!black,
        title=Key Result,
        fonttitle=\bfseries
    },
    foundation/.style={
        colback=green!5!white,
        colframe=green!75!black,
        title=Foundation,
        fonttitle=\bfseries
    },
    alternative/.style={
        colback=orange!5!white,
        colframe=orange!75!black,
        title=Alternative,
        fonttitle=\bfseries
    },
    warningbox/.style={
        colback=red!5!white,
        colframe=red!75!black,
        title=Warning,
        fonttitle=\bfseries
    }
}

% --- Core Environments ---
\newtcolorbox{keyresultbox}[1][]{colback=blue!5!white,colframe=blue!75!black,fonttitle=\bfseries,title={#1},breakable}
\newtcolorbox{keyresult}[1][Key Result]{colback=blue!5!white,colframe=blue!75!black,fonttitle=\bfseries,title={#1},breakable}
\newtcolorbox{foundationbox}[1][]{colback=green!5!white,colframe=green!75!black,fonttitle=\bfseries,title={#1},breakable}
\newtcolorbox{foundation}[1][Foundation]{colback=green!5!white,colframe=green!75!black,fonttitle=\bfseries,title={#1},breakable}
\newtcolorbox{alternativebox}[1][]{colback=orange!5!white,colframe=orange!75!black,fonttitle=\bfseries,title={#1},breakable}
\newtcolorbox{warningboxenv}[1][]{colback=red!5!white,colframe=red!75!black,fonttitle=\bfseries,title={#1},breakable}

% --- Formula Environments ---
\newtcolorbox{fundamental}[1][]{
    colback=boxgray,
    colframe=t0blue,
    fonttitle=\bfseries,
    title=#1,
    sharp corners,
    boxrule=2pt
}

\newtcolorbox{newperspective}[1][]{
    colback=red!5!white,
    colframe=t0red,
    fonttitle=\bfseries,
    title=#1,
    sharp corners,
    boxrule=2pt
}

\newtcolorbox{formula}[1][]{
    colback=blue!5!white,
    colframe=blue!75!black,
    fonttitle=\bfseries,
    title=#1
}

\newtcolorbox{result}[1][]{
    colback=green!5!white,
    colframe=green!75!black,
    fonttitle=\bfseries,
    title=#1
}

\newtcolorbox{derivation}[1][]{
    colback=green!5!white,
    colframe=green!75!black,
    title=#1,
    fonttitle=\bfseries,
    breakable
}

\newtcolorbox{summary}[1][]{
    colback=gray!10!white,
    colframe=gray!75!black,
    title=#1,
    fonttitle=\bfseries,
    breakable
}

\newtcolorbox{comparison}[1][]{
    colback=purple!5!white,
    colframe=purple!75!black,
    title=#1,
    fonttitle=\bfseries,
    breakable
}

\newtcolorbox{relation}[1][]{
    colback=cyan!5!white,
    colframe=cyan!75!black,
    title=#1,
    fonttitle=\bfseries,
    breakable
}

\newtcolorbox{principleBox}[1][]{
    colback=yellow!5!white,
    colframe=yellow!75!black,
    title=#1,
    fonttitle=\bfseries,
    breakable
}

% --- Insight and Discovery Environments ---
\newtcolorbox{insightBox}[1][]{colback=blue!5,colframe=t0blue,title={#1},fonttitle=\bfseries,breakable}
\newtcolorbox{discoveryBox}[1][]{colback=green!5,colframe=t0green,title={#1},fonttitle=\bfseries,breakable}
\newtcolorbox{revelation}[1][]{colback=red!5,colframe=t0red,title={#1},fonttitle=\bfseries,breakable}
\newtcolorbox{keypoint}[1][]{colback=blue!5,colframe=t0blue,title={#1},fonttitle=\bfseries,breakable}
\newtcolorbox{evidence}[1][]{colback=green!5,colframe=t0green,title={#1},fonttitle=\bfseries,breakable}
\newtcolorbox{conclusionBox}[1][]{colback=gray!5,colframe=gray,title={#1},fonttitle=\bfseries,breakable}
\newtcolorbox{significance}[1][]{colback=yellow!5,colframe=orange,title={#1},fonttitle=\bfseries,breakable}
\newtcolorbox{philosophical}[1][]{colback=purple!5,colframe=purple,title={#1},fonttitle=\bfseries,breakable}
\newtcolorbox{implicationBox}[1][]{colback=cyan!5,colframe=cyan,title={#1},fonttitle=\bfseries,breakable}
\newtcolorbox{perspectiveBox}[1][]{colback=blue!5,colframe=t0blue,title={#1},fonttitle=\bfseries,breakable}
\newtcolorbox{revolutionary}[1][]{colback=red!5,colframe=t0red,title={#1},fonttitle=\bfseries,breakable}

% --- Technical Environments ---
\newtcolorbox{technical}[1][]{colback=gray!5,colframe=gray!75!black,title={#1},fonttitle=\bfseries,breakable}
\newtcolorbox{technicalBox}[1][]{colback=gray!5,colframe=gray!75!black,title={#1},fonttitle=\bfseries,breakable}
\newtcolorbox{notationBox}[1][]{colback=yellow!5,colframe=yellow!75!black,title={#1},fonttitle=\bfseries,breakable}
\newtcolorbox{verification}[1][]{colback=orange!5!white,colframe=orange!75!black,fonttitle=\bfseries,title=#1}
\newtcolorbox{explanationBox}[1][]{colback=purple!5!white,colframe=purple!75!black,fonttitle=\bfseries,title=#1}
\newtcolorbox{interpretationBox}[1][]{colback=cyan!5!white,colframe=cyan!75!black,fonttitle=\bfseries,title=#1}
\newtcolorbox{explanation}[1][]{colback=purple!5!white,colframe=purple!75!black,fonttitle=\bfseries,title=#1,breakable}
\newtcolorbox{interpretation}[1][]{colback=cyan!5!white,colframe=cyan!75!black,fonttitle=\bfseries,title=#1,breakable}
\newtcolorbox{proof_step}[1][]{colback=gray!5!white,colframe=gray!75!black,fonttitle=\bfseries,title=#1,breakable}
\newtcolorbox{experimental}[1][]{colback=teal!5!white,colframe=teal!75!black,fonttitle=\bfseries,title=#1,breakable}

% --- Warning and Alert Environments ---
\newtcolorbox{important}[1][]{colback=red!5!white,colframe=red!75!black,title={#1},fonttitle=\bfseries,breakable}
\newtcolorbox{warning}[1][]{colback=orange!5!white,colframe=orange!75!black,title={#1},fonttitle=\bfseries,breakable}
\newtcolorbox{caution}[1][]{colback=yellow!5!white,colframe=yellow!75!black,title={#1},fonttitle=\bfseries,breakable}
\newtcolorbox{highlight}[1][]{colback=yellow!10!white,colframe=yellow!75!black,title={#1},fonttitle=\bfseries,breakable}

% --- Additional German-specific Environments for Matsas documents ---
\newtcolorbox{literatur}[1][Literatur]{colback=blue!5!white,colframe=blue!75!black,title={#1},fonttitle=\bfseries,breakable}
\newtcolorbox{zusammenfassung}[1][Zusammenfassung]{colback=green!5!white,colframe=green!75!black,title={#1},fonttitle=\bfseries,breakable}
\newtcolorbox{frage}[1][Frage]{colback=orange!5!white,colframe=orange!75!black,title={#1},fonttitle=\bfseries,breakable}
\newtcolorbox{erkenntnis}[1][Erkenntnis]{colback=purple!5!white,colframe=purple!75!black,title={#1},fonttitle=\bfseries,breakable}
\newtcolorbox{critical}[1][]{colback=red!10!white,colframe=red!75!black,title={#1},fonttitle=\bfseries,breakable}

% --- Analysis and Application Environments ---
\newtcolorbox{analysis}[1][]{colback=blue!5!white,colframe=blue!75!black,title={#1},fonttitle=\bfseries,breakable}
\newtcolorbox{application}[1][]{colback=green!5!white,colframe=green!75!black,title={#1},fonttitle=\bfseries,breakable}
\newtcolorbox{experiment}[1][]{colback=cyan!5!white,colframe=cyan!75!black,title={#1},fonttitle=\bfseries,breakable}
\newtcolorbox{historical}[1][]{colback=brown!5!white,colframe=brown!75!black,title={#1},fonttitle=\bfseries,breakable}
\newtcolorbox{numerical}[1][]{colback=gray!5!white,colframe=gray!75!black,title={#1},fonttitle=\bfseries,breakable}
\newtcolorbox{overview}[1][]{colback=blue!5!white,colframe=blue!75!black,title={#1},fonttitle=\bfseries,breakable}
\newtcolorbox{speculation}[1][]{colback=purple!5!white,colframe=purple!75!black,title={#1},fonttitle=\bfseries,breakable}
\newtcolorbox{question}[1][]{colback=orange!5!white,colframe=orange!75!black,title={#1},fonttitle=\bfseries,breakable}
\newtcolorbox{method}[1][]{colback=teal!5!white,colframe=teal!75!black,title={#1},fonttitle=\bfseries,breakable}
\newtcolorbox{correct}[1][]{colback=green!10!white,colframe=green!75!black,title={#1},fonttitle=\bfseries,breakable}
\newtcolorbox{units}[1][]{colback=gray!5!white,colframe=gray!75!black,title={#1},fonttitle=\bfseries,breakable}
\newtcolorbox{achievement}[1][]{colback=gold!5!white,colframe=orange!75!black,title={#1},fonttitle=\bfseries,breakable}
\newtcolorbox{equivalence}[1][]{colback=cyan!5!white,colframe=cyan!75!black,title={#1},fonttitle=\bfseries,breakable}
\newtcolorbox{dimensional}[1][]{colback=purple!5!white,colframe=purple!75!black,title={#1},fonttitle=\bfseries,breakable}

% --- Physics-specific Environments ---
\newtcolorbox{photon}[1][]{colback=yellow!5!white,colframe=yellow!75!black,title={#1},fonttitle=\bfseries,breakable}
\newtcolorbox{neutrino}[1][]{colback=blue!5!white,colframe=blue!75!black,title={#1},fonttitle=\bfseries,breakable}
\newtcolorbox{revolution}[1][]{colback=red!5!white,colframe=red!75!black,title={#1},fonttitle=\bfseries,breakable}
\newtcolorbox{t0box}[1][]{colback=blue!5!white,colframe=t0blue,title={#1},fonttitle=\bfseries,breakable}
\newtcolorbox{documentbox}[1][]{colback=gray!5!white,colframe=gray!75!black,title={#1},fonttitle=\bfseries,breakable}
\newtcolorbox{sibox}[1][]{colback=green!5!white,colframe=green!75!black,title={#1},fonttitle=\bfseries,breakable}
\newtcolorbox{smbox}[1][]{colback=blue!5!white,colframe=blue!75!black,title={#1},fonttitle=\bfseries,breakable}
\newtcolorbox{pvbox}[1][]{colback=purple!5!white,colframe=purple!75!black,title={#1},fonttitle=\bfseries,breakable}
\newtcolorbox{koidebox}[1][]{colback=orange!5!white,colframe=orange!75!black,title={#1},fonttitle=\bfseries,breakable}

% --- German Compatibility Environments ---
\newtcolorbox{formel}[1][]{colback=blue!5!white,colframe=blue!75!black,title={#1},fonttitle=\bfseries,breakable}
\newtcolorbox{schluessel}[1][]{colback=blue!5!white,colframe=blue!75!black,title={#1},fonttitle=\bfseries,breakable}
\newtcolorbox{wichtig}[1][]{colback=red!5!white,colframe=red!75!black,title={#1},fonttitle=\bfseries,breakable}
\newtcolorbox{vorsicht}[1][]{colback=orange!5!white,colframe=orange!75!black,title={#1},fonttitle=\bfseries,breakable}
\newtcolorbox{revolutionaer}[1][]{colback=red!5!white,colframe=red!75!black,title={#1},fonttitle=\bfseries,breakable}
\newtcolorbox{numerisch}[1][]{colback=gray!5!white,colframe=gray!75!black,title={#1},fonttitle=\bfseries,breakable}
\newtcolorbox{experimentell}[1][]{colback=cyan!5!white,colframe=cyan!75!black,title={#1},fonttitle=\bfseries,breakable}
\newtcolorbox{anwendung}[1][]{colback=green!5!white,colframe=green!75!black,title={#1},fonttitle=\bfseries,breakable}
\newtcolorbox{alternative}[1][]{colback=orange!5!white,colframe=orange!75!black,title={#1},fonttitle=\bfseries,breakable}
\newtcolorbox{beziehung}[1][]{colback=cyan!5!white,colframe=cyan!75!black,title={#1},fonttitle=\bfseries,breakable}
\newtcolorbox{folgerung}[1][]{colback=green!5!white,colframe=green!75!black,title={#1},fonttitle=\bfseries,breakable}
\newtcolorbox{abhandlung}[1][]{colback=gray!5!white,colframe=gray!75!black,title={#1},fonttitle=\bfseries,breakable}
\newtcolorbox{prinzipBox}[1][]{colback=blue!5!white,colframe=blue!75!black,title={#1},fonttitle=\bfseries,breakable}
\newtcolorbox{prinzip}[1][]{colback=blue!5!white,colframe=blue!75!black,title={#1},fonttitle=\bfseries,breakable}
\newtcolorbox{beweis}[1][]{colback=gray!5!white,colframe=gray!75!black,title={#1},fonttitle=\bfseries,breakable}
\newtcolorbox{key}[2][]{colback=blue!5!white,colframe=blue!75!black,title={#2},fonttitle=\bfseries,breakable}
\newtcolorbox{category}[1][]{colback=purple!5!white,colframe=purple!75!black,title={#1},fonttitle=\bfseries,breakable}

% =============================================================================
% SECTION 16: Additional Simple Environments
% =============================================================================
\newenvironment{treatise}{\begin{quote}}{\end{quote}}
\newenvironment{gemeinsam}{\begin{quote}}{\end{quote}}
\newenvironment{vergleich}{\begin{quote}}{\end{quote}}
\newenvironment{vorteil}{\begin{quote}}{\end{quote}}
\newenvironment{quantum}{\begin{quote}}{\end{quote}}

% =============================================================================
% SECTION 17: Layout Settings (Kindle-compatible)
% =============================================================================
\sloppy  % Allow more flexible line breaking
\hfuzz=65pt  % Suppress overfull warnings up to 65pt (Kindle compatibility)
\vfuzz=65pt  
\tolerance=9999  % High tolerance for bad line breaks
\emergencystretch=3em  % Extra stretch to avoid overfull boxes
\hbadness=10000  % Suppress underfull box warnings
\raggedbottom

% Environment for wide tables/longtables that need scaling
\newenvironment{scaledtable}[1][0.85]{%
  \begingroup\footnotesize\setlength{\LTleft}{0pt}\setlength{\LTright}{0pt}%
}{%
  \endgroup%
}

% Command for inline table scaling
\newcommand{\widetable}[1]{\resizebox{\textwidth}{!}{#1}}

% =============================================================================
% SECTION 18: Table of Contents Formatting
% =============================================================================
\renewcommand{\cftsecfont}{\color{blue}}
\renewcommand{\cftsubsecfont}{\color{blue}}
\renewcommand{\cftsecpagefont}{\color{blue}}
\renewcommand{\cftsubsecpagefont}{\color{blue}}
\renewcommand{\cfttoctitlefont}{\huge\bfseries\color{blue}}

% =============================================================================
% SECTION 19: Default Header and Footer
% =============================================================================
\pagestyle{fancy}
\fancyhf{}
\fancyhead[L]{\textsc{T0 Theory}}
\fancyhead[R]{\textsc{J. Pascher}}
\fancyfoot[C]{\thepage}

% ==============================================================================
% End of Shared Preamble
% ==============================================================================


\begin{document}

\title{Tabellen aus parameterherleitung\_De.tex}
\maketitle

\tableofcontents
\newpage

\section*{Tabelle 1 aus parameterherleitung\_De.tex}
\begin{adjustbox}{max width=\textwidth}
\begin{tabular}{|l|c|c|l|}
			\hline
			\textbf{Parameter} & \textbf{Symbol} & \textbf{Wert} & \textbf{Physikalische Bedeutung} \\
			\hline
			Massenskalierung & $\kappa_{\text{mass}}$ & 1.47 & Fraktaler Exponent, dimensionslos \\
			Gravitationsfeld & $\kappa_{\text{grav}}$ & $4.8 \times 10^{-11}$ m/s$^2$ & Modifikation des Potentials \\
			\hline
		\end{tabular}
\end{adjustbox}

\bigskip
\clearpage

\section*{Tabelle 2 aus parameterherleitung\_De.tex}
\begin{longtable}{p{5cm}p{4cm}p{3.5cm}p{3.5cm}}
	\caption{Standardmodell-Parameter in hierarchischer Ordnung ihrer T0-Ableitung} \\
	\toprule
	\textbf{SM-Parameter} & \textbf{SM-Wert} & \textbf{T0-Formel} & \textbf{T0-Wert} \\
	\midrule
	\endfirsthead
	
	\multicolumn{4}{c}{{\bfseries Fortsetzung der Tabelle}} \\
	\toprule
	\textbf{SM-Parameter} & \textbf{SM-Wert} & \textbf{T0-Formel} & \textbf{T0-Wert} \\
	\midrule
	\endhead
	
	\bottomrule
	\endfoot
	
	\bottomrule
	\endlastfoot
	
	% EBENE 0: FUNDAMENTALE KONSTANTE
	\multicolumn{4}{l}{\textbf{EBENE 0: FUNDAMENTALE GEOMETRISCHE KONSTANTE}} \\
	\midrule
	
	Geometrischer Parameter $\xi$ & -- & $\xi = \frac{4}{3} \times 10^{-4}$ & $1.333 \times 10^{-4}$ \\
	& & (von Geometry) & (exakt) \\[0.3em]
	
	\midrule
	% EBENE 1: DIREKTE ABLEITUNGEN AUS XI
	\multicolumn{4}{l}{\textbf{EBENE 1: PRIMÄRE KOPPLUNGSKONSTANTEN (nur von $\xi$ abhängig)}} \\
	\midrule
	
	Starke Kopplung $\alpha_S$ & $\alpha_S \approx 0.118$ & $\alpha_S = \xi^{-1/3}$ & $9.65$ \\
	& (bei $M_Z$) & $= (1.333 \times 10^{-4})^{-1/3}$ & (nat. Einheiten) \\[0.3em]
	
	Schwache Kopplung $\alpha_W$ & $\alpha_W \approx 1/30$ & $\alpha_W = \xi^{1/2}$ & $1.15 \times 10^{-2}$ \\
	& & $= (1.333 \times 10^{-4})^{1/2}$ & \\[0.3em]
	
	Gravitationskopplung $\alpha_G$ & nicht im SM & $\alpha_G = \xi^{2}$ & $1.78 \times 10^{-8}$ \\
	& & $= (1.333 \times 10^{-4})^{2}$ & \\[0.3em]
	
	Elektromagnetische Kopplung & $\alpha = 1/137.036$ & $\alpha_{EM} = 1$ (Konvention) & $1$ \\
	& & $\varepsilon_T = \xi \cdot \sqrt{3/(4\pi^2)}$ & $3.7 \times 10^{-5}$ \\
	& & (physikalische Kopplung) & (*siehe Anm.) \\[0.3em]
	
	\midrule
	% EBENE 2: ENERGIESKALEN
	\multicolumn{4}{l}{\textbf{EBENE 2: ENERGIESKALEN (von $\xi$ und Planck-Skala)}} \\
	\midrule
	
	Planck-Energie $E_P$ & $1.22 \times 10^{19}$ GeV & Referenzskala & $1.22 \times 10^{19}$ GeV \\
	& & (aus $G, \hbar, c$) & \\[0.3em]
	
Higgs-VEV $v$ & $246.22$ GeV & $v = \frac{4}{3} \cdot \xi_0^{-1/2} \cdot K_{\text{quantum}}$ & $246.2$ GeV \\
& (theoretisch) & (siehe Anhang) & \\[0.3em]

	
	QCD-Skala $\Lambda_{QCD}$ & $\sim 217$ MeV & $\Lambda_{QCD} = v \cdot \xi^{1/3}$ & $200$ MeV \\
	& (freier Parameter) & $= 246 \text{ GeV} \cdot \xi^{1/3}$ & \\[0.3em]
	
	\midrule
	% EBENE 3: HIGGS-SEKTOR
	\multicolumn{4}{l}{\textbf{EBENE 3: HIGGS-SEKTOR (von $v$ abhängig)}} \\
	\midrule
	
	Higgs-Masse $m_h$ & $125.25$ GeV & $m_h = v \cdot \xi^{1/4}$ & $125$ GeV \\
	& (gemessen) & $= 246 \cdot (1.333 \times 10^{-4})^{1/4}$ & \\[0.3em]
	
	Higgs-Selbstkopplung $\lambda_h$ & $0.13$ & $\lambda_h = \frac{m_h^2}{2v^2}$ & $0.129$ \\
	& (abgeleitet) & $= \frac{(125)^2}{2(246)^2}$ & \\[0.3em]
	
	\midrule
	% EBENE 4: FERMION-MASSEN
	\multicolumn{4}{l}{\textbf{EBENE 4: FERMION-MASSEN (von $v$ und $\xi$ abhängig)}} \\
	\midrule
	
	\multicolumn{4}{l}{\textit{Leptonen:}} \\
	
	Elektronmasse $m_e$ & $0.511$ MeV & $m_e = v \cdot \frac{4}{3} \cdot \xi^{3/2}$ & $0.502$ MeV \\
	& (freier Parameter) & $= 246 \text{ GeV} \cdot \frac{4}{3} \cdot \xi^{3/2}$ & \\[0.3em]
	
	Myonmasse $m_\mu$ & $105.66$ MeV & $m_\mu = v \cdot \frac{16}{5} \cdot \xi^1$ & $105.0$ MeV \\
	& (freier Parameter) & $= 246 \text{ GeV} \cdot \frac{16}{5} \cdot \xi$ & \\[0.3em]
	
	Taumasse $m_\tau$ & $1776.86$ MeV & $m_\tau = v \cdot \frac{5}{4} \cdot \xi^{2/3}$ & $1778$ MeV \\
	& (freier Parameter) & $= 246 \text{ GeV} \cdot \frac{5}{4} \cdot \xi^{2/3}$ & \\[0.3em]
	
	\multicolumn{4}{l}{\textit{Up-Typ Quarks:}} \\
	
	Up-Quarkmasse $m_u$ & $2.16$ MeV & $m_u = v \cdot 6 \cdot \xi^{3/2}$ & $2.27$ MeV \\
	
	Charm-Quarkmasse $m_c$ & $1.27$ GeV & $m_c = v \cdot \frac{8}{9} \cdot \xi^{2/3}$ & $1.279$ GeV \\
	
	Top-Quarkmasse $m_t$ & $172.76$ GeV & $m_t = v \cdot \frac{1}{28} \cdot \xi^{-1/3}$ & $173.0$ GeV \\
	
	\multicolumn{4}{l}{\textit{Down-Typ Quarks:}} \\
	
	Down-Quarkmasse $m_d$ & $4.67$ MeV & $m_d = v \cdot \frac{25}{2} \cdot \xi^{3/2}$ & $4.72$ MeV \\
	
	Strange-Quarkmasse $m_s$ & $93.4$ MeV & $m_s = v \cdot 3 \cdot \xi^1$ & $97.9$ MeV \\
	
	Bottom-Quarkmasse $m_b$ & $4.18$ GeV & $m_b = v \cdot \frac{3}{2} \cdot \xi^{1/2}$ & $4.254$ GeV \\
	
	\midrule
	% EBENE 5: NEUTRINO-MASSEN
	\multicolumn{4}{l}{\textbf{EBENE 5: NEUTRINO-MASSEN (von $v$ und doppeltem $\xi$ abhängig)}} \\
	\midrule
	
	Elektron-Neutrino $m_{\nu_e}$ & $< 2$ eV & $m_{\nu_e} = v \cdot r_{\nu_e} \cdot \xi^{3/2} \cdot \xi^3$ & $\sim 10^{-3}$ eV \\
	& (obere Grenze) & mit $r_{\nu_e} \sim 1$ & (Vorhersage) \\[0.3em]
	
	Myon-Neutrino $m_{\nu_\mu}$ & $< 0.19$ MeV & $m_{\nu_\mu} = v \cdot r_{\nu_\mu} \cdot \xi^{1} \cdot \xi^3$ & $\sim 10^{-2}$ eV \\
	
	Tau-Neutrino $m_{\nu_\tau}$ & $< 18.2$ MeV & $m_{\nu_\tau} = v \cdot r_{\nu_\tau} \cdot \xi^{2/3} \cdot \xi^3$ & $\sim 10^{-1}$ eV \\
	
	\midrule
	% EBENE 6: MISCHUNGSPARAMETER
	\multicolumn{4}{l}{\textbf{EBENE 6: MISCHUNGSMATRIZEN (von Massenverhältnissen abhängig)}} \\
	\midrule
	
	\multicolumn{4}{l}{\textit{CKM-Matrix (Quarks):}} \\
	
	$|V_{us}|$ (Cabibbo) & $0.22452$ & $|V_{us}| = \sqrt{\frac{m_d}{m_s}} \cdot f_{Cab}$ & $0.225$ \\
	& & mit $f_{Cab} = \sqrt{\frac{m_s - m_d}{m_s + m_d}}$ & \\[0.3em]
	
	$|V_{ub}|$ & $0.00365$ & $|V_{ub}| = \sqrt{\frac{m_d}{m_b}} \cdot \xi^{1/4}$ & $0.0037$ \\
	
	$|V_{ud}|$ & $0.97446$ & $|V_{ud}| = \sqrt{1 - |V_{us}|^2 - |V_{ub}|^2}$ & $0.974$ \\
	& & (Unitarität) & \\[0.3em]
	
	CKM CP-Phase $\delta_{CKM}$ & $1.20$ rad & $\delta_{CKM} = \arcsin(2\sqrt{2}\xi^{1/2}/3)$ & $1.2$ rad \\
	
	\multicolumn{4}{l}{\textit{PMNS-Matrix (Neutrinos):}} \\
	
	$\theta_{12}$ (Solar) & $33.44°$ & $\theta_{12} = \arcsin\sqrt{m_{\nu_1}/m_{\nu_2}}$ & $33.5°$ \\
	
	$\theta_{23}$ (Atmosphärisch) & $49.2°$ & $\theta_{23} = \arcsin\sqrt{m_{\nu_2}/m_{\nu_3}}$ & $49°$ \\
	
	$\theta_{13}$ (Reaktor) & $8.57°$ & $\theta_{13} = \arcsin(\xi^{1/3})$ & $8.6°$ \\
	
	PMNS CP-Phase $\delta_{CP}$ & unbekannt & $\delta_{CP} = \pi(1 - 2\xi)$ & $1.57$ rad \\
	
	\midrule
	% EBENE 7: ABGELEITETE PARAMETER
	\multicolumn{4}{l}{\textbf{EBENE 7: ABGELEITETE PARAMETER}} \\
	\midrule
	
	Weinberg-Winkel $\sin^2\theta_W$ & $0.2312$ & $\sin^2\theta_W = \frac{1}{4}(1-\sqrt{1-4\alpha_W})$ & $0.231$ \\
	& & mit $\alpha_W$ von Ebene 1 & \\[0.3em]
	
	Starke CP-Phase $\theta_{QCD}$ & $< 10^{-10}$ & $\theta_{QCD} = \xi^{2}$ & $1.78 \times 10^{-8}$ \\
	& (obere Grenze) & & (Vorhersage) \\
	
\end{longtable}

\bigskip
\clearpage

\section*{Tabelle 3 aus parameterherleitung\_De.tex}
\begin{adjustbox}{max width=\textwidth}
\begin{tabular}{lcc}
		\toprule
		\textbf{Parameterkategorie} & \textbf{SM (frei)} & \textbf{T0 (frei)} \\
		\midrule
		Kopplungskonstanten & 3 & 0 \\
		Fermion-Massen (geladen) & 9 & 0 \\
		Neutrino-Massen & 3 & 0 \\
		CKM-Matrix & 4 & 0 \\
		PMNS-Matrix & 4 & 0 \\
		Higgs-Parameter & 2 & 0 \\
		QCD-Parameter & 2 & 0 \\
		\midrule
		\textbf{Gesamt} & \textbf{27+} & \textbf{0} \\
		\bottomrule
	\end{tabular}
\end{adjustbox}

\bigskip
\clearpage

\section*{Tabelle 4 aus parameterherleitung\_De.tex}
\begin{longtable}{p{5cm}p{4cm}p{3.5cm}p{3.5cm}}
	\caption{Kosmologische Parameter in hierarchischer Ordnung} \\
	\toprule
	\textbf{Parameter} & \textbf{$\Lambda$CDM-Wert} & \textbf{T0-Formel} & \textbf{T0-Interpretation} \\
	\midrule
	\endfirsthead
	
	\multicolumn{4}{c}{{\bfseries Fortsetzung der Tabelle}} \\
	\toprule
	\textbf{Parameter} & \textbf{ΛCDM-Wert} & \textbf{T0-Formel} & \textbf{T0-Interpretation} \\
	\midrule
	\endhead
	
	\bottomrule
	\endfoot
	
	\bottomrule
	\endlastfoot
	
	% EBENE 0: FUNDAMENTALE KONSTANTE
	\multicolumn{4}{l}{\textbf{EBENE 0: FUNDAMENTALE GEOMETRISCHE KONSTANTE}} \\
	\midrule
	
	Geometrischer Parameter $\xi$ & nicht existent & $\xi = \frac{4}{3} \times 10^{-4}$ & $1.333 \times 10^{-4}$ \\
	& & (von Geometry) & Basis aller Ableitungen \\[0.3em]
	
	\midrule
	% EBENE 1: PRIMÄRE KOSMISCHE PARAMETER
	\multicolumn{4}{l}{\textbf{EBENE 1: PRIMÄRE ENERGIESKALEN (nur von $\xi$ abhängig)}} \\
	\midrule
	
	Charakteristische Energie & -- & $E_\xi = \frac{1}{\xi} = \frac{3}{4} \times 10^{4}$ & $7500$ (nat. Einh.) \\
	& & & CMB-Energieskala \\[0.3em]
	
	Charakteristische Länge & -- & $L_\xi = \xi$ & $1.33 \times 10^{-4}$ \\
	& & & (nat. Einheiten) \\[0.3em]
	
	$\xi$-Feld Energiedichte & -- & $\rho_\xi = E_\xi^4$ & $3.16 \times 10^{16}$ \\
	& & & Vakuumenergiedichte \\[0.3em]
	
	\midrule
	% EBENE 2: CMB-PARAMETER
	\multicolumn{4}{l}{\textbf{EBENE 2: CMB-PARAMETER (von $\xi$ und $E_\xi$ abhängig)}} \\
	\midrule
	
	CMB-Temperatur heute & $T_0 = 2.7255$ K & $T_{CMB} = \frac{16}{9} \xi^2 \cdot E_\xi$ & $2.725$ K \\
	& (gemessen) & $= \frac{16}{9} \cdot (1.33 \times 10^{-4})^2 \cdot 7500$ & (berechnet) \\[0.3em]
	
	CMB-Energiedichte & $\rho_{CMB} = 4.64 \times 10^{-31}$ kg/m³ & $\rho_{CMB} = \frac{\pi^2}{15} T_{CMB}^4$ & $4.2 \times 10^{-14}$ J/m³ \\
	& & Stefan-Boltzmann & (nat. Einheiten) \\[0.3em]
	
	CMB-Anisotropie & $\Delta T/T \sim 10^{-5}$ & $\delta T = \xi^{1/2} \cdot T_{CMB}$ & $\sim 10^{-5}$ \\
	& (Planck-Satellit) & Quantenfluktuation & (vorhergesagt) \\[0.3em]
	
	\midrule
	% EBENE 3: ROTVERSCHIEBUNG
	\multicolumn{4}{l}{\textbf{EBENE 3: ROTVERSCHIEBUNG (von $\xi$ und Wellenlänge abhängig)}} \\
	\midrule
	
	Hubble-Konstante $H_0$ & $67.4 \pm 0.5$ km/s/Mpc & Nicht expandierend & -- \\
	& (Planck 2020) & Statisches Universum & \\[0.3em]
	
	Rotverschiebung $z$ & $z = \frac{\Delta\lambda}{\lambda}$ & $z(\lambda, d) = \xi \cdot \lambda \cdot d$ & Energieverlust \\
	& (Expansion) & Wellenlängenabhängig! & nicht Expansion \\[0.3em]
	
	Effektive $H_0$ & $67.4$ km/s/Mpc & $H_0^{eff} = c \cdot \xi \cdot \lambda_{ref}$ & $67.45$ km/s/Mpc \\
	(Interpretiert) & & bei $\lambda_{ref} = 550$ nm & (scheinbar) \\[0.3em]
	
	\midrule
	% EBENE 4: DUNKLE MATERIE/ENERGIE
	\multicolumn{4}{l}{\textbf{EBENE 4: DUNKLE KOMPONENTEN}} \\
	\midrule
	
	Dunkle Energie $\Omega_\Lambda$ & $0.6847 \pm 0.0073$ & Nicht erforderlich & $0$ \\
	& (68.47\% des Universums) & Statisches Universum & entfällt \\[0.3em]
	
	Dunkle Materie $\Omega_{DM}$ & $0.2607 \pm 0.0067$ & $\xi$-Feld-Effekte & $0$ \\
	& (26.07\% des Universums) & Modifizierte Gravitation & entfällt \\[0.3em]
	
	Baryonische Materie $\Omega_b$ & $0.0492 \pm 0.0003$ & Gesamte Materie & $1.0$ \\
	& (4.92\% des Universums) & & (100\%) \\[0.3em]
	
	Kosmolog. Konstante $\Lambda$ & $(1.1 \pm 0.02) \times 10^{-52}$ m$^{-2}$ & $\Lambda = 0$ & $0$ \\
	& & Keine Expansion & entfällt \\[0.3em]
	
	\midrule
	% EBENE 5: UNIVERSUMSALTER UND STRUKTUR
	\multicolumn{4}{l}{\textbf{EBENE 5: UNIVERSUMSSTRUKTUR}} \\
	\midrule
	
	Universumsalter & $13.787 \pm 0.020$ Gyr & $t_{univ} = \infty$ & Ewig \\
	& (seit Urknall) & Kein Anfang/Ende & Statisch \\[0.3em]
	
	Urknall & $t = 0$ & Kein Urknall & -- \\
	& Singularität & Heisenberg verbietet & Unmöglich \\[0.3em]
	
	Entkopplung (CMB) & $z \approx 1100$ & CMB aus $\xi$-Feld & Kontinuierlich \\
	& $t = 380,000$ Jahre & Vakuumfluktuation & erzeugt \\[0.3em]
	
	Strukturbildung & Bottom-up & Kontinuierlich & Zyklisch \\
	& (kleine → große) & $\xi$-getrieben & regenerierend \\[0.3em]
	
	\midrule
	% EBENE 6: VORHERSAGEN UND TESTS
	\multicolumn{4}{l}{\textbf{EBENE 6: UNTERSCHEIDBARE VORHERSAGEN}} \\
	\midrule
	
	Hubble-Spannung & Ungelöst & Gelöst durch & Keine Spannung \\
	& $H_0^{lokal} \neq H_0^{CMB}$ & $\xi$-Effekte & $H_0^{eff} = 67.45$ \\[0.3em]
	
	JWST frühe Galaxien & Problem & Kein Problem & Erwartbar in \\
	& (zu früh gebildet) & Ewiges Universum & statischem Univ. \\[0.3em]
	
	$\lambda$-abhängige $z$ & $z$ unabhängig von $\lambda$ & $z \propto \lambda$ & An der Grenze \\
	& Alle $\lambda$ gleiche $z$ & $z_{UV} > z_{Radio}$ & des Testbaren* \\[0.3em]
	
	Casimir-Effekt & Quantenfluktuation & $F_{Cas} = -\frac{\pi^2}{240} \frac{\hbar c}{d^4}$ & $\xi$-Feld \\
	& & aus $\xi$-Geometrie & Manifestation \\[0.3em]
	
	\midrule
	% EBENE 7: ENERGIEERHALTUNG
	\multicolumn{4}{l}{\textbf{EBENE 7: ENERGIEBILANZEN}} \\
	\midrule
	
	Gesamtenergie & Nicht erhalten & $E_{total} = const$ & Strikt erhalten \\
	& (Expansion) & & \\[0.3em]
	
	Materie-Energie & $E = mc^2$ & $E = mc^2$ & Identisch** \\
	Äquivalenz & & & (siehe Anm.) \\[0.3em]
	
	Vakuumenergie & Problem & $\rho_{vac} = \rho_\xi$ & Natürlich aus \\
	& ($10^{120}$ Diskrepanz) & Exakt berechenbar & $\xi$ \\[0.3em]
	
	Entropie & Wächst monoton & $S_{total} = const$ & Zyklisch \\
	& (Wärmetod) & Regeneration & erhalten \\[0.3em]
	
\end{longtable}

\bigskip
\clearpage

\section*{Tabelle 5 aus parameterherleitung\_De.tex}
\begin{adjustbox}{max width=\textwidth}
\begin{tabular}{p{4cm}p{5cm}p{5cm}}
		\toprule
		\textbf{Phänomen} & \textbf{$\Lambda$CDM-Erklärung} & \textbf{T0-Erklärung} \\
		\midrule
		Rotverschiebung & Raumexpansion & Photon-Energieverlust durch $\xi$-Feld \\
		CMB & Rekombination bei $z=1100$ & $\xi$-Feld Gleichgewichtsstrahlung \\
		Dunkle Energie & 68\% des Universums & Nicht existent \\
		Dunkle Materie & 26\% des Universums & $\xi$-Feld Gravitationseffekte \\
		Hubble-Spannung & Ungelöst (4.4$\sigma$) & Natürlich erklärt \\
		JWST-Paradox & Unerklärte frühe Galaxien & Kein Problem im ewigen Universum \\
		\bottomrule
	\end{tabular}
\end{adjustbox}

\bigskip
\clearpage

\section*{Tabelle 6 aus parameterherleitung\_De.tex}
\begin{adjustbox}{max width=\textwidth}
\begin{tabular}{lcc}
		\toprule
		\textbf{Kosmologische Parameter} & \textbf{$\Lambda$CDM (frei)} & \textbf{T0 (frei)} \\
		\midrule
		Hubble-Konstante $H_0$ & 1 & 0 (aus $\xi$) \\
		Dunkle Energie $\Omega_{\Lambda}$ & 1 & 0 (entfällt) \\
		Dunkle Materie $\Omega_{DM}$ & 1 & 0 (entfällt) \\
		Baryonendichte $\Omega_b$ & 1 & 0 (aus $\xi$) \\
		Spektralindex $n_s$ & 1 & 0 (aus $\xi$) \\
		Optische Tiefe $\tau$ & 1 & 0 (aus $\xi$) \\
		\midrule
		\textbf{Gesamt} & \textbf{6+} & \textbf{0} \\
		\bottomrule
	\end{tabular}
\end{adjustbox}

\bigskip
\clearpage

\section*{Tabelle 7 aus parameterherleitung\_De.tex}
\begin{longtable}{lll}
	\toprule
	\textbf{Symbol} & \textbf{Bedeutung} & \textbf{Wert/Einheit} \\
	\midrule
	\endfirsthead
	\multicolumn{3}{c}{{\bfseries Fortsetzung}} \\
	\toprule
	\textbf{Symbol} & \textbf{Bedeutung} & \textbf{Wert/Einheit} \\
	\midrule
	\endhead
	\bottomrule
	\endfoot
	\bottomrule
	\endlastfoot
	
	$\xi$ & Geometrischer Parameter & $\frac{4}{3} \times 10^{-4}$ (dimensionslos) \\
	$c$ & Lichtgeschwindigkeit & $2.998 \times 10^8$ m/s \\
	$\hbar$ & Reduzierte Planck-Konstante & $1.055 \times 10^{-34}$ J·s \\
	$G$ & Gravitationskonstante & $6.674 \times 10^{-11}$ m³/(kg·s²) \\
	$k_B$ & Boltzmann-Konstante & $1.381 \times 10^{-23}$ J/K \\
	$e$ & Elementarladung & $1.602 \times 10^{-19}$ C \\
\end{longtable}

\bigskip
\clearpage

\section*{Tabelle 8 aus parameterherleitung\_De.tex}
\begin{longtable}{lll}
	\toprule
	\textbf{Symbol} & \textbf{Bedeutung} & \textbf{Formel} \\
	\midrule
	$\alpha$ & Feinstrukturkonstante & $1/137.036$ (SI) \\
	$\alpha_{EM}$ & Elektromagnetische Kopplung & $1$ (nat. Einh.) \\
	$\alpha_S$ & Starke Kopplung & $\xi^{-1/3}$ \\
	$\alpha_W$ & Schwache Kopplung & $\xi^{1/2}$ \\
	$\alpha_G$ & Gravitationskopplung & $\xi^{2}$ \\
	$\varepsilon_T$ & T0-Kopplungsparameter & $\xi \cdot E_0^2$ \\
	\bottomrule
\end{longtable}

\bigskip
\clearpage

\section*{Tabelle 9 aus parameterherleitung\_De.tex}
\begin{longtable}{lll}
	\toprule
	\textbf{Symbol} & \textbf{Bedeutung} & \textbf{Wert/Formel} \\
	\midrule
	$E_P$ & Planck-Energie & $1.22 \times 10^{19}$ GeV \\
	$E_\xi$ & Charakteristische Energie & $1/\xi = 7500$ (nat. Einh.) \\
	$E_0$ & Fundamentale EM-Energie & $7.398$ MeV \\
	$v$ & Higgs-VEV & $246.22$ GeV \\
	$m_h$ & Higgs-Masse & $125.25$ GeV \\
	$\Lambda_{QCD}$ & QCD-Skala & $\sim 200$ MeV \\
	$m_e$ & Elektronmasse & $0.511$ MeV \\
	$m_\mu$ & Myonmasse & $105.66$ MeV \\
	$m_\tau$ & Taumasse & $1776.86$ MeV \\
	$m_u, m_d$ & Up-, Down-Quarkmasse & $2.16$, $4.67$ MeV \\
	$m_c, m_s$ & Charm-, Strange-Quarkmasse & $1.27$ GeV, $93.4$ MeV \\
	$m_t, m_b$ & Top-, Bottom-Quarkmasse & $172.76$ GeV, $4.18$ GeV \\
	$m_{\nu_e}, m_{\nu_\mu}, m_{\nu_\tau}$ & Neutrinomassen & $< 2$ eV, $< 0.19$ MeV, $< 18.2$ MeV \\
	\bottomrule
\end{longtable}

\bigskip
\clearpage

\section*{Tabelle 10 aus parameterherleitung\_De.tex}
\begin{longtable}{lll}
	\toprule
	\textbf{Symbol} & \textbf{Bedeutung} & \textbf{Wert/Formel} \\
	\midrule
	$H_0$ & Hubble-Konstante & $67.4$ km/s/Mpc (ΛCDM) \\
	$T_{CMB}$ & CMB-Temperatur & $2.725$ K \\
	$z$ & Rotverschiebung & dimensionslos \\
	$\Omega_\Lambda$ & Dunkle-Energie-Dichte & $0.6847$ (ΛCDM), $0$ (T0) \\
	$\Omega_{DM}$ & Dunkle-Materie-Dichte & $0.2607$ (ΛCDM), $0$ (T0) \\
	$\Omega_b$ & Baryonendichte & $0.0492$ (ΛCDM), $1$ (T0) \\
	$\Lambda$ & Kosmologische Konstante & $(1.1 \pm 0.02) \times 10^{-52}$ m$^{-2}$ \\
	$\rho_\xi$ & ξ-Feld-Energiedichte & $E_\xi^4$ \\
	$\rho_{CMB}$ & CMB-Energiedichte & $4.64 \times 10^{-31}$ kg/m³ \\
	\bottomrule
\end{longtable}

\bigskip
\clearpage

\section*{Tabelle 11 aus parameterherleitung\_De.tex}
\begin{longtable}{lll}
	\toprule
	\textbf{Symbol} & \textbf{Bedeutung} & \textbf{Wert/Formel} \\
	\midrule
	$D_f$ & Fraktale Dimension & $2.94$ \\
	$\kappa_{mass}$ & Massenskalierungsexponent & $D_f/2 = 1.47$ \\
	$\kappa_{grav}$ & Gravitationsfeldparameter & $4.8 \times 10^{-11}$ m/s² \\
	$\lambda_h$ & Higgs-Selbstkopplung & $0.13$ \\
	$\theta_W$ & Weinberg-Winkel & $\sin^2\theta_W = 0.2312$ \\
	$\theta_{QCD}$ & Starke CP-Phase & $< 10^{-10}$ (exp.), $\xi^2$ (T0) \\
	$\ell_P$ & Planck-Länge & $1.616 \times 10^{-35}$ m \\
	$\lambda_C$ & Compton-Wellenlänge & $\hbar/(mc)$ \\
	$r_g$ & Gravitationsradius & $2Gm$ \\
	$L_\xi$ & Charakteristische Länge & $\xi$ (nat. Einh.) \\
	\bottomrule
\end{longtable}

\bigskip
\clearpage

\section*{Tabelle 12 aus parameterherleitung\_De.tex}
\begin{longtable}{lll}
	\toprule
	\textbf{Symbol} & \textbf{Bedeutung} & \textbf{Typischer Wert} \\
	\midrule
	$V_{ij}$ & CKM-Matrixelemente & siehe Tabelle \\
	$|V_{ud}|$ & CKM ud-Element & $0.97446$ \\
	$|V_{us}|$ & CKM us-Element (Cabibbo) & $0.22452$ \\
	$|V_{ub}|$ & CKM ub-Element & $0.00365$ \\
	$\delta_{CKM}$ & CKM CP-Phase & $1.20$ rad \\
	$\theta_{12}$ & PMNS Solar-Winkel & $33.44°$ \\
	$\theta_{23}$ & PMNS Atmosphärisch & $49.2°$ \\
	$\theta_{13}$ & PMNS Reaktor-Winkel & $8.57°$ \\
	$\delta_{CP}$ & PMNS CP-Phase & unbekannt \\
	\bottomrule
\end{longtable}

\bigskip
\clearpage

\section*{Tabelle 13 aus parameterherleitung\_De.tex}
\begin{longtable}{lll}
	\toprule
	\textbf{Symbol} & \textbf{Bedeutung} & \textbf{Kontext} \\
	\midrule
	$n, l, j$ & Quantenzahlen & Teilchenklassifikation \\
	$r_i$ & Rationale Koeffizienten & Yukawa-Kopplungen \\
	$p_i$ & Generationsexponenten & $3/2, 1, 2/3, ...$ \\
	$f(n,l,j)$ & Geometrische Funktion & Massenformel \\
	$\rho_{tet}$ & Tetraeder-Packungsdichte & $0.68$ \\
	$\gamma$ & Universeller Exponent & $1.01$ \\
	$\nu$ & Kristallsymmetrie-Faktor & $0.63$ \\
	$\beta_T$ & Zeit-Feld-Kopplung & $1$ (nat. Einh.) \\
	$y_i$ & Yukawa-Kopplungen & $r_i \cdot \xi^{p_i}$ \\
	$T(x,t)$ & Zeitfeld & Fundamentale Fraktalgeometrische Feldtheorie (FFGFT, früher T0-Theorie) \\
	$E_{field}$ & Energiefeld & Universelles Feld \\
	\bottomrule
\end{longtable}

\bigskip
\clearpage

\section*{Tabelle 14 aus parameterherleitung\_De.tex}
\begin{tcolorbox}[colback=red!5!white,colframe=red!75!black]
	\textbf{Wichtig:} Die Normierung $(1 \text{ MeV})^2$ ist essentiell für dimensionslose Ergebnisse!
\end{tcolorbox}	
	\section{Alternative Herleitung durch fraktale Renormierung}
	
	Als unabh\"angige Best\"atigung kann $\alpha$ auch durch fraktale Renormierung hergeleitet werden:
	
	\begin{equation}
		\alpha_{\text{nackt}}^{-1} = 3\pi \times \xi^{-1} \times \ln\left(\frac{\Lambda_{\text{Planck}}}{m_\mu}\right)
	\end{equation}
	
	Mit dem fraktalen D\"ampfungsfaktor:
	\begin{equation}
		D_{\text{frak}} = \left(\frac{\lambda_C^{(\mu)}}{\ell_P}\right)^{D_f-2} = 4.2 \times 10^{-5}
	\end{equation}
	
	ergibt sich:
	\begin{equation}
		\alpha^{-1} = \alpha_{\text{nackt}}^{-1} \times D_{\text{frak}} = 137.036
	\end{equation}
	
	Diese unabh\"angige Herleitung best\"atigt das Resultat.
	
	\section{Kl\"arung: Die zwei verschiedenen $\kappa$-Parameter}
	
	\subsection{Wichtige Unterscheidung}
	
	In der Fundamentale Fraktalgeometrische Feldtheorie (FFGFT, früher T0-Theorie)-Literatur werden zwei physikalisch unterschiedliche Parameter mit dem Symbol $\kappa$ bezeichnet, was zu Verwirrung f\"uhren kann. Diese m\"ussen klar unterschieden werden:
	
	\begin{enumerate}
		\item $\kappa_{\text{mass}} = 1.47$ - Der fraktale Massenskalierungsexponent
		\item $\kappa_{\text{grav}}$ - Der Gravitationsfeldparameter
	\end{enumerate}
	
	\subsection{Der Massenskalierungsexponent $\kappa_{\text{mass}}$}
	
	Dieser Parameter wurde bereits in Abschnitt 4 hergeleitet:
	
	\begin{equation}
		\kappa_{\text{mass}} = \frac{D_f}{2} = 1.47
	\end{equation}
	
	Er ist dimensionslos und bestimmt die Skalierung in der Formel f\"ur magnetische Momente:
	
	\begin{equation}
		a_x \propto \left(\frac{m_x}{m_\mu}\right)^{\kappa_{\text{mass}}}
	\end{equation}
	
	\subsection{Der Gravitationsfeldparameter $\kappa_{\text{grav}}$}
	
	Dieser Parameter entsteht aus der Kopplung zwischen dem intrinsischen Zeitfeld und Materie. Die T0-Lagrangedichte lautet:
	
	\begin{equation}
		\mathcal{L}_{\text{intrinsic}} = \frac{1}{2}\partial_\mu T \partial^\mu T - \frac{1}{2}T^2 - \frac{\rho}{T}
	\end{equation}
	
	Die resultierende Feldgleichung:
	
	\begin{equation}
		\nabla^2 T = -\frac{\rho}{T^2}
	\end{equation}
	
	f\"uhrt zu einem modifizierten Gravitationspotential:
	
	\begin{equation}
		\Phi(r) = -\frac{GM}{r} + \kappa_{\text{grav}} r
	\end{equation}
	
	\subsection{Beziehung zwischen $\kappa_{\text{grav}}$ und fundamentalen Parametern}
	
	In nat\"urlichen Einheiten gilt:
	
	\begin{equation}
		\kappa_{\text{grav}}^{\text{nat}} = \beta_T^{\text{nat}} \cdot \frac{yv}{r_g^2}
	\end{equation}
	
	Mit $\beta_T = 1$ und $r_g = 2Gm_\mu$:
	
	\begin{equation}
		\kappa_{\text{grav}} = \frac{y_\mu \cdot v}{(2Gm_\mu)^2} = \frac{\sqrt{2} m_\mu \cdot v}{v \cdot 4G^2m_\mu^2} = \frac{\sqrt{2}}{4G^2m_\mu}
	\end{equation}
	
	\subsection{Numerischer Wert und physikalische Bedeutung}
	
	In SI-Einheiten:
	
	\begin{equation}
		\kappa_{\text{grav}}^{\text{SI}} \approx 4.8 \times 10^{-11} \text{ m/s}^2
	\end{equation}
	
	Dieser lineare Term im Gravitationspotential:
	\begin{itemize}
		\item Erkl\"art die beobachteten flachen Rotationskurven von Galaxien
		\item Eliminiert die Notwendigkeit f\"ur Dunkle Materie
		\item Entsteht nat\"urlich aus der Zeitfeld-Materie-Kopplung
	\end{itemize}
	
	\subsection{Zusammenfassung der $\kappa$-Parameter}
	
	\begin{center}
		\begin{tabular}{|l|c|c|l|}
			\hline
			\textbf{Parameter} & \textbf{Symbol} & \textbf{Wert} & \textbf{Physikalische Bedeutung} \\
			\hline
			Massenskalierung & $\kappa_{\text{mass}}$ & 1.47 & Fraktaler Exponent, dimensionslos \\
			Gravitationsfeld & $\kappa_{\text{grav}}$ & $4.8 \times 10^{-11}$ m/s$^2$ & Modifikation des Potentials \\
			\hline
		\end{tabular}
	\end{center}
	
	Die klare Unterscheidung dieser beiden Parameter ist essentiell f\"ur das Verst\"andnis der Fundamentale Fraktalgeometrische Feldtheorie (FFGFT, früher T0-Theorie).
\section{Vollständige Zuordnung: Standardmodell-Parameter zu T0-Entsprechungen}
\label{sec:sm_t0_mapping}

\subsection{Übersicht der Parameterreduktion}
\label{subsec:parameter_overview}

Das Standardmodell benötigt über 20 freie Parameter, die experimentell bestimmt werden müssen. Das T0-System ersetzt alle diese durch Ableitungen aus einer einzigen geometrischen Konstante:

\begin{equation}
	\boxed{\xi = \frac{4}{3} \times 10^{-4}}
\end{equation}

\subsection{Hierarchisch geordnete Parameter-Zuordnungstabelle}
\label{subsec:hierarchical_mapping}

Die Tabelle ist so organisiert, dass jeder Parameter erst definiert wird, bevor er in nachfolgenden Formeln verwendet wird.

\begin{longtable}{p{5cm}p{4cm}p{3.5cm}p{3.5cm}}
	\caption{Standardmodell-Parameter in hierarchischer Ordnung ihrer T0-Ableitung} \\
	\toprule
	\textbf{SM-Parameter} & \textbf{SM-Wert} & \textbf{T0-Formel} & \textbf{T0-Wert} \\
	\midrule
	\endfirsthead
	
	\multicolumn{4}{c}{{\bfseries Fortsetzung der Tabelle}} \\
	\toprule
	\textbf{SM-Parameter} & \textbf{SM-Wert} & \textbf{T0-Formel} & \textbf{T0-Wert} \\
	\midrule
	\endhead
	
	\bottomrule
	\endfoot
	
	\bottomrule
	\endlastfoot
	
	% EBENE 0: FUNDAMENTALE KONSTANTE
	\multicolumn{4}{l}{\textbf{EBENE 0: FUNDAMENTALE GEOMETRISCHE KONSTANTE}} \\
	\midrule
	
	Geometrischer Parameter $\xi$ & -- & $\xi = \frac{4}{3} \times 10^{-4}$ & $1.333 \times 10^{-4}$ \\
	& & (von Geometry) & (exakt) \\[0.3em]
	
	\midrule
	% EBENE 1: DIREKTE ABLEITUNGEN AUS XI
	\multicolumn{4}{l}{\textbf{EBENE 1: PRIMÄRE KOPPLUNGSKONSTANTEN (nur von $\xi$ abhängig)}} \\
	\midrule
	
	Starke Kopplung $\alpha_S$ & $\alpha_S \approx 0.118$ & $\alpha_S = \xi^{-1/3}$ & $9.65$ \\
	& (bei $M_Z$) & $= (1.333 \times 10^{-4})^{-1/3}$ & (nat. Einheiten) \\[0.3em]
	
	Schwache Kopplung $\alpha_W$ & $\alpha_W \approx 1/30$ & $\alpha_W = \xi^{1/2}$ & $1.15 \times 10^{-2}$ \\
	& & $= (1.333 \times 10^{-4})^{1/2}$ & \\[0.3em]
	
	Gravitationskopplung $\alpha_G$ & nicht im SM & $\alpha_G = \xi^{2}$ & $1.78 \times 10^{-8}$ \\
	& & $= (1.333 \times 10^{-4})^{2}$ & \\[0.3em]
	
	Elektromagnetische Kopplung & $\alpha = 1/137.036$ & $\alpha_{EM} = 1$ (Konvention) & $1$ \\
	& & $\varepsilon_T = \xi \cdot \sqrt{3/(4\pi^2)}$ & $3.7 \times 10^{-5}$ \\
	& & (physikalische Kopplung) & (*siehe Anm.) \\[0.3em]
	
	\midrule
	% EBENE 2: ENERGIESKALEN
	\multicolumn{4}{l}{\textbf{EBENE 2: ENERGIESKALEN (von $\xi$ und Planck-Skala)}} \\
	\midrule
	
	Planck-Energie $E_P$ & $1.22 \times 10^{19}$ GeV & Referenzskala & $1.22 \times 10^{19}$ GeV \\
	& & (aus $G, \hbar, c$) & \\[0.3em]
	
Higgs-VEV $v$ & $246.22$ GeV & $v = \frac{4}{3} \cdot \xi_0^{-1/2} \cdot K_{\text{quantum}}$ & $246.2$ GeV \\
& (theoretisch) & (siehe Anhang) & \\[0.3em]

	
	QCD-Skala $\Lambda_{QCD}$ & $\sim 217$ MeV & $\Lambda_{QCD} = v \cdot \xi^{1/3}$ & $200$ MeV \\
	& (freier Parameter) & $= 246 \text{ GeV} \cdot \xi^{1/3}$ & \\[0.3em]
	
	\midrule
	% EBENE 3: HIGGS-SEKTOR
	\multicolumn{4}{l}{\textbf{EBENE 3: HIGGS-SEKTOR (von $v$ abhängig)}} \\
	\midrule
	
	Higgs-Masse $m_h$ & $125.25$ GeV & $m_h = v \cdot \xi^{1/4}$ & $125$ GeV \\
	& (gemessen) & $= 246 \cdot (1.333 \times 10^{-4})^{1/4}$ & \\[0.3em]
	
	Higgs-Selbstkopplung $\lambda_h$ & $0.13$ & $\lambda_h = \frac{m_h^2}{2v^2}$ & $0.129$ \\
	& (abgeleitet) & $= \frac{(125)^2}{2(246)^2}$ & \\[0.3em]
	
	\midrule
	% EBENE 4: FERMION-MASSEN
	\multicolumn{4}{l}{\textbf{EBENE 4: FERMION-MASSEN (von $v$ und $\xi$ abhängig)}} \\
	\midrule
	
	\multicolumn{4}{l}{\textit{Leptonen:}} \\
	
	Elektronmasse $m_e$ & $0.511$ MeV & $m_e = v \cdot \frac{4}{3} \cdot \xi^{3/2}$ & $0.502$ MeV \\
	& (freier Parameter) & $= 246 \text{ GeV} \cdot \frac{4}{3} \cdot \xi^{3/2}$ & \\[0.3em]
	
	Myonmasse $m_\mu$ & $105.66$ MeV & $m_\mu = v \cdot \frac{16}{5} \cdot \xi^1$ & $105.0$ MeV \\
	& (freier Parameter) & $= 246 \text{ GeV} \cdot \frac{16}{5} \cdot \xi$ & \\[0.3em]
	
	Taumasse $m_\tau$ & $1776.86$ MeV & $m_\tau = v \cdot \frac{5}{4} \cdot \xi^{2/3}$ & $1778$ MeV \\
	& (freier Parameter) & $= 246 \text{ GeV} \cdot \frac{5}{4} \cdot \xi^{2/3}$ & \\[0.3em]
	
	\multicolumn{4}{l}{\textit{Up-Typ Quarks:}} \\
	
	Up-Quarkmasse $m_u$ & $2.16$ MeV & $m_u = v \cdot 6 \cdot \xi^{3/2}$ & $2.27$ MeV \\
	
	Charm-Quarkmasse $m_c$ & $1.27$ GeV & $m_c = v \cdot \frac{8}{9} \cdot \xi^{2/3}$ & $1.279$ GeV \\
	
	Top-Quarkmasse $m_t$ & $172.76$ GeV & $m_t = v \cdot \frac{1}{28} \cdot \xi^{-1/3}$ & $173.0$ GeV \\
	
	\multicolumn{4}{l}{\textit{Down-Typ Quarks:}} \\
	
	Down-Quarkmasse $m_d$ & $4.67$ MeV & $m_d = v \cdot \frac{25}{2} \cdot \xi^{3/2}$ & $4.72$ MeV \\
	
	Strange-Quarkmasse $m_s$ & $93.4$ MeV & $m_s = v \cdot 3 \cdot \xi^1$ & $97.9$ MeV \\
	
	Bottom-Quarkmasse $m_b$ & $4.18$ GeV & $m_b = v \cdot \frac{3}{2} \cdot \xi^{1/2}$ & $4.254$ GeV \\
	
	\midrule
	% EBENE 5: NEUTRINO-MASSEN
	\multicolumn{4}{l}{\textbf{EBENE 5: NEUTRINO-MASSEN (von $v$ und doppeltem $\xi$ abhängig)}} \\
	\midrule
	
	Elektron-Neutrino $m_{\nu_e}$ & $< 2$ eV & $m_{\nu_e} = v \cdot r_{\nu_e} \cdot \xi^{3/2} \cdot \xi^3$ & $\sim 10^{-3}$ eV \\
	& (obere Grenze) & mit $r_{\nu_e} \sim 1$ & (Vorhersage) \\[0.3em]
	
	Myon-Neutrino $m_{\nu_\mu}$ & $< 0.19$ MeV & $m_{\nu_\mu} = v \cdot r_{\nu_\mu} \cdot \xi^{1} \cdot \xi^3$ & $\sim 10^{-2}$ eV \\
	
	Tau-Neutrino $m_{\nu_\tau}$ & $< 18.2$ MeV & $m_{\nu_\tau} = v \cdot r_{\nu_\tau} \cdot \xi^{2/3} \cdot \xi^3$ & $\sim 10^{-1}$ eV \\
	
	\midrule
	% EBENE 6: MISCHUNGSPARAMETER
	\multicolumn{4}{l}{\textbf{EBENE 6: MISCHUNGSMATRIZEN (von Massenverhältnissen abhängig)}} \\
	\midrule
	
	\multicolumn{4}{l}{\textit{CKM-Matrix (Quarks):}} \\
	
	$|V_{us}|$ (Cabibbo) & $0.22452$ & $|V_{us}| = \sqrt{\frac{m_d}{m_s}} \cdot f_{Cab}$ & $0.225$ \\
	& & mit $f_{Cab} = \sqrt{\frac{m_s - m_d}{m_s + m_d}}$ & \\[0.3em]
	
	$|V_{ub}|$ & $0.00365$ & $|V_{ub}| = \sqrt{\frac{m_d}{m_b}} \cdot \xi^{1/4}$ & $0.0037$ \\
	
	$|V_{ud}|$ & $0.97446$ & $|V_{ud}| = \sqrt{1 - |V_{us}|^2 - |V_{ub}|^2}$ & $0.974$ \\
	& & (Unitarität) & \\[0.3em]
	
	CKM CP-Phase $\delta_{CKM}$ & $1.20$ rad & $\delta_{CKM} = \arcsin(2\sqrt{2}\xi^{1/2}/3)$ & $1.2$ rad \\
	
	\multicolumn{4}{l}{\textit{PMNS-Matrix (Neutrinos):}} \\
	
	$\theta_{12}$ (Solar) & $33.44°$ & $\theta_{12} = \arcsin\sqrt{m_{\nu_1}/m_{\nu_2}}$ & $33.5°$ \\
	
	$\theta_{23}$ (Atmosphärisch) & $49.2°$ & $\theta_{23} = \arcsin\sqrt{m_{\nu_2}/m_{\nu_3}}$ & $49°$ \\
	
	$\theta_{13}$ (Reaktor) & $8.57°$ & $\theta_{13} = \arcsin(\xi^{1/3})$ & $8.6°$ \\
	
	PMNS CP-Phase $\delta_{CP}$ & unbekannt & $\delta_{CP} = \pi(1 - 2\xi)$ & $1.57$ rad \\
	
	\midrule
	% EBENE 7: ABGELEITETE PARAMETER
	\multicolumn{4}{l}{\textbf{EBENE 7: ABGELEITETE PARAMETER}} \\
	\midrule
	
	Weinberg-Winkel $\sin^2\theta_W$ & $0.2312$ & $\sin^2\theta_W = \frac{1}{4}(1-\sqrt{1-4\alpha_W})$ & $0.231$ \\
	& & mit $\alpha_W$ von Ebene 1 & \\[0.3em]
	
	Starke CP-Phase $\theta_{QCD}$ & $< 10^{-10}$ & $\theta_{QCD} = \xi^{2}$ & $1.78 \times 10^{-8}$ \\
	& (obere Grenze) & & (Vorhersage) \\
	
\end{longtable}

\subsection{Zusammenfassung der Parameterreduktion}
\label{subsec:reduction_summary}

\begin{table}[h]
	\centering
	\begin{tabular}{lcc}
		\toprule
		\textbf{Parameterkategorie} & \textbf{SM (frei)} & \textbf{T0 (frei)} \\
		\midrule
		Kopplungskonstanten & 3 & 0 \\
		Fermion-Massen (geladen) & 9 & 0 \\
		Neutrino-Massen & 3 & 0 \\
		CKM-Matrix & 4 & 0 \\
		PMNS-Matrix & 4 & 0 \\
		Higgs-Parameter & 2 & 0 \\
		QCD-Parameter & 2 & 0 \\
		\midrule
		\textbf{Gesamt} & \textbf{27+} & \textbf{0} \\
		\bottomrule
	\end{tabular}
	\caption{Reduktion von 27+ freien Parametern auf eine einzige Konstante}
\end{table}

\subsection{Die hierarchische Ableitungsstruktur}
\label{subsec:hierarchical_structure}

Die Tabelle zeigt die klare Hierarchie der Parameterableitung:

\begin{enumerate}
	\item \textbf{Ebene 0}: Nur $\xi$ als fundamentale Konstante
	\item \textbf{Ebene 1}: Kopplungskonstanten direkt aus $\xi$
	\item \textbf{Ebene 2}: Energieskalen aus $\xi$ und Referenzskalen
	\item \textbf{Ebene 3}: Higgs-Parameter aus Energieskalen
	\item \textbf{Ebene 4}: Fermion-Massen aus $v$ und $\xi$
	\item \textbf{Ebene 5}: Neutrino-Massen mit zusätzlicher Unterdrückung
	\item \textbf{Ebene 6}: Mischungsparameter aus Massenverhältnissen
	\item \textbf{Ebene 7}: Weitere abgeleitete Parameter
\end{enumerate}

Jede Ebene verwendet nur Parameter, die in vorherigen Ebenen definiert wurden.

\subsection{Kritische Anmerkungen}
\label{subsec:critical_notes}

\textbf{(*) Anmerkung zur Feinstrukturkonstante:}

Die Feinstrukturkonstante hat im T0-System eine Doppelfunktion:
\begin{itemize}
	\item $\alpha_{EM} = 1$ ist eine \textbf{Einheitenkonvention} (wie $c = 1$)
	\item $\varepsilon_T = \xi \cdot f_{geom}$ ist die \textbf{physikalische EM-Kopplung}
\end{itemize}

\textbf{Einheitensystem:}
Alle T0-Werte gelten in natürlichen Einheiten mit $\hbar = c = 1$. Für experimentelle Vergleiche ist eine Transformation in SI-Einheiten erforderlich.

\section{Kosmologische Parameter: Standardkosmologie ($\Lambda$CDM) vs T0-System}
\label{sec:cosmic_t0_mapping}

\subsection{Fundamentaler Paradigmenwechsel}
\label{subsec:paradigm_shift}

\begin{tcolorbox}[colback=red!5!white,colframe=red!75!black,title=Warnung: Fundamentale Unterschiede]
	Das T0-System postuliert ein \textbf{statisches, ewiges Universum} ohne Urknall, während die Standardkosmologie auf einem \textbf{expandierenden Universum} mit Urknall basiert. Die Parameter sind daher oft nicht direkt vergleichbar, sondern repräsentieren unterschiedliche physikalische Konzepte.
\end{tcolorbox}

\bigskip
\clearpage

\end{document}