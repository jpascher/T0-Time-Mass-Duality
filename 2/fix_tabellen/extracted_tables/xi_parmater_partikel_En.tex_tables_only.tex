\documentclass[12pt,a4paper]{book}
% ==============================================================================
% T0 Theory: Shared ENGLISH Preamble – Optimized for eBook/Book
% Version: 2.0 – Final 2026 (LuaLaTeX only) – ENGLISH corrected
% Author: Johann Pascher
% Date: January 2026
% ==============================================================================
%
% IMPORTANT: Compile EXCLUSIVELY with LuaLaTeX!
% In TeXstudio: Options → Configure TeXstudio → Build → Default Compiler → LuaLaTeX
%
% Required Fonts (install once):
% - Inter: https://fonts.google.com/specimen/Inter
% - JetBrains Mono: https://www.jetbrains.com/lp/mono/
% - Libertinus Math: https://github.com/libertinus-fonts/libertinus
% ==============================================================================

% === CHAPTER 1: BASIC PACKAGES (must come FIRST) ===
\RequirePackage{fontspec}
\RequirePackage{unicode-math}
\usepackage{chngcntr}
\setcounter{secnumdepth}{1}  % Nur Sections nummerieren (nicht subsections)
\setcounter{tocdepth}{1}     % Nur Sections im TOC (nicht subsections)
\makeatletter
\@ifundefined{c@chapter}{}{\counterwithout{section}{chapter}}  % Falls Kapitel existieren
\makeatother
\counterwithout{subsection}{section}  % Löse Verknüpfung
% === CHAPTER 2: LANGUAGE (ENGLISH) ===
\usepackage[english]{babel}
\usepackage{microtype}                    % IMPORTANT for better hyphenation!

% Typography settings for better line breaking
\frenchspacing                     % Correct English spacing after punctuation
\emergencystretch=3em              % Allows more stretch for difficult lines
\tolerance=2500                    % Higher tolerance for line breaks
\hbadness=10000                    % Suppresses "underfull hbox" warnings
\hfuzz=2pt                         % Allows minimal overfull
\pretolerance=150                  % Better word breaking

% Prevent bad page breaks
\clubpenalty=10000           % No "orphans"
\widowpenalty=10000          % No "widows"
\displaywidowpenalty=10000   % Also with equations
\brokenpenalty=10000         % No broken words across pages

% Explicit hyphenation for long technical words
\hyphenation{Fun-da-men-tal Frac-tal-Ge-o-met-ric Field The-o-ry Meth-od-o-log-i-cal}
\hyphenation{Re-vi-sion-ism Quan-ti-za-tion U-ni-fi-ca-tion Ef-fec-tive}
\hyphenation{Re-nor-mal-iz-a-bil-i-ty Sin-gu-lar-i-ties Con-cil-i-a-tion}
\hyphenation{E-mer-gence Phe-nom-e-no-log-i-cal Doc-u-men-ta-tion A-nal-y-sis}
\hyphenation{Grav-i-ta-tion Quan-tum Me-chan-ics Dog-ma-tism Con-se-quent}
\hyphenation{Par-al-lel-ism Im-ple-men-ta-tion Per-tur-ba-tions}
\hyphenation{Geo-met-ric Ar-ti-fact In-com-pat-i-bil-i-ty Con-struc-tive}
\hyphenation{Frac-tal Di-men-sion-less In-ves-ti-ga-tion De-scrip-tion}
\hyphenation{In-ter-pre-ta-tion Phe-nom-e-no-log-i-cal Math-e-mat-i-cal}
\hyphenation{Phi-lo-soph-i-cal Le-git-i-ma-tion Ap-pli-ca-tion Der-i-va-tion}
\hyphenation{U-ni-fi-ca-tion As-sump-tion Con-cep-tion Ex-pec-ta-tion}
\hyphenation{Sym-me-try-ex-ten-sion O-ver-all-pic-ture Chal-lenge}
\hyphenation{In-ter-ac-tion Ma-te-ri-al Ap-proach Per-spec-tive Pro-ce-dure}

% === CHAPTER 3: FONTS (with proper ligatures) ===
\setmainfont{Inter}[
Scale=1.02,
UprightFont=*-Regular,
BoldFont=*-Bold,
ItalicFont=*-Italic,
BoldItalicFont=*-BoldItalic,
Ligatures=TeX,           % IMPORTANT for proper typography
Language=English         % Explicit language support
]
\setsansfont{Inter}[
Scale=MatchLowercase,
Ligatures=TeX,
Language=English
]
\setmonofont{JetBrains Mono}[
Scale=0.95,
Language=English
]

% Math Font (simple & stable) – MUST come AFTER language definition
% IMPORTANT: Libertinus Math for correct \underbrace display!
\setmathfont{Libertinus Math}[Scale=1.0]

% === CHAPTER 4: MATHEMATICS PACKAGES (in STRICT order!) ===
% IMPORTANT: mathtools must come BEFORE unicode-math for some commands!
\usepackage{mathtools}           % FIRST mathtools!

% Then the rest
\usepackage{amsmath, amsfonts, amsthm}

% SIUNITX MUST be loaded BEFORE physics!
\usepackage{siunitx}
\sisetup{
	locale=US,                    % ENGLISH settings for SI units!
	group-separator={,},          % Thousands separator comma
	output-decimal-marker={.},    % Decimal separator point
	per-mode=symbol,
	separate-uncertainty=true
}

% Custom SI units used in narrative and books
\DeclareSIUnit\gigalightyear{Gly}
\DeclareSIUnit\mev{MeV}

% physics – MUST be loaded AFTER siunitx and mathtools
\usepackage{physics}

% === CHAPTER 5: ADDITIONS from pdflatex best practices ===
\usepackage{colortbl}        % Colored tables (ESSENTIAL!)
\usepackage{placeins}        % Float control: \FloatBarrier
\usepackage{subcaption}      % Subfigures
\usepackage{xurl}            % Better URL line breaking
% Hyphenation for URLs in bibliography
\def\UrlBreaks{\do\/\do-}

% === CHAPTER 6: PAGE LAYOUT
% =============================================================================
% SECTION 2: Page Geometry – 6" × 9" Buchformat
% =============================================================================
\usepackage[paperwidth=6in, paperheight=9in,
top=0.9in,
bottom=1.1in,
inner=0.9in,            % Größerer Innenrand für Bindung
outer=0.6in,            % Kleinerer Außenrand → mehr Text pro Seite
bindingoffset=0.5in,    % Puffer für Bindung (Steg)
twoside]{geometry}
\setlength{\headheight}{15pt}
%\usepackage[paperwidth=8.25in, paperheight=11in,
%top=1.0in,
%bottom=1.0in,
%left=1.0in,
%right=1.0in,
%twoside=false
% === CHAPTER 7: GRAPHICS AND TABLES ===
\usepackage{graphicx}
\usepackage[table,xcdraw]{xcolor}
% T0 brand colors
\definecolor{gold}{RGB}{255,215,0}
\definecolor{blue}{rgb}{0,0,1}
\definecolor{boxgray}{RGB}{240,240,240}
\definecolor{deepblue}{RGB}{0,0,127}
\definecolor{deepgreen}{RGB}{0,127,0}
\definecolor{deepred}{RGB}{191,0,0}
\definecolor{t0blue}{RGB}{33,150,243}
\definecolor{t0green}{RGB}{76,175,80}
\definecolor{t0orange}{RGB}{255,152,0}
\definecolor{t0purple}{RGB}{156,39,176}
\definecolor{t0red}{RGB}{244,67,54}
\definecolor{t0yellow}{RGB}{255,204,0}
\usepackage{tikz}
\usetikzlibrary{arrows.meta,positioning,shapes.geometric,decorations.pathmorphing,patterns,shapes.arrows,intersections}
\usepackage{pgfplots}
\pgfplotsset{compat=1.18}
\usepackage{quantikz}
\usepackage[most]{tcolorbox}
\tcbuselibrary{breakable}

% === WICHTIG: Algorithm-Konflikt umgehen ===
% Option: algorithmic mit GROSSBUCHSTABEN
% Gemeinsame Box für Experimente
\newtcolorbox{experimentbox}[1][]{
	colback=green!5!white,
	colframe=t0green!80!black,
	fonttitle=\bfseries,
	title={{#1}},
	breakable
}

% Abstract-Fallback
\ifdefined\abstract\else
\newenvironment{abstract}{\section*{\abstractname}\itshape\small\par\bigskip}{\bigskip}
\fi

% === MAKROS SICHER NEU DEFINIEREN / ÜBERSCHREIBEN ===
% Definiere Makros OHNE doppelte Subskripte
\newcommand{\phipar}{\phi_{\mathrm{par}}}
%\newcommand{\xipar}{\xi_{\mathrm{par}}}
\newcommand{\Qphipar}{Q_{\phi_{\mathrm{par}}}}
\newcommand{\rphipar}{r_{\phi_{\mathrm{par}}}}
\newcommand{\logphipar}{\log_{\phi_{\mathrm{par}}}}
\newcommand{\CHSH}{\text{CHSH}}
\usepackage{booktabs}
\usepackage{array}
\usepackage{longtable}
\usepackage{float}
\usepackage{adjustbox}
\usepackage{rotating}
\usepackage{tabularx}
\usepackage{makecell}
\usepackage{multirow}

% === CHAPTER 8: DOCUMENT FORMATTING ===
\usepackage{fancyhdr}
\renewcommand{\headrulewidth}{0.4pt}
\renewcommand{\footrulewidth}{0.4pt}
\usepackage{tocloft}

\usepackage{enumitem}
\setlist[itemize]{leftmargin=*, topsep=2pt, partopsep=0pt, parsep=2pt, itemsep=2pt}
\setlist[enumerate]{leftmargin=*, topsep=2pt, partopsep=0pt, parsep=2pt, itemsep=2pt}
\usepackage{setspace}
\usepackage{ragged2e}
\usepackage{multicol}

% === CHAPTER 9: CODE AND ALGORITHMS ===
\usepackage{algorithm}
\usepackage{algorithmic}
\usepackage{listings}
\lstset{
	basicstyle=\ttfamily\footnotesize,
	breaklines=true,
	breakatwhitespace=true,
	columns=flexible,
	keepspaces=true,
	showstringspaces=false,
	frame=single,
	xleftmargin=0pt,
	xrightmargin=0pt,
	literate=              % For special characters in code listings
	{ä}{{\"a}}1 {ö}{{\"o}}1 {ü}{{\"u}}1 {ß}{{\ss}}1
	{Ä}{{\"A}}1 {Ö}{{\"O}}1 {Ü}{{\"U}}1
}
\usepackage{mdframed}

% === CHAPTER 10: ADDITIONAL PACKAGES ===
\usepackage{pdflscape}
\usepackage{braket}
\usepackage{cancel}
\usepackage{caption}
\captionsetup{format=plain, labelfont=bf, justification=centering}
\usepackage{csquotes}
\usepackage{gensymb}
\usepackage{textcomp}
\usepackage{textgreek}
\usepackage{upgreek}
\usepackage{url}
\usepackage{slashed}
\usepackage{bm}

% === CHAPTER 11: HYPERREF (must come SECOND TO LAST!) ===
\usepackage{hyperref}
\hypersetup{
	colorlinks=true,
	linkcolor=black,
	citecolor=black,
	urlcolor=black,
	breaklinks=true,           % IMPORTANT for special characters in URLs!
	bookmarksnumbered=true,
	unicode=true,
	pdfencoding=auto,
	pdflang=en,                % Set PDF language to English
	pdfsubject={T0 Theory - Fundamental Fractal-Geometric Field Theory}
}

% Fix for unicode-math symbols in PDF bookmarks
\pdfstringdefDisableCommands{%
	\def\xi{xi}%
	\def\alpha{alpha}%
	\def\beta{beta}%
	\def\gamma{gamma}%
	\def\delta{delta}%
	\def\Delta{Delta}%
	\def\epsilon{epsilon}%
	\def\varepsilon{epsilon}%
	\def\theta{theta}%
	\def\kappa{kappa}%
	\def\lambda{lambda}%
	\def\mu{mu}%
	\def\nu{nu}%
	\def\pi{pi}%
	\def\rho{rho}%
	\def\sigma{sigma}%
	\def\tau{tau}%
	\def\phi{phi}%
	\def\chi{chi}%
	\def\psi{psi}%
	\def\omega{omega}%
	\def\Omega{Omega}%
	\def\Lambda{Lambda}%
	\def\times{x}%
	\def\cdot{*}%
	\def\pm{+/-}%
	\def\approx{~}%
	\def\sim{~}%
	\def\equiv{=}%
	\def\ell{l}%
	\def\hbar{h}%
	\def\rightarrow{->}%
	\def\leftarrow{<-}%
	\def\Rightarrow{=>}%
	\def\Leftarrow{<=}%
	\def\propto{~}%
	\def\mitxi{xi}%
	\def\mitalpha{alpha}%
	\def\mitbeta{beta}%
	\def\mitgamma{gamma}%
	\def\mitdelta{delta}%
	\def\mitDelta{Delta}%
	\def\mitepsilon{epsilon}%
	\def\mitvarepsilon{epsilon}%
	\def\mittheta{theta}%
	\def\mitkappa{kappa}%
	\def\mitlambda{lambda}%
	\def\mitLambda{Lambda}%
	\def\mitmu{mu}%
	\def\mitnu{nu}%
	\def\mitpi{pi}%
	\def\mitrho{rho}%
	\def\mitsigma{sigma}%
	\def\mittau{tau}%
	\def\mitphi{phi}%
	\def\mitchi{chi}%
	\def\mitpsi{psi}%
	\def\mitomega{omega}%
	\def\mitOmega{Omega}%
}

% === CHAPTER 12: BOOKMARK (must come AFTER hyperref!) ===
\usepackage{bookmark}

% === CHAPTER 13: CLEVEREF (ENGLISH LABELS) ===
\usepackage[english]{cleveref}
\crefname{equation}{Equation}{Equations}
\crefname{figure}{Figure}{Figures}
\crefname{table}{Table}{Tables}
\crefname{section}{Section}{Sections}
\crefname{chapter}{Chapter}{Chapters}
\crefname{theorem}{Theorem}{Theorems}
\crefname{lemma}{Lemma}{Lemmas}
\crefname{definition}{Definition}{Definitions}
\crefname{example}{Example}{Examples}
\crefname{remark}{Remark}{Remarks}

% === CUSTOM ENVIRONMENTS ===
% Alternative interpretation environment
\newenvironment{alternative}{%
	\begin{mdframed}[linecolor=black!30,linewidth=1pt,roundcorner=4pt,backgroundcolor=black!5]%
	}{%
	\end{mdframed}%
}

% Photon/particle environment
\newenvironment{photon}{%
	\begin{mdframed}[linecolor=blue!30,linewidth=1pt,roundcorner=4pt,backgroundcolor=blue!5]%
	}{%
	\end{mdframed}%
}

% Koide formula box environment
\newenvironment{koidebox}{%
	\begin{mdframed}[linecolor=green!30,linewidth=1pt,roundcorner=4pt,backgroundcolor=green!5]%
	}{%
	\end{mdframed}%
}

% Erkenntnis/insight environment
\newenvironment{erkenntnis}{%
	\begin{mdframed}[linecolor=orange!30,linewidth=1pt,roundcorner=4pt,backgroundcolor=orange!5]%
	}{%
	\end{mdframed}%
}

% Beziehung/relationship environment
\newenvironment{beziehung}{%
	\begin{mdframed}[linecolor=purple!30,linewidth=1pt,roundcorner=4pt,backgroundcolor=purple!5]%
	}{%
	\end{mdframed}%
}

% Derivation environment
\newenvironment{derivation}{%
	\begin{mdframed}[linecolor=teal!30,linewidth=1pt,roundcorner=4pt,backgroundcolor=teal!5]%
	}{%
	\end{mdframed}%
}

% Abhandlung/treatise environment
\newenvironment{abhandlung}{%
	\begin{mdframed}[linecolor=brown!30,linewidth=1pt,roundcorner=4pt,backgroundcolor=brown!5]%
	}{%
	\end{mdframed}%
}

% Anwendung/application environment
\newenvironment{anwendung}{%
	\begin{mdframed}[linecolor=cyan!30,linewidth=1pt,roundcorner=4pt,backgroundcolor=cyan!5]%
	}{%
	\end{mdframed}%
}

% Additional common environments
\newenvironment{konsequenz}{%
	\begin{mdframed}[linecolor=red!30,linewidth=1pt,roundcorner=4pt,backgroundcolor=red!5]%
	}{%
	\end{mdframed}%
}

\newenvironment{schlussfolgerung}{%
	\begin{mdframed}[linecolor=gray!30,linewidth=1pt,roundcorner=4pt,backgroundcolor=gray!5]%
	}{%
	\end{mdframed}%
}

\newenvironment{result}{%
	\begin{mdframed}[linecolor=violet!30,linewidth=1pt,roundcorner=4pt,backgroundcolor=violet!5]%
	}{%
	\end{mdframed}%
}

% Formula environment
\newenvironment{formula}{%
	\begin{mdframed}[linecolor=yellow!30,linewidth=1pt,roundcorner=4pt,backgroundcolor=yellow!5]%
	}{%
	\end{mdframed}%
}

% Revolutionaer/revolutionary environment
\newenvironment{revolutionaer}{%
	\begin{mdframed}[linecolor=red!50,linewidth=2pt,roundcorner=4pt,backgroundcolor=red!10]%
	}{%
	\end{mdframed}%
}

% Formel environment (German version of formula)
\newenvironment{formel}{%
	\begin{mdframed}[linecolor=yellow!30,linewidth=1pt,roundcorner=4pt,backgroundcolor=yellow!5]%
	}{%
	\end{mdframed}%
}

% Prinzip/principle environment
\newenvironment{prinzip}{%
	\begin{mdframed}[linecolor=blue!50,linewidth=2pt,roundcorner=4pt,backgroundcolor=blue!10]%
	}{%
	\end{mdframed}%
}

% Experimentell/experimental environment
\newenvironment{experimentell}{%
	\begin{mdframed}[linecolor=magenta!30,linewidth=1pt,roundcorner=4pt,backgroundcolor=magenta!5]%
	}{%
	\end{mdframed}%
}

% Neutrino environment
\newenvironment{neutrino}{%
	\begin{mdframed}[linecolor=cyan!40,linewidth=1pt,roundcorner=4pt,backgroundcolor=cyan!8]%
	}{%
	\end{mdframed}%
}

% Additional missing environments
\newenvironment{schluessel}{%
	\begin{mdframed}[linecolor=yellow!50,linewidth=1pt,roundcorner=4pt,backgroundcolor=yellow!10]%
	}{%
	\end{mdframed}%
}

\newenvironment{summary}{%
	\begin{mdframed}[linecolor=gray!40,linewidth=1pt,roundcorner=4pt,backgroundcolor=gray!8]%
	}{%
	\end{mdframed}%
}

\newenvironment{category}{%
	\begin{mdframed}[linecolor=pink!40,linewidth=1pt,roundcorner=4pt,backgroundcolor=pink!8]%
	}{%
	\end{mdframed}%
}

\newenvironment{sibox}{%
	\begin{mdframed}[linecolor=lime!40,linewidth=1pt,roundcorner=4pt,backgroundcolor=lime!8]%
	}{%
	\end{mdframed}%
}

% More missing environments
\newenvironment{documentbox}{%
	\begin{mdframed}[linecolor=teal!40,linewidth=1pt,roundcorner=4pt,backgroundcolor=teal!8]%
	}{%
	\end{mdframed}%
}

\newenvironment{t0box}{%
	\begin{mdframed}[linecolor=violet!40,linewidth=1pt,roundcorner=4pt,backgroundcolor=violet!8]%
	}{%
	\end{mdframed}%
}

\newenvironment{wichtig}{%
	\begin{mdframed}[linecolor=red!50,linewidth=2pt,roundcorner=4pt,backgroundcolor=red!10]%
	\textbf{Important:} 
	}{%
	\end{mdframed}%
}

\newenvironment{smbox}{%
	\begin{mdframed}[linecolor=orange!40,linewidth=1pt,roundcorner=4pt,backgroundcolor=orange!8]%
	}{%
	\end{mdframed}%
}

\newenvironment{pvbox}{%
	\begin{mdframed}[linecolor=purple!40,linewidth=1pt,roundcorner=4pt,backgroundcolor=purple!8]%
	}{%
	\end{mdframed}%
}

\newenvironment{numerisch}{%
	\begin{mdframed}[linecolor=blue!40,linewidth=1pt,roundcorner=4pt,backgroundcolor=blue!8]%
	}{%
	\end{mdframed}%
}

% More missing environments
\newenvironment{relation}{%
	\begin{mdframed}[linecolor=green!40,linewidth=1pt,roundcorner=4pt,backgroundcolor=green!8]%
	}{%
	\end{mdframed}%
}

\newenvironment{beweis}{%
	\begin{mdframed}[linecolor=brown!40,linewidth=1pt,roundcorner=4pt,backgroundcolor=brown!8]%
	\textbf{Proof:} 
	}{%
	\end{mdframed}%
}

\newenvironment{revolution}{%
	\begin{mdframed}[linecolor=red!60,linewidth=2pt,roundcorner=4pt,backgroundcolor=red!12]%
	}{%
	\end{mdframed}%
}

\newenvironment{key}{%
	\begin{mdframed}[linecolor=yellow!50,linewidth=1pt,roundcorner=4pt,backgroundcolor=yellow!10]%
	}{%
	\end{mdframed}%
}

\newenvironment{newperspective}{%
	\begin{mdframed}[linecolor=cyan!50,linewidth=1pt,roundcorner=4pt,backgroundcolor=cyan!10]%
	}{%
	\end{mdframed}%
}

\newenvironment{literatur}{%
	\begin{mdframed}[linecolor=gray!50,linewidth=1pt,roundcorner=4pt,backgroundcolor=gray!10]%
	}{%
	\end{mdframed}%
}

\newenvironment{folgerung}{%
	\begin{mdframed}[linecolor=teal!50,linewidth=1pt,roundcorner=4pt,backgroundcolor=teal!10]%
	}{%
	\end{mdframed}%
}

\newenvironment{principle}{%
	\begin{mdframed}[linecolor=blue!60,linewidth=2pt,roundcorner=4pt,backgroundcolor=blue!12]%
	}{%
	\end{mdframed}%
}

% Additional common environments
% ==============================================================================
% FROM HERE: YOUR DEFINITIONS (unchanged)
% ==============================================================================

\setcounter{tocdepth}{3}

% === CITATION COMMANDS ===
\providecommand{\citep}[1]{\cite{#1}}
\providecommand{\citet}[1]{\cite{#1}}

% === COLORS ===
\definecolor{gold}{RGB}{255,215,0}
\definecolor{blue}{rgb}{0,0,1}
\definecolor{boxgray}{RGB}{240,240,240}
\definecolor{deepblue}{RGB}{0,0,127}
\definecolor{deepgreen}{RGB}{0,127,0}
\definecolor{deepred}{RGB}{191,0,0}
\definecolor{t0blue}{RGB}{33,150,243}
\definecolor{t0green}{RGB}{76,175,80}
\definecolor{t0orange}{RGB}{255,152,0}
\definecolor{t0purple}{RGB}{156,39,176}
\definecolor{t0red}{RGB}{244,67,54}
\definecolor{t0yellow}{RGB}{255,204,0}

% === COLUMN TYPES ===
\newcolumntype{L}[1]{>{\raggedright\arraybackslash}p{#1}}
\newcolumntype{C}[1]{>{\centering\arraybackslash}p{#1}}
\newcolumntype{R}[1]{>{\raggedleft\arraybackslash}p{#1}}

% === HYPERREF SETTINGS (updated) ===
\hypersetup{
	colorlinks=true,
	linkcolor=t0blue,
	citecolor=t0blue,
	urlcolor=t0blue,
	breaklinks=true,
	bookmarksnumbered=true,
	pdfstartview=FitH,
	pdfencoding=auto,
	pdfdisplaydoctitle=true
}

% === ENGLISH THEOREM ENVIRONMENTS ===
\theoremstyle{plain}
\newtheorem{theorem}{Theorem}[section]
\newtheorem{lemma}[theorem]{Lemma}
\newtheorem{proposition}[theorem]{Proposition}
\newtheorem{corollary}[theorem]{Corollary}

\theoremstyle{definition}
\newtheorem{definition}[theorem]{Definition}
\newtheorem{example}[theorem]{Example}
\newtheorem{insight}[theorem]{Insight}
\newtheorem{discovery}[theorem]{Discovery}

\theoremstyle{remark}
\newtheorem{remark}[theorem]{Remark}
\newtheorem{axiom}{Axiom}
%\newtheorem{principle}{Principle}  % Commented out to avoid conflicts with document-specific definitions
%\newtheorem{warning}[theorem]{Warning}

% === T0-SPECIFIC COMMANDS ===
% (Here follow all your \newcommand and \providecommand definitions)
% These remain UNCHANGED as in your original preamble
% ==============================================================================
% SECTION 14: T0-Specific Commands
% ==============================================================================

% --- Core T0 Fields ---
\newcommand{\Tfield}{T(x,t)}
\providecommand{\Tfieldt}{T(\vec{x},t)}
\newcommand{\Efield}{E(x,t)}
\newcommand{\mfield}{m(x,t)}
\providecommand{\vecx}{\vec{x}}

% --- Lagrangian ---
\newcommand{\Lag}{\mathcal{L}}
\newcommand{\calL}{\mathcal{L}}

% --- Greek Letters and Constants ---
\newcommand{\alphaem}{\alpha}
\newcommand{\betaT}{\beta_T}
\newcommand{\xiT}{\xi}
\newcommand{\xipar}{\xi}

% --- Energy and Planck Units ---
\newcommand{\Ezero}{E_0}
\newcommand{\E}{E}
\newcommand{\EPlanck}{E_{\text{Pl}}}
\newcommand{\Mpl}{M_{\text{Pl}}}
\newcommand{\mP}{m_{\text{P}}}
\newcommand{\lP}{\ell_{\text{P}}}
\newcommand{\tP}{t_{\text{P}}}
\newcommand{\LPlanck}{\ell_{\text{Pl}}}
\newcommand{\TPlanck}{t_{\text{Pl}}}

% --- Coupling Constants ---
\newcommand{\Gnat}{G_{\text{nat}}}
\newcommand{\alphaEM}{\alpha_{\text{EM}}}
\newcommand{\alphaSI}{\alpha_{\text{SI}}}
\newcommand{\Hubble}{H_0}
\newcommand{\LCDM}{\Lambda\text{CDM}}
\newcommand{\natunits}{(nat. units)}

% --- T0 Model Parameters ---
\newcommand{\xigeom}{\xi_{\mathrm{geom}}}
\newcommand{\rzero}{r_{0}}
\newcommand{\xirat}{\xi_{\mathrm{rat}}}
\newcommand{\tzero}{t_{0}}
\newcommand{\Lambdat}{\Lambda_{\mathrm{t}}}
\newcommand{\EP}{E_{\text{P}}}
\newcommand{\Emu}{E_{\mu}}
\newcommand{\Ee}{E_{e}}
\newcommand{\Etau}{E_{\tau}}
\newcommand{\alphafine}{\alpha_{\mathrm{fine}}}
\newcommand{\alphal}{\alpha_{\ell}}
\newcommand{\Lzero}{\ell_{0}}
\newcommand{\Lp}{\ell_{\mathrm{P}}}

% --- Additional T0 Commands ---
\newcommand{\Kfrak}{K_{\text{frak}}}
\newcommand{\Dfrak}{D_{\text{frak}}}
\newcommand{\betapar}{\ensuremath{\beta_T}}
\newcommand{\alphapar}{\alpha}
\newcommand{\deltafield}{\delta \phi}
\newcommand{\deltam}{\delta m}
\newcommand{\deltaE}{\delta E}
\newcommand{\Exi}{E_{\xi}}
\newcommand{\Lxi}{\ell_{\xi}}
\newcommand{\rhoCMB}{\rho_{\text{CMB}}}
\newcommand{\rhoCasimir}{\rho_{\text{Casimir}}}
\newcommand{\Leff}{L_{\text{eff}}}
\newcommand{\CQCD}{C_{\mathrm{QCD}}}
\newcommand{\Kspec}{K_{\mathrm{spec}}}
\newcommand{\Tzero}{\ensuremath{T_0}}
\newcommand{\Eabs}{E_{\text{abs}}}
\newcommand{\taupar}{\tau}

% --- Provided Commands ---
\providecommand{\xiconst}{\xi_{\text{const}}}
\providecommand{\DhiggsT}{D_{\text{Higgs-T}}}
\providecommand{\rhoE}{\rho_{E}}
\providecommand{\Echar}{E_{\text{char}}}
\providecommand{\kfrac}{k_{\text{frac}}}
\providecommand{\alphaEMSI}{\alpha_{\text{EM,SI}}}
\providecommand{\alphaEMnat}{\alpha_{\text{EM,nat}}}
\providecommand{\betaTSI}{\beta_{T,\text{SI}}}
\providecommand{\betaTnat}{\beta_{T,\text{nat}}}
\providecommand{\Gsi}{G_{\text{SI}}}
\providecommand{\xiparSI}{\xi_{\text{SI}}}
\providecommand{\xiparnat}{\xi_{\text{nat}}}
\providecommand{\meff}{m_{\text{eff}}}
\providecommand{\Tzerot}{T_{0}(t)}
\providecommand{\mzerot}{m_{0}(t)}
\providecommand{\Ezeroabs}{E_{0,\text{abs}}}
\providecommand{\Epar}{E_{\text{par}}}
\providecommand{\Lnat}{\ell_{\text{nat}}}
\providecommand{\Tnat}{T_{\text{nat}}}
\providecommand{\xifrak}{\xi_{\text{frac}}}
\providecommand{\Tfrak}{T_{\text{frac}}}
\providecommand{\mfrak}{m_{\text{frac}}}
\providecommand{\Dfrac}{D_{\text{frac}}}
\providecommand{\EphotSI}{E_{\gamma,\text{SI}}}
\providecommand{\EphotNat}{E_{\gamma,\text{nat}}}
\providecommand{\Eabsint}{E_{\text{abs,int}}}
\providecommand{\mphoton}{m_{\gamma}}
\providecommand{\Evis}{E_{\text{vis}}}
\providecommand{\Cto}{C_{T0}}
\providecommand{\mytimes}{\times}
\providecommand{\lambdah}{\lambda_h}
\providecommand{\checkmarkx}{\checkmark}
\providecommand{\Enorm}{E_{\text{norm}}}
\providecommand{\Tobs}{T_{\text{obs}}}
\providecommand{\mobs}{m_{\text{obs}}}
\providecommand{\Eobs}{E_{\text{obs}}}
\providecommand{\Lobs}{\ell_{\text{obs}}}
\providecommand{\xobs}{\xi_{\text{obs}}}
\providecommand{\calE}{\mathcal{E}}
\providecommand{\calT}{\mathcal{T}}
\providecommand{\calM}{\mathcal{M}}
\providecommand{\alphag}{\alpha_g}
\providecommand{\Tmax}{T_{\text{max}}}
\providecommand{\mmin}{m_{\text{min}}}
\providecommand{\Lmax}{\ell_{\text{max}}}
\providecommand{\Emin}{E_{\text{min}}}
\providecommand{\Geff}{G_{\text{eff}}}
\providecommand{\rhoeff}{\rho_{\text{eff}}}
\providecommand{\xieff}{\xi_{\text{eff}}}
\providecommand{\Teff}{T_{\text{eff}}}
\providecommand{\hPlanck}{h}
\providecommand{\kB}{k_B}
\providecommand{\muB}{\mu_B}
\providecommand{\lambdaC}{\lambda_C}
\providecommand{\omegaP}{\omega_P}
\providecommand{\rhoP}{\rho_P}
\providecommand{\Tref}{T_{\text{ref}}}
\providecommand{\Eref}{E_{\text{ref}}}
\providecommand{\mref}{m_{\text{ref}}}
\providecommand{\Lref}{\ell_{\text{ref}}}
\providecommand{\xikonst}{\xi_0}
\providecommand{\Phiphoton}{\Phi_{\gamma}}
\providecommand{\etavis}{\eta_{\text{vis}}}
\providecommand{\pichar}{\pi}
\providecommand{\primrel}{\mathcal{P}_{\text{rel}}}
\providecommand{\warningx}{\textcolor{orange}{\textbf{!}}}
\providecommand{\phiT}{\phi_T}
\providecommand{\Lorentz}{\Lambda}
\providecommand{\Cconv}{C_{\text{conv}}}
\providecommand{\Df}{\Delta f}
\providecommand{\lambdazero}{\lambda_0}
\providecommand{\myapprox}{\approx}
\providecommand{\checked}{\checkmark}
\providecommand{\alphaWSI}{\alpha_W^{\text{SI}}}
\providecommand{\alphaWnat}{\alpha_W^{\text{nat}}}
\providecommand{\vect}[1]{\vec{#1}}
\providecommand{\Rzero}{R_0}
\providecommand{\Riem}{\mathcal{R}}
\providecommand{\nuzero}{\nu_0}
\providecommand{\mypi}{\pi}

% =============================================================================
% TCOLORBOX STYLES AND ENVIRONMENTS (English titles)
% =============================================================================
\tcbset{
	keyresult/.style={
		colback=blue!5!white,
		colframe=blue!75!black,
		title=Key Result,
		fonttitle=\bfseries
	},
	foundation/.style={
		colback=green!5!white,
		colframe=green!75!black,
		title=Foundation,
		fonttitle=\bfseries
	},
	alternative/.style={
		colback=orange!5!white,
		colframe=orange!75!black,
		title=Alternative,
		fonttitle=\bfseries
	},
	warningbox/.style={
		colback=red!5!white,
		colframe=red!75!black,
		title=Warning,
		fonttitle=\bfseries
	}
}

% (Here follow all your tcolorbox definitions with English titles)
\newtcolorbox{keyresultbox}[1][]{colback=blue!5!white,colframe=blue!75!black,fonttitle=\bfseries,title={#1},breakable}
\newtcolorbox{keyresult}[1][Key Result]{colback=blue!5!white,colframe=blue!75!black,fonttitle=\bfseries,title={#1},breakable}
\newtcolorbox{foundationbox}[1][]{colback=green!5!white,colframe=green!75!black,fonttitle=\bfseries,title={#1},breakable}
\newtcolorbox{foundation}[1][Foundation]{colback=green!5!white,colframe=green!75!black,fonttitle=\bfseries,title={#1},breakable}
\newtcolorbox{alternativebox}[1][]{colback=orange!5!white,colframe=orange!75!black,fonttitle=\bfseries,title={#1},breakable}
\newtcolorbox{warningboxenv}[1][Warning]{colback=red!5!white,colframe=red!75!black,fonttitle=\bfseries,title={#1},breakable}

\newtcolorbox{fundamental}[1][]{
	colback=boxgray,
	colframe=t0blue,
	fonttitle=\bfseries,
	title=#1,
	sharp corners,
	boxrule=2pt
}

\newtcolorbox{insightBox}[1][Insight]{colback=blue!5,colframe=t0blue,title={#1},fonttitle=\bfseries,breakable}
\newtcolorbox{discoveryBox}[1][Discovery]{colback=green!5,colframe=t0green,title={#1},fonttitle=\bfseries,breakable}
\newtcolorbox{revelation}[1][Revelation]{colback=red!5,colframe=t0red,title={#1},fonttitle=\bfseries,breakable}
\newtcolorbox{keypoint}[1][Key Point]{colback=blue!5,colframe=t0blue,title={#1},fonttitle=\bfseries,breakable}
\newtcolorbox{evidence}[1][Evidence]{colback=green!5,colframe=t0green,title={#1},fonttitle=\bfseries,breakable}
\newtcolorbox{conclusionBox}[1][Conclusion]{colback=gray!5,colframe=gray,title={#1},fonttitle=\bfseries,breakable}
\newtcolorbox{significance}[1][Significance]{colback=yellow!5,colframe=orange,title={#1},fonttitle=\bfseries,breakable}
\newtcolorbox{philosophical}[1][Philosophical]{colback=purple!5,colframe=purple,title={#1},fonttitle=\bfseries,breakable}
\newtcolorbox{implicationBox}[1][Implication]{colback=cyan!5,colframe=cyan,title={#1},fonttitle=\bfseries,breakable}
\newtcolorbox{perspectiveBox}[1][Perspective]{colback=blue!5,colframe=t0blue,title={#1},fonttitle=\bfseries,breakable}
\newtcolorbox{revolutionary}[1][Revolutionary]{colback=red!5,colframe=t0red,title={#1},fonttitle=\bfseries,breakable}

\newtcolorbox{technical}[1][Technical]{colback=gray!5,colframe=gray!75!black,title={#1},fonttitle=\bfseries,breakable}
\newtcolorbox{technicalBox}[1][Technical]{colback=gray!5,colframe=gray!75!black,title={#1},fonttitle=\bfseries,breakable}
\newtcolorbox{notationBox}[1][Notation]{colback=yellow!5,colframe=yellow!75!black,title={#1},fonttitle=\bfseries,breakable}
\newtcolorbox{verification}[1][Verification]{colback=orange!5!white,colframe=orange!75!black,fonttitle=\bfseries,title=#1}
\newtcolorbox{explanationBox}[1][Explanation]{colback=purple!5!white,colframe=purple!75!black,fonttitle=\bfseries,title=#1}
\newtcolorbox{interpretationBox}[1][Interpretation]{colback=cyan!5!white,colframe=cyan!75!black,fonttitle=\bfseries,title=#1}
\newtcolorbox{explanation}[1][Explanation]{colback=purple!5!white,colframe=purple!75!black,fonttitle=\bfseries,title=#1,breakable}
\newtcolorbox{interpretation}[1][Interpretation]{colback=cyan!5!white,colframe=cyan!75!black,fonttitle=\bfseries,title=#1,breakable}
\newtcolorbox{proof_step}[1][Proof Step]{colback=gray!5!white,colframe=gray!75!black,fonttitle=\bfseries,title=#1,breakable}
\newtcolorbox{experimental}[1][Experimental]{colback=teal!5!white,colframe=teal!75!black,fonttitle=\bfseries,title=#1,breakable}

\newtcolorbox{important}[1][Important]{colback=red!5!white,colframe=red!75!black,title={#1},fonttitle=\bfseries,breakable}
\newtcolorbox{warning}[1][Warning]{colback=orange!5!white,colframe=orange!75!black,title={#1},fonttitle=\bfseries,breakable}
\newtcolorbox{caution}[1][Caution]{colback=yellow!5!white,colframe=yellow!75!black,title={#1},fonttitle=\bfseries,breakable}
\newtcolorbox{highlight}[1][Highlight]{colback=yellow!10!white,colframe=yellow!75!black,title={#1},fonttitle=\bfseries,breakable}
\newtcolorbox{critical}[1][Critical]{colback=red!10!white,colframe=red!75!black,title={#1},fonttitle=\bfseries,breakable}

\newtcolorbox{analysis}[1][Analysis]{colback=blue!5!white,colframe=blue!75!black,title={#1},fonttitle=\bfseries,breakable}
\newtcolorbox{application}[1][Application]{colback=green!5!white,colframe=green!75!black,title={#1},fonttitle=\bfseries,breakable}
\newtcolorbox{experiment}[1][Experiment]{colback=cyan!5!white,colframe=cyan!75!black,title={#1},fonttitle=\bfseries,breakable}
\newtcolorbox{historical}[1][Historical]{colback=brown!5!white,colframe=brown!75!black,title={#1},fonttitle=\bfseries,breakable}
\newtcolorbox{numerical}[1][Numerical]{colback=gray!5!white,colframe=gray!75!black,title={#1},fonttitle=\bfseries,breakable}
\newtcolorbox{overview}[1][Overview]{colback=blue!5!white,colframe=blue!75!black,title={#1},fonttitle=\bfseries,breakable}
\newtcolorbox{speculation}[1][Speculation]{colback=purple!5!white,colframe=purple!75!black,title={#1},fonttitle=\bfseries,breakable}
\newtcolorbox{question}[1][Question]{colback=orange!5!white,colframe=orange!75!black,title={#1},fonttitle=\bfseries,breakable}
\newtcolorbox{method}[1][Method]{colback=teal!5!white,colframe=teal!75!black,title={#1},fonttitle=\bfseries,breakable}
\newtcolorbox{correct}[1][Correct]{colback=green!10!white,colframe=green!75!black,title={#1},fonttitle=\bfseries,breakable}
\newtcolorbox{units}[1][Units]{colback=gray!5!white,colframe=gray!75!black,title={#1},fonttitle=\bfseries,breakable}
\newtcolorbox{achievement}[1][Achievement]{colback=gold!5!white,colframe=orange!75!black,title={#1},fonttitle=\bfseries,breakable}
\newtcolorbox{equivalence}[1][Equivalence]{colback=cyan!5!white,colframe=cyan!75!black,title={#1},fonttitle=\bfseries,breakable}
\newtcolorbox{dimensional}[1][Dimensional Analysis]{colback=purple!5!white,colframe=purple!75!black,title={#1},fonttitle=\bfseries,breakable}

% === ADDITIONAL SIMPLE ENVIRONMENTS ===
\newenvironment{treatise}{\begin{quote}}{\end{quote}}
\newenvironment{gemeinsam}{\begin{quote}}{\end{quote}}
\newenvironment{vergleich}{\begin{quote}}{\end{quote}}
\newenvironment{vorteil}{\begin{quote}}{\end{quote}}
\newenvironment{common}{\begin{quote}}{\end{quote}}
\newenvironment{comparison}{\begin{quote}}{\end{quote}}
\newenvironment{advantage}{\begin{quote}}{\end{quote}}
\newenvironment{quantum}{\begin{quote}}{\end{quote}}

% === LAYOUT SETTINGS ===
\raggedbottom
\usepackage{environ}
\let\oldtabular\tabular
\let\endoldtabular\endtabular

\newenvironment{scaledtable}[1][0.85]{%
	\begingroup\footnotesize\setlength{\LTleft}{0pt}\setlength{\LTright}{0pt}%
}{%
	\endgroup%
}

\newcommand{\widetable}[1]{\resizebox{\textwidth}{!}{#1}}

% === TABLE OF CONTENTS FORMATTING ===
\renewcommand{\cftsecfont}{\color{blue}}
\renewcommand{\cftsubsecfont}{\color{blue}}
\renewcommand{\cftsecpagefont}{\color{blue}}
\renewcommand{\cftsubsecpagefont}{\color{blue}}
\renewcommand{\cfttoctitlefont}{\huge\bfseries\color{blue}}

% === DEFAULT HEADER AND FOOTER ===
\pagestyle{fancy}
\fancyhf{}
\fancyhead[L]{\textsc{T0 Theory}}
\fancyhead[R]{\textsc{J. Pascher}}
\fancyfoot[C]{\thepage}

% ==============================================================================
% End of Shared Preamble for English
% ==============================================================================

\begin{document}

\title{Tabellen aus xi_parmater_partikel_En.tex}
\maketitle

\tableofcontents
\newpage

\section*{Tabelle 1 aus xi_parmater_partikel_En.tex}
\begin{adjustbox}{max width=\textwidth}
\begin{tabular}{lccc}
					\toprule
					\textbf{Expression} & \textbf{Value} & \textbf{Difference from 1.33} & \textbf{Error [\%]} \\
					\midrule
					4/3 & 1.333333 & +0.003333 & 0.251 \\
					133/100 & 1.330000 & 0.000000 & 0.000 \\
					$\sqrt{7/4}$ & 1.322876 & -0.007124 & 0.536 \\
					21/16 & 1.312500 & -0.017500 & 1.316 \\
					\bottomrule
				\end{tabular}
\end{adjustbox}

\bigskip
\clearpage

\section*{Tabelle 2 aus xi_parmater_partikel_En.tex}
\begin{adjustbox}{max width=\textwidth}
\begin{tabular}{lccc}
			\toprule
			\textbf{Context} & \textbf{Value [$\mytimes 10^{-4}$]} & \textbf{Physical Meaning} & \textbf{Application} \\
			\midrule
			Flat geometry & 1.3165 & QFT in flat spacetime & Local physics \\
			Higgs-calculated & 1.3194 & QFT + minimal corrections & Effective theory \\
			4/3 universal & 1.3300 & 3D space geometry & Universal constant \\
			Spherical geometry & 1.5570 & Curved spacetime & Cosmological physics \\
			\bottomrule
		\end{tabular}
\end{adjustbox}

\bigskip
\clearpage

\section*{Tabelle 3 aus xi_parmater_partikel_En.tex}
\begin{adjustbox}{max width=\textwidth}
\begin{tabular}{ll}
			\toprule
			\textbf{$\xi$ Range} & \textbf{Physical Regime} \\
			\midrule
			Flat $\myrightarrow$ 4/3 & Quantum field theory dominates \\
			4/3 threshold & 3D geometry takes control \\
			4/3 $\myrightarrow$ Spherical & Spacetime curvature dominates \\
			\bottomrule
		\end{tabular}
\end{adjustbox}

\bigskip
\clearpage

\section*{Tabelle 4 aus xi_parmater_partikel_En.tex}
\begin{adjustbox}{max width=\textwidth}
\begin{tabular}{lcc}
			\toprule
			\textbf{Particle} & \textbf{Energy [GeV]} & \textbf{Geometric Context} \\
			\midrule
			Electron & $5.11 \mytimes 10^{-4}$ & Same 4/3 geometry \\
			Proton & $9.38 \mytimes 10^{-1}$ & Same 4/3 geometry \\
			Higgs & $1.25 \mytimes 10^{2}$ & Same 4/3 geometry \\
			Top quark & $1.73 \mytimes 10^{2}$ & Same 4/3 geometry \\
			\bottomrule
		\end{tabular}
\end{adjustbox}

\bigskip
\clearpage

\section*{Tabelle 5 aus xi_parmater_partikel_En.tex}
\begin{adjustbox}{max width=\textwidth}
\begin{tabular}{lcc}
			\toprule
			\textbf{Particle} & \textbf{Energy [GeV]} & \textbf{Frequency Class} \\
			\midrule
			Neutrinos & $\mysim 10^{-12} - 10^{-7}$ & Ultra-low \\
			Electron & $5.11 \mytimes 10^{-4}$ & Low \\
			Proton & $9.38 \mytimes 10^{-1}$ & Medium \\
			W/Z bosons & $\mysim 80-90$ & High \\
			Higgs & $125$ & Very high \\
			\bottomrule
		\end{tabular}
\end{adjustbox}

\bigskip
\clearpage

\section*{Tabelle 6 aus xi_parmater_partikel_En.tex}
\begin{adjustbox}{max width=\textwidth}
\begin{tabular}{lp{5cm}p{4cm}}
			\toprule
			\textbf{Particle} & \textbf{Spatial Pattern} & \textbf{Characteristics} \\
			\midrule
			Electron/Muon & Point-like rotating node & Localized, spin-1/2 \\
			Photon & Extended oscillating pattern & Wave-like, massless \\
			Quarks & Multi-node bound clusters & Confined, color charge \\
			Higgs & Homogeneous background & Scalar, mass-giving \\
			\bottomrule
		\end{tabular}
\end{adjustbox}

\bigskip
\clearpage

\section*{Tabelle 7 aus xi_parmater_partikel_En.tex}
\begin{adjustbox}{max width=\textwidth}
\begin{tabular}{ll}
			\toprule
			\textbf{Musical Concept} & \textbf{T0 Physics Equivalent} \\
			\midrule
			One violin & One universal field $\deltafield(x,t)$ \\
			Different notes & Different particles \\
			Frequency & Particle mass/energy \\
			Harmonics & Excited states \\
			Chords & Composite particles \\
			Resonance & Particle interactions \\
			Amplitude & Field strength/mass \\
			Timbre & Spatial node pattern \\
			\bottomrule
		\end{tabular}
\end{adjustbox}

\bigskip
\clearpage

\section*{Tabelle 8 aus xi_parmater_partikel_En.tex}
\begin{adjustbox}{max width=\textwidth}
\begin{tabular}{lcc}
			\toprule
			\textbf{Aspect} & \textbf{Standard Model} & \textbf{T0 Model} \\
			\midrule
			Fundamental fields & 20+ different & 1 universal ($\deltafield$) \\
			Free parameters & 19+ arbitrary & 1 geometric (4/3) \\
			Particle types & 200+ distinct & Infinite field patterns \\
			Antiparticles & 17 separate fields & Sign flip ($-\deltafield$) \\
			Governing equations & Force-specific & $\partial^2\deltafield = 0$ (universal) \\
			Geometric foundation & None explicit & 4/3 space geometry \\
			Spin origin & Intrinsic property & Node rotation pattern \\
			Mass origin & Higgs mechanism & Field amplitude $|\deltafield|^2$ \\
			\bottomrule
		\end{tabular}
\end{adjustbox}

\bigskip
\clearpage

\section*{Tabelle 9 aus xi_parmater_partikel_En.tex}
\begin{adjustbox}{max width=\textwidth}
\begin{tabular}{lcc}
			\toprule
			\textbf{Parameter} & \textbf{Current Precision} & \textbf{Required for $\xi$ test} \\
			\midrule
			Higgs mass & $\pm 0.17$ GeV & $\pm 0.01$ GeV \\
			Higgs self-coupling & $\pm 20\%$ & $\pm 1\%$ \\
			Higgs VEV & $\pm 0.1$ GeV & $\pm 0.01$ GeV \\
			\bottomrule
		\end{tabular}
\end{adjustbox}

\bigskip
\clearpage

\section*{Tabelle 10 aus xi_parmater_partikel_En.tex}
\begin{adjustbox}{max width=\textwidth}
\begin{tabular}{ll}
			\toprule
			\textbf{Old Paradigm} & \textbf{New T0 Paradigm} \\
			\midrule
			Many fundamental particles & One universal field \\
			Arbitrary parameters & Geometric constants (4/3) \\
			Complex field equations & $\partial^2\deltafield = 0$ \\
			Phenomenological physics & Geometric physics \\
			Separate force descriptions & Unified field dynamics \\
			Quantum vs classical divide & Continuous scale connection \\
			\bottomrule
		\end{tabular}
\end{adjustbox}

\bigskip
\clearpage

\section*{Tabelle 11 aus xi_parmater_partikel_En.tex}
\begin{adjustbox}{max width=\textwidth}
\begin{tcolorbox}[colback=blue!5!white,colframe=blue!75!black,title=Central T0 Principle]
				\textbf{``Every particle is simply a different way the same universal field chooses to dance.''}
				
				\begin{equation}
					\boxed{\text{Reality} = \deltafield(x,t) \text{ dancing in } \xipar \text{-characterized spacetime}}
					\label{eq:fundamental_reality}
				\end{equation}
			\end{tcolorbox}
			
			\section{Mathematical Analysis of the $\xi$ Parameter}
			\label{sec:xi_analysis}
			
			\subsection{Exact vs. Approximated Values}
			\label{subsec:exact_vs_approximated}
			
			\subsubsection{Higgs-Derived Calculation}
			\label{subsubsec:higgs_calculation}
			
			Using Standard Model parameters:
			\begin{align}
				\lambdah &\myapprox 0.13 \quad \text{(Higgs self-coupling)} \\
				v &\myapprox 246 \text{ GeV} \quad \text{(Higgs VEV)} \\
				m_h &\myapprox 125 \text{ GeV} \quad \text{(Higgs mass)}
			\end{align}
			
			The exact calculation yields:
			\begin{equation}
				\xipar_{\text{exact}} = 1.319372 \mytimes 10^{-4}
				\label{eq:xi_exact}
			\end{equation}
			
			\subsubsection{Commonly Used Approximation}
			\label{subsubsec:approximation}
			
			In practical calculations, the value is approximated as:
			\begin{equation}
				\xipar_{\text{approx}} = 1.33 \mytimes 10^{-4}
				\label{eq:xi_approx}
			\end{equation}
			
			\textbf{Relative error}: Only 0.81\%, making this approximation highly accurate for most applications.
			
			\subsection{The Harmonic Meaning of 4/3 - The Universal Fourth}
			\label{subsec:four_thirds_proximity}
			
			\subsubsection{4:3 = THE FOURTH - A Universal Harmonic Ratio}
			\label{subsubsec:four_thirds_connection}
			
			The most striking feature of the $\xi$ parameter is its proximity to the fundamental harmonic constant:
			
			\begin{equation}
				\frac{4}{3} = 1.333333\ldots = \text{Frequency ratio of the perfect fourth}
				\label{eq:four_thirds}
			\end{equation}
			
			The factor 4/3 is not arbitrary but represents the \textbf{perfect fourth}, one of the fundamental harmonic intervals of nature.
			
			\subsubsection{Harmonic Universality}
			\label{subsubsec:harmonic_universality}
			
			Just as musical intervals are universal:
			\begin{itemize}
				\item \textbf{Octave:} 2:1 (always, whether string, air column, or membrane)
				\item \textbf{Fifth:} 3:2 (always)
				\item \textbf{Fourth:} 4:3 (always!)
			\end{itemize}
			
			These ratios are \textbf{geometric/mathematical}, not material-dependent!
			
			\textbf{Why is the fourth universal?}
			
			For a vibrating sphere:
			\begin{itemize}
				\item When divided into 4 equal ``vibration zones''
				\item Compared to 3 zones
				\item The ratio 4:3 emerges
			\end{itemize}
			
			This is \textbf{pure geometry}, independent of material!
			
			\subsubsection{The Harmonic Ratios in the Tetrahedron}
			\label{subsubsec:tetrahedron_harmonics}
			
			The tetrahedron contains BOTH fundamental harmonic intervals:
			\begin{itemize}
				\item \textbf{6 edges : 4 faces = 3:2} (the fifth)
				\item \textbf{4 vertices : 3 edges per vertex = 4:3} (the fourth!)
			\end{itemize}
			
			\textbf{The complementary relationship:}
			Fifth and fourth are complementary intervals - together they form the octave:
			\begin{equation}
				\frac{3}{2} \times \frac{4}{3} = \frac{12}{6} = 2 \quad \text{(Octave)}
			\end{equation}
			
			This demonstrates the complete harmonic structure of space:
			\begin{itemize}
				\item The tetrahedron contains both fundamental intervals
				\item The fourth (4:3) and fifth (3:2) are reciprocally complementary
				\item The harmonic structure is self-consistent and complete
			\end{itemize}
			
			\textbf{Further appearances of the fourth in physics:}
			\begin{itemize}
				\item Crystal lattices (4-fold symmetry)
				\item Spherical harmonics
				\item The sphere volume formula: $V = \frac{4\mypi}{3}r^3$
			\end{itemize}
			
			\subsubsection{The Deeper Meaning}
			\label{subsubsec:deeper_meaning}
			
			\begin{tcolorbox}[colback=green!5!white,colframe=green!75!black,title=The Pythagorean Truth]
				\begin{itemize}
					\item \textbf{Pythagoras was right:} ``Everything is number and harmony''
					\item \textbf{Space itself} has a harmonic structure
					\item \textbf{Particles} are ``tones'' in this cosmic harmony
				\end{itemize}
			\end{tcolorbox}
			
			T0 theory thus reveals: Space is musically/harmonically structured, and 4/3 (the fourth) is its fundamental signature!
			
			If $\xipar = 4/3 \mytimes 10^{-4}$ exactly, this would mean:
			\begin{enumerate}
				\item \textbf{Exact harmonic value}: The fourth as fundamental space constant
				\item \textbf{Parameter-free theory}: No arbitrary constants, all from harmony
				\item \textbf{Unified physics}: Quantum mechanics emerges from harmonic spacetime geometry
			\end{enumerate}
			
			\subsection{Mathematical Structure and Factorization}
			\label{subsec:mathematical_structure}
			
			\subsubsection{Prime Factorization}
			\label{subsubsec:prime_factorization}
			
			The decimal representation reveals interesting structure:
			\begin{equation}
				1.33 = \frac{133}{100} = \frac{7 \mytimes 19}{4 \mytimes 5^2} = \frac{7 \mytimes 19}{100}
				\label{eq:factorization}
			\end{equation}
			
			\textbf{Notable features}:
			\begin{itemize}
				\item Both 7 and 19 are prime numbers
				\item Clean factorization suggests underlying mathematical structure
				\item Factor 100 = $4 \mytimes 5^2$ connects to fundamental geometric ratios
			\end{itemize}
			
			\subsubsection{Rational Approximations}
			\label{subsubsec:rational_approximations}
			
			\begin{table}[htbp]
				\centering
				\begin{tabular}{lccc}
					\toprule
					\textbf{Expression} & \textbf{Value} & \textbf{Difference from 1.33} & \textbf{Error [\%]} \\
					\midrule
					4/3 & 1.333333 & +0.003333 & 0.251 \\
					133/100 & 1.330000 & 0.000000 & 0.000 \\
					$\sqrt{7/4}$ & 1.322876 & -0.007124 & 0.536 \\
					21/16 & 1.312500 & -0.017500 & 1.316 \\
					\bottomrule
				\end{tabular}
				\caption{Rational approximations to $\xi$ coefficient}
				\label{tab:rational_approximations}
			\end{table}
	
	\section{Geometry-Dependent $\xi$ Parameters}
	\label{sec:geometry_dependent_xi}
	
	\subsection{The $\xi$ Parameter Hierarchy}
	\label{subsec:xi_hierarchy}
	
	\subsubsection{Critical Clarification}
	\label{subsubsec:critical_clarification}
	
	\begin{tcolorbox}[colback=red!10!white,colframe=red!75!black,title=CRITICAL WARNING: $\xi$ Parameter Confusion]
		\textbf{COMMON ERROR:} Treating $\xi$ as ``one universal parameter''
		
		\textbf{CORRECT UNDERSTANDING:} $\xi$ is a \textbf{class of dimensionless scale ratios}, not a single value.
		
		$\xi$ represents any dimensionless ratio of the form:
		\begin{equation}
			\xipar = \frac{\text{T0 characteristic scale}}{\text{Reference scale}}
		\end{equation}
	\end{tcolorbox}
\end{adjustbox}

\bigskip
\clearpage

\section*{Tabelle 12 aus xi_parmater_partikel_En.tex}
\begin{adjustbox}{max width=\textwidth}
\begin{tcolorbox}[colback=yellow!5!white,colframe=orange!75!black,title=Three-Dimensional Space Geometry Factor]
		The factor 4/3 in $\xipar \myapprox 4/3 \mytimes 10^{-4}$ represents the \textbf{universal three-dimensional space geometry factor} that:
		\begin{itemize}
			\item Connects quantum field dynamics to 3D spatial structure
			\item Emerges naturally from sphere volume geometry: $V = (4\mypi/3)r^3$
			\item Characterizes how time fields couple to three-dimensional space
			\item Provides the geometric foundation for all particle physics
		\end{itemize}
	\end{tcolorbox}
	
	\subsubsection{Geometric Unity}
	\label{subsubsec:geometric_unity}
	
	This interpretation reveals that:
	\begin{enumerate}
		\item \textbf{Space-time has intrinsic geometric structure} characterized by 4/3
		\item \textbf{Quantum mechanics emerges from geometry}, not vice versa
		\item \textbf{All particles experience the same 3D geometric factor}
		\item \textbf{No free parameters} - everything derives from 3D space geometry
	\end{enumerate}
	
	\subsection{Connection to Particle Physics}
	\label{subsec:connection_particle_physics}
	
	\subsubsection{Universal Geometric Framework}
	\label{subsubsec:universal_framework}
	
	All Standard Model particles exist within the same universal 4/3-characterized spacetime:
	
	\begin{table}[htbp]
		\centering
		\begin{tabular}{lcc}
			\toprule
			\textbf{Particle} & \textbf{Energy [GeV]} & \textbf{Geometric Context} \\
			\midrule
			Electron & $5.11 \mytimes 10^{-4}$ & Same 4/3 geometry \\
			Proton & $9.38 \mytimes 10^{-1}$ & Same 4/3 geometry \\
			Higgs & $1.25 \mytimes 10^{2}$ & Same 4/3 geometry \\
			Top quark & $1.73 \mytimes 10^{2}$ & Same 4/3 geometry \\
			\bottomrule
		\end{tabular}
		\caption{Universal 4/3 geometry for all particles}
		\label{tab:universal_geometry}
	\end{table}
	
	\subsubsection{Unification Principle}
	\label{subsubsec:unification_principle}
	
	The 4/3 geometric factor provides the \textbf{universal foundation} that:
	\begin{itemize}
		\item Unifies all particle types under one geometric principle
		\item Eliminates arbitrary particle classifications
		\item Reduces complex physics to simple geometric relationships
		\item Connects microscopic and cosmological scales
	\end{itemize}
	
	\section{Particle Differentiation in Universal Field}
	\label{sec:particle_differentiation}
	
	\subsection{The Five Fundamental Differentiation Factors}
	\label{subsec:five_factors}
	
	Within the universal 4/3-geometric framework, particles distinguish themselves through five fundamental mechanisms:
	
	\subsubsection{Factor 1: Field Excitation Frequency}
	\label{subsubsec:excitation_frequency}
	
	Particles represent different frequencies of the universal field:
	\begin{equation}
		E = \hbar \myomega \quad \myRightarrow \quad \text{Particle identity} \mypropto \text{Field frequency}
		\label{eq:frequency_identity}
	\end{equation}
	
	\begin{table}[htbp]
		\centering
		\begin{tabular}{lcc}
			\toprule
			\textbf{Particle} & \textbf{Energy [GeV]} & \textbf{Frequency Class} \\
			\midrule
			Neutrinos & $\mysim 10^{-12} - 10^{-7}$ & Ultra-low \\
			Electron & $5.11 \mytimes 10^{-4}$ & Low \\
			Proton & $9.38 \mytimes 10^{-1}$ & Medium \\
			W/Z bosons & $\mysim 80-90$ & High \\
			Higgs & $125$ & Very high \\
			\bottomrule
		\end{tabular}
		\caption{Particle classification by field frequency}
		\label{tab:frequency_classification}
	\end{table}
	
	\subsubsection{Factor 2: Spatial Node Patterns}
	\label{subsubsec:spatial_patterns}
	
	Different particles correspond to distinct spatial field configurations:
	
	\begin{table}[htbp]
		\centering
		\begin{tabular}{lp{5cm}p{4cm}}
			\toprule
			\textbf{Particle} & \textbf{Spatial Pattern} & \textbf{Characteristics} \\
			\midrule
			Electron/Muon & Point-like rotating node & Localized, spin-1/2 \\
			Photon & Extended oscillating pattern & Wave-like, massless \\
			Quarks & Multi-node bound clusters & Confined, color charge \\
			Higgs & Homogeneous background & Scalar, mass-giving \\
			\bottomrule
		\end{tabular}
		\caption{Spatial field patterns for particle types}
		\label{tab:spatial_field_patterns}
	\end{table}
	
	\subsubsection{Factor 3: Rotation/Oscillation Behavior (Spin)}
	\label{subsubsec:spin_behavior}
	
	Spin emerges from field node rotation patterns:
	
	\begin{tcolorbox}[colback=green!5!white,colframe=green!75!black,title=Spin from Field Node Rotation]
		\begin{itemize}
			\item \textbf{Fermions (Spin-1/2)}: $4\mypi$ rotation cycle for field nodes
			\item \textbf{Bosons (Spin-1)}: $2\mypi$ rotation cycle for field nodes
			\item \textbf{Scalars (Spin-0)}: No rotation, spherically symmetric
		\end{itemize}
		
		\textbf{Pauli exclusion}: Identical node patterns cannot occupy same spacetime region
	\end{tcolorbox}
\end{adjustbox}

\bigskip
\clearpage

\section*{Tabelle 13 aus xi_parmater_partikel_En.tex}
\begin{adjustbox}{max width=\textwidth}
\begin{tcolorbox}[colback=green!5!white,colframe=green!75!black,title=T0 Unification Achievement]
		\textbf{From}: 200+ Standard Model particles with arbitrary properties and 19+ free parameters
		
		\textbf{To}: ONE universal field $\deltafield(x,t)$ with infinite pattern expressions in 4/3-characterized spacetime
		
		\textbf{Result}: Complete elimination of fundamental particle taxonomy through geometric unification
	\end{tcolorbox}
	
	\section{Experimental Implications and Predictions}
	\label{sec:experimental_implications}
	
	\subsection{$\xi$ Parameter Precision Tests}
	\label{subsec:xi_precision_tests}
	
	\subsubsection{Testing the 4/3 Hypothesis}
	\label{subsubsec:testing_four_thirds}
	
	Precision measurements of Higgs parameters could resolve whether $\xipar = 4/3 \mytimes 10^{-4}$ exactly:
	
	\begin{table}[htbp]
		\centering
		\begin{tabular}{lcc}
			\toprule
			\textbf{Parameter} & \textbf{Current Precision} & \textbf{Required for $\xi$ test} \\
			\midrule
			Higgs mass & $\pm 0.17$ GeV & $\pm 0.01$ GeV \\
			Higgs self-coupling & $\pm 20\%$ & $\pm 1\%$ \\
			Higgs VEV & $\pm 0.1$ GeV & $\pm 0.01$ GeV \\
			\bottomrule
		\end{tabular}
		\caption{Precision requirements for testing $\xi = 4/3$ hypothesis}
		\label{tab:precision_requirements}
	\end{table}
	
	\subsubsection{Geometric Transition Experiments}
	\label{subsubsec:geometric_transitions}
	
	Experiments could test the geometric $\xi$ hierarchy:
	\begin{itemize}
		\item \textbf{Local measurements}: Should yield $\xipar_{\text{flat}}$ values
		\item \textbf{Cosmological observations}: Should show $\xipar_{\text{spherical}}$ effects
		\item \textbf{Intermediate scales}: Should exhibit geometric transitions
	\end{itemize}
	
	\subsection{Universal Field Pattern Tests}
	\label{subsec:field_pattern_tests}
	
	\subsubsection{Universal Lepton Corrections}
	\label{subsubsec:universal_lepton_corrections}
	
	All leptons should exhibit identical anomalous magnetic moment corrections:
	\begin{equation}
		a_{\ell}^{(T0)} = \frac{\xipar}{2\mypi} \mytimes \frac{1}{12} \myapprox 2.34 \mytimes 10^{-10}
		\label{eq:universal_lepton_prediction}
	\end{equation}
	
	This provides a direct test of universal field theory.
	
	\subsubsection{Field Node Pattern Detection}
	\label{subsubsec:node_pattern_detection}
	
	Advanced experiments might directly observe:
	\begin{itemize}
		\item \textbf{Node rotation signatures}: Spin as physical rotation
		\item \textbf{Field amplitude correlations}: Mass-amplitude relationships
		\item \textbf{Spatial pattern mapping}: Direct field structure visualization
		\item \textbf{Frequency spectrum analysis}: Particle-frequency correspondence
	\end{itemize}
	
	\section{Philosophical and Theoretical Implications}
	\label{sec:philosophical_implications}
	
	\subsection{The Nature of Mathematical Reality}
	\label{subsec:mathematical_reality}
	
	\subsubsection{4/3 as Universal Constant}
	\label{subsubsec:four_thirds_universal}
	
	If $\xipar = 4/3 \mytimes 10^{-4}$ exactly, this suggests that:
	
	\begin{enumerate}
		\item \textbf{Mathematics is the language of nature}: 3D geometry determines physics
		\item \textbf{No arbitrary constants}: All physics emerges from geometric principles
		\item \textbf{Unity of scales}: Same geometry governs quantum and cosmic phenomena
		\item \textbf{Predictive power}: Theory becomes truly parameter-free
	\end{enumerate}
	
	\subsubsection{Geometric Reductionism}
	\label{subsubsec:geometric_reductionism}
	
	The T0 framework achieves ultimate reductionism:
	\begin{equation}
		\boxed{\text{All physics} = \text{3D geometry} + \text{field dynamics}}
		\label{eq:ultimate_reductionism}
	\end{equation}
	
	\subsection{Implications for Fundamental Physics}
	\label{subsec:fundamental_physics}
	
	\subsubsection{Theory of Everything Candidate}
	\label{subsubsec:toe_candidate}
	
	The T0 model exhibits key ``Theory of Everything'' characteristics:
	\begin{itemize}
		\item \textbf{Complete unification}: One field, one equation, one geometric constant
		\item \textbf{Parameter-free}: No arbitrary inputs required
		\item \textbf{Scale invariant}: Same principles from quantum to cosmic scales
		\item \textbf{Experimentally testable}: Makes specific, falsifiable predictions
	\end{itemize}
	
	\subsubsection{Paradigm Shift Summary}
	\label{subsubsec:paradigm_shift}
	
	\begin{table}[htbp]
		\centering
		\begin{tabular}{ll}
			\toprule
			\textbf{Old Paradigm} & \textbf{New T0 Paradigm} \\
			\midrule
			Many fundamental particles & One universal field \\
			Arbitrary parameters & Geometric constants (4/3) \\
			Complex field equations & $\partial^2\deltafield = 0$ \\
			Phenomenological physics & Geometric physics \\
			Separate force descriptions & Unified field dynamics \\
			Quantum vs classical divide & Continuous scale connection \\
			\bottomrule
		\end{tabular}
		\caption{Paradigm shift from Standard Model to T0 theory}
		\label{tab:paradigm_shift}
	\end{table}
	
	\section{Conclusions and Future Directions}
	\label{sec:conclusions}
	
	\subsection{Summary of Key Findings}
	\label{subsec:key_findings}
	
	This comprehensive analysis reveals several profound insights:
	
	\subsubsection{$\xi$ Parameter Mathematical Structure}
	\label{subsubsec:xi_mathematical_summary}
	
	\begin{enumerate}
		\item The calculated value $\xipar = 1.319372 \mytimes 10^{-4}$ lies remarkably close to $4/3 \mytimes 10^{-4}$
		\item Multiple $\xi$ variants (flat, Higgs, 4/3, spherical) form a systematic geometric hierarchy
		\item The 4/3 factor represents the universal three-dimensional space geometry constant
		\item Mathematical factorization $(7 \mytimes 19)/100$ suggests deeper structural relationships
	\end{enumerate}
	
	\subsubsection{Particle Differentiation Mechanisms}
	\label{subsubsec:particle_differentiation_summary}
	
	\begin{enumerate}
		\item All particles are excitation patterns of one universal field $\deltafield(x,t)$
		\item Five fundamental factors distinguish particles: frequency, spatial pattern, rotation, amplitude, coupling
		\item Universal Klein-Gordon equation $\partial^2\deltafield = 0$ governs all particle types
		\item Standard Model complexity reduces to elegant field pattern diversity
	\end{enumerate}
	
	\subsection{Revolutionary Achievements}
	\label{subsec:revolutionary_achievements}
	
	\subsubsection{Unification Success}
	\label{subsubsec:unification_success}
	
	\begin{tcolorbox}[colback=yellow!10!white,colframe=orange!75!black,title=T0 Theory Revolutionary Achievements]
		\begin{itemize}
			\item \textbf{Parameter reduction}: 19+ Standard Model parameters $\myrightarrow$ 1 geometric constant (4/3)
			\item \textbf{Field unification}: 20+ different fields $\myrightarrow$ 1 universal field $\deltafield(x,t)$
			\item \textbf{Equation unification}: Multiple force equations $\myrightarrow$ $\partial^2\deltafield = 0$
			\item \textbf{Geometric foundation}: Arbitrary physics $\myrightarrow$ 3D space geometry
			\item \textbf{Scale connection}: Quantum-classical divide $\myrightarrow$ continuous hierarchy
		\end{itemize}
	\end{tcolorbox}
\end{adjustbox}

\bigskip
\clearpage

\end{document}