\documentclass[12pt,a4paper]{book}
% ==============================================================================
% T0 Theory: Shared ENGLISH Preamble – Optimized for eBook/Book
% Version: 2.0 – Final 2026 (LuaLaTeX only) – ENGLISH corrected
% Author: Johann Pascher
% Date: January 2026
% ==============================================================================
%
% IMPORTANT: Compile EXCLUSIVELY with LuaLaTeX!
% In TeXstudio: Options → Configure TeXstudio → Build → Default Compiler → LuaLaTeX
%
% Required Fonts (install once):
% - Inter: https://fonts.google.com/specimen/Inter
% - JetBrains Mono: https://www.jetbrains.com/lp/mono/
% - Libertinus Math: https://github.com/libertinus-fonts/libertinus
% ==============================================================================

% === CHAPTER 1: BASIC PACKAGES (must come FIRST) ===
\RequirePackage{fontspec}
\RequirePackage{unicode-math}
\usepackage{chngcntr}
\setcounter{secnumdepth}{1}  % Nur Sections nummerieren (nicht subsections)
\setcounter{tocdepth}{1}     % Nur Sections im TOC (nicht subsections)
\makeatletter
\@ifundefined{c@chapter}{}{\counterwithout{section}{chapter}}  % Falls Kapitel existieren
\makeatother
\counterwithout{subsection}{section}  % Löse Verknüpfung
% === CHAPTER 2: LANGUAGE (ENGLISH) ===
\usepackage[english]{babel}
\usepackage{microtype}                    % IMPORTANT for better hyphenation!

% Typography settings for better line breaking
\frenchspacing                     % Correct English spacing after punctuation
\emergencystretch=3em              % Allows more stretch for difficult lines
\tolerance=2500                    % Higher tolerance for line breaks
\hbadness=10000                    % Suppresses "underfull hbox" warnings
\hfuzz=2pt                         % Allows minimal overfull
\pretolerance=150                  % Better word breaking

% Prevent bad page breaks
\clubpenalty=10000           % No "orphans"
\widowpenalty=10000          % No "widows"
\displaywidowpenalty=10000   % Also with equations
\brokenpenalty=10000         % No broken words across pages

% Explicit hyphenation for long technical words
\hyphenation{Fun-da-men-tal Frac-tal-Ge-o-met-ric Field The-o-ry Meth-od-o-log-i-cal}
\hyphenation{Re-vi-sion-ism Quan-ti-za-tion U-ni-fi-ca-tion Ef-fec-tive}
\hyphenation{Re-nor-mal-iz-a-bil-i-ty Sin-gu-lar-i-ties Con-cil-i-a-tion}
\hyphenation{E-mer-gence Phe-nom-e-no-log-i-cal Doc-u-men-ta-tion A-nal-y-sis}
\hyphenation{Grav-i-ta-tion Quan-tum Me-chan-ics Dog-ma-tism Con-se-quent}
\hyphenation{Par-al-lel-ism Im-ple-men-ta-tion Per-tur-ba-tions}
\hyphenation{Geo-met-ric Ar-ti-fact In-com-pat-i-bil-i-ty Con-struc-tive}
\hyphenation{Frac-tal Di-men-sion-less In-ves-ti-ga-tion De-scrip-tion}
\hyphenation{In-ter-pre-ta-tion Phe-nom-e-no-log-i-cal Math-e-mat-i-cal}
\hyphenation{Phi-lo-soph-i-cal Le-git-i-ma-tion Ap-pli-ca-tion Der-i-va-tion}
\hyphenation{U-ni-fi-ca-tion As-sump-tion Con-cep-tion Ex-pec-ta-tion}
\hyphenation{Sym-me-try-ex-ten-sion O-ver-all-pic-ture Chal-lenge}
\hyphenation{In-ter-ac-tion Ma-te-ri-al Ap-proach Per-spec-tive Pro-ce-dure}

% === CHAPTER 3: FONTS (with proper ligatures) ===
\setmainfont{Inter}[
Scale=1.02,
UprightFont=*-Regular,
BoldFont=*-Bold,
ItalicFont=*-Italic,
BoldItalicFont=*-BoldItalic,
Ligatures=TeX,           % IMPORTANT for proper typography
Language=English         % Explicit language support
]
\setsansfont{Inter}[
Scale=MatchLowercase,
Ligatures=TeX,
Language=English
]
\setmonofont{JetBrains Mono}[
Scale=0.95,
Language=English
]

% Math Font (simple & stable) – MUST come AFTER language definition
% IMPORTANT: Libertinus Math for correct \underbrace display!
\setmathfont{Libertinus Math}[Scale=1.0]

% === CHAPTER 4: MATHEMATICS PACKAGES (in STRICT order!) ===
% IMPORTANT: mathtools must come BEFORE unicode-math for some commands!
\usepackage{mathtools}           % FIRST mathtools!

% Then the rest
\usepackage{amsmath, amsfonts, amsthm}

% SIUNITX MUST be loaded BEFORE physics!
\usepackage{siunitx}
\sisetup{
	locale=US,                    % ENGLISH settings for SI units!
	group-separator={,},          % Thousands separator comma
	output-decimal-marker={.},    % Decimal separator point
	per-mode=symbol,
	separate-uncertainty=true
}

% Custom SI units used in narrative and books
\DeclareSIUnit\gigalightyear{Gly}
\DeclareSIUnit\mev{MeV}

% physics – MUST be loaded AFTER siunitx and mathtools
\usepackage{physics}

% === CHAPTER 5: ADDITIONS from pdflatex best practices ===
\usepackage{colortbl}        % Colored tables (ESSENTIAL!)
\usepackage{placeins}        % Float control: \FloatBarrier
\usepackage{subcaption}      % Subfigures
\usepackage{xurl}            % Better URL line breaking
% Hyphenation for URLs in bibliography
\def\UrlBreaks{\do\/\do-}

% === CHAPTER 6: PAGE LAYOUT
% =============================================================================
% SECTION 2: Page Geometry – 6" × 9" Buchformat
% =============================================================================
\usepackage[paperwidth=6in, paperheight=9in,
top=0.9in,
bottom=1.1in,
inner=0.9in,            % Größerer Innenrand für Bindung
outer=0.6in,            % Kleinerer Außenrand → mehr Text pro Seite
bindingoffset=0.5in,    % Puffer für Bindung (Steg)
twoside]{geometry}
\setlength{\headheight}{15pt}
%\usepackage[paperwidth=8.25in, paperheight=11in,
%top=1.0in,
%bottom=1.0in,
%left=1.0in,
%right=1.0in,
%twoside=false
% === CHAPTER 7: GRAPHICS AND TABLES ===
\usepackage{graphicx}
\usepackage[table,xcdraw]{xcolor}
% T0 brand colors
\definecolor{gold}{RGB}{255,215,0}
\definecolor{blue}{rgb}{0,0,1}
\definecolor{boxgray}{RGB}{240,240,240}
\definecolor{deepblue}{RGB}{0,0,127}
\definecolor{deepgreen}{RGB}{0,127,0}
\definecolor{deepred}{RGB}{191,0,0}
\definecolor{t0blue}{RGB}{33,150,243}
\definecolor{t0green}{RGB}{76,175,80}
\definecolor{t0orange}{RGB}{255,152,0}
\definecolor{t0purple}{RGB}{156,39,176}
\definecolor{t0red}{RGB}{244,67,54}
\definecolor{t0yellow}{RGB}{255,204,0}
\usepackage{tikz}
\usetikzlibrary{arrows.meta,positioning,shapes.geometric,decorations.pathmorphing,patterns,shapes.arrows,intersections}
\usepackage{pgfplots}
\pgfplotsset{compat=1.18}
\usepackage{quantikz}
\usepackage[most]{tcolorbox}
\tcbuselibrary{breakable}

% === WICHTIG: Algorithm-Konflikt umgehen ===
% Option: algorithmic mit GROSSBUCHSTABEN
% Gemeinsame Box für Experimente
\newtcolorbox{experimentbox}[1][]{
	colback=green!5!white,
	colframe=t0green!80!black,
	fonttitle=\bfseries,
	title={{#1}},
	breakable
}

% Abstract-Fallback
\ifdefined\abstract\else
\newenvironment{abstract}{\section*{\abstractname}\itshape\small\par\bigskip}{\bigskip}
\fi

% === MAKROS SICHER NEU DEFINIEREN / ÜBERSCHREIBEN ===
% Definiere Makros OHNE doppelte Subskripte
\newcommand{\phipar}{\phi_{\mathrm{par}}}
%\newcommand{\xipar}{\xi_{\mathrm{par}}}
\newcommand{\Qphipar}{Q_{\phi_{\mathrm{par}}}}
\newcommand{\rphipar}{r_{\phi_{\mathrm{par}}}}
\newcommand{\logphipar}{\log_{\phi_{\mathrm{par}}}}
\newcommand{\CHSH}{\text{CHSH}}
\usepackage{booktabs}
\usepackage{array}
\usepackage{longtable}
\usepackage{float}
\usepackage{adjustbox}
\usepackage{rotating}
\usepackage{tabularx}
\usepackage{makecell}
\usepackage{multirow}

% === CHAPTER 8: DOCUMENT FORMATTING ===
\usepackage{fancyhdr}
\renewcommand{\headrulewidth}{0.4pt}
\renewcommand{\footrulewidth}{0.4pt}
\usepackage{tocloft}

\usepackage{enumitem}
\setlist[itemize]{leftmargin=*, topsep=2pt, partopsep=0pt, parsep=2pt, itemsep=2pt}
\setlist[enumerate]{leftmargin=*, topsep=2pt, partopsep=0pt, parsep=2pt, itemsep=2pt}
\usepackage{setspace}
\usepackage{ragged2e}
\usepackage{multicol}

% === CHAPTER 9: CODE AND ALGORITHMS ===
\usepackage{algorithm}
\usepackage{algorithmic}
\usepackage{listings}
\lstset{
	basicstyle=\ttfamily\footnotesize,
	breaklines=true,
	breakatwhitespace=true,
	columns=flexible,
	keepspaces=true,
	showstringspaces=false,
	frame=single,
	xleftmargin=0pt,
	xrightmargin=0pt,
	literate=              % For special characters in code listings
	{ä}{{\"a}}1 {ö}{{\"o}}1 {ü}{{\"u}}1 {ß}{{\ss}}1
	{Ä}{{\"A}}1 {Ö}{{\"O}}1 {Ü}{{\"U}}1
}
\usepackage{mdframed}

% === CHAPTER 10: ADDITIONAL PACKAGES ===
\usepackage{pdflscape}
\usepackage{braket}
\usepackage{cancel}
\usepackage{caption}
\captionsetup{format=plain, labelfont=bf, justification=centering}
\usepackage{csquotes}
\usepackage{gensymb}
\usepackage{textcomp}
\usepackage{textgreek}
\usepackage{upgreek}
\usepackage{url}
\usepackage{slashed}
\usepackage{bm}

% === CHAPTER 11: HYPERREF (must come SECOND TO LAST!) ===
\usepackage{hyperref}
\hypersetup{
	colorlinks=true,
	linkcolor=black,
	citecolor=black,
	urlcolor=black,
	breaklinks=true,           % IMPORTANT for special characters in URLs!
	bookmarksnumbered=true,
	unicode=true,
	pdfencoding=auto,
	pdflang=en,                % Set PDF language to English
	pdfsubject={T0 Theory - Fundamental Fractal-Geometric Field Theory}
}

% Fix for unicode-math symbols in PDF bookmarks
\pdfstringdefDisableCommands{%
	\def\xi{xi}%
	\def\alpha{alpha}%
	\def\beta{beta}%
	\def\gamma{gamma}%
	\def\delta{delta}%
	\def\Delta{Delta}%
	\def\epsilon{epsilon}%
	\def\varepsilon{epsilon}%
	\def\theta{theta}%
	\def\kappa{kappa}%
	\def\lambda{lambda}%
	\def\mu{mu}%
	\def\nu{nu}%
	\def\pi{pi}%
	\def\rho{rho}%
	\def\sigma{sigma}%
	\def\tau{tau}%
	\def\phi{phi}%
	\def\chi{chi}%
	\def\psi{psi}%
	\def\omega{omega}%
	\def\Omega{Omega}%
	\def\Lambda{Lambda}%
	\def\times{x}%
	\def\cdot{*}%
	\def\pm{+/-}%
	\def\approx{~}%
	\def\sim{~}%
	\def\equiv{=}%
	\def\ell{l}%
	\def\hbar{h}%
	\def\rightarrow{->}%
	\def\leftarrow{<-}%
	\def\Rightarrow{=>}%
	\def\Leftarrow{<=}%
	\def\propto{~}%
	\def\mitxi{xi}%
	\def\mitalpha{alpha}%
	\def\mitbeta{beta}%
	\def\mitgamma{gamma}%
	\def\mitdelta{delta}%
	\def\mitDelta{Delta}%
	\def\mitepsilon{epsilon}%
	\def\mitvarepsilon{epsilon}%
	\def\mittheta{theta}%
	\def\mitkappa{kappa}%
	\def\mitlambda{lambda}%
	\def\mitLambda{Lambda}%
	\def\mitmu{mu}%
	\def\mitnu{nu}%
	\def\mitpi{pi}%
	\def\mitrho{rho}%
	\def\mitsigma{sigma}%
	\def\mittau{tau}%
	\def\mitphi{phi}%
	\def\mitchi{chi}%
	\def\mitpsi{psi}%
	\def\mitomega{omega}%
	\def\mitOmega{Omega}%
}

% === CHAPTER 12: BOOKMARK (must come AFTER hyperref!) ===
\usepackage{bookmark}

% === CHAPTER 13: CLEVEREF (ENGLISH LABELS) ===
\usepackage[english]{cleveref}
\crefname{equation}{Equation}{Equations}
\crefname{figure}{Figure}{Figures}
\crefname{table}{Table}{Tables}
\crefname{section}{Section}{Sections}
\crefname{chapter}{Chapter}{Chapters}
\crefname{theorem}{Theorem}{Theorems}
\crefname{lemma}{Lemma}{Lemmas}
\crefname{definition}{Definition}{Definitions}
\crefname{example}{Example}{Examples}
\crefname{remark}{Remark}{Remarks}

% === CUSTOM ENVIRONMENTS ===
% Alternative interpretation environment
\newenvironment{alternative}{%
	\begin{mdframed}[linecolor=black!30,linewidth=1pt,roundcorner=4pt,backgroundcolor=black!5]%
	}{%
	\end{mdframed}%
}

% Photon/particle environment
\newenvironment{photon}{%
	\begin{mdframed}[linecolor=blue!30,linewidth=1pt,roundcorner=4pt,backgroundcolor=blue!5]%
	}{%
	\end{mdframed}%
}

% Koide formula box environment
\newenvironment{koidebox}{%
	\begin{mdframed}[linecolor=green!30,linewidth=1pt,roundcorner=4pt,backgroundcolor=green!5]%
	}{%
	\end{mdframed}%
}

% Erkenntnis/insight environment
\newenvironment{erkenntnis}{%
	\begin{mdframed}[linecolor=orange!30,linewidth=1pt,roundcorner=4pt,backgroundcolor=orange!5]%
	}{%
	\end{mdframed}%
}

% Beziehung/relationship environment
\newenvironment{beziehung}{%
	\begin{mdframed}[linecolor=purple!30,linewidth=1pt,roundcorner=4pt,backgroundcolor=purple!5]%
	}{%
	\end{mdframed}%
}

% Derivation environment
\newenvironment{derivation}{%
	\begin{mdframed}[linecolor=teal!30,linewidth=1pt,roundcorner=4pt,backgroundcolor=teal!5]%
	}{%
	\end{mdframed}%
}

% Abhandlung/treatise environment
\newenvironment{abhandlung}{%
	\begin{mdframed}[linecolor=brown!30,linewidth=1pt,roundcorner=4pt,backgroundcolor=brown!5]%
	}{%
	\end{mdframed}%
}

% Anwendung/application environment
\newenvironment{anwendung}{%
	\begin{mdframed}[linecolor=cyan!30,linewidth=1pt,roundcorner=4pt,backgroundcolor=cyan!5]%
	}{%
	\end{mdframed}%
}

% Additional common environments
\newenvironment{konsequenz}{%
	\begin{mdframed}[linecolor=red!30,linewidth=1pt,roundcorner=4pt,backgroundcolor=red!5]%
	}{%
	\end{mdframed}%
}

\newenvironment{schlussfolgerung}{%
	\begin{mdframed}[linecolor=gray!30,linewidth=1pt,roundcorner=4pt,backgroundcolor=gray!5]%
	}{%
	\end{mdframed}%
}

\newenvironment{result}{%
	\begin{mdframed}[linecolor=violet!30,linewidth=1pt,roundcorner=4pt,backgroundcolor=violet!5]%
	}{%
	\end{mdframed}%
}

% Formula environment
\newenvironment{formula}{%
	\begin{mdframed}[linecolor=yellow!30,linewidth=1pt,roundcorner=4pt,backgroundcolor=yellow!5]%
	}{%
	\end{mdframed}%
}

% Revolutionaer/revolutionary environment
\newenvironment{revolutionaer}{%
	\begin{mdframed}[linecolor=red!50,linewidth=2pt,roundcorner=4pt,backgroundcolor=red!10]%
	}{%
	\end{mdframed}%
}

% Formel environment (German version of formula)
\newenvironment{formel}{%
	\begin{mdframed}[linecolor=yellow!30,linewidth=1pt,roundcorner=4pt,backgroundcolor=yellow!5]%
	}{%
	\end{mdframed}%
}

% Prinzip/principle environment
\newenvironment{prinzip}{%
	\begin{mdframed}[linecolor=blue!50,linewidth=2pt,roundcorner=4pt,backgroundcolor=blue!10]%
	}{%
	\end{mdframed}%
}

% Experimentell/experimental environment
\newenvironment{experimentell}{%
	\begin{mdframed}[linecolor=magenta!30,linewidth=1pt,roundcorner=4pt,backgroundcolor=magenta!5]%
	}{%
	\end{mdframed}%
}

% Neutrino environment
\newenvironment{neutrino}{%
	\begin{mdframed}[linecolor=cyan!40,linewidth=1pt,roundcorner=4pt,backgroundcolor=cyan!8]%
	}{%
	\end{mdframed}%
}

% Additional missing environments
\newenvironment{schluessel}{%
	\begin{mdframed}[linecolor=yellow!50,linewidth=1pt,roundcorner=4pt,backgroundcolor=yellow!10]%
	}{%
	\end{mdframed}%
}

\newenvironment{summary}{%
	\begin{mdframed}[linecolor=gray!40,linewidth=1pt,roundcorner=4pt,backgroundcolor=gray!8]%
	}{%
	\end{mdframed}%
}

\newenvironment{category}{%
	\begin{mdframed}[linecolor=pink!40,linewidth=1pt,roundcorner=4pt,backgroundcolor=pink!8]%
	}{%
	\end{mdframed}%
}

\newenvironment{sibox}{%
	\begin{mdframed}[linecolor=lime!40,linewidth=1pt,roundcorner=4pt,backgroundcolor=lime!8]%
	}{%
	\end{mdframed}%
}

% More missing environments
\newenvironment{documentbox}{%
	\begin{mdframed}[linecolor=teal!40,linewidth=1pt,roundcorner=4pt,backgroundcolor=teal!8]%
	}{%
	\end{mdframed}%
}

\newenvironment{t0box}{%
	\begin{mdframed}[linecolor=violet!40,linewidth=1pt,roundcorner=4pt,backgroundcolor=violet!8]%
	}{%
	\end{mdframed}%
}

\newenvironment{wichtig}{%
	\begin{mdframed}[linecolor=red!50,linewidth=2pt,roundcorner=4pt,backgroundcolor=red!10]%
	\textbf{Important:} 
	}{%
	\end{mdframed}%
}

\newenvironment{smbox}{%
	\begin{mdframed}[linecolor=orange!40,linewidth=1pt,roundcorner=4pt,backgroundcolor=orange!8]%
	}{%
	\end{mdframed}%
}

\newenvironment{pvbox}{%
	\begin{mdframed}[linecolor=purple!40,linewidth=1pt,roundcorner=4pt,backgroundcolor=purple!8]%
	}{%
	\end{mdframed}%
}

\newenvironment{numerisch}{%
	\begin{mdframed}[linecolor=blue!40,linewidth=1pt,roundcorner=4pt,backgroundcolor=blue!8]%
	}{%
	\end{mdframed}%
}

% More missing environments
\newenvironment{relation}{%
	\begin{mdframed}[linecolor=green!40,linewidth=1pt,roundcorner=4pt,backgroundcolor=green!8]%
	}{%
	\end{mdframed}%
}

\newenvironment{beweis}{%
	\begin{mdframed}[linecolor=brown!40,linewidth=1pt,roundcorner=4pt,backgroundcolor=brown!8]%
	\textbf{Proof:} 
	}{%
	\end{mdframed}%
}

\newenvironment{revolution}{%
	\begin{mdframed}[linecolor=red!60,linewidth=2pt,roundcorner=4pt,backgroundcolor=red!12]%
	}{%
	\end{mdframed}%
}

\newenvironment{key}{%
	\begin{mdframed}[linecolor=yellow!50,linewidth=1pt,roundcorner=4pt,backgroundcolor=yellow!10]%
	}{%
	\end{mdframed}%
}

\newenvironment{newperspective}{%
	\begin{mdframed}[linecolor=cyan!50,linewidth=1pt,roundcorner=4pt,backgroundcolor=cyan!10]%
	}{%
	\end{mdframed}%
}

\newenvironment{literatur}{%
	\begin{mdframed}[linecolor=gray!50,linewidth=1pt,roundcorner=4pt,backgroundcolor=gray!10]%
	}{%
	\end{mdframed}%
}

\newenvironment{folgerung}{%
	\begin{mdframed}[linecolor=teal!50,linewidth=1pt,roundcorner=4pt,backgroundcolor=teal!10]%
	}{%
	\end{mdframed}%
}

\newenvironment{principle}{%
	\begin{mdframed}[linecolor=blue!60,linewidth=2pt,roundcorner=4pt,backgroundcolor=blue!12]%
	}{%
	\end{mdframed}%
}

% Additional common environments
% ==============================================================================
% FROM HERE: YOUR DEFINITIONS (unchanged)
% ==============================================================================

\setcounter{tocdepth}{3}

% === CITATION COMMANDS ===
\providecommand{\citep}[1]{\cite{#1}}
\providecommand{\citet}[1]{\cite{#1}}

% === COLORS ===
\definecolor{gold}{RGB}{255,215,0}
\definecolor{blue}{rgb}{0,0,1}
\definecolor{boxgray}{RGB}{240,240,240}
\definecolor{deepblue}{RGB}{0,0,127}
\definecolor{deepgreen}{RGB}{0,127,0}
\definecolor{deepred}{RGB}{191,0,0}
\definecolor{t0blue}{RGB}{33,150,243}
\definecolor{t0green}{RGB}{76,175,80}
\definecolor{t0orange}{RGB}{255,152,0}
\definecolor{t0purple}{RGB}{156,39,176}
\definecolor{t0red}{RGB}{244,67,54}
\definecolor{t0yellow}{RGB}{255,204,0}

% === COLUMN TYPES ===
\newcolumntype{L}[1]{>{\raggedright\arraybackslash}p{#1}}
\newcolumntype{C}[1]{>{\centering\arraybackslash}p{#1}}
\newcolumntype{R}[1]{>{\raggedleft\arraybackslash}p{#1}}

% === HYPERREF SETTINGS (updated) ===
\hypersetup{
	colorlinks=true,
	linkcolor=t0blue,
	citecolor=t0blue,
	urlcolor=t0blue,
	breaklinks=true,
	bookmarksnumbered=true,
	pdfstartview=FitH,
	pdfencoding=auto,
	pdfdisplaydoctitle=true
}

% === ENGLISH THEOREM ENVIRONMENTS ===
\theoremstyle{plain}
\newtheorem{theorem}{Theorem}[section]
\newtheorem{lemma}[theorem]{Lemma}
\newtheorem{proposition}[theorem]{Proposition}
\newtheorem{corollary}[theorem]{Corollary}

\theoremstyle{definition}
\newtheorem{definition}[theorem]{Definition}
\newtheorem{example}[theorem]{Example}
\newtheorem{insight}[theorem]{Insight}
\newtheorem{discovery}[theorem]{Discovery}

\theoremstyle{remark}
\newtheorem{remark}[theorem]{Remark}
\newtheorem{axiom}{Axiom}
%\newtheorem{principle}{Principle}  % Commented out to avoid conflicts with document-specific definitions
%\newtheorem{warning}[theorem]{Warning}

% === T0-SPECIFIC COMMANDS ===
% (Here follow all your \newcommand and \providecommand definitions)
% These remain UNCHANGED as in your original preamble
% ==============================================================================
% SECTION 14: T0-Specific Commands
% ==============================================================================

% --- Core T0 Fields ---
\newcommand{\Tfield}{T(x,t)}
\providecommand{\Tfieldt}{T(\vec{x},t)}
\newcommand{\Efield}{E(x,t)}
\newcommand{\mfield}{m(x,t)}
\providecommand{\vecx}{\vec{x}}

% --- Lagrangian ---
\newcommand{\Lag}{\mathcal{L}}
\newcommand{\calL}{\mathcal{L}}

% --- Greek Letters and Constants ---
\newcommand{\alphaem}{\alpha}
\newcommand{\betaT}{\beta_T}
\newcommand{\xiT}{\xi}
\newcommand{\xipar}{\xi}

% --- Energy and Planck Units ---
\newcommand{\Ezero}{E_0}
\newcommand{\E}{E}
\newcommand{\EPlanck}{E_{\text{Pl}}}
\newcommand{\Mpl}{M_{\text{Pl}}}
\newcommand{\mP}{m_{\text{P}}}
\newcommand{\lP}{\ell_{\text{P}}}
\newcommand{\tP}{t_{\text{P}}}
\newcommand{\LPlanck}{\ell_{\text{Pl}}}
\newcommand{\TPlanck}{t_{\text{Pl}}}

% --- Coupling Constants ---
\newcommand{\Gnat}{G_{\text{nat}}}
\newcommand{\alphaEM}{\alpha_{\text{EM}}}
\newcommand{\alphaSI}{\alpha_{\text{SI}}}
\newcommand{\Hubble}{H_0}
\newcommand{\LCDM}{\Lambda\text{CDM}}
\newcommand{\natunits}{(nat. units)}

% --- T0 Model Parameters ---
\newcommand{\xigeom}{\xi_{\mathrm{geom}}}
\newcommand{\rzero}{r_{0}}
\newcommand{\xirat}{\xi_{\mathrm{rat}}}
\newcommand{\tzero}{t_{0}}
\newcommand{\Lambdat}{\Lambda_{\mathrm{t}}}
\newcommand{\EP}{E_{\text{P}}}
\newcommand{\Emu}{E_{\mu}}
\newcommand{\Ee}{E_{e}}
\newcommand{\Etau}{E_{\tau}}
\newcommand{\alphafine}{\alpha_{\mathrm{fine}}}
\newcommand{\alphal}{\alpha_{\ell}}
\newcommand{\Lzero}{\ell_{0}}
\newcommand{\Lp}{\ell_{\mathrm{P}}}

% --- Additional T0 Commands ---
\newcommand{\Kfrak}{K_{\text{frak}}}
\newcommand{\Dfrak}{D_{\text{frak}}}
\newcommand{\betapar}{\ensuremath{\beta_T}}
\newcommand{\alphapar}{\alpha}
\newcommand{\deltafield}{\delta \phi}
\newcommand{\deltam}{\delta m}
\newcommand{\deltaE}{\delta E}
\newcommand{\Exi}{E_{\xi}}
\newcommand{\Lxi}{\ell_{\xi}}
\newcommand{\rhoCMB}{\rho_{\text{CMB}}}
\newcommand{\rhoCasimir}{\rho_{\text{Casimir}}}
\newcommand{\Leff}{L_{\text{eff}}}
\newcommand{\CQCD}{C_{\mathrm{QCD}}}
\newcommand{\Kspec}{K_{\mathrm{spec}}}
\newcommand{\Tzero}{\ensuremath{T_0}}
\newcommand{\Eabs}{E_{\text{abs}}}
\newcommand{\taupar}{\tau}

% --- Provided Commands ---
\providecommand{\xiconst}{\xi_{\text{const}}}
\providecommand{\DhiggsT}{D_{\text{Higgs-T}}}
\providecommand{\rhoE}{\rho_{E}}
\providecommand{\Echar}{E_{\text{char}}}
\providecommand{\kfrac}{k_{\text{frac}}}
\providecommand{\alphaEMSI}{\alpha_{\text{EM,SI}}}
\providecommand{\alphaEMnat}{\alpha_{\text{EM,nat}}}
\providecommand{\betaTSI}{\beta_{T,\text{SI}}}
\providecommand{\betaTnat}{\beta_{T,\text{nat}}}
\providecommand{\Gsi}{G_{\text{SI}}}
\providecommand{\xiparSI}{\xi_{\text{SI}}}
\providecommand{\xiparnat}{\xi_{\text{nat}}}
\providecommand{\meff}{m_{\text{eff}}}
\providecommand{\Tzerot}{T_{0}(t)}
\providecommand{\mzerot}{m_{0}(t)}
\providecommand{\Ezeroabs}{E_{0,\text{abs}}}
\providecommand{\Epar}{E_{\text{par}}}
\providecommand{\Lnat}{\ell_{\text{nat}}}
\providecommand{\Tnat}{T_{\text{nat}}}
\providecommand{\xifrak}{\xi_{\text{frac}}}
\providecommand{\Tfrak}{T_{\text{frac}}}
\providecommand{\mfrak}{m_{\text{frac}}}
\providecommand{\Dfrac}{D_{\text{frac}}}
\providecommand{\EphotSI}{E_{\gamma,\text{SI}}}
\providecommand{\EphotNat}{E_{\gamma,\text{nat}}}
\providecommand{\Eabsint}{E_{\text{abs,int}}}
\providecommand{\mphoton}{m_{\gamma}}
\providecommand{\Evis}{E_{\text{vis}}}
\providecommand{\Cto}{C_{T0}}
\providecommand{\mytimes}{\times}
\providecommand{\lambdah}{\lambda_h}
\providecommand{\checkmarkx}{\checkmark}
\providecommand{\Enorm}{E_{\text{norm}}}
\providecommand{\Tobs}{T_{\text{obs}}}
\providecommand{\mobs}{m_{\text{obs}}}
\providecommand{\Eobs}{E_{\text{obs}}}
\providecommand{\Lobs}{\ell_{\text{obs}}}
\providecommand{\xobs}{\xi_{\text{obs}}}
\providecommand{\calE}{\mathcal{E}}
\providecommand{\calT}{\mathcal{T}}
\providecommand{\calM}{\mathcal{M}}
\providecommand{\alphag}{\alpha_g}
\providecommand{\Tmax}{T_{\text{max}}}
\providecommand{\mmin}{m_{\text{min}}}
\providecommand{\Lmax}{\ell_{\text{max}}}
\providecommand{\Emin}{E_{\text{min}}}
\providecommand{\Geff}{G_{\text{eff}}}
\providecommand{\rhoeff}{\rho_{\text{eff}}}
\providecommand{\xieff}{\xi_{\text{eff}}}
\providecommand{\Teff}{T_{\text{eff}}}
\providecommand{\hPlanck}{h}
\providecommand{\kB}{k_B}
\providecommand{\muB}{\mu_B}
\providecommand{\lambdaC}{\lambda_C}
\providecommand{\omegaP}{\omega_P}
\providecommand{\rhoP}{\rho_P}
\providecommand{\Tref}{T_{\text{ref}}}
\providecommand{\Eref}{E_{\text{ref}}}
\providecommand{\mref}{m_{\text{ref}}}
\providecommand{\Lref}{\ell_{\text{ref}}}
\providecommand{\xikonst}{\xi_0}
\providecommand{\Phiphoton}{\Phi_{\gamma}}
\providecommand{\etavis}{\eta_{\text{vis}}}
\providecommand{\pichar}{\pi}
\providecommand{\primrel}{\mathcal{P}_{\text{rel}}}
\providecommand{\warningx}{\textcolor{orange}{\textbf{!}}}
\providecommand{\phiT}{\phi_T}
\providecommand{\Lorentz}{\Lambda}
\providecommand{\Cconv}{C_{\text{conv}}}
\providecommand{\Df}{\Delta f}
\providecommand{\lambdazero}{\lambda_0}
\providecommand{\myapprox}{\approx}
\providecommand{\checked}{\checkmark}
\providecommand{\alphaWSI}{\alpha_W^{\text{SI}}}
\providecommand{\alphaWnat}{\alpha_W^{\text{nat}}}
\providecommand{\vect}[1]{\vec{#1}}
\providecommand{\Rzero}{R_0}
\providecommand{\Riem}{\mathcal{R}}
\providecommand{\nuzero}{\nu_0}
\providecommand{\mypi}{\pi}

% =============================================================================
% TCOLORBOX STYLES AND ENVIRONMENTS (English titles)
% =============================================================================
\tcbset{
	keyresult/.style={
		colback=blue!5!white,
		colframe=blue!75!black,
		title=Key Result,
		fonttitle=\bfseries
	},
	foundation/.style={
		colback=green!5!white,
		colframe=green!75!black,
		title=Foundation,
		fonttitle=\bfseries
	},
	alternative/.style={
		colback=orange!5!white,
		colframe=orange!75!black,
		title=Alternative,
		fonttitle=\bfseries
	},
	warningbox/.style={
		colback=red!5!white,
		colframe=red!75!black,
		title=Warning,
		fonttitle=\bfseries
	}
}

% (Here follow all your tcolorbox definitions with English titles)
\newtcolorbox{keyresultbox}[1][]{colback=blue!5!white,colframe=blue!75!black,fonttitle=\bfseries,title={#1},breakable}
\newtcolorbox{keyresult}[1][Key Result]{colback=blue!5!white,colframe=blue!75!black,fonttitle=\bfseries,title={#1},breakable}
\newtcolorbox{foundationbox}[1][]{colback=green!5!white,colframe=green!75!black,fonttitle=\bfseries,title={#1},breakable}
\newtcolorbox{foundation}[1][Foundation]{colback=green!5!white,colframe=green!75!black,fonttitle=\bfseries,title={#1},breakable}
\newtcolorbox{alternativebox}[1][]{colback=orange!5!white,colframe=orange!75!black,fonttitle=\bfseries,title={#1},breakable}
\newtcolorbox{warningboxenv}[1][Warning]{colback=red!5!white,colframe=red!75!black,fonttitle=\bfseries,title={#1},breakable}

\newtcolorbox{fundamental}[1][]{
	colback=boxgray,
	colframe=t0blue,
	fonttitle=\bfseries,
	title=#1,
	sharp corners,
	boxrule=2pt
}

\newtcolorbox{insightBox}[1][Insight]{colback=blue!5,colframe=t0blue,title={#1},fonttitle=\bfseries,breakable}
\newtcolorbox{discoveryBox}[1][Discovery]{colback=green!5,colframe=t0green,title={#1},fonttitle=\bfseries,breakable}
\newtcolorbox{revelation}[1][Revelation]{colback=red!5,colframe=t0red,title={#1},fonttitle=\bfseries,breakable}
\newtcolorbox{keypoint}[1][Key Point]{colback=blue!5,colframe=t0blue,title={#1},fonttitle=\bfseries,breakable}
\newtcolorbox{evidence}[1][Evidence]{colback=green!5,colframe=t0green,title={#1},fonttitle=\bfseries,breakable}
\newtcolorbox{conclusionBox}[1][Conclusion]{colback=gray!5,colframe=gray,title={#1},fonttitle=\bfseries,breakable}
\newtcolorbox{significance}[1][Significance]{colback=yellow!5,colframe=orange,title={#1},fonttitle=\bfseries,breakable}
\newtcolorbox{philosophical}[1][Philosophical]{colback=purple!5,colframe=purple,title={#1},fonttitle=\bfseries,breakable}
\newtcolorbox{implicationBox}[1][Implication]{colback=cyan!5,colframe=cyan,title={#1},fonttitle=\bfseries,breakable}
\newtcolorbox{perspectiveBox}[1][Perspective]{colback=blue!5,colframe=t0blue,title={#1},fonttitle=\bfseries,breakable}
\newtcolorbox{revolutionary}[1][Revolutionary]{colback=red!5,colframe=t0red,title={#1},fonttitle=\bfseries,breakable}

\newtcolorbox{technical}[1][Technical]{colback=gray!5,colframe=gray!75!black,title={#1},fonttitle=\bfseries,breakable}
\newtcolorbox{technicalBox}[1][Technical]{colback=gray!5,colframe=gray!75!black,title={#1},fonttitle=\bfseries,breakable}
\newtcolorbox{notationBox}[1][Notation]{colback=yellow!5,colframe=yellow!75!black,title={#1},fonttitle=\bfseries,breakable}
\newtcolorbox{verification}[1][Verification]{colback=orange!5!white,colframe=orange!75!black,fonttitle=\bfseries,title=#1}
\newtcolorbox{explanationBox}[1][Explanation]{colback=purple!5!white,colframe=purple!75!black,fonttitle=\bfseries,title=#1}
\newtcolorbox{interpretationBox}[1][Interpretation]{colback=cyan!5!white,colframe=cyan!75!black,fonttitle=\bfseries,title=#1}
\newtcolorbox{explanation}[1][Explanation]{colback=purple!5!white,colframe=purple!75!black,fonttitle=\bfseries,title=#1,breakable}
\newtcolorbox{interpretation}[1][Interpretation]{colback=cyan!5!white,colframe=cyan!75!black,fonttitle=\bfseries,title=#1,breakable}
\newtcolorbox{proof_step}[1][Proof Step]{colback=gray!5!white,colframe=gray!75!black,fonttitle=\bfseries,title=#1,breakable}
\newtcolorbox{experimental}[1][Experimental]{colback=teal!5!white,colframe=teal!75!black,fonttitle=\bfseries,title=#1,breakable}

\newtcolorbox{important}[1][Important]{colback=red!5!white,colframe=red!75!black,title={#1},fonttitle=\bfseries,breakable}
\newtcolorbox{warning}[1][Warning]{colback=orange!5!white,colframe=orange!75!black,title={#1},fonttitle=\bfseries,breakable}
\newtcolorbox{caution}[1][Caution]{colback=yellow!5!white,colframe=yellow!75!black,title={#1},fonttitle=\bfseries,breakable}
\newtcolorbox{highlight}[1][Highlight]{colback=yellow!10!white,colframe=yellow!75!black,title={#1},fonttitle=\bfseries,breakable}
\newtcolorbox{critical}[1][Critical]{colback=red!10!white,colframe=red!75!black,title={#1},fonttitle=\bfseries,breakable}

\newtcolorbox{analysis}[1][Analysis]{colback=blue!5!white,colframe=blue!75!black,title={#1},fonttitle=\bfseries,breakable}
\newtcolorbox{application}[1][Application]{colback=green!5!white,colframe=green!75!black,title={#1},fonttitle=\bfseries,breakable}
\newtcolorbox{experiment}[1][Experiment]{colback=cyan!5!white,colframe=cyan!75!black,title={#1},fonttitle=\bfseries,breakable}
\newtcolorbox{historical}[1][Historical]{colback=brown!5!white,colframe=brown!75!black,title={#1},fonttitle=\bfseries,breakable}
\newtcolorbox{numerical}[1][Numerical]{colback=gray!5!white,colframe=gray!75!black,title={#1},fonttitle=\bfseries,breakable}
\newtcolorbox{overview}[1][Overview]{colback=blue!5!white,colframe=blue!75!black,title={#1},fonttitle=\bfseries,breakable}
\newtcolorbox{speculation}[1][Speculation]{colback=purple!5!white,colframe=purple!75!black,title={#1},fonttitle=\bfseries,breakable}
\newtcolorbox{question}[1][Question]{colback=orange!5!white,colframe=orange!75!black,title={#1},fonttitle=\bfseries,breakable}
\newtcolorbox{method}[1][Method]{colback=teal!5!white,colframe=teal!75!black,title={#1},fonttitle=\bfseries,breakable}
\newtcolorbox{correct}[1][Correct]{colback=green!10!white,colframe=green!75!black,title={#1},fonttitle=\bfseries,breakable}
\newtcolorbox{units}[1][Units]{colback=gray!5!white,colframe=gray!75!black,title={#1},fonttitle=\bfseries,breakable}
\newtcolorbox{achievement}[1][Achievement]{colback=gold!5!white,colframe=orange!75!black,title={#1},fonttitle=\bfseries,breakable}
\newtcolorbox{equivalence}[1][Equivalence]{colback=cyan!5!white,colframe=cyan!75!black,title={#1},fonttitle=\bfseries,breakable}
\newtcolorbox{dimensional}[1][Dimensional Analysis]{colback=purple!5!white,colframe=purple!75!black,title={#1},fonttitle=\bfseries,breakable}

% === ADDITIONAL SIMPLE ENVIRONMENTS ===
\newenvironment{treatise}{\begin{quote}}{\end{quote}}
\newenvironment{gemeinsam}{\begin{quote}}{\end{quote}}
\newenvironment{vergleich}{\begin{quote}}{\end{quote}}
\newenvironment{vorteil}{\begin{quote}}{\end{quote}}
\newenvironment{common}{\begin{quote}}{\end{quote}}
\newenvironment{comparison}{\begin{quote}}{\end{quote}}
\newenvironment{advantage}{\begin{quote}}{\end{quote}}
\newenvironment{quantum}{\begin{quote}}{\end{quote}}

% === LAYOUT SETTINGS ===
\raggedbottom
\usepackage{environ}
\let\oldtabular\tabular
\let\endoldtabular\endtabular

\newenvironment{scaledtable}[1][0.85]{%
	\begingroup\footnotesize\setlength{\LTleft}{0pt}\setlength{\LTright}{0pt}%
}{%
	\endgroup%
}

\newcommand{\widetable}[1]{\resizebox{\textwidth}{!}{#1}}

% === TABLE OF CONTENTS FORMATTING ===
\renewcommand{\cftsecfont}{\color{blue}}
\renewcommand{\cftsubsecfont}{\color{blue}}
\renewcommand{\cftsecpagefont}{\color{blue}}
\renewcommand{\cftsubsecpagefont}{\color{blue}}
\renewcommand{\cfttoctitlefont}{\huge\bfseries\color{blue}}

% === DEFAULT HEADER AND FOOTER ===
\pagestyle{fancy}
\fancyhf{}
\fancyhead[L]{\textsc{T0 Theory}}
\fancyhead[R]{\textsc{J. Pascher}}
\fancyfoot[C]{\thepage}

% ==============================================================================
% End of Shared Preamble for English
% ==============================================================================

\begin{document}

\title{Tabellen aus 077_E-mc2_En.tex}
\maketitle

\tableofcontents
\newpage

\section*{Tabelle 1 aus 077_E-mc2_En.tex}
\begin{adjustbox}{max width=\textwidth}
\begin{tabular}{|p{6cm}|p{6cm}|}
			\hline
			\textbf{Practical Measurement Convention} & \textbf{Fundamental Natural Law} \\
			\hline
			For technical applications and everyday experiments, fixing \( c = 299\,792\,458 \ \text{m/s} \) is sensible and useful. & On the most fundamental level, \( c \) is not an absolute natural constant but a dynamic ratio \( L/T \) that follows the time-mass duality \( T \cdot m = 1 \). \\
			\hline
			Corresponds to selecting a stable reference system for our world of experience. & Corresponds to the intrinsic structure of reality before any human conventions. \\
			\hline
			Necessary for building reproducible technology and conducting comparable experiments. & Necessary for understanding the ultimate principles behind the phenomena. \\
			\hline
			Works perfectly for 99.9\% of all current applications. & Shows what remains when all practical conventions are removed. \\
			\hline
		\end{tabular}
\end{adjustbox}

\bigskip
\clearpage

\section*{Tabelle 2 aus 077_E-mc2_En.tex}
\begin{adjustbox}{max width=\textwidth}
\begin{tabular}{|p{3cm}|p{5cm}|p{5cm}|p{5cm}|}
				\hline
				\textbf{Stage} & \textbf{Practical Form} & \textbf{Fundamental Insight} & \textbf{Historical Context} \\
				\hline
				Newtonian Physics & \( F = m \cdot a \) (with absolute time and space) & Approximation for \( v \ll c \) & Industrial Age: Machines, mechanics, predictable motion \\
				\hline
				Einstein (Special Relativity) & \( E = mc^2 \) (with \( c = \text{const.} \)) & Energy-mass equivalence & Early 20th Century: Electromagnetism, early atomic physics \\
				\hline
				Einstein (General Relativity) & \( G_{\mu\nu} = \frac{8\pi G}{c^4} T_{\mu\nu} \) (10 field equations) & Geometry as gravity & Age of astronomy, cosmology \\
				\hline
				Quantum Mechanics & \( i\hbar \frac{\partial}{\partial t} \psi = H \psi \) & Quantization of energy & Atomic and nuclear age \\
				\hline
				T0 Theory & \( E = m \) in natural units (with \( T \cdot m = 1 \)) & Time-mass duality as fundamental & Information Age: Search for unified principles \\
				\hline
		\end{tabular}
\end{adjustbox}

\bigskip
\clearpage

\section*{Tabelle 3 aus 077_E-mc2_En.tex}
\begin{adjustbox}{max width=\textwidth}
\begin{tabular}{|l|c|c|}
			\hline
			\textbf{Phenomenon} & \textbf{Einstein view} & \textbf{Time-constant view} \\
			\hline
			Gravitation & Time slows down & Light slows down \\
			Velocity & Time dilation & c-variation \\
			GPS correction & "Clocks run differently" & "c is different" \\
			Measurements & Same numbers & Same numbers \\
			\hline
		\end{tabular}
\end{adjustbox}

\bigskip
\clearpage

\section*{Tabelle 4 aus 077_E-mc2_En.tex}
\begin{adjustbox}{max width=\textwidth}
\begin{tabular}{|l|c|c|c|}
			\hline
			\textbf{System} & \textbf{Reference Point} & \textbf{Complexity} & \textbf{Elegance} \\
			\hline
			Geocentric & Earth & Epicycles & Low \\
			Heliocentric & Sun & Ellipses & High \\
			Einstein & c-constant & Relativity theory & Medium \\
			T0 & $\Tfield \cdot m = 1$ & $E = m$ & Maximum \\
			\hline
		\end{tabular}
\end{adjustbox}

\bigskip
\clearpage

\section*{Tabelle 5 aus 077_E-mc2_En.tex}
\begin{adjustbox}{max width=\textwidth}
\begin{tabular}{|l|c|c|}
			\hline
			\textbf{Experiment} & \textbf{Einstein (c constant)} & \textbf{T0 (c variable)} \\
			\hline
			Gravitational field & $c = 299792458$ m/s & $c(1 \pm 10^{-15})$ \\
			High energy & $c = $ constant & $c(1 + 10^{-16})$ \\
			\hline
		\end{tabular}
\end{adjustbox}

\bigskip
\clearpage

\section*{Tabelle 6 aus 077_E-mc2_En.tex}
\begin{adjustbox}{max width=\textwidth}
\begin{tcolorbox}[colback=red!5!white,colframe=red!75!black,title=The Central Recognition]
		\textbf{E=mc² and E=m are mathematically identical!}
		
		The only difference: Einstein treats c as a "constant," although c is a dynamic ratio.
		
		\textbf{Einstein's error}: c = 299,792,458 m/s = constant
		
		\textbf{T0 truth}: c = L/T = variable ratio
	\end{tcolorbox}
	
	\subsection{The Mathematical Identity}
	
	\textbf{In natural units}:
	\begin{equation}
		E = mc^2 = m \times c^2 = m \times 1^2 = m
	\end{equation}
	
	\textbf{This is not an approximation - this is exactly the same equation!}
	
	\subsection{What is c really?}
	
	\begin{equation}
		c = \frac{\text{Length}}{\text{Time}} = \frac{L}{T}
	\end{equation}
	
	\textbf{c is a ratio, not a natural constant!}
	
	\section{The Convention Choice: The Constant-Setting of c}
	
	\subsection{The Act of Constant-Setting}
	
	Einstein set: $c = 299,792,458$ m/s = \textbf{constant}
	
	\textbf{What does this mean?}
	\begin{equation}
		c = \frac{L}{T} = \text{constant} \quad \Rightarrow \quad \frac{L}{T} = \text{fixed}
	\end{equation}
	
	\textbf{Implication}: If L and T can vary, their \textbf{ratio} must remain constant.
	
	\subsection{The Problem of Time Variability}
	
	\textbf{Einstein recognized himself}: Time dilates!
	\begin{equation}
		t' = \gamma t \quad \text{(time is variable)}
	\end{equation}
	
	\textbf{But simultaneously he claimed}: 
	\begin{equation}
		c = \frac{L}{T} = \text{constant}
	\end{equation}
	
	\textbf{This is a logical contradiction!}
	
	\subsection{The T0 Resolution}
	
	\textbf{T0 insight}: $\Tfield \cdot m = 1$
	
	This means:
	\begin{itemize}
		\item Time $\Tfield$ \textbf{must} be variable (coupled to mass)
		\item Therefore $c = L/T$ \textbf{cannot} be constant
		\item $c$ is a \textbf{dynamic ratio}, not a constant
	\end{itemize}
	
	\section{The Constants Illusion: How it Works}
	
	\subsection{The Mechanism of the Illusion}
	
	\textbf{Step 1}: Einstein sets c = constant
	\begin{equation}
		c = 299,792,458 \text{ m/s} = \text{fixed}
	\end{equation}
	
	\textbf{Step 2}: Time becomes "frozen" by this
	\begin{equation}
		T = \frac{L}{c} = \frac{L}{\text{constant}} = \text{apparently determined}
	\end{equation}
	
	\textbf{Step 3}: Time dilation becomes "mysterious effect"
	\begin{equation}
		t' = \gamma t \quad \text{(why? $\rightarrow$ complicated relativity theory)}
	\end{equation}
	
	\subsection{What Really Happens (T0 View)}
	
	\textbf{Reality}: Time is naturally variable through $\Tfield \cdot m = 1$
	
	\textbf{Einstein's constant-setting} "freezes" this natural variability artificially
	
	\textbf{Result}: One needs complicated theory to repair the "frozen" dynamics
	
	\section{c as Ratio vs. c as Constant}
	
	\subsection{c as Natural Ratio (T0)}
	
	\begin{equation}
		c(x,t) = \frac{L(x,t)}{T(x,t)}
	\end{equation}
	
	\textbf{Properties}:
	\begin{itemize}
		\item $c$ varies with location and time
		\item $c$ follows the time-mass duality
		\item No artificial constants
		\item Natural simplicity: $E = m$
	\end{itemize}
	
	\subsection{c as Artificial Constant (Einstein)}
	
	\begin{equation}
		c = 299,792,458 \text{ m/s} = \text{constant everywhere}
	\end{equation}
	
	\textbf{Problems}:
	\begin{itemize}
		\item Contradiction to time dilation
		\item Artificial "freezing" of time dynamics
		\item Complicated repair mathematics needed
		\item Inflated formula: $E = mc^2$
	\end{itemize}
	
	\section{The Time Dilation Paradox}
	
	\subsection{Einstein's Contradiction Exposed}
	
	\textbf{Einstein claims simultaneously}:
	\begin{align}
		c &= \text{constant} \\
		t' &= \gamma t \quad \text{(time varies)}
	\end{align}
	
	\textbf{But}:
	\begin{equation}
		c = \frac{L}{T} \quad \text{and} \quad T \text{ varies} \quad \Rightarrow \quad c \text{ cannot be constant!}
	\end{equation}
	
	\subsection{Einstein's Hidden Solution}
	
	Einstein "solves" the contradiction through:
	\begin{itemize}
		\item Complicated Lorentz transformations
		\item Mathematical formalisms
		\item Space-time constructions
		\item \textbf{But the logical contradiction remains!}
	\end{itemize}
	
	\subsection{T0's Natural Solution}
	
	\textbf{No contradiction in T0}:
	\begin{equation}
		\Tfield \cdot m = 1 \quad \Rightarrow \quad \text{time is naturally variable}
	\end{equation}
	
	\begin{equation}
		c = \frac{L}{T} \quad \Rightarrow \quad \text{c is naturally variable}
	\end{equation}
	
	\textbf{No constant-setting $\rightarrow$ No contradictions $\rightarrow$ No complicated repair mathematics}
	
	\section{The Mathematical Demonstration}
	
	\subsection{From E=mc² to E=m}
	
	\textbf{Starting equation}: $E = mc^2$
	
	\textbf{c in natural units}: $c = 1$
	
	\textbf{Substitution}:
	\begin{equation}
		E = mc^2 = m \times 1^2 = m
	\end{equation}
	
	\textbf{Result}: $E = m$
	
	\subsection{The Reverse Direction: From E=m to E=mc²}
	
	\textbf{Starting equation}: $E = m$
	
	\textbf{Artificial constant introduction}: $c = 299,792,458$ m/s
	
	\textbf{Inflating the equation}:
	\begin{equation}
		E = m = m \times 1 = m \times \frac{c^2}{c^2} = m \times c^2 \times \frac{1}{c^2}
	\end{equation}
	
	\textbf{If one defines $c^2$ as "conversion factor"}:
	\begin{equation}
		E = mc^2
	\end{equation}
	
	\textbf{This shows}: $E = mc^2$ is only $E = m$ with \textbf{artificial inflation factor} $c^2$!
	
	\section{The Practical Justification of \( E = mc^2 \) in Our Experiential Realm}
	
	\subsection{\( E = mc^2 \) and \( E = m \) – Same Content in Different Unit Systems}
	
	\begin{tcolorbox}[colback=blue!10!white,colframe=blue!75!black,title=The Pragmatic Perspective: Unit Systems and Conventions]
		The fundamental insight that must be clearly recognized: The equation \( E = mc^2 \) is mathematically equivalent to \( E = m \) when one chooses suitable units.
		
		\textbf{Einstein formulated in SI units (practical for our world):}
		\begin{equation}
			E = m \cdot (299\,792\,458)^2 \ \text{J} 
		\end{equation}
		
		\textbf{T0 formulates in natural units (fundamentally simpler):}
		\begin{equation}
			E = m \quad \text{with} \quad c = 1
		\end{equation}
		
		Both descriptions contain exactly the same physical information – they merely use different measuring rods.
		
		The choice between them is not a question of "right" or "wrong," but of \textit{practical suitability versus fundamental simplicity}.
	\end{tcolorbox}
	
	\subsection{Why the Fixation \( c = \text{const.} \) Is Practically Reasonable}
	
	\textbf{For our everyday experiential world}, the establishment of \( c \) as a constant is not only historically understandable but also \textit{pragmatically justified} from multiple perspectives:
	
	\begin{enumerate}[leftmargin=*, label=\textbf{\arabic*.}]
		\item \textbf{Measurement Practice:} All our measuring instruments (clocks, rulers, electronic devices) utilize physical processes that themselves depend on \( c \). Establishing a fixed \( c \) creates a consistent reference framework for reproducible experiments. Without such a convention, every measurement would require a circular self-calibration.
		
		\item \textbf{Technological Applications:} From GPS navigation to particle accelerator technology, practical applications are based on the assumption of a locally constant \( c \). This assumption works with extremely high precision for the range in which we live and work. The error introduced by this assumption is far below the resolution of our current technology for most applications.
		
		\item \textbf{Scientific Communication:} A uniform convention allows scientists worldwide to compare results and communicate with each other. The SI units with fixed \( c \) provide a practical basis for this. Science requires shared languages, and measurement conventions form an essential part of this language.
		
		\item \textbf{Historical Development:} The establishment of \( c \) as constant did not occur arbitrarily but emerged from centuries of measurement attempts (Roemer, Fizeau, Michelson-Morley) that within their accuracy showed no variation. Einstein built on this empirical foundation.
		
		\item \textbf{Educational Transmission:} Complex theories require entry points. \( E = mc^2 \) with constant \( c \) provides such an entry point into relativistic thinking. The deeper insight \( E = m \) can follow as a second step for those seeking the fundamental structure.
	\end{enumerate}
	
	\subsection{The Crucial Distinction: Practical Convention vs. Fundamental Natural Law}
	
	\textbf{T0 Theory makes a crucial distinction that resolves apparent contradictions:}
	
	\begin{table}[H]
		\centering
		\begin{tabular}{|p{6cm}|p{6cm}|}
			\hline
			\textbf{Practical Measurement Convention} & \textbf{Fundamental Natural Law} \\
			\hline
			For technical applications and everyday experiments, fixing \( c = 299\,792\,458 \ \text{m/s} \) is sensible and useful. & On the most fundamental level, \( c \) is not an absolute natural constant but a dynamic ratio \( L/T \) that follows the time-mass duality \( T \cdot m = 1 \). \\
			\hline
			Corresponds to selecting a stable reference system for our world of experience. & Corresponds to the intrinsic structure of reality before any human conventions. \\
			\hline
			Necessary for building reproducible technology and conducting comparable experiments. & Necessary for understanding the ultimate principles behind the phenomena. \\
			\hline
			Works perfectly for 99.9\% of all current applications. & Shows what remains when all practical conventions are removed. \\
			\hline
		\end{tabular}
		\caption{The dual nature of physical descriptions}
	\end{table}
	
	\subsection{Einstein's Historical Achievement in New Light}
	
	\begin{tcolorbox}[colback=green!10!white,colframe=green!75!black,title=Einstein's Pragmatic Genius Reinterpreted]
		Einstein did not merely discover the energy-mass equivalence \( E = m \), but formulated it in the \textit{units practical for his time} as \( E = mc^2 \).
		
		\textbf{His historical achievement from the T0 perspective consists of three levels:}
		
		\begin{enumerate}
			\item \textbf{Discovery of the fundamental relationship:} Recognizing that energy and mass are different manifestations of the same reality.
			
			\item \textbf{Pragmatic formulation:} Expressing this insight in a form usable for experiments and verifiable by contemporary measurements.
			
			\item \textbf{Conceptual revolution:} Drawing the consequences for our space-time conception and thereby overcoming Newtonian absolutism.
		\end{enumerate}
		
		The T0 Theory does not diminish this achievement but shows that behind the practical form \( E = mc^2 \) lies an even more fundamental simplicity \( E = m \). Einstein stopped one step before the ultimate simplicity – but this step was necessary for his time.
		
		\textbf{Historical irony:} Einstein actually discovered \( E = m \) but packaged it in the form \( E = mc^2 \) because this corresponded to the measurement practices of his time. The physics community then celebrated the packaging and overlooked the simpler content.
	\end{tcolorbox}
\end{adjustbox}

\bigskip
\clearpage

\section*{Tabelle 7 aus 077_E-mc2_En.tex}
\begin{adjustbox}{max width=\textwidth}
\begin{tcolorbox}[colback=yellow!10!white,colframe=yellow!75!black,title=The Peaceful Revolution]
		The insight that \( E = mc^2 = E = m \) does not require us to discard existing physics books or redesign technical systems. It only requires us to recognize that we have been working with a particularly practical form of a fundamentally simpler truth.
		
		\textbf{The revolution is conceptual, not practical.}
	\end{tcolorbox}
	
	\subsection{The Actual Concern of T0 Theory}
	
	\begin{tcolorbox}[colback=orange!10!white,colframe=orange!75!black,title=The True Concern of T0 Theory]
		T0 does not want to abolish \( E = mc^2 \) as a practical equation but to show:
		
		\textbf{That behind the practical form lies a fundamental simplicity that has been obscured by the historical choice of units.}
		
		This insight does not free us from the necessity of practical conventions but opens a deeper understanding of what these conventions actually describe.
		
		\textbf{The goal:} Not to make physics more complicated, but to recognize its inherent simplicity – and then consciously choose which level of description is appropriate for which purpose.
	\end{tcolorbox}
	
	\subsection{The Double Perspective: Practical Engineering vs. Fundamental Science}
	
	\textbf{The beauty of the T0 insight lies in the fact that it allows a double perspective:}
	
	\begin{figure}[H]
		\centering
		\begin{tikzpicture}[node distance=1.5cm]
			\node (practical) [rectangle, draw=blue, thick, fill=blue!5, minimum width=6cm, minimum height=2.5cm, align=center] {\textbf{Practical Engineering Perspective}\\[0.2cm]$E = mc^2$ with $c = 299,792,458$ m/s\\[0.2cm]Works perfectly for technology};
			
			\node (fundamental) [rectangle, draw=red, thick, fill=red!5, minimum width=6cm, minimum height=2.5cm, align=center, below of=practical, yshift=-4.5cm] {\textbf{Fundamental Science Perspective}\\[0.2cm]$E = m$ with $T \cdot m = 1$\\[0.2cm]Shows ultimate simplicity};
			
			\draw[<->, thick] (practical.south) -- (fundamental.north);
			\node at (0,-4.8) [align=center] {\textbf{Connection:}\\[0.1cm]$E = mc^2 = E = m$\\[0.1cm]Same reality, different descriptions};
		\end{tikzpicture}
		\caption{The double perspective of T0 Theory}
	\end{figure}
	
	\textbf{Both perspectives are valid and useful – for different purposes.} The engineer needs the practical perspective to build reliable technology. The theoretical physicist seeks the fundamental perspective to understand ultimate principles. T0 shows that these are not contradictory but complementary views of the same reality.
	
	\subsection{Conclusion: Why This Section Matters}
	
	This extensive section was necessary to clarify a common misunderstanding: When T0 shows that \( E = mc^2 = E = m \), this is not an attack on Einstein or a claim that all previous physics is "wrong." Rather, it is:
	
	\begin{enumerate}
		\item A recognition that Einstein's formulation was \textit{pragmatically optimal} for his time and still is for most applications.
		
		\item A discovery that behind this practical formulation lies a \textit{fundamentally simpler structure}.
		
		\item An invitation to consciously distinguish between \textit{practical conventions} and \textit{fundamental laws}.
		
		\item A proposal for a peaceful coexistence of both levels of description – each valuable in its own domain.
	\end{enumerate}
	
	The following sections now build on this clarified foundation, showing the consequences when we take the fundamental perspective seriously.
	
	\section{The Arbitrariness of Constant Choice: c or Time?}
	
	\subsection{Einstein's Arbitrary Decision}
	
	\begin{tcolorbox}[colback=orange!5!white,colframe=orange!75!black,title=The Fundamental Choice Option]
		\textbf{One can choose what should be "constant"!}
		
		\textbf{Option 1 (Einstein's choice)}: c = constant $\rightarrow$ time becomes variable
		
		\textbf{Option 2 (alternative)}: time = constant $\rightarrow$ c becomes variable
		
		\textbf{Both describe the same physics!}
	\end{tcolorbox}
	
	\subsection{Option 1: Einstein's c-constant}
	
	\textbf{Einstein chose}:
	\begin{align}
		c &= 299,792,458 \text{ m/s} = \text{constant (defined)} \\
		t' &= \gamma t \quad \text{(time becomes automatically variable)}
	\end{align}
	
	\textbf{Language convention}:
	\begin{itemize}
		\item "Speed of light is universally constant"
		\item "Time dilates in strong gravitational fields"
		\item "Clocks run slower at high velocities"
	\end{itemize}
	
	\subsection{Option 2: Time-constant (Einstein could have chosen)}
	
	\textbf{Alternative choice}:
	\begin{align}
		t &= \text{constant (defined)} \\
		c(x,t) &= \frac{L(x,t)}{t} = \text{variable}
	\end{align}
	
	\textbf{Alternative language convention}:
	\begin{itemize}
		\item "Time flows equally everywhere"
		\item "Speed of light varies with location"
		\item "Light becomes slower in strong gravitational fields"
	\end{itemize}
	
	\subsection{Mathematical Equivalence of Both Options}
	
	\textbf{Both descriptions are mathematically identical}:
	
	\begin{table}[htbp]
		\centering
		\begin{tabular}{|l|c|c|}
			\hline
			\textbf{Phenomenon} & \textbf{Einstein view} & \textbf{Time-constant view} \\
			\hline
			Gravitation & Time slows down & Light slows down \\
			Velocity & Time dilation & c-variation \\
			GPS correction & "Clocks run differently" & "c is different" \\
			Measurements & Same numbers & Same numbers \\
			\hline
		\end{tabular}
		\caption{Two views, identical physics}
	\end{table}
	
	\subsection{Why Einstein Chose Option 1}
	
	\textbf{Historical reasons for Einstein's decision}:
	\begin{itemize}
		\item \textbf{Michelson-Morley}: c seemed locally constant
		\item \textbf{Aesthetics}: "Universal constant" sounded elegant
		\item \textbf{Tradition}: Newtonian constant physics
		\item \textbf{Conceivability}: c-constancy easier to imagine than time constancy
		\item \textbf{Authority effect}: Einstein's prestige fixed this choice
	\end{itemize}
	
	\textbf{But it was only a convention, not a natural law!}
	
	\subsection{T0's Overcoming of Both Options}
	
	\textbf{T0 shows}: Both choices are arbitrary!
	
	\begin{equation}
		\Tfield \cdot m = 1 \quad \text{(natural duality without constant constraint)}
	\end{equation}
	
	\textbf{T0 insight}:
	\begin{itemize}
		\item \textbf{Neither} c nor time are "really" constant
		\item \textbf{Both} are aspects of the same T·m dynamics
		\item \textbf{Constancy} is only definition convention
		\item \textbf{E = m} is the constant-free truth
	\end{itemize}
	
	\subsection{Liberation from Constant Constraint}
	
	\textbf{Instead of choosing between}:
	\begin{itemize}
		\item c constant, time variable (Einstein)
		\item Time constant, c variable (alternative)
	\end{itemize}
	
	\textbf{T0 chooses}:
	\begin{itemize}
		\item \textbf{Both dynamically coupled} via T·m = 1
		\item \textbf{No arbitrary fixations}
		\item \textbf{Natural ratios} instead of artificial constants
	\end{itemize}
	
	\section{The Reference Point Revolution: Earth $\rightarrow$ Sun $\rightarrow$ Nature}
	
	\subsection{The Reference Point Analogy: Geocentric $\rightarrow$ Heliocentric $\rightarrow$ T0}
	
	\begin{tcolorbox}[colback=blue!5!white,colframe=blue!75!black,title=The Reference Point Revolution: From Earth $\rightarrow$ Sun $\rightarrow$ Nature]
		\textbf{Geocentric (Ptolemy)}: Earth at center \\
		- Complicated epicycles needed \\
		- Works, but artificially complicated \\
		
		\textbf{Heliocentric (Copernicus)}: Sun at center \\
		- Simple ellipses \\
		- Much more elegant and simple \\
		
		\textbf{T0-centric}: Natural ratios at center \\
		- $\Tfield \cdot m = 1$ (natural reference point) \\
		- Even more elegant: $E = m$
	\end{tcolorbox}
\end{adjustbox}

\bigskip
\clearpage

\section*{Tabelle 8 aus 077_E-mc2_En.tex}
\begin{adjustbox}{max width=\textwidth}
\begin{tcolorbox}[colback=red!5!white,colframe=red!75!black,title=The Reference Point Illusion]
		\textbf{Something only becomes "constant" when we define a reference point!}
		
		\textbf{Without reference point}: All ratios are relative and dynamic
		
		\textbf{With reference point}: One ratio becomes artificially "fixed"
		
		\textbf{Einstein's error}: He defined an absolute reference point for c
	\end{tcolorbox}
	
	\subsection{The Natural Stage: Everything is Relative}
	
	\textbf{Before any reference point definition}:
	\begin{align}
		c_1 &= \frac{L_1}{T_1} \\
		c_2 &= \frac{L_2}{T_2} \\
		c_3 &= \frac{L_3}{T_3} \\
		&\vdots
	\end{align}
	
	\textbf{All c-values are relative to each other}. None is "constant".
	
	\subsection{The Moment of Reference Point Setting}
	
	\textbf{Einstein's fatal step}:
	\begin{equation}
		\text{"I define: } c = 299,792,458 \text{ m/s = reference point"}
	\end{equation}
	
	\textbf{What happens at this moment}:
	\begin{itemize}
		\item An \textbf{arbitrary reference point} is set
		\item All other c-values are measured relative to this
		\item The \textbf{dynamic ratio} becomes a "constant"
		\item The \textbf{natural relativity} is artificially "frozen"
	\end{itemize}
	
	\subsection{The Reference Point Problematic}
	
	\textbf{Every reference point is arbitrary}:
	\begin{itemize}
		\item Why 299,792,458 m/s and not 300,000,000 m/s?
		\item Why in m/s and not in other units?
		\item Why measured on Earth and not in space?
		\item Why at this time and not at another?
	\end{itemize}
	
	\subsection{T0's Reference Point-Free Physics}
	
	\textbf{T0 eliminates all reference points}:
	\begin{equation}
		\Tfield \cdot m = 1 \quad \text{(universal relation without reference point)}
	\end{equation}
	
	\begin{itemize}
		\item No arbitrary fixations
		\item All ratios remain dynamic
		\item Natural relativity is preserved
		\item Fundamental simplicity: $E = m$
	\end{itemize}
	
	\subsection{Example: The Meter Definition}
	
	\textbf{Historical development of meter definition}:
	\begin{enumerate}
		\item \textbf{1793}: 1 meter = 1/10,000,000 of Earth meridian (Earth reference point)
		\item \textbf{1889}: 1 meter = prototype meter in Paris (object reference point)  
		\item \textbf{1960}: 1 meter = 1,650,763.73 wavelengths of krypton-86 (atom reference point)
		\item \textbf{1983}: 1 meter = distance light travels in 1/299,792,458 s (c reference point)
	\end{enumerate}
	
	\textbf{What does this show?}
	\begin{itemize}
		\item Each definition is \textbf{human arbitrariness}
		\item The \textbf{reference point} changes with human technology
		\item There is \textbf{no "natural" length unit} - only human agreements
		\item \textbf{Humans make c "constant" by definition} - not nature!
	\end{itemize}
	
	\subsection{The Circular Error: Humans Define Their Own "Constants"}
	
	\textbf{In 1983 humans defined}:
	\begin{equation}
		1 \text{ meter} = \frac{1}{299,792,458} \times c \times 1 \text{ second}
	\end{equation}
	
	\textbf{This makes c automatically "constant"} - through human definition, not through natural law:
	\begin{equation}
		c = \frac{299,792,458 \text{ meters}}{1 \text{ second}} = 299,792,458 \text{ m/s}
	\end{equation}
	
	\textbf{Circular reasoning}: Humans define c as constant and then "measure" a constant!
	
	\textbf{Nature is not asked in this process!}
	
	\subsection{T0's Resolution of the Reference Point Illusion}
	
	\textbf{T0 recognizes}:
	\begin{itemize}
		\item \textbf{Definition $\neq$ natural law}
		\item \textbf{Measurement reference point $\neq$ physical constant}
		\item \textbf{Practical agreement $\neq$ fundamental truth}
	\end{itemize}
	
	\textbf{T0 solution}:
	\begin{align}
		\text{For measurements:} \quad &\text{Use practical reference points} \\
		\text{For natural laws:} \quad &\text{Use reference point-free relations}
	\end{align}
	
	\section{Why c-Constancy is Not Provable}
	
	\subsection{The Fundamental Measurement Problem}
	
	\textbf{To measure c, we need}:
	\begin{equation}
		c = \frac{L}{T}
	\end{equation}
	
	\textbf{But}: We measure L and T with \textbf{the same physical processes} that depend on c!
	
	\textbf{Circular problem}:
	\begin{itemize}
		\item Light measures distances $\rightarrow$ c determines L
		\item Atomic clocks use EM transitions $\rightarrow$ c influences T
		\item Then we measure c = L/T $\rightarrow$ \textbf{We measure c with c!}
	\end{itemize}
	
	\subsection{The Gauge Definition Problem}
	
	\textbf{Since 1983}: 1 meter = distance light travels in 1/299,792,458 s
	
	\begin{equation}
		c = 299,792,458 \text{ m/s} \quad \text{(not measured, but defined!)}
	\end{equation}
	
	\textbf{One cannot "prove" what one has defined!}
	
	\subsection{The Systematic Compensation Problem}
	
	\textbf{If c varies, ALL measuring devices vary equally}:
	\begin{itemize}
		\item \textbf{Laser interferometers}: use light (c-dependent)
		\item \textbf{Atomic clocks}: use EM transitions (c-dependent)
		\item \textbf{Electronics}: uses EM signals (c-dependent)
	\end{itemize}
	
	\textbf{Result}: All devices \textbf{automatically compensate} the c-variation!
	
	\subsection{The Burden of Proof Problem}
	
	\textbf{Scientifically correct}:
	\begin{itemize}
		\item One \textbf{cannot prove} that something is constant
		\item One can only show that it \textbf{appears constant within measurement precision}
		\item \textbf{Each new precision level} could show variation
	\end{itemize}
	
	\textbf{Einstein's "c-constancy" was belief, not proof!}
	
	\subsection{T0 Prediction for Precise Measurements}
	
	\textbf{T0 predicts}: At highest precision one will find:
	\begin{equation}
		c(x,t) = c_0 \left(1 + \xipar \times \frac{\Tfield(x,t) - \Tfield_0}{\Tfield_0}\right)
	\end{equation}
	
	with $\xipar = 1.33 \times 10^{-4}$ (T0 parameter)
	
	\textbf{c varies tiny ($\sim$10$^{-15}$), but measurable in principle!}
	
	\section{Ontological Consideration: Calculations as Constructs}
	
	\subsection{The Fundamental Epistemological Limit}
	
	\begin{tcolorbox}[colback=purple!5!white,colframe=purple!75!black,title=Ontological Truth]
		\textbf{All calculations are human constructs!}
		
		They can \textbf{at best} give a certain idea of reality.
		
		\textbf{That calculations are internally consistent proves little} about actual reality.
		
		\textbf{Mathematical consistency $\neq$ ontological truth}
	\end{tcolorbox}
	
	\subsection{Einstein's Construct vs. T0's Construct}
	
	\textbf{Both are human thought structures}:
	
	\textbf{Einstein's construct}:
	\begin{itemize}
		\item E = mc² (mathematically consistent)
		\item Relativity theory (internally coherent)
		\item 10 field equations (work computationally)
		\item \textbf{But}: Based on arbitrary c-constant setting
	\end{itemize}
	
	\textbf{T0's construct}:
	\begin{itemize}
		\item E = m (mathematically simpler)
		\item T·m = 1 (internally coherent)
		\item $\partial^2 E = 0$ (works computationally)
		\item \textbf{But}: Also only a human thought model
	\end{itemize}
	
	\subsection{The Ontological Relativity}
	
	\textbf{What is "really" real?}
	\begin{itemize}
		\item \textbf{Einstein's space-time}? (construct)
		\item \textbf{T0's energy field}? (construct)
		\item \textbf{Newton's absolute time}? (construct)
		\item \textbf{Quantum mechanics' probabilities}? (construct)
	\end{itemize}
	
	\textbf{All are human interpretive frameworks of the inaccessible reality!}
	
	\subsection{Why T0 is Still "Better"}
	
	\textbf{Not because of "absolute truth," but because of}:
	
	\textbf{1. Simplicity (Occam's Razor)}:
	\begin{itemize}
		\item E = m is simpler than E = mc²
		\item One equation is simpler than 10 equations
		\item Fewer arbitrary assumptions
	\end{itemize}
	
	\textbf{2. Consistency}:
	\begin{itemize}
		\item No logical contradictions (like Einstein's)
		\item No constant arbitrariness
		\item Unified thought structure
	\end{itemize}
	
	\textbf{3. Predictive power}:
	\begin{itemize}
		\item Testable predictions
		\item Fewer free parameters
		\item Clearer experimental distinction
	\end{itemize}
	
	\textbf{4. Aesthetics}:
	\begin{itemize}
		\item Mathematical elegance
		\item Conceptual clarity
		\item Unity
	\end{itemize}
	
	\subsection{The Epistemological Humility}
	
	\textbf{T0 does NOT claim to be "absolute truth."}
	
	\textbf{T0 only says}:
	\begin{itemize}
		\item "Here is a \textbf{simpler} construct"
		\item "With \textbf{fewer} arbitrary assumptions"
		\item "That is \textbf{more consistent} than Einstein's construct"
		\item "And makes \textbf{more testable} predictions"
	\end{itemize}
	
	\textbf{But ultimately T0 also remains a human thought structure!}
	
	\subsection{The Pragmatic Consequence}
	
	\textbf{Since all theories are constructs}:
	
	\textbf{Evaluation criteria are}:
	\begin{enumerate}
		\item \textbf{Simplicity} (fewer assumptions)
		\item \textbf{Consistency} (no contradictions)
		\item \textbf{Predictive power} (testable consequences)
		\item \textbf{Elegance} (aesthetic criteria)
		\item \textbf{Unity} (fewer separate domains)
	\end{enumerate}
	
	\textbf{By all these criteria T0 is "better" than Einstein - but not "absolutely true".}
	
	\subsection{The Ontological Humility}
	
	\textbf{The deepest insight}:
	\begin{itemize}
		\item \textbf{Reality itself} is inaccessible
		\item \textbf{All theories} are human constructs
		\item \textbf{Mathematical consistency} proves no ontological truth
		\item \textbf{The best} we have: \textbf{Simpler, more consistent constructs}
	\end{itemize}
	
	\textbf{The c-constant setting was a convention decision, associated with the c-constant setting, but also the claim to absolute truth of his mathematical constructs.}
	
	\textbf{T0's advantage is not absolute truth, but relative superiority as a thought model.}
	
	\section{The Practical Consequences}
	
	\subsection{Why E=mc² "Works"}
	
	\textbf{E=mc² works because}:
	\begin{itemize}
		\item It is mathematically identical to $E = m$
		\item $c^2$ compensates the "frozen" time dynamics
		\item The T0 truth is unconsciously contained
		\item Local approximations usually suffice
	\end{itemize}
	
	\subsection{When E=mc² Fails}
	
	\textbf{The constants illusion breaks down at}:
	\begin{itemize}
		\item Very precise measurements
		\item Extreme conditions (high energies/masses)
		\item Cosmological scales
		\item Quantum gravity
	\end{itemize}
	
	\subsection{T0's Universal Validity}
	
	\textbf{E = m is valid everywhere and always}:
	\begin{itemize}
		\item No approximations needed
		\item No constant assumptions
		\item Universal applicability
		\item Fundamental simplicity
	\end{itemize}
	
	\section{The Correction of Physics History}
	
	\subsection{Einstein's True Achievement}
	
	\textbf{Einstein's actual discovery was}:
	\begin{equation}
		E = m \quad \text{(in natural form)}
	\end{equation}
	
	\textbf{His error was}:
	\begin{equation}
		E = mc^2 \quad \text{(with artificial constant inflation)}
	\end{equation}
	
	\subsection{The Historical Irony}
	
	\begin{tcolorbox}[colback=blue!5!white,colframe=blue!75!black,title=The Great Irony]
		Einstein discovered the fundamental simplicity $E = m$, 
		
		but \textbf{hid it behind the constants illusion} $E = mc^2$!
		
		The physics world celebrated the complicated form and overlooked the simple truth.
	\end{tcolorbox}
	
	\section{The T0 Perspective: c as Living Ratio}
	
	\subsection{c as Expression of Time-Mass Duality}
	
	\textbf{In T0 theory}:
	\begin{equation}
		c(x,t) = f\left(\frac{L(x,t)}{\Tfield(x,t)}\right) = f\left(\frac{L(x,t) \cdot m(x,t)}{1}\right)
	\end{equation}
	
	since $\Tfield \cdot m = 1$.
	
	\textbf{c becomes an expression of the fundamental time-mass duality!}
	
	\subsection{The Dynamic Speed of Light}
	
	\textbf{T0 prediction}: 
	\begin{equation}
		c(x,t) = c_0 \sqrt{1 + \xipar \frac{m(x,t) - m_0}{m_0}}
	\end{equation}
	
	\textbf{Light moves faster in more massive regions!}
	
	(Tiny effect, but measurable in principle)
	
	\section{Experimental Tests of c-Variability}
	
	\subsection{Proposed Experiments}
	
	\textbf{Test 1 - Gravitational dependence}:
	\begin{itemize}
		\item Measure c in different gravitational fields
		\item T0 prediction: $c$ varies with $\sim \xipar \times \Delta\Phi_{\text{grav}}$
	\end{itemize}
	

	\textbf{Test 2 - High-energy physics}:
	\begin{itemize}
		\item Measure c in particle accelerators at highest energies
		\item T0 prediction: Tiny deviations at $E \sim$ TeV
	\end{itemize}
	
	\subsection{Expected Results}
	
	\begin{table}[htbp]
		\centering
		\begin{tabular}{|l|c|c|}
			\hline
			\textbf{Experiment} & \textbf{Einstein (c constant)} & \textbf{T0 (c variable)} \\
			\hline
			Gravitational field & $c = 299792458$ m/s & $c(1 \pm 10^{-15})$ \\
			High energy & $c = $ constant & $c(1 + 10^{-16})$ \\
			\hline
		\end{tabular}
		\caption{Predicted c-variations}
	\end{table}
	
	\section{Conclusions}
	
	\subsection{The Central Recognition}
	
	\begin{tcolorbox}[colback=green!5!white,colframe=green!75!black,title=The Fundamental Truth]
		\textbf{E=mc² = E=m}
		
		Einstein's "constant" c is in truth a variable ratio.
		
		The constant-setting was a convention decision.
		
		T0 offers an alternative perspective by returning to natural variability.
	\end{tcolorbox}
\end{adjustbox}

\bigskip
\clearpage

\end{document}