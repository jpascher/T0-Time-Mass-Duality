\documentclass[12pt,a4paper]{book}
% ==============================================================================
% T0 Theory: Shared ENGLISH Preamble – Optimized for eBook/Book
% Version: 2.0 – Final 2026 (LuaLaTeX only) – ENGLISH corrected
% Author: Johann Pascher
% Date: January 2026
% ==============================================================================
%
% IMPORTANT: Compile EXCLUSIVELY with LuaLaTeX!
% In TeXstudio: Options → Configure TeXstudio → Build → Default Compiler → LuaLaTeX
%
% Required Fonts (install once):
% - Inter: https://fonts.google.com/specimen/Inter
% - JetBrains Mono: https://www.jetbrains.com/lp/mono/
% - Libertinus Math: https://github.com/libertinus-fonts/libertinus
% ==============================================================================

% === CHAPTER 1: BASIC PACKAGES (must come FIRST) ===
\RequirePackage{fontspec}
\RequirePackage{unicode-math}
\usepackage{chngcntr}
\setcounter{secnumdepth}{1}  % Nur Sections nummerieren (nicht subsections)
\setcounter{tocdepth}{1}     % Nur Sections im TOC (nicht subsections)
\makeatletter
\@ifundefined{c@chapter}{}{\counterwithout{section}{chapter}}  % Falls Kapitel existieren
\makeatother
\counterwithout{subsection}{section}  % Löse Verknüpfung
% === CHAPTER 2: LANGUAGE (ENGLISH) ===
\usepackage[english]{babel}
\usepackage{microtype}                    % IMPORTANT for better hyphenation!

% Typography settings for better line breaking
\frenchspacing                     % Correct English spacing after punctuation
\emergencystretch=3em              % Allows more stretch for difficult lines
\tolerance=2500                    % Higher tolerance for line breaks
\hbadness=10000                    % Suppresses "underfull hbox" warnings
\hfuzz=2pt                         % Allows minimal overfull
\pretolerance=150                  % Better word breaking

% Prevent bad page breaks
\clubpenalty=10000           % No "orphans"
\widowpenalty=10000          % No "widows"
\displaywidowpenalty=10000   % Also with equations
\brokenpenalty=10000         % No broken words across pages

% Explicit hyphenation for long technical words
\hyphenation{Fun-da-men-tal Frac-tal-Ge-o-met-ric Field The-o-ry Meth-od-o-log-i-cal}
\hyphenation{Re-vi-sion-ism Quan-ti-za-tion U-ni-fi-ca-tion Ef-fec-tive}
\hyphenation{Re-nor-mal-iz-a-bil-i-ty Sin-gu-lar-i-ties Con-cil-i-a-tion}
\hyphenation{E-mer-gence Phe-nom-e-no-log-i-cal Doc-u-men-ta-tion A-nal-y-sis}
\hyphenation{Grav-i-ta-tion Quan-tum Me-chan-ics Dog-ma-tism Con-se-quent}
\hyphenation{Par-al-lel-ism Im-ple-men-ta-tion Per-tur-ba-tions}
\hyphenation{Geo-met-ric Ar-ti-fact In-com-pat-i-bil-i-ty Con-struc-tive}
\hyphenation{Frac-tal Di-men-sion-less In-ves-ti-ga-tion De-scrip-tion}
\hyphenation{In-ter-pre-ta-tion Phe-nom-e-no-log-i-cal Math-e-mat-i-cal}
\hyphenation{Phi-lo-soph-i-cal Le-git-i-ma-tion Ap-pli-ca-tion Der-i-va-tion}
\hyphenation{U-ni-fi-ca-tion As-sump-tion Con-cep-tion Ex-pec-ta-tion}
\hyphenation{Sym-me-try-ex-ten-sion O-ver-all-pic-ture Chal-lenge}
\hyphenation{In-ter-ac-tion Ma-te-ri-al Ap-proach Per-spec-tive Pro-ce-dure}

% === CHAPTER 3: FONTS (with proper ligatures) ===
\setmainfont{Inter}[
Scale=1.02,
UprightFont=*-Regular,
BoldFont=*-Bold,
ItalicFont=*-Italic,
BoldItalicFont=*-BoldItalic,
Ligatures=TeX,           % IMPORTANT for proper typography
Language=English         % Explicit language support
]
\setsansfont{Inter}[
Scale=MatchLowercase,
Ligatures=TeX,
Language=English
]
\setmonofont{JetBrains Mono}[
Scale=0.95,
Language=English
]

% Math Font (simple & stable) – MUST come AFTER language definition
% IMPORTANT: Libertinus Math for correct \underbrace display!
\setmathfont{Libertinus Math}[Scale=1.0]

% === CHAPTER 4: MATHEMATICS PACKAGES (in STRICT order!) ===
% IMPORTANT: mathtools must come BEFORE unicode-math for some commands!
\usepackage{mathtools}           % FIRST mathtools!

% Then the rest
\usepackage{amsmath, amsfonts, amsthm}

% SIUNITX MUST be loaded BEFORE physics!
\usepackage{siunitx}
\sisetup{
	locale=US,                    % ENGLISH settings for SI units!
	group-separator={,},          % Thousands separator comma
	output-decimal-marker={.},    % Decimal separator point
	per-mode=symbol,
	separate-uncertainty=true
}

% Custom SI units used in narrative and books
\DeclareSIUnit\gigalightyear{Gly}
\DeclareSIUnit\mev{MeV}

% physics – MUST be loaded AFTER siunitx and mathtools
\usepackage{physics}

% === CHAPTER 5: ADDITIONS from pdflatex best practices ===
\usepackage{colortbl}        % Colored tables (ESSENTIAL!)
\usepackage{placeins}        % Float control: \FloatBarrier
\usepackage{subcaption}      % Subfigures
\usepackage{xurl}            % Better URL line breaking
% Hyphenation for URLs in bibliography
\def\UrlBreaks{\do\/\do-}

% === CHAPTER 6: PAGE LAYOUT
% =============================================================================
% SECTION 2: Page Geometry – 6" × 9" Buchformat
% =============================================================================
\usepackage[paperwidth=6in, paperheight=9in,
top=0.9in,
bottom=1.1in,
inner=0.9in,            % Größerer Innenrand für Bindung
outer=0.6in,            % Kleinerer Außenrand → mehr Text pro Seite
bindingoffset=0.5in,    % Puffer für Bindung (Steg)
twoside]{geometry}
\setlength{\headheight}{15pt}
%\usepackage[paperwidth=8.25in, paperheight=11in,
%top=1.0in,
%bottom=1.0in,
%left=1.0in,
%right=1.0in,
%twoside=false
% === CHAPTER 7: GRAPHICS AND TABLES ===
\usepackage{graphicx}
\usepackage[table,xcdraw]{xcolor}
% T0 brand colors
\definecolor{gold}{RGB}{255,215,0}
\definecolor{blue}{rgb}{0,0,1}
\definecolor{boxgray}{RGB}{240,240,240}
\definecolor{deepblue}{RGB}{0,0,127}
\definecolor{deepgreen}{RGB}{0,127,0}
\definecolor{deepred}{RGB}{191,0,0}
\definecolor{t0blue}{RGB}{33,150,243}
\definecolor{t0green}{RGB}{76,175,80}
\definecolor{t0orange}{RGB}{255,152,0}
\definecolor{t0purple}{RGB}{156,39,176}
\definecolor{t0red}{RGB}{244,67,54}
\definecolor{t0yellow}{RGB}{255,204,0}
\usepackage{tikz}
\usetikzlibrary{arrows.meta,positioning,shapes.geometric,decorations.pathmorphing,patterns,shapes.arrows,intersections}
\usepackage{pgfplots}
\pgfplotsset{compat=1.18}
\usepackage{quantikz}
\usepackage[most]{tcolorbox}
\tcbuselibrary{breakable}

% === WICHTIG: Algorithm-Konflikt umgehen ===
% Option: algorithmic mit GROSSBUCHSTABEN
% Gemeinsame Box für Experimente
\newtcolorbox{experimentbox}[1][]{
	colback=green!5!white,
	colframe=t0green!80!black,
	fonttitle=\bfseries,
	title={{#1}},
	breakable
}

% Abstract-Fallback
\ifdefined\abstract\else
\newenvironment{abstract}{\section*{\abstractname}\itshape\small\par\bigskip}{\bigskip}
\fi

% === MAKROS SICHER NEU DEFINIEREN / ÜBERSCHREIBEN ===
% Definiere Makros OHNE doppelte Subskripte
\newcommand{\phipar}{\phi_{\mathrm{par}}}
%\newcommand{\xipar}{\xi_{\mathrm{par}}}
\newcommand{\Qphipar}{Q_{\phi_{\mathrm{par}}}}
\newcommand{\rphipar}{r_{\phi_{\mathrm{par}}}}
\newcommand{\logphipar}{\log_{\phi_{\mathrm{par}}}}
\newcommand{\CHSH}{\text{CHSH}}
\usepackage{booktabs}
\usepackage{array}
\usepackage{longtable}
\usepackage{float}
\usepackage{adjustbox}
\usepackage{rotating}
\usepackage{tabularx}
\usepackage{makecell}
\usepackage{multirow}

% === CHAPTER 8: DOCUMENT FORMATTING ===
\usepackage{fancyhdr}
\renewcommand{\headrulewidth}{0.4pt}
\renewcommand{\footrulewidth}{0.4pt}
\usepackage{tocloft}

\usepackage{enumitem}
\setlist[itemize]{leftmargin=*, topsep=2pt, partopsep=0pt, parsep=2pt, itemsep=2pt}
\setlist[enumerate]{leftmargin=*, topsep=2pt, partopsep=0pt, parsep=2pt, itemsep=2pt}
\usepackage{setspace}
\usepackage{ragged2e}
\usepackage{multicol}

% === CHAPTER 9: CODE AND ALGORITHMS ===
\usepackage{algorithm}
\usepackage{algorithmic}
\usepackage{listings}
\lstset{
	basicstyle=\ttfamily\footnotesize,
	breaklines=true,
	breakatwhitespace=true,
	columns=flexible,
	keepspaces=true,
	showstringspaces=false,
	frame=single,
	xleftmargin=0pt,
	xrightmargin=0pt,
	literate=              % For special characters in code listings
	{ä}{{\"a}}1 {ö}{{\"o}}1 {ü}{{\"u}}1 {ß}{{\ss}}1
	{Ä}{{\"A}}1 {Ö}{{\"O}}1 {Ü}{{\"U}}1
}
\usepackage{mdframed}

% === CHAPTER 10: ADDITIONAL PACKAGES ===
\usepackage{pdflscape}
\usepackage{braket}
\usepackage{cancel}
\usepackage{caption}
\captionsetup{format=plain, labelfont=bf, justification=centering}
\usepackage{csquotes}
\usepackage{gensymb}
\usepackage{textcomp}
\usepackage{textgreek}
\usepackage{upgreek}
\usepackage{url}
\usepackage{slashed}
\usepackage{bm}

% === CHAPTER 11: HYPERREF (must come SECOND TO LAST!) ===
\usepackage{hyperref}
\hypersetup{
	colorlinks=true,
	linkcolor=black,
	citecolor=black,
	urlcolor=black,
	breaklinks=true,           % IMPORTANT for special characters in URLs!
	bookmarksnumbered=true,
	unicode=true,
	pdfencoding=auto,
	pdflang=en,                % Set PDF language to English
	pdfsubject={T0 Theory - Fundamental Fractal-Geometric Field Theory}
}

% Fix for unicode-math symbols in PDF bookmarks
\pdfstringdefDisableCommands{%
	\def\xi{xi}%
	\def\alpha{alpha}%
	\def\beta{beta}%
	\def\gamma{gamma}%
	\def\delta{delta}%
	\def\Delta{Delta}%
	\def\epsilon{epsilon}%
	\def\varepsilon{epsilon}%
	\def\theta{theta}%
	\def\kappa{kappa}%
	\def\lambda{lambda}%
	\def\mu{mu}%
	\def\nu{nu}%
	\def\pi{pi}%
	\def\rho{rho}%
	\def\sigma{sigma}%
	\def\tau{tau}%
	\def\phi{phi}%
	\def\chi{chi}%
	\def\psi{psi}%
	\def\omega{omega}%
	\def\Omega{Omega}%
	\def\Lambda{Lambda}%
	\def\times{x}%
	\def\cdot{*}%
	\def\pm{+/-}%
	\def\approx{~}%
	\def\sim{~}%
	\def\equiv{=}%
	\def\ell{l}%
	\def\hbar{h}%
	\def\rightarrow{->}%
	\def\leftarrow{<-}%
	\def\Rightarrow{=>}%
	\def\Leftarrow{<=}%
	\def\propto{~}%
	\def\mitxi{xi}%
	\def\mitalpha{alpha}%
	\def\mitbeta{beta}%
	\def\mitgamma{gamma}%
	\def\mitdelta{delta}%
	\def\mitDelta{Delta}%
	\def\mitepsilon{epsilon}%
	\def\mitvarepsilon{epsilon}%
	\def\mittheta{theta}%
	\def\mitkappa{kappa}%
	\def\mitlambda{lambda}%
	\def\mitLambda{Lambda}%
	\def\mitmu{mu}%
	\def\mitnu{nu}%
	\def\mitpi{pi}%
	\def\mitrho{rho}%
	\def\mitsigma{sigma}%
	\def\mittau{tau}%
	\def\mitphi{phi}%
	\def\mitchi{chi}%
	\def\mitpsi{psi}%
	\def\mitomega{omega}%
	\def\mitOmega{Omega}%
}

% === CHAPTER 12: BOOKMARK (must come AFTER hyperref!) ===
\usepackage{bookmark}

% === CHAPTER 13: CLEVEREF (ENGLISH LABELS) ===
\usepackage[english]{cleveref}
\crefname{equation}{Equation}{Equations}
\crefname{figure}{Figure}{Figures}
\crefname{table}{Table}{Tables}
\crefname{section}{Section}{Sections}
\crefname{chapter}{Chapter}{Chapters}
\crefname{theorem}{Theorem}{Theorems}
\crefname{lemma}{Lemma}{Lemmas}
\crefname{definition}{Definition}{Definitions}
\crefname{example}{Example}{Examples}
\crefname{remark}{Remark}{Remarks}

% === CUSTOM ENVIRONMENTS ===
% Alternative interpretation environment
\newenvironment{alternative}{%
	\begin{mdframed}[linecolor=black!30,linewidth=1pt,roundcorner=4pt,backgroundcolor=black!5]%
	}{%
	\end{mdframed}%
}

% Photon/particle environment
\newenvironment{photon}{%
	\begin{mdframed}[linecolor=blue!30,linewidth=1pt,roundcorner=4pt,backgroundcolor=blue!5]%
	}{%
	\end{mdframed}%
}

% Koide formula box environment
\newenvironment{koidebox}{%
	\begin{mdframed}[linecolor=green!30,linewidth=1pt,roundcorner=4pt,backgroundcolor=green!5]%
	}{%
	\end{mdframed}%
}

% Erkenntnis/insight environment
\newenvironment{erkenntnis}{%
	\begin{mdframed}[linecolor=orange!30,linewidth=1pt,roundcorner=4pt,backgroundcolor=orange!5]%
	}{%
	\end{mdframed}%
}

% Beziehung/relationship environment
\newenvironment{beziehung}{%
	\begin{mdframed}[linecolor=purple!30,linewidth=1pt,roundcorner=4pt,backgroundcolor=purple!5]%
	}{%
	\end{mdframed}%
}

% Derivation environment
\newenvironment{derivation}{%
	\begin{mdframed}[linecolor=teal!30,linewidth=1pt,roundcorner=4pt,backgroundcolor=teal!5]%
	}{%
	\end{mdframed}%
}

% Abhandlung/treatise environment
\newenvironment{abhandlung}{%
	\begin{mdframed}[linecolor=brown!30,linewidth=1pt,roundcorner=4pt,backgroundcolor=brown!5]%
	}{%
	\end{mdframed}%
}

% Anwendung/application environment
\newenvironment{anwendung}{%
	\begin{mdframed}[linecolor=cyan!30,linewidth=1pt,roundcorner=4pt,backgroundcolor=cyan!5]%
	}{%
	\end{mdframed}%
}

% Additional common environments
\newenvironment{konsequenz}{%
	\begin{mdframed}[linecolor=red!30,linewidth=1pt,roundcorner=4pt,backgroundcolor=red!5]%
	}{%
	\end{mdframed}%
}

\newenvironment{schlussfolgerung}{%
	\begin{mdframed}[linecolor=gray!30,linewidth=1pt,roundcorner=4pt,backgroundcolor=gray!5]%
	}{%
	\end{mdframed}%
}

\newenvironment{result}{%
	\begin{mdframed}[linecolor=violet!30,linewidth=1pt,roundcorner=4pt,backgroundcolor=violet!5]%
	}{%
	\end{mdframed}%
}

% Formula environment
\newenvironment{formula}{%
	\begin{mdframed}[linecolor=yellow!30,linewidth=1pt,roundcorner=4pt,backgroundcolor=yellow!5]%
	}{%
	\end{mdframed}%
}

% Revolutionaer/revolutionary environment
\newenvironment{revolutionaer}{%
	\begin{mdframed}[linecolor=red!50,linewidth=2pt,roundcorner=4pt,backgroundcolor=red!10]%
	}{%
	\end{mdframed}%
}

% Formel environment (German version of formula)
\newenvironment{formel}{%
	\begin{mdframed}[linecolor=yellow!30,linewidth=1pt,roundcorner=4pt,backgroundcolor=yellow!5]%
	}{%
	\end{mdframed}%
}

% Prinzip/principle environment
\newenvironment{prinzip}{%
	\begin{mdframed}[linecolor=blue!50,linewidth=2pt,roundcorner=4pt,backgroundcolor=blue!10]%
	}{%
	\end{mdframed}%
}

% Experimentell/experimental environment
\newenvironment{experimentell}{%
	\begin{mdframed}[linecolor=magenta!30,linewidth=1pt,roundcorner=4pt,backgroundcolor=magenta!5]%
	}{%
	\end{mdframed}%
}

% Neutrino environment
\newenvironment{neutrino}{%
	\begin{mdframed}[linecolor=cyan!40,linewidth=1pt,roundcorner=4pt,backgroundcolor=cyan!8]%
	}{%
	\end{mdframed}%
}

% Additional missing environments
\newenvironment{schluessel}{%
	\begin{mdframed}[linecolor=yellow!50,linewidth=1pt,roundcorner=4pt,backgroundcolor=yellow!10]%
	}{%
	\end{mdframed}%
}

\newenvironment{summary}{%
	\begin{mdframed}[linecolor=gray!40,linewidth=1pt,roundcorner=4pt,backgroundcolor=gray!8]%
	}{%
	\end{mdframed}%
}

\newenvironment{category}{%
	\begin{mdframed}[linecolor=pink!40,linewidth=1pt,roundcorner=4pt,backgroundcolor=pink!8]%
	}{%
	\end{mdframed}%
}

\newenvironment{sibox}{%
	\begin{mdframed}[linecolor=lime!40,linewidth=1pt,roundcorner=4pt,backgroundcolor=lime!8]%
	}{%
	\end{mdframed}%
}

% More missing environments
\newenvironment{documentbox}{%
	\begin{mdframed}[linecolor=teal!40,linewidth=1pt,roundcorner=4pt,backgroundcolor=teal!8]%
	}{%
	\end{mdframed}%
}

\newenvironment{t0box}{%
	\begin{mdframed}[linecolor=violet!40,linewidth=1pt,roundcorner=4pt,backgroundcolor=violet!8]%
	}{%
	\end{mdframed}%
}

\newenvironment{wichtig}{%
	\begin{mdframed}[linecolor=red!50,linewidth=2pt,roundcorner=4pt,backgroundcolor=red!10]%
	\textbf{Important:} 
	}{%
	\end{mdframed}%
}

\newenvironment{smbox}{%
	\begin{mdframed}[linecolor=orange!40,linewidth=1pt,roundcorner=4pt,backgroundcolor=orange!8]%
	}{%
	\end{mdframed}%
}

\newenvironment{pvbox}{%
	\begin{mdframed}[linecolor=purple!40,linewidth=1pt,roundcorner=4pt,backgroundcolor=purple!8]%
	}{%
	\end{mdframed}%
}

\newenvironment{numerisch}{%
	\begin{mdframed}[linecolor=blue!40,linewidth=1pt,roundcorner=4pt,backgroundcolor=blue!8]%
	}{%
	\end{mdframed}%
}

% More missing environments
\newenvironment{relation}{%
	\begin{mdframed}[linecolor=green!40,linewidth=1pt,roundcorner=4pt,backgroundcolor=green!8]%
	}{%
	\end{mdframed}%
}

\newenvironment{beweis}{%
	\begin{mdframed}[linecolor=brown!40,linewidth=1pt,roundcorner=4pt,backgroundcolor=brown!8]%
	\textbf{Proof:} 
	}{%
	\end{mdframed}%
}

\newenvironment{revolution}{%
	\begin{mdframed}[linecolor=red!60,linewidth=2pt,roundcorner=4pt,backgroundcolor=red!12]%
	}{%
	\end{mdframed}%
}

\newenvironment{key}{%
	\begin{mdframed}[linecolor=yellow!50,linewidth=1pt,roundcorner=4pt,backgroundcolor=yellow!10]%
	}{%
	\end{mdframed}%
}

\newenvironment{newperspective}{%
	\begin{mdframed}[linecolor=cyan!50,linewidth=1pt,roundcorner=4pt,backgroundcolor=cyan!10]%
	}{%
	\end{mdframed}%
}

\newenvironment{literatur}{%
	\begin{mdframed}[linecolor=gray!50,linewidth=1pt,roundcorner=4pt,backgroundcolor=gray!10]%
	}{%
	\end{mdframed}%
}

\newenvironment{folgerung}{%
	\begin{mdframed}[linecolor=teal!50,linewidth=1pt,roundcorner=4pt,backgroundcolor=teal!10]%
	}{%
	\end{mdframed}%
}

\newenvironment{principle}{%
	\begin{mdframed}[linecolor=blue!60,linewidth=2pt,roundcorner=4pt,backgroundcolor=blue!12]%
	}{%
	\end{mdframed}%
}

% Additional common environments
% ==============================================================================
% FROM HERE: YOUR DEFINITIONS (unchanged)
% ==============================================================================

\setcounter{tocdepth}{3}

% === CITATION COMMANDS ===
\providecommand{\citep}[1]{\cite{#1}}
\providecommand{\citet}[1]{\cite{#1}}

% === COLORS ===
\definecolor{gold}{RGB}{255,215,0}
\definecolor{blue}{rgb}{0,0,1}
\definecolor{boxgray}{RGB}{240,240,240}
\definecolor{deepblue}{RGB}{0,0,127}
\definecolor{deepgreen}{RGB}{0,127,0}
\definecolor{deepred}{RGB}{191,0,0}
\definecolor{t0blue}{RGB}{33,150,243}
\definecolor{t0green}{RGB}{76,175,80}
\definecolor{t0orange}{RGB}{255,152,0}
\definecolor{t0purple}{RGB}{156,39,176}
\definecolor{t0red}{RGB}{244,67,54}
\definecolor{t0yellow}{RGB}{255,204,0}

% === COLUMN TYPES ===
\newcolumntype{L}[1]{>{\raggedright\arraybackslash}p{#1}}
\newcolumntype{C}[1]{>{\centering\arraybackslash}p{#1}}
\newcolumntype{R}[1]{>{\raggedleft\arraybackslash}p{#1}}

% === HYPERREF SETTINGS (updated) ===
\hypersetup{
	colorlinks=true,
	linkcolor=t0blue,
	citecolor=t0blue,
	urlcolor=t0blue,
	breaklinks=true,
	bookmarksnumbered=true,
	pdfstartview=FitH,
	pdfencoding=auto,
	pdfdisplaydoctitle=true
}

% === ENGLISH THEOREM ENVIRONMENTS ===
\theoremstyle{plain}
\newtheorem{theorem}{Theorem}[section]
\newtheorem{lemma}[theorem]{Lemma}
\newtheorem{proposition}[theorem]{Proposition}
\newtheorem{corollary}[theorem]{Corollary}

\theoremstyle{definition}
\newtheorem{definition}[theorem]{Definition}
\newtheorem{example}[theorem]{Example}
\newtheorem{insight}[theorem]{Insight}
\newtheorem{discovery}[theorem]{Discovery}

\theoremstyle{remark}
\newtheorem{remark}[theorem]{Remark}
\newtheorem{axiom}{Axiom}
%\newtheorem{principle}{Principle}  % Commented out to avoid conflicts with document-specific definitions
%\newtheorem{warning}[theorem]{Warning}

% === T0-SPECIFIC COMMANDS ===
% (Here follow all your \newcommand and \providecommand definitions)
% These remain UNCHANGED as in your original preamble
% ==============================================================================
% SECTION 14: T0-Specific Commands
% ==============================================================================

% --- Core T0 Fields ---
\newcommand{\Tfield}{T(x,t)}
\providecommand{\Tfieldt}{T(\vec{x},t)}
\newcommand{\Efield}{E(x,t)}
\newcommand{\mfield}{m(x,t)}
\providecommand{\vecx}{\vec{x}}

% --- Lagrangian ---
\newcommand{\Lag}{\mathcal{L}}
\newcommand{\calL}{\mathcal{L}}

% --- Greek Letters and Constants ---
\newcommand{\alphaem}{\alpha}
\newcommand{\betaT}{\beta_T}
\newcommand{\xiT}{\xi}
\newcommand{\xipar}{\xi}

% --- Energy and Planck Units ---
\newcommand{\Ezero}{E_0}
\newcommand{\E}{E}
\newcommand{\EPlanck}{E_{\text{Pl}}}
\newcommand{\Mpl}{M_{\text{Pl}}}
\newcommand{\mP}{m_{\text{P}}}
\newcommand{\lP}{\ell_{\text{P}}}
\newcommand{\tP}{t_{\text{P}}}
\newcommand{\LPlanck}{\ell_{\text{Pl}}}
\newcommand{\TPlanck}{t_{\text{Pl}}}

% --- Coupling Constants ---
\newcommand{\Gnat}{G_{\text{nat}}}
\newcommand{\alphaEM}{\alpha_{\text{EM}}}
\newcommand{\alphaSI}{\alpha_{\text{SI}}}
\newcommand{\Hubble}{H_0}
\newcommand{\LCDM}{\Lambda\text{CDM}}
\newcommand{\natunits}{(nat. units)}

% --- T0 Model Parameters ---
\newcommand{\xigeom}{\xi_{\mathrm{geom}}}
\newcommand{\rzero}{r_{0}}
\newcommand{\xirat}{\xi_{\mathrm{rat}}}
\newcommand{\tzero}{t_{0}}
\newcommand{\Lambdat}{\Lambda_{\mathrm{t}}}
\newcommand{\EP}{E_{\text{P}}}
\newcommand{\Emu}{E_{\mu}}
\newcommand{\Ee}{E_{e}}
\newcommand{\Etau}{E_{\tau}}
\newcommand{\alphafine}{\alpha_{\mathrm{fine}}}
\newcommand{\alphal}{\alpha_{\ell}}
\newcommand{\Lzero}{\ell_{0}}
\newcommand{\Lp}{\ell_{\mathrm{P}}}

% --- Additional T0 Commands ---
\newcommand{\Kfrak}{K_{\text{frak}}}
\newcommand{\Dfrak}{D_{\text{frak}}}
\newcommand{\betapar}{\ensuremath{\beta_T}}
\newcommand{\alphapar}{\alpha}
\newcommand{\deltafield}{\delta \phi}
\newcommand{\deltam}{\delta m}
\newcommand{\deltaE}{\delta E}
\newcommand{\Exi}{E_{\xi}}
\newcommand{\Lxi}{\ell_{\xi}}
\newcommand{\rhoCMB}{\rho_{\text{CMB}}}
\newcommand{\rhoCasimir}{\rho_{\text{Casimir}}}
\newcommand{\Leff}{L_{\text{eff}}}
\newcommand{\CQCD}{C_{\mathrm{QCD}}}
\newcommand{\Kspec}{K_{\mathrm{spec}}}
\newcommand{\Tzero}{\ensuremath{T_0}}
\newcommand{\Eabs}{E_{\text{abs}}}
\newcommand{\taupar}{\tau}

% --- Provided Commands ---
\providecommand{\xiconst}{\xi_{\text{const}}}
\providecommand{\DhiggsT}{D_{\text{Higgs-T}}}
\providecommand{\rhoE}{\rho_{E}}
\providecommand{\Echar}{E_{\text{char}}}
\providecommand{\kfrac}{k_{\text{frac}}}
\providecommand{\alphaEMSI}{\alpha_{\text{EM,SI}}}
\providecommand{\alphaEMnat}{\alpha_{\text{EM,nat}}}
\providecommand{\betaTSI}{\beta_{T,\text{SI}}}
\providecommand{\betaTnat}{\beta_{T,\text{nat}}}
\providecommand{\Gsi}{G_{\text{SI}}}
\providecommand{\xiparSI}{\xi_{\text{SI}}}
\providecommand{\xiparnat}{\xi_{\text{nat}}}
\providecommand{\meff}{m_{\text{eff}}}
\providecommand{\Tzerot}{T_{0}(t)}
\providecommand{\mzerot}{m_{0}(t)}
\providecommand{\Ezeroabs}{E_{0,\text{abs}}}
\providecommand{\Epar}{E_{\text{par}}}
\providecommand{\Lnat}{\ell_{\text{nat}}}
\providecommand{\Tnat}{T_{\text{nat}}}
\providecommand{\xifrak}{\xi_{\text{frac}}}
\providecommand{\Tfrak}{T_{\text{frac}}}
\providecommand{\mfrak}{m_{\text{frac}}}
\providecommand{\Dfrac}{D_{\text{frac}}}
\providecommand{\EphotSI}{E_{\gamma,\text{SI}}}
\providecommand{\EphotNat}{E_{\gamma,\text{nat}}}
\providecommand{\Eabsint}{E_{\text{abs,int}}}
\providecommand{\mphoton}{m_{\gamma}}
\providecommand{\Evis}{E_{\text{vis}}}
\providecommand{\Cto}{C_{T0}}
\providecommand{\mytimes}{\times}
\providecommand{\lambdah}{\lambda_h}
\providecommand{\checkmarkx}{\checkmark}
\providecommand{\Enorm}{E_{\text{norm}}}
\providecommand{\Tobs}{T_{\text{obs}}}
\providecommand{\mobs}{m_{\text{obs}}}
\providecommand{\Eobs}{E_{\text{obs}}}
\providecommand{\Lobs}{\ell_{\text{obs}}}
\providecommand{\xobs}{\xi_{\text{obs}}}
\providecommand{\calE}{\mathcal{E}}
\providecommand{\calT}{\mathcal{T}}
\providecommand{\calM}{\mathcal{M}}
\providecommand{\alphag}{\alpha_g}
\providecommand{\Tmax}{T_{\text{max}}}
\providecommand{\mmin}{m_{\text{min}}}
\providecommand{\Lmax}{\ell_{\text{max}}}
\providecommand{\Emin}{E_{\text{min}}}
\providecommand{\Geff}{G_{\text{eff}}}
\providecommand{\rhoeff}{\rho_{\text{eff}}}
\providecommand{\xieff}{\xi_{\text{eff}}}
\providecommand{\Teff}{T_{\text{eff}}}
\providecommand{\hPlanck}{h}
\providecommand{\kB}{k_B}
\providecommand{\muB}{\mu_B}
\providecommand{\lambdaC}{\lambda_C}
\providecommand{\omegaP}{\omega_P}
\providecommand{\rhoP}{\rho_P}
\providecommand{\Tref}{T_{\text{ref}}}
\providecommand{\Eref}{E_{\text{ref}}}
\providecommand{\mref}{m_{\text{ref}}}
\providecommand{\Lref}{\ell_{\text{ref}}}
\providecommand{\xikonst}{\xi_0}
\providecommand{\Phiphoton}{\Phi_{\gamma}}
\providecommand{\etavis}{\eta_{\text{vis}}}
\providecommand{\pichar}{\pi}
\providecommand{\primrel}{\mathcal{P}_{\text{rel}}}
\providecommand{\warningx}{\textcolor{orange}{\textbf{!}}}
\providecommand{\phiT}{\phi_T}
\providecommand{\Lorentz}{\Lambda}
\providecommand{\Cconv}{C_{\text{conv}}}
\providecommand{\Df}{\Delta f}
\providecommand{\lambdazero}{\lambda_0}
\providecommand{\myapprox}{\approx}
\providecommand{\checked}{\checkmark}
\providecommand{\alphaWSI}{\alpha_W^{\text{SI}}}
\providecommand{\alphaWnat}{\alpha_W^{\text{nat}}}
\providecommand{\vect}[1]{\vec{#1}}
\providecommand{\Rzero}{R_0}
\providecommand{\Riem}{\mathcal{R}}
\providecommand{\nuzero}{\nu_0}
\providecommand{\mypi}{\pi}

% =============================================================================
% TCOLORBOX STYLES AND ENVIRONMENTS (English titles)
% =============================================================================
\tcbset{
	keyresult/.style={
		colback=blue!5!white,
		colframe=blue!75!black,
		title=Key Result,
		fonttitle=\bfseries
	},
	foundation/.style={
		colback=green!5!white,
		colframe=green!75!black,
		title=Foundation,
		fonttitle=\bfseries
	},
	alternative/.style={
		colback=orange!5!white,
		colframe=orange!75!black,
		title=Alternative,
		fonttitle=\bfseries
	},
	warningbox/.style={
		colback=red!5!white,
		colframe=red!75!black,
		title=Warning,
		fonttitle=\bfseries
	}
}

% (Here follow all your tcolorbox definitions with English titles)
\newtcolorbox{keyresultbox}[1][]{colback=blue!5!white,colframe=blue!75!black,fonttitle=\bfseries,title={#1},breakable}
\newtcolorbox{keyresult}[1][Key Result]{colback=blue!5!white,colframe=blue!75!black,fonttitle=\bfseries,title={#1},breakable}
\newtcolorbox{foundationbox}[1][]{colback=green!5!white,colframe=green!75!black,fonttitle=\bfseries,title={#1},breakable}
\newtcolorbox{foundation}[1][Foundation]{colback=green!5!white,colframe=green!75!black,fonttitle=\bfseries,title={#1},breakable}
\newtcolorbox{alternativebox}[1][]{colback=orange!5!white,colframe=orange!75!black,fonttitle=\bfseries,title={#1},breakable}
\newtcolorbox{warningboxenv}[1][Warning]{colback=red!5!white,colframe=red!75!black,fonttitle=\bfseries,title={#1},breakable}

\newtcolorbox{fundamental}[1][]{
	colback=boxgray,
	colframe=t0blue,
	fonttitle=\bfseries,
	title=#1,
	sharp corners,
	boxrule=2pt
}

\newtcolorbox{insightBox}[1][Insight]{colback=blue!5,colframe=t0blue,title={#1},fonttitle=\bfseries,breakable}
\newtcolorbox{discoveryBox}[1][Discovery]{colback=green!5,colframe=t0green,title={#1},fonttitle=\bfseries,breakable}
\newtcolorbox{revelation}[1][Revelation]{colback=red!5,colframe=t0red,title={#1},fonttitle=\bfseries,breakable}
\newtcolorbox{keypoint}[1][Key Point]{colback=blue!5,colframe=t0blue,title={#1},fonttitle=\bfseries,breakable}
\newtcolorbox{evidence}[1][Evidence]{colback=green!5,colframe=t0green,title={#1},fonttitle=\bfseries,breakable}
\newtcolorbox{conclusionBox}[1][Conclusion]{colback=gray!5,colframe=gray,title={#1},fonttitle=\bfseries,breakable}
\newtcolorbox{significance}[1][Significance]{colback=yellow!5,colframe=orange,title={#1},fonttitle=\bfseries,breakable}
\newtcolorbox{philosophical}[1][Philosophical]{colback=purple!5,colframe=purple,title={#1},fonttitle=\bfseries,breakable}
\newtcolorbox{implicationBox}[1][Implication]{colback=cyan!5,colframe=cyan,title={#1},fonttitle=\bfseries,breakable}
\newtcolorbox{perspectiveBox}[1][Perspective]{colback=blue!5,colframe=t0blue,title={#1},fonttitle=\bfseries,breakable}
\newtcolorbox{revolutionary}[1][Revolutionary]{colback=red!5,colframe=t0red,title={#1},fonttitle=\bfseries,breakable}

\newtcolorbox{technical}[1][Technical]{colback=gray!5,colframe=gray!75!black,title={#1},fonttitle=\bfseries,breakable}
\newtcolorbox{technicalBox}[1][Technical]{colback=gray!5,colframe=gray!75!black,title={#1},fonttitle=\bfseries,breakable}
\newtcolorbox{notationBox}[1][Notation]{colback=yellow!5,colframe=yellow!75!black,title={#1},fonttitle=\bfseries,breakable}
\newtcolorbox{verification}[1][Verification]{colback=orange!5!white,colframe=orange!75!black,fonttitle=\bfseries,title=#1}
\newtcolorbox{explanationBox}[1][Explanation]{colback=purple!5!white,colframe=purple!75!black,fonttitle=\bfseries,title=#1}
\newtcolorbox{interpretationBox}[1][Interpretation]{colback=cyan!5!white,colframe=cyan!75!black,fonttitle=\bfseries,title=#1}
\newtcolorbox{explanation}[1][Explanation]{colback=purple!5!white,colframe=purple!75!black,fonttitle=\bfseries,title=#1,breakable}
\newtcolorbox{interpretation}[1][Interpretation]{colback=cyan!5!white,colframe=cyan!75!black,fonttitle=\bfseries,title=#1,breakable}
\newtcolorbox{proof_step}[1][Proof Step]{colback=gray!5!white,colframe=gray!75!black,fonttitle=\bfseries,title=#1,breakable}
\newtcolorbox{experimental}[1][Experimental]{colback=teal!5!white,colframe=teal!75!black,fonttitle=\bfseries,title=#1,breakable}

\newtcolorbox{important}[1][Important]{colback=red!5!white,colframe=red!75!black,title={#1},fonttitle=\bfseries,breakable}
\newtcolorbox{warning}[1][Warning]{colback=orange!5!white,colframe=orange!75!black,title={#1},fonttitle=\bfseries,breakable}
\newtcolorbox{caution}[1][Caution]{colback=yellow!5!white,colframe=yellow!75!black,title={#1},fonttitle=\bfseries,breakable}
\newtcolorbox{highlight}[1][Highlight]{colback=yellow!10!white,colframe=yellow!75!black,title={#1},fonttitle=\bfseries,breakable}
\newtcolorbox{critical}[1][Critical]{colback=red!10!white,colframe=red!75!black,title={#1},fonttitle=\bfseries,breakable}

\newtcolorbox{analysis}[1][Analysis]{colback=blue!5!white,colframe=blue!75!black,title={#1},fonttitle=\bfseries,breakable}
\newtcolorbox{application}[1][Application]{colback=green!5!white,colframe=green!75!black,title={#1},fonttitle=\bfseries,breakable}
\newtcolorbox{experiment}[1][Experiment]{colback=cyan!5!white,colframe=cyan!75!black,title={#1},fonttitle=\bfseries,breakable}
\newtcolorbox{historical}[1][Historical]{colback=brown!5!white,colframe=brown!75!black,title={#1},fonttitle=\bfseries,breakable}
\newtcolorbox{numerical}[1][Numerical]{colback=gray!5!white,colframe=gray!75!black,title={#1},fonttitle=\bfseries,breakable}
\newtcolorbox{overview}[1][Overview]{colback=blue!5!white,colframe=blue!75!black,title={#1},fonttitle=\bfseries,breakable}
\newtcolorbox{speculation}[1][Speculation]{colback=purple!5!white,colframe=purple!75!black,title={#1},fonttitle=\bfseries,breakable}
\newtcolorbox{question}[1][Question]{colback=orange!5!white,colframe=orange!75!black,title={#1},fonttitle=\bfseries,breakable}
\newtcolorbox{method}[1][Method]{colback=teal!5!white,colframe=teal!75!black,title={#1},fonttitle=\bfseries,breakable}
\newtcolorbox{correct}[1][Correct]{colback=green!10!white,colframe=green!75!black,title={#1},fonttitle=\bfseries,breakable}
\newtcolorbox{units}[1][Units]{colback=gray!5!white,colframe=gray!75!black,title={#1},fonttitle=\bfseries,breakable}
\newtcolorbox{achievement}[1][Achievement]{colback=gold!5!white,colframe=orange!75!black,title={#1},fonttitle=\bfseries,breakable}
\newtcolorbox{equivalence}[1][Equivalence]{colback=cyan!5!white,colframe=cyan!75!black,title={#1},fonttitle=\bfseries,breakable}
\newtcolorbox{dimensional}[1][Dimensional Analysis]{colback=purple!5!white,colframe=purple!75!black,title={#1},fonttitle=\bfseries,breakable}

% === ADDITIONAL SIMPLE ENVIRONMENTS ===
\newenvironment{treatise}{\begin{quote}}{\end{quote}}
\newenvironment{gemeinsam}{\begin{quote}}{\end{quote}}
\newenvironment{vergleich}{\begin{quote}}{\end{quote}}
\newenvironment{vorteil}{\begin{quote}}{\end{quote}}
\newenvironment{common}{\begin{quote}}{\end{quote}}
\newenvironment{comparison}{\begin{quote}}{\end{quote}}
\newenvironment{advantage}{\begin{quote}}{\end{quote}}
\newenvironment{quantum}{\begin{quote}}{\end{quote}}

% === LAYOUT SETTINGS ===
\raggedbottom
\usepackage{environ}
\let\oldtabular\tabular
\let\endoldtabular\endtabular

\newenvironment{scaledtable}[1][0.85]{%
	\begingroup\footnotesize\setlength{\LTleft}{0pt}\setlength{\LTright}{0pt}%
}{%
	\endgroup%
}

\newcommand{\widetable}[1]{\resizebox{\textwidth}{!}{#1}}

% === TABLE OF CONTENTS FORMATTING ===
\renewcommand{\cftsecfont}{\color{blue}}
\renewcommand{\cftsubsecfont}{\color{blue}}
\renewcommand{\cftsecpagefont}{\color{blue}}
\renewcommand{\cftsubsecpagefont}{\color{blue}}
\renewcommand{\cfttoctitlefont}{\huge\bfseries\color{blue}}

% === DEFAULT HEADER AND FOOTER ===
\pagestyle{fancy}
\fancyhf{}
\fancyhead[L]{\textsc{T0 Theory}}
\fancyhead[R]{\textsc{J. Pascher}}
\fancyfoot[C]{\thepage}

% ==============================================================================
% End of Shared Preamble for English
% ==============================================================================

\begin{document}

\title{Tabellen aus 024_T0_netze_En.tex}
\maketitle

\tableofcontents
\newpage

\section*{Tabelle 1 aus 024_T0_netze_En.tex}
\begin{adjustbox}{max width=\textwidth}
\begin{tabular}{cccc}
			\toprule
			\textbf{Dimension ($d$)} & \textbf{Geometric Factor ($G_d$)} & \textbf{Ratio to $G_3$} & \textbf{Application Example} \\
			\midrule
			1 & 1/1 = 1 & 0.75 & Linear chain models in 1D dynamics \\
			2 & 2/2 = 1 & 0.75 & Surface-based Casimir effects \\
			3 & 4/3 = 1.333... & 1.00 & Standard physical space (T0 core) \\
			4 & 8/4 = 2 & 1.50 & Kaluza-Klein-like extensions \\
			5 & 16/5 = 3.2 & 2.40 & Fractal scalings in CMB \\
			6 & 32/6 = 5.333... & 4.00 & Hexagonal networks in quantum computing \\
			10 & 512/10 = 51.2 & 38.40 & High-dimensional information spaces \\
			\bottomrule
		\end{tabular}
\end{adjustbox}

\bigskip
\clearpage

\section*{Tabelle 2 aus 024_T0_netze_En.tex}
\begin{adjustbox}{max width=\textwidth}
\begin{tabular}{cccc}
			\toprule
			\textbf{Number Range} & \textbf{Effective Dimension} & \textbf{Optimal $\xipar_{\text{res}}$} & \textbf{Comparison to RSA Security} \\
			\midrule
			$10^2$ - $10^3$ & 3-4 & 0.05 - 0.1 & Weak (fast factorization) \\
			$10^4$ - $10^6$ & 5-7 & 0.02 - 0.05 & Medium (moderately difficult) \\
			$10^8$ - $10^{12}$ & 8-12 & 0.01 - 0.02 & Strong (RSA-2048 equivalent) \\
			$10^{15}$+ & 15+ & $<0.01$ & Extreme (quantum-resistant scaling) \\
			\bottomrule
		\end{tabular}
\end{adjustbox}

\bigskip
\clearpage

\section*{Tabelle 3 aus 024_T0_netze_En.tex}
\begin{adjustbox}{max width=\textwidth}
\begin{tabular}{lp{8cm}}
			\toprule
			\textbf{Architecture} & \textbf{Advantages for T0 Implementation} \\
			\midrule
			Graph Neural Networks & Natural representation of spacetime network structure with nodes and edges, including $\xi$-weighted propagation \\
			Convolutional Networks & Efficient processing of regular grid patterns in various dimensions, ideal for fractal $D_f$ corrections \\
			Fourier Neural Operators & Handles spectral transformations required for number-field mapping, with fast convergence \\
			Recurrent Networks & Models temporal evolution of field patterns, adhering to $T \cdot E = 1$ duality over timesteps \\
			Transformers & Captures long-range correlations in field values, useful for infinite-dimensional projections \\
			\bottomrule
		\end{tabular}
\end{adjustbox}

\bigskip
\clearpage

\section*{Tabelle 4 aus 024_T0_netze_En.tex}
\begin{adjustbox}{max width=\textwidth}
\begin{tabular}{cccc}
			\toprule
			\textbf{Number Size} & \textbf{Predicted Optimal $\xipar_{\text{res}}$} & \textbf{Predicted Success Rate} & \textbf{Validation Metric} \\
			\midrule
			$10^3$ & 0.05 & 95\% & Hit rate in 100 simulations \\
			$10^6$ & 0.025 & 80\% & Convergence time in ms \\
			$10^9$ & 0.015 & 65\% & Error rate < 5\% \\
			$10^{12}$ & 0.01 & 50\% & Scalability on GPU \\
			\bottomrule
		\end{tabular}
\end{adjustbox}

\bigskip
\clearpage

\section*{Tabelle 5 aus 024_T0_netze_En.tex}
\begin{adjustbox}{max width=\textwidth}
\begin{tabular}{lp{8cm}}
			\toprule
			\textbf{Hardware Platform} & \textbf{Dimensional Implementation Approach} \\
			\midrule
			GPU Arrays & Parallel processing of multiple dimensions with tensor cores, optimized for batch factorization \\
			Quantum Processors & Natural implementation of superposition across dimensions, for exponential speedups \\
			Neuromorphic Chips & Dimension-specific neural circuits with adaptive connectivity, energy-efficient for edge computing \\
			FPGA Systems & Reconfigurable architecture for variable dimensional processing, with real-time $\xi$-adjustment \\
			\bottomrule
		\end{tabular}
\end{adjustbox}

\bigskip
\clearpage

\section*{Tabelle 6 aus 024_T0_netze_En.tex}
\begin{adjustbox}{max width=\textwidth}
\begin{tcolorbox}[colback=blue!5!white,colframe=blue!75!black,title=Dimensional Network Properties,breakable]
		In a $d$-dimensional network:
		\begin{itemize}
			\item Each node has up to $2d$ direct connections, causing connectivity to grow exponentially with dimension;
			\item The geometric factor scales as $G_d = \frac{2^{d-1}}{d}$, normalizing volume and surface measures in higher dimensions;
			\item Field propagation follows $d$-dimensional wave equations: $\partial^2 \deltafield = 0$;
			\item Boundary conditions require $d$-dimensional specification (periodic or Dirichlet-like).
		\end{itemize}
	\end{tcolorbox}
	
	These properties form the basis for dimension-adaptive adjustment, which is detailed in later sections.
	
	\section{Dimensionality and $\xi$-Parameter Variations}
	\label{sec:dimensionality_xi}
	
	\subsection{Geometric Factor Dependence on Dimension}
	\label{subsec:geometric_factor}
	
	One of the most significant discoveries in the T0 theory is the dimensional dependence of the geometric factor, which shapes the fundamental structure of the model across all scales:
	
	\begin{equation}
		G_d = \frac{2^{d-1}}{d}
	\end{equation}
	
	For our familiar 3-dimensional space, we obtain $G_3 = \frac{2^2}{3} = \frac{4}{3}$, which appears as a fundamental geometric constant in the T0 model and directly corresponds to the derivation of the fine-structure constant $\alpha$ in \cite{T0_Feinstruktur}. This formula enables a unified description of volume integrals in variable dimensions, which is particularly useful for cosmological extensions.
	
	\begin{table}[htbp]
		\centering
		\resizebox{\textwidth}{!}{
\begin{tabular}{cccc}
			\toprule
			\textbf{Dimension ($d$)} & \textbf{Geometric Factor ($G_d$)} & \textbf{Ratio to $G_3$} & \textbf{Application Example} \\
			\midrule
			1 & 1/1 = 1 & 0.75 & Linear chain models in 1D dynamics \\
			2 & 2/2 = 1 & 0.75 & Surface-based Casimir effects \\
			3 & 4/3 = 1.333... & 1.00 & Standard physical space (T0 core) \\
			4 & 8/4 = 2 & 1.50 & Kaluza-Klein-like extensions \\
			5 & 16/5 = 3.2 & 2.40 & Fractal scalings in CMB \\
			6 & 32/6 = 5.333... & 4.00 & Hexagonal networks in quantum computing \\
			10 & 512/10 = 51.2 & 38.40 & High-dimensional information spaces \\
			\bottomrule
		\end{tabular}
}
		\caption{Geometric factors for various dimensionalities, extended with application examples}
		\label{tab:geometric_factors}
	\end{table}
	
	\subsection{Dimension-Dependent $\xi$-Parameters}
	\label{subsec:dimension_dependent_xi}
	
	A crucial insight is that the $\xipar$-parameter must be adjusted for different dimensionalities to maintain the consistency of duality relations:
	
	\begin{equation}
		\xipar_d = \frac{G_d}{G_3} \cdot \xipar_3 = \frac{d \cdot 2^{d-3}}{3} \cdot \frac{4}{3} \mytimes 10^{-4}
	\end{equation}
	
	This means that different dimensional contexts require different $\xipar$-values for consistent physical behavior, bridging to the fractal corrections in \cite{T0_g2_erweiterung}, where $D_f = 3 - \xipar$ serves as a sub-dimensional variant.
	
\begin{tcolorbox}[colback=red!5!white,colframe=red!75!black,title={Critical Understanding: Multiple $\xi$-Parameters},width=\textwidth]
	It is a fundamental error to treat $\xipar$ as a single universal constant. Instead:
	\begin{itemize}
		\item $\xipar_{\text{geom}}$: The geometric parameter ($\frac{4}{3} \mytimes 10^{-4}$) in 3D space, derived from space geometry;
		\item $\xipar_{\text{res}}$: The resonance parameter ($\approx 0.1$) for factorization, modulating spectral resolutions;
		\item $\xipar_d$: Dimension-specific parameters scaling with $G_d$ and generating a hierarchy across dimensions.
	\end{itemize}
	Each parameter serves a specific mathematical purpose and scales differently with dimension, making the theory robust against dimensional variations.
\end{tcolorbox}
\end{adjustbox}

\bigskip
\clearpage

\section*{Tabelle 7 aus 024_T0_netze_En.tex}
\begin{adjustbox}{max width=\textwidth}
\begin{tcolorbox}[colback=yellow!10!white,colframe=yellow!50!black,title={Contrasting Dimensional Structures},width=\textwidth]
	\begin{itemize}
		\item \textbf{Physical Space}: 3+1 dimensions (3 spatial + 1 temporal), fixed by observation and consistent with the $\xi$-derivation from 3D geometry;
		\item \textbf{Number Space}: Potentially infinite dimensions (each prime factor represents a dimension), modulated by the Riemann hypothesis and $\zeta$-functions;
		\item \textbf{Effective Dimension}: Determined by problem complexity, not fixed, and dynamically adjustable via $\xi_{\text{res}}$.
	\end{itemize}
\end{tcolorbox}
	
	\subsection{Mathematical Transformation Between Spaces}
	\label{subsec:mathematical_transformation}
	
	The transformation between number space and physical space requires a sophisticated mathematical mapping that establishes isomorphisms between discrete and continuous structures:
	
	\begin{equation}
		\mathcal{T}: \mathbb{Z}_n \to \mathbb{R}^d, \quad \mathcal{T}(n) = \{E_i(x,t)\}
	\end{equation}
	
	This transformation maps numbers from the integer space $\mathbb{Z}_n$ to field configurations in the $d$-dimensional real space $\mathbb{R}^d$ and accounts for $\xi$-dependent rescalings to preserve invariances.
	
	\subsection{Spectral Methods for Dimensional Mapping}
	\label{subsec:spectral_methods}
	
	Spectral methods offer an elegant approach to mapping between spaces by utilizing Fourier-like decompositions to connect frequency domains:
	
	\begin{equation}
		\Psi_n(\omega, \xipar_{\text{res}}) = \sum_i A_i \times \frac{1}{\sqrt{4\pi\xipar_{\text{res}}}} \times \exp\left(-\frac{(\omega-\omega_i)^2}{4\xipar_{\text{res}}}\right)
	\end{equation}
	
	Where:
	\begin{itemize}
		\item $\Psi_n$ represents the spectral representation of the number $n$, encoding prime factors as resonances;
		\item $\omega_i$ represents the frequency associated with the prime factor $p_i$, proportional to $\log(p_i)$;
		\item $A_i$ represents the amplitude coefficient, derived from multiplicity;
		\item $\xipar_{\text{res}}$ controls the spectral resolution and determines the sharpness of the peaks.
	\end{itemize}
	
	This formulation allows efficient numerics and is compatible with quantum algorithms like Shor's.
	
	\section{Neural Network Implementation of the T0 Model}
	\label{sec:neural_network}
	
	\subsection{Optimal Network Architectures}
	\label{subsec:optimal_architectures}
	
	Neural networks offer a promising approach to implementing the T0 model, with several architectures particularly suited to handling dimension-dependent scalings:
	
	\begin{table}[htbp]
		\centering
		\begin{tabular}{lp{8cm}}
			\toprule
			\textbf{Architecture} & \textbf{Advantages for T0 Implementation} \\
			\midrule
			Graph Neural Networks & Natural representation of spacetime network structure with nodes and edges, including $\xi$-weighted propagation \\
			Convolutional Networks & Efficient processing of regular grid patterns in various dimensions, ideal for fractal $D_f$ corrections \\
			Fourier Neural Operators & Handles spectral transformations required for number-field mapping, with fast convergence \\
			Recurrent Networks & Models temporal evolution of field patterns, adhering to $T \cdot E = 1$ duality over timesteps \\
			Transformers & Captures long-range correlations in field values, useful for infinite-dimensional projections \\
			\bottomrule
		\end{tabular}
		\caption{Neural network architectures for T0 implementation, extended with specific T0 advantages}
		\label{tab:network_architectures}
	\end{table}
	
	\subsection{Dimension-Adaptive Networks}
	\label{subsec:dimension_adaptive}
	
	A key innovation for T0 implementation is dimension-adaptive networks that dynamically respond to effective dimensionality:
	
	\begin{formula}[colback=blue!5!white,colframe=blue!75!black,title=Dimension-Adaptive Network Design]
		Effective T0 networks should adapt their dimensionality based on:
		\begin{itemize}
			\item \textbf{Problem Domain}: Physical (3+1D) vs. number space (variable $D$), with automatic switching via layer dropout;
			\item \textbf{Problem Complexity}: Higher dimensions for larger factorization tasks, scaled logarithmically with $n$;
			\item \textbf{Resource Constraints}: Dimensional optimization for computational efficiency through tensor reduction;
			\item \textbf{Accuracy Requirements}: Higher dimensions for more precise results, validated by loss functions with $\xi$-penalty.
		\end{itemize}
	\end{formula}
	
	\subsection{Mathematical Formulation of Neural T0 Networks}
	\label{subsec:mathematical_neural}
	
	For Graph Neural Networks, the T0 model can be implemented as:
	
	\begin{equation}
		h_v^{(l+1)} = \sigma\left(W^{(l)} \cdot h_v^{(l)} + \sum_{u \in \mathcal{N}(v)} \alpha_{vu} \cdot M^{(l)} \cdot h_u^{(l)}\right)
	\end{equation}
	
	Where:
	\begin{itemize}
		\item $h_v^{(l)}$ is the state vector at node $v$ in layer $l$, initialized with $T(v)$ and $E(v)$;
		\item $\mathcal{N}(v)$ is the neighborhood of node $v$, extended by $\xi$-weighted distances;
		\item $W^{(l)}$ and $M^{(l)}$ are learnable weight matrices incorporating $G_d$;
		\item $\alpha_{vu}$ are attention coefficients, computed via softmax over edges;
		\item $\sigma$ is a non-linear activation function, e.g., ReLU with duality constraint.
	\end{itemize}
	
	For spectral methods with Fourier Neural Operators:
	
	\begin{equation}
		(\mathcal{K}\phi)(x) = \int_{\Omega} \kappa(x,y) \phi(y) dy \approx \mathcal{F}^{-1}(R \cdot \mathcal{F}(\phi))
	\end{equation}
	
	Where $\mathcal{F}$ is the Fourier transform, $R$ is a learnable filter, and $\phi$ is the field configuration, with $\xi_{\text{res}}$ as bandwidth parameter.
	
	\section{Dimensional Hierarchy and Scale Relations}
	\label{sec:dimensional_hierarchy}
	
	\subsection{Dimensional Scale Separation}
	\label{subsec:scale_separation}
	
	The T0 model reveals a natural dimensional hierarchy connecting scales from Planck length to cosmological horizons:
	
	\begin{equation}
		\frac{\xipar_{\text{res}}(d)}{\xipar_{\text{geom}}(d)} = \frac{d-1}{d \cdot 2^{d-3}} \cdot \frac{3 \cdot 10^1}{4 \cdot 10^{-4}} \approx \frac{d-1}{d \cdot 2^{d-3}} \cdot 7,5 \cdot 10^4
	\end{equation}
	
	This relation shows how resonance and geometric parameters scale differently with dimension, generating a natural scale separation comparable to the hierarchy in fine-structure constant derivation.
	
	\subsection{Mathematical Relation to Number Space}
	\label{subsec:zahlenraum_relation}
	
	The number space has a fundamentally different dimensional structure than physical space, shaped by infinite prime density:
	
	\begin{equation}
		\dim(\mathbb{Z}_n) = \infty \quad \text{(infinite for prime distribution)}
	\end{equation}
	
	This infinitely-dimensional structure must be projected onto finite-dimensional networks, with the effective dimension:
	
	\begin{equation}
		d_{\text{effective}} = \log_2\left(\frac{n}{\xipar_{\text{res}}}\right)
	\end{equation}
	
	This projection enables treating RSA keys as high-dimensional fields.
	
	\subsection{Information Mapping Between Dimensional Spaces}
	\label{subsec:information_mapping}
	
	The information mapping between number space and physical space can be quantified by:
	
	\begin{equation}
		\mathcal{I}(n, d) = \int \Psi_n(\omega, \xipar_{\text{res}}) \cdot \Phi_d(\omega, \xipar_{\text{geom}}) \, d\omega
	\end{equation}
	
	Where $\Psi_n$ is the spectral representation of number $n$ and $\Phi_d$ is the $d$-dimensional field configuration, with a mutual information metric for evaluating mapping fidelity.
	
	\section{Hybrid Network Models for T0 Implementation}
	\label{sec:hybrid_models}
	
	\subsection{Dual-Space Network Architecture}
	\label{subsec:dual_space}
	
	An optimal T0 implementation requires a hybrid network addressing both physical and number spaces, enabling bidirectional communication:
	
	\begin{equation}
		\mathcal{N}_{\text{hybrid}} = \mathcal{N}_{\text{phys}} \oplus \mathcal{N}_{\text{info}}
	\end{equation}
	
	Where $\mathcal{N}_{\text{phys}}$ is a 3+1D network for physical space and $\mathcal{N}_{\text{info}}$ is a network with variable dimension for information space, connected by a $\xi$-driven interface.
	
	\subsection{Implementation Strategy}
	\label{subsec:implementation_strategy}
	
\begin{tcolorbox}[colback=green!5!white,colframe=green!75!black,title={Optimal T0 Network Implementation Strategy},width=\textwidth]
	\begin{enumerate}
		\item \textbf{Base Layer}: 3D Graph Neural Network with physical time as fourth dimension, initialized with T0 scales;
		\item \textbf{Field Layer}: Node features encoding $E_{\text{field}}$ and $T_{\text{field}}$ values, adhering to duality;
		\item \textbf{Spectral Layer}: Fourier transformations for mapping between spaces, with $\xi_{\text{res}}$ as filter parameter;
		\item \textbf{Dimension Adapter}: Dynamically adjusts network dimensionality based on problem complexity, via autoencoder-like modules;
		\item \textbf{Resonance Detector}: Implements variable $\xipar_{\text{res}}$ based on number size, with feedback loops for convergence.
	\end{enumerate}
\end{tcolorbox}
\end{adjustbox}

\bigskip
\clearpage

\end{document}