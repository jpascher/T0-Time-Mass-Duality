\documentclass[12pt,a4paper]{article}

% Standardized preamble - 066_ParameterSystemdipendent_De.tex
% ==============================================================================
% T0 Theory: Shared English Preamble
% Version: 1.0
% Author: Johann Pascher
% Date: 2025
% ==============================================================================
%
% This is the standardized shared preamble for all English T0 Theory documents.
% Place this file in your document's directory or use a path like:
%   % ==============================================================================
% T0 Theory: Shared ENGLISH Preamble – Optimized for eBook/Book
% Version: 2.0 – Final 2026 (LuaLaTeX only) – ENGLISH corrected
% Author: Johann Pascher
% Date: January 2026
% ==============================================================================
%
% IMPORTANT: Compile EXCLUSIVELY with LuaLaTeX!
% In TeXstudio: Options → Configure TeXstudio → Build → Default Compiler → LuaLaTeX
%
% Required Fonts (install once):
% - Inter: https://fonts.google.com/specimen/Inter
% - JetBrains Mono: https://www.jetbrains.com/lp/mono/
% - Libertinus Math: https://github.com/libertinus-fonts/libertinus
% ==============================================================================

% === CHAPTER 1: BASIC PACKAGES (must come FIRST) ===
\RequirePackage{fontspec}
\RequirePackage{unicode-math}
\usepackage{chngcntr}
\setcounter{secnumdepth}{1}  % Nur Sections nummerieren (nicht subsections)
\setcounter{tocdepth}{1}     % Nur Sections im TOC (nicht subsections)
\makeatletter
\@ifundefined{c@chapter}{}{\counterwithout{section}{chapter}}  % Falls Kapitel existieren
\makeatother
\counterwithout{subsection}{section}  % Löse Verknüpfung
% === CHAPTER 2: LANGUAGE (ENGLISH) ===
\usepackage[english]{babel}
\usepackage{microtype}                    % IMPORTANT for better hyphenation!

% Typography settings for better line breaking
\frenchspacing                     % Correct English spacing after punctuation
\emergencystretch=3em              % Allows more stretch for difficult lines
\tolerance=2500                    % Higher tolerance for line breaks
\hbadness=10000                    % Suppresses "underfull hbox" warnings
\hfuzz=2pt                         % Allows minimal overfull
\pretolerance=150                  % Better word breaking

% Prevent bad page breaks
\clubpenalty=10000           % No "orphans"
\widowpenalty=10000          % No "widows"
\displaywidowpenalty=10000   % Also with equations
\brokenpenalty=10000         % No broken words across pages

% Explicit hyphenation for long technical words
\hyphenation{Fun-da-men-tal Frac-tal-Ge-o-met-ric Field The-o-ry Meth-od-o-log-i-cal}
\hyphenation{Re-vi-sion-ism Quan-ti-za-tion U-ni-fi-ca-tion Ef-fec-tive}
\hyphenation{Re-nor-mal-iz-a-bil-i-ty Sin-gu-lar-i-ties Con-cil-i-a-tion}
\hyphenation{E-mer-gence Phe-nom-e-no-log-i-cal Doc-u-men-ta-tion A-nal-y-sis}
\hyphenation{Grav-i-ta-tion Quan-tum Me-chan-ics Dog-ma-tism Con-se-quent}
\hyphenation{Par-al-lel-ism Im-ple-men-ta-tion Per-tur-ba-tions}
\hyphenation{Geo-met-ric Ar-ti-fact In-com-pat-i-bil-i-ty Con-struc-tive}
\hyphenation{Frac-tal Di-men-sion-less In-ves-ti-ga-tion De-scrip-tion}
\hyphenation{In-ter-pre-ta-tion Phe-nom-e-no-log-i-cal Math-e-mat-i-cal}
\hyphenation{Phi-lo-soph-i-cal Le-git-i-ma-tion Ap-pli-ca-tion Der-i-va-tion}
\hyphenation{U-ni-fi-ca-tion As-sump-tion Con-cep-tion Ex-pec-ta-tion}
\hyphenation{Sym-me-try-ex-ten-sion O-ver-all-pic-ture Chal-lenge}
\hyphenation{In-ter-ac-tion Ma-te-ri-al Ap-proach Per-spec-tive Pro-ce-dure}

% === CHAPTER 3: FONTS (with proper ligatures) ===
\setmainfont{Inter}[
Scale=1.02,
UprightFont=*-Regular,
BoldFont=*-Bold,
ItalicFont=*-Italic,
BoldItalicFont=*-BoldItalic,
Ligatures=TeX,           % IMPORTANT for proper typography
Language=English         % Explicit language support
]
\setsansfont{Inter}[
Scale=MatchLowercase,
Ligatures=TeX,
Language=English
]
\setmonofont{JetBrains Mono}[
Scale=0.95,
Language=English
]

% Math Font (simple & stable) – MUST come AFTER language definition
% IMPORTANT: Libertinus Math for correct \underbrace display!
\setmathfont{Libertinus Math}[Scale=1.0]

% === CHAPTER 4: MATHEMATICS PACKAGES (in STRICT order!) ===
% IMPORTANT: mathtools must come BEFORE unicode-math for some commands!
\usepackage{mathtools}           % FIRST mathtools!

% Then the rest
\usepackage{amsmath, amsfonts, amsthm}

% SIUNITX MUST be loaded BEFORE physics!
\usepackage{siunitx}
\sisetup{
	locale=US,                    % ENGLISH settings for SI units!
	group-separator={,},          % Thousands separator comma
	output-decimal-marker={.},    % Decimal separator point
	per-mode=symbol,
	separate-uncertainty=true
}

% Custom SI units used in narrative and books
\DeclareSIUnit\gigalightyear{Gly}
\DeclareSIUnit\mev{MeV}

% physics – MUST be loaded AFTER siunitx and mathtools
\usepackage{physics}

% === CHAPTER 5: ADDITIONS from pdflatex best practices ===
\usepackage{colortbl}        % Colored tables (ESSENTIAL!)
\usepackage{placeins}        % Float control: \FloatBarrier
\usepackage{subcaption}      % Subfigures
\usepackage{xurl}            % Better URL line breaking
% Hyphenation for URLs in bibliography
\def\UrlBreaks{\do\/\do-}

% === CHAPTER 6: PAGE LAYOUT
% =============================================================================
% SECTION 2: Page Geometry – 6" × 9" Buchformat
% =============================================================================
\usepackage[paperwidth=6in, paperheight=9in,
top=0.9in,
bottom=1.1in,
inner=0.9in,            % Größerer Innenrand für Bindung
outer=0.6in,            % Kleinerer Außenrand → mehr Text pro Seite
bindingoffset=0.5in,    % Puffer für Bindung (Steg)
twoside]{geometry}
\setlength{\headheight}{15pt}
%\usepackage[paperwidth=8.25in, paperheight=11in,
%top=1.0in,
%bottom=1.0in,
%left=1.0in,
%right=1.0in,
%twoside=false
% === CHAPTER 7: GRAPHICS AND TABLES ===
\usepackage{graphicx}
\usepackage[table,xcdraw]{xcolor}
% T0 brand colors
\definecolor{gold}{RGB}{255,215,0}
\definecolor{blue}{rgb}{0,0,1}
\definecolor{boxgray}{RGB}{240,240,240}
\definecolor{deepblue}{RGB}{0,0,127}
\definecolor{deepgreen}{RGB}{0,127,0}
\definecolor{deepred}{RGB}{191,0,0}
\definecolor{t0blue}{RGB}{33,150,243}
\definecolor{t0green}{RGB}{76,175,80}
\definecolor{t0orange}{RGB}{255,152,0}
\definecolor{t0purple}{RGB}{156,39,176}
\definecolor{t0red}{RGB}{244,67,54}
\definecolor{t0yellow}{RGB}{255,204,0}
\usepackage{tikz}
\usetikzlibrary{arrows.meta,positioning,shapes.geometric,decorations.pathmorphing,patterns,shapes.arrows,intersections}
\usepackage{pgfplots}
\pgfplotsset{compat=1.18}
\usepackage{quantikz}
\usepackage[most]{tcolorbox}
\tcbuselibrary{breakable}

% === WICHTIG: Algorithm-Konflikt umgehen ===
% Option: algorithmic mit GROSSBUCHSTABEN
% Gemeinsame Box für Experimente
\newtcolorbox{experimentbox}[1][]{
	colback=green!5!white,
	colframe=t0green!80!black,
	fonttitle=\bfseries,
	title={{#1}},
	breakable
}

% Abstract-Fallback
\ifdefined\abstract\else
\newenvironment{abstract}{\section*{\abstractname}\itshape\small\par\bigskip}{\bigskip}
\fi

% === MAKROS SICHER NEU DEFINIEREN / ÜBERSCHREIBEN ===
% Definiere Makros OHNE doppelte Subskripte
\newcommand{\phipar}{\phi_{\mathrm{par}}}
%\newcommand{\xipar}{\xi_{\mathrm{par}}}
\newcommand{\Qphipar}{Q_{\phi_{\mathrm{par}}}}
\newcommand{\rphipar}{r_{\phi_{\mathrm{par}}}}
\newcommand{\logphipar}{\log_{\phi_{\mathrm{par}}}}
\newcommand{\CHSH}{\text{CHSH}}
\usepackage{booktabs}
\usepackage{array}
\usepackage{longtable}
\usepackage{float}
\usepackage{adjustbox}
\usepackage{rotating}
\usepackage{tabularx}
\usepackage{makecell}
\usepackage{multirow}

% === CHAPTER 8: DOCUMENT FORMATTING ===
\usepackage{fancyhdr}
\renewcommand{\headrulewidth}{0.4pt}
\renewcommand{\footrulewidth}{0.4pt}
\usepackage{tocloft}

\usepackage{enumitem}
\setlist[itemize]{leftmargin=*, topsep=2pt, partopsep=0pt, parsep=2pt, itemsep=2pt}
\setlist[enumerate]{leftmargin=*, topsep=2pt, partopsep=0pt, parsep=2pt, itemsep=2pt}
\usepackage{setspace}
\usepackage{ragged2e}
\usepackage{multicol}

% === CHAPTER 9: CODE AND ALGORITHMS ===
\usepackage{algorithm}
\usepackage{algorithmic}
\usepackage{listings}
\lstset{
	basicstyle=\ttfamily\footnotesize,
	breaklines=true,
	breakatwhitespace=true,
	columns=flexible,
	keepspaces=true,
	showstringspaces=false,
	frame=single,
	xleftmargin=0pt,
	xrightmargin=0pt,
	literate=              % For special characters in code listings
	{ä}{{\"a}}1 {ö}{{\"o}}1 {ü}{{\"u}}1 {ß}{{\ss}}1
	{Ä}{{\"A}}1 {Ö}{{\"O}}1 {Ü}{{\"U}}1
}
\usepackage{mdframed}

% === CHAPTER 10: ADDITIONAL PACKAGES ===
\usepackage{pdflscape}
\usepackage{braket}
\usepackage{cancel}
\usepackage{caption}
\captionsetup{format=plain, labelfont=bf, justification=centering}
\usepackage{csquotes}
\usepackage{gensymb}
\usepackage{textcomp}
\usepackage{textgreek}
\usepackage{upgreek}
\usepackage{url}
\usepackage{slashed}
\usepackage{bm}

% === CHAPTER 11: HYPERREF (must come SECOND TO LAST!) ===
\usepackage{hyperref}
\hypersetup{
	colorlinks=true,
	linkcolor=black,
	citecolor=black,
	urlcolor=black,
	breaklinks=true,           % IMPORTANT for special characters in URLs!
	bookmarksnumbered=true,
	unicode=true,
	pdfencoding=auto,
	pdflang=en,                % Set PDF language to English
	pdfsubject={T0 Theory - Fundamental Fractal-Geometric Field Theory}
}

% Fix for unicode-math symbols in PDF bookmarks
\pdfstringdefDisableCommands{%
	\def\xi{xi}%
	\def\alpha{alpha}%
	\def\beta{beta}%
	\def\gamma{gamma}%
	\def\delta{delta}%
	\def\Delta{Delta}%
	\def\epsilon{epsilon}%
	\def\varepsilon{epsilon}%
	\def\theta{theta}%
	\def\kappa{kappa}%
	\def\lambda{lambda}%
	\def\mu{mu}%
	\def\nu{nu}%
	\def\pi{pi}%
	\def\rho{rho}%
	\def\sigma{sigma}%
	\def\tau{tau}%
	\def\phi{phi}%
	\def\chi{chi}%
	\def\psi{psi}%
	\def\omega{omega}%
	\def\Omega{Omega}%
	\def\Lambda{Lambda}%
	\def\times{x}%
	\def\cdot{*}%
	\def\pm{+/-}%
	\def\approx{~}%
	\def\sim{~}%
	\def\equiv{=}%
	\def\ell{l}%
	\def\hbar{h}%
	\def\rightarrow{->}%
	\def\leftarrow{<-}%
	\def\Rightarrow{=>}%
	\def\Leftarrow{<=}%
	\def\propto{~}%
	\def\mitxi{xi}%
	\def\mitalpha{alpha}%
	\def\mitbeta{beta}%
	\def\mitgamma{gamma}%
	\def\mitdelta{delta}%
	\def\mitDelta{Delta}%
	\def\mitepsilon{epsilon}%
	\def\mitvarepsilon{epsilon}%
	\def\mittheta{theta}%
	\def\mitkappa{kappa}%
	\def\mitlambda{lambda}%
	\def\mitLambda{Lambda}%
	\def\mitmu{mu}%
	\def\mitnu{nu}%
	\def\mitpi{pi}%
	\def\mitrho{rho}%
	\def\mitsigma{sigma}%
	\def\mittau{tau}%
	\def\mitphi{phi}%
	\def\mitchi{chi}%
	\def\mitpsi{psi}%
	\def\mitomega{omega}%
	\def\mitOmega{Omega}%
}

% === CHAPTER 12: BOOKMARK (must come AFTER hyperref!) ===
\usepackage{bookmark}

% === CHAPTER 13: CLEVEREF (ENGLISH LABELS) ===
\usepackage[english]{cleveref}
\crefname{equation}{Equation}{Equations}
\crefname{figure}{Figure}{Figures}
\crefname{table}{Table}{Tables}
\crefname{section}{Section}{Sections}
\crefname{chapter}{Chapter}{Chapters}
\crefname{theorem}{Theorem}{Theorems}
\crefname{lemma}{Lemma}{Lemmas}
\crefname{definition}{Definition}{Definitions}
\crefname{example}{Example}{Examples}
\crefname{remark}{Remark}{Remarks}

% === CUSTOM ENVIRONMENTS ===
% Alternative interpretation environment
\newenvironment{alternative}{%
	\begin{mdframed}[linecolor=black!30,linewidth=1pt,roundcorner=4pt,backgroundcolor=black!5]%
	}{%
	\end{mdframed}%
}

% Photon/particle environment
\newenvironment{photon}{%
	\begin{mdframed}[linecolor=blue!30,linewidth=1pt,roundcorner=4pt,backgroundcolor=blue!5]%
	}{%
	\end{mdframed}%
}

% Koide formula box environment
\newenvironment{koidebox}{%
	\begin{mdframed}[linecolor=green!30,linewidth=1pt,roundcorner=4pt,backgroundcolor=green!5]%
	}{%
	\end{mdframed}%
}

% Erkenntnis/insight environment
\newenvironment{erkenntnis}{%
	\begin{mdframed}[linecolor=orange!30,linewidth=1pt,roundcorner=4pt,backgroundcolor=orange!5]%
	}{%
	\end{mdframed}%
}

% Beziehung/relationship environment
\newenvironment{beziehung}{%
	\begin{mdframed}[linecolor=purple!30,linewidth=1pt,roundcorner=4pt,backgroundcolor=purple!5]%
	}{%
	\end{mdframed}%
}

% Derivation environment
\newenvironment{derivation}{%
	\begin{mdframed}[linecolor=teal!30,linewidth=1pt,roundcorner=4pt,backgroundcolor=teal!5]%
	}{%
	\end{mdframed}%
}

% Abhandlung/treatise environment
\newenvironment{abhandlung}{%
	\begin{mdframed}[linecolor=brown!30,linewidth=1pt,roundcorner=4pt,backgroundcolor=brown!5]%
	}{%
	\end{mdframed}%
}

% Anwendung/application environment
\newenvironment{anwendung}{%
	\begin{mdframed}[linecolor=cyan!30,linewidth=1pt,roundcorner=4pt,backgroundcolor=cyan!5]%
	}{%
	\end{mdframed}%
}

% Additional common environments
\newenvironment{konsequenz}{%
	\begin{mdframed}[linecolor=red!30,linewidth=1pt,roundcorner=4pt,backgroundcolor=red!5]%
	}{%
	\end{mdframed}%
}

\newenvironment{schlussfolgerung}{%
	\begin{mdframed}[linecolor=gray!30,linewidth=1pt,roundcorner=4pt,backgroundcolor=gray!5]%
	}{%
	\end{mdframed}%
}

\newenvironment{result}{%
	\begin{mdframed}[linecolor=violet!30,linewidth=1pt,roundcorner=4pt,backgroundcolor=violet!5]%
	}{%
	\end{mdframed}%
}

% Formula environment
\newenvironment{formula}{%
	\begin{mdframed}[linecolor=yellow!30,linewidth=1pt,roundcorner=4pt,backgroundcolor=yellow!5]%
	}{%
	\end{mdframed}%
}

% Revolutionaer/revolutionary environment
\newenvironment{revolutionaer}{%
	\begin{mdframed}[linecolor=red!50,linewidth=2pt,roundcorner=4pt,backgroundcolor=red!10]%
	}{%
	\end{mdframed}%
}

% Formel environment (German version of formula)
\newenvironment{formel}{%
	\begin{mdframed}[linecolor=yellow!30,linewidth=1pt,roundcorner=4pt,backgroundcolor=yellow!5]%
	}{%
	\end{mdframed}%
}

% Prinzip/principle environment
\newenvironment{prinzip}{%
	\begin{mdframed}[linecolor=blue!50,linewidth=2pt,roundcorner=4pt,backgroundcolor=blue!10]%
	}{%
	\end{mdframed}%
}

% Experimentell/experimental environment
\newenvironment{experimentell}{%
	\begin{mdframed}[linecolor=magenta!30,linewidth=1pt,roundcorner=4pt,backgroundcolor=magenta!5]%
	}{%
	\end{mdframed}%
}

% Neutrino environment
\newenvironment{neutrino}{%
	\begin{mdframed}[linecolor=cyan!40,linewidth=1pt,roundcorner=4pt,backgroundcolor=cyan!8]%
	}{%
	\end{mdframed}%
}

% Additional missing environments
\newenvironment{schluessel}{%
	\begin{mdframed}[linecolor=yellow!50,linewidth=1pt,roundcorner=4pt,backgroundcolor=yellow!10]%
	}{%
	\end{mdframed}%
}

\newenvironment{summary}{%
	\begin{mdframed}[linecolor=gray!40,linewidth=1pt,roundcorner=4pt,backgroundcolor=gray!8]%
	}{%
	\end{mdframed}%
}

\newenvironment{category}{%
	\begin{mdframed}[linecolor=pink!40,linewidth=1pt,roundcorner=4pt,backgroundcolor=pink!8]%
	}{%
	\end{mdframed}%
}

\newenvironment{sibox}{%
	\begin{mdframed}[linecolor=lime!40,linewidth=1pt,roundcorner=4pt,backgroundcolor=lime!8]%
	}{%
	\end{mdframed}%
}

% More missing environments
\newenvironment{documentbox}{%
	\begin{mdframed}[linecolor=teal!40,linewidth=1pt,roundcorner=4pt,backgroundcolor=teal!8]%
	}{%
	\end{mdframed}%
}

\newenvironment{t0box}{%
	\begin{mdframed}[linecolor=violet!40,linewidth=1pt,roundcorner=4pt,backgroundcolor=violet!8]%
	}{%
	\end{mdframed}%
}

\newenvironment{wichtig}{%
	\begin{mdframed}[linecolor=red!50,linewidth=2pt,roundcorner=4pt,backgroundcolor=red!10]%
	\textbf{Important:} 
	}{%
	\end{mdframed}%
}

\newenvironment{smbox}{%
	\begin{mdframed}[linecolor=orange!40,linewidth=1pt,roundcorner=4pt,backgroundcolor=orange!8]%
	}{%
	\end{mdframed}%
}

\newenvironment{pvbox}{%
	\begin{mdframed}[linecolor=purple!40,linewidth=1pt,roundcorner=4pt,backgroundcolor=purple!8]%
	}{%
	\end{mdframed}%
}

\newenvironment{numerisch}{%
	\begin{mdframed}[linecolor=blue!40,linewidth=1pt,roundcorner=4pt,backgroundcolor=blue!8]%
	}{%
	\end{mdframed}%
}

% More missing environments
\newenvironment{relation}{%
	\begin{mdframed}[linecolor=green!40,linewidth=1pt,roundcorner=4pt,backgroundcolor=green!8]%
	}{%
	\end{mdframed}%
}

\newenvironment{beweis}{%
	\begin{mdframed}[linecolor=brown!40,linewidth=1pt,roundcorner=4pt,backgroundcolor=brown!8]%
	\textbf{Proof:} 
	}{%
	\end{mdframed}%
}

\newenvironment{revolution}{%
	\begin{mdframed}[linecolor=red!60,linewidth=2pt,roundcorner=4pt,backgroundcolor=red!12]%
	}{%
	\end{mdframed}%
}

\newenvironment{key}{%
	\begin{mdframed}[linecolor=yellow!50,linewidth=1pt,roundcorner=4pt,backgroundcolor=yellow!10]%
	}{%
	\end{mdframed}%
}

\newenvironment{newperspective}{%
	\begin{mdframed}[linecolor=cyan!50,linewidth=1pt,roundcorner=4pt,backgroundcolor=cyan!10]%
	}{%
	\end{mdframed}%
}

\newenvironment{literatur}{%
	\begin{mdframed}[linecolor=gray!50,linewidth=1pt,roundcorner=4pt,backgroundcolor=gray!10]%
	}{%
	\end{mdframed}%
}

\newenvironment{folgerung}{%
	\begin{mdframed}[linecolor=teal!50,linewidth=1pt,roundcorner=4pt,backgroundcolor=teal!10]%
	}{%
	\end{mdframed}%
}

\newenvironment{principle}{%
	\begin{mdframed}[linecolor=blue!60,linewidth=2pt,roundcorner=4pt,backgroundcolor=blue!12]%
	}{%
	\end{mdframed}%
}

% Additional common environments
% ==============================================================================
% FROM HERE: YOUR DEFINITIONS (unchanged)
% ==============================================================================

\setcounter{tocdepth}{3}

% === CITATION COMMANDS ===
\providecommand{\citep}[1]{\cite{#1}}
\providecommand{\citet}[1]{\cite{#1}}

% === COLORS ===
\definecolor{gold}{RGB}{255,215,0}
\definecolor{blue}{rgb}{0,0,1}
\definecolor{boxgray}{RGB}{240,240,240}
\definecolor{deepblue}{RGB}{0,0,127}
\definecolor{deepgreen}{RGB}{0,127,0}
\definecolor{deepred}{RGB}{191,0,0}
\definecolor{t0blue}{RGB}{33,150,243}
\definecolor{t0green}{RGB}{76,175,80}
\definecolor{t0orange}{RGB}{255,152,0}
\definecolor{t0purple}{RGB}{156,39,176}
\definecolor{t0red}{RGB}{244,67,54}
\definecolor{t0yellow}{RGB}{255,204,0}

% === COLUMN TYPES ===
\newcolumntype{L}[1]{>{\raggedright\arraybackslash}p{#1}}
\newcolumntype{C}[1]{>{\centering\arraybackslash}p{#1}}
\newcolumntype{R}[1]{>{\raggedleft\arraybackslash}p{#1}}

% === HYPERREF SETTINGS (updated) ===
\hypersetup{
	colorlinks=true,
	linkcolor=t0blue,
	citecolor=t0blue,
	urlcolor=t0blue,
	breaklinks=true,
	bookmarksnumbered=true,
	pdfstartview=FitH,
	pdfencoding=auto,
	pdfdisplaydoctitle=true
}

% === ENGLISH THEOREM ENVIRONMENTS ===
\theoremstyle{plain}
\newtheorem{theorem}{Theorem}[section]
\newtheorem{lemma}[theorem]{Lemma}
\newtheorem{proposition}[theorem]{Proposition}
\newtheorem{corollary}[theorem]{Corollary}

\theoremstyle{definition}
\newtheorem{definition}[theorem]{Definition}
\newtheorem{example}[theorem]{Example}
\newtheorem{insight}[theorem]{Insight}
\newtheorem{discovery}[theorem]{Discovery}

\theoremstyle{remark}
\newtheorem{remark}[theorem]{Remark}
\newtheorem{axiom}{Axiom}
%\newtheorem{principle}{Principle}  % Commented out to avoid conflicts with document-specific definitions
%\newtheorem{warning}[theorem]{Warning}

% === T0-SPECIFIC COMMANDS ===
% (Here follow all your \newcommand and \providecommand definitions)
% These remain UNCHANGED as in your original preamble
% ==============================================================================
% SECTION 14: T0-Specific Commands
% ==============================================================================

% --- Core T0 Fields ---
\newcommand{\Tfield}{T(x,t)}
\providecommand{\Tfieldt}{T(\vec{x},t)}
\newcommand{\Efield}{E(x,t)}
\newcommand{\mfield}{m(x,t)}
\providecommand{\vecx}{\vec{x}}

% --- Lagrangian ---
\newcommand{\Lag}{\mathcal{L}}
\newcommand{\calL}{\mathcal{L}}

% --- Greek Letters and Constants ---
\newcommand{\alphaem}{\alpha}
\newcommand{\betaT}{\beta_T}
\newcommand{\xiT}{\xi}
\newcommand{\xipar}{\xi}

% --- Energy and Planck Units ---
\newcommand{\Ezero}{E_0}
\newcommand{\E}{E}
\newcommand{\EPlanck}{E_{\text{Pl}}}
\newcommand{\Mpl}{M_{\text{Pl}}}
\newcommand{\mP}{m_{\text{P}}}
\newcommand{\lP}{\ell_{\text{P}}}
\newcommand{\tP}{t_{\text{P}}}
\newcommand{\LPlanck}{\ell_{\text{Pl}}}
\newcommand{\TPlanck}{t_{\text{Pl}}}

% --- Coupling Constants ---
\newcommand{\Gnat}{G_{\text{nat}}}
\newcommand{\alphaEM}{\alpha_{\text{EM}}}
\newcommand{\alphaSI}{\alpha_{\text{SI}}}
\newcommand{\Hubble}{H_0}
\newcommand{\LCDM}{\Lambda\text{CDM}}
\newcommand{\natunits}{(nat. units)}

% --- T0 Model Parameters ---
\newcommand{\xigeom}{\xi_{\mathrm{geom}}}
\newcommand{\rzero}{r_{0}}
\newcommand{\xirat}{\xi_{\mathrm{rat}}}
\newcommand{\tzero}{t_{0}}
\newcommand{\Lambdat}{\Lambda_{\mathrm{t}}}
\newcommand{\EP}{E_{\text{P}}}
\newcommand{\Emu}{E_{\mu}}
\newcommand{\Ee}{E_{e}}
\newcommand{\Etau}{E_{\tau}}
\newcommand{\alphafine}{\alpha_{\mathrm{fine}}}
\newcommand{\alphal}{\alpha_{\ell}}
\newcommand{\Lzero}{\ell_{0}}
\newcommand{\Lp}{\ell_{\mathrm{P}}}

% --- Additional T0 Commands ---
\newcommand{\Kfrak}{K_{\text{frak}}}
\newcommand{\Dfrak}{D_{\text{frak}}}
\newcommand{\betapar}{\ensuremath{\beta_T}}
\newcommand{\alphapar}{\alpha}
\newcommand{\deltafield}{\delta \phi}
\newcommand{\deltam}{\delta m}
\newcommand{\deltaE}{\delta E}
\newcommand{\Exi}{E_{\xi}}
\newcommand{\Lxi}{\ell_{\xi}}
\newcommand{\rhoCMB}{\rho_{\text{CMB}}}
\newcommand{\rhoCasimir}{\rho_{\text{Casimir}}}
\newcommand{\Leff}{L_{\text{eff}}}
\newcommand{\CQCD}{C_{\mathrm{QCD}}}
\newcommand{\Kspec}{K_{\mathrm{spec}}}
\newcommand{\Tzero}{\ensuremath{T_0}}
\newcommand{\Eabs}{E_{\text{abs}}}
\newcommand{\taupar}{\tau}

% --- Provided Commands ---
\providecommand{\xiconst}{\xi_{\text{const}}}
\providecommand{\DhiggsT}{D_{\text{Higgs-T}}}
\providecommand{\rhoE}{\rho_{E}}
\providecommand{\Echar}{E_{\text{char}}}
\providecommand{\kfrac}{k_{\text{frac}}}
\providecommand{\alphaEMSI}{\alpha_{\text{EM,SI}}}
\providecommand{\alphaEMnat}{\alpha_{\text{EM,nat}}}
\providecommand{\betaTSI}{\beta_{T,\text{SI}}}
\providecommand{\betaTnat}{\beta_{T,\text{nat}}}
\providecommand{\Gsi}{G_{\text{SI}}}
\providecommand{\xiparSI}{\xi_{\text{SI}}}
\providecommand{\xiparnat}{\xi_{\text{nat}}}
\providecommand{\meff}{m_{\text{eff}}}
\providecommand{\Tzerot}{T_{0}(t)}
\providecommand{\mzerot}{m_{0}(t)}
\providecommand{\Ezeroabs}{E_{0,\text{abs}}}
\providecommand{\Epar}{E_{\text{par}}}
\providecommand{\Lnat}{\ell_{\text{nat}}}
\providecommand{\Tnat}{T_{\text{nat}}}
\providecommand{\xifrak}{\xi_{\text{frac}}}
\providecommand{\Tfrak}{T_{\text{frac}}}
\providecommand{\mfrak}{m_{\text{frac}}}
\providecommand{\Dfrac}{D_{\text{frac}}}
\providecommand{\EphotSI}{E_{\gamma,\text{SI}}}
\providecommand{\EphotNat}{E_{\gamma,\text{nat}}}
\providecommand{\Eabsint}{E_{\text{abs,int}}}
\providecommand{\mphoton}{m_{\gamma}}
\providecommand{\Evis}{E_{\text{vis}}}
\providecommand{\Cto}{C_{T0}}
\providecommand{\mytimes}{\times}
\providecommand{\lambdah}{\lambda_h}
\providecommand{\checkmarkx}{\checkmark}
\providecommand{\Enorm}{E_{\text{norm}}}
\providecommand{\Tobs}{T_{\text{obs}}}
\providecommand{\mobs}{m_{\text{obs}}}
\providecommand{\Eobs}{E_{\text{obs}}}
\providecommand{\Lobs}{\ell_{\text{obs}}}
\providecommand{\xobs}{\xi_{\text{obs}}}
\providecommand{\calE}{\mathcal{E}}
\providecommand{\calT}{\mathcal{T}}
\providecommand{\calM}{\mathcal{M}}
\providecommand{\alphag}{\alpha_g}
\providecommand{\Tmax}{T_{\text{max}}}
\providecommand{\mmin}{m_{\text{min}}}
\providecommand{\Lmax}{\ell_{\text{max}}}
\providecommand{\Emin}{E_{\text{min}}}
\providecommand{\Geff}{G_{\text{eff}}}
\providecommand{\rhoeff}{\rho_{\text{eff}}}
\providecommand{\xieff}{\xi_{\text{eff}}}
\providecommand{\Teff}{T_{\text{eff}}}
\providecommand{\hPlanck}{h}
\providecommand{\kB}{k_B}
\providecommand{\muB}{\mu_B}
\providecommand{\lambdaC}{\lambda_C}
\providecommand{\omegaP}{\omega_P}
\providecommand{\rhoP}{\rho_P}
\providecommand{\Tref}{T_{\text{ref}}}
\providecommand{\Eref}{E_{\text{ref}}}
\providecommand{\mref}{m_{\text{ref}}}
\providecommand{\Lref}{\ell_{\text{ref}}}
\providecommand{\xikonst}{\xi_0}
\providecommand{\Phiphoton}{\Phi_{\gamma}}
\providecommand{\etavis}{\eta_{\text{vis}}}
\providecommand{\pichar}{\pi}
\providecommand{\primrel}{\mathcal{P}_{\text{rel}}}
\providecommand{\warningx}{\textcolor{orange}{\textbf{!}}}
\providecommand{\phiT}{\phi_T}
\providecommand{\Lorentz}{\Lambda}
\providecommand{\Cconv}{C_{\text{conv}}}
\providecommand{\Df}{\Delta f}
\providecommand{\lambdazero}{\lambda_0}
\providecommand{\myapprox}{\approx}
\providecommand{\checked}{\checkmark}
\providecommand{\alphaWSI}{\alpha_W^{\text{SI}}}
\providecommand{\alphaWnat}{\alpha_W^{\text{nat}}}
\providecommand{\vect}[1]{\vec{#1}}
\providecommand{\Rzero}{R_0}
\providecommand{\Riem}{\mathcal{R}}
\providecommand{\nuzero}{\nu_0}
\providecommand{\mypi}{\pi}

% =============================================================================
% TCOLORBOX STYLES AND ENVIRONMENTS (English titles)
% =============================================================================
\tcbset{
	keyresult/.style={
		colback=blue!5!white,
		colframe=blue!75!black,
		title=Key Result,
		fonttitle=\bfseries
	},
	foundation/.style={
		colback=green!5!white,
		colframe=green!75!black,
		title=Foundation,
		fonttitle=\bfseries
	},
	alternative/.style={
		colback=orange!5!white,
		colframe=orange!75!black,
		title=Alternative,
		fonttitle=\bfseries
	},
	warningbox/.style={
		colback=red!5!white,
		colframe=red!75!black,
		title=Warning,
		fonttitle=\bfseries
	}
}

% (Here follow all your tcolorbox definitions with English titles)
\newtcolorbox{keyresultbox}[1][]{colback=blue!5!white,colframe=blue!75!black,fonttitle=\bfseries,title={#1},breakable}
\newtcolorbox{keyresult}[1][Key Result]{colback=blue!5!white,colframe=blue!75!black,fonttitle=\bfseries,title={#1},breakable}
\newtcolorbox{foundationbox}[1][]{colback=green!5!white,colframe=green!75!black,fonttitle=\bfseries,title={#1},breakable}
\newtcolorbox{foundation}[1][Foundation]{colback=green!5!white,colframe=green!75!black,fonttitle=\bfseries,title={#1},breakable}
\newtcolorbox{alternativebox}[1][]{colback=orange!5!white,colframe=orange!75!black,fonttitle=\bfseries,title={#1},breakable}
\newtcolorbox{warningboxenv}[1][Warning]{colback=red!5!white,colframe=red!75!black,fonttitle=\bfseries,title={#1},breakable}

\newtcolorbox{fundamental}[1][]{
	colback=boxgray,
	colframe=t0blue,
	fonttitle=\bfseries,
	title=#1,
	sharp corners,
	boxrule=2pt
}

\newtcolorbox{insightBox}[1][Insight]{colback=blue!5,colframe=t0blue,title={#1},fonttitle=\bfseries,breakable}
\newtcolorbox{discoveryBox}[1][Discovery]{colback=green!5,colframe=t0green,title={#1},fonttitle=\bfseries,breakable}
\newtcolorbox{revelation}[1][Revelation]{colback=red!5,colframe=t0red,title={#1},fonttitle=\bfseries,breakable}
\newtcolorbox{keypoint}[1][Key Point]{colback=blue!5,colframe=t0blue,title={#1},fonttitle=\bfseries,breakable}
\newtcolorbox{evidence}[1][Evidence]{colback=green!5,colframe=t0green,title={#1},fonttitle=\bfseries,breakable}
\newtcolorbox{conclusionBox}[1][Conclusion]{colback=gray!5,colframe=gray,title={#1},fonttitle=\bfseries,breakable}
\newtcolorbox{significance}[1][Significance]{colback=yellow!5,colframe=orange,title={#1},fonttitle=\bfseries,breakable}
\newtcolorbox{philosophical}[1][Philosophical]{colback=purple!5,colframe=purple,title={#1},fonttitle=\bfseries,breakable}
\newtcolorbox{implicationBox}[1][Implication]{colback=cyan!5,colframe=cyan,title={#1},fonttitle=\bfseries,breakable}
\newtcolorbox{perspectiveBox}[1][Perspective]{colback=blue!5,colframe=t0blue,title={#1},fonttitle=\bfseries,breakable}
\newtcolorbox{revolutionary}[1][Revolutionary]{colback=red!5,colframe=t0red,title={#1},fonttitle=\bfseries,breakable}

\newtcolorbox{technical}[1][Technical]{colback=gray!5,colframe=gray!75!black,title={#1},fonttitle=\bfseries,breakable}
\newtcolorbox{technicalBox}[1][Technical]{colback=gray!5,colframe=gray!75!black,title={#1},fonttitle=\bfseries,breakable}
\newtcolorbox{notationBox}[1][Notation]{colback=yellow!5,colframe=yellow!75!black,title={#1},fonttitle=\bfseries,breakable}
\newtcolorbox{verification}[1][Verification]{colback=orange!5!white,colframe=orange!75!black,fonttitle=\bfseries,title=#1}
\newtcolorbox{explanationBox}[1][Explanation]{colback=purple!5!white,colframe=purple!75!black,fonttitle=\bfseries,title=#1}
\newtcolorbox{interpretationBox}[1][Interpretation]{colback=cyan!5!white,colframe=cyan!75!black,fonttitle=\bfseries,title=#1}
\newtcolorbox{explanation}[1][Explanation]{colback=purple!5!white,colframe=purple!75!black,fonttitle=\bfseries,title=#1,breakable}
\newtcolorbox{interpretation}[1][Interpretation]{colback=cyan!5!white,colframe=cyan!75!black,fonttitle=\bfseries,title=#1,breakable}
\newtcolorbox{proof_step}[1][Proof Step]{colback=gray!5!white,colframe=gray!75!black,fonttitle=\bfseries,title=#1,breakable}
\newtcolorbox{experimental}[1][Experimental]{colback=teal!5!white,colframe=teal!75!black,fonttitle=\bfseries,title=#1,breakable}

\newtcolorbox{important}[1][Important]{colback=red!5!white,colframe=red!75!black,title={#1},fonttitle=\bfseries,breakable}
\newtcolorbox{warning}[1][Warning]{colback=orange!5!white,colframe=orange!75!black,title={#1},fonttitle=\bfseries,breakable}
\newtcolorbox{caution}[1][Caution]{colback=yellow!5!white,colframe=yellow!75!black,title={#1},fonttitle=\bfseries,breakable}
\newtcolorbox{highlight}[1][Highlight]{colback=yellow!10!white,colframe=yellow!75!black,title={#1},fonttitle=\bfseries,breakable}
\newtcolorbox{critical}[1][Critical]{colback=red!10!white,colframe=red!75!black,title={#1},fonttitle=\bfseries,breakable}

\newtcolorbox{analysis}[1][Analysis]{colback=blue!5!white,colframe=blue!75!black,title={#1},fonttitle=\bfseries,breakable}
\newtcolorbox{application}[1][Application]{colback=green!5!white,colframe=green!75!black,title={#1},fonttitle=\bfseries,breakable}
\newtcolorbox{experiment}[1][Experiment]{colback=cyan!5!white,colframe=cyan!75!black,title={#1},fonttitle=\bfseries,breakable}
\newtcolorbox{historical}[1][Historical]{colback=brown!5!white,colframe=brown!75!black,title={#1},fonttitle=\bfseries,breakable}
\newtcolorbox{numerical}[1][Numerical]{colback=gray!5!white,colframe=gray!75!black,title={#1},fonttitle=\bfseries,breakable}
\newtcolorbox{overview}[1][Overview]{colback=blue!5!white,colframe=blue!75!black,title={#1},fonttitle=\bfseries,breakable}
\newtcolorbox{speculation}[1][Speculation]{colback=purple!5!white,colframe=purple!75!black,title={#1},fonttitle=\bfseries,breakable}
\newtcolorbox{question}[1][Question]{colback=orange!5!white,colframe=orange!75!black,title={#1},fonttitle=\bfseries,breakable}
\newtcolorbox{method}[1][Method]{colback=teal!5!white,colframe=teal!75!black,title={#1},fonttitle=\bfseries,breakable}
\newtcolorbox{correct}[1][Correct]{colback=green!10!white,colframe=green!75!black,title={#1},fonttitle=\bfseries,breakable}
\newtcolorbox{units}[1][Units]{colback=gray!5!white,colframe=gray!75!black,title={#1},fonttitle=\bfseries,breakable}
\newtcolorbox{achievement}[1][Achievement]{colback=gold!5!white,colframe=orange!75!black,title={#1},fonttitle=\bfseries,breakable}
\newtcolorbox{equivalence}[1][Equivalence]{colback=cyan!5!white,colframe=cyan!75!black,title={#1},fonttitle=\bfseries,breakable}
\newtcolorbox{dimensional}[1][Dimensional Analysis]{colback=purple!5!white,colframe=purple!75!black,title={#1},fonttitle=\bfseries,breakable}

% === ADDITIONAL SIMPLE ENVIRONMENTS ===
\newenvironment{treatise}{\begin{quote}}{\end{quote}}
\newenvironment{gemeinsam}{\begin{quote}}{\end{quote}}
\newenvironment{vergleich}{\begin{quote}}{\end{quote}}
\newenvironment{vorteil}{\begin{quote}}{\end{quote}}
\newenvironment{common}{\begin{quote}}{\end{quote}}
\newenvironment{comparison}{\begin{quote}}{\end{quote}}
\newenvironment{advantage}{\begin{quote}}{\end{quote}}
\newenvironment{quantum}{\begin{quote}}{\end{quote}}

% === LAYOUT SETTINGS ===
\raggedbottom
\usepackage{environ}
\let\oldtabular\tabular
\let\endoldtabular\endtabular

\newenvironment{scaledtable}[1][0.85]{%
	\begingroup\footnotesize\setlength{\LTleft}{0pt}\setlength{\LTright}{0pt}%
}{%
	\endgroup%
}

\newcommand{\widetable}[1]{\resizebox{\textwidth}{!}{#1}}

% === TABLE OF CONTENTS FORMATTING ===
\renewcommand{\cftsecfont}{\color{blue}}
\renewcommand{\cftsubsecfont}{\color{blue}}
\renewcommand{\cftsecpagefont}{\color{blue}}
\renewcommand{\cftsubsecpagefont}{\color{blue}}
\renewcommand{\cfttoctitlefont}{\huge\bfseries\color{blue}}

% === DEFAULT HEADER AND FOOTER ===
\pagestyle{fancy}
\fancyhf{}
\fancyhead[L]{\textsc{T0 Theory}}
\fancyhead[R]{\textsc{J. Pascher}}
\fancyfoot[C]{\thepage}

% ==============================================================================
% End of Shared Preamble for English
% ==============================================================================
%
% Usage:
%   \documentclass[12pt,a4paper]{article}  % or book, report, etc.
%   % ==============================================================================
% T0 Theory: Shared ENGLISH Preamble – Optimized for eBook/Book
% Version: 2.0 – Final 2026 (LuaLaTeX only) – ENGLISH corrected
% Author: Johann Pascher
% Date: January 2026
% ==============================================================================
%
% IMPORTANT: Compile EXCLUSIVELY with LuaLaTeX!
% In TeXstudio: Options → Configure TeXstudio → Build → Default Compiler → LuaLaTeX
%
% Required Fonts (install once):
% - Inter: https://fonts.google.com/specimen/Inter
% - JetBrains Mono: https://www.jetbrains.com/lp/mono/
% - Libertinus Math: https://github.com/libertinus-fonts/libertinus
% ==============================================================================

% === CHAPTER 1: BASIC PACKAGES (must come FIRST) ===
\RequirePackage{fontspec}
\RequirePackage{unicode-math}
\usepackage{chngcntr}
\setcounter{secnumdepth}{1}  % Nur Sections nummerieren (nicht subsections)
\setcounter{tocdepth}{1}     % Nur Sections im TOC (nicht subsections)
\makeatletter
\@ifundefined{c@chapter}{}{\counterwithout{section}{chapter}}  % Falls Kapitel existieren
\makeatother
\counterwithout{subsection}{section}  % Löse Verknüpfung
% === CHAPTER 2: LANGUAGE (ENGLISH) ===
\usepackage[english]{babel}
\usepackage{microtype}                    % IMPORTANT for better hyphenation!

% Typography settings for better line breaking
\frenchspacing                     % Correct English spacing after punctuation
\emergencystretch=3em              % Allows more stretch for difficult lines
\tolerance=2500                    % Higher tolerance for line breaks
\hbadness=10000                    % Suppresses "underfull hbox" warnings
\hfuzz=2pt                         % Allows minimal overfull
\pretolerance=150                  % Better word breaking

% Prevent bad page breaks
\clubpenalty=10000           % No "orphans"
\widowpenalty=10000          % No "widows"
\displaywidowpenalty=10000   % Also with equations
\brokenpenalty=10000         % No broken words across pages

% Explicit hyphenation for long technical words
\hyphenation{Fun-da-men-tal Frac-tal-Ge-o-met-ric Field The-o-ry Meth-od-o-log-i-cal}
\hyphenation{Re-vi-sion-ism Quan-ti-za-tion U-ni-fi-ca-tion Ef-fec-tive}
\hyphenation{Re-nor-mal-iz-a-bil-i-ty Sin-gu-lar-i-ties Con-cil-i-a-tion}
\hyphenation{E-mer-gence Phe-nom-e-no-log-i-cal Doc-u-men-ta-tion A-nal-y-sis}
\hyphenation{Grav-i-ta-tion Quan-tum Me-chan-ics Dog-ma-tism Con-se-quent}
\hyphenation{Par-al-lel-ism Im-ple-men-ta-tion Per-tur-ba-tions}
\hyphenation{Geo-met-ric Ar-ti-fact In-com-pat-i-bil-i-ty Con-struc-tive}
\hyphenation{Frac-tal Di-men-sion-less In-ves-ti-ga-tion De-scrip-tion}
\hyphenation{In-ter-pre-ta-tion Phe-nom-e-no-log-i-cal Math-e-mat-i-cal}
\hyphenation{Phi-lo-soph-i-cal Le-git-i-ma-tion Ap-pli-ca-tion Der-i-va-tion}
\hyphenation{U-ni-fi-ca-tion As-sump-tion Con-cep-tion Ex-pec-ta-tion}
\hyphenation{Sym-me-try-ex-ten-sion O-ver-all-pic-ture Chal-lenge}
\hyphenation{In-ter-ac-tion Ma-te-ri-al Ap-proach Per-spec-tive Pro-ce-dure}

% === CHAPTER 3: FONTS (with proper ligatures) ===
\setmainfont{Inter}[
Scale=1.02,
UprightFont=*-Regular,
BoldFont=*-Bold,
ItalicFont=*-Italic,
BoldItalicFont=*-BoldItalic,
Ligatures=TeX,           % IMPORTANT for proper typography
Language=English         % Explicit language support
]
\setsansfont{Inter}[
Scale=MatchLowercase,
Ligatures=TeX,
Language=English
]
\setmonofont{JetBrains Mono}[
Scale=0.95,
Language=English
]

% Math Font (simple & stable) – MUST come AFTER language definition
% IMPORTANT: Libertinus Math for correct \underbrace display!
\setmathfont{Libertinus Math}[Scale=1.0]

% === CHAPTER 4: MATHEMATICS PACKAGES (in STRICT order!) ===
% IMPORTANT: mathtools must come BEFORE unicode-math for some commands!
\usepackage{mathtools}           % FIRST mathtools!

% Then the rest
\usepackage{amsmath, amsfonts, amsthm}

% SIUNITX MUST be loaded BEFORE physics!
\usepackage{siunitx}
\sisetup{
	locale=US,                    % ENGLISH settings for SI units!
	group-separator={,},          % Thousands separator comma
	output-decimal-marker={.},    % Decimal separator point
	per-mode=symbol,
	separate-uncertainty=true
}

% Custom SI units used in narrative and books
\DeclareSIUnit\gigalightyear{Gly}
\DeclareSIUnit\mev{MeV}

% physics – MUST be loaded AFTER siunitx and mathtools
\usepackage{physics}

% === CHAPTER 5: ADDITIONS from pdflatex best practices ===
\usepackage{colortbl}        % Colored tables (ESSENTIAL!)
\usepackage{placeins}        % Float control: \FloatBarrier
\usepackage{subcaption}      % Subfigures
\usepackage{xurl}            % Better URL line breaking
% Hyphenation for URLs in bibliography
\def\UrlBreaks{\do\/\do-}

% === CHAPTER 6: PAGE LAYOUT
% =============================================================================
% SECTION 2: Page Geometry – 6" × 9" Buchformat
% =============================================================================
\usepackage[paperwidth=6in, paperheight=9in,
top=0.9in,
bottom=1.1in,
inner=0.9in,            % Größerer Innenrand für Bindung
outer=0.6in,            % Kleinerer Außenrand → mehr Text pro Seite
bindingoffset=0.5in,    % Puffer für Bindung (Steg)
twoside]{geometry}
\setlength{\headheight}{15pt}
%\usepackage[paperwidth=8.25in, paperheight=11in,
%top=1.0in,
%bottom=1.0in,
%left=1.0in,
%right=1.0in,
%twoside=false
% === CHAPTER 7: GRAPHICS AND TABLES ===
\usepackage{graphicx}
\usepackage[table,xcdraw]{xcolor}
% T0 brand colors
\definecolor{gold}{RGB}{255,215,0}
\definecolor{blue}{rgb}{0,0,1}
\definecolor{boxgray}{RGB}{240,240,240}
\definecolor{deepblue}{RGB}{0,0,127}
\definecolor{deepgreen}{RGB}{0,127,0}
\definecolor{deepred}{RGB}{191,0,0}
\definecolor{t0blue}{RGB}{33,150,243}
\definecolor{t0green}{RGB}{76,175,80}
\definecolor{t0orange}{RGB}{255,152,0}
\definecolor{t0purple}{RGB}{156,39,176}
\definecolor{t0red}{RGB}{244,67,54}
\definecolor{t0yellow}{RGB}{255,204,0}
\usepackage{tikz}
\usetikzlibrary{arrows.meta,positioning,shapes.geometric,decorations.pathmorphing,patterns,shapes.arrows,intersections}
\usepackage{pgfplots}
\pgfplotsset{compat=1.18}
\usepackage{quantikz}
\usepackage[most]{tcolorbox}
\tcbuselibrary{breakable}

% === WICHTIG: Algorithm-Konflikt umgehen ===
% Option: algorithmic mit GROSSBUCHSTABEN
% Gemeinsame Box für Experimente
\newtcolorbox{experimentbox}[1][]{
	colback=green!5!white,
	colframe=t0green!80!black,
	fonttitle=\bfseries,
	title={{#1}},
	breakable
}

% Abstract-Fallback
\ifdefined\abstract\else
\newenvironment{abstract}{\section*{\abstractname}\itshape\small\par\bigskip}{\bigskip}
\fi

% === MAKROS SICHER NEU DEFINIEREN / ÜBERSCHREIBEN ===
% Definiere Makros OHNE doppelte Subskripte
\newcommand{\phipar}{\phi_{\mathrm{par}}}
%\newcommand{\xipar}{\xi_{\mathrm{par}}}
\newcommand{\Qphipar}{Q_{\phi_{\mathrm{par}}}}
\newcommand{\rphipar}{r_{\phi_{\mathrm{par}}}}
\newcommand{\logphipar}{\log_{\phi_{\mathrm{par}}}}
\newcommand{\CHSH}{\text{CHSH}}
\usepackage{booktabs}
\usepackage{array}
\usepackage{longtable}
\usepackage{float}
\usepackage{adjustbox}
\usepackage{rotating}
\usepackage{tabularx}
\usepackage{makecell}
\usepackage{multirow}

% === CHAPTER 8: DOCUMENT FORMATTING ===
\usepackage{fancyhdr}
\renewcommand{\headrulewidth}{0.4pt}
\renewcommand{\footrulewidth}{0.4pt}
\usepackage{tocloft}

\usepackage{enumitem}
\setlist[itemize]{leftmargin=*, topsep=2pt, partopsep=0pt, parsep=2pt, itemsep=2pt}
\setlist[enumerate]{leftmargin=*, topsep=2pt, partopsep=0pt, parsep=2pt, itemsep=2pt}
\usepackage{setspace}
\usepackage{ragged2e}
\usepackage{multicol}

% === CHAPTER 9: CODE AND ALGORITHMS ===
\usepackage{algorithm}
\usepackage{algorithmic}
\usepackage{listings}
\lstset{
	basicstyle=\ttfamily\footnotesize,
	breaklines=true,
	breakatwhitespace=true,
	columns=flexible,
	keepspaces=true,
	showstringspaces=false,
	frame=single,
	xleftmargin=0pt,
	xrightmargin=0pt,
	literate=              % For special characters in code listings
	{ä}{{\"a}}1 {ö}{{\"o}}1 {ü}{{\"u}}1 {ß}{{\ss}}1
	{Ä}{{\"A}}1 {Ö}{{\"O}}1 {Ü}{{\"U}}1
}
\usepackage{mdframed}

% === CHAPTER 10: ADDITIONAL PACKAGES ===
\usepackage{pdflscape}
\usepackage{braket}
\usepackage{cancel}
\usepackage{caption}
\captionsetup{format=plain, labelfont=bf, justification=centering}
\usepackage{csquotes}
\usepackage{gensymb}
\usepackage{textcomp}
\usepackage{textgreek}
\usepackage{upgreek}
\usepackage{url}
\usepackage{slashed}
\usepackage{bm}

% === CHAPTER 11: HYPERREF (must come SECOND TO LAST!) ===
\usepackage{hyperref}
\hypersetup{
	colorlinks=true,
	linkcolor=black,
	citecolor=black,
	urlcolor=black,
	breaklinks=true,           % IMPORTANT for special characters in URLs!
	bookmarksnumbered=true,
	unicode=true,
	pdfencoding=auto,
	pdflang=en,                % Set PDF language to English
	pdfsubject={T0 Theory - Fundamental Fractal-Geometric Field Theory}
}

% Fix for unicode-math symbols in PDF bookmarks
\pdfstringdefDisableCommands{%
	\def\xi{xi}%
	\def\alpha{alpha}%
	\def\beta{beta}%
	\def\gamma{gamma}%
	\def\delta{delta}%
	\def\Delta{Delta}%
	\def\epsilon{epsilon}%
	\def\varepsilon{epsilon}%
	\def\theta{theta}%
	\def\kappa{kappa}%
	\def\lambda{lambda}%
	\def\mu{mu}%
	\def\nu{nu}%
	\def\pi{pi}%
	\def\rho{rho}%
	\def\sigma{sigma}%
	\def\tau{tau}%
	\def\phi{phi}%
	\def\chi{chi}%
	\def\psi{psi}%
	\def\omega{omega}%
	\def\Omega{Omega}%
	\def\Lambda{Lambda}%
	\def\times{x}%
	\def\cdot{*}%
	\def\pm{+/-}%
	\def\approx{~}%
	\def\sim{~}%
	\def\equiv{=}%
	\def\ell{l}%
	\def\hbar{h}%
	\def\rightarrow{->}%
	\def\leftarrow{<-}%
	\def\Rightarrow{=>}%
	\def\Leftarrow{<=}%
	\def\propto{~}%
	\def\mitxi{xi}%
	\def\mitalpha{alpha}%
	\def\mitbeta{beta}%
	\def\mitgamma{gamma}%
	\def\mitdelta{delta}%
	\def\mitDelta{Delta}%
	\def\mitepsilon{epsilon}%
	\def\mitvarepsilon{epsilon}%
	\def\mittheta{theta}%
	\def\mitkappa{kappa}%
	\def\mitlambda{lambda}%
	\def\mitLambda{Lambda}%
	\def\mitmu{mu}%
	\def\mitnu{nu}%
	\def\mitpi{pi}%
	\def\mitrho{rho}%
	\def\mitsigma{sigma}%
	\def\mittau{tau}%
	\def\mitphi{phi}%
	\def\mitchi{chi}%
	\def\mitpsi{psi}%
	\def\mitomega{omega}%
	\def\mitOmega{Omega}%
}

% === CHAPTER 12: BOOKMARK (must come AFTER hyperref!) ===
\usepackage{bookmark}

% === CHAPTER 13: CLEVEREF (ENGLISH LABELS) ===
\usepackage[english]{cleveref}
\crefname{equation}{Equation}{Equations}
\crefname{figure}{Figure}{Figures}
\crefname{table}{Table}{Tables}
\crefname{section}{Section}{Sections}
\crefname{chapter}{Chapter}{Chapters}
\crefname{theorem}{Theorem}{Theorems}
\crefname{lemma}{Lemma}{Lemmas}
\crefname{definition}{Definition}{Definitions}
\crefname{example}{Example}{Examples}
\crefname{remark}{Remark}{Remarks}

% === CUSTOM ENVIRONMENTS ===
% Alternative interpretation environment
\newenvironment{alternative}{%
	\begin{mdframed}[linecolor=black!30,linewidth=1pt,roundcorner=4pt,backgroundcolor=black!5]%
	}{%
	\end{mdframed}%
}

% Photon/particle environment
\newenvironment{photon}{%
	\begin{mdframed}[linecolor=blue!30,linewidth=1pt,roundcorner=4pt,backgroundcolor=blue!5]%
	}{%
	\end{mdframed}%
}

% Koide formula box environment
\newenvironment{koidebox}{%
	\begin{mdframed}[linecolor=green!30,linewidth=1pt,roundcorner=4pt,backgroundcolor=green!5]%
	}{%
	\end{mdframed}%
}

% Erkenntnis/insight environment
\newenvironment{erkenntnis}{%
	\begin{mdframed}[linecolor=orange!30,linewidth=1pt,roundcorner=4pt,backgroundcolor=orange!5]%
	}{%
	\end{mdframed}%
}

% Beziehung/relationship environment
\newenvironment{beziehung}{%
	\begin{mdframed}[linecolor=purple!30,linewidth=1pt,roundcorner=4pt,backgroundcolor=purple!5]%
	}{%
	\end{mdframed}%
}

% Derivation environment
\newenvironment{derivation}{%
	\begin{mdframed}[linecolor=teal!30,linewidth=1pt,roundcorner=4pt,backgroundcolor=teal!5]%
	}{%
	\end{mdframed}%
}

% Abhandlung/treatise environment
\newenvironment{abhandlung}{%
	\begin{mdframed}[linecolor=brown!30,linewidth=1pt,roundcorner=4pt,backgroundcolor=brown!5]%
	}{%
	\end{mdframed}%
}

% Anwendung/application environment
\newenvironment{anwendung}{%
	\begin{mdframed}[linecolor=cyan!30,linewidth=1pt,roundcorner=4pt,backgroundcolor=cyan!5]%
	}{%
	\end{mdframed}%
}

% Additional common environments
\newenvironment{konsequenz}{%
	\begin{mdframed}[linecolor=red!30,linewidth=1pt,roundcorner=4pt,backgroundcolor=red!5]%
	}{%
	\end{mdframed}%
}

\newenvironment{schlussfolgerung}{%
	\begin{mdframed}[linecolor=gray!30,linewidth=1pt,roundcorner=4pt,backgroundcolor=gray!5]%
	}{%
	\end{mdframed}%
}

\newenvironment{result}{%
	\begin{mdframed}[linecolor=violet!30,linewidth=1pt,roundcorner=4pt,backgroundcolor=violet!5]%
	}{%
	\end{mdframed}%
}

% Formula environment
\newenvironment{formula}{%
	\begin{mdframed}[linecolor=yellow!30,linewidth=1pt,roundcorner=4pt,backgroundcolor=yellow!5]%
	}{%
	\end{mdframed}%
}

% Revolutionaer/revolutionary environment
\newenvironment{revolutionaer}{%
	\begin{mdframed}[linecolor=red!50,linewidth=2pt,roundcorner=4pt,backgroundcolor=red!10]%
	}{%
	\end{mdframed}%
}

% Formel environment (German version of formula)
\newenvironment{formel}{%
	\begin{mdframed}[linecolor=yellow!30,linewidth=1pt,roundcorner=4pt,backgroundcolor=yellow!5]%
	}{%
	\end{mdframed}%
}

% Prinzip/principle environment
\newenvironment{prinzip}{%
	\begin{mdframed}[linecolor=blue!50,linewidth=2pt,roundcorner=4pt,backgroundcolor=blue!10]%
	}{%
	\end{mdframed}%
}

% Experimentell/experimental environment
\newenvironment{experimentell}{%
	\begin{mdframed}[linecolor=magenta!30,linewidth=1pt,roundcorner=4pt,backgroundcolor=magenta!5]%
	}{%
	\end{mdframed}%
}

% Neutrino environment
\newenvironment{neutrino}{%
	\begin{mdframed}[linecolor=cyan!40,linewidth=1pt,roundcorner=4pt,backgroundcolor=cyan!8]%
	}{%
	\end{mdframed}%
}

% Additional missing environments
\newenvironment{schluessel}{%
	\begin{mdframed}[linecolor=yellow!50,linewidth=1pt,roundcorner=4pt,backgroundcolor=yellow!10]%
	}{%
	\end{mdframed}%
}

\newenvironment{summary}{%
	\begin{mdframed}[linecolor=gray!40,linewidth=1pt,roundcorner=4pt,backgroundcolor=gray!8]%
	}{%
	\end{mdframed}%
}

\newenvironment{category}{%
	\begin{mdframed}[linecolor=pink!40,linewidth=1pt,roundcorner=4pt,backgroundcolor=pink!8]%
	}{%
	\end{mdframed}%
}

\newenvironment{sibox}{%
	\begin{mdframed}[linecolor=lime!40,linewidth=1pt,roundcorner=4pt,backgroundcolor=lime!8]%
	}{%
	\end{mdframed}%
}

% More missing environments
\newenvironment{documentbox}{%
	\begin{mdframed}[linecolor=teal!40,linewidth=1pt,roundcorner=4pt,backgroundcolor=teal!8]%
	}{%
	\end{mdframed}%
}

\newenvironment{t0box}{%
	\begin{mdframed}[linecolor=violet!40,linewidth=1pt,roundcorner=4pt,backgroundcolor=violet!8]%
	}{%
	\end{mdframed}%
}

\newenvironment{wichtig}{%
	\begin{mdframed}[linecolor=red!50,linewidth=2pt,roundcorner=4pt,backgroundcolor=red!10]%
	\textbf{Important:} 
	}{%
	\end{mdframed}%
}

\newenvironment{smbox}{%
	\begin{mdframed}[linecolor=orange!40,linewidth=1pt,roundcorner=4pt,backgroundcolor=orange!8]%
	}{%
	\end{mdframed}%
}

\newenvironment{pvbox}{%
	\begin{mdframed}[linecolor=purple!40,linewidth=1pt,roundcorner=4pt,backgroundcolor=purple!8]%
	}{%
	\end{mdframed}%
}

\newenvironment{numerisch}{%
	\begin{mdframed}[linecolor=blue!40,linewidth=1pt,roundcorner=4pt,backgroundcolor=blue!8]%
	}{%
	\end{mdframed}%
}

% More missing environments
\newenvironment{relation}{%
	\begin{mdframed}[linecolor=green!40,linewidth=1pt,roundcorner=4pt,backgroundcolor=green!8]%
	}{%
	\end{mdframed}%
}

\newenvironment{beweis}{%
	\begin{mdframed}[linecolor=brown!40,linewidth=1pt,roundcorner=4pt,backgroundcolor=brown!8]%
	\textbf{Proof:} 
	}{%
	\end{mdframed}%
}

\newenvironment{revolution}{%
	\begin{mdframed}[linecolor=red!60,linewidth=2pt,roundcorner=4pt,backgroundcolor=red!12]%
	}{%
	\end{mdframed}%
}

\newenvironment{key}{%
	\begin{mdframed}[linecolor=yellow!50,linewidth=1pt,roundcorner=4pt,backgroundcolor=yellow!10]%
	}{%
	\end{mdframed}%
}

\newenvironment{newperspective}{%
	\begin{mdframed}[linecolor=cyan!50,linewidth=1pt,roundcorner=4pt,backgroundcolor=cyan!10]%
	}{%
	\end{mdframed}%
}

\newenvironment{literatur}{%
	\begin{mdframed}[linecolor=gray!50,linewidth=1pt,roundcorner=4pt,backgroundcolor=gray!10]%
	}{%
	\end{mdframed}%
}

\newenvironment{folgerung}{%
	\begin{mdframed}[linecolor=teal!50,linewidth=1pt,roundcorner=4pt,backgroundcolor=teal!10]%
	}{%
	\end{mdframed}%
}

\newenvironment{principle}{%
	\begin{mdframed}[linecolor=blue!60,linewidth=2pt,roundcorner=4pt,backgroundcolor=blue!12]%
	}{%
	\end{mdframed}%
}

% Additional common environments
% ==============================================================================
% FROM HERE: YOUR DEFINITIONS (unchanged)
% ==============================================================================

\setcounter{tocdepth}{3}

% === CITATION COMMANDS ===
\providecommand{\citep}[1]{\cite{#1}}
\providecommand{\citet}[1]{\cite{#1}}

% === COLORS ===
\definecolor{gold}{RGB}{255,215,0}
\definecolor{blue}{rgb}{0,0,1}
\definecolor{boxgray}{RGB}{240,240,240}
\definecolor{deepblue}{RGB}{0,0,127}
\definecolor{deepgreen}{RGB}{0,127,0}
\definecolor{deepred}{RGB}{191,0,0}
\definecolor{t0blue}{RGB}{33,150,243}
\definecolor{t0green}{RGB}{76,175,80}
\definecolor{t0orange}{RGB}{255,152,0}
\definecolor{t0purple}{RGB}{156,39,176}
\definecolor{t0red}{RGB}{244,67,54}
\definecolor{t0yellow}{RGB}{255,204,0}

% === COLUMN TYPES ===
\newcolumntype{L}[1]{>{\raggedright\arraybackslash}p{#1}}
\newcolumntype{C}[1]{>{\centering\arraybackslash}p{#1}}
\newcolumntype{R}[1]{>{\raggedleft\arraybackslash}p{#1}}

% === HYPERREF SETTINGS (updated) ===
\hypersetup{
	colorlinks=true,
	linkcolor=t0blue,
	citecolor=t0blue,
	urlcolor=t0blue,
	breaklinks=true,
	bookmarksnumbered=true,
	pdfstartview=FitH,
	pdfencoding=auto,
	pdfdisplaydoctitle=true
}

% === ENGLISH THEOREM ENVIRONMENTS ===
\theoremstyle{plain}
\newtheorem{theorem}{Theorem}[section]
\newtheorem{lemma}[theorem]{Lemma}
\newtheorem{proposition}[theorem]{Proposition}
\newtheorem{corollary}[theorem]{Corollary}

\theoremstyle{definition}
\newtheorem{definition}[theorem]{Definition}
\newtheorem{example}[theorem]{Example}
\newtheorem{insight}[theorem]{Insight}
\newtheorem{discovery}[theorem]{Discovery}

\theoremstyle{remark}
\newtheorem{remark}[theorem]{Remark}
\newtheorem{axiom}{Axiom}
%\newtheorem{principle}{Principle}  % Commented out to avoid conflicts with document-specific definitions
%\newtheorem{warning}[theorem]{Warning}

% === T0-SPECIFIC COMMANDS ===
% (Here follow all your \newcommand and \providecommand definitions)
% These remain UNCHANGED as in your original preamble
% ==============================================================================
% SECTION 14: T0-Specific Commands
% ==============================================================================

% --- Core T0 Fields ---
\newcommand{\Tfield}{T(x,t)}
\providecommand{\Tfieldt}{T(\vec{x},t)}
\newcommand{\Efield}{E(x,t)}
\newcommand{\mfield}{m(x,t)}
\providecommand{\vecx}{\vec{x}}

% --- Lagrangian ---
\newcommand{\Lag}{\mathcal{L}}
\newcommand{\calL}{\mathcal{L}}

% --- Greek Letters and Constants ---
\newcommand{\alphaem}{\alpha}
\newcommand{\betaT}{\beta_T}
\newcommand{\xiT}{\xi}
\newcommand{\xipar}{\xi}

% --- Energy and Planck Units ---
\newcommand{\Ezero}{E_0}
\newcommand{\E}{E}
\newcommand{\EPlanck}{E_{\text{Pl}}}
\newcommand{\Mpl}{M_{\text{Pl}}}
\newcommand{\mP}{m_{\text{P}}}
\newcommand{\lP}{\ell_{\text{P}}}
\newcommand{\tP}{t_{\text{P}}}
\newcommand{\LPlanck}{\ell_{\text{Pl}}}
\newcommand{\TPlanck}{t_{\text{Pl}}}

% --- Coupling Constants ---
\newcommand{\Gnat}{G_{\text{nat}}}
\newcommand{\alphaEM}{\alpha_{\text{EM}}}
\newcommand{\alphaSI}{\alpha_{\text{SI}}}
\newcommand{\Hubble}{H_0}
\newcommand{\LCDM}{\Lambda\text{CDM}}
\newcommand{\natunits}{(nat. units)}

% --- T0 Model Parameters ---
\newcommand{\xigeom}{\xi_{\mathrm{geom}}}
\newcommand{\rzero}{r_{0}}
\newcommand{\xirat}{\xi_{\mathrm{rat}}}
\newcommand{\tzero}{t_{0}}
\newcommand{\Lambdat}{\Lambda_{\mathrm{t}}}
\newcommand{\EP}{E_{\text{P}}}
\newcommand{\Emu}{E_{\mu}}
\newcommand{\Ee}{E_{e}}
\newcommand{\Etau}{E_{\tau}}
\newcommand{\alphafine}{\alpha_{\mathrm{fine}}}
\newcommand{\alphal}{\alpha_{\ell}}
\newcommand{\Lzero}{\ell_{0}}
\newcommand{\Lp}{\ell_{\mathrm{P}}}

% --- Additional T0 Commands ---
\newcommand{\Kfrak}{K_{\text{frak}}}
\newcommand{\Dfrak}{D_{\text{frak}}}
\newcommand{\betapar}{\ensuremath{\beta_T}}
\newcommand{\alphapar}{\alpha}
\newcommand{\deltafield}{\delta \phi}
\newcommand{\deltam}{\delta m}
\newcommand{\deltaE}{\delta E}
\newcommand{\Exi}{E_{\xi}}
\newcommand{\Lxi}{\ell_{\xi}}
\newcommand{\rhoCMB}{\rho_{\text{CMB}}}
\newcommand{\rhoCasimir}{\rho_{\text{Casimir}}}
\newcommand{\Leff}{L_{\text{eff}}}
\newcommand{\CQCD}{C_{\mathrm{QCD}}}
\newcommand{\Kspec}{K_{\mathrm{spec}}}
\newcommand{\Tzero}{\ensuremath{T_0}}
\newcommand{\Eabs}{E_{\text{abs}}}
\newcommand{\taupar}{\tau}

% --- Provided Commands ---
\providecommand{\xiconst}{\xi_{\text{const}}}
\providecommand{\DhiggsT}{D_{\text{Higgs-T}}}
\providecommand{\rhoE}{\rho_{E}}
\providecommand{\Echar}{E_{\text{char}}}
\providecommand{\kfrac}{k_{\text{frac}}}
\providecommand{\alphaEMSI}{\alpha_{\text{EM,SI}}}
\providecommand{\alphaEMnat}{\alpha_{\text{EM,nat}}}
\providecommand{\betaTSI}{\beta_{T,\text{SI}}}
\providecommand{\betaTnat}{\beta_{T,\text{nat}}}
\providecommand{\Gsi}{G_{\text{SI}}}
\providecommand{\xiparSI}{\xi_{\text{SI}}}
\providecommand{\xiparnat}{\xi_{\text{nat}}}
\providecommand{\meff}{m_{\text{eff}}}
\providecommand{\Tzerot}{T_{0}(t)}
\providecommand{\mzerot}{m_{0}(t)}
\providecommand{\Ezeroabs}{E_{0,\text{abs}}}
\providecommand{\Epar}{E_{\text{par}}}
\providecommand{\Lnat}{\ell_{\text{nat}}}
\providecommand{\Tnat}{T_{\text{nat}}}
\providecommand{\xifrak}{\xi_{\text{frac}}}
\providecommand{\Tfrak}{T_{\text{frac}}}
\providecommand{\mfrak}{m_{\text{frac}}}
\providecommand{\Dfrac}{D_{\text{frac}}}
\providecommand{\EphotSI}{E_{\gamma,\text{SI}}}
\providecommand{\EphotNat}{E_{\gamma,\text{nat}}}
\providecommand{\Eabsint}{E_{\text{abs,int}}}
\providecommand{\mphoton}{m_{\gamma}}
\providecommand{\Evis}{E_{\text{vis}}}
\providecommand{\Cto}{C_{T0}}
\providecommand{\mytimes}{\times}
\providecommand{\lambdah}{\lambda_h}
\providecommand{\checkmarkx}{\checkmark}
\providecommand{\Enorm}{E_{\text{norm}}}
\providecommand{\Tobs}{T_{\text{obs}}}
\providecommand{\mobs}{m_{\text{obs}}}
\providecommand{\Eobs}{E_{\text{obs}}}
\providecommand{\Lobs}{\ell_{\text{obs}}}
\providecommand{\xobs}{\xi_{\text{obs}}}
\providecommand{\calE}{\mathcal{E}}
\providecommand{\calT}{\mathcal{T}}
\providecommand{\calM}{\mathcal{M}}
\providecommand{\alphag}{\alpha_g}
\providecommand{\Tmax}{T_{\text{max}}}
\providecommand{\mmin}{m_{\text{min}}}
\providecommand{\Lmax}{\ell_{\text{max}}}
\providecommand{\Emin}{E_{\text{min}}}
\providecommand{\Geff}{G_{\text{eff}}}
\providecommand{\rhoeff}{\rho_{\text{eff}}}
\providecommand{\xieff}{\xi_{\text{eff}}}
\providecommand{\Teff}{T_{\text{eff}}}
\providecommand{\hPlanck}{h}
\providecommand{\kB}{k_B}
\providecommand{\muB}{\mu_B}
\providecommand{\lambdaC}{\lambda_C}
\providecommand{\omegaP}{\omega_P}
\providecommand{\rhoP}{\rho_P}
\providecommand{\Tref}{T_{\text{ref}}}
\providecommand{\Eref}{E_{\text{ref}}}
\providecommand{\mref}{m_{\text{ref}}}
\providecommand{\Lref}{\ell_{\text{ref}}}
\providecommand{\xikonst}{\xi_0}
\providecommand{\Phiphoton}{\Phi_{\gamma}}
\providecommand{\etavis}{\eta_{\text{vis}}}
\providecommand{\pichar}{\pi}
\providecommand{\primrel}{\mathcal{P}_{\text{rel}}}
\providecommand{\warningx}{\textcolor{orange}{\textbf{!}}}
\providecommand{\phiT}{\phi_T}
\providecommand{\Lorentz}{\Lambda}
\providecommand{\Cconv}{C_{\text{conv}}}
\providecommand{\Df}{\Delta f}
\providecommand{\lambdazero}{\lambda_0}
\providecommand{\myapprox}{\approx}
\providecommand{\checked}{\checkmark}
\providecommand{\alphaWSI}{\alpha_W^{\text{SI}}}
\providecommand{\alphaWnat}{\alpha_W^{\text{nat}}}
\providecommand{\vect}[1]{\vec{#1}}
\providecommand{\Rzero}{R_0}
\providecommand{\Riem}{\mathcal{R}}
\providecommand{\nuzero}{\nu_0}
\providecommand{\mypi}{\pi}

% =============================================================================
% TCOLORBOX STYLES AND ENVIRONMENTS (English titles)
% =============================================================================
\tcbset{
	keyresult/.style={
		colback=blue!5!white,
		colframe=blue!75!black,
		title=Key Result,
		fonttitle=\bfseries
	},
	foundation/.style={
		colback=green!5!white,
		colframe=green!75!black,
		title=Foundation,
		fonttitle=\bfseries
	},
	alternative/.style={
		colback=orange!5!white,
		colframe=orange!75!black,
		title=Alternative,
		fonttitle=\bfseries
	},
	warningbox/.style={
		colback=red!5!white,
		colframe=red!75!black,
		title=Warning,
		fonttitle=\bfseries
	}
}

% (Here follow all your tcolorbox definitions with English titles)
\newtcolorbox{keyresultbox}[1][]{colback=blue!5!white,colframe=blue!75!black,fonttitle=\bfseries,title={#1},breakable}
\newtcolorbox{keyresult}[1][Key Result]{colback=blue!5!white,colframe=blue!75!black,fonttitle=\bfseries,title={#1},breakable}
\newtcolorbox{foundationbox}[1][]{colback=green!5!white,colframe=green!75!black,fonttitle=\bfseries,title={#1},breakable}
\newtcolorbox{foundation}[1][Foundation]{colback=green!5!white,colframe=green!75!black,fonttitle=\bfseries,title={#1},breakable}
\newtcolorbox{alternativebox}[1][]{colback=orange!5!white,colframe=orange!75!black,fonttitle=\bfseries,title={#1},breakable}
\newtcolorbox{warningboxenv}[1][Warning]{colback=red!5!white,colframe=red!75!black,fonttitle=\bfseries,title={#1},breakable}

\newtcolorbox{fundamental}[1][]{
	colback=boxgray,
	colframe=t0blue,
	fonttitle=\bfseries,
	title=#1,
	sharp corners,
	boxrule=2pt
}

\newtcolorbox{insightBox}[1][Insight]{colback=blue!5,colframe=t0blue,title={#1},fonttitle=\bfseries,breakable}
\newtcolorbox{discoveryBox}[1][Discovery]{colback=green!5,colframe=t0green,title={#1},fonttitle=\bfseries,breakable}
\newtcolorbox{revelation}[1][Revelation]{colback=red!5,colframe=t0red,title={#1},fonttitle=\bfseries,breakable}
\newtcolorbox{keypoint}[1][Key Point]{colback=blue!5,colframe=t0blue,title={#1},fonttitle=\bfseries,breakable}
\newtcolorbox{evidence}[1][Evidence]{colback=green!5,colframe=t0green,title={#1},fonttitle=\bfseries,breakable}
\newtcolorbox{conclusionBox}[1][Conclusion]{colback=gray!5,colframe=gray,title={#1},fonttitle=\bfseries,breakable}
\newtcolorbox{significance}[1][Significance]{colback=yellow!5,colframe=orange,title={#1},fonttitle=\bfseries,breakable}
\newtcolorbox{philosophical}[1][Philosophical]{colback=purple!5,colframe=purple,title={#1},fonttitle=\bfseries,breakable}
\newtcolorbox{implicationBox}[1][Implication]{colback=cyan!5,colframe=cyan,title={#1},fonttitle=\bfseries,breakable}
\newtcolorbox{perspectiveBox}[1][Perspective]{colback=blue!5,colframe=t0blue,title={#1},fonttitle=\bfseries,breakable}
\newtcolorbox{revolutionary}[1][Revolutionary]{colback=red!5,colframe=t0red,title={#1},fonttitle=\bfseries,breakable}

\newtcolorbox{technical}[1][Technical]{colback=gray!5,colframe=gray!75!black,title={#1},fonttitle=\bfseries,breakable}
\newtcolorbox{technicalBox}[1][Technical]{colback=gray!5,colframe=gray!75!black,title={#1},fonttitle=\bfseries,breakable}
\newtcolorbox{notationBox}[1][Notation]{colback=yellow!5,colframe=yellow!75!black,title={#1},fonttitle=\bfseries,breakable}
\newtcolorbox{verification}[1][Verification]{colback=orange!5!white,colframe=orange!75!black,fonttitle=\bfseries,title=#1}
\newtcolorbox{explanationBox}[1][Explanation]{colback=purple!5!white,colframe=purple!75!black,fonttitle=\bfseries,title=#1}
\newtcolorbox{interpretationBox}[1][Interpretation]{colback=cyan!5!white,colframe=cyan!75!black,fonttitle=\bfseries,title=#1}
\newtcolorbox{explanation}[1][Explanation]{colback=purple!5!white,colframe=purple!75!black,fonttitle=\bfseries,title=#1,breakable}
\newtcolorbox{interpretation}[1][Interpretation]{colback=cyan!5!white,colframe=cyan!75!black,fonttitle=\bfseries,title=#1,breakable}
\newtcolorbox{proof_step}[1][Proof Step]{colback=gray!5!white,colframe=gray!75!black,fonttitle=\bfseries,title=#1,breakable}
\newtcolorbox{experimental}[1][Experimental]{colback=teal!5!white,colframe=teal!75!black,fonttitle=\bfseries,title=#1,breakable}

\newtcolorbox{important}[1][Important]{colback=red!5!white,colframe=red!75!black,title={#1},fonttitle=\bfseries,breakable}
\newtcolorbox{warning}[1][Warning]{colback=orange!5!white,colframe=orange!75!black,title={#1},fonttitle=\bfseries,breakable}
\newtcolorbox{caution}[1][Caution]{colback=yellow!5!white,colframe=yellow!75!black,title={#1},fonttitle=\bfseries,breakable}
\newtcolorbox{highlight}[1][Highlight]{colback=yellow!10!white,colframe=yellow!75!black,title={#1},fonttitle=\bfseries,breakable}
\newtcolorbox{critical}[1][Critical]{colback=red!10!white,colframe=red!75!black,title={#1},fonttitle=\bfseries,breakable}

\newtcolorbox{analysis}[1][Analysis]{colback=blue!5!white,colframe=blue!75!black,title={#1},fonttitle=\bfseries,breakable}
\newtcolorbox{application}[1][Application]{colback=green!5!white,colframe=green!75!black,title={#1},fonttitle=\bfseries,breakable}
\newtcolorbox{experiment}[1][Experiment]{colback=cyan!5!white,colframe=cyan!75!black,title={#1},fonttitle=\bfseries,breakable}
\newtcolorbox{historical}[1][Historical]{colback=brown!5!white,colframe=brown!75!black,title={#1},fonttitle=\bfseries,breakable}
\newtcolorbox{numerical}[1][Numerical]{colback=gray!5!white,colframe=gray!75!black,title={#1},fonttitle=\bfseries,breakable}
\newtcolorbox{overview}[1][Overview]{colback=blue!5!white,colframe=blue!75!black,title={#1},fonttitle=\bfseries,breakable}
\newtcolorbox{speculation}[1][Speculation]{colback=purple!5!white,colframe=purple!75!black,title={#1},fonttitle=\bfseries,breakable}
\newtcolorbox{question}[1][Question]{colback=orange!5!white,colframe=orange!75!black,title={#1},fonttitle=\bfseries,breakable}
\newtcolorbox{method}[1][Method]{colback=teal!5!white,colframe=teal!75!black,title={#1},fonttitle=\bfseries,breakable}
\newtcolorbox{correct}[1][Correct]{colback=green!10!white,colframe=green!75!black,title={#1},fonttitle=\bfseries,breakable}
\newtcolorbox{units}[1][Units]{colback=gray!5!white,colframe=gray!75!black,title={#1},fonttitle=\bfseries,breakable}
\newtcolorbox{achievement}[1][Achievement]{colback=gold!5!white,colframe=orange!75!black,title={#1},fonttitle=\bfseries,breakable}
\newtcolorbox{equivalence}[1][Equivalence]{colback=cyan!5!white,colframe=cyan!75!black,title={#1},fonttitle=\bfseries,breakable}
\newtcolorbox{dimensional}[1][Dimensional Analysis]{colback=purple!5!white,colframe=purple!75!black,title={#1},fonttitle=\bfseries,breakable}

% === ADDITIONAL SIMPLE ENVIRONMENTS ===
\newenvironment{treatise}{\begin{quote}}{\end{quote}}
\newenvironment{gemeinsam}{\begin{quote}}{\end{quote}}
\newenvironment{vergleich}{\begin{quote}}{\end{quote}}
\newenvironment{vorteil}{\begin{quote}}{\end{quote}}
\newenvironment{common}{\begin{quote}}{\end{quote}}
\newenvironment{comparison}{\begin{quote}}{\end{quote}}
\newenvironment{advantage}{\begin{quote}}{\end{quote}}
\newenvironment{quantum}{\begin{quote}}{\end{quote}}

% === LAYOUT SETTINGS ===
\raggedbottom
\usepackage{environ}
\let\oldtabular\tabular
\let\endoldtabular\endtabular

\newenvironment{scaledtable}[1][0.85]{%
	\begingroup\footnotesize\setlength{\LTleft}{0pt}\setlength{\LTright}{0pt}%
}{%
	\endgroup%
}

\newcommand{\widetable}[1]{\resizebox{\textwidth}{!}{#1}}

% === TABLE OF CONTENTS FORMATTING ===
\renewcommand{\cftsecfont}{\color{blue}}
\renewcommand{\cftsubsecfont}{\color{blue}}
\renewcommand{\cftsecpagefont}{\color{blue}}
\renewcommand{\cftsubsecpagefont}{\color{blue}}
\renewcommand{\cfttoctitlefont}{\huge\bfseries\color{blue}}

% === DEFAULT HEADER AND FOOTER ===
\pagestyle{fancy}
\fancyhf{}
\fancyhead[L]{\textsc{T0 Theory}}
\fancyhead[R]{\textsc{J. Pascher}}
\fancyfoot[C]{\thepage}

% ==============================================================================
% End of Shared Preamble for English
% ==============================================================================
%   \begin{document}
%   ...
%   \end{document}
%
% ==============================================================================

% =============================================================================
% SECTION 1: Encoding and Language
% =============================================================================
\usepackage[utf8]{inputenc}
\usepackage[T1]{fontenc}
\usepackage[ngerman]{babel}
\usepackage{lmodern}

% =============================================================================
% SECTION 2: Page Geometry
% =============================================================================
\usepackage[a4paper, left=2.5cm, right=2.5cm, top=2.5cm, bottom=3.5cm]{geometry}
\setlength{\headheight}{15pt}

% =============================================================================
% SECTION 3: Mathematics and Physics
% =============================================================================
\usepackage{amsmath,amssymb,amsfonts,amsthm}
\usepackage{mathtools}
\usepackage{physics}
\usepackage{siunitx}
\sisetup{
    locale=US,
    group-separator={,},
    output-decimal-marker={.},
    per-mode=symbol
}

% =============================================================================
% SECTION 4: Graphics and Tables
% =============================================================================
\usepackage{graphicx}
\usepackage[table,xcdraw]{xcolor}
\usepackage{tikz}
\usetikzlibrary{arrows.meta,positioning,shapes.geometric,decorations.pathmorphing,patterns,shapes.arrows,intersections}
\usepackage{pgfplots}
\pgfplotsset{compat=1.18}
\usepackage[most]{tcolorbox}
\tcbuselibrary{breakable}
\usepackage{booktabs}
\usepackage{array}
\usepackage{longtable}
\usepackage{float}
\usepackage{adjustbox}
\usepackage{rotating}
\usepackage{tabularx}
\usepackage{makecell}
\usepackage{multirow}

% =============================================================================
% SECTION 5: Document Formatting
% =============================================================================
\usepackage{fancyhdr}
\renewcommand{\headrulewidth}{0.4pt}
\renewcommand{\footrulewidth}{0.4pt}
\usepackage{tocloft}
\usepackage{hyperref}
\hypersetup{
  colorlinks=true,
  linkcolor=black,
  citecolor=black,
  urlcolor=black,
  breaklinks=true,
  bookmarksnumbered=true,
  unicode=true
}
\usepackage{bookmark}
\usepackage{cleveref}

% Table of contents: only show chapters (not sections/subsections)
\setcounter{tocdepth}{3}  % Show sections, subsections, and subsubsections
\usepackage{microtype}
\usepackage{enumitem}
\usepackage{setspace}
\usepackage{ragged2e}
\usepackage{multicol}

% =============================================================================
% SECTION 6: Code and Algorithms
% =============================================================================
\usepackage{algorithm}
\usepackage{algorithmic}
\usepackage{listings}
\lstset{
  basicstyle=\ttfamily\footnotesize,
  breaklines=true,
  breakatwhitespace=true,
  columns=flexible,
  keepspaces=true,
  showstringspaces=false,
  frame=single,
  xleftmargin=0pt,
  xrightmargin=0pt
}
\usepackage{mdframed}

% =============================================================================
% SECTION 7: Additional Packages
% =============================================================================
\usepackage{pdflscape}
\usepackage{braket}
\usepackage{cancel}
\usepackage{caption}
\usepackage{csquotes}
\usepackage{gensymb}
\usepackage{hyphenat}
\usepackage{textcomp}
\usepackage{textgreek}
\usepackage{upgreek}
\usepackage{url}
\usepackage{slashed}
\usepackage{bm}
\usepackage{newunicodechar}

% =============================================================================
% SECTION 8: Citation Commands (Compatibility)
% =============================================================================
\providecommand{\citep}[1]{\cite{#1}}
\providecommand{\citet}[1]{\cite{#1}}

% =============================================================================
% SECTION 9: Colors
% =============================================================================
\definecolor{gold}{RGB}{255,215,0}
\definecolor{blue}{rgb}{0,0,1}
\definecolor{boxgray}{RGB}{240,240,240}
\definecolor{deepblue}{RGB}{0,0,127}
\definecolor{deepgreen}{RGB}{0,127,0}
\definecolor{deepred}{RGB}{191,0,0}
\definecolor{t0blue}{RGB}{33,150,243}
\definecolor{t0green}{RGB}{76,175,80}
\definecolor{t0orange}{RGB}{255,152,0}
\definecolor{t0purple}{RGB}{156,39,176}
\definecolor{t0red}{RGB}{244,67,54}
\definecolor{t0yellow}{RGB}{255,204,0}

% =============================================================================
% SECTION 10: Column Types
% =============================================================================
\newcolumntype{L}[1]{>{\raggedright\arraybackslash}p{#1}}
\newcolumntype{C}[1]{>{\centering\arraybackslash}p{#1}}

% =============================================================================
% SECTION 11: Unicode Character Mappings
% =============================================================================
\newunicodechar{ħ}{$\hbar$}
\newunicodechar{↔}{$\leftrightarrow$}
\newunicodechar{⇐}{$\Leftarrow$}
\newunicodechar{⇒}{$\Rightarrow$}
\newunicodechar{⇔}{$\Leftrightarrow$}
\newunicodechar{∂}{$\partial$}
\newunicodechar{∅}{$\emptyset$}
\newunicodechar{∇}{$\nabla$}
\newunicodechar{∈}{$\in$}
\newunicodechar{∉}{$\notin$}
\newunicodechar{∏}{$\prod$}
\newunicodechar{∑}{$\sum$}
% Note: √ is mapped to an empty sqrt; use \sqrt{x} for proper usage
\newunicodechar{√}{\ensuremath{\sqrt{}}}
\newunicodechar{∝}{$\propto$}
\newunicodechar{∞}{$\infty$}
\newunicodechar{∩}{$\cap$}
\newunicodechar{∪}{$\cup$}
\newunicodechar{∫}{$\int$}
\newunicodechar{≈}{$\approx$}
\newunicodechar{≠}{$\neq$}
\newunicodechar{≤}{$\leq$}
\newunicodechar{≥}{$\geq$}
\newunicodechar{ξ}{\ensuremath{\xi}}
\newunicodechar{μ}{\ensuremath{\mu}}
\newunicodechar{ψ}{\ensuremath{\psi}}
\newunicodechar{φ}{\ensuremath{\phi}}
\newunicodechar{π}{\ensuremath{\pi}}
\newunicodechar{λ}{\ensuremath{\lambda}}
\newunicodechar{Δ}{\ensuremath{\Delta}}

% =============================================================================
% SECTION 12: Hyperref Settings
% =============================================================================
\hypersetup{
    colorlinks=true,
    linkcolor=blue,
    citecolor=blue,
    urlcolor=blue,
    breaklinks=true,
    bookmarksnumbered=true,
    pdfstartview=FitH
}

% =============================================================================
% SECTION 13: Theorem Environments (English)
% =============================================================================
\theoremstyle{plain}
\newtheorem{theorem}{Theorem}[section]
\newtheorem{lemma}[theorem]{Lemma}
\newtheorem{proposition}[theorem]{Proposition}
\newtheorem{corollary}[theorem]{Corollary}

\theoremstyle{definition}
\newtheorem{definition}[theorem]{Definition}
\newtheorem{example}[theorem]{Example}
\newtheorem{insight}[theorem]{Insight}
\newtheorem{discovery}[theorem]{Discovery}
% \newtheorem{erkenntnis}[theorem]{Insight}  % Commented out - conflicts with tcolorbox environment below

\theoremstyle{remark}
\newtheorem{remark}[theorem]{Remark}
\newtheorem{axiom}{Axiom}
\newtheorem{principle}{Principle}
\newtheorem{bemerkung}[theorem]{Remark}
\newtheorem{warnung}[theorem]{Warning}

% =============================================================================
% SECTION 14: T0-Specific Commands
% =============================================================================

% --- Core T0 Fields ---
\newcommand{\Tfield}{T(x,t)}
\providecommand{\Tfieldt}{T(\vec{x},t)}
\newcommand{\Efield}{E(x,t)}
\newcommand{\mfield}{m(x,t)}
\providecommand{\vecx}{\vec{x}}

% --- Lagrangian ---
\newcommand{\Lag}{\mathcal{L}}
\newcommand{\calL}{\mathcal{L}}

% --- Greek Letters and Constants ---
\newcommand{\alphaem}{\alpha}
\newcommand{\betaT}{\beta_T}
\newcommand{\xiT}{\xi}
\newcommand{\xipar}{\xi}

% --- Energy and Planck Units ---
\newcommand{\Ezero}{E_0}
\newcommand{\EPlanck}{E_{\text{Pl}}}
\newcommand{\Mpl}{M_{\text{Pl}}}
\newcommand{\mP}{m_{\text{P}}}
\newcommand{\lP}{\ell_{\text{P}}}
\newcommand{\tP}{t_{\text{P}}}
\newcommand{\LPlanck}{\ell_{\text{Pl}}}
\newcommand{\TPlanck}{t_{\text{Pl}}}

% --- Coupling Constants ---
\newcommand{\Gnat}{G_{\text{nat}}}
\newcommand{\alphaEM}{\alpha_{\text{EM}}}
\newcommand{\alphaSI}{\alpha_{\text{SI}}}
\newcommand{\Hubble}{H_0}
\newcommand{\LCDM}{\Lambda\text{CDM}}
\newcommand{\natunits}{(nat. units)}

% --- T0 Model Parameters ---
\newcommand{\xigeom}{\xi_{\mathrm{geom}}}
\newcommand{\rzero}{r_{0}}
\newcommand{\xirat}{\xi_{\mathrm{rat}}}
\newcommand{\tzero}{t_{0}}
\newcommand{\Lambdat}{\Lambda_{\mathrm{t}}}
\newcommand{\EP}{E_{\mathrm{P}}}
\newcommand{\Emu}{E_{\mu}}
\newcommand{\Ee}{E_{e}}
\newcommand{\Etau}{E_{\tau}}
\newcommand{\alphafine}{\alpha_{\mathrm{fine}}}
\newcommand{\alphal}{\alpha_{\ell}}
\newcommand{\Lzero}{\ell_{0}}
\newcommand{\Lp}{\ell_{\mathrm{P}}}

% --- Additional T0 Commands ---
\newcommand{\Kfrak}{K_{\text{frak}}}
\newcommand{\Dfrak}{D_{\text{frak}}}
\newcommand{\betapar}{\beta_T}
\newcommand{\alphapar}{\alpha}
\newcommand{\deltafield}{\delta \phi}
\newcommand{\deltam}{\delta m}
\newcommand{\deltaE}{\delta E}
\newcommand{\Exi}{E_{\xi}}
\newcommand{\Lxi}{\ell_{\xi}}
\newcommand{\rhoCMB}{\rho_{\text{CMB}}}
\newcommand{\rhoCasimir}{\rho_{\text{Casimir}}}
\newcommand{\Leff}{L_{\text{eff}}}
\newcommand{\CQCD}{C_{\mathrm{QCD}}}
\newcommand{\Kspec}{K_{\mathrm{spec}}}
\newcommand{\Tzero}{\ensuremath{T_0}}
\newcommand{\Eabs}{E_{\text{abs}}}
\newcommand{\taupar}{\tau}

% --- Provided Commands (may be redefined elsewhere) ---
\providecommand{\xiconst}{\xi_{\text{const}}}
\providecommand{\DhiggsT}{D_{\text{Higgs-T}}}
\providecommand{\rhoE}{\rho_{E}}
\providecommand{\Echar}{E_{\text{char}}}
\providecommand{\kfrac}{k_{\text{frac}}}
\providecommand{\alphaEMSI}{\alpha_{\text{EM,SI}}}
\providecommand{\alphaEMnat}{\alpha_{\text{EM,nat}}}
\providecommand{\betaTSI}{\beta_{T,\text{SI}}}
\providecommand{\betaTnat}{\beta_{T,\text{nat}}}
\providecommand{\Gsi}{G_{\text{SI}}}
\providecommand{\xiparSI}{\xi_{\text{SI}}}
\providecommand{\xiparnat}{\xi_{\text{nat}}}
\providecommand{\meff}{m_{\text{eff}}}
\providecommand{\Tzerot}{T_{0}(t)}
\providecommand{\mzerot}{m_{0}(t)}
\providecommand{\Ezeroabs}{E_{0,\text{abs}}}
\providecommand{\Epar}{E_{\text{par}}}
\providecommand{\Lnat}{\ell_{\text{nat}}}
\providecommand{\Tnat}{T_{\text{nat}}}
\providecommand{\xifrak}{\xi_{\text{frac}}}
\providecommand{\Tfrak}{T_{\text{frac}}}
\providecommand{\mfrak}{m_{\text{frac}}}
\providecommand{\Dfrac}{D_{\text{frac}}}
\providecommand{\EphotSI}{E_{\gamma,\text{SI}}}
\providecommand{\EphotNat}{E_{\gamma,\text{nat}}}
\providecommand{\Eabsint}{E_{\text{abs,int}}}
\providecommand{\mphoton}{m_{\gamma}}
\providecommand{\Evis}{E_{\text{vis}}}
\providecommand{\Cto}{C_{T0}}
\providecommand{\mytimes}{\times}
\providecommand{\lambdah}{\lambda_h}
\providecommand{\checkmarkx}{\checkmark}
\providecommand{\Enorm}{E_{\text{norm}}}
\providecommand{\Tobs}{T_{\text{obs}}}
\providecommand{\mobs}{m_{\text{obs}}}
\providecommand{\Eobs}{E_{\text{obs}}}
\providecommand{\Lobs}{\ell_{\text{obs}}}
\providecommand{\xobs}{\xi_{\text{obs}}}
\providecommand{\calE}{\mathcal{E}}
\providecommand{\calT}{\mathcal{T}}
\providecommand{\calM}{\mathcal{M}}
\providecommand{\alphag}{\alpha_g}
\providecommand{\Tmax}{T_{\text{max}}}
\providecommand{\mmin}{m_{\text{min}}}
\providecommand{\Lmax}{\ell_{\text{max}}}
\providecommand{\Emin}{E_{\text{min}}}
\providecommand{\Geff}{G_{\text{eff}}}
\providecommand{\rhoeff}{\rho_{\text{eff}}}
\providecommand{\xieff}{\xi_{\text{eff}}}
\providecommand{\Teff}{T_{\text{eff}}}
\providecommand{\hPlanck}{h}
\providecommand{\kB}{k_B}
\providecommand{\muB}{\mu_B}
\providecommand{\lambdaC}{\lambda_C}
\providecommand{\omegaP}{\omega_P}
\providecommand{\rhoP}{\rho_P}
\providecommand{\Tref}{T_{\text{ref}}}
\providecommand{\Eref}{E_{\text{ref}}}
\providecommand{\mref}{m_{\text{ref}}}
\providecommand{\Lref}{\ell_{\text{ref}}}
\providecommand{\xikonst}{\xi_0}
\providecommand{\Phiphoton}{\Phi_{\gamma}}
\providecommand{\etavis}{\eta_{\text{vis}}}
\providecommand{\pichar}{\pi}
\providecommand{\primrel}{\mathcal{P}_{\text{rel}}}
\providecommand{\warningx}{\textcolor{orange}{\textbf{!}}}
\providecommand{\phiT}{\phi_T}
\providecommand{\Lorentz}{\Lambda}
\providecommand{\Cconv}{C_{\text{conv}}}
\providecommand{\Df}{\Delta f}
\providecommand{\lambdazero}{\lambda_0}
\providecommand{\myapprox}{\approx}
\providecommand{\checked}{\checkmark}
\providecommand{\alphaWSI}{\alpha_W^{\text{SI}}}
\providecommand{\alphaWnat}{\alpha_W^{\text{nat}}}
\providecommand{\vect}[1]{\vec{#1}}
\providecommand{\Rzero}{R_0}
\providecommand{\Riem}{\mathcal{R}}
\providecommand{\nuzero}{\nu_0}
\providecommand{\mypi}{\pi}

% =============================================================================
% SECTION 15: tcolorbox Styles and Environments
% =============================================================================

% --- Predefined Styles ---
\tcbset{
    keyresult/.style={
        colback=blue!5!white,
        colframe=blue!75!black,
        title=Key Result,
        fonttitle=\bfseries
    },
    foundation/.style={
        colback=green!5!white,
        colframe=green!75!black,
        title=Foundation,
        fonttitle=\bfseries
    },
    alternative/.style={
        colback=orange!5!white,
        colframe=orange!75!black,
        title=Alternative,
        fonttitle=\bfseries
    },
    warningbox/.style={
        colback=red!5!white,
        colframe=red!75!black,
        title=Warning,
        fonttitle=\bfseries
    }
}

% --- Core Environments ---
\newtcolorbox{keyresultbox}[1][]{colback=blue!5!white,colframe=blue!75!black,fonttitle=\bfseries,title={#1},breakable}
\newtcolorbox{keyresult}[1][Key Result]{colback=blue!5!white,colframe=blue!75!black,fonttitle=\bfseries,title={#1},breakable}
\newtcolorbox{foundationbox}[1][]{colback=green!5!white,colframe=green!75!black,fonttitle=\bfseries,title={#1},breakable}
\newtcolorbox{foundation}[1][Foundation]{colback=green!5!white,colframe=green!75!black,fonttitle=\bfseries,title={#1},breakable}
\newtcolorbox{alternativebox}[1][]{colback=orange!5!white,colframe=orange!75!black,fonttitle=\bfseries,title={#1},breakable}
\newtcolorbox{warningboxenv}[1][]{colback=red!5!white,colframe=red!75!black,fonttitle=\bfseries,title={#1},breakable}

% --- Formula Environments ---
\newtcolorbox{fundamental}[1][]{
    colback=boxgray,
    colframe=t0blue,
    fonttitle=\bfseries,
    title=#1,
    sharp corners,
    boxrule=2pt
}

\newtcolorbox{newperspective}[1][]{
    colback=red!5!white,
    colframe=t0red,
    fonttitle=\bfseries,
    title=#1,
    sharp corners,
    boxrule=2pt
}

\newtcolorbox{formula}[1][]{
    colback=blue!5!white,
    colframe=blue!75!black,
    fonttitle=\bfseries,
    title=#1
}

\newtcolorbox{result}[1][]{
    colback=green!5!white,
    colframe=green!75!black,
    fonttitle=\bfseries,
    title=#1
}

\newtcolorbox{derivation}[1][]{
    colback=green!5!white,
    colframe=green!75!black,
    title=#1,
    fonttitle=\bfseries,
    breakable
}

\newtcolorbox{summary}[1][]{
    colback=gray!10!white,
    colframe=gray!75!black,
    title=#1,
    fonttitle=\bfseries,
    breakable
}

\newtcolorbox{comparison}[1][]{
    colback=purple!5!white,
    colframe=purple!75!black,
    title=#1,
    fonttitle=\bfseries,
    breakable
}

\newtcolorbox{relation}[1][]{
    colback=cyan!5!white,
    colframe=cyan!75!black,
    title=#1,
    fonttitle=\bfseries,
    breakable
}

\newtcolorbox{principleBox}[1][]{
    colback=yellow!5!white,
    colframe=yellow!75!black,
    title=#1,
    fonttitle=\bfseries,
    breakable
}

% --- Insight and Discovery Environments ---
\newtcolorbox{insightBox}[1][]{colback=blue!5,colframe=t0blue,title={#1},fonttitle=\bfseries,breakable}
\newtcolorbox{discoveryBox}[1][]{colback=green!5,colframe=t0green,title={#1},fonttitle=\bfseries,breakable}
\newtcolorbox{revelation}[1][]{colback=red!5,colframe=t0red,title={#1},fonttitle=\bfseries,breakable}
\newtcolorbox{keypoint}[1][]{colback=blue!5,colframe=t0blue,title={#1},fonttitle=\bfseries,breakable}
\newtcolorbox{evidence}[1][]{colback=green!5,colframe=t0green,title={#1},fonttitle=\bfseries,breakable}
\newtcolorbox{conclusionBox}[1][]{colback=gray!5,colframe=gray,title={#1},fonttitle=\bfseries,breakable}
\newtcolorbox{significance}[1][]{colback=yellow!5,colframe=orange,title={#1},fonttitle=\bfseries,breakable}
\newtcolorbox{philosophical}[1][]{colback=purple!5,colframe=purple,title={#1},fonttitle=\bfseries,breakable}
\newtcolorbox{implicationBox}[1][]{colback=cyan!5,colframe=cyan,title={#1},fonttitle=\bfseries,breakable}
\newtcolorbox{perspectiveBox}[1][]{colback=blue!5,colframe=t0blue,title={#1},fonttitle=\bfseries,breakable}
\newtcolorbox{revolutionary}[1][]{colback=red!5,colframe=t0red,title={#1},fonttitle=\bfseries,breakable}

% --- Technical Environments ---
\newtcolorbox{technical}[1][]{colback=gray!5,colframe=gray!75!black,title={#1},fonttitle=\bfseries,breakable}
\newtcolorbox{technicalBox}[1][]{colback=gray!5,colframe=gray!75!black,title={#1},fonttitle=\bfseries,breakable}
\newtcolorbox{notationBox}[1][]{colback=yellow!5,colframe=yellow!75!black,title={#1},fonttitle=\bfseries,breakable}
\newtcolorbox{verification}[1][]{colback=orange!5!white,colframe=orange!75!black,fonttitle=\bfseries,title=#1}
\newtcolorbox{explanationBox}[1][]{colback=purple!5!white,colframe=purple!75!black,fonttitle=\bfseries,title=#1}
\newtcolorbox{interpretationBox}[1][]{colback=cyan!5!white,colframe=cyan!75!black,fonttitle=\bfseries,title=#1}
\newtcolorbox{explanation}[1][]{colback=purple!5!white,colframe=purple!75!black,fonttitle=\bfseries,title=#1,breakable}
\newtcolorbox{interpretation}[1][]{colback=cyan!5!white,colframe=cyan!75!black,fonttitle=\bfseries,title=#1,breakable}
\newtcolorbox{proof_step}[1][]{colback=gray!5!white,colframe=gray!75!black,fonttitle=\bfseries,title=#1,breakable}
\newtcolorbox{experimental}[1][]{colback=teal!5!white,colframe=teal!75!black,fonttitle=\bfseries,title=#1,breakable}

% --- Warning and Alert Environments ---
\newtcolorbox{important}[1][]{colback=red!5!white,colframe=red!75!black,title={#1},fonttitle=\bfseries,breakable}
\newtcolorbox{warning}[1][]{colback=orange!5!white,colframe=orange!75!black,title={#1},fonttitle=\bfseries,breakable}
\newtcolorbox{caution}[1][]{colback=yellow!5!white,colframe=yellow!75!black,title={#1},fonttitle=\bfseries,breakable}
\newtcolorbox{highlight}[1][]{colback=yellow!10!white,colframe=yellow!75!black,title={#1},fonttitle=\bfseries,breakable}

% --- Additional German-specific Environments for Matsas documents ---
\newtcolorbox{literatur}[1][Literatur]{colback=blue!5!white,colframe=blue!75!black,title={#1},fonttitle=\bfseries,breakable}
\newtcolorbox{zusammenfassung}[1][Zusammenfassung]{colback=green!5!white,colframe=green!75!black,title={#1},fonttitle=\bfseries,breakable}
\newtcolorbox{frage}[1][Frage]{colback=orange!5!white,colframe=orange!75!black,title={#1},fonttitle=\bfseries,breakable}
\newtcolorbox{erkenntnis}[1][Erkenntnis]{colback=purple!5!white,colframe=purple!75!black,title={#1},fonttitle=\bfseries,breakable}
\newtcolorbox{critical}[1][]{colback=red!10!white,colframe=red!75!black,title={#1},fonttitle=\bfseries,breakable}

% --- Analysis and Application Environments ---
\newtcolorbox{analysis}[1][]{colback=blue!5!white,colframe=blue!75!black,title={#1},fonttitle=\bfseries,breakable}
\newtcolorbox{application}[1][]{colback=green!5!white,colframe=green!75!black,title={#1},fonttitle=\bfseries,breakable}
\newtcolorbox{experiment}[1][]{colback=cyan!5!white,colframe=cyan!75!black,title={#1},fonttitle=\bfseries,breakable}
\newtcolorbox{historical}[1][]{colback=brown!5!white,colframe=brown!75!black,title={#1},fonttitle=\bfseries,breakable}
\newtcolorbox{numerical}[1][]{colback=gray!5!white,colframe=gray!75!black,title={#1},fonttitle=\bfseries,breakable}
\newtcolorbox{overview}[1][]{colback=blue!5!white,colframe=blue!75!black,title={#1},fonttitle=\bfseries,breakable}
\newtcolorbox{speculation}[1][]{colback=purple!5!white,colframe=purple!75!black,title={#1},fonttitle=\bfseries,breakable}
\newtcolorbox{question}[1][]{colback=orange!5!white,colframe=orange!75!black,title={#1},fonttitle=\bfseries,breakable}
\newtcolorbox{method}[1][]{colback=teal!5!white,colframe=teal!75!black,title={#1},fonttitle=\bfseries,breakable}
\newtcolorbox{correct}[1][]{colback=green!10!white,colframe=green!75!black,title={#1},fonttitle=\bfseries,breakable}
\newtcolorbox{units}[1][]{colback=gray!5!white,colframe=gray!75!black,title={#1},fonttitle=\bfseries,breakable}
\newtcolorbox{achievement}[1][]{colback=gold!5!white,colframe=orange!75!black,title={#1},fonttitle=\bfseries,breakable}
\newtcolorbox{equivalence}[1][]{colback=cyan!5!white,colframe=cyan!75!black,title={#1},fonttitle=\bfseries,breakable}
\newtcolorbox{dimensional}[1][]{colback=purple!5!white,colframe=purple!75!black,title={#1},fonttitle=\bfseries,breakable}

% --- Physics-specific Environments ---
\newtcolorbox{photon}[1][]{colback=yellow!5!white,colframe=yellow!75!black,title={#1},fonttitle=\bfseries,breakable}
\newtcolorbox{neutrino}[1][]{colback=blue!5!white,colframe=blue!75!black,title={#1},fonttitle=\bfseries,breakable}
\newtcolorbox{revolution}[1][]{colback=red!5!white,colframe=red!75!black,title={#1},fonttitle=\bfseries,breakable}
\newtcolorbox{t0box}[1][]{colback=blue!5!white,colframe=t0blue,title={#1},fonttitle=\bfseries,breakable}
\newtcolorbox{documentbox}[1][]{colback=gray!5!white,colframe=gray!75!black,title={#1},fonttitle=\bfseries,breakable}
\newtcolorbox{sibox}[1][]{colback=green!5!white,colframe=green!75!black,title={#1},fonttitle=\bfseries,breakable}
\newtcolorbox{smbox}[1][]{colback=blue!5!white,colframe=blue!75!black,title={#1},fonttitle=\bfseries,breakable}
\newtcolorbox{pvbox}[1][]{colback=purple!5!white,colframe=purple!75!black,title={#1},fonttitle=\bfseries,breakable}
\newtcolorbox{koidebox}[1][]{colback=orange!5!white,colframe=orange!75!black,title={#1},fonttitle=\bfseries,breakable}

% --- German Compatibility Environments ---
\newtcolorbox{formel}[1][]{colback=blue!5!white,colframe=blue!75!black,title={#1},fonttitle=\bfseries,breakable}
\newtcolorbox{schluessel}[1][]{colback=blue!5!white,colframe=blue!75!black,title={#1},fonttitle=\bfseries,breakable}
\newtcolorbox{wichtig}[1][]{colback=red!5!white,colframe=red!75!black,title={#1},fonttitle=\bfseries,breakable}
\newtcolorbox{vorsicht}[1][]{colback=orange!5!white,colframe=orange!75!black,title={#1},fonttitle=\bfseries,breakable}
\newtcolorbox{revolutionaer}[1][]{colback=red!5!white,colframe=red!75!black,title={#1},fonttitle=\bfseries,breakable}
\newtcolorbox{numerisch}[1][]{colback=gray!5!white,colframe=gray!75!black,title={#1},fonttitle=\bfseries,breakable}
\newtcolorbox{experimentell}[1][]{colback=cyan!5!white,colframe=cyan!75!black,title={#1},fonttitle=\bfseries,breakable}
\newtcolorbox{anwendung}[1][]{colback=green!5!white,colframe=green!75!black,title={#1},fonttitle=\bfseries,breakable}
\newtcolorbox{alternative}[1][]{colback=orange!5!white,colframe=orange!75!black,title={#1},fonttitle=\bfseries,breakable}
\newtcolorbox{beziehung}[1][]{colback=cyan!5!white,colframe=cyan!75!black,title={#1},fonttitle=\bfseries,breakable}
\newtcolorbox{folgerung}[1][]{colback=green!5!white,colframe=green!75!black,title={#1},fonttitle=\bfseries,breakable}
\newtcolorbox{abhandlung}[1][]{colback=gray!5!white,colframe=gray!75!black,title={#1},fonttitle=\bfseries,breakable}
\newtcolorbox{prinzipBox}[1][]{colback=blue!5!white,colframe=blue!75!black,title={#1},fonttitle=\bfseries,breakable}
\newtcolorbox{prinzip}[1][]{colback=blue!5!white,colframe=blue!75!black,title={#1},fonttitle=\bfseries,breakable}
\newtcolorbox{beweis}[1][]{colback=gray!5!white,colframe=gray!75!black,title={#1},fonttitle=\bfseries,breakable}
\newtcolorbox{key}[2][]{colback=blue!5!white,colframe=blue!75!black,title={#2},fonttitle=\bfseries,breakable}
\newtcolorbox{category}[1][]{colback=purple!5!white,colframe=purple!75!black,title={#1},fonttitle=\bfseries,breakable}

% =============================================================================
% SECTION 16: Additional Simple Environments
% =============================================================================
\newenvironment{treatise}{\begin{quote}}{\end{quote}}
\newenvironment{gemeinsam}{\begin{quote}}{\end{quote}}
\newenvironment{vergleich}{\begin{quote}}{\end{quote}}
\newenvironment{vorteil}{\begin{quote}}{\end{quote}}
\newenvironment{quantum}{\begin{quote}}{\end{quote}}

% =============================================================================
% SECTION 17: Layout Settings (Kindle-compatible)
% =============================================================================
\sloppy  % Allow more flexible line breaking
\hfuzz=65pt  % Suppress overfull warnings up to 65pt (Kindle compatibility)
\vfuzz=65pt  
\tolerance=9999  % High tolerance for bad line breaks
\emergencystretch=3em  % Extra stretch to avoid overfull boxes
\hbadness=10000  % Suppress underfull box warnings
\raggedbottom

% Environment for wide tables/longtables that need scaling
\newenvironment{scaledtable}[1][0.85]{%
  \begingroup\footnotesize\setlength{\LTleft}{0pt}\setlength{\LTright}{0pt}%
}{%
  \endgroup%
}

% Command for inline table scaling
\newcommand{\widetable}[1]{\resizebox{\textwidth}{!}{#1}}

% =============================================================================
% SECTION 18: Table of Contents Formatting
% =============================================================================
\renewcommand{\cftsecfont}{\color{blue}}
\renewcommand{\cftsubsecfont}{\color{blue}}
\renewcommand{\cftsecpagefont}{\color{blue}}
\renewcommand{\cftsubsecpagefont}{\color{blue}}
\renewcommand{\cfttoctitlefont}{\huge\bfseries\color{blue}}

% =============================================================================
% SECTION 19: Default Header and Footer
% =============================================================================
\pagestyle{fancy}
\fancyhf{}
\fancyhead[L]{\textsc{T0 Theory}}
\fancyhead[R]{\textsc{J. Pascher}}
\fancyfoot[C]{\thepage}

% ==============================================================================
% End of Shared Preamble
% ==============================================================================


\title{Parameter-Systemabhängigkeit im T0-Modell: \\
		SI- vs. natürliche Einheiten und die Gefahr \\
		der direkten Übertragung von Formelsymbolen}
\author{}
\date{}

\begin{document}

	
	\title{Parameter-Systemabhängigkeit im T0-Modell: \\
		SI- vs. natürliche Einheiten und die Gefahr \\
		der direkten Übertragung von Formelsymbolen}
	\author{}
	\date{}
	
	\maketitle
	
	\begin{abstract}
		Diese Arbeit analysiert systematisch die Parameterabhängigkeit zwischen SI-Einheiten und natürlichen T0-Modell-Einheiten und offenbart, dass fundamentale Parameter wie $\xipar$, $\alpha_{\text{EM}}$, $\beta_{\text{T}}$ und Yukawa-Kopplungen dramatisch verschiedene numerische Werte in verschiedenen Einheitensystemen haben. Durch detaillierte Berechnungen demonstrieren wir, dass direkte Übertragung von Parameterwerten zwischen Systemen zu Fehlern führt, die mehrere Größenordnungen umspannen. Die Analyse erstreckt sich über spezifische Parameter hinaus zur Etablierung universeller Transformationsregeln und liefert kritische Warnungen gegen naive Parameterübertragung. Diese Arbeit etabliert, dass die scheinbaren Inkonsistenzen in T0-Modell-Parametern tatsächlich systematische Einheitensystem-Abhängigkeiten sind, die sorgfältige Transformationsprotokolle für experimentelle Verifikation erfordern.
	\end{abstract}
	
	\tableofcontents
	
	\section{Einleitung}
	\label{sec:einleitung}
	
	\subsection{Das Parameter-Übertragungsproblem}
	\label{subsec:parameter_problem}
	
	Das T0-Modell, formuliert in natürlichen Einheiten wo $\hbar = c = G = k_B = \alpha_{\text{EM}} = \alpha_{\text{W}} = \beta_{\text{T}} = 1$, präsentiert eine fundamentale Herausforderung beim Vergleich mit experimentellen Daten, die in SI-Einheiten ausgedrückt sind. Diese Arbeit demonstriert, dass die scheinbaren Inkonsistenzen zwischen T0-Modell-Vorhersagen und experimentellen Beobachtungen keine physikalischen Widersprüche sind, sondern systematische Einheitensystem-Abhängigkeiten.
	
	Die Kernerkenntnis ist, dass Parameter wie $\xipar$, $\alpha_{\text{EM}}$ und $\beta_{\text{T}}$ fundamental verschiedene Größen repräsentieren, wenn sie in verschiedenen Einheitensystemen ausgedrückt werden:
	
	$$\xipar_{\text{SI}} \neq \xipar_{\text{nat}}, \quad \alphaEMSI \neq \alphaEMnat, \quad \betaTSI \neq \betaTnat$$
	
	\subsection{Umfang und Methodik}
	\label{subsec:umfang}
	
	Diese Analyse umfasst:
	\begin{itemize}
		\item Systematische Berechnung von Parameterverhältnissen zwischen SI- und T0-natürlichen Einheiten
		\item Demonstration von Transformationsinvarianz für dimensionslose Verhältnisse
		\item Erweiterung auf variable Parameter wie $\xipar$ und Yukawa-Kopplungen
		\item Universelle Warnungen gegen direkte Parameterübertragung
		\item Richtlinien für korrekte experimentelle Vergleichsprotokolle
	\end{itemize}
	
	\section{Der $\xipar$-Parameter: Variabel über Massenskalen}
	\label{sec:xi_parameter}
	
	\subsection{Definition und physikalische Bedeutung}

	
	Der Grundstein des T0-Modells ist die universelle geometrische Konstante, die als fundamentaler Parameter für alle physikalischen Berechnungen dient.
	

		Die universelle geometrische Konstante:
		\begin{equation}
			\xi = \frac{4}{3} \times 10^{-4} = 1,3333... \times 10^{-4}
		\end{equation}

	
	Diese dimensionslose Konstante wird in der gesamten Fundamentale Fraktalgeometrische Feldtheorie (FFGFT, früher T0-Theorie) verwendet, um quantenmechanische und gravitative Phänomene zu verbinden. Sie legt die charakteristische Stärke der Feldwechselwirkungen fest und bildet die Grundlage für einheitliche Feldbeschreibungen.
	

		Für die detaillierte Herleitung und physikalische Begründung dieses Parameters siehe das Dokument "Parameterherleitung" (verfügbar unter:\\ \url{https://github.com/jpascher/T0-Time-Mass-Duality/2/pdf/parameterherleitung_De.pdf}).

	
	Diese geometrische Konstante bestimmt eine charakteristische Energieskala für das $\xi$-Feld:
	
	\begin{equation}
		E_\xi = \frac{1}{\xi} = \frac{3}{4 \times 10^{-4}} = 7500 \text{ (natürliche Einheiten)}
	\end{equation}
	
	Der Parameter $\xipar$ ist auch das Verhältnis des Schwarzschild-Radius zur Planck-Länge:
	
	\begin{equation}
		\xipar = \frac{r_0}{\lP} = \frac{2Gm}{\lP}
		\label{eq:xi_definition}
	\end{equation}
	
\textbf{Entscheidend:} Der Parameter $\xipar$ skaliert mit der Masse des betrachteten Objekts gemäß $\xipar(m) = 2Gm/\lP$. Die Higgs-Masse definiert die fundamentale Referenzskala $\xipar_0 = 1.33 \times 10^{-4}$, auf die alle anderen Massen im T0-Modell normiert werden.
	
	\subsection{Verbindung zur Higgs-Physik}
	\label{subsec:xi_higgs_verbindung}
	
	Das T0-Modell etabliert eine fundamentale Verbindung zwischen $\xipar$ und Higgs-Sektor-Physik durch die Beziehung, die im vollständigen feldtheoretischen Framework hergeleitet wurde.
	
	\begin{equation}
		\xipar = \frac{\lambdah^2 v^2}{16\pichar^3 m_h^2} \approx 1.33 \times 10^{-4}
		\label{eq:xi_higgs_fundamental}
	\end{equation}
	
	wobei:
	\begin{itemize}
		\item $\lambdah \approx 0.13$ (Higgs-Selbstkopplung)
		\item $v \approx 246$ GeV (Higgs-VEV)
		\item $m_h \approx 125$ GeV (Higgs-Masse)
	\end{itemize}
	
	Dies repräsentiert den universellen Skalenparameter, der aus fundamentaler Standardmodell-Physik hervorgeht, während die massenabhängige Form $\xipar = 2Gm/\lP$ auf spezifische Objekte anwendbar ist.
	
	\subsection{$\xipar$-Werte im SI-System}
	\label{subsec:xi_si_werte}
	
	Verwendung von SI-Konstanten:
	\begin{align}
		G &= 6.674 \times 10^{-11} \text{ m}^3/(\text{kg} \cdot \text{s}^2) \\
		\lP &= 1.616 \times 10^{-35} \text{ m}
	\end{align}
	
	Wir berechnen $\xipar_{\text{SI}}$ für verschiedene Objekte:
	
	\begin{table}[htbp]
		\centering
		\begin{adjustbox}{max width=\textwidth}
\begin{tabular}{lcc}
			\toprule
			\textbf{Objekt} & \textbf{Masse} & \textbf{$\xipar_{\text{SI}}$} \\
			\midrule
			Elektron & $9.109 \times 10^{-31}$ kg & $7.52 \times 10^{-7}$ \\
			Proton & $1.673 \times 10^{-27}$ kg & $1.38 \times 10^{-3}$ \\
			Mensch (70 kg) & $7.0 \times 10^{1}$ kg & $6.4 \times 10^{6}$ \\
			Erde & $5.972 \times 10^{24}$ kg & $4.1 \times 10^{28}$ \\
			Sonne & $1.989 \times 10^{30}$ kg & $1.8 \times 10^{38}$ \\
			Planck-Masse & $2.176 \times 10^{-8}$ kg & $2.0$ \\
			\bottomrule
		\end{tabular}
\end{adjustbox}

\bigskip
\clearpage
		\caption{$\xipar$-Werte für verschiedene Objekte in SI-Einheiten}
		\label{tab:xi_si_werte}
	\end{table}
	
	\textbf{Der Parameter $\xipar$ variiert über 46 Größenordnungen!}
	
	\subsection{$\xipar$-Transformation zu T0-natürlichen Einheiten}
	\label{subsec:xi_transformation}
	
	Basierend auf der umfassenden Transformationsanalyse ist der Umwandlungsfaktor zwischen Systemen ungefähr:
	
	$$\frac{\xipar_{\text{nat}}}{\xipar_{\text{SI}}} \approx 4100$$
	
	Dies ergibt T0-natürliche Einheitenwerte:
	
	\begin{table}[htbp]
		\centering
		\begin{adjustbox}{max width=\textwidth}
\begin{tabular}{lcc}
			\toprule
			\textbf{Objekt} & \textbf{$\xipar_{\text{SI}}$} & \textbf{$\xipar_{\text{nat}}$} \\
			\midrule
			Elektron & $7.52 \times 10^{-7}$ & $3.1 \times 10^{-3}$ \\
			Proton & $1.38 \times 10^{-3}$ & $5.7$ \\
			Mensch (70 kg) & $6.4 \times 10^{6}$ & $2.6 \times 10^{10}$ \\
			Sonne & $1.8 \times 10^{38}$ & $7.4 \times 10^{41}$ \\
			\bottomrule
		\end{tabular}
\end{adjustbox}

\bigskip
\clearpage
		\caption{$\xipar$-Transformation zwischen Einheitensystemen}
		\label{tab:xi_transformation}
	\end{table}
	
	\subsection{Invarianz der Verhältnisse}
	\label{subsec:xi_verhaeltnis_invarianz}
	
	\textbf{Kritische Verifikation:} Die Verhältnisse zwischen verschiedenen Objekten bleiben in beiden Systemen identisch:
	
	\begin{align}
		\frac{\xipar_{\text{Sonne},\text{SI}}}{\xipar_{\text{e},\text{SI}}} &= \frac{1.8 \times 10^{38}}{7.52 \times 10^{-7}} = 2.4 \times 10^{44} \\
		\frac{\xipar_{\text{Sonne},\text{nat}}}{\xipar_{\text{e},\text{nat}}} &= \frac{7.4 \times 10^{41}}{3.1 \times 10^{-3}} = 2.4 \times 10^{44}
	\end{align}
	
	\boxed{\text{Verhältnisse sind invariant unter Systemtransformation!}}
	
\section{Die Feinstrukturkonstante $\alpha_{\text{EM}}$}
\label{sec:alpha_em}

\subsection{Die Mystifizierung von 1/137}
\label{subsec:alpha_mystification}

Die Feinstrukturkonstante $\alpha_{\text{EM}} \approx 1/137$ wurde von prominenten Physikern zu einem der größten Mysterien der Physik erklärt:

\begin{itemize}
	\item \textbf{Richard Feynman}: ``Es ist eines der größten verdammten Mysterien der Physik: eine magische Zahl, die zu uns kommt ohne jegliches Verständnis.''
	\item \textbf{Wolfgang Pauli}: ``Wenn ich sterbe, werde ich Gott zwei Fragen stellen: Warum Relativität? Und warum 137? Ich glaube, er wird eine Antwort auf die erste haben.''
	\item \textbf{Max Born}: ``Wenn $\alpha$ größer wäre, könnten keine Moleküle existieren, und es gäbe kein Leben.''
\end{itemize}

\subsection{Die elektromagnetische Dualität als Schlüssel}
\label{subsec:electromagnetic_duality}

Was all diese Aussagen übersehen: Die Feinstrukturkonstante besitzt zwei mathematisch äquivalente Darstellungen, die ihre wahre Natur offenbaren:

\begin{align}
	\alpha_{\text{EM}} &= \frac{e^2}{4\pi\varepsilon_0\hbar c} \quad \text{(Standardform)} \label{eq:alpha_standard}\\
	\alpha_{\text{EM}} &= \frac{e^2 \mu_0 c}{4\pi \hbar} \quad \text{(Duale Form)} \label{eq:alpha_dual}
\end{align}

Diese Äquivalenz beruht auf der Maxwell-Relation $c^2 = \frac{1}{\varepsilon_0\mu_0}$ und offenbart eine fundamentale elektromagnetische Dualität:

\begin{equation}
	\frac{1}{\varepsilon_0 c} = \mu_0 c
	\label{eq:em_duality}
\end{equation}

\subsection{Die doppelte Natur von $\alpha$: Systemabhängig und doch invariant}
\label{subsec:double_nature}

Die Feinstrukturkonstante besitzt eine bemerkenswerte Doppelnatur:

\subsubsection{Als invariantes Verhältnis physikalischer Größen}
\label{subsubsec:invariant_ratio}

Unabhängig vom gewählten Einheitensystem bleibt $\alpha$ als \textbf{Verhältnis} fundamentaler Längen konstant:

\begin{equation}
	\alpha_{\text{EM}} = \frac{r_e}{\lambda_C} = \frac{\text{Klassischer Elektronenradius}}{\text{Compton-Wellenlänge}}
	\label{eq:alpha_ratio_re}
\end{equation}

Ebenso das inverse Verhältnis:

\begin{equation}
	\alpha_{\text{EM}}^{-1} = \frac{a_0}{\lambda_C/2\pi} = \frac{\text{Bohr-Radius}}{\text{Reduzierte Compton-Wellenlänge}} = 137.036...
	\label{eq:alpha_ratio_bohr}
\end{equation}

Diese Verhältnisse sind \textbf{einheitensystem-invariant} -- sie haben denselben numerischen Wert in jedem konsistenten Einheitensystem, da sich die Einheiten im Verhältnis herauskürzen.

\subsubsection{Als systemabhängiger numerischer Wert}
\label{subsubsec:system_dependent}

Gleichzeitig hängt der numerische Wert von $\alpha$ von der Wahl der fundamentalen Einheiten ab:

\begin{itemize}
	\item \textbf{SI-System}: $\alpha = \frac{e^2}{4\pi\varepsilon_0\hbar c} \approx 1/137$
	\item \textbf{Natürliche Einheiten}: $\alpha = 1$ (durch geeignete Wahl)
	\item \textbf{Gaußsche Einheiten}: $\alpha = \frac{e^2}{\hbar c} \approx 1/137$
\end{itemize}

\subsection{Die Systemabhängigkeit von $\alpha$}
\label{subsec:alpha_system_dependency}

Der numerische Wert $\alpha_{\text{EM}} = 1/137$ ist \textbf{ausschließlich im SI-System gültig}:

\begin{align}
	\text{SI-System:} \quad &\alpha_{\text{EM}}^{\text{SI}} = \frac{e^2}{4\pi\varepsilon_0\hbar c} \approx \frac{1}{137.036} \\
	\text{Natürliches Einheitensystem:} \quad &\alpha_{\text{EM}}^{\text{nat}} = 1 \text{ (durch geeignete Wahl der Einheiten)}
\end{align}

\textbf{Transformationsfaktor:}
\begin{equation}
	\frac{\alpha_{\text{EM}}^{\text{nat}}}{\alpha_{\text{EM}}^{\text{SI}}} = 137.036
\end{equation}

\subsection{Das natürliche Einheitensystem mit $\alpha = 1$}
\label{subsec:natural_units}

In einem natürlichen Einheitensystem, das die elektromagnetische Dualität respektiert, erhalten wir:

\begin{itemize}
	\item $\hbar_{\text{nat}} = 1$ (quantenmechanische Skala)
	\item $c_{\text{nat}} = 1$ (relativistische Skala)
	\item $\varepsilon_{0,\text{nat}} = 1$ (elektrische Konstante)
	\item $\mu_{0,\text{nat}} = 1$ (magnetische Konstante)
	\item $e_{\text{nat}}^2 = 4\pi$ (Elementarladung)
\end{itemize}

Mit diesen Werten verifiziert sich $\alpha = 1$ sowohl in der Standardform als auch in der dualen Form:

\begin{equation}
	\alpha = \frac{4\pi}{4\pi \cdot 1 \cdot 1 \cdot 1} = 1
\end{equation}

\subsection{Die Auflösung des ``Mysteriums''}
\label{subsec:mystery_resolution}

Die scheinbare Mystifizierung von $1/137$ entsteht durch:

\begin{enumerate}
	\item \textbf{Verwechslung zweier Aspekte}: Die Invarianz der Verhältnisse wird mit der Systemabhängigkeit der numerischen Darstellung vermischt.
	
	\item \textbf{Behandlung des SI-Systems als absolut}: Die historisch gewachsenen SI-Einheiten (Meter, Sekunde, Kilogramm, Ampere) zwingen elektromagnetische Konstanten zu ``unnatürlichen'' Werten.
	
	\item \textbf{Vergessen der Einheitensystem-Konstruktion}: Alle Einheitensysteme sind menschliche Konstrukte. Die Natur kennt keine bevorzugten Einheiten.
	
	\item \textbf{Suche nach tiefer Bedeutung in Umrechnungsfaktoren}: Die Zahl 137 hat keine tiefere kosmische Bedeutung als etwa der Faktor 1609.344 zwischen Meilen und Metern.
\end{enumerate}

\subsection{Die anthropische Fehlinterpretation}
\label{subsec:anthropic_fallacy}

Typische anthropische Argumente behaupten:
\begin{itemize}
	\item ``Wenn $\alpha_{\text{EM}} = 1/200$ $\rightarrow$ keine Atome $\rightarrow$ kein Leben''
	\item ``Wenn $\alpha_{\text{EM}} = 1/80$ $\rightarrow$ keine Sterne $\rightarrow$ kein Leben''
	\item ``Daher ist $\alpha_{\text{EM}} = 1/137$ `feinabgestimmt' für Leben''
\end{itemize}

\textbf{Das Problem}: Diese Argumente setzen das SI-System als absolut voraus!

\textbf{In natürlichen Einheiten}: $\alpha_{\text{EM}} = 1$ ist perfekt natürlich und benötigt keinerlei Feinabstimmung. Die elektromagnetische Wechselwirkung hat Einheitsstärke im natürlichen Einheitensystem, das die fundamentale Struktur der Quantenmechanik und Relativität respektiert.

\subsection{Sommerfelds harmonische Prägung}
\label{subsec:sommerfeld_harmonic}

Ein oft übersehener historischer Aspekt: Arnold Sommerfeld suchte 1916 aktiv nach \textbf{harmonischen Verhältnissen} in Atomspektren, geleitet von der philosophischen Überzeugung, dass die Natur musikalischen Prinzipien folgt.

Seine methodische Herangehensweise:
\begin{enumerate}
	\item \textbf{Erwartung} musikalischer Verhältnisse in Quantenübergängen
	\item \textbf{Kalibrierung} der Messsysteme zur Erzeugung harmonischer Werte
	\item \textbf{Definition} von $\alpha_{\text{EM}}$ basierend auf harmonischen spektroskopischen Anpassungen
	\item \textbf{Zuordnung} des resultierenden Verhältnisses zur fundamentalen Physik
\end{enumerate}

Die scheinbare ``Harmonie'' in $\alpha_{\text{EM}}^{-1} = 137 \approx (6/5)^{27}$ ist daher keine kosmische Entdeckung, sondern das Resultat von Sommerfelds harmonischen Erwartungen, die in die Einheitensystem-Definition eingebettet wurden.

\subsection{Physikalische Interpretation}
\label{subsec:physical_interpretation}

In natürlichen Einheiten repräsentiert $\alpha = 1$ die perfekte Balance zwischen:

\begin{itemize}
	\item \textbf{Elektrischer Feldkopplung} (durch $\varepsilon_0$ mit $c^{-1}$)
	\item \textbf{Magnetischer Feldkopplung} (durch $\mu_0$ mit $c^{+1}$)
	\item \textbf{Quantenmechanischer Skala} (durch $\hbar$)
	\item \textbf{Relativistischer Skala} (durch $c$)
\end{itemize}

Die elektromagnetische Dualität $\frac{1}{\varepsilon_0 c} = \mu_0 c$ gewährleistet diese perfekte Balance.

\subsection{Zusammenfassung: Die wahre Lektion}
\label{subsec:true_lesson}

Die Feinstrukturkonstante lehrt uns eine tiefgreifende Lektion über die Natur physikalischer Gesetze:

\textbf{Die fundamentalen Beziehungen des Universums sind elegant und einfach, wenn sie in ihrer natürlichen Sprache ausgedrückt werden.}

Die scheinbare Komplexität und das Mysterium von ``1/137'' sind lediglich Artefakte unserer historischen Entscheidung, elektromagnetische Phänomene mit Einheiten zu messen, die ursprünglich für mechanische Größen definiert wurden.

Das ``Feinabstimmungsproblem'' löst sich vollständig auf, sobald wir erkennen:

\begin{itemize}
	\item $\alpha = 1/137$ ist keine fundamentale Zahl, sondern ein Einheiten-Umrechnungsfaktor
	\item $\alpha = 1$ repräsentiert die natürliche Stärke der elektromagnetischen Kopplung
	\item Das scheinbare ``Mysterium'' entsteht durch die Behandlung willkürlicher SI-Einheiten als absolut
	\item Die fundamentalen Beziehungen der Natur sind einfach in ihrer natürlichen Sprache
\end{itemize}

\subsection{Historische Warnung: Die Eddington-Saga}
\label{subsec:eddington_warning}

Arthur Eddington (1882-1944) versuchte, $\alpha_{\text{EM}} = 1/137$ aus ersten Prinzipien zu ``beweisen'' und entwickelte aufwendige numerologische Theorien. Das Ergebnis war vollständig spekulativ und falsch -- eine Warnung davor, systemabhängige Zahlen zu mystifizieren.

Die moderne Analyse zeigt jedoch, dass die Feinstrukturkonstante tatsächlich aus fundamentalen elektromagnetischen Vakuumkonstanten ableitbar ist und dass $\alpha_{\text{EM}} = 1$ in natürlichen Einheiten nicht nur möglich ist, sondern die willkürliche Natur unserer Einheitensystem-Wahl offenbart.

\section{Der $\beta_T$ Parameter -- Ein zweites Beispiel der Systemabhängigkeit}
\label{sec:beta_t}

\subsection{Die Parallele zur Feinstrukturkonstante}
\label{subsec:beta_parallel}

Genau wie die Feinstrukturkonstante zeigt auch der $\beta_T$ Parameter des T0-Modells dieselbe fundamentale Systemabhängigkeit:

\begin{itemize}
	\item \textbf{SI-System}: $\beta_T^{\text{SI}} \approx 0.008$ (aus astrophysikalischen Beobachtungen)
	\item \textbf{T0-natürliche Einheiten}: $\beta_T^{\text{nat}} = 1$ (durch Definition)
\end{itemize}

\textbf{Transformationsfaktor}: 
\begin{equation}
	\frac{\beta_T^{\text{nat}}}{\beta_T^{\text{SI}}} = \frac{1}{0.008} = 125
\end{equation}

\subsection{Theoretische Grundlage aus der Feldtheorie}
\label{subsec:beta_field_theory}

Der $\beta_T$ Parameter wird im T0-Modell durch die fundamentale feldtheoretische Beziehung definiert:

\begin{equation}
	\beta_T = \frac{2Gm}{r}
	\label{eq:beta_definition}
\end{equation}

wobei $G$ die Gravitationskonstante, $m$ die Quellmasse und $r$ der Abstand von der Quelle ist.

In natürlichen Einheiten ($\hbar = c = 1$) wird dieser Parameter dimensionslos und kann durch geeignete Wahl der Einheiten auf $\beta_T = 1$ normiert werden. Dies etabliert eine direkte Verbindung zwischen gravitativen und elektromagnetischen Wechselwirkungen.

\subsection{Die Zirkularität in der SI-Bestimmung}
\label{subsec:beta_circularity}

Die Bestimmung von $\beta_T^{\text{SI}}$ erfolgt über kosmologische Beobachtungen:

\begin{equation}
	z(\lambda) = z_0\left(1 + \beta_T \ln\frac{\lambda}{\lambda_0}\right)
\end{equation}

Diese Bestimmung involviert jedoch:
\begin{itemize}
	\item Hubble-Konstante $H_0$ $\rightarrow$ Distanzmessungen
	\item Distanzleiter $\rightarrow$ Standardkerzen
	\item Photometrie $\rightarrow$ Plancksches Strahlungsgesetz $\rightarrow$ Fundamentalkonstanten
\end{itemize}

\textbf{Die Bestimmung ist zirkulär durch kosmologische Parameter!}

\subsection{Physikalische Interpretation}
\label{subsec:beta_physical}

Der $\beta$-Parameter misst die Stärke des dynamischen Zeitfeldes im T0-Modell:

\begin{itemize}
	\item \textbf{Schwache Gravitation} (Erdoberfläche): $\beta \sim 10^{-9}$
	\item \textbf{Stellare Physik} (Sonnenoberfläche): $\beta \sim 10^{-6}$
	\item \textbf{Starke Gravitation} (Neutronenstern): $\beta \sim 0.1$
	\item \textbf{Schwarzschild-Horizont}: $\beta = 1$ (Grenzfall)
\end{itemize}

\subsection{Die gemeinsame Lektion}
\label{subsec:common_lesson}

Sowohl $\alpha_{\text{EM}}$ als auch $\beta_T$ demonstrieren dasselbe fundamentale Prinzip:

\textbf{Was wir für mysteriöse Naturkonstanten halten, sind oft nur Umrechnungsfaktoren zwischen verschiedenen Einheitensystemen.}

Die scheinbare ``Feinabstimmung'' dieser Parameter verschwindet vollständig, wenn wir sie in ihren natürlichen Einheiten betrachten, wo beide den Wert 1 annehmen -- die einfachste und eleganteste mögliche Wahl.
\section{Der $\beta_T$ Parameter -- Ein zweites Beispiel der Systemabhängigkeit}
\label{sec:beta_t}

\subsection{Die Parallele zur Feinstrukturkonstante}
\label{subsec:beta_parallel}

Genau wie die Feinstrukturkonstante zeigt auch der $\beta_T$ Parameter des T0-Modells dieselbe fundamentale Systemabhängigkeit:

\begin{itemize}
	\item \textbf{SI-System}: $\beta_T^{\text{SI}} \approx 0.008$ (aus astrophysikalischen Beobachtungen)
	\item \textbf{T0-natürliche Einheiten}: $\beta_T^{\text{nat}} = 1$ (durch Definition)
\end{itemize}

\textbf{Transformationsfaktor}: 
\begin{equation}
	\frac{\beta_T^{\text{nat}}}{\beta_T^{\text{SI}}} = \frac{1}{0.008} = 125
\end{equation}

\subsection{Theoretische Grundlage aus der Feldtheorie}
\label{subsec:beta_field_theory}

Der $\beta_T$ Parameter wird im T0-Modell durch die fundamentale feldtheoretische Beziehung definiert:

\begin{equation}
	\beta_T = \frac{2Gm}{r}
	\label{eq:beta_definition}
\end{equation}

wobei $G$ die Gravitationskonstante, $m$ die Quellmasse und $r$ der Abstand von der Quelle ist.

In natürlichen Einheiten ($\hbar = c = 1$) wird dieser Parameter dimensionslos und kann durch geeignete Wahl der Einheiten auf $\beta_T = 1$ normiert werden. Dies etabliert eine direkte Verbindung zwischen gravitativen und elektromagnetischen Wechselwirkungen.

\subsection{Die Zirkularität in der SI-Bestimmung}
\label{subsec:beta_circularity}

Die Bestimmung von $\beta_T^{\text{SI}}$ erfolgt über kosmologische Beobachtungen:

\begin{equation}
	z(\lambda) = z_0\left(1 + \beta_T \ln\frac{\lambda}{\lambda_0}\right)
\end{equation}

Diese Bestimmung involviert jedoch:
\begin{itemize}
	\item Hubble-Konstante $H_0$ $\rightarrow$ Distanzmessungen
	\item Distanzleiter $\rightarrow$ Standardkerzen
	\item Photometrie $\rightarrow$ Plancksches Strahlungsgesetz $\rightarrow$ Fundamentalkonstanten
\end{itemize}

\textbf{Die Bestimmung ist zirkulär durch kosmologische Parameter!}

\subsection{Physikalische Interpretation}
\label{subsec:beta_physical}

Der $\beta$-Parameter misst die Stärke des dynamischen Zeitfeldes im T0-Modell:

\begin{itemize}
	\item \textbf{Schwache Gravitation} (Erdoberfläche): $\beta \sim 10^{-9}$
	\item \textbf{Stellare Physik} (Sonnenoberfläche): $\beta \sim 10^{-6}$
	\item \textbf{Starke Gravitation} (Neutronenstern): $\beta \sim 0.1$
	\item \textbf{Schwarzschild-Horizont}: $\beta = 1$ (Grenzfall)
\end{itemize}

\subsection{Die gemeinsame Lektion}
\label{subsec:common_lesson}

Sowohl $\alpha_{\text{EM}}$ als auch $\beta_T$ demonstrieren dasselbe fundamentale Prinzip:

\textbf{Was wir für mysteriöse Naturkonstanten halten, sind oft nur Umrechnungsfaktoren zwischen verschiedenen Einheitensystemen.}

Die scheinbare ``Feinabstimmung'' dieser Parameter verschwindet vollständig, wenn wir sie in ihren natürlichen Einheiten betrachten, wo beide den Wert 1 annehmen -- die einfachste und eleganteste mögliche Wahl.
	\section{Der $\beta_{\text{T}}$-Parameter}
	\label{sec:beta_t}
	
	\subsection{Empirische vs. theoretische Werte}
	\label{subsec:beta_empirisch_theoretisch}
	
	Der $\beta_{\text{T}}$-Parameter zeigt dieselbe Systemabhängigkeit:
	
	\begin{align}
		\betaTSI &\approx 0.008 \text{ (aus astrophysikalischen Beobachtungen)} \\
		\betaTnat &= 1 \text{ (in T0-natürlichen Einheiten)}
	\end{align}
	
	\textbf{Transformationsfaktor:}
	$$\frac{\betaTnat}{\betaTSI} = \frac{1}{0.008} = 125$$
	
	\subsection{Theoretische Grundlage aus der Feldtheorie}
	\label{subsec:beta_feldtheorie}
	
	Das T0-Modell etabliert $\beta_{\text{T}} = 1$ durch die fundamentale feldtheoretische Beziehung \cite{pascher_derivation_beta_2025}:
	
	\begin{equation}
		\beta_{\text{T}} = \frac{\lambdah^2 v^2}{16\pichar^3 m_h^2 \xipar} = 1
		\label{eq:beta_t_feldtheorie}
	\end{equation}
	
	Diese Beziehung, kombiniert mit dem Higgs-hergeleiteten Wert von $\xipar$, bestimmt eindeutig $\beta_{\text{T}} = 1$ in natürlichen Einheiten und eliminiert alle freien Parameter aus der Theorie.
	
	\subsection{Zirkularität in der SI-Bestimmung}
	\label{subsec:beta_zirkularitaet}
	
	Der SI-Wert $\betaTSI$ wird bestimmt durch:
	$$z(\lambda) = z_0\left(1 + \beta_{\text{T}} \ln\frac{\lambda}{\lambda_0}\right)$$
	
	Aber dies beinhaltet:
	\begin{itemize}
		\item Hubble-Konstante $H_0$ $\rightarrow$ Entfernungsmessungen
		\item Entfernungsleiter $\rightarrow$ Standardkerzen
		\item Photometrie $\rightarrow$ Planck-Strahlungsgesetz $\rightarrow$ fundamentale Konstanten
	\end{itemize}
	
	\textbf{Die Bestimmung ist zirkulär durch kosmologische Parameter!}
	
	\section{Die Wien-Konstante $\alpha_{\text{W}}$}
	\label{sec:alpha_w}
	
	\subsection{Mathematische vs. konventionelle Werte}
	\label{subsec:wien_werte}
	
	Das Wien-Verschiebungsgesetz ergibt:
	
	\begin{align}
		\text{SI-System:} \quad &\alphaWSI = 2.8977719... \\
		\text{T0-System:} \quad &\alphaWnat = 1
	\end{align}
	
	\textbf{Transformationsfaktor:}
	$$\frac{\alphaWSI}{\alphaWnat} = 2.898$$
	
	\section{Parameter-Vergleichstabelle}
	\label{sec:parameter_vergleich}
	
	\begin{table}[htbp]
		\centering
		\begin{adjustbox}{max width=\textwidth}
\begin{tabular}{lcccc}
			\toprule
			\textbf{Parameter} & \textbf{SI-Wert} & \textbf{T0-nat-Wert} & \textbf{Verhältnis} & \textbf{Faktor} \\
			\midrule
			$\xipar$ (Elektron) & $7.5 \times 10^{-6}$ & $3.1 \times 10^{-2}$ & 4100 & $10^{3.6}$ \\
			$\alpha_{\text{EM}}$ & $7.3 \times 10^{-3}$ & $1$ & 137 & $10^{2.1}$ \\
			$\beta_{\text{T}}$ & $0.008$ & $1$ & 125 & $10^{2.1}$ \\
			$\alpha_{\text{W}}$ & $2.898$ & $1$ & 2.9 & $10^{0.5}$ \\
			\bottomrule
		\end{tabular}
\end{adjustbox}

\bigskip
\clearpage
		\caption{Systematische Parameterunterschiede zwischen Einheitensystemen}
		\label{tab:parameter_vergleich}
	\end{table}
	
	\textbf{Alle Parameter zeigen 0.5-4 Größenordnungen Unterschied zwischen Systemen!}
	
	\section{Yukawa-Parameter: Variabel und systemabhängig}
	\label{sec:yukawa_parameter}
	
	\subsection{Die Hierarchie der Yukawa-Kopplungen}
	\label{subsec:yukawa_hierarchie}
	
	Im Standardmodell variieren Yukawa-Kopplungen dramatisch:
	
	\begin{table}[htbp]
		\centering
		\begin{adjustbox}{max width=\textwidth}
\begin{tabular}{lc}
			\toprule
			\textbf{Teilchen} & \textbf{$y_i$ (SI-System)} \\
			\midrule
			Elektron & $2.94 \times 10^{-6}$ \\
			Myon & $6.09 \times 10^{-4}$ \\
			Tau & $1.03 \times 10^{-2}$ \\
			Up-Quark & $1.27 \times 10^{-5}$ \\
			Top-Quark & $1.00$ \\
			Bottom-Quark & $2.25 \times 10^{-2}$ \\
			\bottomrule
		\end{tabular}
\end{adjustbox}

\bigskip
\clearpage
		\caption{Yukawa-Kopplungshierarchie (5 Größenordnungen Variation)}
		\label{tab:yukawa_hierarchie}
	\end{table}
	
	\subsection{Transformationsunsicherheit}
	\label{subsec:yukawa_transformation}
	
	Die Transformation von Yukawa-Parametern zwischen Systemen erfordert sorgfältige Betrachtung des Higgs-Mechanismus. Die allgemeine Form wäre:
	
	$$y_{i,\text{nat}} = y_{i,\text{SI}} \times T_{\text{Yukawa}}$$
	
	wobei $T_{\text{Yukawa}}$ von der Transformation des Higgs-Vakuumerwartungswerts und Teilchenmassen abhängt.
	
	\subsection{Konsistenzbedingungen}
	\label{subsec:yukawa_konsistenz}
	
	Der Higgs-Mechanismus erfordert:
	$$m_h^2 = \frac{\lambdah v^2}{2}$$
	
	Für Transformationskonsistenz:
	$$T_m^2 = T_\lambda \times T_v^2$$
	
	Dies ergibt:
	$$y_{i,\text{nat}} = y_{i,\text{SI}} \times \sqrt{T_\lambda}$$
	
	\textbf{Jedoch erfordert $T_\lambda$ detaillierte Spezifikation der T0-natürlichen Einheitensystem-Transformationsregeln.}
	
	\section{Universelle Warnung: Keine direkte Parameterübertragung}
	\label{sec:universelle_warnung}
	
	\subsection{Das systematische Problem}
	\label{subsec:systematisches_problem}
	
	\begin{warning}
		\textbf{JEDER Parametersymbol in T0-Modell-Dokumenten kann verschiedene Werte haben als in SI-System-Berechnungen!}
	\end{warning}
	
	\textbf{Konkrete Gefahrenzonen:}
	
	\begin{align}
		G_{\text{nat}} &= 1 \quad \text{vs.} \quad G_{\text{SI}} = 6.674 \times 10^{-11} \text{ m}^3/(\text{kg} \cdot \text{s}^2) \\
		\alpha_{\text{EM,nat}} &= 1 \quad \text{vs.} \quad \alpha_{\text{EM,SI}} = 1/137 \\
		e_{\text{nat}} &= 2\sqrt{\pichar} \quad \text{vs.} \quad e_{\text{SI}} = 1.602 \times 10^{-19} \text{ C}
	\end{align}
	
	\textbf{Direkte Übertragung führt zu Fehlern von Faktoren $10^2$ bis $10^{11}$!}
	
	\subsection{Erforderliches Transformationsprotokoll}
	\label{subsec:transformationsprotokoll}
	
	Für jeden Parameter explizit spezifizieren:
	
	\begin{enumerate}
		\item \textbf{Welches Einheitensystem} verwendet wird
		\item \textbf{Wie Transformation erfolgt} zwischen Systemen
		\item \textbf{Welche Faktoren berücksichtigt werden müssen}
		\item \textbf{Welche Konsistenzbedingungen} erfüllt sein müssen
	\end{enumerate}
	
	\textbf{Beispiel vollständiger Spezifikation:}
	\begin{adjustbox}{max width=\textwidth}
\begin{tabular}{lcc}
			\toprule
			\textbf{Testtyp} & \textbf{SI-basierte Vorhersage} & \textbf{T0-basierte Vorhersage} \\
			\midrule
			QED-Anomalie & $a_e \propto \alphaEMSI = 1/137$ & $a_e \propto \alphaEMnat = 1$ \\
			Galaxienrotation & $v^2 \propto \xipar_{\text{SI}} \sim 10^{38}$ & $v^2 \propto \xipar_{\text{nat}} \sim 10^{41}$ \\
			CMB-Temperatur & $T \propto \betaTSI = 0.008$ & $T \propto \betaTnat = 1$ \\
			\bottomrule
		\end{tabular}
\end{adjustbox}

\bigskip
\clearpage

\end{document}
	
	\subsection{Experimentelle Vorhersage-Richtlinien}
	\label{subsec:experimentelle_richtlinien}
	
	\textbf{Für QED-Berechnungen:}
	\begin{align}
		\text{FALSCH:} \quad &\alpha_{\text{EM}} = 1 \text{ aus T0-Modell direkt in SI-Formeln} \\
		\text{RICHTIG:} \quad &\alphaEMSI = 1/137 \text{ mit Transformation zu } \alphaEMnat = 1
	\end{align}
	
	\textbf{Für Gravitationsberechnungen:}
	\begin{align}
		\text{FALSCH:} \quad &G = 1 \text{ aus T0-Modell direkt in Newton-Formeln} \\
		\text{RICHTIG:} \quad &G_{\text{SI}} = 6.674 \times 10^{-11} \text{ mit Transformation zu } G_{\text{nat}} = 1
	\end{align}
	
	\section{Die Zirkularitäts-Auflösung}
	\label{sec:zirkularitaets_aufloesung}
	
	\subsection{Scheinbare vs. reale Zirkularität}
	\label{subsec:scheinbare_reale_zirkularitaet}
	
	Das Zirkularitätsproblem, das die T0-Modell-Parameterbestimmung zu plagen schien, wird durch Erkennen aufgelöst:
	
	\begin{enumerate}
		\item \textbf{Keine reale Zirkularität existiert} innerhalb jedes konsistenten Systems
		\item \textbf{Sowohl SI- als auch T0-Systeme sind intern konsistent}
		\item \textbf{Der scheinbare Widerspruch} entstand aus dem Vergleich von Parametern über verschiedene Systeme hinweg
		\item \textbf{Ordnungsgemäße Transformation} eliminiert alle scheinbaren Inkonsistenzen
	\end{enumerate}
	
	\subsection{Systemkonsistenz-Verifikation}
	\label{subsec:systemkonsistenz}
	
	\textbf{SI-Systemkonsistenz:}
	$$\Rzero = \frac{m_e c \left(\alphaEMSI\right)^2}{2\hbar} \quad \checkmark \text{ (experimentell verifiziert zu 0.000001\%)}$$
	
	\textbf{T0-Systemkonsistenz:}
	$$\text{Alle Parameter = 1} \quad \checkmark \text{ (per Konstruktion)}$$
	
	\textbf{Beide Systeme funktionieren perfekt innerhalb ihrer eigenen Frameworks!}
	
	\section{Implikationen für T0-Modell-Tests}
	\label{sec:test_implikationen}
	
	\subsection{Systemspezifische Vorhersagen}
	\label{subsec:systemspezifische_vorhersagen}
	
	Experimentelle Tests müssen klar spezifizieren, welches Parametersystem verwendet wird:
	
	\begin{table}[htbp]
		\centering
		\begin{tabular}{lcc}
			\toprule
			\textbf{Testtyp} & \textbf{SI-basierte Vorhersage} & \textbf{T0-basierte Vorhersage} \\
			\midrule
			QED-Anomalie & $a_e \propto \alphaEMSI = 1/137$ & $a_e \propto \alphaEMnat = 1$ \\
			Galaxienrotation & $v^2 \propto \xipar_{\text{SI}} \sim 10^{38}$ & $v^2 \propto \xipar_{\text{nat}} \sim 10^{41}$ \\
			CMB-Temperatur & $T \propto \betaTSI = 0.008$ & $T \propto \betaTnat = 1$ \\
			\bottomrule
		\end{tabular}
		\caption{Systemspezifische experimentelle Vorhersagen}
		\label{tab:system_vorhersagen}
	\end{table}
	
	\subsection{Transformations-Validierung}
	\label{subsec:transformations_validierung}
	
	Die Transformationsfaktoren können validiert werden durch Überprüfung:
	
	\begin{enumerate}
		\item \textbf{Dimensionale Konsistenz} in beiden Systemen
		\item \textbf{Bekannte Grenzwerte} werden korrekt reproduziert
		\item \textbf{Verhältnisse bleiben invariant} zwischen Systemen
		\item \textbf{Interne Konsistenz} jedes Systems
	\end{enumerate}
	

	\end{document}