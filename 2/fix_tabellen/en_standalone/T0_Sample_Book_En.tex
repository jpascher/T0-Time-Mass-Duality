\documentclass[a4paper,11pt]{book}
\usepackage[utf8]{inputenc}
\usepackage[T1]{fontenc}
\usepackage[english]{babel}
\usepackage{graphicx}
\usepackage{amsmath,amssymb,amsthm}
\usepackage{hyperref}
\usepackage{geometry}
\usepackage{fancyhdr}
\usepackage{booktabs}
\usepackage{longtable}
\usepackage{array}
\usepackage{tcolorbox}
\tcbuselibrary{breakable,skins}
\usepackage{xcolor}
\usepackage{tikz}

\setlength{\headheight}{14pt}

% Abstract environment
\newenvironment{abstract}{\section*{Abstract}}{}

% Colors
\definecolor{t0blue}{RGB}{33,150,243}
\definecolor{gold}{RGB}{255,215,0}

% T0 specific commands
\newcommand{\Tzero}{T_0}
\newcommand{\betaT}{\beta_T}
\newcommand{\xipar}{\xi}
\providecommand{\Kfrak}{K_{\text{frak}}}
\providecommand{\hbar}{\hslash}
\providecommand{\kB}{k_B}
\providecommand{\Tfield}{T}
\providecommand{\alphaEM}{\alpha}
\providecommand{\Lp}{l_P}
\providecommand{\Tp}{t_P}
\providecommand{\Mp}{M_P}

% tcolorbox environments
\newtcolorbox{keyresult}[1][Key Result]{colback=blue!5,colframe=blue!75!black,title=#1,breakable}
\newtcolorbox{foundation}[1][Foundation]{colback=green!5,colframe=green!75!black,title=#1,breakable}
\newtcolorbox{alternative}[1][Alternative]{colback=orange!5,colframe=orange!75!black,title=#1,breakable}
\newtcolorbox{summary}[1][Summary]{colback=gray!5,colframe=gray!75!black,title=#1,breakable}
\newtcolorbox{important}[1][Important]{colback=red!5,colframe=red!75!black,title=#1,breakable}

\geometry{margin=2.5cm}
\pagestyle{fancy}

\title{\Huge\textbf{FFGFT: Time-Mass Duality}\\[1cm]\Large All Natural Constants from One Number: $\alpha \approx 1/137$}
\author{Johann Pascher}
\date{2024}

\begin{document}

% Cover page with image
\begin{titlepage}
\centering
\includegraphics[width=\textwidth,height=\textheight,keepaspectratio]{T0_deckblatt_En.png}
\end{titlepage}

\frontmatter
\tableofcontents

\mainmatter

\chapter{Introduction to T0-Theory}

The T0-Theory is a new approach to unifying fundamental physics. The central thesis states:

\begin{keyresult}[Central Theorem]
All natural constants and physical parameters can be derived from a single dimensionless number: the fine-structure constant $\alpha \approx 1/137$.
\end{keyresult}

\section{Time-Mass Duality}

The core principle of T0-Theory is the Time-Mass Duality:
\begin{equation}
T(x) = \frac{\hbar}{E(x)} = \frac{\hbar}{m(x)c^2}
\end{equation}

This relationship shows that time and mass are intrinsically linked.

\begin{foundation}[Fundamental Principle]
In regions with higher energy density, intrinsic time runs slower - exactly as General Relativity predicts for gravitation.
\end{foundation}

\section{The Scaling Parameter $\xi$}

The dimensionless scaling parameter $\xi$ connects all natural constants:
\begin{equation}
\xi = \frac{4}{3} \times 10^{-4} \approx \sqrt{\alpha}
\end{equation}

\chapter{Particle Masses and Fundamental Constants}

The masses of all elementary particles can be derived from the scaling parameter $\xi$ and the Planck mass.

\section{Lepton Masses}

The Koide formula finds its natural explanation in the T0 framework:
\begin{equation}
\frac{m_e + m_\mu + m_\tau}{(\sqrt{m_e} + \sqrt{m_\mu} + \sqrt{m_\tau})^2} = \frac{2}{3}
\end{equation}

\chapter{Cosmological Implications}

\section{The Hubble Constant}

The T0-Theory provides a geometric derivation of the Hubble constant:
\begin{equation}
H_0 \approx \frac{c}{\xi \cdot L_P} \cdot \alpha^2
\end{equation}

\section{Dark Energy}

The cosmological constant is explained as a consequence of intrinsic time.

\chapter{Experimental Predictions}

\begin{summary}[Testable Predictions]
\begin{itemize}
\item Anomalous magnetic moment of the electron: $(g-2)_e$
\item Koide formula extensions for quarks
\item Frequency-independent effects
\end{itemize}
\end{summary}

\backmatter

\chapter*{Appendix: Formula Collection}
\addcontentsline{toc}{chapter}{Appendix: Formula Collection}

Here are the key formulas of T0-Theory summarized.

\end{document}
