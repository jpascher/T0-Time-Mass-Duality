
\section{Die Natur der Zeit als physikalischer Parameter}

In vielen physikalischen Theorien tritt die Zeit lediglich als mathematischer Parameter auf, der die Ordnung von Ereignissen beschreibt. Sie ist keine eigenständige physikalische Substanz wie Energie oder Masse. Um Missverständnisse in der Interpretation physikalischer Gleichungen zu vermeiden, sollte die Rolle der Zeit explizit formuliert werden.

\subsection{Axiomatische Festlegung der Zeit}

Um eine korrekte Interpretation physikalischer Theorien sicherzustellen, schlagen wir folgendes Axiom zur Zeit vor:

\begin{itemize}
    \item Zeit ist ein mathematischer Parameter, der die Abfolge von Ereignissen beschreibt, aber selbst keine physikalische Substanz darstellt.
    \item Zeit besitzt eine natürliche Richtung, die sich makroskopisch durch die Irreversibilität von Prozessen zeigt.
    \item Die mathematische Umkehrbarkeit der Zeit in vielen physikalischen Gleichungen bedeutet nicht, dass Prozesse in der Realität umkehrbar sind.
    \item Innerhalb eines Systems ist die eigene Zeit immer konstant erlebbar, unabhängig von der Sicht eines externen Beobachters.
    \item Jede physikalische Interpretation von Zeit muss berücksichtigen, dass sie relational existiert und nicht absolut ist.
\end{itemize}

\subsection{Praktische Konsequenzen}

Obwohl Zeit in physikalischen Gleichungen oft symmetrisch behandelt wird, existiert eine makroskopisch beobachtbare Richtung der Zeit. Die Ursache-Wirkungs-Beziehung, der thermodynamische Zeitpfeil und die Expansion des Universums zeigen, dass die Zeit in realen Systemen eine eindeutige Richtung besitzt. Die Einführung dieses Axioms stellt sicher, dass die Zeit in physikalischen Theorien nicht falsch interpretiert oder mit substantiellen Größen wie Energie oder Materie gleichgesetzt wird.

