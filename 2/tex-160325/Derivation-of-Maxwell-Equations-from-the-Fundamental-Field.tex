\documentclass{article}
\usepackage{amsmath}
\usepackage{amsfonts}
\usepackage{amssymb}
\usepackage{hyperref}

\title{Derivation of Maxwell's Equations from a Fundamental Vector Field \(\psi\) with Enhanced Consistency}
\author{Johann Pascher}
\date{24-05-2025}

\begin{document}
	
	\maketitle
	
	\section{Introduction}
	Maxwell’s equations govern classical electrodynamics, relating electric (\(\mathbf{E}\)) and magnetic (\(\mathbf{B}\)) fields to their sources—charges (\(\rho\)) and currents (\(\mathbf{j}\)). This document derives these equations from a fundamental four-vector field \(\psi_\mu\), identified with the electromagnetic four-potential \(A_\mu = (\phi/c, \mathbf{A})\), where \(\mathbf{E} = -\nabla \phi - \partial \mathbf{A}/\partial t\) and \(\mathbf{B} = \nabla \times \mathbf{A}\). In doing so, we explicitly introduce and justify a set of assumptions that, although necessary, are well motivated by symmetry principles and experimental evidence.
	
	\section{Fundamental Field Equation}
	We start with the field equation:
	\[
	\left( \frac{\partial^2}{\partial t^2} - c^2 \nabla^2 + V'(\psi_\nu \psi^\nu) \right) \psi_\mu = J_\mu,
	\]
	where:
	\begin{itemize}
		\item \(\square = \partial_\mu \partial^\mu = \frac{1}{c^2} \frac{\partial^2}{\partial t^2} - \nabla^2\) is the d'Alembert operator,
		\item \(\psi_\nu \psi^\nu = \eta_{\mu\nu} \psi^\mu \psi^\nu\) with \(\eta_{\mu\nu} = \text{diag}(1, -1, -1, -1)\),
		\item \(V'(\psi_\nu \psi^\nu) = \frac{dV}{d(\psi_\nu \psi^\nu)}\) is the derivative of a potential,
		\item \(J_\mu = (\rho c, \mathbf{j})\) is the four-current, explicitly introduced as a source term.
	\end{itemize}
	
	\section{Potential and Gauge Invariance}
	Maxwell’s equations require gauge invariance under \(A_\mu \to A_\mu + \partial_\mu \lambda\). A potential \(V(\psi_\nu \psi^\nu)\) depending directly on \(\psi_\mu\) is not gauge-invariant since \(\psi_\nu \psi^\nu\) changes under a gauge transformation. To address this, we define the potential in terms of the gauge-invariant field strength tensor:
	\[
	F_{\mu\nu} = \partial_\mu \psi_\nu - \partial_\nu \psi_\mu.
	\]
	We then propose:
	\[
	V = V(F_{\rho\sigma} F^{\rho\sigma}),
	\]
	which ensures that \(V\) depends only on physical fields. For small fields (in the linear approximation), we assume:
	\[
	V(F_{\rho\sigma} F^{\rho\sigma}) \approx 0,
	\]
	with \(V'(0) = 0\) and \(V''(0) = 0\) as required for a massless photon—a consequence of gauge invariance.
	
	\section{Perturbation and Linearization}
	We decompose the field as:
	\[
	\psi_\mu = \psi_{0\mu} + \delta \psi_\mu,
	\]
	where \(\psi_{0\mu} = 0\) in vacuum and \(\delta \psi_\mu = A_\mu\) is identified with the electromagnetic four-potential. Substituting this decomposition into the fundamental equation gives:
	\[
	\square (\psi_{0\mu} + \delta \psi_\mu) + V'(\psi_\nu \psi^\nu) (\psi_{0\mu} + \delta \psi_\mu) = J_\mu.
	\]
	For \(\psi_{0\mu} = 0\) and to first order (with \(\psi_\nu \psi^\nu \approx \delta \psi_\mu \delta \psi^\mu\)), expanding \(V'\) yields:
	\[
	V'(\psi_\nu \psi^\nu) \approx V'(0) + V''(0) (\delta \psi_\mu \delta \psi^\mu).
	\]
	Since \(V'(0) = 0\) and \(V''(0) = 0\) (as demanded by gauge symmetry), neglecting higher-order terms leads to:
	\[
	\square \delta \psi_\mu = J_\mu.
	\]
	Identifying \(\delta \psi_\mu = A_\mu\) results in:
	\[
	\square A_\mu = J_\mu.
	\]
	
	\section{Maxwell’s Equations in Vacuum}
	In the Lorentz gauge (\(\partial_\mu A^\mu = 0\)), and in the absence of sources (\(J_\mu = 0\)), the equation reduces to:
	\[
	\square A_\mu = 0.
	\]
	With the definition:
	\[
	F_{\mu\nu} = \partial_\mu A_\nu - \partial_\nu A_\mu,
	\]
	we obtain:
	\[
	\partial_\mu F_{\mu\nu} = \square A_\nu - \partial_\nu (\partial_\mu A^\mu) = 0,
	\]
	which reproduces the homogeneous Maxwell equations:
	\[
	\nabla \cdot \mathbf{B} = 0, \quad \nabla \times \mathbf{E} + \frac{\partial \mathbf{B}}{\partial t} = 0,
	\]
	\[
	\nabla \cdot \mathbf{E} = 0, \quad \nabla \times \mathbf{B} - \frac{1}{c^2} \frac{\partial \mathbf{E}}{\partial t} = 0.
	\]
	
	\section{Sources and Full Equations}
	Including sources and the appropriate constants (with \(\mu_0\) and \(\epsilon_0\) such that \(\epsilon_0 \mu_0 = 1/c^2\)), the equation becomes:
	\[
	\square A_\mu = \mu_0 J_\mu,
	\]
	which implies:
	\[
	\partial_\mu F_{\mu\nu} = \mu_0 J_\nu.
	\]
	This yields the complete set of Maxwell's equations:
	\[
	\nabla \cdot \mathbf{E} = \frac{\rho}{\epsilon_0}, \quad \nabla \times \mathbf{B} - \frac{1}{c^2} \frac{\partial \mathbf{E}}{\partial t} = \mu_0 \mathbf{j}.
	\]
	
	\section{Lagrangian Formulation}
	The corresponding Lagrangian is given by:
	\[
	\mathcal{L} = -\frac{1}{4} F_{\mu\nu} F^{\mu\nu} + V(F_{\rho\sigma} F^{\rho\sigma}) - J_\mu \psi^\mu.
	\]
	In the linearized limit, where \(V \approx 0\), this reduces to the standard electrodynamics Lagrangian:
	\[
	\mathcal{L} = -\frac{1}{4} F_{\mu\nu} F^{\mu\nu} - J_\mu A^\mu.
	\]
	
	\section{Discussion on Model Assumptions}
	It is important to note that no physical model can be completely free of assumptions. Even in an idealized framework, certain postulates are necessary to construct a coherent theory. In this derivation:
	\begin{itemize}
		\item The conditions \(V'(0) = 0\) and \(V''(0) = 0\) are imposed to ensure the masslessness of the photon, a requirement that follows from gauge invariance and is supported by experimental observations.
		\item The explicit introduction of the source term \(J_\mu\) is a phenomenological input. In a more complete theory, \(J_\mu\) would arise from the coupling of the electromagnetic field to matter, and its conservation would be embedded in the continuity equation.
		\item The linearization of the field around the vacuum state (\(\psi_{0\mu} = 0\)) is valid for small perturbations, but the full non-linear dynamics may involve additional effects. Nonetheless, this approximation is well justified in the regime where classical electrodynamics applies.
	\end{itemize}
	While a model free from any assumptions would be ideal, the explicit declaration and justification of well-motivated assumptions are not only acceptable but also essential for clarity, transparency, and further refinement as new theoretical or experimental insights emerge.
	
	\section{Conclusion}
	We derive Maxwell’s equations by:
	\begin{enumerate}
		\item Defining the fundamental vector field \(\psi_\mu\) with a gauge-invariant potential \(V(F_{\rho\sigma} F^{\rho\sigma})\),
		\item Decomposing the field as \(\psi_\mu = \psi_{0\mu} + \delta \psi_\mu\),
		\item Linearizing the field equation to obtain \(\square \delta \psi_\mu = J_\mu\),
		\item Identifying \(\delta \psi_\mu = A_\mu\) and enforcing the Lorentz gauge,
		\item Arriving at the full equations \(\partial_\mu F_{\mu\nu} = \mu_0 J_\nu\).
	\end{enumerate}
	The final equations are:
	\[
	\nabla \cdot \mathbf{E} = \frac{\rho}{\epsilon_0}, \quad \nabla \times \mathbf{B} - \frac{1}{c^2} \frac{\partial \mathbf{E}}{\partial t} = \mu_0 \mathbf{j}, \quad \nabla \cdot \mathbf{B} = 0, \quad \nabla \times \mathbf{E} + \frac{\partial \mathbf{B}}{\partial t} = 0.
	\]
	
	\subsection{Summary Table}
	\begin{table}[h]
		\centering
		\begin{tabular}{|c|p{8cm}|}
			\hline
			\textbf{Step} & \textbf{Description} \\
			\hline
			1 & \(\psi_\mu\) defined with a gauge-invariant potential \(V(F_{\rho\sigma} F^{\rho\sigma})\). \\
			2 & Perturbation: \(\psi_\mu = \psi_{0\mu} + \delta \psi_\mu\). \\
			3 & Linearization: \(\square \delta \psi_\mu = J_\mu\). \\
			4 & Identification \(\delta \psi_\mu = A_\mu\) in the Lorentz gauge. \\
			5 & Full equations: \(\partial_\mu F_{\mu\nu} = \mu_0 J_\nu\). \\
			\hline
		\end{tabular}
		\caption{Revised derivation steps.}
	\end{table}
	
	\subsection{References}
	\begin{itemize}
		\item \href{https://physics.stackexchange.com/questions/3005}{Maxwell’s Equations from Lagrangian - Physics Stack Exchange}
	\end{itemize}
	
\end{document}
