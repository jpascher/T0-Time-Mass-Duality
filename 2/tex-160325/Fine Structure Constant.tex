\documentclass{article}
\usepackage[utf8]{inputenc}
\usepackage{amsmath}
\usepackage{amssymb}
\usepackage{physics}
\usepackage{hyperref}
\usepackage{geometry}

\geometry{a4paper, margin=2.5cm}

\title{The Fine Structure Constant: Different Representations and Relationships}
\author{Johann Pascher}
\date{25.2.2025}

\begin{document}
	
	\maketitle
	\tableofcontents
	\section{Introduction to the Fine Structure Constant}
	
	The fine structure constant ($\alpha$) is a dimensionless physical constant that plays a fundamental role in quantum electrodynamics. It describes the strength of the electromagnetic interaction between elementary particles. In its most well-known form, the formula is:
	
	\begin{equation}
		\alpha = \frac{e^2}{4\pi\varepsilon_0\hbar c} \approx \frac{1}{137.035999}
	\end{equation}
	
	\section{Differences between the Fine Inequality and the Fine Structure Constant}
	
	\subsection{Fine Inequality}
	\begin{itemize}
		\item Relates to local hidden variables and Bell inequalities
		\item Investigates whether a classical theory can replace quantum mechanics
		\item Shows that quantum entanglement cannot be described by classical probabilities
	\end{itemize}
	
	\subsection{Fine Structure Constant ($\alpha$)}
	\begin{itemize}
		\item A fundamental natural constant of quantum field theory
		\item Describes the strength of electromagnetic interaction
		\item Determines, for example, the energy separation of fine-structure split spectral lines in atoms
	\end{itemize}
	
	\subsection{Possible Connection}
	Although the Fine inequality and the fine structure constant are fundamentally unrelated, there is an interesting connection through quantum mechanics and field theory:
	
	\begin{itemize}
		\item The fine structure constant plays a central role in quantum electrodynamics (QED), which has a non-local structure
		\item The violation of the Fine inequality suggests that quantum theories are non-local
		\item The fine structure constant influences the strength of these quantum interactions
	\end{itemize}
	
	\section{Alternative Formulations of the Fine Structure Constant}
	
	\subsection{Representation with Permeability}
	Starting from the standard form, we can replace the electric field constant $\varepsilon_0$ with the magnetic field constant $\mu_0$ by using the relationship $c^2 = \frac{1}{\varepsilon_0\mu_0}$:
	
	\begin{align}
		\varepsilon_0 &= \frac{1}{\mu_0c^2}\\
		\alpha &= \frac{e^2}{4\pi\left(\frac{1}{\mu_0c^2}\right)\hbar c}\\
		&= \frac{e^2\mu_0c^2}{4\pi\hbar c}\\
		&= \frac{e^2\mu_0c}{4\pi\hbar}
	\end{align}
	
	With the relationship $\hbar = \frac{h}{2\pi}$, we obtain an alternative form:
	
	\begin{equation}
		\alpha = \frac{\mu_0e^2}{2h}
	\end{equation}
	
	\subsection{Formulation with Electron Mass and Compton Wavelength}
	Planck's constant $h$ can be expressed through other physical quantities:
	
	\begin{equation}
		h = \frac{m_e c \lambda_C}{2\pi}
	\end{equation}
	
	where $\lambda_C$ is the Compton wavelength of the electron:
	
	\begin{equation}
		\lambda_C = \frac{h}{m_e c}
	\end{equation}
	
	Substituting this into the fine structure constant:
	
	\begin{align}
		\alpha &= \frac{\mu_0e^2}{2h}\\
		&= \frac{\mu_0e^2}{2\frac{m_e c \lambda_C}{2\pi}}\\
		&= \frac{\mu_0e^2 \cdot 2\pi}{2m_e c \lambda_C}\\
		&= \frac{\mu_0e^2\pi}{m_e c \lambda_C}
	\end{align}
	
	\subsection{Expression with Classical Electron Radius}
	The classical electron radius is defined as:
	
	\begin{equation}
		r_e = \frac{e^2}{4\pi\varepsilon_0 m_e c^2}
	\end{equation}
	
	With $\varepsilon_0 = \frac{1}{\mu_0c^2}$, we get:
	
	\begin{equation}
		r_e = \frac{e^2\mu_0}{4\pi m_e c^2}
	\end{equation}
	
	The fine structure constant can be written as the ratio of the classical electron radius to the Compton wavelength:
	
	\begin{equation}
		\alpha = \frac{r_e}{\lambda_C}
	\end{equation}
	
	This leads to another form:
	
	\begin{align}
		\alpha &= \frac{e^2\mu_0}{4\pi m_e c^2} \cdot \frac{2\pi m_e c}{h}\\
		&= \frac{e^2\mu_0}{2hc}
	\end{align}
	
	\subsection{Formulation with $\mu_0$ and $\varepsilon_0$ as Fundamental Constants}
	Using the relationship $c = \frac{1}{\sqrt{\mu_0\varepsilon_0}}$, the fine structure constant can be expressed as:
	
	\begin{align}
		\alpha &= \frac{e^2}{4\pi\varepsilon_0\hbar c} \cdot \sqrt{\mu_0\varepsilon_0}\\
		&= \frac{e^2}{4\pi\varepsilon_0\hbar} \cdot \sqrt{\mu_0\varepsilon_0}
	\end{align}
	
	\section{Summary}
	The fine structure constant can be represented in various forms:
	
	\begin{align}
		\alpha &= \frac{e^2}{4\pi\varepsilon_0\hbar c} \approx \frac{1}{137.035999}\\
		\alpha &= \frac{\mu_0e^2}{2h}\\
		\alpha &= \frac{r_e}{\lambda_C}\\
		\alpha &= \frac{e^2}{4\pi\varepsilon_0\hbar} \cdot \sqrt{\mu_0\varepsilon_0}
	\end{align}
	
	These different representations enable various physical interpretations and show the connections between fundamental natural constants.
	
	\section{Questions for Further Investigation}
	
	\begin{enumerate}
		\item How would a change in the fine structure constant affect atomic spectra?
		\item What experimental methods exist to precisely determine the fine structure constant?
		\item Discuss the cosmological significance of a possibly time-varying fine structure constant.
		\item What role does the fine structure constant play in the theory of electroweak unification?
		\item How can the representation of the fine structure constant through the classical electron radius and the Compton wavelength be physically interpreted?
		\item Compare the approaches of Dirac and Feynman to interpret the fine structure constant.
	\end{enumerate}
	\section{Derivation of Planck's Constant Through Fundamental Electromagnetic Constants}
	
	The discussion begins with the question of whether Planck's constant $h$ can be expressed through the fundamental electromagnetic constants $\mu_0$ (magnetic permeability of vacuum) and $\varepsilon_0$ (electric permittivity of vacuum).
	
	\subsection{Relationship Between $h$, $\mu_0$ and $\varepsilon_0$}
	
	First, we consider the fundamental relationship between the speed of light $c$, permeability $\mu_0$, and permittivity $\varepsilon_0$:
	
	\begin{equation}
		c = \frac{1}{\sqrt{\mu_0\varepsilon_0}}
	\end{equation}
	
	We also use the fundamental relation between Planck's constant $h$ and the Compton wavelength $\lambda_C$ of the electron:
	
	\begin{equation}
		h = \frac{m_e c \lambda_C}{2\pi}
	\end{equation}
	
	The Compton wavelength is defined as:
	
	\begin{equation}
		\lambda_C = \frac{h}{m_e c}
	\end{equation}
	
	By substituting the speed of light $c = \frac{1}{\sqrt{\mu_0\varepsilon_0}}$, we obtain:
	
	\begin{equation}
		h = \frac{m_e}{2\pi} \cdot \frac{\lambda_C}{\sqrt{\mu_0\varepsilon_0}}
	\end{equation}
	
	Now we replace $\lambda_C$ with its definition:
	
	\begin{equation}
		h = \frac{m_e}{2\pi} \cdot \frac{h}{m_e c \sqrt{\mu_0\varepsilon_0}}
	\end{equation}
	
	This leads to:
	
	\begin{equation}
		h^2 = \frac{1}{\mu_0\varepsilon_0} \cdot \frac{m_e^2 \lambda_C^2}{4\pi^2}
	\end{equation}
	
	With $\lambda_C = \frac{h}{m_e c}$, it follows:
	
	\begin{equation}
		h^2 = \frac{1}{\mu_0\varepsilon_0} \cdot \frac{m_e^2}{4\pi^2} \cdot \frac{h^2}{m_e^2c^2}
	\end{equation}
	
	After canceling $m_e^2$ and substituting $c^2 = \frac{1}{\mu_0\varepsilon_0}$, we finally get:
	
	\begin{equation}
		h = \frac{1}{2\pi\sqrt{\mu_0\varepsilon_0}}
	\end{equation}
	
	This equation shows that Planck's constant $h$ can indeed be expressed through the electromagnetic vacuum constants $\mu_0$ and $\varepsilon_0$.
	
	\section{Redefinition of the Fine Structure Constant}
	
	\subsection{Question: What does the elementary charge $e$ mean?}
	
	The elementary charge $e$ represents the electric charge of an electron or proton and equals approximately $e \approx 1.602 \times 10^{-19}$ C (Coulomb).
	
	\subsection{The Fine Structure Constant Through Electromagnetic Vacuum Constants}
	
	The fine structure constant $\alpha$ is traditionally defined as:
	
	\begin{equation}
		\alpha = \frac{e^2}{4\pi\varepsilon_0\hbar c}
	\end{equation}
	
	By substituting the derivation for $h$, we get:
	
	\begin{equation}
		\alpha = \frac{e^2}{4\pi\varepsilon_0} \cdot \frac{2\pi\sqrt{\mu_0\varepsilon_0}}{1}
	\end{equation}
	
	This leads to:
	
	\begin{equation}
		\alpha = \frac{e^2}{2} \cdot \frac{\mu_0}{\varepsilon_0}
	\end{equation}
	
	This representation shows that the fine structure constant can be derived directly from the electromagnetic structure of the vacuum, without $h$ appearing explicitly.
	
	\section{Consequences of a Redefinition of the Coulomb}
	
	\subsection{Question: Is Coulomb incorrectly defined if one sets $\alpha = 1$?}
	
	The hypothesis is that if one were to set the fine structure constant $\alpha = 1$, the definition of the Coulomb and thus the elementary charge $e$ would need to be adjusted.
	
	\subsection{New Definition of the Elementary Charge}
	
	If we set $\alpha = 1$, then the elementary charge $e$ would be:
	
	\begin{equation}
		e^2 = 4\pi\varepsilon_0\hbar c
	\end{equation}
	
	\begin{equation}
		e = \sqrt{4\pi\varepsilon_0\hbar c}
	\end{equation}
	
	This would mean that the numerical value of $e$ would change because it would then depend directly on $\hbar$, $c$, and $\varepsilon_0$.
	
	\subsection{Physical Meaning}
	
	The unit Coulomb (C) is an arbitrary definition in the SI system. If one instead chooses $\alpha = 1$, the definition of $e$ would change. In natural unit systems (as commonly used in high-energy physics), $\alpha = 1$ is often set, which means that charge is measured in a unit other than Coulomb.
	
	The current value of the fine structure constant $\alpha \approx \frac{1}{137}$ is not "wrong," but a consequence of our historical definitions of units. Originally, the electromagnetic unit system could have been defined such that $\alpha = 1$ applies.
	
	\section{Effects on Other SI Units}
	
	\subsection{Question: What impact would a Coulomb adjustment have on other units?}
	
	An adjustment of the charge unit so that $\alpha = 1$ would have consequences for numerous other physical units:
	
	\subsubsection{New Charge Unit}
	The new elementary charge would be:
	\begin{equation}
		e = \sqrt{4\pi\varepsilon_0\hbar c}
	\end{equation}
	
	\subsubsection{Change of Electric Current (Ampere)}
	Since $1 \text{ A} = 1 \text{ C}/\text{s}$, the unit of Ampere would also change accordingly.
	
	\subsubsection{Changes of Electromagnetic Constants}
	Since $\varepsilon_0$ and $\mu_0$ are linked to the speed of light:
	\begin{equation}
		c^2 = \frac{1}{\mu_0\varepsilon_0}
	\end{equation}
	either $\mu_0$ or $\varepsilon_0$ would need to be adjusted.
	
	\subsubsection{Effects on Capacitance (Farad)}
	Capacitance is defined as $C = \frac{Q}{V}$. Since $Q$ (charge) changes, the unit of Farad would also change.
	
	\subsubsection{Changes of Voltage Unit (Volt)}
	Electric voltage is defined as $1 \text{ V} = 1 \text{ J}/\text{C}$. Since Coulomb would have a different value, the size of Volt would also shift.
	
	\subsubsection{Indirect Effects on Mass}
	In quantum field theory, the fine structure constant is linked to the rest mass energy of electrons, which could have indirect effects on mass definition.
	
	\section{Natural Units and Fundamental Physics}
	
	\subsection{Question: Why can one set $h$ and $c$ to 1?}
	
	Setting $\hbar = 1$ and $c = 1$ is a simplification with deeper meaning. It's about choosing natural units that follow directly from fundamental physical laws, instead of using human-made units like meters, kilograms, or seconds.
	
	\subsubsection{The Speed of Light $c = 1$}
	The speed of light has the unit meters per second: $c = 299,792,458 \text{ m/s}$. In relativity theory, space and time are inseparable (spacetime). If we measure length units in light-seconds, then meters and seconds fall away as separate concepts – and $c = 1$ becomes a pure ratio.
	
\section{Consequences of a Redefinition of the Coulomb}

\subsection{Question: Is Coulomb incorrectly defined if one sets $\alpha = 1$?}

The hypothesis is that if one were to set the fine structure constant $\alpha = 1$, the definition of the Coulomb and thus the elementary charge $e$ would need to be adjusted.

\subsection{New Definition of the Elementary Charge}

If we set $\alpha = 1$, then the elementary charge $e$ would be:

\begin{equation}
	e^2 = 4\pi\varepsilon_0\hbar c
\end{equation}

\begin{equation}
	e = \sqrt{4\pi\varepsilon_0\hbar c}
\end{equation}

This would mean that the numerical value of $e$ would change because it would then depend directly on $\hbar$, $c$, and $\varepsilon_0$.

\subsection{Physical Meaning}

The unit Coulomb (C) is an arbitrary definition in the SI system. If one instead chooses $\alpha = 1$, the definition of $e$ would change. In natural unit systems (as commonly used in high-energy physics), $\alpha = 1$ is often set, which means that charge is measured in a unit other than Coulomb.

The current value of the fine structure constant $\alpha \approx \frac{1}{137}$ is not "wrong," but a consequence of our historical definitions of units. Originally, the electromagnetic unit system could have been defined such that $\alpha = 1$ applies.

\section{Effects on Other SI Units}

\subsection{Question: What impact would a Coulomb adjustment have on other units?}

An adjustment of the charge unit so that $\alpha = 1$ would have consequences for numerous other physical units:

\subsubsection{New Charge Unit}
The new elementary charge would be:
\begin{equation}
	e = \sqrt{4\pi\varepsilon_0\hbar c}
\end{equation}

\subsubsection{Change of Electric Current (Ampere)}
Since $1 \text{ A} = 1 \text{ C}/\text{s}$, the unit of Ampere would also change accordingly.

\subsubsection{Changes of Electromagnetic Constants}
Since $\varepsilon_0$ and $\mu_0$ are linked to the speed of light:
\begin{equation}
	c^2 = \frac{1}{\mu_0\varepsilon_0}
\end{equation}
either $\mu_0$ or $\varepsilon_0$ would need to be adjusted.

\subsubsection{Effects on Capacitance (Farad)}
Capacitance is defined as $C = \frac{Q}{V}$. Since $Q$ (charge) changes, the unit of Farad would also change.

\subsubsection{Changes of Voltage Unit (Volt)}
Electric voltage is defined as $1 \text{ V} = 1 \text{ J}/\text{C}$. Since Coulomb would have a different value, the size of Volt would also shift.

\subsubsection{Indirect Effects on Mass}
In quantum field theory, the fine structure constant is linked to the rest mass energy of electrons, which could have indirect effects on mass definition.

\section{Natural Units and Fundamental Physics}

\subsection{Question: Why can one set $h$ and $c$ to 1?}

Setting $\hbar = 1$ and $c = 1$ is a simplification with deeper meaning. It's about choosing natural units that follow directly from fundamental physical laws, instead of using human-made units like meters, kilograms, or seconds.

\subsubsection{The Speed of Light $c = 1$}
The speed of light has the unit meters per second: $c = 299,792,458 \text{ m/s}$. In relativity theory, space and time are inseparable (spacetime). If we measure length units in light-seconds, then meters and seconds fall away as separate concepts – and $c = 1$ becomes a pure ratio.

\subsubsection{Planck's Constant $\hbar = 1$}
The reduced Planck constant $\hbar$ has the unit Joule-seconds $= \frac{\text{kg} \cdot \text{m}^2}{\text{s}}$. In quantum mechanics, $\hbar$ determines how large the smallest possible angular momentum or the smallest action can be. If we choose a new unit for action such that the smallest action is simply "1", then $\hbar = 1$.

\subsection{Consequences for Other Units}
When we set $c = 1$ and $\hbar = 1$, the units of everything else change automatically:

\begin{itemize}
	\item Energy and mass are equated: $E = mc^2 \Rightarrow m = E$
	\item Length is measured in units of the Compton wavelength
	\item Time is often measured in inverse energy units
\end{itemize}

This means that we actually only need one fundamental unit – energy – because lengths, times, and masses can all be converted to energy.

\subsection{Significance for Physics}
It's more than just a simplification! It shows that our familiar units (meter, kilogram, second, Coulomb, etc.) are not actually fundamental. They are just human conventions based on our everyday experience.

In natural units, all human-made units of measurement disappear, and physics looks "simpler." The laws of nature themselves have no preferred units – those only come from us!

\section{Energy as a Fundamental Field}

\subsection{Question: Can everything be explained by an energy field?}

If all physical quantities can ultimately be traced back to energy, then there is much to suggest that energy is the most fundamental concept in physics. This would mean:

\begin{itemize}
	\item Space, time, mass, and charge are only different manifestations of energy
	\item A unified energy field could be the basis for all known interactions and particles
\end{itemize}

\subsection{Arguments for a Fundamental Energy Field}

\subsubsection{Mass is a Form of Energy}
According to Einstein, $E = mc^2$, which means that mass is just a bound form of energy.

\subsubsection{Space and Time Arise from Energy}
In General Relativity, energy (or energy-momentum) curves space, suggesting that space itself is only an emergent property of an energy field.

\subsubsection{Charge is a Property of Fields}
In quantum field theory, there are no fundamental particles – only fields. Electrons, for example, are just excitations of the electron field. The electric charge is a property of these excitations, thus also just a manifestation of the energy field.

\subsubsection{All Known Forces are Field Phenomena}
\begin{itemize}
	\item Electromagnetism → Electromagnetic field
	\item Gravitation → Curvature of the spacetime field
	\item Strong force → Gluon field
	\item Weak force → W and Z boson field
\end{itemize}

All these fields ultimately describe only different forms of energy distributions.

\subsection{Theoretical Approaches and Outlook}

The idea of a universal energy field has been discussed in various theoretical approaches:

\begin{itemize}
	\item Quantum Field Theory (QFT): Here, particles are nothing but excitations of fields
	\item Unified Field Theories (e.g., Kaluza-Klein, string theory): These attempt to derive all forces from a single fundamental field
	\item Emergent Gravity (Erik Verlinde): Here, gravity is not considered a fundamental force but an emergent property of an energetic background field
	\item Holographic Principle: This suggests that the entire spacetime can be described by a deeper, energy-related mechanism
\end{itemize}

\begin{itemize}
	\item Formulating a new field theory that derives all known interactions and particles from a single energy distribution
	\item Showing that space and time themselves are only emergent effects of this field (similar to how temperature is only an emergent property of many particle movements)
	\item Explaining how the fine structure constant and other fundamental numerical values follow from this field
\end{itemize}

\section{Summary and Outlook}

The analysis of the fine structure constant and its relationship to other fundamental constants has shown that physics can be simplified at various levels. We have gained the following insights:

\begin{itemize}
	\item Planck's constant $h$ can be expressed through the electromagnetic vacuum constants $\mu_0$ and $\varepsilon_0$.
	\item The fine structure constant $\alpha$ could be normalized to 1, which would lead to a redefinition of the unit Coulomb and other electromagnetic units.
	\item The choice of $\hbar = 1$ and $c = 1$ reveals that our units are ultimately arbitrary conventions and do not fundamentally belong to nature.
	\item The possibility of reducing all fundamental quantities to energy suggests a universal energy field as a fundamental construct.
\end{itemize}

Our discussion has shown that nature can possibly be described much more simply than our current system of units suggests. The necessity of numerous conversion constants between different physical quantities could be an indication that we have not yet captured physics in its most natural form.

\subsection{Historical Context}

The current SI units were developed to facilitate practical measurements in everyday life. They arose from historical conventions and were gradually adjusted to create consistent systems of measurement. The fine structure constant $\alpha \approx \frac{1}{137}$ appears in this system as a fundamental natural constant, although it is actually a consequence of our choice of units.

The development of natural unit systems in theoretical physics shows the striving for a simpler, more fundamental description of nature. The realization that all units can ultimately be reduced to a single one (typically energy) supports the idea of a universal energy field as the basis of all physical phenomena.

\subsection{Outlook for a Unified Theory}

The next major step in theoretical physics could be the development of a completely unified field theory that derives all known interactions and particles from a single fundamental energy field. This would not only include the unification of the four fundamental forces but also explain how space, time, and matter emerge from this field.

The challenge is to formulate a mathematically consistent theory that:

\begin{itemize}
	\item Explains all known physical phenomena
	\item Derives the values of dimensionless natural constants (such as $\alpha$) from first principles
	\item Makes experimentally verifiable predictions
\end{itemize}

Such a theory would potentially revolutionize our understanding of nature and bring us closer to a "theory of everything" that derives the entire universe from a single basic principle.

\section{Mathematical Appendix}

\subsection{Alternative Representation of the Fine Structure Constant}

We can represent the fine structure constant $\alpha$ in various ways:

\begin{equation}
	\alpha = \frac{e^2}{4\pi\varepsilon_0\hbar c} = \frac{e^2}{2} \cdot \frac{\mu_0}{\varepsilon_0} = \frac{1}{137.035999...}
\end{equation}

In a system where $\alpha = 1$ is set, the elementary charge would be redefined as:

\begin{equation}
	e = \sqrt{4\pi\varepsilon_0\hbar c} = \sqrt{\frac{2\varepsilon_0}{\mu_0}}
\end{equation}

\subsection{Natural Units and Dimensional Analysis}

In natural units with $\hbar = c = 1$, we get for the fine structure constant:

\begin{equation}
	\alpha = \frac{e^2}{4\pi\varepsilon_0} = \frac{e^2}{2} \cdot \frac{\mu_0}{\varepsilon_0}
\end{equation}

Planck units go one step further and set $\hbar = c = G = 1$, which leads to the following definitions:

\begin{align}
	\text{Planck length: } l_P &= \sqrt{\frac{\hbar G}{c^3}} \\
	\text{Planck time: } t_P &= \sqrt{\frac{\hbar G}{c^5}} \\
	\text{Planck mass: } m_P &= \sqrt{\frac{\hbar c}{G}} \\
	\text{Planck charge: } q_P &= \sqrt{4\pi\varepsilon_0\hbar c}
\end{align}

These units represent the natural scales of physics and significantly simplify the fundamental equations.

\subsection{Dimensional Analysis of Electromagnetic Units}

The following table shows the dimensions of the most important electromagnetic quantities in different unit systems:

\begin{center}
	\begin{tabular}{|l|c|c|}
		\hline
		\textbf{Quantity} & \textbf{SI Units} & \textbf{Natural Units} ($\hbar = c = 1$) \\
		\hline
		Electric charge $e$ & $\text{C} = \text{A} \cdot \text{s}$ & $\sqrt{\alpha}$ (dimensionless) \\
		Electric field strength $E$ & $\text{V/m} = \text{N/C}$ & $\text{Energy}^2$ \\
		Magnetic field strength $B$ & $\text{T} = \text{Vs/m}^2$ & $\text{Energy}^2$ \\
		Electric permittivity $\varepsilon_0$ & $\text{F/m} = \text{C}^2/(\text{N} \cdot \text{m}^2)$ & $\text{Energy}^{-2}$ \\
		Magnetic permeability $\mu_0$ & $\text{H/m} = \text{N}/\text{A}^2$ & $\text{Energy}^{-2}$ \\
		\hline
	\end{tabular}
\end{center}

This shows that in natural units all electromagnetic quantities can ultimately be reduced to a single dimension – energy.

\section{Philosophical Considerations}

The possibility of reducing all physical quantities to a single energy field has profound philosophical implications:

\subsection{Unity of Nature}

If all physical phenomena – from gravitation to quantum mechanics – are ultimately different manifestations of a single fundamental energy field, this would confirm the ancient idea of the unity of nature. The apparent diversity of natural laws and forces would then be only a consequence of our fragmented perception and description.

\subsection{Emergence and Reductionism}

The idea that complex phenomena like space, time, and matter emerge from a fundamental energy field raises interesting questions about the relationship between emergence and reductionism. While physics traditionally proceeds reductionistically, emergence suggests that the whole is more than the sum of its parts.

\subsection{Epistemological Limits}

The search for a universal energy field as the basis of all physics might encounter epistemological limits: To what extent can we understand a concept that is so fundamental that it even transcends space and time? We might need new mathematical and conceptual tools to grasp this level of reality.

\subsection{The Role of Mathematics}

The fact that physical laws appear simpler and more elegant in natural units raises the question of whether mathematics is the language of nature or just a human-made tool. The discovery that dimensionless constants like $\alpha$ can be derived directly from the structure of the vacuum could indicate a deeper mathematical structure of reality.

\section{Conclusion}

The investigation of the fine structure constant and its relationship to other fundamental constants has led us to a deeper insight into the possible structure of physics. The possibility of redefining the Coulomb and other SI units to set $\alpha = 1$ shows the arbitrariness of our current unit systems.

The realization that all physical quantities can ultimately be reduced to a single dimension – energy – supports the revolutionary idea of a universal energy field as the basis of all physics. This perspective could pave the way for a unified theory that derives all known natural forces and phenomena from a single principle.

While these ideas are still speculative, they offer a fascinating outlook on a possibly more fundamental description of reality that goes beyond our current theories and reveals the unity of nature at a deeper level.

\end{document}