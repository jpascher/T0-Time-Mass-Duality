\documentclass[12pt,landscape]{article}
\usepackage[utf8]{inputenc}
\usepackage[T1]{fontenc}
\usepackage[english]{babel}
\usepackage{amsmath,amssymb,amsfonts}
\usepackage{graphicx}
\usepackage{hyperref}
\usepackage{geometry}
\geometry{a4paper, left=2cm, right=2cm, top=2cm, bottom=2cm}
\usepackage{cite}

\title{On the Non-Proof of Instantaneity in Bell Tests and Implications for a Local Field Model}
\author{Johann Pascher}
\date{25.02.2025}

\begin{document}
	\maketitle
	
	\section*{Introduction}
	The existing Bell tests—including satellite experiments such as Micius—do not conclusively prove instantaneous influence. In this document, I summarize and examine why this is the case and how it relates to the local field model that I propose.
	
	\section*{Why Instantaneity Is Not Proven}
	First, consider the statistical nature of the tests. The Bell inequality is verified over many measurements spanning a significant time interval (the "measurement package"), often lasting from several seconds to minutes. For example, in the Micius experiment the light travel time between Alice and Bob is roughly 4~ms for a separation of 1200~km, which is much shorter than the overall measurement duration. Consequently, any local influence propagating at the speed of light could be hidden within the statistical averaging, as the delay is not resolved.
	
	Furthermore, quantum mechanics does not provide an exact timestamp for measurement events (e.g., precisely when the wave function collapses). Although the simultaneity of measurements at distant locations (Alice and Bob) is defined in a relativistic context, the time-energy uncertainty principle makes precise timing difficult. This temporal imprecision leaves room for doubt regarding the assumption of instantaneity.
	
	Additionally, even in satellite experiments—where the separation (e.g., 1200~km, corresponding to a 4~ms light travel time) is considerable—the measurement duration is still on the order of seconds. This means that for a truly local effect, the entire measurement period would need to be shorter than the light travel time, which has not yet been achieved.
	
	Consider the Micius experiment as an example:
	\begin{itemize}
		\item \textbf{Setup:} 1200~km separation, approximately 4~ms light travel time, with measurement durations spanning several seconds.
		\item \textbf{Result:} A violation of Bell's inequality (with $S>2$) is interpreted as evidence of non-locality.
		\item \textbf{Interpretation:} Given that 4~ms is extremely short compared to the seconds-long data collection, a local effect propagating at speed $c$ cannot be definitively excluded. Thus, the assumption of instantaneity is not directly proven because the timing resolution is insufficient.
	\end{itemize}
	
	\section*{Implications of a Local Field Model}
	The local field model I propose posits that disturbances generated by a measurement propagate through space at the speed of light. This implies:
	\begin{itemize}
		\item \textbf{Local Dynamics:} A measurement at one location (e.g., at Alice) generates a disturbance that propagates to another location (e.g., at Bob) at $c$. At short distances (from meters to kilometers), the light travel time is so brief (nanoseconds to milliseconds) that it would not be resolved in the statistical data. Hence, the observed correlations might result from a delayed field effect rather than an instantaneous one.
		\item \textbf{Plausible Alternative:} Although current tests show strong correlations, they do not definitively rule out a local model because the measurement duration exceeds the light travel time. The local field model remains a plausible alternative until the timing issue is explicitly resolved.
		\item \textbf{Proposed Test:} A more stringent test would require that the separation between measurement stations be very large (e.g., on the order of light-seconds or even millions of kilometers) and that the measurement duration be shorter than the light travel time (e.g., under one second). If the correlations remain unchanged under these conditions, that would support true instantaneity; if they diminish, it would lend support to the local field model.
	\end{itemize}
	
	\section*{What the Tests Prove and What They Do Not}
	Current Bell tests have shown that the observed correlations exceed the limits imposed by local hidden variable theories, thereby excluding classical local models. However, these tests do not directly prove that the influence is instantaneous. The assumption of non-locality is based on the interpretation that no local explanation can account for the data, but the time dynamics are not explicitly tested. In other words, a local field effect propagating at $c$ is not ruled out because the delay is not resolved within the measurement period.
	
	\section*{Conclusion}
	In summary, the existing tests do not conclusively prove instantaneity because:
	\begin{itemize}
		\item The measurement duration exceeds the light travel time, potentially masking a local influence.
		\item The probabilistic nature of quantum mechanics and the inherent uncertainty in time measurements leave room for alternative interpretations.
	\end{itemize}
	The local field model, which posits that influences propagate at the speed of light, remains a plausible alternative until experiments with vastly greater separations (e.g., Earth-Moon distances or beyond) and ultra-short measurement durations (shorter than the light travel time) are performed. Satellite experiments like Micius are a significant step, but they are not sufficient to resolve the time dynamics conclusively. Thus, while non-locality is often assumed, it is not directly proven—instantaneity remains an assumption, not an experimentally confirmed fact.
	
	\begin{thebibliography}{9}
		\bibitem{ScienceAdvances}
		Author(s), \emph{A GHZ-type Paradox in a High-Dimensional Quantum System}, Science Advances, \textbf{Volume}, Pages, Year, DOI: \href{https://doi.org/10.1126/sciadv.abd8080}{10.1126/sciadv.abd8080}.
	\end{thebibliography}
	
\end{document}
