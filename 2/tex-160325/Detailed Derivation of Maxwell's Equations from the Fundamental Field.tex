\documentclass{article}
\usepackage{amsmath}
\usepackage{amsfonts}
\usepackage{amssymb}
\usepackage{hyperref}

\title{Detaillierte Ableitung der Maxwell-Gleichungen aus dem fundamentalen Feld \(\psi\)}
\author{Johann Pascher}
\date{24.02.2025}

\begin{document}
	
	\maketitle
	
	\section{Einleitung}
	Die Maxwell-Gleichungen sind der Grundpfeiler der klassischen Elektrodynamik und setzen elektrische und magnetische Felder in Beziehung zu ihren Quellen – Ladungen und Strömen. Dieses Dokument leitet diese Gleichungen aus einem fundamentalen Feld \(\psi\) ab, das als vierdimensionales Vektorfeld \(\psi_\mu\) angenommen wird. Der Ausgangspunkt ist die Feldgleichung:
	\[
	\left( \frac{\partial^2}{\partial t^2} - c^2 \nabla^2 + V'(\psi) \right) \psi(x,t) = 0,
	\]
	wobei \(V'(\psi)\) die Ableitung eines Potentials \(V(\psi)\) ist und \(\psi = \psi_0 + \delta \psi\), wobei \(\delta \psi\) elektromagnetische Störungen darstellt. Da die Maxwell-Gleichungen Vektorfelder (\(\mathbf{E}\) und \(\mathbf{B}\)) umfassen, interpretieren wir \(\psi\) als ein vierdimensionales Vektorfeld, um es mit dem elektromagnetischen Viererpotential \(A_\mu\) zu verbinden.
	
	\section{Annahme über \(\psi\)}
	Die Maxwell-Gleichungen in relativistischer Form verwenden das Viererpotential \(A_\mu = (\phi/c, \mathbf{A})\), wobei \(\mathbf{E} = -\nabla \phi - \partial \mathbf{A}/\partial t\) und \(\mathbf{B} = \nabla \times \mathbf{A}\). Daher nehmen wir an, dass \(\psi_\mu\) ein vierdimensionales Vektorfeld ist und seine Gleichung lautet:
	\[
	\left( \frac{\partial^2}{\partial t^2} - c^2 \nabla^2 + V'(\psi_\nu \psi^\nu) \right) \psi_\mu = 0,
	\]
	wobei \(\psi_\nu \psi^\nu = \eta_{\mu\nu} \psi^\mu \psi^\nu\) (mit \(\eta_{\mu\nu} = \text{diag}(1, -1, -1, -1)\)) die Lorentz-Invarianz sicherstellt und \(V\) von dieser skalaren Norm abhängt.
	
	\subsection{Equation \#1 in Maxwell's Theory}
	Wir zerlegen \(\psi_\mu = \psi_{0\mu} + \delta \psi_\mu\), wobei \(\psi_{0\mu}\) eine Hintergrundlösung ist (z. B. null im Vakuum) und \(\delta \psi_\mu\) die elektromagnetische Störung ist, die als \(A_\mu\) identifiziert wird. Durch Einsetzen in die Feldgleichung linearisieren wir für kleine \(\delta \psi_\mu\).
	
	\section{Linearisierung}
	Für \(\psi_{0\mu} = 0\) gilt \(\psi_\nu \psi^\nu = \delta \psi_\mu \delta \psi^\mu\). Durch Erweitern von \(V'(\psi_\nu \psi^\nu)\) um null:
	\[
	V'(\psi_\nu \psi^\nu) \approx V'(0) + V''(0) (\delta \psi_\mu \delta \psi^\mu) + \text{höhere Ordnungsterme},
	\]
	nehmen wir an, dass \(V'(0) = 0\) (Minimum bei null) und vernachlässigen höhere Ordnungsterme. Für ein masseloses Feld (wie Elektromagnetismus) nehmen wir \(V''(0) = 0\), was zu Folgendem vereinfacht:
	\[
	\left( \frac{\partial^2}{\partial t^2} - c^2 \nabla^2 \right) \delta \psi_\mu = 0,
	\]
	oder relativistisch:
	\[
	\square \delta \psi_\mu = 0,
	\]
	wobei \(\square = \partial_\mu \partial^\mu\) der d'Alembert-Operator ist.
	
	\section{Verbindung zu den Maxwell-Gleichungen}
	Durch Identifizieren von \(\delta \psi_\mu = A_\mu\) entspricht die Gleichung \(\square A_\mu = 0\) der Wellengleichung für \(A_\mu\) in der Lorentz-Eichung (\(\partial_\mu A^\mu = 0\)). Der Feldstärketensor ist:
	\[
	F_{\mu\nu} = \partial_\mu A_\nu - \partial_\nu A_\mu,
	\]
	und:
	\[
	\partial_\mu F_{\mu\nu} = \square A_\nu - \partial_\nu (\partial_\mu A^\mu) = 0,
	\]
	wodurch die homogenen Maxwell-Gleichungen im Vakuum ergeben:
	- \(\nabla \cdot \mathbf{B} = 0\),
	- \(\nabla \times \mathbf{E} + \frac{\partial \mathbf{B}}{\partial t} = 0\),
	- \(\nabla \cdot \mathbf{E} = 0\),
	- \(\nabla \times \mathbf{B} - \frac{1}{c^2} \frac{\partial \mathbf{E}}{\partial t} = 0\).
	
	\section{Einbeziehung von Quellen}
	Für die inhomogenen Gleichungen (\(\nabla \cdot \mathbf{E} = \rho / \epsilon_0\), \(\nabla \times \mathbf{B} - \frac{1}{c^2} \frac{\partial \mathbf{E}}{\partial t} = \mu_0 \mathbf{j}\)) können nichtlineare Terme in \(V'(\psi_\nu \psi^\nu) \psi_\mu\) als Quellen wirken. Betrachten Sie:
	\[
	\square \delta \psi_\mu = -V'(\psi_\nu \psi^\nu) \psi_\mu,
	\]
	wobei die rechte Seite \(J_\mu = (\rho c, \mathbf{j})\) approximiert, was zu Folgendem führt:
	\[
	\square A_\mu = J_\mu,
	\]
	und somit:
	\[
	\partial_\mu F_{\mu\nu} = J_\nu,
	\]
	was den vollständigen Maxwell-Gleichungen mit Quellen entspricht.
	
	\section{Eichinvarianz}
	Die Maxwell-Gleichungen sind unter \(A_\mu \rightarrow A_\mu + \partial_\mu \lambda\) eichinvariant. Die linearisierte Gleichung \(\square A_\mu = 0\) gilt in der Lorentz-Eichung. Alternativ, mit einem Lagrangian:
	\[
	\mathcal{L} = -\frac{1}{4} F_{\mu\nu} F^{\mu\nu} + V(\psi_\nu \psi^\nu),
	\]
	wird die Eichinvarianz sichergestellt, wobei \(F_{\mu\nu} = \partial_\mu \psi_\nu - \partial_\nu \psi_\mu\).
	
	\section{Diskussion und Schlussfolgerung}
	Wir leiten die Maxwell-Gleichungen ab durch:
	1. Annahme, dass \(\psi_\mu\) ein vierdimensionales Vektorfeld ist,
	2. Perturbation: \(\psi_\mu = \psi_{0\mu} + \delta \psi_\mu\),
	3. Linearisierung zu \(\square \delta \psi_\mu = 0\),
	4. Identifizierung von \(\delta \psi_\mu = A_\mu\),
	5. Definition von \(F_{\mu\nu}\),
	6. Hinzufügen von Quellen durch nichtlineare Terme.
	
	Dieser Ansatz funktioniert für ein Vektorfeld \(\psi_\mu\) und liefert direkt die Maxwell-Gleichungen im Vakuum und mit Quellen.
	
	\subsection{Zusammenfassungstabelle}
	\begin{table}[h]
		\centering
		\begin{tabular}{|c|p{8cm}|}
			\hline
			\textbf{Schritt} & \textbf{Beschreibung} \\
			\hline
			1 & \(\psi\) als vierdimensionales Vektor\(\psi_\mu\). \\
			2 & Perturbation: \(\psi_\mu = \psi_{0\mu} + \delta \psi_\mu\). \\
			3 & Linearisierung: \(\square \delta \psi_\mu = 0\). \\
			4 & \(\delta \psi_\mu = A_\mu\), Lorentz-Eichung. \\
			5 & Homogene Gleichungen: \(\partial_\mu F_{\mu\nu} = 0\). \\
			6 & Quellen: \(\partial_\mu F_{\mu\nu} = J_\mu\). \\
			\hline
		\end{tabular}
		\caption{Ableitungsschritte.}
	\end{table}
	
	\subsection{Wichtige Erkenntnisse}
	- Homogene Gleichungen ergeben sich aus der linearisierten Wellengleichung.
	- Quellen entstehen durch nichtlineare Terme oder stabile Feldkonfigurationen.
	
	\subsection{Referenzen}
	- \href{https://en.wikipedia.org/wiki/Maxwell\%27s_equations}{Maxwell-Gleichungen - Wikipedia}
	- \href{https://physics.stackexchange.com/questions/3005/derivation-of-maxwells-equations-from-field-tensor-lagrangian}{Maxwell-Gleichungen aus Lagrangian - Physics Stack Exchange}
	
\end{document}