\documentclass[a4paper,12pt]{article}
\usepackage[utf8]{inputenc}
\usepackage{amsmath, amssymb}

\title{Sphärische Geometrie und Hypersphäre}
\author{Johann Pascher}
\date{März 19, 2025}

\begin{document}
	
	\maketitle
	
	\section{Sphärische Geometrie}
	Die Entfernung auf einer Kugeloberfläche (Geodät) wird entlang eines Großkreises berechnet:
	\[
	d = R \cdot \arccos(\cos\theta_A \cos\theta_B + \sin\theta_A \sin\theta_B \cos(\phi_B - \phi_A))
	\]
	Die Sehnenlänge (direkt durch die Kugel) ist:
	\[
	d_{\text{Sehne}} = 2R \sin\left(\frac{\alpha}{2}\right)
	\]
	
	\subsection{Unterschiede zur euklidischen Geometrie}
	\begin{itemize}
		\item Winkelsumme eines Dreiecks: \(> 180^\circ\).
		\item Keine Parallelen: Großkreise schneiden sich.
	\end{itemize}
	
	\section{Hypersphäre als Universum}
	Falls der Raum eine Hypersphäre (\(k = +1\)) ist:
	\begin{itemize}
		\item Geodäten: \(d_{\text{Geodät}} = R_0 \alpha\).
		\item Verhältnis: \(\frac{d_{\text{Geodät}}}{d_{\text{Sehne}}} = \frac{\alpha}{2 \sin(\alpha/2)}\).
	\end{itemize}
	Für kleine \(\alpha\): \(\approx 1\), für große \(\alpha\): signifikant größer.
	
	\section{Ausdehnung des Universums}
	Mit der Ausdehnung wird die Geodät dynamisch:
	\[
	d_{\text{Geodät}}(t) = a(t) R_0 \alpha
	\]
	wobei \(a(t)\) der Skalenfaktor ist, der mit der Zeit wächst.
	
\end{document}