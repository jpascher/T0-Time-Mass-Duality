\documentclass[12pt,landscape]{article}
\usepackage[utf8]{inputenc}
\usepackage[T1]{fontenc}
\usepackage[english]{babel}
\usepackage{amsmath,amssymb}
\usepackage{geometry}
\geometry{a4paper, left=2cm, right=2cm, top=2cm, bottom=2cm}
\usepackage{graphicx}
\usepackage{hyperref}

\title{An Extended Approach to Bell's Inequality: A Field-Theoretical Perspective}
\author{Johann Pascher}
\date{25.2.2025}

\begin{document}
	\maketitle
	
	\begin{abstract}
		In this document, I present an extended formulation of Bell's inequality that transforms it into an equation by incorporating continuous transitions. This approach aims to bridge the gap between the predictions of local realistic models and the observed quantum mechanical correlations. A field-theoretical perspective is postulated as a potential framework that may more closely represent the underlying reality.
	\end{abstract}
	
	\section*{Introduction}
	Traditional Bell inequalities, such as the CHSH inequality, are derived under the assumption that measurement outcomes are dichotomic (typically $\pm1$) and predetermined by local hidden variables. Under these assumptions, the classical limit is expressed as
	\[
	S \leq 2\,.
	\]
	However, quantum mechanical predictions exhibit a continuous dependence on the measurement settings. For instance, the correlation function
	\[
	E(\theta) = -\cos(2\theta)
	\]
	demonstrates that the strength of the correlations varies gradually with the measurement angles.
	
	\section*{The Extended Equation}
	To capture this continuous behavior, I propose the following extended equation:
	\[
	\bigl|E(a,b) + E(a,b') + E(a',b) - E(a',b')\bigr| = 2 + 2\bigl(\sqrt{2}-1\bigr)\,\sin^2\Bigl(2\bigl(\theta_{ab}-\theta_{ab'}\bigr)\Bigr)\,.
	\]
	This formulation possesses two key features:
	\begin{enumerate}
		\item When $\theta_{ab} = \theta_{ab'}$, the sine term vanishes, and the equation reduces to the classical limit of 2.
		\item When $\sin^2\Bigl(2\bigl(\theta_{ab}-\theta_{ab'}\bigr)\Bigr)=1$, the expression attains the quantum mechanical maximum of $2\sqrt{2}$.
	\end{enumerate}
	Thus, this equation continuously connects the classical limit and the quantum mechanical predictions based on the relative difference in the measurement angles.
	
	\section*{A Field-Theoretical Perspective}
	From a field-theoretical point of view, physical processes are inherently dynamic and continuous. In such a framework:
	\begin{itemize}
		\item \textbf{Dynamic Field Disturbances:} Rather than experiencing a sudden collapse into fixed values, the measurement process can be viewed as a dynamic propagation of field disturbances that gradually influence the outcome. This results in a continuous variation of the measured correlations.
		\item \textbf{Integrated Measurement Systems:} Instead of considering the measurement device as an external observer that forces a binary outcome, the measurement and the field can be treated as an integrated system. In this view, discrete outcomes emerge from an underlying continuous process, which is better captured by the extended equation.
	\end{itemize}
	This perspective suggests that the extended equation is not merely a mathematical reformulation of Bell's inequality—it represents an attempt to model the measurement process in a way that more closely approximates the true ontological nature of the physical world.
	
	\section*{Conclusion}
	The extended equation
	\[
	\bigl|E(a,b) + E(a,b') + E(a',b) - E(a',b')\bigr| = 2 + 2\bigl(\sqrt{2}-1\bigr)\,\sin^2\Bigl(2\bigl(\theta_{ab}-\theta_{ab'}\bigr)\Bigr)
	\]
	offers a continuous framework that bridges the gap between local realistic models and the quantum mechanical behavior observed in experiments. By allowing for continuous transitions, this formulation extends the conventional Bell inequality into an equation that incorporates gradual changes in correlation strengths. I propose that a fundamental field-theoretical approach may underlie this behavior, suggesting that our measurement outcomes result from dynamic field interactions rather than from static, predetermined values.
	
	Sincerely,\\
	Johann Pascher
	
\end{document}
