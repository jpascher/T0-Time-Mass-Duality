\documentclass[a4paper,12pt]{article}
\usepackage[utf8]{inputenc}
\usepackage{amsmath, amssymb}

\title{Friedmann-Gleichungen und kosmologische Dynamik}
\author{Johann Pascher}
\date{März 19, 2025}

\begin{document}
	
	\maketitle
	
	\section{Friedmann-Gleichungen}
	Die Dynamik der Ausdehnung wird durch zwei Gleichungen beschrieben:
	
	\subsection{Erste Gleichung}
	\[
	\left(\frac{\dot{a}}{a}\right)^2 = \frac{8\pi G}{3} \rho - \frac{k c^2}{R_0^2 a^2} + \frac{\Lambda c^2}{3}
	\]
	Anteile:
	\begin{itemize}
		\item \(\rho\): Materie/Energie, bremsend.
		\item \(k\): Krümmung (\(k = +1\) für Hypersphäre).
		\item \(\Lambda\): Dunkle Energie, fördernd.
	\end{itemize}
	
	\subsection{Zweite Gleichung}
	\[
	\frac{\ddot{a}}{a} = -\frac{4\pi G}{3} \left(\rho + \frac{3p}{c^2}\right) + \frac{\Lambda c^2}{3}
	\]
	Materie bremst, \(\Lambda\) beschleunigt.
	
	\section{Ohne Gravitation}
	Ohne Materie (\(\rho_m = \rho_r = 0\)):
	\[
	\left(\frac{\dot{a}}{a}\right)^2 = \frac{\Lambda c^2}{3} - \frac{c^2}{R_0^2 a^2}, \quad \frac{\ddot{a}}{a} = \frac{\Lambda c^2}{3}
	\]
	Relativistischer Anteil: \(\Lambda\) treibt die Ausdehnung.
	
	\section{Sehnen vs. Geodäten}
	Mit Ausdehnung:
	\[
	\frac{d_{\text{Geodät}}(t)}{d_{\text{Sehne}}} = a(t) \cdot \frac{\alpha}{2 \sin(\alpha/2)}
	\]
	Kleine Distanzen: \(\approx a(t)\), große Distanzen: deutlich größer.
	
	\section{Fazit}
	Die ART umfasst die Ausdehnung und Krümmung, die SRT nicht. Die Zukunft von \(a(t)\) ist ungewiss, aber \(\Lambda\) dominiert die Dynamik.
	
\end{document}