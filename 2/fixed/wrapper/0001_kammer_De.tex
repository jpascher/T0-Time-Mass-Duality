\documentclass[12pt,a4paper]{report}
% Minimale T0 Standalone Preamble - A4 Format - 25 Zeilen
\RequirePackage{fontspec}
\RequirePackage{unicode-math}
\usepackage[ngerman]{babel}
\usepackage{microtype}
\setmainfont{Inter}
\setmonofont{JetBrains Mono}
\setmathfont{Libertinus Math}
\usepackage{amsmath,amsfonts,amsthm}
\usepackage{mathtools}
\usepackage{graphicx}
\usepackage{xcolor}
\definecolor{t0blue}{RGB}{0,102,204}
\definecolor{t0green}{RGB}{34,139,34}
\definecolor{t0red}{RGB}{204,0,0}
\usepackage{geometry}
\geometry{a4paper,margin=2.5cm}
\usepackage[most]{tcolorbox}
\newtcolorbox{keyresult}[1][]{colback=yellow!5,colframe=t0blue!80,fonttitle=\bfseries,title={#1},breakable}
\newtcolorbox{important}[1][]{colback=red!5,colframe=t0red!80,fonttitle=\bfseries,title={#1},breakable}
\newcommand{\Tfield}{\ensuremath{\mathcal{T}}}
\usepackage{hyperref}
\hypersetup{colorlinks=true,linkcolor=t0blue}

\title{Analyse zur Instrumentenakustik und Wahrnehmungspsychologie}
\author{Johann Pascher \\ \textit{Technische und empirische Betrachtung}}
\date{\today}

\begin{document}
	
	\maketitle
	
	\section{Einleitung}
	Die Erzeugung eines harmonischen Klangbildes bei Musikinstrumenten ist ein komplexes Zusammenspiel aus physikalischen Gegebenheiten und subjektiver Wahrnehmung. Diese Arbeit untersucht die Grenzen der rechnerischen Modellierbarkeit und die Rolle des menschlichen Gehörs als finale Instanz.
	
	\section{Physikalische und strukturelle Parameter}
	\subsection{Material und Schwingungsverhalten}
	Die Effizienz eines Instruments hängt maßgeblich von der Steifigkeit und Masse des Korpus ab. Jede interne Vibration, die nicht von der Stimmzunge initiiert wird, stellt einen passiven Energieverlust dar. 
	\begin{itemize}
		\item \textbf{Wandstärken:} Eine höhere Masse wirkt sich positiv auf die Energieübertragung aus.
		\item \textbf{Materialwahl:} Hölzer wie Fichte und Zirbe haben sich als klanglich ideal erwiesen.
		\item \textbf{Basskammern:} Die Maserung (Ausrichtung oder Wuchsform) der Trennwände erwies sich in der empirischen Praxis als vernachlässigbar für das Klangresultat.
	\end{itemize}
	
	\section{Die Grenzen der Modellierbarkeit}
	Die mathematische Simulation mittels Finite-Elemente-Methoden (FEM) stößt bei komplexen Musikinstrumenten an ihre prinzipielle Grenze. Die Varianz der natürlichen Werkstoffe kombiniert mit der schieren Anzahl an Einzelteilen erzeugt ein System, dessen Komplexität an Berechnungen auf dem Niveau der sub-Planck-Länge erinnert. Eine KI-gestützte Modellierung kann aufgrund der Einzigartigkeit jedes Holzstücks nur Näherungswerte liefern.
	
	\section{Psychoakustik und subjektive Wahrnehmung}
	Über die rein physikalische Schwingung hinaus definiert sich das Klangbild durch:
	\begin{itemize}
		\item \textbf{Formatierung und Frequenzverhalten:} Das subjektive Empfinden wird durch spezifische Frequenzbereiche bestimmt, die als angenehmer oder "natürlicher" wahrgenommen werden.
		\item \textbf{Konditionierung:} Die individuelle Wahrnehmung ist stark durch kulturelle und persönliche Prägung beeinflusst. Was als "optimales Klangbild" empfunden wird, variiert zwischen Individuen signifikant.
	\end{itemize}
	
	\section{Fazit}
	Das Streben nach einem perfekten Klangbild ist eine Balance aus der Minimierung passiver Energieverluste (durch physikalische Optimierung) und der Gestaltung harmonischer Frequenzspektren. Da die menschliche Wahrnehmung jedoch ein hochgradig individueller Filter ist, bleibt die endgültige "Bewertung" eines Instruments ein subjektiver Prozess, der sich durch kein Berechnungsmodell vollständig vorhersagen lässt.
	
\end{document}