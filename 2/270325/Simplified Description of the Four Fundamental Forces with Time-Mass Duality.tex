\documentclass{article}
\usepackage[utf8]{inputenc}
\usepackage{amsmath}
\usepackage{amssymb}
\usepackage{hyperref}
\usepackage{geometry}
\usepackage{tocloft}
\geometry{a4paper, margin=2cm}
\title{Simplified Description of the Four Fundamental Forces with Time-Mass Duality}
\author{Johann Pascher}
\date{March 26, 2025}
\begin{document}
	\maketitle
	
	\tableofcontents
	\newpage
	
	\section{Unified Lagrangian Density with Dual Time-Mass Concept}
	
	The Lagrangian density for the four fundamental forces (strong nuclear force, electromagnetic force, weak nuclear force, and gravitation) can be summarized in a simplified form that now incorporates the time-mass duality:
	
	\begin{equation}
		\mathcal{L}_\text{total} = \mathcal{L}_\text{Gravitation} + \mathcal{L}_\text{SM} + \mathcal{L}_\text{Higgs} + \mathcal{L}_\text{intrinsic},
	\end{equation}
	
	where:
	\begin{itemize}
		\item \(\mathcal{L}_\text{Gravitation}\) describes the Lagrangian density of gravitation,
		\item \(\mathcal{L}_\text{SM}\) represents the Lagrangian density of the Standard Model (strong, electromagnetic, and weak forces),
		\item \(\mathcal{L}_\text{Higgs}\) is the Lagrangian density of the Higgs field,
		\item \(\mathcal{L}_\text{intrinsic}\) is the new Lagrangian density accounting for intrinsic time.
	\end{itemize}
	
	\subsection{Gravitation}
	The gravitation is described by the Einstein-Hilbert action but can now be expressed in two complementary forms:
	
	\begin{equation}
		\mathcal{L}_\text{Gravitation} = -\frac{1}{16\pi G} \sqrt{-g} R,
	\end{equation}
	
	in the Standard Model (with time dilation), and:
	
	\begin{equation}
		\mathcal{L}_\text{Gravitation-T} = -\frac{1}{16\pi G_T} \sqrt{-g_T} R_T,
	\end{equation}
	
	in the complementary model (with absolute time and mass variation), where \(G_T = G \cdot \frac{T_0}{T}\) is a modified Newton constant dependent on the intrinsic time \(T = \frac{\hbar}{mc^2}\), and \(T_0\) is a reference time scale (e.g., the Planck time).
	
	\subsection{Standard Model}
	The Lagrangian density of the Standard Model encompasses the strong, electromagnetic, and weak forces and can also be formulated dually:
	
	\begin{equation}
		\mathcal{L}_\text{SM} = \mathcal{L}_\text{strong} + \mathcal{L}_\text{em} + \mathcal{L}_\text{weak},
	\end{equation}
	
	where:
	\begin{itemize}
		\item \(\mathcal{L}_\text{strong} = -\frac{1}{4} F_{\mu\nu}^a F^{a\mu\nu} + \bar{\psi}(i \gamma^\mu D_\mu - m_\psi(\phi))\psi\) describes the strong nuclear force,
		\item \(\mathcal{L}_\text{em} = -\frac{1}{4} F_{\mu\nu} F^{\mu\nu} + \bar{\psi}(i \gamma^\mu D_\mu - m_\psi(\phi))\psi\) describes the electromagnetic force,
		\item \(\mathcal{L}_\text{weak} = -\frac{1}{4} W_{\mu\nu}^a W^{a\mu\nu} + \bar{\psi}(i \gamma^\mu D_\mu - m_\psi(\phi))\psi\) describes the weak nuclear force.
	\end{itemize}
	
	The complementary formulation with intrinsic time is:
	
	\begin{equation}
		\mathcal{L}_\text{SM-T} = \mathcal{L}_\text{strong-T} + \mathcal{L}_\text{em-T} + \mathcal{L}_\text{weak-T},
	\end{equation}
	
	where the time derivative is now with respect to intrinsic time \(T\): \(\partial_t \rightarrow \partial_{t/T}\).
	
	\subsection{Higgs Field}
	The Lagrangian density of the Higgs field is:
	
	\begin{equation}
		\mathcal{L}_\text{Higgs} = (D_\mu \phi)^\dagger (D^\mu \phi) - V(\phi),
	\end{equation}
	
	where \(\phi\) is the Higgs field and \(V(\phi) = \mu^2 \phi^\dagger \phi + \lambda (\phi^\dagger \phi)^2\) describes the Higgs potential.
	
	In the complementary formulation with intrinsic time, this becomes:
	
	\begin{equation}
		\mathcal{L}_\text{Higgs-T} = (D_{T\mu} \phi_T)^\dagger (D_T^\mu \phi_T) - V_T(\phi_T),
	\end{equation}
	
	where the covariant derivative \(D_{T\mu}\) accounts for intrinsic time.
	
	\subsection{Lagrangian Density for Intrinsic Time}
	The new component incorporating the time-mass duality is:
	
	\begin{equation}
		\mathcal{L}_\text{intrinsic} = \bar{\psi}\left(i\hbar\gamma^0 \frac{\partial}{\partial (t/T)} - i\hbar\gamma^0 \frac{\partial}{\partial t}\right)\psi,
	\end{equation}
	
	where \(T = \frac{\hbar}{mc^2}\) is the intrinsic time, dependent on the mass of the particle under consideration.
	
	\section{Simplified Description of Mass Terms with Time-Mass Duality}
	
	The mass terms of particles can now be presented in dual forms:
	
	\begin{itemize}
		\item Standard Model (time dilation): \(m_\psi(\phi) = y_\psi \phi\) with constant mass and variable time,
		\item Complementary Model (mass variation): \(m_\psi(\phi_T) = y_\psi \phi_T \cdot \gamma\) with absolute time and variable mass,
	\end{itemize}
	
	where \(\gamma = \frac{1}{\sqrt{1-v^2/c^2}}\) is the Lorentz factor.
	
	\section{Asymptotic Safety with Intrinsic Time}
	
	Asymptotic safety in quantum gravitation can be described by the modified renormalization group flow equation:
	
	\begin{equation}
		\partial_{t/T} \Gamma_k[g] = \frac{1}{2} \text{Tr}\left[\left(\Gamma_k^{(2)}[g] + R_k\right)^{-1} \partial_{t/T} R_k\right]
	\end{equation}
	
	with the effective action \(\Gamma_k\) at scale \(k\) and the regulator term \(R_k\).
	
	The dimensionless couplings are adjusted accordingly:
	
	\begin{align}
		g_k &= G_k k^2 \rightarrow g_{k,T} = G_k (kT)^2 \\
		\lambda_k &= \Lambda_k/k^2 \rightarrow \lambda_{k,T} = \Lambda_k/(kT)^2
	\end{align}
	
	This leads to modified beta functions:
	
	\begin{align}
		\beta_g &= (2 + \eta_N)g_k \rightarrow \beta_{g,T} = (2 + \eta_N + \eta_T)g_{k,T} \\
		\beta_\lambda &= -2\lambda_k + f(g_k,\lambda_k) \rightarrow \beta_{\lambda,T} = -2\lambda_{k,T} + f_T(g_{k,T},\lambda_{k,T})
	\end{align}
	
	where \(\eta_T\) represents the anomalous dimension with respect to intrinsic time.
	
	\section{The Higgs Field as a Universal Medium with Intrinsic Time}
	
	The concept of the Higgs field as a medium influencing all other particles and fields is extended by the notion of intrinsic time. The Higgs field may not only be responsible for mass generation but also for the intrinsic time scale of particles:
	
	\begin{equation}
		T = \frac{\hbar}{m(\phi)c^2} = \frac{\hbar}{y_\psi \phi \cdot c^2}
	\end{equation}
	
	This relation indicates that a particle's intrinsic time is inversely proportional to its mass generated by the Higgs field.
	
	\section{The Higgs Field and the Vacuum: A Complex Relationship with Intrinsic Time}
	
	The relationship between the Higgs field and the vacuum becomes more complex with the concept of intrinsic time. The vacuum energy could be reinterpreted as:
	
	\begin{equation}
		E_\text{vac} = \sum_i \frac{\hbar \omega_i}{2} = \sum_i \frac{\hbar}{2T_i}
	\end{equation}
	
	This formulation directly links vacuum energy to the intrinsic time of quantum fluctuations.
	
	\section{Quantum Entanglement and Nonlocality in Time-Mass Duality}
	
	The apparent instantaneity in quantum entanglement can be reinterpreted through the time-mass duality:
	
	\begin{itemize}
		\item In the absolute time model (\(T_0\)-model), correlations do not occur instantaneously but through mass variation.
		\item In the intrinsic time model, entangled particles of different masses would experience different time evolutions proportional to their intrinsic time scales.
		\item For photons, the intrinsic time could be defined as \(T = \frac{\hbar}{E_{\gamma}} e^{\alpha x}\), where \(\alpha = \frac{H_0}{c} \approx 2.3 \times 10^{-28} \text{ m}^{-1}\) accounts for energy loss over distance \(x\), consistent with the T0 model. This absorption coefficient determines the rate of energy loss of photons to the dark energy field.
	\end{itemize}
	
	\section{Cosmological Implications of Time-Mass Duality}
	
	The time-mass duality framework provides natural explanations for several cosmological phenomena through the following key parameters:
	
	\begin{itemize}
		\item The absorption coefficient \(\alpha = H_0/c \approx 2.3 \times 10^{-28} \text{ m}^{-1}\) determines the rate of energy loss of photons to the dark energy field, explaining cosmological redshift beyond the standard Doppler interpretation.
		
		\item The parameter \(\kappa \approx 4.8 \times 10^{-7} \text{ GeV/cm}\cdot\text{s}^{-2}\) characterizes the strength of the dark energy field in galactic dynamics, providing a modified gravitational potential that can explain flat rotation curves without dark matter:
		\[
		\Phi(r) = -\frac{GM}{r} + \kappa r
		\]
		
		\item The dimensionless coupling constant \(\beta \approx 10^{-3}\) describes the interaction strength between the dark energy field and baryonic matter. These parameters are related through:
		\[
		\kappa = \frac{\beta^2 H_0^2 M_{\text{Pl}}^2}{c^2 \rho_0}
		\]
		where \(\rho_0\) is the critical density of the universe.
	\end{itemize}
	
	This leads to the prediction that Bell tests with particles of different masses or photons of different frequencies might reveal measurable delays in correlations, proportional to the mass ratio \(\frac{m_1}{m_2}\) or energy ratio \(\frac{E_1}{E_2}\).
	
	\section{Summary of the Unified Theory}
	
	The complete unified theory can be described by the following action:
	
	\begin{equation}
		S_\text{unified} = \int \left( \mathcal{L}_\text{standard} + \mathcal{L}_\text{complementary} + \mathcal{L}_\text{coupling} \right) d^4x
	\end{equation}
	
	where:
	\begin{align}
		\mathcal{L}_\text{standard} &= -\frac{1}{16\pi G} \sqrt{-g} R + \mathcal{L}_\text{SM} + (D_\mu \phi)^\dagger (D^\mu \phi) - V(\phi) \\
		\mathcal{L}_\text{complementary} &= -\frac{1}{16\pi G_T} \sqrt{-g_T} R_T + \mathcal{L}_\text{SM-T} + (D_{T\mu} \phi_T)^\dagger (D_T^\mu \phi_T) - V_T(\phi_T) \\
		\mathcal{L}_\text{coupling} &= \int \mathcal{D}[\Psi] \, \Psi^* \left( i\hbar \frac{\partial}{\partial t} - i\hbar \frac{\partial}{\partial (t/T)} \right) \Psi
	\end{align}
	
	This unified theory offers several significant advantages:
	\begin{itemize}
		\item It bridges gaps between quantum mechanics and quantum field theory.
		\item It provides a new perspective on quantum entanglement and nonlocality.
		\item It opens new avenues for quantum gravitation.
		\item It offers deeper insights into the Higgs field and vacuum.
		\item It leads to experimentally testable predictions.
	\end{itemize}
	
	\section{Experimental Testability}
	
	The proposed unified theory with time-mass duality leads to several experimentally testable predictions:
	
	\begin{enumerate}
		\item Measurement of photon energy loss consistent with \(\alpha = H_0/c\) at cosmological distances
		\item Detection of modified gravitational potentials in galaxies characterized by \(\kappa \approx 4.8 \times 10^{-7} \text{ GeV/cm}\cdot\text{s}^{-2}\)
		\item Precision tests of the matter-dark energy coupling constant \(\beta \approx 10^{-3}\)
		\item Mass-dependent time evolution in quantum systems, measurable as different coherence times.
		\item Differences in entanglement speed for particles of different masses.
		\item Scale-dependent gravitational constant correlated with intrinsic time.
		\item Modified energy-momentum relation for very massive particles.
		\item Measurable deviations in high-precision experiments typically explained by time dilation.
	\end{enumerate}
	
	\section{References to Further Works}
	
	The unified theory presented here builds on a series of detailed studies addressing various aspects of the time-mass duality and its applications:
	
	\begin{itemize}
		\item \textbf{``Complementary Extensions of Physics: Absolute Time and Intrinsic Time'' (03/24/2025)} \\
		This foundational work introduces the \(T_0\)-model with absolute time and variable mass, providing the theoretical basis for the time-mass duality. It includes the full mathematical derivation of the transformation between the Standard Model (time dilation) and the complementary model (mass variation).
		
		\item \textbf{``A Model with Absolute Time and Variable Energy: A Comprehensive Investigation of the Foundations'' (03/24/2025)} \\
		This work presents the detailed physical justification for the \(T_0\)-model and demonstrates its compatibility with experimental observations. It explains how phenomena typically attributed to time dilation can also be explained by mass variation.
		
		\item \textbf{``Extensions of Quantum Mechanics through Intrinsic Time'' (03/24/2025)} \\
		This work develops the complete reformulation of quantum mechanics incorporating intrinsic time. It includes the derivation of the modified Schrödinger equation \(i\hbar \frac{\partial}{\partial (t/T)} \Psi = \hat{H} \Psi\) and demonstrates its applications to various quantum mechanical phenomena.
		
		\item \textbf{``Integration of Time-Mass Duality into Quantum Field Theory'' (03/25/2025)} \\
		This work presents the complete reformulation of quantum field theory with intrinsic time. It shows how field operators can be reformulated with respect to intrinsic time and how renormalization can be reinterpreted through mass-dependent time scales.
		
		\item \textbf{``Dynamic Mass of Photons and Their Implications for Nonlocality'' (03/25/2025)} \\
		This work examines the application of time-mass duality to massless particles like photons and develops the concept of frequency-dependent intrinsic time for photons. It includes predictions for experiments measuring delays in quantum correlations.
		
		\item \textbf{``Fundamental Constants and Their Derivation from Natural Units'' (03/25/2025)} \\
		This work demonstrates how the fundamental constants of physics can be reinterpreted within the framework of time-mass duality and how they relate to the concept of intrinsic time.
		
		\item \textbf{``Real Consequences of Reformulating Time and Mass in Physics: Beyond the Planck Scale'' (03/24/2025)} \\
		This work explores the consequences of time-mass duality for phenomena beyond the Planck scale and demonstrates how it could resolve potential issues in quantum gravitation.
	\end{itemize}
	
\end{document}

