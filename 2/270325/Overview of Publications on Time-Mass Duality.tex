\documentclass[a4paper,12pt]{article}
\usepackage[utf8]{inputenc}
\usepackage[T1]{fontenc}
\usepackage{lmodern}
\usepackage{csquotes}
\usepackage{hyperref}
\usepackage{xcolor}
\usepackage{geometry}
\usepackage{booktabs}
\usepackage{array}
\usepackage{tabularx}
\usepackage{fancyhdr}

\geometry{a4paper, margin=2.5cm}
\hypersetup{
	colorlinks=true,
	linkcolor=blue,
	filecolor=magenta,      
	urlcolor=blue,
	pdftitle={Overview of Publications on Time-Mass Duality},
	pdfauthor={Johann Pascher},
	pdfcreator={LaTeX}
}

% Repository base URL
\newcommand{\repobase}{https://github.com/jpascher/T0-Time-Mass-Duality/tree/main/2/}

\pagestyle{fancy}
\fancyhf{}
\rhead{Johann Pascher}
\lhead{Time-Mass Duality}
\cfoot{\thepage}

\title{Overview of Publications on Time-Mass Duality \\ \Large{A Theoretical Framework for Extending Modern Physics}}
\author{Johann Pascher}
\date{March 2025}

\begin{document}
	
	\maketitle
	
	\begin{abstract}
		This overview presents a cohesive collection of works that develop a new theoretical framework for extending modern physics. At its center is the concept of time-mass duality, which proposes a fundamental reformulation of the relationship between time and mass. This approach not only offers a possible path toward unifying quantum mechanics, quantum field theory, and general relativity, but also provides new perspectives on fundamental phenomena such as nonlocality, dark matter, and dark energy. The publications listed here form a coherent research program that ranges from theoretical foundations to concrete applications and experimentally verifiable predictions.
	\end{abstract}
	
	\section{Introduction}
	
	The publications listed below represent a cohesive corpus of works that develop various aspects of a new theoretical framework for modern physics. Central to this framework is the concept of time-mass duality, which proposes a fundamental reinterpretation of the relationship between time and mass, with far-reaching implications for our understanding of physical reality—from quantum mechanics to cosmology.
	
	The works are organized into five thematic areas:
	\begin{enumerate}
		\item Fundamental Theory Development
		\item Specific Applications and Implications
		\item Fundamental Constants and Units
		\item Cosmological and Boundary Areas
		\item Summary Papers and Overviews
	\end{enumerate}
	
	All documents are available in the repository and can be accessed directly via the provided links.
	
	\section{Fundamental Theory Development}
	
	These publications establish the basic conceptual and mathematical foundations of time-mass duality and its extensions to standard physics.
	
	\subsection{\href{\repobase pdf/English/The Necessity of Extending Standard Quantum Mechanics and Quantum Field Theory.pdf}{The Necessity of Extending Standard Quantum Mechanics and Quantum Field Theory}}
	\textit{(276,307 bytes, 27.03.2025)}
	
	This foundational work identifies critical conceptual limitations of existing quantum theories, particularly regarding the asymmetric treatment of time and space, and the static consideration of mass. It introduces the concept of intrinsic time $T = \hbar/mc^2$ and develops an extended Lagrangian formalism that integrates time-mass duality into quantum field theory. A central aspect is the possibility of returning to a deterministic understanding of the quantum world through variable mass as a fundamental hidden variable, without conflicting with experimental findings.
	
	\subsection{\href{\repobase pdf/English/complementary-extensions.pdf}{Complementary Extensions of Physics}}
	\textit{(190,434 bytes, 25.03.2025)}
	
	This work presents the two complementary perspectives that form the core of time-mass duality: the standard model with constant mass and variable time (time dilation) and the $T_0$ model with absolute time and variable mass. It demonstrates how these dual perspectives are mathematically interconvertible and can describe the same physical phenomena, albeit with different conceptual foundations. The paper develops the formal structure for transformation between both models and discusses the philosophical implications of this duality.
	
	\subsection{\href{\repobase pdf/English/Simplified Description of the Four Fundamental Forces with Time-Mass Duality.pdf}{Simplified Description of the Four Fundamental Forces with Time-Mass Duality}}
	\textit{(225,943 bytes, 27.03.2025)}
	
	This publication offers a comprehensive mathematical formulation of all four fundamental forces (gravitation, electromagnetic, strong, and weak force) within the framework of time-mass duality. It develops an extended Lagrangian formalism that connects the standard treatment of fundamental forces with the concept of intrinsic time. Special attention is given to modified gravitational theory and the reformulation of the standard model of particle physics, with the aim of achieving a coherent unification of all forces.
	
	\subsection{\href{\repobase pdf/English/Time as an Emergent Property in Quantum Mechanics.pdf}{Time as an Emergent Property in Quantum Mechanics}}
	\textit{(238,903 bytes, 25.03.2025)}
	
	This work explores how time itself might be understood as an emergent property from more fundamental quantum processes. It establishes connections between relativistic theories, the fine structure constant, and quantum dynamics through the concept of intrinsic time. The paper provides a detailed mathematical analysis of how this perspective resolves several conceptual issues in the standard formulation of quantum mechanics.
	
	\subsection{\href{\repobase pdf/English/Unified Lagrangian Density with Dual Time-Mass Concept.pdf}{Unified Lagrangian Density with Dual Time-Mass Concept}}
	\textit{(210,546 bytes, 25.03.2025)}
	
	This publication presents a unified mathematical framework using Lagrangian density formalism to incorporate the time-mass duality concept. It provides a detailed formulation for integrating this approach with existing physical theories and demonstrates how this extension can naturally accommodate both quantum phenomena and relativistic effects in a single coherent framework.
	
	\section{Specific Applications and Implications}
	
	These publications examine specific applications of time-mass duality to concrete physical phenomena and their implications.
	
	\subsection{\href{\repobase pdf/English/Dynamic Mass of Photons and its Implications for Nonlocality.pdf}{Dynamic Mass of Photons and its Implications for Nonlocality}}
	\textit{(177,299 bytes, 25.03.2025)}
	
	This work extends the concept of time-mass duality to massless particles, particularly photons. By developing a dynamic mass concept for photons that correlates with their frequency, it offers a new perspective on phenomena such as quantum entanglement and nonlocality. It argues that the apparent instantaneous correlation of entangled photons can be explained through subtle, mass-dependent time structures without violating classical causality principles. The paper includes quantitative predictions for experimental tests of this hypothesis.
	
	\subsection{Mass Variation in Galaxies}
	\textit{(278,143 bytes, 27.03.2025)}
	
	This publication applies time-mass duality to galactic structures and develops a modified gravitational model based on systematic mass variations dependent on galactic radius. It demonstrates how this model can explain flat rotation curves of galaxies without resorting to the concept of dark matter. The proposed modified gravitational potential $\Phi(r) = -GM/r + \kappa r$ is analyzed in detail and compared with astronomical observational data.
	
	\subsection{\href{\repobase pdf/English/Unification of the T0 Model Foundations - Dark Energy and Galaxy Dynamics.pdf}{Unification of the T0 Model: Foundations - Dark Energy and Galaxy Dynamics}}
	\textit{(264,279 bytes, 27.03.2025)}
	
	This comprehensive work synthesizes the applications of the $T_0$ model to cosmological phenomena. It develops a theoretical framework that unifies cosmic expansion, the nature of dark energy, and galaxy dynamics in a coherent model. The central thesis is that systematic mass changes on cosmic scales can explain both the accelerated expansion of the universe and the observed anomalies in galaxy dynamics. The work quantifies the interaction between baryonic matter and the postulated dark energy field through the coupling parameter $\beta \approx 10^{-3}$.
	
	\subsection{\href{\repobase pdf/English/A Mathematical Analysis of Energy Dynamics.pdf}{A Mathematical Analysis of Energy Dynamics}}
	\textit{(265,907 bytes, 26.03.2025)}
	
	This paper provides a detailed mathematical formulation of energy dynamics within the time-mass duality framework. It explores how energy transformations and flows can be reinterpreted when mass is considered as a dynamic rather than static property. The analysis leads to novel insights into energy conservation principles and potential experimental signatures.
	
	\section{Fundamental Constants and Units}
	
	These publications examine the relationships between fundamental physical constants and develop new perspectives on natural unit systems.
	
	\subsection{\href{\repobase pdf/English/fundamental-constants-derivation.pdf}{Fundamental Constants and Their Derivation from Natural Units}}
	\textit{(236,985 bytes, 25.03.2025)}
	
	This work analyzes fundamental physical constants such as the fine structure constant, the gravitational constant, and Planck's constant from the perspective of time-mass duality. It investigates how these constants can be derived as emergent quantities from a more fundamental theoretical framework. Special attention is given to the relationship between intrinsic time and these constants, with the goal of reducing the number of truly fundamental constants.
	
	\subsection{\href{\repobase pdf/English/Energy as Fundamental Unit alpha = 1.pdf}{Energy as Fundamental Unit with alpha = 1}}
	\textit{(194,272 bytes, 26.03.2025)}
	
	This publication develops a revolutionary natural unit system in which the fine structure constant $\alpha = 1$ is set, in contrast to its empirical value of approximately 1/137. It argues that this unit system enables deeper theoretical insights and offers mathematical simplifications. The paper examines the consequences of this choice for the formulation of the fundamental laws of physics and the interpretation of empirical measurements, particularly in the context of time-mass duality.
	
	\section{Cosmological and Boundary Areas}
	
	These publications explore the implications of time-mass duality for our understanding of the most fundamental structures of the universe.
	
	\subsection{\href{\repobase pdf/English/Beyond the Planck Scale.pdf}{Beyond the Planck Scale}}
	\textit{(225,409 bytes, 25.03.2025)}
	
	This work examines the consequences of time-mass duality for phenomena beyond the Planck scale, where conventional theories reach their limits. It argues that the new theoretical framework can potentially avoid singularities and lead to a more coherent understanding of extreme physical conditions. The paper develops mathematical models for the transition between classical and quantum mechanical regimes and discusses implications for the early universe and black holes.
	
	\section{Summary Papers and Overviews}
	
	These publications provide condensed overviews of the main concepts and findings from the broader research program.
	
	\subsection{\href{\repobase pdf/English/A New Perspective on Time and Space Johann Pascher's Revolutionary Ideas.pdf}{A New Perspective on Time and Space: Johann Pascher's Revolutionary Ideas}}
	\textit{(58,675 bytes, 25.03.2025)}
	
	This concise overview paper introduces the fundamental concepts of the time-mass duality framework to a broader audience. It summarizes the key innovations and potential impacts on our understanding of physics in an accessible format.
	
	\subsection{\href{\repobase pdf/English/Summary - Complementary Dualism in Physics.pdf}{Summary - Complementary Dualism in Physics}}
	\textit{(145,939 bytes, 25.03.2025)}
	
	This summary document focuses specifically on the complementary nature of the standard model and T0-model approaches. It provides a condensed explanation of how these dual perspectives can describe the same physical reality from different conceptual starting points.
	
	\subsection{\href{\repobase pdf/English/summary-fundamental-constants.pdf}{Summary - Fundamental Constants}}
	\textit{(87,437 bytes, 25.03.2025)}
	
	This paper presents a compact overview of how fundamental physical constants are reinterpreted within the time-mass duality framework, highlighting the potential for reducing the number of truly fundamental constants in physics.
	
	\subsection{\href{\repobase pdf/English/Time and Mass A New Perspective on Old Formulas – and Liberation from Traditional Constraints.pdf}{Time and Mass: A New Perspective on Old Formulas – and Liberation from Traditional Constraints}}
	\textit{(87,137 bytes, 25.03.2025)}
	
	This overview document examines how traditional physical formulas can be reinterpreted through the lens of time-mass duality, potentially freeing theoretical physics from long-standing conceptual constraints and opening new avenues for research.
	
	\section{German Publications}
	
	In addition to the English versions, many of the papers are also available in German:
	
	\begin{itemize}
		\item \href{\repobase pdf/Deutsch/Die Notwendigkeit einer Erweiterung der Standard-Quantenmechanik und Quantenfeldtheorie.pdf}{Die Notwendigkeit einer Erweiterung der Standard-Quantenmechanik und Quantenfeldtheorie}
		\item \href{\repobase pdf/Deutsch/Komplementäre Erweiterungen der Physik.pdf}{Komplementäre Erweiterungen der Physik}
		\item \href{\repobase pdf/Deutsch/Vereinfachte Beschreibung der vier Grundkräfte mit Zeit-Masse-Dualität.pdf}{Vereinfachte Beschreibung der vier Grundkräfte mit Zeit-Masse-Dualität}
		\item \href{\repobase pdf/Deutsch/Dynamische Masse von Photonen und ihre Implikationen für Nichtlokalität.pdf}{Dynamische Masse von Photonen und ihre Implikationen für Nichtlokalität}
		\item \href{\repobase pdf/Deutsch/Massenvariation in Galaxien.pdf}{Massenvariation in Galaxien}
		\item \href{\repobase pdf/Deutsch/Vereinheitlichung des T0-Modells Grundlagen - Dunkle Energie und Galaxiendynamik.pdf}{Vereinheitlichung des T0-Modells: Grundlagen - Dunkle Energie und Galaxiendynamik}
		\item \href{\repobase pdf/Deutsch/Fundamentale Konstanten und deren Herleitung aus natürlichen Einheiten.pdf}{Fundamentale Konstanten und deren Herleitung aus natürlichen Einheiten}
		\item \href{\repobase pdf/Deutsch/Natürliche Einheiten mit Feinstrukturkonstante alpha = 1.pdf}{Natürliche Einheiten mit Feinstrukturkonstante alpha = 1}
		\item \href{\repobase pdf/Deutsch/Jenseits der Planck-Skala.pdf}{Jenseits der Planck-Skala}
	\end{itemize}
	
	\subsection{German Summaries}
	
	\begin{itemize}
		\item \href{\repobase pdf/Deutsch kurzgefasst/Eine neue Perspektive auf Zeit und Raum Johann Paschers revolutionäre Ideen.pdf}{Eine neue Perspektive auf Zeit und Raum: Johann Paschers revolutionäre Ideen}
		\item \href{\repobase pdf/Deutsch kurzgefasst/Kurzgefasst - Komplementärer Dualismus in der Physik - Von Welle-Teilchen zum Zeit-Masse-Konzept.pdf}{Kurzgefasst - Komplementärer Dualismus in der Physik - Von Welle-Teilchen zum Zeit-Masse-Konzept}
		\item \href{\repobase pdf/Deutsch kurzgefasst/Zusammenfassung - Fundamentale Konstanten.pdf}{Zusammenfassung - Fundamentale Konstanten}
		\item \href{\repobase pdf/Deutsch kurzgefasst/Zeit und Masse Ein neuer Blick auf alte Formeln – und die Befreiung von traditionellen Fesseln.pdf}{Zeit und Masse: Ein neuer Blick auf alte Formeln – und die Befreiung von traditionellen Fesseln}
	\end{itemize}
	
	\section{Summary and Outlook}
	
	The presented publications together form a coherent research program that develops a fundamentally new approach to unifying and extending modern physics. Unlike other unifying approaches such as string theory or loop quantum gravity, which introduce complex additional structures, time-mass duality focuses on reformulating the most fundamental concepts—time and mass.
	
	The central idea of intrinsic time $T = \hbar/mc^2$ and the complementary models (standard model and $T_0$ model) offers not only new theoretical perspectives but also concrete experimentally verifiable predictions. These include mass-dependent time evolution in quantum systems, subtle effects in entanglement experiments, and alternative understanding of cosmological phenomena such as dark matter and dark energy.
	
	Future research directions could include the following areas:
	\begin{itemize}
		\item Development of detailed experimental protocols to verify mass-dependent time evolution
		\item Refinement of the mathematical formulation, particularly with regard to quantum gravitation
		\item Extended numerical simulations to verify modified galaxy dynamics
		\item Inclusion of additional fundamental particles and interactions in the theoretical framework
		\item Application of the theory to fundamental problems in quantum information and quantum computing
	\end{itemize}
	
	The overarching goal remains the development of a comprehensive, mathematically elegant, and experimentally confirmed "All-in-One" theory that unites quantum mechanics, quantum field theory, and general relativity in a coherent framework while simultaneously addressing existing problems in modern physics such as nonlocality, dark matter, and dark energy.
	
\end{document}