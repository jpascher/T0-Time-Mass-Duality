\documentclass[12pt,a4paper]{article}  % Explicit specification of font size
\usepackage[utf8]{inputenc}
\usepackage[T1]{fontenc}
\usepackage[english]{babel}
\usepackage{lmodern}  % Latin Modern fonts for better scaling
\usepackage{csquotes}
\usepackage{amsmath}
\usepackage{amssymb}
\usepackage{geometry}
\usepackage{hyperref}
\usepackage{tocloft}
\geometry{a4paper, margin=2cm}
\title{The Necessity of Extending Standard Quantum Mechanics and Quantum Field Theory}
\author{Johann Pascher}
\date{March 27, 2025}

\begin{document}
	
	\maketitle
	
	\tableofcontents
	\newpage
	
	\section{Introduction: Conceptual Limits of Established Theories}
	
	Quantum mechanics and quantum field theory have proven extraordinarily successful in describing the microscopic world. Nevertheless, both theories reach fundamental conceptual limits, particularly when it comes to complete integration with General Relativity (GR) and clarifying the fundamental nature of time and mass. Recent theoretical developments suggest that certain core assumptions of standard QM/QFT require critical review and extension.
	
	Previous attempts to develop a unified physical theory encompassing all fundamental forces and phenomena have not been fully successful despite decades of intensive efforts by outstanding scientists. Both string theory and loop quantum gravity, causal set theory and other approaches have provided significant insights, but no comprehensive solution to the unification problem. The time-mass duality with intrinsic time presented here offers a novel approach that has the potential to function as a true "all-in-one" theory, as it addresses fundamental concepts at their roots rather than merely extending existing theories.
	
	\subsection{Inherent Dualism Between QM and QFT}
	
	A fundamental problem of the current theoretical framework is the inherent dualism between quantum mechanics and quantum field theory themselves. These two theories represent different, partially complementary perspectives on the quantum world:
	
	\begin{itemize}
		\item \textbf{Quantum mechanics} primarily views reality from the perspective of particles. It describes the dynamics of quantum objects through wavefunctions and focuses on discrete states, transitions, and measurement quantities of single or few particles.
		
		\item \textbf{Quantum field theory}, in contrast, takes a field-based view. It treats particles as excitations of continuous quantum fields and is better suited to describe relativistic effects as well as particle creation and annihilation processes.
	\end{itemize}
	
	This dualism reflects in some way the wave-particle duality that has existed since the beginnings of quantum theory. Although both approaches are successful in their respective application areas, a unified conceptual framework that fully integrates the particle-based perspective of QM and the field-based view of QFT has been lacking so far. Each approach captures only part of the overall picture and both have specific limitations: QM neglects relativistic effects and field aspects, while QFT has difficulties describing localized quantum phenomena and coherent quantum states.
	
	The time-mass duality presented here could provide a way to overcome this fundamental dualism by establishing a deeper connection between particle-oriented properties (such as mass) and wave-like, extended aspects (represented by temporal evolution).
	
	\subsection{Overinterpretation Due to Incomplete Theoretical Foundations}
	
	A critical and often overlooked aspect in the development of quantum theory is the tendency to overinterpret certain phenomena that may merely be artifacts of the theory's own incompleteness. This excessive interpretation has led to conceptual "paradoxes" and philosophical discussions that perhaps say more about the limits of our theoretical frameworks than about the nature of reality.
	
	\begin{itemize}
		\item The \textbf{measurement problem} and the associated \textbf{wavefunction collapse} could be artifacts of the incomplete treatment of time in quantum mechanics. The sudden, discontinuous change of the wavefunction upon measurement contradicts the otherwise continuous, unitary time evolution and appears as a fundamental paradox, but might merely reflect the inadequacy of the temporal description.
		
		\item The apparent \textbf{non-locality} in quantum entanglement and the associated "spooky action at a distance" (Einstein) could be based on an overly simplistic conception of time and causality. With a mass-dependent intrinsic time scale, what appears as instantaneous information transfer might have a subtler temporal structure.
		
		\item The \textbf{incompatibility} of various interpretations of quantum mechanics (Copenhagen interpretation, many-worlds interpretation, Bohmian mechanics etc.) could indicate that all these approaches attempt to interpret a fundamentally incomplete theory rather than extending the theoretical framework itself.
	\end{itemize}
	
	The approach of time-mass duality presented here offers the possibility that some of these supposed paradoxes and philosophical problems are not fundamental properties of the quantum world, but rather indicators of the incompleteness of existing theoretical formulations. Many of the puzzling aspects of quantum theory might find natural explanations if time is no longer considered as a universal parameter but as a mass-dependent, emergent property.
	
	\section{Asymmetric Treatment of Time and Space}
	
	\subsection{Time as Parameter versus Space as Operator}
	
	In non-relativistic quantum mechanics, as described by the Schrödinger equation:
	
	\begin{equation}
		i\hbar \frac{\partial}{\partial t}\Psi(x,t) = \hat{H}\Psi(x,t)
	\end{equation}
	
	time ($t$) functions exclusively as an external, classical parameter that controls the unitary evolution of the quantum state. In contrast, spatial coordinates are represented by operators. This asymmetric treatment stands in direct contradiction to the four-dimensional spacetime continuum of relativity theory, where time and space are equal.
	
	\subsection{The Problem of the Time Operator}
	
	Unlike for position and momentum, no time operator exists in standard QM. Attempts to introduce such an operator encounter mathematical and conceptual difficulties, particularly regarding canonical quantization and the energy-time uncertainty relation.
	
	\subsection{Relativistic Extensions and Their Limits}
	
	Although quantum field theory integrates special relativity, the fundamental difficulty remains to unite this with GR, where spacetime itself is dynamic and influenced by mass distributions. The rigid, parametric role of time in the standard formalism makes this unification considerably more difficult.
	
	\section{Static Treatment of Mass}
	
	\subsection{Mass as Unchanging Parameter}
	
	In standard QM and QFT, mass ($m$) is typically treated as an intrinsic, unchanging property of a particle. It appears as a fixed parameter in fundamental equations such as:
	
	\begin{equation}
		\hat{H} = \frac{\hat{p}^2}{2m} + V(\hat{x})
	\end{equation}
	
	This treatment of mass as a static property might represent an oversimplification, particularly regarding a deeper connection between time, mass and energy.
	
	\subsection{Mass Renormalization and Its Limits}
	
	Although concepts like mass renormalization exist in QFT, the rest mass of a specific particle type remains constant in the Standard Model. QFT offers no natural explanation for the observed particle masses or their possible dynamics.
	
	\subsection{Higgs Mechanism and Mass Generation}
	
	The Higgs mechanism explains mass generation but continues to treat mass as a static property once generated. The possibility of a dynamic mass-time relationship is not considered.
	
	\section{The Concept of Intrinsic Time}
	
	\subsection{Derivation from Fundamental Relations}
	
	Starting from Einstein's mass-energy equivalence and the quantum mechanical energy-frequency relation:
	
	\begin{equation}
		E = mc^2 \quad \text{and} \quad E = h\nu = \frac{h}{T}
	\end{equation}
	
	an intrinsic time $T$ can be derived by equating:
	
	\begin{equation}
		T = \frac{h}{mc^2} = \frac{\hbar}{mc^2}
	\end{equation}
	
	This characteristic time can be interpreted as a fundamental time scale associated with a mass $m$.
	
	\subsection{Connection to the Fine-Structure Constant}
	
	The intrinsic time can be connected to the fine-structure constant $\alpha$:
	
	\begin{equation}
		T = \frac{\hbar}{mc^2} \propto \alpha
	\end{equation}
	
	This relationship suggests a deeper connection between time, mass and fundamental interactions.
	
	\subsection{Natural Units and the $T = 1/m$ Relationship}
	
	In a system of natural units where $c = \hbar = 1$, the relationship simplifies to:
	
	\begin{equation}
		T = \frac{1}{m}
	\end{equation}
	
	This elegant relation shows that the intrinsic time of an object in such a theoretical framework is simply the reciprocal of its mass.
	
	\section{Time-Mass Duality: A New Theoretical Framework}
	
	\subsection{Complementary Models: The $T_0$-Model and Intrinsic Time}
	
	The time-mass duality postulates two complementary perspectives:
	\begin{itemize}
		\item The standard model: Constant mass and variable time (time dilation)
		\item The complementary $T_0$-model: Absolute time and variable mass
	\end{itemize}
	
	This duality enables a new interpretation of relativistic phenomena and quantum mechanical processes.
	
	\subsection{A Path to an "All-in-One" Theory}
	
	The time-mass duality represents a fundamentally different approach than previous unification theories:
	
	\begin{itemize}
		\item While \textbf{string theory} introduces additional spatial dimensions and complicated mathematical structures but continues to rely on constant masses and covariant time description, the time-mass duality reformulates these fundamental concepts themselves.
		
		\item \textbf{Loop quantum gravity} quantizes spacetime and leads to a granular structure but does not question the fundamental nature of time as a one-dimensional parameter or mass as an intrinsic, static property.
		
		\item \textbf{Supersymmetric theories} extend the particle spectrum with partner particles but retain the standard treatment of time and mass.
	\end{itemize}
	
	In contrast, the time-mass duality does not introduce complicated additional structures but questions and reformulates the most basic concepts of physics. It offers the potential of a true "all-in-one" theory because it:
	
	\begin{enumerate}
		\item Naturally connects the foundations of QM and QFT with GR
		\item Provides an explanatory framework for cosmological phenomena like dark matter and dark energy
		\item Delivers experimentally verifiable predictions that operate at fundamental levels
		\item Links the concept of intrinsic scales with observed physical constants
		\item Offers an elegantly simple and conceptually deep foundation
	\end{enumerate}
	
	This approach could overcome the fragmentation of modern physics that has persisted despite decades of intensive research in various unification approaches.
	
	\subsection{Overcoming the QM-QFT Dualism}
	
	The time-mass duality offers a promising approach to overcoming the existing dualism between QM and QFT:
	
	\begin{enumerate}
		\item \textbf{Integration of particle and field perspectives}: By the relationship between mass (traditionally associated with particles) and intrinsic time (connected with wave propagation and field evolution), a natural bridge is created between the particle-oriented approach of QM and the field-based view of QFT.
		
		\item \textbf{Unifying time scale}: The intrinsic time $T = \hbar/mc^2$ directly connects particle mass with the characteristic time scale of its quantum mechanical evolution, thereby establishing a fundamental connection between the discrete-local and continuous-extended aspects of quantum reality.
		
		\item \textbf{Complementary descriptions}: Similar to Bohr's complementarity between wave and particle pictures, the time-mass duality offers complementary perspectives ($T_0$-model and standard model) that together provide a more complete picture of quantum reality.
	\end{enumerate}
	
	This unifying view could overcome the traditional separation between non-relativistic quantum mechanics and relativistic quantum field theory and provide a more coherent framework for understanding quantum mechanical phenomena.
	
	\subsection{Reformulation of the Schrödinger Equation}
	
	With the concept of intrinsic time, the Schrödinger equation can be modified:
	
	\begin{equation}
		i\hbar \frac{\partial}{\partial (t/T)}\Psi = \hat{H}\Psi
	\end{equation}
	
	This modification would mean that time evolution is no longer uniform for all objects but depends on their mass.
	
	\subsection{Implications for Many-Particle Systems and Entanglement}
	
	For a system with particles of different masses, the wavefunction could have different intrinsic time scales, with a modified Schrödinger equation:
	
	\begin{equation}
		i (m_1 + m_2) c^2 \frac{\partial}{\partial t} \Psi(x_1, x_2, t) = \hat{H} \Psi(x_1, x_2, t)
	\end{equation}
	
	This would have profound implications for entangled states and coherence phenomena.
	
	\section{Consequences for Fundamental Phenomena}
	
	\subsection{Quantum Coherence and Decoherence}
	
	The mass-dependent intrinsic time would lead to a modified decoherence rate:
	
	\begin{equation}
		\Gamma_{\text{dec}} = \Gamma_0 \cdot \frac{mc^2}{\hbar}
	\end{equation}
	
	This implies that heavier systems decohere more slowly in their intrinsic time scale but faster in external laboratory time.
	
	\subsection{Modified Dispersion Relation}
	
	With intrinsic time, a modified dispersion relation would result:
	
	\begin{equation}
		\omega_{\text{eff}} = \frac{\hbar^2 k^2}{2 m^2 c^2}
	\end{equation}
	
	This form differs from standard QM where $\omega \propto 1/m$. The new form $\omega_{\text{eff}} \propto 1/m^2$ could produce experimental differences in matter wave propagation.
	
	\subsection{Limits of Instantaneity}
	
	The intrinsic time $T = \hbar/mc^2$ establishes a fundamental minimal time scale. From the energy-time uncertainty relation follows:
	
	\begin{equation}
		\Delta t \gtrsim \frac{\hbar}{mc^2} = T
	\end{equation}
	
	This implies that no information can be transferred in exactly zero time - there is a fundamental lower limit for any quantum interaction.
	
	\subsection{EPR Paradox and Bell's Inequalities}
	
	A mass-dependent time theory could offer new interpretational possibilities for the EPR paradox and Bell's inequalities. If time is an emergent, mass-dependent property, the question arises whether "instantaneous" is a well-defined concept at the fundamental quantum level.
	
	\subsection{Resolution of Apparent Paradoxes by More Complete Theory}
	
	Many of the supposed paradoxes and mysteries of quantum physics might turn out to be artifacts of an incomplete theoretical description, similar to how the "paradoxes" of special relativity resulted from the incompleteness of Newtonian mechanics. The mass-dependent time theory could naturally resolve these paradoxes:
	
	\begin{itemize}
		\item \textbf{Bell's inequalities} and their experimental violation are often interpreted as evidence for non-locality or non-causality. With intrinsic time, however, these correlations could be explained by subtler temporal structures without abandoning classical causality.
		
		\item \textbf{Quantum teleportation} appears as instantaneous information transfer, leading to philosophical problems with causality and relativity theory. The intrinsic time scale $T = \hbar/mc^2$, however, establishes a natural minimal duration for any "instantaneous" interaction that conforms with the energy-time uncertainty relation.
		
		\item The \textbf{quantum mechanical tunneling effect}, where particles overcome classically impassable barriers, could be reinterpreted through mass-dependent time evolution. The time a particle needs to penetrate a barrier would then depend on its mass and the associated intrinsic time scale.
	\end{itemize}
	
	These reinterpretations demonstrate that many of the seemingly strange aspects of quantum mechanics may not be fundamental properties of nature but rather indicators of the conceptual deficiencies of our current theories. A more complete theory that captures time and mass in their mutual relationship could enable a more natural and intuitive description of quantum reality without resorting to philosophically problematic concepts like wavefunction collapse, many-worlds or nonlocal effects.
	
	\section{Variable Mass as Hidden Variable: A Path to Determinism?}
	
	\subsection{The Problem of Indeterminism in Quantum Mechanics}
	
	One of the most fundamental and philosophically challenging aspects of quantum mechanics is its inherent indeterminism. Standard quantum mechanics in principle provides only probabilistic statements about possible measurement outcomes and thus seems to make a profound break with classical, deterministic physics. Einstein expressed his resistance to this probabilistic nature in his famous statement "God does not play dice" and suspected that quantum mechanics might be an incomplete theory, with deeper, deterministic processes underlying it.
	
	\subsection{Hidden Variables and Their Previous Limits}
	
	The search for "hidden variables" - directly unobservable parameters that could determine the apparently random outcome of quantum measurements - has a long history in the development of quantum physics:
	
	\begin{itemize}
		\item \textbf{Bohmian mechanics} introduces a "pilot wave" that deterministically guides particle motion. Although mathematically equivalent to standard QM, it leads to a non-local theory.
		
		\item \textbf{Local hidden variables} were largely ruled out by the experimental violation of Bell's inequalities, leading many physicists to accept indeterminism as a fundamental property of nature.
		
		\item The \textbf{Kochen-Specker theorem} and related no-go theorems place further restrictions on hidden variables, particularly for context-independent variables.
	\end{itemize}
	
	\subsection{Variable Mass as Fundamental Hidden Variable}
	
	The time-mass duality approach presented here opens a completely new perspective on the problem of indeterminism in quantum mechanics. The variable mass in the complementary $T_0$-model could function as a kind of fundamental hidden variable, but in a way fundamentally different from previous hidden variable theories:
	
	\begin{enumerate}
		\item Unlike classical hidden variables introduced as additional parameters, variable mass is a fundamental quantity already anchored in physics.
		
		\item While local hidden variables were ruled out by Bell's experiments, variable mass might escape these restrictions by fundamentally changing how we understand time and causality.
		
		\item Since variable mass is directly connected to intrinsic time, it leads to a deeper connection between quantum mechanics and relativity theory rather than further separating them.
	\end{enumerate}
	
	In this model, the apparent randomness of quantum mechanical processes would not be fundamental but an artifact of our incomplete description and measurement of the system. The true dynamic evolution of the system would be deterministic but controlled by mass variation not considered in standard QM.
	
	\subsection{Modified Quantum Dynamics and Deterministic Evolution}
	
	A modified formalism incorporating mass variation could enable deterministic quantum dynamics:
	
	\begin{equation}
		i\hbar \frac{\partial}{\partial t}\Psi(x,t) = \hat{H}(m(t))\Psi(x,t)
	\end{equation}
	
	where $m(t)$ is the time-dependent mass function. This modified Schrödinger equation would describe a deterministic evolution whose apparently probabilistic behavior results from our inability to measure or control the exact mass variation.
	
	The probability interpretation of the wavefunction $|\Psi|^2$ in this picture would not reflect the fundamental nature of reality but would be a statistical description resulting from our ignorance of the exact mass variation - similar to how statistical mechanics results from our ignorance of exact microscopic states in a thermodynamic system.
	
	\subsection{Compatibility with Existing No-Go Theorems}
	
	A crucial point is that this form of determinism need not contradict existing no-go theorems like Bell's theorem or the Kochen-Specker theorem:
	
	\begin{itemize}
		\item Bell's theorem rules out local hidden variables, but variable mass as hidden variable could modify the locality assumption itself by introducing a deeper connection between space, time and mass.
		
		\item The Kochen-Specker theorem rules out context-independent hidden variables, but mass variation could be inherently context-dependent as it relates to the measuring interaction.
		
		\item Mass variation would not necessarily lead to a theory that is obviously non-local (like Bohmian mechanics) but could introduce a subtler form of causality compatible with relativistic principles.
	\end{itemize}
	
	\subsection{Philosophical and Conceptual Implications}
	
	The idea that variable mass could function as a fundamental hidden variable has profound philosophical implications:
	
	\begin{enumerate}
		\item It would confirm Einstein's intuition that the indeterminism of quantum mechanics is not fundamental but based on an incomplete description.
		
		\item It could cast the long debate between Bohr and Einstein about the nature of quantum reality in a new light by integrating elements of both positions.
		
		\item It would offer a new perspective on the measurement problem, as the "collapse" of the wavefunction could be understood as a deterministic process controlled by mass variation.
		
		\item It could lead to a deeper understanding of entanglement by providing a causal mechanism for the observed non-local correlations.
	\end{enumerate}
	
	The time-mass duality thus opens not only a new theoretical perspective but also the possibility to reevaluate the most fundamental philosophical questions of quantum physics. It could pave the way for a theory that preserves both the mathematical success of quantum mechanics and enables a more intuitive, causally closed description of reality.
	
	\subsection{Experimental Accessibility and Verifiability}
	
	A central question is whether mass variation as hidden variable is experimentally accessible or remains fundamentally unobservable. Unlike traditional hidden variable theories, the time-mass duality approach offers concrete predictions:
	
	\begin{itemize}
		\item Precision measurements might detect subtle deviations from standard quantum mechanics in systems with strongly differing masses.
		
		\item The mass-dependent time evolution could lead to measurable differences in coherence time or entanglement dynamics.
		
		\item High-precision Bell tests with particles of different masses could provide evidence for the underlying deterministic structure.
	\end{itemize}
	
	Should such deviations be experimentally confirmed, this would not only strengthen the theoretical foundation of time-mass duality but also initiate a revolutionary paradigm shift in our understanding of quantum mechanical reality - from a fundamentally indeterministic to a deterministic but subtler worldview.
	
	\section{Unified Lagrangian Density with Time-Mass Duality}
	
	A complete unified theory can be described by the following extended Lagrangian density:
	
	\begin{equation}
		\mathcal{L}_\text{total} = \mathcal{L}_\text{Gravitation} + \mathcal{L}_\text{SM} + \mathcal{L}_\text{Higgs} + \mathcal{L}_\text{intrinsic}
	\end{equation}
	
	where the additional term accounts for intrinsic time:
	
	\begin{equation}
		\mathcal{L}_\text{intrinsic} = \bar{\psi}\left(i\hbar\gamma^0 \frac{\partial}{\partial (t/T)} - i\hbar\gamma^0 \frac{\partial}{\partial t}\right)\psi
	\end{equation}
	
	This extended Lagrangian density enables a natural integration of time-mass duality into the existing theory framework.
	
	\subsection{Connection Between Deterministic and Probabilistic Aspects}
	
	The unified Lagrangian density provides a formal framework to connect the deterministic foundation with the probabilistic appearance of the quantum world:
	
	\begin{itemize}
		\item The term $\mathcal{L}_\text{intrinsic}$ captures the discrepancy between time evolution in absolute time and mass-dependent intrinsic time.
		
		\item This difference manifests in our measurements as apparent indeterminism, although the underlying dynamics could be completely deterministic.
		
		\item The deterministic evolution is controlled by the mass-dependent time evolution, while the probabilistic interpretation results from our inability to measure this exact evolution.
	\end{itemize}
	
	In this way, time-mass duality could offer an elegant solution to the tension between determinism and indeterminism in quantum physics without compromising the empirical successes of existing theories.
	
	\section{Experimental Verifiability}
	
	The extended theory leads to several experimentally testable predictions:
	
	\begin{enumerate}
		\item Mass-dependent time evolution in quantum systems, measurable as different coherence times
		\item Differences in entanglement speed for particles of different masses
		\item Modified energy-momentum relation for very massive particles
		\item Measurable deviations in high-precision experiments
	\end{enumerate}
	
	\subsection{Excursus: Empirical Tests for Determinism}
	
	In addition to the tests already mentioned, specific experiments could be designed to verify the hypothesis of hidden determinism:
	
	\begin{itemize}
		\item \textbf{Mass-dependent statistical fluctuations}: If mass variation functions as hidden variable, statistical analyses of quantum measurements on systems with different masses might detect subtle patterns incompatible with standard randomness.
		
		\item \textbf{Correlation experiments with mass generation}: Tests where particle mass is changed during measurement (e.g., by energy input) could provide evidence for the connection between mass variation and measurement outcomes.
		
		\item \textbf{Precision Bell tests with mass effects}: Extended Bell tests specifically designed to detect mass-dependent effects might reveal deviations from standard predictions, particularly if the analysis considers the potential mass-dependent time evolution.
	\end{itemize}
	
	Although such experiments would be extremely challenging and require highest precision, they could provide crucial insights into whether quantum mechanics can be understood at a more fundamental, deterministic level.
	
	\section{Cosmological Implications}
	
	The concept of intrinsic time and time-mass duality offers new perspectives on cosmological phenomena:
	
	\begin{enumerate}
		\item An alternative explanation for cosmological redshift through the absorption coefficient $\alpha = H_0/c$
		\item A modified gravitational potential $\Phi(r) = -GM/r + \kappa r$ that could explain flat rotation curves without dark matter
		\item A natural connection between dark energy and the intrinsic time scale of the universe
	\end{enumerate}
	
	\subsection{Deterministic Cosmology}
	
	The idea of an underlying determinism through variable mass also has far-reaching implications for cosmology:
	
	\begin{itemize}
		\item The apparent randomness of quantum mechanical fluctuations in the early universe serving as starting point for cosmic structure formation could be explained by deterministic but complex mass variation patterns.
		
		\item The expansion of the universe might be connected with systematic mass changes of all particles, offering an alternative explanation to dark energy.
		
		\item The arrow of time and entropy increase could be understood as emergent phenomena resulting from the relationship between absolute and intrinsic time.
	\end{itemize}
	
	These cosmological aspects demonstrate that time-mass duality might not only reinterpret microscopic quantum phenomena but could also provide a coherent picture of cosmic evolution on all scales - from quantum fluctuation to cosmological expansion.
	
	\section{Conclusions}
	
	Standard quantum mechanics and quantum field theory have proven extremely successful but reach fundamental conceptual limits. The asymmetric treatment of time and space as well as the static consideration of mass present significant obstacles to complete unification with general relativity. Moreover, the inherent dualism between particle-oriented QM and field-based QFT persists - both theories capture only partial aspects of quantum physical reality.
	
	The presented concept of intrinsic time $T = \hbar/mc^2$ and time-mass duality offers a promising framework for extending these theories. This extension could not only lead to a deeper understanding of the fundamental nature of time and mass but also overcome the existing dualism between QM and QFT by establishing a natural connection between the particle-oriented and field-based views. 
	
	Unlike other unifying approaches like string theory introducing additional dimensions and complex mathematical structures, or loop quantum gravity postulating a fundamental granularity of spacetime, the approach presented here goes back directly to the conceptual foundations. It does not primarily modify the mathematical structures of existing theories but questions and extends the fundamental concepts of time and mass themselves. This fundamental reorientation could represent the key to a genuine "all-in-one" theory not achieved so far despite intensive efforts.
	
	Particularly noteworthy is the potential of time-mass duality, through the introduction of variable mass as a kind of fundamental hidden variable, to open a path back to deterministic physics. This would not only confirm Einstein's intuition that "God does not play dice" but also provide a deeper, conceptually more satisfying explanation for the apparent randomness of the quantum world. Unlike earlier hidden variable theories, this approach would not contradict experimental findings but reinterpret them in a new light.
	
	By directly linking mass (a particle-associated property) with the intrinsic time scale (a wave-like, processual aspect), it creates a conceptual framework that not only connects quantum mechanics and relativity theory but also enables a deeper understanding of cosmological phenomena like dark matter and dark energy - a feature largely missing in previous unification approaches.
	
	This theoretical extension also provides experimentally verifiable predictions and opens new paths to solving existing problems in theoretical physics. Overcoming the conceptual limits of standard QM and QFT through the integration of intrinsic time represents an important step toward a more comprehensive theory that unifies quantum mechanics, quantum field theory and general relativity in a coherent framework and overcomes the previous fragmentation of our theoretical understanding.
	
	Ultimately, time-mass duality might not only provide a unified description of the physical world but also grant deeper insight into the fundamental nature of reality itself - a reality that might be more deterministic, causal and intuitive than current theories suggest.
	
	\begin{thebibliography}{99}
		\bibitem{pascher1} Pascher, J. (2025). Time as Emergent Property in Quantum Mechanics: A Connection Between Relativity Theory, Fine-Structure Constant and Quantum Dynamics.
		
		\bibitem{pascher2} Pascher, J. (2025). Simplified Description of the Four Fundamental Forces with Time-Mass Duality.
		
		\bibitem{pascher3} Pascher, J. (2025). Simplified Description of the Four Fundamental Forces with Time-Mass Duality [German version].
		
		\bibitem{pascher4} Pascher, J. (2025). Complementary Extensions of Physics: Absolute Time and Intrinsic Time.
		
		\bibitem{pascher5} Pascher, J. (2025). A Model with Absolute Time and Variable Energy: A Comprehensive Investigation of the Foundations.
		
		\bibitem{pascher6} Pascher, J. (2025). Extensions of Quantum Mechanics through Intrinsic Time.
		
		\bibitem{pascher7} Pascher, J. (2025). Integration of Time-Mass Duality into Quantum Field Theory.
		
		\bibitem{einstein} Einstein, A. (1905). Does the Inertia of a Body Depend Upon Its Energy Content? \textit{Annalen der Physik}, 323(13), 639-641.
		
		\bibitem{planck} Planck, M. (1901). On the Law of Distribution of Energy in the Normal Spectrum. \textit{Annalen der Physik}, 309(3), 553-563.
		
		\bibitem{schrodinger} Schrödinger, E. (1926). An Undulatory Theory of the Mechanics of Atoms and Molecules. \textit{Physical Review}, 28(6), 1049-1070.
		
		\bibitem{bell} Bell, J. S. (1964). On the Einstein Podolsky Rosen Paradox. \textit{Physics}, 1(3), 195-200.
		
		\bibitem{aspect} Aspect, A., Dalibard, J., \& Roger, G. (1982). Experimental Test of Bell's Inequalities Using Time-Varying Analyzers. \textit{Physical Review Letters}, 49(25), 1804-1807.
		
		\bibitem{bohm} Bohm, D. (1952). A Suggested Interpretation of the Quantum Theory in Terms of "Hidden" Variables. \textit{Physical Review}, 85(2), 166-179.
		
		\bibitem{kochen} Kochen, S., Specker, E. P. (1967). The Problem of Hidden Variables in Quantum Mechanics. \textit{Journal of Mathematics and Mechanics}, 17(1), 59-87.
		
		\bibitem{einstein2} Einstein, A., Podolsky, B., Rosen, N. (1935). Can Quantum-Mechanical Description of Physical Reality Be Considered Complete? \textit{Physical Review}, 47(10), 777-780.
	\end{thebibliography}
	
\end{document}