\documentclass[11pt,a4paper]{article}
\usepackage[utf8]{inputenc}
\usepackage[T1]{fontenc}
% \usepackage[german]{babel}  % not installed
\usepackage{amsmath,amssymb}
\usepackage{geometry}
\geometry{left=25mm, right=25mm, top=25mm, bottom=25mm}
\usepackage{booktabs}
\usepackage{array}
\usepackage{tabularx}
\usepackage{graphicx}
\usepackage{xcolor}
\usepackage{float}
\usepackage{enumitem}
\usepackage{fancyhdr}
\usepackage{titlesec}
\usepackage{hyperref}
\usepackage{siunitx}
\usepackage{tcolorbox}

% Farben
\definecolor{darkblue}{HTML}{16213e}
\definecolor{accentred}{HTML}{e94560}
\definecolor{keygreen}{HTML}{2e7d32}
\definecolor{warnred}{HTML}{c62828}
\definecolor{warnorange}{HTML}{e65100}
\definecolor{lightgray}{HTML}{f5f5f5}

% Boxen
\newtcolorbox{keybox}{colback=keygreen!8, colframe=keygreen!60!black, 
  left=3mm, right=3mm, top=2mm, bottom=2mm, fonttitle=\bfseries}
\newtcolorbox{warnbox}{colback=warnred!8, colframe=warnred!60!black,
  left=3mm, right=3mm, top=2mm, bottom=2mm, fonttitle=\bfseries}

% Header
\pagestyle{fancy}
\fancyhf{}
\fancyhead[L]{\small Strömungsanalyse Bass-Stimmzunge 50\,Hz -- v6}
\fancyhead[R]{\small\thepage}
\fancyfoot[C]{\small Stationäre \& instationäre Analyse}

% Titel-Formatierung
\titleformat{\section}{\Large\bfseries\color{darkblue}}{Kapitel \thesection:}{0.5em}{}
\titleformat{\subsection}{\large\bfseries\color{darkblue!80}}{}{0em}{}

\begin{document}

% ============================================================
% TITEL
% ============================================================
\begin{center}
{\LARGE\bfseries\color{darkblue} Strömungsanalyse einer 50-Hz-Bass-Stimmzunge}\\[6pt]
{\large Keilförmige Stimmplatte, 1{,}5\,mm Aufbiegung, Folie vs.\ Furnier, Ausblas vs.\ Umlenk}\\[4pt]
\textcolor{accentred}{\rule{0.8\textwidth}{2pt}}
\end{center}
\vspace{4mm}

% ============================================================
\section{Aufbau}
% ============================================================

\begin{keybox}
Im Bassteil eines Akkordeons sitzen Dutzende Stimmzungen in eigenen Kammern, alle am selben Balg. Der Balg erzeugt den Druck, die Tasten steuern über mechanische Hebel die Klappen. Die Klappe wird \textbf{nicht} vom Balgdruck geöffnet -- der Druck steht an allen geschlossenen Klappen an. Erst der Tastendruck öffnet die Klappe mechanisch.
\end{keybox}

% ============================================================
\section{Kammer, Stimmplatte und Aufbiegung}
% ============================================================

Die Kammer ist ein Holzquader von $94 \times 48 \times 40$\,mm. Die gesamte Oberseite wird von einer Aluminium-Stimmplatte verschlossen. Diese Platte ist \textbf{keilförmig}: am freien Zungenende (über der Klappe) 13\,mm dick, am Einspannende nur 2\,mm. Der Schlitz in der Platte ist daher am freien Ende 13\,mm tief -- die Luft durchquert einen kurzen Kanal, bevor sie die Zunge erreicht.

Die Stahlzunge ($70 \times 8 \times 0{,}354$\,mm) ist mit \textbf{zwei Schrauben} auf der Platte befestigt (kleinere Stimmplatten werden genietet). Das freie Ende ist um \textbf{1{,}5\,mm aufgebogen} -- es steht also im Ruhezustand 1{,}5\,mm über der Plattenoberfläche. Dadurch entsteht ein dreieckiger Spalt: 0\,mm an der Einspannung, 1{,}5\,mm am freien Ende.

Die effektive Spaltfläche (Integral über die Zungenlänge) beträgt:
\begin{equation}
S_\text{Spalt} = \frac{W \cdot h_\text{max}}{2} = \frac{8 \times 1{,}5}{2} = \mathbf{6\,\text{mm}^2}
\end{equation}
Zum Vergleich: Die Klappenöffnung hat ca.\ 244\,mm$^2$ -- rund 41-mal größer. Der Spalt bleibt der Engpass.

\begin{figure}[H]
\centering
\includegraphics[width=0.85\textwidth]{abb1_laengsschnitt.png}
\caption{Längsschnitt. Keilförmige Platte (2$\rightarrow$13\,mm), Aufbiegung 1{,}5\,mm am freien Ende.}
\end{figure}

% ============================================================
\section{Der gefaltete Luftweg}
% ============================================================

Eine Trennwand teilt die Kammer in zwei Kanäle. Der Luftweg:

\medskip
\centerline{Klappe (Boden rechts) $\rightarrow$ Kanal B nach links $\rightarrow$ 180\textdegree{}-Faltung $\rightarrow$ Kanal A nach rechts $\rightarrow$ Schlitz $\rightarrow$ Spalt}
\medskip

Gesamtweg über 200\,mm.

% ============================================================
\section{Trennwand -- Folie oder Furnier?}
% ============================================================

Für die Trennwand kommen zwei Materialien in Frage: \textbf{Kunststofffolie} ($\sim$0{,}1\,mm) oder \textbf{Holzfurnier} ($\sim$0{,}5\,mm). Die Wanddicke beeinflusst die Kanalbreite: Bei gerader Wand ergibt Folie 23{,}95\,mm pro Kanal, Furnier 23{,}75\,mm -- eine Differenz von 0{,}20\,mm (0{,}8\,\%). Strömungstechnisch ist das vernachlässigbar.

\begin{table}[H]
\centering\small
\begin{tabular}{l p{50mm} p{50mm}}
\toprule
& \textbf{Folie (0,1\,mm)} & \textbf{Furnier (0,5\,mm)} \\
\midrule
Oberfläche & Glatt $\rightarrow$ weniger Reibung ($f \approx 0{,}02$) & Rau $\rightarrow$ mehr Reibung ($f \approx 0{,}03$) \\
Stabilität & Flexibel $\rightarrow$ kann flattern, braucht Stützpunkte & Formstabil $\rightarrow$ bleibt in Position \\
Akustik & Kann als Membran schwingen $\rightarrow$ Nebengeräusche & Akustisch inert \\
Verarbeitung & Kleben, spannen & Kleben, klemmen, einfacher \\
\bottomrule
\end{tabular}
\caption{Folie vs.\ Furnier als Trennwandmaterial}
\end{table}

Da die Kanalreibung nur einen verschwindenden Anteil des Gesamtverlusts ausmacht, ist der Reibungsunterschied zwischen Folie und Furnier für den Durchfluss bedeutungslos. Der Unterschied liegt ausschließlich in Stabilität und Akustik. Furnier ist die sicherere Wahl; Folie nur sinnvoll, wenn der Platz extrem knapp ist.

% ============================================================
\section{Klappenorientierung -- Ausblas oder Umlenk?}
% ============================================================

\textbf{Ausblas-Richtung:} Die Klappe öffnet so, dass der 30\textdegree{}-Strahl direkt in Kanal B zeigt. Der Strahl hat eine horizontale Komponente von 87\,\% und eine vertikale von 50\,\%. Verlustbeiwert: $\zeta_\text{Klappe} = 0{,}498$ (Vena contracta + Richtungsänderung).

\textbf{Umlenk-Richtung:} Die Klappe öffnet nach außen. Die Luft muss $\sim$150\textdegree{} um die Klappenkante kehrtmachen, bevor sie in die Kammer eintritt. Verlustbeiwert: $\zeta_\text{Klappe} = 1{,}698$ -- das \textbf{3{,}4-fache} der Ausblas-Richtung.

\begin{table}[H]
\centering\small
\begin{tabular}{l c c}
\toprule
& \textbf{Ausblas} & \textbf{Umlenk} \\
\midrule
Strahl & Gerichtet in Kammer & Erst weg, dann 150${}^\circ$-Kurve \\
$\zeta_\text{Klappe}$ & 0{,}498 & 1{,}698 \\
Wandform-Nutzung & Voll (Düse, Coanda) & Reduziert (breiter Eintritt) \\
Nebengeräusche & Weniger & Mehr (Ablösung an Kante) \\
Anwendung & Standardwahl & Nur wenn Bauform es erfordert \\
\bottomrule
\end{tabular}
\caption{Vergleich Ausblas vs.\ Umlenk-Klappenorientierung}
\end{table}

Der höhere Verlust der Umlenk-Richtung wirkt sich auf die Spaltgeschwindigkeit kaum aus (Spalt dominiert), aber auf die Strömungsqualität: Der Strahl tritt breiter und langsamer in die Kammer ein, die Turbulenz am Eintritt ist höher, und die Wandform (Düse bei B, Coanda bei C) kann nicht voll wirken.

% ---- 5.1 Klappenöffnung und Lautstärke ----
\subsection{Klappenöffnung und Lautstärke}

Praktische Tests zeigen: Bei Verringerung der Klappenöffnung treten Lautstärke-Einbußen auf. Die stationäre Berechnung zeigt, warum das \textbf{nicht} am Durchfluss liegt:

\begin{table}[H]
\centering\small
\begin{tabular}{r r r r r r}
\toprule
\textbf{Winkel} & $S_\text{Klap,eff}$ [mm$^2$] & $S_\text{Klap}/S_\text{Spalt}$ & $\zeta_\text{ges}$ & $Q$ [ml/s] & $Q/Q_{30^\circ}$ \\
\midrule
30${}^\circ$ & 244 & 40{,}7$\times$ & 2{,}0004 & 912 & 100{,}0\,\% \\
25${}^\circ$ & 204 & 33{,}9$\times$ & 2{,}0005 & 912 & 100{,}0\,\% \\
20${}^\circ$ & 167 & 27{,}8$\times$ & 2{,}0008 & 912 & 100{,}0\,\% \\
15${}^\circ$ & 126 & 21{,}0$\times$ & 2{,}0013 & 911 & 99{,}9\,\% \\
10${}^\circ$ & 85 & 14{,}1$\times$ & 2{,}0032 & 911 & 99{,}9\,\% \\
7${}^\circ$ & 59 & 9{,}9$\times$ & 2{,}0062 & 910 & 99{,}8\,\% \\
5${}^\circ$ & 43 & 7{,}1$\times$ & 2{,}0113 & 910 & 99{,}7\,\% \\
3${}^\circ$ & 25 & 4{,}3$\times$ & 2{,}0297 & 907 & 99{,}5\,\% \\
2${}^\circ$ & 17 & 2{,}8$\times$ & 2{,}0635 & 903 & 99{,}0\,\% \\
1${}^\circ$ & 9 & 1{,}4$\times$ & 2{,}2362 & 885 & 97{,}0\,\% \\
\bottomrule
\end{tabular}
\caption{Volumenstrom bei verschiedenen Klappenöffnungswinkeln (bei 1000\,Pa, Wand A)}
\end{table}

\begin{warnbox}
Die stationäre Strömung ist erstaunlich unempfindlich gegen die Klappenöffnung -- selbst bei 5${}^\circ$ sind noch über 99\,\% des maximalen Durchflusses verfügbar. Dass die praktisch beobachteten Lautstärke-Einbußen \textbf{nicht durch stationären Durchflussverlust} erklärbar sind (der Spalt dominiert zu stark), deutet auf einen \textbf{instationären Mechanismus} hin.
\end{warnbox}

\textbf{1.\ Langsamerer Druckaufbau:} Eine kleinere Klappenöffnung verzögert den Druckanstieg in der Kammer. Die Zunge braucht länger, um die volle Amplitude zu erreichen. Bei schnellen Passagen wird die Maximalamplitude nie erreicht -- das klingt leiser.

\textbf{2.\ Veränderte Jet-Kopplung:} Bei kleinerem Winkel ist der Strahl flacher und schneller. Bei 30${}^\circ$ hat der Strahl 50\,\% vertikale Komponente, bei 10${}^\circ$ nur noch 17\,\%. Weniger Vertikalkomponente bedeutet, dass der Strahl weniger direkt in den Kanal B gerichtet wird.

\textbf{3.\ Akustische Impedanzänderung:} Die Klappenöffnung ist der akustische Eingang der Kammer. Eine kleinere Öffnung erhöht die akustische Impedanz am Kammereingang, was die Resonanzeigenschaften der Kammer verändert. Bei bestimmten Öffnungswinkeln können destruktive Interferenzen entstehen, die die Zungenamplitude reduzieren.

\begin{keybox}
Alle drei Effekte wirken zusammen und erklären, warum die Lautstärke-Einbußen in der Praxis deutlicher ausfallen als es die stationäre Berechnung erwarten lässt. Die Berechnung zeigt, dass der Effekt nicht im Durchfluss liegt -- also muss er im instationären Verhalten liegen.
\end{keybox}

% ---- 5.2 Viertelkreis an der 180${}^\circ$-Faltung ----
\subsection{Viertelkreis an der 180\textdegree-Faltung}

Die 180\textdegree{}-Faltung am Ende der Trennwand hat scharfe Ecken. Kann man durch Einsetzen eines Viertelkreis-Profils (Radius $R$) an den beiden 90\textdegree{}-Ecken den Verlust reduzieren? Der hydraulische Durchmesser der Faltung ist $D_h = 25{,}8$\,mm (Spalt 19\,mm $\times$ Höhe 40\,mm).

\begin{table}[H]
\centering\small
\begin{tabular}{l r r r r}
\toprule
\textbf{Radius $R$} & $R/D_h$ & $\zeta_\text{Faltung}$ & $\zeta$ auf $S_\text{Spalt}$ bezogen & Anteil an $\zeta_\text{ges}$ \\
\midrule
scharf & 0 & 2{,}50 & 0{,}000016 & 0{,}0008\,\% \\
3\,mm & 0{,}12 & 1{,}56 & 0{,}000010 & 0{,}0005\,\% \\
5\,mm & 0{,}19 & 1{,}22 & 0{,}000008 & 0{,}0004\,\% \\
8\,mm & 0{,}31 & 0{,}94 & 0{,}000006 & 0{,}0003\,\% \\
10\,mm & 0{,}39 & 0{,}82 & 0{,}000005 & 0{,}0003\,\% \\
15\,mm & 0{,}58 & 0{,}64 & 0{,}000004 & 0{,}0002\,\% \\
\bottomrule
\end{tabular}
\caption{Faltungsverlust bei verschiedenen Verrundungsradien}
\end{table}

\textbf{Stationäres Ergebnis:} Die Faltung trägt bei \textbf{allen Radien weniger als 0{,}01\,\% zum Gesamtwiderstand} bei. Der Grund: Die Faltungsquerschnittsfläche $S_\text{Faltung} = 760\,\text{mm}^2$ ist 127-mal größer als der Spalt ($6\,\text{mm}^2$). Der Verlustbeiwert wird mit $(S_\text{Spalt}/S_\text{Faltung})^2 = 6{,}2 \times 10^{-5}$ skaliert -- praktisch Null.

\textbf{Instationäres Bild:} Für den Druckstoss-Durchgang ist die Frage differenzierter:

\textbf{a) Reflexionen reduzieren:} An einer scharfen 90${}^\circ$-Ecke wird ein Druckpuls teilweise reflektiert (akustischer Impedanzsprung). Ein Viertelkreis glättet den Impedanzübergang und lässt mehr Pulsenergie passieren.

\textbf{b) Wirbelbildung reduzieren:} Bei scharfen Ecken löst die Strömung bei jedem Druckstoß sofort ab und bildet Wirbel, die Energie dissipieren. Mit Radius $R > 5$\,mm bleibt die Strömung länger angelegt.

\begin{warnbox}
\textbf{Aber Vorsicht:} Eine \textbf{zu kohärente} Strömung am Spalt ist nicht wünschenswert! Die Bernoulli-Selbsterregung braucht eine Anfangsstörung, um die Schwingung auszulösen. Eine perfekt gleichmäßige, laminare Strömung würde die Zunge nur statisch auslenken, ohne sie in Schwingung zu versetzen. Ein gewisses Maß an Turbulenz und räumlicher Ungleichmäßigkeit ist notwendig.
\end{warnbox}

\textbf{Lokale Druckschwankungen} an der Zungenspitze stoßen die erste Auslenkung an. \textbf{Wirbelablösungen} an den Spaltkanten liefern breitbandige Störungen, die die Zunge zum Schwingen anregen. Wäre die Strömung perfekt laminar und gleichförmig, müsste die Zunge auf eine infinitesimale Instabilität warten -- der Einschwingvorgang würde sich deutlich verzögern.

Das Optimum liegt also nicht bei maximaler Kohärenz, sondern bei \textbf{schnellem Druckaufbau mit ausreichend Störungen}. Der Druckstoß soll schnell am Spalt ankommen (wenig Reflexion in der Faltung, gute Wandführung), aber dort soll genügend Turbulenz entstehen, um die Selbsterregung sofort auszulösen. Die Turbulenz entsteht ohnehin am Spalt selbst ($\text{Re} \sim 1400$, Ablösungen an den Spaltkanten) -- sie muss nicht zusätzlich aus der Kammer angeliefert werden.

\begin{keybox}
\textbf{Empfehlung:} Ein Viertelkreis-Profil $R = 8$--$10$\,mm an den Ecken der Faltung reduziert Reflexionen und Energieverluste im Druckstoß-Transport, ohne die Turbulenz am Spalt zu beeinträchtigen (der Spalt erzeugt seine eigene Turbulenz). Baulich einfach (eingeleimter Rundstab, Wachshohlkehle oder Epoxy-Verrundung). Ob der Effekt hörbar ist, lässt sich nur durch Versuch klären.
\end{keybox}

% ============================================================
\section[Düsenwinkel-Experiment]{Düsenwinkel-Experiment -- warum der Anströmwinkel die Amplitude bestimmt}
% ============================================================

Ein aufschlussreicher Praxistest: Stimmplatte montiert, Klappe entfernt, Anblasen mit einer Druckluftdüse aus verschiedenen Winkeln. Ergebnis: Die Zungenamplitude ändert sich \textbf{von Null bis Maximum} allein durch Änderung des Anströmwinkels.

\subsection{Die zwei Komponenten des Jets}

Ein Luftstrahl, der unter dem Winkel $\alpha$ auf den Spalt trifft, hat zwei Komponenten:

\begin{table}[H]
\centering\small
\begin{tabular}{r r r l}
\toprule
\textbf{Winkel $\alpha$} & $\sin\alpha$ (normal) & $\cos\alpha$ (tangential) & \textbf{Charakter} \\
\midrule
0${}^\circ$ & 0{,}00 & 1{,}00 & Rein tangential -- kein Eintritt \\
10${}^\circ$ & 0{,}17 & 0{,}98 & Flach -- viel Überströmung \\
20${}^\circ$ & 0{,}34 & 0{,}94 & Optimal-Bereich? \\
30${}^\circ$ & 0{,}50 & 0{,}87 & Optimal-Bereich? \\
45${}^\circ$ & 0{,}71 & 0{,}71 & Optimal-Bereich? \\
60${}^\circ$ & 0{,}87 & 0{,}50 & Steil -- wenig Überströmung \\
75${}^\circ$ & 0{,}97 & 0{,}26 & Steil -- wenig Überströmung \\
90${}^\circ$ & 1{,}00 & 0{,}00 & Rein normal -- statischer Druck \\
\bottomrule
\end{tabular}
\caption{Zerlegung des Anströmprofils in Normal- und Tangentialkomponente}
\end{table}

\textbf{Normalkomponente} (senkrecht zur Platte, $v \cdot \sin\alpha$): Treibt Luft durch den Spalt in den Schlitz hinein. Erzeugt statischen Überdruck unter der Zunge. Dieser Druck biegt die Zunge nach oben aus -- aber er schwingt nicht von selbst. Ohne weitere Kopplung hält er die Zunge nur statisch in einer Gleichgewichtslage.

\textbf{Tangentialkomponente} (parallel zur Platte, $v \cdot \cos\alpha$): Strömt über die Spalt\-\textbf{öffnung} hinweg. Das ist der entscheidende Mechanismus.

\subsection{Der Flöten-Mechanismus am Zungenspalt}

Wenn Luft über eine Öffnung streicht, entsteht an der Öffnung ein Unterdruck (Bernoulli). Genau das passiert bei der Tangentialkomponente des Jets: Sie strömt über den Spalt und erzeugt einen Sog, der Luft aus dem Schlitz herauszieht -- also die Zunge nach \textbf{unten} zieht. Gleichzeitig drückt die Normalkomponente die Zunge nach \textbf{oben}.

Entscheidend ist: Beide Kräfte ändern sich, wenn die Zunge schwingt. Bewegt sich die Zunge nach unten (Spalt wird größer), kann mehr Luft tangential über den Spalt strömen $\rightarrow$ stärkerer Sog $\rightarrow$ Zunge wird weiter nach unten gezogen. Bewegt sie sich nach oben (Spalt wird enger), verengt sich der Kanal für die Tangentialströmung $\rightarrow$ weniger Sog $\rightarrow$ Zunge federt zurück. Das ist eine \textbf{positive Rückkopplung}, die die Schwingung antreibt.

Dieser Mechanismus ist physikalisch identisch mit der Tonerzeugung bei Flöteninstrumenten: Dort strömt ein flacher Luftstrahl über eine scharfe Kante (Labium), und die Wechselwirkung zwischen Jet und Resonator erzeugt den Ton. Die Zungenkante übernimmt hier die Rolle des Labiums.

\subsection{Warum der Winkel die Amplitude bestimmt}

\textbf{$\alpha = 0^\circ$ (rein tangential):} Maximale Überströmung, aber kein Druck im Schlitz. Die Zunge hat keine Vorauslenkung -- der Bernoulli-Sog muss gegen die volle Federsteifigkeit arbeiten. Außerdem fehlt der Durchfluss durch den Spalt, der für die klassische Bernoulli-Selbsterregung nötig ist. Ergebnis: Keine oder minimale Schwingung.

\textbf{$\alpha = 90^\circ$ (rein normal):} Maximaler statischer Druck im Schlitz, aber keine Tangentialströmung. Die Zunge wird nach oben gedrückt und bleibt dort. Es fehlt der Bernoulli-Unterdruck an der Spaltoberseite, der die Zunge zurückziehen würde. Ergebnis: Statische Auslenkung, keine Schwingung.

\textbf{Optimaler Winkel ($\sim 20^\circ\text{--}40^\circ$):} Beide Komponenten sind in der richtigen Balance. Die Normalkomponente liefert den Grunddruck, der die Zunge in den Arbeitsbereich vorspannt. Die Tangentialkomponente liefert den geschwindigkeitsabhängigen Bernoulli-Sog, der die Schwingung antreibt. Das Verhältnis entscheidet über die Phasenlage zwischen Druck und Zungenposition -- und damit über die Energiezufuhr pro Zyklus.

Die Amplitude ändert sich stufenlos mit dem Winkel, weil die Energiezufuhr pro Schwingungszyklus vom Verhältnis der beiden Komponenten abhängt. Am optimalen Winkel ist die Netto-Energiezufuhr maximal.

\subsection{Konsequenz für die Kammerauslegung}

Dieses Experiment erklärt unmittelbar, warum die Klappenorientierung und die Trennwandform so wichtig sind: Sie bestimmen den \textbf{effektiven Anströmwinkel} am Spalt.

Im normalen Betrieb (mit Klappe und Kammer) wird der Anströmwinkel nicht direkt vom Balgdruck bestimmt, sondern von der Geometrie des Luftwegs. Die Luft kommt aus Kanal A (parallel zur Platte) und muss um 90\textdegree{} in den Schlitz einbiegen. Der effektive Anströmwinkel hängt davon ab, wie die Strömung in Kanal A verteilt ist:

\textbf{Schnelle, laminare Kanalströmung} (gute Wandführung, wenig Wirbel): Die Luft biegt gleichmäßig in den Spalt ein. Es entsteht eine definierte Tangentialkomponente, die den Bernoulli-Sog zuverlässig erzeugt. \textbf{Langsame, verwirbelte Kanalströmung} (schlechte Wandführung, viele Reflexionen): Die Richtungsinformation ist verloren. Die Luft nähert sich dem Spalt aus allen Richtungen -- die Tangentialkomponente mittelt sich teilweise heraus. Der Bernoulli-Antrieb wird schwächer.

\begin{keybox}
Das Düsenexperiment zeigt \textbf{direkt}, warum die Trennwand das Einschwingverhalten beeinflusst: Nicht über den Durchfluss (der ist immer gleich), sondern über den Anströmwinkel am Spalt. Eine saubere Strömungsführung in der Kammer erhält die Richtungsinformation des Druckstoßes -- und damit den optimalen Anströmwinkel für die Bernoulli-Rückkopplung. Das erklärt auch, warum Variante C (Parabel) in der Praxis die beste Ansprache liefert: Sie erhält die Strömungsrichtung am besten.
\end{keybox}

% ============================================================
\section{Drei Trennwandvarianten}
% ============================================================

\textbf{A -- Gerade:} Beide Kanäle 23{,}8\,mm. 30\textdegree{}-Strahl prallt $\sim$90\textdegree{} auf Wand. Stagnationspunkt, Ablösung, höchste Turbulenz.

\textbf{B -- Schräg ($\beta = 3{,}7^\circ$):} Kanal B bei Klappe 14{,}2\,mm (Düse), bei Faltung 23{,}8\,mm. Kanal A bei Spalt 33{,}4\,mm (weit). Diffusor-Halbwinkel nur 3{,}7\textdegree{} $\rightarrow$ sicher ablösungsfrei.

\textbf{C -- Parabel:} Gleiche Endpunkte wie B, aber stetige Krümmung. Coanda-Effekt führt Strahl entlang konvexer Wand. Sauberste Strömung.

\begin{figure}[H]
\centering
\includegraphics[width=0.85\textwidth]{abb2_varianten.png}
\caption{Trennwandvarianten. Bei B und C: eng bei Klappe (rechts), weit bei Faltung (links).}
\end{figure}

Alle drei Varianten liefern bei 1000\,Pa denselben Volumenstrom ($\sim$912\,ml/s) und dieselbe Spaltgeschwindigkeit ($\sim$28{,}8\,m/s). Die Unterschiede in $\zeta_\text{ges}$ liegen im Promillebereich.

\begin{warnbox}
\textbf{Achtung: Diese Ergebnisse zeigen nur den stationären Zustand.} Die nahezu identischen Volumenströme täuschen darüber hinweg, dass die Trennwand das Einschwingverhalten in der Praxis entscheidend beeinflusst.
\end{warnbox}

\subsection{Warum die Trennwand trotzdem entscheidend ist}

Der Widerspruch löst sich auf, wenn man den \textbf{instationären Anlauf} betrachtet statt den Gleichgewichtszustand. In den ersten Millisekunden nach Klappenöffnung passiert folgendes: Ein Druckstoß läuft vom Ventil durch Kanal B, um die Faltung, durch Kanal A zum Spalt. Die Trennwand ist der Wellenleiter für diesen Druckstoß. Ihre Form bestimmt, wie schnell der Druck am Spalt ankommt und wie viel Energie auf dem Weg durch Reflexionen und Wirbelbildung verlorengeht.

\textbf{Gerade Wand (A):} Der 30-Grad-Strahl prallt auf die Wand, erzeugt einen Stagnationspunkt und Wirbel. Der Druckstoß wird breit gestreut und teilweise reflektiert. Am Spalt kommt er verspätet und mit reduzierter Amplitude an. Der Druckaufbau ist langsam -- die Zunge erreicht den Schwellendruck für die Bernoulli-Rückkopplung später.

\textbf{Schräge Wand (B):} Die Düse am Eintritt beschleunigt den Druckstoß. Weniger Reflexionen auf dem Weg. Am Spalt kommt der Druck schneller auf den Schwellenwert. Der Anlauf ist kürzer.

\textbf{Parabolische Wand (C):} Die stetige Krümmung führt den Druckstoß ohne Ablösung entlang der Wand (Coanda-Effekt). Minimale Reflexionen, schnellster Druckaufbau am Spalt. Die Turbulenz, die die Selbsterregung auslöst, entsteht ohnehin am Spalt selbst (Ablösungen an den Kanten bei $\text{Re} \sim 1400$) und muss nicht aus der Kammer angeliefert werden.

In der Elektrotechnik-Analogie (Kapitel~10): Die stationäre Berechnung entspricht dem Gleichstromwiderstand eines Kabels -- für alle drei Varianten fast gleich. Aber die \textbf{Impulsantwort} (wie schnell ein Spannungssprung am Ende ankommt) hängt von der Wellenimpedanz, Reflexionen und Dispersion ab -- und die sind völlig verschieden. Für die Ansprache zählt nicht der Gleichstromwiderstand, sondern die Impulsantwort.

% ============================================================
\section[Dynamische Spaltfläche]{Die dynamische Spaltfläche -- warum 6\,mm$^2$ nur die halbe Wahrheit ist}
% ============================================================

Alle bisherigen Berechnungen verwenden die Ruhelage-Spaltfläche $S_\text{Spalt} = 6\,\text{mm}^2$. Aber die Zunge schwingt -- und bei voller Lautstärke schwingt die Zungenspitze ca.\ 20\,mm nach unten durch den Schlitz. Dabei gibt die Zunge den \textbf{gesamten Schlitzquerschnitt} frei.

\subsection{Geometrie des Durchschwingvorgangs}

Der Schlitz hat die Breite $W_\text{Schlitz} = 9$\,mm und verläuft durch die gesamte keilförmige Platte (2--13\,mm dick). Der maximale Schlitzquerschnitt (wenn die Zunge komplett entfernt wäre) beträgt $W_\text{Schlitz} \times t_\text{mittel} = 9 \times 7{,}5 = \mathbf{67{,}5\,\text{mm}^2}$ -- das 11-fache der Ruhelage-Spaltfläche.

Drei Phasen beim Durchschwingen:

\begin{table}[H]
\centering\small
\begin{tabular}{r l r r r}
\toprule
\textbf{Tip-Pos.} & \textbf{Zustand} & $S_\text{eff}$ [mm$^2$] & $S/S_\text{Ruhe}$ & \textbf{Freigeg.\ Länge} \\
\midrule
$+1{,}5$\,mm & Über Platte & 4{,}5 & 0{,}8$\times$ & 0\,mm \\
$+0{,}5$\,mm & Über Platte & 1{,}5 & 0{,}3$\times$ & 0\,mm \\
$0$\,mm & Spalt schließt & 7{,}5 & 1{,}3$\times$ & 0\,mm \\
$-2$\,mm & Im Schlitz & 7{,}5 & 1{,}3$\times$ & 0\,mm \\
$-5$\,mm & Im Schlitz & 7{,}5 & 1{,}3$\times$ & 0\,mm \\
$-10$\,mm & Im Schlitz & 7{,}5 & 1{,}3$\times$ & 0\,mm \\
$-15$\,mm & Großteils frei & 24{,}7 & 4{,}1$\times$ & 13\,mm \\
$-20$\,mm & Großteils frei & 42{,}4 & 7{,}1$\times$ & 28\,mm \\
$-25$\,mm & Großteils frei & 49{,}7 & 8{,}3$\times$ & 36\,mm \\
$-30$\,mm & Großteils frei & 53{,}7 & 8{,}9$\times$ & 41\,mm \\
\bottomrule
\end{tabular}
\caption{Effektiver Durchströmquerschnitt bei verschiedenen Tip-Positionen}
\end{table}

\textbf{Phase 1 -- Zunge über Platte ($+1{,}5$ bis $0$\,mm):} Normaler Spaltbetrieb. $S$ sinkt von 4{,}5 auf 0\,mm$^2$. In dieser Phase gilt die bisherige Berechnung.

\textbf{Phase 2 -- Zunge im Schlitz ($0$ bis $\sim$$-13$\,mm):} Die Zunge taucht in den Schlitz ein. Der Hauptquerschnitt ist blockiert, nur die schmalen Seitenspalte (je 0{,}5\,mm) bleiben offen. $S$ ist auf ca.\ 7{,}5\,mm$^2$ begrenzt (Seitenspalt-Querschnitt). Diese Phase ist der \textbf{Engpass im Schwingzyklus}.

\textbf{Phase 3 -- Zunge unter der Platte (unter $-13$\,mm):} Die Zungenspitze hat den Schlitz verlassen. Von der Spitze her wird der volle Schlitzquerschnitt freigegeben ($W_\text{Schlitz} \times t_\text{Platte}$). Bei $-20$\,mm Tip: $S_\text{eff} = 42$\,mm$^2$ (\textbf{7-fach}), 28\,mm der Zungenlänge sind frei. Bei $-30$\,mm: $S_\text{eff} = 54$\,mm$^2$ (\textbf{9-fach}).

\subsection{Zeitlicher Verlauf im Schwingzyklus}

\begin{table}[H]
\centering\small
\begin{tabular}{r r r r r r}
\toprule
\textbf{Ampl.} & $\text{Tip}_\text{min}$ & $S_\text{min}$ & $S_\text{max}$ & $S_\text{mittel}$ & Mittel$/S_\text{Ruhe}$ \\
\midrule
1\,mm & $-0{,}5$\,mm & 0{,}0 & 7{,}5 & 4{,}3 & 0{,}7$\times$ \\
2\,mm & $-2{,}5$\,mm & 0{,}1 & 7{,}5 & 5{,}6 & 0{,}9$\times$ \\
5\,mm & $-8{,}5$\,mm & 0{,}1 & 7{,}5 & 6{,}4 & 1{,}1$\times$ \\
10\,mm & $-18{,}5$\,mm & 0{,}1 & 38{,}8 & 14{,}9 & 2{,}5$\times$ \\
15\,mm & $-28{,}5$\,mm & 0{,}2 & 52{,}7 & 25{,}4 & 4{,}2$\times$ \\
20\,mm & $-38{,}5$\,mm & 0{,}3 & 57{,}5 & 31{,}5 & 5{,}2$\times$ \\
25\,mm & $-48{,}5$\,mm & 0{,}0 & 59{,}9 & 35{,}4 & 5{,}9$\times$ \\
\bottomrule
\end{tabular}
\caption{Zeitgemittelte Spaltfläche über einen Schwingzyklus (alle $S$ in mm$^2$)}
\end{table}

\begin{warnbox}
Bei voller Lautstärke ($A \sim 20$\,mm) oszilliert der Durchströmquerschnitt zwischen fast Null und 58\,mm$^2$ -- ein \textbf{Faktor von über 100} innerhalb einer einzigen Schwingungsperiode von 20\,ms. Der zeitgemittelte Querschnitt liegt bei 32\,mm$^2$ -- mehr als \textbf{5-mal} so groß wie in der Ruhelage.
\end{warnbox}

\subsection{Konsequenzen für die Strömungsanalyse}

\textbf{1.\ Der Spalt dominiert nur beim Einschwingen:} Zu Beginn der Schwingung (kleine Amplitude) ist der Spalt mit 6\,mm$^2$ tatsächlich der Engpass -- die Kammer mit $\sim$950\,mm$^2$ Kanalquerschnitt ist irrelevant. Aber sobald die volle Amplitude erreicht ist, ist der Spalt in jeder Halbperiode für einen Bruchteil der Zeit weit offen. In dieser Phase werden die Kammer, die Klappe (244\,mm$^2$) und die Trennwand zu den relativ engsten Stellen.

\textbf{2.\ Der Volumenstrom pulsiert massiv:} Während eines Zyklus wechselt der Durchfluss zwischen praktisch Null (Zunge schließt Spalt) und einem Maximum, das weit über dem stationären Wert liegt (weil $S$ so viel größer ist). Der instationäre Volumenstrom ist kein sanftes Sinussignal -- er hat extrem steile Flanken. Diese Pulse müssen durch die Kammer.

\textbf{3.\ Die Kammer wird zum Puffer:} Die Kammer mit ihrem Volumen von ca.\ 180\,cm$^3$ muss die extremen Volumenstrom-Schwankungen abpuffern. Wenn der Spalt kurzzeitig $10\times$ so viel Luft durchlässt wie im stationären Fall, muss diese Luft irgendwo herkommen -- aus dem Kammervolumen. Der Druck in der Kammer fällt bei jedem Durchschwing-Puls kurzzeitig ab und wird durch die Klappe nachgespeist.

\begin{keybox}
\textbf{4.\ Die Klappenfläche wird bei voller Lautstärke relevant:} Im stationären Bild ist die Klappe mit $S_\text{Klap}/S_\text{Spalt} = 41\times$ völlig überdimensioniert. Aber wenn der Spalt kurzzeitig 58\,mm$^2$ hat, ist das Verhältnis nur noch $S_\text{Klap}/S_\text{Spalt,dyn} = 4{,}2\times$. Damit wird die Klappenöffnung relevant -- und erklärt direkt, warum eine kleinere Klappenöffnung bei voller Lautstärke die Amplitude begrenzt (Kapitel~5).
\end{keybox}

\textbf{5.\ Die Trennwand sieht pulsierenden Gegendruck:} Bei jeder Durchschwing-Phase entsteht ein Unterdruck-Puls in der Kammer (Luft wird durch den offenen Spalt abgesaugt). Dieser Puls läuft durch die Kammer zurück zur Klappe. Wie die Kammer diesen Puls transportiert (Reflexionen, Dämpfung, Dispersion) hängt von der Trennwandform ab -- genau wie der Druckstoß beim Einschwingen. Die Trennwandform beeinflusst also nicht nur das Einschwingen, sondern auch den \textbf{eingeschwungenen Betrieb bei voller Lautstärke}.

\begin{keybox}
Die stationäre Berechnung mit $S_\text{Spalt} = 6$\,mm$^2$ beschreibt zuverlässig das Einschwingen (kleine Amplitude) und gibt die richtigen Relationen zwischen den Varianten. Bei voller Lautstärke ändert sich das Bild fundamental: Der Spalt dominiert nicht mehr allein, die Kammer und die Klappe werden zu Mitspielern, und der Volumenstrom wird extrem pulsierend. In dieser Phase wird die Kammergeometrie auch für den \textbf{stationären} (zeitgemittelten) Widerstand relevant -- nicht nur für die Impulsantwort.
\end{keybox}

% ============================================================
\section{Bernoulli-Selbsterregung und Einschwingzeit}
% ============================================================

Die Selbsterregung erfolgt über den Bernoulli-Mechanismus: Verengt sich der Spalt, steigt $v$, sinkt $p$, Unterdruck verstärkt Auslenkung. Am Umkehrpunkt bricht die Strömung zusammen, Feder treibt Zunge zurück.

\begin{figure}[H]
\centering
\includegraphics[width=0.85\textwidth]{abb3_bernoulli.png}
\caption{Bernoulli-Kreislauf (Querschnitt quer zur Zunge).}
\end{figure}

Schwellendruck: $\Delta P_\text{min} \approx 2{,}6$\,Pa (mit 1{,}5\,mm Aufbiegung). Niedrig gegenüber 500--5000\,Pa Balgdruck.

Die Einschwingzeit $\tau = Q/(\pi f)$ ist der eigentliche Engpass:

\begin{table}[H]
\centering\small
\begin{tabular}{r r r r}
\toprule
$\zeta$ & $Q$ (Güte) & $\tau$ [ms] & Perioden \\
\midrule
0{,}005 & 100 & 637 & 100 $\times$ 20\,ms \\
0{,}01 & 50 & 318 & 50 $\times$ 20\,ms \\
0{,}02 & 25 & 159 & 25 $\times$ 20\,ms \\
0{,}04 & 12 & 80 & 12 $\times$ 20\,ms \\
\bottomrule
\end{tabular}
\caption{Einschwingzeit bei 50\,Hz für verschiedene Güten}
\end{table}

$Q$ hängt von Material, Geometrie und Einspannung ab und muss gemessen werden.

% ============================================================
\section{Dämpfung -- was sie beeinflusst und wie}
% ============================================================

Die Güte $Q$ bestimmt die Einschwingzeit und damit die Ansprache. $Q$ ist keine Materialkonstante, sondern die Summe aller Energieverluste pro Schwingungszyklus. Diese lassen sich in vier physikalisch unterscheidbare Mechanismen zerlegen:

\textbf{1.\ Materialdämpfung (innere Reibung):} Jedes Metall wandelt bei Verformung einen Bruchteil der elastischen Energie in Wärme um. Der Verlustfaktor $\eta$ ist eine Materialeigenschaft. Für Federstahl liegt $\eta$ bei 0{,}0002--0{,}001, für Messing bei 0{,}001--0{,}003. In der Güte-Zerlegung: $Q_\text{Material} = 1/\eta$. Federstahl kommt allein auf $Q_\text{Material} \sim 1000$--$5000$, Messing auf $300$--$1000$. Frequenzunabhängig.

\textbf{2.\ Luftdämpfung (viskose Verluste im Spalt):} Die schwingende Zunge verdrängt Luft durch den engen Spalt (Squeeze-Film-Effekt). Die dissipierte Leistung skaliert mit $P_\text{visc} \sim \mu W L^3 \omega^2 a^2 / h^3$. Mit 1{,}5\,mm Aufbiegung sinkt der Squeeze-Film-Beitrag mit $h^{-3}$, also um Faktor $(1{,}5/0{,}4)^3 \approx 53$ gegenüber einem 0{,}4-mm-Spalt. Untergeordneter Beitrag.

\textbf{3.\ Einspannungsdämpfung:} An der Verschraubung (zwei Schrauben bei Basszungen) wird bei jeder Schwingung mikroskopisch Energie in Reibung umgewandelt. Einer der größten Einzelposten. Saubere Auflagefläche und gleichmäßiger Schrauben-Anpressdruck minimieren ihn.

\textbf{4.\ Schallabstrahlung:} Die schwingende Zunge strahlt Schall ab. Bei 50\,Hz und 70$\times$8\,mm Fläche ist der Strahlungswiderstand gering (Wellenlänge 6{,}9\,m). Kleiner Beitrag bei tiefen Bässen, wächst mit $f^2$.

\begin{keybox}
Die Gesamt-Güte ergibt sich als harmonische Summe: $\frac{1}{Q_\text{ges}} = \frac{1}{Q_\text{Mat}} + \frac{1}{Q_\text{Luft}} + \frac{1}{Q_\text{Einsp}} + \frac{1}{Q_\text{Strahl}}$. Bei einer typischen Basszunge aus Federstahl dürfte $Q_\text{ges}$ zwischen 50 und 150 liegen.
\end{keybox}

$Q$ lässt sich \textbf{senken} (schnellere Ansprache) durch: Materialwahl (Messing statt Stahl), Massebeladung am Tip, geringere Aufbiegung, weichere Einspannung. $Q$ lässt sich \textbf{erhöhen} (längeres Nachklingen, lauterer Ton) durch: härteren Stahl, präzisere Einspannung, größere Aufbiegung. Messung (Ausschwingversuch) ist der einzig zuverlässige Weg.

% ============================================================
\section{Lehren aus der Propeller- und Windkraft-Aerodynamik}
% ============================================================

\textbf{Hinterkanten-Optimierung (Trailing Edge):} Windkraftrotoren verwenden gezackte Hinterkanten (Serrations), um tonale Geräusche zu reduzieren. An einer scharfen Kante lösen sich Wirbel kohärenter in einer einzigen Frequenz ab (Kármán-Wirbel). Gezackte Kanten brechen diese Kohärenz auf. Dasselbe Prinzip gilt für die Trennwand-Hinterkante am Faltspalt.

\textbf{Vorderkanten-Optimierung (Leading Edge):} Buckelwal-inspirierte Vorderkanten (Tubercles) verhindern gleichzeitigen Strömungsabriss über die gesamte Spannweite. Auf die Kammer übertragen: Eine wellenförmige Wandkante (anstatt der geraden Kante bei Variante A) könnte die großflächige Ablösung in viele kleine, örtlich begrenzte Ablösungen aufteilen.

\textbf{Diffusor-Geometrie:} Bei Windkraft-Diffusoren muss der Halbwinkel unter 7\textdegree{} bleiben. Die schräge Trennwand (Variante B) mit 3{,}7\textdegree{} liegt sicher darunter. Stetige Krümmung (Variante C) ist einem geraden Diffusor überlegen, weil der Druckgradient gleichmäßiger verteilt wird.

\begin{keybox}
Die Übertragbarkeit hat Grenzen: Propeller arbeiten bei $\text{Re} = 10^5$--$10^7$, die Stimmzungenkammer bei $\text{Re} \sim 50$--$1500$. Dennoch sind die geometrischen Prinzipien (stetige Krümmung, Vermeidung scharfer Kanten, Diffusor-Winkel) Reynolds-unabhängig gültig.
\end{keybox}

% ============================================================
\section{Warum resonante Kammern die Ansprache verschlechtern}
% ============================================================

Ein verbreiteter Gedanke ist, dass eine resonante Kammer die Zunge unterstützen könnte -- ähnlich wie ein Resonanzkörper einer Gitarre. Praktische Messungen zeigen das Gegenteil.

Die Erklärung liegt in der Art der Kopplung:

\textbf{Parallelkopplung} (L und C parallel): Beide Elemente werden gleichzeitig erregt, Energie pendelt. Impedanz maximal bei Resonanz. Ein Gitarrenkörper arbeitet so.

\textbf{Serienkopplung} (L und C in Reihe): Energie muss \textit{durch} beide Elemente hindurch. Bei Resonanz sinkt die Gesamtimpedanz, aber jedes Element hat eigene Zeitkonstante.

\begin{keybox}
Die Stimmzungenkammer ist eine \textbf{Serienkopplung}. Die Luft muss den Weg Balg $\rightarrow$ Klappe $\rightarrow$ Kanal B $\rightarrow$ Faltung $\rightarrow$ Kanal A $\rightarrow$ Schlitz $\rightarrow$ Spalt $\rightarrow$ Zunge in genau dieser Reihenfolge durchlaufen. Jede Kammer-Resonanz fügt dem System eine eigene Zeitkonstante hinzu.
\end{keybox}

Was passiert bei einer resonanten Kammer? Wenn die Helmholtz-Frequenz ein Vielfaches der Zungenfrequenz ist, kann die Kammer mitschwingen. Dabei wird Energie aus der Hauptschwingung in die Kammer-Eigenschwingung umgeleitet. Für den Amplitudenaufbau wirkt das wie eine zusätzliche Dämpfung.

Praktische Konsequenz: \textbf{Die Kammer soll akustisch möglichst ,,unsichtbar'' sein.} Ihre Eigenfrequenzen sollen weit von der Zungenfrequenz und deren Obertönen entfernt liegen. Mit $f_H = 461$\,Hz bei einer 50-Hz-Zunge ist das Verhältnis 9{,}2 -- kein ganzzahliges Vielfaches. Das ist günstig.

Praktische Erfahrung zeigt, dass die Trennwand das Einschwingverhalten entscheidend beeinflusst -- obwohl alle Varianten im stationären Zustand nahezu identische Volumenströme liefern. Die Erklärung liegt in der Serienkopplung: Die Kammer ist kein Verstärker, sondern ein Wellenleiter. Ihre Form bestimmt nicht den Durchfluss, sondern die Impulsantwort.

\begin{keybox}
Die Optimierung hat zwei Ebenen: Die Zunge selbst ($Q$, Aufbiegung, Material) bestimmt die Grundansprache. Die Kammergeometrie (Trennwandform, Klappenrichtung) bestimmt, wie viel davon im Instrument tatsächlich ankommt. Beides muss stimmen.
\end{keybox}

% ============================================================
\section{Welche Basstöne sind durch die Kammerresonanz benachteiligt?}
% ============================================================

Die Helmholtz-Frequenz dieser Kammer liegt bei 461\,Hz. Welche Basstöne haben einen Oberton, der in die Nähe dieser Resonanz fällt? Denn wie in Kapitel~10 erläutert, entzieht eine resonante Kammer der Zunge Energie über die Serienkopplung.

Die Bandbreite der Helmholtz-Resonanz hängt von ihrer eigenen Güte $Q_H$ ab. Eine Holzkammer mit Klappe hat typisch $Q_H \sim 5$--$10$. Bei $Q_H = 10$ ist das kritische Band $461 \pm 23$\,Hz, bei $Q_H = 5$ erweitert es sich auf $\pm 46$\,Hz.

\begin{figure}[H]
\centering
\includegraphics[width=0.9\textwidth]{resonanz_diag.png}
\caption{Oberes Diagramm: Abstand des nächsten Obertons zu $f_H$. Rot = kritisch, orange = ungünstig, grün = unbeeinflusst. Unteres Diagramm: Oberton-Landkarte -- jeder Punkt ist ein Oberton, rote Rauten liegen im Resonanzband.}
\end{figure}

\begin{keybox}
\textbf{Unter $\sim$70\,Hz (C2) ist fast jeder Ton kritisch betroffen.} Der Grund: Bei tiefen Tönen liegen die Obertöne dicht beieinander (Abstand = Grundfrequenz). Ein Ton von 40\,Hz hat Obertöne bei 40, 80, 120, \ldots, 440, 460, 480\,Hz -- das Raster ist so fein, dass immer ein Oberton in das Resonanzband fällt. Je tiefer der Ton, desto unvermeidlicher ist der Konflikt.
\end{keybox}

Besonders getroffen sind Töne, deren Grundfrequenz ein ganzzahliger Teiler von $f_H$ ist:

\begin{table}[H]
\centering\small
\begin{tabular}{l r l r r}
\toprule
$f_H / n$ & $f$ [Hz] & Nächste Note & Differenz & Oberton $n$ \\
\midrule
461/2 & 230{,}5 & A\#3 & 3{,}2\,Hz & 2 \\
461/3 & 153{,}7 & D\#3 & 2{,}4\,Hz & 3 \\
461/4 & 115{,}3 & A\#2 & 1{,}6\,Hz & 4 \\
461/5 & 92{,}2 & F\#2 & 0{,}3\,Hz & 5 \\
461/6 & 76{,}8 & D\#2 & 1{,}2\,Hz & 6 \\
461/7 & 65{,}9 & C2 & 0{,}7\,Hz & 7 \\
461/8 & 57{,}6 & A\#1 & 0{,}8\,Hz & 8 \\
461/9 & 51{,}2 & G\#1 & 0{,}8\,Hz & 9 \\
461/10 & 46{,}1 & F\#1 & 0{,}1\,Hz & 10 \\
\bottomrule
\end{tabular}
\caption{Ganzzahlige Teiler von $f_H$ und zugehörige Basstöne}
\end{table}

Ab C3 (130{,}8\,Hz) aufwärts werden die Lücken zwischen den Obertönen größer, und die meisten Töne liegen außerhalb des Resonanzbandes. Das erklärt, warum das Problem in der Praxis vor allem bei den tiefsten Basstönen auffällt -- und warum erfahrene Instrumentenbauer die Kammergröße für verschiedene Tonbereiche variieren.

% ============================================================
\section{Was Berechnungen leisten -- und was nicht}
% ============================================================

Die vorgestellten Berechnungen stützen sich auf gut etablierte physikalische Modelle: Euler-Bernoulli-Balkentheorie für die Zungenfrequenz, Bernoulli-Gleichung für die Spaltströmung, Helmholtz-Resonator für die Kammerakustik, Verlustbeiwerte aus der Strömungsmechanik (Idelchik). Diese Modelle liefern Ergebnisse in der richtigen Größenordnung.

Dennoch ersetzen sie keinen praktischen Test:

\textbf{1.\ Idealisierte Geometrie:} Exakt plane Flächen, scharfe Kanten, gleichmäßige Spalte. Real: Holzfasern, Bearbeitungsspuren, individuelle Aufbiegung. Schon 0{,}1\,mm Unterschied in der Aufbiegung ändert $S_\text{eff}$ um $\sim$7\,\%.

\textbf{2.\ Stationäre Strömung:} \textbf{Dies ist die gravierendste Einschränkung.} Die Verlustberechnung nimmt stationäre Strömung an. Tatsächlich pulsiert die Strömung mit 50\,Hz. Praktische Erfahrung zeigt, dass die Trennwandform das Einschwingverhalten entscheidend beeinflusst -- ein Effekt, der im instationären Anlauf (Impulsantwort) liegt und von der stationären Berechnung grundsätzlich nicht erfasst wird (siehe Kapitel~6).

\textbf{3.\ Dämpfung:} $Q$ ist der empfindlichste Parameter. Unterschiede von Faktor~2 zwischen scheinbar identischen Zungen sind normal.

\textbf{4.\ Kammerakustik:} Das Helmholtz-Modell ist vereinfacht. Real hat die Kammer weitere Moden, und die akustischen Eigenschaften ändern sich mit Balgdruck, Temperatur und Luftfeuchtigkeit.

\textbf{5.\ Wechselwirkungen:} Jeder Effekt ist einzeln modelliert. Eine vollständige Lösung würde eine gekoppelte Fluid-Struktur-Akustik-Simulation (FSI) erfordern.

Die Berechnungen identifizieren zuverlässig, welche Parameter wichtig sind und auf welcher Ebene sie wirken. Sie ordnen die drei Trennwandvarianten in eine physikalisch begründete Rangfolge (C besser als B besser als A -- bestätigt durch praktische Erfahrung). Wo die Berechnung an ihre Grenzen stößt, ist die quantitative Vorhersage des Trennwand-Effekts: Dass die Wand wirkt, ist physikalisch begründbar. Wie stark, lässt sich nur messen.

\begin{keybox}
Für den Instrumentenbauer bedeutet das: \textbf{Die Berechnung sagt, wo man suchen soll. Der praktische Test sagt, was man findet.} Beides zusammen ist mehr als jedes für sich allein.
\end{keybox}

% ============================================================
\section{Zusammenfassung}
% ============================================================

\begin{itemize}[leftmargin=5mm, itemsep=3pt]
\item \textbf{Stimmplatte:} Keilförmig (2$\rightarrow$13\,mm). Am freien Ende wirkt der 13-mm-Schlitz als kurzer Kanal mit eigenem Strömungswiderstand ($\zeta \approx 0{,}152$).

\item \textbf{Spalt:} Dreieckig (0$\rightarrow$1{,}5\,mm), $S_\text{eff} = 6\,\text{mm}^2$. Dominiert den Gesamtwiderstand. Alle Varianten: $v \approx 29$\,m/s, $Q \approx 912$\,ml/s bei 1\,kPa.

\item \textbf{Trennwand-Material:} Folie und Furnier sind strömungstechnisch gleichwertig. Furnier ist stabiler und akustisch neutraler.

\item \textbf{Klappenorientierung:} Ausblas ($\zeta = 0{,}50$) besser als Umlenk ($\zeta = 1{,}70$). Ausblas nutzt Wandgeometrie voll.

\item \textbf{Trennwandform:} Praktisch entscheidend für die Ansprache (instationärer Anlauf). C~(Parabel) am saubersten, B~(schräg) guter Kompromiss, A~(gerade) am einfachsten aber am trägsten. Stationärer Durchfluss bei allen gleich -- der Unterschied liegt in der Impulsantwort.

\item \textbf{Ansprache:} $\tau = Q/(\pi f)$. Bei $Q = 100$: 637\,ms, bei $Q = 25$: 159\,ms. Die Kammergeometrie beeinflusst nicht den stationären Durchfluss, aber entscheidend den instationären Anlauf.

\item \textbf{Akustik:} Helmholtz 461\,Hz, Kammer akustisch klein.
\end{itemize}

\end{document}
