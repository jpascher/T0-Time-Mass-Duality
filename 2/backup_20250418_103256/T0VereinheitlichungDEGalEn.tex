\documentclass[a4paper,12pt]{article}
\usepackage[utf8]{inputenc}
\usepackage[T1]{fontenc}
\usepackage{lmodern}
\usepackage[ngerman]{babel}
\usepackage{amsmath, amssymb, amsthm, physics}
\usepackage{graphicx}
\usepackage{xcolor}
\usepackage{tikz}
\usepackage{setspace}
\usepackage{tcolorbox}
\usepackage{booktabs}
\usepackage{siunitx}
\DeclareSIUnit{\year}{yr}
\DeclareSIUnit{\parsec}{pc}
\usepackage[margin=2cm]{geometry}
\usepackage{tocloft}
\usepackage{fancyhdr}

% Headers and Footers
\pagestyle{fancy}
\fancyhf{}
\fancyhead[L]{Johann Pascher}
\fancyhead[R]{Time-Mass Duality}
\fancyfoot[C]{\thepage}
\renewcommand{\headrulewidth}{0.4pt}
\renewcommand{\footrulewidth}{0.4pt}

% Table of Contents Styling
\renewcommand{\cftsecfont}{\color{blue}}
\renewcommand{\cftsubsecfont}{\color{blue}}
\renewcommand{\cftsecpagefont}{\color{blue}}
\renewcommand{\cftsubsecpagefont}{\color{blue}}
\setlength{\cftsecindent}{1cm}
\setlength{\cftsubsecindent}{2cm}

\usepackage{hyperref}
\hypersetup{
	colorlinks=true,
	linkcolor=blue,
	filecolor=blue,
	citecolor=blue, 
	urlcolor=blue,
	bookmarks=true,
	bookmarksopen=true,
	pdftitle={Unification of the T0 Model: Foundations, Dark Energy, and Galaxy Dynamics},
	pdfauthor={Johann Pascher},
}

\usepackage{cleveref}

\newtheorem{theorem}{Theorem}[section]
\newtheorem{lemma}[theorem]{Lemma}
\newtheorem{proposition}[theorem]{Proposition}
\newtheorem{corollary}[theorem]{Corollary}

\theoremstyle{definition}
\newtheorem{definition}{Definition}[theorem]

\theoremstyle{remark}
\newtheorem{remark}{Remark}

% Custom commands
\newcommand{\Tfield}{T(x)}
\newcommand{\DcovT}[1]{\Tfield D_\mu #1 + #1 \partial_\mu \Tfield}
\newcommand{\DhiggsT}{\Tfield (\partial_\mu + ig A_\mu) \Phi + \Phi \partial_\mu \Tfield}
\newcommand{\betaT}{\beta_{\text{T}}}
\newcommand{\alphaEM}{\alpha_{\text{EM}}}
\newcommand{\alphaW}{\alpha_{\text{W}}}
\newcommand{\Mpl}{M_{\text{Pl}}}
\newcommand{\Tzerot}{T_0(\Tfield)}
\newcommand{\Tzero}{T_0}
\newcommand{\vecx}{\vec{x}}
\newcommand{\gammaf}{\gamma_{\text{Lorentz}}}

\begin{document}
	
	\title{Unification of the T0 Model: \\ Foundations, Dark Energy, and Galaxy Dynamics}
	\author{Johann Pascher}
	\date{March 27, 2025}
	\maketitle
	
	\begin{abstract}
		This work presents a unified framework for the T0 model, integrating its foundational principles with applications to dark energy and galaxy dynamics in a static universe. Based on absolute time and variable mass, the T0 model contrasts with relativity’s relative time and constant mass, offering alternative explanations for cosmic redshift (via energy loss), dark energy (emergent from the intrinsic time field \(\Tfield\)), and galaxy dynamics (through mass variation without dark matter). This paper ensures mathematical consistency across these domains and provides a comprehensive theory with experimentally testable predictions.
	\end{abstract}
	
	\tableofcontents
	\newpage
	
	\section{Introduction to the T0 Model: Core Concepts}
	
	\subsection{Fundamental Assumptions of the T0 Model}
	
	The T0 model is based on assumptions detailed in \cite{pascher_params_2025} and \cite{pascher_galaxies_2025}:
	
	\begin{tcolorbox}[colback=blue!5!white,colframe=blue!75!black,title=Fundamental Assumptions of the T0 Model]
		\begin{itemize}
			\item Time is absolute and universally constant (\cite{pascher_params_2025}, Section “Time-Mass Duality”).
			\item Mass varies as \(m = \frac{\hbar}{\Tfield c^2}\), where \(\Tfield\) is the intrinsic time field (\cite{pascher_params_2025}, Section “Intrinsic Time”).
			\item Gravity emerges from gradients of \(\Tfield\) (\cite{pascher_galaxies_2025}, Section “Emergent Gravity”).
			\item Redshift results from energy loss: \(1 + z = e^{\alpha d}\) (\cite{pascher_messdifferenzen_2025}, Section “Energy Loss”).
		\end{itemize}
	\end{tcolorbox}
	
	\subsection{Intrinsic Time and Time-Mass Duality}
	
	The intrinsic time \(\Tfield\) is defined as:
	\begin{equation}
		\Tfield = \frac{\hbar}{m c^2}
	\end{equation}
	Details in \cite{pascher_params_2025} (Section “Definition of Intrinsic Time”). This leads to the duality:
	\begin{itemize}
		\item \textbf{Standard Model}: Relative time, constant mass.
		\item \textbf{T0 Model}: Absolute time, variable mass (\cite{pascher_params_2025}).
	\end{itemize}
	
	\subsection{Unified Lagrangian Density}
	
	The Lagrangian density is derived in \cite{pascher_lagrange_2025} (Section “Total Lagrangian Density”):
	\begin{equation}
		\mathcal{L}_\text{total} = \mathcal{L}_\text{SM} + \mathcal{L}_\text{Higgs} + \mathcal{L}_\text{intrinsic}
	\end{equation}
	With \(\mathcal{L}_\text{intrinsic} = \frac{1}{2} \partial_\mu \Tfield \partial^\mu \Tfield - V(\Tfield)\).
	
	\subsection{The Role of Gravity in the T0 Model}
	
	Gravity emerges from \(\Tfield\):
	\begin{theorem}[Emergence of Gravity]
		\begin{equation}
			\nabla \Tfield = -\frac{\hbar}{m^2 c^2} \nabla m \sim \nabla \Phi_g
		\end{equation}
		See \cite{pascher_galaxies_2025} (Section “Emergent Gravity”).
	\end{theorem}
	
	\section{Dark Energy in the T0 Model}
	
	\subsection{Reinterpretation of Dark Energy}
	
	Dark energy is an emergent effect of \(\Tfield\):
	\begin{itemize}
		\item \textbf{\(\Lambda\)CDM}: Cosmological constant.
		\item \textbf{T0 Model}: Energy exchange via \(\Tfield\) (\cite{pascher_energy_2025}, Section “Dark Energy”).
	\end{itemize}
	Energy density:
	\begin{equation}
		\rho_{DE}(r) = \frac{\kappa}{r^2}
	\end{equation}
	
	\subsection{Field-Theoretic Description}
	
	\begin{equation}
		\mathcal{L}_\text{intrinsic} = \frac{1}{2} \partial_\mu \Tfield \partial^\mu \Tfield - V(\Tfield)
	\end{equation}
	Field equation:
	\begin{equation}
		\Box \Tfield - \frac{dV}{d\Tfield} = 0
	\end{equation}
	See \cite{pascher_lagrange_2025}.
	
	\subsection{Energy Transfer and Redshift}
	
	Redshift due to energy loss:
	\begin{equation}
		\frac{d E_{\gamma}}{d x} = -\alpha E_{\gamma}, \quad 1 + z = e^{\alpha d}
	\end{equation}
	With \(\alpha \approx \SI{2.3e-18}{\per\meter}\) (\cite{pascher_messdifferenzen_2025}).
	
	\section{Galaxy Dynamics in the T0 Model}
	
	\subsection{Flat Rotation Curves Without Dark Matter}
	

%---
Rotation curves:
\begin{equation}
	v^2(r) = \frac{M(r)}{r} + \kappa r
\end{equation}
in natural units where \(G = 1\). In SI units, an empirical value of \(\kappa \approx \SI{4.8e-11}{\meter\per\second\squared}\) is used (\cite{pascher_galaxies_2025}).
%---	
	\subsection{Effective Gravitational Constant}
	
	\begin{equation}
		G_{\text{eff}}(r) = G \left(1 + \betaT \frac{\kappa}{r}\right)
	\end{equation}
	With \(\betaT^{\text{SI}} \approx 0.008\) (\cite{pascher_params_2025}).
	
	\section{Unified Mathematical Formulation}
	
	\subsection{Common Field Equations}
	
	Action:
	\begin{equation}
		S_\text{unified} = \int \mathcal{L}_\text{total} \, d^4x
	\end{equation}
	Static universe:
	\begin{align}
		\left(\frac{\dot{m}}{m}\right)^2 &= \frac{8\pi G}{3} \rho_{\text{eff}} \\
		\frac{\ddot{m}}{m} &= -\frac{4\pi G}{3} (\rho_{\text{eff}} + 3p_{\text{eff}})
	\end{align}
	

	\subsection{Consistent Parameterization}
	
	Parameters:
	\begin{itemize}
		\item \(\alpha \approx \SI{2.3e-18}{\per\meter}\)
		\item \(\kappa \approx \SI{4.8e-11}{\meter\per\second\squared}\)
		\item \(\betaT^{\text{SI}} \approx 0.008\), \(\betaT^{\text{nat}} = 1\) (\cite{pascher_params_2025}).
	\end{itemize}
	Relationship:
	\begin{equation}
		\kappa = \betaT \frac{y v c^2}{r_g^2}
	\end{equation}
	

	\section{Experimental Tests of the T0 Model}
	
	\subsection{Common Predictions}
	
	\begin{enumerate}
		\item Mass-dependent time evolution (\cite{pascher_photons_2025}).
		\item Environment-dependent redshift: \(\frac{z_{\text{Cluster}}}{z_{\text{Void}}} \approx 1 + 0.003\).
		\item Differential redshift: \(\frac{z(\lambda_1)}{z(\lambda_2)} \approx 1 + \betaT^{\text{SI}} \ln\left(\frac{\lambda_1}{\lambda_2}\right)\), where \(\lambda_1\) and \(\lambda_2\) are different wavelengths.

	\end{enumerate}
	
	\subsection{Tests for Galaxy Dynamics}
	
	\begin{enumerate}
		\item Tully-Fisher Relation: \(L \propto v_{\text{max}}^{4 + \epsilon}\), \(\epsilon \approx \betaT\).
		\item Gravitational lensing effects: \(\alpha_{\text{lens}} \propto \int \nabla \Phi \, dz\) (\cite{pascher_galaxies_2025}).
	\end{enumerate}
	
	\section{Comparison with the \(\Lambda\)CDM Standard Model}
	
	\begin{tcolorbox}[colback=yellow!5!white,colframe=yellow!75!black,title=Model Comparison]
		\begin{tabular}{p{0.45\textwidth}|p{0.45\textwidth}}
			\toprule
			\textbf{\(\Lambda\)CDM Model} & \textbf{T0 Model} \\
			\midrule
			Dark matter as particles & No dark matter, mass variation \\
			NFW profile: \(\rho_{\text{DM}}(r)\) & \(\rho_{\text{eff}}(r) \approx \frac{\kappa}{r^2}\) \\
			Relative time, constant mass & Absolute time, variable mass \\
			Dark energy drives expansion & Dark energy from \(\Tfield\) exchange \\
			Redshift from expansion & Redshift from energy loss \\
			Expanding universe & Static universe \\
			\bottomrule
		\end{tabular}
	\end{tcolorbox}
	
	\section{Summary}
	
	The T0 model unifies absolute time and variable mass to explain cosmic phenomena, supported by internal consistency and references to \cite{pascher_galaxies_2025, pascher_params_2025, pascher_messdifferenzen_2025}.
	
	\begin{thebibliography}{99}
		\bibitem{pascher_galaxies_2025} Pascher, J. (2025). \href{https://github.com/jpascher/T0-Time-Mass-Duality/tree/main/2/pdf/English/MassVarGalaxienEn.pdf}{Mass Variation in Galaxies: An Analysis in the T0 Model with Emergent Gravity}. March 30, 2025.
		\bibitem{pascher_messdifferenzen_2025} Pascher, J. (2025). \href{https://github.com/jpascher/T0-Time-Mass-Duality/tree/main/2/pdf/English/MessdifferenzenT0StandardEn.pdf}{Compensatory and Additive Effects: An Analysis of Measurement Differences Between the T0 Model and the \(\Lambda\)CDM Standard Model}. April 2, 2025.
		\bibitem{pascher_params_2025} Pascher, J. (2025). \href{https://github.com/jpascher/T0-Time-Mass-Duality/tree/main/2/pdf/English/ZeitMasseT0ParamsEn.pdf}{Time-Mass 
	\bibitem{pascher_temp_2025} Pascher, J. (2025). \href{https://github.com/jpascher/T0-Time-Mass-Duality/tree/main/2/pdf/English/TempEinheitenCMBEn.pdf}{Adjustment of Temperature Units in Natural Units and CMB Measurements}. April 2, 2025.		
			Duality Theory (T0 Model): Derivation of Parameters \(\kappa\), \(\alpha\), and \(\beta\)}. April 4, 2025.
		\bibitem{pascher_temp_2025} Pascher, J. (2025).
		
		 \href{https://github.com/jpascher/T0-Time-Mass-Duality/tree/main/2/pdf/English/NatEinheitenAlpha1En.pdf}{Adjustment of Temperature Units in Natural Units and CMB Measurements}. April 2, 2025.
		\bibitem{pascher_lagrange_2025} Pascher, J. (2025). \href{https://github.com/jpascher/T0-Time-Mass-Duality/tree/main/2/pdf/English/MathZeitMasseLagrange.pdf}{From Time Dilation to Mass Variation: Mathematical Core Formulations of Time-Mass Duality Theory}. March 29, 2025.
		\bibitem{pascher_higgs_2025} Pascher, J. (2025). \href{https://github.com/jpascher/T0-Time-Mass-Duality/tree/main/2/pdf/English/MathHiggsZeitMasseEn.pdf}{Mathematical Formulation of the Higgs Mechanism in Time-Mass Duality}. March 28, 2025.
		\bibitem{pascher_photons_2025} Pascher, J. (2025). \href{https://github.com/jpascher/T0-Time-Mass-Duality/tree/main/2/pdf/English/DynMassePhotonenNichtlokalEn.pdf}{Dynamic Mass of Photons and Its Implications for Nonlocality in the T0 Model}. March 25, 2025.
		\bibitem{pascher_energy_2025} Pascher, J. (2025). \href{https://github.com/jpascher/T0-Time-Mass-Duality/tree/main/2/pdf/English/MathEnergiedynamikEn.pdf}{Dark Energy in the T0 Model: A Mathematical Analysis of Energy Dynamics}. April 3, 2025.
	\end{thebibliography}
	
\end{document}