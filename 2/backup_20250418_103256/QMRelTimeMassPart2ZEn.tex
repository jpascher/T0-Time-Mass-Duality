\documentclass[twocolumn,aps,prl]{revtex4-2}
\usepackage[utf8]{inputenc}
\usepackage[T1]{fontenc}
\usepackage[ngerman]{babel}
\usepackage{lmodern}
\usepackage{amsmath}
\usepackage{amssymb}
\usepackage{physics}
\usepackage{hyperref}
\usepackage{booktabs}
\usepackage{enumitem}
\usepackage[table,xcdraw]{xcolor}
\usepackage{pgfplots}
\pgfplotsset{compat=1.18}
\usepackage{graphicx}
\usepackage{siunitx}
\usepackage{array} % Added for custom column types

% Custom commands
\newcommand{\Tfield}{T(x)}
\newcommand{\alphaEM}{\alpha_{\text{EM}}}
\newcommand{\alphaW}{\alpha_{\text{W}}}
\newcommand{\betaT}{\beta_{\text{T}}}
\newcommand{\Mpl}{M_{\text{Pl}}}
\newcommand{\Tzerot}{T_0(\Tfield)}
\newcommand{\Tzero}{T_0}
\newcommand{\vecx}{\vec{x}}
\newcommand{\gammaf}{\gamma_{\text{Lorentz}}}
\newcommand{\DhiggsT}{\Tfield (\partial_\mu + ig A_\mu) \Phi + \Phi \partial_\mu \Tfield}
\newcommand{\LCDM}{\Lambda\text{CDM}}
\newcommand{\DTmu}{D_{T,\mu}}
\newcommand{\calL}{\mathcal{L}}
\newcommand{\deq}{\displaystyle}
\newcommand{\e}{\mathrm{e}}

\hypersetup{
	colorlinks=true,
	linkcolor=blue,
	citecolor=blue,
	urlcolor=blue,
	pdftitle={Bridging Quantum Mechanics and Relativity through Time-Mass Duality: Part II},
	pdfauthor={Johann Pascher},
	pdfsubject={Theoretical Physics},
	pdfkeywords={T0 Model, natural units, time-mass duality, cosmology}
}

\begin{document}
	
	\title{Bridging Quantum Mechanics and Relativity through Time-Mass Duality: A Unified Framework with Natural Units \(\alpha = \beta = 1\) \\ Part II: Cosmological Implications and Experimental Validation}
	\author{Johann Pascher}
	\affiliation{Department of Communications Engineering, Höhere Technische Bundeslehranstalt (HTL), Leonding, Austria}
	\email{johann.pascher@gmail.com}
	\date{April 7, 2025}
	
	\begin{abstract}
		This paper extends the T0 model introduced in Part I into the realms of cosmology and experimental validation, building on a unified natural unit system where \(\hbar = c = G = k_B = \alphaEM = \alphaW = \betaT = 1\). In contrast to the expanding universe of the \(\LCDM\) model, we propose a static cosmos where redshift arises from photon energy loss mediated by the intrinsic time field \(\Tfield\). This framework reinterprets dark matter and dark energy through emergent gravitational effects, enhancing the Standard Model (SM) with a consistent gravitational theory. Key predictions include a wavelength-dependent redshift with a variation of approximately \(2.3\%\) per decade, a cosmic microwave background (CMB) temperature of \(24000 \, \text{K}\) at \(z = 1100\), and speculative extensions beyond the speed of light. These predictions are testable using instruments like the James Webb Space Telescope (JWST) and future CMB missions. We address measurement challenges, such as the frequency-dependent biases in GPS precision and cosmological observations, which obscure distinctions between mass variation and time dilation, offering a philosophically coherent alternative to \(\LCDM\) that aligns theoretical elegance with empirical rigor.
	\end{abstract}
	
	\maketitle
	
	\section{Introduction}
	\label{sec:introduction}
	
	In Part I (\textit{Bridging Quantum Mechanics and Relativity through Time-Mass Duality: Part I}, \cite{pascher_part1_2025}), we established the T0 model as a unified framework for quantum mechanics (QM) and relativity theory (RT), leveraging the intrinsic time field \(\Tfield = \frac{\hbar}{\max(mc^2, \omega)}\) within a natural unit system (\(\hbar = c = G = k_B = \alphaEM = \alphaW = \betaT = 1\)). This system, detailed in Part I, Section 2 "Unification of Constants with Natural Units" \href{https://github.com/jpascher/T0-Time-Mass-Duality/tree/main/2/pdf/English/QMRelTimeMassPart1En.pdf}{[Teil I]}, eliminates empirically determined constants, achieving consistency with measured values (e.g., \(c \approx 3 \times 10^8 \, \text{m/s}\), \(\alphaEM \approx 1/137.036\)) with deviations below \(10^{-6}\). It enabled a mass-dependent Schrödinger equation (Part I, Equation (4.5) \href{https://github.com/jpascher/T0-Time-Mass-Duality/tree/main/2/pdf/English/QMRelTimeMassPart1En.pdf}{[Teil I]}) and emergent gravitation (Part I, Section 5 "Emergent Gravitation" \href{https://github.com/jpascher/T0-Time-Mass-Duality/tree/main/2/pdf/English/QMRelTimeMassPart1En.pdf}{[Teil I]}), bridging micro- and macroscopic scales.
	
	Part II extends these foundations into cosmology and experimental validation, contrasting with the \(\LCDM\) model’s expanding universe, which originates from a Big Bang approximately 13.8 billion years ago \cite{Planck2020}. In \(\LCDM\), cosmic redshift is a kinematic effect (\(z \approx H_0 d / c\)), requiring inflation and dark energy \cite{Riess1998,Perlmutter1999}. The T0 model proposes a static, infinite, and eternal universe where redshift stems from photon energy loss via \(\Tfield\), enhancing the Standard Model (SM) with a consistent gravitational theory while retaining its particle physics core.
	
	Key predictions include:
	- Wavelength-dependent redshift (\(\sim 2.3\%\) per decade),
	- CMB temperature of \(24000 \, \text{K}\) at \(z = 1100\),
	- Speculative superluminal extensions.
	
	These are testable with JWST spectroscopy and CMB distortion measurements, though frequency-based methods (e.g., GPS, redshift) conflate mass variation and time dilation, necessitating careful reassessment \cite{pascher_quantum_2025}. Philosophically, T0 avoids singularities, offering a coherent eternal cosmos \cite{pascher_perspective_2025}.
	
	This paper is structured as:
	- Section 2: Static universe and redshift mechanism.
	- Section 3: Cosmological phenomena and predictions.
	- Section 4: Quantitative predictions.
	- Section 5: Experimental tests and measurement challenges.
	- Section 6: Implications of \(\betaT = 1\).
	- Section 7: Integration with T0 principles.
	- Section 8: Speculative extensions and philosophy.
	
	\section{Static Universe Model}
	\label{sec:static_universe}
	
	\subsection{Concept of a Static Universe}
	\label{subsec:static_concept}
	
	The T0 model envisions a static universe, infinite in space and eternal in time, contrasting with \(\LCDM\)’s expanding cosmos from a Big Bang. In \(\LCDM\), redshift (\(z \approx H_0 d / c\)) reflects expansion (\(H_0 \approx 70 \, \text{km/s/Mpc}\)) \cite{Planck2020}, requiring inflation for uniformity and dark energy for acceleration (\(\Omega_{\Lambda} \approx 0.7\)) \cite{Riess1998}. T0 eliminates these, positing a stable cosmos where \(\Tfield\) governs dynamics without expansion.
	
	Advantages include:
	- \textbf{Horizon Problem:} Infinite time ensures thermal equilibrium across scales \cite{pascher_messdifferenzen_2025}.
	- \textbf{Flatness:} No expansion eliminates curvature tuning.
	- \textbf{Singularity-Free:} Eternal existence avoids infinite density \cite{pascher_perspective_2025}.
	
	This complements SM particle physics with a static gravitational framework derived in Part I, Section 5 "Emergent Gravitation" \href{https://github.com/jpascher/T0-Time-Mass-Duality/tree/main/2/pdf/English/QMRelTimeMassPart1En.pdf}{[Teil I]}.
	
	\subsection{Redshift through Energy Loss}
	\label{subsec:redshift_energy_loss}
	
	Redshift in T0 is:
	\begin{equation}
		1 + z = e^{\alpha d},
		\label{eq:redshift_distance}
	\end{equation}
	with \(\alpha = H_0 / c \approx 2.3 \times 10^{-18} \, \text{m}^{-1}\) (SI) or 1 (natural units). At low \(z\):
	\begin{equation}
		z \approx \alpha d,
		\label{eq:hubble_approx}
	\end{equation}
	matching \(\LCDM\) locally. The mechanism is photon energy loss:
	\begin{equation}
		\frac{dE}{dx} = -\alpha E,
		\label{eq:energy_loss_rate}
	\end{equation}
	yielding \(E = E_0 e^{-\alpha d}\), and thus \(1 + z = e^{\alpha d}\), as derived from \(\Tfield\) properties in Part I, Section 3.1 "Definition and Physical Basis" \href{https://github.com/jpascher/T0-Time-Mass-Duality/tree/main/2/pdf/English/QMRelTimeMassPart1En.pdf}{[Teil I]} \cite{pascher_messdifferenzen_2025}.
	
	\section{Cosmological Phenomena}
	\label{sec:cosmological_phenomena}
	
	\subsection{Temperature-Redshift Relation and CMB}
	\label{subsec:cmb_temp}
	
	\(\LCDM\)’s \(T(z) = T_0 (1 + z)\) gives \(T \approx 3000 \, \text{K}\) at \(z = 1100\) (\(T_0 = 2.725 \, \text{K}\)) \cite{Fixsen2009}. 
%---
T0 predicts:
\begin{equation}
	T(z) = T_0 (1 + z) (1 + \ln(1 + z)),
	\label{eq:temperature_redshift_simplified}
\end{equation}
in natural units with \(\beta_T = 1\), so \(T(1100) \approx \SI{24000}{\kelvin}\), reflecting enhanced energy loss (Equation \ref{eq:energy_loss_rate}). In SI units, an empirical value of \(\beta_T \approx 0.008\) is used \cite{pascher_temp_2025}.
%---	
	\subsection{Wavelength-Dependent Redshift}
	\label{subsec:wavelength_redshift}
	
	T0 predicts:
	\begin{equation}
		z(\lambda) = z_0 \left(1 + \ln\left(\frac{\lambda}{\lambda_0}\right)\right),
		\label{eq:wavelength_redshift}
	\end{equation}
	with \(\Delta z / z_0 \approx 3.85\%\) over 0.6-28 \(\mu\text{m}\) (JWST range), due to:
	\begin{equation}
		\frac{dE}{dx} = -\alpha E \left(1 + \ln\left(\frac{\lambda}{\lambda_0}\right)\right),
		\label{eq:wavelength_energy_loss}
	\end{equation}
	contrasting \(\LCDM\)’s uniformity \cite{pascher_params_2025}.
	
	\subsection{Dark Matter and Dark Energy Reinterpretation}
	\label{subsec:dark_reinterpretation}
%----
The potential:
\begin{equation}
	\Phi(r) = -\frac{M}{r} + \kappa r,
	\label{eq:grav_potential_t0}
\end{equation}
in natural units where \(G = 1\) (empirical SI value: \(\kappa \approx \SI{4.8e-11}{\meter\per\second\squared}\)) reinterprets:
%----	
	- \textbf{Dark Matter:} \(v(r) = \sqrt{\frac{M}{r} + \kappa r}\), as derived in Part I, Section 5.1 "Derivation from \(\Tfield\)" \href{https://github.com/jpascher/T0-Time-Mass-Duality/tree/main/2/pdf/English/QMRelTimeMassPart1En.pdf}{[Teil I]}.
	- \textbf{Dark Energy:} \(\rho_{\text{DE}} \approx \frac{\kappa}{r^2}\) \cite{pascher_galaxies_2025}.
	
	\subsection{Influence on Galaxy Dynamics}
	\label{subsec:galaxy_dynamics}
	
	The T0 model shapes galaxy dynamics through \(\Tfield\), offering an alternative to \(\LCDM\) by reinterpreting gravitational effects without dark matter or expansion.
	
	\subsubsection{Rotation Curves}

	
	%----
	The potential (Equation \ref{eq:grav_potential_t0}) yields:
	\begin{equation}
		v(r) = \sqrt{\frac{M}{r} + \kappa r},
		\label{eq:rotation_velocity}
	\end{equation}
	in natural units where \(G = 1\), reproducing flat rotation curves (e.g., Milky Way: \(v(30 \, \text{kpc}) \approx \SI{211}{\kilo\meter\per\second}\) in SI units) \cite{pascher_galaxies_2025}.
	%----

	
	\begin{figure}[ht]
		\centering
		\begin{tikzpicture}
			\begin{axis}[
				xlabel={Radius [kpc]},
				ylabel={Rotation Velocity [km/s]},
				xlabel style={font=\large},
				ylabel style={font=\large},
				tick label style={font=\normalsize},
				xmin=0, xmax=30,
				ymin=0, ymax=300,
				legend pos=south east,
				legend style={font=\large},
				grid=both,
				minor tick num=4,
				major grid style={line width=0.8pt, gray!50},
				minor grid style={line width=0.4pt, gray!20},
				width=\columnwidth
				]
				\addplot[blue, ultra thick, domain=0.1:30, samples=100] {220*sqrt(10/x)};
				\addplot[red, dashed, ultra thick, domain=0.1:30, samples=100] {sqrt(220^2*10/x + 4.8*x^2)};
				\legend{Newtonian Prediction, T0 Model}
			\end{axis}
		\end{tikzpicture}
		\caption{Rotation curves comparing Newtonian (blue) and T0 model (red) predictions for a galaxy with \(M = 10^{11} M_{\odot}\), \(\kappa_{\text{SI}} = 4.8 \times 10^{-11} \, \text{m/s}^2\).}
		\label{fig:rotation_curves}
	\end{figure}
	
	\subsubsection{Galaxy Formation and Evolution}
	In \(\LCDM\), galaxy formation relies on gravitational collapse of dark matter halos over 13.8 billion years \cite{Planck2020}. T0 proposes gradual baryonic aggregation in an infinite-time universe, driven by \(\Tfield\) and the emergent potential (Equation \ref{eq:grav_potential_t0}), enhancing SM dynamics without dark matter \cite{pascher_galaxies_2025}.
	
	\subsubsection{Cluster Dynamics and Large-Scale Structure}
	For clusters like the Bullet Cluster, T0’s \(\kappa r\) term reduces mass discrepancies:
	\begin{equation}
		v_{\text{cluster}}(r) = \sqrt{\frac{M_{\text{total}}}{r} + \kappa r},
		\label{eq:cluster_velocity}
	\end{equation}
	aligning lensing and dynamical mass estimates without dark matter, testable with precise surveys \cite{pascher_emergente_2025}. Large-scale structure emerges from infinite-time \(\Tfield\) dynamics (Part I, Section 5 \href{https://github.com/jpascher/T0-Time-Mass-Duality/tree/main/2/pdf/English/QMRelTimeMassPart1En.pdf}{[Teil I]}).
	
	\begin{table}[ht]
		\centering
		\caption{Comparison of \(\LCDM\) and T0 Model Predictions for Galaxy Dynamics}
		\label{tab:galaxy_dynamics_comparison}
		\small
		\begin{tabular}{p{0.35\columnwidth} p{0.3\columnwidth} p{0.25\columnwidth}}
			\hline
			\textbf{Phenomenon} & \textbf{\(\LCDM\)} & \textbf{T0 Model} \\
			\hline
			Rotation Curve & Dark matter halo & \(\kappa r\) term \\
			Galaxy Formation & Dark matter collapse & Baryonic aggregation \\
			Cluster Mass & Dark matter dominant & Baryonic + \(\Tfield\) \\
			Large-Scale Structure & Expansion-driven & \(\Tfield\)-driven \\
			\hline
		\end{tabular}
	\end{table}
	
	\section{Quantitative Predictions}
	\label{sec:predictions}
	
	\subsection{CMB Temperature Prediction}
	\label{subsec:cmb_temp_prediction}
	

%----
T0 predicts a CMB temperature at \(z = 1100\) of:
\begin{equation}
	T(1100) \approx \SI{24000}{\kelvin},
	\label{eq:cmb_temp_t0}
\end{equation}
in natural units, versus \(\LCDM\)’s \(\SI{3000}{\kelvin}\), due to \(\Tfield\)’s logarithmic enhancement (Equation \ref{eq:temperature_redshift_simplified}).
%----	
	\subsection{Wavelength-Dependent Redshift Variation}
	\label{subsec:wavelength_redshift_prediction}
	
	Across JWST’s range (0.6-28 \(\mu\text{m}\)):
	\begin{equation}
		\Delta z / z_0 \approx 3.85\%,
		\label{eq:wavelength_variation}
	\end{equation}
	or \(2.3\%\) per decade, testable via quasar emission lines \cite{pascher_params_2025}.
	
	\subsection{Galaxy Rotation Velocities}
	\label{subsec:rotation_velocity_prediction}
	

%----
For the Milky Way:
\begin{equation}
	v(r) = \sqrt{\frac{M}{r} + \kappa r},
	\label{eq:rotation_velocity_repeat}
\end{equation}
in natural units where \(G = 1\), e.g., \(v(30 \, \text{kpc}) \approx \SI{211}{\kilo\meter\per\second}\) in SI units, consistent with observations \cite{McGaugh2016}.
%----	
	\begin{figure}[ht]
		\centering
		\begin{tikzpicture}
			\begin{axis}[
				xlabel={Redshift \(z\)},
				ylabel={Distance Modulus \(\mu\)},
				xmin=0, xmax=2,
				ymin=30, ymax=50,
				legend pos=north west,
				grid=both,
				width=\columnwidth,
				height=6cm,
				samples=100
				]
				\addplot[blue, thick, domain=0.01:2] {5*log10(3e8/70e3*ln(1+x)*(1+x)*0.1) + 25};
				\addplot[red, dashed, domain=0.01:2] {5*log10(3e8/70e3*(1+x)*(2-(1/(1+x)))*1) + 25};
				\legend{T0 Model, \(\LCDM\)}
			\end{axis}
		\end{tikzpicture}
		\caption{Distance modulus vs. redshift comparing T0 (blue) and \(\LCDM\) (red) predictions.}
		\label{fig:distance_modulus}
	\end{figure}
	
	\section{Experimental Tests}
	\label{sec:tests}
	
	\subsection{JWST Spectroscopy}
	\label{subsec:jwst_test}
	
	\(\Delta z / z \approx 3.85\%\) at \(z = 7\) is detectable with JWST’s precision, distinguishing T0 from \(\LCDM\) \cite{pascher_params_2025}.
	
	\subsection{CMB Distortions}
	\label{subsec:cmb_distortions_test}

	T0 predicts:
	\begin{equation}
		\mu \approx 1.4 \times 10^{-5}, \quad y \approx 1.6 \times 10^{-6},
		\label{eq:distortion_parameters}
	\end{equation}
	versus \(\LCDM\)’s \(\mu \approx 2 \times 10^{-8}\), \(y \approx 4 \times 10^{-9}\), measurable with PIXIE \cite{pascher_temp_2025}.
	
	\subsection{Measurement Problem: GPS and Clock Precision}
	\label{subsec:gps_clock_problem}
	
	GPS clocks show a shift of \(\Delta t \approx 38 \, \mu\text{s/day}\), interpreted as mass variation in T0, indistinguishable from GR’s time dilation with current methods \cite{pascher_quantum_2025}.
	
	\subsection{Measurement Problem: Cosmological Observations}
	\label{subsec:cosmological_measurement_problem}
	
	Redshift in T0 is energy loss (Equation \ref{eq:redshift_distance}), not Doppler, requiring non-frequency-based tests like decay rates \cite{pascher_alphabeta_2025}.
	
	\subsection{Reassessment of Measurements}
	\label{subsec:reassessment_measurements}
	
	\(\betaT = 1\) aligns cosmological data with T0’s static framework, resolving tensions like \(H_0\) discrepancies \cite{DiValentino2021}.
	
	\section{Consequences of Setting \(\beta = 1\)}
	\label{sec:consequences_beta}
	
	\subsection{Theoretical Elegance}
	\label{subsec:theoretical_elegance}
	
	\(\betaT = 1\) unifies constants, simplifies interactions (Part I, Section 4.1 \href{https://github.com/jpascher/T0-Time-Mass-Duality/tree/main/2/pdf/English/QMRelTimeMassPart1En.pdf}{[Teil I]}), and enhances coherence \cite{pascher_alphabeta_2025}.
	
	\subsection{Conversion to SI Units}
	\label{subsec:conversion_si}
	
	\begin{equation}
		\betaT^{\text{SI}} = \betaT^{\text{nat}} \cdot \frac{\xi \cdot l_{P,\text{SI}}}{r_{0,\text{SI}}},
		\label{eq:beta_conversion}
	\end{equation}
	yields \(\betaT^{\text{SI}} \approx 0.008\), consistent with observations \cite{pascher_alphabeta_2025}.
	
	\section{Integration into the Time-Mass Duality Theory}
	\label{sec:integration_t0}
	
	\subsection{Consistency with Basic Principles}
	\label{subsec:consistency_principles}
	
	\(\betaT = 1\) supports absolute time, mass variation, and emergent gravitation (Part I, Section 5.1 \href{https://github.com/jpascher/T0-Time-Mass-Duality/tree/main/2/pdf/English/QMRelTimeMassPart1En.pdf}{[Teil I]}) \cite{pascher_lagrange_2025}.
	
	\section{Beyond the Limits}
	\label{sec:beyond_limits}
	
	\subsection{Speculative Extensions}
	\label{subsec:speculative_extensions}
	
	\(\Tfield = \frac{\hbar}{m c^2}\) suggests slower dynamics below \(m_P\), testable near black holes \cite{pascher_planck_2025}.
	
	\begin{figure}[ht]
		\centering
		\begin{tikzpicture}
			\draw[->] (0,0) -- (6,0) node[right] {Mass \(m\)};
			\draw[->] (0,0) -- (0,4) node[above] {Time \(T\)};
			\draw[scale=0.5, domain=0.1:10, smooth, variable=\x, blue, thick] plot ({\x}, {1/\x});
			\draw[dotted, red] (1.5,0) -- (1.5,1.5) -- (0,1.5);
			\node at (1.5,-0.3) {\(m_P\)};
			\node at (-0.3,1.5) {\(t_P\)};
		\end{tikzpicture}
		\caption{Mass vs. intrinsic time near Planck scale.}
		\label{fig:mass_time}
	\end{figure}
	
	\subsection{Philosophical Reflections}
	\label{subsec:philosophical_reflections}
	
	The T0 model’s static, eternal cosmos fundamentally departs from \(\Lambda\)CDM’s finite, expanding universe, offering profound philosophical implications. By avoiding singularities and infinite density, T0 presents a unified reality where time is an intrinsic property (\(\Tfield\)) rather than a relativistic variable, and mass adapts dynamically to local conditions. This contrasts with \(\Lambda\)CDM’s fragmented ontology—featuring a Big Bang origin, dark components, and an uncertain fate—by proposing a coherent, infinite framework that aligns with intuitive notions of existence without beginning or end.
	
	The elimination of expansion and dark entities simplifies cosmology while preserving empirical consistency (Sections 4, 5), suggesting that the universe’s apparent complexity may stem from misinterpretations of frequency-based measurements (Section 5.4). Philosophically, T0 resonates with a holistic view of nature, where quantum and relativistic phenomena emerge from a single principle—time-mass duality—enhancing the SM’s explanatory power across all scales \cite{pascher_perspective_2025}.
	
	\section{Conclusion}
	\label{sec:conclusion}
	
	Part II demonstrates that the T0 model extends the SM with a static, testable cosmology, reinterpreting redshift, dark matter, and dark energy through \(\Tfield\)-mediated effects. Its predictions—wavelength-dependent redshift, a hotter CMB, and galaxy dynamics without dark matter—offer empirical pathways to distinguish it from \(\Lambda\)CDM, while its philosophical coherence provides a compelling alternative to the standard paradigm. Future work will refine experimental tests and explore speculative extensions, solidifying T0’s role as a unified framework bridging QM and RT \cite{pascher_perspective_2025}.
	
	\begin{acknowledgments}
		Thanks to Reinsprecht Martin Dipl.-Ing. Dr. for critical feedback.
	\end{acknowledgments}
	
	\begin{thebibliography}{99}
		\bibitem{pascher_part1_2025} J. Pascher, \href{https://github.com/jpascher/T0-Time-Mass-Duality/tree/main/2/pdf/English/QMRelTimeMassPart1ZEn.pdf}{Bridging Quantum Mechanics and Relativity through Time-Mass Duality: Part I}, April 7, 2025.
		\bibitem{pascher_lagrange_2025} J. Pascher, \href{https://github.com/jpascher/T0-Time-Mass-Duality/tree/main/2/pdf/English/MathZeitMasseLagrange.pdf}{From Time Dilation to Mass Variation}, March 29, 2025.
		\bibitem{pascher_messdifferenzen_2025} J. Pascher, \href{https://github.com/jpascher/T0-Time-Mass-Duality/tree/main/2/pdf/English/MessdifferenzenT0StandardEn.pdf}{Analysis of Measurement Differences Between T0 and \(\LCDM\)}, April 2, 2025.
		\bibitem{pascher_temp_2025} J. Pascher, \href{https://github.com/jpascher/T0-Time-Mass-Duality/tree/main/2/pdf/English/NatEinheitenAlpha1En.pdf}{Adjustment of Temperature Units and CMB Measurements}, April 2, 2025.
		\bibitem{pascher_params_2025} J. Pascher, \href{https://github.com/jpascher/T0-Time-Mass-Duality/tree/main/2/pdf/English/ZeitMasseT0ParamsEn.pdf}{Derivation of Parameters \(\kappa\), \(\alpha\), and \(\beta\)}, April 4, 2025.
		\bibitem{pascher_galaxies_2025} J. Pascher, \href{https://github.com/jpascher/T0-Time-Mass-Duality/tree/main/2/pdf/English/MassVarGalaxienEn.pdf}{Mass Variation in Galaxies}, March 30, 2025.
		\bibitem{pascher_quantum_2025} J. Pascher, \href{https://github.com/jpascher/T0-Time-Mass-Duality/tree/main/2/pdf/English/NotwendigkeitQMErweiterungEn.pdf}{Extending Quantum Mechanics and QFT}, March 27, 2025.
		\bibitem{pascher_planck_2025} J. Pascher, \href{https://github.com/jpascher/T0-Time-Mass-Duality/tree/main/2/pdf/English/JenseitsPlanckEn.pdf}{Beyond the Planck Scale}, March 24, 2025.
		\bibitem{pascher_perspective_2025} J. Pascher, \href{https://github.com/jpascher/T0-Time-Mass-Duality/tree/main/2/pdf/English/ZeitRaumPascherEn.pdf}{A New Perspective on Time and Space}, March 25, 2025.
		\bibitem{pascher_alphabeta_2025} J. Pascher, \href{https://github.com/jpascher/T0-Time-Mass-Duality/tree/main/2/pdf/English/Alpha1Beta1KonsistenzEn.pdf}{Consistency of \(\alpha = 1\) and \(\beta = 1\)}, April 5, 2025.
		\bibitem{pascher_emergente_2025} J. Pascher, \href{https://github.com/jpascher/T0-Time-Mass-Duality/tree/main/2/pdf/English/EmergentGravT0En.pdf}{Emergent Gravitation in the T0 Model}, April 1, 2025.
		\bibitem{pascher_qft_2025} J. Pascher, \href{https://github.com/jpascher/T0-Time-Mass-Duality/tree/main/2/pdf/English/QFTIntrinsischesZeitT0En.pdf}{Quantum Field Theoretical Treatment of the Intrinsic Time Field in the T0 Model}, April 8, 2025.
		\bibitem{Planck2020} Planck Collaboration, Astron. Astrophys. \textbf{641}, A6 (2020).
		\bibitem{Riess1998} A. G. Riess et al., Astron. J. \textbf{116}, 1009 (1998).
		\bibitem{Perlmutter1999} S. Perlmutter et al., Astrophys. J. \textbf{517}, 565 (1999).
		\bibitem{Fixsen2009} D. J. Fixsen, Astrophys. J. \textbf{707}, 916 (2009).
		\bibitem{McGaugh2016} S. S. McGaugh et al., Phys. Rev. Lett. \textbf{117}, 201101 (2016).
		\bibitem{Will2014} C. M. Will, Living Rev. Relativ. \textbf{17}, 4 (2014).
		\bibitem{DiValentino2021} E. Di Valentino et al., Class. Quantum Grav. \textbf{38}, 153001 (2021).
	\end{thebibliography}
	
\end{document}