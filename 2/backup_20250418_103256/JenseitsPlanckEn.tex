\documentclass[a4paper,12pt]{article}
\usepackage[utf8]{inputenc}
\usepackage[T1]{fontenc}
\usepackage{lmodern}
\usepackage[ngerman]{babel}
\usepackage{amsmath}
\usepackage{amssymb}
\usepackage{geometry}
\usepackage{tocloft}
\usepackage{tikz}
\usepackage{tcolorbox}
\usepackage[colorlinks=true, linkcolor=blue, citecolor=blue, urlcolor=blue]{hyperref}
\usepackage{siunitx}
\DeclareSIUnit{\year}{yr}
\DeclareSIUnit{\parsec}{pc}
\usepackage{fancyhdr}

\geometry{a4paper, margin=2cm}

% Headers and Footers
\pagestyle{fancy}
\fancyhf{}
\fancyhead[L]{Johann Pascher}
\fancyhead[R]{Time-Mass Duality}
\fancyfoot[C]{\thepage}
\renewcommand{\headrulewidth}{0.4pt}
\renewcommand{\footrulewidth}{0.4pt}

\renewcommand{\cftsecfont}{\color{blue}}
\renewcommand{\cftsubsecfont}{\color{blue}}
\renewcommand{\cftsecpagefont}{\color{blue}}
\renewcommand{\cftsubsecpagefont}{\color{blue}}
\setlength{\cftsecindent}{1cm}
\setlength{\cftsubsecindent}{2cm}

% Custom commands
\newcommand{\Tfield}{T(x)}
\newcommand{\DcovT}[1]{\Tfield D_\mu #1 + #1 \partial_\mu \Tfield}
\newcommand{\DhiggsT}{\Tfield (\partial_\mu + ig A_\mu) \Phi + \Phi \partial_\mu \Tfield}
\newcommand{\betaT}{\beta_{\text{T}}}
\newcommand{\alphaEM}{\alpha_{\text{EM}}}
\newcommand{\Mpl}{M_{\text{Pl}}}
\newcommand{\Tzerot}{T_0(\Tfield)}
\newcommand{\Tzero}{T_0}
\newcommand{\vecx}{\vec{x}}
\newcommand{\gammaf}{\gamma_{\text{Lorentz}}}

\title{Real Consequences of Reformulating Time and Mass in Physics: \\Beyond the Planck Scale}
\author{Johann Pascher}
\date{March 24, 2025}

\begin{document}
	
	\maketitle
	
	\begin{abstract}
		This work examines the real consequences of reformulating time and mass within the T0 model, which is based on absolute time and an intrinsic time field. Within the limits of the speed of light and the Planck mass, cosmological, quantum mechanical, and gravitational implications are analyzed, while speculative extensions beyond these limits open new perspectives on singularities and causality. The model provides testable predictions and a philosophical reinterpretation of physical reality.
	\end{abstract}
	
	\tableofcontents
	\newpage
	
	\section{Introduction}
	
	The foundations of physics have long rested on the concepts of time and mass, but what happens when we redefine them? In this work, I present the real consequences of such a reformulation as developed in the T0 model—a framework based on absolute time and variable mass, extensively described in my previous studies such as “Time-Mass Duality Theory: Derivation of Parameters” \cite{pascher_params_2025} and “Mass Variation in Galaxies” \cite{pascher_galaxies_2025}. These models challenge the conventional interpretations of special relativity and quantum mechanics by treating time as a fixed quantity and mass as dynamic. They establish boundaries at the speed of light \(c_0 \approx \SI{3e8}{\meter\per\second}\) and the Planck mass \(m_P = \sqrt{\frac{\hbar c_0}{G}} \approx \SI{2.176e-8}{\kilo\gram}\), yet venture speculatively beyond these limits to offer new perspectives on the nature of the universe.
	
	My approach begins with the idea that time is not relative, as in special relativity, but absolute—a universal constant \(T_0\)—while mass varies and is determined by an intrinsic time field \(\Tfield\). This reformulation has far-reaching implications for cosmology, quantum mechanics, and gravitation, inviting us to think beyond the traditional boundaries of physics. In the following sections, I will explore these consequences, starting with the established limits, moving to speculative extensions, and concluding with the real implications for our understanding of the physical world.
	
	\section{Establishing the Limits: \\Speed of Light and Planck Mass}
	
	The speed of light \(c_0\) and the Planck mass \(m_P\) form the cornerstones of modern physics, marking the realms where quantum gravitational effects become significant. They are linked to the Planck time \(t_P = \sqrt{\frac{\hbar G}{c_0^5}} \approx \SI{5.39e-44}{\second}\) and the Planck length \(l_P = \sqrt{\frac{\hbar G}{c_0^3}} \approx \SI{1.616e-35}{\meter}\), often regarded as fundamental limits of measurable reality. In the T0 model, these quantities remain central, but their interpretation shifts.
	
	In the standard model of special relativity, we experience time dilation (\(t' = \gamma t\)) and a constant rest mass (\(m_0\)), with relativistic mass defined as \(m_{rel} = \gamma m_0\) and energy as \(E = m_{rel} c_0^2\). The T0 model reverses this: time remains absolute (\(T_0 = \text{const.}\)), while mass is variable (\(m = \gamma m_0\)) and energy is given by \(E = \frac{\hbar}{T_0}\). A third approach, the intrinsic time model, introduces a mass-dependent time field defined as:
	
	\begin{equation}
		\Tfield = \frac{\hbar}{\max(m c^2, \omega)}
	\end{equation}
	
	This time field governs the time evolution of a system through a modified Schrödinger equation, detailed in “The Necessity of Extending Standard Quantum Mechanics” \cite{pascher_quantum_2025}:
	
	\begin{equation}
		i\hbar \Tfield \frac{\partial}{\partial t} \Psi + i\hbar \Psi \frac{\partial \Tfield}{\partial t} = \hat{H} \Psi
	\end{equation}
	
	These models offer an alternative perspective within the limits of \(c_0\) and \(m_P\), while also inviting exploration beyond these boundaries and their consequences.
	
	\section{Beyond the Limits}
	
	Despite the clearly defined limits at the speed of light and Planck mass, the T0 models open the door to speculative extensions. What happens as we approach singularities or states beyond these thresholds? In the absolute time model, the mass \(m = \frac{\hbar}{T_0 c_0^2}\) might suggest a finite energy state near a singularity rather than an infinite density as in the standard model. Similarly, the intrinsic time field at sub-Planck masses (\(\Tfield > t_P\)) leads to slower time evolution for lighter particles, offering a new perspective on physics at small scales.
	
	\begin{figure}[h]
		\centering
		\begin{tikzpicture}
			\draw[->] (0,0) -- (6,0) node[right] {Mass \(m\)};
			\draw[->] (0,0) -- (0,4) node[above] {Intrinsic Time \(T\)};
			\draw[scale=0.5, domain=0.1:10, smooth, variable=\x, blue, thick] plot ({\x}, {1/\x});
			\draw[dotted, red] (1.5,0) -- (1.5,1.5) -- (0,1.5);
			\node at (1.5,-0.3) {\(m_P\)};
			\node at (-0.3,1.5) {\(t_P\)};
			\node[blue] at (4.5,2) {\(T = \frac{\hbar}{m c^2}\)};
		\end{tikzpicture}
		\caption{Relationship between mass and intrinsic time.}
	\end{figure}
	
	The figure illustrates how \(\Tfield\) increases as mass decreases, indicating a slowdown in dynamics at extremely small masses—a concept that extends beyond the Planck scale and raises speculative questions about the nature of time and space.
	
	\section{Real Interpretive Consequences}
	
	The reformulation of time and mass in the T0 model has profound implications across various domains of physics. Starting with cosmology: instead of an expanding universe as in the standard model, the T0 model interprets redshift as an energy loss of photons, described by \(1 + z = e^{\alpha d}\), where \(\alpha \approx \SI{2.3e-18}{\per\meter}\), as derived in “Measurement Differences” \cite{pascher_messdifferenzen_2025}. The cosmic microwave background (CMB) is not seen as a remnant of an expanding universe but as a static field with mass gradients, and the origin of the universe is reinterpreted as a high-energy state without a singularity. This perspective offers testable predictions, such as deviations in the redshift-distance relationship or mass-dependent anisotropies in the CMB.
	
	In quantum mechanics and gravitation, a new connection emerges through the gradients of the intrinsic time field. The gravitational potential is modified to:
	
	\begin{equation}
		\Phi(r) = -\frac{G M}{r} + \kappa r, \quad \kappa \approx \SI{4.8e-11}{\meter\per\second\squared}
	\end{equation}
	
	This approach, detailed in “Mass Variation in Galaxies” \cite{pascher_galaxies_2025}, bridges to quantum gravitation by viewing gravity as an emergent property of the time field. For nonlocality in quantum physics, the model shows that correlations may be governed by mass variations, with lighter particles exhibiting longer intrinsic times, potentially leading to delayed correlations—an effect explored in “Dynamic Mass of Photons” \cite{pascher_photons_2025}.
	
	\section{Lagrangian Formulation}
	
	The mathematical foundation of the T0 model is provided by a total Lagrangian density, elaborated in “Mathematical Core Formulations” \cite{pascher_lagrange_2025}:
	
	\begin{equation}
		\mathcal{L}_{\text{Total}} = \mathcal{L}_{\text{Boson}} + \mathcal{L}_{\text{Fermion}} + \mathcal{L}_{\text{Higgs-T}} + \mathcal{L}_{\text{intrinsic}}, \quad \mathcal{L}_{\text{intrinsic}} = \frac{1}{2} \partial_\mu \Tfield \partial^\mu \Tfield - V(\Tfield)
	\end{equation}
	
	This formulation integrates the dynamics of the time field into existing field theories, providing a unified description of observed phenomena.
	
	\section{Effects on the Light Cone}
	
	Causality in the T0 model is redefined through mass variation. In the absolute time model, the light cone remains determined by \(c_0^2 T_0^2 - |\vec{x}|^2\), but with intrinsic time, the metric shifts to:
	
	\begin{equation}
		ds^2 = \frac{\hbar^2}{m^2} dt^2 - d\vec{x}^2
	\end{equation}
	
	This change suggests that the causal structure depends on mass, raising new questions about information transfer and causality.
	
	\section{Conclusions and Outlook}
	
	The T0 model offers an alternative perspective on physical phenomena by redefining time and mass, looking beyond the traditional boundaries of the Planck scale. It provides testable predictions—from cosmological deviations to quantum mechanical effects—and challenges us to rethink the philosophical foundations of physics. Integration with works such as “Parameter Derivations” \cite{pascher_params_2025} and “Measurement Differences” \cite{pascher_messdifferenzen_2025} demonstrates that these models are not merely speculative but constitute a coherent and verifiable alternative.
	
	\begin{thebibliography}{99}
		\bibitem{pascher_params_2025} Pascher, J. (2025). \href{https://github.com/jpascher/T0-Time-Mass-Duality/tree/main/2/pdf/English/ZeitMasseT0ParamsEn.pdf}{Time-Mass Duality Theory (T0 Model): Derivation of Parameters \(\kappa\), \(\alpha\), and \(\beta\)}. April 4, 2025.
		\bibitem{pascher_galaxies_2025} Pascher, J. (2025). \href{https://github.com/jpascher/T0-Time-Mass-Duality/tree/main/2/pdf/English/MassVarGalaxienEn.pdf}{Mass Variation in Galaxies: An Analysis in the T0 Model with Emergent Gravitation}. March 30, 2025.
		\bibitem{pascher_messdifferenzen_2025} Pascher, J. (2025). \href{https://github.com/jpascher/T0-Time-Mass-Duality/tree/main/2/pdf/English/MessdifferenzenT0StandardEn.pdf}{Compensatory and Additive Effects: An Analysis of Measurement Differences Between the T0 Model and the \(\Lambda\)CDM Standard Model}. April 2, 2025.
		\bibitem{pascher_lagrange_2025} Pascher, J. (2025). \href{https://github.com/jpascher/T0-Time-Mass-Duality/tree/main/2/pdf/English/MathZeitMasseLagrange.pdf}{From Time Dilation to Mass Variation: Mathematical Core Formulations of Time-Mass Duality Theory}. March 29, 2025.
		\bibitem{pascher_photons_2025} Pascher, J. (2025). \href{https://github.com/jpascher/T0-Time-Mass-Duality/tree/main/2/pdf/English/DynMassePhotonenNichtlokalEn.pdf}{Dynamic Mass of Photons and Its Implications for Nonlocality in the T0 Model}. March 25, 2025.
		\bibitem{pascher_quantum_2025} Pascher, J. (2025). \href{https://github.com/jpascher/T0-Time-Mass-Duality/tree/main/2/pdf/English/NotwendigkeitQMErweiterungEn.pdf}{The Necessity of Extending Standard Quantum Mechanics and Quantum Field Theory}. March 27, 2025.
	\end{thebibliography}
	
\end{document}