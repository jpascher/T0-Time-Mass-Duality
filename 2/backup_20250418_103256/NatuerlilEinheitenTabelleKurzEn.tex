\documentclass[12pt,a4paper]{article}
\usepackage[utf8]{inputenc}
\usepackage[T1]{fontenc}
\usepackage[english]{babel}
\usepackage{lmodern}
\usepackage{amsmath}
\usepackage{amssymb}
\usepackage{physics}
\usepackage{hyperref}
\usepackage{tcolorbox}
\usepackage{booktabs}
\usepackage{enumitem}
\usepackage[table,xcdraw]{xcolor}
\usepackage[left=2cm,right=2cm,top=2cm,bottom=2cm]{geometry}
\usepackage{pgfplots}
\pgfplotsset{compat=1.18}
\usepackage{graphicx}
\usepackage{float}
\usepackage{fancyhdr}
\usepackage{siunitx}
\usepackage{tikz}
\usepackage{adjustbox}
\usetikzlibrary{shapes.geometric}

% Custom Commands
\newcommand{\Tfield}{T(x)}
\newcommand{\alphaEM}{\alpha_{\text{EM}}}
\newcommand{\betaT}{\beta_{\text{T}}}
\newcommand{\Mpl}{M_{\text{Pl}}}
\newcommand{\Tzerot}{T_0(\Tfield)}
\newcommand{\e}{\mathrm{e}}
\newcommand{\alphaEMSI}{\alpha_{\text{EM,SI}}}

% Header and Footer Configuration
\pagestyle{fancy}
\fancyhf{}
\fancyhead[L]{Johann Pascher}
\fancyhead[R]{Systematic Compilation of Natural Units}
\fancyfoot[C]{\thepage}
\renewcommand{\headrulewidth}{0.4pt}
\renewcommand{\footrulewidth}{0.4pt}

\hypersetup{
	colorlinks=true,
	linkcolor=blue,
	citecolor=blue,
	urlcolor=blue,
	pdftitle={Systematic Compilation of Natural Units with Energy as the Base Unit},
	pdfauthor={Johann Pascher},
	pdfsubject={Theoretical Physics},
	pdfkeywords={T0 Model, natural units, fine-structure constant, unified unit system, time-mass duality}
}

\begin{document}
	
	\title{Hierarchical Compilation of Units in the T0 Model with Energy as the Base Unit}
	\author{Johann Pascher}
	\date{13th April 2025}
	\maketitle
	
	\section*{Part 1: Overview of Units and Scales}
	
	\subsection*{Level 1: Primary Dimensional Constants (Value = 1)}
	\begin{itemize}[itemsep=0.5em]
		\item \textbf{Planck Constant} ($\hbar = 1$)
		\item \textbf{Speed of Light} ($c = 1$)
		\item \textbf{Gravitational Constant} ($G = 1$)
		\item \textbf{Boltzmann Constant} ($k_B = 1$)
	\end{itemize}
	
	\subsection*{Level 2: Dimensionless Coupling Constants (Value = 1)}
	\begin{itemize}[itemsep=0.5em]
		\item \textbf{Fine-Structure Constant} (\(\alphaEM = 1\)) \\
		Corresponds to the SI value \(\alphaEMSI \approx \frac{1}{137.036}\).
		\item \textbf{Wien Constant} (\(\alpha_W = 1\)) \\
		Corresponds to the SI value \(\alpha_{W,\mathrm{SI}} \approx 2.82\).
		\item \textbf{T0 Parameter} (\(\betaT = 1\)) \\
		Corresponds to the SI value \(\beta_{T,\mathrm{SI}} \approx 0.008\).
	\end{itemize}
	
	\subsection*{Level 2.5: Derived Electromagnetic Constants}
	\begin{itemize}[itemsep=0.5em]
		\item \textbf{Vacuum Magnetic Permeability} (\(\mu_0 = 1\))
		\item \textbf{Vacuum Permittivity} (\(\varepsilon_0 = 1\))
		\item \textbf{Vacuum Impedance} (\(Z_0 = 1\))
		\item \textbf{Elementary Charge} (\(e = \sqrt{4\pi}\))
		\item[] \textit{Note: For $\alphaEM = e^2/(4\pi\varepsilon_0\hbar c) = 1$ and $\varepsilon_0 = \hbar = c = 1$, it follows that $e = \sqrt{4\pi} \approx 3.5$}
		\item \textbf{Planck Pressure} (\(p_P = 1\))
		\item \textbf{Planck Force} (\(F_P = 1\))
		\item \textbf{Einstein-Hilbert Action}
		\[
		S_{\mathrm{EH}} = \frac{1}{16\pi} \int R \sqrt{-g} \, \mathrm{d}^4x
		\]
	\end{itemize}
	
	\subsection*{Explanation of the Einstein-Hilbert Action}
	
	The Einstein-Hilbert action holds a special position in the T0 model, as it describes gravitation as a geometric property of spacetime. In natural units with $G = c = 1$, the Einstein-Hilbert action simplifies to:
	
	\[
	S_{\mathrm{EH}} = \frac{1}{16\pi}\int R\sqrt{-g}d^4x
	\]
	
	where:
	\begin{itemize}
		\item $R$ is the Ricci scalar (curvature scalar of spacetime)
		\item $g$ is the determinant of the metric tensor $g_{\mu\nu}$
		\item $d^4x$ is the four-dimensional spacetime volume element
	\end{itemize}
	
	In the T0 model, gravitation is not considered a fundamental interaction but an emergent phenomenon from the intrinsic time field $T(x)$. The Einstein-Hilbert action forms the mathematical bridge between the conventional geometric description of gravitation (General Relativity) and the T0 representation with emergent gravitation.
%----
The modified gravitational potential in the T0 model:
\[
\Phi(r) = -\frac{M}{r} + \kappa r
\]
in natural units where \(G = 1\) (empirical SI value: \(\kappa \approx 4.8 \times 10^{-11} \, \text{m/s}^2\)).
%----	

	is directly related to the curvature of spacetime, captured in the Einstein-Hilbert action through the Ricci scalar $R$. The linear term $\kappa r$, which supplements Newtonian gravitation in the T0 model, corresponds to a modified spacetime geometry and manifests in the Einstein-Hilbert action through modified field equations.
\subsection*{Equivalence between Einstein-Hilbert Action and Time Field Derivation}
The T0 model offers two complementary descriptions of gravitation: The formal Einstein-Hilbert action $S_{\mathrm{EH}} = \frac{1}{16\pi} \int (R - 2\kappa) \sqrt{-g} \, d^4x$ and the more fundamental time field derivation $\Phi(\vec{x}) = -\ln\left(\frac{\Tfield}{\Tfield_0}\right)$. Both lead to the identical gravitational potential $\Phi(r) = -\frac{M}{r} + \kappa r$ in natural units where \(G = 1\) (empirical SI value: \(\kappa \approx 4.8 \times 10^{-11} \, \text{m/s}^2\)).


	
	\subsection*{Level 3: Derived Constants with Simple Values}
	\begin{itemize}[itemsep=0.5em]
		\item \textbf{Compton Wavelength of the Electron} (\(\lambda_{C,e} = \frac{1}{m_e}\))
		\item \textbf{Rydberg Constant} (\(R_\infty = \frac{\alphaEM^2 \cdot m_e}{2} = \frac{m_e}{2}\))
		\item[] \textit{Derived from the relation $R_\infty = m_e\cdot e^4/(8\varepsilon_0^2h^3c)$ with $\alphaEM = 1$}
		\item \textbf{Josephson Constant} (\(K_J = \frac{2e}{h} = \frac{2\sqrt{4\pi}}{2\pi} = \sqrt{\frac{4}{\pi}} \approx 1.13\))
		\item[] \textit{With $h = 2\pi$ and $e = \sqrt{4\pi}$}
		\item \textbf{von Klitzing Constant} (\(R_K = \frac{h}{e^2} = \frac{2\pi}{4\pi} = \frac{1}{2}\))
		\item[] \textit{With $h = 2\pi$ and $e^2 = 4\pi$}
		\item \textbf{Schwinger Limit} (\(E_S = \frac{m_e^2c^3}{e\sqrt{\hbar}} = m_e^2\))
		\item[] \textit{With $c = \hbar = 1$ and $e = \sqrt{4\pi}$}
		\item \textbf{Stefan-Boltzmann Constant} (\(\sigma = \frac{\pi^2k_B^4}{60\hbar^3c^2} = \frac{\pi^2}{60}\))
		\item[] \textit{With $\hbar = c = k_B = 1$}
		\item \textbf{Hawking Temperature} (\(T_H = \frac{\hbar c^3}{8\pi GMk_B} = \frac{1}{8\pi M}\))
		\item[] \textit{With $\hbar = c = G = k_B = 1$}
		\item \textbf{Bekenstein-Hawking Entropy} (\(S_{\mathrm{BH}} = \frac{4\pi GM^2}{\hbar c} = 4\pi M^2\))
		\item[] \textit{With $\hbar = c = G = 1$}
	\end{itemize}


	
	\subsection*{Planck Units in the T0 Model}
	
	In the T0 model, all Planck units are set to the value 1, making them natural reference points for physical quantities:
	
	\begin{table}[H]
		\centering
		\begin{adjustbox}{width=0.95\textwidth}
			\begin{tabular}{lcccl}
				\toprule
				\textbf{Planck Unit} & \textbf{Symbol} & \textbf{Definition in SI System} & \textbf{Value in T0 Model} & \textbf{Significance} \\
				\midrule
				Planck Length & \(l_P\) & \(\sqrt{\frac{\hbar G}{c^3}}\) & 1 & Fundamental length unit \\
				Planck Time & \(t_P\) & \(\sqrt{\frac{\hbar G}{c^5}}\) & 1 & Fundamental time unit \\
				Planck Mass & \(m_P\) & \(\sqrt{\frac{\hbar c}{G}}\) & 1 & Fundamental mass unit \\
				Planck Energy & \(E_P\) & \(\sqrt{\frac{\hbar c^5}{G}}\) & 1 & Fundamental energy unit \\
				Planck Temperature & \(T_P\) & \(\frac{\sqrt{\frac{\hbar c^5}{G}}}{k_B}\) & 1 & Fundamental temperature unit \\
				Planck Pressure & \(p_P\) & \(\frac{c^7}{\hbar G^2}\) & 1 & Fundamental pressure unit \\
				Planck Density & \(\rho_P\) & \(\frac{c^5}{\hbar G^2}\) & 1 & Fundamental density unit \\
				Planck Charge & \(q_P\) & \(\sqrt{4\pi \varepsilon_0 \hbar c}\) & 1 & Fundamental charge unit \\
				\bottomrule
			\end{tabular}
		\end{adjustbox}
		\caption{Planck Units in the T0 Model}
		\label{tab:planck_units}
	\end{table}
	
	\subsection*{Length Scales with Hierarchical Relationships}
	
	\begin{table}[H]
		\centering
		\begin{adjustbox}{width=\textwidth}
			\begin{tabular}{lccc}
				\toprule
				\textbf{Physical Structure} & \textbf{With \(l_P = 1\)} & \textbf{With \(r_0 = 1\)} & \textbf{Hierarchical Relationship} \\
				\midrule
				Planck Length (\(l_P\)) & 1 & \(\frac{l_P}{r_0} = \frac{1}{\xi} \approx 7519\) & Base unit \\
				T0 Length (\(r_0\)) & \(\frac{r_0}{l_P} = \xi \approx 1.33 \times 10^{-4}\) & 1 & \(\xi \cdot l_P = \frac{\lambda_h}{32\pi^3} \cdot l_P\) \\
				Strong Scale & \(\sim 10^{-19}\) & \(\sim 10^{-15}\) & \(\sim \alpha_s \cdot \lambda_{C,h}\) \\
				Higgs Length (\(\lambda_{C,h}\)) & \(\sim 1.6 \times 10^{-20}\) & \(\sim 1.2 \times 10^{-16}\) & \(\frac{m_P}{m_h} \cdot l_P\) \\
				Proton Radius & \(\sim 5.2 \times 10^{-20}\) & \(\sim 3.9 \times 10^{-16}\) & \(\sim \frac{\alpha_s}{2\pi} \cdot \lambda_{C,p}\) \\
				Electron Radius (\(r_e\)) & \(\sim 2.4 \times 10^{-23}\) & \(\sim 1.8 \times 10^{-19}\) & \(\frac{\alphaEMSI}{2\pi} \cdot \lambda_{C,e}\) \\
				Compton Length (\(\lambda_{C,e}\)) & \(\sim 2.1 \times 10^{-23}\) & \(\sim 1.6 \times 10^{-19}\) & \(\frac{m_P}{m_e} \cdot l_P\) \\
				Bohr Radius (\(a_0\)) & \(\sim 4.2 \times 10^{-23}\) & \(\sim 3.2 \times 10^{-19}\) & \(\frac{\lambda_{C,e}}{\alphaEMSI} = \frac{m_P}{\alphaEMSI \cdot m_e} \cdot l_P\) \\
				DNA Width & \(\sim 1.2 \times 10^{-26}\) & \(\sim 9.0 \times 10^{-23}\) & \(\sim \lambda_{C,e} \cdot \frac{m_e}{m_{\mathrm{DNA}}}\) \\
				Cell & \(\sim 6.2 \times 10^{-30}\) & \(\sim 4.7 \times 10^{-26}\) & \(\sim 10^7 \cdot \text{DNA Width}\) \\
				Human & \(\sim 6.2 \times 10^{-35}\) & \(\sim 4.7 \times 10^{-31}\) & \(\sim 10^5 \cdot \text{Cell}\) \\
				Earth Radius & \(\sim 3.9 \times 10^{-41}\) & \(\sim 2.9 \times 10^{-37}\) & \(\sim \left(\frac{m_P}{m_{\mathrm{Earth}}}\right)^2 \cdot l_P\) \\
				Sun Radius & \(\sim 4.3 \times 10^{-43}\) & \(\sim 3.2 \times 10^{-39}\) & \(\sim \left(\frac{m_P}{m_{\mathrm{Sun}}}\right)^2 \cdot l_P\) \\
				Solar System & \(\sim 6.2 \times 10^{-47}\) & \(\sim 4.7 \times 10^{-43}\) & \(\sim \alpha_G^{-1/2} \cdot \text{Sun Radius}\) \\
				Galaxy & \(\sim 6.2 \times 10^{-56}\) & \(\sim 4.7 \times 10^{-52}\) & \(\sim \left(\frac{m_P}{m_{\mathrm{Galaxy}}}\right)^2 \cdot l_P\) \\
				Cluster & \(\sim 6.2 \times 10^{-58}\) & \(\sim 4.7 \times 10^{-54}\) & \(\sim 10^2 \cdot \text{Galaxy}\) \\
				Horizon (\(d_H\)) & \(\sim 5.4 \times 10^{61}\) & \(\sim 4.1 \times 10^{65}\) & \(\sim \frac{1}{H_0} = \frac{c}{H_0}\) \\
				Correlation Length (\(L_T\)) & \(\sim 3.9 \times 10^{62}\) & \(\sim 2.9 \times 10^{66}\) & \(\sim \betaT^{-1/4} \cdot \xi^{-1/2} \cdot l_P\) \\
				\bottomrule
			\end{tabular}
		\end{adjustbox}
		\caption{Length Scales with Hierarchical Relationships}
		\label{tab:length_scales}
	\end{table}
	
	\subsection*{Quantized Length Scales and Forbidden Zones}
	
	The preferred length scales in the T0 model follow the pattern:
	\[
	L_n = l_P \times \prod \alpha_i^{n_i}
	\]
	where:
	\begin{itemize}
		\item \(\alpha_i = \text{dimensionless constants} \, (\alphaEM, \betaT, \xi)\)
		\item \(n_i = \text{integer or rational exponents}\)
	\end{itemize}
	
	\subsection*{Biological Anomalies in the Length Scale Hierarchy}
	
	A remarkable discovery in the T0 model is that biological structures preferentially exist in "forbidden zones" of the length scale:
	
	\begin{table}[H]
		\centering
		\begin{tabular}{lccc}
			\toprule
			\textbf{Biological Structure} & \textbf{Typical Size} & \textbf{Ratio to \(l_P\)} & \textbf{Position} \\
			\midrule
			DNA Diameter & \(\sim 2 \times 10^{-9} \, \text{m}\) & \(\sim 10^{-26}\) & Forbidden Zone \\
			Protein & \(\sim 10^{-8} \, \text{m}\) & \(\sim 10^{-27}\) & Forbidden Zone \\
			Bacterium & \(\sim 10^{-6} \, \text{m}\) & \(\sim 10^{-29}\) & Forbidden Zone \\
			Typical Cell & \(\sim 10^{-5} \, \text{m}\) & \(\sim 10^{-30}\) & Forbidden Zone \\
			Multicellular Organism & \(\sim 10^{-3} - 10^{0} \, \text{m}\) & \(\sim 10^{-32} - 10^{-35}\) & Forbidden Zone \\
			\bottomrule
		\end{tabular}
		\caption{Biological Structures in Forbidden Zones}
		\label{tab:biological_anomalies}
	\end{table}
	
	These "forbidden zones" lie between the preferred quantized length scales and form gaps of often several orders of magnitude:
	\begin{itemize}
		\item Between \(10^{-30} \, \text{m}\) and \(10^{-23} \, \text{m}\) (between T0 length and Compton wavelength)
		\item Between \(10^{-9} \, \text{m}\) and \(10^{-6} \, \text{m}\) (between molecular and cellular levels)
		\item Between \(10^{-3} \, \text{m}\) and \(10^{0} \, \text{m}\) (macroscopic range, where biological organisms dominate)
	\end{itemize}
	
	This anomaly can be explained by special stabilization mechanisms that allow biological systems to exist in these forbidden zones:
	
	\begin{enumerate}
		\item \textbf{Information-Based Stabilization}: Biological structures utilize genetic and epigenetic information.
		\item \textbf{Topological Stabilization}: Biological systems often exhibit topologically protected configurations.
		\item \textbf{Dynamic Stabilization}: Operating far from thermodynamic equilibrium.
	\end{enumerate}
	
	In the T0 model, this is formalized through modified time field equations:
	\[
	\nabla^2 \Tfield_{\mathrm{bio}} \approx -\frac{\rho}{\Tfield^2} + \delta_{\mathrm{bio}}(x,t)
	\]
	where \(\delta_{\mathrm{bio}}\) represents a biological correction term that enables stability in forbidden zones.
	
	\section*{Part 2: Detailed Explanations and Derivations}
	
	\subsection*{Dimensional Analysis and Derivation of the Einstein-Hilbert Action in the T0 Model}
	
	\subsubsection*{1. Original Form in SI Units}
	
	In General Relativity, the Einstein-Hilbert action in SI units is:
	
	\[
	S_{\mathrm{EH}} = \frac{c^4}{16\pi G} \int R \sqrt{-g} \, d^4x
	\]
	
	where:
	\begin{itemize}
		\item $c$ is the speed of light
		\item $G$ is the gravitational constant
		\item $R$ is the Ricci scalar with dimension $[L^{-2}]$ (curvature)
		\item $\sqrt{-g} \, d^4x$ is the spacetime volume element with dimension $[L^4]$
		\item $\frac{c^4}{16\pi G}$ is the prefactor with dimension $[L^{-1} M]$
	\end{itemize}
	
	The dimension of the entire action is:
	\[
	[L^{-2}] \cdot [L^4] \cdot [L^{-1} M] = [L M]
	\]
	
	which corresponds to the dimension of energy $\times$ time and, in SI units, matches the physical dimension of an action (e.g., $\hbar$).
	
	\subsubsection*{2. Transition to the T0 Model with Natural Units}
	
	In the T0 model, the fundamental assumptions are:
	\begin{itemize}
		\item $\hbar = 1$ (normalization of the action)
		\item $c = 1$ (unifies space and time)
		\item $G = 1$ (unifies gravitational physics with other interactions)
	\end{itemize}
	
	With energy $[E]$ as the base unit, the dimensions are:
	\begin{itemize}
		\item Length: $[L] = [E^{-1}]$
		\item Time: $[T] = [E^{-1}]$
		\item Mass: $[M] = [E]$
	\end{itemize}
	
	Thus, the Ricci scalar $R$ has the dimension $[L^{-2}] = [E^2]$.
	
	The volume element $\sqrt{-g} \, d^4x$ has the dimension $[L^4] = [E^{-4}]$.
	
	The integrand $R\sqrt{-g} \, d^4x$ thus has the dimension $[E^2] \cdot [E^{-4}] = [E^{-2}]$.
	
	\subsubsection*{3. The Prefactor in the Natural System}
	
	In the T0 model, the prefactor $\frac{c^4}{16\pi G}$ transforms to:
	\begin{itemize}
		\item In SI units, it has the dimension $[L^{-1} M]$
		\item This corresponds in natural units to $[E^{-1} \cdot E] = [E^0] = 1$
	\end{itemize}
	
	The numerical value becomes $\frac{1}{16\pi}$ due to the settings $c = G = 1$.
	
	The action takes the form:
	\[
	S_{\mathrm{EH}} = \frac{1}{16\pi} \int R \sqrt{-g} \, d^4x
	\]
	
	The dimension of this action in the T0 model is:
	\[
	[1] \cdot [E^{-2}] \cdot [E^2] = [E^0] = 1
	\]
	
	\subsubsection*{4. Field Equations in the T0 Model}
	
	Variation of the Einstein-Hilbert action leads to the field equations:
	\[
	R_{\mu\nu} - \frac{1}{2}Rg_{\mu\nu} = 8\pi T_{\mu\nu}
	\]
	
	where the factor $8\pi$ directly results from the prefactor $\frac{1}{16\pi}$ of the action. The energy-momentum tensor $T_{\mu\nu}$ in the T0 model has the dimension $[E^2]$ (energy per volume).
	
	\subsubsection*{5. Connection to the Modified Gravitational Potential}
	
	The connection between the modified potential $\Phi(r) = -\frac{GM}{r} + \kappa r$ and the Einstein-Hilbert action arises through the following derivation:
	
	\begin{enumerate}
		\item The modified potential can be represented as a solution to a modified Poisson equation:
		\[
		\nabla^2\Phi = 4\pi G\rho - 2\kappa
		\]
		
		\item In General Relativity, such a modification corresponds to an energy-momentum tensor that includes a term equivalent to a cosmological constant:
		\[
		T_{\mu\nu} = T_{\mu\nu}(\text{matter}) + \Lambda_{\text{eff}} \cdot g_{\mu\nu}
		\]
		where $\Lambda_{\text{eff}} = \frac{\kappa}{G}$ represents an effective cosmological constant.
		
		\item This additional term in the Einstein equation corresponds to an additional term in the Einstein-Hilbert action:
		\[
		S_{\mathrm{EH}} = \frac{1}{16\pi G}\int(R - 2\Lambda_{\text{eff}})\sqrt{-g}d^4x
		\]
		
		\item In natural units with $G = 1$, this becomes:
		\[
		S_{\mathrm{EH}} = \frac{1}{16\pi}\int(R - 2\kappa)\sqrt{-g}d^4x
		\]
		
		\item Variation of this modified action leads to the field equations:
		\[
		R_{\mu\nu} - \frac{1}{2}Rg_{\mu\nu} + \kappa g_{\mu\nu} = 8\pi T_{\mu\nu}
		\]
		
		\item In the weak-field approximation, this yields exactly the modified potential:
		\[
		ds^2 = -(1+2\Phi)dt^2 + (1-2\Phi)(dx^2 + dy^2 + dz^2)
		\]
		with $\Phi(r) = -\frac{M}{r} + \frac{\kappa r}{2}$ (with $G = 1$).
	\end{enumerate}
	
	\subsection*{Connection to Observed Dark Energy}

%---
The linear term $\kappa r$ in the gravitational potential corresponds to an effective cosmological constant $\Lambda_{\text{eff}} = \kappa$ in natural units where \(G = 1\). This has important implications for observed dark energy:
\begin{enumerate}
	\item The measured energy density of dark energy is approximately $\rho_\Lambda \approx 10^{-123}$ in Planck units.
	
	\item In the T0 model, this value arises naturally as a consequence of the parameter $\kappa \approx \frac{1}{L_T}$ with \(L_T \approx 3.9 \times 10^{62}\) (empirical SI value: \(\kappa \approx 4.8 \times 10^{-11} \, \text{m/s}^2\)):
	\[
	\rho_\Lambda = \frac{\Lambda_{\text{eff}}}{8\pi} = \frac{\kappa}{8\pi} \approx 10^{-123} m_P^4
	\]
	
	\item This agreement naturally resolves the cosmological constant problem, as $\kappa$ does not require fine-tuning but arises from the fundamental structure of the T0 model:
	\[
	\kappa = \beta_T \cdot \frac{1}{L_T}
	\]
\end{enumerate}
%---

	
	This formulation explains both observed galaxy rotation curves and cosmic acceleration without introducing additional dark components and enables direct experimental comparison with MOND (Modified Newtonian Dynamics) and f(R) gravity theories.
	
	\subsection*{Derivation of Gravitation in the Natural System of the T0 Model}
	
	In the T0 model, gravitation is not postulated as a fundamental property but derived directly from the intrinsic time field $T(x)$:
	
	\begin{enumerate}
		\item \textbf{Fundamental Derivation:} Gravitation arises from gradients of the intrinsic time field:
		\[
		\nabla T(x) = -\frac{\hbar}{m^2c^2} \cdot \nabla m
		\]
		
		\item \textbf{Connection to the Einstein-Hilbert Action:} In the natural system with $\hbar = c = G = 1$, it can be shown that the effective gravitational potential $\Phi(x)$ is linked to the time field by:
		\[
		\Phi(x) = -\ln\left(\frac{T(x)}{T_0}\right)
		\]
		where $T_0$ is a reference value of the time field.
		
		\item \textbf{Emergent Field Equations:} The dynamics of the time field lead to modified field equations equivalent to a modified Einstein-Hilbert action:
		\[
		\nabla^2T(x) \approx -\frac{\rho}{T(x)^2}
		\]
		This equation is equivalent to a modified Poisson equation in the weak-field limit, producing the linear term $\kappa r$.
		
		\item \textbf{Unit Relationship:} In the natural unit system of the T0 model, all terms in the Einstein-Hilbert action have the dimension $[E^0]$, i.e., dimensionless. This results from:
		\begin{itemize}
			\item Ricci scalar $R$: $[E^2]$
			\item Determinant $\sqrt{-g}$: dimensionless
			\item Volume element $d^4x$: $[E^{-4}]$
			\item Prefactor $\frac{1}{16\pi}$: dimensionless
		\end{itemize}
	\end{enumerate}
	
	The uniqueness of the T0 model lies in the fact that the Einstein-Hilbert action and General Relativity appear as effective descriptions of gravitation, while the more fundamental description is provided by the intrinsic time field. This enables a unified treatment of gravitation with other interactions and explains observed anomalies in galaxy dynamics without invoking dark matter.
	
	\subsection*{Comparison with Established Gravitation Theories}
	
	The T0 model offers an alternative to established gravitation theories and can be directly compared with them:
	
	\begin{table}[H]
		\centering
		
	%----
	\begin{tabular}{p{3cm}p{3cm}p{4cm}p{4cm}}
		\toprule
		\textbf{Theory} & \textbf{Principle} & \textbf{Modified Potential} & \textbf{Comparison with T0} \\
		\midrule
		Newtonian Gravitation & Force between masses & $\Phi(r) = -\frac{M}{r}$ & Special case of T0 for $\kappa=0$ \\
		General Relativity & Spacetime curvature & Schwarzschild solution & Phenomenologically equivalent in weak fields \\
		MOND & Modified dynamics at low acceleration & $\Phi(r)$ satisfies: $\nabla^2\Phi = 4\pi \rho\cdot\mu(\frac{\nabla\Phi}{a_0})$ & T0 provides a more fundamental basis for MOND effects \\
		f(R) Theories & Modified gravitational action & Depends on specific f(R) function & T0 corresponds to f(R) = R - 2$\kappa$ for weak fields \\
		T0 Model & Emergent gravitation from time field & $\Phi(r) = -\frac{M}{r} + \kappa r$ & Unifies quantum mechanics and gravitation \\
		\bottomrule
	\end{tabular}
	%----	

		\caption{Comparison of the T0 Model with Established Gravitation Theories}
		\label{tab:theory_comparison}
	\end{table}
	
	The T0 model offers the following advantages over these theories:
	
	\begin{enumerate}
		\item \textbf{Unified Treatment of Quantum and Macroscopic Physics} through the intrinsic time field $T(x)$
		\item \textbf{Natural Explanation for Galaxy Dynamics} without assuming dark matter
		\item \textbf{Solution to the Cosmological Constant Problem} by deriving $\kappa$ from fundamental parameters
		\item \textbf{Mathematical Consistency} with quantum field theory and the Standard Model through modified Lagrangian densities
		\item \textbf{Testable Predictions} for deviations from the 1/r potential at various scales
	\end{enumerate}
	
	Experimental tests to distinguish between these theories include:
	\begin{itemize}
		\item Precision measurements of planetary perihelion precession
		\item Gravitational lensing effects in distant galaxies
		\item Satellite measurements of the Pioneer anomaly
		\item Observations of galaxy rotation curves across different morphologies
	\end{itemize}
	
	\subsection*{Practical Equivalents in Energy Units}
	
	\textbf{Important Note}: The energy unit "electronvolt" (abbreviated as "eV") must not be confused with the SI unit "volt" (abbreviated as "V"). In the T0 model with natural units, the electronvolt is used as the fundamental energy unit, from which other units are derived.
	
	\begin{itemize}
		\item \textbf{Length:} (eV)$^{-1}$, (GeV)$^{-1}$, (TeV)$^{-1}$
		\item \textbf{Time:} (eV)$^{-1}$, (GeV)$^{-1}$, (TeV)$^{-1}$
		\item \textbf{Mass/Energy:} eV, MeV, GeV, TeV
		\item \textbf{Temperature:} eV, MeV
		\item \textbf{Momentum:} eV/c, GeV/c (where $c=1$ in natural units)
		\item \textbf{Cross Section:} (GeV)$^{-2}$, mb, pb, fb
		\item \textbf{Decay Rate:} eV, MeV
	\end{itemize}
	
	In the T0 model, length scales are often expressed as inverse energies, reflecting the fundamental relationship between energy and length in natural units (length $\sim$ 1/energy).
	
	\subsection*{Conversion of Common SI Units to T0 Model Units}
	
	Common SI units can be reduced to energy as the base unit in the T0 model. This allows all physical quantities to be represented in a unified system:
	
	\begin{table}[H]
		\centering
		\begin{adjustbox}{width=\textwidth}
			\begin{tabular}{lcccc}
				\toprule
				\textbf{SI Unit} & \textbf{Dimension in SI System} & \textbf{T0 Model Equivalent} & \textbf{Conversion Relationship} & \textbf{Typical Measurement Accuracy} \\
				\midrule
				Meter (m) & $[L]$ & $[E^{-1}]$ & 1 m $\leftrightarrow$ (197 MeV)$^{-1}$ & $<$ 0.001\% \\
				Second (s) & $[T]$ & $[E^{-1}]$ & 1 s $\leftrightarrow$ (6.58 $\times$ 10$^{-22}$ MeV)$^{-1}$ & $<$ 0.00001\% \\
				Kilogram (kg) & $[M]$ & $[E]$ & 1 kg $\leftrightarrow$ 5.61 $\times$ 10$^{26}$ MeV & $<$ 0.001\% \\
				Ampere (A) & $[I]$ & $[E]$ & 1 A $\leftrightarrow$ charge per time $\leftrightarrow$ $[E^2]$ & $<$ 0.005\% \\
				Kelvin (K) & $[\Theta]$ & $[E]$ & 1 K $\leftrightarrow$ 8.62 $\times$ 10$^{-5}$ eV & $<$ 0.01\% \\
				Volt (V) & $[ML^2T^{-3}I^{-1}]$ & $[E]$ & 1 V $\leftrightarrow$ 1 eV/e (with $e = \sqrt{4\pi}$) & $<$ 0.0001\% \\
				Tesla (T) & $[MT^{-2}I^{-1}]$ & $[E^2]$ & 1 T $\leftrightarrow$ energy per area & $<$ 0.01\% \\
				Pascal (Pa) & $[ML^{-1}T^{-2}]$ & $[E^4]$ & 1 Pa $\leftrightarrow$ energy per volume & $<$ 0.005\% \\
				Watt (W) & $[ML^2T^{-3}]$ & $[E^2]$ & 1 W $\leftrightarrow$ energy per time & $<$ 0.001\% \\
				Coulomb (C) & $[TI]$ & $[1]$ & 1 C $\leftrightarrow$ e/$\sqrt{4\pi}$ & $<$ 0.0001\% \\
				Ohm ($\Omega$) & $[ML^2T^{-3}I^{-2}]$ & $[E^{-1}]$ & 1 $\Omega$ $\leftrightarrow$ h/e$^2$ = 1/2 (with h=2$\pi$, e=$\sqrt{4\pi}$) & $<$ 0.0000001\% \\
				Farad (F) & $[M^{-1}L^{-2}T^4I^2]$ & $[E^{-1}]$ & 1 F $\leftrightarrow$ inverse energy & $<$ 0.01\% \\
				Henry (H) & $[ML^2T^{-2}I^{-2}]$ & $[E^{-1}]$ & 1 H $\leftrightarrow$ inverse energy & $<$ 0.01\% \\
				\bottomrule
			\end{tabular}
		\end{adjustbox}
		\caption{Conversion of SI Units to T0 Model Units}
		\label{tab:conversion}
	\end{table}
	
	\subsection*{Special Role of Electric Charge (Coulomb)}
	
	The Coulomb unit holds a special position in the T0 model, as it provides the most direct connection to the electromagnetic constants $\mu_0$ and $\varepsilon_0$. With $\alphaEM = \frac{e^2}{4\pi\varepsilon_0\hbar c} = 1$ in the T0 model, it follows:
	
	\[
	e^2 = 4\pi\varepsilon_0\hbar c
	\]
	
	Since $\hbar = c = \varepsilon_0 = 1$ in the T0 model, we get:
	\[
	e^2 = 4\pi
	\]
	\[
	e = \sqrt{4\pi} \approx 3.5
	\]
	
	With $\varepsilon_0\mu_0c^2 = 1$ and $c = 1$, it further follows:
	\[
	\varepsilon_0\mu_0 = 1
	\]
	
	These relationships give electric charge a special significance in the T0 model. The value $e = \sqrt{4\pi}$ is a natural consequence of the normalization $\alphaEM = 1$ and is consistent with the Maxwell equations in their simplest form.
	
	The effects of the normalization $e = \sqrt{4\pi}$ are:
	\begin{enumerate}
		\item Electric charges are measured in units of $\sqrt{4\pi}$
		\item Electric and magnetic fields can be expressed in pure energy units
		\item The Maxwell equations take their most elegant form
	\end{enumerate}
	
	This natural representation reveals the deep connection between electromagnetism and the fundamental energy structure of the universe.
	
	\subsection*{Concluding Remarks on the Completeness and Accuracy of the T0 Model}
	
	A central strength of the T0 model is that \textbf{all SI units} can be fully and precisely mapped in this system. It is not an approximate or simplified system but a more fundamental representation of physical reality.
	
	The apparent "deviations" between measurements in the SI system and the theoretical predictions of the T0 model are not errors of the natural unit system but reflect inaccuracies in measurement evaluation and the underlying metrology of the SI system. These deviations are in most cases extremely small:
	
	\begin{table}[H]
		\centering
		\begin{adjustbox}{width=0.95\textwidth}
			\begin{tabular}{lcc}
				\toprule
				\textbf{Domain} & \textbf{Typical Deviation} & \textbf{Note} \\
				\midrule
				Atomic Scale & $\sim10^{-9}$ to $10^{-8}$ & Extremely high agreement (0.0000001\% - 0.000001\%) \\
				Nuclear Scale & $\sim10^{-7}$ to $10^{-6}$ & Very high agreement (0.00001\% - 0.0001\%) \\
				Macroscopic Scale & $\sim10^{-5}$ to $10^{-4}$ & High agreement (0.001\% - 0.01\%) \\
				Astronomical Scale & $\sim10^{-3}$ to $10^{-2}$ & Good agreement (0.1\% - 1\%) \\
				Cosmological Scale & $\sim10^{-2}$ to $10^{-1}$ & Larger deviations (1\% - 10\%) \\
				\bottomrule
			\end{tabular}
		\end{adjustbox}
		\caption{Deviations between SI System and T0 Model}
		\label{tab:deviations}
	\end{table}
	
	The larger deviations in cosmological dimensions are not due to deficiencies in the T0 model but to fundamental challenges in cosmological measurement techniques and the interpretation of observational data in the context of the conventional cosmological standard model.
	
	The T0 model, with its system of natural units, not only provides a mathematically more elegant and physically more fundamental framework but also enables new insights into the structure of the universe that remain hidden in the SI system. The quantized structure of length scales, the special role of biological systems, and the unified treatment of all interactions are aspects that fully unfold their significance only in the T0 model.
	
\end{document}