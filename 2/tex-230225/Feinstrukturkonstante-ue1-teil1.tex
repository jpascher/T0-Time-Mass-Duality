\documentclass{article}
\usepackage[utf8]{inputenc}
\usepackage{amsmath}
\usepackage{amssymb}
\usepackage{geometry}
\usepackage{hyperref}

\title{Die Feinstrukturkonstante: Verschiedene Darstellungen und Zusammenhänge}
\author{Johann Pascher}
\date{21-03-2025}

\begin{document}
	
	\maketitle
	
	\section{Einführung zur Feinstrukturkonstante}
	
	Die Feinstrukturkonstante ($\alpha$) ist eine dimensionslose physikalische Konstante, die eine fundamentale Rolle in der Quantenelektrodynamik spielt. Sie beschreibt die Stärke der elektromagnetischen Wechselwirkung zwischen Elementarteilchen. In ihrer bekanntesten Form lautet die Formel:
	
	$$\alpha = \frac{e^2}{4\pi\varepsilon_0\hbar c} \approx \frac{1}{137.035999}$$
	
	\section{Unterschiede zwischen der Fine-Ungleichung und der Feinstrukturkonstante}
	
	\subsection{Fine-Ungleichung}
	\begin{itemize}
		\item Bezieht sich auf lokale verborgene Variablen und Bell-Ungleichungen
		\item Untersucht, ob eine klassische Theorie die Quantenmechanik ersetzen kann
		\item Zeigt, dass Quantenverschränkung nicht durch klassische Wahrscheinlichkeiten beschrieben werden kann
	\end{itemize}
	
	\subsection{Feinstrukturkonstante ($\alpha$)}
	\begin{itemize}
		\item Eine fundamentale Naturkonstante der Quantenfeldtheorie
		\item Beschreibt die Stärke der elektromagnetischen Wechselwirkung
		\item Bestimmt z. B. den Energieabstand feinstrukturaufgespaltener Spektrallinien in Atomen
	\end{itemize}
	
	\subsection{Möglicher Zusammenhang}
	Obwohl die Fine-Ungleichung und die Feinstrukturkonstante grundsätzlich nichts miteinander zu tun haben, gibt es eine interessante Verbindung über die Quantenmechanik und Feldtheorie:
	
	\begin{itemize}
		\item Die Feinstrukturkonstante spielt eine zentrale Rolle in der Quantenelektrodynamik (QED), die eine nicht-lokale Struktur besitzt
		\item Die Verletzung der Fine-Ungleichung deutet darauf hin, dass Quantentheorien nicht-lokal sind
		\item Die Feinstrukturkonstante beeinflusst die Stärke dieser Quantenwechselwirkungen
	\end{itemize}
	
	\section{Alternative Formulierungen der Feinstrukturkonstante}
	
	\subsection{Darstellung mit Permeabilität}
	Ausgehend von der Standardform können wir die elektrische Feldkonstante $\varepsilon_0$ durch die magnetische Feldkonstante $\mu_0$ ersetzen, indem wir die Beziehung $c^2 = \frac{1}{\varepsilon_0\mu_0}$ nutzen:
	
	$$\varepsilon_0 = \frac{1}{\mu_0c^2}$$
	$$\alpha = \frac{e^2}{4\pi\left(\frac{1}{\mu_0c^2}\right)\hbar c}$$
	$$= \frac{e^2\mu_0c^2}{4\pi\hbar c}$$
	$$= \frac{e^2\mu_0c}{4\pi\hbar}$$
	
	Mit der Beziehung $\hbar = \frac{h}{2\pi}$ erhalten wir eine alternative Form:
	
	$$\alpha = \frac{\mu_0e^2}{2h}$$
	
	\subsection{Formulierung mit Elektronenmasse und Compton-Wellenlänge}
	Das Plancksche Wirkungsquantum $h$ lässt sich durch andere physikalische Größen ausdrücken:
	
	$$h = \frac{m_e c \lambda_C}{2\pi}$$
	
	wobei $\lambda_C$ die Compton-Wellenlänge des Elektrons ist:
	
	$$\lambda_C = \frac{h}{m_e c}$$
	
	Setzt man dies in die Feinstrukturkonstante ein:
	
	$$\alpha = \frac{\mu_0e^2}{2h}$$
	$$= \frac{\mu_0e^2}{2\frac{m_e c \lambda_C}{2\pi}}$$
	$$= \frac{\mu_0e^2 \cdot 2\pi}{2m_e c \lambda_C}$$
	$$= \frac{\mu_0e^2\pi}{m_e c \lambda_C}$$
	
	\subsection{Ausdruck mit klassischem Elektronenradius}
	Der klassische Elektronenradius ist definiert als:
	
	$$r_e = \frac{e^2}{4\pi\varepsilon_0 m_e c^2}$$
	
	Mit $\varepsilon_0 = \frac{1}{\mu_0c^2}$ ergibt sich:
	
	$$r_e = \frac{e^2\mu_0}{4\pi m_e c^2}$$
	
	Die Feinstrukturkonstante kann als Verhältnis des klassischen Elektronenradius zur Compton-Wellenlänge geschrieben werden:
	
	$$\alpha = \frac{r_e}{\lambda_C}$$
	
	Dies führt zu einer weiteren Form:
	
	$$\alpha = \frac{e^2\mu_0}{4\pi m_e c^2} \cdot \frac{2\pi m_e c}{h}$$
	$$= \frac{e^2\mu_0}{2hc}$$
	
	\subsection{Formulierung mit $\mu_0$ und $\varepsilon_0$ als fundamentale Konstanten}
	Unter Verwendung der Beziehung $c = \frac{1}{\sqrt{\mu_0\varepsilon_0}}$ kann die Feinstrukturkonstante als:
	
	$$\alpha = \frac{e^2}{4\pi\varepsilon_0\hbar c} \cdot \sqrt{\mu_0\varepsilon_0}$$
	$$= \frac{e^2}{4\pi\varepsilon_0\hbar} \cdot \sqrt{\mu_0\varepsilon_0}$$
	
	ausgedrückt werden.
	
	\section{Zusammenfassung}
	Die Feinstrukturkonstante kann in verschiedenen Formen dargestellt werden:
	
	$$\alpha = \frac{e^2}{4\pi\varepsilon_0\hbar c} \approx \frac{1}{137.035999}$$
	$$\alpha = \frac{\mu_0e^2}{2h}$$
	$$\alpha = \frac{r_e}{\lambda_C}$$
	$$\alpha = \frac{e^2}{4\pi\varepsilon_0\hbar} \cdot \sqrt{\mu_0\varepsilon_0}$$
	
	Diese verschiedenen Darstellungen ermöglichen unterschiedliche physikalische Interpretationen und zeigen die Verbindungen zwischen fundamentalen Naturkonstanten.
	
	\section{Fragestellungen zur Vertiefung}
	
	\begin{enumerate}
		\item Wie würde sich eine Änderung der Feinstrukturkonstante auf atomare Spektren auswirken?
		\item Welche experimentellen Methoden gibt es, um die Feinstrukturkonstante präzise zu bestimmen?
		\item Diskutieren Sie die kosmologische Bedeutung einer möglicherweise zeitlich veränderlichen Feinstrukturkonstante.
		\item Welche Rolle spielt die Feinstrukturkonstante in der Theorie der elektroschwachen Vereinigung?
		\item Inwiefern kann die Darstellung der Feinstrukturkonstante durch den klassischen Elektronenradius und die Compton-Wellenlänge physikalisch interpretiert werden?
		\item Vergleichen Sie die Ansätze von Dirac und Feynman zur Interpretation der Feinstrukturkonstante.
	\end{enumerate}
	
	\section{Herleitung des Planckschen Wirkungsquantums durch fundamentale elektromagnetische Konstanten}
	
	\subsection{Dimensionale Austauschbarkeit: Eine neue Perspektive}
	
	Bevor wir mit der Herleitung beginnen, ist es wichtig, die grundlegende Annahme dieser Arbeit zu erläutern: In einer fundamentalen Theorie könnten die Dimensionen von Quantengrößen (wie dem Planckschen Wirkungsquantum) und elektromagnetischen Größen (wie Permittivität und Permeabilität) austauschbar oder auf eine gemeinsame Basis zurückführbar sein.
	
	Diese Annahme basiert auf der Beobachtung, dass die Natur auf ihrer fundamentalsten Ebene eine vereinheitlichte Beschreibung erfordern könnte, in der die scheinbar unterschiedlichen Dimensionen von Quantenmechanik und Elektrodynamik als verschiedene Manifestationen derselben grundlegenden Struktur erscheinen. In einer solchen Theorie wären die etablierten Dimensionsbarrieren nicht absolut, sondern Artefakte unserer makroskopischen Perspektive.
	
	Diese Austauschbarkeit der Dimensionen ermöglicht es uns, eine direkte Verbindung zwischen h und den elektromagnetischen Konstanten herzustellen, die über die konventionelle Physik hinausgeht und möglicherweise auf eine tiefere Struktur des Universums hindeutet.
	
	\subsection{Beziehung zwischen $h$, $\mu_0$ und $\varepsilon_0$}
	
	Zunächst betrachten wir die grundlegende Beziehung zwischen der Lichtgeschwindigkeit $c$, der Permeabilität $\mu_0$ und der Permittivität $\varepsilon_0$:
	
	$$c = \frac{1}{\sqrt{\mu_0\varepsilon_0}}$$
	
	Wir nutzen außerdem die fundamentale Relation zwischen dem Planckschen Wirkungsquantum $h$ und der Compton-Wellenlänge $\lambda_C$ des Elektrons:
	
	$$h = \frac{m_e c \lambda_C}{2\pi}$$
	
	Die Compton-Wellenlänge ist definiert als:
	
	$$\lambda_C = \frac{h}{m_e c}$$
	
	Durch Einsetzen der Lichtgeschwindigkeit $c = \frac{1}{\sqrt{\mu_0\varepsilon_0}}$ erhalten wir:
	
	$$h = \frac{m_e}{2\pi} \cdot \frac{\lambda_C}{\sqrt{\mu_0\varepsilon_0}}$$
	
	Nun ersetzen wir $\lambda_C$ durch seine Definition:
	
	$$h = \frac{m_e}{2\pi} \cdot \frac{h}{m_e c \sqrt{\mu_0\varepsilon_0}}$$
	
	Dies führt zu:
	
	$$h^2 = \frac{1}{\mu_0\varepsilon_0} \cdot \frac{m_e^2 \lambda_C^2}{4\pi^2}$$
	
	Mit $\lambda_C = \frac{h}{m_e c}$ folgt:
	
	$$h^2 = \frac{1}{\mu_0\varepsilon_0} \cdot \frac{m_e^2}{4\pi^2} \cdot \frac{h^2}{m_e^2c^2}$$
	
	Nach dem Kürzen von $m_e^2$ und Einsetzen von $c^2 = \frac{1}{\mu_0\varepsilon_0}$ erhalten wir schließlich:
	
	$$h = \frac{1}{2\pi\sqrt{\mu_0\varepsilon_0}}$$
	
	Diese Gleichung zeigt, dass unter der Annahme der dimensionalen Austauschbarkeit das Plancksche Wirkungsquantum $h$ tatsächlich durch die elektromagnetischen Vakuumkonstanten $\mu_0$ und $\varepsilon_0$ ausgedrückt werden kann. Diese Beziehung deutet auf eine tiefere Verbindung zwischen Quantenmechanik und Elektrodynamik hin.
	
	\section{Neudefinition der Feinstrukturkonstante}
	
	\subsection{Frage: Was bedeutet die Elementarladung $e$?}
	
	Die Elementarladung $e$ steht für die elektrische Ladung eines Elektrons oder Protons und beträgt etwa $e \approx 1,602 \times 10^{-19}$ C (Coulomb).
	
	\subsection{Die Feinstrukturkonstante durch elektromagnetische Vakuumkonstanten}
	
	Die Feinstrukturkonstante $\alpha$ wird traditionell definiert als:
	
	$$\alpha = \frac{e^2}{4\pi\varepsilon_0\hbar c}$$
	
	Durch Einsetzen der Herleitung für $h$ erhalten wir:
	
	$$\alpha = \frac{e^2}{4\pi\varepsilon_0} \cdot \frac{2\pi\sqrt{\mu_0\varepsilon_0}}{1}$$
	
	Dies führt zu:
	
	$$\alpha = \frac{e^2}{2} \cdot \frac{\mu_0}{\varepsilon_0}$$
	
	Diese Darstellung zeigt, dass die Feinstrukturkonstante direkt aus der elektromagnetischen Struktur des Vakuums abgeleitet werden kann, ohne dass $h$ explizit vorkommen muss.
	
	\section{Konsequenzen einer Neudefinition von Coulomb}
	
	\subsection{Frage: Ist Coulomb falsch definiert, wenn man $\alpha = 1$ setzt?}
	
	Die Hypothese ist, dass wenn man die Feinstrukturkonstante $\alpha = 1$ setzen würde, die Definition des Coulomb und damit der Elementarladung $e$ angepasst werden müsste.
	
	\subsection{Neue Definition der Elementarladung}
	
	Wenn wir $\alpha = 1$ setzen, dann ergibt sich für die Elementarladung $e$:
	
	$$e^2 = 4\pi\varepsilon_0\hbar c$$
	
	$$e = \sqrt{4\pi\varepsilon_0\hbar c}$$
	
	Dies würde bedeuten, dass der Zahlenwert von $e$ sich ändern würde, weil er dann direkt von $\hbar$, $c$ und $\varepsilon_0$ abhängt.
	
	\subsection{Physikalische Bedeutung}
	
	Die Einheit Coulomb (C) ist eine willkürliche Festlegung im SI-System. Wenn man stattdessen $\alpha = 1$ wählt, würde sich die Definition von $e$ ändern. In natürlichen Einheitensystemen (wie in der Hochenergiephysik üblich) wird oft $\alpha = 1$ gesetzt, was bedeutet, dass Ladung in einer anderen Einheit als Coulomb gemessen wird.
	
	Der aktuelle Wert der Feinstrukturkonstanten $\alpha \approx \frac{1}{137}$ ist nicht "falsch", sondern eine Konsequenz unserer historischen Definitionen von Einheiten. Man hätte ursprünglich das elektromagnetische Einheitensystem auch so definieren können, dass $\alpha = 1$ gilt.
	
	\section{Auswirkungen auf andere SI-Einheiten}
	
	\subsection{Frage: Welche Auswirkungen hätte eine Coulomb-Anpassung auf andere Einheiten?}
	
	Eine Anpassung der Ladungseinheit, so dass $\alpha = 1$ gilt, hätte Konsequenzen für zahlreiche andere physikalische Einheiten:
	
	\subsubsection{Neue Ladungseinheit}
	Die neue Elementarladung wäre:
	$$e = \sqrt{4\pi\varepsilon_0\hbar c}$$
	
	\subsubsection{Änderung der elektrischen Stromstärke (Ampere)}
	Da $1 \text{ A} = 1 \text{ C}/\text{s}$, würde sich auch die Einheit des Ampere entsprechend ändern.
	
	\subsubsection{Änderungen der elektromagnetischen Konstanten}
	Da $\varepsilon_0$ und $\mu_0$ mit der Lichtgeschwindigkeit verknüpft sind:
	$$c^2 = \frac{1}{\mu_0\varepsilon_0}$$
	müssten entweder $\mu_0$ oder $\varepsilon_0$ angepasst werden.
	
	\subsubsection{Auswirkungen auf die Kapazität (Farad)}
	Die Kapazität ist definiert als $C = \frac{Q}{V}$. Da sich $Q$ (Ladung) ändert, würde sich auch die Einheit des Farad ändern.
	
	\subsubsection{Änderungen der Spannungseinheit (Volt)}
	Die elektrische Spannung ist definiert als $1 \text{ V} = 1 \text{ J}/\text{C}$. Da Coulomb eine andere Größe hätte, würde sich auch die Größe des Volt verschieben.
	
	\subsubsection{Indirekte Auswirkungen auf die Masse}
	In der Quantenfeldtheorie ist die Feinstrukturkonstante mit der Ruhemassenenergie von Elektronen verknüpft, was indirekte Auswirkungen auf die Massendefinition haben könnte.
	
	\section{Natürliche Einheiten und fundamentale Physik}
	
	\subsection{Frage: Warum kann man $h$ und $c$ auf 1 setzen?}
	
	Das Setzen von $\hbar = 1$ und $c = 1$ ist eine Vereinfachung mit tieferer Bedeutung. Es geht darum, natürliche Einheiten zu wählen, die direkt aus fundamentalen physikalischen Gesetzen folgen, anstatt von Menschen geschaffene Einheiten wie Meter, Kilogramm oder Sekunden zu verwenden.
	
	\subsubsection{Die Lichtgeschwindigkeit $c = 1$}
	Die Lichtgeschwindigkeit hat die Einheit Meter pro Sekunde: $c = 299\,792\,458 \text{ m/s}$. In der Relativitätstheorie sind Raum und Zeit untrennbar (Raumzeit). Wenn wir Längeneinheiten in Lichtsekunden messen, dann fallen Meter und Sekunde als getrennte Konzepte weg – und $c = 1$ wird eine reine Verhältniszahl.
	
	\subsubsection{Das Plancksche Wirkungsquantum $\hbar = 1$}
	Das reduzierte Plancksche Wirkungsquantum $\hbar$ hat die Einheit Joule-Sekunden $= \frac{\text{kg} \cdot \text{m}^2}{\text{s}}$. In der Quantenmechanik bestimmt $\hbar$, wie groß der kleinstmögliche Drehimpuls oder die kleinste Wirkung sein kann. Wenn wir eine neue Einheit für Wirkung wählen, sodass die kleinste Wirkung einfach "1" ist, dann wird $\hbar = 1$.
	
	\subsection{Konsequenzen für andere Einheiten}
	Wenn wir $c = 1$ und $\hbar = 1$ setzen, ändern sich die Einheiten von allem anderen automatisch:
	
	\begin{itemize}
		\item Energie und Masse werden gleichgesetzt: $E = mc^2 \Rightarrow m = E$
		\item Länge wird in Einheiten der Compton-Wellenlänge gemessen
		\item Zeit wird oft in inversen Energieeinheiten gemessen
	\end{itemize}
	
	Dies bedeutet, dass wir eigentlich nur noch eine fundamentale Einheit brauchen – Energie –, weil Längen, Zeiten und Massen alle als Energie umgerechnet werden können.
	
	\subsection{Bedeutung für die Physik}
	Es ist mehr als nur eine Vereinfachung! Es zeigt, dass unsere gewohnten Einheiten (Meter, Kilogramm, Sekunde, Coulomb, etc.) eigentlich nicht fundamental sind. Sie sind nur menschliche Konventionen, die auf unserer alltäglichen Erfahrung basieren.
	
	Bei natürlichen Einheiten verschwinden alle menschengemachten Maßeinheiten, und die Physik sieht "einfacher" aus. Die Naturgesetze selbst haben keine bevorzugten Einheiten – die kommen nur von uns!
	
\end{document}