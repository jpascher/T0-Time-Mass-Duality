\maketitle
	
	\section*{Kapitel 42: Planck-Einheiten und universelle Konstanten in der fraktalen T0-Geometrie}
	
	\subsection*{Kurze Einführung}
	
	Dieses Kapitel leitet die Planck-Einheiten und alle fundamentalen Konstanten aus dem einzigen Parameter \(\xi\) ab.
	
	\subsection*{Mathematische Grundlage}
	
	Planck-Einheiten gelten als natürliche Skalen, bleiben aber im Standardmodell willkürlich. In der FFGFT emergieren sie aus der fraktalen Vakuumstruktur mit \(\xi = \frac{4}{3} \times 10^{-4}\).
	
	\subsection*{Symbolverzeichnis und Einheiten}
	
	\begin{tcolorbox}[title={\textbf{Wichtige Symbole und ihre Einheiten}}, colback=blue!5!white, colframe=blue!75!black]
		\begin{tabular}{>{\raggedright}p{0.25\textwidth}>{\raggedright}p{0.35\textwidth}>{\raggedright\arraybackslash}p{0.35\textwidth}}
			\textbf{Symbol} & \textbf{Bedeutung} & \textbf{Einheit (SI)} \\
			\hline
			\(\xi\) & Fraktaler Skalenparameter & dimensionslos \\
			\(\ell_P\) & Planck-Länge & \si{\meter} \\
			\(m_P\) & Planck-Masse & \si{\kilo\gram} \\
			\(t_P\) & Planck-Zeit & \si{\second} \\
			\(l_0\) & Fraktale Korrelationslänge & \si{\meter} \\
			\(\rho_0\) & Vakuumgleichgewichtsdichte & \si{\kilo\gram\per\meter^{3}} \\
			\(G\) & Gravitationskonstante & \si{\meter\cubed\per\kilo\gram\per\second\squared} \\
			\(\hbar\) & Reduziertes Plancksches Wirkungsquantum & \si{\joule\second} \\
			\(c\) & Lichtgeschwindigkeit & \si{\meter\per\second} \\
		\end{tabular}
	\end{tcolorbox}
	
	\subsection*{Planck-Länge aus Korrelationsskala}
	
	Die Planck-Länge:
	
	\begin{equation}
		\ell_P = l_0 \cdot \xi^{1/2}.
	\end{equation}
	
	Die fraktale Korrelationslänge \(l_0\) wird durch \(\xi^{1/2}\) auf Planck-Skala skaliert – erklärt die winzige Größe.
	
	\textbf{Einheitenprüfung:}
	\begin{align*}
		[\ell_P] &= \si{\meter}.
	\end{align*}
	
	\subsection*{Planck-Masse}
	
	Planck-Masse:
	
	\begin{equation}
		m_P = \rho_0 \cdot l_0^3 \cdot \xi^{-3/2}.
	\end{equation}
	
	Die Dichte \(\rho_0\) im Volumen \(l_0^3\), verstärkt durch \(\xi^{-3/2}\).
	
	\subsection*{Planck-Zeit}
	
	Planck-Zeit:
	
	\begin{equation}
		t_P = \frac{\ell_P}{c} = \frac{l_0 \xi^{1/2}}{c}.
	\end{equation}
	
	Direkt aus Länge und Lichtgeschwindigkeit.
	
	\subsection*{Emergenz von \(G\), \(\hbar\), \(c\)} % Changed to use math mode
	
	Gravitationskonstante:
	
	\begin{equation}
		G = \frac{\hbar c}{m_P^2} \cdot \xi^2.
	\end{equation}
	
	Schwäche durch \(\xi^2\).
	
	Alle Konstanten reduzieren auf \(\xi\), \(l_0\), \(\rho_0\), \(c\).
	
	\subsection*{Vergleich Standard – FFGFT}
	
	\begin{center}
		\begin{tabular}{p{0.45\textwidth}p{0.45\textwidth}}
			\textbf{Standard} & \textbf{FFGFT (T0)} \\
			\hline
			Planck-Einheiten willkürlich & Emergent aus \(\xi\) \\
			19 freie Konstanten & Reduziert auf \(\xi\) \\
			Keine Hierarchie & Geometrisch erklärt \\
			Ad-hoc & Parameterfrei \\
		\end{tabular}
	\end{center}
	
	\subsection*{Schlussfolgerung}
	
	Die FFGFT leitet Planck-Einheiten und alle universellen Konstanten aus der fraktalen Skala \(\xi\) ab. Die Hierarchieprobleme verschwinden – alles ist geometrische Konsequenz eines einzigen Parameters.