\chapter*{Notation and Symbols}
\addcontentsline{toc}{chapter}{Notation and Symbols}

This central notation guide lists all important mathematical symbols and physical quantities used throughout the FFGFT Narrative Edition.

\section*{Fundamental Constants}

\begin{itemize}
    \item $c$ -- Speed of light in vacuum ($c \approx 3 \times 10^8$ m/s)
    \item $G$ -- Gravitational constant ($G \approx 6.674 \times 10^{-11}$ m$^3$ kg$^{-1}$ s$^{-2}$)
    \item $\hbar$ -- Reduced Planck constant ($\hbar = h/(2\pi) \approx 1.055 \times 10^{-34}$ J·s)
    \item $h$ -- Planck constant ($h \approx 6.626 \times 10^{-34}$ J·s)
    \item $k_B$ -- Boltzmann constant ($k_B \approx 1.381 \times 10^{-23}$ J/K)
\end{itemize}

\section*{T0-Specific Quantities}

\begin{itemize}
    \item $T_0$ -- Fundamental fractal time scale (characteristic time of the cosmic brain)
    \item $M_0$ -- Fundamental fractal mass scale
    \item $L_0$ -- Fundamental fractal length scale ($L_0 = c T_0$)
    \item $\mathcal{D}$ -- Fractal dimension
    \item $\alpha$ -- Fine-structure constant ($\alpha \approx 1/137$)
\end{itemize}

\section*{Spacetime and Geometry}

\begin{itemize}
    \item $g_{\mu\nu}$ -- Metric tensor
    \item $R_{\mu\nu}$ -- Ricci tensor
    \item $R$ -- Ricci scalar (curvature scalar)
    \item $G_{\mu\nu}$ -- Einstein tensor
    \item $T_{\mu\nu}$ -- Energy-momentum tensor
    \item $\Gamma^\lambda_{\mu\nu}$ -- Christoffel symbols (connection coefficients)
    \item $ds^2$ -- Line element (infinitesimal spacetime interval)
\end{itemize}

\section*{Special Relativity}

\begin{itemize}
    \item $\gamma$ -- Lorentz factor ($\gamma = 1/\sqrt{1-v^2/c^2}$)
    \item $\beta$ -- Relativistic velocity ($\beta = v/c$)
    \item $E$ -- Energy
    \item $E_0$ -- Rest energy ($E_0 = m_0 c^2$)
    \item $p$ -- Momentum
    \item $m_0$ -- Rest mass
    \item $\tau$ -- Proper time
\end{itemize}

\section*{Quantum Mechanics}

\begin{itemize}
    \item $\psi$ -- Wave function
    \item $|\psi\rangle$ -- State vector (ket vector in Dirac notation)
    \item $\langle\psi|$ -- Dual state vector (bra vector)
    \item $\hat{H}$ -- Hamiltonian operator
    \item $\hat{p}$ -- Momentum operator
    \item $\hat{x}$ -- Position operator
    \item $[\hat{A}, \hat{B}]$ -- Commutator ($[\hat{A}, \hat{B}] = \hat{A}\hat{B} - \hat{B}\hat{A}$)
    \item $\Delta x$ -- Position uncertainty
    \item $\Delta p$ -- Momentum uncertainty
\end{itemize}

\section*{Cosmology}

\begin{itemize}
    \item $H_0$ -- Hubble constant today ($H_0 \approx 70$ km/s/Mpc)
    \item $\Omega_m$ -- Matter density parameter
    \item $\Omega_\Lambda$ -- Dark energy density parameter
    \item $\Omega_k$ -- Curvature density parameter
    \item $a(t)$ -- Scale factor of the universe
    \item $z$ -- Redshift
    \item $\rho$ -- Energy density
    \item $\Lambda$ -- Cosmological constant
\end{itemize}

\section*{Fractal Geometry}

\begin{itemize}
    \item $\mathcal{D}_H$ -- Hausdorff dimension
    \item $\mathcal{D}_f$ -- Fractal dimension
    \item $N(\epsilon)$ -- Number of boxes of size $\epsilon$ (box-counting)
    \item $\epsilon$ -- Resolution scale
    \item $\mathcal{F}$ -- Fractal measure
\end{itemize}

\section*{Thermodynamics}

\begin{itemize}
    \item $S$ -- Entropy
    \item $T$ -- Temperature
    \item $U$ -- Internal energy
    \item $F$ -- Free energy (Helmholtz)
    \item $Q$ -- Heat
    \item $W$ -- Work
\end{itemize}

\section*{Electrodynamics}

\begin{itemize}
    \item $E$ -- Electric field
    \item $B$ -- Magnetic field
    \item $F_{\mu\nu}$ -- Electromagnetic field strength tensor
    \item $A_\mu$ -- Four-potential
    \item $j^\mu$ -- Four-current density
    \item $q$ -- Electric charge
\end{itemize}

\section*{Field Theory}

\begin{itemize}
    \item $\phi$ -- Scalar field
    \item $\mathcal{L}$ -- Lagrangian density
    \item $S$ -- Action
    \item $\partial_\mu$ -- Partial derivative ($\partial_\mu = \partial/\partial x^\mu$)
    \item $D_\mu$ -- Covariant derivative
    \item $\nabla_\mu$ -- Covariant derivative (in curved spacetime)
\end{itemize}

\section*{Statistical Mechanics}

\begin{itemize}
    \item $Z$ -- Partition function
    \item $P$ -- Probability
    \item $\langle A \rangle$ -- Expectation value of observable $A$
    \item $\beta$ -- Inverse temperature ($\beta = 1/(k_B T)$)
\end{itemize}

\section*{Units and Scales}

\begin{itemize}
    \item Gly -- Gigalightyear ($10^9$ light-years)
    \item ly -- Lightyear
    \item Mpc -- Megaparsec ($1$ Mpc $\approx 3.26$ Mly)
    \item MeV -- Mega-electronvolt ($10^6$ eV)
    \item GeV -- Giga-electronvolt ($10^9$ eV)
    \item $l_P$ -- Planck length ($l_P = \sqrt{\hbar G/c^3} \approx 1.616 \times 10^{-35}$ m)
    \item $t_P$ -- Planck time ($t_P = l_P/c \approx 5.391 \times 10^{-44}$ s)
    \item $m_P$ -- Planck mass ($m_P = \sqrt{\hbar c/G} \approx 2.176 \times 10^{-8}$ kg)
\end{itemize}

\section*{Mathematical Operations}

\begin{itemize}
    \item $\nabla$ -- Nabla operator (gradient)
    \item $\nabla \cdot$ -- Divergence
    \item $\nabla \times$ -- Curl
    \item $\nabla^2$ -- Laplace operator
    \item $\Box$ -- d'Alembert operator ($\Box = \partial_\mu \partial^\mu$)
    \item $\int$ -- Integral
    \item $\sum$ -- Sum
    \item $\prod$ -- Product
\end{itemize}

\section*{Special Functions}

\begin{itemize}
    \item $\delta(x)$ -- Dirac delta function
    \item $\Theta(x)$ -- Heaviside step function
    \item $\Gamma(x)$ -- Gamma function
    \item $\exp(x)$ or $e^x$ -- Exponential function
    \item $\ln(x)$ -- Natural logarithm
\end{itemize}

\section*{Index Conventions}

\begin{itemize}
    \item Greek indices ($\mu, \nu, \rho, \sigma$) run from 0 to 3 (spacetime indices)
    \item Latin indices ($i, j, k, l$) run from 1 to 3 (spatial indices)
    \item Einstein summation convention: Repeated indices are summed over
    \item Minkowski metric: $\eta_{\mu\nu} = \text{diag}(-1, +1, +1, +1)$ (mostly used signature)
\end{itemize}

\vspace{1em}
\noindent\textit{Note: This notation guide applies to all chapters of the FFGFT Narrative Edition.}
