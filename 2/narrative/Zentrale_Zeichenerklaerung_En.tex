% Central Symbol Legend for FFGFT Narrative (English)
% This file is included in the master document

\chapter*{Central Symbol Legend}
\addcontentsline{toc}{chapter}{Central Symbol Legend}

This central notation guide lists all important mathematical symbols and physical quantities used throughout the FFGFT Narrative Edition.

\begin{longtable}{@{}p{0.18\textwidth}p{0.22\textwidth}p{0.50\textwidth}@{}}
  \toprule
  \textbf{Symbol} & \textbf{Unit} & \textbf{Meaning} \\
  \midrule
  \endfirsthead
  
  \toprule
  \textbf{Symbol} & \textbf{Unit} & \textbf{Meaning} \\
  \midrule
  \endhead
  
  \multicolumn{3}{l}{\textbf{Fundamental Constants}} \\
  \midrule
  $c$ & \si{\meter\per\second} & Speed of light in vacuum ($c \approx 2.998 \times 10^8$ m/s) \\
  $G$ & \si{\meter\cubed\per\kilo\gram\per\second\squared} & Gravitational constant ($G \approx 6.674 \times 10^{-11}$ m$^3$ kg$^{-1}$ s$^{-2}$) \\
  $h$ & \si{\joule\second} & Planck constant ($h \approx 6.626 \times 10^{-34}$ J·s) \\
  $\hbar$ & \si{\joule\second} & Reduced Planck constant ($\hbar = h/(2\pi)$) \\
  $k_B$ & \si{\joule\per\kelvin} & Boltzmann constant ($k_B \approx 1.381 \times 10^{-23}$ J/K) \\
  $\alpha$ & -- & Fine-structure constant ($\alpha \approx 1/137$) \\
  \addlinespace
  
  \multicolumn{3}{l}{\textbf{FFGFT-Specific Quantities}} \\
  \midrule
  $\xi$ & -- & Fractal parameter ($\xi = \frac{4}{3} \times 10^{-4}$) \\
  $D_f$ & -- & Fractal dimension ($D_f = 3 - \xi$) \\
  $T_0$ & \si{\second} & Fundamental fractal time scale ($T_0 = 1.31 \times 10^{-16}$ s) \\
  $M_0$ & \si{\kilo\gram} & Fundamental fractal mass scale \\
  $L_0$ & \si{\meter} & Fundamental fractal length scale ($L_0 = c T_0$) \\
  $l_0$ & \si{\meter} & Characteristic length scale \\
  $a_0$ & \si{\meter\per\second\squared} & Characteristic acceleration \\
  \addlinespace
  
  \multicolumn{3}{l}{\textbf{Spacetime and Geometry}} \\
  \midrule
  $g_{\mu\nu}$ & -- & Metric tensor \\
  $R_{\mu\nu}$ & \si{\per\meter\squared} & Ricci tensor \\
  $R$ & \si{\per\meter\squared} & Ricci scalar (curvature scalar) \\
  $G_{\mu\nu}$ & \si{\per\meter\squared} & Einstein tensor \\
  $T_{\mu\nu}$ & \si{\joule\per\meter\cubed} & Energy-momentum tensor \\
  $\Gamma^\lambda_{\mu\nu}$ & \si{\per\meter} & Christoffel symbols \\
  $ds^2$ & \si{\meter\squared} & Line element \\
  \addlinespace
  
  \multicolumn{3}{l}{\textbf{Special Relativity}} \\
  \midrule
  $\gamma$ & -- & Lorentz factor ($\gamma = 1/\sqrt{1-v^2/c^2}$) \\
  $\beta$ & -- & Relativistic velocity ($\beta = v/c$) \\
  $E$ & \si{\joule} & Energy \\
  $E_0$ & \si{\joule} & Rest energy ($E_0 = m_0 c^2$) \\
  $p$ & \si{\kilo\gram\meter\per\second} & Momentum \\
  $m_0$ & \si{\kilo\gram} & Rest mass \\
  $\tau$ & \si{\second} & Proper time \\
  \addlinespace
  
  \multicolumn{3}{l}{\textbf{Quantum Mechanics}} \\
  \midrule
  $\psi$ & -- & Wave function \\
  $|\psi\rangle$ & -- & State vector (ket vector) \\
  $\langle\psi|$ & -- & Dual state vector (bra vector) \\
  $\hat{H}$ & \si{\joule} & Hamiltonian operator \\
  $\hat{p}$ & \si{\kilo\gram\meter\per\second} & Momentum operator \\
  $\hat{x}$ & \si{\meter} & Position operator \\
  $[\hat{A}, \hat{B}]$ & -- & Commutator ($[\hat{A}, \hat{B}] = \hat{A}\hat{B} - \hat{B}\hat{A}$) \\
  $\Delta x$ & \si{\meter} & Position uncertainty \\
  $\Delta p$ & \si{\kilo\gram\meter\per\second} & Momentum uncertainty \\
  \addlinespace
  
  \multicolumn{3}{l}{\textbf{Cosmology}} \\
  \midrule
  $H_0$ & km/s/Mpc & Hubble constant today ($H_0 \approx 70$ km/s/Mpc) \\
  $H(t)$ & \si{\per\second} & Hubble parameter \\
  $\Omega_m$ & -- & Matter density parameter \\
  $\Omega_\Lambda$ & -- & Dark energy density parameter \\
  $\Omega_k$ & -- & Curvature density parameter \\
  $\Omega_r$ & -- & Radiation density parameter \\
  $a(t)$ & -- & Scale factor of the universe \\
  $z$ & -- & Redshift ($z = \frac{\lambda_{obs} - \lambda_{em}}{\lambda_{em}}$) \\
  $\rho$ & \si{\joule\per\meter\cubed} & Energy density \\
  $\rho_c$ & \si{\joule\per\meter\cubed} & Critical density \\
  $\Lambda$ & \si{\per\meter\squared} & Cosmological constant \\
  $w$ & -- & Equation of state parameter ($p = w \rho c^2$) \\
  \addlinespace
  
  \multicolumn{3}{l}{\textbf{Fractal Geometry}} \\
  \midrule
  $\mathcal{D}_H$ & -- & Hausdorff dimension \\
  $\mathcal{D}_f$ & -- & Fractal dimension \\
  $N(\epsilon)$ & -- & Number of boxes of size $\epsilon$ (box-counting) \\
  $\epsilon$ & \si{\meter} & Resolution scale \\
  $\mathcal{F}$ & -- & Fractal measure \\
  \addlinespace
  
  \multicolumn{3}{l}{\textbf{Thermodynamics}} \\
  \midrule
  $S$ & \si{\joule\per\kelvin} & Entropy \\
  $T$ & \si{\kelvin} & Temperature \\
  $U$ & \si{\joule} & Internal energy \\
  $F$ & \si{\joule} & Free energy (Helmholtz) \\
  $Q$ & \si{\joule} & Heat \\
  $W$ & \si{\joule} & Work \\
  \addlinespace
  
  \multicolumn{3}{l}{\textbf{Electrodynamics}} \\
  \midrule
  $E$ & \si{\volt\per\meter} & Electric field \\
  $B$ & \si{\tesla} & Magnetic field \\
  $F_{\mu\nu}$ & -- & Electromagnetic field strength tensor \\
  $A_\mu$ & \si{\volt\second\per\meter} & Four-potential \\
  $j^\mu$ & \si{\ampere\per\meter\squared} & Four-current density \\
  $q$ & \si{\coulomb} & Electric charge \\
  \addlinespace
  
  \multicolumn{3}{l}{\textbf{Field Theory}} \\
  \midrule
  $\phi$ & -- & Scalar field \\
  $\Phi$ & -- & Field variable \\
  $\mathcal{L}$ & \si{\joule\per\meter\cubed} & Lagrangian density \\
  $S$ & \si{\joule\second} & Action \\
  $\partial_\mu$ & \si{\per\meter} & Partial derivative ($\partial_\mu = \partial/\partial x^\mu$) \\
  $D_\mu$ & \si{\per\meter} & Covariant derivative \\
  $\nabla_\mu$ & \si{\per\meter} & Covariant derivative (in curved spacetime) \\
  \addlinespace
  
  \multicolumn{3}{l}{\textbf{Statistical Mechanics}} \\
  \midrule
  $Z$ & -- & Partition function \\
  $P$ & -- & Probability \\
  $\langle A \rangle$ & -- & Expectation value of observable $A$ \\
  $\beta$ & \si{\per\joule} & Inverse temperature ($\beta = 1/(k_B T)$) \\
  \addlinespace
  
  \multicolumn{3}{l}{\textbf{Particle Physics}} \\
  \midrule
  $m_e$ & \si{\kilo\gram} & Electron mass \\
  $m_\mu$ & \si{\kilo\gram} & Muon mass \\
  $m_\tau$ & \si{\kilo\gram} & Tauon mass \\
  $m_\nu$ & eV/c$^2$ & Neutrino mass \\
  $\theta_{ij}$ & -- & Mixing angle \\
  $\delta_{CP}$ & -- & CP-violating phase \\
  \addlinespace
  
  \multicolumn{3}{l}{\textbf{Units and Scales}} \\
  \midrule
  Gly & -- & Gigalightyear ($10^9$ light-years) \\
  ly & -- & Lightyear (1 ly $\approx 9.461 \times 10^{15}$ m) \\
  Mpc & -- & Megaparsec (1 Mpc $\approx 3.26$ Mly) \\
  eV & -- & Electronvolt (1 eV $\approx 1.602 \times 10^{-19}$ J) \\
  MeV & -- & Mega-electronvolt ($10^6$ eV) \\
  GeV & -- & Giga-electronvolt ($10^9$ eV) \\
  $l_P$ & \si{\meter} & Planck length ($l_P = \sqrt{\hbar G/c^3} \approx 1.616 \times 10^{-35}$ m) \\
  $t_P$ & \si{\second} & Planck time ($t_P = l_P/c \approx 5.391 \times 10^{-44}$ s) \\
  $m_P$ & \si{\kilo\gram} & Planck mass ($m_P = \sqrt{\hbar c/G} \approx 2.176 \times 10^{-8}$ kg) \\
  \addlinespace
  
  \multicolumn{3}{l}{\textbf{Mathematical Operations}} \\
  \midrule
  $\nabla$ & \si{\per\meter} & Nabla operator (gradient) \\
  $\nabla \cdot$ & \si{\per\meter} & Divergence \\
  $\nabla \times$ & \si{\per\meter} & Curl \\
  $\nabla^2$ & \si{\per\meter\squared} & Laplace operator \\
  $\Box$ & \si{\per\meter\squared} & d'Alembert operator ($\Box = \partial_\mu \partial^\mu$) \\
  $\int$ & -- & Integral \\
  $\sum$ & -- & Sum \\
  $\prod$ & -- & Product \\
  \addlinespace
  
  \multicolumn{3}{l}{\textbf{Special Functions}} \\
  \midrule
  $\delta(x)$ & -- & Dirac delta function \\
  $\Theta(x)$ & -- & Heaviside step function \\
  $\Gamma(x)$ & -- & Gamma function \\
  $\exp(x)$ or $e^x$ & -- & Exponential function \\
  $\ln(x)$ & -- & Natural logarithm \\
  \addlinespace
  
  \bottomrule
\end{longtable}

\vspace{1em}
\section*{Index Conventions}

\begin{itemize}
    \item Greek indices ($\mu, \nu, \rho, \sigma$) run from 0 to 3 (spacetime indices)
    \item Latin indices ($i, j, k, l$) run from 1 to 3 (spatial indices)
    \item Einstein summation convention: Repeated indices are summed over
    \item Minkowski metric: $\eta_{\mu\nu} = \text{diag}(-1, +1, +1, +1)$ (mostly used signature)
\end{itemize}

\vspace{1em}
\noindent\textit{Note: This notation guide applies to all chapters of the FFGFT Narrative Edition.}



