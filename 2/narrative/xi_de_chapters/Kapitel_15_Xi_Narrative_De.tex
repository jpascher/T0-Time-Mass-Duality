% Auto-reconstructed from FFGFT_Xi_Narrative_Master_De_print.pdf
% RAW source: 2\narrative\xi_de_chapters_raw\Kapitel_15_Xi_De_raw.txt

\chapter{Kapitel 15: Quellen und weiterführende Literatur}



Quellen und

weiterführende

Literatur



Dieses Kapitel führt die wichtigsten externen Quel-

len auf, die im Xi-Narrativ zitiert werden, und ver-

weist auf ergänzende T0-Dokumente im Reposito-

ry.




Literaturverzeichnis



[Modesto(2008)]     L.   Modesto,    “Fractal   Structure

     of Loop Quantum Gravity,” Class. Quantum

     Grav. 26 (2009) 242002,           arXiv:0812.2214
     [gr-qc].

[Modesto(2009)]     L.   Modesto,    “Fractal   Quantum

     Space-Time,”    arXiv:0905.1665 [gr-qc].

[Calcagni(2010)]   G.    Calcagni,   “Fractal   universe

     and quantum gravity,” Phys. Rev. Lett. 104

     (2010) 251301,      arXiv:0912.3142 [hep-th].

[Calcagni(2010b)]   G.     Calcagni,    “Quantum    field

     theory, gravity and cosmology in a fractal uni-

    verse,” JHEP 03 (2010) 120,        arXiv:1001.0571
     [hep-th].

[Calcagni(2012)]   G. Calcagni, “Introduction to mul-

     tifractional spacetimes,” AIP Conf. Proc. 1483

     (2012) 31,   arXiv:1209.1110 [hep-th].

[Hořava(2009)]    P. Hořava, “Spectral Dimension of

     the Universe in Quantum Gravity at a Lifs-

     hitz Point,” Phys. Rev. Lett. 102 (2009) 161301,

     arXiv:0902.3657 [hep-th].

[Thürigen(2015)]    J. Thürigen, “Discrete Quantum

     Geometries,”   arXiv:1511.08737 [gr-qc].





[Jiang et al.(2024)]   W.-C. Jiang, M.-C. Zhong, Y.-

     K. Fang, S. Donsa, I. Březinová, L.-Y. Peng,

     J. Burgdörfer, “Time Delays as Attosecond

     Probe of Interelectronic Coherence and Entan-

     glement,” Phys. Rev. Lett. 133 (2024) 163201,

     doi:10.1103/PhysRevLett.133.163201.


[NASA Space News(2026)]          NASA     Space    News,

     “Scientists Measure Quantum Entanglement

     Speed – And It Breaks Physics,” YouTube-

                         https://www.youtub
     Video, 14. Januar 2026,

     e.com/watch?v=t3wjY95zvNM (abgerufen am
     15. Januar 2026).


[Pascher(2026a)]   J. Pascher, “Fraktale Raumzeit

     und ihre Implikationen in der Quantengravi-

     tation,” internes T0-Dokument 141\_Renormie-

     rung\_De (2026), als PDF im GitHub-Repository

     unter   /141\_Renormierung\_De.pdf.

[Pascher(2026b)]   J.    Pascher,       “Attosekunden-

     Vorhersage    zur     Entstehung      von     Quan-

     tenverschränkung      als    Beleg    für   die   T0 -

     Time-Mass-Duality-Theorie,”          internes     T0-

     Dokument 142\_Experimet-verschränkung\_De

     (2026), als PDF im GitHub-Repository unter

     /142\_Experimet-verschränkung\_De.pdf.

[Pascher(2025a)]   J. Pascher, “T0-Teilchenmassen

     und     Leptonenhierarchie,”       internes       T0-

     Dokument           006\_T0\_Teilchenmassen\_De

     (2025), als PDF im GitHub-Repository unter

     /006\_T0\_Teilchenmassen\_De.pdf.

[Pascher(2025b)]   J.   Pascher,    “Feinstrukturkon-

     stante und fraktale Geometrie,” interne T0-

     Dokumente     044\_Feinstrukturkonstante\_De








    und       043\_ResolvingTheConstantsAlfa\_De

    (2025), als PDFs im GitHub-Repository unter

    /044\_Feinstrukturkonstante\_De.pdf und
    /043\_ResolvingTheConstantsAlfa\_De.pdf.

[Pascher(2025c)]      J. Pascher, “Natürliche Einhei-

    ten   und    ihre      Systematik,”      internes   T0-

    Dokument      015\_NatEinheitenSystematik\_De

    (2025), als PDF im GitHub-Repository unter

    /015\_NatEinheitenSystematik\_De.pdf.

[Pascher(2025d)]      J. Pascher, “T0, natürliche Ein-

    heiten     und      SI,”    interne     T0-Dokumente

    014\_T0\_nat-si\_De                und      013\_T0\_SI\_De

    (2025),     als     PDFs        im   GitHub-Repository

    unter    /014\_T0\_nat-si\_De.pdf                      und

    /013\_T0\_SI\_De.pdf.

[Pascher(2025e)]      J.   Pascher,        “T0-Kosmologie

    und      fraktale      Geometrie,”       interne    T0-

    Dokumente              026\_T0\_Geometrische\_Kos-

    mologie\_De          und     025\_T0\_Kosmologie\_De

    (2025), als PDFs im GitHub-Repository unter

    /026\_T0\_Geometrische\_Kosmologie\_De.pdf
    und /025\_T0\_Kosmologie\_De.pdf.





