% Auto-reconstructed from FFGFT_Xi_Narrative_Master_De_print.pdf
% RAW source: 2\narrative\xi_de_chapters_raw\Kapitel_08_Xi_De_raw.txt

\chapter{Kapitel 8: Singularitäten und natürlicher UV-Cutoff}



Singularitäten und

natürlicher UV-Cutoff



In vielen Standardmodellen der Physik treten for-

male Unendlichkeiten auf: Divergierende Integrale

in der Quantenfeldtheorie, Singularitäten in schwar-

zen Löchern oder ein punktförmiger Anfang des

Universums. Üblicherweise werden diese Proble-

me durch Hilfsverfahren wie Renormierung, künstli-

che UV-Cutoffs oder spezielle Anfangsbedingun-

gen entschärft. Die Zeit-Masse-Dualität und die

fraktale Raumzeitstruktur der FFGFT schlagen ei-

nen anderen Weg ein: Die zugrunde liegende Geo-

metrie ist so organisiert, dass echte physikalische

Unendlichkeiten gar nicht erst entstehen.








8.1    Mathematische                Singularitä-

       ten als Artefakte


Singularitäten entstehen in der Regel dann, wenn

eine Theorie außerhalb ihres Gültigkeitsbereichs ex-

trapoliert wird. Ein klassisches Beispiel ist die Punkt-

ladung in der Elektrodynamik, deren Feldenergie

formal divergiert, wenn man den Abstand exakt auf

Null setzt. Auch in der Allgemeinen Relativitätstheo-

rie treten bei der Beschreibung schwarzer Löcher

und im Standard-Big-Bang-Modell Divergenzen der

Krümmung auf.

   Die FFGFT interpretiert diese Singularitäten als

Hinweis darauf, dass die Annahme einer exakt glat-

ten, kontinuierlichen Raumzeit bis zu beliebig klei-

nen Skalen unphysikalisch ist. Sobald man die frak-

tale Dimension

                      𝐷𝑓 = 3 − 𝜉                   (8.1)


und eine minimale effektive Längenskala berück-

sichtigt, verschwinden die formalen Unendlichkei-

ten und werden durch große, aber endliche Beiträge

ersetzt.




8.2        Fraktale Dimension und UV-

           Verhalten


Wie im vorherigen Kapitel erläutert, führt die Ab-

senkung von   𝐷𝑓 gegenüber 3 dazu, dass Integrale,
die in exakt dreidimensionalen Theorien divergieren

würden, abgeschwächt werden. Auf sehr kleinen

Skalen wirkt die fraktale Struktur wie ein eingebau-

ter UV-Cutoff:








• Volumenelemente wachsen etwas anders als in

    der glatten 3D-Geometrie.


•   Effektive Phasenräume für hochenergetische Mo-

    den werden reduziert.


•   Selbstenergien   und   Schleifenbeiträge   bleiben

    endlich und werden durch     𝜉 und die zugehöri-
    gen Skalen fixiert.

    In dieser Sicht ist ein UV-Cutoff keine frei ge-

wählte Rechengröße, sondern Ausdruck der realen

geometrischen Struktur der Raumzeit. Die Theo-

rie selbst kennt keine unendlichen Energiedichten,

sondern nur die Grenze ihrer effektiven Beschrei-

bung auf Skalen unterhalb der durch    𝜉 bestimmten
Längen.




8.3      Minimale Längenskalen und

         Zeit-Masse-Struktur


Die FFGFT arbeitet mit einer Hierarchie von Län-

genskalen: von sehr kleinen, fraktal organisierten

Tiefenstrukturen bis hin zu makroskopischen Be-

reichen, in denen die Raumzeit praktisch glatt er-

scheint. Auf den tiefsten Ebenen gibt es eine mi-

nimale effektive Längenskala, unterhalb derer es

keinen Sinn mehr ergibt, von klassischen Punkten

zu sprechen.

    Narrativ gesprochen bedeutet das:

•   Die Zeit-Masse-Struktur besitzt eine endliche Fal-

    tungsdichte; sie kann dichter, aber nicht unend-

    lich dicht werden.








•   Regionen großer effektiver Masse entsprechen

    stark gefalteter Zeit, nicht einem ”Loch” mit un-

    endlicher Krümmung.


• Auch im frühen Universum wird eine extrem dich-

    te, aber endliche Anfangskonfiguration beschrie-

    ben, keine mathematische Singularität.

    Damit wird der Begriff der Singularität durch ei-

ne geometrisch organisierte Sättigung ersetzt: Wo

klassische Theorien unendliche Größen vorhersa-

gen, beschreibt die FFGFT Bereiche, in denen die

fraktale Struktur ihre maximale Dichte erreicht.




8.4      Konsequenzen für schwarze

         Löcher und den Urknall


Für schwarze Löcher bedeutet dies, dass der innere

Bereich nicht als Punkt mit unendlicher Krümmung

verstanden wird, sondern als Zone, in der die Zeit-

Masse-Struktur maximal gefaltet ist. Die klassische

Horizontstruktur bleibt als effektive Grenze für Be-

obachter erhalten, aber im Inneren verhindert die

fraktale Geometrie das Auftreten unendlicher Ener-

giedichten.

    Ähnlich wird der Anfang des Universums nicht

als unendliche Dichte beschrieben, sondern als

Übergangsphase, in der sich die fraktale Tiefen-

struktur der Raumzeit von einem nahezu homoge-

nen Zustand zu der heutigen, hierarchisch organi-

sierten Struktur entwickelt. Skalen wie die CMB-

Temperatur und charakteristische Hubble-Größen

erscheinen in diesem Bild als Folge dieser Entwick-

lung, nicht als Folge einer mathematischen Singula-

rität.








   Insgesamt ersetzt die Zeit-Masse-Dualität die

Vorstellung physikalischer Unendlichkeiten durch

eine konsistente, durch   𝜉 gesteuerte Geometrie mit
natürlichem UV-Cutoff. Dies schließt an die bereits

diskutierten Zusammenhänge zwischen       𝜉, Massen,
Kopplungen, Casimir-Effekt und Kosmologie an und

verbindet mikroskopische und kosmologische Ska-

len in einem gemeinsamen Rahmen.




