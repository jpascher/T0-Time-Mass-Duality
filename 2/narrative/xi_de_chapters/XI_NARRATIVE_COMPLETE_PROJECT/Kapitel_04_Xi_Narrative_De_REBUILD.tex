% Kapitel 04: Quanteninformation und Grundfunktionen
% Komplett neu geschrieben mit korrekten Formeln
% Basis: 020_T0_QM-QFT-RT_De.tex, 147_quantum_computing_En.tex

\chapter{Quanteninformation und Grundfunktionen in der Zeit-Masse-Dualität}

\section{Einführung}

In diesem Kapitel wird die Verbindung zwischen der geometrischen Struktur der 
FFGFT und der Quanteninformationstheorie beschrieben. Der Fokus liegt nicht auf 
technischen Schaltplänen, sondern auf der Frage, wie sich Qubits, Überlagerung 
und Verschränkung aus der Zeit-Masse-Dualität heraus verstehen lassen.

\section{Qubits als effektive Freiheitsgrade}

\subsection{Standardformulierung}

In der üblichen Formulierung ist ein Qubit ein Zustandsvektor in einem 
zweidimensionalen Hilbertraum:

\begin{equation}
|\psi\rangle = \alpha |0\rangle + \beta |1\rangle, \quad |\alpha|^2 + |\beta|^2 = 1
\label{eq:qubit_state}
\end{equation}

wobei $|0\rangle$ und $|1\rangle$ die Basiszustände und $\alpha, \beta \in \mathbb{C}$ 
komplexe Amplituden sind.

\subsection{FFGFT-Interpretation}

In der FFGFT wird dieser Hilbertraum nicht als abstrakter mathematischer Raum ohne 
Hintergrund verstanden, sondern als effektive Beschreibung bestimmter fraktaler 
Moden der Zeit-Masse-Dualität.

Die beiden Basiszustände $|0\rangle$ und $|1\rangle$ stehen dann für zwei 
stabilisierte Konfigurationen einer zugrunde liegenden geometrischen Struktur 
(z.B. zwei lokal verschiedene Phasen des Feldes), während die Koeffizienten 
$\alpha$ und $\beta$ die Verteilung der Aktivierung in dieser Struktur wiedergeben.

\subsection{Bloch-Sphären-Darstellung}

Ein reiner Qubit-Zustand kann auf der Bloch-Sphäre dargestellt werden:

\begin{equation}
|\psi\rangle = \cos\left(\frac{\theta}{2}\right)|0\rangle + e^{i\phi}\sin\left(\frac{\theta}{2}\right)|1\rangle
\label{eq:bloch_sphere}
\end{equation}

mit $\theta \in [0,\pi]$ und $\phi \in [0,2\pi)$. Diese Interpretation ändert 
an der formalen Verwendung der Qubit-Algebra nichts; sie macht nur explizit, 
dass die Parameter letztlich durch $\xipar$ und die daraus folgenden Skalen 
festgelegt sind.

\section{Überlagerung und Interferenz}

\subsection{Quantenüberlagerung}

Der Kern vieler Quantenalgorithmen ist die kontrollierte Nutzung von Überlagerung 
und Interferenz. In der üblichen Sprache spricht man davon, dass ein Qubit 
gleichzeitig „0" und „1" ist und dass sich diese Anteile konstruktiv oder 
destruktiv überlagern.

In der Zeit-Masse-Dualität beschreibt dies keine mysteriöse Nicht-Lokalität, 
sondern die Tatsache, dass die zugrunde liegende fraktale Zeitstruktur mehrere 
Pfade parallel unterstützt.

\subsection{Hadamard-Transformation}

Die Hadamard-Transformation ist fundamental für Quantenalgorithmen:

\begin{equation}
H = \frac{1}{\sqrt{2}}\begin{pmatrix} 1 & 1 \\ 1 & -1 \end{pmatrix}
\label{eq:hadamard}
\end{equation}

Sie erzeugt aus einem Basiszustand eine gleichmäßige Überlagerung:

\begin{align}
H|0\rangle &= \frac{1}{\sqrt{2}}(|0\rangle + |1\rangle) \label{eq:h_on_0}\\
H|1\rangle &= \frac{1}{\sqrt{2}}(|0\rangle - |1\rangle) \label{eq:h_on_1}
\end{align}

\section{Verschränkung und Bell-Zustände}

\subsection{Zwei-Qubit-Systeme}

Für zwei Qubits ist der Hilbertraum vierdimensional mit Basis 
$\{|00\rangle, |01\rangle, |10\rangle, |11\rangle\}$. Ein allgemeiner Zustand ist:

\begin{equation}
|\Psi\rangle = \alpha_{00}|00\rangle + \alpha_{01}|01\rangle + \alpha_{10}|10\rangle + \alpha_{11}|11\rangle
\label{eq:two_qubit_state}
\end{equation}

mit $\sum_{ij}|\alpha_{ij}|^2 = 1$.

\subsection{Bell-Zustände}

Die maximally entangled Bell-Zustände sind:

\begin{align}
|\Phi^+\rangle &= \frac{1}{\sqrt{2}}(|00\rangle + |11\rangle) \label{eq:bell_phi_plus}\\
|\Phi^-\rangle &= \frac{1}{\sqrt{2}}(|00\rangle - |11\rangle) \label{eq:bell_phi_minus}\\
|\Psi^+\rangle &= \frac{1}{\sqrt{2}}(|01\rangle + |10\rangle) \label{eq:bell_psi_plus}\\
|\Psi^-\rangle &= \frac{1}{\sqrt{2}}(|01\rangle - |10\rangle) \label{eq:bell_psi_minus}
\end{align}

Diese Zustände sind nicht als Produkt $|\psi_1\rangle \otimes |\psi_2\rangle$ 
darstellbar und repräsentieren maximale Verschränkung.

\subsection{T0-Modifikation der Bell-Korrelationen}

In der T0-Theorie werden Bell-Korrelationen durch $\xipar$ modifiziert. Die 
Korrelationsfunktion für verschränkte Photonen mit Messrichtungen $a$ und $b$ ist:

\begin{equation}
E(a,b) = -\cos(a-b) \cdot \left(1 - \xipar \cdot f(n,l,j)\right)
\label{eq:bell_correlation_t0}
\end{equation}

wobei $f(n,l,j)$ eine Funktion der Quantenzahlen ist. Dies führt zu einer 
Dämpfung der Verletzung der Bell-Ungleichung:

\begin{equation}
S_{\text{CHSH}} = 2\sqrt{2} \cdot \left(1 - \xipar \cdot g(n)\right) \approx 2.827
\label{eq:chsh_t0}
\end{equation}

verglichen mit dem Standardwert $S_{\text{CHSH}}^{\text{QM}} = 2\sqrt{2} \approx 2.828$.

\section{Quantengatter}

\subsection{Einqubit-Gatter}

Die fundamentalen Einqubit-Gatter sind:

\textbf{Pauli-Matrizen:}
\begin{align}
X = \begin{pmatrix} 0 & 1 \\ 1 & 0 \end{pmatrix}, \quad
Y = \begin{pmatrix} 0 & -i \\ i & 0 \end{pmatrix}, \quad
Z = \begin{pmatrix} 1 & 0 \\ 0 & -1 \end{pmatrix}
\label{eq:pauli_matrices}
\end{align}

\textbf{Phasen-Gatter:}
\begin{equation}
S = \begin{pmatrix} 1 & 0 \\ 0 & i \end{pmatrix}, \quad
T = \begin{pmatrix} 1 & 0 \\ 0 & e^{i\pi/4} \end{pmatrix}
\label{eq:phase_gates}
\end{equation}

\subsection{Zwei-Qubit-Gatter: CNOT}

Das Controlled-NOT Gatter ist fundamental für Verschränkung:

\begin{equation}
\text{CNOT} = \begin{pmatrix} 
1 & 0 & 0 & 0 \\
0 & 1 & 0 & 0 \\
0 & 0 & 0 & 1 \\
0 & 0 & 1 & 0
\end{pmatrix}
\label{eq:cnot}
\end{equation}

Es wirkt auf zwei Qubits als:
\begin{equation}
\text{CNOT}|a\rangle|b\rangle = |a\rangle|a \oplus b\rangle
\label{eq:cnot_action}
\end{equation}

wobei $\oplus$ die Addition modulo 2 ist.

\section{Quantenalgorithmen}

\subsection{Quanten-Fourier-Transformation}

Die Quanten-Fourier-Transformation (QFT) ist zentral für viele Algorithmen:

\begin{equation}
\text{QFT}|j\rangle = \frac{1}{\sqrt{N}}\sum_{k=0}^{N-1} e^{2\pi ijk/N}|k\rangle
\label{eq:qft}
\end{equation}

für ein $n$-Qubit-System mit $N = 2^n$ Basiszuständen.

\subsection{Shors Algorithmus}

Der Kern von Shors Algorithmus für Faktorisierung ist die Abbildung:

\begin{equation}
|x\rangle|0\rangle \mapsto |x\rangle|f(x)\rangle, \quad f(x) = a^x \mod N
\label{eq:shor_modular_exp}
\end{equation}

gefolgt von einer Quanten-Fourier-Transformation. Diese nutzt die Periodizität 
von $f(x)$ um Faktoren von $N$ zu finden.

\subsection{T0-Implikationen}

In der T0-Formulierung sind Quantenalgorithmen deterministisch auf der Ebene 
der Zeitfeld-Dynamik. Die scheinbare Probabilität entsteht durch die 
Projektion auf den effektiven Hilbertraum. Dies hat Implikationen für:

\begin{itemize}
\item \textbf{Dekohärenz:} Geometrisch als Dämpfung durch $\xipar$-Korrekturen
\item \textbf{Fehlerkorrektur:} Optimierung durch Ausnutzung der fraktalen Struktur
\item \textbf{Skalierung:} $\xi$-abhängige Limits für große Quantencomputer
\end{itemize}

\section{Zusammenfassung}

In diesem Kapitel haben wir die Grundlagen der Quanteninformation im Rahmen 
der Zeit-Masse-Dualität entwickelt:

\begin{enumerate}
\item Qubits als effektive Freiheitsgrade der fraktalen Zeitstruktur
\item Überlagerung und Interferenz als parallele Pfade in der Geometrie
\item Verschränkung mit $\xipar$-modifizierten Bell-Korrelationen
\item Quantengatter (Hadamard, Pauli, CNOT) mit geometrischer Interpretation
\item Quantenalgorithmen (QFT, Shor) als deterministische Zeitfeld-Dynamik
\end{enumerate}

Diese Formulierung zeigt, wie $\xipar$ nicht nur klassische Physik, sondern 
auch Quanteninformation fundamental bestimmt – eine vollständige geometrische 
Grundlage der Quantencomputer-Technologie.
