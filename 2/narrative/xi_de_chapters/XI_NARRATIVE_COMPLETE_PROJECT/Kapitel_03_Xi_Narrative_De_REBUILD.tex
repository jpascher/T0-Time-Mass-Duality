% Kapitel 03: Zeit-Masse-Dualität in Quantenmechanik und Feldtheorie
% Komplett neu geschrieben mit korrekten Formeln
% Basis: 019_T0_lagrndian_De.tex, 020_T0_QM-QFT-RT_De.tex

\chapter{Zeit-Masse-Dualität in Quantenmechanik und Feldtheorie}

\section{Einführung}

In den bisherigen Kapiteln stand die Geometrie im Vordergrund: die Zahl $\xipar$, 
die fraktale Dimension $D_f$ und die daraus folgenden Skalen. Nun wenden wir 
diese Struktur auf die vertrauten Gleichungen der Quantenmechanik und 
Quantenfeldtheorie an.

\section{Schrödingergleichung als effektive Beschreibung}

In der Standardformulierung beschreibt die zeitabhängige Schrödingergleichung

\begin{equation}
i\hbar \frac{\partial}{\partial t} \psi(t,\vec{x}) = \hat{H} \psi(t,\vec{x})
\label{eq:schroedinger}
\end{equation}

die Entwicklung einer Wellenfunktion $\psi$ unter einem Hamiltonoperator $\hat{H}$. 
Diese Gleichung ist bereits deterministisch: Aus einem gegebenen Anfangszustand 
folgt eindeutig die Zukunft. Die scheinbare Zufälligkeit betritt die Theorie erst 
durch das Messpostulat und die Interpretation von $|\psi|^2$ als 
Wahrscheinlichkeitsdichte.

\subsection{T0-Interpretation}

Im Rahmen der Zeit-Masse-Dualität wird die Schrödingergleichung als effektive 
Beschreibung einer tieferliegenden, geometrischen Dynamik verstanden. Vereinfacht 
gesagt beschreibt $\psi$ nicht ein mysteriöses „Feld der Möglichkeiten", sondern 
eine statistische Projektion der zugrunde liegenden fraktalen Zeitstruktur.

Die Parameter im Hamiltonoperator – insbesondere Massen und Kopplungsstärken – 
sind in der FFGFT nicht fundamental, sondern durch $\xipar$ und die daraus 
folgenden Skalen bestimmt.

\section{Von Schrödinger zu Dirac}

Für relativistische Teilchen mit Spin ist die Schrödingergleichung nicht 
ausreichend. Dort tritt die Dirac-Gleichung auf:

\begin{equation}
(i\gamma^\mu \partial_\mu - m)\psi = 0
\label{eq:dirac}
\end{equation}

mit den Dirac-Matrizen $\gamma^\mu$ und der Masse $m$. In der FFGFT wird $m$ 
nicht als Eingabeparameter betrachtet, sondern als abgeleitete Größe aus der 
Zeit-Masse-Dualität:

\begin{equation}
T(x,t) \cdot m(x,t) = 1
\label{eq:time_mass_duality_field}
\end{equation}

\subsection{Geometrische Deutung}

Damit ändert sich auch die Lesart der Dirac-Gleichung: Sie ist nicht die 
fundamentale Gleichung, sondern eine effektive Feldgleichung auf einem 
Hintergrund, dessen Geometrie bereits durch $\xipar$ festgelegt ist.

Die bekannten Eigenschaften – Spin, Antimaterie, Zitterbewegung – bleiben 
erhalten, erhalten aber eine geometrische Deutung im Rahmen der fraktalen 
Raumzeit.

\section{Lagrangedichte und Rolle von $\xipar$}

\subsection{Erweiterter Lagrangian mit Zeitfeld}

Die vollständige T0-Formulierung verwendet einen erweiterten Lagrangian, 
der das dynamische Zeitfeld $T(x,t)$ oder äquivalent die Massenvariation 
$\Delta m$ enthält:

Die Gleichung ist sehr breit und kann in mehreren Zeilen dargestellt werden. Hier ein Vorschlag zur Lesbarkeit (ohne den Inhalt zu ändern):

\[
\begin{aligned}
	\mathcal{L}_{\text{erweitert}} = 
	&-\frac{1}{4}F_{\mu\nu}F^{\mu\nu} 
	+ \bar{\psi}(i\gamma^\mu D_\mu - m)\psi \\
	&+ \frac{1}{2}(\partial_\mu \Delta m)(\partial^\mu \Delta m) 
	- \frac{1}{2}m_T^2 \Delta m^2 \\
	&+ \xi_{\text{par}} \, m_\ell \, \bar{\psi}_\ell \psi_\ell \, \Delta m
	\label{eq:lagrangian_extended}
\end{aligned}
\]

wobei:
\begin{itemize}
\item $F_{\mu\nu}$: Elektromagnetischer Feldstärketensor
\item $\psi$: Fermionfeld (Leptonen/Quarks)
\item $\Delta m$: Dynamische Massenvariation (Zeitfeld)
\item $m_T$: Charakteristische Masse des Zeitfeldes
\item $\xipar m_\ell$: Fundamentale Kopplungsstärke
\end{itemize}

\subsection{Massenproportionale Kopplung}

Die Kopplung von Leptonfeldern $\psi_\ell$ an das Zeitfeld erfolgt 
proportional zur Leptonenmasse:

\begin{align}
\mathcal{L}_{\text{Wechselwirkung}} &= g_T^\ell \bar{\psi}_\ell \psi_\ell \Delta m \label{eq:interaction}\\
g_T^\ell &= \xipar m_\ell \label{eq:coupling}
\end{align}

Diese massenproportionale Kopplung ist zentral für die T0-Vorhersagen und 
führt direkt zur quadratischen Massenskalierung anomaler magnetischer Momente.

\section{Fundamentale T0-Beiträge}

\subsection{Ein-Schleifen-Beitrag}

Vom Wechselwirkungsterm $\mathcal{L}_{\text{int}} = \xipar m_\ell \bar{\psi}_\ell \psi_\ell \Delta m$ 
folgt der Vertex-Faktor $-i g_T^\ell = -i \xipar m_\ell$.

Der allgemeine Ein-Schleifen-Beitrag für einen skalaren Mediator ist:

\begin{equation}
\Delta a_\ell = \frac{(g_T^\ell)^2}{8\pi^2} \int_0^1 dx \frac{m_\ell^2 (1-x)(1-x^2)}{m_\ell^2 x^2 + m_T^2 (1-x)}
\label{eq:one_loop}
\end{equation}

Im Grenzfall schwerer Mediatoren $m_T \gg m_\ell$:

\begin{align}
\Delta a_\ell &\approx \frac{(\xipar m_\ell)^2}{8\pi^2 m_T^2} \cdot \frac{5}{12} \notag\\
&= \frac{5\xipar^2 m_\ell^2}{96\pi^2 m_T^2}
\label{eq:one_loop_heavy}
\end{align}

\subsection{Higgs-Zeitfeld-Verbindung}

Mit $m_T = \lambda/\xipar$ aus der Higgs-Zeitfeld-Verbindung (wobei $\lambda$ 
der Higgs-Zeitfeld-Kopplungsparameter ist):

\begin{equation}
\boxed{\Delta a_\ell^{\text{T0}} = \frac{5\xipar^4}{96\pi^2\lambda^2} \cdot m_\ell^2}
\label{eq:t0_fundamental}
\end{equation}

\subsection{Numerische Formulierung}

Die vollständig abgeleitete T0-Beitragsformel lautet:

\begin{equation}
\Delta a_\ell^{\text{T0}} = 2.246 \times 10^{-13} \cdot m_\ell^2
\label{eq:t0_numeric}
\end{equation}

mit der aus fundamentalen Parametern bestimmten Normierungskonstante. Diese 
Formel enthält \textbf{keine freien Parameter} und liefert testbare Vorhersagen 
für alle Leptonen.

\section{Vorhersagen für Leptonen}

\subsection{Numerische Werte}

Verwendung der fundamentalen Formel mit Leptonenmassen in MeV:

\begin{align}
\Delta a_e^{\text{T0}} &= 2.246 \times 10^{-13} \cdot (0.511)^2 = 5.86 \times 10^{-14} \label{eq:ae_t0}\\
\Delta a_\mu^{\text{T0}} &= 2.246 \times 10^{-13} \cdot (105.658)^2 = 2.51 \times 10^{-9} \label{eq:amu_t0}\\
\Delta a_\tau^{\text{T0}} &= 2.246 \times 10^{-13} \cdot (1776.86)^2 = 7.09 \times 10^{-7} \label{eq:atau_t0}
\end{align}

\subsection{Interpretation}

\begin{itemize}
\item \textbf{Elektron:} $\Delta a_e^{\text{T0}} = 5.86 \times 10^{-14}$ -- 
      vernachlässigbar für aktuelle Experimente (Präzision $\sim 10^{-12}$)

\item \textbf{Myon:} $\Delta a_\mu^{\text{T0}} = 2.51 \times 10^{-9}$ -- 
      entspricht exakt der historischen Diskrepanz zwischen Experiment und 
      Standardmodell (Fermilab 2021: $\sim 2.5 \times 10^{-9}$)

\item \textbf{Tau:} $\Delta a_\tau^{\text{T0}} = 7.09 \times 10^{-7}$ -- 
      klare Vorhersage für zukünftige Experimente (Belle II ab 2026)
\end{itemize}

\section{Zusammenfassung}

In diesem Kapitel haben wir gezeigt, wie die Zeit-Masse-Dualität in die 
Quantenmechanik und Quantenfeldtheorie integriert wird:

\begin{enumerate}
\item Die Schrödingergleichung als effektive Beschreibung einer tieferliegenden 
      geometrischen Dynamik

\item Die Dirac-Gleichung mit geometrisch abgeleiteter Masse $m$ aus $T \cdot m = 1$

\item Der erweiterte Lagrangian mit Zeitfeld $\Delta m$ und massenproportionaler 
      Kopplung $g_T^\ell = \xipar m_\ell$

\item Die fundamentale T0-Formel $\Delta a_\ell^{\text{T0}} = \frac{5\xipar^4}{96\pi^2\lambda^2} \cdot m_\ell^2$ 
      ohne freie Parameter

\item Vorhersagen für alle Leptonen, insbesondere die Erklärung der Myon g-2 
      Anomalie
\end{enumerate}

Diese Formulierung zeigt, wie $\xipar$ nicht nur Massen, sondern auch 
Quantenkorrekturen und Kopplungsstärken bestimmt – eine umfassende geometrische 
Grundlage der Teilchenphysik.
