\documentclass[12pt,a4paper]{article}
\usepackage[utf8]{inputenc}
\usepackage[T1]{fontenc}
\usepackage[english]{babel}
\usepackage{amsmath}
\usepackage{amsfonts}
\usepackage{amssymb}
\usepackage{geometry}
\geometry{a4paper,left=2.5cm,right=2.5cm,top=2.5cm,bottom=2.5cm}
\setlength{\headheight}{30pt}
\usepackage{fancyhdr}
\usepackage{enumitem}
\usepackage{tcolorbox}
\usepackage{physics}
\usepackage{hyperref}
\usepackage{siunitx}

\hypersetup{
	unicode=true,
	pdfencoding=unicode,
	bookmarksopen=true
}

\DeclareSIUnit\radian{rad}

\pdfstringdefDisableCommands{%
	\def\Lambda{Lambda}%
	\def\Delta{Delta}%
	\def\approx{approx}%
	\def\Sigma{Sigma}%
	\def\eta{eta}%
	\def\psi{psi}%
	\def\xi{xi}%
}

\title{Chapter 29: The Delayed-Choice Quantum Eraser Experiment in Fractal T0-Geometry}
\author{}
\date{}

\begin{document}
	
	\maketitle
	
	\section{Chapter 29: The Delayed-Choice Quantum Eraser Experiment in Fractal T0-Geometry}
	
	
\subsection*{Progressive Narrative Introduction}

This chapter builds on the preceding insights. In the first 28 chapters, we have learned the fundamental principles of FFGFT: the Time-Mass Duality, the fractal geometry with parameter $\xi = \frac{4}{3} \times 10^{-4}$, the emergence of space, and numerous applications of these principles.

In this chapter, we expand our understanding with further aspects that follow from the established principles. We will see how the already known concepts enable new insights and how the image of the cosmic brain continues to be refined.

The results presented here assume understanding of the previous chapters and systematically advance the argumentation.

\subsection*{The Mathematical Framework}

The **Delayed-Choice Quantum Eraser (DCQE)** experiment (Kim et al., 2000; Walborn et al., 2002) vividly demonstrates quantum complementarity and entanglement. It appears to imply retrocausality: A delayed decision to erase or retain which-path information seemingly influences the interference behavior of a photon in the past. In the fractal **Fundamental Fractal-Geometric Field Theory (FFGFT)** with **T0-Time-Mass Duality**, this paradox completely resolves. The phenomenon emerges from the global, fractal coherence of the vacuum phase field \(\theta(x,t)\), regulated by the single fundamental parameter \(\xi = \frac{4}{3} \times 10^{-4}\) (dimensionless). There is no retrocausality – merely a nonlocal but causal correlation in the fractal vacuum structure.
	
	In T0, quantum states are excitations of the complex vacuum field \(\Phi(x,t) = \rho(x,t) e^{i\theta(x,t)}\). Photons are pure phase vortices (\(\delta\rho \approx 0\)), whose propagation is guided by gradients of time density \(T(x,t)\) (duality \(T(x,t) \cdot m(x,t) = 1\)). Entanglement is global phase coherence: \(\theta_{\text{signal}} + \theta_{\text{idler}} = \theta_{\text{total}} =\) const.
	
	\subsection{Symbol Directory and Units}
	
	\begin{tcolorbox}[title={\textbf{Important Symbols and their Units}}, colback=blue!5!white, colframe=blue!75!black]
		\begin{tabular}{p{0.3\textwidth}p{0.3\textwidth}p{0.35\textwidth}}
			\textbf{Symbol} & \textbf{Meaning} & \textbf{Unit (SI)} \\
			\hline
			\(\xi\) & Fractal scale parameter & dimensionless \\
			\(\Phi(x,t)\) & Complex vacuum field & \si{\kilo\gram^{1/2}\per\meter^{3/2}} \\
			\(\rho(x,t)\) & Vacuum amplitude density & \si{\kilo\gram^{1/2}\per\meter^{3/2}} \\
			\(\theta(x,t)\) & Vacuum phase field & \si{\radian} (dimensionless) \\
			\(T(x,t)\) & Time density & \si{\second\per\meter^3} \\
			\(\psi(x,t)\) & Effective wave function & dimensionless \\
			\(\Delta\theta\) & Phase perturbation & \si{\radian} \\
			\(l_0\) & Fractal correlation length & \si{\meter} \\
			\(\theta_{\text{total}}\) & Global entangled phase & \si{\radian} \\
			\(\langle \theta(x) \theta(x') \rangle\) & Phase correlation & \si{\radian^2} \\
			\(V\) & Visibility of interference & dimensionless \\
		\end{tabular}
	\end{tcolorbox}
	
	\textbf{Unit check (phase correlation):}
	\begin{align*}
		[\langle \theta \theta \rangle] &= \text{dimensionless} + \text{dimensionless} \cdot \ln(\si{\meter}/\si{\meter}) = \text{dimensionless}
	\end{align*}
	Units are consistent.
	
	\subsection{The Problem of Apparent Retrocausality}
	
	In the standard model of quantum mechanics, DCQE appears paradoxical: The total distribution at signal detector D0 never shows interference. Only with post-selection (correlation with idler detectors) do subsets with interference (erased) or clumping (which-path) occur – even if the idler measurement is delayed.
	
	This leads to misunderstandings about retrocausality. T0 resolves this parameter-free through fractal nonlocality.
	
	\subsection{Description of the Experiment}
	
	Entangled photon pairs from parametric down-conversion (PDC):
	- Signal photon → double slit → detector D0 (movable for scanning).
	- Idler photon → delayed setup with beam splitters and detectors (D1–D4).
	
	Without erasure (which-path detectors): No interference in correlated subsets.  
	With erasure (e.g., beam splitter before detectors): Interference in subsets – delayed choice only classifies the data.
	
	\subsection{Phase Coherence in the T0 Vacuum Structure}
	
	The effective wave function is a phase modulation:
	\begin{equation}
		\psi(x,t) = e^{i \theta(x,t)/\xi},
	\end{equation}
	since photons are pure phase (\(\rho \approx \rho_0\)).
	
	Fractal correlation:
	\begin{equation}
		\langle \theta(x) \theta(x') \rangle = \theta_0 + \xi \cdot \ln(|x - x'| / l_0).
	\end{equation}
	
	\textbf{Unit check:}
	\begin{align*}
		[\xi \cdot \ln(|x-x'|/l_0)] &= \text{dimensionless}
	\end{align*}
	
	For entangled pairs:
	\begin{equation}
		\theta_{\text{signal}}(x) + \theta_{\text{idler}}(x') = \theta_{\text{total}} = \text{constant}.
	\end{equation}
	
	\subsection{Derivation of the Erasure Effect}
	
	Which-path marking disturbs the idler phase:
	\begin{equation}
		\Delta \theta_{\text{idler}} \approx \pi \quad \Rightarrow \quad \Delta \theta_{\text{signal}} \approx \pi \quad (\text{through duality}),
	\end{equation}
	randomizes the phase at D0 → reduced visibility \(V \approx 0\).
	
	Erasure (e.g., 50/50 beam splitter):
	\begin{equation}
		\Delta \theta_{\text{idler}} \approx 0 \quad \Rightarrow \quad \Delta \theta_{\text{signal}} \approx 0,
	\end{equation}
	coherence maintained → \(V \approx 1\) in correlated subsets.
	
	The "delayed choice" only affects post-selection of events – the global phase \(\theta_{\text{total}}\) is always coherent.
	
	Minimal phase uncertainty from fractality:
	\begin{equation}
		\Delta \theta_{\min} \approx \xi^{3/2} \sqrt{\ln(\xi^{-1})} \approx 4.6 \times 10^{-6}.
	\end{equation}
	
	\subsection{Nonlocal Correlation Without Retrocausality}
	
	The correlation is fractally conditioned:
	\begin{equation}
		\Delta \theta_{\text{signal}} \cdot \Delta \theta_{\text{idler}} \geq \xi.
	\end{equation}
	
	This is deterministic and causal – no signal transmission backwards.
	
	\subsection{Comparison with Other Interpretations}
	
	\begin{center}
		\begin{tabular}{p{0.45\textwidth}p{0.45\textwidth}}
			\textbf{Other Interpretations} & \textbf{T0-Fractal FFGFT} \\
			\hline
			Copenhagen: Collapse, observer & Deterministic, vacuum-geometric \\
			Many-Worlds: Branching & Unified fractal phase \\
			Retrocausality models: Time travel & No retrocausality needed \\
			Additional assumptions & Parameter-free from \(\xi\) \\
		\end{tabular}
	\end{center}
	
	\subsection{Conclusion}
	
	The DCQE experiment is no longer a paradox in T0-theory: The apparent retrocausality arises from the global, fractal coherence of the vacuum phase field \(\theta(x,t)\). Erasure restores coherence in correlated subsets without changing the past event – merely the classification of data. Everything emerges parameter-free from the single scale parameter \(\xi = \frac{4}{3} \times 10^{-4}\), and unifies quantum entanglement with Time-Mass Duality as a geometric necessity of the dynamic vacuum.
	

\subsection*{Progressive Narrative Summary}

This chapter has expanded our journey through FFGFT with important aspects. The concepts developed here build directly on the insights from chapters 1-28 and prepare the ground for the following investigations.

In the cosmic brain, each new chapter corresponds to a deeper layer of understanding – similar to how in a neural network, higher processing levels build on the activations of lower levels. The mathematical structures presented here are not isolated, but an integral part of the overall picture that unfolds through all 44 chapters.

In the coming chapters, we will see how these insights find further applications and how the unified picture of FFGFT continues to be completed. Each step brings us closer to a comprehensive understanding of the universe as a self-organizing, fractally structured system – a cosmic brain that creates and maintains its own structure through the Time-Mass Duality at every moment.

\end{document}
