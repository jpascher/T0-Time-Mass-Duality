\documentclass[12pt,a4paper]{article}
% Die Preamble liegt drei Verzeichnisse höher (../../../)
% ==============================================================================
% T0 Theory: Shared English Preamble
% Version: 1.0
% Author: Johann Pascher
% Date: 2025
% ==============================================================================
%
% This is the standardized shared preamble for all English T0 Theory documents.
% Place this file in your document's directory or use a path like:
%   % ==============================================================================
% T0 Theory: Shared ENGLISH Preamble – Optimized for eBook/Book
% Version: 2.0 – Final 2026 (LuaLaTeX only) – ENGLISH corrected
% Author: Johann Pascher
% Date: January 2026
% ==============================================================================
%
% IMPORTANT: Compile EXCLUSIVELY with LuaLaTeX!
% In TeXstudio: Options → Configure TeXstudio → Build → Default Compiler → LuaLaTeX
%
% Required Fonts (install once):
% - Inter: https://fonts.google.com/specimen/Inter
% - JetBrains Mono: https://www.jetbrains.com/lp/mono/
% - Libertinus Math: https://github.com/libertinus-fonts/libertinus
% ==============================================================================

% === CHAPTER 1: BASIC PACKAGES (must come FIRST) ===
\RequirePackage{fontspec}
\RequirePackage{unicode-math}
\usepackage{chngcntr}
\setcounter{secnumdepth}{1}  % Nur Sections nummerieren (nicht subsections)
\setcounter{tocdepth}{1}     % Nur Sections im TOC (nicht subsections)
\makeatletter
\@ifundefined{c@chapter}{}{\counterwithout{section}{chapter}}  % Falls Kapitel existieren
\makeatother
\counterwithout{subsection}{section}  % Löse Verknüpfung
% === CHAPTER 2: LANGUAGE (ENGLISH) ===
\usepackage[english]{babel}
\usepackage{microtype}                    % IMPORTANT for better hyphenation!

% Typography settings for better line breaking
\frenchspacing                     % Correct English spacing after punctuation
\emergencystretch=3em              % Allows more stretch for difficult lines
\tolerance=2500                    % Higher tolerance for line breaks
\hbadness=10000                    % Suppresses "underfull hbox" warnings
\hfuzz=2pt                         % Allows minimal overfull
\pretolerance=150                  % Better word breaking

% Prevent bad page breaks
\clubpenalty=10000           % No "orphans"
\widowpenalty=10000          % No "widows"
\displaywidowpenalty=10000   % Also with equations
\brokenpenalty=10000         % No broken words across pages

% Explicit hyphenation for long technical words
\hyphenation{Fun-da-men-tal Frac-tal-Ge-o-met-ric Field The-o-ry Meth-od-o-log-i-cal}
\hyphenation{Re-vi-sion-ism Quan-ti-za-tion U-ni-fi-ca-tion Ef-fec-tive}
\hyphenation{Re-nor-mal-iz-a-bil-i-ty Sin-gu-lar-i-ties Con-cil-i-a-tion}
\hyphenation{E-mer-gence Phe-nom-e-no-log-i-cal Doc-u-men-ta-tion A-nal-y-sis}
\hyphenation{Grav-i-ta-tion Quan-tum Me-chan-ics Dog-ma-tism Con-se-quent}
\hyphenation{Par-al-lel-ism Im-ple-men-ta-tion Per-tur-ba-tions}
\hyphenation{Geo-met-ric Ar-ti-fact In-com-pat-i-bil-i-ty Con-struc-tive}
\hyphenation{Frac-tal Di-men-sion-less In-ves-ti-ga-tion De-scrip-tion}
\hyphenation{In-ter-pre-ta-tion Phe-nom-e-no-log-i-cal Math-e-mat-i-cal}
\hyphenation{Phi-lo-soph-i-cal Le-git-i-ma-tion Ap-pli-ca-tion Der-i-va-tion}
\hyphenation{U-ni-fi-ca-tion As-sump-tion Con-cep-tion Ex-pec-ta-tion}
\hyphenation{Sym-me-try-ex-ten-sion O-ver-all-pic-ture Chal-lenge}
\hyphenation{In-ter-ac-tion Ma-te-ri-al Ap-proach Per-spec-tive Pro-ce-dure}

% === CHAPTER 3: FONTS (with proper ligatures) ===
\setmainfont{Inter}[
Scale=1.02,
UprightFont=*-Regular,
BoldFont=*-Bold,
ItalicFont=*-Italic,
BoldItalicFont=*-BoldItalic,
Ligatures=TeX,           % IMPORTANT for proper typography
Language=English         % Explicit language support
]
\setsansfont{Inter}[
Scale=MatchLowercase,
Ligatures=TeX,
Language=English
]
\setmonofont{JetBrains Mono}[
Scale=0.95,
Language=English
]

% Math Font (simple & stable) – MUST come AFTER language definition
% IMPORTANT: Libertinus Math for correct \underbrace display!
\setmathfont{Libertinus Math}[Scale=1.0]

% === CHAPTER 4: MATHEMATICS PACKAGES (in STRICT order!) ===
% IMPORTANT: mathtools must come BEFORE unicode-math for some commands!
\usepackage{mathtools}           % FIRST mathtools!

% Then the rest
\usepackage{amsmath, amsfonts, amsthm}

% SIUNITX MUST be loaded BEFORE physics!
\usepackage{siunitx}
\sisetup{
	locale=US,                    % ENGLISH settings for SI units!
	group-separator={,},          % Thousands separator comma
	output-decimal-marker={.},    % Decimal separator point
	per-mode=symbol,
	separate-uncertainty=true
}

% Custom SI units used in narrative and books
\DeclareSIUnit\gigalightyear{Gly}
\DeclareSIUnit\mev{MeV}

% physics – MUST be loaded AFTER siunitx and mathtools
\usepackage{physics}

% === CHAPTER 5: ADDITIONS from pdflatex best practices ===
\usepackage{colortbl}        % Colored tables (ESSENTIAL!)
\usepackage{placeins}        % Float control: \FloatBarrier
\usepackage{subcaption}      % Subfigures
\usepackage{xurl}            % Better URL line breaking
% Hyphenation for URLs in bibliography
\def\UrlBreaks{\do\/\do-}

% === CHAPTER 6: PAGE LAYOUT
% =============================================================================
% SECTION 2: Page Geometry – 6" × 9" Buchformat
% =============================================================================
\usepackage[paperwidth=6in, paperheight=9in,
top=0.9in,
bottom=1.1in,
inner=0.9in,            % Größerer Innenrand für Bindung
outer=0.6in,            % Kleinerer Außenrand → mehr Text pro Seite
bindingoffset=0.5in,    % Puffer für Bindung (Steg)
twoside]{geometry}
\setlength{\headheight}{15pt}
%\usepackage[paperwidth=8.25in, paperheight=11in,
%top=1.0in,
%bottom=1.0in,
%left=1.0in,
%right=1.0in,
%twoside=false
% === CHAPTER 7: GRAPHICS AND TABLES ===
\usepackage{graphicx}
\usepackage[table,xcdraw]{xcolor}
% T0 brand colors
\definecolor{gold}{RGB}{255,215,0}
\definecolor{blue}{rgb}{0,0,1}
\definecolor{boxgray}{RGB}{240,240,240}
\definecolor{deepblue}{RGB}{0,0,127}
\definecolor{deepgreen}{RGB}{0,127,0}
\definecolor{deepred}{RGB}{191,0,0}
\definecolor{t0blue}{RGB}{33,150,243}
\definecolor{t0green}{RGB}{76,175,80}
\definecolor{t0orange}{RGB}{255,152,0}
\definecolor{t0purple}{RGB}{156,39,176}
\definecolor{t0red}{RGB}{244,67,54}
\definecolor{t0yellow}{RGB}{255,204,0}
\usepackage{tikz}
\usetikzlibrary{arrows.meta,positioning,shapes.geometric,decorations.pathmorphing,patterns,shapes.arrows,intersections}
\usepackage{pgfplots}
\pgfplotsset{compat=1.18}
\usepackage{quantikz}
\usepackage[most]{tcolorbox}
\tcbuselibrary{breakable}

% === WICHTIG: Algorithm-Konflikt umgehen ===
% Option: algorithmic mit GROSSBUCHSTABEN
% Gemeinsame Box für Experimente
\newtcolorbox{experimentbox}[1][]{
	colback=green!5!white,
	colframe=t0green!80!black,
	fonttitle=\bfseries,
	title={{#1}},
	breakable
}

% Abstract-Fallback
\ifdefined\abstract\else
\newenvironment{abstract}{\section*{\abstractname}\itshape\small\par\bigskip}{\bigskip}
\fi

% === MAKROS SICHER NEU DEFINIEREN / ÜBERSCHREIBEN ===
% Definiere Makros OHNE doppelte Subskripte
\newcommand{\phipar}{\phi_{\mathrm{par}}}
%\newcommand{\xipar}{\xi_{\mathrm{par}}}
\newcommand{\Qphipar}{Q_{\phi_{\mathrm{par}}}}
\newcommand{\rphipar}{r_{\phi_{\mathrm{par}}}}
\newcommand{\logphipar}{\log_{\phi_{\mathrm{par}}}}
\newcommand{\CHSH}{\text{CHSH}}
\usepackage{booktabs}
\usepackage{array}
\usepackage{longtable}
\usepackage{float}
\usepackage{adjustbox}
\usepackage{rotating}
\usepackage{tabularx}
\usepackage{makecell}
\usepackage{multirow}

% === CHAPTER 8: DOCUMENT FORMATTING ===
\usepackage{fancyhdr}
\renewcommand{\headrulewidth}{0.4pt}
\renewcommand{\footrulewidth}{0.4pt}
\usepackage{tocloft}

\usepackage{enumitem}
\setlist[itemize]{leftmargin=*, topsep=2pt, partopsep=0pt, parsep=2pt, itemsep=2pt}
\setlist[enumerate]{leftmargin=*, topsep=2pt, partopsep=0pt, parsep=2pt, itemsep=2pt}
\usepackage{setspace}
\usepackage{ragged2e}
\usepackage{multicol}

% === CHAPTER 9: CODE AND ALGORITHMS ===
\usepackage{algorithm}
\usepackage{algorithmic}
\usepackage{listings}
\lstset{
	basicstyle=\ttfamily\footnotesize,
	breaklines=true,
	breakatwhitespace=true,
	columns=flexible,
	keepspaces=true,
	showstringspaces=false,
	frame=single,
	xleftmargin=0pt,
	xrightmargin=0pt,
	literate=              % For special characters in code listings
	{ä}{{\"a}}1 {ö}{{\"o}}1 {ü}{{\"u}}1 {ß}{{\ss}}1
	{Ä}{{\"A}}1 {Ö}{{\"O}}1 {Ü}{{\"U}}1
}
\usepackage{mdframed}

% === CHAPTER 10: ADDITIONAL PACKAGES ===
\usepackage{pdflscape}
\usepackage{braket}
\usepackage{cancel}
\usepackage{caption}
\captionsetup{format=plain, labelfont=bf, justification=centering}
\usepackage{csquotes}
\usepackage{gensymb}
\usepackage{textcomp}
\usepackage{textgreek}
\usepackage{upgreek}
\usepackage{url}
\usepackage{slashed}
\usepackage{bm}

% === CHAPTER 11: HYPERREF (must come SECOND TO LAST!) ===
\usepackage{hyperref}
\hypersetup{
	colorlinks=true,
	linkcolor=black,
	citecolor=black,
	urlcolor=black,
	breaklinks=true,           % IMPORTANT for special characters in URLs!
	bookmarksnumbered=true,
	unicode=true,
	pdfencoding=auto,
	pdflang=en,                % Set PDF language to English
	pdfsubject={T0 Theory - Fundamental Fractal-Geometric Field Theory}
}

% Fix for unicode-math symbols in PDF bookmarks
\pdfstringdefDisableCommands{%
	\def\xi{xi}%
	\def\alpha{alpha}%
	\def\beta{beta}%
	\def\gamma{gamma}%
	\def\delta{delta}%
	\def\Delta{Delta}%
	\def\epsilon{epsilon}%
	\def\varepsilon{epsilon}%
	\def\theta{theta}%
	\def\kappa{kappa}%
	\def\lambda{lambda}%
	\def\mu{mu}%
	\def\nu{nu}%
	\def\pi{pi}%
	\def\rho{rho}%
	\def\sigma{sigma}%
	\def\tau{tau}%
	\def\phi{phi}%
	\def\chi{chi}%
	\def\psi{psi}%
	\def\omega{omega}%
	\def\Omega{Omega}%
	\def\Lambda{Lambda}%
	\def\times{x}%
	\def\cdot{*}%
	\def\pm{+/-}%
	\def\approx{~}%
	\def\sim{~}%
	\def\equiv{=}%
	\def\ell{l}%
	\def\hbar{h}%
	\def\rightarrow{->}%
	\def\leftarrow{<-}%
	\def\Rightarrow{=>}%
	\def\Leftarrow{<=}%
	\def\propto{~}%
	\def\mitxi{xi}%
	\def\mitalpha{alpha}%
	\def\mitbeta{beta}%
	\def\mitgamma{gamma}%
	\def\mitdelta{delta}%
	\def\mitDelta{Delta}%
	\def\mitepsilon{epsilon}%
	\def\mitvarepsilon{epsilon}%
	\def\mittheta{theta}%
	\def\mitkappa{kappa}%
	\def\mitlambda{lambda}%
	\def\mitLambda{Lambda}%
	\def\mitmu{mu}%
	\def\mitnu{nu}%
	\def\mitpi{pi}%
	\def\mitrho{rho}%
	\def\mitsigma{sigma}%
	\def\mittau{tau}%
	\def\mitphi{phi}%
	\def\mitchi{chi}%
	\def\mitpsi{psi}%
	\def\mitomega{omega}%
	\def\mitOmega{Omega}%
}

% === CHAPTER 12: BOOKMARK (must come AFTER hyperref!) ===
\usepackage{bookmark}

% === CHAPTER 13: CLEVEREF (ENGLISH LABELS) ===
\usepackage[english]{cleveref}
\crefname{equation}{Equation}{Equations}
\crefname{figure}{Figure}{Figures}
\crefname{table}{Table}{Tables}
\crefname{section}{Section}{Sections}
\crefname{chapter}{Chapter}{Chapters}
\crefname{theorem}{Theorem}{Theorems}
\crefname{lemma}{Lemma}{Lemmas}
\crefname{definition}{Definition}{Definitions}
\crefname{example}{Example}{Examples}
\crefname{remark}{Remark}{Remarks}

% === CUSTOM ENVIRONMENTS ===
% Alternative interpretation environment
\newenvironment{alternative}{%
	\begin{mdframed}[linecolor=black!30,linewidth=1pt,roundcorner=4pt,backgroundcolor=black!5]%
	}{%
	\end{mdframed}%
}

% Photon/particle environment
\newenvironment{photon}{%
	\begin{mdframed}[linecolor=blue!30,linewidth=1pt,roundcorner=4pt,backgroundcolor=blue!5]%
	}{%
	\end{mdframed}%
}

% Koide formula box environment
\newenvironment{koidebox}{%
	\begin{mdframed}[linecolor=green!30,linewidth=1pt,roundcorner=4pt,backgroundcolor=green!5]%
	}{%
	\end{mdframed}%
}

% Erkenntnis/insight environment
\newenvironment{erkenntnis}{%
	\begin{mdframed}[linecolor=orange!30,linewidth=1pt,roundcorner=4pt,backgroundcolor=orange!5]%
	}{%
	\end{mdframed}%
}

% Beziehung/relationship environment
\newenvironment{beziehung}{%
	\begin{mdframed}[linecolor=purple!30,linewidth=1pt,roundcorner=4pt,backgroundcolor=purple!5]%
	}{%
	\end{mdframed}%
}

% Derivation environment
\newenvironment{derivation}{%
	\begin{mdframed}[linecolor=teal!30,linewidth=1pt,roundcorner=4pt,backgroundcolor=teal!5]%
	}{%
	\end{mdframed}%
}

% Abhandlung/treatise environment
\newenvironment{abhandlung}{%
	\begin{mdframed}[linecolor=brown!30,linewidth=1pt,roundcorner=4pt,backgroundcolor=brown!5]%
	}{%
	\end{mdframed}%
}

% Anwendung/application environment
\newenvironment{anwendung}{%
	\begin{mdframed}[linecolor=cyan!30,linewidth=1pt,roundcorner=4pt,backgroundcolor=cyan!5]%
	}{%
	\end{mdframed}%
}

% Additional common environments
\newenvironment{konsequenz}{%
	\begin{mdframed}[linecolor=red!30,linewidth=1pt,roundcorner=4pt,backgroundcolor=red!5]%
	}{%
	\end{mdframed}%
}

\newenvironment{schlussfolgerung}{%
	\begin{mdframed}[linecolor=gray!30,linewidth=1pt,roundcorner=4pt,backgroundcolor=gray!5]%
	}{%
	\end{mdframed}%
}

\newenvironment{result}{%
	\begin{mdframed}[linecolor=violet!30,linewidth=1pt,roundcorner=4pt,backgroundcolor=violet!5]%
	}{%
	\end{mdframed}%
}

% Formula environment
\newenvironment{formula}{%
	\begin{mdframed}[linecolor=yellow!30,linewidth=1pt,roundcorner=4pt,backgroundcolor=yellow!5]%
	}{%
	\end{mdframed}%
}

% Revolutionaer/revolutionary environment
\newenvironment{revolutionaer}{%
	\begin{mdframed}[linecolor=red!50,linewidth=2pt,roundcorner=4pt,backgroundcolor=red!10]%
	}{%
	\end{mdframed}%
}

% Formel environment (German version of formula)
\newenvironment{formel}{%
	\begin{mdframed}[linecolor=yellow!30,linewidth=1pt,roundcorner=4pt,backgroundcolor=yellow!5]%
	}{%
	\end{mdframed}%
}

% Prinzip/principle environment
\newenvironment{prinzip}{%
	\begin{mdframed}[linecolor=blue!50,linewidth=2pt,roundcorner=4pt,backgroundcolor=blue!10]%
	}{%
	\end{mdframed}%
}

% Experimentell/experimental environment
\newenvironment{experimentell}{%
	\begin{mdframed}[linecolor=magenta!30,linewidth=1pt,roundcorner=4pt,backgroundcolor=magenta!5]%
	}{%
	\end{mdframed}%
}

% Neutrino environment
\newenvironment{neutrino}{%
	\begin{mdframed}[linecolor=cyan!40,linewidth=1pt,roundcorner=4pt,backgroundcolor=cyan!8]%
	}{%
	\end{mdframed}%
}

% Additional missing environments
\newenvironment{schluessel}{%
	\begin{mdframed}[linecolor=yellow!50,linewidth=1pt,roundcorner=4pt,backgroundcolor=yellow!10]%
	}{%
	\end{mdframed}%
}

\newenvironment{summary}{%
	\begin{mdframed}[linecolor=gray!40,linewidth=1pt,roundcorner=4pt,backgroundcolor=gray!8]%
	}{%
	\end{mdframed}%
}

\newenvironment{category}{%
	\begin{mdframed}[linecolor=pink!40,linewidth=1pt,roundcorner=4pt,backgroundcolor=pink!8]%
	}{%
	\end{mdframed}%
}

\newenvironment{sibox}{%
	\begin{mdframed}[linecolor=lime!40,linewidth=1pt,roundcorner=4pt,backgroundcolor=lime!8]%
	}{%
	\end{mdframed}%
}

% More missing environments
\newenvironment{documentbox}{%
	\begin{mdframed}[linecolor=teal!40,linewidth=1pt,roundcorner=4pt,backgroundcolor=teal!8]%
	}{%
	\end{mdframed}%
}

\newenvironment{t0box}{%
	\begin{mdframed}[linecolor=violet!40,linewidth=1pt,roundcorner=4pt,backgroundcolor=violet!8]%
	}{%
	\end{mdframed}%
}

\newenvironment{wichtig}{%
	\begin{mdframed}[linecolor=red!50,linewidth=2pt,roundcorner=4pt,backgroundcolor=red!10]%
	\textbf{Important:} 
	}{%
	\end{mdframed}%
}

\newenvironment{smbox}{%
	\begin{mdframed}[linecolor=orange!40,linewidth=1pt,roundcorner=4pt,backgroundcolor=orange!8]%
	}{%
	\end{mdframed}%
}

\newenvironment{pvbox}{%
	\begin{mdframed}[linecolor=purple!40,linewidth=1pt,roundcorner=4pt,backgroundcolor=purple!8]%
	}{%
	\end{mdframed}%
}

\newenvironment{numerisch}{%
	\begin{mdframed}[linecolor=blue!40,linewidth=1pt,roundcorner=4pt,backgroundcolor=blue!8]%
	}{%
	\end{mdframed}%
}

% More missing environments
\newenvironment{relation}{%
	\begin{mdframed}[linecolor=green!40,linewidth=1pt,roundcorner=4pt,backgroundcolor=green!8]%
	}{%
	\end{mdframed}%
}

\newenvironment{beweis}{%
	\begin{mdframed}[linecolor=brown!40,linewidth=1pt,roundcorner=4pt,backgroundcolor=brown!8]%
	\textbf{Proof:} 
	}{%
	\end{mdframed}%
}

\newenvironment{revolution}{%
	\begin{mdframed}[linecolor=red!60,linewidth=2pt,roundcorner=4pt,backgroundcolor=red!12]%
	}{%
	\end{mdframed}%
}

\newenvironment{key}{%
	\begin{mdframed}[linecolor=yellow!50,linewidth=1pt,roundcorner=4pt,backgroundcolor=yellow!10]%
	}{%
	\end{mdframed}%
}

\newenvironment{newperspective}{%
	\begin{mdframed}[linecolor=cyan!50,linewidth=1pt,roundcorner=4pt,backgroundcolor=cyan!10]%
	}{%
	\end{mdframed}%
}

\newenvironment{literatur}{%
	\begin{mdframed}[linecolor=gray!50,linewidth=1pt,roundcorner=4pt,backgroundcolor=gray!10]%
	}{%
	\end{mdframed}%
}

\newenvironment{folgerung}{%
	\begin{mdframed}[linecolor=teal!50,linewidth=1pt,roundcorner=4pt,backgroundcolor=teal!10]%
	}{%
	\end{mdframed}%
}

\newenvironment{principle}{%
	\begin{mdframed}[linecolor=blue!60,linewidth=2pt,roundcorner=4pt,backgroundcolor=blue!12]%
	}{%
	\end{mdframed}%
}

% Additional common environments
% ==============================================================================
% FROM HERE: YOUR DEFINITIONS (unchanged)
% ==============================================================================

\setcounter{tocdepth}{3}

% === CITATION COMMANDS ===
\providecommand{\citep}[1]{\cite{#1}}
\providecommand{\citet}[1]{\cite{#1}}

% === COLORS ===
\definecolor{gold}{RGB}{255,215,0}
\definecolor{blue}{rgb}{0,0,1}
\definecolor{boxgray}{RGB}{240,240,240}
\definecolor{deepblue}{RGB}{0,0,127}
\definecolor{deepgreen}{RGB}{0,127,0}
\definecolor{deepred}{RGB}{191,0,0}
\definecolor{t0blue}{RGB}{33,150,243}
\definecolor{t0green}{RGB}{76,175,80}
\definecolor{t0orange}{RGB}{255,152,0}
\definecolor{t0purple}{RGB}{156,39,176}
\definecolor{t0red}{RGB}{244,67,54}
\definecolor{t0yellow}{RGB}{255,204,0}

% === COLUMN TYPES ===
\newcolumntype{L}[1]{>{\raggedright\arraybackslash}p{#1}}
\newcolumntype{C}[1]{>{\centering\arraybackslash}p{#1}}
\newcolumntype{R}[1]{>{\raggedleft\arraybackslash}p{#1}}

% === HYPERREF SETTINGS (updated) ===
\hypersetup{
	colorlinks=true,
	linkcolor=t0blue,
	citecolor=t0blue,
	urlcolor=t0blue,
	breaklinks=true,
	bookmarksnumbered=true,
	pdfstartview=FitH,
	pdfencoding=auto,
	pdfdisplaydoctitle=true
}

% === ENGLISH THEOREM ENVIRONMENTS ===
\theoremstyle{plain}
\newtheorem{theorem}{Theorem}[section]
\newtheorem{lemma}[theorem]{Lemma}
\newtheorem{proposition}[theorem]{Proposition}
\newtheorem{corollary}[theorem]{Corollary}

\theoremstyle{definition}
\newtheorem{definition}[theorem]{Definition}
\newtheorem{example}[theorem]{Example}
\newtheorem{insight}[theorem]{Insight}
\newtheorem{discovery}[theorem]{Discovery}

\theoremstyle{remark}
\newtheorem{remark}[theorem]{Remark}
\newtheorem{axiom}{Axiom}
%\newtheorem{principle}{Principle}  % Commented out to avoid conflicts with document-specific definitions
%\newtheorem{warning}[theorem]{Warning}

% === T0-SPECIFIC COMMANDS ===
% (Here follow all your \newcommand and \providecommand definitions)
% These remain UNCHANGED as in your original preamble
% ==============================================================================
% SECTION 14: T0-Specific Commands
% ==============================================================================

% --- Core T0 Fields ---
\newcommand{\Tfield}{T(x,t)}
\providecommand{\Tfieldt}{T(\vec{x},t)}
\newcommand{\Efield}{E(x,t)}
\newcommand{\mfield}{m(x,t)}
\providecommand{\vecx}{\vec{x}}

% --- Lagrangian ---
\newcommand{\Lag}{\mathcal{L}}
\newcommand{\calL}{\mathcal{L}}

% --- Greek Letters and Constants ---
\newcommand{\alphaem}{\alpha}
\newcommand{\betaT}{\beta_T}
\newcommand{\xiT}{\xi}
\newcommand{\xipar}{\xi}

% --- Energy and Planck Units ---
\newcommand{\Ezero}{E_0}
\newcommand{\E}{E}
\newcommand{\EPlanck}{E_{\text{Pl}}}
\newcommand{\Mpl}{M_{\text{Pl}}}
\newcommand{\mP}{m_{\text{P}}}
\newcommand{\lP}{\ell_{\text{P}}}
\newcommand{\tP}{t_{\text{P}}}
\newcommand{\LPlanck}{\ell_{\text{Pl}}}
\newcommand{\TPlanck}{t_{\text{Pl}}}

% --- Coupling Constants ---
\newcommand{\Gnat}{G_{\text{nat}}}
\newcommand{\alphaEM}{\alpha_{\text{EM}}}
\newcommand{\alphaSI}{\alpha_{\text{SI}}}
\newcommand{\Hubble}{H_0}
\newcommand{\LCDM}{\Lambda\text{CDM}}
\newcommand{\natunits}{(nat. units)}

% --- T0 Model Parameters ---
\newcommand{\xigeom}{\xi_{\mathrm{geom}}}
\newcommand{\rzero}{r_{0}}
\newcommand{\xirat}{\xi_{\mathrm{rat}}}
\newcommand{\tzero}{t_{0}}
\newcommand{\Lambdat}{\Lambda_{\mathrm{t}}}
\newcommand{\EP}{E_{\text{P}}}
\newcommand{\Emu}{E_{\mu}}
\newcommand{\Ee}{E_{e}}
\newcommand{\Etau}{E_{\tau}}
\newcommand{\alphafine}{\alpha_{\mathrm{fine}}}
\newcommand{\alphal}{\alpha_{\ell}}
\newcommand{\Lzero}{\ell_{0}}
\newcommand{\Lp}{\ell_{\mathrm{P}}}

% --- Additional T0 Commands ---
\newcommand{\Kfrak}{K_{\text{frak}}}
\newcommand{\Dfrak}{D_{\text{frak}}}
\newcommand{\betapar}{\ensuremath{\beta_T}}
\newcommand{\alphapar}{\alpha}
\newcommand{\deltafield}{\delta \phi}
\newcommand{\deltam}{\delta m}
\newcommand{\deltaE}{\delta E}
\newcommand{\Exi}{E_{\xi}}
\newcommand{\Lxi}{\ell_{\xi}}
\newcommand{\rhoCMB}{\rho_{\text{CMB}}}
\newcommand{\rhoCasimir}{\rho_{\text{Casimir}}}
\newcommand{\Leff}{L_{\text{eff}}}
\newcommand{\CQCD}{C_{\mathrm{QCD}}}
\newcommand{\Kspec}{K_{\mathrm{spec}}}
\newcommand{\Tzero}{\ensuremath{T_0}}
\newcommand{\Eabs}{E_{\text{abs}}}
\newcommand{\taupar}{\tau}

% --- Provided Commands ---
\providecommand{\xiconst}{\xi_{\text{const}}}
\providecommand{\DhiggsT}{D_{\text{Higgs-T}}}
\providecommand{\rhoE}{\rho_{E}}
\providecommand{\Echar}{E_{\text{char}}}
\providecommand{\kfrac}{k_{\text{frac}}}
\providecommand{\alphaEMSI}{\alpha_{\text{EM,SI}}}
\providecommand{\alphaEMnat}{\alpha_{\text{EM,nat}}}
\providecommand{\betaTSI}{\beta_{T,\text{SI}}}
\providecommand{\betaTnat}{\beta_{T,\text{nat}}}
\providecommand{\Gsi}{G_{\text{SI}}}
\providecommand{\xiparSI}{\xi_{\text{SI}}}
\providecommand{\xiparnat}{\xi_{\text{nat}}}
\providecommand{\meff}{m_{\text{eff}}}
\providecommand{\Tzerot}{T_{0}(t)}
\providecommand{\mzerot}{m_{0}(t)}
\providecommand{\Ezeroabs}{E_{0,\text{abs}}}
\providecommand{\Epar}{E_{\text{par}}}
\providecommand{\Lnat}{\ell_{\text{nat}}}
\providecommand{\Tnat}{T_{\text{nat}}}
\providecommand{\xifrak}{\xi_{\text{frac}}}
\providecommand{\Tfrak}{T_{\text{frac}}}
\providecommand{\mfrak}{m_{\text{frac}}}
\providecommand{\Dfrac}{D_{\text{frac}}}
\providecommand{\EphotSI}{E_{\gamma,\text{SI}}}
\providecommand{\EphotNat}{E_{\gamma,\text{nat}}}
\providecommand{\Eabsint}{E_{\text{abs,int}}}
\providecommand{\mphoton}{m_{\gamma}}
\providecommand{\Evis}{E_{\text{vis}}}
\providecommand{\Cto}{C_{T0}}
\providecommand{\mytimes}{\times}
\providecommand{\lambdah}{\lambda_h}
\providecommand{\checkmarkx}{\checkmark}
\providecommand{\Enorm}{E_{\text{norm}}}
\providecommand{\Tobs}{T_{\text{obs}}}
\providecommand{\mobs}{m_{\text{obs}}}
\providecommand{\Eobs}{E_{\text{obs}}}
\providecommand{\Lobs}{\ell_{\text{obs}}}
\providecommand{\xobs}{\xi_{\text{obs}}}
\providecommand{\calE}{\mathcal{E}}
\providecommand{\calT}{\mathcal{T}}
\providecommand{\calM}{\mathcal{M}}
\providecommand{\alphag}{\alpha_g}
\providecommand{\Tmax}{T_{\text{max}}}
\providecommand{\mmin}{m_{\text{min}}}
\providecommand{\Lmax}{\ell_{\text{max}}}
\providecommand{\Emin}{E_{\text{min}}}
\providecommand{\Geff}{G_{\text{eff}}}
\providecommand{\rhoeff}{\rho_{\text{eff}}}
\providecommand{\xieff}{\xi_{\text{eff}}}
\providecommand{\Teff}{T_{\text{eff}}}
\providecommand{\hPlanck}{h}
\providecommand{\kB}{k_B}
\providecommand{\muB}{\mu_B}
\providecommand{\lambdaC}{\lambda_C}
\providecommand{\omegaP}{\omega_P}
\providecommand{\rhoP}{\rho_P}
\providecommand{\Tref}{T_{\text{ref}}}
\providecommand{\Eref}{E_{\text{ref}}}
\providecommand{\mref}{m_{\text{ref}}}
\providecommand{\Lref}{\ell_{\text{ref}}}
\providecommand{\xikonst}{\xi_0}
\providecommand{\Phiphoton}{\Phi_{\gamma}}
\providecommand{\etavis}{\eta_{\text{vis}}}
\providecommand{\pichar}{\pi}
\providecommand{\primrel}{\mathcal{P}_{\text{rel}}}
\providecommand{\warningx}{\textcolor{orange}{\textbf{!}}}
\providecommand{\phiT}{\phi_T}
\providecommand{\Lorentz}{\Lambda}
\providecommand{\Cconv}{C_{\text{conv}}}
\providecommand{\Df}{\Delta f}
\providecommand{\lambdazero}{\lambda_0}
\providecommand{\myapprox}{\approx}
\providecommand{\checked}{\checkmark}
\providecommand{\alphaWSI}{\alpha_W^{\text{SI}}}
\providecommand{\alphaWnat}{\alpha_W^{\text{nat}}}
\providecommand{\vect}[1]{\vec{#1}}
\providecommand{\Rzero}{R_0}
\providecommand{\Riem}{\mathcal{R}}
\providecommand{\nuzero}{\nu_0}
\providecommand{\mypi}{\pi}

% =============================================================================
% TCOLORBOX STYLES AND ENVIRONMENTS (English titles)
% =============================================================================
\tcbset{
	keyresult/.style={
		colback=blue!5!white,
		colframe=blue!75!black,
		title=Key Result,
		fonttitle=\bfseries
	},
	foundation/.style={
		colback=green!5!white,
		colframe=green!75!black,
		title=Foundation,
		fonttitle=\bfseries
	},
	alternative/.style={
		colback=orange!5!white,
		colframe=orange!75!black,
		title=Alternative,
		fonttitle=\bfseries
	},
	warningbox/.style={
		colback=red!5!white,
		colframe=red!75!black,
		title=Warning,
		fonttitle=\bfseries
	}
}

% (Here follow all your tcolorbox definitions with English titles)
\newtcolorbox{keyresultbox}[1][]{colback=blue!5!white,colframe=blue!75!black,fonttitle=\bfseries,title={#1},breakable}
\newtcolorbox{keyresult}[1][Key Result]{colback=blue!5!white,colframe=blue!75!black,fonttitle=\bfseries,title={#1},breakable}
\newtcolorbox{foundationbox}[1][]{colback=green!5!white,colframe=green!75!black,fonttitle=\bfseries,title={#1},breakable}
\newtcolorbox{foundation}[1][Foundation]{colback=green!5!white,colframe=green!75!black,fonttitle=\bfseries,title={#1},breakable}
\newtcolorbox{alternativebox}[1][]{colback=orange!5!white,colframe=orange!75!black,fonttitle=\bfseries,title={#1},breakable}
\newtcolorbox{warningboxenv}[1][Warning]{colback=red!5!white,colframe=red!75!black,fonttitle=\bfseries,title={#1},breakable}

\newtcolorbox{fundamental}[1][]{
	colback=boxgray,
	colframe=t0blue,
	fonttitle=\bfseries,
	title=#1,
	sharp corners,
	boxrule=2pt
}

\newtcolorbox{insightBox}[1][Insight]{colback=blue!5,colframe=t0blue,title={#1},fonttitle=\bfseries,breakable}
\newtcolorbox{discoveryBox}[1][Discovery]{colback=green!5,colframe=t0green,title={#1},fonttitle=\bfseries,breakable}
\newtcolorbox{revelation}[1][Revelation]{colback=red!5,colframe=t0red,title={#1},fonttitle=\bfseries,breakable}
\newtcolorbox{keypoint}[1][Key Point]{colback=blue!5,colframe=t0blue,title={#1},fonttitle=\bfseries,breakable}
\newtcolorbox{evidence}[1][Evidence]{colback=green!5,colframe=t0green,title={#1},fonttitle=\bfseries,breakable}
\newtcolorbox{conclusionBox}[1][Conclusion]{colback=gray!5,colframe=gray,title={#1},fonttitle=\bfseries,breakable}
\newtcolorbox{significance}[1][Significance]{colback=yellow!5,colframe=orange,title={#1},fonttitle=\bfseries,breakable}
\newtcolorbox{philosophical}[1][Philosophical]{colback=purple!5,colframe=purple,title={#1},fonttitle=\bfseries,breakable}
\newtcolorbox{implicationBox}[1][Implication]{colback=cyan!5,colframe=cyan,title={#1},fonttitle=\bfseries,breakable}
\newtcolorbox{perspectiveBox}[1][Perspective]{colback=blue!5,colframe=t0blue,title={#1},fonttitle=\bfseries,breakable}
\newtcolorbox{revolutionary}[1][Revolutionary]{colback=red!5,colframe=t0red,title={#1},fonttitle=\bfseries,breakable}

\newtcolorbox{technical}[1][Technical]{colback=gray!5,colframe=gray!75!black,title={#1},fonttitle=\bfseries,breakable}
\newtcolorbox{technicalBox}[1][Technical]{colback=gray!5,colframe=gray!75!black,title={#1},fonttitle=\bfseries,breakable}
\newtcolorbox{notationBox}[1][Notation]{colback=yellow!5,colframe=yellow!75!black,title={#1},fonttitle=\bfseries,breakable}
\newtcolorbox{verification}[1][Verification]{colback=orange!5!white,colframe=orange!75!black,fonttitle=\bfseries,title=#1}
\newtcolorbox{explanationBox}[1][Explanation]{colback=purple!5!white,colframe=purple!75!black,fonttitle=\bfseries,title=#1}
\newtcolorbox{interpretationBox}[1][Interpretation]{colback=cyan!5!white,colframe=cyan!75!black,fonttitle=\bfseries,title=#1}
\newtcolorbox{explanation}[1][Explanation]{colback=purple!5!white,colframe=purple!75!black,fonttitle=\bfseries,title=#1,breakable}
\newtcolorbox{interpretation}[1][Interpretation]{colback=cyan!5!white,colframe=cyan!75!black,fonttitle=\bfseries,title=#1,breakable}
\newtcolorbox{proof_step}[1][Proof Step]{colback=gray!5!white,colframe=gray!75!black,fonttitle=\bfseries,title=#1,breakable}
\newtcolorbox{experimental}[1][Experimental]{colback=teal!5!white,colframe=teal!75!black,fonttitle=\bfseries,title=#1,breakable}

\newtcolorbox{important}[1][Important]{colback=red!5!white,colframe=red!75!black,title={#1},fonttitle=\bfseries,breakable}
\newtcolorbox{warning}[1][Warning]{colback=orange!5!white,colframe=orange!75!black,title={#1},fonttitle=\bfseries,breakable}
\newtcolorbox{caution}[1][Caution]{colback=yellow!5!white,colframe=yellow!75!black,title={#1},fonttitle=\bfseries,breakable}
\newtcolorbox{highlight}[1][Highlight]{colback=yellow!10!white,colframe=yellow!75!black,title={#1},fonttitle=\bfseries,breakable}
\newtcolorbox{critical}[1][Critical]{colback=red!10!white,colframe=red!75!black,title={#1},fonttitle=\bfseries,breakable}

\newtcolorbox{analysis}[1][Analysis]{colback=blue!5!white,colframe=blue!75!black,title={#1},fonttitle=\bfseries,breakable}
\newtcolorbox{application}[1][Application]{colback=green!5!white,colframe=green!75!black,title={#1},fonttitle=\bfseries,breakable}
\newtcolorbox{experiment}[1][Experiment]{colback=cyan!5!white,colframe=cyan!75!black,title={#1},fonttitle=\bfseries,breakable}
\newtcolorbox{historical}[1][Historical]{colback=brown!5!white,colframe=brown!75!black,title={#1},fonttitle=\bfseries,breakable}
\newtcolorbox{numerical}[1][Numerical]{colback=gray!5!white,colframe=gray!75!black,title={#1},fonttitle=\bfseries,breakable}
\newtcolorbox{overview}[1][Overview]{colback=blue!5!white,colframe=blue!75!black,title={#1},fonttitle=\bfseries,breakable}
\newtcolorbox{speculation}[1][Speculation]{colback=purple!5!white,colframe=purple!75!black,title={#1},fonttitle=\bfseries,breakable}
\newtcolorbox{question}[1][Question]{colback=orange!5!white,colframe=orange!75!black,title={#1},fonttitle=\bfseries,breakable}
\newtcolorbox{method}[1][Method]{colback=teal!5!white,colframe=teal!75!black,title={#1},fonttitle=\bfseries,breakable}
\newtcolorbox{correct}[1][Correct]{colback=green!10!white,colframe=green!75!black,title={#1},fonttitle=\bfseries,breakable}
\newtcolorbox{units}[1][Units]{colback=gray!5!white,colframe=gray!75!black,title={#1},fonttitle=\bfseries,breakable}
\newtcolorbox{achievement}[1][Achievement]{colback=gold!5!white,colframe=orange!75!black,title={#1},fonttitle=\bfseries,breakable}
\newtcolorbox{equivalence}[1][Equivalence]{colback=cyan!5!white,colframe=cyan!75!black,title={#1},fonttitle=\bfseries,breakable}
\newtcolorbox{dimensional}[1][Dimensional Analysis]{colback=purple!5!white,colframe=purple!75!black,title={#1},fonttitle=\bfseries,breakable}

% === ADDITIONAL SIMPLE ENVIRONMENTS ===
\newenvironment{treatise}{\begin{quote}}{\end{quote}}
\newenvironment{gemeinsam}{\begin{quote}}{\end{quote}}
\newenvironment{vergleich}{\begin{quote}}{\end{quote}}
\newenvironment{vorteil}{\begin{quote}}{\end{quote}}
\newenvironment{common}{\begin{quote}}{\end{quote}}
\newenvironment{comparison}{\begin{quote}}{\end{quote}}
\newenvironment{advantage}{\begin{quote}}{\end{quote}}
\newenvironment{quantum}{\begin{quote}}{\end{quote}}

% === LAYOUT SETTINGS ===
\raggedbottom
\usepackage{environ}
\let\oldtabular\tabular
\let\endoldtabular\endtabular

\newenvironment{scaledtable}[1][0.85]{%
	\begingroup\footnotesize\setlength{\LTleft}{0pt}\setlength{\LTright}{0pt}%
}{%
	\endgroup%
}

\newcommand{\widetable}[1]{\resizebox{\textwidth}{!}{#1}}

% === TABLE OF CONTENTS FORMATTING ===
\renewcommand{\cftsecfont}{\color{blue}}
\renewcommand{\cftsubsecfont}{\color{blue}}
\renewcommand{\cftsecpagefont}{\color{blue}}
\renewcommand{\cftsubsecpagefont}{\color{blue}}
\renewcommand{\cfttoctitlefont}{\huge\bfseries\color{blue}}

% === DEFAULT HEADER AND FOOTER ===
\pagestyle{fancy}
\fancyhf{}
\fancyhead[L]{\textsc{T0 Theory}}
\fancyhead[R]{\textsc{J. Pascher}}
\fancyfoot[C]{\thepage}

% ==============================================================================
% End of Shared Preamble for English
% ==============================================================================
%
% Usage:
%   \documentclass[12pt,a4paper]{article}  % or book, report, etc.
%   % ==============================================================================
% T0 Theory: Shared ENGLISH Preamble – Optimized for eBook/Book
% Version: 2.0 – Final 2026 (LuaLaTeX only) – ENGLISH corrected
% Author: Johann Pascher
% Date: January 2026
% ==============================================================================
%
% IMPORTANT: Compile EXCLUSIVELY with LuaLaTeX!
% In TeXstudio: Options → Configure TeXstudio → Build → Default Compiler → LuaLaTeX
%
% Required Fonts (install once):
% - Inter: https://fonts.google.com/specimen/Inter
% - JetBrains Mono: https://www.jetbrains.com/lp/mono/
% - Libertinus Math: https://github.com/libertinus-fonts/libertinus
% ==============================================================================

% === CHAPTER 1: BASIC PACKAGES (must come FIRST) ===
\RequirePackage{fontspec}
\RequirePackage{unicode-math}
\usepackage{chngcntr}
\setcounter{secnumdepth}{1}  % Nur Sections nummerieren (nicht subsections)
\setcounter{tocdepth}{1}     % Nur Sections im TOC (nicht subsections)
\makeatletter
\@ifundefined{c@chapter}{}{\counterwithout{section}{chapter}}  % Falls Kapitel existieren
\makeatother
\counterwithout{subsection}{section}  % Löse Verknüpfung
% === CHAPTER 2: LANGUAGE (ENGLISH) ===
\usepackage[english]{babel}
\usepackage{microtype}                    % IMPORTANT for better hyphenation!

% Typography settings for better line breaking
\frenchspacing                     % Correct English spacing after punctuation
\emergencystretch=3em              % Allows more stretch for difficult lines
\tolerance=2500                    % Higher tolerance for line breaks
\hbadness=10000                    % Suppresses "underfull hbox" warnings
\hfuzz=2pt                         % Allows minimal overfull
\pretolerance=150                  % Better word breaking

% Prevent bad page breaks
\clubpenalty=10000           % No "orphans"
\widowpenalty=10000          % No "widows"
\displaywidowpenalty=10000   % Also with equations
\brokenpenalty=10000         % No broken words across pages

% Explicit hyphenation for long technical words
\hyphenation{Fun-da-men-tal Frac-tal-Ge-o-met-ric Field The-o-ry Meth-od-o-log-i-cal}
\hyphenation{Re-vi-sion-ism Quan-ti-za-tion U-ni-fi-ca-tion Ef-fec-tive}
\hyphenation{Re-nor-mal-iz-a-bil-i-ty Sin-gu-lar-i-ties Con-cil-i-a-tion}
\hyphenation{E-mer-gence Phe-nom-e-no-log-i-cal Doc-u-men-ta-tion A-nal-y-sis}
\hyphenation{Grav-i-ta-tion Quan-tum Me-chan-ics Dog-ma-tism Con-se-quent}
\hyphenation{Par-al-lel-ism Im-ple-men-ta-tion Per-tur-ba-tions}
\hyphenation{Geo-met-ric Ar-ti-fact In-com-pat-i-bil-i-ty Con-struc-tive}
\hyphenation{Frac-tal Di-men-sion-less In-ves-ti-ga-tion De-scrip-tion}
\hyphenation{In-ter-pre-ta-tion Phe-nom-e-no-log-i-cal Math-e-mat-i-cal}
\hyphenation{Phi-lo-soph-i-cal Le-git-i-ma-tion Ap-pli-ca-tion Der-i-va-tion}
\hyphenation{U-ni-fi-ca-tion As-sump-tion Con-cep-tion Ex-pec-ta-tion}
\hyphenation{Sym-me-try-ex-ten-sion O-ver-all-pic-ture Chal-lenge}
\hyphenation{In-ter-ac-tion Ma-te-ri-al Ap-proach Per-spec-tive Pro-ce-dure}

% === CHAPTER 3: FONTS (with proper ligatures) ===
\setmainfont{Inter}[
Scale=1.02,
UprightFont=*-Regular,
BoldFont=*-Bold,
ItalicFont=*-Italic,
BoldItalicFont=*-BoldItalic,
Ligatures=TeX,           % IMPORTANT for proper typography
Language=English         % Explicit language support
]
\setsansfont{Inter}[
Scale=MatchLowercase,
Ligatures=TeX,
Language=English
]
\setmonofont{JetBrains Mono}[
Scale=0.95,
Language=English
]

% Math Font (simple & stable) – MUST come AFTER language definition
% IMPORTANT: Libertinus Math for correct \underbrace display!
\setmathfont{Libertinus Math}[Scale=1.0]

% === CHAPTER 4: MATHEMATICS PACKAGES (in STRICT order!) ===
% IMPORTANT: mathtools must come BEFORE unicode-math for some commands!
\usepackage{mathtools}           % FIRST mathtools!

% Then the rest
\usepackage{amsmath, amsfonts, amsthm}

% SIUNITX MUST be loaded BEFORE physics!
\usepackage{siunitx}
\sisetup{
	locale=US,                    % ENGLISH settings for SI units!
	group-separator={,},          % Thousands separator comma
	output-decimal-marker={.},    % Decimal separator point
	per-mode=symbol,
	separate-uncertainty=true
}

% Custom SI units used in narrative and books
\DeclareSIUnit\gigalightyear{Gly}
\DeclareSIUnit\mev{MeV}

% physics – MUST be loaded AFTER siunitx and mathtools
\usepackage{physics}

% === CHAPTER 5: ADDITIONS from pdflatex best practices ===
\usepackage{colortbl}        % Colored tables (ESSENTIAL!)
\usepackage{placeins}        % Float control: \FloatBarrier
\usepackage{subcaption}      % Subfigures
\usepackage{xurl}            % Better URL line breaking
% Hyphenation for URLs in bibliography
\def\UrlBreaks{\do\/\do-}

% === CHAPTER 6: PAGE LAYOUT
% =============================================================================
% SECTION 2: Page Geometry – 6" × 9" Buchformat
% =============================================================================
\usepackage[paperwidth=6in, paperheight=9in,
top=0.9in,
bottom=1.1in,
inner=0.9in,            % Größerer Innenrand für Bindung
outer=0.6in,            % Kleinerer Außenrand → mehr Text pro Seite
bindingoffset=0.5in,    % Puffer für Bindung (Steg)
twoside]{geometry}
\setlength{\headheight}{15pt}
%\usepackage[paperwidth=8.25in, paperheight=11in,
%top=1.0in,
%bottom=1.0in,
%left=1.0in,
%right=1.0in,
%twoside=false
% === CHAPTER 7: GRAPHICS AND TABLES ===
\usepackage{graphicx}
\usepackage[table,xcdraw]{xcolor}
% T0 brand colors
\definecolor{gold}{RGB}{255,215,0}
\definecolor{blue}{rgb}{0,0,1}
\definecolor{boxgray}{RGB}{240,240,240}
\definecolor{deepblue}{RGB}{0,0,127}
\definecolor{deepgreen}{RGB}{0,127,0}
\definecolor{deepred}{RGB}{191,0,0}
\definecolor{t0blue}{RGB}{33,150,243}
\definecolor{t0green}{RGB}{76,175,80}
\definecolor{t0orange}{RGB}{255,152,0}
\definecolor{t0purple}{RGB}{156,39,176}
\definecolor{t0red}{RGB}{244,67,54}
\definecolor{t0yellow}{RGB}{255,204,0}
\usepackage{tikz}
\usetikzlibrary{arrows.meta,positioning,shapes.geometric,decorations.pathmorphing,patterns,shapes.arrows,intersections}
\usepackage{pgfplots}
\pgfplotsset{compat=1.18}
\usepackage{quantikz}
\usepackage[most]{tcolorbox}
\tcbuselibrary{breakable}

% === WICHTIG: Algorithm-Konflikt umgehen ===
% Option: algorithmic mit GROSSBUCHSTABEN
% Gemeinsame Box für Experimente
\newtcolorbox{experimentbox}[1][]{
	colback=green!5!white,
	colframe=t0green!80!black,
	fonttitle=\bfseries,
	title={{#1}},
	breakable
}

% Abstract-Fallback
\ifdefined\abstract\else
\newenvironment{abstract}{\section*{\abstractname}\itshape\small\par\bigskip}{\bigskip}
\fi

% === MAKROS SICHER NEU DEFINIEREN / ÜBERSCHREIBEN ===
% Definiere Makros OHNE doppelte Subskripte
\newcommand{\phipar}{\phi_{\mathrm{par}}}
%\newcommand{\xipar}{\xi_{\mathrm{par}}}
\newcommand{\Qphipar}{Q_{\phi_{\mathrm{par}}}}
\newcommand{\rphipar}{r_{\phi_{\mathrm{par}}}}
\newcommand{\logphipar}{\log_{\phi_{\mathrm{par}}}}
\newcommand{\CHSH}{\text{CHSH}}
\usepackage{booktabs}
\usepackage{array}
\usepackage{longtable}
\usepackage{float}
\usepackage{adjustbox}
\usepackage{rotating}
\usepackage{tabularx}
\usepackage{makecell}
\usepackage{multirow}

% === CHAPTER 8: DOCUMENT FORMATTING ===
\usepackage{fancyhdr}
\renewcommand{\headrulewidth}{0.4pt}
\renewcommand{\footrulewidth}{0.4pt}
\usepackage{tocloft}

\usepackage{enumitem}
\setlist[itemize]{leftmargin=*, topsep=2pt, partopsep=0pt, parsep=2pt, itemsep=2pt}
\setlist[enumerate]{leftmargin=*, topsep=2pt, partopsep=0pt, parsep=2pt, itemsep=2pt}
\usepackage{setspace}
\usepackage{ragged2e}
\usepackage{multicol}

% === CHAPTER 9: CODE AND ALGORITHMS ===
\usepackage{algorithm}
\usepackage{algorithmic}
\usepackage{listings}
\lstset{
	basicstyle=\ttfamily\footnotesize,
	breaklines=true,
	breakatwhitespace=true,
	columns=flexible,
	keepspaces=true,
	showstringspaces=false,
	frame=single,
	xleftmargin=0pt,
	xrightmargin=0pt,
	literate=              % For special characters in code listings
	{ä}{{\"a}}1 {ö}{{\"o}}1 {ü}{{\"u}}1 {ß}{{\ss}}1
	{Ä}{{\"A}}1 {Ö}{{\"O}}1 {Ü}{{\"U}}1
}
\usepackage{mdframed}

% === CHAPTER 10: ADDITIONAL PACKAGES ===
\usepackage{pdflscape}
\usepackage{braket}
\usepackage{cancel}
\usepackage{caption}
\captionsetup{format=plain, labelfont=bf, justification=centering}
\usepackage{csquotes}
\usepackage{gensymb}
\usepackage{textcomp}
\usepackage{textgreek}
\usepackage{upgreek}
\usepackage{url}
\usepackage{slashed}
\usepackage{bm}

% === CHAPTER 11: HYPERREF (must come SECOND TO LAST!) ===
\usepackage{hyperref}
\hypersetup{
	colorlinks=true,
	linkcolor=black,
	citecolor=black,
	urlcolor=black,
	breaklinks=true,           % IMPORTANT for special characters in URLs!
	bookmarksnumbered=true,
	unicode=true,
	pdfencoding=auto,
	pdflang=en,                % Set PDF language to English
	pdfsubject={T0 Theory - Fundamental Fractal-Geometric Field Theory}
}

% Fix for unicode-math symbols in PDF bookmarks
\pdfstringdefDisableCommands{%
	\def\xi{xi}%
	\def\alpha{alpha}%
	\def\beta{beta}%
	\def\gamma{gamma}%
	\def\delta{delta}%
	\def\Delta{Delta}%
	\def\epsilon{epsilon}%
	\def\varepsilon{epsilon}%
	\def\theta{theta}%
	\def\kappa{kappa}%
	\def\lambda{lambda}%
	\def\mu{mu}%
	\def\nu{nu}%
	\def\pi{pi}%
	\def\rho{rho}%
	\def\sigma{sigma}%
	\def\tau{tau}%
	\def\phi{phi}%
	\def\chi{chi}%
	\def\psi{psi}%
	\def\omega{omega}%
	\def\Omega{Omega}%
	\def\Lambda{Lambda}%
	\def\times{x}%
	\def\cdot{*}%
	\def\pm{+/-}%
	\def\approx{~}%
	\def\sim{~}%
	\def\equiv{=}%
	\def\ell{l}%
	\def\hbar{h}%
	\def\rightarrow{->}%
	\def\leftarrow{<-}%
	\def\Rightarrow{=>}%
	\def\Leftarrow{<=}%
	\def\propto{~}%
	\def\mitxi{xi}%
	\def\mitalpha{alpha}%
	\def\mitbeta{beta}%
	\def\mitgamma{gamma}%
	\def\mitdelta{delta}%
	\def\mitDelta{Delta}%
	\def\mitepsilon{epsilon}%
	\def\mitvarepsilon{epsilon}%
	\def\mittheta{theta}%
	\def\mitkappa{kappa}%
	\def\mitlambda{lambda}%
	\def\mitLambda{Lambda}%
	\def\mitmu{mu}%
	\def\mitnu{nu}%
	\def\mitpi{pi}%
	\def\mitrho{rho}%
	\def\mitsigma{sigma}%
	\def\mittau{tau}%
	\def\mitphi{phi}%
	\def\mitchi{chi}%
	\def\mitpsi{psi}%
	\def\mitomega{omega}%
	\def\mitOmega{Omega}%
}

% === CHAPTER 12: BOOKMARK (must come AFTER hyperref!) ===
\usepackage{bookmark}

% === CHAPTER 13: CLEVEREF (ENGLISH LABELS) ===
\usepackage[english]{cleveref}
\crefname{equation}{Equation}{Equations}
\crefname{figure}{Figure}{Figures}
\crefname{table}{Table}{Tables}
\crefname{section}{Section}{Sections}
\crefname{chapter}{Chapter}{Chapters}
\crefname{theorem}{Theorem}{Theorems}
\crefname{lemma}{Lemma}{Lemmas}
\crefname{definition}{Definition}{Definitions}
\crefname{example}{Example}{Examples}
\crefname{remark}{Remark}{Remarks}

% === CUSTOM ENVIRONMENTS ===
% Alternative interpretation environment
\newenvironment{alternative}{%
	\begin{mdframed}[linecolor=black!30,linewidth=1pt,roundcorner=4pt,backgroundcolor=black!5]%
	}{%
	\end{mdframed}%
}

% Photon/particle environment
\newenvironment{photon}{%
	\begin{mdframed}[linecolor=blue!30,linewidth=1pt,roundcorner=4pt,backgroundcolor=blue!5]%
	}{%
	\end{mdframed}%
}

% Koide formula box environment
\newenvironment{koidebox}{%
	\begin{mdframed}[linecolor=green!30,linewidth=1pt,roundcorner=4pt,backgroundcolor=green!5]%
	}{%
	\end{mdframed}%
}

% Erkenntnis/insight environment
\newenvironment{erkenntnis}{%
	\begin{mdframed}[linecolor=orange!30,linewidth=1pt,roundcorner=4pt,backgroundcolor=orange!5]%
	}{%
	\end{mdframed}%
}

% Beziehung/relationship environment
\newenvironment{beziehung}{%
	\begin{mdframed}[linecolor=purple!30,linewidth=1pt,roundcorner=4pt,backgroundcolor=purple!5]%
	}{%
	\end{mdframed}%
}

% Derivation environment
\newenvironment{derivation}{%
	\begin{mdframed}[linecolor=teal!30,linewidth=1pt,roundcorner=4pt,backgroundcolor=teal!5]%
	}{%
	\end{mdframed}%
}

% Abhandlung/treatise environment
\newenvironment{abhandlung}{%
	\begin{mdframed}[linecolor=brown!30,linewidth=1pt,roundcorner=4pt,backgroundcolor=brown!5]%
	}{%
	\end{mdframed}%
}

% Anwendung/application environment
\newenvironment{anwendung}{%
	\begin{mdframed}[linecolor=cyan!30,linewidth=1pt,roundcorner=4pt,backgroundcolor=cyan!5]%
	}{%
	\end{mdframed}%
}

% Additional common environments
\newenvironment{konsequenz}{%
	\begin{mdframed}[linecolor=red!30,linewidth=1pt,roundcorner=4pt,backgroundcolor=red!5]%
	}{%
	\end{mdframed}%
}

\newenvironment{schlussfolgerung}{%
	\begin{mdframed}[linecolor=gray!30,linewidth=1pt,roundcorner=4pt,backgroundcolor=gray!5]%
	}{%
	\end{mdframed}%
}

\newenvironment{result}{%
	\begin{mdframed}[linecolor=violet!30,linewidth=1pt,roundcorner=4pt,backgroundcolor=violet!5]%
	}{%
	\end{mdframed}%
}

% Formula environment
\newenvironment{formula}{%
	\begin{mdframed}[linecolor=yellow!30,linewidth=1pt,roundcorner=4pt,backgroundcolor=yellow!5]%
	}{%
	\end{mdframed}%
}

% Revolutionaer/revolutionary environment
\newenvironment{revolutionaer}{%
	\begin{mdframed}[linecolor=red!50,linewidth=2pt,roundcorner=4pt,backgroundcolor=red!10]%
	}{%
	\end{mdframed}%
}

% Formel environment (German version of formula)
\newenvironment{formel}{%
	\begin{mdframed}[linecolor=yellow!30,linewidth=1pt,roundcorner=4pt,backgroundcolor=yellow!5]%
	}{%
	\end{mdframed}%
}

% Prinzip/principle environment
\newenvironment{prinzip}{%
	\begin{mdframed}[linecolor=blue!50,linewidth=2pt,roundcorner=4pt,backgroundcolor=blue!10]%
	}{%
	\end{mdframed}%
}

% Experimentell/experimental environment
\newenvironment{experimentell}{%
	\begin{mdframed}[linecolor=magenta!30,linewidth=1pt,roundcorner=4pt,backgroundcolor=magenta!5]%
	}{%
	\end{mdframed}%
}

% Neutrino environment
\newenvironment{neutrino}{%
	\begin{mdframed}[linecolor=cyan!40,linewidth=1pt,roundcorner=4pt,backgroundcolor=cyan!8]%
	}{%
	\end{mdframed}%
}

% Additional missing environments
\newenvironment{schluessel}{%
	\begin{mdframed}[linecolor=yellow!50,linewidth=1pt,roundcorner=4pt,backgroundcolor=yellow!10]%
	}{%
	\end{mdframed}%
}

\newenvironment{summary}{%
	\begin{mdframed}[linecolor=gray!40,linewidth=1pt,roundcorner=4pt,backgroundcolor=gray!8]%
	}{%
	\end{mdframed}%
}

\newenvironment{category}{%
	\begin{mdframed}[linecolor=pink!40,linewidth=1pt,roundcorner=4pt,backgroundcolor=pink!8]%
	}{%
	\end{mdframed}%
}

\newenvironment{sibox}{%
	\begin{mdframed}[linecolor=lime!40,linewidth=1pt,roundcorner=4pt,backgroundcolor=lime!8]%
	}{%
	\end{mdframed}%
}

% More missing environments
\newenvironment{documentbox}{%
	\begin{mdframed}[linecolor=teal!40,linewidth=1pt,roundcorner=4pt,backgroundcolor=teal!8]%
	}{%
	\end{mdframed}%
}

\newenvironment{t0box}{%
	\begin{mdframed}[linecolor=violet!40,linewidth=1pt,roundcorner=4pt,backgroundcolor=violet!8]%
	}{%
	\end{mdframed}%
}

\newenvironment{wichtig}{%
	\begin{mdframed}[linecolor=red!50,linewidth=2pt,roundcorner=4pt,backgroundcolor=red!10]%
	\textbf{Important:} 
	}{%
	\end{mdframed}%
}

\newenvironment{smbox}{%
	\begin{mdframed}[linecolor=orange!40,linewidth=1pt,roundcorner=4pt,backgroundcolor=orange!8]%
	}{%
	\end{mdframed}%
}

\newenvironment{pvbox}{%
	\begin{mdframed}[linecolor=purple!40,linewidth=1pt,roundcorner=4pt,backgroundcolor=purple!8]%
	}{%
	\end{mdframed}%
}

\newenvironment{numerisch}{%
	\begin{mdframed}[linecolor=blue!40,linewidth=1pt,roundcorner=4pt,backgroundcolor=blue!8]%
	}{%
	\end{mdframed}%
}

% More missing environments
\newenvironment{relation}{%
	\begin{mdframed}[linecolor=green!40,linewidth=1pt,roundcorner=4pt,backgroundcolor=green!8]%
	}{%
	\end{mdframed}%
}

\newenvironment{beweis}{%
	\begin{mdframed}[linecolor=brown!40,linewidth=1pt,roundcorner=4pt,backgroundcolor=brown!8]%
	\textbf{Proof:} 
	}{%
	\end{mdframed}%
}

\newenvironment{revolution}{%
	\begin{mdframed}[linecolor=red!60,linewidth=2pt,roundcorner=4pt,backgroundcolor=red!12]%
	}{%
	\end{mdframed}%
}

\newenvironment{key}{%
	\begin{mdframed}[linecolor=yellow!50,linewidth=1pt,roundcorner=4pt,backgroundcolor=yellow!10]%
	}{%
	\end{mdframed}%
}

\newenvironment{newperspective}{%
	\begin{mdframed}[linecolor=cyan!50,linewidth=1pt,roundcorner=4pt,backgroundcolor=cyan!10]%
	}{%
	\end{mdframed}%
}

\newenvironment{literatur}{%
	\begin{mdframed}[linecolor=gray!50,linewidth=1pt,roundcorner=4pt,backgroundcolor=gray!10]%
	}{%
	\end{mdframed}%
}

\newenvironment{folgerung}{%
	\begin{mdframed}[linecolor=teal!50,linewidth=1pt,roundcorner=4pt,backgroundcolor=teal!10]%
	}{%
	\end{mdframed}%
}

\newenvironment{principle}{%
	\begin{mdframed}[linecolor=blue!60,linewidth=2pt,roundcorner=4pt,backgroundcolor=blue!12]%
	}{%
	\end{mdframed}%
}

% Additional common environments
% ==============================================================================
% FROM HERE: YOUR DEFINITIONS (unchanged)
% ==============================================================================

\setcounter{tocdepth}{3}

% === CITATION COMMANDS ===
\providecommand{\citep}[1]{\cite{#1}}
\providecommand{\citet}[1]{\cite{#1}}

% === COLORS ===
\definecolor{gold}{RGB}{255,215,0}
\definecolor{blue}{rgb}{0,0,1}
\definecolor{boxgray}{RGB}{240,240,240}
\definecolor{deepblue}{RGB}{0,0,127}
\definecolor{deepgreen}{RGB}{0,127,0}
\definecolor{deepred}{RGB}{191,0,0}
\definecolor{t0blue}{RGB}{33,150,243}
\definecolor{t0green}{RGB}{76,175,80}
\definecolor{t0orange}{RGB}{255,152,0}
\definecolor{t0purple}{RGB}{156,39,176}
\definecolor{t0red}{RGB}{244,67,54}
\definecolor{t0yellow}{RGB}{255,204,0}

% === COLUMN TYPES ===
\newcolumntype{L}[1]{>{\raggedright\arraybackslash}p{#1}}
\newcolumntype{C}[1]{>{\centering\arraybackslash}p{#1}}
\newcolumntype{R}[1]{>{\raggedleft\arraybackslash}p{#1}}

% === HYPERREF SETTINGS (updated) ===
\hypersetup{
	colorlinks=true,
	linkcolor=t0blue,
	citecolor=t0blue,
	urlcolor=t0blue,
	breaklinks=true,
	bookmarksnumbered=true,
	pdfstartview=FitH,
	pdfencoding=auto,
	pdfdisplaydoctitle=true
}

% === ENGLISH THEOREM ENVIRONMENTS ===
\theoremstyle{plain}
\newtheorem{theorem}{Theorem}[section]
\newtheorem{lemma}[theorem]{Lemma}
\newtheorem{proposition}[theorem]{Proposition}
\newtheorem{corollary}[theorem]{Corollary}

\theoremstyle{definition}
\newtheorem{definition}[theorem]{Definition}
\newtheorem{example}[theorem]{Example}
\newtheorem{insight}[theorem]{Insight}
\newtheorem{discovery}[theorem]{Discovery}

\theoremstyle{remark}
\newtheorem{remark}[theorem]{Remark}
\newtheorem{axiom}{Axiom}
%\newtheorem{principle}{Principle}  % Commented out to avoid conflicts with document-specific definitions
%\newtheorem{warning}[theorem]{Warning}

% === T0-SPECIFIC COMMANDS ===
% (Here follow all your \newcommand and \providecommand definitions)
% These remain UNCHANGED as in your original preamble
% ==============================================================================
% SECTION 14: T0-Specific Commands
% ==============================================================================

% --- Core T0 Fields ---
\newcommand{\Tfield}{T(x,t)}
\providecommand{\Tfieldt}{T(\vec{x},t)}
\newcommand{\Efield}{E(x,t)}
\newcommand{\mfield}{m(x,t)}
\providecommand{\vecx}{\vec{x}}

% --- Lagrangian ---
\newcommand{\Lag}{\mathcal{L}}
\newcommand{\calL}{\mathcal{L}}

% --- Greek Letters and Constants ---
\newcommand{\alphaem}{\alpha}
\newcommand{\betaT}{\beta_T}
\newcommand{\xiT}{\xi}
\newcommand{\xipar}{\xi}

% --- Energy and Planck Units ---
\newcommand{\Ezero}{E_0}
\newcommand{\E}{E}
\newcommand{\EPlanck}{E_{\text{Pl}}}
\newcommand{\Mpl}{M_{\text{Pl}}}
\newcommand{\mP}{m_{\text{P}}}
\newcommand{\lP}{\ell_{\text{P}}}
\newcommand{\tP}{t_{\text{P}}}
\newcommand{\LPlanck}{\ell_{\text{Pl}}}
\newcommand{\TPlanck}{t_{\text{Pl}}}

% --- Coupling Constants ---
\newcommand{\Gnat}{G_{\text{nat}}}
\newcommand{\alphaEM}{\alpha_{\text{EM}}}
\newcommand{\alphaSI}{\alpha_{\text{SI}}}
\newcommand{\Hubble}{H_0}
\newcommand{\LCDM}{\Lambda\text{CDM}}
\newcommand{\natunits}{(nat. units)}

% --- T0 Model Parameters ---
\newcommand{\xigeom}{\xi_{\mathrm{geom}}}
\newcommand{\rzero}{r_{0}}
\newcommand{\xirat}{\xi_{\mathrm{rat}}}
\newcommand{\tzero}{t_{0}}
\newcommand{\Lambdat}{\Lambda_{\mathrm{t}}}
\newcommand{\EP}{E_{\text{P}}}
\newcommand{\Emu}{E_{\mu}}
\newcommand{\Ee}{E_{e}}
\newcommand{\Etau}{E_{\tau}}
\newcommand{\alphafine}{\alpha_{\mathrm{fine}}}
\newcommand{\alphal}{\alpha_{\ell}}
\newcommand{\Lzero}{\ell_{0}}
\newcommand{\Lp}{\ell_{\mathrm{P}}}

% --- Additional T0 Commands ---
\newcommand{\Kfrak}{K_{\text{frak}}}
\newcommand{\Dfrak}{D_{\text{frak}}}
\newcommand{\betapar}{\ensuremath{\beta_T}}
\newcommand{\alphapar}{\alpha}
\newcommand{\deltafield}{\delta \phi}
\newcommand{\deltam}{\delta m}
\newcommand{\deltaE}{\delta E}
\newcommand{\Exi}{E_{\xi}}
\newcommand{\Lxi}{\ell_{\xi}}
\newcommand{\rhoCMB}{\rho_{\text{CMB}}}
\newcommand{\rhoCasimir}{\rho_{\text{Casimir}}}
\newcommand{\Leff}{L_{\text{eff}}}
\newcommand{\CQCD}{C_{\mathrm{QCD}}}
\newcommand{\Kspec}{K_{\mathrm{spec}}}
\newcommand{\Tzero}{\ensuremath{T_0}}
\newcommand{\Eabs}{E_{\text{abs}}}
\newcommand{\taupar}{\tau}

% --- Provided Commands ---
\providecommand{\xiconst}{\xi_{\text{const}}}
\providecommand{\DhiggsT}{D_{\text{Higgs-T}}}
\providecommand{\rhoE}{\rho_{E}}
\providecommand{\Echar}{E_{\text{char}}}
\providecommand{\kfrac}{k_{\text{frac}}}
\providecommand{\alphaEMSI}{\alpha_{\text{EM,SI}}}
\providecommand{\alphaEMnat}{\alpha_{\text{EM,nat}}}
\providecommand{\betaTSI}{\beta_{T,\text{SI}}}
\providecommand{\betaTnat}{\beta_{T,\text{nat}}}
\providecommand{\Gsi}{G_{\text{SI}}}
\providecommand{\xiparSI}{\xi_{\text{SI}}}
\providecommand{\xiparnat}{\xi_{\text{nat}}}
\providecommand{\meff}{m_{\text{eff}}}
\providecommand{\Tzerot}{T_{0}(t)}
\providecommand{\mzerot}{m_{0}(t)}
\providecommand{\Ezeroabs}{E_{0,\text{abs}}}
\providecommand{\Epar}{E_{\text{par}}}
\providecommand{\Lnat}{\ell_{\text{nat}}}
\providecommand{\Tnat}{T_{\text{nat}}}
\providecommand{\xifrak}{\xi_{\text{frac}}}
\providecommand{\Tfrak}{T_{\text{frac}}}
\providecommand{\mfrak}{m_{\text{frac}}}
\providecommand{\Dfrac}{D_{\text{frac}}}
\providecommand{\EphotSI}{E_{\gamma,\text{SI}}}
\providecommand{\EphotNat}{E_{\gamma,\text{nat}}}
\providecommand{\Eabsint}{E_{\text{abs,int}}}
\providecommand{\mphoton}{m_{\gamma}}
\providecommand{\Evis}{E_{\text{vis}}}
\providecommand{\Cto}{C_{T0}}
\providecommand{\mytimes}{\times}
\providecommand{\lambdah}{\lambda_h}
\providecommand{\checkmarkx}{\checkmark}
\providecommand{\Enorm}{E_{\text{norm}}}
\providecommand{\Tobs}{T_{\text{obs}}}
\providecommand{\mobs}{m_{\text{obs}}}
\providecommand{\Eobs}{E_{\text{obs}}}
\providecommand{\Lobs}{\ell_{\text{obs}}}
\providecommand{\xobs}{\xi_{\text{obs}}}
\providecommand{\calE}{\mathcal{E}}
\providecommand{\calT}{\mathcal{T}}
\providecommand{\calM}{\mathcal{M}}
\providecommand{\alphag}{\alpha_g}
\providecommand{\Tmax}{T_{\text{max}}}
\providecommand{\mmin}{m_{\text{min}}}
\providecommand{\Lmax}{\ell_{\text{max}}}
\providecommand{\Emin}{E_{\text{min}}}
\providecommand{\Geff}{G_{\text{eff}}}
\providecommand{\rhoeff}{\rho_{\text{eff}}}
\providecommand{\xieff}{\xi_{\text{eff}}}
\providecommand{\Teff}{T_{\text{eff}}}
\providecommand{\hPlanck}{h}
\providecommand{\kB}{k_B}
\providecommand{\muB}{\mu_B}
\providecommand{\lambdaC}{\lambda_C}
\providecommand{\omegaP}{\omega_P}
\providecommand{\rhoP}{\rho_P}
\providecommand{\Tref}{T_{\text{ref}}}
\providecommand{\Eref}{E_{\text{ref}}}
\providecommand{\mref}{m_{\text{ref}}}
\providecommand{\Lref}{\ell_{\text{ref}}}
\providecommand{\xikonst}{\xi_0}
\providecommand{\Phiphoton}{\Phi_{\gamma}}
\providecommand{\etavis}{\eta_{\text{vis}}}
\providecommand{\pichar}{\pi}
\providecommand{\primrel}{\mathcal{P}_{\text{rel}}}
\providecommand{\warningx}{\textcolor{orange}{\textbf{!}}}
\providecommand{\phiT}{\phi_T}
\providecommand{\Lorentz}{\Lambda}
\providecommand{\Cconv}{C_{\text{conv}}}
\providecommand{\Df}{\Delta f}
\providecommand{\lambdazero}{\lambda_0}
\providecommand{\myapprox}{\approx}
\providecommand{\checked}{\checkmark}
\providecommand{\alphaWSI}{\alpha_W^{\text{SI}}}
\providecommand{\alphaWnat}{\alpha_W^{\text{nat}}}
\providecommand{\vect}[1]{\vec{#1}}
\providecommand{\Rzero}{R_0}
\providecommand{\Riem}{\mathcal{R}}
\providecommand{\nuzero}{\nu_0}
\providecommand{\mypi}{\pi}

% =============================================================================
% TCOLORBOX STYLES AND ENVIRONMENTS (English titles)
% =============================================================================
\tcbset{
	keyresult/.style={
		colback=blue!5!white,
		colframe=blue!75!black,
		title=Key Result,
		fonttitle=\bfseries
	},
	foundation/.style={
		colback=green!5!white,
		colframe=green!75!black,
		title=Foundation,
		fonttitle=\bfseries
	},
	alternative/.style={
		colback=orange!5!white,
		colframe=orange!75!black,
		title=Alternative,
		fonttitle=\bfseries
	},
	warningbox/.style={
		colback=red!5!white,
		colframe=red!75!black,
		title=Warning,
		fonttitle=\bfseries
	}
}

% (Here follow all your tcolorbox definitions with English titles)
\newtcolorbox{keyresultbox}[1][]{colback=blue!5!white,colframe=blue!75!black,fonttitle=\bfseries,title={#1},breakable}
\newtcolorbox{keyresult}[1][Key Result]{colback=blue!5!white,colframe=blue!75!black,fonttitle=\bfseries,title={#1},breakable}
\newtcolorbox{foundationbox}[1][]{colback=green!5!white,colframe=green!75!black,fonttitle=\bfseries,title={#1},breakable}
\newtcolorbox{foundation}[1][Foundation]{colback=green!5!white,colframe=green!75!black,fonttitle=\bfseries,title={#1},breakable}
\newtcolorbox{alternativebox}[1][]{colback=orange!5!white,colframe=orange!75!black,fonttitle=\bfseries,title={#1},breakable}
\newtcolorbox{warningboxenv}[1][Warning]{colback=red!5!white,colframe=red!75!black,fonttitle=\bfseries,title={#1},breakable}

\newtcolorbox{fundamental}[1][]{
	colback=boxgray,
	colframe=t0blue,
	fonttitle=\bfseries,
	title=#1,
	sharp corners,
	boxrule=2pt
}

\newtcolorbox{insightBox}[1][Insight]{colback=blue!5,colframe=t0blue,title={#1},fonttitle=\bfseries,breakable}
\newtcolorbox{discoveryBox}[1][Discovery]{colback=green!5,colframe=t0green,title={#1},fonttitle=\bfseries,breakable}
\newtcolorbox{revelation}[1][Revelation]{colback=red!5,colframe=t0red,title={#1},fonttitle=\bfseries,breakable}
\newtcolorbox{keypoint}[1][Key Point]{colback=blue!5,colframe=t0blue,title={#1},fonttitle=\bfseries,breakable}
\newtcolorbox{evidence}[1][Evidence]{colback=green!5,colframe=t0green,title={#1},fonttitle=\bfseries,breakable}
\newtcolorbox{conclusionBox}[1][Conclusion]{colback=gray!5,colframe=gray,title={#1},fonttitle=\bfseries,breakable}
\newtcolorbox{significance}[1][Significance]{colback=yellow!5,colframe=orange,title={#1},fonttitle=\bfseries,breakable}
\newtcolorbox{philosophical}[1][Philosophical]{colback=purple!5,colframe=purple,title={#1},fonttitle=\bfseries,breakable}
\newtcolorbox{implicationBox}[1][Implication]{colback=cyan!5,colframe=cyan,title={#1},fonttitle=\bfseries,breakable}
\newtcolorbox{perspectiveBox}[1][Perspective]{colback=blue!5,colframe=t0blue,title={#1},fonttitle=\bfseries,breakable}
\newtcolorbox{revolutionary}[1][Revolutionary]{colback=red!5,colframe=t0red,title={#1},fonttitle=\bfseries,breakable}

\newtcolorbox{technical}[1][Technical]{colback=gray!5,colframe=gray!75!black,title={#1},fonttitle=\bfseries,breakable}
\newtcolorbox{technicalBox}[1][Technical]{colback=gray!5,colframe=gray!75!black,title={#1},fonttitle=\bfseries,breakable}
\newtcolorbox{notationBox}[1][Notation]{colback=yellow!5,colframe=yellow!75!black,title={#1},fonttitle=\bfseries,breakable}
\newtcolorbox{verification}[1][Verification]{colback=orange!5!white,colframe=orange!75!black,fonttitle=\bfseries,title=#1}
\newtcolorbox{explanationBox}[1][Explanation]{colback=purple!5!white,colframe=purple!75!black,fonttitle=\bfseries,title=#1}
\newtcolorbox{interpretationBox}[1][Interpretation]{colback=cyan!5!white,colframe=cyan!75!black,fonttitle=\bfseries,title=#1}
\newtcolorbox{explanation}[1][Explanation]{colback=purple!5!white,colframe=purple!75!black,fonttitle=\bfseries,title=#1,breakable}
\newtcolorbox{interpretation}[1][Interpretation]{colback=cyan!5!white,colframe=cyan!75!black,fonttitle=\bfseries,title=#1,breakable}
\newtcolorbox{proof_step}[1][Proof Step]{colback=gray!5!white,colframe=gray!75!black,fonttitle=\bfseries,title=#1,breakable}
\newtcolorbox{experimental}[1][Experimental]{colback=teal!5!white,colframe=teal!75!black,fonttitle=\bfseries,title=#1,breakable}

\newtcolorbox{important}[1][Important]{colback=red!5!white,colframe=red!75!black,title={#1},fonttitle=\bfseries,breakable}
\newtcolorbox{warning}[1][Warning]{colback=orange!5!white,colframe=orange!75!black,title={#1},fonttitle=\bfseries,breakable}
\newtcolorbox{caution}[1][Caution]{colback=yellow!5!white,colframe=yellow!75!black,title={#1},fonttitle=\bfseries,breakable}
\newtcolorbox{highlight}[1][Highlight]{colback=yellow!10!white,colframe=yellow!75!black,title={#1},fonttitle=\bfseries,breakable}
\newtcolorbox{critical}[1][Critical]{colback=red!10!white,colframe=red!75!black,title={#1},fonttitle=\bfseries,breakable}

\newtcolorbox{analysis}[1][Analysis]{colback=blue!5!white,colframe=blue!75!black,title={#1},fonttitle=\bfseries,breakable}
\newtcolorbox{application}[1][Application]{colback=green!5!white,colframe=green!75!black,title={#1},fonttitle=\bfseries,breakable}
\newtcolorbox{experiment}[1][Experiment]{colback=cyan!5!white,colframe=cyan!75!black,title={#1},fonttitle=\bfseries,breakable}
\newtcolorbox{historical}[1][Historical]{colback=brown!5!white,colframe=brown!75!black,title={#1},fonttitle=\bfseries,breakable}
\newtcolorbox{numerical}[1][Numerical]{colback=gray!5!white,colframe=gray!75!black,title={#1},fonttitle=\bfseries,breakable}
\newtcolorbox{overview}[1][Overview]{colback=blue!5!white,colframe=blue!75!black,title={#1},fonttitle=\bfseries,breakable}
\newtcolorbox{speculation}[1][Speculation]{colback=purple!5!white,colframe=purple!75!black,title={#1},fonttitle=\bfseries,breakable}
\newtcolorbox{question}[1][Question]{colback=orange!5!white,colframe=orange!75!black,title={#1},fonttitle=\bfseries,breakable}
\newtcolorbox{method}[1][Method]{colback=teal!5!white,colframe=teal!75!black,title={#1},fonttitle=\bfseries,breakable}
\newtcolorbox{correct}[1][Correct]{colback=green!10!white,colframe=green!75!black,title={#1},fonttitle=\bfseries,breakable}
\newtcolorbox{units}[1][Units]{colback=gray!5!white,colframe=gray!75!black,title={#1},fonttitle=\bfseries,breakable}
\newtcolorbox{achievement}[1][Achievement]{colback=gold!5!white,colframe=orange!75!black,title={#1},fonttitle=\bfseries,breakable}
\newtcolorbox{equivalence}[1][Equivalence]{colback=cyan!5!white,colframe=cyan!75!black,title={#1},fonttitle=\bfseries,breakable}
\newtcolorbox{dimensional}[1][Dimensional Analysis]{colback=purple!5!white,colframe=purple!75!black,title={#1},fonttitle=\bfseries,breakable}

% === ADDITIONAL SIMPLE ENVIRONMENTS ===
\newenvironment{treatise}{\begin{quote}}{\end{quote}}
\newenvironment{gemeinsam}{\begin{quote}}{\end{quote}}
\newenvironment{vergleich}{\begin{quote}}{\end{quote}}
\newenvironment{vorteil}{\begin{quote}}{\end{quote}}
\newenvironment{common}{\begin{quote}}{\end{quote}}
\newenvironment{comparison}{\begin{quote}}{\end{quote}}
\newenvironment{advantage}{\begin{quote}}{\end{quote}}
\newenvironment{quantum}{\begin{quote}}{\end{quote}}

% === LAYOUT SETTINGS ===
\raggedbottom
\usepackage{environ}
\let\oldtabular\tabular
\let\endoldtabular\endtabular

\newenvironment{scaledtable}[1][0.85]{%
	\begingroup\footnotesize\setlength{\LTleft}{0pt}\setlength{\LTright}{0pt}%
}{%
	\endgroup%
}

\newcommand{\widetable}[1]{\resizebox{\textwidth}{!}{#1}}

% === TABLE OF CONTENTS FORMATTING ===
\renewcommand{\cftsecfont}{\color{blue}}
\renewcommand{\cftsubsecfont}{\color{blue}}
\renewcommand{\cftsecpagefont}{\color{blue}}
\renewcommand{\cftsubsecpagefont}{\color{blue}}
\renewcommand{\cfttoctitlefont}{\huge\bfseries\color{blue}}

% === DEFAULT HEADER AND FOOTER ===
\pagestyle{fancy}
\fancyhf{}
\fancyhead[L]{\textsc{T0 Theory}}
\fancyhead[R]{\textsc{J. Pascher}}
\fancyfoot[C]{\thepage}

% ==============================================================================
% End of Shared Preamble for English
% ==============================================================================
%   \begin{document}
%   ...
%   \end{document}
%
% ==============================================================================

% =============================================================================
% SECTION 1: Encoding and Language
% =============================================================================
\usepackage[utf8]{inputenc}
\usepackage[T1]{fontenc}
\usepackage[ngerman]{babel}
\usepackage{lmodern}

% =============================================================================
% SECTION 2: Page Geometry
% =============================================================================
\usepackage[a4paper, left=2.5cm, right=2.5cm, top=2.5cm, bottom=3.5cm]{geometry}
\setlength{\headheight}{15pt}

% =============================================================================
% SECTION 3: Mathematics and Physics
% =============================================================================
\usepackage{amsmath,amssymb,amsfonts,amsthm}
\usepackage{mathtools}
\usepackage{physics}
\usepackage{siunitx}
\sisetup{
    locale=US,
    group-separator={,},
    output-decimal-marker={.},
    per-mode=symbol
}

% =============================================================================
% SECTION 4: Graphics and Tables
% =============================================================================
\usepackage{graphicx}
\usepackage[table,xcdraw]{xcolor}
\usepackage{tikz}
\usetikzlibrary{arrows.meta,positioning,shapes.geometric,decorations.pathmorphing,patterns,shapes.arrows,intersections}
\usepackage{pgfplots}
\pgfplotsset{compat=1.18}
\usepackage[most]{tcolorbox}
\tcbuselibrary{breakable}
\usepackage{booktabs}
\usepackage{array}
\usepackage{longtable}
\usepackage{float}
\usepackage{adjustbox}
\usepackage{rotating}
\usepackage{tabularx}
\usepackage{makecell}
\usepackage{multirow}

% =============================================================================
% SECTION 5: Document Formatting
% =============================================================================
\usepackage{fancyhdr}
\renewcommand{\headrulewidth}{0.4pt}
\renewcommand{\footrulewidth}{0.4pt}
\usepackage{tocloft}
\usepackage{hyperref}
\hypersetup{
  colorlinks=true,
  linkcolor=black,
  citecolor=black,
  urlcolor=black,
  breaklinks=true,
  bookmarksnumbered=true,
  unicode=true
}
\usepackage{bookmark}
\usepackage{cleveref}

% Table of contents: only show chapters (not sections/subsections)
\setcounter{tocdepth}{3}  % Show sections, subsections, and subsubsections
\usepackage{microtype}
\usepackage{enumitem}
\usepackage{setspace}
\usepackage{ragged2e}
\usepackage{multicol}

% =============================================================================
% SECTION 6: Code and Algorithms
% =============================================================================
\usepackage{algorithm}
\usepackage{algorithmic}
\usepackage{listings}
\lstset{
  basicstyle=\ttfamily\footnotesize,
  breaklines=true,
  breakatwhitespace=true,
  columns=flexible,
  keepspaces=true,
  showstringspaces=false,
  frame=single,
  xleftmargin=0pt,
  xrightmargin=0pt
}
\usepackage{mdframed}

% =============================================================================
% SECTION 7: Additional Packages
% =============================================================================
\usepackage{pdflscape}
\usepackage{braket}
\usepackage{cancel}
\usepackage{caption}
\usepackage{csquotes}
\usepackage{gensymb}
\usepackage{hyphenat}
\usepackage{textcomp}
\usepackage{textgreek}
\usepackage{upgreek}
\usepackage{url}
\usepackage{slashed}
\usepackage{bm}
\usepackage{newunicodechar}

% =============================================================================
% SECTION 8: Citation Commands (Compatibility)
% =============================================================================
\providecommand{\citep}[1]{\cite{#1}}
\providecommand{\citet}[1]{\cite{#1}}

% =============================================================================
% SECTION 9: Colors
% =============================================================================
\definecolor{gold}{RGB}{255,215,0}
\definecolor{blue}{rgb}{0,0,1}
\definecolor{boxgray}{RGB}{240,240,240}
\definecolor{deepblue}{RGB}{0,0,127}
\definecolor{deepgreen}{RGB}{0,127,0}
\definecolor{deepred}{RGB}{191,0,0}
\definecolor{t0blue}{RGB}{33,150,243}
\definecolor{t0green}{RGB}{76,175,80}
\definecolor{t0orange}{RGB}{255,152,0}
\definecolor{t0purple}{RGB}{156,39,176}
\definecolor{t0red}{RGB}{244,67,54}
\definecolor{t0yellow}{RGB}{255,204,0}

% =============================================================================
% SECTION 10: Column Types
% =============================================================================
\newcolumntype{L}[1]{>{\raggedright\arraybackslash}p{#1}}
\newcolumntype{C}[1]{>{\centering\arraybackslash}p{#1}}

% =============================================================================
% SECTION 11: Unicode Character Mappings
% =============================================================================
\newunicodechar{ħ}{$\hbar$}
\newunicodechar{↔}{$\leftrightarrow$}
\newunicodechar{⇐}{$\Leftarrow$}
\newunicodechar{⇒}{$\Rightarrow$}
\newunicodechar{⇔}{$\Leftrightarrow$}
\newunicodechar{∂}{$\partial$}
\newunicodechar{∅}{$\emptyset$}
\newunicodechar{∇}{$\nabla$}
\newunicodechar{∈}{$\in$}
\newunicodechar{∉}{$\notin$}
\newunicodechar{∏}{$\prod$}
\newunicodechar{∑}{$\sum$}
% Note: √ is mapped to an empty sqrt; use \sqrt{x} for proper usage
\newunicodechar{√}{\ensuremath{\sqrt{}}}
\newunicodechar{∝}{$\propto$}
\newunicodechar{∞}{$\infty$}
\newunicodechar{∩}{$\cap$}
\newunicodechar{∪}{$\cup$}
\newunicodechar{∫}{$\int$}
\newunicodechar{≈}{$\approx$}
\newunicodechar{≠}{$\neq$}
\newunicodechar{≤}{$\leq$}
\newunicodechar{≥}{$\geq$}
\newunicodechar{ξ}{\ensuremath{\xi}}
\newunicodechar{μ}{\ensuremath{\mu}}
\newunicodechar{ψ}{\ensuremath{\psi}}
\newunicodechar{φ}{\ensuremath{\phi}}
\newunicodechar{π}{\ensuremath{\pi}}
\newunicodechar{λ}{\ensuremath{\lambda}}
\newunicodechar{Δ}{\ensuremath{\Delta}}

% =============================================================================
% SECTION 12: Hyperref Settings
% =============================================================================
\hypersetup{
    colorlinks=true,
    linkcolor=blue,
    citecolor=blue,
    urlcolor=blue,
    breaklinks=true,
    bookmarksnumbered=true,
    pdfstartview=FitH
}

% =============================================================================
% SECTION 13: Theorem Environments (English)
% =============================================================================
\theoremstyle{plain}
\newtheorem{theorem}{Theorem}[section]
\newtheorem{lemma}[theorem]{Lemma}
\newtheorem{proposition}[theorem]{Proposition}
\newtheorem{corollary}[theorem]{Corollary}

\theoremstyle{definition}
\newtheorem{definition}[theorem]{Definition}
\newtheorem{example}[theorem]{Example}
\newtheorem{insight}[theorem]{Insight}
\newtheorem{discovery}[theorem]{Discovery}
% \newtheorem{erkenntnis}[theorem]{Insight}  % Commented out - conflicts with tcolorbox environment below

\theoremstyle{remark}
\newtheorem{remark}[theorem]{Remark}
\newtheorem{axiom}{Axiom}
\newtheorem{principle}{Principle}
\newtheorem{bemerkung}[theorem]{Remark}
\newtheorem{warnung}[theorem]{Warning}

% =============================================================================
% SECTION 14: T0-Specific Commands
% =============================================================================

% --- Core T0 Fields ---
\newcommand{\Tfield}{T(x,t)}
\providecommand{\Tfieldt}{T(\vec{x},t)}
\newcommand{\Efield}{E(x,t)}
\newcommand{\mfield}{m(x,t)}
\providecommand{\vecx}{\vec{x}}

% --- Lagrangian ---
\newcommand{\Lag}{\mathcal{L}}
\newcommand{\calL}{\mathcal{L}}

% --- Greek Letters and Constants ---
\newcommand{\alphaem}{\alpha}
\newcommand{\betaT}{\beta_T}
\newcommand{\xiT}{\xi}
\newcommand{\xipar}{\xi}

% --- Energy and Planck Units ---
\newcommand{\Ezero}{E_0}
\newcommand{\EPlanck}{E_{\text{Pl}}}
\newcommand{\Mpl}{M_{\text{Pl}}}
\newcommand{\mP}{m_{\text{P}}}
\newcommand{\lP}{\ell_{\text{P}}}
\newcommand{\tP}{t_{\text{P}}}
\newcommand{\LPlanck}{\ell_{\text{Pl}}}
\newcommand{\TPlanck}{t_{\text{Pl}}}

% --- Coupling Constants ---
\newcommand{\Gnat}{G_{\text{nat}}}
\newcommand{\alphaEM}{\alpha_{\text{EM}}}
\newcommand{\alphaSI}{\alpha_{\text{SI}}}
\newcommand{\Hubble}{H_0}
\newcommand{\LCDM}{\Lambda\text{CDM}}
\newcommand{\natunits}{(nat. units)}

% --- T0 Model Parameters ---
\newcommand{\xigeom}{\xi_{\mathrm{geom}}}
\newcommand{\rzero}{r_{0}}
\newcommand{\xirat}{\xi_{\mathrm{rat}}}
\newcommand{\tzero}{t_{0}}
\newcommand{\Lambdat}{\Lambda_{\mathrm{t}}}
\newcommand{\EP}{E_{\mathrm{P}}}
\newcommand{\Emu}{E_{\mu}}
\newcommand{\Ee}{E_{e}}
\newcommand{\Etau}{E_{\tau}}
\newcommand{\alphafine}{\alpha_{\mathrm{fine}}}
\newcommand{\alphal}{\alpha_{\ell}}
\newcommand{\Lzero}{\ell_{0}}
\newcommand{\Lp}{\ell_{\mathrm{P}}}

% --- Additional T0 Commands ---
\newcommand{\Kfrak}{K_{\text{frak}}}
\newcommand{\Dfrak}{D_{\text{frak}}}
\newcommand{\betapar}{\beta_T}
\newcommand{\alphapar}{\alpha}
\newcommand{\deltafield}{\delta \phi}
\newcommand{\deltam}{\delta m}
\newcommand{\deltaE}{\delta E}
\newcommand{\Exi}{E_{\xi}}
\newcommand{\Lxi}{\ell_{\xi}}
\newcommand{\rhoCMB}{\rho_{\text{CMB}}}
\newcommand{\rhoCasimir}{\rho_{\text{Casimir}}}
\newcommand{\Leff}{L_{\text{eff}}}
\newcommand{\CQCD}{C_{\mathrm{QCD}}}
\newcommand{\Kspec}{K_{\mathrm{spec}}}
\newcommand{\Tzero}{\ensuremath{T_0}}
\newcommand{\Eabs}{E_{\text{abs}}}
\newcommand{\taupar}{\tau}

% --- Provided Commands (may be redefined elsewhere) ---
\providecommand{\xiconst}{\xi_{\text{const}}}
\providecommand{\DhiggsT}{D_{\text{Higgs-T}}}
\providecommand{\rhoE}{\rho_{E}}
\providecommand{\Echar}{E_{\text{char}}}
\providecommand{\kfrac}{k_{\text{frac}}}
\providecommand{\alphaEMSI}{\alpha_{\text{EM,SI}}}
\providecommand{\alphaEMnat}{\alpha_{\text{EM,nat}}}
\providecommand{\betaTSI}{\beta_{T,\text{SI}}}
\providecommand{\betaTnat}{\beta_{T,\text{nat}}}
\providecommand{\Gsi}{G_{\text{SI}}}
\providecommand{\xiparSI}{\xi_{\text{SI}}}
\providecommand{\xiparnat}{\xi_{\text{nat}}}
\providecommand{\meff}{m_{\text{eff}}}
\providecommand{\Tzerot}{T_{0}(t)}
\providecommand{\mzerot}{m_{0}(t)}
\providecommand{\Ezeroabs}{E_{0,\text{abs}}}
\providecommand{\Epar}{E_{\text{par}}}
\providecommand{\Lnat}{\ell_{\text{nat}}}
\providecommand{\Tnat}{T_{\text{nat}}}
\providecommand{\xifrak}{\xi_{\text{frac}}}
\providecommand{\Tfrak}{T_{\text{frac}}}
\providecommand{\mfrak}{m_{\text{frac}}}
\providecommand{\Dfrac}{D_{\text{frac}}}
\providecommand{\EphotSI}{E_{\gamma,\text{SI}}}
\providecommand{\EphotNat}{E_{\gamma,\text{nat}}}
\providecommand{\Eabsint}{E_{\text{abs,int}}}
\providecommand{\mphoton}{m_{\gamma}}
\providecommand{\Evis}{E_{\text{vis}}}
\providecommand{\Cto}{C_{T0}}
\providecommand{\mytimes}{\times}
\providecommand{\lambdah}{\lambda_h}
\providecommand{\checkmarkx}{\checkmark}
\providecommand{\Enorm}{E_{\text{norm}}}
\providecommand{\Tobs}{T_{\text{obs}}}
\providecommand{\mobs}{m_{\text{obs}}}
\providecommand{\Eobs}{E_{\text{obs}}}
\providecommand{\Lobs}{\ell_{\text{obs}}}
\providecommand{\xobs}{\xi_{\text{obs}}}
\providecommand{\calE}{\mathcal{E}}
\providecommand{\calT}{\mathcal{T}}
\providecommand{\calM}{\mathcal{M}}
\providecommand{\alphag}{\alpha_g}
\providecommand{\Tmax}{T_{\text{max}}}
\providecommand{\mmin}{m_{\text{min}}}
\providecommand{\Lmax}{\ell_{\text{max}}}
\providecommand{\Emin}{E_{\text{min}}}
\providecommand{\Geff}{G_{\text{eff}}}
\providecommand{\rhoeff}{\rho_{\text{eff}}}
\providecommand{\xieff}{\xi_{\text{eff}}}
\providecommand{\Teff}{T_{\text{eff}}}
\providecommand{\hPlanck}{h}
\providecommand{\kB}{k_B}
\providecommand{\muB}{\mu_B}
\providecommand{\lambdaC}{\lambda_C}
\providecommand{\omegaP}{\omega_P}
\providecommand{\rhoP}{\rho_P}
\providecommand{\Tref}{T_{\text{ref}}}
\providecommand{\Eref}{E_{\text{ref}}}
\providecommand{\mref}{m_{\text{ref}}}
\providecommand{\Lref}{\ell_{\text{ref}}}
\providecommand{\xikonst}{\xi_0}
\providecommand{\Phiphoton}{\Phi_{\gamma}}
\providecommand{\etavis}{\eta_{\text{vis}}}
\providecommand{\pichar}{\pi}
\providecommand{\primrel}{\mathcal{P}_{\text{rel}}}
\providecommand{\warningx}{\textcolor{orange}{\textbf{!}}}
\providecommand{\phiT}{\phi_T}
\providecommand{\Lorentz}{\Lambda}
\providecommand{\Cconv}{C_{\text{conv}}}
\providecommand{\Df}{\Delta f}
\providecommand{\lambdazero}{\lambda_0}
\providecommand{\myapprox}{\approx}
\providecommand{\checked}{\checkmark}
\providecommand{\alphaWSI}{\alpha_W^{\text{SI}}}
\providecommand{\alphaWnat}{\alpha_W^{\text{nat}}}
\providecommand{\vect}[1]{\vec{#1}}
\providecommand{\Rzero}{R_0}
\providecommand{\Riem}{\mathcal{R}}
\providecommand{\nuzero}{\nu_0}
\providecommand{\mypi}{\pi}

% =============================================================================
% SECTION 15: tcolorbox Styles and Environments
% =============================================================================

% --- Predefined Styles ---
\tcbset{
    keyresult/.style={
        colback=blue!5!white,
        colframe=blue!75!black,
        title=Key Result,
        fonttitle=\bfseries
    },
    foundation/.style={
        colback=green!5!white,
        colframe=green!75!black,
        title=Foundation,
        fonttitle=\bfseries
    },
    alternative/.style={
        colback=orange!5!white,
        colframe=orange!75!black,
        title=Alternative,
        fonttitle=\bfseries
    },
    warningbox/.style={
        colback=red!5!white,
        colframe=red!75!black,
        title=Warning,
        fonttitle=\bfseries
    }
}

% --- Core Environments ---
\newtcolorbox{keyresultbox}[1][]{colback=blue!5!white,colframe=blue!75!black,fonttitle=\bfseries,title={#1},breakable}
\newtcolorbox{keyresult}[1][Key Result]{colback=blue!5!white,colframe=blue!75!black,fonttitle=\bfseries,title={#1},breakable}
\newtcolorbox{foundationbox}[1][]{colback=green!5!white,colframe=green!75!black,fonttitle=\bfseries,title={#1},breakable}
\newtcolorbox{foundation}[1][Foundation]{colback=green!5!white,colframe=green!75!black,fonttitle=\bfseries,title={#1},breakable}
\newtcolorbox{alternativebox}[1][]{colback=orange!5!white,colframe=orange!75!black,fonttitle=\bfseries,title={#1},breakable}
\newtcolorbox{warningboxenv}[1][]{colback=red!5!white,colframe=red!75!black,fonttitle=\bfseries,title={#1},breakable}

% --- Formula Environments ---
\newtcolorbox{fundamental}[1][]{
    colback=boxgray,
    colframe=t0blue,
    fonttitle=\bfseries,
    title=#1,
    sharp corners,
    boxrule=2pt
}

\newtcolorbox{newperspective}[1][]{
    colback=red!5!white,
    colframe=t0red,
    fonttitle=\bfseries,
    title=#1,
    sharp corners,
    boxrule=2pt
}

\newtcolorbox{formula}[1][]{
    colback=blue!5!white,
    colframe=blue!75!black,
    fonttitle=\bfseries,
    title=#1
}

\newtcolorbox{result}[1][]{
    colback=green!5!white,
    colframe=green!75!black,
    fonttitle=\bfseries,
    title=#1
}

\newtcolorbox{derivation}[1][]{
    colback=green!5!white,
    colframe=green!75!black,
    title=#1,
    fonttitle=\bfseries,
    breakable
}

\newtcolorbox{summary}[1][]{
    colback=gray!10!white,
    colframe=gray!75!black,
    title=#1,
    fonttitle=\bfseries,
    breakable
}

\newtcolorbox{comparison}[1][]{
    colback=purple!5!white,
    colframe=purple!75!black,
    title=#1,
    fonttitle=\bfseries,
    breakable
}

\newtcolorbox{relation}[1][]{
    colback=cyan!5!white,
    colframe=cyan!75!black,
    title=#1,
    fonttitle=\bfseries,
    breakable
}

\newtcolorbox{principleBox}[1][]{
    colback=yellow!5!white,
    colframe=yellow!75!black,
    title=#1,
    fonttitle=\bfseries,
    breakable
}

% --- Insight and Discovery Environments ---
\newtcolorbox{insightBox}[1][]{colback=blue!5,colframe=t0blue,title={#1},fonttitle=\bfseries,breakable}
\newtcolorbox{discoveryBox}[1][]{colback=green!5,colframe=t0green,title={#1},fonttitle=\bfseries,breakable}
\newtcolorbox{revelation}[1][]{colback=red!5,colframe=t0red,title={#1},fonttitle=\bfseries,breakable}
\newtcolorbox{keypoint}[1][]{colback=blue!5,colframe=t0blue,title={#1},fonttitle=\bfseries,breakable}
\newtcolorbox{evidence}[1][]{colback=green!5,colframe=t0green,title={#1},fonttitle=\bfseries,breakable}
\newtcolorbox{conclusionBox}[1][]{colback=gray!5,colframe=gray,title={#1},fonttitle=\bfseries,breakable}
\newtcolorbox{significance}[1][]{colback=yellow!5,colframe=orange,title={#1},fonttitle=\bfseries,breakable}
\newtcolorbox{philosophical}[1][]{colback=purple!5,colframe=purple,title={#1},fonttitle=\bfseries,breakable}
\newtcolorbox{implicationBox}[1][]{colback=cyan!5,colframe=cyan,title={#1},fonttitle=\bfseries,breakable}
\newtcolorbox{perspectiveBox}[1][]{colback=blue!5,colframe=t0blue,title={#1},fonttitle=\bfseries,breakable}
\newtcolorbox{revolutionary}[1][]{colback=red!5,colframe=t0red,title={#1},fonttitle=\bfseries,breakable}

% --- Technical Environments ---
\newtcolorbox{technical}[1][]{colback=gray!5,colframe=gray!75!black,title={#1},fonttitle=\bfseries,breakable}
\newtcolorbox{technicalBox}[1][]{colback=gray!5,colframe=gray!75!black,title={#1},fonttitle=\bfseries,breakable}
\newtcolorbox{notationBox}[1][]{colback=yellow!5,colframe=yellow!75!black,title={#1},fonttitle=\bfseries,breakable}
\newtcolorbox{verification}[1][]{colback=orange!5!white,colframe=orange!75!black,fonttitle=\bfseries,title=#1}
\newtcolorbox{explanationBox}[1][]{colback=purple!5!white,colframe=purple!75!black,fonttitle=\bfseries,title=#1}
\newtcolorbox{interpretationBox}[1][]{colback=cyan!5!white,colframe=cyan!75!black,fonttitle=\bfseries,title=#1}
\newtcolorbox{explanation}[1][]{colback=purple!5!white,colframe=purple!75!black,fonttitle=\bfseries,title=#1,breakable}
\newtcolorbox{interpretation}[1][]{colback=cyan!5!white,colframe=cyan!75!black,fonttitle=\bfseries,title=#1,breakable}
\newtcolorbox{proof_step}[1][]{colback=gray!5!white,colframe=gray!75!black,fonttitle=\bfseries,title=#1,breakable}
\newtcolorbox{experimental}[1][]{colback=teal!5!white,colframe=teal!75!black,fonttitle=\bfseries,title=#1,breakable}

% --- Warning and Alert Environments ---
\newtcolorbox{important}[1][]{colback=red!5!white,colframe=red!75!black,title={#1},fonttitle=\bfseries,breakable}
\newtcolorbox{warning}[1][]{colback=orange!5!white,colframe=orange!75!black,title={#1},fonttitle=\bfseries,breakable}
\newtcolorbox{caution}[1][]{colback=yellow!5!white,colframe=yellow!75!black,title={#1},fonttitle=\bfseries,breakable}
\newtcolorbox{highlight}[1][]{colback=yellow!10!white,colframe=yellow!75!black,title={#1},fonttitle=\bfseries,breakable}

% --- Additional German-specific Environments for Matsas documents ---
\newtcolorbox{literatur}[1][Literatur]{colback=blue!5!white,colframe=blue!75!black,title={#1},fonttitle=\bfseries,breakable}
\newtcolorbox{zusammenfassung}[1][Zusammenfassung]{colback=green!5!white,colframe=green!75!black,title={#1},fonttitle=\bfseries,breakable}
\newtcolorbox{frage}[1][Frage]{colback=orange!5!white,colframe=orange!75!black,title={#1},fonttitle=\bfseries,breakable}
\newtcolorbox{erkenntnis}[1][Erkenntnis]{colback=purple!5!white,colframe=purple!75!black,title={#1},fonttitle=\bfseries,breakable}
\newtcolorbox{critical}[1][]{colback=red!10!white,colframe=red!75!black,title={#1},fonttitle=\bfseries,breakable}

% --- Analysis and Application Environments ---
\newtcolorbox{analysis}[1][]{colback=blue!5!white,colframe=blue!75!black,title={#1},fonttitle=\bfseries,breakable}
\newtcolorbox{application}[1][]{colback=green!5!white,colframe=green!75!black,title={#1},fonttitle=\bfseries,breakable}
\newtcolorbox{experiment}[1][]{colback=cyan!5!white,colframe=cyan!75!black,title={#1},fonttitle=\bfseries,breakable}
\newtcolorbox{historical}[1][]{colback=brown!5!white,colframe=brown!75!black,title={#1},fonttitle=\bfseries,breakable}
\newtcolorbox{numerical}[1][]{colback=gray!5!white,colframe=gray!75!black,title={#1},fonttitle=\bfseries,breakable}
\newtcolorbox{overview}[1][]{colback=blue!5!white,colframe=blue!75!black,title={#1},fonttitle=\bfseries,breakable}
\newtcolorbox{speculation}[1][]{colback=purple!5!white,colframe=purple!75!black,title={#1},fonttitle=\bfseries,breakable}
\newtcolorbox{question}[1][]{colback=orange!5!white,colframe=orange!75!black,title={#1},fonttitle=\bfseries,breakable}
\newtcolorbox{method}[1][]{colback=teal!5!white,colframe=teal!75!black,title={#1},fonttitle=\bfseries,breakable}
\newtcolorbox{correct}[1][]{colback=green!10!white,colframe=green!75!black,title={#1},fonttitle=\bfseries,breakable}
\newtcolorbox{units}[1][]{colback=gray!5!white,colframe=gray!75!black,title={#1},fonttitle=\bfseries,breakable}
\newtcolorbox{achievement}[1][]{colback=gold!5!white,colframe=orange!75!black,title={#1},fonttitle=\bfseries,breakable}
\newtcolorbox{equivalence}[1][]{colback=cyan!5!white,colframe=cyan!75!black,title={#1},fonttitle=\bfseries,breakable}
\newtcolorbox{dimensional}[1][]{colback=purple!5!white,colframe=purple!75!black,title={#1},fonttitle=\bfseries,breakable}

% --- Physics-specific Environments ---
\newtcolorbox{photon}[1][]{colback=yellow!5!white,colframe=yellow!75!black,title={#1},fonttitle=\bfseries,breakable}
\newtcolorbox{neutrino}[1][]{colback=blue!5!white,colframe=blue!75!black,title={#1},fonttitle=\bfseries,breakable}
\newtcolorbox{revolution}[1][]{colback=red!5!white,colframe=red!75!black,title={#1},fonttitle=\bfseries,breakable}
\newtcolorbox{t0box}[1][]{colback=blue!5!white,colframe=t0blue,title={#1},fonttitle=\bfseries,breakable}
\newtcolorbox{documentbox}[1][]{colback=gray!5!white,colframe=gray!75!black,title={#1},fonttitle=\bfseries,breakable}
\newtcolorbox{sibox}[1][]{colback=green!5!white,colframe=green!75!black,title={#1},fonttitle=\bfseries,breakable}
\newtcolorbox{smbox}[1][]{colback=blue!5!white,colframe=blue!75!black,title={#1},fonttitle=\bfseries,breakable}
\newtcolorbox{pvbox}[1][]{colback=purple!5!white,colframe=purple!75!black,title={#1},fonttitle=\bfseries,breakable}
\newtcolorbox{koidebox}[1][]{colback=orange!5!white,colframe=orange!75!black,title={#1},fonttitle=\bfseries,breakable}

% --- German Compatibility Environments ---
\newtcolorbox{formel}[1][]{colback=blue!5!white,colframe=blue!75!black,title={#1},fonttitle=\bfseries,breakable}
\newtcolorbox{schluessel}[1][]{colback=blue!5!white,colframe=blue!75!black,title={#1},fonttitle=\bfseries,breakable}
\newtcolorbox{wichtig}[1][]{colback=red!5!white,colframe=red!75!black,title={#1},fonttitle=\bfseries,breakable}
\newtcolorbox{vorsicht}[1][]{colback=orange!5!white,colframe=orange!75!black,title={#1},fonttitle=\bfseries,breakable}
\newtcolorbox{revolutionaer}[1][]{colback=red!5!white,colframe=red!75!black,title={#1},fonttitle=\bfseries,breakable}
\newtcolorbox{numerisch}[1][]{colback=gray!5!white,colframe=gray!75!black,title={#1},fonttitle=\bfseries,breakable}
\newtcolorbox{experimentell}[1][]{colback=cyan!5!white,colframe=cyan!75!black,title={#1},fonttitle=\bfseries,breakable}
\newtcolorbox{anwendung}[1][]{colback=green!5!white,colframe=green!75!black,title={#1},fonttitle=\bfseries,breakable}
\newtcolorbox{alternative}[1][]{colback=orange!5!white,colframe=orange!75!black,title={#1},fonttitle=\bfseries,breakable}
\newtcolorbox{beziehung}[1][]{colback=cyan!5!white,colframe=cyan!75!black,title={#1},fonttitle=\bfseries,breakable}
\newtcolorbox{folgerung}[1][]{colback=green!5!white,colframe=green!75!black,title={#1},fonttitle=\bfseries,breakable}
\newtcolorbox{abhandlung}[1][]{colback=gray!5!white,colframe=gray!75!black,title={#1},fonttitle=\bfseries,breakable}
\newtcolorbox{prinzipBox}[1][]{colback=blue!5!white,colframe=blue!75!black,title={#1},fonttitle=\bfseries,breakable}
\newtcolorbox{prinzip}[1][]{colback=blue!5!white,colframe=blue!75!black,title={#1},fonttitle=\bfseries,breakable}
\newtcolorbox{beweis}[1][]{colback=gray!5!white,colframe=gray!75!black,title={#1},fonttitle=\bfseries,breakable}
\newtcolorbox{key}[2][]{colback=blue!5!white,colframe=blue!75!black,title={#2},fonttitle=\bfseries,breakable}
\newtcolorbox{category}[1][]{colback=purple!5!white,colframe=purple!75!black,title={#1},fonttitle=\bfseries,breakable}

% =============================================================================
% SECTION 16: Additional Simple Environments
% =============================================================================
\newenvironment{treatise}{\begin{quote}}{\end{quote}}
\newenvironment{gemeinsam}{\begin{quote}}{\end{quote}}
\newenvironment{vergleich}{\begin{quote}}{\end{quote}}
\newenvironment{vorteil}{\begin{quote}}{\end{quote}}
\newenvironment{quantum}{\begin{quote}}{\end{quote}}

% =============================================================================
% SECTION 17: Layout Settings (Kindle-compatible)
% =============================================================================
\sloppy  % Allow more flexible line breaking
\hfuzz=65pt  % Suppress overfull warnings up to 65pt (Kindle compatibility)
\vfuzz=65pt  
\tolerance=9999  % High tolerance for bad line breaks
\emergencystretch=3em  % Extra stretch to avoid overfull boxes
\hbadness=10000  % Suppress underfull box warnings
\raggedbottom

% Environment for wide tables/longtables that need scaling
\newenvironment{scaledtable}[1][0.85]{%
  \begingroup\footnotesize\setlength{\LTleft}{0pt}\setlength{\LTright}{0pt}%
}{%
  \endgroup%
}

% Command for inline table scaling
\newcommand{\widetable}[1]{\resizebox{\textwidth}{!}{#1}}

% =============================================================================
% SECTION 18: Table of Contents Formatting
% =============================================================================
\renewcommand{\cftsecfont}{\color{blue}}
\renewcommand{\cftsubsecfont}{\color{blue}}
\renewcommand{\cftsecpagefont}{\color{blue}}
\renewcommand{\cftsubsecpagefont}{\color{blue}}
\renewcommand{\cfttoctitlefont}{\huge\bfseries\color{blue}}

% =============================================================================
% SECTION 19: Default Header and Footer
% =============================================================================
\pagestyle{fancy}
\fancyhf{}
\fancyhead[L]{\textsc{T0 Theory}}
\fancyhead[R]{\textsc{J. Pascher}}
\fancyfoot[C]{\thepage}

% ==============================================================================
% End of Shared Preamble
% ==============================================================================


\title{\textbf{Kapitel 12: Kosmologie und der Big-Bang-Phasenübergang} \\
\large Das Universum als sich vertiefendes Gehirn \\
\normalsize Narrative Version der FFGFT}
\author{}
\\\date{1. Januar 2025}

\begin{document}

\maketitle

\section*{Einleitung}

Stellen Sie sich vor, Sie betrachten ein sich entwickelndes Gehirn -- nicht von außen, sondern Sie befinden sich mitten darin. Was würden Sie wahrnehmen? Keine Expansion, kein Wachstum nach außen, sondern etwas viel Faszinierend

eres: Die Oberfläche faltet sich, Windungen vertiefen sich, neue Verbindungen entstehen überall gleichzeitig. Das Volumen bleibt konstant, doch die Komplexität -- die innere Struktur -- wächst dramatisch.

Genau so verhält es sich mit unserem Universum in der Fundamentalen Fraktalgeometrischen Feldtheorie (FFGFT). Was wir als ``kosmische Expansion'' interpretieren, ist in Wahrheit eine Vertiefung der fraktalen Struktur der Raumzeit selbst. \textbf{Der Raum dehnt sich nicht aus -- er entfaltet sich in zunehmender fraktaler Komplexität.}

\textbf{Zentrale Metapher:} Das Universum verhält sich wie ein wachsendes Gehirn, dessen Windungen (fraktale Komplexität) zunehmen, während das Gesamtvolumen konstant bleibt. Der Big Bang war kein explosiver Anfang, sondern ein Phasenübergang -- der Moment, in dem das ``kosmische Gehirn'' zu ``denken'' begann.

\section{Die fundamentale Täuschung: Expansion ohne Bewegung}

In der Standardkosmologie wird uns gelehrt, dass sich der Raum selbst ausdehnt, dass Galaxien wie Rosinen in einem aufgehenden Teig auseinanderdriften. Doch diese Vorstellung beruht auf einer grundlegenden Fehlinterpretation der Beobachtungen.

\subsection{Was wir wirklich beobachten}

Wenn Astronomen ferne Galaxien beobachten, sehen sie eine systematische Verschiebung von Spektrallinien ins Rote -- die sogenannte Rotverschiebung $z$. Je weiter die Galaxie, desto größer die Rotverschiebung. Im Standardmodell wird dies als Doppler-Effekt gedeutet: Die Galaxien fliehen von uns weg, weil der Raum sich ausdehnt.

Doch die FFGFT bietet eine radikal andere Erklärung. Die Rotverschiebung entsteht nicht durch Bewegung im Raum, sondern durch eine \textit{Änderung der fraktalen Skalenstruktur} der Raumzeit selbst zwischen Emission und Beobachtung des Lichts.

\subsection{Fraktale Rotverschiebung}

Die mathematische Beschreibung ist präzise und elegant:

\begin{equation}
1 + z = \frac{\lambda_{\text{obs}}}{\lambda_{\text{em}}} = \left(\frac{\xi(t_{\text{em}})}{\xi(t_{\text{obs}})}\right)^{-k} = e^{k \cdot \Delta \ln \xi}
\end{equation}

Lassen Sie uns diese Gleichung Schritt für Schritt verstehen:

\begin{itemize}[leftmargin=*]
\item $z$ ist die beobachtete Rotverschiebung -- eine dimensionslose Zahl, die angibt, wie stark das Licht ins Rote verschoben ist
\item $\lambda_{\text{obs}}$ ist die Wellenlänge, die wir heute messen, $\lambda_{\text{em}}$ die ursprünglich emittierte Wellenlänge
\item $\xi(t)$ ist unser fundamentaler fraktaler Skalenparameter (erinnern Sie sich: $\xi = \frac{4}{3} \times 10^{-4}$), der aber leicht mit der Zeit variiert
\item $k$ beschreibt die Hierarchiestufe in der fraktalen Selbstähnlichkeit -- eine ganze Zahl
\item $\Delta \ln \xi$ ist die Änderung des logarithmischen Skalenparameters zwischen Emission und Beobachtung
\end{itemize}

\textbf{Die physikalische Interpretation:} Das Licht einer fernen Galaxie reist nicht einfach durch expandierenden Raum. Stattdessen durchquert es Schichten zunehmend veränderter fraktaler Tiefe. Wie eine Melodie, die durch ein sich langsam verformendes Medium wandert und dadurch tiefer klingt, so wird das Licht durch die sich vertiefende fraktale Struktur rotverschoben.

Es gibt keine Bewegung, keine Flucht -- nur eine Perspektivänderung durch die dynamische Geometrie.

\subsection{Die scheinbare Hubble-Konstante}

Aus dieser fraktalen Rotverschiebung folgt direkt das, was wir als Hubble-Expansion interpretieren:

\begin{equation}
H_0 = \left|\frac{\dot{\xi}}{\xi}\right|_{t_0} \cdot c \approx 70 \, \text{km/s/Mpc}
\end{equation}

Hier ist $\dot{\xi}$ die Änderungsrate von $\xi$ (das Pünktchen bedeutet Zeitableitung), und $c$ ist die Lichtgeschwindigkeit. Der Wert $\dot{\xi}/\xi \approx -2.27 \times 10^{-18} \, \text{s}^{-1}$ ist winzig -- das entspricht einer Änderung von etwa 0.000007\% pro Million Jahre.

Dennoch summiert sich diese winzige Änderung über kosmische Zeitskalen zu dem auf, was wir als Hubble-Expansion beobachten. Der entscheidende Unterschied: Es ist keine wirkliche Expansion, sondern eine geometrische Skalenverschiebung.

\section{Der Big Bang als fraktaler Phasenübergang}

In der FFGFT ist der Big Bang kein Moment der Schöpfung aus dem Nichts, keine explodierende Singularität. Stattdessen war er ein \textit{Phasenübergang} -- vergleichbar mit dem Moment, in dem Wasser zu Eis gefriert oder eine übersättigte Lösung plötzlich auskristallisiert.

\subsection{Das fundamentale Vakuumfeld}

Das Vakuum -- der scheinbar leere Raum -- ist in der FFGFT alles andere als leer. Es ist ein dynamisches Feld, beschrieben durch:

\begin{equation}
\Phi = \rho(x,t) e^{i\theta(x,t)}
\end{equation}

Dies ist eine komplexe Zahl mit zwei Komponenten:
\begin{itemize}[leftmargin=*]
\item $\rho(x,t)$ -- die Amplitude, die Dichte des Vakuumsubstrats (denken Sie an die ``Dicke'' des Gewebes)
\item $\theta(x,t)$ -- die Phase, die Zeitstruktur (denken Sie an die ``Schwingung'' oder den ``Rhythmus'')
\end{itemize}

Die \textbf{Zeit-Masse-Dualität} manifestiert sich in diesem Feld als fundamentale Beziehung:

\begin{equation}
T(x,t) \cdot m(x,t) = 1
\end{equation}

mit $T \propto \theta$ (Zeitstruktur) und $m \propto \rho^2$ (Massendichte).

Diese Gleichung sagt etwas Tiefgründiges: Dort wo viel Zeit ``ist'', gibt es wenig Masse -- und umgekehrt. Zeit und Masse sind komplementäre Aspekte desselben Vakuumfeldes, wie zwei Seiten einer Münze.

\subsection{Die drei Phasen des Universums}

Der Big Bang war der Übergang zwischen drei fundamentalen Zuständen des Vakuums:

\textbf{1. Prä-Phasenübergang ($t < t_{\text{BB}}$):} Das ``schlafende'' Universum

\begin{itemize}[leftmargin=*]
\item $\rho \approx 0$: Das Vakuum ist nahezu ohne Substanz, wie ein extrem dünnes Gewebe
\item $\theta$: Die Phase fluktuiert wild und ungeordnet -- chaotische Zeitstruktur ohne Kohärenz
\item Fraktale Tiefe: Minimal, $D_f \approx 2$ -- das Universum ist stark ``unterdimensioniert'', flach wie ein Blatt Papier
\end{itemize}

Stellen Sie sich ein Gehirn vor der Entwicklung vor -- eine glatte Oberfläche ohne Windungen, ohne Struktur, ohne Funktion.

\textbf{2. Der Phasenübergang ($t = t_{\text{BB}}$):} Das ``Erwachen''

\begin{itemize}[leftmargin=*]
\item Instabilität: $\rho$ wächst plötzlich exponentiell -- das Vakuum verdichtet sich
\item $\theta$ ordnet sich: Aus Chaos entsteht Ordnung, eine kohärente Zeitstruktur
\item Die fraktale Dimension stabilisiert: $D_f = 3 - \xi_0 \approx 2.999867$
\end{itemize}

Dies ist der Moment, in dem das ``kosmische Gehirn'' zu ``denken'' beginnt -- aus ungeordneter Potentialität wird strukturierte Realität. Keine Explosion, sondern Organisation.

\textbf{3. Post-Phasenübergang ($t > t_{\text{BB}}$):} Das sich entwickelnde Universum

\begin{itemize}[leftmargin=*]
\item $\rho = \rho_0 = \frac{\sqrt{\hbar c}}{l_P^{3/2}} \cdot \xi^{-2}$: Die Vakuumdichte stabilisiert sich auf einem konstanten Wert
\item $\theta$: Gleichmäßige, kohärente Zeitentwicklung
\item Fraktale Tiefe: $D_f = 3 - \xi(t)$ mit langsam variierendem $\xi(t)$ -- das Universum ``vertieft'' sich weiter
\end{itemize}

Wie ein reifendes Gehirn bildet das Universum immer komplexere Strukturen aus, ohne sein grundlegendes Volumen zu ändern.

\section{Die fraktale Metrik: Statisch, aber dynamisch}

Die Metrik -- die mathematische Beschreibung der Raumzeit-Geometrie -- sieht in der FFGFT anders aus als im Standardmodell:

\begin{equation}
ds^2 = -c^2 dt^2 + \left(\frac{\xi(t_0)}{\xi(t)}\right)^{2/D_f} \left[dr^2 + r^2 d\Omega^2\right]
\end{equation}

Diese Gleichung beschreibt, wie Abstände in Raum und Zeit gemessen werden. Lassen Sie uns die Komponenten verstehen:

\begin{itemize}[leftmargin=*]
\item $ds^2$ ist das ``Linienelement'' -- der infinitesimale Abstand zwischen zwei Ereignissen in der Raumzeit
\item $-c^2 dt^2$ ist der zeitliche Anteil (das Minus-Zeichen ist eine Konvention der Relativitätstheorie)
\item Der räumliche Anteil wird durch den Faktor $(\xi(t_0)/\xi(t))^{2/D_f}$ modifiziert
\end{itemize}

\textbf{Der entscheidende Punkt:} Wenn $\xi$ konstant wäre, würde diese Metrik zur flachen Minkowski-Metrik der speziellen Relativitätstheorie reduzieren -- keinerlei Expansion. Aber $\xi$ ändert sich leicht mit der Zeit, und dieser Faktor erzeugt die \textit{Illusion} von Expansion.

Die ``Skalenfunktion'' des Standardmodells, normalerweise $a(t)$ genannt, wird hier ersetzt durch:

\begin{equation}
a_{\text{eff}}(t) = \left(\frac{\xi(t_0)}{\xi(t)}\right)^{1/D_f}
\end{equation}

Diese Größe beschreibt keine physikalische Ausdehnung, sondern unsere \textit{Wahrnehmung} der fraktalen Skalen. Es ist wie bei einem Zoom in ein Fraktal: Die Struktur ändert sich, erscheint größer oder kleiner, aber das Fraktal selbst expandiert nicht.

\section{Wie sich $\xi$ entwickelt}

Die Zeitabhängigkeit von $\xi$ ist nicht willkürlich, sondern folgt aus der Stabilität des Vakuums. Die Differentialgleichung lautet:

\begin{equation}
\frac{d\xi}{dt} = -\frac{\xi^2}{\tau_0} \cdot \left(1 - \frac{\xi}{\xi_{\infty}}\right)
\end{equation}

Diese Gleichung sagt: $\xi$ nimmt mit der Zeit ab (das Minuszeichen), aber die Abnahmerate wird kleiner, je näher $\xi$ dem Endwert $\xi_{\infty}$ kommt. Es ist wie ein Pendel, das zur Ruhe kommt, oder Wasser, das in ein Tal fließt und dort zur Ruhe kommt.

Die Lösung dieser Gleichung ist:

\begin{equation}
\xi(t) = \frac{\xi_0 \xi_{\infty} e^{-t/\tau_0}}{\xi_{\infty} - \xi_0 + \xi_0 e^{-t/\tau_0}}
\end{equation}

Mit den Parametern:
\begin{itemize}[leftmargin=*]
\item $\xi_0 = \frac{4}{3} \times 10^{-4}$: Der Anfangswert beim Big Bang
\item $\xi_{\infty} \approx 1.2 \times 10^{-4}$: Der Endwert für $t \to \infty$ (in ferner Zukunft)
\item $\tau_0 = \frac{\hbar}{m_P c^2 \xi_0^2} \approx 4.3 \times 10^{17} \, \text{s}$: Die charakteristische Zeit (etwa 14 Milliarden Jahre!)
\end{itemize}

Das Universum ist also in einem langsamen Übergang begriffen -- es ``vertieft'' sich asymptotisch einem Endzustand, den es nie ganz erreichen wird.

\section{Der kosmische Mikrowellenhintergrund: Echos des Phasenübergangs}

Der kosmische Mikrowellenhintergrund (CMB) -- die 2.7 Kelvin-Strahlung, die aus allen Richtungen kommt -- gilt als ``Echo des Urknalls''. Doch in der FFGFT ist seine Herkunft anders:

Die CMB entsteht nicht aus einer heißen Urphase (die es nie gab), sondern aus \textit{fraktalen Vakuumfluktuationen} unmittelbar nach dem Phasenübergang.

Die Temperaturverteilung am Himmel wird beschrieben durch:

\begin{equation}
T_{\text{CMB}}(\theta, \phi) = T_0 \left[1 + \sum_{l,m} a_{lm} Y_{lm}(\theta, \phi)\right]
\end{equation}

Hier sind $Y_{lm}$ sphärische Harmonische -- mathematische Funktionen, die Muster auf einer Kugel beschreiben, ähnlich wie Obertöne auf einer Gitarrensaite. Die Koeffizienten $a_{lm}$ geben an, wie stark jedes Muster beiträgt.

In der FFGFT kommen diese Koeffizienten von fraktalen Dichtefluktuationen:

\begin{equation}
a_{lm} \propto \int \frac{\delta \rho(\vec{x})}{\rho_0} \cdot j_l(kr) \cdot Y_{lm}^*(\theta, \phi) d^3x
\end{equation}

mit den fraktalen Dichtefluktuationen:

\begin{equation}
\frac{\delta \rho(\vec{x})}{\rho_0} = \xi \cdot \sum_n \frac{\cos(2\pi |\vec{x} - \vec{x}_n|/\lambda_n)}{|\vec{x} - \vec{x}_n|^{D_f/2}}
\end{equation}

\textbf{Die physikalische Bedeutung:} Die Temperatur-Anisotropien im CMB sind keine Relikte einer heißen Phase, sondern \textit{stehende Wellen} in der fraktalen Vakuumstruktur -- ähnlich wie die charakteristischen Klangmuster einer Kirchenglocke ihre Form widerspiegeln.

Das Maximum bei $l \approx 220$ (beobachtet und bestätigt durch Satelliten wie WMAP und Planck) entsteht aus fraktaler Resonanz bei der Skala:

\begin{equation}
\lambda_{\text{res}} = \frac{2\pi c}{H_0} \cdot \frac{D_f}{2} \approx 1.1 \times 10^{26} \, \text{m}
\end{equation}

Dies ist die natürliche Resonanzskala des fraktalen Vakuums -- kein Zufall, sondern geometrische Notwendigkeit.

\section{Baryonische Akustische Oszillationen: Das kosmische Netz}

Wenn Sie die Verteilung von Millionen Galaxien im Raum kartieren, sehen Sie etwas Erstaunliches: Sie sind nicht zufällig verteilt, sondern bilden ein Netz -- Filamente und Voids, Fäden und Blasen, wie Schaum oder wie... ein neuronales Netzwerk.

Diese Struktur zeigt charakteristische Skalen, die sogenannten Baryonischen Akustischen Oszillationen (BAO). In der FFGFT entstehen diese aus stehenden fraktalen Wellen:

\begin{equation}
r_{\text{BAO}} = \frac{\pi c}{H_0} \cdot \frac{1}{\sqrt{1 - \xi/2}} \approx 150 \, \text{Mpc}
\end{equation}

Diese Skala (etwa 150 Megaparsec, rund 490 Millionen Lichtjahre) erscheint als Peak in der Galaxienkorrelationsfunktion:

\begin{equation}
\xi_{\text{gal}}(r) \propto \frac{\sin(r/r_{\text{BAO}})}{r/r_{\text{BAO}}} \cdot r^{-(3-D_f)}
\end{equation}

Die Galaxienverteilung ist also kein evolutionäres Produkt der Gravitation, die aus winzigen Dichteschwankungen Struktur schafft. Sie ist ein \textit{stehendes Muster} im fraktalen Vakuum -- eingeprägt beim Phasenübergang, manifestiert durch die Zeit-Masse-Dualität.

Das ``kosmische Netz'' ist wörtlich ein Netz -- ein Resonanzmuster, analog zu den neuronalen Verbindungen in einem Gehirn.

\section{Dunkle Energie: Der Metabolismus des Kosmos}

Eine der größten Rätsel der modernen Kosmologie ist die ``Dunkle Energie'' -- eine mysteriöse Kraft, die die Expansion des Universums beschleunigt. Sie macht etwa 70\% des Energiebudgets des Universums aus, aber niemand weiß, was sie ist.

In der FFGFT gibt es keine separate ``Dunkle Energie''. Was wir beobachten, ist einfach die fortgesetzte fraktale Entwicklung -- der energetische ``Metabolismus'' des sich vertiefenden Universums.

Die effektive Dichte dieser ``Dunklen Energie'' ist:

\begin{equation}
\rho_{\Lambda}^{\text{eff}} = \frac{3H_0^2}{8\pi G} \cdot \left(\frac{\dot{\xi}}{\xi H_0}\right)^2 \approx 0.7 \rho_c
\end{equation}

Hier ist $\rho_c = 3H_0^2/(8\pi G)$ die kritische Dichte, und der Term $(\dot{\xi}/\xi H_0)^2$ erfasst, wie viel Energie in der Skalenänderung steckt.

Die Zustandsgleichung -- das Verhältnis von Druck zu Dichte -- ist:

\begin{equation}
w_{\text{eff}} = -1 + \frac{2}{3} \cdot \frac{\ddot{\xi}\xi}{\dot{\xi}^2} \approx -0.98
\end{equation}

Der Wert $w \approx -1$ ist genau das, was beobachtet wird und was die Beschleunigung erklärt. Aber in der FFGFT ist dies keine separate Energiekomponente, sondern ein geometrischer Effekt -- der ``Grundumsatz'' des sich vertiefenden fraktalen Gewebes.

Wie ein aktives Gehirn Energie verbraucht, um seine Strukturen aufrechtzuerhalten und zu entwickeln, so ``verbraucht'' das fraktale Vakuum Energie für seine fortgesetzte Vertiefung.

\section{Strukturbildung ohne Inflation}

Das Standardmodell der Kosmologie hat mehrere gravierende Probleme, die es durch eine zusätzliche Hypothese -- die ``Inflation'' -- zu lösen versucht. In der FFGFT lösen sich diese Probleme von selbst:

\textbf{Das Horizontproblem:} Warum ist das Universum in alle Richtungen so gleichförmig, obwohl viele Regionen nie in kausalem Kontakt waren?

\textit{Lösung in der FFGFT:} Fraktale Nichtlokalität. Auf kleinen Skalen sind alle Punkte durch die fraktale Struktur verbunden -- es gibt keine echten ``Horizonte''. Das Vakuum ist intrinsisch kohärent.

\textbf{Das Flachheitsproblem:} Warum hat das Universum genau die kritische Dichte, die es flach macht?

\textit{Lösung in der FFGFT:} Die fraktale Metrik ist intrinsisch flach ($k=0$) auf allen Skalen. Flachheit ist keine Feinabstimmung, sondern geometrische Notwendigkeit.

\textbf{Das Monopolproblem:} Warum sehen wir keine magnetischen Monopole?

\textit{Lösung in der FFGFT:} Die fraktale Topologie erlaubt keine topologischen Defekte mit gefährlicher Dichte. Das Vakuum ist ``glatt'' auf allen Skalen.

Inflation wird überflüssig. Die Homogenität und Struktur des Universums sind direkte Konsequenzen der fraktalen Geometrie.

\section{Testbare Vorhersagen}

Theorien sind nur so gut wie ihre Vorhersagen. Die FFGFT macht mehrere präzise, testbare Vorhersagen, die sie von der Standardkosmologie unterscheiden:

\textbf{1. Abweichungen im CMB-Spektrum:}

Bei hohen Multipolen ($l > 100$) sagt die FFGFT kleine Abweichungen vom Standard-$\Lambda$CDM voraus:

\begin{equation}
\frac{\Delta C_l}{C_l^{\Lambda\text{CDM}}} = \xi \cdot \ln\left(\frac{l}{l_0}\right)
\end{equation}

Bei $l = 2000$ wäre $\Delta C_l/C_l \approx 0.1\%$ -- klein, aber mit zukünftigen hochpräzisen Messungen nachweisbar.

\textbf{2. Zeitvariation fundamentaler Konstanten:}

Wenn $\xi$ sich ändert, müssen sich auch abgeleitete Größen ändern -- etwa die Feinstrukturkonstante $\alpha$:

\begin{equation}
\frac{\dot{\alpha}}{\alpha} = -2 \frac{\dot{\xi}}{\xi} \approx 4.5 \times 10^{-18} \, \text{s}^{-1}
\end{equation}

Dies ist eine Änderung von etwa 0.000014\% pro Million Jahre -- winzig, aber prinzipiell messbar mit Atomuhren und durch Analyse von Quasarabsorptionslinien.

\textbf{3. Fraktale Korrelationen in der großräumigen Struktur:}

Das Leistungsspektrum der Materie-Verteilung sollte fraktale Signaturen zeigen:

\begin{equation}
P(k) = P_{\Lambda\text{CDM}}(k) \cdot \left[1 + \xi \cdot (k/k_0)^{-D_f+3}\right]
\end{equation}

Für $k_0 = 0.1 \, \text{h/Mpc}$ sollten Abweichungen bei kleinen $k$ (großen Skalen) sichtbar sein.

\section{Vergleich: Standard-$\Lambda$CDM vs. Fraktale T0-Kosmologie}

Lassen Sie uns die beiden Paradigmen direkt gegenüberstellen:

\begin{center}
\small
\begin{tabular}{p{0.45\textwidth}|p{0.45\textwidth}}
\toprule
\textbf{Standard-$\Lambda$CDM} & \textbf{Fraktale T0-Kosmologie} \\
\midrule
Raum expandiert physikalisch & Raum ist statisch, fraktale Tiefe ändert sich \\
Big Bang: Singularität & Big Bang: Phasenübergang \\
Dunkle Materie: Teilchen & Dunkle Materie: Fraktale Geometrie \\
Dunkle Energie: Konstante $\Lambda$ & Dunkle Energie: Fraktale Skalenentwicklung \\
Inflation nötig für Homogenität & Fraktale Selbstähnlichkeit garantiert Homogenität \\
6+ freie Parameter & 1 Parameter: $\xi_0 = \frac{4}{3} \times 10^{-4}$ \\
Horizonte durch kausale Verzögerung & Fraktale Nichtlokalität verbindet alle Punkte \\
Rotverschiebung: Doppler-Effekt & Rotverschiebung: Fraktale Skalenänderung \\
\bottomrule
\end{tabular}
\end{center}

Der Kontrast könnte nicht deutlicher sein. Wo das Standardmodell multiple Komponenten und Parameter benötigt, reduziert die FFGFT alles auf ein einziges geometrisches Prinzip.

\section{Die zeitliche Entwicklung in vier Epochen}

Die Geschichte des Universums in der FFGFT lässt sich in vier Phasen einteilen:

\begin{enumerate}[leftmargin=*]
\item \textbf{Frühe fraktale Ära} ($t < 10^{-32}$ s): 

Unmittelbar nach dem Phasenübergang. $\xi \approx \xi_0$, $D_f \approx 3 - \xi_0 \approx 2.999867$. Das Vakuum ist noch ``jung'', die fraktale Struktur gerade erst entstanden. Analoge Phase: Ein neugeborenes Gehirn, noch ohne Windungen.

\item \textbf{Strahlungs-ähnliche Phase} ($10^{-32}$ s $ < t < 4.7 \times 10^4$ Jahre):

$\xi$ nimmt langsam ab, das Universum ``kühlt'' geometrisch. Die Zeit-Masse-Dualität sorgt dafür, dass Energie dominiert, die sich wie Strahlung verhält. Analoge Phase: Neuronale Migration und erste Verbindungsbildung.

\item \textbf{Materie-ähnliche Phase} ($4.7 \times 10^4$ Jahre $ < t < 9.8 \times 10^9$ Jahre):

$\dot{\xi}/\xi$ ist annähernd konstant. Strukturen bilden sich aus, Galaxien entstehen als Manifestationen der fraktalen Resonanzmuster. Analoge Phase: Hauptphase der Synaptogenese -- massive Bildung von Verbindungen.

\item \textbf{Skalenänderungs-dominiert} ($t > 9.8 \times 10^9$ Jahre):

$\dot{\xi}/\xi$ dominiert die Energiebilanz -- die ``beschleunigte Expansion''. Die fraktale Vertiefung wird zum primären Prozess. Analoge Phase: Reifung und Optimierung -- Pruning und Verfeinderung der Strukturen.
\end{enumerate}

\section{Das Universum als sich vertiefendes Gehirn: Eine Synthese}

Die gesamte Kosmologie der FFGFT kulminiert in einem Bild von außerordentlicher Schönheit und Kohärenz:

\textbf{Das Universum ist ein sich vertiefendes, faltendes, selbstähnliches Gewebe -- ein kosmisches Gehirn, dessen ``Windungen'' sich durch die fraktale Zeit-Masse-Dualität ständig weiter ausprägen.}

Diese Metapher ist nicht nur poetisch, sie ist mathematisch präzise:

\begin{itemize}[leftmargin=*]
\item \textbf{Windungen statt Expansion:} Wie ein sich entwickelndes Gehirn wächst das Universum nicht als Ganzes, sondern bildet komplexe Furchungen aus, die seine ``Oberfläche'' bei konstantem Volumen dramatisch vergrößern. Die fraktale Dimension $D_f = 3 - \xi(t)$ beschreibt genau diese zunehmende Komplexität.

\item \textbf{Neuronales Netz \& Kosmisches Netz:} Die großräumige Struktur mit ihren Galaxienfilamenten ist kein Zufallsprodukt, sondern ein stehendes fraktales Muster -- analog zu neuronalen Verbindungen.

\item \textbf{Informationsverarbeitung:} Das Vakuum ``verarbeitet'' über die Zeit-Masse-Dualität reine Zeitstruktur ($\theta$) in manifeste Masse/Energie ($\rho$). Der Big Bang war der Moment, in dem das ``universale Gehirn'' zu ``denken'' begann.

\item \textbf{Selbstähnlichkeit:} Wie ein Gehirn auf verschiedenen Skalen selbstähnlich organisiert ist, so ist das Universum durch $D_f$ auf allen Skalen selbstähnlich -- von der Planck-Länge bis zum kosmischen Horizont.

\item \textbf{Globale Vernetzung:} Die fraktale Nichtlokalität sorgt für instantane Korrelationen auf allen Skalen -- das ``Horizontproblem'' existiert nicht.

\item \textbf{Dunkle Energie als Metabolismus:} Die beobachtete ``beschleunigte Expansion'' ist der energetische Grundumsatz des sich vertiefenden Systems -- analog zum Stoffwechsel eines aktiven Gehirns.
\end{itemize}

\section{Schlussfolgerung: Ein neues Paradigma}

Die fraktale Kosmologie der FFGFT revolutioniert unser Verständnis des Universums durch eine radikale Uminterpretation:

\begin{center}
\textbf{Wir leben nicht in einem expandierenden Ballon,} \\
\textbf{sondern in einem sich vertiefenden, faltenden, selbstähnlichen Gewebe --} \\
\textbf{einem kosmischen Gehirn, dessen ``Windungen'' sich durch die} \\
\textbf{fraktale Zeit-Masse-Dualität ständig weiter ausprägen.}
\end{center}

Die beobachtete ``Expansion'' ist lediglich unser Perspektiveneffekt, während wir in diese zunehmende fraktale Tiefe hinein-``zoomen''. Diese Sichtweise:

\begin{itemize}[leftmargin=*]
\item Eliminiert Singularitäten (der Big Bang ist ein Phasenübergang, keine Schöpfung aus dem Nichts)
\item Macht Dunkle Energie als separate Entität überflüssig (sie ist geometrischer Effekt)
\item Erklärt die Struktur des Universums ohne Inflation
\item Reduziert die gesamte Kosmologie auf ein einziges geometrisches Prinzip: die dynamische Selbstorganisation eines fraktalen Vakuums
\item Benötigt nur einen fundamentalen Parameter: $\xi_0 = \frac{4}{3} \times 10^{-4}$
\end{itemize}

In den folgenden Kapiteln werden wir sehen, wie dieses Bild -- das Universum als sich vertiefendes Gehirn -- noch reichere und tiefere Implikationen für Quantenmechanik, Teilchenphysik und die Vereinheitlichung aller Kräfte hat.

\textbf{Das Gehirn denkt weiter. Das Universum vertieft sich. Und wir -- mitten darin -- beginnen gerade erst zu verstehen, was das bedeutet.}

\end{document}






