\chapter{Chapter 25: The Neutrino Mass Problem in Fractal T0 Geometry  Narrative Version of FFGFT}
\label{chap:25-en}

\section*{Chapter 25: The Neutrino Mass Problem in Fractal T0 Geometry}
	
	\subsection*{Brief Introduction}
	
	This chapter resolves the open questions regarding neutrino masses — their smallness, the three generations, hierarchy, mixing, and Majorana nature — through pure phase excitations of the vacuum field.
	
	\subsection*{Mathematical Foundation}
	
	In the FFGFT, neutrinos are not Dirac or Majorana fields with amplitude but pure phase excitations of the vacuum field \(\Phi = \rho(x,t) e^{i\theta(x,t)}\). All properties emerge from the fundamental parameter \(\xi = \frac{4}{3} \times 10^{-4}\).
	
	\subsection*{Symbol List and Units}
	
	\begin{tcolorbox}[title={\textbf{Important Symbols and Their Units}}, colback=blue!5!white, colframe=blue!75!black]
		\begin{tabular}{p{0.3\textwidth}p{0.3\textwidth}p{0.35\textwidth}}
			\textbf{Symbol} & \textbf{Meaning} & \textbf{Unit (SI)} \\
			\hline
			\(\xi\) & Fractal scaling parameter & dimensionless \\
			\(m_{\nu_i}\) & Mass of the $i$-th neutrino & \si{\ev\per c\squared} \\
			\(m_0^\nu\) & Reference mass for neutrinos & \si{\ev\per c\squared} \\
			\(\theta_{\nu_i}\) & Characteristic phase of the $i$-th neutrino & dimensionless (radians) \\
			\(\Delta \theta_{\min}\) & Minimal phase shift & dimensionless (radians) \\
			\(m_1, m_2, m_3\) & Masses of the three generations & \si{\ev\per c\squared} \\
			\(U_{ij}\) & PMNS mixing element & dimensionless \\
			\(\Delta \theta_{ij}\) & Phase difference between modes $i$ and $j$ & dimensionless (radians) \\
			\(\sum m_\nu\) & Sum of neutrino masses & \si{\ev\per c\squared} \\
			\(\hbar\) & Reduced Planck's constant & \si{\joule\second} \\
			\(c\) & Speed of light & \si{\meter\per\second} \\
			\(l_0\) & Fractal correlation length & \si{\meter} \\
		\end{tabular}
	\end{tcolorbox}
	
	\subsection*{Neutrinos as Pure Phase Excitations}
	
	Neutrinos have almost no amplitude component — their mass arises solely from phase windings. The minimal stable phase shift is limited by fractal fluctuations:
	
	\begin{equation}
		\Delta \theta_{\min} \approx \xi^{3/2} \cdot \sqrt{\ln(\xi^{-1})}.
	\end{equation}
	
	The term \(\xi^{3/2}\) comes from the triple hierarchy of fractal scaling, the logarithm from resummation over infinitely many levels. This small shift makes neutrinos almost massless compared to charged leptons.
	
	\subsection*{Mass Hierarchy of the Three Generations}
	
	The masses result from trigonometric projections of 120°-offset phases:
	
	\begin{align}
		m_1 &\approx 2 m_0^\nu \cdot \sin^2(\theta_0 / 2), \\
		m_2 &\approx 2 m_0^\nu \cdot \sin^2((\theta_0 + 120^\circ)/2), \\
		m_3 &\approx 2 m_0^\nu \cdot \sin^2((\theta_0 + 240^\circ)/2).
	\end{align}
	
	The factor \(2 m_0^\nu\) sets the overall scale, the squared sine describes the effective mass from the phase deviation from equilibrium. The 120° offset is the natural symmetry of the three fractal generations.
	
	With a small fractal correction \(\theta_0 \approx \pi + \xi \cdot \Delta\), the observed hierarchy emerges:
	
	\begin{equation}
		m_1 : m_2 : m_3 \approx 1 : 3 : 8
	\end{equation}
	
	in first order — consistent with the normal hierarchy.
	
	The absolute scale:
	
	\begin{equation}
		m_0^\nu \approx \frac{\hbar}{c l_0} \cdot \xi^3 \approx \SI{0.05}{\ev\per c\squared}.
	\end{equation}
	
	The factor \(\xi^3\) arises from the triple fractal suppression of the phase-amplitude coupling.
	
	The sum of the masses:
	
	\begin{equation}
		\sum m_\nu \approx \SI{0.12}{\ev\per c\squared}
	\end{equation}
	
	lies within the cosmologically allowed range.
	
	\textbf{Unit check:}
	\begin{align*}
		[m_0^\nu] &= \si{\joule\second} / (\si{\meter\per\second} \cdot \si{\meter}) = \si{\kilo\gram} \quad (\text{converted to eV}/c^2).
	\end{align*}
	
	\subsection*{PMNS Mixing from Phase Overlap}
	
	The mixing matrix arises from the overlap of adjacent phase modes:
	
	\begin{equation}
		U_{ij} \approx \cos(\Delta \theta_{ij}) + i \xi \cdot \sin(\Delta \theta_{ij}).
	\end{equation}
	
	The cosine term gives the main mixing (tribimaximal), the imaginary \(\xi\) term small perturbations — exactly the observed PMNS structure with large mixing angles.
	
	\subsection*{Majorana Nature}
	
	Since neutrinos are pure phases, charge conjugation is equivalent to phase reversal \(\theta \to -\theta\):
	
	\begin{equation}
		\nu = \nu^c.
	\end{equation}
	
	They are necessarily Majorana particles.
	
	\subsection*{Comparison Standard Model – FFGFT}
	
	\begin{center}
		\begin{tabular}{p{0.45\textwidth}p{0.45\textwidth}}
			\textbf{Standard Model} & \textbf{FFGFT (T0)} \\
			\hline
			Masses ad-hoc & Emergent from phase \\
			Seesaw postulated & No amplitude \\
			Three generations arbitrary & 120° symmetry \\
			PMNS free & From phase overlap \\
			Majorana unclear & Necessarily Majorana \\
		\end{tabular}
	\end{center}
	
	\subsection*{Conclusion}
	
	The FFGFT completely resolves the neutrino problem: small masses from pure phase, three generations from fractal 120° symmetry, hierarchy and mixing from \(\xi\)-perturbations, Majorana nature from self-conjugation. All values emerge naturally from the single parameter \(\xi\), elegantly closing the lepton sector.