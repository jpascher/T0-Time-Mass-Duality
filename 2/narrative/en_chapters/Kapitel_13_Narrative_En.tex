\chapter{Chapter 13: The Chronology of Universe Formation  From the Null Vacuum to Structured Reality  Narrative Version of FFGFT}
\label{chap:13-en}

\section*{Introduction}
	
	What happened in the beginning? This ancient question has fascinated philosophers, theologians, and physicists for millennia. Modern cosmology answers with the "Big Bang" -- an explosive singularity from which space, time, matter, and energy suddenly emerged. But the closer we look, the more mysterious this "beginning" becomes. A true singularity -- a point of infinite density and temperature -- is physically problematic, if not impossible.
	
	The Fundamental Fractal Geometric Field Theory (FFGFT) tells a different story. There was no explosion, no singularity, no mystical moment of creation from absolute nothingness. Instead, there was a \textit{phase transition} -- a deterministic, traceable transition from a minimal state to a structured one. Like water freezing into ice. Like a supersaturated solution suddenly forming crystals.
	
	\textbf{Central Metaphor:} The universe behaves like a growing brain whose convolutions increase while the overall volume remains constant. The "Big Bang" was not an explosive start, but the moment when the "cosmic brain" began to "think" -- the transition from potential to manifest structure.
	
	In this chapter, we reconstruct the chronology of this transition, step by step, based on a single fundamental parameter: $\xi = \frac{4}{3} \times 10^{-4}$.
	
	\section{The Pre-Big-Bang Phase: The Null Vacuum}
	
	\subsection{A Universe Before the Universe}
	
	Before there were galaxies, before there were atoms, before there was space and time in the form we know -- what was there?
	
	In the Standard Model, this question is unanswerable. There was no "before" the Big Bang because time itself only arose with the Big Bang. This is logically consistent but unsatisfying.
	
	FFGFT offers a concrete answer: There was a \textit{Pre-Vacuum} -- a minimal state of the fractal field, characterized by:
	
	\[
	\begin{aligned}
		\rho &\approx 0 \quad \text{(nearly massless vacuum)} \\
		D_f &\approx 2 \quad \text{(strongly under-dimensional fractal structure)} \\
		\theta &= \text{constant} \quad \text{(static, disordered time structure)} \\
		a_{\min} &\approx l_P \cdot \xi^{-1} \approx 1.2 \times 10^{-31} \, \text{m}
	\end{aligned}
	\]
	
	Let us understand each of these statements:
	
	\begin{itemize}[leftmargin=*]
		\item $\rho \approx 0$: The amplitude of the vacuum field -- its "substance" -- is nearly zero. The vacuum is like an extremely thin, almost transparent fabric.
		
		\item $D_f \approx 2$: The fractal dimension is not 3 (like our space), but close to 2. The universe was effectively \textit{two-dimensional} -- flat like a sheet of paper, without depth, without the third dimension. Imagine a Flatlander living in a 2D world, unable to even conceive of the third dimension.
		
		\item $\theta = \text{constant}$: The phase field -- which encodes the time structure -- is static and disordered. There is no coherent time evolution, no causality, no history.
		
		\item $a_{\min} \approx 1.2 \times 10^{-31}$ m: The minimal effective scale is about 10,000 times larger than the Planck length $l_P$, determined by the relationship $l_P \cdot \xi^{-1}$.
	\end{itemize}
	
	\subsection{Perfect Coherence Without Structure}
	
	This null vacuum is perfectly coherent -- but in a trivial way. It is like a perfectly smooth water surface without waves, without movement. There are no gradients, no fluctuations, no structure.
	
	Why? Because any gradient or fluctuation would require a non-zero amplitude $\rho > 0$. To have a wave, you need water. To have structure, you need substance. And in the pre-vacuum, there is (almost) no substance.
	
	The extremely low fractal dimension $D_f \approx 2$ means that spacetime is almost two-dimensional -- highly constrained, unable to carry the complexity and diversity that characterize a three-dimensional universe.
	
	It is like a brain before development -- a smooth surface without furrows, without convolutions, without the fractal complexity that enables thought.
	
	\section{The Trigger: The Critical Instability}
	
	\subsection{The Hidden Instability of Duality}
	
	But this perfectly coherent null vacuum is not stable. It carries the seed of its own transformation within itself -- the \textit{Time-Mass Duality}:
	
	\begin{equation}
		T(x,t) \cdot m(x,t) = 1
	\end{equation}
	
	This equation says: The product of time structure and mass must be constantly one. When mass approaches zero, the time structure must approach infinity:
	
	\begin{equation}
		\text{For } \rho \to 0: \quad T(x,t) \to \infty \quad \text{(infinite time density)}
	\end{equation}
	
	This is not physically stable. It is like a pendulum balanced perfectly upright -- any tiny disturbance makes it fall. The state $\rho \approx 0$ is an equilibrium, but an \textit{unstable} one.
	
	\subsection{The Triggering Fluctuation}
	
	What triggers the transition? A fluctuation -- but not an arbitrary, mystical fluctuation. It is a \textit{fractal quantum fluctuation}, whose magnitude is determined by $\xi$ itself:
	
	\begin{equation}
		\Delta\rho \approx \xi^2 \cdot \rho_P \approx 2.1 \times 10^{-96} \, \text{kg}^{1/2}\text{m}^{-3/2}
	\end{equation}
	
	Here $\rho_P = \sqrt{\hbar c}/l_P^{3/2} \approx 1.2 \times 10^{88}$ is the Planck density -- the maximum density that makes quantum mechanical sense. The factor $\xi^2 \approx 1.78 \times 10^{-8}$ reduces this to a tiny but non-zero fluctuation.
	
	\textbf{The Physical Meaning:} Even in the "empty" pre-vacuum, there are quantum fluctuations -- unavoidable tremors of the vacuum field due to the Heisenberg uncertainty relation. Normally, these fluctuations are insignificant. But in the unstable state $\rho \approx 0$, such a fluctuation acts like the famous butterfly wing that triggers a tornado.
	
	\subsection{The Phase Transition Potential}
	
	The dynamics of the transition is described by an effective potential:
	
	\begin{equation}
		V(\rho) = \lambda (\rho^2 - \rho_0^2)^2 \cdot \left(1 + \xi \ln(\rho/\rho_0)\right)
	\end{equation}
	
	Imagine a landscape where $V(\rho)$ represents height:
	
	\begin{itemize}[leftmargin=*]
		\item At $\rho = 0$ (the pre-vacuum), the potential is high -- an unstable peak
		\item At $\rho = \rho_0$ (the stable vacuum), the potential is minimal -- a stable valley
		\item $\lambda$ is the coupling constant (proportional to the fine structure constant $\alpha$)
		\item The term $1 + \xi \ln(\rho/\rho_0)$ is a fractal correction
	\end{itemize}
	
	Like a ball balanced on a hill, the field $\rho$ is unstable in the state $\rho = 0$. The slightest disturbance makes it roll into the valley -- the phase transition begins.
	
	\section{The Chronology of the Transition}
	
	\subsection{A Timeline of Becoming}
	
	Let us now reconstruct step by step how our structured universe emerged from the minimal pre-vacuum:
	
	\textbf{Phase 1: Pre-Vacuum ($t \ll t_P \approx 10^{-43}$ s)}
	
	\begin{itemize}[leftmargin=*]
		\item $\rho \approx 0$: No substance
		\item $D_f \approx 2$: Almost two-dimensional spacetime
		\item $\theta$ constant and disordered: No coherent time
		\item Time-Mass duality not yet active (since $m \approx 0$)
		\item No measurable time, no measurable mass
	\end{itemize}
	
	This is the "primordial state" -- but not an absolute nothing. It is a minimal something, a potential waiting to be actualized.
	
	Like a brain before birth -- present, but without function, without structure, without consciousness.
	
	\textbf{Phase 2: Critical Point ($t \approx 10^{-43}$ s)}
	
	\begin{itemize}[leftmargin=*]
		\item Fractal quantum fluctuation reaches $\Delta\rho \approx \xi^2\rho_P$
		\item The Time-Mass duality becomes active: $T \cdot m > 0$
		\item The instability in the potential $V(\rho)$ becomes relevant
		\item The phase transition begins
	\end{itemize}
	
	This is the "Planck moment" -- the smallest time scale on which physical processes make sense: $t_P = \sqrt{\hbar G/c^5} \approx 5.4 \times 10^{-44}$ s.
	
	It is the moment of "awakening" -- the system recognizes its own instability and begins to transform.
	
	\textbf{Phase 3: Exponential Growth ($10^{-43} < t < 10^{-42}$ s)}
	
	\begin{itemize}[leftmargin=*]
		\item $\rho$ grows exponentially: $\rho(t) \approx \Delta\rho \cdot e^{t/\tau}$
		\item $\tau = \hbar/(m_P c^2 \xi^2) \approx 10^{-43}$ s is the characteristic time
		\item $D_f$ evolves from $\approx 2$ to $3-\xi \approx 2.999867$
		\item Time emerges as phase evolution: $d\tau \propto d\theta/\rho$
	\end{itemize}
	
	This is FFGFT's "inflation phase" -- but not a separate, mysterious inflation with an inflaton field. It is simply the natural dynamics of exponential growth of $\rho$ as it rolls from the unstable state to stable equilibrium.
	
	In this tiny time span -- less than one hundredth of a Planck time -- the universe fundamentally transforms. Spacetime "unfolds" from 2D to 3D. Time as a coherent structure emerges. The "cosmic brain" begins to form its first convolutions.
	
	\textbf{Phase 4: Stabilization ($t > 10^{-36}$ s)}
	
	\begin{itemize}[leftmargin=*]
		\item $\rho$ reaches equilibrium: $\rho_0 = \sqrt{\hbar c}/(l_P^{3/2} \xi^2)$
		\item $D_f$ stabilizes at $3 - \xi \approx 2.999867$
		\item The speed of light establishes itself: $c = \sqrt{K_0/\rho_0} \cdot (1 - \xi/2)$
		\item Time-Mass duality is established: $T(x,t) \cdot m(x,t) = 1$
	\end{itemize}
	
	After about $10^{-36}$ seconds (a thousand trillion trillion Planck times), the field has reached its stable equilibrium. The universe is now in the form it retains to this day -- a three-dimensional fractal vacuum with fractal dimension $D_f = 3 - \xi$.
	
	The fundamental transformation is complete. What follows is "just" the elaboration of details -- the formation of structures, galaxies, stars, planets, life, consciousness.
	
	\section{How Fundamental Quantities Emerge}
	
	One of the deepest insights of FFGFT is that all fundamental physical quantities are not "given" but \textit{emerge} -- they arise as consequences of the phase transition.
	
	\subsection{The Emergence of Time}
	
	Time is not fundamental. It emerges as a derivative of phase evolution:
	
	\begin{equation}
		d\tau = \frac{\hbar}{m_P c^2} \cdot \frac{d\theta}{\rho/\rho_0} \cdot \xi^{-1}
	\end{equation}
	
	\textbf{The Interpretation:} An infinitesimal time interval $d\tau$ corresponds to an infinitesimal change in phase $d\theta$, scaled with amplitude $\rho$ and parameter $\xi$.
	
	Before the transition, when $\rho \approx 0$, this relationship is singular -- there is no coherent time. After the transition, with $\rho = \rho_0$ stabilized, time flows uniformly.
	
	Time is thus not a container in which events occur, but a \textit{structure} that emerges from the phase evolution of the vacuum field.
	
	\subsection{The Emergence of the Speed of Light}
	
	The speed of light is not fundamental but emerges from vacuum stiffness:
	
	\begin{equation}
		c = \sqrt{\frac{K_0}{\rho_0}} \cdot \left(1 - \frac{\xi}{2}\right) \approx 2.9979 \times 10^8 \, \text{m/s}
	\end{equation}
	
	Here $K_0$ is the "stiffness" of the vacuum -- its resistance to deformations. The speed of light is the velocity at which disturbances propagate in this medium.
	
	The correction factor $(1 - \xi/2)$ is tiny -- about 0.99993 -- but it is there. Without this fractal correction factor, the speed of light would be slightly higher.
	
	\subsection{The Emergence of Gravitation}
	
	The gravitational constant is not fundamental but follows from the fractal spacetime structure:
	
	\begin{equation}
		G = \frac{c^3 l_P^2}{\hbar} \cdot \xi^2 \approx 6.674 \times 10^{-11} \, \text{m}^3\text{kg}^{-1}\text{s}^{-2}
	\end{equation}
	
	The factor $\xi^2$ is crucial. Without it -- if $\xi = 1$ -- gravitation would be stronger by a factor $(1/\xi)^2 \approx 5.6 \times 10^7$. The universe would immediately collapse. Galaxies, stars, planets -- none of this could exist.
	
	The tiny value $\xi = \frac{4}{3} \times 10^{-4}$ is thus essential for gravitation to be as weak as it is -- and thus enables structure on large scales.
	
	\subsection{The Emergence of Particle Masses}
	
	The masses of all particles -- from the electron to the Higgs boson -- also emerge from the fractal parameter:
	
	\begin{equation}
		m_i = m_P \cdot f_i(\xi) \cdot \xi^{k_i}
	\end{equation}
	
	Here $m_P = \sqrt{\hbar c/G} \approx 2.18 \times 10^{-8}$ kg is the Planck mass, $f_i(\xi)$ are specific fractal form factors, and $k_i$ are hierarchy levels (integers).
	
	The mass hierarchy -- why the electron is so light (about $10^{-30}$ kg) and the top quark so heavy (about $10^{-25}$ kg) -- is encoded in the different hierarchy levels $k_i$ and the fractal form factors.
	
	\section{The Entropy Puzzle}
	
	One of the greatest unsolved mysteries of cosmology is the \textit{extremely low initial entropy} of the universe.
	
	\subsection{The Problem}
	
	Entropy measures disorder. According to the second law of thermodynamics, entropy in a closed system always increases. The universe thus had lower entropy at the beginning than today.
	
	But how low? The initial entropy of the observable universe is estimated at about $S_{\text{initial}} \approx 10^{88} k_B$ (where $k_B$ is Boltzmann's constant). This sounds large but is tiny compared to the \textit{maximum} entropy that a universe of this size could have: about $10^{120} k_B$.
	
	The ratio is $10^{88}/10^{120} = 10^{-32}$ -- an extremely special initial condition. Why? The Standard Model has no answer.
	
	\subsection{The Natural Explanation in FFGFT}
	
	In FFGFT, the low initial entropy follows naturally:
	
	\begin{equation}
		S_{\text{initial}} \approx k_B \cdot \ln\left(\frac{V_{\text{eff}}}{l_P^3}\right) \cdot \xi^3 \approx 10^{88} k_B
	\end{equation}
	
	The factor $\xi^3 \approx 2.37 \times 10^{-10}$ dramatically reduces the maximum possible entropy. Why?
	
	\begin{itemize}[leftmargin=*]
		\item The pre-vacuum has nearly zero entropy due to its fractal self-similarity -- it is perfectly ordered (trivially ordered, but ordered)
		\item Entropy only grows with the emergence of $\rho > 0$ -- with substance also comes the possibility of disorder
		\item The factor $\xi^3$ encodes how many independent degrees of freedom the vacuum has
	\end{itemize}
	
	There is no fine-tuning, no mystery. The low initial entropy is a direct consequence of the fractal structure.
	
	\section{Testable Predictions}
	
	Theory without testable predictions is speculation. FFGFT makes several precise predictions that distinguish it from alternative theories:
	
	\subsection{1. Fractal Traces in the CMB}
	
	The temperature anisotropies in the cosmic microwave background should show fractal self-similarity:
	
	\begin{equation}
		\frac{\delta T}{T}(\vec{n}) \propto \xi \cdot \sum_{n} \frac{\cos(2\pi |\vec{x}_n|/\lambda_n)}{|\vec{x}_n|^{D_f/2}}
	\end{equation}
	
	with a scaling exponent $D_f/2 \approx 1.5$.
	
	\textbf{How to test:} Analyze CMB data from Planck and future missions for fractal correlations. Search for deviations from Gaussian statistics with a characteristic exponent 1.5.
	
	\subsection{2. Time Variation of $\xi$}
	
	The parameter $\xi$ is not absolutely constant but changes slightly with time:
	
	\begin{equation}
		\left|\frac{\dot{\xi}}{\xi}\right| \approx 2.3 \times 10^{-18} \, \text{s}^{-1}
	\end{equation}
	
	This is a change of about 0.000007\% per million years -- tiny but in principle measurable.
	
	\textbf{How to test:} Compare ultra-precise atomic clocks over decades. Search for systematic drifts in fundamental constants. Analyze absorption lines in distant quasars for hints of variation in the fine structure constant.
	
	\subsection{3. Modified Early Expansion}
	
	Instead of a separate inflation phase with an inflaton field, FFGFT predicts:
	
	\begin{equation}
		a(t) \propto t^{2/D_f} \approx t^{0.6667} \quad \text{(early era)}
	\end{equation}
	
	This is a slightly different scaling than standard inflation ($a(t) \propto e^{Ht}$).
	
	\textbf{How to test:} Search for characteristic signatures in the B-mode polarization spectrum of the CMB. FFGFT predicts a somewhat different ratio of tensor to scalar modes.
	
	\section{Comparison with Alternative Theories}
	
	How does FFGFT compare to other approaches that want to avoid the initial singularity?
	
	\subsection{Loop Quantum Cosmology (LQC)}
	
	\textbf{Loop Quantum Cosmology} quantizes spacetime itself and replaces the singularity with a "Big Bounce" -- the universe collapses, reaches a critical density $\rho_{\text{crit}}$, and bounces back into an expansion phase.
	
	\begin{center}
		\small
		\resizebox{\textwidth}{!}{%
			\begin{tabular}{p{0.28\textwidth}|p{0.32\textwidth}|p{0.32\textwidth}}
				\toprule
				\textbf{Aspect} & \textbf{Loop Quantum Cosmology} & \textbf{Fractal FFGFT} \\
				\midrule
				Pre-Phase & Quantum geometry with Immirzi parameter $\gamma$ & Fractal null vacuum with $D_f \approx 2$ \\
				Transition & Big Bounce at $\rho = \rho_{\text{crit}}$ & Phase transition at $\rho \approx \xi^2\rho_P$ \\
				Parameters & $\gamma \approx 0.2375$, $\rho_{\text{crit}}$ & Only $\xi = \frac{4}{3} \times 10^{-4}$ \\
				Dimensions & 3+1 & 3+1 with fractal structure $D_f = 3-\xi$ \\
				Entropy problem & Requires special initial conditions & Naturally explained by $\xi^3$ \\
				\bottomrule
			\end{tabular}
		}%
	\end{center}
	
	FFGFT is simpler -- one parameter instead of several -- and explains more (the low entropy).
	
	\subsection{String Theory Cosmology}
	
	\textbf{String Theory} postulates higher-dimensional spaces (10 or 11 dimensions), with the extra dimensions compactified. The Big Bang could be a brane collision or a tunneling process.
	
	\begin{center}
		\small
		\resizebox{\textwidth}{!}{%
			\begin{tabular}{p{0.28\textwidth}|p{0.32\textwidth}|p{0.32\textwidth}}
				\toprule
				\textbf{Aspect} & \textbf{String Theory Cosmology} & \textbf{Fractal FFGFT} \\
				\midrule
				Pre-Phase & Higher-dimensional branes/compactification & Fractal 4D null vacuum \\
				Transition & Brane collision/tunneling & Deterministic phase transition \\
				Parameters & Many (moduli, dilaton, etc.) & Only $\xi$ \\
				Dimensions & 10-11 (must be compactified) & 3+1 with fractal structure \\
				Predictions & Complex, often multiverse & Precise, testable deviations \\
				\bottomrule
			\end{tabular}
		}%
	\end{center}
	
	FFGFT is radically simpler and makes more precise predictions.
	
	\section{Philosophical Implications}
	
	FFGFT's chronology has profound philosophical consequences:
	
	\subsection{No Singularity}
	
	The "beginning" is not a point of infinite density, no mathematical pathology. It is a regular physical transition -- comprehensible, calculable, non-singular.
	
	This eliminates one of the greatest conceptual problems of modern physics: the inability to describe the moment $t=0$.
	
	\subsection{Determinism}
	
	The phase transition follows inevitably from the Time-Mass duality and the parameter $\xi$. There is no arbitrariness, no fine-tuning, no mysterious choice of initial conditions.
	
	The universe had to become as it is -- given $\xi$.
	
	\subsection{Parameter-free (almost)}
	
	All fundamental constants -- $c$, $G$, $\hbar$, particle masses -- emerge from a single parameter $\xi$. This is a drastic reduction in complexity.
	
	In the Standard Model of particle physics, there are about 19 free parameters. In FFGFT: one.
	
	\subsection{Static Universe}
	
	The universe does not expand in the conventional sense. It deepens fractally. This shift in perspective is radical -- it solves cosmological puzzles (dark energy, low entropy) without additional assumptions.
	
	\subsection{Natural Fine-Tuning}
	
	The "fine-tuned" constants -- why is gravitation so weak? Why is the universe so flat? Why is the cosmological constant so small? -- are no longer mysteries. They are direct consequences of $\xi$.
	
	\section{Conclusion: A New Genesis}
	
	FFGFT's chronology of universe formation offers the simplest and most parameter-poor description of cosmological origins:
	
	\begin{itemize}[leftmargin=*]
		\item \textbf{One Parameter}: Everything emerges from $\xi = \frac{4}{3} \times 10^{-4}$
		\item \textbf{No Singularity}: The Big Bang is a regular fractal phase transition
		\item \textbf{Time-Mass Duality as Engine}: $T(x,t) \cdot m(x,t) = 1$ drives the transition
		\item \textbf{Natural Explanation for Fine-Tuning}: All "fine-tuned" constants follow from $\xi$
		\item \textbf{Testable Predictions}: Fractal patterns in CMB, time variation of fundamental constants, modified B-modes
	\end{itemize}
	
	Instead of an explosive beginning from a singularity, FFGFT describes a gentle, deterministic transition from a minimal fractal state. The universe does not "begin" in the conventional sense, but \textit{unfolds} from a highly symmetric pre-phase through the self-consistent dynamics of the Time-Mass duality.
	
	\textbf{The "cosmic brain" does not awaken through a bang, but through a gentle, inevitable transformation -- from potential to manifestation, from simplicity to complexity, from two-dimensionality to fractal three-dimensionality.}
	
	This perspective eliminates not only the problem of the initial singularity but also provides a natural explanation for the puzzling fine-tuning of natural constants and the extremely low initial entropy of the cosmos -- all emergent consequences of the single fundamental parameter $\xi$.
	
	In the following chapters, we will see how this genesis -- this emergence from fractal duality -- explains all other phenomena of physics: quantum mechanics, particle physics, the unification of forces.
	
	\textbf{The beginning is no longer a mystery. It is a calculable, elegant, inevitable phase transition.}