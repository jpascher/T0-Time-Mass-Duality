\chapter{Chapter 32: Reactor Antineutrino Anomaly – Updated Consideration (as of January 2026)  Narrative Version of FFGFT}


\section*{Chapter 32: Reactor Antineutrino Anomaly – Updated Consideration (as of January 2026)}
	
	\subsection*{Brief Introduction}
	
	This chapter examines the reactor antineutrino anomaly (RAA) in the light of current data and shows how the FFGFT offers a coherent alternative to the mainstream resolution.
	
	\subsection*{Mathematical Foundation}
	
	The RAA described a historical deficit of about 6% in the rate of detected electron antineutrinos at short baselines. Newer flux models have largely explained the deficit, yet the FFGFT provides a geometric interpretation of the numerical value, regulated by \(\xi = \frac{4}{3} \times 10^{-4}\).
	
	\subsection*{Historical Anomaly}
	
	The rate was about 6% lower than predicted:
	
	\begin{equation}
		\frac{R_{\text{obs}}}{R_{\text{pred}}} \approx 0.94.
	\end{equation}
	
	This value was based on older flux models and short baselines (approx. 10–100 m).
	
	\subsection*{Current Status (January 2026)}
	
	Improved summation methods and new measurements (e.g., Daya Bay, RENO, PROSPECT) have eliminated the global deficit. A small “bump” at 4–6 MeV, however, remains discussed in some datasets.
	
	\subsection*{FFGFT Interpretation}
	
	The local vacuum amplitude is modified by the reactor flux:
	
	\begin{equation}
		\frac{\delta \rho}{\rho_0} \approx \xi^2 \cdot \frac{\Phi_{\text{reactor}}}{\rho_0}.
	\end{equation}
	
	The flux \(\Phi_{\text{reactor}}\) generates a small density change, scaled by \(\xi^2\).
	
	The oscillation probability is modified:
	
	\begin{equation}
		P(\bar{\nu}_e \to \bar{\nu}_e) \approx 1 - \sin^2(2\theta) \sin^2\left(1.27 \frac{\Delta m^2 L}{E_\nu}\right) - \xi \cdot \frac{\delta \rho}{\rho_0}.
	\end{equation}
	
	The additional term \(\xi \cdot \frac{\delta \rho}{\rho_0}\) simulates an effective deficit of about 6% in the historical era.
	
	\textbf{Unit check:}
	
	\begin{equation}
		[P] = \text{dimensionless}.
	\end{equation}
	
	\subsection*{Energy Dependence}
	
	The effect maximises at resonance:
	
	\begin{equation}
		E_{\text{res}} \approx \frac{\hbar c}{l_0 \cdot \xi^{-1}} \approx \SIrange{4}{6}{MeV}.
	\end{equation}
	
	The fractally extended correlation length \(l_0 \xi^{-1}\) sets the resonance energy — matching the remaining “bump”.
	
	\textbf{Unit check:}
	
	\begin{equation}
		[E_{\text{res}}] = \si{J \cdot s} \cdot \si{m/s} / \si{m} = \si{J}.
	\end{equation}
	
	\subsection*{Comparison with Sterile Neutrino Hypothesis}
	
	\begin{center}
		\begin{tabular}{p{0.45\textwidth}p{0.45\textwidth}}
			\textbf{Sterile Neutrinos} & \textbf{FFGFT (T0)} \\
			\hline
			\(\Delta m^2 \approx \SI{1}{eV^{2}}\) & No new particles \\
			Constrained by PROSPECT/STEREO & Consistent with all data \\
			Oscillation in vacuum & Vacuum amplitude modification \\
			Ad-hoc scale & Natural from \(\xi\) \\
		\end{tabular}
	\end{center}
	
	\subsection*{Conclusion}
	
	Even after the mainstream resolution of the RAA through improved flux models, the FFGFT remains an elegant alternative: The numerical 6% deficit and the bump at 4–6 MeV are direct consequences of the fractal vacuum modification through \(\delta \rho\). This underlines the universal role of \(\xi\) in unifying particle physics and cosmology.