\chapter{Chapter 03: Problems of General Relativity  and How FFGFT Solves Them  Narrative Version of FFGFT}
\label{chap:03-en}

\section*{Introduction}

Einstein's General Relativity (GR) is one of the most successful scientific theories of all time. It has made countless predictions, all of which have been confirmed: the bending of light rays by massive objects, time dilation in gravitational fields, the existence of gravitational waves, the perihelion shift of Mercury -- the list is impressive.

And yet, GR suffers from fundamental problems that have remained unsolved for decades. In this chapter, we will examine these problems and show how FFGFT elegantly resolves them.

\section{Problem 1: Singularities and Information Loss}

Perhaps the most famous problem of GR is \textbf{singularities}. What happens at the center of a black hole? What was ``before'' the Big Bang? The equations of GR give us a clear answer: at these points, the curvature of spacetime becomes infinite. Density becomes infinite. All physical quantities diverge.

Mathematically expressed: in GR, the curvature $R$ diverges as $R \propto 1/r^4$, where $r$ is the distance to the center. When $r$ approaches zero, $R$ explodes to infinity. This means: the theory breaks down. It cannot tell us what really happens in these regions.

\subsection{The FFGFT Solution}

FFGFT solves this problem elegantly. In FFGFT, the effective curvature always remains finite:
\begin{equation}
R_{\text{eff}} \leq \frac{c^4}{G \hbar} \cdot \xi^2
\end{equation}

The right side of this inequality is a fixed, finite number. It depends on the natural constants $c$ (speed of light, $3 \times 10^8$ m/s), $G$ (gravitational constant), $\hbar$ (Planck's constant, $1.05 \times 10^{-34}$ J$\cdot$s) and of course $\xi$. No matter how close we approach the center of a black hole, the curvature cannot exceed this maximum value.

\textbf{Why?} Because the fractal structure of spacetime possesses a kind of built-in ``damping mechanism.'' Think again of the brain: if you try to fold the cerebral cortex infinitely strongly in a tiny region, the tissue eventually reaches its physical limits. There is a maximum curvature that cannot be exceeded. It is the same with fractal spacetime: the graininess at Planck scales prevents infinite curvature.

\textbf{Validation:} The maximum value is finite, avoids information loss, and is consistent with quantum information principles.

\subsection{Information Preservation}

The absence of singularities has profound implications for the information paradox. In classical GR, information that falls into a black hole is lost forever once it crosses the event horizon. This violates a fundamental principle of quantum mechanics: information must be conserved.

FFGFT resolves this paradox: since there is no singularity, information is never destroyed. It is encoded in the fractal structure and can, in principle, be recovered through Hawking radiation -- though scrambled in extremely complex ways.

\section{Problem 2: The Cosmological Constant Problem}

The second major problem is the \textbf{cosmological constant}. Quantum field theory predicts a vacuum energy density of approximately:
\begin{equation}
\rho_{\text{vac, QFT}} \sim \frac{c^5}{\hbar G^2} \approx 10^{97} \text{ kg/m}^3
\end{equation}

Observations, however, give us:
\begin{equation}
\rho_{\text{vac, obs}} \approx 10^{-27} \text{ kg/m}^3
\end{equation}

The discrepancy is a factor of $10^{124}$ -- the worst prediction in the history of physics!

\subsection{The FFGFT Solution}

In FFGFT, the vacuum energy is not fundamental but emerges from the fractal structure. The effective vacuum energy density is:
\begin{equation}
\rho_{\text{vac, eff}} = \xi \cdot \rho_{\text{vac, QFT}}
\end{equation}

With $\xi = (4/3) \times 10^{-4}$, this brings the theoretical prediction remarkably close to observations. The factor $\xi$ acts as a natural regulator that suppresses the enormous vacuum fluctuations predicted by quantum field theory.

\section{Problem 3: The Hierarchy Problem}

Why is gravity so much weaker than the other fundamental forces? The ratio of gravitational to electromagnetic force between two protons is approximately $10^{-36}$. This enormous hierarchy is unexplained in the Standard Model.

\subsection{The FFGFT Explanation}

In FFGFT, gravity is not a fundamental force but an emergent phenomenon arising from the fractal geometry of spacetime. The weakness of gravity is a direct consequence of the fractal dimension being slightly less than 3:
\begin{equation}
\frac{F_{\text{grav}}}{F_{\text{em}}} \sim \xi^{2}
\end{equation}

The gravitational constant $G$ itself is not fundamental but emerges from $\xi$, the speed of light $c$, and the fundamental time scale $T_0$:
\begin{equation}
G = \frac{c^3 T_0}{\xi}
\end{equation}

\section{Problem 4: Dark Matter}

Observations show that galaxies rotate too fast -- their outer regions move much faster than gravity from visible matter alone would allow. The conventional explanation: there must be invisible ``dark matter'' that provides additional gravitational pull.

\subsection{The FFGFT Alternative}

FFGFT offers an alternative explanation: the apparent dark matter is not a new type of particle, but a manifestation of the fractal structure of spacetime. On galactic scales, the effective gravitational potential is modified:
\begin{equation}
\Phi_{\text{eff}}(r) = -\frac{GM}{r} \left( 1 + \xi \ln\left(1 + \frac{r}{r_\xi}\right) \right)
\end{equation}

This logarithmic correction, arising from fractality, produces exactly the flat rotation curves observed in galaxies -- without requiring dark matter particles.

\section{Summary and Outlook}

Chapter 3 has shown how FFGFT solves the major problems of General Relativity:

\begin{itemize}[leftmargin=*]
\item Singularities are eliminated through the finite maximum curvature
\item The cosmological constant problem is resolved by the natural suppression factor $\xi$
\item The hierarchy problem is explained by gravity emerging from fractal geometry
\item Dark matter phenomena arise naturally from fractal corrections to gravity
\end{itemize}

In the next chapter, we will examine black holes in detail and derive the modified Schwarzschild metric that describes these fascinating objects without singularities.

\vspace{1cm}
\hrule
\vspace{0.5cm}
\noindent\textbf{Technical Note:} The modified gravitational potential and the resolution of the cosmological constant problem are derived rigorously in the technical supplements. The factor $\xi$ appears naturally in all corrections and requires no fine-tuning (see repository: \url{https://github.com/jpascher/T0-Time-Mass-Duality/tree/main/2/pdf}).