\chapter{Black Holes and Quantum Singularities – T0 Perspective (as of December 2025)  Narrative Version of FFGFT}


\section*{Black Holes and Quantum Singularities – T0 Perspective (as of December 2025)}
	
	\subsection*{Brief Introduction}
	
	This chapter examines black holes and singularities as central challenges in theoretical physics. In general relativity (GR), collapse scenarios lead to singularities with infinite curvature (e.g., Schwarzschild radius \(r=0\)). Quantum field theory (QFT) suffers from point-particle singularities (e.g., self-energy divergences). Both problems signal the need for quantum gravity.
	
	Current status (December 2025): Observations (Event Horizon Telescope, gravitational waves from LIGO/Virgo/KAGRA) confirm black holes, but singularities are not directly accessible. Approaches like Loop Quantum Gravity (LQG), string theory, and asymptotic safety propose resolutions, but remain unverified. The T0-based FFGFT offers a fractal-geometric alternative, resolving both types of singularities without new quantum degrees of freedom.
	
	\subsection*{Mathematical Foundation}
	
	In the FFGFT, singularities are eliminated through fractal regularisation of the vacuum field, regulated by \(\xi = \frac{4}{3} \times 10^{-4}\). There is no instantaneous action — all processes are causal and propagate at light speed.
	
	\subsection*{Black Holes in General Relativity}
	
	The Schwarzschild metric has a singularity at \(r=0\):
	
	\begin{equation}
		ds^2 = \left(1 - \frac{2GM}{c^2 r}\right) c^2 dt^2 - \left(1 - \frac{2GM}{c^2 r}\right)^{-1} dr^2 - r^2 d\Omega^2.
	\end{equation}
	
	Curvature diverges as \(r \to 0\), leading to breakdown of GR.
	
	\subsection*{Resolution in T0 Geometry}
	
	In the FFGFT, the vacuum amplitude saturates at high densities:
	
	\begin{equation}
		\rho(r) = \rho_0 \cdot \tanh\left(\frac{r_s}{r \xi}\right),
	\end{equation}
	
	where \(r_s = 2GM/c^2\). The hyperbolic tangent prevents divergence — density approaches a finite maximum \(\rho_0\), avoiding singularity.
	
	\textbf{Unit check:}
	
	\begin{equation}
		[\rho(r)] = \si{kg^{1/2}/m^{3/2}}.
	\end{equation}
	
	The interior becomes a stable “fractal star” with radius \(r \approx l_0 / \xi \approx 10^{-31} \ \si{m}\).
	
	\subsection*{Quantum Singularities in QFT}
	
	Point particles cause UV divergences, e.g., electron self-energy:
	
	\begin{equation}
		\Delta m \propto \frac{e^2}{\hbar c} \int^\Lambda \frac{dk}{k}.
	\end{equation}
	
	Logarithmic divergence requires renormalisation.
	
	\subsection*{Fractal Regularisation of Point Particles}
	
	Particles are extended phase windings:
	
	\begin{equation}
		\theta(r) = \pi + \xi \ln(r/l_0).
	\end{equation}
	
	The logarithmic profile smears the point — effective radius \(l_0 / \xi\), cutting off divergences.
	
	Amplitude deformation:
	
	\begin{equation}
		\delta \rho(x) = \frac{m c^2}{l_0^3} \cdot \xi \cdot \exp\left(-r^2 / (l_0^2 \xi^2)\right),
	\end{equation}
	
	Self-energy finite:
	
	\begin{equation}
		\Delta E \approx \frac{G m^2}{c^2 l_0 \xi}.
	\end{equation}
	
	Validation: Small and negligible; resolves UV divergences without renormalisation.
	
	\subsection*{Comparison with Other Approaches}
	
	\begin{itemize}
		\item LQG: Discrete spacetime, bounce instead of singularity,
		\item String theory: Minimal string length \(l_s\),
		\item Asymptotic safety: UV fixed point of gravity,
		\item T0: Fractal cutoff through \(\xi\), purely classical from vacuum dynamics.
	\end{itemize}
	
	T0 is minimal — no new quantum degrees of freedom or dimensions.
	
	Validation: Consistent with observed black holes (shadow, waves); predictions for echo chambers in mergers testable.
	
	\subsection*{Conclusion}
	
	While mainstream approaches (LQG, strings) regularise singularities through quantisation, T0 offers a coherent alternative: Classical and quantum singularities are uniformly eliminated through saturation of the vacuum amplitude \(\rho\) and fractal effects with \(\xi\). Everything remains finite — a natural consequence of the fractal vacuum structure.
	
	Validation: Conceptually consistent with GR and QFT; testable through gravitational wave echoes and future black hole images.



