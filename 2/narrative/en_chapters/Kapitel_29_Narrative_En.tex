\chapter{Chapter 29: The Delayed-Choice Quantum Eraser Experiment in Fractal T0 Geometry  Narrative Version of FFGFT}


\section*{Chapter 29: The Delayed-Choice Quantum Eraser Experiment in Fractal T0 Geometry}
	
	\subsection*{Brief Introduction}
	
	This chapter resolves the apparent paradox of the delayed-choice quantum eraser (DCQE) experiment through the global coherence of the fractal vacuum phase field.
	
	\subsection*{Mathematical Foundation}
	
	The DCQE experiment demonstrates that the decision to erase or retain which-path information influences the interference pattern of a photon — even if this decision is made after the photon has been detected at the screen. In the FFGFT, this arises from the global, fractal coherence of the vacuum phase field \(\theta(x,t)\), regulated by \(\xi = \frac{4}{3} \times 10^{-4}\).
	
	\subsection*{The DCQE Experiment – Setup and Observation}
	
	An entangled photon pair (signal and idler) is generated. The signal photon passes through a double slit and is registered at the screen detector \(D_0\). The idler photon can carry which-path information (detectors \(D_1, D_2\)) or erase it (erasure detectors \(D_3, D_4\)).
	
	The phase difference between signal and idler:
	
	\begin{equation}
		\Delta \theta = \theta_s - \theta_i.
	\end{equation}
	
	This difference \(\Delta \theta\) determines the interference pattern at the screen. When which-path information is available (\(D_1\) or \(D_2\)), \(\Delta \theta\) is known and no interference pattern appears. With erasure (\(D_3\) or \(D_4\)), \(\Delta \theta\) is unknown and the pattern emerges — even if the erasure decision is made after detection at the screen.
	
	\textbf{Unit check:}
	
	\begin{equation}
		[\Delta \theta] = \si{rad}.
	\end{equation}
	
	\subsection*{Fractal Global Coherence}
	
	The vacuum phase field \(\theta(x,t)\) is fractally correlated:
	
	\begin{equation}
		C(\Delta x) = \xi \ln(|\Delta x|/l_0) + \frac{\xi^2}{2} [\ln(|\Delta x|/l_0)]^2.
	\end{equation}
	
	The correlation function \(C(\Delta x)\) grows logarithmically with distance \(\Delta x\). The leading term \(\xi \ln(|\Delta x|/l_0)\) arises from summation over fractal levels, the quadratic term from higher-order corrections. This maintains phase coherence over large distances but with controlled, weak non-locality due to the small factor \(\xi\).
	
	\textbf{Unit check:}
	
	\begin{equation}
		[C(\Delta x)] = \text{dimensionless}.
	\end{equation}
	
	\subsection*{Erasure and Coherence Restoration}
	
	With erasure, which-path information is deleted:
	
	\begin{equation}
		V = |\langle e^{i \Delta \theta} \rangle| \approx 1 - \xi \cdot \Delta x / l_0.
	\end{equation}
	
	The visibility \(V\) is the magnitude of the expectation value of the phase factor exponential. The subtraction term \(\xi \cdot \Delta x / l_0\) slightly damps coherence at large separations, but \(V\) remains close to 1 — interference is fully restored.
	
	With which-path information:
	
	\begin{equation}
		V \approx \xi \cdot \Delta x / l_0 \ll 1.
	\end{equation}
	
	Visibility almost completely vanishes because the phase is known.
	
	\subsection*{No Retrocausality}
	
	The delayed decision does not change the past:
	
	\begin{equation}
		P(\text{click}|t_d) = P(\text{click}),
	\end{equation}
	
	The single-click probability at the screen is independent of the delay \(t_d\). Only post-selection of the data (which subset of clicks is considered) determines the pattern — the fractal phase remains globally consistent and deterministic.
	
	\subsection*{Comparison with Other Interpretations}
	
	\begin{center}
		\begin{tabular}{p{0.45\textwidth}p{0.45\textwidth}}
			\textbf{Other Interpretations} & \textbf{FFGFT (T0)} \\
			\hline
			Copenhagen: Collapse & Deterministic \\
			Many-Worlds: Branching & Unified phase \\
			Retrocausality & No time travel \\
			Ad-hoc & Parameter-free from \(\xi\) \\
		\end{tabular}
	\end{center}
	
	\subsection*{Conclusion}
	
	The DCQE is no paradox in the FFGFT: the apparent retrocausality arises from global fractal coherence of the vacuum phase. Erasure restores coherence in subsets without altering the past. Everything emerges from \(\xi\), unifying entanglement with the Time-Mass Duality.
