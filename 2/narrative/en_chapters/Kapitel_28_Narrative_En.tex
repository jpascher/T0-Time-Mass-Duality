\chapter{Why Newton's Law Does Not Apply to Quantum Particles in Fractal T0 Geometry  Narrative Version of FFGFT}


\section*{Why Newton's Law Does Not Apply to Quantum Particles in Fractal T0 Geometry}
	
	\subsection*{Brief Introduction}
	
	This chapter shows why Newton's classical law of gravitation does not hold on quantum scales and how the FFGFT provides a consistent quantum gravity at the particle level.
	
	\subsection*{Mathematical Foundation}
	
	Newton's law \(F = G m_1 m_2 / r^2\) assumes well-defined distances and point-like masses. For quantum particles in delocalised states, this assumption breaks down. In the FFGFT, gravity acts as a deformation of the vacuum amplitude \(\delta \rho\), regulated by \(\xi = \frac{4}{3} \times 10^{-4}\).
	
	\subsection*{The Classical Newtonian Law}
	
	Newton's law describes the force between two point-like masses:
	
	\begin{equation}
		F = G \frac{m_1 m_2}{r^2}.
	\end{equation}
	
	The formula assumes that both masses are localised at exact positions and that the distance \(r\) is unambiguously defined. The force acts instantaneously along the line connecting them.
	
	\textbf{Unit check:}
	
	\begin{equation}
		[F] = \si{m^{3}/kg/s^{2}} \cdot \si{kg^{2}} / \si{m^{2}} = \si{N}.
	\end{equation}
	
	For macroscopic objects, this works excellently because delocalisation is negligible.
	
	\subsection*{Problem on Quantum Scales}
	
	For quantum particles, the state is described by the wave function \(\psi(x)\). This is not an ontological object (no real “particle at multiple places simultaneously”) but a purely mathematical construct that encodes the probability distribution of measurement outcomes. The mass is therefore delocalised over the distribution \(|\psi(x)|^2\).
	
	A single proton has no fixed position — the distance \(r\) to another proton is undefined. The classical formula cannot be applied because there is no unambiguous \(r\).
	
	The term “superposition” in the FFGFT also denotes no ontological superposition of real states but a mathematical linear combination of possibilities in the description. The vacuum field itself is always in a single, deterministic state — the apparent superposition is an artefact of the epistemic description.
	
	\subsection*{Gravity as Amplitude Deformation}
	
	In the FFGFT, mass generates a deformation of the vacuum amplitude:
	
	\begin{equation}
		\delta \rho(x) = \xi^2 \cdot \rho_0 \cdot |\psi(x)|^2.
	\end{equation}
	
	The deformation \(\delta \rho\) is proportional to the probability density \(|\psi(x)|^2\) (the mathematical construct), scaled by \(\xi^2\) because the fractal structure strongly damps the coupling. The vacuum responds as a whole — gravity is non-local and follows the distribution of the wave function without requiring an ontological superposition.
	
	\textbf{Unit check:}
	
     \begin{equation}
	\begin{aligned}
		[\delta \rho]
		&= \text{dimensionless}
		\cdot \si{kg^{1/2}/m^{3/2}}
		\cdot \text{dimensionless} \\
		&= \si{kg^{1/2}/m^{3/2}}.
	\end{aligned}
\end{equation}
	
	\subsection*{Effective Force in Delocalised States}
	
	For two delocalised protons, an effective attraction emerges:
	
	\begin{equation}
		F_{\text{eff}} = \xi \cdot G \int |\psi_1(x)|^2 |\psi_2(y)|^2 \frac{m_p^2}{|x-y|^2} \, d^3x \, d^3y.
	\end{equation}
	
	The integral averages over all possible positions — the force is weaker and no longer point-like. The factor \(\xi\) arises from the fractal regularisation.
	
	\subsection*{Example: Gravity Between Two Protons}
	
	For a typical Fermi distance \(r = \SI{1}{fm} = \SI{e-15}{m}\):
	
	\begin{equation}
		F_g \approx \xi \cdot G \frac{m_p^2}{r^2} \approx \SI{e-40}{N}.
	\end{equation}
	
	The classical force would be enormous, but it is extremely damped by \(\xi\). The formula applies only approximately for delocalised states — the true gravity is the integral deformation.
	
	\textbf{Unit check:}
	
	\begin{equation}
		[F_g] = \text{dimensionless} \cdot \si{N} = \si{N}.
	\end{equation}
	
	\subsection*{Comparison Classical – Quantum Gravity}
	
	\begin{center}
		\begin{tabular}{p{0.45\textwidth}p{0.45\textwidth}}
			\textbf{Classical Gravitation} & \textbf{FFGFT Quantum Gravity} \\
			\hline
			Point-like, instantaneous & Delocalised, non-local \\
			Defined $r$ & Integral over $|\psi|^2$ \\
			Paradoxes in superposition & Unified field \\
			Only macroscopic & Consistent on all scales \\
		\end{tabular}
	\end{center}
	
	\subsection*{Conclusion}
	
	The FFGFT defines gravity on quantum scales as amplitude deformation \(\delta \rho \propto |\psi|^2\). The wave function \(\psi\) and superpositions are purely mathematical constructs for describing probabilities — not ontological realities. The vacuum field is always deterministic and unified. This resolves every paradox, and gravity acts consistently on all scales, all from the single fundamental parameter \(\xi = \frac{4}{3} \times 10^{-4}\).



