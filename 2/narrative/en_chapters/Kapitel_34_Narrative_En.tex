\chapter{Solution to the Strong CP Problem in Fractal T0 Geometry  Narrative Version of FFGFT}


\section*{Solution to the Strong CP Problem in Fractal T0 Geometry}
	
	\subsection*{Brief Introduction}
	
	This chapter resolves the strong CP problem through intrinsic regularisation of the vacuum phase field — without an axion or fine-tuning.
	
	\subsection*{Mathematical Foundation}
	
	The strong CP problem asks why the CP-violating parameter \(\theta_{\text{QCD}}\) in QCD is smaller than \(10^{-10}\), although it should naturally be O(1). In the FFGFT, \(\theta_{\text{QCD}}\) is relaxed to zero by fractal non-locality, regulated by \(\xi = \frac{4}{3} \times 10^{-4}\).
	
	\subsection*{The CP-Violating Term in QCD}
	
	The QCD Lagrangian contains the topological term:
	
	\begin{equation}
		\mathcal{L}_{\theta} = \theta_{\text{QCD}} \frac{g^2}{32\pi^2} G \tilde{G}.
	\end{equation}
	
	This term violates CP symmetry and induces a neutron electric dipole moment (EDM).
	
	The experimental limit:
	
	\begin{equation}
		|\theta_{\text{QCD}}| < 10^{-10}.
	\end{equation}
	
	Without a mechanism, this is extreme fine-tuning.
	
	\textbf{Unit check:}
	
	\begin{equation}
		[\mathcal{L}_{\theta}] = \text{dimensionless} \cdot \si{1/m^{4}} = \si{1/m^{4}}.
	\end{equation}
	
	\subsection*{Fractal Regularisation of the Phase}
	
	The vacuum phase field \(\theta(x,t)\) is fractally correlated:
	
	\begin{equation}
		\langle \theta(x) \theta(y) \rangle = \xi \ln(|x-y|/l_0) + \frac{\xi^2}{2} [\ln(|x-y|/l_0)]^2.
	\end{equation}
	
	The logarithmic term sums over hierarchy levels and relaxes global \(\theta\) to zero — local fluctuations remain small.
	
	\textbf{Unit check:}
	
	\begin{equation}
		[\langle \theta \theta \rangle] = \text{dimensionless}.
	\end{equation}
	
	\subsection*{Relaxation of the \(\theta\)-Term}
	
	The effective \(\theta_{\text{QCD}}\):
	
	\begin{equation}
		\theta_{\text{QCD}}^{\text{eff}} \approx \xi^2 \cdot \langle \delta \theta \rangle \approx 10^{-8}.
	\end{equation}
	
	The double \(\xi^2\) factor naturally suppresses the parameter below the EDM limit.
	
	\subsection*{Neutron EDM}
	
	The induced dipole moment:
	
	\begin{equation}
		d_n \approx \theta_{\text{QCD}} \cdot 10^{-16} \, e \cdot \text{cm}.
	\end{equation}
	
	With \(\theta_{\text{QCD}}^{\text{eff}} < 10^{-8}\), \(d_n < 10^{-24} \, e \cdot \text{cm}\) — far below current limits but testable in the future.
	
	\subsection*{Comparison with Axion Solution}
	
	\begin{center}
		\begin{tabular}{p{0.45\textwidth}p{0.45\textwidth}}
			\textbf{Axion} & \textbf{FFGFT (T0)} \\
			\hline
			New particle & No new field \\
			Avoids fine-tuning & Geometrically relaxed \\
			Cold dark matter & Vacuum effect \\
			Testable through search & EDM prediction \\
		\end{tabular}
	\end{center}
	
	\subsection*{Conclusion}
	
	The FFGFT resolves the strong CP problem through fractal relaxation of the vacuum phase — \(\theta_{\text{QCD}}\) is geometrically set near zero, without axion or fine-tuning. The prediction \(|\theta_{\text{QCD}}| \approx \xi^2\) is testable through more precise neutron EDM measurements and underscores the universal role of \(\xi\).
