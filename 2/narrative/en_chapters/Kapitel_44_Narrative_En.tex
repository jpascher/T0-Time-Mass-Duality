\chapter{Qubits, Schrödinger Equation and Dirac Equation in Fractal T0 Geometry  Narrative Version of FFGFT}


\section*{Qubits, Schrödinger Equation and Dirac Equation in Fractal T0 Geometry}
	
	\subsection*{Narrative Introduction: The Cosmic Brain in Detail}
	
	We continue our journey through the cosmic brain. In this chapter, we examine further aspects of the fractal structure of the universe, which – like the complex windings of a brain – manifest on all scales. The quantum bit (qubit), the Schrödinger equation, and the Dirac equation emerge as mathematical constructs from the dynamics of the vacuum phase field. There is no ontological superposition — the vacuum remains deterministic.
	
	\subsection*{Mathematical Foundation}
	
	The qubit, Schrödinger equation, and Dirac equation are central to quantum mechanics and relativistic quantum theory. In the FFGFT, they emerge from the phase excitations of the vacuum field \(\Phi = \rho e^{i\theta}\), regulated by \(\xi = \frac{4}{3} \times 10^{-4}\).
	
	\subsection*{Qubits as Phase Excitations}
	
	A qubit is a local phase excitation:
	
	\begin{equation}
		|q\rangle = \cos(\theta/2) |0\rangle + e^{i\phi} \sin(\theta/2) |1\rangle \approx e^{i \delta \theta}.
	\end{equation}
	
	The phase \(\delta \theta\) encodes the state — superposition is a mathematical construct, not ontological.
	
	\textbf{Unit check:}
	
	\begin{equation}
		[\delta \theta] = \text{dimensionless}.
	\end{equation}
	
	\subsection*{Schrödinger Equation from Phase Dynamics}
	
	The non-relativistic Schrödinger equation emerges as:
	
	\begin{equation}
		i \hbar \partial_t \psi = -\frac{\hbar^2}{2m} \nabla^2 \psi + V \psi + \xi \cdot \hbar^2 (\nabla \ln \rho) \cdot \nabla \psi.
	\end{equation}
	
	The additional term \(\xi \cdot \hbar^2 (\nabla \ln \rho) \cdot \nabla \psi\) couples to amplitude gradients — small fractal correction.
	
	\subsection*{Dirac Equation from Relativistic Phase}
	
	The Dirac equation simplifies to phase dynamics:
	
	\begin{equation}
		i \hbar \gamma^\mu \partial_\mu \psi - m c \psi = \xi \cdot i \hbar \gamma^\mu (\partial_\mu \theta) \psi.
	\end{equation}
	
	The correction term \(\xi \cdot i \hbar \gamma^\mu (\partial_\mu \theta) \psi\) incorporates vacuum phase gradients.
	
	\subsection*{Spin as Topological Phase Winding}
	
	Spin-1/2 arises from half-integer winding:
	
	\begin{equation}
		\theta_s = \pi + 2\pi n.
	\end{equation}
	
	Rotation by 360° reverses sign — explains fermionic statistics without postulate.
	
	\subsection*{Qubit Gates as Phase Manipulations}
	
	Example Hadamard gate:
	
	\begin{equation}
		H: \delta \theta \to \frac{\delta \theta + \pi/2}{\sqrt{2}}.
	\end{equation}
	
	It rotates the phase by 90° and normalises — creates uniform superposition. All gates are local phase operations on the vacuum field.
	
	\subsection*{Comparison Standard – FFGFT}
	
	\begin{center}
		\begin{tabular}{p{0.45\textwidth}p{0.45\textwidth}}
			\textbf{Standard} & \textbf{FFGFT (T0)} \\
			\hline
			Qubit postulated & Local phase excitation \\
			Schrödinger/Dirac axiomatic & Extended and simplified form \\
			Spin ad-hoc & Topological winding of phase \\
			No gravity & Unified with amplitude \\
			Ontological superposition & Mathematical construct \\
		\end{tabular}
	\end{center}
	
	\subsection*{Conclusion}
	
	The FFGFT derives qubits as phase excitations, the Schrödinger equation as extended non-relativistic form, and the Dirac equation as simplified relativistic approximation. Superposition and wave function are purely mathematical constructs — the vacuum field remains deterministic. Spin is a topological property of the phase. Everything emerges parameter-free from \(\xi\), unifies quantum computing with fundamental physics.



