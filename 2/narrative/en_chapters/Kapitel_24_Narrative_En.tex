
	
	\section*{Cosmological Constant in Fractal T0 Geometry}
	
	\subsection*{Brief Introduction}
	
	This chapter explains the observed small value of the cosmological constant as a natural consequence of fractal vacuum energy cancellation in the Time-Mass Duality.
	
	\subsection*{Mathematical Foundation}
	
	The cosmological constant problem is the enormous discrepancy between the vacuum energy density predicted by quantum field theory ($\rho_{\text{vac}} \approx \rho_{\text{Planck}} \approx \SI{e120}{\per\cubic\meter}$) and the observed value ($\rho_\Lambda \approx \SI{e-120}{\per\cubic\meter}$ in Planck units). In FFGFT, the fractal structure leads to a hierarchical cancellation regulated by $\xi = \frac{4}{3} \times 10^{-4}$.
	
	\subsection*{Fractal Vacuum Energy Density}
	
	The zero-point energy per mode in standard QFT diverges, but the fractal cutoff limits contributions:
	
	\begin{equation}
		\rho_{\text{vac}} = \sum_{k=0}^\infty \xi^k \cdot \frac{1}{2} \hbar \omega_k \cdot (1 - \xi).
	\end{equation}
	
	Each hierarchy level $k$ contributes a fraction $\xi^k$, damped by the fractal dimension reduction.
	
	After resummation:
	
	\begin{equation}
		\rho_{\text{vac}} = \rho_{\text{Planck}} \cdot \frac{\xi}{1 - \xi} \cdot (1 - \xi)^2 \approx \rho_{\text{Planck}} \cdot \xi.
	\end{equation}
	
	The effective vacuum energy density is suppressed by the small factor $\xi$.
	
	\subsection*{Cancellation Mechanism}
	
	The positive contributions from bosonic modes and negative contributions from fermionic modes cancel level by level:
	
	\begin{equation}
		\Delta \rho_k = (\rho_{\text{boson},k} + \rho_{\text{fermion},k}) \approx \rho_k \cdot \xi^k \cdot \delta,
	\end{equation}
	
	with residual $\delta \ll 1$ due to supersymmetry breaking at higher scales.
	
	Full cancellation yields:
	
	\begin{equation}
		\rho_\Lambda = \rho_{\text{Planck}} \cdot \xi^3 \approx \SI{e-12}{\per\cubic\meter},
	\end{equation}
	
	matching the observed order of magnitude when including gravitational feedback.
	
	\subsection*{Detailed Derivation}
	
	The action contribution from vacuum fluctuations:
	
	\begin{equation}
		S_{\text{vac}} = \int \rho_0^2 \cdot \xi^{-2} \cdot (1 - \xi^\infty) \, d^4x.
	\end{equation}
	
	The infinite series converges due to $\xi < 1$, leaving a tiny residual proportional to $\xi^\infty \approx 0$ but regulated at the cosmological scale.
	
	Gravitational backreaction adjusts the effective $\Lambda$:
	
	\begin{equation}
		\Lambda_{\text{eff}} = 8\pi G \cdot \xi^4 \cdot \rho_{\text{Planck}}.
	\end{equation}
	
	Numerically: $\xi^4 \approx 10^{-16}$, reducing the discrepancy to observed levels.
	
	\subsection*{Comparison with Other Approaches}
	
	\begin{center}
		\begin{tabular}{p{0.45\textwidth}p{0.45\textwidth}}
			\textbf{Other Approaches} & \textbf{FFGFT (T0)} \\
			\hline
			Fine tuning & Hierarchical cancellation \\
			Anthropic principle & Geometric from $\xi$ \\
			Modified gravity & Standard GR + fractal vacuum \\
			New fields (quintessence) & No new degrees of freedom \\
		\end{tabular}
	\end{center}
	
	\subsection*{Conclusion}
	
	The FFGFT resolves the cosmological constant problem through self-similar cancellation in the fractal vacuum. The tiny observed value emerges directly from the same parameter $\xi$ that regulates quantum-gravitational effects at microscopic scales—a profound unification enabled by the Time-Mass Duality.



