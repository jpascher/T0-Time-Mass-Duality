\documentclass[12pt,a4paper]{article}
\usepackage[utf8]{inputenc}
\usepackage[T1]{fontenc}
\usepackage[english]{babel}
\usepackage{lmodern}
\usepackage[
  paperwidth=6in,
  paperheight=9in,
  inner=0.625in,
  outer=0.625in,
  top=0.625in,
  bottom=1.0in,
  twoside
]{geometry}
\usepackage{amsmath,amssymb,amsfonts,amsthm}
\usepackage{mathtools}
\usepackage{physics}
\usepackage{graphicx}
\usepackage{hyperref}
\usepackage{enumitem}
\usepackage{siunitx}
\usepackage{hyphenat}

% Kindle text/wrapping optimization
\sloppy
\emergencystretch=3em
\hyphenpenalty=500
\tolerance=2000
\hbadness=2000

% Kindle math optimization
\everymath{\displaystyle}
\everydisplay{\displaystyle}
\relpenalty=10
\binoppenalty=10
\DeclareMathSizes{11}{11}{9}{8}
\DeclareMathSizes{10.95}{11}{9}{8}

% siunitx custom units
\DeclareSIUnit\gigalightyear{Gly}
\DeclareSIUnit\lightyear{ly}
\DeclareSIUnit\mev{MeV}
\DeclareSIUnit\gev{GeV}

\title{\textbf{Chapter 1: The Fundamental Fractal-Geometric Field} \\
\large A New View of Reality \\
\normalsize Narrative Version of FFGFT}
\author{}
\date{}

\begin{document}

\maketitle

\section*{Introduction: One Number to Describe the Universe}

Imagine you could describe the entire universe with just a single number. Not with dozens of natural constants, not with complex systems of equations spanning pages, but with a single geometric parameter -- a magic number that determines the very fabric of spacetime itself. This is precisely the revolutionary idea behind the Fundamental Fractal-Geometric Field Theory, or FFGFT for short (formerly known as T0 Theory).

This magic number is:
\begin{equation}
\xi = \frac{4}{3} \times 10^{-4}
\end{equation}

It is dimensionless, a pure number without units -- approximately 0.000133 or more precisely: four-thirds of one ten-thousandth. And from this tiny number, which appears completely inconspicuous at first glance, all fundamental properties of our universe emerge: the speed of light, the gravitational constant, Planck's constant, the fine-structure constant -- simply everything.

\section{The Universe as a Fractal Structure}

To understand what this number means, we must first look at fractal structures. Think of a snowflake: the closer you zoom in, the more details reveal themselves. Its structure repeats on ever smaller scales, yet it remains essentially similar -- self-similar, as mathematicians say. Or think of a coastline: whether you view it from space or walk along the beach, you find the same jagged patterns everywhere, just at different scales.

FFGFT now states something astonishing: spacetime itself -- the fabric from which our universe is woven -- possesses such a fractal structure. It is not smooth and continuous, as Einstein imagined it, but has a finely structured, self-similar architecture on the very smallest scales. And the parameter $\xi$ describes precisely this structure.

\subsection{The Fractal Dimension of Spacetime}

Specifically, $\xi$ defines the \textbf{fractal dimension} of spacetime:
\begin{equation}
D_f = 3 - \xi \approx 2.999867
\end{equation}

In our everyday life, we experience spacetime as three-dimensional -- left-right, forward-backward, up-down. But on the very smallest scales, near the so-called Planck length (about $10^{-35}$ meters, an unimaginably tiny distance), the dimensionality deviates slightly from the number 3. It amounts to approximately 2.999867. This tiny difference -- only 0.000133 -- may seem negligible, but it has dramatic consequences: it regularizes the otherwise infinite divergences of quantum field theory, prevents singularities in black holes, and explains phenomena we have previously attributed to dark matter -- all without additional, mysterious components.

\subsection{The Central Metaphor: The Universe as a Growing Brain}

