% Chapter file: 016_T0_Vollstaendige_Berchnungen_En_ch.tex
% Source: 016_T0_Vollstaendige_Berchnungen_En.tex

% Original: \chapter{\textbf{FFGFT: Calculation of Particle Masses and Physical Constants}
\chapter{FFGFT: Calculation of Particle Masses and Physical Co...}

\hfuzz=200pt
\allowdisplaybreaks

\section*{Abstract}
		the FFGFT presents a new approach to unifying particle physics and cosmology by deriving all fundamental masses and physical constants from just three geometric parameters: the constant $\xi = \frac{4}{3} \times 10^{-4}$, the Planck length $\ell_P = 1.616e-35$ m, and the characteristic energy $E_0 = 7.398$ MeV, where energy can also be derived. This version demonstrates the remarkable precision of the T0 framework with over 99\% accuracy for fundamental constants.
	
	
	\section{Introduction}
	
	the FFGFT is based on the fundamental hypothesis of a geometric constant $\xi$ that unifies all physical phenomena on macroscopic and microscopic scales. Unlike standard approaches based on empirical adjustments, T0 derives all parameters from exact mathematical relationships.
	
	\subsection{Fundamental Parameters}
	
	The entire T0 system is based solely on three input values:
	
	\begin{align}
		\xi &= \frac{4}{3} \times 10^{-4} \approx 1.33333333e-04 \quad \text{(geometric constant)} \\
		\ell_P &= 1.616e-35 \text{ m} \quad \text{(Planck length)} \\
		E_0 &= 7.398 \text{ MeV} \quad \text{(characteristic energy)} \\
		v &= 246.0 \text{ GeV} \quad \text{(Higgs VEV)}
	\end{align}
	
	\section{T0 Fundamental Formula for the Gravitational Constant}
	
	\subsection{Mathematical Derivation}
	
	The central insight of the FFGFT is the relationship:
	\begin{equation}
		\xi = 2\sqrt{G \cdot m_{\text{char}}}
	\end{equation}
	
	where $m_{\text{char}} = \xi/2$ is the characteristic mass. Solving for $G$ yields:
	
	\begin{equation}
		\boxed{G = \frac{\xi^2}{4m_{\text{char}}} = \frac{\xi^2}{4 \cdot (\xi/2)} = \frac{\xi}{2}}
	\end{equation}
	
	\subsection{Dimensional Analysis}
	
	In natural units ($\hbar = c = 1$), the T0 basic formula initially gives:
	\begin{equation}
		[G_{\text{T0}}] = \frac{[\xi^2]}{[m]} = \frac{[1]}{[E]} = [E^{-1}]
	\end{equation}
	
	Since th
% TABLE CONVERTED TO LIST FORMAT FOR KDP COMPLIANCE
% Original table was too complex (many columns/rows)

\begin{itemize}
    \item Fundamental -- 1 -- 0.0005 -- 0.0005 -- 0.0005 -- Excellent
    \item Gravitation -- 1 -- 0.0125 -- 0.0125 -- 0.0125 -- Excellent
    \item Planck -- 6 -- 0.0131 -- 0.0062 -- 0.0220 -- Excellent
    \item Electromagnetic -- 4 -- 0.0001 -- 0.0000 -- 0.0002 -- Excellent
    \item Atomic Physics -- 7 -- 0.0005 -- 0.0000 -- 0.0009 -- Excellent
    \item Metrology -- 5 -- 0.0002 -- 0.0000 -- 0.0005 -- Excellent
    \item Thermodynamics -- 3 -- 0.0008 -- 0.0000 -- 0.0023 -- Excellent
    \item Cosmology -- 4 -- 11.6528 -- 0.0601 -- 45.6741 -- Acceptable
    \item \textbf{Constant} -- \textbf{Symbol} -- \textbf{T0 Value} -- \textbf{Reference Value} -- \textbf{Error [\%]} -- \textbf{Unit}
    \item \textbf{Constant} -- \textbf{Symbol} -- \textbf{T0 Value} -- \textbf{Reference Value} -- \textbf{Error [\%]} -- \textbf{Unit}
    \item Fine-structure constant -- $\alpha$ -- 7.297e-03 -- 7.297e-03 -- 0.0005 -- \text{dimensionless}
    \item Gravitational constant -- $G$ -- 6.673e-11 -- 6.674e-11 -- 0.0125 -- $\si{\meter^3 \kilogram^{-1} \second^{-2}}$
    \item Planck mass -- $m_P$ -- 2.177e-08 -- 2.176e-08 -- 0.0062 -- $\si{\kilogram}$
    \item Planck time -- $t_P$ -- 5.390e-44 -- 5.391e-44 -- 0.0158 -- $\si{\second}$
    \item Planck temperature -- $T_P$ -- 1.417e+32 -- 1.417e+32 -- 0.0062 -- $\si{\kelvin}$
    \item Speed of light -- $c$ -- 2.998e+08 -- 2.998e+08 -- 0.0000 -- $\si{\meter \per \second}$
    \item Reduced Planck constant -- $\hbar$ -- 1.055e-34 -- 1.055e-34 -- 0.0000 -- $\si{\joule \second}$
    \item Planck energy -- $E_P$ -- 1.956e+09 -- 1.956e+09 -- 0.0062 -- $\si{\joule}$
    \item Planck force -- $F_P$ -- 1.211e+44 -- 1.210e+44 -- 0.0220 -- $\si{\newton}$
    \item Planck power -- $P_P$ -- 3.629e+52 -- 3.628e+52 -- 0.0220 -- $\si{\watt}$
    \item Magnetic constant -- $\mu_0$ -- 1.257e-06 -- 1.257e-06 -- 0.0000 -- $\si{\henry \per \meter}$
    \item Electric constant -- $\epsilon_0$ -- 8.854e-12 -- 8.854e-12 -- 0.0000 -- $\si{\farad \per \meter}$
    \item Elementary charge -- $e$ -- 1.602e-19 -- 1.602e-19 -- 0.0002 -- $\si{\coulomb}$
    \item Impedance of free space -- $Z_0$ -- 3.767e+02 -- 3.767e+02 -- 0.0000 -- $\si{\ohm}$
    \item Coulomb constant -- $k_e$ -- 8.988e+09 -- 8.988e+09 -- 0.0000 -- $\si{\newton \meter^2 \per \coulomb^2}$
    \item Stefan-Boltzmann constant -- $\sigma_{SB}$ -- 5.670e-08 -- 5.670e-08 -- 0.0000 -- $\si{\watt \per \meter^2 \kelvin^4}$
    \item Wien constant -- $b$ -- 2.898e-03 -- 2.898e-03 -- 0.0023 -- $\si{\meter \kelvin}$
    \item Planck constant -- $h$ -- 6.626e-34 -- 6.626e-34 -- 0.0000 -- $\si{\joule \second}$
    \item Bohr radius -- $a_0$ -- 5.292e-11 -- 5.292e-11 -- 0.0005 -- $\si{\meter}$
    \item Rydberg constant -- $R_\infty$ -- 1.097e+07 -- 1.097e+07 -- 0.0009 -- $\si{\meter^{-1}}$
    \item Bohr magneton -- $\mu_B$ -- 9.274e-24 -- 9.274e-24 -- 0.0002 -- $\si{\joule \per \tesla}$
    \item Nuclear magneton -- $\mu_N$ -- 5.051e-27 -- 5.051e-27 -- 0.0002 -- $\si{\joule \per \tesla}$
    \item Hartree energy -- $E_h$ -- 4.360e-18 -- 4.360e-18 -- 0.0009 -- $\si{\joule}$
    \item Compton wavelength -- $\lambda_C$ -- 2.426e-12 -- 2.426e-12 -- 0.0000 -- $\si{\meter}$
    \item Classical electron radius -- $r_e$ -- 2.818e-15 -- 2.818e-15 -- 0.0005 -- $\si{\meter}$
    \item Faraday constant -- $F$ -- 9.649e+04 -- 9.649e+04 -- 0.0002 -- $\si{\coulomb \per \mole}$
    \item von Klitzing constant -- $R_K$ -- 2.581e+04 -- 2.581e+04 -- 0.0005 -- $\si{\ohm}$
    \item Josephson constant -- $K_J$ -- 4.836e+14 -- 4.836e+14 -- 0.0002 -- $\si{\hertz \per \volt}$
    \item Magnetic flux quantum -- $\Phi_0$ -- 2.068e-15 -- 2.068e-15 -- 0.0002 -- $\si{\weber}$
    \item Gas constant -- $R$ -- 8.314e+00 -- 8.314e+00 -- 0.0000 -- $\si{\joule \per \mole \kelvin}$
    \item Loschmidt constant -- $n_0$ -- 2.687e+22 -- 2.687e+25 -- 99.9000 -- $\si{\meter^{-3}}$
    \item Hubble constant -- $H_0$ -- 2.196e-18 -- 2.196e-18 -- 0.0000 -- $\si{\second^{-1}}$
    \item Cosmological constant -- $\Lambda$ -- 1.610e-52 -- 1.105e-52 -- 45.6741 -- $\si{\meter^{-2}}$
    \item Age of Universe -- $t_{\text{Universe}}$ -- 4.554e+17 -- 4.551e+17 -- 0.0601 -- $\si{\second}$
    \item Critical density -- $\rho_{\text{crit}}$ -- 8.626e-27 -- 8.558e-27 -- 0.7911 -- $\si{\kilogram \per \meter^3}$
    \item Hubble length -- $l_{\text{Hubble}}$ -- 1.365e+26 -- 1.364e+26 -- 0.0862 -- $\si{\meter}$
    \item Boltzmann constant -- $k_B$ -- 1.381e-23 -- 1.381e-23 -- 0.0000 -- $\si{\joule \per \kelvin}$
    \item Avogadro constant -- $N_A$ -- 6.022e+23 -- 6.022e+23 -- 0.0000 -- $\si{\mole^{-1}}$
    \item \text{Factor 1: } -- 3{.}521 \times 10^{-2} \quad \text{[E}^{-1} \rightarrow \text{E}^{-2}\text{]}
    \item \text{Factor 2: } -- 2{.}843 \times 10^{-5} \quad \text{[E}^{-2} \rightarrow \si{\meter^3 \kilogram^{-1} \second^{-2}}\text{]}
    \item \textbf{Fundamental} -- $\alpha$, $m_{\text{char}}$ (directly from $\xi$)
    \item \textbf{Gravitation} -- $G$, $G_{\text{nat}}$, conversion factors
    \item \textbf{Planck} -- $m_P$, $t_P$, $T_P$, $E_P$, $F_P$, $P_P$
    \item \textbf{Electromagnetic} -- $e$, $\epsilon_0$, $\mu_0$, $Z_0$, $k_e$
    \item \textbf{Atomic Physics} -- $a_0$, $R_\infty$, $\mu_B$, $\mu_N$, $E_h$, $\lambda_C$, $r_e$
    \item \textbf{Metrology} -- $R_K$, $K_J$, $\Phi_0$, $F$, $R_{\text{gas}}$
    \item \textbf{Thermodynamics} -- $\sigma_{SB}$, Wien constant, $h$
    \item \textbf{Cosmology} -- $H_0$, $\Lambda$, $t_{\text{Universe}}$, $\rho_{\text{crit}}$
    \item \textbf{Category} -- \textbf{Count} -- \textbf{Average Error [\%]}
    \item Fundamental -- 1 -- 0.0005
    \item Gravitation -- 1 -- 0.0125
    \item Planck -- 6 -- 0.0131
    \item Electromagnetic -- 4 -- 0.0001
    \item Atomic Physics -- 7 -- 0.0005
    \item Metrology -- 5 -- 0.0002
    \item Thermodynamics -- 3 -- 0.0008
    \item Cosmology -- 4 -- 11.6528
    \item \textbf{Total} -- 45 -- 1.4600
\end{itemize}
