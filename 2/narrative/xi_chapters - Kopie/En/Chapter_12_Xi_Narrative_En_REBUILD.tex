% Auto-reconstructed from FFGFT_Xi_Narrative_Master_De_print.pdf
% RAW source: 2\narrative\xi_de_chapters_raw\Kapitel_11_Xi_De_raw.txt
% English translation

\chapter{Calculating with Time-Mass Duality}


This chapter offers some extended calculation examples that demonstrate how concrete quantities can be estimated using a few formulas of time-mass duality. The examples are deliberately kept simple and do not replace complete technical derivations, but they make the working of the approach transparent.

\section{From $\xi$ and $E_0$ to the Fine-Structure Constant}

The starting point is the number
\begin{equation}
	\xi = \frac{4}{3} \times 10^{-4}
	\label{eq:xi_value}
\end{equation}
and the scale obtained from the lepton hierarchy
\begin{equation}
	E_0 \approx 7.4\ \text{MeV}.
	\label{eq:E0_value}
\end{equation}

The relation introduced in earlier chapters is
\begin{equation}
	\alpha(\xi, E_0) = \xi \left( \frac{E_0}{1\ \text{MeV}} \right)^2.
	\label{eq:alpha_relation}
\end{equation}

Inserting the values gives schematically
\begin{equation}
	\alpha \approx (4/3 \times 10^{-4}) \times (7.4)^2.
	\label{eq:alpha_schematic}
\end{equation}

The squaring yields
\begin{equation}
	(7.4)^2 \approx 54.76,
\end{equation}
so that
\begin{equation}
	\alpha \approx 1.333 \times 10^{-4} \times 54.76 \approx 0.007297
\end{equation}
and thus
\begin{equation}
	\frac{1}{\alpha} \approx 137.0.
\end{equation}

Fine details such as rounding errors and higher-order corrections shift the last decimal place; what matters here is that the structure
\begin{equation}
	\alpha \sim \xi E_0^2
	\label{eq:alpha_structure}
\end{equation}
is consistent with the observed fine-structure constant. The example shows how directly $\xi$ and a single scale $E_0$ enter into a central constant of nature.

\section{From CMB Energy Density to the Scale $L_\xi$}

A second example concerns the connection between the CMB and the Casimir effect. Starting from the observed energy density of the cosmic microwave background $\rho_{\text{CMB}}$ and the relation
\begin{equation}
	\rho_{\text{CMB}} = \frac{\xi \hbar c}{L_\xi^4}
	\label{eq:cmb_relation}
\end{equation}
the possibility opens up to estimate a characteristic vacuum length $L_\xi$.

Solving the equation for $L_\xi$ gives
\begin{equation}
	L_\xi = \left( \frac{\xi \hbar c}{\rho_{\text{CMB}}} \right)^{1/4}.
	\label{eq:Lxi_definition}
\end{equation}

Inserting the known values for $\hbar$, $c$ and $\rho_{\text{CMB}}$ yields a value on the order of
\begin{equation}
	L_\xi \sim 100\ \mu\text{m}.
	\label{eq:Lxi_value}
\end{equation}

This is precisely the scale at which precise Casimir experiments are particularly sensitive. Thus, time-mass duality connects a cosmological quantity (CMB energy density) with a laboratory phenomenon on the micrometer scale.

\section{Fractal Dimension as an Everyday Approximation}

The fractal dimension of spacetime is
\begin{equation}
	D_f = 3 - \xi \approx 2.999867.
	\label{eq:fractal_dimension}
\end{equation}

In everyday life, this difference from smooth 3D geometry appears vanishingly small. However, for integrals over extremely high momenta or very small distances, it acts like an additional exponent that decides convergence or divergence.

A simple heuristic is:
\begin{itemize}
	\item Where classical theories use integrals of the form $\int d^3k$, in FFGFT an effectively slightly changed measure $\int d^{D_f}k$ appears.
	\item The tiny reduction of $D_f$ is sufficient to convert many divergent contributions into finite, regulated quantities.
\end{itemize}

This everyday perspective makes clear that the numerical values of $\xi$ and $D_f$ are not detached from the known dimensions, but only shift them minimally – with large effects in the UV regime.

\section{How to Continue Calculating}

The examples shown here are deliberately kept simple and are intended to invite readers to perform their own estimation calculations. Those who wish to delve deeper into the details will find complete derivations and numerical studies in the technical volumes of FFGFT.

For practical work, it is advisable to
\begin{itemize}
	\item take central formulas of time-mass duality (e.g., for $\alpha$, $E_0$, $L_\xi$) as a starting point,
	\item initially calculate purely based on ratios and with integer or rational numbers (without early floating-point approximations and without early introduction of constants like $\pi$) to maintain numerical precision for very small quantities,
	\item estimate the effects of small variations in $\xi$ or the scales, and
	\item systematically test new data – for example, on precise constants or Casimir measurements – against these structures.
\end{itemize}

In this way, time-mass duality becomes a manageable tool: It provides not only a conceptual explanation but also concrete computational pathways with which known and new phenomena can be quantitatively classified.


