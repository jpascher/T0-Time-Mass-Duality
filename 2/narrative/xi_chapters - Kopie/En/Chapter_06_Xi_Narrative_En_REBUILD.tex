% Chapter 06: Units, Scales, and Constants from ξ
% Completely rewritten with correct formulas
% Base: 013_T0_SI_De.tex, 015_NatEinheitenSystematik_De.tex
% English translation

\chapter{Units, Scales, and Constants from $\xi$}

\section{Introduction}

A central promise of FFGFT is that all fundamental constants of physics 
can be derived from the single parameter $\xi$. In this chapter, we show 
how this works concretely – from the gravitational constant $G$ through 
the Planck length $l_P$ to the Boltzmann constant $k_B$.

\section{Natural Units}

\subsection{The Concept}

In theoretical physics, \textbf{natural units} are frequently used, where 
fundamental constants are set to 1:

\begin{equation}
	\hbar = c = 1
	\label{eq:natural_units}
\end{equation}

In this system, all quantities have dimensions of energy $E$ (or powers thereof):

\begin{align}
	[M] &= [E] \quad \text{(from } E = mc^2\text{)} \\
	[L] &= [E^{-1}] \quad \text{(from } \lambda = \hbar/p\text{)} \\
	[T] &= [E^{-1}] \quad \text{(from } \omega = E/\hbar\text{)}
\end{align}

\subsection{Dimensional Analysis of the Gravitational Constant}

The gravitational constant has the dimension in natural units:

\begin{equation}
	[G] = [M^{-1}L^3T^{-2}] = [E^{-1}][E^{-3}][E^2] = [E^{-2}]
	\label{eq:G_dimension}
\end{equation}

\section{Derivation of the Gravitational Constant}

\subsection{Fundamental T0 Formula}

The gravitational constant follows from $\xi$ and the electron mass:

\begin{equation}
	G = \frac{\xi^2}{4 m_e}
	\label{eq:G_fundamental}
\end{equation}

in natural units.

\subsection{Complete Formula with SI Conversion}

For conversion to SI units, we need additional factors:

\begin{equation}
	\boxed{G_{\text{SI}} = \frac{\xi^2}{4 m_e} \times C_{\text{conv}} \times \mathcal{K}}
	\label{eq:G_complete}
\end{equation}

where:
\begin{itemize}
	\item $\xi = \frac{4}{3} \times 10^{-4}$ (geometric parameter)
	\item $m_e = 0.511$ MeV (electron mass)
	\item $C_{\text{conv}} = 7.783 \times 10^{-3}$ (conversion factor from $\hbar$, $c$)
	\item $\mathcal{K} = 0.986$ (fractal correction)
\end{itemize}

\subsection{Numerical Result}

\begin{equation}
	G_{\text{SI}} = 6.674 \times 10^{-11}\,\text{m}^3/(\text{kg}\cdot\text{s}^2)
	\label{eq:G_result}
\end{equation}

with $< 0.0002\%$ deviation from the CODATA-2018 value!

\section{The Planck Length}

\subsection{Standard Definition}

The Planck length is defined as:

\begin{equation}
	l_P = \sqrt{\frac{\hbar G}{c^3}}
	\label{eq:planck_length_standard}
\end{equation}

In natural units ($\hbar = c = 1$), this simplifies to:

\begin{equation}
	l_P = \sqrt{G}
	\label{eq:planck_length_natural}
\end{equation}

\subsection{T0 Derivation from $\xi$}

Since $G$ is derived from $\xi$, the Planck length follows directly:

\begin{equation}
	l_P = \sqrt{G} = \sqrt{\frac{\xi^2}{4 m_e}} = \frac{\xi}{2\sqrt{m_e}}
	\label{eq:planck_from_xi}
\end{equation}

In natural units with $m_e = 0.511$ MeV:

\begin{equation}
	l_P = \frac{1.333 \times 10^{-4}}{2\sqrt{0.511}} \approx 9.33 \times 10^{-5}
	\label{eq:planck_nat}
\end{equation}

Conversion to SI units:

\begin{equation}
	\boxed{l_P = 1.616 \times 10^{-35}\,\text{m}}
	\label{eq:planck_si}
\end{equation}

\section{Characteristic T0 Length Scales}

\subsection{The Sub-Planck Scale}

The minimal Sub-Planck length scale is:

\begin{equation}
	L_0 = \xi \cdot l_P = \frac{4}{3} \times 10^{-4} \times 1.616 \times 10^{-35}\,\text{m} = 2.155 \times 10^{-39}\,\text{m}
	\label{eq:sub_planck}
\end{equation}

This scale is about $10^4$ times smaller than the Planck length and marks the 
absolute lower bound of spacetime granulation.

\subsection{Energy-Dependent Length Scales}

The characteristic T0 length for an energy $E$ is:

\begin{equation}
	r_0(E) = 2GE
	\label{eq:r0_energy}
\end{equation}

In natural units ($G = 1$):

\begin{equation}
	r_0(E) = \frac{1}{E}
	\label{eq:r0_natural}
\end{equation}

For the fundamental energy scale $\E_0 = \sqrt{m_e \cdot m_\mu}$:

\begin{equation}
	r_0(\E_0) = 2G\E_0 \approx 2.7 \times 10^{-14}\,\text{m}
	\label{eq:r0_E0}
\end{equation}

\section{The Boltzmann Constant}

\subsection{Connection to Temperature}

The Boltzmann constant connects temperature with energy:

\begin{equation}
	E = k_B T
	\label{eq:boltzmann_relation}
\end{equation}

In the T0 theory, this is a manifestation of time-mass duality on 
thermodynamic scales.

\subsection{Derivation from $\xi$}

In natural units, $k_B$ is dimensionless. The SI conversion follows from 
the energy unit:

\begin{equation}
	k_B^{\text{SI}} = \frac{\text{1 eV}}{\text{11604.5 K}} = 1.381 \times 10^{-23}\,\text{J/K}
	\label{eq:boltzmann_si}
\end{equation}

The T0 theory reproduces this through the connection between energy and 
temperature scales via $\xi$-derived masses.

\section{The 2019 SI Reform}

\subsection{Fundamental Redefinition}

The 2019 SI reform defined the kilogram via the Planck constant:

\begin{equation}
	\hbar = 6.62607015 \times 10^{-34}\,\text{J}\cdot\text{s} \quad \text{(exact)}
	\label{eq:planck_const_exact}
\end{equation}

and the Boltzmann constant:

\begin{equation}
	k_B = 1.380649 \times 10^{-23}\,\text{J/K} \quad \text{(exact)}
	\label{eq:boltzmann_exact}
\end{equation}

\subsection{T0 Consequence}

This reform unwittingly implemented the unique calibration consistent with 
the T0 geometric foundation. The SI units are now implicitly fixed by $\xi$:

\begin{equation}
	\text{SI system} \leftrightarrow \xi = \frac{4}{3} \times 10^{-4}
	\label{eq:si_xi_connection}
\end{equation}

\section{Scale Hierarchy}

The various length scales in the T0 theory:

\begin{align}
	L_0 &= 2.155 \times 10^{-39}\,\text{m} \quad \text{(minimal T0 scale)} \\
	l_P &= 1.616 \times 10^{-35}\,\text{m} \quad \text{(Planck length)} \\
	r_0(\E_0) &= 2.7 \times 10^{-14}\,\text{m} \quad \text{(characteristic scale)} \\
	r_e &= 2.818 \times 10^{-15}\,\text{m} \quad \text{(classical electron radius)}
\end{align}

This hierarchy emerges completely from $\xi$ and the fractal structure of 
spacetime.

\section{Summary}

In this chapter, we have shown how all fundamental units and constants 
follow from $\xi$:

\begin{enumerate}
	\item Natural units: $\hbar = c = 1$ simplify the derivations
	\item Gravitational constant: $G = \frac{\xi^2}{4m_e} \times C_{\text{conv}} \times \mathcal{K}$
	\item Planck length: $l_P = \frac{\xi}{2\sqrt{m_e}}$
	\item Sub-Planck scale: $L_0 = \xi \cdot l_P$
	\item 2019 SI reform: Consistent with T0 geometry
\end{enumerate}

The complete derivation chain $\xi \to m_e \to G \to l_P$ demonstrates the 
parameter-free nature of the theory. All physical quantities emerge from the 
geometry of three-dimensional space.


