% Kapitel 09: Kosmologie, Rotverschiebung und CMB
% Komplett neu geschrieben mit korrekten Formeln

\chapter{Kosmologie, Rotverschiebung und CMB in der Zeit-Masse-Dualität}

\section{Einführung}

In den vorangegangenen Kapiteln stand die mikroskopische Seite der 
Zeit-Masse-Dualität im Mittelpunkt: Massen, Kopplungen und Quantenphänomene. 
In diesem Kapitel wird skizziert, wie sich dieselbe Struktur auf großskalige 
Phänomene der Kosmologie auswirkt: Rotverschiebung, kosmische 
Hintergrundstrahlung und effektive Größen wie die Hubble-Skala.

\section{Rotverschiebung ohne expandierenden Raum}

\subsection{Standard-Interpretation}

Die Standardkosmologie deutet die kosmologische Rotverschiebung hauptsächlich 
als Folge einer expandierenden Raumzeit. Die Wellenlänge eines Photons wird 
mit dem kosmischen Skalenfaktor $a(t)$ mitgedehnt:

\begin{equation}
\frac{\lambda_{\text{obs}}}{\lambda_{\text{emit}}} = \frac{a(t_{\text{obs}})}{a(t_{\text{emit}})} = 1 + z
\end{equation}

\subsection{Zeit-Masse-Dualität Interpretation}

Im Rahmen der Zeit-Masse-Dualität wird ein alternatives Bild vorgeschlagen. 
Die beobachtete Rotverschiebung wird als Folge der fraktalen Tiefenstruktur 
verstanden.

Die T0-Rotverschiebung:

\begin{equation}
z_{\text{T0}} = \int_0^d \xipar(r) \frac{E_\gamma(r)}{E_{\gamma,0}} dr
\end{equation}

Für homogenes $\xipar$-Feld:

\begin{equation}
z_{\text{T0}} \approx \xipar \cdot d \cdot \left(1 - \frac{E_\gamma}{2E_{\gamma,0}}\right)
\end{equation}

Hubble-Relation:

\begin{equation}
H_0^{\text{T0}} = \xipar \cdot c \approx 40\,\text{km/s/Mpc}
\end{equation}

\section{CMB-Temperatur}

Die CMB-Temperatur:

\begin{equation}
T_{\text{CMB}} = 2.7255\,\text{K}
\end{equation}

wird in T0 als Gleichgewichtszustand der $\xipar$-Geometrie interpretiert, 
nicht als Relikt eines Urknalls.

\section{Statisches Universum}

Die T0-Theorie favorisiert ein statisches Universum. JWST-Beobachtungen 
entwickelter Galaxien bei $z > 10$ sind konsistent mit unbegrenzter 
Entwicklungszeit.

\section{Zusammenfassung}

Kosmologische Phänomene als Manifestationen der $\xipar$-Geometrie, nicht als 
Relikte einer Urknall-Vergangenheit.



