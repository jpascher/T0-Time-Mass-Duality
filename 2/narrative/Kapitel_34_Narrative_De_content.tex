\maketitle
	
	\section*{Kapitel 34: Lösung des Strong-CP-Problems in der fraktalen T0-Geometrie}
	
	\subsection*{Kurze Einführung}
	
	Dieses Kapitel löst das Strong-CP-Problem durch intrinsische Regularisierung des Vakuumphasenfeldes – ohne Axion oder Feinabstimmung.
	
	\subsection*{Mathematische Grundlage}
	
	Das Strong-CP-Problem fragt, warum der CP-verletzende Parameter \(\theta_{\text{QCD}}\) in QCD kleiner als \(10^{-10}\) ist, obwohl er natürlich O(1) sein sollte. In der FFGFT wird \(\theta_{\text{QCD}}\) durch fraktale Nichtlokalität auf Null relaxiert, reguliert durch \(\xi = \frac{4}{3} \times 10^{-4}\).
	
	\subsection*{Symbolverzeichnis und Einheiten}
	
	
	
	\subsection*{Das Strong-CP-Problem}
	
	Der topologische Term in QCD:
	
	\begin{equation}
		\mathcal{L}_{\theta} = \theta_{\text{QCD}} \frac{g^2}{32\pi^2} G^a_{\mu\nu} \tilde{G}^{a\mu\nu}.
	\end{equation}
	
	Dieser Term verletzt CP, wenn \(\theta_{\text{QCD}} \neq 0\). Natürlich erwartet man \(\theta_{\text{QCD}} \sim O(1)\), doch das Neutronen-EDM begrenzt:
	
	\begin{equation}
		|\theta_{\text{QCD}}| < 10^{-10}.
	\end{equation}
	
	Ohne Mechanismus ist dies extreme Feinabstimmung.
	
	\textbf{Einheitenprüfung:}
	\begin{align*}
		[\mathcal{L}_{\theta}] &= \text{dimensionslos} \cdot \si{\per\meter\tothe{4}} = \si{\per\meter\tothe{4}}.
	\end{align*}
	
	\subsection*{Fraktale Regularisierung der Phase}
	
	Das Vakuumphasenfeld \(\theta(x,t)\) ist fraktal korreliert:
	
	\begin{equation}
		\langle \theta(x) \theta(y) \rangle = \xi \ln(|x-y|/l_0) + \frac{\xi^2}{2} [\ln(|x-y|/l_0)]^2.
	\end{equation}
	
	Der logarithmische Term summiert über Hierarchiestufen und relaxiert globale \(\theta\) auf Null – lokale Fluktuationen bleiben klein.
	
	\textbf{Einheitenprüfung:}
	\begin{align*}
		[\langle \theta \theta \rangle] &= \text{dimensionslos}.
	\end{align*}
	
	\subsection*{Relaxation des \(\theta\)-Terms}
	
	Der effektive \(\theta_{\text{QCD}}\):
	
	\begin{equation}
		\theta_{\text{QCD}}^{\text{eff}} \approx \xi^2 \cdot \langle \delta \theta \rangle \approx 10^{-8}.
	\end{equation}
	
	Der doppelte \(\xi^2\)-Faktor unterdrückt den Parameter natürlich unter die EDM-Grenze.
	
	\subsection*{Neutronen-EDM}
	
	Das induzierte Dipolmoment:
	
	\begin{equation}
		d_n \approx \theta_{\text{QCD}} \cdot 10^{-16} \, e \cdot \text{cm}.
	\end{equation}
	
	Mit \(\theta_{\text{QCD}}^{\text{eff}} < 10^{-8}\) liegt \(d_n < 10^{-24} \, e \cdot \text{cm}\) – weit unter aktuellen Grenzen, aber testbar in Zukunft.
	
	\subsection*{Vergleich mit Axion-Lösung}
	
	\begin{center}
		\begin{tabular}{p{0.45\textwidth}p{0.45\textwidth}}
			\textbf{Axion} & \textbf{FFGFT (T0)} \\
			\hline
			Neues Teilchen & Kein neues Feld \\
			Feinabstimmung vermieden & Geometrisch relaxiert \\
			Kalte Dunkle Materie & Vakuum-Effekt \\
			Testbar durch Suche & EDM-Vorhersage \\
		\end{tabular}
	\end{center}
	
	\subsection*{Schlussfolgerung}
	
	Die FFGFT löst das Strong-CP-Problem durch fraktale Relaxation der Vakuumphase – \(\theta_{\text{QCD}}\) wird geometrisch auf nahe Null gesetzt, ohne Axion oder Feinabstimmung. Die Vorhersage \(|\theta_{\text{QCD}}| \approx \xi^2\) ist testbar durch präzisere Neutronen-EDM-Messungen und unterstreicht die universelle Rolle von \(\xi\).