\chapter{Kapitel 35: Erklärung quantenmechanischer Phänomene in der fraktalen T0-Geometrie}
\label{chap:35}

\section*{Kapitel 35: Erklärung quantenmechanischer Phänomene in der fraktalen T0-Geometrie}
	
	\subsection*{Kurze Einführung}
	
	Dieses Kapitel erklärt zentrale Quantenphänomene wie Interferenz, Verschränkung und Tunneleffekt aus der Dynamik des fraktalen Vakuumfeldes – ohne ontologische Superposition.
	
	\subsection*{Mathematische Grundlage}
	
	Die Quantenmechanik basiert auf Wellenfunktionen und Superposition. In der FFGFT emergieren diese als mathematische Hilfskonstrukte aus der Phase und Amplitude des Vakuumfeldes \(\Phi = \rho e^{i\theta}\), reguliert durch \(\xi = \frac{4}{3} \times 10^{-4}\). Es gibt keine ontologische Überlagerung realer Zustände – das Vakuumfeld ist immer deterministisch.
	
	\subsection*{Symbolverzeichnis und Einheiten}
	
	\begin{tcolorbox}[title={\textbf{Wichtige Symbole und ihre Einheiten}}, colback=blue!5!white, colframe=blue!75!black]
		\begin{tabular}{p{0.3\textwidth}p{0.3\textwidth}p{0.35\textwidth}}
			\textbf{Symbol} & \textbf{Bedeutung} & \textbf{Einheit (SI)} \\
			\hline
			\(\xi\) & Fraktaler Skalenparameter & dimensionslos \\
			\(\theta(x,t)\) & Vakuumphasenfeld (deterministischer Träger der Kohärenz) & dimensionslos (radiant) \\
			\(\rho(x,t)\) & Vakuum-Amplitudendichte & \si{\kilo\gram^{1/2}\per\meter^{3/2}} \\
			\(\Phi\) & Komplexes Vakuumfeld & \si{\kilo\gram^{1/2}\per\meter^{3/2}} \\
			\(\Delta \theta\) & Phasenunterschied zwischen Pfaden & dimensionslos (radiant) \\
			\(P\) & Übergangswahrscheinlichkeit & dimensionslos \\
			\(V(x)\) & Potenzialbarriere & \si{\joule} \\
			\(E\) & Energie des Teilchens & \si{\joule} \\
			\(d\) & Barrierendicke & \si{\meter} \\
			\(\kappa\) & Tunnelexponent & \si{\per\meter} \\
			\(C(\Delta x)\) & Fraktale Korrelationsfunktion & dimensionslos \\
			\(\psi(x)\) & Wellenfunktion (mathematisches Hilfskonstrukt) & dimensionslos \\
		\end{tabular}
	\end{tcolorbox}
	
	\subsection*{Doppelspalt-Interferenz}
	
	Das Photon nimmt beide Pfade:
	
	\begin{equation}
		\Delta \theta = \theta_1 - \theta_2 = \frac{2\pi \Delta L}{\lambda}.
	\end{equation}
	
	Der Phasenunterschied \(\Delta \theta\) zwischen den Pfaden 1 und 2 entsteht aus der Weglängendifferenz \(\Delta L\). Die fraktale Phase bleibt kohärent über beide Pfade – kein ontologisches „beide Pfade gleichzeitig“.
	
	Die Intensität am Schirm:
	
	\begin{equation}
		I \propto 1 + \cos(\Delta \theta).
	\end{equation}
	
	Der Kosinus-Term erzeugt das Interferenzmuster – klassische Welle aus globaler Vakuumphase.
	
	\textbf{Einheitenprüfung:}
	\begin{align*}
		[\Delta \theta] &= \text{dimensionslos}.
	\end{align*}
	
	\subsection*{Verschränkung}
	
	Verschränkte Teilchen teilen Phase:
	
	\begin{equation}
		\theta_{12} = \theta_1 + \theta_2 = \text{konstant}.
	\end{equation}
	
	Die Summe der Phasen ist fest – Messung an einem fixiert die Phase lokal, aber das Feld war bereits global kohärent. Es gibt keine instantane Signalübertragung, sondern vorbestehende fraktale Nichtlokalität.
	
	\subsection*{Tunneleffekt}
	
	Unter der Barriere:
	
	\begin{equation}
		P \approx \exp\left( -2 \kappa d \right), \quad \kappa = \sqrt{2m(V-E)} / \hbar \cdot (1 + \xi \ln(d/l_0)).
	\end{equation}
	
	Der exponentielle Abfall entsteht aus Phasenakkumulation unter der Barriere, mit fraktaler Korrektur \(\xi \ln(d/l_0)\) für Nichtlokalität.
	
	\textbf{Einheitenprüfung:}
	\begin{align*}
		[\kappa] &= \si{\per\meter}.
	\end{align*}
	
	\subsection*{Fraktale Kohärenz}
	
	Korrelationsfunktion:
	
	\begin{equation}
		C(\Delta x) = \xi \ln(\Delta x / l_0).
	\end{equation}
	
	Logarithmische Kohärenz ermöglicht Interferenz über große Distanzen – ohne ontologische Superposition.
	
	\subsection*{Vergleich Standard-QM – FFGFT}
	
	\begin{center}
		\begin{tabular}{p{0.45\textwidth}p{0.45\textwidth}}
			\textbf{Standard-QM} & \textbf{FFGFT (T0)} \\
			\hline
			Postulate & Emergent aus Phase \\
			Wellen-Teilchen-Dualität & Amplitude-Phase-Trennung \\
			Kollaps & Deterministische Dynamik \\
			Keine Gravitation & Einheitlich \\
			Ontologische Superposition & Mathematisches Hilfskonstrukt \\
		\end{tabular}
	\end{center}
	
	\subsection*{Schlussfolgerung}
	
	Die FFGFT erklärt Quantenphänomene als Dynamik der Vakuumphase \(\theta\): Interferenz aus Pfadphasen, Verschränkung aus globaler Kohärenz, Tunneln aus Nichtlokalität. Die Wellenfunktion \(\psi\) ist ein rein mathematisches Hilfskonstrukt zur Beschreibung von Wahrscheinlichkeiten – keine ontologische Realität. Es gibt keine instantane Wirkung oder Retrokausalität. Alles parameterfrei aus \(\xi\), vereinheitlicht QM mit Gravitation.