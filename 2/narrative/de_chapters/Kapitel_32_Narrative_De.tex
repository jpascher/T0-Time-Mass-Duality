\chapter{Reaktor-Antineutrino-Anomalie – Aktualisierte Betrachtung (Stand Januar 2026)}


\section*{Reaktor-Antineutrino-Anomalie – Aktualisierte Betrachtung (Stand Januar 2026)}
	
	\subsection*{Kurze Einführung}
	
	Dieses Kapitel betrachtet die Reaktor-Antineutrino-Anomalie (RAA) im Licht aktueller Daten und zeigt, wie die FFGFT eine kohärente Alternative zur mainstream-Auflösung bietet.
	
	\subsection*{Mathematische Grundlage}
	
	Die RAA beschrieb ein historisches Defizit von etwa 6 % in der Rate gemessener Elektron-Antineutrinos bei kurzen Basislinien. Neuere Flussmodelle haben das Defizit weitgehend erklärt, doch die FFGFT liefert eine geometrische Interpretation des numerischen Werts, reguliert durch \(\xi = \frac{4}{3} \times 10^{-4}\).
	
	\subsection*{Historische Anomalie}
	
	Die Rate war um etwa 6 % niedriger als vorhergesagt:
	
	\begin{equation}
		\frac{R_{\text{obs}}}{R_{\text{pred}}} \approx 0.94.
	\end{equation}
	
	Dieser Wert basierte auf älteren Flussmodellen und kurzen Basislinien (ca. 10–100 m).
	
	\subsection*{Aktueller Stand (Januar 2026)}
	
	Verbesserte Summationsmethoden und neue Messungen (z. B. Daya Bay, RENO, PROSPECT) haben das globale Defizit eliminiert. Ein kleiner ``Bump" bei 4–6 MeV bleibt jedoch in einigen Datensätzen diskutiert.
	
	\subsection*{FFGFT-Interpretation}
	
	Die lokale Vakuum-Amplitude wird durch den Reaktorfluss modifiziert:
	
	\begin{equation}
		\frac{\delta \rho}{\rho_0} \approx \xi^2 \cdot \frac{\Phi_{\text{reactor}}}{\rho_0}.
	\end{equation}
	
	Der Fluss \(\Phi_{\text{reactor}}\) erzeugt eine kleine Dichteänderung, skaliert durch \(\xi^2\).
	
	Die Oszillationswahrscheinlichkeit wird modifiziert:
	
	\begin{equation}
		P(\bar{\nu}_e \to \bar{\nu}_e) \approx 1 - \sin^2(2\theta) \sin^2\left(1.27 \frac{\Delta m^2 L}{E_\nu}\right) - \xi \cdot \frac{\delta \rho}{\rho_0}.
	\end{equation}
	
	Der zusätzliche Term \(\xi \cdot \frac{\delta \rho}{\rho_0}\) simuliert ein effektives Defizit von etwa 6 % in der historischen Epoche.
	
	\textbf{Einheitenprüfung:}
	\[
	[P] = \text{dimensionslos}.
	\]
	
	\subsection*{Energieabhängigkeit}
	
	Der Effekt maximiert bei Resonanz:
	
	\begin{equation}
		E_{\text{res}} \approx \frac{\hbar c}{l_0 \cdot \xi^{-1}} \approx \SIrange{4}{6}{\mev}.
	\end{equation}
	
	Die fraktal erweiterte Korrelationslänge \(l_0 \xi^{-1}\) setzt die Resonanzenergie – passend zum verbleibenden ``Bump".
	
	\textbf{Einheitenprüfung:}
	\[
	[E_{\text{res}}] = \si{\joule\second} \cdot \si{\meter\per\second} / \si{\meter} = \si{\joule}.
	\]
	
	\subsection*{Vergleich mit Sterile-Neutrino-Hypothese}
	
	\begin{center}
		\begin{tabular}{p{0.45\textwidth}p{0.45\textwidth}}
			\textbf{Sterile Neutrinos} & \textbf{FFGFT (T0)} \\
			\hline
			\(\Delta m^2 \approx \SI{1}{\electronvolt\squared}\) & Keine neuen Teilchen \\
			Eingeschränkt durch PROSPECT/STEREO & Konsistent mit allen Daten \\
			Oszillation in Vakuum & Vakuum-Amplitude-Modifikation \\
			Ad-hoc Skala & Natürlich aus \(\xi\) \\
		\end{tabular}
	\end{center}
	
	\subsection*{Schlussfolgerung}
	
	Selbst nach der mainstream-Auflösung der RAA durch verbesserte Flussmodelle bleibt die FFGFT eine elegante Alternative: Das numerische 6 %-Defizit und der Bump bei 4–6 MeV sind direkte Konsequenzen der fraktalen Vakuum-Modifikation durch \(\delta \rho\). Dies unterstreicht die universelle Rolle von \(\xi\) in der Vereinheitlichung von Teilchenphysik und Kosmologie.




