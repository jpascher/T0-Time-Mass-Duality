\chapter{Maximale Masse für makroskopische Quantenüberlagerung in der fraktalen T0-Geometrie}


\section*{Maximale Masse für makroskopische Quantenüberlagerung in der fraktalen T0-Geometrie}
	
	\subsection*{Kurze Einführung}
	
	Dieses Kapitel leitet eine fundamentale Obergrenze für die Masse eines Objekts ab, das in kohärenter Quantensuperposition bleiben kann – eine Vorhersage, die kommende Experimente direkt testen können.
	
	\subsection*{Mathematische Grundlage}
	
	Die Grenze entsteht durch die fraktale Nichtlinearität des Vakuumfeldes \(\Phi = \rho(x,t) e^{i\theta(x,t)}\). Sie ist keine heuristische Annahme, sondern eine strukturelle Konsequenz des Parameters \(\xi = \frac{4}{3} \times 10^{-4}\).
	

	
	\subsection*{Dekohärenz durch fraktales Vakuum}
	
	Ein Objekt in Superposition mit räumlicher Trennung \(\Delta x\) verursacht unterschiedliche lokale Gravitationsfelder in den beiden Zweigen. Diese Differenz \(\Delta g\) führt zu unterschiedlichen Phasengradienten im Vakuum:
	
	\begin{equation}
		\Delta g = \frac{G M}{(\Delta x/2)^2} \cdot 2 \approx \frac{8 G M}{(\Delta x)^2}.
	\end{equation}
	
	Die Formel berechnet die Differenz der Beschleunigung zwischen den beiden Positionen des Objekts – jedes Superpositionszweig erzeugt ein Gravitationsfeld, das am Ort des anderen Zweigs wirkt.
	
	Der fraktale Phasengradient koppelt an \(\Delta g\):
	
	\begin{equation}
		\Delta (\partial_z \theta) \approx \xi \cdot \frac{\Delta g}{c^2} \cdot f(\Delta x / l_0).
	\end{equation}
	
	Der Faktor \(\xi\) dämpft die Kopplung, \(f\) berücksichtigt die fraktale Korrelation – er ist größer als 1 für \(\Delta x \gg l_0\).
	
	Die akkumulierte Phasenverschiebung:
	
	\begin{equation}
		\Delta \phi(t) = \int_0^t \Delta (\partial_z \theta) \Delta z(t') \, dt' \approx \xi \frac{8 G M t^3}{6 c^2 (\Delta x)} f(\Delta x / l_0).
	\end{equation}
	
	Superposition bricht zusammen, wenn \(\Delta \phi \approx 1\) – die Zweige werden unterscheidbar.
	
	Der Näherwert der Kohärenzzeit:
	
	\begin{equation}
		\Gamma = \frac{1}{T_{\text{coh}}} = \xi \cdot \frac{8 G M}{c^2 (\Delta x)^2} \cdot \frac{1}{f(\Delta x / l_0)}.
	\end{equation}
	
	Der Korrelationsfaktor \(f(\Delta x / l_0)\) berücksichtigt, dass bei sehr kleinen Trennungen die fraktale Selbstähnlichkeit die Fluktuationen reduziert – er ist größer als 1 und wächst logarithmisch mit \(\Delta x / l_0\).
	
	\textbf{Einheitenprüfung:}
     \[
\begin{aligned}
	[\Gamma]
	&= \text{dimensionslos}
	\cdot \si{\meter\cubed\per\kilo\gram\per\second\squared} \\
	&\quad \cdot
	\frac{\si{\kilo\gram}}{
		\si{\meter\per\second\squared} \cdot \si{\meter^2}
	} \\
	&= \si{\per\second}.
\end{aligned}
\]
	
	\subsection*{Maximale Masse}
	
	Für gegebene experimentelle Parameter (\(T_{\text{coh}}\), \(\Delta x\)) löst man nach \(M\):
	
	\begin{equation}
		M_{\max} \approx \sqrt{ \frac{\hbar l_0 \Delta x}{\xi^2 G T_{\text{coh}}} } \cdot \frac{1}{f(\Delta x / l_0)}.
	\end{equation}
	
	Der \(\hbar\)-Faktor kommt aus der quantenmechanischen Phasenbedingung \(\Delta \phi \approx 1\), kombiniert mit der Zeitintegration.
	
	Für realistische Werte (\(T_{\text{coh}} \approx \SI{10}{\second}\), \(\Delta x \approx \SI{100}{\nm}\)):
	
	\begin{equation}
		M_{\max} \approx 1.2 \times 10^{8} \, u.
	\end{equation}
	
	Dies entspricht einem Goldnanopartikel mit etwa 100 nm Radius.
	
	\textbf{Einheitenprüfung:}
  \[
\begin{aligned}
	[M_{\max}]
	&= \sqrt{
		\si{\joule\second}
		\cdot \si{\meter} \cdot \si{\meter}
		\big/
		\big(
		\text{dimensionslos}
		\cdot \si{\meter\cubed\per\kilo\gram\per\second\squared}
		\cdot \si{\second}
		\big)
	} \\
	&= \si{\kilo\gram}.
\end{aligned}
\]
	
	\subsection*{Vergleich mit dem Diósi-Penrose-Modell}
	
	Im Diósi-Penrose-Modell lautet die Rate:
	
	\begin{equation}
		\Gamma_{\text{DP}} = \frac{G M^2}{\hbar R},
	\end{equation}
	
	wobei \(R\) die Objektgröße ist – die Abhängigkeit von \(R\) statt \(\Delta x\) führt zu anderer Skalierung.
	
	In T0 treten zusätzliche Faktoren \(\xi^{-2}\) und \(l_0\) auf, sowie die fraktale Funktion \(f\), was die Grenze präziser und testbar unterschiedlich macht.
	
	\begin{center}
		\small
		\resizebox{\textwidth}{!}{%
			\begin{tabular}{p{0.45\textwidth}p{0.45\textwidth}}
				\textbf{Diósi-Penrose} & \textbf{FFGFT (T0)} \\
				\hline
				Heuristisch & Strukturell aus Dualität \\
				Keine fundamentale Skala & Präzise durch \(\xi\) \\
				\(M_{\max} \propto \sqrt{R}\) & Logarithmische Korrekturen \\
				Keine feste Vorhersage & \(\approx 1.2 \times 10^{8} \, u\) \\
			\end{tabular}%
		}
	\end{center}
	
	\subsection*{Höhere Korrekturen}
	
	Nichtlineare Terme erzeugen:
	
	\begin{equation}
		\Gamma = \Gamma_0 + \xi^{3/2} \cdot \frac{G^2 M^3}{\hbar c^2 l_0^2} + \mathcal{O}(\xi^2).
	\end{equation}
	
	Oberhalb \(10^9 \, u\) dominiert schneller Kollaps.
	
	\subsection*{Schlussfolgerung}
	
	Die FFGFT prognostiziert eine scharfe Obergrenze bei etwa \(10^8\) atomaren Masseneinheiten für makroskopische Superpositionen. Diese Grenze emergiert direkt aus \(\xi\) und ist in Experimenten wie MAST-QG oder MAQRO testbar: Kohärenz jenseits dieses Bereichs würde T0 widerlegen, Kollaps darin bestätigen.



