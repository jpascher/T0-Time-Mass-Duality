\chapter{Kapitel 25: Das Neutrinomassen-Problem in der fraktalen T0-Geometrie}


\section*{Kapitel 25: Das Neutrinomassen-Problem in der fraktalen T0-Geometrie}
	
	\subsection*{Kurze Einführung}
	
	Dieses Kapitel löst die offenen Fragen zu Neutrinomassen – ihre Kleinheit, die drei Generationen, Hierarchie, Mischung und Majorana-Natur – durch reine Phasen-Anregungen des Vakuumfeldes.
	
	\subsection*{Mathematische Grundlage}
	
	Neutrinos sind in der FFGFT keine Dirac- oder Majorana-Felder mit Amplitude, sondern reine Phasen-Excitationen des Vakuumfeldes \(\Phi = \rho(x,t) e^{i\theta(x,t)}\). Alle Eigenschaften emergieren aus dem fundamentalen Parameter \(\xi = \frac{4}{3} \times 10^{-4}\).
	
	
	\subsection*{Neutrinos als reine Phasen-Excitationen}
	
	Neutrinos haben fast keine Amplitude-Komponente – ihre Masse entsteht allein aus Phasenwindungen. Die minimale stabile Phasenverschiebung ist durch fraktale Fluktuationen begrenzt:
	
	\begin{equation}
		\Delta \theta_{\min} \approx \xi^{3/2} \cdot \sqrt{\ln(\xi^{-1})}.
	\end{equation}
	
	Der Term \(\xi^{3/2}\) kommt von der dreifachen Hierarchie der fraktalen Skalierung, der Logarithmus aus der Resummation über unendlich viele Stufen. Diese kleine Verschiebung macht Neutrinos fast masselos im Vergleich zu geladenen Leptonen.
	
	\subsection*{Massenhierarchie der drei Generationen}
	
	Die Massen ergeben sich aus trigonometrischen Projektionen der 120°-versetzten Phasen:
	
	\begin{equation}
		m_1 \approx 2 m_0^\nu \cdot \sin^2(\theta_0 / 2),
	\end{equation}
	\begin{equation}
		m_2 \approx 2 m_0^\nu \cdot \sin^2((\theta_0 + 120^\circ)/2),
	\end{equation}
	\begin{equation}
		m_3 \approx 2 m_0^\nu \cdot \sin^2((\theta_0 + 240^\circ)/2).
	\end{equation}
	
	Der Faktor 2 \(m_0^\nu\) setzt die Gesamtskala, der Sinus-Quadrat beschreibt die effektive Masse aus der Phasenabweichung vom Gleichgewicht. Die 120°-Versatz ist die natürliche Symmetrie der drei fraktalen Generationen.
	
	Mit einer kleinen fraktalen Korrektur \(\theta_0 \approx \pi + \xi \cdot \Delta\) entsteht die beobachtete Hierarchie:
	
	\begin{equation}
		m_1 : m_2 : m_3 \approx 1 : 3 : 8
	\end{equation}
	
	in erster Ordnung – passend zur normalen Hierarchie.
	
	Die absolute Skala:
	
	\begin{equation}
		m_0^\nu \approx \frac{\hbar}{c l_0} \cdot \xi^3 \approx \SI{0.05}{\eV}.
	\end{equation}
	
	Der Faktor \(\xi^3\) entsteht aus der dreifachen fraktalen Unterdrückung der Phase-Amplitude-Kopplung.
	
	Die Summe der Massen:
	
	\begin{equation}
		\sum m_\nu \approx \SI{0.12}{\eV}
	\end{equation}
	
	liegt im kosmologisch erlaubten Bereich.
	
	\textbf{Einheitenprüfung:}
	\[
	[m_0^\nu] = \frac{\si{\joule\second}}{\si{\meter\per\second} \cdot \si{\meter}} = \si{\kilo\gram}
	\quad \text{(umgerechnet in eV/$c^2$).}
	\]
	
	\subsection*{PMNS-Mischung aus Phasen-Überlapp}
	
	Die Mischungsmatrix entsteht aus dem Überlapp benachbarter Phasenmoden:
	
	\begin{equation}
		U_{ij} \approx \cos(\Delta \theta_{ij}) + i \xi \cdot \sin(\Delta \theta_{ij}).
	\end{equation}
	
	Der Kosinus-Term gibt die Hauptmischung (tribimaximal), der imaginäre \(\xi\)-Term kleine Perturbationen – exakt die beobachtete PMNS-Struktur mit großen Mischungswinkeln.
	
	\subsection*{Majorana-Natur}
	
	Da Neutrinos reine Phasen sind, ist Ladungskonjugation äquivalent zu Phasenwechsel \(\theta \to -\theta\):
	
	\begin{equation}
		\nu = \nu^c.
	\end{equation}
	
	Sie sind zwangsläufig Majorana-Teilchen.
	
	\subsection*{Vergleich Standardmodell – FFGFT}
	
	\begin{center}
		\begin{tabular}{p{0.45\textwidth}p{0.45\textwidth}}
			\textbf{Standardmodell} & \textbf{FFGFT (T0)} \\
			\hline
			Massen ad-hoc & Emergent aus Phase \\
			Seesaw postuliert & Keine Amplitude \\
			Drei Generationen willkürlich & 120°-Symmetrie \\
			PMNS frei & Aus Phasenüberlapp \\
			Majorana unklar & Zwangsläufig Majorana \\
		\end{tabular}
	\end{center}
	
	\subsection*{Schlussfolgerung}
	
	Die FFGFT löst das Neutrino-Problem vollständig: Kleine Massen durch reine Phase, drei Generationen aus fraktaler 120°-Symmetrie, Hierarchie und Mischung aus \(\xi\)-Perturbationen, Majorana-Natur aus Selbstkonjugation. Alle Werte emergieren natürlich aus dem einzigen Parameter \(\xi\), und schließen den Leptonsektor elegant ab.
