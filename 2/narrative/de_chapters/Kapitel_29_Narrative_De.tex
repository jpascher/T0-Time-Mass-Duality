\chapter{Kapitel 29: Das Delayed-Choice-Quantum-Eraser-Experiment in der fraktalen T0-Geometrie}
\label{chap:29}

\section*{Kapitel 29: Das Delayed-Choice-Quantum-Eraser-Experiment in der fraktalen T0-Geometrie}
	
	\subsection*{Kurze Einführung}
	
	Dieses Kapitel löst das scheinbare Paradoxon des Delayed-Choice-Quantum-Eraser-Experiments durch die globale Kohärenz des fraktalen Vakuumphasenfeldes.
	
	\subsection*{Mathematische Grundlage}
	
	Das DCQE-Experiment demonstriert, dass die Entscheidung, Which-Path-Information zu löschen oder zu behalten, das Interferenzmuster eines Photons beeinflusst – auch wenn diese Entscheidung nach der Detektion am Schirm erfolgt. In der FFGFT entsteht dies aus der globalen, fraktalen Kohärenz des Vakuumphasenfeldes \(\theta(x,t)\), reguliert durch \(\xi = \frac{4}{3} \times 10^{-4}\).
	
	\subsection*{Das DCQE-Experiment – Aufbau und Beobachtung}
	
	Ein verschränktes Photonpaar (Signal und Idler) wird erzeugt. Das Signal-Photon trifft einen Doppelspalt und wird am Schirm-Detektor \(D_0\) registriert. Das Idler-Photon kann Which-Path-Information tragen (Detektoren \(D_1, D_2\)) oder löschen (Erasure-Detektoren \(D_3, D_4\)).
	
	Die Phasendifferenz zwischen Signal und Idler:
	
	\begin{equation}
		\Delta \theta = \theta_s - \theta_i.
	\end{equation}
	
	Diese Differenz \(\Delta \theta\) bestimmt das Interferenzmuster am Schirm. Wenn Which-Path-Information verfügbar ist (\(D_1\) oder \(D_2\)), ist \(\Delta \theta\) bekannt und es gibt kein Interferenzmuster. Bei Erasure (\(D_3\) oder \(D_4\)) ist \(\Delta \theta\) unbekannt und das Muster erscheint – auch wenn die Erasure-Entscheidung nach der Detektion am Schirm erfolgt.
	
	\textbf{Einheitenprüfung:}
	\[
	[\Delta \theta] = \text{dimensionslos (in \si{\radian})}.
	\]
	
	\subsection*{Fraktale globale Kohärenz}
	
	Das Vakuumphasenfeld \(\theta(x,t)\) ist fraktal korreliert:
	
	\begin{equation}
		C(\Delta x) = \xi \ln(|\Delta x|/l_0) + \frac{\xi^2}{2} [\ln(|\Delta x|/l_0)]^2.
	\end{equation}
	
	Die Korrelationsfunktion \(C(\Delta x)\) wächst logarithmisch mit dem Abstand \(\Delta x\). Der führende Term \(\xi \ln(|\Delta x|/l_0)\) entsteht aus der Summation über fraktale Stufen, der quadratische Term aus höheren Korrekturen. Dadurch bleibt die Phase über große Distanzen kohärent, aber mit kontrollierter, schwacher Nichtlokalität durch den kleinen Faktor \(\xi\).
	
	\textbf{Einheitenprüfung:}
	\[
	[C(\Delta x)] = \text{dimensionslos}.
	\]
	
	\subsection*{Erasure und Kohärenz-Wiederherstellung}
	
	Bei Erasure wird Which-Path-Information gelöscht:
	
	\begin{equation}
		V = |\langle e^{i \Delta \theta} \rangle| \approx 1 - \xi \cdot \Delta x / l_0.
	\end{equation}
	
	Die Sichtbarkeit \(V\) ist der Betrag des Erwartungswerts der Phasenfaktor-Exponentialfunktion. Der Subtraktionsterm \(\xi \cdot \Delta x / l_0\) dämpft die Kohärenz leicht bei großen Trennungen, aber \(V\) bleibt nahe 1 – die Interferenz wird vollständig wiederhergestellt.
	
	Bei Which-Path-Information:
	
	\begin{equation}
		V \approx \xi \cdot \Delta x / l_0 \ll 1.
	\end{equation}
	
	Die Sichtbarkeit verschwindet fast vollständig, da die Phase bekannt ist.
	
	\subsection*{Keine Retrokausalität}
	
	Die verzögerte Entscheidung ändert nicht die Vergangenheit:
	
	\begin{equation}
		P(\text{click}|t_d) = P(\text{click}),
	\end{equation}
	
	Die Einzelklick-Wahrscheinlichkeit am Schirm ist unabhängig von der Verzögerung \(t_d\). Nur die Postselektion der Daten (welche Subset von Klicks man betrachtet) entscheidet über das Muster – die fraktale Phase bleibt global konsistent und deterministisch.
	
	\subsection*{Vergleich mit anderen Interpretationen}
	
	\begin{center}
		\begin{tabular}{p{0.45\textwidth}p{0.45\textwidth}}
			\textbf{Andere Interpretationen} & \textbf{FFGFT (T0)} \\
			\hline
			Kopenhagen: Kollaps & Deterministisch \\
			Many-Worlds: Branching & Einheitliche Phase \\
			Retrokausalität & Keine Zeitreise \\
			Ad-hoc & Parameterfrei aus \(\xi\) \\
		\end{tabular}
	\end{center}
	
	\subsection*{Schlussfolgerung}
	
	Das DCQE ist in der FFGFT kein Paradoxon: Die scheinbare Retrokausalität entsteht aus globaler fraktaler Kohärenz der Vakuumphase. Erasure stellt Kohärenz in Subsets wieder her, ohne Vergangenes zu ändern. Alles emergiert aus \(\xi\), vereinheitlicht Verschränkung mit Time-Mass-Dualität.