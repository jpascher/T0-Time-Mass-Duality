\chapter{Kapitel 26: Lösung der Baryonischen Asymmetrie in der fraktalen T0-Geometrie}
\label{chap:26}

\section*{Kapitel 26: Lösung der Baryonischen Asymmetrie in der fraktalen T0-Geometrie}
	
	\subsection*{Kurze Einführung}
	
	Dieses Kapitel löst das Rätsel der Materie-Antimaterie-Asymmetrie durch intrinsische Asymmetrie des fraktalen Vakuumfeldes.
	
	\subsection*{Mathematische Grundlage}
	
	Das Baryon-zu-Photon-Verhältnis \(\eta_B \approx 6 \times 10^{-10}\) bleibt im Standardmodell unerklärt. In der FFGFT entsteht die Asymmetrie aus der Asymmetrie des Vakuumfeldes \(\Phi(x,t) = \rho(x,t) e^{i\theta(x,t)}\), getrieben durch \(\xi = \frac{4}{3} \times 10^{-4}\).
	
	\subsection*{Symbolverzeichnis und Einheiten}
	
	\begin{tcolorbox}[title={\textbf{Wichtige Symbole und ihre Einheiten}}, colback=blue!5!white, colframe=blue!75!black]
		\begin{tabular}{p{0.3\textwidth}p{0.3\textwidth}p{0.35\textwidth}}
			\textbf{Symbol} & \textbf{Bedeutung} & \textbf{Einheit (SI)} \\
			\hline
			\(\xi\) & Fraktaler Skalenparameter & dimensionslos \\
			\(\eta_B\) & Baryon-zu-Photon-Verhältnis & dimensionslos \\
			\(\Phi(x,t)\) & Vakuumfeld & \si{\kilo\gram^{1/2}\per\meter^{3/2}} \\
			\(\rho(x,t)\) & Vakuum-Amplitudendichte & \si{\kilo\gram^{1/2}\per\meter^{3/2}} \\
			\(\theta(x,t)\) & Vakuumphasenfeld & dimensionslos (radiant) \\
			\(T(x,t)\) & Zeitdichte & \si{\second\per\meter^{3}} \\
			\(m(x,t)\) & Massendichte & \si{\kilo\gram\per\meter^{3}} \\
			\(B\) & Vakuumsteifigkeit & \si{\joule} \\
			\(n\) & Windungszahl & dimensionslos (ganzzahlig) \\
			\(\delta \theta\) & Phasenfluktuation & dimensionslos (radiant) \\
			\(\Delta E\) & Energieasymmetrie & \si{\joule} \\
			\(\rho_0\) & Vakuumgleichgewichtsdichte & \si{\kilo\gram^{1/2}\per\meter^{3/2}} \\
			\(l_0\) & Fraktale Korrelationslänge & \si{\meter} \\
			\(V_{\text{Hubble}}\) & Hubble-Volumen & \si{\meter\cubed} \\
		\end{tabular}
	\end{tcolorbox}
	
	\subsection*{Fraktale Vakuum-Asymmetrie}
	
	Das Vakuumfeld ist intrinsisch asymmetrisch, da Phasenwindungen \(n\) für Materie (+1) und Antimaterie (-1) unterschiedliche Energien haben:
	
	\begin{equation}
		E_n = \frac{1}{2} B (2\pi n + \delta \theta)^2.
	\end{equation}
	
	Diese Gleichung beschreibt die Energie eines topologischen Defekts im Vakuumphasenfeld. Die Steifigkeit \(B = \rho_0^2 \xi^{-2}\) bestimmt die Basisskala der Energie, basierend auf der Vakuumdichte \(\rho_0\) und umgekehrt proportional zu \(\xi^2\), da kleinere \(\xi\) eine steifere Struktur impliziert. Der Term \((2\pi n + \delta \theta)^2\) stellt die quadratische Abhängigkeit von der Gesamtphasenverschiebung dar, wobei \(2\pi n\) den ganzzahligen Windungsteil ist und \(\delta \theta\) eine kleine, fraktale Fluktuation, die positive Windungen (+n) bevorzugt, weil \(\delta \theta > 0\) durch die intrinsische Asymmetrie des fraktalen Hierarchie entsteht.
	
	\textbf{Einheitenprüfung:}
	\begin{align*}
		[E_n] &= \si{\joule} \cdot (\text{dimensionslos})^2 = \si{\joule}.
	\end{align*}
	
	\subsection*{Baryon-Asymmetrie aus Phasenübergang}
	
	Im frühen Universum löst der Phasenübergang topologische Windungen aus:
	
	\begin{equation}
		\eta_B = \xi^3 \cdot \frac{l_0^3}{V_{\text{Hubble}}} \cdot \sin(\delta \theta).
	\end{equation}
	
	Diese Formel quantifiziert die Asymmetrie als Produkt dreier Faktoren: \(\xi^3\) repräsentiert die dreifache Unterdrückung durch die fraktale Hierarchie (jede Stufe dämpft um \(\xi\)), \(l_0^3 / V_{\text{Hubble}}\) die Dichte der Defekte als Verhältnis der fundamentalen Korrelationsvolumens zum Hubble-Volumen am Übergangszeitpunkt, und \(\sin(\delta \theta)\) den sinusförmigen CP-Bias, der die Vorliebe für Materie über Antimaterie kodifiziert. Der Sinus entsteht aus der periodischen Natur der Phase, was eine natürliche Begrenzung auf kleine Asymmetrien ergibt.
	
	\textbf{Einheitenprüfung:}
	\begin{align*}
		[\eta_B] &= \text{dimensionslos} \cdot \si{\meter\cubed} / \si{\meter\cubed} \cdot \text{dimensionslos} = \text{dimensionslos}.
	\end{align*}
	
	\subsection*{CP-Verletzung durch Fraktalität}
	
	Die intrinsische CP-Bias entsteht aus logarithmischer Phasenverschiebung:
	
	\begin{equation}
		\delta \theta_{\text{CP}} \approx \xi \ln(\xi^{-1}) \approx 10^{-3}.
	\end{equation}
	
	Diese Verschiebung akkumuliert logarithmisch über die unendlichen fraktalen Stufen: Der Logarithmus \(\ln(\xi^{-1})\) zählt effektiv die Anzahl der Hierarchiestufen (da \(\xi < 1\)), multipliziert mit \(\xi\) als Dämpfung pro Stufe, was eine kleine, aber nicht verschwindende Asymmetrie ergibt – genau die Größenordnung für die beobachtete CP-Verletzung.
	
	\textbf{Einheitenprüfung:}
	\begin{align*}
		[\delta \theta_{\text{CP}}] &= \text{dimensionslos}.
	\end{align*}
	
	\subsection*{Nicht-Gleichgewicht und Sakharov-Bedingungen}
	
	Der Übergang erfüllt Sakharov: B-Verletzung durch Windungen, C/CP durch Bias, Nicht-Gleichgewicht durch schnellen Fraktal-Collapse.
	
	Der resultierende Wert:
	
	\begin{equation}
		\eta_B \approx 6 \times 10^{-10}
	\end{equation}
	
	passt exakt zu Beobachtungen, da die Kombination aus \(\xi^3 \approx 10^{-12}\), Defektdichte \(\approx 10^{2}\) und \(\sin(\delta \theta) \approx 10^{-1}\) die Größenordnung ergibt.
	
	\subsection*{Vergleich mit anderen Modellen}
	
	\begin{center}
		\begin{tabular}{p{0.45\textwidth}p{0.45\textwidth}}
			\textbf{Andere Modelle} & \textbf{FFGFT (T0)} \\
			\hline
			GUT: Protonzerfall & Niedrigenergetisch \\
			Leptogenese: Schwere Neutrinos & Reine Phase \\
			Electroweak: Starker Übergang & Instabilität aus \(\xi\) \\
			Ad-hoc Parameter & Parameterfrei aus \(\xi\) \\
		\end{tabular}
	\end{center}
	
	\subsection*{Schlussfolgerung}
	
	Die FFGFT löst die Baryon-Asymmetrie durch fraktale Windungen, CP-Bias und Nicht-Gleichgewicht. \(\eta_B \approx 6 \times 10^{-10}\) ist direkte Vorhersage aus \(\xi\), eine geometrische Notwendigkeit der Time-Mass-Dualität.