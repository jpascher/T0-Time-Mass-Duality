
\maketitle

\section*{Introduction}

Einstein's most famous equation, $E = mc^2$, tells us that energy and mass are equivalent. But in FFGFT, this relationship takes on a deeper meaning through the time-mass duality: time and mass are two aspects of the same fundamental field $T(x,t)$.

\section{The Time-Mass Duality}

In FFGFT, the vacuum field $T(x,t)$ can be interpreted in two equivalent ways:
\begin{itemize}
\item As a time field: describing temporal dynamics and fluctuations
\item As a mass field $m(x,t)$: describing mass distributions
\end{itemize}

This duality is expressed mathematically as:
\begin{equation}
T(x,t) \leftrightarrow m(x,t)
\end{equation}

\section{Energy from Fractal Geometry}

Energy in FFGFT arises from the fractal structure's dynamics. The fractal corrections modify the energy-momentum relation, leading to new insights about $E = mc^2$.

The total energy includes:
\begin{equation}
E_{\text{tot}} = E_{\text{classical}} + E_{\text{fractal}}
\end{equation}

where $E_{\text{fractal}}$ accounts for contributions from different fractal levels.

\section{Implications}

\begin{itemize}
\item Mass is not an intrinsic property but emerges from the fractal geometry
\item Time dilation and mass increase are unified phenomena
\item The speed of light limit arises naturally from the fractal structure
\end{itemize}

\section{Conclusion}

In FFGFT, $E = mc^2$ gains a deeper geometric meaning through the time-mass duality. Mass and time are not separate entities but manifestations of the fundamental fractal field.

Our central metaphor: The universe as a brain with increasing convolutions but constant volume. Space doesn't expand – the fractal structure becomes more complex.

\vfill
\noindent
\textit{Source:} \url{https://github.com/jpascher/T0-Time-Mass-Duality}

