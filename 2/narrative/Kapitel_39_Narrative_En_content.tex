
\maketitle

\section*{Introduction}

This chapter explores Physics at Planck length in the context of the Fundamental Fractal-Geometric Field Theory. Building on our understanding from previous chapters (our already known concepts of tensors, metric tensor, and energy-momentum tensor), we delve deeper into this specific aspect of FFGFT.

\section{Main Concepts}

Comparing FFGFT with other approaches to quantum gravity reveals both similarities and crucial differences. While other theories introduce new structures (strings, loops, extra dimensions), FFGFT modifies the geometry of spacetime itself through its fractal nature.

\section{Connection to Fractal Geometry}

The fractal parameter $\xi = 4/3 \times 10^{-4}$ plays a crucial role in understanding these phenomena. The fractal dimension $D_f = 3 - \xi \approx 2.999867$ modifies the classical predictions and leads to new insights.

\section{Implications and Predictions}

The fractal structure of spacetime leads to testable predictions and explains observations that are puzzling in standard theories. The time-mass duality $T(x,t) \leftrightarrow m(x,t)$ provides a unified framework for understanding these phenomena.

\section{Conclusion}

In this chapter, we have seen how Physics at Planck length fits into the larger picture of FFGFT. Our central metaphor remains: the universe is like a brain with constant volume but increasing convolutions. Space doesn't expand – the fractal structure becomes more complex.

The next chapters will build on these insights to explore further aspects of the theory.

\vfill
\noindent
\textit{Source:} \url{https://github.com/jpascher/T0-Time-Mass-Duality}

