% Zentrale Zeichenerklärung für FFGFT Narrative (Deutsch)
% Diese Datei wird in das Master-Dokument eingebunden

\chapter*{Zentrale Zeichenerklärung}
\addcontentsline{toc}{chapter}{Zentrale Zeichenerklärung}

Diese zentrale Zeichenerklärung listet alle wichtigen mathematischen Symbole und physikalischen Größen auf, die in der FFGFT Narrative Edition verwendet werden.

\begin{longtable}{@{}p{0.18\textwidth}p{0.22\textwidth}p{0.50\textwidth}@{}}
  \toprule
  \textbf{Symbol} & \textbf{Einheit} & \textbf{Bedeutung} \\
  \midrule
  \endfirsthead
  
  \toprule
  \textbf{Symbol} & \textbf{Einheit} & \textbf{Bedeutung} \\
  \midrule
  \endhead
  
  \multicolumn{3}{l}{\textbf{Fundamentale Konstanten}} \\
  \midrule
  $c$ & \si{\meter\per\second} & Lichtgeschwindigkeit im Vakuum ($c \approx 2{,}998 \times 10^8$ m/s) \\
  $G$ & \si{\meter\cubed\per\kilo\gram\per\second\squared} & Gravitationskonstante ($G \approx 6{,}674 \times 10^{-11}$ m$^3$ kg$^{-1}$ s$^{-2}$) \\
  $h$ & \si{\joule\second} & Plancksches Wirkungsquantum ($h \approx 6{,}626 \times 10^{-34}$ J·s) \\
  $\hbar$ & \si{\joule\second} & Reduziertes Plancksches Wirkungsquantum ($\hbar = h/(2\pi)$) \\
  $k_B$ & \si{\joule\per\kelvin} & Boltzmann-Konstante ($k_B \approx 1{,}381 \times 10^{-23}$ J/K) \\
  $\alpha$ & -- & Feinstrukturkonstante ($\alpha \approx 1/137$) \\
  \addlinespace
  
  \multicolumn{3}{l}{\textbf{FFGFT-spezifische Größen}} \\
  \midrule
  $\xi$ & -- & Fraktaler Parameter ($\xi = \frac{4}{3} \times 10^{-4}$) \\
  $D_f$ & -- & Fraktale Dimension ($D_f = 3 - \xi$) \\
  $T_0$ & \si{\second} & Fundamentale fraktale Zeitskala ($T_0 = 1{,}31 \times 10^{-16}$ s) \\
  $M_0$ & \si{\kilo\gram} & Fundamentale fraktale Massenskala \\
  $L_0$ & \si{\meter} & Fundamentale fraktale Längenskala ($L_0 = c T_0$) \\
  $l_0$ & \si{\meter} & Charakteristische Längenskala \\
  $a_0$ & \si{\meter\per\second\squared} & Charakteristische Beschleunigung \\
  \addlinespace
  
  \multicolumn{3}{l}{\textbf{Raumzeit und Geometrie}} \\
  \midrule
  $g_{\mu\nu}$ & -- & Metrischer Tensor \\
  $R_{\mu\nu}$ & \si{\per\meter\squared} & Ricci-Tensor \\
  $R$ & \si{\per\meter\squared} & Ricci-Skalar (Krümmungsskalar) \\
  $G_{\mu\nu}$ & \si{\per\meter\squared} & Einstein-Tensor \\
  $T_{\mu\nu}$ & \si{\joule\per\meter\cubed} & Energie-Impuls-Tensor \\
  $\Gamma^\lambda_{\mu\nu}$ & \si{\per\meter} & Christoffel-Symbole \\
  $ds^2$ & \si{\meter\squared} & Linienelelement \\
  \addlinespace
  
  \multicolumn{3}{l}{\textbf{Spezielle Relativitätstheorie}} \\
  \midrule
  $\gamma$ & -- & Lorentz-Faktor ($\gamma = 1/\sqrt{1-v^2/c^2}$) \\
  $\beta$ & -- & Relativistische Geschwindigkeit ($\beta = v/c$) \\
  $E$ & \si{\joule} & Energie \\
  $E_0$ & \si{\joule} & Ruheenergie ($E_0 = m_0 c^2$) \\
  $p$ & \si{\kilo\gram\meter\per\second} & Impuls \\
  $m_0$ & \si{\kilo\gram} & Ruhemasse \\
  $\tau$ & \si{\second} & Eigenzeit \\
  \addlinespace
  
  \multicolumn{3}{l}{\textbf{Quantenmechanik}} \\
  \midrule
  $\psi$ & -- & Wellenfunktion \\
  $|\psi\rangle$ & -- & Zustandsvektor (Ket-Vektor) \\
  $\langle\psi|$ & -- & Dualer Zustandsvektor (Bra-Vektor) \\
  $\hat{H}$ & \si{\joule} & Hamilton-Operator \\
  $\hat{p}$ & \si{\kilo\gram\meter\per\second} & Impulsoperator \\
  $\hat{x}$ & \si{\meter} & Ortsoperator \\
  $[\hat{A}, \hat{B}]$ & -- & Kommutator ($[\hat{A}, \hat{B}] = \hat{A}\hat{B} - \hat{B}\hat{A}$) \\
  $\Delta x$ & \si{\meter} & Unschärfe in der Position \\
  $\Delta p$ & \si{\kilo\gram\meter\per\second} & Unschärfe im Impuls \\
  \addlinespace
  
  \multicolumn{3}{l}{\textbf{Kosmologie}} \\
  \midrule
  $H_0$ & km/s/Mpc & Hubble-Konstante heute ($H_0 \approx 70$ km/s/Mpc) \\
  $H(t)$ & \si{\per\second} & Hubble-Parameter \\
  $\Omega_m$ & -- & Materie-Dichteparameter \\
  $\Omega_\Lambda$ & -- & Dunkle-Energie-Dichteparameter \\
  $\Omega_k$ & -- & Krümmungs-Dichteparameter \\
  $\Omega_r$ & -- & Strahlungs-Dichteparameter \\
  $a(t)$ & -- & Skalenfaktor des Universums \\
  $z$ & -- & Rotverschiebung (Redshift, $z = \frac{\lambda_{obs} - \lambda_{em}}{\lambda_{em}}$) \\
  $\rho$ & \si{\joule\per\meter\cubed} & Energiedichte \\
  $\rho_c$ & \si{\joule\per\meter\cubed} & Kritische Dichte \\
  $\Lambda$ & \si{\per\meter\squared} & Kosmologische Konstante \\
  $w$ & -- & Zustandsgleichungsparameter ($p = w \rho c^2$) \\
  \addlinespace
  
  \multicolumn{3}{l}{\textbf{Fraktale Geometrie}} \\
  \midrule
  $\mathcal{D}_H$ & -- & Hausdorff-Dimension \\
  $\mathcal{D}_f$ & -- & Fraktale Dimension \\
  $N(\epsilon)$ & -- & Anzahl der Boxen der Größe $\epsilon$ (Box-Counting) \\
  $\epsilon$ & \si{\meter} & Auflösungsskala \\
  $\mathcal{F}$ & -- & Fraktales Maß \\
  \addlinespace
  
  \multicolumn{3}{l}{\textbf{Thermodynamik}} \\
  \midrule
  $S$ & \si{\joule\per\kelvin} & Entropie \\
  $T$ & \si{\kelvin} & Temperatur \\
  $U$ & \si{\joule} & Innere Energie \\
  $F$ & \si{\joule} & Freie Energie (Helmholtz) \\
  $Q$ & \si{\joule} & Wärme \\
  $W$ & \si{\joule} & Arbeit \\
  \addlinespace
  
  \multicolumn{3}{l}{\textbf{Elektrodynamik}} \\
  \midrule
  $E$ & \si{\volt\per\meter} & Elektrisches Feld \\
  $B$ & \si{\tesla} & Magnetisches Feld \\
  $F_{\mu\nu}$ & -- & Elektromagnetischer Feldstärketensor \\
  $A_\mu$ & \si{\volt\second\per\meter} & Viererpotential \\
  $j^\mu$ & \si{\ampere\per\meter\squared} & Viererstromdichte \\
  $q$ & \si{\coulomb} & Elektrische Ladung \\
  \addlinespace
  
  \multicolumn{3}{l}{\textbf{Feldtheorie}} \\
  \midrule
  $\phi$ & -- & Skalarfeld \\
  $\Phi$ & -- & Feldvariable \\
  $\mathcal{L}$ & \si{\joule\per\meter\cubed} & Lagrange-Dichte \\
  $S$ & \si{\joule\second} & Wirkung (Action) \\
  $\partial_\mu$ & \si{\per\meter} & Partielle Ableitung ($\partial_\mu = \partial/\partial x^\mu$) \\
  $D_\mu$ & \si{\per\meter} & Kovariante Ableitung \\
  $\nabla_\mu$ & \si{\per\meter} & Kovariante Ableitung (in gekrümmter Raumzeit) \\
  \addlinespace
  
  \multicolumn{3}{l}{\textbf{Statistische Mechanik}} \\
  \midrule
  $Z$ & -- & Zustandssumme (Partitionsfunktion) \\
  $P$ & -- & Wahrscheinlichkeit \\
  $\langle A \rangle$ & -- & Erwartungswert der Observablen $A$ \\
  $\beta$ & \si{\per\joule} & Inverse Temperatur ($\beta = 1/(k_B T)$) \\
  \addlinespace
  
  \multicolumn{3}{l}{\textbf{Teilchenphysik}} \\
  \midrule
  $m_e$ & \si{\kilo\gram} & Elektronenmasse \\
  $m_\mu$ & \si{\kilo\gram} & Myonmasse \\
  $m_\tau$ & \si{\kilo\gram} & Tauonmasse \\
  $m_\nu$ & \si{\eV/c\squared} & Neutrinomasse \\
  $\theta_{ij}$ & -- & Mischungswinkel \\
  $\delta_{CP}$ & -- & CP-verletzende Phase \\
  \addlinespace
  
  \multicolumn{3}{l}{\textbf{Einheiten und Skalen}} \\
  \midrule
  Gly & -- & Gigalightyear ($10^9$ Lichtjahre) \\
  ly & -- & Lightyear (Lichtjahr, 1 ly $\approx 9{,}461 \times 10^{15}$ m) \\
  Mpc & -- & Megaparsec (1 Mpc $\approx 3{,}26$ Mly) \\
  eV & -- & Elektronenvolt (1 eV $\approx 1{,}602 \times 10^{-19}$ J) \\
  MeV & -- & Mega-Elektronenvolt ($10^6$ eV) \\
  GeV & -- & Giga-Elektronenvolt ($10^9$ eV) \\
  $l_P$ & \si{\meter} & Planck-Länge ($l_P = \sqrt{\hbar G/c^3} \approx 1{,}616 \times 10^{-35}$ m) \\
  $t_P$ & \si{\second} & Planck-Zeit ($t_P = l_P/c \approx 5{,}391 \times 10^{-44}$ s) \\
  $m_P$ & \si{\kilo\gram} & Planck-Masse ($m_P = \sqrt{\hbar c/G} \approx 2{,}176 \times 10^{-8}$ kg) \\
  \addlinespace
  
  \multicolumn{3}{l}{\textbf{Mathematische Operationen}} \\
  \midrule
  $\nabla$ & \si{\per\meter} & Nabla-Operator (Gradient) \\
  $\nabla \cdot$ & \si{\per\meter} & Divergenz \\
  $\nabla \times$ & \si{\per\meter} & Rotation \\
  $\nabla^2$ & \si{\per\meter\squared} & Laplace-Operator \\
  $\Box$ & \si{\per\meter\squared} & d'Alembert-Operator ($\Box = \partial_\mu \partial^\mu$) \\
  $\int$ & -- & Integral \\
  $\sum$ & -- & Summe \\
  $\prod$ & -- & Produkt \\
  \addlinespace
  
  \multicolumn{3}{l}{\textbf{Spezielle Funktionen}} \\
  \midrule
  $\delta(x)$ & -- & Dirac-Delta-Funktion \\
  $\Theta(x)$ & -- & Heaviside-Stufenfunktion \\
  $\Gamma(x)$ & -- & Gammafunktion \\
  $\exp(x)$ oder $e^x$ & -- & Exponentialfunktion \\
  $\ln(x)$ & -- & Natürlicher Logarithmus \\
  \addlinespace
  
  \bottomrule
\end{longtable}

\vspace{1em}
\section*{Indizes und Konventionen}

\begin{itemize}
    \item Griechische Indizes ($\mu, \nu, \rho, \sigma$) laufen von 0 bis 3 (Raumzeit-Indizes)
    \item Lateinische Indizes ($i, j, k, l$) laufen von 1 bis 3 (räumliche Indizes)
    \item Einstein'sche Summationskonvention: Über doppelt auftretende Indizes wird summiert
    \item Minkowski-Metrik: $\eta_{\mu\nu} = \text{diag}(-1, +1, +1, +1)$ (meist gebrauchte Signatur)
\end{itemize}

\vspace{1em}
\noindent\textit{Hinweis: Diese Zeichenerklärung gilt für alle Kapitel der FFGFT Narrative Edition.}



