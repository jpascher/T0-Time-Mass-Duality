% Zentrale Zeichenerklärung für FFGFT Narrative (Deutsch)
% Diese Datei wird in das Master-Dokument eingebunden

\chapter*{Zentrale Zeichenerklärung}
\addcontentsline{toc}{chapter}{Zentrale Zeichenerklärung}

\section*{Fundamentale Parameter}

\begin{description}
\item[$\xi$] Fraktaler Parameter: $\xi = \frac{4}{3} \times 10^{-4}$ \\
Der fundamentale geometrische Parameter, der die fraktale Struktur der Raumzeit bestimmt.

\item[$D_f$] Fraktale Dimension: $D_f = 3 - \xi$ \\
Die effektive Dimension der fraktalen Raumzeit.

\item[$T_0$] Fundamentale Zeitskala: $T_0 = 1,31 \times 10^{-16}$ s \\
Die kleinste physikalisch bedeutsame Zeitskala in der FFGFT.

\item[$c$] Lichtgeschwindigkeit im Vakuum: $c \approx 2{,}998 \times 10^8$ m/s \\
Emergiert aus der fraktalen Struktur: $c^2 = \frac{1}{\xi \cdot T_0^2}$

\end{description}

\section*{Quantenmechanische Größen}

\begin{description}
\item[$\hbar$] Reduziertes Plancksches Wirkungsquantum: $\hbar = \frac{h}{2\pi}$ \\
Fundamentale Konstante der Quantenmechanik.

\item[$\psi$] Wellenfunktion \\
Beschreibt den Quantenzustand eines Systems.

\item[$\hat{H}$] Hamilton-Operator \\
Operator für die Gesamtenergie eines quantenmechanischen Systems.

\end{description}

\section*{TODO}
% Weitere Abschnitte werden hinzugefügt:
% - Relativistische Größen
% - Kosmologische Parameter
% - Teilchenphysik-Notation
% - Mathematische Symbole und Operatoren
