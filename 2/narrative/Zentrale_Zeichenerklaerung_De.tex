<<<<<<< HEAD
% Zentrale Zeichenerklärung für FFGFT Narrative (Deutsch)
% Diese Datei wird in das Master-Dokument eingebunden

\chapter*{Zentrale Zeichenerklärung}
\addcontentsline{toc}{chapter}{Zentrale Zeichenerklärung}

\section*{Fundamentale Parameter}

\begin{description}
\item[$\xi$] Fraktaler Parameter: $\xi = \frac{4}{3} \times 10^{-4}$ \\
Der fundamentale geometrische Parameter, der die fraktale Struktur der Raumzeit bestimmt.

\item[$D_f$] Fraktale Dimension: $D_f = 3 - \xi$ \\
Die effektive Dimension der fraktalen Raumzeit.

\item[$T_0$] Fundamentale Zeitskala: $T_0 = 1,31 \times 10^{-16}$ s \\
Die kleinste physikalisch bedeutsame Zeitskala in der FFGFT.

\item[$c$] Lichtgeschwindigkeit im Vakuum: $c \approx 2{,}998 \times 10^8$ m/s \\
Emergiert aus der fraktalen Struktur: $c^2 = \frac{1}{\xi \cdot T_0^2}$

\end{description}

\section*{Quantenmechanische Größen}

\begin{description}
\item[$\hbar$] Reduziertes Plancksches Wirkungsquantum: $\hbar = \frac{h}{2\pi}$ \\
Fundamentale Konstante der Quantenmechanik.

\item[$\psi$] Wellenfunktion \\
Beschreibt den Quantenzustand eines Systems.

\item[$\hat{H}$] Hamilton-Operator \\
Operator für die Gesamtenergie eines quantenmechanischen Systems.

\end{description}

\section*{TODO}
% Weitere Abschnitte werden hinzugefügt:
% - Relativistische Größen
% - Kosmologische Parameter
% - Teilchenphysik-Notation
% - Mathematische Symbole und Operatoren
=======
\chapter*{Zeichenerklärung}
\addcontentsline{toc}{chapter}{Zeichenerklärung}

Diese zentrale Zeichenerklärung listet alle wichtigen mathematischen Symbole und physikalischen Größen auf, die in der FFGFT Narrative Edition verwendet werden.

\section*{Fundamentale Konstanten}

\begin{itemize}
    \item $c$ -- Lichtgeschwindigkeit im Vakuum ($c \approx 3 \times 10^8$ m/s)
    \item $G$ -- Gravitationskonstante ($G \approx 6.674 \times 10^{-11}$ m$^3$ kg$^{-1}$ s$^{-2}$)
    \item $\hbar$ -- Reduziertes Plancksches Wirkungsquantum ($\hbar = h/(2\pi) \approx 1.055 \times 10^{-34}$ J·s)
    \item $h$ -- Plancksches Wirkungsquantum ($h \approx 6.626 \times 10^{-34}$ J·s)
    \item $k_B$ -- Boltzmann-Konstante ($k_B \approx 1.381 \times 10^{-23}$ J/K)
\end{itemize}

\section*{T0-spezifische Größen}

\begin{itemize}
    \item $T_0$ -- Fundamentale fraktale Zeitskala (charakteristische Zeit des kosmischen Gehirns)
    \item $M_0$ -- Fundamentale fraktale Massenskala
    \item $L_0$ -- Fundamentale fraktale Längenskala ($L_0 = c T_0$)
    \item $\mathcal{D}$ -- Fraktale Dimension
    \item $\alpha$ -- Feinstrukturkonstante ($\alpha \approx 1/137$)
\end{itemize}

\section*{Raumzeit und Geometrie}

\begin{itemize}
    \item $g_{\mu\nu}$ -- Metrischer Tensor
    \item $R_{\mu\nu}$ -- Ricci-Tensor
    \item $R$ -- Ricci-Skalar (Krümmungsskalar)
    \item $G_{\mu\nu}$ -- Einstein-Tensor
    \item $T_{\mu\nu}$ -- Energie-Impuls-Tensor
    \item $\Gamma^\lambda_{\mu\nu}$ -- Christoffel-Symbole (Zusammenhangskoeffizienten)
    \item $ds^2$ -- Linienelelement (infinitesimaler Abstand in der Raumzeit)
\end{itemize}

\section*{Spezielle Relativitätstheorie}

\begin{itemize}
    \item $\gamma$ -- Lorentz-Faktor ($\gamma = 1/\sqrt{1-v^2/c^2}$)
    \item $\beta$ -- Relativistische Geschwindigkeit ($\beta = v/c$)
    \item $E$ -- Energie
    \item $E_0$ -- Ruheenergie ($E_0 = m_0 c^2$)
    \item $p$ -- Impuls
    \item $m_0$ -- Ruhemasse
    \item $\tau$ -- Eigenzeit
\end{itemize}

\section*{Quantenmechanik}

\begin{itemize}
    \item $\psi$ -- Wellenfunktion
    \item $|\psi\rangle$ -- Zustandsvektor (Ket-Vektor in der Dirac-Notation)
    \item $\langle\psi|$ -- Dualer Zustandsvektor (Bra-Vektor)
    \item $\hat{H}$ -- Hamilton-Operator
    \item $\hat{p}$ -- Impulsoperator
    \item $\hat{x}$ -- Ortsoperator
    \item $[\hat{A}, \hat{B}]$ -- Kommutator ($[\hat{A}, \hat{B}] = \hat{A}\hat{B} - \hat{B}\hat{A}$)
    \item $\Delta x$ -- Unschärfe in der Position
    \item $\Delta p$ -- Unschärfe im Impuls
\end{itemize}

\section*{Kosmologie}

\begin{itemize}
    \item $H_0$ -- Hubble-Konstante heute ($H_0 \approx 70$ km/s/Mpc)
    \item $\Omega_m$ -- Materie-Dichteparameter
    \item $\Omega_\Lambda$ -- Dunkle-Energie-Dichteparameter
    \item $\Omega_k$ -- Krümmungs-Dichteparameter
    \item $a(t)$ -- Skalenfaktor des Universums
    \item $z$ -- Rotverschiebung (Redshift)
    \item $\rho$ -- Energiedichte
    \item $\Lambda$ -- Kosmologische Konstante
\end{itemize}

\section*{Fraktale Geometrie}

\begin{itemize}
    \item $\mathcal{D}_H$ -- Hausdorff-Dimension
    \item $\mathcal{D}_f$ -- Fraktale Dimension
    \item $N(\epsilon)$ -- Anzahl der Boxen der Größe $\epsilon$ (Box-Counting)
    \item $\epsilon$ -- Auflösungsskala
    \item $\mathcal{F}$ -- Fraktales Maß
\end{itemize}

\section*{Thermodynamik}

\begin{itemize}
    \item $S$ -- Entropie
    \item $T$ -- Temperatur
    \item $U$ -- Innere Energie
    \item $F$ -- Freie Energie (Helmholtz)
    \item $Q$ -- Wärme
    \item $W$ -- Arbeit
\end{itemize}

\section*{Elektrodynamik}

\begin{itemize}
    \item $E$ -- Elektrisches Feld
    \item $B$ -- Magnetisches Feld
    \item $F_{\mu\nu}$ -- Elektromagnetischer Feldstärketensor
    \item $A_\mu$ -- Viererpotential
    \item $j^\mu$ -- Viererstromdichte
    \item $q$ -- Elektrische Ladung
\end{itemize}

\section*{Feldtheorie}

\begin{itemize}
    \item $\phi$ -- Skalarfeld
    \item $\mathcal{L}$ -- Lagrange-Dichte
    \item $S$ -- Wirkung (Action)
    \item $\partial_\mu$ -- Partielle Ableitung ($\partial_\mu = \partial/\partial x^\mu$)
    \item $D_\mu$ -- Kovariante Ableitung
    \item $\nabla_\mu$ -- Kovariante Ableitung (in gekrümmter Raumzeit)
\end{itemize}

\section*{Statistische Mechanik}

\begin{itemize}
    \item $Z$ -- Zustandssumme (Partitionsfunktion)
    \item $P$ -- Wahrscheinlichkeit
    \item $\langle A \rangle$ -- Erwartungswert der Observablen $A$
    \item $\beta$ -- Inverse Temperatur ($\beta = 1/(k_B T)$)
\end{itemize}

\section*{Einheiten und Skalen}

\begin{itemize}
    \item Gly -- Gigalightyear ($10^9$ Lichtjahre)
    \item ly -- Lightyear (Lichtjahr)
    \item Mpc -- Megaparsec ($1$ Mpc $\approx 3.26$ Mly)
    \item MeV -- Mega-Elektronenvolt ($10^6$ eV)
    \item GeV -- Giga-Elektronenvolt ($10^9$ eV)
    \item $l_P$ -- Planck-Länge ($l_P = \sqrt{\hbar G/c^3} \approx 1.616 \times 10^{-35}$ m)
    \item $t_P$ -- Planck-Zeit ($t_P = l_P/c \approx 5.391 \times 10^{-44}$ s)
    \item $m_P$ -- Planck-Masse ($m_P = \sqrt{\hbar c/G} \approx 2.176 \times 10^{-8}$ kg)
\end{itemize}

\section*{Mathematische Operationen}

\begin{itemize}
    \item $\nabla$ -- Nabla-Operator (Gradient)
    \item $\nabla \cdot$ -- Divergenz
    \item $\nabla \times$ -- Rotation
    \item $\nabla^2$ -- Laplace-Operator
    \item $\Box$ -- d'Alembert-Operator ($\Box = \partial_\mu \partial^\mu$)
    \item $\int$ -- Integral
    \item $\sum$ -- Summe
    \item $\prod$ -- Produkt
\end{itemize}

\section*{Spezielle Funktionen}

\begin{itemize}
    \item $\delta(x)$ -- Dirac-Delta-Funktion
    \item $\Theta(x)$ -- Heaviside-Stufenfunktion
    \item $\Gamma(x)$ -- Gammafunktion
    \item $\exp(x)$ oder $e^x$ -- Exponentialfunktion
    \item $\ln(x)$ -- Natürlicher Logarithmus
\end{itemize}

\section*{Indizes und Konventionen}

\begin{itemize}
    \item Griechische Indizes ($\mu, \nu, \rho, \sigma$) laufen von 0 bis 3 (Raumzeit-Indizes)
    \item Lateinische Indizes ($i, j, k, l$) laufen von 1 bis 3 (räumliche Indizes)
    \item Einstein'sche Summationskonvention: Über doppelt auftretende Indizes wird summiert
    \item Minkowski-Metrik: $\eta_{\mu\nu} = \text{diag}(-1, +1, +1, +1)$ (meist gebrauchte Signatur)
\end{itemize}

\vspace{1em}
\noindent\textit{Hinweis: Diese Zeichenerklärung gilt für alle Kapitel der FFGFT Narrative Edition.}
>>>>>>> copilot/narrative-reset
