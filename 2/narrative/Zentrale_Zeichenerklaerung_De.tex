% ==============================================================================
% Zentrale Zeichenerklärung für alle Kapitel
% Diese Liste ersetzt alle individuellen tcolorbox-Zeichenerklärungen
% ==============================================================================

\chapter*{Notation und Zeichenerklärung}
\addcontentsline{toc}{chapter}{Notation und Zeichenerklärung}

Diese zentrale Übersicht erklärt alle mathematischen Symbole und Notationen, die in diesem Buch verwendet werden.

\section*{Grundlegende Größen}

\begin{itemize}[leftmargin=2cm, labelsep=0.5cm]
    \item[$T_0$] Fundamentale Zeitkonstante (Urknallzeit)
    \item[$M_{\text{tot}}$] Gesamtmasse des Universums
    \item[$c$] Lichtgeschwindigkeit
    \item[$G$] Gravitationskonstante
    \item[$\hbar$] Reduziertes Planck'sches Wirkungsquantum
    \item[$k_B$] Boltzmann-Konstante
\end{itemize}

\section*{Geometrische Größen}

\begin{itemize}[leftmargin=2cm, labelsep=0.5cm]
    \item[$R_{\text{Uni}}$] Radius des beobachtbaren Universums
    \item[$R_S$] Schwarzschild-Radius
    \item[$a(t)$] Skalenfaktor
    \item[$H(t)$] Hubble-Parameter
    \item[$g_{\mu\nu}$] Metrischer Tensor
    \item[$\Gamma^\lambda_{\mu\nu}$] Christoffel-Symbole
    \item[$R_{\mu\nu}$] Ricci-Tensor
    \item[$R$] Ricci-Skalar (Krümmungsskalar)
\end{itemize}

\section*{Energie und Dichte}

\begin{itemize}[leftmargin=2cm, labelsep=0.5cm]
    \item[$\rho$] Energiedichte
    \item[$\rho_c$] Kritische Dichte
    \item[$\rho_m$] Materiedichte
    \item[$\rho_r$] Strahlungsdichte
    \item[$\rho_\Lambda$] Dunkle-Energie-Dichte (kosmologische Konstante)
    \item[$p$] Druck
    \item[$w$] Zustandsgleichungsparameter ($w = p/\rho$)
\end{itemize}

\section*{Kosmologische Parameter}

\begin{itemize}[leftmargin=2cm, labelsep=0.5cm]
    \item[$H_0$] Hubble-Konstante heute
    \item[$\Omega_m$] Materiedichte-Parameter
    \item[$\Omega_r$] Strahlungsdichte-Parameter
    \item[$\Omega_\Lambda$] Dunkle-Energie-Parameter
    \item[$\Omega_k$] Krümmungsparameter
    \item[$\Omega_{\text{tot}}$] Gesamtdichte-Parameter
    \item[$z$] Rotverschiebung
\end{itemize}

\section*{Feldtheoretische Größen}

\begin{itemize}[leftmargin=2cm, labelsep=0.5cm]
    \item[$\phi$] Skalarfeld
    \item[$\psi$] Wellenfunktion
    \item[$A_\mu$] Vektorpotential
    \item[$F_{\mu\nu}$] Feldstärketensor
    \item[$T_{\mu\nu}$] Energie-Impuls-Tensor
    \item[$\mathcal{L}$] Lagrange-Dichte
    \item[$S$] Wirkung
\end{itemize}

\section*{Quantenmechanische Größen}

\begin{itemize}[leftmargin=2cm, labelsep=0.5cm]
    \item[$\hat{H}$] Hamilton-Operator
    \item[$\hat{p}$] Impulsoperator
    \item[$\hat{x}$] Ortsoperator
    \item[$\langle \cdot \rangle$] Erwartungswert
    \item[$[\hat{A}, \hat{B}]$] Kommutator
    \item[$\Delta x$] Unschärfe/Standardabweichung
\end{itemize}

\section*{Thermodynamische Größen}

\begin{itemize}[leftmargin=2cm, labelsep=0.5cm]
    \item[$S$] Entropie
    \item[$T$] Temperatur
    \item[$U$] Innere Energie
    \item[$F$] Freie Energie
    \item[$\beta$] Inverse Temperatur ($\beta = 1/(k_B T)$)
    \item[$Z$] Zustandssumme
\end{itemize}

\section*{Spezielle Symbole und Operatoren}

\begin{itemize}[leftmargin=2cm, labelsep=0.5cm]
    \item[$\nabla$] Nabla-Operator (Gradient)
    \item[$\nabla^2$, $\Delta$] Laplace-Operator
    \item[$\Box$] d'Alembert-Operator
    \item[$\partial_\mu$] Partielle Ableitung ($\partial/\partial x^\mu$)
    \item[$\nabla_\mu$] Kovariante Ableitung
    \item[$\delta_{ij}$] Kronecker-Delta
    \item[$\epsilon_{\mu\nu\rho\sigma}$] Levi-Civita-Tensor
\end{itemize}

\section*{Einheiten und Konstanten}

\begin{itemize}[leftmargin=2cm, labelsep=0.5cm]
    \item[$\SI{}{\meter}$] Meter
    \item[$\SI{}{\second}$] Sekunde
    \item[$\SI{}{\kilogram}$] Kilogramm
    \item[$\SI{}{\joule}$] Joule (Energie)
    \item[$\SI{}{\electronvolt}$] Elektronenvolt
    \item[$\SI{}{\year}$] Jahr
    \item[$\text{Gly}$] Gigalichtjahr ($10^9$ Lichtjahre)
    \item[$\text{Mpc}$] Megaparsec
\end{itemize}

\section*{Mathematische Konventionen}

\begin{itemize}[leftmargin=2cm, labelsep=0.5cm]
    \item Griechische Indizes ($\mu, \nu, \rho, \sigma$): Raumzeit-Indizes $0,1,2,3$
    \item Lateinische Indizes ($i, j, k$): Räumliche Indizes $1,2,3$
    \item Einstein-Summationskonvention: Über doppelt auftretende Indizes wird summiert
    \item Signatur der Metrik: $(-,+,+,+)$ (meist-plus-Konvention)
    \item Natürliche Einheiten: $c = \hbar = k_B = 1$ (wo nicht anders angegeben)
\end{itemize}

\vspace{1cm}
\noindent
Diese Zeichenerklärung dient als schnelle Referenz. Detaillierte Definitionen und physikalische Bedeutungen werden in den jeweiligen Kapiteln erläutert.

\clearpage
