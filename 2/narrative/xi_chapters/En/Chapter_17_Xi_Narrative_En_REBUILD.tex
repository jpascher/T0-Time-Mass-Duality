**Chapter: Ratios as the Fundamental Language of Nature**
\label{chap:ratios_fundamental}

\begin{quote}
	\textit{This chapter summarizes a fundamental insight that runs through the entire T0 theory and extends far beyond it:} \textbf{Ratios, not absolute values, are the fundamental language of nature}. \textit{This insight, which originates in music theory (Euler's Tonnetz), not only explains why the ratio-based formulation of the T0 theory works, but also reveals a profound truth about the structure of reality itself. We show that all measurements can, in principle, only capture relations, that physics' obsession with $\alpha = 1/137$ was a century-long distraction, and that even seemingly fixed standards (like atomic clocks) only measure ratios.}
\end{quote}

\section{Introduction: The Question of Simplicity}

At the beginning of this investigation was a seemingly simple question: Why are ratios in the T0 theory so simple, while our world is so complex?

Our world is:
\begin{itemize}[nosep]
	\item Geometrically three-dimensional
	\item Fractal ($D_f = 3 - \xi$)
	\item Hierarchically structured (torus modes)
	\item Discretely quantized
	\item A multi-scale system
\end{itemize}

Yet we obtain surprisingly simple ratios in the T0 theory:
\begin{equation}
	\frac{a_\tau}{a_\mu} = \left(\frac{m_\tau}{m_\mu}\right)^2 = 283
	\label{eq:simple_ratio}
\end{equation}

\textbf{Why?} The answer leads us to a profound truth about the nature of measurability and reality itself.

\section{The Historical Perspective: From the Tonnetz to Physics}

\subsection{Euler's Tonnetz (1739)}

The journey began almost 40 years ago with the study of Euler's Tonnetz -- a mathematical lattice describing the structure of musical harmony.

\textbf{Basic Principle:} From two simple generators (fifth 3:2 and major third 5:4) emerge, through combination and octave reduction, all musical tones:

\textbf{Euler's Tonnetz:}
\begin{center}
	\begin{small}
		\begin{tabular}{ccccccccc}
			& & \multicolumn{7}{c}{Fifth $\rightarrow$} \\
			& & F & -- & C & -- & G & -- & D \\
			& & $\vert$ & & $\vert$ & & $\vert$ & & $\vert$ \\
			Third $\downarrow$ & & A & -- & E & -- & B & -- & F$^\sharp$ \\
		\end{tabular}
	\end{small}
\end{center}

\textbf{The First Insight:} Few simple \textit{ratios} generate the entire musical diversity through combination.

\textbf{The Second Insight:} The ear hears \textit{intervals} (ratios), not absolute frequencies. The octave (2:1) sounds the same, whether at 220 Hz or 440 Hz.

\subsection{Transfer to Physics}

The big question was: \textit{If ratios are fundamental in music, are they also in physics?}

The T0 theory gives the answer: \textbf{Yes!}

\begin{table}[h]
	\centering
	\begin{tabularx}{\textwidth}{>{\raggedright\arraybackslash}X>{\raggedright\arraybackslash}X>{\raggedright\arraybackslash}X}
		\toprule
		\textbf{Aspect} & \textbf{Music} & \textbf{Physics (T0)} \\
		\midrule
		Generators & Fifth (3:2), Major Third (5:4) & $r$-values, $\xi^p$ \\
		Scaling & Octaves ($\times 2$) & Generations ($\xi$ powers) \\
		Lattice & Tonnetz & Particle spectrum \\
		Fundamental & Intervals & Mass ratios \\
		Arbitrary & 440 Hz (concert pitch) & 105.658 MeV ($m_\mu$ in chosen units) \\
		Detector & Ear (perceives intervals) & Nature (responds to ratios) \\
		\bottomrule
	\end{tabularx}
	\caption{Parallel Structures: Music and Physics}
	\label{tab:music_physics}
\end{table}

\section{Why Ratios are So Simple}

\subsection{Mathematical Reason: Multiplicative Scaling}

All corrections in the T0 theory act multiplicatively:

\begin{align}
	m_\ell^\text{(ideal)} &= r_\ell \times \xi^{p_\ell} \\
	m_\ell^\text{(fractal)} &= m_\ell^\text{(ideal)} \times K_\text{frak}(D_f) \\
	m_\ell^\text{(hierarchical)} &= m_\ell^\text{(fractal)} \times K_\text{mode}(n,l,j) \\
	m_\ell^\text{(quantized)} &= m_\ell^\text{(hierarchical)} \times K_\text{quant}
\end{align}

\textbf{In ratios:}
\begin{equation}
	\frac{m_\tau}{m_\mu} = \frac{r_\tau \xi^{p_\tau}}{r_\mu \xi^{p_\mu}} \times \frac{K_\text{frak}}{K_\text{frak}} \times \frac{K_\text{mode}(\tau)}{K_\text{mode}(\mu)} \times \frac{K_\text{quant}(\tau)}{K_\text{quant}(\mu)}
\end{equation}

If the corrections are \textit{universal} (equal for all particles):
\begin{equation}
	\frac{m_\tau}{m_\mu} = \frac{r_\tau \xi^{p_\tau}}{r_\mu \xi^{p_\mu}}
\end{equation}

\textbf{All corrections cancel!}

\subsection{Physical Reason: Universality}

\textbf{Fractal Dimension $D_f$:}
\begin{itemize}[nosep]
	\item Property of spacetime
	\item Applies equally to all particles
	\item $\Rightarrow K_\text{frak}(\tau) = K_\text{frak}(\mu)$
\end{itemize}

\textbf{Hierarchical Structure:}
\begin{itemize}[nosep]
	\item Torus geometry is universal
	\item All leptons on the same torus
	\item $\Rightarrow$ If $(n,l,j)$ are equal: $K_\text{mode}(\tau) = K_\text{mode}(\mu)$
\end{itemize}

\textbf{Quantization:}
\begin{itemize}[nosep]
	\item Discretization is universal
	\item $\Rightarrow K_\text{quant}(\tau) = K_\text{quant}(\mu)$
\end{itemize}

\subsection{Geometric Reason: Fractal Self-Similarity}

Fractals are self-similar on all scales. Mathematically, this means:
\begin{equation}
	F(\lambda x) = \lambda^\alpha F(x)
\end{equation}

For ratios:
\begin{equation}
	\frac{F(\lambda x_1)}{F(\lambda x_2)} = \frac{\lambda^\alpha F(x_1)}{\lambda^\alpha F(x_2)} = \frac{F(x_1)}{F(x_2)}
\end{equation}

\textbf{Ratios are scale-invariant!} The fractal structure cancels out.

\subsection{Quantum Theoretical Reason: Renormalization}

From the perspective of the renormalization group, physical quantities depend on the scale $\mu$:
\begin{equation}
	m(\mu) = m_0 \times Z_m(\mu)
\end{equation}

But ratios are RG-invariant:
\begin{equation}
	\frac{m_1(\mu)}{m_2(\mu)} = \frac{m_1^0 \times Z_m(\mu)}{m_2^0 \times Z_m(\mu)} = \frac{m_1^0}{m_2^0}
\end{equation}

The renormalization factors cancel! In the T0 theory, the fractal/hierarchical corrections correspond precisely to such renormalization effects.

\subsection{Symmetry Reason}

Ratios are protected by symmetries:

\begin{itemize}[nosep]
	\item \textbf{Scale Symmetry:} $x \to \lambda x$ for all $x$ $\Rightarrow$ ratios invariant
	\item \textbf{Units Symmetry:} $m \to \text{factor} \times m$ for all $m$ $\Rightarrow$ ratios invariant
	\item \textbf{Fractal Symmetry:} Self-similarity $\Rightarrow$ ratios invariant
\end{itemize}

\subsection{Information-Theoretic Reason}

\textbf{Absolute values contain:}
\begin{itemize}[nosep]
	\item Choice of units ($\hbar$, $c$, $G$, $\alpha$)
	\item Renormalization ($K_\text{frak}$, $K_\text{mode}$)
	\item Scale choice ($\mu$)
	\item $\Rightarrow$ Lots of noise
\end{itemize}

\textbf{Ratios contain:}
\begin{itemize}[nosep]
	\item Only relative geometry ($r_\tau/r_\mu$, $p_\tau - p_\mu$)
	\item Unit-invariant
	\item Renormalization-invariant
	\item $\Rightarrow$ Only signal
\end{itemize}

The signal-to-noise ratio is optimal!

\section{The Great Deception: $\alpha = 1/137$}

\subsection{Can one REALLY set ALL constants to 1?}

Before analyzing the obsession with $\alpha = 1/137$, we must clarify a fundamental question:

\begin{important}
	\textbf{Can one REALLY set ALL fundamental constants to 1?}
	
	\textbf{Answer: YES!}
	
	In pure natural units one can set:
	\begin{equation}
		\hbar = c = G = \alpha = \alpha_s = k_B = \ldots = 1
	\end{equation}
	
	\textbf{BUT:} This has consequences for the definition of certain units.
\end{important}

\subsubsection{Two Types of Constants}

There is an important distinction:

\textbf{1. Conversion Factors} (always settable to 1):
\begin{itemize}[nosep]
	\item $\hbar$, $c$, $G$, $k_B$
	\item These only connect different units
	\item Eliminable by unit choice
\end{itemize}

\textbf{2. Coupling Constants} (dimensionless, but...):
\begin{itemize}[nosep]
	\item $\alpha \approx 1/137$ (electromagnetic)
	\item $\alpha_s$ (strong)
	\item These seem to describe physical strength
\end{itemize}

\textbf{The Question:} Can coupling constants also be set to 1?

\subsubsection{The Answer: Yes, by redefining units}

One \textit{can} set $\alpha = 1$, but this means:

\textbf{Standard definition of $\alpha$:}
\begin{equation}
	\alpha = \frac{e^2}{4\pi\epsilon_0 \hbar c}
\end{equation}

\textbf{In SI units:}
\begin{align}
	e &= 1.602 \times 10^{-19} \text{ C (Coulomb)} \\
	\alpha &= \frac{1}{137.036} \approx 0.00729735
\end{align}

\textbf{If one wants to set $\alpha = 1$:}

One must redefine the charge unit. The fine-structure constant is:
\begin{equation}
	\alpha = \frac{e^2}{4\pi\epsilon_0 \hbar c}
\end{equation}

To enforce $\alpha = 1$:
\begin{equation}
	1 = \frac{e_{\text{new}}^2}{4\pi\epsilon_0 \hbar c} \quad \Rightarrow \quad e_{\text{new}}^2 = 4\pi\epsilon_0 \hbar c
\end{equation}

In natural units one already sets $\hbar = c = 1$. Additionally, one can define the electrical units so that $4\pi\epsilon_0 = 1$ (rationalized Heaviside-Lorentz units). Then:
\begin{equation}
	e_{\text{new}}^2 = 1 \quad \Rightarrow \quad e_{\text{new}} = 1 \quad \text{(dimensionless)}
\end{equation}

\textbf{What does this mean physically?}

The consequences are clear:
\begin{itemize}[nosep]
	\item The elementary charge is no longer measured as $1.602 \times 10^{-19}$ C, but as a dimensionless 1
	\item The strength of the EM interaction is now encoded in the definition of the charge unit
	\item All electric fields become dimensionless
\end{itemize}

\vspace{0.5cm}
\noindent
\textbf{Comparison SI vs. Natural Units:}

\vspace{0.3cm}
\noindent
\begin{minipage}[t]{0.48\textwidth}
	\textbf{SI Units} ($\alpha \approx 1/137$):
	\begin{itemize}[nosep]
		\item $e = 1.602 \times 10^{-19}$ C
		\item Coulomb fixed by definition
		\item E-field in V/m
		\item $\alpha \approx 1/137.036$
		\item EM appears weak
	\end{itemize}
\end{minipage}
\hfill
\begin{minipage}[t]{0.48\textwidth}
	\textbf{Natural Units} ($\alpha = 1$):
	\begin{itemize}[nosep]
		\item $e = 1$ (dimensionless)
		\item Coulomb rescaled
		\item E-field dimensionless
		\item $\alpha = 1$
		\item EM strength in units
	\end{itemize}
\end{minipage}
\vspace{0.5cm}

\textbf{Unit System Conversion: Where does $\sqrt{4\pi}$ come from?}

The factor $\sqrt{4\pi}$ appears when switching between different electromagnetic unit systems. To understand this, we must distinguish three historical systems:

\textbf{1. Gaussian units (historically oldest system):}
\begin{itemize}[nosep]
	\item \textit{Not rationalized}: Factors $4\pi$ appear in the field equations
	\item Coulomb's law: $F = \frac{q_1 q_2}{r^2}$
	\item Maxwell's equations contain $4\pi$, e.g.: $\nabla \cdot \mathbf{E} = 4\pi\rho$
\end{itemize}

\textbf{2. Heaviside-Lorentz units (rationalized system):}
\begin{itemize}[nosep]
	\item The factor $4\pi$ was removed from the field equations
	\item Coulomb's law: $F = \frac{q_1 q_2}{4\pi r^2}$
	\item Maxwell's equations are more elegant, e.g.: $\nabla \cdot \mathbf{E} = \rho$
\end{itemize}

\textbf{3. SI system (standardized today):}
\begin{itemize}[nosep]
	\item Uses $\epsilon_0$ and $\mu_0$ explicitly
	\item Practical for engineers
	\item Theoretically less elegant
\end{itemize}

\textbf{Why rationalized?}

The term rationalized refers to removing the factor $4\pi$ from the fundamental equations of electrodynamics. The $4\pi$ originally comes from the surface area of a sphere ($4\pi r^2$) and appears naturally in spherically symmetric problems.

Through \textit{rationalization}, this geometric constant is shifted into the definition of the charge unit:
\begin{itemize}[nosep]
	\item Gaussian: $\nabla \cdot \mathbf{E} = 4\pi\rho$ (factor $4\pi$ in equation)
	\item Heaviside-Lorentz: $\nabla \cdot \mathbf{E} = \rho$ (factor $4\pi$ in charge definition)
\end{itemize}

\textbf{Historical Background:}

\textit{Oliver Heaviside} (1850--1925), English autodidact, simplified Maxwell's original 20 equations into the 4 vector equations known today. He introduced the rationalized units.

\textit{Hendrik Lorentz} (1853--1928), Dutch physicist, used and popularized this system in his work on electron theory.

The combined name Heaviside-Lorentz units honors both pioneers.

\textbf{Conversion between systems:}

Charge transforms as:
\begin{equation}
	e_{\text{HL}} = \frac{e_{\text{Gaussian}}}{\sqrt{4\pi}}
\end{equation}

The fine-structure constant in both systems:
\begin{align}
	\text{Gaussian:} \quad &\alpha = \frac{e_G^2}{\hbar c} \\
	\text{Heaviside-Lorentz:} \quad &\alpha = \frac{e_{HL}^2}{4\pi\hbar c}
\end{align}

In rationalized natural units ($\hbar = c = 1$, $4\pi\epsilon_0 = 1$) with $\alpha = 1$:
\begin{equation}
	\alpha = \frac{e_{HL}^2}{4\pi} = 1 \quad \Rightarrow \quad e_{HL} = \sqrt{4\pi} \approx 3.545
\end{equation}

But in a consistent natural system, one would simply set $e = 1$ and use the above equation as a \textit{definitional equation} for the unit system.

\textbf{The Core Statement:}

The choice between Gaussian, Heaviside-Lorentz, and SI units is a \textit{convention} -- like the choice between degrees Celsius and Kelvin. The physics remains the same. The T0 theory implicitly uses a kind of geometrically rationalized system in which \textit{all} fundamental constants can be set to 1, because the actual physics resides in the dimensionless ratios.
\subsubsection{Is this legitimate?}

\textbf{Yes, completely!} Why?

\begin{enumerate}
	\item \textbf{What is a Coulomb absolutely?}
	
	Historically: The charge that flows in 1 second at 1 ampere.
	
	But: What is 1 ampere absolutely? A \textit{definition}!
	
	\item \textbf{One can freely choose charge units}
	
	Just as one can freely choose meters, kilograms, seconds, one can also freely choose the charge unit.
	
	\item \textbf{The physics does not change}
	
	Charge ratios remain constant:
	\begin{equation}
		\frac{Q_1}{Q_2} = \text{constant (in all unit systems)}
	\end{equation}
\end{enumerate}

\subsubsection{Why is this not normally done?}

\textbf{Practical reasons:}
\begin{itemize}[nosep]
	\item SI units are historically established
	\item Engineering convention
	\item $\alpha \approx 1/137$ shows the EM force is weak (relative to what? That's the problem!)
\end{itemize}

\textbf{But physically:} There is \textit{no} fundamental reason to set $\alpha \neq 1$!

\subsubsection{The Deeper Truth}

If one sets $\alpha = 1$ \textit{and} $\alpha_s = 1$:

\textbf{Question:} Where then is the information that the EM force is weaker than the strong force?

\textbf{Answer:} In the \textit{ratios} of other measurable quantities!

For example:
\begin{itemize}[nosep]
	\item Ratio of binding energies
	\item Ratio of interaction ranges
	\item Ratio of couplings to different particles
\end{itemize}

The strength of an interaction is \textit{always} relative to other interactions!

\begin{keypoint}
	\textbf{Core Statement:}
	
	One can set \textit{all} fundamental constants ($\hbar$, $c$, $G$, $\alpha$, $\alpha_s$, ...) to 1.
	
	This requires redefining certain units (like the Coulomb for $\alpha$), but it is \textbf{physically legitimate}.
	
	The \textit{entire} physics then resides in:
	\begin{itemize}[nosep]
		\item \textbf{Ratios} of masses, lengths, times
		\item \textbf{Geometric factors} ($r$, $p$, $\xi$ in T0)
		\item \textbf{Topological properties} (torus windings)
	\end{itemize}
	
	In natural units there are \textbf{no} constants $\neq 1$!
\end{keypoint}

\subsection{100 Years of Obsession}

\begin{quote}
	\textit{All these fifty years of conscious brooding have brought me no nearer to the answer to the question, 'What are light quanta?' Nowadays every Tom, Dick and Harry thinks he knows it, but he is mistaken.} -- \textbf{Richard Feynman} on $\alpha$
\end{quote}

\begin{quote}
	\textit{When I die my first question to the Devil will be: What is the meaning of the fine structure constant?} -- \textbf{Wolfgang Pauli}
\end{quote}

Generations of physicists have tried to:
\begin{itemize}[nosep]
	\item Calculate $\alpha$ from a fundamental theory
	\item Number mysticism (137 = prime number?, Kabbalah?)
	\item Complicated models (Eddington, Wyler, string theory, GUTs, ...)
\end{itemize}

\textbf{Result: 100 years wasted!}

\subsection{The Truth About $\alpha$}

$\alpha = 1/137$ is \textbf{not fundamental!}

It is a \textbf{conversion factor} between:
\begin{itemize}[nosep]
	\item Arbitrarily chosen SI units
	\item The natural structure
\end{itemize}

\textbf{In natural units:} $\alpha = 1$

The puzzle disappears!

\subsection{The Real Question}

\textbf{Wrong question:} Why is $\alpha = 1/137.035999084...$?

\textbf{Right question:} Which \textit{ratios} (mass ratios, geometric factors) are fundamental?

\begin{important}
	Science stared at the \textit{wrong} number for 100 years!
	
	While everyone was fixated on $\alpha = 1/137$, the following were overlooked:
	\begin{itemize}[nosep]
		\item Mass ratios ($m_\tau/m_\mu = 16.8$)
		\item Geometric factors ($r$, $p$, $\xi$)
		\item Fractal structure ($D_f$)
		\item Torus topology
	\end{itemize}
\end{important}

\subsection{The Standard Model Problem}

The Standard Model has 19 free parameters:
\begin{itemize}[nosep]
	\item 3 coupling constants ($\alpha$, $\alpha_s$, $\alpha_w$)
	\item 6 quark masses
	\item 3 lepton masses
	\item 4 CKM parameters
	\item 3 neutrino masses
\end{itemize}

Everyone tries to explain $\alpha$, but \textbf{ignores} the 17 mass ratios!

\textbf{T0 approach:}
\begin{itemize}[nosep]
	\item Ratios from geometry
	\item $m_\tau/m_\mu$, $m_\mu/m_e$, $a_\tau/a_\mu$
	\item $\alpha$ is a conversion factor
\end{itemize}

\section{The Ultimate Truth: Only Relations are Measurable}

\subsection{The Fundamental Principle}

\begin{theorem}[Fundamental Measurement Principle]
	\textbf{Every measurement is, in principle, a comparison.}
	
	One \textit{cannot} measure:
	\begin{itemize}[nosep]
		\item One kilogram (absolutely)
		\item One meter (absolutely)
		\item One second (absolutely)
	\end{itemize}
	
	One \textit{can} measure:
	\begin{itemize}[nosep]
		\item Mass A / Mass B
		\item Length A / Length B
		\item Time A / Time B
	\end{itemize}
	
	\textbf{All measurements are ratios!}
\end{theorem}

\subsection{Practical Examples}

\subsubsection{Length Measurement}

\textbf{Historical (Prototype Meter):}
One compares with the prototype meter in Paris:
\begin{equation}
	\frac{L_\text{object}}{L_\text{prototype meter}} = ?
\end{equation}

\textbf{Modern (Speed of Light):}
One measures the light travel time, but $c$ is \textit{defined} as 299,792,458 m/s. So one measures:
\begin{equation}
	\frac{t_\text{object}}{t_\text{standard}} = ?
\end{equation}

\textbf{Always a ratio!}

\subsubsection{Mass Measurement}

\textbf{Balance:}
\begin{equation}
	\frac{m_\text{object}}{m_\text{calibration weight}} = ?
\end{equation}

\textbf{Mass Spectrometer:}
\begin{equation}
	\frac{m}{q} = \text{(ratio)}
\end{equation}

\textbf{Modern Definition (Planck Constant):}
1 kg is defined via $\hbar = 6.62607015 \times 10^{-34}$ kg$\cdot$m$^2$/s. But that \textit{is} a relation!

\textbf{Always a ratio!}

\subsubsection{Time Measurement: The Atomic Clock Paradox}

The atomic clock measures Cs-133 hyperfine transitions:
\begin{equation}
	N_\text{oscillations} = ?
\end{equation}

What does it \textit{really} measure?

The \textbf{frequency:}
\begin{equation}
	f = \frac{\Delta E}{h}
\end{equation}

where $\Delta E$ = energy difference between states.

\textbf{The clock measures a ratio: $E/h$}

\begin{critical}
	\textbf{The atomic clock does not know whether mass or time is changing!}
	
	If there is a change in:
	\begin{itemize}[nosep]
		\item $m_e$ $\Rightarrow$ $\Delta E$ changes $\Rightarrow$ $f$ changes
		\item $h$ $\Rightarrow$ $f$ changes
		\item Time $\Rightarrow$ ??? (What is absolute time?)
	\end{itemize}
	
	\textbf{The clock cannot distinguish!}
\end{critical}

\subsection{Philosophical Consequence}

We can only measure ratios, \textbf{not} because we are not clever enough, but because it is \textbf{in principle impossible}!

\textbf{Reason:}
\begin{itemize}[nosep]
	\item Every measurement requires a standard
	\item The standard is part of nature
	\item If \textit{everything} changes proportionally, we cannot detect it
\end{itemize}

\subsection{Thought Experiments}

\textbf{Scenario 1: Time Slows Down}

Suppose true time slows down:
\begin{equation}
	t_\text{true}(\text{today}) = 0.9 \times t_\text{true}(\text{yesterday})
\end{equation}

\textit{Question:} Would the atomic clock notice?

\textit{Answer:} \textbf{No!} The Cs atoms still oscillate the same \textit{relative} to their internal dynamics. The clock shows normal time.

\textbf{We cannot detect the slowdown!}

\textbf{Scenario 2: All Masses Double}

Suppose:
\begin{equation}
	m(\text{today}) = 2 \times m(\text{yesterday})
\end{equation}

\textit{Question:} Would our balance notice?

\textit{Answer:} \textbf{No!} The calibration weight also doubles. The balance shows:
\begin{equation}
	\frac{m_\text{object}}{m_\text{calibration weight}} = \text{unchanged}
\end{equation}

\textbf{We cannot detect the change!}

\textbf{Scenario 3: Speed of Light Doubles}

Suppose:
\begin{equation}
	c(\text{today}) = 2 \times c(\text{yesterday})
\end{equation}

\textit{Question:} Would we notice?

\textit{Answer:} \textbf{No!} We have \textit{defined} $c = 299,792,458$ m/s. If $c$ changes, our meters change.

\textbf{We cannot detect the change!}

\section{Consequences for the T0 Theory}

\subsection{Time-Mass Duality and Measurability}

In the T0 theory:
\begin{equation}
	T(x) \cdot m(x) = 1
\end{equation}

\textbf{Question:} What does this mean for measurements?

\textbf{Answer:} We \textit{cannot} distinguish:
\begin{itemize}[nosep]
	\item Mass changes (at fixed time)
	\item Time changes (at fixed mass)
\end{itemize}

\textbf{Both interpretations are equivalent!}

What we measure is the \textit{product}:
\begin{equation}
	T \times m = \text{constant}
\end{equation}

\textbf{That is the ratio!}

\subsection{Why a Ratio-Based Formulation is Necessary}

The ratio-based formulation of the T0 theory is \textbf{not} merely elegant or practical, but \textbf{compulsory} because:

\begin{enumerate}
	\item All measurements are ratios (in principle)
	\item Absolute values are definitions (arbitrary)
	\item Nature knows only ratios (fundamental)
\end{enumerate}

\textbf{T0 predicts:}
\begin{equation}
	\frac{a_\tau}{a_\mu} = \left(\frac{m_\tau}{m_\mu}\right)^2 = 283
\end{equation}

This \textit{is} measurable, because:
\begin{itemize}[nosep]
	\item One measures frequencies in a Penning trap
	\item One calculates the ratio
	\item \textbf{No} absolute energy needed!
\end{itemize}

\textbf{T0 does not predict:}
\begin{equation}
	a_\mu = 37.5 \times 10^{-11} \quad \text{(absolutely)}
\end{equation}

Because that would require:
\begin{itemize}[nosep]
	\item Definition of a unit
	\item Conversion via $\alpha$, $\hbar$, $c$
	\item Arbitrary conventions
\end{itemize}

\subsection{The Fractal Correction $K_\text{frak}$}

A common misunderstanding is that one would need to calculate $K_\text{frak}$ exactly. But:

\begin{important}
	An exact derivation of $K_\text{frak}$ is \textbf{not necessary}, because:
	\begin{enumerate}
		\item Measurement uncertainty dominates ($\pm 17\%$ for $\Delta a_\mu$)
		\item Phenomenology is legitimate (like QCD hadronic contributions)
		\item $K_\text{frak}$ cancels in ratios
	\end{enumerate}
\end{important}

Rounding errors ($\sim 10^{-15}$) vs. measurement errors ($\sim 10^{-1}$) show: Numerical precision is \textbf{irrelevant} compared to experimental uncertainties.

\subsection{SI Units and Fractal Correction}

A deep question is: Do SI units already contain $K_\text{frak}$?

\textbf{Answer:} Presumably yes.

SI measurements measure the \textit{real} world:
\begin{itemize}[nosep]
	\item Space is fractal ($D_f = 3 - \xi$)
	\item All measurements occur in this space
	\item Mass integrals: $m \propto \int \rho(r) r^{D_f-1} \, dr$
\end{itemize}

Therefore:
\begin{equation}
	m_\mu[\text{SI measured}] = \tilde{m}_\mu[\text{ideal}] \times K_\text{frak}
\end{equation}

\textbf{But:} For ratios, it doesn't matter!
\begin{equation}
	\frac{m_\tau[\text{SI}]}{m_\mu[\text{SI}]} = \frac{\tilde{m}_\tau \times K_\text{frak}}{\tilde{m}_\mu \times K_\text{frak}} = \frac{\tilde{m}_\tau}{\tilde{m}_\mu}
\end{equation}

\textbf{$K_\text{frak}$ cancels!}

\section{Extended Mach Principle}

\subsection{Classical Mach Principle}

Ernst Mach (1893):
\begin{quote}
	\textit{Absolute motion is meaningless. Only relative motion is measurable.}
\end{quote}

\subsection{Extension by T0}

\begin{theorem}[Extended Mach Principle]
	\textbf{Absolute mass is meaningless.} \\
	\textbf{Absolute time is meaningless.} \\
	\textbf{Absolute charge is meaningless.} \\
	\textbf{Only ratios are measurable.}
\end{theorem}

This is not philosophy, but \textbf{operative reality}!

\subsection{Practical Consequence}

If someone asks: Has the speed of light changed?

\textbf{Answer:} The question is meaningless!

\textbf{Because:}
\begin{itemize}[nosep]
	\item $c$ is \textit{defined} as 299,792,458 m/s
	\item The meter is defined by $c$
	\item Circular!
\end{itemize}

\textbf{The right question:} Has $c/\alpha$ changed? or Has $c$ changed relative to atomic scales?

$\Rightarrow$ \textbf{Ratios} are the only meaningful questions!

\section{Summary: The Fundamental Insights}

\subsection{Seven Pillars of Truth}

\begin{enumerate}
	\item \textbf{Ratios are fundamental} \\
	Not absolute values, but ratios are the language of nature
	
	\item \textbf{All measurements are relations} \\
	In principle, not just in practice
	
	\item \textbf{Absolute values are conventions} \\
	kg, m, s are arbitrarily defined
	
	\item \textbf{$\alpha = 1/137$ was a distraction} \\
	100 years focused on the wrong question
	
	\item \textbf{Universal corrections cancel} \\
	$K_\text{frak}$, $K_\text{mode}$, $K_\text{quant}$ in ratios
	
	\item \textbf{Atomic clocks measure ratios} \\
	$f = \Delta E / h$, not absolute time
	
	\item \textbf{Time-mass duality is measurable as a product} \\
	$T \times m = \text{constant}$, individual quantities are conventions
\end{enumerate}

\subsection{From the Tonnetz to TOE}

The journey of 40 years:

\begin{center}
	\begin{tabular}{rcl}
		\textbf{~1985} & $\longrightarrow$ & Euler's Tonnetz \\
		&  & Intervals are fundamental \\
		&  & \\
		\textbf{~2000} & $\longrightarrow$ & Transfer to physics \\
		&  & Are ratios also fundamental here? \\
		&  & \\
		\textbf{~2020} & $\longrightarrow$ & T0 theory developed \\
		&  & $m = r \times \xi^p$ (like intervals!) \\
		&  & \\
		\textbf{2026} & $\longrightarrow$ & Insight completes \\
		&  & Ratios \textit{are} fundamental -- \\
		&  & as in the Tonnetz 40 years ago! \\
	\end{tabular}
\end{center}

\subsection{The Revolutionary Consequence}

\vspace{0.5cm}
\noindent
\begin{minipage}[t]{0.48\textwidth}
	\textbf{Standard Physics:}
	\begin{itemize}[nosep]
		\item We measure absolute quantities
		\item Why is $\alpha = 1/137$?
		\item $c$, $\hbar$, $e$ are constants of nature
		\item 19 free parameters in the SM
		\item $\alpha$ is explained
		\item Mass ratios ignored
	\end{itemize}
\end{minipage}
\hfill
\begin{minipage}[t]{0.48\textwidth}
	\textbf{T0/Ratios:}
	\begin{itemize}[nosep]
		\item We measure ONLY ratios
		\item Why is $m_\tau/m_\mu = 16.8$?
		\item Those are just conventions!
		\item Ratios from geometry
		\item $\alpha$ is a conversion factor
		\item Ratios are fundamental
	\end{itemize}
\end{minipage}
\vspace{0.5cm}

\section{Outlook: The True Constants}

\subsection{What are the True Constants?}

\textbf{Not:}
\begin{itemize}[nosep]
	\item $c = 299,792,458$ m/s (definition)
	\item $\hbar = 6.626 \times 10^{-34}$ J$\cdot$s (definition)
	\item $\alpha = 1/137$ (conversion factor)
	\item $m_\mu = 105.658$ MeV (relative to unit)
\end{itemize}

\textbf{But:}
\begin{itemize}[nosep]
	\item $m_\tau / m_\mu = 16.817$ (dimensionless, fundamental)
	\item $m_\mu / m_e = 206.768$ (dimensionless, fundamental)
	\item $a_\tau / a_\mu = 283$ (dimensionless, testable)
	\item $\xi = 4/(3 \times 10^4)$ (geometric factor)
	\item $r_e = 4/3$, $r_\mu = 16/5$, $r_\tau = 8/3$ (geometric ratios)
\end{itemize}

\subsection{The Analogy to Music (Final)}

\begin{table}[h]
	\centering
	\begin{tabular}{p{0.3\textwidth}p{0.3\textwidth}p{0.3\textwidth}}
		\toprule
		\textbf{Question} & \textbf{Music} & \textbf{Physics} \\
		\midrule
		What is fundamental? & Intervals (2:1, 3:2) & Ratios ($m_\tau/m_\mu$) \\
		What is arbitrary? & 440 Hz & 105.658 MeV \\
		What does one hear/measure? & Ratios & Ratios \\
		What is A4? & Definition & Convention \\
		What is 1 kg? & -- & Convention \\
		\bottomrule
	\end{tabular}
	\caption{The Fundamental Parallel}
\end{table}

\begin{keypoint}
	The ear hears intervals, not absolute frequencies. \\
	Nature knows ratios, not absolute values.
	
	\textbf{The harmony lies in the ratios -- in music AND physics!}
\end{keypoint}

\subsection{The Test: Belle II (2027-2028)}

The fundamental prediction:
\begin{equation}
	\boxed{\frac{a_\tau}{a_\mu} = \left(\frac{m_\tau}{m_\mu}\right)^2 = 283}
\end{equation}

This is:
\begin{itemize}[nosep]
	\item A \textbf{ratio} (fundamentally measurable)
	\item \textbf{Independent} of $\alpha$, $\hbar$, $c$, $K_\text{frak}$
	\item \textbf{Testable} at Belle II
	\item The \textbf{right} kind of prediction
\end{itemize}

If confirmed: 40 years from the Tonnetz to TOE!

\section{Conclusion}

\begin{tcolorbox}[colback=blue!5!white,colframe=blue!75!black,title=\textbf{Conclusion}]
	The simplicity of ratios in the T0 theory is \textbf{not a coincidence}, but a hint at a profound truth:
	
	\textbf{Ratios are the fundamental language of nature.}
	
	This insight:
	\begin{itemize}[nosep]
		\item Explains why ratios are simple despite a complex world
		\item Shows that $\alpha = 1/137$ was a century-long distraction
		\item Proves that only relations are measurable in principle
		\item Extends Mach's principle to mass and time
		\item Justifies the ratio-based T0 formulation
		\item Closes the circle from the Tonnetz to physics
	\end{itemize}
	
	\vspace{0.5cm}
	
	Science asked for 100 years: Why 137?
	
	The right question is: Why $m_\tau/m_\mu = 16.8$?
	
	\vspace{0.5cm}
	
	\textbf{From the C major chord (C:E:G = 4:5:6) to the lepton triplet (e:$\mu$:$\tau$).}
	
	\textbf{The same structure, the same beauty, the same truth.}
\end{tcolorbox}