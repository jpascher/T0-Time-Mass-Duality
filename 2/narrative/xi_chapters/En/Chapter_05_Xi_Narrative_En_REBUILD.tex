% Chapter 05: Predictions and Experimental Tests
% Completely revised with ratio-based formulation
% Status: January 2026

\chapter{Predictions and Experimental Tests}

\section{Introduction}

A physical theory demonstrates its strength through testable predictions. FFGFT provides predictions for a wide range of experiments. We distinguish between:

\begin{itemize}
	\item \textbf{Fundamental predictions:} Ratios that are independent of unit systems and fractal corrections
	\item \textbf{Phenomenological predictions:} Absolute values in SI units, which require conversion factors
\end{itemize}

\section{Anomalous Magnetic Moments of Leptons}

\subsection{Fundamental Prediction: The Ratio}

T0 theory provides a \textbf{fundamental, parameter-free} prediction for the ratio of anomalous magnetic moments:

\begin{equation}
	\boxed{\frac{a_\tau}{a_\mu} = \left(\frac{m_\tau}{m_\mu}\right)^2}
	\label{eq:g2_ratio_fundamental}
\end{equation}

This prediction is:
\begin{itemize}
	\item \textbf{System-of-units independent:} Valid in natural and SI units
	\item \textbf{Correction independent:} Fractal corrections cancel out
	\item \textbf{Parameter-free:} Only mass ratios, no fitting parameters
	\item \textbf{Pure geometry:} Follows directly from quadratic mass scaling
\end{itemize}

\subsection{Numerical Evaluation}

With the measured lepton masses:
\begin{align}
	m_\mu &= 105.658\,\text{MeV} \\
	m_\tau &= 1776.86\,\text{MeV}
\end{align}

we obtain the ratio:
\begin{equation}
	\frac{m_\tau}{m_\mu} = 16.818 \quad \Rightarrow \quad 
	\frac{a_\tau}{a_\mu} = (16.818)^2 = 282.8
\end{equation}

\subsection{Experimental Status (January 2026)}

\textbf{Muon:} Fermilab final measurement (June 2025)
\begin{align}
	a_\mu^{\text{exp}} &= 116\,592\,070.5(11.4) \times 10^{-11} \\
	a_\mu^{\text{SM}} &= 116\,592\,033(62) \times 10^{-11} \\
	\Delta a_\mu &= 37.5(6.3) \times 10^{-11}
\end{align}

The discrepancy has been reduced from $5\sigma$ (2023) to $0.6\sigma$ (2025) through improved lattice QCD calculations.

\textbf{Tau:} Only upper limit known
\begin{equation}
	|a_\tau| < 9.5 \times 10^{-3} \quad \text{(DELPHI 2004)}
\end{equation}

Belle II expects sensitivity $\sim 10^{-7}$ by 2027-2028.

\subsection{T0 Prediction for Belle II}

From the fundamental ratio it follows:
\begin{equation}
	a_\tau^{\text{T0}} = 282.8 \times \Delta a_\mu = 282.8 \times 37.5 \times 10^{-11} 
	\approx 1.06 \times 10^{-7}
\end{equation}

\textbf{Test of the theory:}
\begin{itemize}
	\item \textbf{If confirmed} ($a_\tau \approx 10^{-7}$): Strong evidence for quadratic mass scaling
	\item \textbf{If contradicted:} The assumption $a_\tau/a_\mu = (m_\tau/m_\mu)^2$ must be revised
	\item \textbf{If null result} ($a_\tau < 10^{-8}$): T0 contributions are suppressed
\end{itemize}

\subsection{Electron g-2: Why No T0 Contributions?}

For the electron, T0 theory predicts:
\begin{equation}
	\frac{a_e^{\text{T0}}}{a_\mu^{\text{T0}}} = \left(\frac{m_e}{m_\mu}\right)^2 
	= (0.00484)^2 \approx 2.3 \times 10^{-5}
\end{equation}

If $\Delta a_\mu^{\text{T0}} \approx 37.5 \times 10^{-11}$, then:
\begin{equation}
	\Delta a_e^{\text{T0}} \approx 37.5 \times 10^{-11} \times 2.3 \times 10^{-5} 
	\approx 8.6 \times 10^{-15}
\end{equation}

This is far below the experimental precision ($\sim 10^{-13}$). The Standard Model explains electron g-2 perfectly at the ppb level.

\begin{keypoint}[Why the Quadratic Suppression?]
	Time-mass duality leads to a coupling $\propto m^2$. For the electron, this means an additional suppression of $(m_e/m_\mu)^4 \approx 5 \times 10^{-10}$ compared to the muon. T0 effects are only relevant for heavy leptons.
\end{keypoint}

\subsection{Philosophical Remark}

The T0 prediction is \textbf{not}:
\begin{itemize}
	\item[$\times$] "$a_\mu = 37.5 \times 10^{-11}$" (SI-dependent, phenomenological)
	\item[$\times$] An ab-initio calculation of absolute values
\end{itemize}

The T0 prediction \textbf{is}:
\begin{itemize}
	\item[\checkmark] "$a_\tau/a_\mu = (m_\tau/m_\mu)^2$" (fundamental, SI-independent)
	\item[\checkmark] A structural statement about ratios
	\item[\checkmark] Testable without knowledge of absolute values
\end{itemize}

\section{Further Testable Predictions}

\subsection{Lepton Mass Ratios}

T0 theory predicts mass ratios from geometric factors:
\begin{align}
	\frac{m_\mu}{m_e} &= \frac{r_\mu}{r_e} \xi^{p_\mu - p_e} = \frac{16/5}{4/3} \xi^{-1/2} 
	\approx 207 \quad \checkmark \\
	\frac{m_\tau}{m_\mu} &= \frac{r_\tau}{r_\mu} \xi^{p_\tau - p_\mu} = \frac{8/3}{16/5} \xi^{-1/3} 
	\approx 16.8 \quad \checkmark
\end{align}

These are \textbf{genuine predictions}, since $(r,p)$ are systematically derived from quantum numbers, not fitted.

\subsection{Fine-Structure Constant (Ratio Statement)}

T0 theory does not make a statement about the absolute value $\alpha = 1/137$ (this is an SI conversion factor). But it predicts a \textbf{structural relation}:

In natural units:
\begin{equation}
	\tilde{\alpha} = \xi \times \tilde{E}_0^2 = 1 \quad \text{(normalized)}
\end{equation}

The transformation to SI units is phenomenological.

\subsection{Spectroscopic Tests}

\subsubsection{Hydrogen Spectrum}

T0 corrections to hydrogen energy levels are extremely small:
\begin{equation}
	\Delta E_n^{\text{T0}} \approx \xi \frac{E_n^2}{E_{\text{Planck}}} 
	\approx 10^{-31}\,\text{eV}
\end{equation}

This is below current precision but in principle accessible with ultra-precision spectroscopy.

\subsubsection{Rydberg Atoms}

For highly excited states ($n \gg 1$), the fractal damping becomes relevant:
\begin{equation}
	\frac{E_n^{\text{Rydberg}}}{E_n^{\text{Bohr}}} = \exp\left(-\xi \frac{n^2}{D_f}\right)
\end{equation}

where $D_f = 3 - \xi$. This is a ratio statement and thus independent of SI units.

\section{Quantum Entanglement}

\subsection{T0-Modified Bell Correlation}

T0 theory modifies the correlation function of entangled particles:
\begin{equation}
	E(a,b)^{\text{T0}} = E(a,b)^{\text{QM}} \times \left(1 - \xi \cdot f(n,l,j)\right)
\end{equation}

This leads to a slight reduction of the CHSH violation. The \textbf{ratio}:
\begin{equation}
	\frac{S_{\text{CHSH}}^{\text{T0}}}{S_{\text{CHSH}}^{\text{QM}}} = 1 - \xi \cdot g(n) 
	\approx 0.9999
\end{equation}

is again a fundamental statement.

\section{Cosmological Implications}

\subsection{Redshift Relation}

T0 theory modifies the interpretation of cosmological redshift. In a static universe with fractal structure:

\begin{equation}
	\frac{\lambda_{\text{observed}}}{\lambda_{\text{emitted}}} = 1 + \xi \cdot f(d,t)
\end{equation}

where $d$ is the distance and $t$ is the light travel time.

\subsection{JWST Observations}

The James Webb Space Telescope observations (2024-2025) show evolved galaxies at high redshifts ($z > 10$). This is more consistent with a static T0 universe than with $\Lambda$CDM, where these structures did not have enough time to evolve.

This is a qualitative, not quantitative, prediction.

\section{Summary of Tests}

\begin{table}[h]
	\centering
	\caption{T0 Predictions by Type}
	\begin{tabularx}{\textwidth}{|X|X|X|X|}
		\hline
		\textbf{Observable} & \textbf{Type} & \textbf{T0 Prediction} & \textbf{Status} \\
		\hline
		$a_\tau/a_\mu$ & Fundamental & $(m_\tau/m_\mu)^2 = 283$ & Belle II 2027-28 \\
		\hline
		$m_\tau/m_\mu$ & Fundamental & $16.8$ (from $r,p$) & Confirmed \checkmark \\
		\hline
		$m_\mu/m_e$ & Fundamental & $207$ (from $r,p$) & Confirmed \checkmark \\
		\hline
		CHSH ratio & Fundamental & $\approx 0.9999$ & 73-Qubit tests \\
		\hline
		$\Delta a_\mu$ absolute & Phenomenol. & Normalization needed & 37.5 × 10⁻¹¹ \\
		\hline
		H spectrum & Phenomenol. & $10^{-31}$ eV & Ultra-precision \\
		\hline
		JWST z>10 & Qualitative & Static universe & Supported \\
		\hline
	\end{tabularx}
\end{table}

\section{Future Experiments}

\subsection{Priority 1: Belle II Tau g-2 (2027-2028)}

This is the \textbf{most critical test} of T0 theory:
\begin{itemize}
	\item Test of the fundamental prediction $a_\tau/a_\mu = 283$
	\item Independent of phenomenological parameters
	\item Direct test of quadratic mass scaling
	\item If contradictory: T0 theory must be revised
\end{itemize}

\subsection{Priority 2: High-Precision Mass Ratios}

\begin{itemize}
	\item More precise measurement of $m_\tau/m_\mu$ and $m_\mu/m_e$
	\item Test whether $(r,p)$ values are exactly rational
	\item Search for generation-dependent corrections
\end{itemize}

\subsection{Priority 3: Fundamental Constant Ratios}

\begin{itemize}
	\item Test whether $\alpha/\alpha_G$ (electromagnetic/gravitational) is determined by $\xi$
	\item Search for time variation of ratios (should be zero in T0)
	\item Comparison of different methods for $\xi$ determination
\end{itemize}

\begin{keypoint}[Experimental Strategy]
	T0 theory should primarily be tested through \textbf{ratio measurements}, not through absolute values. Ratios are fundamental, SI-independent, and free from conversion factors. The Belle II test of $a_\tau/a_\mu$ is the clearest and most direct test of the core statements of the theory.
\end{keypoint}

\section{Limits of Predictive Power}

\subsection{What T0 Theory Does NOT Predict}

\begin{itemize}
	\item \textbf{Absolute values in SI:} These require conversion factors that are phenomenological (e.g., $\alpha = 1/137$, $v = 246$ GeV)
	
	\item \textbf{Absolute g-2 values:} $a_\mu = 37.5 \times 10^{-11}$ cannot be calculated ab initio, only ratios
	
	\item \textbf{Quantitative QCD effects:} Hadronic physics is too complex for ab-initio calculation (as in SM)
\end{itemize}

\subsection{What T0 Theory Predicts}

\begin{itemize}
	\item \textbf{Ratios:} $m_\tau/m_\mu$, $a_\tau/a_\mu$, etc. from geometric factors
	
	\item \textbf{Structural relations:} Quadratic mass scaling, fractal damping
	
	\item \textbf{Qualitative effects:} Direction of corrections, orders of magnitude
\end{itemize}

This is analogous to the Standard Model: There, too, one cannot calculate, e.g., quark mass ratios ab initio, but one can calculate their electroweak couplings.

T0 theory goes one step further: It derives mass ratios from geometry – but absolute values remain phenomenological.