% Chapter 03: Time-Mass Duality in Quantum Mechanics and Field Theory
% Revised with ratio-based philosophy
% Status: January 2026

\chapter{Time-Mass Duality in Quantum Mechanics and Field Theory}

\section{Introduction}

In previous chapters, geometry has been at the forefront: the number $\xipar$, the fractal dimension $D_f$, and the resulting scales. We now apply this structure to the familiar equations of quantum mechanics and quantum field theory.

\section{Schrödinger Equation as an Effective Description}

In standard formulation, the time-dependent Schrödinger equation

\begin{equation}
	i\hbar \frac{\partial}{\partial t} \psi(t,\vec{x}) = \hat{H} \psi(t,\vec{x})
	\label{eq:schroedinger}
\end{equation}

describes the evolution of a wavefunction $\psi$ under a Hamiltonian $\hat{H}$. This equation is already deterministic: from a given initial state, the future follows uniquely. The apparent randomness enters the theory only through the measurement postulate and the interpretation of $|\psi|^2$ as a probability density.

\subsection{T0 Interpretation}

Within the framework of time-mass duality, the Schrödinger equation is understood as an effective description of a deeper, geometric dynamics. Simplifying, $\psi$ does not describe a mysterious "field of possibilities," but a statistical projection of the underlying fractal time structure.

The parameters in the Hamiltonian – particularly masses and coupling strengths – are not fundamental in FFGFT, but are determined by $\xipar$ and the resulting scales.

\section{From Schrödinger to Dirac}

For relativistic particles with spin, the Schrödinger equation is insufficient. There, the Dirac equation appears:

\begin{equation}
	(i\gamma^\mu \partial_\mu - m)\psi = 0
	\label{eq:dirac}
\end{equation}

with Dirac matrices $\gamma^\mu$ and mass $m$. In FFGFT, $m$ is not considered an input parameter but a derived quantity from time-mass duality:

\begin{equation}
	T(x,t) \cdot m(x,t) = 1
	\label{eq:time_mass_duality_field}
\end{equation}

\subsection{Geometric Interpretation}

This changes the interpretation of the Dirac equation: It is not the fundamental equation but an effective field equation on a background whose geometry is already fixed by $\xipar$.

The known properties – spin, antimatter, zitterbewegung – remain preserved but receive a geometric interpretation within the framework of fractal spacetime.

\subsection{Simplified Interpretation: Clifford Algebra Instead of 4×4 Matrices}

The traditional Dirac equation uses complex 4×4 matrices ($\gamma^\mu$) and abstract spinors ($\psi$). This matrix representation, however, is not the fundamental physics but only a **specific representation**.

\textbf{Fundamental structure without explicit matrices:}

The Dirac equation is actually a Clifford algebra equation:
\begin{equation}
	(i \mathbf{e}_\mu \partial^\mu - m)\Psi = 0
	\label{eq:clifford_dirac}
\end{equation}

where:
\begin{itemize}
	\item $\mathbf{e}_\mu$: Abstract basis vectors of spacetime (not matrices!)
	\item $\Psi$: Element in spin space (geometric object)
	\item The algebra rule: $\mathbf{e}_\mu \mathbf{e}_\nu + \mathbf{e}_\nu \mathbf{e}_\mu = 2g_{\mu\nu}$
\end{itemize}

\textbf{In T0 theory:}

Within the framework of fractal spacetime, this becomes:
\begin{equation}
	(i \partial\!\!\!/_{\text{frak}} - m(x))\Psi(x) = 0
	\label{eq:t0_dirac}
\end{equation}

with:
\begin{itemize}
	\item $\partial\!\!\!/_{\text{frak}}$: Differential operator in fractal geometry ($D_f = 3 - \xi$)
	\item $m(x) = 1/(c^2 T(x))$: Time-dependent mass from time-mass duality
	\item $\Psi(x)$: Spinor field in the spin bundle over the fractal manifold
\end{itemize}

\textbf{Spin as geometric property:}

The spin-1/2 character is not a matrix property but:
\begin{itemize}
	\item A **topological winding number** on the torus
	\item A **geometric property** of the solutions
	\item $\Psi$ transforms into itself under 720° rotation (not 360°)
	\item This follows from the Clifford algebra structure, not from matrices
\end{itemize}

\begin{important}{Fundamental vs. Representation Level}
	The 4×4 matrices ($\gamma^\mu$) are a **calculation tool**, not the fundamental physics. The physics is:
	\begin{enumerate}
		\item Clifford algebra structure of spacetime
		\item Spin as topological/geometric property
		\item Time-mass duality: $m(x) = 1/(c^2 T(x))$
	\end{enumerate}
	
	In T0 theory, the $\gamma^\mu$ represent the **geometric structure of the fractal space** with $D_f = 3 - \xi$, not abstract algebraic objects.
	
	For calculations, one can use the standard matrix representation, but the **interpretation** is geometric: the spinor structure follows from torus topology, not from arbitrary matrices.
\end{important}

\textbf{Comparison of formulations:}

\begin{center}
	\begin{tabularx}{\textwidth}{>{\raggedright\arraybackslash}X >{\raggedright\arraybackslash}X >{\raggedright\arraybackslash}X}
		\toprule
		\textbf{Aspect} & \textbf{Matrix Representation} & \textbf{Geometric Clifford Form} \\
		\midrule
		Mathematics & 4×4 matrices & Clifford algebra \\
		Spin & Encoded in matrices & Topological property \\
		Lorentz invariance & Explicit in matrices & In algebra structure \\
		T0 integration & Difficult & Natural (fractal geometry) \\
		Status & Representation & Fundamental \\
		\bottomrule
	\end{tabularx}
\end{center}

\vspace{0.5cm}

This geometric formulation is not only pedagogical but shows the **fundamental nature** of the Dirac equation as a statement about the geometric structure of spacetime.

\section{Lagrangian Density and Role of $\xipar$}

\subsection{Extended Lagrangian with Time Field}

The complete T0 formulation uses an extended Lagrangian containing the dynamic time field $T(x,t)$ or equivalently the mass variation $\Delta m$:

\[
\begin{aligned}
	\mathcal{L}_{\text{extended}} = 
	&-\frac{1}{4}F_{\mu\nu}F^{\mu\nu} 
	+ \bar{\psi}(i\gamma^\mu D_\mu - m)\psi \\
	&+ \frac{1}{2}(\partial_\mu \Delta m)(\partial^\mu \Delta m) 
	- \frac{1}{2}m_T^2 \Delta m^2 \\
	&+ \xi_{\text{par}} \, m_\ell \, \bar{\psi}_\ell \psi_\ell \, \Delta m
	\label{eq:lagrangian_extended}
\end{aligned}
\]

where:
\begin{itemize}
	\item $F_{\mu\nu}$: Electromagnetic field strength tensor
	\item $\psi$: Fermion field (leptons/quarks)
	\item $\Delta m$: Dynamic mass variation (time field)
	\item $m_T$: Characteristic mass of the time field
	\item $\xipar m_\ell$: Fundamental coupling strength
\end{itemize}

\subsection{Mass-Proportional Coupling}

The coupling of lepton fields $\psi_\ell$ to the time field occurs proportionally to the lepton mass:

\begin{align}
	\mathcal{L}_{\text{interaction}} &= g_T^\ell \bar{\psi}_\ell \psi_\ell \Delta m \label{eq:interaction}\\
	g_T^\ell &= \xipar m_\ell \label{eq:coupling}
\end{align}

This mass-proportional coupling is central to T0 structure and leads directly to quadratic mass scaling.

\section{Structure of T0 Contributions}

\subsection{One-Loop Diagram}

From the interaction term $\mathcal{L}_{\text{int}} = \xipar m_\ell \bar{\psi}_\ell \psi_\ell \Delta m$, a one-loop contribution to the anomalous magnetic moment follows.

The general expression is:

\begin{equation}
	\Delta a_\ell \propto \frac{(g_T^\ell)^2 \cdot m_\ell^2}{m_T^2} 
	= \frac{\xipar^2 m_\ell^4}{m_T^2}
	\label{eq:one_loop_structure}
\end{equation}

\subsection{Fundamental Structural Statement}

The essential statement of T0 theory is the **scaling**:

\begin{equation}
	\boxed{\Delta a_\ell \propto m_\ell^2}
	\label{eq:t0_scaling}
\end{equation}

This leads to the fundamental ratio prediction:

\begin{equation}
	\boxed{\frac{\Delta a_{\ell_1}}{\Delta a_{\ell_2}} = \left(\frac{m_{\ell_1}}{m_{\ell_2}}\right)^2}
	\label{eq:t0_ratio}
\end{equation}

This prediction is:
\begin{itemize}
	\item **System-of-units independent:** Ratios are invariant
	\item **Correction independent:** Fractal corrections cancel out
	\item **Parameter-free:** Only mass ratios
	\item **Pure geometry:** Follows directly from $g_T \propto m$
\end{itemize}

\section{Predictions for Leptons}

\subsection{Fundamental Ratio Prediction}

With the measured lepton masses it follows:

\begin{align}
	\frac{m_\mu}{m_e} &= \frac{105.658}{0.511} \approx 207 \quad \Rightarrow \quad 
	\frac{\Delta a_\mu}{\Delta a_e} \approx 42800 \\
	\frac{m_\tau}{m_\mu} &= \frac{1776.86}{105.658} \approx 16.8 \quad \Rightarrow \quad 
	\frac{\Delta a_\tau}{\Delta a_\mu} \approx 283
\end{align}

\subsection{Interpretation of Scaling}

The quadratic mass scaling $\Delta a \propto m^2$ means:
\begin{itemize}
	\item Heavier leptons have **quadratically** larger T0 contributions
	\item The ratio is **independent** of systems of units
	\item The ratio is **independent** of fractal corrections
	\item Pure **geometric** statement from the coupling structure
\end{itemize}

Detailed experimental comparisons and measurements are treated in Chapter 5 (Predictions and Experimental Tests).

\section{Limits of the Theory}

\subsection{What T0 Theory Does NOT Provide at This Level}

From Lagrangian~\eqref{eq:lagrangian_extended} follows the **structure** $\Delta a \propto m^2$, but **not** the absolute value without further assumptions:

\begin{itemize}
	\item The mass $m_T$ of the time field mediator is not calculable ab initio
	\item Complete calculation of loop integrals in fractal spacetime ($D_f = 3 - \xi$) is extremely complex
	\item Recursive interactions between time field, Higgs, and other fields are difficult to handle
	\item Renormalization in non-integer dimension is not yet fully developed
\end{itemize}

\subsection{Analogy to the Standard Model}

This is analogous to the situation in the Standard Model:
\begin{itemize}
	\item SM defines the QCD Lagrangian density
	\item But hadronic contributions to g-2 are not calculable ab initio
	\item Phenomenological methods are used (dispersion relations, lattice)
	\item The **structure** is clear, the **amplitude** is phenomenological
\end{itemize}

\subsection{What T0 Theory Provides}

\begin{itemize}
	\item **Structural statement:** $\Delta a \propto m^2$ (quadratic scaling)
	\item **Ratio prediction:** $\Delta a_\tau / \Delta a_\mu = (m_\tau/m_\mu)^2$
	\item **Qualitative explanation:** Why heavier leptons have larger contributions
	\item **Testable prediction:** Belle II can test the quadratic scaling
\end{itemize}

\section{Phenomenological Formulation}

\subsection{Normalization at the Muon}

If one wants to calculate absolute SI values, one normalizes at the muon:

\begin{equation}
	\Delta a_\ell^{\text{SI}} = \Delta a_\mu^{\text{exp}} \times \left(\frac{m_\ell}{m_\mu}\right)^2
\end{equation}

where $\Delta a_\mu^{\text{exp}} \approx 37.5 \times 10^{-11}$ (as of 2025) is the experimental muon discrepancy.

This is **phenomenological** (like hadronic contributions in SM), but the **structure** $(m_\ell/m_\mu)^2$ is fundamentally derived from the Lagrangian.

\subsection{Alternative: Natural Units}

In natural units ($\alpha = 1$), the dependence on SI constants vanishes:

\begin{equation}
	\tilde{a}_\ell = \tilde{C} \times \xi \times \tilde{m}_\ell^2
\end{equation}

where $\tilde{C}$ is a geometric constant (from $m_T/\xi$ and loop integral).

The ratio is then:
\begin{equation}
	\frac{\tilde{a}_\tau}{\tilde{a}_\mu} = \left(\frac{\tilde{m}_\tau}{\tilde{m}_\mu}\right)^2
\end{equation}

Identical to the SI version – ratios are invariant!

\section{Summary}

In this chapter, we have shown how time-mass duality is integrated into quantum field theory:

\begin{enumerate}
	\item The Schrödinger equation as an effective description of a deeper geometric dynamics
	
	\item The Dirac equation with geometrically derived mass $m$ from $T \cdot m = 1$
	
	\item The extended Lagrangian with time field $\Delta m$ and mass-proportional coupling $g_T^\ell = \xipar m_\ell$
	
	\item The fundamental structural statement $\Delta a \propto m^2$ from the Lagrangian
	
	\item The resulting ratio prediction $\Delta a_\tau/\Delta a_\mu = (m_\tau/m_\mu)^2$
	
	\item The limits of ab-initio calculation (analogous to QCD in SM)
\end{enumerate}

\begin{keypoint}[Fundamental vs. Phenomenological Predictions]
	The Lagrangian provides the **structure** $\Delta a \propto m^2$ as a fundamental statement. The **amplitude** (absolute value) requires normalization to experiment, i.e., is phenomenological. This is analogous to the situation of hadronic contributions in SM.
	
	The testable core prediction is the **ratio** $\Delta a_\tau/\Delta a_\mu = 283$, not the absolute value.
\end{keypoint}

This formulation shows how $\xipar$ determines the structure of quantum corrections without providing all numerical details ab initio – a realistic picture of theoretical possibilities.