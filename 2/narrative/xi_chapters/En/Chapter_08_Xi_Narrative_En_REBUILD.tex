% Chapter 08: Singularities and the Natural UV Cutoff
% Completely rewritten with correct formulas
% English translation

\chapter{Singularities and the Natural UV Cutoff}

\section{Introduction}

In many standard models of physics, formal infinities appear: 
diverging integrals in quantum field theory, singularities in black 
holes, or a point-like beginning of the universe. Time-mass duality 
and the fractal spacetime structure of FFGFT propose a different 
path: The underlying geometry is organized such that true physical 
infinities never arise in the first place.

\section{The Natural UV Cutoff}

\subsection{Emergence from the Fractal Dimension}

The fractal dimension of spacetime:

\begin{equation}
	D_f = 3 - \xi \approx 2.999867
	\label{eq:fractal_dim_ch8}
\end{equation}

implies a natural UV cutoff at the energy:

\begin{equation}
	\boxed{\Lambda_{\text{T0}} = \frac{E_{\text{Planck}}}{\xi} \approx 7.5 \times 10^{15}\,\text{GeV}}
	\label{eq:uv_cutoff}
\end{equation}

where $E_{\text{Planck}} = 1.221 \times 10^{19}$ GeV is the Planck energy.

\subsection{Physical Significance}

At energies above $\Lambda_{\text{T0}}$, the fractal structure of 
spacetime becomes dominant. All loop integrals automatically converge 
at this fundamental scale.

\section{Renormalization in the T0 Theory}

\subsection{Modified Beta Functions}

The renormalization group (RG) beta functions are modified by T0 corrections:

\begin{equation}
	\beta_g^{\text{T0}} = \beta_g^{\text{SM}} + \xi \cdot \frac{g^3}{(4\pi)^2} \cdot f_{\text{T0}}(g)
	\label{eq:beta_function_t0}
\end{equation}

where $f_{\text{T0}}(g)$ is a universal geometric function.

\subsection{One-Loop Integral}

A typical one-loop integral in QFT:

\begin{equation}
	I = \int \frac{d^4k}{(2\pi)^4} \frac{1}{k^2 - m^2}
	\label{eq:loop_integral_standard}
\end{equation}

diverges in the UV. In the T0 theory, it becomes:

\begin{equation}
	I^{\text{T0}} = \int_0^{\Lambda_{\text{T0}}} \frac{d^4k}{(2\pi)^4} \frac{1}{k^2 - m^2} \cdot \exp\left(-\frac{\xi k^4}{E_{\text{Planck}}^4}\right)
	\label{eq:loop_integral_t0}
\end{equation}

The exponential damping factor guarantees convergence.

\section{Black Holes without Singularity}

\subsection{Modified Metric}

The Schwarzschild metric becomes, as $r \to 0$:

\begin{equation}
	\begin{split}
		ds^2 &= \left(1 - \frac{r_S}{r} f_{\text{T0}}(r)\right) dt^2 - \left(1 - \frac{r_S}{r} f_{\text{T0}}(r)\right)^{-1} dr^2 \\
		&\quad - r^2 d\Omega^2
		\label{eq:metric_t0}
	\end{split}
\end{equation}

with the regularization function:

\begin{equation}
	f_{\text{T0}}(r) = \exp\left(-\frac{L_0}{r}\right)
	\label{eq:regularization}
\end{equation}

where $L_0 = \xi \cdot l_P$ is the minimal T0 length scale.

\subsection{Avoidance of the Central Singularity}

At $r \sim L_0$, $f_{\text{T0}}(r) \to 0$ and the metric remains regular. 
There is no true singularity, but a smooth transition to a geometric 
core of size $L_0 \approx 10^{-39}$ m.

\section{Big Bang without Singularity}

\subsection{Static vs. Expanding Universe}

The T0 theory favors a static universe with a $\xi$-field instead of 
cosmological expansion. The "Big Bang" is reinterpreted as an epoch of 
high energy density, not an actual singularity at $t=0$.

\subsection{Minimal Cosmological Time}

The minimal meaningful cosmological time scale is:

\begin{equation}
	t_{\text{min}} = \frac{L_0}{c} = \xi \cdot t_P \approx 7.2 \times 10^{-48}\,\text{s}
	\label{eq:t_min}
\end{equation}

Earlier "times" are geometrically meaningless.

\section{Fractal Damping}

\subsection{General Formula}

For highly excited states or large quantum numbers $n$, fractal damping 
occurs:

\begin{equation}
	f(n) = f_0(n) \cdot \exp\left(-\xi \frac{n^2}{D_f}\right)
	\label{eq:fractal_damping}
\end{equation}

where $f_0(n)$ is the undamped function.

\subsection{Application to Rydberg States}

For hydrogen Rydberg states:

\begin{equation}
	E_n^{\text{Rydberg}} = -\frac{13.6\,\text{eV}}{n^2} \cdot \exp\left(-\xi \frac{n^2}{D_f}\right)
	\label{eq:rydberg_damped}
\end{equation}

This prevents unphysical accumulation of states at large $n$.

\section{Summary}

FFGFT avoids singularities through:

\begin{enumerate}
	\item Natural UV cutoff: $\Lambda_{\text{T0}} = \frac{E_{\text{Planck}}}{\xi}$
	\item Regularized black holes with core radius $L_0 = \xi \cdot l_P$
	\item Static universe without Big Bang singularity
	\item Fractal damping at high energies/quantum numbers
	\item Minimal time/length scales: $t_{\text{min}}, L_0$
\end{enumerate}

The geometry itself prevents infinities – no ad-hoc regularization needed.