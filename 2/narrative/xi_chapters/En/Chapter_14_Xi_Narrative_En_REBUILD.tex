% Chapter 13
% Auto-reconstructed from FFGFT_Xi_Narrative_Master_De_print.pdf
% RAW source: 2\narrative\xi_de_chapters_raw\Kapitel_13_Xi_De_raw.txt
% English translation

\chapter{Why Unit Verification Is Essential}

Natural units make many formulas visually simpler: Constants like $c$ and $\hbar$ disappear from the notation, and couplings like $\alpha$ become seemingly pure numbers. Especially within the framework of time-mass duality, this is useful – but it also carries the danger of forgetting which physical scales are at work in the background. This chapter explains why systematic unit verification is indispensable and how the fractal structure reveals itself fully only through it.

\section{Natural Units as an Intermediate Space}

When calculating in natural units with $c = \hbar = 1$, many relationships become very compact. For example, the fine-structure constant appears, in a suitable normalization, simply as
\begin{equation}
	\alpha = \frac{e^2}{4\pi},
	\label{eq:alpha_natural_compact}
\end{equation}
and the structure organized by $\xi$ as
\begin{equation}
	\alpha = \xi\left(\frac{E_0}{\SI{1}{\mega\electronvolt}}\right)^2.
	\label{eq:alpha_xi_compact}
\end{equation}

In this intermediate space of natural units, the geometry is particularly clear to see. However, for a statement to become physically convincing, one must take the return path: from the compact notation back to the actual measurable quantity in SI units.

\section{Reconversion as a Stress Test}

The fractal structure and the scales defined by $\xi$ demonstrate their robustness only when conversion back to SI units consistently reproduces all known numbers. This means concretely:
\begin{itemize}
	\item One starts with a simple relation in natural units (e.g., $\alpha \sim \xi E_0^2$).
	\item One systematically reinserts all factors of $c$, $\hbar$, and the chosen base quantities.
	\item In particular, one fully reinserts $\alpha$ in the form $\alpha = \xi(E_0/\SI{1}{\mega\electronvolt})^2$, rather than treating it as a mere number.
	\item One checks whether the resulting values for energies, lengths, and times agree with experimental data.
\end{itemize}

Only this stress test reveals whether a seemingly elegant formula is truly more than number play. For time-mass duality, this means: The shortcut through natural units is helpful, but the physical content is decided upon reconversion to concrete units. Dangerous here are "clever" cancellations: If constants like $c$, $\hbar$, or even $\alpha$ are prematurely eliminated, the fractal structure can become invisible and seemingly compelling but physically false scales can arise. Precisely in natural units, it is tempting to immediately deduce $E = m$ from $E = mc^2$ or to turn $\alpha = \xi(E_0/\SI{1}{\mega\electronvolt})^2$ into a pure number; however, the correct physical conclusion always requires keeping in mind the underlying assumptions (rest frame, momentum, concrete scales) and explicitly reinserting them at the end.

\section{Example: CMB, Casimir, and $L_\xi$}

A particularly illustrative example is the relation
\begin{equation}
	\rho_{\text{CMB}} = \frac{\xi \hbar c}{L_\xi^4},
	\label{eq:cmb_relation_units}
\end{equation}
from which a characteristic length scale $L_\xi$ can be estimated.

In natural units, $\hbar$ and $c$ appear as harmless factors. Only when inserting the SI values for $\hbar$, $c$, and $\rho_{\text{CMB}}$ and carefully tracking the dimensions does it become clear that $L_\xi$ indeed lies in the range of $\SI{100}{\micro\meter}$ – exactly where Casimir experiments measure with high precision.

Without consistent unit verification, one could easily overlook or misjudge this connection. Thus, the fractal structure becomes visible not only conceptually but in the concrete back-calculation to real measurable quantities.

\section{Avoiding Spurious Correlations}

Conversely, strict unit verification helps distinguish random numerical overlaps from genuine relationships. Two numbers may look similar in natural units; if their dimensions differ, it is clear that they are not directly comparable.

Therefore, time-mass duality consistently works with dimensionless combinations (like $\alpha$) and clearly defined scales (like $E_0$, $L_0$, $L_\xi$) before drawing comparisons. Every step is accompanied by unit accounting:
\begin{itemize}
	\item Which quantity is truly dimensionless?
	\item Which combinations of $c$, $\hbar$, and base units appear?
	\item Where might seemingly similar numbers actually have different physical content?
\end{itemize}

\section{Units as an Integrity Check of the Theory}

Ultimately, unit verification is more than a technical formality. It serves as an integrity check of the entire theory:
\begin{itemize}
	\item It enforces consistency between geometric picture and measurable quantities.
	\item It reveals whether a proposed relationship is truly scale-compatible.
	\item It protects against overstretched interpretations of seemingly beautiful numbers.
\end{itemize}

For FFGFT and time-mass duality, this means: Only the combination of natural units and consistent back-checking into SI units exposes how deeply the fractal structure intervenes in observed physics. Thus, natural units are a useful working space – the reality check occurs in the familiar units of our measuring instruments.

Simultaneously, a philosophical caveat remains: Every measurement ultimately compares frequencies or counting rates and thus provides only relative statements; what is ontologically "really" slowing down or becoming heavier eludes direct testability. For FFGFT, this means: What is crucial is not whether we can absolutely determine whether time slows down or mass increases; what is crucial is that the mathematical structure is consistent and reproduces all observable relations (frequencies, scales, ratios).