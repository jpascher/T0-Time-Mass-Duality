% Chapter 09: Cosmology, Redshift and CMB in Time-Mass Duality
% Completely rewritten with correct formulas
% English translation

\chapter{Cosmology, Redshift and CMB in Time-Mass Duality}

\section{Introduction}

In the preceding chapters, the microscopic side of time-mass duality 
was the focus: masses, couplings, and quantum phenomena. This chapter 
outlines how the same structure affects large-scale cosmological 
phenomena: redshift, cosmic microwave background, and effective scales 
such as the Hubble scale.

\section{Redshift without Expanding Space}

\subsection{Standard Interpretation}

Standard cosmology interprets cosmological redshift primarily as a 
consequence of expanding spacetime. The wavelength of a photon is 
stretched along with the cosmic scale factor $a(t)$:

\begin{equation}
	\frac{\lambda_{\text{obs}}}{\lambda_{\text{emit}}} = \frac{a(t_{\text{obs}})}{a(t_{\text{emit}})} = 1 + z
	\label{eq:standard_redshift}
\end{equation}

\subsection{Time-Mass Duality Interpretation}

Within the framework of time-mass duality, an alternative picture is 
proposed. The observed redshift is understood as a consequence of the 
fractal deep structure.

The T0 redshift:

\begin{equation}
	z_{\text{T0}} = \int_0^d \xi(r) \frac{E_\gamma(r)}{E_{\gamma,0}} dr
	\label{eq:t0_redshift_integral}
\end{equation}

For a homogeneous $\xi$ field:

\begin{equation}
	z_{\text{T0}} \approx \xi \cdot d \cdot \left(1 - \frac{E_\gamma}{2E_{\gamma,0}}\right)
	\label{eq:t0_redshift_homogeneous}
\end{equation}

Hubble relation:

\begin{equation}
	H_0^{\text{T0}} = \xi \cdot c \approx 40\,\text{km/s/Mpc}
	\label{eq:t0_hubble}
\end{equation}

\section{CMB Temperature}

The CMB temperature:

\begin{equation}
	T_{\text{CMB}} = 2.7255\,\text{K}
	\label{eq:cmb_temperature}
\end{equation}

is interpreted in T0 as an equilibrium state of the $\xi$-geometry, 
not as a relic of a Big Bang.

\section{Static Universe}

The T0 theory favors a static universe. JWST observations of developed 
galaxies at $z > 10$ are consistent with unlimited development time.

\section{Summary}

Cosmological phenomena as manifestations of $\xi$-geometry, not as 
relics of a Big Bang past.