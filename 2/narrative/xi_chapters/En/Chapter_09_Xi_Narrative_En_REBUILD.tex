% Chapter 09: Cosmology, Redshift and CMB
% Extended with ξ_eff, H₀ discussion, CMB derivation
% Consistent with Ref 201 (DVFT-alles)

\chapter{Cosmology, Redshift and CMB in Time-Mass Duality}

\section{Introduction}

In the preceding chapters, the microscopic side of time-mass duality was the 
focus: masses, couplings, and quantum phenomena. This chapter shows how the 
same structure affects large-scale cosmological phenomena: redshift, cosmic 
microwave background radiation, and effective quantities such as the Hubble scale.

Crucial here is the transition from local to cosmological scales. While at the 
particle level the full parameter $\xipar = \frac{4}{3} \times 10^{-4}$ is 
operative, the DVFT (Dynamic Vacuum Field Theory, Ref.\ 201) shows that at 
cosmological scales an effective parameter becomes relevant:

\begin{equation}
\boxed{\xi_{\text{eff}} = \frac{\xipar}{2} \approx 6.667 \times 10^{-5}}
\label{eq:xi_eff}
\end{equation}

This factor of $1/2$ arises from averaging the dynamic vacuum field over 
cosmological distances in an infinitely homogeneous geometry.

\section{Redshift Without Expanding Space}

\subsection{Standard Interpretation}

Standard cosmology interprets the cosmological redshift primarily as a 
consequence of expanding spacetime. A photon's wavelength is stretched 
with the cosmic scale factor $a(t)$:

\begin{equation}
\frac{\lambda_{\text{obs}}}{\lambda_{\text{emit}}} = \frac{a(t_{\text{obs}})}{a(t_{\text{emit}})} = 1 + z
\end{equation}

\subsection{Time-Mass Duality Interpretation}

Within the framework of time-mass duality, the observed redshift is 
understood as a consequence of the fractal deep structure of spacetime. A 
photon propagating through the fractal space with $D_f = 3 - \xipar$ 
continuously loses energy to the dynamic vacuum field.

The T0 redshift:

\begin{equation}
z_{\text{T0}} = \int_0^d \xipar(r) \frac{E_\gamma(r)}{E_{\gamma,0}} dr
\end{equation}

For a homogeneous $\xipar$ field, this simplifies to:

\begin{equation}
z_{\text{T0}} \approx \xipar \cdot d \cdot \left(1 - \frac{E_\gamma}{2E_{\gamma,0}}\right)
\end{equation}

\subsection{Effective Hubble Parameter}

For the Hubble relation, one must distinguish between the local and 
cosmological $\xi$ value:

\begin{equation}
H_0^{\text{T0}} = \xi_{\text{eff}} \cdot c = \frac{\xipar}{2} \cdot c
\label{eq:H0_T0}
\end{equation}

\begin{remark}[On the discrepancy with observed $H_0$]
	The Hubble parameter calculated purely from $\xipar$, $H_0 = \xipar \cdot c 
	\approx 40\,$km/s/Mpc, lies significantly below the observed value of 
	$H_0^{\text{exp}} \approx 67$--$73\,$km/s/Mpc. With $\xi_{\text{eff}} = \xipar/2$, 
	one obtains $H_0 \approx 20\,$km/s/Mpc, which initially increases the discrepancy.
	
	This difference has a physical explanation in T0 theory: the observed $H_0$ 
	value of standard cosmology does not measure an expansion velocity, but rather 
	an effective redshift rate, which in the T0 picture contains additional 
	contributions from local vacuum dynamics and the interaction of photons with 
	the dynamic vacuum field $\Phi = \rho e^{i\theta}$. A detailed treatment of 
	these effects can be found in the redshift chapter (Ch.\ 16) and in Ref.\ 201.
\end{remark}

\section{Cosmological Vacuum Density}

In the DVFT (Ref.\ 201), the vacuum possesses an equilibrium amplitude 
determined at cosmological scales by $\xi_{\text{eff}}$:

\begin{equation}
\rho_0^{\text{cosmo}} = \frac{1}{(\xi_{\text{eff}})^2} = \frac{4}{\xipar^2} \approx 2.25 \times 10^8
\label{eq:rho_cosmo}
\end{equation}

while at local scales $\rho_0 = 1/\xipar^2 \approx 5.625 \times 10^7$ holds. 
The factor of 4 between cosmological and local vacuum density is a direct 
consequence of $\xi_{\text{eff}} = \xipar/2$.

\section{CMB Temperature}

The CMB temperature:

\begin{equation}
T_{\text{CMB}} = 2.7255\,\text{K}
\end{equation}

is interpreted in T0 theory as a thermodynamic equilibrium state of the 
$\xipar$ geometry, not as a relic of a Big Bang. The dynamic vacuum field 
$\Phi = \rho e^{i\theta}$ has an intrinsic phase evolution $\dot{\theta} = m 
= 1/T$ (from time-mass duality). The CMB radiation is the thermal equilibrium 
spectrum of this universal vacuum field, whose temperature is determined by the 
geometric parameters $\xipar$ and $f = 7500$.

\section{Static Universe}

T0 theory describes a static, infinitely homogeneous universe without global 
expansion (Ref.\ 201). In this view:

\begin{itemize}
\item The universe evolves through the dynamic vacuum field from T0 duality
\item Redshift arises from energy loss in the vacuum field, not from expansion
\item Large-scale coherence is explained by the infinite homogeneous geometry 
      ($\xi_{\text{eff}} = \xipar/2$), without inflation
\item Dark energy is not a separate substance but manifests as an effective 
      negative-pressure component of the vacuum field
\end{itemize}

JWST observations of evolved galaxies at $z > 10$, which appear unexpectedly 
early in the Standard Model, are natural in the T0 picture since the 
development time is unlimited.

\section{Summary}

The cosmological predictions of T0 theory follow from the transition 
$\xipar \to \xi_{\text{eff}} = \xipar/2$ at cosmological scales:

\begin{itemize}
\item Redshift as energy loss in the dynamic vacuum field
\item Effective Hubble parameter $H_0^{\text{T0}} = \xi_{\text{eff}} \cdot c$ 
      (connection to observed $H_0$ via vacuum dynamics)
\item Cosmological vacuum density $\rho_0^{\text{cosmo}} = 4/\xipar^2$
\item CMB as equilibrium state of the vacuum geometry
\item Static, infinitely homogeneous universe without expansion
\end{itemize}

