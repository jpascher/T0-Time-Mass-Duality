% Kapitel 06: Einheiten, Skalen und Konstanten aus xi
% Komplett neu geschrieben mit korrekten Formeln
% Basis: 013_T0_SI_De.tex, 015_NatEinheitenSystematik_De.tex

\chapter{Einheiten, Skalen und Konstanten aus $\xi$}

\section{Einführung}

Ein zentrales Versprechen der FFGFT ist, dass alle fundamentalen Konstanten der 
Physik aus dem einzigen Parameter $\xipar$ ableitbar sind. In diesem Kapitel 
zeigen wir, wie dies konkret funktioniert – von der Gravitations konstanten $G$ 
über die Planck-Länge $l_P$ bis zur Boltzmann-Konstante $k_B$.

\section{Natürliche Einheiten}

\subsection{Das Konzept}

In der theoretischen Physik werden häufig \textbf{natürliche Einheiten} verwendet, 
bei denen fundamentale Konstanten auf 1 gesetzt werden:

\begin{equation}
\hbar = c = 1
\label{eq:natural_units}
\end{equation}

In diesem System haben alle Größen Dimensionen von Energie $E$ (oder Potenzen davon):

\begin{align}
[M] &= [E] \quad \text{(aus } E = mc^2\text{)} \\
[L] &= [E^{-1}] \quad \text{(aus } \lambda = \hbar/p\text{)} \\
[T] &= [E^{-1}] \quad \text{(aus } \omega = E/\hbar\text{)}
\end{align}

\subsection{Dimensionsanalyse der Gravitationskonstante}

Die Gravitationskonstante hat in natürlichen Einheiten die Dimension:

\begin{equation}
[G] = [M^{-1}L^3T^{-2}] = [E^{-1}][E^{-3}][E^2] = [E^{-2}]
\label{eq:G_dimension}
\end{equation}

\section{Herleitung der Gravitationskonstante}

\subsection{Fundamentale T0-Formel}

Die Gravitationskonstante folgt aus $\xipar$ und der Elektronmasse (eine 
ausführliche Herleitung mit Dimensionsanalyse und Vergleich alternativer 
Darstellungen findet sich in Kapitel 7):

\begin{equation}
G = \frac{\xipar^2}{4 m_e}
\label{eq:G_fundamental}
\end{equation}

in natürlichen Einheiten.

\begin{remark}[Alternative Darstellung: $G = \xi/2$]
	Im Torsionskristall-Formalismus (Ref.\ 149) wird $G$ auch als $G = \xi/2$ 
	geschrieben. Dies ist konsistent, da dort die Referenzmasse $m = \xi/2$ 
	gesetzt wird (natürliche Massenskala des Gitters), sodass 
	$G = \xi^2/(4 \cdot \xi/2) = \xi/2$. Die hier verwendete Form $G = \xi^2/(4 m_e)$ 
	macht die Abhängigkeit von der Elektronmasse explizit, was für SI-Umrechnungen 
	und numerische Verifikation nützlicher ist. Beide Darstellungen sind äquivalent 
	in ihren jeweiligen Einheitensystemen.
\end{remark}

\subsection{Vollständige Formel mit SI-Umrechnung}

Für die Umrechnung in SI-Einheiten benötigen wir den Konversionsfaktor:

\begin{equation}
\boxed{G_{\text{SI}} = \frac{\xipar^2}{4 m_e} \times C_{\text{conv}}}
\label{eq:G_complete}
\end{equation}

wobei:
\begin{itemize}
\item $\xipar = \frac{4}{3} \times 10^{-4}$ (geometrischer Parameter)
\item $m_e = 0.511$ MeV (Elektronmasse, bereits fraktal korrigiert über $v = 246\,$GeV)
\item $C_{\text{conv}} = 7.783 \times 10^{-3}$ (Umrechnungsfaktor aus $\hbar$, $c$)
\end{itemize}

\begin{remark}[Historischer Faktor $K_{\text{frak}}$]
	In früheren Formulierungen erschien ein zusätzlicher Faktor $K_{\text{frak}} = 0.986$ 
	in der $G$-Formel. In der modernen Formulierung ist diese fraktale Korrektur 
	im gemessenen Higgs-VEV $v = 246\,$GeV und damit in $m_e$ bereits absorbiert. 
	Die Massenformeln $m_i = r_i \times \xi^{p_i} \times v$ verwenden den gemessenen 
	$v$-Wert direkt, sodass kein separater $K_{\text{frak}}$-Faktor mehr benötigt wird.
\end{remark}

\subsection{Numerisches Ergebnis}

\begin{equation}
G_{\text{SI}} = 6.674 \times 10^{-11}\,\text{m}^3/(\text{kg}\cdot\text{s}^2)
\label{eq:G_result}
\end{equation}

mit $< 0.001\%$ Abweichung vom CODATA-2018-Wert!

\section{Die Planck-Länge}

\subsection{Standarddefinition}

Die Planck-Länge ist definiert als:

\begin{equation}
l_P = \sqrt{\frac{\hbar G}{c^3}}
\label{eq:planck_length_standard}
\end{equation}

In natürlichen Einheiten ($\hbar = c = 1$) vereinfacht sich dies zu:

\begin{equation}
l_P = \sqrt{G}
\label{eq:planck_length_natural}
\end{equation}

\subsection{T0-Herleitung aus $\xipar$}

Da $G$ von $\xipar$ abgeleitet wird, folgt die Planck-Länge direkt:

\begin{equation}
l_P = \sqrt{G} = \sqrt{\frac{\xipar^2}{4 m_e}} = \frac{\xipar}{2\sqrt{m_e}}
\label{eq:planck_from_xi}
\end{equation}

In natürlichen Einheiten mit $m_e = 0.511$ MeV:

\begin{equation}
l_P = \frac{1.333 \times 10^{-4}}{2\sqrt{0.511}} \approx 9.33 \times 10^{-5}
\label{eq:planck_nat}
\end{equation}

Umrechnung in SI-Einheiten:

\begin{equation}
\boxed{l_P = 1.616 \times 10^{-35}\,\text{m}}
\label{eq:planck_si}
\end{equation}

\section{Charakteristische T0-Längenskalen}

\subsection{Die Sub-Planck-Skala}

Die minimale Sub-Planck-Längenskala ist:

\begin{equation}
L_0 = \xipar \cdot l_P = \frac{4}{3} \times 10^{-4} \times 1.616 \times 10^{-35}\,\text{m} = 2.155 \times 10^{-39}\,\text{m}
\label{eq:sub_planck}
\end{equation}

Diese Skala ist etwa $10^4$ mal kleiner als die Planck-Länge und markiert die 
absolute Untergrenze der Raumzeit-Granulation.

\subsection{Energieabhängige Längenskalen}

Die charakteristische T0-Länge für eine Energie $E$ ist:

\begin{equation}
r_0(E) = 2GE
\label{eq:r0_energy}
\end{equation}

In natürlichen Einheiten ($G = 1$):

\begin{equation}
r_0(E) = \frac{1}{E}
\label{eq:r0_natural}
\end{equation}

Für die fundamentale Energieskala $\Ezero = \sqrt{m_e \cdot m_\mu}$:

\begin{equation}
r_0(\Ezero) = 2G\Ezero \approx 2.7 \times 10^{-14}\,\text{m}
\label{eq:r0_E0}
\end{equation}

\section{Die Boltzmann-Konstante}

\subsection{Verbindung zur Temperatur}

Die Boltzmann-Konstante verbindet Temperatur mit Energie:

\begin{equation}
E = k_B T
\label{eq:boltzmann_relation}
\end{equation}

In der T0-Theorie ist dies eine Manifestation der Zeit-Masse-Dualität auf 
thermodynamischen Skalen.

\subsection{Ableitung aus $\xipar$}

In natürlichen Einheiten ist $k_B$ dimensionslos. Die SI-Umrechnung folgt aus 
der Energieeinheit:

\begin{equation}
k_B^{\text{SI}} = \frac{\text{1 eV}}{\text{11604.5 K}} = 1.381 \times 10^{-23}\,\text{J/K}
\label{eq:boltzmann_si}
\end{equation}

Die T0-Theorie reproduziert dies durch die Verbindung zwischen Energie- und 
Temperaturskalen über $\xipar$-abgeleitete Massen.

\section{Die SI-Reform 2019}

\subsection{Fundamentale Neudefinition}

Die SI-Reform 2019 definierte das Kilogramm über die Planck-Konstante:

\begin{equation}
\hbar = 6.62607015 \times 10^{-34}\,\text{J}\cdot\text{s} \quad \text{(exakt)}
\label{eq:planck_const_exact}
\end{equation}

und die Boltzmann-Konstante:

\begin{equation}
k_B = 1.380649 \times 10^{-23}\,\text{J/K} \quad \text{(exakt)}
\label{eq:boltzmann_exact}
\end{equation}

\subsection{T0-Konsequenz}

Diese Reform implementierte unwissentlich die eindeutige Kalibration, die mit 
der T0-geometrischen Grundlage konsistent ist. Die SI-Einheiten sind jetzt 
implizit durch $\xipar$ festgelegt:

\begin{equation}
\text{SI-System} \leftrightarrow \xipar = \frac{4}{3} \times 10^{-4}
\label{eq:si_xi_connection}
\end{equation}

\section{Skalenhierarchie}

Die verschiedenen Längenskalen in der T0-Theorie:

\begin{align}
L_0 &= 2.155 \times 10^{-39}\,\text{m} \quad \text{(minimale T0-Skala)} \\
l_P &= 1.616 \times 10^{-35}\,\text{m} \quad \text{(Planck-Länge)} \\
r_0(\Ezero) &= 2.7 \times 10^{-14}\,\text{m} \quad \text{(charakteristische Skala)} \\
r_e &= 2.818 \times 10^{-15}\,\text{m} \quad \text{(klassischer Elektronradius)}
\end{align}

Diese Hierarchie emergiert vollständig aus $\xipar$ und der fraktalen Struktur 
der Raumzeit.

\section{Zusammenfassung}

In diesem Kapitel haben wir gezeigt, wie alle fundamentalen Einheiten und 
Konstanten aus $\xipar$ folgen:

\begin{enumerate}
\item Natürliche Einheiten: $\hbar = c = 1$ vereinfachen die Ableitungen
\item Gravitationskonstante: $G = \frac{\xipar^2}{4m_e} \times C_{\text{conv}}$ (fraktale Korrektur in $m_e$ absorbiert)
\item Planck-Länge: $l_P = \frac{\xipar}{2\sqrt{m_e}}$
\item Sub-Planck-Skala: $L_0 = \xipar \cdot l_P$
\item SI-Reform 2019: Konsistent mit T0-Geometrie
\end{enumerate}

Die vollständige Ableitungskette $\xipar \to m_e \to G \to l_P$ zeigt die 
Parameterfreiheit der Theorie. Alle physikalischen Größen emergieren aus der 
Geometrie des dreidimensionalen Raums.



