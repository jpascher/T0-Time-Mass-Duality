% Kapitel 09: Kosmologie, Rotverschiebung und CMB
% Erweitert mit ξ_eff, H₀-Diskussion, CMB-Herleitung
% Konsistent mit Ref 201 (DVFT-alles)

\chapter{Kosmologie, Rotverschiebung und CMB in der Zeit-Masse-Dualität}

\section{Einführung}

In den vorangegangenen Kapiteln stand die mikroskopische Seite der 
Zeit-Masse-Dualität im Mittelpunkt: Massen, Kopplungen und Quantenphänomene. 
In diesem Kapitel wird gezeigt, wie sich dieselbe Struktur auf großskalige 
Phänomene der Kosmologie auswirkt: Rotverschiebung, kosmische 
Hintergrundstrahlung und effektive Größen wie die Hubble-Skala.

Entscheidend ist dabei der Übergang von der lokalen zur kosmologischen Skala. 
Während auf Teilchenebene der volle Parameter $\xipar = \frac{4}{3} \times 10^{-4}$ 
wirkt, zeigt die DVFT (Dynamische Vakuumfeldtheorie, Ref.\ 201), dass auf 
kosmologischen Skalen ein effektiver Parameter relevant wird:

\begin{equation}
\boxed{\xi_{\text{eff}} = \frac{\xipar}{2} \approx 6{,}667 \times 10^{-5}}
\label{eq:xi_eff}
\end{equation}

Dieser Faktor $1/2$ entsteht aus der Mittelung des dynamischen Vakuumfelds 
über kosmologische Distanzen in einer unendlich homogenen Geometrie.

\section{Rotverschiebung ohne expandierenden Raum}

\subsection{Standard-Interpretation}

Die Standardkosmologie deutet die kosmologische Rotverschiebung hauptsächlich 
als Folge einer expandierenden Raumzeit. Die Wellenlänge eines Photons wird 
mit dem kosmischen Skalenfaktor $a(t)$ mitgedehnt:

\begin{equation}
\frac{\lambda_{\text{obs}}}{\lambda_{\text{emit}}} = \frac{a(t_{\text{obs}})}{a(t_{\text{emit}})} = 1 + z
\end{equation}

\subsection{Zeit-Masse-Dualität Interpretation}

Im Rahmen der Zeit-Masse-Dualität wird die beobachtete Rotverschiebung 
als Folge der fraktalen Tiefenstruktur der Raumzeit verstanden. Ein Photon, 
das durch den fraktalen Raum mit $D_f = 3 - \xipar$ propagiert, verliert 
kontinuierlich Energie an das dynamische Vakuumfeld.

Die T0-Rotverschiebung:

\begin{equation}
z_{\text{T0}} = \int_0^d \xipar(r) \frac{E_\gamma(r)}{E_{\gamma,0}} dr
\end{equation}

Für ein homogenes $\xipar$-Feld vereinfacht sich dies zu:

\begin{equation}
z_{\text{T0}} \approx \xipar \cdot d \cdot \left(1 - \frac{E_\gamma}{2E_{\gamma,0}}\right)
\end{equation}

\subsection{Effektiver Hubble-Parameter}

Für die Hubble-Relation muss unterschieden werden zwischen dem lokalen und 
dem kosmologischen $\xi$-Wert:

\begin{equation}
H_0^{\text{T0}} = \xi_{\text{eff}} \cdot c = \frac{\xipar}{2} \cdot c
\label{eq:H0_T0}
\end{equation}

\begin{remark}[Zur Diskrepanz mit dem beobachteten $H_0$]
	Der rein aus $\xipar$ berechnete Hubble-Parameter $H_0 = \xipar \cdot c 
	\approx 40\,$km/s/Mpc liegt deutlich unter dem beobachteten Wert von 
	$H_0^{\text{exp}} \approx 67$--$73\,$km/s/Mpc. Mit $\xi_{\text{eff}} = \xipar/2$ 
	ergibt sich $H_0 \approx 20\,$km/s/Mpc, was die Diskrepanz zunächst vergrößert.
	
	Diese Differenz hat in der T0-Theorie eine physikalische Erklärung: Der 
	beobachtete $H_0$-Wert der Standardkosmologie misst nicht eine 
	Expansionsgeschwindigkeit, sondern eine effektive Rotverschiebungsrate, die 
	im T0-Bild zusätzliche Beiträge aus der lokalen Vakuumdynamik und der 
	Wechselwirkung von Photonen mit dem dynamischen Vakuumfeld $\Phi = \rho 
	e^{i\theta}$ enthält. Eine detaillierte Behandlung dieser Effekte findet 
	sich im Kapitel zur Rotverschiebung (Kap.\ 16) und in Ref.\ 201.
\end{remark}

\section{Kosmologische Vakuumdichte}

In der DVFT (Ref.\ 201) besitzt das Vakuum eine Gleichgewichtsamplitude, die 
auf kosmologischen Skalen durch $\xi_{\text{eff}}$ bestimmt wird:

\begin{equation}
\rho_0^{\text{cosmo}} = \frac{1}{(\xi_{\text{eff}})^2} = \frac{4}{\xipar^2} \approx 2{,}25 \times 10^8
\label{eq:rho_cosmo}
\end{equation}

während auf lokaler Skala $\rho_0 = 1/\xipar^2 \approx 5{,}625 \times 10^7$ gilt. 
Der Faktor 4 zwischen kosmologischer und lokaler Vakuumdichte ist eine direkte 
Konsequenz von $\xi_{\text{eff}} = \xipar/2$.

\section{CMB-Temperatur}

Die CMB-Temperatur:

\begin{equation}
T_{\text{CMB}} = 2.7255\,\text{K}
\end{equation}

wird in der T0-Theorie als thermodynamischer Gleichgewichtszustand der 
$\xipar$-Geometrie interpretiert, nicht als Relikt eines Urknalls. Das 
dynamische Vakuumfeld $\Phi = \rho e^{i\theta}$ hat eine intrinsische 
Phasenentwicklung $\dot{\theta} = m = 1/T$ (aus der Zeit-Masse-Dualität). 
Die CMB-Strahlung ist das thermische Gleichgewichtsspektrum dieses 
universellen Vakuumfelds, dessen Temperatur durch die geometrischen 
Parameter $\xipar$ und $f = 7500$ festgelegt wird.

\section{Statisches Universum}

Die T0-Theorie beschreibt ein statisches, unendlich homogenes Universum ohne 
globale Expansion (Ref.\ 201). In dieser Sicht:

\begin{itemize}
\item Das Universum evolviert durch das dynamische Vakuumfeld aus der T0-Dualität
\item Rotverschiebung entsteht durch Energieverlust im Vakuumfeld, nicht durch Expansion
\item Die großskalige Kohärenz wird durch die unendliche homogene Geometrie 
      ($\xi_{\text{eff}} = \xipar/2$) erklärt, ohne Inflation
\item Dunkle Energie ist keine separate Substanz, sondern manifestiert sich 
      als effektive negative Druckkomponente des Vakuumfelds
\end{itemize}

JWST-Beobachtungen entwickelter Galaxien bei $z > 10$, die im 
Standardmodell unerwartet früh erscheinen, sind im T0-Bild natürlich, da 
die Entwicklungszeit unbegrenzt ist.

\section{Zusammenfassung}

Die kosmologischen Vorhersagen der T0-Theorie folgen aus dem Übergang 
$\xipar \to \xi_{\text{eff}} = \xipar/2$ auf kosmologischen Skalen:

\begin{itemize}
\item Rotverschiebung als Energieverlust im dynamischen Vakuumfeld
\item Effektiver Hubble-Parameter $H_0^{\text{T0}} = \xi_{\text{eff}} \cdot c$ 
      (Verbindung zum beobachteten $H_0$ über Vakuumdynamik)
\item Kosmologische Vakuumdichte $\rho_0^{\text{cosmo}} = 4/\xipar^2$
\item CMB als Gleichgewichtszustand der Vakuumgeometrie
\item Statisches, unendlich homogenes Universum ohne Expansion
\end{itemize}

