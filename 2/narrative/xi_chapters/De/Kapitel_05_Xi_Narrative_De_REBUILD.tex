% Kapitel 05: Vorhersagen und experimentelle Tests
% Komplett überarbeitet mit verhältnisbasierter Formulierung
% Stand: Januar 2026

\chapter{Vorhersagen und experimentelle Tests}

\section{Einführung}

Eine physikalische Theorie zeigt ihre Stärke in überprüfbaren Vorhersagen. Die 
FFGFT liefert Vorhersagen für eine Vielzahl von Experimenten. Dabei unterscheiden 
wir zwischen:

\begin{itemize}
	\item \textbf{Fundamentalen Vorhersagen:} Verhältnisse, die unabhängig von 
	Einheitensystemen und fraktalen Korrekturen sind
	\item \textbf{Phänomenologischen Vorhersagen:} Absolute Werte in SI-Einheiten, 
	die Umrechnungsfaktoren erfordern
\end{itemize}

\section{Anomale magnetische Momente der Leptonen}

Eine ausführliche quantitative Diskussion der anomalen magnetischen Momente der Leptonen – einschließlich Verhältnissen, Zahlenwerten und experimentellem Status – findet sich im dedizierten T0-Dokument \texttt{018\_T0\_Anomale-g2-10\_De.tex}.
Dieses Kapitel vermerkt nur, dass solche Präzisionstests existieren und als konzeptioneller Benchmark dienen; Formeln, Zahlen und detaillierte Belle-II-Prognosen werden hier nicht wiederholt.


\section{Weitere testbare Vorhersagen}

\subsection{Leptonmassen-Verhältnisse}

Die T0-Theorie sagt die Massenverhältnisse aus geometrischen Faktoren vorher:
\begin{align}
	\frac{m_\mu}{m_e} &= \frac{r_\mu}{r_e} \xi^{p_\mu - p_e} = \frac{16/5}{4/3} \xi^{-1/2} 
	\approx 207 \quad \checkmark \\
	\frac{m_\tau}{m_\mu} &= \frac{r_\tau}{r_\mu} \xi^{p_\tau - p_\mu} = \frac{8/3}{16/5} \xi^{-1/3} 
	\approx 16.8 \quad \checkmark
\end{align}

Diese sind \textbf{echte Vorhersagen}, da $(r,p)$ aus Quantenzahlen systematisch 
hergeleitet werden, nicht gefittet.

\subsection{Feinstrukturkonstante (Verhältnisaussage)}

Die T0-Theorie macht keine Aussage über den absoluten Wert $\alpha = 1/137$ (dieser 
ist ein SI-Umrechnungsfaktor). Aber sie sagt eine \textbf{Strukturrelation} vorher:

In natürlichen Einheiten gilt:
\begin{equation}
	\tilde{\alpha} = \xi \times \tilde{E}_0^2 = 1 \quad \text{(normiert)}
\end{equation}

Die Transformation zu SI-Einheiten ist phänomenologisch.

\subsection{Spektroskopische Tests}

\subsubsection{Wasserstoff-Spektrum}

Die T0-Korrekturen zu Wasserstoff-Energieniveaus sind extrem klein:
\begin{equation}
	\Delta E_n^{\text{T0}} \approx \xi \frac{E_n^2}{E_{\text{Planck}}} 
	\approx 10^{-31}\,\text{eV}
\end{equation}

Dies ist unterhalb aktueller Präzision, aber prinzipiell zugänglich mit 
Ultrapräzisions-Spektroskopie.

\subsubsection{Rydberg-Atome}

Für hochangeregte Zustände ($n \gg 1$) wird die fraktale Dämpfung relevant:
\begin{equation}
	\frac{E_n^{\text{Rydberg}}}{E_n^{\text{Bohr}}} = \exp\left(-\xi \frac{n^2}{D_f}\right)
\end{equation}

wobei $D_f = 3 - \xi$. Dies ist eine Verhältnisaussage und damit unabhängig von 
SI-Einheiten.

\section{Quantenverschränkung}

\subsection{T0-modifizierte Bell-Korrelation}

Die T0-Theorie modifiziert die Korrelationsfunktion verschränkter Teilchen:
\begin{equation}
	E(a,b)^{\text{T0}} = E(a,b)^{\text{QM}} \times \left(1 - \xi \cdot f(n,l,j)\right)
\end{equation}

Dies führt zu einer leichten Reduktion der CHSH-Verletzung. Das \textbf{Verhältnis}:
\begin{equation}
	\frac{S_{\text{CHSH}}^{\text{T0}}}{S_{\text{CHSH}}^{\text{QM}}} = 1 - \xi \cdot g(n) 
	\approx 0.9999
\end{equation}

ist wiederum eine fundamentale Aussage.

\section{Kosmologische Implikationen}

\subsection{Rotverschiebungs-Relation}

Die T0-Theorie modifiziert die Interpretation der kosmologischen Rotverschiebung. 
In einem statischen Universum mit fraktaler Struktur:

\begin{equation}
	\frac{\lambda_{\text{beobachtet}}}{\lambda_{\text{emittiert}}} = 1 + \xi \cdot f(d,t)
\end{equation}

wobei $d$ die Distanz und $t$ die Lichtlaufzeit ist.

\subsection{JWST-Beobachtungen}

Die James Webb Space Telescope Beobachtungen (2024-2025) zeigen entwickelte 
Galaxien bei hohen Rotverschiebungen ($z > 10$). Dies ist konsistenter mit einem 
statischen T0-Universum als mit $\Lambda$CDM, wo diese Strukturen nicht genug 
Zeit zur Entwicklung hatten.

Dies ist eine qualitative, aber keine quantitative Vorhersage.

\section{Zusammenfassung der Tests}

\begin{table}[h]
	\centering
	\caption{T0-Vorhersagen nach Typ}
	\begin{tabularx}{\textwidth}{|X|X|X|X|}
		\hline
		\textbf{Observable} & \textbf{Typ} & \textbf{T0-Vorhersage} & \textbf{Status} \\
		\hline
		$a_\tau/a_\mu$ & Fundamental & $(m_\tau/m_\mu)^2 = 283$ & Belle II 2027-28 \\
		\hline
		$m_\tau/m_\mu$ & Fundamental & $16.8$ (aus $r,p$) & Bestätigt \checkmark \\
		\hline
		$m_\mu/m_e$ & Fundamental & $207$ (aus $r,p$) & Bestätigt \checkmark \\
		\hline
		CHSH-Verhältnis & Fundamental & $\approx 0.9999$ & 73-Qubit Tests \\
		\hline
		$\Delta a_\mu$ absolut & Phänomenolog. & Normierung nötig & 37.5 × 10⁻¹¹ \\
		\hline
		H-Spektrum & Phänomenolog. & $10^{-31}$ eV & Ultrapräzision \\
		\hline
		JWST z>10 & Qualitativ & Statisches Universum & Unterstützt \\
		\hline
	\end{tabularx}
\end{table}

\section{Zukünftige Experimente}

\subsection{Priorität 1: Belle II Tau g-2 (2027-2028)}

Dies ist der \textbf{kritischste Test} der T0-Theorie:
\begin{itemize}
	\item Test der fundamentalen Vorhersage $a_\tau/a_\mu = 283$
	\item Unabhängig von phänomenologischen Parametern
	\item Direkter Test der quadratischen Massenskalierung
	\item Bei Widerspruch: T0-Theorie muss revidiert werden
\end{itemize}

\subsection{Priorität 2: Hochpräzisions-Massenverhältnisse}

\begin{itemize}
	\item Präzisere Messung von $m_\tau/m_\mu$ und $m_\mu/m_e$
	\item Test ob $(r,p)$-Werte exakt rational sind
	\item Suche nach generationsabhängigen Korrekturen
\end{itemize}

\subsection{Priorität 3: Fundamentale Konstanten-Verhältnisse}

\begin{itemize}
	\item Test ob $\alpha/\alpha_G$ (elektromagnetisch/gravitativ) durch $\xi$ bestimmt ist
	\item Suche nach Zeitvariation von Verhältnissen (sollte Null sein in T0)
	\item Vergleich verschiedener Methoden zur $\xi$-Bestimmung
\end{itemize}

\begin{keypoint}[Experimentelle Strategie]
	Die T0-Theorie sollte primär durch \textbf{Verhältnismessungen} getestet werden, 
	nicht durch absolute Werte. Verhältnisse sind fundamental, SI-unabhängig und 
	frei von Umrechnungsfaktoren. Der Belle II Test von $a_\tau/a_\mu$ ist der 
	klarste und direkteste Test der Kernaussagen der Theorie.
\end{keypoint}

\section{Grenzen der Vorhersagekraft}

\subsection{Was die T0-Theorie NICHT vorhersagt}

\begin{itemize}
	\item \textbf{Absolute Werte in SI:} Diese erfordern Umrechnungsfaktoren, die 
	phänomenologisch sind (z.B. $\alpha = 1/137$, $v = 246$ GeV)
	
	\item \textbf{Absolute g-2 Werte:} k\"onnen in der T0-Theorie nicht ab initio berechnet werden; nur Verh\"altnisse sind fundamental, und detaillierte Zahlenwerte werden in \texttt{018\_T0\_Anomale-g2-10\_De.tex} diskutiert
	
	\item \textbf{Quantitative QCD-Effekte:} Hadronische Physik ist zu komplex für 
	ab-initio Berechnung (wie im SM)
\end{itemize}

\subsection{Was die T0-Theorie vorhersagt}

\begin{itemize}
	\item \textbf{Verhältnisse:} $m_\tau/m_\mu$, $a_\tau/a_\mu$, etc. aus geometrischen 
	Faktoren
	
	\item \textbf{Strukturrelationen:} Quadratische Massenskalierung, fraktale Dämpfung
	
	\item \textbf{Qualitative Effekte:} Richtung von Korrekturen, Größenordnungen
\end{itemize}

Dies ist analog zum Standardmodell: Auch dort kann man z.B. Massenverhältnisse der 
Quarks nicht ab initio berechnen, wohl aber ihre elektroschwachen Kopplungen.

Die T0-Theorie geht einen Schritt weiter: Sie leitet Massenverhältnisse aus 
Geometrie her -- aber absolute Werte bleiben phänomenologisch.


