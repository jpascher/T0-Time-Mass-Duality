% Kapitel 03: Zeit-Masse-Dualität in Quantenmechanik und Feldtheorie
% Überarbeitet mit verhältnisbasierter Philosophie
% Stand: Januar 2026

\chapter{Zeit-Masse-Dualität in Quantenmechanik und Feldtheorie}

\section{Einführung}

In den bisherigen Kapiteln stand die Geometrie im Vordergrund: die Zahl $\xipar$, 
die fraktale Dimension $D_f$ und die daraus folgenden Skalen. Nun wenden wir 
diese Struktur auf die vertrauten Gleichungen der Quantenmechanik und 
Quantenfeldtheorie an.

\section{Schrödingergleichung als effektive Beschreibung}

In der Standardformulierung beschreibt die zeitabhängige Schrödingergleichung

\begin{equation}
	i\hbar \frac{\partial}{\partial t} \psi(t,\vec{x}) = \hat{H} \psi(t,\vec{x})
	\label{eq:schroedinger}
\end{equation}

die Entwicklung einer Wellenfunktion $\psi$ unter einem Hamiltonoperator $\hat{H}$. 
Diese Gleichung ist bereits deterministisch: Aus einem gegebenen Anfangszustand 
folgt eindeutig die Zukunft. Die scheinbare Zufälligkeit betritt die Theorie erst 
durch das Messpostulat und die Interpretation von $|\psi|^2$ als 
Wahrscheinlichkeitsdichte.

\subsection{T0-Interpretation}

Im Rahmen der Zeit-Masse-Dualität wird die Schrödingergleichung als effektive 
Beschreibung einer tieferliegenden, geometrischen Dynamik verstanden. Vereinfacht 
gesagt beschreibt $\psi$ nicht ein mysteriöses „Feld der Möglichkeiten'', sondern 
eine statistische Projektion der zugrunde liegenden fraktalen Zeitstruktur.

Die Parameter im Hamiltonoperator – insbesondere Massen und Kopplungsstärken – 
sind in der FFGFT nicht fundamental, sondern durch $\xipar$ und die daraus 
folgenden Skalen bestimmt.

\section{Von Schrödinger zu Dirac}

Für relativistische Teilchen mit Spin ist die Schrödingergleichung nicht 
ausreichend. Dort tritt die Dirac-Gleichung auf:

\begin{equation}
	(i\gamma^\mu \partial_\mu - m)\psi = 0
	\label{eq:dirac}
\end{equation}

mit den Dirac-Matrizen $\gamma^\mu$ und der Masse $m$. In der FFGFT wird $m$ 
nicht als Eingabeparameter betrachtet, sondern als abgeleitete Größe aus der 
Zeit-Masse-Dualität:

\begin{equation}
	T(x,t) \cdot m(x,t) = 1
	\label{eq:time_mass_duality_field}
\end{equation}

\subsection{Geometrische Deutung}

Damit ändert sich auch die Lesart der Dirac-Gleichung: Sie ist nicht die 
fundamentale Gleichung, sondern eine effektive Feldgleichung auf einem 
Hintergrund, dessen Geometrie bereits durch $\xipar$ festgelegt ist.

Die bekannten Eigenschaften – Spin, Antimaterie, Zitterbewegung – bleiben 
erhalten, erhalten aber eine geometrische Deutung im Rahmen der fraktalen 
Raumzeit.

\subsection{Vereinfachte Interpretation: Clifford-Algebra statt 4×4-Matrizen}

Die traditionelle Dirac-Gleichung verwendet komplexe 4×4-Matrizen ($\gamma^\mu$) 
und abstrakte Spinoren ($\psi$). Diese Matrixdarstellung ist jedoch nicht die 
fundamentale Physik, sondern nur eine **spezifische Repräsentation**.

\textbf{Fundamentale Struktur ohne explizite Matrizen:}

Die Dirac-Gleichung ist eigentlich eine Clifford-Algebra-Gleichung:
\begin{equation}
	(i \mathbf{e}_\mu \partial^\mu - m)\Psi = 0
	\label{eq:clifford_dirac}
\end{equation}

wobei:
\begin{itemize}
	\item $\mathbf{e}_\mu$: Abstrakte Basisvektoren der Raumzeit (keine Matrizen!)
	\item $\Psi$: Element im Spin-Raum (geometrisches Objekt)
	\item Die Algebra-Regel: $\mathbf{e}_\mu \mathbf{e}_\nu + \mathbf{e}_\nu \mathbf{e}_\mu = 2g_{\mu\nu}$
\end{itemize}

\textbf{In der T0-Theorie:}

Im Rahmen der fraktalen Raumzeit wird dies zu:
\begin{equation}
	(i \partial\!\!\!/_{\text{frak}} - m(x))\Psi(x) = 0
	\label{eq:t0_dirac}
\end{equation}

mit:
\begin{itemize}
	\item $\partial\!\!\!/_{\text{frak}}$: Differentialoperator in fraktaler Geometrie ($D_f = 3 - \xi$)
	\item $m(x) = 1/(c^2 T(x))$: Zeitabhängige Masse aus Zeit-Masse-Dualität
	\item $\Psi(x)$: Spinor-Feld im Spin-Bündel über fraktaler Mannigfaltigkeit
\end{itemize}

\textbf{Spin als geometrische Eigenschaft:}

Der Spin-1/2 Charakter ist keine Matrixeigenschaft, sondern:
\begin{itemize}
	\item Eine **topologische Wicklungszahl** auf dem Torus
	\item Eine **geometrische Eigenschaft** der Lösungen
	\item $\Psi$ geht unter 720°-Rotation in sich über (nicht 360°)
	\item Dies folgt aus der Clifford-Algebra-Struktur, nicht aus den Matrizen
\end{itemize}

\begin{important}{Fundamentale vs. Darstellungs-Ebene}
	Die 4×4-Matrizen ($\gamma^\mu$) sind ein **Berechnungswerkzeug**, nicht die 
	fundamentale Physik. Die Physik ist:
	\begin{enumerate}
		\item Clifford-Algebra-Struktur der Raumzeit
		\item Spin als topologische/geometrische Eigenschaft
		\item Zeit-Masse-Dualität: $m(x) = 1/(c^2 T(x))$
	\end{enumerate}
	
	In der T0-Theorie repräsentieren die $\gamma^\mu$ die **geometrische Struktur 
	des fraktalen Raums** mit $D_f = 3 - \xi$, nicht abstrakte algebraische Objekte.
	
	Für Berechnungen kann man die Standard-Matrixdarstellung verwenden, aber die 
	**Interpretation** ist geometrisch: Die Spinor-Struktur folgt aus der 
	Torus-Topologie, nicht aus willkürlichen Matrizen.
\end{important}

\textbf{Vergleich der Formulierungen:}

\begin{center}
	\begin{tabularx}{\textwidth}{>{\raggedright\arraybackslash}X >{\raggedright\arraybackslash}X >{\raggedright\arraybackslash}X}
		\toprule
		\textbf{Aspekt} & \textbf{Matrix-Darstellung} & \textbf{Geometrische Clifford-Form} \\
		\midrule
		Mathematik & 4×4-Matrizen & Clifford-Algebra \\
		Spin & In Matrizen kodiert & Topologische Eigenschaft \\
		Lorentz-Inv. & Explizit in Matrizen & In Algebra-Struktur \\
		T0-Integration & Schwierig & Natürlich (fraktale Geometrie) \\
		Status & Darstellung & Fundamental \\
		\bottomrule
	\end{tabularx}
\end{center}

\vspace{0.5cm}

Diese geometrische Formulierung ist nicht nur pädagogisch, sondern zeigt die 
**fundamentale Natur** der Dirac-Gleichung als Aussage über die geometrische 
Struktur der Raumzeit.

\section{Lagrangedichte und Rolle von $\xipar$}

\subsection{Erweiterter Lagrangian mit Zeitfeld}

Die vollständige T0-Formulierung verwendet einen erweiterten Lagrangian, 
der das dynamische Zeitfeld $T(x,t)$ oder äquivalent die Massenvariation 
$\Delta m$ enthält:

\[
\begin{aligned}
	\mathcal{L}_{\text{erweitert}} = 
	&-\frac{1}{4}F_{\mu\nu}F^{\mu\nu} 
	+ \bar{\psi}(i\gamma^\mu D_\mu - m)\psi \\
	&+ \frac{1}{2}(\partial_\mu \Delta m)(\partial^\mu \Delta m) 
	- \frac{1}{2}m_T^2 \Delta m^2 \\
	&+ \xi_{\text{par}} \, m_\ell \, \bar{\psi}_\ell \psi_\ell \, \Delta m
	\label{eq:lagrangian_extended}
\end{aligned}
\]

wobei:
\begin{itemize}
	\item $F_{\mu\nu}$: Elektromagnetischer Feldstärketensor
	\item $\psi$: Fermionfeld (Leptonen/Quarks)
	\item $\Delta m$: Dynamische Massenvariation (Zeitfeld)
	\item $m_T$: Charakteristische Masse des Zeitfeldes
	\item $\xipar m_\ell$: Fundamentale Kopplungsstärke
\end{itemize}

\subsection{Massenproportionale Kopplung}

Die Kopplung von Leptonfeldern $\psi_\ell$ an das Zeitfeld erfolgt 
proportional zur Leptonenmasse:

\begin{align}
	\mathcal{L}_{\text{Wechselwirkung}} &= g_T^\ell \bar{\psi}_\ell \psi_\ell \Delta m \label{eq:interaction}\\
	g_T^\ell &= \xipar m_\ell \label{eq:coupling}
\end{align}

Diese massenproportionale Kopplung ist zentral für die T0-Struktur und 
führt direkt zur quadratischen Massenskalierung.

\section{Struktur der T0-Beiträge}

\subsection{Ein-Schleifen-Diagramm}

Vom Wechselwirkungsterm $\mathcal{L}_{\text{int}} = \xipar m_\ell \bar{\psi}_\ell \psi_\ell \Delta m$ 
folgt ein Ein-Schleifen-Beitrag zum anomalen magnetischen Moment.

Der allgemeine Ausdruck ist:

\begin{equation}
	\Delta a_\ell \propto \frac{(g_T^\ell)^2 \cdot m_\ell^2}{m_T^2} 
	= \frac{\xipar^2 m_\ell^4}{m_T^2}
	\label{eq:one_loop_structure}
\end{equation}

\subsection{Fundamentale Strukturaussage}

Die wesentliche Aussage der T0-Theorie ist die \textbf{Skalierung}:

\begin{equation}
	\boxed{\Delta a_\ell \propto m_\ell^2}
	\label{eq:t0_scaling}
\end{equation}

Dies führt zu der fundamentalen Verhältnisvorhersage:

\begin{equation}
	\boxed{\frac{\Delta a_{\ell_1}}{\Delta a_{\ell_2}} = \left(\frac{m_{\ell_1}}{m_{\ell_2}}\right)^2}
	\label{eq:t0_ratio}
\end{equation}

Diese Vorhersage ist:
\begin{itemize}
	\item \textbf{Einheitensystem-unabhängig:} Verhältnisse sind invariant
	\item \textbf{Korrektur-unabhängig:} Fraktale Korrekturen kürzen sich
	\item \textbf{Parameterfrei:} Nur Massenverhältnisse
	\item \textbf{Pure Geometrie:} Folgt direkt aus $g_T \propto m$
\end{itemize}

\section{Vorhersagen für Leptonen}

\subsection{Fundamentale Verhältnisvorhersage}

Mit den gemessenen Leptonmassen folgt:

\begin{align}
	\frac{m_\mu}{m_e} &= \frac{105.658}{0.511} \approx 207 \quad \Rightarrow \quad 
	\frac{\Delta a_\mu}{\Delta a_e} \approx 42800 \\
	\frac{m_\tau}{m_\mu} &= \frac{1776.86}{105.658} \approx 16.8 \quad \Rightarrow \quad 
	\frac{\Delta a_\tau}{\Delta a_\mu} \approx 283
\end{align}

\subsection{Interpretation der Skalierung}

Die quadratische Massenskalierung $\Delta a \propto m^2$ bedeutet:
\begin{itemize}
	\item Schwerere Leptonen haben \textbf{quadratisch} größere T0-Beiträge
	\item Das Verhältnis ist \textbf{unabhängig} von Einheitensystemen
	\item Das Verhältnis ist \textbf{unabhängig} von fraktalen Korrekturen
	\item Pure \textbf{geometrische} Aussage aus der Kopplungsstruktur
\end{itemize}

Detaillierte experimentelle Vergleiche und Messungen werden in Kapitel 5 
(Vorhersagen und experimentelle Tests) behandelt.

\section{Grenzen der Theorie}

\subsection{Was die T0-Theorie auf dieser Ebene NICHT liefert}

Aus dem Lagrangian~\eqref{eq:lagrangian_extended} folgt die \textbf{Struktur} 
$\Delta a \propto m^2$, aber \textbf{nicht} der absolute Wert ohne weitere Annahmen:

\begin{itemize}
	\item Die Masse $m_T$ des Zeitfeld-Mediators ist nicht ab initio berechenbar
	\item Die vollständige Berechnung der Schleifenintegrale in fraktaler Raumzeit 
	($D_f = 3 - \xi$) ist extrem komplex
	\item Rekursive Wechselwirkungen zwischen Zeitfeld, Higgs und anderen Feldern 
	sind schwer zu behandeln
	\item Renormierung in nicht-ganzzahliger Dimension ist noch nicht vollständig 
	entwickelt
\end{itemize}

\subsection{Analogie zum Standardmodell}

Dies ist analog zur Situation im Standardmodell:
\begin{itemize}
	\item SM definiert die Lagrange-Dichte der QCD
	\item Aber hadronische Beiträge zu g-2 sind nicht ab initio berechenbar
	\item Man verwendet phänomenologische Methoden (Dispersionsrelationen, Lattice)
	\item Die \textbf{Struktur} ist klar, die \textbf{Amplitude} phänomenologisch
\end{itemize}

\subsection{Was die T0-Theorie liefert}

\begin{itemize}
	\item \textbf{Strukturaussage:} $\Delta a \propto m^2$ (quadratische Skalierung)
	\item \textbf{Verhältnisvorhersage:} $\Delta a_\tau / \Delta a_\mu = (m_\tau/m_\mu)^2$
	\item \textbf{Qualitative Erklärung:} Warum schwere Leptonen größere Beiträge haben
	\item \textbf{Testbare Vorhersage:} Belle II kann die quadratische Skalierung testen
\end{itemize}

\section{Phänomenologische Formulierung}

\subsection{Normierung am Myon}

Wenn man absolute SI-Werte berechnen möchte, normiert man am Myon:

\begin{equation}
	\Delta a_\ell^{\text{SI}} = \Delta a_\mu^{\text{exp}} \times \left(\frac{m_\ell}{m_\mu}\right)^2
\end{equation}

wobei $\Delta a_\mu^{\text{exp}} \approx 37.5 \times 10^{-11}$ (Stand 2025) die 
experimentelle Myon-Diskrepanz ist.

Dies ist \textbf{phänomenologisch} (wie hadronische Beiträge im SM), aber die 
\textbf{Struktur} $(m_\ell/m_\mu)^2$ ist fundamental aus dem Lagrangian hergeleitet.

\subsection{Alternative: Natürliche Einheiten}

In natürlichen Einheiten ($\alpha = 1$) verschwindet die Abhängigkeit von SI-Konstanten:

\begin{equation}
	\tilde{a}_\ell = \tilde{C} \times \xi \times \tilde{m}_\ell^2
\end{equation}

wobei $\tilde{C}$ eine geometrische Konstante ist (aus $m_T/\xi$ und Schleifenintegral).

Das Verhältnis ist dann:
\begin{equation}
	\frac{\tilde{a}_\tau}{\tilde{a}_\mu} = \left(\frac{\tilde{m}_\tau}{\tilde{m}_\mu}\right)^2
\end{equation}

Identisch mit der SI-Version -- Verhältnisse sind invariant!

\section{Zusammenfassung}

In diesem Kapitel haben wir gezeigt, wie die Zeit-Masse-Dualität in die 
Quantenfeldtheorie integriert wird:

\begin{enumerate}
	\item Die Schrödingergleichung als effektive Beschreibung einer tieferliegenden 
	geometrischen Dynamik
	
	\item Die Dirac-Gleichung mit geometrisch abgeleiteter Masse $m$ aus $T \cdot m = 1$
	
	\item Der erweiterte Lagrangian mit Zeitfeld $\Delta m$ und massenproportionaler 
	Kopplung $g_T^\ell = \xipar m_\ell$
	
	\item Die fundamentale Strukturaussage $\Delta a \propto m^2$ aus dem Lagrangian
	
	\item Die daraus folgende Verhältnisvorhersage $\Delta a_\tau/\Delta a_\mu = (m_\tau/m_\mu)^2$
	
	\item Die Grenzen der ab-initio Berechnung (analog zu QCD im SM)
\end{enumerate}

\begin{keypoint}[Fundamentale vs. phänomenologische Vorhersagen]
	Der Lagrangian liefert die \textbf{Struktur} $\Delta a \propto m^2$ als fundamentale 
	Aussage. Die \textbf{Amplitude} (absoluter Wert) erfordert Normierung am Experiment, 
	ist also phänomenologisch. Dies ist analog zur Situation hadronischer Beiträge im SM.
	
	Die testbare Kernvorhersage ist das \textbf{Verhältnis} $\Delta a_\tau/\Delta a_\mu = 283$, 
	nicht der absolute Wert.
\end{keypoint}

Diese Formulierung zeigt, wie $\xipar$ die Struktur der Quantenkorrekturen bestimmt, 
ohne alle numerischen Details ab initio zu liefern -- ein realistisches Bild der 
theoretischen Möglichkeiten.