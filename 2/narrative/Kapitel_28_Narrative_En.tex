\documentclass[12pt,a4paper]{article}
\usepackage[utf8]{inputenc}
\usepackage[T1]{fontenc}
\usepackage[english]{babel}
\usepackage{amsmath}
\usepackage{amsfonts}
\usepackage{amssymb}
\usepackage{geometry}
\geometry{a4paper,left=2.5cm,right=2.5cm,top=2.5cm,bottom=2.5cm}
\setlength{\headheight}{30pt}
\usepackage{fancyhdr}
\usepackage{enumitem}
\usepackage{tcolorbox}
\usepackage{physics}
\usepackage{hyperref}
\usepackage{siunitx}

% Define new units
\DeclareSIUnit\newton{N}
\DeclareSIUnit\meter{m}
\DeclareSIUnit\fermi{fm}

% Load hyperref as one of the last packages
\hypersetup{
	unicode=true,
	pdfencoding=unicode,
	bookmarksopen=true
}

% Clean PDF bookmarks
\pdfstringdefDisableCommands{%
	\def\Lambda{Lambda}%
	\def\Delta{Delta}%
	\def\approx{approx}%
	\def\Sigma{Sigma}%
	\def\eta{eta}%
	\def\psi{psi}%
	\def\xi{xi}%
}

\title{Chapter 28: Why Newton's Law Does Not Apply to Quantum Particles in Fractal T0-Geometry}
\author{}
\date{}

\begin{document}
	
	\maketitle
	
	\section{Chapter 28: Why Newton's Law Does Not Apply to Quantum Particles in Fractal T0-Geometry}
	
	
\subsection*{Progressive Narrative Introduction}

This chapter builds on the preceding insights. In the first 27 chapters, we have learned the fundamental principles of FFGFT: the Time-Mass Duality, the fractal geometry with parameter $\xi = \frac{4}{3} \times 10^{-4}$, the emergence of space, and numerous applications of these principles.

In this chapter, we expand our understanding with further aspects that follow from the established principles. We will see how the already known concepts enable new insights and how the image of the cosmic brain continues to be refined.

The results presented here assume understanding of the previous chapters and systematically advance the argumentation.

\subsection*{The Mathematical Framework}

Newton's law \(F = G m_1 m_2 / r^2\) works excellently for planets, stars, and galaxies. But does it apply to a single proton attracting another proton? The answer is: No, not fundamentally.
	
	Newton's law assumes: defined distance \(r\), point-like masses, classical trajectories. In quantum mechanics, these are absent.
	
	In the fractal Fundamental Fractal-Geometric Field Theory (FFGFT) with T0-Time-Mass Duality, gravitation is not spacetime curvature but deformation of the vacuum amplitude field \(\rho(x,t) \propto 1/T(x,t)\). Gravitation is defined for localized, delocalized, or superposed quantum states.
	
	Gravitational field \(\delta\rho(x)\) follows quantum wave function \(|\psi(x)|^2\). Classical limit emerges through decoherence. No singularities: \(\rho_0 = 1/\xi^2\) provides minimum.
	
	T0 achieves self-consistent quantum gravity framework, in which gravitation follows quantum mechanics. Everything from the single fundamental parameter \(\xi = \frac{4}{3} \times 10^{-4}\).
	
	\subsection{Symbol Directory and Units}
	
	\begin{tcolorbox}[title={\textbf{Important Symbols and their Units}}, colback=blue!5!white, colframe=blue!75!black]
		\begin{tabular}{p{0.3\textwidth}p{0.3\textwidth}p{0.35\textwidth}}
			\textbf{Symbol} & \textbf{Meaning} & \textbf{Unit (SI)} \\
			\hline
			\(\xi\) & Fractal scale parameter & dimensionless \\
			\(F\) & Gravitational force & \si{\newton} \\
			\(G\) & Gravitational constant & \si{\meter\cubed\per\kilo\gram\per\second\squared} \\
			\(m_1, m_2\) & Particle masses & \si{\kilo\gram} \\
			\(r\) & Distance between particles & \si{\meter} \\
			\(\rho(x,t)\) & Vacuum amplitude density & \si{\kilo\gram^{1/2}\per\meter^{3/2}} \\
			\(T(x,t)\) & Time density & \si{\second\per\meter^{3}} \\
			\(m(x,t)\) & Mass density & \si{\kilo\gram\per\meter^{3}} \\
			\(\delta \rho(x)\) & Gravitational field (amplitude deformation) & \si{\kilo\gram^{1/2}\per\meter^{3/2}} \\
			\(T^{00}(x)\) & Energy density component & \si{\joule\per\meter^3} \\
			\(|\psi(x)|^2\) & Probability density of wave function & \si{\per\meter^3} \\
			\(g(x)\) & Gravitational acceleration & \si{\meter\per\second^2} \\
			\(\rho_0\) & Vacuum equilibrium density & \si{\kilo\gram^{1/2}\per\meter^{3/2}} \\
			\(E_{\text{self}}\) & Self-gravitational energy & \si{\joule} \\
			\(c^2\) & Speed of light squared & \si{\meter^2\per\second^2} \\
			\(\alpha, \beta\) & Superposition coefficients & dimensionless \\
			\(\phi_1, \phi_2\) & Superposition states & dimensionless \\
			\(\Re\) & Real part & -- \\
			\(m_p\) & Proton mass & \si{\kilo\gram} \\
			\(\psi(x)\) & Wave function & dimensionless \\
		\end{tabular}
	\end{tcolorbox}
	
	\textbf{Unit check (Newton's law):}
	\begin{align*}
		[F] &= \si{\meter\cubed\per\kilo\gram\per\second\squared} \cdot \si{\kilo\gram} \cdot \si{\kilo\gram} / \si{\meter\squared} = \si{\newton}
	\end{align*}
	Units are consistent.
	
	\subsection{Problems of Classical Gravitation on Quantum Scale}
	
	Classical gravitation assumes defined positions and distances – in quantum mechanics, particles are delocalized.
	
	For superposition: Unclear what force acts.
	
	GR: Gravitation as spacetime curvature – but the metric for a superposed wave packet is not defined.
	
	\subsection{Gravitation as Amplitude Deformation in T0 – Complete Derivation}
	
	In T0, matter couples to vacuum amplitude:
	\begin{equation}
		\delta \rho(x) = \frac{G}{c^2} \cdot T^{00}(x) \cdot \xi^{-1}
	\end{equation}
	where \(T^{00} = m c^2 |\psi(x)|^2\) for non-relativistic particles.
	
	The effective gravitational acceleration:
	\begin{equation}
		g(x) = -\xi \cdot \nabla \ln \rho(x) \approx -\xi \cdot \frac{\nabla \delta \rho}{\rho_0}
	\end{equation}
	
	For a quantum mechanical system:
	\begin{equation}
		\delta \rho(x) = \frac{G m}{c^2} \cdot |\psi(x)|^2 \cdot \xi^{-1}
	\end{equation}
	
	\textbf{Unit check:}
	\begin{align*}
		[\delta \rho(x)] &= \si{\meter\cubed\per\kilo\gram\per\second\squared} / \si{\meter\squared\per\second\squared} \cdot \si{\joule\per\meter^3} \cdot \text{dimensionless} = \si{\kilo\gram\per\meter^3}
	\end{align*}
	Adapted to the unit of \(\rho\).
	
	The self-gravitational energy:
	\begin{equation}
		E_{\text{self}} = \int \frac{G m^2}{c^2} \cdot \frac{|\psi(x)|^2 |\psi(y)|^2}{|x-y|} \, d^3x d^3y \cdot \xi^{-2}
	\end{equation}
	
	\textbf{Unit check:}
	\begin{align*}
		[E_{\text{self}}] &= \si{\meter\cubed\per\kilo\gram\per\second\squared} \cdot \si{\kilo\gram^2} / \si{\meter\squared\per\second\squared} \cdot \si{\per\meter^6} \cdot \si{\meter^6} \cdot \text{dimensionless} = \si{\joule}
	\end{align*}
	
	\subsection{Superposition and Nonlocality}
	
	For superposition \(|\psi\rangle = \alpha |\phi_1\rangle + \beta |\phi_2\rangle\):
	\begin{equation}
		\delta \rho(x) = \frac{G m}{c^2 \xi} \left( |\alpha|^2 |\phi_1(x)|^2 + |\beta|^2 |\phi_2(x)|^2 + 2 \Re(\alpha^* \beta \phi_1^*(x) \phi_2(x)) \right)
	\end{equation}
	
	The interference term creates nonlocal gravitation – no "two fields" problem.
	
	\textbf{Unit check:}
	\begin{align*}
		[\Re(\alpha^* \beta \phi_1^*(x) \phi_2(x))] &= \si{\per\meter^3}
	\end{align*}
	
	\subsection{Comparison with Other Approaches}
	
	\begin{center}
		\begin{tabular}{p{0.45\textwidth}p{0.45\textwidth}}
			\textbf{Other Approaches} & \textbf{T0-Fractal FFGFT} \\
			\hline
			Newton-Schrödinger: Nonlinear, collapses superposition & Linear, deterministic \\
			Post-quantum GR: Ad-hoc collapse models & Nonlocal through \(\xi\) \\
			No quantum gravity & Complete framework from duality \\
		\end{tabular}
	\end{center}
	
	\subsection{Example: Gravitation Between Two Protons}
	
	For \(r = \SI{e-15}{\meter}\) (Fermi distance):
	\begin{equation}
		F_g \approx \xi \cdot G \frac{m_p^2}{r^2} \approx \SI{e-40}{\newton}
	\end{equation}
	negligible, but defined for delocalized states.
	
	\textbf{Unit check:}
	\begin{align*}
		[F_g] &= \text{dimensionless} \cdot \si{\meter\cubed\per\kilo\gram\per\second\squared} \cdot \si{\kilo\gram^2} / \si{\meter\squared} = \si{\newton}
	\end{align*}
	
	\subsection{Conclusion}
	
	T0-theory defines gravitation on quantum scale consistently as amplitude deformation \(\delta \rho \propto |\psi|^2\). Superpositions create a unified, nonlocal field – no paradox. This is the first fully coherent quantum gravity on particle scale, everything from the single fundamental parameter \(\xi = \frac{4}{3} \times 10^{-4}\).
	

\subsection*{Progressive Narrative Summary}

This chapter has expanded our journey through FFGFT with important aspects. The concepts developed here build directly on the insights from chapters 1-27 and prepare the ground for the following investigations.

In the cosmic brain, each new chapter corresponds to a deeper layer of understanding – similar to how in a neural network, higher processing levels build on the activations of lower levels. The mathematical structures presented here are not isolated, but an integral part of the overall picture that unfolds through all 44 chapters.

In the coming chapters, we will see how these insights find further applications and how the unified picture of FFGFT continues to be completed. Each step brings us closer to a comprehensive understanding of the universe as a self-organizing, fractally structured system – a cosmic brain that creates and maintains its own structure through the Time-Mass Duality at every moment.

\end{document}
