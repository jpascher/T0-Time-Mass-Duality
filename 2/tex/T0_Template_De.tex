% ============================================================
% T0_Template_De.tex - Vorlage für neue deutsche T0 Dokumente
% T0 Zeit-Masse-Dualität Framework
% Autor: Johann Pascher
% ============================================================

\documentclass[12pt,a4paper]{article}

% Einheitliche T0 Präambel laden
% ============================================================
% T0_Preamble_De.tex - Einheitliche Präambel für deutsche Dokumente
% T0 Zeit-Masse-Dualität Framework
% Autor: Johann Pascher
% ============================================================

% Dokumentenklasse (muss vor \input dieser Datei gesetzt werden)
% \documentclass[12pt,a4paper]{article}

% ============================================================
% GRUNDLEGENDE PAKETE
% ============================================================

% Eingabe und Schriftcodierung
\usepackage[utf8]{inputenc}
\usepackage[T1]{fontenc}
\usepackage{lmodern}

% Deutsche Sprache
\usepackage[ngerman]{babel}

% ============================================================
% SEITENGEOMETRIE
% ============================================================

\usepackage{geometry}
\geometry{a4paper, margin=2.5cm}

% ============================================================
% MATHEMATIK-PAKETE
% ============================================================

\usepackage{amsmath}
\usepackage{amssymb}
\usepackage{amsthm}
\usepackage{mathtools}
\usepackage{physics}

% ============================================================
% GRAFIK UND FARBEN
% ============================================================

\usepackage{graphicx}
\usepackage{float}
\usepackage[table,xcdraw]{xcolor}
\usepackage{tikz}
\usetikzlibrary{positioning,shapes,arrows,arrows.meta}

% Standardfarben definieren
\definecolor{deepblue}{RGB}{0,0,127}
\definecolor{deepred}{RGB}{191,0,0}
\definecolor{deepgreen}{RGB}{0,127,0}

% ============================================================
% TABELLEN UND LISTEN
% ============================================================

\usepackage{booktabs}
\usepackage{array}
\usepackage{longtable}
\usepackage{enumitem}

% ============================================================
% HYPERLINKS
% ============================================================

\usepackage{hyperref}
\hypersetup{
	colorlinks=true,
	linkcolor=blue,
	citecolor=blue,
	urlcolor=blue,
	pdftitle={T0-Theorie Dokument},
	pdfauthor={Johann Pascher},
	pdfsubject={T0 Zeit-Masse-Dualität Framework}
}

% ============================================================
% KOPF- UND FUSSZEILEN
% ============================================================

\usepackage{fancyhdr}
\pagestyle{fancy}
\fancyhf{}
\fancyhead[L]{\textsc{T0-Theorie}}
\fancyhead[R]{\textsc{Johann Pascher}}
\fancyfoot[C]{\thepage}
\renewcommand{\headrulewidth}{0.4pt}
\renewcommand{\footrulewidth}{0.4pt}
\setlength{\headheight}{14.5pt}

% ============================================================
% TCOLORBOX FÜR HERVORGEHOBENE BEREICHE
% ============================================================

\usepackage{tcolorbox}
\tcbuselibrary{theorems,skins,breakable}

% Standardisierte Box-Umgebungen
\newtcolorbox{keyresult}[1][Schlüsselergebnis]{
	colback=blue!5!white,
	colframe=blue!75!black,
	fonttitle=\bfseries,
	title=#1,
	breakable
}

\newtcolorbox{warning}[1][Wichtiger Hinweis]{
	colback=red!5!white,
	colframe=red!75!black,
	fonttitle=\bfseries,
	title=#1,
	breakable
}

\newtcolorbox{formula}[1][Formel]{
	colback=blue!5!white,
	colframe=blue!75!black,
	fonttitle=\bfseries,
	title=#1,
	breakable
}

\newtcolorbox{result}[1][Ergebnis]{
	colback=green!5!white,
	colframe=green!75!black,
	fonttitle=\bfseries,
	title=#1,
	breakable
}

\newtcolorbox{summary}[1][Zusammenfassung]{
	colback=yellow!5!white,
	colframe=orange!75!black,
	fonttitle=\bfseries,
	title=#1,
	breakable
}

\newtcolorbox{important}[1][Wichtig]{
	colback=green!5!white,
	colframe=green!35!black,
	fonttitle=\bfseries,
	title=#1,
	breakable
}

% ============================================================
% THEOREM-UMGEBUNGEN
% ============================================================

\theoremstyle{definition}
\newtheorem{definition}{Definition}[section]
\newtheorem{theorem}{Theorem}[section]
\newtheorem{lemma}{Lemma}[section]
\newtheorem{corollary}{Korollar}[section]

% ============================================================
% TYPOGRAFISCHE EINSTELLUNGEN
% ============================================================

\usepackage{microtype}
\emergencystretch=2em
\tolerance=2000
\hyphenpenalty=500

% ============================================================
% ZUSÄTZLICHE NÜTZLICHE PAKETE
% ============================================================

\usepackage{siunitx}
\sisetup{
	locale=DE,
	per-mode=fraction,
	separate-uncertainty=true
}

% ============================================================
% BENUTZERDEFINIERTE BEFEHLE FÜR T0-THEORIE
% ============================================================

% Grundlegende T0 Parameter
\newcommand{\xipar}{\xi}
\newcommand{\Tfield}{T(x,t)}
\newcommand{\Efield}{E(x,t)}
\newcommand{\betaT}{\beta_{T}}
\newcommand{\alphaEM}{\alpha_{\text{EM}}}

% Planck-Einheiten
\newcommand{\EP}{E_{\text{P}}}
\newcommand{\lP}{\ell_{\text{P}}}
\newcommand{\tP}{t_{\text{P}}}
\newcommand{\mP}{m_{\text{P}}}
\newcommand{\Tzero}{T_0}

% Kosmologische Parameter
\newcommand{\LCDM}{\Lambda\text{CDM}}
\newcommand{\OmegaLambda}{\Omega_{\Lambda}}
\newcommand{\OmegaDM}{\Omega_{\text{DM}}}

% Einheiten-Kurzformen
\newcommand{\GeV}{\,\text{GeV}}
\newcommand{\MeV}{\,\text{MeV}}
\newcommand{\eV}{\,\text{eV}}
\newcommand{\natunits}{\text{(nat. Einh.)}}

% ============================================================
% ENDE DER PRÄAMBEL
% ============================================================


% ============================================================
% DOKUMENTSPEZIFISCHE EINSTELLUNGEN
% ============================================================

% Hier können dokumentspezifische Pakete hinzugefügt werden
% \usepackage{...}

% Dokumentspezifische PDF-Metadaten anpassen
\hypersetup{
	pdftitle={TITEL DES DOKUMENTS},
	pdfsubject={Themenbereich des Dokuments}
}

% Dokumentspezifische Header anpassen (optional)
% \fancyhead[L]{\textsc{Spezifischer Titel}}
% \fancyhead[R]{\textsc{Johann Pascher}}

% ============================================================
% DOKUMENTSPEZIFISCHE BEFEHLE (falls erforderlich)
% ============================================================

% \newcommand{\meinBefehl}{...}

% ============================================================
% TITEL UND AUTOR
% ============================================================

\title{\textbf{Titel des Dokuments}\\[0.5cm]
	\large Untertitel\\[0.3cm]
	\normalsize Dokumentenreihe der T0-Serie}
\author{Johann Pascher\\
	Abteilung für Kommunikationstechnologie\\
	Höhere Technische Lehranstalt (HTL), Leonding, Österreich\\
	\texttt{johann.pascher@gmail.com}}
\date{\today}

% ============================================================
% DOKUMENTINHALT
% ============================================================

\begin{document}
	
\maketitle
\thispagestyle{fancy}

\begin{abstract}
	Hier steht die Zusammenfassung des Dokuments.
\end{abstract}

\tableofcontents
\newpage

\section{Einleitung}

Einführungstext...

\section{Hauptteil}

Hauptinhalt...

% Beispiel für keyresult-Box
\begin{keyresult}[Wichtiges Ergebnis]
	Hier steht ein wichtiges Ergebnis der Analyse.
\end{keyresult}

% Beispiel für Formel
\begin{formula}[Grundlegende Gleichung]
	\begin{equation}
		T \cdot m = 1 \quad \text{(Zeit-Masse-Dualität)}
	\end{equation}
\end{formula}

\section{Schlussfolgerung}

Abschließende Bemerkungen...

% ============================================================
% BIBLIOGRAPHIE
% ============================================================

\begin{thebibliography}{99}
	
	% Einheitliche T0 Bibliographie-Einträge laden
	% ============================================================
% T0_Bibliography_Items_De.tex - Einheitliche Bibliographie-Einträge (Deutsch)
% T0 Zeit-Masse-Dualität Framework
% Zur Verwendung mit: % ============================================================
% T0_Bibliography_Items_De.tex - Einheitliche Bibliographie-Einträge (Deutsch)
% T0 Zeit-Masse-Dualität Framework
% Zur Verwendung mit: % ============================================================
% T0_Bibliography_Items_De.tex - Einheitliche Bibliographie-Einträge (Deutsch)
% T0 Zeit-Masse-Dualität Framework
% Zur Verwendung mit: \input{T0_Bibliography_Items_De}
% Diese Datei enthält nur die \bibitem-Einträge
% Muss innerhalb einer thebibliography-Umgebung verwendet werden
% ============================================================

% ========================================
% Grundlegende Dokumente
% ========================================

\bibitem{t0sicomplete}
Pascher, J. (2025).
\textit{Der vollständige Abschluss der T0-Theorie: Von $\xi$ zur SI-Reform 2019}.
HTL Leonding, Österreich.
\url{https://github.com/jpascher/T0-Time-Mass-Duality/blob/main/2/pdf/T0_SI_De.pdf}

\bibitem{t0grundlagen}
Pascher, J. (2025).
\textit{T0 Grundlagen}.
HTL Leonding, Österreich.
\url{https://github.com/jpascher/T0-Time-Mass-Duality/blob/main/2/pdf/T0_Grundlagen_De.pdf}

\bibitem{hdokument}
Pascher, J. (2025).
\textit{H-Dokument: Vollständiges T0 Framework Master-Dokument}.
HTL Leonding, Österreich.
\url{https://github.com/jpascher/T0-Time-Mass-Duality/blob/main/2/pdf/HdokumentDe.pdf}

\bibitem{t0energie}
Pascher, J. (2025).
\textit{T0-Energie: Umfassende energiebasierte Formulierung}.
HTL Leonding, Österreich.
\url{https://github.com/jpascher/T0-Time-Mass-Duality/blob/main/2/pdf/T0_Energie_De.pdf}

\bibitem{zusammenfassung}
Pascher, J. (2025).
\textit{Zusammenfassung: Umfassendes Übersichtsdokument}.
HTL Leonding, Österreich.
\url{https://github.com/jpascher/T0-Time-Mass-Duality/blob/main/2/pdf/Zusammenfassung_De.pdf}

% ========================================
% Mathematische Grundlagen
% ========================================

\bibitem{mathzeitmasse}
Pascher, J. (2025).
\textit{Mathematische Grundlagen der Zeit-Masse-Dualität mit Lagrange-Formalismus}.
HTL Leonding, Österreich.
\url{https://github.com/jpascher/T0-Time-Mass-Duality/blob/main/2/pdf/MathZeitMasseLagrangeDe.pdf}

\bibitem{eliminationmass}
Pascher, J. (2025).
\textit{Eliminierung der Masse: Mathematischer Rahmen}.
HTL Leonding, Österreich.
\url{https://github.com/jpascher/T0-Time-Mass-Duality/blob/main/2/pdf/EliminationOfMassDe.pdf}

\bibitem{lagrandianvergleich}
Pascher, J. (2025).
\textit{Lagrangian-Vergleich: Von Komplexität zu Eleganz}.
HTL Leonding, Österreich.
\url{https://github.com/jpascher/T0-Time-Mass-Duality/blob/main/2/pdf/LagrandianVergleichDe.pdf}

% ========================================
% Feinstrukturkonstante
% ========================================

\bibitem{t0feinstruktur}
Pascher, J. (2025).
\textit{T0 Feinstruktur: Mathematische Herleitung der Feinstrukturkonstante}.
HTL Leonding, Österreich.
\url{https://github.com/jpascher/T0-Time-Mass-Duality/blob/main/2/pdf/T0_Feinstruktur_De.pdf}

\bibitem{e137}
Pascher, J. (2025).
\textit{Umfassende Analyse der Zahl 137}.
HTL Leonding, Österreich.
\url{https://github.com/jpascher/T0-Time-Mass-Duality/blob/main/2/pdf/137_De.pdf}

% ========================================
% Teilchenmassen
% ========================================

\bibitem{t0teilchenmassen}
Pascher, J. (2025).
\textit{T0 Teilchenmassen: Systematische Massenberechnung aller Fermionen}.
HTL Leonding, Österreich.
\url{https://github.com/jpascher/T0-Time-Mass-Duality/blob/main/2/pdf/T0_Teilchenmassen_De.pdf}

\bibitem{t0neutrinos}
Pascher, J. (2025).
\textit{T0 Neutrinos: Spezielle Behandlung der Neutrinophysik}.
HTL Leonding, Österreich.
\url{https://github.com/jpascher/T0-Time-Mass-Duality/blob/main/2/pdf/T0_Neutrinos_De.pdf}

% ========================================
% Anomale Magnetische Momente
% ========================================

\bibitem{t0anomale}
Pascher, J. (2025).
\textit{T0 Anomale Magnetische Momente: Lösung des Myon g-2 Problems}.
HTL Leonding, Österreich.
\url{https://github.com/jpascher/T0-Time-Mass-Duality/blob/main/2/pdf/T0_Anomale_Magnetische_Momente_De.pdf}

% ========================================
% Kosmologie
% ========================================

\bibitem{t0kosmologie}
Pascher, J. (2025).
\textit{T0 Kosmologie: Kosmologische Anwendungen der T0-Theorie}.
HTL Leonding, Österreich.
\url{https://github.com/jpascher/T0-Time-Mass-Duality/blob/main/2/pdf/T0_Kosmologie_De.pdf}

\bibitem{hubble}
Pascher, J. (2025).
\textit{Hubble-Konstante Analyse im T0 Framework}.
HTL Leonding, Österreich.
\url{https://github.com/jpascher/T0-Time-Mass-Duality/blob/main/2/pdf/Ho_De.pdf}

% ========================================
% Quantenmechanik
% ========================================

\bibitem{t0qmqftrt}
Pascher, J. (2025).
\textit{T0 QM-QFT-RT: Vollständige Quantenfeldtheorie im T0 Framework}.
HTL Leonding, Österreich.
\url{https://github.com/jpascher/T0-Time-Mass-Duality/blob/main/2/pdf/T0_QM-QFT-RT_De.pdf}

\bibitem{qft}
Pascher, J. (2025).
\textit{Quantenfeldtheorie im T0 Framework}.
HTL Leonding, Österreich.
\url{https://github.com/jpascher/T0-Time-Mass-Duality/blob/main/2/pdf/QFT_De.pdf}

\bibitem{qmdeterministic}
Pascher, J. (2025).
\textit{Deterministische Quantenmechanik in T0}.
HTL Leonding, Österreich.
\url{https://github.com/jpascher/T0-Time-Mass-Duality/blob/main/2/pdf/QM-DetrmisticDe.pdf}

% ========================================
% Repository und Online-Ressourcen
% ========================================

\bibitem{t0repository}
Pascher, J. (2025).
\textit{T0-Time-Mass-Duality: Vollständiges Framework-Repository}.
GitHub Repository.
\url{https://github.com/jpascher/T0-Time-Mass-Duality}

\bibitem{t0website}
Pascher, J. (2025).
\textit{Interaktive T0 Framework-Exploration}.
Interaktive Website.
\url{https://jpascher.github.io/T0-Time-Mass-Duality/}

% ============================================================
% ENDE DER BIBLIOGRAPHIE-EINTRÄGE
% ============================================================

% Diese Datei enthält nur die \bibitem-Einträge
% Muss innerhalb einer thebibliography-Umgebung verwendet werden
% ============================================================

% ========================================
% Grundlegende Dokumente
% ========================================

\bibitem{t0sicomplete}
Pascher, J. (2025).
\textit{Der vollständige Abschluss der T0-Theorie: Von $\xi$ zur SI-Reform 2019}.
HTL Leonding, Österreich.
\url{https://github.com/jpascher/T0-Time-Mass-Duality/blob/main/2/pdf/T0_SI_De.pdf}

\bibitem{t0grundlagen}
Pascher, J. (2025).
\textit{T0 Grundlagen}.
HTL Leonding, Österreich.
\url{https://github.com/jpascher/T0-Time-Mass-Duality/blob/main/2/pdf/T0_Grundlagen_De.pdf}

\bibitem{hdokument}
Pascher, J. (2025).
\textit{H-Dokument: Vollständiges T0 Framework Master-Dokument}.
HTL Leonding, Österreich.
\url{https://github.com/jpascher/T0-Time-Mass-Duality/blob/main/2/pdf/HdokumentDe.pdf}

\bibitem{t0energie}
Pascher, J. (2025).
\textit{T0-Energie: Umfassende energiebasierte Formulierung}.
HTL Leonding, Österreich.
\url{https://github.com/jpascher/T0-Time-Mass-Duality/blob/main/2/pdf/T0_Energie_De.pdf}

\bibitem{zusammenfassung}
Pascher, J. (2025).
\textit{Zusammenfassung: Umfassendes Übersichtsdokument}.
HTL Leonding, Österreich.
\url{https://github.com/jpascher/T0-Time-Mass-Duality/blob/main/2/pdf/Zusammenfassung_De.pdf}

% ========================================
% Mathematische Grundlagen
% ========================================

\bibitem{mathzeitmasse}
Pascher, J. (2025).
\textit{Mathematische Grundlagen der Zeit-Masse-Dualität mit Lagrange-Formalismus}.
HTL Leonding, Österreich.
\url{https://github.com/jpascher/T0-Time-Mass-Duality/blob/main/2/pdf/MathZeitMasseLagrangeDe.pdf}

\bibitem{eliminationmass}
Pascher, J. (2025).
\textit{Eliminierung der Masse: Mathematischer Rahmen}.
HTL Leonding, Österreich.
\url{https://github.com/jpascher/T0-Time-Mass-Duality/blob/main/2/pdf/EliminationOfMassDe.pdf}

\bibitem{lagrandianvergleich}
Pascher, J. (2025).
\textit{Lagrangian-Vergleich: Von Komplexität zu Eleganz}.
HTL Leonding, Österreich.
\url{https://github.com/jpascher/T0-Time-Mass-Duality/blob/main/2/pdf/LagrandianVergleichDe.pdf}

% ========================================
% Feinstrukturkonstante
% ========================================

\bibitem{t0feinstruktur}
Pascher, J. (2025).
\textit{T0 Feinstruktur: Mathematische Herleitung der Feinstrukturkonstante}.
HTL Leonding, Österreich.
\url{https://github.com/jpascher/T0-Time-Mass-Duality/blob/main/2/pdf/T0_Feinstruktur_De.pdf}

\bibitem{e137}
Pascher, J. (2025).
\textit{Umfassende Analyse der Zahl 137}.
HTL Leonding, Österreich.
\url{https://github.com/jpascher/T0-Time-Mass-Duality/blob/main/2/pdf/137_De.pdf}

% ========================================
% Teilchenmassen
% ========================================

\bibitem{t0teilchenmassen}
Pascher, J. (2025).
\textit{T0 Teilchenmassen: Systematische Massenberechnung aller Fermionen}.
HTL Leonding, Österreich.
\url{https://github.com/jpascher/T0-Time-Mass-Duality/blob/main/2/pdf/T0_Teilchenmassen_De.pdf}

\bibitem{t0neutrinos}
Pascher, J. (2025).
\textit{T0 Neutrinos: Spezielle Behandlung der Neutrinophysik}.
HTL Leonding, Österreich.
\url{https://github.com/jpascher/T0-Time-Mass-Duality/blob/main/2/pdf/T0_Neutrinos_De.pdf}

% ========================================
% Anomale Magnetische Momente
% ========================================

\bibitem{t0anomale}
Pascher, J. (2025).
\textit{T0 Anomale Magnetische Momente: Lösung des Myon g-2 Problems}.
HTL Leonding, Österreich.
\url{https://github.com/jpascher/T0-Time-Mass-Duality/blob/main/2/pdf/T0_Anomale_Magnetische_Momente_De.pdf}

% ========================================
% Kosmologie
% ========================================

\bibitem{t0kosmologie}
Pascher, J. (2025).
\textit{T0 Kosmologie: Kosmologische Anwendungen der T0-Theorie}.
HTL Leonding, Österreich.
\url{https://github.com/jpascher/T0-Time-Mass-Duality/blob/main/2/pdf/T0_Kosmologie_De.pdf}

\bibitem{hubble}
Pascher, J. (2025).
\textit{Hubble-Konstante Analyse im T0 Framework}.
HTL Leonding, Österreich.
\url{https://github.com/jpascher/T0-Time-Mass-Duality/blob/main/2/pdf/Ho_De.pdf}

% ========================================
% Quantenmechanik
% ========================================

\bibitem{t0qmqftrt}
Pascher, J. (2025).
\textit{T0 QM-QFT-RT: Vollständige Quantenfeldtheorie im T0 Framework}.
HTL Leonding, Österreich.
\url{https://github.com/jpascher/T0-Time-Mass-Duality/blob/main/2/pdf/T0_QM-QFT-RT_De.pdf}

\bibitem{qft}
Pascher, J. (2025).
\textit{Quantenfeldtheorie im T0 Framework}.
HTL Leonding, Österreich.
\url{https://github.com/jpascher/T0-Time-Mass-Duality/blob/main/2/pdf/QFT_De.pdf}

\bibitem{qmdeterministic}
Pascher, J. (2025).
\textit{Deterministische Quantenmechanik in T0}.
HTL Leonding, Österreich.
\url{https://github.com/jpascher/T0-Time-Mass-Duality/blob/main/2/pdf/QM-DetrmisticDe.pdf}

% ========================================
% Repository und Online-Ressourcen
% ========================================

\bibitem{t0repository}
Pascher, J. (2025).
\textit{T0-Time-Mass-Duality: Vollständiges Framework-Repository}.
GitHub Repository.
\url{https://github.com/jpascher/T0-Time-Mass-Duality}

\bibitem{t0website}
Pascher, J. (2025).
\textit{Interaktive T0 Framework-Exploration}.
Interaktive Website.
\url{https://jpascher.github.io/T0-Time-Mass-Duality/}

% ============================================================
% ENDE DER BIBLIOGRAPHIE-EINTRÄGE
% ============================================================

% Diese Datei enthält nur die \bibitem-Einträge
% Muss innerhalb einer thebibliography-Umgebung verwendet werden
% ============================================================

% ========================================
% Grundlegende Dokumente
% ========================================

\bibitem{t0sicomplete}
Pascher, J. (2025).
\textit{Der vollständige Abschluss der T0-Theorie: Von $\xi$ zur SI-Reform 2019}.
HTL Leonding, Österreich.
\url{https://github.com/jpascher/T0-Time-Mass-Duality/blob/main/2/pdf/T0_SI_De.pdf}

\bibitem{t0grundlagen}
Pascher, J. (2025).
\textit{T0 Grundlagen}.
HTL Leonding, Österreich.
\url{https://github.com/jpascher/T0-Time-Mass-Duality/blob/main/2/pdf/T0_Grundlagen_De.pdf}

\bibitem{hdokument}
Pascher, J. (2025).
\textit{H-Dokument: Vollständiges T0 Framework Master-Dokument}.
HTL Leonding, Österreich.
\url{https://github.com/jpascher/T0-Time-Mass-Duality/blob/main/2/pdf/HdokumentDe.pdf}

\bibitem{t0energie}
Pascher, J. (2025).
\textit{T0-Energie: Umfassende energiebasierte Formulierung}.
HTL Leonding, Österreich.
\url{https://github.com/jpascher/T0-Time-Mass-Duality/blob/main/2/pdf/T0_Energie_De.pdf}

\bibitem{zusammenfassung}
Pascher, J. (2025).
\textit{Zusammenfassung: Umfassendes Übersichtsdokument}.
HTL Leonding, Österreich.
\url{https://github.com/jpascher/T0-Time-Mass-Duality/blob/main/2/pdf/Zusammenfassung_De.pdf}

% ========================================
% Mathematische Grundlagen
% ========================================

\bibitem{mathzeitmasse}
Pascher, J. (2025).
\textit{Mathematische Grundlagen der Zeit-Masse-Dualität mit Lagrange-Formalismus}.
HTL Leonding, Österreich.
\url{https://github.com/jpascher/T0-Time-Mass-Duality/blob/main/2/pdf/MathZeitMasseLagrangeDe.pdf}

\bibitem{eliminationmass}
Pascher, J. (2025).
\textit{Eliminierung der Masse: Mathematischer Rahmen}.
HTL Leonding, Österreich.
\url{https://github.com/jpascher/T0-Time-Mass-Duality/blob/main/2/pdf/EliminationOfMassDe.pdf}

\bibitem{lagrandianvergleich}
Pascher, J. (2025).
\textit{Lagrangian-Vergleich: Von Komplexität zu Eleganz}.
HTL Leonding, Österreich.
\url{https://github.com/jpascher/T0-Time-Mass-Duality/blob/main/2/pdf/LagrandianVergleichDe.pdf}

% ========================================
% Feinstrukturkonstante
% ========================================

\bibitem{t0feinstruktur}
Pascher, J. (2025).
\textit{T0 Feinstruktur: Mathematische Herleitung der Feinstrukturkonstante}.
HTL Leonding, Österreich.
\url{https://github.com/jpascher/T0-Time-Mass-Duality/blob/main/2/pdf/T0_Feinstruktur_De.pdf}

\bibitem{e137}
Pascher, J. (2025).
\textit{Umfassende Analyse der Zahl 137}.
HTL Leonding, Österreich.
\url{https://github.com/jpascher/T0-Time-Mass-Duality/blob/main/2/pdf/137_De.pdf}

% ========================================
% Teilchenmassen
% ========================================

\bibitem{t0teilchenmassen}
Pascher, J. (2025).
\textit{T0 Teilchenmassen: Systematische Massenberechnung aller Fermionen}.
HTL Leonding, Österreich.
\url{https://github.com/jpascher/T0-Time-Mass-Duality/blob/main/2/pdf/T0_Teilchenmassen_De.pdf}

\bibitem{t0neutrinos}
Pascher, J. (2025).
\textit{T0 Neutrinos: Spezielle Behandlung der Neutrinophysik}.
HTL Leonding, Österreich.
\url{https://github.com/jpascher/T0-Time-Mass-Duality/blob/main/2/pdf/T0_Neutrinos_De.pdf}

% ========================================
% Anomale Magnetische Momente
% ========================================

\bibitem{t0anomale}
Pascher, J. (2025).
\textit{T0 Anomale Magnetische Momente: Lösung des Myon g-2 Problems}.
HTL Leonding, Österreich.
\url{https://github.com/jpascher/T0-Time-Mass-Duality/blob/main/2/pdf/T0_Anomale_Magnetische_Momente_De.pdf}

% ========================================
% Kosmologie
% ========================================

\bibitem{t0kosmologie}
Pascher, J. (2025).
\textit{T0 Kosmologie: Kosmologische Anwendungen der T0-Theorie}.
HTL Leonding, Österreich.
\url{https://github.com/jpascher/T0-Time-Mass-Duality/blob/main/2/pdf/T0_Kosmologie_De.pdf}

\bibitem{hubble}
Pascher, J. (2025).
\textit{Hubble-Konstante Analyse im T0 Framework}.
HTL Leonding, Österreich.
\url{https://github.com/jpascher/T0-Time-Mass-Duality/blob/main/2/pdf/Ho_De.pdf}

% ========================================
% Quantenmechanik
% ========================================

\bibitem{t0qmqftrt}
Pascher, J. (2025).
\textit{T0 QM-QFT-RT: Vollständige Quantenfeldtheorie im T0 Framework}.
HTL Leonding, Österreich.
\url{https://github.com/jpascher/T0-Time-Mass-Duality/blob/main/2/pdf/T0_QM-QFT-RT_De.pdf}

\bibitem{qft}
Pascher, J. (2025).
\textit{Quantenfeldtheorie im T0 Framework}.
HTL Leonding, Österreich.
\url{https://github.com/jpascher/T0-Time-Mass-Duality/blob/main/2/pdf/QFT_De.pdf}

\bibitem{qmdeterministic}
Pascher, J. (2025).
\textit{Deterministische Quantenmechanik in T0}.
HTL Leonding, Österreich.
\url{https://github.com/jpascher/T0-Time-Mass-Duality/blob/main/2/pdf/QM-DetrmisticDe.pdf}

% ========================================
% Repository und Online-Ressourcen
% ========================================

\bibitem{t0repository}
Pascher, J. (2025).
\textit{T0-Time-Mass-Duality: Vollständiges Framework-Repository}.
GitHub Repository.
\url{https://github.com/jpascher/T0-Time-Mass-Duality}

\bibitem{t0website}
Pascher, J. (2025).
\textit{Interaktive T0 Framework-Exploration}.
Interaktive Website.
\url{https://jpascher.github.io/T0-Time-Mass-Duality/}

% ============================================================
% ENDE DER BIBLIOGRAPHIE-EINTRÄGE
% ============================================================

	
	% Dokumentspezifische Referenzen hinzufügen
	% \bibitem{eigeneReferenz}
	% Autor, A. (Jahr).
	% \textit{Titel}.
	% Verlag.
	
\end{thebibliography}

\end{document}
