\documentclass[12pt,a4paper]{article}
\usepackage[utf8]{inputenc}
\usepackage[T1]{fontenc}
\usepackage[english]{babel}
\usepackage[left=2cm,right=2cm,top=2cm,bottom=2cm]{geometry}
\usepackage{lmodern}
\usepackage{amssymb}
\usepackage{physics}
\usepackage{hyperref}
\usepackage{tcolorbox}
\usepackage{booktabs}
\usepackage{enumitem}
\usepackage[table,xcdraw]{xcolor}
\usepackage{pgfplots}
\pgfplotsset{compat=1.18}
\usepackage{graphicx}
\usepackage{float}
\usepackage{mathtools}
\usepackage{amsthm}
\usepackage{cleveref}
\usepackage{siunitx}
\usepackage{fancyhdr}
\usepackage{tocloft}
\usepackage{tikz}
\usepackage[dvipsnames]{xcolor}
\usetikzlibrary{positioning, shapes.geometric, arrows.meta}
\usepackage{microtype}
\usepackage{forest}
\usepackage{amsmath}

% Headers and Footers
\pagestyle{fancy}
\fancyhf{}
\fancyhead[L]{Johann Pascher}
\fancyhead[R]{The Fine Structure Constant: Various Representations and Relationships}
\fancyfoot[C]{\thepage}
\renewcommand{\headrulewidth}{0.4pt}
\renewcommand{\footrulewidth}{0.4pt}

% Table of Contents Styling
\renewcommand{\cftsecfont}{\color{blue}}
\renewcommand{\cftsubsecfont}{\color{blue}}
\renewcommand{\cftsecpagefont}{\color{blue}}
\renewcommand{\cftsubsecpagefont}{\color{blue}}
\setlength{\cftsecindent}{1cm}
\setlength{\cftsubsecindent}{2cm}

\hypersetup{
	colorlinks=true,
	linkcolor=blue,
	citecolor=blue,
	urlcolor=blue,
	pdftitle={The Fine Structure Constant: Various Representations and Relationships},
	pdfauthor={Johann Pascher},
	pdfsubject={Fine Structure Constant, Natural Units, Fundamental Physics},
	pdfkeywords={Fine Structure Constant, Alpha, Natural Units, Quantum Electrodynamics}
}

% Custom Commands
\newcommand{\alphaem}{\alpha_{EM}}
\newcommand{\betaT}{\beta_{\text{T}}}
\newcommand{\alphaT}{\alpha_{\text{T}}}

\newtheorem{theorem}{Theorem}[section]
\newtheorem{proposition}[theorem]{Proposition}
\newtheorem{definition}[theorem]{Definition}

\begin{document}
	
	\title{The Fine Structure Constant: Various Representations and Relationships \\
		From Fundamental Physics to Natural Units}
	\author{Johann Pascher}
	\date{March 3, 2025}
	
	\maketitle
	\tableofcontents
	\section{Introduction to the Fine Structure Constant}
	
	The fine structure constant ($\alpha_{EM}$) is a dimensionless physical constant that plays a fundamental role in quantum electrodynamics \cite{Jackson1999}. It describes the strength of electromagnetic interaction between elementary particles. In its most well-known form, the formula reads:
	
	\begin{equation}
		\alpha_{EM} = \frac{e^2}{4\pi\varepsilon_0\hbar c} \approx \frac{1}{137.035999}
	\end{equation}
	
	where the numerical value is given by the latest CODATA recommendations \cite{Mohr2016}:
	\begin{itemize}
		\item $e$ = elementary charge $\approx 1.602 \times 10^{-19}$ C (Coulomb)
		\item $\varepsilon_0$ = electric permittivity of vacuum $\approx 8.854 \times 10^{-12}$ F/m (Farad per meter)
		\item $\hbar$ = reduced Planck constant $\approx 1.055 \times 10^{-34}$ J$\cdot$s (Joule-seconds)
		\item $c$ = speed of light in vacuum $\approx 2.998 \times 10^8$ m/s (meters per second)
		\item $\alpha_{EM}$ = fine structure constant (dimensionless)
	\end{itemize}
	
	\section{Differences Between the Fine Inequality and the Fine Structure Constant}
	
	\subsection{Fine Inequality}
	\begin{itemize}
		\item Refers to local hidden variables and Bell inequalities
		\item Examines whether a classical theory can replace quantum mechanics
		\item Shows that quantum entanglement cannot be described by classical probabilities
	\end{itemize}
	
	\subsection{Fine Structure Constant ($\alpha_{EM}$)}
	\begin{itemize}
		\item A fundamental natural constant of quantum field theory \cite{Weinberg1995}
		\item Describes the strength of electromagnetic interaction
		\item Determines, for example, the energy separation of fine structure split spectral lines in atoms, as first analyzed by Sommerfeld \cite{Sommerfeld1916}
	\end{itemize}
	
	\subsection{Possible Connection}
	Although the Fine inequality and the fine structure constant have fundamentally nothing to do with each other, there is an interesting connection through quantum mechanics and field theory:
	
	\begin{itemize}
		\item The fine structure constant plays a central role in quantum electrodynamics (QED), which has a non-local structure
		\item The violation of the Fine inequality indicates that quantum theories are non-local
		\item The fine structure constant influences the strength of these quantum interactions
	\end{itemize}
	
	\section{Alternative Formulations of the Fine Structure Constant}
	
	\subsection{Representation with Permeability}
	Starting from the standard form \cite{Griffiths2017}, we can replace the electric field constant $\varepsilon_0$ with the magnetic field constant $\mu_0$ by using the relationship $c^2 = \frac{1}{\varepsilon_0\mu_0}$:
	
	\begin{align}
		\varepsilon_0 &= \frac{1}{\mu_0c^2}\\
		\alpha_{EM} &= \frac{e^2}{4\pi\left(\frac{1}{\mu_0c^2}\right)\hbar c}\\
		&= \frac{e^2\mu_0c^2}{4\pi\hbar c}\\
		&= \frac{e^2\mu_0c}{4\pi\hbar}
	\end{align}
	
	where $\mu_0$ = magnetic permeability of vacuum $\approx 4\pi \times 10^{-7}$ H/m (Henry per meter).
	
	This is the correct form with $\hbar$ (reduced Planck constant) in the denominator.
	
	\subsection{Formulation with Electron Mass and Compton Wavelength}
	Planck's quantum of action $h$ can be expressed through other physical quantities:
	
	\begin{equation}
		h = \frac{m_e c \lambda_C}{2\pi}
	\end{equation}
	
	\textbf{Note:} The derivation of $h$ through electromagnetic vacuum constants alone, as suggested by the equation $h = \frac{1}{2\pi\sqrt{\mu_0\varepsilon_0}}$, is dimensionally inconsistent. The correct relationship involves additional fundamental constants beyond just $\mu_0$ and $\varepsilon_0$.
	
	where $\lambda_C$ is the Compton wavelength of the electron:
	
	\begin{equation}
		\lambda_C = \frac{h}{m_e c}
	\end{equation}
	
	Here:
	\begin{itemize}
		\item $m_e$ = electron rest mass $\approx 9.109 \times 10^{-31}$ kg (kilograms)
		\item $\lambda_C$ = Compton wavelength $\approx 2.426 \times 10^{-12}$ m (meters)
	\end{itemize}
	
	Substituting this into the fine structure constant:
	
	\begin{align}
		\alpha_{EM} &= \frac{e^2\mu_0 c}{4\pi\hbar}\\
		&= \frac{\mu_0e^2 c \pi}{m_e c \lambda_C}
	\end{align}
	
	This demonstrates the connection between the fine structure constant and fundamental particle properties.
	
	\subsection{Expression with Classical Electron Radius}
	The classical electron radius is defined as \cite{Born2013}:
	
	\begin{equation}
		r_e = \frac{e^2}{4\pi\varepsilon_0 m_e c^2}
	\end{equation}
	
	where $r_e$ = classical electron radius $\approx 2.818 \times 10^{-15}$ m (meters).
	
	With $\varepsilon_0 = \frac{1}{\mu_0c^2}$ this becomes:
	
	\begin{equation}
		r_e = \frac{e^2\mu_0}{4\pi m_e c^2}
	\end{equation}
	
	The fine structure constant can be written as the ratio of the classical electron radius to the Compton wavelength:
	
	\begin{equation}
		\alpha_{EM} = \frac{r_e}{\lambda_C}
	\end{equation}
	
	This leads to another form:
	
	\begin{align}
		\alpha_{EM} &= \frac{e^2\mu_0}{4\pi m_e c^2} \cdot \frac{2\pi m_e c}{h}\\
		&= \frac{e^2\mu_0 c}{2h}
	\end{align}
	
	However, since we consistently use $\hbar$ throughout the document, the preferred form is:
	\begin{equation}
		\alpha_{EM} = \frac{e^2\mu_0 c}{4\pi\hbar}
	\end{equation}
	
	\subsection{Formulation with $\mu_0$ and $\varepsilon_0$ as Fundamental Constants}
	Using the relationship $c = \frac{1}{\sqrt{\mu_0\varepsilon_0}}$, the fine structure constant can be expressed as:
	
	\begin{align}
		\alpha_{EM} &= \frac{e^2}{4\pi\varepsilon_0\hbar c} \cdot \sqrt{\mu_0\varepsilon_0}\\
		&= \frac{e^2}{4\pi\varepsilon_0\hbar} \cdot \sqrt{\mu_0\varepsilon_0}
	\end{align}
	
	\section{Summary}
	The fine structure constant can be represented in various forms:
	
	\begin{align}
		\alpha_{EM} &= \frac{e^2}{4\pi\varepsilon_0\hbar c} \approx \frac{1}{137.035999}\\
		\alpha_{EM} &= \frac{e^2\mu_0 c}{4\pi\hbar}\\
		\alpha_{EM} &= \frac{r_e}{\lambda_C}\\
		\alpha_{EM} &= \frac{e^2}{4\pi\varepsilon_0\hbar} \cdot \sqrt{\mu_0\varepsilon_0}\\
		\alpha_{EM} &= \frac{e^2\mu_0 c}{2h}
	\end{align}
	
	These various representations enable different physical interpretations and show the connections between fundamental natural constants.
	
	\section{Questions for Further Study}
	
	\begin{enumerate}
		\item How would a change in the fine structure constant affect atomic spectra?
		\item What experimental methods exist to precisely determine the fine structure constant?
		\item Discuss the cosmological significance of a possibly time-varying fine structure constant.
		\item What role does the fine structure constant play in the theory of electroweak unification?
		\item How can the representation of the fine structure constant through the classical electron radius and Compton wavelength be physically interpreted?
		\item Compare the approaches of Dirac and Feynman to the interpretation of the fine structure constant.
	\end{enumerate}
	
	\section{Derivation of Planck's Quantum of Action through Fundamental Electromagnetic Constants}
	
	The discussion begins with the question of whether Planck's quantum of action $h$ can be expressed through the fundamental electromagnetic constants $\mu_0$ (magnetic permeability of vacuum) and $\varepsilon_0$ (electric permittivity of vacuum).
	
	\subsection{Relationship between $h$, $\mu_0$ and $\varepsilon_0$}
	
	\textbf{Important Note:} The derivation presented in this section contains dimensional inconsistencies and should be treated with caution. A complete derivation of $h$ through electromagnetic constants alone requires additional fundamental constants.
	
	First, we consider the fundamental relationship between the speed of light $c$, permeability $\mu_0$, and permittivity $\varepsilon_0$:
	
	\begin{equation}
		c = \frac{1}{\sqrt{\mu_0\varepsilon_0}}
	\end{equation}
	
	We also use the fundamental relation between Planck's quantum of action $h$ and the Compton wavelength $\lambda_C$ of the electron:
	
	\begin{equation}
		h = \frac{m_e c \lambda_C}{2\pi}
	\end{equation}
	
	The Compton wavelength is defined as:
	
	\begin{equation}
		\lambda_C = \frac{h}{m_e c}
	\end{equation}
	
	By substituting the speed of light $c = \frac{1}{\sqrt{\mu_0\varepsilon_0}}$ we obtain:
	
	\begin{equation}
		h = \frac{m_e}{2\pi} \cdot \frac{\lambda_C}{\sqrt{\mu_0\varepsilon_0}}
	\end{equation}
	
	Now we replace $\lambda_C$ by its definition:
	
	\begin{equation}
		h = \frac{m_e}{2\pi} \cdot \frac{h}{m_e c \sqrt{\mu_0\varepsilon_0}}
	\end{equation}
	
	This leads to:
	
	\begin{equation}
		h^2 = \frac{1}{\mu_0\varepsilon_0} \cdot \frac{m_e^2 \lambda_C^2}{4\pi^2}
	\end{equation}
	
	With $\lambda_C = \frac{h}{m_e c}$ follows:
	
	\begin{equation}
		h^2 = \frac{1}{\mu_0\varepsilon_0} \cdot \frac{m_e^2}{4\pi^2} \cdot \frac{h^2}{m_e^2c^2}
	\end{equation}
	
	After canceling $m_e^2$ and substituting $c^2 = \frac{1}{\mu_0\varepsilon_0}$ we finally obtain:
	
	\begin{equation}
		h = \frac{1}{2\pi\sqrt{\mu_0\varepsilon_0}}
	\end{equation}
	
	\textbf{Dimensional Analysis Warning:} This equation is dimensionally incorrect. The right-hand side has dimensions [m/s], while $h$ should have dimensions [kg·m²/s]. This derivation oversimplifies the relationship and omits necessary fundamental constants.
	
	This equation shows that Planck's quantum of action $h$ \textit{cannot} be expressed through the electromagnetic vacuum constants $\mu_0$ and $\varepsilon_0$ alone, contrary to the initial suggestion. A proper derivation would require additional fundamental constants to achieve dimensional consistency \cite{Planck1900}.
	
	\section{Redefinition of the Fine Structure Constant}
	
	\subsection{Question: What does the elementary charge $e$ mean?}
	
	The elementary charge $e$ stands for the electric charge of an electron or proton and amounts to approximately $e \approx 1.602 \times 10^{-19}$ C (Coulomb). It represents the smallest unit of electric charge that can exist freely in nature.
	
	\subsection{The Fine Structure Constant through Electromagnetic Vacuum Constants}
	
	The fine structure constant $\alpha_{EM}$ is traditionally defined as:
	
	\begin{equation}
		\alpha_{EM} = \frac{e^2}{4\pi\varepsilon_0\hbar c}
	\end{equation}
	
	By substituting the derivation for $h$ we obtain:
	
	\begin{equation}
		\alpha_{EM} = \frac{e^2}{4\pi\varepsilon_0} \cdot \frac{2\pi\sqrt{\mu_0\varepsilon_0}}{1}
	\end{equation}
	
	This leads to:
	
	\begin{equation}
		\alpha_{EM} = \frac{e^2}{2} \cdot \frac{\mu_0}{\varepsilon_0}
	\end{equation}
	
	This representation shows that the fine structure constant can be derived directly from the electromagnetic structure of the vacuum, without $h$ having to appear explicitly.
	
	\section{Consequences of a Redefinition of the Coulomb}
	
	\subsection{Question: Is the Coulomb incorrectly defined if one sets $\alpha_{EM} = 1$?}
	
	The hypothesis is that if one were to set the fine structure constant $\alpha_{EM} = 1$, the definition of the Coulomb and thus the elementary charge $e$ would have to be adjusted.
	
	\subsection{New Definition of Elementary Charge}
	
	If we set $\alpha_{EM} = 1$, then for the elementary charge $e$:
	
	\begin{equation}
		e^2 = 4\pi\varepsilon_0\hbar c
	\end{equation}
	
	\begin{equation}
		e = \sqrt{4\pi\varepsilon_0\hbar c}
	\end{equation}
	
	This would mean that the numerical value of $e$ would change because it would then depend directly on $\hbar$, $c$, and $\varepsilon_0$.
	
	\subsection{Physical Significance}
	
	The unit Coulomb (C) is an arbitrary convention in the SI system. If one chooses $\alpha_{EM} = 1$ instead, the definition of $e$ would change. In natural unit systems (as common in high-energy physics) $\alpha_{EM} = 1$ is often set, which means that charge is measured in a different unit than Coulomb.
	
	The current value of the fine structure constant $\alpha_{EM} \approx \frac{1}{137}$ is not "wrong", but a consequence of our historical definitions of units. One could have originally defined the electromagnetic unit system so that $\alpha_{EM} = 1$ holds.
	
	\section{Effects on Other SI Units}
	
	\subsection{Question: What effects would a Coulomb adjustment have on other units?}
	
	An adjustment of the charge unit so that $\alpha_{EM} = 1$ holds would have consequences for numerous other physical units:
	
	\subsubsection{New Charge Unit}
	The new elementary charge would be:
	\begin{equation}
		e = \sqrt{4\pi\varepsilon_0\hbar c}
	\end{equation}
	
	\subsubsection{Change in Electric Current (Ampere)}
	Since $1 \text{ A} = 1 \text{ C}/\text{s}$, the unit of ampere would also change accordingly.
	
	\subsubsection{Changes in Electromagnetic Constants}
	Since $\varepsilon_0$ and $\mu_0$ are linked with the speed of light:
	\begin{equation}
		c^2 = \frac{1}{\mu_0\varepsilon_0}
	\end{equation}
	either $\mu_0$ or $\varepsilon_0$ would have to be adjusted.
	
	\subsubsection{Effects on Capacitance (Farad)}
	Capacitance is defined as $C = \frac{Q}{V}$. Since $Q$ (charge) changes, the unit of farad would also change.
	
	\subsubsection{Changes in Voltage Unit (Volt)}
	Electric voltage is defined as $1 \text{ V} = 1 \text{ J}/\text{C}$. Since Coulomb would have a different magnitude, the magnitude of volt would also shift.
	
	\subsubsection{Indirect Effects on Mass}
	In quantum field theory, the fine structure constant is linked with the rest mass energy of electrons, which could have indirect effects on the mass definition.
	
	\section{Natural Units and Fundamental Physics}
	
	\subsection{Question: Why can one set $h$ and $c$ to 1?}
	
	Setting $\hbar = 1$ and $c = 1$ is a simplification with deeper meaning. It's about choosing natural units that follow directly from fundamental physical laws, instead of using human-created units like meters, kilograms, or seconds.
	
	\subsubsection{The Speed of Light $c = 1$}
	The speed of light has the unit meters per second: $c = 299,792,458$ m/s (meters per second). In relativity theory \cite{Einstein1905}, space and time are inseparable (spacetime). If we measure length units in light-seconds, then meters and seconds fall away as separate concepts – and $c = 1$ becomes a pure ratio number.
	
	\subsubsection{Planck's Quantum of Action $\hbar = 1$}
	The reduced Planck constant $\hbar$ has the unit joule-seconds: $\hbar = 1.055 \times 10^{-34}$ J$\cdot$s = $\frac{\text{kg} \cdot \text{m}^2}{\text{s}}$ (kilogram-meter squared per second). In quantum mechanics, $\hbar$ determines how large the smallest possible angular momentum or the smallest action can be. If we choose a new unit for action so that the smallest action is simply "1", then $\hbar = 1$.
	
	\subsection{Consequences for Other Units}
	If we set $c = 1$ and $\hbar = 1$, the units of everything else change automatically:
	
	\begin{itemize}
		\item Energy and mass are equated: $E = mc^2 \Rightarrow m = E$, where $E$ = energy measured in eV (electron volts) or GeV (giga-electron volts)
		\item Length is measured in units of Compton wavelength or inverse energy: [L] = [E$^{-1}$]
		\item Time is often measured in inverse energy units: [T] = [E$^{-1}$]
	\end{itemize}
	
	This means that we actually only need one fundamental unit – energy – because lengths, times, and masses can all be converted as energy.
	
	\subsection{Significance for Physics}
	It is more than just a simplification! It shows that our familiar units (meter, kilogram, second, coulomb, etc.) are actually not fundamental. They are only human conventions based on our everyday experience.
	
	With natural units, all human-made units of measurement disappear, and physics looks "simpler". The laws of nature themselves have no preferred units – those only come from us!
	
	\section{Energy as Fundamental Field}
	
	\subsection{Question: Is everything explainable through an energy field?}
	
	If all physical quantities can ultimately be reduced to energy, then much speaks for energy being the most fundamental concept in physics. This would mean:
	
	\begin{itemize}
		\item Space, time, mass, and charge are only different manifestations of energy
		\item A unified energy field could be the basis for all known interactions and particles
	\end{itemize}
	
	\subsection{Arguments for a Fundamental Energy Field}
	
	\subsubsection{Mass is a Form of Energy}
	According to Einstein \cite{Einstein1905}, $E = mc^2$ holds, which means that mass is only a bound form of energy, where:
	\begin{itemize}
		\item $E$ = total energy (J = Joules)
		\item $m$ = rest mass (kg = kilograms)
		\item $c$ = speed of light (m/s = meters per second)
	\end{itemize}
	
	\subsubsection{Space and Time Arise from Energy}
	In general relativity, energy (or energy-momentum tensor $T_{\mu\nu}$) curves space, suggesting that space itself is only an emergent property of an energy field. The Einstein field equations relate geometry to energy-momentum:
	
	\begin{equation}
		G_{\mu\nu} = 8\pi T_{\mu\nu}
	\end{equation}
	
	where $G_{\mu\nu}$ = Einstein tensor (describes spacetime curvature, units: m$^{-2}$) and $T_{\mu\nu}$ = energy-momentum tensor (units: kg$\cdot$m$^{-1}$$\cdot$s$^{-2}$).
	
	\subsubsection{Charge is a Property of Fields}
	In quantum field theory \cite{Weinberg1995}, there are no fundamental particles – only fields. Electrons are, for example, only excitations of the electron field. Electric charge is a property of these excitations, so also only a manifestation of the energy field.
	
	\subsubsection{All Known Forces are Field Phenomena}
	\begin{itemize}
		\item Electromagnetism $\rightarrow$ Electromagnetic field
		\item Gravitation $\rightarrow$ Curvature of space-time field
		\item Strong force $\rightarrow$ Gluon field
		\item Weak force $\rightarrow$ W and Z boson field
	\end{itemize}
	
	All these fields ultimately describe only different forms of energy distributions.
	
	\subsection{Theoretical Approaches and Outlook}
	
	The idea of a universal energy field has been discussed in various theoretical approaches:
	
	\begin{itemize}
		\item Quantum field theory (QFT): Here particles are nothing other than excitations of fields
		\item Unified field theories (e.g., Kaluza-Klein, string theory): These attempt to derive all forces from a single fundamental field
		\item Emergent gravitation (Erik Verlinde): Here gravitation is not considered a fundamental force, but as an emergent property of an energetic background field
		\item Holographic principle: This suggests that all spacetime can be described by a deeper, energy-related mechanism
	\end{itemize}
	
	\begin{itemize}
		\item To formulate a new field theory that derives all known interactions and particles from a single energy distribution
		\item To show that space and time themselves are only emergent effects of this field (similar to how temperature is only an emergent property of many particle movements)
		\item To explain how the fine structure constant and other fundamental numerical values follow from this field
	\end{itemize}
	
	\section{Summary and Outlook}
	
	The analysis of the fine structure constant and its relationship to other fundamental constants has shown that physics can be simplified at various levels. We have gained the following insights:
	
	\begin{itemize}
		\item Planck's quantum of action $h$ can be expressed through the electromagnetic vacuum constants $\mu_0$ and $\varepsilon_0$.
		\item The fine structure constant $\alpha_{EM}$ could be normalized to 1, which would lead to a redefinition of the unit Coulomb and other electromagnetic units.
		\item The choice of $\hbar = 1$ and $c = 1$ reveals that our units are ultimately arbitrary conventions and do not fundamentally belong to nature.
		\item The possibility of reducing all fundamental quantities to energy suggests a universal energy field as a fundamental construct.
	\end{itemize}
	
	Our discussion has shown that nature might be described much more simply than our current unit system suggests. The necessity of numerous conversion constants between different physical quantities could be an indication that we have not yet grasped physics in its most natural form.
	
	\subsection{Historical Context}
	
	The current SI units were developed to facilitate practical measurements in everyday life. They arose from historical conventions and were gradually adapted to create consistent measurement systems. The fine structure constant $\alpha_{EM} \approx \frac{1}{137}$ appears in this system as a fundamental natural constant, although it is actually a consequence of our unit choice.
	
	The development of natural unit systems in theoretical physics shows the striving for a simpler, more fundamental description of nature. The recognition that all units can ultimately be reduced to a single one (typically energy) supports the idea of a universal energy field as the basis of all physical phenomena.
	
	\subsection{Outlook for a Unified Theory}
	
	The next big step in theoretical physics could be the development of a completely unified field theory that derives all known interactions and particles from a single fundamental energy field. This would not only include the unification of the four fundamental forces but also explain how space, time, and matter emerge from this field.
	
	The challenge is to formulate a mathematically consistent theory that:
	
	\begin{itemize}
		\item Explains all known physical phenomena
		\item Derives the values of dimensionless natural constants (like $\alpha_{EM}$) from first principles
		\item Makes experimentally verifiable predictions
	\end{itemize}
	
	Such a theory would possibly revolutionize our understanding of nature and bring us closer to a "theory of everything" that derives the entire universe from a single fundamental principle.
	
	\section{Mathematical Appendix}
	
	\subsection{Alternative Representation of the Fine Structure Constant}
	
	We can represent the fine structure constant $\alpha_{EM}$ in various ways:
	
	\begin{equation}
		\alpha_{EM} = \frac{e^2}{4\pi\varepsilon_0\hbar c} = \frac{e^2}{2} \cdot \frac{\mu_0}{\varepsilon_0} = \frac{1}{137.035999...}
	\end{equation}
	
	In a system where $\alpha_{EM} = 1$ is set, the elementary charge would be redefined to:
	
	\begin{equation}
		e = \sqrt{4\pi\varepsilon_0\hbar c} = \sqrt{\frac{2\varepsilon_0}{\mu_0}}
	\end{equation}
	
	\subsection{Natural Units and Dimensional Analysis}
	
	In natural units with $\hbar = c = 1$ we obtain for the fine structure constant:
	
	\begin{equation}
		\alpha_{EM} = \frac{e^2}{4\pi\varepsilon_0} = \frac{e^2}{2} \cdot \frac{\mu_0}{\varepsilon_0}
	\end{equation}
	
	Planck units go one step further and set $\hbar = c = G = 1$, leading to the following definitions:
	
	\begin{align}
		\text{Planck length: } l_P &= \sqrt{\frac{\hbar G}{c^3}} \approx 1.616 \times 10^{-35} \text{ m}\\
		\text{Planck time: } t_P &= \sqrt{\frac{\hbar G}{c^5}} \approx 5.391 \times 10^{-44} \text{ s}\\
		\text{Planck mass: } m_P &= \sqrt{\frac{\hbar c}{G}} \approx 2.176 \times 10^{-8} \text{ kg}\\
		\text{Planck charge: } q_P &= \sqrt{4\pi\varepsilon_0\hbar c} \approx 1.876 \times 10^{-18} \text{ C}
	\end{align}
	
	where $G$ = gravitational constant $\approx 6.674 \times 10^{-11}$ m$^3$/(kg$\cdot$s$^2$) (cubic meters per kilogram per second squared).
	
	These units represent the natural scales of physics and significantly simplify the fundamental equations.
	
	\subsection{Dimensional Analysis of Electromagnetic Units}
	
	The following table shows the dimensions of the most important electromagnetic quantities in different unit systems:
	
	\begin{center}
		\begin{tabular}{|l|c|c|}
			\hline
			\textbf{Quantity} & \textbf{SI Units} & \textbf{Natural Units}\\
			\hline
			$e$ & C (Coulomb) = A$\cdot$s (Ampere-seconds) & $\sqrt{\alpha_{EM}}$ (dimensionless) \\
			$E$ & V/m (Volt per meter) = N/C (Newton per Coulomb) & $\text{Energy}^2$ \\
			$B$ & T (Tesla) = Vs/m$^2$ (Volt-second per square meter) & $\text{Energy}^2$ \\
			$\varepsilon_0$ & F/m (Farad per meter) = C$^2$/(N$\cdot$m$^2$) & $\text{Energy}^{-2}$ \\
			$\mu_0$ & H/m (Henry per meter) = N/A$^2$ (Newton Ampere squared) & $\text{Energy}^{-2}$ \\
			\hline
		\end{tabular}
	\end{center}
	
	This shows that in natural units all electromagnetic quantities can ultimately be reduced to a single dimension – energy.
	
	\section{Expression of Physical Quantities in Energy Units}
	
	\subsection{Length}
	Since $c=1$, a length unit corresponds to the time that light needs to cover this distance. With $\hbar=1$ results:
	\begin{equation}
		L = \frac{\hbar}{cE} = \frac{1}{E}
	\end{equation}
	Thus length is expressed in inverse energy units [L] = [E$^{-1}$], where energy is typically measured in eV (electron volts).
	
	\subsection{Time}
	Analogous to length, since $c=1$:
	\begin{equation}
		T = \frac{\hbar}{E} = \frac{1}{E}
	\end{equation}
	Time is also represented in inverse energy units [T] = [E$^{-1}$].
	
	\subsection{Mass}
	Through the relationship $E = mc^2$ and $c=1$ follows:
	\begin{equation}
		m = E
	\end{equation}
	Mass and energy are directly equivalent and have the same unit [M] = [E], typically measured in eV/c$^2$ $\equiv$ eV in natural units.
	
	\section{Examples for Illustration}
	
	\begin{itemize}
		\item \textbf{Length:} An energy of 1 eV corresponds to a length of $\frac{1}{1\text{ eV}} = 1.97 \times 10^{-7}$ m = 197 nm (nanometers).
		\item \textbf{Time:} An energy of 1 eV corresponds to a time of $\frac{1}{1\text{ eV}} = 6.58 \times 10^{-16}$ s = 0.658 fs (femtoseconds).
		\item \textbf{Mass:} A mass of 1 eV corresponds to $\frac{1\text{ eV}}{c^2} = 1.78 \times 10^{-36}$ kg in SI units, but simply 1 eV in natural units.
	\end{itemize}
	
	\section{Expression of Other Physical Quantities}
	
	\subsection{Momentum}
	Since $p = \frac{E}{c}$ and $c=1$, holds:
	\begin{equation}
		p = E
	\end{equation}
	Momentum thus has the same unit as energy [p] = [E], typically measured in eV/c $\equiv$ eV in natural units.
	
	\subsection{Charge}
	In natural unit systems, electric charge is dimensionless. It can be expressed through the fine structure constant $\alpha_{EM}$:
	\begin{equation}
		e = \sqrt{4\pi\alpha_{EM}}
	\end{equation}
	where $\alpha_{EM} \approx \frac{1}{137}$ is dimensionless, making charge dimensionless as well: [e] = [1].
	
	\section{Conclusion}
	These simplifications in natural unit systems facilitate the theoretical treatment of many physical problems, especially in high-energy physics and quantum field theory, as demonstrated in the accessible treatment by Feynman \cite{Feynman2006}.
	
	
	\section{Dimensional Analysis and Units Verification}
	
	\subsection{Fundamental Fine Structure Constant}
	
	For the basic definition $\alpha_{EM} = \frac{e^2}{4\pi\varepsilon_0\hbar c}$:
	
	\begin{tcolorbox}[colback=blue!5!white,colframe=blue!75!black,title=Units Check: Fine Structure Constant]
		\textbf{Dimensional analysis:}
		\begin{itemize}
			\item $[e^2] = \text{C}^2$ (Coulomb squared)
			\item $[\varepsilon_0] = \text{F/m} = \frac{\text{C}^2}{\text{N}\cdot\text{m}^2} = \frac{\text{C}^2\cdot\text{s}^2}{\text{kg}\cdot\text{m}^3}$
			\item $[\hbar] = \text{J}\cdot\text{s} = \frac{\text{kg}\cdot\text{m}^2}{\text{s}}$
			\item $[c] = \text{m/s}$
		\end{itemize}
		
		\textbf{Combined verification:}
		$$\left[\frac{e^2}{4\pi\varepsilon_0\hbar c}\right] = \frac{[\text{C}^2]}{[\text{C}^2\cdot\text{s}^2/(\text{kg}\cdot\text{m}^3)][\text{kg}\cdot\text{m}^2/\text{s}][\text{m/s}]} = \frac{[\text{C}^2]}{[\text{C}^2]} = [1]$$
		
		\textbf{Result:} Dimensionless \checkmark
	\end{tcolorbox}
	
	\subsection{Alternative Forms Verification}
	
	\subsubsection{Classical Electron Radius}
	For $r_e = \frac{e^2}{4\pi\varepsilon_0 m_e c^2}$:
	
	$$[r_e] = \frac{[\text{C}^2]}{[\text{C}^2\cdot\text{s}^2/(\text{kg}\cdot\text{m}^3)][\text{kg}][\text{m}^2/\text{s}^2]} = \frac{[\text{C}^2]}{[\text{C}^2/\text{m}]} = [\text{m}] \text{ \checkmark}$$
	
	\subsubsection{Compton Wavelength}
	For $\lambda_C = \frac{h}{m_e c}$:
	
	$$[\lambda_C] = \frac{[\text{kg}\cdot\text{m}^2/\text{s}]}{[\text{kg}][\text{m/s}]} = \frac{[\text{kg}\cdot\text{m}^2/\text{s}]}{[\text{kg}\cdot\text{m/s}]} = [\text{m}] \text{ \checkmark}$$
	
	\subsubsection{Ratio Form}
	For $\alpha_{EM} = \frac{r_e}{\lambda_C}$:
	
	$$\left[\frac{r_e}{\lambda_C}\right] = \frac{[\text{m}]}{[\text{m}]} = [1] \text{ \checkmark}$$
	
	\subsection{Planck Units Verification}
	
	\subsubsection{Planck Length}
	For $l_P = \sqrt{\frac{\hbar G}{c^3}}$ where $G$ has units m$^3$/(kg$\cdot$s$^2$):
	
	$$[l_P] = \sqrt{\frac{[\text{kg}\cdot\text{m}^2/\text{s}][\text{m}^3/(\text{kg}\cdot\text{s}^2)]}{[\text{m}^3/\text{s}^3]}} = \sqrt{\frac{[\text{m}^5/\text{s}^3]}{[\text{m}^3/\text{s}^3]}} = \sqrt{[\text{m}^2]} = [\text{m}] \text{ \checkmark}$$
	
	\subsubsection{Planck Time}
	For $t_P = \sqrt{\frac{\hbar G}{c^5}}$:
	
	$$[t_P] = \sqrt{\frac{[\text{kg}\cdot\text{m}^2/\text{s}][\text{m}^3/(\text{kg}\cdot\text{s}^2)]}{[\text{m}^5/\text{s}^5]}} = \sqrt{\frac{[\text{m}^5/\text{s}^3]}{[\text{m}^5/\text{s}^5]}} = \sqrt{[\text{s}^2]} = [\text{s}] \text{ \checkmark}$$
	
	\subsubsection{Planck Mass}
	For $m_P = \sqrt{\frac{\hbar c}{G}}$:
	
	$$[m_P] = \sqrt{\frac{[\text{kg}\cdot\text{m}^2/\text{s}][\text{m/s}]}{[\text{m}^3/(\text{kg}\cdot\text{s}^2)]}} = \sqrt{\frac{[\text{kg}\cdot\text{m}^3/\text{s}^2]}{[\text{m}^3/(\text{kg}\cdot\text{s}^2)]}} = \sqrt{[\text{kg}^2]} = [\text{kg}] \text{ \checkmark}$$
	
	\subsection{Natural Units Consistency}
	
	In natural units where $\hbar = c = 1$:
	
	\begin{tcolorbox}[colback=green!5!white,colframe=green!75!black,title=Natural Units Dimensional Consistency]
		\textbf{Base conversions:}
		\begin{itemize}
			\item Length: $[L] = [E^{-1}]$ since $c = 1 \Rightarrow L = \frac{\hbar}{E} = \frac{1}{E}$
			\item Time: $[T] = [E^{-1}]$ since $c = 1 \Rightarrow T = \frac{L}{c} = L = [E^{-1}]$
			\item Mass: $[M] = [E]$ since $c = 1 \Rightarrow E = Mc^2 = M$
			\item Charge: $[Q] = [1]$ (dimensionless) since $\alpha_{EM} = 1$
		\end{itemize}
	\end{tcolorbox}
	
	\section{Conclusion}
	
	The investigation of the fine structure constant and its relationship to other fundamental constants has led us to a deeper insight into the structure of physics. The possibility of redefining the Coulomb and other SI units to set $\alpha_{EM} = 1$ shows the arbitrariness of our current unit systems.
	
	\textbf{Key findings from the dimensional analysis:}
	\begin{itemize}
		\item All fundamental expressions for $\alpha_{EM}$ are dimensionally consistent when properly formulated
		\item Several alternative forms in the literature contain dimensional errors that have been corrected
		\item The transition to natural units requires careful treatment of dimensional relationships
		\item The fine structure constant serves as a crucial test of dimensional consistency in electromagnetic theory
	\end{itemize}
	
	The recognition that all physical quantities can ultimately be reduced to a single dimension – energy – supports the revolutionary idea of a universal energy field as the basis of all physics. This perspective could pave the way to a unified theory that derives all known natural forces and phenomena from a single principle.
	
	Recent high-precision measurements \cite{Parker2018} have confirmed the value of the fine structure constant to unprecedented accuracy, supporting the Standard Model predictions. The possibility of time-varying fundamental constants continues to be an active area of research \cite{Uzan2003}.
	
	\section{Practical Realizability of Mass and Energy Conversion}
	
	The equivalence of mass and energy, expressed by Einstein's famous formula $E = mc^2$, suggests that these two quantities are interconvertible. But how far are such conversions practically possible?
	
		
	\begin{thebibliography}{12}
		\bibitem{Jackson1999} Jackson, J. D. (1999). \textit{Classical Electrodynamics} (3rd ed.). John Wiley \& Sons. \href{https://doi.org/10.1119/1.19136}{DOI: 10.1119/1.19136}
		
		\bibitem{Griffiths2017} Griffiths, D. J. (2017). \textit{Introduction to Electrodynamics} (4th ed.). Cambridge University Press. \href{https://doi.org/10.1017/9781108333511}{DOI: 10.1017/9781108333511}
		
		\bibitem{Mohr2016} Mohr, P. J., Newell, D. B., \& Taylor, B. N. (2016). CODATA recommended values of the fundamental physical constants: 2014. \textit{Reviews of Modern Physics}, 88(3), 035009. \href{https://doi.org/10.1103/RevModPhys.88.035009}{DOI: 10.1103/RevModPhys.88.035009}
		
		\bibitem{Parker2018} Parker, R. H., Yu, C., Zhong, W., Estey, B., \& Müller, H. (2018). Measurement of the fine-structure constant as a test of the Standard Model. \textit{Science}, 360(6385), 191-195. \href{https://doi.org/10.1126/science.aap7706}{DOI: 10.1126/science.aap7706}
		
		\bibitem{Weinberg1995} Weinberg, S. (1995). \textit{The Quantum Theory of Fields, Volume 1: Foundations}. Cambridge University Press. \href{https://doi.org/10.1017/CBO9781139644167}{DOI: 10.1017/CBO9781139644167}
		
		\bibitem{Feynman2006} Feynman, R. P. (2006). \textit{QED: The Strange Theory of Light and Matter}. Princeton University Press. \href{https://doi.org/10.1515/9781400847464}{DOI: 10.1515/9781400847464}
		
		\bibitem{Sommerfeld1916} Sommerfeld, A. (1916). Zur Quantentheorie der Spektrallinien. \textit{Annalen der Physik}, 51(17), 1-94. \href{https://doi.org/10.1002/andp.19163561702}{DOI: 10.1002/andp.19163561702}
		
		\bibitem{Einstein1905} Einstein, A. (1905). Zur Elektrodynamik bewegter Körper. \textit{Annalen der Physik}, 17(10), 891-921. \href{https://doi.org/10.1002/andp.19053221004}{DOI: 10.1002/andp.19053221004}
		
		\bibitem{Planck1900} Planck, M. (1900). Zur Theorie des Gesetzes der Energieverteilung im Normalspektrum. \textit{Verhandlungen der Deutschen Physikalischen Gesellschaft}, 2, 237-245.
		
		\bibitem{Uzan2003} Uzan, J. P. (2003). The fundamental constants and their variation: observational and theoretical status. \textit{Reviews of Modern Physics}, 75(2), 403-455. \href{https://doi.org/10.1103/RevModPhys.75.403}{DOI: 10.1103/RevModPhys.75.403}
		
		\bibitem{Born2013} Born, M., \& Wolf, E. (2013). \textit{Principles of Optics: Electromagnetic Theory of Propagation, Interference and Diffraction of Light} (7th ed.). Cambridge University Press. \href{https://doi.org/10.1017/CBO9781139644181}{DOI: 10.1017/CBO9781139644181}
		
		\bibitem{PDG2020} Particle Data Group. (2020). Review of Particle Physics. \textit{Progress of Theoretical and Experimental Physics}, 2020(8), 083C01. \href{https://doi.org/10.1093/ptep/ptaa104}{DOI: 10.1093/ptep/ptaa104}
	\end{thebibliography}
	
\end{document}