\documentclass[12pt,a4paper]{article}
\usepackage[utf8]{inputenc}
\usepackage[T1]{fontenc}
\usepackage[ngerman]{babel}
\usepackage{lmodern}
\usepackage{amsmath}
\usepackage{amssymb}
\usepackage{physics}
\usepackage{hyperref}
\usepackage{tcolorbox}
\usepackage{booktabs}
\usepackage{enumitem}
\usepackage[table,xcdraw]{xcolor}
\usepackage[left=2cm,right=2cm,top=2cm,bottom=2cm]{geometry}
\usepackage{pgfplots}
\pgfplotsset{compat=1.18}
\usepackage{graphicx}
\usepackage{float}
\usepackage{fancyhdr}
\usepackage{siunitx}
\usepackage{mathtools}
\usepackage{amsthm}
\usepackage{cleveref}
\usepackage{tocloft}
\usepackage{tikz}
\usepackage[dvipsnames]{xcolor}
\usetikzlibrary{positioning, shapes.geometric, arrows.meta}
\usepackage{microtype}
\usepackage{natbib}
\usepackage{doi}

% Erweiterte Cross-Referencing-Konfiguration
\crefname{equation}{Gl.}{Gln.}
\crefname{section}{Abschn.}{Abschn.}
\crefname{subsection}{Abschn.}{Abschn.}
\crefname{table}{Tab.}{Tabs.}
\crefname{figure}{Abb.}{Abbn.}

% Benutzerdefinierte Befehle
\newcommand{\Tfield}{T(x)}
\newcommand{\alphaEM}{\alpha_{\text{EM}}}
\newcommand{\alphaW}{\alpha_{\text{W}}}
\newcommand{\betaT}{\beta_{\text{T}}}
\newcommand{\Mpl}{M_{\text{Pl}}}
\newcommand{\Tzerot}{T_0(\Tfield)}
\newcommand{\Tzero}{T_0}
\newcommand{\vecx}{\vec{x}}
\newcommand{\vr}{\vec{r}}
\newcommand{\gammaf}{\gamma_{\text{Lorentz}}}
\newcommand{\LCDM}{\Lambda\text{CDM}}
\newcommand{\DTmu}{D_{T,\mu}}
\newcommand{\calL}{\mathcal{L}}
\newcommand{\deq}{\displaystyle}
\newcommand{\e}{\mathrm{e}}
\newcommand{\alphaT}{\alpha_{\text{T}}}
\newcommand{\lP}{\ell_{\text{P}}}

% Kopf- und Fußzeilen-Konfiguration
\pagestyle{fancy}
\fancyhf{}
\fancyhead[L]{Johann Pascher}
\fancyhead[R]{Feldtheoretische Herleitung des $\beta$-Parameters}
\fancyfoot[C]{\thepage}
\renewcommand{\headrulewidth}{0.4pt}
\renewcommand{\footrulewidth}{0.4pt}

% Hyperref-Konfiguration
\hypersetup{
	colorlinks=true,
	linkcolor=blue,
	citecolor=red,
	urlcolor=blue,
	bookmarks=true,
	bookmarksnumbered=true,
	pdfstartview=FitH,
	pdftitle={T0-Modell - Feldtheoretische Herleitung des Beta-Parameters},
	pdfauthor={Johann Pascher},
	pdfsubject={T0-Modell, Beta-Parameter, Natürliche Einheiten, Quantenfeldtheorie},
	pdfkeywords={Zeitfeld, Beta-Parameter, Planck-Einheiten, Allgemeine Relativitätstheorie}
}

% Theorem-Umgebungen
\newtheorem{theorem}{Theorem}[section]
\newtheorem{proposition}[theorem]{Proposition}
\newtheorem{definition}[theorem]{Definition}

\begin{document}
	
	\title{T0-Modell: Feldtheoretische Herleitung des $\beta$-Parameters \\
		in natürlichen Einheiten ($\hbar = c = 1$)}
	\author{Johann Pascher\\
		Abteilung für Kommunikationstechnik\\
		Höhere Technische Bundeslehranstalt (HTL), Leonding, Österreich\\
		\texttt{johann.pascher@gmail.com}}
	\date{\today}
	
	\maketitle
	\tableofcontents
	\newpage
	
	\section{Einführung und Motivation}
	\label{sec:introduction}
	
	Das T0-Modell führt eine fundamentale neue Betrachtungsweise der Raumzeit ein, bei der die Zeit selbst zu einem dynamischen Feld wird. Im Zentrum dieser Theorie steht der dimensionslose $\beta$-Parameter, der die Stärke des Zeitfeldes charakterisiert und eine direkte Verbindung zwischen Gravitation und elektromagnetischen Wechselwirkungen herstellt.
	
	Diese Arbeit konzentriert sich ausschließlich auf die mathematisch rigorose Herleitung des $\beta$-Parameters aus den grundlegenden Feldgleichungen des T0-Modells, ohne die Komplexität zusätzlicher Skalierungsparameter.
	
	\begin{tcolorbox}[colback=blue!5!white,colframe=blue!75!black,title=Zentrales Ergebnis]
		Der $\beta$-Parameter wird hergeleitet als:
		\begin{equation}
			\boxed{\beta = \frac{2Gm}{r}}
		\end{equation}
		wobei $G$ die Gravitationskonstante, $m$ die Masse der Quelle und $r$ die Entfernung zur Quelle ist.
	\end{tcolorbox}
	
	\section{Rahmenwerk natürlicher Einheiten}
	\label{sec:natural_units}
	
	Das T0-Modell verwendet das in der modernen Quantenfeldtheorie \citep{peskin1995,weinberg1995} etablierte System natürlicher Einheiten:
	
	\begin{itemize}
		\item $\hbar = 1$ (reduzierte Planck-Konstante)
		\item $c = 1$ (Lichtgeschwindigkeit)
	\end{itemize}
	
	Dieses System reduziert alle physikalischen Größen auf Energiedimensionen und folgt der von Dirac \citep{dirac1958} etablierten Tradition.
	
	\begin{tcolorbox}[colback=blue!5!white,colframe=blue!75!black,title=Dimensionen in natürlichen Einheiten]
		\begin{itemize}
			\item Länge: $[L] = [E^{-1}]$
			\item Zeit: $[T] = [E^{-1}]$ 
			\item Masse: $[M] = [E]$
			\item Der $\beta$-Parameter: $[\beta] = [1]$ (dimensionslos)
		\end{itemize}
	\end{tcolorbox}
	
	\section{Fundamentale Struktur des T0-Modells}
	\label{sec:fundamental_structure}
	
	\subsection{Zeit-Masse-Dualität}
	\label{subsec:time_mass_duality}
	
	Das zentrale Prinzip des T0-Modells ist die Zeit-Masse-Dualität, die besagt, dass Zeit und Masse invers miteinander verknüpft sind. Diese Beziehung unterscheidet sich fundamental von der konventionellen Behandlung in der allgemeinen Relativitätstheorie \citep{einstein1915,misner1973}.
	
	\begin{table}[htbp]
		\centering
		\begin{tabular}{|l|c|c|c|}
			\hline
			\textbf{Theorie} & \textbf{Zeit} & \textbf{Masse} & \textbf{Referenz} \\
			\hline
			Einstein ART & $dt' = \sqrt{g_{00}} dt$ & $m_0 = \text{const}$ & \citep{einstein1915,misner1973} \\
			Spezielle Relativität & $t' = \gamma t$ & $m_0 = \text{const}$ & \citep{einstein1905} \\
			T0-Modell & $T(x) = \frac{1}{m(x)}$ & $m(x) = \text{dynamisch}$ & Diese Arbeit \\
			\hline
		\end{tabular}
		\caption{Vergleich der Zeit-Masse-Behandlung verschiedener Theorien}
		\label{tab:theory_comparison}
	\end{table}
	
	\subsection{Grundlegende Feldgleichung}
	\label{subsec:field_equation}
	
	Die fundamentale Feldgleichung des T0-Modells wird aus Variationsprinzipien hergeleitet, analog zum Ansatz für Skalärfeldtheorien \citep{weinberg1995}:
	
	\begin{equation}
		\label{eq:field_equation_fundamental}
		\nabla^2 m(x) = 4\pi G \rho(x) \cdot m(x)
	\end{equation}
	
	Diese Gleichung zeigt strukturelle Ähnlichkeit zur Poisson-Gleichung der Gravitation $\nabla^2 \phi = 4\pi G \rho$ \citep{jackson1998}, ist jedoch nichtlinear aufgrund des Faktors $m(x)$ auf der rechten Seite.
	
	Das Zeitfeld folgt direkt aus der inversen Beziehung:
	\begin{equation}
		\label{eq:time_field_definition}
		T(x) = \frac{1}{m(x)}
	\end{equation}
	
	\section{Geometrische Herleitung des $\beta$-Parameters}
	\label{sec:beta_derivation}
	
	\subsection{Sphärisch symmetrische Punktquelle}
	\label{subsec:spherical_solution}
	
	Für eine Punktmassenquelle verwenden wir die etablierte Methodik der Lösung von Einsteins Feldgleichungen \citep{schwarzschild1916,misner1973}. Die Massendichte einer Punktquelle wird durch die Dirac-Deltafunktion beschrieben:
	
	\begin{equation}
		\rho(\vec{x}) = m_0 \cdot \delta^3(\vec{x})
	\end{equation}
	
	wobei $m_0$ die Masse der Punktquelle ist.
	
	\subsection{Lösung der Feldgleichung}
	\label{subsec:field_solution}
	
	Außerhalb der Quelle ($r > 0$), wo $\rho = 0$, reduziert sich die Feldgleichung zu:
	
	\begin{equation}
		\nabla^2 m(r) = 0
	\end{equation}
	
	Der sphärisch symmetrische Laplace-Operator \citep{jackson1998,griffiths1999} ergibt:
	
	\begin{equation}
		\frac{1}{r^2}\frac{d}{dr}\left(r^2 \frac{dm}{dr}\right) = 0
	\end{equation}
	
	Die allgemeine Lösung dieser Gleichung ist:
	
	\begin{equation}
		m(r) = \frac{C_1}{r} + C_2
	\end{equation}
	
	\subsection{Bestimmung der Integrationskonstanten}
	\label{subsec:integration_constants}
	
	\textbf{Asymptotische Randbedingung}: Für große Entfernungen soll das Zeitfeld einen konstanten Wert $T_0$ annehmen:
	\begin{equation}
		\lim_{r \to \infty} T(r) = T_0 \quad \Rightarrow \quad \lim_{r \to \infty} m(r) = \frac{1}{T_0}
	\end{equation}
	
	Daraus folgt: $C_2 = \frac{1}{T_0}$
	
	\textbf{Verhalten am Ursprung}: Verwendung des Gaußschen Satzes \citep{griffiths1999,jackson1998} für eine kleine Kugel um den Ursprung:
	\begin{equation}
		\oint_S \nabla m \cdot d\vec{S} = 4\pi G \int_V \rho(r) m(r) \, dV
	\end{equation}
	
	Für einen kleinen Radius $\epsilon$:
	\begin{equation}
		4\pi \epsilon^2 \left.\frac{dm}{dr}\right|_{r=\epsilon} = 4\pi G m_0 \cdot m(\epsilon)
	\end{equation}
	
	Mit $\frac{dm}{dr} = -\frac{C_1}{r^2}$ und $m(\epsilon) \approx \frac{1}{T_0}$ für kleine $\epsilon$:
	\begin{equation}
		4\pi \epsilon^2 \cdot \left(-\frac{C_1}{\epsilon^2}\right) = 4\pi G m_0 \cdot \frac{1}{T_0}
	\end{equation}
	
	Daraus folgt: $C_1 = \frac{G m_0}{T_0}$
	
	\subsection{Die charakteristische Längenskala}
	\label{subsec:characteristic_length}
	
	Die vollständige Lösung lautet:
	\begin{equation}
		m(r) = \frac{1}{T_0}\left(1 + \frac{G m_0}{r}\right)
	\end{equation}
	
	Das entsprechende Zeitfeld ist:
	\begin{equation}
		T(r) = \frac{T_0}{1 + \frac{G m_0}{r}}
	\end{equation}
	
	Für den praktisch wichtigen Fall $G m_0 \ll r$ erhalten wir die Näherung:
	\begin{equation}
		T(r) \approx T_0\left(1 - \frac{G m_0}{r}\right)
	\end{equation}
	
	Die charakteristische Längenskala, bei der das Zeitfeld signifikant von $T_0$ abweicht, ist:
	\begin{equation}
		\boxed{r_0 = G m_0}
	\end{equation}
	
	Diese Skala ist proportional zum halben Schwarzschild-Radius $r_s = 2GM/c^2 = 2Gm$ in geometrischen Einheiten \citep{misner1973,carroll2004}.
	
	\subsection{Definition des $\beta$-Parameters}
	\label{subsec:beta_definition}
	
	Der dimensionslose $\beta$-Parameter wird definiert als das Verhältnis der charakteristischen Längenskala zur aktuellen Entfernung:
	
	\begin{equation}
		\boxed{\beta = \frac{r_0}{r} = \frac{G m_0}{r}}
	\end{equation}
	
	Dieser Parameter misst die relative Stärke des Zeitfeldes an einem gegebenen Punkt. Für astronomische Objekte können wir die allgemeinere Form schreiben:
	
	\begin{equation}
		\boxed{\beta = \frac{2Gm}{r}}
	\end{equation}
	
	wobei der Faktor 2 aus der vollständigen relativistischen Behandlung stammt, analog zur Entstehung des Schwarzschild-Radius.
	
	\section{Physikalische Interpretation des $\beta$-Parameters}
	\label{sec:physical_interpretation}
	
	\subsection{Dimensionsanalyse}
	\label{subsec:dimensional_analysis}
	
	Die Dimensionslosigkeit des $\beta$-Parameters in natürlichen Einheiten:
	\begin{equation}
		[\beta] = \frac{[G][m]}{[r]} = \frac{[E^{-2}][E]}{[E^{-1}]} = [1]
	\end{equation}
	
	\subsection{Verbindung zur klassischen Physik}
	\label{subsec:classical_connection}
	
	Der $\beta$-Parameter zeigt direkte Verbindungen zu etablierten physikalischen Konzepten:
	
	\begin{itemize}
		\item \textbf{Gravitationspotential}: $\beta$ ist proportional zum Newtonschen Potential $\Phi = -Gm/r$
		\item \textbf{Schwarzschild-Radius}: $\beta = r_s/(2r)$ in geometrischen Einheiten
		\item \textbf{Fluchtgeschwindigkeit}: $\beta$ ist verwandt mit $v_{\text{esc}}^2/c^2$
	\end{itemize}
	
	\subsection{Grenzfälle und Anwendungsbereiche}
	\label{subsec:limiting_cases}
	
	\begin{table}[htbp]
		\centering
		\begin{tabular}{lcc}
			\toprule
			\textbf{Physikalisches System} & \textbf{Typischer $\beta$-Wert} & \textbf{Regime} \\
			\midrule
			Wasserstoffatom & $\sim 10^{-39}$ & Quantenmechanik \\
			Erde (Oberfläche) & $\sim 10^{-9}$ & Schwache Gravitation \\
			Sonne (Oberfläche) & $\sim 10^{-6}$ & Stellare Physik \\
			Neutronenstern & $\sim 0.1$ & Starke Gravitation \\
			Schwarzschild-Horizont & $\beta = 1$ & Grenzfall \\
			\bottomrule
		\end{tabular}
		\caption{Typische $\beta$-Werte für verschiedene physikalische Systeme}
		\label{tab:beta_values}
	\end{table}
	
	\section{Vergleich mit etablierten Theorien}
	\label{sec:theory_comparison}
	
	\subsection{Verbindung zur allgemeinen Relativitätstheorie}
	\label{subsec:gr_connection}
	
	In der allgemeinen Relativitätstheorie charakterisiert der Parameter $rs/r = 2Gm/r$ die Stärke des Gravitationsfeldes. Der T0-Parameter $\beta = 2Gm/r$ ist identisch mit diesem Ausdruck, was eine tiefe Verbindung zwischen beiden Theorien aufzeigt.
	
	\subsection{Unterschiede zum Standardmodell}
	\label{subsec:sm_differences}
	
	Während das Standardmodell der Teilchenphysik die Zeit als externe Parameter behandelt, macht das T0-Modell die Zeit zu einem dynamischen Feld. Der $\beta$-Parameter quantifiziert diese Dynamik und stellt eine messbare Abweichung von der Standardphysik dar.
	
	\section{Experimentelle Vorhersagen}
	\label{sec:experimental_predictions}
	
	\subsection{Zeitdilatationseffekte}
	\label{subsec:time_dilation}
	
	Das T0-Modell sagt eine modifizierte Zeitdilatation vorher:
	\begin{equation}
		\frac{dt}{dt_0} = 1 - \beta = 1 - \frac{2Gm}{r}
	\end{equation}
	
	Diese Beziehung ist identisch mit der Gravitationszeitdilatation der ART in erster Ordnung, bietet jedoch eine fundamentally andere theoretische Grundlage.
	
	\subsection{Spektroskopische Tests}
	\label{subsec:spectroscopic_tests}
	
	Der $\beta$-Parameter könnte durch hochpräzise Spektroskopie getestet werden:
	\begin{itemize}
		\item Gravitationsrotverschiebung in stellaren Spektren
		\item Atomuhr-Experimente in verschiedenen Gravitationspotentialen
		\item Interferometrie mit hoher Präzision
	\end{itemize}
	
	\section{Mathematische Konsistenz}
	\label{sec:mathematical_consistency}
	
	\subsection{Erhaltungssätze}
	\label{subsec:conservation_laws}
	
	Die Herleitung des $\beta$-Parameters respektiert fundamentale Erhaltungssätze:
	\begin{itemize}
		\item \textbf{Energieerhaltung}: Durch die Lagrange-Formulierung gewährleistet
		\item \textbf{Impulserhaltung}: Aus der räumlichen Translationsinvarianz
		\item \textbf{Dimensionskonsistenz}: In allen Herleitungsschritten verifiziert
	\end{itemize}
	
	\subsection{Stabilität der Lösung}
	\label{subsec:solution_stability}
	
	Die sphärisch symmetrische Lösung ist stabil gegen kleine Störungen, was durch Linearisierung um die Grundzustandslösung gezeigt werden kann.
	
	\section{Schlussfolgerungen}
	\label{sec:conclusions}
	
	Diese Arbeit hat den $\beta$-Parameter des T0-Modells aus ersten Prinzipien hergeleitet:
	
	\begin{tcolorbox}[colback=green!5!white,colframe=green!75!black,title=Hauptergebnisse]
		\begin{enumerate}
			\item \textbf{Exakte Herleitung}: $\beta = \frac{2Gm}{r}$ aus der fundamentalen Feldgleichung
			\item \textbf{Dimensionskonsistenz}: Der Parameter ist dimensionslos in natürlichen Einheiten
			\item \textbf{Physikalische Interpretation}: $\beta$ misst die Stärke des dynamischen Zeitfeldes
			\item \textbf{Verbindung zur ART}: Identität mit dem Gravitationsparameter der allgemeinen Relativitätstheorie
			\item \textbf{Testbare Vorhersagen}: Spezifische experimentelle Signaturen vorhergesagt
		\end{enumerate}
	\end{tcolorbox}
	
	Der $\beta$-Parameter stellt somit eine fundamentale dimensionslose Konstante des T0-Modells dar, die eine Brücke zwischen der Quantenfeldtheorie und der Gravitation schlägt.
	
	\subsection{Zukünftige Arbeiten}
	\label{subsec:future_work}
	
	\textbf{Theoretische Entwicklungen}:
	\begin{itemize}
		\item Quantenkorrekturen zum klassischen $\beta$-Parameter
		\item Kosmologische Anwendungen des T0-Modells
		\item Schwarze-Loch-Physik im T0-Rahmenwerk
	\end{itemize}
	
	\textbf{Experimentelle Programme}:
	\begin{itemize}
		\item Präzisionsmessungen der Gravitationszeitdilatation
		\item Laborexperimente mit kontrollierten Massenkonfigurationen
		\item Astrophysikalische Tests mit kompakten Objekten
	\end{itemize}
	
	% Bibliographie
	\bibliographystyle{natbib}
	\begin{thebibliography}{99}
		
		\bibitem[Carroll(2004)]{carroll2004}
		Carroll, S.~M.
		\newblock \textit{Spacetime and Geometry: An Introduction to General Relativity}.
		\newblock Addison-Wesley, San Francisco, CA (2004).
		
		\bibitem[Dirac(1958)]{dirac1958}
		Dirac, P.~A.~M.
		\newblock \textit{The Principles of Quantum Mechanics}.
		\newblock Oxford University Press, Oxford, 4th edition (1958).
		
		\bibitem[Einstein(1905)]{einstein1905}
		Einstein, A.
		\newblock Zur Elektrodynamik bewegter Körper.
		\newblock \textit{Annalen der Physik}, \textbf{17}, 891--921 (1905).
		
		\bibitem[Einstein(1915)]{einstein1915}
		Einstein, A.
		\newblock Die Feldgleichungen der Gravitation.
		\newblock \textit{Sitzungsberichte der Königlich Preußischen Akademie der Wissenschaften}, 844--847 (1915).
		
		\bibitem[Griffiths(1999)]{griffiths1999}
		Griffiths, D.~J.
		\newblock \textit{Introduction to Electrodynamics}.
		\newblock Prentice Hall, Upper Saddle River, NJ, 3rd edition (1999).
		
		\bibitem[Jackson(1998)]{jackson1998}
		Jackson, J.~D.
		\newblock \textit{Classical Electrodynamics}.
		\newblock John Wiley \& Sons, New York, 3rd edition (1998).
		
		\bibitem[Misner et al.(1973)]{misner1973}
		Misner, C.~W., Thorne, K.~S., and Wheeler, J.~A.
		\newblock \textit{Gravitation}.
		\newblock W. H. Freeman and Company, New York (1973).
		
		\bibitem[Peskin \& Schroeder(1995)]{peskin1995}
		Peskin, M.~E. and Schroeder, D.~V.
		\newblock \textit{An Introduction to Quantum Field Theory}.
		\newblock Addison-Wesley, Reading, MA (1995).
		
		\bibitem[Schwarzschild(1916)]{schwarzschild1916}
		Schwarzschild, K.
		\newblock Über das Gravitationsfeld eines Massenpunktes nach der Einsteinschen Theorie.
		\newblock \textit{Sitzungsberichte der Königlich Preußischen Akademie der Wissenschaften}, 189--196 (1916).
		
		\bibitem[Weinberg(1995)]{weinberg1995}
		Weinberg, S.
		\newblock \textit{The Quantum Theory of Fields, Volume I: Foundations}.
		\newblock Cambridge University Press, Cambridge (1995).
		
	\end{thebibliography}
	
\end{document}