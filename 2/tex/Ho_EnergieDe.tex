\documentclass[12pt,a4paper]{article}
\usepackage[utf8]{inputenc}
\usepackage[T1]{fontenc}
\usepackage[ngerman]{babel}
\usepackage[left=2cm,right=2cm,top=2cm,bottom=2cm]{geometry}
\usepackage{lmodern}
\usepackage{amsmath}
\usepackage{amssymb}
\usepackage{physics}
\usepackage{booktabs}
\usepackage{tcolorbox}
\usepackage{siunitx}
\usepackage[table,xcdraw]{xcolor}
\usepackage{hyperref}

\title{$H_0$ und $\kappa$ Parameter: T0-Modell Referenzdokument\\
	\large Mathematische Ableitungen und experimentelle Vergleiche}
\author{Johann Pascher}
\date{\today}

\begin{document}
	
	\maketitle
	
	\section{Einleitung}
	
	Das T0-Modell bietet einen einheitlichen Rahmen zur Ableitung kosmologischer Parameter aus der fundamentalen Feldtheorie. Dieses Dokument präsentiert die mathematischen Ableitungen des Hubble-Parameters $H_0$ und des linearen Potentialparameters $\kappa$ zusammen mit experimentellen Vergleichen. Die Schlüsselerkenntnis ist, dass beide Parameter aus geometrieabhängiger Energiefelddynamik hervorgehen, anstatt empirisch bestimmte Konstanten zu sein.
	
	\section{T0-Modell-Rahmen}
	
	\subsection{Natürliche Einheiten-Konvention}
	In den natürlichen Einheiten des T0-Modells:
	\begin{align}
		\hbar = c = \alpha_{\text{em}} = \beta_t = 1
	\end{align}
	
	\subsection{Fundamentale Feldgleichungen}
	Das T0-Energiefeld erfüllt:
	\begin{align}
		E(x,t) &= \frac{1}{\max(m(x,t), \omega)} \\
		\nabla^2 E &= 4\pi G \rho_E
	\end{align}
	
	wobei $\omega$ die fundamentale Frequenzskala repräsentiert und $\rho_E$ die Energiedichte ist.
	
	\section{Geometrieabhängige $\xi$ Parameter}
	
	\subsection{Kritische Entdeckung: $4\pi$-Faktorkorrekturen}
	
	Durch systematische Analyse wurden geometrieabhängige Korrekturen zum fundamentalen $\xi$-Parameter identifiziert:
	
	\begin{tcolorbox}[colback=blue!5!white,colframe=blue!75!black,title=Geometrieabhängige $\xi$-Parameter]
		\textbf{Flache Geometrie (lokale Physik):}
		\begin{equation}
			\xi_{\text{flach}} = \frac{\lambda_h^2 v^2}{16\pi^3 E_h^2} = 1{,}3165 \times 10^{-4}
		\end{equation}
		
		\textbf{Sphärische Geometrie (kosmologische Physik):}
		\begin{equation}
			\xi_{\text{sphärisch}} = \frac{\lambda_h^2 v^2}{24\pi^{5/2} E_h^2} = 1{,}557 \times 10^{-4}
		\end{equation}
		
		\textbf{Geometrischer Korrekturfaktor:}
		\begin{equation}
			\frac{\xi_{\text{sphärisch}}}{\xi_{\text{flach}}} = \sqrt{\frac{4\pi}{9}} = 1{,}1827
		\end{equation}
	\end{tcolorbox}
	
	\subsection{Physikalischer Ursprung}
	Der Korrekturfaktor $\sqrt{4\pi/9}$ entsteht durch:
	\begin{itemize}
		\item $4\pi$-Faktor: Vollständige Raumwinkelintegration über sphärische Geometrie
		\item Faktor $9 = 3^2$: Dreidimensionale räumliche Normierung
		\item Kombinierter Effekt: Elektromagnetische Feldkorrekturen für sphärische vs. flache Geometrie
	\end{itemize}
	
	\section{Ableitung des $H_0$-Parameters}
	
	\subsection{T0-Theoretische Vorhersage}
	Der Hubble-Parameter ergibt sich aus der Energiefeldhierarchie:
	\begin{align}
		H_0 &= \xi_{\text{sphärisch}}^{15{,}697} \times E_P \\
		&= (1{,}557 \times 10^{-4})^{15{,}697} \times 1{,}2209 \times 10^{19} \text{ GeV} \\
		&= 1{,}490 \times 10^{-42} \text{ GeV} \\
		&= \boxed{69{,}9 \text{ km/s/Mpc}}
	\end{align}
	
	wobei $E_P$ die Planck-Energie ist und der Exponent 15{,}697 aus der Energiekaskadenanalyse hervorgeht.
	
	\subsection{Einheitenumrechnung}
	Von natürlichen Einheiten zu SI-Einheiten:
	\begin{align}
		H_0 &= 1{,}490 \times 10^{-42} \text{ GeV} \times \frac{1{,}602 \times 10^{-10} \text{ J}}{\text{GeV}} \times \frac{1}{1{,}055 \times 10^{-34} \text{ J·s}} \\
		&= 2{,}264 \times 10^{-18} \text{ s}^{-1} \\
		&= 69{,}9 \text{ km/s/Mpc}
	\end{align}
	
	\section{$\kappa$-Parameter}
	
	\subsection{Energieverlustmechanismus}
	Der $\kappa$-Parameter ergibt sich aus dem Energieverlust in Feldgradienten:
	\begin{equation}
		\frac{dE}{dr} = -\xi^2 \omega^2 \frac{2G}{r^2}
	\end{equation}
	
	\subsection{Regimeklassifikation}
	\textbf{Lokales Regime} ($r \ll H_0^{-1}$):
	\begin{equation}
		\kappa = \alpha_\kappa H_0 \xi_{\text{flach}}^2
	\end{equation}
	
	\textbf{Kosmisches Regime} ($r \gg H_0^{-1}$):
	\begin{equation}
		\boxed{\kappa = H_0}
	\end{equation}
	
	\section{Unendliche Energiefelder und $\Lambda_E$-Term}
	
	\subsection{Mathematische Konsistenzanforderung}
	Für unendliche, homogene Energieverteilungen mit $\rho_E(x) = \rho_{E0} = \text{konstant}$ hat die Standard-Energiefeldgleichung keine begrenzte Lösung. Dies erfordert die Einführung eines $\Lambda_E$-Terms:
	
	\begin{equation}
		\nabla^2 E = 4\pi G \rho_{E0} \cdot E + \Lambda_E \cdot E
	\end{equation}
	
	\subsection{Bestimmung von $\Lambda_E$}
	Für einen stabilen homogenen Energiehintergrund $E = E_0 = \text{konstant}$:
	\begin{equation}
		\Lambda_E = -4\pi G \rho_{E0}
	\end{equation}
	
	Unter Verwendung der Friedmann-Gleichungsbeziehung $H_0^2 = \frac{8\pi G \rho_{E0}}{3}$:
	\begin{equation}
		\Lambda_E = -\frac{3H_0^2}{2}
	\end{equation}
	
	\section{Experimentelle Vergleiche}
	
	\subsection{Hubble-Parameter-Messungen}
	
	\begin{table}[htbp]
		\centering
		\begin{tabular}{lccc}
			\toprule
			\textbf{Quelle} & \textbf{$H_0$ (km/s/Mpc)} & \textbf{Unsicherheit} & \textbf{Methode} \\
			\midrule
			\rowcolor{green!20}
			\textbf{T0-Vorhersage} & \textbf{69{,}9} & \textbf{Theorie} & \textbf{Reine Energietheorie} \\
			Planck 2018 (CMB) & 67{,}4 & $\pm$ 0{,}5 & CMB \\
			SH0ES (Riess et al.) & 74{,}0 & $\pm$ 1{,}4 & Cepheiden \\
			H0LiCOW & 73{,}3 & $\pm$ 1{,}7 & Lensing \\
			DES-SN3YR & 67{,}8 & $\pm$ 1{,}3 & Supernovae \\
			\bottomrule
		\end{tabular}
		\caption{T0-Vorhersage vs. experimentelle Messungen von $H_0$}
		\label{tab:h0_comparison}
	\end{table}
	
	\subsection{Übereinstimmungsanalyse}
	\begin{itemize}
		\item \textbf{T0 vs. Planck}: $69{,}9$ vs. $67{,}4$ km/s/Mpc $\rightarrow$ $103{,}7\%$ Übereinstimmung
		\item \textbf{T0 vs. SH0ES}: $69{,}9$ vs. $74{,}0$ km/s/Mpc $\rightarrow$ $94{,}4\%$ Übereinstimmung
		\item \textbf{T0 vs. H0LiCOW}: $69{,}9$ vs. $73{,}3$ km/s/Mpc $\rightarrow$ $95{,}3\%$ Übereinstimmung
		\item \textbf{T0 vs. Durchschnitt}: $69{,}9$ vs. $71{,}6$ km/s/Mpc $\rightarrow$ $97{,}6\%$ Übereinstimmung
	\end{itemize}
	
	\subsection{Auflösung der Hubble-Spannung}
	Die T0-Vorhersage von $H_0 = 69{,}9$ km/s/Mpc bietet einen optimalen Kompromiss:
	\begin{itemize}
		\item Nur $2{,}5$ km/s/Mpc von der Planck-Messung entfernt
		\item Nur $4{,}1$ km/s/Mpc von der SH0ES-Messung entfernt
		\item Liegt innerhalb des Bereichs der meisten experimentellen Unsicherheiten
	\end{itemize}
	
	\section{Skalenhierarchie-Analyse}
	
	\subsection{Energiebasierte Skalenbeziehungen}
	
	\begin{table}[htbp]
		\centering
		\begin{tabular}{lccc}
			\toprule
			\textbf{Skala} & \textbf{Charakteristische Energie} & \textbf{$\xi$-Parameter} & \textbf{Regime} \\
			\midrule
			Planck & $E_P = 1{,}22 \times 10^{19}$ GeV & $\xi = 2$ & Referenz \\
			Higgs (lokal) & $E_h = 125$ GeV & $\xi_{\text{flach}} = 1{,}32 \times 10^{-4}$ & Lokale Physik \\
			Higgs (kosmologisch) & Effektive Skala & $\xi_{\text{sphärisch}} = 1{,}557 \times 10^{-4}$ & Kosmische Physik \\
			Proton & $E_p = 0{,}938$ GeV & $1{,}54 \times 10^{-19}$ & Lokale Physik \\
			Elektron & $E_e = 0{,}511$ MeV & $8{,}37 \times 10^{-23}$ & Lokale Physik \\
			\bottomrule
		\end{tabular}
		\caption{Energieskalen und entsprechende $\xi$-Parameter}
		\label{tab:energy_scales}
	\end{table}
	
	\subsection{Übergangssskala}
	Der Übergang zwischen lokalen und kosmischen Regimen erfolgt bei:
	\begin{equation}
		r_{\text{Übergang}} \sim H_0^{-1} = 1{,}28 \times 10^{26} \text{ m}
	\end{equation}
	
	Diese Skala markiert, wo elektromagnetische Geometriekorrekturen wichtig werden.
	
	\section{Planck-Strom-Verifikation}
	
	\subsection{Standard vs. vollständige Formulierung}
	\textbf{Standard-Literatur (unvollständig):}
	\begin{equation}
		I_P^{\text{unvollständig}} = \sqrt{\frac{c^6\varepsilon_0}{G}} = 9{,}81 \times 10^{24} \text{ A}
	\end{equation}
	
	\textbf{Geometrisch vollständig:}
	\begin{equation}
		I_P^{\text{vollständig}} = \sqrt{\frac{4\pi c^6\varepsilon_0}{G}} = 3{,}479 \times 10^{25} \text{ A}
	\end{equation}
	
	\textbf{CODATA-Referenz:} $I_P = 3{,}479 \times 10^{25}$ A
	
	\textbf{Übereinstimmung:} Vollständige Formulierung erreicht $99{,}98\%$ Genauigkeit vs. $28{,}2\%$ für unvollständige Version.
	
	\section{Mathematischer Rahmen}
	
	\subsection{Energiefeldgleichung}
	\begin{equation}
		\nabla^2 E = 4\pi G \rho_E(x,t) \cdot E
	\end{equation}
	
	\subsection{Modifiziertes Energiepotential}
	\begin{equation}
		\Phi_E(r) = -\frac{GE_{\text{Quelle}}}{r} + \kappa r
	\end{equation}
	
	\subsection{Skalenhierarchie}
	Das T0-Modell verbindet Skalen durch:
	\begin{equation}
		\text{Planck-Skala} \xrightarrow{15{,}697 \text{ Schritte}} \text{Hubble-Skala}
	\end{equation}
	
	wobei jeder Schritt eine Faktor $\xi_{\text{sphärisch}}$-Reduktion beinhaltet.
	
	\section{Universums-Altersberechnung}
	
	Aus dem T0-abgeleiteten $H_0$:
	\begin{align}
		t_{\text{Universum}}^{(T0)} &= \frac{1}{H_0} = \frac{1}{2{,}264 \times 10^{-18} \text{ s}^{-1}} \\
		&= 4{,}42 \times 10^{17} \text{ s} \\
		&= 14{,}0 \text{ Milliarden Jahre}
	\end{align}
	
	\textbf{Beobachtungswert:} $13{,}8 \pm 0{,}2$ Milliarden Jahre
	
	\textbf{Übereinstimmung:} $98{,}6\%$
	
	\section{Wichtige physikalische Erkenntnisse}
	
	\subsection{Keine räumliche Expansion}
	Das T0-Modell interpretiert $H_0$ nicht als Expansionsrate, sondern als:
	\begin{itemize}
		\item Charakteristische Energieskala für Regimeübergänge
		\item Energieverlustrate an das Hintergrund-Zeitfeld
		\item Schwellenwert für kosmische Abschirmungseffekte
	\end{itemize}
	
	\subsection{Rotverschiebungsmechanismus}
	\begin{equation}
		z = \frac{\Delta E}{E} = \frac{H_0 \cdot r}{c} \quad \text{(Energieverlust)}
	\end{equation}
	
	\subsection{Geometrieabhängigkeit}
	Verschiedene physikalische Regime erfordern verschiedene geometrische Behandlungen:
	\begin{itemize}
		\item Lokale Physik: Flache Geometrie ($\xi_{\text{flach}}$)
		\item Kosmologische Physik: Sphärische Geometrie ($\xi_{\text{sphärisch}}$)
		\item Übergang bei Skala $r \sim H_0^{-1}$
	\end{itemize}
	
	\section{Mathematische Konsistenz}
	
	\subsection{Dimensionsverifikation}
	Alle T0-Gleichungen behalten die Dimensionskonsistenz in natürlichen Einheiten bei:
	
	\begin{table}[htbp]
		\centering
		\begin{tabular}{lccc}
			\toprule
			\textbf{Gleichung} & \textbf{Linke Seite} & \textbf{Rechte Seite} & \textbf{Status} \\
			\midrule
			Energiefeld & $[E] = [E]$ & $[1/\max(m,\omega)] = [E^{-1}]$ & \checkmark \\
			Feldgleichung & $[\nabla^2 E] = [E^3]$ & $[4\pi G \rho_E E] = [E^3]$ & \checkmark \\
			Energieverlust & $[dE/dr] = [E^2]$ & $[\xi^2 \omega^2 2G/r^2] = [E^2]$ & \checkmark \\
			$\Lambda_E$-Term & $[\Lambda_E] = [E^2]$ & $[4\pi G \rho_{E0}] = [E^2]$ & \checkmark \\
			$\kappa$-Parameter & $[\kappa] = [E^2]$ & $[H_0 \hbar] = [E^2]$ & \checkmark \\
			\bottomrule
		\end{tabular}
		\caption{Dimensionskonsistenz-Verifikation}
		\label{tab:dimensional_check}
	\end{table}
	
	\subsection{Interne Konsistenz}
	Schlüsselbeziehungen, die das T0-Modell erfüllt:
	\begin{align}
		\Lambda_E &= -\frac{3H_0^2}{2} \quad \text{(Friedmann-Beziehung)} \\
		\kappa &= H_0 \quad \text{(kosmisches Regime)} \\
		\xi_{\text{sphärisch}} &= \xi_{\text{flach}} \times \sqrt{\frac{4\pi}{9}} \quad \text{(elektromagnetische Geometrie)} \\
		H_0 &= 69{,}9 \text{ km/s/Mpc} \quad \text{(theoretische Vorhersage)}
	\end{align}
	
	\section{Schlussfolgerungen}
	
	Die energiebasierte T0-Formulierung leitet erfolgreich den Hubble-Parameter $H_0 = 69{,}9$ km/s/Mpc aus ersten Prinzipien ab und bietet eine optimale Auflösung der Hubble-Spannung. Die Schlüsselentdeckungen umfassen:
	
	\begin{itemize}
		\item Geometrieabhängige $\xi$-Parameter mit $4\pi$-Korrekturen
		\item Direkte Verbindung zwischen Quanten- und kosmologischen Energieskalen
		\item Parameterfreie Ableitung mit über 95\% experimenteller Übereinstimmung
		\item Alternative Interpretation kosmologischer Beobachtungen ohne räumliche Expansion
		\item Energiefeld-Vereinheitlichung von Planck- bis Hubble-Skalen
	\end{itemize}
	
	Die fundamentale Beziehung $\kappa = H_0$ im kosmischen Regime stellt eine direkte Brücke zwischen Energiefeldtheorie und Kosmologie her und deutet darauf hin, dass großräumige kosmische Phänomene aus denselben Prinzipien hervorgehen, die Quanten-Energiefeld-Wechselwirkungen regieren.
	
	\begin{thebibliography}{99}
		
		\bibitem{planck2020}
		Planck Collaboration (2020). Planck 2018 results. VI. Cosmological parameters. \textit{Astronomy \& Astrophysics}, 641, A6.
		
		\bibitem{riess2019}
		Riess, A. G., et al. (2019). Large Magellanic Cloud Cepheid Standards Provide a 1\% Foundation for the Determination of the Hubble Constant and Stronger Evidence for Physics beyond $\Lambda$CDM. \textit{The Astrophysical Journal}, 876, 85.
		
		\bibitem{wong2020}
		Wong, K. C., et al. (2020). H0LiCOW -- XIII. A 2.4 per cent measurement of $H_0$ from lensed quasars: 5.3$\sigma$ tension between early- and late-Universe probes. \textit{Monthly Notices of the Royal Astronomical Society}, 498, 1420-1439.
		
		\bibitem{codata2018}
		CODATA (2018). \textit{CODATA International empfohlene 2018-Werte der fundamentalen physikalischen Konstanten}. NIST.
		
		\bibitem{weinberg2008}
		Weinberg, S. (2008). \textit{Kosmologie}. Oxford University Press.
		
		\bibitem{pascher2025}
		Pascher, J. (2025). \textit{Reine Energie-Formulierung der T0-Theorie: Massenfreier Ansatz zur Fundamentalphysik}.
		
	\end{thebibliography}
	
\end{document}