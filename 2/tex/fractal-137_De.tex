\documentclass[12pt,a4paper]{article}
\usepackage[utf8]{inputenc}
\usepackage[ngerman]{babel}
\usepackage{amsmath,amssymb,amsthm}
\usepackage[T1]{fontenc}
\usepackage{xcolor}
\usepackage{geometry}
\usepackage{fancyhdr}
\usepackage{setspace}
\usepackage{booktabs}
\usepackage{tcolorbox}
\usepackage{siunitx}
\usepackage{hyperref}

\hypersetup{
	colorlinks=true,
	linkcolor=blue,
	citecolor=blue,
	urlcolor=blue,
}

\definecolor{deepblue}{RGB}{0,0,127}
\definecolor{deepred}{RGB}{191,0,0}
\definecolor{deepgreen}{RGB}{0,127,0}

% Header Definition
\pagestyle{fancy}
\fancyhf{}
\fancyhead[L]{\textbf{T0-Theorie: Fraktale Renormierung}}
\fancyhead[R]{\textbf{Johann Pascher, 2025}}
\fancyfoot[C]{\thepage}
\renewcommand{\headrulewidth}{0.4pt}
\setlength{\headheight}{15pt}

% Theoreme und Definitionen
\theoremstyle{definition}
\newtheorem{definition}{Definition}[section]
\newtheorem{theorem}{Theorem}[section]
\newtheorem{lemma}{Lemma}[section]
\newtheorem{corollary}{Korollar}[section]

% Abstände
\setstretch{1.2}

\title{\textbf{Die Fraktale Renormierung der Feinstrukturkonstante in der T0-Theorie}\\[0.5cm]
	\large Überprüfung der Berechnungen mit Fehleranalyse\\[0.3cm]
	\normalsize Basierend auf der Herleitung von Johann Pascher}
\author{Anonymer Überprüfer\\
	\small Basierend auf der Arbeit von Johann Pascher, 2025}
\date{September 2025}

\begin{document}
	
	\maketitle
	
	\begin{abstract}
		Dieses Dokument überprüft die Berechnungen der Feinstrukturkonstanten \(\alpha \approx 1/137.036\) in der T0-Theorie, basierend auf der geometrischen Konstante \(\xi = \frac{4}{3} \times 10^{-4}\), der charakteristischen Energie \(E_0 = \SI{7.398}{\MeV}\) und der fraktalen Dimension \(D_f = 2.94\). Drei Methoden werden analysiert: die elementare Herleitung, die direkte geometrische Berechnung (Weg 1) und die fraktale Renormierung (Weg 2). Bei jeder Berechnung wird vermerkt, ob sie korrekt ist oder Fehler enthält, mit einer detaillierten Analyse der Probleme.
	\end{abstract}
	
	\tableofcontents
	\newpage
	
	\section{Einführung}
	Die T0-Theorie leitet die Feinstrukturkonstante \(\alpha \approx 1/137.036\) aus geometrischen Prinzipien ab. Dieses Dokument überprüft die Berechnungen und hebt Fehler hervor, die in den Formeln für Weg 1 und Weg 2 auftreten. Die elementare Herleitung wird als die robusteste Methode identifiziert.
	
	\section{Grundkonstanten der T0-Theorie}
	Die fundamentalen Parameter sind:
	\begin{align}
		\xi &= \frac{4}{3} \times 10^{-4} \approx 1.333 \times 10^{-4}, \\
		E_0 &= \SI{7.398}{\MeV}, \\
		D_f &= 2.94, \quad D_f^{-1} = \frac{1}{2.94} \approx 0.340136.
	\end{align}
	
	\section{Elementare Herleitung: \(\alpha = \xi \cdot \frac{E_0^2}{(1 \, \text{MeV})^2}\)}
	\subsection{Berechnung}
	Die einfachste Herleitung lautet:
	\begin{equation}
		\alpha = \xi \cdot \frac{E_0^2}{(1 \, \text{MeV})^2}.
	\end{equation}
	Mit \(\xi = 1.333 \times 10^{-4}\), \(E_0 = 7.398 \, \text{MeV}\):
	\begin{align}
		E_0^2 &= (7.398)^2 \approx 54.7296 \, \text{MeV}^2, \\
		\frac{E_0^2}{(1 \, \text{MeV})^2} &= 54.7296, \\
		\alpha &= 1.333 \times 10^{-4} \times 54.7296 \approx 0.007297, \\
		\alpha^{-1} &\approx \frac{1}{0.007297} \approx 137.0.
	\end{align}
	
	\subsection{Fehleranalyse}
	\begin{tcolorbox}[colback=green!5!white,colframe=deepgreen,title=Korrektheit]
		Die Berechnung ist \textbf{korrekt} und liefert \(\alpha^{-1} \approx 137.0\), was nur 0.026\% vom experimentellen Wert \(\alpha^{-1} \approx 137.036\) abweicht. Die Formel ist dimensional konsistent und verwendet nur zwei messbare Parameter (\(\xi\), \(E_0\)). Der Fehler durch die Vereinfachung zu \(\alpha \propto \xi^{11/2}\) wird vermieden, da \(E_0\) ein unabhängiger Parameter ist.
	\end{tcolorbox}
	
	\section{Weg 1: Direkte geometrische Berechnung}
	\subsection{Berechnung}
	Die Formel lautet:
	\begin{equation}
		\alpha^{-1} = 3\pi \times \frac{3}{4} \times 10^4 \times \ln(10^4) \times D_f^{-1} = 137.036,
	\end{equation}
	mit \(\ln(10^4) \approx 9.210\), \(D_f^{-1} \approx 0.340136\).
	
	Schrittweise:
	\begin{align}
		3\pi &\approx 9.4248, \\
		3\pi \times \frac{3}{4} &= 9.4248 \times 0.75 \approx 7.0686, \\
		7.0686 \times 10^4 &= 70686, \\
		70686 \times 9.2104 &\approx 651019.3, \\
		\alpha^{-1} &\approx 651019.3 \times 0.340136 \approx 221291.7.
	\end{align}
	
	\subsection{Fehleranalyse}
	\begin{tcolorbox}[colback=red!5!white,colframe=deepred,title=Fehler]
		Die Berechnung ist \textbf{fehlerhaft}. Der berechnete Wert \(\alpha^{-1} \approx 221291.7\) ist weit entfernt von \(137.036\). Der Faktor \(10^4\) scheint falsch zu sein. Eine Korrektur zu \(10^{-4}\) liefert:
		\begin{align*}
			7.0686 \times 10^{-4} \times 9.2104 \times 0.340136 \approx 0.02214, \\
			\alpha^{-1} \approx \frac{1}{0.02214} \approx 45.17,
		\end{align*}
		was ebenfalls nicht korrekt ist. Die Formel oder die Koeffizienten (z. B. \(10^4\)) sind vermutlich falsch definiert.
	\end{tcolorbox}
	
	\section{Weg 2: Fraktale Renormierung}
	\subsection{Berechnung}
	Die Formel lautet:
	\begin{align}
		\alpha^{-1} &= 1 + \Delta_{\text{frac}}, \\
		\Delta_{\text{frac}} &= \frac{3}{4\pi} \times \xi^{-2} \times D_{\text{frac}}^{-1}, \\
		D_{\text{frac}} &= \left( \frac{\lambda_C^{(\mu)}}{\ell_P} \right)^{D_f - 2},
	\end{align}
	mit \(D_f = 2.94\), \(\xi = \frac{4}{3} \times 10^{-4}\), und \(\alpha^{-1} = 137.0\).
	
	1. **Fraktaler Dämpfungsfaktor**:
	\begin{align}
		\lambda_C^{(\mu)} &\approx \frac{1.973 \times 10^{-13}}{105.66} \approx 1.867 \times 10^{-15} \, \text{m}, \\
		\ell_P &\approx 1.616 \times 10^{-35} \, \text{m}, \\
		\frac{\lambda_C^{(\mu)}}{\ell_P} &\approx 1.155 \times 10^{20}, \\
		D_{\text{frac}} &= (1.155 \times 10^{20})^{0.94} \approx 6.93 \times 10^{18}, \\
		D_{\text{frac}}^{-1} &\approx \frac{1}{6.93 \times 10^{18}} \approx 1.443 \times 10^{-19}.
	\end{align}
	
	2. **Fraktale Korrektur**:
	\begin{align}
		\xi^{-2} &= (7500)^2 = 5.625 \times 10^7, \\
		\frac{3}{4\pi} &\approx 0.23873, \\
		\Delta_{\text{frac}} &\approx 0.23873 \times 5.625 \times 10^7 \times 1.443 \times 10^{-19} \approx 1.938 \times 10^{-12}, \\
		\alpha^{-1} &\approx 1 + 1.938 \times 10^{-12} \approx 1.
	\end{align}
	
	\subsection{Fehleranalyse}
	\begin{tcolorbox}[colback=red!5!white,colframe=deepred,title=Fehler]
		Die Berechnung ist \textbf{fehlerhaft}. Die fraktale Korrektur ergibt \(\Delta_{\text{frac}} \approx 1.938 \times 10^{-12}\), nicht 136, wie im Originaldokument angegeben. Daher ist \(\alpha^{-1} \approx 1\), weit entfernt von 137.0. Der Fehler liegt vermutlich in der Definition von \(\Delta_{\text{frac}}\) oder den verwendeten Werten für \(D_{\text{frac}}\). Selbst mit \(D_{\text{frac}} = 6.7 \times 10^{18}\) (wie im Original) ergibt sich kein korrekter Wert.
	\end{tcolorbox}
	
	\section{Vermeidung des Fehlschlusses \(\alpha \propto \xi^{11/2}\)}
	\subsection{Berechnung}
	Eine falsche Vereinfachung wäre:
	\begin{align}
		\xi &= 1.333 \times 10^{-4}, \\
		\xi^{11/2} &= (1.333 \times 10^{-4})^{5.5} \approx 2.34 \times 10^{-21}, \\
		\alpha^{-1} &\sim \frac{1}{2.34 \times 10^{-21}} \approx 10^{21}.
	\end{align}
	
	\subsection{Fehleranalyse}
	\begin{tcolorbox}[colback=red!5!white,colframe=deepred,title=Fehler]
		Diese Vereinfachung ist \textbf{fehlerhaft}. Sie ignoriert die physikalische Bedeutung von \(E_0 = \SI{7.398}{\MeV}\) als messbaren Parameter (geometrisches Mittel von Elektronen- und Myonmasse). Die korrekte Formel \(\alpha = \xi \cdot \frac{E_0^2}{(1 \, \text{MeV})^2}\) respektiert die Dimensionen und liefert das richtige Ergebnis.
	\end{tcolorbox}
	
	\section{Zusammenfassung}
	\begin{tcolorbox}[colback=deepblue!5!white,colframe=deepblue,title=Zusammenfassung]
		\begin{enumerate}
			\item \textbf{Elementare Herleitung}: \(\alpha = \xi \cdot \frac{E_0^2}{(1 \, \text{MeV})^2}\) ist korrekt und liefert \(\alpha^{-1} \approx 137.0\), mit nur 0.026\% Abweichung vom experimentellen Wert.
			\item \textbf{Weg 1}: Die direkte geometrische Berechnung ist fehlerhaft, da sie \(\alpha^{-1} \approx 221291.7\) ergibt. Der Faktor \(10^4\) ist vermutlich falsch.
			\item \textbf{Weg 2}: Die fraktale Renormierung ist fehlerhaft, da \(\Delta_{\text{frac}} \approx 10^{-12}\) statt 136 ergibt, was zu \(\alpha^{-1} \approx 1\) führt.
			\item \textbf{Fehlschluss \(\xi^{11/2}\)}: Diese Vereinfachung ist dimensionsanalytisch falsch und führt zu absurden Ergebnissen (\(\alpha^{-1} \sim 10^{21}\)).
			\item Die elementare Herleitung ist die robusteste Methode, da sie transparent, dimensional korrekt und nahe am experimentellen Wert ist.
		\end{enumerate}
	\end{tcolorbox}
	
\end{document}