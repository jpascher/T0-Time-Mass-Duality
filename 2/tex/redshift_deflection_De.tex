\documentclass[12pt,a4paper]{article}
\usepackage[utf8]{inputenc}
\usepackage[ngerman]{babel}
\usepackage{amsmath,amssymb,amsfonts,amsthm}
\usepackage{physics}
\usepackage{siunitx}
\usepackage{geometry}
\usepackage{fancyhdr}
\usepackage{enumitem}
\usepackage{booktabs}
\usepackage{longtable}
\usepackage{array}
\usepackage{xcolor}
\usepackage{tcolorbox}
\usepackage{mdframed}
\usepackage{graphicx}
\usepackage{hyperref}

\geometry{margin=2.5cm}
\pagestyle{fancy}
\fancyhf{}
\fancyhead[L]{T0-Theorie: Mathematische Äquivalenzformulierung}
\fancyhead[R]{\thepage}
\fancyfoot[C]{\textit{Energieverlust, Rotverschiebung und Lichtablenkung vereinheitlicht}}

\hypersetup{
	colorlinks=true,
	linkcolor=blue,
	filecolor=magenta,
	urlcolor=cyan,
}

\newcommand{\ts}{\textsuperscript}
\newcommand{\xired}{\xi_{\text{red}}}
\newcommand{\ee}{\text{$\mathrm{e}$}}
\newcommand{\mmu}{\text{$\mu$}}
\newcommand{\ttau}{\text{$\tau$}}
\newcommand{\tfield}{T_{\text{Feld}}}
\newcommand{\efield}{E_{\text{Feld}}}
\newcommand{\dfield}{\delta E}
\newcommand{\echar}{E_{\text{char}}}
\newcommand{\eratio}[2]{\frac{E_{#1}}{E_{#2}}}
\newcommand{\T}[1]{\text{#1}}
\newcommand{\vektor}[1]{\vec{#1}}
\newcommand{\dimcheck}[1]{\textcolor{blue}{[#1]}}
\newcommand{\lp}{\ell_{\text{P}}}
\newcommand{\ep}{E_{\text{P}}}
\newcommand{\alphae}{\alpha_{\text{EM}}}
\newcommand{\alphag}{\alpha_{\text{G}}}
\newcommand{\alphaw}{\alpha_{\text{W}}}
\newcommand{\alphas}{\alpha_{\text{S}}}
\newcommand{\xisi}{\xi_{\text{SI}}}
\newcommand{\xit}{\xi_{\text{T0}}}
\newcommand{\epst}{\varepsilon_{\text{T0}}}

\newmdenv[
linecolor=black,
frametitle={Dimensionsanalyse:},
frametitlebackgroundcolor=gray!20,
backgroundcolor=gray!5,
]{dimanalysis}

\newtcolorbox{wichtig}[1][]{
	colback=yellow!10!white,
	colframe=yellow!50!black,
	fonttitle=\bfseries,
	title=Wichtiger Hinweis,
	#1
}

\newtcolorbox{formel}[1][]{
	colback=blue!5!white,
	colframe=blue!75!black,
	fonttitle=\bfseries,
	title=Schlüsselformel,
	#1
}

\theoremstyle{definition}
\newtheorem{prinzip}{Prinzip}
\newtheorem{beobachtung}{Beobachtung}

\title{\Huge\textbf{Mathematische Äquivalenz in der T0-Theorie}\\\Large Einheitliche Beschreibung von Energieverlust, Rotverschiebung und Lichtablenkung}
\author{Basierend auf der Arbeit von Johann Pascher\\
	Abteilung für Kommunikationstechnik, \\Höhere Technische Bundeslehranstalt (HTL), Leonding, Österreich}
\date{\today}

\begin{document}
	
	\maketitle
	\tableofcontents
	\thispagestyle{fancy}
	\newpage
	
	\section{Einleitung}
	
	Dieses Dokument präsentiert die mathematische Äquivalenz von drei Phänomenen, die in der Standardphysik als separate Effekte behandelt werden, im T0-Modell jedoch vereinheitlicht sind:
	
	\begin{enumerate}
		\item Energieverlust von Photonen während der Ausbreitung
		\item Kosmologische Rotverschiebung
		\item Gravitationsbedingte Lichtablenkung
	\end{enumerate}
	
	Die zentrale Erkenntnis der T0-Theorie besteht darin, dass diese Phänomene verschiedene Manifestationen derselben zugrundeliegenden Feldgleichung sind, nicht separate physikalische Prozesse. Diese Vereinheitlichung wird durch einen einzigen geometrischen Parameter $\xi = \frac{4}{3} \times 10^{-4}$ erreicht, der die Kopplung zwischen dem Energiefeld und der Raumzeitgeometrie bestimmt.
	
	\section{Grundlegende Formeln}
	
	\subsection{Energieverlust von Photonen}
	
	\begin{formel}
		Energieverlustrate:
		\begin{equation}
			\boxed{\frac{dE_\gamma}{dr} = -\xi \frac{E_\gamma^2}{\efield \cdot r}}
		\end{equation}
		wobei $\xi = \frac{4}{3} \times 10^{-4}$ der universelle geometrische Parameter ist.
	\end{formel}
	
	\begin{dimanalysis}
		$\left[\frac{dE_\gamma}{dr}\right] = \frac{[E]}{[L]} = \frac{[E]}{[E^{-1}]} = [E^2]$\\
		$[\xi] = [1]$ (dimensionslos)\\
		$\left[\frac{E_\gamma^2}{\efield \cdot r}\right] = \frac{[E^2]}{[E] \cdot [E^{-1}]} = \frac{[E^2]}{[1]} = [E^2]$ \checkmark
	\end{dimanalysis}
	
	Da $E_\gamma = \frac{hc}{\lambda}$ (oder $E_\gamma = \frac{1}{\lambda}$ in natürlichen Einheiten), kann dies in Bezug auf die Wellenlänge ausgedrückt werden:
	
	\begin{equation}
		\frac{d(1/\lambda)}{dr} = -\xi \frac{(1/\lambda)^2}{\efield \cdot r}
	\end{equation}
	
	Umgestellt:
	\begin{equation}
		\frac{d\lambda}{dr} = \xi \frac{\lambda^2 \cdot \efield}{r}
	\end{equation}
	
	Integration der wellenlängenabhängigen Energieverlustgleichung:
	\begin{equation}
		\int_{\lambda_0}^{\lambda(r)} \frac{d\lambda'}{\lambda'^2} = \xi \efield \int_0^r \frac{dr'}{r'}
	\end{equation}
	
	Dies ergibt:
	\begin{equation}
		\frac{1}{\lambda_0} - \frac{1}{\lambda(r)} = \xi \efield \ln\left(\frac{r}{r_0}\right)
	\end{equation}
	
	Für kleine Korrekturen:
	\begin{equation}
		\lambda(r) \approx \lambda_0 \left(1 + \xi \efield \lambda_0 \ln\left(\frac{r}{r_0}\right)\right)
	\end{equation}
	
	\subsection{Rotverschiebungsformulierung}
	
	Die Rotverschiebung ist definiert als:
	\begin{equation}
		z = \frac{\lambda_{\text{beobachtet}} - \lambda_{\text{emittiert}}}{\lambda_{\text{emittiert}}} = \frac{\lambda(r) - \lambda_0}{\lambda_0}
	\end{equation}
	
	Mit dem zuvor abgeleiteten Ausdruck:
	\begin{equation}
		z \approx \xi \efield \lambda_0 \ln\left(\frac{r}{r_0}\right)
	\end{equation}
	
	Da $\lambda_0 \propto \frac{1}{E_{\gamma,0}}$, können wir schreiben:
	
	\begin{formel}
		Wellenlängenabhängige Rotverschiebung:
		\begin{equation}
			\boxed{z(\lambda) = z_0\left(1 - \alpha \ln\frac{\lambda}{\lambda_0}\right)}
		\end{equation}
		wobei $z_0$ die Referenz-Rotverschiebung und $\alpha$ ein dimensionsloser Parameter ist, der mit $\xi$ zusammenhängt.
	\end{formel}
	
	\begin{dimanalysis}
		$[z(\lambda)] = [1]$\\
		$[z_0] = [1]$\\
		$[\alpha] = [1]$\\
		$\left[\ln\frac{\lambda}{\lambda_0}\right] = \ln\left(\frac{[L]}{[L]}\right) = \ln([1]) = [1]$\\
		$\left[z_0\left(1 - \alpha \ln\frac{\lambda}{\lambda_0}\right)\right] = [1] \cdot ([1] - [1] \cdot [1]) = [1]$ \checkmark
	\end{dimanalysis}
	
	Ein charakteristisches Merkmal dieser Rotverschiebungsformel ist ihre Wellenlängenabhängigkeit, die eine überprüfbare Vorhersage liefert:
	
	\begin{equation}
		\frac{dz}{d\ln\lambda} = -\alpha z_0
	\end{equation}
	
	Dies unterscheidet das T0-Modell von Standardmodellen der Kosmologie, die keine Wellenlängenabhängigkeit vorhersagen ($\frac{dz}{d\ln\lambda} = 0$).
	
	\subsection{Gravitationsbedingte Lichtablenkung}
	
	\begin{formel}
		Modifizierte Gravitationsablenkung:
		\begin{equation}
			\boxed{\theta = \frac{4GM}{bc^2}\left(1 + \xi \frac{E_\gamma}{E_0}\right)}
		\end{equation}
		wobei $\theta$ der Ablenkungswinkel, $M$ die Masse des ablenkenden Objekts, $b$ der Stoßparameter, $E_\gamma$ die Photonenenergie und $E_0$ eine Referenzenergie ist.
	\end{formel}
	
	\begin{dimanalysis}
		$[G] = [E^{-2}]$\\
		$[M] = [E]$\\
		$[b] = [E^{-1}]$\\
		$[c^2] = [1]$ (in natürlichen Einheiten)\\
		$\left[\frac{4GM}{bc^2}\right] = \frac{[E^{-2}][E]}{[E^{-1}][1]} = [1]$ (dimensionslos)\\
		$\left[\xi \frac{E_\gamma}{E_0}\right] = [1] \cdot \frac{[E]}{[E]} = [1]$ (dimensionslos)\\
		$[\theta] = [1] \cdot ([1] + [1]) = [1]$ (dimensionslos) \checkmark
	\end{dimanalysis}
	
	Im Gegensatz zur Allgemeinen Relativitätstheorie, die eine wellenlängenunabhängige Lichtablenkung vorhersagt, führt das T0-Modell eine explizite Energieabhängigkeit ein. Diese energieabhängige Gravitationslinsenbildung führt zu einem modifizierten Einstein-Ring-Radius:
	
	\begin{equation}
		\theta_E(\lambda) = \theta_{E,0} \sqrt{1 + \xi \frac{hc}{\lambda E_0}}
	\end{equation}
	
	Für zwei verschiedene Photonenenergien ist das Verhältnis der Ablenkungswinkel:
	
	\begin{equation}
		\frac{\theta(E_1)}{\theta(E_2)} = \frac{1 + \xi \frac{E_1}{E_0}}{1 + \xi \frac{E_2}{E_0}}
	\end{equation}
	
	Für Fälle, in denen $\xi \frac{E}{E_0} \ll 1$ (typisch für astrophysikalische Beobachtungen), kann dies angenähert werden als:
	
	\begin{equation}
		\frac{\theta(E_1)}{\theta(E_2)} \approx 1 + \xi \frac{E_1 - E_2}{E_0}
	\end{equation}
	
	\section{Vereinheitlichende Geodätengleichung}
	
	Die drei oben beschriebenen Phänomene (Energieverlust, Rotverschiebung und Lichtablenkung) werden im T0-Modell durch eine einzige Geodätengleichung mit Zeitfeldkorrekturen vereinheitlicht:
	
	\begin{formel}
		Universelle Geodätengleichung:
		\begin{equation}
			\boxed{\frac{d^2 x^\mu}{d\lambda^2} + \Gamma^\mu_{\alpha\beta}\frac{dx^\alpha}{d\lambda}\frac{dx^\beta}{d\lambda} = \xi \cdot \partial^\mu \ln(\efield)}
		\end{equation}
		wobei $x^\mu$ die Raumzeitposition, $\lambda$ ein affiner Parameter entlang des Photonenpfades, $\Gamma^\mu_{\alpha\beta}$ die Christoffel-Symbole und $\efield$ das lokale Energiefeld ist.
	\end{formel}
	
	\begin{dimanalysis}
		$[\Gamma^\mu_{\alpha\beta}] = [E]$ (Christoffel-Symbole)\\
		$\left[\frac{dx^\alpha}{d\lambda}\right] = \frac{[E^{-1}]}{[E^{-1}]} = [1]$ (dimensionslos)\\
		$[\partial^\mu \ln(\efield)] = [E] \cdot [1] = [E]$\\
		$[\xi \cdot \partial^\mu \ln(\efield)] = [1] \cdot [E] = [E]$ \checkmark
	\end{dimanalysis}
	
	Die Christoffel-Symbole selbst erhalten Zeitfeldkorrekturen:
	
	\begin{equation}
		\Gamma^\lambda_{\mu\nu} = \Gamma^\lambda_{\mu\nu|0} + \frac{\xi}{2} \left(\delta^\lambda_\mu \partial_\nu \tfield + \delta^\lambda_\nu \partial_\mu \tfield - g_{\mu\nu} \partial^\lambda \tfield\right)
	\end{equation}
	
	wobei $\Gamma^\lambda_{\mu\nu|0}$ die Standard-Christoffel-Symbole, $\tfield$ das Zeitfeld, $\delta^\lambda_\mu$ das Kronecker-Delta und $g_{\mu\nu}$ der metrische Tensor sind.
	
	\begin{wichtig}
		Die mathematische Äquivalenz dieser drei Phänomene bedeutet, dass die T0-Theorie mit einem einzigen Mechanismus erklärt, was das Standardmodell durch verschiedene physikalische Prozesse erklärt. Insbesondere:
		
		\begin{enumerate}
			\item Die kosmologische Rotverschiebung ist nicht die Folge einer räumlichen Ausdehnung, sondern eines allmählichen Energieverlusts der Photonen
			\item Dieser Energieverlust folgt derselben Feldgleichung, die auch die gravitationsbedingte Ablenkung von Licht beschreibt
			\item Beide Phänomene sind Manifestationen der lokalen Variation des Energiefelds, beschrieben durch den Parameter $\xi$
		\end{enumerate}
		
		Diese Vereinheitlichung ist ein zentraler konzeptioneller Vorteil des T0-Modells gegenüber dem Standardmodell.
	\end{wichtig}
	
	\section{Experimentelle Signaturen und Tests}
	
	Die mathematische Äquivalenz von Energieverlust, Rotverschiebung und Lichtablenkung führt zu spezifischen experimentellen Vorhersagen, die das T0-Modell von der Standardphysik unterscheiden können:
	
	\subsection{Wellenlängenabhängige Rotverschiebung}
	
	Für einen Quasar mit einer Rotverschiebung von $z_0 = 2$ und $\alpha = 0,1$:
	\begin{align}
		z(\text{blau}) &= 2,0 \times (1 - 0,1 \times \ln(0,5)) = 2,0 \times (1 + 0,069) = 2,14 \\
		z(\text{rot}) &= 2,0 \times (1 - 0,1 \times \ln(2,0)) = 2,0 \times (1 - 0,069) = 1,86
	\end{align}
	
	Dies sagt eine systematische Variation der Rotverschiebung mit der Wellenlänge voraus, die durch Messung der Rotverschiebung desselben astronomischen Objekts bei verschiedenen Wellenlängen getestet werden könnte.
	
	\subsection{Energieabhängige Lichtablenkung}
	
	Für Röntgen- (10 keV) und optische (2 eV) Photonen bei einer Ablenkung durch die Sonne:
	\begin{equation}
		\frac{\theta_{\text{Röntgen}}}{\theta_{\text{optisch}}} \approx 1 + \frac{4}{3} \times 10^{-4} \cdot \frac{10^4 \text{ eV} - 2 \text{ eV}}{511 \times 10^3 \text{ eV}} \approx 1 + 2,6 \times 10^{-6}
	\end{equation}
	
	Dieser kleine, aber potenziell messbare Unterschied im Ablenkungswinkel könnte mit zukünftigen hochpräzisen Beobachtungen nachgewiesen werden.
	
	\subsection{Korrelation zwischen Rotverschiebung und Lichtablenkung}
	
	Die Korrelation zwischen Rotverschiebung und gravitationsbedingter Ablenkung wird beschrieben durch:
	\begin{equation}
		\frac{\Delta z}{\Delta \theta} = \frac{\xi E_{\gamma,0}}{\efield} \cdot \frac{bc^2}{4GM} \cdot \frac{1}{\ln\left(\frac{r}{r_0}\right)} \cdot \frac{1}{\xi \frac{E_\gamma}{E_0}}
	\end{equation}
	
	Bei der Beobachtung der Gravitationslinsenbildung entfernter Objekte sollte eine spezifische Korrelation zwischen dem Grad der Lichtablenkung und der Rotverschiebung nachweisbar sein, die sich von der Vorhersage des Standardmodells unterscheidet.
	
	\section{Fazit}
	
	Die T0-Theorie vereinheitlicht die Phänomene des Energieverlusts, der Rotverschiebung und der Lichtablenkung durch eine einzige Geodätengleichung mit Zeitfeldkorrekturen. Diese Vereinheitlichung wird durch den universellen geometrischen Parameter $\xi = \frac{4}{3} \times 10^{-4}$ erreicht, der die Kopplung zwischen dem Energiefeld und der Raumzeitgeometrie bestimmt.
	
	Die mathematische Äquivalenz dieser Phänomene führt zu spezifischen experimentellen Vorhersagen, die potenziell mit hochpräzisen astronomischen Beobachtungen getestet werden könnten und eine Möglichkeit bieten, zwischen dem T0-Modell und der Standardphysik zu unterscheiden.
	
	Dieser vereinheitlichte Ansatz stellt einen konzeptionellen Fortschritt gegenüber dem Standardmodell dar, das diese Phänomene als unterschiedliche Effekte behandelt, die separate theoretische Rahmen erfordern.
	
\end{document}