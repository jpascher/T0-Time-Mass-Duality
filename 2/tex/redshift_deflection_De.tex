\documentclass[12pt,a4paper]{article}
\usepackage[utf8]{inputenc}
\usepackage[T1]{fontenc}
\usepackage[ngerman]{babel}
\usepackage{amsmath,amssymb,amsfonts,amsthm}
\usepackage{physics}
\usepackage{siunitx}
\usepackage{geometry}
\usepackage{fancyhdr}
\usepackage{enumitem}
\usepackage{booktabs}
\usepackage{longtable}
\usepackage{array}
\usepackage{xcolor}
\usepackage{tcolorbox}
\usepackage{mdframed}
\usepackage{graphicx}
\usepackage{hyperref}

\geometry{margin=2.5cm}
\pagestyle{fancy}
\fancyhf{}
\fancyhead[L]{T0-Theorie: Rotverschiebungs-Mechanismus}
\fancyhead[R]{\thepage}
\fancyfoot[C]{\textit{Wellenlaengen-abhaengige Rotverschiebung ohne Distanzannahmen}}

\hypersetup{
	colorlinks=true,
	linkcolor=blue,
	filecolor=magenta,
	urlcolor=cyan,
}

% Custom commands
\newcommand{\xiconst}{\xi = \frac{4}{3} \times 10^{-4}}
\newcommand{\Exi}{E_\xi}
\newcommand{\xicoupling}{f(E/\Exi)}
\newcommand{\lambdazero}{\lambda_0}
\newcommand{\nuzero}{\nu_0}

% Custom environments
\newtcolorbox{wichtig}[1][]{colback=yellow!10!white,colframe=yellow!50!black,fonttitle=\bfseries,title=Zentrale Erkenntnis,#1}
\newtcolorbox{formel}[1][]{colback=blue!5!white,colframe=blue!75!black,fonttitle=\bfseries,title=T0-Vorhersage,#1}
\newtcolorbox{experiment}[1][]{colback=green!5!white,colframe=green!75!black,fonttitle=\bfseries,title=Experimenteller Test,#1}

\theoremstyle{definition}
\newtheorem{prinzip}{Prinzip}

\title{\Huge\textbf{T0-Theorie: Rotverschiebungs-Mechanismus}\\
	\Large Wellenlaengen-abhaengige Rotverschiebung \\
	ohne Distanzannahmen}

\author{Basierend auf dem T0-Theorie-Framework\\
	Spektroskopische Tests mit kosmischen Objektmassen}

\date{\today}

\begin{document}
	
	\maketitle
	
	\begin{abstract}
		Das T0-Modell erklaert kosmische Rotverschiebung durch $\xi$-Feld-Energieverlust waehrend der Photonenpropagation, ohne Raumexpansion oder Distanzmessungen zu benoetigen. Dieser Mechanismus sagt wellenlaengen-abhaengige Rotverschiebung $z \propto \lambda$ voraus, die mit spektroskopischen Beobachtungen kosmischer Objekte getestet werden kann. Unter Verwendung der universellen Konstante $\xiconst$ und gemessener Massen astronomischer Objekte bietet die Theorie modell-unabhaengige Tests, die sich von der Standard-Kosmologie unterscheiden lassen.
	\end{abstract}
	
	\tableofcontents
	\newpage
	
	\section{Fundamentaler $\xi$-Feld-Energieverlust}
	
	\subsection{Grundlegender Mechanismus}
	
	\begin{prinzip}[$\xi$-Feld-Photon-Wechselwirkung]
		Photonen verlieren Energie durch Wechselwirkung mit dem universellen $\xi$-Feld waehrend der Propagation:
		\begin{equation}
			\frac{dE}{dx} = -\xi \cdot f\left(\frac{E}{\Exi}\right) \cdot E
		\end{equation}
		wobei $\xiconst$ die universelle geometrische Konstante ist und $\Exi = \frac{1}{\xi} = 7500$ (natuerliche Einheiten).
	\end{prinzip}
	
	Die Kopplungsfunktion $f(E/\Exi)$ ist dimensionslos und beschreibt die energie-abhaengige Wechselwirkungsstaerke. Fuer den linearen Kopplungsfall:
	\begin{equation}
		f\left(\frac{E}{\Exi}\right) = \frac{E}{\Exi}
	\end{equation}
	
	Dies ergibt die vereinfachte Energieverlust-Gleichung:
	\begin{equation}
		\frac{dE}{dx} = -\frac{\xi E^2}{\Exi}
	\end{equation}
	
	\subsection{Energie-zu-Wellenlaenge-Umwandlung}
	
	Da $E = \frac{hc}{\lambda}$ (oder $E = \frac{1}{\lambda}$ in natuerlichen Einheiten), koennen wir den Energieverlust in Wellenlaenge ausdruecken. Einsetzen von $E = \frac{1}{\lambda}$:
	
	\begin{equation}
		\frac{d(1/\lambda)}{dx} = -\frac{\xi}{\Exi} \cdot \frac{1}{\lambda^2}
	\end{equation}
	
	Umformen zur Wellenlaengen-Evolution:
	\begin{equation}
		\frac{d\lambda}{dx} = \frac{\xi \lambda^2}{\Exi}
	\end{equation}
	
	\section{Herleitung der Rotverschiebungsformel}
	
	\subsection{Integration fuer kleine $\xi$-Effekte}
	
	Fuer die Wellenlaengen-Evolutionsgleichung:
	\begin{equation}
		\frac{d\lambda}{dx} = \frac{\xi \lambda^2}{\Exi}
	\end{equation}
	
	Trennung der Variablen und Integration:
	\begin{equation}
		\int_{\lambdazero}^{\lambda} \frac{d\lambda'}{\lambda'^2} = \frac{\xi}{\Exi} \int_0^x dx'
	\end{equation}
	
	Dies ergibt:
	\begin{equation}
		\frac{1}{\lambdazero} - \frac{1}{\lambda} = \frac{\xi x}{\Exi}
	\end{equation}
	
	Aufloesen nach der beobachteten Wellenlaenge:
	\begin{equation}
		\lambda = \frac{\lambdazero}{1 - \frac{\xi x \lambdazero}{\Exi}}
	\end{equation}
	
	\subsection{Rotverschiebungs-Definition und Formel}
	
	\begin{formel}
		Rotverschiebungs-Definition:
		\begin{equation}
			z = \frac{\lambda_{\text{beobachtet}} - \lambda_{\text{emittiert}}}{\lambda_{\text{emittiert}}} = \frac{\lambda}{\lambdazero} - 1
		\end{equation}
	\end{formel}
	
	Fuer kleine $\xi$-Effekte mit $\frac{\xi x \lambdazero}{\Exi} \ll 1$ koennen wir entwickeln:
	\begin{equation}
		z \approx \frac{\xi x \lambdazero}{\Exi} = \frac{\xi x}{\Exi} \cdot \lambdazero
	\end{equation}
	
	\begin{wichtig}
		\textbf{Zentrale T0-Vorhersage: Wellenlaengen-abhaengige Rotverschiebung}
		\begin{equation}
			\boxed{z(\lambdazero) = \frac{\xi x}{\Exi} \cdot \lambdazero}
		\end{equation}
		
		Dies ist die fundamentale Vorhersage der T0-Theorie: \textbf{Rotverschiebung ist proportional zur emittierten Wellenlaenge!}
	\end{wichtig}
	
	\section{Frequenz-basierte Formulierung}
	
	\subsection{Frequenz-Energieverlust}
	
	Da $E = h\nu$, wird die Energieverlust-Gleichung zu:
	\begin{equation}
		\frac{d(h\nu)}{dx} = -\frac{\xi (h\nu)^2}{\Exi}
	\end{equation}
	
	Vereinfachung:
	\begin{equation}
		\frac{d\nu}{dx} = -\frac{\xi h \nu^2}{\Exi}
	\end{equation}
	
	\subsection{Frequenz-Rotverschiebungsformel}
	
	Integration der Frequenz-Evolution:
	\begin{equation}
		\int_{\nuzero}^{\nu} \frac{d\nu'}{\nu'^2} = -\frac{\xi h}{\Exi} \int_0^x dx'
	\end{equation}
	
	Dies ergibt:
	\begin{equation}
		\frac{1}{\nu} - \frac{1}{\nuzero} = \frac{\xi h x}{\Exi}
	\end{equation}
	
	Daher:
	\begin{equation}
		\nu = \frac{\nuzero}{1 + \frac{\xi h x \nuzero}{\Exi}}
	\end{equation}
	
	\begin{formel}
		Frequenz-Rotverschiebung:
		\begin{equation}
			z = \frac{\nuzero}{\nu} - 1 \approx \frac{\xi h x \nuzero}{\Exi}
		\end{equation}
	\end{formel}
	
	\begin{wichtig}
		Da $\nu = \frac{c}{\lambda}$, haben wir $h\nu = \frac{hc}{\lambda}$, was bestaetigt:
		\begin{equation}
			z \propto \nu \propto \frac{1}{\lambda}
		\end{equation}
		\textbf{Photonen hoeherer Frequenz zeigen groessere Rotverschiebung!}
	\end{wichtig}
	
	\section{Beobachtbare Vorhersagen ohne Distanzannahmen}
	
	\subsection{Spektrallinien-Verhaeltnisse}
	
	Verschiedene atomare Uebergaenge sollten unterschiedliche Rotverschiebungen entsprechend ihrer Wellenlaengen zeigen:
	
	\begin{equation}
		\frac{z(\lambda_1)}{z(\lambda_2)} = \frac{\lambda_1}{\lambda_2}
	\end{equation}
	
	\begin{experiment}
		\textbf{Wasserstofflinien-Test:}
		\begin{itemize}
			\item Lyman-$\alpha$ (121,6 nm) vs. H$\alpha$ (656,3 nm)
			\item Vorhergesagtes Verhaeltnis: $\frac{z_{\text{Ly}\alpha}}{z_{\text{H}\alpha}} = \frac{121,6}{656,3} = 0,185$
			\item \textbf{Standard-Kosmologie sagt vorher: 1,000}
		\end{itemize}
	\end{experiment}
	
	\subsection{Frequenz-abhaengige Effekte}
	
	Fuer Radio- vs. optische Beobachtungen desselben Objekts:
	\begin{equation}
		\frac{z_{\text{Radio}}}{z_{\text{optisch}}} = \frac{\nu_{\text{Radio}}}{\nu_{\text{optisch}}}
	\end{equation}
	
	\begin{experiment}
		\textbf{21cm vs. H$\alpha$-Test:}
		\begin{itemize}
			\item 21cm Wasserstofflinie: $\nu = 1420$ MHz
			\item Optische H$\alpha$-Linie: $\nu = 457$ THz
			\item Vorhergesagtes Verhaeltnis: $\frac{z_{21\text{cm}}}{z_{\text{H}\alpha}} = \frac{1,42 \times 10^9}{4,57 \times 10^{14}} = 3,1 \times 10^{-6}$
		\end{itemize}
	\end{experiment}
	
	\section{Massen-basierte Energieskalen-Kalibrierung}
	
	\subsection{Verwendung bekannter kosmischer Objektmassen}
	
	Anstatt Distanzen anzunehmen, verwenden wir gemessene Massen kosmischer Objekte zur Kalibrierung der Energieskala:
	
	\begin{longtable}{lll}
		\caption{Gut bestimmte kosmische Massen} \\
		\toprule
		\textbf{Objekttyp} & \textbf{Beispiel} & \textbf{Masse} \\
		\midrule
		\endfirsthead
		\multicolumn{3}{c}{\tablename\ \thetable{} -- Fortsetzung} \\
		\toprule
		\textbf{Objekttyp} & \textbf{Beispiel} & \textbf{Masse} \\
		\midrule
		\endhead
		\multicolumn{3}{l}{\emph{Stellare Massen (Praezise)}} \\
		Sonne & Sol & $1,989 \times 10^{30}$ kg \\
		Sirius A & $\alpha$ CMa A & $2,02\,M_\odot$ \\
		Alpha Centauri A & $\alpha$ Cen A & $1,1\,M_\odot$ \\
		\midrule
		\multicolumn{3}{l}{\emph{Galaxienmassen (Aus Dynamik)}} \\
		Milchstrasse & Unsere Galaxie & $10^{12}\,M_\odot$ \\
		Andromeda & M31 & $1,5 \times 10^{12}\,M_\odot$ \\
		Lokale Gruppe & Gesamt & $\approx 3 \times 10^{12}\,M_\odot$ \\
		\bottomrule
	\end{longtable}
	
	\subsection{Masse-Energie-Beziehung im $\xi$-Feld}
	
	Die charakteristische Energieskala ist:
	\begin{equation}
		\Exi = \xi^{-1} = \frac{3}{4 \times 10^{-4}} = 7500 \text{ (natuerliche Einheiten)}
	\end{equation}
	
	Umwandlung in konventionelle Einheiten:
	\begin{equation}
		\Exi = 7500 \times (\hbar c) \approx 1,5 \text{ GeV}
	\end{equation}
	
	Diese Energieskala ist vergleichbar mit Kernbindungsenergien und deutet darauf hin, dass das $\xi$-Feld an fundamentale Massenskalen in kosmischen Strukturen koppelt.
	
	\section{Experimentelle Tests mittels Spektroskopie}
	
	\subsection{Multi-Wellenlaengen-Beobachtungen}
	
	\begin{experiment}
		\textbf{Simultane Multi-Band-Spektroskopie:}
		\begin{enumerate}
			\item Beobachte Quasar/Galaxie simultan in UV, optisch, IR
			\item Messe Rotverschiebung von verschiedenen Spektrallinien
			\item Teste ob $z \propto \lambda$-Beziehung gilt
			\item Vergleiche mit Standard-Kosmologie-Vorhersage ($z = \text{konstant}$)
		\end{enumerate}
	\end{experiment}
	
	\subsection{Radio- vs. optische Rotverschiebung}
	
	\begin{experiment}
		\textbf{21cm vs. optische Linien-Vergleich:}
		\begin{itemize}
			\item \textbf{Radio-Durchmusterungen}: ALFALFA, HIPASS (21cm-Rotverschiebungen)
			\item \textbf{Optische Durchmusterungen}: SDSS, 2dF (H$\alpha$-, H$\beta$-Rotverschiebungen)
			\item \textbf{Methode}: Vergleiche Objekte, die in beiden Surveys beobachtet wurden
			\item \textbf{Vorhersage}: $z_{21\text{cm}} \neq z_{\text{optisch}}$ (T0) vs. $z_{21\text{cm}} = z_{\text{optisch}}$ (Standard)
		\end{itemize}
	\end{experiment}
	
	\subsection{Erwartete Signalstaerke}
	
	Fuer typische kosmische Objekte mit $\xiconst$:
	
	\begin{equation}
		\frac{\Delta z}{z} = \frac{\lambda_1 - \lambda_2}{\lambda_{\text{mittel}}} \times \xi \approx 10^{-4} \text{ bis } 10^{-5}
	\end{equation}
	
	\begin{wichtig}
		Dieser Wellenlaengen-Effekt liegt an der Grenze aktueller spektroskopischer Praezision, ist aber potenziell nachweisbar mit Instrumenten der naechsten Generation wie:
		\begin{itemize}
			\item Extremely Large Telescope (ELT)
			\item James Webb Space Telescope (JWST)
			\item Square Kilometre Array (SKA)
		\end{itemize}
	\end{wichtig}
	
	\section{Vorteile gegenueber Standard-Kosmologie}
	
	\subsection{Modell-unabhaengiger Ansatz}
	
	\begin{longtable}{lcc}
		\caption{T0-Theorie vs. Standard-Kosmologie} \\
		\toprule
		\textbf{Aspekt} & \textbf{Standard-Kosmologie} & \textbf{T0-Theorie} \\
		\midrule
		\endfirsthead
		\multicolumn{3}{c}{\tablename\ \thetable{} -- Fortsetzung} \\
		\toprule
		\textbf{Aspekt} & \textbf{Standard-Kosmologie} & \textbf{T0-Theorie} \\
		\midrule
		\endhead
		Distanz-Erfordernis & $z \rightarrow d$ (via Hubble) & Direkter spektroskopischer Test \\
		Wellenlaengen-Abhaengigkeit & $\frac{dz}{d\lambda} = 0$ & $\frac{dz}{d\lambda} \propto \xi$ \\
		Freie Parameter & $\Omega_m, \Omega_\Lambda, H_0, \ldots$ & Einzelner Parameter $\xi$ \\
		Exotische Komponenten & Dunkle Energie (69\%) & Nur $\xi$-Feld \\
		Testbarkeit & Indirekt (via Distanzleiter) & Direkt (Spektroskopie) \\
		\bottomrule
	\end{longtable}
	
	\subsection{Testbare Vorhersagen}
	
	\begin{formel}
		\textbf{Unterscheidender Test:}
		\begin{align}
			\text{Standard:} \quad &z_{\text{blau}} = z_{\text{rot}} \\
			\text{T0:} \quad &\frac{z_{\text{blau}}}{z_{\text{rot}}} = \frac{\lambda_{\text{blau}}}{\lambda_{\text{rot}}} < 1
		\end{align}
	\end{formel}
	
	\section{Beobachtungsstrategie}
	
	\subsection{Zielobjekt-Auswahl}
	
	Fokus auf Objekte mit:
	\begin{enumerate}
		\item \textbf{Starken Spektrallinien} ueber breiten Wellenlaengenbereich
		\item \textbf{Gut bestimmten Massen} aus stellarer/galaktischer Dynamik
		\item \textbf{Hohem Signal-zu-Rausch-Verhaeltnis} verfuegbare Spektren
	\end{enumerate}
	
	\textbf{Ideale Ziele:}
	\begin{itemize}
		\item Helle Quasare mit breiter spektraler Abdeckung
		\item Nahe Galaxien mit mehreren Emissionslinien
		\item Doppelsternsysteme mit praezisen Massenbestimmungen
	\end{itemize}
	
	\subsection{Datenanalyse-Protokoll}
	
	\begin{experiment}
		\textbf{Analyse-Schritte:}
		\begin{enumerate}
			\item Messe Rotverschiebungen von mehreren Spektrallinien
			\item Trage $z$ vs. $\lambda$ fuer jedes Objekt auf
			\item Fitte lineare Beziehung: $z = \alpha \cdot \lambda + \beta$
			\item Vergleiche Steigung $\alpha$ mit T0-Vorhersage: $\alpha = \frac{\xi x}{\Exi}$
			\item Teste gegen Standard-Kosmologie: $\alpha = 0$
		\end{enumerate}
	\end{experiment}
	
	\subsection{Erforderliche Praezision}
	
	Zum Nachweis von T0-Effekten mit $\xiconst$:
	
	\begin{itemize}
		\item \textbf{Mindest-Praezision erforderlich}: $\frac{\Delta z}{z} \approx 10^{-5}$
		\item \textbf{Aktuell beste Praezision}: $\frac{\Delta z}{z} \approx 10^{-4}$ (knapp ausreichend)
		\item \textbf{Naechste Generation Instrumente}: $\frac{\Delta z}{z} \approx 10^{-6}$ (klar nachweisbar)
	\end{itemize}
	
	\section{Schlussfolgerung}
	
	\subsection{Zusammenfassung des T0-Rotverschiebungs-Mechanismus}
	
	Die T0-Theorie bietet einen \textbf{distanz-unabhaengigen} Mechanismus fuer kosmische Rotverschiebung durch $\xi$-Feld-Energieverlust. Die Schluesseleigenschaften sind:
	
	\begin{enumerate}
		\item \textbf{Universelle Konstante}: $\xiconst$ bestimmt alle Rotverschiebungseffekte
		\item \textbf{Wellenlaengen-Abhaengigkeit}: $z \propto \lambda$ (fundamentale Vorhersage)
		\item \textbf{Massen-basierte Kalibrierung}: Verwendet gemessene kosmische Objektmassen
		\item \textbf{Modell-unabhaengige Tests}: Direkte spektroskopische Verifikation
	\end{enumerate}
	
	\subsection{Experimentelle Zugaenglichkeit}
	
	Die Theorie liefert konkrete, testbare Vorhersagen:
	
	\begin{formel}
		\textbf{Zentrale experimentelle Signatur:}
		\begin{equation}
			\boxed{\frac{z_{\text{blau}}}{z_{\text{rot}}} = \frac{\lambda_{\text{blau}}}{\lambda_{\text{rot}}} \neq 1}
		\end{equation}
	\end{formel}
	
	Diese Vorhersage kann getestet werden mit:
	\begin{itemize}
		\item Multi-Wellenlaengen-Spektroskopie derselben Objekte
		\item Radio- vs. optische Rotverschiebungs-Vergleiche
		\item Hochpraezisions-Messungen von Spektrallinien-Verhaeltnissen
	\end{itemize}
	
	\subsection{Revolutionaere Implikationen}
	
	\begin{wichtig}
		Falls bestaetigt, wuerde wellenlaengen-abhaengige Rotverschiebung unser Verstaendnis revolutionieren von:
		\begin{itemize}
			\item \textbf{Kosmischer Rotverschiebungs-Ursprung}: Energieverlust vs. Raumexpansion
			\item \textbf{Distanzmessungen}: Modell-unabhaengige spektroskopische Distanzen
			\item \textbf{Dunkler Energie}: Nicht mehr erforderlich zur Erklaerung kosmischer Beschleunigung
			\item \textbf{Fundamentaler Physik}: Neue Feldwechselwirkungen auf kosmischen Skalen
		\end{itemize}
	\end{wichtig}
	
	Der T0-Rotverschiebungs-Mechanismus bietet eine \textbf{testbare Alternative} zur Standard-Kosmologie, die durch spektroskopische Beobachtungen verifiziert werden kann, wodurch sie experimentell zugaenglich wird mit aktuellen oder zukuenftigen astronomischen Instrumenten.
	
	\bibliographystyle{plain}
	\begin{thebibliography}{9}
		
		\bibitem{pascher2024}
		Pascher, J. (2024). \textit{T0-Theorie: Mathematische Aequivalenz-Formulierung}. HTL Leonding, Abteilung Nachrichtentechnik.
		
		\bibitem{planck2020}
		Planck Collaboration (2020). \textit{Planck 2018 results. VI. Cosmological parameters}. Astron. Astrophys. 641, A6.
		
		\bibitem{sdss2020}
		SDSS Collaboration (2020). \textit{The Sloan Digital Sky Survey: Technical Summary}. Astron. J. 120, 1579.
		
		\bibitem{alfalfa2018}
		ALFALFA Team (2018). \textit{The Arecibo Legacy Fast ALFA Survey}. Astrophys. J. Suppl. 232, 21.
		
	\end{thebibliography}
	
\end{document}