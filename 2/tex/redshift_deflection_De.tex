\documentclass[12pt,a4paper]{article}
\usepackage[utf8]{inputenc}
\usepackage[T1]{fontenc}
\usepackage[ngerman]{babel}
\usepackage{amsmath,amssymb,amsfonts,amsthm}
\usepackage{physics}
\usepackage{siunitx}
\usepackage{geometry}
\usepackage{fancyhdr}
\usepackage{enumitem}
\usepackage{booktabs}
\usepackage{longtable}
\usepackage{array}
\usepackage{xcolor}
\usepackage{tcolorbox}
\usepackage{mdframed}
\usepackage{graphicx}
\usepackage{hyperref}

\geometry{margin=2.5cm}
\pagestyle{fancy}
\fancyhf{}
\fancyhead[L]{T0-Theorie: Rotverschiebungsmechanismus}
\fancyhead[R]{\thepage}
\fancyfoot[C]{\textit{Wellenl\"angenabh\"angige Rotverschiebung ohne Entfernungsannahmen}}

\hypersetup{
	colorlinks=true,
	linkcolor=blue,
	filecolor=magenta,
	urlcolor=cyan,
}

% Benutzerdefinierte Befehle
\newcommand{\xiconst}{\xi = \frac{4}{3} \times 10^{-4}}
\newcommand{\Exi}{E_\xi}
\newcommand{\xicoupling}{f(E/\Exi)}
\newcommand{\lambdazero}{\lambda_0}
\newcommand{\nuzero}{\nu_0}

% Benutzerdefinierte Umgebungen
\newtcolorbox{important}[1][]{colback=yellow!10!white,colframe=yellow!50!black,fonttitle=\bfseries,title=Schl\"usseleinsicht,#1}
\newtcolorbox{formula}[1][]{colback=blue!5!white,colframe=blue!75!black,fonttitle=\bfseries,title=T0-Vorhersage,#1}
\newtcolorbox{experiment}[1][]{colback=green!5!white,colframe=green!75!black,fonttitle=\bfseries,title=Experimenteller Test,#1}

\theoremstyle{definition}
\newtheorem{principle}{Prinzip}

\title{\Huge\textbf{T0-Theorie: Rotverschiebungsmechanismus}\\
	\Large Wellenl\"angenabh\"angige Rotverschiebung \\
	ohne Entfernungsannahmen oder r\"aumliche Expansion}

\author{Basierend auf dem T0-Theorie-Rahmenwerk\\
	Spektroskopische Tests unter Verwendung kosmischer Objektmassen}

\date{\today}

\begin{document}
	
	\maketitle
	
	\begin{abstract}
		Das T0-Modell erkl\"art die kosmologische Rotverschiebung durch $\xi$-Feld-Energieverlust w\"ahrend der Photonenausbreitung, ohne r\"aumliche Expansion oder Entfernungsmessungen zu ben\"otigen. Dieser Mechanismus sagt eine wellenl\"angenabh\"angige Rotverschiebung $z \propto \lambda$ vorher, die mit spektroskopischen Beobachtungen kosmischer Objekte getestet werden kann. Unter Verwendung der universellen Konstante $\xiconst$ und gemessener Massen astronomischer Objekte liefert die Theorie modellunabh\"angige Tests, die von der Standardkosmologie unterscheidbar sind. Das $\xi$-Feld erkl\"art auch die kosmische Mikrowellen-Hintergrundtemperatur ($T_{\text{CMB}} = 2,7255$ K) in einem statischen, ewig existierenden Universum, wie in \cite{pascher2025} detailliert beschrieben.
	\end{abstract}
	
	\tableofcontents
	\newpage
	
	\section{Fundamentaler $\xi$-Feld-Energieverlust}
	\label{sec:xi_field}
	
	\subsection{Grundmechanismus}
	
	\begin{principle}[$\xi$-Feld-Photonen-Wechselwirkung]
		Photonen verlieren Energie durch Wechselwirkung mit dem universellen $\xi$-Feld w\"ahrend der Ausbreitung:
		\begin{equation}
			\frac{dE}{dx} = -\xi \cdot f\left(\frac{E}{\Exi}\right) \cdot E
		\end{equation}
		wobei $\xiconst$ die universelle geometrische Konstante ist und $\Exi = \frac{1}{\xi} = 7500$ (nat\"urliche Einheiten).
	\end{principle}
	
	Die Kopplungsfunktion $f(E/\Exi)$ ist dimensionslos und beschreibt die energieabh\"angige Wechselwirkungsst\"arke. F\"ur den linearen Kopplungsfall:
	\begin{equation}
		f\left(\frac{E}{\Exi}\right) = \frac{E}{\Exi}
	\end{equation}
	
	Dies ergibt die vereinfachte Energieverlustgleichung:
	\begin{equation}
		\frac{dE}{dx} = -\frac{\xi E^2}{\Exi}
	\end{equation}
	
	\subsection{Energie-zu-Wellenl\"ange-Umwandlung}
	
	Da $E = \frac{hc}{\lambda}$ (oder $E = \frac{1}{\lambda}$ in nat\"urlichen Einheiten, $\hbar = c = 1$), k\"onnen wir den Energieverlust in Bezug auf die Wellenl\"ange ausdr\"ucken. Einsetzen von $E = \frac{1}{\lambda}$:
	\begin{equation}
		\frac{d(1/\lambda)}{dx} = -\frac{\xi}{\Exi} \cdot \frac{1}{\lambda^2}
	\end{equation}
	
	Umstellung zur Wellenl\"angenentwicklung:
	\begin{equation}
		\frac{d\lambda}{dx} = \frac{\xi \lambda^2}{\Exi}
	\end{equation}
	
	\section{Rotverschiebungsformel-Ableitung}
	
	\subsection{Integration f\"ur kleine $\xi$-Effekte}
	
	F\"ur die Wellenl\"angenentwicklungsgleichung:
	\begin{equation}
		\frac{d\lambda}{dx} = \frac{\xi \lambda^2}{\Exi}
	\end{equation}
	
	Trennung der Variablen und Integration:
	\begin{equation}
		\int_{\lambdazero}^{\lambda} \frac{d\lambda'}{\lambda'^2} = \frac{\xi}{\Exi} \int_0^x dx'
	\end{equation}
	
	Dies ergibt:
	\begin{equation}
		\frac{1}{\lambdazero} - \frac{1}{\lambda} = \frac{\xi x}{\Exi}
	\end{equation}
	
	L\"osung f\"ur die beobachtete Wellenl\"ange:
	\begin{equation}
		\lambda = \frac{\lambdazero}{1 - \frac{\xi x \lambdazero}{\Exi}}
	\end{equation}
	
	\subsection{Rotverschiebungsdefinition und Formel}
	
	\begin{formula}
		Rotverschiebungsdefinition:
		\begin{equation}
			z = \frac{\lambda_{\text{beobachtet}} - \lambda_{\text{emittiert}}}{\lambda_{\text{emittiert}}} = \frac{\lambda}{\lambdazero} - 1
		\end{equation}
	\end{formula}
	
	F\"ur kleine $\xi$-Effekte, wo $\frac{\xi x \lambdazero}{\Exi} \ll 1$, k\"onnen wir entwickeln:
	\begin{equation}
		z \approx \frac{\xi x \lambdazero}{\Exi} = \frac{\xi x}{\Exi / (\hbar c)} \cdot \lambdazero \quad (\text{in konventionellen Einheiten})
	\end{equation}
	
	\begin{important}
		\textbf{Schl\"ussel-T0-Vorhersage: Wellenl\"angenabh\"angige Rotverschiebung}
		\begin{equation}
			\boxed{z(\lambdazero) = \frac{\xi x}{\Exi} \cdot \lambdazero \quad (\text{nat\"urliche Einheiten, } \hbar = c = 1)}
		\end{equation}
		Dies funktioniert OHNE r\"aumliche Expansion! In konventionellen Einheiten wird $\Exi$ mit $\hbar c \approx 197,3$ MeV$\cdot$fm skaliert, sodass $\Exi \approx 1,5$ GeV $\Exi / (\hbar c) \approx 7500$ m$^{-1}$ entspricht, was dimensionale Konsistenz gew\"ahrleistet.
	\end{important}
	
	\subsection{Konsistenz mit beobachteten Rotverschiebungen}
	Die wellenl\"angenabh\"angige Rotverschiebung, gegeben durch $z \propto \frac{\xi x}{\Exi} \cdot \lambdazero$, erkl\"art beobachtete kosmologische Rotverschiebungen in Kombination mit erg\"anzenden Effekten wie Doppler-Verschiebungen, Gravitationsrotverschiebung und nichtlinearen $\xi$-Feld-Wechselwirkungen. F\"ur Objekte mit hoher Rotverschiebung ($z > 10$, z.B. \cite{jwst_early}) kann die Kopplungsfunktion $f\left(\frac{E}{\Exi}\right)$ h\"ohere Ordnungsterme enthalten, die Konsistenz mit Beobachtungen ohne kosmische Expansion gew\"ahrleisten. Laufende spektroskopische Tests, wie in Abschnitt \ref{sec:experimental_tests} beschrieben, zielen darauf ab, diesen Mechanismus zu validieren.
	
	\section{Frequenzbasierte Formulierung}
	
	\subsection{Frequenz-Energieverlust}
	
	Da $E = h\nu$, wird die Energieverlustgleichung zu:
	\begin{equation}
		\frac{d(h\nu)}{dx} = -\frac{\xi (h\nu)^2}{\Exi}
	\end{equation}
	
	Vereinfachung:
	\begin{equation}
		\frac{d\nu}{dx} = -\frac{\xi h \nu^2}{\Exi}
	\end{equation}
	
	\subsection{Frequenz-Rotverschiebungsformel}
	
	Integration der Frequenzentwicklung:
	\begin{equation}
		\int_{\nuzero}^{\nu} \frac{d\nu'}{\nu'^2} = -\frac{\xi h}{\Exi} \int_0^x dx'
	\end{equation}
	
	Dies ergibt:
	\begin{equation}
		\frac{1}{\nu} - \frac{1}{\nuzero} = \frac{\xi h x}{\Exi}
	\end{equation}
	
	Daher:
	\begin{equation}
		\nu = \frac{\nuzero}{1 + \frac{\xi h x \nuzero}{\Exi}}
	\end{equation}
	
	\begin{formula}
		Frequenz-Rotverschiebung:
		\begin{equation}
			z = \frac{\nuzero}{\nu} - 1 \approx \frac{\xi h x \nuzero}{\Exi} \quad (\text{nat\"urliche Einheiten, } h = 1; \text{konventionelle Einheiten, } h = \hbar)
		\end{equation}
	\end{formula}
	
	\begin{important}
		Da $\nu = \frac{c}{\lambda}$, haben wir $h\nu = \frac{hc}{\lambda}$, was best\"atigt:
		\begin{equation}
			z \propto \nu \propto \frac{1}{\lambda}
		\end{equation}
		\textbf{H\"oherfrequente Photonen zeigen gr\"o\ss{}ere Rotverschiebung!} In konventionellen Einheiten wird $\Exi$ mit $\hbar c$ skaliert, um dimensionale Konsistenz zu erhalten.
	\end{important}
	
	\section{Beobachtbare Vorhersagen ohne Entfernungsannahmen}
	
	\subsection{Spektrallinienverh\"altnisse}
	
	Verschiedene atomare \"Uberg\"ange sollten unterschiedliche Rotverschiebungen gem\"a\ss{} ihrer Wellenl\"angen zeigen:
	\begin{equation}
		\frac{z(\lambda_1)}{z(\lambda_2)} = \frac{\lambda_1}{\lambda_2}
	\end{equation}
	
	\begin{experiment}
		\textbf{Wasserstofflinien-Test:}
		\begin{itemize}
			\item Lyman-$\alpha$ (121,6 nm) vs. H$\alpha$ (656,3 nm)
			\item Vorhergesagtes Verh\"altnis: $\frac{z_{\text{Ly}\alpha}}{z_{\text{H}\alpha}} = \frac{121,6}{656,3} = 0,185$
			\item \textbf{Standardkosmologie sagt vorher: 1,000}
		\end{itemize}
	\end{experiment}
	
	\subsection{Frequenzabh\"angige Effekte}
	
	F\"ur Radio- vs. optische Beobachtungen desselben Objekts:
	\begin{equation}
		\frac{z_{\text{Radio}}}{z_{\text{optisch}}} = \frac{\nu_{\text{Radio}}}{\nu_{\text{optisch}}}
	\end{equation}
	
	\begin{experiment}
		\textbf{21cm vs. H$\alpha$ Test:}
		\begin{itemize}
			\item 21cm Wasserstofflinie: $\nu = 1420$ MHz
			\item Optische H$\alpha$ Linie: $\nu = 457$ THz
			\item Vorhergesagtes Verh\"altnis: $\frac{z_{21\text{cm}}}{z_{\text{H}\alpha}} = \frac{1,42 \times 10^9}{4,57 \times 10^{14}} = 3,1 \times 10^{-6}$
		\end{itemize}
	\end{experiment}
	
	\section{Massenbasierte Energieskalen-Kalibrierung}
	
	\subsection{Verwendung bekannter kosmischer Objektmassen}
	
	Anstatt Entfernungen anzunehmen, verwenden wir gemessene Massen kosmischer Objekte zur Kalibrierung der Energieskala:
	
	\begin{longtable}{l l r}
		\caption{Gut bestimmte kosmische Massen} \\
		\toprule
		\textbf{Objekttyp} & \textbf{Beispiel} & \textbf{Masse} \\
		\midrule
		\endfirsthead
		\multicolumn{3}{c}{\tablename\ \thetable{} -- Fortsetzung} \\
		\toprule
		\textbf{Objekttyp} & \textbf{Beispiel} & \textbf{Masse} \\
		\midrule
		\endhead
		\multicolumn{3}{l}{\emph{Sternmassen (pr\"azise)}} \\
		Sonne & Sol & $1,989 \times 10^{30}$ kg \\
		Sirius A & Alpha CMa A & $2,02\,M_\odot$ \\
		Alpha Centauri A & $\alpha$ Cen A & $1,1\,M_\odot$ \\
		\midrule
		\multicolumn{3}{l}{\emph{Galaxienmassen (aus Dynamik)}} \\
		Milchstra\ss{}e & Unsere Galaxie & $10^{12}\,M_\odot$ \\
		Andromeda & M31 & $1,5 \times 10^{12}\,M_\odot$ \\
		Lokale Gruppe & Gesamt & $\approx 3 \times 10^{12}\,M_\odot$ \\
		\bottomrule
	\end{longtable}
	
	\subsection{Masse-Energie-Beziehung im $\xi$-Feld}
	
	Die charakteristische Energieskala ist:
	\begin{equation}
		\Exi = \xi^{-1} = \frac{3}{4 \times 10^{-4}} = 7500 \text{ (nat\"urliche Einheiten)}
	\end{equation}
	
	Umrechnung in konventionelle Einheiten:
	\begin{equation}
		\Exi = 7500 \times (\hbar c) \approx 7500 \times 197,3 \text{ MeV} \cdot \text{fm} \approx 1,5 \text{ GeV}
	\end{equation}
	
	Diese Energieskala ist vergleichbar mit nuklearen Bindungsenergien, was darauf hindeutet, dass das $\xi$-Feld an fundamentale Massenskalen in kosmischen Strukturen koppelt.
	
	\section{Experimentelle Tests mittels Spektroskopie}
	\label{sec:experimental_tests}
	
	\subsection{Multiwellenl\"angen-Beobachtungen}
	
	\begin{experiment}
		\textbf{Simultane Multiband-Spektroskopie:}
		\begin{enumerate}
			\item Beobachtung von Quasar/Galaxie simultan in UV, optisch, IR
			\item Messung der Rotverschiebung aus verschiedenen Spektrallinien
			\item Test ob $z \propto \lambda$ Beziehung gilt
			\item Vergleich mit Standardkosmologie-Vorhersage ($z = \text{konstant}$)
		\end{enumerate}
	\end{experiment}
	
	\subsection{Radio vs. optische Rotverschiebung}
	
	\begin{experiment}
		\textbf{21cm vs. optische Linien-Vergleich:}
		\begin{itemize}
			\item \textbf{Radio-Durchmusterungen}: ALFALFA, HIPASS (21cm Rotverschiebungen)
			\item \textbf{Optische Durchmusterungen}: SDSS, 2dF (H$\alpha$, H$\beta$ Rotverschiebungen)
			\item \textbf{Methode}: Vergleich von Objekten in beiden Durchmusterungen beobachtet
			\item \textbf{Vorhersage}: $z_{21\text{cm}} \neq z_{\text{optisch}}$ (T0) vs. $z_{21\text{cm}} = z_{\text{optisch}}$ (Standard)
		\end{itemize}
	\end{experiment}
	
	\subsection{Erwartete Signalst\"arke}
	
	F\"ur typische kosmische Objekte mit $\xiconst$ ist der relative Unterschied in der Rotverschiebung zwischen zwei Spektrallinien:
	\begin{equation}
		\frac{\Delta z}{z} = \left| \frac{z(\lambda_1) - z(\lambda_2)}{z(\lambda_{\text{mittel}})} \right| = \left| \frac{\lambda_1 - \lambda_2}{\lambda_{\text{mittel}}} \right| \times \xi \approx 10^{-4} \text{ bis } 10^{-5}
	\end{equation}
	
	\begin{important}
		Dieser Wellenl\"angeneffekt liegt an der Grenze der aktuellen spektroskopischen Pr\"azision, ist aber potenziell nachweisbar mit Instrumenten der n\"achsten Generation wie:
		\begin{itemize}
			\item Extremely Large Telescope (ELT)
			\item James Webb Space Telescope (JWST)
			\item Square Kilometre Array (SKA)
		\end{itemize}
	\end{important}
	
	\section{Vorteile gegen\"uber der Standardkosmologie}
	
	\subsection{Modellunabh\"angiger Ansatz}
	
	\begin{longtable}{lcc}
		\caption{T0-Theorie vs. Standardkosmologie} \\
		\toprule
		\textbf{Aspekt} & \textbf{Standardkosmologie} & \textbf{T0-Theorie} \\
		\midrule
		\endfirsthead
		\multicolumn{3}{c}{\tablename\ \thetable{} -- Fortsetzung} \\
		\toprule
		\textbf{Aspekt} & \textbf{Standardkosmologie} & \textbf{T0-Theorie} \\
		\midrule
		\endhead
		Entfernungsanforderung & $z \rightarrow d$ (\"uber Hubble) & Direkter spektroskopischer Test \\
		Wellenl\"angenabh\"angigkeit & $\frac{dz}{d\lambda} = 0$ & $\frac{dz}{d\lambda} \propto \xi$ \\
		Freie Parameter & $\Omega_m, \Omega_\Lambda, H_0, \ldots$ & Einzelner Parameter $\xi$ \\
		Exotische Komponenten & Dunkle Energie (69\%) & Nur $\xi$-Feld \\
		Testbarkeit & Indirekt (\"uber Entfernungsleiter) & Direkt (Spektroskopie) \\
		Universum & Expandierend & Statisch, ewig \\
		\bottomrule
	\end{longtable}
	
	\subsection{Testbare Vorhersagen}
	
	\begin{formula}
		\textbf{Unterscheidungstest:}
		\begin{align}
			\text{Standard:} \quad &z_{\text{blau}} = z_{\text{rot}} \\
			\text{T0:} \quad &\frac{z_{\text{blau}}}{z_{\text{rot}}} = \frac{\lambda_{\text{blau}}}{\lambda_{\text{rot}}} < 1
		\end{align}
	\end{formula}
	
	\section{Beobachtungsstrategie}
	
	\subsection{Zielauswahl}
	
	Fokus auf Objekte mit:
	\begin{enumerate}
		\item \textbf{Starken Spektrallinien} \"uber einen weiten Wellenl\"angenbereich
		\item \textbf{Gut bestimmten Massen} aus stellarer/galaktischer Dynamik
		\item \textbf{Hohem Signal-zu-Rausch} verf\"ugbaren Spektren
	\end{enumerate}
	
	\textbf{Ideale Ziele:}
	\begin{itemize}
		\item Helle Quasare mit breiter spektraler Abdeckung
		\item Nahe Galaxien mit mehreren Emissionslinien
		\item Doppelsternsysteme mit pr\"azisen Massenbestimmungen
	\end{itemize}
	
	\subsection{Datenanalyse-Protokoll}
	
	\begin{experiment}
		\textbf{Analyseschritte:}
		\begin{enumerate}
			\item Messung der Rotverschiebungen aus mehreren Spektrallinien
			\item Auftragung $z$ vs. $\lambda$ f\"ur jedes Objekt
			\item Anpassung linearer Beziehung: $z = \alpha \cdot \lambda + \beta$
			\item Vergleich der Steigung $\alpha$ mit T0-Vorhersage: $\alpha = \frac{\xi x}{\Exi}$
			\item Test gegen Standardkosmologie: $\alpha = 0$
		\end{enumerate}
	\end{experiment}
	
	\subsection{Erforderliche Pr\"azision}
	
	Um T0-Effekte mit $\xiconst$ zu detektieren:
	\begin{itemize}
		\item \textbf{Minimal ben\"otigte Pr\"azision}: $\frac{\Delta z}{z} \approx 10^{-5}$
		\item \textbf{Aktuelle beste Pr\"azision}: $\frac{\Delta z}{z} \approx 10^{-4}$ (kaum ausreichend)
		\item \textbf{N\"achste Generation Instrumente}: $\frac{\Delta z}{z} \approx 10^{-6}$ (klar nachweisbar)
	\end{itemize}
\section{Mathematische Äquivalenz von Raumdehnung, Energieverlust und Beugung}
\label{sec:equivalence}

\subsection{Formale Äquivalenzbeweise}
\label{subsec:equivalence_proofs}

Die drei fundamentalen Mechanismen zur Erklärung der kosmologischen Rotverschiebung lassen sich durch unterschiedliche physikalische Prozesse beschreiben, führen aber unter bestimmten Bedingungen zu mathematisch äquivalenten Ergebnissen.

\begin{table}[h]
	\centering
	\caption{Vergleich der Rotverschiebungsmechanismen mit erweiterten Entwicklungen}
	\scalebox{0.75}{
		\begin{tabular}{lllc}
			\toprule
			\textbf{Mechanismus} & \textbf{Physikalischer Prozess} & \textbf{Rotverschiebungsformel} & \textbf{Taylor-Entwicklung} \\
			\midrule
			Raumdehnung (\(\Lambda\)CDM) & Metrische Expansion & \(1+z = \frac{a(t_0)}{a(t_e)}\) & \(z \approx H_0 D + \frac{1}{2}q_0(H_0 D)^2\) \\
			Energieverlust (T0-E) & Photonenermüdung & \(1+z = \exp\left(\int_0^D \xi \frac{H}{T} dl\right)\) & \(z \approx \xi \frac{H_0 D}{T_0} + \frac{1}{2}\xi^2\left(\frac{H_0 D}{T_0}\right)^2\) \\
			Vakuumbeugung (T0-B) & Brechungsindexänderung & \(1+z = \frac{n(t_e)}{n(t_0)}\) & \(z \approx \xi \ln\left(1+\frac{H_0 D}{c}\right)\left(1+\frac{\xi\lambda_0}{2\lambda_{crit}}\right)\) \\
			\bottomrule
		\end{tabular}
	}
\end{table}

\subsubsection{Mathematische Äquivalenzbedingungen}

Für die Äquivalenz der drei Mechanismen müssen folgende Bedingungen erfüllt sein:

\begin{equation}
	\boxed{\frac{1}{a}\frac{da}{dt} = -\frac{1}{n}\frac{dn}{dt} = \xi \frac{H}{T_0}}
\end{equation}

Dies führt zu den Beziehungen:
\begin{itemize}
	\item \textbf{\(\Lambda\)CDM \(\leftrightarrow\) T0-B}: \(n(t) = a^{-1}(t)\)
	\item \textbf{\(\Lambda\)CDM \(\leftrightarrow\) T0-E}: \(\dot{E}/E = -H(t)\)
	\item \textbf{T0-B \(\leftrightarrow\) T0-E}: \(n(t) \propto E^{-1}(t)\)
\end{itemize}

\subsubsection{Störungstheoretische Entwicklung}

Die Äquivalenz gilt exakt nur in erster Ordnung. In höheren Ordnungen ergeben sich charakteristische Unterschiede:

\begin{equation}
	z_{total} = z^{(1)} + z^{(2)} + z^{(3)} + \mathcal{O}(\xi^4)
\end{equation}

\textbf{Erste Ordnung (identisch für alle Mechanismen):}
\begin{equation}
	z^{(1)} = \xi \int_0^D H(l) \, dl \approx \xi H_0 D
\end{equation}

\textbf{Zweite Ordnung (mechanismusspezifisch):}
\begin{align}
	z^{(2)}_{\Lambda CDM} &= \frac{1}{2}(1-q_0)(H_0 D)^2 \\
	z^{(2)}_{T0-B} &= \frac{\xi^2}{2}\left(\frac{\lambda_0}{\lambda_{crit}}\right)(H_0 D)^2 \\
	z^{(2)}_{T0-E} &= \frac{\xi^2}{2}\left(\frac{H_0 D}{T_0}\right)^2
\end{align}

\subsection{Geometrische Interpretation}
\label{subsec:geometric}

\subsubsection{Konforme Transformation der Metrik}

Die geometrische Äquivalenz wird durch eine konforme Transformation vermittelt:

\begin{equation}
	ds^2_{T0} = \Omega^2(t,\vec{x}) \, ds^2_{\Lambda CDM}
\end{equation}

mit dem konformen Faktor:
\begin{equation}
	\Omega^2(t,\vec{x}) = n^{-2}(t) \times \left[1 + \delta\Omega(\vec{x})\right]
\end{equation}

Für homogene und isotrope Fälle (\(\delta\Omega = 0\)) ergibt sich:
\begin{equation}
	ds^2_{T0} = n^{-2}(t)\left[-dt^2 + a^2(t)d\vec{x}^2\right]
\end{equation}

\subsubsection{Geodätengleichung und Lichtausbreitung}

Die Christoffel-Symbole transformieren sich gemäß:
\begin{equation}
	\Gamma^\mu_{\nu\rho}\bigg|_{T0} = \Gamma^\mu_{\nu\rho}\bigg|_{\Lambda\text{CDM}} + \delta^\mu_\nu \partial_\rho \ln n + \delta^\mu_\rho \partial_\nu \ln n - g_{\nu\rho}g^{\mu\sigma}\partial_\sigma \ln n
\end{equation}

\subsubsection{Effektiver Brechungsindex}

Der wellenlängenabhängige Brechungsindex in der T0-Theorie:

\begin{equation}
	n(t,\lambda) = 1 + \frac{\xi}{2}\left(\frac{T(t)}{T_0}\right)^2 \times \left[1 + \beta\left(\frac{\lambda}{\lambda_0}\right)^\alpha\right]
\end{equation}

mit:
\begin{itemize}
	\item \(\alpha = 1\) für lineare Dispersion
	\item \(\beta = \xi/2\) als Kopplungsstärke
	\item \(T(t)/T_0\) als zeitliche Modulation
\end{itemize}

\subsection{Energieerhaltung und Thermodynamik}
\label{subsec:energy_conservation}

\subsubsection{Energiebilanz in verschiedenen Formalismen}

\textbf{\(\Lambda\)CDM (scheinbarer Energieverlust):}
\begin{equation}
	E_{photon} = \frac{h\nu_0}{1+z} = \frac{h\nu_0 a(t_e)}{a(t_0)}
\end{equation}

\textbf{T0-Beugung (Energieerhaltung):}
\begin{equation}
	E_{photon} = \frac{h\nu}{n(t)} = \frac{h\nu_0}{(1+z)n(t)} = \text{const}
\end{equation}

\textbf{T0-Energieverlust (realer Verlust):}
\begin{equation}
	\frac{dE}{dt} = -\xi H E \quad \Rightarrow \quad E(t) = E_0 \exp\left(-\int_0^t \xi H(t') dt'\right)
\end{equation}

\subsubsection{Thermodynamische Konsistenz}

Die Entropieänderung für die verschiedenen Mechanismen:

\begin{equation}
	\Delta S = \begin{cases}
		0 & \text{(\(\Lambda\)CDM: adiabatisch)} \\
		k_B \xi N_{photon} \ln(1+z) & \text{(T0-Energieverlust)} \\
		0 & \text{(T0-Beugung: reversibel)}
	\end{cases}
\end{equation}

\subsection{Beobachtbare Konsequenzen}
\label{subsec:observables}

\subsubsection{Unterscheidbare Signaturen}

\begin{table}[h]
	\centering
	\caption{Experimentell unterscheidbare Effekte zweiter Ordnung}
	\scalebox{0.75}{
		\begin{tabular}{lcccc}
			\toprule
			\textbf{Observable} & \textbf{\(\Lambda\)CDM} & \textbf{T0-Beugung} & \textbf{T0-Energieverlust} & \textbf{Nachweisbarkeit} \\
			\midrule
			FRB-Dispersion & \(\Delta t \propto \nu^{-2}\) & \(\Delta t \propto \nu^{-2}(1+\xi\nu)\) & \(\Delta t \propto \nu^{-2}(1+\xi^2\ln\nu)\) & CHIME: \(5\sigma\) bei \(\xi>10^{-5}\) \\
			Spektrallinien & \(z\) unabhängig von \(\lambda\) & \(\Delta z/z \approx \xi\lambda/\lambda_{crit}\) & \(z\) unabhängig von \(\lambda\) & ELT: \(\xi \sim 10^{-6}\) \\
			CMB \(\mu\)-Distortion & \(<2.3 \times 10^{-8}\) & \(\sim (2.3+10^4\xi) \times 10^{-8}\) & \(\sim (2.3+10^4\xi^2) \times 10^{-8}\) & PIXIE: \(3\sigma\) bei \(\xi>10^{-4}\) \\
			Sandage-Test & \(\dot{z} = H_0(1+z)-H(z)\) & \(\dot{z} = H_0(1+z)(1+\xi\lambda)\) & \(\dot{z} = H_0(1+z)e^{-\xi t}\) & ELT: \(10^{-7}\)/Jahr \\
			\bottomrule
		\end{tabular}
	}
\end{table}

\subsubsection{Kritische Experimente zur Unterscheidung}

\begin{experiment}
	\textbf{Entscheidungsexperimente:}
	\begin{enumerate}
		\item \textbf{Wellenlängenabhängige Rotverschiebung}
		\begin{itemize}
			\item Test: Vergleich von \(z\) bei verschiedenen \(\lambda\) für identische Quellen
			\item T0-Beugung: \(z(\lambda_2) - z(\lambda_1) = \xi \ln(\lambda_2/\lambda_1)\)
			\item Messgenauigkeit: \(\Delta z/z \sim 10^{-6}\) (ELT-HIRES)
		\end{itemize}
		
		\item \textbf{Zeitliche Variation der Feinstrukturkonstante}
		\begin{itemize}
			\item T0-Beugung: \(\Delta\alpha/\alpha = \xi(z) \times f(\lambda)\)
			\item Quasar-Absorptionslinien bei \(z > 2\)
			\item Aktuelle Grenze: \(|\Delta\alpha/\alpha| < 10^{-6}\)
		\end{itemize}
		
		\item \textbf{Transiente Ereignisse (GRBs, SNe)}
		\begin{itemize}
			\item Lichtkurvenverzerrung durch dispersive Effekte
			\item T0-spezifische Zeitdilatation: \(\Delta t_{obs} = (1+z)(1+\xi\lambda)\Delta t_{em}\)
		\end{itemize}
	\end{enumerate}
\end{experiment}

\subsection{Konsistenz mit CMB-Berechnungen}
\label{subsec:cmb_consistency}

\subsubsection{Modifizierte Boltzmann-Gleichungen}

Die T0-Theorie modifiziert die CMB-Anisotropie-Entwicklung:

\begin{equation}
	\dot{\Theta} + ik\mu\Theta + \dot{\Phi} = \tau'[\Theta_0 - \Theta + \mu v_b - \frac{1}{2}P_2(\mu)\Pi] + \xi\frac{\dot{T}}{T_0}\Theta_1 + \mathcal{S}_{beugung}
\end{equation}

mit dem Beugungsquellterm:
\begin{equation}
	\mathcal{S}_{beugung} = \xi k^2 \int \frac{d^3k'}{(2\pi)^3} G(k,k') \Theta(k')
\end{equation}

\subsubsection{Modifikation des Leistungsspektrums}

Das CMB-Leistungsspektrum wird modifiziert:
\begin{equation}
	C_\ell^{T0} = C_\ell^{\Lambda CDM} \times \left[1 + \xi f_\ell(\lambda_{CMB})\right]
\end{equation}

mit der Korrekturfunktion:
\begin{equation}
	f_\ell(\lambda) = \begin{cases}
		\ell^{-0.3} & \ell < 100 \text{ (große Winkel)} \\
		\ell^{0.1}\sin(\ell/300) & 100 < \ell < 2000 \text{ (akustische Peaks)} \\
		\ell^{-0.5} & \ell > 2000 \text{ (Dämpfung)}
	\end{cases}
\end{equation}

Für \(\xi = 1.33 \times 10^{-4}\):
\begin{itemize}
	\item Verschiebung der Peak-Positionen: \(\Delta\ell/\ell \approx 0.02\%\)
	\item Amplitudenmodulation: \(\Delta C_\ell/C_\ell \approx 0.1\%\)
	\item Beide Effekte liegen innerhalb der Planck-Fehlerbalken
\end{itemize}

\subsection{Quantenfeldtheoretische Grundlagen}
\label{subsec:qft_foundations}

\subsubsection{Vakuumfluktuationen und T-Feld}

Das T-Feld-Vakuum zeigt Quantenfluktuationen:
\begin{equation}
	\langle 0|T^2|0\rangle = \int \frac{d^3k}{(2\pi)^3} \frac{1}{2\omega_k} = \frac{\xi}{4\pi^2}\Lambda_{UV}^2
\end{equation}

Diese führen zu einem effektiven Brechungsindex:
\begin{equation}
	n_{eff} = 1 + \frac{\alpha\xi}{2\pi} \ln\left(\frac{\Lambda_{UV}}{\omega}\right)
\end{equation}

\subsubsection{Renormierungsgruppen-Fluss}

Die Skalenabhängigkeit von \(\xi\):
\begin{equation}
	\xi(\mu) = \frac{\xi_0}{1 + \frac{\xi_0}{4\pi}\ln(\mu/\mu_0)}
\end{equation}

Dies erklärt die beobachtete Hierarchie:
\begin{itemize}
	\item Laborskala: \(\xi \sim 10^{-4}\)
	\item Kosmologische Skala: \(\xi_{eff} \sim 10^{-5}\)
\end{itemize}

\subsection{Robustheit der T0-Theorie gegenüber experimentellen Tests}
\label{subsec:robustness}

\begin{important}
	\textbf{Strukturelle Widerstandsfähigkeit der Theorie}
	
	Ein fundamentaler Aspekt der T0-Theorie ist ihre strukturelle Robustheit. Selbst wenn spezifische Vorhersagen zur Rotverschiebung nicht vollständig bestätigt werden sollten, bleiben die Kernaussagen der Theorie gültig. Dies unterscheidet die T0-Theorie fundamental von monolithischen Theorien, die bei einer einzigen experimentellen Widerlegung kollabieren.
\end{important}

\subsubsection{Hierarchie der theoretischen Vorhersagen}

Die T0-Theorie basiert auf einer mehrstufigen Struktur von Vorhersagen, die unterschiedliche Grade der Abhängigkeit voneinander aufweisen:

\begin{enumerate}
	\item \textbf{Fundamentale geometrische Basis} (unabhängig von Rotverschiebungsmechanismus):
	\begin{itemize}
		\item Fraktale Dimension \(D_f = 2.94\) aus kritischen Exponenten
		\item Geometrischer Parameter \(\xi = 4/3 \times 10^{-4}\) aus Tetraeder-Quantisierung
		\item Massenverhältnisse der Leptonen aus geometrischen Quantenzahlen
	\end{itemize}
	
	\item \textbf{Abgeleitete Größen} (teilweise unabhängig):
	\begin{itemize}
		\item Feinstrukturkonstante \(\alpha \approx 1/137\) aus fraktaler Renormierung
		\item Magnetische Momente (g-2) aus geometrischen Korrekturen
		\item Hubble-Spannung durch 4/3-Skalierung
	\end{itemize}
	
	\item \textbf{Spezifische Implementierung} (modellabhängig):
	\begin{itemize}
		\item Vakuumbeugung vs. Energieverlust vs. modifizierte Metrik
		\item Wellenlängenabhängigkeit der Rotverschiebung
		\item Dispersionsrelationen für FRBs
	\end{itemize}
\end{enumerate}

\subsubsection{Szenarien bei experimenteller Nicht-Bestätigung}

\textbf{Szenario 1: Keine nachweisbare \(\lambda\)-Abhängigkeit von \(z\)}

Falls zukünftige Experimente keine Wellenlängenabhängigkeit der Rotverschiebung finden:
\begin{itemize}
	\item \textbf{Was sich ändert}: Der spezifische Beugungsmechanismus müsste modifiziert werden, möglicherweise \(\xi < 10^{-6}\) statt \(10^{-4}\) für kosmologische Skalen
	\item \textbf{Was bleibt}: 
	\begin{itemize}
		\item Die geometrische Herleitung von \(\xi = 4/3 \times 10^{-4}\) für lokale Phänomene
		\item Die exakte Vorhersage der Muon g-2 Anomalie
		\item Die Erklärung der Hubble-Spannung durch lokale vs. globale Skalierung
		\item Die parameterfreie Berechnung der Leptonen-Massenverhältnisse
	\end{itemize}
\end{itemize}

\textbf{Szenario 2: Sandage-Test zeigt keine T0-Signatur}

Falls \(d z/dt\) exakt den \(\Lambda\)CDM-Vorhersagen folgt:
\begin{itemize}
	\item \textbf{Was sich ändert}: Die zeitliche Evolution des T-Feldes müsste angepasst werden (\(dn/dt \approx 0\) auf kosmologischen Zeitskalen)
	\item \textbf{Was bleibt}:
	\begin{itemize}
		\item Die fraktale Struktur der Raumzeit mit \(D_f = 2.94\)
		\item Die geometrische Interpretation der Fundamentalkonstanten
		\item Alle Laborexperiment-Vorhersagen (g-2, Massenverhältnisse)
	\end{itemize}
\end{itemize}

\textbf{Szenario 3: CMB-Spektrum zeigt keine \(\mu\)-Distortion}

Falls PIXIE/LiteBIRD keine T0-spezifischen Verzerrungen finden:
\begin{itemize}
	\item \textbf{Was sich ändert}: Die Kopplung des T-Feldes an Photonen bei CMB-Temperaturen
	\item \textbf{Was bleibt}: Alle anderen Vorhersagen, da CMB-Physik nur einen Aspekt der Theorie testet
\end{itemize}

\subsubsection{Vergleich mit der Robustheit des Standardmodells}

\begin{table}[h]
	\centering
	\caption{Robustheit gegenüber experimentellen Tests}
	\begin{tabular}{lcc}
		\toprule
		\textbf{Experimenteller Test} & \textbf{\(\Lambda\)CDM-Konsequenz} & \textbf{T0-Konsequenz} \\
		\midrule
		Keine Dunkle Materie gefunden & Theorie kollabiert & Unberührt (keine DM postuliert) \\
		Keine Dunkle Energie & Theorie kollabiert & Unberührt (geometrische Erklärung) \\
		g-2 Anomalie bestätigt & Neue Physik nötig & Bereits erklärt \\
		\(z(\lambda)\) nicht gefunden & Konsistent & Anpassung der Beugungsparameter \\
		Hubble-Spannung persistent & Ungeklärt & Natürlich erklärt durch 4/3 \\
		\bottomrule
	\end{tabular}
\end{table}

\subsubsection{Kernaussagen bleiben unabhängig von Einzeltests}

Die fundamentalen Erfolge der T0-Theorie sind voneinander unabhängig:

\begin{enumerate}
	\item \textbf{Parameterfreie Präzision}: Die Vorhersage von \(g_\mu - 2 = 2.00233184\) (exakt auf 8 Stellen) bleibt gültig, unabhängig von kosmologischen Tests
	
	\item \textbf{Geometrische Konstanten}: Die Herleitung von \(\alpha \approx 1/137\) aus \((4/3)^3\)-Skalierung bleibt bestehen
	
	\item \textbf{Massenhierarchie}: \(m_e : m_\mu : m_\tau = 1 : 206.768 : 3477.15\) folgt aus Quantenzahlen, nicht aus Rotverschiebung
	
	\item \textbf{Hubble-Spannung}: Die 4/3-Erklärung funktioniert unabhängig vom spezifischen Mechanismus
\end{enumerate}

\subsubsection{Adaptivität der theoretischen Struktur}

Die T0-Theorie verfügt über natürliche Anpassungsmechanismen:

\begin{equation}
	\xi_{eff}(\text{Skala}) = \xi_0 \times f(\text{Umgebung}) \times g(\text{Energie})
\end{equation}

wobei:
\begin{itemize}
	\item \(f(\text{Umgebung}) = 4/3\) in Galaxienhaufen, \(= 1\) im intergalaktischen Medium
	\item \(g(\text{Energie})\) die Renormierungsgruppen-Laufkopplung beschreibt
\end{itemize}

Diese Flexibilität ist keine ad-hoc Anpassung, sondern folgt aus der geometrischen Struktur der Theorie.


% Bibliography
\bibliographystyle{unsrt}
\begin{thebibliography}{99}
	
	% Primary T0-Theory Documents
	\bibitem{pascher_lagrangian_en}
	Pascher, Johann (2025). 
	\textit{Simplified Lagrangian Density and Time-Mass Duality in T0-Theory}. 
	T0-Theory Project. 
	\url{https://jpascher.github.io/T0-Time-Mass-Duality/2/pdf/lagrandian-einfachDe.pdf}
	
	\bibitem{pascher_cosmos_en}
	Pascher, Johann (2025). 
	\textit{T0-Model: A unified, static, cyclic, dark-matter-free and dark-energy-free universe}. 
	T0-Theory Project. 
	\url{https://jpascher.github.io/T0-Time-Mass-Duality/2/pdf/cos_De.pdf}
	
	\bibitem{pascher_cmb_en}
	Pascher, Johann (2025). 
	\textit{Temperature Units in Natural Units: T0-Theory and Static Universe}. 
	T0-Theory Project. 
	\url{https://jpascher.github.io/T0-Time-Mass-Duality/2/pdf/TempEinheitenCMBDe.pdf}
	
	\bibitem{pascher_gravitation_en}
	Pascher, Johann (2025). 
	\textit{Geometric Determination of the Gravitational Constant: From the T0-Model}. 
	T0-Theory Project. 
	\url{https://jpascher.github.io/T0-Time-Mass-Duality/2/pdf/gravitationskonstnte_De.pdf}
	

	
	\bibitem{pascher_derivation_beta}
	Pascher, J. (2025). 
	\textit{Field-Theoretic Derivation of the $\beta_T$ Parameter in Natural Units ($\hbar = c = 1$)}. 
	GitHub Repository: T0-Time-Mass-Duality.
	\url{https://github.com/jpascher/T0-Time-Mass-Duality/blob/main/2/pdf/DerivationVonBetaDe.pdf}
	
	\bibitem{pascher_unified}
	J. Pascher (2025).
	\textit{Mathematical Proof: The Fine Structure Constant $\alpha = 1$ in Natural Units}.
	\url{https://github.com/jpascher/T0-Time-Mass-Duality/blob/main/2/pdf/ResolvingTheConstantsAlfaDe.pdf}
	
	\bibitem{pascher_muon_g2}
	J. Pascher (2025).
	\textit{Complete Calculation of the Muon's Anomalous Magnetic Moment in the Unified Natural Unit System}.
	\url{https://github.com/jpascher/T0-Time-Mass-Duality/blob/main/2/pdf/CompleteMuon_g-2_AnalysisDe.pdf}
	
	\bibitem{pascher_pragmatic}
	J. Pascher (2025).
	\textit{Established Calculations in the Unified Natural Unit System: Reinterpretation Rather Than Rejection}.
	\url{https://github.com/jpascher/T0-Time-Mass-Duality/blob/main/2/pdf/PragmaticApproachT0-ModelDe.pdf}
	
	% Fundamental Physics References
	\bibitem{heisenberg1927}
	Heisenberg, W. (1927). 
	\textit{On the intuitive content of quantum theoretical kinematics and mechanics}. 
	Zeitschrift f\"ur Physik, 43(3-4), 172--198.
	
	\bibitem{einstein1915}
	Einstein, A. (1915). 
	\textit{Die Feldgleichungen der Gravitation}. 
	Sitzungsberichte der Preu\ss{}ischen Akademie der Wissenschaften, 844--847.
	
	\bibitem{dirac1928}
	Dirac, P. A. M. (1928). 
	\textit{The Quantum Theory of the Electron}. 
	Proc. R. Soc. London A, 117, 610.
	
	\bibitem{feynman1949}
	Feynman, R. P. (1949). 
	\textit{Space-Time Approach to Quantum Electrodynamics}. 
	Phys. Rev., 76, 769.
	
	\bibitem{higgs1964}
	Higgs, P. W. (1964).
	\textit{Broken Symmetries and the Masses of Gauge Bosons}.
	Phys. Rev. Lett., 13, 508.
	
	\bibitem{weinberg1967}
	Weinberg, S. (1967).
	\textit{A Model of Leptons}.
	Phys. Rev. Lett., 19, 1264.
	
	\bibitem{yang1954}
	Yang, C. N. and Mills, R. L. (1954).
	\textit{Conservation of Isotopic Spin and Isotopic Gauge Invariance}.
	Phys. Rev., 96, 191.
	
	% Cosmological Observations
	\bibitem{planck2020}
	Planck Collaboration (2020). 
	\textit{Planck 2018 results. VI. Cosmological parameters}. 
	Astronomy \& Astrophysics, 641, A6. 
	\url{https://doi.org/10.1051/0004-6361/201833910}
	
	\bibitem{riess2022}
	Riess, A. G., et al. (2022). 
	\textit{A Comprehensive Measurement of the Local Value of the Hubble Constant with 1 km s$^{-1}$ Mpc$^{-1}$ Uncertainty from the Hubble Space Telescope and the SH0ES Team}. 
	The Astrophysical Journal Letters, 934(1), L7. 
	\url{https://doi.org/10.3847/2041-8213/ac5c5b}
	
	\bibitem{jwst_early}
	Naidu, R. P., et al. (2022). 
	\textit{Two Remarkably Luminous Galaxy Candidates at z $\approx$ 11--13 Revealed by JWST}. 
	The Astrophysical Journal Letters, 940(1), L14. 
	\url{https://doi.org/10.3847/2041-8213/ac9b22}
	
	\bibitem{cobe1992}
	COBE Collaboration (1992). 
	\textit{Structure in the COBE differential microwave radiometer first-year maps}. 
	The Astrophysical Journal Letters, 396, L1--L5. 
	\url{https://doi.org/10.1086/186504}
	
	% Experimental Physics
	\bibitem{codata2018}
	CODATA (2018). 
	\textit{The 2018 CODATA Recommended Values of the Fundamental Physical Constants}. 
	National Institute of Standards and Technology. 
	\url{https://physics.nist.gov/cuu/Constants/}
	
	\bibitem{casimir1948}
	Casimir, H. B. G. (1948). 
	\textit{On the attraction between two perfectly conducting plates}. 
	Proceedings of the Royal Netherlands Academy of Arts and Sciences, 51(7), 793--795.
	
	\bibitem{lamoreaux1997}
	Lamoreaux, S. K. (1997). 
	\textit{Demonstration of the Casimir force in the 0.6 to 6 $\mu$m range}. 
	Physical Review Letters, 78(1), 5--8. 
	\url{https://doi.org/10.1103/PhysRevLett.78.5}
	
	\bibitem{muon_g2_2021}
	Muon g-2 Collaboration (2021). 
	\textit{Measurement of the Positive Muon Anomalous Magnetic Moment to 0.46 ppm}. 
	Physical Review Letters, 126(14), 141801. 
	\url{https://doi.org/10.1103/PhysRevLett.126.141801}
	
	\bibitem{katrin_2024}
	KATRIN Collaboration (2024). 
	\textit{Direct neutrino-mass measurement based on 259 days of KATRIN data}. 
	arXiv:2406.13516.
	
	\bibitem{pound1960}
	Pound, R. V. and Rebka Jr., G. A. (1960).
	\textit{Apparent Weight of Photons}.
	Phys. Rev. Lett., 4, 337--341.
	
	% Additional Theoretical Works
	\bibitem{kaluza1921}
	Kaluza, T. (1921).
	\textit{Zum Unitätsproblem der Physik}.
	Sitzungsber. Preuss. Akad. Wiss. Berlin (Math. Phys.), 966--972.
	
	\bibitem{klein1926}
	Klein, O. (1926).
	\textit{Quantentheorie und fünfdimensionale Relativitätstheorie}.
	Z. Phys., 37, 895--906.
	
	\bibitem{yukawa1935}
	Yukawa, H. (1935).
	\textit{On the Interaction of Elementary Particles}.
	Proc. Phys. Math. Soc. Japan, 17, 48.
	
	\bibitem{bohr1928}
	Bohr, N. (1928).
	\textit{The Quantum Postulate and the Recent Development of Atomic Theory}.
	Nature, 121, 580.
	
\end{thebibliography}

\end{document}