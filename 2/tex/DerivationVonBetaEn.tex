\documentclass[12pt,a4paper]{article}
\usepackage[utf8]{inputenc}
\usepackage[T1]{fontenc}
\usepackage{lmodern}
\usepackage{amsmath}
\usepackage{amssymb}
\usepackage{physics}
\usepackage{hyperref}
\usepackage{tcolorbox}
\usepackage{booktabs}
\usepackage{enumitem}
\usepackage[table,xcdraw]{xcolor}
\usepackage[left=2cm,right=2cm,top=2cm,bottom=2cm]{geometry}
\usepackage{pgfplots}
\pgfplotsset{compat=1.18}
\usepackage{graphicx}
\usepackage{float}
\usepackage{fancyhdr}
\usepackage{siunitx}
\usepackage{mathtools}
\usepackage{amsthm}
\usepackage{cleveref}
\usepackage{tocloft}
\usepackage{tikz}
\usepackage[dvipsnames]{xcolor}
\usetikzlibrary{positioning, shapes.geometric, arrows.meta}
\usepackage{microtype}
\usepackage{natbib}
\usepackage{doi}

% Cross-referencing configuration
\crefname{equation}{Eq.}{Eqs.}
\crefname{section}{Sec.}{Secs.}
\crefname{subsection}{Sec.}{Secs.}
\crefname{table}{Tab.}{Tabs.}
\crefname{figure}{Fig.}{Figs.}

% Custom commands
\newcommand{\Tfield}{T(x)}
\newcommand{\alphaEM}{\alpha_{\text{EM}}}
\newcommand{\alphaW}{\alpha_{\text{W}}}
\newcommand{\betaT}{\beta_{\text{T}}}
\newcommand{\Mpl}{M_{\text{Pl}}}
\newcommand{\Tzerot}{T_0(\Tfield)}
\newcommand{\Tzero}{T_0}
\newcommand{\vecx}{\vec{x}}
\newcommand{\vr}{\vec{r}}
\newcommand{\gammaf}{\gamma_{\text{Lorentz}}}
\newcommand{\LCDM}{\Lambda\text{CDM}}
\newcommand{\DTmu}{D_{T,\mu}}
\newcommand{\calL}{\mathcal{L}}
\newcommand{\deq}{\displaystyle}
\newcommand{\e}{\mathrm{e}}
\newcommand{\alphaT}{\alpha_{\text{T}}}
\newcommand{\lP}{\ell_{\text{P}}}

% Header and footer configuration
\pagestyle{fancy}
\fancyhf{}
\fancyhead[L]{Johann Pascher}
\fancyhead[R]{Field-Theoretic Derivation of the $\beta$-Parameter}
\fancyfoot[C]{\thepage}
\renewcommand{\headrulewidth}{0.4pt}
\renewcommand{\footrulewidth}{0.4pt}

% Hyperref configuration
\hypersetup{
	colorlinks=true,
	linkcolor=blue,
	citecolor=red,
	urlcolor=blue,
	bookmarks=true,
	bookmarksnumbered=true,
	pdfstartview=FitH,
	pdftitle={T0 Model - Field-Theoretic Derivation of the Beta Parameter},
	pdfauthor={Johann Pascher},
	pdfsubject={T0 Model, Beta Parameter, Natural Units, Quantum Field Theory},
	pdfkeywords={Time Field, Beta Parameter, Planck Units, General Relativity}
}

% Theorem environments
\newtheorem{theorem}{Theorem}[section]
\newtheorem{proposition}[theorem]{Proposition}
\newtheorem{definition}[theorem]{Definition}

\begin{document}
	
	\title{T0 Model: Field-Theoretic Derivation of the $\beta$-Parameter \\
		in Natural Units ($\hbar = c = 1$)}
	\author{Johann Pascher\\
		Department of Communication Technology\\
		Higher Technical Federal Institute (HTL), Leonding, Austria\\
		\texttt{johann.pascher@gmail.com}}
	\date{\today}
	
	\maketitle
	\tableofcontents
	\newpage
	
	\section{Introduction and Motivation}
	\label{sec:introduction}
	
	The T0 model introduces a fundamentally new perspective on spacetime, where time itself becomes a dynamic field. At the center of this theory lies the dimensionless $\beta$-parameter, which characterizes the strength of the time field and establishes a direct connection between gravitational and electromagnetic interactions.
	
	This work focuses exclusively on the mathematically rigorous derivation of the $\beta$-parameter from the fundamental field equations of the T0 model, avoiding the complexity of additional scaling parameters.
	
	\begin{tcolorbox}[colback=blue!5!white,colframe=blue!75!black,title=Central Result]
		The $\beta$-parameter is derived as:
		\begin{equation}
			\boxed{\beta = \frac{2Gm}{r}}
		\end{equation}
		where $G$ is the gravitational constant, $m$ is the source mass, and $r$ is the distance from the source.
	\end{tcolorbox}
	
	\section{Natural Units Framework}
	\label{sec:natural_units}
	
	The T0 model employs the system of natural units established in modern quantum field theory \citep{peskin1995,weinberg1995}:
	
	\begin{itemize}
		\item $\hbar = 1$ (reduced Planck constant)
		\item $c = 1$ (speed of light)
	\end{itemize}
	
	This system reduces all physical quantities to energy dimensions and follows the tradition established by Dirac \citep{dirac1958}.
	
	\begin{tcolorbox}[colback=blue!5!white,colframe=blue!75!black,title=Dimensions in Natural Units]
		\begin{itemize}
			\item Length: $[L] = [E^{-1}]$
			\item Time: $[T] = [E^{-1}]$ 
			\item Mass: $[M] = [E]$
			\item The $\beta$-parameter: $[\beta] = [1]$ (dimensionless)
		\end{itemize}
	\end{tcolorbox}
	
	\section{Fundamental Structure of the T0 Model}
	\label{sec:fundamental_structure}
	
	\subsection{Time-Mass Duality}
	\label{subsec:time_mass_duality}
	
	The central principle of the T0 model is the time-mass duality, which states that time and mass are inversely linked. This relationship differs fundamentally from the conventional treatment in general relativity \citep{einstein1915,misner1973}.
	
	\begin{table}[htbp]
		\centering
		\begin{tabular}{|l|c|c|c|}
			\hline
			\textbf{Theory} & \textbf{Time} & \textbf{Mass} & \textbf{Reference} \\
			\hline
			Einstein GR & $dt' = \sqrt{g_{00}} dt$ & $m_0 = \text{const}$ & \citep{einstein1915,misner1973} \\
			Special Relativity & $t' = \gamma t$ & $m_0 = \text{const}$ & \citep{einstein1905} \\
			T0 Model & $T(x) = \frac{1}{m(x)}$ & $m(x) = \text{dynamic}$ & This work \\
			\hline
		\end{tabular}
		\caption{Comparison of time-mass treatment in different theories}
		\label{tab:theory_comparison}
	\end{table}
	
	\subsection{Fundamental Field Equation}
	\label{subsec:field_equation}
	
	The fundamental field equation of the T0 model is derived from variational principles, analogous to the approach for scalar field theories \citep{weinberg1995}:
	
	\begin{equation}
		\label{eq:field_equation_fundamental}
		\nabla^2 m(x) = 4\pi G \rho(x) \cdot m(x)
	\end{equation}
	
	This equation shows structural similarity to the Poisson equation of gravitation $\nabla^2 \phi = 4\pi G \rho$ \citep{jackson1998}, but is nonlinear due to the factor $m(x)$ on the right-hand side.
	
	The time field follows directly from the inverse relationship:
	\begin{equation}
		\label{eq:time_field_definition}
		T(x) = \frac{1}{m(x)}
	\end{equation}
	
	\section{Geometric Derivation of the $\beta$-Parameter}
	\label{sec:beta_derivation}
	
	\subsection{Spherically Symmetric Point Source}
	\label{subsec:spherical_solution}
	
	For a point mass source, we use the established methodology for solving Einstein's field equations \citep{schwarzschild1916,misner1973}. The mass density of a point source is described by the Dirac delta function:
	
	\begin{equation}
		\rho(\vec{x}) = m_0 \cdot \delta^3(\vec{x})
	\end{equation}
	
	where $m_0$ is the mass of the point source.
	
	\subsection{Solution of the Field Equation}
	\label{subsec:field_solution}
	
	Outside the source ($r > 0$), where $\rho = 0$, the field equation reduces to:
	
	\begin{equation}
		\nabla^2 m(r) = 0
	\end{equation}
	
	The spherically symmetric Laplace operator \citep{jackson1998,griffiths1999} yields:
	
	\begin{equation}
		\frac{1}{r^2}\frac{d}{dr}\left(r^2 \frac{dm}{dr}\right) = 0
	\end{equation}
	
	The general solution to this equation is:
	
	\begin{equation}
		m(r) = \frac{C_1}{r} + C_2
	\end{equation}
	
	\subsection{Determination of Integration Constants}
	\label{subsec:integration_constants}
	
	\textbf{Asymptotic boundary condition}: For large distances, the time field should assume a constant value $T_0$:
	\begin{equation}
		\lim_{r \to \infty} T(r) = T_0 \quad \Rightarrow \quad \lim_{r \to \infty} m(r) = \frac{1}{T_0}
	\end{equation}
	
	This gives us: $C_2 = \frac{1}{T_0}$
	
	\textbf{Behavior at the origin}: Using Gauss's theorem \citep{griffiths1999,jackson1998} for a small sphere around the origin:
	\begin{equation}
		\oint_S \nabla m \cdot d\vec{S} = 4\pi G \int_V \rho(r) m(r) \, dV
	\end{equation}
	
	For a small radius $\epsilon$:
	\begin{equation}
		4\pi \epsilon^2 \left.\frac{dm}{dr}\right|_{r=\epsilon} = 4\pi G m_0 \cdot m(\epsilon)
	\end{equation}
	
	With $\frac{dm}{dr} = -\frac{C_1}{r^2}$ and $m(\epsilon) \approx \frac{1}{T_0}$ for small $\epsilon$:
	\begin{equation}
		4\pi \epsilon^2 \cdot \left(-\frac{C_1}{\epsilon^2}\right) = 4\pi G m_0 \cdot \frac{1}{T_0}
	\end{equation}
	
	This yields: $C_1 = \frac{G m_0}{T_0}$
	
	\subsection{The Characteristic Length Scale}
	\label{subsec:characteristic_length}
	
	The complete solution reads:
	\begin{equation}
		m(r) = \frac{1}{T_0}\left(1 + \frac{G m_0}{r}\right)
	\end{equation}
	
	The corresponding time field is:
	\begin{equation}
		T(r) = \frac{T_0}{1 + \frac{G m_0}{r}}
	\end{equation}
	
	For the practically important case $G m_0 \ll r$, we obtain the approximation:
	\begin{equation}
		T(r) \approx T_0\left(1 - \frac{G m_0}{r}\right)
	\end{equation}
	
	The characteristic length scale at which the time field significantly deviates from $T_0$ is:
	\begin{equation}
		\boxed{r_0 = G m_0}
	\end{equation}
	
	This scale is proportional to half the Schwarzschild radius $r_s = 2GM/c^2 = 2Gm$ in geometric units \citep{misner1973,carroll2004}.
	
	\subsection{Definition of the $\beta$-Parameter}
	\label{subsec:beta_definition}
	
	The dimensionless $\beta$-parameter is defined as the ratio of the characteristic length scale to the actual distance:
	
	\begin{equation}
		\boxed{\beta = \frac{r_0}{r} = \frac{G m_0}{r}}
	\end{equation}
	
	This parameter measures the relative strength of the time field at a given point. For astronomical objects, we can write the more general form:
	
	\begin{equation}
		\boxed{\beta = \frac{2Gm}{r}}
	\end{equation}
	
	where the factor of 2 arises from the complete relativistic treatment, analogous to the emergence of the Schwarzschild radius.
	
	\section{Physical Interpretation of the $\beta$-Parameter}
	\label{sec:physical_interpretation}
	
	\subsection{Dimensional Analysis}
	\label{subsec:dimensional_analysis}
	
	The dimensionlessness of the $\beta$-parameter in natural units:
	\begin{equation}
		[\beta] = \frac{[G][m]}{[r]} = \frac{[E^{-2}][E]}{[E^{-1}]} = [1]
	\end{equation}
	
	\subsection{Connection to Classical Physics}
	\label{subsec:classical_connection}
	
	The $\beta$-parameter shows direct connections to established physical concepts:
	
	\begin{itemize}
		\item \textbf{Gravitational potential}: $\beta$ is proportional to the Newtonian potential $\Phi = -Gm/r$
		\item \textbf{Schwarzschild radius}: $\beta = r_s/(2r)$ in geometric units
		\item \textbf{Escape velocity}: $\beta$ is related to $v_{\text{esc}}^2/c^2$
	\end{itemize}
	
	\subsection{Limiting Cases and Application Domains}
	\label{subsec:limiting_cases}
	
	\begin{table}[htbp]
		\centering
		\begin{tabular}{lcc}
			\toprule
			\textbf{Physical System} & \textbf{Typical $\beta$-Value} & \textbf{Regime} \\
			\midrule
			Hydrogen atom & $\sim 10^{-39}$ & Quantum mechanics \\
			Earth (surface) & $\sim 10^{-9}$ & Weak gravitation \\
			Sun (surface) & $\sim 10^{-6}$ & Stellar physics \\
			Neutron star & $\sim 0.1$ & Strong gravitation \\
			Schwarzschild horizon & $\beta = 1$ & Limiting case \\
			\bottomrule
		\end{tabular}
		\caption{Typical $\beta$-values for various physical systems}
		\label{tab:beta_values}
	\end{table}
	
	\section{Comparison with Established Theories}
	\label{sec:theory_comparison}
	
	\subsection{Connection to General Relativity}
	\label{subsec:gr_connection}
	
	In general relativity, the parameter $rs/r = 2Gm/r$ characterizes the strength of the gravitational field. The T0 parameter $\beta = 2Gm/r$ is identical to this expression, revealing a deep connection between both theories.
	
	\subsection{Differences from the Standard Model}
	\label{subsec:sm_differences}
	
	While the Standard Model of particle physics treats time as an external parameter, the T0 model makes time a dynamic field. The $\beta$-parameter quantifies this dynamics and represents a measurable deviation from standard physics.
	
	\section{Experimental Predictions}
	\label{sec:experimental_predictions}
	
	\subsection{Time Dilation Effects}
	\label{subsec:time_dilation}
	
	The T0 model predicts a modified time dilation:
	\begin{equation}
		\frac{dt}{dt_0} = 1 - \beta = 1 - \frac{2Gm}{r}
	\end{equation}
	
	This relationship is identical to the gravitational time dilation of GR in first order, but offers a fundamentally different theoretical foundation.
	
	\subsection{Spectroscopic Tests}
	\label{subsec:spectroscopic_tests}
	
	The $\beta$-parameter could be tested through high-precision spectroscopy:
	\begin{itemize}
		\item Gravitational redshift in stellar spectra
		\item Atomic clock experiments in different gravitational potentials
		\item High-precision interferometry
	\end{itemize}
	
	\section{Mathematical Consistency}
	\label{sec:mathematical_consistency}
	
	\subsection{Conservation Laws}
	\label{subsec:conservation_laws}
	
	The derivation of the $\beta$-parameter respects fundamental conservation laws:
	\begin{itemize}
		\item \textbf{Energy conservation}: Guaranteed by the Lagrangian formulation
		\item \textbf{Momentum conservation}: From spatial translation invariance
		\item \textbf{Dimensional consistency}: Verified in all derivation steps
	\end{itemize}
	
	\subsection{Solution Stability}
	\label{subsec:solution_stability}
	
	The spherically symmetric solution is stable against small perturbations, which can be shown by linearization around the ground state solution.
	
	\section{Conclusions}
	\label{sec:conclusions}
	
	This work has derived the $\beta$-parameter of the T0 model from first principles:
	
	\begin{tcolorbox}[colback=green!5!white,colframe=green!75!black,title=Main Results]
		\begin{enumerate}
			\item \textbf{Exact derivation}: $\beta = \frac{2Gm}{r}$ from the fundamental field equation
			\item \textbf{Dimensional consistency}: The parameter is dimensionless in natural units
			\item \textbf{Physical interpretation}: $\beta$ measures the strength of the dynamic time field
			\item \textbf{Connection to GR}: Identity with the gravitational parameter of general relativity
			\item \textbf{Testable predictions}: Specific experimental signatures predicted
		\end{enumerate}
	\end{tcolorbox}
	
	The $\beta$-parameter thus represents a fundamental dimensionless constant of the T0 model that bridges quantum field theory and gravitation.
	
	\subsection{Future Work}
	\label{subsec:future_work}
	
	\textbf{Theoretical developments}:
	\begin{itemize}
		\item Quantum corrections to the classical $\beta$-parameter
		\item Cosmological applications of the T0 model
		\item Black hole physics in the T0 framework
	\end{itemize}
	
	\textbf{Experimental programs}:
	\begin{itemize}
		\item Precision measurements of gravitational time dilation
		\item Laboratory experiments with controlled mass configurations
		\item Astrophysical tests with compact objects
	\end{itemize}
	
	% Bibliography
	\bibliographystyle{natbib}
	\begin{thebibliography}{99}
		
		\bibitem[Carroll(2004)]{carroll2004}
		Carroll, S.~M.
		\newblock \textit{Spacetime and Geometry: An Introduction to General Relativity}.
		\newblock Addison-Wesley, San Francisco, CA (2004).
		
		\bibitem[Dirac(1958)]{dirac1958}
		Dirac, P.~A.~M.
		\newblock \textit{The Principles of Quantum Mechanics}.
		\newblock Oxford University Press, Oxford, 4th edition (1958).
		
		\bibitem[Einstein(1905)]{einstein1905}
		Einstein, A.
		\newblock Zur Elektrodynamik bewegter Körper.
		\newblock \textit{Annalen der Physik}, \textbf{17}, 891--921 (1905).
		
		\bibitem[Einstein(1915)]{einstein1915}
		Einstein, A.
		\newblock Die Feldgleichungen der Gravitation.
		\newblock \textit{Sitzungsberichte der Königlich Preußischen Akademie der Wissenschaften}, 844--847 (1915).
		
		\bibitem[Griffiths(1999)]{griffiths1999}
		Griffiths, D.~J.
		\newblock \textit{Introduction to Electrodynamics}.
		\newblock Prentice Hall, Upper Saddle River, NJ, 3rd edition (1999).
		
		\bibitem[Jackson(1998)]{jackson1998}
		Jackson, J.~D.
		\newblock \textit{Classical Electrodynamics}.
		\newblock John Wiley \& Sons, New York, 3rd edition (1998).
		
		\bibitem[Misner et al.(1973)]{misner1973}
		Misner, C.~W., Thorne, K.~S., and Wheeler, J.~A.
		\newblock \textit{Gravitation}.
		\newblock W. H. Freeman and Company, New York (1973).
		
		\bibitem[Peskin \& Schroeder(1995)]{peskin1995}
		Peskin, M.~E. and Schroeder, D.~V.
		\newblock \textit{An Introduction to Quantum Field Theory}.
		\newblock Addison-Wesley, Reading, MA (1995).
		
		\bibitem[Schwarzschild(1916)]{schwarzschild1916}
		Schwarzschild, K.
		\newblock Über das Gravitationsfeld eines Massenpunktes nach der Einsteinschen Theorie.
		\newblock \textit{Sitzungsberichte der Königlich Preußischen Akademie der Wissenschaften}, 189--196 (1916).
		
		\bibitem[Weinberg(1995)]{weinberg1995}
		Weinberg, S.
		\newblock \textit{The Quantum Theory of Fields, Volume I: Foundations}.
		\newblock Cambridge University Press, Cambridge (1995).
		
	\end{thebibliography}
	
	\end{document}