% Documento autónomo: T0_Introduction_Es
% Utiliza el encabezado T0 común para español
\input{../../T0_standalone_header_es}

\title{Introducción a la Teoría T0}
\author{Johann Pascher}
\date{2025}

\begin{document}

\maketitle

\chapter{Introducción a la Teoría T0}



\chapter*{Introducción}
\addcontentsline{toc}{chapter}{Introducción}

Este libro presenta el estado actual del marco de dualidad tiempo-masa T0 y sus aplicaciones a las
masas de partículas, a las constantes fundamentales, a la mecánica cuántica, a la gravedad y a la cosmología.

La parte principal del libro consiste en una serie de documentos centrales T0. Estos capítulos reflejan la
comprensión actual de la teoría y sus consecuencias cuantitativas. En la medida de lo posible, el
material ha sido reorganizado y unificado para que la estructura de la teoría sea lo más transparente posible.

Al final del libro, varios documentos más antiguos están incluidos en un apéndice. Estos textos representan
etapas anteriores del desarrollo del marco T0. No se han eliminado porque hacen visible la evolución de las
ideas y el refinamiento de las fórmulas. En muchos casos, se puede ver cómo las aproximaciones
han sido mejoradas, cómo los casos especiales han sido generalizados y cómo los nuevos datos empíricos han contribuido a
refinar o corregir los argumentos anteriores.

La versión "viva" de la teoría se mantiene en un repositorio público de GitHub:

\begin{center}
\url{https://github.com/jpascher/T0-Time-Mass-Duality}
\end{center}

Las fuentes LaTeX de los capítulos de este libro provienen de este repositorio. Si se encuentran errores conceptuales o
numéricos, se corregirán primero allí. Esto significa que la versión PDF del
libro que estás leyendo es una instantánea de un proyecto en continua evolución. Para la
versión más reciente de los documentos, incluyendo nuevos apéndices o correcciones, el repositorio GitHub
siempre debe ser considerado como la referencia principal.

La intención de esta compilación es doble:
\begin{itemize}
\item ofrecer un camino coherente y legible a través de las ideas y resultados centrales del marco T0;
\item documentar en el apéndice el desarrollo histórico de estas ideas, incluyendo los comienzos en falso,
las formulaciones intermedias y los primeros ajustes a los datos experimentales.
\end{itemize}

\end{document}
