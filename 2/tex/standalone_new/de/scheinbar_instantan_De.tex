% Standalone document: scheinbar_instantan_En
% Uses shared T0 header
% T0 Standalone Header - German Version
% Gemeinsamer Header für alle deutschen Standalone-Dokumente

\documentclass[12pt,a4paper]{article}
\usepackage[utf8]{inputenc}
\usepackage[T1]{fontenc}
\usepackage[ngerman]{babel}
\usepackage{lmodern}

% Mathematics
\usepackage{amsmath,amssymb,amsthm}
\usepackage{physics}
\usepackage{siunitx}

% Layout
\usepackage[left=2.5cm,right=2.5cm,top=2.5cm,bottom=2.5cm,headheight=15pt]{geometry}
\usepackage{fancyhdr}
\usepackage{titlesec}

% Tables and Graphics
\usepackage{booktabs}
\usepackage{array}
\usepackage{longtable}
\usepackage{graphicx}
\usepackage{tikz}
\usetikzlibrary{arrows.meta,positioning,shapes.geometric}

% Colors and Boxes
\usepackage{xcolor}
\usepackage[most]{tcolorbox}
\usepackage{mdframed}

% Additional packages
\usepackage{enumitem}
\usepackage{float}
\usepackage{caption}
\usepackage{subcaption}
\usepackage{multirow}
\usepackage{colortbl}
\usepackage{pdflscape}
\usepackage{algorithm}
\usepackage{algpseudocode}
\usepackage{listings}
\usepackage{hyperref}

% Define colors
\definecolor{t0blue}{RGB}{0,51,102}
\definecolor{t0green}{RGB}{0,102,51}
\definecolor{t0red}{RGB}{153,0,0}
\definecolor{deepblue}{RGB}{0,51,102}
\definecolor{deepgreen}{RGB}{0,102,51}
\definecolor{deepred}{RGB}{153,0,0}
\definecolor{boxgray}{RGB}{240,240,240}
\definecolor{t0yellow}{RGB}{255,200,0}
\definecolor{boxblue}{RGB}{230,240,255}
\definecolor{boxgreen}{RGB}{230,255,230}
\definecolor{boxorange}{RGB}{255,240,230}
\definecolor{boxyellow}{RGB}{255,255,230}

% Custom tcolorbox environments
\newtcolorbox{fundamental}[1][]{
  colback=blue!5!white,
  colframe=blue!75!black,
  title=#1,
  fonttitle=\bfseries,
  breakable
}

\newtcolorbox{derivation}[1][]{
  colback=green!5!white,
  colframe=green!75!black,
  title=#1,
  fonttitle=\bfseries,
  breakable
}

\newtcolorbox{result}[1][]{
  colback=orange!5!white,
  colframe=orange!75!black,
  title=#1,
  fonttitle=\bfseries,
  breakable
}

\newtcolorbox{summary}[1][]{
  colback=gray!10!white,
  colframe=gray!75!black,
  title=#1,
  fonttitle=\bfseries,
  breakable
}

\newtcolorbox{comparison}[1][]{
  colback=purple!5!white,
  colframe=purple!75!black,
  title=#1,
  fonttitle=\bfseries,
  breakable
}

\newtcolorbox{relation}[1][]{
  colback=cyan!5!white,
  colframe=cyan!75!black,
  title=#1,
  fonttitle=\bfseries,
  breakable
}

\newtcolorbox{principle}[1][]{
  colback=yellow!5!white,
  colframe=yellow!75!black,
  title=#1,
  fonttitle=\bfseries,
  breakable
}

\newtcolorbox{insight}[1][]{colback=blue!5,colframe=t0blue,title={#1},fonttitle=\bfseries,breakable}
\newtcolorbox{discovery}[1][]{colback=green!5,colframe=t0green,title={#1},fonttitle=\bfseries,breakable}
\newtcolorbox{newperspective}[1][]{colback=yellow!5,colframe=orange,title={#1},fonttitle=\bfseries,breakable}
\newtcolorbox{revelation}[1][]{colback=red!5,colframe=t0red,title={#1},fonttitle=\bfseries,breakable}
\newtcolorbox{keypoint}[1][]{colback=blue!5,colframe=t0blue,title={#1},fonttitle=\bfseries,breakable}
\newtcolorbox{evidence}[1][]{colback=green!5,colframe=t0green,title={#1},fonttitle=\bfseries,breakable}
\newtcolorbox{conclusion}[1][]{colback=gray!5,colframe=gray,title={#1},fonttitle=\bfseries,breakable}
\newtcolorbox{significance}[1][]{colback=yellow!5,colframe=orange,title={#1},fonttitle=\bfseries,breakable}
\newtcolorbox{philosophical}[1][]{colback=purple!5,colframe=purple,title={#1},fonttitle=\bfseries,breakable}
\newtcolorbox{implication}[1][]{colback=cyan!5,colframe=cyan,title={#1},fonttitle=\bfseries,breakable}
\newtcolorbox{perspective}[1][]{colback=blue!5,colframe=t0blue,title={#1},fonttitle=\bfseries,breakable}
\newtcolorbox{revolutionary}[1][]{colback=red!5,colframe=t0red,title={#1},fonttitle=\bfseries,breakable}
\newtcolorbox{technical}[1][]{colback=gray!5,colframe=gray!75!black,title={#1},fonttitle=\bfseries,breakable}
\newtcolorbox{notation}[1][]{colback=yellow!5,colframe=yellow!75!black,title={#1},fonttitle=\bfseries,breakable}

% Theorem environments
\newtheorem{theorem}{Satz}[section]
\newtheorem{lemma}[theorem]{Lemma}
\newtheorem{corollary}[theorem]{Korollar}
\newtheorem{proposition}[theorem]{Proposition}
\newtheorem{definition}[theorem]{Definition}
\newtheorem{example}[theorem]{Beispiel}
\newtheorem{remark}[theorem]{Bemerkung}
\newtheorem{note}[theorem]{Anmerkung}

% Additional environments
\newenvironment{treatise}{\begin{quote}}{\end{quote}}
\newenvironment{gemeinsam}{\begin{quote}}{\end{quote}}
\newenvironment{vergleich}{\begin{quote}}{\end{quote}}
\newenvironment{vorteil}{\begin{quote}}{\end{quote}}
\newenvironment{quantum}{\begin{quote}}{\end{quote}}

% T0-specific commands
\newcommand{\Tzero}{T$_0$}
\newcommand{\xipar}{\xi}
\newcommand{\Tfield}{T}
\newcommand{\Efield}{\mathcal{E}}
\newcommand{\meff}{m_{\text{eff}}}
\newcommand{\Eabs}{E_{\text{abs}}}
\newcommand{\taupar}{\tau}

% Header setup
\pagestyle{fancy}
\fancyhf{}
\fancyhead[L]{\leftmark}
\fancyhead[R]{\thepage}
\renewcommand{\headrulewidth}{0.4pt}

% Hyperref setup
\hypersetup{
    colorlinks=true,
    linkcolor=blue,
    filecolor=magenta,
    urlcolor=cyan,
    citecolor=blue,
    pdftitle={T0 Theory Document},
    pdfauthor={Johann Pascher}
}

% German quotation marks
%\newcommand{\dq}[1]{\glqq{}#1\grqq{}}


\title{Apparent Instantaneity}
\author{Johann Pascher}
\date{2025}

\begin{document}

\maketitle

\chapter{Apparent Instantaneity}

	
	
	\begin{abstract}
		This Arbeit demonstrates das the apparent instantaneity in the T0 formalism arises from the notation of the local Einschränkung Bedingung $T \cdot E = 1$. Through Analyse of the underlying Feld Gleichungen and hierarchical Zeit Skalen, es wird gezeigt das T0 theory provides a vollständig causal Beschreibung of Quanten Phänomene das is fully compatible with speziell Relativität. All Parameter of the theory follow from purely geometrisch Prinzipien. The Arbeit extends the Analyse to the complete duality zwischen Zeit, Masse, Energie, and Länge, and critically discusses the Grenzen of Interpretation in extreme situations.
	\end{abstract}
	
	\newpage
	\hypersetup{linkcolor=blue}
	\newpage
	
	\section{Einleitung: The Instantaneity Problem}
	
	Since the groundbreaking Arbeit of Einstein, Podolsky, and Rosen in the 1930s, physics has struggled with a fundamental paradox: Quanten Mechanik appears to require instantaneous correlations zwischen arbitrarily distant Teilchen, welche Einstein called ``spooky action at a Entfernung.'' This apparent instantaneity manifests in various Phänomene—from Welle Funktion collapse through Bell inequality violations to Quanten entanglement.
	
	The T0 formalism offers an alternative resolution to dies paradox. The core idea is das the fundamental Zusammenhang zwischen Zeit and Energie, expressed by the Gleichung $T \cdot E = 1$, is oft misunderstood. What appears at erst glance to be an instantaneous Kopplung proves upon closer examination to be a local Einschränkung Bedingung das implies no action at a Entfernung.
	
	To understand dies, we must distinguish zwischen two fundamentally unterschiedlich types of physikalisch relationships: local Einschränkung Bedingungen das apply at the gleich spatial point, and Feld Gleichungen das describe the propagation of disturbances through Raum. This distinction is the key to resolving the instantaneity paradox.
	
	\section{Apparent Instantaneity in the T0 Formalism}
	
	The T0 Gleichungen appear to imply instantaneity at erst glance, but dies is refuted through detailed Analyse of the Feld Gleichungen. The fundamental challenge is Verständnis wie a theory basierend auf the strict Zusammenhang $T \cdot E = 1$ can trotzdem respect causality. This apparent paradox has its roots in a misunderstanding ungefähr the nature of mathematisch Einschränkung Bedingungen in physics.
	
	\subsection{The Apparent Problem}
	
	The fundamental Gleichungen of the T0 formalism are:
	\begin{align}
		T(\mathbf{x},t) \cdot E(\mathbf{x},t) &= 1 \label{scheinbar_instantan:eq:TE_constraint} \\
		T &= \frac{1}{m} \quad \text{where } \omega = \frac{mc^2}{\hbar}, \text{ so } T = \frac{\hbar}{E} \label{scheinbar_instantan:eq:T_definition} \\
		E &= mc^2 \label{scheinbar_instantan:eq:E_definition}
	\end{align}
	
	These Gleichungen suggest das a change in $E$ requires an immediate adjustment of $T$. If we double the Energie at a point, zum Beispiel, the Zeit Feld seems to have to halve instantaneously. This Interpretation would indeed Mittelwert a violation of relativistisch causality and stands in apparent contradiction to the fundamental Prinzipien of modern physics.
	
	The confusion arises from the fact das diese Gleichungen are oft interpreted as dynamic relationships—as if a change in one Größe causes an instantaneous reaction in the andere. This Interpretation is fundamentally wrong and leads to the apparent paradoxes of Quanten Mechanik.
	
	\subsection{The Resolution: Field Equations Have Dynamics}
	
	The resolution of dies paradox lies in recognizing das the T0 Gleichungen contain two unterschiedlich types of relationships: local Einschränkung Bedingungen and dynamic Feld Gleichungen. This distinction is fundamental to Verständnis warum no reell instantaneity occurs.
	
	\textbf{1. The complete Feld Gleichung:}
	\begin{equation}
		\nabla^2 m = 4\pi G \rho(\mathbf{x},t) \cdot m \label{scheinbar_instantan:eq:field_equation}
	\end{equation}
	wo $\rho(\mathbf{x},t)$ is the Masse Dichte. This Gleichung is \emph{not} instantaneous but eher a Welle Gleichung with endlich propagation Geschwindigkeit $v \leq c$.
	
	This Feld Gleichung describes wie disturbances in the Masse Feld (and somit in the Zeit Feld via $T = 1/m$) propagate through Raum. Crucially, dies propagation occurs at endlich Geschwindigkeit, limited by the Geschwindigkeit of Licht. The Gleichung is zweit-Ordnung in spatial derivatives, welche is Charakteristik of Welle propagation. No information, no Energie, and no Effekt can propagate faster than the Geschwindigkeit of Licht.
	
	\textbf{2. The modified Schrödinger Gleichung:}
	\begin{equation}
		i \cdot T(\mathbf{x},t) \frac{\partial \psi}{\partial t} = H_0 \psi + V_{T0} \psi \label{scheinbar_instantan:eq:schroedinger}
	\end{equation}
	wo $H_0 = -\frac{\hbar^2}{2m}\nabla^2$ is the free Hamiltonian and $V_{T0} = \hbar^2 \delta E(\mathbf{x},t)$ is the T0-specific Potential.
	
	This modified Schrödinger Gleichung explizit shows the temporal evolution of the Welle Funktion under the Einfluss of the Zeit Feld. The presence of the Zeit derivative $\partial/\partial t$ makes clear das dies is a causal evolution, not an instantaneous adjustment. The Welle Funktion evolves kontinuierlich in Zeit gemäß local Feld Bedingungen.
	
	\section{The Critical Insight: Local vs. Global Relations}
	
	The key to Verständnis lies in distinguishing zwischen local and global physikalisch relationships. This distinction is ubiquitous in physics but oft not emphasized explizit enough. The confusion zwischen diese two types of relationships is the source of viele conceptual problems in Quanten Mechanik.
	
	\subsection{Visualization of Local vs. Global Relations}
	
	\begin{center}
		\begin{tikzpicture}[Skala=1.2]
			% Title
			\node at (6, 7) {\Large \textbf{Local Constraint vs. Global Propagation}};
			
			% Local Einschränkung (left)
			\draw[thick, fill=t0blue!20] (0,0) circle (2);
			\node at (0, 3) {\textbf{Local Level}};
			\node at (0, 2.3) {At point $\mathbf{x}_0$};
			\draw[thick, <->] (-0.8, 0.3) -- (0.8, 0.3);
			\node at (0, 0.5) {$T \cdot E = 1$};
			\node at (0, -0.2) {\klein instantaneous};
			\node at (0, -0.6) {\klein (on Planck Skala)};
			\draw[thick, t0blue] (0,0) node[circle, fill, inner sep=2pt]{};
			\node at (0, -1.2) {\klein No Dynamik};
			\node at (0, -1.6) {\klein Only Einschränkung};
			
			% Arrow to the right
			\draw[thick, ->, t0red] (2.5, 0) -- (4.5, 0);
			\node[oben] at (3.5, 0.2) {\klein Disturbance};
			
			% Global propagation (right)
			\draw[thick, fill=t0green!20] (7,0) circle (2);
			\node at (7, 3) {\textbf{Global Level}};
			\node at (7, 2.3) {Propagation to $\mathbf{x}_1$};
			% Wave propagation
			\draw[thick, t0green, ->] (5.5, 0) -- (6.5, 0);
			\draw[thick, t0green] (6.5, -0.3) sin (7, 0) cos (7.5, 0.3) sin (8, 0) cos (8.5, -0.3);
			\node at (7, -0.8) {\klein $v \leq c$};
			\node at (7, -1.2) {\klein Field Gleichung:};
			\node at (7, -1.6) {\klein $\nabla^2 m = 4\pi G \rho m$};
			
			% Time axis unten
			\draw[thick, ->] (0, -3) -- (9, -3) node[right] {Time};
			\draw[thick] (0, -3.1) -- (0, -2.9);
			\node[unten] at (0, -3.1) {$t = 0$};
			\draw[thick] (7, -3.1) -- (7, -2.9);
			\node[unten] at (7, -3.1) {$t = r/c$};
			
			% Distance
			\draw[<->, t0yellow] (0, -4) -- (7, -4);
			\node[unten] at (3.5, -4) {Distance $r = |\mathbf{x}_1 - \mathbf{x}_0|$};
			
			% Legend
			\draw[thick, t0blue, fill=t0blue!20] (10, 1) rectangle (10.3, 1.3);
			\node[right] at (10.4, 1.15) {\klein Local};
			\draw[thick, t0green, fill=t0green!20] (10, 0.3) rectangle (10.3, 0.6);
			\node[right] at (10.4, 0.45) {\klein Global};
			\draw[thick, t0red, ->] (10, -0.4) -- (10.3, -0.4);
			\node[right] at (10.4, -0.4) {\klein Disturbance};
		\end{tikzpicture}
	\end{center}
	
	This diagram illustrates the fundamental difference zwischen local and global Prozesse. On the left, we see the local Einschränkung Bedingung $T \cdot E = 1$, welche holds instantaneously (on the Planck Zeit Skala) at the gleich spatial point. On the right, we see the global propagation of a disturbance, welche occurs at endlich Geschwindigkeit $v \leq c$ and requires Zeit $t = r/c$ to bridge the Entfernung $r$.
	
	\subsection{Local Constraint Condition}
	
	\begin{equation}
		T(\mathbf{x},t) \cdot E(\mathbf{x},t) = 1 \quad \text{[AT THE SAME SPATIAL POINT]} \label{scheinbar_instantan:eq:local_constraint}
	\end{equation}
	
	This is a local Einschränkung Bedingung—analogous to $\nabla \cdot \mathbf{E} = \rho/\epsilon_0$ in Elektrodynamik. It holds instantaneously at the gleich point but does not enforce instantaneous action at a Entfernung.
	
	To deepen dies Analogie: In Elektrodynamik, Gauss's law means das the divergence of the elektrisch Feld at jeder point is proportional to the local Ladung Dichte. This is not a statement ungefähr wie changes propagate, but a Bedingung das must be satisfied locally at jeder moment in Zeit. When the Ladung Dichte changes at a point, the elektrisch Feld dort adjusts sofort, but dies change dann propagates to andere points at the Geschwindigkeit of Licht.
	
	The gleich applies to the T-E Zusammenhang in the T0 formalism. The Gleichung $T \cdot E = 1$ is a local Bedingung das must be satisfied at jeder spatial point at jeder moment. It does not describe wie changes propagate, nur the local Zusammenhang zwischen the Felder.
	
	\subsection{Causal Field Propagation}
	
	\begin{equation}
		\text{Change at } \mathbf{x}_1 \rightarrow \text{Propagation with } v \leq c \rightarrow \text{Effect at } \mathbf{x}_2
	\end{equation}
	\begin{equation}
		\text{Time delay: } \Delta t = \frac{|\mathbf{x}_2 - \mathbf{x}_1|}{c} \label{scheinbar_instantan:eq:time_delay}
	\end{equation}
	
	The tatsächlich propagation of Feld changes follows the dynamic Feld Gleichungen. When the Energie Feld changes at point $\mathbf{x}_1$, the Zeit Feld dort must sofort satisfy the Einschränkung Bedingung. However, dies local change creates a disturbance in the Feld das propagates at endlich Geschwindigkeit.
	
	The crucial point is das local adjustment and global propagation are two vollständig unterschiedlich Prozesse. Local adjustment occurs on the Planck Zeit Skala and is practically instantaneous for alle measurable purposes. Global propagation, jedoch, is limited by the Geschwindigkeit of Licht and can take considerable Zeit over macroscopic distances.
	
	\section{The Geometric Origin of T0 Parameters}
	
	A fundamental Aspekt of T0 theory is das its Parameter are not empirically adjusted but derived from geometrisch Prinzipien. This fundamentally distinguishes it from phenomenological theories and makes it a truly predictive theory.
	
	\subsection{Fundamental Geometric Derivation}
	
	T0 theory derives alle physikalisch Parameter from the Geometrie of three-dimensional Raum. The central Parameter is:
	
	\begin{tcolorbox}[colback=t0blue!5!white, colframe=t0blue!75!black, title=T0 Prediction]
		The universal Parameter
		\begin{equation}
			\xi = \frac{4}{3} \times 10^{-4}
		\end{equation}
		follows from purely geometrisch Prinzipien:
		\begin{itemize}
			\item Fractal Dimension of physikalisch Raum: $D_f = 2.94$
			\item Ratio of Charakteristik Skalen to Planck Länge
			\item Topological Eigenschaften of the Quanten Vakuum
		\end{itemize}
		This is \emph{not} an empirical adjustment but a geometrisch Vorhersage.
	\end{tcolorbox}
	
	The Bedeutung of dies geometrisch Ableitung cannot be overstated. While meist physikalisch theories contain free Parameter das must be determined from Experimente, T0 Parameter follow from the fundamental Struktur of Raum itself. This makes the theory predictive eher than descriptive in a deep sense.
	
	The Parameter $\xi$ appears in various contexts and connects scheinbar unrelated Phänomene. It determines the strength of Quanten Korrekturen, the size of Vakuum fluctuations, and the Charakteristik Skalen at welche new physics appears. This universality is strong Evidenz das we are dealing with a fundamental Konstante of nature.
	
	\subsection{Experimentell Confirmation}
	
	The geometrisch Vorhersagen of T0 theory are confirmed by various precision Experimente without requiring Parameter adjustment. This agreement zwischen geometrisch Vorhersage and experimentell Beobachtung is strong Evidenz for the validity of the T0 Ansatz.
	
	The fact das a Parameter derived from pure Geometrie can be experimentally verified is remarkable. It shows das the Struktur of Raum itself determines the beobachtet physikalisch Phänomene. This is a profound Einsicht das revolutionizes our Verständnis of fundamental physics.
	
	\section{Mathematical Specification of Field Dynamics}
	
	The complete mathematisch Struktur of T0 Feld Dynamik klar shows das alle Prozesse occur causally. This mathematisch precision is essential to resolve the apparent paradoxes and show das T0 theory is fully compatible with Relativität.
	
	\subsection{Complete Wave Gleichung}
	
	T0 Feld Dynamik follows the Gleichung:
	\begin{equation}
		\frac{\partial^2 T}{\partial t^2} = c^2\nabla^2 T + Q(T, E, \rho) \label{scheinbar_instantan:eq:wave_equation}
	\end{equation}
	wo the source Funktion
	\begin{equation}
		Q(T, E, \rho) = -4\pi G \rho \cdot T
	\end{equation}
	describes the self-Wechselwirkung of the Zeit Feld.
	
	This Welle Gleichung is of fundamental Wichtigkeit. It explizit shows das the Zeit Feld follows a hyperbolic differential Gleichung Charakteristik of Welle propagation at endlich Geschwindigkeit. The zweit derivatives in Bezug auf Zeit and Raum are in a fixed Verhältnis given by the Geschwindigkeit of Licht $c$. This guarantees das no information can be transmitted faster than Licht.
	
	\subsection{Beispiel: Energy Change and Field Propagation}
	
	To illustrate the causal nature of Feld propagation, consider a concrete example:
	
	\begin{align}
		t &= 0: \quad E(\mathbf{x}_0) \text{ changes} \\
		&\rightarrow T(\mathbf{x}_0) = \frac{1}{E(\mathbf{x}_0)} \quad \text{[local, constraint]} \\
		&\rightarrow \nabla^2 T \neq 0 \quad \text{[creates field disturbance]} \\
		&\rightarrow \text{Wave propagates with } v = c \\
		t &= \frac{r}{c}: \quad \text{Disturbance reaches point } \mathbf{x}_1
	\end{align}
	
	This Prozess klar shows the hierarchy of events: local adjustment occurs sofort (on the Planck Zeit Skala), but propagation to distant points is limited by the Geschwindigkeit of Licht.
	
	\section{Green's Function and Causality}
	
	The Green's Funktion is the mathematisch tool das vollständig characterizes the causal Struktur of Feld propagation. It describes wie a point disturbance propagates through the Feld and is somit fundamental to Verständnis causality in T0 theory.
	
	The Green's Funktion of the T0 Feld Gleichung:
	\begin{equation}
		G(\mathbf{x},\mathbf{x}',t-t') = \theta(t-t') \cdot \frac{\delta(|\mathbf{x}-\mathbf{x}'| - c(t-t'))}{4\pi|\mathbf{x}-\mathbf{x}'|} \label{scheinbar_instantan:eq:green}
	\end{equation}
	
	The Komponenten have the folgend meaning:
	\begin{itemize}
		\item $\theta(t-t')$: Heaviside Funktion guarantees causality (Effekt nach cause)
		\item $\delta$ Funktion: encodes propagation at Geschwindigkeit of Licht
		\item $1/4\pi r$: geometrisch Faktor for 3D propagation
	\end{itemize}
	
	The Struktur of dies Green's Funktion is remarkable. The Heaviside Funktion $\theta(t-t')$ is zero for $t < t'$, meaning no Effekt can occur vor its cause. This is the mathematisch Implementierung of the causality Prinzip. The delta Funktion $\delta(|\mathbf{x}-\mathbf{x}'| - c(t-t'))$ is non-zero nur wann the Entfernung equals $c$ times the elapsed Zeit—dies describes a disturbance propagating exactly at the Geschwindigkeit of Licht.
	
	\section{The Hierarchy of Time Scales}
	
	Apparent instantaneity in Quanten Mechanik results from the extreme separation of unterschiedlich Zeit Skalen. This hierarchy is fundamental to Verständnis warum viele Quanten Prozesse appear instantaneous sogar obwohl they are not.
	
	\begin{center}
		\begin{tikzpicture}[Skala=1.3]
			\draw[thick,->] (0,0) -- (0,7) node[oben] {Time Skala [s]};
			
			% Time Skalen
			\draw[thick] (-0.1,1) -- (0.1,1);
			\node[right] at (0.2,1) {$t_{\text{Planck}} \sim 10^{-43}$ s};
			\node[right] at (4,1) {\klein Local T-E adjustment};
			
			\draw[thick] (-0.1,3) -- (0.1,3);
			\node[right] at (0.2,3) {$t_{\text{QM}} \sim 10^{-15}$ s};
			\node[right] at (4,3) {\klein Wave Funktion evolution};
			
			\draw[thick] (-0.1,5) -- (0.1,5);
			\node[right] at (0.2,5) {$t_{\text{rel}} = r/c$};
			\node[right] at (4,5) {\klein Causal Feld propagation};
			
			% Regions
			\draw[dashed, gray] (-0.5,0.5) rectangle (8,1.5);
			\node[gray] at (9,1) {\footnotesize Unmeasurable};
			
			\draw[dashed, blue] (-0.5,2.5) rectangle (8,3.5);
			\node[blue] at (9.4,3) {\footnotesize Quantum regime};
			
			\draw[dashed, red] (-0.5,4.5) rectangle (8,5.5);
			\node[red] at (9,5) {\footnotesize Relativistic};
		\end{tikzpicture}
	\end{center}
	
	This hierarchy explains viele scheinbar paradoxical Aspekte of Quanten Mechanik. Processes on the Planck Skala are so fast das they cannot be temporally resolved with irgendein conceivable technology. For alle practical purposes, they appear instantaneous. The Quanten Skala is accessible to modern Experimente but noch extremely fast compared to macroscopic Zeit Skalen. Finally, the relativistisch Skala determines propagation over macroscopic distances.
	
	\section{The Complete Duality: Time, Mass, Energy, and Length}
	
	T0 theory describes not nur a Zeit-Masse duality but a comprehensive System of dualities in welche alle fundamental Größen are interconnected. This extended Perspektive is essential for a complete Verständnis of apparent instantaneity and shows das unterschiedlich physikalisch Größen are nur unterschiedlich Aspekte of the gleich underlying reality.
	
	\subsection{Visualization of Energy-Time Duality}
	
	\begin{center}
		\begin{tikzpicture}[Skala=1.3]
			% Title
			\node at (0, 6) {\Large \textbf{The Fundamental Energy-Time Duality}};
			
			% Main Gleichung in center
			\draw[thick, t0blue, fill=t0blue!10] (-2, 3.5) rectangle (2.2, 4.5);
			\node at (0, 4) {\Large $T \cdot E = 1$};
			
			% Time side (left)
			\draw[thick, t0red, fill=t0red!10] (-6, 1.5) rectangle (-3, 3);
			\node at (-4.5, 2.6) {\textbf{Time Aspect}};
			\node at (-4.5, 2.1) {$T = \frac{1}{m}$};
			\node at (-4.5, 1.6) {\klein Long times};
			\draw[thick, ->] (-3, 2.25) -- (-2.2, 3.5);
			
			% Energy side (right)
			\draw[thick, t0green, fill=t0green!10] (3, 1.5) rectangle (6, 3);
			\node at (4.5, 2.6) {\textbf{Energy Aspect}};
			\node at (4.5, 2.1) {$E = mc^2$};
			\node at (4.5, 1.6) {\klein High energies};
			\draw[thick, ->] (3, 2.25) -- (2.2, 3.5);
			
			% Length Beziehung (bottom left)
			\draw[thick, t0yellow, fill=t0yellow!10] (-6, -0.5) rectangle (-3, 1);
			\node at (-4.5, 0.6) {\textbf{Length Aspect}};
			\node at (-4.5, 0.1) {$\ell = \frac{\hbar}{mc}$};
			\node at (-4.5, -0.4) {\klein Large distances};
			\draw[thick, ->] (-4.5, 1) -- (-4.5, 1.5);
			
			% Mass Beziehung (bottom right)
			\draw[thick, t0purple, fill=t0purple!10] (3, -0.5) rectangle (6, 1);
			\node at (4.5, 0.6) {\textbf{Mass Aspect}};
			\node at (4.5, 0.1) {$m = \frac{E}{c^2}$};
			\node at (4.5, -0.4) {\klein Heavy Teilchen};
			\draw[thick, ->] (4.5, 1) -- (4.5, 1.5);
			
			% Complementarity (bottom)
			\draw[thick, dashed, gray] (-2, -2) -- (2, -2);
			\node at (0, -2.5) {\textbf{Complementarity Principle:}};
			\node at (0, -3) {The mehr precisely $T$ is determined, the weniger präzise $E$};
			\node at (0, -3.5) {$\Delta T \cdot \Delta E \geq \frac{\hbar}{2}$};
			
			% Arrows for relationships
			\draw[thick, <->, gray] (-3, 0) -- (3, 0);
			\node[oben] at (0, 0) {\klein reciprocal};
			
			% Planck Skala box
			\draw[thick, double, fill=white] (-1.8, -1.3) rectangle (1.8, -0.3);
			\node at (0, -0.8) {\klein \textbf{Planck Scale:} All equal};
			
			% Scale dependence
			\node[right] at (-1, 2.2) {\klein \textbf{Dominant at:}};
			\node[right] at (-1, 1.7) {\klein Atomic Skala: $E$-$T$};
			\node[right] at (-1, 1.2) {\klein Macroscopic: $m$};
			\node[right] at (-1, 0.7) {\klein Cosmological: $\ell$-$t$};
		\end{tikzpicture}
	\end{center}
	
	This diagram shows the fundamental Energie-Zeit duality and its connections to Masse and Länge. The central Zusammenhang $T \cdot E = 1$ connects alle Aspekte. Depending on the Skala considered, unterschiedlich Aspekte of dies duality dominate, but alle are linked by the fundamental relationships.
	
	\subsection{The Fundamental Equivalences}
	
	In the T0 formalism, the basic physikalisch Größen are linked by the folgend relationships:
	
	\begin{align}
		T \cdot E &= 1 \quad \text{(Time-Energy duality)} \\
		T &= \frac{1}{m} \quad \text{(Time-Mass relation)} \\
		E &= mc^2 \quad \text{(Mass-Energy equivalence)} \\
		\ell &= \frac{\hbar}{mc} = \frac{\hbar}{E/c} \quad \text{(Length as energy)}
	\end{align}
	
	These relationships show das lengths can auch be interpreted as Energie Skalen. The Compton Wellenlänge $\lambda_C = \hbar/(mc)$ is the paradigmatic example: it represents the Charakteristik Länge Skala at welche the Quanten nature of a Teilchen with Masse $m$ (or equivalently, Energie $E = mc^2$) becomes manifest.
	
	\subsection{The Planck Scale as Universal Reference}
	
	All diese dualities converge at the Planck Skala:
	
	\begin{align}
		\lP &= \sqrt{\frac{\hbar G}{c^3}} \quad \text{(Planck length)} \\
		\tP &= \sqrt{\frac{\hbar G}{c^5}} \quad \text{(Planck time)} \\
		\mP &= \sqrt{\frac{\hbar c}{G}} \quad \text{(Planck mass)} \\
		\EP &= \sqrt{\frac{\hbar c^5}{G}} \quad \text{(Planck energy)}
	\end{align}
	
	Remarkably, diese Größen satisfy the fundamental relationships:
	\begin{align}
		\tP \cdot \EP &= \hbar \\
		\lP &= c \cdot \tP \\
		\EP &= \mP c^2 \\
		\lP &= \frac{\hbar}{\mP c}
	\end{align}
	
	This consistency shows das the T0 dualities are not arbitrary but deeply rooted in the Struktur of Raumzeit.
	
	\section{Scale Dependence and Limits of Interpretation}
	
	T0 theory shows das the unterschiedlich Aspekte of duality—Zeit, Masse, Energie, Länge—are differently pronounced depending on the Skala considered. This Skala dependence is fundamental and calls for caution wann interpreting extreme situations.
	
	\subsection{Complementarity of Aspects}
	
	Different Aspekte dominate at unterschiedlich Skalen:
	\begin{itemize}
		\item \textbf{Planck Skala:} All Aspekte are equivalent, no Näherung gültig
		\item \textbf{Atomic Skala:} Energy-Zeit duality dominates, Gravitation negligible
		\item \textbf{Macroscopic Skala:} Mass Aspekt dominant, Quanten Effekte suppressed
		\item \textbf{Cosmological Skala:} Space-Zeit Struktur dominant, local Quanten Effekte irrelevant
	\end{itemize}
	
	\subsection{The Role of Small Corrections}
	
	Although the $\xi$ Parameter ($\xi = 4/3 \times 10^{-4}$) and gravitativ Effekte are oft extremely klein, they noch have measurable Effekte. These klein Korrekturen are not negligible but essential for complete Verständnis:
	
	\begin{equation}
		\text{Observable effect} = \text{Main contribution} + \xi \cdot \text{Correction} + \text{Gravitational contribution}
	\end{equation}
	
	\subsection{Caution with Singularities}
	
	\begin{tcolorbox}[colback=t0yellow!10!white, colframe=t0yellow!75!black, title=Important Insight]
		Singularities are \textbf{not} the goal of T0 theory. They eher represent Grenzen of applicability:
		\begin{itemize}
			\item As $r \to 0$: The local Näherung breaks down
			\item As $E \to \infty$: The Feld Gleichungen become nichtlinear
			\item As $T \to 0$: Time-Energie duality loses its meaning
		\end{itemize}
		These Grenzen show wo the theory needs to be extended.
	\end{tcolorbox}
	
	\subsection{The Complementarity Principle in T0}
	
	Analogous to Bohr's complementarity Prinzip in Quanten Mechanik, T0 theory Zustände:
	
	\begin{equation}
		\text{Precision}(T) \times \text{Precision}(E) \leq \text{constant}
	\end{equation}
	
	The mehr precisely we determine one Aspekt (e.g., Zeit), the weniger präzise the complementary Aspekt (Energie) becomes. This is not a weakness of the theory but a fundamental Eigenschaft of reality.
	
	\subsection{Interpretation Guidelines}
	
	For korrekt Anwendung of T0 theory, the folgend guidelines apply:
	
	\begin{enumerate}
		\item \textbf{Scale awareness:} Always check welche Skala is dominant
		\item \textbf{Take klein Effekte seriously:} Don't ignore $\xi$ Korrekturen and gravitativ Effekte
		\item \textbf{Avoid singularities:} Understand them as hints at theoretisch Grenzen
		\item \textbf{Respect complementarity:} Not alle Aspekte can be sharp gleichzeitig
		\item \textbf{Experimentell verifiability:} Only make Vorhersagen das are measurable in Prinzip
	\end{enumerate}
	
	\section{Resolution of Quantum Paradoxes}
	
	T0 theory offers elegant Lösungen to the classic paradoxes of Quanten Mechanik by showing das they result from an incomplete Beschreibung of the underlying Feld Struktur.
	
	\subsection{Bell Correlations}
	
	The anscheinend instantaneous Bell correlations are resolved by T0 theory:
	
	\begin{itemize}
		\item \textbf{Local Bedingung:} $T \cdot E = 1$ at beide Messung locations
		\item \textbf{Shared Feld:} Entangled Teilchen share Feld configuration
		\item \textbf{Causal propagation:} Field changes propagate with $c$
		\item \textbf{Correlation without communication:} Pre-structured Feld, no signal transmission
	\end{itemize}
	
	The crucial Einsicht is das entangled Teilchen are not correlated through mysterious instantaneous connections, but through a shared Feld established wann they were created. This Feld exists throughout the spatial region and evolves causally gemäß the Feld Gleichungen. The beobachtet correlations result from dies pre-existing Feld Struktur, not instantaneous communication.
	
	\subsection{Wave Function Collapse}
	
	The supposedly instantaneous collapse is an illusion:
	\begin{align}
		\text{Measurement} &\rightarrow \text{Local field disturbance} \quad (t \sim t_{\text{Planck}}) \\
		&\rightarrow \text{Field propagation} \quad (v = c) \\
		&\rightarrow \text{Appears instantaneous since } t_{\text{Planck}} \ll t_{\text{meas}}
	\end{align}
	
	What appears as discontinuous collapse is actually a kontinuierlich Prozess occurring on a Zeit Skala far unten our Messung resolution. The Messung Prozess is a local Wechselwirkung zwischen measuring device and Feld das creates a disturbance propagating causally.
	
	\section{Experimentell Consequences}
	
	Although meist T0 Effekte occur on immeasurably klein Zeit Skalen, the theory noch makes testable Vorhersagen for extreme Bedingungen.
	
	\subsection{Prediction of Measurable Delays}
	
	For cosmic Bell tests with Entfernung $r$:
	\begin{equation}
		\Delta t_{\text{measurable}} = \xi \cdot \frac{r}{c}
	\end{equation}
	wo $\xi = \frac{4}{3} \times 10^{-4}$ is the geometrisch Parameter.
	
	\textbf{Numerical example:}
	\begin{itemize}
		\item Satellite Experiment with $r = 1000$ km:
		\begin{equation}
			\Delta t = 1.333 \times 10^{-4} \times \frac{10^6 \text{ m}}{3 \times 10^8 \text{ m/s}} \approx 0.44 \, \mu\text{s}
		\end{equation}
		\item This delay is measurable with modern atomic clocks ($\Delta t_{\text{resolution}} \sim 10^{-9}$ s)
	\end{itemize}
	
	\subsection{Proposed Experiments}
	
	\begin{enumerate}
		\item \textbf{Satellite Bell test:} Entangled Photonen zwischen ground station and satellite
		\item \textbf{Lunar laser ranging:} Precision Messung of Quanten correlations Earth-Moon
		\item \textbf{Deep Raum Quanten network:} Test at interplanetary distances
	\end{enumerate}
	
	\section{Philosophical Implications}
	
	The resolution of apparent instantaneity has profound Konsequenzen for our Verständnis of physikalisch reality.
	
	\subsection{New Interpretation of Quantum Mechanics}
	
	T0 theory offers an alternative Perspektive on Quanten Mechanik:
	
	\begin{tcolorbox}[colback=t0red!5!white, colframe=t0red!75!black, title=New Perspective]
		\textbf{Standard Interpretation:}
		\begin{itemize}
			\item Quantum Mechanik requires non-locality
			\item Spooky action at a Entfernung (Einstein)
			\item Wave Funktion collapse
		\end{itemize}
		
		\textbf{T0 Interpretation:}
		\begin{itemize}
			\item Everything is local in a shared Feld
			\item Correlations through Feld pre-Struktur
			\item Continuous, causal evolution
		\end{itemize}
	\end{tcolorbox}
	
	This paradigm shift solves viele conceptual problems das have plagued Quanten Mechanik since its inception. The need for unterschiedlich interpretations disappears wann one recognizes das the apparent paradoxes result from an incomplete Beschreibung.
	
	\subsection{Unification of Quantum Mechanics and Relativity}
	
	T0 theory resolves the apparent conflict:
	\begin{itemize}
		\item Preserves Lorentz Invarianz vollständig
		\item No faster-than-Licht information transmission
		\item Quantum correlations through causal Feld Struktur
	\end{itemize}
	
	This unification is not nur formal but conceptual. Both theories are understood as unterschiedlich Aspekte of the gleich underlying Feld Struktur. Quantum Mechanik describes the coherent Eigenschaften of Felder, while Relativität characterizes their causal Struktur.
	
	\section{The Measurement Process in Detail}
	
	The Messung Prozess in Quanten Mechanik has immer been one of the greatest conceptual problems. Wave Funktion collapse appears to be a non-unitary, instantaneous Prozess fundamentally unterschiedlich from normal Schrödinger evolution. The T0 formalism offers an alternative Beschreibung das avoids diese problems.
	
	In the T0 picture, a Messung is a local Wechselwirkung zwischen the measuring device and the Feld at the Messung location. This Wechselwirkung occurs on the Planck Zeit Skala—extremely fast but not instantaneous. The apparent collapse is actually a very rapid but kontinuierlich reorganization of the local Feld Struktur.
	
	Crucially, dies local reorganization does not require instantaneous change of the Feld at distant locations. Information ungefähr the Messung propagates as a Feld disturbance at the Geschwindigkeit of Licht. When dies disturbance reaches andere Teile of an entangled System, it influences their further evolution, but dies happens causally and at endlich Geschwindigkeit.
	
	This Beschreibung eliminates the conceptual problems of the Messung Prozess. There is no mysterious collapse, no violation of unitarity, and no instantaneous action at a Entfernung. Everything is described by local Feld Wechselwirkungen and causal Feld propagation.
	
	\section{Quantum Entanglement Without Instantaneity}
	
	Quantum entanglement is oft considered the paradigmatic example of non-local Quanten Phänomene. When two Teilchen are entangled, Messung of one Teilchen seems to instantly determine the Zustand of the andere, ungeachtet of Entfernung. Bell's inequalities and their experimentell violation seem to prove das local realistic theories cannot reproduce Quanten Mechanik.
	
	The T0 formalism offers a new Perspektive on diese Phänomene. Entanglement is not interpreted as a mysterious instantaneous Verbindung but as the result of a shared Feld configuration established wann the entangled Teilchen were created. This Feld configuration exists throughout the spatial region zwischen the Teilchen and evolves gemäß causal Feld Gleichungen.
	
	When a Messung is performed on one of the entangled Teilchen, the measuring apparatus interacts locally with the Feld at das location. This Wechselwirkung creates a disturbance in the Feld das propagates at the Geschwindigkeit of Licht. The correlations zwischen Messung results arise not from instantaneous communication but from the pre-existing Struktur of the shared Feld.
	
	This Interpretation resolves the EPR paradox in a way fully compatible with beide Quanten Mechanik and Relativität. There is no spooky action at a Entfernung, nur local Wechselwirkungen with an extended Feld. The beobachtet correlations result from coherent Feld Struktur, not instantaneous information transmission.
	
	\section{Zusammenfassung and Outlook}
	
	The Analyse of the T0 formalism klar shows das the apparent instantaneity of Quanten Mechanik is an illusion arising from several Faktoren.
	
	\subsection{Central Ergebnisse}
	
	T0 theory eliminates instantaneity through a hierarchical Struktur:
	
	\begin{enumerate}
		\item \textbf{Local Ebene:} $T \cdot E = 1$ as Einschränkung Bedingung (no Dynamik)
		\item \textbf{Field Ebene:} Wave Gleichung with propagation $v \leq c$ (causal Dynamik)
		\item \textbf{Measurable Ebene:} Appears instantaneous because $\Delta t < $ resolution
	\end{enumerate}
	
	This hierarchy is key to Verständnis warum Quanten Mechanik appears non-local while the underlying physics remains vollständig local and causal.
	
	\subsection{The Fundamental Insight}
	
	\begin{tcolorbox}[colback=t0yellow!10!white, colframe=t0yellow!75!black, title=Core Message]
		The apparent instantaneity of Quanten Mechanik is an illusion arising from:
		\begin{itemize}
			\item The notation of local Einschränkung Bedingungen
			\item The extreme smallness of Planck Zeit
			\item The pre-structuring of shared Felder
		\end{itemize}
		T0 theory shows das alle Phänomene are strictly causal and local wann the complete Feld Dynamik is considered.
	\end{tcolorbox}
	
	The implications of dies Einsicht extend far beyond technical details. It shows das nature, trotz its Quanten character, is fundamentally understandable and causally structured. The apparent mysteries of Quanten Mechanik dissolve wann one takes the right theoretisch Perspektive.
	
	\subsection{Outlook}
	
	T0 theory opens new research directions:
	\begin{itemize}
		\item Precision tests of vorhergesagt delays
		\item Quantum information theory with Feld correlations
		\item Cosmological implications of Zeit Feld Dynamik
		\item Technological Anwendungen in Quanten communication
	\end{itemize}
	
	Each of diese directions promises new insights into the fundamental nature of reality. T0 theory is not nur a mathematisch reformulation but a new conceptual foundation for our Verständnis of the Quanten world. The resolution of apparent instantaneity is an important step in the further development of our physikalisch worldview.
	
	The future of physics may lie in recognizing das the apparent mysteries of the Quanten world are not fundamental but result from an incomplete Beschreibung. T0 theory shows a path to a mehr complete Verständnis in welche locality, causality, and beobachtet Quanten Phänomene coexist harmoniously.
	

\begin{thebibliography}{99}

% ============================================
% Core T0 Theory References (J. Pascher)
% GitHub Repository: https://github.com/jpascher/T0-Time-Mass-Duality
% ============================================

\bibitem{pascher2024}
J. Pascher, \emph{T0 Theory: Time-Mass Duality}, 2024.
\url{https://github.com/jpascher/T0-Time-Mass-Duality/blob/main/2/pdf/T0_unified_report.pdf}

\bibitem{pascher2025t0}
J. Pascher, \emph{T0 Theory: Fundamentals}, 2025.
\url{https://github.com/jpascher/T0-Time-Mass-Duality/blob/main/2/pdf/T0_Grundlagen_En.pdf}

\bibitem{pascher2025qm}
J. Pascher, \emph{T0 Theory: Quantum Mechanics}, 2025.
\url{https://github.com/jpascher/T0-Time-Mass-Duality/blob/main/2/pdf/QM_En.pdf}

\bibitem{pascher2025si}
J. Pascher, \emph{T0 Theory: SI Units}, 2025.
\url{https://github.com/jpascher/T0-Time-Mass-Duality/blob/main/2/pdf/T0_SI_En.pdf}

\bibitem{pascher2025g2}
J. Pascher, \emph{T0 Theory: The g-2 Anomaly}, 2025.
\url{https://github.com/jpascher/T0-Time-Mass-Duality/blob/main/2/pdf/T0_Anomale-g2-9_En.pdf}

\bibitem{pascher2025cmb}
J. Pascher, \emph{T0 Theory: CMB Analysis}, 2025.
\url{https://github.com/jpascher/T0-Time-Mass-Duality/blob/main/2/pdf/Zwei-Dipole-CMB_En.pdf}

% Historical Physics
\bibitem{einstein1905}
A. Einstein, \emph{On the Electrodynamics of Moving Bodies}, Annalen der Physik, 1905.
\url{https://doi.org/10.1002/andp.19053221004}

\bibitem{dirac1928}
P.A.M. Dirac, \emph{The Quantum Theory of the Electron}, Proc. Roy. Soc. A, 1928.
\url{https://doi.org/10.1098/rspa.1928.0023}

\bibitem{planck1900}
M. Planck, \emph{On the Theory of the Energy Distribution Law}, 1900.
\url{https://doi.org/10.1002/andp.19013090310}

\bibitem{mach1883}
E. Mach, \emph{Die Mechanik in ihrer Entwicklung}, 1883.

\bibitem{hundert1931}
Various Authors, \emph{100 Authors Against Einstein}, 1931.

\bibitem{dingle1972}
H. Dingle, \emph{Science at the Crossroads}, 1972.

% Penrose and Terrell Effect
\bibitem{terrell1959}
J. Terrell, \emph{Invisibility of the Lorentz Contraction}, Phys. Rev., 1959.
\url{https://doi.org/10.1103/PhysRev.116.1041}

\bibitem{penrose1959}
R. Penrose, \emph{The Apparent Shape of a Relativistically Moving Sphere}, Proc. Cambridge Phil. Soc., 1959.
\url{https://doi.org/10.1017/S0305004100033776}

\bibitem{penrose1967}
R. Penrose, \emph{Twistor Algebra}, J. Math. Phys., 1967.
\url{https://doi.org/10.1063/1.1705200}

\bibitem{penrose2004}
R. Penrose, \emph{The Road to Reality}, 2004.

\bibitem{terrell2025}
J. Terrell et al., \emph{Modern Terrell-Penrose Visualization}, 2025.

\bibitem{weiskopf2000}
D. Weiskopf, \emph{Visualization of Four-dimensional Spacetimes}, 2000.

\bibitem{mueller2014}
T. Müller, \emph{Visual Appearance of Relativistically Moving Objects}, 2014.

\bibitem{hossenfelder2025}
S. Hossenfelder, \emph{YouTube: The Terrell Effect}, 2025.

% Quantum Gravity and String Theory
\bibitem{rovelli2004}
C. Rovelli, \emph{Quantum Gravity}, Cambridge University Press, 2004.

\bibitem{thiemann2007}
T. Thiemann, \emph{Modern Canonical Quantum Gravity}, Cambridge University Press, 2007.

\bibitem{ashtekar2004}
A. Ashtekar, J. Lewandowski, \emph{Background Independent Quantum Gravity}, Class. Quant. Grav., 2004.
\url{https://doi.org/10.1088/0264-9381/21/15/R01}

\bibitem{jacobson1995}
T. Jacobson, \emph{Thermodynamics of Spacetime}, Phys. Rev. Lett., 1995.
\url{https://doi.org/10.1103/PhysRevLett.75.1260}

\bibitem{maldacena1998}
J. Maldacena, \emph{The Large N Limit of Superconformal Field Theories}, Adv. Theor. Math. Phys., 1998.
\url{https://doi.org/10.4310/ATMP.1998.v2.n2.a1}

\bibitem{polchinski1998}
J. Polchinski, \emph{String Theory}, Cambridge University Press, 1998.

\bibitem{susskind1995}
L. Susskind, \emph{The World as a Hologram}, J. Math. Phys., 1995.
\url{https://doi.org/10.1063/1.531249}

\bibitem{verlinde2011}
E. Verlinde, \emph{On the Origin of Gravity}, JHEP, 2011.
\url{https://doi.org/10.1007/JHEP04(2011)029}

% Cosmology
\bibitem{hoyle1948}
F. Hoyle, \emph{A New Model for the Expanding Universe}, MNRAS, 1948.
\url{https://doi.org/10.1093/mnras/108.5.372}

\bibitem{bondi1948}
H. Bondi, T. Gold, \emph{The Steady-State Theory}, MNRAS, 1948.
\url{https://doi.org/10.1093/mnras/108.3.252}

\bibitem{zwicky1929}
F. Zwicky, \emph{On the Redshift of Spectral Lines}, Proc. Nat. Acad. Sci., 1929.
\url{https://doi.org/10.1073/pnas.15.10.773}

\bibitem{lopez2010}
C. Lopez-Corredoira, \emph{Tests of Cosmological Models}, Int. J. Mod. Phys. D, 2010.

\bibitem{lerner2014}
E. Lerner, \emph{Evidence for a Non-Expanding Universe}, 2014.

\bibitem{albrecht1999}
A. Albrecht, J. Magueijo, \emph{Variable Speed of Light}, Phys. Rev. D, 1999.
\url{https://doi.org/10.1103/PhysRevD.59.043516}

\bibitem{barrow1999}
J. Barrow, \emph{Cosmologies with Varying Light Speed}, Phys. Rev. D, 1999.
\url{https://doi.org/10.1103/PhysRevD.59.043515}

\bibitem{riess2022}
A. Riess et al., \emph{A Comprehensive Measurement of the Local Value of the Hubble Constant}, ApJ, 2022.
\url{https://doi.org/10.3847/2041-8213/ac5c5b}

\bibitem{desi2025}
DESI Collaboration, \emph{DESI Year 1 Results}, 2025.
\url{https://arxiv.org/abs/2404.03002}

\bibitem{divalentino2021}
E. Di Valentino et al., \emph{Planck Evidence for a Closed Universe}, Nat. Astron., 2021.
\url{https://doi.org/10.1038/s41550-019-0906-9}

% Conformal Field Theory
\bibitem{francesco1997}
P. Di Francesco et al., \emph{Conformal Field Theory}, Springer, 1997.

% Experimental Physics
\bibitem{pdg2024}
Particle Data Group, \emph{Review of Particle Physics}, 2024.
\url{https://pdg.lbl.gov/}

\bibitem{codata2019}
CODATA, \emph{Recommended Values of Fundamental Constants}, 2019.
\url{https://physics.nist.gov/cuu/Constants/}

\bibitem{newell2018}
D. Newell et al., \emph{The CODATA 2017 Values of h, e, k, and $N_A$}, Metrologia, 2018.
\url{https://doi.org/10.1088/1681-7575/aa950a}

\bibitem{muong2_2023}
Muon g-2 Collaboration, \emph{Measurement of the Anomalous Magnetic Moment of the Muon}, Phys. Rev. Lett., 2023.
\url{https://doi.org/10.1103/PhysRevLett.131.161802}

\bibitem{fermilab2023}
Fermilab, \emph{Muon g-2 Results}, 2023.
\url{https://muon-g-2.fnal.gov/}

\bibitem{atlas2023}
ATLAS Collaboration, \emph{Measurements at the LHC}, 2023.
\url{https://atlas.cern/}

\bibitem{atlas2023higgs}
ATLAS Collaboration, \emph{Higgs Boson Properties}, 2023.
\url{https://atlas.cern/}

\bibitem{cms2023top}
CMS Collaboration, \emph{Top Quark Measurements}, 2023.
\url{https://cms.cern/}

\bibitem{cms2024}
CMS Collaboration, \emph{Heavy Ion Collisions}, 2024.
\url{https://cms.cern/}

\bibitem{alice2023}
ALICE Collaboration, \emph{Quark-Gluon Plasma Studies}, 2023.
\url{https://alice-collaboration.web.cern.ch/}

\bibitem{kasevich2023}
M. Kasevich et al., \emph{Atom Interferometry}, 2023.

\bibitem{ludlow2015}
A. Ludlow et al., \emph{Optical Atomic Clocks}, Rev. Mod. Phys., 2015.
\url{https://doi.org/10.1103/RevModPhys.87.637}

\bibitem{brewer2019}
S. Brewer et al., \emph{Al$^+$ Optical Clock}, Phys. Rev. Lett., 2019.
\url{https://doi.org/10.1103/PhysRevLett.123.033201}

\bibitem{lisa2017}
LISA Collaboration, \emph{LISA Mission}, 2017.
\url{https://www.lisamission.org/}

% Fractal Physics
\bibitem{nottale1993}
L. Nottale, \emph{Fractal Space-Time and Microphysics}, World Scientific, 1993.

\bibitem{elnaschie2004}
M.S. El Naschie, \emph{E-Infinity Theory}, Chaos Solitons Fractals, 2004.

% Philosophy and Foundations
\bibitem{wheeler1990}
J.A. Wheeler, \emph{Information, Physics, Quantum}, 1990.

\bibitem{barbour1999}
J. Barbour, \emph{The End of Time}, Oxford University Press, 1999.

\bibitem{sciama1953}
D. Sciama, \emph{On the Origin of Inertia}, MNRAS, 1953.
\url{https://doi.org/10.1093/mnras/113.1.34}

% String Theory Extensions
\bibitem{becker2007}
K. Becker et al., \emph{String Theory and M-Theory}, Cambridge University Press, 2007.

% Missing References for g-2 Chapter
\bibitem{sm_g2_2025}
Muon g-2 Theory Initiative, \emph{Standard Model Prediction for g-2}, arXiv, 2025.
\url{https://arxiv.org/abs/2006.04822}

\bibitem{mug2_final_2025}
Muon g-2 Collaboration, \emph{Final Report on the Anomalous Magnetic Moment of the Muon}, Fermilab, 2025.
\url{https://muon-g-2.fnal.gov/}

\bibitem{pascher_t0_theory_2025}
J. Pascher, \emph{T0 Theory: Complete Framework}, 2025.
\url{https://github.com/jpascher/T0-Time-Mass-Duality/blob/main/2/pdf/systemEn.pdf}

\bibitem{peskin_schroeder_1995}
M.E. Peskin and D.V. Schroeder, \emph{An Introduction to Quantum Field Theory}, Westview Press, 1995.

\bibitem{parker_somov_2018}
R.H. Parker et al., \emph{Measurement of the Fine-Structure Constant}, Science, 2018.
\url{https://doi.org/10.1126/science.aap7706}

\bibitem{morel_rubidium_2020}
L. Morel et al., \emph{Determination of $\alpha$ from Rubidium Atom Recoil}, Nature, 2020.
\url{https://doi.org/10.1038/s41586-020-2964-7}

\bibitem{aoyama_theory_2020}
T. Aoyama et al., \emph{Theory of the Electron Anomalous Magnetic Moment}, Phys. Rep., 2020.
\url{https://doi.org/10.1016/j.physrep.2020.07.006}

\bibitem{fan_lattice_2023}
X. Fan et al., \emph{Hadronic Contributions from Lattice QCD}, Phys. Rev. D, 2023.

\bibitem{hanneke_electron_2008}
D. Hanneke et al., \emph{New Measurement of the Electron g-2}, Phys. Rev. Lett., 2008.
\url{https://doi.org/10.1103/PhysRevLett.100.120801}

% Additional T0 Theory References
\bibitem{pascher_higgs_connection_2025}
J. Pascher, \emph{Higgs Connection in T0 Theory}, 2025.
\url{https://github.com/jpascher/T0-Time-Mass-Duality/blob/main/2/pdf/T0_Energie_En.pdf}

\bibitem{T0_SI}
J. Pascher, \emph{T0 Theory and SI Units}, 2025.
\url{https://github.com/jpascher/T0-Time-Mass-Duality/blob/main/2/pdf/T0_SI_En.pdf}

\bibitem{T0_gravitational_constant}
J. Pascher, \emph{Gravitational Constant in T0 Framework}, 2025.
\url{https://github.com/jpascher/T0-Time-Mass-Duality/blob/main/2/pdf/T0_Gravitationskonstante_En.pdf}

\bibitem{T0_fine_structure}
J. Pascher, \emph{Fine Structure Constant Analysis}, 2025.
\url{https://github.com/jpascher/T0-Time-Mass-Duality/blob/main/2/pdf/T0_Feinstruktur_En.pdf}

\bibitem{bell_muon}
J.S. Bell, \emph{Muon Studies}, 1966.

\bibitem{QFT_T0}
J. Pascher, \emph{Quantum Field Theory in T0}, 2025.
\url{https://github.com/jpascher/T0-Time-Mass-Duality/blob/main/2/pdf/QFT_En.pdf}

\bibitem{planck2018}
Planck Collaboration, \emph{Planck 2018 Results}, A\&A, 2018.
\url{https://doi.org/10.1051/0004-6361/201833910}

\bibitem{pascher:t0_foundations}
J. Pascher, \emph{T0 Theory Foundations}, 2025.
\url{https://github.com/jpascher/T0-Time-Mass-Duality/blob/main/2/pdf/T0_Grundlagen_En.pdf}

\bibitem{pascher:geometric_formalism}
J. Pascher, \emph{Geometric Formalism in T0}, 2025.
\url{https://github.com/jpascher/T0-Time-Mass-Duality/blob/main/2/pdf/T0_Geometrische_Kosmologie_En.pdf}

\bibitem{riess2019}
A. Riess et al., \emph{Hubble Constant Measurements}, ApJ, 2019.
\url{https://doi.org/10.3847/1538-4357/ab1422}

\bibitem{t0_kosmologie}
J. Pascher, \emph{T0 Kosmologie}, 2025.
\url{https://github.com/jpascher/T0-Time-Mass-Duality/blob/main/2/pdf/T0_Kosmologie_En.pdf}

\bibitem{hossenfelder_single_clock_video}
S. Hossenfelder, \emph{Single Clock Video}, YouTube, 2025.
\url{https://www.youtube.com/c/SabineHossenfelder}

\bibitem{video2025}
Various, \emph{Video References}, 2025.

\bibitem{unnikrishnan2004}
C.S. Unnikrishnan, \emph{Gravity Studies}, 2004.

\bibitem{peratt1992}
A. Peratt, \emph{Plasma Cosmology}, 1992.
\url{https://github.com/jpascher/T0-Time-Mass-Duality/blob/main/2/pdf/T0_peratt_En.pdf}

\bibitem{T0_tm_erweiterung}
J. Pascher, \emph{T0 Time-Mass Extension}, 2025.
\url{https://github.com/jpascher/T0-Time-Mass-Duality/blob/main/2/pdf/T0_tm-erweiterung-x6_En.pdf}

\bibitem{T0_g2_erweiterung}
J. Pascher, \emph{T0 g-2 Extension}, 2025.
\url{https://github.com/jpascher/T0-Time-Mass-Duality/blob/main/2/pdf/T0_g2-erweiterung-4_En.pdf}

\bibitem{T0_netze_en}
J. Pascher, \emph{T0 Networks}, 2025.
\url{https://github.com/jpascher/T0-Time-Mass-Duality/blob/main/2/pdf/T0_netze_En.pdf}

\bibitem{Adams1925}
W. Adams, \emph{Gravitational Redshift}, 1925.
\url{https://doi.org/10.1073/pnas.11.7.382}

\bibitem{Ashby2003}
N. Ashby, \emph{Relativity in GPS}, Living Rev. Rel., 2003.
\url{https://doi.org/10.12942/lrr-2003-1}

\bibitem{Bertotti2003}
B. Bertotti et al., \emph{Cassini Doppler Test}, Nature, 2003.
\url{https://doi.org/10.1038/nature01997}

\bibitem{Bolton2008}
A. Bolton et al., \emph{Gravitational Lensing}, 2008.

\bibitem{Born2013}
M. Born, \emph{Einstein's Theory of Relativity}, Dover, 2013.

\bibitem{Brans1961}
C. Brans and R.H. Dicke, \emph{Mach's Principle}, Phys. Rev., 1961.
\url{https://doi.org/10.1103/PhysRev.124.925}

\bibitem{Dirac1927}
P.A.M. Dirac, \emph{Quantum Mechanics}, Proc. Roy. Soc., 1927.
\url{https://doi.org/10.1098/rspa.1927.0039}

\bibitem{Duhem1906}
P. Duhem, \emph{Theory of Physics}, 1906.

\bibitem{Einstein1905}
A. Einstein, \emph{Special Relativity}, Ann. Phys., 1905.
\url{https://doi.org/10.1002/andp.19053221004}

\bibitem{Feynman2006}
R. Feynman, \emph{QED: The Strange Theory of Light and Matter}, 2006.

\bibitem{Griffiths2017}
D. Griffiths, \emph{Introduction to Quantum Mechanics}, 2017.

\bibitem{Jackson1999}
J.D. Jackson, \emph{Classical Electrodynamics}, 1999.

\bibitem{Kaluza1921}
T. Kaluza, \emph{Five-Dimensional Theory}, 1921.

\bibitem{Klein1926}
O. Klein, \emph{Quantum Theory and Relativity}, 1926.

\bibitem{Kuhn1962}
T. Kuhn, \emph{Structure of Scientific Revolutions}, 1962.

\bibitem{Kuhn1977}
T. Kuhn, \emph{Essential Tension}, 1977.

\bibitem{Ludlow2015}
A. Ludlow et al., \emph{Optical Atomic Clocks}, Rev. Mod. Phys., 2015.
\url{https://doi.org/10.1103/RevModPhys.87.637}

\bibitem{Maxwell1873}
J.C. Maxwell, \emph{Treatise on Electricity and Magnetism}, 1873.

\bibitem{McGaugh2016}
S. McGaugh et al., \emph{Radial Acceleration Relation}, Phys. Rev. Lett., 2016.
\url{https://doi.org/10.1103/PhysRevLett.117.201101}

\bibitem{Mohr2016}
P. Mohr et al., \emph{CODATA Values}, Rev. Mod. Phys., 2016.
\url{https://doi.org/10.1103/RevModPhys.88.035009}

\bibitem{PDG2020}
Particle Data Group, \emph{Review of Particle Physics}, Prog. Theor. Exp. Phys., 2020.
\url{https://pdg.lbl.gov/}

\bibitem{Parker2018}
R. Parker et al., \emph{Measurement of $\alpha$}, Science, 2018.
\url{https://doi.org/10.1126/science.aap7706}

\bibitem{Peskin1995}
M. Peskin and D. Schroeder, \emph{QFT}, 1995.

\bibitem{Planck1900}
M. Planck, \emph{Quantum Theory}, 1900.

\bibitem{Planck2020}
Planck Collaboration, \emph{Planck 2020 Results}, 2020.
\url{https://doi.org/10.1051/0004-6361/201833910}

\bibitem{Poincare1905}
H. Poincaré, \emph{Dynamics of the Electron}, 1905.

\bibitem{Pound1960}
R.V. Pound and G.A. Rebka, \emph{Gravitational Redshift}, Phys. Rev. Lett., 1960.
\url{https://doi.org/10.1103/PhysRevLett.4.337}

\bibitem{Quine1951}
W.V. Quine, \emph{Two Dogmas of Empiricism}, 1951.

\bibitem{Quinn2013}
T. Quinn et al., \emph{Gravitational Constant}, 2013.
\url{https://doi.org/10.1103/PhysRevLett.111.101102}

\bibitem{Randall1999}
L. Randall and R. Sundrum, \emph{Extra Dimensions}, Phys. Rev. Lett., 1999.
\url{https://doi.org/10.1103/PhysRevLett.83.3370}

\bibitem{Riess1998}
A. Riess et al., \emph{Type Ia Supernovae}, AJ, 1998.
\url{https://doi.org/10.1086/300499}

\bibitem{Shapiro1971}
I. Shapiro et al., \emph{Time Delay Test}, Phys. Rev. Lett., 1971.
\url{https://doi.org/10.1103/PhysRevLett.26.1132}

\bibitem{Sommerfeld1916}
A. Sommerfeld, \emph{Fine Structure}, 1916.

\bibitem{Suyu2017}
S. Suyu et al., \emph{Time Delay Cosmography}, MNRAS, 2017.
\url{https://doi.org/10.1093/mnras/stx483}

\bibitem{T0Theory}
J. Pascher, \emph{T0 Theory}, 2025.
\url{https://github.com/jpascher/T0-Time-Mass-Duality/blob/main/2/pdf/systemEn.pdf}

\bibitem{T0_Feinstruktur}
J. Pascher, \emph{Fine Structure in T0}, 2025.
\url{https://github.com/jpascher/T0-Time-Mass-Duality/blob/main/2/pdf/T0_Feinstruktur_En.pdf}

\bibitem{Uzan2003}
J.-P. Uzan, \emph{Constants Variation}, Rev. Mod. Phys., 2003.
\url{https://doi.org/10.1103/RevModPhys.75.403}

\bibitem{Webb2001}
J.K. Webb et al., \emph{Fine Structure Constant}, Phys. Rev. Lett., 2001.
\url{https://doi.org/10.1103/PhysRevLett.87.091301}

\bibitem{Weinberg1979}
S. Weinberg, \emph{Cosmological Constant}, Rev. Mod. Phys., 1979.

\bibitem{Weinberg1989}
S. Weinberg, \emph{Cosmological Constant Problem}, 1989.
\url{https://doi.org/10.1103/RevModPhys.61.1}

\bibitem{Weinberg1995}
S. Weinberg, \emph{Quantum Theory of Fields}, 1995.

\bibitem{Will2014}
C. Will, \emph{Theory and Experiment in Gravitational Physics}, 2014.
\url{https://doi.org/10.12942/lrr-2014-4}

\bibitem{dirac_principles}
P.A.M. Dirac, \emph{Principles of Quantum Mechanics}, 1930.

\bibitem{einstein_1917}
A. Einstein, \emph{Cosmological Considerations}, 1917.

\bibitem{jwst_early}
JWST Collaboration, \emph{Early Universe Observations}, 2023.
\url{https://www.jwst.nasa.gov/}

\bibitem{katrin_2022}
KATRIN Collaboration, \emph{Neutrino Mass}, 2022.
\url{https://doi.org/10.1038/s41567-021-01463-1}

\bibitem{pascher:fundamentals}
J. Pascher, \emph{T0 Fundamentals}, 2025.
\url{https://github.com/jpascher/T0-Time-Mass-Duality/blob/main/2/pdf/T0_Grundlagen_En.pdf}

\bibitem{pascher:g2_rev9}
J. Pascher, \emph{g-2 Analysis Rev9}, 2025.
\url{https://github.com/jpascher/T0-Time-Mass-Duality/blob/main/2/pdf/T0_Anomale-g2-9_En.pdf}

\bibitem{pascher:ml_addendum}
J. Pascher, \emph{ML Addendum}, 2025.
\url{https://github.com/jpascher/T0-Time-Mass-Duality/blob/main/2/pdf/T0-QFT-ML_Addendum_En.pdf}

\bibitem{pascher_beta_derivation_2025}
J. Pascher, \emph{Beta Derivation}, 2025.
\url{https://github.com/jpascher/T0-Time-Mass-Duality/blob/main/2/pdf/DerivationVonBetaEn.pdf}

\bibitem{pascher_cmb_en}
J. Pascher, \emph{CMB Analysis in T0}, 2025.
\url{https://github.com/jpascher/T0-Time-Mass-Duality/blob/main/2/pdf/Zwei-Dipole-CMB_En.pdf}

\bibitem{pascher_cosmos_en}
J. Pascher, \emph{Cosmos in T0 Theory}, 2025.
\url{https://github.com/jpascher/T0-Time-Mass-Duality/blob/main/2/pdf/cosmic_En.pdf}

\bibitem{pascher_derivation_beta_2025}
J. Pascher, \emph{Derivation of Beta}, 2025.
\url{https://github.com/jpascher/T0-Time-Mass-Duality/blob/main/2/pdf/DerivationVonBetaEn.pdf}

\bibitem{pascher_gravitation_en}
J. Pascher, \emph{Gravitation in T0}, 2025.
\url{https://github.com/jpascher/T0-Time-Mass-Duality/blob/main/2/pdf/gravitationskonstante_En.pdf}

\bibitem{pascher_lagrangian_2025}
J. Pascher, \emph{Lagrangian in T0}, 2025.
\url{https://github.com/jpascher/T0-Time-Mass-Duality/blob/main/2/pdf/T0_lagrndian_En.pdf}

\bibitem{pascher_lagrangian_en}
J. Pascher, \emph{Lagrangian Framework}, 2025.
\url{https://github.com/jpascher/T0-Time-Mass-Duality/blob/main/2/pdf/LagrandianVergleichEn.pdf}

\bibitem{pascher_lagrangian_extended_2025}
J. Pascher, \emph{Extended Lagrangian Formalism}, 2025.
\url{https://github.com/jpascher/T0-Time-Mass-Duality/blob/main/2/pdf/T0_lagrndian_En.pdf}

\bibitem{pascher_mathematical_structure_2025}
J. Pascher, \emph{Mathematical Structure of T0 Theory}, 2025.
\url{https://github.com/jpascher/T0-Time-Mass-Duality/blob/main/2/pdf/Mathematische_struktur_En.pdf}

\bibitem{pascher_muon_g2_2025}
J. Pascher, \emph{Muon g-2 in T0}, 2025.
\url{https://github.com/jpascher/T0-Time-Mass-Duality/blob/main/2/pdf/T0_Anomale-g2-9_En.pdf}

\bibitem{pascher_pragmatic_2025}
J. Pascher, \emph{Pragmatic Approach}, 2025.

\bibitem{pascher_t0_energy_2025}
J. Pascher, \emph{T0 Energy Formalism}, 2025.
\url{https://github.com/jpascher/T0-Time-Mass-Duality/blob/main/2/pdf/T0-Energie_En.pdf}

\bibitem{pascher_unified_2025}
J. Pascher, \emph{Unified T0 Theory}, 2025.
\url{https://github.com/jpascher/T0-Time-Mass-Duality/blob/main/2/pdf/T0_unified_report.pdf}

\bibitem{sciencedaily2025}
Science Daily, \emph{Physics News}, 2025.
\url{https://www.sciencedaily.com/}

\bibitem{weinberg_1989}
S. Weinberg, \emph{The Cosmological Constant Problem}, Rev. Mod. Phys., 1989.
\url{https://doi.org/10.1103/RevModPhys.61.1}

\bibitem{wiki_bell}
Wikipedia, \emph{Bell's Theorem}, 2025.
\url{https://en.wikipedia.org/wiki/Bell\%27s_theorem}

\bibitem{vanFraassen1980}
B. van Fraassen, \emph{The Scientific Image}, Oxford University Press, 1980.

\bibitem{terrell_single_clock_nature_2024}
J. Terrell, \emph{Single Clock Nature}, Nature, 2024.

% Additional T0 Documents
\bibitem{137_doc}
J. Pascher, \emph{The Number 137 in T0 Theory}, 2025.
\url{https://github.com/jpascher/T0-Time-Mass-Duality/blob/main/2/pdf/137_En.pdf}

\bibitem{ampere_low}
J. Pascher, \emph{Ampere's Law in T0}, 2025.
\url{https://github.com/jpascher/T0-Time-Mass-Duality/blob/main/2/pdf/Amper_Low_En.pdf}

\bibitem{bell_theorem}
J. Pascher, \emph{Bell's Theorem in T0}, 2025.
\url{https://github.com/jpascher/T0-Time-Mass-Duality/blob/main/2/pdf/Bell_En.pdf}

\bibitem{bewegungsenergie}
J. Pascher, \emph{Kinetic Energy in T0}, 2025.
\url{https://github.com/jpascher/T0-Time-Mass-Duality/blob/main/2/pdf/Bewegungsenergie_En.pdf}

\bibitem{emc2}
J. Pascher, \emph{E=mc² in T0 Framework}, 2025.
\url{https://github.com/jpascher/T0-Time-Mass-Duality/blob/main/2/pdf/E-mc2_En.pdf}

\bibitem{formeln_energiebasiert}
J. Pascher, \emph{Energy-Based Formulas}, 2025.
\url{https://github.com/jpascher/T0-Time-Mass-Duality/blob/main/2/pdf/Formeln_Energiebasiert_En.pdf}

\bibitem{hannah}
J. Pascher, \emph{Hannah Document}, 2025.
\url{https://github.com/jpascher/T0-Time-Mass-Duality/blob/main/2/pdf/Hannah_En.pdf}

\bibitem{ho_doc}
J. Pascher, \emph{H0 Analysis}, 2025.
\url{https://github.com/jpascher/T0-Time-Mass-Duality/blob/main/2/pdf/Ho_En.pdf}

\bibitem{markov}
J. Pascher, \emph{Markov Processes in T0}, 2025.
\url{https://github.com/jpascher/T0-Time-Mass-Duality/blob/main/2/pdf/Markov_En.pdf}

\bibitem{elimination_mass}
J. Pascher, \emph{Elimination of Mass}, 2025.
\url{https://github.com/jpascher/T0-Time-Mass-Duality/blob/main/2/pdf/EliminationOfMassEn.pdf}

\bibitem{elimination_mass_dirac}
J. Pascher, \emph{Dirac Equation Mass Elimination}, 2025.
\url{https://github.com/jpascher/T0-Time-Mass-Duality/blob/main/2/pdf/Elimination_Of_Mass_Dirac_TabelleEn.pdf}

\bibitem{feinstrukturkonstante}
J. Pascher, \emph{Fine Structure Constant}, 2025.
\url{https://github.com/jpascher/T0-Time-Mass-Duality/blob/main/2/pdf/FeinstrukturkonstanteEn.pdf}

\bibitem{neutrino_formel}
J. Pascher, \emph{Neutrino Formula}, 2025.
\url{https://github.com/jpascher/T0-Time-Mass-Duality/blob/main/2/pdf/neutrino-Formel_En.pdf}

\bibitem{neutrinos}
J. Pascher, \emph{Neutrinos in T0}, 2025.
\url{https://github.com/jpascher/T0-Time-Mass-Duality/blob/main/2/pdf/T0_Neutrinos_En.pdf}

\bibitem{koide_formel}
J. Pascher, \emph{Koide Formula in T0}, 2025.
\url{https://github.com/jpascher/T0-Time-Mass-Duality/blob/main/2/pdf/T0_koide-formel-3_En.pdf}

\bibitem{teilchenmassen}
J. Pascher, \emph{Particle Masses}, 2025.
\url{https://github.com/jpascher/T0-Time-Mass-Duality/blob/main/2/pdf/Teilchenmassen_En.pdf}

\bibitem{t0_teilchenmassen}
J. Pascher, \emph{T0 Particle Masses}, 2025.
\url{https://github.com/jpascher/T0-Time-Mass-Duality/blob/main/2/pdf/T0_Teilchenmassen_En.pdf}

\bibitem{penrose_doc}
J. Pascher, \emph{Penrose Analysis in T0}, 2025.
\url{https://github.com/jpascher/T0-Time-Mass-Duality/blob/main/2/pdf/T0_penrose_En.pdf}

\bibitem{photonenchip}
J. Pascher, \emph{Photon Chip Implementation}, 2025.
\url{https://github.com/jpascher/T0-Time-Mass-Duality/blob/main/2/pdf/T0_photonenchip-china_En.pdf}

\bibitem{threeclock}
J. Pascher, \emph{Three Clock Experiment}, 2025.
\url{https://github.com/jpascher/T0-Time-Mass-Duality/blob/main/2/pdf/T0_threeclock_En.pdf}

\bibitem{redshift_deflection}
J. Pascher, \emph{Redshift and Deflection}, 2025.
\url{https://github.com/jpascher/T0-Time-Mass-Duality/blob/main/2/pdf/redshift_deflection_En.pdf}

\bibitem{scheinbar_instantan}
J. Pascher, \emph{Apparent Instantaneity}, 2025.
\url{https://github.com/jpascher/T0-Time-Mass-Duality/blob/main/2/pdf/scheinbar_instantan_En.pdf}

\bibitem{universale_ableitung}
J. Pascher, \emph{Universal Derivation}, 2025.
\url{https://github.com/jpascher/T0-Time-Mass-Duality/blob/main/2/pdf/universale-ableitung_En.pdf}

\bibitem{xi_parameter}
J. Pascher, \emph{Xi Parameter for Particles}, 2025.
\url{https://github.com/jpascher/T0-Time-Mass-Duality/blob/main/2/pdf/xi_parmater_partikel_En.pdf}

\bibitem{xi_ursprung}
J. Pascher, \emph{Origin of Xi}, 2025.
\url{https://github.com/jpascher/T0-Time-Mass-Duality/blob/main/2/pdf/T0_xi_ursprung_En.pdf}

\bibitem{zeit}
J. Pascher, \emph{Time in T0 Theory}, 2025.
\url{https://github.com/jpascher/T0-Time-Mass-Duality/blob/main/2/pdf/Zeit_En.pdf}

\bibitem{zeit_konstant}
J. Pascher, \emph{Time Constant}, 2025.
\url{https://github.com/jpascher/T0-Time-Mass-Duality/blob/main/2/pdf/Zeit-konstant_En.pdf}

\bibitem{zusammenfassung}
J. Pascher, \emph{Summary of T0 Theory}, 2025.
\url{https://github.com/jpascher/T0-Time-Mass-Duality/blob/main/2/pdf/Zusammenfassung_En.pdf}

\bibitem{rsa}
J. Pascher, \emph{RSA in T0 Framework}, 2025.
\url{https://github.com/jpascher/T0-Time-Mass-Duality/blob/main/2/pdf/RSA_En.pdf}

\bibitem{qat}
J. Pascher, \emph{Quantum Atomic Theory}, 2025.
\url{https://github.com/jpascher/T0-Time-Mass-Duality/blob/main/2/pdf/T0_QAT_En.pdf}

\bibitem{qm_qft_rt}
J. Pascher, \emph{QM, QFT and RT Unification}, 2025.
\url{https://github.com/jpascher/T0-Time-Mass-Duality/blob/main/2/pdf/T0_QM-QFT-RT_En.pdf}

\bibitem{qm_optimierung}
J. Pascher, \emph{QM Optimization}, 2025.
\url{https://github.com/jpascher/T0-Time-Mass-Duality/blob/main/2/pdf/T0_QM-optimierung_En.pdf}

\bibitem{vollstaendige_berechnungen}
J. Pascher, \emph{Complete Calculations}, 2025.
\url{https://github.com/jpascher/T0-Time-Mass-Duality/blob/main/2/pdf/T0_Vollstaendige_Berchnungen_En.pdf}

\bibitem{synergetics}
J. Pascher, \emph{T0 Theory vs Synergetics}, 2025.
\url{https://github.com/jpascher/T0-Time-Mass-Duality/blob/main/2/pdf/T0-Theory-vs-Synergetics_En.pdf}

\bibitem{modell_uebersicht}
J. Pascher, \emph{T0 Model Overview}, 2025.
\url{https://github.com/jpascher/T0-Time-Mass-Duality/blob/main/2/pdf/T0_Modell_Uebersicht_En.pdf}

\bibitem{mnras_widerlegung}
J. Pascher, \emph{MNRAS Analysis}, 2025.
\url{https://github.com/jpascher/T0-Time-Mass-Duality/blob/main/2/pdf/T0_Analyse_MNRAS_Widerlegung_En.pdf}

\bibitem{anomale_magnetische_momente}
J. Pascher, \emph{Anomalous Magnetic Moments}, 2025.
\url{https://github.com/jpascher/T0-Time-Mass-Duality/blob/main/2/pdf/T0_Anomale_Magnetische_Momente_En.pdf}

\bibitem{sieben_fragen}
J. Pascher, \emph{Seven Questions in T0}, 2025.
\url{https://github.com/jpascher/T0-Time-Mass-Duality/blob/main/2/pdf/T0_7-fragen-3_En.pdf}

\bibitem{detailierte_leptonen}
J. Pascher, \emph{Detailed Lepton Anomaly}, 2025.
\url{https://github.com/jpascher/T0-Time-Mass-Duality/blob/main/2/pdf/detailierte_formel_leptonen_anemal_En.pdf}

\bibitem{parameterherleitung}
J. Pascher, \emph{Parameter Derivation}, 2025.
\url{https://github.com/jpascher/T0-Time-Mass-Duality/blob/main/2/pdf/parameterherleitung_En.pdf}

\bibitem{verhaeltnis_absolut}
J. Pascher, \emph{Absolute Ratios in T0}, 2025.
\url{https://github.com/jpascher/T0-Time-Mass-Duality/blob/main/2/pdf/T0_verhaeltnis-absolut_En.pdf}

\bibitem{xi_und_e}
J. Pascher, \emph{Xi and Energy}, 2025.
\url{https://github.com/jpascher/T0-Time-Mass-Duality/blob/main/2/pdf/T0_xi-und-e_En.pdf}

\bibitem{umkehrung}
J. Pascher, \emph{Inversion in T0}, 2025.
\url{https://github.com/jpascher/T0-Time-Mass-Duality/blob/main/2/pdf/T0_umkehrung_En.pdf}

\bibitem{esm_analysis}
J. Pascher, \emph{T0 vs ESM Conceptual Analysis}, 2025.
\url{https://github.com/jpascher/T0-Time-Mass-Duality/blob/main/2/pdf/T0vsESM_ConceptualAnalysis_En.pdf}

\end{thebibliography}

\end{document}
