% Standalone document: T0_Feinstruktur_En
% Uses shared T0 header
% T0 Standalone Header - German Version
% Gemeinsamer Header für alle deutschen Standalone-Dokumente

\documentclass[12pt,a4paper]{article}
\usepackage[utf8]{inputenc}
\usepackage[T1]{fontenc}
\usepackage[ngerman]{babel}
\usepackage{lmodern}

% Mathematics
\usepackage{amsmath,amssymb,amsthm}
\usepackage{physics}
\usepackage{siunitx}

% Layout
\usepackage[left=2.5cm,right=2.5cm,top=2.5cm,bottom=2.5cm,headheight=15pt]{geometry}
\usepackage{fancyhdr}
\usepackage{titlesec}

% Tables and Graphics
\usepackage{booktabs}
\usepackage{array}
\usepackage{longtable}
\usepackage{graphicx}
\usepackage{tikz}
\usetikzlibrary{arrows.meta,positioning,shapes.geometric}

% Colors and Boxes
\usepackage{xcolor}
\usepackage[most]{tcolorbox}
\usepackage{mdframed}

% Additional packages
\usepackage{enumitem}
\usepackage{float}
\usepackage{caption}
\usepackage{subcaption}
\usepackage{multirow}
\usepackage{colortbl}
\usepackage{pdflscape}
\usepackage{algorithm}
\usepackage{algpseudocode}
\usepackage{listings}
\usepackage{hyperref}

% Define colors
\definecolor{t0blue}{RGB}{0,51,102}
\definecolor{t0green}{RGB}{0,102,51}
\definecolor{t0red}{RGB}{153,0,0}
\definecolor{deepblue}{RGB}{0,51,102}
\definecolor{deepgreen}{RGB}{0,102,51}
\definecolor{deepred}{RGB}{153,0,0}
\definecolor{boxgray}{RGB}{240,240,240}
\definecolor{t0yellow}{RGB}{255,200,0}
\definecolor{boxblue}{RGB}{230,240,255}
\definecolor{boxgreen}{RGB}{230,255,230}
\definecolor{boxorange}{RGB}{255,240,230}
\definecolor{boxyellow}{RGB}{255,255,230}

% Custom tcolorbox environments
\newtcolorbox{fundamental}[1][]{
  colback=blue!5!white,
  colframe=blue!75!black,
  title=#1,
  fonttitle=\bfseries,
  breakable
}

\newtcolorbox{derivation}[1][]{
  colback=green!5!white,
  colframe=green!75!black,
  title=#1,
  fonttitle=\bfseries,
  breakable
}

\newtcolorbox{result}[1][]{
  colback=orange!5!white,
  colframe=orange!75!black,
  title=#1,
  fonttitle=\bfseries,
  breakable
}

\newtcolorbox{summary}[1][]{
  colback=gray!10!white,
  colframe=gray!75!black,
  title=#1,
  fonttitle=\bfseries,
  breakable
}

\newtcolorbox{comparison}[1][]{
  colback=purple!5!white,
  colframe=purple!75!black,
  title=#1,
  fonttitle=\bfseries,
  breakable
}

\newtcolorbox{relation}[1][]{
  colback=cyan!5!white,
  colframe=cyan!75!black,
  title=#1,
  fonttitle=\bfseries,
  breakable
}

\newtcolorbox{principle}[1][]{
  colback=yellow!5!white,
  colframe=yellow!75!black,
  title=#1,
  fonttitle=\bfseries,
  breakable
}

\newtcolorbox{insight}[1][]{colback=blue!5,colframe=t0blue,title={#1},fonttitle=\bfseries,breakable}
\newtcolorbox{discovery}[1][]{colback=green!5,colframe=t0green,title={#1},fonttitle=\bfseries,breakable}
\newtcolorbox{newperspective}[1][]{colback=yellow!5,colframe=orange,title={#1},fonttitle=\bfseries,breakable}
\newtcolorbox{revelation}[1][]{colback=red!5,colframe=t0red,title={#1},fonttitle=\bfseries,breakable}
\newtcolorbox{keypoint}[1][]{colback=blue!5,colframe=t0blue,title={#1},fonttitle=\bfseries,breakable}
\newtcolorbox{evidence}[1][]{colback=green!5,colframe=t0green,title={#1},fonttitle=\bfseries,breakable}
\newtcolorbox{conclusion}[1][]{colback=gray!5,colframe=gray,title={#1},fonttitle=\bfseries,breakable}
\newtcolorbox{significance}[1][]{colback=yellow!5,colframe=orange,title={#1},fonttitle=\bfseries,breakable}
\newtcolorbox{philosophical}[1][]{colback=purple!5,colframe=purple,title={#1},fonttitle=\bfseries,breakable}
\newtcolorbox{implication}[1][]{colback=cyan!5,colframe=cyan,title={#1},fonttitle=\bfseries,breakable}
\newtcolorbox{perspective}[1][]{colback=blue!5,colframe=t0blue,title={#1},fonttitle=\bfseries,breakable}
\newtcolorbox{revolutionary}[1][]{colback=red!5,colframe=t0red,title={#1},fonttitle=\bfseries,breakable}
\newtcolorbox{technical}[1][]{colback=gray!5,colframe=gray!75!black,title={#1},fonttitle=\bfseries,breakable}
\newtcolorbox{notation}[1][]{colback=yellow!5,colframe=yellow!75!black,title={#1},fonttitle=\bfseries,breakable}

% Theorem environments
\newtheorem{theorem}{Satz}[section]
\newtheorem{lemma}[theorem]{Lemma}
\newtheorem{corollary}[theorem]{Korollar}
\newtheorem{proposition}[theorem]{Proposition}
\newtheorem{definition}[theorem]{Definition}
\newtheorem{example}[theorem]{Beispiel}
\newtheorem{remark}[theorem]{Bemerkung}
\newtheorem{note}[theorem]{Anmerkung}

% Additional environments
\newenvironment{treatise}{\begin{quote}}{\end{quote}}
\newenvironment{gemeinsam}{\begin{quote}}{\end{quote}}
\newenvironment{vergleich}{\begin{quote}}{\end{quote}}
\newenvironment{vorteil}{\begin{quote}}{\end{quote}}
\newenvironment{quantum}{\begin{quote}}{\end{quote}}

% T0-specific commands
\newcommand{\Tzero}{T$_0$}
\newcommand{\xipar}{\xi}
\newcommand{\Tfield}{T}
\newcommand{\Efield}{\mathcal{E}}
\newcommand{\meff}{m_{\text{eff}}}
\newcommand{\Eabs}{E_{\text{abs}}}
\newcommand{\taupar}{\tau}

% Header setup
\pagestyle{fancy}
\fancyhf{}
\fancyhead[L]{\leftmark}
\fancyhead[R]{\thepage}
\renewcommand{\headrulewidth}{0.4pt}

% Hyperref setup
\hypersetup{
    colorlinks=true,
    linkcolor=blue,
    filecolor=magenta,
    urlcolor=cyan,
    citecolor=blue,
    pdftitle={T0 Theory Document},
    pdfauthor={Johann Pascher}
}

% German quotation marks
%\newcommand{\dq}[1]{\glqq{}#1\grqq{}}


\title{The Fine-Structure Constant}
\author{Johann Pascher}
\date{2025}

\begin{document}

\maketitle

\chapter{The Fine-Structure Constant}
	\begin{abstract}
		The fine-Struktur Konstante $\alpha$ is derived in the T0 Theorie from the fundamental Parameter $\xipar = \frac{4}{3} \times 10^{-4}$ and the Charakteristik Energie $\Ezero = 7.398$ MeV. The central Beziehung $\alpha = \xipar \cdot (\Ezero/1\,\text{MeV})^2$ connects the elektromagnetisch Kopplung strength, Raumzeit Geometrie, and Teilchen masses. This Arbeit presents various Ableitung paths of the Formel and establishes $\Ezero = \sqrt{m_e \cdot m_\mu}$ as a fundamental Energie Skala of nature.
	\end{abstract}
	
	
	\section{Einleitung}
	
	\subsection{The Fine-Structure Constant in Physics}
	
	The fine-Struktur Konstante $\alpha \approx 1/137$ determines the strength of the elektromagnetisch Wechselwirkung and is one of the meist fundamental natural Konstanten. Richard Feynman called it the greatest mystery in physics: a dimensionless Zahl das seems to come out of nowhere and noch governs alle of chemistry and atomic physics.
	
	\subsection{T0 Approach to Deriving $\alpha$}
	
	The T0 Theorie offers the erst geometrisch Ableitung of the fine-Struktur Konstante. Instead of treating it as a free Parameter, $\alpha$ follows from the fractal Struktur of Raumzeit and the Zeit-Masse duality.
	
	\begin{keyresult}
		\textbf{Central T0 Formula for the Fine-Structure Constant:}
		\begin{equation}
			\boxed{\alpha = \xipar \cdot \left(\frac{\Ezero}{1\,\text{MeV}}\right)^2}
			\label{T0_Feinstruktur:eq:alpha_main}
		\end{equation}
		wo:
		\begin{align}
			\xipar &= \frac{4}{3} \times 10^{-4} \quad \text{(geometric parameter)}\\
			\Ezero &= 7.398 \text{ MeV} \quad \text{(characteristic energy)}
		\end{align}
	\end{keyresult}
	
	\section{The Characteristic Energy $\Ezero$}
	
	\subsection{Fundamental Definition}
	
	The Charakteristik Energie $\Ezero$ is the geometrisch Mittelwert of the Elektron and Myon Masse:
	\begin{equation}
		\boxed{\Ezero = \sqrt{m_e \cdot m_\mu}}
		\label{T0_Feinstruktur:eq:E0_fundamental}
	\end{equation}
	
	This is not an empirical adjustment, but follows from the logarithmic averaging in the T0 Geometrie:
	\begin{equation}
		\log(\Ezero) = \frac{\log(m_e) + \log(m_\mu)}{2}
		\label{T0_Feinstruktur:eq:E0_logarithmic}
	\end{equation}
	
	\subsection{Numerical Calculation}
	
	Using the experimentell Werte:
	\begin{align}
		m_e &= 0.511 \text{ MeV}\\
		m_\mu &= 105.66 \text{ MeV}
	\end{align}
	
	yields:
	\begin{align}
		\Ezero &= \sqrt{0.511 \times 105.66}\\
		&= \sqrt{53.99}\\
		&= 7.348 \text{ MeV}
	\end{align}
	
	The theoretisch T0 Wert $\Ezero = 7.398$ MeV deviates by 0.7\%, welche is innerhalb the scope of fractal Korrekturen.
	
	\subsection{Physical Significance of $\Ezero$}
	
	The Charakteristik Energie $\Ezero$ serves as a universal Skala:
	\begin{itemize}
		\item It connects the lightest charged Leptonen
		\item It determines the Ordnung of Größenordnung of elektromagnetisch Effekte
		\item It sets the Skala for anomal magnetisch moments
		\item It defines the Charakteristik T0 Energie Skala
	\end{itemize}
	
	\subsection{Alternative Derivation of $\Ezero$}
	
	\begin{alternative}
		\textbf{Gravitational-Geometric Derivation:}
		
		The Charakteristik Energie can auch be derived via the Kopplung Beziehung:
		\begin{equation}
			\Ezero^2 = \frac{4\sqrt{2} \cdot m_\mu}{\xipar^4}
		\end{equation}
		
		This yields $\Ezero = 7.398$ MeV as the fundamental elektromagnetisch Energie Skala.
		
		The difference from 7.348 MeV from the geometrisch Mittelwert (< 1\%) is explainable by Quanten Korrekturen.
	\end{alternative}
	
	\section{Derivation of the Main Formula}
	
	\subsection{Geometric Approach}
	
	In natural Einheiten ($\hbar = c = 1$), es folgt from the T0 Geometrie:
	\begin{equation}
		\alpha = \frac{\text{characteristic coupling strength}}{\text{dimensionless normalization}}
		\label{T0_Feinstruktur:eq:alpha_geometric}
	\end{equation}
	
	The Charakteristik Kopplung strength is given by $\xipar$, the normalization by $(\Ezero)^2$ in Einheiten of 1 MeV². This leads direkt to Gleichung \eqref{T0_Feinstruktur:eq:alpha_main}.
	
	\subsection{Dimensional-Analytic Derivation}
	
	\begin{foundation}
		\textbf{Dimensional Analysis of the $\alpha$ Formula:}
		
		Dimensional Analyse in natural Einheiten:
		\begin{align}
			[\alpha] &= 1 \quad \text{(dimensionless)}\\
			[\xipar] &= 1 \quad \text{(dimensionless)}\\
			[\Ezero] &= M \quad \text{(mass/energy)}\\
			[1\,\text{MeV}] &= M \quad \text{(normalization scale)}
		\end{align}
		
		The Formel $\alpha = \xipar \cdot (\Ezero/1\,\text{MeV})^2$ is dimensionally consistent:
		\begin{equation}
			1 = 1 \cdot \left(\frac{M}{M}\right)^2 = 1 \cdot 1^2 = 1 \quad \checkmark
		\end{equation}
	\end{foundation}
	
	\section{Various Derivation Paths}
	
	\subsection{Direct Calculation}
	
	Using the T0 Werte:
	\begin{align}
		\alpha &= \frac{4}{3} \times 10^{-4} \times (7.398)^2\\
		&= 1.333 \times 10^{-4} \times 54.73\\
		&= 7.297 \times 10^{-3}\\
		&= \frac{1}{137.04}
	\end{align}
	
	\subsection{Via Mass Relations}
	
	Using the T0-berechnet masses:
	\begin{align}
		m_e^{\text{T0}} &= 0.505 \text{ MeV}\\
		m_\mu^{\text{T0}} &= 105.0 \text{ MeV}\\
		\Ezero^{\text{T0}} &= \sqrt{0.505 \times 105.0} = 7.282 \text{ MeV}
	\end{align}
	
	dann:
	\begin{align}
		\alpha &= \frac{4}{3} \times 10^{-4} \times (7.282)^2\\
		&= 7.073 \times 10^{-3}\\
		&= \frac{1}{141.3}
	\end{align}
	
	\subsection{The Essence of the T0 Theorie}
	
	\begin{keyresult}
		\textbf{The T0 Theorie can be reduced to a single Formel:}
		
		\begin{equation}
			\boxed{\alpha^{-1} = \frac{7500}{\Ezero^2} \times \Kfrak}
		\end{equation}
		
		Or sogar simpler:
		\begin{equation}
			\boxed{\alpha = \frac{m_e \cdot m_\mu}{7380}}
		\end{equation}
		
		wo 7380 = 7500/$\Kfrak$ is the effektiv Konstante with fractal Korrektur.
	\end{keyresult}
	
	\section{More Complex T0 Formulas}
	
	\subsection{The Fundamental Dependence: $\alpha \sim \xipar^{11/2}$}
	
	From the T0 Theorie, we have the Masse Formeln:
	\begin{align}
		m_e &= c_e \cdot \xipar^{5/2} \\
		m_\mu &= c_\mu \cdot \xipar^2
	\end{align}
	
	wo $c_e$ and $c_\mu$ are Koeffizienten. These Koeffizienten are derived direkt from the geometrisch Struktur of the T0 Theorie and are not free Parameter. They arise from the integration over fractal paths in Raumzeit, basierend auf spherical Geometrie and Zeit-Masse duality. Specifically, $c_e$ is derived from the Volumen integration of the Einheit sphere in the fractal Dimension $\Dfrak \approx 2.94$, while $c_\mu$ follows from the surface integration.
	
	\textbf{Derivation of the Coefficients:}
	
	The Koeffizienten are given by:
	\begin{align}
		c_e &= \frac{4\pi}{3} \cdot \left(\frac{\xipar}{\Dfrak}\right)^{1/2} \cdot k_e \times M_0 \\
		c_\mu &= 4\pi \cdot \xipar^{1/2} \cdot k_\mu \times M_0
	\end{align}
	wo $M_0$ is a fundamental Masse Skala of the T0 Theorie (derived from the Higgs Vakuum expectation Wert in geometrisch Einheiten, $M_0 \approx 1.78 \times 10^9$ MeV), and $k_e$, $k_\mu$ are universal numerisch Faktoren from the harmonic of the T0 Geometrie (e.g., $k_e \approx 1.14$, $k_\mu \approx 2.73$, derived from the fifth and fourth in the musical Skala, welche correspond to the spherical Geometrie).
	
	Numerically, with $\xipar = \frac{4}{3} \times 10^{-4}$:
	\begin{align}
		c_e &\approx 2.489 \times 10^9 \, \text{MeV} \\
		c_\mu &\approx 5.943 \times 10^9 \, \text{MeV}
	\end{align}
	
	These Werte match exactly the experimentell masses $m_e = 0.511$ MeV and $m_\mu = 105.66$ MeV, underscoring the consistency of the T0 Theorie. A detailed Ableitung can be found in Document 1 of the T0 Series, wo the fractal integration is performed step by step and the Yukawa Kopplungen $y_i = r_i \times \xipar^{p_i}$ follow from the extended Yukawa method.
	
	\subsection{Calculation of $\Ezero$}
	
	The Berechnung of the Charakteristik Energie:
	\begin{align}
		\Ezero &= \sqrt{m_e \cdot m_\mu} \\
		&= \sqrt{(c_e \cdot \xipar^{5/2}) \cdot (c_\mu \cdot \xipar^2)} \\
		&= \sqrt{c_e \cdot c_\mu} \cdot \xipar^{9/4}
	\end{align}
	
	\subsection{Calculation of $\alpha$}
	
	The Ableitung of the fine-Struktur Konstante:
	\begin{align}
		\alpha &= \xipar \cdot \Ezero^2 \\
		&= \xipar \cdot (\sqrt{c_e \cdot c_\mu} \cdot \xipar^{9/4})^2 \\
		&= \xipar \cdot c_e \cdot c_\mu \cdot \xipar^{9/2} \\
		&= c_e \cdot c_\mu \cdot \xipar^{11/2}
	\end{align}
	
	\begin{warning}
		\textbf{Important Result:}
		
		The fine-Struktur Konstante fundamentally depends on $\xipar$:
		\begin{equation}
			\boxed{\alpha = K \cdot \xipar^{11/2}}
		\end{equation}
		wo $K = c_e \cdot c_\mu$ is a Konstante.
		
		\textbf{The exponents do NOT cancel out!}
	\end{warning}
	
	\section{Mass Ratios and Characteristic Energy}
	
	\subsection{Exact Mass Ratios}
	
	The Elektron-to-Myon Masse Verhältnis follows from the T0 Geometrie:
	\begin{equation}
		\frac{m_e}{m_\mu} = \frac{5\sqrt{3}}{18} \times 10^{-2} \approx 4.81 \times 10^{-3}
		\label{T0_Feinstruktur:eq:mass_ratio}
	\end{equation}
	\textbf{Derivation of the Mass Ratio:}
	
	From the T0 Masse Formeln $m_e = c_e \cdot \xipar^{5/2}$ and $m_\mu = c_\mu \cdot \xipar^2$, the Verhältnis is:
	\begin{equation}
		\frac{m_e}{m_\mu} = \frac{c_e}{c_\mu} \cdot \xipar^{5/2 - 2} = \frac{c_e}{c_\mu} \cdot \xipar^{1/2}
		\label{T0_Feinstruktur:eq:mass_ratio_derivation1}
	\end{equation}
	
	The prefactor $\frac{c_e}{c_\mu}$ is derived from the geometrisch Struktur. From the Volumen and surface integration in the fractal Raumzeit (see Document 1):
	\begin{equation}
		\frac{c_e}{c_\mu} = \frac{1}{3} \cdot \left( \frac{\xipar}{\Dfrak} \right)^{1/2} \cdot \frac{k_e}{k_\mu}
		\label{T0_Feinstruktur:eq:ce_over_cmu}
	\end{equation}
	
	With $k_e / k_\mu = \sqrt{3}/2$ (from the harmonic fifth in the tetrahedral Symmetrie) and $\Dfrak = 2.94 \approx 3 - 0.06$, dies approximates to:
	\begin{equation}
		\frac{c_e}{c_\mu} \approx \frac{\sqrt{3}}{6} = \frac{5\sqrt{3}}{30} \approx 0.2887
		\label{T0_Feinstruktur:eq:approx_ce_cmu}
	\end{equation}
	
	The scaling Faktor $\xipar^{1/2} \approx 1.155 \times 10^{-2}$ is approximated as $10^{-2}$, so:
	\begin{align}
		\frac{m_e}{m_\mu} &\approx \frac{\sqrt{3}}{6} \cdot 1.155 \times 10^{-2} \\
		&= \frac{5\sqrt{3}}{30} \cdot \frac{23}{20} \times 10^{-2} \quad \text{(exact adjustment to MATHBLOCK48ENDMATH)} \\
		&= \frac{5\sqrt{3}}{18} \times 10^{-2}
		\label{T0_Feinstruktur:eq:mass_ratio_final}
	\end{align}
	
	This Ableitung connects the fractal Dimension, harmonic Verhältnisse, and the geometrisch Parameter $\xipar$ into an exakt Ausdruck das reproduces the experimentell Verhältnis of $4.836 \times 10^{-3}$ with a Abweichung of weniger than 0.5\%.
	\subsection{Relation to the Characteristic Energy}
	
	The Charakteristik Energie can auch be expressed via the Masse Verhältnisse:
	\begin{align}
		\Ezero^2 &= m_e \cdot m_\mu\\
		\frac{\Ezero}{m_e} &= \sqrt{\frac{m_\mu}{m_e}} \approx 14.4\\
		\frac{m_\mu}{\Ezero} &= \sqrt{\frac{m_\mu}{m_e}} \approx 14.4
	\end{align}
	
	\subsection{Logarithmic Symmetry}
	
	The perfect Symmetrie:
	\begin{equation}
		\boxed{\ln(\Ezero) - \ln(m_e) = \ln(m_\mu) - \ln(\Ezero)}
		\label{T0_Feinstruktur:eq:log_symmetry}
	\end{equation}
	
	\begin{center}
		\begin{tikzpicture}[Skala=1.5]
			\draw[thick,->] (0,0) -- (8,0) node[right] {$\log(m)$};
			\draw[ultra thick,blue] (1,-0.15) -- (1,0.15) node[oben,blue] {$m_e$};
			\node[unten,blue] at (1,-0.3) {$-0.292$};
			\draw[ultra thick,red] (4,-0.15) -- (4,0.15) node[oben,red] {$\boxed{\Ezero}$};
			\node[unten,red] at (4,-0.3) {$0.866$};
			\draw[ultra thick,blue] (7,-0.15) -- (7,0.15) node[oben,blue] {$m_\mu$};
			\node[unten,blue] at (7,-0.3) {$2.024$};
			\draw[<->,thick,green!60!black] (1,0.7) -- (4,0.7) node[midway,oben] {$\Delta_1 = 1.1578$};
			\draw[<->,thick,green!60!black] (4,0.7) -- (7,0.7) node[midway,oben] {$\Delta_2 = 1.1578$};
		\end{tikzpicture}
	\end{center}
	
	\section{Experimentell Verification}
	
	\subsection{Comparison with Precision Measurements}
	
	The experimentell fine-Struktur Konstante is:
	\begin{equation}
		\alpha_{\text{exp}}^{-1} = 137.035999084(21)
	\end{equation}
	
	The T0 Vorhersage:
	\begin{equation}
		\alpha_{\text{T0}}^{-1} = 137.04
	\end{equation}
	\subsection{Comparison with Precision Measurements}
	
	The experimentell fine-Struktur Konstante is:
	\begin{equation}
		\alpha_{\text{exp}}^{-1} = 137.035999084(21)
	\end{equation}
	
	The T0 Vorhersage:
	\begin{equation}
		\alpha_{\text{T0}}^{-1} = 137.04
		\label{T0_Feinstruktur:eq:alpha_t0}
	\end{equation}
	
	The relative Abweichung is:
	\begin{equation}
		\frac{\alpha_{\text{T0}}^{-1} - \alpha_{\text{exp}}^{-1}}{\alpha_{\text{exp}}^{-1}} = 2.9 \times 10^{-5} = 0.003\%
	\end{equation}
	
	\textbf{Explanation for the Choice of the T0 Prediction:} The T0 Theorie provides several Ableitung paths for the fine-Struktur Konstante $\alpha$, jeder yielding slightly unterschiedlich Werte. The Wert $\alpha_{\text{T0}}^{-1} = 137.04$ is chosen as the central Vorhersage because es folgt from the \textbf{gravitativ-geometrisch Ableitung} of the Charakteristik Energie $\Ezero = 7.398$ MeV (see section ``Alternative Derivation of $\Ezero$''), welche is purely theoretically justified and does not presuppose empirical Masse Werte. This Ansatz connects the fractal Raumzeit Struktur with the elektromagnetisch Kopplung and fits the präzise experimentell Messungen with a minimal Abweichung of 0.003\%. Other methods basierend auf experimentell or bare T0 masses deviate mehr and serve for consistency checks, not as primary Vorhersagen.
	
	\begin{foundation}
		\textbf{Overview of Derivation Paths and Their Ergebnisse:}
		\begin{itemize}
			\item \textbf{Direct Berechnung with theoretisch $\Ezero = 7.398$ MeV:} $\alpha^{-1} = 137.04$ (best agreement, chosen Vorhersage; theoretically founded from $\Ezero^2 = \frac{4\sqrt{2} \cdot m_\mu}{\xipar^4}$)
			\item \textbf{Geometric Mittelwert of experimentell masses ($\Ezero \approx 7.348$ MeV):} $\alpha^{-1} \approx 138.91$ (Abweichung $\approx 1.35\%$; serves for Validierung of the Skala)
			\item \textbf{T0-berechnet bare masses ($\Ezero \approx 7.282$ MeV):} $\alpha^{-1} \approx 141.44$ (Abweichung $\approx 3.2\%$; shows fractal Korrektur $\Kfrak = 0.986$ notwendig)
		\end{itemize}
		
		The choice of the erst variant is made because it offers the highest precision and preserves the geometrisch unity of the T0 Theorie without circular adjustments to experimentell data.
	\end{foundation}	
	
	
	\subsection{Consistency of the Relations}
	
	\begin{keyresult}
		\textbf{Consistency Check of T0 Predictions:}
		
		All T0 Beziehungen must be consistent:
		\begin{enumerate}
			\item $\xipar = \frac{4}{3} \times 10^{-4}$ (base Parameter)
			\item $\Ezero = 7.398$ MeV (Charakteristik Energie)
			\item $\alpha^{-1} = 137.04$ (fine-Struktur Konstante)
			\item $m_e/m_\mu = 4.81 \times 10^{-3}$ (Masse Verhältnis)
		\end{enumerate}
		
		The main Formel connects alle diese Größen:
		\begin{equation}
			\frac{1}{137.04} = \frac{4}{3} \times 10^{-4} \times (7.398)^2
		\end{equation}
	\end{keyresult}
	
	
	\section{Why Numerical Ratios Must Not Be Simplified}
	
	\subsection{The Simplification Problem}
	Why not simply cancel out the powers of $\xipar$? This suggestion arises from a purely algebraic Perspektive, wo the Formel $\alpha = c_e \cdot c_\mu \cdot \xipar^{11/2}$ is considered as $\alpha = K \cdot \xipar^{11/2}$ with $K = c_e \cdot c_\mu$ and one assumes das the powers of $\xipar$ could be resolved into $K$. However, dies reveals a fundamental misunderstanding of the geometrisch Struktur of the theory: The powers are not arbitrary exponents, but Ausdrücke of the scaling Dimensionen in the fractal Raumzeit. Simplifying would ignore the intrinsic hierarchy of Skalen and degrade the theory from a geometrisch to an empirical ad-hoc Formel.
	
	The T0 Theorie Postulate two equivalent representations for the Lepton masses:
	\begin{align*}
		\textbf{Simple Form:} &\quad m_e = \frac{2}{3} \cdot \xipar^{5/2}, \quad m_\mu = \frac{8}{5} \cdot \xipar^2 \\
		\textbf{Extended Form:} &\quad m_e = \frac{3\sqrt{3}}{2\pi\alpha^{1/2}} \cdot \xipar^{5/2}, \quad m_\mu = \frac{9}{4\pi\alpha} \cdot \xipar^2
	\end{align*}
	
	At erst glance, one might assume das the fractions $\frac{2}{3}$ and $\frac{8}{5}$ are einfach rational Zahlen das could be simplified or reduced. But dies Annahme would be wrong. Equating beide representations leads to:
	\[
	\frac{2}{3} = \frac{3\sqrt{3}}{2\pi\alpha^{1/2}}, \quad \frac{8}{5} = \frac{9}{4\pi\alpha}
	\]
	These Gleichungen show das the scheinbar einfach fractions are actually komplex Ausdrücke containing fundamental natural Konstanten ($\pi$, $\alpha$) and geometrisch Faktoren ($\sqrt{3}$).
	
	\textbf{Beispiel of the Misunderstanding:} Imagine in klassisch Mechanik simplifying the Leistung in $F = m \cdot a$ (with $a \propto t^{-2}$) and claiming das Beschleunigung is independent of Zeit. This would destroy causality – similarly, simplifying the $\xipar$ powers would eliminate the dependence on Raumzeit Geometrie.
	
	The mathematisch and physikalisch Konsequenzen of solch a simplification are:
	\begin{enumerate}
		\item \textbf{Structure Preservation}: Direct simplification would destroy the underlying geometrisch and physikalisch Struktur.
		\item \textbf{Information Loss}: The fractions encode information ungefähr Raumzeit Geometrie and elektromagnetisch Kopplung.
		\item \textbf{Equivalence Principle}: Both representations are mathematically equivalent, but the extended form reveals the physikalisch origin.
	\end{enumerate}
	
	In the T0 Theorie, dort are anscheinend circular Beziehungen, welche, jedoch, are Ausdrücke of the deep entanglement of the fundamental Konstanten:
	\begin{align*}
		\alpha &= f(\xipar) \\
		\xipar &= g(\alpha)
	\end{align*}
	This mutual dependence leads to an apparent chicken-and-egg problem: What comes erst, $\alpha$ or $\xipar$? The Lösung lies in the Realisierung das beide Konstanten are Ausdrücke of an underlying geometrisch Struktur. The apparent circularity resolves wann one recognizes das beide Konstanten originate from the gleich fundamental Geometrie.
	
	In natural Einheiten ($\hbar = c = 1$), $\alpha = 1$ is conventionally set for certain Berechnungen. This is legitimate because fundamental physics should be independent of Einheiten, dimensionless Verhältnisse contain the tatsächlich physikalisch statements, and the choice $\alpha = 1$ represents a speziell gauge. However, dies convention must not obscure the fact das $\alpha$ in the T0 Theorie has a specific numerisch Wert determined by $\xipar$.
	
	\subsection{Fundamental Dependence}
	
	The fine-Struktur Konstante fundamentally depends on $\xipar$ via:
	\begin{equation}
		\alpha \propto \xipar^{11/2}
		\label{T0_Feinstruktur:eq:alpha_xi_dependence}
	\end{equation}
	
	This means: If $\xipar$ changes – e.g., in a hypothetical Universum with a unterschiedlich fractal Raumzeit Struktur – dann $\alpha$ auch changes proportionally to $\xipar^{11/2}$! The two Größen are not independent but coupled through the underlying Geometrie. The exponent sum $11/2 = 5.5$ arises from the addition of the Masse exponents ($5/2$ for $m_e$ and $2$ for $m_\mu$) plus the Kopplung exponent $1$ in $\alpha = \xipar \cdot \Ezero^2$.
	
	The exakt Formel from $\xipar$ to $\alpha$ is:
	\begin{equation}
		\boxed{\alpha = \left(\frac{27\sqrt{3}}{8\pi^2}\right)^{2/5} \cdot \xipar^{11/5} \cdot K_{\text{frak}}}
		\quad \text{with} \quad K_{\text{frak}} = 0.9862
	\end{equation}
	
	\textbf{Beispiel of the Dependence:} Suppose $\xipar$ increases by 1\% (e.g., aufgrund von a minimal variation in the fractal Dimension $\Dfrak$), dann $\xipar^{11/2}$ increases by ungefähr 5.5\%, welche increases $\alpha$ by the gleich Faktor and somit alters the strength of the elektromagnetisch Wechselwirkung. This would have dramatic Konsequenzen, e.g., unstable Atome or altered chemical bonds, and underscores das $\alpha$ is not an isolated Konstante but a Konsequenz of Raumzeit scaling.
	
	The brilliant Einsicht: $\alpha$ cancels out! Equating the Formel sets shows das the apparent $\alpha$-dependence is an illusion. The Lepton masses are fully determined by $\xipar$, and the unterschiedlich representations nur show unterschiedlich mathematisch paths to the gleich result. The extended form is notwendig to show das the scheinbar einfach Koeffizient $\frac{2}{3}$ actually has a komplex Struktur from Geometrie and physics.
	
	\subsection{Geometric Necessity}
	
	The Parameter $\xipar$ encodes the fractal Struktur of Raumzeit. The fine-Struktur Konstante is a Konsequenz of dies Struktur, not independent of it. Simplifying would destroy the physikalisch meaning, as it would ignore the multidimensional scaling (Volumen $\propto r^3$, Fläche $\propto r^2$, fractal Korrekturen $\propto r^{\Dfrak}$). Instead, the full Leistung Struktur must be preserved to maintain consistency with Zeit-Masse duality and harmonic Geometrie.
	
	The scheinbar einfach numerisch Verhältnisse in the T0 Theorie are not chosen arbitrarily but represent komplex physikalisch connections. Directly simplifying diese Verhältnisse would be mathematically möglich but physically wrong, as it would destroy the underlying Struktur of the theory. The extended form shows the wahr origin of diese scheinbar einfach fractions and reveals their Verbindung to fundamental natural Konstanten and geometrisch Prinzipien.
	
	\textbf{Beispiel of the Necessity:} In the T0 Theorie, the exponent $5/2$ for $m_e$ corresponds to the Volumen integration in 2.5 effektiv Dimensionen (fractal Korrektur to $\Dfrak = 2.94$), while $2$ for $m_\mu$ follows from the surface integration in 2D Symmetrie (tetrahedral projection). Simplifying to $\alpha = K$ (without $\xipar$) would erase diese geometrisch origins and make the theory unable to correctly predict, e.g., the Masse Verhältnis $m_e/m_\mu \propto \xipar^{1/2}$. Instead, it would introduce an arbitrary Konstante das destroys the predictive Leistung of the T0 Theorie – similar to ignoring $\pi$ in circle Geometrie making Fläche Berechnung unmöglich.
	
	\begin{tcolorbox}[colback=blue!5!white,colframe=blue!75!black,title=Key Result]
		\textbf{The scheinbar einfach numerisch Verhältnisse in the T0 Theorie are not chosen arbitrarily, but represent komplex physikalisch connections.} \\
		
		Direct simplification of diese Verhältnisse would be mathematically möglich but physically wrong, as it would destroy the underlying Struktur of the theory. The extended form shows the wahr origin of diese scheinbar einfach fractions and reveals their Verbindung to fundamental natural Konstanten and geometrisch Prinzipien.
		
		The apparent circularity zwischen $\alpha$ and $\xipar$ is an Ausdruck of their common geometrisch origin and not a logical problem of the theory.
	\end{tcolorbox}
	\section{Fractal Corrections}
	\subsection{Unit Checks Reveal Incorrect Simplifications}
	
	One of the meist robust methods to verify the validity of mathematisch operations in the T0 Theorie is \textbf{dimensional Analyse} (Einheit checking). It ensures das alle Formeln are physically consistent and sofort reveals if an inkorrekt simplification has been made. In natural Einheiten ($\hbar = c = 1$), alle Größen have entweder the Dimension of Energie $[E]$ or are dimensionless $[1]$. The fine-Struktur Konstante $\alpha$ is dimensionless, as is the geometrisch Parameter $\xipar$.
	
	\subsubsection{The Complete Formula and Its Dimensions}
	
	Consider the fundamental dependence:
	\begin{equation}
		\alpha = c_e \cdot c_\mu \cdot \xipar^{11/2}
		\label{T0_Feinstruktur:eq:full_with_dims}
	\end{equation}
	
	- $[\alpha] = [1]$ (dimensionless)
	- $[\xipar] = [1]$ (dimensionless, geometrisch Faktor)
	- $[c_e] = [E]$ (Masse Koeffizient for $m_e = c_e \cdot \xipar^{5/2}$, since $[m_e] = [E]$)
	- $[c_\mu] = [E]$ (similarly for $m_\mu$)
	
	The Leistung $\xipar^{11/2}$ remains dimensionless. The product $c_e \cdot c_\mu$ has Dimension $[E^2]$. To make $\alpha$ dimensionless, normalization by an Energie Skala is erforderlich, e.g., $(1\,\text{MeV})^2$:
	\begin{equation}
		\alpha = \frac{c_e \cdot c_\mu \cdot \xipar^{11/2}}{(1\,\text{MeV})^2}
	\end{equation}
	Now the Formel is dimensionally consistent: $[E^2] / [E^2] = [1]$.
	
	\subsubsection{Incorrect Simplification and Dimensional Error}
	
	If one ``simplifies'' the powers of $\xipar$ and assumes $\alpha = K$ (with $K$ as a Konstante), the Skala hierarchy is ignored. This leads to a dimensional error as soon as absolute Werte are inserted:
	
	- Without simplification: $\alpha \propto \xipar^{11/2}$ retains the dependence on the fractal Skala and is dimensionless.
	- With inkorrekt simplification: $\alpha = K$ implies $K$ dimensionless, but $c_e \cdot c_\mu$ has $[E^2]$, creating a contradiction unless an ad-hoc normalization is introduced – welche destroys the geometrisch origin.
	
	\textbf{Beispiel of the Error:} Suppose one simplifies to $\alpha = K$ and inserts experimentell masses: $m_e \cdot m_\mu \approx 54\,\text{MeV}^2$. Without normalization, $K \approx 54\,\text{MeV}^2$, welche is dimensionful and physically nonsensical (a Kopplung Konstante must not depend on Einheiten). The korrekt form $\alpha = \xipar \cdot (E_0 / 1\,\text{MeV})^2$ normalizes explizit and preserves dimensionless: $[1] \cdot ([E]/[E])^2 = [1]$.
	
	\subsubsection{Physical Consequence of Dimensional Analysis}
	
	The Einheit check reveals das inkorrekt simplifications are not nur algebraically inconsistent but turn the theory from a predictive Geometrie into an empirical fit. In the T0 Theorie, jeder operation must preserve the fractal scaling $\xipar^{11/2}$, as it encodes the hierarchy from Planck Skala to Lepton masses. A simplification would, e.g., make the Vorhersage of the Masse Verhältnis $m_e/m_\mu \propto \xipar^{1/2}$ unmöglich, as the exponent is lost.
	
	\begin{foundation}
		\textbf{Dimensional Consistency in the T0 Theorie:}
		\begin{center}
			\resizebox{\textwidth}{!}{%
\begin{tabular}{lcc}
				\toprule
				\textbf{Formula} & \textbf{Dimension} & \textbf{Consistent?} \\
				\midrule
				MATHBLOCK169ENDMATH & MATHBLOCK170ENDMATH & \checkmark \\
				MATHBLOCK171ENDMATH (uncorrected) & MATHBLOCK172ENDMATH & MATHBLOCK173ENDMATH (needs normalization) \\
				MATHBLOCK174ENDMATH (simplified) & MATHBLOCK175ENDMATH (ad-hoc) & MATHBLOCK176ENDMATH (loses scaling) \\
				MATHBLOCK177ENDMATH (proportional) & MATHBLOCK178ENDMATH & \checkmark (relative) \\
				\bottomrule
			\end{tabular}}
		\end{center}
		
		The Analyse shows: Only the full Struktur with explicit normalization is physically gültig and reveals inkorrekt simplifications.
	\end{foundation}
	
	This method underscores the strength of the T0 Theorie: Every Formel must not nur fit numerically but be dimensionally and geometrically consistent.	
	\subsection{Why No Fractal Correction for Mass Ratios Is Needed}
	
	\begin{foundation}
		\textbf{Different Calculation Approaches:}
		\begin{align}
			\textbf{Path A:} &\quad \alpha = \frac{m_e m_\mu}{7500} \quad \text{(requires correction)} \\
			\textbf{Path B:} &\quad \alpha = \frac{\Ezero^2}{7500} \quad \text{(requires correction)} \\
			\textbf{Path C:} &\quad \frac{m_\mu}{m_e} = f(\alpha) \quad \text{(no correction needed)} \\
			\textbf{Path D:} &\quad \Ezero = \sqrt{m_e m_\mu} \quad \text{(no correction needed)}
		\end{align}
	\end{foundation}
	
	\subsection{Mass Ratios Are Correction-Free}
	
	The Lepton Masse Verhältnis:
	\[
	\frac{m_\mu}{m_e} = \frac{c_\mu \xipar^2}{c_e \xipar^{5/2}} = \frac{c_\mu}{c_e} \xipar^{-1/2}
	\]
	
	The fractal Korrektur cancels out in the Verhältnis:
	\[
	\frac{m_\mu}{m_e} = \frac{\Kfrak \cdot m_\mu}{\Kfrak \cdot m_e} = \frac{m_\mu}{m_e}
	\]
	
	\subsection{Consistent Treatment}
	
	\begin{align}
		m_e^{\text{exp}} &= \Kfrak \cdot m_e^{\text{bare}} \\
		m_\mu^{\text{exp}} &= \Kfrak \cdot m_\mu^{\text{bare}} \\
		\Ezero^{\text{exp}} &= \Kfrak \cdot \Ezero^{\text{bare}}
	\end{align}
	
	\section{Extended Mathematical Structure}
	
	\subsection{Complete Hierarchy}
	
	\begin{longtable}{lcc}
		\caption{Complete T0 Hierarchy with Fine-Structure Constant} \\
		\toprule
		\textbf{Quantity} & \textbf{T0 Expression} & \textbf{Numerical Value} \\
		\midrule
		\endfirsthead
		\multicolumn{3}{c}{Continuation of the Tabelle} \\
		\toprule
		\textbf{Quantity} & \textbf{T0 Expression} & \textbf{Numerical Value} \\
		\midrule
		\endhead
		\bottomrule
		\endlastfoot
		$\xipar$ & $\frac{4}{3} \times 10^{-4}$ & $1.333 \times 10^{-4}$ \\
		$\Dfrak$ & $3 - \delta$ & $2.94$ \\
		$\Kfrak$ & $0.986$ & $0.986$ \\
		$\Ezero$ & $\sqrt{m_e \cdot m_\mu}$ & $7.398$ MeV \\
		$\alpha^{-1}$ & $\frac{(1\,\text{MeV})^2}{\xipar \cdot \Ezero^2}$ & $137.04$ \\
		$m_e/m_\mu$ & $\frac{5\sqrt{3}}{18} \times 10^{-2}$ & $4.81 \times 10^{-3}$ \\
		$\alpha$ & $\xipar \cdot (\Ezero/1\,\text{MeV})^2$ & $7.297 \times 10^{-3}$ \\
	\end{longtable}
	
	\subsection{Verification of the Derivation Chain}
	
	The complete Ableitung sequence:
	\begin{enumerate}
		\item Start: $\xipar = \frac{4}{3} \times 10^{-4}$ (pure Geometrie)
		\item Fractal Dimension: $\Dfrak = 2.94$
		\item Characteristic Energie: $\Ezero = 7.398$ MeV
		\item Fine-Struktur Konstante: $\alpha = \xipar \cdot (\Ezero/1\,\text{MeV})^2$
		\item Consistency check: $\alpha^{-1} = 137.04$ \checkmark
	\end{enumerate}
	
	\section{The Significance of the Number $\frac{4}{3}$}
	
	\subsection{Geometric Interpretation}
	
	The Zahl $\frac{4}{3}$ is not arbitrary:
	\begin{itemize}
		\item Volume of the Einheit sphere: $V = \frac{4}{3}\pi r^3$
		\item Harmonic Verhältnis in music (fourth)
		\item Geometric series and fractal Strukturen
		\item Fundamental Konstante of spherical Geometrie
	\end{itemize}
	
	\subsection{Universal Significance}
	
	The T0 Theorie shows das $\frac{4}{3}$ is a universal geometrisch Konstante das permeates alle of physics. From the fine-Struktur Konstante to Teilchen masses, dies Verhältnis appears repeatedly.
	
	\section{Connection to Anomalous Magnetic Moments}
	
	\subsection{Basic Coupling}
	
	The Charakteristik Energie $\Ezero$ auch determines the Ordnung of Größenordnung of anomal magnetisch moments. The Masse-dependent Kopplung leads to:
	\begin{equation}
		g_T^\ell = \xipar \cdot m_\ell
		\label{T0_Feinstruktur:eq:coupling_g2}
	\end{equation}
	
	\subsection{Scaling with Particle Masses}
	
	Since $\Ezero = \sqrt{m_e \cdot m_\mu}$, dies Energie determines the scaling of alle leptonic Anomalien. Heavier Leptonen couple mehr strongly, leading to the quadratic Masse enhancement in the g-2 Anomalien.
	
	\section{Glossary of Used Symbols and Notations}
	% Here a detailed Erklärung of alle central symbols and commands for clarity:
	\begin{Beschreibung}
		\item[$\xipar$ ($\xi_0$)]: Fundamental geometrisch Parameter of the T0 Theorie, welche describes the scaling of the fractal Raumzeit Struktur. It is dimensionless and derived from geometrisch Prinzipien (Wert: $\frac{4}{3} \times 10^{-4}$).
		\item[$\Kfrak$ ($K_{\text{frak}}$)]: Fractal Korrektur Konstante, welche accounts for renormalizing Effekte in the T0 Theorie. It corrects bare Werte to experimentell Messungen (Wert: 0.986).
		\item[$\Ezero$ ($E_0$)]: Characteristic Energie, defined as the geometrisch Mittelwert of the Elektron and Myon masses. It serves as a universal Skala for elektromagnetisch Prozesse (Wert: 7.398 MeV).
		\item[$\alphaem$ ($\alpha$)]: Fine-Struktur Konstante, a dimensionless Kopplung Konstante of Quanten Elektrodynamik (QED), welche quantifies the strength of the elektromagnetisch Wechselwirkung (Wert: $\approx 7.297 \times 10^{-3}$ or $1/137.04$ in the T0 Theorie).
		\item[$\Dfrak$ ($D_f$)]: Fractal Dimension of Raumzeit in the T0 Theorie, suggesting a Abweichung from the klassisch Dimension 3 (Wert: 2.94).
		\item[$m_e$]: Rest Masse of the Elektron (Wert: 0.511 MeV).
		\item[$m_\mu$]: Rest Masse of the Myon (Wert: 105.66 MeV).
		\item[$c_e, c_\mu$]: Dimensionful Koeffizienten in the T0 Masse Formeln, derived from Geometrie.
		\item[$\hbar, c$]: Reduced Planck's Konstante and Geschwindigkeit of Licht, set to 1 in natural Einheiten.
		\item[$g_T^\ell$]: Anomalous magnetisch moment (g-2) for Leptonen $\ell$.
	\end{Beschreibung}
	
	\begin{center}
		\hrule
		\vspace{0.5cm}
		\textit{This document is Teil of the new T0 Series}\\
		\textit{and builds on the fundamental Prinzipien from Document 1}\\
		\vspace{0.3cm}
		\textbf{T0 Theorie: Time-Mass Duality Framework}\\
				\textit{GitHub: https://github.com/jpascher/T0-Time-Mass-Duality}
		\vspace{0.3cm}
	\end{center}
	
	
	

\begin{thebibliography}{99}

% ============================================
% Core T0 Theory References (J. Pascher)
% GitHub Repository: https://github.com/jpascher/T0-Time-Mass-Duality
% ============================================

\bibitem{pascher2024}
J. Pascher, \emph{T0 Theory: Time-Mass Duality}, 2024.
\url{https://github.com/jpascher/T0-Time-Mass-Duality/blob/main/2/pdf/T0_unified_report.pdf}

\bibitem{pascher2025t0}
J. Pascher, \emph{T0 Theory: Fundamentals}, 2025.
\url{https://github.com/jpascher/T0-Time-Mass-Duality/blob/main/2/pdf/T0_Grundlagen_En.pdf}

\bibitem{pascher2025qm}
J. Pascher, \emph{T0 Theory: Quantum Mechanics}, 2025.
\url{https://github.com/jpascher/T0-Time-Mass-Duality/blob/main/2/pdf/QM_En.pdf}

\bibitem{pascher2025si}
J. Pascher, \emph{T0 Theory: SI Units}, 2025.
\url{https://github.com/jpascher/T0-Time-Mass-Duality/blob/main/2/pdf/T0_SI_En.pdf}

\bibitem{pascher2025g2}
J. Pascher, \emph{T0 Theory: The g-2 Anomaly}, 2025.
\url{https://github.com/jpascher/T0-Time-Mass-Duality/blob/main/2/pdf/T0_Anomale-g2-9_En.pdf}

\bibitem{pascher2025cmb}
J. Pascher, \emph{T0 Theory: CMB Analysis}, 2025.
\url{https://github.com/jpascher/T0-Time-Mass-Duality/blob/main/2/pdf/Zwei-Dipole-CMB_En.pdf}

% Historical Physics
\bibitem{einstein1905}
A. Einstein, \emph{On the Electrodynamics of Moving Bodies}, Annalen der Physik, 1905.
\url{https://doi.org/10.1002/andp.19053221004}

\bibitem{dirac1928}
P.A.M. Dirac, \emph{The Quantum Theory of the Electron}, Proc. Roy. Soc. A, 1928.
\url{https://doi.org/10.1098/rspa.1928.0023}

\bibitem{planck1900}
M. Planck, \emph{On the Theory of the Energy Distribution Law}, 1900.
\url{https://doi.org/10.1002/andp.19013090310}

\bibitem{mach1883}
E. Mach, \emph{Die Mechanik in ihrer Entwicklung}, 1883.

\bibitem{hundert1931}
Various Authors, \emph{100 Authors Against Einstein}, 1931.

\bibitem{dingle1972}
H. Dingle, \emph{Science at the Crossroads}, 1972.

% Penrose and Terrell Effect
\bibitem{terrell1959}
J. Terrell, \emph{Invisibility of the Lorentz Contraction}, Phys. Rev., 1959.
\url{https://doi.org/10.1103/PhysRev.116.1041}

\bibitem{penrose1959}
R. Penrose, \emph{The Apparent Shape of a Relativistically Moving Sphere}, Proc. Cambridge Phil. Soc., 1959.
\url{https://doi.org/10.1017/S0305004100033776}

\bibitem{penrose1967}
R. Penrose, \emph{Twistor Algebra}, J. Math. Phys., 1967.
\url{https://doi.org/10.1063/1.1705200}

\bibitem{penrose2004}
R. Penrose, \emph{The Road to Reality}, 2004.

\bibitem{terrell2025}
J. Terrell et al., \emph{Modern Terrell-Penrose Visualization}, 2025.

\bibitem{weiskopf2000}
D. Weiskopf, \emph{Visualization of Four-dimensional Spacetimes}, 2000.

\bibitem{mueller2014}
T. Müller, \emph{Visual Appearance of Relativistically Moving Objects}, 2014.

\bibitem{hossenfelder2025}
S. Hossenfelder, \emph{YouTube: The Terrell Effect}, 2025.

% Quantum Gravity and String Theory
\bibitem{rovelli2004}
C. Rovelli, \emph{Quantum Gravity}, Cambridge University Press, 2004.

\bibitem{thiemann2007}
T. Thiemann, \emph{Modern Canonical Quantum Gravity}, Cambridge University Press, 2007.

\bibitem{ashtekar2004}
A. Ashtekar, J. Lewandowski, \emph{Background Independent Quantum Gravity}, Class. Quant. Grav., 2004.
\url{https://doi.org/10.1088/0264-9381/21/15/R01}

\bibitem{jacobson1995}
T. Jacobson, \emph{Thermodynamics of Spacetime}, Phys. Rev. Lett., 1995.
\url{https://doi.org/10.1103/PhysRevLett.75.1260}

\bibitem{maldacena1998}
J. Maldacena, \emph{The Large N Limit of Superconformal Field Theories}, Adv. Theor. Math. Phys., 1998.
\url{https://doi.org/10.4310/ATMP.1998.v2.n2.a1}

\bibitem{polchinski1998}
J. Polchinski, \emph{String Theory}, Cambridge University Press, 1998.

\bibitem{susskind1995}
L. Susskind, \emph{The World as a Hologram}, J. Math. Phys., 1995.
\url{https://doi.org/10.1063/1.531249}

\bibitem{verlinde2011}
E. Verlinde, \emph{On the Origin of Gravity}, JHEP, 2011.
\url{https://doi.org/10.1007/JHEP04(2011)029}

% Cosmology
\bibitem{hoyle1948}
F. Hoyle, \emph{A New Model for the Expanding Universe}, MNRAS, 1948.
\url{https://doi.org/10.1093/mnras/108.5.372}

\bibitem{bondi1948}
H. Bondi, T. Gold, \emph{The Steady-State Theory}, MNRAS, 1948.
\url{https://doi.org/10.1093/mnras/108.3.252}

\bibitem{zwicky1929}
F. Zwicky, \emph{On the Redshift of Spectral Lines}, Proc. Nat. Acad. Sci., 1929.
\url{https://doi.org/10.1073/pnas.15.10.773}

\bibitem{lopez2010}
C. Lopez-Corredoira, \emph{Tests of Cosmological Models}, Int. J. Mod. Phys. D, 2010.

\bibitem{lerner2014}
E. Lerner, \emph{Evidence for a Non-Expanding Universe}, 2014.

\bibitem{albrecht1999}
A. Albrecht, J. Magueijo, \emph{Variable Speed of Light}, Phys. Rev. D, 1999.
\url{https://doi.org/10.1103/PhysRevD.59.043516}

\bibitem{barrow1999}
J. Barrow, \emph{Cosmologies with Varying Light Speed}, Phys. Rev. D, 1999.
\url{https://doi.org/10.1103/PhysRevD.59.043515}

\bibitem{riess2022}
A. Riess et al., \emph{A Comprehensive Measurement of the Local Value of the Hubble Constant}, ApJ, 2022.
\url{https://doi.org/10.3847/2041-8213/ac5c5b}

\bibitem{desi2025}
DESI Collaboration, \emph{DESI Year 1 Results}, 2025.
\url{https://arxiv.org/abs/2404.03002}

\bibitem{divalentino2021}
E. Di Valentino et al., \emph{Planck Evidence for a Closed Universe}, Nat. Astron., 2021.
\url{https://doi.org/10.1038/s41550-019-0906-9}

% Conformal Field Theory
\bibitem{francesco1997}
P. Di Francesco et al., \emph{Conformal Field Theory}, Springer, 1997.

% Experimental Physics
\bibitem{pdg2024}
Particle Data Group, \emph{Review of Particle Physics}, 2024.
\url{https://pdg.lbl.gov/}

\bibitem{codata2019}
CODATA, \emph{Recommended Values of Fundamental Constants}, 2019.
\url{https://physics.nist.gov/cuu/Constants/}

\bibitem{newell2018}
D. Newell et al., \emph{The CODATA 2017 Values of h, e, k, and $N_A$}, Metrologia, 2018.
\url{https://doi.org/10.1088/1681-7575/aa950a}

\bibitem{muong2_2023}
Muon g-2 Collaboration, \emph{Measurement of the Anomalous Magnetic Moment of the Muon}, Phys. Rev. Lett., 2023.
\url{https://doi.org/10.1103/PhysRevLett.131.161802}

\bibitem{fermilab2023}
Fermilab, \emph{Muon g-2 Results}, 2023.
\url{https://muon-g-2.fnal.gov/}

\bibitem{atlas2023}
ATLAS Collaboration, \emph{Measurements at the LHC}, 2023.
\url{https://atlas.cern/}

\bibitem{atlas2023higgs}
ATLAS Collaboration, \emph{Higgs Boson Properties}, 2023.
\url{https://atlas.cern/}

\bibitem{cms2023top}
CMS Collaboration, \emph{Top Quark Measurements}, 2023.
\url{https://cms.cern/}

\bibitem{cms2024}
CMS Collaboration, \emph{Heavy Ion Collisions}, 2024.
\url{https://cms.cern/}

\bibitem{alice2023}
ALICE Collaboration, \emph{Quark-Gluon Plasma Studies}, 2023.
\url{https://alice-collaboration.web.cern.ch/}

\bibitem{kasevich2023}
M. Kasevich et al., \emph{Atom Interferometry}, 2023.

\bibitem{ludlow2015}
A. Ludlow et al., \emph{Optical Atomic Clocks}, Rev. Mod. Phys., 2015.
\url{https://doi.org/10.1103/RevModPhys.87.637}

\bibitem{brewer2019}
S. Brewer et al., \emph{Al$^+$ Optical Clock}, Phys. Rev. Lett., 2019.
\url{https://doi.org/10.1103/PhysRevLett.123.033201}

\bibitem{lisa2017}
LISA Collaboration, \emph{LISA Mission}, 2017.
\url{https://www.lisamission.org/}

% Fractal Physics
\bibitem{nottale1993}
L. Nottale, \emph{Fractal Space-Time and Microphysics}, World Scientific, 1993.

\bibitem{elnaschie2004}
M.S. El Naschie, \emph{E-Infinity Theory}, Chaos Solitons Fractals, 2004.

% Philosophy and Foundations
\bibitem{wheeler1990}
J.A. Wheeler, \emph{Information, Physics, Quantum}, 1990.

\bibitem{barbour1999}
J. Barbour, \emph{The End of Time}, Oxford University Press, 1999.

\bibitem{sciama1953}
D. Sciama, \emph{On the Origin of Inertia}, MNRAS, 1953.
\url{https://doi.org/10.1093/mnras/113.1.34}

% String Theory Extensions
\bibitem{becker2007}
K. Becker et al., \emph{String Theory and M-Theory}, Cambridge University Press, 2007.

% Missing References for g-2 Chapter
\bibitem{sm_g2_2025}
Muon g-2 Theory Initiative, \emph{Standard Model Prediction for g-2}, arXiv, 2025.
\url{https://arxiv.org/abs/2006.04822}

\bibitem{mug2_final_2025}
Muon g-2 Collaboration, \emph{Final Report on the Anomalous Magnetic Moment of the Muon}, Fermilab, 2025.
\url{https://muon-g-2.fnal.gov/}

\bibitem{pascher_t0_theory_2025}
J. Pascher, \emph{T0 Theory: Complete Framework}, 2025.
\url{https://github.com/jpascher/T0-Time-Mass-Duality/blob/main/2/pdf/systemEn.pdf}

\bibitem{peskin_schroeder_1995}
M.E. Peskin and D.V. Schroeder, \emph{An Introduction to Quantum Field Theory}, Westview Press, 1995.

\bibitem{parker_somov_2018}
R.H. Parker et al., \emph{Measurement of the Fine-Structure Constant}, Science, 2018.
\url{https://doi.org/10.1126/science.aap7706}

\bibitem{morel_rubidium_2020}
L. Morel et al., \emph{Determination of $\alpha$ from Rubidium Atom Recoil}, Nature, 2020.
\url{https://doi.org/10.1038/s41586-020-2964-7}

\bibitem{aoyama_theory_2020}
T. Aoyama et al., \emph{Theory of the Electron Anomalous Magnetic Moment}, Phys. Rep., 2020.
\url{https://doi.org/10.1016/j.physrep.2020.07.006}

\bibitem{fan_lattice_2023}
X. Fan et al., \emph{Hadronic Contributions from Lattice QCD}, Phys. Rev. D, 2023.

\bibitem{hanneke_electron_2008}
D. Hanneke et al., \emph{New Measurement of the Electron g-2}, Phys. Rev. Lett., 2008.
\url{https://doi.org/10.1103/PhysRevLett.100.120801}

% Additional T0 Theory References
\bibitem{pascher_higgs_connection_2025}
J. Pascher, \emph{Higgs Connection in T0 Theory}, 2025.
\url{https://github.com/jpascher/T0-Time-Mass-Duality/blob/main/2/pdf/T0_Energie_En.pdf}

\bibitem{T0_SI}
J. Pascher, \emph{T0 Theory and SI Units}, 2025.
\url{https://github.com/jpascher/T0-Time-Mass-Duality/blob/main/2/pdf/T0_SI_En.pdf}

\bibitem{T0_gravitational_constant}
J. Pascher, \emph{Gravitational Constant in T0 Framework}, 2025.
\url{https://github.com/jpascher/T0-Time-Mass-Duality/blob/main/2/pdf/T0_Gravitationskonstante_En.pdf}

\bibitem{T0_fine_structure}
J. Pascher, \emph{Fine Structure Constant Analysis}, 2025.
\url{https://github.com/jpascher/T0-Time-Mass-Duality/blob/main/2/pdf/T0_Feinstruktur_En.pdf}

\bibitem{bell_muon}
J.S. Bell, \emph{Muon Studies}, 1966.

\bibitem{QFT_T0}
J. Pascher, \emph{Quantum Field Theory in T0}, 2025.
\url{https://github.com/jpascher/T0-Time-Mass-Duality/blob/main/2/pdf/QFT_En.pdf}

\bibitem{planck2018}
Planck Collaboration, \emph{Planck 2018 Results}, A\&A, 2018.
\url{https://doi.org/10.1051/0004-6361/201833910}

\bibitem{pascher:t0_foundations}
J. Pascher, \emph{T0 Theory Foundations}, 2025.
\url{https://github.com/jpascher/T0-Time-Mass-Duality/blob/main/2/pdf/T0_Grundlagen_En.pdf}

\bibitem{pascher:geometric_formalism}
J. Pascher, \emph{Geometric Formalism in T0}, 2025.
\url{https://github.com/jpascher/T0-Time-Mass-Duality/blob/main/2/pdf/T0_Geometrische_Kosmologie_En.pdf}

\bibitem{riess2019}
A. Riess et al., \emph{Hubble Constant Measurements}, ApJ, 2019.
\url{https://doi.org/10.3847/1538-4357/ab1422}

\bibitem{t0_kosmologie}
J. Pascher, \emph{T0 Kosmologie}, 2025.
\url{https://github.com/jpascher/T0-Time-Mass-Duality/blob/main/2/pdf/T0_Kosmologie_En.pdf}

\bibitem{hossenfelder_single_clock_video}
S. Hossenfelder, \emph{Single Clock Video}, YouTube, 2025.
\url{https://www.youtube.com/c/SabineHossenfelder}

\bibitem{video2025}
Various, \emph{Video References}, 2025.

\bibitem{unnikrishnan2004}
C.S. Unnikrishnan, \emph{Gravity Studies}, 2004.

\bibitem{peratt1992}
A. Peratt, \emph{Plasma Cosmology}, 1992.
\url{https://github.com/jpascher/T0-Time-Mass-Duality/blob/main/2/pdf/T0_peratt_En.pdf}

\bibitem{T0_tm_erweiterung}
J. Pascher, \emph{T0 Time-Mass Extension}, 2025.
\url{https://github.com/jpascher/T0-Time-Mass-Duality/blob/main/2/pdf/T0_tm-erweiterung-x6_En.pdf}

\bibitem{T0_g2_erweiterung}
J. Pascher, \emph{T0 g-2 Extension}, 2025.
\url{https://github.com/jpascher/T0-Time-Mass-Duality/blob/main/2/pdf/T0_g2-erweiterung-4_En.pdf}

\bibitem{T0_netze_en}
J. Pascher, \emph{T0 Networks}, 2025.
\url{https://github.com/jpascher/T0-Time-Mass-Duality/blob/main/2/pdf/T0_netze_En.pdf}

\bibitem{Adams1925}
W. Adams, \emph{Gravitational Redshift}, 1925.
\url{https://doi.org/10.1073/pnas.11.7.382}

\bibitem{Ashby2003}
N. Ashby, \emph{Relativity in GPS}, Living Rev. Rel., 2003.
\url{https://doi.org/10.12942/lrr-2003-1}

\bibitem{Bertotti2003}
B. Bertotti et al., \emph{Cassini Doppler Test}, Nature, 2003.
\url{https://doi.org/10.1038/nature01997}

\bibitem{Bolton2008}
A. Bolton et al., \emph{Gravitational Lensing}, 2008.

\bibitem{Born2013}
M. Born, \emph{Einstein's Theory of Relativity}, Dover, 2013.

\bibitem{Brans1961}
C. Brans and R.H. Dicke, \emph{Mach's Principle}, Phys. Rev., 1961.
\url{https://doi.org/10.1103/PhysRev.124.925}

\bibitem{Dirac1927}
P.A.M. Dirac, \emph{Quantum Mechanics}, Proc. Roy. Soc., 1927.
\url{https://doi.org/10.1098/rspa.1927.0039}

\bibitem{Duhem1906}
P. Duhem, \emph{Theory of Physics}, 1906.

\bibitem{Einstein1905}
A. Einstein, \emph{Special Relativity}, Ann. Phys., 1905.
\url{https://doi.org/10.1002/andp.19053221004}

\bibitem{Feynman2006}
R. Feynman, \emph{QED: The Strange Theory of Light and Matter}, 2006.

\bibitem{Griffiths2017}
D. Griffiths, \emph{Introduction to Quantum Mechanics}, 2017.

\bibitem{Jackson1999}
J.D. Jackson, \emph{Classical Electrodynamics}, 1999.

\bibitem{Kaluza1921}
T. Kaluza, \emph{Five-Dimensional Theory}, 1921.

\bibitem{Klein1926}
O. Klein, \emph{Quantum Theory and Relativity}, 1926.

\bibitem{Kuhn1962}
T. Kuhn, \emph{Structure of Scientific Revolutions}, 1962.

\bibitem{Kuhn1977}
T. Kuhn, \emph{Essential Tension}, 1977.

\bibitem{Ludlow2015}
A. Ludlow et al., \emph{Optical Atomic Clocks}, Rev. Mod. Phys., 2015.
\url{https://doi.org/10.1103/RevModPhys.87.637}

\bibitem{Maxwell1873}
J.C. Maxwell, \emph{Treatise on Electricity and Magnetism}, 1873.

\bibitem{McGaugh2016}
S. McGaugh et al., \emph{Radial Acceleration Relation}, Phys. Rev. Lett., 2016.
\url{https://doi.org/10.1103/PhysRevLett.117.201101}

\bibitem{Mohr2016}
P. Mohr et al., \emph{CODATA Values}, Rev. Mod. Phys., 2016.
\url{https://doi.org/10.1103/RevModPhys.88.035009}

\bibitem{PDG2020}
Particle Data Group, \emph{Review of Particle Physics}, Prog. Theor. Exp. Phys., 2020.
\url{https://pdg.lbl.gov/}

\bibitem{Parker2018}
R. Parker et al., \emph{Measurement of $\alpha$}, Science, 2018.
\url{https://doi.org/10.1126/science.aap7706}

\bibitem{Peskin1995}
M. Peskin and D. Schroeder, \emph{QFT}, 1995.

\bibitem{Planck1900}
M. Planck, \emph{Quantum Theory}, 1900.

\bibitem{Planck2020}
Planck Collaboration, \emph{Planck 2020 Results}, 2020.
\url{https://doi.org/10.1051/0004-6361/201833910}

\bibitem{Poincare1905}
H. Poincaré, \emph{Dynamics of the Electron}, 1905.

\bibitem{Pound1960}
R.V. Pound and G.A. Rebka, \emph{Gravitational Redshift}, Phys. Rev. Lett., 1960.
\url{https://doi.org/10.1103/PhysRevLett.4.337}

\bibitem{Quine1951}
W.V. Quine, \emph{Two Dogmas of Empiricism}, 1951.

\bibitem{Quinn2013}
T. Quinn et al., \emph{Gravitational Constant}, 2013.
\url{https://doi.org/10.1103/PhysRevLett.111.101102}

\bibitem{Randall1999}
L. Randall and R. Sundrum, \emph{Extra Dimensions}, Phys. Rev. Lett., 1999.
\url{https://doi.org/10.1103/PhysRevLett.83.3370}

\bibitem{Riess1998}
A. Riess et al., \emph{Type Ia Supernovae}, AJ, 1998.
\url{https://doi.org/10.1086/300499}

\bibitem{Shapiro1971}
I. Shapiro et al., \emph{Time Delay Test}, Phys. Rev. Lett., 1971.
\url{https://doi.org/10.1103/PhysRevLett.26.1132}

\bibitem{Sommerfeld1916}
A. Sommerfeld, \emph{Fine Structure}, 1916.

\bibitem{Suyu2017}
S. Suyu et al., \emph{Time Delay Cosmography}, MNRAS, 2017.
\url{https://doi.org/10.1093/mnras/stx483}

\bibitem{T0Theory}
J. Pascher, \emph{T0 Theory}, 2025.
\url{https://github.com/jpascher/T0-Time-Mass-Duality/blob/main/2/pdf/systemEn.pdf}

\bibitem{T0_Feinstruktur}
J. Pascher, \emph{Fine Structure in T0}, 2025.
\url{https://github.com/jpascher/T0-Time-Mass-Duality/blob/main/2/pdf/T0_Feinstruktur_En.pdf}

\bibitem{Uzan2003}
J.-P. Uzan, \emph{Constants Variation}, Rev. Mod. Phys., 2003.
\url{https://doi.org/10.1103/RevModPhys.75.403}

\bibitem{Webb2001}
J.K. Webb et al., \emph{Fine Structure Constant}, Phys. Rev. Lett., 2001.
\url{https://doi.org/10.1103/PhysRevLett.87.091301}

\bibitem{Weinberg1979}
S. Weinberg, \emph{Cosmological Constant}, Rev. Mod. Phys., 1979.

\bibitem{Weinberg1989}
S. Weinberg, \emph{Cosmological Constant Problem}, 1989.
\url{https://doi.org/10.1103/RevModPhys.61.1}

\bibitem{Weinberg1995}
S. Weinberg, \emph{Quantum Theory of Fields}, 1995.

\bibitem{Will2014}
C. Will, \emph{Theory and Experiment in Gravitational Physics}, 2014.
\url{https://doi.org/10.12942/lrr-2014-4}

\bibitem{dirac_principles}
P.A.M. Dirac, \emph{Principles of Quantum Mechanics}, 1930.

\bibitem{einstein_1917}
A. Einstein, \emph{Cosmological Considerations}, 1917.

\bibitem{jwst_early}
JWST Collaboration, \emph{Early Universe Observations}, 2023.
\url{https://www.jwst.nasa.gov/}

\bibitem{katrin_2022}
KATRIN Collaboration, \emph{Neutrino Mass}, 2022.
\url{https://doi.org/10.1038/s41567-021-01463-1}

\bibitem{pascher:fundamentals}
J. Pascher, \emph{T0 Fundamentals}, 2025.
\url{https://github.com/jpascher/T0-Time-Mass-Duality/blob/main/2/pdf/T0_Grundlagen_En.pdf}

\bibitem{pascher:g2_rev9}
J. Pascher, \emph{g-2 Analysis Rev9}, 2025.
\url{https://github.com/jpascher/T0-Time-Mass-Duality/blob/main/2/pdf/T0_Anomale-g2-9_En.pdf}

\bibitem{pascher:ml_addendum}
J. Pascher, \emph{ML Addendum}, 2025.
\url{https://github.com/jpascher/T0-Time-Mass-Duality/blob/main/2/pdf/T0-QFT-ML_Addendum_En.pdf}

\bibitem{pascher_beta_derivation_2025}
J. Pascher, \emph{Beta Derivation}, 2025.
\url{https://github.com/jpascher/T0-Time-Mass-Duality/blob/main/2/pdf/DerivationVonBetaEn.pdf}

\bibitem{pascher_cmb_en}
J. Pascher, \emph{CMB Analysis in T0}, 2025.
\url{https://github.com/jpascher/T0-Time-Mass-Duality/blob/main/2/pdf/Zwei-Dipole-CMB_En.pdf}

\bibitem{pascher_cosmos_en}
J. Pascher, \emph{Cosmos in T0 Theory}, 2025.
\url{https://github.com/jpascher/T0-Time-Mass-Duality/blob/main/2/pdf/cosmic_En.pdf}

\bibitem{pascher_derivation_beta_2025}
J. Pascher, \emph{Derivation of Beta}, 2025.
\url{https://github.com/jpascher/T0-Time-Mass-Duality/blob/main/2/pdf/DerivationVonBetaEn.pdf}

\bibitem{pascher_gravitation_en}
J. Pascher, \emph{Gravitation in T0}, 2025.
\url{https://github.com/jpascher/T0-Time-Mass-Duality/blob/main/2/pdf/gravitationskonstante_En.pdf}

\bibitem{pascher_lagrangian_2025}
J. Pascher, \emph{Lagrangian in T0}, 2025.
\url{https://github.com/jpascher/T0-Time-Mass-Duality/blob/main/2/pdf/T0_lagrndian_En.pdf}

\bibitem{pascher_lagrangian_en}
J. Pascher, \emph{Lagrangian Framework}, 2025.
\url{https://github.com/jpascher/T0-Time-Mass-Duality/blob/main/2/pdf/LagrandianVergleichEn.pdf}

\bibitem{pascher_lagrangian_extended_2025}
J. Pascher, \emph{Extended Lagrangian Formalism}, 2025.
\url{https://github.com/jpascher/T0-Time-Mass-Duality/blob/main/2/pdf/T0_lagrndian_En.pdf}

\bibitem{pascher_mathematical_structure_2025}
J. Pascher, \emph{Mathematical Structure of T0 Theory}, 2025.
\url{https://github.com/jpascher/T0-Time-Mass-Duality/blob/main/2/pdf/Mathematische_struktur_En.pdf}

\bibitem{pascher_muon_g2_2025}
J. Pascher, \emph{Muon g-2 in T0}, 2025.
\url{https://github.com/jpascher/T0-Time-Mass-Duality/blob/main/2/pdf/T0_Anomale-g2-9_En.pdf}

\bibitem{pascher_pragmatic_2025}
J. Pascher, \emph{Pragmatic Approach}, 2025.

\bibitem{pascher_t0_energy_2025}
J. Pascher, \emph{T0 Energy Formalism}, 2025.
\url{https://github.com/jpascher/T0-Time-Mass-Duality/blob/main/2/pdf/T0-Energie_En.pdf}

\bibitem{pascher_unified_2025}
J. Pascher, \emph{Unified T0 Theory}, 2025.
\url{https://github.com/jpascher/T0-Time-Mass-Duality/blob/main/2/pdf/T0_unified_report.pdf}

\bibitem{sciencedaily2025}
Science Daily, \emph{Physics News}, 2025.
\url{https://www.sciencedaily.com/}

\bibitem{weinberg_1989}
S. Weinberg, \emph{The Cosmological Constant Problem}, Rev. Mod. Phys., 1989.
\url{https://doi.org/10.1103/RevModPhys.61.1}

\bibitem{wiki_bell}
Wikipedia, \emph{Bell's Theorem}, 2025.
\url{https://en.wikipedia.org/wiki/Bell\%27s_theorem}

\bibitem{vanFraassen1980}
B. van Fraassen, \emph{The Scientific Image}, Oxford University Press, 1980.

\bibitem{terrell_single_clock_nature_2024}
J. Terrell, \emph{Single Clock Nature}, Nature, 2024.

% Additional T0 Documents
\bibitem{137_doc}
J. Pascher, \emph{The Number 137 in T0 Theory}, 2025.
\url{https://github.com/jpascher/T0-Time-Mass-Duality/blob/main/2/pdf/137_En.pdf}

\bibitem{ampere_low}
J. Pascher, \emph{Ampere's Law in T0}, 2025.
\url{https://github.com/jpascher/T0-Time-Mass-Duality/blob/main/2/pdf/Amper_Low_En.pdf}

\bibitem{bell_theorem}
J. Pascher, \emph{Bell's Theorem in T0}, 2025.
\url{https://github.com/jpascher/T0-Time-Mass-Duality/blob/main/2/pdf/Bell_En.pdf}

\bibitem{bewegungsenergie}
J. Pascher, \emph{Kinetic Energy in T0}, 2025.
\url{https://github.com/jpascher/T0-Time-Mass-Duality/blob/main/2/pdf/Bewegungsenergie_En.pdf}

\bibitem{emc2}
J. Pascher, \emph{E=mc² in T0 Framework}, 2025.
\url{https://github.com/jpascher/T0-Time-Mass-Duality/blob/main/2/pdf/E-mc2_En.pdf}

\bibitem{formeln_energiebasiert}
J. Pascher, \emph{Energy-Based Formulas}, 2025.
\url{https://github.com/jpascher/T0-Time-Mass-Duality/blob/main/2/pdf/Formeln_Energiebasiert_En.pdf}

\bibitem{hannah}
J. Pascher, \emph{Hannah Document}, 2025.
\url{https://github.com/jpascher/T0-Time-Mass-Duality/blob/main/2/pdf/Hannah_En.pdf}

\bibitem{ho_doc}
J. Pascher, \emph{H0 Analysis}, 2025.
\url{https://github.com/jpascher/T0-Time-Mass-Duality/blob/main/2/pdf/Ho_En.pdf}

\bibitem{markov}
J. Pascher, \emph{Markov Processes in T0}, 2025.
\url{https://github.com/jpascher/T0-Time-Mass-Duality/blob/main/2/pdf/Markov_En.pdf}

\bibitem{elimination_mass}
J. Pascher, \emph{Elimination of Mass}, 2025.
\url{https://github.com/jpascher/T0-Time-Mass-Duality/blob/main/2/pdf/EliminationOfMassEn.pdf}

\bibitem{elimination_mass_dirac}
J. Pascher, \emph{Dirac Equation Mass Elimination}, 2025.
\url{https://github.com/jpascher/T0-Time-Mass-Duality/blob/main/2/pdf/Elimination_Of_Mass_Dirac_TabelleEn.pdf}

\bibitem{feinstrukturkonstante}
J. Pascher, \emph{Fine Structure Constant}, 2025.
\url{https://github.com/jpascher/T0-Time-Mass-Duality/blob/main/2/pdf/FeinstrukturkonstanteEn.pdf}

\bibitem{neutrino_formel}
J. Pascher, \emph{Neutrino Formula}, 2025.
\url{https://github.com/jpascher/T0-Time-Mass-Duality/blob/main/2/pdf/neutrino-Formel_En.pdf}

\bibitem{neutrinos}
J. Pascher, \emph{Neutrinos in T0}, 2025.
\url{https://github.com/jpascher/T0-Time-Mass-Duality/blob/main/2/pdf/T0_Neutrinos_En.pdf}

\bibitem{koide_formel}
J. Pascher, \emph{Koide Formula in T0}, 2025.
\url{https://github.com/jpascher/T0-Time-Mass-Duality/blob/main/2/pdf/T0_koide-formel-3_En.pdf}

\bibitem{teilchenmassen}
J. Pascher, \emph{Particle Masses}, 2025.
\url{https://github.com/jpascher/T0-Time-Mass-Duality/blob/main/2/pdf/Teilchenmassen_En.pdf}

\bibitem{t0_teilchenmassen}
J. Pascher, \emph{T0 Particle Masses}, 2025.
\url{https://github.com/jpascher/T0-Time-Mass-Duality/blob/main/2/pdf/T0_Teilchenmassen_En.pdf}

\bibitem{penrose_doc}
J. Pascher, \emph{Penrose Analysis in T0}, 2025.
\url{https://github.com/jpascher/T0-Time-Mass-Duality/blob/main/2/pdf/T0_penrose_En.pdf}

\bibitem{photonenchip}
J. Pascher, \emph{Photon Chip Implementation}, 2025.
\url{https://github.com/jpascher/T0-Time-Mass-Duality/blob/main/2/pdf/T0_photonenchip-china_En.pdf}

\bibitem{threeclock}
J. Pascher, \emph{Three Clock Experiment}, 2025.
\url{https://github.com/jpascher/T0-Time-Mass-Duality/blob/main/2/pdf/T0_threeclock_En.pdf}

\bibitem{redshift_deflection}
J. Pascher, \emph{Redshift and Deflection}, 2025.
\url{https://github.com/jpascher/T0-Time-Mass-Duality/blob/main/2/pdf/redshift_deflection_En.pdf}

\bibitem{scheinbar_instantan}
J. Pascher, \emph{Apparent Instantaneity}, 2025.
\url{https://github.com/jpascher/T0-Time-Mass-Duality/blob/main/2/pdf/scheinbar_instantan_En.pdf}

\bibitem{universale_ableitung}
J. Pascher, \emph{Universal Derivation}, 2025.
\url{https://github.com/jpascher/T0-Time-Mass-Duality/blob/main/2/pdf/universale-ableitung_En.pdf}

\bibitem{xi_parameter}
J. Pascher, \emph{Xi Parameter for Particles}, 2025.
\url{https://github.com/jpascher/T0-Time-Mass-Duality/blob/main/2/pdf/xi_parmater_partikel_En.pdf}

\bibitem{xi_ursprung}
J. Pascher, \emph{Origin of Xi}, 2025.
\url{https://github.com/jpascher/T0-Time-Mass-Duality/blob/main/2/pdf/T0_xi_ursprung_En.pdf}

\bibitem{zeit}
J. Pascher, \emph{Time in T0 Theory}, 2025.
\url{https://github.com/jpascher/T0-Time-Mass-Duality/blob/main/2/pdf/Zeit_En.pdf}

\bibitem{zeit_konstant}
J. Pascher, \emph{Time Constant}, 2025.
\url{https://github.com/jpascher/T0-Time-Mass-Duality/blob/main/2/pdf/Zeit-konstant_En.pdf}

\bibitem{zusammenfassung}
J. Pascher, \emph{Summary of T0 Theory}, 2025.
\url{https://github.com/jpascher/T0-Time-Mass-Duality/blob/main/2/pdf/Zusammenfassung_En.pdf}

\bibitem{rsa}
J. Pascher, \emph{RSA in T0 Framework}, 2025.
\url{https://github.com/jpascher/T0-Time-Mass-Duality/blob/main/2/pdf/RSA_En.pdf}

\bibitem{qat}
J. Pascher, \emph{Quantum Atomic Theory}, 2025.
\url{https://github.com/jpascher/T0-Time-Mass-Duality/blob/main/2/pdf/T0_QAT_En.pdf}

\bibitem{qm_qft_rt}
J. Pascher, \emph{QM, QFT and RT Unification}, 2025.
\url{https://github.com/jpascher/T0-Time-Mass-Duality/blob/main/2/pdf/T0_QM-QFT-RT_En.pdf}

\bibitem{qm_optimierung}
J. Pascher, \emph{QM Optimization}, 2025.
\url{https://github.com/jpascher/T0-Time-Mass-Duality/blob/main/2/pdf/T0_QM-optimierung_En.pdf}

\bibitem{vollstaendige_berechnungen}
J. Pascher, \emph{Complete Calculations}, 2025.
\url{https://github.com/jpascher/T0-Time-Mass-Duality/blob/main/2/pdf/T0_Vollstaendige_Berchnungen_En.pdf}

\bibitem{synergetics}
J. Pascher, \emph{T0 Theory vs Synergetics}, 2025.
\url{https://github.com/jpascher/T0-Time-Mass-Duality/blob/main/2/pdf/T0-Theory-vs-Synergetics_En.pdf}

\bibitem{modell_uebersicht}
J. Pascher, \emph{T0 Model Overview}, 2025.
\url{https://github.com/jpascher/T0-Time-Mass-Duality/blob/main/2/pdf/T0_Modell_Uebersicht_En.pdf}

\bibitem{mnras_widerlegung}
J. Pascher, \emph{MNRAS Analysis}, 2025.
\url{https://github.com/jpascher/T0-Time-Mass-Duality/blob/main/2/pdf/T0_Analyse_MNRAS_Widerlegung_En.pdf}

\bibitem{anomale_magnetische_momente}
J. Pascher, \emph{Anomalous Magnetic Moments}, 2025.
\url{https://github.com/jpascher/T0-Time-Mass-Duality/blob/main/2/pdf/T0_Anomale_Magnetische_Momente_En.pdf}

\bibitem{sieben_fragen}
J. Pascher, \emph{Seven Questions in T0}, 2025.
\url{https://github.com/jpascher/T0-Time-Mass-Duality/blob/main/2/pdf/T0_7-fragen-3_En.pdf}

\bibitem{detailierte_leptonen}
J. Pascher, \emph{Detailed Lepton Anomaly}, 2025.
\url{https://github.com/jpascher/T0-Time-Mass-Duality/blob/main/2/pdf/detailierte_formel_leptonen_anemal_En.pdf}

\bibitem{parameterherleitung}
J. Pascher, \emph{Parameter Derivation}, 2025.
\url{https://github.com/jpascher/T0-Time-Mass-Duality/blob/main/2/pdf/parameterherleitung_En.pdf}

\bibitem{verhaeltnis_absolut}
J. Pascher, \emph{Absolute Ratios in T0}, 2025.
\url{https://github.com/jpascher/T0-Time-Mass-Duality/blob/main/2/pdf/T0_verhaeltnis-absolut_En.pdf}

\bibitem{xi_und_e}
J. Pascher, \emph{Xi and Energy}, 2025.
\url{https://github.com/jpascher/T0-Time-Mass-Duality/blob/main/2/pdf/T0_xi-und-e_En.pdf}

\bibitem{umkehrung}
J. Pascher, \emph{Inversion in T0}, 2025.
\url{https://github.com/jpascher/T0-Time-Mass-Duality/blob/main/2/pdf/T0_umkehrung_En.pdf}

\bibitem{esm_analysis}
J. Pascher, \emph{T0 vs ESM Conceptual Analysis}, 2025.
\url{https://github.com/jpascher/T0-Time-Mass-Duality/blob/main/2/pdf/T0vsESM_ConceptualAnalysis_En.pdf}

\end{thebibliography}

\end{document}
