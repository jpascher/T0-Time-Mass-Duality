% Standalone document: redshift_deflection_En
% Uses shared T0 header
% T0 Standalone Header - German Version
% Gemeinsamer Header für alle deutschen Standalone-Dokumente

\documentclass[12pt,a4paper]{article}
\usepackage[utf8]{inputenc}
\usepackage[T1]{fontenc}
\usepackage[ngerman]{babel}
\usepackage{lmodern}

% Mathematics
\usepackage{amsmath,amssymb,amsthm}
\usepackage{physics}
\usepackage{siunitx}

% Layout
\usepackage[left=2.5cm,right=2.5cm,top=2.5cm,bottom=2.5cm,headheight=15pt]{geometry}
\usepackage{fancyhdr}
\usepackage{titlesec}

% Tables and Graphics
\usepackage{booktabs}
\usepackage{array}
\usepackage{longtable}
\usepackage{graphicx}
\usepackage{tikz}
\usetikzlibrary{arrows.meta,positioning,shapes.geometric}

% Colors and Boxes
\usepackage{xcolor}
\usepackage[most]{tcolorbox}
\usepackage{mdframed}

% Additional packages
\usepackage{enumitem}
\usepackage{float}
\usepackage{caption}
\usepackage{subcaption}
\usepackage{multirow}
\usepackage{colortbl}
\usepackage{pdflscape}
\usepackage{algorithm}
\usepackage{algpseudocode}
\usepackage{listings}
\usepackage{hyperref}

% Define colors
\definecolor{t0blue}{RGB}{0,51,102}
\definecolor{t0green}{RGB}{0,102,51}
\definecolor{t0red}{RGB}{153,0,0}
\definecolor{deepblue}{RGB}{0,51,102}
\definecolor{deepgreen}{RGB}{0,102,51}
\definecolor{deepred}{RGB}{153,0,0}
\definecolor{boxgray}{RGB}{240,240,240}
\definecolor{t0yellow}{RGB}{255,200,0}
\definecolor{boxblue}{RGB}{230,240,255}
\definecolor{boxgreen}{RGB}{230,255,230}
\definecolor{boxorange}{RGB}{255,240,230}
\definecolor{boxyellow}{RGB}{255,255,230}

% Custom tcolorbox environments
\newtcolorbox{fundamental}[1][]{
  colback=blue!5!white,
  colframe=blue!75!black,
  title=#1,
  fonttitle=\bfseries,
  breakable
}

\newtcolorbox{derivation}[1][]{
  colback=green!5!white,
  colframe=green!75!black,
  title=#1,
  fonttitle=\bfseries,
  breakable
}

\newtcolorbox{result}[1][]{
  colback=orange!5!white,
  colframe=orange!75!black,
  title=#1,
  fonttitle=\bfseries,
  breakable
}

\newtcolorbox{summary}[1][]{
  colback=gray!10!white,
  colframe=gray!75!black,
  title=#1,
  fonttitle=\bfseries,
  breakable
}

\newtcolorbox{comparison}[1][]{
  colback=purple!5!white,
  colframe=purple!75!black,
  title=#1,
  fonttitle=\bfseries,
  breakable
}

\newtcolorbox{relation}[1][]{
  colback=cyan!5!white,
  colframe=cyan!75!black,
  title=#1,
  fonttitle=\bfseries,
  breakable
}

\newtcolorbox{principle}[1][]{
  colback=yellow!5!white,
  colframe=yellow!75!black,
  title=#1,
  fonttitle=\bfseries,
  breakable
}

\newtcolorbox{insight}[1][]{colback=blue!5,colframe=t0blue,title={#1},fonttitle=\bfseries,breakable}
\newtcolorbox{discovery}[1][]{colback=green!5,colframe=t0green,title={#1},fonttitle=\bfseries,breakable}
\newtcolorbox{newperspective}[1][]{colback=yellow!5,colframe=orange,title={#1},fonttitle=\bfseries,breakable}
\newtcolorbox{revelation}[1][]{colback=red!5,colframe=t0red,title={#1},fonttitle=\bfseries,breakable}
\newtcolorbox{keypoint}[1][]{colback=blue!5,colframe=t0blue,title={#1},fonttitle=\bfseries,breakable}
\newtcolorbox{evidence}[1][]{colback=green!5,colframe=t0green,title={#1},fonttitle=\bfseries,breakable}
\newtcolorbox{conclusion}[1][]{colback=gray!5,colframe=gray,title={#1},fonttitle=\bfseries,breakable}
\newtcolorbox{significance}[1][]{colback=yellow!5,colframe=orange,title={#1},fonttitle=\bfseries,breakable}
\newtcolorbox{philosophical}[1][]{colback=purple!5,colframe=purple,title={#1},fonttitle=\bfseries,breakable}
\newtcolorbox{implication}[1][]{colback=cyan!5,colframe=cyan,title={#1},fonttitle=\bfseries,breakable}
\newtcolorbox{perspective}[1][]{colback=blue!5,colframe=t0blue,title={#1},fonttitle=\bfseries,breakable}
\newtcolorbox{revolutionary}[1][]{colback=red!5,colframe=t0red,title={#1},fonttitle=\bfseries,breakable}
\newtcolorbox{technical}[1][]{colback=gray!5,colframe=gray!75!black,title={#1},fonttitle=\bfseries,breakable}
\newtcolorbox{notation}[1][]{colback=yellow!5,colframe=yellow!75!black,title={#1},fonttitle=\bfseries,breakable}

% Theorem environments
\newtheorem{theorem}{Satz}[section]
\newtheorem{lemma}[theorem]{Lemma}
\newtheorem{corollary}[theorem]{Korollar}
\newtheorem{proposition}[theorem]{Proposition}
\newtheorem{definition}[theorem]{Definition}
\newtheorem{example}[theorem]{Beispiel}
\newtheorem{remark}[theorem]{Bemerkung}
\newtheorem{note}[theorem]{Anmerkung}

% Additional environments
\newenvironment{treatise}{\begin{quote}}{\end{quote}}
\newenvironment{gemeinsam}{\begin{quote}}{\end{quote}}
\newenvironment{vergleich}{\begin{quote}}{\end{quote}}
\newenvironment{vorteil}{\begin{quote}}{\end{quote}}
\newenvironment{quantum}{\begin{quote}}{\end{quote}}

% T0-specific commands
\newcommand{\Tzero}{T$_0$}
\newcommand{\xipar}{\xi}
\newcommand{\Tfield}{T}
\newcommand{\Efield}{\mathcal{E}}
\newcommand{\meff}{m_{\text{eff}}}
\newcommand{\Eabs}{E_{\text{abs}}}
\newcommand{\taupar}{\tau}

% Header setup
\pagestyle{fancy}
\fancyhf{}
\fancyhead[L]{\leftmark}
\fancyhead[R]{\thepage}
\renewcommand{\headrulewidth}{0.4pt}

% Hyperref setup
\hypersetup{
    colorlinks=true,
    linkcolor=blue,
    filecolor=magenta,
    urlcolor=cyan,
    citecolor=blue,
    pdftitle={T0 Theory Document},
    pdfauthor={Johann Pascher}
}

% German quotation marks
%\newcommand{\dq}[1]{\glqq{}#1\grqq{}}


\title{Redshift and Deflection}
\author{Johann Pascher}
\date{2025}

\begin{document}

\maketitle

\chapter{Redshift and Deflection}

	
	
	\begin{abstract}
		The T0 Modell explains kosmologisch Rotverschiebung through $\xi$-Feld Energie loss during Photon propagation, without requiring spatial Expansion or Entfernung Messungen. This Mechanismus predicts a Wellenlänge-dependent Rotverschiebung $z \propto \lambda$ das can be tested with spectroscopic Beobachtungen of cosmic objects. Using the universal Konstante $\xiconst$ and gemessen masses of astronomical objects, the theory provides Modell-independent tests distinguishable from Standard Kosmologie. The $\xi$-Feld auch explains the cosmic microwave background Temperatur ($T_{\text{CMB}} = 2.7255$ K) in a static, eternally existing Universum, as detailed in \cite{pascher_cmb_en,pascher_cosmos_en}.
	\end{abstract}
	
	\newpage
	
	\section{Einleitung}
	
	\subsection{Universal $\xi$-Constant}
	
	The T0-theory is basierend auf a single fundamental Konstante \cite{pascher_lagrangian_en}:
	\begin{equation}
		\boxed{\xiconst}
	\end{equation}
	
	This Wert arises from geometrisch considerations and determines alle fundamental Wechselwirkungen in the Universum \cite{pascher_gravitation_en}. The geometrisch origin stems from the Verhältnis of Charakteristik Skalen in the Universum, connecting Quanten Mechanik to Kosmologie through a single Parameter.
	
	\subsection{$\xi$-Field Structure}
	
	The $\xi$-Feld permeates the entire Universum and manifests in three fundamental forms:
	\begin{enumerate}
		\item \textbf{Cosmic Microwave Hintergrund (CMB)}: Free $\xi$-Feld Strahlung at $T = 2.7255$ K
		\item \textbf{Casimir Vacuum}: Geometrically constrained $\xi$-Feld zwischen conducting plates
		\item \textbf{Gravitational Interaction}: $\xi$-Feld Kopplung to Materie determines $G$
	\end{enumerate}
	
	The Zusammenhang zwischen diese manifestations is given by:
	\begin{equation}
		\frac{|\rho_{\text{Casimir}}|}{\rho_{\text{CMB}}} = \frac{\pi^2}{240 \xi} = \frac{\pi^2 \times 10^4}{320} \approx 308
	\end{equation}
	
	\section{Energy Loss Mechanism}
	
	\subsection{Photon-$\xi$-Field Interaction}
	
	\begin{Prinzip}[$\xi$-Field Energy Loss]
		Photons propagating through the omnipresent $\xi$-Feld lose Energie gemäß:
		\begin{equation}
			\frac{dE}{dx} = -\xi \cdot \xicoupling \cdot E
		\end{equation}
		wo $\xicoupling$ is the Energie-dependent Kopplung Funktion.
	\end{Prinzip}
	
	For the linear Kopplung case:
	\begin{equation}
		f\left(\frac{E}{\Exi}\right) = \frac{E}{\Exi}
	\end{equation}
	
	This yields the simplified Energie loss Gleichung:
	\begin{equation}
		\frac{dE}{dx} = -\frac{\xi E^2}{\Exi}
	\end{equation}
	
	\subsection{Energy-to-Wavelength Conversion}
	
	Since $E = \frac{hc}{\lambda}$ (or $E = \frac{1}{\lambda}$ in natural Einheiten, $\hbar = c = 1$), we can express the Energie loss in Bezug auf Wellenlänge. Substituting $E = \frac{1}{\lambda}$:
	\begin{equation}
		\frac{d(1/\lambda)}{dx} = -\frac{\xi}{\Exi} \cdot \frac{1}{\lambda^2}
	\end{equation}
	
	Rearranging for Wellenlänge evolution:
	\begin{equation}
		\frac{d\lambda}{dx} = \frac{\xi \lambda^2}{\Exi}
	\end{equation}
	
	\section{Redshift Formula Derivation}
	
	\subsection{Integration for Small $\xi$-Effects}
	
	For the Wellenlänge evolution Gleichung:
	\begin{equation}
		\frac{d\lambda}{dx} = \frac{\xi \lambda^2}{\Exi}
	\end{equation}
	
	Separating Variablen and integrating:
	\begin{equation}
		\int_{\lambdazero}^{\lambda} \frac{d\lambda'}{\lambda'^2} = \frac{\xi}{\Exi} \int_0^x dx'
	\end{equation}
	
	This yields:
	\begin{equation}
		\frac{1}{\lambdazero} - \frac{1}{\lambda} = \frac{\xi x}{\Exi}
	\end{equation}
	
	Solving for the beobachtet Wellenlänge:
	\begin{equation}
		\lambda = \frac{\lambdazero}{1 - \frac{\xi x \lambdazero}{\Exi}}
	\end{equation}
	
	\subsection{Redshift Definition and Formula}
	
	\begin{Formel}
		Redshift definition:
		\begin{equation}
			z = \frac{\lambda_{\text{observed}} - \lambda_{\text{emitted}}}{\lambda_{\text{emitted}}} = \frac{\lambda}{\lambdazero} - 1
		\end{equation}
	\end{Formel}
	
	For klein $\xi$-Effekte wo $\frac{\xi x \lambdazero}{\Exi} \ll 1$, we can expand:
	\begin{equation}
		z \approx \frac{\xi x \lambdazero}{\Exi} = \frac{\xi x}{\Exi / (\hbar c)} \cdot \lambdazero \quad (\text{in conventional units})
	\end{equation}
	
	\begin{important}
		\textbf{Key T0 Prediction: Wavelength-Dependent Redshift}
		\begin{equation}
			\boxed{z(\lambdazero) = \frac{\xi x}{\Exi} \cdot \lambdazero \quad (\text{natural units, } \hbar = c = 1)}
		\end{equation}
		This Wellenlänge dependence is the KEY DISTINGUISHING FEATURE from Standard Kosmologie:
		\begin{itemize}
			\item Standard Kosmologie: $z$ is the gleich for ALL wavelengths from the gleich source
			\item T0 theory: $z$ varies with Wellenlänge - testable Vorhersage!
		\end{itemize}
		In conventional Einheiten, $\Exi$ Skalen with $\hbar c \approx 197.3$ MeV$\cdot$fm, so $\Exi \approx 1.5$ GeV corresponds to $\Exi / (\hbar c) \approx 7500$ m$^{-1}$, ensuring dimensional consistency.
	\end{important}
	
	\subsection{Consistency with Observed Redshifts}
	Current Beobachtungen weder confirm nor refute the Wellenlänge dependence aufgrund von Messung limitations at the detection threshold. The Wellenlänge-dependent Rotverschiebung, given by $z \propto \frac{\xi x}{\Exi} \cdot \lambdazero$, explains beobachtet kosmologisch redshifts in combination with complementary Effekte solch as Doppler shifts, gravitativ Rotverschiebung, and nichtlinear $\xi$-Feld Wechselwirkungen. For high-Rotverschiebung objects ($z > 10$), solch as jene beobachtet by JWST \cite{jwst_early}, the Kopplung Funktion $f\left(\frac{E}{\Exi}\right)$ may contain higher-Ordnung Terme ensuring consistency with Beobachtungen without cosmic Expansion. Future spectroscopic tests, as described in Abschnitt \ref{redshift_deflection:sec:experimental_tests}, will provide definitive Validierung or refutation of dies Mechanismus.
	
	\section{Frequency-Based Formulation}
	
	\subsection{Frequency Energy Loss}
	
	Since $E = h\nu$, the Energie loss Gleichung becomes:
	\begin{equation}
		\frac{d(h\nu)}{dx} = -\frac{\xi (h\nu)^2}{\Exi}
	\end{equation}
	
	Simplifying:
	\begin{equation}
		\frac{d\nu}{dx} = -\frac{\xi h \nu^2}{\Exi}
	\end{equation}
	
	\subsection{Frequency Redshift Formula}
	
	Integrating the Frequenz evolution:
	\begin{equation}
		\int_{\nuzero}^{\nu} \frac{d\nu'}{\nu'^2} = -\frac{\xi h}{\Exi} \int_0^x dx'
	\end{equation}
	
	This yields:
	\begin{equation}
		\frac{1}{\nu} - \frac{1}{\nuzero} = \frac{\xi h x}{\Exi}
	\end{equation}
	
	Therefore:
	\begin{equation}
		\nu = \frac{\nuzero}{1 + \frac{\xi h x \nuzero}{\Exi}}
	\end{equation}
	
	\begin{Formel}
		Frequency Rotverschiebung:
		\begin{equation}
			z = \frac{\nuzero}{\nu} - 1 \approx \frac{\xi h x \nuzero}{\Exi} \quad (\text{natural units, } h = 1; \text{conventional units, } h = \hbar)
		\end{equation}
	\end{Formel}
	
	\begin{important}
		Since $\nu = \frac{c}{\lambda}$, we have $h\nu = \frac{hc}{\lambda}$, confirming:
		\begin{equation}
			z \propto \nu \propto \frac{1}{\lambda}
		\end{equation}
		\textbf{Higher-Frequenz Photonen show greater Rotverschiebung!} In conventional Einheiten, $\Exi$ Skalen with $\hbar c$ to maintain dimensional consistency.
	\end{important}
	
	\section{Observable Predictions without Distance Assumptions}
	
	\subsection{Spectral Line Ratios}
	
	Different atomic Übergänge should show unterschiedlich redshifts gemäß their wavelengths:
	\begin{equation}
		\frac{z(\lambda_1)}{z(\lambda_2)} = \frac{\lambda_1}{\lambda_2}
	\end{equation}
	
	\begin{Experiment}
		\textbf{Hydrogen Line Test:}
		\begin{itemize}
			\item Lyman-$\alpha$ (121.6 nm) vs. H$\alpha$ (656.3 nm)
			\item Predicted Verhältnis: $\frac{z_{\text{Ly}\alpha}}{z_{\text{H}\alpha}} = \frac{121.6}{656.3} = 0.185$
			\item \textbf{Standard Kosmologie predicts: 1.000}
		\end{itemize}
	\end{Experiment}
	
	\subsection{Frequency-Dependent Effects}
	
	For radio vs. optical Beobachtungen of the gleich cosmic object:
	\begin{itemize}
		\item 21 cm line: $\lambda = 0.21$ m
		\item H$\alpha$ line: $\lambda = 6.563 \times 10^{-7}$ m
		\item Predicted Verhältnis: $\frac{z_{21\text{cm}}}{z_{\text{H}\alpha}} = \frac{\lambda_{21\text{cm}}}{\lambda_{\text{H}\alpha}} = \frac{0.21}{6.563 \times 10^{-7}} = 3.2 \times 10^5$
	\end{itemize}
	
	This enormous difference should be detectable sogar with Strom technology if the T0 Mechanismus is korrekt.
	
	\section{Experimentell Tests via Spectroscopy}
	\label{redshift_deflection:sec:experimental_tests}
	
	\subsection{Multi-Wavelength Observations}
	
	\begin{Experiment}
		\textbf{Simultaneous Multiband Spectroscopy:}
		\begin{enumerate}
			\item Observe quasar/galaxy gleichzeitig in UV, optical, IR
			\item Measure Rotverschiebung from unterschiedlich spectral lines
			\item Test whether $z \propto \lambda$ Zusammenhang holds
			\item Compare with Standard Kosmologie Vorhersage ($z = \text{constant}$)
		\end{enumerate}
	\end{Experiment}
	
	\subsection{Radio vs. Optical Redshift}
	
	\begin{Experiment}
		\textbf{21cm vs. Optical Line Comparison:}
		\begin{itemize}
			\item \textbf{Radio surveys}: ALFALFA, HIPASS (21cm redshifts)
			\item \textbf{Optical surveys}: SDSS, 2dF (H$\alpha$, H$\beta$ redshifts)
			\item \textbf{Method}: Compare objects beobachtet in beide surveys
			\item \textbf{Prediction}: $z_{21\text{cm}} \neq z_{\text{optical}}$ (T0) vs. $z_{21\text{cm}} = z_{\text{optical}}$ (Standard)
		\end{itemize}
	\end{Experiment}
	
	\section{Advantages over Standard Cosmology}
	
	\subsection{Model-Independent Approach}
	
	\begin{longtable}{lcc}
		\caption{T0-Theorie vs. Standard Cosmology} \\
		\toprule
		\textbf{Aspect} & \textbf{T0-Theorie} & \textbf{$\Lambda$CDM} \\
		\midrule
		\endfirsthead
		\multicolumn{3}{c}%
		{{\tablename\ \thetable{} -- continued from vorherig page}} \\
		\toprule
		\textbf{Aspect} & \textbf{T0-Theorie} & \textbf{$\Lambda$CDM} \\
		\midrule
		\endhead
		\bottomrule
		\endfoot
		\bottomrule
		\endlastfoot
		Universal Konstante & $\xi = 4/3 \times 10^{-4}$ & None \\
		Dark Energie erforderlich & No & Yes (70\%) \\
		Dark Materie erforderlich & No & Yes (25\%) \\
		Number of Parameter & 1 & 6+ \\
		Hubble tension & Resolved & Unresolved \\
		JWST Beobachtungen & Consistent & Problematic \\
		Big Bang Singularität & None & Required \\
		Horizon problem & None & Unresolved \\
		Flatness problem & Natural & Fine-tuning erforderlich \\
	\end{longtable}
	
	\subsection{Unified Explanations}
	
	The single $\xi$-Konstante explains:
	\begin{enumerate}
		\item \textbf{Gravitational Konstante}: $G = \frac{\xi^2 c^3}{16\pi m_p^2}$
		\item \textbf{CMB Temperatur}: $T_{\text{CMB}} = \frac{16}{9} \xi^2 \times E_\xi$
		\item \textbf{Casimir Effekt}: Related to $\xi$-Feld Vakuum
		\item \textbf{Cosmological Rotverschiebung}: Energy loss through $\xi$-Feld
		\item \textbf{Particle masses}: Geometric resonances in $\xi$-Feld
		\item \textbf{Fine Struktur Konstante}: $\alpha = (4/3)^3 \approx 1/137$
		\item \textbf{Muon anomal magnetisch moment}: $a_\mu = \frac{\xi}{2\pi} \left(\frac{E_\mu}{E_e}\right)^2$
	\end{enumerate}
	
	\section{Critical Assessment: Wavelength Dependence at the Detection Threshold}
	\label{redshift_deflection:sec:wavelength_assessment}
	
	\subsection{Current Experimentell Status and Measurement Limitations}
	
	The T0 theory's Vorhersage of Wellenlänge-dependent Rotverschiebung represents one of its meist distinctive and testable Merkmale. However, the Strom experimentell situation is komplex and requires careful Analyse.
	
	\subsubsection{Precision at the Critical Boundary}
	
	Current spectroscopic Messungen achieve precision of $\Delta z/z \approx 10^{-4}$ to $10^{-5}$, while the T0 Effekt with $\xi = 4/3 \times 10^{-4}$ predicts variations of the gleich Größenordnung. This places us precisely at the detection threshold - a critical situation wo weder Bestätigung nor refutation is currently möglich.
	
	For typical cosmic objects with $\xiconst$, the relative difference in Rotverschiebung zwischen two spectral lines:
	\begin{equation}
		\frac{\Delta z}{z} = \left| \frac{z(\lambda_1) - z(\lambda_2)}{z(\lambda_{\text{mean}})} \right| = \left| \frac{\lambda_1 - \lambda_2}{\lambda_{\text{mean}}} \right| \times \xi \approx 10^{-4} \text{ to } 10^{-5}
	\end{equation}
	
	\begin{important}
		This Wellenlänge Effekt is at the Grenze of Strom spectroscopic precision but potentially detectable with nächst-generation instruments:
		\begin{itemize}
			\item Extremely Large Telescope (ELT): $\Delta z/z \approx 10^{-6}$ to $10^{-7}$
			\item James Webb Space Telescope (JWST): Extended IR spectroscopy
			\item Square Kilometre Array (SKA): Precise 21cm Messungen
		\end{itemize}
	\end{important}
	
	\subsection{Future Experimentell Outcomes and Their Implications}
	
	The nächst generation of instruments will achieve precision $\Delta z/z \approx 10^{-6}$ to $10^{-7}$, schließlich enabling definitive tests. Two primary outcomes are möglich:
	
	\subsubsection{Primary Outcome A: Wavelength Dependence CONFIRMED}
	\label{subsubsec:confirmed}
	
	If Messungen detect $z \propto \lambda_0$ as vorhergesagt:
	
	\textbf{Immediate Implications:}
	\begin{itemize}
		\item \textbf{Fundamental Validierung} of T0 theory's core Mechanismus
		\item \textbf{Paradigm shift}: Redshift from Energie loss, not Expansion
		\item \textbf{New physics confirmed}: Photon-$\xi$-Feld Wechselwirkung is reell
		\item \textbf{Cosmology revolution}: Static Universum Modell validated
	\end{itemize}
	
	\textbf{Required Follow-up Measurements:}
	\begin{itemize}
		\item Precise determination of proportionality Konstante to verify $\xi = 4/3 \times 10^{-4}$
		\item Distance dependence to confirm linear Zusammenhang
		\item Search for Abweichungen at extreme wavelengths (gamma-ray to radio)
	\end{itemize}
	
	\subsubsection{Primary Outcome B: Wavelength Dependence NOT DETECTED}
	\label{subsubsec:not_detected}
	
	If no Wellenlänge dependence is found sogar at $10^{-6}$ precision, two distinct sub-scenarios must be considered:
	
	\subsection{Sub-Scenario B1: Fundamental T0 Mechanism Incorrect}
	\label{redshift_deflection:subsec:scenario_b1}
	
	\textbf{Interpretation:} The nichtlinear Energie loss Mechanismus $dE/dx = -\xi E^2/E_\xi$ is fundamentally wrong.
	
	\textbf{Required Theoretical Adaptation:}
	\begin{itemize}
		\item \textbf{Modified Energie loss Gleichung:} Replace with linear form
		\begin{equation}
			\frac{dE}{dx} = -\xi_{eff} \cdot E
		\end{equation}
		This yields $z = e^{\xi_{eff} x} - 1$, independent of $\lambda_0$
		
		\item \textbf{Reinterpretation of $E_\xi$:} No longer a fundamental Energie Skala for Photon Wechselwirkung
		
		\item \textbf{Alternative Kopplung Funktion:} Instead of $f(E/E_\xi) = E/E_\xi$, use
		\begin{equation}
			f(E/E_\xi) = \text{constant} = \xi_0
		\end{equation}
	\end{itemize}
	
	\textbf{What Remains Valid:}
	\begin{itemize}
		\item Geometric Konstante $\xi = 4/3 \times 10^{-4}$ (from tetrahedron quantization)
		\item Gravitational Konstante Ableitung: $G = \xi^2 c^3/(16\pi m_p^2)$
		\item Particle Masse Verhältnisse from geometrisch Quanten Zahlen
		\item Muon g-2 Anomalie Vorhersage
		\item CMB Temperatur Erklärung
	\end{itemize}
	
	\textbf{What Changes:}
	\begin{itemize}
		\item Loss of unique T0 signature (Wellenlänge dependence)
		\item Harder to distinguish from modified $\Lambda$CDM Modelle
		\item Photon propagation Mechanismus simplified
		\item Need alternative tests to validate static Universum Modell
	\end{itemize}
	
	\subsection{Sub-Scenario B2: Wavelength Dependence Exists but is COMPENSATED}
	\label{redshift_deflection:subsec:scenario_b2}
	
	\textbf{Interpretation:} The T0 Mechanismus is korrekt, but compensating Effekte mask the Wellenlänge dependence.
	
	\subsubsection{Detailed Compensation Mechanisms}
	
	\begin{Formel}[title=Three Compensation Mechanisms]
		The T0 Wellenlänge dependence could be masked by:
		\begin{enumerate}
			\item \textbf{IGM Dispersion}: $z_{\text{IGM}} \propto -\lambda^{-2}$ (opposes $z_{\text{T0}} \propto +\lambda$)
			\item \textbf{Gravitational Layering}: $z_{\text{grav}}(r(\lambda))$ varies with Emission depth
			\item \textbf{Nonlinear Corrections}: Higher-Ordnung Terme $\propto (\xi x \lambda_0/E_\xi)^n$ flatten response
		\end{enumerate}
		Net Effekt: $z_{\text{observed}} = z_{\text{T0}} + z_{\text{comp}} \approx$ Konstante
	\end{Formel}
	
	\textbf{1. Intergalactic Medium (IGM) Dispersion Compensation:}
	\begin{equation}
		z_{\text{observed}} = z_{\text{T0}}(\lambda) + z_{\text{IGM}}(\lambda) + z_{\text{other}}
	\end{equation}
	
	The IGM could provide inverse Wellenlänge dependence:
	\begin{itemize}
		\item T0 Effekt: $z_{\text{T0}} \propto +\lambda$ (longer wavelengths mehr redshifted)
		\item IGM Effekt: $z_{\text{IGM}} \propto -\lambda^{-2}$ (plasma dispersion favors shorter wavelengths)
		\item Net result: $z_{\text{observed}} \approx$ Konstante
	\end{itemize}
	
	\textbf{Physical Mechanismus:} Free Elektronen in IGM create Frequenz-dependent refractive index:
	\begin{equation}
		n(\omega) = 1 - \frac{\omega_p^2}{2\omega^2} \implies z_{\text{IGM}} \propto -\frac{1}{\lambda^2}
	\end{equation}
	
	For appropriate IGM Dichte, dies could precisely cancel T0's linear $\lambda$ dependence.
	
	\textbf{2. Source-Dependent Compensation:}
	
	Different spectral lines originate at unterschiedlich depths in stellar/galactic atmospheres:
	\begin{itemize}
		\item \textbf{UV lines} (e.g., Lyman-$\alpha$): Outer atmosphere, lower Gravitation, weniger gravitativ Rotverschiebung
		\item \textbf{Optical lines} (e.g., H-$\alpha$): Mid-photosphere, moderate gravitativ Feld
		\item \textbf{IR lines}: Deep atmosphere, stronger gravitativ Rotverschiebung
	\end{itemize}
	
	This creates an effektiv compensation:
	\begin{equation}
		z_{\text{total}} = z_{\text{T0}}(\lambda) + z_{\text{grav}}(r(\lambda)) \approx \text{constant}
	\end{equation}
	
	\textbf{3. Nonlinear Field Corrections:}
	
	The complete T0 Lösung might include self-compensation Terme:
	\begin{equation}
		z = \frac{\xi x \lambda_0}{E_\xi}\left[1 - \alpha\left(\frac{\xi x \lambda_0}{E_\xi}\right) + \beta\left(\frac{\xi x \lambda_0}{E_\xi}\right)^2 + ...\right]
	\end{equation}
	
	For specific Werte of $\alpha$ and $\beta$, the Wellenlänge dependence could flatten at kosmologisch distances while remaining visible locally.
	
	\subsubsection{How to Test for Compensation}
	
	\textbf{Observational Strategies:}
	\begin{enumerate}
		\item \textbf{Distance-dependent studies:}
		\begin{itemize}
			\item Measure $\Delta z/\Delta\lambda$ at unterschiedlich distances
			\item Compensation Effekte should vary with Entfernung
			\item T0 Effekt linear with Entfernung, compensation may not be
		\end{itemize}
		
		\item \textbf{Environment-dependent Messungen:}
		\begin{itemize}
			\item Compare objects in voids vs. clusters
			\item Different IGM densities → unterschiedlich compensation
			\item Clean sight lines vs. dense regions
		\end{itemize}
		
		\item \textbf{Source-type variations:}
		\begin{itemize}
			\item Quasars vs. galaxies vs. supernovae
			\item Different Emission Mechanismen
			\item Different atmospheric Strukturen
		\end{itemize}
		
		\item \textbf{Extreme Wellenlänge tests:}
		\begin{itemize}
			\item Gamma-ray bursts (shortest $\lambda$)
			\item Radio galaxies (longest $\lambda$)
			\item Compensation may break down at extremes
		\end{itemize}
	\end{enumerate}
	
	\subsubsection{Required Theoretical Adaptations for B2}
	
	If compensation is confirmed, the T0 theory needs:
	
	\textbf{1. Extended Framework:}
	\begin{equation}
		z_{\text{total}} = z_{\text{T0}}(\lambda, x) + \sum_i z_{\text{comp},i}(\lambda, x, \rho, T, ...)
	\end{equation}
	
	\textbf{2. Environmental Parameters:}
	\begin{itemize}
		\item IGM Dichte profile: $\rho_{\text{IGM}}(x)$
		\item Temperature Verteilung: $T(x)$
		\item Magnetic Feld Effekte: $B(x)$
	\end{itemize}
	
	\textbf{3. Refined Predictions:}
	\begin{itemize}
		\item Residual Wellenlänge dependence in specific Bedingungen
		\item Optimal Beobachtung strategies to reveal T0 Effekt
		\item Predictions for wann compensation fails
	\end{itemize}
	
	\subsection{The Suspicious Coincidence}
	
	The fact das the vorhergesagt T0 Effekt Größenordnung ($\xi = 4/3 \times 10^{-4}$) places the Wellenlänge dependence \textit{exactly} at the Strom detection threshold deserves speziell attention:
	
	\begin{itemize}
		\item \textbf{Probability argument}: The chance das a fundamental Konstante would zufällig place an Effekt precisely at our Strom technological Grenze is extremely klein
		\item \textbf{Historical precedent}: Similar "coincidences" in physics oft indicated reell Effekte masked by complications (e.g., solar Neutrino problem)
		\item \textbf{Anthropic consideration}: No anthropic reason constrains $\xi$ to dies specific Wert
		\item \textbf{Most wahrscheinlich Interpretation}: The Effekt exists but is teilweise compensated, keeping it nur unten clear detection
	\end{itemize}
	
	\begin{Experiment}[title=Testing the Coincidence]
		To resolve whether dies coincidence is meaningful:
		\begin{enumerate}
			\item Compare Messungen from unterschiedlich epochs as technology improves
			\item Look for systematic trends in "non-detections" near the threshold
			\item Search for environmental correlations in marginal detections
			\item Perform meta-Analyse of alle Wellenlänge-dependence studies
		\end{enumerate}
	\end{Experiment}
	
	\subsection{Decision Tree for Future Observations}
	
	\begin{center}
		\resizebox{\textwidth}{!}{%
\begin{tabular}{l}
			\textbf{High-precision measurement} (MATHBLOCK106ENDMATH) \\
			\midrule
			MATHBLOCK107ENDMATH \\
			\textbf{Question:} Wavelength dependence detected? \\
			\midrule
			\textbf{YES} MATHBLOCK108ENDMATH T0 CONFIRMED (Outcome A) \\
			\hspace{1cm} • Measure MATHBLOCK109ENDMATH precisely \\
			\hspace{1cm} • Test distance dependence \\
			\midrule
			\textbf{NO} MATHBLOCK110ENDMATH Further investigation required \\
			\hspace{1cm} \textbf{Test:} Universal across all conditions? \\
			\hspace{2cm} YES MATHBLOCK111ENDMATH B1: Modify T0 (linear mechanism) \\
			\hspace{2cm} NO MATHBLOCK112ENDMATH B2: Compensation (refine theory)
		\end{tabular}}
	\end{center}
	
	\subsection{Schlussfolgerung: A Theorie at the Crossroads}
	
	The T0 theory stands at a critical juncture. The Wellenlänge-dependent Rotverschiebung Vorhersage will entweder:
	
	\begin{itemize}
		\item \textbf{Revolutionize Kosmologie} if confirmed (Outcome A)
		\item \textbf{Require simplification} if absent (Sub-scenario B1)
		\item \textbf{Reveal hidden complexity} if compensated (Sub-scenario B2)
	\end{itemize}
	
	\begin{important}[title=Critical Insight: The Coincidence Problem]
		\textbf{The remarkably präzise coincidence das $\xi = 4/3 \times 10^{-4}$ places the Effekt exactly at Strom detection Grenzen suggests dies is not accidental.} The meist wahrscheinlich scenario may be B2 - the Effekt exists but is teilweise compensated, explaining warum we are precisely at the threshold wo the Effekt is weder klar visible nor klar absent.
	\end{important}
	
	Each outcome advances our Verständnis: Bestätigung validates a new kosmologisch paradigm, absence simplifies the theory while preserving its geometrisch foundations, and compensation reveals additional physics we must account for. This is science at its best - clear Vorhersagen, definitive tests, and the flexibility to learn from whatever nature reveals.
	
	\begin{revolutionary}[title=A Historic Moment in Physics]
		We stand at a unique juncture in the history of Kosmologie. Within the nächst decade, humanity will definitively know whether:
		\begin{itemize}
			\item The Universum is static with Photon Energie loss (T0 confirmed)
			\item The Universum expands as currently believed (T0 refuted via B1)
			\item Reality is mehr komplex than entweder Modell alone (T0 with compensation via B2)
		\end{itemize}
		Each outcome revolutionizes our Verständnis. This is not merely a test of a theory - it is a fundamental verdict on the nature of the cosmos itself.
	\end{revolutionary}	
	
	\section{Statistical Analysis Method}
	
	\subsection{Multi-Line Regression}
	
	\begin{Experiment}
		\textbf{Wavelength-Redshift Correlation Test:}
		\begin{enumerate}
			\item Collect Rotverschiebung Messungen: $\{z_i, \lambda_i\}$ for jeder object
			\item Fit linear Zusammenhang: $z = \alpha \cdot \lambda + \beta$
			\item Compare slope $\alpha$ with T0 Vorhersage: $\alpha = \frac{\xi x}{\Exi}$
			\item Test against Standard Kosmologie: $\alpha = 0$
		\end{enumerate}
	\end{Experiment}
	
	\subsection{Required Precision}
	
	To detect T0 Effekte with $\xiconst$:
	\begin{itemize}
		\item \textbf{Minimum erforderlich precision}: $\frac{\Delta z}{z} \approx 10^{-5}$
		\item \textbf{Current best precision}: $\frac{\Delta z}{z} \approx 10^{-4}$ (barely ausreichend)
		\item \textbf{Next generation instruments}: $\frac{\Delta z}{z} \approx 10^{-6}$ (klar detectable)
	\end{itemize}
	
	\section{Mathematical Equivalence of Space Expansion, Energy Loss, and Diffraction}
	\label{redshift_deflection:sec:equivalence}
	
	\subsection{Formal Equivalence Proofs}
	\label{redshift_deflection:subsec:equivalence_proofs}
	
	The three fundamental Mechanismen for explaining kosmologisch Rotverschiebung can be described by unterschiedlich physikalisch Prozesse but lead to mathematically equivalent results under certain Bedingungen.
	
	\begin{table}[h]
		\centering
		\caption{Comparison of Redshift Mechanisms with Extended Developments}
		\scalebox{0.75}{
			\resizebox{\textwidth}{!}{%
MATHBLOCK194ENDMATH}
		}
	\end{table}
	
	\subsubsection{Mathematical Equivalence Conditions}
	
	For the Äquivalenz of the three Mechanismen, the folgend Bedingungen must be satisfied:
	
	\begin{equation}
		\boxed{\frac{1}{a}\frac{da}{dt} = -\frac{1}{n}\frac{dn}{dt} = \xi \frac{H}{T_0}}
	\end{equation}
	
	This leads to the relationships:
	\begin{itemize}
		\item \textbf{$\Lambda$CDM $\leftrightarrow$ T0-B}: $n(t) = a^{-1}(t)$
		\item \textbf{$\Lambda$CDM $\leftrightarrow$ T0-E}: $\dot{E}/E = -H(t)$
		\item \textbf{T0-B $\leftrightarrow$ T0-E}: $n(t) \propto E^{-1}(t)$
	\end{itemize}
	
	\subsubsection{Perturbative Development}
	
	The Äquivalenz holds exactly nur in erst Ordnung. Higher-Ordnung Abweichungen provide distinguishing signatures:
	
	\begin{equation}
		z_{total} = z_0 + \Delta z_{mechanism} + O(\xi^2)
	\end{equation}
	
	wo $\Delta z_{mechanism}$ depends on the specific physikalisch Prozess.
	
	\subsection{Energy Conservation and Thermodynamics}
	\label{redshift_deflection:subsec:energy_conservation}
	
	\subsubsection{Energy Balance in Different Formalisms}
	
	\textbf{$\Lambda$CDM (apparent Energie loss):}
	\begin{equation}
		E_{photon} = \frac{h\nu_0}{1+z} = \frac{h\nu_0 a(t_e)}{a(t_0)}
	\end{equation}
	
	\textbf{T0-Diffraction (Energie Erhaltung):}
	\begin{equation}
		E_{photon} = \frac{h\nu}{n(t)} = \frac{h\nu_0}{(1+z)n(t)} = \text{const}
	\end{equation}
	
	\textbf{T0-Energy Loss (reell loss):}
	\begin{equation}
		\frac{dE}{dt} = -\xi H E \quad \Rightarrow \quad E(t) = E_0 \exp\left(-\int_0^t \xi H(t') dt'\right)
	\end{equation}
	
	\subsubsection{Thermodynamic Consistency}
	
	The entropy change for the unterschiedlich Mechanismen:
	
	\begin{equation}
		\Delta S = \begin{cases}
			0 & \text{(MATHBLOCK140ENDMATHCDM: adiabatic)} \\
			k_B \xi N_{photon} \ln(1+z) & \text{(T0-Energy Loss)} \\
			0 & \text{(T0-Diffraction: reversible)}
		\end{cases}
	\end{equation}
	
	\section{Implications for Cosmology}
	
	\subsection{Static Universe Model}
	
	The T0-theory describes a static, eternally existing Universum wo:
	\begin{itemize}
		\item Redshift arises from Energie loss, not Expansion
		\item CMB is equilibrium Strahlung of the $\xi$-Feld
		\item No Big Bang Singularität erforderlich
		\item No dunkel Energie or dunkel Materie needed
		\item Cyclic Prozesse möglich innerhalb static Rahmenwerk
	\end{itemize}
	
	\subsection{Resolution of Cosmological Tensions}
	
	The T0 Modell resolves:
	\begin{enumerate}
		\item \textbf{Hubble tension}: Different Messungen reconciled through $\xi$-Effekte
		\item \textbf{JWST early galaxies}: No formation Zeit paradox in static Universum
		\item \textbf{Cosmic coincidence}: Natural Erklärung through $\xi$-Geometrie
		\item \textbf{Horizon problem}: No Horizont in eternal Universum
		\item \textbf{Flatness problem}: Natural Konsequenz of static Geometrie
	\end{enumerate}
	
	\section{Robustness of Core T0 Predictions}
	
	\subsection{Independent of Redshift Mechanism}
	
	Even if spectroscopic tests fail to detect Wellenlänge-dependent Rotverschiebung, the folgend T0 Vorhersagen remain gültig:
	
	\begin{enumerate}
		\item \textbf{Gravitational Konstante}: $G = \frac{\xi^2 c^3}{16\pi m_p^2} = 6.674 \times 10^{-11}$ m$^3$kg$^{-1}$s$^{-2}$ (genau to 8 digits) remains gültig, independent of kosmologisch tests
		
		\item \textbf{Geometric Konstanten}: The Ableitung of $\alpha \approx 1/137$ from $(4/3)^3$ scaling remains
		
		\item \textbf{Mass hierarchy}: $m_e : m_\mu : m_\tau = 1 : 206.768 : 3477.15$ follows from Quanten Zahlen, not Rotverschiebung
		
		\item \textbf{Hubble tension}: The 4/3 Erklärung works ungeachtet of specific Mechanismus
	\end{enumerate}
	
	\subsection{Adaptivity of Theoretical Structure}
	
	The T0-theory has natural adaptation Mechanismen:
	
	\begin{equation}
		\xi_{eff}(\text{Scale}) = \xi_0 \times f(\text{Environment}) \times g(\text{Energy})
	\end{equation}
	
	wo:
	\begin{itemize}
		\item $f(\text{Environment}) = 4/3$ in galaxy clusters, $= 1$ in intergalactic medium
		\item $g(\text{Energy})$ describes renormalization group running
	\end{itemize}
	
	This flexibility is not an ad-hoc adjustment but follows from the geometrisch Struktur of the theory.
	
	\section{Schlussfolgerungen}
	
	The T0-theory provides a revolutionary alternative to Expansion-based Kosmologie through a single universal Konstante $\xiconst$. The Wellenlänge-dependent Rotverschiebung Vorhersage offers a clear experimentell test to distinguish zwischen T0 and Standard Kosmologie. While Strom precision barely reaches the detection threshold, nächst-generation spectroscopic instruments should definitively test dies fundamental Vorhersage.
	
	The unification of gravitativ, elektromagnetisch, and Quanten Phänomene through the $\xi$-Feld represents a paradigm shift from komplex multi-Parameter Modelle to elegant geometrisch simplicity. The experimentell tests proposed hier, besonders multi-Wellenlänge spectroscopy of cosmic objects, provide clear pathways to validate or refute the theory.
	
	\begin{important}[title=Final Perspective]
		The T0-theory demonstrates das alle cosmic Phänomene can be understood through a single geometrisch Konstante, eliminating the need for dunkel Materie, dunkel Energie, inflation, and the Big Bang Singularität. This represents the meist significant simplification in physics since Newton's unification of terrestrial and celestial Mechanik.
	\end{important}
	
	% Bibliography
	\bibliographystyle{unsrt}

\begin{thebibliography}{99}

% ============================================
% Core T0 Theory References (J. Pascher)
% GitHub Repository: https://github.com/jpascher/T0-Time-Mass-Duality
% ============================================

\bibitem{pascher2024}
J. Pascher, \emph{T0 Theory: Time-Mass Duality}, 2024.
\url{https://github.com/jpascher/T0-Time-Mass-Duality/blob/main/2/pdf/T0_unified_report.pdf}

\bibitem{pascher2025t0}
J. Pascher, \emph{T0 Theory: Fundamentals}, 2025.
\url{https://github.com/jpascher/T0-Time-Mass-Duality/blob/main/2/pdf/T0_Grundlagen_En.pdf}

\bibitem{pascher2025qm}
J. Pascher, \emph{T0 Theory: Quantum Mechanics}, 2025.
\url{https://github.com/jpascher/T0-Time-Mass-Duality/blob/main/2/pdf/QM_En.pdf}

\bibitem{pascher2025si}
J. Pascher, \emph{T0 Theory: SI Units}, 2025.
\url{https://github.com/jpascher/T0-Time-Mass-Duality/blob/main/2/pdf/T0_SI_En.pdf}

\bibitem{pascher2025g2}
J. Pascher, \emph{T0 Theory: The g-2 Anomaly}, 2025.
\url{https://github.com/jpascher/T0-Time-Mass-Duality/blob/main/2/pdf/T0_Anomale-g2-9_En.pdf}

\bibitem{pascher2025cmb}
J. Pascher, \emph{T0 Theory: CMB Analysis}, 2025.
\url{https://github.com/jpascher/T0-Time-Mass-Duality/blob/main/2/pdf/Zwei-Dipole-CMB_En.pdf}

% Historical Physics
\bibitem{einstein1905}
A. Einstein, \emph{On the Electrodynamics of Moving Bodies}, Annalen der Physik, 1905.
\url{https://doi.org/10.1002/andp.19053221004}

\bibitem{dirac1928}
P.A.M. Dirac, \emph{The Quantum Theory of the Electron}, Proc. Roy. Soc. A, 1928.
\url{https://doi.org/10.1098/rspa.1928.0023}

\bibitem{planck1900}
M. Planck, \emph{On the Theory of the Energy Distribution Law}, 1900.
\url{https://doi.org/10.1002/andp.19013090310}

\bibitem{mach1883}
E. Mach, \emph{Die Mechanik in ihrer Entwicklung}, 1883.

\bibitem{hundert1931}
Various Authors, \emph{100 Authors Against Einstein}, 1931.

\bibitem{dingle1972}
H. Dingle, \emph{Science at the Crossroads}, 1972.

% Penrose and Terrell Effect
\bibitem{terrell1959}
J. Terrell, \emph{Invisibility of the Lorentz Contraction}, Phys. Rev., 1959.
\url{https://doi.org/10.1103/PhysRev.116.1041}

\bibitem{penrose1959}
R. Penrose, \emph{The Apparent Shape of a Relativistically Moving Sphere}, Proc. Cambridge Phil. Soc., 1959.
\url{https://doi.org/10.1017/S0305004100033776}

\bibitem{penrose1967}
R. Penrose, \emph{Twistor Algebra}, J. Math. Phys., 1967.
\url{https://doi.org/10.1063/1.1705200}

\bibitem{penrose2004}
R. Penrose, \emph{The Road to Reality}, 2004.

\bibitem{terrell2025}
J. Terrell et al., \emph{Modern Terrell-Penrose Visualization}, 2025.

\bibitem{weiskopf2000}
D. Weiskopf, \emph{Visualization of Four-dimensional Spacetimes}, 2000.

\bibitem{mueller2014}
T. Müller, \emph{Visual Appearance of Relativistically Moving Objects}, 2014.

\bibitem{hossenfelder2025}
S. Hossenfelder, \emph{YouTube: The Terrell Effect}, 2025.

% Quantum Gravity and String Theory
\bibitem{rovelli2004}
C. Rovelli, \emph{Quantum Gravity}, Cambridge University Press, 2004.

\bibitem{thiemann2007}
T. Thiemann, \emph{Modern Canonical Quantum Gravity}, Cambridge University Press, 2007.

\bibitem{ashtekar2004}
A. Ashtekar, J. Lewandowski, \emph{Background Independent Quantum Gravity}, Class. Quant. Grav., 2004.
\url{https://doi.org/10.1088/0264-9381/21/15/R01}

\bibitem{jacobson1995}
T. Jacobson, \emph{Thermodynamics of Spacetime}, Phys. Rev. Lett., 1995.
\url{https://doi.org/10.1103/PhysRevLett.75.1260}

\bibitem{maldacena1998}
J. Maldacena, \emph{The Large N Limit of Superconformal Field Theories}, Adv. Theor. Math. Phys., 1998.
\url{https://doi.org/10.4310/ATMP.1998.v2.n2.a1}

\bibitem{polchinski1998}
J. Polchinski, \emph{String Theory}, Cambridge University Press, 1998.

\bibitem{susskind1995}
L. Susskind, \emph{The World as a Hologram}, J. Math. Phys., 1995.
\url{https://doi.org/10.1063/1.531249}

\bibitem{verlinde2011}
E. Verlinde, \emph{On the Origin of Gravity}, JHEP, 2011.
\url{https://doi.org/10.1007/JHEP04(2011)029}

% Cosmology
\bibitem{hoyle1948}
F. Hoyle, \emph{A New Model for the Expanding Universe}, MNRAS, 1948.
\url{https://doi.org/10.1093/mnras/108.5.372}

\bibitem{bondi1948}
H. Bondi, T. Gold, \emph{The Steady-State Theory}, MNRAS, 1948.
\url{https://doi.org/10.1093/mnras/108.3.252}

\bibitem{zwicky1929}
F. Zwicky, \emph{On the Redshift of Spectral Lines}, Proc. Nat. Acad. Sci., 1929.
\url{https://doi.org/10.1073/pnas.15.10.773}

\bibitem{lopez2010}
C. Lopez-Corredoira, \emph{Tests of Cosmological Models}, Int. J. Mod. Phys. D, 2010.

\bibitem{lerner2014}
E. Lerner, \emph{Evidence for a Non-Expanding Universe}, 2014.

\bibitem{albrecht1999}
A. Albrecht, J. Magueijo, \emph{Variable Speed of Light}, Phys. Rev. D, 1999.
\url{https://doi.org/10.1103/PhysRevD.59.043516}

\bibitem{barrow1999}
J. Barrow, \emph{Cosmologies with Varying Light Speed}, Phys. Rev. D, 1999.
\url{https://doi.org/10.1103/PhysRevD.59.043515}

\bibitem{riess2022}
A. Riess et al., \emph{A Comprehensive Measurement of the Local Value of the Hubble Constant}, ApJ, 2022.
\url{https://doi.org/10.3847/2041-8213/ac5c5b}

\bibitem{desi2025}
DESI Collaboration, \emph{DESI Year 1 Results}, 2025.
\url{https://arxiv.org/abs/2404.03002}

\bibitem{divalentino2021}
E. Di Valentino et al., \emph{Planck Evidence for a Closed Universe}, Nat. Astron., 2021.
\url{https://doi.org/10.1038/s41550-019-0906-9}

% Conformal Field Theory
\bibitem{francesco1997}
P. Di Francesco et al., \emph{Conformal Field Theory}, Springer, 1997.

% Experimental Physics
\bibitem{pdg2024}
Particle Data Group, \emph{Review of Particle Physics}, 2024.
\url{https://pdg.lbl.gov/}

\bibitem{codata2019}
CODATA, \emph{Recommended Values of Fundamental Constants}, 2019.
\url{https://physics.nist.gov/cuu/Constants/}

\bibitem{newell2018}
D. Newell et al., \emph{The CODATA 2017 Values of h, e, k, and $N_A$}, Metrologia, 2018.
\url{https://doi.org/10.1088/1681-7575/aa950a}

\bibitem{muong2_2023}
Muon g-2 Collaboration, \emph{Measurement of the Anomalous Magnetic Moment of the Muon}, Phys. Rev. Lett., 2023.
\url{https://doi.org/10.1103/PhysRevLett.131.161802}

\bibitem{fermilab2023}
Fermilab, \emph{Muon g-2 Results}, 2023.
\url{https://muon-g-2.fnal.gov/}

\bibitem{atlas2023}
ATLAS Collaboration, \emph{Measurements at the LHC}, 2023.
\url{https://atlas.cern/}

\bibitem{atlas2023higgs}
ATLAS Collaboration, \emph{Higgs Boson Properties}, 2023.
\url{https://atlas.cern/}

\bibitem{cms2023top}
CMS Collaboration, \emph{Top Quark Measurements}, 2023.
\url{https://cms.cern/}

\bibitem{cms2024}
CMS Collaboration, \emph{Heavy Ion Collisions}, 2024.
\url{https://cms.cern/}

\bibitem{alice2023}
ALICE Collaboration, \emph{Quark-Gluon Plasma Studies}, 2023.
\url{https://alice-collaboration.web.cern.ch/}

\bibitem{kasevich2023}
M. Kasevich et al., \emph{Atom Interferometry}, 2023.

\bibitem{ludlow2015}
A. Ludlow et al., \emph{Optical Atomic Clocks}, Rev. Mod. Phys., 2015.
\url{https://doi.org/10.1103/RevModPhys.87.637}

\bibitem{brewer2019}
S. Brewer et al., \emph{Al$^+$ Optical Clock}, Phys. Rev. Lett., 2019.
\url{https://doi.org/10.1103/PhysRevLett.123.033201}

\bibitem{lisa2017}
LISA Collaboration, \emph{LISA Mission}, 2017.
\url{https://www.lisamission.org/}

% Fractal Physics
\bibitem{nottale1993}
L. Nottale, \emph{Fractal Space-Time and Microphysics}, World Scientific, 1993.

\bibitem{elnaschie2004}
M.S. El Naschie, \emph{E-Infinity Theory}, Chaos Solitons Fractals, 2004.

% Philosophy and Foundations
\bibitem{wheeler1990}
J.A. Wheeler, \emph{Information, Physics, Quantum}, 1990.

\bibitem{barbour1999}
J. Barbour, \emph{The End of Time}, Oxford University Press, 1999.

\bibitem{sciama1953}
D. Sciama, \emph{On the Origin of Inertia}, MNRAS, 1953.
\url{https://doi.org/10.1093/mnras/113.1.34}

% String Theory Extensions
\bibitem{becker2007}
K. Becker et al., \emph{String Theory and M-Theory}, Cambridge University Press, 2007.

% Missing References for g-2 Chapter
\bibitem{sm_g2_2025}
Muon g-2 Theory Initiative, \emph{Standard Model Prediction for g-2}, arXiv, 2025.
\url{https://arxiv.org/abs/2006.04822}

\bibitem{mug2_final_2025}
Muon g-2 Collaboration, \emph{Final Report on the Anomalous Magnetic Moment of the Muon}, Fermilab, 2025.
\url{https://muon-g-2.fnal.gov/}

\bibitem{pascher_t0_theory_2025}
J. Pascher, \emph{T0 Theory: Complete Framework}, 2025.
\url{https://github.com/jpascher/T0-Time-Mass-Duality/blob/main/2/pdf/systemEn.pdf}

\bibitem{peskin_schroeder_1995}
M.E. Peskin and D.V. Schroeder, \emph{An Introduction to Quantum Field Theory}, Westview Press, 1995.

\bibitem{parker_somov_2018}
R.H. Parker et al., \emph{Measurement of the Fine-Structure Constant}, Science, 2018.
\url{https://doi.org/10.1126/science.aap7706}

\bibitem{morel_rubidium_2020}
L. Morel et al., \emph{Determination of $\alpha$ from Rubidium Atom Recoil}, Nature, 2020.
\url{https://doi.org/10.1038/s41586-020-2964-7}

\bibitem{aoyama_theory_2020}
T. Aoyama et al., \emph{Theory of the Electron Anomalous Magnetic Moment}, Phys. Rep., 2020.
\url{https://doi.org/10.1016/j.physrep.2020.07.006}

\bibitem{fan_lattice_2023}
X. Fan et al., \emph{Hadronic Contributions from Lattice QCD}, Phys. Rev. D, 2023.

\bibitem{hanneke_electron_2008}
D. Hanneke et al., \emph{New Measurement of the Electron g-2}, Phys. Rev. Lett., 2008.
\url{https://doi.org/10.1103/PhysRevLett.100.120801}

% Additional T0 Theory References
\bibitem{pascher_higgs_connection_2025}
J. Pascher, \emph{Higgs Connection in T0 Theory}, 2025.
\url{https://github.com/jpascher/T0-Time-Mass-Duality/blob/main/2/pdf/T0_Energie_En.pdf}

\bibitem{T0_SI}
J. Pascher, \emph{T0 Theory and SI Units}, 2025.
\url{https://github.com/jpascher/T0-Time-Mass-Duality/blob/main/2/pdf/T0_SI_En.pdf}

\bibitem{T0_gravitational_constant}
J. Pascher, \emph{Gravitational Constant in T0 Framework}, 2025.
\url{https://github.com/jpascher/T0-Time-Mass-Duality/blob/main/2/pdf/T0_Gravitationskonstante_En.pdf}

\bibitem{T0_fine_structure}
J. Pascher, \emph{Fine Structure Constant Analysis}, 2025.
\url{https://github.com/jpascher/T0-Time-Mass-Duality/blob/main/2/pdf/T0_Feinstruktur_En.pdf}

\bibitem{bell_muon}
J.S. Bell, \emph{Muon Studies}, 1966.

\bibitem{QFT_T0}
J. Pascher, \emph{Quantum Field Theory in T0}, 2025.
\url{https://github.com/jpascher/T0-Time-Mass-Duality/blob/main/2/pdf/QFT_En.pdf}

\bibitem{planck2018}
Planck Collaboration, \emph{Planck 2018 Results}, A\&A, 2018.
\url{https://doi.org/10.1051/0004-6361/201833910}

\bibitem{pascher:t0_foundations}
J. Pascher, \emph{T0 Theory Foundations}, 2025.
\url{https://github.com/jpascher/T0-Time-Mass-Duality/blob/main/2/pdf/T0_Grundlagen_En.pdf}

\bibitem{pascher:geometric_formalism}
J. Pascher, \emph{Geometric Formalism in T0}, 2025.
\url{https://github.com/jpascher/T0-Time-Mass-Duality/blob/main/2/pdf/T0_Geometrische_Kosmologie_En.pdf}

\bibitem{riess2019}
A. Riess et al., \emph{Hubble Constant Measurements}, ApJ, 2019.
\url{https://doi.org/10.3847/1538-4357/ab1422}

\bibitem{t0_kosmologie}
J. Pascher, \emph{T0 Kosmologie}, 2025.
\url{https://github.com/jpascher/T0-Time-Mass-Duality/blob/main/2/pdf/T0_Kosmologie_En.pdf}

\bibitem{hossenfelder_single_clock_video}
S. Hossenfelder, \emph{Single Clock Video}, YouTube, 2025.
\url{https://www.youtube.com/c/SabineHossenfelder}

\bibitem{video2025}
Various, \emph{Video References}, 2025.

\bibitem{unnikrishnan2004}
C.S. Unnikrishnan, \emph{Gravity Studies}, 2004.

\bibitem{peratt1992}
A. Peratt, \emph{Plasma Cosmology}, 1992.
\url{https://github.com/jpascher/T0-Time-Mass-Duality/blob/main/2/pdf/T0_peratt_En.pdf}

\bibitem{T0_tm_erweiterung}
J. Pascher, \emph{T0 Time-Mass Extension}, 2025.
\url{https://github.com/jpascher/T0-Time-Mass-Duality/blob/main/2/pdf/T0_tm-erweiterung-x6_En.pdf}

\bibitem{T0_g2_erweiterung}
J. Pascher, \emph{T0 g-2 Extension}, 2025.
\url{https://github.com/jpascher/T0-Time-Mass-Duality/blob/main/2/pdf/T0_g2-erweiterung-4_En.pdf}

\bibitem{T0_netze_en}
J. Pascher, \emph{T0 Networks}, 2025.
\url{https://github.com/jpascher/T0-Time-Mass-Duality/blob/main/2/pdf/T0_netze_En.pdf}

\bibitem{Adams1925}
W. Adams, \emph{Gravitational Redshift}, 1925.
\url{https://doi.org/10.1073/pnas.11.7.382}

\bibitem{Ashby2003}
N. Ashby, \emph{Relativity in GPS}, Living Rev. Rel., 2003.
\url{https://doi.org/10.12942/lrr-2003-1}

\bibitem{Bertotti2003}
B. Bertotti et al., \emph{Cassini Doppler Test}, Nature, 2003.
\url{https://doi.org/10.1038/nature01997}

\bibitem{Bolton2008}
A. Bolton et al., \emph{Gravitational Lensing}, 2008.

\bibitem{Born2013}
M. Born, \emph{Einstein's Theory of Relativity}, Dover, 2013.

\bibitem{Brans1961}
C. Brans and R.H. Dicke, \emph{Mach's Principle}, Phys. Rev., 1961.
\url{https://doi.org/10.1103/PhysRev.124.925}

\bibitem{Dirac1927}
P.A.M. Dirac, \emph{Quantum Mechanics}, Proc. Roy. Soc., 1927.
\url{https://doi.org/10.1098/rspa.1927.0039}

\bibitem{Duhem1906}
P. Duhem, \emph{Theory of Physics}, 1906.

\bibitem{Einstein1905}
A. Einstein, \emph{Special Relativity}, Ann. Phys., 1905.
\url{https://doi.org/10.1002/andp.19053221004}

\bibitem{Feynman2006}
R. Feynman, \emph{QED: The Strange Theory of Light and Matter}, 2006.

\bibitem{Griffiths2017}
D. Griffiths, \emph{Introduction to Quantum Mechanics}, 2017.

\bibitem{Jackson1999}
J.D. Jackson, \emph{Classical Electrodynamics}, 1999.

\bibitem{Kaluza1921}
T. Kaluza, \emph{Five-Dimensional Theory}, 1921.

\bibitem{Klein1926}
O. Klein, \emph{Quantum Theory and Relativity}, 1926.

\bibitem{Kuhn1962}
T. Kuhn, \emph{Structure of Scientific Revolutions}, 1962.

\bibitem{Kuhn1977}
T. Kuhn, \emph{Essential Tension}, 1977.

\bibitem{Ludlow2015}
A. Ludlow et al., \emph{Optical Atomic Clocks}, Rev. Mod. Phys., 2015.
\url{https://doi.org/10.1103/RevModPhys.87.637}

\bibitem{Maxwell1873}
J.C. Maxwell, \emph{Treatise on Electricity and Magnetism}, 1873.

\bibitem{McGaugh2016}
S. McGaugh et al., \emph{Radial Acceleration Relation}, Phys. Rev. Lett., 2016.
\url{https://doi.org/10.1103/PhysRevLett.117.201101}

\bibitem{Mohr2016}
P. Mohr et al., \emph{CODATA Values}, Rev. Mod. Phys., 2016.
\url{https://doi.org/10.1103/RevModPhys.88.035009}

\bibitem{PDG2020}
Particle Data Group, \emph{Review of Particle Physics}, Prog. Theor. Exp. Phys., 2020.
\url{https://pdg.lbl.gov/}

\bibitem{Parker2018}
R. Parker et al., \emph{Measurement of $\alpha$}, Science, 2018.
\url{https://doi.org/10.1126/science.aap7706}

\bibitem{Peskin1995}
M. Peskin and D. Schroeder, \emph{QFT}, 1995.

\bibitem{Planck1900}
M. Planck, \emph{Quantum Theory}, 1900.

\bibitem{Planck2020}
Planck Collaboration, \emph{Planck 2020 Results}, 2020.
\url{https://doi.org/10.1051/0004-6361/201833910}

\bibitem{Poincare1905}
H. Poincaré, \emph{Dynamics of the Electron}, 1905.

\bibitem{Pound1960}
R.V. Pound and G.A. Rebka, \emph{Gravitational Redshift}, Phys. Rev. Lett., 1960.
\url{https://doi.org/10.1103/PhysRevLett.4.337}

\bibitem{Quine1951}
W.V. Quine, \emph{Two Dogmas of Empiricism}, 1951.

\bibitem{Quinn2013}
T. Quinn et al., \emph{Gravitational Constant}, 2013.
\url{https://doi.org/10.1103/PhysRevLett.111.101102}

\bibitem{Randall1999}
L. Randall and R. Sundrum, \emph{Extra Dimensions}, Phys. Rev. Lett., 1999.
\url{https://doi.org/10.1103/PhysRevLett.83.3370}

\bibitem{Riess1998}
A. Riess et al., \emph{Type Ia Supernovae}, AJ, 1998.
\url{https://doi.org/10.1086/300499}

\bibitem{Shapiro1971}
I. Shapiro et al., \emph{Time Delay Test}, Phys. Rev. Lett., 1971.
\url{https://doi.org/10.1103/PhysRevLett.26.1132}

\bibitem{Sommerfeld1916}
A. Sommerfeld, \emph{Fine Structure}, 1916.

\bibitem{Suyu2017}
S. Suyu et al., \emph{Time Delay Cosmography}, MNRAS, 2017.
\url{https://doi.org/10.1093/mnras/stx483}

\bibitem{T0Theory}
J. Pascher, \emph{T0 Theory}, 2025.
\url{https://github.com/jpascher/T0-Time-Mass-Duality/blob/main/2/pdf/systemEn.pdf}

\bibitem{T0_Feinstruktur}
J. Pascher, \emph{Fine Structure in T0}, 2025.
\url{https://github.com/jpascher/T0-Time-Mass-Duality/blob/main/2/pdf/T0_Feinstruktur_En.pdf}

\bibitem{Uzan2003}
J.-P. Uzan, \emph{Constants Variation}, Rev. Mod. Phys., 2003.
\url{https://doi.org/10.1103/RevModPhys.75.403}

\bibitem{Webb2001}
J.K. Webb et al., \emph{Fine Structure Constant}, Phys. Rev. Lett., 2001.
\url{https://doi.org/10.1103/PhysRevLett.87.091301}

\bibitem{Weinberg1979}
S. Weinberg, \emph{Cosmological Constant}, Rev. Mod. Phys., 1979.

\bibitem{Weinberg1989}
S. Weinberg, \emph{Cosmological Constant Problem}, 1989.
\url{https://doi.org/10.1103/RevModPhys.61.1}

\bibitem{Weinberg1995}
S. Weinberg, \emph{Quantum Theory of Fields}, 1995.

\bibitem{Will2014}
C. Will, \emph{Theory and Experiment in Gravitational Physics}, 2014.
\url{https://doi.org/10.12942/lrr-2014-4}

\bibitem{dirac_principles}
P.A.M. Dirac, \emph{Principles of Quantum Mechanics}, 1930.

\bibitem{einstein_1917}
A. Einstein, \emph{Cosmological Considerations}, 1917.

\bibitem{jwst_early}
JWST Collaboration, \emph{Early Universe Observations}, 2023.
\url{https://www.jwst.nasa.gov/}

\bibitem{katrin_2022}
KATRIN Collaboration, \emph{Neutrino Mass}, 2022.
\url{https://doi.org/10.1038/s41567-021-01463-1}

\bibitem{pascher:fundamentals}
J. Pascher, \emph{T0 Fundamentals}, 2025.
\url{https://github.com/jpascher/T0-Time-Mass-Duality/blob/main/2/pdf/T0_Grundlagen_En.pdf}

\bibitem{pascher:g2_rev9}
J. Pascher, \emph{g-2 Analysis Rev9}, 2025.
\url{https://github.com/jpascher/T0-Time-Mass-Duality/blob/main/2/pdf/T0_Anomale-g2-9_En.pdf}

\bibitem{pascher:ml_addendum}
J. Pascher, \emph{ML Addendum}, 2025.
\url{https://github.com/jpascher/T0-Time-Mass-Duality/blob/main/2/pdf/T0-QFT-ML_Addendum_En.pdf}

\bibitem{pascher_beta_derivation_2025}
J. Pascher, \emph{Beta Derivation}, 2025.
\url{https://github.com/jpascher/T0-Time-Mass-Duality/blob/main/2/pdf/DerivationVonBetaEn.pdf}

\bibitem{pascher_cmb_en}
J. Pascher, \emph{CMB Analysis in T0}, 2025.
\url{https://github.com/jpascher/T0-Time-Mass-Duality/blob/main/2/pdf/Zwei-Dipole-CMB_En.pdf}

\bibitem{pascher_cosmos_en}
J. Pascher, \emph{Cosmos in T0 Theory}, 2025.
\url{https://github.com/jpascher/T0-Time-Mass-Duality/blob/main/2/pdf/cosmic_En.pdf}

\bibitem{pascher_derivation_beta_2025}
J. Pascher, \emph{Derivation of Beta}, 2025.
\url{https://github.com/jpascher/T0-Time-Mass-Duality/blob/main/2/pdf/DerivationVonBetaEn.pdf}

\bibitem{pascher_gravitation_en}
J. Pascher, \emph{Gravitation in T0}, 2025.
\url{https://github.com/jpascher/T0-Time-Mass-Duality/blob/main/2/pdf/gravitationskonstante_En.pdf}

\bibitem{pascher_lagrangian_2025}
J. Pascher, \emph{Lagrangian in T0}, 2025.
\url{https://github.com/jpascher/T0-Time-Mass-Duality/blob/main/2/pdf/T0_lagrndian_En.pdf}

\bibitem{pascher_lagrangian_en}
J. Pascher, \emph{Lagrangian Framework}, 2025.
\url{https://github.com/jpascher/T0-Time-Mass-Duality/blob/main/2/pdf/LagrandianVergleichEn.pdf}

\bibitem{pascher_lagrangian_extended_2025}
J. Pascher, \emph{Extended Lagrangian Formalism}, 2025.
\url{https://github.com/jpascher/T0-Time-Mass-Duality/blob/main/2/pdf/T0_lagrndian_En.pdf}

\bibitem{pascher_mathematical_structure_2025}
J. Pascher, \emph{Mathematical Structure of T0 Theory}, 2025.
\url{https://github.com/jpascher/T0-Time-Mass-Duality/blob/main/2/pdf/Mathematische_struktur_En.pdf}

\bibitem{pascher_muon_g2_2025}
J. Pascher, \emph{Muon g-2 in T0}, 2025.
\url{https://github.com/jpascher/T0-Time-Mass-Duality/blob/main/2/pdf/T0_Anomale-g2-9_En.pdf}

\bibitem{pascher_pragmatic_2025}
J. Pascher, \emph{Pragmatic Approach}, 2025.

\bibitem{pascher_t0_energy_2025}
J. Pascher, \emph{T0 Energy Formalism}, 2025.
\url{https://github.com/jpascher/T0-Time-Mass-Duality/blob/main/2/pdf/T0-Energie_En.pdf}

\bibitem{pascher_unified_2025}
J. Pascher, \emph{Unified T0 Theory}, 2025.
\url{https://github.com/jpascher/T0-Time-Mass-Duality/blob/main/2/pdf/T0_unified_report.pdf}

\bibitem{sciencedaily2025}
Science Daily, \emph{Physics News}, 2025.
\url{https://www.sciencedaily.com/}

\bibitem{weinberg_1989}
S. Weinberg, \emph{The Cosmological Constant Problem}, Rev. Mod. Phys., 1989.
\url{https://doi.org/10.1103/RevModPhys.61.1}

\bibitem{wiki_bell}
Wikipedia, \emph{Bell's Theorem}, 2025.
\url{https://en.wikipedia.org/wiki/Bell\%27s_theorem}

\bibitem{vanFraassen1980}
B. van Fraassen, \emph{The Scientific Image}, Oxford University Press, 1980.

\bibitem{terrell_single_clock_nature_2024}
J. Terrell, \emph{Single Clock Nature}, Nature, 2024.

% Additional T0 Documents
\bibitem{137_doc}
J. Pascher, \emph{The Number 137 in T0 Theory}, 2025.
\url{https://github.com/jpascher/T0-Time-Mass-Duality/blob/main/2/pdf/137_En.pdf}

\bibitem{ampere_low}
J. Pascher, \emph{Ampere's Law in T0}, 2025.
\url{https://github.com/jpascher/T0-Time-Mass-Duality/blob/main/2/pdf/Amper_Low_En.pdf}

\bibitem{bell_theorem}
J. Pascher, \emph{Bell's Theorem in T0}, 2025.
\url{https://github.com/jpascher/T0-Time-Mass-Duality/blob/main/2/pdf/Bell_En.pdf}

\bibitem{bewegungsenergie}
J. Pascher, \emph{Kinetic Energy in T0}, 2025.
\url{https://github.com/jpascher/T0-Time-Mass-Duality/blob/main/2/pdf/Bewegungsenergie_En.pdf}

\bibitem{emc2}
J. Pascher, \emph{E=mc² in T0 Framework}, 2025.
\url{https://github.com/jpascher/T0-Time-Mass-Duality/blob/main/2/pdf/E-mc2_En.pdf}

\bibitem{formeln_energiebasiert}
J. Pascher, \emph{Energy-Based Formulas}, 2025.
\url{https://github.com/jpascher/T0-Time-Mass-Duality/blob/main/2/pdf/Formeln_Energiebasiert_En.pdf}

\bibitem{hannah}
J. Pascher, \emph{Hannah Document}, 2025.
\url{https://github.com/jpascher/T0-Time-Mass-Duality/blob/main/2/pdf/Hannah_En.pdf}

\bibitem{ho_doc}
J. Pascher, \emph{H0 Analysis}, 2025.
\url{https://github.com/jpascher/T0-Time-Mass-Duality/blob/main/2/pdf/Ho_En.pdf}

\bibitem{markov}
J. Pascher, \emph{Markov Processes in T0}, 2025.
\url{https://github.com/jpascher/T0-Time-Mass-Duality/blob/main/2/pdf/Markov_En.pdf}

\bibitem{elimination_mass}
J. Pascher, \emph{Elimination of Mass}, 2025.
\url{https://github.com/jpascher/T0-Time-Mass-Duality/blob/main/2/pdf/EliminationOfMassEn.pdf}

\bibitem{elimination_mass_dirac}
J. Pascher, \emph{Dirac Equation Mass Elimination}, 2025.
\url{https://github.com/jpascher/T0-Time-Mass-Duality/blob/main/2/pdf/Elimination_Of_Mass_Dirac_TabelleEn.pdf}

\bibitem{feinstrukturkonstante}
J. Pascher, \emph{Fine Structure Constant}, 2025.
\url{https://github.com/jpascher/T0-Time-Mass-Duality/blob/main/2/pdf/FeinstrukturkonstanteEn.pdf}

\bibitem{neutrino_formel}
J. Pascher, \emph{Neutrino Formula}, 2025.
\url{https://github.com/jpascher/T0-Time-Mass-Duality/blob/main/2/pdf/neutrino-Formel_En.pdf}

\bibitem{neutrinos}
J. Pascher, \emph{Neutrinos in T0}, 2025.
\url{https://github.com/jpascher/T0-Time-Mass-Duality/blob/main/2/pdf/T0_Neutrinos_En.pdf}

\bibitem{koide_formel}
J. Pascher, \emph{Koide Formula in T0}, 2025.
\url{https://github.com/jpascher/T0-Time-Mass-Duality/blob/main/2/pdf/T0_koide-formel-3_En.pdf}

\bibitem{teilchenmassen}
J. Pascher, \emph{Particle Masses}, 2025.
\url{https://github.com/jpascher/T0-Time-Mass-Duality/blob/main/2/pdf/Teilchenmassen_En.pdf}

\bibitem{t0_teilchenmassen}
J. Pascher, \emph{T0 Particle Masses}, 2025.
\url{https://github.com/jpascher/T0-Time-Mass-Duality/blob/main/2/pdf/T0_Teilchenmassen_En.pdf}

\bibitem{penrose_doc}
J. Pascher, \emph{Penrose Analysis in T0}, 2025.
\url{https://github.com/jpascher/T0-Time-Mass-Duality/blob/main/2/pdf/T0_penrose_En.pdf}

\bibitem{photonenchip}
J. Pascher, \emph{Photon Chip Implementation}, 2025.
\url{https://github.com/jpascher/T0-Time-Mass-Duality/blob/main/2/pdf/T0_photonenchip-china_En.pdf}

\bibitem{threeclock}
J. Pascher, \emph{Three Clock Experiment}, 2025.
\url{https://github.com/jpascher/T0-Time-Mass-Duality/blob/main/2/pdf/T0_threeclock_En.pdf}

\bibitem{redshift_deflection}
J. Pascher, \emph{Redshift and Deflection}, 2025.
\url{https://github.com/jpascher/T0-Time-Mass-Duality/blob/main/2/pdf/redshift_deflection_En.pdf}

\bibitem{scheinbar_instantan}
J. Pascher, \emph{Apparent Instantaneity}, 2025.
\url{https://github.com/jpascher/T0-Time-Mass-Duality/blob/main/2/pdf/scheinbar_instantan_En.pdf}

\bibitem{universale_ableitung}
J. Pascher, \emph{Universal Derivation}, 2025.
\url{https://github.com/jpascher/T0-Time-Mass-Duality/blob/main/2/pdf/universale-ableitung_En.pdf}

\bibitem{xi_parameter}
J. Pascher, \emph{Xi Parameter for Particles}, 2025.
\url{https://github.com/jpascher/T0-Time-Mass-Duality/blob/main/2/pdf/xi_parmater_partikel_En.pdf}

\bibitem{xi_ursprung}
J. Pascher, \emph{Origin of Xi}, 2025.
\url{https://github.com/jpascher/T0-Time-Mass-Duality/blob/main/2/pdf/T0_xi_ursprung_En.pdf}

\bibitem{zeit}
J. Pascher, \emph{Time in T0 Theory}, 2025.
\url{https://github.com/jpascher/T0-Time-Mass-Duality/blob/main/2/pdf/Zeit_En.pdf}

\bibitem{zeit_konstant}
J. Pascher, \emph{Time Constant}, 2025.
\url{https://github.com/jpascher/T0-Time-Mass-Duality/blob/main/2/pdf/Zeit-konstant_En.pdf}

\bibitem{zusammenfassung}
J. Pascher, \emph{Summary of T0 Theory}, 2025.
\url{https://github.com/jpascher/T0-Time-Mass-Duality/blob/main/2/pdf/Zusammenfassung_En.pdf}

\bibitem{rsa}
J. Pascher, \emph{RSA in T0 Framework}, 2025.
\url{https://github.com/jpascher/T0-Time-Mass-Duality/blob/main/2/pdf/RSA_En.pdf}

\bibitem{qat}
J. Pascher, \emph{Quantum Atomic Theory}, 2025.
\url{https://github.com/jpascher/T0-Time-Mass-Duality/blob/main/2/pdf/T0_QAT_En.pdf}

\bibitem{qm_qft_rt}
J. Pascher, \emph{QM, QFT and RT Unification}, 2025.
\url{https://github.com/jpascher/T0-Time-Mass-Duality/blob/main/2/pdf/T0_QM-QFT-RT_En.pdf}

\bibitem{qm_optimierung}
J. Pascher, \emph{QM Optimization}, 2025.
\url{https://github.com/jpascher/T0-Time-Mass-Duality/blob/main/2/pdf/T0_QM-optimierung_En.pdf}

\bibitem{vollstaendige_berechnungen}
J. Pascher, \emph{Complete Calculations}, 2025.
\url{https://github.com/jpascher/T0-Time-Mass-Duality/blob/main/2/pdf/T0_Vollstaendige_Berchnungen_En.pdf}

\bibitem{synergetics}
J. Pascher, \emph{T0 Theory vs Synergetics}, 2025.
\url{https://github.com/jpascher/T0-Time-Mass-Duality/blob/main/2/pdf/T0-Theory-vs-Synergetics_En.pdf}

\bibitem{modell_uebersicht}
J. Pascher, \emph{T0 Model Overview}, 2025.
\url{https://github.com/jpascher/T0-Time-Mass-Duality/blob/main/2/pdf/T0_Modell_Uebersicht_En.pdf}

\bibitem{mnras_widerlegung}
J. Pascher, \emph{MNRAS Analysis}, 2025.
\url{https://github.com/jpascher/T0-Time-Mass-Duality/blob/main/2/pdf/T0_Analyse_MNRAS_Widerlegung_En.pdf}

\bibitem{anomale_magnetische_momente}
J. Pascher, \emph{Anomalous Magnetic Moments}, 2025.
\url{https://github.com/jpascher/T0-Time-Mass-Duality/blob/main/2/pdf/T0_Anomale_Magnetische_Momente_En.pdf}

\bibitem{sieben_fragen}
J. Pascher, \emph{Seven Questions in T0}, 2025.
\url{https://github.com/jpascher/T0-Time-Mass-Duality/blob/main/2/pdf/T0_7-fragen-3_En.pdf}

\bibitem{detailierte_leptonen}
J. Pascher, \emph{Detailed Lepton Anomaly}, 2025.
\url{https://github.com/jpascher/T0-Time-Mass-Duality/blob/main/2/pdf/detailierte_formel_leptonen_anemal_En.pdf}

\bibitem{parameterherleitung}
J. Pascher, \emph{Parameter Derivation}, 2025.
\url{https://github.com/jpascher/T0-Time-Mass-Duality/blob/main/2/pdf/parameterherleitung_En.pdf}

\bibitem{verhaeltnis_absolut}
J. Pascher, \emph{Absolute Ratios in T0}, 2025.
\url{https://github.com/jpascher/T0-Time-Mass-Duality/blob/main/2/pdf/T0_verhaeltnis-absolut_En.pdf}

\bibitem{xi_und_e}
J. Pascher, \emph{Xi and Energy}, 2025.
\url{https://github.com/jpascher/T0-Time-Mass-Duality/blob/main/2/pdf/T0_xi-und-e_En.pdf}

\bibitem{umkehrung}
J. Pascher, \emph{Inversion in T0}, 2025.
\url{https://github.com/jpascher/T0-Time-Mass-Duality/blob/main/2/pdf/T0_umkehrung_En.pdf}

\bibitem{esm_analysis}
J. Pascher, \emph{T0 vs ESM Conceptual Analysis}, 2025.
\url{https://github.com/jpascher/T0-Time-Mass-Duality/blob/main/2/pdf/T0vsESM_ConceptualAnalysis_En.pdf}

\end{thebibliography}

\end{document}
