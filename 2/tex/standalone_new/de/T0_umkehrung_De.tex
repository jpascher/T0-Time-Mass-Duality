% Standalone document: T0_umkehrung_En
% Uses shared T0 header
% T0 Standalone Header - German Version
% Gemeinsamer Header für alle deutschen Standalone-Dokumente

\documentclass[12pt,a4paper]{article}
\usepackage[utf8]{inputenc}
\usepackage[T1]{fontenc}
\usepackage[ngerman]{babel}
\usepackage{lmodern}

% Mathematics
\usepackage{amsmath,amssymb,amsthm}
\usepackage{physics}
\usepackage{siunitx}

% Layout
\usepackage[left=2.5cm,right=2.5cm,top=2.5cm,bottom=2.5cm,headheight=15pt]{geometry}
\usepackage{fancyhdr}
\usepackage{titlesec}

% Tables and Graphics
\usepackage{booktabs}
\usepackage{array}
\usepackage{longtable}
\usepackage{graphicx}
\usepackage{tikz}
\usetikzlibrary{arrows.meta,positioning,shapes.geometric}

% Colors and Boxes
\usepackage{xcolor}
\usepackage[most]{tcolorbox}
\usepackage{mdframed}

% Additional packages
\usepackage{enumitem}
\usepackage{float}
\usepackage{caption}
\usepackage{subcaption}
\usepackage{multirow}
\usepackage{colortbl}
\usepackage{pdflscape}
\usepackage{algorithm}
\usepackage{algpseudocode}
\usepackage{listings}
\usepackage{hyperref}

% Define colors
\definecolor{t0blue}{RGB}{0,51,102}
\definecolor{t0green}{RGB}{0,102,51}
\definecolor{t0red}{RGB}{153,0,0}
\definecolor{deepblue}{RGB}{0,51,102}
\definecolor{deepgreen}{RGB}{0,102,51}
\definecolor{deepred}{RGB}{153,0,0}
\definecolor{boxgray}{RGB}{240,240,240}
\definecolor{t0yellow}{RGB}{255,200,0}
\definecolor{boxblue}{RGB}{230,240,255}
\definecolor{boxgreen}{RGB}{230,255,230}
\definecolor{boxorange}{RGB}{255,240,230}
\definecolor{boxyellow}{RGB}{255,255,230}

% Custom tcolorbox environments
\newtcolorbox{fundamental}[1][]{
  colback=blue!5!white,
  colframe=blue!75!black,
  title=#1,
  fonttitle=\bfseries,
  breakable
}

\newtcolorbox{derivation}[1][]{
  colback=green!5!white,
  colframe=green!75!black,
  title=#1,
  fonttitle=\bfseries,
  breakable
}

\newtcolorbox{result}[1][]{
  colback=orange!5!white,
  colframe=orange!75!black,
  title=#1,
  fonttitle=\bfseries,
  breakable
}

\newtcolorbox{summary}[1][]{
  colback=gray!10!white,
  colframe=gray!75!black,
  title=#1,
  fonttitle=\bfseries,
  breakable
}

\newtcolorbox{comparison}[1][]{
  colback=purple!5!white,
  colframe=purple!75!black,
  title=#1,
  fonttitle=\bfseries,
  breakable
}

\newtcolorbox{relation}[1][]{
  colback=cyan!5!white,
  colframe=cyan!75!black,
  title=#1,
  fonttitle=\bfseries,
  breakable
}

\newtcolorbox{principle}[1][]{
  colback=yellow!5!white,
  colframe=yellow!75!black,
  title=#1,
  fonttitle=\bfseries,
  breakable
}

\newtcolorbox{insight}[1][]{colback=blue!5,colframe=t0blue,title={#1},fonttitle=\bfseries,breakable}
\newtcolorbox{discovery}[1][]{colback=green!5,colframe=t0green,title={#1},fonttitle=\bfseries,breakable}
\newtcolorbox{newperspective}[1][]{colback=yellow!5,colframe=orange,title={#1},fonttitle=\bfseries,breakable}
\newtcolorbox{revelation}[1][]{colback=red!5,colframe=t0red,title={#1},fonttitle=\bfseries,breakable}
\newtcolorbox{keypoint}[1][]{colback=blue!5,colframe=t0blue,title={#1},fonttitle=\bfseries,breakable}
\newtcolorbox{evidence}[1][]{colback=green!5,colframe=t0green,title={#1},fonttitle=\bfseries,breakable}
\newtcolorbox{conclusion}[1][]{colback=gray!5,colframe=gray,title={#1},fonttitle=\bfseries,breakable}
\newtcolorbox{significance}[1][]{colback=yellow!5,colframe=orange,title={#1},fonttitle=\bfseries,breakable}
\newtcolorbox{philosophical}[1][]{colback=purple!5,colframe=purple,title={#1},fonttitle=\bfseries,breakable}
\newtcolorbox{implication}[1][]{colback=cyan!5,colframe=cyan,title={#1},fonttitle=\bfseries,breakable}
\newtcolorbox{perspective}[1][]{colback=blue!5,colframe=t0blue,title={#1},fonttitle=\bfseries,breakable}
\newtcolorbox{revolutionary}[1][]{colback=red!5,colframe=t0red,title={#1},fonttitle=\bfseries,breakable}
\newtcolorbox{technical}[1][]{colback=gray!5,colframe=gray!75!black,title={#1},fonttitle=\bfseries,breakable}
\newtcolorbox{notation}[1][]{colback=yellow!5,colframe=yellow!75!black,title={#1},fonttitle=\bfseries,breakable}

% Theorem environments
\newtheorem{theorem}{Satz}[section]
\newtheorem{lemma}[theorem]{Lemma}
\newtheorem{corollary}[theorem]{Korollar}
\newtheorem{proposition}[theorem]{Proposition}
\newtheorem{definition}[theorem]{Definition}
\newtheorem{example}[theorem]{Beispiel}
\newtheorem{remark}[theorem]{Bemerkung}
\newtheorem{note}[theorem]{Anmerkung}

% Additional environments
\newenvironment{treatise}{\begin{quote}}{\end{quote}}
\newenvironment{gemeinsam}{\begin{quote}}{\end{quote}}
\newenvironment{vergleich}{\begin{quote}}{\end{quote}}
\newenvironment{vorteil}{\begin{quote}}{\end{quote}}
\newenvironment{quantum}{\begin{quote}}{\end{quote}}

% T0-specific commands
\newcommand{\Tzero}{T$_0$}
\newcommand{\xipar}{\xi}
\newcommand{\Tfield}{T}
\newcommand{\Efield}{\mathcal{E}}
\newcommand{\meff}{m_{\text{eff}}}
\newcommand{\Eabs}{E_{\text{abs}}}
\newcommand{\taupar}{\tau}

% Header setup
\pagestyle{fancy}
\fancyhf{}
\fancyhead[L]{\leftmark}
\fancyhead[R]{\thepage}
\renewcommand{\headrulewidth}{0.4pt}

% Hyperref setup
\hypersetup{
    colorlinks=true,
    linkcolor=blue,
    filecolor=magenta,
    urlcolor=cyan,
    citecolor=blue,
    pdftitle={T0 Theory Document},
    pdfauthor={Johann Pascher}
}

% German quotation marks
%\newcommand{\dq}[1]{\glqq{}#1\grqq{}}


\title{Inversion}
\author{Johann Pascher}
\date{2025}

\begin{document}

\maketitle

\chapter{Inversion}

	
	
	\begin{abstract}
		The T0-Time-Mass-Duality theory derives fundamental Konstanten and masses Parameter-free from the universal geometrisch Parameter $\xi = 4/30000$. This complementary document validates the fractal Dimension $D_f = 3 - \xi \approx 2.99987$ through backward Ableitung from the experimentell Masse Verhältnis $r = m_{\mu} / m_e \approx 206.768$ (CODATA 2025). While \emph{ParticleMasses\_En.pdf} presents the systematic Masse Berechnung, dies document demonstrates the compelling geometrisch foundation. The independent Validierung confirms the consistency of T0-theory and demonstrates complete Parameter freedom.
	\end{abstract}
	
	\newpage
	
	\section{Einleitung}
	\label{T0_umkehrung:sec:introduction}
	
	\begin{important}{Document Complementarity}{}
		This document focuses on the \textbf{Validierung of fractal Dimension} $D_f$ from experimentell Lepton masses. It complements the main document \emph{ParticleMasses\_En.pdf}, welche presents the complete systematic Masse Berechnung for alle Fermionen.
	\end{important}
	
	Particle physics faces the fundamental problem of arbitrary Masse Parameter in the Standard Model. The T0-Time-Mass-Duality theory revolutionizes dies Ansatz through a vollständig Parameter-free Beschreibung.
	
	\section{Parameters and Basic Formulas}
	\label{T0_umkehrung:sec:parameters}
	
	The theory is basierend auf Zeit-Energie duality and fractal Raumzeit Struktur.
	
	\subsection{Exact Geometric Parameters}
	\label{T0_umkehrung:subsec:exact_parameters}
	
	\begin{align}
		\xi &= \frac{4}{30000} = \frac{1}{7500} \approx 1.333 \times 10^{-4}, \label{T0_umkehrung:eq:xi} \\
		D_f &= 3 - \xi \approx 2.99986667, \label{T0_umkehrung:eq:Df} \\
		\alpha &= \frac{1 - \xi}{137} \approx 7.298 \times 10^{-3}, \label{T0_umkehrung:eq:alpha} \\
		K_{\text{frac}} &= 1 - 100 \xi \approx 0.9867, \label{T0_umkehrung:eq:K} \\
		g_{T0}^2 &= \alpha K_{\text{frac}}, \label{T0_umkehrung:eq:gT0} \\
		E_0 &= \frac{1}{\xi} \approx \SI{7500}{\giga\electronvolt}, \label{T0_umkehrung:eq:E0} \\
		p &= -\frac{2}{3}. \label{T0_umkehrung:eq:p}
	\end{align}
	
	\begin{result}{Fine Structure Constant Precision}{}
		The Abweichung of $\alpha$ from CODATA is nur $\approx 0.013\%$ -- strong Evidenz for the fractal Korrektur.
	\end{result}
	
	\section{Geometric Mass Derivation - Direct Method}
	\label{T0_umkehrung:sec:geometric_derivation}
	
	T0-theory offers several mathematically equivalent methods for Masse Berechnung. In dies document we use the \textbf{direct geometrisch method} spezifisch to validate the fractal Dimension.
	
	\subsection{Electron Mass $m_e$ - Direct Geometric Method}
	\label{T0_umkehrung:subsec:electron_mass}
	
	In the direct geometrisch method:
	\begin{align}
		m_e &= E_0 \cdot \xi \cdot \sqrt{\alpha} \cdot \frac{\Gamma(D_f)}{\Gamma(3)} \approx \SI{5.10e-4}{\giga\electronvolt}. \label{T0_umkehrung:eq:me_direct}
	\end{align}
	
	\textbf{Experimentell Validation:} Deviation from CODATA ($\SI{0.000511}{\giga\electronvolt}$): $-0.20\%$.
	
	\subsection{Consistency Check with Main Document}
	\label{T0_umkehrung:subsec:consistency_check}
	
	\begin{table}[H]
		\centering
		\resizebox{\textwidth}{!}{%
MATHBLOCK47ENDMATH}
		\caption{Consistency of mass calculation methods in T0-theory}
		\label{T0_umkehrung:tab:method_consistency}
	\end{table}
	
	\begin{result}{Method Equivalence}{}
		Both Berechnung methods yield identical results innerhalb $0.2\%$ -- excellent consistency for a Parameter-free theory. The direct geometrisch method validates the fractal Dimension, while the Yukawa method bridges to the Standard Model.
	\end{result}
	
	\subsection{Effective Torsion Mass $m_T$}
	\label{T0_umkehrung:subsec:torsion_mass}
	
	\begin{align}
		R_f &= \frac{\Gamma(D_f)}{\Gamma(3)} \sqrt{\frac{E_0}{m_e}}, \label{T0_umkehrung:eq:Rf} \\
		m_T &= \frac{m_e}{\xi} \sin(\pi \xi) \, \pi^2 \sqrt{\frac{\alpha}{K_{\text{frac}}}} \, R_f \approx \SI{5.220}{\giga\electronvolt}. \label{T0_umkehrung:eq:mT}
	\end{align}
	
	\subsection{Muon Mass $m_{\mu}$}
	\label{T0_umkehrung:subsec:muon_mass}
	
	From RG-duality and loop integral $I$:
	\begin{align}
		I &= \int_0^1 \frac{m_e^2 x (1-x)^2}{m_e^2 x^2 + m_T^2 (1-x)}  dx \approx 6.82 \times 10^{-5}, \label{T0_umkehrung:eq:I} \\
		r &\approx \sqrt{6 I}, \label{T0_umkehrung:eq:r} \\
		m_{\mu} &\approx m_T \cdot r \approx \SI{0.10566}{\giga\electronvolt}. \label{T0_umkehrung:eq:mmu}
	\end{align}
	
	\textbf{Experimentell Validation:} Deviation from CODATA ($\SI{0.105658}{\giga\electronvolt}$): $+0.002\%$.
	
	\begin{important}{Mass Ratio Validation}{}
		The berechnet Masse Verhältnis $r = m_{\mu} / m_e \approx 207.00$ deviates nur $+0.11\%$ from CODATA -- excellent agreement. This independent Validierung confirms the geometrisch foundation.
	\end{important}
	
	\section{Backward Validation: $D_f$ from $r$ and Nambu Formula}
	\label{T0_umkehrung:sec:backward_validation}
	
	The klassisch Nambu Formel $r \approx (3/2)/\alpha$ (dev. $-0.58\%$) is refined by the $\xi$-Korrektur.
	
	\subsection{Nambu Inversion}
	\label{T0_umkehrung:subsec:nambu_inversion}
	
	\begin{align}
		m_T^{\text{target}} &= \frac{m_{\mu}}{\sqrt{\alpha} \cdot (3/2) \cdot (1 - \xi)} \approx \SI{5.220}{\giga\electronvolt}. \label{T0_umkehrung:eq:mTtarget}
	\end{align}
	
	\subsection{Optimization for $D_f$}
	\label{T0_umkehrung:subsec:optimization_df}
	
	Define $m_T(D_f)$ gemäß Gleichung~\ref{T0_umkehrung:eq:mT} and solve:
	\begin{align}
		D_f = \arg\min \left| m_T(D_f) - m_T^{\text{target}} \right|. \label{T0_umkehrung:eq:optDf}
	\end{align}
	
	\begin{keyresult}{Compelling Fractal Dimension}{}
		Result: $D_f \approx 2.99986667$ (Abweichung from $3 - \xi$: $0.000000\%$). \\
		\textbf{This proves:} The experimentell Masse Verhältnis compels the fractal Geometrie -- no free Parameter! This independent Validierung confirms the foundations of \emph{ParticleMasses\_En.pdf}.
	\end{keyresult}
	
	\section{Application: Anomalous Magnetic Moment $a_{\mu}^{\text{T0}}$}
	\label{T0_umkehrung:sec:application_g2}
	
	With the derived fractal Dimension $D_f$ and geometrisch masses:
	\begin{align}
		F_2^{\text{T0}}(0) &= \frac{g_{T0}^2}{8 \pi^2} I_{\mu} K_{\text{frac}}, \label{T0_umkehrung:eq:F2} \\
		\text{term} &= \left( \frac{\xi E_0}{m_T} \right)^p = m_T^{2/3}, \label{T0_umkehrung:eq:term} \\
		F_{\text{dual}} &= \frac{1}{1 + \text{term}} \approx 0.249, \label{T0_umkehrung:eq:Fdual} \\
		a_{\mu}^{\text{T0}} &= F_2^{\text{T0}}(0) \cdot F_{\text{dual}} \approx 1.53 \times 10^{-9} = 153 \times 10^{-11}. \label{T0_umkehrung:eq:amu}
	\end{align}
	
	\begin{result}{Experimentell Validation}{}
		Deviation from benchmark ($143 \times 10^{-11}$): $\sim 7\%$ ($0.15\sigma$ to 2025 data).
	\end{result}
	
	\section{Python Implementation and Reproducibility}
	\label{T0_umkehrung:sec:python_implementation}
	
	\begin{important}{Full Transparency}{}
		For reproduction of alle numerisch Berechnungen see the external script \texttt{t0\_df\_from\_masses\_geometry.py} in the repository folder.
	\end{important}
	
	\section{Zusammenfassung and Scientific Significance}
	\label{T0_umkehrung:sec:summary}
	
	\subsection{Theoretical Significance of Validation}
	\label{T0_umkehrung:subsec:theoretical_significance}
	
	This document provides independent Validierung of the geometrisch foundations:
	\begin{itemize}
		\item \textbf{Parameter Freedom:} $D_f$ is compelled by experimentell masses
		\item \textbf{Method Consistency:} Independent Bestätigung of \emph{ParticleMasses\_En.pdf}
		\item \textbf{Geometric Foundation:} Experimentell data determines Raumzeit Struktur
		\item \textbf{Predictive Power:} Testable Konsequenzen for g-2 and new physics
	\end{itemize}
	
	\subsection{Complementary Document Structure}
	\label{T0_umkehrung:subsec:document_structure}
	
	\begin{table}[H]
		\centering
		\resizebox{\textwidth}{!}{%
MATHBLOCK48ENDMATH}
		\caption{Complementary roles of T0-theory documents}
		\label{T0_umkehrung:tab:document_complementarity}
	\end{table}
	
	\begin{important}{Scientific Strategy}{}
		This complementary document Struktur follows proven scientific methodology: A main document presents the complete System, while Validierung documents independently confirm specific Aspekte.
	\end{important}
	
	\section{Literaturverzeichnis}
	\label{T0_umkehrung:sec:references}
	
	\begin{itemize}
		\item Pascher, J. (2025). \emph{T0-Model: Complete Parameter-Free Particle Mass Calculation} (ParticleMasses\_En.pdf). Available at: \url{https://github.com/jpascher/T0-Time-Mass-Duality/tree/main/2/pdf/ParticleMasses_En.pdf}
		
		\item Pascher, J. (2025). \emph{T0-Time-Mass-Duality Repository}, GitHub v1.6. Available at: \url{https://github.com/jpascher/T0-Time-Mass-Duality}
		
		\item CODATA (2025). \emph{Fundamental Physical Constants}, NIST.
	\end{itemize}
	

\begin{thebibliography}{99}

% ============================================
% Core T0 Theory References (J. Pascher)
% GitHub Repository: https://github.com/jpascher/T0-Time-Mass-Duality
% ============================================

\bibitem{pascher2024}
J. Pascher, \emph{T0 Theory: Time-Mass Duality}, 2024.
\url{https://github.com/jpascher/T0-Time-Mass-Duality/blob/main/2/pdf/T0_unified_report.pdf}

\bibitem{pascher2025t0}
J. Pascher, \emph{T0 Theory: Fundamentals}, 2025.
\url{https://github.com/jpascher/T0-Time-Mass-Duality/blob/main/2/pdf/T0_Grundlagen_En.pdf}

\bibitem{pascher2025qm}
J. Pascher, \emph{T0 Theory: Quantum Mechanics}, 2025.
\url{https://github.com/jpascher/T0-Time-Mass-Duality/blob/main/2/pdf/QM_En.pdf}

\bibitem{pascher2025si}
J. Pascher, \emph{T0 Theory: SI Units}, 2025.
\url{https://github.com/jpascher/T0-Time-Mass-Duality/blob/main/2/pdf/T0_SI_En.pdf}

\bibitem{pascher2025g2}
J. Pascher, \emph{T0 Theory: The g-2 Anomaly}, 2025.
\url{https://github.com/jpascher/T0-Time-Mass-Duality/blob/main/2/pdf/T0_Anomale-g2-9_En.pdf}

\bibitem{pascher2025cmb}
J. Pascher, \emph{T0 Theory: CMB Analysis}, 2025.
\url{https://github.com/jpascher/T0-Time-Mass-Duality/blob/main/2/pdf/Zwei-Dipole-CMB_En.pdf}

% Historical Physics
\bibitem{einstein1905}
A. Einstein, \emph{On the Electrodynamics of Moving Bodies}, Annalen der Physik, 1905.
\url{https://doi.org/10.1002/andp.19053221004}

\bibitem{dirac1928}
P.A.M. Dirac, \emph{The Quantum Theory of the Electron}, Proc. Roy. Soc. A, 1928.
\url{https://doi.org/10.1098/rspa.1928.0023}

\bibitem{planck1900}
M. Planck, \emph{On the Theory of the Energy Distribution Law}, 1900.
\url{https://doi.org/10.1002/andp.19013090310}

\bibitem{mach1883}
E. Mach, \emph{Die Mechanik in ihrer Entwicklung}, 1883.

\bibitem{hundert1931}
Various Authors, \emph{100 Authors Against Einstein}, 1931.

\bibitem{dingle1972}
H. Dingle, \emph{Science at the Crossroads}, 1972.

% Penrose and Terrell Effect
\bibitem{terrell1959}
J. Terrell, \emph{Invisibility of the Lorentz Contraction}, Phys. Rev., 1959.
\url{https://doi.org/10.1103/PhysRev.116.1041}

\bibitem{penrose1959}
R. Penrose, \emph{The Apparent Shape of a Relativistically Moving Sphere}, Proc. Cambridge Phil. Soc., 1959.
\url{https://doi.org/10.1017/S0305004100033776}

\bibitem{penrose1967}
R. Penrose, \emph{Twistor Algebra}, J. Math. Phys., 1967.
\url{https://doi.org/10.1063/1.1705200}

\bibitem{penrose2004}
R. Penrose, \emph{The Road to Reality}, 2004.

\bibitem{terrell2025}
J. Terrell et al., \emph{Modern Terrell-Penrose Visualization}, 2025.

\bibitem{weiskopf2000}
D. Weiskopf, \emph{Visualization of Four-dimensional Spacetimes}, 2000.

\bibitem{mueller2014}
T. Müller, \emph{Visual Appearance of Relativistically Moving Objects}, 2014.

\bibitem{hossenfelder2025}
S. Hossenfelder, \emph{YouTube: The Terrell Effect}, 2025.

% Quantum Gravity and String Theory
\bibitem{rovelli2004}
C. Rovelli, \emph{Quantum Gravity}, Cambridge University Press, 2004.

\bibitem{thiemann2007}
T. Thiemann, \emph{Modern Canonical Quantum Gravity}, Cambridge University Press, 2007.

\bibitem{ashtekar2004}
A. Ashtekar, J. Lewandowski, \emph{Background Independent Quantum Gravity}, Class. Quant. Grav., 2004.
\url{https://doi.org/10.1088/0264-9381/21/15/R01}

\bibitem{jacobson1995}
T. Jacobson, \emph{Thermodynamics of Spacetime}, Phys. Rev. Lett., 1995.
\url{https://doi.org/10.1103/PhysRevLett.75.1260}

\bibitem{maldacena1998}
J. Maldacena, \emph{The Large N Limit of Superconformal Field Theories}, Adv. Theor. Math. Phys., 1998.
\url{https://doi.org/10.4310/ATMP.1998.v2.n2.a1}

\bibitem{polchinski1998}
J. Polchinski, \emph{String Theory}, Cambridge University Press, 1998.

\bibitem{susskind1995}
L. Susskind, \emph{The World as a Hologram}, J. Math. Phys., 1995.
\url{https://doi.org/10.1063/1.531249}

\bibitem{verlinde2011}
E. Verlinde, \emph{On the Origin of Gravity}, JHEP, 2011.
\url{https://doi.org/10.1007/JHEP04(2011)029}

% Cosmology
\bibitem{hoyle1948}
F. Hoyle, \emph{A New Model for the Expanding Universe}, MNRAS, 1948.
\url{https://doi.org/10.1093/mnras/108.5.372}

\bibitem{bondi1948}
H. Bondi, T. Gold, \emph{The Steady-State Theory}, MNRAS, 1948.
\url{https://doi.org/10.1093/mnras/108.3.252}

\bibitem{zwicky1929}
F. Zwicky, \emph{On the Redshift of Spectral Lines}, Proc. Nat. Acad. Sci., 1929.
\url{https://doi.org/10.1073/pnas.15.10.773}

\bibitem{lopez2010}
C. Lopez-Corredoira, \emph{Tests of Cosmological Models}, Int. J. Mod. Phys. D, 2010.

\bibitem{lerner2014}
E. Lerner, \emph{Evidence for a Non-Expanding Universe}, 2014.

\bibitem{albrecht1999}
A. Albrecht, J. Magueijo, \emph{Variable Speed of Light}, Phys. Rev. D, 1999.
\url{https://doi.org/10.1103/PhysRevD.59.043516}

\bibitem{barrow1999}
J. Barrow, \emph{Cosmologies with Varying Light Speed}, Phys. Rev. D, 1999.
\url{https://doi.org/10.1103/PhysRevD.59.043515}

\bibitem{riess2022}
A. Riess et al., \emph{A Comprehensive Measurement of the Local Value of the Hubble Constant}, ApJ, 2022.
\url{https://doi.org/10.3847/2041-8213/ac5c5b}

\bibitem{desi2025}
DESI Collaboration, \emph{DESI Year 1 Results}, 2025.
\url{https://arxiv.org/abs/2404.03002}

\bibitem{divalentino2021}
E. Di Valentino et al., \emph{Planck Evidence for a Closed Universe}, Nat. Astron., 2021.
\url{https://doi.org/10.1038/s41550-019-0906-9}

% Conformal Field Theory
\bibitem{francesco1997}
P. Di Francesco et al., \emph{Conformal Field Theory}, Springer, 1997.

% Experimental Physics
\bibitem{pdg2024}
Particle Data Group, \emph{Review of Particle Physics}, 2024.
\url{https://pdg.lbl.gov/}

\bibitem{codata2019}
CODATA, \emph{Recommended Values of Fundamental Constants}, 2019.
\url{https://physics.nist.gov/cuu/Constants/}

\bibitem{newell2018}
D. Newell et al., \emph{The CODATA 2017 Values of h, e, k, and $N_A$}, Metrologia, 2018.
\url{https://doi.org/10.1088/1681-7575/aa950a}

\bibitem{muong2_2023}
Muon g-2 Collaboration, \emph{Measurement of the Anomalous Magnetic Moment of the Muon}, Phys. Rev. Lett., 2023.
\url{https://doi.org/10.1103/PhysRevLett.131.161802}

\bibitem{fermilab2023}
Fermilab, \emph{Muon g-2 Results}, 2023.
\url{https://muon-g-2.fnal.gov/}

\bibitem{atlas2023}
ATLAS Collaboration, \emph{Measurements at the LHC}, 2023.
\url{https://atlas.cern/}

\bibitem{atlas2023higgs}
ATLAS Collaboration, \emph{Higgs Boson Properties}, 2023.
\url{https://atlas.cern/}

\bibitem{cms2023top}
CMS Collaboration, \emph{Top Quark Measurements}, 2023.
\url{https://cms.cern/}

\bibitem{cms2024}
CMS Collaboration, \emph{Heavy Ion Collisions}, 2024.
\url{https://cms.cern/}

\bibitem{alice2023}
ALICE Collaboration, \emph{Quark-Gluon Plasma Studies}, 2023.
\url{https://alice-collaboration.web.cern.ch/}

\bibitem{kasevich2023}
M. Kasevich et al., \emph{Atom Interferometry}, 2023.

\bibitem{ludlow2015}
A. Ludlow et al., \emph{Optical Atomic Clocks}, Rev. Mod. Phys., 2015.
\url{https://doi.org/10.1103/RevModPhys.87.637}

\bibitem{brewer2019}
S. Brewer et al., \emph{Al$^+$ Optical Clock}, Phys. Rev. Lett., 2019.
\url{https://doi.org/10.1103/PhysRevLett.123.033201}

\bibitem{lisa2017}
LISA Collaboration, \emph{LISA Mission}, 2017.
\url{https://www.lisamission.org/}

% Fractal Physics
\bibitem{nottale1993}
L. Nottale, \emph{Fractal Space-Time and Microphysics}, World Scientific, 1993.

\bibitem{elnaschie2004}
M.S. El Naschie, \emph{E-Infinity Theory}, Chaos Solitons Fractals, 2004.

% Philosophy and Foundations
\bibitem{wheeler1990}
J.A. Wheeler, \emph{Information, Physics, Quantum}, 1990.

\bibitem{barbour1999}
J. Barbour, \emph{The End of Time}, Oxford University Press, 1999.

\bibitem{sciama1953}
D. Sciama, \emph{On the Origin of Inertia}, MNRAS, 1953.
\url{https://doi.org/10.1093/mnras/113.1.34}

% String Theory Extensions
\bibitem{becker2007}
K. Becker et al., \emph{String Theory and M-Theory}, Cambridge University Press, 2007.

% Missing References for g-2 Chapter
\bibitem{sm_g2_2025}
Muon g-2 Theory Initiative, \emph{Standard Model Prediction for g-2}, arXiv, 2025.
\url{https://arxiv.org/abs/2006.04822}

\bibitem{mug2_final_2025}
Muon g-2 Collaboration, \emph{Final Report on the Anomalous Magnetic Moment of the Muon}, Fermilab, 2025.
\url{https://muon-g-2.fnal.gov/}

\bibitem{pascher_t0_theory_2025}
J. Pascher, \emph{T0 Theory: Complete Framework}, 2025.
\url{https://github.com/jpascher/T0-Time-Mass-Duality/blob/main/2/pdf/systemEn.pdf}

\bibitem{peskin_schroeder_1995}
M.E. Peskin and D.V. Schroeder, \emph{An Introduction to Quantum Field Theory}, Westview Press, 1995.

\bibitem{parker_somov_2018}
R.H. Parker et al., \emph{Measurement of the Fine-Structure Constant}, Science, 2018.
\url{https://doi.org/10.1126/science.aap7706}

\bibitem{morel_rubidium_2020}
L. Morel et al., \emph{Determination of $\alpha$ from Rubidium Atom Recoil}, Nature, 2020.
\url{https://doi.org/10.1038/s41586-020-2964-7}

\bibitem{aoyama_theory_2020}
T. Aoyama et al., \emph{Theory of the Electron Anomalous Magnetic Moment}, Phys. Rep., 2020.
\url{https://doi.org/10.1016/j.physrep.2020.07.006}

\bibitem{fan_lattice_2023}
X. Fan et al., \emph{Hadronic Contributions from Lattice QCD}, Phys. Rev. D, 2023.

\bibitem{hanneke_electron_2008}
D. Hanneke et al., \emph{New Measurement of the Electron g-2}, Phys. Rev. Lett., 2008.
\url{https://doi.org/10.1103/PhysRevLett.100.120801}

% Additional T0 Theory References
\bibitem{pascher_higgs_connection_2025}
J. Pascher, \emph{Higgs Connection in T0 Theory}, 2025.
\url{https://github.com/jpascher/T0-Time-Mass-Duality/blob/main/2/pdf/T0_Energie_En.pdf}

\bibitem{T0_SI}
J. Pascher, \emph{T0 Theory and SI Units}, 2025.
\url{https://github.com/jpascher/T0-Time-Mass-Duality/blob/main/2/pdf/T0_SI_En.pdf}

\bibitem{T0_gravitational_constant}
J. Pascher, \emph{Gravitational Constant in T0 Framework}, 2025.
\url{https://github.com/jpascher/T0-Time-Mass-Duality/blob/main/2/pdf/T0_Gravitationskonstante_En.pdf}

\bibitem{T0_fine_structure}
J. Pascher, \emph{Fine Structure Constant Analysis}, 2025.
\url{https://github.com/jpascher/T0-Time-Mass-Duality/blob/main/2/pdf/T0_Feinstruktur_En.pdf}

\bibitem{bell_muon}
J.S. Bell, \emph{Muon Studies}, 1966.

\bibitem{QFT_T0}
J. Pascher, \emph{Quantum Field Theory in T0}, 2025.
\url{https://github.com/jpascher/T0-Time-Mass-Duality/blob/main/2/pdf/QFT_En.pdf}

\bibitem{planck2018}
Planck Collaboration, \emph{Planck 2018 Results}, A\&A, 2018.
\url{https://doi.org/10.1051/0004-6361/201833910}

\bibitem{pascher:t0_foundations}
J. Pascher, \emph{T0 Theory Foundations}, 2025.
\url{https://github.com/jpascher/T0-Time-Mass-Duality/blob/main/2/pdf/T0_Grundlagen_En.pdf}

\bibitem{pascher:geometric_formalism}
J. Pascher, \emph{Geometric Formalism in T0}, 2025.
\url{https://github.com/jpascher/T0-Time-Mass-Duality/blob/main/2/pdf/T0_Geometrische_Kosmologie_En.pdf}

\bibitem{riess2019}
A. Riess et al., \emph{Hubble Constant Measurements}, ApJ, 2019.
\url{https://doi.org/10.3847/1538-4357/ab1422}

\bibitem{t0_kosmologie}
J. Pascher, \emph{T0 Kosmologie}, 2025.
\url{https://github.com/jpascher/T0-Time-Mass-Duality/blob/main/2/pdf/T0_Kosmologie_En.pdf}

\bibitem{hossenfelder_single_clock_video}
S. Hossenfelder, \emph{Single Clock Video}, YouTube, 2025.
\url{https://www.youtube.com/c/SabineHossenfelder}

\bibitem{video2025}
Various, \emph{Video References}, 2025.

\bibitem{unnikrishnan2004}
C.S. Unnikrishnan, \emph{Gravity Studies}, 2004.

\bibitem{peratt1992}
A. Peratt, \emph{Plasma Cosmology}, 1992.
\url{https://github.com/jpascher/T0-Time-Mass-Duality/blob/main/2/pdf/T0_peratt_En.pdf}

\bibitem{T0_tm_erweiterung}
J. Pascher, \emph{T0 Time-Mass Extension}, 2025.
\url{https://github.com/jpascher/T0-Time-Mass-Duality/blob/main/2/pdf/T0_tm-erweiterung-x6_En.pdf}

\bibitem{T0_g2_erweiterung}
J. Pascher, \emph{T0 g-2 Extension}, 2025.
\url{https://github.com/jpascher/T0-Time-Mass-Duality/blob/main/2/pdf/T0_g2-erweiterung-4_En.pdf}

\bibitem{T0_netze_en}
J. Pascher, \emph{T0 Networks}, 2025.
\url{https://github.com/jpascher/T0-Time-Mass-Duality/blob/main/2/pdf/T0_netze_En.pdf}

\bibitem{Adams1925}
W. Adams, \emph{Gravitational Redshift}, 1925.
\url{https://doi.org/10.1073/pnas.11.7.382}

\bibitem{Ashby2003}
N. Ashby, \emph{Relativity in GPS}, Living Rev. Rel., 2003.
\url{https://doi.org/10.12942/lrr-2003-1}

\bibitem{Bertotti2003}
B. Bertotti et al., \emph{Cassini Doppler Test}, Nature, 2003.
\url{https://doi.org/10.1038/nature01997}

\bibitem{Bolton2008}
A. Bolton et al., \emph{Gravitational Lensing}, 2008.

\bibitem{Born2013}
M. Born, \emph{Einstein's Theory of Relativity}, Dover, 2013.

\bibitem{Brans1961}
C. Brans and R.H. Dicke, \emph{Mach's Principle}, Phys. Rev., 1961.
\url{https://doi.org/10.1103/PhysRev.124.925}

\bibitem{Dirac1927}
P.A.M. Dirac, \emph{Quantum Mechanics}, Proc. Roy. Soc., 1927.
\url{https://doi.org/10.1098/rspa.1927.0039}

\bibitem{Duhem1906}
P. Duhem, \emph{Theory of Physics}, 1906.

\bibitem{Einstein1905}
A. Einstein, \emph{Special Relativity}, Ann. Phys., 1905.
\url{https://doi.org/10.1002/andp.19053221004}

\bibitem{Feynman2006}
R. Feynman, \emph{QED: The Strange Theory of Light and Matter}, 2006.

\bibitem{Griffiths2017}
D. Griffiths, \emph{Introduction to Quantum Mechanics}, 2017.

\bibitem{Jackson1999}
J.D. Jackson, \emph{Classical Electrodynamics}, 1999.

\bibitem{Kaluza1921}
T. Kaluza, \emph{Five-Dimensional Theory}, 1921.

\bibitem{Klein1926}
O. Klein, \emph{Quantum Theory and Relativity}, 1926.

\bibitem{Kuhn1962}
T. Kuhn, \emph{Structure of Scientific Revolutions}, 1962.

\bibitem{Kuhn1977}
T. Kuhn, \emph{Essential Tension}, 1977.

\bibitem{Ludlow2015}
A. Ludlow et al., \emph{Optical Atomic Clocks}, Rev. Mod. Phys., 2015.
\url{https://doi.org/10.1103/RevModPhys.87.637}

\bibitem{Maxwell1873}
J.C. Maxwell, \emph{Treatise on Electricity and Magnetism}, 1873.

\bibitem{McGaugh2016}
S. McGaugh et al., \emph{Radial Acceleration Relation}, Phys. Rev. Lett., 2016.
\url{https://doi.org/10.1103/PhysRevLett.117.201101}

\bibitem{Mohr2016}
P. Mohr et al., \emph{CODATA Values}, Rev. Mod. Phys., 2016.
\url{https://doi.org/10.1103/RevModPhys.88.035009}

\bibitem{PDG2020}
Particle Data Group, \emph{Review of Particle Physics}, Prog. Theor. Exp. Phys., 2020.
\url{https://pdg.lbl.gov/}

\bibitem{Parker2018}
R. Parker et al., \emph{Measurement of $\alpha$}, Science, 2018.
\url{https://doi.org/10.1126/science.aap7706}

\bibitem{Peskin1995}
M. Peskin and D. Schroeder, \emph{QFT}, 1995.

\bibitem{Planck1900}
M. Planck, \emph{Quantum Theory}, 1900.

\bibitem{Planck2020}
Planck Collaboration, \emph{Planck 2020 Results}, 2020.
\url{https://doi.org/10.1051/0004-6361/201833910}

\bibitem{Poincare1905}
H. Poincaré, \emph{Dynamics of the Electron}, 1905.

\bibitem{Pound1960}
R.V. Pound and G.A. Rebka, \emph{Gravitational Redshift}, Phys. Rev. Lett., 1960.
\url{https://doi.org/10.1103/PhysRevLett.4.337}

\bibitem{Quine1951}
W.V. Quine, \emph{Two Dogmas of Empiricism}, 1951.

\bibitem{Quinn2013}
T. Quinn et al., \emph{Gravitational Constant}, 2013.
\url{https://doi.org/10.1103/PhysRevLett.111.101102}

\bibitem{Randall1999}
L. Randall and R. Sundrum, \emph{Extra Dimensions}, Phys. Rev. Lett., 1999.
\url{https://doi.org/10.1103/PhysRevLett.83.3370}

\bibitem{Riess1998}
A. Riess et al., \emph{Type Ia Supernovae}, AJ, 1998.
\url{https://doi.org/10.1086/300499}

\bibitem{Shapiro1971}
I. Shapiro et al., \emph{Time Delay Test}, Phys. Rev. Lett., 1971.
\url{https://doi.org/10.1103/PhysRevLett.26.1132}

\bibitem{Sommerfeld1916}
A. Sommerfeld, \emph{Fine Structure}, 1916.

\bibitem{Suyu2017}
S. Suyu et al., \emph{Time Delay Cosmography}, MNRAS, 2017.
\url{https://doi.org/10.1093/mnras/stx483}

\bibitem{T0Theory}
J. Pascher, \emph{T0 Theory}, 2025.
\url{https://github.com/jpascher/T0-Time-Mass-Duality/blob/main/2/pdf/systemEn.pdf}

\bibitem{T0_Feinstruktur}
J. Pascher, \emph{Fine Structure in T0}, 2025.
\url{https://github.com/jpascher/T0-Time-Mass-Duality/blob/main/2/pdf/T0_Feinstruktur_En.pdf}

\bibitem{Uzan2003}
J.-P. Uzan, \emph{Constants Variation}, Rev. Mod. Phys., 2003.
\url{https://doi.org/10.1103/RevModPhys.75.403}

\bibitem{Webb2001}
J.K. Webb et al., \emph{Fine Structure Constant}, Phys. Rev. Lett., 2001.
\url{https://doi.org/10.1103/PhysRevLett.87.091301}

\bibitem{Weinberg1979}
S. Weinberg, \emph{Cosmological Constant}, Rev. Mod. Phys., 1979.

\bibitem{Weinberg1989}
S. Weinberg, \emph{Cosmological Constant Problem}, 1989.
\url{https://doi.org/10.1103/RevModPhys.61.1}

\bibitem{Weinberg1995}
S. Weinberg, \emph{Quantum Theory of Fields}, 1995.

\bibitem{Will2014}
C. Will, \emph{Theory and Experiment in Gravitational Physics}, 2014.
\url{https://doi.org/10.12942/lrr-2014-4}

\bibitem{dirac_principles}
P.A.M. Dirac, \emph{Principles of Quantum Mechanics}, 1930.

\bibitem{einstein_1917}
A. Einstein, \emph{Cosmological Considerations}, 1917.

\bibitem{jwst_early}
JWST Collaboration, \emph{Early Universe Observations}, 2023.
\url{https://www.jwst.nasa.gov/}

\bibitem{katrin_2022}
KATRIN Collaboration, \emph{Neutrino Mass}, 2022.
\url{https://doi.org/10.1038/s41567-021-01463-1}

\bibitem{pascher:fundamentals}
J. Pascher, \emph{T0 Fundamentals}, 2025.
\url{https://github.com/jpascher/T0-Time-Mass-Duality/blob/main/2/pdf/T0_Grundlagen_En.pdf}

\bibitem{pascher:g2_rev9}
J. Pascher, \emph{g-2 Analysis Rev9}, 2025.
\url{https://github.com/jpascher/T0-Time-Mass-Duality/blob/main/2/pdf/T0_Anomale-g2-9_En.pdf}

\bibitem{pascher:ml_addendum}
J. Pascher, \emph{ML Addendum}, 2025.
\url{https://github.com/jpascher/T0-Time-Mass-Duality/blob/main/2/pdf/T0-QFT-ML_Addendum_En.pdf}

\bibitem{pascher_beta_derivation_2025}
J. Pascher, \emph{Beta Derivation}, 2025.
\url{https://github.com/jpascher/T0-Time-Mass-Duality/blob/main/2/pdf/DerivationVonBetaEn.pdf}

\bibitem{pascher_cmb_en}
J. Pascher, \emph{CMB Analysis in T0}, 2025.
\url{https://github.com/jpascher/T0-Time-Mass-Duality/blob/main/2/pdf/Zwei-Dipole-CMB_En.pdf}

\bibitem{pascher_cosmos_en}
J. Pascher, \emph{Cosmos in T0 Theory}, 2025.
\url{https://github.com/jpascher/T0-Time-Mass-Duality/blob/main/2/pdf/cosmic_En.pdf}

\bibitem{pascher_derivation_beta_2025}
J. Pascher, \emph{Derivation of Beta}, 2025.
\url{https://github.com/jpascher/T0-Time-Mass-Duality/blob/main/2/pdf/DerivationVonBetaEn.pdf}

\bibitem{pascher_gravitation_en}
J. Pascher, \emph{Gravitation in T0}, 2025.
\url{https://github.com/jpascher/T0-Time-Mass-Duality/blob/main/2/pdf/gravitationskonstante_En.pdf}

\bibitem{pascher_lagrangian_2025}
J. Pascher, \emph{Lagrangian in T0}, 2025.
\url{https://github.com/jpascher/T0-Time-Mass-Duality/blob/main/2/pdf/T0_lagrndian_En.pdf}

\bibitem{pascher_lagrangian_en}
J. Pascher, \emph{Lagrangian Framework}, 2025.
\url{https://github.com/jpascher/T0-Time-Mass-Duality/blob/main/2/pdf/LagrandianVergleichEn.pdf}

\bibitem{pascher_lagrangian_extended_2025}
J. Pascher, \emph{Extended Lagrangian Formalism}, 2025.
\url{https://github.com/jpascher/T0-Time-Mass-Duality/blob/main/2/pdf/T0_lagrndian_En.pdf}

\bibitem{pascher_mathematical_structure_2025}
J. Pascher, \emph{Mathematical Structure of T0 Theory}, 2025.
\url{https://github.com/jpascher/T0-Time-Mass-Duality/blob/main/2/pdf/Mathematische_struktur_En.pdf}

\bibitem{pascher_muon_g2_2025}
J. Pascher, \emph{Muon g-2 in T0}, 2025.
\url{https://github.com/jpascher/T0-Time-Mass-Duality/blob/main/2/pdf/T0_Anomale-g2-9_En.pdf}

\bibitem{pascher_pragmatic_2025}
J. Pascher, \emph{Pragmatic Approach}, 2025.

\bibitem{pascher_t0_energy_2025}
J. Pascher, \emph{T0 Energy Formalism}, 2025.
\url{https://github.com/jpascher/T0-Time-Mass-Duality/blob/main/2/pdf/T0-Energie_En.pdf}

\bibitem{pascher_unified_2025}
J. Pascher, \emph{Unified T0 Theory}, 2025.
\url{https://github.com/jpascher/T0-Time-Mass-Duality/blob/main/2/pdf/T0_unified_report.pdf}

\bibitem{sciencedaily2025}
Science Daily, \emph{Physics News}, 2025.
\url{https://www.sciencedaily.com/}

\bibitem{weinberg_1989}
S. Weinberg, \emph{The Cosmological Constant Problem}, Rev. Mod. Phys., 1989.
\url{https://doi.org/10.1103/RevModPhys.61.1}

\bibitem{wiki_bell}
Wikipedia, \emph{Bell's Theorem}, 2025.
\url{https://en.wikipedia.org/wiki/Bell\%27s_theorem}

\bibitem{vanFraassen1980}
B. van Fraassen, \emph{The Scientific Image}, Oxford University Press, 1980.

\bibitem{terrell_single_clock_nature_2024}
J. Terrell, \emph{Single Clock Nature}, Nature, 2024.

% Additional T0 Documents
\bibitem{137_doc}
J. Pascher, \emph{The Number 137 in T0 Theory}, 2025.
\url{https://github.com/jpascher/T0-Time-Mass-Duality/blob/main/2/pdf/137_En.pdf}

\bibitem{ampere_low}
J. Pascher, \emph{Ampere's Law in T0}, 2025.
\url{https://github.com/jpascher/T0-Time-Mass-Duality/blob/main/2/pdf/Amper_Low_En.pdf}

\bibitem{bell_theorem}
J. Pascher, \emph{Bell's Theorem in T0}, 2025.
\url{https://github.com/jpascher/T0-Time-Mass-Duality/blob/main/2/pdf/Bell_En.pdf}

\bibitem{bewegungsenergie}
J. Pascher, \emph{Kinetic Energy in T0}, 2025.
\url{https://github.com/jpascher/T0-Time-Mass-Duality/blob/main/2/pdf/Bewegungsenergie_En.pdf}

\bibitem{emc2}
J. Pascher, \emph{E=mc² in T0 Framework}, 2025.
\url{https://github.com/jpascher/T0-Time-Mass-Duality/blob/main/2/pdf/E-mc2_En.pdf}

\bibitem{formeln_energiebasiert}
J. Pascher, \emph{Energy-Based Formulas}, 2025.
\url{https://github.com/jpascher/T0-Time-Mass-Duality/blob/main/2/pdf/Formeln_Energiebasiert_En.pdf}

\bibitem{hannah}
J. Pascher, \emph{Hannah Document}, 2025.
\url{https://github.com/jpascher/T0-Time-Mass-Duality/blob/main/2/pdf/Hannah_En.pdf}

\bibitem{ho_doc}
J. Pascher, \emph{H0 Analysis}, 2025.
\url{https://github.com/jpascher/T0-Time-Mass-Duality/blob/main/2/pdf/Ho_En.pdf}

\bibitem{markov}
J. Pascher, \emph{Markov Processes in T0}, 2025.
\url{https://github.com/jpascher/T0-Time-Mass-Duality/blob/main/2/pdf/Markov_En.pdf}

\bibitem{elimination_mass}
J. Pascher, \emph{Elimination of Mass}, 2025.
\url{https://github.com/jpascher/T0-Time-Mass-Duality/blob/main/2/pdf/EliminationOfMassEn.pdf}

\bibitem{elimination_mass_dirac}
J. Pascher, \emph{Dirac Equation Mass Elimination}, 2025.
\url{https://github.com/jpascher/T0-Time-Mass-Duality/blob/main/2/pdf/Elimination_Of_Mass_Dirac_TabelleEn.pdf}

\bibitem{feinstrukturkonstante}
J. Pascher, \emph{Fine Structure Constant}, 2025.
\url{https://github.com/jpascher/T0-Time-Mass-Duality/blob/main/2/pdf/FeinstrukturkonstanteEn.pdf}

\bibitem{neutrino_formel}
J. Pascher, \emph{Neutrino Formula}, 2025.
\url{https://github.com/jpascher/T0-Time-Mass-Duality/blob/main/2/pdf/neutrino-Formel_En.pdf}

\bibitem{neutrinos}
J. Pascher, \emph{Neutrinos in T0}, 2025.
\url{https://github.com/jpascher/T0-Time-Mass-Duality/blob/main/2/pdf/T0_Neutrinos_En.pdf}

\bibitem{koide_formel}
J. Pascher, \emph{Koide Formula in T0}, 2025.
\url{https://github.com/jpascher/T0-Time-Mass-Duality/blob/main/2/pdf/T0_koide-formel-3_En.pdf}

\bibitem{teilchenmassen}
J. Pascher, \emph{Particle Masses}, 2025.
\url{https://github.com/jpascher/T0-Time-Mass-Duality/blob/main/2/pdf/Teilchenmassen_En.pdf}

\bibitem{t0_teilchenmassen}
J. Pascher, \emph{T0 Particle Masses}, 2025.
\url{https://github.com/jpascher/T0-Time-Mass-Duality/blob/main/2/pdf/T0_Teilchenmassen_En.pdf}

\bibitem{penrose_doc}
J. Pascher, \emph{Penrose Analysis in T0}, 2025.
\url{https://github.com/jpascher/T0-Time-Mass-Duality/blob/main/2/pdf/T0_penrose_En.pdf}

\bibitem{photonenchip}
J. Pascher, \emph{Photon Chip Implementation}, 2025.
\url{https://github.com/jpascher/T0-Time-Mass-Duality/blob/main/2/pdf/T0_photonenchip-china_En.pdf}

\bibitem{threeclock}
J. Pascher, \emph{Three Clock Experiment}, 2025.
\url{https://github.com/jpascher/T0-Time-Mass-Duality/blob/main/2/pdf/T0_threeclock_En.pdf}

\bibitem{redshift_deflection}
J. Pascher, \emph{Redshift and Deflection}, 2025.
\url{https://github.com/jpascher/T0-Time-Mass-Duality/blob/main/2/pdf/redshift_deflection_En.pdf}

\bibitem{scheinbar_instantan}
J. Pascher, \emph{Apparent Instantaneity}, 2025.
\url{https://github.com/jpascher/T0-Time-Mass-Duality/blob/main/2/pdf/scheinbar_instantan_En.pdf}

\bibitem{universale_ableitung}
J. Pascher, \emph{Universal Derivation}, 2025.
\url{https://github.com/jpascher/T0-Time-Mass-Duality/blob/main/2/pdf/universale-ableitung_En.pdf}

\bibitem{xi_parameter}
J. Pascher, \emph{Xi Parameter for Particles}, 2025.
\url{https://github.com/jpascher/T0-Time-Mass-Duality/blob/main/2/pdf/xi_parmater_partikel_En.pdf}

\bibitem{xi_ursprung}
J. Pascher, \emph{Origin of Xi}, 2025.
\url{https://github.com/jpascher/T0-Time-Mass-Duality/blob/main/2/pdf/T0_xi_ursprung_En.pdf}

\bibitem{zeit}
J. Pascher, \emph{Time in T0 Theory}, 2025.
\url{https://github.com/jpascher/T0-Time-Mass-Duality/blob/main/2/pdf/Zeit_En.pdf}

\bibitem{zeit_konstant}
J. Pascher, \emph{Time Constant}, 2025.
\url{https://github.com/jpascher/T0-Time-Mass-Duality/blob/main/2/pdf/Zeit-konstant_En.pdf}

\bibitem{zusammenfassung}
J. Pascher, \emph{Summary of T0 Theory}, 2025.
\url{https://github.com/jpascher/T0-Time-Mass-Duality/blob/main/2/pdf/Zusammenfassung_En.pdf}

\bibitem{rsa}
J. Pascher, \emph{RSA in T0 Framework}, 2025.
\url{https://github.com/jpascher/T0-Time-Mass-Duality/blob/main/2/pdf/RSA_En.pdf}

\bibitem{qat}
J. Pascher, \emph{Quantum Atomic Theory}, 2025.
\url{https://github.com/jpascher/T0-Time-Mass-Duality/blob/main/2/pdf/T0_QAT_En.pdf}

\bibitem{qm_qft_rt}
J. Pascher, \emph{QM, QFT and RT Unification}, 2025.
\url{https://github.com/jpascher/T0-Time-Mass-Duality/blob/main/2/pdf/T0_QM-QFT-RT_En.pdf}

\bibitem{qm_optimierung}
J. Pascher, \emph{QM Optimization}, 2025.
\url{https://github.com/jpascher/T0-Time-Mass-Duality/blob/main/2/pdf/T0_QM-optimierung_En.pdf}

\bibitem{vollstaendige_berechnungen}
J. Pascher, \emph{Complete Calculations}, 2025.
\url{https://github.com/jpascher/T0-Time-Mass-Duality/blob/main/2/pdf/T0_Vollstaendige_Berchnungen_En.pdf}

\bibitem{synergetics}
J. Pascher, \emph{T0 Theory vs Synergetics}, 2025.
\url{https://github.com/jpascher/T0-Time-Mass-Duality/blob/main/2/pdf/T0-Theory-vs-Synergetics_En.pdf}

\bibitem{modell_uebersicht}
J. Pascher, \emph{T0 Model Overview}, 2025.
\url{https://github.com/jpascher/T0-Time-Mass-Duality/blob/main/2/pdf/T0_Modell_Uebersicht_En.pdf}

\bibitem{mnras_widerlegung}
J. Pascher, \emph{MNRAS Analysis}, 2025.
\url{https://github.com/jpascher/T0-Time-Mass-Duality/blob/main/2/pdf/T0_Analyse_MNRAS_Widerlegung_En.pdf}

\bibitem{anomale_magnetische_momente}
J. Pascher, \emph{Anomalous Magnetic Moments}, 2025.
\url{https://github.com/jpascher/T0-Time-Mass-Duality/blob/main/2/pdf/T0_Anomale_Magnetische_Momente_En.pdf}

\bibitem{sieben_fragen}
J. Pascher, \emph{Seven Questions in T0}, 2025.
\url{https://github.com/jpascher/T0-Time-Mass-Duality/blob/main/2/pdf/T0_7-fragen-3_En.pdf}

\bibitem{detailierte_leptonen}
J. Pascher, \emph{Detailed Lepton Anomaly}, 2025.
\url{https://github.com/jpascher/T0-Time-Mass-Duality/blob/main/2/pdf/detailierte_formel_leptonen_anemal_En.pdf}

\bibitem{parameterherleitung}
J. Pascher, \emph{Parameter Derivation}, 2025.
\url{https://github.com/jpascher/T0-Time-Mass-Duality/blob/main/2/pdf/parameterherleitung_En.pdf}

\bibitem{verhaeltnis_absolut}
J. Pascher, \emph{Absolute Ratios in T0}, 2025.
\url{https://github.com/jpascher/T0-Time-Mass-Duality/blob/main/2/pdf/T0_verhaeltnis-absolut_En.pdf}

\bibitem{xi_und_e}
J. Pascher, \emph{Xi and Energy}, 2025.
\url{https://github.com/jpascher/T0-Time-Mass-Duality/blob/main/2/pdf/T0_xi-und-e_En.pdf}

\bibitem{umkehrung}
J. Pascher, \emph{Inversion in T0}, 2025.
\url{https://github.com/jpascher/T0-Time-Mass-Duality/blob/main/2/pdf/T0_umkehrung_En.pdf}

\bibitem{esm_analysis}
J. Pascher, \emph{T0 vs ESM Conceptual Analysis}, 2025.
\url{https://github.com/jpascher/T0-Time-Mass-Duality/blob/main/2/pdf/T0vsESM_ConceptualAnalysis_En.pdf}

\end{thebibliography}

\end{document}
