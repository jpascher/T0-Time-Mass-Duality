% Standalone document: Zeit_En
% Uses shared T0 header
% T0 Standalone Header - German Version
% Gemeinsamer Header für alle deutschen Standalone-Dokumente

\documentclass[12pt,a4paper]{article}
\usepackage[utf8]{inputenc}
\usepackage[T1]{fontenc}
\usepackage[ngerman]{babel}
\usepackage{lmodern}

% Mathematics
\usepackage{amsmath,amssymb,amsthm}
\usepackage{physics}
\usepackage{siunitx}

% Layout
\usepackage[left=2.5cm,right=2.5cm,top=2.5cm,bottom=2.5cm,headheight=15pt]{geometry}
\usepackage{fancyhdr}
\usepackage{titlesec}

% Tables and Graphics
\usepackage{booktabs}
\usepackage{array}
\usepackage{longtable}
\usepackage{graphicx}
\usepackage{tikz}
\usetikzlibrary{arrows.meta,positioning,shapes.geometric}

% Colors and Boxes
\usepackage{xcolor}
\usepackage[most]{tcolorbox}
\usepackage{mdframed}

% Additional packages
\usepackage{enumitem}
\usepackage{float}
\usepackage{caption}
\usepackage{subcaption}
\usepackage{multirow}
\usepackage{colortbl}
\usepackage{pdflscape}
\usepackage{algorithm}
\usepackage{algpseudocode}
\usepackage{listings}
\usepackage{hyperref}

% Define colors
\definecolor{t0blue}{RGB}{0,51,102}
\definecolor{t0green}{RGB}{0,102,51}
\definecolor{t0red}{RGB}{153,0,0}
\definecolor{deepblue}{RGB}{0,51,102}
\definecolor{deepgreen}{RGB}{0,102,51}
\definecolor{deepred}{RGB}{153,0,0}
\definecolor{boxgray}{RGB}{240,240,240}
\definecolor{t0yellow}{RGB}{255,200,0}
\definecolor{boxblue}{RGB}{230,240,255}
\definecolor{boxgreen}{RGB}{230,255,230}
\definecolor{boxorange}{RGB}{255,240,230}
\definecolor{boxyellow}{RGB}{255,255,230}

% Custom tcolorbox environments
\newtcolorbox{fundamental}[1][]{
  colback=blue!5!white,
  colframe=blue!75!black,
  title=#1,
  fonttitle=\bfseries,
  breakable
}

\newtcolorbox{derivation}[1][]{
  colback=green!5!white,
  colframe=green!75!black,
  title=#1,
  fonttitle=\bfseries,
  breakable
}

\newtcolorbox{result}[1][]{
  colback=orange!5!white,
  colframe=orange!75!black,
  title=#1,
  fonttitle=\bfseries,
  breakable
}

\newtcolorbox{summary}[1][]{
  colback=gray!10!white,
  colframe=gray!75!black,
  title=#1,
  fonttitle=\bfseries,
  breakable
}

\newtcolorbox{comparison}[1][]{
  colback=purple!5!white,
  colframe=purple!75!black,
  title=#1,
  fonttitle=\bfseries,
  breakable
}

\newtcolorbox{relation}[1][]{
  colback=cyan!5!white,
  colframe=cyan!75!black,
  title=#1,
  fonttitle=\bfseries,
  breakable
}

\newtcolorbox{principle}[1][]{
  colback=yellow!5!white,
  colframe=yellow!75!black,
  title=#1,
  fonttitle=\bfseries,
  breakable
}

\newtcolorbox{insight}[1][]{colback=blue!5,colframe=t0blue,title={#1},fonttitle=\bfseries,breakable}
\newtcolorbox{discovery}[1][]{colback=green!5,colframe=t0green,title={#1},fonttitle=\bfseries,breakable}
\newtcolorbox{newperspective}[1][]{colback=yellow!5,colframe=orange,title={#1},fonttitle=\bfseries,breakable}
\newtcolorbox{revelation}[1][]{colback=red!5,colframe=t0red,title={#1},fonttitle=\bfseries,breakable}
\newtcolorbox{keypoint}[1][]{colback=blue!5,colframe=t0blue,title={#1},fonttitle=\bfseries,breakable}
\newtcolorbox{evidence}[1][]{colback=green!5,colframe=t0green,title={#1},fonttitle=\bfseries,breakable}
\newtcolorbox{conclusion}[1][]{colback=gray!5,colframe=gray,title={#1},fonttitle=\bfseries,breakable}
\newtcolorbox{significance}[1][]{colback=yellow!5,colframe=orange,title={#1},fonttitle=\bfseries,breakable}
\newtcolorbox{philosophical}[1][]{colback=purple!5,colframe=purple,title={#1},fonttitle=\bfseries,breakable}
\newtcolorbox{implication}[1][]{colback=cyan!5,colframe=cyan,title={#1},fonttitle=\bfseries,breakable}
\newtcolorbox{perspective}[1][]{colback=blue!5,colframe=t0blue,title={#1},fonttitle=\bfseries,breakable}
\newtcolorbox{revolutionary}[1][]{colback=red!5,colframe=t0red,title={#1},fonttitle=\bfseries,breakable}
\newtcolorbox{technical}[1][]{colback=gray!5,colframe=gray!75!black,title={#1},fonttitle=\bfseries,breakable}
\newtcolorbox{notation}[1][]{colback=yellow!5,colframe=yellow!75!black,title={#1},fonttitle=\bfseries,breakable}

% Theorem environments
\newtheorem{theorem}{Satz}[section]
\newtheorem{lemma}[theorem]{Lemma}
\newtheorem{corollary}[theorem]{Korollar}
\newtheorem{proposition}[theorem]{Proposition}
\newtheorem{definition}[theorem]{Definition}
\newtheorem{example}[theorem]{Beispiel}
\newtheorem{remark}[theorem]{Bemerkung}
\newtheorem{note}[theorem]{Anmerkung}

% Additional environments
\newenvironment{treatise}{\begin{quote}}{\end{quote}}
\newenvironment{gemeinsam}{\begin{quote}}{\end{quote}}
\newenvironment{vergleich}{\begin{quote}}{\end{quote}}
\newenvironment{vorteil}{\begin{quote}}{\end{quote}}
\newenvironment{quantum}{\begin{quote}}{\end{quote}}

% T0-specific commands
\newcommand{\Tzero}{T$_0$}
\newcommand{\xipar}{\xi}
\newcommand{\Tfield}{T}
\newcommand{\Efield}{\mathcal{E}}
\newcommand{\meff}{m_{\text{eff}}}
\newcommand{\Eabs}{E_{\text{abs}}}
\newcommand{\taupar}{\tau}

% Header setup
\pagestyle{fancy}
\fancyhf{}
\fancyhead[L]{\leftmark}
\fancyhead[R]{\thepage}
\renewcommand{\headrulewidth}{0.4pt}

% Hyperref setup
\hypersetup{
    colorlinks=true,
    linkcolor=blue,
    filecolor=magenta,
    urlcolor=cyan,
    citecolor=blue,
    pdftitle={T0 Theory Document},
    pdfauthor={Johann Pascher}
}

% German quotation marks
%\newcommand{\dq}[1]{\glqq{}#1\grqq{}}


\title{Time}
\author{Johann Pascher}
\date{2025}

\begin{document}

\maketitle

\chapter{Time}

	
	
	\begin{abstract}
		The T0 Modell describes a fundamental granulation of Raumzeit at the sub-Planck Skala $\Lzero = \xipar \times \Lp$ with $\xipar \approx 1.333 \times 10^{-4}$. This Arbeit examines the Konsequenzen for Skala hierarchies, Zeit continuity, and the mathematisch completeness of various gravitativ theories. The Zeit-Masse duality $T(x,t) \cdot m(x,t) = 1$ requires beide Felder to be coupled and Variable, while the fundamental $\xipar$-asymmetry enables alle developmental Prozesse.
	\end{abstract}
	
	\newpage
	
	\section{Granulation as Fundamental Principle of Reality}
	
	\subsection{Minimum Length Scale $\Lzero$}
	
	The T0 Modell introduces a fundamental Länge Skala deeper than the Planck Länge:
	
	\begin{equation}
		\Lzero = \xipar \times \Lp \approx \frac{4}{3} \times 10^{-4} \times 1.616 \times 10^{-35} \text{ m} \approx 2.155 \times 10^{-39} \text{ m}
	\end{equation}
	
	\textbf{Significance of $\Lzero$}:
	\begin{itemize}
		\item Absolute physikalisch lower Grenze for spatial Strukturen
		\item Granulated Raumzeit Struktur - not kontinuierlich
		\item Sub-Planck physics with new fundamental laws
		\item Universal Skala for alle physikalisch Phänomene
	\end{itemize}
	
	\subsection{The Extreme Scale Hierarchy}
	
	From $\Lzero$ to kosmologisch Skalen extends a hierarchy of over 60 orders of Größenordnung:
	
	\begin{align}
		\Lzero &\approx 10^{-39} \text{ m} \quad \text{(Sub-Planck minimum)} \\
		\Lp &\approx 10^{-35} \text{ m} \quad \text{(Planck length)} \\
		L_{\text{Casimir}} &\approx 100 \text{ micrometers} \quad \text{(Casimir scale)} \\
		L_{\text{Atom}} &\approx 10^{-10} \text{ m} \quad \text{(Atomic scale)} \\
		L_{\text{Macro}} &\approx 1 \text{ m} \quad \text{(Human scale)} \\
		L_{\text{Cosmo}} &\approx 10^{26} \text{ m} \quad \text{(Cosmological scale)}
	\end{align}
	
	\subsection{Casimir Scale as Evidence of Granulation}
	
	At the Casimir Charakteristik Skala, erst measurable Effekte appear:
	
	\begin{equation}
		L_{\xipar} \approx \frac{1}{\sqrt{\xipar \times \Lp}} \approx 100 \text{ micrometers}
	\end{equation}
	
	\textbf{Experimentell Evidenz}:
	\begin{itemize}
		\item Deviations from $1/d^4$ law at distances $\approx 10$ nm
		\item $\xipar$-Korrekturen in Casimir Kraft Messungen
		\item Limits of continuum physics become visible
	\end{itemize}
	
	\section{Limit Systems and Scale Hierarchies}
	
	\subsection{Three-Scale Hierarchy}
	
	The T0 Modell organizes alle physikalisch Skalen into three fundamental domains:
	
	\begin{enumerate}
		\item \textbf{$\Lzero$-domain}: Granulated physics, universal laws
		\item \textbf{Planck domain}: Quantum Gravitation, Übergang Dynamik
		\item \textbf{Macro domain}: Classical physics with $\xipar$-Korrekturen
	\end{enumerate}
	
	\subsection{Relational Number System}
	
	Prime Zahl Verhältnisse organize Teilchen into natural generations:
	
	\begin{itemize}
		\item \textbf{3-Grenze}: u-, d-Quarks (1st generation)
		\item \textbf{5-Grenze}: c-, s-Quarks (2nd generation)
		\item \textbf{7-Grenze}: t-, b-Quarks (3rd generation)
	\end{itemize}
	
	The nächst prime Zahl (11) leads to $\xipar^{11}$-Korrekturen $\approx 10^{-44}$, welche lie unten the Planck Skala.
	
	\subsection{CP Violation from Universal Asymmetry}
	
	The $\xipar$-asymmetry explains:
	\begin{itemize}
		\item CP violation in weak Wechselwirkungen
		\item Matter-Antimaterie asymmetry in the Universum
		\item Chiral Symmetrie breaking in nature
	\end{itemize}
	
	\section{Fundamental Asymmetry as Motion Principle}
	
	\subsection{The Universal $\xipar$-Constant}
	
	\begin{equation}
		\xipar = \frac{4}{3} \times 10^{-4} \approx 1.333 \times 10^{-4}
	\end{equation}
	
	\textbf{Origin}: Geometric 4/3-Konstante from optimal 3D Raum packing
	
	\textbf{Effect}: Universal asymmetry enabling alle development
	
	\subsection{Eternal Universe Without Big Bang}
	
	The T0 Modell describes an eternal, unendlich, non-expanding Universum:
	
	\begin{itemize}
		\item No beginning, no end - timeless existence
		\item Heisenberg's Unschärfe Prinzip forbids Big Bang: $\Delta E \times \Delta t \geq \hbar/2$
		\item Structured development stattdessen of chaotic explosion
		\item Continuous $\xipar$-Feld Dynamik stattdessen of Big Bang
	\end{itemize}
	
	\subsection{Time Exists Only After Field-Asymmetry Excitation}
	
	\textbf{Hierarchy of Zeit emergence}:
	\begin{enumerate}
		\item \textbf{Timeless Universum}: Perfect Symmetrie, no Zeit
		\item \textbf{$\xipar$-asymmetry arises}: Symmetry breaking activates Zeit Feld
		\item \textbf{Time-Energie duality}: $T(x,t) \cdot E(x,t) = 1$ becomes active
		\item \textbf{Manifested Zeit}: Local Zeit emerges through Feld Dynamik
		\item \textbf{Directed Zeit}: Thermodynamic arrow of Zeit stabilizes
	\end{enumerate}
	
	Time is not fundamental but emergent from Feld asymmetry.
	
	\section{Hierarchical Structure: Universe > Field > Space}
	
	\subsection{The Fundamental Order Hierarchy}
	
	\textbf{Universe (highest Ordnung Ebene)}:
	\begin{itemize}
		\item Superordinate Struktur with eternal, unendlich Eigenschaften
		\item Global organizational Prinzipien determine everything unten
		\item $\xipar$-asymmetry as universal guiding Struktur
		\item Thermodynamic overall balance of alle Prozesse
	\end{itemize}
	
	\textbf{Field (middle organizational Ebene)}:
	\begin{itemize}
		\item Universal $\xipar$-Feld as mediator zwischen Universum and Raum
		\item Local Dynamik innerhalb global Einschränkungen
		\item Time-Energie duality as Feld Prinzip
		\item Structure-forming Prozesse through asymmetry
	\end{itemize}
	
	\textbf{Space (manifestation Ebene)}:
	\begin{itemize}
		\item 3D Geometrie as stage for Feld manifestations
		\item Granulation at $\Lzero$-Skala
		\item Local Wechselwirkungen zwischen Feld excitations
	\end{itemize}
	
	\subsection{Causal Downward Coupling}
	
	\begin{equation}
		\text{UNIVERSE} \rightarrow \text{FIELD} \rightarrow \text{SPACE} \rightarrow \text{PARTICLES}
	\end{equation}
	
	The Universum is not nur the sum of its spatial Teile. Superordinate Eigenschaften emerge nur at the highest Ebene. The $\xipar$-Konstante is universal, not a Raum Eigenschaft.
	
	\section{Continuous Time Beyond Certain Scales}
	
	\subsection{The Crucial Scale Hierarchy of Time}
	
	In the T0 Modell, unterschiedlich Zeit domains exist with fundamentally unterschiedlich Eigenschaften. The further we move from $\Lzero$, the mehr kontinuierlich and Konstante Zeit becomes.
	
	\subsubsection{Granulated Zone (unten $\Lzero$)}
	
	\begin{equation}
		\Lzero = \xipar \times \Lp \approx 2.155 \times 10^{-39} \text{ m}
	\end{equation}
	
	\begin{itemize}
		\item Time is discretely granulated, not kontinuierlich
		\item Chaotic Quanten fluctuations dominate
		\item Physics loses klassisch meaning
		\item All fundamental Kräfte equally strong
	\end{itemize}
	
	\subsubsection{Transition Zone (around $\Lzero$)}
	
	\begin{itemize}
		\item Time-Masse duality $T \cdot m = 1$ becomes fully active
		\item Intensive Wechselwirkung of alle Felder
		\item Transition from granulated to kontinuierlich
	\end{itemize}
	
	\subsubsection{Continuous Zone (oben $\Lzero$)}
	
	\begin{tcolorbox}[colback=blue!5!white,colframe=blue!75!black,title=Central Insight]
		\begin{equation}
			\text{Distance to } \Lzero \uparrow \quad \Rightarrow \quad \text{Time continuity} \uparrow \quad \Rightarrow \quad \text{Constant direction} \uparrow
		\end{equation}
	\end{tcolorbox}
	
	\begin{itemize}
		\item Beyond a certain point, Zeit becomes kontinuierlich
		\item Constant directed flow direction emerges
		\item The greater the Entfernung to $\Lzero$, the mehr stable the Zeit direction
		\item Emergent klassisch physics with $\xipar$-Korrekturen
	\end{itemize}
	
	\subsection{Quantitative Scaling of Time Continuity}
	
	\textbf{Time continuity as Funktion of Entfernung to $\Lzero$}:
	\begin{equation}
		\text{Time continuity} \propto \log\left(\frac{L}{\Lzero}\right) \quad \text{for } L \gg \Lzero
	\end{equation}
	
	\textbf{Practical Skalen}:
	\begin{align}
		L = 10^{-35}\text{ m (Planck)}: &\quad \text{Still granulated} \\
		L = 10^{-15}\text{ m (Nuclear)}: &\quad \text{Transition to continuity} \\
		L = 10^{-10}\text{ m (Atomic)}: &\quad \text{Practically continuous} \\
		L = 10^{-3}\text{ m (mm)}: &\quad \text{Completely continuous, constant direction} \\
		L = 1\text{ m (Meter)}: &\quad \text{Perfectly linear, directed time}
	\end{align}
	
	\subsection{Thermodynamic Arrow of Time}
	
	\textbf{Scale-dependent entropy}:
	\begin{itemize}
		\item \textbf{Granulated Ebene ($\Lzero$)}: Maximum entropy, perfect Symmetrie
		\item \textbf{Transition Ebene}: Entropy gradients emerge
		\item \textbf{Continuous Ebene}: Second law becomes active
		\item \textbf{Macroscopic Ebene}: Irreversible Zeit direction
	\end{itemize}
	
	\section{Practical vs. Fundamental Physics}
	
	\subsection{Time is Practically Experienced as Constant}
	
	De facto for us: Time flows ständig in our experience domain
	\begin{itemize}
		\item \textbf{Local Skalen (m to km)}: Time is practically perfectly linear and Konstante
		\item \textbf{Measurable variations}: Only under extreme Bedingungen (GPS satellites, Teilchen accelerators)
		\item \textbf{Everyday physics}: Time constancy is a good Näherung
	\end{itemize}
	
	\subsection{Speed of Light as Clear Upper Limit}
	
	\textbf{Observed reality}:
	\begin{itemize}
		\item $c = 299,792,458$ m/s is measurable upper Grenze for information transfer
		\item \textbf{Causality}: No signals faster than $c$ beobachtet
		\item \textbf{Relativistic Effekte}: Clearly measurable at $v \rightarrow c$
		\item \textbf{Particle accelerators}: Confirm $c$-Grenze daily
	\end{itemize}
	
	\subsection{Resolution of the Apparent Contradiction}
	
	\textbf{Macroscopic Ebene (our world)}:
	\begin{equation}
		L = 1 \text{ m to } 10^6 \text{ m (km range)}
	\end{equation}
	
	\begin{itemize}
		\item Time flows ständig: $dt/dt_0 \approx 1 + 10^{-16}$ (immeasurable)
		\item $c$ is practically Konstante: $\Delta c/c \approx 10^{-16}$ (immeasurable)
		\item Einstein physics works perfectly
	\end{itemize}
	
	\textbf{Fundamental Ebene (T0 Modell)}:
	\begin{equation}
		\Lzero = 10^{-39} \text{ m to } \Lp = 10^{-35} \text{ m}
	\end{equation}
	
	\begin{itemize}
		\item Time-Masse duality: $T \cdot m = 1$ is fundamental
		\item $c$ is Verhältnis: $c = L/T$ (must be Variable)
		\item Mathematical consistency requires coupled variation
	\end{itemize}
	
	\textbf{These variations are $10^6$ times smaller than our best Messung precision!}
	
	\section{Gravitation: Mass Variation vs. Space Curvature}
	
	\subsection{Two Equivalent Interpretations}
	
	\textbf{Einstein Interpretation}:
	\begin{itemize}
		\item $m = $ Konstante (fixed Masse)
		\item $g_{\mu\nu} = $ Variable (curved Raumzeit)
		\item Mass causes Raum Krümmung
	\end{itemize}
	
	\textbf{T0 Interpretation}:
	\begin{itemize}
		\item $m(x,t) = $ Variable (dynamic Masse)
		\item $g_{\mu\nu} = $ fixed (flat Euclidean Raum)
		\item Mass varies locally through $\xipar$-Feld
	\end{itemize}
	
	\subsection{Important Insight: We Don't Know!}
	
	\begin{tcolorbox}[colback=red!5!white,colframe=red!75!black,title=Attention - Fundamental Point]
		We DO NOT KNOW whether Masse causes Raum Krümmung or whether Masse itself varies!
		
		This is an Annahme, not a proven fact!
	\end{tcolorbox}
	
	\textbf{Both interpretations are equally gültig}:
	
	\textbf{Einstein Annahme}:
	\begin{align}
		\text{Mass/energy} &\rightarrow \text{Space curvature} \rightarrow \text{Gravitation} \\
		G_{\mu\nu} &= 8\pi T_{\mu\nu}
	\end{align}
	
	\textbf{T0 alternative}:
	\begin{align}
		\xipar\text{-field} &\rightarrow \text{Mass variation} \rightarrow \text{Gravitational effects} \\
		m(x,t) &= m_0 \cdot (1 + \xipar \cdot \Phi(x,t))
	\end{align}
	
	\subsection{Experimentell Indistinguishability}
	
	\textbf{All Messungen are Frequenz-based}:
	\begin{itemize}
		\item \textbf{Clocks}: Hyperfine Übergang frequencies
		\item \textbf{Scales}: Spring Oszillationen/resonance frequencies
		\item \textbf{Spectrometers}: Light frequencies and Übergänge
		\item \textbf{Interferometers}: Phases = Frequenz integrals
	\end{itemize}
	
	\textbf{Identical Frequenz shifts}:
	\begin{align}
		\text{Einstein}: \quad \nu' &= \nu_0 \sqrt{1 + 2\Phi/c^2} \approx \nu_0 (1 + \Phi/c^2) \\
		\text{T0}: \quad \nu' &= \nu_0 \cdot \frac{m(x,t)}{T(x,t)} \approx \nu_0 (1 + \Phi/c^2)
	\end{align}
	
	Only Frequenz Verhältnisse are measurable - absolute frequencies are fundamentally inaccessible!
	
	\section{Mathematical Completeness: Both Fields Coupled Variable}
	
	\subsection{The Correct Mathematical Formulation}
	
	\textbf{Mathematically korrekt in T0 Modell}:
	\begin{align}
		T(x,t) &= \text{variable} \quad \text{(Time as dynamic field)} \\
		m(x,t) &= \text{variable} \quad \text{(Mass as dynamic field)}
	\end{align}
	
	\textbf{Coupled through fundamental duality}:
	\begin{equation}
		T(x,t) \cdot m(x,t) = 1
	\end{equation}
	
	\textbf{Both Felder vary TOGETHER}:
	\begin{align}
		T(x,t) &= T_0 \cdot (1 + \xipar \cdot \Phi(x,t)) \\
		m(x,t) &= m_0 \cdot (1 - \xipar \cdot \Phi(x,t))
	\end{align}
	
	\subsection{Verification of Mathematical Consistency}
	
	\textbf{Duality check}:
	\begin{align}
		T(x,t) \cdot m(x,t) &= T_0 m_0 \cdot (1 + \xipar \Phi)(1 - \xipar \Phi) \\
		&= T_0 m_0 \cdot (1 - \xipar^2 \Phi^2) \\
		&\approx T_0 m_0 = 1 \quad \text{(for } \xipar \Phi \ll 1\text{)}
	\end{align}
	
	Mathematical consistency confirmed!
	
	\subsection{Why Both Fields Must Be Variable}
	
	\textbf{Lagrange formalism requires}:
	\begin{equation}
		\delta S = \int \delta \mathcal{L} \, d^4x = 0
	\end{equation}
	
	\textbf{Complete variation}:
	\begin{equation}
		\delta \mathcal{L} = \frac{\partial \mathcal{L}}{\partial T}\delta T + \frac{\partial \mathcal{L}}{\partial m}\delta m + \frac{\partial \mathcal{L}}{\partial \partial_\mu T}\delta \partial_\mu T + \frac{\partial \mathcal{L}}{\partial \partial_\mu m}\delta \partial_\mu m
	\end{equation}
	
	For mathematisch completeness:
	\begin{itemize}
		\item $\delta T \neq 0$ (Time must be Variable)
		\item $\delta m \neq 0$ (Mass must be Variable)
		\item Both coupled through $T \cdot m = 1$
	\end{itemize}
	
	\subsection{Einstein's Arbitrary Constant Setting}
	
	Einstein arbitrarily sets:
	\begin{equation}
		m_0 = \text{constant} \quad \Rightarrow \quad \delta m = 0
	\end{equation}
	
	\textbf{Mathematical problem}:
	\begin{itemize}
		\item Incomplete variation of the Lagrangian
		\item Violates variation Prinzip of Feld theory
		\item Arbitrary Symmetrie breaking without justification
	\end{itemize}
	
	\subsection{Parameter Elegance}
	
	\begin{align}
		\text{Einstein}: \quad &m_0, c, G, \hbar, \Lambda, \alpha_{\text{EM}}, \ldots \quad (\gg 10 \text{ free parameters}) \\
		\text{T0}: \quad &\xipar \quad (1 \text{ universal parameter})
	\end{align}
	
	\section{Pragmatic Preference: Variable Mass with Constant Time}
	
	\subsection{The Pragmatic Alternative for Our Experience Space}
	
	As pragmatists, one can sicherlich prefer:
	\begin{align}
		\text{Time}: \quad t &= \text{constant} \quad \text{(practical experience)} \\
		\text{Mass}: \quad m(x,t) &= \text{variable} \quad \text{(dynamic adjustment)}
	\end{align}
	
	\textbf{Why dies is pragmatically sensible}:
	\begin{itemize}
		\item Time constancy corresponds to our direct experience
		\item Mass variation is conceptually easier to imagine
		\item Practical Berechnungen oft become simpler
		\item Intuitive understandability for Anwendungen
	\end{itemize}
	
	\subsection{Practical Advantages of Constant Time}
	
	In our experienceable Raum (m to km):
	\begin{itemize}
		\item Time flows linearly and ständig - our direct experience
		\item Clocks tick gleichförmig - practical Zeit Messung
		\item Causal sequences are klar defined
		\item Technical Anwendungen (GPS, navigation) Funktion
	\end{itemize}
	
	\textbf{Language convention}:
	\begin{itemize}
		\item Time passes ständig
		\item Mass adapts to the Felder
		\item Matter becomes heavier/lighter depending on location
	\end{itemize}
	
	\subsection{Variable Mass as Intuitive Concept}
	
	\textbf{Pragmatic Interpretation}:
	\begin{equation}
		m(x) = m_0 \cdot (1 + \xipar \cdot \text{Gravitational field}(x))
	\end{equation}
	
	\textbf{Intuitive conception}:
	\begin{itemize}
		\item Mass increases in strong gravitativ Felder
		\item Mass decreases in weaker Felder
		\item Matter feels the local $\xipar$-Feld
		\item Dynamic adaptation to environment
	\end{itemize}
	
	\subsection{Scientific Legitimacy of Preference}
	
	\begin{tcolorbox}[colback=green!5!white,colframe=green!75!black,title=Important Insight]
		Pragmatic preferences are scientifically justified wann beide approaches are experimentally equivalent!
	\end{tcolorbox}
	
	\textbf{Justification}:
	\begin{itemize}
		\item Scientifically equivalent to Einstein Ansatz
		\item Often practically advantageous for Anwendungen
		\item Didactically easier to teach
		\item Technically mehr efficient to implement
	\end{itemize}
	
	The choice zwischen Konstante Zeit + Variable Masse vs. Einstein is a Materie of taste - beide are scientifically equally justified!
	
	\section{The Eternal Philosophical Boundary}
	
	\subsection{What the T0 Model Explains}
	
	\begin{itemize}
		\item HOW the $\xipar$-asymmetry works
		\item WHAT the Konsequenzen are
		\item WHICH laws follow from it
		\item WHEN Zeit and development emerge
	\end{itemize}
	
	\subsection{What the T0 Model CANNOT Explain}
	
	The fundamental questions remain:
	\begin{itemize}
		\item WHY does the $\xipar$-asymmetry exist?
		\item WHERE does the original Energie come from?
		\item WHO/WHAT gave the erst impulse?
		\item WHY does anything exist at alle stattdessen of nothing?
	\end{itemize}
	
	\subsection{Scientific Humility}
	
	\textbf{The eternal Rand}:
	Every Erklärung needs unexplained Axiome. The ultimate reason immer remains mysterious. The das of existence is given, the warum remains open.
	
	\textbf{The elegant shift}:
	The T0 Modell shifts the mystery to a deeper, mehr elegant Ebene - but it cannot resolve the fundamental riddle of existence.
	
	And das is good. Because a Universum without mystery would be a boring Universum.
	
	\section{Experimentell Predictions and Tests}
	
	\subsection{Casimir Effect Modifications}
	
	\begin{itemize}
		\item Deviations from $1/d^4$ law at $d \approx 10$ nm
		\item $\xipar$-Korrekturen in precision Messungen
		\item Frequency-dependent Casimir Kräfte
	\end{itemize}
	
	\subsection{Atom Interferometry}
	
	\begin{itemize}
		\item $\xipar$-resonances in Quanten interferometers
		\item Mass variations in gravitativ Felder
		\item Time-Masse duality in precision Experimente
	\end{itemize}
	
	\subsection{Gravitational Wave Detection}
	
	\begin{itemize}
		\item $\xipar$-Korrekturen in LIGO/Virgo data
		\item Modifications of Welle dispersion
		\item Sub-Planck Strukturen in gravitativ Wellen
	\end{itemize}
	
	\section{Schlussfolgerung: Asymmetry as Engine of Reality}
	
	The T0 Modell shows das granulation, Grenzen, and fundamental asymmetry are inseparably connected with the Skala-dependent nature of Zeit:
	
	\begin{enumerate}
		\item \textbf{Granulation} at $\Lzero$ defines the base Skala of alle physics
		\item \textbf{Limit Systeme} organize Teilchen into natural generations
		\item \textbf{Fundamental asymmetry} generates Zeit, development, and Struktur formation
		\item \textbf{Hierarchical organization} from Universum through Feld to Raum
		\item \textbf{Continuous Zeit} emerges beyond certain Skalen through Entfernung to $\Lzero$
		\item \textbf{Mathematical completeness} requires T0 formulation over Einstein
		\item \textbf{Experimentell indistinguishability} of unterschiedlich interpretations
		\item \textbf{Pragmatic preferences} are scientifically justified
		\item \textbf{Philosophical boundaries} remain and preserve the mystery
	\end{enumerate}
	
	The $\xipar$-asymmetry is the engine of reality - without it, the Universum would remain in perfect, timeless Symmetrie. With it emerges the entire diversity and Dynamik of our observable world.
	
	The T0 Modell somit offers a unified Erklärung for fundamental puzzles of physics - from the granulation of Raumzeit to the emergence of Zeit itself.
	% Mathematical Beweis: The Formula T·m = 1 Excludes Singularities
	% This segment can be inserted into an existing LaTeX document
	
	\section{Mathematical Beweis: The Formula $T \cdot m = 1$ Excludes Singularities}
	
	\subsection{Important Clarification: $T$ as Oscillation Period}
	
	\textbf{ATTENTION:} In dies Analyse, $T$ does not Mittelwert the experienced, kontinuierlich flowing Zeit, but the \textbf{Oszillation period} or \textbf{Charakteristik Zeit Konstante} of a System. This is a fundamental difference:
	
	\begin{itemize}
		\item $T =$ Oszillation period (diskret, Charakteristik Zeit Einheit)
		\item Not: $T =$ kontinuierlich Zeit coordinate (our everyday experience)
	\end{itemize}
	
	\subsection{The Fundamental Exclusion Property}
	
	The Gleichung $T \cdot m = 1$ is not nur a mathematisch Zusammenhang -- it is an \textbf{exclusion theorem}. Through its algebraic Struktur, it makes certain Zustände mathematically unmöglich.
	
	\subsection{Beweis 1: Exclusion of Infinite Mass}
	
	\textbf{Assumption:} There exists an unendlich Masse $m = \infty$
	
	\textbf{Mathematical Konsequenz:}
	\begin{align}
		T \cdot m &= 1\\
		T \cdot \infty &= 1\\
		T &= \frac{1}{\infty} = 0
	\end{align}
	
	\textbf{Contradiction:} $T = 0$ is not in the domain of the Gleichung $T \cdot m = 1$, since:
	\begin{itemize}
		\item The product $0 \cdot \infty$ is mathematically undefined
		\item The original Gleichung $T \cdot m = 1$ would be violated $(0 \cdot \infty \neq 1)$
	\end{itemize}
	
	\textbf{Schlussfolgerung:} $m = \infty$ is excluded by the Formel.
	
	\subsection{Beweis 2: Exclusion of Infinite Time}
	
	\textbf{Assumption:} There exists an unendlich Zeit $T = \infty$
	
	\textbf{Mathematical Konsequenz:}
	\begin{align}
		T \cdot m &= 1\\
		\infty \cdot m &= 1\\
		m &= \frac{1}{\infty} = 0
	\end{align}
	
	\textbf{Contradiction:} $m = 0$ is not in the domain, since:
	\begin{itemize}
		\item The product $\infty \cdot 0$ is mathematically undefined
		\item The Gleichung $T \cdot m = 1$ would be violated $(\infty \cdot 0 \neq 1)$
	\end{itemize}
	
	\textbf{Schlussfolgerung:} $T = \infty$ is excluded by the Formel.
	
	\subsection{Beweis 3: Exclusion of Zero Values}
	
	\textbf{Assumption:} There exists $T = 0$ or $m = 0$
	
	\textbf{Case 1:} $T = 0$
	\begin{equation}
		T \cdot m = 1 \Rightarrow 0 \cdot m = 1
	\end{equation}
	This is unmöglich for irgendein endlich Wert of $m$, since $0 \cdot m = 0 \neq 1$.
	
	\textbf{Case 2:} $m = 0$
	\begin{equation}
		T \cdot m = 1 \Rightarrow T \cdot 0 = 1
	\end{equation}
	This is unmöglich for irgendein endlich Wert of $T$, since $T \cdot 0 = 0 \neq 1$.
	
	\textbf{Schlussfolgerung:} Both $T = 0$ and $m = 0$ are excluded by the Formel.
	
	\subsection{Beweis 4: Exclusion of Mathematical Singularities}
	
	\textbf{Definition of a Singularität:} A point wo a Funktion becomes undefined or unendlich.
	
	\textbf{Analysis of the Funktion} $T = \frac{1}{m}$:
	
	\textbf{Potential singularities could occur at:}
	\begin{itemize}
		\item $m = 0$ (division by zero)
		\item $T \to \infty$ (unendlich Funktion Werte)
	\end{itemize}
	
	\textbf{Exclusion by the Einschränkung} $T \cdot m = 1$:
	\begin{enumerate}
		\item \textbf{At} $m = 0$: The Gleichung $T \cdot m = 1$ cannot be satisfied
		\item \textbf{At} $T \to \infty$: Would require $m \to 0$, welche is bereits excluded
	\end{enumerate}
	
	\textbf{Mathematical Beweis of Singularität freedom:}
	
	For jeder point $(T,m)$ with $T \cdot m = 1$:
	\begin{align}
		T &= \frac{1}{m} \text{ with } m \in (0, +\infty)\\
		m &= \frac{1}{T} \text{ with } T \in (0, +\infty)
	\end{align}
	
	Both Funktionen are on their entire domain:
	\begin{itemize}
		\item \textbf{Continuous}
		\item \textbf{Differentiable}
		\item \textbf{Finite}
		\textbf{Well-defined}
	\end{itemize}
	
	\subsection{The Algebraic Protection Function}
	
	The Gleichung $T \cdot m = 1$ acts like an \textbf{algebraic protection} against singularities:
	
	\subsubsection{Automatic Correction}
	\begin{align}
		\text{If } m \text{ becomes very small} &\Rightarrow T \text{ automatically becomes very large}\\
		\text{If } T \text{ becomes very small} &\Rightarrow m \text{ automatically becomes very large}\\
		\text{But: } T \cdot m &\text{ always remains exactly } 1
	\end{align}
	
	\subsubsection{Mathematical Stability}
	\begin{align}
		\lim_{m \to 0^+} T &= +\infty, \text{ but } T \cdot m = 1 \text{ remains satisfied}\\
		\lim_{T \to 0^+} m &= +\infty, \text{ but } T \cdot m = 1 \text{ remains satisfied}
	\end{align}
	
	The Einschränkung \textbf{Kräfte} the Variablen into a endlich, well-defined region.
	
	\subsection{Beweis 5: Positive Definiteness}
	
	\textbf{Satz:} All Lösungen of $T \cdot m = 1$ are positiv.
	
	\textbf{Beweis:}
	\begin{equation}
		T \cdot m = 1 > 0
	\end{equation}
	
	Since the product is positiv, beide Faktoren must have the gleich sign.
	
	\textbf{Exclusion of negativ Werte:}
	\begin{itemize}
		\item If $T < 0$ and $m < 0$, dann $T \cdot m > 0$, but physically meaningless
		\item If $T > 0$ and $m < 0$, dann $T \cdot m < 0 \neq 1$
		\item If $T < 0$ and $m > 0$, dann $T \cdot m < 0 \neq 1$
	\end{itemize}
	
	\textbf{Schlussfolgerung:} Only $T > 0$ and $m > 0$ satisfy the Gleichung.
	
	\subsection{The Fundamental Insight About Time and Continuity}
	
	\textbf{Important physikalisch clarification:}
	
	The Formel $T \cdot m = 1$ describes \textbf{diskret, Charakteristik Eigenschaften} of Systeme, not the kontinuierlich Zeit flow of our experience. This means:
	
	\subsubsection{What $T \cdot m = 1$ does NOT Zustand:}
	\begin{itemize}
		\item \glqq Time stands noch\grqq\ $(T = 0)$
		\item \glqq Processes take infinitely long\grqq\ $(T = \infty)$
		\item \glqq The Zeit flow is interrupted\grqq
		\item \glqq Our experienced Zeit disappears\grqq
	\end{itemize}
	
	\subsubsection{What $T \cdot m = 1$ actually describes:}
	\begin{itemize}
		\item \textbf{Oscillation periods} have mathematisch Grenzen
		\item \textbf{Characteristic Zeit Konstanten} cannot become arbitrary
		\item \textbf{Discrete Zeit Einheiten} stand in fixed Beziehung to Masse
		\item \textbf{Periodic Prozesse} follow the Einschränkung $T \cdot m = 1$
	\end{itemize}
	
	\subsubsection{The kontinuierlich Zeit flow remains unaffected}
	
	The kontinuierlich Zeit coordinate $t$ (our \glqq arrow Zeit\grqq) is \textbf{not affected} by dies Zusammenhang. $T \cdot m = 1$ regulates nur the \textbf{intrinsic Zeit Skalen} of physikalisch Systeme, not the superordinate Zeit flow in welche diese Systeme exist.
	
	\textbf{Important Einsicht ungefähr our Zeit perception:}
	
	Our kontinuierlich Zeit perception could practically be nur a \textbf{tiny excerpt} of a much larger period -- an Oszillation period so immense das it far exceeds anything humans could ever experience or conceive.
	
	\textbf{Conceivable orders of Größenordnung:}
	\begin{itemize}
		\item \textbf{Human life:} $\sim 10^2$ years
		\item \textbf{Human history:} $\sim 10^4$ years
		\item \textbf{Earth age:} $\sim 10^9$ years
		\item \textbf{Universe age:} $\sim 10^{10}$ years
		\textbf{Possible cosmic period:} $10^{50}$, $10^{100}$ or sogar larger Zeit Skalen
	\end{itemize}
	
	In solch a scenario, our entire observable Universum would experience nur an \textbf{infinitesimal klein fraction} of a fundamental Oszillation period. For us, Zeit appears linear and kontinuierlich because we perceive nur a vanishingly klein section of a huge cosmic \glqq Oszillation\grqq.
	
	\textbf{Analogy:} Just as a bacterium on a clock hand would perceive the movement as \glqq straight ahead\grqq, obwohl it moves on a circular path, we might experience \glqq linear Zeit\grqq, obwohl we are in a gigantic periodic Struktur.
	
	This Perspektive shows das $T \cdot m = 1$ and our Zeit perception can operate on vollständig unterschiedlich Skalen without contradicting jeder andere.
	
	\subsection{Cosmological Implications}
	
	\textbf{This Standpunkt opens new possibilities:}
	
	What we observe as cosmic development and change could be nur a \textbf{klein section} in a much larger cyclic pattern das follows the fundamental Zusammenhang $T \cdot m = 1$.
	
	\textbf{Possible cosmic Struktur:}
	\begin{itemize}
		\item \textbf{Local Zeit perception:} Linear, kontinuierlich (our experience domain)
		\item \textbf{Middle Zeit Skalen:} Observable cosmic developments
		\item \textbf{Fundamental Zeit Skala:} Gigantic period gemäß $T \cdot m = 1$
	\end{itemize}
	
	\textbf{Implications:}
	\begin{itemize}
		\item Nature could be organized in \textbf{layered-periodic} fashion
		\item Different Zeit Skalen follow unterschiedlich regularities
		\item $T \cdot m = 1$ could be the \textbf{master Einschränkung} for the largest Skala
		\item Our observable cosmic development would be a fragment of a cyclic System
	\end{itemize}
	
	This Interpretation shows wie mathematisch Einschränkungen $(T \cdot m = 1)$ and physikalisch Beobachtungen (linear Zeit perception) can coexist in a \textbf{hierarchical Zeit Modell}.
	
	\subsection{Schlussfolgerung: Mathematical Certainty}
	
	The Formel $T \cdot m = 1$ is not nur an Gleichung -- it is an \textbf{existence Beweis} for Singularität-free physics. It proves mathematically das:
	
	\begin{itemize}
		\item \textbf{Infinite masses do not exist}
		\item \textbf{Infinite Oszillation periods do not exist}
		\item \textbf{Zero masses are excluded}
		\item \textbf{Zero Oszillation periods are excluded}
		\item \textbf{Singularities in Charakteristik Zeit Skalen cannot occur}
	\end{itemize}
	
	\textbf{Mathematics itself protects physics from singularities -- without affecting the kontinuierlich Zeit flow.}    

\begin{thebibliography}{99}

% ============================================
% Core T0 Theory References (J. Pascher)
% GitHub Repository: https://github.com/jpascher/T0-Time-Mass-Duality
% ============================================

\bibitem{pascher2024}
J. Pascher, \emph{T0 Theory: Time-Mass Duality}, 2024.
\url{https://github.com/jpascher/T0-Time-Mass-Duality/blob/main/2/pdf/T0_unified_report.pdf}

\bibitem{pascher2025t0}
J. Pascher, \emph{T0 Theory: Fundamentals}, 2025.
\url{https://github.com/jpascher/T0-Time-Mass-Duality/blob/main/2/pdf/T0_Grundlagen_En.pdf}

\bibitem{pascher2025qm}
J. Pascher, \emph{T0 Theory: Quantum Mechanics}, 2025.
\url{https://github.com/jpascher/T0-Time-Mass-Duality/blob/main/2/pdf/QM_En.pdf}

\bibitem{pascher2025si}
J. Pascher, \emph{T0 Theory: SI Units}, 2025.
\url{https://github.com/jpascher/T0-Time-Mass-Duality/blob/main/2/pdf/T0_SI_En.pdf}

\bibitem{pascher2025g2}
J. Pascher, \emph{T0 Theory: The g-2 Anomaly}, 2025.
\url{https://github.com/jpascher/T0-Time-Mass-Duality/blob/main/2/pdf/T0_Anomale-g2-9_En.pdf}

\bibitem{pascher2025cmb}
J. Pascher, \emph{T0 Theory: CMB Analysis}, 2025.
\url{https://github.com/jpascher/T0-Time-Mass-Duality/blob/main/2/pdf/Zwei-Dipole-CMB_En.pdf}

% Historical Physics
\bibitem{einstein1905}
A. Einstein, \emph{On the Electrodynamics of Moving Bodies}, Annalen der Physik, 1905.
\url{https://doi.org/10.1002/andp.19053221004}

\bibitem{dirac1928}
P.A.M. Dirac, \emph{The Quantum Theory of the Electron}, Proc. Roy. Soc. A, 1928.
\url{https://doi.org/10.1098/rspa.1928.0023}

\bibitem{planck1900}
M. Planck, \emph{On the Theory of the Energy Distribution Law}, 1900.
\url{https://doi.org/10.1002/andp.19013090310}

\bibitem{mach1883}
E. Mach, \emph{Die Mechanik in ihrer Entwicklung}, 1883.

\bibitem{hundert1931}
Various Authors, \emph{100 Authors Against Einstein}, 1931.

\bibitem{dingle1972}
H. Dingle, \emph{Science at the Crossroads}, 1972.

% Penrose and Terrell Effect
\bibitem{terrell1959}
J. Terrell, \emph{Invisibility of the Lorentz Contraction}, Phys. Rev., 1959.
\url{https://doi.org/10.1103/PhysRev.116.1041}

\bibitem{penrose1959}
R. Penrose, \emph{The Apparent Shape of a Relativistically Moving Sphere}, Proc. Cambridge Phil. Soc., 1959.
\url{https://doi.org/10.1017/S0305004100033776}

\bibitem{penrose1967}
R. Penrose, \emph{Twistor Algebra}, J. Math. Phys., 1967.
\url{https://doi.org/10.1063/1.1705200}

\bibitem{penrose2004}
R. Penrose, \emph{The Road to Reality}, 2004.

\bibitem{terrell2025}
J. Terrell et al., \emph{Modern Terrell-Penrose Visualization}, 2025.

\bibitem{weiskopf2000}
D. Weiskopf, \emph{Visualization of Four-dimensional Spacetimes}, 2000.

\bibitem{mueller2014}
T. Müller, \emph{Visual Appearance of Relativistically Moving Objects}, 2014.

\bibitem{hossenfelder2025}
S. Hossenfelder, \emph{YouTube: The Terrell Effect}, 2025.

% Quantum Gravity and String Theory
\bibitem{rovelli2004}
C. Rovelli, \emph{Quantum Gravity}, Cambridge University Press, 2004.

\bibitem{thiemann2007}
T. Thiemann, \emph{Modern Canonical Quantum Gravity}, Cambridge University Press, 2007.

\bibitem{ashtekar2004}
A. Ashtekar, J. Lewandowski, \emph{Background Independent Quantum Gravity}, Class. Quant. Grav., 2004.
\url{https://doi.org/10.1088/0264-9381/21/15/R01}

\bibitem{jacobson1995}
T. Jacobson, \emph{Thermodynamics of Spacetime}, Phys. Rev. Lett., 1995.
\url{https://doi.org/10.1103/PhysRevLett.75.1260}

\bibitem{maldacena1998}
J. Maldacena, \emph{The Large N Limit of Superconformal Field Theories}, Adv. Theor. Math. Phys., 1998.
\url{https://doi.org/10.4310/ATMP.1998.v2.n2.a1}

\bibitem{polchinski1998}
J. Polchinski, \emph{String Theory}, Cambridge University Press, 1998.

\bibitem{susskind1995}
L. Susskind, \emph{The World as a Hologram}, J. Math. Phys., 1995.
\url{https://doi.org/10.1063/1.531249}

\bibitem{verlinde2011}
E. Verlinde, \emph{On the Origin of Gravity}, JHEP, 2011.
\url{https://doi.org/10.1007/JHEP04(2011)029}

% Cosmology
\bibitem{hoyle1948}
F. Hoyle, \emph{A New Model for the Expanding Universe}, MNRAS, 1948.
\url{https://doi.org/10.1093/mnras/108.5.372}

\bibitem{bondi1948}
H. Bondi, T. Gold, \emph{The Steady-State Theory}, MNRAS, 1948.
\url{https://doi.org/10.1093/mnras/108.3.252}

\bibitem{zwicky1929}
F. Zwicky, \emph{On the Redshift of Spectral Lines}, Proc. Nat. Acad. Sci., 1929.
\url{https://doi.org/10.1073/pnas.15.10.773}

\bibitem{lopez2010}
C. Lopez-Corredoira, \emph{Tests of Cosmological Models}, Int. J. Mod. Phys. D, 2010.

\bibitem{lerner2014}
E. Lerner, \emph{Evidence for a Non-Expanding Universe}, 2014.

\bibitem{albrecht1999}
A. Albrecht, J. Magueijo, \emph{Variable Speed of Light}, Phys. Rev. D, 1999.
\url{https://doi.org/10.1103/PhysRevD.59.043516}

\bibitem{barrow1999}
J. Barrow, \emph{Cosmologies with Varying Light Speed}, Phys. Rev. D, 1999.
\url{https://doi.org/10.1103/PhysRevD.59.043515}

\bibitem{riess2022}
A. Riess et al., \emph{A Comprehensive Measurement of the Local Value of the Hubble Constant}, ApJ, 2022.
\url{https://doi.org/10.3847/2041-8213/ac5c5b}

\bibitem{desi2025}
DESI Collaboration, \emph{DESI Year 1 Results}, 2025.
\url{https://arxiv.org/abs/2404.03002}

\bibitem{divalentino2021}
E. Di Valentino et al., \emph{Planck Evidence for a Closed Universe}, Nat. Astron., 2021.
\url{https://doi.org/10.1038/s41550-019-0906-9}

% Conformal Field Theory
\bibitem{francesco1997}
P. Di Francesco et al., \emph{Conformal Field Theory}, Springer, 1997.

% Experimental Physics
\bibitem{pdg2024}
Particle Data Group, \emph{Review of Particle Physics}, 2024.
\url{https://pdg.lbl.gov/}

\bibitem{codata2019}
CODATA, \emph{Recommended Values of Fundamental Constants}, 2019.
\url{https://physics.nist.gov/cuu/Constants/}

\bibitem{newell2018}
D. Newell et al., \emph{The CODATA 2017 Values of h, e, k, and $N_A$}, Metrologia, 2018.
\url{https://doi.org/10.1088/1681-7575/aa950a}

\bibitem{muong2_2023}
Muon g-2 Collaboration, \emph{Measurement of the Anomalous Magnetic Moment of the Muon}, Phys. Rev. Lett., 2023.
\url{https://doi.org/10.1103/PhysRevLett.131.161802}

\bibitem{fermilab2023}
Fermilab, \emph{Muon g-2 Results}, 2023.
\url{https://muon-g-2.fnal.gov/}

\bibitem{atlas2023}
ATLAS Collaboration, \emph{Measurements at the LHC}, 2023.
\url{https://atlas.cern/}

\bibitem{atlas2023higgs}
ATLAS Collaboration, \emph{Higgs Boson Properties}, 2023.
\url{https://atlas.cern/}

\bibitem{cms2023top}
CMS Collaboration, \emph{Top Quark Measurements}, 2023.
\url{https://cms.cern/}

\bibitem{cms2024}
CMS Collaboration, \emph{Heavy Ion Collisions}, 2024.
\url{https://cms.cern/}

\bibitem{alice2023}
ALICE Collaboration, \emph{Quark-Gluon Plasma Studies}, 2023.
\url{https://alice-collaboration.web.cern.ch/}

\bibitem{kasevich2023}
M. Kasevich et al., \emph{Atom Interferometry}, 2023.

\bibitem{ludlow2015}
A. Ludlow et al., \emph{Optical Atomic Clocks}, Rev. Mod. Phys., 2015.
\url{https://doi.org/10.1103/RevModPhys.87.637}

\bibitem{brewer2019}
S. Brewer et al., \emph{Al$^+$ Optical Clock}, Phys. Rev. Lett., 2019.
\url{https://doi.org/10.1103/PhysRevLett.123.033201}

\bibitem{lisa2017}
LISA Collaboration, \emph{LISA Mission}, 2017.
\url{https://www.lisamission.org/}

% Fractal Physics
\bibitem{nottale1993}
L. Nottale, \emph{Fractal Space-Time and Microphysics}, World Scientific, 1993.

\bibitem{elnaschie2004}
M.S. El Naschie, \emph{E-Infinity Theory}, Chaos Solitons Fractals, 2004.

% Philosophy and Foundations
\bibitem{wheeler1990}
J.A. Wheeler, \emph{Information, Physics, Quantum}, 1990.

\bibitem{barbour1999}
J. Barbour, \emph{The End of Time}, Oxford University Press, 1999.

\bibitem{sciama1953}
D. Sciama, \emph{On the Origin of Inertia}, MNRAS, 1953.
\url{https://doi.org/10.1093/mnras/113.1.34}

% String Theory Extensions
\bibitem{becker2007}
K. Becker et al., \emph{String Theory and M-Theory}, Cambridge University Press, 2007.

% Missing References for g-2 Chapter
\bibitem{sm_g2_2025}
Muon g-2 Theory Initiative, \emph{Standard Model Prediction for g-2}, arXiv, 2025.
\url{https://arxiv.org/abs/2006.04822}

\bibitem{mug2_final_2025}
Muon g-2 Collaboration, \emph{Final Report on the Anomalous Magnetic Moment of the Muon}, Fermilab, 2025.
\url{https://muon-g-2.fnal.gov/}

\bibitem{pascher_t0_theory_2025}
J. Pascher, \emph{T0 Theory: Complete Framework}, 2025.
\url{https://github.com/jpascher/T0-Time-Mass-Duality/blob/main/2/pdf/systemEn.pdf}

\bibitem{peskin_schroeder_1995}
M.E. Peskin and D.V. Schroeder, \emph{An Introduction to Quantum Field Theory}, Westview Press, 1995.

\bibitem{parker_somov_2018}
R.H. Parker et al., \emph{Measurement of the Fine-Structure Constant}, Science, 2018.
\url{https://doi.org/10.1126/science.aap7706}

\bibitem{morel_rubidium_2020}
L. Morel et al., \emph{Determination of $\alpha$ from Rubidium Atom Recoil}, Nature, 2020.
\url{https://doi.org/10.1038/s41586-020-2964-7}

\bibitem{aoyama_theory_2020}
T. Aoyama et al., \emph{Theory of the Electron Anomalous Magnetic Moment}, Phys. Rep., 2020.
\url{https://doi.org/10.1016/j.physrep.2020.07.006}

\bibitem{fan_lattice_2023}
X. Fan et al., \emph{Hadronic Contributions from Lattice QCD}, Phys. Rev. D, 2023.

\bibitem{hanneke_electron_2008}
D. Hanneke et al., \emph{New Measurement of the Electron g-2}, Phys. Rev. Lett., 2008.
\url{https://doi.org/10.1103/PhysRevLett.100.120801}

% Additional T0 Theory References
\bibitem{pascher_higgs_connection_2025}
J. Pascher, \emph{Higgs Connection in T0 Theory}, 2025.
\url{https://github.com/jpascher/T0-Time-Mass-Duality/blob/main/2/pdf/T0_Energie_En.pdf}

\bibitem{T0_SI}
J. Pascher, \emph{T0 Theory and SI Units}, 2025.
\url{https://github.com/jpascher/T0-Time-Mass-Duality/blob/main/2/pdf/T0_SI_En.pdf}

\bibitem{T0_gravitational_constant}
J. Pascher, \emph{Gravitational Constant in T0 Framework}, 2025.
\url{https://github.com/jpascher/T0-Time-Mass-Duality/blob/main/2/pdf/T0_Gravitationskonstante_En.pdf}

\bibitem{T0_fine_structure}
J. Pascher, \emph{Fine Structure Constant Analysis}, 2025.
\url{https://github.com/jpascher/T0-Time-Mass-Duality/blob/main/2/pdf/T0_Feinstruktur_En.pdf}

\bibitem{bell_muon}
J.S. Bell, \emph{Muon Studies}, 1966.

\bibitem{QFT_T0}
J. Pascher, \emph{Quantum Field Theory in T0}, 2025.
\url{https://github.com/jpascher/T0-Time-Mass-Duality/blob/main/2/pdf/QFT_En.pdf}

\bibitem{planck2018}
Planck Collaboration, \emph{Planck 2018 Results}, A\&A, 2018.
\url{https://doi.org/10.1051/0004-6361/201833910}

\bibitem{pascher:t0_foundations}
J. Pascher, \emph{T0 Theory Foundations}, 2025.
\url{https://github.com/jpascher/T0-Time-Mass-Duality/blob/main/2/pdf/T0_Grundlagen_En.pdf}

\bibitem{pascher:geometric_formalism}
J. Pascher, \emph{Geometric Formalism in T0}, 2025.
\url{https://github.com/jpascher/T0-Time-Mass-Duality/blob/main/2/pdf/T0_Geometrische_Kosmologie_En.pdf}

\bibitem{riess2019}
A. Riess et al., \emph{Hubble Constant Measurements}, ApJ, 2019.
\url{https://doi.org/10.3847/1538-4357/ab1422}

\bibitem{t0_kosmologie}
J. Pascher, \emph{T0 Kosmologie}, 2025.
\url{https://github.com/jpascher/T0-Time-Mass-Duality/blob/main/2/pdf/T0_Kosmologie_En.pdf}

\bibitem{hossenfelder_single_clock_video}
S. Hossenfelder, \emph{Single Clock Video}, YouTube, 2025.
\url{https://www.youtube.com/c/SabineHossenfelder}

\bibitem{video2025}
Various, \emph{Video References}, 2025.

\bibitem{unnikrishnan2004}
C.S. Unnikrishnan, \emph{Gravity Studies}, 2004.

\bibitem{peratt1992}
A. Peratt, \emph{Plasma Cosmology}, 1992.
\url{https://github.com/jpascher/T0-Time-Mass-Duality/blob/main/2/pdf/T0_peratt_En.pdf}

\bibitem{T0_tm_erweiterung}
J. Pascher, \emph{T0 Time-Mass Extension}, 2025.
\url{https://github.com/jpascher/T0-Time-Mass-Duality/blob/main/2/pdf/T0_tm-erweiterung-x6_En.pdf}

\bibitem{T0_g2_erweiterung}
J. Pascher, \emph{T0 g-2 Extension}, 2025.
\url{https://github.com/jpascher/T0-Time-Mass-Duality/blob/main/2/pdf/T0_g2-erweiterung-4_En.pdf}

\bibitem{T0_netze_en}
J. Pascher, \emph{T0 Networks}, 2025.
\url{https://github.com/jpascher/T0-Time-Mass-Duality/blob/main/2/pdf/T0_netze_En.pdf}

\bibitem{Adams1925}
W. Adams, \emph{Gravitational Redshift}, 1925.
\url{https://doi.org/10.1073/pnas.11.7.382}

\bibitem{Ashby2003}
N. Ashby, \emph{Relativity in GPS}, Living Rev. Rel., 2003.
\url{https://doi.org/10.12942/lrr-2003-1}

\bibitem{Bertotti2003}
B. Bertotti et al., \emph{Cassini Doppler Test}, Nature, 2003.
\url{https://doi.org/10.1038/nature01997}

\bibitem{Bolton2008}
A. Bolton et al., \emph{Gravitational Lensing}, 2008.

\bibitem{Born2013}
M. Born, \emph{Einstein's Theory of Relativity}, Dover, 2013.

\bibitem{Brans1961}
C. Brans and R.H. Dicke, \emph{Mach's Principle}, Phys. Rev., 1961.
\url{https://doi.org/10.1103/PhysRev.124.925}

\bibitem{Dirac1927}
P.A.M. Dirac, \emph{Quantum Mechanics}, Proc. Roy. Soc., 1927.
\url{https://doi.org/10.1098/rspa.1927.0039}

\bibitem{Duhem1906}
P. Duhem, \emph{Theory of Physics}, 1906.

\bibitem{Einstein1905}
A. Einstein, \emph{Special Relativity}, Ann. Phys., 1905.
\url{https://doi.org/10.1002/andp.19053221004}

\bibitem{Feynman2006}
R. Feynman, \emph{QED: The Strange Theory of Light and Matter}, 2006.

\bibitem{Griffiths2017}
D. Griffiths, \emph{Introduction to Quantum Mechanics}, 2017.

\bibitem{Jackson1999}
J.D. Jackson, \emph{Classical Electrodynamics}, 1999.

\bibitem{Kaluza1921}
T. Kaluza, \emph{Five-Dimensional Theory}, 1921.

\bibitem{Klein1926}
O. Klein, \emph{Quantum Theory and Relativity}, 1926.

\bibitem{Kuhn1962}
T. Kuhn, \emph{Structure of Scientific Revolutions}, 1962.

\bibitem{Kuhn1977}
T. Kuhn, \emph{Essential Tension}, 1977.

\bibitem{Ludlow2015}
A. Ludlow et al., \emph{Optical Atomic Clocks}, Rev. Mod. Phys., 2015.
\url{https://doi.org/10.1103/RevModPhys.87.637}

\bibitem{Maxwell1873}
J.C. Maxwell, \emph{Treatise on Electricity and Magnetism}, 1873.

\bibitem{McGaugh2016}
S. McGaugh et al., \emph{Radial Acceleration Relation}, Phys. Rev. Lett., 2016.
\url{https://doi.org/10.1103/PhysRevLett.117.201101}

\bibitem{Mohr2016}
P. Mohr et al., \emph{CODATA Values}, Rev. Mod. Phys., 2016.
\url{https://doi.org/10.1103/RevModPhys.88.035009}

\bibitem{PDG2020}
Particle Data Group, \emph{Review of Particle Physics}, Prog. Theor. Exp. Phys., 2020.
\url{https://pdg.lbl.gov/}

\bibitem{Parker2018}
R. Parker et al., \emph{Measurement of $\alpha$}, Science, 2018.
\url{https://doi.org/10.1126/science.aap7706}

\bibitem{Peskin1995}
M. Peskin and D. Schroeder, \emph{QFT}, 1995.

\bibitem{Planck1900}
M. Planck, \emph{Quantum Theory}, 1900.

\bibitem{Planck2020}
Planck Collaboration, \emph{Planck 2020 Results}, 2020.
\url{https://doi.org/10.1051/0004-6361/201833910}

\bibitem{Poincare1905}
H. Poincaré, \emph{Dynamics of the Electron}, 1905.

\bibitem{Pound1960}
R.V. Pound and G.A. Rebka, \emph{Gravitational Redshift}, Phys. Rev. Lett., 1960.
\url{https://doi.org/10.1103/PhysRevLett.4.337}

\bibitem{Quine1951}
W.V. Quine, \emph{Two Dogmas of Empiricism}, 1951.

\bibitem{Quinn2013}
T. Quinn et al., \emph{Gravitational Constant}, 2013.
\url{https://doi.org/10.1103/PhysRevLett.111.101102}

\bibitem{Randall1999}
L. Randall and R. Sundrum, \emph{Extra Dimensions}, Phys. Rev. Lett., 1999.
\url{https://doi.org/10.1103/PhysRevLett.83.3370}

\bibitem{Riess1998}
A. Riess et al., \emph{Type Ia Supernovae}, AJ, 1998.
\url{https://doi.org/10.1086/300499}

\bibitem{Shapiro1971}
I. Shapiro et al., \emph{Time Delay Test}, Phys. Rev. Lett., 1971.
\url{https://doi.org/10.1103/PhysRevLett.26.1132}

\bibitem{Sommerfeld1916}
A. Sommerfeld, \emph{Fine Structure}, 1916.

\bibitem{Suyu2017}
S. Suyu et al., \emph{Time Delay Cosmography}, MNRAS, 2017.
\url{https://doi.org/10.1093/mnras/stx483}

\bibitem{T0Theory}
J. Pascher, \emph{T0 Theory}, 2025.
\url{https://github.com/jpascher/T0-Time-Mass-Duality/blob/main/2/pdf/systemEn.pdf}

\bibitem{T0_Feinstruktur}
J. Pascher, \emph{Fine Structure in T0}, 2025.
\url{https://github.com/jpascher/T0-Time-Mass-Duality/blob/main/2/pdf/T0_Feinstruktur_En.pdf}

\bibitem{Uzan2003}
J.-P. Uzan, \emph{Constants Variation}, Rev. Mod. Phys., 2003.
\url{https://doi.org/10.1103/RevModPhys.75.403}

\bibitem{Webb2001}
J.K. Webb et al., \emph{Fine Structure Constant}, Phys. Rev. Lett., 2001.
\url{https://doi.org/10.1103/PhysRevLett.87.091301}

\bibitem{Weinberg1979}
S. Weinberg, \emph{Cosmological Constant}, Rev. Mod. Phys., 1979.

\bibitem{Weinberg1989}
S. Weinberg, \emph{Cosmological Constant Problem}, 1989.
\url{https://doi.org/10.1103/RevModPhys.61.1}

\bibitem{Weinberg1995}
S. Weinberg, \emph{Quantum Theory of Fields}, 1995.

\bibitem{Will2014}
C. Will, \emph{Theory and Experiment in Gravitational Physics}, 2014.
\url{https://doi.org/10.12942/lrr-2014-4}

\bibitem{dirac_principles}
P.A.M. Dirac, \emph{Principles of Quantum Mechanics}, 1930.

\bibitem{einstein_1917}
A. Einstein, \emph{Cosmological Considerations}, 1917.

\bibitem{jwst_early}
JWST Collaboration, \emph{Early Universe Observations}, 2023.
\url{https://www.jwst.nasa.gov/}

\bibitem{katrin_2022}
KATRIN Collaboration, \emph{Neutrino Mass}, 2022.
\url{https://doi.org/10.1038/s41567-021-01463-1}

\bibitem{pascher:fundamentals}
J. Pascher, \emph{T0 Fundamentals}, 2025.
\url{https://github.com/jpascher/T0-Time-Mass-Duality/blob/main/2/pdf/T0_Grundlagen_En.pdf}

\bibitem{pascher:g2_rev9}
J. Pascher, \emph{g-2 Analysis Rev9}, 2025.
\url{https://github.com/jpascher/T0-Time-Mass-Duality/blob/main/2/pdf/T0_Anomale-g2-9_En.pdf}

\bibitem{pascher:ml_addendum}
J. Pascher, \emph{ML Addendum}, 2025.
\url{https://github.com/jpascher/T0-Time-Mass-Duality/blob/main/2/pdf/T0-QFT-ML_Addendum_En.pdf}

\bibitem{pascher_beta_derivation_2025}
J. Pascher, \emph{Beta Derivation}, 2025.
\url{https://github.com/jpascher/T0-Time-Mass-Duality/blob/main/2/pdf/DerivationVonBetaEn.pdf}

\bibitem{pascher_cmb_en}
J. Pascher, \emph{CMB Analysis in T0}, 2025.
\url{https://github.com/jpascher/T0-Time-Mass-Duality/blob/main/2/pdf/Zwei-Dipole-CMB_En.pdf}

\bibitem{pascher_cosmos_en}
J. Pascher, \emph{Cosmos in T0 Theory}, 2025.
\url{https://github.com/jpascher/T0-Time-Mass-Duality/blob/main/2/pdf/cosmic_En.pdf}

\bibitem{pascher_derivation_beta_2025}
J. Pascher, \emph{Derivation of Beta}, 2025.
\url{https://github.com/jpascher/T0-Time-Mass-Duality/blob/main/2/pdf/DerivationVonBetaEn.pdf}

\bibitem{pascher_gravitation_en}
J. Pascher, \emph{Gravitation in T0}, 2025.
\url{https://github.com/jpascher/T0-Time-Mass-Duality/blob/main/2/pdf/gravitationskonstante_En.pdf}

\bibitem{pascher_lagrangian_2025}
J. Pascher, \emph{Lagrangian in T0}, 2025.
\url{https://github.com/jpascher/T0-Time-Mass-Duality/blob/main/2/pdf/T0_lagrndian_En.pdf}

\bibitem{pascher_lagrangian_en}
J. Pascher, \emph{Lagrangian Framework}, 2025.
\url{https://github.com/jpascher/T0-Time-Mass-Duality/blob/main/2/pdf/LagrandianVergleichEn.pdf}

\bibitem{pascher_lagrangian_extended_2025}
J. Pascher, \emph{Extended Lagrangian Formalism}, 2025.
\url{https://github.com/jpascher/T0-Time-Mass-Duality/blob/main/2/pdf/T0_lagrndian_En.pdf}

\bibitem{pascher_mathematical_structure_2025}
J. Pascher, \emph{Mathematical Structure of T0 Theory}, 2025.
\url{https://github.com/jpascher/T0-Time-Mass-Duality/blob/main/2/pdf/Mathematische_struktur_En.pdf}

\bibitem{pascher_muon_g2_2025}
J. Pascher, \emph{Muon g-2 in T0}, 2025.
\url{https://github.com/jpascher/T0-Time-Mass-Duality/blob/main/2/pdf/T0_Anomale-g2-9_En.pdf}

\bibitem{pascher_pragmatic_2025}
J. Pascher, \emph{Pragmatic Approach}, 2025.

\bibitem{pascher_t0_energy_2025}
J. Pascher, \emph{T0 Energy Formalism}, 2025.
\url{https://github.com/jpascher/T0-Time-Mass-Duality/blob/main/2/pdf/T0-Energie_En.pdf}

\bibitem{pascher_unified_2025}
J. Pascher, \emph{Unified T0 Theory}, 2025.
\url{https://github.com/jpascher/T0-Time-Mass-Duality/blob/main/2/pdf/T0_unified_report.pdf}

\bibitem{sciencedaily2025}
Science Daily, \emph{Physics News}, 2025.
\url{https://www.sciencedaily.com/}

\bibitem{weinberg_1989}
S. Weinberg, \emph{The Cosmological Constant Problem}, Rev. Mod. Phys., 1989.
\url{https://doi.org/10.1103/RevModPhys.61.1}

\bibitem{wiki_bell}
Wikipedia, \emph{Bell's Theorem}, 2025.
\url{https://en.wikipedia.org/wiki/Bell\%27s_theorem}

\bibitem{vanFraassen1980}
B. van Fraassen, \emph{The Scientific Image}, Oxford University Press, 1980.

\bibitem{terrell_single_clock_nature_2024}
J. Terrell, \emph{Single Clock Nature}, Nature, 2024.

% Additional T0 Documents
\bibitem{137_doc}
J. Pascher, \emph{The Number 137 in T0 Theory}, 2025.
\url{https://github.com/jpascher/T0-Time-Mass-Duality/blob/main/2/pdf/137_En.pdf}

\bibitem{ampere_low}
J. Pascher, \emph{Ampere's Law in T0}, 2025.
\url{https://github.com/jpascher/T0-Time-Mass-Duality/blob/main/2/pdf/Amper_Low_En.pdf}

\bibitem{bell_theorem}
J. Pascher, \emph{Bell's Theorem in T0}, 2025.
\url{https://github.com/jpascher/T0-Time-Mass-Duality/blob/main/2/pdf/Bell_En.pdf}

\bibitem{bewegungsenergie}
J. Pascher, \emph{Kinetic Energy in T0}, 2025.
\url{https://github.com/jpascher/T0-Time-Mass-Duality/blob/main/2/pdf/Bewegungsenergie_En.pdf}

\bibitem{emc2}
J. Pascher, \emph{E=mc² in T0 Framework}, 2025.
\url{https://github.com/jpascher/T0-Time-Mass-Duality/blob/main/2/pdf/E-mc2_En.pdf}

\bibitem{formeln_energiebasiert}
J. Pascher, \emph{Energy-Based Formulas}, 2025.
\url{https://github.com/jpascher/T0-Time-Mass-Duality/blob/main/2/pdf/Formeln_Energiebasiert_En.pdf}

\bibitem{hannah}
J. Pascher, \emph{Hannah Document}, 2025.
\url{https://github.com/jpascher/T0-Time-Mass-Duality/blob/main/2/pdf/Hannah_En.pdf}

\bibitem{ho_doc}
J. Pascher, \emph{H0 Analysis}, 2025.
\url{https://github.com/jpascher/T0-Time-Mass-Duality/blob/main/2/pdf/Ho_En.pdf}

\bibitem{markov}
J. Pascher, \emph{Markov Processes in T0}, 2025.
\url{https://github.com/jpascher/T0-Time-Mass-Duality/blob/main/2/pdf/Markov_En.pdf}

\bibitem{elimination_mass}
J. Pascher, \emph{Elimination of Mass}, 2025.
\url{https://github.com/jpascher/T0-Time-Mass-Duality/blob/main/2/pdf/EliminationOfMassEn.pdf}

\bibitem{elimination_mass_dirac}
J. Pascher, \emph{Dirac Equation Mass Elimination}, 2025.
\url{https://github.com/jpascher/T0-Time-Mass-Duality/blob/main/2/pdf/Elimination_Of_Mass_Dirac_TabelleEn.pdf}

\bibitem{feinstrukturkonstante}
J. Pascher, \emph{Fine Structure Constant}, 2025.
\url{https://github.com/jpascher/T0-Time-Mass-Duality/blob/main/2/pdf/FeinstrukturkonstanteEn.pdf}

\bibitem{neutrino_formel}
J. Pascher, \emph{Neutrino Formula}, 2025.
\url{https://github.com/jpascher/T0-Time-Mass-Duality/blob/main/2/pdf/neutrino-Formel_En.pdf}

\bibitem{neutrinos}
J. Pascher, \emph{Neutrinos in T0}, 2025.
\url{https://github.com/jpascher/T0-Time-Mass-Duality/blob/main/2/pdf/T0_Neutrinos_En.pdf}

\bibitem{koide_formel}
J. Pascher, \emph{Koide Formula in T0}, 2025.
\url{https://github.com/jpascher/T0-Time-Mass-Duality/blob/main/2/pdf/T0_koide-formel-3_En.pdf}

\bibitem{teilchenmassen}
J. Pascher, \emph{Particle Masses}, 2025.
\url{https://github.com/jpascher/T0-Time-Mass-Duality/blob/main/2/pdf/Teilchenmassen_En.pdf}

\bibitem{t0_teilchenmassen}
J. Pascher, \emph{T0 Particle Masses}, 2025.
\url{https://github.com/jpascher/T0-Time-Mass-Duality/blob/main/2/pdf/T0_Teilchenmassen_En.pdf}

\bibitem{penrose_doc}
J. Pascher, \emph{Penrose Analysis in T0}, 2025.
\url{https://github.com/jpascher/T0-Time-Mass-Duality/blob/main/2/pdf/T0_penrose_En.pdf}

\bibitem{photonenchip}
J. Pascher, \emph{Photon Chip Implementation}, 2025.
\url{https://github.com/jpascher/T0-Time-Mass-Duality/blob/main/2/pdf/T0_photonenchip-china_En.pdf}

\bibitem{threeclock}
J. Pascher, \emph{Three Clock Experiment}, 2025.
\url{https://github.com/jpascher/T0-Time-Mass-Duality/blob/main/2/pdf/T0_threeclock_En.pdf}

\bibitem{redshift_deflection}
J. Pascher, \emph{Redshift and Deflection}, 2025.
\url{https://github.com/jpascher/T0-Time-Mass-Duality/blob/main/2/pdf/redshift_deflection_En.pdf}

\bibitem{scheinbar_instantan}
J. Pascher, \emph{Apparent Instantaneity}, 2025.
\url{https://github.com/jpascher/T0-Time-Mass-Duality/blob/main/2/pdf/scheinbar_instantan_En.pdf}

\bibitem{universale_ableitung}
J. Pascher, \emph{Universal Derivation}, 2025.
\url{https://github.com/jpascher/T0-Time-Mass-Duality/blob/main/2/pdf/universale-ableitung_En.pdf}

\bibitem{xi_parameter}
J. Pascher, \emph{Xi Parameter for Particles}, 2025.
\url{https://github.com/jpascher/T0-Time-Mass-Duality/blob/main/2/pdf/xi_parmater_partikel_En.pdf}

\bibitem{xi_ursprung}
J. Pascher, \emph{Origin of Xi}, 2025.
\url{https://github.com/jpascher/T0-Time-Mass-Duality/blob/main/2/pdf/T0_xi_ursprung_En.pdf}

\bibitem{zeit}
J. Pascher, \emph{Time in T0 Theory}, 2025.
\url{https://github.com/jpascher/T0-Time-Mass-Duality/blob/main/2/pdf/Zeit_En.pdf}

\bibitem{zeit_konstant}
J. Pascher, \emph{Time Constant}, 2025.
\url{https://github.com/jpascher/T0-Time-Mass-Duality/blob/main/2/pdf/Zeit-konstant_En.pdf}

\bibitem{zusammenfassung}
J. Pascher, \emph{Summary of T0 Theory}, 2025.
\url{https://github.com/jpascher/T0-Time-Mass-Duality/blob/main/2/pdf/Zusammenfassung_En.pdf}

\bibitem{rsa}
J. Pascher, \emph{RSA in T0 Framework}, 2025.
\url{https://github.com/jpascher/T0-Time-Mass-Duality/blob/main/2/pdf/RSA_En.pdf}

\bibitem{qat}
J. Pascher, \emph{Quantum Atomic Theory}, 2025.
\url{https://github.com/jpascher/T0-Time-Mass-Duality/blob/main/2/pdf/T0_QAT_En.pdf}

\bibitem{qm_qft_rt}
J. Pascher, \emph{QM, QFT and RT Unification}, 2025.
\url{https://github.com/jpascher/T0-Time-Mass-Duality/blob/main/2/pdf/T0_QM-QFT-RT_En.pdf}

\bibitem{qm_optimierung}
J. Pascher, \emph{QM Optimization}, 2025.
\url{https://github.com/jpascher/T0-Time-Mass-Duality/blob/main/2/pdf/T0_QM-optimierung_En.pdf}

\bibitem{vollstaendige_berechnungen}
J. Pascher, \emph{Complete Calculations}, 2025.
\url{https://github.com/jpascher/T0-Time-Mass-Duality/blob/main/2/pdf/T0_Vollstaendige_Berchnungen_En.pdf}

\bibitem{synergetics}
J. Pascher, \emph{T0 Theory vs Synergetics}, 2025.
\url{https://github.com/jpascher/T0-Time-Mass-Duality/blob/main/2/pdf/T0-Theory-vs-Synergetics_En.pdf}

\bibitem{modell_uebersicht}
J. Pascher, \emph{T0 Model Overview}, 2025.
\url{https://github.com/jpascher/T0-Time-Mass-Duality/blob/main/2/pdf/T0_Modell_Uebersicht_En.pdf}

\bibitem{mnras_widerlegung}
J. Pascher, \emph{MNRAS Analysis}, 2025.
\url{https://github.com/jpascher/T0-Time-Mass-Duality/blob/main/2/pdf/T0_Analyse_MNRAS_Widerlegung_En.pdf}

\bibitem{anomale_magnetische_momente}
J. Pascher, \emph{Anomalous Magnetic Moments}, 2025.
\url{https://github.com/jpascher/T0-Time-Mass-Duality/blob/main/2/pdf/T0_Anomale_Magnetische_Momente_En.pdf}

\bibitem{sieben_fragen}
J. Pascher, \emph{Seven Questions in T0}, 2025.
\url{https://github.com/jpascher/T0-Time-Mass-Duality/blob/main/2/pdf/T0_7-fragen-3_En.pdf}

\bibitem{detailierte_leptonen}
J. Pascher, \emph{Detailed Lepton Anomaly}, 2025.
\url{https://github.com/jpascher/T0-Time-Mass-Duality/blob/main/2/pdf/detailierte_formel_leptonen_anemal_En.pdf}

\bibitem{parameterherleitung}
J. Pascher, \emph{Parameter Derivation}, 2025.
\url{https://github.com/jpascher/T0-Time-Mass-Duality/blob/main/2/pdf/parameterherleitung_En.pdf}

\bibitem{verhaeltnis_absolut}
J. Pascher, \emph{Absolute Ratios in T0}, 2025.
\url{https://github.com/jpascher/T0-Time-Mass-Duality/blob/main/2/pdf/T0_verhaeltnis-absolut_En.pdf}

\bibitem{xi_und_e}
J. Pascher, \emph{Xi and Energy}, 2025.
\url{https://github.com/jpascher/T0-Time-Mass-Duality/blob/main/2/pdf/T0_xi-und-e_En.pdf}

\bibitem{umkehrung}
J. Pascher, \emph{Inversion in T0}, 2025.
\url{https://github.com/jpascher/T0-Time-Mass-Duality/blob/main/2/pdf/T0_umkehrung_En.pdf}

\bibitem{esm_analysis}
J. Pascher, \emph{T0 vs ESM Conceptual Analysis}, 2025.
\url{https://github.com/jpascher/T0-Time-Mass-Duality/blob/main/2/pdf/T0vsESM_ConceptualAnalysis_En.pdf}

\end{thebibliography}

\end{document}
