% Standalone document: E-mc2_En
% Uses shared T0 header
% T0 Standalone Header - German Version
% Gemeinsamer Header für alle deutschen Standalone-Dokumente

\documentclass[12pt,a4paper]{article}
\usepackage[utf8]{inputenc}
\usepackage[T1]{fontenc}
\usepackage[ngerman]{babel}
\usepackage{lmodern}

% Mathematics
\usepackage{amsmath,amssymb,amsthm}
\usepackage{physics}
\usepackage{siunitx}

% Layout
\usepackage[left=2.5cm,right=2.5cm,top=2.5cm,bottom=2.5cm,headheight=15pt]{geometry}
\usepackage{fancyhdr}
\usepackage{titlesec}

% Tables and Graphics
\usepackage{booktabs}
\usepackage{array}
\usepackage{longtable}
\usepackage{graphicx}
\usepackage{tikz}
\usetikzlibrary{arrows.meta,positioning,shapes.geometric}

% Colors and Boxes
\usepackage{xcolor}
\usepackage[most]{tcolorbox}
\usepackage{mdframed}

% Additional packages
\usepackage{enumitem}
\usepackage{float}
\usepackage{caption}
\usepackage{subcaption}
\usepackage{multirow}
\usepackage{colortbl}
\usepackage{pdflscape}
\usepackage{algorithm}
\usepackage{algpseudocode}
\usepackage{listings}
\usepackage{hyperref}

% Define colors
\definecolor{t0blue}{RGB}{0,51,102}
\definecolor{t0green}{RGB}{0,102,51}
\definecolor{t0red}{RGB}{153,0,0}
\definecolor{deepblue}{RGB}{0,51,102}
\definecolor{deepgreen}{RGB}{0,102,51}
\definecolor{deepred}{RGB}{153,0,0}
\definecolor{boxgray}{RGB}{240,240,240}
\definecolor{t0yellow}{RGB}{255,200,0}
\definecolor{boxblue}{RGB}{230,240,255}
\definecolor{boxgreen}{RGB}{230,255,230}
\definecolor{boxorange}{RGB}{255,240,230}
\definecolor{boxyellow}{RGB}{255,255,230}

% Custom tcolorbox environments
\newtcolorbox{fundamental}[1][]{
  colback=blue!5!white,
  colframe=blue!75!black,
  title=#1,
  fonttitle=\bfseries,
  breakable
}

\newtcolorbox{derivation}[1][]{
  colback=green!5!white,
  colframe=green!75!black,
  title=#1,
  fonttitle=\bfseries,
  breakable
}

\newtcolorbox{result}[1][]{
  colback=orange!5!white,
  colframe=orange!75!black,
  title=#1,
  fonttitle=\bfseries,
  breakable
}

\newtcolorbox{summary}[1][]{
  colback=gray!10!white,
  colframe=gray!75!black,
  title=#1,
  fonttitle=\bfseries,
  breakable
}

\newtcolorbox{comparison}[1][]{
  colback=purple!5!white,
  colframe=purple!75!black,
  title=#1,
  fonttitle=\bfseries,
  breakable
}

\newtcolorbox{relation}[1][]{
  colback=cyan!5!white,
  colframe=cyan!75!black,
  title=#1,
  fonttitle=\bfseries,
  breakable
}

\newtcolorbox{principle}[1][]{
  colback=yellow!5!white,
  colframe=yellow!75!black,
  title=#1,
  fonttitle=\bfseries,
  breakable
}

\newtcolorbox{insight}[1][]{colback=blue!5,colframe=t0blue,title={#1},fonttitle=\bfseries,breakable}
\newtcolorbox{discovery}[1][]{colback=green!5,colframe=t0green,title={#1},fonttitle=\bfseries,breakable}
\newtcolorbox{newperspective}[1][]{colback=yellow!5,colframe=orange,title={#1},fonttitle=\bfseries,breakable}
\newtcolorbox{revelation}[1][]{colback=red!5,colframe=t0red,title={#1},fonttitle=\bfseries,breakable}
\newtcolorbox{keypoint}[1][]{colback=blue!5,colframe=t0blue,title={#1},fonttitle=\bfseries,breakable}
\newtcolorbox{evidence}[1][]{colback=green!5,colframe=t0green,title={#1},fonttitle=\bfseries,breakable}
\newtcolorbox{conclusion}[1][]{colback=gray!5,colframe=gray,title={#1},fonttitle=\bfseries,breakable}
\newtcolorbox{significance}[1][]{colback=yellow!5,colframe=orange,title={#1},fonttitle=\bfseries,breakable}
\newtcolorbox{philosophical}[1][]{colback=purple!5,colframe=purple,title={#1},fonttitle=\bfseries,breakable}
\newtcolorbox{implication}[1][]{colback=cyan!5,colframe=cyan,title={#1},fonttitle=\bfseries,breakable}
\newtcolorbox{perspective}[1][]{colback=blue!5,colframe=t0blue,title={#1},fonttitle=\bfseries,breakable}
\newtcolorbox{revolutionary}[1][]{colback=red!5,colframe=t0red,title={#1},fonttitle=\bfseries,breakable}
\newtcolorbox{technical}[1][]{colback=gray!5,colframe=gray!75!black,title={#1},fonttitle=\bfseries,breakable}
\newtcolorbox{notation}[1][]{colback=yellow!5,colframe=yellow!75!black,title={#1},fonttitle=\bfseries,breakable}

% Theorem environments
\newtheorem{theorem}{Satz}[section]
\newtheorem{lemma}[theorem]{Lemma}
\newtheorem{corollary}[theorem]{Korollar}
\newtheorem{proposition}[theorem]{Proposition}
\newtheorem{definition}[theorem]{Definition}
\newtheorem{example}[theorem]{Beispiel}
\newtheorem{remark}[theorem]{Bemerkung}
\newtheorem{note}[theorem]{Anmerkung}

% Additional environments
\newenvironment{treatise}{\begin{quote}}{\end{quote}}
\newenvironment{gemeinsam}{\begin{quote}}{\end{quote}}
\newenvironment{vergleich}{\begin{quote}}{\end{quote}}
\newenvironment{vorteil}{\begin{quote}}{\end{quote}}
\newenvironment{quantum}{\begin{quote}}{\end{quote}}

% T0-specific commands
\newcommand{\Tzero}{T$_0$}
\newcommand{\xipar}{\xi}
\newcommand{\Tfield}{T}
\newcommand{\Efield}{\mathcal{E}}
\newcommand{\meff}{m_{\text{eff}}}
\newcommand{\Eabs}{E_{\text{abs}}}
\newcommand{\taupar}{\tau}

% Header setup
\pagestyle{fancy}
\fancyhf{}
\fancyhead[L]{\leftmark}
\fancyhead[R]{\thepage}
\renewcommand{\headrulewidth}{0.4pt}

% Hyperref setup
\hypersetup{
    colorlinks=true,
    linkcolor=blue,
    filecolor=magenta,
    urlcolor=cyan,
    citecolor=blue,
    pdftitle={T0 Theory Document},
    pdfauthor={Johann Pascher}
}

% German quotation marks
%\newcommand{\dq}[1]{\glqq{}#1\grqq{}}


\title{E=mc² Revisited}
\author{Johann Pascher}
\date{2025}

\begin{document}

\maketitle

\chapter{E=mc² Revisited}

	
	\title{E=mc² = E=m: The Constants Illusion Exposed \\
		Why Einstein's c-Konstante conceals the fundamental error \\
		\groß From Dynamic Ratios to the Constants Illusion}
	\author{Johann Pascher\\
		Department of Communications Engineering, \\Higher Technical Federal Institute (HTL), Leonding, Austria\\
	\date{\today}
	
	
	\begin{abstract}
		This Arbeit reveals the central point of Einstein's Relativität theory: E=mc² is mathematically identical to E=m. The nur difference lies in Einstein's treatment of c as a "Konstante" stattdessen of a dynamic Verhältnis. By fixing c = 299,792,458 m/s, the natural Zeit-Masse duality T·m = 1 is artificially "frozen," leading to apparent complexity. The T0 theory shows: c is not a fundamental law of nature, but nur a Verhältnis das must be Variable if Zeit is Variable. Einstein's error was not E=mc² itself, but the Konstante-setting of c.
	\end{abstract}
	
	\newpage
	
	\section{The Central Thesis: E=mc² = E=m}
	
	\begin{tcolorbox}[colback=red!5!white,colframe=red!75!black,title=The Fundamental Recognition]
		\textbf{E=mc² and E=m are mathematically identical!}
		
		The nur difference: Einstein treats c as a "Konstante," obwohl c is a dynamic Verhältnis.
		
		\textbf{Einstein's error}: c = 299,792,458 m/s = Konstante
		
		\textbf{T0 truth}: c = L/T = Variable Verhältnis
	\end{tcolorbox}
	
	\subsection{The Mathematical Identity}
	
	\textbf{In natural Einheiten}:
	\begin{equation}
		E = mc^2 = m \times c^2 = m \times 1^2 = m
	\end{equation}
	
	\textbf{This is not an Näherung - dies is exactly the gleich Gleichung!}
	
	\subsection{What is c really?}
	
	\begin{equation}
		c = \frac{\text{Length}}{\text{Time}} = \frac{L}{T}
	\end{equation}
	
	\textbf{c is a Verhältnis, not a natural Konstante!}
	
	\section{Einstein's Fundamental Error: The Constant-Setting}
	
	\subsection{The Act of Constant-Setting}
	
	Einstein set: $c = 299,792,458$ m/s = \textbf{Konstante}
	
	\textbf{What does dies Mittelwert?}
	\begin{equation}
		c = \frac{L}{T} = \text{constant} \quad \Rightarrow \quad \frac{L}{T} = \text{fixed}
	\end{equation}
	
	\textbf{Implication}: If L and T can vary, their \textbf{Verhältnis} must remain Konstante.
	
	\subsection{The Problem of Time Variability}
	
	\textbf{Einstein recognized himself}: Time dilates!
	\begin{equation}
		t' = \gamma t \quad \text{(time is variable)}
	\end{equation}
	
	\textbf{But gleichzeitig he claimed}: 
	\begin{equation}
		c = \frac{L}{T} = \text{constant}
	\end{equation}
	
	\textbf{This is a logical contradiction!}
	
	\subsection{The T0 Resolution}
	
	\textbf{T0 Einsicht}: $\Tfield \cdot m = 1$
	
	This means:
	\begin{itemize}
		\item Time $\Tfield$ \textbf{must} be Variable (coupled to Masse)
		\item Therefore $c = L/T$ \textbf{cannot} be Konstante
		\item $c$ is a \textbf{dynamic Verhältnis}, not a Konstante
	\end{itemize}
	
	\section{The Constants Illusion: How it Works}
	
	\subsection{The Mechanism of the Illusion}
	
	\textbf{Step 1}: Einstein sets c = Konstante
	\begin{equation}
		c = 299,792,458 \text{ m/s} = \text{fixed}
	\end{equation}
	
	\textbf{Step 2}: Time becomes "frozen" by dies
	\begin{equation}
		T = \frac{L}{c} = \frac{L}{\text{constant}} = \text{apparently determined}
	\end{equation}
	
	\textbf{Step 3}: Time dilation becomes "mysterious Effekt"
	\begin{equation}
		t' = \gamma t \quad \text{(why? MATHBLOCK5ENDMATH complicated relativity theory)}
	\end{equation}
	
	\subsection{What Really Happens (T0 View)}
	
	\textbf{Reality}: Time is naturally Variable through $\Tfield \cdot m = 1$
	
	\textbf{Einstein's Konstante-setting} "freezes" dies natural variability artificially
	
	\textbf{Result}: One needs complicated theory to repair the "frozen" Dynamik
	
	\section{c as Ratio vs. c as Constant}
	
	\subsection{c as Natural Ratio (T0)}
	
	\begin{equation}
		c(x,t) = \frac{L(x,t)}{T(x,t)}
	\end{equation}
	
	\textbf{Properties}:
	\begin{itemize}
		\item $c$ varies with location and Zeit
		\item $c$ follows the Zeit-Masse duality
		\item No artificial Konstanten
		\item Natural simplicity: $E = m$
	\end{itemize}
	
	\subsection{c as Artificial Constant (Einstein)}
	
	\begin{equation}
		c = 299,792,458 \text{ m/s} = \text{constant everywhere}
	\end{equation}
	
	\textbf{Problems}:
	\begin{itemize}
		\item Contradiction to Zeit dilation
		\item Artificial "freezing" of Zeit Dynamik
		\item Complicated repair mathematics needed
		\item Inflated Formel: $E = mc^2$
	\end{itemize}
	
	\section{The Time Dilation Paradox}
	
	\subsection{Einstein's Contradiction Exposed}
	
	\textbf{Einstein claims gleichzeitig}:
	\begin{align}
		c &= \text{constant} \\
		t' &= \gamma t \quad \text{(time varies)}
	\end{align}
	
	\textbf{But}:
	\begin{equation}
		c = \frac{L}{T} \quad \text{and} \quad T \text{ varies} \quad \Rightarrow \quad c \text{ cannot be constant!}
	\end{equation}
	
	\subsection{Einstein's Hidden Solution}
	
	Einstein "solves" the contradiction through:
	\begin{itemize}
		\item Complicated Lorentz Transformationen
		\item Mathematical formalisms
		\item Space-Zeit constructions
		\item \textbf{But the logical contradiction remains!}
	\end{itemize}
	
	\subsection{T0's Natural Solution}
	
	\textbf{No contradiction in T0}:
	\begin{equation}
		\Tfield \cdot m = 1 \quad \Rightarrow \quad \text{time is naturally variable}
	\end{equation}
	
	\begin{equation}
		c = \frac{L}{T} \quad \Rightarrow \quad \text{c is naturally variable}
	\end{equation}
	
	\textbf{No Konstante-setting $\rightarrow$ No contradictions $\rightarrow$ No complicated repair mathematics}
	
	\section{The Mathematical Demonstration}
	
	\subsection{From E=mc² to E=m}
	
	\textbf{Starting Gleichung}: $E = mc^2$
	
	\textbf{c in natural Einheiten}: $c = 1$
	
	\textbf{Substitution}:
	\begin{equation}
		E = mc^2 = m \times 1^2 = m
	\end{equation}
	
	\textbf{Result}: $E = m$
	
	\subsection{The Reverse Direction: From E=m to E=mc²}
	
	\textbf{Starting Gleichung}: $E = m$
	
	\textbf{Artificial Konstante introduction}: $c = 299,792,458$ m/s
	
	\textbf{Inflating the Gleichung}:
	\begin{equation}
		E = m = m \times 1 = m \times \frac{c^2}{c^2} = m \times c^2 \times \frac{1}{c^2}
	\end{equation}
	
	\textbf{If one defines $c^2$ as "conversion Faktor"}:
	\begin{equation}
		E = mc^2
	\end{equation}
	
	\textbf{This shows}: $E = mc^2$ is nur $E = m$ with \textbf{artificial inflation Faktor} $c^2$!
	
	\section{The Arbitrariness of Constant Choice: c or Time?}
	
	\subsection{Einstein's Arbitrary Decision}
	
	\begin{tcolorbox}[colback=orange!5!white,colframe=orange!75!black,title=The Fundamental Choice Option]
		\textbf{One can choose was should be "Konstante"!}
		
		\textbf{Option 1 (Einstein's choice)}: c = Konstante $\rightarrow$ Zeit becomes Variable
		
		\textbf{Option 2 (alternative)}: Zeit = Konstante $\rightarrow$ c becomes Variable
		
		\textbf{Both describe the gleich physics!}
	\end{tcolorbox}
	
	\subsection{Option 1: Einstein's c-Konstante}
	
	\textbf{Einstein chose}:
	\begin{align}
		c &= 299,792,458 \text{ m/s} = \text{constant (defined)} \\
		t' &= \gamma t \quad \text{(time becomes automatically variable)}
	\end{align}
	
	\textbf{Language convention}:
	\begin{itemize}
		\item "Speed of Licht is universally Konstante"
		\item "Time dilates in strong gravitativ Felder"
		\item "Clocks run slower at high velocities"
	\end{itemize}
	
	\subsection{Option 2: Time-Konstante (Einstein could have chosen)}
	
	\textbf{Alternative choice}:
	\begin{align}
		t &= \text{constant (defined)} \\
		c(x,t) &= \frac{L(x,t)}{t} = \text{variable}
	\end{align}
	
	\textbf{Alternative language convention}:
	\begin{itemize}
		\item "Time flows equally everywhere"
		\item "Speed of Licht varies with location"
		\item "Light becomes slower in strong gravitativ Felder"
	\end{itemize}
	
	\subsection{Mathematical Equivalence of Both Options}
	
	\textbf{Both descriptions are mathematically identical}:
	
	\begin{table}[htbp]
		\centering
		\resizebox{\textwidth}{!}{%
MATHBLOCK99ENDMATH}
		\caption{Two views, identical physics}
	\end{table}
	
	\subsection{Why Einstein Chose Option 1}
	
	\textbf{Historical reasons for Einstein's decision}:
	\begin{itemize}
		\item \textbf{Michelson-Morley}: c seemed locally Konstante
		\item \textbf{Aesthetics}: "Universal Konstante" sounded elegant
		\item \textbf{Tradition}: Newtonian Konstante physics
		\item \textbf{Conceivability}: c-constancy easier to imagine than Zeit constancy
		\item \textbf{Authority Effekt}: Einstein's prestige fixed dies choice
	\end{itemize}
	
	\textbf{But it was nur a convention, not a natural law!}
	
	\subsection{T0's Overcoming of Both Options}
	
	\textbf{T0 shows}: Both choices are arbitrary!
	
	\begin{equation}
		\Tfield \cdot m = 1 \quad \text{(natural duality without constant constraint)}
	\end{equation}
	
	\textbf{T0 Einsicht}:
	\begin{itemize}
		\item \textbf{Neither} c nor Zeit are "really" Konstante
		\item \textbf{Both} are Aspekte of the gleich T·m Dynamik
		\item \textbf{Constancy} is nur definition convention
		\item \textbf{E = m} is the Konstante-free truth
	\end{itemize}
	
	\subsection{Liberation from Constant Constraint}
	
	\textbf{Instead of choosing zwischen}:
	\begin{itemize}
		\item c Konstante, Zeit Variable (Einstein)
		\item Time Konstante, c Variable (alternative)
	\end{itemize}
	
	\textbf{T0 chooses}:
	\begin{itemize}
		\item \textbf{Both dynamically coupled} via T·m = 1
		\item \textbf{No arbitrary fixations}
		\item \textbf{Natural Verhältnisse} stattdessen of artificial Konstanten
	\end{itemize}
	
	\section{The Reference Point Revolution: Earth $\rightarrow$ Sun $\rightarrow$ Nature}
	
	\subsection{The Reference Point Analogy: Geocentric $\rightarrow$ Heliocentric $\rightarrow$ T0}
	
	\begin{tcolorbox}[colback=blue!5!white,colframe=blue!75!black,title=The Reference Point Revolution: From Earth $\rightarrow$ Sun $\rightarrow$ Nature]
		\textbf{Geocentric (Ptolemy)}: Earth at center \\
		- Complicated epicycles needed \\
		- Works, but artificially complicated \\
		
		\textbf{Heliocentric (Copernicus)}: Sun at center \\
		- Simple ellipses \\
		- Much mehr elegant and einfach \\
		
		\textbf{T0-centric}: Natural Verhältnisse at center \\
		- $\Tfield \cdot m = 1$ (natural reference point) \\
		- Even mehr elegant: $E = m$
	\end{tcolorbox}
	
	\textbf{Einstein's c-Konstante corresponds to the geocentric System}:
	\begin{itemize}
		\item \textbf{Human} reference point at center (like Earth at center)
		\item \textbf{Complicated} mathematics needed (like epicycles)
		\item \textbf{Works} locally, but artificially inflated
	\end{itemize}
	
	\textbf{T0's natural Verhältnisse correspond to the heliocentric System}:
	\begin{itemize}
		\item \textbf{Natural} reference point at center (like Sun at center)
		\item \textbf{Simple} mathematics (like ellipses)
		\item \textbf{Universally} gültig and elegant
	\end{itemize}
	
	\subsection{Why We Need Reference Points}
	
	\textbf{Reference points are notwendig and natural}:
	\begin{itemize}
		\item \textbf{For Messungen}: We need standards for Vergleich
		\item \textbf{For communication}: Common basis for exchange
		\item \textbf{For technology}: Practical Anwendungen require Einheiten
		\item \textbf{For science}: Reproducible Experimente need standards
	\end{itemize}
	
	\textbf{The question is not WHETHER, but WHICH reference point}:
	
	\begin{table}[htbp]
		\centering
		\resizebox{\textwidth}{!}{%
MATHBLOCK100ENDMATH}
		\caption{Reference point systems comparison}
	\end{table}
	
	\subsection{The Right vs. Wrong Reference Point}
	
	\textbf{Einstein's error was not to choose a reference point}:
	\begin{itemize}
		\item \textbf{But to choose the wrong reference point!}
	\end{itemize}
	
	\textbf{Wrong reference point (Einstein)}: c = 299,792,458 m/s = Konstante
	\begin{itemize}
		\item Basierend auf human definition
		\item Leads to complicated mathematics
		\item Creates logical contradictions
	\end{itemize}
	
	\textbf{Right reference point (T0)}: $\Tfield \cdot m = 1$
	\begin{itemize}
		\item Basierend auf natural Verhältnis
		\item Leads to einfach mathematics: $E = m$
		\item No contradictions, pure elegance
	\end{itemize}
	
	\section{When Something Becomes "Constant"}
	
	\subsection{The Fundamental Reference Point Problem}
	
	\begin{tcolorbox}[colback=red!5!white,colframe=red!75!black,title=The Reference Point Illusion]
		\textbf{Something nur becomes "Konstante" wann we define a reference point!}
		
		\textbf{Without reference point}: All Verhältnisse are relative and dynamic
		
		\textbf{With reference point}: One Verhältnis becomes artificially "fixed"
		
		\textbf{Einstein's error}: He defined an absolute reference point for c
	\end{tcolorbox}
	
	\subsection{The Natural Stage: Everything is Relative}
	
	\textbf{Before irgendein reference point definition}:
	\begin{align}
		c_1 &= \frac{L_1}{T_1} \\
		c_2 &= \frac{L_2}{T_2} \\
		c_3 &= \frac{L_3}{T_3} \\
		&\vdots
	\end{align}
	
	\textbf{All c-Werte are relative to jeder andere}. None is "Konstante".
	
	\subsection{The Moment of Reference Point Setting}
	
	\textbf{Einstein's fatal step}:
	\begin{equation}
		\text{"I define: } c = 299,792,458 \text{ m/s = reference point"}
	\end{equation}
	
	\textbf{What happens at dies moment}:
	\begin{itemize}
		\item An \textbf{arbitrary reference point} is set
		\item All andere c-Werte are gemessen relative to dies
		\item The \textbf{dynamic Verhältnis} becomes a "Konstante"
		\item The \textbf{natural Relativität} is artificially "frozen"
	\end{itemize}
	
	\subsection{The Reference Point Problematic}
	
	\textbf{Every reference point is arbitrary}:
	\begin{itemize}
		\item Why 299,792,458 m/s and not 300,000,000 m/s?
		\item Why in m/s and not in andere Einheiten?
		\item Why gemessen on Earth and not in Raum?
		\item Why at dies Zeit and not at ein anderer?
	\end{itemize}
	
	\subsection{T0's Reference Point-Free Physics}
	
	\textbf{T0 eliminates alle reference points}:
	\begin{equation}
		\Tfield \cdot m = 1 \quad \text{(universal relation without reference point)}
	\end{equation}
	
	\begin{itemize}
		\item No arbitrary fixations
		\item All Verhältnisse remain dynamic
		\item Natural Relativität is preserved
		\item Fundamental simplicity: $E = m$
	\end{itemize}
	
	\subsection{Beispiel: The Meter Definition}
	
	\textbf{Historical development of meter definition}:
	\begin{enumerate}
		\item \textbf{1793}: 1 meter = 1/10,000,000 of Earth meridian (Earth reference point)
		\item \textbf{1889}: 1 meter = prototype meter in Paris (object reference point)  
		\item \textbf{1960}: 1 meter = 1,650,763.73 wavelengths of krypton-86 (Atom reference point)
		\item \textbf{1983}: 1 meter = Entfernung Licht travels in 1/299,792,458 s (c reference point)
	\end{enumerate}
	
	\textbf{What does dies show?}
	\begin{itemize}
		\item Each definition is \textbf{human arbitrariness}
		\item The \textbf{reference point} changes with human technology
		\item There is \textbf{no "natural" Länge Einheit} - nur human agreements
		\item \textbf{Humans make c "Konstante" by definition} - not nature!
	\end{itemize}
	
	\subsection{The Circular Error: Humans Define Their Own "Constants"}
	
	\textbf{In 1983 humans defined}:
	\begin{equation}
		1 \text{ meter} = \frac{1}{299,792,458} \times c \times 1 \text{ second}
	\end{equation}
	
	\textbf{This makes c automatically "Konstante"} - through human definition, not through natural law:
	\begin{equation}
		c = \frac{299,792,458 \text{ meters}}{1 \text{ second}} = 299,792,458 \text{ m/s}
	\end{equation}
	
	\textbf{Circular reasoning}: Humans define c as Konstante and dann "measure" a Konstante!
	
	\textbf{Nature is not asked in dies Prozess!}
	
	\subsection{T0's Resolution of the Reference Point Illusion}
	
	\textbf{T0 recognizes}:
	\begin{itemize}
		\item \textbf{Definition $\neq$ natural law}
		\item \textbf{Measurement reference point $\neq$ physikalisch Konstante}
		\item \textbf{Practical agreement $\neq$ fundamental truth}
	\end{itemize}
	
	\textbf{T0 Lösung}:
	\begin{align}
		\text{For measurements:} \quad &\text{Use practical reference points} \\
		\text{For natural laws:} \quad &\text{Use reference point-free relations}
	\end{align}
	
	\section{Why c-Constancy is Not Provable}
	
	\subsection{The Fundamental Measurement Problem}
	
	\textbf{To measure c, we need}:
	\begin{equation}
		c = \frac{L}{T}
	\end{equation}
	
	\textbf{But}: We measure L and T with \textbf{the gleich physikalisch Prozesse} das depend on c!
	
	\textbf{Circular problem}:
	\begin{itemize}
		\item Light measures distances $\rightarrow$ c determines L
		\item Atomic clocks use EM Übergänge $\rightarrow$ c influences T
		\item Then we measure c = L/T $\rightarrow$ \textbf{We measure c with c!}
	\end{itemize}
	
	\subsection{The Gauge Definition Problem}
	
	\textbf{Since 1983}: 1 meter = Entfernung Licht travels in 1/299,792,458 s
	
	\begin{equation}
		c = 299,792,458 \text{ m/s} \quad \text{(not measured, but defined!)}
	\end{equation}
	
	\textbf{One cannot "prove" was one has defined!}
	
	\subsection{The Systematic Compensation Problem}
	
	\textbf{If c varies, ALL measuring devices vary equally}:
	\begin{itemize}
		\item \textbf{Laser interferometers}: use Licht (c-dependent)
		\item \textbf{Atomic clocks}: use EM Übergänge (c-dependent)
		\item \textbf{Electronics}: uses EM signals (c-dependent)
	\end{itemize}
	
	\textbf{Result}: All devices \textbf{automatically compensate} the c-variation!
	
	\subsection{The Burden of Beweis Problem}
	
	\textbf{Scientifically korrekt}:
	\begin{itemize}
		\item One \textbf{cannot prove} das something is Konstante
		\item One can nur show das it \textbf{appears Konstante innerhalb Messung precision}
		\item \textbf{Each new precision Ebene} could show variation
	\end{itemize}
	
	\textbf{Einstein's "c-constancy" was belief, not Beweis!}
	
	\subsection{T0 Prediction for Precise Measurements}
	
	\textbf{T0 predicts}: At highest precision one will find:
	\begin{equation}
		c(x,t) = c_0 \left(1 + \xipar \times \frac{\Tfield(x,t) - \Tfield_0}{\Tfield_0}\right)
	\end{equation}
	
	with $\xipar = 1.33 \times 10^{-4}$ (T0 Parameter)
	
	\textbf{c varies tiny ($\sim$10$^{-15}$), but measurable in Prinzip!}
	
	\section{Ontological Consideration: Calculations as Constructs}
	
	\subsection{The Fundamental Epistemological Limit}
	
	\begin{tcolorbox}[colback=purple!5!white,colframe=purple!75!black,title=Ontological Truth]
		\textbf{All Berechnungen are human constructs!}
		
		They can \textbf{at best} give a certain idea of reality.
		
		\textbf{That Berechnungen are internally consistent proves little} ungefähr tatsächlich reality.
		
		\textbf{Mathematical consistency $\neq$ ontological truth}
	\end{tcolorbox}
	
	\subsection{Einstein's Construct vs. T0's Construct}
	
	\textbf{Both are human thought Strukturen}:
	
	\textbf{Einstein's construct}:
	\begin{itemize}
		\item E = mc² (mathematically consistent)
		\item Relativity theory (internally coherent)
		\item 10 Feld Gleichungen (Arbeit computationally)
		\item \textbf{But}: Basierend auf arbitrary c-Konstante setting
	\end{itemize}
	
	\textbf{T0's construct}:
	\begin{itemize}
		\item E = m (mathematically simpler)
		\item T·m = 1 (internally coherent)
		\item $\partial^2 E = 0$ (works computationally)
		\item \textbf{But}: Also nur a human thought Modell
	\end{itemize}
	
	\subsection{The Ontological Relativity}
	
	\textbf{What is "really" reell?}
	\begin{itemize}
		\item \textbf{Einstein's Raum-Zeit}? (construct)
		\item \textbf{T0's Energie Feld}? (construct)
		\item \textbf{Newton's absolute Zeit}? (construct)
		\item \textbf{Quantum Mechanik' probabilities}? (construct)
	\end{itemize}
	
	\textbf{All are human interpretive frameworks of the inaccessible reality!}
	
	\subsection{Why T0 is Still "Better"}
	
	\textbf{Not because of "absolute truth," but because of}:
	
	\textbf{1. Simplicity (Occam's Razor)}:
	\begin{itemize}
		\item E = m is simpler than E = mc²
		\item One Gleichung is simpler than 10 Gleichungen
		\item Fewer arbitrary Annahmen
	\end{itemize}
	
	\textbf{2. Consistency}:
	\begin{itemize}
		\item No logical contradictions (like Einstein's)
		\item No Konstante arbitrariness
		\item Unified thought Struktur
	\end{itemize}
	
	\textbf{3. Predictive Leistung}:
	\begin{itemize}
		\item Testable Vorhersagen
		\item Fewer free Parameter
		\item Clearer experimentell distinction
	\end{itemize}
	
	\textbf{4. Aesthetics}:
	\begin{itemize}
		\item Mathematical elegance
		\item Conceptual clarity
		\item Unity
	\end{itemize}
	
	\subsection{The Epistemological Humility}
	
	\textbf{T0 does NOT claim to be "absolute truth."}
	
	\textbf{T0 nur says}:
	\begin{itemize}
		\item "Here is a \textbf{simpler} construct"
		\item "With \textbf{fewer} arbitrary Annahmen"
		\item "That is \textbf{mehr consistent} than Einstein's construct"
		\item "And makes \textbf{mehr testable} Vorhersagen"
	\end{itemize}
	
	\textbf{But letztendlich T0 auch remains a human thought Struktur!}
	
	\subsection{The Pragmatic Consequence}
	
	\textbf{Since alle theories are constructs}:
	
	\textbf{Evaluation criteria are}:
	\begin{enumerate}
		\item \textbf{Simplicity} (fewer Annahmen)
		\item \textbf{Consistency} (no contradictions)
		\item \textbf{Predictive Leistung} (testable Konsequenzen)
		\item \textbf{Elegance} (aesthetic criteria)
		\item \textbf{Unity} (fewer separate domains)
	\end{enumerate}
	
	\textbf{By alle diese criteria T0 is "better" than Einstein - but not "absolutely wahr".}
	
	\subsection{The Ontological Humility}
	
	\textbf{The deepest Einsicht}:
	\begin{itemize}
		\item \textbf{Reality itself} is inaccessible
		\item \textbf{All theories} are human constructs
		\item \textbf{Mathematical consistency} proves no ontological truth
		\item \textbf{The best} we have: \textbf{Simpler, mehr consistent constructs}
	\end{itemize}
	
	\textbf{Einstein's error was not nur the c-Konstante setting, but auch the claim to absolute truth of his mathematisch constructs.}
	
	\textbf{T0's advantage is not absolute truth, but relative superiority as a thought Modell.}
	
	\section{The Practical Consequences}
	
	\subsection{Why E=mc² "Works"}
	
	\textbf{E=mc² works because}:
	\begin{itemize}
		\item It is mathematically identical to $E = m$
		\item $c^2$ compensates the "frozen" Zeit Dynamik
		\item The T0 truth is unconsciously contained
		\item Local Näherungen gewöhnlich suffice
	\end{itemize}
	
	\subsection{When E=mc² Fails}
	
	\textbf{The Konstanten illusion breaks down at}:
	\begin{itemize}
		\item Very präzise Messungen
		\item Extreme Bedingungen (high energies/masses)
		\item Cosmological Skalen
		\item Quantum Gravitation
	\end{itemize}
	
	\subsection{T0's Universal Validity}
	
	\textbf{E = m is gültig everywhere and immer}:
	\begin{itemize}
		\item No Näherungen needed
		\item No Konstante Annahmen
		\item Universal applicability
		\item Fundamental simplicity
	\end{itemize}
	
	\section{The Correction of Physics History}
	
	\subsection{Einstein's True Achievement}
	
	\textbf{Einstein's tatsächlich discovery was}:
	\begin{equation}
		E = m \quad \text{(in natural form)}
	\end{equation}
	
	\textbf{His error was}:
	\begin{equation}
		E = mc^2 \quad \text{(with artificial constant inflation)}
	\end{equation}
	
	\subsection{The Historical Irony}
	
	\begin{tcolorbox}[colback=blue!5!white,colframe=blue!75!black,title=The Great Irony]
		Einstein discovered the fundamental simplicity $E = m$, 
		
		but \textbf{hid it behind the Konstanten illusion} $E = mc^2$!
		
		The physics world celebrated the complicated form and overlooked the einfach truth.
	\end{tcolorbox}
	
	\section{The T0 Perspective: c as Living Ratio}
	
	\subsection{c as Expression of Time-Mass Duality}
	
	\textbf{In T0 theory}:
	\begin{equation}
		c(x,t) = f\left(\frac{L(x,t)}{\Tfield(x,t)}\right) = f\left(\frac{L(x,t) \cdot m(x,t)}{1}\right)
	\end{equation}
	
	since $\Tfield \cdot m = 1$.
	
	\textbf{c becomes an Ausdruck of the fundamental Zeit-Masse duality!}
	
	\subsection{The Dynamic Speed of Light}
	
	\textbf{T0 Vorhersage}: 
	\begin{equation}
		c(x,t) = c_0 \sqrt{1 + \xipar \frac{m(x,t) - m_0}{m_0}}
	\end{equation}
	
	\textbf{Light moves faster in mehr massive regions!}
	
	(Tiny Effekt, but measurable in Prinzip)
	
	\section{Experimentell Tests of c-Variability}
	
	\subsection{Proposed Experiments}
	
	\textbf{Test 1 - Gravitational dependence}:
	\begin{itemize}
		\item Measure c in unterschiedlich gravitativ Felder
		\item T0 Vorhersage: $c$ varies with $\sim \xipar \times \Delta\Phi_{\text{grav}}$
	\end{itemize}
	
	\textbf{Test 2 - Cosmological variation}:
	\begin{itemize}
		\item Measure c over kosmologisch Zeit periods
		\item T0 Vorhersage: $c$ changes with Universum Expansion
	\end{itemize}
	
	\textbf{Test 3 - High-Energie physics}:
	\begin{itemize}
		\item Measure c in Teilchen accelerators at highest energies
		\item T0 Vorhersage: Tiny Abweichungen at $E \sim$ TeV
	\end{itemize}
	
	\subsection{Expected Ergebnisse}
	
	\begin{table}[htbp]
		\centering
		\resizebox{\textwidth}{!}{%
MATHBLOCK101ENDMATH}
		\caption{Predicted c-variations}
	\end{table}
	
	\section{Schlussfolgerungen}
	
	\subsection{The Central Recognition}
	
	\begin{tcolorbox}[colback=green!5!white,colframe=green!75!black,title=The Fundamental Truth]
		\textbf{E=mc² = E=m}
		
		Einstein's "Konstante" c is in truth a Variable Verhältnis.
		
		The Konstante-setting was Einstein's fundamental error.
		
		T0 corrects dies error by returning to natural variability.
	\end{tcolorbox}
	
	\subsection{Physics After the Constants Illusion}
	
	\textbf{The future of physics}:
	\begin{itemize}
		\item No artificial Konstanten
		\item Dynamic Verhältnisse everywhere
		\item Living, Variable natural laws
		\item Fundamental simplicity: $E = m$
	\end{itemize}
	
	\subsection{Einstein's Corrected Legacy}
	
	\textbf{Einstein's wahr discovery}: $E = m$ (Energie-Masse identity)
	
	\textbf{Einstein's error}: Constant-setting of c
	
	\textbf{T0's Korrektur}: Return to natural form $E = m$
	
	\textbf{Einstein was brilliant - he nur stopped one step auch early!}

\begin{thebibliography}{99}

% ============================================
% Core T0 Theory References (J. Pascher)
% GitHub Repository: https://github.com/jpascher/T0-Time-Mass-Duality
% ============================================

\bibitem{pascher2024}
J. Pascher, \emph{T0 Theory: Time-Mass Duality}, 2024.
\url{https://github.com/jpascher/T0-Time-Mass-Duality/blob/main/2/pdf/T0_unified_report.pdf}

\bibitem{pascher2025t0}
J. Pascher, \emph{T0 Theory: Fundamentals}, 2025.
\url{https://github.com/jpascher/T0-Time-Mass-Duality/blob/main/2/pdf/T0_Grundlagen_En.pdf}

\bibitem{pascher2025qm}
J. Pascher, \emph{T0 Theory: Quantum Mechanics}, 2025.
\url{https://github.com/jpascher/T0-Time-Mass-Duality/blob/main/2/pdf/QM_En.pdf}

\bibitem{pascher2025si}
J. Pascher, \emph{T0 Theory: SI Units}, 2025.
\url{https://github.com/jpascher/T0-Time-Mass-Duality/blob/main/2/pdf/T0_SI_En.pdf}

\bibitem{pascher2025g2}
J. Pascher, \emph{T0 Theory: The g-2 Anomaly}, 2025.
\url{https://github.com/jpascher/T0-Time-Mass-Duality/blob/main/2/pdf/T0_Anomale-g2-9_En.pdf}

\bibitem{pascher2025cmb}
J. Pascher, \emph{T0 Theory: CMB Analysis}, 2025.
\url{https://github.com/jpascher/T0-Time-Mass-Duality/blob/main/2/pdf/Zwei-Dipole-CMB_En.pdf}

% Historical Physics
\bibitem{einstein1905}
A. Einstein, \emph{On the Electrodynamics of Moving Bodies}, Annalen der Physik, 1905.
\url{https://doi.org/10.1002/andp.19053221004}

\bibitem{dirac1928}
P.A.M. Dirac, \emph{The Quantum Theory of the Electron}, Proc. Roy. Soc. A, 1928.
\url{https://doi.org/10.1098/rspa.1928.0023}

\bibitem{planck1900}
M. Planck, \emph{On the Theory of the Energy Distribution Law}, 1900.
\url{https://doi.org/10.1002/andp.19013090310}

\bibitem{mach1883}
E. Mach, \emph{Die Mechanik in ihrer Entwicklung}, 1883.

\bibitem{hundert1931}
Various Authors, \emph{100 Authors Against Einstein}, 1931.

\bibitem{dingle1972}
H. Dingle, \emph{Science at the Crossroads}, 1972.

% Penrose and Terrell Effect
\bibitem{terrell1959}
J. Terrell, \emph{Invisibility of the Lorentz Contraction}, Phys. Rev., 1959.
\url{https://doi.org/10.1103/PhysRev.116.1041}

\bibitem{penrose1959}
R. Penrose, \emph{The Apparent Shape of a Relativistically Moving Sphere}, Proc. Cambridge Phil. Soc., 1959.
\url{https://doi.org/10.1017/S0305004100033776}

\bibitem{penrose1967}
R. Penrose, \emph{Twistor Algebra}, J. Math. Phys., 1967.
\url{https://doi.org/10.1063/1.1705200}

\bibitem{penrose2004}
R. Penrose, \emph{The Road to Reality}, 2004.

\bibitem{terrell2025}
J. Terrell et al., \emph{Modern Terrell-Penrose Visualization}, 2025.

\bibitem{weiskopf2000}
D. Weiskopf, \emph{Visualization of Four-dimensional Spacetimes}, 2000.

\bibitem{mueller2014}
T. Müller, \emph{Visual Appearance of Relativistically Moving Objects}, 2014.

\bibitem{hossenfelder2025}
S. Hossenfelder, \emph{YouTube: The Terrell Effect}, 2025.

% Quantum Gravity and String Theory
\bibitem{rovelli2004}
C. Rovelli, \emph{Quantum Gravity}, Cambridge University Press, 2004.

\bibitem{thiemann2007}
T. Thiemann, \emph{Modern Canonical Quantum Gravity}, Cambridge University Press, 2007.

\bibitem{ashtekar2004}
A. Ashtekar, J. Lewandowski, \emph{Background Independent Quantum Gravity}, Class. Quant. Grav., 2004.
\url{https://doi.org/10.1088/0264-9381/21/15/R01}

\bibitem{jacobson1995}
T. Jacobson, \emph{Thermodynamics of Spacetime}, Phys. Rev. Lett., 1995.
\url{https://doi.org/10.1103/PhysRevLett.75.1260}

\bibitem{maldacena1998}
J. Maldacena, \emph{The Large N Limit of Superconformal Field Theories}, Adv. Theor. Math. Phys., 1998.
\url{https://doi.org/10.4310/ATMP.1998.v2.n2.a1}

\bibitem{polchinski1998}
J. Polchinski, \emph{String Theory}, Cambridge University Press, 1998.

\bibitem{susskind1995}
L. Susskind, \emph{The World as a Hologram}, J. Math. Phys., 1995.
\url{https://doi.org/10.1063/1.531249}

\bibitem{verlinde2011}
E. Verlinde, \emph{On the Origin of Gravity}, JHEP, 2011.
\url{https://doi.org/10.1007/JHEP04(2011)029}

% Cosmology
\bibitem{hoyle1948}
F. Hoyle, \emph{A New Model for the Expanding Universe}, MNRAS, 1948.
\url{https://doi.org/10.1093/mnras/108.5.372}

\bibitem{bondi1948}
H. Bondi, T. Gold, \emph{The Steady-State Theory}, MNRAS, 1948.
\url{https://doi.org/10.1093/mnras/108.3.252}

\bibitem{zwicky1929}
F. Zwicky, \emph{On the Redshift of Spectral Lines}, Proc. Nat. Acad. Sci., 1929.
\url{https://doi.org/10.1073/pnas.15.10.773}

\bibitem{lopez2010}
C. Lopez-Corredoira, \emph{Tests of Cosmological Models}, Int. J. Mod. Phys. D, 2010.

\bibitem{lerner2014}
E. Lerner, \emph{Evidence for a Non-Expanding Universe}, 2014.

\bibitem{albrecht1999}
A. Albrecht, J. Magueijo, \emph{Variable Speed of Light}, Phys. Rev. D, 1999.
\url{https://doi.org/10.1103/PhysRevD.59.043516}

\bibitem{barrow1999}
J. Barrow, \emph{Cosmologies with Varying Light Speed}, Phys. Rev. D, 1999.
\url{https://doi.org/10.1103/PhysRevD.59.043515}

\bibitem{riess2022}
A. Riess et al., \emph{A Comprehensive Measurement of the Local Value of the Hubble Constant}, ApJ, 2022.
\url{https://doi.org/10.3847/2041-8213/ac5c5b}

\bibitem{desi2025}
DESI Collaboration, \emph{DESI Year 1 Results}, 2025.
\url{https://arxiv.org/abs/2404.03002}

\bibitem{divalentino2021}
E. Di Valentino et al., \emph{Planck Evidence for a Closed Universe}, Nat. Astron., 2021.
\url{https://doi.org/10.1038/s41550-019-0906-9}

% Conformal Field Theory
\bibitem{francesco1997}
P. Di Francesco et al., \emph{Conformal Field Theory}, Springer, 1997.

% Experimental Physics
\bibitem{pdg2024}
Particle Data Group, \emph{Review of Particle Physics}, 2024.
\url{https://pdg.lbl.gov/}

\bibitem{codata2019}
CODATA, \emph{Recommended Values of Fundamental Constants}, 2019.
\url{https://physics.nist.gov/cuu/Constants/}

\bibitem{newell2018}
D. Newell et al., \emph{The CODATA 2017 Values of h, e, k, and $N_A$}, Metrologia, 2018.
\url{https://doi.org/10.1088/1681-7575/aa950a}

\bibitem{muong2_2023}
Muon g-2 Collaboration, \emph{Measurement of the Anomalous Magnetic Moment of the Muon}, Phys. Rev. Lett., 2023.
\url{https://doi.org/10.1103/PhysRevLett.131.161802}

\bibitem{fermilab2023}
Fermilab, \emph{Muon g-2 Results}, 2023.
\url{https://muon-g-2.fnal.gov/}

\bibitem{atlas2023}
ATLAS Collaboration, \emph{Measurements at the LHC}, 2023.
\url{https://atlas.cern/}

\bibitem{atlas2023higgs}
ATLAS Collaboration, \emph{Higgs Boson Properties}, 2023.
\url{https://atlas.cern/}

\bibitem{cms2023top}
CMS Collaboration, \emph{Top Quark Measurements}, 2023.
\url{https://cms.cern/}

\bibitem{cms2024}
CMS Collaboration, \emph{Heavy Ion Collisions}, 2024.
\url{https://cms.cern/}

\bibitem{alice2023}
ALICE Collaboration, \emph{Quark-Gluon Plasma Studies}, 2023.
\url{https://alice-collaboration.web.cern.ch/}

\bibitem{kasevich2023}
M. Kasevich et al., \emph{Atom Interferometry}, 2023.

\bibitem{ludlow2015}
A. Ludlow et al., \emph{Optical Atomic Clocks}, Rev. Mod. Phys., 2015.
\url{https://doi.org/10.1103/RevModPhys.87.637}

\bibitem{brewer2019}
S. Brewer et al., \emph{Al$^+$ Optical Clock}, Phys. Rev. Lett., 2019.
\url{https://doi.org/10.1103/PhysRevLett.123.033201}

\bibitem{lisa2017}
LISA Collaboration, \emph{LISA Mission}, 2017.
\url{https://www.lisamission.org/}

% Fractal Physics
\bibitem{nottale1993}
L. Nottale, \emph{Fractal Space-Time and Microphysics}, World Scientific, 1993.

\bibitem{elnaschie2004}
M.S. El Naschie, \emph{E-Infinity Theory}, Chaos Solitons Fractals, 2004.

% Philosophy and Foundations
\bibitem{wheeler1990}
J.A. Wheeler, \emph{Information, Physics, Quantum}, 1990.

\bibitem{barbour1999}
J. Barbour, \emph{The End of Time}, Oxford University Press, 1999.

\bibitem{sciama1953}
D. Sciama, \emph{On the Origin of Inertia}, MNRAS, 1953.
\url{https://doi.org/10.1093/mnras/113.1.34}

% String Theory Extensions
\bibitem{becker2007}
K. Becker et al., \emph{String Theory and M-Theory}, Cambridge University Press, 2007.

% Missing References for g-2 Chapter
\bibitem{sm_g2_2025}
Muon g-2 Theory Initiative, \emph{Standard Model Prediction for g-2}, arXiv, 2025.
\url{https://arxiv.org/abs/2006.04822}

\bibitem{mug2_final_2025}
Muon g-2 Collaboration, \emph{Final Report on the Anomalous Magnetic Moment of the Muon}, Fermilab, 2025.
\url{https://muon-g-2.fnal.gov/}

\bibitem{pascher_t0_theory_2025}
J. Pascher, \emph{T0 Theory: Complete Framework}, 2025.
\url{https://github.com/jpascher/T0-Time-Mass-Duality/blob/main/2/pdf/systemEn.pdf}

\bibitem{peskin_schroeder_1995}
M.E. Peskin and D.V. Schroeder, \emph{An Introduction to Quantum Field Theory}, Westview Press, 1995.

\bibitem{parker_somov_2018}
R.H. Parker et al., \emph{Measurement of the Fine-Structure Constant}, Science, 2018.
\url{https://doi.org/10.1126/science.aap7706}

\bibitem{morel_rubidium_2020}
L. Morel et al., \emph{Determination of $\alpha$ from Rubidium Atom Recoil}, Nature, 2020.
\url{https://doi.org/10.1038/s41586-020-2964-7}

\bibitem{aoyama_theory_2020}
T. Aoyama et al., \emph{Theory of the Electron Anomalous Magnetic Moment}, Phys. Rep., 2020.
\url{https://doi.org/10.1016/j.physrep.2020.07.006}

\bibitem{fan_lattice_2023}
X. Fan et al., \emph{Hadronic Contributions from Lattice QCD}, Phys. Rev. D, 2023.

\bibitem{hanneke_electron_2008}
D. Hanneke et al., \emph{New Measurement of the Electron g-2}, Phys. Rev. Lett., 2008.
\url{https://doi.org/10.1103/PhysRevLett.100.120801}

% Additional T0 Theory References
\bibitem{pascher_higgs_connection_2025}
J. Pascher, \emph{Higgs Connection in T0 Theory}, 2025.
\url{https://github.com/jpascher/T0-Time-Mass-Duality/blob/main/2/pdf/T0_Energie_En.pdf}

\bibitem{T0_SI}
J. Pascher, \emph{T0 Theory and SI Units}, 2025.
\url{https://github.com/jpascher/T0-Time-Mass-Duality/blob/main/2/pdf/T0_SI_En.pdf}

\bibitem{T0_gravitational_constant}
J. Pascher, \emph{Gravitational Constant in T0 Framework}, 2025.
\url{https://github.com/jpascher/T0-Time-Mass-Duality/blob/main/2/pdf/T0_Gravitationskonstante_En.pdf}

\bibitem{T0_fine_structure}
J. Pascher, \emph{Fine Structure Constant Analysis}, 2025.
\url{https://github.com/jpascher/T0-Time-Mass-Duality/blob/main/2/pdf/T0_Feinstruktur_En.pdf}

\bibitem{bell_muon}
J.S. Bell, \emph{Muon Studies}, 1966.

\bibitem{QFT_T0}
J. Pascher, \emph{Quantum Field Theory in T0}, 2025.
\url{https://github.com/jpascher/T0-Time-Mass-Duality/blob/main/2/pdf/QFT_En.pdf}

\bibitem{planck2018}
Planck Collaboration, \emph{Planck 2018 Results}, A\&A, 2018.
\url{https://doi.org/10.1051/0004-6361/201833910}

\bibitem{pascher:t0_foundations}
J. Pascher, \emph{T0 Theory Foundations}, 2025.
\url{https://github.com/jpascher/T0-Time-Mass-Duality/blob/main/2/pdf/T0_Grundlagen_En.pdf}

\bibitem{pascher:geometric_formalism}
J. Pascher, \emph{Geometric Formalism in T0}, 2025.
\url{https://github.com/jpascher/T0-Time-Mass-Duality/blob/main/2/pdf/T0_Geometrische_Kosmologie_En.pdf}

\bibitem{riess2019}
A. Riess et al., \emph{Hubble Constant Measurements}, ApJ, 2019.
\url{https://doi.org/10.3847/1538-4357/ab1422}

\bibitem{t0_kosmologie}
J. Pascher, \emph{T0 Kosmologie}, 2025.
\url{https://github.com/jpascher/T0-Time-Mass-Duality/blob/main/2/pdf/T0_Kosmologie_En.pdf}

\bibitem{hossenfelder_single_clock_video}
S. Hossenfelder, \emph{Single Clock Video}, YouTube, 2025.
\url{https://www.youtube.com/c/SabineHossenfelder}

\bibitem{video2025}
Various, \emph{Video References}, 2025.

\bibitem{unnikrishnan2004}
C.S. Unnikrishnan, \emph{Gravity Studies}, 2004.

\bibitem{peratt1992}
A. Peratt, \emph{Plasma Cosmology}, 1992.
\url{https://github.com/jpascher/T0-Time-Mass-Duality/blob/main/2/pdf/T0_peratt_En.pdf}

\bibitem{T0_tm_erweiterung}
J. Pascher, \emph{T0 Time-Mass Extension}, 2025.
\url{https://github.com/jpascher/T0-Time-Mass-Duality/blob/main/2/pdf/T0_tm-erweiterung-x6_En.pdf}

\bibitem{T0_g2_erweiterung}
J. Pascher, \emph{T0 g-2 Extension}, 2025.
\url{https://github.com/jpascher/T0-Time-Mass-Duality/blob/main/2/pdf/T0_g2-erweiterung-4_En.pdf}

\bibitem{T0_netze_en}
J. Pascher, \emph{T0 Networks}, 2025.
\url{https://github.com/jpascher/T0-Time-Mass-Duality/blob/main/2/pdf/T0_netze_En.pdf}

\bibitem{Adams1925}
W. Adams, \emph{Gravitational Redshift}, 1925.
\url{https://doi.org/10.1073/pnas.11.7.382}

\bibitem{Ashby2003}
N. Ashby, \emph{Relativity in GPS}, Living Rev. Rel., 2003.
\url{https://doi.org/10.12942/lrr-2003-1}

\bibitem{Bertotti2003}
B. Bertotti et al., \emph{Cassini Doppler Test}, Nature, 2003.
\url{https://doi.org/10.1038/nature01997}

\bibitem{Bolton2008}
A. Bolton et al., \emph{Gravitational Lensing}, 2008.

\bibitem{Born2013}
M. Born, \emph{Einstein's Theory of Relativity}, Dover, 2013.

\bibitem{Brans1961}
C. Brans and R.H. Dicke, \emph{Mach's Principle}, Phys. Rev., 1961.
\url{https://doi.org/10.1103/PhysRev.124.925}

\bibitem{Dirac1927}
P.A.M. Dirac, \emph{Quantum Mechanics}, Proc. Roy. Soc., 1927.
\url{https://doi.org/10.1098/rspa.1927.0039}

\bibitem{Duhem1906}
P. Duhem, \emph{Theory of Physics}, 1906.

\bibitem{Einstein1905}
A. Einstein, \emph{Special Relativity}, Ann. Phys., 1905.
\url{https://doi.org/10.1002/andp.19053221004}

\bibitem{Feynman2006}
R. Feynman, \emph{QED: The Strange Theory of Light and Matter}, 2006.

\bibitem{Griffiths2017}
D. Griffiths, \emph{Introduction to Quantum Mechanics}, 2017.

\bibitem{Jackson1999}
J.D. Jackson, \emph{Classical Electrodynamics}, 1999.

\bibitem{Kaluza1921}
T. Kaluza, \emph{Five-Dimensional Theory}, 1921.

\bibitem{Klein1926}
O. Klein, \emph{Quantum Theory and Relativity}, 1926.

\bibitem{Kuhn1962}
T. Kuhn, \emph{Structure of Scientific Revolutions}, 1962.

\bibitem{Kuhn1977}
T. Kuhn, \emph{Essential Tension}, 1977.

\bibitem{Ludlow2015}
A. Ludlow et al., \emph{Optical Atomic Clocks}, Rev. Mod. Phys., 2015.
\url{https://doi.org/10.1103/RevModPhys.87.637}

\bibitem{Maxwell1873}
J.C. Maxwell, \emph{Treatise on Electricity and Magnetism}, 1873.

\bibitem{McGaugh2016}
S. McGaugh et al., \emph{Radial Acceleration Relation}, Phys. Rev. Lett., 2016.
\url{https://doi.org/10.1103/PhysRevLett.117.201101}

\bibitem{Mohr2016}
P. Mohr et al., \emph{CODATA Values}, Rev. Mod. Phys., 2016.
\url{https://doi.org/10.1103/RevModPhys.88.035009}

\bibitem{PDG2020}
Particle Data Group, \emph{Review of Particle Physics}, Prog. Theor. Exp. Phys., 2020.
\url{https://pdg.lbl.gov/}

\bibitem{Parker2018}
R. Parker et al., \emph{Measurement of $\alpha$}, Science, 2018.
\url{https://doi.org/10.1126/science.aap7706}

\bibitem{Peskin1995}
M. Peskin and D. Schroeder, \emph{QFT}, 1995.

\bibitem{Planck1900}
M. Planck, \emph{Quantum Theory}, 1900.

\bibitem{Planck2020}
Planck Collaboration, \emph{Planck 2020 Results}, 2020.
\url{https://doi.org/10.1051/0004-6361/201833910}

\bibitem{Poincare1905}
H. Poincaré, \emph{Dynamics of the Electron}, 1905.

\bibitem{Pound1960}
R.V. Pound and G.A. Rebka, \emph{Gravitational Redshift}, Phys. Rev. Lett., 1960.
\url{https://doi.org/10.1103/PhysRevLett.4.337}

\bibitem{Quine1951}
W.V. Quine, \emph{Two Dogmas of Empiricism}, 1951.

\bibitem{Quinn2013}
T. Quinn et al., \emph{Gravitational Constant}, 2013.
\url{https://doi.org/10.1103/PhysRevLett.111.101102}

\bibitem{Randall1999}
L. Randall and R. Sundrum, \emph{Extra Dimensions}, Phys. Rev. Lett., 1999.
\url{https://doi.org/10.1103/PhysRevLett.83.3370}

\bibitem{Riess1998}
A. Riess et al., \emph{Type Ia Supernovae}, AJ, 1998.
\url{https://doi.org/10.1086/300499}

\bibitem{Shapiro1971}
I. Shapiro et al., \emph{Time Delay Test}, Phys. Rev. Lett., 1971.
\url{https://doi.org/10.1103/PhysRevLett.26.1132}

\bibitem{Sommerfeld1916}
A. Sommerfeld, \emph{Fine Structure}, 1916.

\bibitem{Suyu2017}
S. Suyu et al., \emph{Time Delay Cosmography}, MNRAS, 2017.
\url{https://doi.org/10.1093/mnras/stx483}

\bibitem{T0Theory}
J. Pascher, \emph{T0 Theory}, 2025.
\url{https://github.com/jpascher/T0-Time-Mass-Duality/blob/main/2/pdf/systemEn.pdf}

\bibitem{T0_Feinstruktur}
J. Pascher, \emph{Fine Structure in T0}, 2025.
\url{https://github.com/jpascher/T0-Time-Mass-Duality/blob/main/2/pdf/T0_Feinstruktur_En.pdf}

\bibitem{Uzan2003}
J.-P. Uzan, \emph{Constants Variation}, Rev. Mod. Phys., 2003.
\url{https://doi.org/10.1103/RevModPhys.75.403}

\bibitem{Webb2001}
J.K. Webb et al., \emph{Fine Structure Constant}, Phys. Rev. Lett., 2001.
\url{https://doi.org/10.1103/PhysRevLett.87.091301}

\bibitem{Weinberg1979}
S. Weinberg, \emph{Cosmological Constant}, Rev. Mod. Phys., 1979.

\bibitem{Weinberg1989}
S. Weinberg, \emph{Cosmological Constant Problem}, 1989.
\url{https://doi.org/10.1103/RevModPhys.61.1}

\bibitem{Weinberg1995}
S. Weinberg, \emph{Quantum Theory of Fields}, 1995.

\bibitem{Will2014}
C. Will, \emph{Theory and Experiment in Gravitational Physics}, 2014.
\url{https://doi.org/10.12942/lrr-2014-4}

\bibitem{dirac_principles}
P.A.M. Dirac, \emph{Principles of Quantum Mechanics}, 1930.

\bibitem{einstein_1917}
A. Einstein, \emph{Cosmological Considerations}, 1917.

\bibitem{jwst_early}
JWST Collaboration, \emph{Early Universe Observations}, 2023.
\url{https://www.jwst.nasa.gov/}

\bibitem{katrin_2022}
KATRIN Collaboration, \emph{Neutrino Mass}, 2022.
\url{https://doi.org/10.1038/s41567-021-01463-1}

\bibitem{pascher:fundamentals}
J. Pascher, \emph{T0 Fundamentals}, 2025.
\url{https://github.com/jpascher/T0-Time-Mass-Duality/blob/main/2/pdf/T0_Grundlagen_En.pdf}

\bibitem{pascher:g2_rev9}
J. Pascher, \emph{g-2 Analysis Rev9}, 2025.
\url{https://github.com/jpascher/T0-Time-Mass-Duality/blob/main/2/pdf/T0_Anomale-g2-9_En.pdf}

\bibitem{pascher:ml_addendum}
J. Pascher, \emph{ML Addendum}, 2025.
\url{https://github.com/jpascher/T0-Time-Mass-Duality/blob/main/2/pdf/T0-QFT-ML_Addendum_En.pdf}

\bibitem{pascher_beta_derivation_2025}
J. Pascher, \emph{Beta Derivation}, 2025.
\url{https://github.com/jpascher/T0-Time-Mass-Duality/blob/main/2/pdf/DerivationVonBetaEn.pdf}

\bibitem{pascher_cmb_en}
J. Pascher, \emph{CMB Analysis in T0}, 2025.
\url{https://github.com/jpascher/T0-Time-Mass-Duality/blob/main/2/pdf/Zwei-Dipole-CMB_En.pdf}

\bibitem{pascher_cosmos_en}
J. Pascher, \emph{Cosmos in T0 Theory}, 2025.
\url{https://github.com/jpascher/T0-Time-Mass-Duality/blob/main/2/pdf/cosmic_En.pdf}

\bibitem{pascher_derivation_beta_2025}
J. Pascher, \emph{Derivation of Beta}, 2025.
\url{https://github.com/jpascher/T0-Time-Mass-Duality/blob/main/2/pdf/DerivationVonBetaEn.pdf}

\bibitem{pascher_gravitation_en}
J. Pascher, \emph{Gravitation in T0}, 2025.
\url{https://github.com/jpascher/T0-Time-Mass-Duality/blob/main/2/pdf/gravitationskonstante_En.pdf}

\bibitem{pascher_lagrangian_2025}
J. Pascher, \emph{Lagrangian in T0}, 2025.
\url{https://github.com/jpascher/T0-Time-Mass-Duality/blob/main/2/pdf/T0_lagrndian_En.pdf}

\bibitem{pascher_lagrangian_en}
J. Pascher, \emph{Lagrangian Framework}, 2025.
\url{https://github.com/jpascher/T0-Time-Mass-Duality/blob/main/2/pdf/LagrandianVergleichEn.pdf}

\bibitem{pascher_lagrangian_extended_2025}
J. Pascher, \emph{Extended Lagrangian Formalism}, 2025.
\url{https://github.com/jpascher/T0-Time-Mass-Duality/blob/main/2/pdf/T0_lagrndian_En.pdf}

\bibitem{pascher_mathematical_structure_2025}
J. Pascher, \emph{Mathematical Structure of T0 Theory}, 2025.
\url{https://github.com/jpascher/T0-Time-Mass-Duality/blob/main/2/pdf/Mathematische_struktur_En.pdf}

\bibitem{pascher_muon_g2_2025}
J. Pascher, \emph{Muon g-2 in T0}, 2025.
\url{https://github.com/jpascher/T0-Time-Mass-Duality/blob/main/2/pdf/T0_Anomale-g2-9_En.pdf}

\bibitem{pascher_pragmatic_2025}
J. Pascher, \emph{Pragmatic Approach}, 2025.

\bibitem{pascher_t0_energy_2025}
J. Pascher, \emph{T0 Energy Formalism}, 2025.
\url{https://github.com/jpascher/T0-Time-Mass-Duality/blob/main/2/pdf/T0-Energie_En.pdf}

\bibitem{pascher_unified_2025}
J. Pascher, \emph{Unified T0 Theory}, 2025.
\url{https://github.com/jpascher/T0-Time-Mass-Duality/blob/main/2/pdf/T0_unified_report.pdf}

\bibitem{sciencedaily2025}
Science Daily, \emph{Physics News}, 2025.
\url{https://www.sciencedaily.com/}

\bibitem{weinberg_1989}
S. Weinberg, \emph{The Cosmological Constant Problem}, Rev. Mod. Phys., 1989.
\url{https://doi.org/10.1103/RevModPhys.61.1}

\bibitem{wiki_bell}
Wikipedia, \emph{Bell's Theorem}, 2025.
\url{https://en.wikipedia.org/wiki/Bell\%27s_theorem}

\bibitem{vanFraassen1980}
B. van Fraassen, \emph{The Scientific Image}, Oxford University Press, 1980.

\bibitem{terrell_single_clock_nature_2024}
J. Terrell, \emph{Single Clock Nature}, Nature, 2024.

% Additional T0 Documents
\bibitem{137_doc}
J. Pascher, \emph{The Number 137 in T0 Theory}, 2025.
\url{https://github.com/jpascher/T0-Time-Mass-Duality/blob/main/2/pdf/137_En.pdf}

\bibitem{ampere_low}
J. Pascher, \emph{Ampere's Law in T0}, 2025.
\url{https://github.com/jpascher/T0-Time-Mass-Duality/blob/main/2/pdf/Amper_Low_En.pdf}

\bibitem{bell_theorem}
J. Pascher, \emph{Bell's Theorem in T0}, 2025.
\url{https://github.com/jpascher/T0-Time-Mass-Duality/blob/main/2/pdf/Bell_En.pdf}

\bibitem{bewegungsenergie}
J. Pascher, \emph{Kinetic Energy in T0}, 2025.
\url{https://github.com/jpascher/T0-Time-Mass-Duality/blob/main/2/pdf/Bewegungsenergie_En.pdf}

\bibitem{emc2}
J. Pascher, \emph{E=mc² in T0 Framework}, 2025.
\url{https://github.com/jpascher/T0-Time-Mass-Duality/blob/main/2/pdf/E-mc2_En.pdf}

\bibitem{formeln_energiebasiert}
J. Pascher, \emph{Energy-Based Formulas}, 2025.
\url{https://github.com/jpascher/T0-Time-Mass-Duality/blob/main/2/pdf/Formeln_Energiebasiert_En.pdf}

\bibitem{hannah}
J. Pascher, \emph{Hannah Document}, 2025.
\url{https://github.com/jpascher/T0-Time-Mass-Duality/blob/main/2/pdf/Hannah_En.pdf}

\bibitem{ho_doc}
J. Pascher, \emph{H0 Analysis}, 2025.
\url{https://github.com/jpascher/T0-Time-Mass-Duality/blob/main/2/pdf/Ho_En.pdf}

\bibitem{markov}
J. Pascher, \emph{Markov Processes in T0}, 2025.
\url{https://github.com/jpascher/T0-Time-Mass-Duality/blob/main/2/pdf/Markov_En.pdf}

\bibitem{elimination_mass}
J. Pascher, \emph{Elimination of Mass}, 2025.
\url{https://github.com/jpascher/T0-Time-Mass-Duality/blob/main/2/pdf/EliminationOfMassEn.pdf}

\bibitem{elimination_mass_dirac}
J. Pascher, \emph{Dirac Equation Mass Elimination}, 2025.
\url{https://github.com/jpascher/T0-Time-Mass-Duality/blob/main/2/pdf/Elimination_Of_Mass_Dirac_TabelleEn.pdf}

\bibitem{feinstrukturkonstante}
J. Pascher, \emph{Fine Structure Constant}, 2025.
\url{https://github.com/jpascher/T0-Time-Mass-Duality/blob/main/2/pdf/FeinstrukturkonstanteEn.pdf}

\bibitem{neutrino_formel}
J. Pascher, \emph{Neutrino Formula}, 2025.
\url{https://github.com/jpascher/T0-Time-Mass-Duality/blob/main/2/pdf/neutrino-Formel_En.pdf}

\bibitem{neutrinos}
J. Pascher, \emph{Neutrinos in T0}, 2025.
\url{https://github.com/jpascher/T0-Time-Mass-Duality/blob/main/2/pdf/T0_Neutrinos_En.pdf}

\bibitem{koide_formel}
J. Pascher, \emph{Koide Formula in T0}, 2025.
\url{https://github.com/jpascher/T0-Time-Mass-Duality/blob/main/2/pdf/T0_koide-formel-3_En.pdf}

\bibitem{teilchenmassen}
J. Pascher, \emph{Particle Masses}, 2025.
\url{https://github.com/jpascher/T0-Time-Mass-Duality/blob/main/2/pdf/Teilchenmassen_En.pdf}

\bibitem{t0_teilchenmassen}
J. Pascher, \emph{T0 Particle Masses}, 2025.
\url{https://github.com/jpascher/T0-Time-Mass-Duality/blob/main/2/pdf/T0_Teilchenmassen_En.pdf}

\bibitem{penrose_doc}
J. Pascher, \emph{Penrose Analysis in T0}, 2025.
\url{https://github.com/jpascher/T0-Time-Mass-Duality/blob/main/2/pdf/T0_penrose_En.pdf}

\bibitem{photonenchip}
J. Pascher, \emph{Photon Chip Implementation}, 2025.
\url{https://github.com/jpascher/T0-Time-Mass-Duality/blob/main/2/pdf/T0_photonenchip-china_En.pdf}

\bibitem{threeclock}
J. Pascher, \emph{Three Clock Experiment}, 2025.
\url{https://github.com/jpascher/T0-Time-Mass-Duality/blob/main/2/pdf/T0_threeclock_En.pdf}

\bibitem{redshift_deflection}
J. Pascher, \emph{Redshift and Deflection}, 2025.
\url{https://github.com/jpascher/T0-Time-Mass-Duality/blob/main/2/pdf/redshift_deflection_En.pdf}

\bibitem{scheinbar_instantan}
J. Pascher, \emph{Apparent Instantaneity}, 2025.
\url{https://github.com/jpascher/T0-Time-Mass-Duality/blob/main/2/pdf/scheinbar_instantan_En.pdf}

\bibitem{universale_ableitung}
J. Pascher, \emph{Universal Derivation}, 2025.
\url{https://github.com/jpascher/T0-Time-Mass-Duality/blob/main/2/pdf/universale-ableitung_En.pdf}

\bibitem{xi_parameter}
J. Pascher, \emph{Xi Parameter for Particles}, 2025.
\url{https://github.com/jpascher/T0-Time-Mass-Duality/blob/main/2/pdf/xi_parmater_partikel_En.pdf}

\bibitem{xi_ursprung}
J. Pascher, \emph{Origin of Xi}, 2025.
\url{https://github.com/jpascher/T0-Time-Mass-Duality/blob/main/2/pdf/T0_xi_ursprung_En.pdf}

\bibitem{zeit}
J. Pascher, \emph{Time in T0 Theory}, 2025.
\url{https://github.com/jpascher/T0-Time-Mass-Duality/blob/main/2/pdf/Zeit_En.pdf}

\bibitem{zeit_konstant}
J. Pascher, \emph{Time Constant}, 2025.
\url{https://github.com/jpascher/T0-Time-Mass-Duality/blob/main/2/pdf/Zeit-konstant_En.pdf}

\bibitem{zusammenfassung}
J. Pascher, \emph{Summary of T0 Theory}, 2025.
\url{https://github.com/jpascher/T0-Time-Mass-Duality/blob/main/2/pdf/Zusammenfassung_En.pdf}

\bibitem{rsa}
J. Pascher, \emph{RSA in T0 Framework}, 2025.
\url{https://github.com/jpascher/T0-Time-Mass-Duality/blob/main/2/pdf/RSA_En.pdf}

\bibitem{qat}
J. Pascher, \emph{Quantum Atomic Theory}, 2025.
\url{https://github.com/jpascher/T0-Time-Mass-Duality/blob/main/2/pdf/T0_QAT_En.pdf}

\bibitem{qm_qft_rt}
J. Pascher, \emph{QM, QFT and RT Unification}, 2025.
\url{https://github.com/jpascher/T0-Time-Mass-Duality/blob/main/2/pdf/T0_QM-QFT-RT_En.pdf}

\bibitem{qm_optimierung}
J. Pascher, \emph{QM Optimization}, 2025.
\url{https://github.com/jpascher/T0-Time-Mass-Duality/blob/main/2/pdf/T0_QM-optimierung_En.pdf}

\bibitem{vollstaendige_berechnungen}
J. Pascher, \emph{Complete Calculations}, 2025.
\url{https://github.com/jpascher/T0-Time-Mass-Duality/blob/main/2/pdf/T0_Vollstaendige_Berchnungen_En.pdf}

\bibitem{synergetics}
J. Pascher, \emph{T0 Theory vs Synergetics}, 2025.
\url{https://github.com/jpascher/T0-Time-Mass-Duality/blob/main/2/pdf/T0-Theory-vs-Synergetics_En.pdf}

\bibitem{modell_uebersicht}
J. Pascher, \emph{T0 Model Overview}, 2025.
\url{https://github.com/jpascher/T0-Time-Mass-Duality/blob/main/2/pdf/T0_Modell_Uebersicht_En.pdf}

\bibitem{mnras_widerlegung}
J. Pascher, \emph{MNRAS Analysis}, 2025.
\url{https://github.com/jpascher/T0-Time-Mass-Duality/blob/main/2/pdf/T0_Analyse_MNRAS_Widerlegung_En.pdf}

\bibitem{anomale_magnetische_momente}
J. Pascher, \emph{Anomalous Magnetic Moments}, 2025.
\url{https://github.com/jpascher/T0-Time-Mass-Duality/blob/main/2/pdf/T0_Anomale_Magnetische_Momente_En.pdf}

\bibitem{sieben_fragen}
J. Pascher, \emph{Seven Questions in T0}, 2025.
\url{https://github.com/jpascher/T0-Time-Mass-Duality/blob/main/2/pdf/T0_7-fragen-3_En.pdf}

\bibitem{detailierte_leptonen}
J. Pascher, \emph{Detailed Lepton Anomaly}, 2025.
\url{https://github.com/jpascher/T0-Time-Mass-Duality/blob/main/2/pdf/detailierte_formel_leptonen_anemal_En.pdf}

\bibitem{parameterherleitung}
J. Pascher, \emph{Parameter Derivation}, 2025.
\url{https://github.com/jpascher/T0-Time-Mass-Duality/blob/main/2/pdf/parameterherleitung_En.pdf}

\bibitem{verhaeltnis_absolut}
J. Pascher, \emph{Absolute Ratios in T0}, 2025.
\url{https://github.com/jpascher/T0-Time-Mass-Duality/blob/main/2/pdf/T0_verhaeltnis-absolut_En.pdf}

\bibitem{xi_und_e}
J. Pascher, \emph{Xi and Energy}, 2025.
\url{https://github.com/jpascher/T0-Time-Mass-Duality/blob/main/2/pdf/T0_xi-und-e_En.pdf}

\bibitem{umkehrung}
J. Pascher, \emph{Inversion in T0}, 2025.
\url{https://github.com/jpascher/T0-Time-Mass-Duality/blob/main/2/pdf/T0_umkehrung_En.pdf}

\bibitem{esm_analysis}
J. Pascher, \emph{T0 vs ESM Conceptual Analysis}, 2025.
\url{https://github.com/jpascher/T0-Time-Mass-Duality/blob/main/2/pdf/T0vsESM_ConceptualAnalysis_En.pdf}

\end{thebibliography}

\end{document}
