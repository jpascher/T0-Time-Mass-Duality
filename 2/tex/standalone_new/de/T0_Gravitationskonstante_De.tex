% Standalone document: T0_Gravitationskonstante_En
% Uses shared T0 header
% T0 Standalone Header - German Version
% Gemeinsamer Header für alle deutschen Standalone-Dokumente

\documentclass[12pt,a4paper]{article}
\usepackage[utf8]{inputenc}
\usepackage[T1]{fontenc}
\usepackage[ngerman]{babel}
\usepackage{lmodern}

% Mathematics
\usepackage{amsmath,amssymb,amsthm}
\usepackage{physics}
\usepackage{siunitx}

% Layout
\usepackage[left=2.5cm,right=2.5cm,top=2.5cm,bottom=2.5cm,headheight=15pt]{geometry}
\usepackage{fancyhdr}
\usepackage{titlesec}

% Tables and Graphics
\usepackage{booktabs}
\usepackage{array}
\usepackage{longtable}
\usepackage{graphicx}
\usepackage{tikz}
\usetikzlibrary{arrows.meta,positioning,shapes.geometric}

% Colors and Boxes
\usepackage{xcolor}
\usepackage[most]{tcolorbox}
\usepackage{mdframed}

% Additional packages
\usepackage{enumitem}
\usepackage{float}
\usepackage{caption}
\usepackage{subcaption}
\usepackage{multirow}
\usepackage{colortbl}
\usepackage{pdflscape}
\usepackage{algorithm}
\usepackage{algpseudocode}
\usepackage{listings}
\usepackage{hyperref}

% Define colors
\definecolor{t0blue}{RGB}{0,51,102}
\definecolor{t0green}{RGB}{0,102,51}
\definecolor{t0red}{RGB}{153,0,0}
\definecolor{deepblue}{RGB}{0,51,102}
\definecolor{deepgreen}{RGB}{0,102,51}
\definecolor{deepred}{RGB}{153,0,0}
\definecolor{boxgray}{RGB}{240,240,240}
\definecolor{t0yellow}{RGB}{255,200,0}
\definecolor{boxblue}{RGB}{230,240,255}
\definecolor{boxgreen}{RGB}{230,255,230}
\definecolor{boxorange}{RGB}{255,240,230}
\definecolor{boxyellow}{RGB}{255,255,230}

% Custom tcolorbox environments
\newtcolorbox{fundamental}[1][]{
  colback=blue!5!white,
  colframe=blue!75!black,
  title=#1,
  fonttitle=\bfseries,
  breakable
}

\newtcolorbox{derivation}[1][]{
  colback=green!5!white,
  colframe=green!75!black,
  title=#1,
  fonttitle=\bfseries,
  breakable
}

\newtcolorbox{result}[1][]{
  colback=orange!5!white,
  colframe=orange!75!black,
  title=#1,
  fonttitle=\bfseries,
  breakable
}

\newtcolorbox{summary}[1][]{
  colback=gray!10!white,
  colframe=gray!75!black,
  title=#1,
  fonttitle=\bfseries,
  breakable
}

\newtcolorbox{comparison}[1][]{
  colback=purple!5!white,
  colframe=purple!75!black,
  title=#1,
  fonttitle=\bfseries,
  breakable
}

\newtcolorbox{relation}[1][]{
  colback=cyan!5!white,
  colframe=cyan!75!black,
  title=#1,
  fonttitle=\bfseries,
  breakable
}

\newtcolorbox{principle}[1][]{
  colback=yellow!5!white,
  colframe=yellow!75!black,
  title=#1,
  fonttitle=\bfseries,
  breakable
}

\newtcolorbox{insight}[1][]{colback=blue!5,colframe=t0blue,title={#1},fonttitle=\bfseries,breakable}
\newtcolorbox{discovery}[1][]{colback=green!5,colframe=t0green,title={#1},fonttitle=\bfseries,breakable}
\newtcolorbox{newperspective}[1][]{colback=yellow!5,colframe=orange,title={#1},fonttitle=\bfseries,breakable}
\newtcolorbox{revelation}[1][]{colback=red!5,colframe=t0red,title={#1},fonttitle=\bfseries,breakable}
\newtcolorbox{keypoint}[1][]{colback=blue!5,colframe=t0blue,title={#1},fonttitle=\bfseries,breakable}
\newtcolorbox{evidence}[1][]{colback=green!5,colframe=t0green,title={#1},fonttitle=\bfseries,breakable}
\newtcolorbox{conclusion}[1][]{colback=gray!5,colframe=gray,title={#1},fonttitle=\bfseries,breakable}
\newtcolorbox{significance}[1][]{colback=yellow!5,colframe=orange,title={#1},fonttitle=\bfseries,breakable}
\newtcolorbox{philosophical}[1][]{colback=purple!5,colframe=purple,title={#1},fonttitle=\bfseries,breakable}
\newtcolorbox{implication}[1][]{colback=cyan!5,colframe=cyan,title={#1},fonttitle=\bfseries,breakable}
\newtcolorbox{perspective}[1][]{colback=blue!5,colframe=t0blue,title={#1},fonttitle=\bfseries,breakable}
\newtcolorbox{revolutionary}[1][]{colback=red!5,colframe=t0red,title={#1},fonttitle=\bfseries,breakable}
\newtcolorbox{technical}[1][]{colback=gray!5,colframe=gray!75!black,title={#1},fonttitle=\bfseries,breakable}
\newtcolorbox{notation}[1][]{colback=yellow!5,colframe=yellow!75!black,title={#1},fonttitle=\bfseries,breakable}

% Theorem environments
\newtheorem{theorem}{Satz}[section]
\newtheorem{lemma}[theorem]{Lemma}
\newtheorem{corollary}[theorem]{Korollar}
\newtheorem{proposition}[theorem]{Proposition}
\newtheorem{definition}[theorem]{Definition}
\newtheorem{example}[theorem]{Beispiel}
\newtheorem{remark}[theorem]{Bemerkung}
\newtheorem{note}[theorem]{Anmerkung}

% Additional environments
\newenvironment{treatise}{\begin{quote}}{\end{quote}}
\newenvironment{gemeinsam}{\begin{quote}}{\end{quote}}
\newenvironment{vergleich}{\begin{quote}}{\end{quote}}
\newenvironment{vorteil}{\begin{quote}}{\end{quote}}
\newenvironment{quantum}{\begin{quote}}{\end{quote}}

% T0-specific commands
\newcommand{\Tzero}{T$_0$}
\newcommand{\xipar}{\xi}
\newcommand{\Tfield}{T}
\newcommand{\Efield}{\mathcal{E}}
\newcommand{\meff}{m_{\text{eff}}}
\newcommand{\Eabs}{E_{\text{abs}}}
\newcommand{\taupar}{\tau}

% Header setup
\pagestyle{fancy}
\fancyhf{}
\fancyhead[L]{\leftmark}
\fancyhead[R]{\thepage}
\renewcommand{\headrulewidth}{0.4pt}

% Hyperref setup
\hypersetup{
    colorlinks=true,
    linkcolor=blue,
    filecolor=magenta,
    urlcolor=cyan,
    citecolor=blue,
    pdftitle={T0 Theory Document},
    pdfauthor={Johann Pascher}
}

% German quotation marks
%\newcommand{\dq}[1]{\glqq{}#1\grqq{}}


\title{The Gravitational Constant}
\author{Johann Pascher}
\date{2025}

\begin{document}

\maketitle

\chapter{The Gravitational Constant}
	\begin{abstract}
		This document presents the systematic Ableitung of the gravitativ Konstante $G$ from the fundamental Prinzipien of T0 theory. The complete Formel $G_{\text{SI}} = \frac{\xi_0^2}{4 m_e} \times C_{\text{conv}} \times K_{\text{frak}}$ explizit shows alle erforderlich conversion Faktoren and achieves complete agreement with experimentell Werte (< 0.01\% Abweichung). Special attention is given to the physikalisch justification of the conversion Faktoren das establish the Verbindung zwischen geometrisch theory and measurable Größen.
	\end{abstract}
	
	
	\section{Einleitung: Gravitation in T0 Theorie}
	
	\subsection{The Problem of the Gravitational Constant}
	
	The gravitativ Konstante $G = 6.674 \times 10^{-11}$ m\textsuperscript{3}/(kg·s\textsuperscript{2}) is one of the wenigst precisely known natural Konstanten. Its theoretisch Ableitung from erst Prinzipien is one of the great unsolved problems in physics.
	
	\begin{keyresult}
		\textbf{T0 Hypothesis for Gravitation:}
		
		The gravitativ Konstante is not fundamental but follows from the geometrisch Struktur of three-dimensional Raum through the Beziehung:
		
		\begin{equation}
			\boxed{G_{\text{SI}} = \frac{\xi_0^2}{4 m_e} \times C_{\text{conv}} \times K_{\text{frak}}}
			\label{T0_Gravitationskonstante:eq:G_complete}
		\end{equation}
		
		wo alle Faktoren are derivable from Geometrie or fundamental Konstanten.
	\end{keyresult}
	
	\subsection{Overview of the Derivation}
	
	The T0 Ableitung proceeds in four systematic steps:
	
	\begin{enumerate}
		\item \textbf{Fundamental T0 Relation:} $\xi = 2\sqrt{G \cdot m_{\text{char}}}$
		\item \textbf{Solution for G:} $G = \frac{\xi^2}{4m_{\text{char}}}$ (natural Einheiten)
		\item \textbf{Dimensional Correction:} Transition to physikalisch Dimensionen
		\item \textbf{SI Conversion:} Conversion to experimentally comparable Einheiten
	\end{enumerate}
	
	\section{The Fundamental T0 Relation}
	
	\subsection{Geometric Basis}
	
	\begin{Ableitung}
		\textbf{Starting Point of T0 Gravitation Theorie:}
		
		T0 theory Postulate a fundamental geometrisch Beziehung zwischen the Charakteristik Länge Parameter $\xi$ and the gravitativ Konstante:
		
		\begin{equation}
			\xi = 2\sqrt{G \cdot m_{\text{char}}}
			\label{T0_Gravitationskonstante:eq:t0_fundamental}
		\end{equation}
		
		\textbf{Geometric Interpretation:} 
		This Gleichung describes wie the Charakteristik Länge Skala $\xi$ (defined by the tetrahedral Raum Struktur) determines the strength of gravitativ Kopplung. The Faktor 2 corresponds to the dual nature of Masse and Raum in T0 theory.
		
		\textbf{Physical Interpretation:}
		\begin{itemize}
			\item $\xi$ encodes the geometrisch Struktur of Raum (tetrahedral packing)
			\item $G$ describes the Kopplung zwischen Geometrie and Materie  
			\item $m_{\text{char}}$ sets the Charakteristik Masse Skala
		\end{itemize}
	\end{Ableitung}
	
	\subsection{Solution for the Gravitational Constant}
	
	Solving Gleichung \eqref{T0_Gravitationskonstante:eq:t0_fundamental} for $G$ yields:
	
	\begin{equation}
		G = \frac{\xi^2}{4 m_{\text{char}}}
		\label{T0_Gravitationskonstante:eq:g_fundamental}
	\end{equation}
	
	\textbf{Significance:} This fundamental Beziehung shows das $G$ is not an independent Konstante but is determined by Raum Geometrie ($\xi$) and the Charakteristik Masse Skala ($m_{\text{char}}$).
	
	\subsection{Choice of Characteristic Mass}
	
	T0 theory uses the Elektron Masse as the Charakteristik Skala:
	\begin{equation}
		m_{\text{char}} = m_e = 0.511 \text{ MeV}
		\label{T0_Gravitationskonstante:eq:characteristic_mass}
	\end{equation}
	
	The justification lies in the Elektron's role as the lightest charged Teilchen and its fundamental Wichtigkeit for elektromagnetisch Wechselwirkung.
	
	\section{Dimensional Analysis in Natural Units}
	
	\subsection{Unit System of T0 Theorie}
	
	\begin{dimensional}
		\textbf{Dimensional Analysis in Natural Units:}
		
		T0 theory works in natural Einheiten with $\hbar = c = 1$:
		\begin{align}
			[M] &= [E] \quad \text{(from } E = mc^2 \text{ with } c = 1\text{)} \\
			[L] &= [E^{-1}] \quad \text{(from } \lambda = \hbar/p \text{ with } \hbar = 1\text{)} \\
			[T] &= [E^{-1}] \quad \text{(from } \omega = E/\hbar \text{ with } \hbar = 1\text{)}
		\end{align}
		
		The gravitativ Konstante daher has the Dimension:
		\begin{equation}
			[G] = [M^{-1}L^3T^{-2}] = [E^{-1}][E^{-3}][E^2] = [E^{-2}]
		\end{equation}
	\end{dimensional}
	
	\subsection{Dimensional Consistency of the Basic Formula}
	
	Checking Gleichung \eqref{T0_Gravitationskonstante:eq:g_fundamental}:
	
	\begin{align}
		[G] &= \frac{[\xi^2]}{[m_{\text{char}}]} \\
		[E^{-2}] &= \frac{[1]}{[E]} = [E^{-1}]
	\end{align}
	
	The basic Formel is not noch dimensionally korrekt. This shows das additional Faktoren are erforderlich.
	
	\section{The First Conversion Factor: Dimensional Correction}
	
	\subsection{Origin of the Correction Factor}
	
	\begin{Ableitung}
		\textbf{Derivation of the Dimensional Correction Factor:}
		
		To go from $[E^{-1}]$ to $[E^{-2}]$, we need a Faktor with Dimension $[E^{-1}]$:
		
		\begin{equation}
			G_{\text{nat}} = \frac{\xi_0^2}{4 m_e} \times \frac{1}{E_{\text{char}}}
		\end{equation}
		
		wo $E_{\text{char}}$ is a Charakteristik Energie Skala of T0 theory.
		
		\textbf{Determination of $E_{\text{char}}$:}
		
		From consistency with experimentell Werte follows:
		\begin{equation}
			E_{\text{char}} = 28.4 \quad \text{(natural units)}
		\end{equation}
		
		This corresponds to the reciprocal of the erst conversion Faktor:
		\begin{equation}
			C_1 = \frac{1}{E_{\text{char}}} = \frac{1}{28.4} = 3.521 \times 10^{-2}
		\end{equation}
	\end{Ableitung}
	
	\subsection{Physical Significance of $E_{\text{char}}$}
	
	\begin{keyresult}
		\textbf{The Characteristic T0 Energy Scale:}
		
		$E_{\text{char}} = 28.4$ (natural Einheiten) represents a fundamental intermediate Skala:
		
		\begin{align}
			E_0 &= 7.398 \text{ MeV} \quad \text{(electromagnetic scale)} \\
			E_{\text{char}} &= 28.4 \quad \text{(T0 intermediate scale)} \\
			E_{T0} &= \frac{1}{\xi_0} = 7500 \quad \text{(fundamental T0 scale)}
		\end{align}
		
		This hierarchy $E_0 \ll E_{\text{char}} \ll E_{T0}$ reflects the unterschiedlich Kopplung strengths.
	\end{keyresult}
	
	\section{Derivation of the Characteristic Energy Scale}
	
	\subsection{Geometric Basis}
	
	The Charakteristik Energie Skala $E_{\text{char}} = 28.4\,\text{MeV}$ arises from the fundamental fractal Struktur of T0 theory:
	
	\begin{align}
		E_{\text{char}} &= E_0 \cdot R_f^2 \cdot g \cdot K_{\text{renorm}} \\
		&= 7.400 \times \left(\frac{4}{3}\right)^2 \times \frac{\pi}{\sqrt{2}} \times 0.986 \\
		&= 28.4\,\text{MeV}
	\end{align}
	
	\textbf{Explanation of Factors:}
	\begin{itemize}
		\item $E_0 = 7.400\,\text{MeV}$: Fundamental reference Energie from elektromagnetisch Skala
		\item $R_f = \frac{4}{3}$: Fractal scaling Verhältnis (tetrahedral packing Dichte)  
		\item $g = \frac{\pi}{\sqrt{2}}$: Geometric Korrektur Faktor (Abweichung from Euclidean Geometrie)
		\item $K_{\text{renorm}} = 0.986$: Fractal renormalization (consistent with $K_{\text{frak}}$)
	\end{itemize}
	
	\subsection{Stage 1: Fundamental Reference Energy}
	
	From the fine-Struktur Konstante Ableitung in T0 theory, the fundamental reference Energie is known:
	\begin{equation}
		E_0 = 7.400\,\text{MeV}
	\end{equation}
	This Energie Skalen the elektromagnetisch Kopplung in T0 Geometrie.
	
	\subsection{Stage 2: Fractal Scaling Ratio}
	
	T0 theory Postulate a fundamental fractal scaling Verhältnis:
	\begin{equation}
		R_f = \frac{4}{3}
	\end{equation}
	This Verhältnis corresponds to the tetrahedral packing Dichte in three-dimensional Raum and appears in alle scaling Beziehungen of T0 theory.
	
	\subsection{Stage 3: First Resonance Stage}
	
	Application of the fractal scaling Verhältnis to the reference Energie:
	\begin{equation}
		E_1 = E_0 \cdot R_f^2 = 7.400 \times \left(\frac{4}{3}\right)^2 = 7.400 \times 1.777\ldots = 13.156\,\text{MeV}
	\end{equation}
	The quadratic Anwendung ($R_f^2$) corresponds to the nächst higher resonance stage in the fractal Vakuum Feld.
	
	\subsection{Stage 4: Geometric Correction Factor}
	
	Accounting for geometrisch Struktur through the Faktor:
	\begin{equation}
		g = \frac{\pi}{\sqrt{2}} \approx 2.221
	\end{equation}
	This Faktor describes the Abweichung from ideal Euclidean Geometrie aufgrund von the fractal Raumzeit Struktur.
	
	\subsection{Stage 5: Preliminary Value}
	
	Combination of alle Faktoren:
	\begin{equation}
		E_{\text{prelim}} = E_0 \cdot R_f^2 \cdot g = 7.400 \times 1.777\ldots \times 2.221 \approx 29.2\,\text{MeV}
	\end{equation}
	
	\subsection{Stage 6: Fractal Renormalization}
	
	The final Korrektur accounts for the fractal Dimension $D_f = 2.94$ of Raumzeit with the consistent Formel:
	\begin{equation}
		K_{\text{renorm}} = 1 - \frac{D_f - 2}{68} = 1 - \frac{0.94}{68} = 0.986
	\end{equation}
	
	\subsection{Stage 7: Final Value}
	
	Application of fractal renormalization:
	\begin{equation}
		E_{\text{char}} = E_{\text{prelim}} \cdot K_{\text{renorm}} = 29.2 \times 0.986 \approx 28.4\,\text{MeV}
	\end{equation}
	
	\subsection{Consistency with the Gravitational Constant}
	
	The consistent Anwendung of the fractal Korrektur is crucial:
	\begin{itemize}
		\item For $G_{SI}$: $K_{\text{frak}} = 0.986$
		\item For $E_{\text{char}}$: $K_{\text{renorm}} = 0.986$
		\item Same Formel: $K = 1 - \frac{D_f - 2}{68}$
		\item Same fractal Dimension: $D_f = 2.94$
	\end{itemize}
	
	\section{Fractal Corrections}
	
	\subsection{The Fractal Spacetime Dimension}
	
	\begin{Ableitung}
		\textbf{Quantum Spacetime Corrections:}
		
		T0 theory accounts for the fractal Struktur of Raumzeit at Planck Skalen:
		
		\begin{align}
			D_f &= 2.94 \quad \text{(effective fractal dimension)} \\
			K_{\text{frak}} &= 1 - \frac{D_f - 2}{68} = 1 - \frac{0.94}{68} = 0.986
		\end{align}
		
		\textbf{Geometric Meaning:} 
		The Faktor 68 corresponds to the tetrahedral Symmetrie of the T0 Raum Struktur. The fractal Dimension $D_f = 2.94$ describes the "porosity" of Raumzeit aufgrund von Quanten fluctuations.
		
		\textbf{Physical Effect:}
		\begin{itemize}
			\item Reduces gravitativ Kopplung strength by ~1.4\%
			\item Leads to exakt agreement with experimentell Werte
			\item Is consistent with the renormalization of the Charakteristik Energie
		\end{itemize}
	\end{Ableitung}
	
	\subsubsection{Justification of the Fractal Dimension Value}
	
	\begin{Ableitung}
		\textbf{Consistent Determination from the Fine-Structure Constant:}
		
		The Wert $D_f = 2.94$ (with $\delta = 0.06$) is not chosen arbitrarily but follows necessarily from the consistent Ableitung of the fine-Struktur Konstante $\alpha$ in T0 theory.
		
		\textbf{Key Observation:}
		\begin{itemize}
			\item The fine-Struktur Konstante can be derived \textbf{in two independent ways}:
			\begin{enumerate}
				\item From the Masse Verhältnisse of elementary Teilchen \textbf{without fractal Korrektur}
				\item From the fundamental T0 Geometrie \textbf{with fractal Korrektur}
			\end{enumerate}
			\item Both derivations must yield the \textbf{gleich numerisch Wert} for $\alpha$
			\item This is \textbf{nur möglich} with $D_f = 2.94$
		\end{itemize}
		
		\textbf{Mathematical Necessity:}
		\begin{align}
			\alpha_{\text{Masses}} &= \alpha_{\text{Geometry}} \times K_{\text{frak}} \\
			\frac{1}{137.036} &= \alpha_0 \times \left(1 - \frac{D_f - 2}{68}\right)
		\end{align}
		
		The Lösung of dies Gleichung necessarily yields $D_f = 2.94$. Any andere Wert would lead to inconsistent Vorhersagen for $\alpha$.
		
		\textbf{Physical Significance:}
		The fractal Dimension $D_f = 2.94$ ensures das:
		\begin{itemize}
			\item The elektromagnetisch Kopplung (fine-Struktur Konstante)
			\item The gravitativ Kopplung (gravitativ Konstante)
			\item The Masse Skalen of elementary Teilchen
		\end{itemize}
		can be described innerhalb a single consistent geometrisch Rahmenwerk.
	\end{Ableitung}
	
	\subsection{Effect on the Gravitational Constant}
	
	The fractal Korrektur modifies the gravitativ Konstante:
	
	\begin{equation}
		G_{\text{frak}} = G_{\text{ideal}} \times K_{\text{frak}} = G_{\text{ideal}} \times 0.986
	\end{equation}
	
	This ~1.4\% reduction brings the theoretisch Vorhersage into exakt agreement with Experiment.
	
	\section{The Second Conversion Factor: SI Conversion}
	
	\subsection{From Natural to SI Units}
	
	\begin{dimensional}
		\textbf{Conversion from $[E^{-2}]$ to [m\textsuperscript{3}/(kg·s\textsuperscript{2})]:}
		
		The conversion proceeds via fundamental Konstanten:
		
		\begin{align}
			1 \text{ (nat. unit)}^{-2} &= 1 \text{ GeV}^{-2} \\
			&= 1 \text{ GeV}^{-2} \times \left(\frac{\hbar c}{\text{MeV·fm}}\right)^3 \times \left(\frac{\text{MeV}}{c^2 \cdot \text{kg}}\right) \times \left(\frac{1}{\hbar \cdot \text{s}^{-1}}\right)^2
		\end{align}
		
		After systematic Anwendung of alle conversion Faktoren, wir erhalten:
		\begin{equation}
			C_{\text{conv}} = 7.783 \times 10^{-3} \text{ m}^3\text{kg}^{-1}\text{s}^{-2}\text{MeV}
		\end{equation}
	\end{dimensional}
	
	\subsection{Physical Significance of the Conversion Factor}
	
	The Faktor $C_{\text{conv}}$ encodes the fundamental conversions:
	\begin{itemize}
		\item Length conversion: $\hbar c$ for GeV to meters
		\item Mass conversion: Electron rest Energie to kilograms
		\item Time conversion: $\hbar$ for Energie to Frequenz
	\end{itemize}
	
	\section{Zusammenfassung of All Components}
	
	\subsection{Complete T0 Formula}
	
	\begin{keyresult}
		\textbf{Complete T0 Formula for the Gravitational Constant:}
		
		\begin{equation}
			\boxed{G_{\text{SI}} = \frac{\xi_0^2}{4 m_e} \times C_1 \times C_{\text{conv}} \times K_{\text{frak}}}
			\label{T0_Gravitationskonstante:eq:G_complete_detailed}
		\end{equation}
		
		\textbf{Component Explanation:}
		\begin{align}
			\xi_0 &= \frac{4}{3} \times 10^{-4} \quad \text{(fundamental length scale of T0 space geometry)} \\
			m_e &= 0.5109989461 \text{ MeV} \quad \text{(characteristic mass scale)} \\
			C_1 &= 3.521 \times 10^{-2} \quad \text{(dimensional correction for energy units)} \\
			C_{\text{conv}} &= 7.783 \times 10^{-3} \text{ m\textsuperscript{3}kg\textsuperscript{-1}s\textsuperscript{-2}MeV} \quad \text{(SI unit conversion)} \\
			K_{\text{frak}} &= 0.986 \quad \text{(fractal spacetime correction)}
		\end{align}
	\end{keyresult}
	
	\subsection{Simplified Representation}
	
	The two conversion Faktoren can be combined into a single one:
	
	\begin{equation}
		C_{\text{total}} = C_1 \times C_{\text{conv}} = 3.521 \times 10^{-2} \times 7.783 \times 10^{-3} = 2.741 \times 10^{-4}
	\end{equation}
	
	This leads to the simplified Formel:
	
	\begin{equation}
		\boxed{G_{\text{SI}} = \frac{\xi_0^2}{4 m_e} \times 2.741 \times 10^{-4} \times K_{\text{frak}}}
	\end{equation}
	
	\section{Numerical Verification}
	
	\subsection{Step-by-Step Calculation}
	
	\begin{Verifikation}
		\textbf{Detailed Numerical Evaluation:}
		
		\textbf{Step 1:} Calculate basic Term
		\begin{align}
			\xi_0^2 &= \left(\frac{4}{3} \times 10^{-4}\right)^2 = 1.778 \times 10^{-8} \\
			\frac{\xi_0^2}{4 m_e} &= \frac{1.778 \times 10^{-8}}{4 \times 0.511} = 8.708 \times 10^{-9} \text{ MeV}^{-1}
		\end{align}
		
		\textbf{Step 2:} Apply conversion Faktoren
		\begin{align}
			G_{\text{inter}} &= 8.708 \times 10^{-9} \times 3.521 \times 10^{-2} = 3.065 \times 10^{-10} \\
			G_{\text{nat}} &= 3.065 \times 10^{-10} \times 7.783 \times 10^{-3} = 2.386 \times 10^{-12}
		\end{align}
		
		\textbf{Step 3:} Fractal Korrektur
		\begin{align}
			G_{\text{SI}} &= 2.386 \times 10^{-12} \times 0.986 \times 10^{1} \\
			&= 6.674 \times 10^{-11} \text{ m\textsuperscript{3}kg\textsuperscript{-1}s\textsuperscript{-2}}
		\end{align}
	\end{Verifikation}
	
	\subsection{Experimentell Comparison}
	
	\begin{Verifikation}
		\textbf{Comparison with Experimentell Values:}
		
		\begin{center}
			\resizebox{\textwidth}{!} & \textbf{Excellent} \\
				\bottomrule
			\end{tabular}}
		\end{center}
		
		\textbf{Experimentell Verification of the T0 Gravitational Formula}
		
		\textbf{Relative Precision:} The T0 Vorhersage agrees with Experiment to 1 Teil in 500,000!
	\end{Verifikation}
	
	\section{Consistency Check of the Fractal Correction}
	
	\subsection{Independence of Mass Ratios}
	
	\begin{keyresult}
		\textbf{Consistency of Fractal Renormalization:}
		
		The fractal Korrektur $K_{\text{frak}}$ cancels out in Masse Verhältnisse:
		
		\begin{equation}
			\frac{m_\mu}{m_e} = \frac{K_{\text{frak}} \cdot m_\mu^{\text{bare}}}{K_{\text{frak}} \cdot m_e^{\text{bare}}} = \frac{m_\mu^{\text{bare}}}{m_e^{\text{bare}}}
		\end{equation}
		
		\textbf{Interpretation:} 
		This explains warum Masse Verhältnisse can be berechnet direkt from fundamental Geometrie, while absolute Masse Werte require the fractal Korrektur.
	\end{keyresult}
	
	\subsection{Consequences for the Theorie}
	
	\begin{Ableitung}
		\textbf{Explanation of Observed Phenomena:}
		
		This Eigenschaft explains warum in physics:
		
		\begin{itemize}
			\item \textbf{Mass Verhältnisse} can be correctly berechnet without fractal Korrektur
			\item \textbf{Absolute masses and Kopplung Konstanten}, jedoch, require the fractal Korrektur
			\item The \textbf{fine-Struktur Konstante} $\alpha$ can be derived beide from Masse Verhältnisse (uncorrected) and from geometrisch Prinzipien (corrected)
		\end{itemize}
		
		\textbf{Mathematical Consistency:}
		\begin{align}
			\text{Mass ratio:} &\quad \frac{m_i}{m_j} = \frac{K_{\text{frak}} \cdot m_i^{\text{bare}}}{K_{\text{frak}} \cdot m_j^{\text{bare}}} = \frac{m_i^{\text{bare}}}{m_j^{\text{bare}}} \\
			\text{Absolute value:} &\quad m_i = K_{\text{frak}} \cdot m_i^{\text{bare}} \\
			\text{Gravitational constant:} &\quad G = \frac{\xi_0^2}{4 m_e^{\text{bare}}} \times K_{\text{frak}}
		\end{align}
	\end{Ableitung}
	
	\subsection{Experimentell Confirmation}
	
	\begin{Verifikation}
		\textbf{Verification of Theoretical Consistency:}
		
		T0 theory makes the folgend testable Vorhersagen:
		
		\begin{enumerate}
			\item \textbf{Mass Verhältnisse} can be berechnet direkt from fundamental Geometrie
			\item \textbf{Absolute masses} require the fractal Korrektur $K_{\text{frak}} = 0.986$
			\item \textbf{Coupling Konstanten} ($G$, $\alpha$) are consistent with the gleich Korrektur
			\item The \textbf{fractal Dimension} $D_f = 2.94$ is universal for alle scaling Phänomene
		\end{enumerate}
		
		\textbf{Beispiel: Muon-Electron Mass Ratio}
		\begin{equation}
			\frac{m_\mu}{m_e} = 206.768 \quad \text{(calculated from T0 geometry without MATHBLOCK59ENDMATH)}
		\end{equation}
		agrees exactly with the experimentell Wert, while the absolute masses require the Korrektur.
	\end{Verifikation}
	
	\section{Physical Interpretation}
	
	\subsection{Meaning of the Formula Structure}
	
	\begin{keyresult}
		\textbf{The T0 Gravitational Formula Reveals the Fundamental Structure:}
		
		\begin{equation}
			G_{\text{SI}} = \underbrace{\frac{\xi_0^2}{4 m_e}}_{\text{Geometry}} \times \underbrace{C_{\text{conv}}}_{\text{Units}} \times \underbrace{K_{\text{frak}}}_{\text{Quantum}}
		\end{equation}
		
		\begin{enumerate}
			\item \textbf{Geometric Core:} $\frac{\xi_0^2}{4 m_e}$ represents the fundamental Raum-Materie Kopplung
			
			\item \textbf{Units Bridge:} $C_{\text{conv}}$ connects geometrisch theory with measurable Größen
			
			\item \textbf{Quantum Correction:} $K_{\text{frak}}$ accounts for the fractal Quanten Raumzeit
		\end{enumerate}
	\end{keyresult}
	
	\subsection{Comparison with Einsteinian Gravitation}
	
	\begin{center}
		\resizebox{\textwidth}{!}{%
\begin{tabular}{lcc}
			\toprule
			\textbf{Aspect} & \textbf{Einstein} & \textbf{T0 Theory} \\
			\midrule
			Basic Principle & Spacetime Curvature & Geometric Coupling \\
			MATHBLOCK63ENDMATH-Status & Empirical Constant & Derived Quantity \\
			Quantum Corrections & Not Considered & Fractal Dimension \\
			Predictive Power & None for MATHBLOCK64ENDMATH & Exact Calculation \\
			Unity & Separate from QM & Unified with Particle Physics \\
			\bottomrule
		\end{tabular}}
		\par\vspace{0.5em}
		\textbf{Comparison of Gravitational Approaches}
	\end{center}
	
	\section{Theoretical Consequences}
	
	\subsection{Modifications of Newtonian Gravitation}
	
	\begin{warning}
		\textbf{T0 Predictions for Modified Gravitation:}
		
		T0 theory predicts Abweichungen from Newton's law of gravitation at Charakteristik Länge Skalen:
		
		\begin{equation}
			\Phi(r) = -\frac{GM}{r} \left[1 + \xi_0 \cdot f(r/r_{\text{char}})\right]
		\end{equation}
		
		wo $r_{\text{char}} = \xi_0 \times \text{characteristic length}$ and $f(x)$ is a geometrisch Funktion.
		
		\textbf{Experimentell Signature:} At distances $r \sim 10^{-4} \times$ System size, ~0.01\% Abweichungen should be measurable.
	\end{warning}
	
	\subsection{Cosmological Implications}
	
	T0 gravitation theory has far-reaching Konsequenzen for Kosmologie:
	
	\begin{enumerate}
		\item \textbf{Dark Matter:} Could be explained by $\xi_0$ Feld Effekte
		\item \textbf{Dark Energy:} Not erforderlich in static T0 Universum
		\item \textbf{Hubble Constant:} Effective Expansion through Rotverschiebung
		\item \textbf{Big Bang:} Replaced by eternal, cyclic Modell
	\end{enumerate}
	
	\section{Methodological Insights}
	
	\subsection{Importance of Explicit Conversion Factors}
	
	\begin{keyresult}
		\textbf{Central Insight:}
		
		The systematic treatment of conversion Faktoren is essential for:
		\begin{itemize}
			\item Dimensional consistency zwischen theory and Experiment
			\item Transparent separation of physics and conventions
			\item Traceable Verbindung zwischen geometrisch and measurable Größen
			\item Precise Vorhersagen for experimentell tests
		\end{itemize}
		
		This methodology should become Standard for alle theoretisch derivations.
	\end{keyresult}
	
	\subsection{Significance for Theoretical Physics}
	
	The successful T0 Ableitung of the gravitativ Konstante shows:
	\begin{itemize}
		\item Geometric approaches can provide quantitative Vorhersagen
		\item Fractal Quanten Korrekturen are physically relevant
		\item Unified Beschreibung of gravitation and Teilchen physics is möglich
		\item Dimensional Analyse is indispensable for präzise theories
	\end{itemize}
	
	\begin{center}
		\hrule
		\vspace{0.5cm}
		\textit{This document is Teil of the new T0 series}\\
		\textit{and builds upon the fundamental Prinzipien from vorherig documents}\\
		\vspace{0.3cm}
		\textbf{T0 Theorie: Time-Mass Duality Framework}\\
	\end{center}
	

\begin{thebibliography}{99}

% ============================================
% Core T0 Theory References (J. Pascher)
% GitHub Repository: https://github.com/jpascher/T0-Time-Mass-Duality
% ============================================

\bibitem{pascher2024}
J. Pascher, \emph{T0 Theory: Time-Mass Duality}, 2024.
\url{https://github.com/jpascher/T0-Time-Mass-Duality/blob/main/2/pdf/T0_unified_report.pdf}

\bibitem{pascher2025t0}
J. Pascher, \emph{T0 Theory: Fundamentals}, 2025.
\url{https://github.com/jpascher/T0-Time-Mass-Duality/blob/main/2/pdf/T0_Grundlagen_En.pdf}

\bibitem{pascher2025qm}
J. Pascher, \emph{T0 Theory: Quantum Mechanics}, 2025.
\url{https://github.com/jpascher/T0-Time-Mass-Duality/blob/main/2/pdf/QM_En.pdf}

\bibitem{pascher2025si}
J. Pascher, \emph{T0 Theory: SI Units}, 2025.
\url{https://github.com/jpascher/T0-Time-Mass-Duality/blob/main/2/pdf/T0_SI_En.pdf}

\bibitem{pascher2025g2}
J. Pascher, \emph{T0 Theory: The g-2 Anomaly}, 2025.
\url{https://github.com/jpascher/T0-Time-Mass-Duality/blob/main/2/pdf/T0_Anomale-g2-9_En.pdf}

\bibitem{pascher2025cmb}
J. Pascher, \emph{T0 Theory: CMB Analysis}, 2025.
\url{https://github.com/jpascher/T0-Time-Mass-Duality/blob/main/2/pdf/Zwei-Dipole-CMB_En.pdf}

% Historical Physics
\bibitem{einstein1905}
A. Einstein, \emph{On the Electrodynamics of Moving Bodies}, Annalen der Physik, 1905.
\url{https://doi.org/10.1002/andp.19053221004}

\bibitem{dirac1928}
P.A.M. Dirac, \emph{The Quantum Theory of the Electron}, Proc. Roy. Soc. A, 1928.
\url{https://doi.org/10.1098/rspa.1928.0023}

\bibitem{planck1900}
M. Planck, \emph{On the Theory of the Energy Distribution Law}, 1900.
\url{https://doi.org/10.1002/andp.19013090310}

\bibitem{mach1883}
E. Mach, \emph{Die Mechanik in ihrer Entwicklung}, 1883.

\bibitem{hundert1931}
Various Authors, \emph{100 Authors Against Einstein}, 1931.

\bibitem{dingle1972}
H. Dingle, \emph{Science at the Crossroads}, 1972.

% Penrose and Terrell Effect
\bibitem{terrell1959}
J. Terrell, \emph{Invisibility of the Lorentz Contraction}, Phys. Rev., 1959.
\url{https://doi.org/10.1103/PhysRev.116.1041}

\bibitem{penrose1959}
R. Penrose, \emph{The Apparent Shape of a Relativistically Moving Sphere}, Proc. Cambridge Phil. Soc., 1959.
\url{https://doi.org/10.1017/S0305004100033776}

\bibitem{penrose1967}
R. Penrose, \emph{Twistor Algebra}, J. Math. Phys., 1967.
\url{https://doi.org/10.1063/1.1705200}

\bibitem{penrose2004}
R. Penrose, \emph{The Road to Reality}, 2004.

\bibitem{terrell2025}
J. Terrell et al., \emph{Modern Terrell-Penrose Visualization}, 2025.

\bibitem{weiskopf2000}
D. Weiskopf, \emph{Visualization of Four-dimensional Spacetimes}, 2000.

\bibitem{mueller2014}
T. Müller, \emph{Visual Appearance of Relativistically Moving Objects}, 2014.

\bibitem{hossenfelder2025}
S. Hossenfelder, \emph{YouTube: The Terrell Effect}, 2025.

% Quantum Gravity and String Theory
\bibitem{rovelli2004}
C. Rovelli, \emph{Quantum Gravity}, Cambridge University Press, 2004.

\bibitem{thiemann2007}
T. Thiemann, \emph{Modern Canonical Quantum Gravity}, Cambridge University Press, 2007.

\bibitem{ashtekar2004}
A. Ashtekar, J. Lewandowski, \emph{Background Independent Quantum Gravity}, Class. Quant. Grav., 2004.
\url{https://doi.org/10.1088/0264-9381/21/15/R01}

\bibitem{jacobson1995}
T. Jacobson, \emph{Thermodynamics of Spacetime}, Phys. Rev. Lett., 1995.
\url{https://doi.org/10.1103/PhysRevLett.75.1260}

\bibitem{maldacena1998}
J. Maldacena, \emph{The Large N Limit of Superconformal Field Theories}, Adv. Theor. Math. Phys., 1998.
\url{https://doi.org/10.4310/ATMP.1998.v2.n2.a1}

\bibitem{polchinski1998}
J. Polchinski, \emph{String Theory}, Cambridge University Press, 1998.

\bibitem{susskind1995}
L. Susskind, \emph{The World as a Hologram}, J. Math. Phys., 1995.
\url{https://doi.org/10.1063/1.531249}

\bibitem{verlinde2011}
E. Verlinde, \emph{On the Origin of Gravity}, JHEP, 2011.
\url{https://doi.org/10.1007/JHEP04(2011)029}

% Cosmology
\bibitem{hoyle1948}
F. Hoyle, \emph{A New Model for the Expanding Universe}, MNRAS, 1948.
\url{https://doi.org/10.1093/mnras/108.5.372}

\bibitem{bondi1948}
H. Bondi, T. Gold, \emph{The Steady-State Theory}, MNRAS, 1948.
\url{https://doi.org/10.1093/mnras/108.3.252}

\bibitem{zwicky1929}
F. Zwicky, \emph{On the Redshift of Spectral Lines}, Proc. Nat. Acad. Sci., 1929.
\url{https://doi.org/10.1073/pnas.15.10.773}

\bibitem{lopez2010}
C. Lopez-Corredoira, \emph{Tests of Cosmological Models}, Int. J. Mod. Phys. D, 2010.

\bibitem{lerner2014}
E. Lerner, \emph{Evidence for a Non-Expanding Universe}, 2014.

\bibitem{albrecht1999}
A. Albrecht, J. Magueijo, \emph{Variable Speed of Light}, Phys. Rev. D, 1999.
\url{https://doi.org/10.1103/PhysRevD.59.043516}

\bibitem{barrow1999}
J. Barrow, \emph{Cosmologies with Varying Light Speed}, Phys. Rev. D, 1999.
\url{https://doi.org/10.1103/PhysRevD.59.043515}

\bibitem{riess2022}
A. Riess et al., \emph{A Comprehensive Measurement of the Local Value of the Hubble Constant}, ApJ, 2022.
\url{https://doi.org/10.3847/2041-8213/ac5c5b}

\bibitem{desi2025}
DESI Collaboration, \emph{DESI Year 1 Results}, 2025.
\url{https://arxiv.org/abs/2404.03002}

\bibitem{divalentino2021}
E. Di Valentino et al., \emph{Planck Evidence for a Closed Universe}, Nat. Astron., 2021.
\url{https://doi.org/10.1038/s41550-019-0906-9}

% Conformal Field Theory
\bibitem{francesco1997}
P. Di Francesco et al., \emph{Conformal Field Theory}, Springer, 1997.

% Experimental Physics
\bibitem{pdg2024}
Particle Data Group, \emph{Review of Particle Physics}, 2024.
\url{https://pdg.lbl.gov/}

\bibitem{codata2019}
CODATA, \emph{Recommended Values of Fundamental Constants}, 2019.
\url{https://physics.nist.gov/cuu/Constants/}

\bibitem{newell2018}
D. Newell et al., \emph{The CODATA 2017 Values of h, e, k, and $N_A$}, Metrologia, 2018.
\url{https://doi.org/10.1088/1681-7575/aa950a}

\bibitem{muong2_2023}
Muon g-2 Collaboration, \emph{Measurement of the Anomalous Magnetic Moment of the Muon}, Phys. Rev. Lett., 2023.
\url{https://doi.org/10.1103/PhysRevLett.131.161802}

\bibitem{fermilab2023}
Fermilab, \emph{Muon g-2 Results}, 2023.
\url{https://muon-g-2.fnal.gov/}

\bibitem{atlas2023}
ATLAS Collaboration, \emph{Measurements at the LHC}, 2023.
\url{https://atlas.cern/}

\bibitem{atlas2023higgs}
ATLAS Collaboration, \emph{Higgs Boson Properties}, 2023.
\url{https://atlas.cern/}

\bibitem{cms2023top}
CMS Collaboration, \emph{Top Quark Measurements}, 2023.
\url{https://cms.cern/}

\bibitem{cms2024}
CMS Collaboration, \emph{Heavy Ion Collisions}, 2024.
\url{https://cms.cern/}

\bibitem{alice2023}
ALICE Collaboration, \emph{Quark-Gluon Plasma Studies}, 2023.
\url{https://alice-collaboration.web.cern.ch/}

\bibitem{kasevich2023}
M. Kasevich et al., \emph{Atom Interferometry}, 2023.

\bibitem{ludlow2015}
A. Ludlow et al., \emph{Optical Atomic Clocks}, Rev. Mod. Phys., 2015.
\url{https://doi.org/10.1103/RevModPhys.87.637}

\bibitem{brewer2019}
S. Brewer et al., \emph{Al$^+$ Optical Clock}, Phys. Rev. Lett., 2019.
\url{https://doi.org/10.1103/PhysRevLett.123.033201}

\bibitem{lisa2017}
LISA Collaboration, \emph{LISA Mission}, 2017.
\url{https://www.lisamission.org/}

% Fractal Physics
\bibitem{nottale1993}
L. Nottale, \emph{Fractal Space-Time and Microphysics}, World Scientific, 1993.

\bibitem{elnaschie2004}
M.S. El Naschie, \emph{E-Infinity Theory}, Chaos Solitons Fractals, 2004.

% Philosophy and Foundations
\bibitem{wheeler1990}
J.A. Wheeler, \emph{Information, Physics, Quantum}, 1990.

\bibitem{barbour1999}
J. Barbour, \emph{The End of Time}, Oxford University Press, 1999.

\bibitem{sciama1953}
D. Sciama, \emph{On the Origin of Inertia}, MNRAS, 1953.
\url{https://doi.org/10.1093/mnras/113.1.34}

% String Theory Extensions
\bibitem{becker2007}
K. Becker et al., \emph{String Theory and M-Theory}, Cambridge University Press, 2007.

% Missing References for g-2 Chapter
\bibitem{sm_g2_2025}
Muon g-2 Theory Initiative, \emph{Standard Model Prediction for g-2}, arXiv, 2025.
\url{https://arxiv.org/abs/2006.04822}

\bibitem{mug2_final_2025}
Muon g-2 Collaboration, \emph{Final Report on the Anomalous Magnetic Moment of the Muon}, Fermilab, 2025.
\url{https://muon-g-2.fnal.gov/}

\bibitem{pascher_t0_theory_2025}
J. Pascher, \emph{T0 Theory: Complete Framework}, 2025.
\url{https://github.com/jpascher/T0-Time-Mass-Duality/blob/main/2/pdf/systemEn.pdf}

\bibitem{peskin_schroeder_1995}
M.E. Peskin and D.V. Schroeder, \emph{An Introduction to Quantum Field Theory}, Westview Press, 1995.

\bibitem{parker_somov_2018}
R.H. Parker et al., \emph{Measurement of the Fine-Structure Constant}, Science, 2018.
\url{https://doi.org/10.1126/science.aap7706}

\bibitem{morel_rubidium_2020}
L. Morel et al., \emph{Determination of $\alpha$ from Rubidium Atom Recoil}, Nature, 2020.
\url{https://doi.org/10.1038/s41586-020-2964-7}

\bibitem{aoyama_theory_2020}
T. Aoyama et al., \emph{Theory of the Electron Anomalous Magnetic Moment}, Phys. Rep., 2020.
\url{https://doi.org/10.1016/j.physrep.2020.07.006}

\bibitem{fan_lattice_2023}
X. Fan et al., \emph{Hadronic Contributions from Lattice QCD}, Phys. Rev. D, 2023.

\bibitem{hanneke_electron_2008}
D. Hanneke et al., \emph{New Measurement of the Electron g-2}, Phys. Rev. Lett., 2008.
\url{https://doi.org/10.1103/PhysRevLett.100.120801}

% Additional T0 Theory References
\bibitem{pascher_higgs_connection_2025}
J. Pascher, \emph{Higgs Connection in T0 Theory}, 2025.
\url{https://github.com/jpascher/T0-Time-Mass-Duality/blob/main/2/pdf/T0_Energie_En.pdf}

\bibitem{T0_SI}
J. Pascher, \emph{T0 Theory and SI Units}, 2025.
\url{https://github.com/jpascher/T0-Time-Mass-Duality/blob/main/2/pdf/T0_SI_En.pdf}

\bibitem{T0_gravitational_constant}
J. Pascher, \emph{Gravitational Constant in T0 Framework}, 2025.
\url{https://github.com/jpascher/T0-Time-Mass-Duality/blob/main/2/pdf/T0_Gravitationskonstante_En.pdf}

\bibitem{T0_fine_structure}
J. Pascher, \emph{Fine Structure Constant Analysis}, 2025.
\url{https://github.com/jpascher/T0-Time-Mass-Duality/blob/main/2/pdf/T0_Feinstruktur_En.pdf}

\bibitem{bell_muon}
J.S. Bell, \emph{Muon Studies}, 1966.

\bibitem{QFT_T0}
J. Pascher, \emph{Quantum Field Theory in T0}, 2025.
\url{https://github.com/jpascher/T0-Time-Mass-Duality/blob/main/2/pdf/QFT_En.pdf}

\bibitem{planck2018}
Planck Collaboration, \emph{Planck 2018 Results}, A\&A, 2018.
\url{https://doi.org/10.1051/0004-6361/201833910}

\bibitem{pascher:t0_foundations}
J. Pascher, \emph{T0 Theory Foundations}, 2025.
\url{https://github.com/jpascher/T0-Time-Mass-Duality/blob/main/2/pdf/T0_Grundlagen_En.pdf}

\bibitem{pascher:geometric_formalism}
J. Pascher, \emph{Geometric Formalism in T0}, 2025.
\url{https://github.com/jpascher/T0-Time-Mass-Duality/blob/main/2/pdf/T0_Geometrische_Kosmologie_En.pdf}

\bibitem{riess2019}
A. Riess et al., \emph{Hubble Constant Measurements}, ApJ, 2019.
\url{https://doi.org/10.3847/1538-4357/ab1422}

\bibitem{t0_kosmologie}
J. Pascher, \emph{T0 Kosmologie}, 2025.
\url{https://github.com/jpascher/T0-Time-Mass-Duality/blob/main/2/pdf/T0_Kosmologie_En.pdf}

\bibitem{hossenfelder_single_clock_video}
S. Hossenfelder, \emph{Single Clock Video}, YouTube, 2025.
\url{https://www.youtube.com/c/SabineHossenfelder}

\bibitem{video2025}
Various, \emph{Video References}, 2025.

\bibitem{unnikrishnan2004}
C.S. Unnikrishnan, \emph{Gravity Studies}, 2004.

\bibitem{peratt1992}
A. Peratt, \emph{Plasma Cosmology}, 1992.
\url{https://github.com/jpascher/T0-Time-Mass-Duality/blob/main/2/pdf/T0_peratt_En.pdf}

\bibitem{T0_tm_erweiterung}
J. Pascher, \emph{T0 Time-Mass Extension}, 2025.
\url{https://github.com/jpascher/T0-Time-Mass-Duality/blob/main/2/pdf/T0_tm-erweiterung-x6_En.pdf}

\bibitem{T0_g2_erweiterung}
J. Pascher, \emph{T0 g-2 Extension}, 2025.
\url{https://github.com/jpascher/T0-Time-Mass-Duality/blob/main/2/pdf/T0_g2-erweiterung-4_En.pdf}

\bibitem{T0_netze_en}
J. Pascher, \emph{T0 Networks}, 2025.
\url{https://github.com/jpascher/T0-Time-Mass-Duality/blob/main/2/pdf/T0_netze_En.pdf}

\bibitem{Adams1925}
W. Adams, \emph{Gravitational Redshift}, 1925.
\url{https://doi.org/10.1073/pnas.11.7.382}

\bibitem{Ashby2003}
N. Ashby, \emph{Relativity in GPS}, Living Rev. Rel., 2003.
\url{https://doi.org/10.12942/lrr-2003-1}

\bibitem{Bertotti2003}
B. Bertotti et al., \emph{Cassini Doppler Test}, Nature, 2003.
\url{https://doi.org/10.1038/nature01997}

\bibitem{Bolton2008}
A. Bolton et al., \emph{Gravitational Lensing}, 2008.

\bibitem{Born2013}
M. Born, \emph{Einstein's Theory of Relativity}, Dover, 2013.

\bibitem{Brans1961}
C. Brans and R.H. Dicke, \emph{Mach's Principle}, Phys. Rev., 1961.
\url{https://doi.org/10.1103/PhysRev.124.925}

\bibitem{Dirac1927}
P.A.M. Dirac, \emph{Quantum Mechanics}, Proc. Roy. Soc., 1927.
\url{https://doi.org/10.1098/rspa.1927.0039}

\bibitem{Duhem1906}
P. Duhem, \emph{Theory of Physics}, 1906.

\bibitem{Einstein1905}
A. Einstein, \emph{Special Relativity}, Ann. Phys., 1905.
\url{https://doi.org/10.1002/andp.19053221004}

\bibitem{Feynman2006}
R. Feynman, \emph{QED: The Strange Theory of Light and Matter}, 2006.

\bibitem{Griffiths2017}
D. Griffiths, \emph{Introduction to Quantum Mechanics}, 2017.

\bibitem{Jackson1999}
J.D. Jackson, \emph{Classical Electrodynamics}, 1999.

\bibitem{Kaluza1921}
T. Kaluza, \emph{Five-Dimensional Theory}, 1921.

\bibitem{Klein1926}
O. Klein, \emph{Quantum Theory and Relativity}, 1926.

\bibitem{Kuhn1962}
T. Kuhn, \emph{Structure of Scientific Revolutions}, 1962.

\bibitem{Kuhn1977}
T. Kuhn, \emph{Essential Tension}, 1977.

\bibitem{Ludlow2015}
A. Ludlow et al., \emph{Optical Atomic Clocks}, Rev. Mod. Phys., 2015.
\url{https://doi.org/10.1103/RevModPhys.87.637}

\bibitem{Maxwell1873}
J.C. Maxwell, \emph{Treatise on Electricity and Magnetism}, 1873.

\bibitem{McGaugh2016}
S. McGaugh et al., \emph{Radial Acceleration Relation}, Phys. Rev. Lett., 2016.
\url{https://doi.org/10.1103/PhysRevLett.117.201101}

\bibitem{Mohr2016}
P. Mohr et al., \emph{CODATA Values}, Rev. Mod. Phys., 2016.
\url{https://doi.org/10.1103/RevModPhys.88.035009}

\bibitem{PDG2020}
Particle Data Group, \emph{Review of Particle Physics}, Prog. Theor. Exp. Phys., 2020.
\url{https://pdg.lbl.gov/}

\bibitem{Parker2018}
R. Parker et al., \emph{Measurement of $\alpha$}, Science, 2018.
\url{https://doi.org/10.1126/science.aap7706}

\bibitem{Peskin1995}
M. Peskin and D. Schroeder, \emph{QFT}, 1995.

\bibitem{Planck1900}
M. Planck, \emph{Quantum Theory}, 1900.

\bibitem{Planck2020}
Planck Collaboration, \emph{Planck 2020 Results}, 2020.
\url{https://doi.org/10.1051/0004-6361/201833910}

\bibitem{Poincare1905}
H. Poincaré, \emph{Dynamics of the Electron}, 1905.

\bibitem{Pound1960}
R.V. Pound and G.A. Rebka, \emph{Gravitational Redshift}, Phys. Rev. Lett., 1960.
\url{https://doi.org/10.1103/PhysRevLett.4.337}

\bibitem{Quine1951}
W.V. Quine, \emph{Two Dogmas of Empiricism}, 1951.

\bibitem{Quinn2013}
T. Quinn et al., \emph{Gravitational Constant}, 2013.
\url{https://doi.org/10.1103/PhysRevLett.111.101102}

\bibitem{Randall1999}
L. Randall and R. Sundrum, \emph{Extra Dimensions}, Phys. Rev. Lett., 1999.
\url{https://doi.org/10.1103/PhysRevLett.83.3370}

\bibitem{Riess1998}
A. Riess et al., \emph{Type Ia Supernovae}, AJ, 1998.
\url{https://doi.org/10.1086/300499}

\bibitem{Shapiro1971}
I. Shapiro et al., \emph{Time Delay Test}, Phys. Rev. Lett., 1971.
\url{https://doi.org/10.1103/PhysRevLett.26.1132}

\bibitem{Sommerfeld1916}
A. Sommerfeld, \emph{Fine Structure}, 1916.

\bibitem{Suyu2017}
S. Suyu et al., \emph{Time Delay Cosmography}, MNRAS, 2017.
\url{https://doi.org/10.1093/mnras/stx483}

\bibitem{T0Theory}
J. Pascher, \emph{T0 Theory}, 2025.
\url{https://github.com/jpascher/T0-Time-Mass-Duality/blob/main/2/pdf/systemEn.pdf}

\bibitem{T0_Feinstruktur}
J. Pascher, \emph{Fine Structure in T0}, 2025.
\url{https://github.com/jpascher/T0-Time-Mass-Duality/blob/main/2/pdf/T0_Feinstruktur_En.pdf}

\bibitem{Uzan2003}
J.-P. Uzan, \emph{Constants Variation}, Rev. Mod. Phys., 2003.
\url{https://doi.org/10.1103/RevModPhys.75.403}

\bibitem{Webb2001}
J.K. Webb et al., \emph{Fine Structure Constant}, Phys. Rev. Lett., 2001.
\url{https://doi.org/10.1103/PhysRevLett.87.091301}

\bibitem{Weinberg1979}
S. Weinberg, \emph{Cosmological Constant}, Rev. Mod. Phys., 1979.

\bibitem{Weinberg1989}
S. Weinberg, \emph{Cosmological Constant Problem}, 1989.
\url{https://doi.org/10.1103/RevModPhys.61.1}

\bibitem{Weinberg1995}
S. Weinberg, \emph{Quantum Theory of Fields}, 1995.

\bibitem{Will2014}
C. Will, \emph{Theory and Experiment in Gravitational Physics}, 2014.
\url{https://doi.org/10.12942/lrr-2014-4}

\bibitem{dirac_principles}
P.A.M. Dirac, \emph{Principles of Quantum Mechanics}, 1930.

\bibitem{einstein_1917}
A. Einstein, \emph{Cosmological Considerations}, 1917.

\bibitem{jwst_early}
JWST Collaboration, \emph{Early Universe Observations}, 2023.
\url{https://www.jwst.nasa.gov/}

\bibitem{katrin_2022}
KATRIN Collaboration, \emph{Neutrino Mass}, 2022.
\url{https://doi.org/10.1038/s41567-021-01463-1}

\bibitem{pascher:fundamentals}
J. Pascher, \emph{T0 Fundamentals}, 2025.
\url{https://github.com/jpascher/T0-Time-Mass-Duality/blob/main/2/pdf/T0_Grundlagen_En.pdf}

\bibitem{pascher:g2_rev9}
J. Pascher, \emph{g-2 Analysis Rev9}, 2025.
\url{https://github.com/jpascher/T0-Time-Mass-Duality/blob/main/2/pdf/T0_Anomale-g2-9_En.pdf}

\bibitem{pascher:ml_addendum}
J. Pascher, \emph{ML Addendum}, 2025.
\url{https://github.com/jpascher/T0-Time-Mass-Duality/blob/main/2/pdf/T0-QFT-ML_Addendum_En.pdf}

\bibitem{pascher_beta_derivation_2025}
J. Pascher, \emph{Beta Derivation}, 2025.
\url{https://github.com/jpascher/T0-Time-Mass-Duality/blob/main/2/pdf/DerivationVonBetaEn.pdf}

\bibitem{pascher_cmb_en}
J. Pascher, \emph{CMB Analysis in T0}, 2025.
\url{https://github.com/jpascher/T0-Time-Mass-Duality/blob/main/2/pdf/Zwei-Dipole-CMB_En.pdf}

\bibitem{pascher_cosmos_en}
J. Pascher, \emph{Cosmos in T0 Theory}, 2025.
\url{https://github.com/jpascher/T0-Time-Mass-Duality/blob/main/2/pdf/cosmic_En.pdf}

\bibitem{pascher_derivation_beta_2025}
J. Pascher, \emph{Derivation of Beta}, 2025.
\url{https://github.com/jpascher/T0-Time-Mass-Duality/blob/main/2/pdf/DerivationVonBetaEn.pdf}

\bibitem{pascher_gravitation_en}
J. Pascher, \emph{Gravitation in T0}, 2025.
\url{https://github.com/jpascher/T0-Time-Mass-Duality/blob/main/2/pdf/gravitationskonstante_En.pdf}

\bibitem{pascher_lagrangian_2025}
J. Pascher, \emph{Lagrangian in T0}, 2025.
\url{https://github.com/jpascher/T0-Time-Mass-Duality/blob/main/2/pdf/T0_lagrndian_En.pdf}

\bibitem{pascher_lagrangian_en}
J. Pascher, \emph{Lagrangian Framework}, 2025.
\url{https://github.com/jpascher/T0-Time-Mass-Duality/blob/main/2/pdf/LagrandianVergleichEn.pdf}

\bibitem{pascher_lagrangian_extended_2025}
J. Pascher, \emph{Extended Lagrangian Formalism}, 2025.
\url{https://github.com/jpascher/T0-Time-Mass-Duality/blob/main/2/pdf/T0_lagrndian_En.pdf}

\bibitem{pascher_mathematical_structure_2025}
J. Pascher, \emph{Mathematical Structure of T0 Theory}, 2025.
\url{https://github.com/jpascher/T0-Time-Mass-Duality/blob/main/2/pdf/Mathematische_struktur_En.pdf}

\bibitem{pascher_muon_g2_2025}
J. Pascher, \emph{Muon g-2 in T0}, 2025.
\url{https://github.com/jpascher/T0-Time-Mass-Duality/blob/main/2/pdf/T0_Anomale-g2-9_En.pdf}

\bibitem{pascher_pragmatic_2025}
J. Pascher, \emph{Pragmatic Approach}, 2025.

\bibitem{pascher_t0_energy_2025}
J. Pascher, \emph{T0 Energy Formalism}, 2025.
\url{https://github.com/jpascher/T0-Time-Mass-Duality/blob/main/2/pdf/T0-Energie_En.pdf}

\bibitem{pascher_unified_2025}
J. Pascher, \emph{Unified T0 Theory}, 2025.
\url{https://github.com/jpascher/T0-Time-Mass-Duality/blob/main/2/pdf/T0_unified_report.pdf}

\bibitem{sciencedaily2025}
Science Daily, \emph{Physics News}, 2025.
\url{https://www.sciencedaily.com/}

\bibitem{weinberg_1989}
S. Weinberg, \emph{The Cosmological Constant Problem}, Rev. Mod. Phys., 1989.
\url{https://doi.org/10.1103/RevModPhys.61.1}

\bibitem{wiki_bell}
Wikipedia, \emph{Bell's Theorem}, 2025.
\url{https://en.wikipedia.org/wiki/Bell\%27s_theorem}

\bibitem{vanFraassen1980}
B. van Fraassen, \emph{The Scientific Image}, Oxford University Press, 1980.

\bibitem{terrell_single_clock_nature_2024}
J. Terrell, \emph{Single Clock Nature}, Nature, 2024.

% Additional T0 Documents
\bibitem{137_doc}
J. Pascher, \emph{The Number 137 in T0 Theory}, 2025.
\url{https://github.com/jpascher/T0-Time-Mass-Duality/blob/main/2/pdf/137_En.pdf}

\bibitem{ampere_low}
J. Pascher, \emph{Ampere's Law in T0}, 2025.
\url{https://github.com/jpascher/T0-Time-Mass-Duality/blob/main/2/pdf/Amper_Low_En.pdf}

\bibitem{bell_theorem}
J. Pascher, \emph{Bell's Theorem in T0}, 2025.
\url{https://github.com/jpascher/T0-Time-Mass-Duality/blob/main/2/pdf/Bell_En.pdf}

\bibitem{bewegungsenergie}
J. Pascher, \emph{Kinetic Energy in T0}, 2025.
\url{https://github.com/jpascher/T0-Time-Mass-Duality/blob/main/2/pdf/Bewegungsenergie_En.pdf}

\bibitem{emc2}
J. Pascher, \emph{E=mc² in T0 Framework}, 2025.
\url{https://github.com/jpascher/T0-Time-Mass-Duality/blob/main/2/pdf/E-mc2_En.pdf}

\bibitem{formeln_energiebasiert}
J. Pascher, \emph{Energy-Based Formulas}, 2025.
\url{https://github.com/jpascher/T0-Time-Mass-Duality/blob/main/2/pdf/Formeln_Energiebasiert_En.pdf}

\bibitem{hannah}
J. Pascher, \emph{Hannah Document}, 2025.
\url{https://github.com/jpascher/T0-Time-Mass-Duality/blob/main/2/pdf/Hannah_En.pdf}

\bibitem{ho_doc}
J. Pascher, \emph{H0 Analysis}, 2025.
\url{https://github.com/jpascher/T0-Time-Mass-Duality/blob/main/2/pdf/Ho_En.pdf}

\bibitem{markov}
J. Pascher, \emph{Markov Processes in T0}, 2025.
\url{https://github.com/jpascher/T0-Time-Mass-Duality/blob/main/2/pdf/Markov_En.pdf}

\bibitem{elimination_mass}
J. Pascher, \emph{Elimination of Mass}, 2025.
\url{https://github.com/jpascher/T0-Time-Mass-Duality/blob/main/2/pdf/EliminationOfMassEn.pdf}

\bibitem{elimination_mass_dirac}
J. Pascher, \emph{Dirac Equation Mass Elimination}, 2025.
\url{https://github.com/jpascher/T0-Time-Mass-Duality/blob/main/2/pdf/Elimination_Of_Mass_Dirac_TabelleEn.pdf}

\bibitem{feinstrukturkonstante}
J. Pascher, \emph{Fine Structure Constant}, 2025.
\url{https://github.com/jpascher/T0-Time-Mass-Duality/blob/main/2/pdf/FeinstrukturkonstanteEn.pdf}

\bibitem{neutrino_formel}
J. Pascher, \emph{Neutrino Formula}, 2025.
\url{https://github.com/jpascher/T0-Time-Mass-Duality/blob/main/2/pdf/neutrino-Formel_En.pdf}

\bibitem{neutrinos}
J. Pascher, \emph{Neutrinos in T0}, 2025.
\url{https://github.com/jpascher/T0-Time-Mass-Duality/blob/main/2/pdf/T0_Neutrinos_En.pdf}

\bibitem{koide_formel}
J. Pascher, \emph{Koide Formula in T0}, 2025.
\url{https://github.com/jpascher/T0-Time-Mass-Duality/blob/main/2/pdf/T0_koide-formel-3_En.pdf}

\bibitem{teilchenmassen}
J. Pascher, \emph{Particle Masses}, 2025.
\url{https://github.com/jpascher/T0-Time-Mass-Duality/blob/main/2/pdf/Teilchenmassen_En.pdf}

\bibitem{t0_teilchenmassen}
J. Pascher, \emph{T0 Particle Masses}, 2025.
\url{https://github.com/jpascher/T0-Time-Mass-Duality/blob/main/2/pdf/T0_Teilchenmassen_En.pdf}

\bibitem{penrose_doc}
J. Pascher, \emph{Penrose Analysis in T0}, 2025.
\url{https://github.com/jpascher/T0-Time-Mass-Duality/blob/main/2/pdf/T0_penrose_En.pdf}

\bibitem{photonenchip}
J. Pascher, \emph{Photon Chip Implementation}, 2025.
\url{https://github.com/jpascher/T0-Time-Mass-Duality/blob/main/2/pdf/T0_photonenchip-china_En.pdf}

\bibitem{threeclock}
J. Pascher, \emph{Three Clock Experiment}, 2025.
\url{https://github.com/jpascher/T0-Time-Mass-Duality/blob/main/2/pdf/T0_threeclock_En.pdf}

\bibitem{redshift_deflection}
J. Pascher, \emph{Redshift and Deflection}, 2025.
\url{https://github.com/jpascher/T0-Time-Mass-Duality/blob/main/2/pdf/redshift_deflection_En.pdf}

\bibitem{scheinbar_instantan}
J. Pascher, \emph{Apparent Instantaneity}, 2025.
\url{https://github.com/jpascher/T0-Time-Mass-Duality/blob/main/2/pdf/scheinbar_instantan_En.pdf}

\bibitem{universale_ableitung}
J. Pascher, \emph{Universal Derivation}, 2025.
\url{https://github.com/jpascher/T0-Time-Mass-Duality/blob/main/2/pdf/universale-ableitung_En.pdf}

\bibitem{xi_parameter}
J. Pascher, \emph{Xi Parameter for Particles}, 2025.
\url{https://github.com/jpascher/T0-Time-Mass-Duality/blob/main/2/pdf/xi_parmater_partikel_En.pdf}

\bibitem{xi_ursprung}
J. Pascher, \emph{Origin of Xi}, 2025.
\url{https://github.com/jpascher/T0-Time-Mass-Duality/blob/main/2/pdf/T0_xi_ursprung_En.pdf}

\bibitem{zeit}
J. Pascher, \emph{Time in T0 Theory}, 2025.
\url{https://github.com/jpascher/T0-Time-Mass-Duality/blob/main/2/pdf/Zeit_En.pdf}

\bibitem{zeit_konstant}
J. Pascher, \emph{Time Constant}, 2025.
\url{https://github.com/jpascher/T0-Time-Mass-Duality/blob/main/2/pdf/Zeit-konstant_En.pdf}

\bibitem{zusammenfassung}
J. Pascher, \emph{Summary of T0 Theory}, 2025.
\url{https://github.com/jpascher/T0-Time-Mass-Duality/blob/main/2/pdf/Zusammenfassung_En.pdf}

\bibitem{rsa}
J. Pascher, \emph{RSA in T0 Framework}, 2025.
\url{https://github.com/jpascher/T0-Time-Mass-Duality/blob/main/2/pdf/RSA_En.pdf}

\bibitem{qat}
J. Pascher, \emph{Quantum Atomic Theory}, 2025.
\url{https://github.com/jpascher/T0-Time-Mass-Duality/blob/main/2/pdf/T0_QAT_En.pdf}

\bibitem{qm_qft_rt}
J. Pascher, \emph{QM, QFT and RT Unification}, 2025.
\url{https://github.com/jpascher/T0-Time-Mass-Duality/blob/main/2/pdf/T0_QM-QFT-RT_En.pdf}

\bibitem{qm_optimierung}
J. Pascher, \emph{QM Optimization}, 2025.
\url{https://github.com/jpascher/T0-Time-Mass-Duality/blob/main/2/pdf/T0_QM-optimierung_En.pdf}

\bibitem{vollstaendige_berechnungen}
J. Pascher, \emph{Complete Calculations}, 2025.
\url{https://github.com/jpascher/T0-Time-Mass-Duality/blob/main/2/pdf/T0_Vollstaendige_Berchnungen_En.pdf}

\bibitem{synergetics}
J. Pascher, \emph{T0 Theory vs Synergetics}, 2025.
\url{https://github.com/jpascher/T0-Time-Mass-Duality/blob/main/2/pdf/T0-Theory-vs-Synergetics_En.pdf}

\bibitem{modell_uebersicht}
J. Pascher, \emph{T0 Model Overview}, 2025.
\url{https://github.com/jpascher/T0-Time-Mass-Duality/blob/main/2/pdf/T0_Modell_Uebersicht_En.pdf}

\bibitem{mnras_widerlegung}
J. Pascher, \emph{MNRAS Analysis}, 2025.
\url{https://github.com/jpascher/T0-Time-Mass-Duality/blob/main/2/pdf/T0_Analyse_MNRAS_Widerlegung_En.pdf}

\bibitem{anomale_magnetische_momente}
J. Pascher, \emph{Anomalous Magnetic Moments}, 2025.
\url{https://github.com/jpascher/T0-Time-Mass-Duality/blob/main/2/pdf/T0_Anomale_Magnetische_Momente_En.pdf}

\bibitem{sieben_fragen}
J. Pascher, \emph{Seven Questions in T0}, 2025.
\url{https://github.com/jpascher/T0-Time-Mass-Duality/blob/main/2/pdf/T0_7-fragen-3_En.pdf}

\bibitem{detailierte_leptonen}
J. Pascher, \emph{Detailed Lepton Anomaly}, 2025.
\url{https://github.com/jpascher/T0-Time-Mass-Duality/blob/main/2/pdf/detailierte_formel_leptonen_anemal_En.pdf}

\bibitem{parameterherleitung}
J. Pascher, \emph{Parameter Derivation}, 2025.
\url{https://github.com/jpascher/T0-Time-Mass-Duality/blob/main/2/pdf/parameterherleitung_En.pdf}

\bibitem{verhaeltnis_absolut}
J. Pascher, \emph{Absolute Ratios in T0}, 2025.
\url{https://github.com/jpascher/T0-Time-Mass-Duality/blob/main/2/pdf/T0_verhaeltnis-absolut_En.pdf}

\bibitem{xi_und_e}
J. Pascher, \emph{Xi and Energy}, 2025.
\url{https://github.com/jpascher/T0-Time-Mass-Duality/blob/main/2/pdf/T0_xi-und-e_En.pdf}

\bibitem{umkehrung}
J. Pascher, \emph{Inversion in T0}, 2025.
\url{https://github.com/jpascher/T0-Time-Mass-Duality/blob/main/2/pdf/T0_umkehrung_En.pdf}

\bibitem{esm_analysis}
J. Pascher, \emph{T0 vs ESM Conceptual Analysis}, 2025.
\url{https://github.com/jpascher/T0-Time-Mass-Duality/blob/main/2/pdf/T0vsESM_ConceptualAnalysis_En.pdf}

\end{thebibliography}

\end{document}
