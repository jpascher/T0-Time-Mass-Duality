% Standalone-Dokument: QFT_De
% Verwendet gemeinsamen T0-Header
% T0 Standalone Header - German Version
% Gemeinsamer Header für alle deutschen Standalone-Dokumente

\documentclass[12pt,a4paper]{article}
\usepackage[utf8]{inputenc}
\usepackage[T1]{fontenc}
\usepackage[ngerman]{babel}
\usepackage{lmodern}

% Mathematics
\usepackage{amsmath,amssymb,amsthm}
\usepackage{physics}
\usepackage{siunitx}

% Layout
\usepackage[left=2.5cm,right=2.5cm,top=2.5cm,bottom=2.5cm,headheight=15pt]{geometry}
\usepackage{fancyhdr}
\usepackage{titlesec}

% Tables and Graphics
\usepackage{booktabs}
\usepackage{array}
\usepackage{longtable}
\usepackage{graphicx}
\usepackage{tikz}
\usetikzlibrary{arrows.meta,positioning,shapes.geometric}

% Colors and Boxes
\usepackage{xcolor}
\usepackage[most]{tcolorbox}
\usepackage{mdframed}

% Additional packages
\usepackage{enumitem}
\usepackage{float}
\usepackage{caption}
\usepackage{subcaption}
\usepackage{multirow}
\usepackage{colortbl}
\usepackage{pdflscape}
\usepackage{algorithm}
\usepackage{algpseudocode}
\usepackage{listings}
\usepackage{hyperref}

% Define colors
\definecolor{t0blue}{RGB}{0,51,102}
\definecolor{t0green}{RGB}{0,102,51}
\definecolor{t0red}{RGB}{153,0,0}
\definecolor{deepblue}{RGB}{0,51,102}
\definecolor{deepgreen}{RGB}{0,102,51}
\definecolor{deepred}{RGB}{153,0,0}
\definecolor{boxgray}{RGB}{240,240,240}
\definecolor{t0yellow}{RGB}{255,200,0}
\definecolor{boxblue}{RGB}{230,240,255}
\definecolor{boxgreen}{RGB}{230,255,230}
\definecolor{boxorange}{RGB}{255,240,230}
\definecolor{boxyellow}{RGB}{255,255,230}

% Custom tcolorbox environments
\newtcolorbox{fundamental}[1][]{
  colback=blue!5!white,
  colframe=blue!75!black,
  title=#1,
  fonttitle=\bfseries,
  breakable
}

\newtcolorbox{derivation}[1][]{
  colback=green!5!white,
  colframe=green!75!black,
  title=#1,
  fonttitle=\bfseries,
  breakable
}

\newtcolorbox{result}[1][]{
  colback=orange!5!white,
  colframe=orange!75!black,
  title=#1,
  fonttitle=\bfseries,
  breakable
}

\newtcolorbox{summary}[1][]{
  colback=gray!10!white,
  colframe=gray!75!black,
  title=#1,
  fonttitle=\bfseries,
  breakable
}

\newtcolorbox{comparison}[1][]{
  colback=purple!5!white,
  colframe=purple!75!black,
  title=#1,
  fonttitle=\bfseries,
  breakable
}

\newtcolorbox{relation}[1][]{
  colback=cyan!5!white,
  colframe=cyan!75!black,
  title=#1,
  fonttitle=\bfseries,
  breakable
}

\newtcolorbox{principle}[1][]{
  colback=yellow!5!white,
  colframe=yellow!75!black,
  title=#1,
  fonttitle=\bfseries,
  breakable
}

\newtcolorbox{insight}[1][]{colback=blue!5,colframe=t0blue,title={#1},fonttitle=\bfseries,breakable}
\newtcolorbox{discovery}[1][]{colback=green!5,colframe=t0green,title={#1},fonttitle=\bfseries,breakable}
\newtcolorbox{newperspective}[1][]{colback=yellow!5,colframe=orange,title={#1},fonttitle=\bfseries,breakable}
\newtcolorbox{revelation}[1][]{colback=red!5,colframe=t0red,title={#1},fonttitle=\bfseries,breakable}
\newtcolorbox{keypoint}[1][]{colback=blue!5,colframe=t0blue,title={#1},fonttitle=\bfseries,breakable}
\newtcolorbox{evidence}[1][]{colback=green!5,colframe=t0green,title={#1},fonttitle=\bfseries,breakable}
\newtcolorbox{conclusion}[1][]{colback=gray!5,colframe=gray,title={#1},fonttitle=\bfseries,breakable}
\newtcolorbox{significance}[1][]{colback=yellow!5,colframe=orange,title={#1},fonttitle=\bfseries,breakable}
\newtcolorbox{philosophical}[1][]{colback=purple!5,colframe=purple,title={#1},fonttitle=\bfseries,breakable}
\newtcolorbox{implication}[1][]{colback=cyan!5,colframe=cyan,title={#1},fonttitle=\bfseries,breakable}
\newtcolorbox{perspective}[1][]{colback=blue!5,colframe=t0blue,title={#1},fonttitle=\bfseries,breakable}
\newtcolorbox{revolutionary}[1][]{colback=red!5,colframe=t0red,title={#1},fonttitle=\bfseries,breakable}
\newtcolorbox{technical}[1][]{colback=gray!5,colframe=gray!75!black,title={#1},fonttitle=\bfseries,breakable}
\newtcolorbox{notation}[1][]{colback=yellow!5,colframe=yellow!75!black,title={#1},fonttitle=\bfseries,breakable}

% Theorem environments
\newtheorem{theorem}{Satz}[section]
\newtheorem{lemma}[theorem]{Lemma}
\newtheorem{corollary}[theorem]{Korollar}
\newtheorem{proposition}[theorem]{Proposition}
\newtheorem{definition}[theorem]{Definition}
\newtheorem{example}[theorem]{Beispiel}
\newtheorem{remark}[theorem]{Bemerkung}
\newtheorem{note}[theorem]{Anmerkung}

% Additional environments
\newenvironment{treatise}{\begin{quote}}{\end{quote}}
\newenvironment{gemeinsam}{\begin{quote}}{\end{quote}}
\newenvironment{vergleich}{\begin{quote}}{\end{quote}}
\newenvironment{vorteil}{\begin{quote}}{\end{quote}}
\newenvironment{quantum}{\begin{quote}}{\end{quote}}

% T0-specific commands
\newcommand{\Tzero}{T$_0$}
\newcommand{\xipar}{\xi}
\newcommand{\Tfield}{T}
\newcommand{\Efield}{\mathcal{E}}
\newcommand{\meff}{m_{\text{eff}}}
\newcommand{\Eabs}{E_{\text{abs}}}
\newcommand{\taupar}{\tau}

% Header setup
\pagestyle{fancy}
\fancyhf{}
\fancyhead[L]{\leftmark}
\fancyhead[R]{\thepage}
\renewcommand{\headrulewidth}{0.4pt}

% Hyperref setup
\hypersetup{
    colorlinks=true,
    linkcolor=blue,
    filecolor=magenta,
    urlcolor=cyan,
    citecolor=blue,
    pdftitle={T0 Theory Document},
    pdfauthor={Johann Pascher}
}

% German quotation marks
%\newcommand{\dq}[1]{\glqq{}#1\grqq{}}


\title{Quantenfeldtheorie in der T0-Theorie}
\author{Johann Pascher}
\date{2025}

% Dokument-spezifische tcolorbox-Umgebungen
\newtcolorbox{important}[1][]{colback=yellow!10!white,colframe=yellow!50!black,fonttitle=\bfseries,title=Wichtiger Hinweis,#1}
\newtcolorbox{formula}[1][]{colback=blue!5!white,colframe=blue!75!black,fonttitle=\bfseries,title=Zentrale Formel,#1}
\newtcolorbox{experimental}[1][]{colback=green!5!white,colframe=green!75!black,fonttitle=\bfseries,title=Experimentelle Analyse,#1}

\begin{document}

\maketitle

\section{Quantenfeldtheorie in der T0-Theorie}

\begin{abstract}
Diese Arbeit untersucht die Quantenfeldtheorie im Rahmen der T0-Theorie. Das Modell bietet eine alternative Formulierung der QFT basierend auf dem universellen Energiefeld.

\textbf{Kernaussagen:}
\begin{itemize}
\item Felder entstehen aus dem Energiefeld
\item Wechselwirkungen haben geometrischen Ursprung
\item Renormierung wird natürlich
\end{itemize}
\end{abstract}

\section{Grundlagen}\label{QFT:sec:grundlagen}

\subsection{Feldoperatoren}\label{QFT:subsec:operatoren}

Die Feldoperatoren werden mit dem Energiefeld verknüpft:
\begin{equation}
\hat{\phi}(x) = \int \frac{d^3k}{(2\pi)^3} \frac{1}{\sqrt{2E_k}} \left( a_k e^{-ikx} + a_k^\dagger e^{ikx} \right)
\label{QFT:eq:feldoperator}
\end{equation}

\subsection{Propagatoren}\label{QFT:subsec:propagatoren}

Der Feynman-Propagator im T0-Framework:
\begin{equation}
D_F(x-y) = \langle 0 | T\{\phi(x)\phi(y)\} | 0 \rangle
\label{QFT:eq:propagator}
\end{equation}

\section{Wechselwirkungen}\label{QFT:sec:wechselwirkungen}

\subsection{Kopplungen}\label{QFT:subsec:kopplungen}

Wechselwirkungskopplungen entstehen aus der Feldgeometrie:
\begin{equation}
g = \xigeom \cdot g_0
\label{QFT:eq:kopplung}
\end{equation}

\subsection{Streuamplituden}\label{QFT:subsec:amplituden}

Die S-Matrix wird modifiziert:
\begin{equation}
\mathcal{M} = -i\lambda \cdot f(\xigeom)
\label{QFT:eq:amplitude}
\end{equation}

\section{Renormierung}\label{QFT:sec:renormierung}

Im T0-Modell wird die Renormierung natürlich durch die Energiefelddynamik:
\begin{equation}
\Lambda_{\text{cut}} = \frac{1}{\xigeom} \cdot E_P
\label{QFT:eq:cutoff}
\end{equation}

\section{Schlussfolgerungen}\label{QFT:sec:schluss}

Das T0-Modell bietet eine geometrische Grundlage für die Quantenfeldtheorie.


\begin{thebibliography}{99}

\bibitem{pascher2024}
J. Pascher, \emph{T0 Theory: Time-Mass Duality}, 2024.

\bibitem{t0grundlagen}
J. Pascher, \emph{Grundlagen der T0-Theorie}, T0 Theory Collection (2025).

\bibitem{t0kosmologie}
J. Pascher, \emph{T0-Kosmologie: Ein statisches Universum-Modell}, T0 Theory Collection (2025).

\bibitem{parameterherleitung}
J. Pascher, \emph{Parameterherleitung im T0-Modell}, T0 Theory Collection (2025).

\bibitem{teilchenmassen}
J. Pascher, \emph{Teilchenmassen im T0-Modell}, T0 Theory Collection (2025).

\bibitem{feinstruktur}
J. Pascher, \emph{Die Feinstrukturkonstante im T0-Rahmenwerk}, T0 Theory Collection (2025).

\bibitem{pdg2024}
Particle Data Group, \emph{Review of Particle Physics}, 2024.

\bibitem{codata2019}
CODATA, \emph{Recommended Values of Fundamental Constants}, 2019.

\end{thebibliography}

\end{document}
