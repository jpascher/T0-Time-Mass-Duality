% Standalone-Dokument: T0_Teilchenmassen_De
% Verwendet gemeinsamen T0-Header
% T0 Standalone Header - German Version
% Gemeinsamer Header für alle deutschen Standalone-Dokumente

\documentclass[12pt,a4paper]{article}
\usepackage[utf8]{inputenc}
\usepackage[T1]{fontenc}
\usepackage[ngerman]{babel}
\usepackage{lmodern}

% Mathematics
\usepackage{amsmath,amssymb,amsthm}
\usepackage{physics}
\usepackage{siunitx}

% Layout
\usepackage[left=2.5cm,right=2.5cm,top=2.5cm,bottom=2.5cm,headheight=15pt]{geometry}
\usepackage{fancyhdr}
\usepackage{titlesec}

% Tables and Graphics
\usepackage{booktabs}
\usepackage{array}
\usepackage{longtable}
\usepackage{graphicx}
\usepackage{tikz}
\usetikzlibrary{arrows.meta,positioning,shapes.geometric}

% Colors and Boxes
\usepackage{xcolor}
\usepackage[most]{tcolorbox}
\usepackage{mdframed}

% Additional packages
\usepackage{enumitem}
\usepackage{float}
\usepackage{caption}
\usepackage{subcaption}
\usepackage{multirow}
\usepackage{colortbl}
\usepackage{pdflscape}
\usepackage{algorithm}
\usepackage{algpseudocode}
\usepackage{listings}
\usepackage{hyperref}

% Define colors
\definecolor{t0blue}{RGB}{0,51,102}
\definecolor{t0green}{RGB}{0,102,51}
\definecolor{t0red}{RGB}{153,0,0}
\definecolor{deepblue}{RGB}{0,51,102}
\definecolor{deepgreen}{RGB}{0,102,51}
\definecolor{deepred}{RGB}{153,0,0}
\definecolor{boxgray}{RGB}{240,240,240}
\definecolor{t0yellow}{RGB}{255,200,0}
\definecolor{boxblue}{RGB}{230,240,255}
\definecolor{boxgreen}{RGB}{230,255,230}
\definecolor{boxorange}{RGB}{255,240,230}
\definecolor{boxyellow}{RGB}{255,255,230}

% Custom tcolorbox environments
\newtcolorbox{fundamental}[1][]{
  colback=blue!5!white,
  colframe=blue!75!black,
  title=#1,
  fonttitle=\bfseries,
  breakable
}

\newtcolorbox{derivation}[1][]{
  colback=green!5!white,
  colframe=green!75!black,
  title=#1,
  fonttitle=\bfseries,
  breakable
}

\newtcolorbox{result}[1][]{
  colback=orange!5!white,
  colframe=orange!75!black,
  title=#1,
  fonttitle=\bfseries,
  breakable
}

\newtcolorbox{summary}[1][]{
  colback=gray!10!white,
  colframe=gray!75!black,
  title=#1,
  fonttitle=\bfseries,
  breakable
}

\newtcolorbox{comparison}[1][]{
  colback=purple!5!white,
  colframe=purple!75!black,
  title=#1,
  fonttitle=\bfseries,
  breakable
}

\newtcolorbox{relation}[1][]{
  colback=cyan!5!white,
  colframe=cyan!75!black,
  title=#1,
  fonttitle=\bfseries,
  breakable
}

\newtcolorbox{principle}[1][]{
  colback=yellow!5!white,
  colframe=yellow!75!black,
  title=#1,
  fonttitle=\bfseries,
  breakable
}

\newtcolorbox{insight}[1][]{colback=blue!5,colframe=t0blue,title={#1},fonttitle=\bfseries,breakable}
\newtcolorbox{discovery}[1][]{colback=green!5,colframe=t0green,title={#1},fonttitle=\bfseries,breakable}
\newtcolorbox{newperspective}[1][]{colback=yellow!5,colframe=orange,title={#1},fonttitle=\bfseries,breakable}
\newtcolorbox{revelation}[1][]{colback=red!5,colframe=t0red,title={#1},fonttitle=\bfseries,breakable}
\newtcolorbox{keypoint}[1][]{colback=blue!5,colframe=t0blue,title={#1},fonttitle=\bfseries,breakable}
\newtcolorbox{evidence}[1][]{colback=green!5,colframe=t0green,title={#1},fonttitle=\bfseries,breakable}
\newtcolorbox{conclusion}[1][]{colback=gray!5,colframe=gray,title={#1},fonttitle=\bfseries,breakable}
\newtcolorbox{significance}[1][]{colback=yellow!5,colframe=orange,title={#1},fonttitle=\bfseries,breakable}
\newtcolorbox{philosophical}[1][]{colback=purple!5,colframe=purple,title={#1},fonttitle=\bfseries,breakable}
\newtcolorbox{implication}[1][]{colback=cyan!5,colframe=cyan,title={#1},fonttitle=\bfseries,breakable}
\newtcolorbox{perspective}[1][]{colback=blue!5,colframe=t0blue,title={#1},fonttitle=\bfseries,breakable}
\newtcolorbox{revolutionary}[1][]{colback=red!5,colframe=t0red,title={#1},fonttitle=\bfseries,breakable}
\newtcolorbox{technical}[1][]{colback=gray!5,colframe=gray!75!black,title={#1},fonttitle=\bfseries,breakable}
\newtcolorbox{notation}[1][]{colback=yellow!5,colframe=yellow!75!black,title={#1},fonttitle=\bfseries,breakable}

% Theorem environments
\newtheorem{theorem}{Satz}[section]
\newtheorem{lemma}[theorem]{Lemma}
\newtheorem{corollary}[theorem]{Korollar}
\newtheorem{proposition}[theorem]{Proposition}
\newtheorem{definition}[theorem]{Definition}
\newtheorem{example}[theorem]{Beispiel}
\newtheorem{remark}[theorem]{Bemerkung}
\newtheorem{note}[theorem]{Anmerkung}

% Additional environments
\newenvironment{treatise}{\begin{quote}}{\end{quote}}
\newenvironment{gemeinsam}{\begin{quote}}{\end{quote}}
\newenvironment{vergleich}{\begin{quote}}{\end{quote}}
\newenvironment{vorteil}{\begin{quote}}{\end{quote}}
\newenvironment{quantum}{\begin{quote}}{\end{quote}}

% T0-specific commands
\newcommand{\Tzero}{T$_0$}
\newcommand{\xipar}{\xi}
\newcommand{\Tfield}{T}
\newcommand{\Efield}{\mathcal{E}}
\newcommand{\meff}{m_{\text{eff}}}
\newcommand{\Eabs}{E_{\text{abs}}}
\newcommand{\taupar}{\tau}

% Header setup
\pagestyle{fancy}
\fancyhf{}
\fancyhead[L]{\leftmark}
\fancyhead[R]{\thepage}
\renewcommand{\headrulewidth}{0.4pt}

% Hyperref setup
\hypersetup{
    colorlinks=true,
    linkcolor=blue,
    filecolor=magenta,
    urlcolor=cyan,
    citecolor=blue,
    pdftitle={T0 Theory Document},
    pdfauthor={Johann Pascher}
}

% German quotation marks
%\newcommand{\dq}[1]{\glqq{}#1\grqq{}}


\title{Teilchenmassen in der T0-Theorie}
\author{Johann Pascher}
\date{2025}

\begin{document}

\maketitle

\chapter{Teilchenmassen in der T0-Theorie}

\begin{abstract}
Diese Arbeit präsentiert eine umfassende Analyse der Teilchenmassen im Rahmen der T0-Theorie. Im Gegensatz zum Standardmodell, das Teilchenmassen als freie Parameter behandelt, leitet das T0-Modell Massenbeziehungen aus fundamentalen geometrischen Prinzipien ab. Der zentrale Parameter $\xigeom = \frac{4}{3} \times 10^{-4}$ bestimmt die Skalenhierarchie aller bekannten Teilchen.

\textbf{Kernaussagen:}
\begin{itemize}
\item Teilchenmassen entstehen aus der Zeit-Energie-Dualität $T \cdot E = 1$
\item Die Massenspektren folgen präzisen geometrischen Verhältnissen
\item Vorhersagen stimmen mit experimentellen Daten überein
\end{itemize}
\end{abstract}

\section{Grundlagen der Massenberechnung}\label{T0_Teilchenmassen:sec:grundlagen}

\subsection{Masse als emergentes Phänomen}\label{T0_Teilchenmassen:subsec:emergent}

In der T0-Theorie ist Masse keine fundamentale Eigenschaft, sondern eine emergente Größe, die aus den Wechselwirkungen des Energiefeldes entsteht:
\begin{equation}
m = \frac{E}{c^2} = \frac{1}{T \cdot c^2}
\label{T0_Teilchenmassen:eq:masse_energie}
\end{equation}

\textbf{Physikalische Interpretation:} Die Masse eines Teilchens ist invers proportional zu seinem intrinsischen Zeitfeld. Schwere Teilchen haben kürzere charakteristische Zeitskalen.

\subsection{Der geometrische Parameter}\label{T0_Teilchenmassen:subsec:xi_parameter}

Der fundamentale Parameter $\xigeom$ verbindet die Planck-Skala mit den beobachteten Teilchenmassen:
\begin{equation}
\xigeom = \frac{4}{3} \times 10^{-4}
\label{T0_Teilchenmassen:eq:xi_geom}
\end{equation}

Dieser Parameter ist nicht willkürlich, sondern folgt aus der geometrischen Struktur des Energiefeldes.

\section{Leptonenmassen}\label{T0_Teilchenmassen:sec:leptonen}

\subsection{Elektronmasse}\label{T0_Teilchenmassen:subsec:elektron}

Das Elektron als leichtestes geladenes Lepton definiert die fundamentale Massenskala:
\begin{equation}
m_e = \xigeom \cdot m_P
\label{T0_Teilchenmassen:eq:elektron_masse}
\end{equation}

wobei $m_P$ die Planck-Masse ist.

\subsection{Myon und Tau}\label{T0_Teilchenmassen:subsec:myon_tau}

Die schwereren Leptonen folgen einem geometrischen Progressionsmuster:
\begin{align}
m_\mu &\approx 207 \cdot m_e \\
m_\tau &\approx 3477 \cdot m_e
\label{T0_Teilchenmassen:eq:lepton_verhaeltnisse}
\end{align}

Diese Verhältnisse können aus der T0-Geometrie abgeleitet werden.

\section{Quarkmassen}\label{T0_Teilchenmassen:sec:quarks}

Die Quarkmassen zeigen eine komplexere Hierarchie, die ebenfalls dem T0-Framework folgt:
\begin{equation}
\frac{m_t}{m_u} \sim \left(\frac{1}{\xigeom}\right)^n
\label{T0_Teilchenmassen:eq:quark_hierarchie}
\end{equation}

wobei $n$ eine ganzzahlige Potenz ist, die die Generation des Quarks widerspiegelt.

\section{Schlussfolgerungen}\label{T0_Teilchenmassen:sec:schluss}

Das T0-Modell bietet einen einheitlichen Rahmen zum Verständnis der Teilchenmassenhierarchie ohne freie Parameter. Die Vorhersagen stimmen mit den experimentellen Beobachtungen überein und legen nahe, dass Masse ein geometrisches Phänomen ist.

\end{document}
