% Standalone-Dokument: Amper_Low_De
% Verwendet gemeinsamen T0-Header
% T0 Standalone Header - German Version
% Gemeinsamer Header für alle deutschen Standalone-Dokumente

\documentclass[12pt,a4paper]{article}
\usepackage[utf8]{inputenc}
\usepackage[T1]{fontenc}
\usepackage[ngerman]{babel}
\usepackage{lmodern}

% Mathematics
\usepackage{amsmath,amssymb,amsthm}
\usepackage{physics}
\usepackage{siunitx}

% Layout
\usepackage[left=2.5cm,right=2.5cm,top=2.5cm,bottom=2.5cm,headheight=15pt]{geometry}
\usepackage{fancyhdr}
\usepackage{titlesec}

% Tables and Graphics
\usepackage{booktabs}
\usepackage{array}
\usepackage{longtable}
\usepackage{graphicx}
\usepackage{tikz}
\usetikzlibrary{arrows.meta,positioning,shapes.geometric}

% Colors and Boxes
\usepackage{xcolor}
\usepackage[most]{tcolorbox}
\usepackage{mdframed}

% Additional packages
\usepackage{enumitem}
\usepackage{float}
\usepackage{caption}
\usepackage{subcaption}
\usepackage{multirow}
\usepackage{colortbl}
\usepackage{pdflscape}
\usepackage{algorithm}
\usepackage{algpseudocode}
\usepackage{listings}
\usepackage{hyperref}

% Define colors
\definecolor{t0blue}{RGB}{0,51,102}
\definecolor{t0green}{RGB}{0,102,51}
\definecolor{t0red}{RGB}{153,0,0}
\definecolor{deepblue}{RGB}{0,51,102}
\definecolor{deepgreen}{RGB}{0,102,51}
\definecolor{deepred}{RGB}{153,0,0}
\definecolor{boxgray}{RGB}{240,240,240}
\definecolor{t0yellow}{RGB}{255,200,0}
\definecolor{boxblue}{RGB}{230,240,255}
\definecolor{boxgreen}{RGB}{230,255,230}
\definecolor{boxorange}{RGB}{255,240,230}
\definecolor{boxyellow}{RGB}{255,255,230}

% Custom tcolorbox environments
\newtcolorbox{fundamental}[1][]{
  colback=blue!5!white,
  colframe=blue!75!black,
  title=#1,
  fonttitle=\bfseries,
  breakable
}

\newtcolorbox{derivation}[1][]{
  colback=green!5!white,
  colframe=green!75!black,
  title=#1,
  fonttitle=\bfseries,
  breakable
}

\newtcolorbox{result}[1][]{
  colback=orange!5!white,
  colframe=orange!75!black,
  title=#1,
  fonttitle=\bfseries,
  breakable
}

\newtcolorbox{summary}[1][]{
  colback=gray!10!white,
  colframe=gray!75!black,
  title=#1,
  fonttitle=\bfseries,
  breakable
}

\newtcolorbox{comparison}[1][]{
  colback=purple!5!white,
  colframe=purple!75!black,
  title=#1,
  fonttitle=\bfseries,
  breakable
}

\newtcolorbox{relation}[1][]{
  colback=cyan!5!white,
  colframe=cyan!75!black,
  title=#1,
  fonttitle=\bfseries,
  breakable
}

\newtcolorbox{principle}[1][]{
  colback=yellow!5!white,
  colframe=yellow!75!black,
  title=#1,
  fonttitle=\bfseries,
  breakable
}

\newtcolorbox{insight}[1][]{colback=blue!5,colframe=t0blue,title={#1},fonttitle=\bfseries,breakable}
\newtcolorbox{discovery}[1][]{colback=green!5,colframe=t0green,title={#1},fonttitle=\bfseries,breakable}
\newtcolorbox{newperspective}[1][]{colback=yellow!5,colframe=orange,title={#1},fonttitle=\bfseries,breakable}
\newtcolorbox{revelation}[1][]{colback=red!5,colframe=t0red,title={#1},fonttitle=\bfseries,breakable}
\newtcolorbox{keypoint}[1][]{colback=blue!5,colframe=t0blue,title={#1},fonttitle=\bfseries,breakable}
\newtcolorbox{evidence}[1][]{colback=green!5,colframe=t0green,title={#1},fonttitle=\bfseries,breakable}
\newtcolorbox{conclusion}[1][]{colback=gray!5,colframe=gray,title={#1},fonttitle=\bfseries,breakable}
\newtcolorbox{significance}[1][]{colback=yellow!5,colframe=orange,title={#1},fonttitle=\bfseries,breakable}
\newtcolorbox{philosophical}[1][]{colback=purple!5,colframe=purple,title={#1},fonttitle=\bfseries,breakable}
\newtcolorbox{implication}[1][]{colback=cyan!5,colframe=cyan,title={#1},fonttitle=\bfseries,breakable}
\newtcolorbox{perspective}[1][]{colback=blue!5,colframe=t0blue,title={#1},fonttitle=\bfseries,breakable}
\newtcolorbox{revolutionary}[1][]{colback=red!5,colframe=t0red,title={#1},fonttitle=\bfseries,breakable}
\newtcolorbox{technical}[1][]{colback=gray!5,colframe=gray!75!black,title={#1},fonttitle=\bfseries,breakable}
\newtcolorbox{notation}[1][]{colback=yellow!5,colframe=yellow!75!black,title={#1},fonttitle=\bfseries,breakable}

% Theorem environments
\newtheorem{theorem}{Satz}[section]
\newtheorem{lemma}[theorem]{Lemma}
\newtheorem{corollary}[theorem]{Korollar}
\newtheorem{proposition}[theorem]{Proposition}
\newtheorem{definition}[theorem]{Definition}
\newtheorem{example}[theorem]{Beispiel}
\newtheorem{remark}[theorem]{Bemerkung}
\newtheorem{note}[theorem]{Anmerkung}

% Additional environments
\newenvironment{treatise}{\begin{quote}}{\end{quote}}
\newenvironment{gemeinsam}{\begin{quote}}{\end{quote}}
\newenvironment{vergleich}{\begin{quote}}{\end{quote}}
\newenvironment{vorteil}{\begin{quote}}{\end{quote}}
\newenvironment{quantum}{\begin{quote}}{\end{quote}}

% T0-specific commands
\newcommand{\Tzero}{T$_0$}
\newcommand{\xipar}{\xi}
\newcommand{\Tfield}{T}
\newcommand{\Efield}{\mathcal{E}}
\newcommand{\meff}{m_{\text{eff}}}
\newcommand{\Eabs}{E_{\text{abs}}}
\newcommand{\taupar}{\tau}

% Header setup
\pagestyle{fancy}
\fancyhf{}
\fancyhead[L]{\leftmark}
\fancyhead[R]{\thepage}
\renewcommand{\headrulewidth}{0.4pt}

% Hyperref setup
\hypersetup{
    colorlinks=true,
    linkcolor=blue,
    filecolor=magenta,
    urlcolor=cyan,
    citecolor=blue,
    pdftitle={T0 Theory Document},
    pdfauthor={Johann Pascher}
}

% German quotation marks
%\newcommand{\dq}[1]{\glqq{}#1\grqq{}}


\title{Ampere und Niederenergie}
\author{Johann Pascher}
\date{2025}

\begin{document}

\maketitle

\chapter{Ampere und Niederenergie}

\begin{abstract}
Diese Arbeit führt das T0-Modell ein, eine erweiterte klassische Feldtheorie basierend auf dem Prinzip der lokalen Konjugation von Basisgrößen (Zeit--Masse, Länge--Steifigkeit, Energie--Dichte). Diese Konjugation wirkt als fundamentale Nebenbedingung, während die Dynamik der zugehörigen Abweichungen $\sigma_i$ kausalen Wellengleichungen gehorcht. Die Theorie koppelt natürlich elektromagnetische Ströme an die Geometrie des Leiters und erklärt die Existenz longitudinaler Kraftkomponenten, die Ampère-Helix-Anomalie, die nichtlineare $I^4$-Skalierung der Kraft bei hohen Strömen und die fraktale Skalierung $F \propto r^{2D_f - 4}$ ohne Verletzung der Kausalität.
\end{abstract}

\section{Einführung}
Die Maxwell'sche Theorie der Elektrodynamik ist eine der erfolgreichsten Theorien der Physik. Experimentelle Untersuchungen von Kräften zwischen Strömen, insbesondere in komplexen Leitergeometrien, zeigen jedoch systematische Abweichungen, die auf zusätzliche physikalische Mechanismen hindeuten. Beobachtete longitudinale Kraftkomponenten, die nichtlineare Abhängigkeit der Kraftstärke vom Strom und geometrieabhängige Effekte wie die Ampère-Helix-Anomalie können im konventionellen Rahmen nicht vollständig erklärt werden.

Diese Arbeit präsentiert das T0-Modell, einen neuartigen theoretischen Rahmen, der diese Phänomene durch Einführung konjugierter Basisgrößen erklärt.

\section{Das Prinzip der lokalen Konjugation}
\subsection{Fundamentale Nebenbedingungen}
Das T0-Modell postuliert, dass physikalische Basisgrößen an jedem Raumzeit-Punkt $(x,t)$ durch lokale Konjugationsbedingungen verknüpft sind:
\begin{align}
T(x,t) \cdot m(x,t) &= 1 \quad \text{mit } [T] = \text{s}, [m] = 1/\text{s} \\
L(x,t) \cdot \kappa(x,t) &= 1 \quad \text{mit } [L] = \text{m}, [\kappa] = 1/\text{m} \\
E(x,t) \cdot \rho(x,t) &= 1 \quad \text{mit } [E] = \text{J}, [\rho] = 1/\text{J}
\end{align}

Diese Gleichungen sind als \textbf{lokale Nebenbedingungen} zu interpretieren. Eine Änderung einer Größe auf der linken Seite erzwingt eine sofortige, rein lokale Neudefinition der konjugierten Größe auf der rechten Seite, um die Gleichung zu erfüllen.

\subsection{Dynamische Abweichungen}
Um diese Nebenbedingungen dynamisch zu machen, führen wir ein Abweichungsfeld $\sigma_i(x,t)$ für jedes Paar ein:
\begin{align}
T \cdot m &= 1 + \sigma_{Tm} \\
L \cdot \kappa &= 1 + \sigma_{L\kappa} \\
E \cdot \rho &= 1 + \sigma_{E\rho}
\end{align}

Die Dynamik dieser $\sigma$-Felder wird durch eine Wirkung beschrieben, die Abweichungen vom Idealwert $\sigma_i = 0$ bestraft:
\begin{equation}
\mathcal{L}_{\sigma} = \sum_i \left[ \frac{1}{2} (\partial_\mu \sigma_i)(\partial^\mu \sigma_i) - \frac{\mu_i^2}{2} \sigma_i^2 \right]
\end{equation}

\section{Zusammenfassung}
Das T0-Modell bietet einen kausalen Rahmen zur Erklärung experimentell beobachteter Kraftanomalien bei Leiterströmen.

\end{document}
