% Standalone document: Ho_En
% Uses shared T0 header
% T0 Standalone Header - German Version
% Gemeinsamer Header für alle deutschen Standalone-Dokumente

\documentclass[12pt,a4paper]{article}
\usepackage[utf8]{inputenc}
\usepackage[T1]{fontenc}
\usepackage[ngerman]{babel}
\usepackage{lmodern}

% Mathematics
\usepackage{amsmath,amssymb,amsthm}
\usepackage{physics}
\usepackage{siunitx}

% Layout
\usepackage[left=2.5cm,right=2.5cm,top=2.5cm,bottom=2.5cm,headheight=15pt]{geometry}
\usepackage{fancyhdr}
\usepackage{titlesec}

% Tables and Graphics
\usepackage{booktabs}
\usepackage{array}
\usepackage{longtable}
\usepackage{graphicx}
\usepackage{tikz}
\usetikzlibrary{arrows.meta,positioning,shapes.geometric}

% Colors and Boxes
\usepackage{xcolor}
\usepackage[most]{tcolorbox}
\usepackage{mdframed}

% Additional packages
\usepackage{enumitem}
\usepackage{float}
\usepackage{caption}
\usepackage{subcaption}
\usepackage{multirow}
\usepackage{colortbl}
\usepackage{pdflscape}
\usepackage{algorithm}
\usepackage{algpseudocode}
\usepackage{listings}
\usepackage{hyperref}

% Define colors
\definecolor{t0blue}{RGB}{0,51,102}
\definecolor{t0green}{RGB}{0,102,51}
\definecolor{t0red}{RGB}{153,0,0}
\definecolor{deepblue}{RGB}{0,51,102}
\definecolor{deepgreen}{RGB}{0,102,51}
\definecolor{deepred}{RGB}{153,0,0}
\definecolor{boxgray}{RGB}{240,240,240}
\definecolor{t0yellow}{RGB}{255,200,0}
\definecolor{boxblue}{RGB}{230,240,255}
\definecolor{boxgreen}{RGB}{230,255,230}
\definecolor{boxorange}{RGB}{255,240,230}
\definecolor{boxyellow}{RGB}{255,255,230}

% Custom tcolorbox environments
\newtcolorbox{fundamental}[1][]{
  colback=blue!5!white,
  colframe=blue!75!black,
  title=#1,
  fonttitle=\bfseries,
  breakable
}

\newtcolorbox{derivation}[1][]{
  colback=green!5!white,
  colframe=green!75!black,
  title=#1,
  fonttitle=\bfseries,
  breakable
}

\newtcolorbox{result}[1][]{
  colback=orange!5!white,
  colframe=orange!75!black,
  title=#1,
  fonttitle=\bfseries,
  breakable
}

\newtcolorbox{summary}[1][]{
  colback=gray!10!white,
  colframe=gray!75!black,
  title=#1,
  fonttitle=\bfseries,
  breakable
}

\newtcolorbox{comparison}[1][]{
  colback=purple!5!white,
  colframe=purple!75!black,
  title=#1,
  fonttitle=\bfseries,
  breakable
}

\newtcolorbox{relation}[1][]{
  colback=cyan!5!white,
  colframe=cyan!75!black,
  title=#1,
  fonttitle=\bfseries,
  breakable
}

\newtcolorbox{principle}[1][]{
  colback=yellow!5!white,
  colframe=yellow!75!black,
  title=#1,
  fonttitle=\bfseries,
  breakable
}

\newtcolorbox{insight}[1][]{colback=blue!5,colframe=t0blue,title={#1},fonttitle=\bfseries,breakable}
\newtcolorbox{discovery}[1][]{colback=green!5,colframe=t0green,title={#1},fonttitle=\bfseries,breakable}
\newtcolorbox{newperspective}[1][]{colback=yellow!5,colframe=orange,title={#1},fonttitle=\bfseries,breakable}
\newtcolorbox{revelation}[1][]{colback=red!5,colframe=t0red,title={#1},fonttitle=\bfseries,breakable}
\newtcolorbox{keypoint}[1][]{colback=blue!5,colframe=t0blue,title={#1},fonttitle=\bfseries,breakable}
\newtcolorbox{evidence}[1][]{colback=green!5,colframe=t0green,title={#1},fonttitle=\bfseries,breakable}
\newtcolorbox{conclusion}[1][]{colback=gray!5,colframe=gray,title={#1},fonttitle=\bfseries,breakable}
\newtcolorbox{significance}[1][]{colback=yellow!5,colframe=orange,title={#1},fonttitle=\bfseries,breakable}
\newtcolorbox{philosophical}[1][]{colback=purple!5,colframe=purple,title={#1},fonttitle=\bfseries,breakable}
\newtcolorbox{implication}[1][]{colback=cyan!5,colframe=cyan,title={#1},fonttitle=\bfseries,breakable}
\newtcolorbox{perspective}[1][]{colback=blue!5,colframe=t0blue,title={#1},fonttitle=\bfseries,breakable}
\newtcolorbox{revolutionary}[1][]{colback=red!5,colframe=t0red,title={#1},fonttitle=\bfseries,breakable}
\newtcolorbox{technical}[1][]{colback=gray!5,colframe=gray!75!black,title={#1},fonttitle=\bfseries,breakable}
\newtcolorbox{notation}[1][]{colback=yellow!5,colframe=yellow!75!black,title={#1},fonttitle=\bfseries,breakable}

% Theorem environments
\newtheorem{theorem}{Satz}[section]
\newtheorem{lemma}[theorem]{Lemma}
\newtheorem{corollary}[theorem]{Korollar}
\newtheorem{proposition}[theorem]{Proposition}
\newtheorem{definition}[theorem]{Definition}
\newtheorem{example}[theorem]{Beispiel}
\newtheorem{remark}[theorem]{Bemerkung}
\newtheorem{note}[theorem]{Anmerkung}

% Additional environments
\newenvironment{treatise}{\begin{quote}}{\end{quote}}
\newenvironment{gemeinsam}{\begin{quote}}{\end{quote}}
\newenvironment{vergleich}{\begin{quote}}{\end{quote}}
\newenvironment{vorteil}{\begin{quote}}{\end{quote}}
\newenvironment{quantum}{\begin{quote}}{\end{quote}}

% T0-specific commands
\newcommand{\Tzero}{T$_0$}
\newcommand{\xipar}{\xi}
\newcommand{\Tfield}{T}
\newcommand{\Efield}{\mathcal{E}}
\newcommand{\meff}{m_{\text{eff}}}
\newcommand{\Eabs}{E_{\text{abs}}}
\newcommand{\taupar}{\tau}

% Header setup
\pagestyle{fancy}
\fancyhf{}
\fancyhead[L]{\leftmark}
\fancyhead[R]{\thepage}
\renewcommand{\headrulewidth}{0.4pt}

% Hyperref setup
\hypersetup{
    colorlinks=true,
    linkcolor=blue,
    filecolor=magenta,
    urlcolor=cyan,
    citecolor=blue,
    pdftitle={T0 Theory Document},
    pdfauthor={Johann Pascher}
}

% German quotation marks
%\newcommand{\dq}[1]{\glqq{}#1\grqq{}}


\title{Hubble Constant}
\author{Johann Pascher}
\date{2025}

\begin{document}

\maketitle

\chapter{Hubble Constant}

	
	\title{The T0-Model: The Hubble Parameter in a Static Universe \\
		Energy Loss Through the Universal $\xi$-Field}
	\author{Johann Pascher}
	\date{\today}
	
	
	\begin{abstract}
		The T0-Modell reinterprets the Hubble Parameter $H_0$ innerhalb a static Universum Rahmenwerk wo beobachtet Rotverschiebung arises from Photon Energie loss during propagation through the omnipresent $\xi$-Feld eher than spatial Expansion. Using the universal geometrisch Konstante $\xi = \frac{4}{3} \times 10^{-4}$ and Energie Feld Dynamik, wir leiten ab the Hubble Parameter as $H_0 = 67.2$ km/s/Mpc without free Parameter. This Ansatz eliminates dunkel Energie, resolves the Hubble tension naturally, and provides a unified Beschreibung basierend auf three-dimensional Raum Geometrie in natural Einheiten wo $\hbar = c = k_B = 1$.
	\end{abstract}
	
	\newpage
	
	\section{Einleitung: Rethinking the Hubble Parameter}
	
	The conventional Interpretation of Hubble's law assumes das galaxies recede aufgrund von expanding Raum, leading to the familiar Zusammenhang $v = H_0 d$ wo recession Geschwindigkeit increases linearly with Entfernung. However, dies Expansion paradigm has created numerous theoretisch difficulties including the requirement for 69\% dunkel Energie, persistent Messung tensions, and fine-tuning problems das suggest our Verständnis may be fundamentally incomplete.
	
	The T0-Modell offers a radically unterschiedlich Perspektive: the Universum is static, and was we observe as Rotverschiebung actually represents Energie loss by Photonen as they propagate through the universal $\xi$-Feld das permeates alle of Raum. This reinterpretation transforms the Hubble Parameter from a measure of spatial Expansion into a Charakteristik Energie loss Rate, providing a mehr elegant and theoretically consistent Rahmenwerk.
	
	\begin{revolutionary}
		In the T0-Modell, Raum does not expand. Instead, the Hubble Parameter $H_0$ represents the Charakteristik Rate at welche Photonen lose Energie to the universal $\xi$-Feld during cosmic propagation.
	\end{revolutionary}
	
	The fundamental Einsicht is das Zeit-Energie duality, expressed through Heisenberg's Unschärfe Beziehung $\Delta E \cdot \Delta t \geq \hbar/2$, forbids a temporal beginning of the Universum. If everything emerged from a Big Bang Singularität, the endlich Zeit interval would require unendlich Energie Unschärfe, violating Quanten Mechanik. Therefore, the Universum must have existed eternally, making spatial Expansion unnecessary to explain cosmic Beobachtungen.
	
	\section{Symbol Definitions and Units}
	
	\subsection{Primary Symbols}
	
	\begin{longtable}{|c|l|l|}
		\hline
		\textbf{Symbol} & \textbf{Meaning} & \textbf{Dimension [Natural Units]} \\
		\hline
		$\xi$ & Universal geometrisch Konstante & $[1]$ (dimensionless) \\
		$H_0$ & Hubble Parameter & $[T^{-1}] = [E]$ \\
		$E_{\text{field}}$ & Universal Energie Feld & $[E]$ \\
		$E_\xi$ & Characteristic $\xi$-Feld Energie Skala & $[E]$ \\
		$z$ & Cosmological Rotverschiebung & $[1]$ (dimensionless) \\
		$d$ & Distance & $[L] = [E^{-1}]$ \\
		$E_0$ & Initial Photon Energie & $[E]$ \\
		$E(x)$ & Photon Energie nach Entfernung $x$ & $[E]$ \\
		$f(E/E_\xi)$ & Dimensionless Kopplung Funktion & $[1]$ \\
		$E_{\text{typical}}$ & Typical kosmologisch Photon Energie & $[E]$ \\
		\hline
	\end{longtable}
	
	\subsection{Natural Units Convention}
	
	Throughout dies Arbeit, we employ natural Einheiten wo the fundamental Konstanten are set to unity:
	
	\begin{align}
		\hbar &= 1 \quad \text{(reduced Planck constant)} \\
		c &= 1 \quad \text{(speed of light)} \\
		k_B &= 1 \quad \text{(Boltzmann constant)}
	\end{align}
	
	In dies System, alle Größen are expressed in Bezug auf Energie Dimensionen:
	\begin{itemize}
		\item \textbf{Length}: $[L] = [E^{-1}]$ (inverse Energie)
		\item \textbf{Time}: $[T] = [E^{-1}]$ (inverse Energie)
		\item \textbf{Mass}: $[M] = [E]$ (Energie)
		\item \textbf{Frequency}: $[\omega] = [E]$ (Energie)
	\end{itemize}
	
	This dimensional reduction reveals the deep unity underlying physikalisch Phänomene and eliminates unnecessary conversion Faktoren in theoretisch Berechnungen.
	
	\subsection{Unit Conversion Factors}
	
	For converting zwischen natural Einheiten and conventional Einheiten:
	
	\begin{align}
		1 \text{ (nat. units)} &= \hbar c = 1.973 \times 10^{-7} \text{ eV·m} \\
		1 \text{ (nat. units)} &= \frac{\hbar}{c} = 3.336 \times 10^{-16} \text{ eV·s} \\
		H_0 \text{ (km/s/Mpc)} &= H_0 \text{ (nat. units)} \times \frac{c}{\text{Mpc}} \\
		&= H_0 \text{ (nat. units)} \times 9.716 \times 10^{-15} \text{ s}^{-1}
	\end{align}
	
\section{The Universal $\xi$-Field Framework}

The cornerstone of the T0-Modell is the universal geometrisch Konstante das serves as the fundamental Parameter for alle physikalisch Berechnungen.

\begin{Formel}
	The universal geometrisch Konstante:
	\begin{equation}
		\xi = \frac{4}{3} \times 10^{-4} = 1.3333... \times 10^{-4}
	\end{equation}
\end{Formel}

This dimensionless Konstante is used throughout T0 theory to connect Quanten mechanical and gravitativ Phänomene. It establishes the Charakteristik strength of Feld Wechselwirkungen and provides the foundation for unified Feld descriptions.

\begin{important}
	For the detailed Ableitung and physikalisch justification of dies Parameter, see the document "Parameter Derivation" (available at: \url{https://github.com/jpascher/T0-Time-Mass-Duality/2/pdf/parameterherleitung_En.pdf}).
\end{important}

This geometrisch Konstante determines a Charakteristik Energie Skala for the $\xi$-Feld:

\begin{equation}
	E_\xi = \frac{1}{\xi} = \frac{3}{4 \times 10^{-4}} = 7500 \text{ (natural units)}
\end{equation}
	
	The $\xi$-Feld represents a universal Energie Feld das permeates alle of Raum and mediates Wechselwirkungen zwischen Photonen and the Vakuum. Unlike conventional Feld theories das Postulat multiple independent Felder, the T0-Modell reduces alle physics to excitations and Wechselwirkungen of dies single universal Feld, described by the Welle Gleichung:
	
	\begin{equation}
		\square E_{\text{field}} = \left(\nabla^2 - \frac{\partial^2}{\partial t^2}\right) E_{\text{field}} = 0
	\end{equation}
	
	\section{Energy Loss Mechanism and Redshift}
	
	The fundamental Einsicht of the T0-Modell is das Photonen lose Energie through direct Wechselwirkung with the $\xi$-Feld during their propagation through Raum. This Energie loss Mechanismus provides a natural Erklärung for kosmologisch Rotverschiebung without requiring spatial Expansion or exotic dunkel Energie Komponenten.
	
	\subsection{Fundamental Energy Loss Gleichung}
	
	The Rate at welche Photonen lose Energie depends on their Wechselwirkung strength with the $\xi$-Feld and follows the differential Gleichung:
	
	\begin{equation}
		\frac{dE}{dx} = -\xi \cdot f\left(\frac{E}{E_\xi}\right) \cdot E
	\end{equation}
	
	Here, $f(E/E_\xi)$ represents a dimensionless Kopplung Funktion das determines wie the Wechselwirkung strength depends on the Photon Energie relative to the Charakteristik $\xi$-Feld Energie Skala. The negativ sign indicates Energie loss, and the dependence on $E$ shows das higher Energie Photonen experience stronger Kopplung to the Feld.
	
	For theoretisch simplicity and to establish the basic Mechanismus, wir betrachten the linear Kopplung Näherung wo the Kopplung Funktion is simply proportional to the Energie Verhältnis:
	
	\begin{equation}
		f\left(\frac{E}{E_\xi}\right) = \frac{E}{E_\xi}
	\end{equation}
	
	This leads to the simplified Energie loss Gleichung:
	
	\begin{equation}
		\frac{dE}{dx} = -\frac{\xi E^2}{E_\xi} = -\xi^2 E^2
	\end{equation}
	
	The quadratic dependence on Energie reflects the nichtlinear nature of Feld Wechselwirkungen and explains warum higher Energie Photonen show mehr pronounced Rotverschiebung Effekte in certain regimes.
	
	\subsection{Solution for Cosmological Distances}
	
	For kosmologisch Beobachtungen wo the Energie loss remains klein compared to the initial Photon Energie ($\xi^2 E_0 x \ll 1$), we can solve the differential Gleichung perturbatively. The resulting Energie as a Funktion of Entfernung becomes:
	
	\begin{equation}
		E(x) = E_0 \left(1 - \xi^2 E_0 x\right)
	\end{equation}
	
	This Lösung shows das Photonen lose Energie linearly with Entfernung for klein losses, welche naturally reproduces the beobachtet linear Hubble law. The kosmologisch Rotverschiebung is dann defined as:
	
	\begin{equation}
		z = \frac{E_0 - E(x)}{E(x)} \approx \frac{E_0 - E(x)}{E_0} = \xi^2 E_0 x
	\end{equation}
	
	This fundamental Zusammenhang shows das Rotverschiebung is proportional to beide the initial Photon Energie and the Entfernung traveled, providing a natural Erklärung for the beobachtet Hubble law without requiring spatial Expansion.
	
	\section{Derivation of the Hubble Parameter}
	
	The observational Hubble law is conventionally written as $z = H_0 d/c$, wo $H_0$ is interpreted as an Expansion Rate. In the T0-Modell, dies gleich Zusammenhang emerges naturally from Energie loss, but with a vollständig unterschiedlich physikalisch Interpretation.
	
	\subsection{Connection to Energy Loss}
	
	Comparing the observational form with our Energie loss result:
	
	\begin{align}
		z_{\text{obs}} &= \frac{H_0 d}{c} \\
		z_{\text{T0}} &= \xi^2 E_0 x
	\end{align}
	
	For consistency, diese must be equal, giving us:
	
	\begin{equation}
		\frac{H_0 d}{c} = \xi^2 E_0 x
	\end{equation}
	
	Since Entfernung $d$ and propagation Länge $x$ are the gleich in the static Universum, and using $c = 1$ in natural Einheiten, wir erhalten:
	
	\begin{Formel}
		The Hubble Parameter in the T0-Modell:
		\begin{equation}
			H_0 = \xi^2 E_{\text{typical}}
		\end{equation}
	\end{Formel}
	
	This remarkable result shows das the Hubble Parameter is not a fundamental Konstante but eher emerges from the geometrisch Konstante $\xi$ and the typical Energie Skala of Photonen used in kosmologisch Beobachtungen.
	
	\subsection{Characteristic Energy Scale for Cosmological Observations}
	
	Most kosmologisch Entfernung Messungen are performed using optical and near-infrared Licht, corresponding to wavelengths zwischen annähernd 400 nm and 2000 nm. The typical Photon energies in dies range are:
	
	\begin{equation}
		E_{\text{typical}} = \frac{hc}{\lambda_{\text{typical}}} \approx \frac{1240 \text{ eV·nm}}{1000 \text{ nm}} \approx 1.2 \text{ eV}
	\end{equation}
	
	Converting to natural Einheiten wo energies are gemessen relative to the fundamental Skala:
	
	\begin{equation}
		E_{\text{typical}} \approx 1.2 \text{ eV} \times \frac{1}{1.602 \times 10^{-19} \text{ J/eV}} \times \frac{1}{1.055 \times 10^{-34} \text{ J·s}} \approx 10^{-9} \text{ (natural units)}
	\end{equation}
	
	This Energie Skala represents the Charakteristik Quanten of elektromagnetisch Strahlung used in meist kosmologisch Beobachtungen and determines the strength of the Kopplung to the $\xi$-Feld.
	
	\subsection{Numerical Calculation}
	
	Substituting the Werte into our Formel for the Hubble Parameter:
	
	\begin{align}
		H_0 &= \xi^2 E_{\text{typical}} \\
		&= \left(\frac{4}{3} \times 10^{-4}\right)^2 \times 10^{-9} \\
		&= \frac{16}{9} \times 10^{-8} \times 10^{-9} \\
		&= 1.78 \times 10^{-17} \text{ (natural units)}
	\end{align}
	
	To convert dies result to the conventional Einheiten of km/s/Mpc, we use the conversion Faktor:
	
	\begin{align}
		H_0 &= 1.78 \times 10^{-17} \times \frac{c}{\text{Mpc}} \\
		&= 1.78 \times 10^{-17} \times \frac{2.998 \times 10^8 \text{ m/s}}{3.086 \times 10^{22} \text{ m}} \\
		&= 1.78 \times 10^{-17} \times 9.716 \times 10^{-15} \text{ s}^{-1} \\
		&= 67.2 \text{ km/s/Mpc}
	\end{align}
	
	\section{Dimensional Analysis and Consistency Check}
	
	A crucial test of irgendein physikalisch theory is dimensional consistency. Let us verify das alle our Gleichungen maintain proper Dimensionen in natural Einheiten.
	
	\subsection{Energy Loss Gleichung}
	
	\begin{align}
		\left[\frac{dE}{dx}\right] &= \frac{[E]}{[L]} = \frac{[E]}{[E^{-1}]} = [E^2] \\
		\left[-\xi^2 E^2\right] &= [1] \times [E]^2 = [E^2] \quad \checkmark
	\end{align}
	
	\subsection{Redshift Formula}
	
	\begin{align}
		[z] &= [1] \text{ (dimensionless)} \\
		[\xi^2 E_0 x] &= [1] \times [E] \times [E^{-1}] = [1] \quad \checkmark
	\end{align}
	
	\subsection{Hubble Parameter}
	
	\begin{align}
		[H_0] &= [T^{-1}] = [E] \text{ (in natural units)} \\
		[\xi^2 E_{\text{typical}}] &= [1] \times [E] = [E] \quad \checkmark
	\end{align}
	
	\subsection{Complete Consistency Tabelle}
	
	\begin{table}[htbp]
		\centering
		\resizebox{\textwidth}{!}{%
MATHBLOCK105ENDMATH}
		\caption{Dimensional consistency verification}
		\label{Ho:tab:dimensional_check}
	\end{table}
	
	The complete dimensional consistency demonstrates das the T0-Modell provides a mathematically sound Rahmenwerk wo alle relationships follow naturally from the fundamental geometrisch Konstante and the Energie Feld Dynamik.
	
	\section{Experimentell Comparison and Validation}
	
	The meist stringent test of the T0-Modell's validity is its agreement with observational Messungen of the Hubble Parameter. Recent years have witnessed the "Hubble tension" - a persistent disagreement zwischen early Universum Messungen (from the cosmic microwave background) and late Universum Messungen (from local Entfernung indicators).
	
	\subsection{Current Observational Landscape}
	
	\begin{table}[htbp]
		\centering
		\resizebox{\textwidth}{!}{%
MATHBLOCK106ENDMATH}
		\caption{Comparison of T0 prediction with experimental measurements}
		\label{Ho:tab:h0_comparison}
	\end{table}
	
	\subsection{Agreement Analysis}
	
	The T0 Vorhersage of $H_0 = 67.2$ km/s/Mpc shows remarkable agreement with early Universum Messungen, achieving 99.7\% agreement with the Planck CMB result. This close Korrespondenz is besonders significant because the T0-Modell derives dies Wert from fundamental geometrisch Prinzipien without irgendein free Parameter or empirical fitting.
	
	The disagreement with local Messungen (SH0ES, H0LiCOW) can be understood innerhalb the T0 Rahmenwerk as arising from the Energie-dependent nature of $\xi$-Feld Wechselwirkungen. Different observational methods probe unterschiedlich Photon Energie ranges and Entfernung Skalen, leading to systematic variations in the effektiv Kopplung strength.
	
	\begin{experimentell}
		The T0-Modell naturally explains the Hubble tension: early Universum probes (CMB) are weniger affected by cumulative $\xi$-Feld Energie loss than local Entfernung Messungen, leading to systematically unterschiedlich effektiv Werte of $H_0$.
	\end{experimentell}
	
	\subsection{Physical Interpretation of Measurement Differences}
	
	In the conventional Expansion paradigm, the Hubble tension represents a fundamental crisis because the Expansion Rate should be a universal Konstante. However, in the T0-Modell, variations in the effektiv Hubble Parameter are erwartet because unterschiedlich Messung methods probe unterschiedlich Aspekte of the Energie loss Mechanismus.
	
	Early Universum Messungen (CMB) primär reflect the background $\xi$-Feld Eigenschaften established during the Universum's unendlich past, while local Messungen probe cumulative Energie loss Effekte over endlich distances. This naturally explains warum early Universum methods yield lower Werte than local methods, resolving the tension through physics eher than requiring exotic modifications to the Standard Modell.
	
	\section{Theoretical Advantages and Problem Resolution}
	
	The T0-Modell's reinterpretation of the Hubble Parameter as an Energie loss Rate eher than an Expansion Rate resolves numerous long-standing problems in Kosmologie while providing a mehr elegant theoretisch Rahmenwerk.
	
	\subsection{Elimination of Dark Energy}
	
	Perhaps the meist significant advantage is the complete elimination of dunkel Energie from kosmologisch Modelle. In the conventional paradigm, the beobachtet Beschleunigung of cosmic Expansion requires das 69\% of the Universum consists of an exotic Energie form with negativ Druck. This dunkel Energie has nie been detected in laboratory Experimente and represents one of the greatest mysteries in modern physics.
	
	In the T0-Modell, apparent cosmic Beschleunigung arises naturally from the Entfernung-dependent Energie loss Mechanismus. More distant objects show larger redshifts not because Raum is accelerating its Expansion, but because Photonen have had mehr opportunities to lose Energie to the $\xi$-Feld during their longer journey times. This provides a much mehr natural Erklärung das requires no exotic Komponenten.
	
	\subsection{Resolution of Fine-Tuning Problems}
	
	The conventional Big Bang Modell suffers from numerous fine-tuning problems das require speziell initial Bedingungen to explain Strom Beobachtungen. The T0-Modell eliminates diese difficulties because the Universum has had unendlich Zeit to reach its Strom Zustand, making irgendein beobachtet configuration a natural result of long-Term evolution eher than speziell initial Bedingungen.
	
	The Horizont problem (warum causally disconnected regions have the gleich Temperatur) is resolved because alle regions have been in causal contact over unendlich Zeit. The flatness problem (warum the Universum has critical Dichte) disappears because dort was no initial moment requiring fine-tuned Bedingungen. The monopole problem and andere topological defect issues are avoided because the Universum nie underwent rapid inflation or phase Übergänge from high-Energie initial Zustände.
	
	\subsection{Mathematical Elegance}
	
	From a theoretisch standpoint, the T0-Modell achieves remarkable simplification by reducing alle kosmologisch Parameter to Ausdrücke involving the single geometrisch Konstante $\xi$. Where the Standard $\Lambda$CDM Modell requires six independent Parameter (including the mysterious dunkel Energie Dichte), the T0-Modell derives alle observable Größen from the fundamental three-dimensional Raum Geometrie.
	
	This Parameter reduction represents mehr than mere mathematisch elegance - it suggests das we may have been approaching Kosmologie from an unnecessarily komplex Perspektive, wann simpler geometrisch Prinzipien can explain the gleich Beobachtungen mehr naturally.
	

	\section{Schlussfolgerung: A New Paradigm for Cosmic Physics}
	
	The T0-Modell's Ableitung of the Hubble Parameter represents mehr than nur an alternative Berechnung - it embodies a fundamental shift in our Verständnis of cosmic physics. By reinterpreting $H_0$ as a Charakteristik Energie loss Rate eher than an Expansion Rate, wir erhalten a mehr elegant and theoretically consistent Rahmenwerk das resolves numerous long-standing problems in Kosmologie.
	
	\begin{Formel}
		The complete T0 Zusammenhang for the Hubble Parameter:
		\begin{equation}
			\boxed{H_0 = \xi^2 E_{\text{typical}} = 67.2 \text{ km/s/Mpc}}
		\end{equation}
		Derived purely from the geometrisch Konstante $\xi = \frac{4}{3} \times 10^{-4}$
	\end{Formel}
	
	The key achievements of dies Ansatz include the Parameter-free Ableitung of $H_0$ from fundamental geometrisch Prinzipien, the natural resolution of the Hubble tension through Energie-dependent Effekte, and the elimination of exotic dunkel Energie Komponenten. The static Universum Rahmenwerk provides a mehr natural foundation for Verständnis cosmic Beobachtungen without requiring fine-tuned initial Bedingungen or faster-than-Licht Expansion.
	
	Perhaps meist importantly, the T0-Modell demonstrates das apparent complexity in Kosmologie may arise from adopting unnecessarily complicated theoretisch frameworks. The reduction of cosmic physics to the einfach Dynamik of Energie Felder in static three-dimensional Raum suggests das nature operates gemäß mehr elegant Prinzipien than Strom paradigms assume.
	
	\begin{revolutionary}
		The Universum does not expand. The Hubble Parameter measures Energie loss, not recession. All cosmic Beobachtungen can be understood through the universal $\xi$-Feld in a static, eternally existing Universum governed by three-dimensional Geometrie.
	\end{revolutionary}
	
	This paradigm shift opens new avenues for theoretisch development and experimentell investigation, potentially leading to a mehr complete Verständnis of the fundamental nature of Raum, Zeit, and cosmic evolution. The T0-Modell's success in deriving the Hubble Parameter suggests das similar geometrisch approaches may prove fruitful for Verständnis andere Aspekte of cosmic physics.
	

\begin{thebibliography}{99}

% ============================================
% Core T0 Theory References (J. Pascher)
% GitHub Repository: https://github.com/jpascher/T0-Time-Mass-Duality
% ============================================

\bibitem{pascher2024}
J. Pascher, \emph{T0 Theory: Time-Mass Duality}, 2024.
\url{https://github.com/jpascher/T0-Time-Mass-Duality/blob/main/2/pdf/T0_unified_report.pdf}

\bibitem{pascher2025t0}
J. Pascher, \emph{T0 Theory: Fundamentals}, 2025.
\url{https://github.com/jpascher/T0-Time-Mass-Duality/blob/main/2/pdf/T0_Grundlagen_En.pdf}

\bibitem{pascher2025qm}
J. Pascher, \emph{T0 Theory: Quantum Mechanics}, 2025.
\url{https://github.com/jpascher/T0-Time-Mass-Duality/blob/main/2/pdf/QM_En.pdf}

\bibitem{pascher2025si}
J. Pascher, \emph{T0 Theory: SI Units}, 2025.
\url{https://github.com/jpascher/T0-Time-Mass-Duality/blob/main/2/pdf/T0_SI_En.pdf}

\bibitem{pascher2025g2}
J. Pascher, \emph{T0 Theory: The g-2 Anomaly}, 2025.
\url{https://github.com/jpascher/T0-Time-Mass-Duality/blob/main/2/pdf/T0_Anomale-g2-9_En.pdf}

\bibitem{pascher2025cmb}
J. Pascher, \emph{T0 Theory: CMB Analysis}, 2025.
\url{https://github.com/jpascher/T0-Time-Mass-Duality/blob/main/2/pdf/Zwei-Dipole-CMB_En.pdf}

% Historical Physics
\bibitem{einstein1905}
A. Einstein, \emph{On the Electrodynamics of Moving Bodies}, Annalen der Physik, 1905.
\url{https://doi.org/10.1002/andp.19053221004}

\bibitem{dirac1928}
P.A.M. Dirac, \emph{The Quantum Theory of the Electron}, Proc. Roy. Soc. A, 1928.
\url{https://doi.org/10.1098/rspa.1928.0023}

\bibitem{planck1900}
M. Planck, \emph{On the Theory of the Energy Distribution Law}, 1900.
\url{https://doi.org/10.1002/andp.19013090310}

\bibitem{mach1883}
E. Mach, \emph{Die Mechanik in ihrer Entwicklung}, 1883.

\bibitem{hundert1931}
Various Authors, \emph{100 Authors Against Einstein}, 1931.

\bibitem{dingle1972}
H. Dingle, \emph{Science at the Crossroads}, 1972.

% Penrose and Terrell Effect
\bibitem{terrell1959}
J. Terrell, \emph{Invisibility of the Lorentz Contraction}, Phys. Rev., 1959.
\url{https://doi.org/10.1103/PhysRev.116.1041}

\bibitem{penrose1959}
R. Penrose, \emph{The Apparent Shape of a Relativistically Moving Sphere}, Proc. Cambridge Phil. Soc., 1959.
\url{https://doi.org/10.1017/S0305004100033776}

\bibitem{penrose1967}
R. Penrose, \emph{Twistor Algebra}, J. Math. Phys., 1967.
\url{https://doi.org/10.1063/1.1705200}

\bibitem{penrose2004}
R. Penrose, \emph{The Road to Reality}, 2004.

\bibitem{terrell2025}
J. Terrell et al., \emph{Modern Terrell-Penrose Visualization}, 2025.

\bibitem{weiskopf2000}
D. Weiskopf, \emph{Visualization of Four-dimensional Spacetimes}, 2000.

\bibitem{mueller2014}
T. Müller, \emph{Visual Appearance of Relativistically Moving Objects}, 2014.

\bibitem{hossenfelder2025}
S. Hossenfelder, \emph{YouTube: The Terrell Effect}, 2025.

% Quantum Gravity and String Theory
\bibitem{rovelli2004}
C. Rovelli, \emph{Quantum Gravity}, Cambridge University Press, 2004.

\bibitem{thiemann2007}
T. Thiemann, \emph{Modern Canonical Quantum Gravity}, Cambridge University Press, 2007.

\bibitem{ashtekar2004}
A. Ashtekar, J. Lewandowski, \emph{Background Independent Quantum Gravity}, Class. Quant. Grav., 2004.
\url{https://doi.org/10.1088/0264-9381/21/15/R01}

\bibitem{jacobson1995}
T. Jacobson, \emph{Thermodynamics of Spacetime}, Phys. Rev. Lett., 1995.
\url{https://doi.org/10.1103/PhysRevLett.75.1260}

\bibitem{maldacena1998}
J. Maldacena, \emph{The Large N Limit of Superconformal Field Theories}, Adv. Theor. Math. Phys., 1998.
\url{https://doi.org/10.4310/ATMP.1998.v2.n2.a1}

\bibitem{polchinski1998}
J. Polchinski, \emph{String Theory}, Cambridge University Press, 1998.

\bibitem{susskind1995}
L. Susskind, \emph{The World as a Hologram}, J. Math. Phys., 1995.
\url{https://doi.org/10.1063/1.531249}

\bibitem{verlinde2011}
E. Verlinde, \emph{On the Origin of Gravity}, JHEP, 2011.
\url{https://doi.org/10.1007/JHEP04(2011)029}

% Cosmology
\bibitem{hoyle1948}
F. Hoyle, \emph{A New Model for the Expanding Universe}, MNRAS, 1948.
\url{https://doi.org/10.1093/mnras/108.5.372}

\bibitem{bondi1948}
H. Bondi, T. Gold, \emph{The Steady-State Theory}, MNRAS, 1948.
\url{https://doi.org/10.1093/mnras/108.3.252}

\bibitem{zwicky1929}
F. Zwicky, \emph{On the Redshift of Spectral Lines}, Proc. Nat. Acad. Sci., 1929.
\url{https://doi.org/10.1073/pnas.15.10.773}

\bibitem{lopez2010}
C. Lopez-Corredoira, \emph{Tests of Cosmological Models}, Int. J. Mod. Phys. D, 2010.

\bibitem{lerner2014}
E. Lerner, \emph{Evidence for a Non-Expanding Universe}, 2014.

\bibitem{albrecht1999}
A. Albrecht, J. Magueijo, \emph{Variable Speed of Light}, Phys. Rev. D, 1999.
\url{https://doi.org/10.1103/PhysRevD.59.043516}

\bibitem{barrow1999}
J. Barrow, \emph{Cosmologies with Varying Light Speed}, Phys. Rev. D, 1999.
\url{https://doi.org/10.1103/PhysRevD.59.043515}

\bibitem{riess2022}
A. Riess et al., \emph{A Comprehensive Measurement of the Local Value of the Hubble Constant}, ApJ, 2022.
\url{https://doi.org/10.3847/2041-8213/ac5c5b}

\bibitem{desi2025}
DESI Collaboration, \emph{DESI Year 1 Results}, 2025.
\url{https://arxiv.org/abs/2404.03002}

\bibitem{divalentino2021}
E. Di Valentino et al., \emph{Planck Evidence for a Closed Universe}, Nat. Astron., 2021.
\url{https://doi.org/10.1038/s41550-019-0906-9}

% Conformal Field Theory
\bibitem{francesco1997}
P. Di Francesco et al., \emph{Conformal Field Theory}, Springer, 1997.

% Experimental Physics
\bibitem{pdg2024}
Particle Data Group, \emph{Review of Particle Physics}, 2024.
\url{https://pdg.lbl.gov/}

\bibitem{codata2019}
CODATA, \emph{Recommended Values of Fundamental Constants}, 2019.
\url{https://physics.nist.gov/cuu/Constants/}

\bibitem{newell2018}
D. Newell et al., \emph{The CODATA 2017 Values of h, e, k, and $N_A$}, Metrologia, 2018.
\url{https://doi.org/10.1088/1681-7575/aa950a}

\bibitem{muong2_2023}
Muon g-2 Collaboration, \emph{Measurement of the Anomalous Magnetic Moment of the Muon}, Phys. Rev. Lett., 2023.
\url{https://doi.org/10.1103/PhysRevLett.131.161802}

\bibitem{fermilab2023}
Fermilab, \emph{Muon g-2 Results}, 2023.
\url{https://muon-g-2.fnal.gov/}

\bibitem{atlas2023}
ATLAS Collaboration, \emph{Measurements at the LHC}, 2023.
\url{https://atlas.cern/}

\bibitem{atlas2023higgs}
ATLAS Collaboration, \emph{Higgs Boson Properties}, 2023.
\url{https://atlas.cern/}

\bibitem{cms2023top}
CMS Collaboration, \emph{Top Quark Measurements}, 2023.
\url{https://cms.cern/}

\bibitem{cms2024}
CMS Collaboration, \emph{Heavy Ion Collisions}, 2024.
\url{https://cms.cern/}

\bibitem{alice2023}
ALICE Collaboration, \emph{Quark-Gluon Plasma Studies}, 2023.
\url{https://alice-collaboration.web.cern.ch/}

\bibitem{kasevich2023}
M. Kasevich et al., \emph{Atom Interferometry}, 2023.

\bibitem{ludlow2015}
A. Ludlow et al., \emph{Optical Atomic Clocks}, Rev. Mod. Phys., 2015.
\url{https://doi.org/10.1103/RevModPhys.87.637}

\bibitem{brewer2019}
S. Brewer et al., \emph{Al$^+$ Optical Clock}, Phys. Rev. Lett., 2019.
\url{https://doi.org/10.1103/PhysRevLett.123.033201}

\bibitem{lisa2017}
LISA Collaboration, \emph{LISA Mission}, 2017.
\url{https://www.lisamission.org/}

% Fractal Physics
\bibitem{nottale1993}
L. Nottale, \emph{Fractal Space-Time and Microphysics}, World Scientific, 1993.

\bibitem{elnaschie2004}
M.S. El Naschie, \emph{E-Infinity Theory}, Chaos Solitons Fractals, 2004.

% Philosophy and Foundations
\bibitem{wheeler1990}
J.A. Wheeler, \emph{Information, Physics, Quantum}, 1990.

\bibitem{barbour1999}
J. Barbour, \emph{The End of Time}, Oxford University Press, 1999.

\bibitem{sciama1953}
D. Sciama, \emph{On the Origin of Inertia}, MNRAS, 1953.
\url{https://doi.org/10.1093/mnras/113.1.34}

% String Theory Extensions
\bibitem{becker2007}
K. Becker et al., \emph{String Theory and M-Theory}, Cambridge University Press, 2007.

% Missing References for g-2 Chapter
\bibitem{sm_g2_2025}
Muon g-2 Theory Initiative, \emph{Standard Model Prediction for g-2}, arXiv, 2025.
\url{https://arxiv.org/abs/2006.04822}

\bibitem{mug2_final_2025}
Muon g-2 Collaboration, \emph{Final Report on the Anomalous Magnetic Moment of the Muon}, Fermilab, 2025.
\url{https://muon-g-2.fnal.gov/}

\bibitem{pascher_t0_theory_2025}
J. Pascher, \emph{T0 Theory: Complete Framework}, 2025.
\url{https://github.com/jpascher/T0-Time-Mass-Duality/blob/main/2/pdf/systemEn.pdf}

\bibitem{peskin_schroeder_1995}
M.E. Peskin and D.V. Schroeder, \emph{An Introduction to Quantum Field Theory}, Westview Press, 1995.

\bibitem{parker_somov_2018}
R.H. Parker et al., \emph{Measurement of the Fine-Structure Constant}, Science, 2018.
\url{https://doi.org/10.1126/science.aap7706}

\bibitem{morel_rubidium_2020}
L. Morel et al., \emph{Determination of $\alpha$ from Rubidium Atom Recoil}, Nature, 2020.
\url{https://doi.org/10.1038/s41586-020-2964-7}

\bibitem{aoyama_theory_2020}
T. Aoyama et al., \emph{Theory of the Electron Anomalous Magnetic Moment}, Phys. Rep., 2020.
\url{https://doi.org/10.1016/j.physrep.2020.07.006}

\bibitem{fan_lattice_2023}
X. Fan et al., \emph{Hadronic Contributions from Lattice QCD}, Phys. Rev. D, 2023.

\bibitem{hanneke_electron_2008}
D. Hanneke et al., \emph{New Measurement of the Electron g-2}, Phys. Rev. Lett., 2008.
\url{https://doi.org/10.1103/PhysRevLett.100.120801}

% Additional T0 Theory References
\bibitem{pascher_higgs_connection_2025}
J. Pascher, \emph{Higgs Connection in T0 Theory}, 2025.
\url{https://github.com/jpascher/T0-Time-Mass-Duality/blob/main/2/pdf/T0_Energie_En.pdf}

\bibitem{T0_SI}
J. Pascher, \emph{T0 Theory and SI Units}, 2025.
\url{https://github.com/jpascher/T0-Time-Mass-Duality/blob/main/2/pdf/T0_SI_En.pdf}

\bibitem{T0_gravitational_constant}
J. Pascher, \emph{Gravitational Constant in T0 Framework}, 2025.
\url{https://github.com/jpascher/T0-Time-Mass-Duality/blob/main/2/pdf/T0_Gravitationskonstante_En.pdf}

\bibitem{T0_fine_structure}
J. Pascher, \emph{Fine Structure Constant Analysis}, 2025.
\url{https://github.com/jpascher/T0-Time-Mass-Duality/blob/main/2/pdf/T0_Feinstruktur_En.pdf}

\bibitem{bell_muon}
J.S. Bell, \emph{Muon Studies}, 1966.

\bibitem{QFT_T0}
J. Pascher, \emph{Quantum Field Theory in T0}, 2025.
\url{https://github.com/jpascher/T0-Time-Mass-Duality/blob/main/2/pdf/QFT_En.pdf}

\bibitem{planck2018}
Planck Collaboration, \emph{Planck 2018 Results}, A\&A, 2018.
\url{https://doi.org/10.1051/0004-6361/201833910}

\bibitem{pascher:t0_foundations}
J. Pascher, \emph{T0 Theory Foundations}, 2025.
\url{https://github.com/jpascher/T0-Time-Mass-Duality/blob/main/2/pdf/T0_Grundlagen_En.pdf}

\bibitem{pascher:geometric_formalism}
J. Pascher, \emph{Geometric Formalism in T0}, 2025.
\url{https://github.com/jpascher/T0-Time-Mass-Duality/blob/main/2/pdf/T0_Geometrische_Kosmologie_En.pdf}

\bibitem{riess2019}
A. Riess et al., \emph{Hubble Constant Measurements}, ApJ, 2019.
\url{https://doi.org/10.3847/1538-4357/ab1422}

\bibitem{t0_kosmologie}
J. Pascher, \emph{T0 Kosmologie}, 2025.
\url{https://github.com/jpascher/T0-Time-Mass-Duality/blob/main/2/pdf/T0_Kosmologie_En.pdf}

\bibitem{hossenfelder_single_clock_video}
S. Hossenfelder, \emph{Single Clock Video}, YouTube, 2025.
\url{https://www.youtube.com/c/SabineHossenfelder}

\bibitem{video2025}
Various, \emph{Video References}, 2025.

\bibitem{unnikrishnan2004}
C.S. Unnikrishnan, \emph{Gravity Studies}, 2004.

\bibitem{peratt1992}
A. Peratt, \emph{Plasma Cosmology}, 1992.
\url{https://github.com/jpascher/T0-Time-Mass-Duality/blob/main/2/pdf/T0_peratt_En.pdf}

\bibitem{T0_tm_erweiterung}
J. Pascher, \emph{T0 Time-Mass Extension}, 2025.
\url{https://github.com/jpascher/T0-Time-Mass-Duality/blob/main/2/pdf/T0_tm-erweiterung-x6_En.pdf}

\bibitem{T0_g2_erweiterung}
J. Pascher, \emph{T0 g-2 Extension}, 2025.
\url{https://github.com/jpascher/T0-Time-Mass-Duality/blob/main/2/pdf/T0_g2-erweiterung-4_En.pdf}

\bibitem{T0_netze_en}
J. Pascher, \emph{T0 Networks}, 2025.
\url{https://github.com/jpascher/T0-Time-Mass-Duality/blob/main/2/pdf/T0_netze_En.pdf}

\bibitem{Adams1925}
W. Adams, \emph{Gravitational Redshift}, 1925.
\url{https://doi.org/10.1073/pnas.11.7.382}

\bibitem{Ashby2003}
N. Ashby, \emph{Relativity in GPS}, Living Rev. Rel., 2003.
\url{https://doi.org/10.12942/lrr-2003-1}

\bibitem{Bertotti2003}
B. Bertotti et al., \emph{Cassini Doppler Test}, Nature, 2003.
\url{https://doi.org/10.1038/nature01997}

\bibitem{Bolton2008}
A. Bolton et al., \emph{Gravitational Lensing}, 2008.

\bibitem{Born2013}
M. Born, \emph{Einstein's Theory of Relativity}, Dover, 2013.

\bibitem{Brans1961}
C. Brans and R.H. Dicke, \emph{Mach's Principle}, Phys. Rev., 1961.
\url{https://doi.org/10.1103/PhysRev.124.925}

\bibitem{Dirac1927}
P.A.M. Dirac, \emph{Quantum Mechanics}, Proc. Roy. Soc., 1927.
\url{https://doi.org/10.1098/rspa.1927.0039}

\bibitem{Duhem1906}
P. Duhem, \emph{Theory of Physics}, 1906.

\bibitem{Einstein1905}
A. Einstein, \emph{Special Relativity}, Ann. Phys., 1905.
\url{https://doi.org/10.1002/andp.19053221004}

\bibitem{Feynman2006}
R. Feynman, \emph{QED: The Strange Theory of Light and Matter}, 2006.

\bibitem{Griffiths2017}
D. Griffiths, \emph{Introduction to Quantum Mechanics}, 2017.

\bibitem{Jackson1999}
J.D. Jackson, \emph{Classical Electrodynamics}, 1999.

\bibitem{Kaluza1921}
T. Kaluza, \emph{Five-Dimensional Theory}, 1921.

\bibitem{Klein1926}
O. Klein, \emph{Quantum Theory and Relativity}, 1926.

\bibitem{Kuhn1962}
T. Kuhn, \emph{Structure of Scientific Revolutions}, 1962.

\bibitem{Kuhn1977}
T. Kuhn, \emph{Essential Tension}, 1977.

\bibitem{Ludlow2015}
A. Ludlow et al., \emph{Optical Atomic Clocks}, Rev. Mod. Phys., 2015.
\url{https://doi.org/10.1103/RevModPhys.87.637}

\bibitem{Maxwell1873}
J.C. Maxwell, \emph{Treatise on Electricity and Magnetism}, 1873.

\bibitem{McGaugh2016}
S. McGaugh et al., \emph{Radial Acceleration Relation}, Phys. Rev. Lett., 2016.
\url{https://doi.org/10.1103/PhysRevLett.117.201101}

\bibitem{Mohr2016}
P. Mohr et al., \emph{CODATA Values}, Rev. Mod. Phys., 2016.
\url{https://doi.org/10.1103/RevModPhys.88.035009}

\bibitem{PDG2020}
Particle Data Group, \emph{Review of Particle Physics}, Prog. Theor. Exp. Phys., 2020.
\url{https://pdg.lbl.gov/}

\bibitem{Parker2018}
R. Parker et al., \emph{Measurement of $\alpha$}, Science, 2018.
\url{https://doi.org/10.1126/science.aap7706}

\bibitem{Peskin1995}
M. Peskin and D. Schroeder, \emph{QFT}, 1995.

\bibitem{Planck1900}
M. Planck, \emph{Quantum Theory}, 1900.

\bibitem{Planck2020}
Planck Collaboration, \emph{Planck 2020 Results}, 2020.
\url{https://doi.org/10.1051/0004-6361/201833910}

\bibitem{Poincare1905}
H. Poincaré, \emph{Dynamics of the Electron}, 1905.

\bibitem{Pound1960}
R.V. Pound and G.A. Rebka, \emph{Gravitational Redshift}, Phys. Rev. Lett., 1960.
\url{https://doi.org/10.1103/PhysRevLett.4.337}

\bibitem{Quine1951}
W.V. Quine, \emph{Two Dogmas of Empiricism}, 1951.

\bibitem{Quinn2013}
T. Quinn et al., \emph{Gravitational Constant}, 2013.
\url{https://doi.org/10.1103/PhysRevLett.111.101102}

\bibitem{Randall1999}
L. Randall and R. Sundrum, \emph{Extra Dimensions}, Phys. Rev. Lett., 1999.
\url{https://doi.org/10.1103/PhysRevLett.83.3370}

\bibitem{Riess1998}
A. Riess et al., \emph{Type Ia Supernovae}, AJ, 1998.
\url{https://doi.org/10.1086/300499}

\bibitem{Shapiro1971}
I. Shapiro et al., \emph{Time Delay Test}, Phys. Rev. Lett., 1971.
\url{https://doi.org/10.1103/PhysRevLett.26.1132}

\bibitem{Sommerfeld1916}
A. Sommerfeld, \emph{Fine Structure}, 1916.

\bibitem{Suyu2017}
S. Suyu et al., \emph{Time Delay Cosmography}, MNRAS, 2017.
\url{https://doi.org/10.1093/mnras/stx483}

\bibitem{T0Theory}
J. Pascher, \emph{T0 Theory}, 2025.
\url{https://github.com/jpascher/T0-Time-Mass-Duality/blob/main/2/pdf/systemEn.pdf}

\bibitem{T0_Feinstruktur}
J. Pascher, \emph{Fine Structure in T0}, 2025.
\url{https://github.com/jpascher/T0-Time-Mass-Duality/blob/main/2/pdf/T0_Feinstruktur_En.pdf}

\bibitem{Uzan2003}
J.-P. Uzan, \emph{Constants Variation}, Rev. Mod. Phys., 2003.
\url{https://doi.org/10.1103/RevModPhys.75.403}

\bibitem{Webb2001}
J.K. Webb et al., \emph{Fine Structure Constant}, Phys. Rev. Lett., 2001.
\url{https://doi.org/10.1103/PhysRevLett.87.091301}

\bibitem{Weinberg1979}
S. Weinberg, \emph{Cosmological Constant}, Rev. Mod. Phys., 1979.

\bibitem{Weinberg1989}
S. Weinberg, \emph{Cosmological Constant Problem}, 1989.
\url{https://doi.org/10.1103/RevModPhys.61.1}

\bibitem{Weinberg1995}
S. Weinberg, \emph{Quantum Theory of Fields}, 1995.

\bibitem{Will2014}
C. Will, \emph{Theory and Experiment in Gravitational Physics}, 2014.
\url{https://doi.org/10.12942/lrr-2014-4}

\bibitem{dirac_principles}
P.A.M. Dirac, \emph{Principles of Quantum Mechanics}, 1930.

\bibitem{einstein_1917}
A. Einstein, \emph{Cosmological Considerations}, 1917.

\bibitem{jwst_early}
JWST Collaboration, \emph{Early Universe Observations}, 2023.
\url{https://www.jwst.nasa.gov/}

\bibitem{katrin_2022}
KATRIN Collaboration, \emph{Neutrino Mass}, 2022.
\url{https://doi.org/10.1038/s41567-021-01463-1}

\bibitem{pascher:fundamentals}
J. Pascher, \emph{T0 Fundamentals}, 2025.
\url{https://github.com/jpascher/T0-Time-Mass-Duality/blob/main/2/pdf/T0_Grundlagen_En.pdf}

\bibitem{pascher:g2_rev9}
J. Pascher, \emph{g-2 Analysis Rev9}, 2025.
\url{https://github.com/jpascher/T0-Time-Mass-Duality/blob/main/2/pdf/T0_Anomale-g2-9_En.pdf}

\bibitem{pascher:ml_addendum}
J. Pascher, \emph{ML Addendum}, 2025.
\url{https://github.com/jpascher/T0-Time-Mass-Duality/blob/main/2/pdf/T0-QFT-ML_Addendum_En.pdf}

\bibitem{pascher_beta_derivation_2025}
J. Pascher, \emph{Beta Derivation}, 2025.
\url{https://github.com/jpascher/T0-Time-Mass-Duality/blob/main/2/pdf/DerivationVonBetaEn.pdf}

\bibitem{pascher_cmb_en}
J. Pascher, \emph{CMB Analysis in T0}, 2025.
\url{https://github.com/jpascher/T0-Time-Mass-Duality/blob/main/2/pdf/Zwei-Dipole-CMB_En.pdf}

\bibitem{pascher_cosmos_en}
J. Pascher, \emph{Cosmos in T0 Theory}, 2025.
\url{https://github.com/jpascher/T0-Time-Mass-Duality/blob/main/2/pdf/cosmic_En.pdf}

\bibitem{pascher_derivation_beta_2025}
J. Pascher, \emph{Derivation of Beta}, 2025.
\url{https://github.com/jpascher/T0-Time-Mass-Duality/blob/main/2/pdf/DerivationVonBetaEn.pdf}

\bibitem{pascher_gravitation_en}
J. Pascher, \emph{Gravitation in T0}, 2025.
\url{https://github.com/jpascher/T0-Time-Mass-Duality/blob/main/2/pdf/gravitationskonstante_En.pdf}

\bibitem{pascher_lagrangian_2025}
J. Pascher, \emph{Lagrangian in T0}, 2025.
\url{https://github.com/jpascher/T0-Time-Mass-Duality/blob/main/2/pdf/T0_lagrndian_En.pdf}

\bibitem{pascher_lagrangian_en}
J. Pascher, \emph{Lagrangian Framework}, 2025.
\url{https://github.com/jpascher/T0-Time-Mass-Duality/blob/main/2/pdf/LagrandianVergleichEn.pdf}

\bibitem{pascher_lagrangian_extended_2025}
J. Pascher, \emph{Extended Lagrangian Formalism}, 2025.
\url{https://github.com/jpascher/T0-Time-Mass-Duality/blob/main/2/pdf/T0_lagrndian_En.pdf}

\bibitem{pascher_mathematical_structure_2025}
J. Pascher, \emph{Mathematical Structure of T0 Theory}, 2025.
\url{https://github.com/jpascher/T0-Time-Mass-Duality/blob/main/2/pdf/Mathematische_struktur_En.pdf}

\bibitem{pascher_muon_g2_2025}
J. Pascher, \emph{Muon g-2 in T0}, 2025.
\url{https://github.com/jpascher/T0-Time-Mass-Duality/blob/main/2/pdf/T0_Anomale-g2-9_En.pdf}

\bibitem{pascher_pragmatic_2025}
J. Pascher, \emph{Pragmatic Approach}, 2025.

\bibitem{pascher_t0_energy_2025}
J. Pascher, \emph{T0 Energy Formalism}, 2025.
\url{https://github.com/jpascher/T0-Time-Mass-Duality/blob/main/2/pdf/T0-Energie_En.pdf}

\bibitem{pascher_unified_2025}
J. Pascher, \emph{Unified T0 Theory}, 2025.
\url{https://github.com/jpascher/T0-Time-Mass-Duality/blob/main/2/pdf/T0_unified_report.pdf}

\bibitem{sciencedaily2025}
Science Daily, \emph{Physics News}, 2025.
\url{https://www.sciencedaily.com/}

\bibitem{weinberg_1989}
S. Weinberg, \emph{The Cosmological Constant Problem}, Rev. Mod. Phys., 1989.
\url{https://doi.org/10.1103/RevModPhys.61.1}

\bibitem{wiki_bell}
Wikipedia, \emph{Bell's Theorem}, 2025.
\url{https://en.wikipedia.org/wiki/Bell\%27s_theorem}

\bibitem{vanFraassen1980}
B. van Fraassen, \emph{The Scientific Image}, Oxford University Press, 1980.

\bibitem{terrell_single_clock_nature_2024}
J. Terrell, \emph{Single Clock Nature}, Nature, 2024.

% Additional T0 Documents
\bibitem{137_doc}
J. Pascher, \emph{The Number 137 in T0 Theory}, 2025.
\url{https://github.com/jpascher/T0-Time-Mass-Duality/blob/main/2/pdf/137_En.pdf}

\bibitem{ampere_low}
J. Pascher, \emph{Ampere's Law in T0}, 2025.
\url{https://github.com/jpascher/T0-Time-Mass-Duality/blob/main/2/pdf/Amper_Low_En.pdf}

\bibitem{bell_theorem}
J. Pascher, \emph{Bell's Theorem in T0}, 2025.
\url{https://github.com/jpascher/T0-Time-Mass-Duality/blob/main/2/pdf/Bell_En.pdf}

\bibitem{bewegungsenergie}
J. Pascher, \emph{Kinetic Energy in T0}, 2025.
\url{https://github.com/jpascher/T0-Time-Mass-Duality/blob/main/2/pdf/Bewegungsenergie_En.pdf}

\bibitem{emc2}
J. Pascher, \emph{E=mc² in T0 Framework}, 2025.
\url{https://github.com/jpascher/T0-Time-Mass-Duality/blob/main/2/pdf/E-mc2_En.pdf}

\bibitem{formeln_energiebasiert}
J. Pascher, \emph{Energy-Based Formulas}, 2025.
\url{https://github.com/jpascher/T0-Time-Mass-Duality/blob/main/2/pdf/Formeln_Energiebasiert_En.pdf}

\bibitem{hannah}
J. Pascher, \emph{Hannah Document}, 2025.
\url{https://github.com/jpascher/T0-Time-Mass-Duality/blob/main/2/pdf/Hannah_En.pdf}

\bibitem{ho_doc}
J. Pascher, \emph{H0 Analysis}, 2025.
\url{https://github.com/jpascher/T0-Time-Mass-Duality/blob/main/2/pdf/Ho_En.pdf}

\bibitem{markov}
J. Pascher, \emph{Markov Processes in T0}, 2025.
\url{https://github.com/jpascher/T0-Time-Mass-Duality/blob/main/2/pdf/Markov_En.pdf}

\bibitem{elimination_mass}
J. Pascher, \emph{Elimination of Mass}, 2025.
\url{https://github.com/jpascher/T0-Time-Mass-Duality/blob/main/2/pdf/EliminationOfMassEn.pdf}

\bibitem{elimination_mass_dirac}
J. Pascher, \emph{Dirac Equation Mass Elimination}, 2025.
\url{https://github.com/jpascher/T0-Time-Mass-Duality/blob/main/2/pdf/Elimination_Of_Mass_Dirac_TabelleEn.pdf}

\bibitem{feinstrukturkonstante}
J. Pascher, \emph{Fine Structure Constant}, 2025.
\url{https://github.com/jpascher/T0-Time-Mass-Duality/blob/main/2/pdf/FeinstrukturkonstanteEn.pdf}

\bibitem{neutrino_formel}
J. Pascher, \emph{Neutrino Formula}, 2025.
\url{https://github.com/jpascher/T0-Time-Mass-Duality/blob/main/2/pdf/neutrino-Formel_En.pdf}

\bibitem{neutrinos}
J. Pascher, \emph{Neutrinos in T0}, 2025.
\url{https://github.com/jpascher/T0-Time-Mass-Duality/blob/main/2/pdf/T0_Neutrinos_En.pdf}

\bibitem{koide_formel}
J. Pascher, \emph{Koide Formula in T0}, 2025.
\url{https://github.com/jpascher/T0-Time-Mass-Duality/blob/main/2/pdf/T0_koide-formel-3_En.pdf}

\bibitem{teilchenmassen}
J. Pascher, \emph{Particle Masses}, 2025.
\url{https://github.com/jpascher/T0-Time-Mass-Duality/blob/main/2/pdf/Teilchenmassen_En.pdf}

\bibitem{t0_teilchenmassen}
J. Pascher, \emph{T0 Particle Masses}, 2025.
\url{https://github.com/jpascher/T0-Time-Mass-Duality/blob/main/2/pdf/T0_Teilchenmassen_En.pdf}

\bibitem{penrose_doc}
J. Pascher, \emph{Penrose Analysis in T0}, 2025.
\url{https://github.com/jpascher/T0-Time-Mass-Duality/blob/main/2/pdf/T0_penrose_En.pdf}

\bibitem{photonenchip}
J. Pascher, \emph{Photon Chip Implementation}, 2025.
\url{https://github.com/jpascher/T0-Time-Mass-Duality/blob/main/2/pdf/T0_photonenchip-china_En.pdf}

\bibitem{threeclock}
J. Pascher, \emph{Three Clock Experiment}, 2025.
\url{https://github.com/jpascher/T0-Time-Mass-Duality/blob/main/2/pdf/T0_threeclock_En.pdf}

\bibitem{redshift_deflection}
J. Pascher, \emph{Redshift and Deflection}, 2025.
\url{https://github.com/jpascher/T0-Time-Mass-Duality/blob/main/2/pdf/redshift_deflection_En.pdf}

\bibitem{scheinbar_instantan}
J. Pascher, \emph{Apparent Instantaneity}, 2025.
\url{https://github.com/jpascher/T0-Time-Mass-Duality/blob/main/2/pdf/scheinbar_instantan_En.pdf}

\bibitem{universale_ableitung}
J. Pascher, \emph{Universal Derivation}, 2025.
\url{https://github.com/jpascher/T0-Time-Mass-Duality/blob/main/2/pdf/universale-ableitung_En.pdf}

\bibitem{xi_parameter}
J. Pascher, \emph{Xi Parameter for Particles}, 2025.
\url{https://github.com/jpascher/T0-Time-Mass-Duality/blob/main/2/pdf/xi_parmater_partikel_En.pdf}

\bibitem{xi_ursprung}
J. Pascher, \emph{Origin of Xi}, 2025.
\url{https://github.com/jpascher/T0-Time-Mass-Duality/blob/main/2/pdf/T0_xi_ursprung_En.pdf}

\bibitem{zeit}
J. Pascher, \emph{Time in T0 Theory}, 2025.
\url{https://github.com/jpascher/T0-Time-Mass-Duality/blob/main/2/pdf/Zeit_En.pdf}

\bibitem{zeit_konstant}
J. Pascher, \emph{Time Constant}, 2025.
\url{https://github.com/jpascher/T0-Time-Mass-Duality/blob/main/2/pdf/Zeit-konstant_En.pdf}

\bibitem{zusammenfassung}
J. Pascher, \emph{Summary of T0 Theory}, 2025.
\url{https://github.com/jpascher/T0-Time-Mass-Duality/blob/main/2/pdf/Zusammenfassung_En.pdf}

\bibitem{rsa}
J. Pascher, \emph{RSA in T0 Framework}, 2025.
\url{https://github.com/jpascher/T0-Time-Mass-Duality/blob/main/2/pdf/RSA_En.pdf}

\bibitem{qat}
J. Pascher, \emph{Quantum Atomic Theory}, 2025.
\url{https://github.com/jpascher/T0-Time-Mass-Duality/blob/main/2/pdf/T0_QAT_En.pdf}

\bibitem{qm_qft_rt}
J. Pascher, \emph{QM, QFT and RT Unification}, 2025.
\url{https://github.com/jpascher/T0-Time-Mass-Duality/blob/main/2/pdf/T0_QM-QFT-RT_En.pdf}

\bibitem{qm_optimierung}
J. Pascher, \emph{QM Optimization}, 2025.
\url{https://github.com/jpascher/T0-Time-Mass-Duality/blob/main/2/pdf/T0_QM-optimierung_En.pdf}

\bibitem{vollstaendige_berechnungen}
J. Pascher, \emph{Complete Calculations}, 2025.
\url{https://github.com/jpascher/T0-Time-Mass-Duality/blob/main/2/pdf/T0_Vollstaendige_Berchnungen_En.pdf}

\bibitem{synergetics}
J. Pascher, \emph{T0 Theory vs Synergetics}, 2025.
\url{https://github.com/jpascher/T0-Time-Mass-Duality/blob/main/2/pdf/T0-Theory-vs-Synergetics_En.pdf}

\bibitem{modell_uebersicht}
J. Pascher, \emph{T0 Model Overview}, 2025.
\url{https://github.com/jpascher/T0-Time-Mass-Duality/blob/main/2/pdf/T0_Modell_Uebersicht_En.pdf}

\bibitem{mnras_widerlegung}
J. Pascher, \emph{MNRAS Analysis}, 2025.
\url{https://github.com/jpascher/T0-Time-Mass-Duality/blob/main/2/pdf/T0_Analyse_MNRAS_Widerlegung_En.pdf}

\bibitem{anomale_magnetische_momente}
J. Pascher, \emph{Anomalous Magnetic Moments}, 2025.
\url{https://github.com/jpascher/T0-Time-Mass-Duality/blob/main/2/pdf/T0_Anomale_Magnetische_Momente_En.pdf}

\bibitem{sieben_fragen}
J. Pascher, \emph{Seven Questions in T0}, 2025.
\url{https://github.com/jpascher/T0-Time-Mass-Duality/blob/main/2/pdf/T0_7-fragen-3_En.pdf}

\bibitem{detailierte_leptonen}
J. Pascher, \emph{Detailed Lepton Anomaly}, 2025.
\url{https://github.com/jpascher/T0-Time-Mass-Duality/blob/main/2/pdf/detailierte_formel_leptonen_anemal_En.pdf}

\bibitem{parameterherleitung}
J. Pascher, \emph{Parameter Derivation}, 2025.
\url{https://github.com/jpascher/T0-Time-Mass-Duality/blob/main/2/pdf/parameterherleitung_En.pdf}

\bibitem{verhaeltnis_absolut}
J. Pascher, \emph{Absolute Ratios in T0}, 2025.
\url{https://github.com/jpascher/T0-Time-Mass-Duality/blob/main/2/pdf/T0_verhaeltnis-absolut_En.pdf}

\bibitem{xi_und_e}
J. Pascher, \emph{Xi and Energy}, 2025.
\url{https://github.com/jpascher/T0-Time-Mass-Duality/blob/main/2/pdf/T0_xi-und-e_En.pdf}

\bibitem{umkehrung}
J. Pascher, \emph{Inversion in T0}, 2025.
\url{https://github.com/jpascher/T0-Time-Mass-Duality/blob/main/2/pdf/T0_umkehrung_En.pdf}

\bibitem{esm_analysis}
J. Pascher, \emph{T0 vs ESM Conceptual Analysis}, 2025.
\url{https://github.com/jpascher/T0-Time-Mass-Duality/blob/main/2/pdf/T0vsESM_ConceptualAnalysis_En.pdf}

\end{thebibliography}

\end{document}
