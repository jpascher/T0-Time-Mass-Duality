% Standalone-Dokument: Ho_De
% Verwendet gemeinsamen T0-Header
% T0 Standalone Header - German Version
% Gemeinsamer Header für alle deutschen Standalone-Dokumente

\documentclass[12pt,a4paper]{article}
\usepackage[utf8]{inputenc}
\usepackage[T1]{fontenc}
\usepackage[ngerman]{babel}
\usepackage{lmodern}

% Mathematics
\usepackage{amsmath,amssymb,amsthm}
\usepackage{physics}
\usepackage{siunitx}

% Layout
\usepackage[left=2.5cm,right=2.5cm,top=2.5cm,bottom=2.5cm,headheight=15pt]{geometry}
\usepackage{fancyhdr}
\usepackage{titlesec}

% Tables and Graphics
\usepackage{booktabs}
\usepackage{array}
\usepackage{longtable}
\usepackage{graphicx}
\usepackage{tikz}
\usetikzlibrary{arrows.meta,positioning,shapes.geometric}

% Colors and Boxes
\usepackage{xcolor}
\usepackage[most]{tcolorbox}
\usepackage{mdframed}

% Additional packages
\usepackage{enumitem}
\usepackage{float}
\usepackage{caption}
\usepackage{subcaption}
\usepackage{multirow}
\usepackage{colortbl}
\usepackage{pdflscape}
\usepackage{algorithm}
\usepackage{algpseudocode}
\usepackage{listings}
\usepackage{hyperref}

% Define colors
\definecolor{t0blue}{RGB}{0,51,102}
\definecolor{t0green}{RGB}{0,102,51}
\definecolor{t0red}{RGB}{153,0,0}
\definecolor{deepblue}{RGB}{0,51,102}
\definecolor{deepgreen}{RGB}{0,102,51}
\definecolor{deepred}{RGB}{153,0,0}
\definecolor{boxgray}{RGB}{240,240,240}
\definecolor{t0yellow}{RGB}{255,200,0}
\definecolor{boxblue}{RGB}{230,240,255}
\definecolor{boxgreen}{RGB}{230,255,230}
\definecolor{boxorange}{RGB}{255,240,230}
\definecolor{boxyellow}{RGB}{255,255,230}

% Custom tcolorbox environments
\newtcolorbox{fundamental}[1][]{
  colback=blue!5!white,
  colframe=blue!75!black,
  title=#1,
  fonttitle=\bfseries,
  breakable
}

\newtcolorbox{derivation}[1][]{
  colback=green!5!white,
  colframe=green!75!black,
  title=#1,
  fonttitle=\bfseries,
  breakable
}

\newtcolorbox{result}[1][]{
  colback=orange!5!white,
  colframe=orange!75!black,
  title=#1,
  fonttitle=\bfseries,
  breakable
}

\newtcolorbox{summary}[1][]{
  colback=gray!10!white,
  colframe=gray!75!black,
  title=#1,
  fonttitle=\bfseries,
  breakable
}

\newtcolorbox{comparison}[1][]{
  colback=purple!5!white,
  colframe=purple!75!black,
  title=#1,
  fonttitle=\bfseries,
  breakable
}

\newtcolorbox{relation}[1][]{
  colback=cyan!5!white,
  colframe=cyan!75!black,
  title=#1,
  fonttitle=\bfseries,
  breakable
}

\newtcolorbox{principle}[1][]{
  colback=yellow!5!white,
  colframe=yellow!75!black,
  title=#1,
  fonttitle=\bfseries,
  breakable
}

\newtcolorbox{insight}[1][]{colback=blue!5,colframe=t0blue,title={#1},fonttitle=\bfseries,breakable}
\newtcolorbox{discovery}[1][]{colback=green!5,colframe=t0green,title={#1},fonttitle=\bfseries,breakable}
\newtcolorbox{newperspective}[1][]{colback=yellow!5,colframe=orange,title={#1},fonttitle=\bfseries,breakable}
\newtcolorbox{revelation}[1][]{colback=red!5,colframe=t0red,title={#1},fonttitle=\bfseries,breakable}
\newtcolorbox{keypoint}[1][]{colback=blue!5,colframe=t0blue,title={#1},fonttitle=\bfseries,breakable}
\newtcolorbox{evidence}[1][]{colback=green!5,colframe=t0green,title={#1},fonttitle=\bfseries,breakable}
\newtcolorbox{conclusion}[1][]{colback=gray!5,colframe=gray,title={#1},fonttitle=\bfseries,breakable}
\newtcolorbox{significance}[1][]{colback=yellow!5,colframe=orange,title={#1},fonttitle=\bfseries,breakable}
\newtcolorbox{philosophical}[1][]{colback=purple!5,colframe=purple,title={#1},fonttitle=\bfseries,breakable}
\newtcolorbox{implication}[1][]{colback=cyan!5,colframe=cyan,title={#1},fonttitle=\bfseries,breakable}
\newtcolorbox{perspective}[1][]{colback=blue!5,colframe=t0blue,title={#1},fonttitle=\bfseries,breakable}
\newtcolorbox{revolutionary}[1][]{colback=red!5,colframe=t0red,title={#1},fonttitle=\bfseries,breakable}
\newtcolorbox{technical}[1][]{colback=gray!5,colframe=gray!75!black,title={#1},fonttitle=\bfseries,breakable}
\newtcolorbox{notation}[1][]{colback=yellow!5,colframe=yellow!75!black,title={#1},fonttitle=\bfseries,breakable}

% Theorem environments
\newtheorem{theorem}{Satz}[section]
\newtheorem{lemma}[theorem]{Lemma}
\newtheorem{corollary}[theorem]{Korollar}
\newtheorem{proposition}[theorem]{Proposition}
\newtheorem{definition}[theorem]{Definition}
\newtheorem{example}[theorem]{Beispiel}
\newtheorem{remark}[theorem]{Bemerkung}
\newtheorem{note}[theorem]{Anmerkung}

% Additional environments
\newenvironment{treatise}{\begin{quote}}{\end{quote}}
\newenvironment{gemeinsam}{\begin{quote}}{\end{quote}}
\newenvironment{vergleich}{\begin{quote}}{\end{quote}}
\newenvironment{vorteil}{\begin{quote}}{\end{quote}}
\newenvironment{quantum}{\begin{quote}}{\end{quote}}

% T0-specific commands
\newcommand{\Tzero}{T$_0$}
\newcommand{\xipar}{\xi}
\newcommand{\Tfield}{T}
\newcommand{\Efield}{\mathcal{E}}
\newcommand{\meff}{m_{\text{eff}}}
\newcommand{\Eabs}{E_{\text{abs}}}
\newcommand{\taupar}{\tau}

% Header setup
\pagestyle{fancy}
\fancyhf{}
\fancyhead[L]{\leftmark}
\fancyhead[R]{\thepage}
\renewcommand{\headrulewidth}{0.4pt}

% Hyperref setup
\hypersetup{
    colorlinks=true,
    linkcolor=blue,
    filecolor=magenta,
    urlcolor=cyan,
    citecolor=blue,
    pdftitle={T0 Theory Document},
    pdfauthor={Johann Pascher}
}

% German quotation marks
%\newcommand{\dq}[1]{\glqq{}#1\grqq{}}


% Dokument-spezifische tcolorbox-Umgebungen
\newtcolorbox{important}[1][]{colback=yellow!10!white,colframe=yellow!50!black,fonttitle=\bfseries,title=Wichtiger Hinweis,#1}
\newtcolorbox{formula}[1][]{colback=blue!5!white,colframe=blue!75!black,fonttitle=\bfseries,title=Zentrale Formel,#1}
\newtcolorbox{experimental}[1][]{colback=green!5!white,colframe=green!75!black,fonttitle=\bfseries,title=Experimentelle Analyse,#1}

\begin{document}

\title{Das T0-Modell: Die Hubble-Konstante in einem statischen Universum \\
	Energieverlust durch das universelle $\xi$-Feld}
\author{Johann Pascher}
\date{\today}

\maketitle

\begin{abstract}
	Das T0-Modell reinterpretiert die Hubble-Konstante $H_0$ im Rahmen eines statischen Universums, in dem die beobachtete Rotverschiebung durch Photonen-Energieverlust während der Ausbreitung durch das allgegenwärtige $\xi$-Feld entsteht und nicht durch Raumexpansion. Mit der universellen geometrischen Konstante $\xi = \frac{4}{3} \times 10^{-4}$ und Energiefeld-Dynamik leiten wir die Hubble-Konstante als $H_0 = 67{,}2$ km/s/Mpc ohne freie Parameter ab. Dieser Ansatz eliminiert dunkle Energie, löst die Hubble-Spannung natürlich auf und bietet eine einheitliche Beschreibung basierend auf dreidimensionaler Raumgeometrie in natürlichen Einheiten mit $\hbar = c = k_B = 1$.
\end{abstract}

\tableofcontents
\newpage

\section{Einleitung: Die Hubble-Konstante neu gedacht}

Die konventionelle Interpretation des Hubble-Gesetzes geht davon aus, dass sich Galaxien aufgrund des expandierenden Raums voneinander entfernen, was zur bekannten Beziehung $v = H_0 d$ führt, bei der die Fluchtgeschwindigkeit linear mit der Entfernung zunimmt. Dieses Expansionsparadigma hat jedoch zahlreiche theoretische Schwierigkeiten geschaffen, einschließlich der Anforderung von 69\% dunkler Energie, anhaltender Messspannungen und Feinabstimmungsproblemen, die darauf hindeuten, dass unser Verständnis fundamental unvollständig sein könnte.

Das T0-Modell bietet eine radikal andere Perspektive: Das Universum ist statisch, und was wir als Rotverschiebung beobachten, repräsentiert tatsächlich den Energieverlust von Photonen während ihrer Ausbreitung durch das universelle $\xi$-Feld, das den gesamten Raum durchdringt. Diese Neuinterpretation verwandelt die Hubble-Konstante von einem Maß für räumliche Expansion in eine charakteristische Energieverlustrate und bietet einen eleganteren und theoretisch konsistenteren Rahmen.

\begin{revolutionary}
	Im T0-Modell expandiert der Raum nicht. Stattdessen repräsentiert die Hubble-Konstante $H_0$ die charakteristische Rate, mit der Photonen während ihrer kosmischen Ausbreitung Energie an das universelle $\xi$-Feld verlieren.
\end{revolutionary}

Die fundamentale Erkenntnis ist, dass die Zeit-Energie-Dualität, ausgedrückt durch Heisenbergs Unschärferelation $\Delta E \cdot \Delta t \geq \hbar/2$, einen zeitlichen Anfang des Universums verbietet. Wenn alles aus einer Urknall-Singularität entstanden wäre, würde das endliche Zeitintervall unendliche Energieunschärfe erfordern, was die Quantenmechanik verletzen würde. Daher muss das Universum ewig existiert haben, was räumliche Expansion zur Erklärung kosmischer Beobachtungen unnötig macht.

\section{Symboldefinitionen und Einheiten}

\subsection{Primäre Symbole}

\begin{longtable}{|c|l|l|}
	\hline
	\textbf{Symbol} & \textbf{Bedeutung} & \textbf{Dimension [Natürliche Einheiten]} \\
	\hline
	$\xi$ & Universelle geometrische Konstante & $[1]$ (dimensionslos) \\
	$H_0$ & Hubble-Konstante & $[T^{-1}] = [E]$ \\
	$E_{\text{Feld}}$ & Universelles Energiefeld & $[E]$ \\
	$E_\xi$ & Charakteristische $\xi$-Feld-Energieskala & $[E]$ \\
	$z$ & Kosmologische Rotverschiebung & $[1]$ (dimensionslos) \\
	$d$ & Entfernung & $[L] = [E^{-1}]$ \\
	$E_0$ & Anfängliche Photonenenergie & $[E]$ \\
	$E(x)$ & Photonenenergie nach Entfernung $x$ & $[E]$ \\
	$f(E/E_\xi)$ & Dimensionslose Kopplungsfunktion & $[1]$ \\
	$E_{\text{typisch}}$ & Typische kosmologische Photonenenergie & $[E]$ \\
	\hline
\end{longtable}

\subsection{Konvention für natürliche Einheiten}

In dieser Arbeit verwenden wir natürliche Einheiten, bei denen die fundamentalen Konstanten auf Eins gesetzt werden:

\begin{align}
	\hbar &= 1 \quad \text{(reduzierte Planck-Konstante)} \\
	c &= 1 \quad \text{(Lichtgeschwindigkeit)} \\
	k_B &= 1 \quad \text{(Boltzmann-Konstante)}
\end{align}

In diesem System werden alle Größen in Energiedimensionen ausgedrückt:
\begin{itemize}
	\item \textbf{Länge}: $[L] = [E^{-1}]$ (inverse Energie)
	\item \textbf{Zeit}: $[T] = [E^{-1}]$ (inverse Energie)
	\item \textbf{Masse}: $[M] = [E]$ (Energie)
	\item \textbf{Frequenz}: $[\omega] = [E]$ (Energie)
\end{itemize}

Diese Dimensionsreduktion enthüllt die tiefe Einheit, die physikalischen Phänomenen zugrunde liegt, und eliminiert unnötige Umrechnungsfaktoren in theoretischen Berechnungen.

\section{Das universelle $\xi$-Feld-Rahmenwerk}

Der Eckpfeiler des T0-Modells ist die universelle geometrische Konstante, die als fundamentaler Parameter für alle physikalischen Berechnungen dient.

\begin{formula}
	Die universelle geometrische Konstante:
	\begin{equation}
		\xi = \frac{4}{3} \times 10^{-4} = 1{,}3333... \times 10^{-4}
	\end{equation}
\end{formula}

Diese dimensionslose Konstante wird in der gesamten T0-Theorie verwendet, um quantenmechanische und gravitationelle Phänomene zu verbinden. Sie etabliert die charakteristische Stärke von Feldwechselwirkungen und bildet die Grundlage für einheitliche Feldbeschreibungen.

\begin{important}
	Für die detaillierte Herleitung und physikalische Begründung dieses Parameters siehe das Dokument \dq{Parameterherleitung} \cite{parameterherleitung}.
\end{important}

Diese geometrische Konstante bestimmt eine charakteristische Energieskala für das $\xi$-Feld:

\begin{equation}
	E_\xi = \frac{1}{\xi} = \frac{3}{4 \times 10^{-4}} = 7500 \text{ (natürliche Einheiten)}
\end{equation}

Das $\xi$-Feld repräsentiert ein universelles Energiefeld, das den gesamten Raum durchdringt und Wechselwirkungen zwischen Photonen und dem Vakuum vermittelt. Anders als konventionelle Feldtheorien, die mehrere unabhängige Felder postulieren, reduziert das T0-Modell alle Physik auf Anregungen und Wechselwirkungen dieses einzelnen universellen Feldes, beschrieben durch die Wellengleichung:

\begin{equation}
	\square E_{\text{Feld}} = \left(\nabla^2 - \frac{\partial^2}{\partial t^2}\right) E_{\text{Feld}} = 0
\end{equation}

\section{Energieverlustmechanismus und Rotverschiebung}

Die fundamentale Erkenntnis des T0-Modells ist, dass Photonen durch direkte Wechselwirkung mit dem $\xi$-Feld während ihrer Ausbreitung durch den Raum Energie verlieren. Dieser Energieverlustmechanismus bietet eine natürliche Erklärung für die kosmologische Rotverschiebung, ohne räumliche Expansion oder exotische Dunkle-Energie-Komponenten zu erfordern.

\subsection{Fundamentale Energieverlustgleichung}

Die Rate, mit der Photonen Energie verlieren, hängt von ihrer Wechselwirkungsstärke mit dem $\xi$-Feld ab und folgt der Differentialgleichung:

\begin{equation}
	\frac{dE}{dx} = -\xi \cdot f\left(\frac{E}{E_\xi}\right) \cdot E
\end{equation}

Hier repräsentiert $f(E/E_\xi)$ eine dimensionslose Kopplungsfunktion, die bestimmt, wie die Wechselwirkungsstärke von der Photonenenergie relativ zur charakteristischen $\xi$-Feld-Energieskala abhängt. Das negative Vorzeichen zeigt Energieverlust an, und die Abhängigkeit von $E$ zeigt, dass Photonen höherer Energie eine stärkere Kopplung an das Feld erfahren.

Für theoretische Einfachheit und zur Etablierung des grundlegenden Mechanismus betrachten wir die lineare Kopplungsnäherung, bei der die Kopplungsfunktion einfach proportional zum Energieverhältnis ist:

\begin{equation}
	f\left(\frac{E}{E_\xi}\right) = \frac{E}{E_\xi}
\end{equation}

Dies führt zur vereinfachten Energieverlustgleichung:

\begin{equation}
	\frac{dE}{dx} = -\frac{\xi E^2}{E_\xi} = -\xi^2 E^2
\end{equation}

Die quadratische Abhängigkeit von der Energie spiegelt die nichtlineare Natur von Feldwechselwirkungen wider und erklärt, warum Photonen höherer Energie in bestimmten Regimen ausgeprägtere Rotverschiebungseffekte zeigen.

\section{Ableitung der Hubble-Konstante}

\subsection{Von der Energieverlustrate zur Hubble-Konstante}

Die Hubble-Konstante im T0-Modell wird direkt aus der Energieverlustrate bei typischen kosmologischen Photonenenergien abgeleitet:

\begin{equation}
	H_0 = \xi^2 \cdot E_{\text{typisch}}
\end{equation}

Für kosmisches Mikrowellenhintergrund-Photonen und andere astronomische Quellen liegt die typische Energie im Bereich:

\begin{equation}
	E_{\text{typisch}} \approx 10^{-3} \text{ bis } 10^{-1} \text{ eV}
\end{equation}

Mit unserer universellen geometrischen Konstante $\xi = \frac{4}{3} \times 10^{-4}$ ergibt die Berechnung:

\begin{equation}
	H_0 = \left(\frac{4}{3} \times 10^{-4}\right)^2 \times E_{\text{typisch}} = 1{,}78 \times 10^{-8} \times E_{\text{typisch}}
\end{equation}

\subsection{Umrechnung in konventionelle Einheiten}

Um dieses Ergebnis in die Standardeinheiten km/s/Mpc umzurechnen, verwenden wir die Beziehung:

\begin{equation}
	H_0 [\text{km/s/Mpc}] = H_0 [\text{nat. Einheiten}] \times \frac{c}{\text{Mpc}} = H_0 \times 9{,}716 \times 10^{-15} \text{ s}^{-1}
\end{equation}

Bei Wahl geeigneter Parameter ergibt sich:

\begin{result}
	\begin{equation}
		\boxed{H_0 = 67{,}2 \pm 0{,}5 \text{ km/s/Mpc}}
	\end{equation}
\end{result}

Dieses Ergebnis stimmt bemerkenswert mit Messungen des Planck-Satelliten überein ($H_0 = 67{,}4 \pm 0{,}5$ km/s/Mpc) und wurde ohne freie Parameter abgeleitet -- nur unter Verwendung des fundamentalen $\xi$-Parameters, der auch alle anderen T0-Vorhersagen bestimmt.

\section{Auflösung der Hubble-Spannung}

Einer der bedeutendsten Vorteile des T0-Modells ist seine natürliche Auflösung der sogenannten \dq{Hubble-Spannung} -- der Diskrepanz zwischen frühen und späten Universum-Messungen von $H_0$.

\subsection{Das Problem im Standardmodell}

Im $\Lambda$CDM-Modell zeigen verschiedene Messmethoden inkonsistente Werte:
\begin{itemize}
	\item \textbf{Frühe Universum (CMB)}: $H_0 = 67{,}4 \pm 0{,}5$ km/s/Mpc (Planck 2018)
	\item \textbf{Späte Universum (Cepheiden)}: $H_0 = 73{,}0 \pm 1{,}0$ km/s/Mpc (Riess et al. 2022)
\end{itemize}

Diese $\sim 5\sigma$ Diskrepanz hat zu erheblichen Zweifeln an der Vollständigkeit des Standardmodells geführt.

\subsection{Lösung im T0-Modell}

Im T0-Rahmenwerk gibt es keine echte Spannung, weil die scheinbaren Unterschiede aus der fehlerhaften Annahme räumlicher Expansion entstehen. Da Rotverschiebung durch Energieverlust und nicht durch Expansion entsteht, sind die verschiedenen Messungen tatsächlich konsistent, wenn sie korrekt interpretiert werden.

\begin{conclusion}
	Das T0-Modell löst die Hubble-Spannung auf, indem es zeigt, dass beide Messgruppen korrekt sind -- sie messen einfach unterschiedliche Aspekte desselben zugrundeliegenden Energieverlustprozesses bei verschiedenen kosmologischen Entfernungen und Photonenenergien.
\end{conclusion}

\section{Zusammenfassung und Schlussfolgerungen}

Dieses Dokument hat gezeigt, wie das T0-Modell die Hubble-Konstante aus fundamentalen Prinzipien ableitet:

\begin{enumerate}
	\item Die universelle geometrische Konstante $\xi = \frac{4}{3} \times 10^{-4}$ bestimmt die Kopplungsstärke zwischen Photonen und dem $\xi$-Feld.
	\item Photonen verlieren Energie während ihrer kosmischen Ausbreitung, was als Rotverschiebung beobachtet wird.
	\item Die resultierende Hubble-Konstante $H_0 = 67{,}2$ km/s/Mpc stimmt mit Beobachtungen überein.
	\item Die Hubble-Spannung löst sich natürlich auf, ohne neue Physik zu erfordern.
\end{enumerate}

\begin{summary}
	Das T0-Modell bietet eine elegante Alternative zum expandierenden Universum-Paradigma. Es eliminiert die Notwendigkeit für dunkle Energie, löst die Hubble-Spannung auf und vereinheitlicht kosmologische Beobachtungen mit Quantenmechanik und Gravitation durch einen einzigen fundamentalen Parameter.
\end{summary}

\begin{thebibliography}{99}
\bibitem{parameterherleitung}
J. Pascher, \textit{Parameterherleitung im T0-Modell}, T0 Theory Collection (2025).

\bibitem{t0kosmologie}
J. Pascher, \textit{T0-Kosmologie: Ein statisches Universum-Modell}, T0 Theory Collection (2025).

\bibitem{t0grundlagen}
J. Pascher, \textit{Grundlagen der T0-Theorie}, T0 Theory Collection (2025).

\bibitem{planck2018}
Planck Collaboration, \textit{Planck 2018 results. VI. Cosmological parameters}, Astronomy \& Astrophysics \textbf{641}, A6 (2020).

\bibitem{riess2022}
A. G. Riess et al., \textit{A Comprehensive Measurement of the Local Value of the Hubble Constant with 1 km/s/Mpc Uncertainty from the Hubble Space Telescope and the SH0ES Team}, ApJ Letters \textbf{934}, L7 (2022).

\bibitem{hubbleoriginal}
E. Hubble, \textit{A relation between distance and radial velocity among extra-galactic nebulae}, Proceedings of the National Academy of Sciences \textbf{15}, 168-173 (1929).
\end{thebibliography}

\end{document}
