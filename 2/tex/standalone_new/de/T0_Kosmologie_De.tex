% Standalone-Dokument: T0_Kosmologie_De
% Verwendet gemeinsamen T0-Header
% T0 Standalone Header - German Version
% Gemeinsamer Header für alle deutschen Standalone-Dokumente

\documentclass[12pt,a4paper]{article}
\usepackage[utf8]{inputenc}
\usepackage[T1]{fontenc}
\usepackage[ngerman]{babel}
\usepackage{lmodern}

% Mathematics
\usepackage{amsmath,amssymb,amsthm}
\usepackage{physics}
\usepackage{siunitx}

% Layout
\usepackage[left=2.5cm,right=2.5cm,top=2.5cm,bottom=2.5cm,headheight=15pt]{geometry}
\usepackage{fancyhdr}
\usepackage{titlesec}

% Tables and Graphics
\usepackage{booktabs}
\usepackage{array}
\usepackage{longtable}
\usepackage{graphicx}
\usepackage{tikz}
\usetikzlibrary{arrows.meta,positioning,shapes.geometric}

% Colors and Boxes
\usepackage{xcolor}
\usepackage[most]{tcolorbox}
\usepackage{mdframed}

% Additional packages
\usepackage{enumitem}
\usepackage{float}
\usepackage{caption}
\usepackage{subcaption}
\usepackage{multirow}
\usepackage{colortbl}
\usepackage{pdflscape}
\usepackage{algorithm}
\usepackage{algpseudocode}
\usepackage{listings}
\usepackage{hyperref}

% Define colors
\definecolor{t0blue}{RGB}{0,51,102}
\definecolor{t0green}{RGB}{0,102,51}
\definecolor{t0red}{RGB}{153,0,0}
\definecolor{deepblue}{RGB}{0,51,102}
\definecolor{deepgreen}{RGB}{0,102,51}
\definecolor{deepred}{RGB}{153,0,0}
\definecolor{boxgray}{RGB}{240,240,240}
\definecolor{t0yellow}{RGB}{255,200,0}
\definecolor{boxblue}{RGB}{230,240,255}
\definecolor{boxgreen}{RGB}{230,255,230}
\definecolor{boxorange}{RGB}{255,240,230}
\definecolor{boxyellow}{RGB}{255,255,230}

% Custom tcolorbox environments
\newtcolorbox{fundamental}[1][]{
  colback=blue!5!white,
  colframe=blue!75!black,
  title=#1,
  fonttitle=\bfseries,
  breakable
}

\newtcolorbox{derivation}[1][]{
  colback=green!5!white,
  colframe=green!75!black,
  title=#1,
  fonttitle=\bfseries,
  breakable
}

\newtcolorbox{result}[1][]{
  colback=orange!5!white,
  colframe=orange!75!black,
  title=#1,
  fonttitle=\bfseries,
  breakable
}

\newtcolorbox{summary}[1][]{
  colback=gray!10!white,
  colframe=gray!75!black,
  title=#1,
  fonttitle=\bfseries,
  breakable
}

\newtcolorbox{comparison}[1][]{
  colback=purple!5!white,
  colframe=purple!75!black,
  title=#1,
  fonttitle=\bfseries,
  breakable
}

\newtcolorbox{relation}[1][]{
  colback=cyan!5!white,
  colframe=cyan!75!black,
  title=#1,
  fonttitle=\bfseries,
  breakable
}

\newtcolorbox{principle}[1][]{
  colback=yellow!5!white,
  colframe=yellow!75!black,
  title=#1,
  fonttitle=\bfseries,
  breakable
}

\newtcolorbox{insight}[1][]{colback=blue!5,colframe=t0blue,title={#1},fonttitle=\bfseries,breakable}
\newtcolorbox{discovery}[1][]{colback=green!5,colframe=t0green,title={#1},fonttitle=\bfseries,breakable}
\newtcolorbox{newperspective}[1][]{colback=yellow!5,colframe=orange,title={#1},fonttitle=\bfseries,breakable}
\newtcolorbox{revelation}[1][]{colback=red!5,colframe=t0red,title={#1},fonttitle=\bfseries,breakable}
\newtcolorbox{keypoint}[1][]{colback=blue!5,colframe=t0blue,title={#1},fonttitle=\bfseries,breakable}
\newtcolorbox{evidence}[1][]{colback=green!5,colframe=t0green,title={#1},fonttitle=\bfseries,breakable}
\newtcolorbox{conclusion}[1][]{colback=gray!5,colframe=gray,title={#1},fonttitle=\bfseries,breakable}
\newtcolorbox{significance}[1][]{colback=yellow!5,colframe=orange,title={#1},fonttitle=\bfseries,breakable}
\newtcolorbox{philosophical}[1][]{colback=purple!5,colframe=purple,title={#1},fonttitle=\bfseries,breakable}
\newtcolorbox{implication}[1][]{colback=cyan!5,colframe=cyan,title={#1},fonttitle=\bfseries,breakable}
\newtcolorbox{perspective}[1][]{colback=blue!5,colframe=t0blue,title={#1},fonttitle=\bfseries,breakable}
\newtcolorbox{revolutionary}[1][]{colback=red!5,colframe=t0red,title={#1},fonttitle=\bfseries,breakable}
\newtcolorbox{technical}[1][]{colback=gray!5,colframe=gray!75!black,title={#1},fonttitle=\bfseries,breakable}
\newtcolorbox{notation}[1][]{colback=yellow!5,colframe=yellow!75!black,title={#1},fonttitle=\bfseries,breakable}

% Theorem environments
\newtheorem{theorem}{Satz}[section]
\newtheorem{lemma}[theorem]{Lemma}
\newtheorem{corollary}[theorem]{Korollar}
\newtheorem{proposition}[theorem]{Proposition}
\newtheorem{definition}[theorem]{Definition}
\newtheorem{example}[theorem]{Beispiel}
\newtheorem{remark}[theorem]{Bemerkung}
\newtheorem{note}[theorem]{Anmerkung}

% Additional environments
\newenvironment{treatise}{\begin{quote}}{\end{quote}}
\newenvironment{gemeinsam}{\begin{quote}}{\end{quote}}
\newenvironment{vergleich}{\begin{quote}}{\end{quote}}
\newenvironment{vorteil}{\begin{quote}}{\end{quote}}
\newenvironment{quantum}{\begin{quote}}{\end{quote}}

% T0-specific commands
\newcommand{\Tzero}{T$_0$}
\newcommand{\xipar}{\xi}
\newcommand{\Tfield}{T}
\newcommand{\Efield}{\mathcal{E}}
\newcommand{\meff}{m_{\text{eff}}}
\newcommand{\Eabs}{E_{\text{abs}}}
\newcommand{\taupar}{\tau}

% Header setup
\pagestyle{fancy}
\fancyhf{}
\fancyhead[L]{\leftmark}
\fancyhead[R]{\thepage}
\renewcommand{\headrulewidth}{0.4pt}

% Hyperref setup
\hypersetup{
    colorlinks=true,
    linkcolor=blue,
    filecolor=magenta,
    urlcolor=cyan,
    citecolor=blue,
    pdftitle={T0 Theory Document},
    pdfauthor={Johann Pascher}
}

% German quotation marks
%\newcommand{\dq}[1]{\glqq{}#1\grqq{}}


\title{Kosmologie in der T0-Theorie}
\author{Johann Pascher}
\date{2025}

\begin{document}

\maketitle

\chapter{Kosmologie in der T0-Theorie}

\begin{abstract}
Diese Arbeit untersucht die kosmologischen Implikationen der T0-Theorie. Das Modell bietet eine alternative Erklärung für kosmologische Beobachtungen ohne Dunkle Energie oder kosmologische Konstante. Die Rotverschiebung entsteht als natürliche Konsequenz der Zeit-Energie-Dualität.

\textbf{Kernaussagen:}
\begin{itemize}
\item Die kosmische Rotverschiebung hat geometrischen Ursprung
\item Keine Dunkle Energie erforderlich
\item Übereinstimmung mit Beobachtungsdaten
\end{itemize}
\end{abstract}

\section{Kosmische Rotverschiebung}\label{T0_Kosmologie:sec:rotverschiebung}

\subsection{Zeit-Energie-Interpretation}\label{T0_Kosmologie:subsec:interpretation}

Im T0-Modell entsteht die kosmische Rotverschiebung aus der Variation des Energiefeldes:
\begin{equation}
z = \frac{\lambda_{\text{beobachtet}} - \lambda_{\text{emittiert}}}{\lambda_{\text{emittiert}}} = \frac{E_{\text{emittiert}}}{E_{\text{beobachtet}}} - 1
\label{T0_Kosmologie:eq:rotverschiebung}
\end{equation}

\textbf{Physikalische Bedeutung:} Die Rotverschiebung reflektiert die Änderung des lokalen Energiefeldes zwischen Emission und Beobachtung.

\subsection{Hubble-Parameter}\label{T0_Kosmologie:subsec:hubble}

Der Hubble-Parameter erhält eine neue Interpretation:
\begin{equation}
H_0 = \frac{\dot{E}}{E} = \frac{1}{T_H}
\label{T0_Kosmologie:eq:hubble}
\end{equation}

wobei $T_H$ die charakteristische kosmische Zeitskala ist.

\section{Dunkle Energie und kosmologische Konstante}\label{T0_Kosmologie:sec:dunkle_energie}

\subsection{Keine zusätzliche Energieform erforderlich}\label{T0_Kosmologie:subsec:keine_de}

Im T0-Framework wird die beschleunigte Expansion des Universums durch die natürliche Evolution des Energiefeldes erklärt:
\begin{equation}
\Lambda_{\text{eff}} = 0
\label{T0_Kosmologie:eq:lambda_null}
\end{equation}

Die scheinbare beschleunigte Expansion ist ein geometrischer Effekt, keine fundamentale Kraft.

\section{Kosmische Mikrowellenhintergrundstrahlung}\label{T0_Kosmologie:sec:cmb}

Die T0-Theorie macht spezifische Vorhersagen für das CMB-Spektrum:
\begin{equation}
\Delta T = \xigeom \cdot T_{\text{CMB}}
\label{T0_Kosmologie:eq:cmb_fluktuation}
\end{equation}

Diese Vorhersage steht im Einklang mit den Planck-Satellitenmessungen.

\section{Schlussfolgerungen}\label{T0_Kosmologie:sec:schluss}

Das T0-Modell bietet eine elegante kosmologische Beschreibung ohne freie Parameter. Die Übereinstimmung mit Beobachtungsdaten unterstützt die Gültigkeit des Ansatzes.

\end{document}
