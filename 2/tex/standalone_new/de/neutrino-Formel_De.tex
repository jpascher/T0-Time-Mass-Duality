% Standalone document: Neutrino-Formel_En
% Uses shared T0 header
% T0 Standalone Header - German Version
% Gemeinsamer Header für alle deutschen Standalone-Dokumente

\documentclass[12pt,a4paper]{article}
\usepackage[utf8]{inputenc}
\usepackage[T1]{fontenc}
\usepackage[ngerman]{babel}
\usepackage{lmodern}

% Mathematics
\usepackage{amsmath,amssymb,amsthm}
\usepackage{physics}
\usepackage{siunitx}

% Layout
\usepackage[left=2.5cm,right=2.5cm,top=2.5cm,bottom=2.5cm,headheight=15pt]{geometry}
\usepackage{fancyhdr}
\usepackage{titlesec}

% Tables and Graphics
\usepackage{booktabs}
\usepackage{array}
\usepackage{longtable}
\usepackage{graphicx}
\usepackage{tikz}
\usetikzlibrary{arrows.meta,positioning,shapes.geometric}

% Colors and Boxes
\usepackage{xcolor}
\usepackage[most]{tcolorbox}
\usepackage{mdframed}

% Additional packages
\usepackage{enumitem}
\usepackage{float}
\usepackage{caption}
\usepackage{subcaption}
\usepackage{multirow}
\usepackage{colortbl}
\usepackage{pdflscape}
\usepackage{algorithm}
\usepackage{algpseudocode}
\usepackage{listings}
\usepackage{hyperref}

% Define colors
\definecolor{t0blue}{RGB}{0,51,102}
\definecolor{t0green}{RGB}{0,102,51}
\definecolor{t0red}{RGB}{153,0,0}
\definecolor{deepblue}{RGB}{0,51,102}
\definecolor{deepgreen}{RGB}{0,102,51}
\definecolor{deepred}{RGB}{153,0,0}
\definecolor{boxgray}{RGB}{240,240,240}
\definecolor{t0yellow}{RGB}{255,200,0}
\definecolor{boxblue}{RGB}{230,240,255}
\definecolor{boxgreen}{RGB}{230,255,230}
\definecolor{boxorange}{RGB}{255,240,230}
\definecolor{boxyellow}{RGB}{255,255,230}

% Custom tcolorbox environments
\newtcolorbox{fundamental}[1][]{
  colback=blue!5!white,
  colframe=blue!75!black,
  title=#1,
  fonttitle=\bfseries,
  breakable
}

\newtcolorbox{derivation}[1][]{
  colback=green!5!white,
  colframe=green!75!black,
  title=#1,
  fonttitle=\bfseries,
  breakable
}

\newtcolorbox{result}[1][]{
  colback=orange!5!white,
  colframe=orange!75!black,
  title=#1,
  fonttitle=\bfseries,
  breakable
}

\newtcolorbox{summary}[1][]{
  colback=gray!10!white,
  colframe=gray!75!black,
  title=#1,
  fonttitle=\bfseries,
  breakable
}

\newtcolorbox{comparison}[1][]{
  colback=purple!5!white,
  colframe=purple!75!black,
  title=#1,
  fonttitle=\bfseries,
  breakable
}

\newtcolorbox{relation}[1][]{
  colback=cyan!5!white,
  colframe=cyan!75!black,
  title=#1,
  fonttitle=\bfseries,
  breakable
}

\newtcolorbox{principle}[1][]{
  colback=yellow!5!white,
  colframe=yellow!75!black,
  title=#1,
  fonttitle=\bfseries,
  breakable
}

\newtcolorbox{insight}[1][]{colback=blue!5,colframe=t0blue,title={#1},fonttitle=\bfseries,breakable}
\newtcolorbox{discovery}[1][]{colback=green!5,colframe=t0green,title={#1},fonttitle=\bfseries,breakable}
\newtcolorbox{newperspective}[1][]{colback=yellow!5,colframe=orange,title={#1},fonttitle=\bfseries,breakable}
\newtcolorbox{revelation}[1][]{colback=red!5,colframe=t0red,title={#1},fonttitle=\bfseries,breakable}
\newtcolorbox{keypoint}[1][]{colback=blue!5,colframe=t0blue,title={#1},fonttitle=\bfseries,breakable}
\newtcolorbox{evidence}[1][]{colback=green!5,colframe=t0green,title={#1},fonttitle=\bfseries,breakable}
\newtcolorbox{conclusion}[1][]{colback=gray!5,colframe=gray,title={#1},fonttitle=\bfseries,breakable}
\newtcolorbox{significance}[1][]{colback=yellow!5,colframe=orange,title={#1},fonttitle=\bfseries,breakable}
\newtcolorbox{philosophical}[1][]{colback=purple!5,colframe=purple,title={#1},fonttitle=\bfseries,breakable}
\newtcolorbox{implication}[1][]{colback=cyan!5,colframe=cyan,title={#1},fonttitle=\bfseries,breakable}
\newtcolorbox{perspective}[1][]{colback=blue!5,colframe=t0blue,title={#1},fonttitle=\bfseries,breakable}
\newtcolorbox{revolutionary}[1][]{colback=red!5,colframe=t0red,title={#1},fonttitle=\bfseries,breakable}
\newtcolorbox{technical}[1][]{colback=gray!5,colframe=gray!75!black,title={#1},fonttitle=\bfseries,breakable}
\newtcolorbox{notation}[1][]{colback=yellow!5,colframe=yellow!75!black,title={#1},fonttitle=\bfseries,breakable}

% Theorem environments
\newtheorem{theorem}{Satz}[section]
\newtheorem{lemma}[theorem]{Lemma}
\newtheorem{corollary}[theorem]{Korollar}
\newtheorem{proposition}[theorem]{Proposition}
\newtheorem{definition}[theorem]{Definition}
\newtheorem{example}[theorem]{Beispiel}
\newtheorem{remark}[theorem]{Bemerkung}
\newtheorem{note}[theorem]{Anmerkung}

% Additional environments
\newenvironment{treatise}{\begin{quote}}{\end{quote}}
\newenvironment{gemeinsam}{\begin{quote}}{\end{quote}}
\newenvironment{vergleich}{\begin{quote}}{\end{quote}}
\newenvironment{vorteil}{\begin{quote}}{\end{quote}}
\newenvironment{quantum}{\begin{quote}}{\end{quote}}

% T0-specific commands
\newcommand{\Tzero}{T$_0$}
\newcommand{\xipar}{\xi}
\newcommand{\Tfield}{T}
\newcommand{\Efield}{\mathcal{E}}
\newcommand{\meff}{m_{\text{eff}}}
\newcommand{\Eabs}{E_{\text{abs}}}
\newcommand{\taupar}{\tau}

% Header setup
\pagestyle{fancy}
\fancyhf{}
\fancyhead[L]{\leftmark}
\fancyhead[R]{\thepage}
\renewcommand{\headrulewidth}{0.4pt}

% Hyperref setup
\hypersetup{
    colorlinks=true,
    linkcolor=blue,
    filecolor=magenta,
    urlcolor=cyan,
    citecolor=blue,
    pdftitle={T0 Theory Document},
    pdfauthor={Johann Pascher}
}

% German quotation marks
%\newcommand{\dq}[1]{\glqq{}#1\grqq{}}


\title{Neutrino Formula}
\author{Johann Pascher}
\date{2025}

\begin{document}

\maketitle

\chapter{Neutrino Formula}

	
	
	\begin{abstract}
		This document presents a mathematically consistent Formel Struktur for Neutrino Berechnungen innerhalb the T0 Modell, basierend auf the Hypothese of equal masses for alle flavor Zustände (\(\nu_e, \nu_\mu, \nu_\Tau\)). The Neutrino Masse is derived from the Photon Analogie (\(\frac{\xipar^2}{2}\)-suppression), and Oszillationen are explained by geometrisch phases basierend auf \( T_x \cdot m_x = 1 \), with Quanten Zahlen (\(n, \ell, j\)) determining phase differences. A plausible target Wert for the Neutrino Masse (\(m_\nu = 15 \text{ meV}\)) is derived from empirical data (kosmologisch Einschränkungen). The T0 Modell is basierend auf speculative geometrisch harmonies without empirical support and is highly wahrscheinlich to be incomplete or inkorrekt. Scientific integrity requires a clear distinction zwischen mathematisch correctness and physikalisch validity.
	\end{abstract}
	
	\newpage
	
	\section{Preamble: Scientific Integrity}
	
	\begin{warning}
		\textbf{CRITICAL LIMITATION:} The folgend Formeln for Neutrino masses are \textbf{speculative extrapolations} basierend auf the untested Hypothese das Neutrinos follow geometrisch harmonies and alle flavor Zustände have equal masses. This Hypothese has \textbf{no empirical basis} and is highly wahrscheinlich to be incomplete or inkorrekt. The mathematisch Formeln are trotzdem internally consistent and error-free.
		
		\vspace{0.5cm}
		\textbf{Scientific Integrity Requires:}
		\begin{itemize}
			\item Honesty ungefähr the speculative nature of Vorhersagen
			\item Mathematical correctness trotz physikalisch Unschärfe
			\item Clear separation zwischen Hypothesen and verified facts
		\end{itemize}
	\end{warning}
	
	\section{Neutrinos as ''Near-Massless Photons'': The T0 Photon Analogy}
	
	\begin{speculation}
		\textbf{Fundamental T0 Insight:} Neutrinos can be understood as ''damped Photonen.''
		
		The remarkable similarity zwischen Photonen and Neutrinos suggests a deeper geometrisch kinship:
		\begin{itemize}
			\item \textbf{Speed:} Both propagate at nahezu the Geschwindigkeit of Licht
			\item \textbf{Penetration:} Both have extreme penetration capabilities
			\item \textbf{Mass:} Photon is exactly massless, Neutrino is nahezu massless
			\item \textbf{Interaction:} Photon interacts electromagnetically, Neutrino interacts weakly
		\end{itemize}
	\end{speculation}
	
	\subsection{Photon-Neutrino Correspondence}
	
	\begin{important}
		\textbf{Physical Parallels:}
		\begin{align}
			\text{Photon:} \quad &E^2 = (pc)^2 + 0 \quad \text{(perfectly massless)} \\
			\text{Neutrino:} \quad &E^2 = (pc)^2 + \left(\sqrt{\frac{\xipar^2}{2}} m c^2\right)^2 \quad \text{(nearly massless)}
		\end{align}
		
		\textbf{Speed Comparison:}
		\begin{align}
			v_\gamma &= c \quad \text{(exact)} \\
			v_\nu &= c \times \left(1 - \frac{\xipar^2}{2}\right) \approx 0.9999999911 \times c
		\end{align}
		
		The Geschwindigkeit difference is nur \(8.89 \times 10^{-9}\) -- practically unmeasurable!
	\end{important}
	
	\subsection{Double \(\xipar\)-Suppression from Photon Analogy}
	
	\begin{Formel}
		\textbf{T0 Hypothesis:} Neutrino = Photon with Geometric Double Damping
		
		If Neutrinos are ''near-Photonen,'' two suppression Faktoren arise:
		\begin{itemize}
			\item \textbf{First \(\xipar\) Factor:} ''Near massless'' (like a Photon, but not perfect)
			\item \textbf{Second \(\xipar\) Factor:} ''Weak Wechselwirkung'' (geometrisch Kopplung)
			\item \textbf{Result:} \(m_\nu \propto \frac{\xipar^2}{2}\), consistent with the Geschwindigkeit difference \(v_\nu = c \times \left(1 - \frac{\xipar^2}{2}\right)\)
		\end{itemize}
		
		\textbf{Interaction Strength Comparison:}
		\begin{align}
			\sigma_\gamma &\sim \alpha_{\text{EM}} \approx \frac{1}{137} \\
			\sigma_\nu &\sim \frac{\xipar^2}{2} \times G_F \approx 8.888888 \times 10^{-9}
		\end{align}
		
		The Verhältnis \(\sigma_\nu/\sigma_\gamma \sim \frac{\xipar^2}{2}\) confirms the geometrisch suppression!
	\end{Formel}
	
	\section{Neutrino Oscillations}
	
	\begin{important}
		\textbf{Neutrino Oscillations:} Neutrinos can change their identity (flavor) during flight -- a Phänomen known as Neutrino Oszillation. A Neutrino produced as an Elektron Neutrino (\(\nu_e\)) can later be detected as a Myon Neutrino (\(\nu_\mu\)) or Tau Neutrino (\(\nu_\Tau\)) and vice versa.
		
		In Standard physics, dies Verhalten is described by the mixing of Masse Eigenzustände (\(\nu_1, \nu_2, \nu_3\)) connected to flavor Zustände (\(\nu_e, \nu_\mu, \nu_\Tau\)) via the PMNS matrix (Pontecorvo-Maki-Nakagawa-Sakata):
		\begin{align}
			\begin{pmatrix}
				\nu_e \\ \nu_\mu \\ \nu_\tau
			\end{pmatrix}
			=
			U_{\text{PMNS}}
			\begin{pmatrix}
				\nu_1 \\ \nu_2 \\ \nu_3
			\end{pmatrix},
		\end{align}
		wo \(U_{\text{PMNS}}\) is the mixing matrix.
		
		Oscillations depend on Masse differences \(\Delta m^2_{ij} = m_i^2 - m_j^2\) and mixing angles. Current experimentell data (2025) provide:
		\begin{align}
			\Delta m^2_{21} &\approx 7.53 \times 10^{-5} \text{ eV}^2 \quad \text{[Solar]} \\
			\Delta m^2_{32} &\approx 2.44 \times 10^{-3} \text{ eV}^2 \quad \text{[Atmospheric]} \\
			m_\nu &> 0.06 \text{ eV} \quad \text{[At least one neutrino, 3}\sigma\text{]}
		\end{align}
		
		\textbf{Implications for T0:}
		\begin{itemize}
			\item The T0 Modell Postulate equal masses for flavor Zustände (\(\nu_e, \nu_\mu, \nu_\Tau\)), implying \(\Delta m^2_{ij} = 0\), welche is incompatible with Standard Oszillationen.
			\item To explain Oszillationen, the T0 Modell uses geometrisch phases basierend auf \( T_x \cdot m_x = 1 \), with Quanten Zahlen (\(n, \ell, j\)) determining phase differences.
		\end{itemize}
	\end{important}
	
	\subsection{Geometric Phases as Oscillation Mechanism}
	
	\begin{speculation}
		\textbf{T0 Hypothesis: Geometric Phases for Oscillations}
		
		To reconcile the Hypothese of equal masses (\(m_{\nu_e} = m_{\nu_\mu} = m_{\nu_\Tau} = m_\nu\)) with Neutrino Oszillationen, it is speculated das Oszillationen in the T0 Modell are caused by geometrisch phases eher than Masse differences. This is basierend auf the T0 Beziehung:
		\[
		T_x \cdot m_x = 1,
		\]
		wo \(m_x = m_\nu = 4.54 \text{ meV}\) is the Neutrino Masse, and \(T_x\) is a Charakteristik Zeit or Frequenz:
		\[
		T_x = \frac{1}{m_\nu} = \frac{1}{4.54 \times 10^{-3} \text{ eV}} \approx 2.2026 \times 10^2 \text{ eV}^{-1} \approx 1.449 \times 10^{-13} \text{ s}.
		\]
		
		The geometrisch phase is determined by the T0 Quanten Zahlen (\(n, \ell, j\)):
		\[
		\phi_{\text{geo}, i} \propto f(n, \ell, j) \cdot \frac{L}{E} \cdot \frac{1}{T_x},
		\]
		wo \(f(n, \ell, j) = \frac{n^6}{\ell^3}\) (or 1 for \(\ell = 0\)) are the geometrisch Faktoren:
		\begin{align}
			f_{\nu_e} &= 1, \\
			f_{\nu_\mu} &= 64, \\
			f_{\nu_\tau} &= 91.125.
		\end{align}
		
		\textbf{Calculated Phase Differences:}
		\begin{align}
			\phi_{\nu_e} &\propto 1 \cdot \frac{L}{E} \cdot \frac{1}{T_x}, \\
			\phi_{\nu_\mu} &\propto 64 \cdot \frac{L}{E} \cdot \frac{1}{T_x}, \\
			\phi_{\nu_\tau} &\propto 91.125 \cdot \frac{L}{E} \cdot \frac{1}{T_x}.
		\end{align}
		
		These phase differences could cause Oszillationen zwischen flavor Zustände without requiring unterschiedlich masses. The exakt form of the Oszillation Wahrscheinlichkeit requires further development but remains highly speculative.
		
		\textbf{WARNING:} This Ansatz is purely hypothetical and lacks empirical Bestätigung. It contradicts the established theory das Oszillationen are caused by \(\Delta m^2_{ij} \neq 0\).
	\end{speculation}
	
	\section{Fundamental Constants and Units}
	
	\subsection{Base Parameters}
	
	\begin{Formel}
		\textbf{T0 Base Constants:}
		\begin{align}
			\xipar &= \frac{4}{3} \times 10^{-4} \approx 1.333333 \times 10^{-4} \quad \text{[dimensionless]} \\
			\frac{\xipar^2}{2} &= \frac{\left(\frac{4}{3} \times 10^{-4}\right)^2}{2} \approx 8.888888 \times 10^{-9} \quad \text{[dimensionless]} \\
			v &= 246.22 \text{ GeV} \quad \text{[Higgs VEV]} \\
			\hbar c &= 0.19733 \text{ GeV·fm} \quad \text{[Conversion constant]} \\
			T_x &= \frac{1}{4.54 \times 10^{-3} \text{ eV}} \approx 2.2026 \times 10^2 \text{ eV}^{-1} \approx 1.449 \times 10^{-13} \text{ s} \quad \text{[T0 Mass]}
		\end{align}
	\end{Formel}
	
	\subsection{Unit Conventions}
	
	\begin{important}
		\textbf{Consistent Unit Hierarchy:}
		\begin{align}
			\text{Standard:} &\quad \text{GeV} \\
			\text{Submultiples:} &\quad 1 \text{ eV} = 10^{-9} \text{ GeV} \\
			&\quad 1 \text{ meV} = 10^{-12} \text{ GeV} = 10^{-3} \text{ eV} \\
			\text{Masses:} &\quad m[\text{GeV}/c^2] = E[\text{GeV}]/c^2 \approx E[\text{GeV}] \text{ (natural units)} \\
			\text{Time:} &\quad 1 \text{ eV}^{-1} \approx 6.582 \times 10^{-16} \text{ s}
		\end{align}
	\end{important}
	
	\section{Charged Lepton Reference Masses}
	
	\subsection{Precise Experimentell Values (PDG 2024)}
	
	\begin{experimentell}
		\textbf{Verified Particle Masses:}
		\begin{align}
			m_e &= 0.51099895000 \times 10^{-3} \text{ GeV} = 510.99895 \text{ keV} \\
			m_\mu &= 105.6583745 \times 10^{-3} \text{ GeV} = 105.6583745 \text{ MeV} \\
			m_\tau &= 1776.86 \times 10^{-3} \text{ GeV} = 1.77686 \text{ GeV}
		\end{align}
		
		\textbf{Unit Conversion to eV:}
		\begin{align}
			m_e &= 510998.95 \text{ eV} = 510998950 \text{ meV} \\
			m_\mu &= 105658374.5 \text{ eV} \\
			m_\tau &= 1776860000 \text{ eV}
		\end{align}
	\end{experimentell}
	
	\section{Neutrino Quantum Numbers (T0 Hypothesis)}
	
	\subsection{Postulated Quantum Number Assignment}
	
	\begin{speculation}
		\textbf{Hypothetical Neutrino Quantum Numbers:}
		\begin{align}
			\nu_e: &\quad n=1, \ell=0, j=1/2 \quad \text{[Ground state neutrino]} \\
			\nu_\mu: &\quad n=2, \ell=1, j=1/2 \quad \text{[First excitation]} \\
			\nu_\tau: &\quad n=3, \ell=2, j=1/2 \quad \text{[Second excitation]}
		\end{align}
		
		\textbf{Role of Quantum Numbers:}
		The Quanten Zahlen do not affect Neutrino masses (since \(m_{\nu_e} = m_{\nu_\mu} = m_{\nu_\Tau}\)) but determine the geometrisch Faktoren \(f(n, \ell, j)\), welche govern the Oszillation phases.
		
		\textbf{WARNING:} These assignments are purely speculative and lack experimentell basis.
	\end{speculation}
	
	\subsection{Geometric Factors}
	
	\begin{Formel}
		\textbf{T0 Geometric Factors:}
		\begin{align}
			f(n,\ell,j) &= \frac{n^6}{\ell^3} \quad \text{for } \ell > 0 \\
			f(1,0,j) &= 1 \quad \text{for } \ell = 0 \text{ (special case)}
		\end{align}
		
		\textbf{Calculated Values:}
		\begin{align}
			f_{\nu_e} &= f(1,0,1/2) = 1 \\
			f_{\nu_\mu} &= f(2,1,1/2) = \frac{2^6}{1^3} = 64 \\
			f_{\nu_\tau} &= f(3,2,1/2) = \frac{3^6}{2^3} = \frac{729}{8} = 91.125
		\end{align}
	\end{Formel}
	
	\section{Neutrino Mass Formula}
	
	\subsection{T0 Hypothesis: Equal Masses with Geometric Phases}
	
	\begin{speculation}
		\textbf{T0 Hypothesis: Equal Neutrino Masses with Geometric Phases}
		
		The T0 Modell Postulate das alle flavor Zustände (\(\nu_e, \nu_\mu, \nu_\Tau\)) have the gleich Masse:
		\[
		m_{\nu_e} = m_{\nu_\mu} = m_{\nu_\tau} = m_\nu = 4.54 \text{ meV}.
		\]
		The Masse is derived from the Photon Analogie:
		\[
		m_\nu = \frac{\xipar^2}{2} \times m_e = \left(8.888888 \times 10^{-9}\right) \times (0.51099895 \times 10^{-3} \text{ GeV}) = 4.54 \text{ meV}.
		\]
		
		To explain Oszillationen, a geometrisch Mechanismus is postulated basierend auf the T0 Beziehung:
		\[
		T_x \cdot m_x = 1, \quad m_x = 4.54 \text{ meV}, \quad T_x \approx 2.2026 \times 10^2 \text{ eV}^{-1} \approx 1.449 \times 10^{-13} \text{ s}.
		\]
		
		The Oszillation phases are determined by geometrisch Faktoren \(f(n, \ell, j)\):
		\[
		\phi_{\text{geo}, i} \propto f_{\nu_i} \cdot \frac{L}{E} \cdot \frac{1}{T_x},
		\]
		wo \(f_{\nu_e} = 1\), \(f_{\nu_\mu} = 64\), \(f_{\nu_\Tau} = 91.125\).
		
		\textbf{Rationale:}
		\begin{itemize}
			\item The Masse \(4.54 \text{ meV}\) is consistent with the kosmologisch Einschränkung (\(\Sigma m_\nu = 0.01362 \text{ eV} < 0.07 \text{ eV}\)).
			\item Geometric phases enable Oszillationen without Masse differences, supporting the equal-Masse Hypothese.
			\item This Hypothese is highly speculative and lacks empirical Bestätigung.
		\end{itemize}
	\end{speculation}
	
	\begin{Formel}
		\textbf{Formula:} \(m_{\nu_i} = 4.54 \text{ meV}\)
		
		\textbf{Total Mass:}
		\[
		\Sigma m_\nu = 3 \times 4.54 \text{ meV} = 13.62 \text{ meV} = 0.01362 \text{ eV}
		\]
		
		\textbf{Comparison with Plausible Target Value:}
		\begin{itemize}
			\item \(\nu_e, \nu_\mu, \nu_\Tau\): \(4.54 \text{ meV}\) vs. \(15 \text{ meV}\) (Agreement: \(30.3\%\))
			\item \(\Sigma m_\nu\): \(13.62 \text{ meV}\) vs. \(45 \text{ meV}\) (Deviation: Factor \(\approx 3.30\))
		\end{itemize}
	\end{Formel}
	
	\begin{warning}
		\textbf{CRITICAL FINDING:} The Hypothese of equal masses with geometrisch phases is incompatible with experimentell Oszillation data (\(\Delta m^2_{21} \approx 7.53 \times 10^{-5} \text{ eV}^2\), \(\Delta m^2_{32} \approx 2.44 \times 10^{-3} \text{ eV}^2\)), as it implies \(\Delta m^2_{ij} = 0\). The geometrisch Ansatz is purely speculative and requires further theoretisch and experimentell Validierung.
	\end{warning}
	
	\section{Plausible Target Value Basierend auf Empirical Data}
	
	\subsection{Derivation from Measurements}
	
	\begin{experimentell}
		\textbf{Plausible Target Value:}
		The T0 Modell Postulate equal masses for alle flavor Zustände (\(\nu_e, \nu_\mu, \nu_\Tau\)). Thus, a single target Wert for the Neutrino Masse \(m_\nu\) is derived basierend auf empirical data (as of 2025):
		\begin{itemize}
			\item Cosmological Constraint: \(\Sigma m_\nu = 3 m_\nu < 0.07 \text{ eV} \implies m_\nu < 23.33 \text{ meV}\).
			\item Oscillation Data: \(\Delta m^2_{21} \approx 7.53 \times 10^{-5} \text{ eV}^2\), \(\Delta m^2_{32} \approx 2.44 \times 10^{-3} \text{ eV}^2\), typisch requiring unterschiedlich masses. The T0 Modell bypasses dies via geometrisch phases.
			\item Plausible Target Value: \(m_\nu \approx 15 \text{ meV}\), lying zwischen the solar (\(8.68 \text{ meV}\)) and atmospheric Skalen (\(50.15 \text{ meV}\)) and satisfying the kosmologisch Einschränkung:
			\[
			\Sigma m_\nu = 3 \times 15 \text{ meV} = 45 \text{ meV} = 0.045 \text{ eV} < 0.07 \text{ eV}.
			\]
		\end{itemize}
		
		\textbf{Rationale:}
		\begin{itemize}
			\item The target Wert is consistent with the kosmologisch Einschränkung and lies innerhalb the Ordnung of Größenordnung of Oszillation data.
			\item The equal-Masse Hypothese is supported by geometrisch phases, distinguishing the T0 Modell from Standard physics.
			\item The Wert is plausible but not direkt gemessen, as flavor masses are mixtures of Eigenzustände.
			\item The T0 Masse (\(4.54 \text{ meV}\)) is unten the target Wert (\(30.3\%\)) but auch cosmologically consistent.
		\end{itemize}
	\end{experimentell}
	
	\section{Experimentell Comparison}
	
	\subsection{Current Experimentell Upper Limits (2025)}
	
	\begin{experimentell}
		\textbf{Experimentell Limits:}
		\begin{align}
			m_{\nu_e} &< 0.45 \text{ eV} \quad \text{[KATRIN, 90\% CL]} \\
			m_{\nu_\mu} &< 0.17 \text{ MeV} \quad \text{[Muon decay, indirect]} \\
			m_{\nu_\tau} &< 18.2 \text{ MeV} \quad \text{[Tau decay, indirect]} \\
			\Sigma m_\nu &< 0.07 \text{ eV} \quad \text{[DESI+Planck, 95\% CL]} \\
			\Delta m^2_{21} &\approx 7.53 \times 10^{-5} \text{ eV}^2 \quad \text{[Solar]} \\
			\Delta m^2_{32} &\approx 2.44 \times 10^{-3} \text{ eV}^2 \quad \text{[Atmospheric]} \\
			m_\nu &> 0.06 \text{ eV} \quad \text{[At least one neutrino, 3}\sigma\text{]}
		\end{align}
	\end{experimentell}
	
	\subsection{Safety Margins for T0 Hypothesis}
	
	\begin{longtable}[c]{@{}lcc@{}}
		\caption{Safety Margins of the T0 Hypothesis Against Experimentell Limits} \\
		\toprule
		\textbf{Parameter} & \textbf{T0 Mass (\(4.54 \text{ meV}\))} & \textbf{Target Value (\(15 \text{ meV}\))} \\
		\midrule
		\endfirsthead
		\toprule
		\textbf{Parameter} & \textbf{T0 Mass (\(4.54 \text{ meV}\))} & \textbf{Target Value (\(15 \text{ meV}\))} \\
		\midrule
		\endhead
		$m_{\nu_e}$ vs 0.45 eV & 99200× & 30× \\
		$m_{\nu_\mu}$ vs 0.17 MeV & 3.74E7× & 11333× \\
		$m_{\nu_\tau}$ vs 18.2 MeV & 4.01E9× & 1.21E6× \\
		\midrule
		$\Sigma m_\nu$ vs 0.07 eV & 5.14× & 1.56× \\
		$\Sigma m_\nu$ vs 0.06 eV & 4.41× & 1.33× \\
		\bottomrule
	\end{longtable}
	
	\begin{important}
		\textbf{T0 Hypothesis:}
		\begin{itemize}
			\item The T0 Masse (\(4.54 \text{ meV}\)) is consistent with kosmologisch Einschränkungen (\(\Sigma m_\nu = 0.01362 \text{ eV} < 0.07 \text{ eV}\)) and lies unten the target Wert (\(15 \text{ meV}\), \(30.3\%\)).
			\item Geometric phases (\(T_x \cdot m_x = 1\)) provide a speculative Mechanismus for Oszillationen but are incompatible with Standard Oszillationen.
			\item Physical Rationale: The Masse is basierend auf \(\frac{\xipar^2}{2}\)-suppression, consistent with the Geschwindigkeit difference \(v_\nu = c \times \left(1 - \frac{\xipar^2}{2}\right)\).
		\end{itemize}
	\end{important}
	
	\section{Consistency Checks and Validation}
	
	\subsection{Dimensional Analysis}
	
	\begin{Formel}
		\textbf{Dimensional Consistency:}
		\begin{align}
			[\xipar] &= 1 \quad \checkmark \text{ dimensionless} \\
			[m_e] &= \text{GeV} \quad \checkmark \text{ energy/mass} \\
			\left[\frac{\xipar^2}{2} \times m_e\right] &= \text{GeV} \quad \checkmark \text{ energy/mass} \\
			[f_{\nu_i}] &= 1 \quad \checkmark \text{ dimensionless} \\
			[m_\nu] &= \text{eV} \quad \checkmark \text{ (fixed mass)} \\
			[T_x] &= \text{eV}^{-1} \quad \checkmark \text{ (time)}
		\end{align}
		All Formeln are dimensionally consistent.
	\end{Formel}
	
	\subsection{Mathematical Consistency}
	
	\begin{important}
		\textbf{Consistency of the Hypothesis:}
		\begin{itemize}
			\item The Formel \(m_\nu = \frac{\xipar^2}{2} \times m_e = 4.54 \text{ meV}\) is physically grounded in the Photon Analogie and consistent with the Geschwindigkeit difference.
			\item Geometric phases basierend auf \(f(n, \ell, j)\) and \(T_x \cdot m_x = 1\) provide a speculative Mechanismus for Oszillationen.
			\item No free Parameter except \(\xipar\), simplifying the theory.
		\end{itemize}
	\end{important}
	
	\subsection{Experimentell Validation}
	
	\begin{experimentell}
		\textbf{Validation Status (as of 2025):}
		\begin{itemize}
			\item The T0 Masse (\(4.54 \text{ meV}\)) satisfies kosmologisch Einschränkungen (\(\Sigma m_\nu = 0.01362 \text{ eV} < 0.07 \text{ eV}\)) and is close to the target Wert (\(15 \text{ meV}\), \(30.3\%\)).
			\item Incompatible with Standard Oszillationen (\(\Delta m^2_{ij} = 0\)), but geometrisch phases offer a speculative workaround.
			\item The target Wert (\(15 \text{ meV}\)) is consistent with kosmologisch Einschränkungen but not direkt gemessen.
		\end{itemize}
	\end{experimentell}
	
	\section{Schlussfolgerung}
	
	\begin{important}
		\textbf{Zusammenfassung and Outlook:}
		\begin{itemize}
			\item The T0 Modell Postulate equal Neutrino masses (\(m_\nu = 4.54 \text{ meV}\)) basierend auf the Photon Analogie (\(\frac{\xipar^2}{2} \times m_e\)), consistent with the Geschwindigkeit difference (\(v_\nu = c \times \left(1 - \frac{\xipar^2}{2}\right)\)).
			\item Geometric phases basierend auf \(T_x \cdot m_x = 1\) and Quanten Zahlen (\(f_{\nu_e} = 1\), \(f_{\nu_\mu} = 64\), \(f_{\nu_\Tau} = 91.125\)) speculatively explain Oszillationen without Masse differences.
			\item The plausible target Wert (\(m_\nu = 15 \text{ meV}\)) is derived from empirical data (kosmologisch Einschränkung) and lies innerhalb the Ordnung of Größenordnung of Oszillation data but is not direkt gemessen.
			\item The T0 Masse (\(4.54 \text{ meV}\)) is reasonably close to the target Wert (\(30.3\%\)), satisfies kosmologisch Einschränkungen, but is incompatible with Standard Oszillationen.
			\item The T0 Modell remains speculative, relying on geometrisch harmonies without empirical basis.
			\item Future Experimente (2025–2030, e.g., KATRIN upgrade, DESI, Euclid) could further test or refute the T0 Hypothese, besonders the geometrisch Oszillation Mechanismus.
			\item Scientific integrity requires klar communicating the speculative nature of the T0 Modell and awaiting further tests.
		\end{itemize}
	\end{important}
	

\begin{thebibliography}{99}

% ============================================
% Core T0 Theory References (J. Pascher)
% GitHub Repository: https://github.com/jpascher/T0-Time-Mass-Duality
% ============================================

\bibitem{pascher2024}
J. Pascher, \emph{T0 Theory: Time-Mass Duality}, 2024.
\url{https://github.com/jpascher/T0-Time-Mass-Duality/blob/main/2/pdf/T0_unified_report.pdf}

\bibitem{pascher2025t0}
J. Pascher, \emph{T0 Theory: Fundamentals}, 2025.
\url{https://github.com/jpascher/T0-Time-Mass-Duality/blob/main/2/pdf/T0_Grundlagen_En.pdf}

\bibitem{pascher2025qm}
J. Pascher, \emph{T0 Theory: Quantum Mechanics}, 2025.
\url{https://github.com/jpascher/T0-Time-Mass-Duality/blob/main/2/pdf/QM_En.pdf}

\bibitem{pascher2025si}
J. Pascher, \emph{T0 Theory: SI Units}, 2025.
\url{https://github.com/jpascher/T0-Time-Mass-Duality/blob/main/2/pdf/T0_SI_En.pdf}

\bibitem{pascher2025g2}
J. Pascher, \emph{T0 Theory: The g-2 Anomaly}, 2025.
\url{https://github.com/jpascher/T0-Time-Mass-Duality/blob/main/2/pdf/T0_Anomale-g2-9_En.pdf}

\bibitem{pascher2025cmb}
J. Pascher, \emph{T0 Theory: CMB Analysis}, 2025.
\url{https://github.com/jpascher/T0-Time-Mass-Duality/blob/main/2/pdf/Zwei-Dipole-CMB_En.pdf}

% Historical Physics
\bibitem{einstein1905}
A. Einstein, \emph{On the Electrodynamics of Moving Bodies}, Annalen der Physik, 1905.
\url{https://doi.org/10.1002/andp.19053221004}

\bibitem{dirac1928}
P.A.M. Dirac, \emph{The Quantum Theory of the Electron}, Proc. Roy. Soc. A, 1928.
\url{https://doi.org/10.1098/rspa.1928.0023}

\bibitem{planck1900}
M. Planck, \emph{On the Theory of the Energy Distribution Law}, 1900.
\url{https://doi.org/10.1002/andp.19013090310}

\bibitem{mach1883}
E. Mach, \emph{Die Mechanik in ihrer Entwicklung}, 1883.

\bibitem{hundert1931}
Various Authors, \emph{100 Authors Against Einstein}, 1931.

\bibitem{dingle1972}
H. Dingle, \emph{Science at the Crossroads}, 1972.

% Penrose and Terrell Effect
\bibitem{terrell1959}
J. Terrell, \emph{Invisibility of the Lorentz Contraction}, Phys. Rev., 1959.
\url{https://doi.org/10.1103/PhysRev.116.1041}

\bibitem{penrose1959}
R. Penrose, \emph{The Apparent Shape of a Relativistically Moving Sphere}, Proc. Cambridge Phil. Soc., 1959.
\url{https://doi.org/10.1017/S0305004100033776}

\bibitem{penrose1967}
R. Penrose, \emph{Twistor Algebra}, J. Math. Phys., 1967.
\url{https://doi.org/10.1063/1.1705200}

\bibitem{penrose2004}
R. Penrose, \emph{The Road to Reality}, 2004.

\bibitem{terrell2025}
J. Terrell et al., \emph{Modern Terrell-Penrose Visualization}, 2025.

\bibitem{weiskopf2000}
D. Weiskopf, \emph{Visualization of Four-dimensional Spacetimes}, 2000.

\bibitem{mueller2014}
T. Müller, \emph{Visual Appearance of Relativistically Moving Objects}, 2014.

\bibitem{hossenfelder2025}
S. Hossenfelder, \emph{YouTube: The Terrell Effect}, 2025.

% Quantum Gravity and String Theory
\bibitem{rovelli2004}
C. Rovelli, \emph{Quantum Gravity}, Cambridge University Press, 2004.

\bibitem{thiemann2007}
T. Thiemann, \emph{Modern Canonical Quantum Gravity}, Cambridge University Press, 2007.

\bibitem{ashtekar2004}
A. Ashtekar, J. Lewandowski, \emph{Background Independent Quantum Gravity}, Class. Quant. Grav., 2004.
\url{https://doi.org/10.1088/0264-9381/21/15/R01}

\bibitem{jacobson1995}
T. Jacobson, \emph{Thermodynamics of Spacetime}, Phys. Rev. Lett., 1995.
\url{https://doi.org/10.1103/PhysRevLett.75.1260}

\bibitem{maldacena1998}
J. Maldacena, \emph{The Large N Limit of Superconformal Field Theories}, Adv. Theor. Math. Phys., 1998.
\url{https://doi.org/10.4310/ATMP.1998.v2.n2.a1}

\bibitem{polchinski1998}
J. Polchinski, \emph{String Theory}, Cambridge University Press, 1998.

\bibitem{susskind1995}
L. Susskind, \emph{The World as a Hologram}, J. Math. Phys., 1995.
\url{https://doi.org/10.1063/1.531249}

\bibitem{verlinde2011}
E. Verlinde, \emph{On the Origin of Gravity}, JHEP, 2011.
\url{https://doi.org/10.1007/JHEP04(2011)029}

% Cosmology
\bibitem{hoyle1948}
F. Hoyle, \emph{A New Model for the Expanding Universe}, MNRAS, 1948.
\url{https://doi.org/10.1093/mnras/108.5.372}

\bibitem{bondi1948}
H. Bondi, T. Gold, \emph{The Steady-State Theory}, MNRAS, 1948.
\url{https://doi.org/10.1093/mnras/108.3.252}

\bibitem{zwicky1929}
F. Zwicky, \emph{On the Redshift of Spectral Lines}, Proc. Nat. Acad. Sci., 1929.
\url{https://doi.org/10.1073/pnas.15.10.773}

\bibitem{lopez2010}
C. Lopez-Corredoira, \emph{Tests of Cosmological Models}, Int. J. Mod. Phys. D, 2010.

\bibitem{lerner2014}
E. Lerner, \emph{Evidence for a Non-Expanding Universe}, 2014.

\bibitem{albrecht1999}
A. Albrecht, J. Magueijo, \emph{Variable Speed of Light}, Phys. Rev. D, 1999.
\url{https://doi.org/10.1103/PhysRevD.59.043516}

\bibitem{barrow1999}
J. Barrow, \emph{Cosmologies with Varying Light Speed}, Phys. Rev. D, 1999.
\url{https://doi.org/10.1103/PhysRevD.59.043515}

\bibitem{riess2022}
A. Riess et al., \emph{A Comprehensive Measurement of the Local Value of the Hubble Constant}, ApJ, 2022.
\url{https://doi.org/10.3847/2041-8213/ac5c5b}

\bibitem{desi2025}
DESI Collaboration, \emph{DESI Year 1 Results}, 2025.
\url{https://arxiv.org/abs/2404.03002}

\bibitem{divalentino2021}
E. Di Valentino et al., \emph{Planck Evidence for a Closed Universe}, Nat. Astron., 2021.
\url{https://doi.org/10.1038/s41550-019-0906-9}

% Conformal Field Theory
\bibitem{francesco1997}
P. Di Francesco et al., \emph{Conformal Field Theory}, Springer, 1997.

% Experimental Physics
\bibitem{pdg2024}
Particle Data Group, \emph{Review of Particle Physics}, 2024.
\url{https://pdg.lbl.gov/}

\bibitem{codata2019}
CODATA, \emph{Recommended Values of Fundamental Constants}, 2019.
\url{https://physics.nist.gov/cuu/Constants/}

\bibitem{newell2018}
D. Newell et al., \emph{The CODATA 2017 Values of h, e, k, and $N_A$}, Metrologia, 2018.
\url{https://doi.org/10.1088/1681-7575/aa950a}

\bibitem{muong2_2023}
Muon g-2 Collaboration, \emph{Measurement of the Anomalous Magnetic Moment of the Muon}, Phys. Rev. Lett., 2023.
\url{https://doi.org/10.1103/PhysRevLett.131.161802}

\bibitem{fermilab2023}
Fermilab, \emph{Muon g-2 Results}, 2023.
\url{https://muon-g-2.fnal.gov/}

\bibitem{atlas2023}
ATLAS Collaboration, \emph{Measurements at the LHC}, 2023.
\url{https://atlas.cern/}

\bibitem{atlas2023higgs}
ATLAS Collaboration, \emph{Higgs Boson Properties}, 2023.
\url{https://atlas.cern/}

\bibitem{cms2023top}
CMS Collaboration, \emph{Top Quark Measurements}, 2023.
\url{https://cms.cern/}

\bibitem{cms2024}
CMS Collaboration, \emph{Heavy Ion Collisions}, 2024.
\url{https://cms.cern/}

\bibitem{alice2023}
ALICE Collaboration, \emph{Quark-Gluon Plasma Studies}, 2023.
\url{https://alice-collaboration.web.cern.ch/}

\bibitem{kasevich2023}
M. Kasevich et al., \emph{Atom Interferometry}, 2023.

\bibitem{ludlow2015}
A. Ludlow et al., \emph{Optical Atomic Clocks}, Rev. Mod. Phys., 2015.
\url{https://doi.org/10.1103/RevModPhys.87.637}

\bibitem{brewer2019}
S. Brewer et al., \emph{Al$^+$ Optical Clock}, Phys. Rev. Lett., 2019.
\url{https://doi.org/10.1103/PhysRevLett.123.033201}

\bibitem{lisa2017}
LISA Collaboration, \emph{LISA Mission}, 2017.
\url{https://www.lisamission.org/}

% Fractal Physics
\bibitem{nottale1993}
L. Nottale, \emph{Fractal Space-Time and Microphysics}, World Scientific, 1993.

\bibitem{elnaschie2004}
M.S. El Naschie, \emph{E-Infinity Theory}, Chaos Solitons Fractals, 2004.

% Philosophy and Foundations
\bibitem{wheeler1990}
J.A. Wheeler, \emph{Information, Physics, Quantum}, 1990.

\bibitem{barbour1999}
J. Barbour, \emph{The End of Time}, Oxford University Press, 1999.

\bibitem{sciama1953}
D. Sciama, \emph{On the Origin of Inertia}, MNRAS, 1953.
\url{https://doi.org/10.1093/mnras/113.1.34}

% String Theory Extensions
\bibitem{becker2007}
K. Becker et al., \emph{String Theory and M-Theory}, Cambridge University Press, 2007.

% Missing References for g-2 Chapter
\bibitem{sm_g2_2025}
Muon g-2 Theory Initiative, \emph{Standard Model Prediction for g-2}, arXiv, 2025.
\url{https://arxiv.org/abs/2006.04822}

\bibitem{mug2_final_2025}
Muon g-2 Collaboration, \emph{Final Report on the Anomalous Magnetic Moment of the Muon}, Fermilab, 2025.
\url{https://muon-g-2.fnal.gov/}

\bibitem{pascher_t0_theory_2025}
J. Pascher, \emph{T0 Theory: Complete Framework}, 2025.
\url{https://github.com/jpascher/T0-Time-Mass-Duality/blob/main/2/pdf/systemEn.pdf}

\bibitem{peskin_schroeder_1995}
M.E. Peskin and D.V. Schroeder, \emph{An Introduction to Quantum Field Theory}, Westview Press, 1995.

\bibitem{parker_somov_2018}
R.H. Parker et al., \emph{Measurement of the Fine-Structure Constant}, Science, 2018.
\url{https://doi.org/10.1126/science.aap7706}

\bibitem{morel_rubidium_2020}
L. Morel et al., \emph{Determination of $\alpha$ from Rubidium Atom Recoil}, Nature, 2020.
\url{https://doi.org/10.1038/s41586-020-2964-7}

\bibitem{aoyama_theory_2020}
T. Aoyama et al., \emph{Theory of the Electron Anomalous Magnetic Moment}, Phys. Rep., 2020.
\url{https://doi.org/10.1016/j.physrep.2020.07.006}

\bibitem{fan_lattice_2023}
X. Fan et al., \emph{Hadronic Contributions from Lattice QCD}, Phys. Rev. D, 2023.

\bibitem{hanneke_electron_2008}
D. Hanneke et al., \emph{New Measurement of the Electron g-2}, Phys. Rev. Lett., 2008.
\url{https://doi.org/10.1103/PhysRevLett.100.120801}

% Additional T0 Theory References
\bibitem{pascher_higgs_connection_2025}
J. Pascher, \emph{Higgs Connection in T0 Theory}, 2025.
\url{https://github.com/jpascher/T0-Time-Mass-Duality/blob/main/2/pdf/T0_Energie_En.pdf}

\bibitem{T0_SI}
J. Pascher, \emph{T0 Theory and SI Units}, 2025.
\url{https://github.com/jpascher/T0-Time-Mass-Duality/blob/main/2/pdf/T0_SI_En.pdf}

\bibitem{T0_gravitational_constant}
J. Pascher, \emph{Gravitational Constant in T0 Framework}, 2025.
\url{https://github.com/jpascher/T0-Time-Mass-Duality/blob/main/2/pdf/T0_Gravitationskonstante_En.pdf}

\bibitem{T0_fine_structure}
J. Pascher, \emph{Fine Structure Constant Analysis}, 2025.
\url{https://github.com/jpascher/T0-Time-Mass-Duality/blob/main/2/pdf/T0_Feinstruktur_En.pdf}

\bibitem{bell_muon}
J.S. Bell, \emph{Muon Studies}, 1966.

\bibitem{QFT_T0}
J. Pascher, \emph{Quantum Field Theory in T0}, 2025.
\url{https://github.com/jpascher/T0-Time-Mass-Duality/blob/main/2/pdf/QFT_En.pdf}

\bibitem{planck2018}
Planck Collaboration, \emph{Planck 2018 Results}, A\&A, 2018.
\url{https://doi.org/10.1051/0004-6361/201833910}

\bibitem{pascher:t0_foundations}
J. Pascher, \emph{T0 Theory Foundations}, 2025.
\url{https://github.com/jpascher/T0-Time-Mass-Duality/blob/main/2/pdf/T0_Grundlagen_En.pdf}

\bibitem{pascher:geometric_formalism}
J. Pascher, \emph{Geometric Formalism in T0}, 2025.
\url{https://github.com/jpascher/T0-Time-Mass-Duality/blob/main/2/pdf/T0_Geometrische_Kosmologie_En.pdf}

\bibitem{riess2019}
A. Riess et al., \emph{Hubble Constant Measurements}, ApJ, 2019.
\url{https://doi.org/10.3847/1538-4357/ab1422}

\bibitem{t0_kosmologie}
J. Pascher, \emph{T0 Kosmologie}, 2025.
\url{https://github.com/jpascher/T0-Time-Mass-Duality/blob/main/2/pdf/T0_Kosmologie_En.pdf}

\bibitem{hossenfelder_single_clock_video}
S. Hossenfelder, \emph{Single Clock Video}, YouTube, 2025.
\url{https://www.youtube.com/c/SabineHossenfelder}

\bibitem{video2025}
Various, \emph{Video References}, 2025.

\bibitem{unnikrishnan2004}
C.S. Unnikrishnan, \emph{Gravity Studies}, 2004.

\bibitem{peratt1992}
A. Peratt, \emph{Plasma Cosmology}, 1992.
\url{https://github.com/jpascher/T0-Time-Mass-Duality/blob/main/2/pdf/T0_peratt_En.pdf}

\bibitem{T0_tm_erweiterung}
J. Pascher, \emph{T0 Time-Mass Extension}, 2025.
\url{https://github.com/jpascher/T0-Time-Mass-Duality/blob/main/2/pdf/T0_tm-erweiterung-x6_En.pdf}

\bibitem{T0_g2_erweiterung}
J. Pascher, \emph{T0 g-2 Extension}, 2025.
\url{https://github.com/jpascher/T0-Time-Mass-Duality/blob/main/2/pdf/T0_g2-erweiterung-4_En.pdf}

\bibitem{T0_netze_en}
J. Pascher, \emph{T0 Networks}, 2025.
\url{https://github.com/jpascher/T0-Time-Mass-Duality/blob/main/2/pdf/T0_netze_En.pdf}

\bibitem{Adams1925}
W. Adams, \emph{Gravitational Redshift}, 1925.
\url{https://doi.org/10.1073/pnas.11.7.382}

\bibitem{Ashby2003}
N. Ashby, \emph{Relativity in GPS}, Living Rev. Rel., 2003.
\url{https://doi.org/10.12942/lrr-2003-1}

\bibitem{Bertotti2003}
B. Bertotti et al., \emph{Cassini Doppler Test}, Nature, 2003.
\url{https://doi.org/10.1038/nature01997}

\bibitem{Bolton2008}
A. Bolton et al., \emph{Gravitational Lensing}, 2008.

\bibitem{Born2013}
M. Born, \emph{Einstein's Theory of Relativity}, Dover, 2013.

\bibitem{Brans1961}
C. Brans and R.H. Dicke, \emph{Mach's Principle}, Phys. Rev., 1961.
\url{https://doi.org/10.1103/PhysRev.124.925}

\bibitem{Dirac1927}
P.A.M. Dirac, \emph{Quantum Mechanics}, Proc. Roy. Soc., 1927.
\url{https://doi.org/10.1098/rspa.1927.0039}

\bibitem{Duhem1906}
P. Duhem, \emph{Theory of Physics}, 1906.

\bibitem{Einstein1905}
A. Einstein, \emph{Special Relativity}, Ann. Phys., 1905.
\url{https://doi.org/10.1002/andp.19053221004}

\bibitem{Feynman2006}
R. Feynman, \emph{QED: The Strange Theory of Light and Matter}, 2006.

\bibitem{Griffiths2017}
D. Griffiths, \emph{Introduction to Quantum Mechanics}, 2017.

\bibitem{Jackson1999}
J.D. Jackson, \emph{Classical Electrodynamics}, 1999.

\bibitem{Kaluza1921}
T. Kaluza, \emph{Five-Dimensional Theory}, 1921.

\bibitem{Klein1926}
O. Klein, \emph{Quantum Theory and Relativity}, 1926.

\bibitem{Kuhn1962}
T. Kuhn, \emph{Structure of Scientific Revolutions}, 1962.

\bibitem{Kuhn1977}
T. Kuhn, \emph{Essential Tension}, 1977.

\bibitem{Ludlow2015}
A. Ludlow et al., \emph{Optical Atomic Clocks}, Rev. Mod. Phys., 2015.
\url{https://doi.org/10.1103/RevModPhys.87.637}

\bibitem{Maxwell1873}
J.C. Maxwell, \emph{Treatise on Electricity and Magnetism}, 1873.

\bibitem{McGaugh2016}
S. McGaugh et al., \emph{Radial Acceleration Relation}, Phys. Rev. Lett., 2016.
\url{https://doi.org/10.1103/PhysRevLett.117.201101}

\bibitem{Mohr2016}
P. Mohr et al., \emph{CODATA Values}, Rev. Mod. Phys., 2016.
\url{https://doi.org/10.1103/RevModPhys.88.035009}

\bibitem{PDG2020}
Particle Data Group, \emph{Review of Particle Physics}, Prog. Theor. Exp. Phys., 2020.
\url{https://pdg.lbl.gov/}

\bibitem{Parker2018}
R. Parker et al., \emph{Measurement of $\alpha$}, Science, 2018.
\url{https://doi.org/10.1126/science.aap7706}

\bibitem{Peskin1995}
M. Peskin and D. Schroeder, \emph{QFT}, 1995.

\bibitem{Planck1900}
M. Planck, \emph{Quantum Theory}, 1900.

\bibitem{Planck2020}
Planck Collaboration, \emph{Planck 2020 Results}, 2020.
\url{https://doi.org/10.1051/0004-6361/201833910}

\bibitem{Poincare1905}
H. Poincaré, \emph{Dynamics of the Electron}, 1905.

\bibitem{Pound1960}
R.V. Pound and G.A. Rebka, \emph{Gravitational Redshift}, Phys. Rev. Lett., 1960.
\url{https://doi.org/10.1103/PhysRevLett.4.337}

\bibitem{Quine1951}
W.V. Quine, \emph{Two Dogmas of Empiricism}, 1951.

\bibitem{Quinn2013}
T. Quinn et al., \emph{Gravitational Constant}, 2013.
\url{https://doi.org/10.1103/PhysRevLett.111.101102}

\bibitem{Randall1999}
L. Randall and R. Sundrum, \emph{Extra Dimensions}, Phys. Rev. Lett., 1999.
\url{https://doi.org/10.1103/PhysRevLett.83.3370}

\bibitem{Riess1998}
A. Riess et al., \emph{Type Ia Supernovae}, AJ, 1998.
\url{https://doi.org/10.1086/300499}

\bibitem{Shapiro1971}
I. Shapiro et al., \emph{Time Delay Test}, Phys. Rev. Lett., 1971.
\url{https://doi.org/10.1103/PhysRevLett.26.1132}

\bibitem{Sommerfeld1916}
A. Sommerfeld, \emph{Fine Structure}, 1916.

\bibitem{Suyu2017}
S. Suyu et al., \emph{Time Delay Cosmography}, MNRAS, 2017.
\url{https://doi.org/10.1093/mnras/stx483}

\bibitem{T0Theory}
J. Pascher, \emph{T0 Theory}, 2025.
\url{https://github.com/jpascher/T0-Time-Mass-Duality/blob/main/2/pdf/systemEn.pdf}

\bibitem{T0_Feinstruktur}
J. Pascher, \emph{Fine Structure in T0}, 2025.
\url{https://github.com/jpascher/T0-Time-Mass-Duality/blob/main/2/pdf/T0_Feinstruktur_En.pdf}

\bibitem{Uzan2003}
J.-P. Uzan, \emph{Constants Variation}, Rev. Mod. Phys., 2003.
\url{https://doi.org/10.1103/RevModPhys.75.403}

\bibitem{Webb2001}
J.K. Webb et al., \emph{Fine Structure Constant}, Phys. Rev. Lett., 2001.
\url{https://doi.org/10.1103/PhysRevLett.87.091301}

\bibitem{Weinberg1979}
S. Weinberg, \emph{Cosmological Constant}, Rev. Mod. Phys., 1979.

\bibitem{Weinberg1989}
S. Weinberg, \emph{Cosmological Constant Problem}, 1989.
\url{https://doi.org/10.1103/RevModPhys.61.1}

\bibitem{Weinberg1995}
S. Weinberg, \emph{Quantum Theory of Fields}, 1995.

\bibitem{Will2014}
C. Will, \emph{Theory and Experiment in Gravitational Physics}, 2014.
\url{https://doi.org/10.12942/lrr-2014-4}

\bibitem{dirac_principles}
P.A.M. Dirac, \emph{Principles of Quantum Mechanics}, 1930.

\bibitem{einstein_1917}
A. Einstein, \emph{Cosmological Considerations}, 1917.

\bibitem{jwst_early}
JWST Collaboration, \emph{Early Universe Observations}, 2023.
\url{https://www.jwst.nasa.gov/}

\bibitem{katrin_2022}
KATRIN Collaboration, \emph{Neutrino Mass}, 2022.
\url{https://doi.org/10.1038/s41567-021-01463-1}

\bibitem{pascher:fundamentals}
J. Pascher, \emph{T0 Fundamentals}, 2025.
\url{https://github.com/jpascher/T0-Time-Mass-Duality/blob/main/2/pdf/T0_Grundlagen_En.pdf}

\bibitem{pascher:g2_rev9}
J. Pascher, \emph{g-2 Analysis Rev9}, 2025.
\url{https://github.com/jpascher/T0-Time-Mass-Duality/blob/main/2/pdf/T0_Anomale-g2-9_En.pdf}

\bibitem{pascher:ml_addendum}
J. Pascher, \emph{ML Addendum}, 2025.
\url{https://github.com/jpascher/T0-Time-Mass-Duality/blob/main/2/pdf/T0-QFT-ML_Addendum_En.pdf}

\bibitem{pascher_beta_derivation_2025}
J. Pascher, \emph{Beta Derivation}, 2025.
\url{https://github.com/jpascher/T0-Time-Mass-Duality/blob/main/2/pdf/DerivationVonBetaEn.pdf}

\bibitem{pascher_cmb_en}
J. Pascher, \emph{CMB Analysis in T0}, 2025.
\url{https://github.com/jpascher/T0-Time-Mass-Duality/blob/main/2/pdf/Zwei-Dipole-CMB_En.pdf}

\bibitem{pascher_cosmos_en}
J. Pascher, \emph{Cosmos in T0 Theory}, 2025.
\url{https://github.com/jpascher/T0-Time-Mass-Duality/blob/main/2/pdf/cosmic_En.pdf}

\bibitem{pascher_derivation_beta_2025}
J. Pascher, \emph{Derivation of Beta}, 2025.
\url{https://github.com/jpascher/T0-Time-Mass-Duality/blob/main/2/pdf/DerivationVonBetaEn.pdf}

\bibitem{pascher_gravitation_en}
J. Pascher, \emph{Gravitation in T0}, 2025.
\url{https://github.com/jpascher/T0-Time-Mass-Duality/blob/main/2/pdf/gravitationskonstante_En.pdf}

\bibitem{pascher_lagrangian_2025}
J. Pascher, \emph{Lagrangian in T0}, 2025.
\url{https://github.com/jpascher/T0-Time-Mass-Duality/blob/main/2/pdf/T0_lagrndian_En.pdf}

\bibitem{pascher_lagrangian_en}
J. Pascher, \emph{Lagrangian Framework}, 2025.
\url{https://github.com/jpascher/T0-Time-Mass-Duality/blob/main/2/pdf/LagrandianVergleichEn.pdf}

\bibitem{pascher_lagrangian_extended_2025}
J. Pascher, \emph{Extended Lagrangian Formalism}, 2025.
\url{https://github.com/jpascher/T0-Time-Mass-Duality/blob/main/2/pdf/T0_lagrndian_En.pdf}

\bibitem{pascher_mathematical_structure_2025}
J. Pascher, \emph{Mathematical Structure of T0 Theory}, 2025.
\url{https://github.com/jpascher/T0-Time-Mass-Duality/blob/main/2/pdf/Mathematische_struktur_En.pdf}

\bibitem{pascher_muon_g2_2025}
J. Pascher, \emph{Muon g-2 in T0}, 2025.
\url{https://github.com/jpascher/T0-Time-Mass-Duality/blob/main/2/pdf/T0_Anomale-g2-9_En.pdf}

\bibitem{pascher_pragmatic_2025}
J. Pascher, \emph{Pragmatic Approach}, 2025.

\bibitem{pascher_t0_energy_2025}
J. Pascher, \emph{T0 Energy Formalism}, 2025.
\url{https://github.com/jpascher/T0-Time-Mass-Duality/blob/main/2/pdf/T0-Energie_En.pdf}

\bibitem{pascher_unified_2025}
J. Pascher, \emph{Unified T0 Theory}, 2025.
\url{https://github.com/jpascher/T0-Time-Mass-Duality/blob/main/2/pdf/T0_unified_report.pdf}

\bibitem{sciencedaily2025}
Science Daily, \emph{Physics News}, 2025.
\url{https://www.sciencedaily.com/}

\bibitem{weinberg_1989}
S. Weinberg, \emph{The Cosmological Constant Problem}, Rev. Mod. Phys., 1989.
\url{https://doi.org/10.1103/RevModPhys.61.1}

\bibitem{wiki_bell}
Wikipedia, \emph{Bell's Theorem}, 2025.
\url{https://en.wikipedia.org/wiki/Bell\%27s_theorem}

\bibitem{vanFraassen1980}
B. van Fraassen, \emph{The Scientific Image}, Oxford University Press, 1980.

\bibitem{terrell_single_clock_nature_2024}
J. Terrell, \emph{Single Clock Nature}, Nature, 2024.

% Additional T0 Documents
\bibitem{137_doc}
J. Pascher, \emph{The Number 137 in T0 Theory}, 2025.
\url{https://github.com/jpascher/T0-Time-Mass-Duality/blob/main/2/pdf/137_En.pdf}

\bibitem{ampere_low}
J. Pascher, \emph{Ampere's Law in T0}, 2025.
\url{https://github.com/jpascher/T0-Time-Mass-Duality/blob/main/2/pdf/Amper_Low_En.pdf}

\bibitem{bell_theorem}
J. Pascher, \emph{Bell's Theorem in T0}, 2025.
\url{https://github.com/jpascher/T0-Time-Mass-Duality/blob/main/2/pdf/Bell_En.pdf}

\bibitem{bewegungsenergie}
J. Pascher, \emph{Kinetic Energy in T0}, 2025.
\url{https://github.com/jpascher/T0-Time-Mass-Duality/blob/main/2/pdf/Bewegungsenergie_En.pdf}

\bibitem{emc2}
J. Pascher, \emph{E=mc² in T0 Framework}, 2025.
\url{https://github.com/jpascher/T0-Time-Mass-Duality/blob/main/2/pdf/E-mc2_En.pdf}

\bibitem{formeln_energiebasiert}
J. Pascher, \emph{Energy-Based Formulas}, 2025.
\url{https://github.com/jpascher/T0-Time-Mass-Duality/blob/main/2/pdf/Formeln_Energiebasiert_En.pdf}

\bibitem{hannah}
J. Pascher, \emph{Hannah Document}, 2025.
\url{https://github.com/jpascher/T0-Time-Mass-Duality/blob/main/2/pdf/Hannah_En.pdf}

\bibitem{ho_doc}
J. Pascher, \emph{H0 Analysis}, 2025.
\url{https://github.com/jpascher/T0-Time-Mass-Duality/blob/main/2/pdf/Ho_En.pdf}

\bibitem{markov}
J. Pascher, \emph{Markov Processes in T0}, 2025.
\url{https://github.com/jpascher/T0-Time-Mass-Duality/blob/main/2/pdf/Markov_En.pdf}

\bibitem{elimination_mass}
J. Pascher, \emph{Elimination of Mass}, 2025.
\url{https://github.com/jpascher/T0-Time-Mass-Duality/blob/main/2/pdf/EliminationOfMassEn.pdf}

\bibitem{elimination_mass_dirac}
J. Pascher, \emph{Dirac Equation Mass Elimination}, 2025.
\url{https://github.com/jpascher/T0-Time-Mass-Duality/blob/main/2/pdf/Elimination_Of_Mass_Dirac_TabelleEn.pdf}

\bibitem{feinstrukturkonstante}
J. Pascher, \emph{Fine Structure Constant}, 2025.
\url{https://github.com/jpascher/T0-Time-Mass-Duality/blob/main/2/pdf/FeinstrukturkonstanteEn.pdf}

\bibitem{neutrino_formel}
J. Pascher, \emph{Neutrino Formula}, 2025.
\url{https://github.com/jpascher/T0-Time-Mass-Duality/blob/main/2/pdf/neutrino-Formel_En.pdf}

\bibitem{neutrinos}
J. Pascher, \emph{Neutrinos in T0}, 2025.
\url{https://github.com/jpascher/T0-Time-Mass-Duality/blob/main/2/pdf/T0_Neutrinos_En.pdf}

\bibitem{koide_formel}
J. Pascher, \emph{Koide Formula in T0}, 2025.
\url{https://github.com/jpascher/T0-Time-Mass-Duality/blob/main/2/pdf/T0_koide-formel-3_En.pdf}

\bibitem{teilchenmassen}
J. Pascher, \emph{Particle Masses}, 2025.
\url{https://github.com/jpascher/T0-Time-Mass-Duality/blob/main/2/pdf/Teilchenmassen_En.pdf}

\bibitem{t0_teilchenmassen}
J. Pascher, \emph{T0 Particle Masses}, 2025.
\url{https://github.com/jpascher/T0-Time-Mass-Duality/blob/main/2/pdf/T0_Teilchenmassen_En.pdf}

\bibitem{penrose_doc}
J. Pascher, \emph{Penrose Analysis in T0}, 2025.
\url{https://github.com/jpascher/T0-Time-Mass-Duality/blob/main/2/pdf/T0_penrose_En.pdf}

\bibitem{photonenchip}
J. Pascher, \emph{Photon Chip Implementation}, 2025.
\url{https://github.com/jpascher/T0-Time-Mass-Duality/blob/main/2/pdf/T0_photonenchip-china_En.pdf}

\bibitem{threeclock}
J. Pascher, \emph{Three Clock Experiment}, 2025.
\url{https://github.com/jpascher/T0-Time-Mass-Duality/blob/main/2/pdf/T0_threeclock_En.pdf}

\bibitem{redshift_deflection}
J. Pascher, \emph{Redshift and Deflection}, 2025.
\url{https://github.com/jpascher/T0-Time-Mass-Duality/blob/main/2/pdf/redshift_deflection_En.pdf}

\bibitem{scheinbar_instantan}
J. Pascher, \emph{Apparent Instantaneity}, 2025.
\url{https://github.com/jpascher/T0-Time-Mass-Duality/blob/main/2/pdf/scheinbar_instantan_En.pdf}

\bibitem{universale_ableitung}
J. Pascher, \emph{Universal Derivation}, 2025.
\url{https://github.com/jpascher/T0-Time-Mass-Duality/blob/main/2/pdf/universale-ableitung_En.pdf}

\bibitem{xi_parameter}
J. Pascher, \emph{Xi Parameter for Particles}, 2025.
\url{https://github.com/jpascher/T0-Time-Mass-Duality/blob/main/2/pdf/xi_parmater_partikel_En.pdf}

\bibitem{xi_ursprung}
J. Pascher, \emph{Origin of Xi}, 2025.
\url{https://github.com/jpascher/T0-Time-Mass-Duality/blob/main/2/pdf/T0_xi_ursprung_En.pdf}

\bibitem{zeit}
J. Pascher, \emph{Time in T0 Theory}, 2025.
\url{https://github.com/jpascher/T0-Time-Mass-Duality/blob/main/2/pdf/Zeit_En.pdf}

\bibitem{zeit_konstant}
J. Pascher, \emph{Time Constant}, 2025.
\url{https://github.com/jpascher/T0-Time-Mass-Duality/blob/main/2/pdf/Zeit-konstant_En.pdf}

\bibitem{zusammenfassung}
J. Pascher, \emph{Summary of T0 Theory}, 2025.
\url{https://github.com/jpascher/T0-Time-Mass-Duality/blob/main/2/pdf/Zusammenfassung_En.pdf}

\bibitem{rsa}
J. Pascher, \emph{RSA in T0 Framework}, 2025.
\url{https://github.com/jpascher/T0-Time-Mass-Duality/blob/main/2/pdf/RSA_En.pdf}

\bibitem{qat}
J. Pascher, \emph{Quantum Atomic Theory}, 2025.
\url{https://github.com/jpascher/T0-Time-Mass-Duality/blob/main/2/pdf/T0_QAT_En.pdf}

\bibitem{qm_qft_rt}
J. Pascher, \emph{QM, QFT and RT Unification}, 2025.
\url{https://github.com/jpascher/T0-Time-Mass-Duality/blob/main/2/pdf/T0_QM-QFT-RT_En.pdf}

\bibitem{qm_optimierung}
J. Pascher, \emph{QM Optimization}, 2025.
\url{https://github.com/jpascher/T0-Time-Mass-Duality/blob/main/2/pdf/T0_QM-optimierung_En.pdf}

\bibitem{vollstaendige_berechnungen}
J. Pascher, \emph{Complete Calculations}, 2025.
\url{https://github.com/jpascher/T0-Time-Mass-Duality/blob/main/2/pdf/T0_Vollstaendige_Berchnungen_En.pdf}

\bibitem{synergetics}
J. Pascher, \emph{T0 Theory vs Synergetics}, 2025.
\url{https://github.com/jpascher/T0-Time-Mass-Duality/blob/main/2/pdf/T0-Theory-vs-Synergetics_En.pdf}

\bibitem{modell_uebersicht}
J. Pascher, \emph{T0 Model Overview}, 2025.
\url{https://github.com/jpascher/T0-Time-Mass-Duality/blob/main/2/pdf/T0_Modell_Uebersicht_En.pdf}

\bibitem{mnras_widerlegung}
J. Pascher, \emph{MNRAS Analysis}, 2025.
\url{https://github.com/jpascher/T0-Time-Mass-Duality/blob/main/2/pdf/T0_Analyse_MNRAS_Widerlegung_En.pdf}

\bibitem{anomale_magnetische_momente}
J. Pascher, \emph{Anomalous Magnetic Moments}, 2025.
\url{https://github.com/jpascher/T0-Time-Mass-Duality/blob/main/2/pdf/T0_Anomale_Magnetische_Momente_En.pdf}

\bibitem{sieben_fragen}
J. Pascher, \emph{Seven Questions in T0}, 2025.
\url{https://github.com/jpascher/T0-Time-Mass-Duality/blob/main/2/pdf/T0_7-fragen-3_En.pdf}

\bibitem{detailierte_leptonen}
J. Pascher, \emph{Detailed Lepton Anomaly}, 2025.
\url{https://github.com/jpascher/T0-Time-Mass-Duality/blob/main/2/pdf/detailierte_formel_leptonen_anemal_En.pdf}

\bibitem{parameterherleitung}
J. Pascher, \emph{Parameter Derivation}, 2025.
\url{https://github.com/jpascher/T0-Time-Mass-Duality/blob/main/2/pdf/parameterherleitung_En.pdf}

\bibitem{verhaeltnis_absolut}
J. Pascher, \emph{Absolute Ratios in T0}, 2025.
\url{https://github.com/jpascher/T0-Time-Mass-Duality/blob/main/2/pdf/T0_verhaeltnis-absolut_En.pdf}

\bibitem{xi_und_e}
J. Pascher, \emph{Xi and Energy}, 2025.
\url{https://github.com/jpascher/T0-Time-Mass-Duality/blob/main/2/pdf/T0_xi-und-e_En.pdf}

\bibitem{umkehrung}
J. Pascher, \emph{Inversion in T0}, 2025.
\url{https://github.com/jpascher/T0-Time-Mass-Duality/blob/main/2/pdf/T0_umkehrung_En.pdf}

\bibitem{esm_analysis}
J. Pascher, \emph{T0 vs ESM Conceptual Analysis}, 2025.
\url{https://github.com/jpascher/T0-Time-Mass-Duality/blob/main/2/pdf/T0vsESM_ConceptualAnalysis_En.pdf}

\end{thebibliography}

\end{document}
