% Documento autonomo: T0_Introduction_It
% Utilizza l'intestazione T0 comune per l'italiano
% Italian Standalone Document Header
\documentclass[12pt,a4paper]{article}
\usepackage[utf8]{inputenc}
\usepackage[T1]{fontenc}
\usepackage[italian]{babel}
\usepackage{lmodern}
\usepackage{amsmath,amssymb,amsthm}
\usepackage{physics}
\usepackage{siunitx}
\usepackage{geometry}
\geometry{margin=2.5cm}
\usepackage{fancyhdr}
\usepackage{titlesec}
\usepackage{booktabs}
\usepackage{longtable}
\usepackage{graphicx}
\usepackage{tikz}
\usepackage{hyperref}
\usepackage{cleveref}
\usepackage{xcolor}
\usepackage{tcolorbox}

% Custom commands
\newcommand{\Tfield}{T(x,t)}
\newcommand{\xipar}{\xi}

% tcolorbox environments
\newtcolorbox{insight}[1][]{colback=blue!5,colframe=blue!75!black,title=Intuizione,#1}
\newtcolorbox{discovery}[1][]{colback=green!5,colframe=green!75!black,title=Scoperta,#1}
\newtcolorbox{keypoint}[1][]{colback=red!5,colframe=red!75!black,title=Punto Chiave,#1}
\newtcolorbox{conclusion}[1][]{colback=gray!5,colframe=gray!75!black,title=Conclusione,#1}
\newtcolorbox{significance}[1][]{colback=yellow!5,colframe=yellow!75!black,title=Significato,#1}
\newtcolorbox{philosophical}[1][]{colback=purple!5,colframe=purple!75!black,title=Riflessione Filosofica,#1}
\newtcolorbox{implication}[1][]{colback=orange!5,colframe=orange!75!black,title=Implicazione,#1}
\newtcolorbox{newperspective}[1][]{colback=cyan!5,colframe=cyan!75!black,title=Nuova Prospettiva,#1}
\newtcolorbox{revelation}[1][]{colback=magenta!5,colframe=magenta!75!black,title=Rivelazione,#1}
\newtcolorbox{evidence}[1][]{colback=teal!5,colframe=teal!75!black,title=Evidenza,#1}
\newtcolorbox{perspective}[1][]{colback=lime!5,colframe=lime!75!black,title=Prospettiva,#1}
\newtcolorbox{revolutionary}[1][]{colback=pink!5,colframe=pink!75!black,title=Rivoluzionario,#1}

% Theorem environments
\newtheorem{theorem}{Teorema}
\newtheorem{lemma}{Lemma}
\newtheorem{proposition}{Proposizione}
\newtheorem{corollary}{Corollario}
\theoremstyle{definition}
\newtheorem{definition}{Definizione}
\newtheorem{example}{Esempio}
\theoremstyle{remark}
\newtheorem{remark}{Osservazione}


\title{Introduzione alla Teoria T0}
\author{Johann Pascher}
\date{2025}

\begin{document}

\maketitle

\chapter{Introduzione alla Teoria T0}



\chapter*{Introduzione}
\addcontentsline{toc}{chapter}{Introduzione}

Questo libro presenta lo stato attuale del quadro di dualità tempo-massa T0 e le sue applicazioni alle
masse delle particelle, alle costanti fondamentali, alla meccanica quantistica, alla gravità e alla cosmologia.

La parte principale del libro consiste in una serie di documenti centrali T0. Questi capitoli riflettono la
comprensione attuale della teoria e delle sue conseguenze quantitative. Nella misura del possibile, il
materiale è stato riorganizzato e unificato affinché la struttura della teoria sia il più trasparente possibile.

Alla fine del libro, diversi documenti più vecchi sono inclusi in un'appendice. Questi testi rappresentano
stadi precedenti dello sviluppo del quadro T0. Non sono stati rimossi perché rendono visibile l'evoluzione delle
idee e il raffinamento delle formule. In molti casi, si può vedere come le approssimazioni
sono state migliorate, come i casi speciali sono stati generalizzati e come i nuovi dati empirici hanno contribuito a
raffinare o correggere gli argomenti precedenti.

La versione "live" della teoria è mantenuta in un repository GitHub pubblico:

\begin{center}
\url{https://github.com/jpascher/T0-Time-Mass-Duality}
\end{center}

Le sorgenti LaTeX dei capitoli di questo libro provengono da questo repository. Se vengono trovati errori concettuali o
numerici, saranno corretti prima lì. Ciò significa che la versione PDF del
libro che stai leggendo è un'istantanea di un progetto in continua evoluzione. Per la
versione più recente dei documenti, incluse nuove appendici o correzioni, il repository GitHub
dovrebbe sempre essere considerato come riferimento principale.

L'intenzione di questa compilazione è duplice:
\begin{itemize}
\item offrire un percorso coerente e leggibile attraverso le idee e i risultati centrali del quadro T0;
\item documentare nell'appendice lo sviluppo storico di queste idee, inclusi i falsi partenze,
le formulazioni intermedie e i primi aggiustamenti ai dati sperimentali.
\end{itemize}

\end{document}
