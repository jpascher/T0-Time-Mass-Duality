% Standalone document: RSAtest_En
% Uses shared T0 header
% T0 Standalone Header - shared preamble for all standalone documents
% Change to article class for standalone documents
\documentclass[12pt,a4paper]{article}
\usepackage[utf8]{inputenc}
\usepackage[T1]{fontenc}
\usepackage[english]{babel}
\usepackage{lmodern}

% Mathematics
\usepackage{amsmath,amssymb,amsthm}
\usepackage{physics}
\usepackage{siunitx}

% Layout
\usepackage[left=2.5cm,right=2.5cm,top=2.5cm,bottom=2.5cm,headheight=15pt]{geometry}
\usepackage{fancyhdr}
\usepackage{titlesec}

% Tables and Graphics
\usepackage{booktabs}
\usepackage{array}
\usepackage{longtable}
\usepackage{graphicx}
\usepackage{tikz}
\usetikzlibrary{arrows.meta,positioning,shapes.geometric}

% Colors and Boxes
\usepackage{xcolor}
\usepackage[most]{tcolorbox}
\usepackage{mdframed}

% Additional packages
\usepackage{enumitem}
\usepackage{float}
\usepackage{caption}
\usepackage{subcaption}
\usepackage{multirow}
\usepackage{colortbl}
\usepackage{pdflscape}
\usepackage{algorithm}
\usepackage{algpseudocode}
\usepackage{listings}
\usepackage{hyperref}

% Define colors
\definecolor{t0blue}{RGB}{0,51,102}
\definecolor{t0green}{RGB}{0,102,51}
\definecolor{t0red}{RGB}{153,0,0}
\definecolor{deepblue}{RGB}{0,51,102}
\definecolor{deepgreen}{RGB}{0,102,51}
\definecolor{deepred}{RGB}{153,0,0}
\definecolor{boxgray}{RGB}{240,240,240}
\definecolor{boxblue}{RGB}{230,240,255}
\definecolor{boxgreen}{RGB}{230,255,230}
\definecolor{boxorange}{RGB}{255,240,230}
\definecolor{boxyellow}{RGB}{255,255,230}

% Custom tcolorbox environments
\newtcolorbox{fundamental}[1][]{
  colback=blue!5!white,
  colframe=blue!75!black,
  title=#1,
  fonttitle=\bfseries,
  breakable
}

\newtcolorbox{derivation}[1][]{
  colback=green!5!white,
  colframe=green!75!black,
  title=#1,
  fonttitle=\bfseries,
  breakable
}

\newtcolorbox{result}[1][]{
  colback=orange!5!white,
  colframe=orange!75!black,
  title=#1,
  fonttitle=\bfseries,
  breakable
}

\newtcolorbox{summary}[1][]{
  colback=gray!10!white,
  colframe=gray!75!black,
  title=#1,
  fonttitle=\bfseries,
  breakable
}

\newtcolorbox{comparison}[1][]{
  colback=purple!5!white,
  colframe=purple!75!black,
  title=#1,
  fonttitle=\bfseries,
  breakable
}

\newtcolorbox{relation}[1][]{
  colback=cyan!5!white,
  colframe=cyan!75!black,
  title=#1,
  fonttitle=\bfseries,
  breakable
}

\newtcolorbox{treatise}[1][]{
  colback=yellow!5!white,
  colframe=yellow!75!black,
  title=#1,
  fonttitle=\bfseries,
  breakable
}

\newtcolorbox{gemeinsam}[1][]{
  colback=teal!5!white,
  colframe=teal!75!black,
  title=#1,
  fonttitle=\bfseries,
  breakable
}

\newtcolorbox{vergleich}[1][]{
  colback=violet!5!white,
  colframe=violet!75!black,
  title=#1,
  fonttitle=\bfseries,
  breakable
}

\newtcolorbox{vorteil}[1][]{
  colback=lime!5!white,
  colframe=lime!75!black,
  title=#1,
  fonttitle=\bfseries,
  breakable
}

\newtcolorbox{quantum}[1][]{
  colback=magenta!5!white,
  colframe=magenta!75!black,
  title=#1,
  fonttitle=\bfseries,
  breakable
}

\newtcolorbox{principle}[1][]{
  colback=brown!5!white,
  colframe=brown!75!black,
  title=#1,
  fonttitle=\bfseries,
  breakable
}

% Custom commands
\newcommand{\xipar}{\xi}
\newcommand{\Tzero}{T_0}
\newcommand{\meff}{m_{\text{eff}}}
\newcommand{\Eint}{E_{\text{int}}}
\newcommand{\xiconst}{\xi}
\newcommand{\betaT}{\beta_T}
\newcommand{\alphaEM}{\alpha}
\newcommand{\lambdazero}{\lambda_0}
\newcommand{\nuzero}{\nu_0}
\newcommand{\vecx}{\vec{x}}
\providecommand{\Etau}{E_\tau}
\providecommand{\Emu}{E_\mu}
\providecommand{\Ee}{E_e}

% Theorem environments
\newtheorem{theorem}{Theorem}
\newtheorem{lemma}[theorem]{Lemma}
\newtheorem{proposition}[theorem]{Proposition}
\newtheorem{corollary}[theorem]{Corollary}
\theoremstyle{definition}
\newtheorem{definition}{Definition}
\newtheorem{example}{Example}
\theoremstyle{remark}
\newtheorem{remark}{Remark}

% Page style
\pagestyle{fancy}
\fancyhf{}
\fancyhead[L]{\leftmark}
\fancyhead[R]{\thepage}
\renewcommand{\headrulewidth}{0.4pt}

% Hyperref setup
\hypersetup{
  colorlinks=true,
  linkcolor=blue,
  citecolor=blue,
  urlcolor=blue,
  pdftitle={T0 Theory Document},
  pdfauthor={Johann Pascher}
}


\title{RSA Testing}
\author{Johann Pascher}
\date{2025}

\begin{document}

\maketitle



	
	
	\begin{abstract}
		This work documents empirical results from systematic testing of various factorization algorithms. 37 test cases were conducted using Trial Division, Fermat's Method, Pollard Rho, Pollard $p-1$, and the T0-Framework. The primary purpose is to demonstrate that deterministic period finding is feasible. All results are based on direct measurements without theoretical evaluations or comparisons.
	\end{abstract}
	
	\newpage
	
	\section{Methodology}
	
	\subsection{Tested Algorithms}
	
	The following factorization algorithms were implemented and tested:
	
	\begin{enumerate}
		\item \textbf{Trial Division}: Systematic division attempts up to $\sqrt{n}$
		\item \textbf{Fermat's Method}: Search for representation as difference of squares
		\item \textbf{Pollard Rho}: Probabilistic period finding in pseudorandom sequences
		\item \textbf{Pollard $p-1$}: Method for numbers with smooth factors
		\item \textbf{T0-Framework}: Deterministic period finding in modular exponentiation (classical Shor-inspired)
	\end{enumerate}
	
	\subsection{Test Configuration}
	
	\begin{table}[H]
		\centering
		\caption{Experimental Parameters}
		\begin{tabular}{ll}
			\toprule
			\textbf{Parameter} & \textbf{Value} \\
			\midrule
			Number of test cases & 37 \\
			Timeout per test & 2.0 seconds \\
			Number range & 15 to 16777213 \\
			Bit size & 4 to 24 bits \\
			Hardware & Standard desktop CPU \\
			Repetitions & 1 per combination \\
			\bottomrule
		\end{tabular}
		\label{RSAtest:tab:test_config}
	\end{table}
	
	\subsection{Metrics}
	
	For each test, the following were recorded:
	\begin{itemize}
		\item \textbf{Success/Failure}: Binary result
		\item \textbf{Execution time}: Millisecond precision
		\item \textbf{Found factors}: For successful tests
		\item \textbf{Algorithm-specific parameters}: Depending on method
	\end{itemize}
	
	\section{T0-Framework Feasibility Demonstration}
	
	\subsection{Purpose of Implementation}
	
	The T0-Framework implementation serves as a proof-of-concept to demonstrate that deterministic period finding is technically feasible on classical hardware.
	
	\subsection{Implementation Components}
	
	The T0-Framework implements the following components to demonstrate deterministic period finding:
	
	\begin{verbatim}
		class UniversalT0Algorithm:
		def __init__(self):
		self.xi_profiles = {
			'universal': Fraction(1, 100),
			'twin_prime_optimized': Fraction(1, 50),
			'medium_size': Fraction(1, 1000),
			'special_cases': Fraction(1, 42)
		}
		self.pi_fraction = Fraction(355, 113)
		self.threshold = Fraction(1, 1000)
	\end{verbatim}
	
	\subsection{Adaptive $\xi$-Strategies}
	
	The system uses different $\xi$-parameters based on number characteristics:
	
	\begin{table}[H]
		\centering
		\caption{$\xi$-Strategies in the T0-Framework}
		\begin{tabular}{lll}
			\toprule
			\textbf{Strategy} & \textbf{$\xi$-Value} & \textbf{Application} \\
			\midrule
			twin\_prime\_optimized & $1/50$ & Twin prime semiprimes \\
			universal & $1/100$ & General semiprimes \\
			medium\_size & $1/1000$ & Medium-sized numbers \\
			special\_cases & $1/42$ & Mathematical constants \\
			\bottomrule
		\end{tabular}
		\label{RSAtest:tab:xi_strategies}
	\end{table}
	
	\subsection{Resonance Calculation}
	
	Resonance evaluation is performed using exact rational arithmetic:
	
	\begin{equation}
		\omega = \frac{2 \cdot \pi_{\text{ratio}}}{r}
	\end{equation}
	
	\begin{equation}
		R(r) = \frac{1}{1 + \left|\frac{-(\omega-\pi)^2}{4\xi}\right|}
	\end{equation}
	
	\section{Experimental Results: Proof of Concept}
	
	The experimental results serve to demonstrate the feasibility of deterministic period finding rather than to compare algorithmic performance.
	
	\subsection{Success Rates by Algorithm}
	
	\begin{table}[H]
		\centering
		\caption{Overall success rates of all algorithms}
		\begin{tabular}{lrr}
			\toprule
			\textbf{Algorithm} & \textbf{Successful tests} & \textbf{Success rate (\%)} \\
			\midrule
			Trial Division & 37/37 & 100.0 \\
			Fermat & 37/37 & 100.0 \\
			Pollard Rho & 36/37 & 97.3 \\
			Pollard $p-1$ & 12/37 & 32.4 \\
			T0-Adaptive & 31/37 & 83.8 \\
			\bottomrule
		\end{tabular}
		\label{RSAtest:tab:success_rates}
	\end{table}
	
	\section{Period-based Factorization: T0, Pollard Rho, and Shor's Algorithm}
	
	\subsection{Comparison of Period Finding Approaches}
	
	T0-Framework, Pollard Rho, and Shor's quantum algorithm are all period-finding algorithms with different computational paradigms:
	
	\begin{table}[H]
		\centering
		\caption{Period-Finding Algorithms Comparison}
		\begin{tabular}{llll}
			\toprule
			\textbf{Aspect} & \textbf{Pollard Rho} & \textbf{T0-Framework} & \textbf{Shor's Algorithm} \\
			\midrule
			Computation & Classical prob. & Classical det. & Quantum \\
			Period detect & Floyd cycle & Resonance analysis & Quantum FT \\
			Arithmetic & Modular & Exact rational & Quantum superpos. \\
			Reproducibility & Variable & 100\% reprod. & Prob. measurement \\
			Sequence gen & $f(x) = x^2 + c \bmod n$ & $a^r \equiv 1 \pmod{n}$ & $a^x \bmod n$ \\
			Success crit & $\gcd(|x_i - x_j|, n) > 1$ & Resonance thresh. & Period from QFT \\
			Complexity & $O(n^{1/4})$ expect. & $O((\log n)^3)$ theor. & $O((\log n)^3)$ theor. \\
			Hardware & Classical comp. & Classical comp. & Quantum comp. \\
			Practical limit & Birthday paradox & Resonance tuning & Quantum decoher. \\
			\bottomrule
		\end{tabular}
		\label{RSAtest:tab:period_comparison}
	\end{table}
	
	\subsection{Shared Period-Finding Principle}
	
	All three algorithms exploit the same mathematical foundation:
	
	\begin{itemize}
		\item \textbf{Core idea}: Find period $r$ where $a^r \equiv 1 \pmod{n}$
		\item \textbf{Factor extraction}: Use period to compute $\gcd(a^{r/2} \pm 1, n)$
		\item \textbf{Mathematical basis}: Euler's theorem and order of elements in $\mathbb{Z}_n^*$
	\end{itemize}
	
	\subsection{Theoretical Complexity Analysis}
	
	Both T0-Framework and Shor's algorithm share the same theoretical complexity advantage:
	
	\begin{itemize}
		\item \textbf{Period search space}: Both search for periods $r$ where $a^r \equiv 1 \pmod{n}$
		\item \textbf{Maximum period}: The order of any element is at most $n-1$, but typically much smaller
		\item \textbf{Expected period length}: $O(\log n)$ for most elements due to Euler's theorem
		\item \textbf{Period testing}: Each period test requires $O((\log n)^2)$ operations for modular exponentiation
		\item \textbf{Total complexity}: $O(\log n) \times O((\log n)^2) = O((\log n)^3)$
	\end{itemize}
	
	\subsection{The Shared Polynomial Advantage}
	
	Both T0 and Shor's algorithm achieve the same theoretical breakthrough:
	
	\begin{equation}
		\text{Classical exponential: } O(2^{\sqrt{\log n \log \log n}}) \rightarrow \text{Polynomial: } O((\log n)^3)
	\end{equation}
	
	The key insight is that **both algorithms exploit the same mathematical structure**:
	\begin{itemize}
		\item Period finding in the group $\mathbb{Z}_n^*$
		\item Expected period length $O(\log n)$ due to smooth numbers
		\item Polynomial-time period verification
		\item Identical factor extraction method
	\end{itemize}
	
	**The only difference**: Shor uses quantum superposition to search periods in parallel, while T0 searches them deterministically in sequence - but both have the same $O((\log n)^3)$ complexity bound.
	
	\subsection{The Implementation Paradox}
	
	Both T0 and Shor's algorithm demonstrate a fundamental paradox in advanced algorithmic design:
	
	\begin{tcolorbox}[colback=yellow!10,colframe=orange!50,title=Core Problem]
		\textbf{Perfect Theory, Imperfect Implementation:} \\
		Both algorithms achieve the same theoretical breakthrough from exponential to polynomial complexity, but practical implementation overhead completely negates these theoretical advantages.
	\end{tcolorbox}
	
	\subsubsection{Shared Implementation Failures}
	\begin{itemize}
		\item \textbf{Shor's quantum overhead}: 
		\begin{itemize}
			\item Quantum error correction requires $\sim 10^6$ physical qubits per logical qubit
			\item Decoherence times limit algorithm execution
			\item Current systems: 1000 qubits → Need: $10^9$ qubits for RSA-2048
		\end{itemize}
		
		\item \textbf{T0's classical overhead}:
		\begin{itemize}
			\item Exact rational arithmetic: Fraction objects grow exponentially in size
			\item Resonance evaluation: Complex mathematical operations per period
			\item Adaptive parameter tuning: Multiple $\xi$-strategies increase computational cost
		\end{itemize}
	\end{itemize}
	
	\section{Philosophical Implications: Information and Determinism}
	
	\subsection{Intrinsic Mathematical Information}
	
	A crucial insight emerges from this analysis that extends beyond computational complexity:
	
	\begin{tcolorbox}[colback=blue!10,colframe=blue!50,title=Fundamental Principle]
		\textbf{No Superdeterminism Required:} \\
		All information that can be extracted from a number through factorization algorithms is intrinsically contained within the number itself. The algorithms merely reveal pre-existing mathematical relationships - they do not create information.
	\end{tcolorbox}
	
	\subsection{Vibrational Modes and Predictive Patterns}
	
	A deeper analysis reveals that number size constrains the possible "vibrational modes" in factorization:
	
	\begin{tcolorbox}[colback=purple!10,colframe=purple!50,title=Vibrational Constraint Principle]
		\textbf{Size-Determined Mode Space:} \\
		The size of a number $n$ predetermines the boundaries of possible oscillation modes. Within these boundaries, only specific resonance patterns are mathematically possible, and these follow predictable patterns that enable "looking into the future" of the factorization process.
	\end{tcolorbox}
	
	\subsubsection{Constrained Oscillation Space}
	
	For a number $n$ with $k = \log_2(n)$ bits:
	
	\begin{itemize}
		\item \textbf{Maximum period}: $r_{\max} = \lambda(n) \leq n-1$ (Carmichael function)
		\item \textbf{Typical period range}: $r_{typical} \in [1, O(\sqrt{n})]$ for most bases
		\item \textbf{Resonance frequencies}: $\omega = 2\pi/r$ constrained to discrete values
		\item \textbf{Vibrational modes}: Only $O(\sqrt{n})$ distinct oscillation patterns possible
	\end{itemize}
	
	\subsection{The Bounded Universe of Oscillations}
	
	\begin{equation}
		\Omega_n = \left\{\omega_r = \frac{2\pi}{r} : r \in \mathbb{Z}, 2 \leq r \leq \lambda(n)\right\}
	\end{equation}
	
	This frequency space $\Omega_n$ is:
	\begin{itemize}
		\item \textbf{Finite}: Constrained by number size
		\item \textbf{Discrete}: Only integer periods allowed
		\item \textbf{Structured}: Follows mathematical patterns based on $n$'s prime structure
		\item \textbf{Predictable}: Resonance peaks cluster in mathematically determined regions
	\end{itemize}
	
	\begin{tcolorbox}[colback=cyan!10,colframe=cyan!50,title=Predictive Principle]
		\textbf{Mathematical Foresight:} \\
		By analyzing the constrained oscillation space and recognizing structural patterns, it becomes possible to predict which periods will yield strong resonances without exhaustively testing all possibilities. This represents a form of mathematical "future sight" - not mystical, but based on deep pattern recognition in number-theoretic structures.
	\end{tcolorbox}
	
	\section{Neural Network Implications: Learning Mathematical Patterns}
	
	\subsection{Machine Learning Potential}
	
	If mathematical patterns in oscillation modes are predictable through pattern recognition, then neural networks should inherently be capable of learning these patterns:
	
	\begin{tcolorbox}[colback=green!10,colframe=green!50,title=Neural Network Hypothesis]
		\textbf{Learnable Mathematical Patterns:} \\
		Since the vibrational modes and resonance patterns follow mathematically deterministic rules within constrained spaces, neural networks should be able to learn to predict optimal factorization strategies without exhaustive search.
	\end{tcolorbox}
	
	\subsection{Training Data Structure}
	
	The experimental data provides perfect training material:
	
	\begin{itemize}
		\item \textbf{Input features}: Number size, bit length, mathematical type (twin prime, smooth, etc.)
		\item \textbf{Target predictions}: Optimal $\xi$-strategy, expected resonance periods, success probability
		\item \textbf{Pattern examples}: 37 test cases with documented success/failure patterns
		\item \textbf{Feature engineering}: Extract mathematical invariants (prime gaps, smoothness, etc.)
	\end{itemize}
	
	\subsection{Learning Mathematical Invariants}
	
	Neural networks could learn to recognize:
	
	\begin{table}[H]
		\centering
		\caption{Learnable Mathematical Patterns}
		\begin{tabular}{ll}
			\toprule
			\textbf{Math Pattern} & \textbf{NN Learning Target} \\
			\midrule
			Twin prime struct. & Predict $\xi = 1/50$ strategy \\
			Prime gap distrib. & Estimate reson. clustering \\
			Smoothness indic. & Predict period distrib. \\
			Math constants & ID multi-reson. patterns \\
			Carmichael patterns & Estimate max period bounds \\
			Factor size ratios & Predict optimal base select. \\
			\bottomrule
		\end{tabular}
		\label{RSAtest:tab:learnable_patterns}
	\end{table}
	
	
	

\begin{thebibliography}{99}
\bibitem{pascher2025t0} J. Pascher, \emph{T0 Theory Overview}, 2025.
\bibitem{einstein1905} A. Einstein, \emph{On the Electrodynamics of Moving Bodies}, Ann. Phys., 1905.
\bibitem{planck1900} M. Planck, \emph{On the Law of Distribution of Energy}, 1900.
\bibitem{feynman2006} R. P. Feynman, \emph{QED: The Strange Theory of Light and Matter}, 2006.
\bibitem{weinberg1995} S. Weinberg, \emph{The Quantum Theory of Fields}, 1995.
\bibitem{pdg2024} Particle Data Group, \emph{Review of Particle Physics}, 2024.
\end{thebibliography}

\end{document}
