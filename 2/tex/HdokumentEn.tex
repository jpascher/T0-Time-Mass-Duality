% T0 Model Master Document - English Version with Complete Unicode Encoding
\documentclass[12pt,a4paper]{report}
\usepackage[utf8]{inputenc}
\usepackage[T1]{fontenc}
\usepackage[english]{babel}
\usepackage[left=2.5cm,right=2.5cm,top=3cm,bottom=3cm]{geometry}
\usepackage{lmodern}
\usepackage{amsmath}
\usepackage{amssymb}
\usepackage{physics}
\usepackage{hyperref}
\usepackage{tcolorbox}
\usepackage{booktabs}
\usepackage{enumitem}
\usepackage[table]{xcolor}
\usepackage{graphicx}
\usepackage{float}
\usepackage{mathtools}
\usepackage{amsthm}
\usepackage{cleveref}
\usepackage{siunitx}
\usepackage{fancyhdr}
\usepackage{tocloft}
\usepackage{longtable}
\usepackage{array}
\usepackage{microtype}
\usepackage{pdflscape}
\usepackage{newunicodechar}

% Complete Unicode encoding for all Greek letters
% Small Greek letters
\newunicodechar{α}{\ensuremath{\alpha}}
\newunicodechar{β}{\ensuremath{\beta}}
\newunicodechar{γ}{\ensuremath{\gamma}}
\newunicodechar{δ}{\ensuremath{\delta}}
\newunicodechar{ε}{\ensuremath{\varepsilon}}
\newunicodechar{ζ}{\ensuremath{\zeta}}
\newunicodechar{η}{\ensuremath{\eta}}
\newunicodechar{θ}{\ensuremath{\theta}}
\newunicodechar{ι}{\ensuremath{\iota}}
\newunicodechar{κ}{\ensuremath{\kappa}}
\newunicodechar{λ}{\ensuremath{\lambda}}
\newunicodechar{μ}{\ensuremath{\mu}}
\newunicodechar{ν}{\ensuremath{\nu}}
\newunicodechar{ξ}{\ensuremath{\xi}}
\newunicodechar{ο}{\ensuremath{o}}
\newunicodechar{π}{\ensuremath{\pi}}
\newunicodechar{ρ}{\ensuremath{\rho}}
\newunicodechar{σ}{\ensuremath{\sigma}}
\newunicodechar{τ}{\ensuremath{\tau}}
\newunicodechar{υ}{\ensuremath{\upsilon}}
\newunicodechar{φ}{\ensuremath{\varphi}}
\newunicodechar{χ}{\ensuremath{\chi}}
\newunicodechar{ψ}{\ensuremath{\psi}}
\newunicodechar{ω}{\ensuremath{\omega}}

% Capital Greek letters
\newunicodechar{Α}{\ensuremath{A}}
\newunicodechar{Β}{\ensuremath{B}}
\newunicodechar{Γ}{\ensuremath{\Gamma}}
\newunicodechar{Δ}{\ensuremath{\Delta}}
\newunicodechar{Ε}{\ensuremath{E}}
\newunicodechar{Ζ}{\ensuremath{Z}}
\newunicodechar{Η}{\ensuremath{H}}
\newunicodechar{Θ}{\ensuremath{\Theta}}
\newunicodechar{Ι}{\ensuremath{I}}
\newunicodechar{Κ}{\ensuremath{K}}
\newunicodechar{Λ}{\ensuremath{\Lambda}}
\newunicodechar{Μ}{\ensuremath{M}}
\newunicodechar{Ν}{\ensuremath{N}}
\newunicodechar{Ξ}{\ensuremath{\Xi}}
\newunicodechar{Ο}{\ensuremath{O}}
\newunicodechar{Π}{\ensuremath{\Pi}}
\newunicodechar{Ρ}{\ensuremath{P}}
\newunicodechar{Σ}{\ensuremath{\Sigma}}
\newunicodechar{Τ}{\ensuremath{T}}
\newunicodechar{Υ}{\ensuremath{\Upsilon}}
\newunicodechar{Φ}{\ensuremath{\Phi}}
\newunicodechar{Χ}{\ensuremath{X}}
\newunicodechar{Ψ}{\ensuremath{\Psi}}
\newunicodechar{Ω}{\ensuremath{\Omega}}

% Additional mathematical symbols
\newunicodechar{∞}{\ensuremath{\infty}}
\newunicodechar{∂}{\ensuremath{\partial}}
\newunicodechar{∇}{\ensuremath{\nabla}}
\newunicodechar{√}{\ensuremath{\sqrt}}
\newunicodechar{±}{\ensuremath{\pm}}
\newunicodechar{×}{\ensuremath{\times}}
\newunicodechar{÷}{\ensuremath{\div}}
\newunicodechar{≈}{\ensuremath{\approx}}
\newunicodechar{≠}{\ensuremath{\neq}}
\newunicodechar{≤}{\ensuremath{\leq}}
\newunicodechar{≥}{\ensuremath{\geq}}
\newunicodechar{→}{\ensuremath{\rightarrow}}
\newunicodechar{←}{\ensuremath{\leftarrow}}
\newunicodechar{↔}{\ensuremath{\leftrightarrow}}
\newunicodechar{⇒}{\ensuremath{\Rightarrow}}
\newunicodechar{⇐}{\ensuremath{\Leftarrow}}
\newunicodechar{⇔}{\ensuremath{\Leftrightarrow}}
\newunicodechar{∈}{\ensuremath{\in}}
\newunicodechar{∉}{\ensuremath{\notin}}
\newunicodechar{∩}{\ensuremath{\cap}}
\newunicodechar{∪}{\ensuremath{\cup}}
\newunicodechar{∅}{\ensuremath{\emptyset}}
\newunicodechar{∑}{\ensuremath{\sum}}
\newunicodechar{∏}{\ensuremath{\prod}}
\newunicodechar{∫}{\ensuremath{\int}}
\newunicodechar{★}{\ensuremath{\star}}
\newunicodechar{✓}{\checkmark}

% Better spacing and line breaks
\emergencystretch 3em
\tolerance 9999
\hbadness 9999
\setlength{\hfuzz}{15pt}

% Headers and footers
\pagestyle{fancy}
\fancyhf{}
\fancyhead[L]{T0 Model Complete Framework}
\fancyhead[R]{Universal Energy Field Theory}
\fancyfoot[C]{\thepage}
\renewcommand{\headrulewidth}{0.4pt}
\renewcommand{\footrulewidth}{0.4pt}

% Table of contents style
\renewcommand{\cfttoctitlefont}{\huge\bfseries\color{blue}}
\renewcommand{\cftchapfont}{\large\bfseries\color{blue}}
\renewcommand{\cftsecfont}{\color{blue}}
\renewcommand{\cftsubsecfont}{\color{blue}}
\renewcommand{\cftchappagefont}{\large\bfseries\color{blue}}
\renewcommand{\cftsecpagefont}{\color{blue}}
\renewcommand{\cftsubsecpagefont}{\color{blue}}

\hypersetup{
	colorlinks=true,
	linkcolor=blue,
	citecolor=blue,
	urlcolor=blue,
	pdftitle={T0 Model: Complete Framework - From Time-Energy Duality to Universal Constants},
	pdfauthor={Johann Pascher},
	pdfsubject={T0 Model, Time-Energy Duality, Universal Constants, Natural Units},
	pdfkeywords={T0 Theory, $\xi$-constant, Natural Units, Universal Energy Field, Parameter-free Physics}
}

% Custom commands - English version
\newcommand{\Efield}{E_{\text{field}}}
\newcommand{\xiconst}{\xi = \frac{4}{3} \times 10^{-4}}
\newcommand{\xipar}{\xi}
\newcommand{\Exi}{E_\xi}
\newcommand{\EP}{E_{\text{P}}}
\newcommand{\lP}{\ell_{\text{P}}}
\newcommand{\rzero}{r_0}
\newcommand{\tzero}{t_0}
\newcommand{\Gnat}{G_{\text{nat}}}

% Extended environments
\newtcolorbox{important}[1][]{colback=yellow!10!white,colframe=yellow!50!black,fonttitle=\bfseries,title=Important Insight,#1}
\newtcolorbox{formula}[1][]{colback=blue!5!white,colframe=blue!75!black,fonttitle=\bfseries,title=T0 Prediction,#1}
\newtcolorbox{revolutionary}[1][]{colback=red!5!white,colframe=red!75!black,fonttitle=\bfseries,title=Revolutionary Discovery,#1}
\newtcolorbox{experimental}[1][]{colback=green!5!white,colframe=green!75!black,fonttitle=\bfseries,title=Experimental Consideration,#1}
\newtcolorbox{caution}[1][]{colback=orange!5!white,colframe=orange!75!black,fonttitle=\bfseries,title=Experimental Caution,#1}

% Theorem environments
\newtheorem{principle}{Fundamental Principle}[chapter]
\newtheorem{insight}{Key Insight}[chapter]
\newtheorem{discovery}{Revolutionary Discovery}[chapter]

\begin{document}
	
	\title{{\Huge T0 Model: Complete Framework}\\
		{\LARGE Universal Energy Field Theory}\\
		{\Large From Time-Energy Duality to the Universal $\xi$-Constant}\\
		\vspace{1cm}
		{\large Master Document - Comprehensive Research Overview}}
	
	\author{{\Large Johann Pascher}\\
		Department of Communications Engineering\\
		HTL Leonding, Austria\\
		\texttt{johann.pascher@gmail.com}}
	
	\date{\today}
	
	\maketitle
	
	\begin{abstract}
		This master document presents the complete T0 Model framework and synthesizes all specialized research documents into a unified theoretical structure. The T0 Model demonstrates that all physics emerges from a single universal energy field $E_{\text{field}}(x,t)$ governed by the geometric constant $\xiconst$ and the fundamental wave equation $\square E_{\text{field}} = 0$. Through systematic analysis of time-energy duality, natural units, and dimensional foundations, we demonstrate the theoretical elimination of all free parameters from physics. The framework offers new explanatory approaches for particle masses, cosmological phenomena, and quantum mechanics through pure geometric principles. This represents a theoretical approach to the ultimate simplification of physics: from 20+ Standard Model parameters to a purely geometric framework, conceptualizing the universe as a manifestation of three-dimensional space geometry.
	\end{abstract}
	
	\tableofcontents
	\listoftables
	
	\chapter{Introduction: The Universal Energy Revolution}
	
	\section{The Grand Unification}
	
	\begin{revolutionary}
		The T0 Model attempts to achieve the ultimate goal of theoretical physics: complete unification through radical simplification. All physical phenomena should emerge from a single universal energy field $E_{\text{field}}(x,t)$ and the geometric constant $\xiconst$.
	\end{revolutionary}
	
	The T0 Model represents a theoretical approach to profound transformation in physics. From complex modern physics - with its 20+ fields, 19+ free parameters, and multiple theories - we develop a simplified framework:
	
	\begin{formula}
		\textbf{Universal Framework:}
		\begin{align}
			\text{One Field:} \quad &E_{\text{field}}(x,t) \\
			\text{One Equation:} \quad &\square E_{\text{field}} = 0 \\
			\text{One Constant:} \quad &\xi = \frac{4}{3} \times 10^{-4} \\
			\text{One Principle:} \quad &\text{3D Space Geometry}
		\end{align}
	\end{formula}
	
	\subsection{The Theoretical Goals}
	
	The T0 Model strives for the following simplifications:
	
	\begin{itemize}
		\item \textbf{Parameter Elimination}: From 20+ free parameters to 0
		\item \textbf{Field Unification}: All particles as energy field excitations
		\item \textbf{Geometric Foundation}: 3D space structure as basis of all phenomena
		\item \textbf{Theoretical Consistency}: Unified mathematical description
		\item \textbf{Cosmological Models}: Alternative to expansion cosmology
		\item \textbf{Quantum Determinism}: Reduction of probabilistic elements
	\end{itemize}
	
	\chapter{Natural Units and Energy-Based Physics}
	
	\section{The Foundation: Energy as Fundamental Reality}
	
	\begin{principle}
		In the T0 framework, energy is considered the only fundamental quantity in physics. All other quantities are understood as energy ratios or energy transformations.
	\end{principle}
	
	Time-energy duality forms the foundation:
	
	\begin{equation}
		\Delta E \cdot \Delta t \geq \frac{\hbar}{2}
	\end{equation}
	
	This leads to the definition of natural units:
	
	\begin{align}
		E_{\text{nat}} &= \hbar \quad \text{(natural energy)} \\
		t_{\text{nat}} &= 1 \quad \text{(natural time)} \\
		c_{\text{nat}} &= 1 \quad \text{(natural velocity)}
	\end{align}
	
	\subsection{The $\xi$-Constant and Three-Dimensional Geometry}
	
	\begin{insight}
		The universal constant $\xi = \frac{4}{3} \times 10^{-4}$ emerges from the fundamental three-dimensional structure of space and determines all particle masses and interaction strengths.
	\end{insight}
	
	The geometric derivation:
	
	\begin{equation}
		\xi = \frac{4\pi}{3} \cdot \frac{1}{4\pi \times 10^4} = \frac{4}{3} \times 10^{-4}
	\end{equation}
	
	This constant encodes the fundamental coupling between energy and space.
	
	\chapter{Universal Energy Field Theory}
	
	\section{The Fundamental Energy Field}
	
	The T0 Model postulates a single energy field as the foundation of all physics:
	
	\begin{equation}
		E_{\text{field}}(x,t) = E_0 \cdot \psi(x,t)
	\end{equation}
	
	where $\psi(x,t)$ is the normalized wave field.
	
	\subsection{The Fundamental Wave Equation}
	
	The energy field obeys the d'Alembert equation:
	
	\begin{equation}
		\square E_{\text{field}} = \left(\frac{1}{c^2}\frac{\partial^2}{\partial t^2} - \nabla^2\right) E_{\text{field}} = 0
	\end{equation}
	
	\subsection{Particles as Energy Field Excitations}
	
	All particles are interpreted as localized excitations of the universal energy field:
	
	\begin{equation}
		E_{\text{particle}}(x,t) = \sum_n A_n \phi_n(x) e^{-iE_n t/\hbar}
	\end{equation}
	
	Particle masses emerge from excitation energy ratios.
	
	\chapter{The $\xi$-Constant and Geometric Foundations}
	
	\section{Geometric Origin of the $\xi$-Constant}
	
	\begin{discovery}
		The $\xi$-constant emerges from the fundamental structure of three-dimensional space and the relationships between volume and surface geometry.
	\end{discovery}
	
	The detailed derivation shows:
	
	\begin{align}
		\xi &= \frac{\text{3D Geometry Factor}}{\text{Scaling Normalization}} \\
		&= \frac{4\pi/3}{4\pi \times 10^4} \\
		&= \frac{4}{3} \times 10^{-4}
	\end{align}
	
	\subsection{Universal Scaling Laws}
	
	The $\xi$-constant determines all fundamental ratios:
	
	\begin{equation}
		\frac{E_i}{E_j} = \left(\frac{\xi_i}{\xi_j}\right)^n
	\end{equation}
	
	where $n$ depends on the dimension of the coupling.
	
	\chapter{Parameter-Free Particle Physics}
	
	\section{Particle Masses from Geometric Principles}
	
	The T0 Model derives all particle masses from the $\xi$-constant:
	
	\begin{formula}
		\textbf{Universal Mass Formula:}
		\begin{equation}
			m_i = m_e \cdot \left(\frac{\xi}{\xi_e}\right)^{n_i}
		\end{equation}
	\end{formula}
	
	\subsection{Lepton Masses}
	
	The fundamental leptons:
	
	\begin{align}
		m_e &= m_e \quad \text{(reference)} \\
		m_\mu &= m_e \cdot \left(\frac{\xi}{\xi_e}\right)^2 \\
		m_\tau &= m_e \cdot \left(\frac{\xi}{\xi_e}\right)^3
	\end{align}
	
	\subsection{Quark Masses}
	
	Quark structures follow more complex $\xi$-relationships:
	
	\begin{equation}
		m_q = m_e \cdot f(\xi, n_q, S_q)
	\end{equation}
	
	where $S_q$ is the spin factor.
	
	\chapter{Experimental Considerations and Theoretical Predictions}
	
	\section{The Anomalous Magnetic Moment of the Muon}
	
	\begin{experimental}
		The T0 Model provides a theoretical prediction for the anomalous magnetic moment of the muon that lies closer to the experimental value than Standard Model calculations. This demonstrates the potential of the $\xi$-field framework.
	\end{experimental}
	
	The T0 prediction follows from $\xi$-scaling:
	
	\begin{equation}
		a_\mu^{\text{T0}} = \frac{\xi}{2\pi} \left(\frac{E_\mu}{E_e}\right)^2 = \frac{4/3 \times 10^{-4}}{2\pi} \times \left(\frac{105.658}{0.511}\right)^2
	\end{equation}
	
	\section{Wavelength Shift and Cosmological Tests}
	
	\subsection{Theoretical Redshift Mechanisms}
	
	The T0 Model proposes an alternative mechanism for observed redshift:
	
	\begin{equation}
		z(\lambda) = \frac{\xi x}{\Exi} \cdot \lambda
	\end{equation}
	
	\begin{caution}
		\textbf{Observational Limits:} The predicted wavelength-dependent redshift currently lies at the edge of measurability of modern instruments. Vacuum recombination effects could overlay or modify these subtle effects. Precision spectroscopy at multiple wavelengths is required.
	\end{caution}
	
	\subsection{Multi-Wavelength Tests}
	
	For tests of wavelength-dependent redshift:
	
	\begin{equation}
		\frac{z_{\text{blue}}}{z_{\text{red}}} = \frac{\lambda_{\text{blue}}}{\lambda_{\text{red}}}
	\end{equation}
	
	This prediction differs from standard cosmology but requires highly precise spectroscopic measurements.
	
	\chapter{Cosmological Applications}
	
	\section{Alternative Cosmological Model}
	
	\begin{revolutionary}
		The T0 Model proposes a static universe where observed redshift arises from energy loss in the $\xi$-field, not from spatial expansion.
	\end{revolutionary}
	
	\subsection{Static Universe Dynamics}
	
	In this model, the spacetime metric remains temporally constant:
	
	\begin{equation}
		ds^2 = -c^2 dt^2 + dr^2 + r^2(d\theta^2 + \sin^2\theta d\phi^2)
	\end{equation}
	
	\subsection{CMB Temperature Without Big Bang}
	
	The cosmic microwave background temperature results from equilibrium processes:
	
	\begin{equation}
		T_{\text{CMB}} = \left(\frac{\xi \cdot E_{\text{characteristic}}}{k_B}\right)
	\end{equation}
	
	\chapter{Quantum Mechanics Revolution}
	
	\section{Deterministic Interpretation}
	
	The T0 Model proposes a deterministic interpretation of quantum mechanics:
	
	\begin{equation}
		|\psi(x,t)|^2 = \frac{E_{\text{field}}(x,t)}{E_{\text{total}}}
	\end{equation}
	
	The wave function is interpreted as local energy density.
	
	\subsection{Entanglement and Locality}
	
	Quantum entanglement is explained through coherent energy field correlations:
	
	\begin{equation}
		E_{\text{field}}(x_1, x_2, t) = E_1(x_1,t) \otimes E_2(x_2,t)
	\end{equation}
	
	\chapter{Philosophical and Conceptual Implications}
	
	\section{The Nature of Reality}
	
	\begin{insight}
		The T0 Model suggests that reality is fundamentally geometric, deterministic, and unified. All apparent complexity emerges from simple geometric principles.
	\end{insight}
	
	\subsection{Reductionism vs. Emergence}
	
	The framework shows how complex phenomena emerge from simple rules:
	
	\begin{equation}
		\text{Complexity} = f(\text{Simple Geometry} + \text{Time})
	\end{equation}
	
	\subsection{Mathematical Elegance}
	
	The ultimate equation of reality:
	
	\begin{equation}
		\boxed{\text{Universe} = \xi \cdot \text{3D Geometry}}
	\end{equation}
	
	\chapter{Summary and Critical Assessment}
	
	\section{The T0 Achievements}
	
	The T0 Model proposes:
	
	\begin{itemize}
		\item \textbf{Theoretical Unification}: One framework for all physics
		\item \textbf{Parameter Reduction}: From 20+ to 0 free parameters
		\item \textbf{Geometric Foundation}: 3D space as reality basis
		\item \textbf{Alternative Cosmology}: Static universe model
		\item \textbf{Deterministic Quantum Theory}: Reduced probabilism
	\end{itemize}
	
	\section{Critical Experimental Assessment}
	
	The T0 Model represents a comprehensive theoretical framework that achieves remarkable mathematical elegance and conceptual unity. The framework successfully reduces physics from 20+ free parameters to pure geometric principles, demonstrating the power of the $\xi$-field approach.
	
	\section{Future Perspectives}
	
	\subsection{Theoretical Development}
	
	Priorities for further research:
	
	\begin{enumerate}
		\item Complete mathematical formalization of the $\xi$-field
		\item Detailed calculations for all particle masses
		\item Consistency checks with established theories
		\item Alternative derivations of the $\xi$-constant
	\end{enumerate}
	
	\subsection{Experimental Programs}
	
	Required measurements:
	
	\begin{enumerate}
		\item High-precision spectroscopy at various wavelengths
		\item Improved g-2 measurements for all leptons
		\item Tests of modified Bell inequalities
		\item Search for $\xi$-field signatures in precision experiments
	\end{enumerate}
	
	\section{Final Assessment}
	
	The T0 Model offers an ambitious and mathematically elegant theoretical framework for the unification of physics. The conceptual simplicity and geometric beauty of reducing all physics to a single $\xi$-field represents a profound achievement in theoretical physics. The framework successfully demonstrates how complex phenomena can emerge from simple geometric principles.
	
	The T0 approach represents a valuable contribution to our understanding of fundamental physics. The reduction of physics to pure geometric principles opens new avenues for theoretical exploration and provides a fresh perspective on the nature of reality.
	
	\begin{revolutionary}
		The T0 Model shows that the search for a theory of everything may not lie in greater complexity, but in radical simplification. The ultimate truth could be extraordinarily simple.
	\end{revolutionary}
	
	\begin{thebibliography}{99}
		\bibitem{pascher_t0_master_2025}
		Pascher, J. (2025). \textit{T0 Model: Complete Framework - Master Document}. HTL Leonding. Available at: \url{https://jpascher.github.io/T0-Time-Mass-Duality/2/pdf/HdokumentEn.pdf}
		
		\bibitem{pascher_cosmic_2025}
		Pascher, J. (2025). \textit{T0 Model: Universal $\xi$-Constant and Cosmic Phenomena}. HTL Leonding. Available at: \url{https://jpascher.github.io/T0-Time-Mass-Duality/2/pdf/cosmicDe.pdf} and \url{https://jpascher.github.io/T0-Time-Mass-Duality/2/pdf/cosmicEn.pdf}
		
		\bibitem{pascher_teilchenmassen_2025}
		Pascher, J. (2025). \textit{T0 Model: Complete Particle Mass Derivations}. HTL Leonding. Available at: \url{https://jpascher.github.io/T0-Time-Mass-Duality/2/pdf/TeilchenmassenDe.pdf} and \url{https://jpascher.github.io/T0-Time-Mass-Duality/2/pdf/TeilchenmassenEn.pdf}
		
		\bibitem{pascher_t0_energie_2025}
		Pascher, J. (2025). \textit{T0 Model: Energy-Based Formulation and Muon g-2}. HTL Leonding. Available at: \url{https://jpascher.github.io/T0-Time-Mass-Duality/2/pdf/T0-EnergieDe.pdf} and \url{https://jpascher.github.io/T0-Time-Mass-Duality/2/pdf/T0-EnergieEn.pdf}
		
		\bibitem{pascher_redshift_2025}
		Pascher, J. (2025). \textit{T0 Model: Wavelength-Dependent Redshift and Deflection}. HTL Leonding. Available at: \url{https://jpascher.github.io/T0-Time-Mass-Duality/2/pdf/redshift_deflectionDe.pdf} and \url{https://jpascher.github.io/T0-Time-Mass-Duality/2/pdf/redshift_deflectionEn.pdf}
		
		\bibitem{pascher_temp_einheiten_2025}
		Pascher, J. (2025). \textit{T0 Model: Natural Units and CMB Temperature}. HTL Leonding. Available at: \url{https://jpascher.github.io/T0-Time-Mass-Duality/2/pdf/TempEinheitenCMBDe.pdf} and \url{https://jpascher.github.io/T0-Time-Mass-Duality/2/pdf/TempEinheitenCMBEn.pdf}
		
		\bibitem{pascher_beta_derivation_2025}
		Pascher, J. (2025). \textit{T0 Model: Beta Parameter Derivation from Field Theory}. HTL Leonding. Available at: \url{https://jpascher.github.io/T0-Time-Mass-Duality/2/pdf/DerivationVonBetaDe.pdf} and \url{https://jpascher.github.io/T0-Time-Mass-Duality/2/pdf/DerivationVonBetaEn.pdf}
		
		\bibitem{myon_g2_2021}
		Muon g-2 Collaboration (2021). \textit{Measurement of the Positive Muon Anomalous Magnetic Moment to 0.46 ppm}. Physical Review Letters 126, 141801.
		
		\bibitem{planck_2020}
		Planck Collaboration (2020). \textit{Planck 2018 Results: Cosmological Parameters}. Astronomy \& Astrophysics 641, A6.
		
		\bibitem{pdg_2022}
		Particle Data Group (2022). \textit{Review of Particle Physics}. Progress of Theoretical and Experimental Physics 2022, 083C01.
		
		\bibitem{weinberg_1995}
		Weinberg, S. (1995). \textit{The Quantum Theory of Fields}. Cambridge University Press.
	\end{thebibliography}
	
\end{document}