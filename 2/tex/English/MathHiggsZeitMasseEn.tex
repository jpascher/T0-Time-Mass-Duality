\documentclass[a4paper,12pt]{article}
\usepackage[utf8]{inputenc}
\usepackage[T1]{fontenc}
\usepackage[ngerman]{babel}
\usepackage{lmodern}
\usepackage{csquotes}
\usepackage{tocloft}
\usepackage{xcolor}
\usepackage{amsmath}
\usepackage{amssymb}
\usepackage{physics}
\usepackage{booktabs}
\usepackage{array}
\usepackage{tabularx}
\usepackage{fancyhdr}
\usepackage[margin=2cm]{geometry}
\usepackage[colorlinks=true, linkcolor=blue, citecolor=blue, urlcolor=blue]{hyperref}
\usepackage{siunitx}

\renewcommand{\cftsecfont}{\color{blue}}
\renewcommand{\cftsubsecfont}{\color{blue}}
\renewcommand{\cftsecpagefont}{\color{blue}}
\renewcommand{\cftsubsecpagefont}{\color{blue}}
\setlength{\cftsecindent}{1cm}
\setlength{\cftsubsecindent}{2cm}

% Custom commands
\newcommand{\Tfield}{T(x)}
\newcommand{\betaT}{\beta_{\text{T}}}
\newcommand{\alphaEM}{\alpha_{\text{EM}}}
\newcommand{\Tzero}{T_0}
\newcommand{\DhiggsT}{\Tfield (\partial_\mu + ig A_\mu) \Phi + \Phi \partial_\mu \Tfield}
\newcommand{\gammaf}{\gamma_{\text{Lorentz}}}

\pagestyle{fancy}
\fancyhf{}
\fancyhead[L]{Johann Pascher}
\fancyhead[R]{Time-Mass Duality}
\fancyfoot[C]{\thepage}
\renewcommand{\headrulewidth}{0.4pt}
\renewcommand{\footrulewidth}{0.4pt}

\title{Mathematical Formulation of the Higgs Mechanism in Time-Mass Duality}
\author{Johann Pascher}
\date{March 28, 2025}

\begin{document}
	
	\maketitle
	
	\begin{abstract}
		This work develops a precise mathematical formulation of the Higgs mechanism within the framework of the T0 model, a novel time-mass duality theory. Assuming that time and mass are complementary aspects of the same fundamental reality, it demonstrates how the Higgs mechanism serves as a mediator between two equivalent descriptions: the conventional view with time dilation and constant rest mass, and an alternative view with absolute time and variable mass. The formulation not only leads to an elegant mathematical structure but also provides concrete, experimentally testable predictions that deviate from the Standard Model of particle physics.
	\end{abstract}
	
	\tableofcontents
	\newpage
	
	\section{Introduction}
	Modern theoretical physics is based on two fundamental yet incompletely reconciled theories: relativity and quantum mechanics. While relativity describes time and space as dynamic, observer-dependent quantities, quantum mechanics treats time as an external parameter. This conceptual tension might hint at a deeper structure that could unify both perspectives.
	
	In this work, we explore an alternative theoretical foundation based on the idea of a fundamental duality between time and mass. Similar to the wave-particle duality in quantum mechanics, we propose that time and mass represent two complementary descriptions of the same physical reality. Whereas conventional relativity treats time as relative (time dilation) and rest mass as constant, we suggest a mathematically equivalent perspective in which time is absolute and mass is variable.
	
	The Higgs mechanism plays a special role in this context, as it is responsible for generating particle masses in the Standard Model. In our dual formulation, the Higgs field becomes the central mediator between both perspectives, defining both the rest mass and the intrinsic timescale of all particles. Particularly noteworthy is that the unique position of the Higgs boson in the particle zoo—as the only particle without a clear “mirror image”—finds a natural explanation within this framework.
	
	In the following, we develop a mathematically precise formalism for this time-mass duality, reformulate the fundamental field equations, and derive concrete experimental implications. This theory does not represent a break with established physics but extends its interpretive framework and could uncover deeper connections between seemingly independent phenomena such as quantum coherence, Higgs interactions, and cosmological observations.
	
	\section{Starting Point: Higgs Mechanism in the Standard Model}
	In the Standard Model, the Higgs field is introduced as a complex scalar doublet:
	\begin{equation}
		\Phi = \begin{pmatrix} \phi^+ \\ \phi^0 \end{pmatrix}
	\end{equation}
	The Lagrangian density for the Higgs field is:
	\begin{equation}
		\mathcal{L}_{\text{Higgs}} = (D_\mu \Phi)^\dagger (D^\mu \Phi) - V(\Phi)
	\end{equation}
	with the Higgs potential:
	\begin{equation}
		V(\Phi) = -\mu^2 \Phi^\dagger \Phi + \lambda (\Phi^\dagger \Phi)^2
	\end{equation}
	The Yukawa coupling describes the interaction of the Higgs field with fermions:
	\begin{equation}
		\mathcal{L}_{\text{Yukawa}} = -y_f \bar{\psi}_L \Phi \psi_R + \text{h.c.}
	\end{equation}
	After spontaneous symmetry breaking, the Higgs field acquires a vacuum expectation value (VEV):
	\begin{equation}
		\langle \Phi \rangle = \frac{1}{\sqrt{2}} \begin{pmatrix} 0 \\ v \end{pmatrix}
	\end{equation}
	The fermion masses are then given by:
	\begin{equation}
		m_f = \frac{y_f v}{\sqrt{2}}
	\end{equation}
	
	\section{Reformulation within the Framework of Time-Mass Duality}
	\subsection{Time Dilation Perspective (Standard Relativity)}
	In this perspective, the rest mass of particles remains constant, while time is relative (time dilation). The mass-energy relation is:
	\begin{equation}
		E = \gammaf m_0 c^2
	\end{equation}
	where \( \gammaf = \frac{1}{\sqrt{1-v^2/c^2}} \) is the Lorentz factor.
	
	Time dilation is described by:
	\begin{equation}
		t' = \gammaf t
	\end{equation}
	The Yukawa coupling in this perspective directly yields a constant rest mass:
	\begin{equation}
		m_0 = \frac{y_f v}{\sqrt{2}}
	\end{equation}
	
	\subsection{Mass Variation Perspective (T0 Model)}
	In this alternative perspective, time \( \Tzero \) is absolute (constant), while mass is variable. The intrinsic time field is defined as:
	\begin{equation}
		\Tfield = \frac{\hbar}{\max(m c^2, \omega)}
	\end{equation}
	where \( m c^2 \) applies to massive particles and \( \omega \) to photons (as energy) to ensure a unified treatment. The transformation relation to the standard perspective is:
	\begin{equation}
		m = \gammaf m_0
	\end{equation}
	and
	\begin{equation}
		\Tfield = \frac{\Tzero}{\gammaf}
	\end{equation}
	where \( \Tzero = \frac{\hbar}{m_0 c^2} \) is the intrinsic time at rest.
	
	\section{The Higgs Field as a Mediator of Time-Mass Duality}
	\subsection{Modified Higgs Lagrangian Density}
	In the T0 model, the Higgs Lagrangian density is modified:
	\begin{equation}
		\mathcal{L}_{\text{Higgs-T}} = |\DhiggsT|^2 - V(\Tfield, \Phi)
	\end{equation}
	where \( V(\Tfield, \Phi) = -\mu^2 \Phi^\dagger \Phi + \lambda (\Phi^\dagger \Phi)^2 \) is the Higgs potential, and the modified covariant derivative is defined as:
	\begin{equation}
		\DhiggsT = \Tfield (\partial_\mu + ig A_\mu) \Phi + \Phi \partial_\mu \Tfield
	\end{equation}
	
	\subsection{Modified Yukawa Coupling}
	The Yukawa coupling is reformulated in the T0 model:
	\begin{equation}
		\mathcal{L}_{\text{Yukawa-T}} = -y_f \bar{\psi}_L \Phi \psi_R + \text{h.c.}
	\end{equation}
	This leads to a velocity-dependent mass:
	\begin{equation}
		m(v) = \gammaf \cdot \frac{y_f v}{\sqrt{2}} = \gammaf m_0
	\end{equation}
	while the intrinsic time field scales accordingly:
	\begin{equation}
		\Tfield(v) = \frac{\hbar}{m(v)c^2} = \frac{\hbar}{\gammaf m_0 c^2} = \frac{\Tzero}{\gammaf}
	\end{equation}
	
	\subsection{Higgs Field as a Bridge Between Perspectives}
	In the new framework, the Higgs field plays a dual role:
	\begin{enumerate}
		\item It generates the rest mass \( m_0 \) through its VEV in the standard perspective.
		\item It defines the intrinsic timescale \( \Tzero = \frac{\hbar}{m_0 c^2} \) in the duality perspective.
	\end{enumerate}
	The fundamental connection is expressed by:
	\begin{equation}
		\Tzero \cdot m_0 c^2 = \hbar
	\end{equation}
	This relationship holds in both perspectives, as:
	\begin{equation}
		\Tfield \cdot m c^2 = \frac{\Tzero}{\gammaf} \cdot \gammaf m_0 c^2 = \Tzero \cdot m_0 c^2 = \hbar
	\end{equation}
	
	\section{Field Equations in Dual Formulation}
	\subsection{Klein-Gordon Equation}
	The standard Klein-Gordon equation for the Higgs boson is:
	\begin{equation}
		(\Box + m_H^2) h(x) = 0
	\end{equation}
	In the T0 model, it is modified to:
	\begin{equation}
		i\hbar \Tfield \frac{\partial h_T}{\partial t} + i\hbar h_T \frac{\partial \Tfield}{\partial t} + \frac{\hbar^2}{2 m_H} \nabla^2 h_T = 0
	\end{equation}
	
	\section{Lagrangian Formulation}
	The total Lagrangian density of the T0 model is:
	\begin{multline}
		\mathcal{L}_{\text{Total}} = \mathcal{L}_{\text{Boson}} + \mathcal{L}_{\text{Fermion}} + \mathcal{L}_{\text{Higgs-T}} + \mathcal{L}_{\text{intrinsic}}, \\
		\mathcal{L}_{\text{intrinsic}} = \frac{1}{2} \partial_\mu \Tfield \partial^\mu \Tfield - V(\Tfield)
	\end{multline}
	
	\section{Cosmological Implications}
	The T0 model implies:
	\begin{itemize}
		\item Modified Gravitational Potential: \( \Phi(r) = -\frac{GM}{r} + \kappa r \), \( \kappa \approx 4.8 \times 10^{-11} \, \text{m/s}^2 \)
		\item Cosmic Redshift: \( 1 + z = e^{\alpha d} \), \( \alpha \approx 2.3 \times 10^{-28} \, \text{m}^{-1} \)
		\item Wavelength Dependence: \( z(\lambda) = z_0 (1 + \betaT \ln(\lambda/\lambda_0)) \), in natural units \(\betaT = 1\)
	\end{itemize}
	Gravitation arises from \( \nabla \Tfield \).
	
	\section{Uncertainty in \(\betaT\)}
	The parameter \(\betaT\) is derived in the T0 model as:
	\begin{equation}
		\betaT = \frac{\lambda_h^2 v^2}{16\pi^3} \cdot \frac{1}{m_h^2} \cdot \frac{1}{\xi}
	\end{equation}
	where \(\xi \approx 1.33 \times 10^{-4}\). In natural units, \(\betaT = 1\) is exact \cite{pascher_alphabeta_2025}, while \(\betaT \approx 0.008\) in SI units is estimated from observations \cite{pascher_massenvariation_2025}, introducing uncertainties. Further tests are required.
	
	\section{Conclusion}
	The dual formulation of the Higgs mechanism in the T0 model offers a mathematically coherent reformulation that is not only conceptually elegant but also provides concrete, testable predictions. The theory interprets the Higgs mechanism not merely as a mass generator but also as a mediator between two complementary perspectives of reality: the conventional view with time dilation and constant rest mass, and an alternative view with absolute time and variable mass.
	
	\begin{thebibliography}{99}
		\bibitem{pascher_zeit_2025} Pascher, J. (2025). \href{https://github.com/jpascher/T0-Time-Mass-Duality/tree/main/2/pdf/English/ZeitEmergentQMEn.pdf}{Time as an Emergent Property in Quantum Mechanics: A Connection Between Relativity, Fine-Structure Constant, and Quantum Dynamics}. March 23, 2025.
		\bibitem{pascher_lagrange_2025} Pascher, J. (2025). \href{https://github.com/jpascher/T0-Time-Mass-Duality/tree/main/2/pdf/English/MathZeitMasseLagrangeEn.pdf}{From Time Dilation to Mass Variation: Mathematical Core Formulations of Time-Mass Duality Theory}. March 29, 2025.
		\bibitem{pascher_photon_2025} Pascher, J. (2025). \href{https://github.com/jpascher/T0-Time-Mass-Duality/tree/main/2/pdf/English/DynMassePhotonenNichtlokalEn.pdf}{Dynamic Mass of Photons and Its Implications for Nonlocality in the T0 Model}.
		\bibitem{pascher_erweiterung_2025} Pascher, J. (2025). \href{https://github.com/jpascher/T0-Time-Mass-Duality/tree/main/2/pdf/English/NotwendigkeitQMErweiterungEn.pdf}{The Necessity of Extending Standard Quantum Mechanics and Quantum Field Theory}. March 27, 2025.
		\bibitem{pascher_massenvariation_2025} Pascher, J. (2025). \href{https://github.com/jpascher/T0-Time-Mass-Duality/tree/main/2/pdf/English/MassVarGalaxienEn.pdf}{Mass Variation in Galaxies: An Analysis in the T0 Model with Emergent Gravitation}. March 30, 2025.
		\bibitem{pascher_higgs_2025} Pascher, J. (2025). \href{https://github.com/jpascher/T0-Time-Mass-Duality/tree/main/2/pdf/English/MathHiggsZeitMasseEn.pdf}{Mathematical Formulation of the Higgs Mechanism in Time-Mass Duality}. March 28, 2025.
		\bibitem{pascher_feldtheorie_2025} Pascher, J. (2025). \href{https://github.com/jpascher/T0-Time-Mass-Duality/tree/main/2/pdf/English/FeldtheorieQuantenEn.pdf}{Field Theory and Quantum Correlations: A New Perspective on Instantaneity}. March 28, 2025.
		\bibitem{pascher_messdifferenzen_2025} Pascher, J. (2025). \href{https://github.com/jpascher/T0-Time-Mass-Duality/tree/main/2/pdf/English/MessdifferenzenT0StandardEn.pdf}{Compensatory and Additive Effects: An Analysis of Measurement Differences Between the T0 Model and the \(\Lambda\)CDM Standard Model}. April 2, 2025.
		\bibitem{pascher_planck_2025} Pascher, J. (2025). \href{https://github.com/jpascher/T0-Time-Mass-Duality/tree/main/2/pdf/English/JenseitsPlanckEn.pdf}{Real Consequences of Reformulating Time and Mass in Physics: Beyond the Planck Scale}. March 24, 2025.
		\bibitem{pascher_alpha_2025} Pascher, J. (2025). \href{https://github.com/jpascher/T0-Time-Mass-Duality/tree/main/2/pdf/English/NatEinheitenAlpha1En.pdf}{Energy as a Fundamental Unit: Natural Units with \(\alphaEM = 1\) in the T0 Model}. March 26, 2025.
		\bibitem{pascher_alphabeta_2025} Pascher, J. (2025). \href{https://github.com/jpascher/T0-Time-Mass-Duality/tree/main/2/pdf/English/Alpha1Beta1KonsistenzEn.pdf}{Unified Unit System in the T0 Model: The Consistency of \(\alpha = 1\) and \(\beta = 1\)}. April 5, 2025.
		\bibitem{pascher_temp_2025} Pascher, J. (2025). \href{https://github.com/jpascher/T0-Time-Mass-Duality/tree/main/2/pdf/English/NatEinheitenAlpha1En.pdf}{Adjustment of Temperature Units in Natural Units and CMB Measurements}. April 2, 2025.
		\bibitem{pascher_params_2025} Pascher, J. (2025). \href{https://github.com/jpascher/T0-Time-Mass-Duality/tree/main/2/pdf/English/ZeitMasseT0ParamsEn.pdf}{Time-Mass Duality Theory (T0 Model): Derivation of Parameters \(\kappa\), \(\alpha\), and \(\beta\)}. April 4, 2025.
		\bibitem{pascher_emergente_gravitation_2025} Pascher, J. (2025). \href{https://github.com/jpascher/T0-Time-Mass-Duality/tree/main/2/pdf/English/EmergentGravT0En.pdf}{Emergent Gravitation in the T0 Model: A Comprehensive Derivation}. April 1, 2025.
	\end{thebibliography}
	
\end{document}