\documentclass[a4paper,12pt]{article}
\usepackage[utf8]{inputenc}
\usepackage[T1]{fontenc}
\usepackage{lmodern}
\usepackage[english]{babel}
\usepackage{amsmath, amssymb, amsthm}
\usepackage{geometry}
\usepackage{xcolor}
\usepackage{tocloft}
\usepackage{siunitx}
\DeclareSIUnit{\year}{yr}
\DeclareSIUnit{\parsec}{pc}
\usepackage{fancyhdr}

\usepackage{hyperref}
\hypersetup{
	colorlinks=true,
	linkcolor=blue,
	filecolor=blue,
	citecolor=blue,
	urlcolor=blue,
	bookmarks=true,
	bookmarksopen=true,
	pdftitle={Simplified Description of Fundamental Forces with Time-Mass Duality},
	pdfauthor={Johann Pascher},
}

\usepackage{cleveref}

\geometry{a4paper, margin=2cm}

% Headers and Footers
\pagestyle{fancy}
\fancyhf{}
\fancyhead[L]{Johann Pascher}
\fancyhead[R]{Time-Mass Duality}
\fancyfoot[C]{\thepage}
\renewcommand{\headrulewidth}{0.4pt}
\renewcommand{\footrulewidth}{0.4pt}

\renewcommand{\cftsecfont}{\color{blue}}
\renewcommand{\cftsubsecfont}{\color{blue}}
\renewcommand{\cftsecpagefont}{\color{blue}}
\renewcommand{\cftsubsecpagefont}{\color{blue}}
\setlength{\cftsecindent}{1cm}
\setlength{\cftsubsecindent}{2cm}

% Custom commands
\newcommand{\Tfield}{T(x)}
\newcommand{\DcovT}[1]{\Tfield D_\mu #1 + #1 \partial_\mu \Tfield}
\newcommand{\DhiggsT}{\Tfield (\partial_\mu + ig A_\mu) \Phi + \Phi \partial_\mu \Tfield}
\newcommand{\betaT}{\beta_{\text{T}}}
\newcommand{\alphaEM}{\alpha_{\text{EM}}}
\newcommand{\alphaW}{\alpha_{\text{W}}}
\newcommand{\Mpl}{M_{\text{Pl}}}
\newcommand{\Tzerot}{T_0(\Tfield)}
\newcommand{\Tzero}{T_0}
\newcommand{\vecx}{\vec{x}}
\newcommand{\gammaf}{\gamma_{\text{Lorentz}}}

\title{Simplified Description of Fundamental Forces with Time-Mass Duality}
\author{Johann Pascher}
\date{March 27, 2025}

\begin{document}
	
	\maketitle
	
	\begin{abstract}
		This work presents a unified approach to the four fundamental forces—strong, weak, electromagnetic, and gravitational—based on the time-mass duality theory. Traditional physics considers these forces separately, but in the T0 model with time-mass duality, they can be unified within a single Lagrangian density that naturally encompasses both established interactions and gravitation. This density is expressed as:
		$\mathcal{L}_\text{total} = \mathcal{L}_\text{SM} + \mathcal{L}_\text{Higgs} + \mathcal{L}_\text{intrinsic}$
		Here, $\mathcal{L}_\text{SM}$ represents the Standard Model interactions—strong, electromagnetic, and weak forces; $\mathcal{L}_\text{Higgs}$ describes the Higgs field dynamics; and $\mathcal{L}_\text{intrinsic}$ introduces the concept of intrinsic time, reflecting time-mass duality. Notably, gravitation is not added as a separate force but emerges from the dynamics of the intrinsic time field.
	\end{abstract}
	
	\tableofcontents
	\newpage
	
	\section{Unified Lagrangian Density with Time-Mass Duality Concept}
	
	Physics describes the world through four fundamental forces—strong, weak, electromagnetic, and gravitational—traditionally considered separately. In the T0 model, based on time-mass duality, these forces can be unified in a single Lagrangian density that naturally encompasses both known interactions and gravitation. This density is given by:
	
	\begin{equation}
		\mathcal{L}_\text{total} = \mathcal{L}_\text{SM} + \mathcal{L}_\text{Higgs} + \mathcal{L}_\text{intrinsic}
	\end{equation}
	
	Here $\mathcal{L}_\text{SM}$ represents the Standard Model interactions—the strong, electromagnetic, and weak forces; $\mathcal{L}_\text{Higgs}$ describes the Higgs field dynamics; and $\mathcal{L}_\text{intrinsic}$ introduces the concept of intrinsic time, reflecting the time-mass duality. Notably, gravitation is not added as a separate force but emerges from the dynamics of the intrinsic time field, as detailed in "Mathematical Core Formulations" \cite{pascher_lagrange_2025} and "Emergent Gravitation in the T0 Model" \cite{pascher_emergente_gravitation_2025}.
	
	\subsection{Standard Model}
	
	The Standard Model forms the basis for describing the three forces that determine particle behavior at the atomic level. Its Lagrangian density consists of:
	
	\begin{equation}
		\mathcal{L}_\text{SM} = \mathcal{L}_\text{strong} + \mathcal{L}_\text{em} + \mathcal{L}_\text{weak}
	\end{equation}
	
	Here $\mathcal{L}_\text{strong} = -\frac{1}{4} F_{\mu\nu}^a F^{a\mu\nu} + \bar{\psi}(i \gamma^\mu D_\mu - m_\psi(\phi))\psi$ represents the strong nuclear force, which binds quarks into protons and neutrons; $\mathcal{L}_\text{em} = -\frac{1}{4} F_{\mu\nu} F^{\mu\nu} + \bar{\psi}(i \gamma^\mu D_\mu - m_\psi(\phi))\psi$ represents the electromagnetic force, which couples electrons to nuclei; and $\mathcal{L}_\text{weak} = -\frac{1}{4} W_{\mu\nu}^a W^{a\mu\nu} + \bar{\psi}(i \gamma^\mu D_\mu - m_\psi(\phi))\psi$ represents the weak force, which governs processes like radioactive decay. This conventional description follows the standard quantum field theory formulation \cite{weinberg1995quantum}.
	
	In the T0 model, this description is adjusted by replacing time dilation with mass variation, leading to a dual formulation:
	
	\begin{equation}
		\mathcal{L}_\text{SM-T} = \mathcal{L}_\text{strong-T} + \mathcal{L}_\text{em-T} + \mathcal{L}_\text{weak-T}
	\end{equation}
	
	Here the time derivative is bound to the intrinsic time $\Tfield$, such that $\partial_t \rightarrow \partial_{t/T}$, an adjustment that reinterprets dynamics under absolute time. This modification is fundamental to the T0 model approach and is directly connected to the concept of time as an emergent property, as described in \cite{pascher_zeit_2025}.
	
	\subsection{Higgs Field}
	
	The Higgs field, responsible for mass generation, is described in the Standard Model by:
	
	\begin{equation}
		\mathcal{L}_\text{Higgs} = (D_\mu \phi)^\dagger (D^\mu \phi) - V(\phi)
	\end{equation}
	
	where $\phi$ is the Higgs field and $V(\phi) = \mu^2 \phi^\dagger \phi + \lambda (\phi^\dagger \phi)^2$ is the potential. This formulation follows the original work of Higgs, Englert, and Brout \cite{higgs1964broken, englert1964broken}.
	
	In the T0 model, this formula is extended to include the intrinsic time:
	
	\begin{equation}
		\mathcal{L}_\text{Higgs-T} = (D_{T\mu} \phi_T)^\dagger (D_T^\mu \phi_T) - V_T(\phi_T)
	\end{equation}
	
	The covariant derivative $D_{T\mu}$ accounts for the time-mass duality and emphasizes the role of the Higgs field as a medium for mass and time, as elaborated in "Mathematical Formulation of the Higgs Mechanism" \cite{pascher_higgs_2025}. This modification reveals the deeper connection between mass generation and the intrinsic timescale of particles, a key insight of the T0 model that goes beyond the standard interpretation of the Higgs mechanism.
	
	\subsection{Lagrangian Density for Intrinsic Time}
	
	The central innovation of the T0 model is the complete Lagrangian density for intrinsic time, which in its complete form can be expressed as:
	
	\begin{equation}
		\mathcal{L}_{\text{intrinsic}}^{\text{complete}} = \underbrace{\frac{1}{2} \partial_\mu \Tfield \partial^\mu \Tfield - \frac{1}{2}\Tfield^2}_{\text{Free field dynamics}} + \underbrace{\bar{\psi} \left( i\hbar \gamma^0 \frac{\partial}{\partial (t/\Tfield)} - i\hbar \gamma^0 \frac{\partial}{\partial t} \right) \psi}_{\text{Interaction with matter}}
	\end{equation}
	
	This formulation encompasses both the free field dynamics of $\Tfield$ and its interaction with matter. In earlier fundamental works, a simplified version was presented, focusing only on the matter interaction term:
	
	\begin{equation}
		\mathcal{L}_\text{intrinsic}^{\text{simplified}} = \bar{\psi} \left( i\hbar \gamma^0 \frac{\partial}{\partial (t/T)} - i\hbar \gamma^0 \frac{\partial}{\partial t} \right) \psi
	\end{equation}
	
	The complete formulation reveals important consequences: When applying the variational principle, we obtain the field equation with source term:
	
	\begin{equation}
		\nabla^2 \Tfield + \Tfield = -\kappa\rho(x)\Tfield^2
	\end{equation}
	
	This equation shows how matter density functions as a source for the intrinsic time field and generates the emergent gravitational effects that are central to the T0 model. The free field component enables a wave-like propagation of the $\Tfield$ field, while the matter interaction component couples this field to particles and generates mass-dependent dynamics.
	
	Here, $\Tfield = \frac{\hbar}{m c^2}$ is the intrinsic time, dependent on mass. This extended formulation, developed in "The Necessity of Extending Standard Quantum Mechanics" \cite{pascher_erweiterung_2025}, connects particle dynamics with their individual timescales and enables a unified description of all forces. The intrinsic time has the dimension $[E^{-1}]$ in natural units, which maintains dimensional consistency throughout the formalism, as discussed in \cite{pascher_alpha_2025} and \cite{pascher_alphabeta_2025}.
	
	\section{Simplified Description of Mass Terms with Time-Mass Duality}
	
	In the Standard Model, a particle's mass is defined by its coupling to the Higgs field: $m_\psi(\phi) = y_\psi \phi$, where the mass remains constant and time is variable. In the T0 model, this view is reversed: time remains absolute, and the mass varies with the Lorentz factor $\gamma$:
	
	\begin{equation}
		m_\psi(\phi_T) = y_\psi \phi_T \cdot \gamma, \quad \gamma = \frac{1}{\sqrt{1 - v^2/c^2}}
	\end{equation}
	
	This dual description, derived in "Time-Mass Duality Theory" \cite{pascher_params_2025}, explains the same phenomena as time dilation but offers a new perspective on the role of mass. This reformulation of the effects of special relativity theory preserves the experimental predictions of Einstein's theory \cite{einstein1905}, but provides a conceptually different framework that aligns with the fundamental time-mass duality, as studied in \cite{pascher_zeit_masse_2025}.
	
	\section{The Higgs Field as a Universal Medium with Intrinsic Time}
	
	The Higgs field is more than a mechanism for mass generation—in the T0 model, it also determines the intrinsic timescales of particles. This relationship is expressed as:
	
	\begin{equation}
		\Tfield = \frac{\hbar}{m(\phi) c^2} = \frac{\hbar}{y_\psi \phi \cdot c^2}
	\end{equation}
	
	The intrinsic time of a particle is thus inversely proportional to its mass, which is generated by the Higgs field. This perspective extends the role of the Higgs field as a universal medium that affects all interactions, as studied in "Higgs Mechanism" \cite{pascher_higgs_2025}. The unique position of the Higgs boson in the particle spectrum gains new meaning in this framework, as it mediates not only mass but also the fundamental temporal properties of all other particles.
	
	In natural units, where $\hbar = c = 1$, this relationship simplifies to $\Tfield = \frac{1}{m}$, highlighting the fundamental duality between time and mass. When additionally $\alphaEM = \betaT = 1$ is set in the unified natural unit system, this relationship reveals further elegance and simplicity, as discussed in \cite{pascher_alphabeta_2025}.
	
	\section{The Higgs Field and the Vacuum: A Complex Relationship with Intrinsic Time}
	
	Vacuum energy, a central problem in modern physics, is reinterpreted in the T0 model. Instead of a sum of zero-point energies, it could be described as:
	
	\begin{equation}
		E_\text{Vacuum} = \sum_i \frac{\hbar}{2 T_i}
	\end{equation}
	
	where $T_i$ is the intrinsic time of the quantum fluctuations. This formulation connects vacuum energy to the dynamics of the Higgs field and time-mass duality, offering new insights into the cosmological constant problem \cite{weinberg1989cosmological}. By directly connecting vacuum energy to the intrinsic timescales of quantum fluctuations, the T0 model offers a potential path to resolving the enormous discrepancy between the predictions of quantum field theory and astronomical observations of vacuum energy.
	
	\section{Quantum Entanglement and Nonlocality in Time-Mass Duality}
	
	The apparent immediacy of quantum entanglement is reconsidered in the T0 model through intrinsic time. In the $T_0$ model, correlations do not arise instantaneously but through mass variations. In entangled particles with different masses, the time evolution varies with their intrinsic times. For photons, this is defined as:
	
	\begin{equation}
		\Tfield = \frac{\hbar}{E_{\gamma}} e^{\alpha^{\text{SI}} x}, \quad \alpha^{\text{SI}} = \frac{H_0}{c} \approx \SI{2,3e-18}{\per\meter}
	\end{equation}
	
	which reflects energy loss over distances, as described in "Dynamic Mass of Photons" \cite{pascher_photons_2025}. This approach offers a novel perspective on the paradoxes of quantum nonlocality \cite{bell1964}, reinterpreting the apparent instantaneous action over distances in terms of the varying intrinsic timescales of quantum systems. Further implications for quantum correlations and Bell's theorem are explored in \cite{pascher_feldtheorie_2025}.
	
	\section{Cosmological Implications of Time-Mass Duality}
	
	The T0 model offers natural explanations for cosmological phenomena through three key parameters: $\alpha^{\text{SI}} \approx \SI{2,3e-18}{\per\meter}$ describes the energy loss of photons over cosmic distances, $\kappa^{\text{SI}} \approx \SI{4,8e-11}{\meter\per\second\squared}$ characterizes the strength of the dark energy field in galactic dynamics, and $\betaT^{\text{SI}} \approx 0.008$ quantifies the coupling to baryonic matter. The gravitational potential becomes:
	
	\begin{equation}
		\Phi(r) = -\frac{G M}{r} + \kappa r
	\end{equation}
	
	where $\kappa$ has the dimension $[E]$ in natural units. These parameters, derived in "Mass Variation in Galaxies" \cite{pascher_galaxies_2025} and "Measurement Differences" \cite{pascher_messdifferenzen_2025}, explain flat rotation curves and redshift without requiring dark matter or cosmic expansion.
	
	This modified gravitational potential agrees with the observational data from galaxy rotation curves \cite{rubin1980} and the radial acceleration relation \cite{McGaugh2016} while providing a more parsimonious explanation than models with dark matter. The linear term $\kappa r$ in the potential leads to an additional constant force component that dominates at large distances, naturally explaining the flattening of galaxy rotation curves.
	
	Moreover, the interpretation of cosmic redshift in the T0 model as energy loss rather than expansion provides an alternative to standard $\Lambda$CDM cosmology \cite{Planck2018} and potentially resolves tensions in Hubble constant measurements without requiring dark energy.
	
	\section{Summary of the Unified Theory}
	
	The unified theory is described by the action:
	
	\begin{equation}
		S_\text{unified} = \int \left( \mathcal{L}_\text{standard} + \mathcal{L}_\text{complementary} + \mathcal{L}_\text{coupling} \right) d^4x
	\end{equation}
	
	where $\mathcal{L}_\text{standard}$ is the Standard Model, $\mathcal{L}_\text{complementary}$ the dual formulation, and $\mathcal{L}_\text{coupling}$ the time-mass interaction. This approach bridges quantum mechanics and gravitation, offers new insights into entanglement and cosmological phenomena, and is experimentally testable.
	
	The T0 model requires no exotic components like dark matter or dark energy, instead explaining these phenomena through the fundamental properties of the intrinsic time field. This unification of seemingly disparate physical phenomena through a single conceptual framework represents a significant step toward a more coherent understanding of nature.
	
	\section{Experimental Testability}
	
	The T0 model makes several specific, experimentally testable predictions that distinguish it from the Standard Model and conventional cosmology:
	
	\begin{itemize}
		\item \textbf{Photon Energy Loss:} The parameter $\alpha^{\text{SI}} \approx \SI{2,3e-18}{\per\meter}$ predicts a distance-dependent energy loss for photons that could be tested through precision spectroscopy of distant sources.
		
		\item \textbf{Modified Gravitational Potential:} The parameter $\kappa^{\text{SI}} \approx \SI{4,8e-11}{\meter\per\second\squared}$ leads to deviations from Newtonian gravitation at large distances, which could be recognized through careful measurements of galaxy dynamics or solar system ephemerides.
		
		\item \textbf{Wavelength-Dependent Redshift:} The model predicts a logarithmic dependence of redshift on wavelength, characterized by $\betaT^{\text{SI}} \approx 0.008$, which could be tested with multi-wavelength observations of distant galaxies.
		
		\item \textbf{Mass-Dependent Quantum Coherence:} The coupling of the intrinsic time field predicts that quantum coherence times should depend on mass, which could be verified through interference experiments with particles of different mass.
	\end{itemize}
	
	These predictions are detailed in "Parameter Derivations" \cite{pascher_params_2025} and "Measurement Differences" \cite{pascher_messdifferenzen_2025} and provide a clear path for experimental verification or falsification of the T0 model.
	
	\section{References to Further Works}
	
	This theory builds on my earlier works, which are listed in the bibliography and examine various aspects of time-mass duality in depth. Together, these works form a comprehensive framework that addresses fundamental questions of physics from a novel perspective and offers potential solutions for long-standing problems in quantum gravitation, cosmology, and the unification of fundamental forces.
	
	\begin{thebibliography}{99}
		\bibitem{pascher_zeit_2025} Pascher, J. (2025). \href{https://github.com/jpascher/T0-Time-Mass-Duality/tree/main/2/pdf/English/ZeitEmergentQMEn.pdf}{Time as an Emergent Property in Quantum Mechanics: A Connection Between Relativity, Fine-Structure Constant, and Quantum Dynamics}. March 23, 2025.
		\bibitem{pascher_zeit_masse_2025} Pascher, J. (2025). \href{https://github.com/jpascher/T0-Time-Mass-Duality/tree/main/2/pdf/English/ZeitMasseNeuerBlickEn.pdf}{Time and Mass: A New Look at Old Formulas – and Liberation from Traditional Constraints}. March 22, 2025.
		\bibitem{pascher_params_2025} Pascher, J. (2025). \href{https://github.com/jpascher/T0-Time-Mass-Duality/tree/main/2/pdf/English/ZeitMasseT0ParamsEn.pdf}{Time-Mass Duality Theory (T0 Model): Derivation of Parameters $\kappa$, $\alpha$ and $\beta$}. April 4, 2025.
		\bibitem{pascher_galaxies_2025} Pascher, J. (2025). \href{https://github.com/jpascher/T0-Time-Mass-Duality/tree/main/2/pdf/English/MassVarGalaxienEn.pdf}{Mass Variation in Galaxies: An Analysis in the T0 Model with Emergent Gravitation}. March 30, 2025.
		\bibitem{pascher_messdifferenzen_2025} Pascher, J. (2025). \href{https://github.com/jpascher/T0-Time-Mass-Duality/tree/main/2/pdf/English/MessdifferenzenT0StandardEn.pdf}{Compensatory and Additive Effects: An Analysis of Measurement Differences Between the T0 Model and the $\Lambda$CDM Standard Model}. April 2, 2025.
		\bibitem{pascher_lagrange_2025} Pascher, J. (2025). \href{https://github.com/jpascher/T0-Time-Mass-Duality/tree/main/2/pdf/English/MathZeitMasseLagrangeEn.pdf}{From Time Dilation to Mass Variation: Mathematical Core Formulations of Time-Mass Duality Theory}. March 29, 2025.
		\bibitem{pascher_photons_2025} Pascher, J. (2025). \href{https://github.com/jpascher/T0-Time-Mass-Duality/tree/main/2/pdf/English/DynMassePhotonenNichtlokalEn.pdf}{Dynamic Mass of Photons and Its Implications for Nonlocality in the T0 Model}. March 25, 2025.
		\bibitem{pascher_erweiterung_2025} Pascher, J. (2025). \href{https://github.com/jpascher/T0-Time-Mass-Duality/tree/main/2/pdf/English/NotwendigkeitQMErweiterungEn.pdf}{The Necessity of Extending Standard Quantum Mechanics and Quantum Field Theory}. March 27, 2025.
		\bibitem{pascher_higgs_2025} Pascher, J. (2025). \href{https://github.com/jpascher/T0-Time-Mass-Duality/tree/main/2/pdf/English/MathHiggsZeitMasseEn.pdf}{Mathematical Formulation of the Higgs Mechanism in Time-Mass Duality}. March 28, 2025.
		\bibitem{pascher_emergente_gravitation_2025} Pascher, J. (2025). \href{https://github.com/jpascher/T0-Time-Mass-Duality/tree/main/2/pdf/English/EmergentGravT0En.pdf}{Emergent Gravitation in the T0 Model: A Comprehensive Derivation}. April 1, 2025.
		\bibitem{pascher_alpha_2025} Pascher, J. (2025). \href{https://github.com/jpascher/T0-Time-Mass-Duality/tree/main/2/pdf/English/NatEinheitenAlpha1En.pdf}{Energy as a Fundamental Unit: Natural Units with $\alphaEM = 1$ in the T0 Model}. March 26, 2025.
		\bibitem{pascher_alphabeta_2025} Pascher, J. (2025). \href{https://github.com/jpascher/T0-Time-Mass-Duality/tree/main/2/pdf/English/Alpha1Beta1KonsistenzEn.pdf}{Unified Unit System in the T0 Model: The Consistency of $\alphaEM = 1$ and $\betaT = 1$}. April 5, 2025.
		\bibitem{pascher_feldtheorie_2025} Pascher, J. (2025). \href{https://github.com/jpascher/T0-Time-Mass-Duality/tree/main/2/pdf/English/FeldtheorieQuantenEn.pdf}{Field Theory and Quantum Correlations: A New Perspective on Instantaneity}. March 28, 2025.
		\bibitem{einstein1905} Einstein, A. (1905). On the Electrodynamics of Moving Bodies. \textit{Annalen der Physik}, 322(10), 891-921.
		\bibitem{higgs1964broken} Higgs, P. W. (1964). Broken Symmetries and the Masses of Gauge Bosons. \textit{Physical Review Letters}, 13(16), 508-509.
		\bibitem{englert1964broken} Englert, F., \& Brout, R. (1964). Broken Symmetry and the Mass of Gauge Vector Mesons. \textit{Physical Review Letters}, 13(9), 321-323.
		\bibitem{weinberg1995quantum} Weinberg, S. (1995). \textit{The Quantum Theory of Fields, Volume I: Foundations}. Cambridge University Press.
		\bibitem{weinberg1989cosmological} Weinberg, S. (1989). The Cosmological Constant Problem. \textit{Reviews of Modern Physics}, 61(1), 1-23.
		\bibitem{bell1964} Bell, J. S. (1964). On the Einstein Podolsky Rosen Paradox. \textit{Physics}, 1(3), 195-200.
		\bibitem{rubin1980} Rubin, V. C., Ford, W. K., \& Thonnard, N. (1980). Rotational Properties of 21 SC Galaxies with a Large Range of Luminosities and Radii. \textit{The Astrophysical Journal}, 238, 471-487.
		\bibitem{McGaugh2016} McGaugh, S. S., Lelli, F., \& Schombert, J. M. (2016). Radial Acceleration Relation in Rotationally Supported Galaxies. \textit{Physical Review Letters}, 117(20), 201101.
		\bibitem{Planck2018} Planck Collaboration. (2020). Planck 2018 Results. VI. Cosmological Parameters. \textit{Astronomy \& Astrophysics}, 641, A6.
	\end{thebibliography}
	
\end{document}