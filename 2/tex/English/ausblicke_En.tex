\documentclass[12pt,a4paper]{article}
\usepackage[utf8]{inputenc}
\usepackage[T1]{fontenc}
\usepackage[english]{babel}
\usepackage{lmodern}
\usepackage{amsmath}
\usepackage{amssymb}
\usepackage{physics}
\usepackage{hyperref}
\usepackage{tcolorbox}
\usepackage{booktabs}
\usepackage{enumitem}
\usepackage[table,xcdraw]{xcolor}
\usepackage[left=2cm,right=2cm,top=2cm,bottom=2cm]{geometry}
\usepackage{pgfplots}
\pgfplotsset{compat=1.18}
\usepackage{graphicx}
\usepackage{float}
\usepackage{fancyhdr}
\usepackage{siunitx}
\usepackage{array}
\usepackage{cleveref}

% Headers and Footers
\pagestyle{fancy}
\fancyhf{}
\fancyhead[L]{Johann Pascher}
\fancyhead[R]{Unifying Fields in the T0 Model}
\fancyfoot[C]{\thepage}
\renewcommand{\headrulewidth}{0.4pt}
\renewcommand{\footrulewidth}{0.4pt}

% Custom commands
\newcommand{\Tfield}{T(x)}
\newcommand{\Tfieldt}{T(x,t)}
\newcommand{\alphaEM}{\alpha_{\text{EM}}}
\newcommand{\alphaW}{\alpha_{\text{W}}}
\newcommand{\betaT}{\beta_{\text{T}}}
\newcommand{\Mpl}{M_{\text{Pl}}}
\newcommand{\Tzerot}{T_0(\Tfield)}
\newcommand{\Tzero}{T_0}
\newcommand{\vecx}{\vec{x}}
\newcommand{\gammaf}{\gamma_{\text{Lorentz}}}
\newcommand{\DhiggsT}{\Tfield (\partial_\mu + ig A_\mu) \Phi + \Phi \partial_\mu \Tfield}
\newcommand{\DhiggsTt}{\Tfieldt (\partial_\mu + ig A_\mu) \Phi + \Phi \partial_\mu \Tfieldt}
\newcommand{\LCDM}{\Lambda\text{CDM}}
\newcommand{\DTmu}{D_{T,\mu}}
\newcommand{\calL}{\mathcal{L}}
\newcommand{\deq}{\displaystyle}
\newcommand{\e}{\mathrm{e}}
\newcommand{\dTdt}{\frac{d\Tfieldt}{dt}}
\newcommand{\pdTdt}{\frac{\partial\Tfieldt}{\partial t}}
\newcommand{\pdTdx}{\nabla\Tfieldt}

\hypersetup{
	colorlinks=true,
	linkcolor=blue,
	citecolor=blue,
	urlcolor=blue,
	pdftitle={The Emerging Unified Framework: Relationships Between Fundamental Fields in the T0 Model},
	pdfauthor={Johann Pascher},
	pdfsubject={Theoretical Physics},
	pdfkeywords={T0 Model, Higgs field, intrinsic time field, vacuum, unified theory, theoretical physics}
}

\begin{document}
	
	\title{The Emerging Unified Framework:\\Relationships Between Fundamental Fields in the T0 Model}
	\author{Johann Pascher\\
		Department of Communications Engineering, \\Höhere Technische Bundeslehranstalt (HTL), Leonding, Austria\\
		\texttt{johann.pascher@gmail.com}}
	\date{\today}
	
	\maketitle
	
	\begin{abstract}
		This paper explores the profound relationships between seemingly distinct fundamental fields—the Higgs field, the vacuum with its electromagnetic constants, and the intrinsic time field—within the T0 model framework. We demonstrate that these connections mirror the previously established relationship between various gravitational theories and the T0 model, suggesting an emergent pattern of theoretical unification. By analyzing the coupled Lagrangian density that directly connects the Higgs field and the intrinsic time field, we reveal their potential identity as different manifestations of the same underlying phenomenon. The mathematical relationship $\xi = \lambda_h/(32\pi^3)$ between the Higgs self-coupling and the fundamental T0 parameter provides a quantitative bridge between particle physics and gravitational phenomena. This paper suggests that the T0 model represents an evolving unification framework rather than a completed theory, systematically integrating diverse physical concepts while preserving their mathematical achievements and reinterpreting their ontological foundations. These insights offer a path toward a more coherent understanding of physical reality with significant implications for quantum gravity, cosmology, and the philosophy of science.
	\end{abstract}
	\newpage
	\tableofcontents
	\newpage
	\section{Introduction}
	\label{sec:introduction}
	
	In the development of the T0 model, a striking pattern has emerged: fundamentally different physical concepts—previously considered as distinct entities—appear to be manifestations of the same underlying reality when viewed through the lens of the intrinsic time field. This paper focuses on three such apparently distinct concepts: the Higgs field of the Standard Model, the vacuum with its electromagnetic constants ($\varepsilon_0$ and $\mu_0$), and the intrinsic time field of the T0 model itself.
	
	The investigation of these relationships is particularly compelling for several reasons:
	
	\begin{enumerate}
		\item Each of these fields or concepts addresses a fundamental aspect of physical reality: the Higgs field explains the origin of mass, the vacuum constants determine electromagnetic interactions, and the intrinsic time field mediates between quantum and gravitational phenomena.
		
		\item Each concept comes from a different theoretical tradition: the Higgs field from quantum field theory, vacuum constants from electromagnetism, and the intrinsic time field from the T0 model's approach to unify quantum mechanics and relativity.
		
		\item Despite their different origins, these concepts show surprising mathematical and conceptual connections when examined within the T0 framework.
	\end{enumerate}
	
	This pattern of unification parallels an earlier finding in the T0 model: that various gravitational theories—from String Theory to Loop Quantum Gravity to Asymptotically Safe Gravity—can be understood as different approximations of the more fundamental T0 description, each valid in specific domains \cite{pascher_completetheory_2025}.
	
	The present analysis suggests that we are witnessing the emergence of a more comprehensive unified framework rather than merely adding another competing theory to the landscape of theoretical physics. The T0 model appears to function as an integrative metatheory that can accommodate diverse theoretical approaches while providing a more coherent ontological foundation.
	
	This paper is structured as follows: Section \ref{sec:fields_overview} provides an overview of the three fundamental fields/concepts and their conventional interpretations. Section \ref{sec:mathematical_connections} explores the mathematical connections between these concepts, focusing particularly on the coupled Lagrangian density linking the Higgs and intrinsic time fields. Section \ref{sec:unified_perspective} develops a unified perspective that integrates these concepts. Section \ref{sec:parallel_gravitation} draws parallels with the relationship between gravitational theories and the T0 model. Section \ref{sec:emerging_framework} discusses the implications for theoretical physics and the nature of scientific theory development. Finally, Section \ref{sec:conclusion} summarizes the findings and outlines directions for future research.
	
	\section{Overview of Fundamental Fields and Concepts}
	\label{sec:fields_overview}
	
	Before exploring their connections, it is essential to understand the conventional interpretations of the Higgs field, the vacuum, and the intrinsic time field.
	
	\subsection{The Higgs Field}
	\label{subsec:higgs_field}
	
	The Higgs field, a cornerstone of the Standard Model of particle physics, is conventionally understood as a scalar field that permeates all of space. Its primary function is to break electroweak symmetry and provide mass to elementary particles through the Higgs mechanism. Key characteristics include:
	
	\begin{itemize}
		\item The Higgs potential $V(\Phi) = \lambda(|\Phi|^2 - v^2)^2$, which drives spontaneous symmetry breaking
		\item The vacuum expectation value $v \approx 246$ GeV, which sets the scale of electroweak symmetry breaking
		\item The Higgs self-coupling $\lambda_h \approx 0.13$, determined from the Higgs boson mass $m_h \approx 125$ GeV
		\item The relation $m_h^2 = 2\lambda_h v^2$, connecting these parameters
	\end{itemize}
	
	Despite its success, the conventional understanding of the Higgs field leaves significant puzzles, particularly the hierarchy problem: why the Higgs mass is so much smaller than the Planck mass, representing a fine-tuning of approximately 16 orders of magnitude \cite{Weinberg1989}.
	
	\subsection{The Vacuum and Its Constants}
	\label{subsec:vacuum}
	
	The vacuum in modern physics is far from empty space. It is characterized by specific constants that determine how electromagnetic fields propagate:
	
	\begin{itemize}
		\item The electric permittivity of free space $\varepsilon_0 \approx 8.85 \times 10^{-12}$ F/m
		\item The magnetic permeability of free space $\mu_0 = 4\pi \times 10^{-7}$ H/m
		\item Their relationship to the speed of light: $c = 1/\sqrt{\varepsilon_0\mu_0}$
		\item The fine-structure constant $\alphaEM = e^2/(4\pi\varepsilon_0\hbar c) \approx 1/137.036$
	\end{itemize}
	
	In quantum field theory, the vacuum is further characterized as the state of lowest energy, filled with virtual particles and quantum fluctuations. This conception leads to the vacuum catastrophe—the enormous discrepancy between the predicted vacuum energy density and the observed cosmological constant \cite{Weinberg1989}.
	
	\subsection{The Intrinsic Time Field}
	\label{subsec:time_field}
	
	The intrinsic time field $\Tfieldt$, central to the T0 model, represents a fundamentally new concept that mediates between quantum mechanics and relativity theory. It is defined as:
	
	\begin{equation}
		\Tfieldt = \frac{\hbar}{\max(m(\vecx,t)c^2, \omega(\vecx,t))}
	\end{equation}
	
	Key characteristics include:
	
	\begin{itemize}
		\item For massive particles: $\Tfieldt = \hbar/(m(\vecx,t)c^2)$
		\item For photons: $\Tfieldt = \hbar/\omega(\vecx,t)$
		\item The field equation: $\partial_{\mu}\partial^{\mu}\Tfieldt + \Tfieldt + \rho(\vecx,t)/\Tfieldt^2 = 0$
		\item Its relationship to gravitational potential: $\Phi(\vecx) = -\ln(\Tfieldt/\Tzero)$
	\end{itemize}
	
	The intrinsic time field inverts the conventional relationship between time and mass: instead of relative time and constant mass (as in relativity), it posits absolute time and variable mass \cite{pascher_part1_2025}.
	
	\section{Mathematical Connections Between Fundamental Fields}
	\label{sec:mathematical_connections}
	
	The T0 model reveals surprising mathematical connections between these seemingly distinct concepts, suggesting they may be different aspects of the same underlying reality.
	
	\subsection{Coupled Lagrangian Density}
	\label{subsec:coupled_lagrangian}
	
	A particularly revealing connection appears in the total Lagrangian density of the T0 model, which includes a term directly coupling the Higgs field and the intrinsic time field:
	
	\begin{equation}
		\mathcal{L}_{\text{Higgs-T}} = |\DhiggsTt|^2 - \lambda(|\Phi|^2 - v^2)^2
	\end{equation}
	
	with:
	
	\begin{equation}
		\DhiggsTt = \Tfieldt (\partial_\mu + ig A_\mu) \Phi + \Phi \partial_\mu \Tfieldt
	\end{equation}
	
	This modified covariant derivative creates a direct interaction between the intrinsic time field and the Higgs field, significantly beyond a mere formal similarity. The coupling involves two crucial terms:
	
	\begin{enumerate}
		\item $\Tfieldt (\partial_\mu + ig A_\mu) \Phi$: The time field scales the ordinary covariant derivative of the Higgs field
		\item $\Phi \partial_\mu \Tfieldt$: Gradients in the time field couple directly to the Higgs field value
	\end{enumerate}
	
	This coupling suggests that the dynamics of the Higgs field and the intrinsic time field are intrinsically interconnected, not merely analogous.
	
	\subsection{The Bridging Relation $\xi = \lambda_h/(32\pi^3)$}
	\label{subsec:bridging_relation}
	
	A quantitative relationship of profound significance emerges in the T0 model: the connection between the Higgs self-coupling $\lambda_h$ and the fundamental T0 parameter $\xi = r_0/l_P \approx 1.33 \times 10^{-4}$, which defines the ratio between the T0 characteristic length and the Planck length:
	
	\begin{equation}
		\xi = \frac{\lambda_h}{32\pi^3} \approx 1.31 \times 10^{-4}
	\end{equation}
	
	This relationship can also be derived from:
	
	\begin{equation}
		\xi = \frac{\lambda_h^2 v^2}{16\pi^3 m_h^2} \approx 1.33 \times 10^{-4}
	\end{equation}
	
	The consistency between these derivations strongly supports the connection between the Higgs mechanism and the intrinsic time field \cite{pascher_alphabeta_2025}.
	
	This mathematical bridge links particle physics (through $\lambda_h$, $v$, and $m_h$) directly to gravitational physics (through $\xi$ and the Planck length), suggesting a common origin for phenomena previously considered distinct.
	
	\subsection{Unification of Constants in the Natural Unit System}
	\label{subsec:unified_constants}
	
	In the unified natural unit system of the T0 model, all fundamental constants are set to unity:
	
	\begin{equation}
		\hbar = c = G = k_B = \alphaEM = \alphaW = \betaT = 1
	\end{equation}
	
	This includes not only dimensional constants ($\hbar$, $c$, $G$, $k_B$) but also dimensionless coupling constants:
	
	\begin{itemize}
		\item $\alphaEM = 1$ (naturally $\approx 1/137.036$)
		\item $\alphaW = 1$ (naturally $\approx 2.82$)
		\item $\betaT = 1$ (naturally $\approx 0.008$)
	\end{itemize}
	
	This unification is not arbitrary but reflects a deeper unity in nature. Particularly revealing is the normalization of the fine-structure constant $\alphaEM = e^2/(4\pi\varepsilon_0\hbar c) = 1$, which connects electromagnetic phenomena to the intrinsic time field framework \cite{pascher_alpha_2025}.
	
	With $\hbar = c = \varepsilon_0 = 1$, setting $\alphaEM = 1$ yields $e = \sqrt{4\pi} \approx 3.544$, making electric charge a derived rather than fundamental quantity. This aligns with the view that electromagnetic interactions emerge from the more fundamental time field dynamics.
	
	\section{A Unified Perspective}
	\label{sec:unified_perspective}
	
	The mathematical connections explored in Section \ref{sec:mathematical_connections} point toward a unified perspective in which the Higgs field, vacuum properties, and the intrinsic time field represent different aspects of the same underlying phenomenon.
	
	\subsection{Hierarchical Integration}
	\label{subsec:hierarchical_integration}
	
	A hierarchical perspective emerges from this analysis:
	
	\begin{enumerate}
		\item \textbf{Most Fundamental Level}: The intrinsic time field $\Tfieldt$ represents the primary entity, from which other phenomena emerge
		
		\item \textbf{Intermediate Level}: The vacuum, characterized by constants $\varepsilon_0$ and $\mu_0$ and the fine-structure constant $\alpha_{EM}$, can be understood as a specific configuration of the time field
		
		\item \textbf{Domain-Specific Level}: The Higgs field can be interpreted as a specialized manifestation of the time field that focuses on the mass-generation mechanism in particle physics
	\end{enumerate}
	
	This hierarchy is not one of importance but of generality and explanatory scope. The intrinsic time field provides a more comprehensive framework that naturally accommodates both vacuum properties and the Higgs mechanism.
	
	\subsection{Unified Field Equations}
	\label{subsec:unified_equations}
	
	From the coupled Lagrangian, we derive field equations that demonstrate the interconnected nature of these concepts:
	
	For the Higgs field:
	\begin{equation}
		(\Tfieldt D_\mu)^2 \Phi - 2\lambda\Phi(|\Phi|^2 - v^2) = 0
	\end{equation}
	
	For the time field:
	\begin{equation}
		\partial_\mu\partial^\mu\Tfieldt + \Tfieldt + \frac{\rho(\vecx,t)}{\Tfieldt^2} + |D_\mu\Phi|^2 = 0
	\end{equation}
	
	These equations reveal a self-consistent feedback loop:
	\begin{itemize}
		\item The time field influences how the Higgs field propagates and interacts
		\item The Higgs field contributes to the energy density that shapes the time field
		\item Both fields evolve in tandem, constraining each other
	\end{itemize}
	
	This mutual influence suggests that rather than being separate entities, these fields represent aspects of a unified field structure.
	
	\subsection{Resolution of Longstanding Problems}
	\label{subsec:resolution_problems}
	
	The unified perspective offers potential resolutions to longstanding problems in physics:
	
	\begin{enumerate}
		\item \textbf{The Hierarchy Problem}: The relation $\xi = \lambda_h/(32\pi^3) \approx 1.33 \times 10^{-4}$ provides a natural explanation for why the Higgs mass is so much smaller than the Planck mass, defining a natural scale transition between Planck physics and Standard Model physics.
		
		\item \textbf{The Cosmological Constant Problem}: The intrinsic time field provides a dynamic mechanism that regulates vacuum energy density, potentially resolving the enormous discrepancy between quantum field theory predictions and cosmological observations.
		
		\item \textbf{Quantum Gravity Incompatibility}: By recognizing the Higgs field and vacuum properties as aspects of the time field, a natural path emerges for integrating quantum and gravitational phenomena within a single framework.
	\end{enumerate}
	
	These resolutions arise not through ad hoc adjustments but as natural consequences of the unified framework.
	
	\section{Parallel with Gravitational Theories}
	\label{sec:parallel_gravitation}
	
	The relationship between the Higgs field, vacuum properties, and the intrinsic time field displays a striking parallel to the previously established relationship between various gravitational theories and the T0 model \cite{pascher_completetheory_2025}.
	
	\subsection{Structural Similarities}
	\label{subsec:structural_similarities}
	
	\begin{table}[h]
		\centering
		\begin{tabular}{|p{0.45\textwidth}|p{0.45\textwidth}|}
			\hline
			\textbf{Gravitational Theories Integration} & \textbf{Fundamental Fields Integration} \\
			\hline
			Various gravitational theories as approximations of the T0 model & Higgs field and vacuum properties as aspects of the intrinsic time field \\
			\hline
			Mathematical equivalence with different conceptual bases & Coupled mathematical structure with unified Lagrangian density \\
			\hline
			Domain-specific validity (e.g., String Theory at high energies) & Domain-specific focus (Higgs field for particle physics) \\
			\hline
			Conceptual simplification through the T0 model & Unified perspective through the intrinsic time field \\
			\hline
		\end{tabular}
		\caption{Parallel structures in theoretical integration within the T0 model}
		\label{tab:parallel_structures}
	\end{table}
	
	This parallel suggests a deeper pattern in how the T0 model relates to established theories: not as a replacement but as an integrative framework that preserves their mathematical achievements while providing a more coherent ontological foundation.
	
	\subsection{Metatheoretical Framework}
	\label{subsec:metatheory}
	
	Both cases demonstrate how the T0 model functions as a metatheoretical framework:
	
	\begin{enumerate}
		\item It accommodates the mathematical structures of established theories
		\item It reinterprets their ontological foundations
		\item It reveals connections between seemingly disparate domains
		\item It simplifies the overall theoretical landscape
	\end{enumerate}
	
	This metatheoretical character represents a significant advancement over conventional approaches that typically develop competing theories rather than integrative frameworks.
	
	\section{The Emerging Unified Framework}
	\label{sec:emerging_framework}
	
	The parallel patterns in how the T0 model relates to both gravitational theories and fundamental fields suggest that we are witnessing the emergence of a broader unification framework rather than merely another competing theory.
	
	\subsection{Evolution Rather Than Completion}
	\label{subsec:evolution}
	
	It is crucial to recognize that the T0 model, despite its unifying potential, represents an evolving framework rather than a completed "theory of everything." The parallels identified in this paper indicate a direction of theoretical development toward greater unification, but the journey is ongoing.
	
	This perspective aligns with the history of theoretical physics, where major advances often involve recognizing deeper connections between previously distinct domains rather than creating entirely new theories ex nihilo. For example, Maxwell's electromagnetism unified electricity and magnetism, and Einstein's special relativity unified electromagnetism with mechanics.
	
	\subsection{Scientific Theory Development}
	\label{subsec:theory_development}
	
	The emerging pattern suggests a particular model of scientific theory development:
	
	\begin{enumerate}
		\item \textbf{Divergent Phase}: Development of specialized theories for different domains (e.g., quantum field theory, general relativity)
		
		\item \textbf{Recognition Phase}: Identification of formal similarities and connections between these theories
		
		\item \textbf{Unification Phase}: Development of a more comprehensive framework that reveals these theories as different aspects of the same deeper reality
		
		\item \textbf{Simplification Phase}: Conceptual clarification leading to greater theoretical elegance and parsimony
	\end{enumerate}
	
	The T0 model appears to represent the transition between the recognition and unification phases for both gravitational theories and fundamental fields.
	
	\subsection{Next Steps in Framework Development}
	\label{subsec:next_steps}
	
	For the continued development of this unified framework, several avenues are particularly promising:
	
	\begin{enumerate}
		\item \textbf{Identifying further connections}: Exploring potential relationships between the intrinsic time field and other fundamental concepts, such as the inflaton field of cosmic inflation or the quantum vacuum fluctuations
		
		\item \textbf{Deriving precise mathematical relationships}: Further developing quantitative connections like $\xi = \lambda_h/(32\pi^3)$ to establish a comprehensive network of theoretical bridges
		
		\item \textbf{Developing experimental tests}: Identifying phenomena where the unified perspective would lead to predictions that differ from conventional approaches
		
		\item \textbf{Conceptual refinement}: Continuing to clarify the ontological status of the intrinsic time field as the foundation of a unified framework
	\end{enumerate}
	
	These steps would further solidify the T0 model as an integrative framework rather than merely another competing theory.
	
	\section{Conclusion}
	\label{sec:conclusion}
	
	This paper has explored the profound connections between the Higgs field, vacuum properties, and the intrinsic time field within the T0 model. The analysis reveals that these seemingly distinct concepts may represent different aspects of the same underlying reality. This parallels the previously established relationship between various gravitational theories and the T0 model, suggesting a broader pattern of theoretical unification.
	
	Key findings include:
	
	\begin{enumerate}
		\item The coupled Lagrangian density directly connects the Higgs field and the intrinsic time field, suggesting they are intrinsically linked rather than merely analogous
		
		\item The mathematical relationship $\xi = \lambda_h/(32\pi^3)$ provides a quantitative bridge between particle physics and gravitational phenomena
		
		\item The unified natural unit system, with $\alphaEM = \betaT = 1$, suggests that electromagnetic interactions and the T0 coupling are fundamentally connected to the intrinsic time field
		
		\item The parallel between how the T0 model relates to gravitational theories and fundamental fields suggests a consistent pattern of theoretical integration
	\end{enumerate}
	
	These connections indicate that the T0 model represents an evolving unification framework rather than a completed theory. It systematically integrates diverse physical concepts while preserving their mathematical achievements and reinterpreting their ontological foundations.
	
	This perspective has significant implications for our understanding of the fundamental nature of reality, suggesting a deeper unity beneath the apparent diversity of physical phenomena. It also offers potential resolutions to longstanding problems such as the hierarchy problem and the cosmological constant problem.
	
	Future work should focus on further developing the mathematical connections, identifying experimental tests, and continuing to refine the conceptual foundations of this emerging unified framework. The T0 model's approach of integration rather than replacement represents a promising direction for theoretical physics, potentially leading toward a more coherent and elegant understanding of the physical universe.
	
	\begin{thebibliography}{99}
		\bibitem{pascher_part1_2025} J. Pascher, \href{https://github.com/jpascher/T0-Time-Mass-Duality/tree/main/2/pdf/English/QMRelTimeMassPart1En.pdf}{Bridging Quantum Mechanics and Relativity through Time-Mass Duality: Part I: Theoretical Foundations}, April 7, 2025.
		\bibitem{pascher_part2_2025} J. Pascher, \href{https://github.com/jpascher/T0-Time-Mass-Duality/tree/main/2/pdf/English/QMRelTimeMassPart2En.pdf}{Bridging Quantum Mechanics and Relativity through Time-Mass Duality: Part II: Cosmological Implications and Experimental Validation}, April 7, 2025.
		\bibitem{pascher_quantum_2025} J. Pascher, \href{https://github.com/jpascher/T0-Time-Mass-Duality/tree/main/2/pdf/English/NotwendigkeitQMErweiterungEn.pdf}{The Necessity of Extending Standard Quantum Mechanics and Quantum Field Theory}, March 27, 2025.
		\bibitem{pascher_lagrange_2025} J. Pascher, \href{https://github.com/jpascher/T0-Time-Mass-Duality/tree/main/2/pdf/English/MathZeitMasseLagrangeEn.pdf}{From Time Dilation to Mass Variation: Mathematical Core Formulations of Time-Mass Duality Theory}, March 29, 2025.
		\bibitem{pascher_alphabeta_2025} J. Pascher, \href{https://github.com/jpascher/T0-Time-Mass-Duality/tree/main/2/pdf/English/Alpha1Beta1KonsistenzEn.pdf}{Unified Unit System in the T0 Model: The Consistency of $\alpha = 1$ and $\beta = 1$}, April 5, 2025.
		\bibitem{pascher_alpha_2025} J. Pascher, \href{https://github.com/jpascher/T0-Time-Mass-Duality/tree/main/2/pdf/English/NatEinheitenAlpha1En.pdf}{Energy as the Fundamental Unit: Natural Units with $\alpha = 1$ in the T0 Model}, March 26, 2025.
		\bibitem{pascher_completetheory_2025} J. Pascher, \href{https://github.com/jpascher/T0-Time-Mass-Duality/tree/main/2/pdf/English/T0-ModelAsCompleteTheory_En.pdf}{The T0 Model as a More Complete Theory Compared to Approximative Gravitational Theories}, May 10, 2025.
		\bibitem{pascher_higgs_2025} J. Pascher, \href{https://github.com/jpascher/T0-Time-Mass-Duality/tree/main/2/pdf/English/MathHiggsZeitMasseEn.pdf}{Higgs Mechanism and Time-Mass Duality: Mathematical Foundations}, March 28, 2025.
		\bibitem{Weinberg1989} S. Weinberg, \textit{The Cosmological Constant Problem}, Rev. Mod. Phys. \textbf{61}, 1 (1989).
		\bibitem{Dirac1938} P. A. M. Dirac, \textit{A New Basis for Cosmology}, Proc. Roy. Soc. London A \textbf{165}, 199 (1938).
		\bibitem{Kuhn1962} T. S. Kuhn, \textit{The Structure of Scientific Revolutions}, University of Chicago Press (1962).
		\bibitem{Feyerabend1975} P. Feyerabend, \textit{Against Method: Outline of an Anarchistic Theory of Knowledge}, New Left Books (1975).
	\end{thebibliography}
	
\end{document}