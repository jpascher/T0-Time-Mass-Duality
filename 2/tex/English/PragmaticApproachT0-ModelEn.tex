\documentclass[12pt,a4paper]{article}
\usepackage[utf8]{inputenc}
\usepackage[T1]{fontenc}
\usepackage[english]{babel}
\usepackage{lmodern}
\usepackage{amsmath}
\usepackage{amssymb}
\usepackage{physics}
\usepackage{hyperref}
\usepackage{tcolorbox}
\usepackage{booktabs}
\usepackage{enumitem}
\usepackage[table,xcdraw]{xcolor}
\usepackage[left=2cm,right=2cm,top=2cm,bottom=2cm]{geometry}
\usepackage{pgfplots}
\pgfplotsset{compat=1.18}
\usepackage{graphicx}
\usepackage{float}
\usepackage{fancyhdr}
\usepackage{siunitx}
\usepackage{array}
\usepackage{cleveref}

% Headers and Footers
\pagestyle{fancy}
\fancyhf{}
\fancyhead[L]{Johann Pascher}
\fancyhead[R]{Pragmatic Application of the T0 Model}
\fancyfoot[C]{\thepage}
\renewcommand{\headrulewidth}{0.4pt}
\renewcommand{\footrulewidth}{0.4pt}

% Custom commands
\newcommand{\Tfield}{T(x)}
\newcommand{\Tfieldt}{T(x,t)}
\newcommand{\alphaEM}{\alpha_{\text{EM}}}
\newcommand{\alphaW}{\alpha_{\text{W}}}
\newcommand{\betaT}{\beta_{\text{T}}}
\newcommand{\Mpl}{M_{\text{Pl}}}
\newcommand{\Tzerot}{T_0(\Tfield)}
\newcommand{\Tzero}{T_0}
\newcommand{\vecx}{\vec{x}}
\newcommand{\gammaf}{\gamma_{\text{Lorentz}}}
\newcommand{\DhiggsT}{\Tfield (\partial_\mu + ig A_\mu) \Phi + \Phi \partial_\mu \Tfield}
\newcommand{\DhiggsTt}{\Tfieldt (\partial_\mu + ig A_\mu) \Phi + \Phi \partial_\mu \Tfieldt}
\newcommand{\LCDM}{\Lambda\text{CDM}}
\newcommand{\DTmu}{D_{T,\mu}}
\newcommand{\calL}{\mathcal{L}}
\newcommand{\deq}{\displaystyle}
\newcommand{\e}{\mathrm{e}}
\newcommand{\dTdt}{\frac{d\Tfieldt}{dt}}
\newcommand{\pdTdt}{\frac{\partial\Tfieldt}{\partial t}}
\newcommand{\pdTdx}{\nabla\Tfieldt}

\hypersetup{
	colorlinks=true,
	linkcolor=blue,
	citecolor=blue,
	urlcolor=blue,
	pdftitle={Pragmatic Application of the T0 Model: Avoiding Direct RT Translations and Necessary Extensions},
	pdfauthor={Johann Pascher},
	pdfsubject={Theoretical Physics},
	pdfkeywords={T0 Model, Relativity Theory, Time Field, Variable Mass, Physical Applications}
}

\begin{document}
	
	\title{Pragmatic Application of the T0 Model:\\Avoiding Direct RT Translations and Necessary Extensions}
	\author{Johann Pascher\\
		Department of Communications Engineering, \\Höhere Technische Bundeslehranstalt (HTL), Leonding, Austria\\
		\texttt{johann.pascher@gmail.com}}
	\date{\today}
	
	\maketitle
	
	\begin{abstract}
		This document addresses the pragmatic application of the T0 model without attempting to create meaningful translations of complex formulations from relativity theory. It explains why a direct translation of relativistic tensor formalisms would be conceptually misleading and would undermine the independent character of the T0 model. Instead, pragmatic applications of already defined T0 equations for important physical phenomena are presented. The document also identifies missing formulas necessary for practical calculations and offers concrete suggestions for their development. Application areas in which relative mass becomes particularly relevant are analyzed. This examination underscores the value of the T0 model as an independent conceptual framework that can describe physical phenomena in a more direct and mathematically simpler way than relativity theory.
	\end{abstract}
	\newpage
	\tableofcontents
	\newpage
	\section{Introduction}
	\label{sec:introduction}
	
	The T0 model of time-mass duality offers a fundamentally different approach to describing relativistic and gravitational phenomena compared to conventional relativity theory (RT). Instead of relative time and constant mass, it postulates absolute time and variable mass, mediated by the intrinsic time field $\Tfieldt$. This conceptual inversion raises the question of how the mathematical structures of RT—particularly the complex tensor formalisms of General Relativity (GR)—could be interpreted or reformulated within the framework of the T0 model.
	
	This document takes the position that attempting a direct "translation" of relativistic formalisms into the T0 model would be conceptually misleading and mathematically unnecessarily complicated. Such a translation would undermine the independent character of the T0 model and obscure its potential elegance and simplicity. Instead, the development of the T0 model should proceed from its own fundamental principles and derive direct mathematical formulations for the relevant physical phenomena.
	
	The goals of this document are:
	\begin{enumerate}
		\item To demonstrate why a direct translation of relativistic formalisms into the T0 model should be avoided
		\item To present pragmatic applications of already defined T0 equations for important physical phenomena
		\item To identify missing formulas necessary for practical calculations
		\item To analyze application areas in which relative mass is particularly relevant
	\end{enumerate}
	
	This approach allows us to fully utilize the unique strengths of the T0 model—particularly its potential conceptual clarity and mathematical simplicity—without forcing it into a conceptually foreign formalism.
	
	\section{Warning Against Direct Translation of Relativistic Formulas}
	\label{sec:warning}
	
	\subsection{The Temptation of Formal Equivalence}
	\label{subsec:temptation}
	
	It is understandable that one might be tempted to seek a direct "translation" of every formula in relativity theory into the T0 model. This temptation stems from the desire for complete mathematical equivalence and the demonstration that the T0 model is at least as powerful as established RT. For example, one might try to construct an equivalent "time-field tensor equation" for the Einstein equations:
	\begin{equation}
		G_{\mu\nu} + \Lambda g_{\mu\nu} = \frac{8\pi G}{c^4} T_{\mu\nu}
	\end{equation}
	or create a complex T0 equivalent for the geodesic equation:
	\begin{equation}
		\frac{d^2 x^\mu}{d\tau^2} + \Gamma^\mu_{\alpha\beta} \frac{dx^\alpha}{d\tau} \frac{dx^\beta}{d\tau} = 0
	\end{equation}
	
	\subsection{Conceptual Problems of Direct Translation}
	\label{subsec:conceptual_problems}
	
	This direct translation strategy is problematic for several reasons:
	
	\begin{enumerate}
		\item \textbf{Conceptual Distortion}: A direct translation would undermine the fundamental distinctiveness of the T0 approach. The T0 model is based on the inversion of a fundamental principle (absolute rather than relative time) and specifically attempts to avoid the complex tensor constructs of RT.
		
		\item \textbf{Artificial Complexity}: The tensors and differential geometric structures of GR were developed specifically for describing curved spacetime. Transferring them to a model based on absolute space and absolute time would introduce artificially complex mathematical structures.
		
		\item \textbf{Obscuring Strengths}: The actual strengths of the T0 model—such as the elegant explanation of galaxy rotation curves without dark matter—would be buried under a mass of RT-like formulas.
		
		\item \textbf{False Basis for Comparison}: RT was developed to describe relative time; its mathematical structures are optimized for this purpose. The T0 model has a different focus and should use mathematical structures optimized for its specific basic assumptions.
	\end{enumerate}
	
	\subsection{Terminological Clarification: The "Intrinsic Time Field"}
	\label{subsec:terminology}
	
	Before proceeding further, a terminological clarification is appropriate. The central concept of the T0 model, $\Tfieldt$, is consistently referred to as the "intrinsic time field" – occasionally abbreviated simply as "time field" for brevity. One might be tempted to consider this field as a type of "gravitational field" due to its effects in producing gravitational phenomena. However, such a renaming would be problematic for several reasons:
	
	\begin{enumerate}
		\item The term "gravitational field" is already strongly associated with specific concepts in classical physics and GR that conceptually differ from the T0 approach.
		
		\item The intrinsic time field has a more fundamental role than merely generating gravity – it determines the basic "clock rate" of physical processes and mediates between mass and time.
		
		\item The designation "intrinsic time field" emphasizes the conceptual novelty of the T0 model and avoids misunderstandings that might arise if an already established term were used.
	\end{enumerate}
	
	The intrinsic time field $\Tfieldt$ is therefore not a gravitational field in the conventional sense, but a more fundamental concept from which gravitational effects emerge as emergent phenomena. This conceptual differentiation is crucial for the correct understanding of the T0 model.
	
	\subsection{Mathematical Example: Geodesic Equation vs. Direct Force Formula}
	\label{subsec:math_example}
	
	A concrete example illustrates the problem of direct translation attempts. In GR, the motion of a test particle is described by the complex geodesic equation:
	
	\begin{equation}
		\frac{d^2 x^\mu}{d\tau^2} + \Gamma^\mu_{\alpha\beta} \frac{dx^\alpha}{d\tau} \frac{dx^\beta}{d\tau} = 0
	\end{equation}
	
	This equation requires calculating 40 Christoffel symbols $\Gamma^\mu_{\alpha\beta}$, which themselves must be derived from partial derivatives of the metric tensor.
	
	In the T0 model, the same motion can be directly described by the force equation:
	
	\begin{equation}
		\vec{F}(\vecx,t) = -\frac{\nabla\Tfieldt(\vecx,t)}{\Tfieldt(\vecx,t)}
	\end{equation}
	
	which leads to the same predictions, but in a conceptually clearer and mathematically simpler way. The attempt to find a "T0 equivalent" for each term of the geodesic equation would destroy this natural simplicity.
	
	\section{Pragmatic Application of Existing T0 Formulas}
	\label{sec:pragmatic_approach}
	
	Instead of trying to find a T0 equivalent for every RT formula, we should start from the T0 equations already formulated and work directly with them to describe physical phenomena. Here are the main areas of application:
	
	\subsection{Prediction of Motion Trajectories}
	\label{subsec:trajectories}
	
	\textbf{RT Approach}: Uses geodesic equations in curved spacetime
	\begin{equation}
		\frac{d^2x^{\mu}}{d\tau^2} + \Gamma^{\mu}_{\alpha\beta} \frac{dx^{\alpha}}{d\tau} \frac{dx^{\beta}}{d\tau} = 0
	\end{equation}
	
	\textbf{T0 Alternative}: Uses the defined force equation directly
	\begin{equation}
		\vec{F}(\vecx,t) = -\frac{\nabla\Tfieldt(\vecx,t)}{\Tfieldt(\vecx,t)}
	\end{equation}
	
	The resulting equation of motion is:
	\begin{equation}
		m(\vecx,t) \cdot \frac{d^2\vecx}{dt^2} = -\frac{\nabla\Tfieldt(\vecx,t)}{\Tfieldt(\vecx,t)}
	\end{equation}
	
	This equation combines Newtonian mechanics with the variable time field and variable mass. For a point-like time field
	\begin{equation}
		\Tfieldt(r) = \Tzero\left(1 - \frac{M}{r} + \kappa r\right)
	\end{equation}
	we obtain the correct planetary orbits including perihelion precession.
	
	\subsection{Description of the Gravitational Potential}
	\label{subsec:gravitational_potential}
	
	\textbf{RT Approach}: Uses the metric component $g_{00}$
	\begin{equation}
		g_{00} \approx 1 + \frac{2\Phi}{c^2}
	\end{equation}
	
	\textbf{T0 Alternative}: The already defined potential
	\begin{equation}
		\Phi(r) = -\ln\left(\frac{\Tfieldt(r)}{\Tzero}\right) = -\frac{GM}{r} + \kappa r
	\end{equation}
	
	This formula serves the same purpose—describing the gravitational potential—in a more direct form that already includes the additional effects (term $\kappa r$) that would require separate constructs like dark matter in RT.
	
	\subsection{Calculation of Redshift}
	\label{subsec:redshift}
	
	\textbf{RT Approach}: Uses spacetime metric and cosmic expansion
	\begin{equation}
		1+z = \frac{a(t_0)}{a(t_{\text{emission}})}
	\end{equation}
	
	\textbf{T0 Alternative}: Uses energy attenuation through the time field
	\begin{equation}
		1+z = e^{\alpha d} \cdot \left(1 + \betaT \ln\left(\frac{\lambda}{\lambda_0}\right)\right)
	\end{equation}
	
	This formula provides a direct calculation of redshift with the additional advantage of capturing the wavelength-dependent component, which is a distinguishing feature of the T0 model.
	
	\subsection{Examination of Astronomical Phenomena}
	\label{subsec:astronomical_phenomena}
	
	\textbf{RT Approach}: Uses specific metric solutions like Schwarzschild
	\begin{equation}
		ds^2 = \left(1-\frac{2GM}{rc^2}\right)c^2dt^2 - \left(1-\frac{2GM}{rc^2}\right)^{-1}dr^2 - r^2d\Omega^2
	\end{equation}
	
	\textbf{T0 Alternative}: Can use existing time field solutions
	\begin{equation}
		\Tfieldt(r) = \Tzero\left(1 - \frac{M}{r} + \kappa r\right)
	\end{equation}
	
	This simpler formula already captures the essential features for testing phenomena such as perihelion precession and light deflection.
	
	\subsection{Analysis of Galaxy Structure}
	\label{subsec:galactic_structure}
	
	\textbf{RT Approach}: Combines GR with dark matter, complex simulations
	
	\textbf{T0 Alternative}: Uses the modified potential directly
	\begin{equation}
		v(r) = \sqrt{\frac{GM}{r} + \kappa r}
	\end{equation}
	
	This formula provides an elegant explanation for galaxy rotation curves without additional dark matter hypotheses.
	
	\subsection{Cosmological Modeling}
	\label{subsec:cosmological_modeling}
	
	\textbf{RT Approach}: Uses Friedmann equations, requires dark energy
	\begin{equation}
		\left(\frac{\dot{a}}{a}\right)^2 = \frac{8\pi G}{3}\rho - \frac{k}{a^2} + \frac{\Lambda}{3}
	\end{equation}
	
	\textbf{T0 Alternative}: Static universe with energy attenuation
	\begin{equation}
		T(z) = T_0 (1+z)(1+\ln(1+z))
	\end{equation}
	
	This formula allows cosmological modeling in a static universe, a conceptually simpler alternative to the expanding universe model.
	
	\subsection{Quantum Mechanical Integration}
	\label{subsec:quantum_integration}
	
	\textbf{RT Approach}: Difficult, leads to divergences and incompatibilities
	
	\textbf{T0 Alternative}: Already formulated extended Schrödinger equation
	\begin{equation}
		i\hbar \Tfieldt \frac{\partial\Psi}{\partial t} + i\hbar \Psi \left[\frac{\partial \Tfieldt}{\partial t} + \vec{v}\cdot\nabla\Tfieldt\right] = \hat{H} \Psi
	\end{equation}
	
	This equation already provides a way to integrate quantum mechanics and gravitation without additional complexity.
	
	\section{Necessary Extensions and Missing Formulas}
	\label{sec:missing_formulas}
	
	For a comprehensive practical application of the T0 model, several important formulas are still missing. The following identifies these and makes concrete suggestions for their development.
	
	\subsection{Quantitative Mass Variation in Gravitational Fields}
	\label{subsec:mass_variation}
	
	\textbf{Missing Formula}: A precise equation describing how the effective mass of an object changes in a gravitational field.
	
	\textbf{Suggestion}:
	\begin{equation}
		m(r) = m_0 \cdot \left(1 - \frac{\Phi(r)}{c^2}\right)^{-1} = m_0 \cdot \left(1 - \frac{GM}{rc^2} + \frac{\kappa r}{c^2}\right)^{-1}
	\end{equation}
	
	This formula would describe the local mass variation as a function of the gravitational potential $\Phi(r)$.
	
	\subsection{Detailed Equation of Motion with Time Field Dynamics}
	\label{subsec:detailed_motion}
	
	\textbf{Missing Formula}: A complete equation of motion that combines inertial effects and time field gradients.
	
	\textbf{Suggestion}:
	\begin{equation}
		\frac{d(m(\vecx,t)\vec{v})}{dt} = -\frac{\nabla\Tfieldt(\vecx,t)}{\Tfieldt(\vecx,t)}
	\end{equation}
	
	This extended form of the equation of motion takes into account that not only position but also mass changes.
	
	\subsection{Mechanism of Wavelength-Dependent Redshift}
	\label{subsec:wavelength_mechanism}
	
	\textbf{Missing Formula}: A detailed formula describing the exact mechanism of wavelength-dependent redshift.
	
	\textbf{Suggestion}:
	\begin{equation}
		z(\lambda,d) = z_0(d) \cdot \left(1 + \betaT \ln\left(\frac{\lambda}{\lambda_0}\right)\right)
	\end{equation}
	\begin{equation}
		z_0(d) = e^{\alpha d} - 1
	\end{equation}
	
	These formulas should be supplemented by a microscopic theory explaining why and exactly how redshift depends on wavelength.
	
	\subsection{Time Field Propagation and Disturbances}
	\label{subsec:field_propagation}
	
	\textbf{Missing Formula}: Equations describing how disturbances in the time field propagate.
	
	\textbf{Suggestion}:
	\begin{equation}
		\frac{\partial^2\Tfieldt}{\partial t^2} - c^2\nabla^2\Tfieldt + c^2\Tfieldt + \frac{c^2\rho(\vecx,t)}{\Tfieldt^2} = 0
	\end{equation}
	
	This wave equation for the time field would describe how changes in the time field propagate with finite speed.
	
	\subsection{Quantitative Calculation of Classical Tests}
	\label{subsec:classical_tests}
	
	\textbf{Missing Formulas}: Precise equations for classical RT tests in the T0 formalism.
	
	\textbf{Suggestions}:
	
	Perihelion precession per orbit:
	\begin{equation}
		\Delta\phi = \frac{6\pi GM}{c^2a(1-e^2)} \cdot \left(1 + \frac{\kappa a^2}{GM}\right)
	\end{equation}
	
	Light deflection:
	\begin{equation}
		\Delta\theta = \frac{4GM}{c^2b} \cdot \left(1 + \frac{\kappa b^2}{2GM}\right)
	\end{equation}
	
	These formulas would make the T0 model directly comparable with precise astronomical measurements.
	
	\subsection{Quantum Effects in Variable Time Field}
	\label{subsec:quantum_effects}
	
	\textbf{Missing Formula}: Precise description of the decoherence rate as a function of the time field.
	
	\textbf{Suggestion}:
	\begin{equation}
		\Gamma_{\text{dec}} = \Gamma_0 \cdot \frac{m(\vecx,t)}{m_0} = \Gamma_0 \cdot \frac{T_0}{\Tfieldt(\vecx,t)}
	\end{equation}
	
	This formula would predict how quantum decoherence varies with local mass.
	
	\subsection{Photon Energy Loss When Traversing the Time Field}
	\label{subsec:photon_energy_loss}
	
	\textbf{Missing Formula}: Detailed mechanism of photon energy loss.
	
	\textbf{Suggestion}:
	\begin{equation}
		\frac{dE}{dx} = -\alpha \cdot E \cdot \left(1 + \betaT \ln\left(\frac{\lambda}{\lambda_0}\right)\right)
	\end{equation}
	
	This differential equation would describe the exact process of how photons lose energy while traversing space.
	
	\subsection{Precise Form of the Modified Potential for Complex Mass Distributions}
	\label{subsec:modified_potential}
	
	\textbf{Missing Formula}: Generalization of the potential for more realistic mass models.
	
	\textbf{Suggestion}:
	\begin{equation}
		\Phi(\vec{r}) = -\int G \frac{\rho(\vec{r}')}{|\vec{r}-\vec{r}'|} d^3r' + \kappa\int|\vec{r}-\vec{r}'|\rho(\vec{r}')d^3r'
	\end{equation}
	
	This integral form would describe the modified potential for arbitrary mass distributions.
	
	\section{Application Areas with Special Relevance of Relative Mass}
	\label{sec:relative_mass_applications}
	
	The variable or "relative" mass in the T0 model—analogous to relative time in RT—is particularly crucial for calculations in the following areas:
	
	\subsection{High-Energy Physics and Particle Accelerators}
	\label{subsec:high_energy_physics}
	
	In particle accelerators, particles reach velocities close to the speed of light, where mass increase becomes dramatic. The T0 model would work directly with mass variation:
	\begin{equation}
		m = \gamma m_0 = \frac{m_0}{\sqrt{1-v^2/c^2}}
	\end{equation}
	
	Applications include:
	\begin{itemize}
		\item Calculation of energy required to achieve certain particle energies
		\item Prediction of collision effects
		\item Interpretation of decay rates of moving particles
	\end{itemize}
	
	In the T0 model, the apparently longer lifetimes of rapidly moving unstable particles would not be explained by "slower time" but by their increased mass, which changes the decay dynamics.
	
	\subsection{Precision Navigation and GPS}
	\label{subsec:gps}
	
	GPS systems must account for relativistic effects to function accurately. While RT calculates a time dilation effect, the T0 model would calculate the frequency shift as a consequence of the mass variation of clock atoms:
	\begin{equation}
		f = f_0 \cdot \frac{T_0}{\Tfieldt(\vecx,t)} = f_0 \cdot \frac{m(\vecx,t)}{m_0}
	\end{equation}
	
	The observed frequency shifts would be predicted quantitatively identically, but from the perspective of variable mass rather than variable time.
	
	\subsection{Astronomical Phenomena Near Massive Objects}
	\label{subsec:massive_objects}
	
	Near black holes and neutron stars, relativistic effects become extreme. The altered frequency of light signals would be calculated through the massive change in the local time field and thus the effective mass of photons:
	\begin{equation}
		\omega' = \omega \cdot \frac{T_0}{\Tfieldt(r)} = \omega \cdot \left(1-\frac{GM}{rc^2} + \frac{\kappa r}{c^2}\right)^{-1}
	\end{equation}
	
	This formula would directly predict redshift from strong gravitational fields.
	
	\subsection{Quantum Coherence in Variable Gravitational Fields}
	\label{subsec:quantum_coherence}
	
	Maintaining quantum coherence depends on precise temporal evolution. Quantum systems in different gravitational environments would show different mass values and thus different coherence times:
	\begin{equation}
		\tau_{\text{coherence}} \propto \frac{\hbar}{m(\vecx,t)} = \Tfieldt(\vecx,t) \cdot c^2
	\end{equation}
	
	This could lead to new predictions about gravitationally induced quantum phase shifts.
	
	\subsection{Analysis of Binary Systems and Gravitational Waves}
	\label{subsec:binary_systems}
	
	Binary systems such as double neutron stars emit gravitational waves. The energy radiation and orbital dynamics would be calculated through time field gradients:
	\begin{equation}
		\dot{E} = -k \left(\frac{d\Tfieldt(\vecx,t)}{dt}\right)^2
	\end{equation}
	
	where $k$ is a constant that depends on the system configuration. This formula would predict orbital decay in binary systems.
	
	\section{Summary and Conclusions}
	\label{sec:conclusion}
	
	This document has shown why attempting a direct "translation" of relativistic formalisms into the T0 model would be conceptually misleading and mathematically unnecessarily complicated. Instead, the development of the T0 model should proceed from its own basic principles and derive direct mathematical formulations for the relevant physical phenomena.
	
	The main findings are:
	
	\begin{enumerate}
		\item \textbf{Conceptual Independence}: The T0 model should be appreciated as an independent approach that offers an alternative, possibly simpler way to describe the same physical phenomena—not as a "translation" of RT into another language.
		
		\item \textbf{Pragmatic Application}: The T0 equations already formulated can be used directly to describe a variety of physical phenomena, from planetary motion to cosmic redshift.
		
		\item \textbf{Necessary Extensions}: For a comprehensive practical application of the T0 model, several important formulas are still missing, particularly for the quantitative description of mass variation in gravitational fields and the detailed equation of motion with time field dynamics.
		
		\item \textbf{Relative Mass}: The variable or "relative" mass in the T0 model is particularly relevant in areas such as high-energy physics, precision navigation, and quantum coherence in gravitational fields.
	\end{enumerate}
	
	The pragmatic approach presented here leverages the strengths of the T0 model—its conceptual simplicity and physical intuition—and avoids the trap of forcing it into an RT-like formalism that would contradict its nature. This enables focusing on the unique predictions of the T0 model, such as wavelength-dependent redshift, and testing them experimentally.
	
	\begin{thebibliography}{99}
		\bibitem{pascher_part1_2025} J. Pascher, \href{https://github.com/jpascher/T0-Time-Mass-Duality/tree/main/2/pdf/English/QMRelTimeMassPart1En.pdf}{Bridging Quantum Mechanics and Relativity through Time-Mass Duality: Part I: Theoretical Foundations}, April 7, 2025.
		\bibitem{pascher_part2_2025} J. Pascher, \href{https://github.com/jpascher/T0-Time-Mass-Duality/tree/main/2/pdf/English/QMRelTimeMassPart2En.pdf}{Bridging Quantum Mechanics and Relativity through Time-Mass Duality: Part II: Cosmological Implications and Experimental Validation}, April 7, 2025.
		\bibitem{pascher_quantum_2025} J. Pascher, \href{https://github.com/jpascher/T0-Time-Mass-Duality/tree/main/2/pdf/English/NotwendigkeitQMErweiterungEn.pdf}{The Necessity of Extending Standard Quantum Mechanics and Quantum Field Theory}, March 27, 2025.
		\bibitem{pascher_lagrange_2025} J. Pascher, \href{https://github.com/jpascher/T0-Time-Mass-Duality/tree/main/2/pdf/English/MathZeitMasseLagrangeEn.pdf}{From Time Dilation to Mass Variation: Mathematical Core Formulations of Time-Mass Duality Theory}, March 29, 2025.
		\bibitem{pascher_emergente_2025} J. Pascher, \href{https://github.com/jpascher/T0-Time-Mass-Duality/tree/main/2/pdf/English/EmergentGravT0En.pdf}{Emergent Gravitation in the T0 Model: A Comprehensive Derivation}, April 1, 2025.
		\bibitem{pascher_galaxies_2025} J. Pascher, \href{https://github.com/jpascher/T0-Time-Mass-Duality/tree/main/2/pdf/English/MassVarGalaxienEn.pdf}{Mass Variation in Galaxies: An Analysis in the T0 Model with Emergent Gravitation}, March 30, 2025.
		\bibitem{pascher_alphabeta_2025} J. Pascher, \href{https://github.com/jpascher/T0-Time-Mass-Duality/tree/main/2/pdf/English/Alpha1Beta1KonsistenzEn.pdf}{Unified Unit System in the T0 Model: The Consistency of $\alpha = 1$ and $\beta = 1$}, April 5, 2025.
		\bibitem{pascher_dynamic_timeField_2025} J. Pascher, \href{https://github.com/jpascher/T0-Time-Mass-Duality/tree/main/2/pdf/English/DynamicTF-SchrodingerExtensions_En.pdf}{Dynamic Extension of the Intrinsic Time Field in the T0 Model: Complete Field-Theoretic Treatment and Implications for Quantum Evolution}, May 5, 2025.
		\bibitem{pascher_esm_comparison_2025} J. Pascher, \href{https://github.com/jpascher/T0-Time-Mass-Duality/tree/main/2/pdf/English/T0vsESM_ConceptualAnalysisEn.pdf}{Conceptual Comparison of T0 Model and Extended Standard Model: Field-Theoretic vs. Dimensional Approaches}, April 25, 2025.
		\bibitem{Will2014} C. M. Will, \textit{The Confrontation between General Relativity and Experiment}, Living Rev. Rel. \textbf{17}, 4 (2014).
		\bibitem{Verlinde2011} E. Verlinde, \textit{On the Origin of Gravity and the Laws of Newton}, J. High Energy Phys. \textbf{2011}, 29 (2011).
	\end{thebibliography}
	
\end{document}