\documentclass[12pt,a4paper]{article}
\usepackage[utf8]{inputenc}
\usepackage[T1]{fontenc}
\usepackage[ngerman]{babel}
\usepackage{lmodern}
\usepackage{csquotes}
\usepackage{amsmath}
\usepackage{amssymb}
\usepackage{physics}
\usepackage{geometry}
\usepackage{tocloft}
\usepackage{xcolor}
\usepackage{graphicx,tikz,pgfplots}
\pgfplotsset{compat=1.18}
\usepackage{booktabs}
\usepackage{siunitx}
\usepackage{amsthm}
\usepackage[colorlinks=true, linkcolor=blue, citecolor=blue, urlcolor=blue]{hyperref}
\usepackage{cleveref}
\usepackage{fancyhdr}

\geometry{a4paper, margin=2cm}

% Headers and Footers
\pagestyle{fancy}
\fancyhf{}
\fancyhead[L]{Johann Pascher}
\fancyhead[R]{Time-Mass Duality}
\fancyfoot[C]{\thepage}
\renewcommand{\headrulewidth}{0.4pt}
\renewcommand{\footrulewidth}{0.4pt}

% Table of Contents Styling
\renewcommand{\cftsecfont}{\color{blue}}
\renewcommand{\cftsubsecfont}{\color{blue}}
\renewcommand{\cftsecpagefont}{\color{blue}}
\renewcommand{\cftsubsecpagefont}{\color{blue}}
\setlength{\cftsecindent}{1cm}
\setlength{\cftsubsecindent}{2cm}

% Custom commands (consistent with other documents)
\newcommand{\Tfield}{T(x)}
\newcommand{\betaT}{\beta_{\text{T}}}
\newcommand{\alphaEM}{\alpha_{\text{EM}}}
\newcommand{\alphaW}{\alpha_{\text{W}}}
\newcommand{\Mpl}{M_{\text{Pl}}}
\newcommand{\Tzerot}{T_0(\Tfield)}
\newcommand{\Tzero}{T_0}
\newcommand{\vecx}{\vec{x}}
\newcommand{\gammaf}{\gamma_{\text{Lorentz}}}
\newcommand{\DhiggsT}{\Tfield (\partial_\mu + ig A_\mu) \Phi + \Phi \partial_\mu \Tfield}

\newtheorem{theorem}{Theorem}[section]
\newtheorem{proposition}[theorem]{Proposition}

\title{The Necessity of Extending Standard Quantum Mechanics and Quantum Field Theory}
\author{Johann Pascher}
\date{March 27, 2025}

\begin{document}
	
	\maketitle
	
	\begin{abstract}
		This work examines the conceptual limitations of standard quantum mechanics (QM) and quantum field theory (QFT) and proposes the time-mass duality with an intrinsic time field as an extension. By introducing \(\Tfield = \frac{\hbar}{\max(m c^2, \omega)}\), a connection between time and mass is established, overcoming the QM-QFT duality and providing a deterministic framework. The theory is supported by experimental predictions and cosmological implications.
	\end{abstract}
	
	\tableofcontents
	\newpage
	
	\section{Introduction: Conceptual Limits of Established Theories}
	Standard quantum mechanics (QM) and quantum field theory (QFT) face limitations in integrating with General Relativity (GR) and in understanding time and mass. The T0 model offers a new perspective, as described in \textit{From Time Dilation to Mass Variation: Mathematical Core Formulations of Time-Mass Duality Theory} \cite{pascher_lagrange_2025}.
	
	\subsection{Inherent Duality Between QM and QFT}
	\begin{itemize}
		\item QM: Particle perspective \cite{schrodinger}.
		\item QFT: Field-based view.
	\end{itemize}
	
	\subsection{Overinterpretation Due to Incomplete Theoretical Foundations}
	\begin{itemize}
		\item Measurement problem \cite{einstein2}.
		\item Nonlocality \cite{bell}.
	\end{itemize}
	
	\section{Asymmetric Treatment of Time and Space}
	\subsection{Time as a Parameter vs. Space as an Operator}
	In standard quantum mechanics, time is treated as a parameter:
	\begin{equation}
		i\hbar \frac{\partial}{\partial t}\Psi(x,t) = \hat{H}\Psi(x,t)
	\end{equation}
	Spatial coordinates, however, are described by operators, resulting in an asymmetric treatment of time and space.
	
	\section{Static Treatment of Mass}
	\subsection{Mass as an Invariable Parameter}
	In the standard formulation, mass remains constant:
	\begin{equation}
		\hat{H} = \frac{\hat{p}^2}{2m} + V(\hat{x})
	\end{equation}
	This static treatment of mass limits the theory’s flexibility and prevents a dynamic integration of mass and time.
	
	\section{The Concept of Intrinsic Time}
	\begin{theorem}[Intrinsic Time]
		The intrinsic time is defined as:
		\begin{equation}
			\Tfield = \frac{\hbar}{\max(m c^2, \omega)}
		\end{equation}
	\end{theorem}
	This definition unifies the treatment of massive particles and photons. The modified Schrödinger equation is:
	\begin{equation}
		i\hbar \Tfield \frac{\partial}{\partial t} \Psi + i\hbar \Psi \frac{\partial \Tfield}{\partial t} = \hat{H} \Psi
	\end{equation}
	This makes time evolution mass-dependent, enabling a more dynamic description.
	
	\section{Time-Mass Duality: A New Theoretical Framework}
	\subsection{Complementary Models}
	\begin{itemize}
		\item Standard Model: Constant mass, variable time.
		\item T0 Model: Absolute time, variable mass.
	\end{itemize}
	Time-mass duality offers an alternative perspective that transcends the limitations of traditional approaches.
	
	\section{Lagrangian Formulation}
	The total Lagrangian density of the T0 model is:
	\begin{equation}
		\mathcal{L}_{\text{Total}} = \mathcal{L}_{\text{Boson}} + \mathcal{L}_{\text{Fermion}} + \mathcal{L}_{\text{Higgs-T}} + \mathcal{L}_{\text{intrinsic}}, \quad \mathcal{L}_{\text{intrinsic}} = \frac{1}{2} \partial_\mu \Tfield \partial^\mu \Tfield - V(\Tfield)
	\end{equation}
	This approach integrates the dynamics of the intrinsic time field into existing field theories.
	
	\section{Implications for Fundamental Phenomena}
	\subsection{Quantum Coherence and Decoherence}
	The decoherence rate becomes mass-dependent:
	\begin{equation}
		\Gamma_{\text{dec}} = \Gamma_0 \cdot \frac{m c^2}{\hbar}
	\end{equation}
	Gravitation emerges as a property from gradients of the intrinsic time field:
	\begin{equation}
		\nabla \Tfield = -\frac{\hbar}{m^2 c^2} \nabla m
	\end{equation}
	with the modified gravitational potential:
	\begin{equation}
		\Phi(r) = -\frac{G M}{r} + \kappa r, \quad \kappa \approx \SI{4.8e-11}{\meter\per\second\squared}
	\end{equation}
	
	\begin{figure}[h]
		\centering
		\begin{tikzpicture}
			\begin{axis}[
				xlabel={Mass [eV]},
				ylabel={Coherence Time [eV\(^{-1}\)]},
				xlabel style={font=\large},
				ylabel style={font=\large},
				tick label style={font=\normalsize},
				xmin=0, xmax=1000,
				ymin=0, ymax=0.01,
				legend pos=north east,
				legend style={font=\large},
				grid=both,
				minor tick num=1
				]
				\addplot[blue, ultra thick, domain=1:1000, samples=100] {1/x};
				\legend{\(\tau \propto 1/m\)}
			\end{axis}
		\end{tikzpicture}
		\caption{Mass-dependent coherence time in the T0 model.}
	\end{figure}
	
	\section{Variable Mass as a Hidden Variable}
	\subsection{Modified Quantum Dynamics}
	Time evolution can also be described by a variable mass:
	\begin{equation}
		i\hbar \frac{\partial}{\partial t}\Psi(x,t) = \hat{H}(m(t))\Psi(x,t)
	\end{equation}
	This suggests that mass could act as a hidden variable explaining the apparent indeterminacy of quantum mechanics.
	
	\section{Cosmological Implications}
	The T0 model has far-reaching cosmological implications:
	\begin{itemize}
		\item Redshift: \( 1 + z = e^{\alpha d} \), \(\alpha \approx \SI{2.3e-18}{\per\meter}\) \cite{pascher_lagrange_2025}.
		\item Gravitational Potential: \(\Phi(r) = -\frac{G M}{r} + \kappa r\), \(\kappa \approx \SI{4.8e-11}{\meter\per\second\squared}\) \cite{pascher_lagrange_2025}.
		\item Wavelength Dependence: \( z(\lambda) = z_0 (1 + \betaT \ln(\lambda/\lambda_0)) \), where \(\betaT^{\text{SI}} \approx 0.008\) in SI units and \(\betaT^{\text{nat}} = 1\) in natural units \cite{pascher_params_2025}.
	\end{itemize}
	
	\section{Conclusion}
	The T0 model extends standard quantum mechanics and quantum field theory by introducing time-mass duality and the intrinsic time field. It provides a deterministic framework that overcomes the traditional QM-QFT duality and is supported by experimental predictions such as mass-dependent coherence times and cosmological effects. This extension could represent a significant step toward a more unified theory of physics, integrating quantum mechanics and gravitation.
	
	\begin{thebibliography}{99}
		\bibitem{pascher_lagrange_2025} Pascher, J. (2025). \href{https://github.com/jpascher/T0-Time-Mass-Duality/tree/main/2/pdf/English/MathZeitMasseLagrange.pdf}{From Time Dilation to Mass Variation: Mathematical Core Formulations of Time-Mass Duality Theory}. March 29, 2025.
		\bibitem{pascher_params_2025} Pascher, J. (2025). \href{https://github.com/jpascher/T0-Time-Mass-Duality/tree/main/2/pdf/English/ZeitMasseT0ParamsEn.pdf}{Time-Mass Duality Theory (T0 Model): Derivation of Parameters \(\kappa\), \(\alpha\), and \(\beta\)}. April 4, 2025.
		\bibitem{einstein} Einstein, A. (1905). \textit{Does the Inertia of a Body Depend Upon Its Energy Content?}. \textit{Annalen der Physik}, 323(13), 639-641.
		\bibitem{planck} Planck, M. (1901). \textit{On the Law of Energy Distribution in the Normal Spectrum}. \textit{Annalen der Physik}, 309(3), 553-563.
		\bibitem{schrodinger} Schrödinger, E. (1926). \textit{An Undulatory Theory of the Mechanics of Atoms and Molecules}. \textit{Physical Review}, 28(6), 1049-1070.
		\bibitem{bell} Bell, J. S. (1964). \textit{On the Einstein-Podolsky-Rosen Paradox}. \textit{Physics}, 1(3), 195-200.
		\bibitem{einstein2} Einstein, A., Podolsky, B., Rosen, N. (1935). \textit{Can Quantum-Mechanical Description of Physical Reality Be Considered Complete?}. \textit{Physical Review}, 47(10), 777-780.
	\end{thebibliography}
	
\end{document}