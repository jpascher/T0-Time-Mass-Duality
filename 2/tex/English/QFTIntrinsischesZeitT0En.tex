\documentclass[12pt,a4paper]{article}
\usepackage[utf8]{inputenc}
\usepackage[T1]{fontenc}
\usepackage[english]{babel}
\usepackage{lmodern}
\usepackage{amsmath}
\usepackage{amssymb}
\usepackage{physics}
\usepackage{hyperref}
\usepackage{bookmark}
\usepackage{tcolorbox}
\usepackage{booktabs}
\usepackage{enumitem}
\usepackage[table,xcdraw]{xcolor}
\usepackage[left=2cm,right=2cm,top=2cm,bottom=2cm]{geometry}
\usepackage{pgfplots}
\pgfplotsset{compat=1.18}
\usepackage{graphicx}
\usepackage{float}
\usepackage{fancyhdr}
\usepackage{siunitx}
\usepackage{url}
\usepackage{bm}

% Acknowledgments environment
\newenvironment{acknowledgments}
{\section*{Acknowledgments}}
{\vspace{1em}}

% Custom commands
\newcommand{\Tfield}{T(x)}
\newcommand{\alphaEM}{\alpha_{\text{EM}}}
\newcommand{\alphaW}{\alpha_{\text{W}}}
\newcommand{\betaT}{\beta_{\text{T}}}
\newcommand{\Mpl}{M_{\text{Pl}}}
\newcommand{\Tzerot}{T_0(\Tfield)}
\newcommand{\Tzero}{T_0}
\newcommand{\vecx}{\vec{x}}
\newcommand{\vr}{\vec{r}}
\newcommand{\gammaf}{\gamma_{\text{Lorentz}}}
\newcommand{\DhiggsT}{\Tfield (\partial_\mu + ig A_\mu) \Phi + \Phi \partial_\mu \Tfield}
\newcommand{\LCDM}{\Lambda\text{CDM}}
\newcommand{\DTmu}{D_{T,\mu}}
\newcommand{\calL}{\mathcal{L}}
\newcommand{\deq}{\displaystyle}

% Header and Footer Configuration
\pagestyle{fancy}
\fancyhf{}
\fancyhead[L]{Johann Pascher}
\fancyhead[R]{Quantum Field Theory of the T0 Model}
\fancyfoot[C]{\thepage}
\renewcommand{\headrulewidth}{0.4pt}
\renewcommand{\footrulewidth}{0.4pt}

\hypersetup{
	colorlinks=true,
	linkcolor=blue,
	citecolor=blue,
	urlcolor=blue,
	pdftitle={Quantum Field Theoretical Treatment of the Intrinsic Time Field in the T0 Model},
	pdfauthor={Johann Pascher},
	pdfsubject={Theoretical Physics},
	pdfkeywords={T0 Model, intrinsic time field, quantum field theory, time-mass duality}
}

\title{Quantum Field Theoretical Treatment of the Intrinsic Time Field in the T0 Model}
\author{Johann Pascher\\
	Department of Communication Technology\\
	Higher Technical Federal Institute (HTL), Leonding, Austria\\
	\texttt{johann.pascher@gmail.com}}
\date{April 8, 2025}

\begin{document}
	
	\maketitle
	
	\begin{abstract}
		This work presents a systematic quantum field theoretical treatment of the intrinsic time field $\Tfield$ in the T0 model. Starting from classical field theory, a complete quantization is developed, encompassing canonical commutation relations, path integral formalism, and interaction dynamics. Particular attention is given to the integration of the quantized time field with Standard Model fields through modified propagators and extended Feynman rules. The theory satisfies the requirements of unitarity and causality while providing a natural bridge between quantum mechanics and relativity according to the time-mass duality principle. The quantization yields specific experimental predictions, including quantum corrections to wavelength-dependent redshift and modified gravitational wave propagation. This consistent quantum theory of the intrinsic time field addresses open questions in foundational physics and establishes the T0 model as a promising alternative to conventional approaches to quantum gravity and unified theories.
	\end{abstract}
	
	\tableofcontents
	\newpage
	
	\section{Quantum Field Theoretical Treatment of the T0 Model}
	\label{sec:qft_treatment}
	
	\subsection{Foundations of the T0 Model}
	\label{subsec:foundations}
	
	The T0 model provides a novel approach to fundamental physics based on the intrinsic time field $T(x)$ and time-mass duality. In the T0 framework, the intrinsic time field is defined as:
	\begin{equation}
		T(x) = \frac{\hbar}{\max(mc^2, \omega)}
		\label{eq:intrinsic_time}
	\end{equation}
	
	In natural units ($\hbar = c = G = k_B = 1$), this simplifies to:
	\begin{equation}
		T(x) = \frac{1}{m}
		\label{eq:natural_units_time}
	\end{equation}
	
	This relationship establishes energy as the fundamental unit, with $T(x)$ having dimension $[E^{-1}]$. The complete T0 model is built on unified natural units with $\alpha_{\text{EM}} = \beta_T = \alpha_W = 1$, as developed in \cite{Pascher2025Alpha1Beta1}.
	
	\subsection{Field Equations and Lagrangian Density}
	\label{subsec:field_equations}
	
	The field equation for the intrinsic time field is:
	\begin{equation}
		\nabla^2 T(x) = -\kappa\rho(x)T(x)^2
		\label{eq:field_equation}
	\end{equation}
	
	where $\kappa$ has dimension $[E]$ and $\rho$ has dimension $[E^2]$ in natural units. As demonstrated in \cite{Pascher2025EmergentGrav}, this equation leads to emergent gravitation.
	
	The total Lagrangian density for the T0 model is:
	\begin{equation}
		\mathcal{L}_{\text{Total}} = \mathcal{L}_{\text{Boson}} + \mathcal{L}_{\text{Fermion}} + \mathcal{L}_{\text{Higgs-T}} + \mathcal{L}_{\text{intrinsic}}
		\label{eq:total_lagrangian}
	\end{equation}
	
	where:
	\begin{align}
		\mathcal{L}_{\text{intrinsic}} &= \frac{1}{2}\partial_{\mu}T(x)\partial^{\mu}T(x) - \frac{1}{2}T(x)^2 - \frac{\rho}{T(x)} \label{eq:intrinsic_lagrangian} \\
		\mathcal{L}_{\text{Boson}} &= -\frac{1}{4}F_{\mu\nu}F^{\mu\nu} \label{eq:boson_lagrangian} \\
		\mathcal{L}_{\text{Fermion}} &= \bar{\psi}i\gamma^{\mu}D_{T\mu}\psi \label{eq:fermion_lagrangian} \\
		\mathcal{L}_{\text{Higgs-T}} &= |D_{T\mu}\Phi|^2 - V(\Phi) \label{eq:higgs_lagrangian}
	\end{align}
	
	with the T-modified derivatives:
	\begin{align}
		D_{T\mu}\psi &= T(x)D_{\mu}\psi + \psi\partial_{\mu}T(x) \label{eq:t_modified_derivative} \\
		D_{T\mu}\Phi &= T(x)(\partial_{\mu} + igA_{\mu})\Phi + \Phi\partial_{\mu}T(x) \label{eq:higgs_t_derivative}
	\end{align}
	
	These expressions maintain complete consistency with the field equations derived in \cite{Pascher2025Lagrange} and \cite{Pascher2025Higgs}.
	
	\subsection{Canonical Quantization}
	\label{subsec:canonical_quantization}
	
	To quantize the intrinsic time field, we separate it into a classical background and quantum fluctuations:
	\begin{equation}
		T(x) = T_c(x) + \hat{T}(x)
		\label{eq:quantum_decomposition}
	\end{equation}
	
	The classical part satisfies:
	\begin{equation}
		\nabla^2 T_c(x) = -\kappa\rho(x)T_c(x)^2
		\label{eq:classical_field_equation}
	\end{equation}
	
	For the quantum fluctuations, the effective Lagrangian becomes:
	\begin{equation}
		\mathcal{L}_{\text{quantum}} \approx \frac{1}{2}\partial_{\mu}\hat{T}(x)\partial^{\mu}\hat{T}(x) - \frac{1}{2}\hat{T}(x)^2 - \frac{\rho}{T_c(x)^3}\hat{T}(x)^2
		\label{eq:quantum_lagrangian}
	\end{equation}
	
	The canonical momentum is:
	\begin{equation}
		\Pi(x) = \frac{\partial\mathcal{L}}{\partial(\partial_0 \hat{T})} = \partial_0 \hat{T}(x)
		\label{eq:canonical_momentum}
	\end{equation}
	
	We impose canonical commutation relations:
	\begin{equation}
		[\hat{T}(\vec{x}, t), \Pi(\vec{y}, t)] = i\delta^3(\vec{x} - \vec{y})
		\label{eq:commutation_relation}
	\end{equation}
	\begin{equation}
		[\hat{T}(\vec{x}, t), \hat{T}(\vec{y}, t)] = [\Pi(\vec{x}, t), \Pi(\vec{y}, t)] = 0
		\label{eq:field_commutators}
	\end{equation}
	
	These relations follow directly from the principles of quantum field theory applied to the intrinsic time field, without importing Standard Model assumptions.
	
	\subsection{Mode Expansion and Hamiltonian}
	\label{subsec:mode_expansion}
	
	The effective mass for quantum fluctuations is:
	\begin{equation}
		m_{\text{eff}}^2(x) = 1 + \frac{2\rho}{T_c(x)^3}
		\label{eq:effective_mass}
	\end{equation}
	
	In regions of approximately uniform mass density, we can express $\hat{T}(x)$ using mode expansion:
	\begin{equation}
		\hat{T}(x) = \int \frac{d^3k}{(2\pi)^3} \frac{1}{\sqrt{2\omega_{\vec{k}}}} \left(a_{\vec{k}} e^{-ik \cdot x} + a_{\vec{k}}^{\dagger} e^{ik \cdot x}\right)
		\label{eq:mode_expansion}
	\end{equation}
	
	where $\omega_{\vec{k}} = \sqrt{\vec{k}^2 + m_{\text{eff}}^2}$ and:
	\begin{equation}
		[a_{\vec{k}}, a_{\vec{k'}}^{\dagger}] = (2\pi)^3 \delta^3(\vec{k} - \vec{k'})
		\label{eq:creation_annihilation}
	\end{equation}
	
	The Hamiltonian density is:
	\begin{equation}
		\mathcal{H} = \frac{1}{2}\Pi(x)^2 + \frac{1}{2}(\nabla \hat{T}(x))^2 + \frac{1}{2}\left(1 + \frac{2\rho}{T_c(x)^3}\right)\hat{T}(x)^2
		\label{eq:hamiltonian_density}
	\end{equation}
	
	After normal ordering, the Hamiltonian becomes:
	\begin{equation}
		H = \int \frac{d^3k}{(2\pi)^3} \omega_{\vec{k}} :a_{\vec{k}}^{\dagger}a_{\vec{k}}: + E_0
		\label{eq:hamiltonian}
	\end{equation}
	
	The vacuum energy $E_0$ depends on the matter distribution through $m_{\text{eff}}$, providing a natural mechanism for vacuum energy to adjust to the presence of matter. This is fundamentally different from the Standard Model approach, which leads to the cosmological constant problem. In the T0 model, this issue is naturally addressed through the coupling between the time field and matter distribution, as detailed in \cite{Pascher2025Energy}.
	
	\subsection{Path Integral Formulation}
	\label{subsec:path_integral}
	
	The generating functional for the quantum time field is:
	\begin{equation}
		Z[J] = \int \mathcal{D}T \exp\left(i\int d^4x (\mathcal{L}_{\text{quantum}} + J(x)\hat{T}(x))\right)
		\label{eq:generating_functional}
	\end{equation}
	
	which can be evaluated as:
	\begin{equation}
		Z[J] = \exp\left(-\frac{i}{2}\int d^4x d^4y J(x)\Delta_F(x-y)J(y)\right)
		\label{eq:evaluated_functional}
	\end{equation}
	
	where $\Delta_F(x-y)$ is the Feynman propagator:
	\begin{equation}
		\Delta_F(x-y) = \int \frac{d^4k}{(2\pi)^4} \frac{i}{k^2 - m_{\text{eff}}^2 + i\epsilon} e^{-ik \cdot (x-y)}
		\label{eq:feynman_propagator}
	\end{equation}
	
	This propagator includes the effective mass $m_{\text{eff}}$ which contains the matter coupling, a feature unique to the T0 model not found in Standard Model approaches.
	
	\subsection{Modified Particle Propagators}
	\label{subsec:modified_propagators}
	
	The time field modifies the propagators of Standard Model particles. For fermions:
	\begin{equation}
		S_F^T(p) = \frac{i(\slash{p} + m)}{p^2 - m^2 + i\epsilon} \cdot \frac{T_c}{T_0}
		\label{eq:fermion_propagator}
	\end{equation}
	
	For gauge bosons:
	\begin{equation}
		D_{\mu\nu}^T(p) = \frac{-ig_{\mu\nu} + \frac{p_{\mu}p_{\nu}}{p^2}}{p^2 + i\epsilon} \cdot \left(\frac{T_c}{T_0}\right)^2
		\label{eq:gauge_propagator}
	\end{equation}
	
	These modifications follow directly from the coupling structure in the T0 model Lagrangian, without introducing additional parameters, consistent with the framework in \cite{Pascher2025Lagrange}.
	
	\subsection{Feynman Rules}
	\label{subsec:feynman_rules}
	
	The T0 model Feynman rules include:
	
	\begin{enumerate}
		\item $T(x)$-propagator: $\frac{i}{p^2 - m_{\text{eff}}^2 + i\epsilon}$
		\item Fermion-$T(x)$ vertex: $i\gamma^{\mu}p_{\mu}$
		\item Gauge boson-$T(x)$ vertex: $-2ig_{\mu\nu}T_c$
		\item Higgs-$T(x)$ vertex: $ip_{\mu}\Phi^*\partial^{\mu}\Phi + \text{h.c.}$
	\end{enumerate}
	
	These rules preserve the dimensional consistency with energy as the fundamental unit and follow directly from the T0 model Lagrangian densities derived in \cite{Pascher2025Fields}.
	
	\subsection{Quantum Corrections to Wavelength-Dependent Redshift}
	\label{subsec:quantum_redshift}
	
	The classical wavelength-dependent redshift in the T0 model is:
	\begin{equation}
		z(\lambda) = z_0\left(1 + \ln\frac{\lambda}{\lambda_0}\right)
		\label{eq:classical_redshift}
	\end{equation}
	
	Quantum fluctuations of the time field introduce corrections:
	\begin{equation}
		z(\lambda) = z_0\left(1 + \ln\frac{\lambda}{\lambda_0} + \frac{\langle \hat{T}(x)^2 \rangle}{T_c(x)^2}\right)
		\label{eq:quantum_redshift}
	\end{equation}
	
	The quantum correction term $\frac{\langle \hat{T}(x)^2 \rangle}{T_c(x)^2}$ is proportional to $\frac{1}{E_{\text{Pl}}}$, making it small for typical astronomical observations but potentially detectable in high-precision measurements, as discussed in \cite{Pascher2025Measurements}.
	
	\subsection{Emergent Spacetime}
	\label{subsec:emergent_spacetime}
	
	The quantum time field gives rise to an emergent spacetime metric:
	\begin{equation}
		g_{\mu\nu}^{\text{eff}} = \eta_{\mu\nu} + 2\kappa\langle T(x) \rangle \partial_{\mu}\partial_{\nu}\langle T(x) \rangle - \kappa\eta_{\mu\nu}\partial_{\alpha}\langle T(x) \rangle \partial^{\alpha}\langle T(x) \rangle
		\label{eq:emergent_metric}
	\end{equation}
	
	This connects directly to the gravitational parameter $\kappa^{\text{nat}} = \betaT^{\text{nat}} \cdot \frac{yv}{r_g^2}\betaT^{\text{nat}} \cdot \frac{yv}{r_g^2}$ with $\beta_T = 1$ in natural units. The metric emerges naturally from the time field dynamics without assuming general relativity, as demonstrated in \cite{Pascher2025EmergentGrav}.
	
	Gravitational waves arise as oscillations in the expectation value of the time field, with propagation speed:
	\begin{equation}
		v_{\text{GW}} = c\left(1 - \frac{1}{2}\frac{\langle \hat{T}(x)^2 \rangle}{T_c(x)^2}\right)
		\label{eq:gw_speed}
	\end{equation}
	
	showing a small quantum correction that could potentially be measured in future gravitational wave observations.
	
	\subsection{Modified Uncertainty Relations}
	\label{subsec:uncertainty}
	
	The time field leads to modified uncertainty relations:
	\begin{equation}
		\Delta x \Delta p \geq \frac{\hbar}{2}\left(1 + \langle \hat{T}(x) \rangle \Delta V\right)
		\label{eq:uncertainty}
	\end{equation}
	
	This provides a natural bridge between quantum and classical regimes, with the standard uncertainty principle recovered in the limit of weak gravitational fields, as explored in \cite{Pascher2025Extensions}.
	
	\section{Experimental Predictions}
	\label{sec:predictions}
	
	The quantum field theoretical treatment of the T0 model leads to several unique experimental predictions that distinguish it from the Standard Model:
	
	\subsection{Wavelength-Dependent Redshift}
	\label{subsec:redshift_prediction}
	
	The T0 model predicts a specific wavelength dependence of cosmic redshift:
	\begin{equation}
		z(\lambda) = z_0\left(1 + \ln\frac{\lambda}{\lambda_0}\right)
		\label{eq:redshift_prediction}
	\end{equation}
	
	This can be tested through multi-wavelength observations of distant galaxies using instruments like the James Webb Space Telescope, as detailed in \cite{Pascher2025Measurements}.
	
	\subsection{Modified Gravitational Potential}
	\label{subsec:potential_prediction}
	
	The gravitational potential in the T0 model takes the form:
	\begin{equation}
		\Phi(r) = -\frac{M}{r} + \kappa r
		\label{eq:grav_potential}
	\end{equation}
	
	where $\kappa^{\text{nat}} = \betaT^{\text{nat}} \cdot \frac{yv}{r_g^2}\betaT^{\text{nat}} \cdot \frac{yv}{r_g^2}$ has dimension $[E]$ in natural units. This modified potential explains galaxy rotation curves without dark matter, as shown in \cite{Pascher2025Galaxies}.
	
	\subsection{Quantum Gravitational Effects}
	\label{subsec:quantum_gravity}
	
	The T0 model predicts quantum gravitational effects at energies approximately:
	\begin{equation}
		E_{\text{QG}} \sim \sqrt{\xi} \cdot M_{\text{Pl}} \approx 10^{-2} M_{\text{Pl}}
		\label{eq:quantum_gravity_scale}
	\end{equation}
	
	where $\xi \approx 1.33 \times 10^{-4}$ relates the characteristic length scale $r_0$ to the Planck length: $r_0 = \xi \cdot l_P$. This places quantum gravitational effects potentially within reach of future experiments, as discussed in \cite{Pascher2025Planck}.
	
	\section{Conclusion}
	\label{sec:conclusion}
	
	The quantum field theoretical treatment of the T0 model provides a consistent framework that:
	
	\begin{enumerate}
		\item Maintains energy as the fundamental unit throughout
		\item Requires no new independent constants beyond those specified in the T0 model
		\item Provides a natural explanation for redshift without cosmic expansion
		\item Offers an elegant solution to the vacuum energy problem
		\item Makes specific, testable predictions that distinguish it from the Standard Model
	\end{enumerate}
	
	This development completes the theoretical structure of the T0 model, establishing it as a viable alternative to conventional approaches to quantum gravity and unified theories.
	
	\begin{thebibliography}{99}
		\bibitem{Pascher2025Alpha1Beta1} Pascher, J. (2025). \href{https://github.com/jpascher/T0-Time-Mass-Duality/tree/main/2/pdf/English/Alpha1Beta1KonsistenzEn.pdf}{Unified Unit System in the T0 Model: The Consistency of $\alpha = 1$ and $\beta = 1$}.
		
		\bibitem{Pascher2025EmergentGrav} Pascher, J. (2025). \href{https://github.com/jpascher/T0-Time-Mass-Duality/tree/main/2/pdf/English/EmergentGravT0En.pdf}{Emergent Gravitation in the T0 Model: A Comprehensive Derivation}.
		
		\bibitem{Pascher2025Lagrange} Pascher, J. (2025). \href{https://github.com/jpascher/T0-Time-Mass-Duality/tree/main/2/pdf/English/MathZeitMasseLagrangeEn.pdf}{From Time Dilation to Mass Variation: Mathematical Core Formulations of Time-Mass Duality Theory}.
		
		\bibitem{Pascher2025Higgs} Pascher, J. (2025). \href{https://github.com/jpascher/T0-Time-Mass-Duality/tree/main/2/pdf/English/MathHiggsZeitMasseEn.pdf}{Mathematical Formulation of the Higgs Mechanism in Time-Mass Duality}.
		
		\bibitem{Pascher2025Energy} Pascher, J. (2025). \href{https://github.com/jpascher/T0-Time-Mass-Duality/tree/main/2/pdf/English/MathEnergiedynamikEn.pdf}{Dark Energy in the T0 Model: A Mathematical Analysis of Energy Dynamics}.
		
		\bibitem{Pascher2025Fields} Pascher, J. (2025). \href{https://github.com/jpascher/T0-Time-Mass-Duality/tree/main/2/pdf/English/FeldtheorieQuantenEn.pdf}{Field Theory and Quantum Correlations: A New Perspective on Instantaneity}.
		
		\bibitem{Pascher2025Measurements} Pascher, J. (2025). \href{https://github.com/jpascher/T0-Time-Mass-Duality/tree/main/2/pdf/English/MessdifferenzenT0StandardEn.pdf}{Compensatory and Additive Effects: An Analysis of Measurement Differences Between the T0 Model and the $\Lambda$CDM Standard Model}.
		
		\bibitem{Pascher2025Galaxies} Pascher, J. (2025). \href{https://github.com/jpascher/T0-Time-Mass-Duality/tree/main/2/pdf/English/MassVarGalaxienEn.pdf}{Mass Variation in Galaxies: An Analysis in the T0 Model with Emergent Gravitation}.
		
		\bibitem{Pascher2025Extensions} Pascher, J. (2025). \href{https://github.com/jpascher/T0-Time-Mass-Duality/tree/main/2/pdf/English/NotwendigkeitQMErweiterungEn.pdf}{The Necessity of Extending Standard Quantum Mechanics and Quantum Field Theory}.
		
		\bibitem{Pascher2025Planck} Pascher, J. (2025). \href{https://github.com/jpascher/T0-Time-Mass-Duality/tree/main/2/pdf/English/JenseitsPlanckEn.pdf}{Real Consequences of Reformulating Time and Mass in Physics: Beyond the Planck Scale}.
		
		\bibitem{Pascher2025Time} Pascher, J. (2025). \href{https://github.com/jpascher/T0-Time-Mass-Duality/tree/main/2/pdf/English/ZeitEmergentQMEn.pdf}{Time as an Emergent Property in Quantum Mechanics: A Connection Between Relativity, Fine-Structure Constant, and Quantum Dynamics}.
		
		\bibitem{Pascher2025TimeMass} Pascher, J. (2025). \href{https://github.com/jpascher/T0-Time-Mass-Duality/tree/main/2/pdf/English/ZeitMasseNeuerBlickEn.pdf}{Time and Mass: A New Look at Old Formulas – and Liberation from Traditional Constraints}.
		
		\bibitem{Pascher2025Parameters} Pascher, J. (2025). \href{https://github.com/jpascher/T0-Time-Mass-Duality/tree/main/2/pdf/English/ZeitMasseT0ParamsEn.pdf}{Time-Mass Duality Theory (T0 Model): Derivation of Parameters $\kappa$, $\alpha$, and $\beta$}.
		
		\bibitem{Pascher2025Photons} Pascher, J. (2025). \href{https://github.com/jpascher/T0-Time-Mass-Duality/tree/main/2/pdf/English/DynMassePhotonenNichtlokalEn.pdf}{Dynamic Mass of Photons and Its Implications for Nonlocality in the T0 Model}.
		
		\bibitem{Pascher2025Forces} Pascher, J. (2025). \href{https://github.com/jpascher/T0-Time-Mass-Duality/tree/main/2/pdf/English/VierKraefteZeitMasseEn.pdf}{Simplified Description of Fundamental Forces with Time-Mass Duality}.
	\end{thebibliography}
	
\end{document}