\documentclass[12pt,a4paper]{article}
\usepackage[utf8]{inputenc}
\usepackage[T1]{fontenc}
\usepackage[ngerman]{babel}
\usepackage{lmodern}
\usepackage{amsmath}
\usepackage{amssymb}
\usepackage{physics}
\usepackage{hyperref}
\usepackage{tcolorbox}
\usepackage{booktabs}
\usepackage{enumitem}
\usepackage[table,xcdraw]{xcolor}
\usepackage[left=2cm,right=2cm,top=2cm,bottom=2cm]{geometry}
\usepackage{pgfplots}
\pgfplotsset{compat=1.18}
\usepackage{graphicx}
\usepackage{float}
\usepackage{fancyhdr}
\usepackage{siunitx}

% Acknowledgments environment
\newenvironment{acknowledgments}
{\section*{Acknowledgments}}
{\vspace{1em}}

% Custom commands
\newcommand{\Tfield}{T(x)}
\newcommand{\alphaEM}{\alpha_{\text{EM}}}
\newcommand{\alphaW}{\alpha_{\text{W}}}
\newcommand{\betaT}{\beta_{\text{T}}}
\newcommand{\Mpl}{M_{\text{Pl}}}
\newcommand{\Tzerot}{T_0(\Tfield)}
\newcommand{\Tzero}{T_0}
\newcommand{\vecx}{\vec{x}}
\newcommand{\vr}{\vec{r}}
\newcommand{\gammaf}{\gamma_{\text{Lorentz}}}
\newcommand{\DhiggsT}{\Tfield (\partial_\mu + ig A_\mu) \Phi + \Phi \partial_\mu \Tfield}
\newcommand{\LCDM}{\Lambda\text{CDM}}
\newcommand{\DTmu}{D_{T,\mu}}
\newcommand{\calL}{\mathcal{L}}
\newcommand{\deq}{\displaystyle}
\newcommand{\e}{\mathrm{e}}

% Header and Footer Configuration
\pagestyle{fancy}
\fancyhf{}
\fancyhead[L]{Johann Pascher}
\fancyhead[R]{Time-Mass Duality: Part II}
\fancyfoot[C]{\thepage}
\renewcommand{\headrulewidth}{0.4pt}
\renewcommand{\footrulewidth}{0.4pt}

\hypersetup{
	colorlinks=true,
	linkcolor=blue,
	citecolor=blue,
	urlcolor=blue,
	pdftitle={Bridging Quantum Mechanics and Relativity through Time-Mass Duality: Part II},
	pdfauthor={Johann Pascher},
	pdfsubject={Theoretical Physics},
	pdfkeywords={T0 Model, natural units, time-mass duality, cosmology}
}

\title{Bridging Quantum Mechanics and Relativity through Time-Mass Duality: \\ A Unified Framework with Natural Units \(\alpha = \beta = 1\) \\ Part II: Cosmological Implications and Experimental Validation}
\author{Johann Pascher\\
	Department of Communications Engineering\\
	Höhere Technische Bundeslehranstalt (HTL), Leonding, Austria\\
	\texttt{johann.pascher@gmail.com}}
\date{April 7, 2025}

\begin{document}
	
	\maketitle
	
	\begin{abstract}
		This paper extends the T0 model introduced in Part I into the realms of cosmology and experimental validation, building on a unified natural unit system where \(\hbar = c = G = k_B = \alphaEM = \alphaW = \betaT = 1\). In contrast to the expanding universe of the \(\Lambda\)CDM model, we propose a static cosmos where redshift arises from photon energy loss mediated by the intrinsic time field \(\Tfield\). This framework reinterprets dark matter and dark energy through emergent gravitational effects, enhancing the Standard Model (SM) with a consistent gravitational theory. Key predictions include a wavelength-dependent redshift with a variation of approximately \(2.3\%\) per decade, a cosmic microwave background (CMB) temperature of \(24000 \, \text{K}\) at \(z = 1100\), and speculative extensions beyond the speed of light. These predictions are testable using instruments like the James Webb Space Telescope (JWST) and future CMB missions. We address measurement challenges, such as the frequency-dependent biases in GPS precision and cosmological observations, which obscure distinctions between mass variation and time dilation, offering a philosophically coherent alternative to \(\Lambda\)CDM that aligns theoretical elegance with empirical rigor.
	\end{abstract}
	
	\section{Introduction}
	\label{sec:introduction}
	
	In Part I (\textit{Bridging Quantum Mechanics and Relativity through Time-Mass Duality: Part I}, \cite{pascher_part1_2025}), we established the T0 model as a unified framework for quantum mechanics (QM) and relativity theory (RT), leveraging the intrinsic time field \(\Tfield = \frac{\hbar}{\max(mc^2, \omega)}\) within a natural unit system (\(\hbar = c = G = k_B = \alphaEM = \alphaW = \betaT = 1\)). This system, detailed in Part I, Section 2 "Unification of Constants with Natural Units" \href{https://github.com/jpascher/T0-Time-Mass-Duality/tree/main/2/pdf/English/Bridging Quantum Mechanics and Relativity through Time-Mass Duality Part I Theoretical Foundations_en.pdf}{[Teil I]}, eliminates empirically determined constants, achieving consistency with measured values (e.g., \(c \approx 3 \times 10^8 \, \text{m/s}\), \(\alphaEM \approx 1/137.036\)) with deviations below \(10^{-6}\). It enabled a mass-dependent Schrödinger equation (Part I, Equation (4.5) \href{https://github.com/jpascher/T0-Time-Mass-Duality/tree/main/2/pdf/English/Bridging Quantum Mechanics and Relativity through Time-Mass Duality Part I Theoretical Foundations_en.pdf}{[Teil I]}) and emergent gravitation (Part I, Section 5 "Emergent Gravitation" \href{https://github.com/jpascher/T0-Time-Mass-Duality/tree/main/2/pdf/English/Bridging Quantum Mechanics and Relativity through Time-Mass Duality Part I Theoretical Foundations_en.pdf}{[Teil I]}), bridging micro- and macroscopic scales.
	
	Part II extends these foundations into cosmology and experimental validation, contrasting with the \(\Lambda\)CDM model’s expanding universe, which originates from a Big Bang approximately 13.8 billion years ago \cite{Planck2020}. In \(\Lambda\)CDM, cosmic redshift is a kinematic effect (\(z \approx H_0 d / c\)), requiring inflation and dark energy \cite{Riess1998,Perlmutter1999}. The T0 model proposes a static, infinite, and eternal universe where redshift stems from photon energy loss via \(\Tfield\), enhancing the Standard Model (SM) with a consistent gravitational theory while retaining its particle physics core.
	
	Key predictions include:
	- Wavelength-dependent redshift (\(\sim 2.3\%\) per decade),
	- CMB temperature of \(24000 \, \text{K}\) at \(z = 1100\),
	- Speculative superluminal extensions.
	
	These are testable with JWST spectroscopy and CMB distortion measurements, though frequency-based methods (e.g., GPS, redshift) conflate mass variation and time dilation, necessitating careful reassessment \cite{pascher_quantum_2025}. Philosophically, T0 avoids singularities, offering a coherent eternal cosmos \cite{pascher_perspective_2025}.
	
	This paper is structured as:
	- Section 2: Static universe and redshift mechanism.
	- Section 3: Cosmological phenomena and predictions.
	- Section 4: Quantitative predictions.
	- Section 5: Experimental tests and measurement challenges.
	- Section 6: Implications of \(\betaT = 1\).
	- Section 7: Integration with T0 principles.
	- Section 8: Speculative extensions and philosophy.
	
	\section{Static Universe Model}
	\label{sec:static_universe}
	
	\subsection{Concept of a Static Universe}
	\label{subsec:static_concept}
	
	The T0 model envisions a static universe, infinite in space and eternal in time, contrasting with \(\Lambda\)CDM’s expanding cosmos from a Big Bang. In \(\Lambda\)CDM, redshift (\(z \approx H_0 d / c\)) reflects expansion (\(H_0 \approx 70 \, \text{km/s/Mpc}\)) \cite{Planck2020}, requiring inflation for uniformity and dark energy for acceleration (\(\Omega_{\Lambda} \approx 0.7\)) \cite{Riess1998}. T0 eliminates these, positing a stable cosmos where \(\Tfield\) governs dynamics without expansion.
	
	Advantages include:
	- **Horizon Problem:** Infinite time ensures thermal equilibrium across scales \cite{pascher_messdifferenzen_2025}.
	- **Flatness:** No expansion eliminates curvature tuning.
	- **Singularity-Free:** Eternal existence avoids infinite densities \cite{pascher_perspective_2025}.
	
	This complements SM particle physics with a static gravitational framework derived in Part I, Section 5 "Emergent Gravitation" \href{https://github.com/jpascher/T0-Time-Mass-Duality/tree/main/2/pdf/English/Bridging Quantum Mechanics and Relativity through Time-Mass Duality Part I Theoretical Foundations_en.pdf}{[Teil I]}.
	
	\subsection{Redshift through Energy Loss}
	\label{subsec:redshift_energy_loss}
	
	Redshift in T0 is:
	\begin{equation}
		1 + z = e^{\alpha d},
		\label{eq:redshift_distance}
	\end{equation}
	with \(\alpha = H_0 / c \approx 2.3 \times 10^{-18} \, \text{m}^{-1}\) (SI) or 1 (natural units). At low \(z\):
	\begin{equation}
		z \approx \alpha d,
		\label{eq:hubble_approx}
	\end{equation}
	matching \(\Lambda\)CDM locally. The mechanism is photon energy loss:
	\begin{equation}
		\frac{dE}{dx} = -\alpha E,
		\label{eq:energy_loss_rate}
	\end{equation}
	yielding \(E = E_0 e^{-\alpha d}\), and thus \(1 + z = e^{\alpha d}\), as derived from \(\Tfield\) properties in Part I, Section 3.1 "Definition and Physical Basis" \href{https://github.com/jpascher/T0-Time-Mass-Duality/tree/main/2/pdf/English/Bridging Quantum Mechanics and Relativity through Time-Mass Duality Part I Theoretical Foundations_en.pdf}{[Teil I]} \cite{pascher_messdifferenzen_2025}.
	
	\section{Cosmological Phenomena}
	\label{sec:cosmological_phenomena}
	
	\subsection{Temperature-Redshift Relation and CMB}
	\label{subsec:cmb_temp}
	
	\(\Lambda\)CDM’s \(T(z) = T_0 (1 + z)\) gives \(T \approx 3000 \, \text{K}\) at \(z = 1100\) (\(T_0 = 2.725 \, \text{K}\)) \cite{Fixsen2009}. T0 predicts:
	\begin{equation}
		T(z) = T_0 (1 + z) (1 + \ln(1 + z)),
		\label{eq:temperature_redshift_simplified}
	\end{equation}
	so \(T(1100) \approx 24000 \, \text{K}\), reflecting enhanced energy loss (Equation \ref{eq:energy_loss_rate}) \cite{pascher_temp_2025}. This impacts nucleosynthesis and recombination, testable via CMB distortions (Section 5.2).
	
	\subsection{Wavelength-Dependent Redshift}
	\label{subsec:wavelength_redshift}
	
	T0 predicts:
	\begin{equation}
		z(\lambda) = z_0 \left(1 + \ln\left(\frac{\lambda}{\lambda_0}\right)\right),
		\label{eq:wavelength_redshift}
	\end{equation}
	with \(\Delta z / z_0 \approx 3.85\%\) over 0.6-28 \(\mu\text{m}\) (JWST range), due to:
	\begin{equation}
		\frac{dE}{dx} = -\alpha E \left(1 + \ln\left(\frac{\lambda}{\lambda_0}\right)\right),
		\label{eq:wavelength_energy_loss}
	\end{equation}
	contrasting \(\Lambda\)CDM’s uniformity \cite{pascher_params_2025}.
	
	\subsection{Dark Matter and Dark Energy Reinterpretation}
	\label{subsec:dark_reinterpretation}
	
	The potential:
	\begin{equation}
		\Phi(r) = -\frac{M}{r} + \kappa r,
		\label{eq:grav_potential_t0}
	\end{equation}
	(\(\kappa \approx 4.8 \times 10^{-11} \, \text{m/s}^2\)) reinterprets:
	- **Dark Matter:** \(v(r) = \sqrt{\frac{M}{r} + \kappa r}\), as derived in Part I, Section 5.1 "Derivation from \(\Tfield\)" \href{https://github.com/jpascher/T0-Time-Mass-Duality/tree/main/2/pdf/English/Bridging Quantum Mechanics and Relativity through Time-Mass Duality Part I Theoretical Foundations_en.pdf}{[Teil I]}.
	- **Dark Energy:** \(\rho_{\text{DE}} \approx \frac{\kappa}{r^2}\) \cite{pascher_galaxies_2025}.
	
	\subsection{Influence on Galaxy Dynamics}
	\label{subsec:galaxy_dynamics}
	
	The T0 model shapes galaxy dynamics through \(\Tfield\), offering an alternative to \(\Lambda\)CDM by reinterpreting gravitational effects without dark matter or expansion.
	
	\subsubsection{Rotation Curves}
	The potential (Equation \ref{eq:grav_potential_t0}) yields:
	\begin{equation}
		v(r) = \sqrt{\frac{M}{r} + \kappa r},
		\label{eq:rotation_velocity}
	\end{equation}
	reproducing flat rotation curves (e.g., Milky Way: \(v(30 \, \text{kpc}) \approx 211 \, \text{km/s}\)) \cite{pascher_galaxies_2025}.
	
	\begin{figure}[h]
		\centering
		\begin{tikzpicture}
			\begin{axis}[
				xlabel={Radius [kpc]},
				ylabel={Rotation Velocity [km/s]},
				xlabel style={font=\large},
				ylabel style={font=\large},
				tick label style={font=\normalsize},
				xmin=0, xmax=30,
				ymin=0, ymax=300,
				legend pos=south east,
				legend style={font=\large},
				grid=both,
				minor tick num=4,
				major grid style={line width=0.8pt, gray!50},
				minor grid style={line width=0.4pt, gray!20}
				]
				\addplot[blue, ultra thick, domain=0.1:30, samples=100] {220*sqrt(10/x)};
				\addplot[red, dashed, ultra thick, domain=0.1:30, samples=100] {sqrt(220^2*10/x + 4.8*x^2)};
				\legend{Newtonian Prediction, T0 Model}
			\end{axis}
		\end{tikzpicture}
		\caption{Rotation curves comparing Newtonian (blue) and T0 model (red) predictions for a galaxy with \(M = 10^{11} M_{\odot}\), \(\kappa_{\text{SI}} = 4.8 \times 10^{-11} \, \text{m/s}^2\). The T0 model produces a flat profile at large radii, consistent with observations such as those of the Milky Way.}
		\label{fig:rotation_curves}
	\end{figure}
	
	\subsubsection{Galaxy Formation and Evolution}
	In \(\Lambda\)CDM, galaxy formation relies on gravitational collapse of dark matter halos seeded by primordial fluctuations, amplified by cosmic expansion over a finite 13.8 billion years \cite{Planck2020}. The T0 model, with its static, infinite-time universe, proposes a different mechanism: galaxies form and evolve through gradual aggregation of baryonic matter under the influence of \(\Tfield\)-mediated gravitation, without the need for dark matter or a temporal origin. The infinite timescale allows for slow, steady processes such as gas cooling, star formation, and dynamical relaxation, driven solely by the observable mass and the emergent potential (Equation \ref{eq:grav_potential_t0}).
	
	This process enhances the SM’s baryonic dynamics by providing a gravitational framework that operates consistently across all scales. For instance, the \(\kappa r\) term introduces a long-range stabilizing force that prevents excessive dispersion of gas clouds, facilitating the formation of spiral arms and galactic disks over extended periods. This contrasts with \(\Lambda\)CDM’s reliance on dark matter to provide the necessary gravitational wells, which T0 replaces with a purely baryonic, \(\Tfield\)-driven mechanism \cite{pascher_galaxies_2025}.
	
	\subsubsection{Cluster Dynamics and Large-Scale Structure}
	On larger scales, such as galaxy clusters (e.g., the Bullet Cluster), \(\Lambda\)CDM invokes dark matter to explain mass discrepancies inferred from gravitational lensing and velocity dispersions \cite{McGaugh2016}. In T0, the \(\kappa r\) term modifies gravitational interactions, reducing these discrepancies:
	\begin{equation}
		v_{\text{cluster}}(r) = \sqrt{\frac{M_{\text{total}}}{r} + \kappa r},
		\label{eq:cluster_velocity}
	\end{equation}
	where \(M_{\text{total}}\) is the total baryonic mass. For a cluster like the Bullet Cluster (\(M_{\text{total}} \approx 10^{14} M_{\odot}\), \(r \approx 1 \, \text{Mpc}\)), the additional \(\kappa r\) term contributes significantly at large radii, aligning lensing and dynamical mass estimates without dark matter. Numerical simulations suggest deviations from \(\Lambda\)CDM predictions are within observational errors (\(\sim 5-10\%\)), testable with precise lensing surveys \cite{pascher_emergente_2025}.
	
	The large-scale structure in T0 emerges from the infinite-time evolution of baryonic matter under \(\Tfield\), forming filaments and walls naturally without expansion-driven growth. This aligns with the SM’s particle interactions, extended by T0’s gravitational theory (Part I, Section 5 "Emergent Gravitation" \href{https://github.com/jpascher/T0-Time-Mass-Duality/tree/main/2/pdf/English/Bridging Quantum Mechanics and Relativity through Time-Mass Duality Part I Theoretical Foundations_en.pdf}{[Teil I]}).
	
	\begin{table}[ht]
		\centering
		\caption{Comparison of \(\Lambda\)CDM and T0 Model Predictions for Galaxy Dynamics}
		\label{tab:galaxy_dynamics_comparison}
		\scalebox{0.8}{
			\begin{tabular}{lcc}
				\hline
				\textbf{Phenomenon} & \textbf{\(\Lambda\)CDM} & \textbf{T0 Model} \\
				\hline
				Rotation Curve & Dark matter halo (\(v \propto r^0\)) & \(\kappa r\) term (\(v \propto \sqrt{\kappa r}\)) \\
				Galaxy Formation & Dark matter collapse, 13.8 Gyr & Baryonic aggregation, infinite time \\
				Cluster Mass & Dark matter dominant & Baryonic + \(\Tfield\) effects \\
				Large-Scale Structure & Expansion + fluctuations & Static \(\Tfield\)-driven \\
				\hline
			\end{tabular}
		}
	\end{table}
	
	\section{Quantitative Predictions}
	\label{sec:predictions}
	
	\subsection{CMB Temperature Prediction}
	\label{subsec:cmb_temp_prediction}
	
	T0 predicts a CMB temperature at \(z = 1100\) of:
	\begin{equation}
		T(1100) \approx 24000 \, \text{K},
		\label{eq:cmb_temp_t0}
	\end{equation}
	versus \(\Lambda\)CDM’s \(3000 \, \text{K}\), a factor of 8 difference due to \(\Tfield\)’s logarithmic enhancement (Equation \ref{eq:temperature_redshift_simplified}).
	
	\subsection{Wavelength-Dependent Redshift Variation}
	\label{subsec:wavelength_redshift_prediction}
	
	Across JWST’s range (0.6-28 \(\mu\text{m}\)):
	\begin{equation}
		\Delta z / z_0 \approx 3.85\%,
		\label{eq:wavelength_variation}
	\end{equation}
	or \(2.3\%\) per decade, a direct test of \(\betaT = 1\) (Equation \ref{eq:wavelength_redshift}).
	
	\subsection{Galaxy Rotation Velocities}
	\label{subsec:rotation_velocity_prediction}
	
	For a galaxy like the Milky Way:
	\begin{equation}
		v(r) = \sqrt{\frac{M}{r} + \kappa r},
		\label{eq:rotation_velocity_repeat}
	\end{equation}
	e.g., \(v(30 \, \text{kpc}) \approx 211 \, \text{km/s}\), consistent with observed flat curves \cite{McGaugh2016}.
	
	\begin{figure}[ht]
		\centering
		\begin{tikzpicture}
			\begin{axis}[
				xlabel={Redshift \(z\)},
				ylabel={Distance Modulus \(\mu = m - M\)},
				xmin=0,
				xmax=2,
				ymin=30,
				ymax=50,
				legend pos=north west,
				grid=both,
				width=\textwidth,
				height=6cm,
				samples=100
				]
				\addplot[blue, thick, domain=0.01:2] {5*log10(3e8/70e3*ln(1+x)*(1+x)*0.1) + 25};
				\addplot[red, dashed, domain=0.01:2] {5*log10(3e8/70e3*(1+x)*(2-(1/(1+x)))*1) + 25};
				\legend{T0 Model, \(\Lambda\)CDM (\(\Omega_m=0.3\), \(\Omega_{\Lambda}=0.7\))}
			\end{axis}
		\end{tikzpicture}
		\caption{Distance modulus vs. redshift comparing T0 (blue) and \(\Lambda\)CDM (red) predictions, with \(H_0 = 70 \, \text{km/s/Mpc}\), illustrating distinct behaviors testable with supernovae data.}
		\label{fig:distance_modulus}
	\end{figure}
	
	\section{Experimental Tests}
	\label{sec:tests}
	
	\subsection{JWST Spectroscopy}
	\label{subsec:jwst_test}
	
	The predicted \(\Delta z / z \approx 3.85\%\) at \(z = 7\) (Equation \ref{eq:wavelength_variation}) is detectable with JWST’s \(0.1\%\) precision, testing \(\betaT\)-induced variations via quasar emission lines \cite{pascher_params_2025}. A sample of high-\(z\) quasars could confirm this effect, distinguishing T0 from \(\Lambda\)CDM’s uniform redshift.
	
	\subsection{CMB Distortions}
	\label{subsec:cmb_distortions_test}
	
	T0 predicts:
	\begin{equation}
		\mu \approx 1.4 \times 10^{-5}, \quad y \approx 1.6 \times 10^{-6},
		\label{eq:distortion_parameters}
	\end{equation}
	versus \(\Lambda\)CDM’s \(\mu \approx 2 \times 10^{-8}\), \(y \approx 4 \times 10^{-9}\), measurable with PIXIE’s sensitivity (\(\sim 10^{-8}\)) \cite{pascher_temp_2025}. This tests the hotter CMB hypothesis (Section 3.1).
	
	\subsection{Measurement Problem: GPS and Clock Precision}
	\label{subsec:gps_clock_problem}
	
	GPS clocks (cesium, \(9.19 \, \text{GHz}\)) show a relativistic shift of \(\Delta t \approx 38 \, \mu\text{s/day}\), traditionally attributed to time dilation in GR. T0 interprets this as mass variation (\(m = \gamma m_0\)), as both affect frequency (\(f = \frac{c}{\lambda}\)) identically. Current frequency-based methods cannot distinguish these, as they measure oscillation rates rather than absolute time, posing a challenge to differentiating T0 from GR locally \cite{pascher_quantum_2025}. Alternative approaches, such as direct mass variation tests (e.g., comparing particle decay rates in gravitational fields), could resolve this ambiguity, though such experiments are not yet feasible with current technology.
	
	\subsection{Measurement Problem: Cosmological Observations}
	\label{subsec:cosmological_measurement_problem}
	
	Cosmological redshift (\(z = \frac{\Delta \lambda}{\lambda_0}\)) is interpreted as a Doppler effect in \(\Lambda\)CDM, but T0 attributes it to energy loss (Equation \ref{eq:redshift_distance}). Frequency-based spectroscopy, reliant on photon wavelength shifts, cannot isolate \(\Tfield\) variations from expansion effects, as both manifest similarly in observed spectra. This methodological limitation biases interpretations toward \(\Lambda\)CDM, necessitating non-frequency-based metrics—such as radioactive decay rates over cosmic distances—which are currently unavailable but could provide a definitive test \cite{pascher_alphabeta_2025}. For instance, comparing decay rates of isotopes in distant supernovae could reveal mass variation independent of frequency shifts.
	
	\subsection{Reassessment of Measurements}
	\label{subsec:reassessment_measurements}
	
	The setting of \(\betaT = 1\) in natural units (Part I, Section 2.2 "Definition of the Unified Natural Unit System" \href{https://github.com/jpascher/T0-Time-Mass-Duality/tree/main/2/pdf/English/Bridging Quantum Mechanics and Relativity through Time-Mass Duality Part I Theoretical Foundations_en.pdf}{[Teil I]}) introduces apparent discrepancies with SI values (e.g., \(\betaT^{\text{SI}} \approx 0.008\)), reflecting biases in \(\Lambda\)CDM calibration methods that assume expansion. Reinterpreting cosmological data through T0’s static framework could resolve tensions, such as the Hubble parameter discrepancy (\(H_0\) ranging from 67 to 73 km/s/Mpc) \cite{DiValentino2021}, by attributing variations to energy loss rather than expansion rates. This reassessment requires reanalyzing existing datasets (e.g., supernovae, CMB) with T0’s predictions, a task facilitated by the model’s mathematical simplicity (Equations \ref{eq:redshift_distance}, \ref{eq:temperature_redshift_simplified}) \cite{pascher_alphabeta_2025}.
	
	\section{Consequences of Setting \(\beta = 1\)}
	\label{sec:consequences_beta}
	
	\subsection{Theoretical Elegance}
	\label{subsec:theoretical_elegance}
	
	Setting \(\betaT = 1\) in natural units (Part I, Section 2.2 "Definition of the Unified Natural Unit System" \href{https://github.com/jpascher/T0-Time-Mass-Duality/tree/main/2/pdf/English/Bridging Quantum Mechanics and Relativity through Time-Mass Duality Part I Theoretical Foundations_en.pdf}{[Teil I]}) enhances the T0 model’s theoretical coherence:
	- **Unified Time-Energy Relation:** The temperature-redshift relation (Equation \ref{eq:temperature_redshift_simplified}) integrates logarithmic energy loss seamlessly with linear redshift terms, reflecting a consistent physical principle across scales.
	- **Dimensionless Consistency:** All fundamental constants (\(\hbar, c, G, k_B, \alphaEM, \alphaW, \betaT\)) are unified at 1, eliminating arbitrary scaling factors and aligning with the energy-centric framework established in Part I, Section 2 "Unification of Constants with Natural Units" \href{https://github.com/jpascher/T0-Time-Mass-Duality/tree/main/2/pdf/English/Bridging Quantum Mechanics and Relativity through Time-Mass Duality Part I Theoretical Foundations_en.pdf}{[Teil I]}.
	- **Simplification of Interactions:** The coupling of \(\Tfield\) to matter and fields (Part I, Section 4.1 "Lagrangian Densities" \href{https://github.com/jpascher/T0-Time-Mass-Duality/tree/main/2/pdf/English/Bridging Quantum Mechanics and Relativity through Time-Mass Duality Part I Theoretical Foundations_en.pdf}{[Teil I]}) becomes parameter-free, enhancing the SM’s elegance by reducing complexity in gravitational and quantum interactions \cite{pascher_alphabeta_2025}.
	
	This elegance contrasts with \(\Lambda\)CDM’s reliance on multiple free parameters (e.g., \(\Omega_m, \Omega_{\Lambda}\)), offering a more unified theoretical structure.
	
	\subsection{Conversion to SI Units}
	\label{subsec:conversion_si}
	
	In natural units, \(\betaT = 1\), but conversion to SI units requires:
	\begin{equation}
		\betaT^{\text{SI}} = \betaT^{\text{nat}} \cdot \frac{\xi \cdot l_{P,\text{SI}}}{r_{0,\text{SI}}},
		\label{eq:beta_conversion}
	\end{equation}
	where \(\xi \approx 1.33 \times 10^{-4}\) (Part I, Section 2.3 "Length Scales and Corresponding Constants" \href{https://github.com/jpascher/T0-Time-Mass-Duality/tree/main/2/pdf/English/Bridging Quantum Mechanics and Relativity through Time-Mass Duality Part I Theoretical Foundations_en.pdf}{[Teil I]}), \(l_{P,\text{SI}} = 1.616 \times 10^{-35} \, \text{m}\), and \(r_{0,\text{SI}}\) is the characteristic T0 length scale, approximately \(10^{-39} \, \text{m}\) based on Higgs parameters \cite{pascher_alphabeta_2025}. This yields \(\betaT^{\text{SI}} \approx 0.008\), consistent with empirical estimates, bridging theoretical purity with experimental applicability, analogous to \(c = 1\) translating to \(3 \times 10^8 \, \text{m/s}\) in SI units. This conversion ensures T0’s predictions (e.g., Equation \ref{eq:wavelength_redshift}) align with observable data while maintaining its foundational elegance.
	
	\section{Integration into the Time-Mass Duality Theory}
	\label{sec:integration_t0}
	
	\subsection{Consistency with Basic Principles}
	\label{subsec:consistency_principles}
	
	The choice of \(\betaT = 1\) aligns seamlessly with the T0 model’s foundational axioms established in Part I, Section 3 "Intrinsic Time Field \(\Tfield\)" \href{https://github.com/jpascher/T0-Time-Mass-Duality/tree/main/2/pdf/English/Bridging Quantum Mechanics and Relativity through Time-Mass Duality Part I Theoretical Foundations_en.pdf}{[Teil I]}:
	- **Absolute Time:** \(\Tfield\) defines a universal, intrinsic timescale for all particles, independent of observer motion, contrasting with GR’s relative time (Part I, Section 3.3 "Physical Interpretation" \href{https://github.com/jpascher/T0-Time-Mass-Duality/tree/main/2/pdf/English/Bridging Quantum Mechanics and Relativity through Time-Mass Duality Part I Theoretical Foundations_en.pdf}{[Teil I]}).
	- **Mass Variation:** Mass is dynamically determined as \(m = \frac{\hbar}{\Tfield c^2}\), mediated by the Higgs mechanism (Part I, Section 4.1 "Lagrangian Densities" \href{https://github.com/jpascher/T0-Time-Mass-Duality/tree/main/2/pdf/English/Bridging Quantum Mechanics and Relativity through Time-Mass Duality Part I Theoretical Foundations_en.pdf}{[Teil I]}), providing a quantum-relativistic link consistent with SM particle physics.
	- **Emergent Gravitation:** Gravitational effects arise from \(\Tfield\) gradients (Part I, Section 5.1 "Derivation from \(\Tfield\)" \href{https://github.com/jpascher/T0-Time-Mass-Duality/tree/main/2/pdf/English/Bridging Quantum Mechanics and Relativity through Time-Mass Duality Part I Theoretical Foundations_en.pdf}{[Teil I]}), unifying QM and RT without spacetime curvature, as validated by galaxy dynamics (Section 3.4) \cite{pascher_lagrange_2025}.
	
	This integration extends the SM by replacing GR’s geometric gravitation with a field-theoretic approach, maintaining consistency across micro- and macroscopic phenomena, from quantum entanglement (Part I, Section 4.2 "Extension of Quantum Mechanics" \href{https://github.com/jpascher/T0-Time-Mass-Duality/tree/main/2/pdf/English/Bridging Quantum Mechanics and Relativity through Time-Mass Duality Part I Theoretical Foundations_en.pdf}{[Teil I]}) to cosmological scales (Section 2).
	
	\section{Beyond the Limits}
	\label{sec:beyond_limits}
	
	\subsection{Speculative Extensions}
	\label{subsec:speculative_extensions}
	
	The T0 model’s definition of \(\Tfield = \frac{\hbar}{m c^2}\) (Part I, Section 3.1 "Definition and Physical Basis" \href{https://github.com/jpascher/T0-Time-Mass-Duality/tree/main/2/pdf/English/Bridging Quantum Mechanics and Relativity through Time-Mass Duality Part I Theoretical Foundations_en.pdf}{[Teil I]}) suggests intriguing possibilities beyond conventional limits, particularly near the Planck scale (\(m_P \approx 2.176 \times 10^{-8} \, \text{kg}\), \(t_P \approx 5.391 \times 10^{-44} \, \text{s}\)). For masses below \(m_P\), \(\Tfield > t_P\), implying slower intrinsic dynamics:
	\begin{equation}
		T = \frac{\hbar}{m c^2} \propto \frac{1}{m},
		\label{eq:intrinsic_time_repeat}
	\end{equation}
	where \(m < m_P\) results in \(T > t_P\). This slowdown could stabilize states near singularities, suggesting finite physical conditions rather than GR’s infinite densities (e.g., black hole interiors).
	
	\begin{figure}[h]
		\centering
		\begin{tikzpicture}
			\draw[->] (0,0) -- (6,0) node[right] {Mass \(m\)};
			\draw[->] (0,0) -- (0,4) node[above] {Intrinsic Time \(T\)};
			\draw[scale=0.5, domain=0.1:10, smooth, variable=\x, blue, thick] plot ({\x}, {1/\x});
			\draw[dotted, red] (1.5,0) -- (1.5,1.5) -- (0,1.5);
			\node at (1.5,-0.3) {\(m_P\)};
			\node at (-0.3,1.5) {\(t_P\)};
			\node[blue] at (4.5,2) {\(T = \frac{\hbar}{m c^2}\)};
		\end{tikzpicture}
		\caption{Mass vs. intrinsic time, illustrating how \(\Tfield\) increases as mass decreases below the Planck scale (\(m_P\)), potentially stabilizing dynamics near singularities.}
		\label{fig:mass_time}
	\end{figure}
	
	This speculative extension challenges the light-speed barrier (\(c = 1\)) by allowing \(\Tfield\) to influence particle behavior beyond conventional constraints, a hypothesis testable with future high-energy experiments or astrophysical observations near extreme conditions (e.g., black hole event horizons) \cite{pascher_planck_2025}.
	
	\subsection{Philosophical Reflections}
	\label{subsec:philosophical_reflections}
	
	The T0 model’s static, eternal cosmos fundamentally departs from \(\Lambda\)CDM’s finite, expanding universe, offering profound philosophical implications. By avoiding singularities and infinite densities, T0 presents a unified reality where time is an intrinsic property (\(\Tfield\)) rather than a relativistic variable, and mass adapts dynamically to local conditions. This contrasts with \(\Lambda\)CDM’s fragmented ontology—featuring a Big Bang origin, dark components, and an uncertain fate—by proposing a coherent, infinite framework that aligns with intuitive notions of existence without beginning or end.
	
	The elimination of expansion and dark entities simplifies cosmology while preserving empirical consistency (Sections 4, 5), suggesting that the universe’s apparent complexity may stem from misinterpretations of frequency-based measurements (Section 5.4). Philosophically, T0 resonates with a holistic view of nature, where quantum and relativistic phenomena emerge from a single principle—time-mass duality—enhancing the SM’s explanatory power across all scales \cite{pascher_perspective_2025}.
	
	\section{Conclusion}
	\label{sec:conclusion}
	
	Part II demonstrates that the T0 model extends the SM with a static, testable cosmology, reinterpreting redshift, dark matter, and dark energy through \(\Tfield\)-mediated effects. Its predictions—wavelength-dependent redshift, a hotter CMB, and galaxy dynamics without dark matter—offer empirical pathways to distinguish it from \(\Lambda\)CDM, while its philosophical coherence provides a compelling alternative to the standard paradigm. Future work will refine experimental tests and explore speculative extensions, solidifying T0’s role as a unified framework bridging QM and RT \cite{pascher_perspective_2025}.
	
	\begin{acknowledgments}
		Thanks to Reinsprecht Martin Dipl.-Ing. Dr. for critical feedback.
	\end{acknowledgments}
	
	\bibliographystyle{apsrev4-2}
	\begin{thebibliography}{99}
		\bibitem{pascher_part1_2025} J. Pascher, \href{https://github.com/jpascher/T0-Time-Mass-Duality/tree/main/2/pdf/English/Bridging Quantum Mechanics and Relativity through Time-Mass Duality Part I Theoretical Foundations_en.pdf}{Bridging Quantum Mechanics and Relativity through Time-Mass Duality: A Unified Framework with Natural Units \(\alpha = \beta = 1\) Part I: Theoretical Foundations}, April 7, 2025.
		\bibitem{pascher_lagrange_2025} J. Pascher, \href{https://github.com/jpascher/T0-Time-Mass-Duality/tree/main/2/pdf/English/Mathematische Formulierungen der Zeit-Masse-Dualit\%C3\%A4tstheorie mit Lagrange_en.pdf}{From Time Dilation to Mass Variation: Mathematical Core Formulations of Time-Mass Duality Theory}, March 29, 2025.
		\bibitem{pascher_messdifferenzen_2025} J. Pascher, \href{https://github.com/jpascher/T0-Time-Mass-Duality/tree/main/2/pdf/English/Analyse der Messdifferenzen zwischen dem T0-Modell und dem Standardmodell_en.pdf}{Compensatory and Additive Effects: An Analysis of Measurement Differences Between the T0 Model and the \(\Lambda\)CDM Standard Model}, April 2, 2025.
		\bibitem{pascher_temp_2025} J. Pascher, \href{https://github.com/jpascher/T0-Time-Mass-Duality/tree/main/2/pdf/English/Anpassung von Temperatureinheiten in nat%C3%BCrlichen Einheiten und CMB-Messungen_en.pdf}{Adjustment of Temperature Units in Natural Units and CMB Measurements}, April 2, 2025.
		\bibitem{pascher_params_2025} J. Pascher, \href{https://github.com/jpascher/T0-Time-Mass-Duality/tree/main/2/pdf/English/Zeit-Masse-Dualit\%C3\%A4tstheorie (T0-Modell) Herleitung der Parameter kappa, alpha und beta_en.pdf}{Time-Mass Duality Theory (T0 Model): Derivation of Parameters \(\kappa\), \(\alpha\), and \(\beta\)}, April 4, 2025.
		\bibitem{pascher_galaxies_2025} J. Pascher, \href{https://github.com/jpascher/T0-Time-Mass-Duality/tree/main/2/pdf/English/Massenvariation in Galaxien_en.pdf}{Mass Variation in Galaxies: An Analysis in the T0 Model with Emergent Gravitation}, March 30, 2025.
		\bibitem{pascher_quantum_2025} J. Pascher, \href{https://github.com/jpascher/T0-Time-Mass-Duality/tree/main/2/pdf/English/Die Notwendigkeit einer Erweiterung der Standard-Quantenmechanik und Quantenfeldtheorie_en.pdf}{The Necessity of Extending Standard Quantum Mechanics and Quantum Field Theory}, March 27, 2025.
		\bibitem{pascher_planck_2025} J. Pascher, \href{https://github.com/jpascher/T0-Time-Mass-Duality/tree/main/2/pdf/English/Jenseits der Planck-Skala_en.pdf}{Real Consequences of Reformulating Time and Mass in Physics: Beyond the Planck Scale}, March 24, 2025.
		\bibitem{pascher_perspective_2025} J. Pascher, \href{https://github.com/jpascher/T0-Time-Mass-Duality/tree/main/2/pdf/English/Eine neue Perspektive auf Zeit und Raum Johann Paschers revolution\%C3\%A4re Ideen_en.pdf}{A New Perspective on Time and Space: Johann Pascher’s Revolutionary Ideas}, March 25, 2025.
		\bibitem{pascher_alphabeta_2025} J. Pascher, \href{https://github.com/jpascher/T0-Time-Mass-Duality/tree/main/2/pdf/English/Die Konsistenz von alpha = 1 und beta = 1_en.pdf}{Unified Unit System in the T0 Model: The Consistency of \(\alpha = 1\) and \(\beta = 1\)}, April 5, 2025.
		\bibitem{pascher_emergente_2025} J. Pascher, \href{https://github.com/jpascher/T0-Time-Mass-Duality/tree/main/2/pdf/English/Emergente Gravitation im T0-Modell Eine formale Herleitung_en.pdf}{Emergent Gravitation in the T0 Model: A Comprehensive Derivation}, April 1, 2025.
		\bibitem{Planck2020} Planck Collaboration, Astron. Astrophys. \textbf{641}, A6 (2020).
		\bibitem{Riess1998} A. G. Riess et al., Astron. J. \textbf{116}, 1009 (1998).
		\bibitem{Perlmutter1999} S. Perlmutter et al., Astrophys. J. \textbf{517}, 565 (1999).
		\bibitem{Fixsen2009} D. J. Fixsen, Astrophys. J. \textbf{707}, 916 (2009).
		\bibitem{McGaugh2016} S. S. McGaugh et al., Phys. Rev. Lett. \textbf{117}, 201101 (2016).
		\bibitem{Will2014} C. M. Will, Living Rev. Relativ. \textbf{17}, 4 (2014).
		\bibitem{DiValentino2021} E. Di Valentino et al., Class. Quantum Grav. \textbf{38}, 153001 (2021).
				\bibitem{pascher_qft_2025} J. Pascher, \href{https://github.com/jpascher/T0-Time-Mass-Duality/tree/main/2/pdf/English/Quantenfeldtheoretische Behandlung des intrinsischen Zeitfelds im T0-Modell_en.pdf}{Quantum Field Theoretical Treatment of the Intrinsic Time Field in the T0 Model}, April 8, 2025.
	\end{thebibliography}
	
\end{document}