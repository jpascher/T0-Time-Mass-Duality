\documentclass[12pt,a4paper]{article}
\usepackage[utf8]{inputenc}
\usepackage[T1]{fontenc}
\usepackage[ngerman]{babel}
\usepackage{lmodern}
\usepackage{csquotes}
\usepackage{amsmath}
\usepackage{amssymb}
\usepackage{physics}
\usepackage{geometry}
\usepackage{tocloft}
\usepackage{xcolor}
\usepackage{graphicx,tikz,pgfplots}
\pgfplotsset{compat=1.18}
\usepackage{booktabs}
\usepackage{siunitx}
\usepackage{amsthm}
\usepackage[colorlinks=true, linkcolor=blue, citecolor=blue, urlcolor=blue]{hyperref}
\usepackage{cleveref}
\usepackage{fancyhdr}

\geometry{a4paper, margin=2cm}

% Headers and Footers
\pagestyle{fancy}
\fancyhf{}
\fancyhead[L]{Johann Pascher}
\fancyhead[R]{Time-Mass Duality}
\fancyfoot[C]{\thepage}
\renewcommand{\headrulewidth}{0.4pt}
\renewcommand{\footrulewidth}{0.4pt}

% Table of Contents Styling
\renewcommand{\cftsecfont}{\color{blue}}
\renewcommand{\cftsubsecfont}{\color{blue}}
\renewcommand{\cftsecpagefont}{\color{blue}}
\renewcommand{\cftsubsecpagefont}{\color{blue}}
\setlength{\cftsecindent}{1cm}
\setlength{\cftsubsecindent}{2cm}

% Custom commands (consistent with other documents)
\newcommand{\Tfield}{T(x)}
\newcommand{\betaT}{\beta_{\text{T}}}
\newcommand{\alphaEM}{\alpha_{\text{EM}}}
\newcommand{\alphaW}{\alpha_{\text{W}}}
\newcommand{\Mpl}{M_{\text{Pl}}}
\newcommand{\Tzerot}{T_0(\Tfield)}
\newcommand{\Tzero}{T_0}
\newcommand{\vecx}{\vec{x}}
\newcommand{\gammaf}{\gamma_{\text{Lorentz}}}
\newcommand{\DhiggsT}{\Tfield (\partial_\mu + ig A_\mu) \Phi + \Phi \partial_\mu \Tfield}

\newtheorem{theorem}{Theorem}[section]
\newtheorem{proposition}[theorem]{Proposition}

\title{Dynamic Mass of Photons and Its Implications for Nonlocality in the T0 Model}
\author{Johann Pascher}
\date{March 25, 2025}

\begin{document}
	
	\maketitle
	
	\begin{abstract}
		This work examines the implications of assigning a dynamic, frequency-dependent effective mass to photons within the framework of the T0 model of time-mass duality, which postulates absolute time and variable mass. By assuming \(m_\gamma = \omega\) in natural units, an energy-dependent intrinsic time is introduced, influencing nonlocality and causality. The theory builds on the T0 model’s framework and is supported by experimental predictions consistent with its principles.
	\end{abstract}
	
	\tableofcontents
	\newpage
	
	\section{Introduction}
	This work analyzes the implications of a dynamic, frequency-dependent effective mass for photons within the T0 model of quantum mechanics, which assumes absolute time and variable mass \cite{pascher_galaxies_2025}. The concept extends the model’s intrinsic time framework to explore nonlocality and causality.
	
	\section{Natural Units as the Foundation}
	\subsection{Definition of Natural Units}
	\begin{theorem}[Natural Units]
		With \(\hbar = c = G = 1\):
		\begin{align}
			[L] &= [E^{-1}] \\
			[T] &= [E^{-1}] \\
			[M] &= [E]
		\end{align}
	\end{theorem}
	
	\subsection{Significance for Mass-Energy Equivalence}
	In the T0 model, mass is dynamic (\(\Tfield = \frac{\hbar}{m c^2}\)). For photons, an effective mass is proposed:
	\begin{equation}
		m_\gamma = \omega
	\end{equation}
	where \(\omega\) is the angular frequency, consistent with \(E = \hbar \omega\) in natural units (\(\hbar = 1\)).
	
	\section{Time Models in Quantum Mechanics}
	\subsection{Limitations of the Standard Model}
	The standard Schrödinger equation assumes a universal time:
	\begin{equation}
		i\hbar\frac{\partial\psi}{\partial t} = H\psi
	\end{equation}
	
	\subsection{The T0 Model with Absolute Time}
	In the T0 model, energy is linked to a constant intrinsic time \(T_0\):
	\begin{equation}
		E = \frac{\hbar}{T_0}
	\end{equation}
	For massive particles, \(\Tfield = \frac{\hbar}{m c^2}\).
	
	\subsection{Extension for Photons}
	For photons, this extends to an energy-dependent intrinsic time:
	\begin{equation}
		\Tfield = \frac{\hbar}{m_\gamma c^2} = \frac{1}{\omega}
	\end{equation}
	This remains consistent with \(m_\gamma = \omega\) (since \(\hbar = c = 1\)).
	
	\section{Unification in the T0 Model}
	To unify massive particles and photons:
	\begin{equation}
		\Tfield = \frac{\hbar}{\max(m c^2, \omega)}
	\end{equation}
	For massive particles, \(m c^2\) dominates; for photons, \(\omega\).
	
	\section{Implications for Nonlocality and Entanglement}
	\subsection{Energy-Dependent Correlations}
	The energy-dependent \(\Tfield\) leads to time delays in entangled systems:
	\begin{itemize}
		\item Delay: \(\left|\frac{1}{\omega_1} - \frac{1}{\omega_2}\right|\)
	\end{itemize}
	This suggests that nonlocality emerges from intrinsic time differences, akin to the energy loss mechanism of redshift in the T0 model \cite{pascher_messdifferenzen_2025}.
	
	\subsection{\(\betaT\) in the T0 Model}
	In the T0 model, wavelength-dependent redshift is described by the parameter \(\betaT\), with \(\betaT^{\text{SI}} \approx 0.008\) in SI units and \(\betaT^{\text{nat}} = 1\) in natural units \cite{pascher_params_2025}. These values are equivalent, reflecting the same physical reality, with conversion via the characteristic length scale \(r_0\) \cite{pascher_temp_2025}. The derivation of \(\betaT\) is well-established in the T0 model, and the choice between \(\betaT^{\text{SI}}\) and \(\betaT^{\text{nat}}\) depends solely on the unit system, without uncertainty in the theoretical foundation.
	
	\begin{figure}[h]
		\centering
		\begin{tikzpicture}
			\begin{axis}[
				xlabel={Energy [eV]},
				ylabel={Time [eV\(^{-1}\)]},
				xlabel style={font=\large},
				ylabel style={font=\large},
				tick label style={font=\normalsize},
				xmin=0, xmax=10,
				ymin=0, ymax=10,
				legend pos=north east,
				legend style={font=\large},
				grid=both,
				minor tick num=1
				]
				\addplot[blue, ultra thick, domain=0.1:10, samples=100] {1/x};
				\legend{\(T = E^{-1}\)}
			\end{axis}
		\end{tikzpicture}
		\caption{Energy-dependent intrinsic time for photons in the T0 model.}
	\end{figure}
	
	\section{Experimental Verification}
	\begin{itemize}
		\item Frequency-dependent Bell tests to measure time delays in entanglement.
		\item Spectroscopic redshift measurements to validate wavelength-dependent redshift with \(\betaT\).
	\end{itemize}
	
	\section{Physics Beyond the Speed of Light}
	A hypothetical modified dispersion relation in the T0 model:
	\begin{equation}
		E^2 = (m_\gamma c^2)^2 + (p c)^2 + \alpha_c p^4 c^2 / E_P^2
	\end{equation}
	where \(\alpha_c\) is a coupling constant and \(E_P\) is the Planck energy, could explain the behavior of high-energy photons and be tested via cosmic ray measurements.
	
	\section{Conclusion}
	The dynamic effective mass of photons in the T0 model offers a novel view of nonlocality as an emergent phenomenon driven by energy-dependent intrinsic time, enhancing the explanatory power of the model.
	
	\begin{thebibliography}{99}
		\bibitem{pascher_galaxies_2025} Pascher, J. (2025). \href{https://github.com/jpascher/T0-Time-Mass-Duality/tree/main/2/pdf/English/MassVarGalaxienEn.pdf}{Mass Variation in Galaxies: An Analysis in the T0 Model with Emergent Gravitation}. March 30, 2025.
		\bibitem{pascher_messdifferenzen_2025} Pascher, J. (2025). \href{https://github.com/jpascher/T0-Time-Mass-Duality/tree/main/2/pdf/English/MessdifferenzenT0StandardEn.pdf}{Compensatory and Additive Effects: An Analysis of Measurement Differences Between the T0 Model and the \(\Lambda\)CDM Standard Model}. April 2, 2025.
		\bibitem{pascher_params_2025} Pascher, J. (2025). \href{https://github.com/jpascher/T0-Time-Mass-Duality/tree/main/2/pdf/English/ZeitMasseT0ParamsEn.pdf}{Time-Mass Duality Theory (T0 Model): Derivation of Parameters \(\kappa\), \(\alpha\), and \(\beta\)}. April 4, 2025.
		\bibitem{pascher_temp_2025} Pascher, J. (2025). \href{https://github.com/jpascher/T0-Time-Mass-Duality/tree/main/2/pdf/English/NatEinheitenAlpha1En.pdf}{Adjustment of Temperature Units in Natural Units and CMB Measurements}. April 2, 2025.
		\bibitem{einstein} Einstein, A. (1905). \textit{On the Electrodynamics of Moving Bodies}. \textit{Annalen der Physik}, 322(10), 891-921.
		\bibitem{planck} Planck, M. (1901). \textit{On the Law of Energy Distribution in the Normal Spectrum}. \textit{Annalen der Physik}, 309(3), 553-563.
		\bibitem{bell} Bell, J. S. (1964). \textit{On the Einstein-Podolsky-Rosen Paradox}. \textit{Physics}, 1(3), 195-200.
		\bibitem{feynman} Feynman, R. P. (1985). \textit{QED: The Strange Theory of Light and Matter}. Princeton University Press.
	\end{thebibliography}
	
\end{document}