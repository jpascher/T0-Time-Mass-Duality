\documentclass[a4paper,12pt]{article}
\usepackage[utf8]{inputenc}
\usepackage[T1]{fontenc}
\usepackage{lmodern}
\usepackage[ngerman]{babel}
\usepackage{amsmath, amssymb, amsthm, physics}
\usepackage{graphicx}
\usepackage{xcolor}
\usepackage{tikz}
\usepackage{setspace}
\usepackage{tcolorbox}
\usepackage{booktabs}
\usepackage{siunitx}
\DeclareSIUnit{\year}{yr}
\DeclareSIUnit{\parsec}{pc}
\usepackage{geometry}
\usepackage{tocloft}
\usepackage{fancyhdr}

\geometry{margin=2cm}

% Headers and Footers
\pagestyle{fancy}
\fancyhf{}
\fancyhead[L]{Johann Pascher}
\fancyhead[R]{Time-Mass Duality}
\fancyfoot[C]{\thepage}
\renewcommand{\headrulewidth}{0.4pt}
\renewcommand{\footrulewidth}{0.4pt}

% Table of Contents Styling
\renewcommand{\cftsecfont}{\color{blue}}
\renewcommand{\cftsubsecfont}{\color{blue}}
\renewcommand{\cftsecpagefont}{\color{blue}}
\renewcommand{\cftsubsecpagefont}{\color{blue}}
\setlength{\cftsecindent}{1cm}
\setlength{\cftsubsecindent}{2cm}

% Colored links in table of contents and document
\usepackage{hyperref}
\hypersetup{
	colorlinks=true,
	linkcolor=blue,
	filecolor=blue,
	citecolor=blue,
	urlcolor=blue,
	bookmarks=true,
	bookmarksopen=true,
	pdftitle={''Dark Energy in the T0 Model: A Mathematical Analysis of Energy Dynamics''},
	pdfauthor={Johann Pascher}
}

% cleveref must be loaded after hyperref
\usepackage{cleveref}

% Theorem styles
\newtheorem{theorem}{Theorem}[section]
\newtheorem{lemma}[theorem]{Lemma}
\newtheorem{proposition}[theorem]{Proposition}
\newtheorem{corollary}[theorem]{Corollary}

\theoremstyle{definition}
\newtheorem{definition}{Definition}[theorem]
\newtheorem{example}{Example}

\theoremstyle{remark}
\newtheorem{remark}{Remark}
\renewcommand{\proofname}{Proof}

% Custom commands (consistent with other documents)
\newcommand{\Tfield}{T(x)}
\newcommand{\DcovT}[1]{\Tfield D_\mu #1 + #1 \partial_\mu \Tfield}
\newcommand{\DhiggsT}{\Tfield (\partial_\mu + ig A_\mu) \Phi + \Phi \partial_\mu \Tfield}
\newcommand{\betaT}{\beta_{\text{T}}}
\newcommand{\alphaEM}{\alpha_{\text{EM}}}
\newcommand{\alphaW}{\alpha_{\text{W}}}
\newcommand{\Mpl}{M_{\text{Pl}}}
\newcommand{\Tzerot}{T_0(\Tfield)}
\newcommand{\Tzero}{T_0}
\newcommand{\vecx}{\vec{x}}
\newcommand{\gammaf}{\gamma_{\text{Lorentz}}}

\begin{document}
	
	\title{''Dark Energy in the T0 Model: \\A Mathematical Analysis of Energy Dynamics''}
	\author{Johann Pascher}
	\date{March 30, 2025}
	\maketitle
	
	\begin{abstract}
		This work develops a detailed mathematical analysis of dark energy within the framework of the T0 model with absolute time and variable mass. Unlike the \(\Lambda\)CDM standard model, dark energy is not considered a driving force of cosmic expansion but emerges as a dynamic effect of energy exchange in a static universe, mediated by the intrinsic time field \(\Tfield\). The document builds on foundations from \cite{pascher_params_2025} and the gravitation theory from \cite{pascher_galaxies_2025}, characterizes energy transfer rates, analyzes the radial density profile of dark energy, and explains the observed redshift as a result of photon energy loss to this field (see \cite{pascher_messdifferenzen_2025}). Experimental tests to distinguish this interpretation from the standard model conclude the analysis.
	\end{abstract}
	
	\tableofcontents
	\newpage
	
	%======================= PART 1: INTRODUCTION ========================
	\section{Introduction}
	
	The discovery of accelerated cosmic expansion through supernova observations in the late 1990s led to the introduction of dark energy as the dominant component of the universe in the \(\Lambda\)CDM standard model, where it is modeled as a cosmological constant (\(\Lambda\)) with negative pressure, accounting for approximately 68\% of the energy content. This work pursues an alternative approach within the T0 model, based on time-mass duality (see \cite{pascher_params_2025}, Section ''Time-Mass Duality''). Here, time is absolute, and mass varies, with dark energy not being a separate entity driving expansion but an emergent effect of the intrinsic time field \(\Tfield\). Cosmic redshift is explained not by spatial expansion but by the energy loss of photons to \(\Tfield\), as detailed in \cite{pascher_messdifferenzen_2025} (Section ''Energy Loss and Redshift'') and \cite{pascher_temp_2025} (Section ''Temperature Scaling''). The energy dynamics are mathematically analyzed below, referencing established derivations such as gravitation theory in \cite{pascher_galaxies_2025} and parameters in \cite{pascher_params_2025}. Experimental tests to differentiate from the standard model conclude the work.
	
	%======================= PART 2: MATHEMATICAL FOUNDATIONS ========================
	\section{Mathematical Foundations of the T0 Model}
	
	\subsection{Time-Mass Duality}
	
	The T0 model postulates a duality between time and mass, enabling two descriptions:
	\begin{enumerate}
		\item \textbf{Standard View}: Time dilation (\(t' = \gamma t\)), constant rest mass (\(m_0\)).
		\item \textbf{T0 Model}: Absolute time (\(T_0\)), variable mass (\(m = \gamma m_0\)).
	\end{enumerate}
	The complete derivation and transformations are provided in \cite{pascher_params_2025} (Section ''Time-Mass Duality'') and \cite{pascher_galaxies_2025} (Section ''Foundations''). An overview is given in the table:
	
	\begin{table}[h]
		\centering
		\begin{tabular}{|l|c|c|}
			\hline
			\textbf{Quantity} & \textbf{Standard View} & \textbf{T0 Model} \\
			\hline
			Time & \(t' = \gamma t\) & \(t = \text{const.}\) \\
			Mass & \(m = \text{const.}\) & \(m = \gamma m_0\) \\
			Intrinsic Time & \(T = \frac{\hbar}{m c^2}\) & \(T = \frac{\hbar}{\gamma m_0 c^2}\) \\
			\hline
		\end{tabular}
		\caption{Transformations in the T0 Model (see \cite{pascher_params_2025})}
	\end{table}
	
	\subsection{Intrinsic Time}
	
	The intrinsic time \(\Tfield\) is central to the T0 model:
	
	\begin{definition}[Intrinsic Time]
		For a particle with mass \(m\):
		\begin{equation}
			\Tfield = \frac{\hbar}{m c^2}
		\end{equation}
		The derivation is detailed in \cite{pascher_params_2025} (Section ''Definition of Intrinsic Time'').
	\end{definition}
	
	\begin{corollary}[Scalar Field]
		As a field:
		\begin{equation}
			\Tfield = \frac{\hbar}{y \langle\Phi\rangle c^2}
		\end{equation}
		Details on the Higgs field are in \cite{pascher_higgs_2025} (Section ''Higgs Mechanism'').
	\end{corollary}
	
	\subsection{Modified Derivative Operators}
	
	The operators were introduced in \cite{pascher_lagrange_2025} (Section ''Lagrangian Formulation''):
	
	\begin{definition}[Modified Time Derivative]
		\begin{equation}
			\partial_{t/T} = T \frac{\partial}{\partial t}
		\end{equation}
	\end{definition}
	
	\begin{definition}[Covariant Derivative]
		For a field \(\Psi\):
		\begin{equation}
			D_{T,\mu} \Psi = \Tfield D_\mu \Psi + \Psi \partial_\mu \Tfield
		\end{equation}
	\end{definition}
	
	\begin{definition}[Higgs Field Derivative]
		\begin{equation}
			D_{T,\mu} \Phi = \DhiggsT
		\end{equation}
	\end{definition}
	
	%======================= PART 3: FIELD EQUATIONS ========================
	\section{Modified Field Equations for Dark Energy}
	
	\subsection{Modified Lagrangian Density}
	
	The Lagrangian density is derived in \cite{pascher_lagrange_2025} (Section ''Total Lagrangian Density''):
	
	\begin{equation}
		\mathcal{L}_{\text{Total}} = \mathcal{L}_{\text{Boson}} + \mathcal{L}_{\text{Fermion}} + \mathcal{L}_{\text{Higgs-T}} + \mathcal{L}_{\text{intrinsic}}
	\end{equation}
	
	With:
	\begin{align}
		\mathcal{L}_{\text{Boson}} &= -\frac{1}{4} F_{\mu\nu} F^{\mu\nu} \\
		\mathcal{L}_{\text{Fermion}} &= \bar{\psi} i \gamma^\mu \DcovT{\psi} - y_f \bar{\psi}_L \Phi \psi_R + \text{h.c.} \\
		\mathcal{L}_{\text{Higgs-T}} &= (D_{T,\mu} \Phi)^\dagger (D_{T,\mu} \Phi) - V(\Tfield, \Phi)
	\end{align}
	
	With the Higgs potential:
	\begin{equation}
		V(\Tfield, \Phi) = -\mu^2 \Phi^\dagger \Phi + \lambda (\Phi^\dagger \Phi)^2
	\end{equation}
	
	And the intrinsic time field Lagrangian density:
	\begin{equation}
		\mathcal{L}_{\text{intrinsic}} = \frac{1}{2} \partial_\mu \Tfield \partial^\mu \Tfield - \frac{1}{2}\Tfield^2 - \frac{\rho}{\Tfield}
	\end{equation}
	
	\subsection{Dark Energy as an Emergent Effect}
	
	Dark energy arises from \(\Tfield\) variations, as described in \cite{pascher_galaxies_2025} (Section ''Emergent Gravitation''):
	
	\begin{equation}
		\rho_{\text{DE}}(r) \approx \frac{1}{2} (\nabla \Tfield)^2
	\end{equation}
	
	Details on \(\kappa\) are in \cite{pascher_params_2025} (Section ''Parameter Derivations'').
	
	\subsection{Energy Density Profile}
	
	The energy density of the time field gradient can be approximated as:
	\begin{equation}
		\rho_{\text{DE}}(r) \approx \frac{1}{2} (\nabla \Tfield)^2
	\end{equation}
	See \cite{pascher_galaxies_2025} (Section ''Energy Density'').
	
	\subsection{Emergent Gravitation}
	
	\begin{theorem}[Emergence of Gravitation]
		\begin{equation}
			\nabla \Tfield = -\frac{\hbar}{m^2 c^2} \nabla m \sim \nabla \Phi_g
		\end{equation}
		Full derivation in \cite{pascher_galaxies_2025} (Section ''Emergent Gravitation'').
	\end{theorem}
	
	\begin{proof}
		In regions with gravitational potential \(\Phi_g\), the effective mass varies as:
		\begin{equation}
			m(\vec{r}) = m_0 \left(1 + \frac{\Phi_g(\vec{r})}{c^2}\right)
		\end{equation}
		
		Thus:
		\begin{equation}
			\nabla m = \frac{m_0}{c^2} \nabla \Phi_g
		\end{equation}
		
		Substituting into the gradient of the intrinsic time field:
		\begin{equation}
			\nabla \Tfield = -\frac{\hbar}{m^2 c^2} \cdot \frac{m_0}{c^2} \nabla \Phi_g
		\end{equation}
	\end{proof}
	
	The field equation for the intrinsic time field is:
	\begin{equation}
		\nabla^2 \Tfield = -\kappa \rho(\vecx) \Tfield^2
	\end{equation}
	
	In natural units with G = 1, the Poisson equation is:
	\begin{equation}
		\nabla^2 \Phi = 4\pi \rho
	\end{equation}
	
	%======================= PART 4: ENERGY TRANSFER AND REDSHIFT ========================
	\section{Energy Transfer and Redshift}
	
	\subsection{Photon Energy Loss}
	
	Redshift results from energy loss, derived in \cite{pascher_messdifferenzen_2025} (Section ''Energy Loss''):
	
	\begin{equation}
		\frac{d E_{\gamma}}{d x} = -\alpha E_{\gamma}, \quad E_{\gamma}(x) = E_{\gamma,0} e^{-\alpha x}
	\end{equation}
	
	\begin{equation}
		1 + z = e^{\alpha d}, \quad \alpha = \frac{H_0}{c}
	\end{equation}
	
	Details on \(\alpha\) in \cite{pascher_params_2025} (Section ''Derivation of \(\alpha\)'').
	
	\subsection{Modified Energy-Momentum Relation}
	
	\begin{theorem}[Energy-Momentum Relation]
		\begin{equation}
			E^2 = p^2 + m^2 + \alpha_c \frac{p^4}{E_P^2}
		\end{equation}
		See \cite{pascher_photons_2025} (Section ''Physics Beyond the Speed of Light'').
	\end{theorem}
	
	\begin{theorem}[Wavelength Dependence]
		\begin{equation}
			z(\lambda) = z_0 (1 + \betaT^{\text{nat}} \ln(\lambda/\lambda_0))
		\end{equation}
		With \(\betaT^{\text{nat}} = 1\) in natural units (see \cite{pascher_params_2025}).
	\end{theorem}
	
	\subsection{Energy Balance Equation}
	
	\begin{equation}
		\rho_{\text{total}} = \rho_{\text{Matter}} + \rho_{\gamma} + \rho_{\text{DE}} = \text{const.}
	\end{equation}
	
	\begin{align}
		\frac{d \rho_{\text{Matter}}}{d t} &= -\alpha \rho_{\text{Matter}} \\
		\frac{d \rho_{\gamma}}{d t} &= -\alpha \rho_{\gamma} \\
		\frac{d \rho_{\text{DE}}}{d t} &= \alpha (\rho_{\text{Matter}} + \rho_{\gamma})
	\end{align}
	See \cite{pascher_messdifferenzen_2025} (Section ''Energy Balance'').
	
	%======================= PART 5: PARAMETERS ========================
	\section{Quantitative Determination of Parameters}
	
	\subsection{Parameters in Natural Units}
	
	\begin{theorem}[Key Parameters]
		\begin{align}
			\kappa &= \betaT^{\text{nat}} \frac{y v}{r_g^2} = \frac{y v}{r_g^2} \\
			\alpha &= \frac{\lambda_h^2 v}{L_T} \\
			\betaT^{\text{nat}} &= \frac{\lambda_h^2 v^2}{16\pi^3 m_h^2 \xi} = 1
		\end{align}
		Derivation in \cite{pascher_params_2025} (Section ''Parameter Derivations'').
	\end{theorem}
	
	\subsection{Gravitational Potential}
	
	\begin{theorem}[Gravitational Potential]
		\begin{equation}
			\Phi(r) = -\frac{M}{r} + \kappa r
		\end{equation}
		See \cite{pascher_galaxies_2025} (Section ''Modified Gravitational Potential'').
	\end{theorem}
	
	%======================= PART 6: OBSERVATIONS AND TESTS ========================
	\section{Dark Energy and Cosmological Observations}
	
	\subsection{Type Ia Supernovae}
	
	\begin{equation}
		d_L = \ln(1+z) (1+z)
	\end{equation}
	See \cite{pascher_messdifferenzen_2025} (Section ''Supernovae'').
	
	\subsection{Energy Density Parameter}
	
	\begin{equation}
		\Omega_{DE}^{\text{eff}} \approx \frac{3 \kappa}{R_U H_0^2} \approx 0.68
	\end{equation}
	
	\section{Experimental Tests}
	
	\subsection{Fine Structure Constant}
	
	\begin{equation}
		\frac{d \alpha_{\text{EM}}}{d t} \approx 10^{-18}
	\end{equation}
	See \cite{pascher_photons_2025} (Section ''Experimental Verification'').
	
	\subsection{Environment-Dependent Redshift}
	
	\begin{equation}
		\frac{z_{\text{Cluster}}}{z_{\text{Void}}} \approx 1 + 0.003
	\end{equation}
	
	\subsection{Differential Redshift}
	
	\begin{equation}
		\frac{z(\lambda_1)}{z(\lambda_2)} \approx 1 + \betaT^{\text{nat}} \frac{\lambda_1 - \lambda_2}{\lambda_0} = 1 + \frac{\lambda_1 - \lambda_2}{\lambda_0}
	\end{equation}
	
	\section{Outlook and Summary}
	
	The T0 model provides a framework for a static universe where dark energy emerges from \(\Tfield\). Future tests (e.g., Euclid) can validate it.
	
	\begin{thebibliography}{99}
		\bibitem{pascher_galaxies_2025} Pascher, J. (2025). \href{https://github.com/jpascher/T0-Time-Mass-Duality/tree/main/2/pdf/English/MassVarGalaxienEn.pdf}{''Mass Variation in Galaxies: An Analysis in the T0 Model with Emergent Gravitation''}. March 30, 2025.
		\bibitem{pascher_messdifferenzen_2025} Pascher, J. (2025). \href{https://github.com/jpascher/T0-Time-Mass-Duality/tree/main/2/pdf/English/MessdifferenzenT0StandardEn.pdf}{''Compensatory and Additive Effects: An Analysis of Measurement Differences Between the T0 Model and the \(\Lambda\)CDM Standard Model''}. April 2, 2025.
		\bibitem{pascher_temp_2025} Pascher, J. (2025). \href{https://github.com/jpascher/T0-Time-Mass-Duality/tree/main/2/pdf/English/NatEinheitenAlpha1En.pdf}{''Adjustment of Temperature Units in Natural Units and CMB Measurements''}. April 2, 2025.
		\bibitem{pascher_params_2025} Pascher, J. (2025). \href{https://github.com/jpascher/T0-Time-Mass-Duality/tree/main/2/pdf/English/ZeitMasseT0ParamsEn.pdf}{''Time-Mass Duality Theory (T0 Model): Derivation of Parameters \(\kappa\), \(\alpha\), and \(\beta\)''}. April 4, 2025.
		\bibitem{pascher_lagrange_2025} Pascher, J. (2025). \href{https://github.com/jpascher/T0-Time-Mass-Duality/tree/main/2/pdf/English/MathZeitMasseLagrange.pdf}{''From Time Dilation to Mass Variation: Mathematical Core Formulations of Time-Mass Duality Theory''}. March 29, 2025.
		\bibitem{pascher_higgs_2025} Pascher, J. (2025). \href{https://github.com/jpascher/T0-Time-Mass-Duality/tree/main/2/pdf/English/MathHiggsZeitMasseEn.pdf}{''Mathematical Formulation of the Higgs Mechanism in Time-Mass Duality''}. March 28, 2025.
		\bibitem{pascher_photons_2025} Pascher, J. (2025). \href{https://github.com/jpascher/T0-Time-Mass-Duality/tree/main/2/pdf/English/DynMassePhotonenNichtlokalEn.pdf}{''Dynamic Mass of Photons and Its Implications for Nonlocality in the T0 Model''}. March 25, 2025.
	\end{thebibliography}
	
\end{document}