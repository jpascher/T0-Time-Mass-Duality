\documentclass[12pt,a4paper]{article}
\usepackage[utf8]{inputenc}
\usepackage[T1]{fontenc}
\usepackage[ngerman]{babel}
\usepackage[left=2cm,right=2cm,top=2cm,bottom=2cm]{geometry}
\usepackage{lmodern}
\usepackage{amsmath}
\usepackage{amssymb}
\usepackage{physics}
\usepackage{hyperref}
\usepackage{tocloft}
\usepackage{tcolorbox}
\usepackage{booktabs}
\usepackage{enumitem}
\usepackage[table,xcdraw]{xcolor}
\usepackage{pgfplots}
\pgfplotsset{compat=1.18}
\usepackage{graphicx}
\usepackage{float}
\usepackage{mathtools}
\usepackage[T2A,T1]{fontenc}
\usepackage{fancyhdr}

\renewcommand{\cftsecfont}{\color{blue}}
\renewcommand{\cftsubsecfont}{\color{blue}}
\renewcommand{\cftsecpagefont}{\color{blue}}
\renewcommand{\cftsubsecpagefont}{\color{blue}}
\setlength{\cftsecindent}{1cm}
\setlength{\cftsubsecindent}{2cm}

\hypersetup{
	colorlinks=true,
	linkcolor=blue,
	citecolor=blue,
	urlcolor=blue,
	pdftitle={Time as an Emergent Property in Quantum Mechanics},
	pdfauthor={Johann Pascher},
	pdfsubject={Theoretical Physics},
	pdfkeywords={T0 Model, Time-Mass Duality, Quantum Mechanics, Fine-Structure Constant}
}

% Custom Commands
\newcommand{\Tfield}{T(x)}
\newcommand{\betaT}{\beta_{\text{T}}}
\newcommand{\alphaEM}{\alpha_{\text{EM}}}
\newcommand{\alphaW}{\alpha_{\text{W}}}
\newcommand{\Mpl}{M_{\text{Pl}}}
\newcommand{\Tzerot}{T_0(\Tfield)}
\newcommand{\Tzero}{T_0}
\newcommand{\vecx}{\vec{x}}
\newcommand{\DhiggsT}{\Tfield (\partial_\mu + ig A_\mu) \Phi + \Phi \partial_\mu \Tfield}
\newcommand{\DcovT}[1]{\Tfield D_\mu #1 + #1 \partial_\mu \Tfield}
\newcommand{\HiggsLagr}{\mathcal{L}_{\text{Higgs-T}}}

% Headers and Footers
\pagestyle{fancy}
\fancyhf{}
\fancyhead[L]{Johann Pascher}
\fancyhead[R]{Time-Mass Duality}
\fancyfoot[C]{\thepage}
\renewcommand{\headrulewidth}{0.4pt}
\renewcommand{\footrulewidth}{0.4pt}

\title{Time as an Emergent Property in Quantum Mechanics: \\A Connection Between Relativity, Fine-Structure Constant, and Quantum Dynamics}
\author{Johann Pascher}
\date{March 23, 2025}

\begin{document}
	
	\maketitle
	
	\tableofcontents
	\newpage
	
	\section{Introduction}
	In modern physics, time and space are treated differently. While spatial coordinates in quantum mechanics are represented by operators, time primarily appears as a parameter. This asymmetric treatment raises fundamental questions about the nature of time. This work explores the extent to which time can be understood as an emergent property, linked to fundamental constants and the mass of the system under consideration.
	
	This investigation is part of a broader conceptual framework, discussed in detail in the companion paper \href{https://github.com/jpascher/T0-Time-Mass-Duality/tree/main/2/pdf/English/KomplementPhysikZeitEn.pdf}{\textit{Complementary Extensions of Physics: Absolute Time and Intrinsic Time}} \cite{pascher1} (March 24, 2025). There, the intrinsic time introduced here is connected with a complementary model of absolute time (the \href{https://github.com/jpascher/T0-Time-Mass-Duality/tree/main/2/pdf/English/ZeitMasseT0ParamsEn.pdf}{T0 Model} \cite{pascher_params_2025}), resulting in a \href{https://github.com/jpascher/T0-Time-Mass-Duality/tree/main/2/pdf/English/MathZeitMasseLagrangeEn.pdf}{time-mass duality} \cite{pascher_lagrange_2025} conceptually similar to the well-known wave-particle duality.
	
	\section{Time in Special Relativity}
	Einstein’s famous formula \( E = mc^2 \) connects energy and mass via the speed of light. To establish a connection to time, we introduce:
	\begin{align}
		E &= mc^2 \\
		E &= h\nu = \frac{h}{T}
	\end{align}
	where \( h \) is the Planck constant, \( \nu \) is the frequency, and \( T \) is the period. Equating these yields:
	\begin{align}
		mc^2 &= \frac{h}{T} \\
		T &= \frac{h}{mc^2}
	\end{align}
	This time \( T \) can be interpreted as a characteristic or intrinsic timescale linked to the mass \( m \).
	
	\section{Connection to the Fine-Structure Constant}
	The fine-structure constant \( \alphaEM \) is a dimensionless physical constant that describes the strength of electromagnetic interaction:
	\begin{equation}
		\alphaEM = \frac{e^2}{4\pi\varepsilon_0\hbar c} \approx \frac{1}{137.035999}
	\end{equation}
	
	\subsection{Derivation via Electromagnetic Constants}
	The Planck constant can be expressed in terms of electromagnetic vacuum constants:
	\begin{equation}
		h = \frac{1}{2\pi\sqrt{\mu_0\varepsilon_0}}
	\end{equation}
	Thus, the intrinsic time \( T \) can be rewritten as:
	\begin{align}
		T &= \frac{h}{mc^2} \\
		&= \frac{1}{2\pi\sqrt{\mu_0\varepsilon_0}} \cdot \frac{1}{mc^2}
	\end{align}
	Since \( c = \frac{1}{\sqrt{\mu_0\varepsilon_0}} \), we obtain:
	\begin{align}
		T &= \frac{1}{2\pi\sqrt{\mu_0\varepsilon_0}} \cdot \frac{1}{m \cdot \frac{1}{\mu_0\varepsilon_0}} \\
		&= \frac{\mu_0\varepsilon_0}{2\pi m c}
	\end{align}
	
	\section{Time in Quantum Mechanics}
	\subsection{Standard Treatment of Time}
	In conventional quantum mechanics, time appears as a parameter in the Schrödinger equation:
	\begin{equation}
		i\hbar \frac{\partial}{\partial t}\Psi(x,t) = \hat{H}\Psi(x,t)
	\end{equation}
	Unlike spatial or momentum coordinates, there is no time operator. Time is treated as a continuous parameter along which quantum states evolve.
	
	\subsection{A New Perspective: Intrinsic Time}
	Consider the characteristic time \( \Tfield = \frac{\hbar}{\max(m c^2, \omega)} \) as the "intrinsic time" of a quantum object, encompassing both massive particles and photons. This time depends on the object’s mass or energy and could be interpreted as the minimal timescale on which the object undergoes quantum mechanical changes. The Schrödinger equation is modified to:
	\begin{equation}
		i\hbar \Tfield \frac{\partial}{\partial t} \Psi + i\hbar \Psi \frac{\partial \Tfield}{\partial t} = \hat{H} \Psi
	\end{equation}
	This implies that time evolution is no longer uniform for all objects but depends on their properties.
	
	\section{Extended Relationships Between Time, Mass, and Fundamental Constants}
	\subsection{A Unified Relation to the Fine-Structure Constant}
	Using \( T = \frac{\hbar}{mc^2} \) for massive particles and extending:
	\begin{align}
		T &= \frac{\hbar}{mc^2} \cdot \frac{4\pi\varepsilon_0\hbar c}{e^2} \cdot \alphaEM
	\end{align}
	This reveals a deep connection between time evolution and electromagnetic interactions.
	
	\subsection{Modified Dispersion Relation}
	In standard quantum mechanics:
	\begin{equation}
		\omega = \frac{\hbar k^2}{2m}
	\end{equation}
	With \( \Tfield = \frac{\hbar}{\max(m c^2, \omega)} \), for massive particles:
	\begin{equation}
		\omega_{\text{eff}} = \frac{\hbar^2 k^2}{2 m^2 c^2}
	\end{equation}
	This differs from the standard \( \omega \propto \frac{1}{m} \).
	
	\subsection{Treatment of Multi-Particle Systems}
	For two particles (\( m_1 \), \( m_2 \)):
	\begin{equation}
		i (m_1 + m_2) c^2 \frac{\partial}{\partial t} \Psi(x_1, x_2, t) = \hat{H} \Psi(x_1, x_2, t)
	\end{equation}
	For entangled states:
	\begin{equation}
		|\Psi(t)\rangle = \frac{1}{\sqrt{2}}(|0(t/T_1)\rangle_{m_1} \otimes |1(t/T_2)\rangle_{m_2} + |1(t/T_1)\rangle_{m_1} \otimes |0(t/T_2)\rangle_{m_2})
	\end{equation}
	where \( T_1 = \frac{\hbar}{m_1 c^2} \), \( T_2 = \frac{\hbar}{m_2 c^2} \).
	
	\subsection{Mass-Dependent Coherence Times and Instantaneity}
	Decoherence rate:
	\begin{equation}
		\Gamma_{\text{dec}} = \Gamma_0 \cdot \frac{m c^2}{\hbar}
	\end{equation}
	Energy-time uncertainty:
	\begin{equation}
		\Delta t \gtrsim \frac{\hbar}{mc^2}
	\end{equation}
	
	\section{Lagrangian Formulation}
	\begin{equation}
		\mathcal{L}_{\text{Total}} = \mathcal{L}_{\text{Boson}} + \mathcal{L}_{\text{Fermion}} + \mathcal{L}_{\text{Higgs-T}} + \mathcal{L}_{\text{intrinsic}}, \quad \mathcal{L}_{\text{intrinsic}} = \frac{1}{2} \partial_\mu \Tfield \partial^\mu \Tfield - V(\Tfield)
	\end{equation}
	
	\section{Implications for Physics}
	\subsection{A New Perspective on Time}
	Time is not derived as fundamental but considered an emergent property.
	\subsection{Connection to Time Dilation}
	The T0 model reflects relativistic effects through mass variation.
	\subsection{Emergent Gravity}
	In the T0 model, gravity emerges as a force from the gradients of the intrinsic time field \(\Tfield\), defined by \( m = \frac{1}{\Tfield} \) in a unified unit system (\(\hbar = c = G = \alphaEM = \betaT = 1\)). The gravitational potential arises as:
	\begin{equation}
		\Phi(r) = -\ln\left(\frac{\Tfield(r)}{\Tzero}\right) \approx \frac{M}{r},
	\end{equation}
	where \(\Tfield(r) = \Tzero \left(1 - \frac{M}{r}\right)\) for a point mass \(M\). The resulting force is:
	\begin{equation}
		\vec{F} = -\grad \Phi \approx -\frac{M}{r^2} \hat{r},
	\end{equation}
	reproducing Newtonian gravity. On larger scales, an additional term \(\kappa r\) may appear, explaining flat rotation curves. A detailed derivation is provided in \cite{pascher_emergente_gravitation_2025, pascher_alphabeta_2025}.
	
	\section{Cosmological Implications in SI Units}
	\begin{itemize}
		\item Redshift: In the T0 model, cosmic redshift \(z\) is determined by the variation of the intrinsic time field \(\Tfield\), with the relation \(1 + z = \frac{\Tfield_0}{\Tfield}\), where \(\Tfield_0\) is the local value of the time field. In SI units, this becomes \(1 + z = e^{\alpha d}\), with \(\alpha \approx 2.3 \times 10^{-28} \, \text{m}^{-1}\), where \(\alpha = H_0/c\) describes the spatial variation rate of \(\Tfield\) \cite{pascher_galaxies_2025, pascher_emergente_gravitation_2025}.
		\item Gravitational Potential: The emergent gravitational potential in the T0 model is \(\Phi(r) = -\frac{GM}{r} + \kappa r\), with \(\kappa = \frac{y v}{r_g}\) \cite{pascher_emergente_gravitation_2025}.
		\item Wavelength Dependence: Redshift exhibits a wavelength-dependent component, described by \(z(\lambda) = z_0 \left(1 + \betaT^{\text{SI}} \ln\left(\frac{\lambda}{\lambda_0}\right)\right)\), with \(\betaT^{\text{SI}} \approx 0.008\). In natural units with \(\betaT^{\text{nat}} = 1\), it becomes \(z(\lambda) = z_0 \left(1 + \ln\left(\frac{\lambda}{\lambda_0}\right)\right)\) \cite{pascher_temp_2025, pascher_alphabeta_2025}.
	\end{itemize}
	
	\section{Role of \(\betaT\)}
	The parameter \(\betaT^{\text{SI}} \approx 0.008\) in SI units serves as a factor in the natural derivation of the wavelength-dependent redshift \( z(\lambda) = z_0 \left(1 + \betaT^{\text{SI}} \ln\left(\frac{\lambda}{\lambda_0}\right)\right) \) in the T0 model \cite{pascher_alphabeta_2025}. In a unified unit system, \(\betaT^{\text{nat}} = 1\) is set, defining the characteristic length scale \( r_0 \approx 1.33 \times 10^{-4} \cdot l_P \) and supporting consistency with \(\alphaEM = 1\).
	
	\section{Experimental Verification Possibilities}
	\begin{itemize}
		\item Differences in Coherence Times: Measurement of temporal deviations in interferometric experiments.
		\item Mass-Dependent Phase Shifts: Investigation of phase differences depending on particle masses.
		\item Spectroscopy Signatures: Detection of wavelength-dependent redshift with high-precision spectroscopy.
		\item Test of Modified Dispersion Relation: \( \omega_{\text{eff}} \propto \frac{1}{m^2} \), verifiable through light propagation experiments.
	\end{itemize}
	
	\section{Effects on Instantaneous Coherence in Quantum Mechanics}
	\subsection{Problem of Instantaneous Coherence}
	Quantum correlations appear instantaneous, challenging causality within standard physics \cite{bell}.
	
	\subsection{Mass-Dependent Coherence Times}
	Heavier particles decohere faster in lab time due to mass variation \( m = \frac{\hbar}{T c^2} \) in the T0 model \cite{pascher_galaxies_2025}.
	
	\subsection{Mathematical Formulation for Multi-Particle Systems}
	Dynamics are described by coupling to \(\Tfield\) (see detailed equations in \cite{pascher_lagrange_2025}).
	
	\subsection{Effects on Entangled States}
	Mass differences influence the coherence of entangled states through the intrinsic time \(\Tfield\).
	
	\subsection{New Interpretation for EPR and Bell}
	Nonlocality could reflect mass-dependent timescales, testable through modified Bell experiments with variable masses \cite{bell}.
	
	\section{Conclusions and Outlook}
	The T0 model views time as emergent and provides a unified, experimentally verifiable framework that connects relativity, quantum mechanics, and the fine-structure constant, with \(\betaT\) playing a central role in deriving cosmological and quantum mechanical effects \cite{pascher_galaxies_2025, pascher_alphabeta_2025}.
	
	\begin{thebibliography}{99}
		\bibitem{pascher1} Pascher, J. (2025). \href{https://github.com/jpascher/T0-Time-Mass-Duality/tree/main/2/pdf/English/KomplementPhysikZeitEn.pdf}{Complementary Extensions of Physics: Absolute Time and Intrinsic Time}. March 24, 2025.
		\bibitem{pascher_zeit_2025} Pascher, J. (2025). \href{https://github.com/jpascher/T0-Time-Mass-Duality/tree/main/2/pdf/English/ZeitEmergentQMEn.pdf}{Time as an Emergent Property in Quantum Mechanics: A Connection Between Relativity, Fine-Structure Constant, and Quantum Dynamics}. March 23, 2025.
		\bibitem{pascher_messdifferenzen_2025} Pascher, J. (2025). \href{https://github.com/jpascher/T0-Time-Mass-Duality/tree/main/2/pdf/English/MessdifferenzenT0StandardEn.pdf}{Compensatory and Additive Effects: An Analysis of Measurement Differences Between the T0 Model and the \(\Lambda\)CDM Standard Model}. April 2, 2025.
		\bibitem{pascher_alpha_2025} Pascher, J. (2025). \href{https://github.com/jpascher/T0-Time-Mass-Duality/tree/main/2/pdf/English/NatEinheitenAlpha1En.pdf}{Energy as a Fundamental Unit: Natural Units with \(\alpha = 1\) in the T0 Model}. March 26, 2025.
		\bibitem{pascher_params_2025} Pascher, J. (2025). \href{https://github.com/jpascher/T0-Time-Mass-Duality/tree/main/2/pdf/English/ZeitMasseT0ParamsEn.pdf}{Time-Mass Duality Theory (T0 Model): Derivation of Parameters \(\kappa\), \(\alpha\), and \(\beta\)}. April 4, 2025.
		\bibitem{pascher_higgs_2025} Pascher, J. (2025). \href{https://github.com/jpascher/T0-Time-Mass-Duality/tree/main/2/pdf/English/MathHiggsZeitMasseEn.pdf}{Mathematical Formulation of the Higgs Mechanism in Time-Mass Duality}. March 28, 2025.
		\bibitem{pascher_lagrange_2025} Pascher, J. (2025). \href{https://github.com/jpascher/T0-Time-Mass-Duality/tree/main/2/pdf/English/MathZeitMasseLagrangeEn.pdf}{From Time Dilation to Mass Variation: Mathematical Core Formulations of Time-Mass Duality Theory}. March 29, 2025.
		\bibitem{pascher_emergente_gravitation_2025} Pascher, J. (2025). \href{https://github.com/jpascher/T0-Time-Mass-Duality/tree/main/2/pdf/English/EmergentGravT0En.pdf}{Emergent Gravity in the T0 Model: A Comprehensive Derivation}. April 1, 2025.
		\bibitem{pascher_galaxies_2025} Pascher, J. (2025). \href{https://github.com/jpascher/T0-Time-Mass-Duality/tree/main/2/pdf/English/MassVarGalaxienEn.pdf}{Mass Variation in Galaxies: An Analysis in the T0 Model with Emergent Gravity}. March 30, 2025.
		\bibitem{pascher_alphabeta_2025} Pascher, J. (2025). \href{https://github.com/jpascher/T0-Time-Mass-Duality/tree/main/2/pdf/English/Alpha1Beta1KonsistenzEn.pdf}{Unified Unit System in the T0 Model: The Consistency of \(\alpha = 1\) and \(\beta = 1\)}. April 5, 2025.
		\bibitem{einstein} Einstein, A. (1905). \textit{On the Electrodynamics of Moving Bodies}. Annalen der Physik, 322(10), 891-921. DOI: 10.1002/andp.19053221004.
		\bibitem{bell} Bell, J. S. (1964). \textit{On the Einstein-Podolsky-Rosen Paradox}. Physics Physique {\fontencoding{T2A}\selectfont Физика}, 1(3), 195-200. DOI: \href{https://doi.org/10.1103/Physics.1.195}{10.1103/Physics.1.195}
		\bibitem{rubin1980} Rubin, V. C., Ford Jr, W. K., \& Thonnard, N. (1980). Rotational properties of 21 SC galaxies with a large range of luminosities and radii. \textit{The Astrophysical Journal}, 238, 471-487. DOI: 10.1086/158003.
		\bibitem{McGaugh2016} McGaugh, S. S., Lelli, F., \& Schombert, J. M. (2016). Radial acceleration relation in rotationally supported galaxies. \textit{Physical Review Letters}, 117(20), 201101. DOI: 10.1103/PhysRevLett.117.201101.
		\bibitem{Milgrom1983} Milgrom, M. (1983). A modification of the Newtonian dynamics. \textit{The Astrophysical Journal}, 270, 365-370. DOI: 10.1086/161130.
		\bibitem{Planck2018} Planck Collaboration (2020). Planck 2018 results. VI. Cosmological parameters. \textit{Astronomy \& Astrophysics}, 641, A6. DOI: 10.1051/0004-6361/201833910.
	\end{thebibliography}
	
\end{document}