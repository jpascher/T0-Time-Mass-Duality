\documentclass[12pt,a4paper]{article}
\usepackage[utf8]{inputenc}
\usepackage[T1]{fontenc}
\usepackage[ngerman]{babel}
\usepackage[left=2cm,right=2cm,top=2cm,bottom=2cm]{geometry}
\usepackage{lmodern}
\usepackage{amsmath}
\usepackage{amssymb}
\usepackage{physics}
\usepackage{hyperref}
\usepackage{tocloft}
\usepackage{tcolorbox}
\usepackage{booktabs}
\usepackage{enumitem}
\usepackage[table,xcdraw]{xcolor}
\usepackage{pgfplots}
\pgfplotsset{compat=1.18}
\usepackage{graphicx}
\usepackage{float}
\usepackage{mathtools}
\usepackage[T2A,T1]{fontenc}
\usepackage{fancyhdr}

\renewcommand{\cftsecfont}{\color{blue}}
\renewcommand{\cftsubsecfont}{\color{blue}}
\renewcommand{\cftsecpagefont}{\color{blue}}
\renewcommand{\cftsubsecpagefont}{\color{blue}}
\setlength{\cftsecindent}{1cm}
\setlength{\cftsubsecindent}{2cm}

\hypersetup{
	colorlinks=true,
	linkcolor=blue,
	citecolor=blue,
	urlcolor=blue,
	pdftitle={Time as an Emergent Property in Quantum Mechanics},
	pdfauthor={Johann Pascher},
	pdfsubject={Theoretical Physics},
	pdfkeywords={T0 Model, Time-Mass Duality, Quantum Mechanics, Fine-Structure Constant}
}

% Custom Commands
\newcommand{\Tfield}{T(x)}
\newcommand{\betaT}{\beta_{\text{T}}}
\newcommand{\alphaEM}{\alpha_{\text{EM}}}
\newcommand{\alphaW}{\alpha_{\text{W}}}
\newcommand{\Mpl}{M_{\text{Pl}}}
\newcommand{\Tzerot}{T_0(\Tfield)}
\newcommand{\Tzero}{T_0}
\newcommand{\vecx}{\vec{x}}
\newcommand{\gammaf}{\gamma_{\text{Lorentz}}}
\newcommand{\DhiggsT}{\Tfield (\partial_\mu + ig A_\mu) \Phi + \Phi \partial_\mu \Tfield}
\newcommand{\DcovT}[1]{\Tfield D_\mu #1 + #1 \partial_\mu \Tfield}
\newcommand{\HiggsLagr}{\mathcal{L}_{\text{Higgs-T}}}

% Headers and Footers
\pagestyle{fancy}
\fancyhf{}
\fancyhead[L]{Johann Pascher}
\fancyhead[R]{Time-Mass Duality}
\fancyfoot[C]{\thepage}
\renewcommand{\headrulewidth}{0.4pt}
\renewcommand{\footrulewidth}{0.4pt}

\title{Time as an Emergent Property in Quantum Mechanics: \\A Connection Between Relativity, Fine-Structure Constant, and Quantum Dynamics}
\author{Johann Pascher}
\date{March 23, 2025}

\begin{document}
	
	\maketitle
	
	\tableofcontents
	\newpage
	
	\section{Introduction}
	In modern physics, time and space are treated differently. While spatial coordinates in quantum mechanics are represented by operators, time primarily appears as a parameter. This asymmetric treatment raises fundamental questions about the nature of time. This work explores the extent to which time can be understood as an emergent property, linked to fundamental constants and the mass of the system under consideration.
	
	This investigation is part of a broader conceptual framework, discussed in detail in the companion paper \href{https://github.com/jpascher/T0-Time-Mass-Duality/tree/main/2/pdf/English/KomplementPhysikZeitEn.pdf}{\textit{Complementary Extensions of Physics: Absolute Time and Intrinsic Time}} \cite{pascher_komplementaer_2025} (March 24, 2025). There, the intrinsic time introduced here is connected with a complementary model of absolute time (the \href{https://github.com/jpascher/T0-Time-Mass-Duality/tree/main/2/pdf/English/ZeitMasseT0ParamsEn.pdf}{T0 Model} \cite{pascher_params_2025}), resulting in a \href{https://github.com/jpascher/T0-Time-Mass-Duality/tree/main/2/pdf/English/MathZeitMasseLagrangeEn.pdf}{time-mass duality} \cite{pascher_lagrange_2025} conceptually similar to the well-known wave-particle duality. The philosophical implications of this duality are further explored in \cite{pascher_zeit_masse_2025}.
	
	\section{Time in Special Relativity}
	Einstein's famous formula \( E = mc^2 \) connects energy and mass via the speed of light \cite{einstein}. To establish a connection to time, we introduce:
	\begin{align}
		E &= mc^2 \\
		E &= h\nu = \frac{h}{T}
	\end{align}
	where \( h \) is the Planck constant, \( \nu \) is the frequency, and \( T \) is the period. Equating these yields:
	\begin{align}
		mc^2 &= \frac{h}{T} \\
		T &= \frac{h}{mc^2}
	\end{align}
	This time \( T \) can be interpreted as a characteristic or intrinsic timescale linked to the mass \( m \). This fundamental relationship forms the basis of the intrinsic time field concept in the T0 model, as elaborated in \cite{pascher_lagrange_2025}.
	
	\section{Connection to the Fine-Structure Constant}
	The fine-structure constant \( \alphaEM \) is a dimensionless physical constant that describes the strength of electromagnetic interaction:
	\begin{equation}
		\alphaEM = \frac{e^2}{4\pi\varepsilon_0\hbar c} \approx \frac{1}{137.035999}
	\end{equation}
	
	\subsection{Derivation via Electromagnetic Constants}
	The Planck constant can be expressed in terms of electromagnetic vacuum constants:
	\begin{equation}
		h = \frac{1}{2\pi\sqrt{\mu_0\varepsilon_0}}
	\end{equation}
	Thus, the intrinsic time \( T \) can be rewritten as:
	\begin{align}
		T &= \frac{h}{mc^2} \\
		&= \frac{1}{2\pi\sqrt{\mu_0\varepsilon_0}} \cdot \frac{1}{mc^2}
	\end{align}
	Since \( c = \frac{1}{\sqrt{\mu_0\varepsilon_0}} \), we obtain:
	\begin{align}
		T &= \frac{1}{2\pi\sqrt{\mu_0\varepsilon_0}} \cdot \frac{1}{m \cdot \frac{1}{\mu_0\varepsilon_0}} \\
		&= \frac{\mu_0\varepsilon_0}{2\pi m c}
	\end{align}
	
	This relationship reveals a profound connection between intrinsic time, electromagnetic constants, and mass, further explored in \cite{pascher_alpha_2025}, where a natural unit system with \(\alphaEM = 1\) is proposed.
	
	\section{Time in Quantum Mechanics}
	\subsection{Standard Treatment of Time}
	In conventional quantum mechanics, time appears as a parameter in the Schrödinger equation:
	\begin{equation}
		i\hbar \frac{\partial}{\partial t}\Psi(x,t) = \hat{H}\Psi(x,t)
	\end{equation}
	Unlike spatial or momentum coordinates, there is no time operator. Time is treated as a continuous parameter along which quantum states evolve. This asymmetric treatment of time versus space has been a long-standing issue in quantum theory \cite{pascher_erweiterung_2025}.
	
	\subsection{A New Perspective: Intrinsic Time}
	Consider the characteristic time \( \Tfield = \frac{\hbar}{\max(m c^2, \omega)} \) as the "intrinsic time" of a quantum object, encompassing both massive particles and photons. This time depends on the object's mass or energy and could be interpreted as the minimal timescale on which the object undergoes quantum mechanical changes. The Schrödinger equation is modified to:
	\begin{equation}
		i\hbar \Tfield \frac{\partial}{\partial t} \Psi + i\hbar \Psi \frac{\partial \Tfield}{\partial t} = \hat{H} \Psi
	\end{equation}
	This implies that time evolution is no longer uniform for all objects but depends on their properties. For photons, the intrinsic time is energy-dependent, with important consequences for nonlocality as detailed in \cite{pascher_photons_2025}.
	
	\section{Extended Relationships Between Time, Mass, and Fundamental Constants}
	\subsection{A Unified Relation to the Fine-Structure Constant}
	Using \( T = \frac{\hbar}{mc^2} \) for massive particles and extending:
	\begin{align}
		T &= \frac{\hbar}{mc^2} \cdot \frac{4\pi\varepsilon_0\hbar c}{e^2} \cdot \alphaEM
	\end{align}
	This reveals a deep connection between time evolution and electromagnetic interactions. When \(\alphaEM = 1\) is set in natural units, this relationship becomes particularly elegant, as discussed in \cite{pascher_alphabeta_2025}.
	
	\subsection{Modified Dispersion Relation}
	In standard quantum mechanics:
	\begin{equation}
		\omega = \frac{\hbar k^2}{2m}
	\end{equation}
	With \( \Tfield = \frac{\hbar}{\max(m c^2, \omega)} \), for massive particles:
	\begin{equation}
		\omega_{\text{eff}} = \frac{\hbar^2 k^2}{2 m^2 c^2}
	\end{equation}
	This differs from the standard \( \omega \propto \frac{1}{m} \). The modified dispersion relation could have detectable consequences for quantum phenomena at high energies or on cosmological scales, as explored in \cite{pascher_emergente_gravitation_2025}.
	
	\subsection{Treatment of Multi-Particle Systems}
	For two particles (\( m_1 \), \( m_2 \)):
	\begin{equation}
		i (m_1 + m_2) c^2 \frac{\partial}{\partial t} \Psi(x_1, x_2, t) = \hat{H} \Psi(x_1, x_2, t)
	\end{equation}
	For entangled states:
	\begin{equation}
		|\Psi(t)\rangle = \frac{1}{\sqrt{2}}(|0(t/T_1)\rangle_{m_1} \otimes |1(t/T_2)\rangle_{m_2} + |1(t/T_1)\rangle_{m_1} \otimes |0(t/T_2)\rangle_{m_2})
	\end{equation}
	where \( T_1 = \frac{\hbar}{m_1 c^2} \), \( T_2 = \frac{\hbar}{m_2 c^2} \). This formalism provides new insights into quantum entanglement and nonlocality, as detailed in \cite{pascher_feldtheorie_2025}.
	
	\subsection{Mass-Dependent Coherence Times and Instantaneity}
	Decoherence rate:
	\begin{equation}
		\Gamma_{\text{dec}} = \Gamma_0 \cdot \frac{m c^2}{\hbar}
	\end{equation}
	Energy-time uncertainty:
	\begin{equation}
		\Delta t \gtrsim \frac{\hbar}{mc^2}
	\end{equation}
	
	These relationships suggest experimentally testable predictions regarding mass-dependent quantum coherence, offering potential insights into the quantum-to-classical transition \cite{pascher_erweiterung_2025}.
	
\section{Lagrangian Formulation}
The complete mathematical structure of the T0 model is captured by the total Lagrangian density:
\begin{equation}
	\mathcal{L}_{\text{Total}} = \mathcal{L}_{\text{Boson}} + \mathcal{L}_{\text{Fermion}} + \mathcal{L}_{\text{Higgs-T}} + \mathcal{L}_{\text{intrinsic}}
\end{equation}

The intrinsic time field Lagrangian in its complete form combines both the free field dynamics and matter interactions:

\begin{equation}
	\mathcal{L}_{\text{intrinsic}}^{\text{complete}} = \underbrace{\frac{1}{2} \partial_\mu \Tfield \partial^\mu \Tfield - \frac{1}{2}\Tfield^2}_{\text{Free field dynamics}} + \underbrace{\bar{\psi} \left( i\hbar \gamma^0 \frac{\partial}{\partial (t/\Tfield)} - i\hbar \gamma^0 \frac{\partial}{\partial t} \right) \psi}_{\text{Interaction with matter}}
\end{equation}

In applications focusing on field interactions with matter distributions, this can be reformulated as:
\begin{equation}
	\mathcal{L}_{\text{intrinsic}}^{\text{matter}} = \frac{1}{2} \partial_\mu \Tfield \partial^\mu \Tfield - \frac{1}{2}\Tfield^2 - \frac{\rho}{\Tfield}
\end{equation}
where the last term represents the coupling to matter with density $\rho$ having dimension $[E^2]$ in natural units.

In many contexts, depending on the specific focus of the analysis, either the free field component or the matter interaction component may be emphasized. For the study of field propagation and potential energy, the free field component with $V(\Tfield) = \frac{1}{2}\Tfield^2$ is commonly used.

This formulation provides a unified description of all fundamental interactions, including the emergent gravitation through the intrinsic time field \(\Tfield\). \(\Tfield\). The Higgs mechanism plays a special role in this theory as a mediator between the standard picture and the T0 perspective, as elaborated in \cite{pascher_higgs_2025}.
\section{Implications for Physics}
	\subsection{A New Perspective on Time}
	Time is not derived as fundamental but considered an emergent property linked to mass through the intrinsic time field \(\Tfield\). This perspective offers a radical reimagining of temporal phenomena in physics \cite{pascher_zeit_masse_2025}.
	
	\subsection{Connection to Time Dilation}
	The T0 model reflects relativistic effects through mass variation rather than time dilation. While mathematically equivalent to special relativity, this approach provides new conceptual insights and potential experimental distinctions \cite{pascher_messdifferenzen_2025}.
	
	\subsection{Emergent Gravitation}
	In the T0 model, gravitation emerges as a force from the gradients of the intrinsic time field \(\Tfield\), defined by \( m = \frac{\hbar}{\Tfield c^2} \) in a unified unit system (\(\hbar = c = G = \alphaEM = \betaT = 1\)). The gravitational potential arises as:
	\begin{equation}
		\Phi(r) = -\frac{G M}{r} + \kappa r,
	\end{equation}
	where \(\kappa\) has dimension \([E]\) in natural units. For a point mass \(M\) at short distances, the potential approximates to \(\Phi(r) \approx -\frac{GM}{r}\). The resulting force is:
	\begin{equation}
		\vec{F} = -\nabla \Phi \approx -\frac{M}{r^2} \hat{r},
	\end{equation}
	reproducing Newtonian gravitation. On larger scales, the term \(\kappa r\) explains flat rotation curves without requiring dark matter, as observed in galaxies \cite{rubin1980, McGaugh2016}. A detailed derivation is provided in \cite{pascher_emergente_gravitation_2025, pascher_alphabeta_2025}.
	
	\section{Cosmological Implications in SI Units}
	\begin{itemize}
		\item \textbf{Redshift:} In the T0 model, cosmic redshift \(z\) is determined by the variation of the intrinsic time field \(\Tfield\), with the relation \(1 + z = \frac{\Tfield_0}{\Tfield}\), where \(\Tfield_0\) is the local value of the time field. In SI units, this becomes \(1 + z = e^{\alpha^{\text{SI}} d}\), with \(\alpha^{\text{SI}} \approx 2.3 \times 10^{-18} \, \text{m}^{-1}\), where \(\alpha = H_0/c\) describes the spatial variation rate of \(\Tfield\) \cite{pascher_galaxies_2025, pascher_emergente_gravitation_2025}. This approach provides an alternative explanation to cosmic expansion for the observed redshift of distant galaxies.
		
		\item \textbf{Gravitational Potential:} The emergent gravitational potential in the T0 model is \(\Phi(r) = -\frac{GM}{r} + \kappa r\), with \(\kappa^{\text{nat}} = \betaT^{\text{nat}} \cdot \frac{yv}{r_g^2}\frac{y v}{r_g^2}\) in natural units, where \(\kappa\) has dimension \([E]\) \cite{pascher_emergente_gravitation_2025}. This modified potential successfully explains galaxy rotation curves without dark matter, offering a more parsimonious alternative to MOND \cite{Milgrom1983} and \(\Lambda\)CDM models \cite{Planck2018}.
		
		\item \textbf{Wavelength Dependence:} Redshift exhibits a wavelength-dependent component, described by \(z(\lambda) = z_0 \left(1 + \betaT^{\text{SI}} \ln\left(\frac{\lambda}{\lambda_0}\right)\right)\), with \(\betaT^{\text{SI}} \approx 0.008\). In natural units with \(\betaT^{\text{nat}} = 1\), it becomes \(z(\lambda) = z_0 \left(1 + \ln\left(\frac{\lambda}{\lambda_0}\right)\right)\) \cite{pascher_temp_2025, pascher_alphabeta_2025}. This distinctive prediction offers a clear experimental test to distinguish the T0 model from standard cosmology.
	\end{itemize}
	
	\section{Role of \(\betaT\)}
	The parameter \(\betaT^{\text{SI}} \approx 0.008\) in SI units serves as a factor in the natural derivation of the wavelength-dependent redshift \( z(\lambda) = z_0 \left(1 + \betaT^{\text{SI}} \ln\left(\frac{\lambda}{\lambda_0}\right)\right) \) in the T0 model \cite{pascher_alphabeta_2025}. In a unified unit system, \(\betaT^{\text{nat}} = 1\) is set, defining the characteristic length scale \( r_0 \approx 1.33 \times 10^{-4} \cdot l_P \) and supporting consistency with \(\alphaEM = 1\). The precise derivation of this parameter and its relationship to the Higgs mechanism is detailed in \cite{pascher_params_2025, pascher_higgs_2025}.
	
	\section{Experimental Verification Possibilities}
	\begin{itemize}
		\item \textbf{Differences in Coherence Times:} Measurement of temporal deviations in interferometric experiments with particles of different masses could reveal the mass-dependent nature of quantum evolution \cite{pascher_erweiterung_2025}.
		
		\item \textbf{Mass-Dependent Phase Shifts:} Investigation of phase differences depending on particle masses in quantum interference experiments could provide direct evidence for the intrinsic time field concept \cite{pascher_feldtheorie_2025}.
		
		\item \textbf{Spectroscopy Signatures:} Detection of wavelength-dependent redshift with high-precision spectroscopy using instruments like the James Webb Space Telescope would offer a definitive test of the T0 model's cosmological predictions \cite{pascher_messdifferenzen_2025}.
		
		\item \textbf{Test of Modified Dispersion Relation:} \( \omega_{\text{eff}} \propto \frac{1}{m^2} \), verifiable through light propagation experiments at different energies, could distinguish the T0 model from standard quantum theory \cite{pascher_photons_2025}.
	\end{itemize}
	
	\section{Effects on Instantaneous Coherence in Quantum Mechanics}
	\subsection{Problem of Instantaneous Coherence}
	Quantum correlations appear instantaneous, challenging causality within standard physics \cite{bell}. The T0 model offers a new perspective on this long-standing problem.
	
	\subsection{Mass-Dependent Coherence Times}
	Heavier particles decohere faster in lab time due to mass variation \( m = \frac{\hbar}{T c^2} \) in the T0 model \cite{pascher_galaxies_2025, pascher_feldtheorie_2025}. This mass dependence could explain apparent nonlocality without requiring instantaneous action at a distance.
	
	\subsection{Mathematical Formulation for Multi-Particle Systems}
	Dynamics are described by coupling to \(\Tfield\) through the modified Schrödinger equation and Lagrangian density (see detailed equations in \cite{pascher_lagrange_2025}). This formalism provides a coherent description of quantum entanglement and its temporal aspects.
	
	\subsection{Effects on Entangled States}
	Mass differences influence the coherence of entangled states through the intrinsic time \(\Tfield\). The T0 model predicts that entangled particles with different masses would exhibit distinctive temporal behaviors that could be experimentally tested \cite{pascher_feldtheorie_2025}.
	
	\subsection{New Interpretation for EPR and Bell}
	Nonlocality could reflect mass-dependent timescales, testable through modified Bell experiments with variable masses \cite{bell, pascher_photons_2025}. This approach offers a potential resolution to the tensions between quantum mechanics and relativity without requiring hidden variables or faster-than-light signaling.
	
	\section{Conclusions and Outlook}
	The T0 model views time as emergent and provides a unified, experimentally verifiable framework that connects relativity, quantum mechanics, and the fine-structure constant, with \(\betaT\) playing a central role in deriving cosmological and quantum mechanical effects \cite{pascher_galaxies_2025, pascher_alphabeta_2025}.
	
	Future work will focus on developing more precise experimental tests, refining the mathematical formalism, and exploring the implications for quantum gravity and unification theory. The comprehensive integration of fundamental forces within the T0 framework, as outlined in \cite{pascher_grundkraefte_2025}, points toward a potentially transformative approach to theoretical physics that merits further investigation.
	
	The T0 model's parsimonious explanation of phenomena traditionally attributed to dark matter and dark energy, combined with its novel predictions regarding wavelength-dependent redshift and mass-dependent quantum evolution, offers a compelling alternative to standard cosmological and quantum theories that deserves serious consideration by the scientific community.
	
	\begin{thebibliography}{99}
		\bibitem{pascher_komplementaer_2025} Pascher, J. (2025). \href{https://github.com/jpascher/T0-Time-Mass-Duality/tree/main/2/pdf/English/KomplementPhysikZeitEn.pdf}{Complementary Extensions of Physics: Absolute Time and Intrinsic Time}. March 24, 2025.
		\bibitem{pascher_zeit_masse_2025} Pascher, J. (2025). \href{https://github.com/jpascher/T0-Time-Mass-Duality/tree/main/2/pdf/English/ZeitMasseNeuerBlickEn.pdf}{Time and Mass: A New Look at Old Formulas – and Liberation from Traditional Constraints}. March 22, 2025.
		\bibitem{pascher_messdifferenzen_2025} Pascher, J. (2025). \href{https://github.com/jpascher/T0-Time-Mass-Duality/tree/main/2/pdf/English/MessdifferenzenT0StandardEn.pdf}{Compensatory and Additive Effects: An Analysis of Measurement Differences Between the T0 Model and the \(\Lambda\)CDM Standard Model}. April 2, 2025.
		\bibitem{pascher_alpha_2025} Pascher, J. (2025). \href{https://github.com/jpascher/T0-Time-Mass-Duality/tree/main/2/pdf/English/NatEinheitenAlpha1En.pdf}{Energy as a Fundamental Unit: Natural Units with \(\alphaEM = 1\) in the T0 Model}. March 26, 2025.
		\bibitem{pascher_params_2025} Pascher, J. (2025). \href{https://github.com/jpascher/T0-Time-Mass-Duality/tree/main/2/pdf/English/ZeitMasseT0ParamsEn.pdf}{Time-Mass Duality Theory (T0 Model): Derivation of Parameters \(\kappa\), \(\alpha\), and \(\beta\)}. April 4, 2025.
		\bibitem{pascher_higgs_2025} Pascher, J. (2025). \href{https://github.com/jpascher/T0-Time-Mass-Duality/tree/main/2/pdf/English/MathHiggsZeitMasseEn.pdf}{Mathematical Formulation of the Higgs Mechanism in Time-Mass Duality}. March 28, 2025.
		\bibitem{pascher_lagrange_2025} Pascher, J. (2025). \href{https://github.com/jpascher/T0-Time-Mass-Duality/tree/main/2/pdf/English/MathZeitMasseLagrangeEn.pdf}{From Time Dilation to Mass Variation: Mathematical Core Formulations of Time-Mass Duality Theory}. March 29, 2025.
		\bibitem{pascher_emergente_gravitation_2025} Pascher, J. (2025). \href{https://github.com/jpascher/T0-Time-Mass-Duality/tree/main/2/pdf/English/EmergentGravT0En.pdf}{Emergent Gravitation in the T0 Model: A Comprehensive Derivation}. April 1, 2025.
		\bibitem{pascher_galaxies_2025} Pascher, J. (2025). \href{https://github.com/jpascher/T0-Time-Mass-Duality/tree/main/2/pdf/English/MassVarGalaxienEn.pdf}{Mass Variation in Galaxies: An Analysis in the T0 Model with Emergent Gravitation}. March 30, 2025.
		\bibitem{pascher_alphabeta_2025} Pascher, J. (2025). \href{https://github.com/jpascher/T0-Time-Mass-Duality/tree/main/2/pdf/English/Alpha1Beta1KonsistenzEn.pdf}{Unified Unit System in the T0 Model: The Consistency of \(\alphaEM = 1\) and \(\betaT = 1\)}. April 5, 2025.
		\bibitem{pascher_temp_2025} Pascher, J. (2025). \href{https://github.com/jpascher/T0-Time-Mass-Duality/tree/main/2/pdf/English/TempEinheitenCMBEn.pdf}{Adjustment of Temperature Units in Natural Units and CMB Measurements}. April 2, 2025.
		\bibitem{pascher_erweiterung_2025} Pascher, J. (2025). \href{https://github.com/jpascher/T0-Time-Mass-Duality/tree/main/2/pdf/English/NotwendigkeitQMErweiterungEn.pdf}{The Necessity of Extending Standard Quantum Mechanics and Quantum Field Theory}. March 27, 2025.
		\bibitem{pascher_feldtheorie_2025} Pascher, J. (2025). \href{https://github.com/jpascher/T0-Time-Mass-Duality/tree/main/2/pdf/English/FeldtheorieQuantenEn.pdf}{Field Theory and Quantum Correlations: A New Perspective on Instantaneity}. March 28, 2025.
		\bibitem{pascher_photons_2025} Pascher, J. (2025). \href{https://github.com/jpascher/T0-Time-Mass-Duality/tree/main/2/pdf/English/DynMassePhotonenNichtlokalEn.pdf}{Dynamic Mass of Photons and Its Implications for Nonlocality in the T0 Model}. March 25, 2025.
		\bibitem{pascher_grundkraefte_2025} Pascher, J. (2025). \href{https://github.com/jpascher/T0-Time-Mass-Duality/tree/main/2/pdf/English/VierKraefteZeitMasseEn.pdf}{Simplified Description of Fundamental Forces with Time-Mass Duality}. March 27, 2025.
		\bibitem{pascher_planck_2025} Pascher, J. (2025). \href{https://github.com/jpascher/T0-Time-Mass-Duality/tree/main/2/pdf/English/JenseitsPlanckEn.pdf}{Real Consequences of Reformulating Time and Mass in Physics: Beyond the Planck Scale}. March 24, 2025.
		\bibitem{einstein} Einstein, A. (1905). \textit{On the Electrodynamics of Moving Bodies}. Annalen der Physik, 322(10), 891-921. DOI: 10.1002/andp.19053221004.
		\bibitem{bell} Bell, J. S. (1964). \textit{On the Einstein-Podolsky-Rosen Paradox}. Physics Physique {\fontencoding{T2A}\selectfont Физика}, 1(3), 195-200. DOI: \href{https://doi.org/10.1103/Physics.1.195}{10.1103/Physics.1.195}
		\bibitem{rubin1980} Rubin, V. C., Ford Jr, W. K., \& Thonnard, N. (1980). Rotational properties of 21 SC galaxies with a large range of luminosities and radii. \textit{The Astrophysical Journal}, 238, 471-487. DOI: 10.1086/158003.
		\bibitem{McGaugh2016} McGaugh, S. S., Lelli, F., \& Schombert, J. M. (2016). Radial acceleration relation in rotationally supported galaxies. \textit{Physical Review Letters}, 117(20), 201101. DOI: 10.1103/PhysRevLett.117.201101.
		\bibitem{Milgrom1983} Milgrom, M. (1983). A modification of the Newtonian dynamics. \textit{The Astrophysical Journal}, 270, 365-370. DOI: 10.1086/161130.
		\bibitem{Planck2018} Planck Collaboration (2020). Planck 2018 results. VI. Cosmological parameters. \textit{Astronomy \& Astrophysics}, 641, A6. DOI: 10.1051/0004-6361/201833910.
		\bibitem{planck1899} Planck, M. (1899). On Irreversible Radiation Processes. \textit{Proceedings of the Prussian Academy of Sciences}, 5, 440-480.
		\bibitem{higgs1964} Higgs, P. W. (1964). Broken Symmetries and the Masses of Gauge Bosons. \textit{Physical Review Letters}, 13(16), 508-509. DOI: 10.1103/PhysRevLett.13.508.
		\bibitem{dirac1928} Dirac, P. A. M. (1928). The Quantum Theory of the Electron. \textit{Proceedings of the Royal Society A}, 117(778), 610-624. DOI: 10.1098/rspa.1928.0023.
	\end{thebibliography}
	
\end{document}