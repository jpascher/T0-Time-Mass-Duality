\documentclass[12pt,a4paper]{article}
\usepackage[margin=2cm]{geometry}
\usepackage[utf8]{inputenc}
\usepackage[T1]{fontenc}
\usepackage{lmodern}
\usepackage[ngerman]{babel}
\usepackage{amsmath,amssymb,physics,graphicx,xcolor,amsthm}
\usepackage{hyperref}
\usepackage{booktabs}
\usepackage{siunitx}
\usepackage{cleveref}
\usepackage{pgfplots}
\pgfplotsset{compat=1.18}
\usepackage{tikz}
\usetikzlibrary{intersections}
\usepgfplotslibrary{fillbetween}
\usepackage{fancyhdr}

% Custom commands
\newcommand{\Tfield}{T(x)}
\newcommand{\betaT}{\beta_{\text{T}}}
\newcommand{\alphaEM}{\alpha_{\text{EM}}}
\newcommand{\Tzero}{T_0}
\newcommand{\DcovT}[1]{\partial_\mu #1 + #1 \partial_\mu \Tfield}
\newcommand{\DhiggsT}{\Tfield (\partial_\mu + ig A_\mu) \Phi + \Phi \partial_\mu \Tfield}
\newcommand{\gammaf}{\gamma_{\text{Lorentz}}}

% Theorem styles
\newtheorem{theorem}{Theorem}[section]
\newtheorem{proposition}[theorem]{Proposition}
\newtheorem{corollary}[theorem]{Corollary}
\newtheorem{lemma}[theorem]{Lemma}
\theoremstyle{definition}
\newtheorem{definition}[theorem]{Definition}
\newtheorem{example}[theorem]{Example}
\theoremstyle{remark}
\newtheorem{remark}[theorem]{Remark}

% Hyperref configuration
\hypersetup{
	colorlinks=true,
	linkcolor=blue,
	urlcolor=blue,
	citecolor=blue,
	pdftitle={From Time Dilation to Mass Variation: Mathematical Core Formulations of Time-Mass Duality Theory},
	pdfauthor={Johann Pascher},
	pdfsubject={Theoretical Physics},
	pdfkeywords={T0 Model, Time-Mass Duality, Emergent Gravitation}
}

% Header and Footer Configuration
\pagestyle{fancy}
\fancyhf{}
\fancyhead[L]{Johann Pascher}
\fancyhead[R]{From Time Dilation to Mass Variation}
\fancyfoot[C]{\thepage}
\renewcommand{\headrulewidth}{0.4pt}
\renewcommand{\footrulewidth}{0.4pt}

\title{From Time Dilation to Mass Variation: \\ Mathematical Core Formulations of Time-Mass Duality Theory}
\author{Johann Pascher}
\date{March 29, 2025}

\begin{document}
	
	\maketitle
	
	\begin{abstract}
		This work presents the essential mathematical formulations of time-mass duality theory, focusing on the fundamental equations and their physical interpretations. The theory establishes a duality between two complementary descriptions of reality: the standard view with time dilation and constant rest mass, and the T0 model with absolute time and variable mass. Central to this framework is the intrinsic time \( \Tfield = \frac{\hbar}{\max(m c^2, \omega)} \), which enables a unified treatment of massive particles and photons. The mathematical formulations include modified Lagrangian densities that emphasize emergent gravitation and energy-loss redshift in a static universe. Building on the conceptual foundation presented in \cite{pascher_zeit_masse_2025}, this paper provides the rigorous mathematical structure required for applications to specific physical scenarios.
	\end{abstract}
	
	\tableofcontents
	\newpage
	
	\section{Introduction to Time-Mass Duality}
	The time-mass duality theory proposes an alternative framework to the conventional relativistic perspective. While both frameworks are mathematically equivalent, they offer complementary insights into physical phenomena:
	\begin{enumerate}
		\item Standard View (Relativistic): \( t' = \gammaf t \), \( m_0 = \text{const.} \)
		\item T0 Model: \( \Tzero = \text{const.} \), \( m = \gammaf m_0 \)
	\end{enumerate}
	
	This duality is analogous to the wave-particle duality in quantum mechanics, as discussed in \cite{pascher_planck_2025}, providing complementary perspectives rather than contradictory descriptions of the same reality. The concept was first introduced in \cite{pascher_zeit_masse_2025} and further elaborated in \cite{pascher_zeit_2025}.
	
	\subsection{Relationship to the Standard Model}
	The T0 model extends the Standard Model with the following key components:
	\begin{enumerate}
		\item Intrinsic Time Field: \( \Tfield = \frac{\hbar}{\max(m c^2, \omega)} \)
		\item Higgs Field: \( \Phi \) with dynamic mass coupling, as detailed in \cite{pascher_higgs_2025}
		\item Fermion Fields: \( \psi \) with Yukawa coupling
		\item Gauge Boson Fields: \( A_\mu \) with \( \Tfield \) interaction
	\end{enumerate}
	
	This extension allows for a unified treatment of massive particles and photons, addressing some of the fundamental challenges in current physics, as discussed in \cite{pascher_erweiterung_2025}.
	
	\section{Emergent Gravitation from the Intrinsic Time Field}
	One of the most significant implications of the T0 model is that gravitation emerges naturally from the dynamics of the intrinsic time field, eliminating the need for a separate fundamental force.
	
	\begin{theorem}[Emergence of Gravitation]
		Gravitation arises from gradients of the intrinsic time field:
		\begin{equation}
			\nabla \Tfield = -\frac{\hbar}{m^2 c^2} \nabla m
		\end{equation}
		with the modified potential:
		\begin{equation}
			\Phi(r) = -\frac{GM}{r} + \kappa r, \quad \kappa^{\text{SI}} \approx 4.8 \times 10^{-11} \, \text{m/s}^2
		\end{equation}
		
		This formulation is detailed in \cite{pascher_emergente_gravitation_2025} and has been applied to galactic dynamics in \cite{pascher_galaxies_2025}.
	\end{theorem}
	
	\begin{proof}
		From \( \Tfield = \frac{\hbar}{m c^2} \) for massive particles, we have:
		\begin{equation}
			\nabla \Tfield = -\frac{\hbar}{m^2 c^2} \nabla m
		\end{equation}
		With the mass variation in a gravitational field given by \( m(\vec{r}) = m_0 (1 + \frac{\Phi_g}{c^2}) \), we can write:
		\begin{equation}
			\nabla m = \frac{m_0}{c^2} \nabla \Phi_g
		\end{equation}
		Thus:
		\begin{equation}
			\nabla \Tfield \approx -\frac{\hbar}{m_0 c^4} \nabla \Phi_g
		\end{equation}
		
		The full derivation with all intermediate steps is provided in \cite{pascher_emergente_gravitation_2025}.
	\end{proof}
	
	\section{Mathematical Foundations: Intrinsic Time}
	The concept of intrinsic time is central to the T0 model and serves as the fundamental bridge between quantum mechanics and relativity.
	
	\begin{theorem}[Intrinsic Time]
		The intrinsic time for a particle with mass \(m\) or a photon with energy \(\hbar\omega\) is defined as:
		\begin{equation}
			\Tfield = \frac{\hbar}{\max(m c^2, \omega)}
		\end{equation}
		
		In natural units where \(\hbar = c = 1\), this simplifies to \(\Tfield = \frac{1}{m}\) for massive particles and \(\Tfield = \frac{1}{\omega}\) for photons, as shown in \cite{pascher_alpha_2025}.
	\end{theorem}
	
	This formulation allows for a unified treatment of both massive particles and photons, addressing one of the key challenges in quantum field theory, as discussed in \cite{pascher_photons_2025}.
	
	\section{Modified Derivative Operators}
	To incorporate the intrinsic time field into the standard formalism of quantum field theory, we need to modify the conventional derivative operators.
	
	\begin{definition}[Modified Derivative]
		The modified covariant derivative in the T0 model is defined as:
		\begin{equation}
			\DcovT{\Psi} = \partial_\mu \Psi + \Psi \partial_\mu \Tfield
		\end{equation}
		where \(\Psi\) is any quantum field and \(\Tfield\) is the intrinsic time field.
		
		This modification is a central component of the T0 model's mathematical framework, as detailed in \cite{pascher_feldtheorie_2025}.
	\end{definition}
	
	\section{Modified Field Equations}
	The standard equations of quantum mechanics must be reformulated to incorporate the intrinsic time field. The most fundamental of these is the Schrödinger equation.
	
	\begin{theorem}[Modified Schrödinger Equation]
		In the T0 model, the Schrödinger equation takes the form:
		\begin{equation}
			i\hbar \Tfield \frac{\partial}{\partial t} \Psi + i\hbar \Psi \frac{\partial \Tfield}{\partial t} = \hat{H} \Psi
		\end{equation}
		
		This equation reflects the coupling between the quantum wavefunction and the intrinsic time field, as discussed in \cite{pascher_zeit_2025}.
	\end{theorem}
	
	\section{Modified Lagrangian Density for the Higgs Field}
	The Higgs mechanism plays a central role in the T0 model, serving as the bridge between the intrinsic time field and particle masses.
	
	\begin{theorem}[Higgs Lagrangian Density]
		The Lagrangian density of the Higgs field with coupling to \(\Tfield\) is:
		\begin{multline}
			\mathcal{L}_{\text{Higgs-T}} = |\DhiggsT|^2 + \frac{1}{2} \partial_\mu \Tfield \partial^\mu \Tfield - V(\Tfield, \Phi), \quad \\
			\DhiggsT = \Tfield (\partial_\mu + ig A_\mu) \Phi + \Phi \partial_\mu \Tfield
		\end{multline}
		
		The detailed derivation and implications of this Lagrangian density are presented in \cite{pascher_higgs_2025}.
	\end{theorem}
	
	\section{Modified Lagrangian Density for Fermions}
	The fermion fields must also be reformulated to incorporate their coupling to the intrinsic time field.
	
	\begin{theorem}[Fermion Lagrangian Density]
		The Lagrangian density for fermions in the T0 model is:
		\begin{equation}
			\mathcal{L}_{\text{Fermion}} = \bar{\psi} i \gamma^\mu (\partial_\mu \psi + \psi \partial_\mu \Tfield) - y \bar{\psi} \Phi \psi
		\end{equation}
		
		This formulation shows how fermion masses arise from the Yukawa coupling to the Higgs field while also interacting with the intrinsic time field, as detailed in \cite{pascher_higgs_2025}.
	\end{theorem}
	
	\section{Modified Lagrangian Density for Gauge Bosons}
	The gauge fields in the T0 model also require reformulation to account for their interaction with the intrinsic time field.
	
	\begin{theorem}[Gauge Boson Lagrangian Density]
		The Lagrangian density for gauge bosons in the T0 model is:
		\begin{equation}
			\mathcal{L}_{\text{Boson}} = -\frac{1}{4} F_{\mu\nu} F^{\mu\nu} + \frac{1}{2} \partial_\mu \Tfield \partial^\mu \Tfield
		\end{equation}
		
		This formulation maintains gauge invariance while incorporating the dynamics of the intrinsic time field, as discussed in \cite{pascher_feldtheorie_2025}.
	\end{theorem}
	
	\section{Complete Total Lagrangian Density}
	Combining all the components, we can formulate the complete Lagrangian 
	%------	
\begin{theorem}[Total Lagrangian Density]
	The total Lagrangian Density is:
	\begin{equation}
		\mathcal{L}_{\text{Total}} = \mathcal{L}_{\text{Boson}} + \mathcal{L}_{\text{Fermion}} + \mathcal{L}_{\text{Higgs-T}} + \mathcal{L}_{\text{intrinsic}},
	\end{equation}
	where the complete intrinsic Lagrangian combines free field dynamics and matter interactions:
	\begin{equation}
		\mathcal{L}_{\text{intrinsic}}^{\text{complete}} = \underbrace{\frac{1}{2} \partial_\mu \Tfield \partial^\mu \Tfield - \frac{1}{2}\Tfield^2}_{\text{Free field dynamics}} + \underbrace{\bar{\psi} \left( i\hbar \gamma^0 \frac{\partial}{\partial (t/\Tfield)} - i\hbar \gamma^0 \frac{\partial}{\partial t} \right) \psi}_{\text{Interaction with matter}}
	\end{equation}
	
	In applications focusing solely on field propagation, the free field component is often used:
	\begin{equation}
		\mathcal{L}_{\text{intrinsic}}^{\text{field}} = \frac{1}{2} \partial_\mu \Tfield \partial^\mu \Tfield - \frac{1}{2} \Tfield^2.
	\end{equation}
\end{theorem}
	%------
	\section{Cosmological Implications}
	The T0 model has profound implications for cosmology, offering alternative explanations for several observed phenomena without requiring dark energy or cosmic inflation.
	
	The key cosmological predictions of the T0 model include:
	\begin{itemize}
		\item Modified Gravitational Potential: \( \Phi(r) = -\frac{GM}{r} + \kappa r \), with \( \kappa^{\text{SI}} \approx 4.8 \times 10^{-11} \, \text{m/s}^2 \), as derived in \cite{pascher_emergente_gravitation_2025}
		\item Cosmic Redshift: \( 1 + z = e^{\alpha d} \), with \( \alpha^{\text{SI}} \approx 2.3 \times 10^{-18} \, \text{m}^{-1} \), as explained in \cite{pascher_galaxies_2025}
		\item Wavelength Dependence: \( z(\lambda) = z_0 (1 + \betaT^{\text{SI}} \ln(\lambda/\lambda_0)) \), with \( \betaT^{\text{SI}} \approx 0.008 \) in SI units, as detailed in \cite{pascher_messdifferenzen_2025}
	\end{itemize}
	
	These cosmological implications offer testable predictions that differentiate the T0 model from the standard cosmological model, providing opportunities for experimental verification as discussed in \cite{pascher_messdifferenzen_2025}.
	
	\section{Derivation of \(\betaT\) in the T0 Model}
	The parameter \(\betaT\) plays a crucial role in the T0 model, describing the coupling of the intrinsic time field \(\Tfield\) to physical phenomena such as wavelength-dependent redshift. In the T0 model, \(\betaT\) is precisely derived as:
	\begin{equation}
		\betaT^{\text{nat}} = \frac{\lambda_h^2 v^2}{16\pi^3 m_h^2 \xi}{16\pi^3} \cdot \frac{1}{m_h^2} \cdot \frac{1}{\xi}
	\end{equation}
	where \(\lambda_h\) is the Higgs self-coupling, \(v\) is the Higgs vacuum expectation value, \(m_h\) is the Higgs mass, and \(\xi \approx 1.33 \times 10^{-4}\) is a dimensionless parameter defining the characteristic length scale \(r_0 = \xi \cdot l_P\) (\(l_P\): Planck length).
	
	In natural units, \(\betaT^{\text{nat}} = 1\) holds exactly, representing a theoretical prediction derived directly from the model parameters. With \(\betaT^{\text{nat}} = 1\), we can determine the value of \(\xi\):
	\begin{equation}
		\xi = \frac{\lambda_h^2 v^2}{16\pi^3 m_h^2} \approx 1.33 \times 10^{-4}
	\end{equation}
	
	This value of \(\xi\) relates the characteristic T0-length \(r_0\) to the Planck length by \(r_0 = \xi \cdot l_P\). In SI units, the parameter value converts to \(\betaT^{\text{SI}} \approx 0.008\), as detailed in \cite{pascher_alphabeta_2025} and \cite{pascher_params_2025}.
	
	\section{Conclusion and Outlook}
	The mathematical formulations presented in this paper provide a rigorous foundation for the time-mass duality theory and the T0 model. By reformulating the conventional equations of quantum field theory to incorporate the intrinsic time field \(\Tfield\), we establish a framework that offers new perspectives on fundamental physics and cosmology.
	
	Key achievements of this formulation include:
	\begin{itemize}
		\item A unified treatment of massive particles and photons through the intrinsic time field concept
		\item The emergence of gravitation from the dynamics of the intrinsic time field, eliminating the need for a separate fundamental force
		\item A consistent explanation for cosmological observations without requiring dark energy or cosmic inflation
		\item Testable predictions that differentiate the T0 model from the standard model of physics
	\end{itemize}
	
	Future work will focus on further refining these mathematical formulations and developing specific applications to address current challenges in theoretical physics, as outlined in \cite{pascher_galaxies_2025} and \cite{pascher_feldtheorie_2025}.
	
	\begin{thebibliography}{99}
		\bibitem{pascher_zeit_2025} Pascher, J. (2025). \href{https://github.com/jpascher/T0-Time-Mass-Duality/tree/main/2/pdf/Deutsch/ZeitEmergentQM.pdf}{Zeit als emergente Eigenschaft in der Quantenmechanik: Eine Verbindung zwischen Relativität, Feinstrukturkonstante und Quantendynamik}. 23. März 2025.
		\bibitem{pascher_lagrange_2025} Pascher, J. (2025). \href{https://github.com/jpascher/T0-Time-Mass-Duality/tree/main/2/pdf/Deutsch/MathZeitMasseLagrange.pdf}{Von Zeitdilatation zur Massenvariation: Mathematische Kernformulierungen der Zeit-Masse-Dualitätstheorie}. 29. März 2025.
		\bibitem{pascher_photons_2025} Pascher, J. (2025). \href{https://github.com/jpascher/T0-Time-Mass-Duality/tree/main/2/pdf/Deutsch/DynMassePhotonenNichtlokal.pdf}{Dynamische Masse von Photonen und ihre Implikationen für Nichtlokalität im T0-Modell}. 25. März 2025.
		\bibitem{pascher_erweiterung_2025} Pascher, J. (2025). \href{https://github.com/jpascher/T0-Time-Mass-Duality/tree/main/2/pdf/Deutsch/NotwendigkeitQMErweiterung.pdf}{Die Notwendigkeit der Erweiterung der Standard-Quantenmechanik und Quantenfeldtheorie}. 27. März 2025.
		\bibitem{pascher_galaxies_2025} Pascher, J. (2025). \href{https://github.com/jpascher/T0-Time-Mass-Duality/tree/main/2/pdf/Deutsch/MassVarGalaxien.pdf}{MassenVariation in Galaxien: Eine Analyse im T0-Modell mit emergenter Gravitation}. 30. März 2025.
		\bibitem{pascher_higgs_2025} Pascher, J. (2025). \href{https://github.com/jpascher/T0-Time-Mass-Duality/tree/main/2/pdf/Deutsch/MathHiggsZeitMasse.pdf}{Mathematische Formulierung des Higgs-Mechanismus in der Zeit-Masse-Dualität}. 28. März 2025.
		\bibitem{pascher_feldtheorie_2025} Pascher, J. (2025). \href{https://github.com/jpascher/T0-Time-Mass-Duality/tree/main/2/pdf/Deutsch/FeldtheorieQuanten.pdf}{Feldtheorie und Quantenkorrelationen: Eine neue Perspektive auf Instantaneität}. 28. März 2025.
		\bibitem{pascher_messdifferenzen_2025} Pascher, J. (2025). \href{https://github.com/jpascher/T0-Time-Mass-Duality/tree/main/2/pdf/Deutsch/MessdifferenzenT0Standard.pdf}{Kompensatorische und additive Effekte: Eine Analyse der Messunterschiede zwischen dem T0-Modell und dem \(\Lambda\)CDM-Standardmodell}. 2. April 2025.
		\bibitem{pascher_planck_2025} Pascher, J. (2025). \href{https://github.com/jpascher/T0-Time-Mass-Duality/tree/main/2/pdf/Deutsch/JenseitsPlanck.pdf}{Reale Konsequenzen der Neuformulierung von Zeit und Masse in der Physik: Jenseits der Planck-Skala}. 24. März 2025.
		\bibitem{pascher_alpha_2025} Pascher, J. (2025). \href{https://github.com/jpascher/T0-Time-Mass-Duality/tree/main/2/pdf/Deutsch/NatEinheitenAlpha1.pdf}{Energie als fundamentale Einheit: Natürliche Einheiten mit \(\alphaEM = 1\) im T0-Modell}. 26. März 2025.
		\bibitem{pascher_alphabeta_2025} Pascher, J. (2025). \href{https://github.com/jpascher/T0-Time-Mass-Duality/tree/main/2/pdf/Deutsch/Alpha1Beta1Konsistenz.pdf}{Einheitliches Einheitensystem im T0-Modell: Die Konsistenz von \(\alpha = 1\) und \(\beta = 1\)}. 5. April 2025.
		\bibitem{pascher_temp_2025} Pascher, J. (2025). \href{https://github.com/jpascher/T0-Time-Mass-Duality/tree/main/2/pdf/Deutsch/TempEinheitenCMB.pdf}{Anpassung der Temperatureinheiten in natürlichen Einheiten und CMB-Messungen}. 2. April 2025.
		\bibitem{pascher_params_2025} Pascher, J. (2025). \href{https://github.com/jpascher/T0-Time-Mass-Duality/tree/main/2/pdf/Deutsch/ZeitMasseT0Params.pdf}{Zeit-Masse-Dualitätstheorie (T0-Modell): Ableitung der Parameter \(\kappa\), \(\alpha\) und \(\beta\)}. 4. April 2025.
		\bibitem{pascher_emergente_gravitation_2025} Pascher, J. (2025). \href{https://github.com/jpascher/T0-Time-Mass-Duality/tree/main/2/pdf/Deutsch/EmergentGravT0.pdf}{Emergente Gravitation im T0-Modell: Eine umfassende Ableitung}. 1. April 2025.
		\bibitem{pascher_zeit_masse_2025} Pascher, J. (2025). \href{https://github.com/jpascher/T0-Time-Mass-Duality/tree/main/2/pdf/Deutsch/ZeitMasseNeuerBlick.pdf}{Zeit und Masse: Ein neuer Blick auf alte Formeln – und Befreiung von traditionellen Zwängen}. 22. März 2025.
		\bibitem{einstein1905} Einstein, A. (1905). Zur Elektrodynamik bewegter Körper. \textit{Annalen der Physik}, 322(10), 891-921.
		\bibitem{higgs1964} Higgs, P. W. (1964). Broken Symmetries and the Masses of Gauge Bosons. \textit{Physical Review Letters}, 13(16), 508-509.
		\bibitem{bohr1928} Bohr, N. (1928). The Quantum Postulate and the Recent Development of Atomic Theory. \textit{Nature}, 121(3050), 580-590.
	\end{thebibliography}
	
\end{document}