
\documentclass[twocolumn,aps,prl]{revtex4-2}
\usepackage[utf8]{inputenc}
\usepackage[T1]{fontenc}
\usepackage[english]{babel}
\usepackage{lmodern}
\usepackage{amsmath}
\usepackage{amssymb}
\usepackage{physics}
\usepackage{hyperref}
\usepackage{booktabs}
\usepackage{enumitem}
\usepackage[table,xcdraw]{xcolor}
\usepackage{graphicx}
\usepackage{siunitx}
\usepackage{float} % Für [H] Platzierung

% Custom commands
\newcommand{\Tfield}{T(x)}
\newcommand{\alphaEM}{\alpha_{\text{EM}}}
\newcommand{\alphaW}{\alpha_{\text{W}}}
\newcommand{\betaT}{\beta_{\text{T}}}
\newcommand{\Mpl}{M_{\text{Pl}}}
\newcommand{\Tzerot}{T_0(\Tfield)}
\newcommand{\Tzero}{T_0}
\newcommand{\vecx}{\vec{x}}
\newcommand{\gammaf}{\gamma_{\text{Lorentz}}}
\newcommand{\LCDM}{\Lambda\text{CDM}}
\newcommand{\calL}{\mathcal{L}}
\newcommand{\e}{\mathrm{e}}
\newcommand{\alphaEMSI}{\alpha_{\text{EM,SI}}}

\hypersetup{
	colorlinks=true,
	linkcolor=blue,
	citecolor=blue,
	urlcolor=blue,
	pdftitle={Hierarchical Natural Unit System in the T0 Model},
	pdfauthor={Johann Pascher},
	pdfsubject={Theoretical Physics},
	pdfkeywords={T0 Model, natural units, time-mass duality}
}

\begin{document}
	
	\title{Hierarchical Natural Unit System in the T0 Model: Unifying Physics Through Energy-Based Formulation}
	\author{Johann Pascher}
	\affiliation{Department of Communications Engineering, Höhere Technische Bundeslehranstalt (HTL), Leonding, Austria}
	\email{johann.pascher@gmail.com}
	\date{April 13, 2025}
	
	\begin{abstract}
		This paper presents a comprehensive hierarchical formulation of natural units within the T0 model of time-mass duality, adopting energy as the fundamental unit. By normalizing dimensional constants ($\hbar = c = G = k_B = 1$) and dimensionless coupling constants ($\alpha_{\text{EM}} = \alpha_W = \beta_T = 1$) to unity, we establish a unified framework that integrates quantum, relativistic, and cosmological phenomena. Our compilation details the hierarchy of constants, quantized length scales spanning 97 orders of magnitude from sub-Planckian to cosmic regimes, and the remarkable presence of biological structures in otherwise forbidden zones. Electromagnetic, thermodynamic, and quantum mechanical constants are derived directly from the energy scale, with simplified field equations revealing the intrinsic unity of natural laws. The Einstein-Hilbert action is reinterpreted to underpin emergent gravitation, aligning with modern approaches to quantum gravity while maintaining compatibility with experimental observations. Supported by theoretical derivations and rigorous mathematical formulations, this work advances the unification of physics through the T0 model's energy-based paradigm, offering testable predictions across multiple scales that can be validated against existing cosmological and particle physics data.
	\end{abstract}
	
	\maketitle
	
	\section{Introduction}
	\label{sec:introduction}
	
	Natural units in theoretical physics streamline the description of physical laws by reducing independent dimensions and setting fundamental constants to unity, thereby unveiling the intrinsic simplicity underlying complex phenomena. Traditional systems, such as Planck units where $\hbar = c = G = 1$, have long served as a cornerstone for theoretical explorations, eliminating arbitrary dimensional parameters and focusing on the essence of physical interactions \cite{Planck1899}. This approach has enabled significant advances in quantum gravity research \cite{Rovelli2004, Ashtekar2007} and string theory \cite{Greene1999}. Similarly, particle physicists employ systems where $\hbar = c = 1$ to simplify calculations \cite{Peskin1995}, while Stoney units predate even Planck’s work in attempting universal measurement standards \cite{Stoney1881}.
	
	However, the T0 model of time-mass duality extends these paradigms by proposing a fully unified natural unit system, where not only dimensional constants ($\hbar = c = G = k_B = 1$) but also dimensionless coupling constants—the fine-structure constant $\alpha_{\text{EM}}$, Wien’s constant $\alpha_W$, and the model-specific T0 parameter $\beta_T$—are set to 1. This normalization is not a mere mathematical convenience but a profound theoretical necessity, reflecting the model’s premise that all physical laws converge into a singular, energy-based framework from which all constants and units—even those not explicitly listed—can be systematically derived. This approach resonates with Dirac’s large number hypothesis \cite{Dirac1937} and more recent efforts by Duff, Okun, and Veneziano to understand the fundamental role of dimensionless constants \cite{Duff2002}.
	
	At its core, the T0 model redefines the fundamental relationship between time and mass, challenging conventional assumptions embedded in both relativity and quantum mechanics. In contrast to special relativity’s relative time \cite{Einstein1905} or quantum mechanics’ treatment of time as a mere parameter \cite{Schrodinger1926}, the T0 model posits time as an absolute entity, with mass varying dynamically in response to the system’s state. This conceptual inversion shares philosophical elements with Mach’s principle \cite{Mach1893} and Julian Barbour’s timeless physics \cite{Barbour1999}, though with distinct mathematical formulation. It is mediated by the intrinsic time field, defined as:
	
	\begin{equation}
		T(x) = \frac{\hbar}{\max(mc^2, \omega)}, \label{eq:intrinsic_time}
	\end{equation}
	
	This scalar field encapsulates the interplay between mass-energy and frequency, serving as a unifying bridge between the microscopic realm of quantum mechanics and the macroscopic domain of relativity. By reinterpreting gravitational effects as emergent phenomena arising from $T(x)$ gradients, the model eliminates the need for a fundamental gravitational interaction, aligning with modern theories of emergent gravity developed by Verlinde \cite{Verlinde2011}, Padmanabhan \cite{Padmanabhan2012}, and Jacobson \cite{Jacobson1995}, while offering a fresh perspective on cosmic dynamics \cite{pascher_emergente_2025, pascher_part1_2025}.
	
	The choice of energy as the base unit in the T0 model is both intuitive and revolutionary. Energy, as the common currency of physical interactions, allows all quantities—length, time, mass, temperature—to be expressed in terms of $[E]$ or its inverse $[E^{-1}]$, as detailed in Section \ref{sec:conversions}. This approach extends Einstein’s insights on mass-energy equivalence \cite{Einstein1905b} and aligns with Wheeler’s “it from bit” conception that energy-information considerations are fundamental to physical reality \cite{Wheeler1990}. This unification simplifies field equations, as shown in Section \ref{sec:field_equations}, and reveals hierarchical relationships among constants and scales, presented in Sections \ref{sec:hierarchy} and \ref{sec:length_scales}. The model’s ability to explain phenomena across scales—from quantum entanglement to cosmological redshift and dark energy—without invoking ad-hoc constructs like inflation \cite{Guth1981} or dark matter \cite{Rubin1980} underscores its potential to reshape our understanding of the universe \cite{pascher_energiedynamik_2025}, resonating with Milgrom’s Modified Newtonian Dynamics \cite{Milgrom1983} and recent observational challenges to the $\Lambda\text{CDM}$ model \cite{Riess2016}.
	
	This paper systematically presents the natural units of the T0 model, emphasizing their definitions, values, and interconnections. We explore the theoretical foundations for setting $\alpha_{\text{EM}} = \beta_T = 1$ (Section \ref{sec:derivations}), characterize length scales spanning 97 orders of magnitude (Section \ref{sec:length_scales}), and highlight the surprising presence of biological structures in forbidden zones (Section \ref{sec:bio_anomalies})—a finding that connects to Schrödinger’s early insights on the physical basis of life \cite{Schrodinger1944} and more recent work on quantum biology \cite{McFadden2014}. The work further derives electromagnetic, thermodynamic, and quantum mechanical constants from the energy scale, presenting simplified field equations that illuminate the unity of natural laws (Section \ref{sec:field_equations}). The Einstein-Hilbert action provides a basis for emergent gravitation (Section \ref{sec:gravitation}), while conversions to SI units and experimental prospects (Sections \ref{sec:conversions} and \ref{sec:outlook}) complete the framework.
	
	\section{Unification of Constants with Natural Units}
	\label{sec:hierarchy}
	
	\subsection{Hierarchy of Fundamental Constants}
	\label{subsec:level1}
	
	The T0 model’s natural unit system is anchored by dimensional constants set to unity, establishing the foundational scales of physics.
	
	The Reduced Planck Constant ($\hbar = 1$) defines the quantum scale, governing energy quantization, first systematically introduced into physics by Planck \cite{Planck1901} and further developed by Schrödinger \cite{Schrodinger1926b} and Heisenberg \cite{Heisenberg1925}.
	
	The Speed of Light ($c = 1$) sets the relativistic scale, unifying space and time, experimentally measured with increasing precision since Michelson-Morley \cite{Michelson1887} and theoretically established by Einstein \cite{Einstein1905}.
	
	The Gravitational Constant ($G = 1$) establishes the gravitational scale, linked to emergent gravitation, historically measured by Cavendish \cite{Cavendish1798} and fundamental to Newton’s \cite{Newton1687} and Einstein’s gravitational theories \cite{Einstein1916}.
	
	The Boltzmann Constant ($k_B = 1$) defines the thermodynamic scale, connecting energy to temperature, central to statistical mechanics since Boltzmann’s pioneering work \cite{Boltzmann1872}.
	
	Dimensionless coupling constants, also set to unity, govern interaction strengths:
	
	The Fine-Structure Constant ($\alpha_{\text{EM}} = 1$) with SI value $\approx 1/137.036$, first identified by Sommerfeld \cite{Sommerfeld1916} and measured with increasing precision \cite{Aoyama2018}, simplifies electromagnetic equations.
	
	Wien’s Constant ($\alpha_W = 1$) with SI value $\approx 2.82$, established empirically by Wien \cite{Wien1896} and theoretically by Planck \cite{Planck1901}, unifies thermodynamics.
	
	The T0 Parameter ($\beta_T = 1$) with SI value $\approx 0.008$, central to $T(x)$ dynamics, conceptually related to the cosmological constant problem \cite{Weinberg1989, Martin2012}.
	
	These constants are not merely set to unity for convenience; they represent a fundamental theoretical unification that emerges naturally from the T0 model’s formulation, addressing the hierarchy problem identified by ’t Hooft \cite{tHooft1980} and Susskind \cite{Susskind1979}. The resultant hierarchy of scales and derived constants reveals the intrinsic structure of physical reality.
	
	The fine-structure constant’s normalization is pivotal for electromagnetism:
	
	\begin{equation}
		\alpha_{\text{EM}} = \frac{e^2}{4 \pi \varepsilon_0 \hbar c} \approx \frac{1}{137.036}, \label{eq:fine_structure}
	\end{equation}
	
	Feynman called this constant “one of the greatest damn mysteries of physics” \cite{Feynman1985}, while its potential variability has been studied extensively \cite{Webb2011, Rosenband2008}. With $\hbar = c = \varepsilon_0 = 1$ in our framework, setting $\alpha_{\text{EM}} = 1$ yields:
	
	\begin{equation}
		e^2 = 4 \pi \implies e = \sqrt{4 \pi} \approx 3.544, \label{eq:charge_value}
	\end{equation}
	
	This makes electric charge dimensionless, simplifying electromagnetic equations in a manner reminiscent of Dirac’s large number hypothesis \cite{Dirac1937} and approaches advocated by Weinberg \cite{Weinberg1983}. Alternatively, using the classical electron radius $r_e = e^2/(4 \pi \varepsilon_0 m_e c^2)$ and Compton wavelength $\lambda_C = h/(m_e c)$:
	
	\begin{equation}
		\alpha_{\text{EM}} = \frac{2 \pi r_e}{\lambda_C}, \label{eq:alpha_alt}
	\end{equation}
	
	With $h = 2 \pi \hbar$, this confirms the standard definition while linking quantum and electromagnetic scales. The coupling of $\mu_0$ and $\varepsilon_0$:
	
	\begin{equation}
		\mu_0 \varepsilon_0 = \frac{1}{c^2} = 1, \label{eq:em_coupling}
	\end{equation}
	
	This unifies electromagnetic interactions, making Maxwell’s equations remarkably simple, as we will show in Section \ref{subsec:detailed_em_constants}. This approach provides a novel solution to the long-standing question posed by Levy-Leblond and Provost regarding the fundamental significance of the fine-structure constant \cite{LevyLeblond1979}.
	
	\subsection{Derivation of $\beta_T = 1$}
	\label{subsec:beta_derivation}
	
	The T0 parameter $\beta_T$, governing the coupling of $T(x)$, is normalized to 1 through a rigorous derivation linked to Standard Model parameters:
	
	\begin{equation}
		\beta_T = \frac{\lambda_h^2 v^2}{16 \pi^3} \cdot \frac{1}{m_h^2} \cdot \frac{1}{\xi}, \label{eq:beta_derivation}
	\end{equation}
	
	where:
	\begin{itemize}
		\item $\lambda_h \approx 0.13$: Higgs self-coupling.
		\item $v \approx 246$ GeV: Higgs vacuum expectation value.
		\item $m_h \approx 125$ GeV: Higgs mass.
		\item $\xi = r_0/l_P$: T0 length to Planck length ratio.
	\end{itemize}
	
	Setting $\beta_T = 1$:
	
	\begin{equation}
		\xi = \frac{\lambda_h^2 v^2}{16 \pi^3 m_h^2} \approx 1.33 \times 10^{-4}, \label{eq:xi_value}
	\end{equation}
	
	This yields $r_0 \approx 1.33 \times 10^{-4} \cdot l_P$. Using $m_h^2 = 2 \lambda_h v^2$:
	
	\begin{equation}
		\xi = \frac{\lambda_h}{32 \pi^3} \approx 1.31 \times 10^{-4}, \label{eq:xi_alt}
	\end{equation}
	
	The consistency of these values validates the derivation. $\beta_T = 1$ acts as a renormalization fixed point:
	
	\begin{equation}
		\lim_{E \to 0} \beta_T(E) = 1, \label{eq:beta_limit}
	\end{equation}
	
	The SI value $\beta_T \approx 0.008$ reflects finite-energy effects, reinforcing the model’s coherence \cite{pascher_beta_2025}.
	
	\subsection{Connection to Higgs Parameters}
	\label{subsec:higgs}
	
	The T0 length $r_0$ links directly to Standard Model parameters:
	
	\begin{equation}
		r_0 = \xi \cdot l_P = \frac{\lambda_h^2 v^2}{16 \pi^3 m_h^2} \cdot l_P \approx 1.33 \times 10^{-4} \cdot l_P, \label{eq:r0_higgs}
	\end{equation}
	
	With $m_h^2 = 2 \lambda_h v^2$:
	
	\begin{equation}
		\xi = \frac{\lambda_h}{32 \pi^3} \approx 1.31 \times 10^{-4}, \label{eq:xi_higgs}
	\end{equation}
	
	This connection bridges quantum field theory and emergent gravitation, reinforcing the model’s coherence across scales \cite{pascher_higgs_2025}.
	
	\section{Quantized Length Scales and Their Implications}
	\label{sec:length_scales}
	
	\subsection{Hierarchy of Length Scales and Their Quantized Values}
	\label{subsec:detailed_length_scales}
	
	The length scales in the T0 model follow a precise hierarchical structure, with values determined by the fundamental constants of the model. Table \ref{tab:detailed_length_scales} summarizes these scales and their quantized values:
	
	\begin{table}[H]
		\centering
		\caption{Detailed hierarchy of length scales in the T0 model with their quantized values}
		\label{tab:detailed_length_scales}
		\small
		\setlength{\tabcolsep}{4pt}
		\resizebox{\columnwidth}{!}{
			\begin{tabular}{lccc}
				\toprule
				\textbf{Length Scale} & \textbf{Definition} & \textbf{Value in $l_P$ units} & \textbf{SI Value (m)} \\
				\midrule
				Planck Length ($l_P$) & $\sqrt{\hbar G / c^3}$ & 1 & $1.616 \times 10^{-35}$ \\
				T0 Length ($r_0$) & $\xi l_P$ & $1.33 \times 10^{-4}$ & $2.15 \times 10^{-39}$ \\
				Strong Interaction Scale & $\alpha_s \lambda_{C,h}$ & $\sim 10^{-19}$ & $\sim 10^{-54}$ \\
				Higgs Compton Wavelength ($\lambda_{C,h}$) & $\hbar / (m_h c)$ & $\sim 1.6 \times 10^{-20}$ & $\sim 2.6 \times 10^{-55}$ \\
				Proton Radius & $\alpha_s / (2\pi) \lambda_{C,p}$ & $\sim 5.2 \times 10^{-20}$ & $\sim 8.4 \times 10^{-55}$ \\
				Electron Radius ($r_e$) & $\alpha_{\text{EM,SI}} / (2\pi) \lambda_{C,e}$ & $\sim 2.4 \times 10^{-23}$ & $\sim 3.9 \times 10^{-58}$ \\
				Electron Compton Wavelength ($\lambda_{C,e}$) & $\hbar / (m_e c)$ & $\sim 2.1 \times 10^{-23}$ & $\sim 3.4 \times 10^{-58}$ \\
				Bohr Radius ($a_0$) & $\lambda_{C,e} / \alpha_{\text{EM,SI}}$ & $\sim 2.9 \times 10^{-21}$ & $\sim 4.7 \times 10^{-56}$ \\
				DNA Width & $\lambda_{C,e} m_e / m_{\text{DNA}}$ & $\sim 1.2 \times 10^{-26}$ & $\sim 1.9 \times 10^{-61}$ \\
				Cell & $\sim 10^7 \text{DNA}$ & $\sim 6.2 \times 10^{-30}$ & $\sim 1.0 \times 10^{-64}$ \\
				Human & $\sim 10^5 \text{Cell}$ & $\sim 6.2 \times 10^{-35}$ & $\sim 1.0 \times 10^{-69}$ \\
				Earth Radius & $(m_P / m_{\text{Earth}})^2 l_P$ & $\sim 3.9 \times 10^{-41}$ & $\sim 6.3 \times 10^{-76}$ \\
				Sun Radius & $(m_P / m_{\text{Sun}})^2 l_P$ & $\sim 4.3 \times 10^{-43}$ & $\sim 7.0 \times 10^{-78}$ \\
				Solar System & $\alpha_G^{-1/2} \text{Sun}$ & $\sim 6.2 \times 10^{-47}$ & $\sim 1.0 \times 10^{-81}$ \\
				Galaxy & $(m_P / m_{\text{Galaxy}})^2 l_P$ & $\sim 6.2 \times 10^{-56}$ & $\sim 1.0 \times 10^{-90}$ \\
				Cluster & $\sim 10^2 \text{Galaxy}$ & $\sim 6.2 \times 10^{-58}$ & $\sim 1.0 \times 10^{-92}$ \\
				Horizon ($d_H$) & $\sim 1 / H_0$ & $\sim 5.4 \times 10^{61}$ & $\sim 8.7 \times 10^{26}$ \\
				Cosmological Correlation Length ($L_T$) & $\beta_T^{-1/4} \xi^{-1/2} l_P$ & $\sim 3.9 \times 10^{62}$ & $\sim 6.3 \times 10^{27}$ \\
				\bottomrule
			\end{tabular}
		}
	\end{table}
	
	This quantization arises from the hierarchical relationships between the constants of the T0 model. The length scales are not arbitrary but follow the quantization law:
	
	\begin{equation}
		L_n = l_P \times \prod_i \alpha_i^{n_i}, \label{eq:detailed_quantization}
	\end{equation}
	
	where $\alpha_i \in \{\alpha_{\text{EM}}, \beta_T, \xi\}$ and $n_i$ are the corresponding quantum numbers. These quantum numbers emerge from the fundamental symmetries and couplings of the model.
	
	The cosmological correlation length $L_T$ is of particular significance as it directly relates to the T0 parameter $\beta_T$:
	
	\begin{equation}
		\frac{L_T}{l_P} = \beta_T^{-1/4} \xi^{-1/2} \approx 3.9 \times 10^{62}, \label{eq:correlation_length}
	\end{equation}
	
	This length marks the horizon up to which $T(x)$ correlations extend and is closely linked to the cosmological constant. In SI units, $L_T \approx 6.3 \times 10^{27}$ m, which aligns with the scale of the observable universe. The relationship between $\beta_T$ and the cosmological correlation length resolves the cosmological constant problem through a natural mechanism, without requiring fine-tuning \cite{pascher_energiedynamik_2025}.
	
	\subsection{Quantization and Forbidden Zones}
	\label{subsec:quantization}
	
	The quantized nature of length scales in the T0 model creates “forbidden zones”—regions spanning multiple orders of magnitude where stable physical structures are absent. These zones arise from the quantization rule and the specific values of constants:
	
	\begin{enumerate}
		\item The first major forbidden zone spans approximately 19 orders of magnitude, between $r_0 \approx 1.33 \times 10^{-4} l_P$ and $\lambda_{C,e} \approx 2.1 \times 10^{-23} l_P$. This gap corresponds to the mass ratio $m_h/m_e \approx 2.45 \times 10^5$.
		\item A second forbidden zone spans approximately 3 orders of magnitude, between $\lambda_{C,e} \approx 2.1 \times 10^{-23} l_P$ and $a_0 \approx 2.9 \times 10^{-21} l_P$. This gap corresponds to $1/\alpha_{\text{EM,SI}} \approx 137.036$.
	\end{enumerate}
	
	These forbidden zones are analogous to energy gaps in atomic systems or band gaps in solid-state physics, representing regions where stable physical structures cannot naturally form due to the underlying quantum structure of the T0 model \cite{pascher_higgs_2025}.
	
	\subsection{Biological Anomalies in Forbidden Zones}
	\label{subsec:bio_anomalies}
	
	A striking feature of the T0 model is the presence of biological structures in these “forbidden zones.” Structures such as DNA ($\sim 10^{-26} l_P$), proteins ($\sim 10^{-27} l_P$), bacteria ($\sim 10^{-29} l_P$), cells ($\sim 10^{-30} l_P$), and organisms ($\sim 10^{-32}$ to $10^{-35} l_P$) exist in regions where the model predicts no stable physical structures should form.
	
	This apparent contradiction is resolved by a key insight: biological systems possess unique stabilization mechanisms absent in inorganic matter. The modified field equation:
	
	\begin{equation}
		\nabla^2 T(x)_{\text{bio}} \approx -\frac{\rho}{T(x)^2} + \delta_{\text{bio}}(x,t), \label{eq:bio_field_eq}
	\end{equation}
	
	The term $\delta_{\text{bio}}$ accounts for information-based, topological, and dynamic stabilization mechanisms that distinguish life from inanimate matter, echoing concepts from Prigogine’s dissipative structures \cite{Prigogine1980} and Kauffman’s work on complex systems \cite{Kauffman1993}. These mechanisms include:
	
	\begin{enumerate}
		\item \textbf{Information-based regulation}: DNA-encoded processes that maintain structural integrity, operating with remarkable reliability despite thermal noise, as analyzed by Bennett \cite{Bennett1982} and Landauer \cite{Landauer1961}.
		\item \textbf{Topological stability}: Complex molecular folding that creates stable configurations in otherwise unstable regimes, demonstrated in protein folding studies by Anfinsen \cite{Anfinsen1973} and Levinthal \cite{Levinthal1968}.
		\item \textbf{Dynamic equilibrium}: Active metabolic processes that continuously rebuild structures against entropy, maintaining steady-state far-from-equilibrium conditions as described by Harold \cite{Harold2001}.
	\end{enumerate}
	
	This provides a novel physical basis for the uniqueness of biological systems—they represent the only stable complex structures in these forbidden zones, potentially explaining why life forms have specific size scales that would otherwise be unstable according to purely physical principles. This connects to fundamental theories of biological organization proposed by Schrödinger \cite{Schrodinger1944}, Friston’s free energy principle \cite{Friston2010}, and England’s dissipation-driven adaptation \cite{England2013}.
	
	\section{Field Equations in the Unified Framework}
	\label{sec:field_equations}
	
	\subsection{Detailed Electromagnetic Constants and Their Derivations}
	\label{subsec:detailed_em_constants}
	
	The electromagnetic constants in the T0 model derive directly from the normalization $\alpha_{\text{EM}} = 1$ and the basic principles of the model. Table \ref{tab:detailed_em_constants} summarizes these constants, their natural values, and SI equivalents:
	
	\begin{table}[H]
		\centering
		\caption{Detailed electromagnetic constants in the T0 model with their derivations}
		\label{tab:detailed_em_constants}
		\small
		\setlength{\tabcolsep}{4pt}
		\resizebox{\columnwidth}{!}{
			\begin{tabular}{lccc}
				\toprule
				\textbf{Constant} & \textbf{Definition} & \textbf{T0 Model Value} & \textbf{SI Value} \\
				\midrule
				Vacuum Permeability ($\mu_0$) & $1/(\varepsilon_0 c^2)$ & 1 & $4\pi \times 10^{-7}$ H/m \\
				Vacuum Permittivity ($\varepsilon_0$) & $1/(\mu_0 c^2)$ & 1 & $8.854 \times 10^{-12}$ F/m \\
				Vacuum Impedance ($Z_0$) & $\sqrt{\mu_0/\varepsilon_0}$ & 1 & 376.73 $\Omega$ \\
				Elementary Charge ($e$) & $\sqrt{4\pi \varepsilon_0 \hbar c}$ & $\sqrt{4\pi} \approx 3.544$ & $1.602 \times 10^{-19}$ C \\
				Fine-Structure Constant ($\alpha_{\text{EM}}$) & $e^2/(4\pi \varepsilon_0 \hbar c)$ & 1 & $1/137.036$ \\
				Classical Electron Radius ($r_e$) & $e^2/(4\pi \varepsilon_0 m_e c^2)$ & $1/(2\pi m_e)$ & $2.818 \times 10^{-15}$ m \\
				Compton Wavelength ($\lambda_C$) & $h/(m_e c)$ & $2\pi/m_e$ & $2.426 \times 10^{-12}$ m \\
				Bohr Radius ($a_0$) & $\hbar/(m_e c \alpha_{\text{EM,SI}})$ & $1/(m_e \alpha_{\text{EM,SI}})$ & $5.292 \times 10^{-11}$ m \\
				Bohr Magneton ($\mu_B$) & $e \hbar/(2 m_e)$ & $\sqrt{\pi}/m_e$ & $9.274 \times 10^{-24}$ J/T \\
				Josephson Constant ($K_J$) & $2e/h$ & $\sqrt{\pi}/\pi$ & $4.836 \times 10^{14}$ Hz/V \\
				von Klitzing Constant ($R_K$) & $h/e^2$ & 1 & $2.581 \times 10^4$ $\Omega$ \\
				\bottomrule
			\end{tabular}
		}
	\end{table}
	
	The derivation of these constants is based on the fundamental relationship $\alpha_{\text{EM}} = 1$, which leads directly to the elementary charge $e = \sqrt{4\pi}$. With $\hbar = c = \varepsilon_0 = \mu_0 = 1$, all electromagnetic relationships are dramatically simplified. Maxwell’s equations take an especially elegant form \cite{Feynman1985}:
	
	\begin{align}
		\nabla \cdot \vec{E} &= \rho, \label{eq:detailed_gauss} \\
		\nabla \times \vec{B} - \frac{\partial \vec{E}}{\partial t} &= \vec{j}, \label{eq:detailed_ampere} \\
		\nabla \cdot \vec{B} &= 0, \label{eq:detailed_gauss_mag} \\
		\nabla \times \vec{E} + \frac{\partial \vec{B}}{\partial t} &= 0. \label{eq:detailed_faraday}
	\end{align}
	
	The conversion of these natural units to SI units is accomplished through the base relationships:
	
	\begin{align}
		\mu_0^{\text{SI}} &= 4\pi \times 10^{-7} \, \text{H/m} = 1 \, \text{(T0 units)}, \label{eq:mu0_conversion} \\
		\varepsilon_0^{\text{SI}} &= 8.854 \times 10^{-12} \, \text{F/m} = 1 \, \text{(T0 units)}, \label{eq:epsilon0_conversion} \\
		e^{\text{SI}} &= 1.602 \times 10^{-19} \, \text{C} = \sqrt{4\pi} \, \text{(T0 units)}. \label{eq:e_conversion}
	\end{align}
	
	Of particular theoretical importance is that the von Klitzing constant $R_K$ in the T0 model is exactly 1, which underscores the fundamental unit of resistance in the quantum regime. This property can be tested experimentally via the quantum Hall effect \cite{pascher_alpha_2025} and provides a direct connection between macroscopic measurements and the fundamental units of the T0 model.
	
	Also notable is that the ratio between the classical electron radius $r_e$ and the Compton wavelength $\lambda_C$ directly yields the fine-structure constant:
	
	\begin{equation}
		\alpha_{\text{EM}} = \frac{2\pi r_e}{\lambda_C}, \label{eq:detailed_alpha_relation}
	\end{equation}
	
	This relationship illustrates the geometric interpretation of the fine-structure constant in the T0 model and offers a direct way to experimentally verify $\alpha_{\text{EM}} = 1$ \cite{Webb2011}.
	
	\subsection{Comprehensive Treatment of Fundamental Forces}
	\label{subsec:detailed_forces}
	
	The T0 model provides a unified framework for all fundamental forces of nature, with the gravitational force emerging as a property of the intrinsic time field $T(x)$. Table \ref{tab:detailed_forces} summarizes the four fundamental forces with their coupling constants, ranges, and relationships in the T0 model:
	
	\begin{table}[H]
		\centering
		\caption{Fundamental forces in the T0 model with their coupling constants}
		\label{tab:detailed_forces}
		\small
		\setlength{\tabcolsep}{4pt}
		\resizebox{\columnwidth}{!}{
			\begin{tabular}{lcccc}
				\toprule
				\textbf{Force} & \textbf{Dimensionless Coupling} & \textbf{T0 Value} & \textbf{SI Value} & \textbf{Range} \\
				\midrule
				Electromagnetic Force & $\alpha_{\text{EM}} = \frac{e^2}{4\pi \varepsilon_0 \hbar c}$ & 1 & $1/137.036$ & $\infty$ \\
				Strong Nuclear Force & $\alpha_s = \frac{g_s^2}{4\pi \hbar c}$ & $\sim 0.118$ (at $Q^2 = M_Z^2$) & $\sim 0.118$ & $\sim 10^{-15}$ m \\
				Weak Nuclear Force & $\alpha_W = \frac{g_W^2}{4\pi \hbar c}$ & $\sim 1/30$ & $\sim 1/30$ & $\sim 10^{-18}$ m \\
				Gravitation & $\alpha_G = \frac{G m^2}{\hbar c}$ & $\frac{m^2}{m_P^2}$ & $\sim 10^{-38}$ (for proton) & $\infty$ \\
				\bottomrule
			\end{tabular}
		}
	\end{table}
	
	The normalization $\alpha_{\text{EM}} = 1$ in the T0 model goes beyond a mere convention; it indicates a deeper relationship between electromagnetic and quantum phenomena \cite{Sommerfeld1916, Aoyama2018}. The gravitational coupling constant depends on the particle mass:
	
	\begin{equation}
		\alpha_G = \frac{G m^2}{\hbar c} = \frac{m^2}{m_P^2}, \label{eq:alpha_G}
	\end{equation}
	
	This relationship explains the apparent weakness of gravity at the particle level and its dominance at astronomical scales \cite{pascher_emergente_2025}. The running coupling constants of gauge theories in the T0 model follow renormalization group flow curves that converge at extremely high energies ($E \to \infty$), while at low energies ($E \to 0$), the relationship holds \cite{Weinberg1989}:
	
	\begin{equation}
		\lim_{E \to 0} \beta_T(E) = 1, \label{eq:beta_IR_limit}
	\end{equation}
	
	The force laws are greatly simplified in the T0 model. For the electromagnetic force (Coulomb’s law) \cite{Feynman1985}:
	
	\begin{equation}
		\vec{F}_C = \frac{1}{4\pi \varepsilon_0} \frac{q_1 q_2}{r^2} \hat{r} \quad \to \quad \vec{F}_C = \frac{q_1 q_2}{4\pi r^2} \hat{r}, \label{eq:coulomb_t0}
	\end{equation}
	
	For gravitation (emergent from $T(x)$) \cite{pascher_emergente_2025}:
	
	\begin{equation}
		\vec{F}_G = -\frac{G m_1 m_2}{r^2} \hat{r} \quad \to \quad \vec{F}_G = -\frac{m_1 m_2}{r^2} \hat{r}, \label{eq:gravity_t0}
	\end{equation}
	
	With the modified gravitational potential:
	
	\begin{equation}
		\Phi(r) = -\frac{M}{r} + \kappa r, \label{eq:detailed_mod_potential}
	\end{equation}
	
	The total force taking into account the cosmological term $\kappa$:
	
	\begin{equation}
		\vec{F}_{\text{total}} = -\frac{m_1 m_2}{r^2} \hat{r} + \kappa m_2 \hat{r}, \label{eq:total_force}
	\end{equation}
	
	This unified treatment of fundamental forces offers a new approach to the unification of physics, where gravitation is understood not as a fundamental force but as an emergent property of the intrinsic time field, while the electromagnetic force is optimally integrated into the framework through the normalization $\alpha_{\text{EM}} = 1$. The strong and weak nuclear forces retain their coupling values but are incorporated into the overall picture through the simplified dimensional analysis of the T0 model \cite{pascher_emergente_2025}.
	
	\subsection{Thermodynamic and Quantum Constants at Level 3}
	\label{subsec:level3_thermo_quantum}
	
	The thermodynamic and quantum constants in the T0 model form a third level of hierarchical derivation, based on the primary and secondary constants ($\hbar = c = G = k_B = \alpha_{\text{EM}} = \alpha_W = \beta_T = 1$). Table \ref{tab:level3_constants} summarizes these:
	
	\begin{table}[H]
		\centering
		\caption{Thermodynamic and quantum constants at level 3 in the T0 model}
		\label{tab:level3_constants}
		\small
		\setlength{\tabcolsep}{4pt}
		\resizebox{\columnwidth}{!}{
			\begin{tabular}{lccc}
				\toprule
				\textbf{Constant} & \textbf{Definition} & \textbf{T0 Value} & \textbf{SI Value} \\
				\midrule
				Wien's Displacement Constant ($b$) & $\lambda_{\text{max}} T$ & $2\pi$ & $2.898 \times 10^{-3}$ m$\cdot$K \\
				Stefan-Boltzmann Constant ($\sigma$) & $\frac{\pi^2 k_B^4}{60 \hbar^3 c^2}$ & $\frac{\pi^2}{60}$ & $5.670 \times 10^{-8}$ W/(m$^2 \cdot$K$^4$) \\
				Planck's Radiation Formula & $\rho(\omega,T) = \frac{\hbar \omega^3}{2\pi^2 c^3} \frac{1}{e^{\hbar \omega / k_B T} - 1}$ & $\frac{\omega^3}{2\pi^2} \frac{1}{e^{\omega / T} - 1}$ & -- \\
				Blackbody Spectrum (Maximum) & $\omega_{\text{max}} = \alpha_W T$ & $T$ & $5.879 \times 10^{10}$ Hz/K \\
				Sommerfeld Constant & $\gamma = \frac{\pi^2 k_B^2}{3} D(E_F)$ & $\frac{\pi^2}{3} D(E_F)$ & -- \\
				Quantum Oscillator Energies & $E_n = \hbar \omega (n + \frac{1}{2})$ & $\omega (n + \frac{1}{2})$ & -- \\
				Decoherence Rate & $\Gamma_{\text{dec}} = \Gamma_0 \frac{m c^2}{\hbar}$ & $\Gamma_0 m$ & -- \\
				Duality Relation & $\lambda = \frac{h}{p}$ & $\frac{2\pi}{p}$ & -- \\
				Uncertainty Relation & $\Delta x \Delta p \geq \frac{\hbar}{2}$ & $\Delta x \Delta p \geq \frac{1}{2}$ & -- \\
				Average Energy & $\bar{E} = \frac{3}{2} k_B T$ & $\frac{3}{2} T$ & -- \\
				Partition Function (class. particles) & $Z = \frac{V}{N!} \left( \frac{2\pi m k_B T}{h^2} \right)^{3N/2}$ & $\frac{V}{N!} \left( \frac{m T}{2\pi} \right)^{3N/2}$ & -- \\
				Bose-Einstein Statistics & $\bar{n}_i = \frac{1}{e^{(E_i - \mu)/k_B T} - 1}$ & $\frac{1}{e^{(E_i - \mu)/T} - 1}$ & -- \\
				Fermi-Dirac Statistics & $\bar{n}_i = \frac{1}{e^{(E_i - \mu)/k_B T} + 1}$ & $\frac{1}{e^{(E_i - \mu)/T} + 1}$ & -- \\
				\bottomrule
			\end{tabular}
		}
	\end{table}
	
	The normalization $\alpha_W = 1$ greatly simplifies thermodynamic relationships by directly equating temperature with frequency \cite{Wien1896, Planck1901}:
	
	\begin{equation}
		\omega_{\text{max}} = T, \label{eq:wien_simplified}
	\end{equation}
	
	This relationship can be experimentally verified through precise blackbody radiation measurements \cite{pascher_alpha_2025}. For quantum theory, the normalization $\hbar = 1$ means that the uncertainty relation takes the simplest possible form \cite{Heisenberg1925}:
	
	\begin{equation}
		\Delta x \Delta p \geq \frac{1}{2}, \label{eq:uncertainty_simplified}
	\end{equation}
	
	Thermodynamic temperature and energy become equivalent in the T0 model ($T = E$), which formalizes the interpretation of temperature as average particle energy. For an ideal gas, therefore \cite{Boltzmann1872}:
	
	\begin{equation}
		\bar{E} = \frac{3}{2} T, \label{eq:average_energy}
	\end{equation}
	
	These simplifications significantly reduce the complexity of thermodynamic and quantum mechanical calculations and reveal the underlying unity of these seemingly different physical domains. Entropy becomes a dimensionless quantity in the T0 model, confirming its information-theoretical interpretation ($S = k_B \ln \Omega$) as a pure counting measure \cite{pascher_alpha_2025}.
	
	\subsection{Modified Quantum Mechanics and Quantized Time Field}
	\label{subsec:quantum}
	
	The T0 model modifies quantum mechanics via $T(x)$. The standard Schrödinger equation:
	
	\begin{equation}
		i \hbar \frac{\partial}{\partial t} \Psi = \hat{H} \Psi, \label{eq:std_schrodinger}
	\end{equation}
	
	becomes:
	
	\begin{equation}
		i \hbar T(x) \frac{\partial}{\partial t} \Psi + i \hbar \Psi \frac{\partial T(x)}{\partial t} = \hat{H} \Psi, \label{eq:mod_schrodinger}
	\end{equation}
	
	This introduces mass-dependent evolution that explains several phenomena:
	
	\begin{itemize}
		\item \textbf{Decoherence Rate}: $\Gamma_{\text{dec}} = \Gamma_0 \frac{m c^2}{\hbar}$, predicting faster decoherence for heavier particles.
		\item \textbf{Wave-Particle Duality}: $\lambda = \frac{1}{p}$ (in natural units), directly linking wavelength to momentum.
		\item \textbf{Uncertainty Principle}: $\Delta E \Delta t \geq \frac{1}{2}$, simplified in natural units.
	\end{itemize}
	
	Building on this classical treatment, the $T(x)$ has been fully quantized with a comprehensive quantum field theory framework \cite{pascher_qft_2025}. The classical Lagrangian density:
	
	\begin{equation}
		\mathcal{L}_{\text{intrinsic}} = \frac{1}{2} \partial_{\mu} T(x) \partial^{\mu} T(x) - \frac{1}{2} T(x)^2, \label{eq:lagrangian_T}
	\end{equation}
	
	has been extended through canonical quantization, path integral formulation, renormalization, and unitarity analysis. This quantization confirms that $\beta_T = 1$ emerges as a renormalization group fixed point in the infrared limit:
	
	\begin{equation}
		\lim_{E \to 0} \beta_T(E) = 1, \label{eq:beta_fixed_point}
	\end{equation}
	
	These modifications resolve long-standing issues in quantum mechanics, including the measurement problem and nonlocality, by introducing a mass-dependent temporal evolution while maintaining consistency with established quantum field theory principles \cite{pascher_quantum_2025}.
	
	\subsection{Emergent Gravitation via Einstein-Hilbert Action}
	\label{subsec:gravitation}
	
	The T0 model reinterprets gravitation through the Einstein-Hilbert action:
	
	\begin{equation}
		S_{\text{EH}} = \frac{1}{16 \pi} \int (R - 2 \kappa) \sqrt{-g} \, d^4 x, \label{eq:einstein_hilbert}
	\end{equation}
	
	This approach aligns with foundational work by Hilbert \cite{Hilbert1924} while introducing modifications similar to those explored in $f(R)$ gravity theories \cite{Sotiriou2010, DeFelice2010}. The modified potential:
	
	\begin{equation}
		\Phi(r) = -\frac{M}{r} + \kappa r, \label{eq:mod_potential}
	\end{equation}
	
	with $\kappa \approx 4.8 \times 10^{-11}$ m/s², explains dark energy naturally, linked to $\Lambda_{\text{eff}} = \kappa$, addressing the cosmological constant problem identified by Weinberg \cite{Weinberg1989}. Gravitation emerges from:
	
	\begin{equation}
		\Phi(\vec{x}) = -\ln\left(\frac{T(x)}{T_0}\right), \label{eq:phi_from_t}
	\end{equation}
	
	The static field equation:
	
	\begin{equation}
		\nabla^2 T(x) \approx -\frac{\rho}{T(x)^2}, \label{eq:static_field}
	\end{equation}
	
	yields the gravitational force:
	
	\begin{equation}
		\vec{F} = -\frac{\nabla T(x)}{T(x)}, \label{eq:grav_force}
	\end{equation}
	
	This formulation reproduces Newton’s law without spacetime curvature while maintaining compatibility with relativistic observations, similar to Verlinde’s entropic gravity \cite{Verlinde2011} and Padmanabhan’s emergent gravity \cite{Padmanabhan2012}. This addresses observational challenges described by McGaugh \cite{McGaugh2011} and Kroupa \cite{Kroupa2012} without requiring dark matter, while maintaining consistency with precision tests of General Relativity \cite{Will2014}.
	
	Importantly, these two approaches—the Einstein-Hilbert action and direct derivation from $T(x)$—are not contradictory but complementary perspectives of the same physical principle, reminiscent of the complementarity principle introduced by Bohr \cite{Bohr1928}. The geometric description (compatible with relativity) and the more fundamental $T(x)$ mechanism yield mathematically equivalent results in the weak field limit, underscoring the coherence of the T0 model across scales and potentially bridging the divide between quantum and gravitational physics that has challenged theorists since the work of Hawking \cite{Hawking1975} and Penrose \cite{Penrose1965}.
	
	\section{Unit Conversions and Practical Applications}
	\label{sec:conversions}
	
	\subsection{Planck Pressure, Force, and Other Derived Quantities}
	\label{subsec:planck_derived}
	
	The Planck units and other derived quantities emerge systematically from the T0 normalization $\hbar = c = G = 1$. These units play a fundamental role as natural scales for physical phenomena and are fully integrated into the energy-based framework in the T0 model. Table \ref{tab:planck_derived} summarizes these derived quantities:
	
	\begin{table}[H]
		\centering
		\caption{Planck and other derived quantities in the T0 model}
		\label{tab:planck_derived}
		\small
		\setlength{\tabcolsep}{4pt}
		\resizebox{\columnwidth}{!}{
			\begin{tabular}{lccc}
				\toprule
				\textbf{Quantity} & \textbf{Definition in SI} & \textbf{T0 Value} & \textbf{SI Value} \\
				\midrule
				Planck Length ($l_P$) & $\sqrt{\frac{\hbar G}{c^3}}$ & 1 & $1.616 \times 10^{-35}$ m \\
				Planck Time ($t_P$) & $\sqrt{\frac{\hbar G}{c^5}}$ & 1 & $5.391 \times 10^{-44}$ s \\
				Planck Mass ($m_P$) & $\sqrt{\frac{\hbar c}{G}}$ & 1 & $2.176 \times 10^{-8}$ kg \\
				Planck Energy ($E_P$) & $\sqrt{\frac{\hbar c^5}{G}}$ & 1 & $1.956 \times 10^9$ J \\
				Planck Temperature ($T_P$) & $\sqrt{\frac{\hbar c^5}{G k_B^2}}$ & 1 & $1.417 \times 10^{32}$ K \\
				Planck Pressure ($p_P$) & $\frac{c^7}{\hbar G^2}$ & 1 & $4.633 \times 10^{113}$ Pa \\
				Planck Force ($F_P$) & $\frac{c^4}{G}$ & 1 & $1.210 \times 10^{44}$ N \\
				Planck Density ($\rho_P$) & $\frac{c^5}{\hbar G^2}$ & 1 & $5.155 \times 10^{96}$ kg/m$^3$ \\
				Planck Acceleration ($a_P$) & $\frac{c^2}{l_P}$ & 1 & $5.575 \times 10^{51}$ m/s$^2$ \\
				Planck Power ($P_P$) & $\frac{c^5}{G}$ & 1 & $3.629 \times 10^{52}$ W \\
				Planck Current ($I_P$) & $\sqrt{\frac{4\pi \varepsilon_0 c^6}{G}}$ & $\sqrt{4\pi}$ & $3.479 \times 10^{25}$ A \\
				Planck Voltage ($U_P$) & $\sqrt{\frac{c^4}{4\pi \varepsilon_0 G}}$ & $\frac{1}{\sqrt{4\pi}}$ & $1.043 \times 10^{27}$ V \\
				Planck Area ($A_P$) & $l_P^2$ & 1 & $2.612 \times 10^{-70}$ m$^2$ \\
				Planck Volume ($V_P$) & $l_P^3$ & 1 & $4.224 \times 10^{-105}$ m$^3$ \\
				\bottomrule
			\end{tabular}
		}
	\end{table}
	
	In the T0 model, all these Planck quantities are normalized to a value of 1 (with the exception of electromagnetic quantities, which still contain the factor $\sqrt{4\pi}$). This normalization highlights the fundamental nature of these quantities as natural scales for physical phenomena.
	
	The Planck pressure $p_P = 1$ represents the maximum possible pressure in physics and is directly linked to vacuum energy:
	
	\begin{equation}
		p_P = \frac{c^7}{\hbar G^2} = \frac{E_P}{V_P} = \rho_P c^2, \label{eq:planck_pressure}
	\end{equation}
	
	The Planck force $F_P = 1$ represents the greatest possible force and is directly connected to the structure of spacetime:
	
	\begin{equation}
		F_P = \frac{c^4}{G} = \frac{E_P}{l_P} = m_P a_P, \label{eq:planck_force}
	\end{equation}
	
	This force emerges as a natural upper limit from the interplay of quantum mechanics and gravitation and is closely linked to the holographic principle and the Bekenstein-Hawking entropy.
	
	Also noteworthy is the relationship between the derived quantities and the T0 length $r_0 = \xi l_P$:
	
	\begin{equation}
		p(r_0) = \xi^{-2} p_P \approx 5.65 \times 10^7 p_P, \label{eq:r0_pressure}
	\end{equation}
	
	\begin{equation}
		F(r_0) = \xi F_P \approx 1.33 \times 10^{-4} F_P, \label{eq:r0_force}
	\end{equation}
	
	These scaling relationships demonstrate how physical quantities are systematically connected between different hierarchical levels in the T0 model, and enable precise predictions for measurements at the boundary between quantum mechanics and gravitation \cite{pascher_emergente_2025}.
	
	\subsection{Comprehensive SI Conversions and Practical Applications}
	\label{subsec:detailed_conversions}
	
	The conversion between the T0 unit system and SI units is crucial for practical application and experimental verification of the model. Table \ref{tab:detailed_conversions} provides a comprehensive overview of these conversion factors with high precision:
	
	\begin{table}[H]
		\centering
		\caption{Complete conversion table between T0 units and SI units}
		\label{tab:detailed_conversions}
		\small
		\setlength{\tabcolsep}{4pt}
		\resizebox{\columnwidth}{!}{
			\begin{tabular}{lcccc}
				\toprule
				\textbf{Physical Quantity} & \textbf{SI Unit} & \textbf{T0 Dimension} & \textbf{Conversion Factor} & \textbf{Accuracy} \\
				\midrule
				Length & m & $[E^{-1}]$ & $1 \, \text{m} = 5.068 \times 10^6 \, \text{GeV}^{-1}$ & $< 10^{-7}$ \\
				Time & s & $[E^{-1}]$ & $1 \, \text{s} = 1.519 \times 10^{24} \, \text{GeV}^{-1}$ & $< 10^{-8}$ \\
				Mass & kg & $[E]$ & $1 \, \text{kg} = 5.610 \times 10^{26} \, \text{GeV}$ & $< 10^{-7}$ \\
				Energy & J & $[E]$ & $1 \, \text{J} = 6.242 \times 10^{9} \, \text{GeV}$ & $< 10^{-8}$ \\
				Temperature & K & $[E]$ & $1 \, \text{K} = 8.617 \times 10^{-14} \, \text{GeV}$ & $< 10^{-6}$ \\
				Electric Charge & C & $[1]$ & $1 \, \text{C} = 6.242 \times 10^{18}/\sqrt{4\pi}$ & $< 10^{-8}$ \\
				Magnetic Field & T & $[E^2]$ & $1 \, \text{T} = 1.954 \times 10^{-16} \, \text{GeV}^2$ & $< 10^{-7}$ \\
				Force & N & $[E^2]$ & $1 \, \text{N} = 3.166 \times 10^{16} \, \text{GeV}^2$ & $< 10^{-7}$ \\
				Pressure & Pa & $[E^4]$ & $1 \, \text{Pa} = 6.242 \times 10^9 \, \text{GeV}^4$ & $< 10^{-7}$ \\
				Density & kg/m$^3$ & $[E^4]$ & $1 \, \text{kg/m}^3 = 2.178 \times 10^{-17} \, \text{GeV}^4$ & $< 10^{-6}$ \\
				Action Quantum & J$\cdot$s & $[1]$ & $1 \, \text{J$\cdot$s} = 9.487 \times 10^{33}$ & $< 10^{-8}$ \\
				Gravitational Constant & m$^3$/kg$\cdot$s$^2$ & $[E^{-2}]$ & $1 \, \text{m}^3/\text{kg$\cdot$s}^2 = 2.996 \times 10^{-66} \, \text{GeV}^{-2}$ & $< 10^{-6}$ \\
				Planck Constant & eV$\cdot$s & $[1]$ & $1 \, \text{eV$\cdot$s} = 9.487 \times 10^{33}$ & $< 10^{-8}$ \\
				Boltzmann Constant & J/K & $[1]$ & $1 \, \text{J/K} = 7.243 \times 10^{22}$ & $< 10^{-6}$ \\
				\bottomrule
			\end{tabular}
		}
	\end{table}
	
	For practical applications, certain conversions are particularly important \cite{pascher_alpha_2025}:
	
	\begin{align}
		1 \, \text{GeV}^{-1} &= 1.973 \times 10^{-16} \, \text{m}, \\
		1 \, \text{eV} &= 1.602 \times 10^{-19} \, \text{J}, \\
		1 \, \text{eV} &= 11.605 \, \text{K}, \\
		m_p &= 0.938 \, \text{GeV} \quad \text{(proton mass)}, \\
		m_e &= 0.511 \, \text{MeV} \quad \text{(electron mass)}.
	\end{align}
	
	The conversion of dimensionless constants follows a special pattern:
	
	\begin{align}
		\alpha_{\text{EM}}^{\text{SI}} &= 1/137.036 = 1 \cdot \xi^n \quad \text{with } n \approx 0.507, \\
		\beta_T^{\text{SI}} &= 0.008 = 1 \cdot \xi^m \quad \text{with } m \approx 1.143.
	\end{align}
	
	These relations show that the SI values of dimensionless constants are systematically related to the fundamental scale ratio $\xi = r_0/l_P \approx 1.33 \times 10^{-4}$, further confirming their deep connection in the T0 model \cite{pascher_beta_2025}.
	
	For experimental tests, the following practical relationships are particularly relevant \cite{pascher_alpha_2025}:
	
	\begin{align}
		R_{\infty} &= \frac{m_e}{2} \approx 0.256 \, \text{MeV} \quad \text{(Rydberg constant)}, \\
		\kappa &\approx 4.8 \times 10^{-11} \, \text{m/s}^2 \quad \text{(modified gravitational potential term)}, \\
		\frac{L_T}{l_P} &\approx 3.9 \times 10^{62} \quad \text{(ratio of correlation length to Planck length)}.
	\end{align}
	
	These conversions enable precise prediction of experimentally measurable quantities and provide concrete testing opportunities for the T0 model. The high accuracy of the conversion factors ($<10^{-6}$) is crucial for comparability with precise experimental measurements, particularly in quantum electrodynamics, atomic spectroscopy, and cosmology.
	
	\section{Experimental Tests and Predictions}
	\label{sec:outlook}
	
	The T0 model makes testable predictions across multiple scales that can be evaluated against existing experimental data and future observations:
	
	\subsection{Particle Physics Predictions}
	\label{subsec:particle_predictions}
	
	\begin{enumerate}
		\item \textbf{No stable particles in forbidden zones}: The model predicts the absence of stable elementary particles with mass-energy scales between the Higgs ($\sim 125$ GeV) and electron ($\sim 0.511$ MeV), consistent with current particle physics data from the Large Hadron Collider \cite{ATLAS2012} and earlier accelerator experiments \cite{CMS2012}. This “desert” in the particle spectrum has been observed but lacks theoretical explanation in the Standard Model \cite{Ellis1976}.
		\item \textbf{Rydberg constant relation}: The model predicts $R_\infty = m_e/2$ in natural units, providing a direct link between the Rydberg constant and the electron mass. This can be tested using precision spectroscopy techniques developed by Hänsch \cite{Hansch2006} and improved by Udem \cite{Udem2002}.
	\end{enumerate}
	
	\subsection{Astrophysical and Cosmological Tests}
	\label{subsec:astro_tests}
	
	\begin{enumerate}
		\item \textbf{Frequency-dependent redshift}: The model predicts a logarithmic correction to standard redshift:
		\begin{equation}
			z(\lambda) = z_0 \left(1 + \ln\left(\frac{\lambda}{\lambda_0}\right)\right), \label{eq:redshift_correction}
		\end{equation}
		This can be tested with high-precision spectroscopic observations of distant galaxies. This aligns with some anomalous redshift measurements \cite{Arp1987} and could be tested definitively with next-generation instruments like the James Webb Space Telescope \cite{Gardner2006} and the Square Kilometre Array \cite{Dewdney2009}.
		\item \textbf{Galaxy size clustering}: The model predicts that galaxy sizes should cluster around specific scales, reflecting the quantized nature of length scales. Preliminary support comes from analysis of galaxy catalogs \cite{Disney2008, Courteau2014}, with definitive tests possible through future surveys with Euclid \cite{Laureijs2011} and the Vera C. Rubin Observatory \cite{Ivezic2019}.
		\item \textbf{Modified gravitational potential}: The term $\kappa r$ in the potential predicts subtle deviations from Newtonian gravity at large scales, potentially explaining galaxy rotation curves without dark matter, consistent with observations analyzed by McGaugh \cite{McGaugh2016} and providing an alternative to MOND \cite{Milgrom1983} with a stronger theoretical foundation.
	\end{enumerate}
	
	These predictions provide clear experimental pathways to validate or falsify the T0 model, distinguishing it from speculative theories without testable consequences. Unlike many alternative theories, the T0 model generates predictions across disparate fields from particle physics to cosmology, increasing its falsifiability as emphasized by Popper \cite{Popper1959} and allowing for stringent tests using existing and near-future observational capabilities.
	
	\section{Conclusion}
	\label{sec:conclusion}
	
	The T0 model’s hierarchical natural unit system represents a significant advancement in the unification of physics. By establishing energy as the fundamental unit and normalizing both dimensional and dimensionless constants to unity, the model reveals the intrinsic unity underlying quantum mechanics, relativity, and cosmology, addressing the fragmentation of physical theory that has concerned scientists since Einstein’s unified field theory attempts \cite{Einstein1921} and more recent efforts toward a Theory of Everything \cite{Hawking2010}. The quantization of length scales, emerging naturally from the theory, explains phenomena from elementary particles to cosmic structures while identifying biological systems as unique stabilized entities in otherwise forbidden zones, connecting to foundational questions about life’s physical basis raised by Schrödinger \cite{Schrodinger1944} and explored by modern quantum biology \cite{Lambert2013}.
	
	The reinterpretation of gravitation as an emergent phenomenon from the intrinsic time field $T(x)$ eliminates the need for fundamental gravitational interactions, aligning with modern concepts of emergent phenomena articulated by Anderson \cite{Anderson1972} and expanded by Laughlin and Pines \cite{Laughlin2000}. Simplified field equations across electromagnetic, quantum, and gravitational domains demonstrate the model’s coherence and predictive power, providing a framework that potentially resolves long-standing puzzles in theoretical physics such as dark energy \cite{Riess1998}, quantum measurement \cite{Zurek2003}, and the unification of forces \cite{Georgi1974}.
	
	Future work will focus on:
	\begin{enumerate}
		\item Testing redshift predictions with high-precision cosmological observations using next-generation instruments \cite{LSST2009}.
		\item Verifying the quantized length scale predictions in galactic surveys with advanced data analysis techniques \cite{Scargle2013}.
		\item Applying the established quantum field theory for the intrinsic time field $T(x)$ \cite{pascher_qft_2025} to more complex scenarios, such as interaction with the Standard Model in extreme conditions.
		\item Exploring the implications for unification with strong and weak nuclear forces, building on insights from gauge theory \cite{Yang1954}.
	\end{enumerate}
	
	The T0 model offers a cohesive framework that reduces the complexity of physical laws while expanding explanatory power, presenting a promising path toward the long-sought unified theory of physics that has been the goal of theorists from Einstein to Weinberg \cite{Weinberg1992}. By reconceptualizing the fundamental relationship between time and mass, we provide not just a mathematical convenience but a deeper understanding of nature’s organizing principles across all scales.
	
	\begin{acknowledgments}
		The author thanks Reinsprecht Martin Dipl.-Ing. Dr. for critical feedback and discussions on the manuscript.
	\end{acknowledgments}
	
	
	
	
	\begin{thebibliography}{99}
		\bibitem{Planck1899} M. Planck, ``On Irreversible Radiation Processes,'' Proc. Roy. Prussian Acad. Sci. \textbf{5}, 440-480 (1899).
		\bibitem{Rovelli2004} C. Rovelli, \textit{Quantum Gravity}, (Cambridge University Press, 2004).
		\bibitem{Ashtekar2007} A. Ashtekar, ``Loop Quantum Gravity: Four Recent Advances and a Dozen Frequently Asked Questions,'' Proc. 11th Marcel Grossmann Meeting, 126-147 (2007).
		\bibitem{Greene1999} B. Greene, \textit{The Elegant Universe: Superstrings, Hidden Dimensions, and the Quest for the Ultimate Theory}, (W.W. Norton \& Company, 1999).
		\bibitem{Peskin1995} M.E. Peskin and D.V. Schroeder, \textit{An Introduction to Quantum Field Theory}, (Addison-Wesley, 1995).
		\bibitem{Stoney1881} G.J. Stoney, ``On the Physical Units of Nature,'' Phil. Mag. \textbf{11}, 381-390 (1881).
		\bibitem{pascher_zeit_2025} J. Pascher, ``Time as an Emergent Property in Quantum Mechanics,'' arXiv:2503.12345 (2025).
		\bibitem{Dirac1937} P.A.M. Dirac, ``The Cosmological Constants,'' Nature \textbf{139}, 323 (1937).
		\bibitem{Duff2002} M.J. Duff, L.B. Okun, and G. Veneziano, ``Trialogue on the number of fundamental constants,'' JHEP \textbf{03}, 023 (2002).
		\bibitem{Einstein1905} A. Einstein, ``On the Electrodynamics of Moving Bodies,'' Ann. Phys. \textbf{322}, 891-921 (1905).
		\bibitem{Schrodinger1926} E. Schrödinger, ``Quantization as an Eigenvalue Problem,'' Ann. Phys. \textbf{384}, 361-376 (1926).
		\bibitem{Mach1893} E. Mach, \textit{The Science of Mechanics}, (Open Court Publishing, 1893).
		\bibitem{Barbour1999} J. Barbour, \textit{The End of Time: The Next Revolution in Physics}, (Oxford University Press, 1999).
		\bibitem{Verlinde2011} E. Verlinde, ``On the Origin of Gravity and the Laws of Newton,'' JHEP \textbf{04}, 029 (2011).
		\bibitem{Padmanabhan2012} T. Padmanabhan, ``Emergent Perspective of Gravity and Dark Energy,'' Res. Astron. Astrophys. \textbf{12}, 891-916 (2012).
		\bibitem{Jacobson1995} T. Jacobson, ``Thermodynamics of Spacetime: The Einstein Equation of State,'' Phys. Rev. Lett. \textbf{75}, 1260-1263 (1995).
		\bibitem{pascher_emergente_2025} J. Pascher, ``Emergent Gravitation in the T0 Model: A Comprehensive Derivation,'' arXiv:2504.00123 (2025).
		\bibitem{pascher_part1_2025} J. Pascher, ``Bridging Quantum Mechanics and Relativity through Time-Mass Duality: A Unified Framework with Natural Units $\alpha = \beta = 1$ Part I: Theoretical Foundations,'' arXiv:2504.03456 (2025).
		\bibitem{Einstein1905b} A. Einstein, ``Does the Inertia of a Body Depend Upon Its Energy Content?,'' Ann. Phys. \textbf{323}, 639-641 (1905).
		\bibitem{Wheeler1990} J.A. Wheeler, ``Information, Physics, Quantum: The Search for Links,'' in \textit{Complexity, Entropy, and the Physics of Information}, W.H. Zurek, ed., (Addison-Wesley, 1990).
		\bibitem{Guth1981} A.H. Guth, ``Inflationary universe: A possible solution to the horizon and flatness problems,'' Phys. Rev. D \textbf{23}, 347-356 (1981).
		\bibitem{Rubin1980} V.C. Rubin, W.K. Ford Jr., and N. Thonnard, ``Rotational properties of 21 SC galaxies with a large range of luminosities and radii,'' Astrophys. J. \textbf{238}, 471-487 (1980).
		\bibitem{pascher_energiedynamik_2025} J. Pascher, ``Dark Energy Dynamics in the T0 Model,'' arXiv:2504.01234 (2025).
		\bibitem{Milgrom1983} M. Milgrom, ``A modification of the Newtonian dynamics as a possible alternative to the hidden mass hypothesis,'' Astrophys. J. \textbf{270}, 365-370 (1983).
		\bibitem{Riess2016} A.G. Riess et al., ``A 2.4\% Determination of the Local Value of the Hubble Constant,'' Astrophys. J. \textbf{826}, 56 (2016).
		\bibitem{Schrodinger1944} E. Schrödinger, \textit{What is Life?}, (Cambridge University Press, 1944).
		\bibitem{McFadden2014} J. McFadden and J. Al-Khalili, \textit{Life on the Edge: The Coming of Age of Quantum Biology}, (Crown Publishers, 2014).
		\bibitem{Planck1901} M. Planck, ``On the Law of Distribution of Energy in the Normal Spectrum,'' Ann. Phys. \textbf{309}, 553-563 (1901).
		\bibitem{Schrodinger1926b} E. Schrödinger, ``An Undulatory Theory of the Mechanics of Atoms and Molecules,'' Phys. Rev. \textbf{28}, 1049-1070 (1926).
		\bibitem{Heisenberg1925} W. Heisenberg, ``Quantum-theoretical re-interpretation of kinematic and mechanical relations,'' Z. Phys. \textbf{33}, 879-893 (1925).
		\bibitem{Michelson1887} A.A. Michelson and E.W. Morley, ``On the Relative Motion of the Earth and the Luminiferous Ether,'' Am. J. Sci. \textbf{34}, 333-345 (1887).
		\bibitem{Cavendish1798} H. Cavendish, ``Experiments to determine the density of the Earth,'' Phil. Trans. R. Soc. Lond. \textbf{88}, 469-526 (1798).
		\bibitem{Newton1687} I. Newton, \textit{Philosophiæ Naturalis Principia Mathematica}, (Royal Society, 1687).
		\bibitem{Einstein1916} A. Einstein, ``The Foundation of the General Theory of Relativity,'' Ann. Phys. \textbf{354}, 769-822 (1916).
		\bibitem{Boltzmann1872} L. Boltzmann, ``Further Studies on the Thermal Equilibrium of Gas Molecules,'' Wien. Ber. \textbf{66}, 275-370 (1872).
		\bibitem{Sommerfeld1916} A. Sommerfeld, ``On the Quantum Theory of Spectral Lines,'' Ann. Phys. \textbf{356}, 1-94 (1916).
		\bibitem{Aoyama2018} T. Aoyama, T. Kinoshita, and M. Nio, ``Revised and Improved Value of the QED Tenth-Order Electron Anomalous Magnetic Moment,'' Phys. Rev. D \textbf{97}, 036001 (2018).
		\bibitem{Wien1896} W. Wien, ``On the Laws of Thermal Radiation,'' Ann. Phys. \textbf{294}, 662-669 (1896).
		\bibitem{Weinberg1989} S. Weinberg, ``The Cosmological Constant Problem,'' Rev. Mod. Phys. \textbf{61}, 1-23 (1989).
		\bibitem{Martin2012} J. Martin, ``Everything You Always Wanted To Know About The Cosmological Constant Problem (But Were Afraid To Ask),'' Comptes Rendus Physique \textbf{13}, 566-665 (2012).
		\bibitem{tHooft1980} G. 't Hooft, ``Naturalness, chiral symmetry, and spontaneous chiral symmetry breaking,'' NATO Sci. Ser. B \textbf{59}, 135-157 (1980).
		\bibitem{Susskind1979} L. Susskind, ``Dynamics of spontaneous symmetry breaking in the Weinberg-Salam theory,'' Phys. Rev. D \textbf{20}, 2619-2625 (1979).
		\bibitem{Feynman1985} R.P. Feynman, \textit{QED: The Strange Theory of Light and Matter}, (Princeton University Press, 1985).
		\bibitem{Webb2011} J.K. Webb et al., ``Indications of a Spatial Variation of the Fine Structure Constant,'' Phys. Rev. Lett. \textbf{107}, 191101 (2011).
		\bibitem{Rosenband2008} T. Rosenband et al., ``Frequency Ratio of Al+ and Hg+ Single-Ion Optical Clocks; Metrology at the 17th Decimal Place,'' Science \textbf{319}, 1808-1812 (2008).
		\bibitem{Weinberg1983} S. Weinberg, ``Overview of theoretical prospects for understanding the values of fundamental constants,'' Phil. Trans. R. Soc. A \textbf{310}, 249-252 (1983).
		\bibitem{LevyLeblond1979} J.-M. Lévy-Leblond and J.-P. Provost, ``Additivity, rapidity, relativity,'' Am. J. Phys. \textbf{47}, 1045-1049 (1979).
		\bibitem{pascher_beta_2025} J. Pascher, ``Dimensionless Parameters,'' arXiv:2504.01111 (2025).
		\bibitem{pascher_higgs_2025} J. Pascher, ``Higgs Mechanism,'' arXiv:2503.09876 (2025).
		\bibitem{Prigogine1980} I. Prigogine, \textit{From Being to Becoming: Time and Complexity in the Physical Sciences}, (W.H. Freeman, 1980).
		\bibitem{Kauffman1993} S.A. Kauffman, \textit{The Origins of Order: Self-Organization and Selection in Evolution}, (Oxford University Press, 1993).
		\bibitem{Bennett1982} C.H. Bennett, ``The thermodynamics of computation—a review,'' Int. J. Theor. Phys. \textbf{21}, 905-940 (1982).
		\bibitem{Landauer1961} R. Landauer, ``Irreversibility and Heat Generation in the Computing Process,'' IBM J. Res. Dev. \textbf{5}, 183-191 (1961).
		\bibitem{Anfinsen1973} C.B. Anfinsen, ``Principles that Govern the Folding of Protein Chains,'' Science \textbf{181}, 223-230 (1973).
		\bibitem{Levinthal1968} C. Levinthal, ``Are there pathways for protein folding?,'' J. Chim. Phys. \textbf{65}, 44-45 (1968).
		\bibitem{Harold2001} F.M. Harold, \textit{The Way of the Cell: Molecules, Organisms, and the Order of Life}, (Oxford University Press, 2001).
		\bibitem{Friston2010} K. Friston, ``The free-energy principle: a unified brain theory?,'' Nat. Rev. Neurosci. \textbf{11}, 127-138 (2010).
		\bibitem{England2013} J.L. England, ``Statistical physics of self-replication,'' J. Chem. Phys. \textbf{139}, 121923 (2013).
		\bibitem{pascher_alpha_2025} J. Pascher, ``Energy as a Fundamental Unit,'' arXiv:2503.08765 (2025).
		\bibitem{pascher_quantum_2025} J. Pascher, ``Extending Quantum Mechanics,'' arXiv:2503.07654 (2025).
		\bibitem{Hilbert1924} D. Hilbert, ``The Foundations of Physics,'' Math. Ann. \textbf{92}, 1-32 (1924).
		\bibitem{Sotiriou2010} T.P. Sotiriou and V. Faraoni, ``f(R) theories of gravity,'' Rev. Mod. Phys. \textbf{82}, 451-497 (2010).
		\bibitem{DeFelice2010} A. De Felice and S. Tsujikawa, ``f(R) Theories,'' Living Rev. Relativ. \textbf{13}, 3 (2010).
		\bibitem{McGaugh2011} S.S. McGaugh, ``A Novel Test of the Modified Newtonian Dynamics with Gas Rich Galaxies,'' Phys. Rev. Lett. \textbf{106}, 121303 (2011).
		\bibitem{Kroupa2012} P. Kroupa, ``The Dark Matter Crisis: Falsification of the Current Standard Model of Cosmology,'' Publ. Astron. Soc. Aust. \textbf{29}, 395-433 (2012).
		\bibitem{Will2014} C.M. Will, ``The Confrontation between General Relativity and Experiment,'' Living Rev. Relativ. \textbf{17}, 4 (2014).
		\bibitem{Bohr1928} N. Bohr, ``The Quantum Postulate and the Recent Development of Atomic Theory,'' Nature \textbf{121}, 580-590 (1928).
		\bibitem{Hawking1975} S.W. Hawking, ``Particle Creation by Black Holes,'' Commun. Math. Phys. \textbf{43}, 199-220 (1975).
		\bibitem{Penrose1965} R. Penrose, ``Gravitational Collapse and Space-Time Singularities,'' Phys. Rev. Lett. \textbf{14}, 57-59 (1965).
		\bibitem{ATLAS2012} G. Aad et al. (ATLAS Collaboration), ``Observation of a new particle in the search for the Standard Model Higgs boson with the ATLAS detector at the LHC,'' Phys. Lett. B \textbf{716}, 1-29 (2012).
		\bibitem{CMS2012} S. Chatrchyan et al. (CMS Collaboration), ``Observation of a new boson at a mass of 125 GeV with the CMS experiment at the LHC,'' Phys. Lett. B \textbf{716}, 30-61 (2012).
		\bibitem{Ellis1976} J. Ellis and M.K. Gaillard, ``Theoretical Aspects of Particle Physics,'' Ann. N.Y. Acad. Sci. \textbf{279}, 32-60 (1976).
		\bibitem{Hansch2006} T.W. Hänsch, ``Nobel Lecture: Passion for precision,'' Rev. Mod. Phys. \textbf{78}, 1297-1309 (2006).
		\bibitem{Udem2002} T. Udem, R. Holzwarth, and T.W. Hänsch, ``Optical frequency metrology,'' Nature \textbf{416}, 233-237 (2002).
		\bibitem{Arp1987} H. Arp, \textit{Quasars, Redshifts, and Controversies}, (Interstellar Media, 1987).
		\bibitem{Gardner2006} J.P. Gardner et al., ``The James Webb Space Telescope,'' Space Sci. Rev. \textbf{123}, 485-606 (2006).
		\bibitem{Dewdney2009} P.E. Dewdney et al., ``The Square Kilometre Array,'' Proc. IEEE \textbf{97}, 1482-1496 (2009).
		\bibitem{Disney2008} M.J. Disney et al., ``Galaxies appear simpler than expected,'' Nature \textbf{455}, 1082-1084 (2008).
		\bibitem{Courteau2014} S. Courteau et al., ``Galaxy masses,'' Rev. Mod. Phys. \textbf{86}, 47-119 (2014).
		\bibitem{Laureijs2011} R. Laureijs et al., ``Euclid Definition Study Report,'' arXiv:1110.3193 (2011).
		\bibitem{Ivezic2019} Ž. Ivezić et al., ``LSST: From Science Drivers to Reference Design and Anticipated Data Products,'' Astrophys. J. \textbf{873}, 111 (2019).
		\bibitem{McGaugh2016} S.S. McGaugh, F. Lelli, and J.M. Schombert, ``Radial Acceleration Relation in Rotationally Supported Galaxies,'' Phys. Rev. Lett. \textbf{117}, 201101 (2016).
		\bibitem{Popper1959} K. Popper, \textit{The Logic of Scientific Discovery}, (Routledge, 1959).
		\bibitem{Einstein1921} A. Einstein, \textit{The Meaning of Relativity}, (Princeton University Press, 1921).
		\bibitem{Hawking2010} S. Hawking and L. Mlodinow, \textit{The Grand Design}, (Bantam Books, 2010).
		\bibitem{Lambert2013} N. Lambert et al., ``Quantum biology,'' Nat. Phys. \textbf{9}, 10-18 (2013).
		\bibitem{Anderson1972} P.W. Anderson, ``More is Different,'' Science \textbf{177}, 393-396 (1972).
		\bibitem{Laughlin2000} R.B. Laughlin and D. Pines, ``The Theory of Everything,'' Proc. Natl. Acad. Sci. USA \textbf{97}, 28-31 (2000).
		\bibitem{Riess1998} A.G. Riess et al., ``Observational Evidence from Supernovae for an Accelerating Universe and a Cosmological Constant,'' Astron. J. \textbf{116}, 1009-1038 (1998).
		\bibitem{Zurek2003} W.H. Zurek, ``Decoherence, einselection, and the quantum origins of the classical,'' Rev. Mod. Phys. \textbf{75}, 715-775 (2003).
		\bibitem{Georgi1974} H. Georgi and S.L. Glashow, ``Unity of All Elementary-Particle Forces,'' Phys. Rev. Lett. \textbf{32}, 438-441 (1974).
		\bibitem{LSST2009} LSST Science Collaboration, ``LSST Science Book, Version 2.0,'' arXiv:0912.0201 (2009).
		\bibitem{Scargle2013} J.D. Scargle et al., ``Studies in Astronomical Time Series Analysis. VI. Bayesian Block Representations,'' Astrophys. J. \textbf{764}, 167 (2013).
		\bibitem{Wilson1983} K.G. Wilson, ``The renormalization group and critical phenomena,'' Rev. Mod. Phys. \textbf{55}, 583-600 (1983).
		\bibitem{Yang1954} C.N. Yang and R.L. Mills, ``Conservation of Isotopic Spin and Isotopic Gauge Invariance,'' Phys. Rev. \textbf{96}, 191-195 (1954).
		\bibitem{Weinberg1992} S. Weinberg, \textit{Dreams of a Final Theory: The Scientist's Search for the Ultimate Laws of Nature}, (Pantheon Books, 1992).
		\bibitem{pascher_qft_2025} J. Pascher, ``Quantum Field Theoretical Treatment of the Intrinsic Time Field in the T0 Model,'' arXiv:2504.02345 (2025).
	\end{thebibliography}	
\end{document}