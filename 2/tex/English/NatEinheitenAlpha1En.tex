\documentclass[12pt,a4paper]{article}
\usepackage[utf8]{inputenc}
\usepackage[T1]{fontenc}
\usepackage[ngerman]{babel}
\usepackage[left=2cm,right=2cm,top=2cm,bottom=2cm]{geometry}
\usepackage{lmodern}
\usepackage{amsmath}
\usepackage{amssymb}
\usepackage{physics}  % Already includes \grad, \dv, \pdv, \e, \ii, \vev
\usepackage{hyperref}
\usepackage{tcolorbox}
\usepackage{booktabs}
\usepackage{enumitem}
\usepackage[table,xcdraw]{xcolor}
\usepackage{pgfplots}
\pgfplotsset{compat=1.18}
\usepackage{graphicx}
\usepackage{float}
\usepackage{mathtools}
\usepackage{amsthm}
\usepackage{cleveref}
\usepackage{siunitx}
\usepackage{fancyhdr} % For headers and footers
\usepackage{tocloft}  % For table of contents styling

% Headers and Footers
\pagestyle{fancy}
\fancyhf{}
\fancyhead[L]{Johann Pascher}
\fancyhead[R]{Time-Mass Duality}
\fancyfoot[C]{\thepage}
\renewcommand{\headrulewidth}{0.4pt}
\renewcommand{\footrulewidth}{0.4pt}

% Table of Contents Styling
\renewcommand{\cftsecfont}{\color{blue}}
\renewcommand{\cftsubsecfont}{\color{blue}}
\renewcommand{\cftsecpagefont}{\color{blue}}
\renewcommand{\cftsubsecpagefont}{\color{blue}}
\setlength{\cftsecindent}{1cm}
\setlength{\cftsubsecindent}{2cm}

\hypersetup{
	colorlinks=true,
	linkcolor=blue,
	citecolor=blue,
	urlcolor=blue,
	pdftitle={Energy as a Fundamental Unit: Natural Units with alphaEM = 1 in the T0 Model},
	pdfauthor={Johann Pascher},
	pdfsubject={Theoretical Physics},
	pdfkeywords={T0 Model, Natural Units, Fine-Structure Constant, Unified Unit System, Time-Mass Duality}
}

% Custom Commands (consistent)
\newcommand{\Tfield}{T(x)}
\newcommand{\betaT}{\beta_{\text{T}}}
\newcommand{\alphaEM}{\alpha_{\text{EM}}}
\newcommand{\alphaW}{\alpha_{\text{W}}}
\newcommand{\Mpl}{M_{\text{Pl}}}
\newcommand{\Tzerot}{T_0(\Tfield)}
\newcommand{\Tzero}{T_0}
\newcommand{\vecx}{\vec{x}}
\newcommand{\gammaf}{\gamma_{\text{Lorentz}}}
\newcommand{\DhiggsT}{\Tfield (\partial_\mu + ig A_\mu) \Phi + \Phi \partial_\mu \Tfield} % Consistent Definition

\newtheorem{theorem}{Theorem}[section]
\newtheorem{proposition}[theorem]{Proposition}

\begin{document}
	
	\title{Energy as a Fundamental Unit: \\ Natural Units with \(\alphaEM = 1\) in the T0 Model}
	\author{Johann Pascher}
	\date{March 25, 2025}
	
	\maketitle
	\tableofcontents
	\newpage
	
	\section{Introduction to the Unified Unit System}
	\label{sec:intro}
	
	\subsection{From Natural Units to a Fully Unified System}
	\label{subsec:natural_units}
	
	In theoretical physics, various systems of natural units are used to simplify the mathematical formulation of physical laws. The most well-known include:
	
	\begin{itemize}
		\item \textbf{Natural Units:} \(\hbar = c = 1\)
		\item \textbf{Planck Units:} \(\hbar = c = G = 1\)
		\item \textbf{Electrodynamic Natural Units:} \(\hbar = c = \alphaEM = 1\)
		\item \textbf{Thermodynamic Natural Units:} \(\hbar = c = k_B = \alphaW = 1\)
	\end{itemize}
	
	The T0 model introduces a fully unified unit system in which additionally:
	\begin{equation}
		\label{eq:unified_system}
		\betaT = \alphaEM = \alphaW = 1
	\end{equation}
	is set. In this system, all physical quantities are reduced to the dimension of energy:
	
	\begin{tcolorbox}[colback=blue!5!white,colframe=blue!75!black,title=Dimensions in the Unified Unit System]
		\begin{itemize}
			\item Length: \([L] = [E^{-1}]\)
			\item Time: \([T] = [E^{-1}]\)
			\item Mass: \([M] = [E]\)
			\item Temperature: \([T_{\text{emp}}] = [E]\)
			\item Electric Charge: \([Q] = [1]\) (dimensionless)
			\item Intrinsic Time: \([\Tfield] = [E^{-1}]\)
		\end{itemize}
	\end{tcolorbox}
	
	This unified system reveals fundamental relationships between seemingly disparate physical phenomena and enables a more elegant mathematical formulation of the T0 model, as demonstrated in \cite{pascher_lagrange_2025} and \cite{pascher_alphabeta_2025}.
	
	\subsection{Concept of Energy as a Fundamental Unit}
	\label{subsec:energy_concept}
	
	This work also explores the consequences of assuming that the fine-structure constant \(\alphaEM = 1\) in a system of natural units (\(\hbar = c = 1\)) is applied to the T0 model of time-mass duality. Here, energy is identified as the fundamental unit to which all physical quantities can be reduced. The analysis includes dimensional reformulations, simplified fundamental equations, and cosmological implications within the context of the T0 model, which postulates absolute time and variable mass as described in \cite{pascher_zeit_masse_2025}.
	
	\section{Extrapolation of Physics Beyond Known Limits}
	\label{sec:beyond_limits}
	
	\subsection{Physics Beyond the Speed of Light}
	\label{subsec:beyond_lightspeed}
	
	The speed of light \(c\) is considered an absolute limit for matter and signal transmission in standard physics, a direct consequence of Lorentz transformation and relativity theory. Within this framework, all fundamental constants and the Planck scale were defined. However, this limit might only apply within our current theoretical model. In the T0 model with its fundamental time-mass duality, an alternative interpretation may be possible:
	
	\begin{itemize}
		\item \textbf{Reinterpretation of Mass Variation:} In the T0 model, mass \(m = \frac{\hbar}{\Tfield c^2}\) is determined by the intrinsic time field. The relativistic mass change \(m = m_0/\sqrt{1-v^2/c^2}\) can be interpreted as a variation of \(\Tfield\), as explained in \cite{pascher_zeit_2025}.
		\item \textbf{Modified Transformation Laws:} In the unified unit system with \(c = 1\), extended transformations could describe regions with \(v > 1\) without causality violations, while preserving the fundamental relation \(m = \frac{\hbar}{\Tfield c^2}\).
		\item \textbf{Extended Constants:} With \(\alphaEM = \betaT = \alphaW = 1\), a consistent framework emerges that might remain valid beyond the speed of light.
	\end{itemize}
	
	These speculative considerations are further explored in \cref{sec:speculative}.
	
	\subsection{Consequences for Causality and Information}
	\label{subsec:causality}
	
	In the T0 model with its time-mass duality, causality could be reinterpreted as elaborated in \cite{pascher_feldtheorie_2025}:
	\begin{itemize}
		\item \textbf{Time Field-Based Causality:} Causal relationships might be determined by the geometry of the time field \(\Tfield\), not by light cone structures.
		\item \textbf{Non-Local Information Transfer:} The apparent non-locality of quantum mechanics could be explained by the intrinsic time field structure without requiring superluminal signal transmission.
		\item \textbf{Mass-Dependent Causal Structure:} Since \(m = \frac{\hbar}{\Tfield c^2}\), causal relationships might depend on mass, potentially providing a natural explanation for quantum correlations.
	\end{itemize}
	
	\section{Introduction to the Fine-Structure Constant \(\alphaEM\)}
	\label{sec:alpha_em}
	
	The fine-structure constant \(\alphaEM\) describes the strength of electromagnetic interaction between elementary particles and is central to quantum electrodynamics. It is defined as:
	\begin{equation}
		\label{eq:alpha_em_def}
		\alphaEM = \frac{e^2}{4\pi \varepsilon_0 \hbar c} \approx \frac{1}{137.035999}.
	\end{equation}
	
	In the unified unit system, we set \(\alphaEM = 1\), meaning the electric charge \(e\) becomes dimensionless and derives its value directly from the electromagnetic vacuum constants:
	\begin{equation}
		\label{eq:charge_relation}
		e = \sqrt{4\pi \varepsilon_0 \hbar c}
	\end{equation}
	
	This setting leads to a significant simplification of electromagnetic equations and reveals the fundamental nature of electromagnetic interaction as part of the unified framework.
	
	\subsection{Natural Units with \(\alphaEM = 1\)}
	\label{subsec:alpha_one}
	
	In theoretical physics, \(c\) and \(\hbar\) are commonly set to one, as introduced by Planck \cite{planck1899}. Here, we investigate the consequences of additionally setting the fine-structure constant \(\alphaEM = 1\).
	
	\begin{theorem}[Definition of \(\alphaEM = 1\)]
		The fine-structure constant is \cite{Feynman1985}:
		\begin{equation}
			\alphaEM = \frac{e^2}{4\pi\varepsilon_0 \hbar c} \approx \frac{1}{137.036}
		\end{equation}
		With \(\alphaEM = 1\), \(\hbar = c = 1\):
		\begin{equation}
			e = \sqrt{4\pi\varepsilon_0}
		\end{equation}
	\end{theorem}
	
	\textbf{Note}: Here, \(\alphaEM\) denotes the fine-structure constant, not the Wien constant \(\alpha_W \approx 2.82\), as examined in \cite{pascher_temp_2025}.
	
	\subsection{Energy as a Fundamental Unit}
	\label{subsec:energy_fundamental}
	
	\begin{theorem}[Energy as the Basis]
		All quantities can be reduced to energy \cite{Duff2002}:
		\begin{itemize}
			\item Length: \([L] = [E^{-1}]\)
			\item Time: \([T] = [E^{-1}]\)
			\item Mass: \([M] = [E]\)
			\item Charge: \([Q] = [\sqrt{4\pi}]\) (dimensionless)
		\end{itemize}
	\end{theorem}
	
	In the T0 model, this is complemented by \(\Tfield = \frac{\hbar}{m c^2}\), where \(m\) is variable, and energy plays a central role as detailed in \cite{pascher_zeit_masse_2025}.
	
	\subsection{Simplified Fundamental Equations}
	\label{subsec:simplified_equations}
	
	\begin{itemize}
		\item Maxwell Equations \cite{Feynman1985}:
		\begin{align}
			\nabla \cdot \vec{E} &= \rho \\
			\nabla \times \vec{B} - \frac{\partial \vec{E}}{\partial t} &= \vec{j}
		\end{align}
		\item Schrödinger Equation:
		\begin{equation}
			i \frac{\partial \psi}{\partial t} = -\frac{1}{2m} \nabla^2 \psi + V \psi
		\end{equation}
	\end{itemize}
	
	These simplified forms emerge naturally from the T0 model's Lagrangian formulation as shown in \cite{pascher_lagrange_2025}.
	
	\subsection{Table of Reformulated Quantities}
	\label{subsec:reformulated_quantities}
	
	\begin{center}
		\begin{tabular}{|l|c|c|}
			\hline
			\textbf{Physical Quantity} & \textbf{SI Units} & \textbf{\(\hbar = c = \alphaEM = 1\)} \\
			\hline
			Length & m & \(\text{eV}^{-1}\) \\
			Time & s & \(\text{eV}^{-1}\) \\
			Mass & kg & eV \\
			Energy & J & eV \\
			Charge & C & dimensionless \\
			El. Field & V/m & \(\text{eV}^2\) \\
			Mag. Field & T & \(\text{eV}^2\) \\
			\hline
		\end{tabular}
	\end{center}
	
	\subsection{Cosmological Implications}
	\label{subsec:cosmological_implications}
	
	The assumption \(\alphaEM = 1\) could, in the T0 model \cite{pascher_galaxies_2025}:
	\begin{itemize}
		\item More strongly connect electromagnetic interactions with gravitation, as \(\Tfield\) emergently explains gravity as demonstrated in \cite{pascher_emergente_gravitation_2025}.
		\item Enable a unified energy description consistent with redshift due to energy loss to \(\Tfield\) \cite{pascher_messdifferenzen_2025}.
	\end{itemize}
	
	In the T0 model, wavelength-dependent redshift is described by the parameter \(\betaT^{\text{SI}} \approx 0.008\) in SI units, while in natural units \(\betaT = 1\) applies \cite{pascher_params_2025}. This is consistent with:
	\begin{equation}
		\label{eq:wavelength_redshift}
		z(\lambda) = z_0 (1 + \betaT \ln(\lambda/\lambda_0))
	\end{equation}
	
	When setting both \(\alphaEM = 1\) and \(\betaT^{\text{nat}} = 1\) simultaneously, significant deviations from Standard Model predictions arise (e.g., \(z(\lambda) \approx 3.3\) for \(\lambda/\lambda_0 = 10\)). These deviations should not be considered "unphysical" but may indicate a Standard Model bias in interpreting cosmological data \cite{pascher_alphabeta_2025}.
	
	\begin{figure}[h]
		\centering
		\begin{tikzpicture}
			\begin{axis}[
				xlabel={Energy [eV]},
				ylabel={Length [eV\(^{-1}\)]},
				xlabel style={font=\large},
				ylabel style={font=\large},
				tick label style={font=\normalsize},
				xmin=0, xmax=10,
				ymin=0, ymax=10,
				legend pos=north east,
				legend style={font=\large},
				grid=both,
				minor tick num=1
				]
				\addplot[blue, ultra thick, domain=0.1:10, samples=100] {1/x};
				\legend{\(L = E^{-1}\)}
			\end{axis}
		\end{tikzpicture}
		\caption{Relationship between energy and length in the \(\alphaEM = 1\) system.}
		\label{fig:energy_length}
	\end{figure}
	
	\section{Derivation of Planck's Quantum of Action}
	\label{sec:planck_quantum}
	
	Planck's quantum of action \(h\) forms a fundamental link between quantum mechanics and electrodynamics. In the T0 model, a deeper relationship between \(h\) and the electromagnetic vacuum constants can be established.
	
	\subsection{Connection to Electromagnetic Constants}
	\label{subsec:electromagnetic_constants}
	
	The speed of light in a vacuum is given by:
	\begin{equation}
		\label{eq:speed_of_light}
		c = \frac{1}{\sqrt{\mu_0 \varepsilon_0}}
	\end{equation}
	
	In the unified unit system with \(c = 1\):
	\begin{equation}
		\label{eq:mu_epsilon}
		\mu_0 \varepsilon_0 = 1
	\end{equation}
	
	A dimensionally consistent relationship between Planck's quantum of action and electromagnetic constants can be formulated via the vacuum impedance:
	\begin{equation}
		\label{eq:vacuum_impedance}
		Z_0 = \sqrt{\frac{\mu_0}{\varepsilon_0}} \approx 376.73 \, \Omega
	\end{equation}
	
	We can now define a fundamental length \(\lambda_0\) as:
	\begin{equation}
		\lambda_0 = \frac{c}{2\pi \nu_0}
	\end{equation}
	where \(\nu_0\) is a characteristic frequency. If we interpret \(\lambda_0\) as the Compton wavelength of an elementary particle, then:
	\begin{equation}
		\lambda_0 = \frac{h}{m_0 c}
	\end{equation}
	
	Combining these relationships and using the vacuum impedance, we obtain:
	\begin{equation}
		h = 2\pi m_0 c \lambda_0 = \frac{2\pi m_0 c^2}{\nu_0} = \frac{2\pi E_0}{\nu_0}
	\end{equation}
	
	where \(E_0 = m_0 c^2\) is the rest energy of the elementary particle.
	
	By introducing a fundamental coupling constant \(\kappa_h\), which represents the ratio between the vacuum impedance and a characteristic quantum impedance:
	\begin{equation}
		\kappa_h = \frac{Z_0}{Z_Q} = \frac{Z_0}{h/e^2} = \frac{e^2 Z_0}{h}
	\end{equation}
	
	we can write:
	\begin{equation}
		h = \frac{e^2 Z_0}{\kappa_h} = \frac{e^2}{\kappa_h} \sqrt{\frac{\mu_0}{\varepsilon_0}}
	\end{equation}
	
	In the unified unit system with \(\alphaEM = 1\), where \(e^2 = 4\pi\varepsilon_0\hbar c\), this becomes:
	\begin{equation}
		h = \frac{4\pi\varepsilon_0\hbar c}{\kappa_h} \sqrt{\frac{\mu_0}{\varepsilon_0}} = \frac{4\pi\hbar}{\kappa_h} \sqrt{\mu_0\varepsilon_0} \cdot c^2 \sqrt{\frac{\mu_0}{\varepsilon_0}} = \frac{4\pi\hbar}{\kappa_h} \mu_0 c^2
	\end{equation}
	
	With \(\kappa_h = 2\), we obtain the simple form:
	\begin{equation}
		h = 2\pi\hbar \mu_0 c^2
	\end{equation}
	
	This relationship demonstrates a deep connection between Planck's quantum of action, electromagnetic vacuum constants, and the structure of spacetime. In the unified unit system with \(\hbar = c = \mu_0 = 1\), we obtain \(h = 2\pi\), as expected.
	
	\subsection{Alternative Derivation Approaches}
	\label{subsec:alternative_derivations}
	
	The connection between \(h\) and electromagnetic constants can be established in various ways, all consistent with the unified unit system:
	
	\begin{enumerate}
		\item \textbf{De Broglie Wavelength:} \(\lambda = \frac{h}{p}\) leads, with \(p = \frac{\hbar}{c \cdot \Tfield}\) for massless particles, to a relationship between the intrinsic time field and wavelength.
		\item \textbf{Compton Scattering:} The Compton wavelength \(\lambda_C = \frac{h}{mc}\) is linked to intrinsic time via \(\lambda_C = \frac{h \cdot \Tfield}{c}\).
		\item \textbf{Uncertainty Relation:} The energy-time uncertainty \(\Delta E \Delta t \geq \frac{\hbar}{2}\) gains deeper significance in the T0 model, as time and energy are connected by the fundamental relation \(E = \frac{\hbar}{\Tfield}\).
	\end{enumerate}
	
	All these approaches confirm the fundamental role of \(h = 2\pi\) in the unified unit system and reveal the profound link between quantum mechanics, electrodynamics, and the time-mass duality of the T0 model as elaborated in \cite{pascher_zeit_2025}.
	
	\section{Alternative Formulations of the Fine-Structure Constant}
	\label{sec:alternative_alpha}
	
	\subsection{Standard Definition of the Fine-Structure Constant}
	\label{subsec:standard_alpha}
	
	The fine-structure constant \(\alphaEM\) is defined as:
	\begin{equation}
		\alphaEM = \frac{e^2}{4\pi \varepsilon_0 \hbar c} \approx \frac{1}{137.035999}
	\end{equation}
	
	This dimensionless constant characterizes the strength of electromagnetic interaction.
	
	\subsection{Using the Classical Electron Radius}
	\label{subsec:electron_radius}
	
	The classical electron radius is defined as:
	\begin{equation}
		r_e = \frac{e^2}{4\pi \varepsilon_0 m_e c^2}
	\end{equation}
	
	The Compton wavelength of the electron is:
	\begin{equation}
		\lambda_C = \frac{h}{m_e c} = \frac{2\pi\hbar}{m_e c}
	\end{equation}
	
	The fine-structure constant can be expressed as the ratio of these characteristic lengths:
	\begin{equation}
		\alphaEM = \frac{r_e}{\lambda_C/2\pi} = \frac{2\pi r_e}{\lambda_C}
	\end{equation}
	
	Substituting the definitions yields:
	\begin{equation}
		\alphaEM = \frac{2\pi \cdot \frac{e^2}{4\pi \varepsilon_0 m_e c^2}}{\frac{h}{m_e c}} = \frac{e^2}{2\varepsilon_0 h c}
	\end{equation}
	
	With \(h = 2\pi\hbar\), this again results in the standard definition:
	\begin{equation}
		\alphaEM = \frac{e^2}{4\pi \varepsilon_0 \hbar c}
	\end{equation}
	
	\section{Wien's Constant \(\alphaW\) in the Unified Unit System}
	\label{sec:wien_constant}
	
	Wien's constant \(\alphaW\) determines the relationship between the frequency of the radiation maximum and temperature in blackbody radiation:
	
	\begin{equation}
		\label{eq:wien_law}
		\nu_{\text{max}} = \alphaW \cdot \frac{k_B T}{h}
	\end{equation}
	
	with \(\alphaW \approx 2.821439\).
	
	In the unified unit system, we set \(k_B = 1\) and \(\hbar = 1\) (thus \(h = 2\pi\)), leading to the following relationship:
	
	\begin{equation}
		\nu_{\text{max}} = \alphaW \cdot \frac{T}{2\pi}
	\end{equation}
	
	With \(\alphaW = 1\), this becomes:
	
	\begin{equation}
		\nu_{\text{max}} = \frac{T}{2\pi}
	\end{equation}
	
	This relationship is consistent with the discussion in \cite{pascher_temp_2025} and demonstrates the direct proportionality between temperature and the frequency of the radiation maximum in the unified unit system.
	
	\section{The T0 Parameter \(\betaT\) in the Unified Unit System}
	\label{sec:beta_t}
	
	The T0 parameter \(\betaT\) is a fundamental dimensionless parameter in the T0 model, describing the coupling between the intrinsic time field \(\Tfield\) and other physical fields. It appears in various contexts and connects seemingly disparate physical phenomena, including the dynamic mass effects in photons as explored in \cite{pascher_photons_2025}.
	
	\subsection{Derivation from Field Equations and Fundamental Parameters}
	\label{subsec:beta_derivation}
	
	The parameter \(\betaT\) emerges naturally from the field equations of the T0 model, as detailed in the comprehensive field-theoretic treatment \cite{pascher_photons_2025}. The derivation from the underlying field equations shows that \(\betaT\) can be expressed in terms of fundamental physical constants:
	
	\begin{equation}
		\betaT^{\text{nat}} = \frac{\lambda_h^2 v^2}{16\pi^3 m_h^2 \xi}
	\end{equation}
	
	where \(\lambda_h\) is the Higgs self-coupling, \(v\) is the Higgs vacuum expectation value, \(m_h\) is the Higgs mass, and \(\xi\) is a dimensionless parameter defining the characteristic length scale \(r_0 = \xi \cdot l_P\), with \(l_P\) as the Planck length.
	
	This derivation from field equations, as presented in \cite{pascher_photons_2025}, demonstrates that \(\betaT\) is not an arbitrary parameter but emerges from the fundamental structure of the intrinsic time field \(\Tfield\) and its interactions with matter fields. The field-theoretic origin explains why the same parameter governs both cosmological redshift phenomena and quantum optical correlations.
	
	In SI units, \(\betaT^{\text{SI}} \approx 0.008\) was derived from cosmological observations and perturbative calculations \cite{pascher_params_2025}. In natural units, we have \(\betaT^{\text{nat}} = 1\), and these values are mathematically equivalent, representing the same physical reality expressed in different unit systems. The conversion between these values is given by:
	
	\begin{equation}
		\betaT^{\text{SI}} = \betaT^{\text{nat}} \cdot \frac{\xi \cdot l_{P,\text{SI}}}{r_{0,\text{SI}}}
	\end{equation}
	
	where \(\xi \approx 1.33 \times 10^{-4}\) \cite{pascher_photons_2025}.
	
	With \(\xi = \frac{\lambda_h^2 v^2}{16\pi^3 m_h^2}\), we obtain \(\betaT^{\text{nat}} = 1\) naturally from the field equation framework. The choice between using \(\betaT^{\text{SI}}\) and \(\betaT^{\text{nat}}\) depends solely on the unit system being employed and does not reflect any uncertainty in the theoretical foundation.
	
	\subsection{Physical Manifestations of \(\betaT\)}
	\label{subsec:beta_manifestations}
	
	The parameter \(\betaT\), derived from the fundamental field equations, manifests in various physical contexts:
	
	\begin{enumerate}
		\item \textbf{Temperature-Redshift Relation:} 
		\begin{equation}
			\label{eq:temp_redshift}
			T(z) = T_0 (1 + z) (1 + \betaT \ln(1 + z))
		\end{equation}
		In the unified unit system with \(\betaT = 1\), this simplifies to:
		\begin{equation}
			T(z) = T_0 (1 + z) (1 + \ln(1 + z))
		\end{equation}
		
		\item \textbf{Wavelength-Dependent Redshift:} 
		\begin{equation}
			z(\lambda) = z_0 (1 + \betaT \ln(\lambda/\lambda_0))
		\end{equation}
		With \(\betaT = 1\) in natural units:
		\begin{equation}
			z(\lambda) = z_0 (1 + \ln(\lambda/\lambda_0))
		\end{equation}
		This represents a distinctive prediction derived from the field equations that could be tested through high-precision spectroscopic observations \cite{pascher_photons_2025}.
		
		\item \textbf{Time Field-Higgs Coupling:} The parameter \(\betaT\) describes the coupling between the intrinsic time field \(\Tfield\) and the Higgs field \(\Phi\) as derived from the Lagrangian field equations:
		\begin{equation}
			\Tfield = \frac{\hbar}{y \langle \Phi \rangle c^2}
		\end{equation}
		where \(y\) is the Yukawa coupling.
		
		\item \textbf{Dynamic Photon Mass Effects:} As demonstrated through the field-theoretic approach in \cite{pascher_photons_2025}, the parameter \(\betaT\) connects to the energy-dependent intrinsic time of photons. For photons with effective mass \(m_\gamma = \omega\), the intrinsic time derived from the field equations is:
		\begin{equation}
			\Tfield_{\gamma} = \frac{\hbar}{\omega c^2} = \frac{1}{\omega}
		\end{equation}
		in natural units. The energy loss mechanism responsible for wavelength-dependent redshift, characterized by \(\betaT\) and derived from the field equations, influences the subtle time delays in quantum correlations between photons of different energies.
	\end{enumerate}
	
	These manifestations are examined in detail in \cite{pascher_params_2025}, \cite{pascher_alphabeta_2025}, and through the field-equation derivations in \cite{pascher_photons_2025}.
	
	\subsection{Field-Theoretic Origin of \(\betaT\) and Quantum Correlations}
	\label{subsec:field_theoretic_origin}
	
	The field-equation derivation of \(\betaT\) in \cite{pascher_photons_2025} reveals a deep connection between cosmological phenomena and quantum optical effects. The same field-theoretic framework that produces the wavelength-dependent redshift also predicts energy-dependent delays in quantum correlations.
	
	For entangled photon systems with frequencies \(\omega_1\) and \(\omega_2\), the field equations predict a difference in their intrinsic times:
	
	\begin{equation}
		\Delta \Tfield = \left|\frac{1}{\omega_1} - \frac{1}{\omega_2}\right|
	\end{equation}
	
	This prediction emerges directly from the field-theoretic treatment that also produces the \(\betaT\) parameter, demonstrating the unified nature of the T0 model. The mechanism underlying both effects is the coupling of the intrinsic time field \(\Tfield\) to energy-momentum, as encoded in the field equations \cite{pascher_photons_2025}.
	
	\subsection{Connection to Other Dimensionless Constants}
	\label{subsec:connection_constants}
	
	In the unified unit system with \(\alphaEM = \betaT = \alphaW = 1\), the field-equation derivation reveals fundamental relationships between seemingly different physical phenomena. The relationship between \(\betaT\) and \(\alphaEM\) emerges from the field equations through the coupling structure:
	
	\begin{equation}
		\betaT \cdot \alphaEM = \frac{\lambda_h^2 v^2 e^2}{64\pi^4\varepsilon_0\hbar c m_h^2}
	\end{equation}
	
	In the unified unit system with \(\betaT = \alphaEM = 1\), this relationship becomes unity, suggesting a deeper unity between electromagnetic and time-field interactions as revealed by the field equations. This connection is explored in detail through the field-theoretic framework in \cite{pascher_photons_2025}.
	
	\subsection{Experimental Validation from Field-Equation Predictions}
	\label{subsec:beta_experimental}
	
	The field-equation derivation of \(\betaT\) leads to specific, testable predictions that distinguish the T0 model from alternative theories:
	
	\begin{itemize}
		\item \textbf{Cross-scale validation:} The same \(\betaT\) value derived from field equations should appear in both cosmological redshift measurements and quantum optical experiments, providing a cross-validation of the field-theoretic framework.
		
		\item \textbf{Energy-dependent correlations:} The field equations predict specific energy dependencies in quantum correlations that can be tested with entangled photons of different frequencies \cite{pascher_photons_2025}.
		
		\item \textbf{Modified dispersion relations:} Field-equation corrections to standard dispersion relations provide testable predictions for high-energy astrophysical observations.
		
		\item \textbf{Wavelength-dependent cosmological observables:} The field-derived formula \(z(\lambda) = z_0 (1 + \betaT \ln(\lambda/\lambda_0))\) offers precise predictions for multi-wavelength astronomical observations.
	\end{itemize}
	
	These experimental tests directly probe the field-theoretic foundations of the T0 model, as detailed in the comprehensive field-equation treatment of \cite{pascher_photons_2025}.
	
	\section{Derivation of the Gravitational Constant \(G\)}
	\label{sec:gravitational_constant}
	
	In Planck units:
	\begin{equation}
		G = \frac{\hbar c}{m_P^2}
	\end{equation}
	
	In the T0 model, gravitation emerges from the gradients of the intrinsic time field \(\Tfield\). In the unified unit system with \(G = 1\), the gravitational potential is derived as in \cite{pascher_emergente_gravitation_2025}:
	
	\begin{equation}
		\label{eq:grav_potential}
		\Phi(r) = -\frac{M}{r} + \kappa r
	\end{equation}
	
	where the first term corresponds to the Newtonian potential, and the second term arises from the global variation of the time field. This linear term accounts for effects attributed to dark energy in the Standard Model, as explained in detail in \cite{pascher_galaxies_2025}.
	
	\section{Speculative Extensions of the T0 Model}
	\label{sec:speculative}
	
	The T0 model with its time-mass duality opens possibilities for speculative extensions that might go beyond the known limits of standard physics. Although not yet experimentally confirmed, these extensions offer conceptually intriguing perspectives.
	
	\subsection{Modified Energy-Momentum Relationship}
	\label{subsec:modified_energy_momentum}
	
	In the unified unit system, the relativistic energy-momentum relationship can be extended by an additional term:
	
	\begin{equation}
		\label{eq:modified_energy_momentum}
		E^2 = m^2 + p^2 + \frac{\alpha_c p^4}{E_P^2}
	\end{equation}
	
	where \(\alpha_c\) is a dimensionless constant characterizing the strength of the modification, and \(E_P\) is the Planck energy. This modification is motivated by various approaches:
	
	\begin{enumerate}
		\item \textbf{Time Field Fluctuations:} In the T0 model, fluctuations of the intrinsic time field \(\Tfield\) at high energies could lead to deviations from the standard dispersion relation. The parameter \(\alpha_c\) quantifies the coupling between these fluctuations and particle dynamics.
		\item \textbf{Quantum Gravity Effects:} Near the Planck scale, deviations due to quantum gravity effects are expected. The \(p^4\) dependence corresponds to a natural extension of the standard relation by Planck-scale corrections.
		\item \textbf{Consistency with Time-Mass Duality:} The modification can be understood as a natural consequence of the fundamental relation \(m = \frac{\hbar}{\Tfield c^2}\) at high energies, where the structure of the time field might deviate from its macroscopic description.
	\end{enumerate}
	
	Based on theoretical considerations and consistency with current experimental limits, we expect \(|\alpha_c| \lesssim 10^{-2}\).
	
	\subsection{Physics Beyond the Speed of Light}
	\label{subsec:beyond_c}
	
	In the T0 model, the conventional light speed barrier (\(v \leq c\)) might be an emergent property resulting from the macroscopic structure of the time field, rather than a fundamental limit. Under certain conditions, \(v > c\) might be possible without violating causality:
	
	\begin{enumerate}
		\item \textbf{Modified Causal Structure:} Causality in the T0 model is determined by the structure of the time field \(\Tfield\), not solely by the light cone structure. With strong gradients of \(\Tfield\), the effective causal structure could be altered.
		\item \textbf{Mass-Dependent Maximum Velocity:} Since \(m = \frac{\hbar}{\Tfield c^2}\), the effective maximum velocity might be mass-dependent, given by \(v_{\text{max}}(m) = c \cdot f(m \cdot \Tzero)\), where \(f\) is a function to be determined.
		\item \textbf{Tunneling Effects in the Time Field:} Quantum mechanical tunneling through "time barriers" could enable apparent superluminal phenomena, similar to quantum tunneling through energy barriers.
	\end{enumerate}
	
	These possibilities do not contradict relativity but extend it within a new conceptual framework where Lorentz invariance is an emergent property, not a fundamental principle, as discussed in \cite{pascher_zeit_masse_2025}.
	
	\subsection{Experimental Signatures}
	\label{subsec:experimental_signatures}
	
	Although speculative, these extensions could leave experimental signatures:
	
	\begin{enumerate}
		\item \textbf{Energy-Dependent Speed of Light:} The modified energy-momentum relationship leads to an energy-dependent speed of light \(c(E) \approx c (1 - \alpha_c \frac{E^2}{2E_P^2})\), testable through observations of high-energy cosmic rays or gamma-ray bursts.
		\item \textbf{Modified Compton Scattering:} The scattering of high-energy photons on electrons could show deviations from standard behavior, detectable through precision measurements.
		\item \textbf{Unexpected Quantum Correlations:} If the causal structure is modified by the time field, quantum correlations might exhibit patterns beyond the predictions of standard quantum mechanics.
	\end{enumerate}
	
	These speculative extensions offer conceptually intriguing possibilities for resolving fundamental problems in theoretical physics but require experimental evidence to be accepted as part of the established scientific framework.
	
	\section{Dimensional Analysis with SI Units}
	\label{sec:dimensional_analysis}
	
	\subsection{Verification of Dimensional Consistency}
	\label{subsec:dimensional_consistency}
	
	The dimensions of the derived quantities are verified:
	
	\begin{center}
		\begin{tabular}{lcc}
			\toprule
			\textbf{Quantity} & \textbf{SI Units} & \textbf{Natural Units} \\
			\midrule
			Length \(L\) & \si{\meter} & \(\text{Energy}^{-1}\) \\
			Time \(T\) & \si{\second} & \(\text{Energy}^{-1}\) \\
			Mass \(M\) & \si{\kilo\gram} & \(\text{Energy}\) \\
			Charge \(e\) & \si{\coulomb} & \(\sqrt{\alphaEM}\) \\
			\(G\) & \si{\meter^3\kilo\gram^{-1}\second^{-2}} & \(\text{Energy}^{-2}\) \\
			\(\varepsilon_0\) & \si{\farad\per\meter} & \(\text{Energy}^{-2}\) \\
			\(\mu_0\) & \si{\henry\per\meter} & \(\text{Energy}^{-2}\) \\
			\(h\) & \(\SI{6.62607015e-34}{\joule\second} = \si{\kilo\gram \meter\squared\per\second}\) & \(2\pi\) (dimensionless) \\
			\bottomrule
		\end{tabular}
	\end{center}
	
	This dimensional analysis confirms the consistency of the unified unit system with SI units, as elaborated in \cite{pascher_alphabeta_2025}.
	
	\subsection{Agreement of Empirical and Theoretical Values}
	\label{subsec:empirical_theoretical}
	
	The theoretical speed of light
	\begin{equation}
		c_{theor} = \frac{1}{\sqrt{\mu_0 \varepsilon_0}}
	\end{equation}
	matches \(c = \SI{299792458}{\meter\per\second}\) when \(\mu_0 = 4\pi \times 10^{-7} \, \si{\henry\per\meter}\) and \(\varepsilon_0 = 8.8541878128 \times 10^{-12} \, \si{\farad\per\meter}\) are used.
	
	\section{Representation as Planck Quantities}
	\label{sec:planck_quantities}
	
	In the unified unit system, physical quantities become dimensionless:
	\begin{align}
		\tilde{m} &= \frac{m}{m_P}, \\
		\tilde{L} &= \frac{L}{l_P}, \\
		\tilde{t} &= \frac{t}{t_P}.
	\end{align}
	
	This representation is particularly useful for quantum gravity considerations, as explored in \cite{pascher_planck_2025}.
	
	\section{Implications for Photons in the T0 Model}
	\label{sec:photons}
	
	\subsection{Dynamic, frequency-dependent Mass of Photons}
	\label{subsec:photon_dynamic_mass}
	
	In natural units (\(c = 1\), \(\hbar = 1\)), for photon energy:
	\begin{equation}
		E = \omega,
	\end{equation}
	and with \(E = m\) in the unified unit system, a frequency-dependent effective mass follows:
	\begin{equation}
		m_{\gamma} = \omega.
	\end{equation}
	
	This assignment of an effective mass to photons is not entirely new—various theoretical frameworks have considered this possibility, particularly in the context of modified electrodynamics. The T0 model approaches this from a unique perspective, viewing the photon's effective mass as a natural consequence of the fundamental time-mass duality principle \cite{pascher_photons_2025}.
	
	\subsection{Unified Treatment of Massive and Massless Particles}
	\label{subsec:unified_treatment}
	
	In the T0 model, the intrinsic time of a photon is:
	\begin{equation}
		\Tfield_{\gamma} = \frac{\hbar}{m_{\gamma} c^2} = \frac{\hbar}{\omega c^2} = \frac{1}{\omega}
	\end{equation}
	where the last equality applies in natural units with \(\hbar = c = 1\). This confirms the fundamental time-mass duality \(m = \frac{\hbar}{\Tfield c^2}\) for photons.
	
	To unify the treatment of massive particles and photons within a single framework, we define the intrinsic time field as:
	\begin{equation}
		\Tfield = \frac{\hbar}{\max(m c^2, \omega)}
	\end{equation}
	
	For massive particles at rest or low velocities, \(m c^2\) dominates and we recover the standard T0 model relation. For photons or ultra-relativistic particles, \(\omega\) becomes the determining factor. This unified treatment is developed in detail in \cite{pascher_photons_2025}.
	
	\subsection{Implications for Quantum Nonlocality}
	\label{subsec:quantum_nonlocality}
	
	The energy-dependent intrinsic time \(\Tfield\) for photons leads to interesting implications for quantum entanglement. In a system with entangled photons of different frequencies \(\omega_1\) and \(\omega_2\), the difference in their intrinsic times would be:
	\begin{equation}
		\Delta \Tfield = \left|\frac{1}{\omega_1} - \frac{1}{\omega_2}\right|
	\end{equation}
	
	This time difference suggests that quantum correlations between entangled photons might not be truly instantaneous but could experience a slight delay related to their energy difference. This perspective suggests that quantum nonlocality might emerge from intrinsic time differences rather than representing true action at a distance, potentially resolving the tension with special relativity regarding instantaneous action at a distance \cite{pascher_photons_2025}.
	
	\subsection{Connection to the \(\betaT\) Parameter}
	\label{subsec:beta_t_connection}
	
	In the T0 model, the wavelength-dependent component of redshift is characterized by the parameter \(\betaT\), with \(\betaT^{\text{SI}} \approx 0.008\) in SI units and \(\betaT^{\text{nat}} = 1\) in natural units \cite{pascher_params_2025}. These values are mathematically equivalent, representing the same physical reality expressed in different unit systems.
	
	The energy-dependent intrinsic time of photons could have a connection to this parameter, as both are based on the fundamental time-mass duality of the T0 model. The mechanism of energy losses to the time field, responsible for wavelength-dependent redshift, might also influence the subtle time delays in quantum correlations between photons of different energies \cite{pascher_photons_2025}.
	
	\subsection{Experimental Verification}
	\label{subsec:experimental_verification}
	
	The concept of dynamic mass for photons in the T0 model leads to several experimentally testable predictions:
	
	\begin{itemize}
		\item \textbf{Frequency-dependent Bell Tests:} Experiments could be designed to measure potential time delays in quantum correlations between entangled photons of different frequencies. The predicted delay \(\Delta \Tfield = \left|\frac{1}{\omega_1} - \frac{1}{\omega_2}\right|\) would be extremely small but might be detectable with ultra-precise timing measurements in quantum optics.
		
		\item \textbf{Wavelength-dependent Redshift:} The formula \(z(\lambda) = z_0 \left(1 + \betaT \ln\frac{\lambda}{\lambda_0}\right)\) predicts a characteristic wavelength dependence of cosmological redshift that could be tested through high-precision spectroscopic observations of distant sources across multiple wavelength bands.
		
		\item \textbf{High-energy Photon Propagation:} The energy-dependent intrinsic time could lead to subtle energy-dependent propagation effects for high-energy gamma rays traveling over cosmological distances, as discussed in \cite{pascher_photons_2025}.
	\end{itemize}
	
	\subsection{Modified Dispersion Relation}
	\label{subsec:modified_dispersion}
	
	The T0 model with dynamic photon mass suggests the possibility of a modified dispersion relation for photons:
	\begin{equation}
		E^2 = (m_{\gamma} c^2)^2 + (p c)^2 + \alpha_c \frac{p^4 c^2}{E_P^2}
	\end{equation}
	
	where \(\alpha_c\) is a dimensionless coupling constant and \(E_P\) is the Planck energy. The additional term represents a quantum gravity correction that becomes significant only at very high energies. This modified relation could explain potential anomalies in the propagation of ultra-high-energy cosmic rays and gamma rays, as analyzed in detail in \cite{pascher_photons_2025}.
	
	\section{Considerations Beyond the Planck Scale}
	\label{sec:beyond_planck}
	
	\subsection{Absolute Time and Intrinsic Time in the T0 Model}
	\label{subsec:absolute_intrinsic}
	
	The T0 model unifies two complementary perspectives:
	\begin{itemize}
		\item \textbf{Absolute Time Perspective:} Time \(\Tzero\) is absolute and constant, while mass varies: \(m = \frac{\hbar}{\Tfield c^2}\).
		\item \textbf{Intrinsic Time Perspective:} \(\Tfield = \frac{\hbar}{m c^2}\) leads to the modified Schrödinger equation: \(i\hbar \Tfield \frac{\partial \psi}{\partial t} = \hat{H} \psi\).
	\end{itemize}
	
	This duality is conceptually akin to wave-particle duality and reveals deeper connections between time, mass, and energy, as explained in \cite{pascher_zeit_2025}.
	
	\subsection{Connection to Planck Units}
	\label{subsec:planck_connection}
	
	For masses near the Planck mass (\(m \approx m_P\)), the intrinsic time \(\Tfield\) approaches the Planck time \(t_P\):
	\begin{equation}
		\Tfield = \frac{\hbar}{m c^2} \approx \frac{\hbar}{m_P c^2} = \frac{\hbar}{\sqrt{\hbar c/G} \cdot c^2} = \sqrt{\frac{\hbar G}{c^5}} = t_P
	\end{equation}
	
	This relation suggests that the Planck time might be a natural threshold for intrinsic time, potentially offering new insights into quantum gravity, as detailed in \cite{pascher_planck_2025}.
	
	\section{Consequences of a Fully Unified Unit System}
	\label{sec:fully_unified}
	
	Setting \(\alphaEM = \betaT = \alphaW = 1\) in the unified unit system leads to profound conceptual simplifications:
	
	\begin{enumerate}
		\item \textbf{Electrodynamics:} Electric charges become dimensionless, and electromagnetic equations take a more elegant form.
		\item \textbf{Thermodynamics:} Temperature becomes directly proportional to frequency, enabling a unified description of thermal and quantum phenomena.
		\item \textbf{Gravitation:} Gravity naturally emerges from the gradients of the time field without requiring additional coupling constants, as shown in \cite{pascher_emergente_gravitation_2025}.
		\item \textbf{Emergent Spacetime:} Spacetime geometry can be understood as an emergent phenomenon from the more fundamental time field.
	\end{enumerate}
	
	\subsection{Philosophical Implications}
	\label{subsec:philosophical}
	
	\begin{itemize}
		\item Energy as the most fundamental property of reality \cite{Wilczek2008}, supported in the T0 model by absolute time and variable mass.
		\item Space and time as emergent properties of an energy field \cite{Verlinde2011}, compatible with \(\Tfield\) as the foundational field.
	\end{itemize}
	
	\section{Summary and Outlook}
	\label{sec:summary}
	
	The unified unit system of the T0 model with \(\hbar = c = G = \alphaEM = \betaT = \alphaW = 1\) provides an elegant framework for unifying fundamental interactions. Deriving fundamental constants from this system suggests they may not truly be "fundamental" but rather artifacts of our chosen unit systems. By setting \(\alphaEM = 1\), energy becomes the fundamental unit, revealing in the T0 model a deeper unity of time, mass, and gravitation. This simplification aligns with the general principle that fundamental dimensionless parameters should take simple values in a fully natural formulation. Similar to setting \(\betaT^{\text{nat}} = 1\), \(\alphaEM = 1\) leads to a conceptually clearer theory where the dimensions of all physical quantities can be reduced to a single fundamental dimension (energy). For a comprehensive analysis of the consistency of both simplifications, refer to \cite{pascher_alphabeta_2025}.
	
	The time-mass duality \(m = \frac{\hbar}{\Tfield c^2}\) uncovers profound connections between seemingly disparate physical phenomena and could be the key to a more comprehensive understanding of nature beyond known limits.
	
	Future research should focus on experimental tests of the T0 model's specific predictions, particularly:
	
	\begin{itemize}
		\item Wavelength-dependent redshift \(z(\lambda) = z_0 (1 + \betaT \ln(\lambda/\lambda_0))\) as elaborated in \cite{pascher_messdifferenzen_2025}.
		\item Modified temperature-redshift relation from \cref{eq:temp_redshift}, as discussed in \cite{pascher_temp_2025}.
		\item Emergence of gravitation from the time field as derived in \cite{pascher_emergente_gravitation_2025}.
		\item Connection between quantum correlations and time field geometry as explored in \cite{pascher_feldtheorie_2025}.
		\item Energy-dependent quantum correlations in entangled photon systems as predicted in \cite{pascher_photons_2025}.
	\end{itemize}
	
	These tests could revolutionize our fundamental understanding of space, time, and matter, paving the way for a complete unification of physics.
	
	\begin{thebibliography}{99}
		\bibitem{pascher_zeit_2025} Pascher, J. (2025). \href{https://github.com/jpascher/T0-Time-Mass-Duality/tree/main/2/pdf/English/ZeitEmergentQMEn.pdf}{Time as an Emergent Property in Quantum Mechanics: A Connection Between Relativity, Fine-Structure Constant, and Quantum Dynamics}. March 23, 2025. \textit{See \cref{subsec:beyond_lightspeed,subsec:alternative_derivations,subsec:absolute_intrinsic} for key concepts on intrinsic time.}
		
		\bibitem{pascher_galaxies_2025} Pascher, J. (2025). \href{https://github.com/jpascher/T0-Time-Mass-Duality/tree/main/2/pdf/English/MassVarGalaxienEn.pdf}{Mass Variation in Galaxies: An Analysis in the T0 Model with Emergent Gravitation}. March 30, 2025. \textit{Referenced in \cref{subsec:cosmological_implications,sec:gravitational_constant} for galaxy dynamics.}
		
		\bibitem{pascher_messdifferenzen_2025} Pascher, J. (2025). \href{https://github.com/jpascher/T0-Time-Mass-Duality/tree/main/2/pdf/English/MessdifferenzenT0StandardEn.pdf}{Compensatory and Additive Effects: An Analysis of Measurement Differences Between the T0 Model and the $\Lambda$CDM Standard Model}. April 2, 2025. \textit{Referenced in \cref{subsec:cosmological_implications,sec:summary} for cosmological implications.}
		
		\bibitem{pascher_params_2025} Pascher, J. (2025). \href{https://github.com/jpascher/T0-Time-Mass-Duality/tree/main/2/pdf/English/ZeitMasseT0ParamsEn.pdf}{Time-Mass Duality Theory (T0 Model): Derivation of Parameters $\kappa$, $\alpha$, and $\beta$}. April 4, 2025. \textit{See \cref{subsec:cosmological_implications,subsec:beta_derivation,subsec:beta_manifestations} for parameter derivations.}
		
		\bibitem{pascher_higgs_2025} Pascher, J. (2025). \href{https://github.com/jpascher/T0-Time-Mass-Duality/tree/main/2/pdf/English/MathHiggsZeitMasseEn.pdf}{Mathematical Formulation of the Higgs Mechanism in Time-Mass Duality}. March 28, 2025. \textit{Relevant for \cref{subsec:connection_constants} on Higgs-time field interactions.}
		
		\bibitem{pascher_alphabeta_2025} Pascher, J. (2025). \href{https://github.com/jpascher/T0-Time-Mass-Duality/tree/main/2/pdf/English/Alpha1Beta1KonsistenzEn.pdf}{Unified Unit System in the T0 Model: The Consistency of $\alpha = 1$ and $\beta = 1$}. April 5, 2025. \textit{Essential for \cref{subsec:natural_units,subsec:cosmological_implications,subsec:beta_manifestations,subsec:dimensional_consistency,sec:summary} on the unified unit system.}
		
		\bibitem{pascher_temp_2025} Pascher, J. (2025). \href{https://github.com/jpascher/T0-Time-Mass-Duality/tree/main/2/pdf/English/TempEinheitenCMBEn.pdf}{Adjustment of Temperature Units in Natural Units and CMB Measurements}. April 2, 2025. \textit{Referenced in \cref{subsec:alpha_one,sec:wien_constant,sec:summary} for temperature scaling.}
		
		\bibitem{pascher_emergente_gravitation_2025} Pascher, J. (2025). \href{https://github.com/jpascher/T0-Time-Mass-Duality/tree/main/2/pdf/English/EmergentGravT0En.pdf}{Emergent Gravitation in the T0 Model: A Comprehensive Derivation}. April 1, 2025. \textit{Key reference for \cref{subsec:cosmological_implications,sec:gravitational_constant,sec:fully_unified,sec:summary} on emergent gravitation.}
		
		\bibitem{pascher_planck_2025} Pascher, J. (2025). \href{https://github.com/jpascher/T0-Time-Mass-Duality/tree/main/2/pdf/English/JenseitsPlanckEn.pdf}{Real Consequences of Reformulating Time and Mass in Physics: Beyond the Planck Scale}. March 24, 2025. \textit{Referenced in \cref{sec:planck_quantities,subsec:planck_connection} for Planck scale implications.}
		
		\bibitem{pascher_lagrange_2025} Pascher, J. (2025). \href{https://github.com/jpascher/T0-Time-Mass-Duality/tree/main/2/pdf/English/MathZeitMasseLagrangeEn.pdf}{From Time Dilation to Mass Variation: Mathematical Core Formulations of Time-Mass Duality Theory}. March 29, 2025. \textit{Important for \cref{subsec:natural_units,subsec:simplified_equations} on mathematical formulations.}
		
		\bibitem{pascher_zeit_masse_2025} Pascher, J. (2025). \href{https://github.com/jpascher/T0-Time-Mass-Duality/tree/main/2/pdf/English/ZeitMasseNeuerBlickEn.pdf}{Time and Mass: A New Look at Old Formulas - and Liberation from Traditional Constraints}. March 22, 2025. \textit{Essential background for \cref{subsec:energy_concept,subsec:energy_fundamental,subsec:beyond_c} on time-mass duality.}
		
		\bibitem{pascher_photons_2025} Pascher, J. (2025). \href{https://github.com/jpascher/T0-Time-Mass-Duality/tree/main/2/pdf/English/DynMassePhotonenNichtlokalEn.pdf}{Dynamic Mass of Photons and Its Implications for Nonlocality in the T0 Model}. March 25, 2025. \textit{Referenced in \cref{sec:beta_t,sec:photons} for photon dynamic mass and field-equation derivations of $\betaT$.}
		
		\bibitem{pascher_feldtheorie_2025} Pascher, J. (2025). \href{https://github.com/jpascher/T0-Time-Mass-Duality/tree/main/2/pdf/English/FeldtheorieQuantenEn.pdf}{Field Theory and Quantum Correlations: A New Perspective on Instantaneity}. March 28, 2025. \textit{Relevant for \cref{subsec:causality,sec:summary} on quantum correlations.}
		
		\bibitem{einstein1905} Einstein, A. (1905). Does the Inertia of a Body Depend Upon Its Energy Content? \textit{Annalen der Physik}, 323(13), 639-641.
		
		\bibitem{planck1899} Planck, M. (1899). On Irreversible Radiation Processes. \textit{Proceedings of the Royal Prussian Academy of Sciences in Berlin}, 5, 440-480.
		
		\bibitem{Feynman1985} Feynman, R. P. (1985). \textit{QED: The Strange Theory of Light and Matter}. Princeton University Press.
		
		\bibitem{Duff2002} Duff, M. J., Okun, L. B., \& Veneziano, G. (2002). \textit{Trialogue on the Number of Fundamental Constants}. \textit{Journal of High Energy Physics}, 2002(03), 023.
		
		\bibitem{Verlinde2011} Verlinde, E. (2011). \textit{On the Origin of Gravity and the Laws of Newton}. \textit{Journal of High Energy Physics}, 2011(4), 29.
		
		\bibitem{Wilczek2008} Wilczek, F. (2008). \textit{The Lightness of Being: Mass, Ether, and the Unification of Forces}. Basic Books.
		
		\bibitem{Rubin1980} Rubin, V. C., \& Ford Jr, W. K. (1980). Rotation of the Andromeda Nebula from a Spectroscopic Survey of Emission Regions. \textit{The Astrophysical Journal}, 159, 379.
		
		\bibitem{McGaugh2016} McGaugh, S. S., Lelli, F., \& Schombert, J. M. (2016). Radial Acceleration Relation in Rotationally Supported Galaxies. \textit{Physical Review Letters}, 117(20), 201101.
	\end{thebibliography}
	
\end{document}