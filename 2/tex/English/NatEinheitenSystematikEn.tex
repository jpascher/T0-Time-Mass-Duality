\documentclass[11pt,a4paper]{article}
\usepackage[utf8]{inputenc}
\usepackage[T1]{fontenc}
\usepackage[english]{babel}
\usepackage{lmodern}
\usepackage{amsmath}
\usepackage{amssymb}
\usepackage{physics}
\usepackage{hyperref}
\usepackage{booktabs}
\usepackage{array}
\usepackage[left=2cm,right=2cm,top=2cm,bottom=2cm]{geometry}
\usepackage{fancyhdr}
\usepackage{siunitx}
\usepackage{amsthm}
\usepackage{adjustbox}

% Custom Commands
\newcommand{\tablescale}{0.9}

% Theorem environments
\newtheorem{definition}{Definition}[section]

% Header and footer
\pagestyle{fancy}
\fancyhf{}
\fancyhead[L]{Johann Pascher}
\fancyhead[R]{Natural Unit Systems}
\fancyfoot[C]{\thepage}
\renewcommand{\headrulewidth}{0.4pt}
\renewcommand{\footrulewidth}{0.4pt}

\hypersetup{
	colorlinks=true,
	linkcolor=blue,
	citecolor=blue,
	urlcolor=blue,
	pdftitle={Natural Unit Systems: Universal Energy Conversion and Fundamental Length Scale Hierarchy},
	pdfauthor={Johann Pascher}
}

\title{Natural Unit Systems:\\
	Universal Energy Conversion and\\
	Fundamental Length Scale Hierarchy}
\author{Johann Pascher\\
	Department of Communication Technology\\
	Higher Technical Federal Institute (HTL), Leonding, Austria\\
	\texttt{johann.pascher@gmail.com}}
\date{April 13, 2025}

\begin{document}
	
	\maketitle
	
	\begin{abstract}
		This foundational document establishes the natural unit system used throughout the T0 model framework. By setting fundamental constants to unity and adopting energy as the base dimension, all physical quantities can be expressed as powers of energy. This document serves as the reference for unit conversions and dimensional analysis across all T0 model applications.
	\end{abstract}
	
	\tableofcontents
	\newpage
	
	\section{List of Symbols and Notation}
	
	{\small
		\begin{table}[htbp]
			\centering
			\begin{adjustbox}{width=0.98\textwidth}
				\begin{tabular}{lll}
					\toprule
					\textbf{Symbol} & \textbf{Meaning} & \textbf{Units/Notes} \\
					\midrule
					\multicolumn{3}{c}{\textbf{Fundamental Constants}} \\
					$\hbar$ & Reduced Planck constant & Set to 1 \\
					$c$ & Speed of light & Set to 1 \\
					$G$ & Gravitational constant & Set to 1 \\
					$k_B$ & Boltzmann constant & Set to 1 \\
					$e$ & Elementary charge & $[E^{1/2}]$ \\
					$\varepsilon_0, \mu_0$ & Vacuum permittivity, permeability & Set to 1 in QED units \\
					\midrule
					\multicolumn{3}{c}{\textbf{Units}} \\
					$l_P, t_P, m_P, E_P, T_P$ & Planck length, time, mass, energy, temp. & Natural base units \\
					$m_e, a_0, E_h$ & Electron mass, Bohr radius, Hartree energy & Atomic units \\
					\midrule
					\multicolumn{3}{c}{\textbf{Coupling Constants}} \\
					$\alpha_{\text{EM}}$ & Fine-structure constant & $e^2/(4\pi) = 1$ (nat.), $\approx 1/137$ (SI) \\
					$\alpha_s, \alpha_W, \alpha_G$ & Strong, weak, gravitational coupling & Dimensionless \\
					\midrule
					\multicolumn{3}{c}{\textbf{Physical Quantities}} \\
					$E, m, \Theta$ & Energy, mass, temperature & $[E]$ \\
					$L, r, \lambda, t$ & Length, radius, wavelength, time & $[E^{-1}]$ \\
					$p, \omega, \nu$ & Momentum, angular freq., frequency & $[E]$ \\
					$F$ & Force & $[E^2]$ \\
					$v$ & Velocity & Dimensionless \\
					$q$ & Electric charge & $[E^{1/2}]$ \\
					\midrule
					\multicolumn{3}{c}{\textbf{Special Scales \& Notation}} \\
					$r_0, \xi$ & T0 length, scaling parameter & $\xi l_P, \xi \approx 1.33 \times 10^{-4}$ \\
					$\lambda_{C,e}, r_e$ & Compton wavelength, classical e radius & $\hbar/(m_e c), e^2/(4\pi\varepsilon_0 m_e c^2)$ \\
					$[X], [E^n]$ & Dimension of X, energy dimension & Dimensional analysis \\
					$\sim, \leftrightarrow$ & Approximately, conversion & Order of magnitude, units \\
					\bottomrule
				\end{tabular}
			\end{adjustbox}
			\caption{Symbols and notation}
			\label{tab:symbols}
		\end{table}
	}
	
	\newpage
	
	\section{Introduction}
	
	Natural units are unit systems where fundamental physical constants are set to unity to simplify calculations and reveal the underlying mathematical structure of physical laws. The most well-known systems are **Planck units** (for gravitation and quantum physics) and **atomic units** (for quantum chemistry).
	
	This document establishes the complete framework for the natural unit system used in the T0 model, which is based on Planck units with energy as the fundamental dimension. The key insight is that energy $[E]$ serves as the universal dimension from which all other physical quantities derive.
	
	\subsection{Comparison with Other Natural Unit Systems}
	
	\begin{table}[htbp]
		\centering
		\begin{adjustbox}{width=0.95\textwidth}
			\begin{tabular}{lllll}
				\toprule
				\textbf{System} & \textbf{Constants Set to 1} & \textbf{Base Units} & \textbf{Applications} & \textbf{Notes} \\
				\midrule
				Planck Units & $\hbar, c, G, k_B = 1$ & $l_P, t_P, m_P, E_P$ & Quantum gravity, cosmology & Universal significance \\
				Atomic Units & $m_e, e, \hbar, \frac{1}{4\pi\varepsilon_0} = 1$ & $a_0, E_h$ & Quantum chemistry, atoms & Chemistry applications \\
				Particle Physics & $\hbar, c = 1$ & GeV & High energy physics & Practical for colliders \\
				T0 Model & $\hbar, c, G, k_B = 1$ & Energy $[E]$ & Unified physics & Energy as base dimension \\
				\bottomrule
			\end{tabular}
		\end{adjustbox}
		\caption{Comparison of natural unit systems}
		\label{tab:unit_systems}
	\end{table}
	
	\section{Fundamentals of Natural Unit Systems}
	
	\subsection{Planck Units}
	
	The Planck units were proposed by Max Planck in 1899 \cite{planck1900,planck1906} and are based on the fundamental natural constants:
	\begin{align}
		G &= 1 \quad \text{(gravitational constant)} \\
		c &= 1 \quad \text{(speed of light)} \\
		\hbar &= 1 \quad \text{(reduced Planck constant)}
	\end{align}
	
	Planck recognized that these units \textit{``retain their meaning for all times and for all, including extraterrestrial and non-human cultures necessarily''} \cite{planck1900}.
	
	\subsection{Atomic Units}
	
	The atomic units, introduced by Hartree in 1927 \cite{hartree1957}, set:
	\begin{align}
		m_e &= 1 \quad \text{(electron mass)} \\
		e &= 1 \quad \text{(elementary charge)} \\
		\hbar &= 1 \\
		\frac{1}{4\pi\varepsilon_0} &= 1 \quad \text{(Coulomb constant)}
	\end{align}
	
	\subsection{Quantum Optical Units}
	
	For quantum field theory applications, quantum optical units are commonly used:
	\begin{align}
		c &= 1 \quad \text{(speed of light)} \\
		\hbar &= 1 \quad \text{(reduced Planck constant)} \\
		\varepsilon_0 &= 1 \quad \text{(permittivity)} \\
		\mu_0 &= 1 \quad \text{(permeability, because } c = 1/\sqrt{\varepsilon_0 \mu_0}\text{)}
	\end{align}
	
	\subsection{Advantages of Natural Units}
	
	Natural units offer several key advantages:
	\begin{itemize}
		\item **Simplified equations** (e.g., $E = m$ instead of $E = mc^2$)
		\item **No superfluous constants** in calculations
		\item **Universal scaling** for fundamental physics
		\item **Reveals fundamental relationships** between physical quantities
		\item **Provides dimensional consistency** checks
		\item **Eliminates arbitrary conversion factors**
		\item **Highlights the universal role** of energy
	\end{itemize}
	
	\section{Mathematical Proof of Energy Equivalence}
	
	\subsection{Fundamental Dimensional Relations}
	
	In natural units, all physical quantities have dimensions that can be expressed as powers of energy $[E]$ \cite{weinberg1995,peskin1995}:
	
	\begin{align}
		[L] &= [E]^{-1} \quad \text{(from } \hbar c = 1\text{)} \\
		[T] &= [E]^{-1} \quad \text{(from } \hbar = 1\text{)} \\
		[M] &= [E] \quad \text{(from } c = 1\text{)}
	\end{align}
	
	\subsection{Conversion of Fundamental Quantities}
	
	\textbf{Length:} From the relation $\hbar c = 1$ it follows:
	\begin{equation}
		[L] = \frac{[\hbar][c]}{[E]} = [E]^{-1}
	\end{equation}
	
	\textbf{Time:} From $\hbar = 1$ and $E = \hbar \omega$ it follows:
	\begin{equation}
		[T] = \frac{[\hbar]}{[E]} = [E]^{-1}
	\end{equation}
	
	\textbf{Mass:} From $E = mc^2$ and $c = 1$ it follows:
	\begin{equation}
		[M] = [E]
	\end{equation}
	
	\textbf{Velocity:} 
	\begin{equation}
		[v] = \frac{[L]}{[T]} = \frac{[E]^{-1}}{[E]^{-1}} = [E]^0 = \text{dimensionless}
	\end{equation}
	
	\textbf{Momentum:}
	\begin{equation}
		[p] = [M][v] = [E] \cdot [E]^0 = [E]
	\end{equation}
	
	\textbf{Force:}
	\begin{equation}
		[F] = [M][a] = [E] \cdot [E]^{-1} = [E]^2
	\end{equation}
	
	\textbf{Charge:} In Planck units from $F = \frac{1}{4\pi\varepsilon_0} \frac{q^2}{r^2}$:
	\begin{equation}
		[q] = [E]^{1/2}
	\end{equation}
	
	\subsection{Generalization}
	
	Any physical quantity $G$ can be represented as a product of powers of the fundamental constants:
	\begin{equation}
		G = c^a \cdot \hbar^b \cdot G^c \cdot k_B^d \cdot \ldots
	\end{equation}
	
	In natural units this becomes:
	\begin{equation}
		[G] = [E]^n \quad \text{for a specific } n \in \mathbb{Q}
	\end{equation}
	
	\begin{table}[htbp]
		\centering
		\begin{adjustbox}{width=0.9\textwidth}
			\begin{tabular}{lccc}
				\toprule
				\textbf{Physical Quantity} & \textbf{SI Dimension} & \textbf{Natural Dimension} & \textbf{Derivation} \\
				\midrule
				Energy & $[ML^2T^{-2}]$ & $[E]$ & Base dimension \\
				Mass & $[M]$ & $[E]$ & $E = mc^2, c = 1$ \\
				Temperature & $[\Theta]$ & $[E]$ & $E = k_BT, k_B = 1$ \\
				Length & $[L]$ & $[E^{-1}]$ & $l_P = \sqrt{\hbar G/c^3} = 1$ \\
				Time & $[T]$ & $[E^{-1}]$ & $t_P = \sqrt{\hbar G/c^5} = 1$ \\
				Momentum & $[MLT^{-1}]$ & $[E]$ & $p = mv, v = [E^0]$ \\
				Force & $[MLT^{-2}]$ & $[E^2]$ & $F = ma = [E][E] = [E^2]$ \\
				Power & $[ML^2T^{-3}]$ & $[E^2]$ & $P = E/t = [E]/[E^{-1}] = [E^2]$ \\
				Charge & $[AT]$ & $[E^0]$ & Set dimensionless \\
				Electric Field & $[MLT^{-3}A^{-1}]$ & $[E^2]$ & $\vec{E} = \vec{F}/q$ \\
				Magnetic Field & $[MT^{-2}A^{-1}]$ & $[E^2]$ & $\vec{B} = \vec{F}/(qv)$ \\
				\bottomrule
			\end{tabular}
		\end{adjustbox}
		\caption{Universal energy dimensions of physical quantities}
		\label{tab:energy_dimensions}
	\end{table}
	
	\subsection{Fundamental Relationships}
	
	The key relationships in natural units become:
	\begin{align}
		E &= m \quad \text{(mass-energy equivalence)} \\
		E &= T \quad \text{(temperature-energy equivalence)} \\
		[L] &= [T] = [E^{-1}] \quad \text{(space-time unity)} \\
		\omega &= E \quad \text{(frequency-energy equivalence)} \\
		p &= E \quad \text{(momentum-energy equivalence for massless particles)}
	\end{align}
	
	\section{Length Scale Hierarchy}
	
	\subsection{Standard Length Scales}
	
	Physical systems organize themselves around characteristic length scales:
	
	\begin{table}[htbp]
		\centering
		\begin{adjustbox}{width=0.95\textwidth}
			\begin{tabular}{lccc}
				\toprule
				\textbf{Scale} & \textbf{Symbol} & \textbf{SI Value (m)} & \textbf{Natural Units ($l_P = 1$)} \\
				\midrule
				Planck Length & $l_P$ & $1.616 \times 10^{-35}$ & $1$ \\
				Compton (electron) & $\lambda_{C,e}$ & $2.426 \times 10^{-12}$ & $1.5 \times 10^{23}$ \\
				Classical electron radius & $r_e$ & $2.818 \times 10^{-15}$ & $1.7 \times 10^{20}$ \\
				Bohr radius & $a_0$ & $5.292 \times 10^{-11}$ & $3.3 \times 10^{24}$ \\
				Nuclear scale & $\sim 10^{-15}$ & $10^{-15}$ & $6.2 \times 10^{19}$ \\
				Atomic scale & $\sim 10^{-10}$ & $10^{-10}$ & $6.2 \times 10^{24}$ \\
				Human scale & $\sim 1$ & $1$ & $6.2 \times 10^{34}$ \\
				Earth radius & $R_\oplus$ & $6.371 \times 10^6$ & $3.9 \times 10^{41}$ \\
				Solar System & $\sim 10^{12}$ & $10^{12}$ & $6.2 \times 10^{46}$ \\
				Galactic scale & $\sim 10^{21}$ & $10^{21}$ & $6.2 \times 10^{55}$ \\
				\bottomrule
			\end{tabular}
		\end{adjustbox}
		\caption{Standard length scales in natural units}
		\label{tab:length_scales}
	\end{table}
	
	\subsection{The T0 Length Scale}
	
	The T0 model introduces a sub-Planckian length scale:
	
	\begin{definition}[T0 Length]
		\begin{equation}
			r_0 = \xi \cdot l_P
		\end{equation}
		where $\xi \approx 1.33 \times 10^{-4}$ is a dimensionless parameter.
	\end{definition}
	
	This gives:
	\begin{align}
		r_0 &= \xi \cdot l_P = 1.33 \times 10^{-4} \times 1.616 \times 10^{-35}\,\text{m} \\
		&= 2.15 \times 10^{-39}\,\text{m}
	\end{align}
	
	In natural units with $l_P = 1$:
	\begin{equation}
		r_0 = \xi \approx 1.33 \times 10^{-4}
	\end{equation}
	
	\section{Unit Conversions}
	
	\subsection{Energy as Reference}
	
	Using the electronvolt (eV) as the practical energy unit:
	
	\begin{table}[htbp]
		\centering
		\begin{adjustbox}{width=0.9\textwidth}
			\begin{tabular}{lll}
				\toprule
				\textbf{Physical Quantity} & \textbf{Conversion to SI} & \textbf{Example (1 GeV)} \\
				\midrule
				Energy & $\SI{1}{\electronvolt} = \SI{1.602e-19}{\joule}$ & $\SI{1.602e-10}{\joule}$ \\
				Mass & $E(\text{eV}) \times \SI{1.783e-36}{\kilogram\per\electronvolt}$ & $\SI{1.783e-27}{\kilogram}$ \\
				Length & $E(\text{eV})^{-1} \times \SI{1.973e-7}{\meter\electronvolt}$ & $\SI{1.973e-16}{\meter}$ \\
				Time & $E(\text{eV})^{-1} \times \SI{6.582e-16}{\second\electronvolt}$ & $\SI{6.582e-25}{\second}$ \\
				Temperature & $E(\text{eV}) \times \SI{1.161e4}{\kelvin\per\electronvolt}$ & $\SI{1.161e13}{\kelvin}$ \\
				\bottomrule
			\end{tabular}
		\end{adjustbox}
		\caption{Conversion factors from natural to SI units}
		\label{tab:conversions}
	\end{table}
	
	\subsection{Planck Scale Conversions}
	
	Converting between Planck units and SI:
	
	\begin{table}[htbp]
		\centering
		\begin{adjustbox}{width=0.8\textwidth}
			\begin{tabular}{lll}
				\toprule
				\textbf{Planck Unit} & \textbf{Natural Value} & \textbf{SI Value} \\
				\midrule
				Length ($l_P$) & $1$ & $\SI{1.616e-35}{\meter}$ \\
				Time ($t_P$) & $1$ & $\SI{5.391e-44}{\second}$ \\
				Mass ($m_P$) & $1$ & $\SI{2.176e-8}{\kilogram}$ \\
				Energy ($E_P$) & $1$ & $\SI{1.220e19}{\giga\electronvolt}$ \\
				Temperature ($T_P$) & $1$ & $\SI{1.417e32}{\kelvin}$ \\
				\bottomrule
			\end{tabular}
		\end{adjustbox}
		\caption{Planck unit conversions}
		\label{tab:planck_conversions}
	\end{table}
	
	\section{Mathematical Framework}
	
	\subsection{Simplified Equations}
	
	In natural units, fundamental equations become elegantly simple:
	
	\subsubsection{Quantum Mechanics}
	\begin{align}
		\text{Schrödinger equation:} \quad & i\frac{\partial\psi}{\partial t} = H\psi \\
		\text{Uncertainty principle:} \quad & \Delta E \Delta t \geq \frac{1}{2} \\
		\text{de Broglie relation:} \quad & \lambda = \frac{1}{p}
	\end{align}
	
	\subsubsection{Special Relativity}
	\begin{align}
		\text{Mass-energy:} \quad & E = m \\
		\text{Energy-momentum:} \quad & E^2 = p^2 + m^2 \\
		\text{Lorentz factor:} \quad & \gamma = \frac{1}{\sqrt{1-v^2}}
	\end{align}
	
	\subsubsection{General Relativity}
	\begin{align}
		\text{Einstein equations:} \quad & G_{\mu\nu} = 8\pi T_{\mu\nu} \\
		\text{Schwarzschild radius:} \quad & r_s = 2M
	\end{align}
	
	\subsubsection{Electromagnetism}
	\begin{align}
		\text{Coulomb's law:} \quad & F = \frac{q_1 q_2}{4\pi r^2} \\
		\text{Fine structure constant:} \quad & \alpha = \frac{e^2}{4\pi}
	\end{align}
	
	\subsubsection{Thermodynamics}
	\begin{align}
		\text{Stefan-Boltzmann:} \quad & j = \sigma T^4 \\
		\text{Wien's law:} \quad & \lambda_{max} T = b \\
		\text{Boltzmann distribution:} \quad & P \propto e^{-E/T}
	\end{align}
	
	\section{Advantages and Applications}
	
	\subsection{Advantages of Natural Units}
	\begin{itemize}
		\item **Simplified equations** (e.g., $E = m$ instead of $E = mc^2$)
		\item **No superfluous constants** in calculations
		\item **Universal scaling** for fundamental physics
		\item **Reveals fundamental relationships** between physical quantities
		\item **Provides dimensional consistency** checks
		\item **Eliminates arbitrary conversion factors**
		\item **Highlights the universal role** of energy
	\end{itemize}
	
	\subsection{Disadvantages}
	\begin{itemize}
		\item **Unintuitive for macroscopic applications**
		\item **Conversion to SI requires knowledge** of fundamental constants
		\item **Initial unfamiliarity** for those used to SI units
		\item **Engineering preference** for practical SI units
	\end{itemize}
	
	\subsection{Practical Applications}
	\begin{itemize}
		\item Particle physics calculations
		\item Quantum field theory
		\item General relativity and cosmology
		\item High-energy astrophysics
		\item String theory and quantum gravity
		\item Fundamental constant relationships
	\end{itemize}
	
	\section{Working with Natural Units}
	
	\subsection{Working with Natural Units}
	
	To convert a calculation from SI to natural units:
	\begin{enumerate}
		\item Express all quantities in terms of energy (eV or GeV)
		\item Set $\hbar = c = G = k_B = 1$
		\item Perform the calculation
		\item Convert results back to SI if needed
	\end{enumerate}
	
	\subsection{Dimensional Check}
	
	Always verify dimensional consistency:
	\begin{itemize}
		\item All terms in an equation must have the same energy dimension
		\item Check that exponents are consistent
		\item Use dimensional analysis to verify results
	\end{itemize}
	
	\subsection{Fundamental Forces in Natural Units}
	
	The four fundamental forces can be characterized by their dimensionless coupling constants:
	
	\begin{table}[htbp]
		\centering
		\begin{adjustbox}{width=0.9\textwidth}
			\begin{tabular}{llll}
				\toprule
				\textbf{Force} & \textbf{Dimensionless Coupling} & \textbf{Typical Value} & \textbf{Range} \\
				\midrule
				Electromagnetic & $\alpha_{\text{EM}}$ & $\sim 1/137$ & $\infty$ \\
				Strong & $\alpha_s$ & $\sim 0.118$ at $Q^2 = M_Z^2$ & $\sim \SI{1e-15}{\meter}$ \\
				Weak & $\alpha_W = g^2/(4\pi)$ & $\sim 1/30$ & $\sim \SI{1e-18}{\meter}$ \\
				Gravitation & $\alpha_G = G m^2/(\hbar c)$ & $m^2/m_P^2$ & $\infty$ \\
				\bottomrule
			\end{tabular}
		\end{adjustbox}
		\caption{Fundamental forces characterized by coupling constants}
		\label{tab:forces}
	\end{table}
	
	\subsection{Comprehensive Unit Conversions}
	
	\begin{table}[htbp]
		\centering
		\begin{adjustbox}{width=0.95\textwidth}
			\begin{tabular}{lcccc}
				\toprule
				\textbf{SI Unit} & \textbf{SI Dimension} & \textbf{Natural Dimension} & \textbf{Conversion} & \textbf{Accuracy} \\
				\midrule
				Meter & $[L]$ & $[E^{-1}]$ & $\SI{1}{\meter} \leftrightarrow (\SI{197}{\mega\electronvolt})^{-1}$ & $< 0.001\%$ \\
				Second & $[T]$ & $[E^{-1}]$ & $\SI{1}{\second} \leftrightarrow (\SI{6.58e-22}{\mega\electronvolt})^{-1}$ & $< 0.00001\%$ \\
				Kilogram & $[M]$ & $[E]$ & $\SI{1}{\kilogram} \leftrightarrow \SI{5.61e26}{\mega\electronvolt}$ & $< 0.001\%$ \\
				Ampere & $[I]$ & $[E]^{1/2}$ & $\SI{1}{\ampere} \leftrightarrow (\SI{6.24e18}{\electronvolt})^{1/2}/\si{\second}$ & $< 0.005\%$ \\
				Kelvin & $[\Theta]$ & $[E]$ & $\SI{1}{\kelvin} \leftrightarrow \SI{8.62e-5}{\electronvolt}$ & $< 0.01\%$ \\
				Volt & $[ML^2 T^{-3} I^{-1}]$ & $[E]$ & $\SI{1}{\volt} \leftrightarrow \SI{1}{\electronvolt}/e$ & $< 0.0001\%$ \\
				Coulomb & $[T I]$ & $[E^0]$ & $\SI{1}{\coulomb} \leftrightarrow 6.24 \times 10^{18} \, e$ & $< 0.0001\%$ \\
				\bottomrule
			\end{tabular}
		\end{adjustbox}
		\caption{Comprehensive unit conversions from SI to natural units}
		\label{tab:conversion}
	\end{table}
	
	\section{Conclusion}
	
	This natural unit system provides the foundation for all T0 model calculations. By establishing energy as the universal dimension and setting fundamental constants to unity, we reveal the underlying unity of physical laws across all scales from the sub-Planckian T0 length to cosmological distances.
	
	Key principles:
	\begin{enumerate}
		\item Energy is the fundamental dimension
		\item All physical quantities are powers of energy
		\item The T0 length extends physics below the Planck scale
		\item Natural units simplify fundamental equations
		\item Dimensional consistency is paramount
	\end{enumerate}
	
	This framework serves as the basis for all further developments in the T0 model, providing both computational tools and conceptual insights into the nature of physical reality.
	
	\bibliographystyle{plain}
	\begin{thebibliography}{10}
		
		\bibitem{planck1900}
		M. Planck,
		\textit{Zur Theorie des Gesetzes der Energieverteilung im Normalspektrum},
		Verhandlungen der Deutschen Physikalischen Gesellschaft 2, 237-245 (1900).
		
		\bibitem{planck1906}
		M. Planck,
		\textit{Vorlesungen über die Theorie der Wärmestrahlung},
		Johann Ambrosius Barth, Leipzig, 1906.
		
		\bibitem{hartree1957}
		D. R. Hartree,
		\textit{The Calculation of Atomic Structures},
		John Wiley \& Sons, New York, 1957.
		
		\bibitem{weinberg1995}
		S. Weinberg,
		\textit{The Quantum Theory of Fields, Vol. 1},
		Cambridge University Press, 1995.
		
		\bibitem{peskin1995}
		M. E. Peskin and D. V. Schroeder,
		\textit{An Introduction to Quantum Field Theory},
		Addison-Wesley, 1995.
		
		\bibitem{misner1973}
		C. W. Misner, K. S. Thorne, and J. A. Wheeler,
		\textit{Gravitation},
		W. H. Freeman and Company, 1973.
		
		\bibitem{jackson1998}
		J. D. Jackson,
		\textit{Classical Electrodynamics},
		3rd edition, John Wiley \& Sons, 1998.
		
		\bibitem{pascher_t0_length_2025}
		J. Pascher,
		\textit{Beyond the Planck Scale: The T0 Length in Quantum Gravity},
		March 24, 2025.
		
	\end{thebibliography}
	
\end{document}