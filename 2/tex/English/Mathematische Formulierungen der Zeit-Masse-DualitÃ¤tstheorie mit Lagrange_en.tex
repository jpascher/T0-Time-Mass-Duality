\documentclass[12pt,a4paper]{article}
\usepackage[margin=2cm]{geometry}
\usepackage[utf8]{inputenc}
\usepackage[T1]{fontenc}
\usepackage{lmodern}
\usepackage[ngerman]{babel}
\usepackage{amsmath,amssymb,physics,graphicx,xcolor,amsthm}
\usepackage{hyperref}
\usepackage{booktabs}
\usepackage{siunitx}
\usepackage{cleveref}
\usepackage{pgfplots}
\pgfplotsset{compat=1.18}
\usepackage{tikz}
\usetikzlibrary{intersections}
\usepgfplotslibrary{fillbetween}
\usepackage{fancyhdr}

% Custom commands
\newcommand{\Tfield}{T(x)}
\newcommand{\betaT}{\beta_{\text{T}}}
\newcommand{\alphaEM}{\alpha_{\text{EM}}}
\newcommand{\Tzero}{T_0}
\newcommand{\DcovT}[1]{\partial_\mu #1 + #1 \partial_\mu \Tfield}
\newcommand{\DhiggsT}{\Tfield (\partial_\mu + ig A_\mu) \Phi + \Phi \partial_\mu \Tfield}
\newcommand{\gammaf}{\gamma_{\text{Lorentz}}}

% Theorem styles
\newtheorem{theorem}{Theorem}[section]
\newtheorem{proposition}[theorem]{Proposition}
\newtheorem{corollary}[theorem]{Corollary}
\newtheorem{lemma}[theorem]{Lemma}
\theoremstyle{definition}
\newtheorem{definition}[theorem]{Definition}
\newtheorem{example}[theorem]{Example}
\theoremstyle{remark}
\newtheorem{remark}[theorem]{Remark}

% Hyperref configuration
\hypersetup{
	colorlinks=true,
	linkcolor=blue,
	urlcolor=blue,
	citecolor=blue,
	pdftitle={From Time Dilation to Mass Variation: Mathematical Core Formulations of Time-Mass Duality Theory},
	pdfauthor={Johann Pascher},
	pdfsubject={Theoretical Physics},
	pdfkeywords={T0 Model, Time-Mass Duality, Emergent Gravitation}
}

% Header and Footer Configuration
\pagestyle{fancy}
\fancyhf{}
\fancyhead[L]{Johann Pascher}
\fancyhead[R]{From Time Dilation to Mass Variation}
\fancyfoot[C]{\thepage}
\renewcommand{\headrulewidth}{0.4pt}
\renewcommand{\footrulewidth}{0.4pt}

\title{From Time Dilation to Mass Variation: \\ Mathematical Core Formulations of Time-Mass Duality Theory}
\author{Johann Pascher}
\date{March 29, 2025}

\begin{document}
	
	\maketitle
	
	\begin{abstract}
		This work presents the essential mathematical formulations of time-mass duality theory, focusing on the fundamental equations and their physical interpretations. The theory establishes a duality between two complementary descriptions of reality: the standard view with time dilation and constant rest mass, and the T0 model with absolute time and variable mass. Central to this framework is the intrinsic time \( \Tfield = \frac{\hbar}{\max(m c^2, \omega)} \), which enables a unified treatment of massive particles and photons. The mathematical formulations include modified Lagrangian densities that emphasize emergent gravitation and energy-loss redshift in a static universe.
	\end{abstract}
	
	\tableofcontents
	\newpage
	
	\section{Introduction to Time-Mass Duality}
	The time-mass duality theory proposes an alternative framework:
	\begin{enumerate}
		\item Standard View: \( t' = \gammaf t \), \( m_0 = \text{const.} \)
		\item T0 Model: \( \Tzero = \text{const.} \), \( m = \gammaf m_0 \)
	\end{enumerate}
	
	\subsection{Relationship to the Standard Model}
	The T0 model extends the Standard Model with:
	\begin{enumerate}
		\item Intrinsic Time Field: \( \Tfield = \frac{\hbar}{\max(m c^2, \omega)} \)
		\item Higgs Field: \( \Phi \) with dynamic mass coupling
		\item Fermion Fields: \( \psi \) with Yukawa coupling
		\item Gauge Boson Fields: \( A_\mu \) with \( \Tfield \) interaction
	\end{enumerate}
	
	\section{Emergent Gravitation from the Intrinsic Time Field}
	\begin{theorem}[Emergence of Gravitation]
		Gravitation arises from gradients of the intrinsic time field:
		\begin{equation}
			\nabla \Tfield = -\frac{\hbar}{m^2 c^2} \nabla m
		\end{equation}
		with the modified potential:
		\begin{equation}
			\Phi(r) = -\frac{GM}{r} + \kappa r, \quad \kappa \approx 4.8 \times 10^{-11} \, \text{m/s}^2
		\end{equation}
	\end{theorem}
	
	\begin{proof}
		From \( \Tfield = \frac{\hbar}{m c^2} \) for massive particles:
		\begin{equation}
			\nabla \Tfield = -\frac{\hbar}{m^2 c^2} \nabla m
		\end{equation}
		With \( m(\vec{r}) = m_0 (1 + \frac{\Phi_g}{c^2}) \):
		\begin{equation}
			\nabla m = \frac{m_0}{c^2} \nabla \Phi_g
		\end{equation}
		Thus:
		\begin{equation}
			\nabla \Tfield \approx -\frac{\hbar}{m_0 c^4} \nabla \Phi_g
		\end{equation}
	\end{proof}
	
	\section{Mathematical Foundations: Intrinsic Time}
	\begin{theorem}[Intrinsic Time]
		\begin{equation}
			\Tfield = \frac{\hbar}{\max(m c^2, \omega)}
		\end{equation}
	\end{theorem}
	
	\section{Modified Derivative Operators}
	\begin{definition}[Modified Derivative]
		The modified covariant derivative in the T0 model is:
		\begin{equation}
			\DcovT{\Psi} = \partial_\mu \Psi + \Psi \partial_\mu \Tfield
		\end{equation}
	\end{definition}
	
	\section{Modified Field Equations}
	\begin{theorem}[Modified Schrödinger Equation]
		\begin{equation}
			i\hbar \Tfield \frac{\partial}{\partial t} \Psi + i\hbar \Psi \frac{\partial \Tfield}{\partial t} = \hat{H} \Psi
		\end{equation}
	\end{theorem}
	
	\section{Modified Lagrangian Density for the Higgs Field}
	\begin{theorem}[Higgs Lagrangian Density]
		The Lagrangian density of the Higgs field with coupling to \(\Tfield\) is:
		\begin{multline}
			\mathcal{L}_{\text{Higgs-T}} = |\DhiggsT|^2 + \frac{1}{2} \partial_\mu \Tfield \partial^\mu \Tfield - V(\Tfield, \Phi), \quad \\
			\DhiggsT = \Tfield (\partial_\mu + ig A_\mu) \Phi + \Phi \partial_\mu \Tfield
		\end{multline}
	\end{theorem}
	
	\section{Modified Lagrangian Density for Fermions}
	\begin{theorem}[Fermion Lagrangian Density]
		\begin{equation}
			\mathcal{L}_{\text{Fermion}} = \bar{\psi} i \gamma^\mu (\partial_\mu \psi + \psi \partial_\mu \Tfield) - y \bar{\psi} \Phi \psi
		\end{equation}
	\end{theorem}
	
	\section{Modified Lagrangian Density for Gauge Bosons}
	\begin{theorem}[Gauge Boson Lagrangian Density]
		\begin{equation}
			\mathcal{L}_{\text{Boson}} = -\frac{1}{4} F_{\mu\nu} F^{\mu\nu} + \frac{1}{2} \partial_\mu \Tfield \partial^\mu \Tfield
		\end{equation}
	\end{theorem}
	
	\section{Complete Total Lagrangian Density}
	\begin{theorem}[Total Lagrangian Density]
		\begin{equation}
			\mathcal{L}_{\text{Total}} = \mathcal{L}_{\text{Boson}} + \mathcal{L}_{\text{Fermion}} + \mathcal{L}_{\text{Higgs-T}} + \mathcal{L}_{\text{intrinsic}}, \quad \mathcal{L}_{\text{intrinsic}} = \frac{1}{2} \partial_\mu \Tfield \partial^\mu \Tfield - V(\Tfield)
		\end{equation}
	\end{theorem}
	
	\section{Cosmological Implications}
	The T0 model has the following implications:
	\begin{itemize}
		\item Modified Gravitational Potential: \( \Phi(r) = -\frac{GM}{r} + \kappa r \), \( \kappa \approx 4.8 \times 10^{-11} \, \text{m/s}^2 \)
		\item Cosmic Redshift: \( 1 + z = e^{\alpha d} \), \( \alpha \approx 2.3 \times 10^{-28} \, \text{m}^{-1} \)
		\item Wavelength Dependence: \( z(\lambda) = z_0 (1 + \betaT \ln(\lambda/\lambda_0)) \), \( \betaT \approx 0.008 \) (SI units)
	\end{itemize}
	
	\section{Derivation of \(\betaT\) in the T0 Model}
	The parameter \(\betaT\) describes the coupling of the intrinsic time field \(\Tfield\) to physical phenomena such as wavelength-dependent redshift. In the T0 model, \(\betaT\) is precisely derived as:
	\begin{equation}
		\betaT = \frac{\lambda_h^2 v^2}{16\pi^3} \cdot \frac{1}{m_h^2} \cdot \frac{1}{\xi}
	\end{equation}
	where \(\lambda_h\) is the Higgs self-coupling, \(v\) is the Higgs vacuum expectation value, \(m_h\) is the Higgs mass, and \(\xi \approx 1.33 \times 10^{-4}\) is a dimensionless parameter defining the characteristic length scale \(r_0 = \xi \cdot l_P\) (\(l_P\): Planck length). In natural units, \(\betaT = 1\) holds, representing an exact theoretical prediction derived directly from the model parameters, as detailed in \cite{pascher_alphabeta_2025}. A comprehensive derivation and discussion of this parameter can be found in \cite{pascher_alphabeta_2025}.
	
	\begin{thebibliography}{99}
		\bibitem{pascher_zeit_2025} Pascher, J. (2025). \href{https://github.com/jpascher/T0-Time-Mass-Duality/tree/main/2/pdf/English/Zeit\%20als\%20emergente\%20Eigenschaft\%20in\%20der\%20Quantenmechanik_en.pdf}{Time as an Emergent Property in Quantum Mechanics: A Connection Between Relativity, Fine-Structure Constant, and Quantum Dynamics}. March 23, 2025.
		\bibitem{pascher_lagrange_2025} Pascher, J. (2025). \href{https://github.com/jpascher/T0-Time-Mass-Duality/tree/main/2/pdf/English/Mathematische\%20Formulierungen\%20der\%20Zeit-Masse-Dualitätstheorie\%20mit\%20Lagrange_en.pdf}{From Time Dilation to Mass Variation: Mathematical Core Formulations of Time-Mass Duality Theory}. March 29, 2025.
		\bibitem{pascher_photon_2025} Pascher, J. (2025). \href{https://github.com/jpascher/T0-Time-Mass-Duality/tree/main/2/pdf/English/Dynamische\%20Masse\%20von\%20Photonen\%20und\%20ihre\%20Implikationen\%20für\%20Nichtlokalität_en.pdf}{Dynamic Mass of Photons and Its Implications for Nonlocality in the T0 Model}.
		\bibitem{pascher_erweiterung_2025} Pascher, J. (2025). \href{https://github.com/jpascher/T0-Time-Mass-Duality/tree/main/2/pdf/English/Die\%20Notwendigkeit\%20einer\%20Erweiterung\%20der\%20Standard-Quantenmechanik\%20und\%20Quantenfeldtheorie_en.pdf}{The Necessity of Extending Standard Quantum Mechanics and Quantum Field Theory}. March 27, 2025.
		\bibitem{pascher_massenvariation_2025} Pascher, J. (2025). \href{https://github.com/jpascher/T0-Time-Mass-Duality/tree/main/2/pdf/English/Massenvariation\%20in\%20Galaxien_en.pdf}{Mass Variation in Galaxies: An Analysis in the T0 Model with Emergent Gravitation}. March 30, 2025.
		\bibitem{pascher_higgs_2025} Pascher, J. (2025). \href{https://github.com/jpascher/T0-Time-Mass-Duality/tree/main/2/pdf/English/Mathematische\%20Formulierung\%20des\%20Higgs-Mechanismus\%20in\%20der\%20Zeit-Masse-Dualität_en.pdf}{Mathematical Formulation of the Higgs Mechanism in Time-Mass Duality}. March 28, 2025.
		\bibitem{pascher_feldtheorie_2025} Pascher, J. (2025). \href{https://github.com/jpascher/T0-Time-Mass-Duality/tree/main/2/pdf/English/Feldtheorie\%20und\%20Quantenkorrelationen_en.pdf}{Field Theory and Quantum Correlations: A New Perspective on Instantaneity}. March 28, 2025.
		\bibitem{pascher_messdifferenzen_2025} Pascher, J. (2025). \href{https://github.com/jpascher/T0-Time-Mass-Duality/tree/main/2/pdf/English/Analyse\%20der\%20Messdifferenzen\%20zwischen\%20dem\%20T0-Modell\%20und\%20dem\%20Standardmodell_en.pdf}{Compensatory and Additive Effects: An Analysis of Measurement Differences Between the T0 Model and the \(\Lambda\)CDM Standard Model}. April 2, 2025.
		\bibitem{pascher_planck_2025} Pascher, J. (2025). \href{https://github.com/jpascher/T0-Time-Mass-Duality/tree/main/2/pdf/English/Jenseits\%20der\%20Planck-Skala_en.pdf}{Real Consequences of Reformulating Time and Mass in Physics: Beyond the Planck Scale}. March 24, 2025.
		\bibitem{pascher_alpha_2025} Pascher, J. (2025). \href{https://github.com/jpascher/T0-Time-Mass-Duality/tree/main/2/pdf/English/Natürliche\%20Einheiten\%20mit\%20Feinstrukturkonstante\%20alpha\%20=\%201_en.pdf}{Energy as a Fundamental Unit: Natural Units with \(\alphaEM = 1\) in the T0 Model}. March 26, 2025.
		\bibitem{pascher_alphabeta_2025} Pascher, J. (2025). \href{https://github.com/jpascher/T0-Time-Mass-Duality/tree/main/2/pdf/English/Die\%20Konsistenz\%20von\%20alpha\%20=\%201\%20und\%20beta\%20=\%201_en.pdf}{Unified Unit System in the T0 Model: The Consistency of \(\alpha = 1\) and \(\beta = 1\)}. April 5, 2025.
		\bibitem{pascher_temp_2025} Pascher, J. (2025). \href{https://github.com/jpascher/T0-Time-Mass-Duality/tree/main/2/pdf/English/Anpassung\%20von\%20Temperatureinheiten\%20in\%20natürlichen\%20Einheiten\%20und\%20CMB-Messungen_en.pdf}{Adjustment of Temperature Units in Natural Units and CMB Measurements}. April 2, 2025.
		\bibitem{pascher_params_2025} Pascher, J. (2025). \href{https://github.com/jpascher/T0-Time-Mass-Duality/tree/main/2/pdf/English/Zeit-Masse-Dualitätstheorie\%20(T0-Modell)\%20Herleitung\%20der\%20Parameter\%20kappa,\%20alpha\%20und\%20beta_en.pdf}{Time-Mass Duality Theory (T0 Model): Derivation of Parameters \(\kappa\), \(\alpha\), and \(\beta\)}. April 4, 2025.
		\bibitem{pascher_emergente_gravitation_2025} Pascher, J. (2025). \href{https://github.com/jpascher/T0-Time-Mass-Duality/tree/main/2/pdf/English/Emergente\%20Gravitation\%20im\%20T0-Modell\%20Eine\%20formale\%20Herleitung_en.pdf}{Emergent Gravitation in the T0 Model: A Comprehensive Derivation}. April 1, 2025.
	\end{thebibliography}
	
\end{document}