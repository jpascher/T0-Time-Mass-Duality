\documentclass[a4paper,12pt]{article}
\usepackage[utf8]{inputenc}
\usepackage[T1]{fontenc}
\usepackage{lmodern}
\usepackage[ngerman]{babel}
\usepackage{amsmath}
\usepackage{amssymb}
\usepackage{geometry}
\usepackage{tocloft}
\usepackage{xcolor}
\usepackage[colorlinks=true, linkcolor=blue, citecolor=blue, urlcolor=blue]{hyperref}
\usepackage{siunitx}
\DeclareSIUnit{\year}{yr}
\DeclareSIUnit{\parsec}{pc}
\usepackage{fancyhdr}

\geometry{a4paper, margin=2cm}

% Headers and Footers
\pagestyle{fancy}
\fancyhf{}
\fancyhead[L]{Johann Pascher}
\fancyhead[R]{Time-Mass Duality}
\fancyfoot[C]{\thepage}
\renewcommand{\headrulewidth}{0.4pt}
\renewcommand{\footrulewidth}{0.4pt}

\renewcommand{\cftsecfont}{\color{blue}}
\renewcommand{\cftsubsecfont}{\color{blue}}
\renewcommand{\cftsecpagefont}{\color{blue}}
\renewcommand{\cftsubsecpagefont}{\color{blue}}
\setlength{\cftsecindent}{1cm}
\setlength{\cftsubsecindent}{2cm}

% Custom commands
\newcommand{\Tfield}{T(x)}
\newcommand{\DcovT}[1]{\Tfield D_\mu #1 + #1 \partial_\mu \Tfield}
\newcommand{\DhiggsT}{\Tfield (\partial_\mu + ig A_\mu) \Phi + \Phi \partial_\mu \Tfield}
\newcommand{\betaT}{\beta_{\text{T}}}
\newcommand{\alphaEM}{\alpha_{\text{EM}}}
\newcommand{\Mpl}{M_{\text{Pl}}}
\newcommand{\Tzerot}{T_0(\Tfield)}
\newcommand{\Tzero}{T_0}
\newcommand{\vecx}{\vec{x}}
\newcommand{\gammaf}{\gamma_{\text{Lorentz}}}

\title{A New Perspective on Time and Space: \\Johann Pascher’s Revolutionary Ideas}
\author{Johann Pascher}
\date{March 25, 2025}

\begin{document}
	
	\maketitle
	
	Imagine looking at a familiar painting, one you’ve seen a hundred times before. Then someone tilts it slightly, and suddenly you notice details and patterns that had previously escaped you. That’s exactly what I aim to do with our understanding of the universe. For over a century, Einstein’s theories have shaped our view of time and space. We’ve accepted that time is malleable—it slows down when you move quickly or enter a strong gravitational field—while an object’s rest mass is considered an unchanging property. This perspective has served us well, from precise navigation with GPS satellites to observing the deflection of light by the sun.
	
	But I propose flipping this picture. In my T0 model, time is absolute and flows uniformly, while mass becomes variable. This isn’t mere speculation but a well-developed model that, with mathematical formulations, explains the same experimental observations as Einstein’s theories—only from an entirely new angle. This work invites you to reconsider the familiar foundations of physics and asks whether a different perspective might yield a clearer, more unified picture of reality.
	
	\section{The Clock in Every Particle}
	
	In the T0 model, every particle in the universe—be it an electron, a proton, or a heavier muon—carries its own characteristic timescale, which I call “intrinsic time.” This time is inversely proportional to the particle’s mass and is defined as:
	
	\begin{equation}
		\Tfield = \frac{\hbar}{\max(m c^2, \omega)}
	\end{equation}
	
	Heavy particles have faster internal clocks, while light particles have slower ones. Take the muon as an example: in classical relativity, we explain its extended lifetime as it races through the atmosphere with time dilation—time stretches for the moving muon. In the T0 model, time remains constant, but the muon’s mass changes. These two descriptions lead to the same measurable outcomes, yet they offer vastly different insights into the nature of reality. The mathematical equivalence is detailed in “Time-Mass Duality Theory: Derivation of Parameters” \cite{pascher_params_2025}, but intrinsic time opens new pathways to understanding quantum phenomena.
	
	\section{When Distant Particles Are Connected}
	
	Quantum entanglement—the phenomenon where two particles appear linked across any distance—is one of physics’ most fascinating enigmas. Einstein called it “spooky action at a distance” because standard quantum mechanics describes it without truly explaining it. In the T0 model, this connection gains a new interpretation. Instead of assuming instantaneous correlation, it depends on the mass of the involved particles. Two entangled particles with different masses evolve with distinct intrinsic times, causing a measurable delay in their correlations—proportional to the ratio of their masses.
	
	This idea, explored in depth in “Dynamic Mass of Photons” \cite{pascher_photons_2025}, diverges from the conventional view and offers a clear, testable prediction. It challenges us to rethink the nature of nonlocality and demonstrates how time-mass duality can provide a concrete alternative to traditional quantum mechanics.
	
	\section{Rethinking Beginning and End}
	
	The T0 model also turns our conception of the universe upside down. Classical cosmology envisions an expanding space where galaxies move apart, observed as the redshift of light. In the T0 model, space remains static, and redshift arises from an energy loss of light over time, described as:
	
	\begin{equation}
		1 + z = e^{\alpha d}, \quad \alpha \approx \SI{2.3e-18}{\per\meter}
	\end{equation}
	
	This approach, elaborated in “Measurement Differences” \cite{pascher_messdifferenzen_2025}, resolves issues like the horizon problem more elegantly than inflation theory and avoids the mathematical singularities of the standard model. The Big Bang is not seen as the beginning of time and space but as a state of extremely high energy and mass evolving over constant time.
	
	For black holes, this means they lack a central singularity. The event horizon marks a boundary of extreme mass variation, not an end to time. This aligns with thermodynamics and sidesteps the information paradox, as examined in “Mass Variation in Galaxies” \cite{pascher_galaxies_2025}.
	
	\section{A Fundamental Building Block: Energy}
	
	In the T0 model, all fundamental constants—the speed of light \(c\), Planck’s constant \(\hbar\), the gravitational constant \(G\)—are reduced to a single quantity: energy. This unification is mathematically precise and reveals that these constants are not independent values but facets of an underlying energetic reality. While the standard model treats them as given, the T0 model derives them from simpler principles, as presented in “Parameter Derivations” \cite{pascher_params_2025}. It’s a simplification reminiscent of the shift from the geocentric to the heliocentric worldview—a profound reorientation of our view of natural laws.
	
	\section{Putting It to the Test}
	
	The T0 model isn’t just theoretical musing—it makes clear, testable predictions that differ from the standard model. Bell tests with particles of different masses could reveal correlation delays proportional to their mass ratio, an effect described in “Dynamic Mass of Photons” \cite{pascher_photons_2025}. In quantum coherence experiments, coherence times should vary with mass, detectable with current technology. The modified Schrödinger equation with intrinsic time, developed in “The Necessity of Extending Standard Quantum Mechanics” \cite{pascher_quantum_2025}, leads to distinct dispersion relations for matter waves:
	
	\begin{equation}
		i\hbar \Tfield \frac{\partial}{\partial t} \Psi + i\hbar \Psi \frac{\partial \Tfield}{\partial t} = \hat{H} \Psi
	\end{equation}
	
	These predictions offer concrete ways to distinguish between models, verifiable with today’s or soon-to-be-available technology.
	
	\section{A New Lens, a Clearer Picture}
	
	My approach inverts the usual perspective without discarding the experimentally confirmed laws of physics. The mathematical foundations remain intact but are interpreted and expanded within a new framework. This inversion echoes the transition from the geocentric to the heliocentric worldview: the observations stay the same, but the explanation becomes more elegant and profound.
	
	While standard physics seeks ways to unify quantum mechanics and gravitation, the T0 model offers a direct solution through the consistent treatment of time and mass. It addresses major unsolved questions—dark matter, dark energy, the black hole information paradox—and resolves many naturally, without the additional assumptions required in the standard model.
	
	The history of science shows that the greatest advances often arise not from new data but from new perspectives. This work is a call to reconsider familiar facts—not to replace them, but to gain a clearer, more unified picture of reality. It’s a step toward a physics that is more intuitive and comprehensive, perhaps the beginning of a revolution in our understanding of the universe.
	
	\begin{thebibliography}{99}
		\bibitem{pascher_params_2025} Pascher, J. (2025). \href{https://github.com/jpascher/T0-Time-Mass-Duality/tree/main/2/pdf/English/Zeit-Masse-Dualitätstheorie (T0-Modell) Herleitung der Parameter kappa, alpha und beta_en.pdf}{Time-Mass Duality Theory (T0 Model): Derivation of Parameters \(\kappa\), \(\alpha\), and \(\beta\)}. April 4, 2025.
		\bibitem{pascher_galaxies_2025} Pascher, J. (2025). \href{https://github.com/jpascher/T0-Time-Mass-Duality/tree/main/2/pdf/English/Massenvariation in Galaxien_en.pdf}{Mass Variation in Galaxies: An Analysis in the T0 Model with Emergent Gravitation}. March 30, 2025.
		\bibitem{pascher_messdifferenzen_2025} Pascher, J. (202elo5). \href{https://github.com/jpascher/T0-Time-Mass-Duality/tree/main/2/pdf/English/Analyse der Messdifferenzen zwischen dem T0-Modell und dem Standardmodell_en.pdf}{Compensatory and Additive Effects: An Analysis of Measurement Differences Between the T0 Model and the \(\Lambda\)CDM Standard Model}. April 2, 2025.
		\bibitem{pascher_photons_2025} Pascher, J. (2025). \href{https://github.com/jpascher/T0-Time-Mass-Duality/tree/main/2/pdf/English/Dynamische Masse von Photonen und ihre Implikationen für Nichtlokalität_en.tex}{Dynamic Mass of Photons and Its Implications for Nonlocality in the T0 Model}. March 25, 2025.
		\bibitem{pascher_quantum_2025} Pascher, J. (2025). \href{https://github.com/jpascher/T0-Time-Mass-Duality/tree/main/2/pdf/English/Die Notwendigkeit einer Erweiterung der Standard-Quantenmechanik und Quantenfeldtheorie_en.pdf}{The Necessity of Extending Standard Quantum Mechanics and Quantum Field Theory}. March 27, 2025.
	\end{thebibliography}
	
\end{document}