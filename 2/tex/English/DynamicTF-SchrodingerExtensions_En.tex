\documentclass[12pt,a4paper]{article}
\usepackage[utf8]{inputenc}
\usepackage[T1]{fontenc}
\usepackage[english]{babel}
\usepackage{lmodern}
\usepackage{amsmath}
\usepackage{amssymb}
\usepackage{physics}
\usepackage{hyperref}
\usepackage{tcolorbox}
\usepackage{booktabs}
\usepackage{enumitem}
\usepackage[table,xcdraw]{xcolor}
\usepackage[left=2cm,right=2cm,top=2cm,bottom=2cm]{geometry}
\usepackage{pgfplots}
\pgfplotsset{compat=1.18}
\usepackage{graphicx}
\usepackage{float}
\usepackage{fancyhdr}
\usepackage{siunitx}
\usepackage{array}
\usepackage{cleveref}

% Headers and Footers
\pagestyle{fancy}
\fancyhf{}
\fancyhead[L]{Johann Pascher}
\fancyhead[R]{Dynamic Time Field Extension}
\fancyfoot[C]{\thepage}
\renewcommand{\headrulewidth}{0.4pt}
\renewcommand{\footrulewidth}{0.4pt}

% Custom commands
\newcommand{\Tfield}{T(x)}
\newcommand{\Tfieldt}{T(x,t)}
\newcommand{\alphaEM}{\alpha_{\text{EM}}}
\newcommand{\alphaW}{\alpha_{\text{W}}}
\newcommand{\betaT}{\beta_{\text{T}}}
\newcommand{\Mpl}{M_{\text{Pl}}}
\newcommand{\Tzerot}{T_0(\Tfield)}
\newcommand{\Tzero}{T_0}
\newcommand{\vecx}{\vec{x}}
\newcommand{\gammaf}{\gamma_{\text{Lorentz}}}
\newcommand{\DhiggsT}{\Tfield (\partial_\mu + ig A_\mu) \Phi + \Phi \partial_\mu \Tfield}
\newcommand{\DhiggsTt}{\Tfieldt (\partial_\mu + ig A_\mu) \Phi + \Phi \partial_\mu \Tfieldt}
\newcommand{\LCDM}{\Lambda\text{CDM}}
\newcommand{\DTmu}{D_{T,\mu}}
\newcommand{\calL}{\mathcal{L}}
\newcommand{\deq}{\displaystyle}
\newcommand{\e}{\mathrm{e}}
\newcommand{\dTdt}{\frac{d\Tfieldt}{dt}}
\newcommand{\pdTdt}{\frac{\partial\Tfieldt}{\partial t}}
\newcommand{\pdTdx}{\nabla\Tfieldt}

\hypersetup{
	colorlinks=true,
	linkcolor=blue,
	citecolor=blue,
	urlcolor=blue,
	pdftitle={Dynamic Extension of the Intrinsic Time Field in the T0 Model},
	pdfauthor={Johann Pascher},
	pdfsubject={Theoretical Physics},
	pdfkeywords={T0 Model, Intrinsic Time Field, Dynamic Field Theory, Quantum Mechanics, Extended Schrödinger Equation}
}

\begin{document}
	
	\title{Dynamic Extension of the Intrinsic Time Field in the T0 Model: \\Complete Field-Theoretic Treatment and Implications for Quantum Evolution}
	\author{Johann Pascher\\
		Department of Communications Engineering, \\Höhere Technische Bundeslehranstalt (HTL), Leonding, Austria\\
		\texttt{johann.pascher@gmail.com}}
	\date{\today}
	
	\maketitle
	
	\begin{abstract}
		This paper presents a significant conceptual extension of the T0 model by fully developing the dynamic nature of the intrinsic time field. Rather than treating the time field as a static spatial configuration $\Tfield$, we advance to a complete space-time dependent field $\Tfieldt$, where both mass and frequency are treated as variables that depend on position and time. This extension deepens the elegant unification of the physical boundary conditions—wave propagation and concentrated mass—through the $\max$ function, and necessitates a corresponding extension to the modified Schrödinger equation. We formulate a complete Lagrangian density that incorporates the field's full dynamics and derive the implications for quantum evolution, including the appearance of the total time derivative in the quantum evolution equation. This advancement represents a natural theoretical progression that enhances the T0 model's capacity to describe quantum systems in varying gravitational environments and transitions between particle and wave-dominated states.
	\end{abstract}
	\newpage
	\tableofcontents
	\newpage
	\section{Introduction}
	\label{sec:introduction}
	
	The T0 model of time-mass duality represents a novel approach to unifying quantum mechanics and relativity theory through the introduction of the intrinsic time field $\Tfield$ and the inversion of the traditional relationship between time and mass \cite{pascher_part1_2025,pascher_part2_2025}. In previous formulations, this field has been primarily treated as a spatial configuration with position dependence, defined as $\Tfield = \frac{\hbar}{\max(mc^2, \omega)}$. This definition elegantly captures both extremes of physical reality—mass-dominated systems where $T = \frac{\hbar}{mc^2}$ and wave-dominated systems where $T = \frac{\hbar}{\omega}$—using a single mathematical construct \cite{pascher_quantum_2025}.
	
	While earlier works have acknowledged the field's potential time dependence, particularly in the context of the extended Schrödinger equation \cite{pascher_quantum_2025}, a comprehensive treatment of the field as a fully dynamic entity $\Tfieldt$ has not been fully developed. This paper addresses this conceptual extension, exploring the implications of treating both mass $m(\vecx,t)$ and frequency $\omega(\vecx,t)$ as variables that depend on both position and time. This approach aligns more naturally with field-theoretic principles, where fields are inherently dynamic entities evolving in both space and time.
	
	This extension is not merely a mathematical refinement but represents a fundamental advancement in the conceptual foundation of the T0 model. By treating the intrinsic time field as fully dynamic, we enhance its capacity to describe physical phenomena across scales, from quantum decoherence in varying gravitational environments to the evolution of entangled states in dynamic spacetimes. Furthermore, this extension provides a more natural framework for understanding transitions between particle-like and wave-like behaviors in quantum systems.
	
	The paper is organized as follows: Section \ref{sec:dynamic_time_field} presents the extended definition and properties of the dynamic intrinsic time field. Section \ref{sec:lagrangian_formulation} develops the complete Lagrangian density for this field. Section \ref{sec:quantum_evolution} explores the implications for quantum evolution through an extended Schrödinger equation. Section \ref{sec:boundary_conditions} examines how this dynamic field treatment enhances the unification of wave and particle boundary conditions. Finally, Section \ref{sec:conclusion} summarizes the conceptual advancements and discusses future research directions.
	
	\section{The Dynamic Intrinsic Time Field}
	\label{sec:dynamic_time_field}
	
	\subsection{Extended Definition}
	\label{subsec:extended_definition}
	
	The dynamic extension of the intrinsic time field is defined as:
	
	\begin{equation}
		\Tfieldt = \frac{\hbar}{\max(m(\vecx,t)c^2, \omega(\vecx,t))}
		\label{eq:dynamic_time_field}
	\end{equation}
	
	where:
	\begin{itemize}
		\item $m(\vecx,t)$ is the position and time-dependent mass
		\item $\omega(\vecx,t)$ is the position and time-dependent frequency/energy
	\end{itemize}
	
	This extension explicitly recognizes that both mass and frequency are not static constants but dynamic quantities that vary across space and time. For massive particles, this gives:
	
	\begin{equation}
		\Tfieldt = \frac{\hbar}{m(\vecx,t)c^2}
		\label{eq:massive_dynamic}
	\end{equation}
	
	and for photons or wave-dominated systems:
	
	\begin{equation}
		\Tfieldt = \frac{\hbar}{\omega(\vecx,t)}
		\label{eq:photon_dynamic}
	\end{equation}
	
	The variation of mass across space and time reflects the fundamental premise of the T0 model—that mass is a variable quantity influenced by the gravitational environment, rather than a constant property as assumed in traditional relativity theory. Similarly, the frequency of photons changes as they interact with the time field during propagation, providing a natural mechanism for phenomena such as redshift without requiring cosmic expansion \cite{pascher_part2_2025}.
	
	\subsection{Physical Significance}
	\label{subsec:physical_significance}
	
	The dynamic nature of the time field has profound physical significance:
	
	\begin{enumerate}
		\item \textbf{Gravitational Response}: The field $\Tfieldt$ actively responds to changes in gravitational environments, with mass increasing (and thus $\Tfieldt$ decreasing) in stronger gravitational fields.
		
		\item \textbf{Energy Conservation}: As photons propagate through space, their interaction with the time field causes energy attenuation, manifesting as cosmic redshift over large distances \cite{pascher_galaxies_2025}.
		
		\item \textbf{Quantum Decoherence}: The rate of quantum decoherence is linked to the local value and rate of change of $\Tfieldt$, explaining why macroscopic objects (with smaller $\Tfieldt$) decohere more rapidly than microscopic quantum systems \cite{pascher_quantum_2025}.
		
		\item \textbf{Field Propagation}: Changes in the time field propagate at finite speed, ensuring causality in the framework's predictions.
	\end{enumerate}
	
	This dynamic treatment also naturally accommodates transitional states where systems may shift between wave-dominated and particle-dominated behaviors. For instance, during particle creation and annihilation, the relevant term in the $\max$ function smoothly transitions, providing a continuous description of such processes.
	
	\section{Complete Lagrangian Formulation}
	\label{sec:lagrangian_formulation}
	
	\subsection{Dynamic Field Lagrangian}
	\label{subsec:dynamic_lagrangian}
	
	The complete Lagrangian density for the dynamic intrinsic time field is:
	
	\begin{equation}
		\mathcal{L}_{\text{intrinsic}} = \frac{1}{2}\partial_{\mu}\Tfieldt\partial^{\mu}\Tfieldt - \frac{1}{2}\Tfieldt^2 - \frac{\rho(\vecx,t)}{\Tfieldt}
		\label{eq:intrinsic_lagrangian}
	\end{equation}
	
	where $\rho(\vecx,t)$ is the position and time-dependent mass-energy density. This Lagrangian includes:
	
	\begin{itemize}
		\item A kinetic term $\frac{1}{2}\partial_{\mu}\Tfieldt\partial^{\mu}\Tfieldt$ representing the field's space-time dynamics
		\item A potential term $\frac{1}{2}\Tfieldt^2$ reflecting the field's self-interaction
		\item A coupling term $\frac{\rho(\vecx,t)}{\Tfieldt}$ representing the interaction between the field and matter/energy distributions
	\end{itemize}
	
	This formulation extends previous work \cite{pascher_lagrange_2025} by explicitly incorporating the field's time dependence and ensuring that the mass-energy density is treated as a dynamic quantity rather than a static distribution.
	
	\subsection{Field Equations}
	\label{subsec:field_equations}
	
	The Euler-Lagrange equation derived from the Lagrangian density yields the field equation:
	
	\begin{equation}
		\partial_{\mu}\partial^{\mu}\Tfieldt + \Tfieldt + \frac{\rho(\vecx,t)}{\Tfieldt^2} = 0
		\label{eq:field_equation}
	\end{equation}
	
	In the static limit where time derivatives vanish, this reduces to:
	
	\begin{equation}
		\nabla^2 \Tfieldt \approx -\frac{\rho(\vecx,t)}{\Tfieldt^2}
		\label{eq:static_approximation}
	\end{equation}
	
	which aligns with previous formulations \cite{pascher_emergente_2025} but maintains the potential time dependence of the mass-energy density.
	
	The dynamic field equation describes how the intrinsic time field evolves in response to changes in the mass-energy distribution, providing a natural framework for understanding phenomena such as gravitational waves and dynamic cosmological evolution within the T0 model.
	
	\section{Implications for Quantum Evolution}
	\label{sec:quantum_evolution}
	
	\subsection{Extended Schrödinger Equation}
	\label{subsec:extended_schrodinger}
	
	The dynamic extension of the time field necessitates a corresponding extension of the modified Schrödinger equation. The previously established form \cite{pascher_quantum_2025}:
	
	\begin{equation}
		i\hbar \Tfield \frac{\partial\Psi}{\partial t} + i\hbar \Psi \frac{\partial \Tfield}{\partial t} = \hat{H} \Psi
		\label{eq:original_schrodinger}
	\end{equation}
	
	must now be extended to account for the full space-time dynamics of the field:
	
	\begin{equation}
		i\hbar \Tfieldt \frac{\partial\Psi}{\partial t} + i\hbar \Psi \left[\frac{\partial \Tfieldt}{\partial t} + \vec{v}\cdot\nabla\Tfieldt\right] = \hat{H} \Psi
		\label{eq:dynamic_schrodinger}
	\end{equation}
	
	where $\vec{v}$ is the velocity of the quantum system. The term in square brackets represents the total time derivative of the field as experienced by the moving quantum system:
	
	\begin{equation}
		\frac{d\Tfieldt}{dt} = \frac{\partial \Tfieldt}{\partial t} + \vec{v}\cdot\nabla\Tfieldt
		\label{eq:total_derivative}
	\end{equation}
	
	This total derivative has profound physical significance. The first term $\pdTdt$ represents how the time field changes explicitly with time at a fixed point in space, while the second term $\vec{v}\cdot\pdTdx$ accounts for how the field changes along the trajectory of the moving quantum system.
	
	\subsection{Physical Interpretation}
	\label{subsec:quantum_interpretation}
	
	The extended Schrödinger equation provides several key insights:
	
	\begin{enumerate}
		\item \textbf{Path Dependency}: Quantum evolution depends not just on the local value of the time field but on how it changes along the system's path, introducing an element of history-dependence to quantum dynamics.
		
		\item \textbf{Velocity Coupling}: The explicit appearance of velocity in the quantum evolution equation creates a direct coupling between a particle's motion and its internal quantum evolution.
		
		\item \textbf{Gravitational Influence}: Since the time field is linked to gravitation, this formulation provides a natural mechanism for how gravitational environments influence quantum behavior.
		
		\item \textbf{Time Dilation Effects}: The equation naturally incorporates effects analogous to time dilation in relativity, but through the mechanism of mass variation rather than relativistic time effects.
	\end{enumerate}
	
	This framework offers a natural explanation for gravitationally induced phase shifts in quantum interference experiments, such as neutron interferometry in gravitational fields \cite{Colella1975}, without requiring separate treatments for quantum and gravitational phenomena.
	
	\section{Unification of Boundary Conditions}
	\label{sec:boundary_conditions}
	
	\subsection{Wave-Particle Transitions}
	\label{subsec:wave_particle}
	
	A particularly elegant feature of the dynamic time field is how it handles the two physical boundary conditions—wave propagation and concentrated mass—through the $\max$ function in its definition. This mechanism deserves special attention as it provides a natural framework for understanding wave-particle duality.
	
	For a quantum system transitioning between wave-like and particle-like behaviors, the time field smoothly adjusts according to which term dominates in the expression:
	
	\begin{equation}
		\Tfieldt = \frac{\hbar}{\max(m(\vecx,t)c^2, \omega(\vecx,t))}
		\label{eq:boundary_transition}
	\end{equation}
	
	During measurement processes or interactions that localize a quantum system, the $m(\vecx,t)c^2$ term may come to dominate, shifting the system toward particle-like behavior. Conversely, during free evolution, the $\omega(\vecx,t)$ term may dominate, leading to wave-like behavior.
	
	This smooth transition provides a more natural description of quantum measurement and wave function collapse than conventional quantum mechanics, where the transition between wave and particle descriptions often appears artificially imposed rather than emerging from the underlying physics.
	
	\subsection{Comparison with Extended Standard Model}
	\label{subsec:comparison_esm}
	
	The direct handling of these boundary conditions through the $\max$ function contrasts favorably with alternative approaches such as the Extended Standard Model (ESM) \cite{pascher_standardmod_2025}. In the ESM, the scalar field $\Theta$ relates to the time field through a logarithmic relationship:
	
	\begin{equation}
		\Theta(\vecx,t) \propto \ln\left(\frac{\Tfieldt}{\Tzero}\right)
		\label{eq:theta_relation}
	\end{equation}
	
	This relationship obscures the elegant boundary condition handling of the T0 model. The extreme states that are captured directly by the $\max$ function in the time field definition must be handled through more complex mathematical machinery in the ESM, making the physical interpretation less intuitive and direct.
	
	The transition between pure energy (wave) states and maximum mass (particle) states lacks the direct, intuitive formulation present in the T0 model. Instead, these transitions are indirectly encoded in the coupling between the scalar field and the energy-momentum tensor, adding a layer of mathematical abstraction that distances the theory from its physical interpretation.
	
	\section{Conclusion and Outlook}
	\label{sec:conclusion}
	
	\subsection{Conceptual Advancements}
	\label{subsec:advancements}
	
	The dynamic extension of the intrinsic time field represents a significant advancement in the conceptual foundation of the T0 model. By treating the time field as a fully dynamic entity $\Tfieldt$ with both spatial and temporal dependence, we have:
	
	\begin{itemize}
		\item Deepened the field-theoretic treatment of the T0 model, aligning it more closely with established quantum field theory principles
		\item Enhanced the model's capacity to describe quantum systems in varying gravitational environments
		\item Provided a more natural framework for understanding wave-particle duality and transitions between these states
		\item Developed a more comprehensive quantum evolution equation that accounts for the total time derivative of the field
	\end{itemize}
	
	These advancements maintain the core principles of the T0 model—absolute time, variable mass, and emergent gravitation—while extending its explanatory power and theoretical consistency.
	
	\subsection{Future Research Directions}
	\label{subsec:future_research}
	
	This dynamic extension opens several promising avenues for future research:
	
	\begin{enumerate}
		\item \textbf{Numerical Simulations}: Developing numerical simulations of quantum evolution in varying gravitational environments using the extended Schrödinger equation.
		
		\item \textbf{Quantum-Gravitational Experiments}: Designing experimental tests that could distinguish between the predictions of the dynamic T0 model and conventional quantum mechanics in gravitational contexts.
		
		\item \textbf{Cosmological Applications}: Extending the model's cosmological predictions to account for time-varying gravitational environments in the early universe.
		
		\item \textbf{Wave-Particle Transitions}: Further exploring how the model's handling of wave-particle duality might provide new insights into quantum measurement and decoherence.
		
		\item \textbf{Field Quantization}: Developing a fully quantized version of the dynamic time field, potentially opening new approaches to quantum gravity.
	\end{enumerate}
	
	The dynamic extension of the intrinsic time field represents not merely a mathematical refinement but a significant conceptual advancement that enhances the T0 model's capacity to provide a unified description of quantum and gravitational phenomena. By treating the time field as a fully dynamic entity, we move closer to a truly unified framework for understanding the fundamental nature of reality.
	
	\bibliographystyle{apsrev4-2}
	\begin{thebibliography}{99}
		\bibitem{pascher_part1_2025} J. Pascher, \href{https://github.com/jpascher/T0-Time-Mass-Duality/tree/main/2/pdf/English/QMRelTimeMassPart1En.pdf}{Bridging Quantum Mechanics and Relativity through Time-Mass Duality: Part I: Theoretical Foundations}, April 7, 2025.
		\bibitem{pascher_part2_2025} J. Pascher, \href{https://github.com/jpascher/T0-Time-Mass-Duality/tree/main/2/pdf/English/QMRelTimeMassPart2En.pdf}{Bridging Quantum Mechanics and Relativity through Time-Mass Duality: Part II: Cosmological Implications and Experimental Validation}, April 7, 2025.
		\bibitem{pascher_quantum_2025} J. Pascher, \href{https://github.com/jpascher/T0-Time-Mass-Duality/tree/main/2/pdf/English/NotwendigkeitQMErweiterungEn.pdf}{The Necessity of Extending Standard Quantum Mechanics and Quantum Field Theory}, March 27, 2025.
		\bibitem{pascher_lagrange_2025} J. Pascher, \href{https://github.com/jpascher/T0-Time-Mass-Duality/tree/main/2/pdf/English/MathZeitMasseLagrangeEn.pdf}{From Time Dilation to Mass Variation: Mathematical Core Formulations of Time-Mass Duality Theory}, March 29, 2025.
		\bibitem{pascher_emergente_2025} J. Pascher, \href{https://github.com/jpascher/T0-Time-Mass-Duality/tree/main/2/pdf/English/EmergentGravT0En.pdf}{Emergent Gravitation in the T0 Model: A Comprehensive Derivation}, April 1, 2025.
		\bibitem{pascher_galaxies_2025} J. Pascher, \href{https://github.com/jpascher/T0-Time-Mass-Duality/tree/main/2/pdf/English/MassVarGalaxienEn.pdf}{Mass Variation in Galaxies: An Analysis in the T0 Model with Emergent Gravitation}, March 30, 2025.
		\bibitem{pascher_standardmod_2025} J. Pascher, \href{https://github.com/jpascher/T0-Time-Mass-Duality/tree/main/2/pdf/English/StandardModKruemmungRotvEn.pdf}{Completing the Standard Model: An Extension Compatible with the T0 Model of Time-Mass Duality}, April 17, 2025.
		\bibitem{pascher_esm_comparison_2025} J. Pascher, \href{https://github.com/jpascher/T0-Time-Mass-Duality/tree/main/2/pdf/English/T0vsESM_ConceptualAnalysisEn.pdf}{Conceptual Comparison of T0 Model and Extended Standard Model: Field-Theoretic vs. Dimensional Approaches}, April 25, 2025.
		\bibitem{Colella1975} R. Colella, A. W. Overhauser, and S. A. Werner, \textit{Observation of Gravitationally Induced Quantum Interference}, Phys. Rev. Lett. \textbf{34}, 1472 (1975).
		\bibitem{Will2014} C. M. Will, \textit{The Confrontation between General Relativity and Experiment}, Living Rev. Rel. \textbf{17}, 4 (2014).
		\bibitem{Dirac1928} P. A. M. Dirac, \textit{The Quantum Theory of the Electron}, Proc. Roy. Soc. London A \textbf{117}, 610--624 (1928).
		\bibitem{Einstein1915} A. Einstein, \textit{The Field Equations of Gravitation}, Proc. Roy. Prussian Acad. Sci., 844--847 (1915).
		\bibitem{schrodinger1926} E. Schrödinger, \textit{An Undulatory Theory of the Mechanics of Atoms and Molecules}, Phys. Rev. \textbf{28}, 1049 (1926).
	\end{thebibliography}
	
\end{document}