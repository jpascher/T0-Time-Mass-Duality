\documentclass[a4paper,12pt]{article}
\usepackage[utf8]{inputenc}
\usepackage[T1]{fontenc}
\usepackage{lmodern}
\usepackage[ngerman]{babel}
\usepackage{amsmath, amssymb, amsthm}
\usepackage{geometry}
\usepackage{xcolor}
\usepackage{tocloft}
\usepackage{siunitx}
\DeclareSIUnit{\year}{yr}
\DeclareSIUnit{\parsec}{pc}
\usepackage{fancyhdr}

\usepackage{hyperref}
\hypersetup{
	colorlinks=true,
	linkcolor=blue,
	filecolor=blue,
	citecolor=blue,
	urlcolor=blue,
	bookmarks=true,
	bookmarksopen=true,
	pdftitle={Simplified Description of Fundamental Forces with Time-Mass Duality},
	pdfauthor={Johann Pascher},
}

\usepackage{cleveref}

\geometry{a4paper, margin=2cm}

% Headers and Footers
\pagestyle{fancy}
\fancyhf{}
\fancyhead[L]{Johann Pascher}
\fancyhead[R]{Time-Mass Duality}
\fancyfoot[C]{\thepage}
\renewcommand{\headrulewidth}{0.4pt}
\renewcommand{\footrulewidth}{0.4pt}

\renewcommand{\cftsecfont}{\color{blue}}
\renewcommand{\cftsubsecfont}{\color{blue}}
\renewcommand{\cftsecpagefont}{\color{blue}}
\renewcommand{\cftsubsecpagefont}{\color{blue}}
\setlength{\cftsecindent}{1cm}
\setlength{\cftsubsecindent}{2cm}

% Custom commands
\newcommand{\Tfield}{T(x)}
\newcommand{\DcovT}[1]{\Tfield D_\mu #1 + #1 \partial_\mu \Tfield}
\newcommand{\DhiggsT}{\Tfield (\partial_\mu + ig A_\mu) \Phi + \Phi \partial_\mu \Tfield}
\newcommand{\betaT}{\beta_{\text{T}}}
\newcommand{\alphaEM}{\alpha_{\text{EM}}}
\newcommand{\Mpl}{M_{\text{Pl}}}
\newcommand{\Tzerot}{T_0(\Tfield)}
\newcommand{\Tzero}{T_0}
\newcommand{\vecx}{\vec{x}}
\newcommand{\gammaf}{\gamma_{\text{Lorentz}}}
\newcommand{\dcomma}{,}

\title{Simplified Description of Fundamental Forces with Time-Mass Duality}
\author{Johann Pascher}
\date{March 26, 2025}

\begin{document}
	
	\maketitle
	
	\tableofcontents
	\newpage
	
	\section{Unified Lagrangian Density with Dual Time-Mass Concept}
	
	Physics describes the world through four fundamental forces—strong, weak, electromagnetic, and gravitational—traditionally considered separately. However, in the T0 model, based on time-mass duality, these forces can be unified in a single Lagrangian density that naturally encompasses both known interactions and gravitation. This density is given by:
	
	\begin{equation}
		\mathcal{L}_\text{total} = \mathcal{L}_\text{SM} + \mathcal{L}_\text{Higgs} + \mathcal{L}_\text{intrinsic}
	\end{equation}
	
	Here, \(\mathcal{L}_\text{SM}\) represents the interactions of the Standard Model—the strong, electromagnetic, and weak forces—\(\mathcal{L}_\text{Higgs}\) describes the dynamics of the Higgs field, and \(\mathcal{L}_\text{intrinsic}\) introduces the concept of intrinsic time, reflecting time-mass duality. Notably, gravitation is not added as a separate force but emerges from the dynamics of the intrinsic time field, as detailed in "Mathematical Core Formulations" \cite{pascher_lagrange_2025}.
	
	\subsection{Standard Model}
	
	The Standard Model forms the basis for describing the three forces that govern particle behavior at the atomic level. Its Lagrangian density is composed of:
	
	\begin{equation}
		\mathcal{L}_\text{SM} = \mathcal{L}_\text{strong} + \mathcal{L}_\text{em} + \mathcal{L}_\text{weak}
	\end{equation}
	
	Here, \(\mathcal{L}_\text{strong} = -\frac{1}{4} F_{\mu\nu}^a F^{a\mu\nu} + \bar{\psi}(i \gamma^\mu D_\mu - m_\psi(\phi))\psi\) represents the strong nuclear force, binding quarks into protons and neutrons; \(\mathcal{L}_\text{em} = -\frac{1}{4} F_{\mu\nu} F^{\mu\nu} + \bar{\psi}(i \gamma^\mu D_\mu - m_\psi(\phi))\psi\) represents the electromagnetic force, coupling electrons to nuclei; and \(\mathcal{L}_\text{weak} = -\frac{1}{4} W_{\mu\nu}^a W^{a\mu\nu} + \bar{\psi}(i \gamma^\mu D_\mu - m_\psi(\phi))\psi\) represents the weak force, governing processes like radioactive decay. In the T0 model, this description is adjusted by replacing time dilation with mass variation, leading to a dual formulation:
	
	\begin{equation}
		\mathcal{L}_\text{SM-T} = \mathcal{L}_\text{strong-T} + \mathcal{L}_\text{em-T} + \mathcal{L}_\text{weak-T}
	\end{equation}
	
	Here, the time derivative is tied to the intrinsic time \(T\), such that \(\partial_t \rightarrow \partial_{t/T}\), an adjustment that reinterprets dynamics under absolute time.
	
	\subsection{Higgs Field}
	
	The Higgs field, responsible for mass generation, is described in the Standard Model by:
	
	\begin{equation}
		\mathcal{L}_\text{Higgs} = (D_\mu \phi)^\dagger (D^\mu \phi) - V(\phi)
	\end{equation}
	
	where \(\phi\) is the Higgs field and \(V(\phi) = \mu^2 \phi^\dagger \phi + \lambda (\phi^\dagger \phi)^2\) is the potential. In the T0 model, this formula is extended to incorporate intrinsic time:
	
	\begin{equation}
		\mathcal{L}_\text{Higgs-T} = (D_{T\mu} \phi_T)^\dagger (D_T^\mu \phi_T) - V_T(\phi_T)
	\end{equation}
	
	The covariant derivative \(D_{T\mu}\) accounts for time-mass duality, highlighting the Higgs field’s role as a medium for mass and time, as elaborated in "Mathematical Formulation of the Higgs Mechanism" \cite{pascher_higgs_2025}.
	
	\subsection{Lagrangian Density for Intrinsic Time}
	
	The central innovation of the T0 model is the Lagrangian density for intrinsic time, given by:
	
	\begin{equation}
		\mathcal{L}_\text{intrinsic} = \bar{\psi} \left( i\hbar \gamma^0 \frac{\partial}{\partial (t/T)} - i\hbar \gamma^0 \frac{\partial}{\partial t} \right) \psi
	\end{equation}
	
	Here, \(T = \frac{\hbar}{m c^2}\) is the intrinsic time, dependent on mass. This formulation, developed in "The Necessity of Extending Standard Quantum Mechanics" \cite{pascher_quantum_2025}, links particle dynamics to their individual timescales, enabling a unified description of all forces.
	
	\section{Simplified Description of Mass Terms with Time-Mass Duality}
	
	In the Standard Model, a particle’s mass is defined by its coupling to the Higgs field: \(m_\psi(\phi) = y_\psi \phi\), where mass remains constant and time is variable. In the T0 model, this view is reversed: time remains absolute, and mass varies with the Lorentz factor \(\gamma\):
	
	\begin{equation}
		m_\psi(\phi_T) = y_\psi \phi_T \cdot \gamma, \quad \gamma = \frac{1}{\sqrt{1 - v^2/c^2}}
	\end{equation}
	
	This dual description, derived in "Time-Mass Duality Theory" \cite{pascher_params_2025}, explains the same phenomena as time dilation but offers a new perspective on the role of mass.
	
	\section{The Higgs Field as a Universal Medium with Intrinsic Time}
	
	The Higgs field is more than a mechanism for mass generation—in the T0 model, it also determines particles’ intrinsic timescales. This relationship is expressed as:
	
	\begin{equation}
		T = \frac{\hbar}{m(\phi) c^2} = \frac{\hbar}{y_\psi \phi \cdot c^2}
	\end{equation}
	
	A particle’s intrinsic time is thus inversely proportional to its mass, generated by the Higgs field. This perspective expands the Higgs field’s role as a universal medium influencing all interactions, as explored in "Higgs Mechanism" \cite{pascher_higgs_2025}.
	
	\section{The Higgs Field and the Vacuum: A Complex Relationship with Intrinsic Time}
	
	Vacuum energy, a central issue in modern physics, is reinterpreted in the T0 model. Instead of a sum of zero-point energies, it could be described as:
	
	\begin{equation}
		E_\text{vacuum} = \sum_i \frac{\hbar}{2 T_i}
	\end{equation}
	
	where \(T_i\) is the intrinsic time of quantum fluctuations. This formulation links vacuum energy to the dynamics of the Higgs field and time-mass duality, offering new insights into the cosmological constant.
	
	\section{Quantum Entanglement and Nonlocality in Time-Mass Duality}
	
	The apparent instantaneity of quantum entanglement is reconsidered in the T0 model through intrinsic time. In the \(T_0\) model, correlations arise not instantaneously but through mass variations. For entangled particles with different masses, time evolution varies with their intrinsic times. For photons, this is defined as:
	
	\begin{equation}
		T = \frac{\hbar}{E_{\gamma}} e^{\alpha x}, \quad \alpha = \frac{H_0}{c} \approx \SI{2.3e-18}{\per\meter}
	\end{equation}
	
	reflecting energy loss over distances, as described in "Dynamic Mass of Photons" \cite{pascher_photons_2025}.
	
	\section{Cosmological Implications of Time-Mass Duality}
	
	The T0 model provides natural explanations for cosmological phenomena through three key parameters: \(\alpha \approx \SI{2.3e-18}{\per\meter}\) describes photon energy loss, \(\kappa \approx \SI{4.8e-11}{\meter\per\second\squared}\) the strength of the dark energy field in galactic dynamics, and \(\betaT^{\text{SI}} \approx 0{,}008\) the coupling to baryonic matter. The gravitational potential becomes:
	
	\begin{equation}
		\Phi(r) = -\frac{G M}{r} + \kappa r
	\end{equation}
	
	These parameters, derived in "Mass Variation in Galaxies" \cite{pascher_galaxies_2025} and "Measurement Differences" \cite{pascher_messdifferenzen_2025}, explain flat rotation curves and redshift without dark matter or expansion.
	
	\section{Summary of the Unified Theory}
	
	The unified theory is described by the action:
	
	\begin{equation}
		S_\text{unified} = \int \left( \mathcal{L}_\text{standard} + \mathcal{L}_\text{complementary} + \mathcal{L}_\text{coupling} \right) d^4x
	\end{equation}
	
	where \(\mathcal{L}_\text{standard}\) is the Standard Model, \(\mathcal{L}_\text{complementary}\) the dual formulation, and \(\mathcal{L}_\text{coupling}\) the time-mass interaction. This approach bridges quantum mechanics and gravitation, offers new insights into entanglement and cosmological phenomena, and is experimentally testable.
	
	\section{Experimental Testability}
	
	The T0 model makes testable predictions, such as photon energy loss with \(\alpha\), modified gravitational potentials with \(\kappa\), and mass-dependent coherence times in quantum systems, verifiable with current technology, as outlined in "Parameter Derivations" \cite{pascher_params_2025}.
	
	\section{References to Further Works}
	
	This theory builds on my previous works, listed in the bibliography, which explore various aspects of time-mass duality in depth.
	
	\begin{thebibliography}{99}
		\bibitem{pascher_params_2025} Pascher, J. (2025). \href{https://github.com/jpascher/T0-Time-Mass-Duality/tree/main/2/pdf/English/Zeit-Masse-Dualitätstheorie (T0-Modell) Herleitung der Parameter kappa, alpha und beta_en.pdf}{Time-Mass Duality Theory (T0 Model): Derivation of Parameters \(\kappa\), \(\alpha\), and \(\beta\)}. April 4, 2025.
		\bibitem{pascher_galaxies_2025} Pascher, J. (2025). \href{https://github.com/jpascher/T0-Time-Mass-Duality/tree/main/2/pdf/English/Massenvariation in Galaxien_en.pdf}{Mass Variation in Galaxies: An Analysis in the T0 Model with Emergent Gravitation}. March 30, 2025.
		\bibitem{pascher_messdifferenzen_2025} Pascher, J. (2025). \href{https://github.com/jpascher/T0-Time-Mass-Duality/tree/main/2/pdf/English/Analyse der Messdifferenzen zwischen dem T0-Modell und dem Standardmodell_en.pdf}{Compensatory and Additive Effects: An Analysis of Measurement Differences Between the T0 Model and the \(\Lambda\)CDM Standard Model}. April 2, 2025.
		\bibitem{pascher_lagrange_2025} Pascher, J. (2025). \href{https://github.com/jpascher/T0-Time-Mass-Duality/tree/main/2/pdf/English/Mathematische Formulierungen der Zeit-Masse-Dualitätstheorie mit Lagrange_en.pdf}{From Time Dilation to Mass Variation: Mathematical Core Formulations of Time-Mass Duality Theory}. March 29, 2025.
		\bibitem{pascher_photons_2025} Pascher, J. (2025). \href{https://github.com/jpascher/T0-Time-Mass-Duality/tree/main/2/pdf/English/Dynamische Masse von Photonen und ihre Implikationen für Nichtlokalität_en.tex}{Dynamic Mass of Photons and Its Implications for Nonlocality in the T0 Model}. March 25, 2025.
		\bibitem{pascher_quantum_2025} Pascher, J. (2025). \href{https://github.com/jpascher/T0-Time-Mass-Duality/tree/main/2/pdf/English/Die Notwendigkeit einer Erweiterung der Standard-Quantenmechanik und Quantenfeldtheorie_en.pdf}{The Necessity of Extending Standard Quantum Mechanics and Quantum Field Theory}. March 27, 2025.
		\bibitem{pascher_higgs_2025} Pascher, J. (2025). \href{https://github.com/jpascher/T0-Time-Mass-Duality/tree/main/2/pdf/English/Mathematische Formulierung des Higgs-Mechanismus in der Zeit-Masse-Dualität_en.pdf}{Mathematical Formulation of the Higgs Mechanism in Time-Mass Duality}. March 28, 2025.
	\end{thebibliography}
	
\end{document}