\documentclass[12pt,a4paper]{article}
\usepackage[utf8]{inputenc}
\usepackage[T1]{fontenc}
\usepackage[ngerman]{babel}
\usepackage[left=2cm,right=2cm,top=2cm,bottom=2cm]{geometry}
\usepackage{lmodern}
\usepackage{amsmath}
\usepackage{amssymb}
\usepackage{physics}
\usepackage{hyperref}
\usepackage{tcolorbox}
\usepackage{booktabs}
\usepackage{enumitem}
\usepackage[table,xcdraw]{xcolor}
\usepackage{pgfplots}
\pgfplotsset{compat=1.18}
\usepackage{graphicx}
\usepackage{float}
\usepackage{mathtools}
\usepackage{tocloft}
\usepackage{fancyhdr}

\renewcommand{\cftsecfont}{\color{blue}}
\renewcommand{\cftsubsecfont}{\color{blue}}
\renewcommand{\cftsecpagefont}{\color{blue}}
\renewcommand{\cftsubsecpagefont}{\color{blue}}
\setlength{\cftsecindent}{1cm}
\setlength{\cftsubsecindent}{2cm}

\hypersetup{
	colorlinks=true,
	linkcolor=blue,
	citecolor=blue,
	urlcolor=blue,
	pdftitle={Mass Variation in Galaxies: An Analysis in the T0 Model with Emergent Gravitation},
	pdfauthor={Johann Pascher},
	pdfsubject={Theoretical Physics},
	pdfkeywords={T0 Model, Time-Mass Duality, Galaxy Dynamics, Dark Matter}
}

% Custom Commands
\newcommand{\Tfield}{T(x)}
\newcommand{\betaT}{\beta_{\text{T}}}
\newcommand{\alphaEM}{\alpha_{\text{EM}}}
\newcommand{\alphaW}{\alpha_{\text{W}}}
\newcommand{\Mpl}{M_{\text{Pl}}}
\newcommand{\Tzerot}{T_0(\Tfield)}
\newcommand{\Tzero}{T_0}
\newcommand{\vecx}{\vec{x}}
\newcommand{\DhiggsT}{\Tfield (\partial_\mu + ig A_\mu) \Phi + \Phi \partial_\mu \Tfield}
\newcommand{\DcovT}[1]{\Tfield D_\mu #1 + #1 \partial_\mu \Tfield}
\newcommand{\HiggsLagr}{\mathcal{L}_{\text{Higgs-T}}}

% Headers and Footers
\pagestyle{fancy}
\fancyhf{}
\fancyhead[L]{Johann Pascher}
\fancyhead[R]{Time-Mass Duality}
\fancyfoot[C]{\thepage}
\renewcommand{\headrulewidth}{0.4pt}
\renewcommand{\footrulewidth}{0.4pt}

\title{Mass Variation in Galaxies: \\An Analysis in the T0 Model with Emergent Gravitation}
\author{Johann Pascher}
\date{March 30, 2025}

\begin{document}
	
	\maketitle
	
	\begin{abstract}
		This work analyzes galaxy dynamics within the framework of the T0 model of time-mass duality theory, where time is absolute and mass varies as \( m = \frac{\hbar}{T c^2} \), with \( \Tfield \) as a dynamic intrinsic time field. Gravitation is not introduced as a fundamental interaction but emerges from the gradients of \( \Tfield \). We formulate a complete total Lagrangian density encompassing contributions from the four fundamental fields (Higgs, fermions, gauge bosons) and the intrinsic time field, demonstrating that flat rotation curves can be explained by the variation of \( \Tfield \) without requiring dark matter or separate dark energy. Experimental tests to validate the model are proposed, including cosmological implications such as the interpretation of the cosmic microwave background.
	\end{abstract}
	
	\tableofcontents
	\newpage
	
	\section{Introduction}
	The rotation curves of galaxies exhibit behavior that cannot be explained by visible matter alone. In the outer regions of spiral galaxies, the rotation velocity \( v(r) \) remains nearly constant instead of decreasing with \( r^{-1/2} \), as predicted by Kepler’s law for isolated masses. The standard cosmological model (\(\Lambda\)CDM) accounts for this phenomenon by assuming an invisible component, dark matter, forming an extended halo around galaxies and controlling the motion of visible matter through its gravitational field, supplemented by dark energy to explain cosmic acceleration.
	
	This work pursues an alternative approach based on the T0 model of time-mass duality theory, where time is absolute and particle mass varies as \( m = \frac{\hbar}{\Tfield c^2} \), with \( \Tfield \) as a dynamic intrinsic time field. In this framework, dark matter is not considered a separate entity; instead, the observed dynamic effects arise from emergent gravitation resulting from the gradients of \( \Tfield \). Similarly, effects traditionally attributed to dark energy, such as redshift, are explained by the spatial variation of \( \Tfield \), eliminating the need for separate dark energy as in the \(\Lambda\)CDM model. This reformulation yields mathematically equivalent predictions for rotation curves and offers a fundamentally different physical interpretation that requires neither dark matter nor separate dark energy. A detailed analysis of the cosmological implications of the T0 model, particularly regarding distance measurements, redshift, and the interpretation of the cosmic microwave background, is provided in \cite{pascher_messdifferenzen_2025}.
	
	\subsection{Redshift in the T0 Model}
	In the T0 model, redshift \( z \) is determined by the variation of the intrinsic time field \( \Tfield \). The relationship between redshift and mass is given by:
	\begin{equation}
		1 + z = \frac{\Tfield_0}{\Tfield} = \frac{m}{m_0},
	\end{equation}
	where \( \Tfield_0 \) and \( m_0 \) are the values of the intrinsic time field and mass at the observer’s location. This interpretation of redshift is based on intrinsic time and does not require cosmic expansion, in contrast to the \(\Lambda\)CDM model, where redshift is explained by the universe’s expansion:
	\begin{equation}
		1 + z = \frac{a(t_0)}{a(t_{\text{emit}})}.
	\end{equation}
	The spatial variation of \( \Tfield \) can be linked to distance \( d \) via \( \Tfield = \Tfield_0 e^{-\alpha d} \), where \( \alpha = H_0/c \), leading to an equivalent form:
	\begin{equation}
		1 + z = e^{\alpha d}.
	\end{equation}
	This formulation aligns with the energy loss of photons due to the dynamics of \( \Tfield \), as detailed in \cite{pascher_messdifferenzen_2025}. The relationship between redshift and distance \( d \) in the T0 model is thus:
	\begin{equation}
		d = \frac{c \ln(1 + z)}{H_0},
	\end{equation}
	where \( H_0 \) is the Hubble constant, reinterpreted in the T0 model as a measure of the spatial variation rate of \( \Tfield \) rather than an expansion rate.
	
	\subsection{Cosmological Implications: Distance Measures and CMB Interpretation}
	The T0 model has far-reaching implications for cosmological measurements, as detailed in \cite{pascher_messdifferenzen_2025}. In particular, distance measures in the T0 model differ from those in the \(\Lambda\)CDM model:
	
	- \textbf{Physical Distance \( d \):}
	\[
	d = \frac{c \ln(1 + z)}{H_0},
	\]
	compared to \(\Lambda\)CDM:
	\[
	d = \frac{c}{H_0} \int_0^z \frac{dz'}{\sqrt{\Omega_m (1 + z')^3 + \Omega_\Lambda}}.
	\]
	
	- \textbf{Luminosity Distance \( d_L \):}
	\[
	d_L = \frac{c}{H_0} \ln(1 + z) (1 + z),
	\]
	compared to \(\Lambda\)CDM:
	\[
	d_L = (1 + z) \cdot \frac{c}{H_0} \int_0^z \frac{dz'}{\sqrt{\Omega_m (1 + z')^3 + \Omega_\Lambda}}.
	\]
	
	- \textbf{Angular Diameter Distance \( d_A \):}
	\[
	d_A = \frac{c \ln(1 + z)}{H_0 (1 + z)},
	\]
	compared to \(\Lambda\)CDM:
	\[
	d_A = \frac{d}{1 + z}.
	\]
	
	Additionally, the CMB temperature-redshift relation in the T0 model is modified due to the dynamics of \( \Tfield \):
	\begin{equation}
		T(z) = T_0 (1 + z) (1 + \betaT \ln(1 + z)),
	\end{equation}
	with \( \betaT \approx 0.008 \) in SI units, compared to the \(\Lambda\)CDM prediction \( T(z) = T_0 (1 + z) \). These differences lead to significant deviations at high redshifts, particularly for the cosmic microwave background (CMB) at \( z = 1100 \). In the T0 model, the angular diameter distance \( d_A \) is approximately twice as large as in the \(\Lambda\)CDM model (28.9 Mpc vs. 13.5 Mpc), resulting in an angular size of structures of about \( 5.8^\circ \) in the T0 model compared to \( 1^\circ \) in the \(\Lambda\)CDM model. These dramatic differences provide an opportunity to experimentally test the models, as further elaborated in \cite{pascher_messdifferenzen_2025}.
	
	\begin{thebibliography}{99}
		\bibitem{pascher_zeit_2025} Pascher, J. (2025). \href{https://github.com/jpascher/T0-Time-Mass-Duality/tree/main/2/pdf/English/ZeitEmergentQMEn.pdf}{Time as an Emergent Property in Quantum Mechanics: A Connection Between Relativity, Fine-Structure Constant, and Quantum Dynamics}. March 23, 2025.
		\bibitem{pascher_messdifferenzen_2025} Pascher, J. (2025). \href{https://github.com/jpascher/T0-Time-Mass-Duality/tree/main/2/pdf/English/MessdifferenzenT0StandardEn.pdf}{Compensatory and Additive Effects: An Analysis of Measurement Differences Between the T0 Model and the \(\Lambda\)CDM Standard Model}. April 2, 2025.
		\bibitem{pascher_alpha_2025} Pascher, J. (2025). \href{https://github.com/jpascher/T0-Time-Mass-Duality/tree/main/2/pdf/English/NatEinheitenAlpha1En.pdf}{Energy as a Fundamental Unit: Natural Units with \(\alpha = 1\) in the T0 Model}. March 26, 2025.
		\bibitem{pascher_params_2025} Pascher, J. (2025). \href{https://github.com/jpascher/T0-Time-Mass-Duality/tree/main/2/pdf/English/ZeitMasseT0ParamsEn.pdf}{Time-Mass Duality Theory (T0 Model): Derivation of Parameters \(\kappa\), \(\alpha\), and \(\beta\)}. April 4, 2025.
		\bibitem{pascher_higgs_2025} Pascher, J. (2025). \href{https://github.com/jpascher/T0-Time-Mass-Duality/tree/main/2/pdf/English/MathHiggsZeitMasseEn.pdf}{Mathematical Formulation of the Higgs Mechanism in Time-Mass Duality}. March 28, 2025.
		\bibitem{pascher_lagrange_2025} Pascher, J. (2025). \href{https://github.com/jpascher/T0-Time-Mass-Duality/tree/main/2/pdf/English/MathZeitMasseLagrangeEn.pdf}{From Time Dilation to Mass Variation: Mathematical Core Formulations of Time-Mass Duality Theory}. March 29, 2025.
		\bibitem{pascher_emergente_gravitation_2025} Pascher, J. (2025). \href{https://github.com/jpascher/T0-Time-Mass-Duality/tree/main/2/pdf/English/EmergentGravT0En.pdf}{Emergent Gravitation in the T0 Model: A Comprehensive Derivation}. April 1, 2025.
		\bibitem{pascher_galaxies_2025} Pascher, J. (2025). \href{https://github.com/jpascher/T0-Time-Mass-Duality/tree/main/2/pdf/English/MassVarGalaxienEn.pdf}{Mass Variation in Galaxies: An Analysis in the T0 Model with Emergent Gravitation}. March 30, 2025.
		\bibitem{pascher_temp_2025} Pascher, J. (2025). \href{https://github.com/jpascher/T0-Time-Mass-Duality/tree/main/2/pdf/English/NatEinheitenAlpha1En.pdf}{Adjustment of Temperature Units in Natural Units and CMB Measurements}. April 2, 2025.
		\bibitem{pascher_alphabeta_2025} Pascher, J. (2025). \href{https://github.com/jpascher/T0-Time-Mass-Duality/tree/main/2/pdf/English/Alpha1Beta1KonsistenzEn.pdf}{Unified Unit System in the T0 Model: The Consistency of \(\alpha = 1\) and \(\beta = 1\)}. April 5, 2025.
		\bibitem{pascher_feldtheorie_2025} Pascher, J. (2025). \href{https://github.com/jpascher/T0-Time-Mass-Duality/tree/main/2/pdf/English/FeldtheorieQuantenEn.pdf}{Field Theory and Quantum Correlations: A New Perspective on Instantaneity}. March 28, 2025.
		\bibitem{pascher_planck_2025} Pascher, J. (2025). \href{https://github.com/jpascher/T0-Time-Mass-Duality/tree/main/2/pdf/English/JenseitsPlanckEn.pdf}{Real Consequences of Reformulating Time and Mass in Physics: Beyond the Planck Scale}. March 24, 2025.
		\bibitem{pascher_erweiterung_2025} Pascher, J. (2025). \href{https://github.com/jpascher/T0-Time-Mass-Duality/tree/main/2/pdf/English/NotwendigkeitQMErweiterungEn.pdf}{The Necessity of Extending Standard Quantum Mechanics and Quantum Field Theory}. March 27, 2025.
		\bibitem{rubin1980} Rubin, V. C., Ford Jr, W. K., \& Thonnard, N. (1980). Rotational properties of 21 SC galaxies with a large range of luminosities and radii, from NGC 4605 (R=4kpc) to UGC 2885 (R=122kpc). \textit{The Astrophysical Journal}, 238, 471-487. DOI: 10.1086/158003.
		\bibitem{McGaugh2016} McGaugh, S. S., Lelli, F., \& Schombert, J. M. (2016). Radial acceleration relation in rotationally supported galaxies. \textit{Physical Review Letters}, 117(20), 201101. DOI: 10.1103/PhysRevLett.117.201101.
		\bibitem{Milgrom1983} Milgrom, M. (1983). A modification of the Newtonian dynamics as a possible alternative to the hidden mass hypothesis. \textit{The Astrophysical Journal}, 270, 365-370. DOI: 10.1086/161130.
		\bibitem{Planck2018} Planck Collaboration, Aghanim, N., et al. (2020). Planck 2018 results. VI. Cosmological parameters. \textit{Astronomy \& Astrophysics}, 641, A6. DOI: 10.1051/0004-6361/201833910.
	\end{thebibliography}
	
\end{document}