\documentclass[a4paper,12pt]{article}
\usepackage[utf8]{inputenc}
\usepackage[T1]{fontenc}
\usepackage[english]{babel}
\usepackage{lmodern}
\usepackage{csquotes}
\usepackage{hyperref}
\usepackage{xcolor}
\usepackage{geometry}
\usepackage{booktabs}
\usepackage{array}
\usepackage{tabularx}
\usepackage{fancyhdr}

\geometry{a4paper, margin=2.5cm}
\hypersetup{
	colorlinks=true,
	linkcolor=blue,
	filecolor=magenta,      
	urlcolor=blue,
	pdftitle={Overview of Publications on Time-Mass Duality},
	pdfauthor={Johann Pascher},
	pdfcreator={LaTeX}
}

% Repository base URL
\newcommand{\repobase}{https://github.com/jpascher/T0-Time-Mass-Duality/tree/main/2/}

\pagestyle{fancy}
\fancyhf{}
\rhead{Johann Pascher}
\lhead{Time-Mass Duality}
\cfoot{\thepage}

\title{Overview of Publications on Time-Mass Duality \\ \Large{A Theoretical Framework for Extending Modern Physics}}
\author{Johann Pascher}
\date{March 2025}

\begin{document}
	
	\maketitle
	
	\begin{abstract}
		This overview presents a collection of works developing a new theoretical framework to extend physics: the Time-Mass Duality. This approach proposes a rethinking of time and mass, offering solutions to unresolved questions in quantum mechanics, quantum field theory, and cosmology—such as nonlocality or dark energy. The documents form a program spanning foundational ideas, mathematical models, and practical applications.
	\end{abstract}
	
	\section{Introduction}
	The following publications develop the Time-Mass Duality, a novel perspective on time and mass with far-reaching implications for physics. They are organized into five sections: starting point, conceptual foundations, mathematical formalization, applications, and frontier questions. All files are available in the repository at \url{\repobase}.
	
	\section{Starting Point: Field Theory as the Initial Idea}
	
	\subsection{\small\href{\repobase/pdf/English/Feldtheorie und Quantenkorrelationen_en.pdf}{Field Theory and Quantum Correlations}}
	\textit{(348.297 Bytes, 31.03.2025)}
	
	This document marks the starting point. It questions why particles seem instantly connected over large distances (nonlocality) and proposes a new field structure as an answer, later leading to the Time-Mass Duality.
	
	\section{Conceptual Foundations and Motivation}
	
	\subsection{\small\href{\repobase/pdf/English/Die Notwendigkeit einer Erweiterung der Standard-Quantenmechanik und Quantenfeldtheorie_en.pdf}{The Necessity of Extending Standard Quantum Mechanics \\ and Quantum Field Theory}}
	\textit{(257.169 Bytes, 31.03.2025)}
	
	It highlights weaknesses in conventional theories, e.g., linking quantum mechanics and gravity, and introduces Time-Mass Duality as a solution.
	
	\subsection{\small\href{\repobase/pdf/English/Kurzgefasst - Komplementärer Dualismus in der Physik - Von Welle-Teilchen zum Zeit-Masse-Konzept_en.pdf}{Summary - Complementary Dualism in Physics - From Wave-Particle to Time-Mass Concept}}
	\textit{(145.857 Bytes, 31.03.2025)}
	
	This document simply explains Time-Mass Duality: just as light is both wave and particle, time and mass might be two sides of the same coin.
	
	\subsection{\small\href{\repobase/pdf/English/Eine neue Perspektive auf Zeit und Raum Johann Paschers revolutionäre Ideen_en.pdf}{A New Perspective on Time and Space: Johann Pascher’s Revolutionary Ideas}}
	\textit{(235.024 Bytes, 31.03.2025)}
	
	Accessible to all, even without math: It introduces the T0 model, where time is fixed and mass varies—unlike Einstein’s view. It explains puzzles like instant particle connections or the universe’s expansion in a simple way.
	
	\section{Mathematical Formalization}
	
	\subsection{\small\href{\repobase/pdf/English/Wesentliche mathematische Formalismen der Zeit-Masse-Dualitätstheorie mit Lagrange-Dichten_en.pdf}{Essential Mathematical Formalisms of Time-Mass Duality Theory with Lagrange Densities}}
	\textit{(349.877 Bytes, 31.03.2025)}
	
	Here begins the precise development. Using simple rules (setting everything to 1), the theory is mathematically described, e.g., with the Lagrange method.
	
	\subsection{\small\href{\repobase/pdf/English/Mathematische Formulierungen der Zeit-Masse-Dualitätstheorie mit Lagrange_en.pdf}{Mathematical Formulations of Time-Mass Duality Theory with Lagrange}}
	\textit{(544.118 Bytes, 31.03.2025)}
	
	It deepens the models for particles like the Higgs field, showing how the theory works mathematically.
	
	\subsection{\small\href{\repobase/pdf/English/Mathematische Formulierung des Higgs-Mechanismus in der Zeit-Masse-Dualität_en.pdf}{Mathematical Formulation of the Higgs Mechanism in Time-Mass Duality}}
	\textit{(316.917 Bytes, 31.03.2025)}
	
	This document explains how the Higgs mechanism (which gives particles mass) fits into the new theory.
	
	\section{Applications and Extensions}
	
	\subsection{\small\href{\repobase/pdf/English/Dynamische Masse von Photonen und ihre Implikationen für Nichtlokalität_en.pdf}{Dynamic Mass of Photons and Its Implications for Nonlocality}}
	\textit{(265.909 Bytes, 31.03.2025)}
	
	It explores whether light (photons) has variable mass and how this explains particle connections.
	
	\subsection{\small\href{\repobase/pdf/English/Eine mathematische Analyse der Energiedynamik_en.pdf}{A Mathematical Analysis of Energy Dynamics}}
	\textit{(377.701 Bytes, 31.03.2025)}
	
	This document applies the theory to the universe, viewing dark energy as distributing energy, not causing expansion.
	
	\section{Cosmological and Frontier Areas}
	
	\subsection{\small\href{\repobase/pdf/English/Jenseits der Planck-Skala_en.pdf}{Beyond the Planck Scale}}
	\textit{(347.870 Bytes, 31.03.2025)}
	
	It asks how the theory might address the smallest (Planck scale) and biggest questions in physics—like black holes or the early universe.
	
	\section{Additional Relevant Documents}
	
	\subsection{\small\href{\repobase/pdf/English/Massenvariation in Galaxien_en.pdf}{Mass Variation in Galaxies}}
	\textit{(347.376 Bytes, 31.03.2025)}
	
	It shows how variable mass affects galaxies, explaining star movements without dark matter.
	
	\subsection{\small\href{\repobase/pdf/English/Vereinheitlichung des T0-Modells Grundlagen - Dunkle Energie und Galaxiendynamik_en.pdf}{Unification of the T0 Model: Foundations - Dark Energy and Galaxy Dynamics}}
	\textit{(351.434 Bytes, 31.03.2025)}
	
	A comprehensive work applying the theory to cosmology and galaxies.
	
	\subsection{\small\href{\repobase/pdf/English/Natürliche Einheiten mit Feinstrukturkonstante alpha = 1_en.pdf}{Natural Units with Fine-Structure Constant alpha = 1}}
	\textit{(336.496 Bytes, 31.03.2025)}
	
	It proposes a simple system where a key number (fine-structure constant) is set to 1 to simplify physics.
	
	\section{German Versions}
	
	Additionally, German versions are available:
	\begin{itemize}
		\item \small\href{\repobase/pdf/Deutsch/Die Notwendigkeit einer Erweiterung der Standard-Quantenmechanik und Quantenfeldtheorie.pdf}{Die Notwendigkeit einer Erweiterung der Standard-Quantenmechanik und Quantenfeldtheorie} (276.670 Bytes)
		\item \small\href{\repobase/pdf/Deutsch/Dynamische Masse von Photonen und ihre Implikationen für Nichtlokalität.pdf}{Dynamische Masse von Photonen und ihre Implikationen für Nichtlokalität} (276.670 Bytes)
		\item \small\href{\repobase/pdf/Deutsch/Eine mathematische Analyse der Energiedynamik.pdf}{Eine mathematische Analyse der Energiedynamik} (388.573 Bytes)
		\item \small\href{\repobase/pdf/Deutsch/Eine neue Perspektive auf Zeit und Raum Johann Paschers revolutionäre Ideen.pdf}{Eine neue Perspektive auf Zeit und Raum} (242.204 Bytes)
		\item \small\href{\repobase/pdf/Deutsch/Feldtheorie und Quantenkorrelationen.pdf}{Feldtheorie und Quantenkorrelationen} (356.638 Bytes)
		\item \small\href{\repobase/pdf/Deutsch/Jenseits der Planck-Skala.pdf}{Jenseits der Planck-Skala} (351.328 Bytes)
		\item \small\href{\repobase/pdf/Deutsch kurzgefasst/Kurzgefasst - Komplementärer Dualismus in der Physik - Von Welle-Teilchen zum Zeit-Masse-Konzept.pdf}{Kurzgefasst - Komplementärer Dualismus in der Physik} (149.923 Bytes)
		\item \small\href{\repobase/pdf/Deutsch/Massenvariation in Galaxien.pdf}{Massenvariation in Galaxien} (362.547 Bytes)
		\item \small\href{\repobase/pdf/Deutsch/Mathematische Formulierung des Higgs-Mechanismus in der Zeit-Masse-Dualität.pdf}{Mathematische Formulierung des Higgs-Mechanismus} (325.463 Bytes)
		\item \small\href{\repobase/pdf/Deutsch/Mathematische Formulierungen der Zeit-Masse-Dualitätstheorie mit Lagrange.pdf}{Mathematische Formulierungen der Zeit-Masse-Dualitätstheorie mit Lagrange} (559.012 Bytes)
	\end{itemize}
	
	\section{Summary and Outlook}
	
	These works form a program rethinking physics. Time-Mass Duality uses simple rules to address big questions like nonlocality or dark energy. Future steps could include testing the theory, refining models, and simulations to create a unified physics.
	
\end{document}