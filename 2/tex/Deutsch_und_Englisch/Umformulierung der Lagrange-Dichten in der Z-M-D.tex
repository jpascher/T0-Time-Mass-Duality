\documentclass{article}
\usepackage[utf8]{inputenc}
\usepackage[ngerman]{babel}
\usepackage{amsmath}
\usepackage{amssymb}
\usepackage{physics}
\usepackage{graphicx}
\usepackage{hyperref}
\usepackage{xcolor}

\title{Umformulierung der Lagrange-Dichten in der Zeit-Masse-Dualität}
\author{Johann Pascher}
\date{29. März 2025}

\begin{document}
	
	\maketitle
	
	\section*{Einleitung}
	Ich werde versuchen, eine konsistente Umformulierung der grundlegenden Lagrange-Dichten zu entwickeln, ausgehend von der Zeit-Masse-Dualitätstheorie. Das Ziel ist, eine mathematisch kohärente und physikalisch sinnvolle Formulierung zu schaffen, die alle wesentlichen Aspekte der Theorie erfasst.
	
	\section{Grundlegende Prinzipien}
	Beginnen wir mit den Fundamentalprinzipien der Zeit-Masse-Dualität:
	
	\begin{itemize}
		\item Intrinsische Zeit: \( T = \frac{\hbar}{m c^2} \)
		\item Modifizierte Zeitableitung: \( \partial_{t/T} = \frac{\partial}{\partial(t/T)} = T \frac{\partial}{\partial t} \)
		\item Dualität zwischen: Standardbild (Zeitdilatation, konstante Masse) und Alternativbild (absolute Zeit, variable Masse)
	\end{itemize}
	
	\section{Modifizierte Lagrange-Dichte für skalare Felder}
	Die Standard-Lagrange-Dichte für ein skalares Feld (wie das Higgs-Feld) lautet:
	
	\begin{equation}
		\mathcal{L}_{\text{skalar}} = \frac{1}{2} (\partial_\mu \phi) (\partial^\mu \phi) - \frac{1}{2} m^2 \phi^2 - V(\phi)
	\end{equation}
	
	In der Zeit-Masse-Dualität wird dies zu:
	
	\begin{equation}
		\mathcal{L}_{\text{skalar-T}} = \frac{1}{2} (D_{T\mu} \phi) (D_T^\mu \phi) - \frac{1}{2} m^2 \phi^2 - V(\phi)
	\end{equation}
	
	wobei die modifizierte kovariante Ableitung definiert ist als:
	
	\begin{equation}
		D_{T\mu} \phi = T(x) \partial_\mu \phi + \phi \partial_\mu T(x)
	\end{equation}
	
	Explizit ausgeschrieben:
	
	\begin{equation}
		\mathcal{L}_{\text{skalar-T}} = \frac{1}{2} T(x)^2 \left( \frac{\partial \phi}{\partial t} \right)^2 + T(x) \phi \frac{\partial \phi}{\partial t} \frac{\partial T(x)}{\partial t} - \frac{1}{2} (\nabla \phi)^2 - \frac{1}{2} m^2 \phi^2 - V(\phi)
	\end{equation}
	
	\section{Vollständige Higgs-Lagrange-Dichte}
	Für das Higgs-Feld als komplexes Dublett erhalten wir:
	
	\begin{equation}
		\mathcal{L}_{\text{Higgs-T}} = (D_{T\mu} \Phi_T)^\dagger (D_T^\mu \Phi_T) - V_T(\Phi_T)
	\end{equation}
	
	mit der kovarianten Ableitung:
	
	\begin{equation}
		D_{T\mu} \Phi_T = T(x) (\partial_\mu + i g \tau^a W_\mu^a + i g' \frac{Y}{2} B_\mu) \Phi_T + \Phi_T \partial_\mu T(x)
	\end{equation}
	
	Das Higgs-Potential behält seine Form:
	
	\begin{equation}
		V_T(\Phi_T) = -\mu^2 \Phi_T^\dagger \Phi_T + \lambda (\Phi_T^\dagger \Phi_T)^2
	\end{equation}
	
	\section{Umformulierte Yukawa-Kopplung}
	Die Yukawa-Kopplung wird modifiziert zu:
	
	\begin{equation}
		\mathcal{L}_{\text{Yukawa-T}} = -y_f \bar{\psi}_L \Phi_T \psi_R + \text{h.c.}
	\end{equation}
	
	Die Transformationsfunktion \( \mathcal{T}(\gamma) \) wird hier nicht explizit benötigt, da die Massenvariation durch \( T(x) \) implizit berücksichtigt wird.
	
	\section{Lagrange-Dichte für Fermionen}
	Die Dirac-Lagrange-Dichte für Fermionen wird zu:
	
	\begin{equation}
		\mathcal{L}_{\text{Dirac-T}} = \bar{\psi} (i \gamma^\mu D_{T\mu} - m) \psi
	\end{equation}
	
	mit:
	
	\begin{equation}
		D_{T\mu} \psi = T(x) D_\mu \psi + \psi \partial_\mu T(x)
	\end{equation}
	
	wobei \( D_\mu \) die übliche kovariante Ableitung mit Eichfeldern ist.
	
	\section{Eichboson-Lagrange-Dichte}
	Für Eichbosonen wird die Lagrange-Dichte modifiziert zu:
	
	\begin{equation}
		\mathcal{L}_{\text{Eich-T}} = -\frac{1}{4} T(x)^2 F_{\mu\nu} F^{\mu\nu}
	\end{equation}
	
	mit dem unveränderten Feldstärketensor:
	
	\begin{equation}
		F_{\mu\nu} = \partial_\mu A_\nu - \partial_\nu A_\mu + i g [A_\mu, A_\nu]
	\end{equation}
	
	\section{Einheitliche Formulierung der vollständigen Lagrange-Dichte}
	Die Gesamt-Lagrange-Dichte lautet nun:
	
	\begin{equation}
		\mathcal{L}_{\text{Gesamt-T}} = \mathcal{L}_{\text{Higgs-T}} + \mathcal{L}_{\text{Dirac-T}} + \mathcal{L}_{\text{Yukawa-T}} + \mathcal{L}_{\text{Eich-T}}
	\end{equation}
	
	\section{Feldgleichungen aus der modifizierten Lagrange-Dichte}
	Die Feldgleichungen ergeben sich durch Anwendung der Euler-Lagrange-Gleichungen:
	
	Für das Higgs-Feld:
	
	\begin{equation}
		D_{T\mu} D_T^\mu \Phi_T + \frac{\partial V_T}{\partial \Phi_T^\dagger} = 0
	\end{equation}
	
	Für Fermionen:
	
	\begin{equation}
		(i \gamma^\mu D_{T\mu} - m) \psi = 0
	\end{equation}
	
	Für Eichbosonen:
	
	\begin{equation}
		\partial_\mu (T(x)^2 F^{\mu\nu}) + i g [A_\mu, T(x)^2 F^{\mu\nu}] = j^\nu
	\end{equation}
	
	\section{Gravitationseinbindung über modifizierte Einstein-Hilbert-Aktion}
	Die Einstein-Hilbert-Aktion wird modifiziert zu:
	
	\begin{equation}
		S_{\text{Grav-T}} = \frac{1}{16\pi G} \int d^4x \sqrt{-g} T(x) R
	\end{equation}
	
	wobei \( R \) der Ricci-Skalar ist, angepasst durch \( T(x) \).
	
	\section{Zusammenfassung und Konsistenzprüfung}
	Die Umformulierung basiert auf der konsistenten Einführung von \( T(x) \) in alle Ableitungen und Feldterme. Die Theorie sollte:
	
	\begin{itemize}
		\item Lorentz-invariant bleiben unter Berücksichtigung der Dualität
		\item Phänomene wie Zeitdilatation und Massenvariation korrekt beschreiben
		\item Testbare Abweichungen vom Standardmodell vorhersagen
	\end{itemize}
	
	\section{Umfassende Lagrange-Dichte der Zeit-Masse-Dualitätstheorie}
	Die vollständige Lagrange-Dichte ist:
	
	\begin{equation}
		\mathcal{L}_{\text{Gesamt-T}} = \mathcal{L}_{\text{Higgs-T}} + \mathcal{L}_{\text{Fermion-T}} + \mathcal{L}_{\text{Eich-T}} + \mathcal{L}_{\text{Yukawa-T}}
	\end{equation}
	
	\subsection{Higgs-Sektor}
	\begin{equation}
		\mathcal{L}_{\text{Higgs-T}} = (D_{T\mu} \Phi_T)^\dagger (D_T^\mu \Phi_T) - V_T(\Phi_T)
	\end{equation}
	
	mit:
	\begin{itemize}
		\item \( D_{T\mu} \Phi_T = T(x) (\partial_\mu + i g \tau^a W_\mu^a + i g' \frac{Y}{2} B_\mu) \Phi_T + \Phi_T \partial_\mu T(x) \)
		\item \( V_T(\Phi_T) = -\mu^2 \Phi_T^\dagger \Phi_T + \lambda (\Phi_T^\dagger \Phi_T)^2 \)
		\item \( T(x) = \frac{\hbar}{m c^2} \)
	\end{itemize}
	
	\subsection{Fermion-Sektor}
	\begin{equation}
		\mathcal{L}_{\text{Fermion-T}} = \sum_f \bar{\psi}_f (i \gamma^\mu D_{T\mu} - m_f) \psi_f
	\end{equation}
	
	\subsection{Eichboson-Sektor}
	\begin{equation}
		\mathcal{L}_{\text{Eich-T}} = -\frac{1}{4} T(x)^2 (G_{\mu\nu}^a G^{a\mu\nu} + W_{\mu\nu}^a W^{a\mu\nu} + B_{\mu\nu} B^{\mu\nu})
	\end{equation}
	
	\subsection{Yukawa-Sektor}
	\begin{equation}
		\mathcal{L}_{\text{Yukawa-T}} = -\sum_f y_f \bar{\psi}_{fL} \Phi_T \psi_{fR} + \text{h.c.}
	\end{equation}
	
	\subsection{Energieimpulsbeziehung}
	Die modifizierte Energieimpulsbeziehung lautet:
	
	\begin{equation}
		E^2 = (p c)^2 + (m c^2)^2 + \alpha \frac{\hbar c}{T}
	\end{equation}
	
\end{document}