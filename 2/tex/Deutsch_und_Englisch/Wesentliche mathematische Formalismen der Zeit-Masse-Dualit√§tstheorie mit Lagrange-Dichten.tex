\documentclass[12pt,a4paper]{article}
\usepackage[margin=2cm]{geometry}
\usepackage[utf8]{inputenc}
\usepackage[T1]{fontenc}
\usepackage{lmodern}
\usepackage[ngerman]{babel}
\usepackage{amsmath,amssymb,physics,graphicx,xcolor,amsthm}
\usepackage{hyperref}
\usepackage{booktabs}
\usepackage{siunitx}
\usepackage{cleveref}
\usepackage{pgfplots}
\pgfplotsset{compat=1.18}
\usepackage{tikz}
\usetikzlibrary{intersections}
\usepgfplotslibrary{fillbetween}

% Custom commands
\newcommand{\Tfield}{T(x)}
\newcommand{\betaT}{\beta_{\text{T}}}
\newcommand{\alphaEM}{\alpha_{\text{EM}}}
\newcommand{\Tzero}{T_0}
\newcommand{\DcovT}[1]{\partial_\mu #1 + #1 \partial_\mu \Tfield}
\newcommand{\DhiggsT}{\Tfield (\partial_\mu + ig A_\mu) \Phi + \Phi \partial_\mu \Tfield} % Korrigierte Definition
\newcommand{\gammaf}{\gamma_{\text{Lorentz}}}

% Theorem styles
\newtheorem{theorem}{Satz}[section]
\newtheorem{proposition}[theorem]{Proposition}
\newtheorem{corollary}[theorem]{Korollar}
\newtheorem{lemma}[theorem]{Lemma}
\theoremstyle{definition}
\newtheorem{definition}[theorem]{Definition}
\newtheorem{example}[theorem]{Beispiel}
\theoremstyle{remark}
\newtheorem{remark}[theorem]{Bemerkung}

% Hyperref configuration
\hypersetup{
	colorlinks=true,
	linkcolor=blue,
	urlcolor=blue,
	citecolor=blue, % Angepasst von red zu blue
	pdftitle={Von Zeitdilatation zu Massenvariation: Mathematische Kernformulierungen der Zeit-Masse-Dualitätstheorie},
	pdfauthor={Johann Pascher},
	pdfsubject={Theoretische Physik}, % Hinzugefügt
	pdfkeywords={T0-Modell, Zeit-Masse-Dualität, emergente Gravitation} % Hinzugefügt
}

\title{Von Zeitdilatation zu Massenvariation: \\ Mathematische Kernformulierungen der Zeit-Masse-Dualitätstheorie}
\author{Johann Pascher}
\date{29. März 2025}

\begin{document}
	
	\maketitle
	
	\begin{abstract}
		Diese Arbeit stellt die wesentlichen mathematischen Formulierungen der Zeit-Masse-Dualitätstheorie vor, mit Fokus auf die grundlegenden Gleichungen und ihre physikalischen Interpretationen. Die Theorie etabliert eine Dualität zwischen zwei komplementären Beschreibungen der Realität: dem Standardbild mit Zeitdilatation und konstanter Ruhemasse und dem T0-Modell mit absoluter Zeit und variabler Masse. Im Zentrum dieses Rahmens steht die intrinsische Zeit \( \Tfield = \frac{\hbar}{\max(m c^2, \omega)} \), die eine einheitliche Behandlung von massiven Teilchen und Photonen ermöglicht. Die mathematischen Formulierungen umfassen modifizierte Lagrangedichten, die emergente Gravitation und Rotverschiebung durch Energieverlust in einem statischen Universum betonen.
	\end{abstract}
	
	\tableofcontents
	\newpage
	
	\section{Einführung in die Zeit-Masse-Dualität}
	Die Zeit-Masse-Dualitätstheorie schlägt einen alternativen Rahmen vor:
	\begin{enumerate}
		\item Standardbild: \( t' = \gammaf t \), \( m_0 = \text{konst.} \)
		\item T0-Modell: \( \Tzero = \text{konst.} \), \( m = \gammaf m_0 \)
	\end{enumerate}
	
	\subsection{Beziehung zum Standardmodell}
	Das T0-Modell erweitert das Standardmodell mit:
	\begin{enumerate}
		\item Intrinsisches Zeitfeld: \( \Tfield = \frac{\hbar}{\max(m c^2, \omega)} \)
		\item Higgs-Feld: \( \Phi \) mit dynamischer Massenkopplung
		\item Fermionenfelder: \( \psi \) mit Yukawa-Kopplung
		\item Eichbosonenfelder: \( A_\mu \) mit \( \Tfield \)-Wechselwirkung
	\end{enumerate}
	
	\section{Emergente Gravitation aus dem intrinsischen Zeitfeld}
	\begin{theorem}[Gravitationsentstehung]
		Gravitation entsteht aus Gradienten des intrinsischen Zeitfelds:
		\begin{equation}
			\nabla \Tfield = -\frac{\hbar}{m^2 c^2} \nabla m
		\end{equation}
		mit dem modifizierten Potential:
		\begin{equation}
			\Phi(r) = -\frac{GM}{r} + \kappa r, \quad \kappa \approx 4.8 \times 10^{-11} \, \text{m/s}^2
		\end{equation}
	\end{theorem}
	
	\begin{proof}
		Aus \( \Tfield = \frac{\hbar}{m c^2} \) für massive Teilchen:
		\begin{equation}
			\nabla \Tfield = -\frac{\hbar}{m^2 c^2} \nabla m
		\end{equation}
		Mit \( m(\vec{r}) = m_0 (1 + \frac{\Phi_g}{c^2}) \):
		\begin{equation}
			\nabla m = \frac{m_0}{c^2} \nabla \Phi_g
		\end{equation}
		Daher:
		\begin{equation}
			\nabla \Tfield \approx -\frac{\hbar}{m_0 c^4} \nabla \Phi_g
		\end{equation}
	\end{proof}
	
	\section{Mathematische Grundlagen: Intrinsische Zeit}
	\begin{theorem}[Intrinsische Zeit]
		\begin{equation}
			\Tfield = \frac{\hbar}{\max(m c^2, \omega)}
		\end{equation}
	\end{theorem}
	
	\section{Modifizierte Ableitungsoperatoren}
	\begin{definition}[Modifizierte Ableitung]
		Die modifizierte kovariante Ableitung im T0-Modell lautet:
		\begin{equation}
			\DcovT{\Psi} = \partial_\mu \Psi + \Psi \partial_\mu \Tfield
		\end{equation}
	\end{definition}
	
	\section{Modifizierte Feldgleichungen}
	\begin{theorem}[Modifizierte Schrödinger-Gleichung]
		\begin{equation}
			i\hbar \Tfield \frac{\partial}{\partial t} \Psi + i\hbar \Psi \frac{\partial \Tfield}{\partial t} = \hat{H} \Psi
		\end{equation}
	\end{theorem}
	
	\section{Modifizierte Lagrangedichte für das Higgs-Feld}
	\begin{theorem}[Higgs-Lagrangedichte]
		Die Lagrangedichte des Higgs-Felds mit Kopplung an \(\Tfield\) lautet:
\section{Modifizierte Lagrangedichte für das Higgs-Feld}
\begin{theorem}[Higgs-Lagrangedichte]
	Die Lagrangedichte des Higgs-Felds mit Kopplung an \(\Tfield\) lautet:
	\begin{multline}
		\mathcal{L}_{\text{Higgs-T}} = |\DhiggsT|^2 + \frac{1}{2} \partial_\mu \Tfield \partial^\mu \Tfield - V(\Tfield, \Phi), \quad \\
		\DhiggsT = \Tfield (\partial_\mu + ig A_\mu) \Phi + \Phi \partial_\mu \Tfield
	\end{multline}
\end{theorem}
	\end{theorem}
	
	\section{Modifizierte Lagrangedichte für Fermionen}
	\begin{theorem}[Fermionen-Lagrangedichte]
		\begin{equation}
			\mathcal{L}_{\text{Fermion}} = \bar{\psi} i \gamma^\mu (\partial_\mu \psi + \psi \partial_\mu \Tfield) - y \bar{\psi} \Phi \psi
		\end{equation}
	\end{theorem}
	
	\section{Modifizierte Lagrangedichte für Eichbosonen}
	\begin{theorem}[Eichbosonen-Lagrangedichte]
		\begin{equation}
			\mathcal{L}_{\text{Boson}} = -\frac{1}{4} F_{\mu\nu} F^{\mu\nu} + \frac{1}{2} \partial_\mu \Tfield \partial^\mu \Tfield
		\end{equation}
	\end{theorem}
	
	\section{Vollständige Gesamt-Lagrangedichte}
	\begin{theorem}[Gesamt-Lagrangedichte]
		\begin{equation}
			\mathcal{L}_{\text{Total}} = \mathcal{L}_{\text{Boson}} + \mathcal{L}_{\text{Fermion}} + \mathcal{L}_{\text{Higgs-T}} + \mathcal{L}_{\text{intrinsic}}, \quad \mathcal{L}_{\text{intrinsic}} = \frac{1}{2} \partial_\mu \Tfield \partial^\mu \Tfield - V(\Tfield)
		\end{equation}
	\end{theorem}
	
	\section{Kosmologische Implikationen}
	Das T0-Modell impliziert:
	\begin{itemize}
		\item Modifiziertes Gravitationspotential: \( \Phi(r) = -\frac{GM}{r} + \kappa r \), \( \kappa \approx 4.8 \times 10^{-11} \, \text{m/s}^2 \)
		\item Kosmische Rotverschiebung: \( 1 + z = e^{\alpha d} \), \( \alpha \approx 2.3 \times 10^{-28} \, \text{m}^{-1} \)
		\item Wellenlängenabhängigkeit: \( z(\lambda) = z_0 (1 + \betaT \ln(\lambda/\lambda_0)) \), \( \betaT \approx 0.008 \) (SI-Einheiten)
	\end{itemize}
	\section{Herleitung von \(\betaT\) im T0-Modell}
	
	Der Parameter \(\betaT\) beschreibt die Kopplung des intrinsischen Zeitfelds \(\Tfield\) an physikalische Phänomene wie die wellenlängenabhängige Rotverschiebung. Im T0-Modell wird \(\betaT\) präzise hergeleitet als:
	\begin{equation}
		\betaT = \frac{\lambda_h^2 v^2}{16\pi^3} \cdot \frac{1}{m_h^2} \cdot \frac{1}{\xi}
	\end{equation}
	wobei \(\lambda_h\) die Higgs-Selbstkopplung, \(v\) der Higgs-Vakuumerwartungswert, \(m_h\) die Higgs-Masse und \(\xi \approx 1.33 \times 10^{-4}\) ein dimensionsloser Parameter ist, der die charakteristische Längenskala \(r_0 = \xi \cdot l_P\) definiert (\(l_P\): Planck-Länge). In natürlichen Einheiten gilt \(\betaT = 1\), was eine exakte theoretische Vorhersage darstellt, da sie direkt aus den Modellparametern folgt, wie in \cite{pascher_alphabeta_2025} detailliert beschrieben. Eine ausführliche Herleitung und Diskussion dieses Parameters findet sich in \cite{pascher_alphabeta_2025}.
	
	\begin{thebibliography}{9}
		\bibitem{pascher_alphabeta_2025} Pascher, J. (2025). \href{https://github.com/jpascher/T0-Time-Mass-Duality/tree/main/2/pdf/Deutsch/Die\%20Konsistenz\%20von\%20alpha\%20=\%201\%20und\%20beta\%20=\%201.pdf}{Vereinheitlichtes Einheitensystem im T0-Modell: Die Konsistenz von \(\alpha = 1\) und \(\beta = 1\)}. 5. April 2025.
		\bibitem{pascher_massenvariation_2025} Pascher, J. (2025). \href{https://github.com/jpascher/T0-Time-Mass-Duality/tree/main/2/pdf/Deutsch/Massenvariation\%20in\%20Galaxien.pdf}{Massenvariation in Galaxien: Eine Analyse im T0-Modell mit emergenter Gravitation}. 30. März 2025.
	\end{thebibliography}
	
\end{document}higgd