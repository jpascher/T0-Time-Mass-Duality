\documentclass[a4paper,12pt]{article}
\usepackage[utf8]{inputenc}
\usepackage[T1]{fontenc}
\usepackage{lmodern}
\usepackage[ngerman]{babel} % Änderung zu Deutsch
\usepackage{amsmath, amssymb, amsthm, physics}
\usepackage{graphicx}
\usepackage{xcolor}
\usepackage{tikz}
\usepackage{pgfplots}
\pgfplotsset{compat=1.18}
\usepackage{setspace}
\usepackage{booktabs}
\usepackage{siunitx}
\usepackage{array}
\usepackage{float}
\usepackage[section]{placeins}

\usepackage{hyperref}
\hypersetup{
	colorlinks=true,
	linkcolor=blue,
	filecolor=blue,
	citecolor=blue, 
	urlcolor=blue,
	bookmarks=true,
	bookmarksopen=true,
	pdftitle={Kompensatorische und additive Effekte: Eine Analyse der Messdifferenzen zwischen dem T0-Modell und dem \(\Lambda\)CDM-Standardmodell},
	pdfauthor={Johann Pascher},
}

% Benutzerdefinierter Befehl für Tfield
\newcommand{\Tfield}{T(x)}

% Repository base URL
\newcommand{\repobase}{https://github.com/jpascher/T0-Time-Mass-Duality/tree/main/2/}

\begin{document}
	
	\title{Kompensatorische und additive Effekte: Eine Analyse der Messdifferenzen zwischen dem T0-Modell und dem \(\Lambda\)CDM-Standardmodell}
	\author{Johann Pascher}
	\date{2. April 2025}
	\maketitle
	
	\begin{abstract}
		Dieses Dokument analysiert die Unterschiede in kosmologischen Messungen zwischen dem Standardmodell (\(\Lambda\)CDM) und dem alternativen T0-Modell. Wir untersuchen, wie die unterschiedlichen theoretischen Grundlagen Distanzmessungen, Rotverschiebungen und die Interpretation des kosmischen Mikrowellenhintergrunds beeinflussen. Besondere Aufmerksamkeit wird darauf gelegt, ob die Effekte sich gegenseitig verstärken (additiv wirken) oder kompensieren. Die Analyse zeigt ein komplexes Zusammenspiel, das das Hubble-Spannungsproblem erklären könnte. Bei niedrigen Rotverschiebungen (z \(\approx\) 1) sind die Unterschiede moderat, während sie bei hohen Rotverschiebungen (z = 1100, CMB) dramatisch werden und zu grundlegend unterschiedlichen Interpretationen führen.
	\end{abstract}
	
	\tableofcontents
	\newpage
	
	\section{Einführung}
	
	Das kosmologische Standardmodell (\(\Lambda\)CDM) und das alternative T0-Modell bieten grundlegend unterschiedliche Erklärungen für dieselben astronomischen Beobachtungen. Während \(\Lambda\)CDM auf einem expandierenden Universum basiert, postuliert das T0-Modell ein statisches Universum mit absoluter Zeit und variabler Masse. Diese Arbeit untersucht, wie diese unterschiedlichen theoretischen Grundlagen kosmologische Messungen beeinflussen und wie diese Effekte sich entweder verstärken oder kompensieren.
	
	\subsection{Das T0-Modell}
	
	Im {\small\href{\repobase/pdf/Deutsch/Wesentliche mathematische Formalismen der Zeit-Masse-Dualitätstheorie mit Lagrange-Dichten_de.pdf}{T0-Modell}} ist Zeit absolut, während die Masse als \( m = \frac{\hbar}{\Tfield c^2} \) variiert, wobei \( \Tfield \) das intrinsische Zeitfeld ist, das über \( \Tfield = \frac{\hbar}{y \langle \Phi \rangle c^2} \) mit dem Higgs-Feld gekoppelt ist \cite{pascher_galaxies_2025}. Die Rotverschiebung entsteht durch die räumliche Variation von \( \Tfield \), die einen Energieverlust von Photonen verursacht:
	
	\begin{equation}
		1 + z = \frac{\Tfield}{\Tfield_0},
	\end{equation}
	
	wobei \( \Tfield_0 \) der Wert am Ort des Beobachters ist. Diese Variation kann als \( \Tfield = \Tfield_0 e^{-\alpha d} \) ausgedrückt werden, mit \( \alpha = H_0/c \), was zu folgender Form führt:
	
	\begin{equation}
		1 + z = e^{\alpha d},
	\end{equation}
	
	wobei \( d \) die physische Distanz und \( H_0 \) die Hubble-Konstante ist, neu interpretiert als die Rate der räumlichen Änderung von \( \Tfield \) anstatt als Expansionsparameter. Im Gegensatz zum \(\Lambda\)CDM-Modell, wo Dunkle Energie (\(\Lambda\)) die kosmische Beschleunigung antreibt, dient \( \Tfield \) im T0-Modell als effektives Feld, das die Rotverschiebung ohne Expansion erklärt, während seine Gradienten auch emergente Gravitation erklären \cite{pascher_galaxies_2025}.
	
	\subsection{T0-Modell (CMB-Temperatur)}
	
	Im T0-Modell weist die CMB-Temperatur eine leichte Modifikation aufgrund der Dynamik von \( \Tfield \) auf:
	
	\begin{equation}
		T(z) = T_0 (1 + z) (1 + \beta \ln(1 + z)),
	\end{equation}
	
	mit \( \beta \approx 0.008 \), was zu einer subtilen Abweichung von der Vorhersage des Standardmodells \( T(z) = T_0 (1 + z) \) führt. Dies ergibt sich, weil \( \Tfield \) sowohl die Massenvariation als auch die Rotverschiebung steuert, wie in \cite{pascher_galaxies_2025} detailliert beschrieben. Der Parameter \( \beta \) spiegelt den logarithmischen Einfluss der räumlichen Variation von \( \Tfield \) auf den Energieverlust von Photonen wider, konsistent mit \( 1 + z = e^{\alpha d} \).
	
	% Rest des Dokuments bleibt unverändert
	
	\begin{thebibliography}{9}
		\bibitem{pascher_galaxies_2025} Pascher, J. (2025). \href{\repobase/pdf/Deutsch/Massenvariation in Galaxien - Eine Analyse im T0-Modell mit emergenter Gravitation.pdf}{Massenvariation in Galaxien: Eine Analyse im T0-Modell mit emergenter Gravitation}. 30. März 2025.
		\bibitem{Planck2018} Planck Collaboration, Aghanim, N., et al. (2020). \textit{Planck 2018 Ergebnisse. VI. Kosmologische Parameter}. Astronomy \& Astrophysics, 641, A6. DOI: 10.1051/0004-6361/201833910.
		% Weitere Einträge wie im Original
	\end{thebibliography}
	
\end{document}