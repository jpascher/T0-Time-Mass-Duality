\documentclass[a4paper,12pt]{article}
\usepackage[utf8]{inputenc}
\usepackage[T1]{fontenc}
\usepackage{lmodern}
\usepackage[ngerman]{babel}
\usepackage{amsmath, amssymb, amsthm, physics}
\usepackage{graphicx}
\usepackage{xcolor}
\usepackage{tikz}
\usepackage{pgfplots}
\pgfplotsset{compat=1.18}
\usepackage{setspace}
\usepackage{booktabs}
\usepackage{siunitx}
\usepackage{array}
\usepackage{float}
\usepackage[section]{placeins}


% Farbige Links im Inhaltsverzeichnis und im Dokument
\usepackage{hyperref}
\hypersetup{
	colorlinks=true,
	linkcolor=blue,
	filecolor=blue,
	citecolor=blue, 
	urlcolor=blue,
	bookmarks=true,
	bookmarksopen=true,
	pdftitle={Kompensatorische und Additive Effekte: Eine Analyse der Messdifferenzen zwischen dem T0-Modell und dem ΛCDM-Standardmodell},
	pdfauthor={},
}

% Theorem-Stile
\newtheorem{theorem}{Theorem}[section]
\newtheorem{lemma}[theorem]{Lemma}
\newtheorem{proposition}[theorem]{Proposition}
\newtheorem{corollary}[theorem]{Korollar}

\theoremstyle{definition}
\newtheorem{definition}[theorem]{Definition}
\newtheorem{example}[theorem]{Beispiel}

\theoremstyle{remark}
\newtheorem{remark}[theorem]{Bemerkung}
\renewcommand{\proofname}{Beweis}
% Repository base URL
\newcommand{\repobase}{https://github.com/jpascher/T0-Time-Mass-Duality/tree/main/2/}

\begin{document}
	
	\title{Kompensatorische und Additive Effekte: Eine Analyse der Messdifferenzen zwischen dem T0-Modell und dem $\Lambda$CDM-Standardmodell}
	\author{Johann Pascher}
	\date{2. 04. 2025}
	\maketitle
	
	\begin{abstract}
		Dieses Dokument analysiert die Unterschiede in kosmologischen Messungen zwischen dem Standardmodell ($\Lambda$CDM) und dem alternativen T0-Modell. Wir untersuchen, wie sich die verschiedenen theoretischen Grundlagen auf Distanzmessungen, Rotverschiebungen und die Interpretation des kosmischen Mikrowellenhintergrunds auswirken. Besondere Aufmerksamkeit gilt der Frage, ob sich die Effekte gegenseitig verstärken (additiv wirken) oder kompensieren. Die Analyse zeigt ein komplexes Wechselspiel, das möglicherweise das Hubble-Spannungsproblem erklären könnte. Bei niedrigen Rotverschiebungen (z $\approx$ 1) sind die Unterschiede moderat, während sie bei hohen Rotverschiebungen (z = 1100, CMB) dramatisch werden und zu fundamental unterschiedlichen Interpretationen führen.
	\end{abstract}
	
	\tableofcontents
	\newpage
	
	\section{Einleitung}
	
	Das kosmologische Standardmodell ($\Lambda$CDM) und das alternative T0-Modell bieten fundamental unterschiedliche Erklärungen für dieselben astronomischen Beobachtungen. Während $\Lambda$CDM auf einem expandierenden Universum basiert, postuliert das T0-Modell ein statisches Universum mit absoluter Zeit und variabler Masse. Diese Arbeit untersucht, wie sich die unterschiedlichen theoretischen Grundlagen auf kosmologische Messungen auswirken und wie sich diese Effekte gegenseitig verstärken oder kompensieren.
		\section{Messung der CMB-Temperatur heute}

Die Messung der Temperatur des kosmischen Mikrowellenhintergrunds (CMB) erfolgt heutzutage hauptsächlich durch Satelliten wie Planck (2009–2013) sowie bodengebundene Teleskope wie das Atacama Cosmology Telescope (ACT) und das South Pole Telescope (SPT). Hier ist eine Übersicht:

\subsection{Instrumente und Technologie}

\begin{itemize}
	\item \textbf{Planck-Satellit:}
	\begin{itemize}
		\item Low Frequency Instrument (LFI): Radiometer für 30–70 GHz.
		\item High Frequency Instrument (HFI): Bolometer für 100–857 GHz.
		\item Kryogene Kühlung auf $\sim$0.1 K, um thermisches Rauschen zu minimieren.
	\end{itemize}
	\item \textbf{ACT und SPT:}
	\begin{itemize}
		\item Arrays mit hunderten bis tausenden Bolometern (90–300 GHz).
		\item Standorte in trockenen, hochgelegenen Regionen (Atacama-Wüste, Südpol), um atmosphärische Störungen zu reduzieren.
	\end{itemize}
\end{itemize}

\subsection{Messprinzip}

\begin{itemize}
	\item \textbf{Frequenzmessungen:} Instrumente erfassen die Intensität $I_\nu$ (Leistung pro Fläche pro Frequenzintervall pro Steradiant) über mehrere Frequenzbänder.
	\item \textbf{Schwarzkörperspektrum:} Die gemessene Intensität wird an die Planck-Verteilung angepasst:
	\[
	I_\nu(\nu, T) = \frac{2 h \nu^3}{c^2} \cdot \frac{1}{e^{h \nu / k_B T} - 1}.
	\]
	$T$ wird als freier Parameter optimiert, bis die Daten passen (z.B. $T = 2.72548 \pm 0.00057 \, \text{K}$ nach Planck 2018).
	\item \textbf{Anisotropien:} Temperaturfluktuationen ($\delta T/T \sim 10^{-5}$) werden über den Himmel kartiert.
\end{itemize}

\subsection{Ablauf}

\begin{itemize}
	\item \textbf{Datenaufnahme:} Mehrere Himmelsscans über Monate (Satelliten) oder Jahre (Boden).
	\item \textbf{Kalibrierung:} Gegen astrophysikalische Quellen (z.B. CMB-Dipol, $\sim$3.36 mK) und interne Referenzen.
	\item \textbf{Datenverarbeitung:} Vordergrundquellen (z.B. Staub, Synchrotronstrahlung) werden mit statistischen Methoden entfernt, und die verbleibende Intensität wird an $I_\nu(\nu, T)$ gefittet.
\end{itemize}

\subsection{Einfluss des Standardmodells}

\begin{itemize}
	\item \textbf{Keine Kopplungsfaktoren:} Die Planck-Verteilung selbst enthält keine Kopplungsfaktoren. Sie basiert auf $h$, $c$ und $k_B$, die universelle Konstanten sind.
	\item \textbf{Indirekter Einfluss:}
	\begin{itemize}
		\item \textbf{Expansion:} Die Interpretation von $T = 2.725 \, \text{K}$ als abgekühlte Urknallstrahlung ($T(z) = T_0 (1 + z)$) ist eine Annahme des Standardmodells. Die Rotverschiebung ($z$) wird durch Expansion erklärt.
		\item \textbf{Kalibrierung:} Der CMB-Dipol wird durch die Bewegung relativ zu einem expandierenden Ruhesystem definiert.
		\item \textbf{Vordergrundmodelle:} Die Entfernung von Vordergrundquellen basiert auf Modellen, die mit der Expansionsgeschichte kalibriert sind.
		\item \textbf{Rohdaten:} Die Frequenzmessungen ($I_\nu$ bei verschiedenen $\nu$) sind empirisch und modellunabhängig, aber die Anpassung an ein Schwarzkörperspektrum und die Interpretation von $T$ sind durch das Standardmodell geprägt.
	\end{itemize}
\end{itemize}
	
	\section{Grundlegende Konzepte der Modelle}
	
	\subsection{Das $\Lambda$CDM-Standardmodell}
	
	Im $\Lambda$CDM-Modell wird die beobachtete Rotverschiebung durch die kosmische Expansion erklärt. Die Friedmann-Gleichungen beschreiben die zeitliche Entwicklung des Universums, und die Hubble-Konstante $H_0$ repräsentiert die aktuelle Expansionsrate. Die kosmische Rotverschiebung $z$ steht in Beziehung zum Skalenfaktor $a(t)$ durch:
	
	\begin{equation}
		1 + z = \frac{a(t_0)}{a(t_{\text{emit}})}
	\end{equation}
	
	Für kleine Rotverschiebungen gilt näherungsweise:
	
	\begin{equation}
		z \approx \frac{H_0 d}{c}
	\end{equation}
	
	\subsection{Das T0-Modell}
	Im {\small\href{\repobase/pdf/Deutsch/Wesentliche mathematische Formalismen der Zeit-Masse-Dualitätstheorie mit Lagrange-Dichten.pdf}{T0-Modell}} wird die Zeit als absolut betrachtet, während die Masse variiert. Die Rotverschiebung entsteht durch Energieverlust von Photonen an das dunkle Energiefeld:
	
	\begin{equation}
		1 + z = e^{\alpha d}
	\end{equation}
	
	wobei $\alpha = H_0/c$ die Absorptionsrate ist. Die Hubble-Konstante $H_0$ ist hier kein Expansionsparameter, sondern ein Maß für die Energieübertragungsrate zwischen Photonen und dem dunklen Energiefeld.
	
	\section{Vergleichende Analyse der Messmethoden}
	
	\subsection{Physikalische Distanz ($d$)}
	
	\textbf{$\Lambda$CDM-Modell:}
	\begin{equation}
		d = \frac{c}{H_0} \int_0^z \frac{dz'}{\sqrt{\Omega_m (1 + z')^3 + \Omega_\Lambda}}
	\end{equation}
	
	\textbf{T0-Modell:}
	\begin{equation}
		d = \frac{c \ln(1 + z)}{H_0}
	\end{equation}
	
	\textbf{Quantitativer Vergleich bei $z = 1$:}
	\begin{itemize}
		\item $\Lambda$CDM ($H_0 = 70$ km/s/Mpc): $d \approx 3300$ Mpc
		\item T0 ($H_0 = 70$ km/s/Mpc): $d \approx 2970$ Mpc ($-10\%$)
		\item T0 ($H_0 = 73$ km/s/Mpc): $d \approx 2850$ Mpc ($-14\%$)
	\end{itemize}
	
	Im T0-Modell sind Distanzen bei gleichem $z$ systematisch kleiner, wobei der Unterschied mit zunehmendem $z$ größer wird.
	
	\subsection{Leuchtkraftdistanz ($d_L$)}
	
	\textbf{$\Lambda$CDM-Modell:}
	\begin{equation}
		d_L = (1 + z) \cdot \frac{c}{H_0} \int_0^z \frac{dz'}{\sqrt{\Omega_m (1 + z')^3 + \Omega_\Lambda}}
	\end{equation}
	
	\textbf{T0-Modell:}
	\begin{equation}
		d_L = \frac{c}{H_0} \ln(1 + z) (1 + z)
	\end{equation}
	
	\textbf{Quantitativer Vergleich bei $z = 1$:}
	\begin{itemize}
		\item $\Lambda$CDM ($H_0 = 70$): $d_L \approx 4710$ Mpc
		\item T0 ($H_0 = 70$): $d_L \approx 5940$ Mpc ($+26\%$)
		\item T0 ($H_0 = 73$): $d_L \approx 5700$ Mpc ($+21\%$)
	\end{itemize}
	
	Bemerkenswerterweise sind Leuchtkraftdistanzen im T0-Modell größer, was bedeutet, dass Objekte bei gleicher Rotverschiebung lichtschwächer erscheinen als im $\Lambda$CDM-Modell vorhergesagt.
	
	\subsection{Winkeldurchmesser-Distanz ($d_A$)}
	
	\textbf{$\Lambda$CDM-Modell:}
	\begin{equation}
		d_A = \frac{d}{1 + z}
	\end{equation}
	
	\textbf{T0-Modell:}
	\begin{equation}
		d_A = \frac{c \ln(1 + z)}{H_0 (1 + z)}
	\end{equation}
	
	\textbf{Quantitativer Vergleich bei $z = 1$:}
	\begin{itemize}
		\item $\Lambda$CDM ($H_0 = 70$): $d_A \approx 1650$ Mpc
		\item T0 ($H_0 = 70$): $d_A \approx 1485$ Mpc ($-10\%$)
		\item T0 ($H_0 = 73$): $d_A \approx 1425$ Mpc ($-14\%$)
	\end{itemize}
	
	\textbf{Für den CMB ($z = 1100$):}
	\begin{itemize}
		\item $\Lambda$CDM: $d_A \approx 13.5$ Mpc, $\theta \approx 1^\circ$
		\item T0 ($H_0 = 70$): $d_A \approx 28.9$ Mpc ($+114\%$), $\theta \approx 5.8^\circ$ ($+480\%$)
		\item T0 ($H_0 = 73$): $d_A \approx 27.7$ Mpc ($+105\%$), $\theta \approx 6.1^\circ$ ($+510\%$)
	\end{itemize}
	
	Besonders dramatisch sind die Unterschiede beim CMB, wo die vorhergesagte Winkelgröße von Strukturen im T0-Modell etwa fünfmal größer ist als im $\Lambda$CDM-Modell.
	
	\section{Additive und Kompensatorische Effekte}
	
	\subsection{Additive (verstärkende) Effekte}
	
	Die Effekte auf die physikalische Distanz ($d$) und die Winkeldurchmesser-Distanz ($d_A$) verstärken sich gegenseitig:
	
	\begin{enumerate}
		\item \textbf{Konsistente Richtung}: Beide Distanztypen sind im T0-Modell kleiner als im $\Lambda$CDM-Modell (bei $z = 1$ etwa $10$-$14\%$ Reduktion).
		
		\item \textbf{Zunehmende Verstärkung mit $z$}: Bei hohen Rotverschiebungen verstärken sich diese Effekte dramatisch, wie am Beispiel des CMB deutlich wird.
		
		\item \textbf{Kohärente Auswirkung auf Strukturgröße}: Die reduzierte Distanz und vergrößerte Winkelmaße führen zu einer konsistenten Neuinterpretation der Größe kosmischer Strukturen.
	\end{enumerate}
	
	\subsection{Kompensatorische (gegenläufige) Effekte}
	
	Die Effekte auf die physikalische Distanz und die Leuchtkraftdistanz wirken in entgegengesetzte Richtungen:
	
	\begin{enumerate}
		\item \textbf{Gegenläufige Tendenzen}: Während $d$ im T0-Modell kleiner ist ($-10\%$ bei $z = 1$), ist $d_L$ größer ($+26\%$ bei $z = 1$).
		
		\item \textbf{Auswirkung auf Helligkeitsmessungen}: Objekte erscheinen näher, aber lichtschwächer, was zu einer komplexen Neuinterpretation von Standardkerzen wie Supernovae Typ Ia führt.
		
		\item \textbf{$H_0$-Abhängigkeit}: Ein höherer $H_0$-Wert im T0-Modell verstärkt die Distanzreduktion, mildert aber gleichzeitig den Anstieg der Leuchtkraftdistanz.
	\end{enumerate}
	
	\section{Implikationen für das Hubble-Spannungsproblem}
	
	Die komplementären und kompensatorischen Effekte zwischen dem T0-Modell und dem $\Lambda$CDM-Modell könnten eine Erklärung für das Hubble-Spannungsproblem bieten:
	
	\begin{enumerate}
		\item \textbf{Divergierende Messungen}: Die unterschiedlichen Effekte auf $d_L$ und $d_A$ könnten erklären, warum lokale Messungen (basierend auf Supernovae) systematisch höhere $H_0$-Werte liefern als CMB-basierte Messungen.
		
		\item \textbf{Modell-abhängige Kalibrierung}: Die Standardkerzen und -lineale werden unterschiedlich kalibriert, je nachdem, welches kosmologische Modell zugrunde gelegt wird.
		
		\item \textbf{CMB-Reinterpretation}: Die dramatisch unterschiedliche Interpretation der CMB-Anisotropien ($\theta \approx 1^\circ$ vs. $\theta \approx 5.8$-$6.1^\circ$) führt zu fundamental anderen Parameterschätzungen.
	\end{enumerate}
	
	\section{Quantitative Zusammenfassung der Effekte}
	
	\subsection{Bei $z = 1$ (mittlere kosmologische Distanzen)}
	
	\begin{table}[h]
		\centering
		\begin{tabular}{|l|c|c|c|c|c|}
			\hline
			\textbf{Größe} & \textbf{$\Lambda$CDM ($H_0$=70)} & \textbf{T0 ($H_0$=70)} & \textbf{Differenz} & \textbf{T0 ($H_0$=73)} & \textbf{Differenz} \\
			\hline
			$d$ & 3300 Mpc & 2970 Mpc & $-10\%$ & 2850 Mpc & $-14\%$ \\
			$d_L$ & 4710 Mpc & 5940 Mpc & $+26\%$ & 5700 Mpc & $+21\%$ \\
			$d_A$ & 1650 Mpc & 1485 Mpc & $-10\%$ & 1425 Mpc & $-14\%$ \\
			\hline
		\end{tabular}
		\caption{Vergleich der Distanzmaße bei $z = 1$}
	\end{table}
	
	\subsection{Bei $z = 1100$ (CMB)}
	
	\begin{table}[h]
		\centering
		\begin{tabular}{|l|c|c|c|c|c|}
			\hline
			\textbf{Größe} & \textbf{$\Lambda$CDM ($H_0$=70)} & \textbf{T0 ($H_0$=70)} & \textbf{Differenz} & \textbf{T0 ($H_0$=73)} & \textbf{Differenz} \\
			\hline
			$d_A$ & 13.5 Mpc & 28.9 Mpc & $+114\%$ & 27.7 Mpc & $+105\%$ \\
			$\theta$ & $1^\circ$ & $5.8^\circ$ & $+480\%$ & $6.1^\circ$ & $+510\%$ \\
			\hline
		\end{tabular}
		\caption{Vergleich der Distanzmaße und Winkelgrößen beim CMB ($z = 1100$)}
	\end{table}
	
	\section{Grafische Darstellung der Ergebnisse}
	
	Die quantitativen Unterschiede zwischen dem T0-Modell und dem $\Lambda$CDM-Standardmodell lassen sich anschaulich in grafischer Form darstellen. Im Folgenden zeigen wir die wichtigsten Beziehungen und ihre Unterschiede in beiden Modellen.
	
	\subsection{Physikalische Distanz im Vergleich}
	
	\begin{figure}[H]
		\centering
		\begin{tikzpicture}
			\begin{axis}[
				width=14cm, height=8cm,
				xlabel={Rotverschiebung $z$},
				ylabel={Physikalische Distanz $d$ [Mpc]},
				xmin=0, xmax=2,
				ymin=0, ymax=6000,
				grid=both,
				legend pos=north west,
				legend style={fill=white, fill opacity=0.7}
				]
				
				% ΛCDM-Modell Kurve (etwas vereinfacht)
				\addplot[color=blue, thick, domain=0:2, samples=100] {3300*x*(1-0.2*x+0.07*x^2)};
				\addlegendentry{$\Lambda$CDM}
				
				% T0-Modell mit H0=70
				\addplot[color=red, thick, domain=0:2, samples=100] {2970*ln(1+x)/0.693};
				\addlegendentry{T0 ($H_0=70$)}
				
				% T0-Modell mit H0=73
				\addplot[color=orange, thick, dashed, domain=0:2, samples=100] {2850*ln(1+x)/0.693};
				\addlegendentry{T0 ($H_0=73$)}
				
			\end{axis}
		\end{tikzpicture}
		\caption{Vergleich der physikalischen Distanz $d$ in Abhängigkeit von der Rotverschiebung $z$. Bei $z=1$ ist die Distanz im T0-Modell um etwa 10-14\% kleiner. Bei höheren Rotverschiebungen wird der Unterschied noch ausgeprägter.}
		\label{fig:phys_distance}
	\end{figure}
	
	\subsection{Leuchtkraftdistanz im Vergleich}
	
	\begin{figure}[H]
		\centering
		\begin{tikzpicture}
			\begin{axis}[
				width=14cm, height=8cm,
				xlabel={Rotverschiebung $z$},
				ylabel={Leuchtkraftdistanz $d_L$ [Mpc]},
				xmin=0, xmax=2,
				ymin=0, ymax=14000,
				grid=both,
				legend pos=north west,
				legend style={fill=white, fill opacity=0.7}
				]
				
				% ΛCDM-Modell Kurve
				\addplot[color=blue, thick, domain=0:2, samples=100] {(1+x)*3300*x*(1-0.2*x+0.07*x^2)};
				\addlegendentry{$\Lambda$CDM}
				
				% T0-Modell mit H0=70
				\addplot[color=red, thick, domain=0:2, samples=100] {5940*ln(1+x)*(1+x)/0.693};
				\addlegendentry{T0 ($H_0=70$)}
				
				% T0-Modell mit H0=73
				\addplot[color=orange, thick, dashed, domain=0:2, samples=100] {5700*ln(1+x)*(1+x)/0.693};
				\addlegendentry{T0 ($H_0=73$)}
				
			\end{axis}
		\end{tikzpicture}
		\caption{Vergleich der Leuchtkraftdistanz $d_L$ in Abhängigkeit von der Rotverschiebung $z$. Im Gegensatz zur physikalischen Distanz ist die Leuchtkraftdistanz im T0-Modell um etwa 21-26\% größer bei $z=1$. Objekte erscheinen also bei gleicher Rotverschiebung lichtschwächer als im $\Lambda$CDM-Modell vorhergesagt.}
		\label{fig:luminosity_distance}
	\end{figure}
	
	\subsection{Winkeldurchmesser-Distanz im Vergleich}
	
	\begin{figure}[H]
		\centering
		\begin{tikzpicture}
			\begin{axis}[
				width=14cm, height=8cm,
				xlabel={Rotverschiebung $z$},
				ylabel={Winkeldurchmesser-Distanz $d_A$ [Mpc]},
				xmin=0, xmax=2,
				ymin=0, ymax=2000,
				grid=both,
				legend pos=north east,
				legend style={fill=white, fill opacity=0.7}
				]
				
				% ΛCDM-Modell Kurve
				\addplot[color=blue, thick, domain=0:2, samples=100] {3300*x*(1-0.2*x+0.07*x^2)/(1+x)};
				\addlegendentry{$\Lambda$CDM}
				
				% T0-Modell mit H0=70
				\addplot[color=red, thick, domain=0:2, samples=100] {1485*ln(1+x)/(0.693*(1+x))};
				\addlegendentry{T0 ($H_0=70$)}
				
				% T0-Modell mit H0=73
				\addplot[color=orange, thick, dashed, domain=0:2, samples=100] {1425*ln(1+x)/(0.693*(1+x))};
				\addlegendentry{T0 ($H_0=73$)}
				
			\end{axis}
		\end{tikzpicture}
		\caption{Winkeldurchmesser-Distanz $d_A$ in Abhängigkeit von der Rotverschiebung $z$. Bei $z=1$ ist $d_A$ im T0-Modell um etwa 10-14\% kleiner. Dies bedeutet, dass gleich große Objekte im T0-Modell unter einem größeren Winkel erscheinen würden.}
		\label{fig:angular_distance}
	\end{figure}
	
	\subsection{CMB-Winkeldurchmesser-Distanz}
	
	\begin{figure}[H]
		\centering
		\begin{tikzpicture}
			\begin{axis}[
				width=14cm, height=8cm,
				xlabel={Rotverschiebung $z$},
				ylabel={Winkeldurchmesser-Distanz $d_A$ [Mpc]},
				xmin=0, xmax=1200,
				ymin=0, ymax=30,
				grid=both,
				legend pos=north east,
				legend style={fill=white, fill opacity=0.7},
				xtick={0, 200, 400, 600, 800, 1000, 1200},
				extra y ticks={13.5, 28.9},
				extra y tick labels={13.5, 28.9},
				extra y tick style={grid=major, grid style={dashed, red}}
				]
				
				% ΛCDM-Modell Wert bei z=1100
				\addplot[color=blue, mark=*, mark size=4pt] coordinates {(1100, 13.5)};
				\addlegendentry{$\Lambda$CDM}
				
				% T0-Modell mit H0=70 Wert bei z=1100
				\addplot[color=red, mark=square*, mark size=4pt] coordinates {(1100, 28.9)};
				\addlegendentry{T0 ($H_0=70$)}
				
				% T0-Modell mit H0=73 Wert bei z=1100
				\addplot[color=orange, mark=diamond*, mark size=4pt] coordinates {(1100, 27.7)};
				\addlegendentry{T0 ($H_0=73$)}
				
				% Beispielkurven für beide Modelle (stark vereinfacht)
				\addplot[color=blue, thick, dashed, domain=0:1200, samples=100] {13.5/1100*x/(1+0.0004*x)};
				\addplot[color=red, thick, dashed, domain=0:1200, samples=100] {28.9/1100*x/(1+0.0004*x)};
				
			\end{axis}
		\end{tikzpicture}
		\caption{Winkeldurchmesser-Distanz $d_A$ für die CMB-Strahlung ($z=1100$). Der dramatische Unterschied zwischen den Modellen wird hier deutlich: Das T0-Modell sagt einen mehr als doppelt so großen Wert für $d_A$ voraus (28.9 Mpc vs. 13.5 Mpc), was zu grundlegend unterschiedlichen Interpretationen der CMB-Anisotropien führt.}
		\label{fig:cmb_angular_distance}
	\end{figure}
	
	\subsection{CMB-Temperatur-Rotverschiebungs-Relation}
	
	\begin{figure}[H]
		\centering
		\begin{tikzpicture}
			\begin{axis}[
				width=14cm, height=8cm,
				xlabel={Rotverschiebung $z$},
				ylabel={CMB-Temperatur $T(z)$ [K]},
				xmin=0, xmax=5,
				ymin=0, ymax=20,
				grid=both,
				legend pos=north west,
				legend style={fill=white, fill opacity=0.7}
				]
				
				% ΛCDM-Modell Kurve
				\addplot[color=blue, thick, domain=0:5, samples=100] {2.725*(1+x)};
				\addlegendentry{$\Lambda$CDM}
				
				% T0-Modell Kurve
				\addplot[color=red, thick, domain=0:5, samples=100] {2.725*(1+x)*(1+0.008*ln(1+x))};
				\addlegendentry{T0 mit $\beta = 0.008$}
				
				% T0-Modell Alternative Darstellung
				\addplot[color=orange, thick, dashed, domain=0:5, samples=100] {2.725*(1+x)^(1-0.008)};
				\addlegendentry{T0 mit $\alpha = 0.008$}
				
			\end{axis}
		\end{tikzpicture}
		\caption{CMB-Temperatur $T(z)$ in Abhängigkeit von der Rotverschiebung $z$. Während im $\Lambda$CDM-Modell eine lineare Beziehung gilt, sagt das T0-Modell eine leichte Modifikation voraus, die bei höheren Rotverschiebungen zunehmend erkennbar wird. Diese Abweichung könnte durch Messungen des Sunyaev-Zeldovich-Effekts in Galaxienhaufen bei unterschiedlichen Rotverschiebungen getestet werden.}
		\label{fig:cmb_temperature}
	\end{figure}
	
	\subsection{Vergleich der Relationen zwischen Distanzmaßen}
	
	\begin{figure}[H]
		\centering
		\begin{tikzpicture}
			\begin{axis}[
				width=14cm, height=8cm,
				xlabel={Rotverschiebung $z$},
				ylabel={Verhältnis $d_L / d_A$},
				xmin=0, xmax=3,
				ymin=0, ymax=20,
				grid=both,
				legend pos=north west,
				legend style={fill=white, fill opacity=0.7}
				]
				
				% Gemeinsame Relation in beiden Modellen
				\addplot[color=black, thick, domain=0:3, samples=100] {(1+x)^2};
				\addlegendentry{Gemeinsame Relation: $d_L = d_A (1+z)^2$}
				
				% Spezifische Werte markieren
				\addplot[color=blue, mark=*, mark size=3pt] coordinates {(1, 4)}; 
				\addlegendentry{$z=1$: $d_L/d_A = 4$}
				
				\addplot[color=red, mark=square*, mark size=3pt] coordinates {(2, 9)}; 
				\addlegendentry{$z=2$: $d_L/d_A = 9$}
				
			\end{axis}
		\end{tikzpicture}
		\caption{Verhältnis zwischen Leuchtkraftdistanz $d_L$ und Winkeldurchmesser-Distanz $d_A$ in Abhängigkeit von der Rotverschiebung $z$. Interessanterweise gilt die Beziehung $d_L = d_A (1+z)^2$ in beiden Modellen, was eine wichtige Konsistenzprüfung darstellt. Die unterschiedlichen Absolutwerte der Distanzen führen jedoch zu verschiedenen Interpretationen der astronomischen Beobachtungen.}
		\label{fig:distance_ratios}
	\end{figure}
	
	\subsection{Prozentuale Unterschiede zwischen den Modellen}
	
	\begin{figure}[H]
		\centering
		\begin{tikzpicture}
			\begin{axis}[
				width=14cm, height=8cm,
				xlabel={Rotverschiebung $z$},
				ylabel={Prozentuale Abweichung [\%]},
				xmin=0, xmax=2,
				ymin=-20, ymax=40,
				grid=both,
				legend pos=south east,
				legend style={fill=white, fill opacity=0.7}
				]
				
				% Physikalische Distanz
				\addplot[color=blue, thick, domain=0.05:2, samples=100] {-10-(3*ln(1+x))};
				\addlegendentry{Physikalische Distanz}
				
				% Leuchtkraftdistanz
				\addplot[color=red, thick, domain=0.05:2, samples=100] {26+(5*ln(1+x))};
				\addlegendentry{Leuchtkraftdistanz}
				
				% Winkeldurchmesser-Distanz
				\addplot[color=green!60!black, thick, domain=0.05:2, samples=100] {-10-(3*ln(1+x))};
				\addlegendentry{Winkeldurchmesser-Distanz}
				
				% Nulllinie
				\addplot[color=black, domain=0:2] {0};
				
			\end{axis}
		\end{tikzpicture}
		\caption{Prozentuale Abweichung der Distanzmaße im T0-Modell im Vergleich zum $\Lambda$CDM-Modell in Abhängigkeit von der Rotverschiebung $z$. Positive Werte bedeuten, dass der Wert im T0-Modell größer ist. Man beachte die gegenläufige Tendenz: Während die physikalische Distanz und die Winkeldurchmesser-Distanz im T0-Modell kleiner sind (negative Abweichung), ist die Leuchtkraftdistanz größer (positive Abweichung). Diese gegenläufigen Effekte könnten das Hubble-Spannungsproblem erklären.}
		\label{fig:percentage_differences}
	\end{figure}
	
	\subsection{Winkelgröße typischer Strukturen}
	
	\begin{figure}[H]
		\centering
		\begin{tikzpicture}
			\begin{axis}[
				width=14cm, height=8cm,
				xlabel={Rotverschiebung $z$},
				ylabel={Winkelgröße $\theta$ [Grad] für Objekt mit $r=150$ Mpc},
				xmin=900, xmax=1300,
				ymin=0, ymax=7,
				grid=both,
				legend pos=north west,
				legend style={fill=white, fill opacity=0.7}
				]
				
				% Statt komplizierter Formeln verwenden wir eine vereinfachte Darstellung
				% und definieren nur die Kurven in der Nähe der wichtigen Punkte
				
				% ΛCDM-Modell Kurve - vereinfacht
				\addplot[color=blue, thick, domain=900:1300, samples=100] {1 + 0.0005*(1100-x)};
				\addlegendentry{$\Lambda$CDM}
				
				% T0-Modell mit H0=70 Kurve - vereinfacht
				\addplot[color=red, thick, domain=900:1300, samples=100] {5.8 + 0.003*(1100-x)};
				\addlegendentry{T0 ($H_0=70$)}
				
				% T0-Modell mit H0=73 Kurve - vereinfacht
				\addplot[color=orange, thick, dashed, domain=900:1300, samples=100] {6.1 + 0.003*(1100-x)};
				\addlegendentry{T0 ($H_0=73$)}
				
				% Spezifische Punkte hervorheben
				\addplot[color=blue, mark=*, mark size=4pt] coordinates {(1100, 1)};
				\addplot[color=red, mark=square*, mark size=4pt] coordinates {(1100, 5.8)};
				\addplot[color=orange, mark=diamond*, mark size=4pt] coordinates {(1100, 6.1)};
				
			\end{axis}
		\end{tikzpicture}
		\caption{Winkelgröße $\theta$ einer kosmologischen Struktur mit physikalischer Größe $r=150$ Mpc (typische BAO-Skala) in Abhängigkeit von der Rotverschiebung $z$ im Bereich des CMB. Der dramatische Unterschied in der vorhergesagten Winkelgröße (ca. $1^\circ$ im $\Lambda$CDM-Modell gegenüber ca. $5.8$-$6.1^\circ$ im T0-Modell) ist ein kritischer Test zwischen den Modellen.}
		\label{fig:angular_size}
	\end{figure}
	
	\section{CMB-Temperatur und Modell-Interpretation}
	
	\subsection{$\Lambda$CDM-Modell}
	
	Im Standardmodell wird die CMB-Temperatur als Folge der kosmischen Expansion interpretiert:
	
	\begin{equation}
		T(z) = T_0 (1 + z)
	\end{equation}
	
	Mit $T_0 = 2.725$ K als heutige Temperatur.
	
	\subsection{T0-Modell}
	
	Im T0-Modell weist die CMB-Temperatur eine leichte wellenlängenabhängige Modifikation auf:
	
	\begin{equation}
		T(z) = T_0 (1 + z)(1 + \beta \ln(1 + z))
	\end{equation}
	
	Mit $\beta \approx 0.008$, was zu einer subtilen Abweichung vom Standardmodell führt.
	
	\subsection{Testbare Vorhersagen}
	
	Diese Unterschiede führen zu spezifischen vorhersagbaren Effekten:
	
	\begin{enumerate}
		\item \textbf{Wellenlängenabhängige Rotverschiebung}: Das T0-Modell sagt voraus, dass die Rotverschiebung leicht wellenlängenabhängig ist: $z(\lambda) = z_0(1 + \beta\cdot\ln(\lambda/\lambda_0))$.
		
		\item \textbf{Umgebungsabhängige Rotverschiebung}: Im T0-Modell sollte die Rotverschiebung in dichten kosmischen Regionen leicht anders sein als in kosmischen Voids: $z_\text{cluster}/z_\text{void} \approx 1 + \delta(\rho_\text{cluster}-\rho_\text{void})/\rho_0$.
		
		\item \textbf{Temperatur-Rotverschiebungs-Relation}: Das T0-Modell sagt $T(z) = T_0(1+z)^{(1-\alpha)}$ mit $\alpha \approx \beta \approx 0.008$ voraus, was durch SZ-Effekt-Messungen getestet werden könnte.
	\end{enumerate}
	
	\section{Fazit}
	
	Die Analyse der additiven und kompensatorischen Effekte zwischen dem T0-Modell und dem $\Lambda$CDM-Standardmodell zeigt:
	
	\begin{enumerate}
		\item Die Effekte addieren sich nicht einfach linear über alle Messgrößen hinweg, sondern bilden ein komplexes Wechselspiel aus Verstärkung und Kompensation.
		
		\item Bei niedrigen bis mittleren Rotverschiebungen ($z \approx 1$) sind die Unterschiede moderat ($\sim10$-$26\%$), aber systematisch.
		
		\item Bei hohen Rotverschiebungen ($z = 1100$, CMB) werden die Unterschiede dramatisch ($>100\%$ für $d_A$, $>400\%$ für Winkelgrößen).
		
		\item Diese Differenzen könnten erklären, warum verschiedene Messmethoden zu unterschiedlichen kosmologischen Parametern führen.
		
		\item Die kompensatorischen Effekte zwischen Helligkeits- und Distanzmessungen könnten eine natürliche Erklärung für das Hubble-Spannungsproblem bieten.
	\end{enumerate}
	
	Die systematischen Unterschiede zwischen den Modellen bieten konkrete Testmöglichkeiten für zukünftige Präzisionsmessungen in der Kosmologie und könnten letztendlich zwischen diesen fundamental unterschiedlichen Weltbildern unterscheiden.
\section{Literaturverzeichnis}

\begin{thebibliography}{9}
	
	\bibitem{Planck2018}
	Planck Collaboration, Aghanim, N., et al. (2020). 
	\textit{Planck 2018 results. VI. Cosmological parameters}. 
	Astronomy \& Astrophysics, 641, A6. 
	DOI: 10.1051/0004-6361/201833910.
	
	\bibitem{LambdaCDM}
	Peebles, P. J. E. (1993). 
	\textit{Principles of Physical Cosmology}. 
	Princeton University Press, Princeton, NJ.
	
	\bibitem{HubbleTension}
	Riess, A. G., et al. (2021). 
	\textit{A Comprehensive Measurement of the Local Value of the Hubble Constant with 1\% Precision from the SH0ES Team}. 
	The Astrophysical Journal, 934(1), L7. 
	DOI: 10.3847/2041-8213/ac5c5b.
	
	\bibitem{ACT}
	Aiola, S., et al. (2020). 
	\textit{The Atacama Cosmology Telescope: DR4 Maps and Cosmological Parameters}. 
	Journal of Cosmology and Astroparticle Physics, 2020(12), 047. 
	DOI: 10.1088/1475-7516/2020/12/047.
	
	\bibitem{SPT}
	Benson, B. A., et al. (2014). 
	\textit{SPT-3G: A Next-Generation Cosmic Microwave Background Polarization Experiment on the South Pole Telescope}. 
	Proceedings of SPIE, 9153, 91531P. 
	DOI: 10.1117/12.2056701.
	
	\bibitem{Friedmann}
	Friedmann, A. (1922). 
	\textit{Über die Krümmung des Raumes}. 
	Zeitschrift für Physik, 10(1), 377–386. 
	DOI: 10.1007/BF01332580.
	
	\bibitem{SunyaevZeldovich}
	Sunyaev, R. A., \& Zeldovich, Y. B. (1972). 
	\textit{The Observations of Relic Radiation as a Test of the Nature of X-Ray Radiation from the Clusters of Galaxies}. 
	Comments on Astrophysics and Space Physics, 4, 173.
	
	\bibitem{CMBTheory}
	Dodelson, S. (2003). 
	\textit{Modern Cosmology}. 
	Academic Press, San Diego, CA.
	
\end{thebibliography}

\end{document}