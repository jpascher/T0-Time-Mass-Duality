\documentclass[a4paper,12pt]{article}
\usepackage[utf8]{inputenc}
\usepackage[T1]{fontenc}
\usepackage{lmodern}
\usepackage[ngerman]{babel}
\usepackage{amsmath, amssymb, amsthm, physics}
\usepackage{graphicx}
\usepackage{xcolor}
\usepackage{tikz}
\usepackage{setspace}
\usepackage{tcolorbox}
\usepackage{booktabs}
\usepackage[margin=2cm]{geometry}


\usepackage{hyperref}
\hypersetup{
	colorlinks=true,
	linkcolor=blue,
	filecolor=blue,
	citecolor=blue, 
	urlcolor=blue,
	bookmarks=true,
	bookmarksopen=true,
	pdftitle={Vereinheitlichung des T0-Modells: Grundlagen, Dunkle Energie und Galaxien-Dynamik},
	pdfauthor={Johann Pascher},
}

\usepackage{cleveref}

\newtheorem{theorem}{Satz}
\newtheorem{lemma}[theorem]{Lemma}
\newtheorem{proposition}[theorem]{Proposition}
\newtheorem{corollary}[theorem]{Korollar}

\theoremstyle{definition}
\newtheorem{definition}{Definition}

\theoremstyle{remark}
\newtheorem{remark}{Bemerkung}

\newcommand{\Tfield}{T(x)}

\begin{document}
	
	\title{Vereinheitlichung des T0-Modells: \\ Grundlagen, Dunkle Energie und Galaxien-Dynamik}
	\author{Johann Pascher}
	\date{27. März 2025}
	\maketitle
	
	\begin{abstract}
		Diese Arbeit präsentiert einen einheitlichen Rahmen für das T0-Modell, der seine grundlegenden Prinzipien mit Anwendungen auf Dunkle Energie und Galaxien-Dynamik in einem statischen Universum integriert. Basierend auf absoluter Zeit und variabler Masse steht das T0-Modell im Gegensatz zur Relativitätstheorie mit relativer Zeit und konstanter Masse und bietet alternative Erklärungen für kosmische Rotverschiebung (durch Energieverlust), Dunkle Energie (emergent aus dem intrinsischen Zeitfeld \(\Tfield\)) und Galaxien-Dynamik (durch Massenvariation ohne Dunkle Materie). Dieses Papier gewährleistet mathematische Konsistenz über diese Bereiche hinweg und bietet eine umfassende Theorie mit experimentell überprüfbaren Vorhersagen.
	\end{abstract}
	
	\tableofcontents
	\newpage
	
	\section{Einführung in das T0-Modell: Grundlegende Konzepte}
	
	\subsection{Grundannahmen des T0-Modells}
	
	Das T0-Modell beruht auf folgenden Kernannahmen:
	
	\begin{tcolorbox}[colback=blue!5!white,colframe=blue!75!black,title=Grundannahmen des T0-Modells]
		\begin{align}
			&\text{1. Zeit ist absolut und universell konstant.} \\
			&\text{2. Masse variiert als $m = \frac{\hbar}{\Tfield c^2}$, wobei $\Tfield$ das intrinsische Zeitfeld ist.} \\
			&\text{3. Gravitation entsteht aus Gradienten von $\Tfield$.} \\
			&\text{4. Rotverschiebung resultiert aus Energieverlust: $1 + z = e^{\alpha d}$.}
		\end{align}
	\end{tcolorbox}
	
	Diese Annahmen führen zu einem komplementären physikalischen Rahmen, der mathematisch äquivalent, aber konzeptionell unterschiedlich ist.
	
	\subsection{Intrinsische Zeit und Zeit-Masse-Dualität}
	
	Ein zentrales Konzept ist die intrinsische Zeit:
	
	\begin{itemize}
		\item Die intrinsische Zeit ist definiert als $\Tfield = \frac{\hbar}{mc^2}$, umgekehrt proportional zur Masse eines Teilchens.
		\item Dies führt zu einer Dualität:
		\begin{itemize}
			\item \textbf{Standardmodell}: Relative Zeit (Zeitdilatation), konstante Masse.
			\item \textbf{T0-Modell}: Absolute Zeit, variable Masse.
		\end{itemize}
	\end{itemize}
	
	Diese Dualität bietet eine Neuinterpretation von Phänomenen, die üblicherweise der Zeitdilatation zugeschrieben werden.
	
	\subsection{Vereinheitlichte Lagrangedichte}
	
	Die Lagrangedichte des T0-Modells lautet:
	
	\begin{equation}
		\mathcal{L}_\text{total} = \mathcal{L}_\text{SM} + \mathcal{L}_\text{Higgs} + \mathcal{L}_\text{intrinsic}
	\end{equation}
	
	wobei:
	\begin{itemize}
		\item \(\mathcal{L}_\text{SM}\): Standardmodell (starke, elektromagnetische, schwache Kräfte).
		\item \(\mathcal{L}_\text{Higgs}\): Dynamik des Higgs-Felds.
		\item \(\mathcal{L}_\text{intrinsic}\): Beitrag des intrinsischen Zeitfelds.
	\end{itemize}
	
	Gravitation wird nicht separat eingeführt, sondern emergiert aus der Dynamik von \(\Tfield\).
	
	\subsection{Die Rolle der Gravitation im T0-Modell}
	
	Gravitation entsteht aus dem intrinsischen Zeitfeld:
	
	\begin{theorem}[Emergenz der Gravitation]
		Gravitationseffekte ergeben sich aus räumlichen und zeitlichen Gradienten von $\Tfield$:
		\begin{equation}
			\nabla \Tfield = \nabla \left(\frac{\hbar}{mc^2}\right) = -\frac{\hbar}{m^2c^2}\nabla m \sim \nabla \Phi_g
		\end{equation}
		wobei \(\Phi_g\) das Gravitationspotential ist.
	\end{theorem}
	
	Dieser Ansatz eliminiert die Notwendigkeit eines separaten Gravitationsterms und vereinfacht die Theorie \cite{pascher_galaxies_2025}.
	
	\section{Dunkle Energie im T0-Modell}
	
	\subsection{Neuinterpretation der Dunklen Energie}
	
	Im T0-Modell ist Dunkle Energie kein kosmologischer Antrieb, sondern ein emergenter Effekt von \(\Tfield\):
	
	\begin{itemize}
		\item \textbf{Standardmodell ($\Lambda$CDM)}: Dunkle Energie als Konstante mit negativem Druck.
		\item \textbf{T0-Modell}: Dunkle Energie als dynamischer Effekt des Energieaustauschs über $\Tfield$.
	\end{itemize}
	
	Die Energiedichte lautet:
	\begin{equation}
		\rho_{DE}(r) = \frac{\kappa}{r^2}
	\end{equation}
	
	\subsection{Feldtheoretische Beschreibung}
	
	Dunkle Energie entsteht aus der Dynamik von \(\Tfield\) in:
	\begin{equation}
		\mathcal{L}_\text{intrinsic} = \frac{1}{2} \partial_\mu \Tfield \partial^\mu \Tfield - V(\Tfield)
	\end{equation}
	
	Die Feldgleichung ist:
	\begin{equation}
		\Box \Tfield - \frac{dV}{d\Tfield} = 0
	\end{equation}
	
	Für große \(r\): \(\rho_{DE}(r) \approx \frac{1}{2} (\nabla \Tfield)^2 \approx \frac{\kappa}{r^2}\).
	
	\subsection{Energietransfer und Rotverschiebung}
	
	Die kosmische Rotverschiebung resultiert aus dem Energieverlust von Photonen:
	\begin{equation}
		\frac{dE_{\gamma}}{dx} = -\alpha E_{\gamma}
	\end{equation}
	
	Lösung:
	\begin{equation}
		E_{\gamma}(x) = E_{\gamma,0} e^{-\alpha x}
	\end{equation}
	
	Rotverschiebung:
	\begin{equation}
		1 + z = e^{\alpha d}, \quad \alpha = \frac{H_0}{c} \approx 2.3 \times 10^{-28} \, \text{m}^{-1}
	\end{equation}
	
	\section{Galaxien-Dynamik im T0-Modell}
	
	\subsection{Flache Rotationskurven ohne Dunkle Materie}
	
	Flache Rotationskurven entstehen durch \(\Tfield\)-induzierte Massenvariation:
	\begin{equation}
		v^2(r) = \frac{G M(r)}{r} + \kappa r
	\end{equation}
	
	wobei \(\kappa \approx 4.8 \times 10^{-11} \, \text{m/s}^2\).
	
	\subsection{Effektive Gravitationskonstante}
	
	Alternativ eine effektive Gravitationskonstante:
	\begin{equation}
		G_{\text{eff}}(r) = G \left(1 + \beta \frac{\kappa}{r}\right)
	\end{equation}
	
	ergibt:
	\begin{equation}
		v^2(r) \approx \frac{G M}{r} + \kappa r
	\end{equation}
	
	Für große \(r\): \(v^2(r) \approx \text{constant}\).
	
	\subsection{Parameterwerte aus Beobachtungen}
	
	Für die Milchstraße (\(v \approx 220 \, \text{km/s}\)):
	\begin{equation}
		\kappa \approx 4.8 \times 10^{-11} \, \text{m/s}^2
	\end{equation}
	\begin{equation}
		\beta \approx 0.008
	\end{equation}
	
	\section{Vereinheitlichte mathematische Formulierung}
	
	\subsection{Gemeinsame Feldgleichungen}
	
	Die vereinheitlichte Wirkung ist:
	\begin{equation}
		S_\text{unified} = \int \left( \mathcal{L}_\text{SM} + \mathcal{L}_\text{Higgs} + \mathcal{L}_\text{intrinsic} \right) d^4x
	\end{equation}
	
	In einem statischen Universum ersetzt Massenvariation die Expansion:
	\begin{align}
		\left(\frac{\dot{m}}{m}\right)^2 &= \frac{8\pi G}{3} \rho_{\text{eff}} \\
		\frac{\ddot{m}}{m} &= -\frac{4\pi G}{3} (\rho_{\text{eff}} + 3p_{\text{eff}})
	\end{align}
	
	\subsection{Konsistente Parametrierung}
	
	Parameter sind:
	\begin{itemize}
		\item \(\alpha = \frac{H_0}{c} \approx 2.3 \times 10^{-28} \, \text{m}^{-1}\)
		\item \(\kappa \approx 4.8 \times 10^{-11} \, \text{m/s}^2\)
		\item \(\beta \approx 0.008\)
	\end{itemize}
	
	Beziehung:
	\begin{equation}
		\kappa = \beta \frac{y v c^2}{r_g^2}
	\end{equation}
	
	Intrinsische Zeit für Photonen:
	\begin{equation}
		\Tfield = \frac{\hbar}{E_{\gamma}} e^{\alpha x}
	\end{equation}
	
	\subsection{Unsicherheit bei \(\beta\) und Modellgrenzen}
	Der Parameter \(\beta \approx 0.008\) stimmt mit Beobachtungen überein \cite{pascher_params_2025}. Vorschläge für \(\beta = 1\) \cite{pascher_temp_2025} führen zu unphysikalischen Ergebnissen (z. B. übermäßige Rotverschiebung), was eine Verfeinerung der \(\Tfield\)-Dynamik nahelegt. Der wahre Wert könnte zwischen 0.008 und 1 liegen und erfordert weitere Tests.
	
	\section{Experimentelle Tests des T0-Modells}
	
	\subsection{Gemeinsame Vorhersagen}
	\begin{enumerate}
		\item Massenabhängige Zeitevolution in Quantensystemen.
		\item Umgebungsabhängige Rotverschiebung: \(\frac{z_{\text{Cluster}}}{z_{\text{Leerraum}}} \approx 1 + 0.003\).
		\item Differentielle Rotverschiebung: \(\frac{z(\lambda_1)}{z(\lambda_2)} \approx 1 + \beta \frac{\lambda_1 - \lambda_2}{\lambda_0}\).
	\end{enumerate}
	
	\subsection{Tests für Galaxien-Dynamik}
	\begin{enumerate}
		\item Modifizierte Tully-Fisher-Relation: \(L \propto v_{\text{max}}^{4 + \epsilon}\), \(\epsilon \approx \beta\).
		\item Massenabhängige Gravitationslinseneffekte: \(\alpha_{\text{lens}} \propto \int \nabla \Phi \, dz\).
	\end{enumerate}
	
	\section{Vergleich mit dem \(\Lambda\)CDM-Standardmodell}
	
	\begin{tcolorbox}[colback=yellow!5!white,colframe=yellow!75!black,title=Vergleich der Modelle]
		\begin{tabular}{p{0.45\textwidth}|p{0.45\textwidth}}
			\toprule
			\textbf{\(\Lambda\)CDM-Modell} & \textbf{T0-Modell} \\
			\midrule
			Dunkle Materie als Teilchen & Keine Dunkle Materie, Massenvariation \\
			NFW-Profil: \(\rho_{\text{DM}}(r)\) & \(\rho_{\text{eff}}(r) \approx \frac{\kappa}{r^2}\) \\
			Relative Zeit, konstante Masse & Absolute Zeit, variable Masse \\
			Dunkle Energie treibt Expansion & Dunkle Energie aus \(\Tfield\)-Austausch \\
			Rotverschiebung durch Expansion & Rotverschiebung durch Energieverlust \\
			Expandierendes Universum & Statisches Universum \\
			\bottomrule
		\end{tabular}
	\end{tcolorbox}
	
	\section{Zusammenfassung}
	
	Das T0-Modell vereint absolute Zeit und variable Masse, um Rotverschiebung, Dunkle Energie und Galaxien-Dynamik ohne Expansion oder Dunkle Materie zu erklären, basierend auf einem konsistenten \(\Tfield\)-Rahmen.
	
	\begin{thebibliography}{9}
		\bibitem{pascher_galaxies_2025} Pascher, J. (2025). \href{https://github.com/jpascher/T0-Time-Mass-Duality/tree/main/2/pdf/Deutsch/Massenvariation in Galaxien - Eine Analyse im T0-Modell mit emergenter Gravitation.pdf}{Massenvariation in Galaxien: Eine Analyse im T0-Modell mit emergenter Gravitation}. 30. März 2025.
		\bibitem{pascher_messdifferenzen_2025} Pascher, J. (2025). \href{https://github.com/jpascher/T0-Time-Mass-Duality/tree/main/2/pdf/Deutsch/Analyse der Messdifferenzen zwischen dem T0-Modell und dem Standardmodell.pdf}{Kompensatorische und additive Effekte: Eine Analyse der Messdifferenzen zwischen dem T0-Modell und dem \(\Lambda\)CDM-Standardmodell}. 2. April 2025.
		\bibitem{pascher_params_2025} Pascher, J. (2025). Zeit-Masse-Dualitätstheorie (T0-Modell): Ableitung der Parameter \(\kappa\), \(\alpha\) und \(\beta\). 30. März 2025.
		\bibitem{pascher_temp_2025} Pascher, J. (2025). Anpassung der Temperatureinheiten in natürlichen Einheiten und CMB-Messungen. 2. April 2025.
	\end{thebibliography}
	
\end{document}