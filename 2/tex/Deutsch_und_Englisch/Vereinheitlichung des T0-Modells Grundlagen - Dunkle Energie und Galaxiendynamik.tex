\documentclass[a4paper,12pt]{article}
\usepackage[utf8]{inputenc}
\usepackage[T1]{fontenc}
\usepackage{lmodern}
\usepackage[ngerman]{babel}
\usepackage{amsmath, amssymb, amsthm, physics}
\usepackage{graphicx}
\usepackage{xcolor}
\usepackage{tikz}
\usepackage{setspace}
\usepackage{tcolorbox}
\usepackage{booktabs}
\usepackage{siunitx}
\DeclareSIUnit{\year}{yr}
\DeclareSIUnit{\parsec}{pc}
\usepackage[margin=2cm]{geometry}
\usepackage{tocloft}
\usepackage{fancyhdr}

% Kopf- und Fußzeilen
\pagestyle{fancy}
\fancyhf{}
\fancyhead[L]{Johann Pascher}
\fancyhead[R]{Zeit-Masse-Dualität}
\fancyfoot[C]{\thepage}
\renewcommand{\headrulewidth}{0.4pt}
\renewcommand{\footrulewidth}{0.4pt}

% Inhaltsverzeichnis-Styling
\renewcommand{\cftsecfont}{\color{blue}}
\renewcommand{\cftsubsecfont}{\color{blue}}
\renewcommand{\cftsecpagefont}{\color{blue}}
\renewcommand{\cftsubsecpagefont}{\color{blue}}
\setlength{\cftsecindent}{1cm}
\setlength{\cftsubsecindent}{2cm}

\usepackage{hyperref}
\hypersetup{
	colorlinks=true,
	linkcolor=blue,
	filecolor=blue,
	citecolor=blue, 
	urlcolor=blue,
	bookmarks=true,
	bookmarksopen=true,
	pdftitle={Vereinheitlichung des T0-Modells: Grundlagen, Dunkle Energie und Galaxien-Dynamik},
	pdfauthor={Johann Pascher},
}

\usepackage{cleveref}

\newtheorem{theorem}{Satz}[section]
\newtheorem{lemma}[theorem]{Lemma}
\newtheorem{proposition}[theorem]{Proposition}
\newtheorem{corollary}[theorem]{Korollar}

\theoremstyle{definition}
\newtheorem{definition}{Definition}[theorem]

\theoremstyle{remark}
\newtheorem{remark}{Bemerkung}

% Custom commands
\newcommand{\Tfield}{T(x)}
\newcommand{\DcovT}[1]{\Tfield D_\mu #1 + #1 \partial_\mu \Tfield}
\newcommand{\DhiggsT}{\Tfield (\partial_\mu + ig A_\mu) \Phi + \Phi \partial_\mu \Tfield}
\newcommand{\betaT}{\beta_{\text{T}}}
\newcommand{\alphaEM}{\alpha_{\text{EM}}}
\newcommand{\alphaW}{\alpha_{\text{W}}}
\newcommand{\Mpl}{M_{\text{Pl}}}
\newcommand{\Tzerot}{T_0(\Tfield)}
\newcommand{\Tzero}{T_0}
\newcommand{\vecx}{\vec{x}}
\newcommand{\gammaf}{\gamma_{\text{Lorentz}}}

\begin{document}
	
	\title{Vereinheitlichung des T0-Modells: \\ Grundlagen, Dunkle Energie und Galaxien-Dynamik}
	\author{Johann Pascher}
	\date{27. März 2025}
	\maketitle
	
	\begin{abstract}
		Diese Arbeit präsentiert einen einheitlichen Rahmen für das T0-Modell, der seine grundlegenden Prinzipien mit Anwendungen auf Dunkle Energie und Galaxien-Dynamik in einem statischen Universum integriert. Basierend auf absoluter Zeit und variabler Masse steht das T0-Modell im Gegensatz zur Relativitätstheorie mit relativer Zeit und konstanter Masse und bietet alternative Erklärungen für kosmische Rotverschiebung (durch Energieverlust), Dunkle Energie (emergent aus dem intrinsischen Zeitfeld \(\Tfield\)) und Galaxien-Dynamik (durch Massenvariation ohne Dunkle Materie). Dieses Papier gewährleistet mathematische Konsistenz über diese Bereiche hinweg und bietet eine umfassende Theorie mit experimentell überprüfbaren Vorhersagen.
	\end{abstract}
	
	\tableofcontents
	\newpage
	
	\section{Einführung in das T0-Modell: Grundlegende Konzepte}
	
	\subsection{Grundannahmen des T0-Modells}
	
	Das T0-Modell basiert auf Annahmen, die in \cite{pascher_params_2025} und \cite{pascher_galaxies_2025} ausführlich hergeleitet sind:
	
	\begin{tcolorbox}[colback=blue!5!white,colframe=blue!75!black,title=Grundannahmen des T0-Modells]
		\begin{itemize}
			\item Zeit ist absolut und universell konstant (\cite{pascher_params_2025}, Abschnitt „Zeit-Masse-Dualität“).
			\item Masse variiert als \(m = \frac{\hbar}{\Tfield c^2}\), wobei \(\Tfield\) das intrinsische Zeitfeld ist (\cite{pascher_params_2025}, Abschnitt „Intrinsische Zeit“).
			\item Gravitation entsteht aus Gradienten von \(\Tfield\) (\cite{pascher_galaxies_2025}, Abschnitt „Emergente Gravitation“).
			\item Rotverschiebung resultiert aus Energieverlust: \(1 + z = e^{\alpha d}\) (\cite{pascher_messdifferenzen_2025}, Abschnitt „Energieverlust“).
		\end{itemize}
	\end{tcolorbox}
	
	\subsection{Intrinsische Zeit und Zeit-Masse-Dualität}
	
	Die intrinsische Zeit \(\Tfield\) ist definiert als:
	\begin{equation}
		\Tfield = \frac{\hbar}{m c^2}
	\end{equation}
	Details in \cite{pascher_params_2025} (Abschnitt „Definition der intrinsischen Zeit“). Dies führt zur Dualität:
	\begin{itemize}
		\item \textbf{Standardmodell}: Relative Zeit, konstante Masse.
		\item \textbf{T0-Modell}: Absolute Zeit, variable Masse (\cite{pascher_params_2025}).
	\end{itemize}
	
	\subsection{Vereinheitlichte Lagrangedichte}
	
	Die Lagrangedichte ist in \cite{pascher_lagrange_2025} (Abschnitt „Gesamt-Lagrangedichte“) hergeleitet:
	\begin{equation}
		\mathcal{L}_\text{total} = \mathcal{L}_\text{SM} + \mathcal{L}_\text{Higgs} + \mathcal{L}_\text{intrinsic}
	\end{equation}
	Mit \(\mathcal{L}_\text{intrinsic} = \frac{1}{2} \partial_\mu \Tfield \partial^\mu \Tfield - V(\Tfield)\).
	
	\subsection{Die Rolle der Gravitation im T0-Modell}
	
	Gravitation emergiert aus \(\Tfield\):
	\begin{theorem}[Emergenz der Gravitation]
		\begin{equation}
			\nabla \Tfield = -\frac{\hbar}{m^2 c^2} \nabla m \sim \nabla \Phi_g
		\end{equation}
		Siehe \cite{pascher_galaxies_2025} (Abschnitt „Emergente Gravitation“).
	\end{theorem}
	
	\section{Dunkle Energie im T0-Modell}
	
	\subsection{Neuinterpretation der Dunklen Energie}
	
	Dunkle Energie ist ein emergenter Effekt von \(\Tfield\):
	\begin{itemize}
		\item \textbf{\(\Lambda\)CDM}: Kosmologische Konstante.
		\item \textbf{T0-Modell}: Energieaustausch über \(\Tfield\) (\cite{pascher_energy_2025}, Abschnitt „Dunkle Energie“).
	\end{itemize}
	Energiedichte:
	\begin{equation}
		\rho_{DE}(r) = \frac{\kappa}{r^2}
	\end{equation}
	
	\subsection{Feldtheoretische Beschreibung}
	
	\begin{equation}
		\mathcal{L}_\text{intrinsic} = \frac{1}{2} \partial_\mu \Tfield \partial^\mu \Tfield - V(\Tfield)
	\end{equation}
	Feldgleichung:
	\begin{equation}
		\Box \Tfield - \frac{dV}{d\Tfield} = 0
	\end{equation}
	Siehe \cite{pascher_lagrange_2025}.
	
	\subsection{Energietransfer und Rotverschiebung}
	
	Rotverschiebung durch Energieverlust:
	\begin{equation}
		\frac{d E_{\gamma}}{d x} = -\alpha E_{\gamma}, \quad 1 + z = e^{\alpha d}
	\end{equation}
	Mit \(\alpha \approx \SI{2.3e-18}{\per\meter}\) (\cite{pascher_messdifferenzen_2025}).
	
	\section{Galaxien-Dynamik im T0-Modell}
	
	\subsection{Flache Rotationskurven ohne Dunkle Materie}
	
	Rotationskurven:
	\begin{equation}
		v^2(r) = \frac{G M(r)}{r} + \kappa r
	\end{equation}
	\(\kappa \approx \SI{4.8e-11}{\meter\per\second\squared}\) (\cite{pascher_galaxies_2025}).
	
	\subsection{Effektive Gravitationskonstante}
	
	\begin{equation}
		G_{\text{eff}}(r) = G \left(1 + \betaT \frac{\kappa}{r}\right)
	\end{equation}
	Mit \(\betaT^{\text{SI}} \approx 0.008\) (\cite{pascher_params_2025}).
	
	\section{Vereinheitlichte mathematische Formulierung}
	
	\subsection{Gemeinsame Feldgleichungen}
	
	Wirkung:
	\begin{equation}
		S_\text{unified} = \int \mathcal{L}_\text{total} \, d^4x
	\end{equation}
	Statisches Universum:
	\begin{align}
		\left(\frac{\dot{m}}{m}\right)^2 &= \frac{8\pi G}{3} \rho_{\text{eff}} \\
		\frac{\ddot{m}}{m} &= -\frac{4\pi G}{3} (\rho_{\text{eff}} + 3p_{\text{eff}})
	\end{align}
	
	\subsection{Konsistente Parametrierung}
	
	Parameter:
	\begin{itemize}
		\item \(\alpha \approx \SI{2.3e-18}{\per\meter}\)
		\item \(\kappa \approx \SI{4.8e-11}{\meter\per\second\squared}\)
		\item \(\betaT^{\text{SI}} \approx 0.008\), \(\betaT^{\text{nat}} = 1\) (\cite{pascher_params_2025}).
	\end{itemize}
	Beziehung:
	\begin{equation}
		\kappa = \betaT \frac{y v c^2}{r_g^2}
	\end{equation}
	
	\section{Experimentelle Tests des T0-Modells}
	
	\subsection{Gemeinsame Vorhersagen}
	
	\begin{enumerate}
		\item Massenabhängige Zeitevolution (\cite{pascher_photons_2025}).
		\item Umgebungsabhängige Rotverschiebung: \(\frac{z_{\text{Cluster}}}{z_{\text{Leerraum}}} \approx 1 + 0.003\).
		\item Differentielle Rotverschiebung: \(\frac{z(\lambda_1)}{z(\lambda_2)} \approx 1 + \betaT \frac{\lambda_1 - \lambda_2}{\lambda_0}\).
	\end{enumerate}
	
	\subsection{Tests für Galaxien-Dynamik}
	
	\begin{enumerate}
		\item Tully-Fisher-Relation: \(L \propto v_{\text{max}}^{4 + \epsilon}\), \(\epsilon \approx \betaT\).
		\item Gravitationslinseneffekte: \(\alpha_{\text{lens}} \propto \int \nabla \Phi \, dz\) (\cite{pascher_galaxies_2025}).
	\end{enumerate}
	
	\section{Vergleich mit dem \(\Lambda\)CDM-Standardmodell}
	
	\begin{tcolorbox}[colback=yellow!5!white,colframe=yellow!75!black,title=Vergleich der Modelle]
		\begin{tabular}{p{0.45\textwidth}|p{0.45\textwidth}}
			\toprule
			\textbf{\(\Lambda\)CDM-Modell} & \textbf{T0-Modell} \\
			\midrule
			Dunkle Materie als Teilchen & Keine Dunkle Materie, Massenvariation \\
			NFW-Profil: \(\rho_{\text{DM}}(r)\) & \(\rho_{\text{eff}}(r) \approx \frac{\kappa}{r^2}\) \\
			Relative Zeit, konstante Masse & Absolute Zeit, variable Masse \\
			Dunkle Energie treibt Expansion & Dunkle Energie aus \(\Tfield\)-Austausch \\
			Rotverschiebung durch Expansion & Rotverschiebung durch Energieverlust \\
			Expandierendes Universum & Statisches Universum \\
			\bottomrule
		\end{tabular}
	\end{tcolorbox}
	
	\section{Zusammenfassung}
	
	Das T0-Modell vereint absolute Zeit und variable Masse, um kosmische Phänomene zu erklären, gestützt durch interne Konsistenz und Verweise auf \cite{pascher_galaxies_2025, pascher_params_2025, pascher_messdifferenzen_2025}.
	
	\begin{thebibliography}{99}
		\bibitem{pascher_galaxies_2025} Pascher, J. (2025). \href{https://github.com/jpascher/T0-Time-Mass-Duality/tree/main/2/pdf/Deutsch/Massenvariation in Galaxien.pdf}{Massenvariation in Galaxien: Eine Analyse im T0-Modell mit emergenter Gravitation}. 30. März 2025.
		\bibitem{pascher_messdifferenzen_2025} Pascher, J. (2025). \href{https://github.com/jpascher/T0-Time-Mass-Duality/tree/main/2/pdf/Deutsch/Analyse der Messdifferenzen zwischen dem T0-Modell und dem Standardmodell.pdf}{Kompensatorische und additive Effekte: Eine Analyse der Messdifferenzen zwischen dem T0-Modell und dem \(\Lambda\)CDM-Standardmodell}. 2. April 2025.
		\bibitem{pascher_params_2025} Pascher, J. (2025). \href{https://github.com/jpascher/T0-Time-Mass-Duality/tree/main/2/pdf/Deutsch/Zeit-Masse-Dualitätstheorie (T0-Modell) Herleitung der Parameter kappa, alpha und beta.pdf}{Zeit-Masse-Dualitätstheorie (T0-Modell): Ableitung der Parameter \(\kappa\), \(\alpha\) und \(\beta\)}. 4. April 2025.
		\bibitem{pascher_temp_2025} Pascher, J. (2025). \href{https://github.com/jpascher/T0-Time-Mass-Duality/tree/main/2/pdf/Deutsch/Anpassung von Temperatureinheiten in natürlichen Einheiten und CMB-Messungen.pdf}{Anpassung der Temperatureinheiten in natürlichen Einheiten und CMB-Messungen}. 2. April 2025.
		\bibitem{pascher_lagrange_2025} Pascher, J. (2025). \href{https://github.com/jpascher/T0-Time-Mass-Duality/tree/main/2/pdf/Deutsch/Mathematische Formulierungen der Zeit-Masse-Dualitätstheorie mit Lagrange.pdf}{Von Zeitdilatation zu Massenvariation: Mathematische Kernformulierungen der Zeit-Masse-Dualitätstheorie}. 29. März 2025.
		\bibitem{pascher_higgs_2025} Pascher, J. (2025). \href{https://github.com/jpascher/T0-Time-Mass-Duality/tree/main/2/pdf/Deutsch/Mathematische Formulierung des Higgs-Mechanismus in der Zeit-Masse-Dualität.pdf}{Mathematische Formulierung des Higgs-Mechanismus in der Zeit-Masse-Dualität}. 28. März 2025.
		\bibitem{pascher_photons_2025} Pascher, J. (2025). \href{https://github.com/jpascher/T0-Time-Mass-Duality/tree/main/2/pdf/Deutsch/Dynamische Masse von Photonen und ihre Implikationen für Nichtlokalität.tex}{Dynamische Masse von Photonen und ihre Auswirkungen auf Nichtlokalität im T0-Modell}. 25. März 2025.
		\bibitem{pascher_energy_2025} Pascher, J. (2025). \href{https://github.com/jpascher/T0-Time-Mass-Duality/tree/main/2/pdf/Deutsch/Eine mathematische Analyse der Energiedynamik.pdf}{Dunkle Energie im T0-Modell: Eine mathematische Analyse der Energiedynamik}. 3. April 2025.
	\end{thebibliography}
	
\end{document}