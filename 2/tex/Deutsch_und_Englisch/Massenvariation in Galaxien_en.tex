\documentclass[a4paper,12pt]{article}
\usepackage[utf8]{inputenc}
\usepackage[T1]{fontenc}
\usepackage[english]{babel}
\usepackage{amsmath, amssymb, amsthm}
\usepackage{physics}
\usepackage{graphicx}
\usepackage{hyperref}
\usepackage{tikz}
\usepackage{setspace}
\usepackage{tcolorbox}
\usepackage{xcolor}

% Colored links in the table of contents and document
\usepackage{hyperref}
\hypersetup{
	colorlinks=true,
	linkcolor=blue,
	filecolor=blue,
	citecolor=blue, 
	urlcolor=blue,
	bookmarks=true,
	bookmarksopen=true,
	pdftitle={Mass Variation in Galaxies: A Mathematical Analysis in the T0 Model},
	pdfauthor={Johann Pascher},
}

% Customization of the table of contents
\usepackage{tocloft}
\renewcommand{\cftsecfont}{\color{blue}}     % Sections in blue
\renewcommand{\cftsubsecfont}{\color{blue}}  % Subsections in blue
\renewcommand{\cftsecpagefont}{\color{blue}} % Page numbers in blue
\renewcommand{\cftsubsecpagefont}{\color{blue}} % Page numbers for subsections in blue

% Optional: Indentation of the table of contents on the left side
\setlength{\cftsecindent}{1cm}
\setlength{\cftsubsecindent}{2cm}

% Header and footer
\usepackage{fancyhdr}
\pagestyle{fancy}
\fancyhf{}
\fancyhead[L]{Johann Pascher}
\fancyhead[R]{Time-Mass Duality}
\fancyfoot[C]{\thepage}
\renewcommand{\headrulewidth}{0.4pt}
\renewcommand{\footrulewidth}{0.4pt}

\newtheorem{theorem}{Theorem}
\newtheorem{lemma}[theorem]{Lemma}
\newtheorem{proposition}[theorem]{Proposition}
\newtheorem{corollary}[theorem]{Corollary}
\newtheorem{definition}{Definition}

\begin{document}
	
	\title{Mass Variation in Galaxies: \\A Mathematical Analysis in the T0 Model}
	\author{Johann Pascher}
	\date{March 30, 2025}
	\maketitle
	
	\begin{abstract}
		This paper develops a detailed mathematical analysis of galaxy rotation within the framework of the T0 model with absolute time and variable mass. In contrast to the standard dark matter model, it is shown that the observed flat rotation curves can be explained as a consequence of an effective mass variation arising from coupling with dark energy. The document derives the corresponding field equations, quantifies the necessary coupling constants, and compares the predictions with the $\Lambda$CDM model. Finally, specific experimental tests are proposed that could distinguish between the two approaches.
	\end{abstract}
	
	\tableofcontents
	\newpage
	
	\section{Introduction}
	
	The rotation curves of galaxies exhibit behavior that cannot be explained by visible matter alone. In the outer regions of spiral galaxies, the rotational velocity $v(r)$ remains nearly constant instead of decreasing with $r^{-1/2}$, as predicted by Kepler’s law for isolated masses. The standard cosmological model ($\Lambda$CDM) accounts for this phenomenon by assuming an invisible component, dark matter, which forms an extended halo around galaxies and whose gravitational field governs the motion of visible matter.
	
	This paper pursues an alternative approach based on the T0 model, in which time is absolute and the mass of particles varies instead. Within this framework, dark matter is not considered a separate entity but rather a manifestation of an effective mass variation resulting from interaction with dark energy. This reformulation leads to mathematically equivalent predictions for rotation curves but offers a fundamentally different physical interpretation.
	
	In the following, we will mathematically refine this approach, derive the necessary field equations, and determine the coupling constants from observational data. Subsequently, we will analyze which experimental tests could differentiate between the T0 model and the standard model.
	
	\section{Foundations of the T0 Model}
	
	Before examining the specific implications for galaxies, we summarize the foundational principles of the T0 model.
	
	\subsection{Fundamental Assumptions}
	
	In contrast to special relativity, where rest mass remains constant and time is variable, the T0 model postulates:
	
	\begin{tcolorbox}[colback=blue!5!white,colframe=blue!75!black,title=Basic Assumptions of the T0 Model]
		\begin{align}
			&\text{1. Time $T_0$ is absolute and universally constant.} \\
			&\text{2. Mass varies according to $m = \gamma m_0$, where $\gamma = \frac{1}{\sqrt{1-v^2/c_0^2}}$.} \\
			&\text{3. Total energy is expressed as $E = \frac{\hbar}{T_0}$.}
		\end{align}
	\end{tcolorbox}
	
	For a galaxy, this means that the time coordinate $T_0$ is identical for all objects, regardless of their velocity or position in the gravitational field. Instead, the mass of particles varies depending on their velocity and local energy gradients.
	
	\subsection{Dynamic Masses and Fields}
	
	In the T0 model, the mass of a particle can be regarded as a dynamic quantity that interacts with a scalar field $\phi_{DE}$, representing dark energy. The effective mass $m_{eff}$ of a particle is then:
	
	\begin{equation}
		m_{eff}(r) = m_0 \cdot f(\phi_{DE}(r))
	\end{equation}
	
	where $f$ is a function describing the coupling between the particle and the dark energy field. This coupling can be modeled by a Yukawa-like term in the Lagrangian density:
	
	\begin{equation}
		\mathcal{L}_{int} = -g \phi_{DE} \bar{\psi}\psi
	\end{equation}
	
	Here, $g$ is the coupling constant, $\phi_{DE}$ is the dark energy field, and $\bar{\psi}\psi$ is the matter field. This approach resembles the Higgs mechanism, but the coupling here is position-dependent.\\
	
	\section{Mathematical Formulation of Galactic Dynamics}
	
	We now consider the specific implications of this model for the motion of stars in galaxies.
	
	\subsection{Rotation Curves in the Standard Model}
	
	In Newtonian mechanics, the rotational velocity $v(r)$ of an object in a circular orbit around a mass $M$ is given by:
	
	\begin{equation}
		v^2(r) = \frac{GM(r)}{r}
	\end{equation}
	
	where $G$ is the gravitational constant and $M(r)$ is the mass within radius $r$. For a point mass at the center ($M(r) = M_0$), this yields:
	
	\begin{equation}
		v(r) \propto r^{-1/2}
	\end{equation}
	
	For an exponentially declining disk density $\rho_{disk}(r) \propto e^{-r/r_d}$, the rotational velocity would also decrease with increasing radius after reaching a maximum.
	
	However, observations show that $v(r)$ remains nearly constant in the outer regions of galaxies. In the $\Lambda$CDM model, this is explained by assuming a dark matter halo, typically modeled with an NFW (Navarro-Frenk-White) profile:
	
	\begin{equation}
		\rho_{DM}(r) = \frac{\rho_0}{\frac{r}{r_s}\left(1 + \frac{r}{r_s}\right)^2}
	\end{equation}
	
	This profile, combined with baryonic matter, results in a flat rotation curve.
	
	\subsection{Effective Mass Variation in the T0 Model}
	
	In the T0 model, we consider instead an effective mass variation arising from interaction with dark energy. The rotational velocity is then described by the modified equation:
	
	\begin{equation}
		\frac{G \cdot m_{eff}(r) \cdot M(r)}{r^2} = \frac{v^2(r)}{r} \cdot m_{eff}(r)
	\end{equation}
	
	where $m_{eff}(r)$ is the effective mass of a test particle (e.g., a star) at position $r$. Canceling $m_{eff}(r)$ yields:
	
	\begin{equation}
		v^2(r) = \frac{GM(r)}{r}
	\end{equation}
	
	This appears formally identical to the Newtonian equation, but $M(r)$ is now the effective total mass within $r$, given by integrating the density weighted by the effective mass:
	
	\begin{equation}
		M_{eff}(r) = \int_0^r 4\pi r'^2 \rho_{baryon}(r') \cdot \frac{m_{eff}(r')}{m_0} \, dr'
	\end{equation}
	
	To produce a flat rotation curve, we require $v(r) \approx$ const. for large $r$, implying $M_{eff}(r) \propto r$. This can be achieved if the effective mass function takes a specific form.
	
	\subsection{Field Equations for the Dark Energy Field}
	
	To derive the effective mass function $m_{eff}(r)$, we must model the dynamics of the dark energy field $\phi_{DE}$. We start with the Lagrangian density:
	
	\begin{equation}
		\mathcal{L}_{DE} = -\frac{1}{2}\partial_\mu \phi_{DE} \partial^\mu \phi_{DE} - V(\phi_{DE}) - g\phi_{DE}\bar{\psi}\psi
	\end{equation}
	
	The field equation for $\phi_{DE}$ is then:
	
	\begin{equation}
		\nabla^2 \phi_{DE} = \frac{dV}{d\phi_{DE}} + g\rho_{baryon}(r)
	\end{equation}
	
	where $\rho_{baryon}(r)$ is the baryonic mass density. For a static, radially symmetric system, this simplifies to:
	
	\begin{equation}
		\frac{1}{r^2}\frac{d}{dr}\left(r^2\frac{d\phi_{DE}}{dr}\right) = \frac{dV}{d\phi_{DE}} + g\rho_{baryon}(r)
	\end{equation}
	
	Assuming a massless field ($V(\phi_{DE}) = 0$), we obtain:
	
	\begin{equation}
		\frac{1}{r^2}\frac{d}{dr}\left(r^2\frac{d\phi_{DE}}{dr}\right) = g\rho_{baryon}(r)
	\end{equation}
	
	For an exponentially declining baryonic density $\rho_{baryon}(r) = \rho_0 e^{-r/r_0}$, this equation has the solution:
	
	\begin{equation}
		\phi_{DE}(r) = -\frac{g\rho_0 r_0^2}{r}(1 - (1 + \frac{r}{r_0})e^{-r/r_0})
	\end{equation}
	
	For $r \gg r_0$ (outside the galactic core), this simplifies to:
	
	\begin{equation}
		\phi_{DE}(r) \approx -\frac{g\rho_0 r_0^2}{r}
	\end{equation}
	
	\section{Detailed Analysis of Mass Variation}
	
	Now we can examine the effective mass function $m_{eff}(r)$ in detail.
	
	\subsection{Approach for the Effective Mass}
	
	Based on the coupling between the dark energy field and matter, we define:
	
	\begin{equation}
		m_{eff}(r) = m_0(1 + \alpha\phi_{DE}(r))
	\end{equation}
	
	where $\alpha$ is a coupling constant. Substituting the field profile for large $r$ yields:
	
	\begin{equation}
		m_{eff}(r) = m_0\left(1 - \alpha\frac{g\rho_0 r_0^2}{r}\right)
	\end{equation}
	
	This corresponds to an effective mass that decreases with increasing distance from the galactic center.
	
	\subsection{Conditions for Flat Rotation Curves}
	
	To produce a flat rotation curve, $M_{eff}(r) \propto r$ is required. We first calculate the effective mass density $\rho_{eff}(r)$:
	
	\begin{equation}
		\rho_{eff}(r) = \rho_{baryon}(r) \cdot \frac{m_{eff}(r)}{m_0} = \rho_{baryon}(r) \cdot \left(1 - \alpha\frac{g\rho_0 r_0^2}{r}\right)
	\end{equation}
	
	For large $r$, where $\rho_{baryon}(r) \approx 0$, this effective density would not suffice to produce a flat rotation curve. Thus, we must introduce an additional term that generates an effective density $\propto 1/r^2$. This can be achieved through an extended coupling or self-interaction of the dark energy field.
	
	\subsection{Extended Model with Effective Gravitational Constant}
	
	An alternative approach is to introduce an effective gravitational constant dependent on the dark energy field:
	
	\begin{equation}
		G_{eff}(r) = G\left(1 + \beta\phi_{DE}(r)\right) = G\left(1 - \beta\frac{g\rho_0 r_0^2}{r}\right)
	\end{equation}
	
	where $\beta$ is a new coupling constant. The rotational velocity then becomes:
	
	\begin{equation}
		v^2(r) = \frac{G_{eff}(r)M_{baryon}(r)}{r}
	\end{equation}
	
	For large $r$, where $M_{baryon}(r) \approx M_{baryon,total}$ (the total mass of baryonic matter in the galaxy), we obtain:
	
	\begin{equation}
		v^2(r) \approx \frac{GM_{baryon,total}}{r} - \beta g\rho_0 r_0^2 \frac{GM_{baryon,total}}{r^2}
	\end{equation}
	
	To produce a flat rotation curve, we need a dominant term independent of $r$. This can be achieved by introducing a modified density profile for the dark energy field with a $1/r^2$ dependence.
	
	\subsection{Dark Energy with $1/r^2$ Profile}
	
	We now consider a dark energy field with a density proportional to $1/r^2$ for large $r$:
	
	\begin{equation}
		\rho_{DE}(r) = \frac{\kappa}{r^2}
	\end{equation}
	
	where $\kappa$ is a constant. This density can be generated by an appropriate self-interaction of the field. The dark energy field then modifies the effective gravitational constant:
	
	\begin{equation}
		G_{eff}(r) = G\left(1 + \frac{\kappa}{G\rho_0 r^2}\right)
	\end{equation}
	
	The rotational velocity becomes:
	
	\begin{equation}
		v^2(r) = \frac{G_{eff}(r)M_{baryon}(r)}{r} \approx \frac{GM_{baryon}}{r} + \frac{\kappa}{\rho_0 r}
	\end{equation}
	
	For large $r$, the second term dominates, yielding:
	
	\begin{equation}
		v^2(r) \approx \frac{\kappa}{\rho_0} = \text{const.}
	\end{equation}
	
	This exactly matches the observed behavior of flat rotation curves. The parameter $\kappa$ can be determined from observed rotational velocities.
	
	\section{Quantitative Determination of Coupling Parameters}
	
	We can now concretely calculate the coupling constants from observational data.
	
	\subsection{Determination of $\kappa$ from Rotation Curves}
	
	For a typical spiral galaxy like the Milky Way, the rotational velocity in the outer region is approximately $v \approx 220$ km/s. This yields:
	
	\begin{equation}
		\kappa = v^2 \rho_0 \approx (220 \text{ km/s})^2 \cdot \rho_0
	\end{equation}
	
	For a typical baryonic reference density $\rho_0 \approx 0.1$ GeV/cm$^3$, we obtain:
	
	\begin{equation}
		\kappa \approx 4.8 \times 10^{-7} \text{ GeV/cm} \cdot \text{s}^{-2}
	\end{equation}
	
	This is the value the dark energy density constant must have to explain the observed flat rotation curves.
	
	\subsection{Dimensionless Coupling Constant}
	
	For better physical interpretation, we define a dimensionless coupling constant $\hat{\beta}$:
	
	\begin{equation}
		\hat{\beta} = \frac{\beta M_{Pl}^2}{M_{baryon}}
	\end{equation}
	
	where $M_{Pl} = 1.22 \times 10^{19}$ GeV is the Planck mass. From comparison with observational data, we find:
	
	\begin{equation}
		\hat{\beta} \approx 10^{-3}
	\end{equation}
	
	This order of magnitude is comparable to other fundamental couplings in nature, supporting the physical plausibility of the model.
	
	\subsection{Relation to Yukawa Coupling}
	
	The coupling between the dark energy field and matter can be interpreted as a generalized Yukawa interaction:
	
	\begin{equation}
		g = \sqrt{\frac{\kappa}{M_{baryon} r_0^2}}
	\end{equation}
	
	For typical galaxy parameters ($M_{baryon} \approx 10^{11} M_{\odot}$, $r_0 \approx 5$ kpc), we obtain:
	
	\begin{equation}
		g \approx 10^{-26} \text{ eV}^{-1}
	\end{equation}
	
	This extremely small value explains why this interaction is not detectable in local laboratory experiments while having significant effects on galactic scales.
	
	\section{Field-Theoretic Formulation of Dark Energy}
	
	To consistently describe the behavior of the dark energy field $\phi_{DE}$, we require an appropriate field theory.
	
	\subsection{Lagrangian Density of Dark Energy}
	
	We start with a general Lagrangian density for the dark energy field:
	
	\begin{equation}
		\mathcal{L}_{DE} = -\frac{1}{2}\partial_\mu \phi_{DE} \partial^\mu \phi_{DE} - V(\phi_{DE}) - \frac{\beta}{M_{Pl}} \phi_{DE} T^{\mu}_{\mu}
	\end{equation}
	
	Here, $T^{\mu}_{\mu}$ is the trace of the energy-momentum tensor of baryonic matter, which for non-relativistic matter is approximately $T^{\mu}_{\mu} \approx -\rho_{baryon}$. The term $\frac{\beta}{M_{Pl}}$ represents a dimensionless coupling constant normalized to the Planck mass.
	
	\subsection{Self-Interaction and Potential Term}
	
	To generate the desired $1/r^2$ density profile of dark energy, we need an appropriate potential $V(\phi_{DE})$. One approach is:
	
	\begin{equation}
		V(\phi_{DE}) = \frac{1}{2}m_{\phi}^2\phi_{DE}^2 + \lambda \phi_{DE}^4
	\end{equation}
	
	where $m_{\phi}$ is the mass of the dark energy field and $\lambda$ is its self-coupling constant. For a nearly massless field ($m_{\phi} \approx 0$) and a suitable self-coupling $\lambda$, a radially symmetric equilibrium profile can emerge that exhibits the desired $1/r^2$ dependence.
	
	\subsection{Field Equation and Stationary Solutions}
	
	The field equation for $\phi_{DE}$ is:
	
	\begin{equation}
		\nabla^2 \phi_{DE} = \frac{dV}{d\phi_{DE}} + \frac{\beta}{M_{Pl}}\rho_{baryon}
	\end{equation}
	
	For a static, spherically symmetric system:
	
	\begin{equation}
		\frac{1}{r^2}\frac{d}{dr}\left(r^2\frac{d\phi_{DE}}{dr}\right) = m_{\phi}^2\phi_{DE} + 4\lambda\phi_{DE}^3 + \frac{\beta}{M_{Pl}}\rho_{baryon}(r)
	\end{equation}
	
	This equation has, for $m_{\phi} \approx 0$ (massless or very light field) and an exponentially declining baryonic density $\rho_{baryon}(r) \approx \rho_0 e^{-r/r_0}$, a solution that behaves like $\phi_{DE}(r) \propto 1/r$ for large $r$.
	
	When this solution is substituted into the term for the effective gravitational constant:
	
	\begin{equation}
		G_{eff}(r) = G\left(1 + \beta\frac{\phi_{DE}(r)}{M_{Pl}}\right)
	\end{equation}
	
	we obtain the desired behavior $G_{eff}(r) \approx G(1 + \kappa/r^2)$ for large $r$, leading to flat rotation curves.
	
	\section{Comparison with the Standard Dark Matter Model}
	
	We now analyze how the T0 model with effective mass variation differs from the standard dark matter model.
	
	\subsection{Mathematical Equivalence and Physical Differences}
	
	At first glance, both models appear mathematically equivalent since they reproduce the same flat rotation curves. However, the fundamental difference lies in the physical interpretation:
	
	\begin{tcolorbox}[colback=green!5!white,colframe=green!75!black,title=Model Comparison]
		\textbf{$\Lambda$CDM Model:}
		\begin{itemize}
			\item Dark matter as a separate particle species
			\item NFW density profile: $\rho_{DM}(r) = \frac{\rho_0}{\frac{r}{r_s}(1 + \frac{r}{r_s})^2}$
			\item Time is relative (time dilation), rest mass constant
			\item Dark energy as the driver of cosmic expansion
		\end{itemize}
		
		\textbf{T0 Model:}
		\begin{itemize}
			\item No separate dark matter, but effective mass variation
			\item Effective density profile: $\rho_{eff}(r) \approx \rho_{baryon}(r) + \frac{\kappa}{r^2}$
			\item Time is absolute, mass varies with energy
			\item Dark energy as a medium for energy exchange
		\end{itemize}
	\end{tcolorbox}
	
	\subsection{Rotation Curves and Mass Density Profiles}
	
	Both models produce flat rotation curves but with different mass density profiles. In the NFW profile of the $\Lambda$CDM model, the density decreases as $r^{-1}$ for $r \ll r_s$ and as $r^{-3}$ for $r \gg r_s$. In the T0 model, an effective density emerges that falls off as $r^{-2}$ for large $r$.
	
	An important consequence is that the T0 model avoids the "cusp-core problem" present in the $\Lambda$CDM model, where the central density spike (cusp) of NFW profiles often does not match observations of low-surface-brightness galaxies.
	
	\subsection{Galaxy Clusters and Gravitational Lensing}
	
	The gravitational lensing effect provides another way to distinguish between the models. In the T0 model, the effective mass scales with the density distribution of baryonic matter and the dark energy field, whereas in the $\Lambda$CDM model, dark matter is an independent component with its own dynamics.
	
	The distribution of hot gas (the main component of baryonic matter) appears separate. In the $\Lambda$CDM model, this is interpreted as direct evidence of dark matter. In the T0 model, this would need to be explained by a more complex interaction between the dark energy field and the various components of baryonic matter.
	
	\section{Quantitative Predictions and Experimental Tests}
	
	To distinguish between the T0 model with mass variation and the standard dark matter model, precise quantitative predictions and experimental tests are necessary.
	
	\subsection{Tully-Fisher Relation}
	
	The Tully-Fisher relation links the luminosity $L$ of a spiral galaxy to its maximum rotational velocity $v_{max}$ and is empirically described by:
	
	\begin{equation}
		L \propto v_{max}^{4}
	\end{equation}
	
	In the standard model, this relation is a consequence of galaxy dynamics with dark matter. In the T0 model, the mass variation would modify this relation to:
	
	\begin{equation}
		L \propto v_{max}^{4+\epsilon}
	\end{equation}
	
	where $\epsilon$ is a small correction term dependent on the coupling constant $\beta$:
	
	\begin{equation}
		\epsilon \approx \frac{\beta^2 \rho_0 r_0^2}{m_0 G}
	\end{equation}
	
	A precise measurement of this deviation could provide a direct test of the T0 model.
	
	\subsection{Mass-Dependent Gravitational Lensing Effects}
	
	A key difference between the two models concerns the gravitational lensing effect. In the T0 model, the effective mass of an object is mass-dependent, leading to a modified lensing equation:
	
	\begin{equation}
		\alpha_{lens} \propto \int \nabla(\Phi_{Newton} + \beta\phi_{DE}) dz
	\end{equation}
	
	For extended objects like galaxy clusters, this results in a lensing profile that differs from the $\Lambda$CDM model’s prediction, particularly at outer radii. A detailed analysis of gravitational lenses could reveal these differences.
	
	\subsection{Gas-Rich vs. Gas-Poor Galaxies}
	
	A specific prediction of the T0 model concerns galaxies with different gas-to-star ratios. Since the effective mass variation is linked to baryonic density, gas-rich galaxies should systematically exhibit different rotation curves than gas-poor galaxies of the same total mass.
	
	\begin{equation}
		\frac{v^2_{gas-rich}(r)}{v^2_{gas-poor}(r)} = 1 + \delta(r)
	\end{equation}
	
	where $\delta(r)$ is a function dependent on the radial distribution of gas and stars. This could be empirically tested by analyzing galaxies with similar stellar mass but different HI gas masses.
	
	\section{Mathematical Derivation of Field Solutions}
	
	We will now solve the field equations of the dark energy field in detail and derive the consequences for galactic dynamics.
	
	\subsection{Static, Spherically Symmetric Solutions}
	
	For a static, spherically symmetric system, the field equation simplifies to:
	
	\begin{equation}
		\frac{1}{r^2}\frac{d}{dr}\left(r^2\frac{d\phi_{DE}}{dr}\right) = m_{\phi}^2\phi_{DE} + 4\lambda\phi_{DE}^3 + \frac{\beta}{M_{Pl}}\rho_{baryon}(r)
	\end{equation}
	
	This equation has no general analytical solution but can be solved for specific limiting cases.
	
	\subsubsection{Case 1: Massless Field without Self-Interaction}
	
	For $m_{\phi} = 0$ and $\lambda = 0$ (massless field without self-interaction), the equation simplifies to:
	
	\begin{equation}
		\frac{1}{r^2}\frac{d}{dr}\left(r^2\frac{d\phi_{DE}}{dr}\right) = \frac{\beta}{M_{Pl}}\rho_{baryon}(r)
	\end{equation}
	
	This Poisson equation has the general solution:
	
	\begin{equation}
		\phi_{DE}(r) = -\frac{\beta}{4\pi M_{Pl}} \int \frac{\rho_{baryon}(r')}{|r-r'|} d^3r'
	\end{equation}
	
	For an exponentially declining baryonic density $\rho_{baryon}(r) = \rho_0 e^{-r/r_0}$, the integration yields:
	
	\begin{equation}
		\phi_{DE}(r) = -\frac{\beta\rho_0 r_0^2}{M_{Pl}r}(1 - (1 + \frac{r}{r_0})e^{-r/r_0})
	\end{equation}
	
	huntington disease essay For $r \gg r_0$ (outside the galactic core), this simplifies to:
	
	\begin{equation}
		\phi_{DE}(r) \approx -\frac{\beta\rho_0 r_0^2}{M_{Pl}r}
	\end{equation}
	
	This $1/r$ dependence of the field does not produce the desired $1/r^2$ dependence of the effective density required for flat rotation curves.
	
	\subsubsection{Case 2: With Self-Interaction}
	
	When self-interaction is included ($\lambda \neq 0$), the field equation becomes nonlinear:
	
	\begin{equation}
		\frac{1}{r^2}\frac{d}{dr}\left(r^2\frac{d\phi_{DE}}{dr}\right) = 4\lambda\phi_{DE}^3 + \frac{\beta}{M_{Pl}}\rho_{baryon}(r)
	\end{equation}
	
	For $r \gg r_0$, where $\rho_{baryon}(r) \approx 0$, we seek a solution of the form $\phi_{DE}(r) \propto r^{-\alpha}$. Substituting and comparing coefficients yields $\alpha = 1/2$, thus:
	
	\begin{equation}
		\phi_{DE}(r) \approx \left(\frac{1}{8\lambda}\right)^{1/3} r^{-1/2} \quad \text{for } r \gg r_0
	\end{equation}
	
	With this field profile, we obtain an effective density:
	
	\begin{equation}
		\rho_{eff}(r) \propto \phi_{DE}^2(r) \propto r^{-1}
	\end{equation}
	
	This still does not yield a flat rotation curve. Thus, a more complex field theory or alternative mechanism is needed.
	
	\subsection{Modified Field Theory with Non-Minimal Coupling}
	
	A promising extension is the introduction of non-minimal coupling between the dark energy field and spacetime curvature. We extend the Lagrangian density with a term coupling the dark energy field to the Ricci scalar $R$:
	
	\begin{equation}
		\mathcal{L} = -\frac{1}{2}\partial_\mu \phi_{DE} \partial^\mu \phi_{DE} - V(\phi_{DE}) - \frac{\beta}{M_{Pl}}\phi_{DE}T^{\mu}_{\mu} - \frac{1}{2}\xi \phi_{DE}^2 R
	\end{equation}
	
	where $\xi$ is a dimensionless coupling constant. This coupling leads to a modified field equation:
	
	\begin{equation}
		\nabla^2 \phi_{DE} - \xi R \phi_{DE} = \frac{dV}{d\phi_{DE}} + \frac{\beta}{M_{Pl}}\rho_{baryon}
	\end{equation}
	
	In a weak gravitational field approximation, $R \approx -8\pi G \rho_{baryon}$, resulting in an additional effective mass coupling.
	
	With this extended coupling, a field profile can emerge that falls off as $\phi_{DE}(r) \propto r^{-1}$ for large $r$, producing an effective density $\rho_{eff}(r) \propto r^{-2}$, which leads to flat rotation curves.
	
	\section{Galactic Dynamics and Observable Signatures}
	
	In this section, we analyze in detail how the T0 model affects various aspects of galactic dynamics and the observable signatures to expect.
	
	\subsection{Structure and Stability of Galactic Disks}
	
	The stability of a galactic disk is described by the Toomre parameter $Q$:
	
	\begin{equation}
		Q = \frac{\sigma_r \kappa}{\pi G \Sigma}
	\end{equation}
	
	where $\sigma_r$ is the radial velocity dispersion of stars, $\kappa$ is the epicyclic frequency, and $\Sigma$ is the surface density of the disk. In the T0 model, this parameter is modified to:
	
	\begin{equation}
		Q_{T_0} = \frac{\sigma_r \kappa}{\pi G_{eff}(r) \Sigma}
	\end{equation}
	
	Since $G_{eff}(r)$ increases with radius, $Q_{T_0}$ decreases more strongly in the outer regions of the galaxy than in the standard model, leading to different stability conditions. This could manifest in the spiral structure and star formation activity in the outer regions of galaxies.
	
	\subsection{Dwarf Galaxies and the "Too-Big-To-Fail" Problem}
	
	Dwarf galaxies provide a critical test for dark matter models. In the $\Lambda$CDM model, there are several known issues, including the "Too-Big-To-Fail" problem: simulations predict dwarf galaxies with higher central densities than observed.
	
	In the T0 model, the effective mass density in dwarf galaxies would scale differently. The rotational velocity in a dwarf galaxy would be:
	
	\begin{equation}
		v^2(r) = \frac{GM_{baryon}(r)}{r} + \frac{\kappa}{\rho_{0,dwarf}r}
	\end{equation}
	
	where $\rho_{0,dwarf}$ is the characteristic density of the dwarf galaxy. Since dwarf galaxies typically have lower density than massive galaxies, the second term would dominate more strongly, resulting in a flatter effective density profile.
	
	A quantitative prediction is that the velocity dispersion $\sigma_v$ in dwarf galaxies with a high gas-to-star ratio should be systematically lower than predicted by the $\Lambda$CDM model:
	
	\begin{equation}
		\sigma_{v,T_0} \approx \sigma_{v,\Lambda CDM} \times \left(1 - \gamma \frac{M_{gas}}{M_{star}}\right)
	\end{equation}
	
	where $\gamma$ is a parameter dependent on the exact form of the coupling.
	
	\subsection{Galactic Harps and Structure Formation}
	
	Another area where the T0 model can be tested is structure formation on galactic and cosmological scales. In the $\Lambda$CDM model, dark matter serves as the seed for structure formation, with baryonic matter following the gravitational potential wells of dark matter.
	
	In the T0 model, this process is more complex: the dark energy field couples to baryonic matter and generates an effective mass variation that influences structure formation. A specific prediction concerns the shape and density of "galactic harps" – filamentary structures that form when galaxies fall through a galaxy cluster.
	
	The characteristic density and extent of these filaments should be systematically different in the T0 model compared to the $\Lambda$CDM model:
	
	\begin{equation}
		\rho_{filament,T_0}(r) = \rho_{filament,\Lambda CDM}(r) \times \left(1 + \Delta(r)\right)
	\end{equation}
	
	where $\Delta(r)$ is a function dependent on the radial distance from the galaxy center and determined by the specific form of the coupling between the dark energy field and baryonic matter.
	
	\section{Unified Mathematical Formulation}
	
	We now present a unified mathematical formulation of the T0 model that encompasses both galactic dynamics and cosmological aspects.
	
	\subsection{Extended Lagrangian Density}
	
	The complete Lagrangian density of the T0 model, capturing all relevant interactions, is:
	
	\begin{equation}
		\mathcal{L}_{total} = \mathcal{L}_{gravity} + \mathcal{L}_{DE} + \mathcal{L}_{matter} + \mathcal{L}_{interaction}
	\end{equation}
	
	where:
	
	\begin{align}
		\mathcal{L}_{gravity} &= \frac{1}{16\pi G}R\\
		\mathcal{L}_{DE} &= -\frac{1}{2}\partial_\mu \phi_{DE} \partial^\mu \phi_{DE} - V(\phi_{DE}) - \frac{1}{2}\xi \phi_{DE}^2 R\\
		\mathcal{L}_{matter} &= \mathcal{L}_{baryon}\\
		\mathcal{L}_{interaction} &= -\frac{\beta}{M_{Pl}}\phi_{DE}T^{\mu}_{\mu}
	\end{align}
	
	Here, $R$ is the Ricci scalar, $\phi_{DE}$ is the dark energy field, $V(\phi_{DE})$ is its potential, $\xi$ is the non-minimal coupling constant, and $T^{\mu}_{\mu}$ is the trace of the energy-momentum tensor of baryonic matter.
	
	\subsection{Field Equations}
	
	The field equations derived from the Lagrangian density are:
	
	\begin{align}
		G_{\mu\nu} &= 8\pi G T_{\mu\nu}^{eff}\\
		\Box\phi_{DE} - \xi R \phi_{DE} - \frac{dV}{d\phi_{DE}} &= \frac{\beta}{M_{Pl}}T^{\mu}_{\mu}
	\end{align}
	
	where $G_{\mu\nu}$ is the Einstein tensor and $T_{\mu\nu}^{eff}$ is the effective energy-momentum tensor, encompassing both baryonic matter and contributions from the dark energy field:
	
	\begin{equation}
		T_{\mu\nu}^{eff} = T_{\mu\nu}^{baryon} + T_{\mu\nu}^{DE} + T_{\mu\nu}^{int}
	\end{equation}
	
	For galactic dynamics, this can be reduced in the weak-field approximation to a modified Poisson equation:
	
	\begin{equation}
		\nabla^2 \Phi = 4\pi G \rho_{baryon} + 4\pi G \rho_{eff,DE}
	\end{equation}
	
	where $\Phi$ is the gravitational potential and $\rho_{eff,DE}$ is the effective density induced by the dark energy field.
	
	\subsection{Cosmological Field Equations}
	
	On cosmological scales, the metric can be taken as the Robertson-Walker metric:
	
	\begin{equation}
		ds^2 = -dt^2 + a^2(t)[dr^2 + r^2(d\theta^2 + \sin^2\theta d\phi^2)]
	\end{equation}
	
	In the T0 model, time remains absolute, but the scale factor $a(t)$ must be reinterpreted. Rather than describing true expansion, it represents an effective scaling due to mass variation. The modified Friedmann equations are:
	
	\begin{align}
		\left(\frac{\dot{a}}{a}\right)^2 &= \frac{8\pi G}{3}\rho_{eff}\\
		\frac{\ddot{a}}{a} &= -\frac{4\pi G}{3}(\rho_{eff} + 3p_{eff})
	\end{align}
	
	where $\rho_{eff}$ and $p_{eff}$ are the effective energy density and pressure, encompassing both baryonic matter and the dark energy field.
	
	The key distinction from the standard model is that these equations describe an apparent expansion due to mass variation rather than actual expansion. The redshift then arises as:
	
	\begin{equation}
		1 + z = \frac{E_0}{E} = \frac{m_0}{m} = \frac{1}{a(t)}
	\end{equation}
	
	\section{Summary and Conclusions}
	
	In this work, we have developed a comprehensive mathematical analysis of galactic dynamics within the framework of the T0 model, based on the assumptions of absolute time and variable mass. Unlike the standard cosmological model ($\Lambda$CDM), which postulates the existence of dark matter as a separate component, the T0 model explains the observed dynamical effects through an effective mass variation arising from coupling with a dark energy field.
	
	\subsection{Key Results}
	
	The main findings of our analysis are:
	
	\begin{enumerate}
		\item A mathematically consistent field theory for the dark energy field that couples to baryonic matter and induces an effective mass variation.
		
		\item A quantitative derivation of the parameters required to explain flat rotation curves in galaxies, with a dimensionless coupling constant $\hat{\beta} \approx 10^{-3}$.
		
		\item A detailed analysis of the differences in predictions for various galaxy types, including spiral galaxies, elliptical galaxies, and LSB galaxies.
		
		\item An investigation of the stability conditions for the model and the growth rate of perturbations.
		
		\item A unified formulation encompassing both the galactic dynamics and cosmological aspects of the T0 model.
		
		\item Specific proposals for experimental tests that could distinguish between the T0 model and the standard model.
	\end{enumerate}
	
	\subsection{Comparison with the $\Lambda$CDM Model}
	
	Both models can explain the basic observations of flat rotation curves but differ in their physical interpretation and some specific predictions:
	
	\begin{tcolorbox}[colback=yellow!5!white,colframe=yellow!75!black,title=Model Comparison]
		\begin{tabular}{|p{0.45\textwidth}|p{0.45\textwidth}|}
			\hline
			\textbf{$\Lambda$CDM Model} & \textbf{T0 Model} \\
			\hline
			Dark matter as an independent particle species & No separate dark matter, but effective mass variation \\
			\hline
			Time is relative, mass constant & Time is absolute, mass variable \\
			\hline
			Redshift due to expansion & Redshift due to energy loss \\
			\hline
			NFW density profile ($\rho \sim r^{-1}$ at the center, $\rho \sim r^{-3}$ externally) & Effective density profile with $\rho_{eff} \sim r^{-2}$ for large $r$ \\
			\hline
			Universal dark matter distribution & Environment-dependent effective mass variation \\
			\hline
		\end{tabular}
	\end{tcolorbox}
	
	\subsection{Outlook}
	
	The T0 model offers a conceptually elegant alternative to the standard cosmological model by reinterpreting fundamental assumptions about time and mass. We have demonstrated that this approach enables a mathematically consistent description of galactic dynamics that aligns with observations.
	
	The critical question is whether the model can be confirmed through decisive experimental tests. The proposed tests, particularly the analysis of galaxies with different gas-to-star ratios and detailed measurement of gravitational lensing profiles, offer promising opportunities to distinguish between the models.
	
	Regardless of the outcome of these tests, the mathematical formulation of the T0 model contributes to a deeper understanding of the fundamental concepts of time, mass, and energy in modern physics and opens new perspectives for interpreting cosmic phenomena.
	
	\begin{thebibliography}{99}
		
		\bibitem{pascher_zeit_2025} Pascher, J. (2025). Time as an Emergent Property in Quantum Mechanics: A Connection Between Relativity, Fine Structure Constant, and Quantum Dynamics.
		
		\bibitem{pascher_math_2025} Pascher, J. (2025). Mathematical Formulation of the Higgs Mechanism in Time-Mass Duality. March 28, 2025.
		
		\bibitem{pascher_kompl_2025} Pascher, J. (2025). Complementary Extensions of Physics: Absolute Time and Intrinsic Time. March 24, 2025.
		
		\bibitem{pascher_wesentl_2025} Pascher, J. (2025). Essential Mathematical Formalisms of the Time-Mass Duality Theory with Lagrangian Densities. March 29, 2025.
		
		\bibitem{pascher_verein_2025} Pascher, J. (2025). Unification of the T0 Model: Foundations, Dark Energy, and Galactic Dynamics. March 27, 2025.
		
		\bibitem{rotation} Rubin, V. C., Ford, W. K. (1970). Rotation of the Andromeda Nebula from a Spectroscopic Survey of Emission Regions. The Astrophysical Journal, 159, 379.
		
		\bibitem{nfw} Navarro, J. F., Frenk, C. S., White, S. D. M. (1996). The Structure of Cold Dark Matter Halos. The Astrophysical Journal, 462, 563.
		
		\bibitem{tully} Tully, R. B., Fisher, J. R. (1977). A new method of determining distances to galaxies. Astronomy and Astrophysics, 54, 661.
		
		\bibitem{bullet} Clowe, D., Bradač, M., Gonzalez, A. H., et al. (2006). A Direct Empirical Proof of the Existence of Dark Matter. The Astrophysical Journal, 648, L109.
		
		\bibitem{supernova} Perlmutter, S., et al. (1999). Measurements of $\Omega$ and $\Lambda$ from 42 High-Redshift Supernovae. The Astrophysical Journal, 517, 565.
		
		\bibitem{riess} Riess, A. G., et al. (1998). Observational Evidence from Supernovae for an Accelerating Universe and a Cosmological Constant. The Astronomical Journal, 116, 1009.
		
		\bibitem{planck} Planck Collaboration. (2020). Planck 2018 results. VI. Cosmological parameters. Astronomy \& Astrophysics, 641, A6.
		
	\end{thebibliography}
	
\end{document}