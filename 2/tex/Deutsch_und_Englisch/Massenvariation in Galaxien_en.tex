\documentclass[a4paper,12pt]{article}
\usepackage[utf8]{inputenc}
\usepackage[T1]{fontenc}
\usepackage[english]{babel}
\usepackage{amsmath, amssymb, amsthm}
\usepackage{physics}
\usepackage{graphicx}
\usepackage{hyperref}
\usepackage{tikz}
\usepackage{setspace}
\usepackage{tcolorbox}
\usepackage{xcolor}
\usepackage{pgfplots}
\pgfplotsset{compat=1.18}

\hypersetup{
	colorlinks=true,
	linkcolor=blue,
	filecolor=blue,
	citecolor=blue, 
	urlcolor=blue,
	bookmarks=true,
	bookmarksopen=true,
	pdftitle={Mass Variation in Galaxies: An Analysis in the T0-Model with Emergent Gravitation},
	pdfauthor={Johann Pascher},
}

% Custom commands (aus Original übernommen und konsistent gemacht)
\newcommand{\Tfield}{T(x)}
\newcommand{\DhiggsT}{\mathcal{D}_\mu\Phi_T}
\newcommand{\DcovT}[1]{\Tfield D_\mu #1 + #1 \partial_\mu \Tfield}
\newcommand{\DhiggsTdef}{\Tfield (\partial_\mu + ig A_\mu) \Phi + \Phi \partial_\mu \Tfield}
\newcommand{\HiggsLagr}{\mathcal{L}_{\text{Higgs-T}}}

% Repository base URL
\newcommand{\repobase}{https://github.com/jpascher/T0-Time-Mass-Duality/tree/main/2/}

\begin{document}
	
	\title{Mass Variation in Galaxies: \\An Analysis in the T0-Model with Emergent Gravitation}
	\author{Johann Pascher}
	\date{March 30, 2025}
	\maketitle
	
	\begin{abstract}
		This work analyzes the dynamics of galaxies within the framework of the T0-model of the time-mass duality theory, where time is absolute and the mass varies as \( m = \frac{\hbar}{T c^2} \), with \( \Tfield \) being a dynamic intrinsic time field. Gravitation is not introduced as a fundamental interaction but emerges from the gradients of \( \Tfield \). We formulate a complete total Lagrangian density that encompasses the contributions of the four fundamental fields (Higgs, fermions, gauge bosons) as well as the intrinsic time field, and demonstrate that flat rotation curves can be explained by the variation of \( \Tfield \), without requiring dark matter or separate dark energy. Experimental tests to validate the model are proposed, including cosmological implications such as the interpretation of the cosmic microwave background.
	\end{abstract}
	
	\tableofcontents
	\newpage
	
	\section{Introduction}
	
	The rotation curves of galaxies exhibit behavior that cannot be explained by visible matter alone. In the outer regions of spiral galaxies, the rotational velocity \( v(r) \) remains nearly constant instead of decreasing with \( r^{-1/2} \), as predicted by Kepler's law for isolated masses. The standard model of cosmology (\(\Lambda\)CDM) accounts for this phenomenon by assuming the presence of an invisible component, dark matter, which forms an extended halo around galaxies and whose gravitational field governs the motion of visible matter, supplemented by dark energy to explain cosmic acceleration.
	
	This work pursues an alternative approach based on the T0-model of the time-mass duality theory, where time is absolute and the mass of particles varies as \( m = \frac{\hbar}{T c^2} \), with \( \Tfield \) being a dynamic intrinsic time field. In this framework, dark matter is not considered a separate entity; instead, the observed dynamical effects arise from emergent gravitation resulting from the gradients of \( \Tfield \). Similarly, effects traditionally attributed to dark energy, such as redshift, are explained by the spatial variation of \( \Tfield \), eliminating the need for a separate dark energy component as in \(\Lambda\)CDM. This reformulation leads to mathematically equivalent predictions for rotation curves and offers a fundamentally different physical interpretation, requiring neither dark matter nor separate dark energy. A detailed analysis of the cosmological implications of the T0-model, particularly regarding distance measurements, redshift, and the interpretation of the cosmic microwave background, can be found in \cite{pascher_messdifferenzen_2025}.
	
	\subsection{Redshift in the T0-Model}
	
	In the T0-model, the redshift \( z \) is determined by the variation of the intrinsic time field \( \Tfield \). The relationship between redshift and mass is given by:
	
	\begin{equation}
		1 + z = \frac{\Tfield}{\Tfield_0} = \frac{m_0}{m},
	\end{equation}
	
	where \( \Tfield_0 \) and \( m_0 \) are the values of the intrinsic time field and mass at the observer's location, respectively. This interpretation of redshift is based on intrinsic time and does not require cosmic expansion, in contrast to the \(\Lambda\)CDM model, where redshift is explained by the expansion of the universe:
	
	\begin{equation}
		1 + z = \frac{a(t_0)}{a(t_{\text{emit}})}.
	\end{equation}
	
	The spatial variation of \( \Tfield \) can be related to distance \( d \) via \( \Tfield = \Tfield_0 e^{-\alpha d} \), where \( \alpha = H_0/c \), leading to an equivalent form:
	
	\begin{equation}
		1 + z = e^{\alpha d}.
	\end{equation}
	
	This formulation aligns with the energy loss of photons due to the dynamics of \( \Tfield \), as detailed in \cite{pascher_messdifferenzen_2025}. The relationship between redshift and distance \( d \) in the T0-model is thus:
	
	\begin{equation}
		d = \frac{c \ln(1 + z)}{H_0},
	\end{equation}
	
	where \( H_0 \) is the Hubble constant, reinterpreted in the T0-model as a measure of the spatial variation rate of \( \Tfield \), rather than an expansion rate.
	
	\subsection{Cosmological Implications: Distance Measures and CMB Interpretation}
	
	The T0-model also has far-reaching implications for cosmological measurements, as detailed in \cite{pascher_messdifferenzen_2025}. In particular, the distance measures in the T0-model differ from those in the \(\Lambda\)CDM model:
	
	- \textbf{Physical Distance \( d \):}
	\[
	d = \frac{c \ln(1 + z)}{H_0},
	\]
	compared to \(\Lambda\)CDM:
	\[
	d = \frac{c}{H_0} \int_0^z \frac{dz'}{\sqrt{\Omega_m (1 + z')^3 + \Omega_\Lambda}}.
	\]
	
	- \textbf{Luminosity Distance \( d_L \):}
	\[
	d_L = \frac{c}{H_0} \ln(1 + z) (1 + z),
	\]
	compared to \(\Lambda\)CDM:
	\[
	d_L = (1 + z) \cdot \frac{c}{H_0} \int_0^z \frac{dz'}{\sqrt{\Omega_m (1 + z')^3 + \Omega_\Lambda}}.
	\]
	
	- \textbf{Angular Diameter Distance \( d_A \):}
	\[
	d_A = \frac{c \ln(1 + z)}{H_0 (1 + z)},
	\]
	compared to \(\Lambda\)CDM:
	\[
	d_A = \frac{d}{1 + z}.
	\]
	
	Additionally, the CMB temperature-redshift relation in the T0-model is modified due to the dynamics of \( \Tfield \):
	
	\begin{equation}
		T(z) = T_0 (1 + z) (1 + \beta \ln(1 + z)),
	\end{equation}
	
	with \( \beta \approx 0.008 \), contrasting with the \(\Lambda\)CDM prediction \( T(z) = T_0 (1 + z) \). These differences lead to significant deviations at high redshifts, particularly for the cosmic microwave background (CMB) at \( z = 1100 \). In the T0-model, the angular diameter distance \( d_A \) is approximately twice as large as in the \(\Lambda\)CDM model (28.9 Mpc vs. 13.5 Mpc), resulting in an angular size of structures of about \( 5.8^\circ \) in the T0-model compared to \( 1^\circ \) in the \(\Lambda\)CDM model. These dramatic differences provide an opportunity to experimentally test the models, as further elaborated in \cite{pascher_messdifferenzen_2025}.
	
	% Rest des Dokuments bleibt unverändert
	
	\begin{thebibliography}{99}
		\bibitem{pascher_messdifferenzen_2025} Pascher, J. (2025). \href{\repobase/pdf/Deutsch/Analyse der Messdifferenzen zwischen dem T0-Modell und dem Standardmodell.pdf}{Compensatory and Additive Effects: An Analysis of Measurement Differences Between the T0-Model and the $\Lambda$CDM Standard Model}. April 2, 2025.
		% Weitere Einträge wie im Original
	\end{thebibliography}
	
\end{document}