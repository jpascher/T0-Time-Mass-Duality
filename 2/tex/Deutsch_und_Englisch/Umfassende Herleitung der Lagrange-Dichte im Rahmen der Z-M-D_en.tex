\documentclass[a4paper,12pt]{article}
\usepackage[utf8]{inputenc}
\usepackage[T1]{fontenc}
\usepackage[english]{babel}
\usepackage{lmodern}
\usepackage{amsmath}
\usepackage{amssymb}
\usepackage{amsthm}
\usepackage{physics}
\usepackage{bm}
\usepackage{csquotes}
\usepackage{hyperref}
\usepackage{xcolor}
\usepackage{geometry}
\usepackage{booktabs}
\usepackage{array}
\usepackage{tabularx}
\usepackage{fancyhdr}
\usepackage{braket}
\usepackage{tcolorbox}
\usepackage{graphicx}
\usepackage{mathtools}
\usepackage{tikz}
\usepackage{float}
\usepackage[backend=biber,style=numeric,sorting=none]{biblatex}

\geometry{a4paper, margin=2.5cm}
\hypersetup{
	colorlinks=true,
	linkcolor=blue,
	filecolor=magenta,      
	urlcolor=blue,
	pdftitle={Comprehensive Derivation of the Lagrangian Density in the Framework of Time-Mass Duality},
	pdfauthor={Johann Pascher},
	pdfcreator={LaTeX}
}

\newtheorem{theorem}{Theorem}[section]
\newtheorem{lemma}[theorem]{Lemma}
\newtheorem{proposition}[theorem]{Proposition}
\newtheorem{corollary}[theorem]{Corollary}
\newtheorem{definition}{Definition}[section]
\newtheorem{remark}{Remark}[section]

\newcommand{\imunit}{\mathrm{i}}
\newcommand{\realm}{\mathbb{R}}
\newcommand{\comp}{\mathbb{C}}
\newcommand{\intr}{T}
\newcommand{\BosonLagr}{\mathcal{L}_{\text{Gauge-T}}}
\newcommand{\FermionLagr}{\mathcal{L}_{\text{Fermion-T}}}
\newcommand{\HiggsLagr}{\mathcal{L}_{\text{Higgs-T}}}
\newcommand{\YukawaLagr}{\mathcal{L}_{\text{Yukawa-T}}}
\newcommand{\TotalLagr}{\mathcal{L}_{\text{Total-T}}}

\begin{document}
	
	\title{Comprehensive Derivation of the Lagrangian Density in the Framework of Time-Mass Duality}
	\author{Johann Pascher}
	\date{March 29, 2025}
	
	\maketitle
	
	\begin{abstract}
		This work presents a detailed derivation and mathematical formulation of the Lagrangian density within the framework of the time-mass duality theory. Starting from the foundational principles of this theory—the duality between a standard picture with time dilation and constant rest mass and an alternative picture with absolute time and variable mass—a consistent field theory is developed. The central innovation is the introduction of intrinsic time \( T = \hbar/mc^2 \) as a fundamental quantity directly linked to mass, governing the temporal evolution of all quantum systems. The resulting Lagrangian density includes modified terms for the Higgs field, fermions, gauge bosons, and their interactions, with the Higgs field playing a special role as a mediator between the two complementary descriptions. The work demonstrates that this reformulation is not only mathematically consistent but also leads to experimentally testable predictions that deviate from the Standard Model of particle physics.
	\end{abstract}
	
	\tableofcontents
	\newpage
	
	\section{Introduction}
	
	The time-mass duality theory represents an innovative approach that reexamines central concepts of modern physics in a new light. In contrast to the conventional view, where time is relative (time dilation) and rest mass is constant, this theory proposes an alternative, mathematically equivalent picture in which time remains absolute while mass varies. This duality necessitates a fundamental reformulation of established physical theories, particularly the Lagrangian density, which describes the dynamics of all fundamental fields and their interactions.
	
	\subsection{Fundamental Principles of Time-Mass Duality}
	
	The time-mass duality theory is based on the following fundamental principles:
	
	\begin{itemize}
		\item \textbf{Intrinsic Time:} For every particle with mass \( m \), a characteristic intrinsic time is defined as \( T = \frac{\hbar}{mc^2} \). This time scales inversely with mass and determines the temporal evolution of the quantum system.
		\item \textbf{Modified Time Derivative:} The conventional time derivative \( \frac{\partial}{\partial t} \) is replaced by a modified time derivative \( \frac{\partial}{\partial(t/T)} = T \frac{\partial}{\partial t} \), accounting for the system’s intrinsic timescale.
		\item \textbf{Duality of Descriptions:} There exist two equivalent descriptions of physical phenomena:
		\begin{enumerate}
			\item The \textbf{Standard Picture} with time dilation (\( t' = \gamma t \)) and constant rest mass (\( m_0 = \text{const.} \))
			\item The \textbf{Alternative Picture} with absolute time (\( T_0 = \text{const.} \)) and variable mass (\( m = \gamma m_0 \))
		\end{enumerate}
		\item \textbf{Higgs Mediation:} The Higgs field plays a central role as a mediator between both descriptions, defining both the rest mass and the intrinsic timescale.
	\end{itemize}
	
	\subsection{Objective and Structure of This Work}
	
	The primary objective of this work is to develop a complete and consistent mathematical formulation of the Lagrangian density within the framework of the time-mass duality theory. This formulation must encompass all fundamental interactions and fields of the Standard Model while accurately reflecting the novel aspects of the duality theory.
	
	The work is structured as follows:
	
	\begin{itemize}
		\item Section 2 mathematically derives intrinsic time and discusses its physical significance.
		\item Section 3 addresses the transformation of fundamental field equations according to the time-mass duality principle.
		\item Section 4 develops the modified Lagrangian density for scalar fields, particularly the Higgs field.
		\item Section 5 deals with the reformulation of the Lagrangian density for fermions and their coupling to the Higgs field.
		\item Section 6 addresses the modified Lagrangian density for gauge bosons.
		\item Section 7 compiles the complete total Lagrangian density and verifies its consistency.
		\item Section 8 examines the experimental consequences and predictions of this theory.
		\item Section 9 summarizes the findings and provides an outlook on future research.
	\end{itemize}
	
	\section{Mathematical Derivation of Intrinsic Time}
	
	\subsection{Basic Concepts and Definitions}
	
	To establish intrinsic time as a fundamental quantity, we begin with the basic relationships from special relativity and quantum mechanics.
	
	\begin{definition}[Energy-Mass Equivalence]
		Special relativity postulates the equivalence of mass and energy according to the famous formula
		\begin{equation}
			E = mc^2
		\end{equation}
		where \( E \) is the energy, \( m \) is the mass, and \( c \) is the speed of light in a vacuum.
	\end{definition}
	
	\begin{definition}[Energy-Frequency Relationship]
		Quantum mechanics relates the energy of a quantum system to its frequency through
		\begin{equation}
			E = h\nu = \frac{h}{T}
		\end{equation}
		where \( h \) is Planck’s constant, \( \nu \) is the frequency, and \( T \) is the period.
	\end{definition}
	
	\subsection{Derivation of Intrinsic Time}
	
	From these two fundamental relationships, we can derive the intrinsic time of a particle with mass \( m \).
	
	\begin{theorem}[Intrinsic Time]
		For a particle with mass \( m \), the intrinsic time \( T \) is defined as
		\begin{equation}
			T = \frac{\hbar}{mc^2}
		\end{equation}
		where \( \hbar = h/2\pi \) is the reduced Planck’s constant.
	\end{theorem}
	
	\begin{proof}
		We equate the energy-mass equivalence and the energy-frequency relationship:
		\begin{align}
			E &= mc^2 \\
			E &= \frac{h}{T}
		\end{align}
		
		By setting them equal, we obtain:
		\begin{align}
			mc^2 &= \frac{h}{T} \\
		\end{align}
		
		Solving for \( T \) yields:
		\begin{align}
			T &= \frac{h}{mc^2} = \frac{\hbar \cdot 2\pi}{mc^2} = \frac{\hbar}{mc^2} \cdot 2\pi
		\end{align}
		
		For the fundamental period of the quantum system, we use \( T = \frac{\hbar}{mc^2} \), which corresponds to the reduced Compton wavelength of the particle divided by the speed of light.
	\end{proof}
	
	\subsection{Physical Interpretation of Intrinsic Time}
	
	The intrinsic time \( T = \frac{\hbar}{mc^2} \) can be interpreted as a fundamental timescale associated with a particle of mass \( m \). It represents the characteristic time during which significant quantum mechanical changes in the particle’s state can occur.
	
	\begin{remark}
		For an electron with \( m_e \approx 9.1 \times 10^{-31} \) kg, the intrinsic time is \( T_e \approx 8.1 \times 10^{-21} \) s, corresponding to the electron’s Compton time.
	\end{remark}
	
	\begin{proposition}[Scaling of Intrinsic Time]
		The intrinsic times of two particles with masses \( m_1 \) and \( m_2 \) are inversely proportional to their masses:
		\begin{equation}
			\frac{T_1}{T_2} = \frac{m_2}{m_1}
		\end{equation}
	\end{proposition}
	
	\begin{proof}
		From the definition of intrinsic time, it follows directly:
		\begin{align}
			\frac{T_1}{T_2} = \frac{\hbar/(m_1 c^2)}{\hbar/(m_2 c^2)} = \frac{m_2}{m_1}
		\end{align}
	\end{proof}
	
	\subsection{Relation to the Fine Structure Constant}
	
	A noteworthy connection exists between intrinsic time and the fine structure constant \( \alpha \), which describes the strength of the electromagnetic interaction.
	
	\begin{theorem}[Intrinsic Time and Fine Structure Constant]
		The intrinsic time can be expressed in terms of the fine structure constant \( \alpha \) as:
		\begin{equation}
			T = \frac{\hbar^2 \cdot 4\pi \varepsilon_0 c}{m c^2 \cdot e^2} \cdot \alpha
		\end{equation}
		where \( e \) is the elementary charge and \( \varepsilon_0 \) is the electric permittivity of free space.
	\end{theorem}
	
	\begin{proof}
		The fine structure constant is defined as:
		\begin{equation}
			\alpha = \frac{e^2}{4\pi \varepsilon_0 \hbar c} \approx \frac{1}{137.036}
		\end{equation}
		
		We multiply and divide the intrinsic time by appropriate factors:
		\begin{align}
			T &= \frac{\hbar}{m c^2} \\
			&= \frac{\hbar}{m c^2} \cdot \frac{4\pi \varepsilon_0 \hbar c}{e^2} \cdot \frac{e^2}{4\pi \varepsilon_0 \hbar c} \\
			&= \frac{\hbar^2 \cdot 4\pi \varepsilon_0 c}{m c^2 \cdot e^2} \cdot \alpha
		\end{align}
	\end{proof}
	
	\begin{corollary}[Natural Units]
		In a system of natural units where \( \hbar = c = 1 \), the relationship simplifies to:
		\begin{equation}
			T = \frac{\alpha}{m} \cdot \frac{4\pi \varepsilon_0}{e^2}
		\end{equation}
		
		If additionally \( \alpha = 1 \) and \( e^2/(4\pi \varepsilon_0) = 1 \), the simple relationship emerges:
		\begin{equation}
			T = \frac{1}{m}
		\end{equation}
	\end{corollary}
	
	\section{Transformation of Field Equations}
	
	\subsection{Modified Time Derivative}
	
	The central innovation of the time-mass duality theory is the introduction of a modified time derivative that accounts for intrinsic time \( T \).
	
	\begin{definition}[Modified Covariant Derivative]
		The modified covariant derivative for a field \( \Psi \) is defined as:
		\begin{equation}
			D_\mu^T \Psi = T(x) D_\mu \Psi + \Psi \partial_\mu T(x)
		\end{equation}
		where \( D_\mu \) is the usual covariant derivative including gauge field interactions, and \( T(x) \) is the intrinsic time field.
	\end{definition}
	
	\begin{remark}
		For the time component, this reduces to:
		\begin{equation}
			\partial_{t/T} = T(x) \frac{\partial}{\partial t}
		\end{equation}
	\end{remark}
	
	\subsection{Transformation of the Schrödinger Equation}
	
	\begin{theorem}[Modified Schrödinger Equation]
		The Schrödinger equation in the time-mass duality theory becomes:
		\begin{equation}
			i\hbar T(x) \frac{\partial}{\partial t} \Psi + i\hbar \Psi \frac{\partial T(x)}{\partial t} = \hat{H} \Psi
		\end{equation}
		where \( T(x) = \frac{\hbar}{m(x) c^2} \) is the intrinsic time field, which can vary spatially and temporally.
	\end{theorem}
	
	\begin{proof}
		From the standard Schrödinger equation \( i\hbar \frac{\partial}{\partial t} \Psi = \hat{H} \Psi \), the time derivative is replaced by \( \frac{\partial}{\partial (t/T(x))} \):
		\begin{equation}
			i\hbar \frac{\partial}{\partial (t/T(x))} \Psi = \hat{H} \Psi
		\end{equation}
		With \( \frac{\partial}{\partial (t/T(x))} = T(x) \frac{\partial}{\partial t} + \Psi \frac{\partial T(x)}{\partial t} \), the given form follows.
	\end{proof}
	
	\section{Modified Lagrangian Density for the Higgs Field}
	
	\begin{theorem}[Consistent Higgs Lagrangian Density]
		The Lagrangian density for the Higgs field is:
		\begin{equation}
			\mathcal{L}_{\text{Higgs-T}} = (D_\mu^T \Phi)^\dagger (D_\mu^T \Phi) - \lambda (|\Phi|^2 - v^2)^2
		\end{equation}
		with:
		\begin{equation}
			D_\mu^T \Phi = T(x) (\partial_\mu + i g A_\mu) \Phi + \Phi \partial_\mu T(x)
		\end{equation}
	\end{theorem}
	
	\section{Modified Lagrangian Density for Fermions}
	
	\begin{theorem}[Consistent Fermion Lagrangian Density]
		The Lagrangian density for fermions is:
		\begin{equation}
			\mathcal{L}_{\text{Fermion-T}} = \bar{\psi} i \gamma^\mu D_\mu^T \psi - y \bar{\psi} \Phi \psi
		\end{equation}
		with:
		\begin{equation}
			D_\mu^T \psi = T(x) D_\mu \psi + \psi \partial_\mu T(x)
		\end{equation}
	\end{theorem}
	
	\section{Modified Lagrangian Density for Gauge Bosons}
	
	\begin{theorem}[Consistent Gauge Boson Lagrangian Density]
		The Lagrangian density for gauge bosons is:
		\begin{equation}
			\mathcal{L}_{\text{Gauge-T}} = -\frac{1}{4} T(x)^2 F_{\mu\nu} F^{\mu\nu}
		\end{equation}
		where \( F_{\mu\nu} = \partial_\mu A_\nu - \partial_\nu A_\mu + i g [A_\mu, A_\nu] \).
	\end{theorem}
	
	\section{Complete Total Lagrangian Density}
	
	\begin{theorem}[Complete Total Lagrangian Density]
		The total Lagrangian density of the time-mass duality theory is:
		\begin{equation}
			\mathcal{L}_{\text{Total-T}} = \mathcal{L}_{\text{Gauge-T}} + \mathcal{L}_{\text{Fermion-T}} + \mathcal{L}_{\text{Higgs-T}}
		\end{equation}
	\end{theorem}
	
	\section{Experimental Consequences and Predictions}
	
	The time-mass duality theory leads to several experimentally testable predictions that deviate from the Standard Model.
	
	\subsection{Modified Energy-Momentum Relation}
	
	\begin{theorem}[Modified Energy-Momentum Relation]
		The time-mass duality leads to a modified energy-momentum relation:
		\begin{equation}
			E^2 = (p c)^2 + (m c^2)^2 + \alpha \frac{\hbar c}{T}
		\end{equation}
		where \( \alpha \) is a dimension-dependent parameter.
	\end{theorem}
	
	\subsection{Mass-Dependent Quantum Coherence}
	
	\begin{theorem}[Mass-Dependent Coherence Times]
		The coherence times \( \tau_1 \) and \( \tau_2 \) of two otherwise identical quantum systems with masses \( m_1 \) and \( m_2 \) should follow the ratio:
		\begin{equation}
			\frac{\tau_1}{\tau_2} = \frac{m_2}{m_1}
		\end{equation}
	\end{theorem}
	
	\subsection{Modified Higgs Couplings}
	
	\begin{theorem}[Nonlinearity in the Mass Hierarchy]
		In the time-mass duality theory, Higgs couplings may exhibit slight nonlinearities:
		\begin{equation}
			g_H \propto m \left(1 + \delta \cdot \ln\left(\frac{m}{m_0}\right)\right)
		\end{equation}
		where \( \delta \) is a small correction and \( m_0 \) is a reference mass.
	\end{theorem}
	
	\subsection{Photon Energy Loss and Cosmological Consequences}
	
	\begin{theorem}[Photon Energy Decrease]
		Photons should experience a slight energy decrease according to
		\begin{equation}
			E(r) = E_0 e^{-\alpha r}
		\end{equation}
		where \( \alpha = \frac{H_0}{c} \) is the absorption coefficient and \( H_0 \) is the Hubble constant.
	\end{theorem}
	
	\subsection{Modified Gravitational Potential}
	
	\begin{theorem}[Modified Gravitational Potential]
		The gravitational potential in the T0 model is:
		\begin{equation}
			\Phi(r) = -\frac{G M}{r} + \kappa r
		\end{equation}
		where \( \kappa \approx 4.8 \times 10^{-7} \, \text{GeV/cm} \cdot \text{s}^{-2} \) emerges from the dynamics of \( T(x) \).
	\end{theorem}
	
	\begin{proof}
		From \( T(x) = \frac{\hbar}{m c^2} \), it follows:
		\begin{equation}
			\nabla T(x) = -\frac{\hbar}{m^2 c^2} \nabla m
		\end{equation}
		With \( m = m_0 (1 + \frac{\Phi(r)}{c^2}) \) and \( \Phi(r) = -\frac{G M}{r} + \kappa r \), the modification arises.
	\end{proof}
	
	\subsection{Entanglement Effects with Unequal Masses}
	
	\begin{theorem}[Mass-Dependent Entanglement Correlations]
		In the time-mass duality theory, correlations between entangled particles of different masses should exhibit a subtle mass dependence.
	\end{theorem}
	
	\section{Summary and Outlook}
	
	\subsection{Summary of Main Results}
	
	\begin{enumerate}
		\item \textbf{Intrinsic Time:} Introduction of \( T = \frac{\hbar}{m c^2} \) as a fundamental quantity.
		\item \textbf{Modified Time Derivative:} Replacement with \( D_\mu^T \).
		\item \textbf{Higgs Field as Mediator:} Dual role in mass and timescale.
		\item \textbf{Comprehensive Lagrangian Density:} Consistent formulation of all interactions.
		\item \textbf{Experimental Predictions:} Testable deviations from the Standard Model.
	\end{enumerate}
	
	\subsection{Philosophical Implications}
	
	\begin{enumerate}
		\item \textbf{Nature of Time:} Emergent or fundamental property?
		\item \textbf{Relative vs. Absolute Character:} Alternative to relativity.
		\item \textbf{Quantum Correlations:} Natural explanation for nonlocality.
		\item \textbf{Cosmology:} Alternative to expansion and dark matter.
	\end{enumerate}
	
	\subsection{Open Questions and Future Research Directions}
	
	\begin{enumerate}
		\item \textbf{Quantum Gravity:} Connection to the Planck scale.
		\item \textbf{Massless Particles:} Treatment of photons.
		\item \textbf{Experimental Tests:} Precision experiments.
		\item \textbf{Numerical Simulations:} Cosmological consequences.
		\item \textbf{Unification of Forces:} Integration of gravity.
	\end{enumerate}
	
\end{document}