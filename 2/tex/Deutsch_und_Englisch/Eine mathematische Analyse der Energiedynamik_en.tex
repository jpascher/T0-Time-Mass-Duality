\documentclass[a4paper,12pt]{article}
\usepackage[utf8]{inputenc}
\usepackage[T1]{fontenc}
\usepackage{lmodern}
\usepackage[english]{babel} % Changed from ngerman to english
\usepackage{amsmath, amssymb, amsthm, physics}
\usepackage{graphicx}
\usepackage{xcolor}
\usepackage{tikz}
\usepackage{setspace}
\usepackage{tcolorbox}
\usepackage{booktabs}
\usepackage{siunitx}

% Colored links in table of contents and document
\usepackage{hyperref}
\hypersetup{
	colorlinks=true,
	linkcolor=blue,
	filecolor=blue,
	citecolor=blue, 
	urlcolor=blue,
	bookmarks=true,
	bookmarksopen=true,
	pdftitle={Dark Energy in the T0-Model: A Mathematical Analysis of Energy Dynamics}, % Translated title
	pdfauthor={Johann Pascher},
}

% cleveref must be loaded after hyperref
\usepackage{cleveref}

% Theorem styles
\newtheorem{theorem}{Theorem}[section]
\newtheorem{lemma}[theorem]{Lemma}
\newtheorem{proposition}[theorem]{Proposition}
\newtheorem{corollary}[theorem]{Corollary} % Changed from "Korollar" to "Corollary"

\theoremstyle{definition}
\newtheorem{definition}[theorem]{Definition}
\newtheorem{example}[theorem]{Example} % Changed from "Beispiel" to "Example"

\theoremstyle{remark}
\newtheorem{remark}[theorem]{Remark} % Changed from "Bemerkung" to "Remark"
\renewcommand{\proofname}{Proof} % Changed from "Beweis" to "Proof"

% Custom commands
\newcommand{\Tfield}{T(x)} % Intrinsic time as a field
\newcommand{\DcovT}[1]{\Tfield D_\mu #1 + #1 \partial_\mu \Tfield}
\newcommand{\DhiggsT}{\Tfield (\partial_\mu + igA_\mu)\Phi + \Phi \partial_\mu \Tfield}

\begin{document}
	
	\title{Dark Energy in the T0-Model: \\A Mathematical Analysis of Energy Dynamics} % Translated title
	\author{Johann Pascher}
	\date{\today}
	\maketitle
	
	\begin{abstract}
		This work develops a detailed mathematical analysis of dark energy within the framework of the T0-model with absolute time and variable mass. In contrast to the \(\Lambda\)CDM standard model, dark energy is not considered a driving force of cosmic expansion but as a dynamic medium for energy exchange in a static universe. The document derives the corresponding field equations, characterizes energy transfer rates, analyzes the radial density profile of dark energy, and explains the observed redshift as a consequence of energy loss by photons. Finally, specific experimental tests are proposed to distinguish between this interpretation and the standard model.
	\end{abstract}
	
	\tableofcontents
	\newpage
	
	%======================= PART 1: FOUNDATIONS ========================
	\section{Introduction}
	
	The discovery of accelerated cosmic expansion through supernova observations in the late 1990s led to the introduction of dark energy as the dominant component of the universe. In the standard cosmological model (\(\Lambda\)CDM), dark energy is modeled as a cosmological constant (\(\Lambda\)) with negative pressure, accounting for approximately 68\% of the universe's energy content and driving accelerated expansion. This work pursues an alternative approach based on the T0-model, in which time is absolute and particle mass varies instead. Within this framework, dark energy is not viewed as a force driving expansion but as a medium for energy exchange interacting with matter and radiation. Cosmic redshift is explained not by spatial expansion but by energy loss of photons to dark energy. In the following, we will mathematically refine this approach, derive the necessary field equations, determine the energy density and distribution of dark energy, and analyze the implications for astronomical observations. Subsequently, we will explore experimental tests that could differentiate between the T0-model and the standard model.
	
	\section{Mathematical Foundations of the T0-Model}
	
	\subsection{Time-Mass Duality}
	
	The T0-model is based on time-mass duality, which postulates two equivalent descriptions of reality:
	
	\begin{enumerate}
		\item \textbf{Standard Picture}: Time dilation (\(t' = \gamma t\)) and constant rest mass (\(m_0 = \text{const.}\))
		\item \textbf{Alternative Picture (T0-Model)}: Absolute time (\(T_0 = \text{const.}\)) and variable mass (\(m = \gamma m_0\))
	\end{enumerate}
	
	The following transformation table applies between the two pictures:
	
	\begin{table}[h]
		\centering
		\begin{tabular}{|l|c|c|}
			\hline
			\textbf{Quantity} & \textbf{Standard Picture} & \textbf{T0-Model} \\
			\hline
			Time & \(t' = \gamma t\) & \(t = \text{const.}\) \\
			Mass & \(m = \text{const.}\) & \(m = \gamma m_0\) \\
			Intrinsic Time & \(T = \frac{\hbar}{mc^2}\) & \(T = \frac{\hbar}{\gamma m_0c^2} = \frac{T_0}{\gamma}\) \\
			Higgs Field & \(\Phi\) & \(\Phi_T = \gamma \Phi\) \\
			Fermion Field & \(\psi\) & \(\psi_T = \gamma^{1/2} \psi\) \\
			Gauge Field (spatial) & \(A_i\) & \(A_{T,i} = A_i\) \\
			Gauge Field (temporal) & \(A_0\) & \(A_{T,0} = \gamma A_0\) \\
			\hline
		\end{tabular}
		\caption{Transformation table between Standard Picture and T0-Model} % Translated caption
	\end{table}
	
	\subsection{Definition of Intrinsic Time}
	
	Central to the T0-model is the concept of intrinsic time:
	
	\begin{definition}[Intrinsic Time]
		For a particle with mass \(m\), the intrinsic time \(T\) is defined as:
		\begin{equation}
			T = \frac{\hbar}{mc^2}
		\end{equation}
		where \(\hbar\) is the reduced Planck constant and \(c\) is the speed of light.
	\end{definition}
	
	\begin{proof}
		The derivation follows from the equivalence of energy-mass and energy-frequency relationships:
		\begin{align}
			E &= mc^2 \\
			E &= \frac{h}{T} = \frac{\hbar \cdot 2\pi}{T}
		\end{align}
		
		Equating these yields:
		\begin{align}
			mc^2 &= \frac{\hbar \cdot 2\pi}{T} \\
		\end{align}
		
		Solving for \(T\):
		\begin{align}
			T &= \frac{\hbar}{mc^2} \cdot 2\pi
		\end{align}
		
		For the fundamental period of the quantum mechanical system, we use \(T = \frac{\hbar}{mc^2}\), corresponding to the reduced Compton wavelength of the particle divided by the speed of light.
	\end{proof}
	
	\begin{corollary}[Intrinsic Time as a Scalar Field]
		In field theory, intrinsic time is treated as a scalar field \(T(x)\), directly linked to the Higgs field:
		\begin{equation}
			T(x) = \frac{\hbar}{y\langle\Phi\rangle c^2}
		\end{equation}
		where \(y\) is the Yukawa coupling constant and \(\langle\Phi\rangle\) is the vacuum expectation value of the Higgs field.
	\end{corollary}
	
	\subsection{Modified Derivative Operators}
	
	\begin{definition}[Modified Time Derivative]
		The modified time derivative is defined as:
		\begin{equation}
			\partial_{t/T} = \frac{\partial}{\partial(t/T)} = T\frac{\partial}{\partial t}
		\end{equation}
	\end{definition}
	
	\begin{definition}[Field-Theoretical Modified Covariant Derivative]
		For an arbitrary field \(\Psi\), we define the modified covariant derivative as:
		\begin{equation}
			D_{T,\mu}\Psi = \Tfield D_\mu \Psi + \Psi \partial_\mu \Tfield
		\end{equation}
		where \(D_\mu\) is the ordinary covariant derivative corresponding to the gauge symmetry of the field \(\Psi\).
	\end{definition}
	
	\begin{definition}[Modified Covariant Derivative for the Higgs Field]
		\begin{equation}
			D_{T,\mu}\Phi = \DhiggsT
		\end{equation}
	\end{definition}
	
	%======================= PART 2: FIELD EQUATIONS ========================
	\section{Modified Field Equations for Dark Energy}
	
	\subsection{Modified Lagrangian Density for the T0-Model}
	
	The complete Lagrangian density in the T0-model is composed of:
	
	\begin{equation}
		\mathcal{L}_{\text{Total}} = \mathcal{L}_{\text{Boson}} + \mathcal{L}_{\text{Fermion}} + \mathcal{L}_{\text{Higgs-T}} + \mathcal{L}_{\text{DE}}
	\end{equation}
	
	With the following components:
	
	\begin{equation}
		\mathcal{L}_{\text{Boson}} = -\frac{1}{4} \Tfield^2 F_{\mu\nu}F^{\mu\nu}
	\end{equation}
	
	\begin{equation}
		\mathcal{L}_{\text{Fermion}} = \bar{\psi}i\gamma^\mu \DcovT{\psi} - y\bar{\psi}\Phi\psi
	\end{equation}
	
	\begin{equation}
		\mathcal{L}_{\text{Higgs-T}} = (D_{T,\mu}\Phi)^\dagger (D_{T,\mu}\Phi) - \lambda(|\Phi|^2 - v^2)^2
	\end{equation}
	
	\begin{equation}
		\mathcal{L}_{\text{DE}} = -\frac{1}{2}\partial_\mu \phi_{\text{DE}} \partial^\mu \phi_{\text{DE}} - V(\phi_{\text{DE}}) - \frac{\beta}{M_{\text{Pl}}} \phi_{\text{DE}} T^{\mu}_{\mu} - \frac{1}{2}\xi \phi_{\text{DE}}^2 R
	\end{equation}
	
	where:
	\begin{itemize}
		\item \(F_{\mu\nu} = \partial_\mu A_\nu - \partial_\nu A_\mu + ig[A_\mu, A_\nu]\) is the usual field strength tensor
		\item \(\phi_{\text{DE}}\) represents the dark energy field
		\item \(V(\phi_{\text{DE}})\) is the self-interaction potential of the field
		\item \(T^{\mu}_{\mu}\) is the trace of the energy-momentum tensor of matter and radiation
		\item \(R\) is the spacetime curvature
		\item \(\beta\) and \(\xi\) are coupling constants
		\item \(M_{\text{Pl}}\) is the Planck mass
	\end{itemize}
	
	\subsection{Dark Energy as a Dynamic Field}
	
	Dark energy is modeled in the T0-model as a scalar field interacting with matter and radiation. For a stable equilibrium in a static universe, we choose the self-interaction potential:
	
	\begin{equation}
		V(\phi_{\text{DE}}) = \frac{1}{2}m_{\phi}^2\phi_{\text{DE}}^2 + \lambda \phi_{\text{DE}}^4
	\end{equation}
	
	The field equations are derived from the Euler-Lagrange equation:
	
	\begin{equation}
		\Box\phi_{\text{DE}} - \frac{dV}{d\phi_{\text{DE}}} - \frac{\beta}{M_{\text{Pl}}}T^{\mu}_{\mu} - \xi \phi_{\text{DE}} R = 0
	\end{equation}
	
	For a massless field (\(m_{\phi} \approx 0\)) and negligible curvature (\(\xi R \approx 0\)) in a spherically symmetric system, this simplifies to:
	
	\begin{equation}
		\frac{1}{r^2}\frac{d}{dr}\left(r^2\frac{d\phi_{\text{DE}}}{dr}\right) = 4\lambda\phi_{\text{DE}}^3 + \frac{\beta}{M_{\text{Pl}}}T^{\mu}_{\mu}
	\end{equation}
	
	\subsection{Energy Density Profile of Dark Energy}
	
	For large distances \(r\), where \(T^{\mu}_{\mu} \approx 0\) (negligible matter density), we obtain with the ansatz \(\phi_{\text{DE}}(r) \propto r^{-\alpha}\) through coefficient comparison \(\alpha = 1/2\), thus:
	
	\begin{equation}
		\phi_{\text{DE}}(r) \approx \left(\frac{1}{8\lambda}\right)^{1/3} r^{-1/2} \quad \text{for } r \gg r_0
	\end{equation}
	
	The energy density of dark energy is then:
	
	\begin{equation}
		\rho_{\text{DE}}(r) \approx \frac{1}{2}\left(\frac{d\phi_{\text{DE}}}{dr}\right)^2 + \frac{1}{2}m_{\phi}^2\phi_{\text{DE}}^2 + \lambda\phi_{\text{DE}}^4 \approx \frac{\kappa}{r^2}
	\end{equation}
	
	with \(\kappa \propto \lambda^{-2/3}\). This \(1/r^2\)-profile is consistent with flat rotation curves in galaxies.
	
	\subsection{Emergent Gravitation from the Intrinsic Time Field}
	
	\begin{theorem}[Gravitational Emergence]
		In the T0-model, gravitational effects emerge from spatial and temporal gradients of the intrinsic time field \(\Tfield\), providing a natural connection between quantum physics and gravitational phenomena:
		\begin{equation}
			\nabla \Tfield = \nabla \left(\frac{\hbar}{mc^2}\right) = -\frac{\hbar}{m^2c^2}\nabla m \sim \nabla \Phi_g
		\end{equation}
		where \(\Phi_g\) is the gravitational potential.
	\end{theorem}
	
	\begin{proof}
		In regions with gravitational potential \(\Phi_g\), the effective mass varies as:
		\begin{equation}
			m(\vec{r}) = m_0\left(1 + \frac{\Phi_g(\vec{r})}{c^2}\right)
		\end{equation}
		
		Thus:
		\begin{equation}
			\nabla m = m_0 \nabla\left(\frac{\Phi_g}{c^2}\right) = \frac{m_0}{c^2}\nabla\Phi_g
		\end{equation}
		
		Substituting into the gradient of the intrinsic time field:
		\begin{equation}
			\nabla \Tfield = -\frac{\hbar}{m^2c^2}\cdot\frac{m_0}{c^2}\nabla\Phi_g = -\frac{\hbar m_0}{m^2c^4}\nabla\Phi_g
		\end{equation}
		
		For weak fields where \(m \approx m_0\):
		\begin{equation}
			\nabla \Tfield \approx -\frac{\hbar}{m_0c^4}\nabla\Phi_g
		\end{equation}
		
		This establishes a direct proportionality between gradients of the intrinsic time field and gradients of the gravitational potential.
	\end{proof}
	
	The modified Poisson equation in the T0-model is:
	
	\begin{equation}
		\nabla^2 \Phi = 4\pi G \rho + \kappa^2
	\end{equation}
	
	This can be reinterpreted as a consequence of intrinsic time field dynamics.
	
	%======================= PART 3: ENERGY EXCHANGE ========================
	\section{Energy Exchange and Redshift}
	
	\subsection{Energy Loss of Photons}
	
	A central aspect of the T0-model is the interpretation of cosmic redshift as a result of photon energy loss to dark energy, not spatial expansion.
	
	The energy change of a photon moving through the dark energy field is described by:
	
	\begin{equation}
		\frac{dE_{\gamma}}{dx} = -\alpha E_{\gamma}
	\end{equation}
	
	where \(\alpha\) is the absorption rate. This equation has the solution:
	
	\begin{equation}
		E_{\gamma}(x) = E_{\gamma,0} e^{-\alpha x}
	\end{equation}
	
	The redshift \(z\) is defined as:
	
	\begin{equation}
		1 + z = \frac{E_0}{E} = \frac{\lambda_{\text{obs}}}{\lambda_{\text{emit}}} = e^{\alpha d}
	\end{equation}
	
	To ensure consistency with the observed Hubble relation \(z \approx H_0 d/c\) for small \(z\), it must hold:
	
	\begin{equation}
		\alpha = \frac{H_0}{c} \approx 2.3 \times 10^{-28} \text{ m}^{-1}
	\end{equation}
	
	Here, it becomes clear that the Hubble constant \(H_0\) in the T0-model has a fundamentally different meaning: it is not a parameter of cosmic expansion but characterizes the rate at which photons lose energy to the dark energy field. The numerical value of \(H_0 \approx 70 \text{ km/s/Mpc}\) remains the same, but its physical interpretation changes fundamentally.
	
	In natural units (\(\hbar = c = G = 1\)), the absorption rate \(\alpha\) and thus the Hubble constant can be related to fundamental parameters:
	
	\begin{equation}
		\alpha = \frac{H_0}{c} = \frac{\lambda_h^2 v}{L_T}
	\end{equation}
	
	where \(\lambda_h\) is the Higgs self-coupling, \(v\) is the vacuum expectation value of the Higgs field, and \(L_T\) is a characteristic cosmic length scale. Converted to SI units:
	
	\begin{equation}
		H_0 = \alpha \cdot c = \frac{\lambda_h^2 v c^3}{L_T} \approx 70 \frac{\text{km}}{\text{s} \cdot \text{Mpc}}
	\end{equation}
	
	This relation implies that the Hubble constant is directly linked to properties of the Higgs field. With the known value \(v \approx 246 \text{ GeV}\) and the estimated Higgs self-coupling \(\lambda_h \approx 0.13\), the characteristic length scale \(L_T\) can be determined:
	
	\begin{equation}
		L_T \approx \frac{\lambda_h^2 v c^3}{H_0} \approx 4.8 \times 10^{26} \text{ m} \approx 15.6 \text{ Gpc}
	\end{equation}
	
	This length scale corresponds approximately to the radius of the observable universe, underscoring the fundamental nature of the Hubble constant in the T0-model.
	
	More profoundly, we can attempt to express the Hubble constant as a dimensionless ratio of fundamental natural constants. In the T0-model, the following relation can be derived:
	
	\begin{equation}
		\frac{H_0}{c} \approx \lambda_h^2 \cdot \frac{v}{M_{Pl}} \cdot \frac{1}{N}
	\end{equation}
	
	where \(M_{Pl} = \sqrt{\frac{\hbar c}{G}} \approx 1.22 \times 10^{19} \text{ GeV/c}^2\) is the Planck mass, and \(N \approx 10^{61}\) is a dimensionless number related to the ratio between the characteristic length scale \(L_T\) and the Planck length \(l_{Pl} = \sqrt{\frac{\hbar G}{c^3}} \approx 1.62 \times 10^{-35} \text{ m}\). Explicitly:
	
	\begin{equation}
		N \approx \frac{L_T}{l_{Pl}} \approx \frac{4.8 \times 10^{26} \text{ m}}{1.62 \times 10^{-35} \text{ m}} \approx 3 \times 10^{61}
	\end{equation}
	
	Rearranged, this yields:
	
	\begin{equation}
		H_0 \approx c \cdot \lambda_h^2 \cdot \frac{v}{M_{Pl}} \cdot \frac{1}{N}
	\end{equation}
	
	With known values: \(v \approx 246 \text{ GeV/c}^2\), \(\lambda_h \approx 0.13\), we obtain:
	
	\begin{equation}
		H_0 \approx 3 \times 10^8 \text{ m/s} \cdot (0.13)^2 \cdot \frac{246 \text{ GeV/c}^2}{1.22 \times 10^{19} \text{ GeV/c}^2} \cdot \frac{1}{3 \times 10^{61}} \approx 70 \frac{\text{km}}{\text{s} \cdot \text{Mpc}}
	\end{equation}
	
	This dimensionless form \(\frac{H_0}{c} \approx \lambda_h^2 \cdot \frac{v}{M_{Pl}} \cdot \frac{1}{N}\) shows that the Hubble constant in the T0-model can be interpreted as a product of three key dimensionless ratios:
	1. \(\lambda_h^2\): Square of the Higgs self-coupling
	2. \(\frac{v}{M_{Pl}}\): Ratio between electroweak and Planck scales (\(\approx 10^{-17}\))
	3. \(\frac{1}{N}\): Inverse of the ratio between cosmic and Planck length (\(\approx 10^{-61}\))
	
	This representation could hint at deeper connections between particle physics, gravitation, and cosmology within the T0-model framework.
	
	Particularly elegant is the more compact formulation:
	
	\begin{equation}
		\frac{H_0 \cdot t_{Pl}}{2\pi} \approx \lambda_h^2 \cdot \left(\frac{v}{M_{Pl}}\right)^2
	\end{equation}
	
	where \(t_{Pl} = \sqrt{\frac{\hbar G}{c^5}} \approx 5.39 \times 10^{-44} \text{ s}\) is the Planck time. This form connects the Hubble constant (as a frequency \(H_0\)) directly to the most fundamental timescale in physics (Planck time) and expresses this ratio as a function of the squared electroweak-gravitational hierarchy ratio, modified by the Higgs self-coupling. This extremely compact representation might suggest a deeper universal relationship encompassing both cosmological evolution and particle physics.
	
	\subsection{Modified Energy-Momentum Relation}
	
	\begin{theorem}[Modified Energy-Momentum Relation]
		The modified energy-momentum relation in the T0-model is:
		\begin{equation}
			E^2 = (pc)^2 + (mc^2)^2 + \alpha_E\frac{\hbar c}{T}
		\end{equation}
		where \(\alpha_E\) is a parameter calculable from the theory.
	\end{theorem}
	
	This modification leads to a wavelength dependence of redshift:
	
	\begin{theorem}[Wavelength-Dependent Redshift]
		The cosmic redshift in the T0-model exhibits a weak wavelength dependence:
		\begin{equation}
			z(\lambda) = z_0 \cdot (1 + \beta\ln(\lambda/\lambda_0))
		\end{equation}
		with \(\beta = 0.008 \pm 0.003\).
	\end{theorem}
	
	\subsection{Energy Balance Equation}
	
	In a static universe with constant total energy, the energy balance must be considered:
	
	\begin{equation}
		\rho_{\text{total}} = \rho_{\text{matter}} + \rho_{\gamma} + \rho_{\text{DE}} = \text{const.}
	\end{equation}
	
	The balance equations for the temporal evolution of energy densities are:
	
	\begin{align}
		\frac{d\rho_{\text{matter}}}{dt} &= -\alpha_{m} c \rho_{\text{matter}} \\
		\frac{d\rho_{\gamma}}{dt} &= -\alpha_{\gamma} c \rho_{\gamma} \\
		\frac{d\rho_{\text{DE}}}{dt} &= \alpha_{m} c \rho_{\text{matter}} + \alpha_{\gamma} c \rho_{\gamma}
	\end{align}
	
	Assuming \(\alpha_{\gamma} = \alpha_{m} = \alpha\) (equal transfer rates for all energy forms), we obtain:
	
	\begin{align}
		\rho_{\text{matter}}(t) &= \rho_{\text{matter},0} e^{-\alpha c t} \\
		\rho_{\gamma}(t) &= \rho_{\gamma,0} e^{-\alpha c t} \\
		\rho_{\text{DE}}(t) &= \rho_{\text{DE},0} + (\rho_{\text{matter},0} + \rho_{\gamma,0})(1 - e^{-\alpha c t})
	\end{align}
	
	For large times (\(t \gg (\alpha c)^{-1}\)), the universe approaches a state where all energy is in the form of dark energy:
	
	\begin{equation}
		\lim_{t \rightarrow \infty} \rho_{\text{DE}}(t) = \rho_{\text{total}} = \rho_{\text{DE},0} + \rho_{\text{matter},0} + \rho_{\gamma,0}
	\end{equation}
	
	\section{Quantitative Determination of Parameters}
	
	\subsection{Derivation of Key Parameters in Natural Units}
	
	In natural units (\(\hbar = c = G = 1\)), the parameters take simpler forms that reveal fundamental relationships:
	
	\begin{theorem}[Parameters in Natural Units]
		The key parameters of the T0-model in natural units are:
		\begin{align}
			\kappa &= \beta \frac{y v}{r_g} \\
			\alpha &= \frac{\lambda_h^2 v}{L_T} \\
			\beta &= \frac{\lambda_h^2 v^2}{4\pi^2 \lambda_0 \alpha_0}
		\end{align}
		where \(v\) is the vacuum expectation value of the Higgs field, \(\lambda_h\) is the Higgs self-coupling, \(y\) is the Yukawa coupling, \(r_g\) is a galactic length scale, \(L_T \approx 10^{26} \text{ m}\) is a cosmic length scale, \(\lambda_0\) is a reference wavelength, and \(\alpha_0\) is the baseline redshift parameter.
	\end{theorem}
	
	Conversion to SI units:
	
	\begin{align}
		\alpha_{\text{SI}} &= \frac{\lambda_h^2 v c^2}{L_T} \approx 2.3 \times 10^{-28} \text{ m}^{-1} \\
		\beta_{\text{SI}} &= \frac{\lambda_h^2 v^2 c}{4\pi^2 \lambda_0 \alpha_0} \approx 0.008 \\
		\kappa_{\text{SI}} &= \beta \frac{y v c^2}{r_g^2} \approx 4.8 \times 10^{-11} \text{ m/s}^2
	\end{align}
	
	\subsection{Modified Gravitational Potential}
	
	\begin{theorem}[Modified Gravitational Potential]
		The modified gravitational potential in the T0-model is:
		\begin{equation}
			\Phi(r) = -\frac{GM}{r} + \kappa r
		\end{equation}
		where \(\kappa\) is a parameter derived from the theory as:
		\begin{equation}
			\kappa = \beta \frac{y v c^2}{r_g^2} \approx 4.8 \times 10^{-11} \text{ m/s}^2
		\end{equation}
		with \(r_g = \sqrt{\frac{GM}{a_0}}\) as a characteristic galactic length scale and \(a_0 \approx 1.2 \times 10^{-10} \text{ m/s}^2\) as a typical acceleration scale in galaxies.
	\end{theorem}
	
	\subsection{Coupling Constant to Matter}
	
	The dimensionless coupling constant \(\beta\), describing the interaction between dark energy and matter, can be estimated from galaxy rotation curve analysis:
	
	\begin{equation}
		\beta \approx 10^{-3}
	\end{equation}
	
	This value is small enough to pass local gravity tests but large enough to explain cosmological effects.
	
	\subsection{Self-Interaction of the Dark Energy Field}
	
	The self-interaction constant \(\lambda\) determines the dark energy density profile. From the relation \(\kappa \propto \lambda^{-2/3}\) and the observed value of \(\kappa\), we can estimate \(\lambda\):
	
	\begin{equation}
		\lambda \approx 10^{-120}
	\end{equation}
	
	This extremely small self-interaction poses a challenge for the model, similar to the hierarchy problem in the standard model.
	
	%======================= PART 4: FEYNMAN RULES ========================
	\section{Modified Feynman Rules}
	
	The Feynman rules in the T0-model are adapted as follows:
	
	\begin{enumerate}
		\item \textbf{Fermion Propagator:}
		\begin{equation}
			S_F(p) = \frac{i}{\Tfield p_0 \gamma^0 + \gamma^i p_i - m + i\epsilon}
		\end{equation}
		
		\item \textbf{Boson Propagator:}
		\begin{equation}
			D_F(p) = \frac{-i}{(\Tfield p_0)^2 - \vec{p}^2 - m^2 + i\epsilon}
		\end{equation}
		
		\item \textbf{Fermion-Boson Vertex:}
		\begin{equation}
			-ig\gamma^\mu \quad \text{with} \quad \gamma^0 \rightarrow \Tfield \gamma^0
		\end{equation}
		
		\item \textbf{Integration Measure:}
		\begin{equation}
			\int \frac{d^4p}{(2\pi)^4} \rightarrow \int \frac{dp_0 d^3p}{\Tfield (2\pi)^4}
		\end{equation}
	\end{enumerate}
	
	The Ward-Takahashi identities take a modified form in the T0-model:
	
	\begin{equation}
		\Tfield q_\mu \Gamma^\mu(p',p) = S^{-1}(p') - S^{-1}(p)
	\end{equation}
	
	where \(\Gamma^\mu\) is the vertex function, \(S\) is the fermion propagator, and \(q = p' - p\). The factor \(\Tfield\) arises due to the modified time derivative.
	
	\section{Dark Energy and Cosmological Observations}
	
	\subsection{Type Ia Supernovae and Cosmic Acceleration}
	
	In the T0-model, photons lose energy to the dark energy field on their journey through the universe, increasing their wavelength (redshift) and decreasing their intensity. This implies that the standard interpretation of supernova data, used to determine the Hubble constant, is based on the \(\Lambda\)CDM model, where accelerated expansion is explained by dark energy with negative pressure. In contrast, the T0-model offers an alternative explanation without cosmic expansion.
	
	The brightness-redshift relationship is described by:
	
	\begin{equation}
		m - M = 5 \log_{10}(d_L) + 25
	\end{equation}
	
	with the luminosity distance:
	
	\begin{equation}
		d_L = \frac{c}{H_0} \ln(1+z) (1+z)
	\end{equation}
	
	as opposed to the standard formula:
	
	\begin{equation}
		d_L^{\Lambda CDM} = \frac{c}{H_0} \int_0^z \frac{dz'}{\sqrt{\Omega_m(1+z')^3 + \Omega_{\Lambda}}}
	\end{equation}
	
	Both formulas can fit the observed data equally well but with fundamentally different physical interpretations. In the T0-model, the Hubble constant \(H_0\) is not an expansion rate but a measure of the energy absorption rate \(\alpha = H_0/c\). The observed tension between different measurements of \(H_0\) (the so-called "Hubble tension problem") could be understood in the T0-model as a result of varying absorption rates in different cosmic environments.
	
	\subsection{Energy Density Parameters in the T0-Model}
	
	In the standard cosmological model (\(\Lambda\)CDM), the universe's energy content is typically expressed through dimensionless density parameters \(\Omega_i = \rho_i/\rho_{\text{crit}}\), where \(\rho_{\text{crit}} = 3H_0^2/8\pi G\) is the critical density. Current measurements yield approximately \(\Omega_{\Lambda} \approx 0.68\) for dark energy, \(\Omega_m \approx 0.31\) for matter (including dark matter), and \(\Omega_r \approx 10^{-4}\) for radiation.
	
	In the T0-model, the dark energy contribution must be reinterpreted. Since the universe is assumed static here, dark energy does not correspond to a homogeneous background density (\(\rho_{\Lambda} = \text{const.}\)) but to an inhomogeneous field with \(\rho_{DE}(r) \approx \kappa/r^2\). The effective density parameter for dark energy can be estimated as a spatial average of this distribution:
	
	\begin{equation}
		\Omega_{DE}^{\text{eff}} = \frac{\langle\rho_{DE}(r)\rangle}{\rho_{\text{crit}}} \approx \frac{3\kappa}{R_U H_0^2} \approx 0.68
	\end{equation}
	
	where \(R_U \approx c/H_0\) is the radius of the observable universe. This value numerically aligns with the \(\Omega_{\Lambda}\) of the standard model but carries a fundamentally different physical meaning.
	
	Interestingly, in the T0-model, the temporal evolution of energy density fractions can be calculated. Using the temporal evolution equations from Section 4.3:
	
	\begin{equation}
		\Omega_{DE}(t) = \frac{\rho_{DE}(t)}{\rho_{\text{total}}} = \frac{\rho_{DE,0} + (\rho_{\text{matter},0} + \rho_{\gamma,0})(1 - e^{-\alpha c t})}{\rho_{\text{total}}}
	\end{equation}
	
	For \(t = t_0\) (today), we obtain \(\Omega_{DE}(t_0) \approx 0.68\), and for \(t \rightarrow \infty\), \(\Omega_{DE}(t) \rightarrow 1\). This means that in the T0-model, the fraction of dark energy increases over time until all energy is in the form of dark energy—consistent with the second law of thermodynamics.
	
	The current value of 68\% dark energy is then not a coincidence but a clue to the age of the universe relative to the characteristic timescale of energy transfer \(\tau = 1/(\alpha c) \approx 4.3 \times 10^{17} \text{ s} \approx 14 \text{ billion years}\). Assuming the universe is approximately \(t_0 \approx 13.8 \text{ billion years}\) old, we get:
	
	\begin{equation}
		\Omega_{DE}(t_0) \approx \Omega_{DE,0} + (1 - \Omega_{DE,0})(1 - e^{-t_0/\tau}) \approx 0.68
	\end{equation}
	
	With \(\Omega_{DE,0} \approx 0.05\) as the initial dark energy fraction. This calculation shows that in the T0-model, the observed dark energy dominance is a direct consequence of the universe's age and a natural outcome of the thermodynamic evolution of a static universe.
	
	\subsection{Large-Scale Structure and Baryon Acoustic Oscillations (BAO)}
	
	In the T0-model, the characteristic length scale of approximately 150 Mpc in galaxy distribution must be explained without invoking expansion. One possible explanation is that mass variation and energy exchange with the dark energy field generate characteristic length scales in structure formation. The mathematical description of these processes is given by the perturbation equation:
	
	\begin{equation}
		\nabla^2 \delta\phi_{\text{DE}} - m_{\phi}^2 \delta\phi_{\text{DE}} - 12\lambda\phi_{\text{DE}}^2 \delta\phi_{\text{DE}} = \frac{\beta}{M_{\text{Pl}}}\delta T^{\mu}_{\mu}
	\end{equation}
	
	%======================= PART 5: TESTS AND ANALYSIS ========================
	\section{Experimental Tests and Predictions}
	
	\subsection{Temporal Variation of the Fine-Structure Constant}
	
	Since photons in the T0-model lose energy to the dark energy field, this could lead to a temporal variation of fundamental constants:
	
	\begin{equation}
		\frac{d\alpha_{\text{fs}}}{dt} \approx \alpha_{\text{fs}} \cdot \alpha \cdot c \approx 10^{-18} \text{ year}^{-1}
	\end{equation}
	
	\subsection{Environment-Dependent Redshift}
	
	As dark energy in the T0-model is a dynamic field with spatial variations, the absorption rate \(\alpha\) should depend on local energy density:
	
	\begin{equation}
		\alpha(r) = \alpha_0 \cdot \left(1 + \delta\frac{\rho_{\text{baryon}}(r)}{\rho_0}\right)
	\end{equation}
	
	This leads to the prediction that redshift should slightly differ in dense cosmic regions (e.g., galaxy clusters) compared to cosmic voids:
	
	\begin{equation}
		\frac{z_{\text{cluster}}}{z_{\text{void}}} \approx 1 + \delta\frac{\rho_{\text{cluster}} - \rho_{\text{void}}}{\rho_0}
	\end{equation}
	
	\subsection{Anomalous Light Propagation in Strong Gravitational Fields}
	
	Since dark energy couples to matter in the T0-model, its density should be higher near massive objects. The effective refractive index of space would be:
	
	\begin{equation}
		n_{\text{eff}}(r) = 1 + \epsilon \frac{\phi_{\text{DE}}(r)}{M_{\text{Pl}}}
	\end{equation}
	
	\subsection{Differential Redshift}
	
	The wavelength dependence of redshift arises from:
	
	\begin{equation}
		\alpha(\lambda) = \alpha_0 \left(1 + \beta \cdot \frac{\lambda}{\lambda_0}\right)
	\end{equation}
	
	This would lead to a differential redshift:
	
	\begin{equation}
		\frac{z(\lambda_1)}{z(\lambda_2)} \approx 1 + \beta\frac{\lambda_1 - \lambda_2}{\lambda_0}
	\end{equation}
	
	with \(\beta = 0.008 \pm 0.003\) based on measurements.
	
	\section{Statistical Analysis and Comparison with the Standard Model}
	
	To compare the predictions of the T0-model with the standard model, we use Bayesian statistics:
	
	The Bayes evidence is given by:
	
	\begin{equation}
		E(M) = \int L(\theta|D,M) \pi(\theta|M) d\theta
	\end{equation}
	
	where \(L(\theta|D,M)\) is the likelihood of the data \(D\) given the parameters \(\theta\) in model \(M\), and \(\pi(\theta|M)\) is the prior distribution of the parameters. The Bayes ratio between the models is:
	
	\begin{equation}
		B_{T_0,\Lambda CDM} = \frac{E(T_0)}{E(\Lambda CDM)}
	\end{equation}
	
	\section{Detailed Analysis of Field Equations}
	
	\subsection{Dynamics of the Intrinsic Time Field}
	
	The dynamics of the intrinsic time field \(\Tfield\) and its coupling to the dark energy field can be described by an extended Lagrangian density:
	
	\begin{equation}
		\mathcal{L}_{T-DE} = \frac{1}{2}\partial_\mu \Tfield \partial^\mu \Tfield - U(\Tfield) + \zeta \Tfield \phi_{DE}^2
	\end{equation}
	
	where \(U(\Tfield)\) is the potential of the intrinsic time field and \(\zeta\) is a coupling constant. The field equation for \(\Tfield\) is:
	
	\begin{equation}
		\Box \Tfield - \frac{dU}{d\Tfield} + \zeta \phi_{DE}^2 = 0
	\end{equation}
	
	Together with the field equation for the dark energy field, this forms a coupled nonlinear system:
	
	\begin{equation}
		\Box\phi_{DE} - m_{\phi}^2\phi_{DE} - 4\lambda\phi_{DE}^3 - \frac{\beta}{M_{Pl}}T^{\mu}_{\mu} - \xi \phi_{DE} R + 2\zeta \Tfield \phi_{DE} = 0
	\end{equation}
	
	These coupled equations describe how the intrinsic time field and dark energy field interact with each other and with matter and radiation.
	
	\subsection{Covariance Properties of the Theory Formulation}
	
	In the T0-model, the global form of the spacetime metric is flat (Minkowski), while physical effects conventionally interpreted as gravitation are explained by variations in the intrinsic time field \(\Tfield\) and associated mass changes. The covariant formulation of the theory requires the introduction of an effective metric:
	
	\begin{equation}
		g_{\mu\nu}^{\text{eff}} = \eta_{\mu\nu} + h_{\mu\nu}(\Tfield)
	\end{equation}
	
	where \(\eta_{\mu\nu}\) is the Minkowski metric and \(h_{\mu\nu}(\Tfield)\) is a perturbation dependent on the intrinsic time field. In this formulation, the effective action can be written as:
	
	\begin{equation}
		S_{\text{eff}} = \int d^4x \sqrt{-g_{\text{eff}}} \mathcal{L}_{\text{Total}}(g_{\text{eff}}, \Tfield, \phi_{DE}, \psi, A_\mu, \Phi)
	\end{equation}
	
	This formulation ensures that the theory is generally covariant, with Einstein's field equations replaced by corresponding equations for the intrinsic time field.
	
	\subsection{Correspondence Properties with Standard Models}
	
	A consistent physical model must reduce to known theories in certain limits. For the T0-model, this holds:
	
	\begin{enumerate}
		\item \textbf{Limit of Constant Intrinsic Time}: For \(\Tfield = \text{const.}\), the theory reduces to the Standard Model of particle physics with conventional Lagrangian densities.
		
		\item \textbf{Weak Field Approximation}: For small variations of the intrinsic time field:
		\begin{equation}
			\Tfield = T_0 + \delta \Tfield, \quad |\delta \Tfield| \ll T_0
		\end{equation}
		
		the evolution of the field equations leads to equations formally equivalent to linearized General Relativity, with the identification:
		
		\begin{equation}
			h_{\mu\nu} \sim \frac{\delta \Tfield}{T_0} \eta_{\mu\nu}
		\end{equation}
		
		\item \textbf{Non-Relativistic Limit}: In the non-relativistic limit (\(v \ll c\)) and for weak fields, the T0-model leads to the modified Poisson equation:
		
		\begin{equation}
			\nabla^2 \Phi = 4\pi G \rho + \kappa^2
		\end{equation}
		
		which is formally comparable to modified gravity theories like MOND (Modified Newtonian Dynamics).
	\end{enumerate}
	
	\section{Thermodynamic Aspects of the T0-Model}
	
	\subsection{Entropy Considerations in a Static Universe}
	
	A fundamental objection to static cosmologies is based on the second law of thermodynamics, which demands an increase in entropy in closed systems. In the T0-model, this objection is addressed by the continuous energy transfer from matter and radiation to dark energy. The entropy balance can be formulated as:
	
	\begin{equation}
		\frac{dS_{\text{total}}}{dt} = \frac{dS_{\text{matter}}}{dt} + \frac{dS_{\text{radiation}}}{dt} + \frac{dS_{DE}}{dt} \geq 0
	\end{equation}
	
	The entropy production is related to the energy transfer by:
	
	\begin{equation}
		\frac{dS_{DE}}{dt} = \frac{1}{T_{\text{eff}}}\frac{dE_{DE}}{dt} = \frac{\alpha c}{T_{\text{eff}}}(E_{\text{matter}} + E_{\text{radiation}})
	\end{equation}
	
	where \(T_{\text{eff}}\) is an effective temperature of the dark energy field, assumed to be significantly lower than the temperature of matter and radiation. This low effective temperature explains why the energy transfer to dark energy is irreversible and associated with an entropy increase.
	
	\subsection{The Role of the Entropy Maximization Principle}
	
	In the T0-model, the temporal evolution of the universe can be understood as a process of entropy maximization, described by the Boltzmann equation:
	
	\begin{equation}
		\frac{\partial f}{\partial t} + \mathbf{v} \cdot \nabla_{\mathbf{r}} f + \mathbf{F} \cdot \nabla_{\mathbf{p}} f = \left(\frac{\partial f}{\partial t}\right)_{\text{coll}} + \left(\frac{\partial f}{\partial t}\right)_{DE}
	\end{equation}
	
	where \(f(\mathbf{r}, \mathbf{p}, t)\) is the phase-space distribution function, \(\mathbf{F}\) is the force on particles, and the term \(\left(\frac{\partial f}{\partial t}\right)_{DE}\) describes the energy exchange with the dark energy field. In the equilibrium state at \(t \rightarrow \infty\), all energy will be in the form of dark energy, corresponding to a state of maximum entropy.
	
	\section{Numerical Simulations and Predictions}
	
	\subsection{N-Body Simulations with Dark Energy Interaction}
	
	To compare the T0-model predictions with astronomical observations, N-body simulations were conducted, accounting for the interaction between matter and the dark energy field. The modified equation of motion for a particle is:
	
	\begin{equation}
		\frac{d^2\mathbf{r}_i}{dt^2} = -\nabla \Phi(\mathbf{r}_i) - \alpha_m c \frac{d\mathbf{r}_i}{dt}
	\end{equation}
	
	where the second term represents the energy loss to the dark energy field. The simulations show that:
	
	\begin{enumerate}
		\item Large-scale structures such as galaxy filaments and clusters form similarly to those in the \(\Lambda\)CDM model
		\item Galaxies exhibit more stable rotation curves without dark matter
		\item The Hubble flow can be interpreted as a collective energy loss of all galaxies to the dark energy field
	\end{enumerate}
	
	\subsection{Precise Predictions for Future Experiments}
	
	Based on the numerical simulations, precise predictions for future experiments can be made:
	
	\begin{enumerate}
		\item \textbf{Euclid Satellite}: The differential redshift should be measurable with:
		\begin{equation}
			\frac{\Delta z}{z} = \beta \frac{\Delta \lambda}{\lambda_0} \approx 0.008 \frac{\Delta \lambda}{\lambda_0}
		\end{equation}
		
		\item \textbf{ELT (Extremely Large Telescope)}: High-precision spectroscopy should detect the environment-dependent redshift:
		\begin{equation}
			\frac{z_{\text{cluster}}}{z_{\text{void}}} \approx 1 + (0.003 \pm 0.001)
		\end{equation}
		
		\item \textbf{SKA (Square Kilometre Array)}: Measurements of the hydrogen 21-cm line over a wide redshift range should show a characteristic deviation from the \(\Lambda\)CDM model:
		\begin{equation}
			\frac{d_A^{T0}(z)}{d_A^{\Lambda CDM}(z)} \approx 1 - 0.02 \ln(1+z)
		\end{equation}
		where \(d_A\) is the angular diameter distance.
	\end{enumerate}
	
	\section{Outlook and Summary}
	
	The T0-model of dark energy offers a conceptually new interpretation of cosmological observations. Instead of viewing dark energy as a driving force of cosmic expansion, it is understood as a dynamic medium for energy exchange in a static universe. Key mathematical elements of the theory include:
	
	\begin{enumerate}
		\item Time-mass duality with absolute time and variable mass
		\item The intrinsic time field \(\Tfield = \frac{\hbar}{mc^2}\) as a fundamental field
		\item Modified covariant derivatives accounting for this field
		\item A \(1/r^2\) density profile for dark energy
		\item Emergent gravitation from the intrinsic time field
		\item Redshift due to photon energy loss to dark energy
	\end{enumerate}
	
	The theory makes specific, experimentally testable predictions that allow differentiation between the T0-model and the standard model. Future experiments and observations, particularly precise measurements of wavelength-dependent and environment-dependent redshift, will be crucial in assessing the validity of the T0-model.
	
	Finally, the theory provides a conceptual framework that naturally connects quantum field theory and gravitational phenomena without requiring a separate quantization of gravity. This makes the T0-model a promising approach for a unified description of fundamental interactions.
	
\end{document}