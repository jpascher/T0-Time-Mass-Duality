\documentclass[12pt,a4paper]{article}
\usepackage[utf8]{inputenc}
\usepackage[T1]{fontenc}
\usepackage[ngerman]{babel}
\usepackage{lmodern}
\usepackage{amsmath}
\usepackage{amssymb}
\usepackage{hyperref}
\hypersetup{
	colorlinks=true,
	linkcolor=blue,
	citecolor=blue,
	urlcolor=blue
}

\title{Anpassung von Temperatureinheiten in natürlichen Einheiten und CMB-Messungen}
\author{Johann Pascher}
\date{2. 04. 2025}

\begin{document}
	
	\maketitle
	
	\section*{Einleitung}
	
	Es ist korrekt, dass alle Einheiten in natürlichen Einheiten gewählt werden könnten, indem man fundamentale Konstanten wie $\hbar$, $c$, $k_B$ und $G$ auf 1 setzt. Diese Vorgehensweise ist zwar möglich, aber in der Praxis nicht üblich. Dieses Dokument erläutert, wie die Temperaturmessung und die Schwarzkörperstrahlung in einem solchen System angepasst werden könnten, insbesondere wenn man zusätzlich $\alpha_W = 1$ setzt. Es wird auch darauf eingegangen, wie die Messungen der CMB-Temperatur heute durchgeführt werden und ob dabei indirekt Konstanten oder Kopplungsfaktoren vom Standardmodell beeinflusst sind. Danach wird die Idee der Anpassung der Temperatureinheit mit $\alpha_W = 1$ kommentiert.
	
	\section{Anpassung der Temperatureinheit mit $\alpha_W = 1$}
	
	Die konsequente Anwendung des Prinzips der maximalen Vereinfachung in natürlichen Einheitensystemen hat tiefgreifende Auswirkungen auf die Interpretation und Skalierung thermodynamischer Größen. Insbesondere die Beziehung zwischen Temperatur und Energie muss neu betrachtet werden. Die Plancksche Strahlungsformel, die die spektrale Energiedichte der Schwarzkörperstrahlung beschreibt:
	
	\begin{equation}
		u(\nu, T) = \frac{2\pi h \nu^3}{c^2} \cdot \frac{1}{e^{h \nu / k_B T} - 1}
	\end{equation}
	
	führt zur Wienschen Verschiebungsgesetz, das die Frequenz des Strahlungsmaximums mit der Temperatur verbindet:
	
	\begin{equation}
		\nu_{\text{max}} = \alpha \cdot \frac{k_B T}{h}
	\end{equation}
	
	wobei $\alpha \approx 2.82$ eine numerisch bestimmte Konstante ist. Um Verwechslungen mit der Feinstrukturkonstante zu vermeiden, bezeichnen wir diese Konstante im Folgenden als $\alpha_W$. Wenn nun zusätzlich zu $k_B = h = c = 1$ auch $\alpha_W = 1$ gesetzt wird, ergibt sich eine direkte Proportionalität zwischen der Frequenz des Strahlungsmaximums und der Temperatur:
	
	\begin{equation}
		\nu_{\text{max}} = T
	\end{equation}
	
	Um diese Beziehung konsistent zu machen, ist eine Anpassung der Temperatureinheit erforderlich. Kelvin würde als Basiseinheit ungeeignet, da die Temperatur dann direkt in Energieeinheiten gemessen und so skaliert würde, dass sie mit der Frequenz des Strahlungsmaximums übereinstimmt. Diese Anpassung ist analog zur Behandlung von Raum und Zeit in der Relativitätstheorie, wo bei $c = 1$ beide in Längeneinheiten gemessen werden können. Die Wahl $\alpha_W = 1$ ist also eine konsequente Fortführung des Prinzips der maximalen Vereinfachung, die jedoch eine Neudefinition der Temperatureinheit erfordert.
	\section{Anpassung der Temperatureinheit mit $\alpha_W = 1$}
	
	Das \href{https://github.com/jpascher/T0-Time-Mass-Duality/tree/main/2/pdf/Deutsch/Natürliche Einheiten mit Feinstrukturkonstante alpha = 1.pdf}{Dokument} schlägt vor, dass in natürlichen Einheiten mit $k_B = h = c = 1$ und zusätzlich $\alpha_W = 1$ die Temperatur direkt mit der Frequenz des Strahlungsmaximums übereinstimmt ($\nu_{\text{max}} = T$). Lassen Sie uns das prüfen:
	
	\subsection{Standardformel}
	
	Das Wien’sche Verschiebungsgesetz in SI-Einheiten lautet:
	\[
	\nu_{\text{max}} = \alpha \cdot \frac{k_B T}{h}, \quad \alpha \approx 2.821439,
	\]
	wobei $\alpha$ aus der Maximierung der Planck-Verteilung ($3 (e^x - 1) = x e^x$) numerisch bestimmt wird.
	\subsection{Natürliche Einheiten}
	
	Mit $k_B = 1$, $h = 2\pi$ (da $\hbar = 1$), $c = 1$:
	\[
	\nu_{\text{max}} = \alpha \cdot \frac{T}{2\pi},
	\]
	\[
	\nu_{\text{max}} = \frac{2.821439}{2\pi} T \approx 0.449 T.
	\]
	In natürlichen Einheiten bleibt $\alpha \approx 2.82$, weil es eine mathematische Konstante ist, die nicht von $h$, $c$ oder $k_B$ abhängt.
	\subsection{$\alpha_W = 1$ setzen}
	
	Wenn $\alpha_W = 1$ gesetzt wird:
	\[
	\nu_{\text{max}} = \frac{T}{2\pi},
	\]
	oder, wenn $\alpha_W$ komplett eliminiert werden soll:
	\[
	\nu_{\text{max}} = T.
	\]
	Das erfordert eine Neudefinition der Temperatureinheit:
	\begin{itemize}
		\item In SI: $T$ in Kelvin, $\nu_{\text{max}}$ in Hz, und $k_B T / h$ skaliert die Beziehung.
		\item Mit $k_B = h = 1$, $\alpha_W = 1$: $T$ muss so skaliert werden, dass $\nu_{\text{max}} = T$ gilt, was bedeutet, dass $T$ eine Frequenz (oder Energie in natürlichen Einheiten) wird, nicht Kelvin.
	\end{itemize}
	
	\subsection{Implikationen}
	
	\begin{itemize}
		\item \textbf{Neue Einheit:} $T$ wäre keine Temperatur im klassischen Sinne (Kelvin), sondern eine Energie/Frequenz (z.B. in GeV oder Hz, wenn $c = 1$ wegfällt). Das ist konsistent mit der \href{https://github.com/jpascher/T0-Time-Mass-Duality/tree/main/2/pdf/Deutsch/Eine neue Perspektive auf Zeit und Raum Johann Paschers revolutionäre Ideen.pdf}{Analogie zur Relativitätstheorie} ($c = 1$, Raum und Zeit in Längeneinheiten).
		\item \textbf{CMB-Temperatur:} Die gemessene $T = 2.725 \, \text{K}$ müsste umgerechnet werden. In natürlichen Einheiten mit $k_B = 1$:
		\[
		T = 2.725 \, \text{K} \cdot k_B = 2.725 \cdot 1.380649 \times 10^{-23} \, \text{J} \approx 3.762 \times 10^{-23} \, \text{J}.
		\]
		Mit $h = 2\pi \hbar = 6.62607015 \times 10^{-34} \, \text{J·s}$:
		\[
		\nu_{\text{max}} = \frac{k_B T}{h} \cdot 2.821439 \approx \frac{3.762 \times 10^{-23}}{6.62607015 \times 10^{-34}} \cdot 2.821439 \approx 1.6 \times 10^{11} \, \text{Hz}.
		\]
		Mit $\alpha_W = 1$:
		\[
		\nu_{\text{max}} = \frac{T}{2\pi} \approx 6 \times 10^{10} \, \text{Hz},
		\]
		und $T$ müsste auf diese Frequenz skaliert werden, was eine neue Einheit erfordert.
	\end{itemize}
	
	\subsection{Warum nicht üblich?}
	
	\begin{itemize}
		\item \textbf{Beobachtungspraxis:} Kosmologen verwenden Kelvin, weil es direkt mit gemessenen Temperaturen (z.B. CMB, Sternoberflächen) verknüpft ist. Natürliche Einheiten mit $\alpha_W = 1$ würden die Kommunikation mit experimentellen Daten erschweren.
		\item \textbf{Mathematische Konstante:} $\alpha \approx 2.82$ ist keine willkürliche Konstante, sondern eine Lösung der Gleichung $3 (e^x - 1) = x e^x$, unabhängig von Einheiten. Sie auf 1 zu setzen, ist eine Vereinfachung, die physikalische Realität verzerrt, es sei denn, $T$ wird entsprechend neu definiert.
	\end{itemize}
	
	\section{Fazit}
	
	Die CMB-Temperaturmessung basiert auf Frequenzmessungen, die an die Planck-Verteilung angepasst werden. In natürlichen Einheiten ($\hbar = c = k_B = 1$) vereinfacht sich die Form, aber $\alpha \approx 2.82$ bleibt. Mit $\alpha_W = 1$ wird $T$ eine Frequenz/Energie, was möglich, aber nicht üblich ist, da es die direkte Verbindung zu beobachtbaren Einheiten (Kelvin) aufgibt. Das Dokument zeigt diese Möglichkeit, aber die Praxis bevorzugt SI-Einheiten für Vergleichbarkeit mit Messdaten.
	
	\begin{thebibliography}{9}
		\bibitem{Planck2018Temp}
		Planck Collaboration, Aghanim, N., et al. (2020). 
		\textit{Planck 2018 results. V. CMB power spectra and likelihoods}. 
		Astronomy \& Astrophysics, 641, A5. 
		DOI: 10.1051/0004-6361/201833887.
		
		\bibitem{Fixsen2009}
		Fixsen, D. J. (2009). \textit{The Temperature of the Cosmic Microwave Background}. 
		The Astrophysical Journal, 707(2), 916–920. 
		DOI: 10.1088/0004-637X/707/2/916.
		
		\bibitem{ACTTemp}
		Choi, S. K., et al. (2020).
		\textit{The Atacama Cosmology Telescope: A Measurement of the Cosmic Microwave Background Power Spectra at 98 and 150 GHz}.
		Journal of Cosmology and Astroparticle Physics, 2020(12), 045. 
		DOI: 10.1088/1475-7516/2020/12/045. 
		
		\bibitem{SPTTemp}
		Reichardt, C. L., et al. (2021). \textit{The South Pole Telescope 3G Survey: CMB Temperature and Polarization Power Spectra}. 
		The Astrophysical Journal, 908(2), 199. 
		DOI: 10.3847/1538-4357/abd407.
		
		\bibitem{Mather1994}
		Mather, J. C., et al. (1994). \textit{Measurement of the Cosmic Microwave Background Spectrum by the COBE FIRAS Instrument}. 
		The Astrophysical Journal, 420, 439–444. 
		DOI: 10.1086/173574.
		
		\bibitem{SunyaevZeldovich}
		Birkinshaw, M. (1999). \textit{The Sunyaev-Zel’dovich Effect}. 
		Physics Reports, 310(2–3), 97–195.
		DOI: 10.1016/S0370-1573(98)00080-5. 
		
		\bibitem{PlanckTech}
		Planck Collaboration, Tauber, J. A., et al. (2010). \textit{Planck Pre-Launch Status: The Planck Mission}. 
		Astronomy \& Astrophysics, 520, A1. 
		DOI: 10.1051/0004-6361/200912983.
		
		\bibitem{CMBTheoryTemp}
		Hu, W., \& Dodelson, S. (2002). \textit{Cosmic Microwave Background Anisotropies}. 
		Annual Review of Astronomy and Astrophysics, 40, 171–216. 
		DOI: 10.1146/annurev.astro.40.060401.093926. 
		
	\end{thebibliography}
	
\end{document}