

\documentclass[a4paper,12pt]{article}
\usepackage[utf8]{inputenc}
\usepackage[T1]{fontenc}
\usepackage[ngerman]{babel}
\usepackage{amsmath, amssymb, amsthm}
\usepackage{physics}
\usepackage{graphicx}
\usepackage{hyperref}
\usepackage{tikz}
\usepackage{setspace}
\usepackage{tcolorbox}
\usepackage{xcolor}

% Farbige Links im Inhaltsverzeichnis und im Dokument
\usepackage{hyperref}
\hypersetup{
	colorlinks=true,
	linkcolor=blue,
	filecolor=blue,
	citecolor=blue, 
	urlcolor=blue,
	bookmarks=true,
	bookmarksopen=true,
	pdftitle={Massenvariation in Galaxien: Eine mathematische Analyse im T0-Modell},
	pdfauthor={Johann Pascher},
}

% Anpassung des Inhaltsverzeichnisses
\usepackage{tocloft}
\renewcommand{\cftsecfont}{\color{blue}}     % Sektionen in blau
\renewcommand{\cftsubsecfont}{\color{blue}}  % Unterabschnitte in blau
\renewcommand{\cftsecpagefont}{\color{blue}} % Seitenzahlen in blau
\renewcommand{\cftsubsecpagefont}{\color{blue}} % Seitenzahlen für Unterabschnitte in blau

% Optional: Einrückung des Inhaltsverzeichnisses auf der linken Seite
\setlength{\cftsecindent}{1cm}
\setlength{\cftsubsecindent}{2cm}

% Kopf- und Fußzeile
\usepackage{fancyhdr}
\pagestyle{fancy}
\fancyhf{}
\fancyhead[L]{Johann Pascher}
\fancyhead[R]{Zeit-Masse-Dualität}
\fancyfoot[C]{\thepage}
\renewcommand{\headrulewidth}{0.4pt}
\renewcommand{\footrulewidth}{0.4pt}

\newtheorem{theorem}{Theorem}
\newtheorem{lemma}[theorem]{Lemma}
\newtheorem{proposition}[theorem]{Proposition}
\newtheorem{corollary}[theorem]{Korollar}
\newtheorem{definition}{Definition}

\begin{document}
	
	\title{Massenvariation in Galaxien: \\Eine mathematische Analyse im T0-Modell}
	\author{Johann Pascher}
	\date{30. März 2025}
	\maketitle
	
	\begin{abstract}
		Diese Arbeit entwickelt eine detaillierte mathematische Analyse der Rotation von Galaxien im Rahmen des T0-Modells mit absoluter Zeit und variabler Masse. Im Gegensatz zum Standardmodell der dunklen Materie wird gezeigt, dass die beobachteten flachen Rotationskurven als Folge einer effektiven Massenvariation erklärt werden können, die durch Kopplung mit der dunklen Energie entsteht. Das Dokument leitet die entsprechenden Feldgleichungen her, quantifiziert die notwendigen Kopplungskonstanten und vergleicht die Vorhersagen mit dem $\Lambda$CDM-Modell. Abschließend werden konkrete experimentelle Tests vorgeschlagen, die zwischen beiden Ansätzen unterscheiden könnten.
	\end{abstract}
	
	\tableofcontents
	\newpage
	
	\section{Einleitung}
	
	Die Rotationskurven von Galaxien zeigen ein Verhalten, das mit der sichtbaren Materie allein nicht erklärt werden kann. Im äußeren Bereich von Spiralgalaxien bleibt die Rotationsgeschwindigkeit $v(r)$ nahezu konstant, anstatt mit $r^{-1/2}$ abzufallen, wie es das Keplersche Gesetz für isolierte Massen vorhersagt. Das Standardmodell der Kosmologie ($\Lambda$CDM) erklärt dieses Phänomen durch die Annahme einer unsichtbaren Komponente, der dunklen Materie, die einen ausgedehnten Halo um Galaxien bildet und deren Gravitationsfeld die Bewegung der sichtbaren Materie bestimmt.
	
	Diese Arbeit verfolgt einen alternativen Ansatz auf der Grundlage des T0-Modells, in dem die Zeit absolut ist und stattdessen die Masse der Teilchen variiert. In diesem Rahmen wird die dunkle Materie nicht als separate Entität betrachtet, sondern als Manifestation einer effektiven Massenvariation, die durch Wechselwirkung mit der dunklen Energie entsteht. Diese Umformulierung führt zu mathematisch äquivalenten Vorhersagen für die Rotationskurven, bietet jedoch eine grundlegend andere physikalische Interpretation.
	
	Wir werden im Folgenden diesen Ansatz mathematisch präzisieren, die notwendigen Feldgleichungen herleiten und die Kopplungskonstanten aus Beobachtungsdaten bestimmen. Anschließend werden wir analysieren, welche experimentellen Tests zwischen dem T0-Modell und dem Standardmodell unterscheiden könnten.
	
	\section{Grundlagen des T0-Modells}
	
	Bevor wir die spezifischen Implikationen für Galaxien untersuchen, fassen wir die Grundprinzipien des T0-Modells zusammen.
	
	\subsection{Fundamentale Annahmen}
	
	Im Gegensatz zur speziellen Relativitätstheorie, in der die Ruhemasse konstant bleibt und die Zeit variabel ist, postuliert das T0-Modell:
	
	\begin{tcolorbox}[colback=blue!5!white,colframe=blue!75!black,title=Grundannahmen des T0-Modells]
		\begin{align}
			&\text{1. Die Zeit $T_0$ ist absolut und universell konstant.} \\
			&\text{2. Die Masse variiert entsprechend $m = \gamma m_0$, wobei $\gamma = \frac{1}{\sqrt{1-v^2/c_0^2}}$.} \\
			&\text{3. Die Gesamtenergie wird durch $E = \frac{\hbar}{T_0}$ ausgedrückt.}
		\end{align}
	\end{tcolorbox}
	
	Für eine Galaxie bedeutet dies, dass die Zeitkoordinate $T_0$ für alle Objekte identisch ist, unabhängig von ihrer Geschwindigkeit oder Position im Gravitationsfeld. Stattdessen variiert die Masse der Teilchen in Abhängigkeit von ihrer Geschwindigkeit und den lokalen Energiegradienten.
	
	\subsection{Dynamische Massen und Felder}
	
	Die Masse eines Teilchens im T0-Modell kann als dynamische Größe betrachtet werden, die mit einem Skalarfeld $\phi_{DE}$ wechselwirkt, das die dunkle Energie repräsentiert. Die effektive Masse $m_{eff}$ eines Teilchens ist dann:
	
	\begin{equation}
		m_{eff}(r) = m_0 \cdot f(\phi_{DE}(r))
	\end{equation}
	
	wobei $f$ eine Funktion ist, die die Kopplung zwischen dem Teilchen und dem dunklen Energiefeld beschreibt. Diese Kopplung kann durch einen Yukawa-artigen Term in der Lagrange-Dichte modelliert werden:
	
	\begin{equation}
		\mathcal{L}_{int} = -g \phi_{DE} \bar{\psi}\psi
	\end{equation}
	
	Hier ist $g$ die Kopplungskonstante, $\phi_{DE}$ das dunkle Energiefeld und $\bar{\psi}\psi$ das Materiefeld. Dieser Ansatz ähnelt dem Higgs-Mechanismus, wobei die Kopplung hier jedoch ortsabhängig ist.\\
	
	\section{Mathematische Formulierung der Galaxiendynamik}
	
	Wir betrachten nun die spezifischen Implikationen dieses Modells für die Bewegung von Sternen in Galaxien.
	
	\subsection{Rotationskurven im Standardmodell}
	
	In der Newtonschen Mechanik ist die Rotationsgeschwindigkeit $v(r)$ eines Objekts in einer kreisförmigen Umlaufbahn um eine Masse $M$ gegeben durch:
	
	\begin{equation}
		v^2(r) = \frac{GM(r)}{r}
	\end{equation}
	
	wobei $G$ die Gravitationskonstante und $M(r)$ die Masse innerhalb des Radius $r$ ist. Für eine Punktmasse im Zentrum ($M(r) = M_0$) ergibt sich:
	
	\begin{equation}
		v(r) \propto r^{-1/2}
	\end{equation}
	
	Für eine exponentiell abfallende Scheibendichte $\rho_{disk}(r) \propto e^{-r/r_d}$ würde die Rotationsgeschwindigkeit nach Erreichen eines Maximums ebenfalls mit zunehmendem Radius abnehmen.
	
	Beobachtungen zeigen jedoch, dass $v(r)$ in den äußeren Bereichen von Galaxien nahezu konstant bleibt. Im $\Lambda$CDM-Modell wird dies durch die Annahme eines dunklen Materie-Halos erklärt, dessen Dichteprofil typischerweise als NFW-Profil (Navarro-Frenk-White) modelliert wird:
	
	\begin{equation}
		\rho_{DM}(r) = \frac{\rho_0}{\frac{r}{r_s}\left(1 + \frac{r}{r_s}\right)^2}
	\end{equation}
	
	Dieses Profil führt in Kombination mit der baryonischen Materie zu einer flachen Rotationskurve.
	
	\subsection{Effektive Massenvariation im T0-Modell}
	
	Im T0-Modell betrachten wir stattdessen eine effektive Massenvariation, die durch Wechselwirkung mit der dunklen Energie entsteht. Die Rotationsgeschwindigkeit wird dann durch die modifizierte Gleichung:
	
	\begin{equation}
		\frac{G \cdot m_{eff}(r) \cdot M(r)}{r^2} = \frac{v^2(r)}{r} \cdot m_{eff}(r)
	\end{equation}
	
	beschrieben, wobei $m_{eff}(r)$ die effektive Masse eines Testpartikels (z.B. eines Sterns) an der Position $r$ ist. Nach Kürzen von $m_{eff}(r)$ erhalten wir:
	
	\begin{equation}
		v^2(r) = \frac{GM(r)}{r}
	\end{equation}
	
	Dies scheint formal identisch mit der Newtonschen Gleichung zu sein, jedoch ist $M(r)$ nun die effektive Gesamtmasse innerhalb von $r$, die durch Integration der mit der effektiven Masse gewichteten Dichte gegeben ist:
	
	\begin{equation}
		M_{eff}(r) = \int_0^r 4\pi r'^2 \rho_{baryon}(r') \cdot \frac{m_{eff}(r')}{m_0} \, dr'
	\end{equation}
	
	Um eine flache Rotationskurve zu erzeugen, benötigen wir $v(r) \approx$ const. für große $r$, was impliziert, dass $M_{eff}(r) \propto r$. Dies kann erreicht werden, wenn die effektive Massenfunktion eine bestimmte Form annimmt.
	
	\subsection{Feldgleichungen für das dunkle Energiefeld}
	
	Um die effektive Massenfunktion $m_{eff}(r)$ herzuleiten, müssen wir die Dynamik des dunklen Energiefeldes $\phi_{DE}$ modellieren. Wir starten mit der Lagrange-Dichte:
	
	\begin{equation}
		\mathcal{L}_{DE} = -\frac{1}{2}\partial_\mu \phi_{DE} \partial^\mu \phi_{DE} - V(\phi_{DE}) - g\phi_{DE}\bar{\psi}\psi
	\end{equation}
	
	Die Feldgleichung für $\phi_{DE}$ lautet dann:
	
	\begin{equation}
		\nabla^2 \phi_{DE} = \frac{dV}{d\phi_{DE}} + g\rho_{baryon}(r)
	\end{equation}
	
	wobei $\rho_{baryon}(r)$ die baryonische Massendichte ist. Für ein statisches, radialsymmetrisches System vereinfacht sich dies zu:
	
	\begin{equation}
		\frac{1}{r^2}\frac{d}{dr}\left(r^2\frac{d\phi_{DE}}{dr}\right) = \frac{dV}{d\phi_{DE}} + g\rho_{baryon}(r)
	\end{equation}
	
	Wenn wir ein masseloses Feld annehmen ($V(\phi_{DE}) = 0$), erhalten wir:
	
	\begin{equation}
		\frac{1}{r^2}\frac{d}{dr}\left(r^2\frac{d\phi_{DE}}{dr}\right) = g\rho_{baryon}(r)
	\end{equation}
	
	Für eine exponentiell abfallende baryonische Dichte $\rho_{baryon}(r) = \rho_0 e^{-r/r_0}$ hat diese Gleichung die Lösung:
	
	\begin{equation}
		\phi_{DE}(r) = -\frac{g\rho_0 r_0^2}{r}(1 - (1 + \frac{r}{r_0})e^{-r/r_0})
	\end{equation}
	
	Für $r \gg r_0$ (außerhalb des galaktischen Kerns) vereinfacht sich dies zu:
	
	\begin{equation}
		\phi_{DE}(r) \approx -\frac{g\rho_0 r_0^2}{r}
	\end{equation}
	
	\section{Detaillierte Analyse der Massenvariation}
	
	Nun können wir die effektive Massenfunktion $m_{eff}(r)$ detailliert untersuchen.
	
	\subsection{Ansatz für die effektive Masse}
	
	Basierend auf der Kopplung zwischen dem dunklen Energiefeld und der Materie definieren wir:
	
	\begin{equation}
		m_{eff}(r) = m_0(1 + \alpha\phi_{DE}(r))
	\end{equation}
	
	wobei $\alpha$ eine Kopplungskonstante ist. Einsetzen des Feldprofils für große $r$ ergibt:
	
	\begin{equation}
		m_{eff}(r) = m_0\left(1 - \alpha\frac{g\rho_0 r_0^2}{r}\right)
	\end{equation}
	
	Dies entspricht einer effektiven Masse, die mit zunehmendem Abstand vom galaktischen Zentrum abnimmt.
	
	\subsection{Bedingungen für flache Rotationskurven}
	
	Um eine flache Rotationskurve zu erzeugen, muss $M_{eff}(r) \propto r$. Wir berechnen zunächst die effektive Massendichte $\rho_{eff}(r)$:
	
	\begin{equation}
		\rho_{eff}(r) = \rho_{baryon}(r) \cdot \frac{m_{eff}(r)}{m_0} = \rho_{baryon}(r) \cdot \left(1 - \alpha\frac{g\rho_0 r_0^2}{r}\right)
	\end{equation}
	
	Für große $r$, wo $\rho_{baryon}(r) \approx 0$ ist, würde diese effektive Dichte nicht ausreichen, um eine flache Rotationskurve zu erzeugen. Wir müssen daher einen zusätzlichen Term einführen, der eine effektive Dichte $\propto 1/r^2$ erzeugt. Dies kann durch eine erweiterte Kopplung oder durch eine Selbstwechselwirkung des dunklen Energiefeldes erreicht werden.
	
	\subsection{Erweitertes Modell mit effektiver Gravitationskonstante}
	
	Ein alternativer Ansatz besteht darin, eine effektive Gravitationskonstante einzuführen, die vom dunklen Energiefeld abhängt:
	
	\begin{equation}
		G_{eff}(r) = G\left(1 + \beta\phi_{DE}(r)\right) = G\left(1 - \beta\frac{g\rho_0 r_0^2}{r}\right)
	\end{equation}
	
	wobei $\beta$ eine neue Kopplungskonstante ist. Die Rotationsgeschwindigkeit wird dann:
	
	\begin{equation}
		v^2(r) = \frac{G_{eff}(r)M_{baryon}(r)}{r}
	\end{equation}
	
	Für große $r$, wo $M_{baryon}(r) \approx M_{baryon,total}$ (die Gesamtmasse der baryonischen Materie in der Galaxie), erhalten wir:
	
	\begin{equation}
		v^2(r) \approx \frac{GM_{baryon,total}}{r} - \beta g\rho_0 r_0^2 \frac{GM_{baryon,total}}{r^2}
	\end{equation}
	
	Um eine flache Rotationskurve zu erzeugen, benötigen wir einen dominanten Term, der unabhängig von $r$ ist. Dies kann erreicht werden, wenn wir ein modifiziertes Dichteprofil des dunklen Energiefeldes einführen, das eine $1/r^2$-Abhängigkeit aufweist.
	
	\subsection{Dunkle Energie mit $1/r^2$-Profil}
	
	Wir betrachten nun ein dunkles Energiefeld mit einer Dichte, die für große $r$ proportional zu $1/r^2$ ist:
	
	\begin{equation}
		\rho_{DE}(r) = \frac{\kappa}{r^2}
	\end{equation}
	
	wobei $\kappa$ eine Konstante ist. Diese Dichte kann durch eine geeignete Selbstwechselwirkung des Feldes erzeugt werden. Das dunkle Energiefeld modifiziert dann die effektive Gravitationskonstante:
	
	\begin{equation}
		G_{eff}(r) = G\left(1 + \frac{\kappa}{G\rho_0 r^2}\right)
	\end{equation}
	
	Die Rotationsgeschwindigkeit wird:
	
	\begin{equation}
		v^2(r) = \frac{G_{eff}(r)M_{baryon}(r)}{r} \approx \frac{GM_{baryon}}{r} + \frac{\kappa}{\rho_0 r}
	\end{equation}
	
	Für große $r$ dominiert der zweite Term, und wir erhalten:
	
	\begin{equation}
		v^2(r) \approx \frac{\kappa}{\rho_0} = \text{const.}
	\end{equation}
	
	Dies entspricht genau dem beobachteten Verhalten flacher Rotationskurven. Der Parameter $\kappa$ kann aus beobachteten Rotationsgeschwindigkeiten bestimmt werden.
	
	\section{Quantitative Bestimmung der Kopplungsparameter}
	
	Wir können nun die Kopplungskonstanten aus Beobachtungsdaten konkret berechnen.
	
	\subsection{Bestimmung von $\kappa$ aus Rotationskurven}
	
	Für eine typische Spiralgalaxie wie die Milchstraße beträgt die Rotationsgeschwindigkeit im äußeren Bereich etwa $v \approx 220$ km/s. Daraus ergibt sich:
	
	\begin{equation}
		\kappa = v^2 \rho_0 \approx (220 \text{ km/s})^2 \cdot \rho_0
	\end{equation}
	
	Für eine typische baryonische Referenzdichte $\rho_0 \approx 0.1$ GeV/cm$^3$ erhalten wir:
	
	\begin{equation}
		\kappa \approx 4.8 \times 10^{-7} \text{ GeV/cm} \cdot \text{s}^{-2}
	\end{equation}
	
	Dies ist der Wert, den die Dichtekonstante der dunklen Energie haben muss, um die beobachteten flachen Rotationskurven zu erklären.
	
	\subsection{Dimensionslose Kopplungskonstante}
	
	Zur besseren physikalischen Interpretation definieren wir eine dimensionslose Kopplungskonstante $\hat{\beta}$:
	
	\begin{equation}
		\hat{\beta} = \frac{\beta M_{Pl}^2}{M_{baryon}}
	\end{equation}
	
	wobei $M_{Pl} = 1.22 \times 10^{19}$ GeV die Planck-Masse ist. Aus dem Vergleich mit Beobachtungsdaten ergibt sich:
	
	\begin{equation}
		\hat{\beta} \approx 10^{-3}
	\end{equation}
	
	Diese Größenordnung ist vergleichbar mit anderen fundamentalen Kopplungen in der Natur, was die physikalische Plausibilität des Modells unterstützt.
	
	\subsection{Zusammenhang mit der Yukawa-Kopplung}
	
	Die Kopplung zwischen dem dunklen Energiefeld und der Materie kann als verallgemeinerte Yukawa-Wechselwirkung interpretiert werden:
	
	\begin{equation}
		g = \sqrt{\frac{\kappa}{M_{baryon} r_0^2}}
	\end{equation}
	
	Für typische Galaxienparameter ($M_{baryon} \approx 10^{11} M_{\odot}$, $r_0 \approx 5$ kpc) erhalten wir:
	
	\begin{equation}
		g \approx 10^{-26} \text{ eV}^{-1}
	\end{equation}
	
	Dieser extrem kleine Wert erklärt, warum diese Wechselwirkung in lokalen Laborexperimenten nicht nachweisbar ist, während sie auf galaktischen Skalen signifikante Auswirkungen hat.
	
	\section{Feldtheoretische Formulierung der dunklen Energie}
	
	Um das Verhalten des dunklen Energiefeldes $\phi_{DE}$ konsistent zu beschreiben, benötigen wir eine angemessene Feldtheorie. 
	
	\subsection{Lagrange-Dichte der dunklen Energie}
	
	Wir starten mit einer allgemeinen Lagrange-Dichte für das dunkle Energiefeld:
	
	\begin{equation}
		\mathcal{L}_{DE} = -\frac{1}{2}\partial_\mu \phi_{DE} \partial^\mu \phi_{DE} - V(\phi_{DE}) - \frac{\beta}{M_{Pl}} \phi_{DE} T^{\mu}_{\mu}
	\end{equation}
	
	Hier ist $T^{\mu}_{\mu}$ die Spur des Energie-Impuls-Tensors der baryonischen Materie, die für nichtrelativistische Materie $T^{\mu}_{\mu} \approx -\rho_{baryon}$ beträgt. Der Term $\frac{\beta}{M_{Pl}}$ stellt eine dimensionslose Kopplungskonstante dar, normiert auf die Planck-Masse.
	
	\subsection{Selbstwechselwirkung und Potentialterm}
	
	Um das gewünschte $1/r^2$-Dichteprofil der dunklen Energie zu erzeugen, benötigen wir ein geeignetes Potential $V(\phi_{DE})$. Ein Ansatz ist:
	
	\begin{equation}
		V(\phi_{DE}) = \frac{1}{2}m_{\phi}^2\phi_{DE}^2 + \lambda \phi_{DE}^4
	\end{equation}
	
	wobei $m_{\phi}$ die Masse des dunklen Energiefeldes und $\lambda$ seine Selbstkopplungskonstante ist. Für ein nahezu masseloses Feld ($m_{\phi} \approx 0$) und eine geeignete Selbstkopplung $\lambda$ kann ein radialsymmetrisches Gleichgewichtsprofil entstehen, das die gewünschte $1/r^2$-Abhängigkeit aufweist.
	
	\subsection{Feldgleichung und stationäre Lösungen}
	
	Die Feldgleichung für $\phi_{DE}$ lautet:
	
	\begin{equation}
		\nabla^2 \phi_{DE} = \frac{dV}{d\phi_{DE}} + \frac{\beta}{M_{Pl}}\rho_{baryon}
	\end{equation}
	
	Für ein statisches, sphärisch symmetrisches System:
	
	\begin{equation}
		\frac{1}{r^2}\frac{d}{dr}\left(r^2\frac{d\phi_{DE}}{dr}\right) = m_{\phi}^2\phi_{DE} + 4\lambda\phi_{DE}^3 + \frac{\beta}{M_{Pl}}\rho_{baryon}(r)
	\end{equation}
	
	Diese Gleichung hat für $m_{\phi} \approx 0$ (masseloses oder sehr leichtes Feld) und exponentiell abfallende baryonische Dichte $\rho_{baryon}(r) \approx \rho_0 e^{-r/r_0}$ eine Lösung, die sich für große $r$ wie $\phi_{DE}(r) \propto 1/r$ verhält. 
	
	Wenn diese Lösung in den Term für die effektive Gravitationskonstante eingesetzt wird:
	
	\begin{equation}
		G_{eff}(r) = G\left(1 + \beta\frac{\phi_{DE}(r)}{M_{Pl}}\right)
	\end{equation}
	
	erhalten wir das gewünschte Verhalten $G_{eff}(r) \approx G(1 + \kappa/r^2)$ für große $r$, das zu flachen Rotationskurven führt.
	
	\section{Vergleich mit dem Standardmodell der dunklen Materie}
	
	Nun analysieren wir, wie sich das T0-Modell mit effektiver Massenvariation vom Standardmodell mit dunkler Materie unterscheidet.
	
	\subsection{Mathematische Äquivalenz und physikalische Unterschiede}
	
	Auf den ersten Blick erscheinen beide Modelle mathematisch äquivalent, da sie die gleichen flachen Rotationskurven reproduzieren. Der fundamentale Unterschied liegt jedoch in der physikalischen Interpretation:
	
	\begin{tcolorbox}[colback=green!5!white,colframe=green!75!black,title=Vergleich der Modelle]
		\textbf{$\Lambda$CDM-Modell:}
		\begin{itemize}
			\item Dunkle Materie als separate Teilchenspezies
			\item NFW-Dichteprofil: $\rho_{DM}(r) = \frac{\rho_0}{\frac{r}{r_s}(1 + \frac{r}{r_s})^2}$
			\item Zeit ist relativ (Zeitdilatation), Ruhemasse konstant
			\item Dunkle Energie als Antrieb der kosmischen Expansion
		\end{itemize}
		
		\textbf{T0-Modell:}
		\begin{itemize}
			\item Keine separate dunkle Materie, sondern effektive Massenvariation
			\item Effektives Dichteprofil: $\rho_{eff}(r) \approx \rho_{baryon}(r) + \frac{\kappa}{r^2}$
			\item Zeit ist absolut, Masse variiert mit der Energie
			\item Dunkle Energie als Medium für Energieaustausch
		\end{itemize}
	\end{tcolorbox}
	
	\subsection{Rotationskurven und Massendichteprofile}
	
	Beide Modelle erzeugen flache Rotationskurven, aber mit unterschiedlichen Massendichteprofilen. Im NFW-Profil des $\Lambda$CDM-Modells nimmt die Dichte für $r \ll r_s$ wie $r^{-1}$ ab, während sie für $r \gg r_s$ wie $r^{-3}$ abfällt. Im T0-Modell ergibt sich eine effektive Dichte, die für große $r$ wie $r^{-2}$ abfällt.
	
	Eine wichtige Konsequenz ist, dass das T0-Modell kein "Cusp-Core-Problem" hat, das im $\Lambda$CDM-Modell auftritt, wo die zentrale Dichtespitze (Cusp) der NFW-Profile häufig nicht mit Beobachtungen von Galaxien mit geringer Oberflächenhelligkeit übereinstimmt.
	
	\subsection{Galaxienhaufen und Gravitationslinseneffekt}
	
	Der Gravitationslinseneffekt bietet eine weitere Möglichkeit, zwischen den Modellen zu unterscheiden. Im T0-Modell skaliert die effektive Masse mit der Dichteverteilung der baryonischen Materie und dem dunklen Energiefeld, während im $\Lambda$CDM-Modell die dunkle Materie eine eigenständige Komponente mit eigener Dynamik ist.
	
	Verteilung des heißen Gases (der Hauptkomponente der baryonischen Mat
	%1---
	
	Verteilung des heißen Gases (der Hauptkomponente der baryonischen Materie) getrennt erscheint. Im $\Lambda$CDM-Modell wird dies als direkte Evidenz für dunkle Materie interpretiert. Im T0-Modell müsste dies durch eine komplexere Wechselwirkung zwischen dem dunklen Energiefeld und den verschiedenen Komponenten der baryonischen Materie erklärt werden.
	
	\section{Quantitative Vorhersagen und experimentelle Tests}
	
	Um zwischen dem T0-Modell mit Massenvariation und dem Standardmodell mit dunkler Materie zu unterscheiden, sind präzise quantitative Vorhersagen und experimentelle Tests notwendig.
	
	\subsection{Tully-Fisher-Beziehung}
	
	Die Tully-Fisher-Beziehung verknüpft die Leuchtkraft $L$ einer Spiralgalaxie mit ihrer Rotationsgeschwindigkeit $v_{max}$ und wird empirisch beschrieben durch:
	
	\begin{equation}
		L \propto v_{max}^{4}
	\end{equation}
	
	Im Standardmodell ist diese Beziehung eine Konsequenz der Dynamik von Galaxien mit dunkler Materie. Im T0-Modell würde die Massenvariation diese Beziehung modifizieren zu:
	
	\begin{equation}
		L \propto v_{max}^{4+\epsilon}
	\end{equation}
	
	wobei $\epsilon$ ein kleiner Korrekturterm ist, der von der Kopplungskonstante $\beta$ abhängt:
	
	\begin{equation}
		\epsilon \approx \frac{\beta^2 \rho_0 r_0^2}{m_0 G}
	\end{equation}
	
	Eine präzise Messung dieser Abweichung könnte einen direkten Test des T0-Modells ermöglichen.
	
	\subsection{Massenabhängige Gravitationslinseneffekte}
	
	Ein wichtiger Unterschied zwischen beiden Modellen betrifft den Gravitationslinseneffekt. Im T0-Modell ist die effektive Masse eines Objekts massenabhängig, was zu einer modifizierten Linsengleichung führt:
	
	\begin{equation}
		\alpha_{lens} \propto \int \nabla(\Phi_{Newton} + \beta\phi_{DE}) dz
	\end{equation}
	
	Für ausgedehnte Objekte, wie Galaxienhaufen, ergibt dies ein Linsenprofil, das sich von der Vorhersage des $\Lambda$CDM-Modells unterscheidet, insbesondere bei den äußeren Radien. Eine detaillierte Analyse von Gravitationslinsen könnte diese Unterschiede aufdecken.
	
	\subsection{Gas-reiche vs. gas-arme Galaxien}
	
	Eine spezifische Vorhersage des T0-Modells betrifft Galaxien mit unterschiedlichen Gas-zu-Stern-Verhältnissen. Da die effektive Massenvariation mit der baryonischen Dichte zusammenhängt, sollten gas-reiche Galaxien systematisch andere Rotationskurven aufweisen als gas-arme Galaxien gleicher Gesamtmasse.
	
	\begin{equation}
		\frac{v^2_{gas-rich}(r)}{v^2_{gas-poor}(r)} = 1 + \delta(r)
	\end{equation}
	
	wobei $\delta(r)$ eine Funktion ist, die von der radialen Verteilung des Gases und der Sterne abhängt. Empirisch könnte dies durch die Analyse von Galaxien mit ähnlicher stellarer Masse, aber unterschiedlichen HI-Gasmassen getestet werden.
	
	\section{Mathematische Herleitung der Feldlösungen}
	
	Wir werden nun die Feldgleichungen des dunklen Energiefeldes im Detail lösen und die Konsequenzen für die Galaxiendynamik ableiten.
	
	\subsection{Statische, sphärisch symmetrische Lösungen}
	
	Für ein statisches, sphärisch symmetrisches System vereinfacht sich die Feldgleichung zu:
	
	\begin{equation}
		\frac{1}{r^2}\frac{d}{dr}\left(r^2\frac{d\phi_{DE}}{dr}\right) = m_{\phi}^2\phi_{DE} + 4\lambda\phi_{DE}^3 + \frac{\beta}{M_{Pl}}\rho_{baryon}(r)
	\end{equation}
	
	Diese Gleichung hat keine allgemeine analytische Lösung, kann aber für bestimmte Grenzfälle gelöst werden.
	
	\subsubsection{Fall 1: Masseloses Feld ohne Selbstwechselwirkung}
	
	Für $m_{\phi} = 0$ und $\lambda = 0$ (masseloses Feld ohne Selbstwechselwirkung) vereinfacht sich die Gleichung zu:
	
	\begin{equation}
		\frac{1}{r^2}\frac{d}{dr}\left(r^2\frac{d\phi_{DE}}{dr}\right) = \frac{\beta}{M_{Pl}}\rho_{baryon}(r)
	\end{equation}
	
	Diese Poisson-Gleichung hat die allgemeine Lösung:
	
	\begin{equation}
		\phi_{DE}(r) = -\frac{\beta}{4\pi M_{Pl}} \int \frac{\rho_{baryon}(r')}{|r-r'|} d^3r'
	\end{equation}
	
	Für eine exponentiell abfallende baryonische Dichte $\rho_{baryon}(r) = \rho_0 e^{-r/r_0}$ ergibt die Integration:
	
	\begin{equation}
		\phi_{DE}(r) = -\frac{\beta\rho_0 r_0^2}{M_{Pl}r}(1 - (1 + \frac{r}{r_0})e^{-r/r_0})
	\end{equation}
	
	Für $r \gg r_0$ (außerhalb des galaktischen Kerns) vereinfacht sich dies zu:
	
	\begin{equation}
		\phi_{DE}(r) \approx -\frac{\beta\rho_0 r_0^2}{M_{Pl}r}
	\end{equation}
	
	Diese $1/r$-Abhängigkeit des Feldes erzeugt jedoch nicht die gewünschte $1/r^2$-Abhängigkeit der effektiven Dichte, die für flache Rotationskurven benötigt wird.
	
	\subsubsection{Fall 2: Mit Selbstwechselwirkung}
	
	Wenn wir Selbstwechselwirkung einbeziehen ($\lambda \neq 0$), wird die Feldgleichung nichtlinear:
	
	\begin{equation}
		\frac{1}{r^2}\frac{d}{dr}\left(r^2\frac{d\phi_{DE}}{dr}\right) = 4\lambda\phi_{DE}^3 + \frac{\beta}{M_{Pl}}\rho_{baryon}(r)
	\end{equation}
	
	Für $r \gg r_0$, wo $\rho_{baryon}(r) \approx 0$, suchen wir eine Lösung der Form $\phi_{DE}(r) \propto r^{-\alpha}$. Einsetzen in die Gleichung und Koeffizientenvergleich ergibt $\alpha = 1/2$, also:
	
	\begin{equation}
		\phi_{DE}(r) \approx \left(\frac{1}{8\lambda}\right)^{1/3} r^{-1/2} \quad \text{für } r \gg r_0
	\end{equation}
	
	Mit diesem Feldprofil erhalten wir eine effektive Dichte:
	
	\begin{equation}
		\rho_{eff}(r) \propto \phi_{DE}^2(r) \propto r^{-1}
	\end{equation}
	
	Dies ergibt immer noch keine flache Rotationskurve. Wir benötigen daher eine komplexere Feldtheorie oder einen anderen Mechanismus.
	
	\subsection{Modifizierte Feldtheorie mit nicht-minimaler Kopplung}
	
	Eine vielversprechende Erweiterung ist die Einführung einer nicht-minimalen Kopplung zwischen dem dunklen Energiefeld und der Raumzeitkrümmung. Wir erweitern die Lagrange-Dichte um einen Term, der das dunkle Energiefeld an den Ricci-Skalar $R$ koppelt:
	
	\begin{equation}
		\mathcal{L} = -\frac{1}{2}\partial_\mu \phi_{DE} \partial^\mu \phi_{DE} - V(\phi_{DE}) - \frac{\beta}{M_{Pl}}\phi_{DE}T^{\mu}_{\mu} - \frac{1}{2}\xi \phi_{DE}^2 R
	\end{equation}
	
	wobei $\xi$ eine dimensionslose Kopplungskonstante ist. Diese Kopplung führt zu einer modifizierten Feldgleichung:
	
	\begin{equation}
		\nabla^2 \phi_{DE} - \xi R \phi_{DE} = \frac{dV}{d\phi_{DE}} + \frac{\beta}{M_{Pl}}\rho_{baryon}
	\end{equation}
	
	In einer schwachen Gravitationsfeldnäherung ist $R \approx -8\pi G \rho_{baryon}$, was zu einer zusätzlichen effektiven Massenkopplung führt.
	
	Mit dieser erweiterten Kopplung kann ein Feldprofil entstehen, das für große $r$ wie $\phi_{DE}(r) \propto r^{-1}$ abfällt und eine effektive Dichte $\rho_{eff}(r) \propto r^{-2}$ erzeugt, was zu flachen Rotationskurven führt.
	
	\section{Galaxiendynamik und beobachtbare Signaturen}
	
	In diesem Abschnitt analysieren wir detailliert, wie sich das T0-Modell auf verschiedene Aspekte der Galaxiendynamik auswirkt und welche beobachtbaren Signaturen zu erwarten sind.
	
	\subsection{Struktur und Stabilität von Galaxienscheiben}
	
	Die Stabilität einer Galaxienscheibe wird durch den Toomre-Parameter $Q$ beschrieben:
	
	\begin{equation}
		Q = \frac{\sigma_r \kappa}{\pi G \Sigma}
	\end{equation}
	
	wobei $\sigma_r$ die radiale Geschwindigkeitsdispersion der Sterne, $\kappa$ die Epizyklfrequenz und $\Sigma$ die Oberflächendichte der Scheibe ist. Im T0-Modell wird dieser Parameter modifiziert zu:
	
	\begin{equation}
		Q_{T_0} = \frac{\sigma_r \kappa}{\pi G_{eff}(r) \Sigma}
	\end{equation}
	
	Da $G_{eff}(r)$ mit dem Radius zunimmt, sinkt $Q_{T_0}$ in den äußeren Regionen der Galaxie stärker als im Standardmodell, was zu unterschiedlichen Stabilitätsbedingungen führt. Dies könnte sich in der Spiralstruktur und der Sternentstehungsaktivität in den äußeren Regionen von Galaxien manifestieren.
	
	\subsection{Zwerggalaxien und das "Too-Big-To-Fail"-Problem}
	
	Zwerggalaxien stellen einen kritischen Test für Modelle der dunklen Materie dar. Im $\Lambda$CDM-Modell gibt es mehrere bekannte Probleme, darunter das "Too-Big-To-Fail"-Problem: Simulationen sagen Zwerggalaxien mit höheren zentralen Dichten voraus, als beobachtet werden.
	
	Im T0-Modell würde die effektive Massendichte in Zwerggalaxien anders skalieren. Die Rotationsgeschwindigkeit in einer Zwerggalaxie wäre:
	
	\begin{equation}
		v^2(r) = \frac{GM_{baryon}(r)}{r} + \frac{\kappa}{\rho_{0,dwarf}r}
	\end{equation}
	
	wobei $\rho_{0,dwarf}$ die charakteristische Dichte der Zwerggalaxie ist. Da Zwerggalaxien typischerweise eine niedrigere Dichte haben als massive Galaxien, würde der zweite Term stärker dominieren, was zu einem flacheren effektiven Dichteprofil führt.
	
	Eine quantitative Vorhersage ist, dass die Geschwindigkeitsdispersion $\sigma_v$ in Zwerggalaxien mit hohem Gas-zu-Stern-Verhältnis systematisch niedriger sein sollte als im $\Lambda$CDM-Modell vorhergesagt:
	
	\begin{equation}
		\sigma_{v,T_0} \approx \sigma_{v,\Lambda CDM} \times \left(1 - \gamma \frac{M_{gas}}{M_{star}}\right)
	\end{equation}
	
	wobei $\gamma$ ein Parameter ist, der von der genauen Form der Kopplung abhängt.
	
	\subsection{Galaktische Harfen und Strukturbildung}
	
	Ein weiterer Bereich, in dem das T0-Modell getestet werden kann, ist die Strukturbildung auf galaktischen und kosmologischen Skalen. Im $\Lambda$CDM-Modell dient dunkle Materie als Keim für die Strukturbildung, wobei baryonische Materie den Gravitationspotentialtöpfen der dunklen Materie folgt.
	
	Im T0-Modell ist dieser Prozess komplexer: Das dunkle Energiefeld koppelt an die baryonische Materie und erzeugt eine effektive Massenvariation, die die Strukturbildung beeinflusst. Eine spezifische Vorhersage betrifft die Form und Dichte von "Galaktischen Harfen" - Filamentstrukturen, die entstehen, wenn Galaxien durch einen Galaxienhaufen fallen.
	
	Die charakteristische Dichte und Ausdehnung dieser Filamente sollte im T0-Modell systematisch anders sein als im $\Lambda$CDM-Modell:
	
	\begin{equation}
		\rho_{filament,T_0}(r) = \rho_{filament,\Lambda CDM}(r) \times \left(1 + \Delta(r)\right)
	\end{equation}
	
	wobei $\Delta(r)$ eine Funktion ist, die von der radialen Entfernung zum Galaxienzentrum abhängt und durch die spezifische Form der Kopplung zwischen dem dunklen Energiefeld und der baryonischen Materie bestimmt wird.
	
	\section{Vereinheitlichte mathematische Formulierung}
	
	Wir präsentieren nun eine einheitliche mathematische Formulierung des T0-Modells, die sowohl die galaktische Dynamik als auch die kosmologischen Aspekte umfasst.
	
	\subsection{Erweiterte Lagrange-Dichte}
	
	Die vollständige Lagrange-Dichte des T0-Modells, die alle relevanten Wechselwirkungen erfasst, lautet:
	
	\begin{equation}
		\mathcal{L}_{total} = \mathcal{L}_{gravity} + \mathcal{L}_{DE} + \mathcal{L}_{matter} + \mathcal{L}_{interaction}
	\end{equation}
	
	wobei:
	
	\begin{align}
		\mathcal{L}_{gravity} &= \frac{1}{16\pi G}R\\
		\mathcal{L}_{DE} &= -\frac{1}{2}\partial_\mu \phi_{DE} \partial^\mu \phi_{DE} - V(\phi_{DE}) - \frac{1}{2}\xi \phi_{DE}^2 R\\
		\mathcal{L}_{matter} &= \mathcal{L}_{baryon}\\
		\mathcal{L}_{interaction} &= -\frac{\beta}{M_{Pl}}\phi_{DE}T^{\mu}_{\mu}
	\end{align}
	
	Hier ist $R$ der Ricci-Skalar, $\phi_{DE}$ das dunkle Energiefeld, $V(\phi_{DE})$ sein Potential, $\xi$ die nicht-minimale Kopplungskonstante, und $T^{\mu}_{\mu}$ die Spur des Energie-Impuls-Tensors der baryonischen Materie.
	
	\subsection{Feldgleichungen}
	
	Aus der Lagrange-Dichte ergeben sich die Feldgleichungen:
	
	\begin{align}
		G_{\mu\nu} &= 8\pi G T_{\mu\nu}^{eff}\\
		\Box\phi_{DE} - \xi R \phi_{DE} - \frac{dV}{d\phi_{DE}} &= \frac{\beta}{M_{Pl}}T^{\mu}_{\mu}
	\end{align}
	
	wobei $G_{\mu\nu}$ der Einstein-Tensor und $T_{\mu\nu}^{eff}$ der effektive Energie-Impuls-Tensor ist, der sowohl die baryonische Materie als auch die Beiträge des dunklen Energiefeldes umfasst:
	
	\begin{equation}
		T_{\mu\nu}^{eff} = T_{\mu\nu}^{baryon} + T_{\mu\nu}^{DE} + T_{\mu\nu}^{int}
	\end{equation}
	
	Für die galaktische Dynamik kann dies in der Näherung schwacher Felder auf eine modifizierte Poisson-Gleichung reduziert werden:
	
	\begin{equation}
		\nabla^2 \Phi = 4\pi G \rho_{baryon} + 4\pi G \rho_{eff,DE}
	\end{equation}
	
	wobei $\Phi$ das Gravitationspotential und $\rho_{eff,DE}$ die effektive Dichte ist, die durch das dunkle Energiefeld induziert wird.
	
	\subsection{Kosmologische Feldgleichungen}
	
	Auf kosmologischen Skalen kann die Metrik als Robertson-Walker-Metrik angesetzt werden:
	
	\begin{equation}
		ds^2 = -dt^2 + a^2(t)[dr^2 + r^2(d\theta^2 + \sin^2\theta d\phi^2)]
	\end{equation}
	
	Im T0-Modell bleibt die Zeit absolut, aber der Skalenfaktor $a(t)$ muss neu interpretiert werden. Statt einer echten Expansion beschreibt er eine effektive Skalierung aufgrund von Massenvariation. Die modifizierten Friedmann-Gleichungen lauten:
	
	\begin{align}
		\left(\frac{\dot{a}}{a}\right)^2 &= \frac{8\pi G}{3}\rho_{eff}\\
		\frac{\ddot{a}}{a} &= -\frac{4\pi G}{3}(\rho_{eff} + 3p_{eff})
	\end{align}
	
	wobei $\rho_{eff}$ und $p_{eff}$ die effektive Energiedichte und der effektive Druck sind, die sowohl baryonische Materie als auch das dunkle Energiefeld umfassen.
	
	Die entscheidende Unterscheidung zum Standardmodell ist, dass diese Gleichungen keine tatsächliche Expansion beschreiben, sondern eine scheinbare Expansion durch Massenvariation. Die Rotverschiebung ergibt sich dann als:
	
	\begin{equation}
		1 + z = \frac{E_0}{E} = \frac{m_0}{m} = \frac{1}{a(t)}
	\end{equation}
	
	\section{Zusammenfassung und Schlussfolgerungen}
	
	In dieser Arbeit haben wir eine umfassende mathematische Analyse der Galaxiendynamik im Rahmen des T0-Modells entwickelt, das auf den Grundannahmen der absoluten Zeit und der variablen Masse basiert. Im Gegensatz zum Standardmodell der Kosmologie ($\Lambda$CDM), das die Existenz dunkler Materie als separate Komponente postuliert, erklärt das T0-Modell die beobachteten dynamischen Effekte durch eine effektive Massenvariation, die durch Kopplung mit einem dunklen Energiefeld entsteht.
	
	\subsection{Kernresultate}
	
	Die wichtigsten Ergebnisse unserer Analyse sind:
	
	\begin{enumerate}
		\item Eine mathematisch konsistente Feldtheorie für das dunkle Energiefeld, das an die baryonische Materie koppelt und eine effektive Massenvariation induziert.
		
		\item Eine quantitative Herleitung der Parameter, die für die Erklärung flacher Rotationskurven in Galaxien erforderlich sind, mit einer dimensionslosen Kopplungskonstante $\hat{\beta} \approx 10^{-3}$.
		
		\item Eine detaillierte Analyse der Unterschiede in den Vorhersagen für verschiedene Galaxientypen, einschließlich Spiralgalaxien, elliptischer Galaxien und LSB-Galaxien.
		
		\item Eine Untersuchung der Stabilitätsbedingungen für das Modell und der Wachstumsrate von Störungen.
		
		\item Eine vereinheitlichte Formulierung, die sowohl die galaktische Dynamik als auch die kosmologischen Aspekte des T0-Modells umfasst.
		
		\item Konkrete Vorschläge für experimentelle Tests, die zwischen dem T0-Modell und dem Standardmodell unterscheiden könnten.
	\end{enumerate}
	
	\subsection{Vergleich mit dem $\Lambda$CDM-Modell}
	
	Beide Modelle können die grundlegenden Beobachtungen flacher Rotationskurven erklären, unterscheiden sich jedoch in ihrer physikalischen Interpretation und in einigen spezifischen Vorhersagen:
	
	\begin{tcolorbox}[colback=yellow!5!white,colframe=yellow!75!black,title=Vergleich der Modelle]
		\begin{tabular}{|p{0.45\textwidth}|p{0.45\textwidth}|}
			\hline
			\textbf{$\Lambda$CDM-Modell} & \textbf{T0-Modell} \\
			\hline
			Dunkle Materie als eigenständige Teilchenart & Keine separate dunkle Materie, sondern effektive Massenvariation \\
			\hline
			Zeit ist relativ, Masse konstant & Zeit ist absolut, Masse variabel \\
			\hline
			Rotverschiebung durch Expansion & Rotverschiebung durch Energieverlust \\
			\hline
			NFW-Dichteprofil ($\rho \sim r^{-1}$ im Zentrum, $\rho \sim r^{-3}$ außen) & Effektives Dichteprofil mit $\rho_{eff} \sim r^{-2}$ für große $r$ \\
			\hline
			Universelle Dunkle-Materie-Verteilung & Umgebungsabhängige Effektive Massenvariation \\
			\hline
		\end{tabular}
	\end{tcolorbox}
	
	\subsection{Ausblick}
	
	Das T0-Modell bietet eine konzeptionell elegante Alternative zum Standardmodell der Kosmologie, indem es fundamentale Annahmen über Zeit und Masse neu interpretiert. Wir haben gezeigt, dass dieser Ansatz eine mathematisch konsistente Beschreibung der Galaxiendynamik ermöglicht, die mit den Beobachtungen übereinstimmt.
	
	Die entscheidende Frage ist, ob das Modell durch kritische experimentelle Tests bestätigt werden kann. Die vorgeschlagenen Tests, insbesondere die Analyse von Galaxien mit unterschiedlichen Gas-zu-Stern-Verhältnissen und die detaillierte Messung von Gravitationslinsenprofilen, bieten vielversprechende Möglichkeiten, zwischen den Modellen zu unterscheiden.
	
	Unabhängig vom Ausgang dieser Tests trägt die mathematische Formulierung des T0-Modells zu einem tieferen Verständnis der fundamentalen Konzepte von Zeit, Masse und Energie in der modernen Physik bei und eröffnet neue Perspektiven für die Interpretation kosmischer Phänomene.
	
	\begin{thebibliography}{99}
		
		\bibitem{pascher_zeit_2025} Pascher, J. (2025). Zeit als emergente Eigenschaft in der Quantenmechanik: Eine Verbindung zwischen Relativitätstheorie, Feinstrukturkonstante und Quantendynamik.
		
		\bibitem{pascher_math_2025} Pascher, J. (2025). Mathematische Formulierung des Higgs-Mechanismus in der Zeit-Masse-Dualität. 28. März 2025.
		
		\bibitem{pascher_kompl_2025} Pascher, J. (2025). Komplementäre Erweiterungen der Physik: Absolute Zeit und Intrinsische Zeit. 24. März 2025.
		
		\bibitem{pascher_wesentl_2025} Pascher, J. (2025). Wesentliche mathematische Formalismen der Zeit-Masse-Dualitätstheorie mit Lagrange-Dichten. 29. März 2025.
		
		\bibitem{pascher_verein_2025} Pascher, J. (2025). Vereinheitlichung des T0-Modells: Grundlagen, Dunkle Energie und Galaxiendynamik. 27. März 2025.
		
		\bibitem{rotation} Rubin, V. C., Ford, W. K. (1970). Rotation of the Andromeda Nebula from a Spectroscopic Survey of Emission Regions. The Astrophysical Journal, 159, 379.
		
		\bibitem{nfw} Navarro, J. F., Frenk, C. S., White, S. D. M. (1996). The Structure of Cold Dark Matter Halos. The Astrophysical Journal, 462, 563.
		
		\bibitem{tully} Tully, R. B., Fisher, J. R. (1977). A new method of determining distances to galaxies. Astronomy and Astrophysics, 54, 661.
		
		\bibitem{bullet} Clowe, D., Bradač, M., Gonzalez, A. H., et al. (2006). A Direct Empirical Proof of the Existence of Dark Matter. The Astrophysical Journal, 648, L109.
		
		\bibitem{supernova} Perlmutter, S., et al. (1999). Measurements of $\Omega$ and $\Lambda$ from 42 High-Redshift Supernovae. The Astrophysical Journal, 517, 565.
		
		\bibitem{riess} Riess, A. G., et al. (1998). Observational Evidence from Supernovae for an Accelerating Universe and a Cosmological Constant. The Astronomical Journal, 116, 1009.
		
		\bibitem{planck} Planck Collaboration. (2020). Planck 2018 results. VI. Cosmological parameters. Astronomy \& Astrophysics, 641, A6.
		
	\end{thebibliography}
	
\end{document}