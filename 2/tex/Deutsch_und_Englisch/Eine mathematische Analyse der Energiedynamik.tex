\documentclass[a4paper,12pt]{article}
\usepackage[utf8]{inputenc}
\usepackage[T1]{fontenc}
\usepackage{lmodern}
\usepackage[ngerman]{babel} % Deutsch
\usepackage{amsmath, amssymb, amsthm, physics}
\usepackage{graphicx}
\usepackage{xcolor}
\usepackage{tikz}
\usepackage{setspace}
\usepackage{tcolorbox}
\usepackage{booktabs}
\usepackage{siunitx}
\DeclareSIUnit{\year}{yr} % Definition der Einheit "Jahr"
\DeclareSIUnit{\parsec}{pc} % Definition der Einheit "Parsec"
\usepackage{geometry}
\usepackage{tocloft}
\usepackage{fancyhdr} % Für Kopf- und Fußzeilen

\geometry{margin=2cm}

% Kopf- und Fußzeilen
\pagestyle{fancy}
\fancyhf{}
\fancyhead[L]{Johann Pascher}
\fancyhead[R]{Zeit-Masse-Dualität}
\fancyfoot[C]{\thepage}
\renewcommand{\headrulewidth}{0.4pt}
\renewcommand{\footrulewidth}{0.4pt}

% Inhaltsverzeichnis-Styling
\renewcommand{\cftsecfont}{\color{blue}}
\renewcommand{\cftsubsecfont}{\color{blue}}
\renewcommand{\cftsecpagefont}{\color{blue}}
\renewcommand{\cftsubsecpagefont}{\color{blue}}
\setlength{\cftsecindent}{1cm}
\setlength{\cftsubsecindent}{2cm}

% Colored links in table of contents and document
\usepackage{hyperref}
\hypersetup{
	colorlinks=true,
	linkcolor=blue,
	filecolor=blue,
	citecolor=blue, 
	urlcolor=blue,
	bookmarks=true,
	bookmarksopen=true,
	pdftitle={Dunkle Energie im T0-Modell: Eine mathematische Analyse der Energiedynamik},
	pdfauthor={Johann Pascher},
}

% cleveref must be loaded after hyperref
\usepackage{cleveref}

% Theorem styles
\newtheorem{theorem}{Satz}[section]
\newtheorem{lemma}[theorem]{Lemma}
\newtheorem{proposition}[theorem]{Proposition}
\newtheorem{corollary}[theorem]{Korollar}

\theoremstyle{definition}
\newtheorem{definition}{Definition}[theorem]
\newtheorem{example}{Beispiel}

\theoremstyle{remark}
\newtheorem{remark}{Bemerkung}
\renewcommand{\proofname}{Beweis}

% Custom commands (konsistent mit anderen Dokumenten)
\newcommand{\Tfield}{T(x)} % Intrinsisches Zeitfeld
\newcommand{\DcovT}[1]{\Tfield D_\mu #1 + #1 \partial_\mu \Tfield}
\newcommand{\DhiggsT}{\Tfield (\partial_\mu + ig A_\mu) \Phi + \Phi \partial_\mu \Tfield}
\newcommand{\betaT}{\beta_{\text{T}}}
\newcommand{\alphaEM}{\alpha_{\text{EM}}}
\newcommand{\alphaW}{\alpha_{\text{W}}}
\newcommand{\Mpl}{M_{\text{Pl}}}
\newcommand{\Tzerot}{T_0(\Tfield)}
\newcommand{\Tzero}{T_0}
\newcommand{\vecx}{\vec{x}}
\newcommand{\gammaf}{\gamma_{\text{Lorentz}}}

\begin{document}
	
	\title{Dunkle Energie im T0-Modell: \\Eine mathematische Analyse der Energiedynamik}
	\author{Johann Pascher}
	\date{30. März 2025}
	\maketitle
	
	\begin{abstract}
		Diese Arbeit entwickelt eine detaillierte mathematische Analyse der Dunklen Energie im Rahmen des T0-Modells mit absoluter Zeit und variabler Masse. Im Gegensatz zum \(\Lambda\)CDM-Standardmodell wird Dunkle Energie nicht als treibende Kraft der kosmischen Expansion betrachtet, sondern entsteht als dynamischer Effekt des Energieaustauschs in einem statischen Universum, vermittelt durch das intrinsische Zeitfeld \(\Tfield\). Das Dokument baut auf den Grundlagen aus \cite{pascher_params_2025} und der Gravitationstheorie aus \cite{pascher_galaxies_2025} auf, charakterisiert Energietransferraten, analysiert das radiale Dichteprofil der Dunklen Energie und erklärt die beobachtete Rotverschiebung als Folge des Energieverlusts von Photonen an dieses Feld (siehe \cite{pascher_messdifferenzen_2025}). Abschließend werden experimentelle Tests vorgeschlagen, um zwischen dieser Interpretation und dem Standardmodell zu unterscheiden.
	\end{abstract}
	
	\tableofcontents
	\newpage
	
	%======================= TEIL 1: EINFÜHRUNG ========================
	\section{Einführung}
	
	Die Entdeckung der beschleunigten kosmischen Expansion durch Supernova-Beobachtungen in den späten 1990er Jahren führte zur Einführung der Dunklen Energie als dominierende Komponente des Universums im \(\Lambda\)CDM-Standardmodell, wo sie als kosmologische Konstante (\(\Lambda\)) mit negativem Druck modelliert wird und etwa 68\% des Energiegehalts ausmacht. Diese Arbeit verfolgt einen alternativen Ansatz im T0-Modell, das auf der Zeit-Masse-Dualität basiert (siehe \cite{pascher_params_2025}, Abschnitt „Zeit-Masse-Dualität“). Hier ist die Zeit absolut, und die Masse variiert, wobei Dunkle Energie keine separate Entität ist, die Expansion antreibt, sondern ein emergenter Effekt des intrinsischen Zeitfelds \(\Tfield\). Die kosmische Rotverschiebung wird nicht durch räumliche Expansion, sondern durch den Energieverlust von Photonen an \(\Tfield\) erklärt, wie in \cite{pascher_messdifferenzen_2025} (Abschnitt „Energieverlust und Rotverschiebung“) und \cite{pascher_temp_2025} (Abschnitt „Temperaturskalierung“) ausgeführt. Im Folgenden wird die Energiedynamik mathematisch analysiert, wobei auf bestehende Herleitungen wie die Gravitationstheorie in \cite{pascher_galaxies_2025} und Parameter in \cite{pascher_params_2025} verwiesen wird. Experimentelle Tests zur Unterscheidung vom Standardmodell schließen die Arbeit ab.
	
	%======================= TEIL 2: MATHEMATISCHE GRUNDLAGEN ========================
	\section{Mathematische Grundlagen des T0-Modells}
	
	\subsection{Zeit-Masse-Dualität}
	
	Das T0-Modell postuliert eine Dualität zwischen Zeit und Masse, die zwei Beschreibungen ermöglicht:
	\begin{enumerate}
		\item \textbf{Standardbild}: Zeitdilatation (\(t' = \gamma t\)), konstante Ruhemasse (\(m_0\)).
		\item \textbf{T0-Modell}: Absolute Zeit (\(T_0\)), variable Masse (\(m = \gamma m_0\)).
	\end{enumerate}
	Die vollständige Herleitung und Transformationen sind in \cite{pascher_params_2025} (Abschnitt „Zeit-Masse-Dualität“) und \cite{pascher_galaxies_2025} (Abschnitt „Grundlagen“) gegeben. Eine Übersicht bietet die Tabelle:
	
	\begin{table}[h]
		\centering
		\begin{tabular}{|l|c|c|}
			\hline
			\textbf{Größe} & \textbf{Standardbild} & \textbf{T0-Modell} \\
			\hline
			Zeit & \(t' = \gamma t\) & \(t = \text{const.}\) \\
			Masse & \(m = \text{const.}\) & \(m = \gamma m_0\) \\
			Intrinsische Zeit & \(T = \frac{\hbar}{m c^2}\) & \(T = \frac{\hbar}{\gamma m_0 c^2}\) \\
			\hline
		\end{tabular}
		\caption{Transformationen im T0-Modell (siehe \cite{pascher_params_2025})}
	\end{table}
	
	\subsection{Intrinsische Zeit}
	
	Die intrinsische Zeit \(\Tfield\) ist zentral für das T0-Modell:
	
	\begin{definition}[Intrinsische Zeit]
		Für ein Teilchen mit Masse \(m\) gilt:
		\begin{equation}
			\Tfield = \frac{\hbar}{m c^2}
		\end{equation}
		Die Herleitung ist in \cite{pascher_params_2025} (Abschnitt „Definition der intrinsischen Zeit“) ausgeführt.
	\end{definition}
	
	\begin{corollary}[Skalarfeld]
		Als Feld gilt:
		\begin{equation}
			\Tfield = \frac{\hbar}{y \langle\Phi\rangle c^2}
		\end{equation}
		Details zum Higgs-Feld siehe \cite{pascher_higgs_2025} (Abschnitt „Higgs-Mechanismus“).
	\end{corollary}
	
	\subsection{Modifizierte Ableitungsoperatoren}
	
	Die Operatoren wurden in \cite{pascher_lagrange_2025} (Abschnitt „Lagrange-Formulierung“) eingeführt:
	
	\begin{definition}[Modifizierte Zeit-Ableitung]
		\begin{equation}
			\partial_{t/T} = T \frac{\partial}{\partial t}
		\end{equation}
	\end{definition}
	
	\begin{definition}[Kovariante Ableitung]
		Für ein Feld \(\Psi\):
		\begin{equation}
			D_{T,\mu} \Psi = \Tfield D_\mu \Psi + \Psi \partial_\mu \Tfield
		\end{equation}
	\end{definition}
	
	\begin{definition}[Higgs-Feld Ableitung]
		\begin{equation}
			D_{T,\mu} \Phi = \DhiggsT
		\end{equation}
	\end{definition}
	
	%======================= TEIL 3: FELDGLEICHUNGEN ========================
	\section{Modifizierte Feldgleichungen für Dunkle Energie}
	
	\subsection{Modifizierte Lagrangedichte}
	
	Die Lagrangedichte ist in \cite{pascher_lagrange_2025} (Abschnitt „Gesamt-Lagrangedichte“) hergeleitet:
	
	\begin{equation}
		\mathcal{L}_{\text{Total}} = \mathcal{L}_{\text{Boson}} + \mathcal{L}_{\text{Fermion}} + \mathcal{L}_{\text{Higgs-T}}
	\end{equation}
	
	Mit:
	\begin{align}
		\mathcal{L}_{\text{Boson}} &= -\frac{1}{4} \Tfield^2 F_{\mu\nu} F^{\mu\nu} \\
		\mathcal{L}_{\text{Fermion}} &= \bar{\psi} i \gamma^\mu \DcovT{\psi} - y \bar{\psi} \Phi \psi \\
		\mathcal{L}_{\text{Higgs-T}} &= (D_{T,\mu} \Phi)^\dagger (D_{T,\mu} \Phi) - \lambda (|\Phi|^2 - v^2)^2
	\end{align}
	
	\subsection{Dunkle Energie als emergenter Effekt}
	
	Dunkle Energie entsteht aus \(\Tfield\)-Variationen, wie in \cite{pascher_galaxies_2025} (Abschnitt „Emergente Gravitation“) beschrieben:
	
	\begin{equation}
		\rho_{\text{DE}}(r) \approx \frac{\kappa}{r^2}
	\end{equation}
	
	Details zu \(\kappa\) siehe \cite{pascher_params_2025} (Abschnitt „Parameterableitungen“).
	
	\subsection{Energiedichteprofil}
	
	\begin{equation}
		\rho_{\text{DE}}(r) \approx \frac{1}{2} (\nabla \Tfield)^2 \approx \frac{\kappa}{r^2}
	\end{equation}
	Siehe \cite{pascher_galaxies_2025} (Abschnitt „Energiedichte“).
	
	\subsection{Emergente Gravitation}
	
\begin{theorem}[Emergenz der Gravitation]
	\begin{equation}
		\nabla \Tfield = -\frac{\hbar}{m^2 c^2} \nabla m \sim \nabla \Phi_g
	\end{equation}
	Vollständige Herleitung in \cite{pascher_galaxies_2025} (Abschnitt „Emergente Gravitation“).
\end{theorem}

\begin{proof}
	In Regionen mit Gravitationspotential \(\Phi_g\) variiert die effektive Masse wie:
	\begin{equation}
		m(\vec{r}) = m_0 \left(1 + \frac{\Phi_g(\vec{r})}{c^2}\right)
	\end{equation}
	
	Daraus folgt:
	\begin{equation}
		\nabla m = \frac{m_0}{c^2} \nabla \Phi_g
	\end{equation}
	
	Einsetzen in den Gradienten des intrinsischen Zeitfelds:
	\begin{equation}
		\nabla \Tfield = -\frac{\hbar}{m^2 c^2} \cdot \frac{m_0}{c^2} \nabla \Phi_g
	\end{equation}
\end{proof}

Die Poisson-Gleichung lautet:
\begin{equation}
	\nabla^2 \Phi = 4\pi G \rho + \kappa^2
\end{equation}

%======================= TEIL 4: ENERGIETRANSFER UND ROTVERSCHIEBUNG ========================
\section{Energietransfer und Rotverschiebung}

\subsection{Energieverlust der Photonen}

Die Rotverschiebung resultiert aus Energieverlust, hergeleitet in \cite{pascher_messdifferenzen_2025} (Abschnitt „Energieverlust“):

\begin{equation}
	\frac{d E_{\gamma}}{d x} = -\alpha E_{\gamma}, \quad E_{\gamma}(x) = E_{\gamma,0} e^{-\alpha x}
\end{equation}

\begin{equation}
	1 + z = e^{\alpha d}, \quad \alpha = \frac{H_0}{c} \approx \SI{2.3e-18}{\per\meter}
\end{equation}

Details zu \(\alpha\) in \cite{pascher_params_2025} (Abschnitt „Ableitung von \(\alpha\)“).

\subsection{Modifizierte Energie-Impuls-Relation}

\begin{theorem}[Energie-Impuls-Relation]
	\begin{equation}
		E^2 = (p c)^2 + (m c^2)^2 + \alpha_E \frac{\hbar c}{T}
	\end{equation}
	Siehe \cite{pascher_photons_2025} (Abschnitt „Physik jenseits der Lichtgeschwindigkeit“).
\end{theorem}

\begin{theorem}[Wellenlängenabhängigkeit]
	\begin{equation}
		z(\lambda) = z_0 (1 + \betaT \ln(\lambda/\lambda_0))
	\end{equation}
	Mit \(\betaT^{\text{SI}} \approx 0.008\), \(\betaT^{\text{nat}} = 1\) (siehe \cite{pascher_params_2025}).
\end{theorem}

\subsection{Energiebilanzgleichung}

\begin{equation}
	\rho_{\text{total}} = \rho_{\text{Materie}} + \rho_{\gamma} + \rho_{\text{DE}} = \text{const.}
\end{equation}

\begin{align}
	\frac{d \rho_{\text{Materie}}}{d t} &= -\alpha c \rho_{\text{Materie}} \\
	\frac{d \rho_{\gamma}}{d t} &= -\alpha c \rho_{\gamma} \\
	\frac{d \rho_{\text{DE}}}{d t} &= \alpha c (\rho_{\text{Materie}} + \rho_{\gamma})
\end{align}
Siehe \cite{pascher_messdifferenzen_2025} (Abschnitt „Energiebilanz“).

%======================= TEIL 5: PARAMETER ========================
\section{Quantitative Bestimmung der Parameter}

\subsection{Parameter in natürlichen Einheiten}

\begin{theorem}[Schlüsselparameter]
	\begin{align}
		\kappa &= \betaT \frac{y v}{r_g} \\
		\alpha &= \frac{\lambda_h^2 v}{L_T} \\
		\betaT &= \frac{\lambda_h^2 v^2}{4\pi^2 \lambda_0 \alpha_0}
	\end{align}
	Herleitung in \cite{pascher_params_2025} (Abschnitt „Parameterableitungen“).
\end{theorem}

In SI-Einheiten:
\begin{align}
	\alpha_{\text{SI}} &\approx \SI{2.3e-18}{\per\meter} \\
	\betaT^{\text{SI}} &\approx 0.008 \\
	\kappa_{\text{SI}} &\approx \SI{4.8e-11}{\meter\per\second\squared}
\end{align}

\subsection{Gravitationspotential}

\begin{theorem}[Gravitationspotential]
	\begin{equation}
		\Phi(r) = -\frac{G M}{r} + \kappa r
	\end{equation}
	Siehe \cite{pascher_galaxies_2025} (Abschnitt „Modifiziertes Gravitationspotential“).
\end{theorem}

%======================= TEIL 6: BEOBACHTUNGEN UND TESTS ========================
\section{Dunkle Energie und kosmologische Beobachtungen}

\subsection{Supernovae Typ Ia}

\begin{equation}
	d_L = \frac{c}{H_0} \ln(1+z) (1+z)
\end{equation}
Siehe \cite{pascher_messdifferenzen_2025} (Abschnitt „Supernovae“).

\subsection{Energiedichte-Parameter}

\begin{equation}
	\Omega_{DE}^{\text{eff}} \approx \frac{3 \kappa}{R_U H_0^2} \approx 0.68
\end{equation}

\section{Experimentelle Tests}

\subsection{Feinstrukturkonstante}

\begin{equation}
	\frac{d \alpha_{\text{EM}}}{d t} \approx \SI{1e-18}{\per\year}
\end{equation}
Siehe \cite{pascher_photons_2025} (Abschnitt „Experimentelle Überprüfung“).

\subsection{Umgebungsabhängige Rotverschiebung}

\begin{equation}
	\frac{z_{\text{Cluster}}}{z_{\text{Leerraum}}} \approx 1 + 0.003
\end{equation}

\subsection{Differentielle Rotverschiebung}

\begin{equation}
	\frac{z(\lambda_1)}{z(\lambda_2)} \approx 1 + \betaT \frac{\lambda_1 - \lambda_2}{\lambda_0}
\end{equation}

\section{Ausblick und Zusammenfassung}

Das T0-Modell bietet einen Rahmen für ein statisches Universum, in dem Dunkle Energie aus \(\Tfield\) emergiert. Zukünftige Tests (z. B. Euclid) können es validieren.

\begin{thebibliography}{99}
	\bibitem{pascher_galaxies_2025} Pascher, J. (2025). \href{https://github.com/jpascher/T0-Time-Mass-Duality/tree/main/2/pdf/Deutsch/Massenvariation in Galaxien.pdf}{Massenvariation in Galaxien: Eine Analyse im T0-Modell mit emergenter Gravitation}. 30. März 2025.
	\bibitem{pascher_messdifferenzen_2025} Pascher, J. (2025). \href{https://github.com/jpascher/T0-Time-Mass-Duality/tree/main/2/pdf/Deutsch/Analyse der Messdifferenzen zwischen dem T0-Modell und dem Standardmodell.pdf}{Kompensatorische und additive Effekte: Eine Analyse der Messdifferenzen zwischen dem T0-Modell und dem \(\Lambda\)CDM-Standardmodell}. 2. April 2025.
	\bibitem{pascher_temp_2025} Pascher, J. (2025). \href{https://github.com/jpascher/T0-Time-Mass-Duality/tree/main/2/pdf/Deutsch/Anpassung von Temperatureinheiten in natürlichen Einheiten und CMB-Messungen.pdf}{Anpassung der Temperatureinheiten in natürlichen Einheiten und CMB-Messungen}. 2. April 2025.
	\bibitem{pascher_params_2025} Pascher, J. (2025). \href{https://github.com/jpascher/T0-Time-Mass-Duality/tree/main/2/pdf/Deutsch/Zeit-Masse-Dualitätstheorie (T0-Modell) Herleitung der Parameter kappa, alpha und beta.pdf}{Zeit-Masse-Dualitätstheorie (T0-Modell): Ableitung der Parameter \(\kappa\), \(\alpha\) und \(\beta\)}. 4. April 2025.
	\bibitem{pascher_lagrange_2025} Pascher, J. (2025). \href{https://github.com/jpascher/T0-Time-Mass-Duality/tree/main/2/pdf/Deutsch/Mathematische Formulierungen der Zeit-Masse-Dualitätstheorie mit Lagrange.pdf}{Von Zeitdilatation zu Massenvariation: Mathematische Kernformulierungen der Zeit-Masse-Dualitätstheorie}. 29. März 2025.
	\bibitem{pascher_higgs_2025} Pascher, J. (2025). \href{https://github.com/jpascher/T0-Time-Mass-Duality/tree/main/2/pdf/Deutsch/Mathematische Formulierung des Higgs-Mechanismus in der Zeit-Masse-Dualität.pdf}{Mathematische Formulierung des Higgs-Mechanismus in der Zeit-Masse-Dualität}. 28. März 2025.
	\bibitem{pascher_photons_2025} Pascher, J. (2025). \href{https://github.com/jpascher/T0-Time-Mass-Duality/tree/main/2/pdf/Deutsch/Dynamische Masse von Photonen und ihre Implikationen für Nichtlokalität.tex}{Dynamische Masse von Photonen und ihre Auswirkungen auf Nichtlokalität im T0-Modell}. 25. März 2025.
\end{thebibliography}

\end{document}