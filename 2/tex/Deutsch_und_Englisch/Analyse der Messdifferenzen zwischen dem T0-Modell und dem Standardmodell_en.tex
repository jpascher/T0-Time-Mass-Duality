\documentclass[a4paper,12pt]{article}
\usepackage[utf8]{inputenc}
\usepackage[T1]{fontenc}
\usepackage{lmodern}
\usepackage[english]{babel}
\usepackage{amsmath, amssymb, amsthm, physics}
\usepackage{graphicx}
\usepackage{xcolor}
\usepackage{tikz}
\usepackage{pgfplots}
\pgfplotsset{compat=1.18}
\usepackage{setspace}
\usepackage{booktabs}
\usepackage{siunitx}
\usepackage{array}
\usepackage{float}
\usepackage[section]{placeins}

\usepackage{hyperref}
\hypersetup{
	colorlinks=true,
	linkcolor=blue,
	filecolor=blue,
	citecolor=blue, 
	urlcolor=blue,
	bookmarks=true,
	bookmarksopen=true,
	pdftitle={Compensatory and Additive Effects: An Analysis of Measurement Differences between the T0 Model and the ΛCDM Standard Model},
	pdfauthor={Johann Pascher},
}

% Theorem-Definitionen bleiben unverändert

% Custom command für Tfield hinzugefügt
\newcommand{\Tfield}{T(x)}

% Repository base URL
\newcommand{\repobase}{https://github.com/jpascher/T0-Time-Mass-Duality/tree/main/2/}

\begin{document}
	
	\title{Compensatory and Additive Effects: An Analysis of Measurement Differences between the T0 Model and the $\Lambda$CDM Standard Model}
	\author{Johann Pascher}
	\date{April 2, 2025}
	\maketitle
	
	\begin{abstract}
		This document analyzes the differences in cosmological measurements between the standard model ($\Lambda$CDM) and the alternative T0 model. We investigate how the differing theoretical foundations affect distance measurements, redshifts, and the interpretation of the cosmic microwave background. Particular attention is given to whether the effects reinforce each other (act additively) or compensate. The analysis reveals a complex interplay that might explain the Hubble tension problem. At low redshifts (z $\approx$ 1), the differences are moderate, while at high redshifts (z = 1100, CMB), they become dramatic, leading to fundamentally different interpretations.
	\end{abstract}
	
	\tableofcontents
	\newpage
	
	\section{Introduction}
	
	The cosmological standard model ($\Lambda$CDM) and the alternative T0 model offer fundamentally different explanations for the same astronomical observations. While $\Lambda$CDM is based on an expanding universe, the T0 model postulates a static universe with absolute time and variable mass. This work examines how these differing theoretical foundations impact cosmological measurements and how these effects either reinforce or compensate each other.
	
	\subsection{The T0 Model}
	
	In the {\small\href{\repobase/pdf/English/Wesentliche mathematische Formalismen der Zeit-Masse-Dualitätstheorie mit Lagrange-Dichten_en.pdf}{T0 model}}, time is considered absolute, while mass varies as \( m = \frac{\hbar}{\Tfield c^2} \), where \( \Tfield \) is the intrinsic time field coupled to the Higgs field via \( \Tfield = \frac{\hbar}{y \langle \Phi \rangle c^2} \) \cite{pascher_galaxies_2025}. The redshift arises from the spatial variation of \( \Tfield \), which causes an energy loss of photons:
	
	\begin{equation}
		1 + z = \frac{\Tfield}{\Tfield_0},
	\end{equation}
	
	where \( \Tfield_0 \) is the value at the observer's location. This variation can be expressed as \( \Tfield = \Tfield_0 e^{-\alpha d} \), with \( \alpha = H_0/c \), leading to:
	
	\begin{equation}
		1 + z = e^{\alpha d},
	\end{equation}
	
	where \( d \) is the physical distance and \( H_0 \) is the Hubble constant, reinterpreted as the rate of \( \Tfield \)'s spatial change rather than an expansion parameter. Unlike the \(\Lambda\)CDM model, where dark energy (\( \Lambda \)) drives cosmic acceleration, in the T0-model, \( \Tfield \) serves as an effective field that accounts for redshift without expansion, while its gradients also explain emergent gravitation \cite{pascher_galaxies_2025}.
	
	\subsection{T0 Model}
	
	In the T0-model, the CMB temperature exhibits a slight modification due to the dynamics of \( \Tfield \):
	
	\begin{equation}
		T(z) = T_0 (1 + z) (1 + \beta \ln(1 + z)),
	\end{equation}
	
	with \( \beta \approx 0.008 \), leading to a subtle deviation from the standard model prediction \( T(z) = T_0 (1 + z) \). This arises because \( \Tfield \) governs both mass variation and redshift, as detailed in \cite{pascher_galaxies_2025}. The parameter \( \beta \) reflects the logarithmic influence of \( \Tfield \)'s spatial variation on photon energy loss, consistent with \( 1 + z = e^{\alpha d} \).
	
	% Rest des Dokuments bleibt unverändert
	
	\begin{thebibliography}{9}
		\bibitem{pascher_galaxies_2025} Pascher, J. (2025). \href{\repobase/pdf/English/Mass Variation in Galaxies - An Analysis in the T0-Model with Emergent Gravitation.pdf}{Mass Variation in Galaxies: An Analysis in the T0-Model with Emergent Gravitation}. March 30, 2025.
		\bibitem{Planck2018} Planck Collaboration, Aghanim, N., et al. (2020). \textit{Planck 2018 results. VI. Cosmological parameters}. Astronomy \& Astrophysics, 641, A6. DOI: 10.1051/0004-6361/201833910.
		% Weitere Einträge wie im Original
	\end{thebibliography}
	
\end{document}