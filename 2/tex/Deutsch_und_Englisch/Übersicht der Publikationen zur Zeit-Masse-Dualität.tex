\documentclass[a4paper,12pt]{article}
\usepackage[utf8]{inputenc}
\usepackage[T1]{fontenc}
\usepackage[german]{babel}
\usepackage{lmodern}
\usepackage{csquotes}
\usepackage{hyperref}
\usepackage{xcolor}
\usepackage{geometry}
\usepackage{booktabs}
\usepackage{array}
\usepackage{tabularx}
\usepackage{fancyhdr}

\geometry{a4paper, margin=2.5cm}
\hypersetup{
	colorlinks=true,
	linkcolor=blue,
	filecolor=magenta,      
	urlcolor=blue,
	pdftitle={Übersicht der Publikationen zur Zeit-Masse-Dualität},
	pdfauthor={Johann Pascher},
	pdfcreator={LaTeX}
}

% Repository base URL
\newcommand{\repobase}{https://github.com/jpascher/T0-Time-Mass-Duality/tree/main/2/}

\pagestyle{fancy}
\fancyhf{}
\rhead{Johann Pascher}
\lhead{Zeit-Masse-Dualität}
\cfoot{\thepage}

\title{Übersicht der Publikationen zur Zeit-Masse-Dualität \\ \Large{Ein theoretischer Rahmen für die Erweiterung der modernen Physik}}
\author{Johann Pascher}
\date{März 2025}

\begin{document}
	
	\maketitle
	
	\begin{abstract}
		Diese Übersicht präsentiert eine Sammlung von Arbeiten, die einen neuen theoretischen Rahmen zur Erweiterung der Physik entwickeln: die Zeit-Masse-Dualität. Dieser Ansatz schlägt vor, Zeit und Masse neu zu betrachten, und bietet Lösungen für offene Fragen in der Quantenmechanik, Quantenfeldtheorie und Kosmologie – etwa zur Nichtlokalität oder dunklen Energie. Die Dokumente bilden ein Programm, das von Grundideen über mathematische Modelle bis zu praktischen Anwendungen reicht.
	\end{abstract}
	
	\section{Einleitung}
	Die folgenden Publikationen entwickeln die Zeit-Masse-Dualität, eine neue Sicht auf Zeit und Masse mit weitreichenden Folgen für die Physik. Sie sind in fünf Abschnitte gegliedert: Ausgangspunkt, konzeptionelle Grundlagen, mathematische Formalisierung, Anwendungen und Grenzfragen. Alle Dateien sind im Repository unter \url{\repobase} verfügbar.
	
	\section{Ausgangspunkt: Feldtheorie als Initialidee}
	
	\subsection{\small\href{\repobase/pdf/Deutsch/Feldtheorie und Quantenkorrelationen.pdf}{Feldtheorie und Quantenkorrelationen}}
	\textit{(356.638 Bytes, 31.03.2025)}
	
	Dieses Dokument ist der Startpunkt. Es fragt, warum Teilchen über große Distanzen sofort verbunden scheinen (Nichtlokalität), und schlägt eine neue Feldstruktur als Antwort vor. Diese Idee führt später zur Zeit-Masse-Dualität.
	
	\section{Konzeptionelle Grundlage und Motivation}
	
	\subsection{\small\href{\repobase/pdf/Deutsch/Die Notwendigkeit einer Erweiterung der Standard-Quantenmechanik und Quantenfeldtheorie.pdf}{Die Notwendigkeit einer Erweiterung der Standard-Quantenmechanik \\ und Quantenfeldtheorie}}
	\textit{(276.670 Bytes, 31.03.2025)}
	
	Es zeigt Schwächen der üblichen Theorien, z. B. bei der Verbindung von Quantenmechanik und Gravitation, und führt die Zeit-Masse-Dualität als Lösung ein.
	
	\subsection{\small\href{\repobase/pdf/Deutsch kurzgefasst/Kurzgefasst - Komplementärer Dualismus in der Physik - Von Welle-Teilchen zum Zeit-Masse-Konzept.pdf}{Kurzgefasst - Komplementärer Dualismus in der Physik - Von Welle-Teilchen zum Zeit-Masse-Konzept}}
	\textit{(149.923 Bytes, 31.03.2025)}
	
	Dieses Dokument erklärt die Zeit-Masse-Dualität einfach: Wie Licht Welle und Teilchen zugleich ist, könnten Zeit und Masse zwei Seiten einer Medaille sein.
	
	\subsection{\small\href{\repobase/pdf/Deutsch/Eine neue Perspektive auf Zeit und Raum Johann Paschers revolutionäre Ideen.pdf}{Eine neue Perspektive auf Zeit und Raum: Johann Paschers revolutionäre Ideen}}
	\textit{(242.204 Bytes, 31.03.2025)}
	
	Für alle verständlich, auch ohne Mathematik: Es stellt das T0-Modell vor, bei dem Zeit fest ist und Masse sich ändert – anders als bei Einstein. Es erklärt Rätsel wie die sofortige Verbindung von Teilchen oder die Ausdehnung des Universums auf einfache Weise.
	
	\section{Mathematische Formalisierung}
	
	\subsection{\small\href{\repobase/pdf/Deutsch/Wesentliche mathematische Formalismen der Zeit-Masse-Dualitätstheorie mit Lagrange-Dichten.pdf}{Wesentliche mathematische Formalismen der Zeit-Masse-Dualitätstheorie mit Lagrange-Dichten}}
	\textit{(370.669 Bytes, 31.03.2025)}
	
	Hier beginnt die präzise Ausarbeitung. Mit einfachen Regeln (alles auf 1 setzen) wird die Theorie mathematisch beschrieben, z. B. mit der Lagrange-Methode.
	
	\subsection{\small\href{\repobase/pdf/Deutsch/Mathematische Formulierungen der Zeit-Masse-Dualitätstheorie mit Lagrange.pdf}{Mathematische Formulierungen der Zeit-Masse-Dualitätstheorie mit Lagrange}}
	\textit{(559.012 Bytes, 31.03.2025)}
	
	Es vertieft die Modelle für Teilchen wie das Higgs-Feld und zeigt, wie die Theorie mathematisch funktioniert.
	
	\subsection{\small\href{\repobase/pdf/Deutsch/Mathematische Formulierung des Higgs-Mechanismus in der Zeit-Masse-Dualität.pdf}{Mathematische Formulierung des Higgs-Mechanismus in der Zeit-Masse-Dualität}}
	\textit{(325.463 Bytes, 31.03.2025)}
	
	Dieses Dokument erklärt, wie der Higgs-Mechanismus (der Teilchen Masse gibt) in die neue Theorie passt.
	
	\section{Anwendungen und Erweiterungen}
	
	\subsection{\small\href{\repobase/pdf/Deutsch/Dynamische Masse von Photonen und ihre Implikationen für Nichtlokalität.pdf}{Dynamische Masse von Photonen und ihre Implikationen für Nichtlokalität}}
	\textit{(276.670 Bytes, 31.03.2025)}
	
	Es untersucht, ob Licht (Photonen) eine veränderliche Masse hat und wie das die Verbindung zwischen Teilchen erklärt.
	
	\subsection{\small\href{\repobase/pdf/Deutsch/Eine mathematische Analyse der Energiedynamik.pdf}{Eine mathematische Analyse der Energiedynamik}}
	\textit{(388.573 Bytes, 31.03.2025)}
	
	Dieses Dokument wendet die Theorie auf das Universum an und sieht dunkle Energie als etwas, das Energie verteilt, nicht als Ursache der Ausdehnung.
	
	\section{Kosmologische und Grenzgebiete}
	
	\subsection{\small\href{\repobase/pdf/Deutsch/Jenseits der Planck-Skala.pdf}{Jenseits der Planck-Skala}}
	\textit{(351.328 Bytes, 31.03.2025)}
	
	Es fragt, wie die Theorie die kleinsten (Planck-Skala) und größten Fragen der Physik – wie schwarze Löcher oder das frühe Universum – beantworten könnte.
	
	\section{Weitere relevante Dokumente}
	
	\subsection{\small\href{\repobase/pdf/Deutsch/Massenvariation in Galaxien.pdf}{Massenvariation in Galaxien}}
	\textit{(362.547 Bytes, 31.03.2025)}
	
	Es zeigt, wie sich veränderliche Masse auf Galaxien auswirkt und die Bewegung von Sternen erklärt, ohne dunkle Materie zu brauchen.
	
	\subsection{\small\href{\repobase/pdf/Deutsch/Vereinheitlichung des T0-Modells Grundlagen - Dunkle Energie und Galaxiendynamik.pdf}{Vereinheitlichung des T0-Modells: Grundlagen - Dunkle Energie und Galaxiendynamik}}
	\textit{(362.682 Bytes, 31.03.2025)}
	
	Eine umfassende Arbeit, die die Theorie auf Kosmologie und Galaxien anwendet.
	
	\subsection{\small\href{\repobase/pdf/Deutsch/Natürliche Einheiten mit Feinstrukturkonstante alpha = 1.pdf}{Natürliche Einheiten mit Feinstrukturkonstante alpha = 1}}
	\textit{(336.496 Bytes, 31.03.2025)}
	
	Es schlägt ein einfaches System vor, in dem eine wichtige Zahl (Feinstrukturkonstante) auf 1 gesetzt wird, um die Physik zu vereinfachen.
	
	\section{Englische Versionen}
	
	Zusätzlich gibt es englische Fassungen:
	\begin{itemize}
		\item \small\href{\repobase/pdf/English/Die Notwendigkeit einer Erweiterung der Standard-Quantenmechanik und Quantenfeldtheorie_en.pdf}{Die Notwendigkeit einer Erweiterung der Standard-Quantenmechanik und Quantenfeldtheorie} (257.169 Bytes)
		\item \small\href{\repobase/pdf/English/Dynamische Masse von Photonen und ihre Implikationen für Nichtlokalität_en.pdf}{Dynamische Masse von Photonen und ihre Implikationen für Nichtlokalität} (265.909 Bytes)
		\item \small\href{\repobase/pdf/English/Eine mathematische Analyse der Energiedynamik_en.pdf}{Eine mathematische Analyse der Energiedynamik} (377.701 Bytes)
		\item \small\href{\repobase/pdf/English/Eine neue Perspektive auf Zeit und Raum Johann Paschers revolutionäre Ideen_en.pdf}{Eine neue Perspektive auf Zeit und Raum} (235.024 Bytes)
		\item \small\href{\repobase/pdf/English/Feldtheorie und Quantenkorrelationen_en.pdf}{Feldtheorie und Quantenkorrelationen} (348.297 Bytes)
		\item \small\href{\repobase/pdf/English/Jenseits der Planck-Skala_en.pdf}{Jenseits der Planck-Skala} (347.870 Bytes)
		\item \small\href{\repobase/pdf/English/Kurzgefasst - Komplementärer Dualismus in der Physik - Von Welle-Teilchen zum Zeit-Masse-Konzept_en.pdf}{Kurzgefasst - Komplementärer Dualismus in der Physik} (145.857 Bytes)
		\item \small\href{\repobase/pdf/English/Massenvariation in Galaxien_en.pdf}{Massenvariation in Galaxien} (347.376 Bytes)
		\item \small\href{\repobase/pdf/English/Mathematische Formulierung des Higgs-Mechanismus in der Zeit-Masse-Dualität_en.pdf}{Mathematische Formulierung des Higgs-Mechanismus} (316.917 Bytes)
		\item \small\href{\repobase/pdf/English/Mathematische Formulierungen der Zeit-Masse-Dualitätstheorie mit Lagrange_en.pdf}{Mathematische Formulierungen der Zeit-Masse-Dualitätstheorie mit Lagrange} (544.118 Bytes)
	\end{itemize}
	
	\section{Zusammenfassung und Ausblick}
	
	Diese Arbeiten bilden ein Programm, das die Physik neu denkt. Die Zeit-Masse-Dualität nutzt einfache Regeln, um große Fragen wie Nichtlokalität oder dunkle Energie zu lösen. Zukünftige Schritte könnten Tests der Theorie, genauere Modelle und Simulationen umfassen, um eine einheitliche Physik zu schaffen.
	
\end{document}