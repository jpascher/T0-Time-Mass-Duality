\documentclass[a4paper,12pt]{article}
\usepackage[utf8]{inputenc}
\usepackage[T1]{fontenc}
\usepackage[ngerman]{babel}
\usepackage{lmodern}
\usepackage{csquotes}
\usepackage{amsmath}
\usepackage{amsfonts}
\usepackage{amssymb}
\usepackage{physics}
\usepackage{geometry}
\usepackage{tocloft}
\usepackage{xcolor}
\usepackage{graphicx,tikz,pgfplots}
\pgfplotsset{compat=1.18}
\usepackage{booktabs}
\usepackage{array}
\usepackage{tabularx}
\usepackage{braket}
\usepackage{siunitx}
\usepackage{amsthm}
\usepackage[colorlinks=true, linkcolor=blue, citecolor=blue, urlcolor=blue]{hyperref}
\usepackage{cleveref}
\usepackage{fancyhdr}

\geometry{a4paper, margin=2cm}

\hypersetup{
	pdftitle={Feldtheorie und Quantenkorrelationen: Eine neue Perspektive auf Instantaneität},
	pdfauthor={Johann Pascher},
	pdfcreator={LaTeX}
}

\renewcommand{\cftsecfont}{\color{blue}}
\renewcommand{\cftsubsecfont}{\color{blue}}
\renewcommand{\cftsecpagefont}{\color{blue}}
\renewcommand{\cftsubsecpagefont}{\color{blue}}
\setlength{\cftsecindent}{1cm}
\setlength{\cftsubsecindent}{2cm}

\newcommand{\DhiggsT}{\partial_\mu \Phi + \Phi \partial_\mu \Tfield}
\newcommand{\Tfield}{T(x)}

\newtheorem{theorem}{Satz}[section]
\newtheorem{proposition}[theorem]{Proposition}

\title{Feldtheorie und Quantenkorrelationen: \\Eine neue Perspektive auf Instantaneität}
\author{Johann Pascher}
\date{28. März 2025}

\begin{document}
	
	\maketitle
	
	\begin{abstract}
		Diese Arbeit entwickelt eine neue Perspektive auf das Phänomen der Quantenkorrelationen und deren scheinbare Instantaneität. Durch Einführung eines fundamentalen Feldansatzes im T0-Modell wird gezeigt, wie die nichtlokalen Eigenschaften der Quantenmechanik als natürliche Folge einer zugrundeliegenden Feldstruktur verstanden werden können. Besondere Aufmerksamkeit wird der Rolle des Quantenhintergrunds und der Interpretation moderner Bell-Experimente gewidmet. Diese Perspektive ergänzt die Zeit-Masse-Dualitätstheorie und bietet einen konsistenten Rahmen zur Erklärung quantenmechanischer Phänomene innerhalb eines umfassenden Feldkonzepts.
	\end{abstract}
	
	\tableofcontents
	\newpage
	
	\section{Einführung}
	Die moderne Quantenphysik steht vor einer grundlegenden Herausforderung: Die scheinbare Instantaneität von Quantenkorrelationen scheint unseren klassischen Vorstellungen von Lokalität und Kausalität zu widersprechen. Seit den bahnbrechenden Bell-Experimenten, insbesondere den lückenlosen Tests seit 2015 \cite{Hensen2015}, wissen wir mit Sicherheit, dass die Quantenwelt nichtlokale Eigenschaften aufweist. Doch die Frage nach der \textit{Natur} dieser Nichtlokalität und ihrer Vereinbarkeit mit der Relativitätstheorie bleibt offen.
	
	\subsection{Ein neuer Ansatz}
	Diese Arbeit entwickelt eine alternative Perspektive auf das Problem der Quantenkorrelationen, indem sie einen fundamentalen Feldansatz im T0-Modell vorschlägt. Statt separater Quantenfelder wird ein einheitliches fundamentales Feld postuliert, in dem Teilchen als Feldknoten und Quantenkorrelationen als Feldeigenschaften erscheinen \cite{Wilczek2008}. Dieser Blickwinkel ermöglicht es, die scheinbare „spukhafte Fernwirkung“ als natürliche Folge der Feldstruktur zu verstehen.
	
	\subsection{Theoretische Grundlagen}
	\begin{theorem}[Feldkonzept]
		Der vorgeschlagene Ansatz basiert auf drei Kernkonzepten:
		\begin{itemize}
			\item Das Vakuum als aktiver Quantenhintergrund mit definierten Eigenschaften (\(\varepsilon_0\), \(\mu_0\)).
			\item Teilchen als stabile Knoten oder Anregungsmuster im fundamentalen Feld.
			\item Quantenkorrelationen als inhärente Eigenschaften der Feldkohärenz.
		\end{itemize}
	\end{theorem}
	Diese Konzepte sind direkt mit der Zeit-Masse-Dualitätstheorie verknüpft \cite{Pascher2024}, wobei das intrinsische Zeitfeld \( \Tfield = \frac{\hbar}{\max(m c^2, \omega)} \) als fundamentale Größe betrachtet wird. Die Eigenschaften des Quantenhintergrunds bestimmen sowohl die Zeitenentwicklung des Systems als auch die Struktur der als Materie wahrgenommenen Feldknoten.
	
	\subsection{Experimentelle Beweise}
	Die Theorie wird durch moderne Experimente gestützt, insbesondere:
	\begin{itemize}
		\item Die Wiener Experimente von 2015, die alle klassischen Lücken schlossen \cite{Giustina2015}.
		\item Der „Big Bell Test“ von 2018 mit seiner einzigartigen Methodik \cite{BigBellTest2018}.
		\item Verschiedene Analogien zu klassischen Feldphänomenen.
	\end{itemize}
	
	\subsection{Mathematischer Rahmen}
	Die fundamentale Feldgleichung kann geschrieben werden als:
	\begin{equation}
		\Box \Psi + V(\Psi) = 0
	\end{equation}
	wobei \( \Box = \frac{\partial^2}{\partial t^2} - c^2 \nabla^2 \) der d’Alembert-Operator ist und \( V(\Psi) \) ein Potentialterm, der die Stabilität der Feldknoten sicherstellt. In Verbindung mit der Zeit-Masse-Dualität wird diese Gleichung reformuliert:
	\begin{equation}
		i\hbar \Tfield \frac{\partial}{\partial t} \Psi + i\hbar \Psi \frac{\partial \Tfield}{\partial t} = \hat{H} \Psi
	\end{equation}
	
	\section{Das Vakuum als Quantenhintergrund}
	Das Vakuum ist nicht bloß „nichts“, sondern ein aktiver Quantenhintergrund mit definierten physikalischen Eigenschaften \cite{Milonni1994}.
	
	\subsection{Fundamentale Vakuumkonstanten}
	Die elektrische Feldkonstante (\(\varepsilon_0\)) und die magnetische Feldkonstante (\(\mu_0\)) charakterisieren die fundamentalen Eigenschaften des Vakuums als Quantenhintergrund. Sie bestimmen Wechselwirkungen im elektromagnetischen Feld und sind direkt mit der Lichtgeschwindigkeit verknüpft:
	\begin{equation}
		c = \frac{1}{\sqrt{\varepsilon_0 \mu_0}}
	\end{equation}
	Diese Konstanten sind nicht nur mathematische Größen, sondern Ausdruck der physikalischen Struktur des Quantenhintergrunds \cite{Aitchison2004}. Im T0-Modell beeinflussen sie direkt das intrinsische Zeitfeld \( \Tfield \) und damit die Energieskala der Feldknoten \cite{Pascher2024}.
	
	\subsection{Das Vakuum als Feldträger}
	Der Quantenhintergrund dient als Trägermedium für das elektromagnetische Feld und alle anderen fundamentalen Felder. Diese Perspektive ermöglicht:
	\begin{itemize}
		\item Die Erklärung der Wellenausbreitung im „leeren“ Raum.
		\item Das Verständnis nichtlokaler Korrelationen als inhärente Feldeigenschaften.
		\item Die Überwindung der Grenzen klassischer Teilchenkonzepte.
	\end{itemize}
	Die Homogenität des Vakuums und seine Eigenschaften (\(\varepsilon_0\), \(\mu_0\)) sind entscheidend für die Konstanz der Lichtgeschwindigkeit und damit die Gültigkeit der speziellen Relativitätstheorie \cite{Weinberg1995}.
	
	\section{Quantenkorrelationen im Feldmodell}
	\subsection{Polarisation und Verschränkung}
	Die Polarisation eines Photons kann als Superposition horizontaler (H) und vertikaler (V) Polarisation beschrieben werden \cite{Fox2006}:
	\begin{equation}
		|\psi\rangle = \alpha |H\rangle + \beta e^{i\phi} |V\rangle
	\end{equation}
	Für verschränkte Photonenpaare entsteht ein gemeinsamer Zustand wie:
	\begin{equation}
		|\psi\rangle = \frac{1}{\sqrt{2}} (|H\rangle_A |H\rangle_B + |V\rangle_A |V\rangle_B)
	\end{equation}
	Im Feldmodell werden diese Zustände nicht als isolierte Teilcheneigenschaften betrachtet, sondern als kohärente Feldmuster, die sich über den Raum erstrecken \cite{Zeilinger2010}. Die Korrelation zwischen Messungen an Teilchen A und B ist eine inhärente Eigenschaft dieses Feldmusters, nicht das Ergebnis einer instantanen „Kommunikation“ zwischen den Teilchen.
	
	\subsection{Bell-Ungleichungen und lokaler Realismus}
	Bell-Ungleichungen zeigen die Grenzen lokaler realistischer Theorien \cite{Bell1964}:
	\begin{equation}
		|E(a,b) - E(a,c)| \leq 1 + E(b,c)
	\end{equation}
	Die experimentell beobachtete Verletzung dieser Ungleichung (für bestimmte Winkel \( a \), \( b \), \( c \)) zeigt, dass die Quantenwelt nicht durch lokale verborgene Variablen beschrieben werden kann \cite{Aspect1982}. Das Feldmodell bietet eine natürliche Erklärung: Das fundamentale Feld ist inhärent nichtlokal, da es den gesamten Raum umspannt.
	
	\subsection{Das Wiener Experiment von 2015}
	Das Experiment der Gruppe von Anton Zeilinger in Wien im Jahr 2015 war einer der ersten wirklich lückenlosen Tests des Bell-Theorems \cite{Giustina2015}. Es kombinierte:
	\begin{itemize}
		\item Sehr hohe Detektionseffizienz (>97\% mit SNSPDs).
		\item Ausreichende räumliche Trennung der Messungen.
		\item Schnelle, unabhängige Quantenzufallszahlengeneratoren.
	\end{itemize}
	Die beobachtete Verletzung der Bell-Ungleichung mit einer statistischen Signifikanz von 11,5 Standardabweichungen bestätigt die Nichtlokalität der Quantenwelt. Im Feldmodell ist diese Nichtlokalität eine natürliche Eigenschaft des fundamentalen Feldes und seiner kohärenten Struktur \cite{Zeilinger2010}.
	
	\subsection{Der „Big Bell Test“ von 2018}
	Der „Big Bell Test“ nutzte Entscheidungen von über 100.000 Menschen weltweit, um Messungseinstellungen in 13 verschiedenen Laboren zu steuern \cite{BigBellTest2018}. Diese menschliche Komponente adressierte die Freiheitswahl-Lücke auf neuartige Weise. Die Ergebnisse zeigten eine Verletzung der Bell-Ungleichungen mit statistischen Signifikanzen bis zu 70 Standardabweichungen.
	
	\section{Feldtheorie und Instantaneität}
	\subsection{Schallwellen als Analogie}
	Schallwellen bieten eine nützliche Analogie zum Verständnis des Feldkonzepts \cite{Bohm1980}:
	\begin{itemize}
		\item Schall existiert als Druckwelle, die den gesamten Raum durchdringt.
		\item Ein Mikrofon misst die Schwingung lokal, doch die Welle selbst ist global präsent.
		\item Die Gleichzeitigkeit von Messungen an verschiedenen Mikrofonen ergibt sich aus der kohärenten Struktur der Schallwelle.
	\end{itemize}
	Im Feldmodell sind verschränkte Teilchen wie Knoten in einem globalen Quantenfeld. Die Korrelationen zwischen ihnen sind keine „Fernwirkung“, sondern inhärente Eigenschaften des Feldes, die lokal bei der Messung abgetastet werden. Das Higgs-Feld spielt eine besondere Rolle als universelles Medium, das nicht nur Masse verleiht, sondern auch die intrinsische Zeitskala aller Teilchen bestimmt, wie in der Zeit-Masse-Dualitätstheorie beschrieben \cite{Pascher2024}.
	
	\subsection{Warum diese Analogie wichtig ist}
	\subsubsection{Auflösung des Paradoxons}
	Nichtlokalität erscheint nur paradox, wenn Teilchen als separate Objekte betrachtet werden. Im Feldmodell sind sie Teile eines Ganzen – wie Schallwellenpunkte in einem Raum \cite{Bohm1980}.
	\subsubsection{Realität des Feldes}
	Das Quantenfeld ist keine Abstraktion, sondern die fundamentale Entität \cite{Weinberg1995}. Seine Eigenschaften (Kohärenz, Nichtlokalität) sind ebenso real wie die eines Schallwellenfeldes.
	\subsubsection{Experimentelle Konsequenz}
	Wenn Alice und Bob verschränkte Photonen messen, „lauschen“ sie im Wesentlichen zwei Mikrofonen, die dieselbe Schallwelle abtasten. Die Korrelationen sind bereits im Feld enthalten, nicht erst durch die Messung erzeugt \cite{Zeilinger2010}.
	
	Im Kontext der Zeit-Masse-Dualität erhält diese Korrelation eine zusätzliche zeitliche Dimension: Das intrinsische Zeitfeld \( \Tfield = \frac{\hbar}{\max(m c^2, \omega)} \) bestimmt die Zeitskala der Feldkorrelationen und bietet eine natürliche Erklärung für die beobachteten Kohärenzzeiten und deren Massenabhängigkeit \cite{Pascher2024}.
	
	\section{Feldgleichungen in dualer Formulierung}
	\subsection{Modifizierte Quantenmechanik mit variabler Masse}
	Im Gegensatz zur herkömmlichen Schrödinger-Gleichung:
	\begin{equation}
		i\hbar \frac{\partial}{\partial t}\Psi(x,t) = \hat{H}\Psi(x,t)
	\end{equation}
	führt die Zeit-Masse-Dualität eine fundamentale Modifikation ein:
	\begin{equation}
		i\hbar \Tfield \frac{\partial}{\partial t} \Psi + i\hbar \Psi \frac{\partial \Tfield}{\partial t} = \hat{H} \Psi
	\end{equation}
	
	\subsection{Lagrange-Formulierung}
	Die Gesamt-Lagrangedichte des T0-Modells lautet:
	\begin{equation}
		\mathcal{L}_{\text{Total}} = \mathcal{L}_{\text{Boson}} + \mathcal{L}_{\text{Fermion}} + \mathcal{L}_{\text{Higgs-T}} + \mathcal{L}_{\text{intrinsic}}, \quad \mathcal{L}_{\text{intrinsic}} = \frac{1}{2} \partial_\mu \Tfield \partial^\mu \Tfield - V(\Tfield)
	\end{equation}
	
	\section{Kosmologische Implikationen}
	Das T0-Modell hat folgende Implikationen:
	\begin{itemize}
		\item Modifiziertes Gravitationspotential: \( \Phi(r) = -\frac{GM}{r} + \kappa r \), \( \kappa \approx 4.8 \times 10^{-11} \, \text{m/s}^2 \)
		\item Kosmische Rotverschiebung: \( 1 + z = e^{\alpha d} \), \( \alpha \approx 2.3 \times 10^{-28} \, \text{m}^{-1} \)
		\item Wellenlängenabhängigkeit: \( z(\lambda) = z_0 (1 + \beta \ln(\lambda/\lambda_0)) \), \( \beta \approx 0.008 \)
	\end{itemize}
	Gravitation entsteht aus \( \nabla \Tfield \).
	
	\section{Unsicherheit bei \(\beta\)}
	Der Parameter \( \beta \approx 0.008 \) ist unsicher; weitere Tests sind erforderlich.
	
	\section{Schlussfolgerung}
	Das T0-Modell bietet eine neue Perspektive auf Quantenkorrelationen als Feldeigenschaften, integriert mit der Zeit-Masse-Dualität, und stellt einen konsistenten Rahmen bereit, der durch moderne Bell-Experimente gestützt wird.
	
	\begin{thebibliography}{9}
		\bibitem{wesentlicheFormalismen} Pascher, J. (2025). \textit{Wesentliche mathematische Formalismen der Zeit-Masse-Dualitätstheorie mit Lagrange-Dichten}. 29. März 2025.
	\end{thebibliography}
	
\end{document}