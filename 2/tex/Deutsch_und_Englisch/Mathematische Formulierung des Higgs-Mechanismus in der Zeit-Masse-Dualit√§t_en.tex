\documentclass[a4paper,12pt]{article}
\usepackage[utf8]{inputenc}
\usepackage[T1]{fontenc}
\usepackage[english]{babel}
\usepackage{lmodern}
\usepackage{csquotes}
\usepackage{tocloft} % For table of contents customization
\usepackage{xcolor}
\usepackage{hyperref}
\usepackage{geometry}
\usepackage{booktabs}
\usepackage{array}
\usepackage{tabularx}
\usepackage{fancyhdr}
\usepackage{amsmath}
\usepackage{amssymb}
\usepackage{physics}
\usepackage{bm}

\geometry{a4paper, margin=2.5cm}
\hypersetup{colorlinks=true, linkcolor=blue, citecolor=blue, urlcolor=blue}

% Table of contents customization
\renewcommand{\cftsecfont}{\color{blue}}     % Sections in blue
\renewcommand{\cftsubsecfont}{\color{blue}}  % Subsections in blue
\renewcommand{\cftsecpagefont}{\color{blue}} % Page numbers in blue
\renewcommand{\cftsubsecpagefont}{\color{blue}} % Subsection page numbers in blue

% Optional: Indentation of table of contents on the left side
\setlength{\cftsecindent}{1cm}
\setlength{\cftsubsecindent}{2cm}

% Definition for the intrinsic time field
\newcommand{\Tfield}{T(x)} % Intrinsic time as a field
\newcommand{\DhiggsT}{\Tfield (\partial_\mu + igA_\mu)\Phi + \Phi \partial_\mu \Tfield}

\pagestyle{fancy}
\fancyhf{}
\fancyhead[L]{Johann Pascher}
\fancyhead[R]{Time-Mass Duality}
\fancyfoot[C]{\thepage}
\renewcommand{\headrulewidth}{0.4pt}
\renewcommand{\footrulewidth}{0.4pt}

\title{Mathematical Formulation of the Higgs Mechanism in Time-Mass Duality}
\author{Johann Pascher}
\date{March 28, 2025}

\begin{document}
	
	\maketitle
	
	\tableofcontents
	\clearpage
	
	\begin{abstract}
		This work develops a precise mathematical formulation of the Higgs mechanism within the framework of a novel time-mass duality theory. Assuming that time and mass are complementary aspects of the same fundamental reality, we show how the Higgs mechanism serves as a mediator between two equivalent descriptions: the conventional picture with time dilation and constant rest mass on one hand, and an alternative picture with absolute time and variable mass on the other. The formulation not only leads to an elegant mathematical structure but also yields concrete, experimentally verifiable predictions that deviate from the Standard Model of particle physics.
	\end{abstract}
	
	\section*{Introduction}
	
	Modern theoretical physics is based on two fundamental yet not fully reconciled theories: relativity and quantum mechanics. While relativity describes time and space as dynamic, observer-dependent quantities, quantum mechanics treats time as an external parameter. This conceptual tension may point to a deeper structure that could unify both perspectives.
	
	In this work, we investigate an alternative theoretical foundation based on the idea of a fundamental duality between time and mass. Similar to wave-particle duality in quantum mechanics, we postulate that time and mass represent two complementary descriptions of the same physical reality. While conventional relativity treats time as relative (time dilation) and rest mass as constant, we propose an alternative, mathematically equivalent picture where time is absolute and mass varies instead.
	
	The Higgs mechanism plays a special role in this context, as it is responsible for generating particle masses in the Standard Model. In our dual formulation, the Higgs field becomes the central mediator between both perspectives, defining both the rest mass and the intrinsic time scale of all particles. Particularly noteworthy is that the unique position of the Higgs boson in the particle zoo—as the only particle without a clear 'mirror image'—finds a natural explanation in this framework.
	
	In the following, we develop a mathematically precise formalism for this time-mass duality, reformulate the fundamental field equations, and derive concrete experimental consequences. This theory does not represent a break with established physics but rather expands its interpretative framework, potentially revealing deeper connections between seemingly independent phenomena such as quantum coherence, Higgs interactions, and cosmological observations.
	
	\section{Starting Point: Higgs Mechanism in the Standard Model}
	
	In the Standard Model, the Higgs field is introduced as a complex scalar doublet:
	\begin{equation}
		\Phi = \begin{pmatrix} \phi^+ \\ \phi^0 \end{pmatrix}
	\end{equation}
	
	The Lagrangian density for the Higgs field is:
	\begin{equation}
		\mathcal{L}_{\text{Higgs}} = (D_\mu \Phi)^\dagger (D^\mu \Phi) - V(\Phi)
	\end{equation}
	
	with the Higgs potential:
	\begin{equation}
		V(\Phi) = -\mu^2 \Phi^\dagger \Phi + \lambda (\Phi^\dagger \Phi)^2
	\end{equation}
	
	The Yukawa coupling, which describes the coupling of the Higgs field to fermions:
	\begin{equation}
		\mathcal{L}_{\text{Yukawa}} = -y_f \bar{\psi}_L \Phi \psi_R + \text{h.c.}
	\end{equation}
	
	After spontaneous symmetry breaking, the Higgs field acquires a vacuum expectation value (VEV):
	\begin{equation}
		\langle \Phi \rangle = \frac{1}{\sqrt{2}} \begin{pmatrix} 0 \\ v \end{pmatrix}
	\end{equation}
	
	The fermion masses then emerge as:
	\begin{equation}
		m_f = \frac{y_f v}{\sqrt{2}}
	\end{equation}
	
	\section{Reformulation in the Time-Mass Duality Framework}
	
	\subsection{Time Dilation Picture (Standard Relativity)}
	
	In this picture, the rest mass of particles is constant, while time is relative (time dilation). The mass-energy relation is:
	\begin{equation}
		E = \gamma m_0 c^2
	\end{equation}
	
	where $\gamma = \frac{1}{\sqrt{1-v^2/c^2}}$ is the Lorentz factor.
	
	Time dilation is described by:
	\begin{equation}
		t' = \gamma t
	\end{equation}
	
	The Yukawa coupling in this picture directly leads to a constant rest mass:
	\begin{equation}
		m_0 = \frac{y_f v}{\sqrt{2}}
	\end{equation}
	
	\subsection{Mass Variation Picture (Time-Mass Duality)}
	
	In this alternative picture, time $T_0$ is absolute (constant), while mass is variable. The intrinsic time field is defined as:
	\begin{equation}
		\Tfield = \frac{\hbar}{m c^2}
	\end{equation}
	
	The transformation relation to the standard picture is:
	\begin{equation}
		m = \gamma m_0
	\end{equation}
	
	and
	\begin{equation}
		\Tfield = \frac{T_0}{\gamma}
	\end{equation}
	
	where $T_0 = \frac{\hbar}{m_0 c^2}$ is the intrinsic time in the rest state.
	
	\section{The Higgs Field as Mediator of Time-Mass Duality}
	
	\subsection{Modified Higgs Lagrangian Density}
	
	In the time-mass duality framework, we modify the Higgs Lagrangian density:
	\begin{equation}
		\mathcal{L}_{\text{Higgs-T}} = (\DhiggsT)^\dagger (\DhiggsT) - \lambda(|\Phi|^2 - v^2)^2
	\end{equation}
	
	where the index $T$ denotes the dependence on the intrinsic time field. The modified covariant derivative is defined as:
	\begin{equation}
		\DhiggsT = \Tfield (\partial_\mu + igA_\mu)\Phi + \Phi \partial_\mu \Tfield
	\end{equation}
	
	This means that the time derivative is taken with respect to the intrinsic time $\Tfield$:
	\begin{equation}
		\partial_{t/T} = \frac{\partial}{\partial(t/\Tfield)} = \Tfield\frac{\partial}{\partial t}
	\end{equation}
	
	\subsection{Modified Yukawa Coupling}
	
	The Yukawa coupling in the mass variation picture is reinterpreted:
	\begin{equation}
		\mathcal{L}_{\text{Yukawa-T}} = -y_f \bar{\psi}_L \Phi \psi_R \cdot \gamma + \text{h.c.}
	\end{equation}
	
	This modified Yukawa coupling leads to a velocity-dependent mass:
	\begin{equation}
		m(v) = \gamma \cdot \frac{y_f v}{\sqrt{2}} = \gamma m_0
	\end{equation}
	
	while the intrinsic time field scales accordingly:
	\begin{equation}
		\Tfield(v) = \frac{\hbar}{m(v)c^2} = \frac{\hbar}{\gamma m_0 c^2} = \frac{T_0}{\gamma}
	\end{equation}
	
	\subsection{Higgs Field as Connection Between the Pictures}
	
	In the new framework, the Higgs field plays a dual role:
	\begin{enumerate}
		\item It generates the rest mass $m_0$ through its VEV in the standard picture
		\item It defines the intrinsic time scale $T_0 = \frac{\hbar}{m_0 c^2}$ in the duality picture
	\end{enumerate}
	
	The fundamental connection is expressed by:
	\begin{equation}
		T_0 \cdot m_0 c^2 = \hbar
	\end{equation}
	
	This relationship is preserved in both pictures, since:
	\begin{equation}
		\Tfield \cdot m c^2 = \frac{T_0}{\gamma} \cdot \gamma m_0 c^2 = T_0 \cdot m_0 c^2 = \hbar
	\end{equation}
	\section{Field Equations in Dual Formulation}
	
	\subsection{Klein-Gordon Equation}
	
	The standard Klein-Gordon equation for the Higgs boson is:
	\begin{equation}
		(\Box + m_H^2) h(x) = 0
	\end{equation}
	
	In the time-mass duality picture, it becomes:
	\begin{equation}
		\left(\frac{\partial^2}{\partial(t/\Tfield)^2} - \nabla^2 + m_H^2\right) h_T(x) = 0
	\end{equation}
	
	This leads to a modified dispersion relation:
	\begin{equation}
		\omega_T^2 = \mathbf{k}^2 + \frac{m_H^2 c^4}{\hbar^2} \cdot \Tfield^2
	\end{equation}
	
	\subsection{Dirac Equation}
	
	The Dirac equation for fermions in the Standard Model:
	\begin{equation}
		(i\gamma^\mu\partial_\mu - m_f) \psi(x) = 0
	\end{equation}
	
	becomes in the time-mass duality picture:
	\begin{equation}
		\left(i\gamma^0\frac{\partial}{\partial(t/\Tfield)} + i\gamma^i\partial_i - m_f\right) \psi_T(x) = 0
	\end{equation}
	
	\subsection{Field Equations for Gauge Bosons}
	
	The Yang-Mills equations for gauge bosons are similarly modified, with the time derivative replaced by $\partial_{t/T}$.
	
	\section{Higgs as Universal Medium}
	
	The Higgs field can be viewed as a universal medium that not only mediates mass but also determines the intrinsic time scale of all particles. Since the Higgs field is present everywhere in space, it in a sense defines a preferred reference frame—not for spacetime coordinates but for time-mass duality.
	
	The intrinsic time field of a particle is determined by its coupling to the Higgs field:
	\begin{equation}
		T_0 = \frac{\hbar}{m_0 c^2} = \frac{\hbar \sqrt{2}}{y_f v c^2}
	\end{equation}
	
	This shows that the intrinsic time scale is inversely proportional to the Yukawa coupling constant.
	
	\section{Symmetry Considerations}
	
	\subsection{Conserved Quantities}
	
	In the standard picture, energy $E = mc^2$ is a conserved quantity. In the time-mass duality picture, the product $\Tfield \cdot m c^2 = \hbar$ is constant, corresponding to a new conserved quantity.
	
	\subsection{Symmetry Transformations}
	
	The Lorentz transformation is reinterpreted:
	\begin{align}
		t &\to t' = \gamma t & &\text{(Standard picture)} \\
		m &\to m' = \gamma m_0 & &\text{(Duality picture)}
	\end{align}
	
	The global phase in quantum mechanics $\psi \to e^{i\theta}\psi$ could acquire a deeper meaning in this context, possibly as a rotation in 'time-mass space'.
	
	\section{Philosophical and Epistemological Implications}
	
	The time-mass duality theory raises profound philosophical questions alongside its physical consequences:
	
	\subsection{Decidability Between Mathematically Equivalent Theories}
	
	Particularly fascinating is that our theory raises a central philosophical question: To what extent can we decide between mathematically equivalent but conceptually different theories? In the standard picture with time dilation and the alternative picture with mass variation, the same results initially emerge for known phenomena such as GPS corrections or the extended lifetime of moving muons.
	
	The answer may lie in subtle experimental effects that appear 'natural' in only one of the two pictures. The predicted nonlinearities in Higgs couplings or mass-dependent coherence times could serve as such distinguishing criteria. This recalls the debate between geocentric and heliocentric worldviews, where both models could mathematically describe planetary motion, but the heliocentric model led to a simpler and more elegant explanation.
	
	\subsection{Emergent Properties of Fundamental Quantities}
	
	The interpretation of time as an emergent property derived from mass and fundamental constants (such as $\hbar$ and $c$) fundamentally questions our understanding of basic quantities. As extensively discussed in earlier works \cite{pascher_zeit_2025, pascher_natur_2025}, the relation $\Tfield = \frac{\hbar}{mc^2}$ suggests that time might not be fundamental but rather an emergent property.
	
	This conception could have far-reaching implications, as it suggests that other seemingly fundamental parameters of physics might likewise be emergent properties of deeper structures. The mass-dependent time scale could indicate that we need to reconsider the hierarchy of physical fundamental quantities, as further elaborated in \cite{pascher_kompl_2025}.
	
	\subsection{The Vacuum Energy Paradox in a New Light}
	
	The so-called 'cosmological constant problem'—the enormous discrepancy between theoretically calculated and observed vacuum energy—might find a fundamentally new interpretation in our framework. The formulation of vacuum energy as 
	\begin{equation}
		E_{\text{Vacuum}} = \sum_i \frac{\hbar}{2T_i} = \sum_i \frac{m_i c^2}{2}
	\end{equation}
	links vacuum energy directly to the intrinsic time of quantum fluctuations. The apparent discrepancy could result from summing the 'wrong' degrees of freedom in the Standard Model.
	
	This new interpretation might explain why the observed vacuum energy (dark energy) is about $10^{-120}$ times smaller than the naive calculation of zero-point energy for all quantum fields. In our model, the Higgs field would provide a natural upper limit for the summation, consistent with the observed cosmological constant.
	
	\subsubsection{Detailed Consideration of the Vacuum Energy Problem}
	
	In conventional quantum field theory, vacuum energy is calculated as the sum of zero-point energies of all field modes:
	\begin{equation}
		E_{\text{Vacuum, conv.}} = \sum_{\text{modes}} \frac{\hbar\omega_k}{2}
	\end{equation}
	
	This sum diverges unless an arbitrary cutoff at high energies is introduced. Even with a Planck-scale cutoff ($\Lambda_{\text{Planck}} \sim 10^{19}$ GeV), the resulting energy density is about $10^{120}$ times larger than the observed value of the cosmological constant—one of the greatest discrepancies between theory and observation in physics.
	
	In our time-mass duality model, a fundamentally different approach emerges. The zero-point energy of a quantum field is directly linked to the intrinsic time field $\Tfield_i$ of the involved particles:
	
	\begin{equation}
		E_i = \frac{\hbar}{2\Tfield_i} = \frac{m_i c^2}{2}
	\end{equation}
	
	This leads to several crucial consequences:
	
	\begin{itemize}
		\item \textbf{Natural Weighting:} The contribution of each quantum fluctuation to vacuum energy is proportional to the mass of the corresponding particle, not to the third power of its energy scale as in the Standard Model.
		
		\item \textbf{Intrinsic Regularization:} Massless particles with $\Tfield \to \infty$ contribute nothing, while very massive particles with small $\Tfield$ automatically provide a finite contribution—a natural regularization without arbitrary cutoffs.
		
		\item \textbf{Higgs Mechanism as Regulator:} Since masses are generated by the Higgs field, the sum of vacuum energy is directly regulated and stabilized by the Higgs field.
	\end{itemize}
	
	The total vacuum energy in our model results as:
	\begin{equation}
		E_{\text{Vacuum}} = \sum_i n_i \frac{m_i c^2}{2}
	\end{equation}
	where $n_i$ represents the number of effective degrees of freedom for each particle type.
	
	This formulation provides several natural mechanisms that could explain why the observed vacuum energy is much smaller than the naive expectation:
	
	\begin{itemize}
		\item \textbf{Partial Cancellation:} Contributions from fermions and bosons could partially cancel each other, where the cancellation does not require perfect supersymmetry but follows from the intrinsic time structure.
		
		\item \textbf{Dynamic Equilibrium:} The vacuum energy could be in a dynamic equilibrium with the Higgs field, leading to a natural 'self-adjustment'.
		
		\item \textbf{Alternative to Cosmic Expansion:} The intrinsic time field $\Tfield$ offers an alternative explanatory framework for phenomena usually attributed to cosmic expansion. In our model, the universe does not expand, but the observed effects result from the energy loss of photons over large distances according to $E(r) = E_0 e^{-\alpha r}$.
	\end{itemize}
	
	The resulting value of vacuum energy in this model would not be an arbitrarily fine-tuned quantity but the outcome of the fundamental structure of particle-time relations in the universe.
	
	\subsection{Realism versus Instrumentalism}
	
	Time-mass duality also sheds new light on the philosophical debate between realism and instrumentalism. Is one of the two descriptions (time dilation or mass variation) 'more real' than the other, or are they merely mathematically equivalent descriptions without ontological difference?
	
	Our approach suggests that both pictures could describe different aspects of the same underlying reality, similar to wave-particle duality in quantum mechanics. This would argue for a 'perspectivist realism', where the choice of picture depends on the context and specific phenomena being studied.
	
	\section{Experimental Signatures and New Predictions}
	
	The dual formulation of the Higgs mechanism leads to several experimentally verifiable predictions that deviate from the Standard Model. These are particularly important as they offer concrete possibilities to distinguish between time-mass duality theory and conventional interpretation:
	
	\subsection{Mass-Dependent Quantum Coherence}
	
	In time-mass duality theory, particles of different mass have different intrinsic time scales ($\Tfield = \frac{\hbar}{mc^2}$). This leads to the following testable predictions:
	
	\begin{itemize}
		\item \textbf{Coherence Time Ratio:} For quantum systems of different mass, the coherence times $\tau_1$ and $\tau_2$ of two otherwise identical quantum systems with masses $m_1$ and $m_2$ should follow the ratio:
		\begin{equation}
			\frac{\tau_1}{\tau_2} = \frac{m_2}{m_1}
		\end{equation}
		This could be tested in precision experiments with molecules of different isotopes or Bose-Einstein condensates of different atomic species.
		
		\item \textbf{Mass-Dependent Interference Patterns:} In double-slit experiments with particles of different mass (at the same velocity), subtle differences in interference patterns should appear beyond the de Broglie wavelength differences.
	\end{itemize}
	
	\subsection{Modified Higgs Couplings}
	
	Time-mass duality should cause deviations in Higgs couplings:
	
	\begin{itemize}
		\item \textbf{Nonlinearity in Mass Hierarchy:} The Standard Model predicts that Higgs couplings are strictly proportional to particle mass. In time-mass duality theory, this relationship could exhibit slight nonlinearities:
		\begin{equation}
			g_H \propto m \left(1 + \delta \cdot \ln\left(\frac{m}{m_0}\right)\right)
		\end{equation}
		where $\delta$ is a small correction and $m_0$ a reference mass.
		
		\item \textbf{Dynamic Higgs Couplings:} At very high energies, Higgs couplings could show slight deviations from relativistic predictions, detectable with precision measurements at the LHC or future accelerators.
	\end{itemize}
	
	\subsection{Entanglement Effects with Unequal Masses}
	
	Time-mass duality makes unique predictions for entangled quantum systems with different masses:
	
	\begin{itemize}
		\item \textbf{Mass-Dependent Entanglement Correlations:} In Bell tests with entangled particles of different mass, the measured correlations should show a subtle mass dependence.
		
		\item \textbf{Delayed Correlations:} The intrinsic time field $\Tfield = \frac{\hbar}{mc^2}$ could lead to measurable delays in quantum correlations, proportional to the mass ratio of the entangled particles.
	\end{itemize}
	
	\subsection{Modified Energy-Momentum Relation}
	
	Time-mass duality leads to a modified energy-momentum relation:
	\begin{equation}
		E^2 = (pc)^2 + (mc^2)^2 + \alpha_E\frac{\hbar c}{\Tfield}
	\end{equation}
	where $\alpha_E$ is a small dimensionless constant and $\Tfield$ the intrinsic time field of the particle. This effect would become visible in very precise measurements of the energy-momentum relation, especially for light particles with large intrinsic time scales.
	
	\subsection{Cosmological Tests}
	
	\begin{itemize}
		\item \textbf{Energy Transfer Coefficient:} The absorption coefficient $\alpha \approx 2.3 \times 10^{-18} \text{ m}^{-1}$ should be experimentally detectable in precise measurements of cosmic redshift and could provide an alternative explanation for the observed cosmic acceleration, as detailed in the theorem on cosmic redshift in \cite{pascher_wesentl_2025}.
		
		\item \textbf{Modified Gravitational Potential:} In galactic rotation curves, the parameter $\kappa \approx 4.8 \times 10^{-11} \text{ m/s}^2$ should be measurable and could explain the observed deviations without dark matter, according to the theorem on modified gravitational potential in \cite{pascher_wesentl_2025}:
		\begin{equation}
			\Phi(r) = -\frac{GM}{r} + \kappa r
		\end{equation}
	\end{itemize}
	
	\subsection{New Interpretation of Vacuum Energy}
	
	Time-mass duality leads to a new interpretation of vacuum energy:
	\begin{equation}
		E_{\text{Vacuum}} = \sum_i \frac{\hbar}{2\Tfield_i} = \sum_i \frac{m_i c^2}{2}
	\end{equation}
	This formulation links vacuum energy directly to the intrinsic time field of quantum fluctuations and could lead to measurable deviations in Casimir force or other vacuum effects.
	
	\subsection{Photon Energy Loss}
	
	According to the theory, photons should experience a slight energy loss according to $E(r) = E_0 e^{-\alpha r}$, where $\alpha \approx 2.3 \times 10^{-18} \text{ m}^{-1}$ is the absorption coefficient. This could alternatively explain cosmic redshift and be verified by precision spectroscopy of distant quasars.
	
	\subsection{Practical Experimental Feasibility}
	
	The most promising experiments to test these predictions would be:
	\begin{enumerate}
		\item High-precision atomic clock comparisons with different elements
		\item Quantum interference experiments with particles of different mass
		\item Precision measurements of Higgs couplings at the LHC or future accelerators
		\item Bell tests with entangled particles of different mass
		\item Detailed analyses of cosmic redshift over large distances
	\end{enumerate}
	
	\section{Conclusion}
	
	The time-mass duality theory offers a mathematically coherent reformulation of the Higgs mechanism that is not only conceptually elegant but also leads to concrete, verifiable predictions. The theory interprets the Higgs mechanism not only as a mass generator but also as a mediator between two complementary views of reality: the conventional picture with time dilation and constant rest mass on one hand, and an alternative picture with absolute time and variable mass on the other.
	
	The mathematical structure developed in this work, with the intrinsic time field $\Tfield = \frac{\hbar}{mc^2}$ as a central quantity, leads to an elegant connection between seemingly independent phenomena such as quantum coherence, Higgs couplings, and cosmological observations. The predicted deviations from the Standard Model, such as nonlinear Higgs couplings, mass-dependent coherence times, or a modified energy-momentum relation, are in principle experimentally verifiable and could serve to distinguish between time-mass duality and conventional interpretation.
	
	Particularly promising is the new perspective on the vacuum energy problem, one of the greatest unsolved mysteries of modern physics. Time-mass duality theory offers a natural mechanism that could explain why the observed vacuum energy is so much smaller than the naive expectation from quantum field theory.
	
	This work is not intended as a radical break with established physics but as an expansion of the interpretative framework that opens new insights and research perspectives. Time-mass duality could ultimately be an important step toward a deeper understanding of the fundamental structure of reality.
	
	\begin{thebibliography}{9}
		
		\bibitem{pascher_zeit_2025} Pascher, J. (2025). Time as an emergent property in quantum mechanics: A connection between relativity, fine-structure constant, and quantum dynamics.
		
		\bibitem{pascher_natur_2025} Pascher, J. (2025). Natural units with fine-structure constant alpha = 1.
		
		\bibitem{pascher_kompl_2025} Pascher, J. (2025). Complementary extensions of physics: Absolute time and intrinsic time.
		
		\bibitem{pascher_blick_2025} Pascher, J. (2025). Time and mass: A new look at old formulas—and liberation from traditional constraints.
		
		\bibitem{pascher_grund_2025} Pascher, J. (2025). Simplified description of the four fundamental forces with time-mass duality.
		
		\bibitem{pascher_wesentl_2025} Pascher, J. (2025). Essential mathematical formalisms of time-mass duality theory with Lagrangian densities. March 29, 2025.
		
	\end{thebibliography}
	
\end{document}