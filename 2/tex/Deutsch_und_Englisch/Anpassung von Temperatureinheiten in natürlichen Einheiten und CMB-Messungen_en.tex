\documentclass[12pt,a4paper]{article}
\usepackage[utf8]{inputenc}
\usepackage[T1]{fontenc}
\usepackage[english]{babel}
\usepackage{lmodern}
\usepackage{amsmath}
\usepackage{amssymb}
\usepackage{hyperref}
\hypersetup{
	colorlinks=true,
	linkcolor=blue,
	citecolor=blue,
	urlcolor=blue
}

\title{Adjustment of Temperature Units in Natural Units and CMB Measurements}
\author{Johann Pascher}
\date{April 2, 2025}

\begin{document}
	
	\maketitle
	
	\section*{Introduction}
	
	It is correct that all units could be chosen in natural units by setting fundamental constants such as $\hbar$, $c$, $k_B$, and $G$ to 1. While this approach is possible, it is not common in practice. This document explains how temperature measurement and blackbody radiation could be adjusted in such a system, particularly when additionally setting $\alpha_W = 1$. It also addresses how CMB temperature measurements are conducted today and whether they are indirectly influenced by constants or coupling factors from the standard model. Subsequently, the idea of adjusting the temperature unit with $\alpha_W = 1$ is discussed.
	
	\section{Adjustment of the Temperature Unit with $\alpha_W = 1$}
	
	The consistent application of the principle of maximum simplification in natural unit systems has profound implications for the interpretation and scaling of thermodynamic quantities. In particular, the relationship between temperature and energy must be reconsidered. The Planck radiation formula, which describes the spectral energy density of blackbody radiation:
	
	\begin{equation}
		u(\nu, T) = \frac{2\pi h \nu^3}{c^2} \cdot \frac{1}{e^{h \nu / k_B T} - 1}
	\end{equation}
	
	leads to Wien's displacement law, which relates the frequency of the radiation maximum to temperature:
	
	\begin{equation}
		\nu_{\text{max}} = \alpha \cdot \frac{k_B T}{h}
	\end{equation}
	
	where $\alpha \approx 2.82$ is a numerically determined constant. To avoid confusion with the fine-structure constant, we will denote this constant as $\alpha_W$ henceforth. If, in addition to $k_B = h = c = 1$, $\alpha_W = 1$ is also set, a direct proportionality between the frequency of the radiation maximum and temperature emerges:
	
	\begin{equation}
		\nu_{\text{max}} = T
	\end{equation}
	
	To make this relationship consistent, an adjustment of the temperature unit is required. Kelvin would be unsuitable as a base unit, as temperature would then be measured directly in energy units and scaled to match the frequency of the radiation maximum. This adjustment is analogous to the treatment of space and time in relativity, where with $c = 1$, both can be measured in length units. The choice of $\alpha_W = 1$ is thus a consistent extension of the principle of maximum simplification, but it necessitates a redefinition of the temperature unit.
	
	\section{Adjustment of the Temperature Unit with $\alpha_W = 1$}
	
	The \href{https://github.com/jpascher/T0-Time-Mass-Duality/tree/main/2/pdf/English/Natürliche Einheiten mit Feinstrukturkonstante alpha = 1_en.pdf}{document} suggests that in natural units with $k_B = h = c = 1$ and additionally $\alpha_W = 1$, the temperature directly corresponds to the frequency of the radiation maximum ($\nu_{\text{max}} = T$). Let us verify this:
	
	\subsection{Standard Formula}
	
	Wien's displacement law in SI units is:
	\[
	\nu_{\text{max}} = \alpha \cdot \frac{k_B T}{h}, \quad \alpha \approx 2.821439,
	\]
	where $\alpha$ is numerically determined from maximizing the Planck distribution ($3 (e^x - 1) = x e^x$).
	
	\subsection{Natural Units}
	
	With $k_B = 1$, $h = 2\pi$ (since $\hbar = 1$), $c = 1$:
	\[
	\nu_{\text{max}} = \alpha \cdot \frac{T}{2\pi},
	\]
	\[
	\nu_{\text{max}} = \frac{2.821439}{2\pi} T \approx 0.449 T.
	\]
	In natural units, $\alpha \approx 2.82$ remains because it is a mathematical constant independent of $h$, $c$, or $k_B$.
	
	\subsection{Setting $\alpha_W = 1$}
	
	If $\alpha_W = 1$ is set:
	\[
	\nu_{\text{max}} = \frac{T}{2\pi},
	\]
	or, if $\alpha_W$ is to be completely eliminated:
	\[
	\nu_{\text{max}} = T.
	\]
	This requires a redefinition of the temperature unit:
	\begin{itemize}
		\item In SI: $T$ in Kelvin, $\nu_{\text{max}}$ in Hz, and $k_B T / h$ scales the relationship.
		\item With $k_B = h = 1$, $\alpha_W = 1$: $T$ must be scaled so that $\nu_{\text{max}} = T$ holds, meaning $T$ becomes a frequency (or energy in natural units), not Kelvin.
	\end{itemize}
	
	\subsection{Implications}
	
	\begin{itemize}
		\item \textbf{New Unit:} $T$ would no longer be a temperature in the classical sense (Kelvin) but an energy/frequency (e.g., in GeV or Hz, since $c = 1$ is omitted). This is consistent with the \href{https://github.com/jpascher/T0-Time-Mass-Duality/tree/main/2/pdf/English/Eine neue Perspektive auf Zeit und Raum Johann Paschers revolutionäre Ideen_en.pdf}{analogy to relativity} ($c = 1$, space and time in length units).
		\item \textbf{CMB Temperature:} The measured $T = 2.725 \, \text{K}$ would need conversion. In natural units with $k_B = 1$:
		\[
		T = 2.725 \, \text{K} \cdot k_B = 2.725 \cdot 1.380649 \times 10^{-23} \, \text{J} \approx 3.762 \times 10^{-23} \, \text{J}.
		\]
		With $h = 2\pi \hbar = 6.62607015 \times 10^{-34} \, \text{J·s}$:
		\[
		\nu_{\text{max}} = \frac{k_B T}{h} \cdot 2.821439 \approx \frac{3.762 \times 10^{-23}}{6.62607015 \times 10^{-34}} \cdot 2.821439 \approx 1.6 \times 10^{11} \, \text{Hz}.
		\]
		With $\alpha_W = 1$:
		\[
		\nu_{\text{max}} = \frac{T}{2\pi} \approx 6 \times 10^{10} \, \text{Hz},
		\]
		and $T$ would need to be scaled to this frequency, requiring a new unit.
	\end{itemize}
	
	\subsection{Why Not Common?}
	
	\begin{itemize}
		\item \textbf{Observational Practice:} Cosmologists use Kelvin because it directly relates to measured temperatures (e.g., CMB, stellar surfaces). Natural units with $\alpha_W = 1$ would complicate communication with experimental data.
		\item \textbf{Mathematical Constant:} $\alpha \approx 2.82$ is not an arbitrary constant but a solution to the equation $3 (e^x - 1) = x e^x$, independent of units. Setting it to 1 is a simplification that distorts physical reality unless $T$ is redefined accordingly.
	\end{itemize}
	
	\section{Conclusion}
	
	CMB temperature measurement relies on frequency measurements fitted to the Planck distribution. In natural units ($\hbar = c = k_B = 1$), the form simplifies, but $\alpha \approx 2.82$ remains. With $\alpha_W = 1$, $T$ becomes a frequency/energy, which is possible but uncommon, as it sacrifices direct connection to observable units (Kelvin). The document demonstrates this possibility, but practice favors SI units for comparability with measurement data.
	
	\begin{thebibliography}{9}
		\bibitem{Planck2018Temp}
		Planck Collaboration, Aghanim, N., et al. (2020). 
		\textit{Planck 2018 results. V. CMB power spectra and likelihoods}. 
		Astronomy \& Astrophysics, 641, A5. 
		DOI: 10.1051/0004-6361/201833887.
		
		\bibitem{Fixsen2009}
		Fixsen, D. J. (2009). \textit{The Temperature of the Cosmic Microwave Background}. 
		The Astrophysical Journal, 707(2), 916–920. 
		DOI: 10.1088/0004-637X/707/2/916.
		
		\bibitem{ACTTemp}
		Choi, S. K., et al. (2020).
		\textit{The Atacama Cosmology Telescope: A Measurement of the Cosmic Microwave Background Power Spectra at 98 and 150 GHz}.
		Journal of Cosmology and Astroparticle Physics, 2020(12), 045. 
		DOI: 10.1088/1475-7516/2020/12/045. 
		
		\bibitem{SPTTemp}
		Reichardt, C. L., et al. (2021). \textit{The South Pole Telescope 3G Survey: CMB Temperature and Polarization Power Spectra}. 
		The Astrophysical Journal, 908(2), 199. 
		DOI: 10.3847/1538-4357/abd407.
		
		\bibitem{Mather1994}
		Mather, J. C., et al. (1994). \textit{Measurement of the Cosmic Microwave Background Spectrum by the COBE FIRAS Instrument}. 
		The Astrophysical Journal, 420, 439–444. 
		DOI: 10.1086/173574.
		
		\bibitem{SunyaevZeldovich}
		Birkinshaw, M. (1999). \textit{The Sunyaev-Zel’dovich Effect}. 
		Physics Reports, 310(2–3), 97–195.
		DOI: 10.1016/S0370-1573(98)00080-5. 
		
		\bibitem{PlanckTech}
		Planck Collaboration, Tauber, J. A., et al. (2010). \textit{Planck Pre-Launch Status: The Planck Mission}. 
		Astronomy \& Astrophysics, 520, A1. 
		DOI: 10.1051/0004-6361/200912983.
		
		\bibitem{CMBTheoryTemp}
		Hu, W., \& Dodelson, S. (2002). \textit{Cosmic Microwave Background Anisotropies}. 
		Annual Review of Astronomy and Astrophysics, 40, 171–216. 
		DOI: 10.1146/annurev.astro.40.060401.093926. 
		
	\end{thebibliography}
	
\end{document}