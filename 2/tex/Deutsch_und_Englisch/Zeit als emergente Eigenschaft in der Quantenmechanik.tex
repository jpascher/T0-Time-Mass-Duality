\documentclass[12pt,a4paper]{article}
\usepackage[utf8]{inputenc}
\usepackage[T1]{fontenc}
\usepackage[ngerman]{babel}
\usepackage[left=2cm,right=2cm,top=2cm,bottom=2cm]{geometry}
\usepackage{lmodern}
\usepackage{amsmath}
\usepackage{amssymb}
\usepackage{physics}
\usepackage{hyperref}
\usepackage{tocloft} % Hinzugefügt für Inhaltsverzeichnis-Styling
\usepackage{tcolorbox}
\usepackage{booktabs}
\usepackage{enumitem}
\usepackage[table,xcdraw]{xcolor}
\usepackage{pgfplots}
\pgfplotsset{compat=1.18}
\usepackage{graphicx}
\usepackage{float}
\usepackage{mathtools}
\usepackage[T2A,T1]{fontenc}
\usepackage{fancyhdr} % Hinzugefügt für Kopf- und Fußzeilen

\renewcommand{\cftsecfont}{\color{blue}}
\renewcommand{\cftsubsecfont}{\color{blue}}
\renewcommand{\cftsecpagefont}{\color{blue}}
\renewcommand{\cftsubsecpagefont}{\color{blue}}
\setlength{\cftsecindent}{1cm}
\setlength{\cftsubsecindent}{2cm}

\hypersetup{
	colorlinks=true,
	linkcolor=blue,
	citecolor=blue,
	urlcolor=blue,
	pdftitle={Zeit als emergente Eigenschaft in der Quantenmechanik},
	pdfauthor={Johann Pascher},
	pdfsubject={Theoretische Physik},
	pdfkeywords={T0-Modell, Zeit-Masse-Dualität, Quantenmechanik, Feinstrukturkonstante}
}

% Benutzerdefinierte Befehle
\newcommand{\Tfield}{T(x)}
\newcommand{\betaT}{\beta_{\text{T}}}
\newcommand{\alphaEM}{\alpha_{\text{EM}}}
\newcommand{\alphaW}{\alpha_{\text{W}}}
\newcommand{\Mpl}{M_{\text{Pl}}}
\newcommand{\Tzerot}{T_0(\Tfield)}
\newcommand{\Tzero}{T_0}
\newcommand{\vecx}{\vec{x}}
\newcommand{\DhiggsT}{\Tfield (\partial_\mu + ig A_\mu) \Phi + \Phi \partial_\mu \Tfield} % Korrigierte Definition
\newcommand{\DcovT}[1]{\Tfield D_\mu #1 + #1 \partial_\mu \Tfield}
\newcommand{\HiggsLagr}{\mathcal{L}_{\text{Higgs-T}}}

% Kopf- und Fußzeilen
\pagestyle{fancy}
\fancyhf{}
\fancyhead[L]{Johann Pascher}
\fancyhead[R]{Zeit-Masse-Dualität}
\fancyfoot[C]{\thepage}
\renewcommand{\headrulewidth}{0.4pt}
\renewcommand{\footrulewidth}{0.4pt}

\title{Zeit als emergente Eigenschaft in der Quantenmechanik: \\Eine Verbindung zwischen Relativität, Feinstrukturkonstante und Quantendynamik}
\author{Johann Pascher}
\date{23. März 2025}

\begin{document}
	
	\maketitle
	
	\tableofcontents
	\newpage
	
	\section{Einführung}
	In der modernen Physik werden Zeit und Raum unterschiedlich behandelt. Während räumliche Koordinaten in der Quantenmechanik durch Operatoren dargestellt werden, erscheint Zeit primär als Parameter. Diese asymmetrische Behandlung wirft grundlegende Fragen zur Natur der Zeit auf. Diese Arbeit untersucht, inwieweit Zeit als emergente Eigenschaft verstanden werden kann, die mit fundamentalen Konstanten und der Masse des betrachteten Systems verknüpft ist.
	
	Diese Untersuchung ist Teil eines umfassenderen konzeptionellen Rahmens, der im Begleitpapier \href{https://github.com/jpascher/T0-Time-Mass-Duality/tree/main/2/pdf/Deutsch/Komplementäre\%20Erweiterungen\%20der\%20Physik\%20Absolute\%20Zeit\%20und\%20Intrinsische\%20Zeit.pdf}{\textit{Komplementäre Erweiterungen der Physik: Absolute Zeit und Intrinsische Zeit}} \cite{pascher1} (24. März 2025) detailliert diskutiert wird. Dort wird die hier eingeführte intrinsische Zeit mit einem komplementären Modell absoluter Zeit (dem \href{https://github.com/jpascher/T0-Time-Mass-Duality/tree/main/2/pdf/Deutsch/Zeit-Masse-Dualitätstheorie\%20(T0-Modell)\%20Herleitung\%20der\%20Parameter\%20kappa,\%20alpha\%20und\%20beta.pdf}{T0-Modell} \cite{pascher_params_2025}) verknüpft, was eine \href{https://github.com/jpascher/T0-Time-Mass-Duality/tree/main/2/pdf/Deutsch/Mathematische\%20Formulierungen\%20der\%20Zeit-Masse-Dualitätstheorie\%20mit\%20Lagrange.pdf}{Zeit-Masse-Dualität} \cite{pascher_lagrange_2025} hervorbringt, die konzeptionell der bekannten Welle-Teilchen-Dualität ähnelt.
	
	\section{Zeit in der speziellen Relativitätstheorie}
	Einsteins berühmte Formel \( E = mc^2 \) verbindet Energie und Masse über die Lichtgeschwindigkeit. Um eine Verbindung zur Zeit herzustellen, führen wir ein:
	\begin{align}
		E &= mc^2 \\
		E &= h\nu = \frac{h}{T}
	\end{align}
	wo \( h \) die Planck-Konstante, \( \nu \) die Frequenz und \( T \) die Periode ist. Durch Gleichsetzen erhalten wir:
	\begin{align}
		mc^2 &= \frac{h}{T} \\
		T &= \frac{h}{mc^2}
	\end{align}
	Diese Zeit \( T \) kann als charakteristische oder intrinsische Zeitskala interpretiert werden, die mit der Masse \( m \) verknüpft ist.
	
	\section{Verbindung zur Feinstrukturkonstante}
	Die Feinstrukturkonstante \( \alphaEM \) ist eine dimensionslose physikalische Konstante, die die Stärke der elektromagnetischen Wechselwirkung beschreibt:
	\begin{equation}
		\alphaEM = \frac{e^2}{4\pi\varepsilon_0\hbar c} \approx \frac{1}{137.035999}
	\end{equation}
	
	\subsection{Ableitung über elektromagnetische Konstanten}
	Die Planck-Konstante kann über elektromagnetische Vakuumkonstanten ausgedrückt werden:
	\begin{equation}
		h = \frac{1}{2\pi\sqrt{\mu_0\varepsilon_0}}
	\end{equation}
	Damit lässt sich die intrinsische Zeit \( T \) umschreiben:
	\begin{align}
		T &= \frac{h}{mc^2} \\
		&= \frac{1}{2\pi\sqrt{\mu_0\varepsilon_0}} \cdot \frac{1}{mc^2}
	\end{align}
	Da \( c = \frac{1}{\sqrt{\mu_0\varepsilon_0}} \), erhalten wir:
	\begin{align}
		T &= \frac{1}{2\pi\sqrt{\mu_0\varepsilon_0}} \cdot \frac{1}{m \cdot \frac{1}{\mu_0\varepsilon_0}} \\
		&= \frac{\mu_0\varepsilon_0}{2\pi m c}
	\end{align}
	
	\section{Zeit in der Quantenmechanik}
	\subsection{Standardbehandlung der Zeit}
	In der konventionellen Quantenmechanik erscheint Zeit als Parameter in der Schrödinger-Gleichung:
	\begin{equation}
		i\hbar \frac{\partial}{\partial t}\Psi(x,t) = \hat{H}\Psi(x,t)
	\end{equation}
	Im Gegensatz zu räumlichen oder Impulskoordinaten gibt es keinen Zeitoperator. Zeit wird als kontinuierlicher Parameter behandelt, entlang dessen Quantenzustände evolvieren.
	
	\subsection{Eine neue Perspektive: Intrinsische Zeit}
	Betrachten wir die charakteristische Zeit \( \Tfield = \frac{\hbar}{\max(m c^2, \omega)} \) als "intrinsische Zeit" eines Quantenobjekts, die sowohl massive Teilchen als auch Photonen einbezieht. Diese Zeit hängt von der Masse oder Energie des Objekts ab und könnte als minimale Zeitskala interpretiert werden, auf der das Objekt quantenmechanische Änderungen durchlaufen kann. Die Schrödinger-Gleichung wird modifiziert zu:
	\begin{equation}
		i\hbar \Tfield \frac{\partial}{\partial t} \Psi + i\hbar \Psi \frac{\partial \Tfield}{\partial t} = \hat{H} \Psi
	\end{equation}
	Dies impliziert, dass die Zeitenentwicklung nicht mehr einheitlich für alle Objekte ist, sondern von ihren Eigenschaften abhängt.
	
	\section{Erweiterte Beziehungen zwischen Zeit, Masse und fundamentalen Konstanten}
	\subsection{Eine einheitliche Beziehung zur Feinstrukturkonstanten}
	Mit \( T = \frac{\hbar}{mc^2} \) für massive Teilchen und erweitert:
	\begin{align}
		T &= \frac{\hbar}{mc^2} \cdot \frac{4\pi\varepsilon_0\hbar c}{e^2} \cdot \alphaEM
	\end{align}
	Dies zeigt eine tiefe Verbindung zwischen Zeitenentwicklung und elektromagnetischen Wechselwirkungen.
	
	\subsection{Modifizierte Dispersionsrelation}
	In der Standard-Quantenmechanik:
	\begin{equation}
		\omega = \frac{\hbar k^2}{2m}
	\end{equation}
	Mit \( \Tfield = \frac{\hbar}{\max(m c^2, \omega)} \), für massive Teilchen:
	\begin{equation}
		\omega_{\text{eff}} = \frac{\hbar^2 k^2}{2 m^2 c^2}
	\end{equation}
	Dies unterscheidet sich von der standardmäßigen \( \omega \propto \frac{1}{m} \).
	
	\subsection{Behandlung von Mehr-Teilchen-Systemen}
	Für zwei Teilchen (\( m_1 \), \( m_2 \)):
	\begin{equation}
		i (m_1 + m_2) c^2 \frac{\partial}{\partial t} \Psi(x_1, x_2, t) = \hat{H} \Psi(x_1, x_2, t)
	\end{equation}
	Für verschränkte Zustände:
	\begin{equation}
		|\Psi(t)\rangle = \frac{1}{\sqrt{2}}(|0(t/T_1)\rangle_{m_1} \otimes |1(t/T_2)\rangle_{m_2} + |1(t/T_1)\rangle_{m_1} \otimes |0(t/T_2)\rangle_{m_2})
	\end{equation}
	wobei \( T_1 = \frac{\hbar}{m_1 c^2} \), \( T_2 = \frac{\hbar}{m_2 c^2} \).
	
	\subsection{Massenabhängige Kohärenzzeiten und Instantaneität}
	Dekohärenzrate:
	\begin{equation}
		\Gamma_{\text{dec}} = \Gamma_0 \cdot \frac{m c^2}{\hbar}
	\end{equation}
	Energie-Zeit-Unsicherheit:
	\begin{equation}
		\Delta t \gtrsim \frac{\hbar}{mc^2}
	\end{equation}
	
	\section{Lagrange-Formulierung}
	\begin{equation}
		\mathcal{L}_{\text{Total}} = \mathcal{L}_{\text{Boson}} + \mathcal{L}_{\text{Fermion}} + \mathcal{L}_{\text{Higgs-T}} + \mathcal{L}_{\text{intrinsic}}, \quad \mathcal{L}_{\text{intrinsic}} = \frac{1}{2} \partial_\mu \Tfield \partial^\mu \Tfield - V(\Tfield)
	\end{equation}
	
	\section{Folgen für die Physik}
	\subsection{Eine neue Perspektive auf Zeit}
	Zeit wird nicht als fundamental abgeleitet, sondern als emergente Eigenschaft betrachtet.
	\subsection{Verbindung zur Zeitdilatation}
	Das T0-Modell spiegelt relativistische Effekte über Massenvariation wider.
	\subsection{Emergente Gravitation}
	Im T0-Modell entsteht Gravitation als emergente Kraft aus den Gradienten des intrinsischen Zeitfelds \(\Tfield\), definiert durch \( m = \frac{1}{\Tfield} \) im vereinheitlichten Einheitensystem (\(\hbar = c = G = \alphaEM = \betaT = 1\)). Das Gravitationspotential ergibt sich aus:
	\begin{equation}
		\Phi(r) = -\ln\left(\frac{\Tfield(r)}{\Tzero}\right) \approx \frac{M}{r},
	\end{equation}
	wobei \(\Tfield(r) = \Tzero \left(1 - \frac{M}{r}\right)\) für eine punktförmige Masse \(M\) gilt. Die resultierende Kraft ist:
	\begin{equation}
		\vec{F} = -\grad \Phi \approx -\frac{M}{r^2} \hat{r},
	\end{equation}
	was die Newtonsche Gravitation reproduziert. Auf größeren Skalen kann ein zusätzlicher Term \(\kappa r\) auftreten, der flache Rotationskurven erklärt. Eine ausführliche Herleitung findet sich in \cite{pascher_emergente_gravitation_2025, pascher_alphabeta_2025}.
	
	\section{Kosmologische Implikationen in SI-Einheiten}
	\begin{itemize}
		\item Rotverschiebung: Im T0-Modell wird die kosmische Rotverschiebung \(z\) durch die Variation des intrinsischen Zeitfelds \(\Tfield\) bestimmt, mit der Beziehung \(1 + z = \frac{\Tfield_0}{\Tfield}\), wobei \(\Tfield_0\) der lokale Wert des Zeitfelds ist. In SI-Einheiten ergibt sich \(1 + z = e^{\alpha d}\), mit \(\alpha \approx 2.3 \times 10^{-28} \, \text{m}^{-1}\), wobei \(\alpha = H_0/c\) die räumliche Variationsrate von \(\Tfield\) beschreibt \cite{pascher_galaxies_2025, pascher_emergente_gravitation_2025}.
		\item Gravitationspotential: Das emergente Gravitationspotential im T0-Modell lautet \(\Phi(r) = -\frac{GM}{r} + \kappa r\), mit \(\kappa = \frac{y v}{r_g}\) \cite{pascher_emergente_gravitation_2025}.
		\item Wellenlängenabhängigkeit: Die Rotverschiebung zeigt eine wellenlängenabhängige Komponente, beschrieben durch \(z(\lambda) = z_0 \left(1 + \betaT^{\text{SI}} \ln\left(\frac{\lambda}{\lambda_0}\right)\right)\), mit \(\betaT^{\text{SI}} \approx 0.008\). In natürlichen Einheiten mit \(\betaT^{\text{nat}} = 1\) wird sie zu \(z(\lambda) = z_0 \left(1 + \ln\left(\frac{\lambda}{\lambda_0}\right)\right)\) \cite{pascher_temp_2025, pascher_alphabeta_2025}.
	\end{itemize}
	
	\section{Rolle von \(\betaT\)}
	Der Parameter \(\betaT^{\text{SI}} \approx 0.008\) in SI-Einheiten dient als Faktor in der natürlichen Herleitung der wellenlängenabhängigen Rotverschiebung \( z(\lambda) = z_0 \left(1 + \betaT^{\text{SI}} \ln\left(\frac{\lambda}{\lambda_0}\right)\right) \) im T0-Modell \cite{pascher_alphabeta_2025}. In einem vereinheitlichten Einheitensystem wird \(\betaT^{\text{nat}} = 1\) gesetzt, was die charakteristische Längenskala \( r_0 \approx 1.33 \times 10^{-4} \cdot l_P \) definiert und die Konsistenz mit \(\alphaEM = 1\) unterstützt.
	
	\section{Experimentelle Überprüfungsmöglichkeiten}
	\begin{itemize}
		\item Unterschiede in Kohärenzzeiten: Messung zeitlicher Abweichungen in interferometrischen Experimenten.
		\item Massenabhängige Phasenverschiebungen: Untersuchung von Phasenunterschieden in Abhängigkeit von Teilchenmassen.
		\item Spektroskopie-Signaturen: Nachweis der wellenlängenabhängigen Rotverschiebung mit hochpräziser Spektroskopie.
		\item Test der modifizierten Dispersionsrelation: \( \omega_{\text{eff}} \propto \frac{1}{m^2} \), überprüfbar durch Lichtausbreitungsexperimente.
	\end{itemize}
	
	\section{Auswirkungen auf instantane Kohärenz in der Quantenmechanik}
	\subsection{Problem der instantanen Kohärenz}
	Quantenkorrelationen erscheinen instantan, was die Kausalität im Rahmen der Standardphysik herausfordert \cite{bell}.
	
	\subsection{Massenabhängige Kohärenzzeiten}
	Schwerere Teilchen dekohärieren schneller in Labzeit aufgrund der Massenvariation \( m = \frac{\hbar}{T c^2} \) im T0-Modell \cite{pascher_galaxies_2025}.
	
	\subsection{Mathematische Formulierung für Mehr-Teilchen-Systeme}
	Die Dynamik wird durch die Kopplung an \(\Tfield\) beschrieben (siehe detaillierte Gleichungen in \cite{pascher_lagrange_2025}).
	
	\subsection{Auswirkungen auf verschränkte Zustände}
	Massendifferenzen beeinflussen die Kohärenz verschränkter Zustände durch die intrinsische Zeit \(\Tfield\).
	
	\subsection{Neue Interpretation für EPR und Bell}
	Nichtlokalität könnte massenabhängige Zeitskalen reflektieren, testbar durch modifizierte Bell-Experimente mit variablen Massen \cite{bell}.
	
	\section{Schlussfolgerungen und Ausblick}
	Das T0-Modell betrachtet Zeit als emergent und bietet einen einheitlichen, experimentell überprüfbaren Rahmen, der Relativität, Quantenmechanik und die Feinstrukturkonstante miteinander verknüpft, wobei \(\betaT\) eine zentrale Rolle in der Herleitung kosmologischer und quantenmechanischer Effekte spielt \cite{pascher_galaxies_2025, pascher_alphabeta_2025}.
	
	\begin{thebibliography}{99}
		\bibitem{pascher1} Pascher, J. (2025). \href{https://github.com/jpascher/T0-Time-Mass-Duality/tree/main/2/pdf/Deutsch/Komplementäre\%20Erweiterungen\%20der\%20Physik\%20Absolute\%20Zeit\%20und\%20Intrinsische\%20Zeit.pdf}{Komplementäre Erweiterungen der Physik: Absolute Zeit und Intrinsische Zeit}. 24. März 2025.
		\bibitem{pascher_zeit_2025} Pascher, J. (2025). \href{https://github.com/jpascher/T0-Time-Mass-Duality/tree/main/2/pdf/Deutsch/Zeit\%20als\%20emergente\%20Eigenschaft\%20in\%20der\%20Quantenmechanik.pdf}{Zeit als emergente Eigenschaft in der Quantenmechanik: Eine Verbindung zwischen Relativität, Feinstrukturkonstante und Quantendynamik}. 23. März 2025.
		\bibitem{pascher_messdifferenzen_2025} Pascher, J. (2025). \href{https://github.com/jpascher/T0-Time-Mass-Duality/tree/main/2/pdf/Deutsch/Analyse\%20der\%20Messdifferenzen\%20zwischen\%20dem\%20T0-Modell\%20und\%20dem\%20Standardmodell.pdf}{Kompensatorische und additive Effekte: Eine Analyse der Messdifferenzen zwischen dem T0-Modell und dem \(\Lambda\)CDM-Standardmodell}. 2. April 2025.
		\bibitem{pascher_alpha_2025} Pascher, J. (2025). \href{https://github.com/jpascher/T0-Time-Mass-Duality/tree/main/2/pdf/Deutsch/Natürliche\%20Einheiten\%20mit\%20Feinstrukturkonstante\%20alpha\%20=\%201.pdf}{Energie als fundamentale Einheit: Natürliche Einheiten mit \(\alpha = 1\) im T0-Modell}. 26. März 2025.
		\bibitem{pascher_params_2025} Pascher, J. (2025). \href{https://github.com/jpascher/T0-Time-Mass-Duality/tree/main/2/pdf/Deutsch/Zeit-Masse-Dualitätstheorie\%20(T0-Modell)\%20Herleitung\%20der\%20Parameter\%20kappa,\%20alpha\%20und\%20beta.pdf}{Zeit-Masse-Dualitätstheorie (T0-Modell): Ableitung der Parameter \(\kappa\), \(\alpha\) und \(\beta\)}. 4. April 2025.
		\bibitem{pascher_higgs_2025} Pascher, J. (2025). \href{https://github.com/jpascher/T0-Time-Mass-Duality/tree/main/2/pdf/Deutsch/Mathematische\%20Formulierung\%20des\%20Higgs-Mechanismus\%20in\%20der\%20Zeit-Masse-Dualität.pdf}{Mathematische Formulierung des Higgs-Mechanismus in der Zeit-Masse-Dualität}. 28. März 2025.
		\bibitem{pascher_lagrange_2025} Pascher, J. (2025). \href{https://github.com/jpascher/T0-Time-Mass-Duality/tree/main/2/pdf/Deutsch/Mathematische\%20Formulierungen\%20der\%20Zeit-Masse-Dualitätstheorie\%20mit\%20Lagrange.pdf}{Von Zeitdilatation zu Massenvariation: Mathematische Kernformulierungen der Zeit-Masse-Dualitätstheorie}. 29. März 2025.
		\bibitem{pascher_emergente_gravitation_2025} Pascher, J. (2025). \href{https://github.com/jpascher/T0-Time-Mass-Duality/tree/main/2/pdf/Deutsch/Emergente\%20Gravitation\%20im\%20T0-Modell\%20Eine\%20formale\%20Herleitung.pdf}{Emergente Gravitation im T0-Modell: Eine umfassende Herleitung}. 1. April 2025.
		\bibitem{pascher_galaxies_2025} Pascher, J. (2025). \href{https://github.com/jpascher/T0-Time-Mass-Duality/tree/main/2/pdf/Deutsch/Massenvariation\%20in\%20Galaxien.pdf}{Massenvariation in Galaxien: Eine Analyse im T0-Modell mit emergenter Gravitation}. 30. März 2025.
		\bibitem{pascher_alphabeta_2025} Pascher, J. (2025). \href{https://github.com/jpascher/T0-Time-Mass-Duality/tree/main/2/pdf/Deutsch/Die\%20Konsistenz\%20von\%20alpha\%20=\%201\%20und\%20beta\%20=\%201.pdf}{Vereinheitlichtes Einheitensystem im T0-Modell: Die Konsistenz von \(\alpha = 1\) und \(\beta = 1\)}. 5. April 2025.
		\bibitem{einstein} Einstein, A. (1905). \textit{Zur Elektrodynamik bewegter Körper}. Annalen der Physik, 322(10), 891-921. DOI: 10.1002/andp.19053221004.
		\bibitem{bell} Bell, J. S. (1964). \textit{Zum Einstein-Podolsky-Rosen-Paradoxon}. Physics Physique {\fontencoding{T2A}\selectfont Физика}, 1(3), 195-200. DOI: \href{https://doi.org/10.1103/Physics.1.195}{10.1103/Physics.1.195}
		\bibitem{rubin1980} Rubin, V. C., Ford Jr, W. K., \& Thonnard, N. (1980). Rotational properties of 21 SC galaxies with a large range of luminosities and radii. \textit{The Astrophysical Journal}, 238, 471-487. DOI: 10.1086/158003.
		\bibitem{McGaugh2016} McGaugh, S. S., Lelli, F., \& Schombert, J. M. (2016). Radial acceleration relation in rotationally supported galaxies. \textit{Physical Review Letters}, 117(20), 201101. DOI: 10.1103/PhysRevLett.117.201101.
		\bibitem{Milgrom1983} Milgrom, M. (1983). A modification of the Newtonian dynamics. \textit{The Astrophysical Journal}, 270, 365-370. DOI: 10.1086/161130.
		\bibitem{Planck2018} Planck Collaboration (2020). Planck 2018 results. VI. Cosmological parameters. \textit{Astronomy \& Astrophysics}, 641, A6. DOI: 10.1051/0004-6361/201833910.
	\end{thebibliography}
	
\end{document}