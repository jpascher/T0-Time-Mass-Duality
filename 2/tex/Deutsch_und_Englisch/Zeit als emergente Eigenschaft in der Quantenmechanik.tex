\documentclass{article}
\usepackage[utf8]{inputenc}
\usepackage{amsmath}
\usepackage{amssymb}
\usepackage[margin=2cm]{geometry}
\usepackage[colorlinks=true, linkcolor=blue, citecolor=blue, urlcolor=blue]{hyperref}
\usepackage{siunitx}
\usepackage[ngerman]{babel}

\newcommand{\Tfield}{T(x)}

\title{Zeit als emergente Eigenschaft in der Quantenmechanik: \\Eine Verbindung zwischen Relativität, Feinstrukturkonstante und Quantendynamik}
\author{Johann Pascher}
\date{23. März 2025}

\begin{document}
	
	\maketitle
	
	\tableofcontents
	
	\section{Einführung}
	In der modernen Physik werden Zeit und Raum unterschiedlich behandelt. Während räumliche Koordinaten in der Quantenmechanik durch Operatoren dargestellt werden, erscheint Zeit primär als Parameter. Diese asymmetrische Behandlung wirft grundlegende Fragen zur Natur der Zeit auf. Diese Arbeit untersucht, inwieweit Zeit als emergente Eigenschaft verstanden werden kann, die mit fundamentalen Konstanten und der Masse des betrachteten Systems verknüpft ist.
	
	Diese Untersuchung ist Teil eines umfassenderen konzeptionellen Rahmens, der im Begleitpapier \textit{Komplementäre Erweiterungen der Physik: Absolute Zeit und Intrinsische Zeit} (24. März 2025) detailliert diskutiert wird. Dort wird die hier eingeführte intrinsische Zeit mit einem komplementären Modell absoluter Zeit (dem T0-Modell) verknüpft, was eine Zeit-Masse-Dualität hervorbringt, die konzeptionell der bekannten Welle-Teilchen-Dualität ähnelt.
	
	\section{Zeit in der speziellen Relativitätstheorie}
	Einsteins berühmte Formel \( E = mc^2 \) verbindet Energie und Masse über die Lichtgeschwindigkeit. Um eine Verbindung zur Zeit herzustellen, führen wir ein:
	\begin{align}
		E &= mc^2 \\
		E &= h\nu = \frac{h}{T}
	\end{align}
	wo \( h \) die Planck-Konstante, \( \nu \) die Frequenz und \( T \) die Periode ist. Durch Gleichsetzen erhalten wir:
	\begin{align}
		mc^2 &= \frac{h}{T} \\
		T &= \frac{h}{mc^2}
	\end{align}
	Diese Zeit \( T \) kann als charakteristische oder intrinsische Zeitskala interpretiert werden, die mit der Masse \( m \) verknüpft ist.
	
	\section{Verbindung zur Feinstrukturkonstanten}
	Die Feinstrukturkonstante \( \alpha \) ist eine dimensionslose physikalische Konstante, die die Stärke der elektromagnetischen Wechselwirkung beschreibt:
	\begin{equation}
		\alpha = \frac{e^2}{4\pi\varepsilon_0\hbar c} \approx \frac{1}{137.035999}
	\end{equation}
	
	\subsection{Ableitung über elektromagnetische Konstanten}
	Die Planck-Konstante kann über elektromagnetische Vakuumkonstanten ausgedrückt werden:
	\begin{equation}
		h = \frac{1}{2\pi\sqrt{\mu_0\varepsilon_0}}
	\end{equation}
	Damit lässt sich die intrinsische Zeit \( T \) umschreiben:
	\begin{align}
		T &= \frac{h}{mc^2} \\
		&= \frac{1}{2\pi\sqrt{\mu_0\varepsilon_0}} \cdot \frac{1}{mc^2}
	\end{align}
	Da \( c = \frac{1}{\sqrt{\mu_0\varepsilon_0}} \), erhalten wir:
	\begin{align}
		T &= \frac{1}{2\pi\sqrt{\mu_0\varepsilon_0}} \cdot \frac{1}{m \cdot \frac{1}{\mu_0\varepsilon_0}} \\
		&= \frac{1}{2\pi m c^3}
	\end{align}
	
	\section{Zeit in der Quantenmechanik}
	\subsection{Standardbehandlung der Zeit}
	In der konventionellen Quantenmechanik erscheint Zeit als Parameter in der Schrödinger-Gleichung:
	\begin{equation}
		i\hbar \frac{\partial}{\partial t}\Psi(x,t) = \hat{H}\Psi(x,t)
	\end{equation}
	Im Gegensatz zu räumlichen oder Impulskoordinaten gibt es keinen Zeitoperator. Zeit wird als kontinuierlicher Parameter behandelt, entlang dessen Quantenzustände evolvieren.
	
	\subsection{Eine neue Perspektive: Intrinsische Zeit}
	Betrachten wir die charakteristische Zeit \( \Tfield = \frac{\hbar}{\max(m c^2, \omega)} \) als "intrinsische Zeit" eines Quantenobjekts, die sowohl massive Teilchen als auch Photonen einbezieht. Diese Zeit hängt von der Masse oder Energie des Objekts ab und könnte als minimale Zeitskala interpretiert werden, auf der das Objekt quantenmechanische Änderungen durchlaufen kann. Die Schrödinger-Gleichung wird modifiziert zu:
	\begin{equation}
		i\hbar \Tfield \frac{\partial}{\partial t} \Psi + i\hbar \Psi \frac{\partial \Tfield}{\partial t} = \hat{H} \Psi
	\end{equation}
	Dies impliziert, dass die Zeitenentwicklung nicht mehr einheitlich für alle Objekte ist, sondern von ihren Eigenschaften abhängt.
	
	\section{Erweiterte Beziehungen zwischen Zeit, Masse und fundamentalen Konstanten}
	\subsection{Eine einheitliche Beziehung zur Feinstrukturkonstanten}
	Mit \( T = \frac{\hbar}{mc^2} \) für massive Teilchen und erweitert:
	\begin{align}
		T &= \frac{\hbar}{mc^2} \cdot \frac{4\pi\varepsilon_0\hbar c}{e^2} \cdot \alpha
	\end{align}
	Dies zeigt eine tiefe Verbindung zwischen Zeitenentwicklung und elektromagnetischen Wechselwirkungen.
	
	\subsection{Modifizierte Dispersionsrelation}
	In der Standard-Quantenmechanik:
	\begin{equation}
		\omega = \frac{\hbar k^2}{2m}
	\end{equation}
	Mit \( \Tfield = \frac{\hbar}{\max(m c^2, \omega)} \), für massive Teilchen:
	\begin{equation}
		\omega_{\text{eff}} = \frac{\hbar^2 k^2}{2 m^2 c^2}
	\end{equation}
	Dies unterscheidet sich von der standardmäßigen \( \omega \propto \frac{1}{m} \) und schlägt experimentelle Tests vor.
	
	\subsection{Behandlung von Mehr-Teilchen-Systemen}
	Für zwei Teilchen (\( m_1 \), \( m_2 \)):
	\begin{equation}
		i (m_1 + m_2) c^2 \frac{\partial}{\partial t} \Psi(x_1, x_2, t) = \hat{H} \Psi(x_1, x_2, t)
	\end{equation}
	Für verschränkte Zustände:
	\begin{equation}
		|\Psi(t)\rangle = \frac{1}{\sqrt{2}}(|0(t/T_1)\rangle_{m_1} \otimes |1(t/T_2)\rangle_{m_2} + |1(t/T_1)\rangle_{m_1} \otimes |0(t/T_2)\rangle_{m_2})
	\end{equation}
	wobei \( T_1 = \frac{\hbar}{m_1 c^2} \), \( T_2 = \frac{\hbar}{m_2 c^2} \).
	
	\subsection{Massenabhängige Kohärenzzeiten und Instantaneität}
	Dekohärenzrate:
	\begin{equation}
		\Gamma_{\text{dec}} = \Gamma_0 \cdot \frac{m c^2}{\hbar}
	\end{equation}
	Energie-Zeit-Unsicherheit:
	\begin{equation}
		\Delta t \gtrsim \frac{\hbar}{mc^2}
	\end{equation}
	
	\section{Lagrange-Formulierung}
	\begin{equation}
		\mathcal{L}_{\text{Total}} = \mathcal{L}_{\text{Boson}} + \mathcal{L}_{\text{Fermion}} + \mathcal{L}_{\text{Higgs-T}} + \mathcal{L}_{\text{intrinsic}}, \quad \mathcal{L}_{\text{intrinsic}} = \frac{1}{2} \partial_\mu \Tfield \partial^\mu \Tfield - V(\Tfield)
	\end{equation}
	
	\section{Folgen für die Physik}
	\subsection{Eine neue Perspektive auf Zeit}
	Zeit könnte abgeleitet sein, nicht fundamental.
	\subsection{Verbindung zur Zeitdilatation}
	Das T0-Modell spiegelt relativistische Effekte über Massenvariation wider.
	\subsection{Emergente Gravitation}
	Gravitation entsteht aus \( \nabla \Tfield \), mit Potential:
	\begin{equation}
		\Phi(r) = -\frac{GM}{r} + \kappa r, \quad \kappa \approx 4.8 \times 10^{-11} \, \text{m/s}^2
	\end{equation}
	
	\section{Kosmologische Implikationen}
	\begin{itemize}
		\item Rotverschiebung: \( 1 + z = e^{\alpha d} \), \( \alpha \approx 2.3 \times 10^{-28} \, \text{m}^{-1} \)
		\item Gravitationspotential: \( \Phi(r) = -\frac{GM}{r} + \kappa r \)
		\item Wellenlängenabhängigkeit: \( z(\lambda) = z_0 (1 + \beta \ln(\lambda/\lambda_0)) \), \( \beta \approx 0.008 \)
	\end{itemize}
	
	\section{Unsicherheit bei \(\beta\)}
	Der Parameter \( \beta \approx 0.008 \) ist unsicher; weitere Tests sind erforderlich.
	
	\section{Experimentelle Überprüfungsmöglichkeiten}
	- Unterschiede in Kohärenzzeiten
	- Massenabhängige Phasenverschiebungen
	- Spektroskopie-Signaturen
	- Test der modifizierten Dispersionsrelation \( \omega_{\text{eff}} \propto \frac{1}{m^2} \)
	
	\section{Auswirkungen auf instantane Kohärenz in der Quantenmechanik}
	\subsection{Problem der instantanen Kohärenz}
	Quantenkorrelationen erscheinen instantan, was Kausalität herausfordert.
	\subsection{Massenabhängige Kohärenzzeiten}
	Schwerere Teilchen dekohärieren schneller in Labzeit.
	\subsection{Mathematische Formulierung für Mehr-Teilchen-Systeme}
	Siehe Mehr-Teilchen-Gleichung oben.
	\subsection{Auswirkungen auf verschränkte Zustände}
	Massendifferenzen beeinflussen Kohärenz.
	\subsection{Neue Interpretation für EPR und Bell}
	Nichtlokalität könnte massenabhängige Zeitskalen widerspiegeln, testbar über Bell-Experimente.
	
	\section{Schlussfolgerungen und Ausblick}
	Das T0-Modell stellt Zeit als emergent dar und bietet einen einheitlichen, testbaren Rahmen, der Relativität, Quantenmechanik und die Feinstrukturkonstante verbindet.
	
	\begin{thebibliography}{9}
		\bibitem{pascher1} Pascher, J. (2025). \textit{Komplementäre Erweiterungen der Physik: Absolute Zeit und Intrinsische Zeit}.
		\bibitem{einstein} Einstein, A. (1905). \textit{Hängt die Trägheit eines Körpers von seinem Energiegehalt ab?}. Annalen der Physik, 323(13), 639-641.
		\bibitem{bell} Bell, J. S. (1964). \textit{Zum Einstein-Podolsky-Rosen-Paradoxon}. Physics, 1(3), 195-200.
	\end{thebibliography}
	
\end{document}