\documentclass{article}
\usepackage[utf8]{inputenc}
\usepackage{amsmath}
\usepackage{amssymb}
\usepackage[margin=2cm]{geometry}
\usepackage{tikz}
\usepackage{pgfplots}
\pgfplotsset{compat=1.18}
\usepackage{booktabs}
\usepackage{siunitx}
\usepackage{amsthm}
\usepackage[colorlinks=true, linkcolor=blue, citecolor=blue, urlcolor=blue]{hyperref}
\usepackage{cleveref}
\usepackage{tocloft}
\usepackage{xcolor}
\usepackage[ngerman]{babel}

\renewcommand{\cftsecfont}{\color{blue}}
\renewcommand{\cftsubsecfont}{\color{blue}}
\renewcommand{\cftsecpagefont}{\color{blue}}
\renewcommand{\cftsubsecpagefont}{\color{blue}}
\setlength{\cftsecindent}{1cm}
\setlength{\cftsubsecindent}{2cm}

\newcommand{\Tfield}{T(x)}

\newtheorem{theorem}{Satz}[section]
\newtheorem{proposition}[theorem]{Proposition}

\title{Die Notwendigkeit der Erweiterung der Standard-Quantenmechanik und Quantenfeldtheorie}
\author{Johann Pascher}
\date{27. März 2025}

\begin{document}
	
	\maketitle
	
	\begin{abstract}
		Diese Arbeit untersucht die konzeptionellen Grenzen der Standard-Quantenmechanik (QM) und Quantenfeldtheorie (QFT) und schlägt die Zeit-Masse-Dualität mit einem intrinsischen Zeitfeld als Erweiterung vor. Durch die Einführung von \(\Tfield = \frac{\hbar}{\max(m c^2, \omega)}\) wird eine Verbindung zwischen Zeit und Masse hergestellt, die QM-QFT-Dualität überwunden und ein deterministischer Rahmen geboten. Die Theorie wird durch experimentelle Vorhersagen und kosmologische Implikationen gestützt.
	\end{abstract}
	
	\tableofcontents
	\newpage
	
	\section{Einführung: Konzeptionelle Grenzen etablierter Theorien}
	Die Standard-Quantenmechanik (QM) und Quantenfeldtheorie (QFT) stoßen bei der Integration mit der Allgemeinen Relativitätstheorie (AR) und beim Verständnis von Zeit und Masse an Grenzen. Das T0-Modell bietet eine neue Perspektive, wie in \textit{Wesentliche mathematische Formalismen der Zeit-Masse-Dualitätstheorie mit Lagrange-Dichten} beschrieben \cite{wesentlicheFormalismen}.
	
	\subsection{Inhärente Dualität zwischen QM und QFT}
	\begin{itemize}
		\item QM: Teilchenperspektive \cite{schrodinger}.
		\item QFT: Feldbasierte Sicht.
	\end{itemize}
	
	\subsection{Überinterpretation durch unvollständige theoretische Grundlagen}
	\begin{itemize}
		\item Messproblem \cite{einstein2}.
		\item Nichtlokalität \cite{bell}.
	\end{itemize}
	
	\section{Asymmetrische Behandlung von Zeit und Raum}
	\subsection{Zeit als Parameter vs. Raum als Operator}
	In der Standard-Quantenmechanik wird Zeit als Parameter behandelt:
	\begin{equation}
		i\hbar \frac{\partial}{\partial t}\Psi(x,t) = \hat{H}\Psi(x,t)
	\end{equation}
	Räumliche Koordinaten hingegen werden durch Operatoren beschrieben, was eine asymmetrische Behandlung von Zeit und Raum zur Folge hat.
	
	\section{Statische Behandlung der Masse}
	\subsection{Masse als unveränderlicher Parameter}
	In der Standardformulierung bleibt die Masse konstant:
	\begin{equation}
		\hat{H} = \frac{\hat{p}^2}{2m} + V(\hat{x})
	\end{equation}
	Diese statische Behandlung der Masse schränkt die Flexibilität der Theorie ein und verhindert eine dynamische Integration von Masse und Zeit.
	
	\section{Das Konzept der intrinsischen Zeit}
	\begin{theorem}[Intrinsische Zeit]
		Die intrinsische Zeit wird definiert als:
		\begin{equation}
			\Tfield = \frac{\hbar}{\max(m c^2, \omega)}
		\end{equation}
	\end{theorem}
	Diese Definition vereinheitlicht die Behandlung von massiven Teilchen und Photonen. Die modifizierte Schrödinger-Gleichung lautet:
	\begin{equation}
		i\hbar \Tfield \frac{\partial}{\partial t} \Psi + i\hbar \Psi \frac{\partial \Tfield}{\partial t} = \hat{H} \Psi
	\end{equation}
	Hierdurch wird die Zeitenentwicklung massenabhängig, was eine dynamischere Beschreibung ermöglicht.
	
	\section{Zeit-Masse-Dualität: Ein neuer theoretischer Rahmen}
	\subsection{Komplementäre Modelle}
	\begin{itemize}
		\item Standardmodell: Konstante Masse, variable Zeit.
		\item T0-Modell: Absolute Zeit, variable Masse.
	\end{itemize}
	Die Zeit-Masse-Dualität bietet eine alternative Perspektive, die die Begrenzungen der traditionellen Ansätze überwindet.
	
	\section{Lagrange-Formulierung}
	Die Gesamt-Lagrangedichte des T0-Modells lautet:
	\begin{equation}
		\mathcal{L}_{\text{Total}} = \mathcal{L}_{\text{Boson}} + \mathcal{L}_{\text{Fermion}} + \mathcal{L}_{\text{Higgs-T}} + \mathcal{L}_{\text{intrinsic}}, \quad \mathcal{L}_{\text{intrinsic}} = \frac{1}{2} \partial_\mu \Tfield \partial^\mu \Tfield - V(\Tfield)
	\end{equation}
	Dieser Ansatz integriert die Dynamik des intrinsischen Zeitfelds in die bestehenden Feldtheorien.
	
	\section{Folgen für fundamentale Phänomene}
	\subsection{Quanten-Kohärenz und Dekohärenz}
	Die Dekohärenzrate wird massenabhängig:
	\begin{equation}
		\Gamma_{\text{dec}} = \Gamma_0 \cdot \frac{m c^2}{\hbar}
	\end{equation}
	Gravitation entsteht als emergente Eigenschaft aus Gradienten des intrinsischen Zeitfelds:
	\begin{equation}
		\nabla \Tfield = -\frac{\hbar}{m^2 c^2} \nabla m
	\end{equation}
	mit dem modifizierten Gravitationspotential:
	\begin{equation}
		\Phi(r) = -\frac{GM}{r} + \kappa r, \quad \kappa \approx 4.8 \times 10^{-11} \, \text{m/s}^2
	\end{equation}
	
	\begin{figure}[h]
		\centering
		\begin{tikzpicture}
			\begin{axis}[
				xlabel={Masse [eV]},
				ylabel={Kohärenzzeit [eV\(^{-1}\)]},
				xlabel style={font=\large},
				ylabel style={font=\large},
				tick label style={font=\normalsize},
				xmin=0, xmax=1000,
				ymin=0, ymax=0.01,
				legend pos=north east,
				legend style={font=\large},
				grid=both,
				minor tick num=1
				]
				\addplot[blue, ultra thick, domain=1:1000, samples=100] {1/x};
				\legend{\(\tau \propto 1/m\)}
			\end{axis}
		\end{tikzpicture}
		\caption{Massenabhängige Kohärenzzeit im T0-Modell.}
	\end{figure}
	
	\section{Variable Masse als verborgene Variable}
	\subsection{Modifizierte Quanten-Dynamik}
	Die Zeitentwicklung kann auch durch eine variable Masse beschrieben werden:
	\begin{equation}
		i\hbar \frac{\partial}{\partial t}\Psi(x,t) = \hat{H}(m(t))\Psi(x,t)
	\end{equation}
	Dies deutet darauf hin, dass Masse als verborgene Variable fungieren könnte, die die scheinbare Unbestimmtheit der Quantenmechanik erklärt.
	
	\section{Kosmologische Implikationen}
	Das T0-Modell hat weitreichende kosmologische Auswirkungen:
	\begin{itemize}
		\item Rotverschiebung: \( 1 + z = e^{\alpha d} \), \( \alpha \approx 2.3 \times 10^{-28} \, \text{m}^{-1} \) \cite{wesentlicheFormalismen}.
		\item Gravitationspotential: \( \Phi(r) = -\frac{GM}{r} + \kappa r \), \( \kappa \approx 4.8 \times 10^{-11} \, \text{m/s}^2 \) \cite{wesentlicheFormalismen}.
		\item Wellenlängenabhängigkeit: \( z(\lambda) = z_0 (1 + \beta \ln(\lambda/\lambda_0)) \), \( \beta \approx 0.008 \).
	\end{itemize}
	
	\section{Unsicherheit bei \(\beta\)}
	Der Parameter \( \beta \approx 0.008 \) ist unsicher; Werte wie \( \beta = 1 \) würden den Beobachtungen widersprechen. Weitere experimentelle Tests sind erforderlich, um \( \beta \) einzugrenzen.
	
	\section{Schlussfolgerung}
	Das T0-Modell erweitert die Standard-Quantenmechanik und Quantenfeldtheorie durch die Einführung der Zeit-Masse-Dualität und des intrinsischen Zeitfelds. Es bietet einen deterministischen Rahmen, der die traditionelle Dualität zwischen QM und QFT überwindet und durch experimentelle Vorhersagen wie massenabhängige Kohärenzzeiten und kosmologische Effekte gestützt wird. Diese Erweiterung könnte einen wichtigen Schritt hin zu einer einheitlicheren Theorie der Physik darstellen, die Quantenmechanik und Gravitation integriert.
	
	\begin{thebibliography}{99}
		\bibitem{wesentlicheFormalismen} Pascher, J. (2025). \textit{Wesentliche mathematische Formalismen der Zeit-Masse-Dualitätstheorie mit Lagrange-Dichten}. 29. März 2025.
		\bibitem{einstein} Einstein, A. (1905). \textit{Hängt die Trägheit eines Körpers von seinem Energiegehalt ab?}. Annalen der Physik, 323(13), 639-641.
		\bibitem{planck} Planck, M. (1901). \textit{Über das Gesetz der Energieverteilung im Normalspektrum}. Annalen der Physik, 309(3), 553-563.
		\bibitem{schrodinger} Schrödinger, E. (1926). \textit{Eine undulatorische Theorie der Mechanik von Atomen und Molekülen}. Physical Review, 28(6), 1049-1070.
		\bibitem{bell} Bell, J. S. (1964). \textit{Zum Einstein-Podolsky-Rosen-Paradoxon}. Physics, 1(3), 195-200.
		\bibitem{einstein2} Einstein, A., Podolsky, B., Rosen, N. (1935). \textit{Kann die quantenmechanische Beschreibung der physikalischen Realität als vollständig betrachtet werden?}. Physical Review, 47(10), 777-780.
	\end{thebibliography}
	
\end{document}