\documentclass[a4paper,12pt]{article}
\usepackage[utf8]{inputenc}
\usepackage{amsmath}
\usepackage{amssymb}
\usepackage[margin=2cm]{geometry}
\usepackage{tikz}
\usepackage{tcolorbox}
\usepackage[colorlinks=true, linkcolor=blue, citecolor=blue, urlcolor=blue]{hyperref}
\usepackage{siunitx}
\usepackage[ngerman]{babel}

\newcommand{\Tfield}{T(x)}

\title{Reale Konsequenzen der Umformulierung von Zeit und Masse in der Physik: Jenseits der Planck-Skala}
\author{Johann Pascher}
\date{24. März 2025}

\begin{document}
	
	\maketitle
	
	\begin{abstract}
		Diese Arbeit untersucht die realen Konsequenzen der Umformulierung von Zeit und Masse im T0-Modell, basierend auf absoluter Zeit und intrinsischem Zeitfeld. Innerhalb der Grenzen von Lichtgeschwindigkeit und Planck-Masse werden kosmologische, quantenmechanische und gravitative Implikationen analysiert, während spekulative Erweiterungen jenseits dieser Grenzen neue Perspektiven auf Singularitäten und Kausalität eröffnen. Die Modelle bieten testbare Vorhersagen und eine philosophische Neuinterpretation der physikalischen Realität.
	\end{abstract}
	
	\tableofcontents
	\newpage
	
	\section{Einleitung}
	Diese Arbeit untersucht die realen Konsequenzen der Umformulierung grundlegender physikalischer Konzepte, insbesondere von Zeit und Masse, wie sie in meinen vorherigen Studien vorgestellt wurden: \textit{Komplementäre Erweiterungen der Physik: Absolute Zeit und Intrinsische Zeit} (24. März 2025) \cite{komplementaer}, \textit{Ein Modell mit absoluter Zeit und variabler Energie: Eine ausführliche Untersuchung der Grundlagen} (24. März 2025) \cite{absolateZeit} und \textit{Erweiterungen der Quantenmechanik durch intrinsische Zeit} (24. März 2025) \cite{erweiterungenQM}. Diese schlagen alternative Rahmenwerke vor – absolute Zeit mit variabler Masse und eine massenabhängige intrinsische Zeit –, die die herkömmlichen Interpretationen der speziellen Relativitätstheorie und der Quantenmechanik herausfordern. Die Gültigkeitsgrenzen sind Lichtgeschwindigkeit (\( c_0 \approx 3 \times 10^8 \, \text{m/s} \)) und Planck-Masse (\( m_P = \sqrt{\frac{\hbar c_0}{G}} \approx 2.176 \times 10^{-8} \, \text{kg} \)), doch die Modelle erstrecken sich spekulativ darüber hinaus.
	
	\section{Festlegung der Grenzen: Lichtgeschwindigkeit und Planck-Masse}
	Die Lichtgeschwindigkeit \( c_0 \) und die Planck-Masse \( m_P \) dienen als fundamentale Einschränkungen und markieren die Bereiche, in denen quantengravitative Effekte bedeutend werden, typischerweise verbunden mit der Planck-Zeit (\( t_P = \sqrt{\frac{\hbar G}{c_0^5}} \approx 5.39 \times 10^{-44} \, \text{s} \)) und der Planck-Länge (\( l_P = \sqrt{\frac{\hbar G}{c_0^3}} \approx 1.616 \times 10^{-35} \, \text{m} \)).
	
	\begin{tcolorbox}[colback=blue!5!white,colframe=blue!75!black,title=Definitionen der Modelle]
		\textbf{Standardmodell der SRT:}
		\begin{itemize}
			\item Zeitdilatation: \( t' = \gamma t \)
			\item Ruhemasse konstant: \( m_0 = \text{konst.} \)
			\item Relativistische Masse: \( m_{rel} = \gamma m_0 \)
			\item Energie: \( E = m_{rel}c_0^2 \)
		\end{itemize}
		\textbf{T0-Modell mit absoluter Zeit:}
		\begin{itemize}
			\item Zeit absolut: \( T_0 = \text{konst.} \)
			\item Masse variabel: \( m = \gamma m_0 \)
			\item Energie: \( E = \frac{\hbar}{T_0} \)
		\end{itemize}
		\textbf{Modell mit intrinsischer Zeit:}
		\begin{itemize}
			\item Intrinsisches Zeitfeld: \( \Tfield = \frac{\hbar}{\max(m c^2, \omega)} \)
			\item Zeitentwicklung: \( i\hbar \Tfield \frac{\partial}{\partial t} \Psi + i\hbar \Psi \frac{\partial \Tfield}{\partial t} = \hat{H} \Psi \)
		\end{itemize}
	\end{tcolorbox}
	
	\section{Über die Grenzen hinaus}
	Trotz der festgelegten Grenzen laden die Modelle dazu ein, über \( c_0 \) und \( m_P \) hinauszugehen:
	- \textbf{Nahe der Singularität}: \( m = \frac{\hbar}{T_0 c_0^2} \) deutet auf einen endlichen Energiezustand hin.
	- \textbf{Sub-Planck-Massen}: \( \Tfield > t_P \) impliziert eine langsamere Zeitentwicklung für leichtere Teilchen.
	
	\begin{figure}[h]
		\centering
		\begin{tikzpicture}
			\draw[->] (0,0) -- (6,0) node[right] {Masse \(m\)};
			\draw[->] (0,0) -- (0,4) node[above] {Intrinsische Zeit \(T\)};
			\draw[scale=0.5, domain=0.1:10, smooth, variable=\x, blue, thick] plot ({\x}, {1/\x});
			\draw[dotted, red] (1.5,0) -- (1.5,1.5) -- (0,1.5);
			\node at (1.5,-0.3) {\(m_P\)};
			\node at (-0.3,1.5) {\(t_P\)};
			\node[blue] at (4.5,2) {\(T = \frac{\hbar}{mc^2}\)};
		\end{tikzpicture}
		\caption{Beziehung zwischen Masse und intrinsischer Zeit.}
	\end{figure}
	
	\section{Reale Interpretative Konsequenzen}
	\subsection{Kosmologische Implikationen}
	Im T0-Modell wird die Rotverschiebung als Energieverlust interpretiert:
	\begin{itemize}
		\item Rotverschiebung: \( 1 + z = e^{\alpha d} \), \( \alpha \approx 2.3 \times 10^{-28} \, \text{m}^{-1} \)
		\item CMB als statisches Feld mit Massengradienten
		\item Hochenergetischer Zustand statt Singularität
	\end{itemize}
	
	\begin{tcolorbox}[colback=green!5!white,colframe=green!75!black,title=Neuinterpretation kosmologischer Phänomene]
		\textbf{Standardmodell:}
		\begin{itemize}
			\item Rotverschiebung \( z = \frac{\lambda_{beobachtet} - \lambda_{emittiert}}{\lambda_{emittiert}} \) als Folge der Expansion.
			\item CMB als abgekühlte Strahlung des frühen Universums.
			\item Big Bang als Anfangssingularität.
		\end{itemize}
		\textbf{T0-Modell:}
		\begin{itemize}
			\item Rotverschiebung als Energieverlust \( E_2 = E_1(1+z)^{-1} \).
			\item CMB als statisches Feld mit Massengradienten.
			\item Hochenergetischer Zustand statt Singularität.
		\end{itemize}
		\textbf{Testbare Vorhersagen:}
		\begin{itemize}
			\item Abweichungen in der Rotverschiebungs-Entfernungs-Beziehung: \( 1 + z = e^{\alpha d} \).
			\item Anisotropien im CMB mit massenabhängiger Charakteristik.
			\item Modifizierte Muster der primordialen Nukleosynthese.
		\end{itemize}
	\end{tcolorbox}
	
	\subsection{Quantenmechanik und Gravitation}
	Gravitation entsteht aus \( \nabla \Tfield \):
	\begin{equation}
		\Phi(r) = -\frac{GM}{r} + \kappa r, \quad \kappa \approx 4.8 \times 10^{-11} \, \text{m/s}^2
	\end{equation}
	Dies bietet eine Brücke zur Quantengravitation.
	
	\subsection{Nichtlokalität in der Quantenphysik}
	Korrelationen über Massenvariation:
	\begin{figure}[h]
		\centering
		\begin{tikzpicture}
			\draw[->] (0,0) -- (5,0) node[right] {Zeit \(t\)};
			\draw[<->] (1,1) -- (4,1);
			\draw[<->] (1,2) -- (3,2);
			\node at (0.5,1) {\(m_1\)};
			\node at (0.5,2) {\(m_2 < m_1\)};
			\draw[dotted] (1,0) -- (1,2.5);
			\draw[dotted] (3,0) -- (3,2.5);
			\draw[dotted] (4,0) -- (4,2.5);
			\node at (2.5,0.5) {\(T_1 = \frac{\hbar}{m_1c^2}\)};
			\node at (2,1.5) {\(T_2 = \frac{\hbar}{m_2c^2}\)};
			\node at (2.5,2.5) {Verzögerung \(\propto \frac{m_1}{m_2}\)};
		\end{tikzpicture}
		\caption{Verzögerte Korrelation bei verschränkten Teilchen.}
	\end{figure}
	
	\section{Lagrange-Formulierung}
	\begin{equation}
		\mathcal{L}_{\text{Total}} = \mathcal{L}_{\text{Boson}} + \mathcal{L}_{\text{Fermion}} + \mathcal{L}_{\text{Higgs-T}} + \mathcal{L}_{\text{intrinsic}}, \quad \mathcal{L}_{\text{intrinsic}} = \frac{1}{2} \partial_\mu \Tfield \partial^\mu \Tfield - V(\Tfield)
	\end{equation}
	
	\section{Auswirkungen auf den Lichtkegel}
	Kausalität wird durch Massenvariation neu definiert:
	\begin{equation}
		\mathcal{O}_{T_0} = c_0^2 T_0^2 - |\vec{x}|^2
	\end{equation}
	Mit intrinsischer Zeit:
	\begin{equation}
		ds^2 = \frac{\hbar^2}{m^2} dt^2 - d\vec{x}^2
	\end{equation}
	
	\section{Schlussfolgerungen und Ausblick}
	Das T0-Modell bietet eine alternative Sicht auf physikalische Phänomene jenseits traditioneller Grenzen.
	
	\section{Unsicherheit bei \(\beta\)}
	Der Parameter \( \beta \approx 0.008 \) ist unsicher; weitere Tests sind nötig.
	
	\begin{thebibliography}{9}
		\bibitem{wesentlicheFormalismen} Pascher, J. (2025). \textit{Wesentliche mathematische Formalismen der Zeit-Masse-Dualitätstheorie mit Lagrange-Dichten}.
		\bibitem{komplementaer} Pascher, J. (2025). \textit{Komplementäre Erweiterungen der Physik: Absolute Zeit und Intrinsische Zeit}.
	\end{thebibliography}
	
\end{document}