% T0 Standalone Header - German Version
% Gemeinsamer Header für alle deutschen Standalone-Dokumente

\documentclass[12pt,a4paper]{report}

% Sprachunterstützung - Deutsch
\usepackage[ngerman]{babel}
\usepackage[utf8]{inputenc}
\usepackage[T1]{fontenc}

% Mathematik
\usepackage{amsmath,amssymb,amsfonts}
\usepackage{mathtools}
\usepackage{physics}

% Grafik und Farben
\usepackage{graphicx}
\usepackage{xcolor}
\usepackage{tikz}
\usetikzlibrary{arrows,positioning,shapes}

% Tabellen
\usepackage{booktabs}
\usepackage{longtable}
\usepackage{array}
\usepackage{multirow}
\usepackage{graphicx} % für resizebox

% Layout
\usepackage{geometry}
\geometry{margin=2.5cm}
\usepackage{fancyhdr}
\usepackage{pdflscape}

% Hyperlinks
\usepackage{hyperref}
\hypersetup{
    colorlinks=true,
    linkcolor=blue,
    filecolor=magenta,
    urlcolor=cyan,
    citecolor=blue
}

% Zitate und Referenzen
\usepackage{cite}

% Boxen
\usepackage{tcolorbox}
\tcbuselibrary{theorems,skins,breakable}

% Farben
\definecolor{t0blue}{RGB}{0,102,204}
\definecolor{t0green}{RGB}{0,153,51}
\definecolor{t0red}{RGB}{204,0,0}
\definecolor{t0yellow}{RGB}{255,204,0}

% tcolorbox Umgebungen
\newtcolorbox{insight}[1][]{colback=blue!5,colframe=t0blue,title={Erkenntnis},#1}
\newtcolorbox{discovery}[1][]{colback=green!5,colframe=t0green,title={Entdeckung},#1}
\newtcolorbox{newperspective}[1][]{colback=yellow!5,colframe=orange,title={Neue Perspektive},#1}
\newtcolorbox{revelation}[1][]{colback=red!5,colframe=t0red,title={Offenbarung},#1}
\newtcolorbox{keypoint}[1][]{colback=blue!5,colframe=t0blue,title={Kernpunkt},#1}
\newtcolorbox{evidence}[1][]{colback=green!5,colframe=t0green,title={Beleg},#1}
\newtcolorbox{conclusion}[1][]{colback=gray!5,colframe=gray,title={Schlussfolgerung},#1}
\newtcolorbox{significance}[1][]{colback=yellow!5,colframe=orange,title={Bedeutung},#1}
\newtcolorbox{philosophical}[1][]{colback=purple!5,colframe=purple,title={Philosophisch},#1}
\newtcolorbox{implication}[1][]{colback=cyan!5,colframe=cyan,title={Implikation},#1}
\newtcolorbox{perspective}[1][]{colback=blue!5,colframe=t0blue,title={Perspektive},#1}
\newtcolorbox{revolutionary}[1][]{colback=red!5,colframe=t0red,title={Revolutionär},#1}

% Theorem-Umgebungen
\newtheorem{theorem}{Satz}[chapter]
\newtheorem{lemma}[theorem]{Lemma}
\newtheorem{corollary}[theorem]{Korollar}
\newtheorem{proposition}[theorem]{Proposition}
\newtheorem{definition}[theorem]{Definition}
\newtheorem{example}[theorem]{Beispiel}
\newtheorem{remark}[theorem]{Bemerkung}
\newtheorem{note}[theorem]{Anmerkung}

% Zusätzliche Umgebungen für T0-Dokumente
\newenvironment{relation}{\begin{quote}\itshape}{\end{quote}}
\newenvironment{treatise}{\begin{quote}}{\end{quote}}
\newenvironment{gemeinsam}{\begin{quote}}{\end{quote}}
\newenvironment{vergleich}{\begin{quote}}{\end{quote}}
\newenvironment{vorteil}{\begin{quote}}{\end{quote}}
\newenvironment{quantum}{\begin{quote}}{\end{quote}}
\newenvironment{principle}{\begin{quote}\bfseries}{\end{quote}}

% T0-spezifische Befehle
\newcommand{\Tzero}{T$_0$}
\newcommand{\xipar}{\xi}
\newcommand{\Tfield}{T}
\newcommand{\Efield}{\mathcal{E}}
\newcommand{\meff}{m_{\text{eff}}}
\newcommand{\Eabs}{E_{\text{abs}}}
\newcommand{\taupar}{\tau}

% Einheiten (falls siunitx nicht verfügbar)
\providecommand{\SI}[2]{#1\,#2}
\providecommand{\si}[1]{#1}
\providecommand{\num}[1]{#1}
\providecommand{\qty}[2]{#1\,#2}

% Deutsche Anführungszeichen
\newcommand{\dq}[1]{\glqq{}#1\grqq{}}

