\documentclass[12pt,a4paper]{article}
\usepackage[utf8]{inputenc}
\usepackage[T1]{fontenc}
\usepackage[ngerman]{babel}
\usepackage{lmodern}
\usepackage{amsmath}
\usepackage{amssymb}
\usepackage{physics}
\usepackage{hyperref}
\usepackage{tcolorbox}
\usepackage{booktabs}
\usepackage{enumitem}
\usepackage[table,xcdraw]{xcolor}
\usepackage[left=2cm,right=2cm,top=2cm,bottom=2cm]{geometry}
\usepackage{pgfplots}
\pgfplotsset{compat=1.18}
\usepackage{graphicx}
\usepackage{float}
\usepackage{fancyhdr}
\usepackage{siunitx}
\usepackage{mathtools}
\usepackage{amsthm}
\usepackage{cleveref}
\usepackage{tocloft}

% Kopf- und Fußzeilen
\pagestyle{fancy}
\fancyhf{}
\fancyhead[L]{Johann Pascher}
\fancyhead[R]{Reine Energie T0-Theorie: Verhältnis-basierte Physik}
\fancyfoot[C]{\thepage}
\renewcommand{\headrulewidth}{0.4pt}
\renewcommand{\footrulewidth}{0.4pt}
\setlength{\headheight}{15pt}

% Inhaltsverzeichnis-Formatierung
\renewcommand{\cftsecfont}{\color{blue}}
\renewcommand{\cftsubsecfont}{\color{blue}}
\renewcommand{\cftsecpagefont}{\color{blue}}
\renewcommand{\cftsubsecpagefont}{\color{blue}}
\setlength{\cftsecindent}{1cm}
\setlength{\cftsubsecindent}{2cm}

% Benutzerdefinierte Befehle
\newcommand{\Lag}{\mathcal{L}}
\newcommand{\deltam}{\delta m}
\newcommand{\Efield}{E}
\newcommand{\xipar}{\xi}

% Theorem-Umgebungen
\newtheorem{theorem}{Theorem}[section]
\newtheorem{proposition}[theorem]{Proposition}
\newtheorem{corollary}[theorem]{Korollar}
\newtheorem{lemma}[theorem]{Lemma}
\theoremstyle{definition}
\newtheorem{definition}[theorem]{Definition}
\newtheorem{example}[theorem]{Beispiel}
\theoremstyle{remark}
\newtheorem{remark}[theorem]{Bemerkung}

\hypersetup{
	colorlinks=true,
	linkcolor=blue,
	citecolor=blue,
	urlcolor=blue,
	pdftitle={Reine Energie T0-Theorie: Verhältnis-basierte Physik mit SI-Referenz},
	pdfauthor={Johann Pascher},
	pdfsubject={T0-Theorie, Verhältnis-basierte Physik, Energie-Skalierung},
	pdfkeywords={T0-Theorie, Energie-Verhältnisse, Skalen-Beziehungen, SI-Referenz}
}

\title{Reine Energie T0-Theorie: Die Verhältnis-basierte Revolution \\
	Von Parameter-Physik zu Skalen-Beziehungen \\
	\large Aufbauend auf vereinfachter Dirac- und universeller Lagrange-Grundlage}
\author{Johann Pascher\\
	Abteilung für Nachrichtentechnik, \\Höhere Technische Bundeslehranstalt (HTL), Leonding, Österreich\\
	\texttt{johann.pascher@gmail.com}}
\date{\today}

\begin{document}
	
	\maketitle
	
	\begin{abstract}
		Diese Arbeit präsentiert den Höhepunkt der T0-theoretischen Revolution: eine vollständig verhältnis-basierte Physik, die die Notwendigkeit multipler experimenteller Parameter eliminiert. Aufbauend auf den vereinfachten Dirac-Gleichungs- und universellen Lagrange-Einsichten demonstrieren wir, dass fundamentale Physik durch dimensionslose Energie-Skalen-Verhältnisse operiert, nicht durch zugewiesene Parameter. Das T0-System benötigt nur einen SI-Referenzwert, um reine verhältnis-basierte Physik mit messbaren Größen zu verbinden. Wir zeigen, dass Einsteins $E = mc^2$ Masse als konzentrierte Energie offenbart und zu universellen Energie-Beziehungen mit 100\% mathematischer Genauigkeit führt, verglichen mit 99.98\% Genauigkeit komplexer Multi-Parameter-Formeln. Alle Physik reduziert sich auf Energie-Skalen-Verhältnisse, regiert von der ultimativen Gleichung $\partial^2 \Efield = 0$, mit quantitativen Vorhersagen ermöglicht durch einen einzigen SI-Referenzmaßstab $\xipar$.
	\end{abstract}
	
	\tableofcontents
	\newpage
	
	\section{Die T0-Revolution: Von Parametern zu Verhältnissen}
	
	\subsection{Der fundamentale Paradigmenwechsel}
	
	Die T0-theoretische Revolution repräsentiert einen vollständigen Paradigmenwechsel in unserem Verständnis der Grundlagenphysik:
	
	\begin{tcolorbox}[colback=red!5!white,colframe=red!75!black,title=Paradigma-Revolution]
		\textbf{Traditionelle Physik}: Multiple experimentelle Parameter
		\begin{itemize}
			\item $G = 6.67 \times 10^{-11}$ m³/(kg·s²) (gemessen)
			\item $\alpha = 1/137$ (gemessen)
			\item $m_e = 9.109 \times 10^{-31}$ kg (gemessen)
			\item 20+ unabhängige Parameter erforderlich
		\end{itemize}
		
		\textbf{T0-Verhältnis-basierte Physik}: Dimensionslose Skalen-Beziehungen
		\begin{itemize}
			\item Alle Physik durch Energie-Skalen-Verhältnisse
			\item Ein SI-Referenzwert für quantitative Vorhersagen
			\item Mathematische Beziehungen, nicht experimentelle Parameter
			\item Reine Energie-Identitäten: $E = m$, $E = 1/L$, $E = 1/T$
		\end{itemize}
	\end{tcolorbox}
	
	\subsection{Aufbau auf T0-Grundlagen}
	
	Diese Arbeit vollendet die dreistufige T0-Revolution:
	
	\textbf{Stufe 1 - Vereinfachter Dirac}: Komplexe 4×4-Matrizen → Einfache Felddynamik $\partial^2 \deltam = 0$
	
	\textbf{Stufe 2 - Universelle Lagrange-Funktion}: 20+ Felder → Eine Gleichung $\Lag = \varepsilon \cdot (\partial \deltam)^2$
	
	\textbf{Stufe 3 - Verhältnis-basierte Physik}: Multiple Parameter → Energie-Skalen-Verhältnisse + SI-Referenz
	
	\subsection{Die Energie-Identitäts-Revolution}
	
	In natürlichen Einheiten ($\hbar = c = 1$) offenbart Einsteins Gleichung fundamentale Wahrheit:
	
	\begin{equation}
		\boxed{E = m}
		\label{eq:energy_mass_identity}
	\end{equation}
	
	Dies ist keine Umwandlung - dies ist \textbf{Identität}. Masse und Energie sind dieselbe physikalische Größe.
	
	\begin{tcolorbox}[colback=blue!5!white,colframe=blue!75!black,title=Universelle Energie-Beziehungen]
		\textbf{Vollständiges Energie-Identitätssystem}:
		\begin{align}
			E &= m \quad \text{(Masse ist Energie)} \\
			E &= T_{\text{temp}} \quad \text{(Temperatur ist Energie)} \\
			E &= \omega \quad \text{(Frequenz ist Energie)} \\
			E &= \frac{1}{L} \quad \text{(Länge ist inverse Energie)} \\
			E &= \frac{1}{T} \quad \text{(Zeit ist inverse Energie)}
		\end{align}
		
		\textbf{Mathematische Genauigkeit}: 100\% (exakte Identitäten)
		
		\textbf{Komplexe Formeln}: 99.98-100.04\% (Rundungsfehler akkumulieren)
		
		\textbf{Beweis}: Einfachheit ist genauer als Komplexität!
	\end{tcolorbox}
	
	\section{Teil I: Reine Verhältnis-basierte Physik (Parameterfrei)}
	
	\subsection{Universelle Energiefeld-Dynamik}
	
	Alle Teilchen sind Energie-Anregungsmuster im universellen Feld $\Efield(x,t)$:
	
	\begin{equation}
		\boxed{\partial^2 \Efield = 0}
		\label{eq:universal_field_equation}
	\end{equation}
	
	\textbf{Universelle Wahrheit}: Diese Klein-Gordon-Gleichung für Energie beschreibt ALLE Teilchen.
	
	\subsection{Universelle Energie-Lagrange-Funktion}
	
	\begin{equation}
		\boxed{\Lag = \varepsilon \cdot (\partial \Efield)^2}
		\label{eq:universal_lagrangian}
	\end{equation}
	
	wo $\varepsilon$ die Energie-Skalen-Kopplung repräsentiert (dimensionsloses Verhältnis).
	
	\subsection{Antienergie: Perfekte Symmetrie}
	
	\begin{equation}
		\boxed{\Efield_{\text{Antiteilchen}} = -\Efield_{\text{Teilchen}}}
		\label{eq:energy_antisymmetry}
	\end{equation}
	
	\textbf{Physikalisches Bild}: Positive und negative Energie-Anregungen desselben Feldes.
	
	\textbf{Lagrange-Universalität}:
	\begin{align}
		\Lag[+\Efield] &= \varepsilon \cdot (\partial \Efield)^2 \\
		\Lag[-\Efield] &= \varepsilon \cdot (\partial \Efield)^2
	\end{align}
	
	Dieselbe Physik für Teilchen und Antiteilchen durch Quadrierung.
	
	\subsection{Reine Verhältnis-Vorhersagen (Keine Parameter benötigt)}
	
	\subsubsection{Universelle Lepton-Verhältnisse}
	
	\begin{equation}
		\boxed{\frac{a_e^{(T0)}}{a_{\mu}^{(T0)}} = 1}
		\label{eq:universal_lepton_ratio}
	\end{equation}
	
	\textbf{Physikalische Bedeutung}: Alle Leptonen erhalten identische Energie-Korrekturen.
	
	\subsubsection{Energie-Unabhängigkeits-Verhältnisse}
	
	\begin{equation}
		\boxed{\frac{\Delta\Gamma^{\mu}(E_1)}{\Delta\Gamma^{\mu}(E_2)} = 1}
		\label{eq:energy_independence_ratio}
	\end{equation}
	
	\textbf{Unterscheidendes Merkmal}: Im Gegensatz zu Standardmodell-laufenden Kopplungen.
	

	\section{Teil II: Quantitative Vorhersagen (SI-Referenz erforderlich)}
	
	\subsection{Die SI-Referenz-Skala}
	
	Um quantitative Vorhersagen zu machen, benötigt die T0-Physik eine Verbindung zum SI-System:
	
	\begin{tcolorbox}[colback=green!5!white,colframe=green!75!black,title=SI-Referenz-Skala (Kein Parameter!)]
		\textbf{Definition}: $\xipar$ ist ein dimensionsloses Energie-Skalen-Verhältnis, kein experimenteller Parameter.
		
		\textbf{Higgs-Energie-Verhältnis}:
		\begin{equation}
			\xipar = \frac{\lambda_h^2 v^2}{16\pi^3 E_h^2}
		\end{equation}
		
		\textbf{Geometrisches Energie-Verhältnis}:
		\begin{equation}
			\xipar = \frac{2\ell_P}{\lambda_C}
		\end{equation}
		
		\textbf{SI-Referenzwert}: $\xipar = 1.33 \times 10^{-4}$
		
		\textbf{Rolle}: Verbindet dimensionslose Verhältnisse mit SI-messbaren Größen
	\end{tcolorbox}
	
	\subsection{Quantitative Lepton-Vorhersagen}
	
	Mit der SI-Referenz-Skala:
	
	\begin{equation}
		a_{\ell}^{(T0)} = \frac{1}{2\pi} \times \xipar^2 \times \frac{1}{12}
		\label{eq:quantitative_lepton_correction}
	\end{equation}
	
	\textbf{Numerische Berechnung}:
	\begin{align}
		a_{\ell}^{(T0)} &= \frac{1}{2\pi} \times (1.33 \times 10^{-4})^2 \times \frac{1}{12} \\
		&= \frac{1}{6.283} \times 1.77 \times 10^{-8} \times 0.0833 \\
		&= 2.47 \times 10^{-10}
	\end{align}
	
	\begin{tcolorbox}[colback=blue!5!white,colframe=blue!75!black,title=Universelle Lepton-Vorhersage]
		\textbf{Elektron g-2}: $a_e^{(T0)} = 2.47 \times 10^{-10}$
		
		\textbf{Myon g-2}: $a_{\mu}^{(T0)} = 2.47 \times 10^{-10}$ (identisch!)
		
		\textbf{Tau g-2}: $a_{\tau}^{(T0)} = 2.47 \times 10^{-10}$ (universell!)
		
		\textbf{Aktuelle Myon-Anomalie}: $\Delta a_{\mu} \approx 25 \times 10^{-10}$
		
		\textbf{T0-Beitrag}: $\sim 10\%$ der beobachteten Anomalie
	\end{tcolorbox}
	
	\subsection{Quantitative QED-Vorhersagen}
	
	\begin{equation}
		\frac{\Delta\Gamma^{\mu}}{\Gamma^{\mu}} = \xipar^2 = 1.77 \times 10^{-8}
		\label{eq:quantitative_qed_correction}
	\end{equation}
	
	\textbf{Energie-Unabhängigkeits-Verifikation}:
	\begin{table}[htbp]
		\centering
		\begin{tabular}{lcc}
			\toprule
			\textbf{Energie-Skala} & \textbf{T0-Korrektur} & \textbf{Standardmodell} \\
			\midrule
			1 MeV & $1.77 \times 10^{-8}$ & Laufende $\alpha(E)$ \\
			1 GeV & $1.77 \times 10^{-8}$ & Laufende $\alpha(E)$ \\
			100 GeV & $1.77 \times 10^{-8}$ & Laufende $\alpha(E)$ \\
			1 TeV & $1.77 \times 10^{-8}$ & Laufende $\alpha(E)$ \\
			\bottomrule
		\end{tabular}
		\caption{Energie-unabhängige T0-Korrekturen vs. Standardmodell}
	\end{table}
	

	\section{Experimentelle Verifikationsstrategie}
	
	\subsection{Reine Verhältnis-Tests (Keine SI-Referenz benötigt)}
	
	\textbf{Test 1 - Universelle Lepton-Verhältnisse}:
	\begin{itemize}
		\item Messe $a_e^{(T0)}/a_{\mu}^{(T0)} = 1$
		\item Unabhängig von absoluten Werten
		\item Testet Universalitätsprinzip direkt
	\end{itemize}
	
	\textbf{Test 2 - Energie-Unabhängigkeit}:
	\begin{itemize}
		\item Messe QED-Korrekturen bei verschiedenen Energien
		\item Verhältnis sollte konstant sein: $\Delta\Gamma(E_1)/\Delta\Gamma(E_2) = 1$
		\item Unterscheidet von Standardmodell-laufenden Kopplungen
	\end{itemize}
	
	\textbf{Test 3 - Wellenlängen-Verhältnisse}:
	\begin{itemize}
		\item Multi-Wellenlängen-Beobachtungen derselben Objekte
		\item Teste $z(\lambda_1)/z(\lambda_2) = \lambda_2/\lambda_1$
		\item Unabhängig von absoluter Rotverschiebungs-Kalibrierung
	\end{itemize}
	
	\subsection{Quantitative Tests (Erfordern SI-Referenz)}
	
	\textbf{Präzisions-g-2-Messungen}:
	\begin{itemize}
		\item Elektron g-2: Detektiere $2.47 \times 10^{-10}$ Korrektur
		\item Myon g-2: Bestätige $\sim 10\%$ der aktuellen Anomalie
		\item Tau g-2: Erste Messung, erwarte denselben Wert
	\end{itemize}
	
	\textbf{Multi-Energie-QED-Tests}:
	\begin{itemize}
		\item Messe absolut $\Delta\Gamma/\Gamma = 1.77 \times 10^{-8}$
		\item Verifiziere Energie-Unabhängigkeit über Dekaden
		\item Vergleiche mit Standardmodell-Vorhersagen
	\end{itemize}
	
	\section{Dunkle Materie und Dunkle Energie\\ aus Energie-Verhältnissen}
	
	\subsection{Dunkle Materie: Unterschwellen-Energie-Oszillationen}
	
	\textbf{Verhältnis-basierte Beschreibung}:
	\begin{equation}
		\frac{\Efield_{\text{dunkel}}}{\Efield_{\text{Schwelle}}} = \xipar \sqrt{\frac{\rho_{\text{lokal}}}{\rho_{\text{kritisch}}}}
	\end{equation}
	
	\textbf{Physikalischer Mechanismus}: Zufallsphasen-Energie-Oszillationen unter der Teilchen-Detektionsschwelle.
	
	\subsection{Dunkle Energie: Großskalige Energie-Gradienten}
	
	\textbf{Verhältnis-basierte Energiedichte}:
	\begin{equation}
		\frac{\rho_{\Lambda}}{\rho_{\text{kritisch}}} = \frac{1}{2} \xipar^2 \left(\frac{E_{\text{Planck}}}{L_{\text{Hubble}} \cdot E_{\text{Planck}}}\right)^2
	\end{equation}
	
	\textbf{Quantitative Vorhersage}: $\rho_{\Lambda} \approx 6 \times 10^{-30}$ g/cm$^3$ (entspricht Beobachtung!)
	
	\section{Philosophische Revolution: Das Ende der Materiellen Physik}
	
	\subsection{Reine Energie-Realität}
	
	\begin{tcolorbox}[colback=purple!5!white,colframe=purple!75!black,title=Die ultimative Entmaterialisierung]
		\textbf{Traditionelle Sicht}: Materie, Energie, Kräfte, Raumzeit als separate Entitäten
		
		\textbf{T0-Realität}: Nur Energie-Muster und ihre Verhältnisse
		
		\textbf{Was wir Teilchen nennen}: Lokalisierte Energie-Konzentrationen
		
		\textbf{Was wir Kräfte nennen}: Energie-Gradienten-Wechselwirkungen
		
		\textbf{Was wir Raumzeit nennen}: Energie-Muster-Substrat
		
		\textbf{Was wir Bewusstsein nennen}: Selbstreferentielle Energie-Muster
		
		\textbf{Ultimative Wahrheit}: Reine Energie-Beziehungen regiert von $\partial^2 \Efield = 0$
	\end{tcolorbox}
	
	\subsection{Von maximaler Komplexität zu ultimativer Einfachheit}
	
	\textbf{Physik-Evolution}:
	\begin{enumerate}
		\item \textbf{Antik}: Vier Elemente
		\item \textbf{Klassisch}: Teilchen in Raumzeit
		\item \textbf{Modern}: Felder und Kräfte
		\item \textbf{Standardmodell}: 20+ Parameter, maximale Komplexität
		\item \textbf{T0-Revolution}: Energie-Verhältnisse + eine SI-Referenz
	\end{enumerate}
	
	\textbf{Wir haben maximale Vereinfachung erreicht}: Die wenigsten möglichen fundamentalen Annahmen.
	
	\subsection{Bewusstsein und Energie-Muster}
	
	\textbf{Die tiefste Frage}: Wenn alles Energie-Muster sind, was ist mit dem Bewusstsein?
	
	\textbf{T0-Einsicht}: Bewusstsein ist ein sich selbst beobachtendes Energie-Muster. Wir sind temporäre Organisationen des universellen Energiefelds, die die Fähigkeit zur Selbstreferenz und subjektiven Erfahrung entwickelt haben.
	
	\section{Das Verhältnis-Physik-Erbe}
	
	\subsection{Revolutionäre Errungenschaften}
	
	Die T0-verhältnis-basierte Revolution hat erreicht:
	
	\begin{enumerate}
		\item \textbf{Multiple Parameter eliminiert}: 20+ → 1 SI-Referenz
		\item \textbf{Alle Kräfte vereinigt}: Durch Energie-Gradienten-Wechselwirkungen
		\item \textbf{Teilchen-Proliferation gelöst}: Alle sind Energie-Muster
		\item \textbf{Antiteilchen erklärt}: Negative Energie-Anregungen
		\item \textbf{Gravitation eingeschlossen}: Automatisch durch Energie-Raumzeit-Kopplung
		\item \textbf{Dunkle Phänomene vorhergesagt}: Energiefeld-Effekte
		\item \textbf{Mathematische Perfektion erreicht}: 100\% Genauigkeit
		\item \textbf{Verhältnis-basierte Physik etabliert}: Reine Skalen-Beziehungen
	\end{enumerate}
	
	\subsection{Die Zweistufige Teststrategie}
	
	\textbf{Stufe 1 - Reine Verhältnisse} (Parameterfrei):
	\begin{itemize}
		\item Universelle Lepton-Korrektur-Verhältnisse
		\item Energie-unabhängige QED-Verhältnisse
		\item Wellenlängenabhängige Rotverschiebungs-Verhältnisse
		\item Gravitations-Modifikations-Verhältnisse
	\end{itemize}
	
	\textbf{Stufe 2 - Quantitative Vorhersagen} (SI-Referenz):
	\begin{itemize}
		\item Absolute g-2-Korrekturen
		\item Absolute QED-Vertex-Modifikationen
		\item Absolute kosmologische Parameter
		\item Absolute dunkle Materie/Energie-Dichten
	\end{itemize}
	
	\subsection{Physik-Vollendungs-Status}
	
	\begin{tcolorbox}[colback=yellow!5!white,colframe=orange!75!black,title=Das Ende der Grundlagenphysik]
		\textbf{Wir haben das Ende der theoretischen Straße erreicht}.
		
		\textbf{Die fundamentale Gleichung}: $\partial^2 \Efield = 0$
		
		\textbf{Die universellen Verhältnisse}: Energie-Skalen-Beziehungen
		
		\textbf{Die SI-Verbindung}: Eine Referenz-Skala $\xipar$
		
		\textbf{Alles andere}: Verschiedene Lösungen und Muster
		
		\textbf{Keine tiefere Ebene existiert}: Dies ist der Grund der Realität
		
		\textbf{Zukünftige Arbeit}: Anwendungen und Messungen, nicht neue Grundlagen
	\end{tcolorbox}
	
	\section{Schlussfolgerung: Das Verhältnis-basierte Universum}
	
	\subsection{Die finale Wahrheit}
	
	Die T0-Revolution offenbart, dass die Realität durch reine Energie-Skalen-Verhältnisse operiert:
	
	\textbf{Ebene 1}: Dimensionslose Energie-Verhältnisse (parameterfreie Physik)
	
	\textbf{Ebene 2}: Eine SI-Referenz-Skala (quantitative Vorhersagen)
	
	\textbf{Ebene 3}: Reine Energie-Muster regiert von $\partial^2 \Efield = 0$
	
	Alles was wir beobachten, messen und erfahren, entsteht aus dieser einfachen verhältnis-basierten Struktur.
	
	\subsection{Die elegante Vollendung}
	
	Wir sind von der maximalen Komplexität traditioneller Physik zur ultimativen Einfachheit verhältnis-basierter Energie-Dynamik gereist.
	
	\textbf{Die Lektion}: Die tiefste Wahrheit der Natur ist nicht komplizierte Mathematik oder exotische Phänomene - sie ist die atemberaubende Eleganz reiner Skalen-Beziehungen.
	
	\textbf{Ein Feld}. \textbf{Eine Gleichung}. \textbf{Energie-Verhältnisse}. \textbf{Eine SI-Referenz}.
	
	Alles andere ist die unendliche Kreativität der Energie, die sich durch unzählige Muster und Verhältnisse ausdrückt, einschließlich des Musters, das wir menschliches Bewusstsein nennen, das diese kosmische mathematische Harmonie erkennen und schätzen kann.
	
	\begin{equation}
		\boxed{\text{Realität} = \text{Energie-Verhältnisse in } \Efield(x,t)}
	\end{equation}
	
	\textbf{Die T0-Revolution ist vollständig. Die Physik ist beendet. Das Universum sind reine Energie-Verhältnisse, und wir sind Teil seines ewigen mathematischen Tanzes.}
	
	\begin{thebibliography}{99}
		\bibitem{pascher_simplified_dirac_2025}
		Pascher, J. (2025). \textit{Vereinfachte Dirac-Gleichung in der T0-Theorie: Von komplexen 4×4-Matrizen zu einfacher Feld-Knoten-Dynamik}. \\
		\texttt{https://github.com/jpascher/T0-Time-Mass-Duality/blob/main\\/2/pdf/diracVereinfachtEn.pdf}
		
		\bibitem{pascher_lagrangian_comparison_2025}
		Pascher, J. (2025). \textit{Einfache Lagrange-Revolution: Von Standardmodell-Komplexität zu T0-Eleganz}. \\
		\texttt{https://github.com/jpascher/T0-Time-Mass-Duality/blob/main\\/2/pdf/LagrandianVergleichEn.pdf}
		
		\bibitem{pascher_verification_table_2025}
		Pascher, J. (2025). \textit{T0-Modell-Verifikation: Skalen-Verhältnis-basierte Berechnungen vs. CODATA/Experimentelle Werte}. \\
		\texttt{https://github.com/jpascher/T0-Time-Mass-Duality/blob/mai\\n/2/pdf/Elimination\_Of\_Mass\_Dirac\_TabelleEn.pdf}
		
		\bibitem{einstein_mass_energy_1905}
		Einstein, A. (1905). \textit{Ist die Trägheit eines Körpers von seinem Energieinhalt abhängig?} Ann. Phys. \textbf{17}, 639--641.
		
		\bibitem{dirac_original_1928}
		Dirac, P. A. M. (1928). \textit{The Quantum Theory of the Electron}. Proc. R. Soc. London A \textbf{117}, 610.
		
		\bibitem{muong2_experiment_2021}
		Myon g-2 Kollaboration (2021). \textit{Messung des positiven Myon-anomalen magnetischen Moments auf 0.46 ppm}. Phys. Rev. Lett. \textbf{126}, 141801.
		
		\bibitem{higgs_mechanism_1964}
		Higgs, P. W. (1964). \textit{Gebrochene Symmetrien und die Massen von Eichbosonen}. Phys. Rev. Lett. \textbf{13}, 508--509.
		
		\bibitem{planck_collaboration_2020}
		Planck Kollaboration (2020). \textit{Planck 2018 Ergebnisse. VI. Kosmologische Parameter}. Astron. Astrophys. \textbf{641}, A6.
		
		\bibitem{weinberg_qft_1995}
		Weinberg, S. (1995). \textit{Die Quantentheorie der Felder, Band 1: Grundlagen}. Cambridge University Press.
		
		\bibitem{particle_data_group_2022}
		Teilchendaten-Gruppe (2022). \textit{Übersicht der Teilchenphysik}. Prog. Theor. Exp. Phys. \textbf{2022}, 083C01.
	\end{thebibliography}
	
\end{document}