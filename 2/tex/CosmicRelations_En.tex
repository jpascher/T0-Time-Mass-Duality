\documentclass[12pt,a4paper]{article}
\usepackage[utf8]{inputenc}
\usepackage[T1]{fontenc}
\usepackage[english]{babel}
\usepackage[left=2cm,right=2cm,top=2cm,bottom=2cm]{geometry}
\usepackage{lmodern}
\usepackage{amsmath}
\usepackage{amssymb}
\usepackage{physics}
\usepackage{graphicx}
\usepackage{hyperref}
\usepackage{tcolorbox}
\usepackage{booktabs}
\usepackage{enumitem}
\usepackage[table,xcdraw]{xcolor}
\usepackage{longtable}
\usepackage{siunitx}
\usepackage{fancyhdr}
\usepackage{textgreek}

% Header and Footer
\pagestyle{fancy}
\fancyhf{}
\fancyhead[L]{T0-Theory: Cosmic Relations}
\fancyhead[R]{\thepage}
\fancyfoot[C]{\textit{From the universal $\xi$-constant to cosmic structures}}
\renewcommand{\headrulewidth}{0.4pt}
\renewcommand{\footrulewidth}{0.4pt}

\hypersetup{
	colorlinks=true,
	linkcolor=blue,
	citecolor=blue,
	urlcolor=blue,
	pdftitle={T0-Theory: Cosmic Relations and universal $\xi$-constant},
	pdfauthor={T0-Theory Project},
	pdfsubject={Cosmology, $\xi$-field, Gravitation, CMB, Casimir Effect}
}

% Custom environments
\newtcolorbox{important}[1][]{colback=yellow!10!white,colframe=yellow!50!black,fonttitle=\bfseries,title=Important Note,#1}
\newtcolorbox{formula}[1][]{colback=blue!5!white,colframe=blue!75!black,fonttitle=\bfseries,title=Key Formula,#1}
\newtcolorbox{revolutionary}[1][]{colback=red!5!white,colframe=red!75!black,fonttitle=\bfseries,title=Insight,#1}
\newtcolorbox{experiment}[1][]{colback=green!5!white,colframe=green!75!black,fonttitle=\bfseries,title=Experimental Test,#1}
\newtcolorbox{sibox}[1][]{colback=orange!10!white,colframe=orange!75!black,fonttitle=\bfseries,title=SI Units (for reference only),#1}

\title{\Huge\textbf{T0-Theory: Cosmic Relations}\\
	\Large The universal $\xi$-constant as key \\ to gravitation, CMB and cosmic structures}
\author{\Large Johann Pascher\\
	Department of Communications Engineering,\\
	Higher Technical Federal Institute (HTL), Leonding, Austria\\
	\texttt{johann.pascher@gmail.com}}
\date{\today}

\numberwithin{equation}{section}

\begin{document}
	
	\maketitle
	\thispagestyle{fancy}
	
	\tableofcontents
	
	\section{Introduction to T0-Theory}
	
	T0-Theory presents a novel framework connecting quantum phenomena with cosmological structures through a universal dimensionless constant $\xi$. This theory establishes fundamental relationships between microscopic quantum scales and macroscopic cosmic dimensions, offering a unified perspective on physics from the quantum realm to the cosmological horizon.
	
	\section{Microscopic Length $L_0$ in T0-Theory}
	
	\subsection{Derivation of the Microscopic Length in Natural Units ($\hbar = c = 1$)}
	
	\begin{table}[h!]
		\centering
		\begin{tabular}{ccc}
			\toprule
			\textbf{Quantity} & \textbf{Dimension} & \textbf{Relation} \\
			\midrule
			Energy $E_0$ & [E] = GeV & $E_0 = 1/\xi$ \\
			Mass $m_0$ & [m] = GeV & $m_0 = E_0$ \\
			Length $L_0$ & [L] = GeV$^{-1}$ & $L_0 = 1/E_0 = \xi$ \\
			\bottomrule
		\end{tabular}
		\caption{Characteristic microscopic quantities in natural units.}
	\end{table}
	
	\[
	\xi = \frac{4}{3} \times 10^{-4} \quad \Rightarrow \quad E_0 = 1/\xi = 7500 \,\text{GeV} \quad \Rightarrow \quad L_0 = \xi
	\]
	
	\subsection{Conversion to Physical Units}
	
	\[
	1 \,\text{GeV}^{-1} = \hbar c = 1.973 \times 10^{-16}\,\text{m}
	\]
	
	\[
	L_0 = \xi \cdot \hbar c = \frac{4}{3} \times 10^{-4} \cdot 1.973 \times 10^{-16}\,\text{m} \approx 2.63 \times 10^{-20}\,\text{m}
	\]
	
	\subsection{Physical Significance}
	
	\begin{itemize}
		\item $L_0$ represents the fundamental microscopic length scale in T0-Theory
		\item It serves as the basis for all other length scales in the theory
		\item Originates from the geometric structure of 3D space and $\xi$-field physics
	\end{itemize}
	
\begin{important}
	Yes, T0-Theory postulates a minimal length $L_0 \approx 2.63 \times 10^{-20}$ m that cannot be exceeded. This minimal length emerges naturally from the Lagrangian density and the maximum field fluctuation, without any arbitrary parameters.
\end{important}
	
	\section{Characteristic Vacuum Length $L_\xi$ and CMB Connection}
	
	\subsection{Fundamental Relationship in T0-Theory}
	
	T0-Theory postulates a fundamental relationship between basic constants:
	
	\begin{formula}
		\[
		\hbar c = \xi \rho_{\text{CMB}} L_\xi^4
		\]
	\end{formula}
	
	This equation connects quantum mechanics ($\hbar c$) with the cosmic microwave background radiation ($\rho_{\text{CMB}}$) through the dimensionless constant $\xi$ and the characteristic vacuum length $L_\xi$.
	
	\subsection{Derivation of the Characteristic Vacuum Length $L_\xi$}
	
	From the fundamental relationship follows:
	
	\[
	L_\xi = \left(\frac{\hbar c}{\xi \rho_{\text{CMB}}}\right)^{1/4}
	\]
	
	\subsubsection{CMB Energy Density}
	
	\[
	T_{\text{CMB}} \approx 2.725\,\text{K} \quad \Rightarrow \quad \rho_{\text{CMB}} = \frac{\pi^2}{15} \frac{(k_B T_{\text{CMB}})^4}{(\hbar c)^3} \approx 4.17 \times 10^{-14}\, \text{J/m}^3
	\]
	
	\subsubsection{Numerical Calculation}
	
	Using the values:
	\begin{itemize}
		\item $\hbar c = 3.16 \times 10^{-26}$ J·m
		\item $\xi = 4/3 \times 10^{-4}$
		\item $\rho_{\text{CMB}} = 4.17 \times 10^{-14}$ J/m³
	\end{itemize}
	
	we obtain:
	
	\[
	L_\xi = \left(\frac{3.16 \times 10^{-26}}{(4/3) \times 10^{-4} \times 4.17 \times 10^{-14}}\right)^{1/4} \approx 1.0 \times 10^{-4}\,\text{m}
	\]
	
	\subsection{Numerical Verification of the Fundamental Relationship}
	
	Back-calculation for verification:
	\[
	\xi \rho_{\text{CMB}} L_\xi^4 = \frac{4}{3} \times 10^{-4} \times 4.17 \times 10^{-14} \times (10^{-4})^4 = 3.13 \times 10^{-26}\,\text{J·m}
	\]
	
	Compared with $\hbar c = 3.16 \times 10^{-26}$ J·m, this shows a deviation of less than 1\%.
	
	\section{Cosmic Length $R_0$ and Scale Hierarchy}
	
	\subsection{Definition of $R_0$}
	
	The cosmic length $R_0$ is theoretically derived through the hierarchy between $L_0$ and the Planck length $L_P$:
	
	\[
	R_0 \sim \frac{L_P^2}{L_0} \sim 10^{26}\,\text{m}
	\]
	
	It can be numerically compared with the Hubble length:
	\[
	L_H = c / H_0 \sim 10^{26}\,\text{m}
	\]
	
	\subsection{Connection between $L_\xi$ and $R_0$ via $\xi$}
	
	T0-Theory postulates a hierarchy:
	
	\[
	\frac{R_0}{L_\xi} \sim \xi^{-N} \quad \Rightarrow \quad R_0 \sim L_\xi \, \xi^{-N}
	\]
	
	With $N \approx 30$ and $L_\xi \sim 10^{-4}$ m, we obtain:
	
	\[
	R_0 \sim 10^{-4} \times (10^4)^{30/4} = 10^{-4} \times 10^{30} = 10^{26}\,\text{m}
	\]
	
	This directly connects the characteristic vacuum length $L_\xi$ with the cosmic length $R_0$.
	
	\section{Derivation via Lagrangian Density and Planck Length}
	
	The microscopic length $L_0$ can be derived from the T0 Lagrangian density. The T0 Lagrangian function contains a term describing the vacuum field:
	
	\[
	\mathcal{L}_{\xi} \sim \frac{1}{2} (\partial_\mu \phi_\xi)^2 - \frac{1}{2} \frac{\phi_\xi^2}{L_0^2}
	\]
	
	Energy minimization yields:
	
	\[
	\phi_\xi \sim L_0^{-1} \quad \Rightarrow \quad L_0 = \xi \sim 10^{-20}\,\text{m (in SI units)}
	\]
	
	The cosmic length results from the Planck length $L_P$ and $L_0$:
	
	\[
	R_0 \sim \frac{L_P^2}{L_0} \sim \frac{(1.616 \times 10^{-35}\,\text{m})^2}{2.6 \times 10^{-20}\,\text{m}} \sim 1.0 \times 10^{25}\,\text{m}
	\]
	
	\section{Percentage Deviation from Hubble Length}
	
	The calculated cosmic length $R_0$ deviates from the Hubble length $L_H$ as follows:
	
	\[
	\Delta_{\%} = \frac{L_H - R_0}{L_H} \times 100\% \approx 4\%
	\]
	
	\section{Remarkable Connection with $\xi$}
	
	\begin{itemize}
		\item The dimensionless constant $\xi \sim 4/3 \times 10^{-4}$ appears in multiple physical contexts
		\item $L_\xi \sim 10^{-4}$ m is consistently derived from $\rho_{\text{CMB}}$ and the fundamental relationship
		\item Casimir effects confirm the characteristic vacuum length $L_\xi$
		\item Small powers of $\xi$ determine average values of observed cosmic parameters and create a hierarchical, self-similar pattern
		\item The hierarchy $R_0 / L_\xi \sim \xi^{-30}$ connects vacuum and cosmic scales
	\end{itemize}
	
	\section{Summary}
	
	\begin{itemize}
		\item The microscopic length $L_0 = \xi \approx 2.63 \times 10^{-20}\,\text{m}$ is fundamental in T0-Theory
		\item The characteristic vacuum length $L_\xi \sim 10^{-4}\,\text{m}$ emerges consistently from CMB energy density via the fundamental relationship $\hbar c = \xi \rho_{\text{CMB}} L_\xi^4$
		\item The cosmic length $R_0 \sim 10^{26}\,\text{m}$ results from powers of $\xi$ and agrees within approximately 4\% with the Hubble length
		\item $\xi$ connects microscopic and cosmological scales and appears repeatedly as a "fingerprint" in physical quantities
		\item Casimir experiments and CMB temperature confirm the consistency of the characteristic vacuum length $L_\xi$
		\item Derivation via Lagrangian density and Planck length shows theoretical consistency of the scale hierarchy
	\end{itemize}
	
	\section{Derivation of Minimal Length from the Lagrangian}
	
	Starting from the T0 theory Lagrangian:
	
	\begin{equation}
		\mathcal{L} = \varepsilon (\partial \delta m)^2, \quad \delta m(x,t) = m(x,t) - m_0
	\end{equation}
	
	where $\delta m$ is the fluctuation of the mass field around a reference mass $m_0$ and $\varepsilon$ is a scaling constant.
	
	\subsection{Euler-Lagrange Equation}
	
	The Euler-Lagrange equation for the mass fluctuation $\delta m$ is
	
	\begin{equation}
		\partial_\mu \frac{\partial \mathcal{L}}{\partial (\partial_\mu \delta m)} - \frac{\partial \mathcal{L}}{\partial \delta m} = 0
	\end{equation}
	
	Since $\mathcal{L} \sim (\partial \delta m)^2$, we have $\frac{\partial \mathcal{L}}{\partial \delta m} = 0$ and
	
	\begin{equation}
		\frac{\partial \mathcal{L}}{\partial (\partial_\mu \delta m)} = 2 \varepsilon \partial_\mu \delta m
	\end{equation}
	
	leading to the classical wave equation:
	
	\begin{equation}
		\partial_\mu \partial^\mu \delta m = 0
	\end{equation}
	
	\subsection{Discrete Structure and Minimal Length}
	
	Considering plane-wave solutions
	
	\begin{equation}
		\delta m(x) \sim e^{i k \cdot x}, \quad k = |k|
	\end{equation}
	
	the field energy scales as
	
	\begin{equation}
		E_k \sim \varepsilon k^2 |\delta m_k|^2
	\end{equation}
	
	so that high frequencies (short wavelengths) are energetically suppressed.  
	
	Imposing a maximal allowed field fluctuation $\delta m_{\mathrm{max}}$ naturally defines a characteristic maximal mass
	
	\begin{equation}
		m_{\mathrm{max}} \sim m_0 + \delta m_{\mathrm{max}}
	\end{equation}
	
	\subsection{Minimal Time and Length via Duality}
	
	Using the fundamental T0-theory duality
	
	\begin{equation}
		T \cdot m = 1 \quad \Rightarrow \quad T_{\mathrm{min}} = \frac{1}{m_{\mathrm{max}}}
	\end{equation}
	
	and in natural units ($c = 1$), this translates directly to a minimal length
	
	\begin{equation}
		r_0 \sim T_{\mathrm{min}} \sim \frac{1}{m_{\mathrm{max}}} \sim \frac{1}{m_0 + \delta m_{\mathrm{max}}}
	\end{equation}
	
	\subsection{Scaling with the Universal Constant $\xi$}
	
	Incorporating the universal scaling constant $\xi \ll 1$ of the T0 theory, the minimal length becomes
	
	\begin{equation}
		r_0 \sim \xi \ell_P \ll \ell_P
	\end{equation}
	
	Thus, the minimal length $r_0$ emerges naturally from the Lagrangian, the maximal field fluctuation, and the intrinsic mass-time duality, without any arbitrary parameters.
\begin{revolutionary}
	T0-Theory predicts a minimal length of $r_0 \sim \xi \ell_P \approx 2.63 \times 10^{-20}$ m that cannot be exceeded. This emerges naturally from the Lagrangian density and the fundamental mass-time duality of the theory.
\end{revolutionary}
	\section*{Characteristic Vacuum Length $L_\xi$ Scale Verification}
	
	\begin{important}
		The characteristic vacuum length $L_\xi$ is indeed approximately 0.1 mm:
		\[
		L_\xi \approx 1.0 \times 10^{-4}\,\text{m} = 0.1\,\text{mm}
		\]
		This length scale is consistently derived from the fundamental relationship of T0-Theory:
		\[
		\hbar c = \xi \rho_{\text{CMB}} L_\xi^4
		\]
		with $\xi = \frac{4}{3} \times 10^{-4}$ and the CMB energy density $\rho_{\text{CMB}} \approx 4.17 \times 10^{-14}\,\text{J/m}^3$.
	\end{important}
	
	\subsection*{Numerical Verification}
	
	\begin{align*}
		L_\xi &= \left(\frac{\hbar c}{\xi \rho_{\text{CMB}}}\right)^{1/4} \\
		&= \left(\frac{3.16 \times 10^{-26}\,\text{J·m}}{\frac{4}{3} \times 10^{-4} \times 4.17 \times 10^{-14}\,\text{J/m}^3}\right)^{1/4} \\
		&\approx \left(\frac{3.16 \times 10^{-26}}{5.56 \times 10^{-18}}\right)^{1/4} \\
		&\approx \left(5.68 \times 10^{-9}\right)^{1/4} \\
		&\approx 1.0 \times 10^{-4}\,\text{m} = 0.1\,\text{mm}
	\end{align*}
	
	\subsection*{Physical Significance}
	
	The length scale of 0.1 mm is particularly significant because it:
	\begin{itemize}
		\item Lies within the observable range of Casimir effects
		\item Represents a natural boundary between microscopic and macroscopic phenomena
		\item Is directly linked to CMB radiation
		\item Mediates the hierarchy between quantum and cosmic scales
	\end{itemize}
\section*{Appendix: Notation and Symbol Explanations}

\subsection*{Symbols and Notation Used in T0-Theory}

\begin{longtable}{p{2cm} p{12cm}}
	\toprule
	\textbf{Symbol} & \textbf{Description} \\
	\midrule
	\endhead
	
	$\xi$ & Universal dimensionless constant, fundamental parameter of T0-Theory: $\xi = \frac{4}{3} \times 10^{-4}$ \\
	$L_0$ & Minimal length scale, fundamental microscopic length: $L_0 \approx 2.63 \times 10^{-20}$ m \\
	$E_0$ & Characteristic energy scale: $E_0 = 1/\xi = 7500$ GeV \\
	$m_0$ & Reference mass scale: $m_0 = E_0$ (in natural units) \\
	$L_\xi$ & Characteristic vacuum length scale: $L_\xi \approx 1.0 \times 10^{-4}$ m \\
	$\rho_{\text{CMB}}$ & Energy density of Cosmic Microwave Background radiation \\
	$T_{\text{CMB}}$ & Temperature of Cosmic Microwave Background: $T_{\text{CMB}} \approx 2.725$ K \\
	$R_0$ & Cosmic length scale: $R_0 \sim 10^{26}$ m \\
	$L_P$ & Planck length: $L_P \approx 1.616 \times 10^{-35}$ m \\
	$L_H$ & Hubble length: $L_H = c/H_0 \sim 10^{26}$ m \\
	$\hbar$ & Reduced Planck constant: $\hbar = h/2\pi$ \\
	$c$ & Speed of light in vacuum \\
	$k_B$ & Boltzmann constant \\
	$\mathcal{L}$ & Lagrangian density \\
	$\mathcal{L}_{\xi}$ & $\xi$-field component of Lagrangian density \\
	$\phi_\xi$ & $\xi$-field scalar field \\
	$\delta m$ & Mass fluctuation field: $\delta m(x,t) = m(x,t) - m_0$ \\
	$\varepsilon$ & The scaling constant corresponds to the fine-structure constant $\alpha$: \\
	$\partial_\mu$ & Partial derivative (4-gradient in spacetime) \\
	$\ell_P$ & Alternative notation for Planck length \\
	$r_0$ & Alternative notation for minimal length scale \\
	$T_{\text{min}}$ & Minimal time scale derived from mass-time duality \\
	$m_{\text{max}}$ & Maximum mass scale from field fluctuations \\
	$N$ & Scaling exponent in hierarchy relation: $N \approx 30$ \\
	$\Delta_{\%}$ & Percentage deviation between theoretical and observed values \\
	\bottomrule
\end{longtable}

\subsection*{Mathematical Notation}

\begin{longtable}{p{2cm} p{12cm}}
	\toprule
	\textbf{Notation} & \textbf{Meaning} \\
	\midrule
	\endhead
	
	$\sim$ & Proportional to or approximately equal \\
	$\approx$ & Approximately equal \\
	$\equiv$ & Defined as \\
	$:=$ & Definition equality \\
	$\partial_\mu$ & Partial derivative with respect to coordinate $x^\mu$ \\
	$\partial^\mu$ & Contravariant partial derivative \\
	$\partial_\mu \partial^\mu$ & d'Alembert operator (wave operator) \\
	$[\text{E}]$ & Dimension of energy (natural units) \\
	$[\text{L}]$ & Dimension of length (natural units) \\
	$[\text{m}]$ & Dimension of mass (natural units) \\
	$\text{GeV}$ & Giga-electronvolt, unit of energy: $1$ GeV $= 10^9$ eV \\
	$\text{GeV}^{-1}$ & Inverse GeV, unit of length in natural units \\
	$\text{J/m}^3$ & Joules per cubic meter, unit of energy density \\
	$\text{K}$ & Kelvin, unit of temperature \\
	\bottomrule
\end{longtable}

\subsection*{Special Constants and Values}

\begin{longtable}{p{4cm} p{10cm}}
	\toprule
	\textbf{Constant/Value} & \textbf{Description} \\
	\midrule
	\endhead
	
	$\xi = \frac{4}{3} \times 10^{-4}$ & Fundamental dimensionless constant of T0-Theory \\
	$L_0 \approx 2.63 \times 10^{-20}$ m & Minimal length scale derived from $\xi$ \\
	$E_0 = 7500$ GeV & Characteristic energy scale \\
	$L_\xi \approx 0.1$ mm & Characteristic vacuum length scale \\
	$R_0 \sim 10^{26}$ m & Cosmic scale comparable to Hubble length \\
	$4\%$ deviation & Difference between $R_0$ and Hubble length $L_H$ \\
	$\hbar c = 3.16 \times 10^{-26}$ J·m & Product of reduced Planck constant and speed of light \\
	$\rho_{\text{CMB}} \approx 4.17 \times 10^{-14}$ J/m³ & CMB energy density \\
	$T_{\text{CMB}} = 2.725$ K & Measured CMB temperature \\
	$1$ GeV$^{-1} = 1.973 \times 10^{-16}$ m & Conversion factor between natural and SI units \\
	\bottomrule
\end{longtable}
	
\end{document}