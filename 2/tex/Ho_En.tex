\documentclass[12pt,a4paper]{article}
\usepackage[utf8]{inputenc}
\usepackage[T1]{fontenc}
\usepackage[english]{babel}
\usepackage[left=2cm,right=2cm,top=2cm,bottom=2cm]{geometry}
\usepackage{lmodern}
\usepackage{amsmath}
\usepackage{amssymb}
\usepackage{physics}
\usepackage{hyperref}
\usepackage{tcolorbox}
\usepackage{booktabs}
\usepackage{enumitem}
\usepackage[table,xcdraw]{xcolor}
\usepackage{pgfplots}
\pgfplotsset{compat=1.18}
\usepackage{graphicx}
\usepackage{float}
\usepackage{mathtools}
\usepackage{amsthm}
\usepackage{cleveref}
\usepackage{siunitx}
\usepackage{fancyhdr}
\usepackage{tocloft}

% Headers and Footers
\pagestyle{fancy}
\fancyhf{}
\fancyhead[L]{Johann Pascher}
\fancyhead[R]{$H_0$ and $\kappa$ Parameter Analysis in T0 Model}
\fancyfoot[C]{\thepage}
\renewcommand{\headrulewidth}{0.4pt}
\renewcommand{\footrulewidth}{0.4pt}

% Table of Contents Styling
\renewcommand{\cftsecfont}{\color{blue}}
\renewcommand{\cftsubsecfont}{\color{blue}}
\renewcommand{\cftsecpagefont}{\color{blue}}
\renewcommand{\cftsubsecpagefont}{\color{blue}}
\setlength{\cftsecindent}{1cm}
\setlength{\cftsubsecindent}{2cm}

\hypersetup{
	colorlinks=true,
	linkcolor=blue,
	citecolor=blue,
	urlcolor=blue,
	pdftitle={Derivation and Analysis of H₀ and κ Parameters in T0 Model},
	pdfauthor={Johann Pascher},
	pdfsubject={T0 Model, Hubble Constant, Field Theory},
	pdfkeywords={T0 Model, Hubble Constant, Kappa Parameter, Natural Units, Cosmological Parameters}
}

% Custom Commands
\newcommand{\Tfield}{T(x,t)}
\newcommand{\betaT}{\beta_{\text{T}}}
\newcommand{\alphaEM}{\alpha_{\text{EM}}}
\newcommand{\Tzero}{T_0}
\newcommand{\vecx}{\vec{x}}
\newcommand{\lP}{\ell_{\text{P}}}
\newcommand{\xipar}{\xi}
\newcommand{\LambdaT}{\Lambda_{\text{T}}}
\newcommand{\Hzero}{H_0}
\newcommand{\kappaparam}{\kappa}

\newtheorem{theorem}{Theorem}[section]
\newtheorem{proposition}[theorem]{Proposition}
\newtheorem{definition}[theorem]{Definition}

\begin{document}
	
	\title{Derivation and Comprehensive Analysis of $H_0$ and $\kappa$ Parameters \\
		in the T0 Model Framework: \\
		From Field Theory to Cosmological Scale Relations}
	\author{Johann Pascher\\
		Department of Communications Engineering, \\H{\"o}here Technische Bundeslehranstalt (HTL), Leonding, Austria\\
		\texttt{johann.pascher@gmail.com}}
	\date{\today}
	
	\maketitle
	
	\begin{abstract}
		This comprehensive document presents the complete derivation and analysis of the Hubble parameter $H_0$ and the linear potential parameter $\kappa$ within the T0 model framework. We demonstrate that $H_0$ is not an external empirical parameter but emerges naturally from the field-theoretic structure through the energy loss mechanism and cosmic regime transitions. The key relationship $\kappa = H_0$ in the infinite geometry limit provides the fundamental connection between microscopic field parameters and macroscopic cosmological scales. Through detailed dimensional analysis and scale hierarchy examination, we establish the relationship between T0 characteristic lengths, Planck scales, and cosmological scales, revealing a unified framework spanning 61 orders of magnitude from quantum gravity to cosmic horizons. All derivations maintain strict dimensional consistency in natural units ($\hbar = c = \alpha_{\text{EM}} = \beta_{\text{T}} = 1$).
	\end{abstract}
	
	\tableofcontents
	\newpage
	
	\section{Introduction: The Challenge of Scale Unification}
	\label{sec:introduction}
	
	One of the fundamental challenges in theoretical physics is connecting the microscopic world of quantum field theory with the macroscopic realm of cosmology. The T0 model addresses this challenge by proposing that the Hubble parameter $H_0$, traditionally viewed as an empirical cosmological constant, actually emerges from the underlying field-theoretic structure through the intrinsic time field dynamics.
	
	\subsection{Historical Context}
	\label{subsec:historical_context}
	
	In standard cosmology, the Hubble constant $H_0 \approx 70$ km/s/Mpc appears as an empirical parameter that characterizes the expansion rate of the universe. Various independent measurements yield slightly different values, leading to the so-called ``Hubble tension.'' The T0 model offers a fundamentally different perspective: $H_0$ emerges as a characteristic transition scale between local and cosmic field regimes.
	
	\subsection{T0 Model Foundation}
	\label{subsec:t0_foundation}
	
	The T0 model is based on the fundamental time field $\Tfield$ satisfying:
	\begin{equation}
		\nabla^2 m(x,t) = 4\pi G \rho(x,t) \cdot m(x,t)
	\end{equation}
	
	where the time field is defined as:
	\begin{equation}
		\Tfield = \frac{1}{\max(m(x,t), \omega)}
	\end{equation}
	
	\textbf{Dimensional verification in natural units} ($\hbar = c = 1$):
	\begin{itemize}
		\item $[\Tfield] = [E^{-1}]$ (time field dimension)
		\item $[m] = [E]$ (mass-energy dimension)
		\item $[\omega] = [E]$ (frequency-energy dimension)
	\end{itemize}
	
	\textbf{Complete T0 model documentation available at:} \\
	\url{https://github.com/jpascher/T0-Time-Mass-Duality/blob/main/2/pdf/}
	
	\section{The $\kappa$ Parameter: From Energy Loss to Linear Potential}
	\label{sec:kappa_derivation}
	
	\subsection{Energy Loss Mechanism Foundation}
	\label{subsec:energy_loss_foundation}
	
	The $\kappa$ parameter emerges from the fundamental energy loss mechanism in the T0 model. When photons propagate through time field gradients, they lose energy according to:
	
	\begin{equation}
		\frac{dE}{dr} = -g_T \omega^2 \frac{2G}{r^2}
		\label{eq:energy_loss_rate}
	\end{equation}
	
	\textbf{Dimensional consistency check}:
	\begin{itemize}
		\item $[dE/dr] = [E]/[E^{-1}] = [E^2]$
		\item $[g_T \omega^2 2G/r^2] = [1][E^2][E^{-2}]/[E^{-2}] = [E^2]$ \checkmark
	\end{itemize}
	
	where $g_T = \alphaEM = 1$ in natural units.
	
	\subsection{Integration Over Cosmological Distances}
	\label{subsec:cosmological_integration}
	
	For propagation over large distances $r$, integration of \cref{eq:energy_loss_rate} yields:
	\begin{equation}
		\Delta E = \int_{\infty}^{r} g_T \omega^2 \frac{2G}{r'^2} dr' = g_T \omega^2 \frac{2G}{r}
	\end{equation}
	
	The redshift becomes:
	\begin{equation}
		z = \frac{\Delta E}{E} = \frac{\Delta E}{\omega} = g_T \omega \frac{2G}{r}
		\label{eq:redshift_basic}
	\end{equation}
	
	\subsection{Definition of $\kappa$ Parameter}
	\label{subsec:kappa_definition}
	
	For the modified gravitational potential:
	\begin{equation}
		\Phi(r) = -\frac{GM}{r} + \kappaparam r
		\label{eq:modified_potential}
	\end{equation}
	
	The $\kappa$ parameter is defined through the energy loss mechanism:
	\begin{equation}
		\kappaparam = g_T \frac{2G}{r_{\text{char}}}
		\label{eq:kappa_definition}
	\end{equation}
	
	where $r_{\text{char}}$ is a characteristic distance scale.
	
	\textbf{Dimensional verification}:
	\begin{itemize}
		\item $[\kappaparam] = [g_T 2G/r] = [1][E^{-2}][E] = [E^{-1}]$ in space
		\item But $[\kappaparam r] = [E^{-1}][E^{-1}] = [E^{-2}]$ (potential energy per unit mass)
		\item This requires $[\kappaparam] = [E^2]$ for dimensional consistency with $[GM/r] = [E^2]$
	\end{itemize}
	
	\textbf{Corrected form}: In natural units, $\kappa$ has dimension $[E^2]$:
	\begin{equation}
		\kappaparam = g_T \omega \frac{2G}{r} \quad \text{with } [\kappaparam] = [E^2]
	\end{equation}
	
	\section{Regime Classification and Cosmic Screening}
	\label{sec:regime_classification}
	
	\subsection{Three Fundamental Field Geometries}
	\label{subsec:three_geometries}
	
	The T0 model requires different treatments for distinct geometric configurations:
	
	\begin{enumerate}
		\item \textbf{Localized spherical fields}: Finite, spherically symmetric mass distributions
		\item \textbf{Localized non-spherical fields}: Finite, asymmetric mass distributions  
		\item \textbf{Infinite homogeneous fields}: Uniform cosmic background
	\end{enumerate}
	
	\subsection{Infinite Fields and the $\LambdaT$ Term}
	\label{subsec:lambda_term}
	
	For infinite, homogeneous matter distributions with $\rho(x) = \rho_0 = \text{constant}$, the standard field equation:
	\begin{equation}
		\nabla^2 m = 4\pi G \rho_0 \cdot m
	\end{equation}
	
	has \textbf{no bounded solution}. Mathematical consistency requires the introduction of a $\LambdaT$ term:
	
	\begin{equation}
		\boxed{\nabla^2 m = 4\pi G \rho_0 \cdot m + \LambdaT \cdot m}
		\label{eq:modified_field_equation}
	\end{equation}
	
	\subsection{Determination of $\LambdaT$}
	\label{subsec:lambda_determination}
	
	For a stable homogeneous background $m = m_0 = \text{constant}$:
	\begin{equation}
		\nabla^2 m_0 = 0 = 4\pi G \rho_0 \cdot m_0 + \LambdaT \cdot m_0
	\end{equation}
	
	This yields:
	\begin{equation}
		\boxed{\LambdaT = -4\pi G \rho_0}
		\label{eq:lambda_value}
	\end{equation}
	
	\textbf{Dimensional verification}:
	\begin{itemize}
		\item $[\LambdaT] = [4\pi G \rho_0] = [1][E^{-2}][E^4] = [E^2]$ \checkmark
	\end{itemize}
	
	\subsection{Connection to Cosmological Parameters}
	\label{subsec:cosmological_connection}
	
	Using the Friedmann equation relationship:
	\begin{equation}
		\Hzero^2 = \frac{8\pi G \rho_0}{3}
	\end{equation}
	
	we can express the critical density as:
	\begin{equation}
		\rho_0 = \frac{3\Hzero^2}{8\pi G}
	\end{equation}
	
	Substituting into \cref{eq:lambda_value}:
	\begin{equation}
		\LambdaT = -4\pi G \cdot \frac{3\Hzero^2}{8\pi G} = -\frac{3\Hzero^2}{2}
		\label{eq:lambda_hubble_relation}
	\end{equation}
	
	\section{Emergence of $H_0$ from Regime Transitions}
	\label{sec:h0_emergence}
	
	\subsection{Local vs. Cosmic Regime Parameters}
	\label{subsec:regime_parameters}
	
	The $\kappa$ parameter exhibits different behavior in different regimes:
	
	\textbf{Local regime} ($r \ll \Hzero^{-1}$):
	\begin{equation}
		\kappaparam = \alpha_{\kappa} \Hzero \xipar
		\label{eq:kappa_local}
	\end{equation}
	
	\textbf{Cosmic regime} ($r \gg \Hzero^{-1}$):
	\begin{equation}
		\boxed{\kappaparam = \Hzero}
		\label{eq:kappa_cosmic}
	\end{equation}
	
	where $\xipar = 2\sqrt{G} \cdot m$ is the T0 scale parameter.
	
	\subsection{Physical Mechanism of Regime Transition}
	\label{subsec:regime_mechanism}
	
	The transition occurs when the $\LambdaT$ term becomes comparable to the local gravitational term:
	\begin{equation}
		|\LambdaT \cdot m| \sim |4\pi G \rho_0 \cdot m|
	\end{equation}
	
	This defines a characteristic transition scale:
	\begin{equation}
		r_{\text{transition}} \sim \frac{1}{\sqrt{|\LambdaT|}} = \frac{1}{\sqrt{4\pi G \rho_0}}
	\end{equation}
	
	Using the Friedmann relation:
	\begin{equation}
		r_{\text{transition}} \sim \frac{1}{\sqrt{4\pi G \cdot \frac{3\Hzero^2}{8\pi G}}} = \frac{1}{\sqrt{\frac{3\Hzero^2}{2}}} \sim \Hzero^{-1}
	\end{equation}
	
	\textbf{Key result}: The transition scale is naturally $\sim \Hzero^{-1}$.
	
	\subsection{Derivation of $H_0$ from $\kappa$}
	\label{subsec:h0_from_kappa}
	
	In the cosmic regime, where cosmic screening dominates, the energy loss mechanism yields:
	\begin{equation}
		\kappaparam = g_T \omega \frac{2G}{r_{\text{cosmic}}}
	\end{equation}
	
	With cosmic screening effects ($\xipar \to \xipar/2$) and $g_T = 1$:
	\begin{equation}
		\kappaparam = \omega \frac{G}{\Hzero^{-1}} = \omega G \Hzero
	\end{equation}
	
	For the universal cosmic scale ($\omega \sim G^{-1}$ in appropriate units):
	\begin{equation}
		\boxed{\kappaparam = \Hzero}
	\end{equation}
	
	\textbf{This is the fundamental emergence of $H_0$ from the T0 field structure.}
	
	\section{Scale Hierarchy and Unit Relations}
	\label{sec:scale_hierarchy}
	
	\subsection{Fundamental Scale Relationships}
	\label{subsec:scale_relationships}
	
	The T0 model connects scales across the entire hierarchy of physics:
	
	\begin{table}[htbp]
		\centering
		\begin{tabular}{lccc}
			\toprule
			\textbf{Scale} & \textbf{Characteristic Length} & \textbf{Value (m)} & \textbf{Ratio to $\lP$} \\
			\midrule
			Planck & $\lP = \sqrt{G\hbar/c^3}$ & $1.62 \times 10^{-35}$ & $1$ \\
			Electron T0 & $r_0(e) = 2Gm_e/c^2$ & $1.35 \times 10^{-57}$ & $8.37 \times 10^{-23}$ \\
			Proton T0 & $r_0(p) = 2Gm_p/c^2$ & $2.48 \times 10^{-54}$ & $1.54 \times 10^{-19}$ \\
			Nuclear & $\sim 10^{-15}$ & $10^{-15}$ & $6.2 \times 10^{19}$ \\
			Atomic & $\sim 10^{-10}$ & $10^{-10}$ & $6.2 \times 10^{24}$ \\
			Macroscopic & $\sim 1$ & $1$ & $6.2 \times 10^{34}$ \\
			Solar System & $\sim 10^{12}$ & $10^{12}$ & $6.2 \times 10^{46}$ \\
			Galactic & $\sim 10^{21}$ & $10^{21}$ & $6.2 \times 10^{55}$ \\
			Hubble & $\ell_H = c/H_0$ & $1.32 \times 10^{26}$ & $8.17 \times 10^{60}$ \\
			\bottomrule
		\end{tabular}
		\caption{Scale hierarchy in the T0 model}
		\label{tab:scale_hierarchy}
	\end{table}
	
	\subsection{$\xi$ Parameter Scaling}
	\label{subsec:xi_scaling}
	
	The dimensionless $\xi$ parameter connects particle masses to the Planck scale:
	\begin{equation}
		\xipar = \frac{r_0}{\lP} = \frac{2Gm}{\sqrt{G\hbar/c^3}} = 2\sqrt{G} \cdot m = \frac{2m}{M_P}
	\end{equation}
	
	\begin{table}[htbp]
		\centering
		\begin{tabular}{lcc}
			\toprule
			\textbf{Particle} & \textbf{Mass} & \textbf{$\xi$ Value} \\
			\midrule
			Electron & $m_e \approx 0.511$ MeV & $\xi_e \approx 8.37 \times 10^{-23}$ \\
			Proton & $m_p \approx 938$ MeV & $\xi_p \approx 1.54 \times 10^{-19}$ \\
			Higgs & $m_h \approx 125$ GeV & $\xi_h \approx 2.05 \times 10^{-17}$ \\
			Top quark & $m_t \approx 173$ GeV & $\xi_t \approx 2.84 \times 10^{-17}$ \\
			Planck mass & $M_P \approx 1.22 \times 10^{19}$ GeV & $\xi_P = 2$ \\
			\bottomrule
		\end{tabular}
		\caption{$\xi$ parameter values for different particles}
		\label{tab:xi_values}
	\end{table}
	
	\textbf{Key observation}: All known particles have $\xi \ll 1$, indicating they operate far below the Planck scale.
	
	\subsection{Cosmic Screening Effects}
	\label{subsec:cosmic_screening}
	
	In infinite, homogeneous fields, the effective $\xi$ parameter is reduced by a factor of 2:
	\begin{equation}
		\xi_{\text{eff}} = \frac{\xi}{2} = \sqrt{G} \cdot m
		\label{eq:xi_effective}
	\end{equation}
	
	This cosmic screening effect leads to:
	\begin{align}
		\beta_{\text{cosmic}} &= \frac{Gm}{r} = \frac{\beta_{\text{local}}}{2} \\
		\kappaparam_{\text{cosmic}} &= \Hzero \quad \text{(independent of local mass)}
	\end{align}
	
	\section{Numerical Analysis and Unit Conversions}
	\label{sec:numerical_analysis}
	
	\subsection{Hubble Parameter in Various Units}
	\label{subsec:hubble_units}
	
	The Hubble parameter $H_0 \approx 70$ km/s/Mpc converts to:
	
	\begin{table}[htbp]
		\centering
		\begin{tabular}{lcc}
			\toprule
			\textbf{Unit System} & \textbf{$H_0$ Value} & \textbf{Dimension} \\
			\midrule
			SI & $2.27 \times 10^{-18}$ s$^{-1}$ & $[T^{-1}]$ \\
			Natural ($\hbar = c = 1$) & $2.27 \times 10^{-18}$ & $[E]$ \\
			Planck units & $1.22 \times 10^{-61}$ $E_P$ & $[E]$ \\
			eV & $1.5 \times 10^{-42}$ eV & $[E]$ \\
			\bottomrule
		\end{tabular}
		\caption{Hubble parameter in different unit systems}
		\label{tab:hubble_units}
	\end{table}
	
	\subsection{Scale Ratios and Hierarchies}
	\label{subsec:scale_ratios}
	
	The fundamental ratio between Hubble and Planck scales:
	\begin{align}
		\frac{\ell_H}{\lP} &= \frac{c/H_0}{\sqrt{G\hbar/c^3}} \approx 8.17 \times 10^{60} \\
		\frac{t_H}{t_P} &= \frac{1/H_0}{\sqrt{G\hbar/c^5}} \approx 8.17 \times 10^{60}
	\end{align}
	
	\textbf{This factor of $\sim 10^{61}$ appears throughout the T0 model hierarchy.}
	
	\subsection{Energy Scale Comparisons}
	\label{subsec:energy_comparisons}
	
	\begin{table}[htbp]
		\centering
		\begin{tabular}{lcc}
			\toprule
			\textbf{Energy Scale} & \textbf{Value (eV)} & \textbf{Ratio to $H_0$} \\
			\midrule
			Planck energy & $E_P \approx 1.22 \times 10^{28}$ & $\sim 10^{70}$ \\
			Electroweak scale & $v \approx 2.46 \times 10^{11}$ & $\sim 10^{53}$ \\
			Proton rest mass & $m_p c^2 \approx 9.38 \times 10^8$ & $\sim 10^{50}$ \\
			Electron rest mass & $m_e c^2 \approx 5.11 \times 10^5$ & $\sim 10^{47}$ \\
			Room temperature & $k_B T \approx 0.025$ & $\sim 10^{40}$ \\
			Hubble energy & $H_0 \hbar \approx 1.5 \times 10^{-42}$ & $1$ \\
			\bottomrule
		\end{tabular}
		\caption{Energy scales relative to Hubble energy}
		\label{tab:energy_scales}
	\end{table}
	
	\section{Physical Implications and Experimental Predictions}
	\label{sec:physical_implications}
	
	\subsection{Modified Gravitational Potential}
	\label{subsec:modified_gravity}
	
	The T0 model predicts a linear correction to the gravitational potential:
	\begin{equation}
		\Phi(r) = -\frac{GM}{r} + \kappaparam r
	\end{equation}
	
	For cosmological scales where $\kappaparam = \Hzero$:
	\begin{equation}
		\Phi(r) = -\frac{GM}{r} + \Hzero r
		\label{eq:cosmological_potential}
	\end{equation}
	
	\textbf{Experimental signature}: This predicts observable deviations from Newtonian gravity at large scales, potentially explaining dark energy phenomena without requiring exotic matter.
	
	\subsection{Wavelength-Dependent Redshift}
	\label{subsec:wavelength_redshift}
	
	The T0 energy loss mechanism predicts:
	\begin{equation}
		z(\lambda) = z_0\left(1 - \ln\frac{\lambda}{\lambda_0}\right)
		\label{eq:wavelength_redshift}
	\end{equation}
	
	\textbf{Observable consequence}: Multi-wavelength observations should reveal systematic wavelength-dependent redshift variations, distinguishing T0 predictions from standard cosmological models.
	
	\subsection{Regime-Dependent Effects}
	\label{subsec:regime_effects}
	
	The transition between local and cosmic regimes at $r \sim \Hzero^{-1}$ should produce observable effects:
	
	\begin{itemize}
		\item \textbf{Galaxy rotation curves}: Transition from local $\kappaparam = \alpha_{\kappa} \Hzero \xipar$ to cosmic $\kappaparam = \Hzero$
		\item \textbf{Cosmic structure formation}: Modified growth rates due to $\LambdaT$ term
		\item \textbf{CMB anisotropies}: Different acoustic oscillation patterns from modified recombination physics
	\end{itemize}
	
	\section{Comparison with Standard Cosmological Models}
	\label{sec:model_comparison}
	
	\subsection{Parameter Count Comparison}
	\label{subsec:parameter_comparison}
	
	\begin{table}[htbp]
		\centering
		\begin{tabular}{lcc}
			\toprule
			\textbf{Model} & \textbf{Free Parameters} & \textbf{$H_0$ Status} \\
			\midrule
			$\Lambda$CDM & $\sim 6$ & Empirical input \\
			Extended $\Lambda$CDM & $>10$ & Empirical input \\
			T0 Model & $0$ & Derived from field theory \\
			\bottomrule
		\end{tabular}
		\caption{Comparison of cosmological models}
		\label{tab:model_comparison}
	\end{table}
	
	\subsection{Physical Mechanisms}
	\label{subsec:mechanism_comparison}
	
	\begin{table}[htbp]
		\centering
		\begin{tabular}{lcc}
			\toprule
			\textbf{Phenomenon} & \textbf{Standard Model} & \textbf{T0 Model} \\
			\midrule
			Cosmic expansion & Spatial metric expansion & No spatial expansion \\
			Redshift mechanism & Doppler + expansion & Energy loss to time field \\
			Dark energy & Cosmological constant $\Lambda$ & Geometric $\LambdaT$ term \\
			Hubble parameter & Empirical constant & Emergent transition scale \\
			Acceleration & Unknown dark energy & Modified potential $\kappaparam r$ \\
			\bottomrule
		\end{tabular}
		\caption{Physical mechanism comparison}
		\label{tab:mechanism_comparison}
	\end{table}
	
	\section{Mathematical Consistency and Dimensional Analysis}
	\label{sec:mathematical_consistency}
	
	\subsection{Complete Dimensional Verification}
	\label{subsec:dimensional_verification}
	
	All T0 model equations maintain dimensional consistency:
	
	\begin{table}[htbp]
		\centering
		\begin{tabular}{lccc}
			\toprule
			\textbf{Equation} & \textbf{Left Side} & \textbf{Right Side} & \textbf{Status} \\
			\midrule
			Time field & $[\Tfield] = [E^{-1}]$ & $[1/\max(m,\omega)] = [E^{-1}]$ & \checkmark \\
			Field equation & $[\nabla^2 m] = [E^3]$ & $[4\pi G \rho m] = [E^3]$ & \checkmark \\
			Energy loss & $[dE/dr] = [E^2]$ & $[g_T \omega^2 2G/r^2] = [E^2]$ & \checkmark \\
			$\LambdaT$ term & $[\LambdaT] = [E^2]$ & $[4\pi G \rho_0] = [E^2]$ & \checkmark \\
			$\kappa$ parameter & $[\kappaparam] = [E^2]$ & $[H_0 \hbar] = [E^2]$ & \checkmark \\
			Modified potential & $[\Phi] = [E^2]$ & $[GM/r + \kappaparam r] = [E^2]$ & \checkmark \\
			\bottomrule
		\end{tabular}
		\caption{Dimensional consistency verification}
		\label{tab:dimensional_check}
	\end{table}
	
	\subsection{Internal Consistency Checks}
	\label{subsec:consistency_checks}
	
	Key relationships that must be satisfied:
	\begin{align}
		\LambdaT &= -\frac{3\Hzero^2}{2} \quad \text{(from Friedmann relation)} \\
		\kappaparam &= \Hzero \quad \text{(cosmic regime limit)} \\
		\xi_{\text{eff}} &= \frac{\xi}{2} \quad \text{(cosmic screening)} \\
		r_{\text{transition}} &\sim \Hzero^{-1} \quad \text{(regime boundary)}
	\end{align}
	
	All relationships are mathematically consistent and dimensionally verified.
	
	\section{Conclusions and Future Directions}
	\label{sec:conclusions}
	
	\subsection{Key Achievements}
	\label{subsec:key_achievements}
	
	This analysis has established:
	
	\begin{enumerate}
		\item \textbf{$H_0$ is not empirical}: The Hubble parameter emerges naturally from T0 field theory as a characteristic transition scale between local and cosmic regimes.
		
		\item \textbf{$\kappa = H_0$ relationship}: The linear potential parameter becomes the Hubble parameter in the cosmic limit, providing a fundamental connection between microscopic and macroscopic physics.
		
		\item \textbf{Scale hierarchy unification}: The T0 model provides a unified framework spanning 61 orders of magnitude from Planck to Hubble scales.
		
		\item \textbf{Parameter-free cosmology}: Unlike standard models requiring multiple empirical parameters, the T0 model derives all cosmological parameters from field theory.
		
		\item \textbf{Dimensional consistency}: All equations maintain perfect dimensional consistency in natural units.
		
		\item \textbf{Testable predictions}: The model makes specific predictions for wavelength-dependent redshift, modified gravity, and regime-dependent effects.
	\end{enumerate}
	
	\subsection{Physical Significance}
	\label{subsec:physical_significance}
	
	The emergence of $H_0$ from field theory represents a paradigm shift in cosmological thinking:
	
	\begin{tcolorbox}[colback=green!5!white,colframe=green!75!black,title=Fundamental Insight]
		The Hubble parameter is not a measure of spatial expansion but rather the characteristic energy scale where local field dynamics transition to cosmic background effects. This transition is governed by the cosmic screening mechanism encoded in the $\LambdaT$ term.
	\end{tcolorbox}
	
	\subsection{Resolution of Cosmological Problems}
	\label{subsec:problem_resolution}
	
	The T0 approach potentially resolves several outstanding cosmological issues:
	
	\begin{itemize}
		\item \textbf{Hubble tension}: Different measurement methods probe different physical regimes with naturally different effective parameters.
		\item \textbf{Dark energy problem}: The acceleration arises geometrically from the $\kappaparam r$ term rather than requiring exotic matter.
		\item \textbf{Fine-tuning problems}: All parameters emerge from field theory without anthropic adjustments.
		\item \textbf{Coincidence problems}: The current epoch is not special; regime transitions occur naturally at the scale $\Hzero^{-1}$.
	\end{itemize}
	
	\subsection{Future Research Directions}
	\label{subsec:future_directions}
	
	\textbf{Theoretical developments}:
	\begin{itemize}
		\item Higher-order corrections to the $\kappa$-$H_0$ relationship
		\item Quantum field theory foundations of the regime transition mechanism
		\item Non-linear effects in strong-field regimes
		\item Extension to non-Abelian gauge theories
	\end{itemize}
	
	\textbf{Observational tests}:
	\begin{itemize}
		\item High-precision multi-wavelength redshift measurements
		\item Tests of modified gravity at intermediate scales ($r \sim \Hzero^{-1}$)
		\item Analysis of cosmic structure formation with $\LambdaT$ effects
		\item Laboratory tests of energy-dependent photon propagation
	\end{itemize}
	
	\textbf{Computational studies}:
	\begin{itemize}
		\item Numerical simulations of regime transitions
		\item N-body simulations with modified gravitational potential
		\item Statistical analysis of existing cosmological data within T0 framework
	\end{itemize}
	
	\section{Final Remarks}
	\label{sec:final_remarks}
	
	The derivation of $H_0$ and $\kappa$ from first principles represents a significant achievement in the development of the T0 model. By showing that these parameters emerge naturally from field-theoretic considerations rather than requiring empirical input, we demonstrate the potential for a truly fundamental approach to cosmology.
	
	The scale hierarchy analysis reveals the remarkable fact that the T0 model provides a unified description spanning from quantum gravity scales ($\sim 10^{-35}$ m) to cosmological horizons ($\sim 10^{26}$ m) — a range of 61 orders of magnitude. This unification is achieved through the elegant mechanism of regime transitions governed by the cosmic screening effect.
	
	Perhaps most significantly, the relationship $\kappa = H_0$ in the cosmic regime provides a direct connection between the microscopic world of particle physics (through the $\xi$ parameter) and the macroscopic realm of cosmology. This connection suggests that the large-scale structure of the universe is not independent of quantum field theory but emerges naturally from the same underlying principles.
	
	The T0 model thus offers not just an alternative to standard cosmology, but a more fundamental framework that derives cosmological parameters from quantum field theory while maintaining mathematical rigor and dimensional consistency throughout. The path from energy loss mechanisms to cosmic-scale phenomena demonstrates the power of field-theoretic approaches to unify apparently disparate domains of physics.
	
	\begin{thebibliography}{99}
		\bibitem{pascher_derivation_beta_2025} 
		Pascher, J. (2025). \textit{Field-Theoretic Derivation of the $\beta_T$ Parameter in Natural Units ($\hbar = c = 1$)}. \\
		Available at: \url{https://github.com/jpascher/T0-Time-Mass-Duality/blob/main/2/pdf/DerivationVonBetaEn.pdf}
		
		\bibitem{pascher_cmb_analysis_2025}
		Pascher, J. (2025). \textit{Temperature Units in Natural Units: Field-Theoretic Foundations and CMB Analysis}. \\
		Available at: \url{https://github.com/jpascher/T0-Time-Mass-Duality/blob/main/2/pdf/TempEinheitenCMBEn.pdf}
		\bibitem{pascher_natural_units_2025}
		Pascher, J. (2025). \textit{Natural Unit Systems: Universal Energy Conversion and Fundamental Length Scale Hierarchy}. \\
		Available at: \url{https://github.com/jpascher/T0-Time-Mass-Duality/blob/main/2/pdf/NatEinheitenSystematikEn.pdf}
		
		\bibitem{pascher_parameter_system_2025}
		Pascher, J. (2025). \textit{Parameter System-Dependency in T0-Model: SI vs. Natural Units}. \\
		Available at: \url{https://github.com/jpascher/T0-Time-Mass-Duality/blob/main/2/pdf/ParameterSystemdipendentEn.pdf}
		
		\bibitem{pascher_alpha_proof_2025}
		Pascher, J. (2025). \textit{Mathematical Proof: The Fine Structure Constant $\alpha = 1$ in Natural Units}. \\
		Available at: \url{https://github.com/jpascher/T0-Time-Mass-Duality/blob/main/2/pdf/ResolvingTheConstantsAlfaEn.pdf}
		Weinberg, S. (2008). \textit{Cosmology}. Oxford University Press.
		
		\bibitem{peebles1993}
		Peebles, P. J. E. (1993). \textit{Principles of Physical Cosmology}. Princeton University Press.
		
		\bibitem{planck2020}
		Planck Collaboration (2020). Planck 2018 results. VI. Cosmological parameters. \textit{Astronomy \& Astrophysics}, 641, A6.
		
		\bibitem{riess2019}
		Riess, A. G., et al. (2019). Large Magellanic Cloud Cepheid Standards Provide a 1\% Foundation for the Determination of the Hubble Constant. \textit{The Astrophysical Journal}, 876(1), 85.
		
		\bibitem{weinberg1995}
		Weinberg, S. (1995). \textit{The Quantum Theory of Fields, Volume I: Foundations}. Cambridge University Press.
		
		\bibitem{misner1973}
		Misner, C. W., Thorne, K. S., and Wheeler, J. A. (1973). \textit{Gravitation}. W. H. Freeman and Company.
	\end{thebibliography}
	
\end{document}