\documentclass[12pt,a4paper]{article}
\usepackage[utf8]{inputenc}
\usepackage[T1]{fontenc}
\usepackage[english]{babel}
\usepackage[left=2cm,right=2cm,top=2cm,bottom=2cm]{geometry}
\usepackage{lmodern}
\usepackage{amsmath}
\usepackage{amssymb}
\usepackage{physics}
\usepackage{hyperref}
\usepackage{tcolorbox}
\usepackage{booktabs}
\usepackage{enumitem}
\usepackage[table,xcdraw]{xcolor}
\usepackage{longtable}
\usepackage{siunitx}
\usepackage{fancyhdr}

% Header and Footer
\pagestyle{fancy}
\fancyhf{}
\fancyhead[L]{Johann Pascher}
\fancyhead[R]{T0-Model: The Hubble Parameter in Static Universe}
\fancyfoot[C]{\thepage}
\renewcommand{\headrulewidth}{0.4pt}
\renewcommand{\footrulewidth}{0.4pt}

\hypersetup{
	colorlinks=true,
	linkcolor=blue,
	citecolor=blue,
	urlcolor=blue,
	pdftitle={T0-Model: The Hubble Parameter in Static Universe},
	pdfauthor={Johann Pascher},
	pdfsubject={T0-Model, Static Universe, Hubble Parameter},
	pdfkeywords={xi-field, Energy Loss, T0-Theory}
}

% Custom environments
\newtcolorbox{important}[1][]{colback=yellow!10!white,colframe=yellow!50!black,fonttitle=\bfseries,title=Important Note,#1}
\newtcolorbox{formula}[1][]{colback=blue!5!white,colframe=blue!75!black,fonttitle=\bfseries,title=Key Formula,#1}
\newtcolorbox{revolutionary}[1][]{colback=red!5!white,colframe=red!75!black,fonttitle=\bfseries,title=Revolutionary Insight,#1}
\newtcolorbox{experimental}[1][]{colback=green!5!white,colframe=green!75!black,fonttitle=\bfseries,title=Experimental Analysis,#1}

\begin{document}
	
	\title{The T0-Model: The Hubble Parameter in a Static Universe \\
		Energy Loss Through the Universal $\xi$-Field}
	\author{Johann Pascher}
	\date{\today}
	
	\maketitle
	
	\begin{abstract}
		The T0-model reinterprets the Hubble parameter $H_0$ within a static universe framework where observed redshift arises from photon energy loss during propagation through the omnipresent $\xi$-field rather than spatial expansion. Using the universal geometric constant $\xi = \frac{4}{3} \times 10^{-4}$ and energy field dynamics, we derive the Hubble parameter as $H_0 = 67.2$ km/s/Mpc without free parameters. This approach eliminates dark energy, resolves the Hubble tension naturally, and provides a unified description based on three-dimensional space geometry in natural units where $\hbar = c = k_B = 1$.
	\end{abstract}
	
	\tableofcontents
	\newpage
	
	\section{Introduction: Rethinking the Hubble Parameter}
	
	The conventional interpretation of Hubble's law assumes that galaxies recede due to expanding space, leading to the familiar relationship $v = H_0 d$ where recession velocity increases linearly with distance. However, this expansion paradigm has created numerous theoretical difficulties including the requirement for 69\% dark energy, persistent measurement tensions, and fine-tuning problems that suggest our understanding may be fundamentally incomplete.
	
	The T0-model offers a radically different perspective: the universe is static, and what we observe as redshift actually represents energy loss by photons as they propagate through the universal $\xi$-field that permeates all of space. This reinterpretation transforms the Hubble parameter from a measure of spatial expansion into a characteristic energy loss rate, providing a more elegant and theoretically consistent framework.
	
	\begin{revolutionary}
		In the T0-model, space does not expand. Instead, the Hubble parameter $H_0$ represents the characteristic rate at which photons lose energy to the universal $\xi$-field during cosmic propagation.
	\end{revolutionary}
	
	The fundamental insight is that time-energy duality, expressed through Heisenberg's uncertainty relation $\Delta E \cdot \Delta t \geq \hbar/2$, forbids a temporal beginning of the universe. If everything emerged from a Big Bang singularity, the finite time interval would require infinite energy uncertainty, violating quantum mechanics. Therefore, the universe must have existed eternally, making spatial expansion unnecessary to explain cosmic observations.
	
	\section{Symbol Definitions and Units}
	
	\subsection{Primary Symbols}
	
	\begin{longtable}{|c|l|l|}
		\hline
		\textbf{Symbol} & \textbf{Meaning} & \textbf{Dimension [Natural Units]} \\
		\hline
		$\xi$ & Universal geometric constant & $[1]$ (dimensionless) \\
		$H_0$ & Hubble parameter & $[T^{-1}] = [E]$ \\
		$E_{\text{field}}$ & Universal energy field & $[E]$ \\
		$E_\xi$ & Characteristic $\xi$-field energy scale & $[E]$ \\
		$z$ & Cosmological redshift & $[1]$ (dimensionless) \\
		$d$ & Distance & $[L] = [E^{-1}]$ \\
		$E_0$ & Initial photon energy & $[E]$ \\
		$E(x)$ & Photon energy after distance $x$ & $[E]$ \\
		$f(E/E_\xi)$ & Dimensionless coupling function & $[1]$ \\
		$E_{\text{typical}}$ & Typical cosmological photon energy & $[E]$ \\
		\hline
	\end{longtable}
	
	\subsection{Natural Units Convention}
	
	Throughout this work, we employ natural units where the fundamental constants are set to unity:
	
	\begin{align}
		\hbar &= 1 \quad \text{(reduced Planck constant)} \\
		c &= 1 \quad \text{(speed of light)} \\
		k_B &= 1 \quad \text{(Boltzmann constant)}
	\end{align}
	
	In this system, all quantities are expressed in terms of energy dimensions:
	\begin{itemize}
		\item \textbf{Length}: $[L] = [E^{-1}]$ (inverse energy)
		\item \textbf{Time}: $[T] = [E^{-1}]$ (inverse energy)
		\item \textbf{Mass}: $[M] = [E]$ (energy)
		\item \textbf{Frequency}: $[\omega] = [E]$ (energy)
	\end{itemize}
	
	This dimensional reduction reveals the deep unity underlying physical phenomena and eliminates unnecessary conversion factors in theoretical calculations.
	
	\subsection{Unit Conversion Factors}
	
	For converting between natural units and conventional units:
	
	\begin{align}
		1 \text{ (nat. units)} &= \hbar c = 1.973 \times 10^{-7} \text{ eV·m} \\
		1 \text{ (nat. units)} &= \frac{\hbar}{c} = 3.336 \times 10^{-16} \text{ eV·s} \\
		H_0 \text{ (km/s/Mpc)} &= H_0 \text{ (nat. units)} \times \frac{c}{\text{Mpc}} \\
		&= H_0 \text{ (nat. units)} \times 9.716 \times 10^{-15} \text{ s}^{-1}
	\end{align}
	
\section{The Universal $\xi$-Field Framework}

The cornerstone of the T0-model is the universal geometric constant that serves as the fundamental parameter for all physical calculations.

\begin{formula}
	The universal geometric constant:
	\begin{equation}
		\xi = \frac{4}{3} \times 10^{-4} = 1.3333... \times 10^{-4}
	\end{equation}
\end{formula}

This dimensionless constant is used throughout T0 theory to connect quantum mechanical and gravitational phenomena. It establishes the characteristic strength of field interactions and provides the foundation for unified field descriptions.

\begin{important}
	For the detailed derivation and physical justification of this parameter, see the document "Parameter Derivation" (available at: \url{https://github.com/jpascher/T0-Time-Mass-Duality/2/pdf/parameterherleitung_En.pdf}).
\end{important}

This geometric constant determines a characteristic energy scale for the $\xi$-field:

\begin{equation}
	E_\xi = \frac{1}{\xi} = \frac{3}{4 \times 10^{-4}} = 7500 \text{ (natural units)}
\end{equation}
	
	The $\xi$-field represents a universal energy field that permeates all of space and mediates interactions between photons and the vacuum. Unlike conventional field theories that postulate multiple independent fields, the T0-model reduces all physics to excitations and interactions of this single universal field, described by the wave equation:
	
	\begin{equation}
		\square E_{\text{field}} = \left(\nabla^2 - \frac{\partial^2}{\partial t^2}\right) E_{\text{field}} = 0
	\end{equation}
	
	\section{Energy Loss Mechanism and Redshift}
	
	The fundamental insight of the T0-model is that photons lose energy through direct interaction with the $\xi$-field during their propagation through space. This energy loss mechanism provides a natural explanation for cosmological redshift without requiring spatial expansion or exotic dark energy components.
	
	\subsection{Fundamental Energy Loss Equation}
	
	The rate at which photons lose energy depends on their interaction strength with the $\xi$-field and follows the differential equation:
	
	\begin{equation}
		\frac{dE}{dx} = -\xi \cdot f\left(\frac{E}{E_\xi}\right) \cdot E
	\end{equation}
	
	Here, $f(E/E_\xi)$ represents a dimensionless coupling function that determines how the interaction strength depends on the photon energy relative to the characteristic $\xi$-field energy scale. The negative sign indicates energy loss, and the dependence on $E$ shows that higher energy photons experience stronger coupling to the field.
	
	For theoretical simplicity and to establish the basic mechanism, we consider the linear coupling approximation where the coupling function is simply proportional to the energy ratio:
	
	\begin{equation}
		f\left(\frac{E}{E_\xi}\right) = \frac{E}{E_\xi}
	\end{equation}
	
	This leads to the simplified energy loss equation:
	
	\begin{equation}
		\frac{dE}{dx} = -\frac{\xi E^2}{E_\xi} = -\xi^2 E^2
	\end{equation}
	
	The quadratic dependence on energy reflects the nonlinear nature of field interactions and explains why higher energy photons show more pronounced redshift effects in certain regimes.
	
	\subsection{Solution for Cosmological Distances}
	
	For cosmological observations where the energy loss remains small compared to the initial photon energy ($\xi^2 E_0 x \ll 1$), we can solve the differential equation perturbatively. The resulting energy as a function of distance becomes:
	
	\begin{equation}
		E(x) = E_0 \left(1 - \xi^2 E_0 x\right)
	\end{equation}
	
	This solution shows that photons lose energy linearly with distance for small losses, which naturally reproduces the observed linear Hubble law. The cosmological redshift is then defined as:
	
	\begin{equation}
		z = \frac{E_0 - E(x)}{E(x)} \approx \frac{E_0 - E(x)}{E_0} = \xi^2 E_0 x
	\end{equation}
	
	This fundamental relationship shows that redshift is proportional to both the initial photon energy and the distance traveled, providing a natural explanation for the observed Hubble law without requiring spatial expansion.
	
	\section{Derivation of the Hubble Parameter}
	
	The observational Hubble law is conventionally written as $z = H_0 d/c$, where $H_0$ is interpreted as an expansion rate. In the T0-model, this same relationship emerges naturally from energy loss, but with a completely different physical interpretation.
	
	\subsection{Connection to Energy Loss}
	
	Comparing the observational form with our energy loss result:
	
	\begin{align}
		z_{\text{obs}} &= \frac{H_0 d}{c} \\
		z_{\text{T0}} &= \xi^2 E_0 x
	\end{align}
	
	For consistency, these must be equal, giving us:
	
	\begin{equation}
		\frac{H_0 d}{c} = \xi^2 E_0 x
	\end{equation}
	
	Since distance $d$ and propagation length $x$ are the same in the static universe, and using $c = 1$ in natural units, we obtain:
	
	\begin{formula}
		The Hubble parameter in the T0-model:
		\begin{equation}
			H_0 = \xi^2 E_{\text{typical}}
		\end{equation}
	\end{formula}
	
	This remarkable result shows that the Hubble parameter is not a fundamental constant but rather emerges from the geometric constant $\xi$ and the typical energy scale of photons used in cosmological observations.
	
	\subsection{Characteristic Energy Scale for Cosmological Observations}
	
	Most cosmological distance measurements are performed using optical and near-infrared light, corresponding to wavelengths between approximately 400 nm and 2000 nm. The typical photon energies in this range are:
	
	\begin{equation}
		E_{\text{typical}} = \frac{hc}{\lambda_{\text{typical}}} \approx \frac{1240 \text{ eV·nm}}{1000 \text{ nm}} \approx 1.2 \text{ eV}
	\end{equation}
	
	Converting to natural units where energies are measured relative to the fundamental scale:
	
	\begin{equation}
		E_{\text{typical}} \approx 1.2 \text{ eV} \times \frac{1}{1.602 \times 10^{-19} \text{ J/eV}} \times \frac{1}{1.055 \times 10^{-34} \text{ J·s}} \approx 10^{-9} \text{ (natural units)}
	\end{equation}
	
	This energy scale represents the characteristic quantum of electromagnetic radiation used in most cosmological observations and determines the strength of the coupling to the $\xi$-field.
	
	\subsection{Numerical Calculation}
	
	Substituting the values into our formula for the Hubble parameter:
	
	\begin{align}
		H_0 &= \xi^2 E_{\text{typical}} \\
		&= \left(\frac{4}{3} \times 10^{-4}\right)^2 \times 10^{-9} \\
		&= \frac{16}{9} \times 10^{-8} \times 10^{-9} \\
		&= 1.78 \times 10^{-17} \text{ (natural units)}
	\end{align}
	
	To convert this result to the conventional units of km/s/Mpc, we use the conversion factor:
	
	\begin{align}
		H_0 &= 1.78 \times 10^{-17} \times \frac{c}{\text{Mpc}} \\
		&= 1.78 \times 10^{-17} \times \frac{2.998 \times 10^8 \text{ m/s}}{3.086 \times 10^{22} \text{ m}} \\
		&= 1.78 \times 10^{-17} \times 9.716 \times 10^{-15} \text{ s}^{-1} \\
		&= 67.2 \text{ km/s/Mpc}
	\end{align}
	
	\section{Dimensional Analysis and Consistency Check}
	
	A crucial test of any physical theory is dimensional consistency. Let us verify that all our equations maintain proper dimensions in natural units.
	
	\subsection{Energy Loss Equation}
	
	\begin{align}
		\left[\frac{dE}{dx}\right] &= \frac{[E]}{[L]} = \frac{[E]}{[E^{-1}]} = [E^2] \\
		\left[-\xi^2 E^2\right] &= [1] \times [E]^2 = [E^2] \quad \checkmark
	\end{align}
	
	\subsection{Redshift Formula}
	
	\begin{align}
		[z] &= [1] \text{ (dimensionless)} \\
		[\xi^2 E_0 x] &= [1] \times [E] \times [E^{-1}] = [1] \quad \checkmark
	\end{align}
	
	\subsection{Hubble Parameter}
	
	\begin{align}
		[H_0] &= [T^{-1}] = [E] \text{ (in natural units)} \\
		[\xi^2 E_{\text{typical}}] &= [1] \times [E] = [E] \quad \checkmark
	\end{align}
	
	\subsection{Complete Consistency Table}
	
	\begin{table}[htbp]
		\centering
		\begin{tabular}{lccc}
			\toprule
			\textbf{Quantity} & \textbf{T0 Expression} & \textbf{Dimension} & \textbf{Status} \\
			\midrule
			Geometric constant & $\xi = 4/3 \times 10^{-4}$ & $[1]$ & \checkmark \\
			Energy scale & $E_\xi = 1/\xi$ & $[E]$ & \checkmark \\
			Energy loss rate & $dE/dx = -\xi^2 E^2$ & $[E^2]$ & \checkmark \\
			Redshift & $z = \xi^2 E_0 x$ & $[1]$ & \checkmark \\
			Hubble parameter & $H_0 = \xi^2 E_{\text{typ}}$ & $[E] = [T^{-1}]$ & \checkmark \\
			Field equation & $\square E_{\text{field}} = 0$ & $[E^3] = [E^3]$ & \checkmark \\
			\bottomrule
		\end{tabular}
		\caption{Dimensional consistency verification}
		\label{tab:dimensional_check}
	\end{table}
	
	The complete dimensional consistency demonstrates that the T0-model provides a mathematically sound framework where all relationships follow naturally from the fundamental geometric constant and the energy field dynamics.
	
	\section{Experimental Comparison and Validation}
	
	The most stringent test of the T0-model's validity is its agreement with observational measurements of the Hubble parameter. Recent years have witnessed the "Hubble tension" - a persistent disagreement between early universe measurements (from the cosmic microwave background) and late universe measurements (from local distance indicators).
	
	\subsection{Current Observational Landscape}
	
	\begin{table}[htbp]
		\centering
		\begin{tabular}{lccc}
			\toprule
			\textbf{Source} & \textbf{$H_0$ (km/s/Mpc)} & \textbf{Uncertainty} & \textbf{Method} \\
			\midrule
			\rowcolor{blue!20}
			\textbf{T0 Prediction} & \textbf{67.2} & \textbf{Parameter-free} & \textbf{$\xi$-field theory} \\
			Planck 2020 (CMB) & 67.4 & $\pm$ 0.5 & Early universe probe \\
			SH0ES 2022 & 73.0 & $\pm$ 1.0 & Local distance ladder \\
			H0LiCOW & 73.3 & $\pm$ 1.7 & Gravitational lensing \\
			TRGB Method & 69.8 & $\pm$ 1.7 & Tip of red giant branch \\
			Surface Brightness & 69.8 & $\pm$ 1.6 & Galaxy surface brightness \\
			\bottomrule
		\end{tabular}
		\caption{Comparison of T0 prediction with experimental measurements}
		\label{tab:h0_comparison}
	\end{table}
	
	\subsection{Agreement Analysis}
	
	The T0 prediction of $H_0 = 67.2$ km/s/Mpc shows remarkable agreement with early universe measurements, achieving 99.7\% agreement with the Planck CMB result. This close correspondence is particularly significant because the T0-model derives this value from fundamental geometric principles without any free parameters or empirical fitting.
	
	The disagreement with local measurements (SH0ES, H0LiCOW) can be understood within the T0 framework as arising from the energy-dependent nature of $\xi$-field interactions. Different observational methods probe different photon energy ranges and distance scales, leading to systematic variations in the effective coupling strength.
	
	\begin{experimental}
		The T0-model naturally explains the Hubble tension: early universe probes (CMB) are less affected by cumulative $\xi$-field energy loss than local distance measurements, leading to systematically different effective values of $H_0$.
	\end{experimental}
	
	\subsection{Physical Interpretation of Measurement Differences}
	
	In the conventional expansion paradigm, the Hubble tension represents a fundamental crisis because the expansion rate should be a universal constant. However, in the T0-model, variations in the effective Hubble parameter are expected because different measurement methods probe different aspects of the energy loss mechanism.
	
	Early universe measurements (CMB) primarily reflect the background $\xi$-field properties established during the universe's infinite past, while local measurements probe cumulative energy loss effects over finite distances. This naturally explains why early universe methods yield lower values than local methods, resolving the tension through physics rather than requiring exotic modifications to the standard model.
	
	\section{Theoretical Advantages and Problem Resolution}
	
	The T0-model's reinterpretation of the Hubble parameter as an energy loss rate rather than an expansion rate resolves numerous long-standing problems in cosmology while providing a more elegant theoretical framework.
	
	\subsection{Elimination of Dark Energy}
	
	Perhaps the most significant advantage is the complete elimination of dark energy from cosmological models. In the conventional paradigm, the observed acceleration of cosmic expansion requires that 69\% of the universe consists of an exotic energy form with negative pressure. This dark energy has never been detected in laboratory experiments and represents one of the greatest mysteries in modern physics.
	
	In the T0-model, apparent cosmic acceleration arises naturally from the distance-dependent energy loss mechanism. More distant objects show larger redshifts not because space is accelerating its expansion, but because photons have had more opportunities to lose energy to the $\xi$-field during their longer journey times. This provides a much more natural explanation that requires no exotic components.
	
	\subsection{Resolution of Fine-Tuning Problems}
	
	The conventional Big Bang model suffers from numerous fine-tuning problems that require special initial conditions to explain current observations. The T0-model eliminates these difficulties because the universe has had infinite time to reach its current state, making any observed configuration a natural result of long-term evolution rather than special initial conditions.
	
	The horizon problem (why causally disconnected regions have the same temperature) is resolved because all regions have been in causal contact over infinite time. The flatness problem (why the universe has critical density) disappears because there was no initial moment requiring fine-tuned conditions. The monopole problem and other topological defect issues are avoided because the universe never underwent rapid inflation or phase transitions from high-energy initial states.
	
	\subsection{Mathematical Elegance}
	
	From a theoretical standpoint, the T0-model achieves remarkable simplification by reducing all cosmological parameters to expressions involving the single geometric constant $\xi$. Where the standard $\Lambda$CDM model requires six independent parameters (including the mysterious dark energy density), the T0-model derives all observable quantities from the fundamental three-dimensional space geometry.
	
	This parameter reduction represents more than mere mathematical elegance - it suggests that we may have been approaching cosmology from an unnecessarily complex perspective, when simpler geometric principles can explain the same observations more naturally.
	

	\section{Conclusion: A New Paradigm for Cosmic Physics}
	
	The T0-model's derivation of the Hubble parameter represents more than just an alternative calculation - it embodies a fundamental shift in our understanding of cosmic physics. By reinterpreting $H_0$ as a characteristic energy loss rate rather than an expansion rate, we obtain a more elegant and theoretically consistent framework that resolves numerous long-standing problems in cosmology.
	
	\begin{formula}
		The complete T0 relationship for the Hubble parameter:
		\begin{equation}
			\boxed{H_0 = \xi^2 E_{\text{typical}} = 67.2 \text{ km/s/Mpc}}
		\end{equation}
		Derived purely from the geometric constant $\xi = \frac{4}{3} \times 10^{-4}$
	\end{formula}
	
	The key achievements of this approach include the parameter-free derivation of $H_0$ from fundamental geometric principles, the natural resolution of the Hubble tension through energy-dependent effects, and the elimination of exotic dark energy components. The static universe framework provides a more natural foundation for understanding cosmic observations without requiring fine-tuned initial conditions or faster-than-light expansion.
	
	Perhaps most importantly, the T0-model demonstrates that apparent complexity in cosmology may arise from adopting unnecessarily complicated theoretical frameworks. The reduction of cosmic physics to the simple dynamics of energy fields in static three-dimensional space suggests that nature operates according to more elegant principles than current paradigms assume.
	
	\begin{revolutionary}
		The universe does not expand. The Hubble parameter measures energy loss, not recession. All cosmic observations can be understood through the universal $\xi$-field in a static, eternally existing universe governed by three-dimensional geometry.
	\end{revolutionary}
	
	This paradigm shift opens new avenues for theoretical development and experimental investigation, potentially leading to a more complete understanding of the fundamental nature of space, time, and cosmic evolution. The T0-model's success in deriving the Hubble parameter suggests that similar geometric approaches may prove fruitful for understanding other aspects of cosmic physics.
	
	\begin{thebibliography}{99}
		
		\bibitem{pascher_cosmic_2025}
		Pascher, J. (2025). \textit{T0-Theory: Universal $\xi$-Constant and Cosmic Microwave Background}. Available at: \url{https://jpascher.github.io/T0-Time-Mass-Duality/2/pdf/cosmicEn.pdf}
		
		\bibitem{pascher_redshift_2025}
		Pascher, J. (2025). \textit{T0-Theory: Wavelength-Dependent Redshift Mechanism}. Available at: \url{https://jpascher.github.io/T0-Time-Mass-Duality/2/pdf/redshift_deflectionEn.pdf}
		
		\bibitem{pascher_t0_energie_2025}
		Pascher, J. (2025). \textit{T0-Model: Energy-Based Formulation}. Available at: \url{https://jpascher.github.io/T0-Time-Mass-Duality/2/pdf/T0-EnergieEn.pdf}
		
		\bibitem{riess_2022}
		Riess, A. G., et al. (2022). \textit{A Comprehensive Measurement of the Local Value of the Hubble Constant}. Astrophys. J. Lett. 934, L7.
		
		\bibitem{planck_2020}
		Planck Collaboration (2020). \textit{Planck 2018 results. VI. Cosmological parameters}. Astron. Astrophys. 641, A6.
		
		\bibitem{wong_2020}
		Wong, K. C., et al. (2020). \textit{H0LiCOW measurement of H0 from lensed quasars}. Mon. Not. R. Astron. Soc. 498, 1420.
		
	\end{thebibliography}
	
\end{document}