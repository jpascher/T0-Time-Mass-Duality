% ============================================================
% T0_Preamble_De.tex - Einheitliche Präambel für deutsche Dokumente
% T0 Zeit-Masse-Dualität Framework
% Autor: Johann Pascher
% ============================================================

% Dokumentenklasse (muss vor \input dieser Datei gesetzt werden)
% \documentclass[12pt,a4paper]{article}

% ============================================================
% GRUNDLEGENDE PAKETE
% ============================================================

% Eingabe und Schriftcodierung
\usepackage[utf8]{inputenc}
\usepackage[T1]{fontenc}
\usepackage{lmodern}

% Deutsche Sprache
\usepackage[ngerman]{babel}

% ============================================================
% SEITENGEOMETRIE
% ============================================================

\usepackage{geometry}
\geometry{a4paper, margin=2.5cm}

% ============================================================
% MATHEMATIK-PAKETE
% ============================================================

\usepackage{amsmath}
\usepackage{amssymb}
\usepackage{amsthm}
\usepackage{mathtools}
\usepackage{physics}

% ============================================================
% GRAFIK UND FARBEN
% ============================================================

\usepackage{graphicx}
\usepackage{float}
\usepackage[table,xcdraw]{xcolor}
\usepackage{tikz}
\usetikzlibrary{positioning,shapes,arrows,arrows.meta}

% Standardfarben definieren
\definecolor{deepblue}{RGB}{0,0,127}
\definecolor{deepred}{RGB}{191,0,0}
\definecolor{deepgreen}{RGB}{0,127,0}

% ============================================================
% TABELLEN UND LISTEN
% ============================================================

\usepackage{booktabs}
\usepackage{array}
\usepackage{longtable}
\usepackage{enumitem}

% ============================================================
% HYPERLINKS
% ============================================================

\usepackage{hyperref}
\hypersetup{
	colorlinks=true,
	linkcolor=blue,
	citecolor=blue,
	urlcolor=blue,
	pdftitle={T0-Theorie Dokument},
	pdfauthor={Johann Pascher},
	pdfsubject={T0 Zeit-Masse-Dualität Framework}
}

% ============================================================
% KOPF- UND FUSSZEILEN
% ============================================================

\usepackage{fancyhdr}
\pagestyle{fancy}
\fancyhf{}
\fancyhead[L]{\textsc{T0-Theorie}}
\fancyhead[R]{\textsc{Johann Pascher}}
\fancyfoot[C]{\thepage}
\renewcommand{\headrulewidth}{0.4pt}
\renewcommand{\footrulewidth}{0.4pt}
\setlength{\headheight}{14.5pt}

% ============================================================
% TCOLORBOX FÜR HERVORGEHOBENE BEREICHE
% ============================================================

\usepackage{tcolorbox}
\tcbuselibrary{theorems,skins,breakable}

% Standardisierte Box-Umgebungen
\newtcolorbox{keyresult}[1][Schlüsselergebnis]{
	colback=blue!5!white,
	colframe=blue!75!black,
	fonttitle=\bfseries,
	title=#1,
	breakable
}

\newtcolorbox{warning}[1][Wichtiger Hinweis]{
	colback=red!5!white,
	colframe=red!75!black,
	fonttitle=\bfseries,
	title=#1,
	breakable
}

\newtcolorbox{formula}[1][Formel]{
	colback=blue!5!white,
	colframe=blue!75!black,
	fonttitle=\bfseries,
	title=#1,
	breakable
}

\newtcolorbox{result}[1][Ergebnis]{
	colback=green!5!white,
	colframe=green!75!black,
	fonttitle=\bfseries,
	title=#1,
	breakable
}

\newtcolorbox{summary}[1][Zusammenfassung]{
	colback=yellow!5!white,
	colframe=orange!75!black,
	fonttitle=\bfseries,
	title=#1,
	breakable
}

\newtcolorbox{important}[1][Wichtig]{
	colback=green!5!white,
	colframe=green!35!black,
	fonttitle=\bfseries,
	title=#1,
	breakable
}

% ============================================================
% THEOREM-UMGEBUNGEN
% ============================================================

\theoremstyle{definition}
\newtheorem{definition}{Definition}[section]
\newtheorem{theorem}{Theorem}[section]
\newtheorem{lemma}{Lemma}[section]
\newtheorem{corollary}{Korollar}[section]

% ============================================================
% TYPOGRAFISCHE EINSTELLUNGEN
% ============================================================

\usepackage{microtype}
\emergencystretch=2em
\tolerance=2000
\hyphenpenalty=500

% ============================================================
% ZUSÄTZLICHE NÜTZLICHE PAKETE
% ============================================================

\usepackage{siunitx}
\sisetup{
	locale=DE,
	per-mode=fraction,
	separate-uncertainty=true
}

% ============================================================
% BENUTZERDEFINIERTE BEFEHLE FÜR T0-THEORIE
% ============================================================

% Grundlegende T0 Parameter
\newcommand{\xipar}{\xi}
\newcommand{\Tfield}{T(x,t)}
\newcommand{\Efield}{E(x,t)}
\newcommand{\betaT}{\beta_{T}}
\newcommand{\alphaEM}{\alpha_{\text{EM}}}

% Planck-Einheiten
\newcommand{\EP}{E_{\text{P}}}
\newcommand{\lP}{\ell_{\text{P}}}
\newcommand{\tP}{t_{\text{P}}}
\newcommand{\mP}{m_{\text{P}}}
\newcommand{\Tzero}{T_0}

% Kosmologische Parameter
\newcommand{\LCDM}{\Lambda\text{CDM}}
\newcommand{\OmegaLambda}{\Omega_{\Lambda}}
\newcommand{\OmegaDM}{\Omega_{\text{DM}}}

% Einheiten-Kurzformen
\newcommand{\GeV}{\,\text{GeV}}
\newcommand{\MeV}{\,\text{MeV}}
\newcommand{\eV}{\,\text{eV}}
\newcommand{\natunits}{\text{(nat. Einh.)}}

% ============================================================
% ENDE DER PRÄAMBEL
% ============================================================
