\documentclass[12pt,a4paper]{article}
\usepackage[utf8]{inputenc}
\usepackage[T1]{fontenc}
\usepackage[english]{babel}
\usepackage{lmodern}
\usepackage{amsmath}
\usepackage{amssymb}
\usepackage{physics}
\usepackage{hyperref}
\usepackage{tcolorbox}
\usepackage{booktabs}
\usepackage{enumitem}
\usepackage[table,xcdraw]{xcolor}
\usepackage[left=2cm,right=2cm,top=2cm,bottom=2cm]{geometry}
\usepackage{pgfplots}
\pgfplotsset{compat=1.18}
\usepackage{graphicx}
\usepackage{float}
\usepackage{fancyhdr}
\usepackage{siunitx}
\usepackage{array}
\usepackage{cleveref}
\usepackage{mathtools}

% Headers and Footers
\pagestyle{fancy}
\fancyhf{}
\fancyhead[L]{Johann Pascher}
\fancyhead[R]{Unified Framework in T0 Model - Updated}
\fancyfoot[C]{\thepage}
\renewcommand{\headrulewidth}{0.4pt}
\renewcommand{\footrulewidth}{0.4pt}

% Custom commands - Updated for T0 model consistency
\newcommand{\Tfield}{T(x,t)}
\newcommand{\Tfieldt}{T(x,t)}
\newcommand{\alphaEM}{\alpha_{\text{EM}}}
\newcommand{\alphaW}{\alpha_{\text{W}}}
\newcommand{\betaT}{\beta_{\text{T}}}
\newcommand{\Mpl}{M_{\text{Pl}}}
\newcommand{\Tzerot}{T_0(\Tfieldt)}
\newcommand{\Tzero}{T_0}
\newcommand{\vecx}{\vec{x}}
\newcommand{\gammaf}{\gamma_{\text{Lorentz}}}
\newcommand{\DhiggsT}{\Tfieldt (\partial_\mu + ig A_\mu) \Phi + \Phi \partial_\mu \Tfieldt}
\newcommand{\DhiggsTt}{\Tfieldt (\partial_\mu + ig A_\mu) \Phi + \Phi \partial_\mu \Tfieldt}
\newcommand{\LCDM}{\Lambda\text{CDM}}
\newcommand{\DTmu}{D_{T,\mu}}
\newcommand{\calL}{\mathcal{L}}
\newcommand{\deq}{\displaystyle}
\newcommand{\e}{\mathrm{e}}
\newcommand{\dTdt}{\frac{d\Tfieldt}{dt}}
\newcommand{\pdTdt}{\frac{\partial\Tfieldt}{\partial t}}
\newcommand{\pdTdx}{\nabla\Tfieldt}
\newcommand{\xipar}{\xi}
\newcommand{\lP}{\ell_{\text{P}}}

\hypersetup{
	colorlinks=true,
	linkcolor=blue,
	citecolor=blue,
	urlcolor=blue,
	pdftitle={The Emerging Unified Framework: Relationships Between Fundamental Fields in the T0 Model - Updated Framework},
	pdfauthor={Johann Pascher},
	pdfsubject={Theoretical Physics},
	pdfkeywords={T0 Model, Unified Framework, Natural Units, Field Theory}
}

\begin{document}
	
	\title{The Emerging Unified Framework:\\Relationships Between Fundamental Fields in the T0 Model\\Updated with Complete Geometric Foundations}
	\author{Johann Pascher\\
		Department of Communications Engineering, \\Höhere Technische Bundeslehranstalt (HTL), Leonding, Austria\\
		\texttt{johann.pascher@gmail.com}}
	\date{\today}
	
	\maketitle
	
	\begin{abstract}
		This updated paper explores the profound relationships between fundamental fields—the Higgs field, the vacuum with its electromagnetic constants, and the intrinsic time field—within the comprehensive T0 model framework established through complete field-theoretic derivation. Building upon the natural units system where $\hbar = c = \alpha_{\text{EM}} = \beta_{\text{T}} = 1$, we demonstrate how these connections emerge from the three fundamental field geometries: localized spherical, localized non-spherical, and infinite homogeneous. The mathematical relationship $\xi = 2\sqrt{G} \cdot m$ between the T0 scale parameter and fundamental constants, combined with the connection to Higgs physics through $\beta_T = \lambda_h^2 v^2/(16\pi^3 m_h^2 \xi) = 1$, provides quantitative bridges between particle physics and gravitational phenomena. The coupled Lagrangian density directly connects the Higgs field and intrinsic time field through dimensionally consistent formulations. This framework represents an evolving unification that systematically integrates diverse physical concepts while maintaining parameter-free predictions and complete dimensional consistency. All formulations are derived from the fundamental field equation $\nabla^2 m = 4\pi G \rho m$ and its geometric solutions across different field configurations.
	\end{abstract}
	
	\newpage
	\tableofcontents
	\newpage
	
	\section{Introduction: Updated T0 Unified Framework}
	\label{sec:introduction}
	
	This updated analysis builds upon the comprehensive T0 model framework established through complete field-theoretic derivation, revealing the profound relationships between fundamental fields within a mathematically rigorous and dimensionally consistent structure. The unified framework emerges naturally from the three fundamental field geometries and their parameter modifications.
	
	The investigation of field relationships within the T0 model reveals a striking pattern: seemingly distinct physical concepts manifest as interconnected aspects of the same underlying field-theoretic reality. This pattern extends beyond mere mathematical similarity to represent genuine unification through the intrinsic time field $\Tfieldt$.
	
	\subsection{T0 Model Foundation for Unification}
	\label{subsec:t0_foundation}
	
	The T0 model provides the foundation for unification through:
	
	\begin{enumerate}
		\item \textbf{Fundamental field equation}: $\nabla^2 m = 4\pi G \rho m$ with complete geometric solutions
		\item \textbf{Three field geometries}: Each with specific parameter modifications and physical applications
		\item \textbf{Natural units system}: $\hbar = c = \alpha_{\text{EM}} = \beta_{\text{T}} = 1$ through deep theoretical connections
		\item \textbf{Parameter-free structure}: All constants derived from first principles without adjustable parameters
		\item \textbf{Dimensional consistency}: Complete verification across all formulations
	\end{enumerate}
	
	\subsection{Unified Field Relationships}
	\label{subsec:unified_relationships}
	
	Within this framework, three fundamental concepts reveal deep connections:
	
	\begin{tcolorbox}[colback=blue!5!white,colframe=blue!75!black,title=T0 Unified Framework Components]
		\begin{itemize}
			\item \textbf{Intrinsic Time Field}: $\Tfieldt = 1/\max(m(x,t), \omega(x,t))$ as the fundamental entity
			\item \textbf{Higgs Field}: Mass generation mechanism through $\Tfieldt = 1/(y\langle\Phi\rangle)$
			\item \textbf{Vacuum Structure}: Electromagnetic constants unified through $\alpha_{\text{EM}} = \beta_{\text{T}} = 1$
		\end{itemize}
	\end{tcolorbox}
	
	\section{Complete Natural Units Framework}
	\label{sec:complete_natural_units}
	
	\subsection{Field-Theoretic Foundation of Unity Values}
	\label{subsec:field_theoretic_foundation}
	
	The natural units system where fundamental constants equal unity emerges from the complete field-theoretic derivation:
	
	\begin{equation}
		\beta_{\text{T}} = \frac{\lambda_h^2 v^2}{16\pi^3 m_h^2 \xi} = 1
		\label{eq:beta_t_derivation}
	\end{equation}
	
	where:
	\begin{itemize}
		\item $\lambda_h \approx 0.13$ (Higgs self-coupling)
		\item $v \approx 246$ GeV (Higgs VEV)
		\item $m_h \approx 125$ GeV (Higgs mass)
		\item $\xi = 2\sqrt{G} \cdot m$ (T0 scale parameter)
	\end{itemize}
	
	\textbf{Dimensional verification}: $[\lambda_h^2 v^2] = [1][E^2] = [E^2]$ and $[16\pi^3 m_h^2 \xi] = [1][E^2][1] = [E^2]$ $\Rightarrow$ $[\beta_T] = [1]$ \checkmark
	
	\subsection{Electromagnetic Coupling Unification}
	\label{subsec:em_coupling_unification}
	
	The electromagnetic coupling unity follows from the shared field-theoretic foundation:
	\begin{equation}
		\alpha_{\text{EM}} = \beta_{\text{T}} = 1
		\label{eq:coupling_unity}
	\end{equation}
	
	This unity reflects the deep connection between electromagnetic phenomena and the intrinsic time field through the vacuum structure.
	
	\subsection{Scale Parameter Network}
	\label{subsec:scale_parameter_network}
	
	The T0 scale parameters form an interconnected network:
	
	\begin{align}
		\xi &= 2\sqrt{G} \cdot m = \frac{2Gm}{\sqrt{G}} = \frac{r_0}{\lP} \\
		\beta &= \frac{2Gm}{r} = \frac{\xi \lP}{r} \\
		\xi_{\text{eff}} &= \frac{\xi}{2} = \sqrt{G} \cdot m \quad \text{(cosmic screening)}
		\label{eq:parameter_network}
	\end{align}
	
	These relationships connect Planck scale physics to macroscopic gravitational effects.
	
	\section{Higgs Field and Time Field Integration}
	\label{sec:higgs_time_integration}
	
	\subsection{Direct Coupling Through Modified Lagrangian}
	\label{subsec:direct_coupling}
	
	The T0 model reveals direct coupling between the Higgs field and intrinsic time field through the modified Lagrangian density:
	
	\begin{equation}
		\mathcal{L}_{\text{Higgs-T}} = |\DhiggsTt|^2 - V(\Phi, \Tfieldt)
		\label{eq:higgs_time_lagrangian}
	\end{equation}
	
	where the modified covariant derivative is:
	\begin{equation}
		\DhiggsTt = \Tfieldt (\partial_\mu + ig A_\mu) \Phi + \Phi \partial_\mu \Tfieldt
		\label{eq:modified_covariant_derivative}
	\end{equation}
	
	\textbf{Dimensional verification}:
	\begin{itemize}
		\item $[\Tfieldt (\partial_\mu + ig A_\mu) \Phi] = [E^{-1}]([E] + [1][E])[E^2] = [E^2]$
		\item $[\Phi \partial_\mu \Tfieldt] = [E^2][E][E^{-1}] = [E^2]$
		\item $[|\DhiggsTt|^2] = [E^4]$ (energy density dimension) \checkmark
	\end{itemize}
	
	\subsection{Time-Higgs Duality Relationship}
	\label{subsec:time_higgs_duality}
	
	The fundamental duality relationship connects the fields:
	\begin{equation}
		\Tfieldt = \frac{1}{y\langle\Phi\rangle}
		\label{eq:time_higgs_duality}
	\end{equation}
	
	where $y$ is the Yukawa coupling and $\langle\Phi\rangle$ is the Higgs VEV.
	
	This establishes:
	\begin{equation}
		\Tfieldt \cdot m_{\text{particle}} = \frac{1}{y\langle\Phi\rangle} \cdot y\langle\Phi\rangle = 1
		\label{eq:time_mass_unity}
	\end{equation}
	
	demonstrating the fundamental time-mass duality principle.
	
	\subsection{Higgs Potential Modification}
	\label{subsec:higgs_potential_modification}
	
	The potential becomes field-dependent:
	\begin{equation}
		V(\Phi, \Tfieldt) = \lambda(\Tfieldt)[\Phi^{\dagger}\Phi - v^2(\Tfieldt)]^2
		\label{eq:modified_higgs_potential}
	\end{equation}
	
	where both the coupling $\lambda(\Tfieldt)$ and VEV $v(\Tfieldt)$ depend on the local time field configuration.
	
	\section{Vacuum Structure and Field Geometry}
	\label{sec:vacuum_structure}
	
	\subsection{Vacuum as Time Field Configuration}
	\label{subsec:vacuum_time_field}
	
	The vacuum in the T0 model represents a specific configuration of the intrinsic time field:
	
	\begin{equation}
		\Tfieldt_{\text{vacuum}} = \Tzero = \text{constant}
		\label{eq:vacuum_time_field}
	\end{equation}
	
	This vacuum state determines the electromagnetic constants:
	\begin{align}
		\varepsilon_0 &= \frac{1}{4\pi\alpha_{\text{EM}}} = \frac{1}{4\pi} \quad \text{(with } \alpha_{\text{EM}} = 1\text{)} \\
		\mu_0 &= 4\pi \quad \text{(natural units)} \\
		c &= \frac{1}{\sqrt{\varepsilon_0\mu_0}} = 1 \quad \text{(natural units)}
		\label{eq:vacuum_constants}
	\end{align}
	
	\subsection{Field Geometry Effects on Vacuum}
	\label{subsec:geometry_vacuum_effects}
	
	Different field geometries modify the vacuum structure:
	
	\subsubsection{Localized Fields}
	\label{subsubsec:localized_vacuum}
	
	Near massive objects, the vacuum structure is modified:
	\begin{equation}
		\Tfieldt(r) = \Tzero(1 - \beta) = \Tzero\left(1 - \frac{2Gm}{r}\right)
		\label{eq:localized_vacuum_modification}
	\end{equation}
	
	This leads to position-dependent vacuum properties and modified electromagnetic propagation.
	
	\subsubsection{Infinite Homogeneous Fields}
	\label{subsubsec:infinite_vacuum}
	
	In cosmological contexts with cosmic screening:
	\begin{equation}
		\xi_{\text{eff}} = \frac{\xi}{2} = \sqrt{G} \cdot m
		\label{eq:cosmic_vacuum_screening}
	\end{equation}
	
	The vacuum experiences cosmic screening effects that modify its large-scale properties.
	
	\section{Mathematical Connections and Quantitative Relationships}
	\label{sec:mathematical_connections}
	
	\subsection{Fundamental Parameter Relationships}
	\label{subsec:parameter_relationships}
	
	The T0 model establishes precise quantitative relationships:
	
	\begin{equation}
		\xi = 2\sqrt{G} \cdot m = \frac{\lambda_h^2 v^2}{16\pi^3 m_h^2} \approx 1.33 \times 10^{-4}
		\label{eq:xi_higgs_connection}
	\end{equation}
	
	This connects:
	\begin{itemize}
		\item Gravitational physics through $G$ and $m$
		\item Particle physics through $\lambda_h$, $v$, and $m_h$
		\item Scale hierarchy through $\xi = r_0/\lP$
	\end{itemize}
	
	\subsection{Coupling Constant Network}
	\label{subsec:coupling_network}
	
	The unified coupling structure reveals:
	\begin{align}
		\alpha_{\text{EM}} &= 1 \quad \text{(electromagnetic)} \\
		\beta_{\text{T}} &= 1 \quad \text{(time field)} \\
		\lambda_h &\approx 0.13 \quad \text{(Higgs self-coupling)} \\
		\xi &= \frac{\lambda_h^2 v^2}{16\pi^3 m_h^2} \quad \text{(scale parameter)}
		\label{eq:coupling_network}
	\end{align}
	
	These relationships are not empirical fits but theoretical predictions from the T0 field equations.
	
	\subsection{Dimensional Consistency Verification}
	\label{subsec:dimensional_verification}
	
	\begin{table}[htbp]
		\centering
		\begin{tabular}{lccl}
			\toprule
			\textbf{Relationship} & \textbf{Left Side} & \textbf{Right Side} & \textbf{Status} \\
			\midrule
			Time-Higgs duality & $[\Tfieldt] = [E^{-1}]$ & $[1/(y\langle\Phi\rangle)] = [E^{-1}]$ & \checkmark \\
			$\xi$ parameter & $[\xi] = [1]$ & $[2\sqrt{G} \cdot m] = [1]$ & \checkmark \\
			$\beta_T$ formula & $[\beta_T] = [1]$ & $[\lambda_h^2 v^2/(16\pi^3 m_h^2 \xi)] = [1]$ & \checkmark \\
			Modified derivative & $[|\DhiggsTt|^2] = [E^4]$ & $[|\text{kinetic terms}|^2] = [E^4]$ & \checkmark \\
			Vacuum constants & $[\varepsilon_0 \mu_0] = [1]$ & $[1/c^2] = [1]$ & \checkmark \\
			\bottomrule
		\end{tabular}
		\caption{Dimensional consistency verification for unified framework relationships}
	\end{table}
	
	\section{Field Equations for Unified Framework}
	\label{sec:unified_field_equations}
	
	\subsection{Coupled Field Evolution}
	\label{subsec:coupled_evolution}
	
	The unified framework leads to coupled field equations:
	
	\textbf{For the time field}:
	\begin{equation}
		\nabla^2 \Tfieldt = -\frac{4\pi G \rho}{\Tfieldt^2} - \lambda_{\text{coupling}}|\Phi|^2
		\label{eq:time_field_equation}
	\end{equation}
	
	\textbf{For the Higgs field}:
	\begin{equation}
		[\Tfieldt^2 (\partial_\mu + ig A_\mu)^2 + m_h^2]\Phi - 2\lambda_h\Phi(|\Phi|^2 - v^2) = 0
		\label{eq:higgs_field_equation}
	\end{equation}
	
	\textbf{Dimensional verification}:
	\begin{itemize}
		\item Time field equation: $[\nabla^2 \Tfieldt] = [E^2][E^{-1}] = [E]$ and $[4\pi G \rho/\Tfieldt^2] = [E^{-2}][E^4]/[E^{-2}] = [E]$ \checkmark
		\item Higgs field equation: $[\Tfieldt^2 (\partial_\mu)^2 \Phi] = [E^{-2}][E^2][E^2] = [E^2]$ and $[m_h^2 \Phi] = [E^2][E^2] = [E^4]/[E^2] = [E^2]$ \checkmark
	\end{itemize}
	
	\subsection{Self-Consistent Solutions}
	\label{subsec:self_consistent_solutions}
	
	The coupled equations admit self-consistent solutions where:
	\begin{align}
		\Tfieldt(x) &= \frac{1}{\sqrt{\lambda_h}\langle\Phi(x)\rangle} \\
		\Phi(x) &= v(x) = \frac{1}{\sqrt{\lambda_h}\Tfieldt(x)}
		\label{eq:self_consistent_solutions}
	\end{align}
	
	These solutions demonstrate the deep interconnection between the fields.
	
	\section{Experimental Predictions of Unified Framework}
	\label{sec:experimental_predictions}
	
	\subsection{Field Coupling Tests}
	\label{subsec:field_coupling_tests}
	
	The unified framework predicts specific experimental signatures:
	
	\begin{enumerate}
		\item \textbf{Higgs-Gravity Coupling}:
		\begin{equation}
			\frac{\Delta m_h}{m_h} = \beta \cdot \frac{\Delta \Phi_{\text{grav}}}{\Phi_0}
			\label{eq:higgs_gravity_coupling}
		\end{equation}
		
		where $\Phi_{\text{grav}}$ is the gravitational potential.
		
		\item \textbf{Vacuum Modification in Strong Fields}:
		\begin{equation}
			\frac{\Delta \alpha_{\text{EM}}}{\alpha_{\text{EM}}} = \xi \cdot \frac{GM}{rc^2}
			\label{eq:vacuum_modification}
		\end{equation}
		
		\item \textbf{Time Field Gradients}:
		\begin{equation}
			\Delta f_{\text{atomic}} = f_0 \cdot \frac{\Delta \Tfieldt}{\Tzero}
			\label{eq:atomic_frequency_shift}
		\end{equation}
	\end{enumerate}
	
	\subsection{Precision Tests}
	\label{subsec:precision_tests}
	
	The parameter-free nature enables stringent tests:
	\begin{itemize}
		\item Atomic clock networks for time field gradient detection
		\item High-energy particle colliders for Higgs-gravity coupling
		\item Precision electromagnetic measurements in strong gravitational fields
		\item Cosmological observations for vacuum structure evolution
	\end{itemize}
	
	\section{Implications for Fundamental Physics}
	\label{sec:fundamental_implications}
	
	\subsection{Resolution of Hierarchy Problems}
	\label{subsec:hierarchy_resolution}
	
	The unified framework naturally resolves several hierarchy problems:
	
	\begin{enumerate}
		\item \textbf{Higgs Mass Hierarchy}: The relationship $\xi = \lambda_h^2 v^2/(16\pi^3 m_h^2) \approx 1.33 \times 10^{-4}$ provides a natural explanation for the Higgs mass scale relative to the Planck mass.
		
		\item \textbf{Cosmological Constant Problem}: The dynamic time field provides a mechanism for vacuum energy regulation through field-dependent vacuum properties.
		
		\item \textbf{Fine-Structure Constant Stability}: The unity $\alpha_{\text{EM}} = \beta_{\text{T}} = 1$ emerges from field-theoretic consistency rather than empirical fitting.
	\end{enumerate}
	
	\subsection{Unification Pathway}
	\label{subsec:unification_pathway}
	
	The T0 framework suggests a systematic pathway toward complete unification:
	
	\begin{enumerate}
		\item \textbf{Geometric Foundation}: Three field geometries provide comprehensive coverage
		\item \textbf{Parameter Derivation}: All constants emerge from field equations
		\item \textbf{Dimensional Consistency}: Natural units reveal fundamental relationships
		\item \textbf{Experimental Testability}: Parameter-free predictions enable decisive tests
	\end{enumerate}
	
	\section{Philosophical and Theoretical Implications}
	\label{sec:philosophical_implications}
	
	\subsection{Emergence vs. Reduction}
	\label{subsec:emergence_reduction}
	
	The unified framework demonstrates that apparent diversity in physics emerges from underlying unity:
	
	\begin{itemize}
		\item Different field manifestations arise from single fundamental field
		\item Complex phenomena emerge from simple geometric principles
		\item Multiple theoretical descriptions converge on unified foundation
	\end{itemize}
	
	\subsection{Theory Development Pattern}
	\label{subsec:theory_development}
	
	The T0 model exemplifies a particular pattern of theory development:
	
	\begin{enumerate}
		\item \textbf{Recognition of connections} between seemingly disparate phenomena
		\item \textbf{Mathematical unification} through field-theoretic frameworks
		\item \textbf{Parameter reduction} to eliminate arbitrary constants
		\item \textbf{Predictive extension} to new experimental domains
	\end{enumerate}
	
	This pattern suggests general principles for theoretical physics development.
	
	\section{Conclusions and Future Directions}
	\label{sec:conclusions}
	
	\subsection{Summary of Unified Framework}
	\label{subsec:summary}
	
	This updated analysis demonstrates that the T0 model provides a comprehensive unified framework where:
	
	\begin{enumerate}
		\item \textbf{Field Integration}: Higgs field, vacuum structure, and time field represent aspects of unified field theory
		\item \textbf{Mathematical Consistency}: All relationships maintain dimensional consistency and parameter-free structure
		\item \textbf{Geometric Foundation}: Three field geometries provide complete theoretical coverage
		\item \textbf{Experimental Testability}: Specific predictions distinguish T0 from conventional approaches
		\item \textbf{Conceptual Coherence}: Unified perspective resolves longstanding theoretical problems
	\end{enumerate}
	
	\subsection{Key Theoretical Achievements}
	\label{subsec:key_achievements}
	
	\begin{tcolorbox}[colback=green!5!white,colframe=green!75!black,title=T0 Unified Framework: Core Results]
		\begin{itemize}
			\item \textbf{Field Unification}: $\Tfieldt = 1/\max(m, \omega)$ as fundamental entity
			\item \textbf{Parameter Network}: $\xi = 2\sqrt{G} \cdot m$, $\beta = 2Gm/r$, $\beta_T = 1$
			\item \textbf{Higgs Connection}: $\Tfieldt = 1/(y\langle\Phi\rangle)$ establishing time-mass duality
			\item \textbf{Vacuum Structure}: $\alpha_{\text{EM}} = \beta_T = 1$ through field-theoretic unity
			\item \textbf{Geometric Modifications}: Three field geometries with specific parameter changes
		\end{itemize}
	\end{tcolorbox}
	
	\subsection{Future Research Directions}
	\label{subsec:future_directions}
	
	\begin{enumerate}
		\item \textbf{Non-Abelian Extensions}: Integration of weak and strong interactions
		\item \textbf{Quantum Gravity}: Full quantization of the unified field framework
		\item \textbf{Cosmological Applications}: Structure formation in unified field cosmology
		\item \textbf{Experimental Programs}: Design of definitive tests for field unification
		\item \textbf{Mathematical Developments}: Higher-order corrections and field dynamics
	\end{enumerate}
	
	The T0 unified framework represents a significant step toward complete theoretical unification, demonstrating that diverse physical phenomena can be understood as manifestations of a single underlying field-theoretic reality. The parameter-free nature and dimensional consistency provide a robust foundation for continued development and experimental validation.
	
	\begin{thebibliography}{99}
		
		\bibitem{pascher_derivation_beta_2025} 
		Pascher, J. (2025). \href{https://github.com/jpascher/T0-Time-Mass-Duality/blob/main/2/pdf/DerivationVonBetaEn.pdf}{\textit{Field-Theoretic Derivation of the $\beta_T$ Parameter in Natural Units ($\hbar = c = 1$)}}. GitHub Repository: T0-Time-Mass-Duality.
		
		\bibitem{pascher_dirac_integration_2025}
		Pascher, J. (2025). \href{https://github.com/jpascher/T0-Time-Mass-Duality/blob/main/2/pdf/diracEn.pdf}{\textit{Integration of the Dirac Equation in the T0 Model: Updated Framework}}. GitHub Repository: T0-Time-Mass-Duality.
		
		\bibitem{pascher_math_zeit_masse_2025}
		Pascher, J. (2025). \href{https://github.com/jpascher/T0-Time-Mass-Duality/tree/main/2/pdf/MathZeitMasseLagrangeEn.pdf}{\textit{Mathematical Core Formulations of Time-Mass Duality Theory: Updated Framework}}. GitHub Repository: T0-Time-Mass-Duality.
		
		\bibitem{pascher_dynamic_photons_2025}
		Pascher, J. (2025). \href{https://github.com/jpascher/T0-Time-Mass-Duality/blob/main/2/pdf/DynMassePhotonenNichtlokalEn.pdf}{\textit{Dynamic Mass of Photons and Nonlocality in T0 Model: Updated Framework}}. GitHub Repository: T0-Time-Mass-Duality.
		
		
		\bibitem{Weinberg1989} 
		S. Weinberg, \textit{The Cosmological Constant Problem}, Rev. Mod. Phys. \textbf{61}, 1 (1989).
		
		\bibitem{Dirac1938} 
		P. A. M. Dirac, \textit{A New Basis for Cosmology}, Proc. Roy. Soc. London A \textbf{165}, 199 (1938).
		
		\bibitem{Kuhn1962} 
		T. S. Kuhn, \textit{The Structure of Scientific Revolutions}, University of Chicago Press (1962).
		
		\bibitem{Feyerabend1975} 
		P. Feyerabend, \textit{Against Method: Outline of an Anarchistic Theory of Knowledge}, New Left Books (1975).
		
	\end{thebibliography}
	
\end{document}