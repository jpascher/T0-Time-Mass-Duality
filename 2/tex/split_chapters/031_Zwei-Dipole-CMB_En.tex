% Kindle-compatible chapter format
% \documentclass[12pt,a4paper]{article}
% % ==============================================================================
% T0 Theory: Standardized English Preamble
% Version: 1.0
% Author: Johann Pascher
% ==============================================================================
% This file contains all necessary packages and definitions for English
% T0 Theory documents. Use % ==============================================================================
% T0 Theory: Standardized English Preamble
% Version: 1.0
% Author: Johann Pascher
% ==============================================================================
% This file contains all necessary packages and definitions for English
% T0 Theory documents. Use % ==============================================================================
% T0 Theory: Standardized English Preamble
% Version: 1.0
% Author: Johann Pascher
% ==============================================================================
% This file contains all necessary packages and definitions for English
% T0 Theory documents. Use \input{T0_preamble_En} after \documentclass.
% ==============================================================================

% --- Encoding and Language ---
\usepackage[utf8]{inputenc}
\usepackage[T1]{fontenc}
\usepackage[english]{babel}
\usepackage{lmodern}

% --- Page Geometry ---
\usepackage[a4paper, margin=2.5cm]{geometry}
\setlength{\headheight}{15pt}

% --- Mathematics and Physics ---
\usepackage{amsmath,amssymb,amsfonts,amsthm}
\usepackage{mathtools}
\usepackage{physics}
\usepackage{siunitx}
\sisetup{
    locale=US,
    group-separator={,},
    output-decimal-marker={.},
    per-mode=symbol
}

% --- Graphics and Tables ---
\usepackage{graphicx}
\usepackage[table,xcdraw]{xcolor}
\usepackage{tikz}
\usetikzlibrary{arrows.meta,positioning,shapes.geometric,decorations.pathmorphing,patterns,shapes.arrows,intersections}
\usepackage{pgfplots}
\pgfplotsset{compat=1.18}
\usepackage{tcolorbox}
\usepackage{booktabs}
\usepackage{array}
\usepackage{longtable}
\usepackage{float}
\usepackage{adjustbox}
\usepackage{tabularx}
\usepackage{multirow}

% --- Document Formatting ---
\usepackage{fancyhdr}
\renewcommand{\headrulewidth}{0.4pt}
\renewcommand{\footrulewidth}{0.4pt}
\usepackage{tocloft}
\usepackage{hyperref}
\usepackage{bookmark}
\usepackage{cleveref}
\usepackage{microtype}
\usepackage{enumitem}
\usepackage{setspace}
\usepackage{ragged2e}
\usepackage{multicol}

% --- Code and Algorithms ---
\usepackage{algorithm}
\usepackage{algorithmic}
\usepackage{listings}
\usepackage{mdframed}

% --- Additional Packages ---
\usepackage{pdflscape}
\usepackage{braket}
\usepackage{cancel}
\usepackage{caption}
\usepackage{csquotes}
\usepackage{gensymb}
\usepackage{hyphenat}
\usepackage{textcomp}
\usepackage{textgreek}
\usepackage{upgreek}
\usepackage{url}
\usepackage{slashed}
\usepackage{bm}

% --- Column Types ---
\newcolumntype{L}[1]{>{\raggedright\arraybackslash}p{#1}}
\newcolumntype{C}[1]{>{\centering\arraybackslash}p{#1}}

% --- Unicode Characters ---
\usepackage{newunicodechar}
\newunicodechar{ħ}{$\hbar$}
\newunicodechar{↔}{$\leftrightarrow$}
\newunicodechar{⇐}{$\Leftarrow$}
\newunicodechar{⇒}{$\Rightarrow$}
\newunicodechar{⇔}{$\Leftrightarrow$}
\newunicodechar{∂}{$\partial$}
\newunicodechar{∅}{$\emptyset$}
\newunicodechar{∇}{$\nabla$}
\newunicodechar{∈}{$\in$}
\newunicodechar{∉}{$\notin$}
\newunicodechar{∏}{$\prod$}
\newunicodechar{∑}{$\sum$}
\newunicodechar{√}{$\sqrt{}$}
\newunicodechar{∝}{$\propto$}
\newunicodechar{∞}{$\infty$}
\newunicodechar{∩}{$\cap$}
\newunicodechar{∪}{$\cup$}
\newunicodechar{∫}{$\int$}
\newunicodechar{≈}{$\approx$}
\newunicodechar{≠}{$\neq$}
\newunicodechar{≤}{$\leq$}
\newunicodechar{≥}{$\geq$}
\newunicodechar{ξ}{\ensuremath{\xi}}
\newunicodechar{μ}{\ensuremath{\mu}}
\newunicodechar{ψ}{\ensuremath{\psi}}
\newunicodechar{φ}{\ensuremath{\phi}}
\newunicodechar{π}{\ensuremath{\pi}}
\newunicodechar{λ}{\ensuremath{\lambda}}
\newunicodechar{Δ}{\ensuremath{\Delta}}

% --- Colors ---
\definecolor{blue}{rgb}{0,0,1}
\definecolor{boxgray}{RGB}{240,240,240}
\definecolor{deepblue}{RGB}{0,0,127}
\definecolor{deepgreen}{RGB}{0,127,0}
\definecolor{deepred}{RGB}{191,0,0}
\definecolor{t0blue}{RGB}{33,150,243}
\definecolor{t0green}{RGB}{76,175,80}
\definecolor{t0orange}{RGB}{255,152,0}
\definecolor{t0purple}{RGB}{156,39,176}
\definecolor{t0red}{RGB}{244,67,54}
\definecolor{t0yellow}{RGB}{255,204,0}

% --- Hyperref Settings ---
\hypersetup{
    colorlinks=true,
    linkcolor=blue,
    citecolor=blue,
    urlcolor=blue,
    breaklinks=true,
    bookmarksnumbered=true,
    pdfstartview=FitH
}

% --- Theorem Environments (English) ---
\theoremstyle{plain}
\newtheorem{theorem}{Theorem}[section]
\newtheorem{lemma}[theorem]{Lemma}
\newtheorem{proposition}[theorem]{Proposition}
\newtheorem{corollary}[theorem]{Corollary}

\theoremstyle{definition}
\newtheorem{definition}[theorem]{Definition}
\newtheorem{example}[theorem]{Example}
\newtheorem{insight}[theorem]{Insight}
\newtheorem{discovery}[theorem]{Discovery}

\theoremstyle{remark}
\newtheorem{remark}[theorem]{Remark}
\newtheorem{warning}[theorem]{Warning}
\newtheorem{axiom}{Axiom}
\newtheorem{principle}{Principle}

% --- T0-Specific Commands ---
\newcommand{\Tfield}{T(x,t)}
\newcommand{\Efield}{E(x,t)}
\newcommand{\mfield}{m(x,t)}
\newcommand{\Lag}{\mathcal{L}}
\newcommand{\calL}{\mathcal{L}}
\newcommand{\alphaem}{\alpha}
\newcommand{\betaT}{\beta_T}
\newcommand{\xiT}{\xi}
\newcommand{\xipar}{\xi}
\newcommand{\Ezero}{E_0}
\newcommand{\EPlanck}{E_{\text{Pl}}}
\newcommand{\Mpl}{M_{\text{Pl}}}
\newcommand{\lP}{\ell_{\text{P}}}
\newcommand{\tP}{t_{\text{P}}}
\newcommand{\LPlanck}{\ell_{\text{Pl}}}
\newcommand{\TPlanck}{t_{\text{Pl}}}
\newcommand{\Gnat}{G_{\text{nat}}}
\newcommand{\alphaEM}{\alpha_{\text{EM}}}
\newcommand{\alphaSI}{\alpha_{\text{SI}}}
\newcommand{\Hubble}{H_0}
\newcommand{\LCDM}{\Lambda\text{CDM}}
\newcommand{\natunits}{(nat. units)}

% T0 Model Parameters
\newcommand{\xigeom}{\xi_{\mathrm{geom}}}
\newcommand{\rzero}{r_{0}}
\newcommand{\xirat}{\xi_{\mathrm{rat}}}
\newcommand{\tzero}{t_{0}}
\newcommand{\Lambdat}{\Lambda_{\mathrm{t}}}
\newcommand{\EP}{E_{\mathrm{P}}}
\newcommand{\Emu}{E_{\mu}}
\newcommand{\Ee}{E_{e}}
\newcommand{\Etau}{E_{\tau}}
\newcommand{\alphafine}{\alpha_{\mathrm{fine}}}
\newcommand{\alphal}{\alpha_{\ell}}

% Additional Commands
\newcommand{\Kfrak}{K_{\text{frak}}}
\newcommand{\Dfrak}{D_{\text{frak}}}
\newcommand{\betapar}{\beta_T}
\newcommand{\alphapar}{\alpha}
\newcommand{\deltafield}{\delta \phi}
\newcommand{\deltam}{\delta m}
\newcommand{\deltaE}{\delta E}
\newcommand{\Exi}{E_{\xi}}
\newcommand{\Lxi}{\ell_{\xi}}
\newcommand{\rhoCMB}{\rho_{\text{CMB}}}
\newcommand{\rhoCasimir}{\rho_{\text{Casimir}}}
\newcommand{\Leff}{L_{\text{eff}}}
\newcommand{\CQCD}{C_{\mathrm{QCD}}}
\newcommand{\Kspec}{K_{\mathrm{spec}}}

% --- tcolorbox Styles ---
\tcbset{
    keyresult/.style={
        colback=blue!5!white,
        colframe=blue!75!black,
        title=Key Result,
        fonttitle=\bfseries
    },
    foundation/.style={
        colback=green!5!white,
        colframe=green!75!black,
        title=Foundation,
        fonttitle=\bfseries
    },
    alternative/.style={
        colback=orange!5!white,
        colframe=orange!75!black,
        title=Alternative,
        fonttitle=\bfseries
    },
    warningbox/.style={
        colback=red!5!white,
        colframe=red!75!black,
        title=Warning,
        fonttitle=\bfseries
    }
}

\newtcolorbox{keyresultbox}[1][]{keyresult, #1}
\newtcolorbox{foundationbox}[1][]{foundation, #1}
\newtcolorbox{alternativebox}[1][]{alternative, #1}
\newtcolorbox{warningboxenv}[1][]{warningbox, #1}

% Custom boxes for formulas
\newtcolorbox{fundamental}[1][]{
    colback=boxgray,
    colframe=t0blue,
    fonttitle=\bfseries,
    title=#1,
    sharp corners,
    boxrule=2pt
}

\newtcolorbox{newperspective}[1][]{
    colback=red!5!white,
    colframe=t0red,
    fonttitle=\bfseries,
    title=#1,
    sharp corners,
    boxrule=2pt
}

\newtcolorbox{formula}[1][]{
    colback=blue!5!white,
    colframe=blue!75!black,
    fonttitle=\bfseries,
    title=#1
}

\newtcolorbox{result}[1][]{
    colback=green!5!white,
    colframe=green!75!black,
    fonttitle=\bfseries,
    title=#1
}

% --- Layout Settings ---
\sloppy
\hfuzz=2pt
\vfuzz=2pt
\tolerance=1000
\emergencystretch=3em
\raggedbottom

% --- TOC Formatting ---
\renewcommand{\cftsecfont}{\color{blue}}
\renewcommand{\cftsubsecfont}{\color{blue}}
\renewcommand{\cftsecpagefont}{\color{blue}}
\renewcommand{\cftsubsecpagefont}{\color{blue}}
\renewcommand{\cfttoctitlefont}{\huge\bfseries\color{blue}}

% --- Default Header and Footer ---
\pagestyle{fancy}
\fancyhf{}
\fancyhead[L]{\textsc{T0 Theory}}
\fancyhead[R]{\textsc{J. Pascher}}
\fancyfoot[C]{\thepage}

% ==============================================================================
% End of Preamble
% ==============================================================================
 after \documentclass.
% ==============================================================================

% --- Encoding and Language ---
\usepackage[utf8]{inputenc}
\usepackage[T1]{fontenc}
\usepackage[english]{babel}
\usepackage{lmodern}

% --- Page Geometry ---
\usepackage[a4paper, margin=2.5cm]{geometry}
\setlength{\headheight}{15pt}

% --- Mathematics and Physics ---
\usepackage{amsmath,amssymb,amsfonts,amsthm}
\usepackage{mathtools}
\usepackage{physics}
\usepackage{siunitx}
\sisetup{
    locale=US,
    group-separator={,},
    output-decimal-marker={.},
    per-mode=symbol
}

% --- Graphics and Tables ---
\usepackage{graphicx}
\usepackage[table,xcdraw]{xcolor}
\usepackage{tikz}
\usetikzlibrary{arrows.meta,positioning,shapes.geometric,decorations.pathmorphing,patterns,shapes.arrows,intersections}
\usepackage{pgfplots}
\pgfplotsset{compat=1.18}
\usepackage{tcolorbox}
\usepackage{booktabs}
\usepackage{array}
\usepackage{longtable}
\usepackage{float}
\usepackage{adjustbox}
\usepackage{tabularx}
\usepackage{multirow}

% --- Document Formatting ---
\usepackage{fancyhdr}
\renewcommand{\headrulewidth}{0.4pt}
\renewcommand{\footrulewidth}{0.4pt}
\usepackage{tocloft}
\usepackage{hyperref}
\usepackage{bookmark}
\usepackage{cleveref}
\usepackage{microtype}
\usepackage{enumitem}
\usepackage{setspace}
\usepackage{ragged2e}
\usepackage{multicol}

% --- Code and Algorithms ---
\usepackage{algorithm}
\usepackage{algorithmic}
\usepackage{listings}
\usepackage{mdframed}

% --- Additional Packages ---
\usepackage{pdflscape}
\usepackage{braket}
\usepackage{cancel}
\usepackage{caption}
\usepackage{csquotes}
\usepackage{gensymb}
\usepackage{hyphenat}
\usepackage{textcomp}
\usepackage{textgreek}
\usepackage{upgreek}
\usepackage{url}
\usepackage{slashed}
\usepackage{bm}

% --- Column Types ---
\newcolumntype{L}[1]{>{\raggedright\arraybackslash}p{#1}}
\newcolumntype{C}[1]{>{\centering\arraybackslash}p{#1}}

% --- Unicode Characters ---
\usepackage{newunicodechar}
\newunicodechar{ħ}{$\hbar$}
\newunicodechar{↔}{$\leftrightarrow$}
\newunicodechar{⇐}{$\Leftarrow$}
\newunicodechar{⇒}{$\Rightarrow$}
\newunicodechar{⇔}{$\Leftrightarrow$}
\newunicodechar{∂}{$\partial$}
\newunicodechar{∅}{$\emptyset$}
\newunicodechar{∇}{$\nabla$}
\newunicodechar{∈}{$\in$}
\newunicodechar{∉}{$\notin$}
\newunicodechar{∏}{$\prod$}
\newunicodechar{∑}{$\sum$}
\newunicodechar{√}{$\sqrt{}$}
\newunicodechar{∝}{$\propto$}
\newunicodechar{∞}{$\infty$}
\newunicodechar{∩}{$\cap$}
\newunicodechar{∪}{$\cup$}
\newunicodechar{∫}{$\int$}
\newunicodechar{≈}{$\approx$}
\newunicodechar{≠}{$\neq$}
\newunicodechar{≤}{$\leq$}
\newunicodechar{≥}{$\geq$}
\newunicodechar{ξ}{\ensuremath{\xi}}
\newunicodechar{μ}{\ensuremath{\mu}}
\newunicodechar{ψ}{\ensuremath{\psi}}
\newunicodechar{φ}{\ensuremath{\phi}}
\newunicodechar{π}{\ensuremath{\pi}}
\newunicodechar{λ}{\ensuremath{\lambda}}
\newunicodechar{Δ}{\ensuremath{\Delta}}

% --- Colors ---
\definecolor{blue}{rgb}{0,0,1}
\definecolor{boxgray}{RGB}{240,240,240}
\definecolor{deepblue}{RGB}{0,0,127}
\definecolor{deepgreen}{RGB}{0,127,0}
\definecolor{deepred}{RGB}{191,0,0}
\definecolor{t0blue}{RGB}{33,150,243}
\definecolor{t0green}{RGB}{76,175,80}
\definecolor{t0orange}{RGB}{255,152,0}
\definecolor{t0purple}{RGB}{156,39,176}
\definecolor{t0red}{RGB}{244,67,54}
\definecolor{t0yellow}{RGB}{255,204,0}

% --- Hyperref Settings ---
\hypersetup{
    colorlinks=true,
    linkcolor=blue,
    citecolor=blue,
    urlcolor=blue,
    breaklinks=true,
    bookmarksnumbered=true,
    pdfstartview=FitH
}

% --- Theorem Environments (English) ---
\theoremstyle{plain}
\newtheorem{theorem}{Theorem}[section]
\newtheorem{lemma}[theorem]{Lemma}
\newtheorem{proposition}[theorem]{Proposition}
\newtheorem{corollary}[theorem]{Corollary}

\theoremstyle{definition}
\newtheorem{definition}[theorem]{Definition}
\newtheorem{example}[theorem]{Example}
\newtheorem{insight}[theorem]{Insight}
\newtheorem{discovery}[theorem]{Discovery}

\theoremstyle{remark}
\newtheorem{remark}[theorem]{Remark}
\newtheorem{warning}[theorem]{Warning}
\newtheorem{axiom}{Axiom}
\newtheorem{principle}{Principle}

% --- T0-Specific Commands ---
\newcommand{\Tfield}{T(x,t)}
\newcommand{\Efield}{E(x,t)}
\newcommand{\mfield}{m(x,t)}
\newcommand{\Lag}{\mathcal{L}}
\newcommand{\calL}{\mathcal{L}}
\newcommand{\alphaem}{\alpha}
\newcommand{\betaT}{\beta_T}
\newcommand{\xiT}{\xi}
\newcommand{\xipar}{\xi}
\newcommand{\Ezero}{E_0}
\newcommand{\EPlanck}{E_{\text{Pl}}}
\newcommand{\Mpl}{M_{\text{Pl}}}
\newcommand{\lP}{\ell_{\text{P}}}
\newcommand{\tP}{t_{\text{P}}}
\newcommand{\LPlanck}{\ell_{\text{Pl}}}
\newcommand{\TPlanck}{t_{\text{Pl}}}
\newcommand{\Gnat}{G_{\text{nat}}}
\newcommand{\alphaEM}{\alpha_{\text{EM}}}
\newcommand{\alphaSI}{\alpha_{\text{SI}}}
\newcommand{\Hubble}{H_0}
\newcommand{\LCDM}{\Lambda\text{CDM}}
\newcommand{\natunits}{(nat. units)}

% T0 Model Parameters
\newcommand{\xigeom}{\xi_{\mathrm{geom}}}
\newcommand{\rzero}{r_{0}}
\newcommand{\xirat}{\xi_{\mathrm{rat}}}
\newcommand{\tzero}{t_{0}}
\newcommand{\Lambdat}{\Lambda_{\mathrm{t}}}
\newcommand{\EP}{E_{\mathrm{P}}}
\newcommand{\Emu}{E_{\mu}}
\newcommand{\Ee}{E_{e}}
\newcommand{\Etau}{E_{\tau}}
\newcommand{\alphafine}{\alpha_{\mathrm{fine}}}
\newcommand{\alphal}{\alpha_{\ell}}

% Additional Commands
\newcommand{\Kfrak}{K_{\text{frak}}}
\newcommand{\Dfrak}{D_{\text{frak}}}
\newcommand{\betapar}{\beta_T}
\newcommand{\alphapar}{\alpha}
\newcommand{\deltafield}{\delta \phi}
\newcommand{\deltam}{\delta m}
\newcommand{\deltaE}{\delta E}
\newcommand{\Exi}{E_{\xi}}
\newcommand{\Lxi}{\ell_{\xi}}
\newcommand{\rhoCMB}{\rho_{\text{CMB}}}
\newcommand{\rhoCasimir}{\rho_{\text{Casimir}}}
\newcommand{\Leff}{L_{\text{eff}}}
\newcommand{\CQCD}{C_{\mathrm{QCD}}}
\newcommand{\Kspec}{K_{\mathrm{spec}}}

% --- tcolorbox Styles ---
\tcbset{
    keyresult/.style={
        colback=blue!5!white,
        colframe=blue!75!black,
        title=Key Result,
        fonttitle=\bfseries
    },
    foundation/.style={
        colback=green!5!white,
        colframe=green!75!black,
        title=Foundation,
        fonttitle=\bfseries
    },
    alternative/.style={
        colback=orange!5!white,
        colframe=orange!75!black,
        title=Alternative,
        fonttitle=\bfseries
    },
    warningbox/.style={
        colback=red!5!white,
        colframe=red!75!black,
        title=Warning,
        fonttitle=\bfseries
    }
}

\newtcolorbox{keyresultbox}[1][]{keyresult, #1}
\newtcolorbox{foundationbox}[1][]{foundation, #1}
\newtcolorbox{alternativebox}[1][]{alternative, #1}
\newtcolorbox{warningboxenv}[1][]{warningbox, #1}

% Custom boxes for formulas
\newtcolorbox{fundamental}[1][]{
    colback=boxgray,
    colframe=t0blue,
    fonttitle=\bfseries,
    title=#1,
    sharp corners,
    boxrule=2pt
}

\newtcolorbox{newperspective}[1][]{
    colback=red!5!white,
    colframe=t0red,
    fonttitle=\bfseries,
    title=#1,
    sharp corners,
    boxrule=2pt
}

\newtcolorbox{formula}[1][]{
    colback=blue!5!white,
    colframe=blue!75!black,
    fonttitle=\bfseries,
    title=#1
}

\newtcolorbox{result}[1][]{
    colback=green!5!white,
    colframe=green!75!black,
    fonttitle=\bfseries,
    title=#1
}

% --- Layout Settings ---
\sloppy
\hfuzz=2pt
\vfuzz=2pt
\tolerance=1000
\emergencystretch=3em
\raggedbottom

% --- TOC Formatting ---
\renewcommand{\cftsecfont}{\color{blue}}
\renewcommand{\cftsubsecfont}{\color{blue}}
\renewcommand{\cftsecpagefont}{\color{blue}}
\renewcommand{\cftsubsecpagefont}{\color{blue}}
\renewcommand{\cfttoctitlefont}{\huge\bfseries\color{blue}}

% --- Default Header and Footer ---
\pagestyle{fancy}
\fancyhf{}
\fancyhead[L]{\textsc{T0 Theory}}
\fancyhead[R]{\textsc{J. Pascher}}
\fancyfoot[C]{\thepage}

% ==============================================================================
% End of Preamble
% ==============================================================================
 after \documentclass.
% ==============================================================================

% --- Encoding and Language ---
\usepackage[utf8]{inputenc}
\usepackage[T1]{fontenc}
\usepackage[english]{babel}
\usepackage{lmodern}

% --- Page Geometry ---
\usepackage[a4paper, margin=2.5cm]{geometry}
\setlength{\headheight}{15pt}

% --- Mathematics and Physics ---
\usepackage{amsmath,amssymb,amsfonts,amsthm}
\usepackage{mathtools}
\usepackage{physics}
\usepackage{siunitx}
\sisetup{
    locale=US,
    group-separator={,},
    output-decimal-marker={.},
    per-mode=symbol
}

% --- Graphics and Tables ---
\usepackage{graphicx}
\usepackage[table,xcdraw]{xcolor}
\usepackage{tikz}
\usetikzlibrary{arrows.meta,positioning,shapes.geometric,decorations.pathmorphing,patterns,shapes.arrows,intersections}
\usepackage{pgfplots}
\pgfplotsset{compat=1.18}
\usepackage{tcolorbox}
\usepackage{booktabs}
\usepackage{array}
\usepackage{longtable}
\usepackage{float}
\usepackage{adjustbox}
\usepackage{tabularx}
\usepackage{multirow}

% --- Document Formatting ---
\usepackage{fancyhdr}
\renewcommand{\headrulewidth}{0.4pt}
\renewcommand{\footrulewidth}{0.4pt}
\usepackage{tocloft}
\usepackage{hyperref}
\usepackage{bookmark}
\usepackage{cleveref}
\usepackage{microtype}
\usepackage{enumitem}
\usepackage{setspace}
\usepackage{ragged2e}
\usepackage{multicol}

% --- Code and Algorithms ---
\usepackage{algorithm}
\usepackage{algorithmic}
\usepackage{listings}
\usepackage{mdframed}

% --- Additional Packages ---
\usepackage{pdflscape}
\usepackage{braket}
\usepackage{cancel}
\usepackage{caption}
\usepackage{csquotes}
\usepackage{gensymb}
\usepackage{hyphenat}
\usepackage{textcomp}
\usepackage{textgreek}
\usepackage{upgreek}
\usepackage{url}
\usepackage{slashed}
\usepackage{bm}

% --- Column Types ---
\newcolumntype{L}[1]{>{\raggedright\arraybackslash}p{#1}}
\newcolumntype{C}[1]{>{\centering\arraybackslash}p{#1}}

% --- Unicode Characters ---
\usepackage{newunicodechar}
\newunicodechar{ħ}{$\hbar$}
\newunicodechar{↔}{$\leftrightarrow$}
\newunicodechar{⇐}{$\Leftarrow$}
\newunicodechar{⇒}{$\Rightarrow$}
\newunicodechar{⇔}{$\Leftrightarrow$}
\newunicodechar{∂}{$\partial$}
\newunicodechar{∅}{$\emptyset$}
\newunicodechar{∇}{$\nabla$}
\newunicodechar{∈}{$\in$}
\newunicodechar{∉}{$\notin$}
\newunicodechar{∏}{$\prod$}
\newunicodechar{∑}{$\sum$}
\newunicodechar{√}{$\sqrt{}$}
\newunicodechar{∝}{$\propto$}
\newunicodechar{∞}{$\infty$}
\newunicodechar{∩}{$\cap$}
\newunicodechar{∪}{$\cup$}
\newunicodechar{∫}{$\int$}
\newunicodechar{≈}{$\approx$}
\newunicodechar{≠}{$\neq$}
\newunicodechar{≤}{$\leq$}
\newunicodechar{≥}{$\geq$}
\newunicodechar{ξ}{\ensuremath{\xi}}
\newunicodechar{μ}{\ensuremath{\mu}}
\newunicodechar{ψ}{\ensuremath{\psi}}
\newunicodechar{φ}{\ensuremath{\phi}}
\newunicodechar{π}{\ensuremath{\pi}}
\newunicodechar{λ}{\ensuremath{\lambda}}
\newunicodechar{Δ}{\ensuremath{\Delta}}

% --- Colors ---
\definecolor{blue}{rgb}{0,0,1}
\definecolor{boxgray}{RGB}{240,240,240}
\definecolor{deepblue}{RGB}{0,0,127}
\definecolor{deepgreen}{RGB}{0,127,0}
\definecolor{deepred}{RGB}{191,0,0}
\definecolor{t0blue}{RGB}{33,150,243}
\definecolor{t0green}{RGB}{76,175,80}
\definecolor{t0orange}{RGB}{255,152,0}
\definecolor{t0purple}{RGB}{156,39,176}
\definecolor{t0red}{RGB}{244,67,54}
\definecolor{t0yellow}{RGB}{255,204,0}

% --- Hyperref Settings ---
\hypersetup{
    colorlinks=true,
    linkcolor=blue,
    citecolor=blue,
    urlcolor=blue,
    breaklinks=true,
    bookmarksnumbered=true,
    pdfstartview=FitH
}

% --- Theorem Environments (English) ---
\theoremstyle{plain}
\newtheorem{theorem}{Theorem}[section]
\newtheorem{lemma}[theorem]{Lemma}
\newtheorem{proposition}[theorem]{Proposition}
\newtheorem{corollary}[theorem]{Corollary}

\theoremstyle{definition}
\newtheorem{definition}[theorem]{Definition}
\newtheorem{example}[theorem]{Example}
\newtheorem{insight}[theorem]{Insight}
\newtheorem{discovery}[theorem]{Discovery}

\theoremstyle{remark}
\newtheorem{remark}[theorem]{Remark}
\newtheorem{warning}[theorem]{Warning}
\newtheorem{axiom}{Axiom}
\newtheorem{principle}{Principle}

% --- T0-Specific Commands ---
\newcommand{\Tfield}{T(x,t)}
\newcommand{\Efield}{E(x,t)}
\newcommand{\mfield}{m(x,t)}
\newcommand{\Lag}{\mathcal{L}}
\newcommand{\calL}{\mathcal{L}}
\newcommand{\alphaem}{\alpha}
\newcommand{\betaT}{\beta_T}
\newcommand{\xiT}{\xi}
\newcommand{\xipar}{\xi}
\newcommand{\Ezero}{E_0}
\newcommand{\EPlanck}{E_{\text{Pl}}}
\newcommand{\Mpl}{M_{\text{Pl}}}
\newcommand{\lP}{\ell_{\text{P}}}
\newcommand{\tP}{t_{\text{P}}}
\newcommand{\LPlanck}{\ell_{\text{Pl}}}
\newcommand{\TPlanck}{t_{\text{Pl}}}
\newcommand{\Gnat}{G_{\text{nat}}}
\newcommand{\alphaEM}{\alpha_{\text{EM}}}
\newcommand{\alphaSI}{\alpha_{\text{SI}}}
\newcommand{\Hubble}{H_0}
\newcommand{\LCDM}{\Lambda\text{CDM}}
\newcommand{\natunits}{(nat. units)}

% T0 Model Parameters
\newcommand{\xigeom}{\xi_{\mathrm{geom}}}
\newcommand{\rzero}{r_{0}}
\newcommand{\xirat}{\xi_{\mathrm{rat}}}
\newcommand{\tzero}{t_{0}}
\newcommand{\Lambdat}{\Lambda_{\mathrm{t}}}
\newcommand{\EP}{E_{\mathrm{P}}}
\newcommand{\Emu}{E_{\mu}}
\newcommand{\Ee}{E_{e}}
\newcommand{\Etau}{E_{\tau}}
\newcommand{\alphafine}{\alpha_{\mathrm{fine}}}
\newcommand{\alphal}{\alpha_{\ell}}

% Additional Commands
\newcommand{\Kfrak}{K_{\text{frak}}}
\newcommand{\Dfrak}{D_{\text{frak}}}
\newcommand{\betapar}{\beta_T}
\newcommand{\alphapar}{\alpha}
\newcommand{\deltafield}{\delta \phi}
\newcommand{\deltam}{\delta m}
\newcommand{\deltaE}{\delta E}
\newcommand{\Exi}{E_{\xi}}
\newcommand{\Lxi}{\ell_{\xi}}
\newcommand{\rhoCMB}{\rho_{\text{CMB}}}
\newcommand{\rhoCasimir}{\rho_{\text{Casimir}}}
\newcommand{\Leff}{L_{\text{eff}}}
\newcommand{\CQCD}{C_{\mathrm{QCD}}}
\newcommand{\Kspec}{K_{\mathrm{spec}}}

% --- tcolorbox Styles ---
\tcbset{
    keyresult/.style={
        colback=blue!5!white,
        colframe=blue!75!black,
        title=Key Result,
        fonttitle=\bfseries
    },
    foundation/.style={
        colback=green!5!white,
        colframe=green!75!black,
        title=Foundation,
        fonttitle=\bfseries
    },
    alternative/.style={
        colback=orange!5!white,
        colframe=orange!75!black,
        title=Alternative,
        fonttitle=\bfseries
    },
    warningbox/.style={
        colback=red!5!white,
        colframe=red!75!black,
        title=Warning,
        fonttitle=\bfseries
    }
}

\newtcolorbox{keyresultbox}[1][]{keyresult, #1}
\newtcolorbox{foundationbox}[1][]{foundation, #1}
\newtcolorbox{alternativebox}[1][]{alternative, #1}
\newtcolorbox{warningboxenv}[1][]{warningbox, #1}

% Custom boxes for formulas
\newtcolorbox{fundamental}[1][]{
    colback=boxgray,
    colframe=t0blue,
    fonttitle=\bfseries,
    title=#1,
    sharp corners,
    boxrule=2pt
}

\newtcolorbox{newperspective}[1][]{
    colback=red!5!white,
    colframe=t0red,
    fonttitle=\bfseries,
    title=#1,
    sharp corners,
    boxrule=2pt
}

\newtcolorbox{formula}[1][]{
    colback=blue!5!white,
    colframe=blue!75!black,
    fonttitle=\bfseries,
    title=#1
}

\newtcolorbox{result}[1][]{
    colback=green!5!white,
    colframe=green!75!black,
    fonttitle=\bfseries,
    title=#1
}

% --- Layout Settings ---
\sloppy
\hfuzz=2pt
\vfuzz=2pt
\tolerance=1000
\emergencystretch=3em
\raggedbottom

% --- TOC Formatting ---
\renewcommand{\cftsecfont}{\color{blue}}
\renewcommand{\cftsubsecfont}{\color{blue}}
\renewcommand{\cftsecpagefont}{\color{blue}}
\renewcommand{\cftsubsecpagefont}{\color{blue}}
\renewcommand{\cfttoctitlefont}{\huge\bfseries\color{blue}}

% --- Default Header and Footer ---
\pagestyle{fancy}
\fancyhf{}
\fancyhead[L]{\textsc{T0 Theory}}
\fancyhead[R]{\textsc{J. Pascher}}
\fancyfoot[C]{\thepage}

% ==============================================================================
% End of Preamble
% ==============================================================================

% \begin{document}
% \maketitle

\chapter{Kommentar: CMB- und Quasar-Dipol-Anomalie -- Eine dramatische Bestätigung der T0-Vorhersagen!}
%\author{}
%\date{}
%\begin{document}



Dieses Video \href{https://www.youtube.com/watch?v=OywWThFmEII}{OywWThFmEII} ist geradezu \textbf{sensationell} für die T0-Theorie, denn es beschreibt genau das kosmologische Rätsel, für das T0 eine elegante Lösung bietet. Die Widersprüche im Video sind für die Standardkosmologie katastrophal, für T0 hingegen \textbf{erwartbar und vorhersagbar}. Neuere Reviews und Studien aus 2025 unterstreichen die anhaltende Krise in der Kosmologie und bestätigen die Relevanz dieser Anomalien \cite{sarkar2025, landstry2025, bengaly2025}.

\section{Das Problem: Zwei Dipole, zwei Richtungen}

Das Video präsentiert den Kern-Widerspruch (basierend auf dem Quaia-Katalog mit 1,3 Mio.\ Quasaren \cite{storey2024}):
\begin{itemize}
	\item \textbf{CMB-Dipol}: Zeigt nach Leo, 370 km/s
	\item \textbf{Quasar-Dipol}: Zeigt zum Galaktischen Zentrum, $\sim$1700 km/s \cite{mittal2024}
	\item \textbf{Winkel zwischen beiden}: 90° (orthogonal!) \cite{secrest2024}
\end{itemize}

Die Standardkosmologie steht vor einem Trilemma:
\begin{enumerate}
	\item Quasare sind falsch $\rightarrow$ schwer zu rechtfertigen bei 1,3 Mio.\ Objekten
	\item Beide sind Artefakte $\rightarrow$ unglaubwürdig
	\item Das Universum ist anisotrop $\rightarrow$ kosmologisches Prinzip kollabiert
\end{enumerate}

\section{Die T0-Lösung: Wellenlängenabhängige Rotverschiebung}

\subsection{1. T0 sagt vorher: Der CMB-Dipol ist KEINE Bewegung}

In meinen Projektdokumenten (\texttt{redshift\_deflection\_En.tex}, \texttt{cosmic\_En.tex}) ist genau beschrieben:

\textbf{CMB im T0-Modell:}
\begin{itemize}
	\item Die CMB-Temperatur ergibt sich als: $T_{\text{CMB}} = \frac{16}{9} \xi^2 \times E_\xi \approx 2.725$ K
	\item Der CMB-Dipol ist \textbf{keine Doppler-Bewegung}, sondern eine \textbf{intrinsische Anisotropie} des $\xi$-Feldes
	\item Das $\xi$-Feld ($\xi = \frac{4}{3} \times 10^{-4}$) ist das fundamentale Vakuumfeld, aus dem die CMB als Gleichgewichtsstrahlung entsteht
\end{itemize}

Das Video sagt bei \textbf{12:19}: \textit{``The cleanest reading is that the CMB dipole is not a velocity at all. It's something else.''}

\textbf{Das ist EXAKT die T0-Interpretation!}

\subsection{2. Wellenlängenabhängige Rotverschiebung erklärt den Quasar-Dipol}

Die T0-Theorie sagt vorher:

$$z(\lambda_0) = \frac{\xi x}{E_\xi} \cdot \lambda_0$$

\textbf{Kritisch:} Die Rotverschiebung hängt von der Wellenlänge ab!

\begin{itemize}
	\item \textbf{Optische Quasar-Spektren} (sichtbares Licht, $\sim$500 nm): Zeigen größere Rotverschiebung
	\item \textbf{Radio-Beobachtungen} (21 cm): Zeigen kleinere Rotverschiebung
	\item \textbf{CMB-Photonen} (Mikrowellen, $\sim$1 mm): Unterschiedliche Energieverlustrate
\end{itemize}

Der Quasar-Dipol könnte entstehen durch:
\begin{enumerate}
	\item \textbf{Strukturelle Asymmetrie} im $\xi$-Feld entlang der galaktischen Ebene
	\item \textbf{Wellenlängenselektionseffekte} im Quaia-Katalog \cite{storey2024}
	\item \textbf{Kombination} aus lokalem $\xi$-Feld-Gradienten und echter Bewegung
\end{enumerate}

\subsection{3. Die 90°-Orthogonalität: Ein Hinweis auf Feldgeometrie}

Das Video erwähnt bei \textbf{13:17}: \textit{``The two dipoles don't just disagree. They're almost exactly 90° apart.''} \cite{secrest2024}

\textbf{T0-Interpretation:}
\begin{itemize}
	\item Der Quasar-Dipol folgt der \textbf{Materieverteilung} (baryonische Strukturen)
	\item Der CMB-Dipol zeigt die \textbf{$\xi$-Feld-Anisotropie} (Vakuumfeld)
	\item Die Orthogonalität könnte eine \textbf{fundamentale Eigenschaft} der Materie-Feld-Kopplung sein
\end{itemize}

In der T0-Theorie gibt es eine duale Struktur:
\begin{itemize}
	\item $T \cdot m = 1$ (Zeit-Masse-Dualität)
	\item $\alpha_{\text{EM}} = \beta_T = 1$ (elektromagnetisch-temporal Einheit)
\end{itemize}

Diese Dualität könnte geometrische Orthogonalitäten zwischen Materie- und Strahlungskomponenten implizieren. 
Neuere Analysen aus 2025 verstärken diese Spannung durch Hinweise auf Superhorizon-Fluktuationen und Residuen-Dipole \cite{sarkar2025, bengaly2025}.

\subsection{4. Statisches Universum löst das ``Great Attractor''-Problem}

Das Video erwähnt ``Dark Flow'' und großskalige Strukturen. Im T0-Modell:

\textbf{Statisches, zyklisches Universum:}
\begin{itemize}
	\item Kein Big Bang $\rightarrow$ keine Expansion
	\item Strukturbildung ist \textbf{kontinuierlich} und \textbf{zyklisch}
	\item Großskalige Flows sind echte gravitationale Bewegungen, nicht ``peculiar velocities'' relativ zur Expansion
	\item Der ``Great Attractor'' ist einfach eine massive Struktur in einem statischen Raum
\end{itemize}

\subsection{5. Testbare Vorhersagen}

Das Video endet frustriert: \textit{``Two compasses, two directions.''} (bei \textbf{13:22})

\textbf{T0 bietet klare Tests:}

\subsubsection{A) Multi-Wellenlängen-Spektroskopie:}

Wasserstofflinien-Test:
\begin{itemize}
	\item Lyman-$\alpha$ (121,6 nm) vs.\ H$\alpha$ (656,3 nm)
	\item T0-Vorhersage: $z_{\mathrm{Ly}\alpha} / z_{\mathrm{H}\alpha} = 0{,}185$
	\item Standardkosmologie: $= 1$
\end{itemize}

\subsubsection{B) Radio vs.\ Optische Rotverschiebung:}
Für dieselben Quasare:
\begin{itemize}
	\item 21 cm HI-Linie
	\item Optische Emissionslinien
	\item \textbf{T0 sagt massive Unterschiede vorher}, Standard erwartet Identität
\end{itemize}

\subsubsection{C) CMB-Temperatur-Rotverschiebung:}
$$T(z) = T_0(1+z)(1+\ln(1+z))$$
Statt der Standard-Relation $T(z) = T_0(1+z)$

\subsection{6. Auflösung der ``Hubble-Spannung''}

Das Video erwähnt nicht direkt die Hubble-Spannung, aber sie ist verwandt. T0 löst sie durch:

\textbf{Effektive Hubble-``Konstante'':}
$$H_0^{\text{eff}} = c \cdot \xi \cdot \lambda_{\text{ref}} \approx 67.45 \text{ km/s/Mpc}$$

bei $\lambda_{\text{ref}} = 550$ nm

Die verschiedenen $H_0$-Messungen nutzen verschiedene Wellenlängen $\rightarrow$ verschiedene scheinbare ``Hubble-Konstanten''! Neuere Untersuchungen zu Dipol-Spannungen aus 2025 unterstützen die Notwendigkeit alternativer Modelle \cite{landstry2025, bengaly2025}.

\section{Alternative Erklärungswege ohne Rotverschiebung}

\subsection{Der grundlegende Paradigmenwechsel}

Falls sich herausstellen sollte, dass die kosmologische Rotverschiebung nicht existiert oder fundamental falsch interpretiert wurde, bietet das T0-Modell alternative Erklärungen, die komplett ohne Expansion auskommen.

\subsection{Berücksichtigung kosmischer Distanzen und minimaler Effekte}

Ein entscheidender physikalischer Aspekt ist die Berücksichtigung der extrem großen Skalen kosmologischer Beobachtungen:

\begin{itemize}
	\item \textbf{Typische Beobachtungsdistanzen:} $1 - 10^4$ Megaparsec ($3 \times 10^{22} - 3 \times 10^{26}$ Meter)
	\item \textbf{Kumulative Effekte:} Selbst minimale prozentuale Änderungen akkumulieren über diese Skalen zu messbaren Größen
\end{itemize}

\subsection{Alternative 1: Energieverlust durch Feldkopplung}

Photonen könnten Energie durch Wechselwirkung mit dem $\xi$-Feld verlieren:

\begin{align}
	\frac{dE}{dt} = -\Gamma(\lambda) \cdot E \cdot \rho_\xi(\vec{x},t)
\end{align}

Mit einer kleinen Kopplungskonstante $\Gamma(\lambda) = 10^{-25} \, \text{m}^{-1}$ ergibt sich über $L = 10^{25} \, \text{m}$:

\begin{align}
	\frac{\Delta E}{E} = -10^{-25} \times 10^{25} = -1 \quad \text{(entspricht z = 1)}
\end{align}

\subsection{Alternative 2: Zeitliche Evolution fundamentaler Konstanten}

\begin{align}
	\frac{\Delta\alpha}{\alpha} = \xi \cdot T
\end{align}

Mit $\xi = 10^{-15} \, \text{Jahr}^{-1}$ und $T = 10^{10}$ Jahren:

\begin{align}
	\frac{\Delta\alpha}{\alpha} = 10^{-5}
\end{align}

\subsection{Alternative 3: Gravitationspotential-Effekte}

\begin{align}
	\frac{\Delta\nu}{\nu} = \frac{\Delta\Phi}{c^2} \cdot h(\lambda)
\end{align}

\subsection{Physikalische Plausibilität}

\begin{quote}
	\textit{„Was auf menschlichen Skalen als vernachlässigbar klein erscheint, wird über kosmologische Distanzen zu einem kumulativ messbaren Effekt. Die scheinbare Stärke kosmologischer Phänomene ist oft mehr ein Maß für die beteiligten Distanzen als für die Stärke der zugrundeliegenden Physik.“}
\end{quote}

Die benötigten Änderungsraten sind extrem klein ($10^{-15} - 10^{-25}$ pro Einheit) und liegen unterhalb aktueller Labor-Nachweisgrenzen, werden aber über kosmologische Skalen messbar.

\subsection{Konsequenzen für die beobachteten Phänomene}

\begin{itemize}
	\item \textbf{Hubble-„Gesetz“}: Resultat kumulativer Energieverluste, nicht Expansion
	\item \textbf{CMB}: Thermisches Gleichgewicht des $\xi$-Feldes
	\item \textbf{Strukturbildung}: Kontinuierlich in einem statischen Raum
\end{itemize}

\section{Fazit: T0 verwandelt Krise in Vorhersage}

\begin{tabular}{p{3.5cm}|p{6cm}|p{5.5cm}}
	\textbf{Problem (Video)} & \textbf{Standardkosmologie} & \textbf{T0-Lösung} \\
	\hline
	CMB-Dipol $\neq$ Quasar-Dipol & Katastrophe \cite{mittal2024} & Erwartet \\
	90° Orthogonalität & Unerklärlich \cite{secrest2024} & Feldgeometrie \\
	Geschwindigkeitswiderspruch & Unmöglich & Verschiedene Phänomene \\
	Anisotropie & Kosmologisches Prinzip bedroht & Lokale $\xi$-Feld-Struktur \\
	Hubble-Spannung & Ungeklärt & Gelöst \\
	JWST frühe Galaxien & Problem & Kein Problem \\
\end{tabular}

Das Video schließt mit: \textit{``Whichever way you turn, something in cosmology doesn't add up.''}

\textbf{T0-Antwort:} Es addiert sich perfekt -- wenn man aufhört, die CMB-Anisotropie als Bewegung zu interpretieren, und stattdessen die wellenlängenabhängige Rotverschiebung im fundamentalen $\xi$-Feld anerkennt.

Die \textbf{1,3 Millionen Quasare} des Quaia-Katalogs sind nicht das Problem -- sie sind der \textbf{Beweis}, dass unsere Interpretation der CMB falsch war. T0 hatte diese Konsequenzen bereits vorhergesagt, bevor diese Beobachtungen gemacht wurden. Aktuelle Entwicklungen aus 2025, wie Tests der Isotropie mit Quasaren, verstärken diese Bestätigung \cite{sarkar2025}.

\textbf{Nächster Schritt:} Die im Video beschriebenen Daten sollten gezielt auf wellenlängenabhängige Effekte analysiert werden. Die T0-Vorhersagen sind so spezifisch, dass sie mit existierenden Multi-Wellenlängen-Katalogen bereits testbar sein könnte.

\begin{thebibliography}{9}
	
	\bibitem{video}
	YouTube-Video: ``Two Compasses Pointing in Different Directions: The CMB and Quasar Dipole Crisis'', 
	URL: \url{https://www.youtube.com/watch?v=OywWThFmEII}, 
	zuletzt abgerufen: 05. Oktober 2025.
	
	\bibitem{storey2024}
	K.~Storey-Fisher, D.~J.~Farrow, D.~W.~Hogg, et al.,
	``Quaia, the Gaia-unWISE Quasar Catalog: An All-sky Spectroscopic Quasar Sample'',
	\emph{The Astrophysical Journal} \textbf{964}, 69 (2024),
	arXiv:2306.17749,
	\url{https://arxiv.org/pdf/2306.17749.pdf}.
	
	\bibitem{mittal2024}
	V.~Mittal, O.~T.~Oayda, G.~F.~Lewis,
	``The Cosmic Dipole in the Quaia Sample of Quasars: A Bayesian Analysis'',
	\emph{Monthly Notices of the Royal Astronomical Society} \textbf{527}, 8497 (2024),
	arXiv:2311.14938,
	\url{https://arxiv.org/pdf/2311.14938.pdf}.
	
	\bibitem{secrest2024}
	A.~Abghari, E.~F.~Bunn, L.~T.~Hergt, et al.,
	``Reassessment of the dipole in the distribution of quasars on the sky'',
	\emph{Journal of Cosmology and Astroparticle Physics} \textbf{11}, 067 (2024),
	arXiv:2405.09762,
	\url{https://arxiv.org/pdf/2405.09762.pdf}.
	
	\bibitem{sarkar2025}
	S.~Sarkar,
	``Colloquium: The Cosmic Dipole Anomaly'',
	arXiv:2505.23526 (2025),
	Accepted for publication in Reviews of Modern Physics,
	\url{https://arxiv.org/pdf/2505.23526.pdf}.
	
	\bibitem{landstry2025}
	M.~Land-Strykowski et al.,
	``Cosmic dipole tensions: confronting the Cosmic Microwave Background with infrared and radio populations of cosmological sources'',
	arXiv:2509.18689 (2025),
	Accepted for publication in MNRAS,
	\url{https://arxiv.org/pdf/2509.18689.pdf}.
	
	\bibitem{bengaly2025}
	J.~Bengaly et al.,
	``The kinematic contribution to the cosmic number count dipole'',
	\emph{Astronomy \& Astrophysics} \textbf{685}, A123 (2025),
	arXiv:2503.02470,
	\url{https://arxiv.org/pdf/2503.02470.pdf}.
	
\end{thebibliography}
%37-------
%\documentclass[11pt,a4paper]{article}

% Standardized preamble - Hannah_En.tex
%% ==============================================================================
% T0 Theory: Standardized English Preamble
% Version: 1.0
% Author: Johann Pascher
% ==============================================================================
% This file contains all necessary packages and definitions for English
% T0 Theory documents. Use % ==============================================================================
% T0 Theory: Standardized English Preamble
% Version: 1.0
% Author: Johann Pascher
% ==============================================================================
% This file contains all necessary packages and definitions for English
% T0 Theory documents. Use % ==============================================================================
% T0 Theory: Standardized English Preamble
% Version: 1.0
% Author: Johann Pascher
% ==============================================================================
% This file contains all necessary packages and definitions for English
% T0 Theory documents. Use \input{T0_preamble_En} after \documentclass.
% ==============================================================================

% --- Encoding and Language ---
\usepackage[utf8]{inputenc}
\usepackage[T1]{fontenc}
\usepackage[english]{babel}
\usepackage{lmodern}

% --- Page Geometry ---
\usepackage[a4paper, margin=2.5cm]{geometry}
\setlength{\headheight}{15pt}

% --- Mathematics and Physics ---
\usepackage{amsmath,amssymb,amsfonts,amsthm}
\usepackage{mathtools}
\usepackage{physics}
\usepackage{siunitx}
\sisetup{
    locale=US,
    group-separator={,},
    output-decimal-marker={.},
    per-mode=symbol
}

% --- Graphics and Tables ---
\usepackage{graphicx}
\usepackage[table,xcdraw]{xcolor}
\usepackage{tikz}
\usetikzlibrary{arrows.meta,positioning,shapes.geometric,decorations.pathmorphing,patterns,shapes.arrows,intersections}
\usepackage{pgfplots}
\pgfplotsset{compat=1.18}
\usepackage{tcolorbox}
\usepackage{booktabs}
\usepackage{array}
\usepackage{longtable}
\usepackage{float}
\usepackage{adjustbox}
\usepackage{tabularx}
\usepackage{multirow}

% --- Document Formatting ---
\usepackage{fancyhdr}
\renewcommand{\headrulewidth}{0.4pt}
\renewcommand{\footrulewidth}{0.4pt}
\usepackage{tocloft}
\usepackage{hyperref}
\usepackage{bookmark}
\usepackage{cleveref}
\usepackage{microtype}
\usepackage{enumitem}
\usepackage{setspace}
\usepackage{ragged2e}
\usepackage{multicol}

% --- Code and Algorithms ---
\usepackage{algorithm}
\usepackage{algorithmic}
\usepackage{listings}
\usepackage{mdframed}

% --- Additional Packages ---
\usepackage{pdflscape}
\usepackage{braket}
\usepackage{cancel}
\usepackage{caption}
\usepackage{csquotes}
\usepackage{gensymb}
\usepackage{hyphenat}
\usepackage{textcomp}
\usepackage{textgreek}
\usepackage{upgreek}
\usepackage{url}
\usepackage{slashed}
\usepackage{bm}

% --- Column Types ---
\newcolumntype{L}[1]{>{\raggedright\arraybackslash}p{#1}}
\newcolumntype{C}[1]{>{\centering\arraybackslash}p{#1}}

% --- Unicode Characters ---
\usepackage{newunicodechar}
\newunicodechar{ħ}{$\hbar$}
\newunicodechar{↔}{$\leftrightarrow$}
\newunicodechar{⇐}{$\Leftarrow$}
\newunicodechar{⇒}{$\Rightarrow$}
\newunicodechar{⇔}{$\Leftrightarrow$}
\newunicodechar{∂}{$\partial$}
\newunicodechar{∅}{$\emptyset$}
\newunicodechar{∇}{$\nabla$}
\newunicodechar{∈}{$\in$}
\newunicodechar{∉}{$\notin$}
\newunicodechar{∏}{$\prod$}
\newunicodechar{∑}{$\sum$}
\newunicodechar{√}{$\sqrt{}$}
\newunicodechar{∝}{$\propto$}
\newunicodechar{∞}{$\infty$}
\newunicodechar{∩}{$\cap$}
\newunicodechar{∪}{$\cup$}
\newunicodechar{∫}{$\int$}
\newunicodechar{≈}{$\approx$}
\newunicodechar{≠}{$\neq$}
\newunicodechar{≤}{$\leq$}
\newunicodechar{≥}{$\geq$}
\newunicodechar{ξ}{\ensuremath{\xi}}
\newunicodechar{μ}{\ensuremath{\mu}}
\newunicodechar{ψ}{\ensuremath{\psi}}
\newunicodechar{φ}{\ensuremath{\phi}}
\newunicodechar{π}{\ensuremath{\pi}}
\newunicodechar{λ}{\ensuremath{\lambda}}
\newunicodechar{Δ}{\ensuremath{\Delta}}

% --- Colors ---
\definecolor{blue}{rgb}{0,0,1}
\definecolor{boxgray}{RGB}{240,240,240}
\definecolor{deepblue}{RGB}{0,0,127}
\definecolor{deepgreen}{RGB}{0,127,0}
\definecolor{deepred}{RGB}{191,0,0}
\definecolor{t0blue}{RGB}{33,150,243}
\definecolor{t0green}{RGB}{76,175,80}
\definecolor{t0orange}{RGB}{255,152,0}
\definecolor{t0purple}{RGB}{156,39,176}
\definecolor{t0red}{RGB}{244,67,54}
\definecolor{t0yellow}{RGB}{255,204,0}

% --- Hyperref Settings ---
\hypersetup{
    colorlinks=true,
    linkcolor=blue,
    citecolor=blue,
    urlcolor=blue,
    breaklinks=true,
    bookmarksnumbered=true,
    pdfstartview=FitH
}

% --- Theorem Environments (English) ---
\theoremstyle{plain}
\newtheorem{theorem}{Theorem}[section]
\newtheorem{lemma}[theorem]{Lemma}
\newtheorem{proposition}[theorem]{Proposition}
\newtheorem{corollary}[theorem]{Corollary}

\theoremstyle{definition}
\newtheorem{definition}[theorem]{Definition}
\newtheorem{example}[theorem]{Example}
\newtheorem{insight}[theorem]{Insight}
\newtheorem{discovery}[theorem]{Discovery}

\theoremstyle{remark}
\newtheorem{remark}[theorem]{Remark}
\newtheorem{warning}[theorem]{Warning}
\newtheorem{axiom}{Axiom}
\newtheorem{principle}{Principle}

% --- T0-Specific Commands ---
\newcommand{\Tfield}{T(x,t)}
\newcommand{\Efield}{E(x,t)}
\newcommand{\mfield}{m(x,t)}
\newcommand{\Lag}{\mathcal{L}}
\newcommand{\calL}{\mathcal{L}}
\newcommand{\alphaem}{\alpha}
\newcommand{\betaT}{\beta_T}
\newcommand{\xiT}{\xi}
\newcommand{\xipar}{\xi}
\newcommand{\Ezero}{E_0}
\newcommand{\EPlanck}{E_{\text{Pl}}}
\newcommand{\Mpl}{M_{\text{Pl}}}
\newcommand{\lP}{\ell_{\text{P}}}
\newcommand{\tP}{t_{\text{P}}}
\newcommand{\LPlanck}{\ell_{\text{Pl}}}
\newcommand{\TPlanck}{t_{\text{Pl}}}
\newcommand{\Gnat}{G_{\text{nat}}}
\newcommand{\alphaEM}{\alpha_{\text{EM}}}
\newcommand{\alphaSI}{\alpha_{\text{SI}}}
\newcommand{\Hubble}{H_0}
\newcommand{\LCDM}{\Lambda\text{CDM}}
\newcommand{\natunits}{(nat. units)}

% T0 Model Parameters
\newcommand{\xigeom}{\xi_{\mathrm{geom}}}
\newcommand{\rzero}{r_{0}}
\newcommand{\xirat}{\xi_{\mathrm{rat}}}
\newcommand{\tzero}{t_{0}}
\newcommand{\Lambdat}{\Lambda_{\mathrm{t}}}
\newcommand{\EP}{E_{\mathrm{P}}}
\newcommand{\Emu}{E_{\mu}}
\newcommand{\Ee}{E_{e}}
\newcommand{\Etau}{E_{\tau}}
\newcommand{\alphafine}{\alpha_{\mathrm{fine}}}
\newcommand{\alphal}{\alpha_{\ell}}

% Additional Commands
\newcommand{\Kfrak}{K_{\text{frak}}}
\newcommand{\Dfrak}{D_{\text{frak}}}
\newcommand{\betapar}{\beta_T}
\newcommand{\alphapar}{\alpha}
\newcommand{\deltafield}{\delta \phi}
\newcommand{\deltam}{\delta m}
\newcommand{\deltaE}{\delta E}
\newcommand{\Exi}{E_{\xi}}
\newcommand{\Lxi}{\ell_{\xi}}
\newcommand{\rhoCMB}{\rho_{\text{CMB}}}
\newcommand{\rhoCasimir}{\rho_{\text{Casimir}}}
\newcommand{\Leff}{L_{\text{eff}}}
\newcommand{\CQCD}{C_{\mathrm{QCD}}}
\newcommand{\Kspec}{K_{\mathrm{spec}}}

% --- tcolorbox Styles ---
\tcbset{
    keyresult/.style={
        colback=blue!5!white,
        colframe=blue!75!black,
        title=Key Result,
        fonttitle=\bfseries
    },
    foundation/.style={
        colback=green!5!white,
        colframe=green!75!black,
        title=Foundation,
        fonttitle=\bfseries
    },
    alternative/.style={
        colback=orange!5!white,
        colframe=orange!75!black,
        title=Alternative,
        fonttitle=\bfseries
    },
    warningbox/.style={
        colback=red!5!white,
        colframe=red!75!black,
        title=Warning,
        fonttitle=\bfseries
    }
}

\newtcolorbox{keyresultbox}[1][]{keyresult, #1}
\newtcolorbox{foundationbox}[1][]{foundation, #1}
\newtcolorbox{alternativebox}[1][]{alternative, #1}
\newtcolorbox{warningboxenv}[1][]{warningbox, #1}

% Custom boxes for formulas
\newtcolorbox{fundamental}[1][]{
    colback=boxgray,
    colframe=t0blue,
    fonttitle=\bfseries,
    title=#1,
    sharp corners,
    boxrule=2pt
}

\newtcolorbox{newperspective}[1][]{
    colback=red!5!white,
    colframe=t0red,
    fonttitle=\bfseries,
    title=#1,
    sharp corners,
    boxrule=2pt
}

\newtcolorbox{formula}[1][]{
    colback=blue!5!white,
    colframe=blue!75!black,
    fonttitle=\bfseries,
    title=#1
}

\newtcolorbox{result}[1][]{
    colback=green!5!white,
    colframe=green!75!black,
    fonttitle=\bfseries,
    title=#1
}

% --- Layout Settings ---
\sloppy
\hfuzz=2pt
\vfuzz=2pt
\tolerance=1000
\emergencystretch=3em
\raggedbottom

% --- TOC Formatting ---
\renewcommand{\cftsecfont}{\color{blue}}
\renewcommand{\cftsubsecfont}{\color{blue}}
\renewcommand{\cftsecpagefont}{\color{blue}}
\renewcommand{\cftsubsecpagefont}{\color{blue}}
\renewcommand{\cfttoctitlefont}{\huge\bfseries\color{blue}}

% --- Default Header and Footer ---
\pagestyle{fancy}
\fancyhf{}
\fancyhead[L]{\textsc{T0 Theory}}
\fancyhead[R]{\textsc{J. Pascher}}
\fancyfoot[C]{\thepage}

% ==============================================================================
% End of Preamble
% ==============================================================================
 after \documentclass.
% ==============================================================================

% --- Encoding and Language ---
\usepackage[utf8]{inputenc}
\usepackage[T1]{fontenc}
\usepackage[english]{babel}
\usepackage{lmodern}

% --- Page Geometry ---
\usepackage[a4paper, margin=2.5cm]{geometry}
\setlength{\headheight}{15pt}

% --- Mathematics and Physics ---
\usepackage{amsmath,amssymb,amsfonts,amsthm}
\usepackage{mathtools}
\usepackage{physics}
\usepackage{siunitx}
\sisetup{
    locale=US,
    group-separator={,},
    output-decimal-marker={.},
    per-mode=symbol
}

% --- Graphics and Tables ---
\usepackage{graphicx}
\usepackage[table,xcdraw]{xcolor}
\usepackage{tikz}
\usetikzlibrary{arrows.meta,positioning,shapes.geometric,decorations.pathmorphing,patterns,shapes.arrows,intersections}
\usepackage{pgfplots}
\pgfplotsset{compat=1.18}
\usepackage{tcolorbox}
\usepackage{booktabs}
\usepackage{array}
\usepackage{longtable}
\usepackage{float}
\usepackage{adjustbox}
\usepackage{tabularx}
\usepackage{multirow}

% --- Document Formatting ---
\usepackage{fancyhdr}
\renewcommand{\headrulewidth}{0.4pt}
\renewcommand{\footrulewidth}{0.4pt}
\usepackage{tocloft}
\usepackage{hyperref}
\usepackage{bookmark}
\usepackage{cleveref}
\usepackage{microtype}
\usepackage{enumitem}
\usepackage{setspace}
\usepackage{ragged2e}
\usepackage{multicol}

% --- Code and Algorithms ---
\usepackage{algorithm}
\usepackage{algorithmic}
\usepackage{listings}
\usepackage{mdframed}

% --- Additional Packages ---
\usepackage{pdflscape}
\usepackage{braket}
\usepackage{cancel}
\usepackage{caption}
\usepackage{csquotes}
\usepackage{gensymb}
\usepackage{hyphenat}
\usepackage{textcomp}
\usepackage{textgreek}
\usepackage{upgreek}
\usepackage{url}
\usepackage{slashed}
\usepackage{bm}

% --- Column Types ---
\newcolumntype{L}[1]{>{\raggedright\arraybackslash}p{#1}}
\newcolumntype{C}[1]{>{\centering\arraybackslash}p{#1}}

% --- Unicode Characters ---
\usepackage{newunicodechar}
\newunicodechar{ħ}{$\hbar$}
\newunicodechar{↔}{$\leftrightarrow$}
\newunicodechar{⇐}{$\Leftarrow$}
\newunicodechar{⇒}{$\Rightarrow$}
\newunicodechar{⇔}{$\Leftrightarrow$}
\newunicodechar{∂}{$\partial$}
\newunicodechar{∅}{$\emptyset$}
\newunicodechar{∇}{$\nabla$}
\newunicodechar{∈}{$\in$}
\newunicodechar{∉}{$\notin$}
\newunicodechar{∏}{$\prod$}
\newunicodechar{∑}{$\sum$}
\newunicodechar{√}{$\sqrt{}$}
\newunicodechar{∝}{$\propto$}
\newunicodechar{∞}{$\infty$}
\newunicodechar{∩}{$\cap$}
\newunicodechar{∪}{$\cup$}
\newunicodechar{∫}{$\int$}
\newunicodechar{≈}{$\approx$}
\newunicodechar{≠}{$\neq$}
\newunicodechar{≤}{$\leq$}
\newunicodechar{≥}{$\geq$}
\newunicodechar{ξ}{\ensuremath{\xi}}
\newunicodechar{μ}{\ensuremath{\mu}}
\newunicodechar{ψ}{\ensuremath{\psi}}
\newunicodechar{φ}{\ensuremath{\phi}}
\newunicodechar{π}{\ensuremath{\pi}}
\newunicodechar{λ}{\ensuremath{\lambda}}
\newunicodechar{Δ}{\ensuremath{\Delta}}

% --- Colors ---
\definecolor{blue}{rgb}{0,0,1}
\definecolor{boxgray}{RGB}{240,240,240}
\definecolor{deepblue}{RGB}{0,0,127}
\definecolor{deepgreen}{RGB}{0,127,0}
\definecolor{deepred}{RGB}{191,0,0}
\definecolor{t0blue}{RGB}{33,150,243}
\definecolor{t0green}{RGB}{76,175,80}
\definecolor{t0orange}{RGB}{255,152,0}
\definecolor{t0purple}{RGB}{156,39,176}
\definecolor{t0red}{RGB}{244,67,54}
\definecolor{t0yellow}{RGB}{255,204,0}

% --- Hyperref Settings ---
\hypersetup{
    colorlinks=true,
    linkcolor=blue,
    citecolor=blue,
    urlcolor=blue,
    breaklinks=true,
    bookmarksnumbered=true,
    pdfstartview=FitH
}

% --- Theorem Environments (English) ---
\theoremstyle{plain}
\newtheorem{theorem}{Theorem}[section]
\newtheorem{lemma}[theorem]{Lemma}
\newtheorem{proposition}[theorem]{Proposition}
\newtheorem{corollary}[theorem]{Corollary}

\theoremstyle{definition}
\newtheorem{definition}[theorem]{Definition}
\newtheorem{example}[theorem]{Example}
\newtheorem{insight}[theorem]{Insight}
\newtheorem{discovery}[theorem]{Discovery}

\theoremstyle{remark}
\newtheorem{remark}[theorem]{Remark}
\newtheorem{warning}[theorem]{Warning}
\newtheorem{axiom}{Axiom}
\newtheorem{principle}{Principle}

% --- T0-Specific Commands ---
\newcommand{\Tfield}{T(x,t)}
\newcommand{\Efield}{E(x,t)}
\newcommand{\mfield}{m(x,t)}
\newcommand{\Lag}{\mathcal{L}}
\newcommand{\calL}{\mathcal{L}}
\newcommand{\alphaem}{\alpha}
\newcommand{\betaT}{\beta_T}
\newcommand{\xiT}{\xi}
\newcommand{\xipar}{\xi}
\newcommand{\Ezero}{E_0}
\newcommand{\EPlanck}{E_{\text{Pl}}}
\newcommand{\Mpl}{M_{\text{Pl}}}
\newcommand{\lP}{\ell_{\text{P}}}
\newcommand{\tP}{t_{\text{P}}}
\newcommand{\LPlanck}{\ell_{\text{Pl}}}
\newcommand{\TPlanck}{t_{\text{Pl}}}
\newcommand{\Gnat}{G_{\text{nat}}}
\newcommand{\alphaEM}{\alpha_{\text{EM}}}
\newcommand{\alphaSI}{\alpha_{\text{SI}}}
\newcommand{\Hubble}{H_0}
\newcommand{\LCDM}{\Lambda\text{CDM}}
\newcommand{\natunits}{(nat. units)}

% T0 Model Parameters
\newcommand{\xigeom}{\xi_{\mathrm{geom}}}
\newcommand{\rzero}{r_{0}}
\newcommand{\xirat}{\xi_{\mathrm{rat}}}
\newcommand{\tzero}{t_{0}}
\newcommand{\Lambdat}{\Lambda_{\mathrm{t}}}
\newcommand{\EP}{E_{\mathrm{P}}}
\newcommand{\Emu}{E_{\mu}}
\newcommand{\Ee}{E_{e}}
\newcommand{\Etau}{E_{\tau}}
\newcommand{\alphafine}{\alpha_{\mathrm{fine}}}
\newcommand{\alphal}{\alpha_{\ell}}

% Additional Commands
\newcommand{\Kfrak}{K_{\text{frak}}}
\newcommand{\Dfrak}{D_{\text{frak}}}
\newcommand{\betapar}{\beta_T}
\newcommand{\alphapar}{\alpha}
\newcommand{\deltafield}{\delta \phi}
\newcommand{\deltam}{\delta m}
\newcommand{\deltaE}{\delta E}
\newcommand{\Exi}{E_{\xi}}
\newcommand{\Lxi}{\ell_{\xi}}
\newcommand{\rhoCMB}{\rho_{\text{CMB}}}
\newcommand{\rhoCasimir}{\rho_{\text{Casimir}}}
\newcommand{\Leff}{L_{\text{eff}}}
\newcommand{\CQCD}{C_{\mathrm{QCD}}}
\newcommand{\Kspec}{K_{\mathrm{spec}}}

% --- tcolorbox Styles ---
\tcbset{
    keyresult/.style={
        colback=blue!5!white,
        colframe=blue!75!black,
        title=Key Result,
        fonttitle=\bfseries
    },
    foundation/.style={
        colback=green!5!white,
        colframe=green!75!black,
        title=Foundation,
        fonttitle=\bfseries
    },
    alternative/.style={
        colback=orange!5!white,
        colframe=orange!75!black,
        title=Alternative,
        fonttitle=\bfseries
    },
    warningbox/.style={
        colback=red!5!white,
        colframe=red!75!black,
        title=Warning,
        fonttitle=\bfseries
    }
}

\newtcolorbox{keyresultbox}[1][]{keyresult, #1}
\newtcolorbox{foundationbox}[1][]{foundation, #1}
\newtcolorbox{alternativebox}[1][]{alternative, #1}
\newtcolorbox{warningboxenv}[1][]{warningbox, #1}

% Custom boxes for formulas
\newtcolorbox{fundamental}[1][]{
    colback=boxgray,
    colframe=t0blue,
    fonttitle=\bfseries,
    title=#1,
    sharp corners,
    boxrule=2pt
}

\newtcolorbox{newperspective}[1][]{
    colback=red!5!white,
    colframe=t0red,
    fonttitle=\bfseries,
    title=#1,
    sharp corners,
    boxrule=2pt
}

\newtcolorbox{formula}[1][]{
    colback=blue!5!white,
    colframe=blue!75!black,
    fonttitle=\bfseries,
    title=#1
}

\newtcolorbox{result}[1][]{
    colback=green!5!white,
    colframe=green!75!black,
    fonttitle=\bfseries,
    title=#1
}

% --- Layout Settings ---
\sloppy
\hfuzz=2pt
\vfuzz=2pt
\tolerance=1000
\emergencystretch=3em
\raggedbottom

% --- TOC Formatting ---
\renewcommand{\cftsecfont}{\color{blue}}
\renewcommand{\cftsubsecfont}{\color{blue}}
\renewcommand{\cftsecpagefont}{\color{blue}}
\renewcommand{\cftsubsecpagefont}{\color{blue}}
\renewcommand{\cfttoctitlefont}{\huge\bfseries\color{blue}}

% --- Default Header and Footer ---
\pagestyle{fancy}
\fancyhf{}
\fancyhead[L]{\textsc{T0 Theory}}
\fancyhead[R]{\textsc{J. Pascher}}
\fancyfoot[C]{\thepage}

% ==============================================================================
% End of Preamble
% ==============================================================================
 after \documentclass.
% ==============================================================================

% --- Encoding and Language ---
\usepackage[utf8]{inputenc}
\usepackage[T1]{fontenc}
\usepackage[english]{babel}
\usepackage{lmodern}

% --- Page Geometry ---
\usepackage[a4paper, margin=2.5cm]{geometry}
\setlength{\headheight}{15pt}

% --- Mathematics and Physics ---
\usepackage{amsmath,amssymb,amsfonts,amsthm}
\usepackage{mathtools}
\usepackage{physics}
\usepackage{siunitx}
\sisetup{
    locale=US,
    group-separator={,},
    output-decimal-marker={.},
    per-mode=symbol
}

% --- Graphics and Tables ---
\usepackage{graphicx}
\usepackage[table,xcdraw]{xcolor}
\usepackage{tikz}
\usetikzlibrary{arrows.meta,positioning,shapes.geometric,decorations.pathmorphing,patterns,shapes.arrows,intersections}
\usepackage{pgfplots}
\pgfplotsset{compat=1.18}
\usepackage{tcolorbox}
\usepackage{booktabs}
\usepackage{array}
\usepackage{longtable}
\usepackage{float}
\usepackage{adjustbox}
\usepackage{tabularx}
\usepackage{multirow}

% --- Document Formatting ---
\usepackage{fancyhdr}
\renewcommand{\headrulewidth}{0.4pt}
\renewcommand{\footrulewidth}{0.4pt}
\usepackage{tocloft}
\usepackage{hyperref}
\usepackage{bookmark}
\usepackage{cleveref}
\usepackage{microtype}
\usepackage{enumitem}
\usepackage{setspace}
\usepackage{ragged2e}
\usepackage{multicol}

% --- Code and Algorithms ---
\usepackage{algorithm}
\usepackage{algorithmic}
\usepackage{listings}
\usepackage{mdframed}

% --- Additional Packages ---
\usepackage{pdflscape}
\usepackage{braket}
\usepackage{cancel}
\usepackage{caption}
\usepackage{csquotes}
\usepackage{gensymb}
\usepackage{hyphenat}
\usepackage{textcomp}
\usepackage{textgreek}
\usepackage{upgreek}
\usepackage{url}
\usepackage{slashed}
\usepackage{bm}

% --- Column Types ---
\newcolumntype{L}[1]{>{\raggedright\arraybackslash}p{#1}}
\newcolumntype{C}[1]{>{\centering\arraybackslash}p{#1}}

% --- Unicode Characters ---
\usepackage{newunicodechar}
\newunicodechar{ħ}{$\hbar$}
\newunicodechar{↔}{$\leftrightarrow$}
\newunicodechar{⇐}{$\Leftarrow$}
\newunicodechar{⇒}{$\Rightarrow$}
\newunicodechar{⇔}{$\Leftrightarrow$}
\newunicodechar{∂}{$\partial$}
\newunicodechar{∅}{$\emptyset$}
\newunicodechar{∇}{$\nabla$}
\newunicodechar{∈}{$\in$}
\newunicodechar{∉}{$\notin$}
\newunicodechar{∏}{$\prod$}
\newunicodechar{∑}{$\sum$}
\newunicodechar{√}{$\sqrt{}$}
\newunicodechar{∝}{$\propto$}
\newunicodechar{∞}{$\infty$}
\newunicodechar{∩}{$\cap$}
\newunicodechar{∪}{$\cup$}
\newunicodechar{∫}{$\int$}
\newunicodechar{≈}{$\approx$}
\newunicodechar{≠}{$\neq$}
\newunicodechar{≤}{$\leq$}
\newunicodechar{≥}{$\geq$}
\newunicodechar{ξ}{\ensuremath{\xi}}
\newunicodechar{μ}{\ensuremath{\mu}}
\newunicodechar{ψ}{\ensuremath{\psi}}
\newunicodechar{φ}{\ensuremath{\phi}}
\newunicodechar{π}{\ensuremath{\pi}}
\newunicodechar{λ}{\ensuremath{\lambda}}
\newunicodechar{Δ}{\ensuremath{\Delta}}

% --- Colors ---
\definecolor{blue}{rgb}{0,0,1}
\definecolor{boxgray}{RGB}{240,240,240}
\definecolor{deepblue}{RGB}{0,0,127}
\definecolor{deepgreen}{RGB}{0,127,0}
\definecolor{deepred}{RGB}{191,0,0}
\definecolor{t0blue}{RGB}{33,150,243}
\definecolor{t0green}{RGB}{76,175,80}
\definecolor{t0orange}{RGB}{255,152,0}
\definecolor{t0purple}{RGB}{156,39,176}
\definecolor{t0red}{RGB}{244,67,54}
\definecolor{t0yellow}{RGB}{255,204,0}

% --- Hyperref Settings ---
\hypersetup{
    colorlinks=true,
    linkcolor=blue,
    citecolor=blue,
    urlcolor=blue,
    breaklinks=true,
    bookmarksnumbered=true,
    pdfstartview=FitH
}

% --- Theorem Environments (English) ---
\theoremstyle{plain}
\newtheorem{theorem}{Theorem}[section]
\newtheorem{lemma}[theorem]{Lemma}
\newtheorem{proposition}[theorem]{Proposition}
\newtheorem{corollary}[theorem]{Corollary}

\theoremstyle{definition}
\newtheorem{definition}[theorem]{Definition}
\newtheorem{example}[theorem]{Example}
\newtheorem{insight}[theorem]{Insight}
\newtheorem{discovery}[theorem]{Discovery}

\theoremstyle{remark}
\newtheorem{remark}[theorem]{Remark}
\newtheorem{warning}[theorem]{Warning}
\newtheorem{axiom}{Axiom}
\newtheorem{principle}{Principle}

% --- T0-Specific Commands ---
\newcommand{\Tfield}{T(x,t)}
\newcommand{\Efield}{E(x,t)}
\newcommand{\mfield}{m(x,t)}
\newcommand{\Lag}{\mathcal{L}}
\newcommand{\calL}{\mathcal{L}}
\newcommand{\alphaem}{\alpha}
\newcommand{\betaT}{\beta_T}
\newcommand{\xiT}{\xi}
\newcommand{\xipar}{\xi}
\newcommand{\Ezero}{E_0}
\newcommand{\EPlanck}{E_{\text{Pl}}}
\newcommand{\Mpl}{M_{\text{Pl}}}
\newcommand{\lP}{\ell_{\text{P}}}
\newcommand{\tP}{t_{\text{P}}}
\newcommand{\LPlanck}{\ell_{\text{Pl}}}
\newcommand{\TPlanck}{t_{\text{Pl}}}
\newcommand{\Gnat}{G_{\text{nat}}}
\newcommand{\alphaEM}{\alpha_{\text{EM}}}
\newcommand{\alphaSI}{\alpha_{\text{SI}}}
\newcommand{\Hubble}{H_0}
\newcommand{\LCDM}{\Lambda\text{CDM}}
\newcommand{\natunits}{(nat. units)}

% T0 Model Parameters
\newcommand{\xigeom}{\xi_{\mathrm{geom}}}
\newcommand{\rzero}{r_{0}}
\newcommand{\xirat}{\xi_{\mathrm{rat}}}
\newcommand{\tzero}{t_{0}}
\newcommand{\Lambdat}{\Lambda_{\mathrm{t}}}
\newcommand{\EP}{E_{\mathrm{P}}}
\newcommand{\Emu}{E_{\mu}}
\newcommand{\Ee}{E_{e}}
\newcommand{\Etau}{E_{\tau}}
\newcommand{\alphafine}{\alpha_{\mathrm{fine}}}
\newcommand{\alphal}{\alpha_{\ell}}

% Additional Commands
\newcommand{\Kfrak}{K_{\text{frak}}}
\newcommand{\Dfrak}{D_{\text{frak}}}
\newcommand{\betapar}{\beta_T}
\newcommand{\alphapar}{\alpha}
\newcommand{\deltafield}{\delta \phi}
\newcommand{\deltam}{\delta m}
\newcommand{\deltaE}{\delta E}
\newcommand{\Exi}{E_{\xi}}
\newcommand{\Lxi}{\ell_{\xi}}
\newcommand{\rhoCMB}{\rho_{\text{CMB}}}
\newcommand{\rhoCasimir}{\rho_{\text{Casimir}}}
\newcommand{\Leff}{L_{\text{eff}}}
\newcommand{\CQCD}{C_{\mathrm{QCD}}}
\newcommand{\Kspec}{K_{\mathrm{spec}}}

% --- tcolorbox Styles ---
\tcbset{
    keyresult/.style={
        colback=blue!5!white,
        colframe=blue!75!black,
        title=Key Result,
        fonttitle=\bfseries
    },
    foundation/.style={
        colback=green!5!white,
        colframe=green!75!black,
        title=Foundation,
        fonttitle=\bfseries
    },
    alternative/.style={
        colback=orange!5!white,
        colframe=orange!75!black,
        title=Alternative,
        fonttitle=\bfseries
    },
    warningbox/.style={
        colback=red!5!white,
        colframe=red!75!black,
        title=Warning,
        fonttitle=\bfseries
    }
}

\newtcolorbox{keyresultbox}[1][]{keyresult, #1}
\newtcolorbox{foundationbox}[1][]{foundation, #1}
\newtcolorbox{alternativebox}[1][]{alternative, #1}
\newtcolorbox{warningboxenv}[1][]{warningbox, #1}

% Custom boxes for formulas
\newtcolorbox{fundamental}[1][]{
    colback=boxgray,
    colframe=t0blue,
    fonttitle=\bfseries,
    title=#1,
    sharp corners,
    boxrule=2pt
}

\newtcolorbox{newperspective}[1][]{
    colback=red!5!white,
    colframe=t0red,
    fonttitle=\bfseries,
    title=#1,
    sharp corners,
    boxrule=2pt
}

\newtcolorbox{formula}[1][]{
    colback=blue!5!white,
    colframe=blue!75!black,
    fonttitle=\bfseries,
    title=#1
}

\newtcolorbox{result}[1][]{
    colback=green!5!white,
    colframe=green!75!black,
    fonttitle=\bfseries,
    title=#1
}

% --- Layout Settings ---
\sloppy
\hfuzz=2pt
\vfuzz=2pt
\tolerance=1000
\emergencystretch=3em
\raggedbottom

% --- TOC Formatting ---
\renewcommand{\cftsecfont}{\color{blue}}
\renewcommand{\cftsubsecfont}{\color{blue}}
\renewcommand{\cftsecpagefont}{\color{blue}}
\renewcommand{\cftsubsecpagefont}{\color{blue}}
\renewcommand{\cfttoctitlefont}{\huge\bfseries\color{blue}}

% --- Default Header and Footer ---
\pagestyle{fancy}
\fancyhf{}
\fancyhead[L]{\textsc{T0 Theory}}
\fancyhead[R]{\textsc{J. Pascher}}
\fancyfoot[C]{\thepage}

% ==============================================================================
% End of Preamble
% ==============================================================================

%\nwepage
% \end{document}
