%\documentclass[12pt,a4paper]{article}

% Standardized preamble - redshift_deflection_De.tex
%% ==============================================================================
% T0-Theorie: Standardisierte Deutsche Präambel
% Version: 1.0
% Autor: Johann Pascher
% ==============================================================================
% Diese Datei enthält alle notwendigen Pakete und Definitionen für deutsche
% T0-Theorie Dokumente. Verwenden Sie % ==============================================================================
% T0-Theorie: Standardisierte Deutsche Präambel
% Version: 1.0
% Autor: Johann Pascher
% ==============================================================================
% Diese Datei enthält alle notwendigen Pakete und Definitionen für deutsche
% T0-Theorie Dokumente. Verwenden Sie % ==============================================================================
% T0-Theorie: Standardisierte Deutsche Präambel
% Version: 1.0
% Autor: Johann Pascher
% ==============================================================================
% Diese Datei enthält alle notwendigen Pakete und Definitionen für deutsche
% T0-Theorie Dokumente. Verwenden Sie \input{T0_preamble_De} nach \documentclass.
% ==============================================================================

% --- Kodierung und Sprache ---
\usepackage[utf8]{inputenc}
\usepackage[T1]{fontenc}
\usepackage[ngerman]{babel}
\usepackage{lmodern}

% --- Seitengeometrie ---
\usepackage[a4paper, margin=2.5cm]{geometry}
\setlength{\headheight}{15pt}

% --- Mathematik und Physik ---
\usepackage{amsmath,amssymb,amsfonts,amsthm}
\usepackage{mathtools}
\usepackage{physics}
\usepackage{siunitx}
\sisetup{
    locale=DE,
    group-separator={.},
    output-decimal-marker={,},
    per-mode=symbol
}

% --- Grafiken und Tabellen ---
\usepackage{graphicx}
\usepackage[table,xcdraw]{xcolor}
\usepackage{tikz}
\usetikzlibrary{arrows.meta,positioning,shapes.geometric,decorations.pathmorphing,patterns,shapes.arrows,intersections}
\usepackage{pgfplots}
\pgfplotsset{compat=1.18}
\usepackage{quantikz}
\usepackage[most]{tcolorbox}
\tcbuselibrary{breakable}

% === WICHTIG: Algorithm-Konflikt umgehen ===
% Option: algorithmic mit GROSSBUCHSTABEN
% Gemeinsame Box für Experimente
\newtcolorbox{experimentbox}[1][]{
	colback=green!5!white,
	colframe=t0green!80!black,
	fonttitle=\bfseries,
	title={{#1}},
	breakable
}

% Abstract-Fallback
\ifdefined\abstract\else
\newenvironment{abstract}{\section*{\abstractname}\itshape\small\par\bigskip}{\bigskip}
\fi

% === MAKROS SICHER NEU DEFINIEREN / ÜBERSCHREIBEN ===
% Definiere Makros OHNE doppelte Subskripte
\newcommand{\phipar}{\phi_{\mathrm{par}}}
%\newcommand{\xipar}{\xi_{\mathrm{par}}}
\newcommand{\Qphipar}{Q_{\phi_{\mathrm{par}}}}
\newcommand{\rphipar}{r_{\phi_{\mathrm{par}}}}
\newcommand{\logphipar}{\log_{\phi_{\mathrm{par}}}}
\newcommand{\CHSH}{\text{CHSH}}
\usepackage{booktabs}
\usepackage{array}
\usepackage{longtable}
\usepackage{float}
\usepackage{adjustbox}
\usepackage{tabularx}
\usepackage{multirow}

% --- Dokumentformatierung ---
\usepackage{fancyhdr}
\renewcommand{\headrulewidth}{0.4pt}
\renewcommand{\footrulewidth}{0.4pt}
\usepackage{tocloft}
\usepackage{hyperref}
\usepackage{bookmark}
\usepackage{cleveref}
\usepackage{microtype}
\usepackage{enumitem}
\usepackage{setspace}
\usepackage{ragged2e}
\usepackage{multicol}

% --- Code und Algorithmen ---
\usepackage{algorithm}
\usepackage{algorithmic}
\usepackage{listings}
\usepackage{mdframed}

% --- Zitationsbefehle (Kompatibilität) ---
\providecommand{\citep}[1]{\cite{#1}}
\providecommand{\citet}[1]{\cite{#1}}

% --- Zusätzliche Pakete ---
\usepackage{pdflscape}
\usepackage{braket}
\usepackage{cancel}
\usepackage{caption}
\usepackage{csquotes}
\usepackage{gensymb}
\usepackage{hyphenat}
\usepackage{textcomp}
\usepackage{textgreek}
\usepackage{upgreek}
\usepackage{url}
% Hyphenation for URLs in bibliography
\def\UrlBreaks{\do\/\do-}
\usepackage{slashed}
\usepackage{bm}

% --- Fehlende Farben definieren ---
\definecolor{gold}{RGB}{255,215,0}

% --- Spaltentypen ---
\newcolumntype{L}[1]{>{\raggedright\arraybackslash}p{#1}}
\newcolumntype{C}[1]{>{\centering\arraybackslash}p{#1}}

% --- Unicode-Zeichen ---
\usepackage{newunicodechar}
\newunicodechar{ħ}{$\hbar$}
\newunicodechar{↔}{$\leftrightarrow$}
\newunicodechar{⇐}{$\Leftarrow$}
\newunicodechar{⇒}{$\Rightarrow$}
\newunicodechar{⇔}{$\Leftrightarrow$}
\newunicodechar{∂}{$\partial$}
\newunicodechar{∅}{$\emptyset$}
\newunicodechar{∇}{$\nabla$}
\newunicodechar{∈}{$\in$}
\newunicodechar{∉}{$\notin$}
\newunicodechar{∏}{$\prod$}
\newunicodechar{∑}{$\sum$}
\newunicodechar{√}{$\sqrt{}$}
\newunicodechar{∝}{$\propto$}
\newunicodechar{∞}{$\infty$}
\newunicodechar{∩}{$\cap$}
\newunicodechar{∪}{$\cup$}
\newunicodechar{∫}{$\int$}
\newunicodechar{≈}{$\approx$}
\newunicodechar{≠}{$\neq$}
\newunicodechar{≤}{$\leq$}
\newunicodechar{≥}{$\geq$}
\newunicodechar{ξ}{\ensuremath{\xi}}
\newunicodechar{μ}{\ensuremath{\mu}}
\newunicodechar{ψ}{\ensuremath{\psi}}
\newunicodechar{φ}{\ensuremath{\phi}}
\newunicodechar{π}{\ensuremath{\pi}}
\newunicodechar{λ}{\ensuremath{\lambda}}
\newunicodechar{Δ}{\ensuremath{\Delta}}

% --- Farben ---
\definecolor{blue}{rgb}{0,0,1}
\definecolor{boxgray}{RGB}{240,240,240}
\definecolor{deepblue}{RGB}{0,0,127}
\definecolor{deepgreen}{RGB}{0,127,0}
\definecolor{deepred}{RGB}{191,0,0}
\definecolor{t0blue}{RGB}{33,150,243}
\definecolor{t0green}{RGB}{76,175,80}
\definecolor{t0orange}{RGB}{255,152,0}
\definecolor{t0purple}{RGB}{156,39,176}
\definecolor{t0red}{RGB}{244,67,54}
\definecolor{t0yellow}{RGB}{255,204,0}

% --- Hyperref-Einstellungen ---
\hypersetup{
    colorlinks=true,
    linkcolor=blue,
    citecolor=blue,
    urlcolor=blue,
    breaklinks=true,
    bookmarksnumbered=true,
    pdfstartview=FitH
}

% --- Theorem-Umgebungen (Deutsch) ---
\theoremstyle{plain}
\newtheorem{satz}{Satz}[section]
\newtheorem{lemma}[satz]{Lemma}
\newtheorem{proposition}[satz]{Proposition}
\newtheorem{korollar}[satz]{Korollar}

\theoremstyle{definition}
\newtheorem{definition}[satz]{Definition}
\newtheorem{beispiel}[satz]{Beispiel}
\newtheorem{erkenntnis}[satz]{Erkenntnis}
\newtheorem{entdeckung}[satz]{Entdeckung}

\theoremstyle{remark}
\newtheorem{bemerkung}[satz]{Bemerkung}
\newtheorem{warnung}[satz]{Warnung}
\newtheorem{axiom}{Axiom}
\newtheorem{prinzip}{Prinzip}

% Aliases für englische Bezeichnungen
\newtheorem{theorem}[satz]{Theorem}
\newtheorem{corollary}[satz]{Corollary}
\newtheorem{remark}[satz]{Remark}
\newtheorem{example}[satz]{Example}
\newtheorem{insight}[satz]{Insight}
\newtheorem{discovery}[satz]{Discovery}
\newtheorem{principle}[satz]{Principle}

% --- T0-spezifische Befehle ---
\newcommand{\Tfield}{T(x,t)}
\providecommand{\Tfieldt}{T(\vec{x},t)}
\newcommand{\Efield}{E(x,t)}
\newcommand{\mfield}{m(x,t)}
\providecommand{\vecx}{\vec{x}}
\newcommand{\Lag}{\mathcal{L}}
\newcommand{\calL}{\mathcal{L}}
\newcommand{\alphaem}{\alpha}
\newcommand{\betaT}{\beta_T}
\newcommand{\xiT}{\xi}
\newcommand{\xipar}{\xi}
\newcommand{\Ezero}{E_0}
\newcommand{\EPlanck}{E_{\text{Pl}}}
\newcommand{\Mpl}{M_{\text{Pl}}}
\newcommand{\lP}{\ell_{\text{P}}}
\newcommand{\tP}{t_{\text{P}}}
\newcommand{\LPlanck}{\ell_{\text{Pl}}}
\newcommand{\TPlanck}{t_{\text{Pl}}}
\newcommand{\Gnat}{G_{\text{nat}}}
\newcommand{\alphaEM}{\alpha_{\text{EM}}}
\newcommand{\alphaSI}{\alpha_{\text{SI}}}
\newcommand{\Hubble}{H_0}
\newcommand{\LCDM}{\Lambda\text{CDM}}
\newcommand{\natunits}{(nat. Einheiten)}

% T0 Modell Parameter
\newcommand{\xigeom}{\xi_{\mathrm{geom}}}
\newcommand{\rzero}{r_{0}}
\newcommand{\xirat}{\xi_{\mathrm{rat}}}
\newcommand{\tzero}{t_{0}}
\newcommand{\Lambdat}{\Lambda_{\mathrm{t}}}
\newcommand{\EP}{E_{\mathrm{P}}}
\newcommand{\Emu}{E_{\mu}}
\newcommand{\Ee}{E_{e}}
\newcommand{\Etau}{E_{\tau}}
\newcommand{\alphafine}{\alpha_{\mathrm{fine}}}
\newcommand{\alphal}{\alpha_{\ell}}
\newcommand{\Lzero}{\ell_{0}}
\newcommand{\Lp}{\ell_{\mathrm{P}}}

% Zusätzliche Befehle
\newcommand{\Kfrak}{K_{\text{frak}}}
\newcommand{\Dfrak}{D_{\text{frak}}}
\newcommand{\betapar}{\beta_T}
\newcommand{\alphapar}{\alpha}
\newcommand{\deltafield}{\delta \phi}
\newcommand{\deltam}{\delta m}
\newcommand{\deltaE}{\delta E}
\newcommand{\Exi}{E_{\xi}}
\newcommand{\Lxi}{\ell_{\xi}}
\newcommand{\rhoCMB}{\rho_{\text{CMB}}}
\newcommand{\rhoCasimir}{\rho_{\text{Casimir}}}
\newcommand{\Leff}{L_{\text{eff}}}
\newcommand{\CQCD}{C_{\mathrm{QCD}}}
\newcommand{\Kspec}{K_{\mathrm{spec}}}

% Fehlende Befehle aus Dokumenten
\providecommand{\xiconst}{\xi_{\text{const}}}
\providecommand{\DhiggsT}{D_{\text{Higgs-T}}}
\providecommand{\rhoE}{\rho_{E}}
\providecommand{\Echar}{E_{\text{char}}}
\providecommand{\kfrac}{k_{\text{frac}}}
\providecommand{\alphaEMSI}{\alpha_{\text{EM,SI}}}
\providecommand{\alphaEMnat}{\alpha_{\text{EM,nat}}}
\providecommand{\betaTSI}{\beta_{T,\text{SI}}}
\providecommand{\betaTnat}{\beta_{T,\text{nat}}}
\providecommand{\Gsi}{G_{\text{SI}}}
\providecommand{\xiparSI}{\xi_{\text{SI}}}
\providecommand{\xiparnat}{\xi_{\text{nat}}}
\providecommand{\meff}{m_{\text{eff}}}
\providecommand{\Tzerot}{T_{0}(t)}
\providecommand{\mzerot}{m_{0}(t)}
\providecommand{\Ezeroabs}{E_{0,\text{abs}}}
\providecommand{\Epar}{E_{\text{par}}}
\providecommand{\Lnat}{\ell_{\text{nat}}}
\providecommand{\Tnat}{T_{\text{nat}}}
\providecommand{\xifrak}{\xi_{\text{frac}}}
\providecommand{\Tfrak}{T_{\text{frac}}}
\providecommand{\mfrak}{m_{\text{frac}}}
\providecommand{\Dfrac}{D_{\text{frac}}}
\providecommand{\EphotSI}{E_{\gamma,\text{SI}}}
\providecommand{\EphotNat}{E_{\gamma,\text{nat}}}
\providecommand{\Eabsint}{E_{\text{abs,int}}}
\providecommand{\mphoton}{m_{\gamma}}

% Zusätzliche fehlende Befehle aus Dokumenten
\providecommand{\Evis}{E_{\text{vis}}}
\providecommand{\Cto}{C_{T0}}
\providecommand{\mytimes}{\times}
\providecommand{\lambdah}{\lambda_h}
\providecommand{\checkmarkx}{\checkmark}
\providecommand{\Enorm}{E_{\text{norm}}}
\providecommand{\Tobs}{T_{\text{obs}}}
\providecommand{\mobs}{m_{\text{obs}}}
\providecommand{\Eobs}{E_{\text{obs}}}
\providecommand{\Lobs}{\ell_{\text{obs}}}
\providecommand{\xobs}{\xi_{\text{obs}}}
\providecommand{\calE}{\mathcal{E}}
\providecommand{\calT}{\mathcal{T}}
\providecommand{\calM}{\mathcal{M}}
\providecommand{\alphag}{\alpha_g}
\providecommand{\Tmax}{T_{\text{max}}}
\providecommand{\mmin}{m_{\text{min}}}
\providecommand{\Lmax}{\ell_{\text{max}}}
\providecommand{\Emin}{E_{\text{min}}}
\providecommand{\Geff}{G_{\text{eff}}}
\providecommand{\rhoeff}{\rho_{\text{eff}}}
\providecommand{\xieff}{\xi_{\text{eff}}}
\providecommand{\Teff}{T_{\text{eff}}}
\providecommand{\hPlanck}{h}
\providecommand{\kB}{k_B}
\providecommand{\muB}{\mu_B}
\providecommand{\lambdaC}{\lambda_C}
\providecommand{\omegaP}{\omega_P}
\providecommand{\rhoP}{\rho_P}
\providecommand{\Tref}{T_{\text{ref}}}
\providecommand{\Eref}{E_{\text{ref}}}
\providecommand{\mref}{m_{\text{ref}}}
\providecommand{\Lref}{\ell_{\text{ref}}}

% --- tcolorbox Stile ---
\tcbset{
    keyresult/.style={
        colback=blue!5!white,
        colframe=blue!75!black,
        title=Kernaussage,
        fonttitle=\bfseries
    },
    foundation/.style={
        colback=green!5!white,
        colframe=green!75!black,
        title=Grundlage,
        fonttitle=\bfseries
    },
    alternative/.style={
        colback=orange!5!white,
        colframe=orange!75!black,
        title=Alternative,
        fonttitle=\bfseries
    },
    warningbox/.style={
        colback=red!5!white,
        colframe=red!75!black,
        title=Warnung,
        fonttitle=\bfseries
    }
}

\newtcolorbox{keyresultbox}[1][]{colback=blue!5!white,colframe=blue!75!black,fonttitle=\bfseries,title={#1},breakable}
\newtcolorbox{keyresult}[1][Kernaussage]{colback=blue!5!white,colframe=blue!75!black,fonttitle=\bfseries,title={#1},breakable}
\newtcolorbox{foundationbox}[1][]{colback=green!5!white,colframe=green!75!black,fonttitle=\bfseries,title={#1},breakable}
\newtcolorbox{foundation}[1][Grundlage]{colback=green!5!white,colframe=green!75!black,fonttitle=\bfseries,title={#1},breakable}
\newtcolorbox{alternativebox}[1][]{colback=orange!5!white,colframe=orange!75!black,fonttitle=\bfseries,title={#1},breakable}
\newtcolorbox{warningboxenv}[1][]{colback=red!5!white,colframe=red!75!black,fonttitle=\bfseries,title={#1},breakable}

% Benutzerdefinierte Boxen für Formeln
\newtcolorbox{fundamental}[1][]{
    colback=boxgray,
    colframe=t0blue,
    fonttitle=\bfseries,
    title=#1,
    sharp corners,
    boxrule=2pt
}

\newtcolorbox{neueperspektive}[1][]{
    colback=red!5!white,
    colframe=t0red,
    fonttitle=\bfseries,
    title=#1,
    sharp corners,
    boxrule=2pt
}

\newtcolorbox{formula}[1][]{
    colback=blue!5!white,
    colframe=blue!75!black,
    fonttitle=\bfseries,
    title=#1
}

\newtcolorbox{result}[1][]{
    colback=green!5!white,
    colframe=green!75!black,
    fonttitle=\bfseries,
    title=#1
}

% Zusätzliche tcolorbox-Umgebungen (aus T0_standalone_header_de.tex)
\newtcolorbox{derivation}[1][]{
    colback=green!5!white,
    colframe=green!75!black,
    title=#1,
    fonttitle=\bfseries,
    breakable
}

\newtcolorbox{summary}[1][]{
    colback=gray!10!white,
    colframe=gray!75!black,
    title=#1,
    fonttitle=\bfseries,
    breakable
}

\newtcolorbox{comparison}[1][]{
    colback=purple!5!white,
    colframe=purple!75!black,
    title=#1,
    fonttitle=\bfseries,
    breakable
}

\newtcolorbox{relation}[1][]{
    colback=cyan!5!white,
    colframe=cyan!75!black,
    title=#1,
    fonttitle=\bfseries,
    breakable
}

\newtcolorbox{principleBox}[1][]{
    colback=yellow!5!white,
    colframe=yellow!75!black,
    title=#1,
    fonttitle=\bfseries,
    breakable
}

% Hinweis: insight und discovery sind als Theorem-Umgebungen definiert
% insightBox und discoveryBox für tcolorbox-Versionen
\newtcolorbox{insightBox}[1][]{colback=blue!5,colframe=t0blue,title={#1},fonttitle=\bfseries,breakable}
\newtcolorbox{discoveryBox}[1][]{colback=green!5,colframe=t0green,title={#1},fonttitle=\bfseries,breakable}
\newtcolorbox{newperspective}[1][]{colback=yellow!5,colframe=orange,title={#1},fonttitle=\bfseries,breakable}
\newtcolorbox{revelation}[1][]{colback=red!5,colframe=t0red,title={#1},fonttitle=\bfseries,breakable}
\newtcolorbox{keypoint}[1][]{colback=blue!5,colframe=t0blue,title={#1},fonttitle=\bfseries,breakable}
\newtcolorbox{evidenceBox}[1][]{colback=green!5,colframe=t0green,title={#1},fonttitle=\bfseries,breakable}
\newtcolorbox{conclusionBox}[1][]{colback=gray!5,colframe=gray,title={#1},fonttitle=\bfseries,breakable}
\newtcolorbox{significance}[1][]{colback=yellow!5,colframe=orange,title={#1},fonttitle=\bfseries,breakable}
\newtcolorbox{philosophical}[1][]{colback=purple!5,colframe=purple,title={#1},fonttitle=\bfseries,breakable}
\newtcolorbox{implicationBox}[1][]{colback=cyan!5,colframe=cyan,title={#1},fonttitle=\bfseries,breakable}
\newtcolorbox{perspectiveBox}[1][]{colback=blue!5,colframe=t0blue,title={#1},fonttitle=\bfseries,breakable}
\newtcolorbox{revolutionary}[1][]{colback=red!5,colframe=t0red,title={#1},fonttitle=\bfseries,breakable}
\newtcolorbox{technical}[1][]{colback=gray!5,colframe=gray!75!black,title={#1},fonttitle=\bfseries,breakable}
\newtcolorbox{technicalBox}[1][]{colback=gray!5,colframe=gray!75!black,title={#1},fonttitle=\bfseries,breakable}
\newtcolorbox{notationBox}[1][]{colback=yellow!5,colframe=yellow!75!black,title={#1},fonttitle=\bfseries,breakable}
\newtcolorbox{verification}[1][]{colback=orange!5!white,colframe=orange!75!black,fonttitle=\bfseries,title=#1}
\newtcolorbox{explanationBox}[1][]{colback=purple!5!white,colframe=purple!75!black,fonttitle=\bfseries,title=#1}
\newtcolorbox{interpretationBox}[1][]{colback=cyan!5!white,colframe=cyan!75!black,fonttitle=\bfseries,title=#1}
\newtcolorbox{explanation}[1][]{colback=purple!5!white,colframe=purple!75!black,fonttitle=\bfseries,title=#1,breakable}
\newtcolorbox{interpretation}[1][]{colback=cyan!5!white,colframe=cyan!75!black,fonttitle=\bfseries,title=#1,breakable}
\newtcolorbox{proof_step}[1][]{colback=gray!5!white,colframe=gray!75!black,fonttitle=\bfseries,title=#1,breakable}
\newtcolorbox{experimental}[1][]{colback=teal!5!white,colframe=teal!75!black,fonttitle=\bfseries,title=#1,breakable}

% Zusätzliche Umgebungen
\newenvironment{treatise}{\begin{quote}}{\end{quote}}
\newenvironment{gemeinsam}{\begin{quote}}{\end{quote}}
\newenvironment{vergleich}{\begin{quote}}{\end{quote}}
\newenvironment{vorteil}{\begin{quote}}{\end{quote}}
\newenvironment{quantum}{\begin{quote}}{\end{quote}}

% Fehlende tcolorbox-Umgebungen
\newtcolorbox{important}[1][]{colback=red!5!white,colframe=red!75!black,title={#1},fonttitle=\bfseries,breakable}
\newtcolorbox{warning}[1][]{colback=orange!5!white,colframe=orange!75!black,title={#1},fonttitle=\bfseries,breakable}
\newtcolorbox{caution}[1][]{colback=yellow!5!white,colframe=yellow!75!black,title={#1},fonttitle=\bfseries,breakable}
\newtcolorbox{highlight}[1][]{colback=yellow!10!white,colframe=yellow!75!black,title={#1},fonttitle=\bfseries,breakable}
\newtcolorbox{critical}[1][]{colback=red!10!white,colframe=red!75!black,title={#1},fonttitle=\bfseries,breakable}
\newtcolorbox{analysis}[1][]{colback=blue!5!white,colframe=blue!75!black,title={#1},fonttitle=\bfseries,breakable}
\newtcolorbox{application}[1][]{colback=green!5!white,colframe=green!75!black,title={#1},fonttitle=\bfseries,breakable}
\newtcolorbox{experiment}[1][]{colback=cyan!5!white,colframe=cyan!75!black,title={#1},fonttitle=\bfseries,breakable}
\newtcolorbox{historical}[1][]{colback=brown!5!white,colframe=brown!75!black,title={#1},fonttitle=\bfseries,breakable}
\newtcolorbox{numerical}[1][]{colback=gray!5!white,colframe=gray!75!black,title={#1},fonttitle=\bfseries,breakable}
\newtcolorbox{overview}[1][]{colback=blue!5!white,colframe=blue!75!black,title={#1},fonttitle=\bfseries,breakable}
\newtcolorbox{speculation}[1][]{colback=purple!5!white,colframe=purple!75!black,title={#1},fonttitle=\bfseries,breakable}
\newtcolorbox{question}[1][]{colback=orange!5!white,colframe=orange!75!black,title={#1},fonttitle=\bfseries,breakable}
\newtcolorbox{method}[1][]{colback=teal!5!white,colframe=teal!75!black,title={#1},fonttitle=\bfseries,breakable}
\newtcolorbox{correct}[1][]{colback=green!10!white,colframe=green!75!black,title={#1},fonttitle=\bfseries,breakable}
\newtcolorbox{units}[1][]{colback=gray!5!white,colframe=gray!75!black,title={#1},fonttitle=\bfseries,breakable}
\newtcolorbox{achievement}[1][]{colback=gold!5!white,colframe=orange!75!black,title={#1},fonttitle=\bfseries,breakable}
\newtcolorbox{equivalence}[1][]{colback=cyan!5!white,colframe=cyan!75!black,title={#1},fonttitle=\bfseries,breakable}
\newtcolorbox{dimensional}[1][]{colback=purple!5!white,colframe=purple!75!black,title={#1},fonttitle=\bfseries,breakable}
\newtcolorbox{photon}[1][]{colback=yellow!5!white,colframe=yellow!75!black,title={#1},fonttitle=\bfseries,breakable}
\newtcolorbox{neutrino}[1][]{colback=blue!5!white,colframe=blue!75!black,title={#1},fonttitle=\bfseries,breakable}
\newtcolorbox{revolution}[1][]{colback=red!5!white,colframe=red!75!black,title={#1},fonttitle=\bfseries,breakable}
\newtcolorbox{t0box}[1][]{colback=blue!5!white,colframe=t0blue,title={#1},fonttitle=\bfseries,breakable}
\newtcolorbox{documentbox}[1][]{colback=gray!5!white,colframe=gray!75!black,title={#1},fonttitle=\bfseries,breakable}
\newtcolorbox{sibox}[1][]{colback=green!5!white,colframe=green!75!black,title={#1},fonttitle=\bfseries,breakable}
\newtcolorbox{smbox}[1][]{colback=blue!5!white,colframe=blue!75!black,title={#1},fonttitle=\bfseries,breakable}
\newtcolorbox{pvbox}[1][]{colback=purple!5!white,colframe=purple!75!black,title={#1},fonttitle=\bfseries,breakable}
\newtcolorbox{koidebox}[1][]{colback=orange!5!white,colframe=orange!75!black,title={#1},fonttitle=\bfseries,breakable}
\newtcolorbox{formel}[1][]{colback=blue!5!white,colframe=blue!75!black,title={#1},fonttitle=\bfseries,breakable}
\newtcolorbox{schluessel}[1][]{colback=blue!5!white,colframe=blue!75!black,title={#1},fonttitle=\bfseries,breakable}
\newtcolorbox{wichtig}[1][]{colback=red!5!white,colframe=red!75!black,title={#1},fonttitle=\bfseries,breakable}
\newtcolorbox{vorsicht}[1][]{colback=orange!5!white,colframe=orange!75!black,title={#1},fonttitle=\bfseries,breakable}
\newtcolorbox{revolutionaer}[1][]{colback=red!5!white,colframe=red!75!black,title={#1},fonttitle=\bfseries,breakable}
\newtcolorbox{numerisch}[1][]{colback=gray!5!white,colframe=gray!75!black,title={#1},fonttitle=\bfseries,breakable}
\newtcolorbox{experimentell}[1][]{colback=cyan!5!white,colframe=cyan!75!black,title={#1},fonttitle=\bfseries,breakable}
\newtcolorbox{anwendung}[1][]{colback=green!5!white,colframe=green!75!black,title={#1},fonttitle=\bfseries,breakable}
\newtcolorbox{alternative}[1][]{colback=orange!5!white,colframe=orange!75!black,title={#1},fonttitle=\bfseries,breakable}
\newtcolorbox{beziehung}[1][]{colback=cyan!5!white,colframe=cyan!75!black,title={#1},fonttitle=\bfseries,breakable}
\newtcolorbox{folgerung}[1][]{colback=green!5!white,colframe=green!75!black,title={#1},fonttitle=\bfseries,breakable}
\newtcolorbox{abhandlung}[1][]{colback=gray!5!white,colframe=gray!75!black,title={#1},fonttitle=\bfseries,breakable}
\newtcolorbox{prinzipBox}[1][]{colback=blue!5!white,colframe=blue!75!black,title={#1},fonttitle=\bfseries,breakable}
\newtcolorbox{beweis}[1][]{colback=gray!5!white,colframe=gray!75!black,title={#1},fonttitle=\bfseries,breakable}
\newtcolorbox{key}[2][]{colback=blue!5!white,colframe=blue!75!black,title={#2},fonttitle=\bfseries,breakable}
\newtcolorbox{category}[1][]{colback=purple!5!white,colframe=purple!75!black,title={#1},fonttitle=\bfseries,breakable}

% Zusätzliche T0-spezifische Befehle
\newcommand{\Tzero}{T$_0$}
\providecommand{\meff}{m_{\text{eff}}}
\newcommand{\Eabs}{E_{\text{abs}}}
\newcommand{\taupar}{\tau}

% Missing commands from various documents
\providecommand{\xikonst}{\xi_0}
\providecommand{\Phiphoton}{\Phi_{\gamma}}
\providecommand{\etavis}{\eta_{\text{vis}}}
\providecommand{\pichar}{\pi}
\providecommand{\primrel}{\mathcal{P}_{\text{rel}}}
\providecommand{\warningx}{\textcolor{orange}{\textbf{!}}}
\providecommand{\phiT}{\phi_T}
\providecommand{\xiT}{\xi_T}
\providecommand{\Lorentz}{\Lambda}
\providecommand{\Cconv}{C_{\text{conv}}}
\providecommand{\Df}{\Delta f}
\providecommand{\lambdazero}{\lambda_0}
\providecommand{\myapprox}{\approx}
\providecommand{\checked}{\checkmark}
\providecommand{\alphaWSI}{\alpha_W^{\text{SI}}}
\providecommand{\alphaWnat}{\alpha_W^{\text{nat}}}
\providecommand{\vect}[1]{\vec{#1}}
\providecommand{\Rzero}{R_0}
\providecommand{\Riem}{\mathcal{R}}
\providecommand{\nuzero}{\nu_0}
\providecommand{\mypi}{\pi}

% --- Layout-Einstellungen ---
\sloppy
\hfuzz=2pt
\vfuzz=2pt
\tolerance=1000
\emergencystretch=3em
\raggedbottom

% --- Inhaltsverzeichnis-Formatierung ---
\renewcommand{\cftsecfont}{\color{blue}}
\renewcommand{\cftsubsecfont}{\color{blue}}
\renewcommand{\cftsecpagefont}{\color{blue}}
\renewcommand{\cftsubsecpagefont}{\color{blue}}
\renewcommand{\cfttoctitlefont}{\huge\bfseries\color{blue}}

% --- Standard Kopf- und Fußzeilen ---
\pagestyle{fancy}
\fancyhf{}
\fancyhead[L]{\textsc{T0-Theorie}}
\fancyhead[R]{\textsc{J. Pascher}}
\fancyfoot[C]{\thepage}

% ==============================================================================
% Ende der Präambel
% ==============================================================================

 nach \documentclass.
% ==============================================================================

% --- Kodierung und Sprache ---
\usepackage[utf8]{inputenc}
\usepackage[T1]{fontenc}
\usepackage[ngerman]{babel}
\usepackage{lmodern}

% --- Seitengeometrie ---
\usepackage[a4paper, margin=2.5cm]{geometry}
\setlength{\headheight}{15pt}

% --- Mathematik und Physik ---
\usepackage{amsmath,amssymb,amsfonts,amsthm}
\usepackage{mathtools}
\usepackage{physics}
\usepackage{siunitx}
\sisetup{
    locale=DE,
    group-separator={.},
    output-decimal-marker={,},
    per-mode=symbol
}

% --- Grafiken und Tabellen ---
\usepackage{graphicx}
\usepackage[table,xcdraw]{xcolor}
\usepackage{tikz}
\usetikzlibrary{arrows.meta,positioning,shapes.geometric,decorations.pathmorphing,patterns,shapes.arrows,intersections}
\usepackage{pgfplots}
\pgfplotsset{compat=1.18}
\usepackage{quantikz}
\usepackage[most]{tcolorbox}
\tcbuselibrary{breakable}

% === WICHTIG: Algorithm-Konflikt umgehen ===
% Option: algorithmic mit GROSSBUCHSTABEN
% Gemeinsame Box für Experimente
\newtcolorbox{experimentbox}[1][]{
	colback=green!5!white,
	colframe=t0green!80!black,
	fonttitle=\bfseries,
	title={{#1}},
	breakable
}

% Abstract-Fallback
\ifdefined\abstract\else
\newenvironment{abstract}{\section*{\abstractname}\itshape\small\par\bigskip}{\bigskip}
\fi

% === MAKROS SICHER NEU DEFINIEREN / ÜBERSCHREIBEN ===
% Definiere Makros OHNE doppelte Subskripte
\newcommand{\phipar}{\phi_{\mathrm{par}}}
%\newcommand{\xipar}{\xi_{\mathrm{par}}}
\newcommand{\Qphipar}{Q_{\phi_{\mathrm{par}}}}
\newcommand{\rphipar}{r_{\phi_{\mathrm{par}}}}
\newcommand{\logphipar}{\log_{\phi_{\mathrm{par}}}}
\newcommand{\CHSH}{\text{CHSH}}
\usepackage{booktabs}
\usepackage{array}
\usepackage{longtable}
\usepackage{float}
\usepackage{adjustbox}
\usepackage{tabularx}
\usepackage{multirow}

% --- Dokumentformatierung ---
\usepackage{fancyhdr}
\renewcommand{\headrulewidth}{0.4pt}
\renewcommand{\footrulewidth}{0.4pt}
\usepackage{tocloft}
\usepackage{hyperref}
\usepackage{bookmark}
\usepackage{cleveref}
\usepackage{microtype}
\usepackage{enumitem}
\usepackage{setspace}
\usepackage{ragged2e}
\usepackage{multicol}

% --- Code und Algorithmen ---
\usepackage{algorithm}
\usepackage{algorithmic}
\usepackage{listings}
\usepackage{mdframed}

% --- Zitationsbefehle (Kompatibilität) ---
\providecommand{\citep}[1]{\cite{#1}}
\providecommand{\citet}[1]{\cite{#1}}

% --- Zusätzliche Pakete ---
\usepackage{pdflscape}
\usepackage{braket}
\usepackage{cancel}
\usepackage{caption}
\usepackage{csquotes}
\usepackage{gensymb}
\usepackage{hyphenat}
\usepackage{textcomp}
\usepackage{textgreek}
\usepackage{upgreek}
\usepackage{url}
% Hyphenation for URLs in bibliography
\def\UrlBreaks{\do\/\do-}
\usepackage{slashed}
\usepackage{bm}

% --- Fehlende Farben definieren ---
\definecolor{gold}{RGB}{255,215,0}

% --- Spaltentypen ---
\newcolumntype{L}[1]{>{\raggedright\arraybackslash}p{#1}}
\newcolumntype{C}[1]{>{\centering\arraybackslash}p{#1}}

% --- Unicode-Zeichen ---
\usepackage{newunicodechar}
\newunicodechar{ħ}{$\hbar$}
\newunicodechar{↔}{$\leftrightarrow$}
\newunicodechar{⇐}{$\Leftarrow$}
\newunicodechar{⇒}{$\Rightarrow$}
\newunicodechar{⇔}{$\Leftrightarrow$}
\newunicodechar{∂}{$\partial$}
\newunicodechar{∅}{$\emptyset$}
\newunicodechar{∇}{$\nabla$}
\newunicodechar{∈}{$\in$}
\newunicodechar{∉}{$\notin$}
\newunicodechar{∏}{$\prod$}
\newunicodechar{∑}{$\sum$}
\newunicodechar{√}{$\sqrt{}$}
\newunicodechar{∝}{$\propto$}
\newunicodechar{∞}{$\infty$}
\newunicodechar{∩}{$\cap$}
\newunicodechar{∪}{$\cup$}
\newunicodechar{∫}{$\int$}
\newunicodechar{≈}{$\approx$}
\newunicodechar{≠}{$\neq$}
\newunicodechar{≤}{$\leq$}
\newunicodechar{≥}{$\geq$}
\newunicodechar{ξ}{\ensuremath{\xi}}
\newunicodechar{μ}{\ensuremath{\mu}}
\newunicodechar{ψ}{\ensuremath{\psi}}
\newunicodechar{φ}{\ensuremath{\phi}}
\newunicodechar{π}{\ensuremath{\pi}}
\newunicodechar{λ}{\ensuremath{\lambda}}
\newunicodechar{Δ}{\ensuremath{\Delta}}

% --- Farben ---
\definecolor{blue}{rgb}{0,0,1}
\definecolor{boxgray}{RGB}{240,240,240}
\definecolor{deepblue}{RGB}{0,0,127}
\definecolor{deepgreen}{RGB}{0,127,0}
\definecolor{deepred}{RGB}{191,0,0}
\definecolor{t0blue}{RGB}{33,150,243}
\definecolor{t0green}{RGB}{76,175,80}
\definecolor{t0orange}{RGB}{255,152,0}
\definecolor{t0purple}{RGB}{156,39,176}
\definecolor{t0red}{RGB}{244,67,54}
\definecolor{t0yellow}{RGB}{255,204,0}

% --- Hyperref-Einstellungen ---
\hypersetup{
    colorlinks=true,
    linkcolor=blue,
    citecolor=blue,
    urlcolor=blue,
    breaklinks=true,
    bookmarksnumbered=true,
    pdfstartview=FitH
}

% --- Theorem-Umgebungen (Deutsch) ---
\theoremstyle{plain}
\newtheorem{satz}{Satz}[section]
\newtheorem{lemma}[satz]{Lemma}
\newtheorem{proposition}[satz]{Proposition}
\newtheorem{korollar}[satz]{Korollar}

\theoremstyle{definition}
\newtheorem{definition}[satz]{Definition}
\newtheorem{beispiel}[satz]{Beispiel}
\newtheorem{erkenntnis}[satz]{Erkenntnis}
\newtheorem{entdeckung}[satz]{Entdeckung}

\theoremstyle{remark}
\newtheorem{bemerkung}[satz]{Bemerkung}
\newtheorem{warnung}[satz]{Warnung}
\newtheorem{axiom}{Axiom}
\newtheorem{prinzip}{Prinzip}

% Aliases für englische Bezeichnungen
\newtheorem{theorem}[satz]{Theorem}
\newtheorem{corollary}[satz]{Corollary}
\newtheorem{remark}[satz]{Remark}
\newtheorem{example}[satz]{Example}
\newtheorem{insight}[satz]{Insight}
\newtheorem{discovery}[satz]{Discovery}
\newtheorem{principle}[satz]{Principle}

% --- T0-spezifische Befehle ---
\newcommand{\Tfield}{T(x,t)}
\providecommand{\Tfieldt}{T(\vec{x},t)}
\newcommand{\Efield}{E(x,t)}
\newcommand{\mfield}{m(x,t)}
\providecommand{\vecx}{\vec{x}}
\newcommand{\Lag}{\mathcal{L}}
\newcommand{\calL}{\mathcal{L}}
\newcommand{\alphaem}{\alpha}
\newcommand{\betaT}{\beta_T}
\newcommand{\xiT}{\xi}
\newcommand{\xipar}{\xi}
\newcommand{\Ezero}{E_0}
\newcommand{\EPlanck}{E_{\text{Pl}}}
\newcommand{\Mpl}{M_{\text{Pl}}}
\newcommand{\lP}{\ell_{\text{P}}}
\newcommand{\tP}{t_{\text{P}}}
\newcommand{\LPlanck}{\ell_{\text{Pl}}}
\newcommand{\TPlanck}{t_{\text{Pl}}}
\newcommand{\Gnat}{G_{\text{nat}}}
\newcommand{\alphaEM}{\alpha_{\text{EM}}}
\newcommand{\alphaSI}{\alpha_{\text{SI}}}
\newcommand{\Hubble}{H_0}
\newcommand{\LCDM}{\Lambda\text{CDM}}
\newcommand{\natunits}{(nat. Einheiten)}

% T0 Modell Parameter
\newcommand{\xigeom}{\xi_{\mathrm{geom}}}
\newcommand{\rzero}{r_{0}}
\newcommand{\xirat}{\xi_{\mathrm{rat}}}
\newcommand{\tzero}{t_{0}}
\newcommand{\Lambdat}{\Lambda_{\mathrm{t}}}
\newcommand{\EP}{E_{\mathrm{P}}}
\newcommand{\Emu}{E_{\mu}}
\newcommand{\Ee}{E_{e}}
\newcommand{\Etau}{E_{\tau}}
\newcommand{\alphafine}{\alpha_{\mathrm{fine}}}
\newcommand{\alphal}{\alpha_{\ell}}
\newcommand{\Lzero}{\ell_{0}}
\newcommand{\Lp}{\ell_{\mathrm{P}}}

% Zusätzliche Befehle
\newcommand{\Kfrak}{K_{\text{frak}}}
\newcommand{\Dfrak}{D_{\text{frak}}}
\newcommand{\betapar}{\beta_T}
\newcommand{\alphapar}{\alpha}
\newcommand{\deltafield}{\delta \phi}
\newcommand{\deltam}{\delta m}
\newcommand{\deltaE}{\delta E}
\newcommand{\Exi}{E_{\xi}}
\newcommand{\Lxi}{\ell_{\xi}}
\newcommand{\rhoCMB}{\rho_{\text{CMB}}}
\newcommand{\rhoCasimir}{\rho_{\text{Casimir}}}
\newcommand{\Leff}{L_{\text{eff}}}
\newcommand{\CQCD}{C_{\mathrm{QCD}}}
\newcommand{\Kspec}{K_{\mathrm{spec}}}

% Fehlende Befehle aus Dokumenten
\providecommand{\xiconst}{\xi_{\text{const}}}
\providecommand{\DhiggsT}{D_{\text{Higgs-T}}}
\providecommand{\rhoE}{\rho_{E}}
\providecommand{\Echar}{E_{\text{char}}}
\providecommand{\kfrac}{k_{\text{frac}}}
\providecommand{\alphaEMSI}{\alpha_{\text{EM,SI}}}
\providecommand{\alphaEMnat}{\alpha_{\text{EM,nat}}}
\providecommand{\betaTSI}{\beta_{T,\text{SI}}}
\providecommand{\betaTnat}{\beta_{T,\text{nat}}}
\providecommand{\Gsi}{G_{\text{SI}}}
\providecommand{\xiparSI}{\xi_{\text{SI}}}
\providecommand{\xiparnat}{\xi_{\text{nat}}}
\providecommand{\meff}{m_{\text{eff}}}
\providecommand{\Tzerot}{T_{0}(t)}
\providecommand{\mzerot}{m_{0}(t)}
\providecommand{\Ezeroabs}{E_{0,\text{abs}}}
\providecommand{\Epar}{E_{\text{par}}}
\providecommand{\Lnat}{\ell_{\text{nat}}}
\providecommand{\Tnat}{T_{\text{nat}}}
\providecommand{\xifrak}{\xi_{\text{frac}}}
\providecommand{\Tfrak}{T_{\text{frac}}}
\providecommand{\mfrak}{m_{\text{frac}}}
\providecommand{\Dfrac}{D_{\text{frac}}}
\providecommand{\EphotSI}{E_{\gamma,\text{SI}}}
\providecommand{\EphotNat}{E_{\gamma,\text{nat}}}
\providecommand{\Eabsint}{E_{\text{abs,int}}}
\providecommand{\mphoton}{m_{\gamma}}

% Zusätzliche fehlende Befehle aus Dokumenten
\providecommand{\Evis}{E_{\text{vis}}}
\providecommand{\Cto}{C_{T0}}
\providecommand{\mytimes}{\times}
\providecommand{\lambdah}{\lambda_h}
\providecommand{\checkmarkx}{\checkmark}
\providecommand{\Enorm}{E_{\text{norm}}}
\providecommand{\Tobs}{T_{\text{obs}}}
\providecommand{\mobs}{m_{\text{obs}}}
\providecommand{\Eobs}{E_{\text{obs}}}
\providecommand{\Lobs}{\ell_{\text{obs}}}
\providecommand{\xobs}{\xi_{\text{obs}}}
\providecommand{\calE}{\mathcal{E}}
\providecommand{\calT}{\mathcal{T}}
\providecommand{\calM}{\mathcal{M}}
\providecommand{\alphag}{\alpha_g}
\providecommand{\Tmax}{T_{\text{max}}}
\providecommand{\mmin}{m_{\text{min}}}
\providecommand{\Lmax}{\ell_{\text{max}}}
\providecommand{\Emin}{E_{\text{min}}}
\providecommand{\Geff}{G_{\text{eff}}}
\providecommand{\rhoeff}{\rho_{\text{eff}}}
\providecommand{\xieff}{\xi_{\text{eff}}}
\providecommand{\Teff}{T_{\text{eff}}}
\providecommand{\hPlanck}{h}
\providecommand{\kB}{k_B}
\providecommand{\muB}{\mu_B}
\providecommand{\lambdaC}{\lambda_C}
\providecommand{\omegaP}{\omega_P}
\providecommand{\rhoP}{\rho_P}
\providecommand{\Tref}{T_{\text{ref}}}
\providecommand{\Eref}{E_{\text{ref}}}
\providecommand{\mref}{m_{\text{ref}}}
\providecommand{\Lref}{\ell_{\text{ref}}}

% --- tcolorbox Stile ---
\tcbset{
    keyresult/.style={
        colback=blue!5!white,
        colframe=blue!75!black,
        title=Kernaussage,
        fonttitle=\bfseries
    },
    foundation/.style={
        colback=green!5!white,
        colframe=green!75!black,
        title=Grundlage,
        fonttitle=\bfseries
    },
    alternative/.style={
        colback=orange!5!white,
        colframe=orange!75!black,
        title=Alternative,
        fonttitle=\bfseries
    },
    warningbox/.style={
        colback=red!5!white,
        colframe=red!75!black,
        title=Warnung,
        fonttitle=\bfseries
    }
}

\newtcolorbox{keyresultbox}[1][]{colback=blue!5!white,colframe=blue!75!black,fonttitle=\bfseries,title={#1},breakable}
\newtcolorbox{keyresult}[1][Kernaussage]{colback=blue!5!white,colframe=blue!75!black,fonttitle=\bfseries,title={#1},breakable}
\newtcolorbox{foundationbox}[1][]{colback=green!5!white,colframe=green!75!black,fonttitle=\bfseries,title={#1},breakable}
\newtcolorbox{foundation}[1][Grundlage]{colback=green!5!white,colframe=green!75!black,fonttitle=\bfseries,title={#1},breakable}
\newtcolorbox{alternativebox}[1][]{colback=orange!5!white,colframe=orange!75!black,fonttitle=\bfseries,title={#1},breakable}
\newtcolorbox{warningboxenv}[1][]{colback=red!5!white,colframe=red!75!black,fonttitle=\bfseries,title={#1},breakable}

% Benutzerdefinierte Boxen für Formeln
\newtcolorbox{fundamental}[1][]{
    colback=boxgray,
    colframe=t0blue,
    fonttitle=\bfseries,
    title=#1,
    sharp corners,
    boxrule=2pt
}

\newtcolorbox{neueperspektive}[1][]{
    colback=red!5!white,
    colframe=t0red,
    fonttitle=\bfseries,
    title=#1,
    sharp corners,
    boxrule=2pt
}

\newtcolorbox{formula}[1][]{
    colback=blue!5!white,
    colframe=blue!75!black,
    fonttitle=\bfseries,
    title=#1
}

\newtcolorbox{result}[1][]{
    colback=green!5!white,
    colframe=green!75!black,
    fonttitle=\bfseries,
    title=#1
}

% Zusätzliche tcolorbox-Umgebungen (aus T0_standalone_header_de.tex)
\newtcolorbox{derivation}[1][]{
    colback=green!5!white,
    colframe=green!75!black,
    title=#1,
    fonttitle=\bfseries,
    breakable
}

\newtcolorbox{summary}[1][]{
    colback=gray!10!white,
    colframe=gray!75!black,
    title=#1,
    fonttitle=\bfseries,
    breakable
}

\newtcolorbox{comparison}[1][]{
    colback=purple!5!white,
    colframe=purple!75!black,
    title=#1,
    fonttitle=\bfseries,
    breakable
}

\newtcolorbox{relation}[1][]{
    colback=cyan!5!white,
    colframe=cyan!75!black,
    title=#1,
    fonttitle=\bfseries,
    breakable
}

\newtcolorbox{principleBox}[1][]{
    colback=yellow!5!white,
    colframe=yellow!75!black,
    title=#1,
    fonttitle=\bfseries,
    breakable
}

% Hinweis: insight und discovery sind als Theorem-Umgebungen definiert
% insightBox und discoveryBox für tcolorbox-Versionen
\newtcolorbox{insightBox}[1][]{colback=blue!5,colframe=t0blue,title={#1},fonttitle=\bfseries,breakable}
\newtcolorbox{discoveryBox}[1][]{colback=green!5,colframe=t0green,title={#1},fonttitle=\bfseries,breakable}
\newtcolorbox{newperspective}[1][]{colback=yellow!5,colframe=orange,title={#1},fonttitle=\bfseries,breakable}
\newtcolorbox{revelation}[1][]{colback=red!5,colframe=t0red,title={#1},fonttitle=\bfseries,breakable}
\newtcolorbox{keypoint}[1][]{colback=blue!5,colframe=t0blue,title={#1},fonttitle=\bfseries,breakable}
\newtcolorbox{evidenceBox}[1][]{colback=green!5,colframe=t0green,title={#1},fonttitle=\bfseries,breakable}
\newtcolorbox{conclusionBox}[1][]{colback=gray!5,colframe=gray,title={#1},fonttitle=\bfseries,breakable}
\newtcolorbox{significance}[1][]{colback=yellow!5,colframe=orange,title={#1},fonttitle=\bfseries,breakable}
\newtcolorbox{philosophical}[1][]{colback=purple!5,colframe=purple,title={#1},fonttitle=\bfseries,breakable}
\newtcolorbox{implicationBox}[1][]{colback=cyan!5,colframe=cyan,title={#1},fonttitle=\bfseries,breakable}
\newtcolorbox{perspectiveBox}[1][]{colback=blue!5,colframe=t0blue,title={#1},fonttitle=\bfseries,breakable}
\newtcolorbox{revolutionary}[1][]{colback=red!5,colframe=t0red,title={#1},fonttitle=\bfseries,breakable}
\newtcolorbox{technical}[1][]{colback=gray!5,colframe=gray!75!black,title={#1},fonttitle=\bfseries,breakable}
\newtcolorbox{technicalBox}[1][]{colback=gray!5,colframe=gray!75!black,title={#1},fonttitle=\bfseries,breakable}
\newtcolorbox{notationBox}[1][]{colback=yellow!5,colframe=yellow!75!black,title={#1},fonttitle=\bfseries,breakable}
\newtcolorbox{verification}[1][]{colback=orange!5!white,colframe=orange!75!black,fonttitle=\bfseries,title=#1}
\newtcolorbox{explanationBox}[1][]{colback=purple!5!white,colframe=purple!75!black,fonttitle=\bfseries,title=#1}
\newtcolorbox{interpretationBox}[1][]{colback=cyan!5!white,colframe=cyan!75!black,fonttitle=\bfseries,title=#1}
\newtcolorbox{explanation}[1][]{colback=purple!5!white,colframe=purple!75!black,fonttitle=\bfseries,title=#1,breakable}
\newtcolorbox{interpretation}[1][]{colback=cyan!5!white,colframe=cyan!75!black,fonttitle=\bfseries,title=#1,breakable}
\newtcolorbox{proof_step}[1][]{colback=gray!5!white,colframe=gray!75!black,fonttitle=\bfseries,title=#1,breakable}
\newtcolorbox{experimental}[1][]{colback=teal!5!white,colframe=teal!75!black,fonttitle=\bfseries,title=#1,breakable}

% Zusätzliche Umgebungen
\newenvironment{treatise}{\begin{quote}}{\end{quote}}
\newenvironment{gemeinsam}{\begin{quote}}{\end{quote}}
\newenvironment{vergleich}{\begin{quote}}{\end{quote}}
\newenvironment{vorteil}{\begin{quote}}{\end{quote}}
\newenvironment{quantum}{\begin{quote}}{\end{quote}}

% Fehlende tcolorbox-Umgebungen
\newtcolorbox{important}[1][]{colback=red!5!white,colframe=red!75!black,title={#1},fonttitle=\bfseries,breakable}
\newtcolorbox{warning}[1][]{colback=orange!5!white,colframe=orange!75!black,title={#1},fonttitle=\bfseries,breakable}
\newtcolorbox{caution}[1][]{colback=yellow!5!white,colframe=yellow!75!black,title={#1},fonttitle=\bfseries,breakable}
\newtcolorbox{highlight}[1][]{colback=yellow!10!white,colframe=yellow!75!black,title={#1},fonttitle=\bfseries,breakable}
\newtcolorbox{critical}[1][]{colback=red!10!white,colframe=red!75!black,title={#1},fonttitle=\bfseries,breakable}
\newtcolorbox{analysis}[1][]{colback=blue!5!white,colframe=blue!75!black,title={#1},fonttitle=\bfseries,breakable}
\newtcolorbox{application}[1][]{colback=green!5!white,colframe=green!75!black,title={#1},fonttitle=\bfseries,breakable}
\newtcolorbox{experiment}[1][]{colback=cyan!5!white,colframe=cyan!75!black,title={#1},fonttitle=\bfseries,breakable}
\newtcolorbox{historical}[1][]{colback=brown!5!white,colframe=brown!75!black,title={#1},fonttitle=\bfseries,breakable}
\newtcolorbox{numerical}[1][]{colback=gray!5!white,colframe=gray!75!black,title={#1},fonttitle=\bfseries,breakable}
\newtcolorbox{overview}[1][]{colback=blue!5!white,colframe=blue!75!black,title={#1},fonttitle=\bfseries,breakable}
\newtcolorbox{speculation}[1][]{colback=purple!5!white,colframe=purple!75!black,title={#1},fonttitle=\bfseries,breakable}
\newtcolorbox{question}[1][]{colback=orange!5!white,colframe=orange!75!black,title={#1},fonttitle=\bfseries,breakable}
\newtcolorbox{method}[1][]{colback=teal!5!white,colframe=teal!75!black,title={#1},fonttitle=\bfseries,breakable}
\newtcolorbox{correct}[1][]{colback=green!10!white,colframe=green!75!black,title={#1},fonttitle=\bfseries,breakable}
\newtcolorbox{units}[1][]{colback=gray!5!white,colframe=gray!75!black,title={#1},fonttitle=\bfseries,breakable}
\newtcolorbox{achievement}[1][]{colback=gold!5!white,colframe=orange!75!black,title={#1},fonttitle=\bfseries,breakable}
\newtcolorbox{equivalence}[1][]{colback=cyan!5!white,colframe=cyan!75!black,title={#1},fonttitle=\bfseries,breakable}
\newtcolorbox{dimensional}[1][]{colback=purple!5!white,colframe=purple!75!black,title={#1},fonttitle=\bfseries,breakable}
\newtcolorbox{photon}[1][]{colback=yellow!5!white,colframe=yellow!75!black,title={#1},fonttitle=\bfseries,breakable}
\newtcolorbox{neutrino}[1][]{colback=blue!5!white,colframe=blue!75!black,title={#1},fonttitle=\bfseries,breakable}
\newtcolorbox{revolution}[1][]{colback=red!5!white,colframe=red!75!black,title={#1},fonttitle=\bfseries,breakable}
\newtcolorbox{t0box}[1][]{colback=blue!5!white,colframe=t0blue,title={#1},fonttitle=\bfseries,breakable}
\newtcolorbox{documentbox}[1][]{colback=gray!5!white,colframe=gray!75!black,title={#1},fonttitle=\bfseries,breakable}
\newtcolorbox{sibox}[1][]{colback=green!5!white,colframe=green!75!black,title={#1},fonttitle=\bfseries,breakable}
\newtcolorbox{smbox}[1][]{colback=blue!5!white,colframe=blue!75!black,title={#1},fonttitle=\bfseries,breakable}
\newtcolorbox{pvbox}[1][]{colback=purple!5!white,colframe=purple!75!black,title={#1},fonttitle=\bfseries,breakable}
\newtcolorbox{koidebox}[1][]{colback=orange!5!white,colframe=orange!75!black,title={#1},fonttitle=\bfseries,breakable}
\newtcolorbox{formel}[1][]{colback=blue!5!white,colframe=blue!75!black,title={#1},fonttitle=\bfseries,breakable}
\newtcolorbox{schluessel}[1][]{colback=blue!5!white,colframe=blue!75!black,title={#1},fonttitle=\bfseries,breakable}
\newtcolorbox{wichtig}[1][]{colback=red!5!white,colframe=red!75!black,title={#1},fonttitle=\bfseries,breakable}
\newtcolorbox{vorsicht}[1][]{colback=orange!5!white,colframe=orange!75!black,title={#1},fonttitle=\bfseries,breakable}
\newtcolorbox{revolutionaer}[1][]{colback=red!5!white,colframe=red!75!black,title={#1},fonttitle=\bfseries,breakable}
\newtcolorbox{numerisch}[1][]{colback=gray!5!white,colframe=gray!75!black,title={#1},fonttitle=\bfseries,breakable}
\newtcolorbox{experimentell}[1][]{colback=cyan!5!white,colframe=cyan!75!black,title={#1},fonttitle=\bfseries,breakable}
\newtcolorbox{anwendung}[1][]{colback=green!5!white,colframe=green!75!black,title={#1},fonttitle=\bfseries,breakable}
\newtcolorbox{alternative}[1][]{colback=orange!5!white,colframe=orange!75!black,title={#1},fonttitle=\bfseries,breakable}
\newtcolorbox{beziehung}[1][]{colback=cyan!5!white,colframe=cyan!75!black,title={#1},fonttitle=\bfseries,breakable}
\newtcolorbox{folgerung}[1][]{colback=green!5!white,colframe=green!75!black,title={#1},fonttitle=\bfseries,breakable}
\newtcolorbox{abhandlung}[1][]{colback=gray!5!white,colframe=gray!75!black,title={#1},fonttitle=\bfseries,breakable}
\newtcolorbox{prinzipBox}[1][]{colback=blue!5!white,colframe=blue!75!black,title={#1},fonttitle=\bfseries,breakable}
\newtcolorbox{beweis}[1][]{colback=gray!5!white,colframe=gray!75!black,title={#1},fonttitle=\bfseries,breakable}
\newtcolorbox{key}[2][]{colback=blue!5!white,colframe=blue!75!black,title={#2},fonttitle=\bfseries,breakable}
\newtcolorbox{category}[1][]{colback=purple!5!white,colframe=purple!75!black,title={#1},fonttitle=\bfseries,breakable}

% Zusätzliche T0-spezifische Befehle
\newcommand{\Tzero}{T$_0$}
\providecommand{\meff}{m_{\text{eff}}}
\newcommand{\Eabs}{E_{\text{abs}}}
\newcommand{\taupar}{\tau}

% Missing commands from various documents
\providecommand{\xikonst}{\xi_0}
\providecommand{\Phiphoton}{\Phi_{\gamma}}
\providecommand{\etavis}{\eta_{\text{vis}}}
\providecommand{\pichar}{\pi}
\providecommand{\primrel}{\mathcal{P}_{\text{rel}}}
\providecommand{\warningx}{\textcolor{orange}{\textbf{!}}}
\providecommand{\phiT}{\phi_T}
\providecommand{\xiT}{\xi_T}
\providecommand{\Lorentz}{\Lambda}
\providecommand{\Cconv}{C_{\text{conv}}}
\providecommand{\Df}{\Delta f}
\providecommand{\lambdazero}{\lambda_0}
\providecommand{\myapprox}{\approx}
\providecommand{\checked}{\checkmark}
\providecommand{\alphaWSI}{\alpha_W^{\text{SI}}}
\providecommand{\alphaWnat}{\alpha_W^{\text{nat}}}
\providecommand{\vect}[1]{\vec{#1}}
\providecommand{\Rzero}{R_0}
\providecommand{\Riem}{\mathcal{R}}
\providecommand{\nuzero}{\nu_0}
\providecommand{\mypi}{\pi}

% --- Layout-Einstellungen ---
\sloppy
\hfuzz=2pt
\vfuzz=2pt
\tolerance=1000
\emergencystretch=3em
\raggedbottom

% --- Inhaltsverzeichnis-Formatierung ---
\renewcommand{\cftsecfont}{\color{blue}}
\renewcommand{\cftsubsecfont}{\color{blue}}
\renewcommand{\cftsecpagefont}{\color{blue}}
\renewcommand{\cftsubsecpagefont}{\color{blue}}
\renewcommand{\cfttoctitlefont}{\huge\bfseries\color{blue}}

% --- Standard Kopf- und Fußzeilen ---
\pagestyle{fancy}
\fancyhf{}
\fancyhead[L]{\textsc{T0-Theorie}}
\fancyhead[R]{\textsc{J. Pascher}}
\fancyfoot[C]{\thepage}

% ==============================================================================
% Ende der Präambel
% ==============================================================================

 nach \documentclass.
% ==============================================================================

% --- Kodierung und Sprache ---
\usepackage[utf8]{inputenc}
\usepackage[T1]{fontenc}
\usepackage[ngerman]{babel}
\usepackage{lmodern}

% --- Seitengeometrie ---
\usepackage[a4paper, margin=2.5cm]{geometry}
\setlength{\headheight}{15pt}

% --- Mathematik und Physik ---
\usepackage{amsmath,amssymb,amsfonts,amsthm}
\usepackage{mathtools}
\usepackage{physics}
\usepackage{siunitx}
\sisetup{
    locale=DE,
    group-separator={.},
    output-decimal-marker={,},
    per-mode=symbol
}

% --- Grafiken und Tabellen ---
\usepackage{graphicx}
\usepackage[table,xcdraw]{xcolor}
\usepackage{tikz}
\usetikzlibrary{arrows.meta,positioning,shapes.geometric,decorations.pathmorphing,patterns,shapes.arrows,intersections}
\usepackage{pgfplots}
\pgfplotsset{compat=1.18}
\usepackage{quantikz}
\usepackage[most]{tcolorbox}
\tcbuselibrary{breakable}

% === WICHTIG: Algorithm-Konflikt umgehen ===
% Option: algorithmic mit GROSSBUCHSTABEN
% Gemeinsame Box für Experimente
\newtcolorbox{experimentbox}[1][]{
	colback=green!5!white,
	colframe=t0green!80!black,
	fonttitle=\bfseries,
	title={{#1}},
	breakable
}

% Abstract-Fallback
\ifdefined\abstract\else
\newenvironment{abstract}{\section*{\abstractname}\itshape\small\par\bigskip}{\bigskip}
\fi

% === MAKROS SICHER NEU DEFINIEREN / ÜBERSCHREIBEN ===
% Definiere Makros OHNE doppelte Subskripte
\newcommand{\phipar}{\phi_{\mathrm{par}}}
%\newcommand{\xipar}{\xi_{\mathrm{par}}}
\newcommand{\Qphipar}{Q_{\phi_{\mathrm{par}}}}
\newcommand{\rphipar}{r_{\phi_{\mathrm{par}}}}
\newcommand{\logphipar}{\log_{\phi_{\mathrm{par}}}}
\newcommand{\CHSH}{\text{CHSH}}
\usepackage{booktabs}
\usepackage{array}
\usepackage{longtable}
\usepackage{float}
\usepackage{adjustbox}
\usepackage{tabularx}
\usepackage{multirow}

% --- Dokumentformatierung ---
\usepackage{fancyhdr}
\renewcommand{\headrulewidth}{0.4pt}
\renewcommand{\footrulewidth}{0.4pt}
\usepackage{tocloft}
\usepackage{hyperref}
\usepackage{bookmark}
\usepackage{cleveref}
\usepackage{microtype}
\usepackage{enumitem}
\usepackage{setspace}
\usepackage{ragged2e}
\usepackage{multicol}

% --- Code und Algorithmen ---
\usepackage{algorithm}
\usepackage{algorithmic}
\usepackage{listings}
\usepackage{mdframed}

% --- Zitationsbefehle (Kompatibilität) ---
\providecommand{\citep}[1]{\cite{#1}}
\providecommand{\citet}[1]{\cite{#1}}

% --- Zusätzliche Pakete ---
\usepackage{pdflscape}
\usepackage{braket}
\usepackage{cancel}
\usepackage{caption}
\usepackage{csquotes}
\usepackage{gensymb}
\usepackage{hyphenat}
\usepackage{textcomp}
\usepackage{textgreek}
\usepackage{upgreek}
\usepackage{url}
% Hyphenation for URLs in bibliography
\def\UrlBreaks{\do\/\do-}
\usepackage{slashed}
\usepackage{bm}

% --- Fehlende Farben definieren ---
\definecolor{gold}{RGB}{255,215,0}

% --- Spaltentypen ---
\newcolumntype{L}[1]{>{\raggedright\arraybackslash}p{#1}}
\newcolumntype{C}[1]{>{\centering\arraybackslash}p{#1}}

% --- Unicode-Zeichen ---
\usepackage{newunicodechar}
\newunicodechar{ħ}{$\hbar$}
\newunicodechar{↔}{$\leftrightarrow$}
\newunicodechar{⇐}{$\Leftarrow$}
\newunicodechar{⇒}{$\Rightarrow$}
\newunicodechar{⇔}{$\Leftrightarrow$}
\newunicodechar{∂}{$\partial$}
\newunicodechar{∅}{$\emptyset$}
\newunicodechar{∇}{$\nabla$}
\newunicodechar{∈}{$\in$}
\newunicodechar{∉}{$\notin$}
\newunicodechar{∏}{$\prod$}
\newunicodechar{∑}{$\sum$}
\newunicodechar{√}{$\sqrt{}$}
\newunicodechar{∝}{$\propto$}
\newunicodechar{∞}{$\infty$}
\newunicodechar{∩}{$\cap$}
\newunicodechar{∪}{$\cup$}
\newunicodechar{∫}{$\int$}
\newunicodechar{≈}{$\approx$}
\newunicodechar{≠}{$\neq$}
\newunicodechar{≤}{$\leq$}
\newunicodechar{≥}{$\geq$}
\newunicodechar{ξ}{\ensuremath{\xi}}
\newunicodechar{μ}{\ensuremath{\mu}}
\newunicodechar{ψ}{\ensuremath{\psi}}
\newunicodechar{φ}{\ensuremath{\phi}}
\newunicodechar{π}{\ensuremath{\pi}}
\newunicodechar{λ}{\ensuremath{\lambda}}
\newunicodechar{Δ}{\ensuremath{\Delta}}

% --- Farben ---
\definecolor{blue}{rgb}{0,0,1}
\definecolor{boxgray}{RGB}{240,240,240}
\definecolor{deepblue}{RGB}{0,0,127}
\definecolor{deepgreen}{RGB}{0,127,0}
\definecolor{deepred}{RGB}{191,0,0}
\definecolor{t0blue}{RGB}{33,150,243}
\definecolor{t0green}{RGB}{76,175,80}
\definecolor{t0orange}{RGB}{255,152,0}
\definecolor{t0purple}{RGB}{156,39,176}
\definecolor{t0red}{RGB}{244,67,54}
\definecolor{t0yellow}{RGB}{255,204,0}

% --- Hyperref-Einstellungen ---
\hypersetup{
    colorlinks=true,
    linkcolor=blue,
    citecolor=blue,
    urlcolor=blue,
    breaklinks=true,
    bookmarksnumbered=true,
    pdfstartview=FitH
}

% --- Theorem-Umgebungen (Deutsch) ---
\theoremstyle{plain}
\newtheorem{satz}{Satz}[section]
\newtheorem{lemma}[satz]{Lemma}
\newtheorem{proposition}[satz]{Proposition}
\newtheorem{korollar}[satz]{Korollar}

\theoremstyle{definition}
\newtheorem{definition}[satz]{Definition}
\newtheorem{beispiel}[satz]{Beispiel}
\newtheorem{erkenntnis}[satz]{Erkenntnis}
\newtheorem{entdeckung}[satz]{Entdeckung}

\theoremstyle{remark}
\newtheorem{bemerkung}[satz]{Bemerkung}
\newtheorem{warnung}[satz]{Warnung}
\newtheorem{axiom}{Axiom}
\newtheorem{prinzip}{Prinzip}

% Aliases für englische Bezeichnungen
\newtheorem{theorem}[satz]{Theorem}
\newtheorem{corollary}[satz]{Corollary}
\newtheorem{remark}[satz]{Remark}
\newtheorem{example}[satz]{Example}
\newtheorem{insight}[satz]{Insight}
\newtheorem{discovery}[satz]{Discovery}
\newtheorem{principle}[satz]{Principle}

% --- T0-spezifische Befehle ---
\newcommand{\Tfield}{T(x,t)}
\providecommand{\Tfieldt}{T(\vec{x},t)}
\newcommand{\Efield}{E(x,t)}
\newcommand{\mfield}{m(x,t)}
\providecommand{\vecx}{\vec{x}}
\newcommand{\Lag}{\mathcal{L}}
\newcommand{\calL}{\mathcal{L}}
\newcommand{\alphaem}{\alpha}
\newcommand{\betaT}{\beta_T}
\newcommand{\xiT}{\xi}
\newcommand{\xipar}{\xi}
\newcommand{\Ezero}{E_0}
\newcommand{\EPlanck}{E_{\text{Pl}}}
\newcommand{\Mpl}{M_{\text{Pl}}}
\newcommand{\lP}{\ell_{\text{P}}}
\newcommand{\tP}{t_{\text{P}}}
\newcommand{\LPlanck}{\ell_{\text{Pl}}}
\newcommand{\TPlanck}{t_{\text{Pl}}}
\newcommand{\Gnat}{G_{\text{nat}}}
\newcommand{\alphaEM}{\alpha_{\text{EM}}}
\newcommand{\alphaSI}{\alpha_{\text{SI}}}
\newcommand{\Hubble}{H_0}
\newcommand{\LCDM}{\Lambda\text{CDM}}
\newcommand{\natunits}{(nat. Einheiten)}

% T0 Modell Parameter
\newcommand{\xigeom}{\xi_{\mathrm{geom}}}
\newcommand{\rzero}{r_{0}}
\newcommand{\xirat}{\xi_{\mathrm{rat}}}
\newcommand{\tzero}{t_{0}}
\newcommand{\Lambdat}{\Lambda_{\mathrm{t}}}
\newcommand{\EP}{E_{\mathrm{P}}}
\newcommand{\Emu}{E_{\mu}}
\newcommand{\Ee}{E_{e}}
\newcommand{\Etau}{E_{\tau}}
\newcommand{\alphafine}{\alpha_{\mathrm{fine}}}
\newcommand{\alphal}{\alpha_{\ell}}
\newcommand{\Lzero}{\ell_{0}}
\newcommand{\Lp}{\ell_{\mathrm{P}}}

% Zusätzliche Befehle
\newcommand{\Kfrak}{K_{\text{frak}}}
\newcommand{\Dfrak}{D_{\text{frak}}}
\newcommand{\betapar}{\beta_T}
\newcommand{\alphapar}{\alpha}
\newcommand{\deltafield}{\delta \phi}
\newcommand{\deltam}{\delta m}
\newcommand{\deltaE}{\delta E}
\newcommand{\Exi}{E_{\xi}}
\newcommand{\Lxi}{\ell_{\xi}}
\newcommand{\rhoCMB}{\rho_{\text{CMB}}}
\newcommand{\rhoCasimir}{\rho_{\text{Casimir}}}
\newcommand{\Leff}{L_{\text{eff}}}
\newcommand{\CQCD}{C_{\mathrm{QCD}}}
\newcommand{\Kspec}{K_{\mathrm{spec}}}

% Fehlende Befehle aus Dokumenten
\providecommand{\xiconst}{\xi_{\text{const}}}
\providecommand{\DhiggsT}{D_{\text{Higgs-T}}}
\providecommand{\rhoE}{\rho_{E}}
\providecommand{\Echar}{E_{\text{char}}}
\providecommand{\kfrac}{k_{\text{frac}}}
\providecommand{\alphaEMSI}{\alpha_{\text{EM,SI}}}
\providecommand{\alphaEMnat}{\alpha_{\text{EM,nat}}}
\providecommand{\betaTSI}{\beta_{T,\text{SI}}}
\providecommand{\betaTnat}{\beta_{T,\text{nat}}}
\providecommand{\Gsi}{G_{\text{SI}}}
\providecommand{\xiparSI}{\xi_{\text{SI}}}
\providecommand{\xiparnat}{\xi_{\text{nat}}}
\providecommand{\meff}{m_{\text{eff}}}
\providecommand{\Tzerot}{T_{0}(t)}
\providecommand{\mzerot}{m_{0}(t)}
\providecommand{\Ezeroabs}{E_{0,\text{abs}}}
\providecommand{\Epar}{E_{\text{par}}}
\providecommand{\Lnat}{\ell_{\text{nat}}}
\providecommand{\Tnat}{T_{\text{nat}}}
\providecommand{\xifrak}{\xi_{\text{frac}}}
\providecommand{\Tfrak}{T_{\text{frac}}}
\providecommand{\mfrak}{m_{\text{frac}}}
\providecommand{\Dfrac}{D_{\text{frac}}}
\providecommand{\EphotSI}{E_{\gamma,\text{SI}}}
\providecommand{\EphotNat}{E_{\gamma,\text{nat}}}
\providecommand{\Eabsint}{E_{\text{abs,int}}}
\providecommand{\mphoton}{m_{\gamma}}

% Zusätzliche fehlende Befehle aus Dokumenten
\providecommand{\Evis}{E_{\text{vis}}}
\providecommand{\Cto}{C_{T0}}
\providecommand{\mytimes}{\times}
\providecommand{\lambdah}{\lambda_h}
\providecommand{\checkmarkx}{\checkmark}
\providecommand{\Enorm}{E_{\text{norm}}}
\providecommand{\Tobs}{T_{\text{obs}}}
\providecommand{\mobs}{m_{\text{obs}}}
\providecommand{\Eobs}{E_{\text{obs}}}
\providecommand{\Lobs}{\ell_{\text{obs}}}
\providecommand{\xobs}{\xi_{\text{obs}}}
\providecommand{\calE}{\mathcal{E}}
\providecommand{\calT}{\mathcal{T}}
\providecommand{\calM}{\mathcal{M}}
\providecommand{\alphag}{\alpha_g}
\providecommand{\Tmax}{T_{\text{max}}}
\providecommand{\mmin}{m_{\text{min}}}
\providecommand{\Lmax}{\ell_{\text{max}}}
\providecommand{\Emin}{E_{\text{min}}}
\providecommand{\Geff}{G_{\text{eff}}}
\providecommand{\rhoeff}{\rho_{\text{eff}}}
\providecommand{\xieff}{\xi_{\text{eff}}}
\providecommand{\Teff}{T_{\text{eff}}}
\providecommand{\hPlanck}{h}
\providecommand{\kB}{k_B}
\providecommand{\muB}{\mu_B}
\providecommand{\lambdaC}{\lambda_C}
\providecommand{\omegaP}{\omega_P}
\providecommand{\rhoP}{\rho_P}
\providecommand{\Tref}{T_{\text{ref}}}
\providecommand{\Eref}{E_{\text{ref}}}
\providecommand{\mref}{m_{\text{ref}}}
\providecommand{\Lref}{\ell_{\text{ref}}}

% --- tcolorbox Stile ---
\tcbset{
    keyresult/.style={
        colback=blue!5!white,
        colframe=blue!75!black,
        title=Kernaussage,
        fonttitle=\bfseries
    },
    foundation/.style={
        colback=green!5!white,
        colframe=green!75!black,
        title=Grundlage,
        fonttitle=\bfseries
    },
    alternative/.style={
        colback=orange!5!white,
        colframe=orange!75!black,
        title=Alternative,
        fonttitle=\bfseries
    },
    warningbox/.style={
        colback=red!5!white,
        colframe=red!75!black,
        title=Warnung,
        fonttitle=\bfseries
    }
}

\newtcolorbox{keyresultbox}[1][]{colback=blue!5!white,colframe=blue!75!black,fonttitle=\bfseries,title={#1},breakable}
\newtcolorbox{keyresult}[1][Kernaussage]{colback=blue!5!white,colframe=blue!75!black,fonttitle=\bfseries,title={#1},breakable}
\newtcolorbox{foundationbox}[1][]{colback=green!5!white,colframe=green!75!black,fonttitle=\bfseries,title={#1},breakable}
\newtcolorbox{foundation}[1][Grundlage]{colback=green!5!white,colframe=green!75!black,fonttitle=\bfseries,title={#1},breakable}
\newtcolorbox{alternativebox}[1][]{colback=orange!5!white,colframe=orange!75!black,fonttitle=\bfseries,title={#1},breakable}
\newtcolorbox{warningboxenv}[1][]{colback=red!5!white,colframe=red!75!black,fonttitle=\bfseries,title={#1},breakable}

% Benutzerdefinierte Boxen für Formeln
\newtcolorbox{fundamental}[1][]{
    colback=boxgray,
    colframe=t0blue,
    fonttitle=\bfseries,
    title=#1,
    sharp corners,
    boxrule=2pt
}

\newtcolorbox{neueperspektive}[1][]{
    colback=red!5!white,
    colframe=t0red,
    fonttitle=\bfseries,
    title=#1,
    sharp corners,
    boxrule=2pt
}

\newtcolorbox{formula}[1][]{
    colback=blue!5!white,
    colframe=blue!75!black,
    fonttitle=\bfseries,
    title=#1
}

\newtcolorbox{result}[1][]{
    colback=green!5!white,
    colframe=green!75!black,
    fonttitle=\bfseries,
    title=#1
}

% Zusätzliche tcolorbox-Umgebungen (aus T0_standalone_header_de.tex)
\newtcolorbox{derivation}[1][]{
    colback=green!5!white,
    colframe=green!75!black,
    title=#1,
    fonttitle=\bfseries,
    breakable
}

\newtcolorbox{summary}[1][]{
    colback=gray!10!white,
    colframe=gray!75!black,
    title=#1,
    fonttitle=\bfseries,
    breakable
}

\newtcolorbox{comparison}[1][]{
    colback=purple!5!white,
    colframe=purple!75!black,
    title=#1,
    fonttitle=\bfseries,
    breakable
}

\newtcolorbox{relation}[1][]{
    colback=cyan!5!white,
    colframe=cyan!75!black,
    title=#1,
    fonttitle=\bfseries,
    breakable
}

\newtcolorbox{principleBox}[1][]{
    colback=yellow!5!white,
    colframe=yellow!75!black,
    title=#1,
    fonttitle=\bfseries,
    breakable
}

% Hinweis: insight und discovery sind als Theorem-Umgebungen definiert
% insightBox und discoveryBox für tcolorbox-Versionen
\newtcolorbox{insightBox}[1][]{colback=blue!5,colframe=t0blue,title={#1},fonttitle=\bfseries,breakable}
\newtcolorbox{discoveryBox}[1][]{colback=green!5,colframe=t0green,title={#1},fonttitle=\bfseries,breakable}
\newtcolorbox{newperspective}[1][]{colback=yellow!5,colframe=orange,title={#1},fonttitle=\bfseries,breakable}
\newtcolorbox{revelation}[1][]{colback=red!5,colframe=t0red,title={#1},fonttitle=\bfseries,breakable}
\newtcolorbox{keypoint}[1][]{colback=blue!5,colframe=t0blue,title={#1},fonttitle=\bfseries,breakable}
\newtcolorbox{evidenceBox}[1][]{colback=green!5,colframe=t0green,title={#1},fonttitle=\bfseries,breakable}
\newtcolorbox{conclusionBox}[1][]{colback=gray!5,colframe=gray,title={#1},fonttitle=\bfseries,breakable}
\newtcolorbox{significance}[1][]{colback=yellow!5,colframe=orange,title={#1},fonttitle=\bfseries,breakable}
\newtcolorbox{philosophical}[1][]{colback=purple!5,colframe=purple,title={#1},fonttitle=\bfseries,breakable}
\newtcolorbox{implicationBox}[1][]{colback=cyan!5,colframe=cyan,title={#1},fonttitle=\bfseries,breakable}
\newtcolorbox{perspectiveBox}[1][]{colback=blue!5,colframe=t0blue,title={#1},fonttitle=\bfseries,breakable}
\newtcolorbox{revolutionary}[1][]{colback=red!5,colframe=t0red,title={#1},fonttitle=\bfseries,breakable}
\newtcolorbox{technical}[1][]{colback=gray!5,colframe=gray!75!black,title={#1},fonttitle=\bfseries,breakable}
\newtcolorbox{technicalBox}[1][]{colback=gray!5,colframe=gray!75!black,title={#1},fonttitle=\bfseries,breakable}
\newtcolorbox{notationBox}[1][]{colback=yellow!5,colframe=yellow!75!black,title={#1},fonttitle=\bfseries,breakable}
\newtcolorbox{verification}[1][]{colback=orange!5!white,colframe=orange!75!black,fonttitle=\bfseries,title=#1}
\newtcolorbox{explanationBox}[1][]{colback=purple!5!white,colframe=purple!75!black,fonttitle=\bfseries,title=#1}
\newtcolorbox{interpretationBox}[1][]{colback=cyan!5!white,colframe=cyan!75!black,fonttitle=\bfseries,title=#1}
\newtcolorbox{explanation}[1][]{colback=purple!5!white,colframe=purple!75!black,fonttitle=\bfseries,title=#1,breakable}
\newtcolorbox{interpretation}[1][]{colback=cyan!5!white,colframe=cyan!75!black,fonttitle=\bfseries,title=#1,breakable}
\newtcolorbox{proof_step}[1][]{colback=gray!5!white,colframe=gray!75!black,fonttitle=\bfseries,title=#1,breakable}
\newtcolorbox{experimental}[1][]{colback=teal!5!white,colframe=teal!75!black,fonttitle=\bfseries,title=#1,breakable}

% Zusätzliche Umgebungen
\newenvironment{treatise}{\begin{quote}}{\end{quote}}
\newenvironment{gemeinsam}{\begin{quote}}{\end{quote}}
\newenvironment{vergleich}{\begin{quote}}{\end{quote}}
\newenvironment{vorteil}{\begin{quote}}{\end{quote}}
\newenvironment{quantum}{\begin{quote}}{\end{quote}}

% Fehlende tcolorbox-Umgebungen
\newtcolorbox{important}[1][]{colback=red!5!white,colframe=red!75!black,title={#1},fonttitle=\bfseries,breakable}
\newtcolorbox{warning}[1][]{colback=orange!5!white,colframe=orange!75!black,title={#1},fonttitle=\bfseries,breakable}
\newtcolorbox{caution}[1][]{colback=yellow!5!white,colframe=yellow!75!black,title={#1},fonttitle=\bfseries,breakable}
\newtcolorbox{highlight}[1][]{colback=yellow!10!white,colframe=yellow!75!black,title={#1},fonttitle=\bfseries,breakable}
\newtcolorbox{critical}[1][]{colback=red!10!white,colframe=red!75!black,title={#1},fonttitle=\bfseries,breakable}
\newtcolorbox{analysis}[1][]{colback=blue!5!white,colframe=blue!75!black,title={#1},fonttitle=\bfseries,breakable}
\newtcolorbox{application}[1][]{colback=green!5!white,colframe=green!75!black,title={#1},fonttitle=\bfseries,breakable}
\newtcolorbox{experiment}[1][]{colback=cyan!5!white,colframe=cyan!75!black,title={#1},fonttitle=\bfseries,breakable}
\newtcolorbox{historical}[1][]{colback=brown!5!white,colframe=brown!75!black,title={#1},fonttitle=\bfseries,breakable}
\newtcolorbox{numerical}[1][]{colback=gray!5!white,colframe=gray!75!black,title={#1},fonttitle=\bfseries,breakable}
\newtcolorbox{overview}[1][]{colback=blue!5!white,colframe=blue!75!black,title={#1},fonttitle=\bfseries,breakable}
\newtcolorbox{speculation}[1][]{colback=purple!5!white,colframe=purple!75!black,title={#1},fonttitle=\bfseries,breakable}
\newtcolorbox{question}[1][]{colback=orange!5!white,colframe=orange!75!black,title={#1},fonttitle=\bfseries,breakable}
\newtcolorbox{method}[1][]{colback=teal!5!white,colframe=teal!75!black,title={#1},fonttitle=\bfseries,breakable}
\newtcolorbox{correct}[1][]{colback=green!10!white,colframe=green!75!black,title={#1},fonttitle=\bfseries,breakable}
\newtcolorbox{units}[1][]{colback=gray!5!white,colframe=gray!75!black,title={#1},fonttitle=\bfseries,breakable}
\newtcolorbox{achievement}[1][]{colback=gold!5!white,colframe=orange!75!black,title={#1},fonttitle=\bfseries,breakable}
\newtcolorbox{equivalence}[1][]{colback=cyan!5!white,colframe=cyan!75!black,title={#1},fonttitle=\bfseries,breakable}
\newtcolorbox{dimensional}[1][]{colback=purple!5!white,colframe=purple!75!black,title={#1},fonttitle=\bfseries,breakable}
\newtcolorbox{photon}[1][]{colback=yellow!5!white,colframe=yellow!75!black,title={#1},fonttitle=\bfseries,breakable}
\newtcolorbox{neutrino}[1][]{colback=blue!5!white,colframe=blue!75!black,title={#1},fonttitle=\bfseries,breakable}
\newtcolorbox{revolution}[1][]{colback=red!5!white,colframe=red!75!black,title={#1},fonttitle=\bfseries,breakable}
\newtcolorbox{t0box}[1][]{colback=blue!5!white,colframe=t0blue,title={#1},fonttitle=\bfseries,breakable}
\newtcolorbox{documentbox}[1][]{colback=gray!5!white,colframe=gray!75!black,title={#1},fonttitle=\bfseries,breakable}
\newtcolorbox{sibox}[1][]{colback=green!5!white,colframe=green!75!black,title={#1},fonttitle=\bfseries,breakable}
\newtcolorbox{smbox}[1][]{colback=blue!5!white,colframe=blue!75!black,title={#1},fonttitle=\bfseries,breakable}
\newtcolorbox{pvbox}[1][]{colback=purple!5!white,colframe=purple!75!black,title={#1},fonttitle=\bfseries,breakable}
\newtcolorbox{koidebox}[1][]{colback=orange!5!white,colframe=orange!75!black,title={#1},fonttitle=\bfseries,breakable}
\newtcolorbox{formel}[1][]{colback=blue!5!white,colframe=blue!75!black,title={#1},fonttitle=\bfseries,breakable}
\newtcolorbox{schluessel}[1][]{colback=blue!5!white,colframe=blue!75!black,title={#1},fonttitle=\bfseries,breakable}
\newtcolorbox{wichtig}[1][]{colback=red!5!white,colframe=red!75!black,title={#1},fonttitle=\bfseries,breakable}
\newtcolorbox{vorsicht}[1][]{colback=orange!5!white,colframe=orange!75!black,title={#1},fonttitle=\bfseries,breakable}
\newtcolorbox{revolutionaer}[1][]{colback=red!5!white,colframe=red!75!black,title={#1},fonttitle=\bfseries,breakable}
\newtcolorbox{numerisch}[1][]{colback=gray!5!white,colframe=gray!75!black,title={#1},fonttitle=\bfseries,breakable}
\newtcolorbox{experimentell}[1][]{colback=cyan!5!white,colframe=cyan!75!black,title={#1},fonttitle=\bfseries,breakable}
\newtcolorbox{anwendung}[1][]{colback=green!5!white,colframe=green!75!black,title={#1},fonttitle=\bfseries,breakable}
\newtcolorbox{alternative}[1][]{colback=orange!5!white,colframe=orange!75!black,title={#1},fonttitle=\bfseries,breakable}
\newtcolorbox{beziehung}[1][]{colback=cyan!5!white,colframe=cyan!75!black,title={#1},fonttitle=\bfseries,breakable}
\newtcolorbox{folgerung}[1][]{colback=green!5!white,colframe=green!75!black,title={#1},fonttitle=\bfseries,breakable}
\newtcolorbox{abhandlung}[1][]{colback=gray!5!white,colframe=gray!75!black,title={#1},fonttitle=\bfseries,breakable}
\newtcolorbox{prinzipBox}[1][]{colback=blue!5!white,colframe=blue!75!black,title={#1},fonttitle=\bfseries,breakable}
\newtcolorbox{beweis}[1][]{colback=gray!5!white,colframe=gray!75!black,title={#1},fonttitle=\bfseries,breakable}
\newtcolorbox{key}[2][]{colback=blue!5!white,colframe=blue!75!black,title={#2},fonttitle=\bfseries,breakable}
\newtcolorbox{category}[1][]{colback=purple!5!white,colframe=purple!75!black,title={#1},fonttitle=\bfseries,breakable}

% Zusätzliche T0-spezifische Befehle
\newcommand{\Tzero}{T$_0$}
\providecommand{\meff}{m_{\text{eff}}}
\newcommand{\Eabs}{E_{\text{abs}}}
\newcommand{\taupar}{\tau}

% Missing commands from various documents
\providecommand{\xikonst}{\xi_0}
\providecommand{\Phiphoton}{\Phi_{\gamma}}
\providecommand{\etavis}{\eta_{\text{vis}}}
\providecommand{\pichar}{\pi}
\providecommand{\primrel}{\mathcal{P}_{\text{rel}}}
\providecommand{\warningx}{\textcolor{orange}{\textbf{!}}}
\providecommand{\phiT}{\phi_T}
\providecommand{\xiT}{\xi_T}
\providecommand{\Lorentz}{\Lambda}
\providecommand{\Cconv}{C_{\text{conv}}}
\providecommand{\Df}{\Delta f}
\providecommand{\lambdazero}{\lambda_0}
\providecommand{\myapprox}{\approx}
\providecommand{\checked}{\checkmark}
\providecommand{\alphaWSI}{\alpha_W^{\text{SI}}}
\providecommand{\alphaWnat}{\alpha_W^{\text{nat}}}
\providecommand{\vect}[1]{\vec{#1}}
\providecommand{\Rzero}{R_0}
\providecommand{\Riem}{\mathcal{R}}
\providecommand{\nuzero}{\nu_0}
\providecommand{\mypi}{\pi}

% --- Layout-Einstellungen ---
\sloppy
\hfuzz=2pt
\vfuzz=2pt
\tolerance=1000
\emergencystretch=3em
\raggedbottom

% --- Inhaltsverzeichnis-Formatierung ---
\renewcommand{\cftsecfont}{\color{blue}}
\renewcommand{\cftsubsecfont}{\color{blue}}
\renewcommand{\cftsecpagefont}{\color{blue}}
\renewcommand{\cftsubsecpagefont}{\color{blue}}
\renewcommand{\cfttoctitlefont}{\huge\bfseries\color{blue}}

% --- Standard Kopf- und Fußzeilen ---
\pagestyle{fancy}
\fancyhf{}
\fancyhead[L]{\textsc{T0-Theorie}}
\fancyhead[R]{\textsc{J. Pascher}}
\fancyfoot[C]{\thepage}

% ==============================================================================
% Ende der Präambel
% ==============================================================================



\title{\Huge\textbf{T0-Theorie: Rotverschiebungsmechanismus}\\
	\Large Wellenl\"angenabh\"angige Rotverschiebung \\
	ohne Entfernungsannahmen oder r\"aumliche Expansion}
\author{}
\date{}
%\begin{document}
	
	\maketitle
	
	\begin{abstract}
		Das T0-Modell erkl\"art die kosmologische Rotverschiebung durch $\xi$-Feld-Energieverlust w\"ahrend der Photonenausbreitung, ohne r\"aumliche Expansion oder Entfernungsmessungen zu ben\"otigen. Dieser Mechanismus sagt eine wellenl\"angenabh\"angige Rotverschiebung $z \propto \lambda$ vorher, die mit spektroskopischen Beobachtungen kosmischer Objekte getestet werden kann. Unter Verwendung der universellen Konstante $\xiconst$ und gemessener Massen astronomischer Objekte liefert die Theorie modellunabh\"angige Tests, die von der Standardkosmologie unterscheidbar sind. Das $\xi$-Feld erkl\"art auch die kosmische Mikrowellen-Hintergrundtemperatur ($T_{\text{CMB}} = 2,7255$ K) in einem statischen, ewig existierenden Universum, wie in \cite{pascher2025} detailliert beschrieben.
	\end{abstract}
	
	\tableofcontents
	\newpage
	
	\section{Fundamentaler $\xi$-Feld-Energieverlust}
	\label{sec:xi_field}
	
	\subsection{Grundmechanismus}
	
	\begin{principle}[$\xi$-Feld-Photonen-Wechselwirkung]
		Photonen verlieren Energie durch Wechselwirkung mit dem universellen $\xi$-Feld w\"ahrend der Ausbreitung:
		\begin{equation}
			\frac{dE}{dx} = -\xi \cdot f\left(\frac{E}{\Exi}\right) \cdot E
		\end{equation}
		wobei $\xiconst$ die universelle geometrische Konstante ist und $\Exi = \frac{1}{\xi} = 7500$ (nat\"urliche Einheiten).
	\end{principle}
	
	Die Kopplungsfunktion $f(E/\Exi)$ ist dimensionslos und beschreibt die energieabh\"angige Wechselwirkungsst\"arke. F\"ur den linearen Kopplungsfall:
	\begin{equation}
		f\left(\frac{E}{\Exi}\right) = \frac{E}{\Exi}
	\end{equation}
	
	Dies ergibt die vereinfachte Energieverlustgleichung:
	\begin{equation}
		\frac{dE}{dx} = -\frac{\xi E^2}{\Exi}
	\end{equation}
	
	\subsection{Energie-zu-Wellenl\"ange-Umwandlung}
	
	Da $E = \frac{hc}{\lambda}$ (oder $E = \frac{1}{\lambda}$ in nat\"urlichen Einheiten, $\hbar = c = 1$), k\"onnen wir den Energieverlust in Bezug auf die Wellenl\"ange ausdr\"ucken. Einsetzen von $E = \frac{1}{\lambda}$:
	\begin{equation}
		\frac{d(1/\lambda)}{dx} = -\frac{\xi}{\Exi} \cdot \frac{1}{\lambda^2}
	\end{equation}
	
	Umstellung zur Wellenl\"angenentwicklung:
	\begin{equation}
		\frac{d\lambda}{dx} = \frac{\xi \lambda^2}{\Exi}
	\end{equation}
	
	\section{Rotverschiebungsformel-Ableitung}
	
	\subsection{Integration f\"ur kleine $\xi$-Effekte}
	
	F\"ur die Wellenl\"angenentwicklungsgleichung:
	\begin{equation}
		\frac{d\lambda}{dx} = \frac{\xi \lambda^2}{\Exi}
	\end{equation}
	
	Trennung der Variablen und Integration:
	\begin{equation}
		\int_{\lambdazero}^{\lambda} \frac{d\lambda'}{\lambda'^2} = \frac{\xi}{\Exi} \int_0^x dx'
	\end{equation}
	
	Dies ergibt:
	\begin{equation}
		\frac{1}{\lambdazero} - \frac{1}{\lambda} = \frac{\xi x}{\Exi}
	\end{equation}
	
	L\"osung f\"ur die beobachtete Wellenl\"ange:
	\begin{equation}
		\lambda = \frac{\lambdazero}{1 - \frac{\xi x \lambdazero}{\Exi}}
	\end{equation}
	
	\subsection{Rotverschiebungsdefinition und Formel}
	
	\begin{formula}
		Rotverschiebungsdefinition:
		\begin{equation}
			z = \frac{\lambda_{\text{beobachtet}} - \lambda_{\text{emittiert}}}{\lambda_{\text{emittiert}}} = \frac{\lambda}{\lambdazero} - 1
		\end{equation}
	\end{formula}
	
	F\"ur kleine $\xi$-Effekte, wo $\frac{\xi x \lambdazero}{\Exi} \ll 1$, k\"onnen wir entwickeln:
	\begin{equation}
		z \approx \frac{\xi x \lambdazero}{\Exi} = \frac{\xi x}{\Exi / (\hbar c)} \cdot \lambdazero \quad (\text{in konventionellen Einheiten})
	\end{equation}
	
	\begin{important}
		\textbf{Schl\"ussel-T0-Vorhersage: Wellenl\"angenabh\"angige Rotverschiebung}
		\begin{equation}
			\boxed{z(\lambdazero) = \frac{\xi x}{\Exi} \cdot \lambdazero \quad (\text{nat\"urliche Einheiten, } \hbar = c = 1)}
		\end{equation}
		Diese Wellenl\"angenabh\"angigkeit ist das ENTSCHEIDENDE UNTERSCHEIDUNGSMERKMAL zur Standardkosmologie:
		\begin{itemize}
			\item Standardkosmologie: $z$ ist gleich f\"ur ALLE Wellenl\"angen derselben Quelle
			\item T0-Theorie: $z$ variiert mit der Wellenl\"ange - testbare Vorhersage!
		\end{itemize}
		In konventionellen Einheiten wird $\Exi$ mit $\hbar c \approx 197,3$ MeV$\cdot$fm skaliert, sodass $\Exi \approx 1,5$ GeV $\Exi / (\hbar c) \approx 7500$ m$^{-1}$ entspricht, was dimensionale Konsistenz gew\"ahrleistet.
	\end{important}
	
	\subsection{Konsistenz mit beobachteten Rotverschiebungen}
	Aktuelle Beobachtungen best\"atigen oder widerlegen die Wellenl\"angenabh\"angigkeit aufgrund von Messbegrenzungen an der Nachweisschwelle weder. Die wellenl\"angenabh\"angige Rotverschiebung, gegeben durch $z \propto \frac{\xi x}{\Exi} \cdot \lambdazero$, erkl\"art beobachtete kosmologische Rotverschiebungen in Kombination mit erg\"anzenden Effekten wie Doppler-Verschiebungen, Gravitationsrotverschiebung und nichtlinearen $\xi$-Feld-Wechselwirkungen. F\"ur Objekte mit hoher Rotverschiebung ($z > 10$), wie sie von JWST beobachtet wurden \cite{jwst_early}, kann die Kopplungsfunktion $f\left(\frac{E}{\Exi}\right)$ h\"ohere Ordnungsterme enthalten, die Konsistenz mit Beobachtungen ohne kosmische Expansion gew\"ahrleisten. Zuk\"unftige spektroskopische Tests, wie in Abschnitt \ref{sec:experimental_tests} beschrieben, werden eine definitive Validierung oder Widerlegung dieses Mechanismus liefern.
	
	\section{Frequenzbasierte Formulierung}
	
	\subsection{Frequenz-Energieverlust}
	
	Da $E = h\nu$, wird die Energieverlustgleichung zu:
	\begin{equation}
		\frac{d(h\nu)}{dx} = -\frac{\xi (h\nu)^2}{\Exi}
	\end{equation}
	
	Vereinfachung:
	\begin{equation}
		\frac{d\nu}{dx} = -\frac{\xi h \nu^2}{\Exi}
	\end{equation}
	
	\subsection{Frequenz-Rotverschiebungsformel}
	
	Integration der Frequenzentwicklung:
	\begin{equation}
		\int_{\nuzero}^{\nu} \frac{d\nu'}{\nu'^2} = -\frac{\xi h}{\Exi} \int_0^x dx'
	\end{equation}
	
	Dies ergibt:
	\begin{equation}
		\frac{1}{\nu} - \frac{1}{\nuzero} = \frac{\xi h x}{\Exi}
	\end{equation}
	
	Daher:
	\begin{equation}
		\nu = \frac{\nuzero}{1 + \frac{\xi h x \nuzero}{\Exi}}
	\end{equation}
	
	\begin{formula}
		Frequenz-Rotverschiebung:
		\begin{equation}
			z = \frac{\nuzero}{\nu} - 1 \approx \frac{\xi h x \nuzero}{\Exi} \quad (\text{nat\"urliche Einheiten, } h = 1; \text{konventionelle Einheiten, } h = \hbar)
		\end{equation}
	\end{formula}
	
	\begin{important}
		Da $\nu = \frac{c}{\lambda}$, haben wir $h\nu = \frac{hc}{\lambda}$, was best\"atigt:
		\begin{equation}
			z \propto \nu \propto \frac{1}{\lambda}
		\end{equation}
		\textbf{H\"oherfrequente Photonen zeigen gr\"o\ss{}ere Rotverschiebung!} In konventionellen Einheiten wird $\Exi$ mit $\hbar c$ skaliert, um dimensionale Konsistenz zu erhalten.
	\end{important}
	
	\section{Beobachtbare Vorhersagen ohne Entfernungsannahmen}
	
	\subsection{Spektrallinienverh\"altnisse}
	
	Verschiedene atomare \"Uberg\"ange sollten unterschiedliche Rotverschiebungen gem\"a\ss{} ihrer Wellenl\"angen zeigen:
	\begin{equation}
		\frac{z(\lambda_1)}{z(\lambda_2)} = \frac{\lambda_1}{\lambda_2}
	\end{equation}
	
	\begin{experiment}
		\textbf{Wasserstofflinien-Test:}
		\begin{itemize}
			\item Lyman-$\alpha$ (121,6 nm) vs. H$\alpha$ (656,3 nm)
			\item Vorhergesagtes Verh\"altnis: $\frac{z_{\text{Ly}\alpha}}{z_{\text{H}\alpha}} = \frac{121,6}{656,3} = 0,185$
			\item \textbf{Standardkosmologie sagt vorher: 1,000}
		\end{itemize}
	\end{experiment}
	
	\subsection{Frequenzabh\"angige Effekte}
	
	F\"ur Radio- vs. optische Beobachtungen desselben kosmischen Objekts:
	\begin{itemize}
		\item 21 cm Linie: $\lambda = 0,21$ m
		\item H$\alpha$ Linie: $\lambda = 6,563 \times 10^{-7}$ m
		\item Vorhergesagtes Verh\"altnis: $\frac{z_{21\text{cm}}}{z_{\text{H}\alpha}} = \frac{\lambda_{21\text{cm}}}{\lambda_{\text{H}\alpha}} = \frac{0,21}{6,563 \times 10^{-7}} = 3,2 \times 10^5$
	\end{itemize}
	
	Dieser enorme Unterschied sollte selbst mit aktueller Technologie nachweisbar sein, wenn der T0-Mechanismus korrekt ist.
	
	\section{Experimentelle Tests mittels Spektroskopie}
	\label{sec:experimental_tests}
	
	\subsection{Multiwellenl\"angen-Beobachtungen}
	
	\begin{experiment}
		\textbf{Simultane Multiband-Spektroskopie:}
		\begin{enumerate}
			\item Beobachtung von Quasar/Galaxie simultan in UV, optisch, IR
			\item Messung der Rotverschiebung aus verschiedenen Spektrallinien
			\item Test ob $z \propto \lambda$ Beziehung gilt
			\item Vergleich mit Standardkosmologie-Vorhersage ($z = \text{konstant}$)
		\end{enumerate}
	\end{experiment}
	
	\subsection{Radio vs. optische Rotverschiebung}
	
	\begin{experiment}
		\textbf{21cm vs. optische Linien-Vergleich:}
		\begin{itemize}
			\item \textbf{Radio-Durchmusterungen}: ALFALFA, HIPASS (21cm Rotverschiebungen)
			\item \textbf{Optische Durchmusterungen}: SDSS, 2dF (H$\alpha$, H$\beta$ Rotverschiebungen)
			\item \textbf{Methode}: Vergleich von Objekten in beiden Durchmusterungen beobachtet
			\item \textbf{Vorhersage}: $z_{21\text{cm}} \neq z_{\text{optisch}}$ (T0) vs. $z_{21\text{cm}} = z_{\text{optisch}}$ (Standard)
		\end{itemize}
	\end{experiment}
	
	\section{Vorteile gegen\"uber der Standardkosmologie}
	
	\subsection{Modellunabh\"angiger Ansatz}
	
	\begin{longtable}{lcc}
		\caption{T0-Theorie vs. Standardkosmologie} \\
		\toprule
		\textbf{Aspekt} & \textbf{T0-Theorie} & \textbf{$\Lambda$CDM} \\
		\midrule
		\endfirsthead
		\multicolumn{3}{c}%
		{{\tablename\ \thetable{} -- Fortsetzung von vorheriger Seite}} \\
		\toprule
		\textbf{Aspekt} & \textbf{T0-Theorie} & \textbf{$\Lambda$CDM} \\
		\midrule
		\endhead
		\bottomrule
		\endfoot
		\bottomrule
		\endlastfoot
		Universelle Konstante & $\xi = 4/3 \times 10^{-4}$ & Keine \\
		Dunkle Energie erforderlich & Nein & Ja (70\%) \\
		Dunkle Materie erforderlich & Nein & Ja (25\%) \\
		Anzahl der Parameter & 1 & 6+ \\
		Hubble-Spannung & Gel\"ost & Ungel\"ost \\
		JWST-Beobachtungen & Konsistent & Problematisch \\
		Urknall-Singularit\"at & Keine & Erforderlich \\
		Horizontproblem & Keines & Ungel\"ost \\
		Flachheitsproblem & Nat\"urlich & Feinabstimmung erforderlich \\
	\end{longtable}
	
	\subsection{Vereinheitlichte Erkl\"arungen}
	
	Die einzelne $\xi$-Konstante erkl\"art:
	\begin{enumerate}
		\item \textbf{Gravitationskonstante}: $G = \frac{\xi^2 c^3}{16\pi m_p^2}$
		\item \textbf{CMB-Temperatur}: $T_{\text{CMB}} = \frac{16}{9} \xi^2 \times E_\xi$
		\item \textbf{Casimir-Effekt}: Bezogen auf $\xi$-Feld-Vakuum
		\item \textbf{Kosmologische Rotverschiebung}: Energieverlust durch $\xi$-Feld
		\item \textbf{Teilchenmassen}: Geometrische Resonanzen im $\xi$-Feld
		\item \textbf{Feinstrukturkonstante}: $\alpha = (4/3)^3 \approx 1/137$
		\item \textbf{Myon anomales magnetisches Moment}: $a_\mu = \frac{\xi}{2\pi} \left(\frac{E_\mu}{E_e}\right)^2$
	\end{enumerate}
	
	\section{Kritische Bewertung: Wellenl\"angenabh\"angigkeit an der Nachweisschwelle}
	\label{sec:wavelength_assessment}
	
	\subsection{Aktueller experimenteller Status und Messbegrenzungen}
	
	Die Vorhersage der T0-Theorie einer wellenl\"angenabh\"angigen Rotverschiebung stellt eines ihrer markantesten und testbarsten Merkmale dar. Die aktuelle experimentelle Situation ist jedoch komplex und erfordert eine sorgf\"altige Analyse.
	
	\subsubsection{Pr\"azision an der kritischen Grenze}
	
	Aktuelle spektroskopische Messungen erreichen eine Pr\"azision von $\Delta z/z \approx 10^{-4}$ bis $10^{-5}$, w\"ahrend der T0-Effekt mit $\xi = 4/3 \times 10^{-4}$ Variationen derselben Gr\"o\ss{}enordnung vorhersagt. Dies platziert uns genau an der Nachweisschwelle - eine kritische Situation, in der weder Best\"atigung noch Widerlegung derzeit m\"oglich ist.
	
	F\"ur typische kosmische Objekte mit $\xiconst$ ist der relative Unterschied in der Rotverschiebung zwischen zwei Spektrallinien:
	\begin{equation}
		\frac{\Delta z}{z} = \left| \frac{z(\lambda_1) - z(\lambda_2)}{z(\lambda_{\text{mittel}})} \right| = \left| \frac{\lambda_1 - \lambda_2}{\lambda_{\text{mittel}}} \right| \times \xi \approx 10^{-4} \text{ bis } 10^{-5}
	\end{equation}
	
	\begin{important}
		Dieser Wellenl\"angeneffekt liegt an der Grenze der aktuellen spektroskopischen Pr\"azision, ist aber potenziell nachweisbar mit Instrumenten der n\"achsten Generation:
		\begin{itemize}
			\item Extremely Large Telescope (ELT): $\Delta z/z \approx 10^{-6}$ bis $10^{-7}$
			\item James Webb Space Telescope (JWST): Erweiterte IR-Spektroskopie
			\item Square Kilometre Array (SKA): Pr\"azise 21cm-Messungen
		\end{itemize}
	\end{important}
	
	\subsection{Zuk\"unftige experimentelle Ergebnisse und ihre Implikationen}
	
	Die n\"achste Generation von Instrumenten wird eine Pr\"azision von $\Delta z/z \approx 10^{-6}$ bis $10^{-7}$ erreichen und endlich definitive Tests erm\"oglichen. Zwei prim\"are Ergebnisse sind m\"oglich:
	
	\subsubsection{Prim\"ares Ergebnis A: Wellenl\"angenabh\"angigkeit BEST\"ATIGT}
	\label{subsubsec:confirmed}
	
	Wenn Messungen $z \propto \lambda_0$ wie vorhergesagt detektieren:
	
	\textbf{Unmittelbare Implikationen:}
	\begin{itemize}
		\item \textbf{Fundamentale Validierung} des T0-Kernmechanismus
		\item \textbf{Paradigmenwechsel}: Rotverschiebung durch Energieverlust, nicht Expansion
		\item \textbf{Neue Physik best\"atigt}: Photon-$\xi$-Feld-Wechselwirkung ist real
		\item \textbf{Kosmologie-Revolution}: Statisches Universumsmodell validiert
	\end{itemize}
	
	\textbf{Erforderliche Folgemessungen:}
	\begin{itemize}
		\item Pr\"azise Bestimmung der Proportionalit\"atskonstante zur Verifikation von $\xi = 4/3 \times 10^{-4}$
		\item Entfernungsabh\"angigkeit zur Best\"atigung der linearen Beziehung
		\item Suche nach Abweichungen bei extremen Wellenl\"angen (Gammastrahlen bis Radio)
	\end{itemize}
	
	\subsubsection{Prim\"ares Ergebnis B: Wellenl\"angenabh\"angigkeit NICHT DETEKTIERT}
	\label{subsubsec:not_detected}
	
	Wenn keine Wellenl\"angenabh\"angigkeit selbst bei $10^{-6}$ Pr\"azision gefunden wird, m\"ussen zwei verschiedene Unterszenarien betrachtet werden:
	
	\subsection{Unter-Szenario B1: Fundamentaler T0-Mechanismus inkorrekt}
	\label{subsec:scenario_b1}
	
	\textbf{Interpretation:} Der nichtlineare Energieverlustmechanismus $dE/dx = -\xi E^2/E_\xi$ ist fundamental falsch.
	
	\textbf{Erforderliche theoretische Anpassung:}
	\begin{itemize}
		\item \textbf{Modifizierte Energieverlustgleichung:} Ersetzen durch lineare Form
		\begin{equation}
			\frac{dE}{dx} = -\xi_{eff} \cdot E
		\end{equation}
		Dies ergibt $z = e^{\xi_{eff} x} - 1$, unabh\"angig von $\lambda_0$
		
		\item \textbf{Neuinterpretation von $E_\xi$:} Nicht l\"anger eine fundamentale Energieskala f\"ur Photonenwechselwirkung
		
		\item \textbf{Alternative Kopplungsfunktion:} Statt $f(E/E_\xi) = E/E_\xi$, verwende
		\begin{equation}
			f(E/E_\xi) = \text{konstant} = \xi_0
		\end{equation}
	\end{itemize}
	
	\textbf{Was g\"ultig bleibt:}
	\begin{itemize}
		\item Geometrische Konstante $\xi = 4/3 \times 10^{-4}$ (aus Tetraeder-Quantisierung)
		\item Gravitationskonstanten-Ableitung: $G = \xi^2 c^3/(16\pi m_p^2)$
		\item Teilchenmassen-Verh\"altnisse aus geometrischen Quantenzahlen
		\item Myon g-2 Anomalie-Vorhersage
		\item CMB-Temperatur-Erkl\"arung
	\end{itemize}
	
	\textbf{Was sich \"andert:}
	\begin{itemize}
		\item Verlust der einzigartigen T0-Signatur (Wellenl\"angenabh\"angigkeit)
		\item Schwieriger von modifizierten $\Lambda$CDM-Modellen zu unterscheiden
		\item Photonen-Ausbreitungsmechanismus vereinfacht
		\item Alternative Tests zur Validierung des statischen Universumsmodells n\"otig
	\end{itemize}
	
	\subsection{Unter-Szenario B2: Wellenl\"angenabh\"angigkeit existiert, ist aber KOMPENSIERT}
	\label{subsec:scenario_b2}
	
	\textbf{Interpretation:} Der T0-Mechanismus ist korrekt, aber kompensierende Effekte maskieren die Wellenl\"angenabh\"angigkeit.
	
	\subsubsection{Detaillierte Kompensationsmechanismen}
	
	\begin{formula}[title=Drei Kompensationsmechanismen]
		Die T0-Wellenl\"angenabh\"angigkeit k\"onnte maskiert sein durch:
		\begin{enumerate}
			\item \textbf{IGM-Dispersion}: $z_{\text{IGM}} \propto -\lambda^{-2}$ (wirkt $z_{\text{T0}} \propto +\lambda$ entgegen)
			\item \textbf{Gravitations-Schichtung}: $z_{\text{grav}}(r(\lambda))$ variiert mit Emissionstiefe
			\item \textbf{Nichtlineare Korrekturen}: H\"ohere Ordnungsterme $\propto (\xi x \lambda_0/E_\xi)^n$ fl\"achen Antwort ab
		\end{enumerate}
		Nettoeffekt: $z_{\text{beobachtet}} = z_{\text{T0}} + z_{\text{komp}} \approx$ konstant
	\end{formula}
	
	\textbf{1. Intergalaktisches Medium (IGM) Dispersionskompensation:}
	\begin{equation}
		z_{\text{beobachtet}} = z_{\text{T0}}(\lambda) + z_{\text{IGM}}(\lambda) + z_{\text{andere}}
	\end{equation}
	
	Das IGM k\"onnte inverse Wellenl\"angenabh\"angigkeit liefern:
	\begin{itemize}
		\item T0-Effekt: $z_{\text{T0}} \propto +\lambda$ (l\"angere Wellenl\"angen st\"arker rotverschoben)
		\item IGM-Effekt: $z_{\text{IGM}} \propto -\lambda^{-2}$ (Plasmadispersion bevorzugt k\"urzere Wellenl\"angen)
		\item Nettoergebnis: $z_{\text{beobachtet}} \approx$ konstant
	\end{itemize}
	
	\textbf{Physikalischer Mechanismus:} Freie Elektronen im IGM erzeugen frequenzabh\"angigen Brechungsindex:
	\begin{equation}
		n(\omega) = 1 - \frac{\omega_p^2}{2\omega^2} \implies z_{\text{IGM}} \propto -\frac{1}{\lambda^2}
	\end{equation}
	
	F\"ur angemessene IGM-Dichte k\"onnte dies T0s lineare $\lambda$-Abh\"angigkeit pr\"azise aufheben.
	
	\textbf{2. Quellenabh\"angige Kompensation:}
	
	Verschiedene Spektrallinien entstehen in verschiedenen Tiefen stellarer/galaktischer Atmosph\"aren:
	\begin{itemize}
		\item \textbf{UV-Linien} (z.B. Lyman-$\alpha$): \"Au\ss{}ere Atmosph\"are, niedrigere Gravitation, weniger Gravitationsrotverschiebung
		\item \textbf{Optische Linien} (z.B. H-$\alpha$): Mittlere Photosph\"are, moderates Gravitationsfeld
		\item \textbf{IR-Linien}: Tiefe Atmosph\"are, st\"arkere Gravitationsrotverschiebung
	\end{itemize}
	
	Dies erzeugt eine effektive Kompensation:
	\begin{equation}
		z_{\text{total}} = z_{\text{T0}}(\lambda) + z_{\text{grav}}(r(\lambda)) \approx \text{konstant}
	\end{equation}
	
	\textbf{3. Nichtlineare Feldkorrekturen:}
	
	Die vollst\"andige T0-L\"osung k\"onnte Selbstkompensationsterme enthalten:
	\begin{equation}
		z = \frac{\xi x \lambda_0}{E_\xi}\left[1 - \alpha\left(\frac{\xi x \lambda_0}{E_\xi}\right) + \beta\left(\frac{\xi x \lambda_0}{E_\xi}\right)^2 + ...\right]
	\end{equation}
	
	F\"ur spezifische Werte von $\alpha$ und $\beta$ k\"onnte die Wellenl\"angenabh\"angigkeit bei kosmologischen Entfernungen abflachen, w\"ahrend sie lokal sichtbar bleibt.
	
	\subsubsection{Wie man auf Kompensation testet}
	
	\textbf{Beobachtungsstrategien:}
	\begin{enumerate}
		\item \textbf{Entfernungsabh\"angige Studien:}
		\begin{itemize}
			\item Messung von $\Delta z/\Delta\lambda$ bei verschiedenen Entfernungen
			\item Kompensationseffekte sollten mit Entfernung variieren
			\item T0-Effekt linear mit Entfernung, Kompensation m\"oglicherweise nicht
		\end{itemize}
		
		\item \textbf{Umgebungsabh\"angige Messungen:}
		\begin{itemize}
			\item Vergleich von Objekten in Voids vs. Haufen
			\item Verschiedene IGM-Dichten $\rightarrow$ verschiedene Kompensation
			\item Saubere Sichtlinien vs. dichte Regionen
		\end{itemize}
		
		\item \textbf{Quellentyp-Variationen:}
		\begin{itemize}
			\item Quasare vs. Galaxien vs. Supernovae
			\item Verschiedene Emissionsmechanismen
			\item Verschiedene atmosph\"arische Strukturen
		\end{itemize}
		
		\item \textbf{Extreme Wellenl\"angentests:}
		\begin{itemize}
			\item Gammastrahlenausbr\"uche (k\"urzeste $\lambda$)
			\item Radiogalaxien (l\"angste $\lambda$)
			\item Kompensation k\"onnte an Extremen zusammenbrechen
		\end{itemize}
	\end{enumerate}
	
	\subsubsection{Erforderliche theoretische Anpassungen f\"ur B2}
	
	Wenn Kompensation best\"atigt wird, ben\"otigt die T0-Theorie:
	
	\textbf{1. Erweitertes Framework:}
	\begin{equation}
		z_{\text{total}} = z_{\text{T0}}(\lambda, x) + \sum_i z_{\text{komp},i}(\lambda, x, \rho, T, ...)
	\end{equation}
	
	\textbf{2. Umgebungsparameter:}
	\begin{itemize}
		\item IGM-Dichteprofil: $\rho_{\text{IGM}}(x)$
		\item Temperaturverteilung: $T(x)$
		\item Magnetfeldeffekte: $B(x)$
	\end{itemize}
	
	\textbf{3. Verfeinerte Vorhersagen:}
	\begin{itemize}
		\item Restliche Wellenl\"angenabh\"angigkeit unter spezifischen Bedingungen
		\item Optimale Beobachtungsstrategien zur Aufdeckung des T0-Effekts
		\item Vorhersagen f\"ur wann Kompensation versagt
	\end{itemize}
	
	\subsection{Die verd\"achtige Koinzidenz}
	
	Die Tatsache, dass die vorhergesagte T0-Effektgr\"o\ss{}e ($\xi = 4/3 \times 10^{-4}$) die Wellenl\"angenabh\"angigkeit \textit{exakt} an die aktuelle Nachweisschwelle platziert, verdient besondere Aufmerksamkeit:
	
	\begin{itemize}
		\item \textbf{Wahrscheinlichkeitsargument}: Die Chance, dass eine fundamentale Konstante einen Effekt zuf\"allig genau an unsere aktuelle technologische Grenze platziert, ist extrem klein
		\item \textbf{Historischer Pr\"azedenzfall}: \"Ahnliche Koinzidenzen in der Physik deuteten oft auf reale Effekte hin, die durch Komplikationen maskiert waren (z.B. solares Neutrinoproblem)
		\item \textbf{Anthropische \"Uberlegung}: Kein anthropischer Grund beschr\"ankt $\xi$ auf diesen spezifischen Wert
		\item \textbf{Wahrscheinlichste Interpretation}: Der Effekt existiert, ist aber teilweise kompensiert und h\"alt ihn knapp unterhalb klarer Detektion
	\end{itemize}
	
	\begin{experiment}[title=Test der Koinzidenz]
		Um zu kl\"aren, ob diese Koinzidenz bedeutsam ist:
		\begin{enumerate}
			\item Vergleich von Messungen aus verschiedenen Epochen bei technologischem Fortschritt
			\item Suche nach systematischen Trends in Nicht-Detektionen nahe der Schwelle
			\item Suche nach Umgebungskorrelationen in marginalen Detektionen
			\item Meta-Analyse aller Wellenl\"angenabh\"angigkeitsstudien
		\end{enumerate}
	\end{experiment}
	
	\subsection{Entscheidungsbaum f\"ur zuk\"unftige Beobachtungen}
	
	\begin{center}
		\begin{tabular}{l}
			\textbf{Hochpr\"azisionsmessung} ($\Delta z/z < 10^{-6}$) \\
			\midrule
			$\downarrow$ \\
			\textbf{Frage:} Wellenl\"angenabh\"angigkeit detektiert? \\
			\midrule
			\textbf{JA} $\rightarrow$ T0 BEST\"ATIGT (Ergebnis A) \\
			\hspace{1cm} $\bullet$ $\xi$ pr\"azise messen \\
			\hspace{1cm} $\bullet$ Entfernungsabh\"angigkeit testen \\
			\midrule
			\textbf{NEIN} $\rightarrow$ Weitere Untersuchung erforderlich \\
			\hspace{1cm} \textbf{Test:} Universal \"uber alle Bedingungen? \\
			\hspace{2cm} JA $\rightarrow$ B1: T0 modifizieren (linearer Mechanismus) \\
			\hspace{2cm} NEIN $\rightarrow$ B2: Kompensation (Theorie verfeinern)
		\end{tabular}
	\end{center}
	
	\subsection{Fazit: Eine Theorie am Scheideweg}
	
	Die T0-Theorie steht an einem kritischen Wendepunkt. Die Vorhersage der wellenl\"angenabh\"angigen Rotverschiebung wird entweder:
	
	\begin{itemize}
		\item \textbf{Die Kosmologie revolutionieren} wenn best\"atigt (Ergebnis A)
		\item \textbf{Vereinfachung erfordern} wenn abwesend (Unter-Szenario B1)
		\item \textbf{Verborgene Komplexit\"at aufdecken} wenn kompensiert (Unter-Szenario B2)
	\end{itemize}
	
	\begin{important}[title=Kritische Einsicht: Das Koinzidenzproblem]
		\textbf{Die bemerkenswert pr\"azise Koinzidenz, dass $\xi = 4/3 \times 10^{-4}$ den Effekt exakt an die aktuellen Nachweisgrenzen platziert, deutet darauf hin, dass dies kein Zufall ist.} Das wahrscheinlichste Szenario k\"onnte B2 sein - der Effekt existiert, ist aber teilweise kompensiert, was erkl\"art, warum wir genau an der Schwelle sind, wo der Effekt weder klar sichtbar noch klar abwesend ist.
	\end{important}
	
	Jedes Ergebnis f\"ordert unser Verst\"andnis: Best\"atigung validiert ein neues kosmologisches Paradigma, Abwesenheit vereinfacht die Theorie unter Bewahrung ihrer geometrischen Grundlagen, und Kompensation enth\"ullt zus\"atzliche Physik, die wir ber\"ucksichtigen m\"ussen. Dies ist Wissenschaft von ihrer besten Seite - klare Vorhersagen, definitive Tests und die Flexibilit\"at, aus dem zu lernen, was die Natur enth\"ullt.
	
	\begin{revolutionary}[title=Ein historischer Moment in der Physik]
		Wir stehen an einem einzigartigen Wendepunkt in der Geschichte der Kosmologie. Innerhalb des n\"achsten Jahrzehnts wird die Menschheit definitiv wissen, ob:
		\begin{itemize}
			\item Das Universum statisch mit Photonenenergieverlust ist (T0 best\"atigt)
			\item Das Universum expandiert wie derzeit angenommen (T0 widerlegt via B1)
			\item Die Realit\"at komplexer ist als jedes Modell allein (T0 mit Kompensation via B2)
		\end{itemize}
		Jedes Ergebnis revolutioniert unser Verst\"andnis. Dies ist nicht nur ein Test einer Theorie - es ist ein fundamentales Urteil \"uber die Natur des Kosmos selbst.
	\end{revolutionary}
	
	\section{Statistische Analysemethode}
	
	\subsection{Multi-Linien-Regression}
	
	\begin{experiment}
		\textbf{Wellenl\"angen-Rotverschiebungs-Korrelationstest:}
		\begin{enumerate}
			\item Sammlung von Rotverschiebungsmessungen: $\{z_i, \lambda_i\}$ f\"ur jedes Objekt
			\item Anpassung linearer Beziehung: $z = \alpha \cdot \lambda + \beta$
			\item Vergleich der Steigung $\alpha$ mit T0-Vorhersage: $\alpha = \frac{\xi x}{\Exi}$
			\item Test gegen Standardkosmologie: $\alpha = 0$
		\end{enumerate}
	\end{experiment}
	
	\subsection{Erforderliche Pr\"azision}
	
	Um T0-Effekte mit $\xiconst$ zu detektieren:
	\begin{itemize}
		\item \textbf{Minimal ben\"otigte Pr\"azision}: $\frac{\Delta z}{z} \approx 10^{-5}$
		\item \textbf{Aktuelle beste Pr\"azision}: $\frac{\Delta z}{z} \approx 10^{-4}$ (kaum ausreichend)
		\item \textbf{N\"achste Generation Instrumente}: $\frac{\Delta z}{z} \approx 10^{-6}$ (klar nachweisbar)
	\end{itemize}
	
	\section{Mathematische \"Aquivalenz von Raumdehnung, Energieverlust und Beugung}
	\label{sec:equivalence}
	
	\subsection{Formale \"Aquivalenzbeweise}
	\label{subsec:equivalence_proofs}
	
	Die drei fundamentalen Mechanismen zur Erkl\"arung der kosmologischen Rotverschiebung lassen sich durch unterschiedliche physikalische Prozesse beschreiben, f\"uhren aber unter bestimmten Bedingungen zu mathematisch \"aquivalenten Ergebnissen.
	
	\begin{table}[h]
		\centering
		\caption{Vergleich der Rotverschiebungsmechanismen mit erweiterten Entwicklungen}
		\scalebox{0.75}{
			\begin{tabular}{lllc}
				\toprule
				\textbf{Mechanismus} & \textbf{Physikalischer Prozess} & \textbf{Rotverschiebungsformel} & \textbf{Taylor-Entwicklung} \\
				\midrule
				Raumdehnung ($\Lambda$CDM) & Metrische Expansion & $1+z = \frac{a(t_0)}{a(t_e)}$ & $z \approx H_0 D + \frac{1}{2}q_0(H_0 D)^2$ \\
				Energieverlust (T0-E) & Photonenerm\"udung & $1+z = \exp\left(\int_0^D \xi \frac{H}{T} dl\right)$ & $z \approx \xi \frac{H_0 D}{T_0} + \frac{1}{2}\xi^2\left(\frac{H_0 D}{T_0}\right)^2$ \\
				Vakuumbeugung (T0-B) & Brechungsindex\"anderung & $1+z = \frac{n(t_e)}{n(t_0)}$ & $z \approx \xi \ln\left(1+\frac{H_0 D}{c}\right)\left(1+\frac{\xi\lambda_0}{2\lambda_{crit}}\right)$ \\
				\bottomrule
			\end{tabular}
		}
	\end{table}
	
	\subsubsection{Mathematische \"Aquivalenzbedingungen}
	
	F\"ur die \"Aquivalenz der drei Mechanismen m\"ussen folgende Bedingungen erf\"ullt sein:
	
	\begin{equation}
		\boxed{\frac{1}{a}\frac{da}{dt} = -\frac{1}{n}\frac{dn}{dt} = \xi \frac{H}{T_0}}
	\end{equation}
	
	Dies f\"uhrt zu den Beziehungen:
	\begin{itemize}
		\item \textbf{$\Lambda$CDM $\leftrightarrow$ T0-B}: $n(t) = a^{-1}(t)$
		\item \textbf{$\Lambda$CDM $\leftrightarrow$ T0-E}: $\dot{E}/E = -H(t)$
		\item \textbf{T0-B $\leftrightarrow$ T0-E}: $n(t) \propto E^{-1}(t)$
	\end{itemize}
	
	\subsubsection{St\"orungstheoretische Entwicklung}
	
	Die \"Aquivalenz gilt exakt nur in erster Ordnung. H\"ohere Ordnung Abweichungen liefern unterscheidende Signaturen:
	
	\begin{equation}
		z_{total} = z_0 + \Delta z_{mechanism} + O(\xi^2)
	\end{equation}
	
	wobei $\Delta z_{mechanism}$ vom spezifischen physikalischen Prozess abh\"angt.
	
	\subsection{Energieerhaltung und Thermodynamik}
	\label{subsec:energy_conservation}
	
	\subsubsection{Energiebilanz in verschiedenen Formalismen}
	
	\textbf{$\Lambda$CDM (scheinbarer Energieverlust):}
	\begin{equation}
		E_{photon} = \frac{h\nu_0}{1+z} = \frac{h\nu_0 a(t_e)}{a(t_0)}
	\end{equation}
	
	\textbf{T0-Beugung (Energieerhaltung):}
	\begin{equation}
		E_{photon} = \frac{h\nu}{n(t)} = \frac{h\nu_0}{(1+z)n(t)} = \text{const}
	\end{equation}
	
	\textbf{T0-Energieverlust (realer Verlust):}
	\begin{equation}
		\frac{dE}{dt} = -\xi H E \quad \Rightarrow \quad E(t) = E_0 \exp\left(-\int_0^t \xi H(t') dt'\right)
	\end{equation}
	
	\subsubsection{Thermodynamische Konsistenz}
	
	Die Entropie\"anderung f\"ur die verschiedenen Mechanismen:
	
	\begin{equation}
		\Delta S = \begin{cases}
			0 & \text{($\Lambda$CDM: adiabatisch)} \\
			k_B \xi N_{photon} \ln(1+z) & \text{(T0-Energieverlust)} \\
			0 & \text{(T0-Beugung: reversibel)}
		\end{cases}
	\end{equation}
	
	\section{Implikationen f\"ur die Kosmologie}
	
	\subsection{Statisches Universumsmodell}
	
	Die T0-Theorie beschreibt ein statisches, ewig existierendes Universum, in dem:
	\begin{itemize}
		\item Rotverschiebung aus Energieverlust entsteht, nicht aus Expansion
		\item CMB ist Gleichgewichtsstrahlung des $\xi$-Feldes
		\item Keine Urknall-Singularit\"at erforderlich
		\item Keine dunkle Energie oder dunkle Materie ben\"otigt
		\item Zyklische Prozesse innerhalb des statischen Rahmens m\"oglich
	\end{itemize}
	
	\subsection{Aufl\"osung kosmologischer Spannungen}
	
	Das T0-Modell l\"ost:
	\begin{enumerate}
		\item \textbf{Hubble-Spannung}: Verschiedene Messungen durch $\xi$-Effekte vers\"ohnt
		\item \textbf{JWST fr\"uhe Galaxien}: Kein Entstehungszeitparadox im statischen Universum
		\item \textbf{Kosmische Koinzidenz}: Nat\"urliche Erkl\"arung durch $\xi$-Geometrie
		\item \textbf{Horizontproblem}: Kein Horizont im ewigen Universum
		\item \textbf{Flachheitsproblem}: Nat\"urliche Konsequenz statischer Geometrie
	\end{enumerate}
	
	\section{Robustheit der T0-Kernvorhersagen}
	
	\subsection{Unabh\"angig vom Rotverschiebungsmechanismus}
	
	Selbst wenn spektroskopische Tests keine wellenl\"angenabh\"angige Rotverschiebung detektieren, bleiben folgende T0-Vorhersagen g\"ultig:
	
	\begin{enumerate}
		\item \textbf{Gravitationskonstante}: $G = \frac{\xi^2 c^3}{16\pi m_p^2} = 6,674 \times 10^{-11}$ m$^3$kg$^{-1}$s$^{-2}$ (genau auf 8 Stellen) bleibt g\"ultig, unabh\"angig von kosmologischen Tests
		
		\item \textbf{Geometrische Konstanten}: Die Herleitung von $\alpha \approx 1/137$ aus $(4/3)^3$-Skalierung bleibt bestehen
		
		\item \textbf{Massenhierarchie}: $m_e : m_\mu : m_\tau = 1 : 206,768 : 3477,15$ folgt aus Quantenzahlen, nicht aus Rotverschiebung
		
		\item \textbf{Hubble-Spannung}: Die 4/3-Erkl\"arung funktioniert unabh\"angig vom spezifischen Mechanismus
	\end{enumerate}
	
	\subsection{Adaptivit\"at der theoretischen Struktur}
	
	Die T0-Theorie hat nat\"urliche Anpassungsmechanismen:
	
	\begin{equation}
		\xi_{eff}(\text{Skala}) = \xi_0 \times f(\text{Umgebung}) \times g(\text{Energie})
	\end{equation}
	
	wobei:
	\begin{itemize}
		\item $f(\text{Umgebung}) = 4/3$ in Galaxienhaufen, $= 1$ im intergalaktischen Medium
		\item $g(\text{Energie})$ beschreibt Renormierungsgruppen-Laufen
	\end{itemize}
	
	Diese Flexibilit\"at ist keine ad-hoc Anpassung, sondern folgt aus der geometrischen Struktur der Theorie.
	
	\section{Schlussfolgerungen}
	
	Die T0-Theorie bietet eine revolution\"are Alternative zur expansionsbasierten Kosmologie durch eine einzige universelle Konstante $\xiconst$. Die Vorhersage der wellenl\"angenabh\"angigen Rotverschiebung bietet einen klaren experimentellen Test zur Unterscheidung zwischen T0 und Standardkosmologie. W\"ahrend die aktuelle Pr\"azision kaum die Nachweisschwelle erreicht, sollten spektroskopische Instrumente der n\"achsten Generation diese fundamentale Vorhersage definitiv testen.
	
	Die Vereinheitlichung von gravitativen, elektromagnetischen und Quantenph\"anomenen durch das $\xi$-Feld repr\"asentiert einen Paradigmenwechsel von komplexen Mehrparameter-Modellen zu eleganter geometrischer Einfachheit. Die hier vorgeschlagenen experimentellen Tests, insbesondere die Multiwellenl\"angen-Spektroskopie kosmischer Objekte, bieten klare Wege zur Validierung oder Widerlegung der Theorie.
	
	\begin{important}[title=Abschlie\ss{}ende Perspektive]
		Die T0-Theorie demonstriert, dass alle kosmischen Ph\"anomene durch eine einzige geometrische Konstante verstanden werden k\"onnen, wodurch die Notwendigkeit f\"ur dunkle Materie, dunkle Energie, Inflation und die Urknall-Singularit\"at eliminiert wird. Dies repr\"asentiert die bedeutendste Vereinfachung in der Physik seit Newtons Vereinheitlichung der terrestrischen und himmlischen Mechanik.
	\end{important}
	
	% Bibliographie
	\bibliographystyle{unsrt}
	\begin{thebibliography}{99}
		
		\bibitem{pascher2025}
		Pascher, Johann (2025). 
		\textit{T0-Theorie: Vollst\"andige Herleitung und experimentelle Tests}. 
		T0-Theory Project. 
		\url{https://jpascher.github.io/T0-Time-Mass-Duality/}
		
		\bibitem{jwst_early}
		Naidu, R. P., et al. (2022). 
		\textit{Two Remarkably Luminous Galaxy Candidates at z $\approx$ 11--13 Revealed by JWST}. 
		The Astrophysical Journal Letters, 940(1), L14.
		
	\end{thebibliography}
	
%\end{document}