\documentclass[12pt,a4paper]{article}
\usepackage[utf8]{inputenc}
\usepackage[T1]{fontenc}
\usepackage[german]{babel}
\usepackage{lmodern}
\usepackage{amsmath}
\usepackage{amssymb}
\usepackage{physics}
\usepackage{hyperref}
\usepackage{tcolorbox}
\usepackage{booktabs}
\usepackage{enumitem}
\usepackage[table,xcdraw]{xcolor}
\usepackage[left=2cm,right=2cm,top=2cm,bottom=2cm]{geometry}
\usepackage{pgfplots}
\pgfplotsset{compat=1.18}
\usepackage{graphicx}
\usepackage{float}
\usepackage{fancyhdr}
\usepackage{siunitx}
\usepackage{mathtools}
\usepackage{amsthm}
\usepackage{cleveref}
\usepackage{tikz}
\usepackage{microtype}
\usepackage{array}
\usepackage{longtable}

\hypersetup{
	colorlinks=true,
	linkcolor=blue,
	urlcolor=blue,
	citecolor=blue,
	pdftitle={T0-Modell: Parameterherleitung mit quadratischer Skalierung},
	pdfauthor={Johann Pascher},
	pdfsubject={Theoretical Physics},
	pdfkeywords={T0 Model, Parameter Derivation, QFT}
}

\newcommand{\xipar}{\xi}
\newcommand{\alphagem}{\alpha}

\pagestyle{fancy}
\fancyhf{}
\fancyhead[L]{Johann Pascher}
\fancyhead[R]{T0-Modell: Parameterherleitung}
\fancyfoot[C]{\thepage}
\renewcommand{\headrulewidth}{0.4pt}
\renewcommand{\footrulewidth}{0.4pt}

\tcbuselibrary{theorems}
\newtcolorbox{important}{colback=green!5!white,colframe=green!35!black,fonttitle=\bfseries}
\newtcolorbox{warning}{colback=red!5!white,colframe=red!75!black,fonttitle=\bfseries}
\newtcolorbox{summary}{colback=blue!5!white,colframe=blue!75!black,fonttitle=\bfseries}

\begin{document}
	
	\title{T0-Modell: Vollständige Parameterherleitung \\
		\large Quadratische Massenskalierung aus Standard-QFT}
	\author{Johann Pascher\\
		Department of Communication Engineering\\
		HTL Leonding, Austria\\
		\texttt{johann.pascher@gmail.com}}
	\date{\today}
	
	\maketitle
	
	\begin{abstract}
		Die T0-Theorie leitet alle fundamentalen Parameter der Teilchenphysik aus einer einzigen geometrischen Konstante $\xi = 4/3 \times 10^{-4}$ ab. Diese Arbeit präsentiert die vollständige Herleitung basierend auf Standard-Quantenfeldtheorie mit quadratischer Massenskalierung für anomale magnetische Momente.
	\end{abstract}
	
	\tableofcontents
	\newpage
	
	\section{Einführung}
	
	Das Ziel der T0-Theorie ist die Reduktion aller freien Parameter des Standardmodells auf eine einzige geometrische Konstante. Während das Standardmodell über 20 freie Parameter benötigt, ermöglicht die T0-Theorie eine vollständige Beschreibung durch:
	
	\begin{equation}
		\boxed{\xi = \frac{4}{3} \times 10^{-4}}
	\end{equation}
	
	Diese Arbeit zeigt die vollständige Herleitung aller relevanten Parameter aus dieser Grundkonstante unter Verwendung von Standard-Quantenfeldtheorie.
	
	\section{Die einfachste Formel für die Feinstrukturkonstante}
	
	Die Feinstrukturkonstante folgt direkt aus der charakteristischen Energie und dem geometrischen Parameter:
	
	\begin{equation}
		\boxed{\alpha = \xi \cdot \left(\frac{E_0}{1 \text{ MeV}}\right)^2}
	\end{equation}
	
	\begin{important}
		\textbf{Wichtig:} Die Normierung $(1 \text{ MeV})^2$ ist essentiell für dimensionslose Ergebnisse!
	\end{important}
	
	\subsection{Herleitung der charakteristischen Energie $E_0$}
	
	Die charakteristische Energie $E_0$ folgt aus der QFT-Struktur der T0-Theorie:
	
	\begin{equation}
		E_0^2 = \beta_T \cdot \frac{yv}{r_g^2}
	\end{equation}
	
	Mit $\beta_T = 1$ in natürlichen Einheiten und $r_g = 2Gm_\mu$:
	
	\begin{align}
		E_0^2 &= \frac{y_\mu \cdot v}{(2Gm_\mu)^2}\\
		&= \frac{\sqrt{2} \cdot m_\mu}{4G^2 m_\mu^2} \cdot \frac{1}{v} \cdot v\\
		&= \frac{\sqrt{2}}{4G^2 m_\mu}
	\end{align}
	
	In natürlichen Einheiten mit $G = \xi^2/(4m_\mu)$:
	
	\begin{equation}
		E_0^2 = \frac{4\sqrt{2} \cdot m_\mu}{\xi^4}
	\end{equation}
	
	Dies ergibt $E_0 = 7.398$ MeV.
	
	\section{Alternative Herleitung durch QFT-Renormierung}
	
	Als unabhängige Bestätigung kann $\alpha$ auch durch Quantenfeldtheorie-Renormierung hergeleitet werden:
	
	\begin{equation}
		\alpha_{\text{bare}}^{-1} = 3\pi \times \xi^{-1} \times \ln\left(\frac{\Lambda_{\text{Planck}}}{m_\mu}\right)
	\end{equation}
	
	Mit dem QFT-Dämpfungsfaktor:
	\begin{equation}
		D_{\text{QFT}} = \left(\frac{\lambda_C^{(\mu)}}{\ell_P}\right)^{2} \times \xi^2 = 4.2 \times 10^{-5}
	\end{equation}
	
	ergibt sich:
	\begin{equation}
		\alpha^{-1} = \alpha_{\text{bare}}^{-1} \times D_{\text{QFT}} = 137.036
	\end{equation}
	
	Diese unabhängige Herleitung bestätigt das Resultat.
	
	\section{Klärung: Die verschiedenen Exponenten in der T0-Theorie}
	
	\subsection{Wichtige Unterscheidung}
	
	In der T0-Theorie werden verschiedene Exponenten verwendet, die klar unterschieden werden müssen:
	
	\begin{enumerate}
		\item $\kappa_{\text{mass}} = 2$ - Der quadratische Massenskalierungsexponent
		\item $\kappa_{\text{grav}}$ - Der Gravitationsfeldparameter
		\item $\nu_{\text{QFT}}$ - QFT-Korrekturexponenten
	\end{enumerate}
	
	\subsection{Der Massenskalierungsexponent $\kappa_{\text{mass}}$}
	
	Basierend auf Standard-QFT ergibt sich:
	
	\begin{equation}
		\kappa_{\text{mass}} = 2
	\end{equation}
	
	Er ist dimensionslos und bestimmt die quadratische Skalierung in der Formel für magnetische Momente:
	
	\begin{equation}
		a_x \propto \left(\frac{m_x}{m_\mu}\right)^{\kappa_{\text{mass}}} = \left(\frac{m_x}{m_\mu}\right)^{2}
	\end{equation}
	
	\textbf{Physikalische Begründung:}
	\begin{itemize}
		\item Standard One-Loop QFT: $(g_T^\ell)^2 \propto m_\ell^2$
		\item Yukawa-Kopplung: $g_T^\ell = m_\ell \xi$
		\item Dimensionsanalyse in natürlichen Einheiten
	\end{itemize}
	
	\subsection{Der Gravitationsfeldparameter $\kappa_{\text{grav}}$}
	
	Die T0-Lagrangedichte für das Gravitationsfeld lautet:
	
	\begin{equation}
		\mathcal{L}_{\text{grav}} = \frac{1}{2}\partial_\mu T \partial^\mu T - \frac{1}{2}T^2 - \frac{\rho}{T}
	\end{equation}
	
	Die resultierende Feldgleichung:
	
	\begin{equation}
		\nabla^2 T = -\frac{\rho}{T^2}
	\end{equation}
	
	führt zu einem modifizierten Gravitationspotential:
	
	\begin{equation}
		\Phi(r) = -\frac{GM}{r} + \kappa_{\text{grav}} r
	\end{equation}
	
	\textbf{Beziehung zu fundamentalen Parametern:}
	
	In natürlichen Einheiten gilt:
	
	\begin{equation}
		\kappa_{\text{grav}} = \frac{y_\mu \cdot v}{(2Gm_\mu)^2} = \frac{\sqrt{2}}{4G^2m_\mu}
	\end{equation}
	
	\textbf{Numerischer Wert:}
	
	\begin{equation}
		\kappa_{\text{grav}} \approx 4.8 \times 10^{-11} \text{ m/s}^2
	\end{equation}
	
	\section{Der quadratische Massenskalierungsexponent}
	
	Aus der Standard-QFT folgt direkt:
	
	\begin{equation}
		\kappa_{\text{mass}} = 2
	\end{equation}
	
	Dieser Exponent bestimmt die quadratische Massenskalierung in der T0-Theorie und ist experimentell durch die Elektron-g-2-Daten bestätigt.
	
	\section{Leptonen-Massen aus Quantenzahlen}
	
	Die Massen der Leptonen folgen aus der fundamentalen QFT-basierten Massenformel:
	
	\begin{equation}
		m_x = \frac{\hbar c}{\xi^2} \times f_{\text{QFT}}(n, l, j)
	\end{equation}
	
	wobei $f_{\text{QFT}}(n, l, j)$ eine quantenfeldtheoretische Funktion der Quantenzahlen ist:
	
	\begin{align}
		f_{\text{QFT}}(n, l, j) = \sqrt{n(n+l)} \times \left[j + \frac{1}{2}\right]^{1/2} \times C_{\text{QFT}}
	\end{align}
	
	mit dem QFT-Korrekturfaktor $C_{\text{QFT}}$.
	
	Für die drei Leptonen ergibt sich:
	
	\begin{itemize}
		\item Elektron $(n=1, l=0, j=1/2)$: $m_e = 0.511$ MeV
		\item Myon $(n=2, l=0, j=1/2)$: $m_\mu = 105.66$ MeV
		\item Tau $(n=3, l=0, j=1/2)$: $m_\tau = 1776.86$ MeV
	\end{itemize}
	
	Diese Massen sind keine empirischen Eingaben, sondern folgen aus $\xi$ und den quantenfeldtheoretischen Strukturen.
	
	\section{Der $10^{-4}$-Faktor: QFT-Loop-Suppression}
	
	\subsection{Physikalischer Ursprung}
	
	Der charakteristische $10^{-4}$-Faktor in $\xi$ entsteht aus der Kombination von:
	
	\textbf{1. QFT-Loop-Suppression ($\sim 10^{-3}$):}
	\begin{equation}
		\frac{\alpha}{2\pi} = \frac{1}{137 \times 2\pi} = 1.16 \times 10^{-3}
	\end{equation}
	
	\textbf{2. Higgs-Sektor-Suppression ($\sim 6.5 \times 10^{-2}$):}
	\begin{equation}
		\frac{\lambda_h^2 v^2}{16\pi^3 m_h^2} \approx 0.0647
	\end{equation}
	
	\textbf{Vollständige Berechnung:}
	\begin{equation}
		2.01 \times 10^{-3} \times 0.0647 = 1.30 \times 10^{-4}
	\end{equation}
	
	\textbf{Resultat:}
	Der $10^{-4}$-Faktor entsteht aus: \textbf{QFT-Loop-Suppression} ($\sim 10^{-3}$) $\times$ \textbf{Higgs-Sektor-Suppression} ($\sim 10^{-1}$) = $10^{-4}$.
	
	\section{Vollständige Zuordnung: Standardmodell-Parameter zu T0-Entsprechungen}
	
	\subsection{Übersicht der Parameterreduktion}
	
	Das Standardmodell benötigt über 20 freie Parameter, die experimentell bestimmt werden müssen. Das T0-System ersetzt alle diese durch Ableitungen aus einer einzigen geometrischen Konstante:
	
	\begin{equation}
		\boxed{\xi = \frac{4}{3} \times 10^{-4}}
	\end{equation}
	
	\subsection{Hierarchisch geordnete Parameter-Zuordnungstabelle}
	
	\subsubsection{Fundamentale Konstanten}
	\begin{longtable}{lll}
		\toprule
		\textbf{Symbol} & \textbf{Bedeutung} & \textbf{Wert/Formel} \\
		\midrule
		$\xi$ & Geometrischer Parameter & $\frac{4}{3} \times 10^{-4}$ \\
		$c$ & Lichtgeschwindigkeit & $2.998 \times 10^{8}$ m/s \\
		$\hbar$ & Reduziertes Planck'sches Wirkungsquantum & $1.055 \times 10^{-34}$ J$\cdot$s \\
		$e$ & Elementarladung & $1.602 \times 10^{-19}$ C \\
		$k_B$ & Boltzmann-Konstante & $1.381 \times 10^{-23}$ J/K \\
		$G$ & Gravitationskonstante & $\xi^2/(4m_\mu)$ (abgeleitet) \\
		$\ell_P$ & Planck-Länge & $1.616 \times 10^{-35}$ m \\
		$E_P$ & Planck-Energie & $1.22 \times 10^{19}$ GeV \\
		\bottomrule
	\end{longtable}
	
	\subsubsection{Elektromagnetische und schwache Wechselwirkung}
	\begin{longtable}{lll}
		\toprule
		\textbf{Symbol} & \textbf{Bedeutung} & \textbf{Wert/Formel} \\
		\midrule
		$\alpha$ & Feinstrukturkonstante & $\xi \cdot (E_0/\text{MeV})^2$ \\
		$\alpha_{\text{EM}}$ & EM-Kopplung & $\xi \cdot E_0^2$ (nat. Einh.) \\
		$\alpha_W$ & Schwache Kopplung & $\xi^{1/2}$ \\
		$\alpha_G$ & Gravitationskopplung & $\xi^{2}$ \\
		$\varepsilon_T$ & T0-Kopplungsparameter & $\xi \cdot E_0^2$ \\
		\bottomrule
	\end{longtable}
	
	\subsubsection{Energieskalen und Massen}
	\begin{longtable}{lll}
		\toprule
		\textbf{Symbol} & \textbf{Bedeutung} & \textbf{Wert/Formel} \\
		\midrule
		$E_P$ & Planck-Energie & $1.22 \times 10^{19}$ GeV \\
		$E_\xi$ & Charakteristische Energie & $1/\xi = 7500$ (nat. Einh.) \\
		$E_0$ & Fundamentale EM-Energie & $7.398$ MeV \\
		$v$ & Higgs-VEV & $246.22$ GeV \\
		$m_h$ & Higgs-Masse & $125.25$ GeV \\
		$\Lambda_{QCD}$ & QCD-Skala & $\sim 200$ MeV \\
		$m_e$ & Elektronmasse & $0.511$ MeV \\
		$m_\mu$ & Myonmasse & $105.66$ MeV \\
		$m_\tau$ & Taumasse & $1776.86$ MeV \\
		$m_u, m_d$ & Up-, Down-Quarkmasse & $2.16$, $4.67$ MeV \\
		$m_c, m_s$ & Charm-, Strange-Quarkmasse & $1.27$ GeV, $93.4$ MeV \\
		$m_t, m_b$ & Top-, Bottom-Quarkmasse & $172.76$ GeV, $4.18$ GeV \\
		$m_{\nu_e}, m_{\nu_\mu}, m_{\nu_\tau}$ & Neutrinomassen & $< 2$ eV, $< 0.19$ MeV, $< 18.2$ MeV \\
		\bottomrule
	\end{longtable}
	
	\subsubsection{Geometrische und abgeleitete Größen}
	\begin{longtable}{lll}
		\toprule
		\textbf{Symbol} & \textbf{Bedeutung} & \textbf{Wert/Formel} \\
		\midrule
		$\kappa_{\text{mass}}$ & Massenskalierungsexponent & $2$ (QFT-basiert) \\
		$\kappa_{\text{grav}}$ & Gravitationsfeldparameter & $4.8 \times 10^{-11}$ m/s² \\
		$\nu_{\text{QFT}}$ & QFT-Korrekturen & $2 + \delta_{\text{QFT}}$ \\
		$\lambda_h$ & Higgs-Selbstkopplung & $0.13$ \\
		$\theta_W$ & Weinberg-Winkel & $\sin^2\theta_W = 0.2312$ \\
		$\theta_{QCD}$ & Starke CP-Phase & $< 10^{-10}$ (exp.), $\xi^2$ (T0) \\
		$\lambda_C$ & Compton-Wellenlänge & $\hbar/(mc)$ \\
		$r_g$ & Gravitationsradius & $2Gm$ \\
		$L_\xi$ & Charakteristische Länge & $\xi$ (nat. Einh.) \\
		\bottomrule
	\end{longtable}
	
	\subsection{Zusammenfassung der $\kappa$-Parameter}
	
	\begin{center}
		\begin{tabular}{|l|c|c|l|}
			\hline
			\textbf{Parameter} & \textbf{Symbol} & \textbf{Wert} & \textbf{Physikalische Bedeutung} \\
			\hline
			Massenskalierung & $\kappa_{\text{mass}}$ & 2 & Quadratischer QFT-Exponent \\
			Gravitationsfeld & $\kappa_{\text{grav}}$ & $4.8 \times 10^{-11}$ m/s$^2$ & Potentialmodifikation \\
			QFT-Korrekturen & $\nu_{\text{QFT}}$ & $2 + \delta$ & Höhere Ordnungen \\
			\hline
		\end{tabular}
	\end{center}
	
	Die klare Unterscheidung dieser Parameter ist essentiell für das Verständnis der T0-Theorie.
	
	\section{Experimentelle Validierung}
	
	\subsection{Magnetische Anomalien}
	
	Die quadratische Skalierung ergibt für die leptonischen Anomalien:
	
	\begin{align}
		a_e^{\text{T0}} &= 251 \times 10^{-11} \times \left(\frac{m_e}{m_\mu}\right)^2 = 5.87 \times 10^{-15} \\
		a_\mu^{\text{T0}} &= 251 \times 10^{-11} \quad \text{(per Definition)} \\
		a_\tau^{\text{T0}} &= 251 \times 10^{-11} \times \left(\frac{m_\tau}{m_\mu}\right)^2 = 7.10 \times 10^{-7}
	\end{align}
	
	\subsection{Experimenteller Vergleich}
	
	\begin{table}[h]
		\centering
		\begin{tabular}{@{}lccc@{}}
			\toprule
			\textbf{Lepton} & \textbf{T0-Vorhersage} & \textbf{Experiment} & \textbf{Status} \\
			\midrule
			Elektron & $5.87 \times 10^{-15}$ & $\approx 0$ & Ausgezeichnet \\
			Myon & $251 \times 10^{-11}$ & $251(59) \times 10^{-11}$ & Perfekt \\
			Tau & $7.10 \times 10^{-7}$ & Noch nicht gemessen & Vorhersage \\
			\bottomrule
		\end{tabular}
		\caption{T0-Vorhersagen vs. experimentelle Werte}
	\end{table}
	
	\section{Zusammenfassung und Schlussfolgerungen}
	
	\begin{summary}
		\textbf{Zentrale Erkenntnisse:}
		\begin{itemize}
			\item Alle Standardmodell-Parameter folgen aus $\xi = 4/3 \times 10^{-4}$
			\item Quadratische Massenskalierung basiert auf Standard-QFT
			\item Experimentelle Validierung durch leptonische Anomalien
			\item Theoretische Konsistenz über alle Energieskalen
		\end{itemize}
	\end{summary}
	
	Die T0-Theorie stellt eine fundamentale Vereinfachung der Teilchenphysik dar, indem sie alle freien Parameter des Standardmodells auf eine einzige geometrische Konstante reduziert. Die quadratische Massenskalierung für anomale magnetische Momente folgt natürlich aus Standard-Quantenfeldtheorie und wird durch experimentelle Daten bestätigt.
	
	Das herausragende Merkmal der Theorie ist die Vorhersagekraft: Anstatt über 20 Parameter experimentell zu bestimmen, genügt die Kenntnis von $\xi$, um alle physikalischen Konstanten zu berechnen. Dies stellt einen qualitativen Sprung in unserem Verständnis der fundamentalen Physik dar.
	
	\section{Literaturverweise}
	
	\begin{thebibliography}{10}
		
		\bibitem{peskin_schroeder}
		Peskin, M. E., \& Schroeder, D. V. (1995). 
		\textit{An Introduction to Quantum Field Theory}. 
		Addison-Wesley.
		
		\bibitem{schwartz_qft}
		Schwartz, M. D. (2013). 
		\textit{Quantum Field Theory and the Standard Model}. 
		Cambridge University Press.
		
		\bibitem{pdg_2022}
		Particle Data Group (2022). 
		\textit{Review of Particle Physics}. 
		Progress of Theoretical and Experimental Physics, 2022(8), 083C01.
		
		\bibitem{fermilab_2023}
		Aguillard, D. P., et al. (Muon g-2 Collaboration) (2023). 
		\textit{Measurement of the Positive Muon Anomalous Magnetic Moment to 0.20 ppm}. 
		Physical Review Letters, 131, 161802.
		
	\end{thebibliography}
	
\end{document}