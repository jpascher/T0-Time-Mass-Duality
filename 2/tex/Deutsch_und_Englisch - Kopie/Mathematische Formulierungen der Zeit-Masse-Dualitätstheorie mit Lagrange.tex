\documentclass{article}
\usepackage[utf8]{inputenc}
\usepackage[T1]{fontenc}
\usepackage[ngerman]{babel}
\usepackage{lmodern}
\usepackage{amsmath}
\usepackage{amssymb}
\usepackage{physics}
\usepackage{graphicx}
\usepackage{xcolor}
\usepackage{tocloft}
\usepackage{amsthm}
\usepackage{hyperref}
\usepackage{cleveref}
\usepackage[margin=2cm]{geometry}
\usepackage{siunitx}

\renewcommand{\cftsecfont}{\color{blue}}
\renewcommand{\cftsubsecfont}{\color{blue}}
\renewcommand{\cftsecpagefont}{\color{blue}}
\renewcommand{\cftsubsecpagefont}{\color{blue}}
\setlength{\cftsecindent}{1cm}
\setlength{\cftsubsecindent}{2cm}

\hypersetup{
	colorlinks=true,
	linkcolor=blue,
	urlcolor=blue,
	citecolor=red,
	pdftitle={Von Zeitdilatation zu Massenvariation: Mathematische Kernformulierungen der Zeit-Masse-Dualitätstheorie},
	pdfauthor={Johann Pascher}
}

\newcommand{\Tfield}{T(x)}
\newcommand{\DhiggsT}{\partial_\mu \Phi + \Phi \partial_\mu \Tfield}
\newcommand{\gammaf}{\gamma_{\text{Lorentz}}}

\newtheorem{theorem}{Satz}[section]
\newtheorem{proposition}[theorem]{Proposition}

\title{Von Zeitdilatation zu Massenvariation: \\ Mathematische Kernformulierungen der Zeit-Masse-Dualitätstheorie}
\author{Johann Pascher}
\date{29. März 2025}

\begin{document}
	
	\maketitle
	
	\begin{abstract}
		Diese Arbeit stellt die wesentlichen mathematischen Formulierungen der Zeit-Masse-Dualitätstheorie vor, mit Fokus auf die grundlegenden Gleichungen und ihre physikalischen Interpretationen. Die Theorie etabliert eine Dualität zwischen zwei komplementären Beschreibungen der Realität: der Standard-Sicht mit Zeitdilatation und konstanter Ruhemasse und dem T0-Modell mit absoluter Zeit und variabler Masse. Zentral für diesen Rahmen ist die intrinsische Zeit \( \Tfield = \frac{\hbar}{\max(m c^2, \omega)} \), die eine einheitliche Behandlung von massiven Teilchen und Photonen ermöglicht. Die mathematischen Formulierungen umfassen modifizierte Lagrange-Dichten, die emergente Gravitation und Energieverlust-Rotverschiebung in einem statischen Universum betonen.
	\end{abstract}
	
	\tableofcontents
	\newpage
	
	\section{Einführung in die Zeit-Masse-Dualität}
	Die Zeit-Masse-Dualitätstheorie schlägt einen alternativen Rahmen vor:
	\begin{enumerate}
		\item Standard-Sicht: \( t' = \gammaf t \), \( m_0 = \text{konst.} \)
		\item T0-Modell: \( T_0 = \text{konst.} \), \( m = \gammaf m_0 \)
	\end{enumerate}
	
	\subsection{Beziehung zum Standardmodell}
	Das T0-Modell erweitert das Standardmodell mit:
	\begin{enumerate}
		\item Intrinsisches Zeitfeld: \( \Tfield = \frac{\hbar}{\max(m c^2, \omega)} \)
		\item Higgs-Feld: \( \Phi \) mit dynamischer Massenkopplung
		\item Fermionenfelder: \( \psi \) mit Yukawa-Kopplung
		\item Eichbosonenfelder: \( A_\mu \) mit \( \Tfield \)-Wechselwirkung
	\end{enumerate}
	
	\section{Emergente Gravitation aus dem intrinsischen Zeitfeld}
	\begin{theorem}[Gravitationsentstehung]
		Gravitation entsteht aus Gradienten des intrinsischen Zeitfelds:
		\begin{equation}
			\nabla \Tfield = -\frac{\hbar}{m^2 c^2} \nabla m \sim \nabla \Phi_g
		\end{equation}
		mit dem modifizierten Potential:
		\begin{equation}
			\Phi(r) = -\frac{GM}{r} + \kappa r, \quad \kappa \approx 4.8 \times 10^{-11} \, \text{m/s}^2
		\end{equation}
	\end{theorem}
	
	\begin{proof}
		Aus \( \Tfield = \frac{\hbar}{m c^2} \) für massive Teilchen:
		\begin{equation}
			\nabla \Tfield = -\frac{\hbar}{m^2 c^2} \nabla m
		\end{equation}
		Mit \( m(\vec{r}) = m_0 (1 + \frac{\Phi_g}{c^2}) \):
		\begin{equation}
			\nabla m = \frac{m_0}{c^2} \nabla \Phi_g
		\end{equation}
		Daher:
		\begin{equation}
			\nabla \Tfield \approx -\frac{\hbar}{m_0 c^4} \nabla \Phi_g
		\end{equation}
	\end{proof}
	
	\section{Mathematische Grundlagen: Intrinsische Zeit}
	\begin{theorem}[Intrinsische Zeit]
		\begin{equation}
			\Tfield = \frac{\hbar}{\max(m c^2, \omega)}
		\end{equation}
	\end{theorem}
	
	\section{Modifizierte Feldgleichungen}
	\begin{theorem}[Modifizierte Schrödinger-Gleichung]
		\begin{equation}
			i\hbar \Tfield \frac{\partial}{\partial t} \Psi + i\hbar \Psi \frac{\partial \Tfield}{\partial t} = \hat{H} \Psi
		\end{equation}
	\end{theorem}
	
	\section{Lagrange-Formulierung}
	Die vollständige Gesamt-Lagrangedichte lautet:
	\begin{equation}
		\mathcal{L}_{\text{Total}} = \mathcal{L}_{\text{Boson}} + \mathcal{L}_{\text{Fermion}} + \mathcal{L}_{\text{Higgs-T}} + \mathcal{L}_{\text{intrinsic}}, \quad \mathcal{L}_{\text{intrinsic}} = \frac{1}{2} \partial_\mu \Tfield \partial^\mu \Tfield - V(\Tfield)
	\end{equation}
	
	\section{Kosmologische Implikationen}
	Das T0-Modell hat folgende Implikationen:
	\begin{itemize}
		\item Modifiziertes Gravitationspotential: \( \Phi(r) = -\frac{GM}{r} + \kappa r \), \( \kappa \approx 4.8 \times 10^{-11} \, \text{m/s}^2 \)
		\item Kosmische Rotverschiebung: \( 1 + z = e^{\alpha d} \), \( \alpha \approx 2.3 \times 10^{-28} \, \text{m}^{-1} \)
		\item Wellenlängenabhängigkeit: \( z(\lambda) = z_0 (1 + \beta \ln(\lambda/\lambda_0)) \), \( \beta \approx 0.008 \)
	\end{itemize}
	
	\section{Unsicherheit bei \(\beta\)}
	Der Parameter \( \beta \approx 0.008 \) ist unsicher; weitere Tests sind erforderlich, um ihn einzugrenzen.
	
	\section{Schlussfolgerung}
	Diese Arbeit dient als präziser mathematischer Bezugsrahmen für die Zeit-Masse-Dualitätstheorie im T0-Modell, indem sie die grundlegenden Gleichungen und deren physikalische Interpretationen bereitstellt.
	
	\begin{thebibliography}{9}
		\bibitem{wesentlicheFormalismen} Pascher, J. (2025). \textit{Wesentliche mathematische Formalismen der Zeit-Masse-Dualitätstheorie mit Lagrange-Dichten}. 29. März 2025.
	\end{thebibliography}
	
\end{document}