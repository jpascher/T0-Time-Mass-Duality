\documentclass[a4paper,12pt]{article}
\usepackage[utf8]{inputenc}
\usepackage[T1]{fontenc}
\usepackage{lmodern}
\usepackage[ngerman]{babel} % Deutsch
\usepackage{amsmath, amssymb, amsthm, physics}
\usepackage{graphicx}
\usepackage{xcolor}
\usepackage{tikz}
\usepackage{pgfplots}
\pgfplotsset{compat=1.18}
\usepackage{setspace}
\usepackage{booktabs}
\usepackage{siunitx}
\usepackage{array}
\usepackage{float}
\usepackage[section]{placeins}

% Colored links in the table of contents and document
\usepackage{hyperref}
\hypersetup{
	colorlinks=true,
	linkcolor=blue,
	filecolor=blue,
	citecolor=blue, 
	urlcolor=blue,
	bookmarks=true,
	bookmarksopen=true,
	pdftitle={Kompensatorische und additive Effekte: Eine Analyse der Messdifferenzen zwischen dem T0-Modell und dem \(\Lambda\)CDM-Standardmodell},
	pdfauthor={Johann Pascher},
}

% Theorem styles
\newtheorem{theorem}{Satz}[section]
\newtheorem{lemma}[theorem]{Lemma}
\newtheorem{proposition}[theorem]{Proposition}
\newtheorem{corollary}[theorem]{Korollar}

\theoremstyle{definition}
\newtheorem{definition}{Definition}[theorem]
\newtheorem{example}{Beispiel}[theorem]

\theoremstyle{remark}
\newtheorem{remark}{Bemerkung}[theorem]
\renewcommand{\proofname}{Beweis}

% Custom command für Tfield
\newcommand{\Tfield}{T(x)}

% Repository base URL
\newcommand{\repobase}{https://github.com/jpascher/T0-Time-Mass-Duality/tree/main/2/}

\begin{document}
	
	\title{Kompensatorische und additive Effekte: Eine Analyse der Messdifferenzen zwischen dem T0-Modell und dem \(\Lambda\)CDM-Standardmodell}
	\author{Johann Pascher}
	\date{2. April 2025}
	\maketitle
	
	\begin{abstract}
		Dieses Dokument analysiert die Unterschiede in kosmologischen Messungen zwischen dem Standardmodell (\(\Lambda\)CDM) und dem alternativen T0-Modell. Wir untersuchen, wie die unterschiedlichen theoretischen Grundlagen Distanzmessungen, Rotverschiebungen und die Interpretation des kosmischen Mikrowellenhintergrunds beeinflussen. Besondere Aufmerksamkeit wird darauf gelegt, ob die Effekte sich gegenseitig verstärken (additiv wirken) oder kompensieren. Die Analyse zeigt ein komplexes Zusammenspiel, das das Hubble-Spannungsproblem erklären könnte. Bei niedrigen Rotverschiebungen (z \(\approx\) 1) sind die Unterschiede moderat, während sie bei hohen Rotverschiebungen (z = 1100, CMB) dramatisch werden und zu grundlegend unterschiedlichen Interpretationen führen.
	\end{abstract}
	
	\tableofcontents
	\newpage
	
	\section{Einführung}
	
	Das kosmologische Standardmodell (\(\Lambda\)CDM) und das alternative T0-Modell bieten grundlegend unterschiedliche Erklärungen für dieselben astronomischen Beobachtungen. Während \(\Lambda\)CDM auf einem expandierenden Universum basiert, postuliert das T0-Modell ein statisches Universum mit absoluter Zeit und variabler Masse. Diese Arbeit untersucht, wie diese unterschiedlichen theoretischen Grundlagen kosmologische Messungen beeinflussen und wie diese Effekte sich entweder verstärken oder kompensieren.
	
	\section{Messung der CMB-Temperatur heute}
	
	Die Messung der Temperatur des kosmischen Mikrowellenhintergrunds (CMB) wird heutzutage hauptsächlich durch Satelliten wie Planck (2009–2013) sowie bodengebundene Teleskope wie das Atacama Cosmology Telescope (ACT) und das South Pole Telescope (SPT) durchgeführt. Hier ist ein Überblick:
	
	\subsection{Instrumente und Technologie}
	
	\begin{itemize}
		\item \textbf{Planck-Satellit:}
		\begin{itemize}
			\item Low Frequency Instrument (LFI): Radiometer für 30–70 GHz.
			\item High Frequency Instrument (HFI): Bolometer für 100–857 GHz.
			\item Kryogene Kühlung auf \(\sim\)0.1 K zur Minimierung von thermischem Rauschen.
		\end{itemize}
		\item \textbf{ACT und SPT:}
		\begin{itemize}
			\item Arrays mit Hunderten bis Tausenden von Bolometern (90–300 GHz).
			\item Standorte in trockenen, hochgelegenen Regionen (Atacama-Wüste, Südpol) zur Reduzierung atmosphärischer Störungen.
		\end{itemize}
	\end{itemize}
	
	\subsection{Messprinzip}
	
	\begin{itemize}
		\item \textbf{Frequenzmessungen:} Instrumente erfassen die Intensität $I_\nu$ (Leistung pro Fläche pro Frequenzintervall pro Steradiant) über mehrere Frequenzbänder.
		\item \textbf{Schwarzkörperspektrum:} Die gemessene Intensität wird an die Planck-Verteilung angepasst:
		\[
		I_\nu(\nu, T) = \frac{2 h \nu^3}{c^2} \cdot \frac{1}{e^{h \nu / k_B T} - 1}.
		\]
		$T$ wird als freier Parameter optimiert, bis die Daten passen (z. B. $T = 2.72548 \pm 0.00057 \, \text{K}$ laut Planck 2018).
		\item \textbf{Anisotropien:} Temperaturfluktuationen ($\delta T/T \sim 10^{-5}$) werden über den Himmel kartiert.
	\end{itemize}
	
	\subsection{Verfahren}
	
	\begin{itemize}
		\item \textbf{Datenerfassung:} Mehrfache Himmelsdurchmusterungen über Monate (Satelliten) oder Jahre (bodengebunden).
		\item \textbf{Kalibrierung:} Gegen astrophysikalische Quellen (z. B. CMB-Dipol, \(\sim\)3.36 mK) und interne Referenzen.
		\item \textbf{Datenverarbeitung:} Vordergrundquellen (z. B. Staub, Synchrotronstrahlung) werden mittels statistischer Methoden entfernt, und die verbleibende Intensität wird an $I_\nu(\nu, T)$ angepasst.
	\end{itemize}
	
	\subsection{Einfluss des Standardmodells}
	
	\begin{itemize}
		\item \textbf{Keine Kopplungsfaktoren:} Die Planck-Verteilung selbst enthält keine Kopplungsfaktoren. Sie basiert auf $h$, $c$ und $k_B$, die universelle Konstanten sind.
		\item \textbf{Indirekter Einfluss:}
		\begin{itemize}
			\item \textbf{Expansion:} Die Interpretation von $T = 2.725 \, \text{K}$ als abgekühlte Urknallstrahlung ($T(z) = T_0 (1 + z)$) ist eine Annahme des Standardmodells. Die Rotverschiebung ($z$) wird durch Expansion erklärt.
			\item \textbf{Kalibrierung:} Der CMB-Dipol wird durch Bewegung relativ zu einem expandierenden Ruhesystem definiert.
			\item \textbf{Vordergrundmodelle:} Die Entfernung von Vordergrundquellen basiert auf Modellen, die mit der Expansionsgeschichte kalibriert sind.
			\item \textbf{Rohdaten:} Die Frequenzmessungen ($I_\nu$ bei verschiedenen $\nu$) sind empirisch und modellunabhängig, aber die Anpassung an ein Schwarzkörperspektrum und die Interpretation von $T$ werden durch das Standardmodell geprägt.
		\end{itemize}
	\end{itemize}
	
	\section{Grundlegende Konzepte der Modelle}
	
	\subsection{Das \(\Lambda\)CDM-Standardmodell}
	
	Im \(\Lambda\)CDM-Modell wird die beobachtete Rotverschiebung durch kosmische Expansion erklärt. Die Friedmann-Gleichungen beschreiben die zeitliche Entwicklung des Universums, und die Hubble-Konstante $H_0$ repräsentiert die aktuelle Expansionsrate. Die kosmische Rotverschiebung $z$ ist mit dem Skalenfaktor $a(t)$ verknüpft durch:
	
	\begin{equation}
		1 + z = \frac{a(t_0)}{a(t_{\text{emit}})}
	\end{equation}
	
	Für kleine Rotverschiebungen gilt näherungsweise:
	
	\begin{equation}
		z \approx \frac{H_0 d}{c}
	\end{equation}
	
	\subsection{Das T0-Modell}
	
	Im {\small\href{\repobase/pdf/Deutsch/Wesentliche mathematische Formalismen der Zeit-Masse-Dualitätstheorie mit Lagrange-Dichten_de.pdf}{T0-Modell}} ist Zeit absolut, während die Masse als \( m = \frac{\hbar}{\Tfield c^2} \) variiert, wobei \( \Tfield \) das intrinsische Zeitfeld ist, das über \( \Tfield = \frac{\hbar}{y \langle \Phi \rangle c^2} \) mit dem Higgs-Feld gekoppelt ist \cite{pascher_galaxies_2025}. Die Rotverschiebung entsteht durch die räumliche Variation von \( \Tfield \), die einen Energieverlust von Photonen verursacht:
	
	\begin{equation}
		1 + z = \frac{\Tfield}{\Tfield_0},
	\end{equation}
	
	wobei \( \Tfield_0 \) der Wert am Ort des Beobachters ist. Diese Variation kann als \( \Tfield = \Tfield_0 e^{-\alpha d} \) ausgedrückt werden, mit \( \alpha = H_0/c \), was zu folgender Form führt:
	
	\begin{equation}
		1 + z = e^{\alpha d},
	\end{equation}
	
	wobei \( d \) die physische Distanz und \( H_0 \) die Hubble-Konstante ist, neu interpretiert als die Rate der räumlichen Änderung von \( \Tfield \) anstatt als Expansionsparameter. Im Gegensatz zum \(\Lambda\)CDM-Modell, wo Dunkle Energie (\(\Lambda\)) die kosmische Beschleunigung antreibt, dient \( \Tfield \) im T0-Modell als effektives Feld, das die Rotverschiebung ohne Expansion erklärt, während seine Gradienten auch emergente Gravitation erklären \cite{pascher_galaxies_2025}.
	
	\section{Vergleichende Analyse der Messmethoden}
	
	\subsection{Physische Distanz ($d$)}
	
	\textbf{\(\Lambda\)CDM-Modell:}
	\begin{equation}
		d = \frac{c}{H_0} \int_0^z \frac{dz'}{\sqrt{\Omega_m (1 + z')^3 + \Omega_\Lambda}}
	\end{equation}
	
	\textbf{T0-Modell:}
	\begin{equation}
		d = \frac{c \ln(1 + z)}{H_0}
	\end{equation}
	
	\textbf{Quantitativer Vergleich bei $z = 1$:}
	\begin{itemize}
		\item \(\Lambda\)CDM ($H_0 = 70$ km/s/Mpc): $d \approx 3300$ Mpc
		\item T0 ($H_0 = 70$ km/s/Mpc): $d \approx 2970$ Mpc ($-10\%$)
		\item T0 ($H_0 = 73$ km/s/Mpc): $d \approx 2850$ Mpc ($-14\%$)
	\end{itemize}
	
	Im T0-Modell sind Distanzen bei gleichem $z$ systematisch kleiner, wobei der Unterschied mit wachsendem $z$ zunimmt.
	
	\subsection{Luminositätsdistanz ($d_L$)}
	
	\textbf{\(\Lambda\)CDM-Modell:}
	\begin{equation}
		d_L = (1 + z) \cdot \frac{c}{H_0} \int_0^z \frac{dz'}{\sqrt{\Omega_m (1 + z')^3 + \Omega_\Lambda}}
	\end{equation}
	
	\textbf{T0-Modell:}
	\begin{equation}
		d_L = \frac{c}{H_0} \ln(1 + z) (1 + z)
	\end{equation}
	
	\textbf{Quantitativer Vergleich bei $z = 1$:}
	\begin{itemize}
		\item \(\Lambda\)CDM ($H_0 = 70$): $d_L \approx 4710$ Mpc
		\item T0 ($H_0 = 70$): $d_L \approx 5940$ Mpc ($+26\%$)
		\item T0 ($H_0 = 73$): $d_L \approx 5700$ Mpc ($+21\%$)
	\end{itemize}
	
	Bemerkenswert ist, dass Luminositätsdistanzen im T0-Modell größer sind, was bedeutet, dass Objekte bei gleicher Rotverschiebung schwächer erscheinen als vom \(\Lambda\)CDM-Modell vorhergesagt.
	
	\subsection{Winkeldurchmesser-Distanz ($d_A$)}
	
	\textbf{\(\Lambda\)CDM-Modell:}
	\begin{equation}
		d_A = \frac{d}{1 + z}
	\end{equation}
	
	\textbf{T0-Modell:}
	\begin{equation}
		d_A = \frac{c \ln(1 + z)}{H_0 (1 + z)}
	\end{equation}
	
	\textbf{Quantitativer Vergleich bei $z = 1$:}
	\begin{itemize}
		\item \(\Lambda\)CDM ($H_0 = 70$): $d_A \approx 1650$ Mpc
		\item T0 ($H_0 = 70$): $d_A \approx 1485$ Mpc ($-10\%$)
		\item T0 ($H_0 = 73$): $d_A \approx 1425$ Mpc ($-14\%$)
	\end{itemize}
	
	\textbf{Für den CMB ($z = 1100$):}
	\begin{itemize}
		\item \(\Lambda\)CDM: $d_A \approx 13.5$ Mpc, $\theta \approx 1^\circ$
		\item T0 ($H_0 = 70$): $d_A \approx 28.9$ Mpc ($+114\%$), $\theta \approx 5.8^\circ$ ($+480\%$)
		\item T0 ($H_0 = 73$): $d_A \approx 27.7$ Mpc ($+105\%$), $\theta \approx 6.1^\circ$ ($+510\%$)
	\end{itemize}
	
	Die Unterschiede sind besonders dramatisch für den CMB, wo die vorhergesagte Winkelgröße von Strukturen im T0-Modell etwa fünfmal größer ist als im \(\Lambda\)CDM-Modell.
	
	\section{Additive und kompensatorische Effekte}
	
	\subsection{Additive (verstärkende) Effekte}
	
	Die Effekte auf physische Distanz ($d$) und Winkeldurchmesser-Distanz ($d_A$) verstärken sich gegenseitig:
	
	\begin{enumerate}
		\item \textbf{Konsistente Richtung}: Beide Distanztypen sind im T0-Modell kleiner als im \(\Lambda\)CDM-Modell (bei $z = 1$ etwa 10–14\% Reduktion).
		\item \textbf{Steigende Verstärkung mit $z$}: Bei hohen Rotverschiebungen verstärken sich diese Effekte dramatisch, wie am CMB-Beispiel ersichtlich.
		\item \textbf{Kohärenter Einfluss auf Strukturgröße}: Die reduzierte Distanz und vergrößerte Winkelmaße führen zu einer konsistenten Neuinterpretation der Größe kosmischer Strukturen.
	\end{enumerate}
	
	\subsection{Kompensatorische (entgegenwirkende) Effekte}
	
	Die Effekte auf physische Distanz und Luminositätsdistanz wirken in entgegengesetzte Richtungen:
	
	\begin{enumerate}
		\item \textbf{Gegensätzliche Trends}: Während $d$ im T0-Modell kleiner ist ($-10\%$ bei $z = 1$), ist $d_L$ größer ($+26\%$ bei $z = 1$).
		\item \textbf{Einfluss auf Helligkeitsmessungen}: Objekte erscheinen näher, aber schwächer, was zu einer komplexen Neuinterpretation von Standardkerzen wie Typ-Ia-Supernovae führt.
		\item \textbf{$H_0$-Abhängigkeit}: Ein höherer $H_0$-Wert im T0-Modell verstärkt die Distanzreduktion, mildert jedoch die Zunahme der Luminositätsdistanz.
	\end{enumerate}
	
	\section{Implikationen für das Hubble-Spannungsproblem}
	
	Die komplementären und kompensatorischen Effekte zwischen dem T0-Modell und dem \(\Lambda\)CDM-Modell könnten eine Erklärung für das Hubble-Spannungsproblem bieten:
	
	\begin{enumerate}
		\item \textbf{Abweichende Messungen}: Die unterschiedlichen Effekte auf $d_L$ und $d_A$ könnten erklären, warum lokale Messungen (basierend auf Supernovae) systematisch höhere $H_0$-Werte liefern als CMB-basierte Messungen.
		\item \textbf{Modellabhängige Kalibrierung}: Standardkerzen und -maße werden je nach zugrunde liegendem kosmologischen Modell unterschiedlich kalibriert.
		\item \textbf{CMB-Neuinterpretation}: Die dramatisch unterschiedliche Interpretation von CMB-Anisotropien ($\theta \approx 1^\circ$ vs. $\theta \approx 5.8$-$6.1^\circ$) führt zu grundlegend unterschiedlichen Parameterschätzungen.
	\end{enumerate}
	
	\section{Quantitative Zusammenfassung der Effekte}
	
	\subsection{Bei $z = 1$ (mittlere kosmologische Distanzen)}
	
	\begin{table}[h]
		\centering
		\begin{tabular}{|l|c|c|c|c|c|}
			\hline
			\textbf{Größe} & \textbf{\(\Lambda\)CDM ($H_0$=70)} & \textbf{T0 ($H_0$=70)} & \textbf{Differenz} & \textbf{T0 ($H_0$=73)} & \textbf{Differenz} \\
			\hline
			$d$ & 3300 Mpc & 2970 Mpc & $-10\%$ & 2850 Mpc & $-14\%$ \\
			$d_L$ & 4710 Mpc & 5940 Mpc & $+26\%$ & 5700 Mpc & $+21\%$ \\
			$d_A$ & 1650 Mpc & 1485 Mpc & $-10\%$ & 1425 Mpc & $-14\%$ \\
			\hline
		\end{tabular}
		\caption{Vergleich der Distanzmaße bei $z = 1$}
	\end{table}
	
	\subsection{Bei $z = 1100$ (CMB)}
	
	\begin{table}[h]
		\centering
		\begin{tabular}{|l|c|c|c|c|c|}
			\hline
			\textbf{Größe} & \textbf{\(\Lambda\)CDM ($H_0$=70)} & \textbf{T0 ($H_0$=70)} & \textbf{Differenz} & \textbf{T0 ($H_0$=73)} & \textbf{Differenz} \\
			\hline
			$d_A$ & 13.5 Mpc & 28.9 Mpc & $+114\%$ & 27.7 Mpc & $+105\%$ \\
			$\theta$ & $1^\circ$ & $5.8^\circ$ & $+480\%$ & $6.1^\circ$ & $+510\%$ \\
			\hline
		\end{tabular}
		\caption{Vergleich der Distanzmaße und Winkelgrößen beim CMB ($z = 1100$)}
	\end{table}
	
	\section{Grafische Darstellung der Ergebnisse}
	
	Die quantitativen Unterschiede zwischen dem T0-Modell und dem \(\Lambda\)CDM-Standardmodell lassen sich anschaulich grafisch darstellen. Im Folgenden präsentieren wir die wichtigsten Beziehungen und ihre Unterschiede in beiden Modellen.
	
	\subsection{Vergleich der physischen Distanz}
	
	\begin{figure}[H]
		\centering
		\begin{tikzpicture}
			\begin{axis}[
				width=14cm, height=8cm,
				xlabel={Rotverschiebung $z$},
				ylabel={Physische Distanz $d$ [Mpc]},
				xmin=0, xmax=2,
				ymin=0, ymax=6000,
				grid=both,
				legend pos=north west,
				legend style={fill=white, fill opacity=0.7}
				]
				% ΛCDM-Modellkurve (etwas vereinfacht)
				\addplot[color=blue, thick, domain=0:2, samples=100] {3300*x*(1-0.2*x+0.07*x^2)};
				\addlegendentry{\(\Lambda\)CDM}
				% T0-Modell mit H0=70
				\addplot[color=red, thick, domain=0:2, samples=100] {2970*ln(1+x)/0.693};
				\addlegendentry{T0 ($H_0=70$)}
				% T0-Modell mit H0=73
				\addplot[color=orange, thick, dashed, domain=0:2, samples=100] {2850*ln(1+x)/0.693};
				\addlegendentry{T0 ($H_0=73$)}
			\end{axis}
		\end{tikzpicture}
		\caption{Vergleich der physischen Distanz $d$ als Funktion der Rotverschiebung $z$. Bei $z=1$ ist die Distanz im T0-Modell etwa 10–14\% kleiner. Bei höheren Rotverschiebungen wird der Unterschied noch ausgeprägter.}
		\label{fig:phys_distanz}
	\end{figure}
	
	\subsection{Vergleich der Luminositätsdistanz}
	
	\begin{figure}[H]
		\centering
		\begin{tikzpicture}
			\begin{axis}[
				width=14cm, height=8cm,
				xlabel={Rotverschiebung $z$},
				ylabel={Luminositätsdistanz $d_L$ [Mpc]},
				xmin=0, xmax=2,
				ymin=0, ymax=14000,
				grid=both,
				legend pos=north west,
				legend style={fill=white, fill opacity=0.7}
				]
				% ΛCDM-Modellkurve
				\addplot[color=blue, thick, domain=0:2, samples=100] {(1+x)*3300*x*(1-0.2*x+0.07*x^2)};
				\addlegendentry{\(\Lambda\)CDM}
				% T0-Modell mit H0=70
				\addplot[color=red, thick, domain=0:2, samples=100] {5940*ln(1+x)*(1+x)/0.693};
				\addlegendentry{T0 ($H_0=70$)}
				% T0-Modell mit H0=73
				\addplot[color=orange, thick, dashed, domain=0:2, samples=100] {5700*ln(1+x)*(1+x)/0.693};
				\addlegendentry{T0 ($H_0=73$)}
			\end{axis}
		\end{tikzpicture}
		\caption{Vergleich der Luminositätsdistanz $d_L$ als Funktion der Rotverschiebung $z$. Im Gegensatz zur physischen Distanz ist die Luminositätsdistanz im T0-Modell bei $z=1$ etwa 21–26\% größer. Somit erscheinen Objekte bei gleicher Rotverschiebung schwächer als vom \(\Lambda\)CDM-Modell vorhergesagt.}
		\label{fig:luminositaetsdistanz}
	\end{figure}
	
	\subsection{Vergleich der Winkeldurchmesser-Distanz}
	
	\begin{figure}[H]
		\centering
		\begin{tikzpicture}
			\begin{axis}[
				width=14cm, height=8cm,
				xlabel={Rotverschiebung $z$},
				ylabel={Winkeldurchmesser-Distanz $d_A$ [Mpc]},
				xmin=0, xmax=2,
				ymin=0, ymax=2000,
				grid=both,
				legend pos=north east,
				legend style={fill=white, fill opacity=0.7}
				]
				% ΛCDM-Modellkurve
				\addplot[color=blue, thick, domain=0:2, samples=100] {3300*x*(1-0.2*x+0.07*x^2)/(1+x)};
				\addlegendentry{\(\Lambda\)CDM}
				% T0-Modell mit H0=70
				\addplot[color=red, thick, domain=0:2, samples=100] {1485*ln(1+x)/(0.693*(1+x))};
				\addlegendentry{T0 ($H_0=70$)}
				% T0-Modell mit H0=73
				\addplot[color=orange, thick, dashed, domain=0:2, samples=100] {1425*ln(1+x)/(0.693*(1+x))};
				\addlegendentry{T0 ($H_0=73$)}
			\end{axis}
		\end{tikzpicture}
		\caption{Winkeldurchmesser-Distanz $d_A$ als Funktion der Rotverschiebung $z$. Bei $z=1$ ist $d_A$ im T0-Modell etwa 10–14\% kleiner. Dies bedeutet, dass Objekte gleicher Größe im T0-Modell unter einem größeren Winkel erscheinen würden.}
		\label{fig:winkeldistanz}
	\end{figure}
	
	\subsection{CMB-Winkeldurchmesser-Distanz}
	
	\begin{figure}[H]
		\centering
		\begin{tikzpicture}
			\begin{axis}[
				width=14cm, height=8cm,
				xlabel={Rotverschiebung $z$},
				ylabel={Winkeldurchmesser-Distanz $d_A$ [Mpc]},
				xmin=0, xmax=1200,
				ymin=0, ymax=30,
				grid=both,
				legend pos=north east,
				legend style={fill=white, fill opacity=0.7},
				xtick={0, 200, 400, 600, 800, 1000, 1200},
				extra y ticks={13.5, 28.9},
				extra y tick labels={13.5, 28.9},
				extra y tick style={grid=major, grid style={dashed, red}}
				]
				% ΛCDM-Modellwert bei z=1100
				\addplot[color=blue, mark=*, mark size=4pt] coordinates {(1100, 13.5)};
				\addlegendentry{\(\Lambda\)CDM}
				% T0-Modell mit H0=70 Wert bei z=1100
				\addplot[color=red, mark=square*, mark size=4pt] coordinates {(1100, 28.9)};
				\addlegendentry{T0 ($H_0=70$)}
				% T0-Modell mit H0=73 Wert bei z=1100
				\addplot[color=orange, mark=diamond*, mark size=4pt] coordinates {(1100, 27.7)};
				\addlegendentry{T0 ($H_0=73$)}
				% Beispielkurven für beide Modelle (stark vereinfacht)
				\addplot[color=blue, thick, dashed, domain=0:1200, samples=100] {13.5/1100*x/(1+0.0004*x)};
				\addplot[color=red, thick, dashed, domain=0:1200, samples=100] {28.9/1100*x/(1+0.0004*x)};
			\end{axis}
		\end{tikzpicture}
		\caption{Winkeldurchmesser-Distanz $d_A$ für die CMB-Strahlung ($z=1100$). Der dramatische Unterschied zwischen den Modellen ist hier offensichtlich: Das T0-Modell sagt einen mehr als doppelt so großen Wert für $d_A$ (28.9 Mpc vs. 13.5 Mpc) voraus, was zu grundlegend unterschiedlichen Interpretationen der CMB-Anisotropien führt.}
		\label{fig:cmb_winkeldistanz}
	\end{figure}
	
	\subsection{CMB-Temperatur-Rotverschiebungs-Relation}
	
	\begin{figure}[H]
		\centering
		\begin{tikzpicture}
			\begin{axis}[
				width=14cm, height=8cm,
				xlabel={Rotverschiebung $z$},
				ylabel={CMB-Temperatur $T(z)$ [K]},
				xmin=0, xmax=5,
				ymin=0, ymax=20,
				grid=both,
				legend pos=north west,
				legend style={fill=white, fill opacity=0.7}
				]
				% ΛCDM-Modellkurve
				\addplot[color=blue, thick, domain=0:5, samples=100] {2.725*(1+x)};
				\addlegendentry{\(\Lambda\)CDM}
				% T0-Modellkurve
				\addplot[color=red, thick, domain=0:5, samples=100] {2.725*(1+x)*(1+0.008*ln(1+x))};
				\addlegendentry{T0 mit \(\beta = 0.008\)}
				% T0-Modell alternative Darstellung
				\addplot[color=orange, thick, dashed, domain=0:5, samples=100] {2.725*(1+x)^(1-0.008)};
				\addlegendentry{T0 mit \(\alpha = 0.008\)}
			\end{axis}
		\end{tikzpicture}
		\caption{CMB-Temperatur $T(z)$ als Funktion der Rotverschiebung $z$. Während das \(\Lambda\)CDM-Modell eine lineare Beziehung annimmt, sagt das T0-Modell eine leichte Modifikation voraus, die bei höheren Rotverschiebungen zunehmend auffällig wird. Diese Abweichung könnte durch Messungen des Sunyaev-Zeldovich-Effekts in Galaxienhaufen bei verschiedenen Rotverschiebungen getestet werden.}
		\label{fig:cmb_temperatur}
	\end{figure}
	
	\subsection{Vergleich der Distanzmaßverhältnisse}
	
	\begin{figure}[H]
		\centering
		\begin{tikzpicture}
			\begin{axis}[
				width=14cm, height=8cm,
				xlabel={Rotverschiebung $z$},
				ylabel={Verhältnis $d_L / d_A$},
				xmin=0, xmax=3,
				ymin=0, ymax=20,
				grid=both,
				legend pos=north west,
				legend style={fill=white, fill opacity=0.7}
				]
				% Gemeinsame Beziehung in beiden Modellen
				\addplot[color=black, thick, domain=0:3, samples=100] {(1+x)^2};
				\addlegendentry{Gemeinsame Beziehung: $d_L = d_A (1+z)^2$}
				% Markierte spezifische Werte
				\addplot[color=blue, mark=*, mark size=3pt] coordinates {(1, 4)}; 
				\addlegendentry{$z=1$: $d_L/d_A = 4$}
				\addplot[color=red, mark=square*, mark size=3pt] coordinates {(2, 9)}; 
				\addlegendentry{$z=2$: $d_L/d_A = 9$}
			\end{axis}
		\end{tikzpicture}
		\caption{Verhältnis zwischen Luminositätsdistanz $d_L$ und Winkeldurchmesser-Distanz $d_A$ als Funktion der Rotverschiebung $z$. Interessanterweise gilt die Beziehung $d_L = d_A (1+z)^2$ in beiden Modellen, was eine wichtige Konsistenzprüfung darstellt. Jedoch führen die unterschiedlichen absoluten Werte der Distanzen zu variierenden Interpretationen astronomischer Beobachtungen.}
		\label{fig:distanzverhaeltnisse}
	\end{figure}
	
	\subsection{Prozentuale Unterschiede zwischen den Modellen}
	
	\begin{figure}[H]
		\centering
		\begin{tikzpicture}
			\begin{axis}[
				width=14cm, height=8cm,
				xlabel={Rotverschiebung $z$},
				ylabel={Prozentuale Abweichung [\%]},
				xmin=0, xmax=2,
				ymin=-20, ymax=40,
				grid=both,
				legend pos=south east,
				legend style={fill=white, fill opacity=0.7}
				]
				% Physische Distanz
				\addplot[color=blue, thick, domain=0.05:2, samples=100] {-10-(3*ln(1+x))};
				\addlegendentry{Physische Distanz}
				% Luminositätsdistanz
				\addplot[color=red, thick, domain=0:2, samples=100] {26+(5*ln(1+x))};
				\addlegendentry{Luminositätsdistanz}
				% Winkeldurchmesser-Distanz
				\addplot[color=green!60!black, thick, domain=0.05:2, samples=100] {-10-(3*ln(1+x))};
				\addlegendentry{Winkeldurchmesser-Distanz}
				% Null-Linie
				\addplot[color=black, domain=0:2] {0};
			\end{axis}
		\end{tikzpicture}
		\caption{Prozentuale Abweichung der Distanzmaße im T0-Modell im Vergleich zum \(\Lambda\)CDM-Modell als Funktion der Rotverschiebung $z$. Positive Werte zeigen an, dass der Wert im T0-Modell größer ist. Beachten Sie die gegensätzlichen Trends: Während physische Distanz und Winkeldurchmesser-Distanz im T0-Modell kleiner sind (negative Abweichung), ist die Luminositätsdistanz größer (positive Abweichung). Diese gegensätzlichen Effekte könnten das Hubble-Spannungsproblem erklären.}
		\label{fig:prozentuale_abweichungen}
	\end{figure}
	
	\subsection{Winkelgröße typischer Strukturen}
	
	\begin{figure}[H]
		\centering
		\begin{tikzpicture}
			\begin{axis}[
				width=14cm, height=8cm,
				xlabel={Rotverschiebung $z$},
				ylabel={Winkelgröße $\theta$ [Grad] für Objekt mit $r=150$ Mpc},
				xmin=900, xmax=1300,
				ymin=0, ymax=7,
				grid=both,
				legend pos=north west,
				legend style={fill=white, fill opacity=0.7}
				]
				% Vereinfachte Darstellung statt komplexer Formeln
				% ΛCDM-Modellkurve - vereinfacht
				\addplot[color=blue, thick, domain=900:1300, samples=100] {1 + 0.0005*(1100-x)};
				\addlegendentry{\(\Lambda\)CDM}
				% T0-Modell mit H0=70 Kurve - vereinfacht
				\addplot[color=red, thick, domain=900:1300, samples=100] {5.8 + 0.003*(1100-x)};
				\addlegendentry{T0 ($H_0=70$)}
				% T0-Modell mit H0=73 Kurve - vereinfacht
				\addplot[color=orange, thick, dashed, domain=900:1300, samples=100] {6.1 + 0.003*(1100-x)};
				\addlegendentry{T0 ($H_0=73$)}
				% Hervorgehobene spezifische Punkte
				\addplot[color=blue, mark=*, mark size=4pt] coordinates {(1100, 1)};
				\addplot[color=red, mark=square*, mark size=4pt] coordinates {(1100, 5.8)};
				\addplot[color=orange, mark=diamond*, mark size=4pt] coordinates {(1100, 6.1)};
			\end{axis}
		\end{tikzpicture}
		\caption{Winkelgröße $\theta$ einer kosmologischen Struktur mit einer physischen Größe $r=150$ Mpc (typische BAO-Skala) als Funktion der Rotverschiebung $z$ im CMB-Bereich. Der dramatische Unterschied in der vorhergesagten Winkelgröße (etwa $1^\circ$ im \(\Lambda\)CDM-Modell vs. etwa $5.8$-$6.1^\circ$ im T0-Modell) ist ein kritischer Test zwischen den Modellen.}
		\label{fig:winkelgroesse}
	\end{figure}
	
	\section{CMB-Temperatur und Modellinterpretation}
	
	\subsection{\(\Lambda\)CDM-Modell}
	
	Im Standardmodell wird die CMB-Temperatur als Folge der kosmischen Expansion interpretiert:
	
	\begin{equation}
		T(z) = T_0 (1 + z)
	\end{equation}
	
	Mit $T_0 = 2.725$ K als aktueller Temperatur.
	
	\subsection{T0-Modell}
	
	Im T0-Modell zeigt die CMB-Temperatur eine leichte wellenlängenabhängige Modifikation aufgrund der Dynamik von \( \Tfield \):
	
	\begin{equation}
		T(z) = T_0 (1 + z) (1 + \beta \ln(1 + z))
	\end{equation}
	
	Mit \(\beta \approx 0.008\), was zu einer subtilen Abweichung vom Standardmodell führt.
	
	\subsubsection{Unsicherheit bei \(\beta\) und Modellgrenzen}
	
	Der Wert \(\beta \approx 0.008\) in \( T(z) = T_0 (1 + z) (1 + \beta \ln(1 + z)) \) und \( z(\lambda) = z_0 (1 + \beta \ln(\lambda/\lambda_0)) \) wird aus kosmologischen Einschränkungen und perturbativen Berechnungen abgeleitet \cite{pascher_params_2025}, was subtile Abweichungen ermöglicht, die über den Sunyaev-Zeldovich-Effekt oder JWST-Beobachtungen testbar sind. Das Setzen von \(\beta = 1\) als Vereinfachung in natürlichen Einheiten \cite{pascher_temp_2025} führt jedoch zu erheblichen Inkonsistenzen: \( T(1100) \approx 24,000 \, \text{K} \) (vs. \(\sim 3000 \, \text{K}\)) und eine übertriebene wellenlängenabhängige Rotverschiebung (z. B. \( z(\lambda) \approx 3.3 \) für \(\lambda/\lambda_0 = 10\)). Dies deutet auf eine potenzielle Schwäche in der Grundlage des T0-Modells hin, da die Verknüpfung zwischen \(\Tfield\), Rotverschiebung und Temperatur möglicherweise nicht vollständig geklärt ist. Zudem birgt die Abhängigkeit des Modells von im Standardmodell interpretierten Daten (z. B. CMB-Kalibrierung) das Risiko einer zirkulären Argumentation, bei der \(\beta\) möglicherweise \(\Lambda\)CDM-Voreingenommenheiten widerspiegelt. Der wahre kosmologische Rahmen könnte zwischen diesen Modellen liegen, was eine unabhängige Validierung erforderlich macht, um diese Zirkularität aufzulösen.
	
	\subsection{Testbare Vorhersagen}
	
	Diese Unterschiede führen zu spezifischen vorhersehbaren Effekten:
	
	\begin{enumerate}
		\item \textbf{Wellenlängenabhängige Rotverschiebung}: Das T0-Modell sagt eine leichte Wellenlängenabhängigkeit der Rotverschiebung voraus: $z(\lambda) = z_0(1 + \beta\cdot\ln(\lambda/\lambda_0))$.
		\item \textbf{Umgebungsabhängige Rotverschiebung}: Im T0-Modell sollte die Rotverschiebung in dichten kosmischen Regionen leicht von der in kosmischen Leerräumen abweichen: $z_\text{Cluster}/z_\text{Leerraum} \approx 1 + \delta(\rho_\text{Cluster}-\rho_\text{Leerraum})/\rho_0$.
		\item \textbf{Temperatur-Rotverschiebungs-Relation}: Das T0-Modell sagt $T(z) = T_0(1+z)^{(1-\alpha)}$ mit $\alpha \approx \beta \approx 0.008$ voraus, was durch SZ-Effektmessungen getestet werden könnte.
	\end{enumerate}
	
	\section{Schlussfolgerung}
	
	Die Analyse der additiven und kompensatorischen Effekte zwischen dem T0-Modell und dem \(\Lambda\)CDM-Standardmodell zeigt:
	
	\begin{enumerate}
		\item Die Effekte addieren sich nicht einfach linear über alle gemessenen Größen, sondern bilden ein komplexes Zusammenspiel aus Verstärkung und Kompensation.
		\item Bei niedrigen bis mittleren Rotverschiebungen ($z \approx 1$) sind die Unterschiede moderat ($\sim10$-$26\%$), aber systematisch.
		\item Bei hohen Rotverschiebungen ($z = 1100$, CMB) werden die Unterschiede dramatisch ($>100\%$ für $d_A$, $>400\%$ für Winkelgrößen).
		\item Diese Diskrepanzen könnten erklären, warum verschiedene Messmethoden unterschiedliche kosmologische Parameter liefern.
		\item Die kompensatorischen Effekte zwischen Helligkeits- und Distanzmessungen könnten eine natürliche Erklärung für das Hubble-Spannungsproblem bieten.
	\end{enumerate}
	
	Die systematischen Unterschiede zwischen den Modellen bieten konkrete Testmöglichkeiten für zukünftige Präzisionsmessungen in der Kosmologie und könnten letztlich zwischen diesen grundlegend unterschiedlichen Weltbildern unterscheiden.
	
	\begin{thebibliography}{9}
		\bibitem{pascher_galaxies_2025} Pascher, J. (2025). \href{\repobase/pdf/Deutsch/Massenvariation in Galaxien - Eine Analyse im T0-Modell mit emergenter Gravitation.pdf}{Massenvariation in Galaxien: Eine Analyse im T0-Modell mit emergenter Gravitation}. 30. März 2025.
		\bibitem{pascher_params_2025} Pascher, J. (2025). Zeit-Masse-Dualitätstheorie (T0-Modell): Ableitung der Parameter \(\kappa\), \(\alpha\) und \(\beta\). 30. März 2025.
		\bibitem{pascher_temp_2025} Pascher, J. (2025). Anpassung der Temperatureinheiten in natürlichen Einheiten und CMB-Messungen. 2. April 2025.
		\bibitem{Planck2018}
		Planck Collaboration, Aghanim, N., et al. (2020). 
		\textit{Planck 2018 Ergebnisse. VI. Kosmologische Parameter}. 
		Astronomy \& Astrophysics, 641, A6. 
		DOI: 10.1051/0004-6361/201833910.
		\bibitem{LambdaCDM}
		Peebles, P. J. E. (1993). 
		\textit{Grundlagen der physikalischen Kosmologie}. 
		Princeton University Press, Princeton, NJ.
		\bibitem{HubbleTension}
		Riess, A. G., et al. (2021). 
		\textit{Eine umfassende Messung des lokalen Werts der Hubble-Konstanten mit 1\% Präzision vom SH0ES-Team}. 
		The Astrophysical Journal, 934(1), L7. 
		DOI: 10.3847/2041-8213/ac5c5b.
		\bibitem{ACT}
		Aiola, S., et al. (2020). 
		\textit{Das Atacama Cosmology Telescope: DR4 Karten und kosmologische Parameter}. 
		Journal of Cosmology and Astroparticle Physics, 2020(12), 047. 
		DOI: 10.1088/1475-7516/2020/12/047.
		\bibitem{SPT}
		Benson, B. A., et al. (2014). 
		\textit{SPT-3G: Ein nächstgenerations CMB-Polarisationsexperiment am Südpol-Teleskop}. 
		Proceedings of SPIE, 9153, 91531P. 
		DOI: 10.1117/12.2056701.
		\bibitem{Friedmann}
		Friedmann, A. (1922). 
		\textit{Über die Krümmung des Raumes}. 
		Zeitschrift für Physik, 10(1), 377–386. 
		DOI: 10.1007/BF01332580.
		\bibitem{SunyaevZeldovich}
		Sunyaev, R. A., \& Zeldovich, Y. B. (1972). 
		\textit{Die Beobachtungen der Reliktstrahlung als Test der Natur der Röntgenstrahlung von Galaxienhaufen}. 
		Comments on Astrophysics and Space Physics, 4, 173.
		\bibitem{CMBTheory}
		Dodelson, S. (2003). 
		\textit{Moderne Kosmologie}. 
		Academic Press, San Diego, CA.
	\end{thebibliography}
	
\end{document}