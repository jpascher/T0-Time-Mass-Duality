\documentclass{article}
\usepackage[utf8]{inputenc}
\usepackage[T1]{fontenc}
\usepackage{lmodern}
\usepackage[ngerman]{babel}
\usepackage{amsmath,amssymb,physics}
\usepackage{graphicx,tikz,pgfplots}
\pgfplotsset{compat=1.18}
\usepackage{geometry}
\usepackage[colorlinks=true, linkcolor=blue, citecolor=blue, urlcolor=blue]{hyperref}
\usepackage{booktabs}
\usepackage{siunitx}
\usepackage{cleveref}
\usepackage{amsthm}

\geometry{a4paper, margin=2.5cm}

% Theorem styles
\newtheorem{theorem}{Satz}[section]
\newtheorem{proposition}[theorem]{Proposition}

% Custom commands
\newcommand{\Tfield}{T(x)}
\newcommand{\alphaEM}{\alpha_{\text{EM}}}
\newcommand{\betaT}{\beta_{\text{T}}}

\title{Energie als fundamentale Einheit: \\ Natürliche Einheiten mit \(\alphaEM = 1\) im T0-Modell}
\author{Johann Pascher}
\date{26. März 2025}

\begin{document}
	
	\maketitle
	
	\begin{abstract}
		Diese Arbeit untersucht die Konsequenzen der Annahme, dass die Feinstrukturkonstante \(\alphaEM = 1\) in einem System natürlicher Einheiten (\(\hbar = c = 1\)) gesetzt wird, mit Anwendung auf das T0-Modell der Zeit-Masse-Dualität. Dabei wird Energie als fundamentale Einheit identifiziert, auf die alle physikalischen Größen zurückgeführt werden können. Die Analyse umfasst dimensionale Umformulierungen, vereinfachte Grundgleichungen und kosmologische Implikationen im Kontext des T0-Modells, das absolute Zeit und variable Masse postuliert.
	\end{abstract}
	
	\tableofcontents
	\newpage
	
	\section{Einleitung}
	In der theoretischen Physik werden üblicherweise \(c\) und \(\hbar\) auf eins gesetzt, wie von Planck eingeführt \cite{Planck1899}. Diese Arbeit untersucht die Konsequenzen, wenn zusätzlich die Feinstrukturkonstante \(\alphaEM = 1\) gesetzt wird, und wendet dies auf das T0-Modell an, das eine absolute Zeit und variable Masse annimmt \cite{pascher_galaxies_2025}.
	
	\section{Natürliche Einheiten und \(\alphaEM = 1\)}
	\begin{theorem}[Definition von \(\alphaEM = 1\)]
		Die Feinstrukturkonstante ist \cite{Feynman1985}:
		\begin{equation}
			\alphaEM = \frac{e^2}{4\pi\varepsilon_0 \hbar c} \approx \frac{1}{137.036}
		\end{equation}
		Mit \(\alphaEM = 1\), \(\hbar = c = 1\):
		\begin{equation}
			e = \sqrt{4\pi\varepsilon_0}
		\end{equation}
	\end{theorem}
	
	\textbf{Hinweis}: Hier bezeichnet \(\alphaEM\) die Feinstrukturkonstante, nicht die Wien-Konstante \(\alpha_W \approx 2.82\), wie in \cite{pascher_temp_2025} untersucht.
	
	\section{Energie als fundamentale Einheit}
	\begin{theorem}[Energie als Basis]
		Alle Größen lassen sich auf Energie zurückführen \cite{Duff2002}:
		\begin{itemize}
			\item Länge: \([L] = [E^{-1}]\)
			\item Zeit: \([T] = [E^{-1}]\)
			\item Masse: \([M] = [E]\)
			\item Ladung: \([Q] = [\sqrt{4\pi}]\) (dimensionslos)
		\end{itemize}
	\end{theorem}
	
	Im T0-Modell wird dies durch \(\Tfield = \frac{\hbar}{mc^2}\) ergänzt, wobei \(m\) variabel ist und Energie eine zentrale Rolle spielt.
	
	\section{Vereinfachte Grundgleichungen}
	\begin{itemize}
		\item Maxwell-Gleichungen \cite{Feynman1985}:
		\begin{align}
			\nabla \cdot \vec{E} &= \rho \\
			\nabla \times \vec{B} - \frac{\partial \vec{E}}{\partial t} &= \vec{j}
		\end{align}
		\item Schrödinger-Gleichung:
		\begin{equation}
			i \frac{\partial \psi}{\partial t} = -\frac{1}{2m} \nabla^2 \psi + V \psi
		\end{equation}
	\end{itemize}
	
	\section{Tabelle der umgeformten Größen}
	\begin{center}
		\begin{tabular}{|l|c|c|}
			\hline
			\textbf{Physikalische Größe} & \textbf{SI-Einheiten} & \textbf{\(\hbar = c = \alphaEM = 1\)} \\
			\hline
			Länge & m & \(\text{eV}^{-1}\) \\
			Zeit & s & \(\text{eV}^{-1}\) \\
			Masse & kg & eV \\
			Energie & J & eV \\
			Ladung & C & dimensionslos \\
			El. Feld & V/m & \(\text{eV}^2\) \\
			Mag. Feld & T & \(\text{eV}^2\) \\
			\hline
		\end{tabular}
	\end{center}
	
	\section{Kosmologische Implikationen}
	Die Annahme \(\alphaEM = 1\) könnte im T0-Modell \cite{pascher_galaxies_2025}:
	\begin{itemize}
		\item Elektromagnetische Wechselwirkungen stärker mit Gravitation verbinden, da \(\Tfield\) Gravitation emergent erklärt.
		\item Eine einheitliche Energiebeschreibung ermöglichen, konsistent mit der Rotverschiebung durch Energieverlust an \(\Tfield\) \cite{pascher_messdifferenzen_2025}.
	\end{itemize}
	
	\subsection{Dimensionslose Parameter im T0-Modell}
	Im T0-Modell wird die wellenlängenabhängige Rotverschiebung durch den Parameter \(\betaT^{\text{SI}} \approx 0.008\) beschrieben \cite{pascher_params_2025}. In natürlichen Einheiten kann analog zu \(\alphaEM = 1\) auch \(\betaT^{\text{nat}} = 1\) gesetzt werden, wie in \cite{pascher_beta_2025} diskutiert.
	
	Bei gleichzeitiger Setzung von \(\alphaEM = 1\) und \(\betaT^{\text{nat}} = 1\) ergeben sich signifikante Abweichungen von Standardmodell-Vorhersagen (z.B. \(z(\lambda) \approx 3.3\) für \(\lambda/\lambda_0 = 10\)). Diese Abweichungen sind jedoch nicht als "unphysikalisch" zu verstehen, sondern können auf einen Standardmodell-Bias in der Interpretation kosmologischer Daten hindeuten \cite{pascher_alphabeta_2025}.
	
	Die Konsistenz der gleichzeitigen Setzung von \(\alphaEM = 1\) und \(\betaT^{\text{nat}} = 1\) und die Umrechnung in SI-Einheiten werden in \cite{pascher_alphabeta_2025} detailliert untersucht.
	
	\begin{figure}[h]
		\centering
		\begin{tikzpicture}
			\begin{axis}[
				xlabel={Energie [eV]},
				ylabel={Länge [eV\(^{-1}\)]},
				xlabel style={font=\large},
				ylabel style={font=\large},
				tick label style={font=\normalsize},
				xmin=0, xmax=10,
				ymin=0, ymax=10,
				legend pos=north east,
				legend style={font=\large},
				grid=both,
				minor tick num=1
				]
				\addplot[blue, ultra thick, domain=0.1:10, samples=100] {1/x};
				\legend{\(L = E^{-1}\)}
			\end{axis}
		\end{tikzpicture}
		\caption{Beziehung zwischen Energie und Länge im \(\alphaEM = 1\)-System.}
	\end{figure}
	
	\section{Philosophische Implikationen}
	\begin{itemize}
		\item Energie als fundamentalste Eigenschaft der Realität \cite{Wilczek2008}, im T0-Modell durch absolute Zeit und variable Masse unterstützt.
		\item Raum und Zeit als emergente Eigenschaften eines Energiefeldes \cite{Verlinde2011}, kompatibel mit \(\Tfield\) als Grundfeld.
	\end{itemize}
	
	\section{Zusammenfassung}
	Durch \(\alphaEM = 1\) wird Energie zur fundamentalen Einheit, die im T0-Modell eine tiefere Einheit von Zeit, Masse und Gravitation offenbart. Diese Vereinfachung steht im Einklang mit dem allgemeinen Prinzip, dass fundamentale dimensionslose Parameter in einer vollständig natürlichen Formulierung einfache Werte annehmen sollten. Ähnlich wie die Setzung \(\betaT^{\text{nat}} = 1\) führt auch \(\alphaEM = 1\) zu einer konzeptionell klareren Theorie, in der die Dimensionen aller physikalischen Größen auf eine einzige fundamentale Dimension (Energie) zurückgeführt werden können. Für eine umfassende Analyse der Konsistenz beider Vereinfachungen wird auf \cite{pascher_alphabeta_2025} verwiesen.
	
	\begin{thebibliography}{10}
		\bibitem{Planck1899} Planck, M. (1899). \textit{Über irreversible Strahlungsvorgänge}. Sitzungsberichte der Preußischen Akademie der Wissenschaften, 5, 440-480.
		\bibitem{Feynman1985} Feynman, R. P. (1985). \textit{QED: Die seltsame Theorie des Lichts und der Materie}. Princeton University Press.
		\bibitem{Duff2002} Duff, M. J., Okun, L. B., \& Veneziano, G. (2002). \textit{Trialog über die Anzahl fundamentaler Konstanten}. Journal of High Energy Physics, 2002(03), 023.
		\bibitem{Verlinde2011} Verlinde, E. (2011). \textit{Über den Ursprung der Gravitation und die Gesetze Newtons}. Journal of High Energy Physics, 2011(4), 29.
		\bibitem{Wilczek2008} Wilczek, F. (2008). \textit{Die Leichtigkeit des Seins: Masse, Äther und die Vereinigung der Kräfte}. Basic Books.
		\bibitem{pascher_galaxies_2025} Pascher, J. (2025). \href{https://github.com/jpascher/T0-Time-Mass-Duality/tree/main/2/pdf/Deutsch/Massenvariation in Galaxien - Eine Analyse im T0-Modell mit emergenter Gravitation.pdf}{Massenvariation in Galaxien: Eine Analyse im T0-Modell mit emergenter Gravitation}. 30. März 2025.
		\bibitem{pascher_messdifferenzen_2025} Pascher, J. (2025). \href{https://github.com/jpascher/T0-Time-Mass-Duality/tree/main/2/pdf/Deutsch/Analyse der Messdifferenzen zwischen dem T0-Modell und dem Standardmodell.pdf}{Kompensatorische und additive Effekte: Eine Analyse der Messdifferenzen zwischen dem T0-Modell und dem \(\Lambda\)CDM-Standardmodell}. 2. April 2025.
		\bibitem{pascher_params_2025} Pascher, J. (2025). Zeit-Masse-Dualitätstheorie (T0-Modell): Ableitung der Parameter \(\kappa\), \(\alpha\) und \(\beta\). 30. März 2025.
		\bibitem{pascher_temp_2025} Pascher, J. (2025). Anpassung der Temperatureinheiten in natürlichen Einheiten und CMB-Messungen. 2. April 2025.
		\bibitem{pascher_beta_2025} Pascher, J. (2025). Dimensionslose Parameter im T0-Modell: Die Setzung von \(\beta = 1\) in natürlichen Einheiten. 4. April 2025.
		\bibitem{pascher_alphabeta_2025} Pascher, J. (2025). Vereinheitlichtes Einheitensystem im T0-Modell: Die Konsistenz von \(\alpha = 1\) und \(\beta = 1\). 5. April 2025.
	\end{thebibliography}
	
\end{document}