\documentclass[12pt,a4paper]{article}
\usepackage[utf8]{inputenc}
\usepackage[T1]{fontenc}
\usepackage[ngerman]{babel}
\usepackage[left=2cm,right=2cm,top=2cm,bottom=2cm]{geometry}
\usepackage{lmodern}
\usepackage{parskip}

% Mathematische Pakete
\usepackage{amsmath, amssymb, amsthm, mathtools, physics}
\usepackage{siunitx}

% Grafik- und Diagrammpakete
\usepackage{graphicx}
\usepackage{tikz, tikz-feynman}
\usepackage{pgfplots}
\pgfplotsset{compat=1.18}

% Tabellen und Formatierung
\usepackage{booktabs}
\usepackage{array}
\usepackage[table,xcdraw]{xcolor}

% Theoreme und Referenzen
\usepackage{thmtools}


% Boxen und spezielle Formatierungen
\usepackage{tcolorbox}


\tcbuselibrary{theorems, breakable}

% Hyperlinks und PDF-Metadaten
\usepackage{hyperref}
\usepackage{cleveref}
\hypersetup{
	colorlinks=true,
	linkcolor=blue,
	citecolor=blue,
	urlcolor=blue,
	pdftitle={Zeit-Masse-Dualitätstheorie (T0-Modell)},
	pdfauthor={Johann Pascher},
	pdfsubject={Theoretische Physik},
	pdfkeywords={T0-Modell, natürliche Einheiten, Zeit-Masse-Dualität}
}

% Benutzerdefinierte Befehle
\newcommand{\Tfield}{T(x)}
\newcommand{\DcovT}[1]{\Tfield D_\mu #1 + #1 \partial_\mu \Tfield}
\newcommand{\HiggsLagr}{\mathcal{L}_{\text{Higgs-T}}}
\newcommand{\FermionLagr}{\mathcal{L}_{\text{Fermion-T}}}
\newcommand{\BosonLagr}{\mathcal{L}_{\text{Boson-T}}}
\newcommand{\Mpl}{M_{\text{Pl}}}
\newcommand{\alphaEM}{\alpha_{\text{EM}}}
\newcommand{\betaT}{\beta_{\text{T}}}
\newcommand{\e}{\mathrm{e}} % Exponentialfunktion
\newcommand{\di}{\mathrm{d}} % Differential (alternative zu \dd aus physics)

% Theoreme
\declaretheorem[name=Theorem,numberwithin=section]{theorem}
\declaretheorem[name=Lemma,sibling=theorem]{lemma}

\title{Zeit-Masse-Dualitätstheorie (T0-Modell): \\ Herleitung der Parameter \(\kappa\), \(\alpha\) und \(\beta\)}
\author{Johann Pascher}
\date{4.4.2025}

\begin{document}
	
	\maketitle
	
	\section*{Einführung}
	
	Diese Arbeit untersucht die Verbindung zwischen natürlichen Einheitensystemen und dimensionslosen Konstanten im T0-Modell der Zeit-Masse-Dualitätstheorie. Es wird argumentiert, dass der Parameter \(\beta \approx 0.008\) in der Temperatur-Rotverschiebungs-Relation \(T(z) = T_0 (1+z)(1+\beta\ln(1+z))\) in natürlichen Einheiten auf \(\beta = 1\) gesetzt werden kann, analog zur Wienschen Konstante \(\alpha_W\) \cite{pascher_temp_2025}. Zusätzlich werden die Parameter \(\kappa\), \(\alpha\) und \(\beta\) des T0-Modells detailliert abgeleitet und mit kosmologischen Implikationen verknüpft. Für eine weiterführende Analyse der Konsistenz bei gleichzeitiger Setzung der Feinstrukturkonstante \(\alphaEM = 1\) und des Parameters \(\betaT = 1\) wird auf \cite{pascher_alphabeta_2025} verwiesen.
	
	\section{Dimensionslose Parameter in fundamentalen Theorien}
	
	\subsection{Historische Entwicklung und Prinzipien}
	
	Die Physik zeigt eine Entwicklung hin zu Einheitensystemen, in denen Naturkonstanten auf 1 gesetzt werden:
	\begin{itemize}
		\item Maxwell: \(c\) als fundamentale Konstante
		\item Relativitätstheorie: \(c = 1\)
		\item Quantenmechanik: \(\hbar = 1\)
		\item Quantengravitation: \(G = 1\)
	\end{itemize}
	Dimensionslose Parameter sollten einfach sein (z. B. 1, \(\pi\)). \(\betaT^{\text{SI}} \approx 0.008\) deutet auf ein nicht optimales System hin.
	
	\subsection{Die Bedeutung der „richtigen" natürlichen Einheiten}
	
	Komplexe Werte wie \(\betaT^{\text{SI}} \approx 0.008\) suggerieren, dass die Formulierung nicht fundamental ist. Historische Beispiele:
	\begin{itemize}
		\item \(c = 1\) in geeigneten Einheiten
		\item \(\hbar = 1\) in Quanteneinheiten
		\item \(G = 1\) in Planck-Einheiten
	\end{itemize}
	
	\section{Die charakteristische Längenskala \(r_0\)}
	
	\subsection{Neudefinition von \(r_0\) in natürlichen Einheiten}
	
	Die Längenskala \(r_0\) wird als \(r_0 = \xi \cdot l_P\) definiert, wobei \(\xi\) eine dimensionslose Konstante und \(l_P = \sqrt{\frac{\hbar G}{c^3}}\) die Planck-Länge ist. In natürlichen Einheiten (\(\hbar = c = G = 1\)) ist \(l_P = 1\), also \(r_0 = \xi\).
	
	Aus \(\betaT^{\text{nat}} = 1\) und:
	\begin{equation}
		\betaT^{\text{nat}} = \frac{\lambda_h^2 v^2}{16\pi^3 m_h^2} \cdot \frac{1}{r_0}
	\end{equation}
	folgt:
	\begin{equation}
		\xi = \frac{\lambda_h^2 v^2}{16\pi^3 m_h^2} \approx 1.33 \times 10^{-4}
	\end{equation}
	\begin{equation}
		r_0 \approx \frac{1}{7519} \cdot l_P
	\end{equation}
	
	\subsection{Physikalische Interpretation}
	
	\(r_0\) ist die Wechselwirkungslänge zwischen \(\Tfield\) und Higgs-Feld:
	\begin{itemize}
		\item Korrelation von Fluktuationen
		\item Übergang zwischen Quanten- und klassischer Gravitation
		\item Kopplung zum elektroschwachen Sektor
	\end{itemize}
	Dies deutet auf eine Planck-Skala-Verbindung hin.
	
	\subsection{Umrechnung zwischen natürlichen Einheiten und SI-Einheiten}
	
	\begin{align}
		r_{0,\text{SI}} &= \xi \cdot l_{P,\text{SI}} \\
		&= 1.33 \times 10^{-4} \cdot 1.616255 \times 10^{-35} \text{ m} \\
		&\approx 2.15 \times 10^{-39} \text{ m}
	\end{align}
	\begin{align}
		\betaT^{\text{SI}} &= \betaT^{\text{nat}} \cdot \frac{r_{0,\text{nat}}}{r_{0,\text{SI}}/l_{P,\text{SI}}} \\
		&= 1 \cdot \frac{\xi \cdot l_{P,\text{SI}}}{r_{0,\text{SI}}} \\
		&\approx 0.008
	\end{align}
	
	\subsection{Konsistenz mit der kosmologischen Längenskala \(L_T\)}
	
	\begin{equation}
		L_T \sim \frac{\Mpl}{m_h^2 v} \approx 6.3 \times 10^{27} \text{ m}
	\end{equation}
	\begin{equation}
		\frac{r_0}{L_T} \sim \frac{\lambda_h^2 v^4}{16\pi^3 \Mpl} \approx 3.41 \times 10^{-67}
	\end{equation}
	
	Dieses Verhältnis ist bemerkenswert, da es in der Größenordnung von $(m_e/M_{Pl})^2$ liegt, was möglicherweise auf eine tiefere Verbindung zur Elektronen-Masse hindeutet.
	
	\section{Parameterableitungen im T0-Modell}
	
	\subsection{Ableitung von \(\kappa\)}
	
	\begin{theorem}[Ableitung von \(\kappa\)]
		In natürlichen Einheiten:
		\begin{equation}
			\kappa = \betaT^{\text{nat}} \frac{y v}{r_g}, \quad r_g = \sqrt{\frac{M}{a_0}}
		\end{equation}
		In SI-Einheiten:
		\begin{equation}
			\kappa_{\text{SI}} = \betaT^{\text{SI}} \frac{y v c^2}{r_g^2} \approx 4.8 \times 10^{-11} \text{ m/s}^2
		\end{equation}
	\end{theorem}
	
	\subsection{Ableitung von \(\alpha\)}
	
	\begin{theorem}[Ableitung von \(\alpha\)]
		In natürlichen Einheiten:
		\begin{equation}
			\alpha = \frac{\lambda_h^2 v}{L_T}
		\end{equation}
		In SI-Einheiten:
		\begin{equation}
			\alpha_{\text{SI}} = \frac{\lambda_h^2 v c^2}{L_T} \approx 2.3 \times 10^{-18} \text{ m}^{-1}
		\end{equation}
	\end{theorem}
	
	\subsection{Ableitung von \(\beta\): Von natürlichen zu SI-Einheiten}
	
	\begin{theorem}[Ableitung von \(\beta\)]
		In natürlichen Einheiten: \(\betaT^{\text{nat}} = 1\). Perturbativ:
		\begin{equation}
			\betaT^{\text{nat}} = \frac{\lambda_h^2 v^2}{4\pi^2 \lambda_0 \alpha_0}
		\end{equation}
		In SI-Einheiten:
		\begin{equation}
			\betaT^{\text{SI}} = \frac{(2\pi)^4 m_h^2}{16 \pi^2 v^4 y^2 \Mpl^2 \lambda_0^4 \alpha_0} \approx 0.008
		\end{equation}
	\end{theorem}
	
	Hierbei sind $\lambda_0$ und $\alpha_0$ Parameter, die mit der Strukturkonstante des T0-Modells zusammenhängen. Es ist zu beachten, dass $\alpha_0$ nicht notwendigerweise mit der Feinstrukturkonstante $\alphaEM$ identisch ist, obwohl eine Beziehung zwischen beiden existieren könnte (siehe \cite{pascher_alphabeta_2025}).
	
	\subsection{Anwendung: Wellenlängenabhängige Rotverschiebung und Temperaturentwicklung}
	
	Aus der Setzung \(\betaT^{\text{nat}} = 1\) ergibt sich die Rotverschiebungs-Wellenlängen-Relation:
	\begin{equation}
		z(\lambda) = z_0 \left(1 + \betaT^{\text{SI}} \ln \frac{\lambda}{\lambda_0}\right)
	\end{equation}
	
	Und die Temperatur-Rotverschiebungs-Relation:
	\begin{equation}
		T(z) = T_0 (1 + z) (1 + \betaT^{\text{SI}} \ln(1 + z))
	\end{equation}
	
	\subsubsection{Feynman-Diagramm-Analyse}
	
	\begin{center}
		\feynmandiagram [horizontal=a to b] {
			a [particle=\(\gamma\)] -- [photon] b -- [photon] f [particle=\(\gamma\)],
			b -- [scalar, half left] c -- [scalar, half left] b,
			c -- [photon] d,
		};
	\end{center}
	
	Die quantenfeldtheoretische Analyse führt zum perturbativen Wert von \(\betaT^{\text{SI}} \approx 0.008\), der mit kosmologischen Beobachtungen konsistent ist. Eine tiefere theoretische Betrachtung legt jedoch nahe, dass \(\betaT^{\text{nat}} = 1\) in natürlichen Einheiten der fundamentalere Wert ist.
	
	\section{Interpretation und Kohärenz natürlicher Parameter}
	
	\subsection{Hierarchie der Einheiten und dimensionslosen Konstanten}
	
	\begin{enumerate}
		\item Naturkonstanten: \(c = \hbar = G = k_B = 1\)
		\item Dimensionslose Parameter: \(\alphaEM \approx 1/137\), \(\alpha_W \approx 2.82\) \cite{pascher_temp_2025}, \(\betaT^{\text{nat}} = 1\)
		\item Längenskalen: \(r_0 = \xi \cdot l_P\), \(\xi \approx 1.33 \times 10^{-4}\); \(L_T = \zeta \cdot l_P\), \(\zeta \sim 10^{62}\)
	\end{enumerate}
	
	\subsection{Verhältniszahlen zwischen Längenskalen im T0-Modell}
	
	\begin{itemize}
		\item \(l_{P,\text{SI}} \approx 1.616 \times 10^{-35} \text{ m}\)
		\item \(\lambda_h \approx 1.576 \times 10^{-18} \text{ m}\)
		\item \(r_{0,\text{SI}} \approx 2.15 \times 10^{-39} \text{ m}\)
		\item \(L_T \approx 6.3 \times 10^{27} \text{ m}\)
	\end{itemize}
	\begin{align}
		\frac{r_0}{l_P} &\approx 1.33 \times 10^{-4} \\
		\frac{\lambda_h}{l_P} &\approx 9.75 \times 10^{16} \\
		\frac{L_T}{l_P} &\approx 3.9 \times 10^{62}
	\end{align}
	
	Diese Verhältniszahlen sind rein dimensionslos und unabhängig von der Wahl des Einheitensystems. Sie repräsentieren fundamentale Aspekte der Theorie und könnten auf tiefere Strukturen hindeuten.
	
	\subsection{Umrechnung zwischen Einheitensystemen}
	
	\begin{tcolorbox}[colback=blue!5!white, colframe=blue!75!black, title=Umrechnungsschema]
		\begin{enumerate}
			\item Längenskalen: \(L_{\text{SI}} = L_{\text{nat}} \cdot l_{P,\text{SI}}\)
			\item Energieskalen: \(E_{\text{SI}} = E_{\text{nat}} \cdot \sqrt{\frac{\hbar c^5}{G}}\)
			\item Dimensionslose Parameter: \(\betaT^{\text{SI}} = \betaT^{\text{nat}} \cdot \frac{\xi \cdot l_{P,\text{SI}}}{r_{0,\text{SI}}}\)
		\end{enumerate}
	\end{tcolorbox}
	
	\subsection{Anwendung: Berechnung von \(\kappa\)}
	
	Das modifizierte Gravitationspotential im T0-Modell lautet:
	\begin{equation}
		\Phi(r) = -\frac{G M}{r} + \kappa r
	\end{equation}
	
	In natürlichen Einheiten mit \(\betaT^{\text{nat}} = 1\):
	\begin{equation}
		\kappa_{\text{nat}} = \frac{y v}{r_g}
	\end{equation}
	
	In SI-Einheiten mit \(\betaT^{\text{SI}} \approx 0.008\):
	\begin{equation}
		\kappa_{\text{SI}} = \betaT^{\text{SI}} \frac{y v c^2}{r_g^2} \approx 4.8 \times 10^{-11} \text{ m/s}^2
	\end{equation}
	
	\section{Kosmologische Implikationen}
	
	\begin{itemize}
		\item \(\kappa_{\text{SI}}\): Erklärt Rotationskurven ohne Dunkle Materie
		\item \(\alpha_{\text{SI}}\): Beschreibt Expansion ohne Dunkle Energie
		\item \(\betaT^{\text{SI}}\): Wellenlängenabhängige Rotverschiebung, testbar mit JWST
	\end{itemize}
	
	\begin{figure}[h]
		\centering
		\begin{tikzpicture}
			\begin{axis}[
				xlabel={Radius [kpc]},
				ylabel={Rotationsgeschwindigkeit [km/s]},
				xlabel style={font=\large},
				ylabel style={font=\large},
				tick label style={font=\normalsize},
				xmin=0, xmax=30,
				ymin=0, ymax=300,
				legend pos=south east,
				legend style={font=\large},
				grid=both,
				minor tick num=4,
				major grid style={line width=0.8pt, gray!50},
				minor grid style={line width=0.4pt, gray!20}
				]
				\addplot[blue, ultra thick, domain=0.1:30, samples=100] {220*sqrt(10/x)};
				\addplot[red, dashed, ultra thick, domain=0.1:30, samples=100] {sqrt(220^2*10/x + 4.8*x^2)};
				\legend{Newtonsche Vorhersage, T0-Modell}
			\end{axis}
		\end{tikzpicture}
		\caption{Rotationskurven mit \(\kappa_{\text{SI}}\).}
	\end{figure}
	
	\section{Konsequenzen der Setzung \(\beta = 1\)}
	
	\subsection{Theoretische Eleganz}
	
	\begin{itemize}
		\item Einfachheit der Temperatur-Rotverschiebungs-Relation
		\item Kohärenz dimensionsloser Parameter
		\item Klarheit der Beziehungen zwischen fundamentalen Größen
	\end{itemize}
	
	\subsection{Umrechnung in SI-Einheiten}
	
	Die Umrechnungsvorschrift:
	\begin{equation}
		\betaT^{\text{SI}} = \betaT^{\text{nat}} \cdot \frac{\xi \cdot l_{P,\text{SI}}}{r_{0,\text{SI}}}
	\end{equation}
	
	Dies ist analog zu \(c = 1\) in der Relativitätstheorie, wo wir zwischen der theoretischen Formulierung mit \(c = 1\) und der experimentellen Messung mit \(c = 3 \times 10^8\) m/s wechseln können.
	
	\subsection{Abweichungen von aktuellen Messungen}
	
	Mit \(\betaT^{\text{nat}} = 1\) wird die Temperatur-Rotverschiebungs-Relation zu:
	\begin{equation}
		T(z) = T_0 (1 + z) (1 + \ln(1 + z))
	\end{equation}
	
	Bei \(z = 1100\) (CMB-Entkopplung) ergibt dies:
	\begin{equation}
		T(1100) \approx 8800 \cdot T_0
	\end{equation}
	
	Im Vergleich dazu:
	\begin{itemize}
		\item \(\betaT^{\text{SI}} = 0.008\): \(T(1100) \approx 1163 \cdot T_0\)
		\item Standardmodell: \(T(1100) \approx 1101 \cdot T_0\)
	\end{itemize}
	
	\subsection{Neubewertung von Messungen}
	
	Die signifikante Diskrepanz zwischen den Vorhersagen mit \(\betaT^{\text{nat}} = 1\) und den aktuellen "Messungen" könnte auf einen Standardmodell-Bias in der Interpretation kosmologischer Daten hindeuten. Es ist zu beachten, dass:
	
	\begin{itemize}
		\item Kosmologische Messungen werden typischerweise im Rahmen des \(\Lambda\)CDM-Modells kalibriert
		\item Die "gemessenen" Werte könnten implizite Annahmen enthalten
		\item Eine vollständige Neubewertung im Rahmen des T0-Modells mit \(\betaT^{\text{nat}} = 1\) könnte zu einer konsistenten Interpretation führen
	\end{itemize}
	
	Die quantitativen Auswirkungen dieser Neubewertung werden in \cite{pascher_alphabeta_2025} detailliert analysiert.
	
	\section{Integration in die Zeit-Masse-Dualitätstheorie}
	
	\subsection{Konsistenz mit den Grundprinzipien}
	
	Die Setzung \(\betaT^{\text{nat}} = 1\) steht im Einklang mit den Grundprinzipien der Zeit-Masse-Dualitätstheorie:
	\begin{itemize}
		\item Zeit ist absolut: Die fundamentale Zeitskala wird durch das intrinsische Zeitfeld \(\Tfield\) bestimmt
		\item Masse variiert: \(m = \frac{\hbar}{\Tfield c^2}\), wobei die Variation durch das Higgs-Feld vermittelt wird
		\item Emergente Gravitation: Gravitation entsteht aus den Gradienten von \(\Tfield\)
	\end{itemize}
	
	\subsection{Implikationen für andere Parameter}
	
	Die Setzung \(\betaT^{\text{nat}} = 1\) beeinflusst andere Parameter des T0-Modells, insbesondere:
	\begin{itemize}
		\item \(\kappa\): Direkte Abhängigkeit über die Gleichung \(\kappa = \frac{y v}{r_g}\)
		\item \(\alpha\): Verbindung über die charakteristischen Längenskalen \(r_0\) und \(L_T\)
	\end{itemize}
	
	\section{Experimentelle Tests und Perspektiven}
	
	\subsection{Direkte Tests der Setzung \(\beta = 1\)}
	
	\begin{itemize}
		\item \textbf{Präzisionsmessungen des CMB-Spektrums:} Eine detaillierte Analyse von Abweichungen vom perfekten Schwarzkörperspektrum könnte Hinweise auf die wahre Form der Temperatur-Rotverschiebungs-Relation liefern.
		
		\item \textbf{Suche nach Signaturen höherer Temperaturen in der frühen kosmischen Geschichte:} Die Untersuchung von Isotopenverteilungen aus der primordialen Nukleosynthese könnte Hinweise auf höhere Temperaturen liefern.
		
		\item \textbf{Direkte Temperaturmessungen bei mittleren Rotverschiebungen:} Die Abweichung zwischen den Modellen wächst mit \(z\) und könnte bei mittleren Rotverschiebungen bereits messbar sein.
	\end{itemize}
	
	\subsection{Indirekte Tests und kosmologische Parameter}
	
	\begin{itemize}
		\item \textbf{Hubble-Spannung:} Eine Neuinterpretation der CMB-Daten mit \(\betaT^{\text{nat}} = 1\) könnte das Hubble-Spannungsproblem lösen.
		
		\item \textbf{Baryonische Akustische Oszillationen (BAO):} Die veränderte Temperatur-Rotverschiebungs-Relation würde die Interpretation von BAO-Messungen beeinflussen.
		
		\item \textbf{Galaxienformation:} Höhere Temperaturen im frühen Universum würden die Struktur- und Galaxienbildung beeinflussen.
	\end{itemize}
	
	Für eine detaillierte quantitative Analyse dieser Tests wird auf \cite{pascher_alphabeta_2025} verwiesen, wo spezifische Vorhersagen und Vergleiche mit dem Standardmodell präsentiert werden.
	
	\section{Schlussfolgerungen}
	
	Die Setzung \(\betaT^{\text{nat}} = 1\) in natürlichen Einheiten des T0-Modells stellt eine konzeptionell elegante und physikalisch motivierte Vereinfachung dar, analog zur Setzung von \(c = 1\) in der Relativitätstheorie oder \(\hbar = 1\) in der Quantenmechanik. Diese Vereinfachung erfordert eine spezifische Interpretation der charakteristischen Längenskala \(r_0\) als \(r_0 \approx 1.33 \times 10^{-4} \cdot l_P\), was einem bestimmten Verhältnis zur Planck-Länge entspricht.
	
	Die daraus resultierende Diskrepanz zu aktuellen „Messungen" kann als Hinweis darauf verstanden werden, dass unsere Interpretation kosmologischer Daten möglicherweise zu stark vom paradigmatischen Rahmen des Standardmodells beeinflusst ist. Dies öffnet die Tür für neue Perspektiven und experimentelle Tests, die zwischen verschiedenen kosmologischen Modellen unterscheiden könnten.
	
	Für die praktische Anwendung und den Vergleich mit experimentellen Daten können alle Ergebnisse problemlos in SI-Einheiten zurückübersetzt werden. Die konzeptionelle Eleganz einer Theorie mit einfachen dimensionslosen Parametern (\(\betaT^{\text{nat}} = 1\)) gegenüber komplexen Werten (\(\betaT^{\text{SI}} \approx 0.008\)) spricht für eine tiefere Untersuchung dieser Möglichkeit, insbesondere im Kontext der Zeit-Masse-Dualitätstheorie, die bereits fundamentale Neuinterpretationen physikalischer Konzepte vorschlägt.
	
	\begin{thebibliography}{99}
		\bibitem{pascher_messdifferenzen_2025} Pascher, J. (2025). \href{https://github.com/jpascher/T0-Time-Mass-Duality/tree/main/2/pdf/Deutsch/Analyse der Messdifferenzen zwischen dem T0-Modell und dem Standardmodell.pdf}{Analyse der Messdifferenzen zwischen dem T0-Modell und dem \(\Lambda\)CDM-Standardmodell}. 2. April 2025.
		\bibitem{pascher_temp_2025} Pascher, J. (2025). Anpassung der Temperatureinheiten in natürlichen Einheiten und CMB-Messungen. 2. April 2025.
		\bibitem{pascher_galaxies_2025} Pascher, J. (2025). Massenvariation in Galaxien: Eine Analyse im T0-Modell mit emergenter Gravitation. 30. März 2025.
		\bibitem{pascher_params_2025} Pascher, J. (2025). Zeit-Masse-Dualitätstheorie (T0-Modell): Ableitung der Parameter \(\kappa\), \(\alpha\) und \(\beta\). 30. März 2025.
		\bibitem{pascher_alpha_2025} Pascher, J. (2025). Energie als fundamentale Einheit: Natürliche Einheiten mit \(\alpha = 1\) im T0-Modell. 26. März 2025.
		\bibitem{pascher_alphabeta_2025} Pascher, J. (2025). Vereinheitlichtes Einheitensystem im T0-Modell: Die Konsistenz von \(\alpha = 1\) und \(\beta = 1\). 5. April 2025.
	\end{thebibliography}
	
\end{document}	