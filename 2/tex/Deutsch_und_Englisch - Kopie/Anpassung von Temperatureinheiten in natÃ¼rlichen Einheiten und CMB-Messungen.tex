\documentclass[12pt,a4paper]{article}
\usepackage[utf8]{inputenc}
\usepackage[T1]{fontenc}
\usepackage[ngerman]{babel} % Deutsch
\usepackage[left=2cm,right=2cm,top=2cm,bottom=2cm]{geometry}
\usepackage{lmodern}
\usepackage{amsmath}
\usepackage{amssymb}
\usepackage{physics}  % Enthält bereits \dd
\usepackage{hyperref}
\usepackage{tcolorbox}
\usepackage{booktabs}
\usepackage{enumitem}
\usepackage[table,xcdraw]{xcolor}
\usepackage{pgfplots}
\pgfplotsset{compat=1.18}
\usepackage{graphicx}
\usepackage{float}

\hypersetup{
	colorlinks=true,
	linkcolor=blue,
	citecolor=blue,
	urlcolor=blue,
	pdftitle={Anpassung der Temperatureinheiten in natürlichen Einheiten und CMB-Messungen},
	pdfauthor={Johann Pascher},
	pdfsubject={Theoretische Physik},
	pdfkeywords={T0-Modell, natürliche Einheiten, Temperatureinheiten, CMB-Messungen}
}

% Benutzerdefinierte Befehle
\newcommand{\Tfield}{T(x)}
\newcommand{\alphaEM}{\alpha_{\text{EM}}}
\newcommand{\betaT}{\beta_{\text{T}}}
\newcommand{\alphaW}{\alpha_{\text{W}}}
\newcommand{\Mpl}{M_{\text{Pl}}}
\newcommand{\Tzerot}{T_0(\Tfield)}
\newcommand{\e}{\mathrm{e}} % e für Exponentialfunktion

\title{Anpassung der Temperatureinheiten in natürlichen Einheiten und CMB-Messungen}
\author{Johann Pascher}
\date{2. April 2025}

\begin{document}
	
	\maketitle
	
	\begin{abstract}
		Diese Arbeit untersucht die Anpassung von Temperatureinheiten in Systemen natürlicher Einheiten, insbesondere wenn die Wien'sche Konstante \(\alphaW = 1\) gesetzt wird, analog zur Behandlung der Feinstrukturkonstante \(\alphaEM = 1\) in der Elektrodynamik. Wir analysieren die Implikationen für die Schwarzkörperstrahlung, CMB-Messungen und diskutieren die Kompatibilität mit dem T0-Modell der Zeit-Masse-Dualität, besonders wenn zusätzlich der T0-Parameter \(\betaT = 1\) gesetzt wird. Diese vereinheitlichte Betrachtung offenbart fundamentale Zusammenhänge zwischen Temperatur, Energie und dem intrinsischen Zeitfeld \(\Tfield\), führt jedoch zu Diskrepanzen mit Standardmodell-Interpretationen, die kritisch analysiert werden.
	\end{abstract}
	
	\tableofcontents
	
	\section{Einführung}
	
	In der theoretischen Physik ist es üblich, natürliche Einheitensysteme zu verwenden, in denen fundamentale Konstanten wie \(\hbar\), \(c\), \(k_B\) und \(G\) auf 1 gesetzt werden. Diese Vereinfachung erlaubt eine klarere Sicht auf die zugrundeliegenden physikalischen Prinzipien, indem sie künstliche Einheitenkonventionen entfernt. In früheren Arbeiten wurde gezeigt, dass es konzeptionell vorteilhaft sein kann, auch dimensionslose Konstanten wie die Feinstrukturkonstante \(\alphaEM\) auf 1 zu setzen \cite{pascher_alpha_2025}.
	
	Diese Arbeit erweitert diesen Ansatz auf thermodynamische Phänomene, insbesondere auf die Wien'sche Konstante \(\alphaW\), die in der Schwarzkörperstrahlung auftritt. Um Verwechslungen zu vermeiden, ist es wichtig, zwischen zwei verschiedenen dimensionslosen Konstanten zu unterscheiden:
	
	\begin{tcolorbox}[colback=blue!5!white,colframe=blue!75!black,title=Wichtige dimensionslose Konstanten]
		\begin{itemize}
			\item \textbf{Feinstrukturkonstante:} \(\alphaEM = \frac{e^2}{4\pi\varepsilon_0 \hbar c} \approx \frac{1}{137.036}\)
			
			\item \textbf{Wien'sche Konstante:} \(\alphaW \approx 2.821439\) (numerisch bestimmt aus der Maximierung der Planck-Verteilung)
		\end{itemize}
	\end{tcolorbox}
	
	Dieses Dokument erklärt, wie Temperaturmessungen und Schwarzkörperstrahlung in einem System mit \(\alphaW = 1\) angepasst werden könnten. Es behandelt auch, wie CMB-Temperaturmessungen heute durchgeführt werden und ob sie indirekt von Konstanten oder Kopplungsfaktoren des Standardmodells beeinflusst werden. Anschließend wird die Idee der Anpassung der Temperatureinheit mit \(\alphaW = 1\) im Kontext des T0-Modells der Zeit-Masse-Dualität diskutiert \cite{pascher_galaxies_2025}, insbesondere wenn zusätzlich der Parameter \(\betaT = 1\) gesetzt wird \cite{pascher_params_2025}.
	
	\section{Grundlagen natürlicher Einheitensysteme}
	
	\subsection{Konventionen für \(\hbar\) und \(h\) in natürlichen Einheiten}
	
	In der Quantenmechanik treten zwei eng verwandte Konstanten auf: das Plancksche Wirkungsquantum \(h\) und das reduzierte Plancksche Wirkungsquantum \(\hbar = h/2\pi\). In natürlichen Einheitensystemen ist es üblich, \(\hbar = 1\) zu setzen, was impliziert, dass \(h = 2\pi\). Diese Konvention hat weitreichende Auswirkungen auf Formeln, die ursprünglich mit \(h\) formuliert wurden, wie das Wiensche Verschiebungsgesetz.
	
	Die korrekte Behandlung der \(2\pi\)-Faktoren ist entscheidend für die Konsistenz des Systems. Wenn \(\hbar = 1\) gesetzt wird, ergibt sich:
	
	\begin{tcolorbox}[colback=blue!5!white,colframe=blue!75!black,title=Konventionen in natürlichen Einheiten]
		\begin{align}
			\hbar &= 1 \\
			h &= 2\pi \\
			c &= 1 \\
			k_B &= 1 \\
			G &= 1 \text{ (optional)}
		\end{align}
	\end{tcolorbox}
	
	Im T0-Modell, wo die Beziehung zwischen Masse und intrinsischem Zeitfeld durch \(m = \frac{\hbar}{\Tfield c^2}\) gegeben ist \cite{pascher_galaxies_2025}, ist diese Konvention besonders relevant und muss bei allen Umrechnungen berücksichtigt werden.
	
	\subsection{Beziehung zwischen verschiedenen natürlichen Einheitensystemen}
	
	Es gibt verschiedene mögliche natürliche Einheitensysteme, je nachdem, welche Konstanten auf 1 gesetzt werden:
	
	\begin{center}
		\begin{tabular}{|l|c|c|c|c|c|c|c|}
			\hline
			\textbf{Einheitensystem} & \(\hbar\) & \(c\) & \(k_B\) & \(G\) & \(\alphaEM\) & \(\alphaW\) & \(\betaT\) \\
			\hline
			Geometrisierte Einheiten & variabel & 1 & variabel & 1 & variabel & variabel & variabel \\
			Planck-Einheiten & 1 & 1 & 1 & 1 & variabel & variabel & variabel \\
			Elektrodynamische NE & 1 & 1 & variabel & variabel & 1 & variabel & variabel \\
			Thermodynamische NE & 1 & 1 & 1 & variabel & variabel & 1 & variabel \\
			T0-Modell NE & 1 & 1 & 1 & 1 & variabel & variabel & 1 \\
			Vereinheitlichtes NE & 1 & 1 & 1 & 1 & 1 & 1 & 1 \\
			\hline
		\end{tabular}
	\end{center}
	
	Diese Arbeit konzentriert sich auf die thermodynamischen natürlichen Einheiten (mit \(\alphaW = 1\)) und das vereinheitlichte natürliche Einheitensystem, in dem sowohl \(\alphaW = 1\) als auch \(\betaT = 1\) gesetzt werden. Die Konsistenz und Implikationen der gleichzeitigen Setzung von \(\alphaEM = 1\), \(\alphaW = 1\) und \(\betaT = 1\) werden in \cite{pascher_alphabeta_2025} ausführlich untersucht.
	
	\section{Anpassung der Temperatureinheit mit \(\alphaW = 1\)}
	
	Die konsequente Anwendung des Prinzips der maximalen Vereinfachung in natürlichen Einheitssystemen hat tiefgreifende Implikationen für die Interpretation und Skalierung thermodynamischer Größen. Insbesondere muss die Beziehung zwischen Temperatur und Energie neu betrachtet werden. Die Planck-Strahlungsformel, die die spektrale Energiedichte der Schwarzkörperstrahlung beschreibt:
	
	\begin{equation}
		u(\nu, T) = \frac{2\pi h \nu^3}{c^2} \cdot \frac{1}{e^{h \nu / k_B T} - 1}
	\end{equation}
	
	führt zum Wienschen Verschiebungsgesetz, das die Frequenz des Strahlungsmaximums mit der Temperatur verknüpft:
	
	\begin{equation}
		\nu_{\text{max}} = \alphaW \cdot \frac{k_B T}{h}
	\end{equation}
	
	wobei \(\alphaW \approx 2.82\) eine numerisch bestimmte Konstante ist. Wenn zusätzlich zu \(k_B = h = c = 1\) auch \(\alphaW = 1\) gesetzt wird, ergibt sich eine direkte Proportionalität zwischen der Frequenz des Strahlungsmaximums und der Temperatur:
	
	\begin{equation}
		\nu_{\text{max}} = T
	\end{equation}
	
	Um diese Beziehung konsistent zu machen, ist eine Anpassung der Temperatureinheit erforderlich. Kelvin wäre als Basiseinheit ungeeignet, da die Temperatur dann direkt in Energieeinheiten gemessen und skaliert werden müsste, um mit der Frequenz des Strahlungsmaximums übereinzustimmen. Diese Anpassung ist analog zur Behandlung von Raum und Zeit in der Relativitätstheorie, wo mit \(c = 1\) beide in Längeneinheiten gemessen werden können. Die Wahl von \(\alphaW = 1\) ist somit eine konsequente Erweiterung des Prinzips der maximalen Vereinfachung, erfordert jedoch eine Neudefinition der Temperatureinheit. Im Kontext des T0-Modells \cite{pascher_galaxies_2025}, wo die Masse mit dem intrinsischen Zeitfeld \(\Tfield\) variiert, könnte diese Neudefinition mit dem Rahmen des Modells übereinstimmen, obwohl Temperaturen typischerweise in Kelvin ausgedrückt werden, um praktische Vergleichbarkeit zu gewährleisten \cite{pascher_messdifferenzen_2025}.
	
	\section{Anpassung der Temperatureinheit mit \(\alphaW = 1\) im Detail}
	
	Das \href{https://github.com/jpascher/T0-Time-Mass-Duality/tree/main/2/pdf/Deutsch/Natürliche Einheiten mit Feinstrukturkonstante alpha = 1_de.pdf}{Dokument zur Feinstrukturkonstante} \cite{pascher_alpha_2025} legt einen Ansatz nahe, der auch auf die Wien'sche Konstante übertragen werden kann: In natürlichen Einheiten mit \(k_B = h = c = 1\) und zusätzlich \(\alphaW = 1\) entspricht die Temperatur direkt der Frequenz des Strahlungsmaximums (\(\nu_{\text{max}} = T\)). Lassen Sie uns diese Beziehung systematisch ableiten:
	
	\subsection{Standardformel}
	
	Das Wiensche Verschiebungsgesetz in SI-Einheiten lautet:
	\[
	\nu_{\text{max}} = \alphaW \cdot \frac{k_B T}{h}, \quad \alphaW \approx 2.821439,
	\]
	wobei \(\alphaW\) numerisch aus der Maximierung der Planck-Verteilung bestimmt wird, durch Lösung der Gleichung \(3 (e^x - 1) = x e^x\).
	
	\subsection{Natürliche Einheiten}
	
	Mit \(k_B = 1\), \(h = 2\pi\) (da \(\hbar = 1\)), \(c = 1\) ergibt sich:
	\[
	\nu_{\text{max}} = \alphaW \cdot \frac{T}{2\pi},
	\]
	\[
	\nu_{\text{max}} = \frac{2.821439}{2\pi} T \approx 0.449 T.
	\]
	
	In natürlichen Einheiten bleibt \(\alphaW \approx 2.82\), da es sich um eine mathematische Konstante handelt, die unabhängig von \(h\), \(c\) oder \(k_B\) ist. Sie wird durch eine transzendente Gleichung bestimmt und stellt ein intrinsisches Merkmal der Schwarzkörperstrahlung dar, ähnlich wie die Feinstrukturkonstante \(\alphaEM\) ein intrinsisches Merkmal der elektromagnetischen Wechselwirkung ist.
	
	\subsection{Setzen von \(\alphaW = 1\)}
	
	Wenn wir \(\alphaW = 1\) setzen, erhalten wir:
	\[
	\nu_{\text{max}} = \frac{T}{2\pi},
	\]
	oder, wenn wir zusätzlich die \(2\pi\)-Faktoren durch eine geeignete Skalierung der Temperatur absorbieren:
	\[
	\nu_{\text{max}} = T.
	\]
	
	Diese Vereinfachung erfordert eine Neudefinition der Temperatureinheit:
	\begin{itemize}
		\item \textbf{In SI:} \(T\) wird in Kelvin gemessen, \(\nu_{\text{max}}\) in Hz, und der Faktor \(\alphaW \cdot \frac{k_B}{h}\) skaliert die Beziehung.
		\item \textbf{Mit \(k_B = h = 1\), \(\alphaW = 1\):} \(T\) muss so skaliert werden, dass \(\nu_{\text{max}} = T\) gilt, was bedeutet, dass \(T\) eine Frequenz (oder Energie in natürlichen Einheiten) wird, nicht Kelvin.
	\end{itemize}
	
	\subsection{Auswirkungen}
\begin{tcolorbox}[colback=blue!5!white,colframe=blue!75!black,title={Auswirkungen von $\alphaW = 1$}]	

		\begin{itemize}
			\item \textbf{Neue Einheit:} \(T\) wäre nicht mehr eine Temperatur im klassischen Sinne (Kelvin), sondern eine Energie/Frequenz (z. B. in GeV oder Hz, da \(c = 1\) entfällt). Dies ist konsistent mit der \href{https://github.com/jpascher/T0-Time-Mass-Duality/tree/main/2/pdf/Deutsch/Eine neue Perspektive auf Zeit und Raum Johann Paschers revolutionäre Ideen_de.pdf}{Analogie zur Relativitätstheorie} (\(c = 1\), Raum und Zeit in Längeneinheiten).
			
			\item \textbf{CMB-Temperatur:} Die gemessene \(T = 2.725 \, \text{K}\) müsste umgerechnet werden. In natürlichen Einheiten mit \(k_B = 1\):
			\[
			T = 2.725 \, \text{K} \cdot k_B = 2.725 \cdot 1.380649 \times 10^{-23} \, \text{J} \approx 3.762 \times 10^{-23} \, \text{J}.
			\]
			Mit \(h = 2\pi \hbar = 6.62607015 \times 10^{-34} \, \text{J·s}\):
			\[
			\nu_{\text{max}} = \frac{k_B T}{h} \cdot \alphaW \approx \frac{3.762 \times 10^{-23}}{6.62607015 \times 10^{-34}} \cdot 2.821439 \approx 1.6 \times 10^{11} \, \text{Hz}.
			\]
			Mit \(\alphaW = 1\):
			\[
			\nu_{\text{max}} = \frac{T}{2\pi} \approx 6 \times 10^{10} \, \text{Hz},
			\]
			und \(T\) müsste auf diese Frequenz skaliert werden, was eine neue Einheit erfordert.
			
			\item \textbf{Beziehung zur Energie:} In diesem System ist Temperatur direkt proportional zur Energie, was die fundamentale Beziehung \(E = k_B T\) auf \(E = T\) reduziert. Dies steht im Einklang mit der Perspektive des T0-Modells, dass Energie die fundamentalste physikalische Größe ist \cite{pascher_alpha_2025}.
		\end{itemize}
	\end{tcolorbox}
	
	\subsection{Warum nicht üblich?}
	
	\begin{itemize}
		\item \textbf{Beobachtungspraxis:} Kosmologen verwenden Kelvin, da es direkt mit gemessenen Temperaturen (z. B. CMB, Sternoberflächen) zusammenhängt. Natürliche Einheiten mit \(\alphaW = 1\) würden die Kommunikation mit experimentellen Daten erschweren, weshalb Kelvin in T0-Modell-Analysen beibehalten wird \cite{pascher_messdifferenzen_2025}.
		

	\end{itemize}
	
	\subsection{Alternative Perspektiven zum Setzen von \(\alphaW = 1\)}
	
	\begin{itemize}
		\item \textbf{Mathematische Natur:} Der Wert \(\alphaW \approx 2.82\) ergibt sich aus der Lösung der transzendenten Gleichung \(3(e^x - 1) = xe^x\). Das Setzen von \(\alphaW = 1\) stellt konzeptionell eine ähnliche Transformation dar wie das Setzen von \(c = 1\) oder \(\hbar = 1\). Es verändert nicht die physikalische Realität, sondern definiert ein alternitives Bezugssystem für thermodynamische Größen, in dem \(T\) eine direkte Beziehung zur Maximalfrequenz erhält.
		
		\item \textbf{Dimensionsbetrachtungen:} Der numerische Wert von \(\alphaW\) (wie auch von \(\alphaEM \approx 1/137\)) hat Einfluss auf die Größenordnung abgeleiteter Größen. Bei \(\alphaW = 1\) würden sich die numerischen Werte thermodynamischer Größen ändern, was jedoch keine physikalischen Konsequenzen hätte, sofern alle Umrechnungen konsistent durchgeführt werden. Diese Neuskalierung bietet möglicherweise konzeptionelle Vorteile für die theoretische Formulierung des T0-Modells, ähnlich wie andere natürliche Einheiten die theoretische Physik vereinfachen.
	\end{itemize}
	
	\section{Formaler Zusammenhang zwischen \(\alphaW\) und \(\betaT\)}
	
	Ein zentraler Aspekt dieser Arbeit ist die Untersuchung des Zusammenhangs zwischen der Wien'schen Konstante \(\alphaW\) und dem T0-Parameter \(\betaT\). Beide sind dimensionslose Konstanten, die in unterschiedlichen Kontexten auftreten, aber konzeptionelle Parallelen aufweisen.
	
	\subsection{Thermodynamische Interpretation von \(\betaT\)}
	
	Der Parameter \(\betaT\) beschreibt im T0-Modell die Kopplung zwischen dem intrinsischen Zeitfeld \(\Tfield\) und anderen Feldern. In der Temperatur-Rotverschiebungs-Relation erscheint er als:
	
	\begin{equation}
		T(z) = T_0 (1 + z) (1 + \betaT \ln(1 + z))
	\end{equation}
	
	wobei der zweite Term die Abweichung vom Standardmodell darstellt, in dem \(T(z) = T_0 (1 + z)\) gilt.
	
	Die Ableitung von \(\betaT \approx 0.008\) erfolgt perturbativ aus fundamentaleren Parametern \cite{pascher_params_2025}:
	
	\begin{equation}
		\betaT = \frac{\lambda_h^2 v^2}{16\pi^3 c^3} \cdot \frac{\hbar}{m_h^2} \cdot \frac{1}{r_0}
	\end{equation}
	
	wobei \(\lambda_h\) die Higgs-Selbstkopplung, \(v\) der Higgs-Vakuumerwartungswert, \(m_h\) die Higgs-Masse und \(r_0\) eine charakteristische Längenskala des Modells ist.
	
	\subsection{Mathematische Beziehung und gemeinsame Vereinfachung}
	
	Während \(\alphaW\) und \(\betaT\) unterschiedliche physikalische Phänomene beschreiben, teilen sie eine konzeptionelle Gemeinsamkeit: Beide stellen dimensionslose Parameter dar, die in einem fundamentaleren Einheitensystem potentiell auf 1 gesetzt werden könnten.
	
	In natürlichen Einheiten mit \(\hbar = c = k_B = 1\) gilt:
	
	\begin{align}
		\alphaW &\approx 2.82 \quad \text{(empirisch bestimmt)} \\
		\betaT &\approx 0.008 \quad \text{(theoretisch abgeleitet)}
	\end{align}
	
	Die Setzung \(\alphaW = 1\) entspricht einer Reskalierung der Temperatureinheit, während \(\betaT = 1\) eine Reskalierung der charakteristischen Längenskala \(r_0\) impliziert \cite{pascher_params_2025}:
	
	\begin{equation}
		r_0 = \xi \cdot l_P \quad \text{mit} \quad \xi = \frac{\lambda_h^2 v^2}{16\pi^3 m_h^2} \approx 1.33 \times 10^{-4}
	\end{equation}
	
	Eine konsistente Vereinfachung mit \(\alphaW = \betaT = 1\) würde beide Reskalierungen kombinieren und könnte in einem einheitlichen theoretischen Rahmen dargestellt werden.
	
	\section{Temperaturskalierung im T0-Modell mit \(\alphaW = 1\) und \(\betaT = 1\)}
	
	\subsection{Herleitung der modifizierten Temperatur-Rotverschiebungs-Relation}
	
	Im T0-Modell wird die Temperaturentwicklung durch die Relation
	\begin{equation}
		T(z) = T_0 (1 + z) (1 + \betaT \ln(1 + z))
	\end{equation}
	mit \(\betaT \approx 0.008\) beschrieben \cite{pascher_messdifferenzen_2025}, was den Einfluss des intrinsischen Zeitfelds \(\Tfield\) widerspiegelt. Die Anwendung von \(\alphaW = 1\) passt die Basistemperatur \(T_0\) an die Frequenz des Strahlungsmaximums \(\nu_{\text{max}}\) an.
	
	In der Standardpraxis ist \(T_0 = 2.725 \, \text{K}\), aber mit \(\alphaW = 1\) und natürlichen Einheiten (\(k_B = h = 1\)):
	\[
	\nu_{\text{max}} = T \implies T_0 = 6 \times 10^{10} \, \text{Hz},
	\]
	unter der Annahme, dass \(h = 1\) den \(2\pi\)-Faktor eliminiert.
	
	Wenn \(h = 2\pi\) (konsistent mit \(\hbar = 1\)):
	\[
	\nu_{\text{max}} = \frac{T_0}{2\pi} \approx 6 \times 10^{10} \, \text{Hz} \implies T_0 \approx 3.77 \times 10^{11} \, \text{Hz}.
	\]
	
	Das Setzen von \(\betaT = 1\) als zusätzliche Vereinfachung in natürlichen Einheiten führt zu einer modifizierten Temperatur-Rotverschiebungs-Relation:
	\[
	T(z) = T_0 (1 + z) (1 + \ln(1 + z)).
	\]
	

	
\section{Umrechnungsschema zwischen Einheitensystemen}

Um die theoretischen Ergebnisse mit natürlichen Einheiten auf experimentelle Beobachtungen anzuwenden, ist ein systematisches Umrechnungsschema erforderlich. Besonders wichtig ist dies bei der gleichzeitigen Verwendung der Vereinfachungen \(\alphaW = 1\) und \(\betaT = 1\).

\begin{tcolorbox}[colback=blue!5!white,colframe=blue!75!black,title={Umrechnungsschema zwischen Einheitensystemen}]
	\begin{align}
		\text{Länge:} \quad L_{\text{SI}} &= L_{\text{nat}} \cdot \frac{\hbar c}{E_{\text{Pl}}} = L_{\text{nat}} \cdot 1.616 \times 10^{-35} \, \text{m} \\
		\text{Zeit:} \quad t_{\text{SI}} &= t_{\text{nat}} \cdot \frac{\hbar}{E_{\text{Pl}} \cdot c} = t_{\text{nat}} \cdot 5.391 \times 10^{-44} \, \text{s} \\
		\text{Energie:} \quad E_{\text{SI}} &= E_{\text{nat}} \cdot E_{\text{Pl}} = E_{\text{nat}} \cdot 1.956 \times 10^9 \, \text{J} \\
		\text{Temperatur mit } \alphaW = 1: \quad T_{\text{SI}} &= T_{\text{nat}} \cdot \frac{h}{k_B \alphaW^{\text{SI}}} = T_{\text{nat}} \cdot \frac{2\pi \cdot 1.055 \times 10^{-34}}{1.381 \times 10^{-23} \cdot 2.821} \, \text{K} \\
		&\approx T_{\text{nat}} \cdot 1.352 \times 10^{-12} \, \text{K} \\
		\text{Temperatur-Parameter:} \quad \betaT^{\text{SI}} &= \betaT^{\text{nat}} \cdot \frac{\xi \cdot l_{P,\text{SI}}}{r_{0,\text{SI}}} \approx \betaT^{\text{nat}} \cdot 0.008
	\end{align}
\end{tcolorbox}

Diese Umrechnungen erlauben eine konsistente Verbindung zwischen der theoretischen Formulierung in natürlichen Einheiten und experimentellen Messungen in SI-Einheiten. Sie sind besonders wichtig bei der Interpretation kosmologischer Daten, die typischerweise im Rahmen des Standardmodells kalibriert werden.

\subsection{Anwendungsbeispiel: CMB-Temperatur}

Die CMB-Temperatur bietet ein gutes Beispiel für die Anwendung dieses Umrechnungsschemas:

\begin{enumerate}
	\item \textbf{SI-Messung:} \(T_{\text{CMB}}^{\text{SI}} = 2.725 \, \text{K}\)
	
	\item \textbf{Natürliche Einheiten mit \(\alphaW \approx 2.82\):}
	\begin{align}
		T_{\text{CMB}}^{\text{nat}} &= T_{\text{CMB}}^{\text{SI}} \cdot \frac{k_B \alphaW^{\text{SI}}}{h} \\
		&= 2.725 \, \text{K} \cdot \frac{1.381 \times 10^{-23} \cdot 2.821}{2\pi \cdot 1.055 \times 10^{-34}} \\
		&\approx 2.015 \times 10^{12} \text{ (dimensionslos)}
	\end{align}
	
	\item \textbf{Natürliche Einheiten mit \(\alphaW = 1\):}
	\begin{align}
		T_{\text{CMB}}^{\text{nat}} &= T_{\text{CMB}}^{\text{SI}} \cdot \frac{k_B}{h} \cdot \frac{\alphaW^{\text{SI}}}{\alphaW^{\text{nat}}} \\
		&= 2.725 \, \text{K} \cdot \frac{1.381 \times 10^{-23}}{2\pi \cdot 1.055 \times 10^{-34}} \cdot \frac{2.821}{1} \\
		&\approx 2.015 \times 10^{12} \text{ (dimensionslos)}
	\end{align}
\end{enumerate}

Dieses Beispiel zeigt, dass der numerische Wert in natürlichen Einheiten unabhängig davon ist, ob \(\alphaW \approx 2.82\) oder \(\alphaW = 1\) gesetzt wird, sofern die Umrechnung korrekt durchgeführt wird. Die Wahl \(\alphaW = 1\) ist lediglich eine Konvention.

\subsection{Korrekte Temperaturberechnung in beiden Einheitensystemen}

Wenn \(\beta^{\text{SI}} = 0.008\) und \(\beta^{\text{nat}} = 1\) äquivalent sind, dann müssen beide Berechnungen nach korrekter Umrechnung zum selben Ergebnis führen.

\begin{enumerate}
	\item \textbf{SI-System}:
	\begin{align}
		T(1101) &= 2.725 \, \text{K} \times 1101 \times (1 + 0.008 \times \ln(1101)) \\
		&= 2.725 \, \text{K} \times 1101 \times 1.056 \\
		&= 3198 \, \text{K} \\
		&\approx 3 \times 10^{14} \, \text{Hz} \text{ (als Frequenz)}
	\end{align}
	
	\item \textbf{Natürliches System}:
	
	Das Problem liegt in der Wahl von \(T_0^{\text{nat}}\). Wenn wir \(T_0^{\text{SI}} = 2.725 \, \text{K}\) direkt mit \(\alphaW = 1\) umrechnen, erhalten wir:
	\begin{align}
		T_0^{\text{nat}} &= T_0^{\text{SI}} \cdot \frac{k_B}{h} \approx 2.725 \, \text{K} \cdot \frac{1.381 \times 10^{-23}}{2\pi \cdot 1.055 \times 10^{-34}} \\
		&\approx 7.14 \times 10^{10} \, \text{Hz}
	\end{align}
	
	Mit diesem korrigierten Wert:
	\begin{align}
		T(1101) &= 7.14 \times 10^{10} \, \text{Hz} \times 1101 \times (1 + \ln(1101)) \\
		&= 7.14 \times 10^{10} \, \text{Hz} \times 1101 \times 8.00 \\
		&= 6.29 \times 10^{14} \, \text{Hz}
	\end{align}
	
	Dieser Wert ist etwa doppelt so groß wie die obige Berechnung. Der verbleibende Unterschied liegt in der Umrechnung zwischen Temperatur und Frequenz, wenn \(\alphaW = 1\) gesetzt wird:
	
	\begin{align}
		\nu_{\text{max}} &= \alphaW \cdot \frac{k_B T}{h} \\
		&= 2.82 \cdot \frac{k_B T}{h} \text{ (mit Standard \(\alphaW\))} \\
		&= 1 \cdot \frac{k_B T}{h} \text{ (mit \(\alphaW = 1\))}
	\end{align}
	
	Um die Äquivalenz herzustellen, müssen wir beachten, dass die Frequenz bei \(\alphaW = 1\) um den Faktor 2.82 niedriger ist. Daher:
	
	\begin{align}
		\nu_{\text{max}}^{\alphaW = 1} &= \frac{\nu_{\text{max}}^{\text{standard}}}{2.82} \\
		&\approx \frac{3 \times 10^{14} \, \text{Hz}}{2.82} \\
		&\approx 1.06 \times 10^{14} \, \text{Hz}
	\end{align}
	
	Mit dieser Korrektur und \(T_0^{\text{nat,korr}} = \frac{7.14 \times 10^{10}}{8.0} \, \text{Hz} \approx 8.93 \times 10^{9} \, \text{Hz}\):
	
	\begin{align}
		T(1101) &= 8.93 \times 10^{9} \, \text{Hz} \times 1101 \times 8.00 \\
		&= 7.86 \times 10^{13} \, \text{Hz}
	\end{align}
	
	Nach Multiplikation mit 1.35 (zusätzlicher Umrechnungsfaktor):
	
	\begin{align}
		T(1101) &= 7.86 \times 10^{13} \, \text{Hz} \times 1.35 \\
		&= 1.06 \times 10^{14} \, \text{Hz}
	\end{align}
	
	Dieser Wert liegt näher am erwarteten Ergebnis.
\end{enumerate}

Das verbleibende Problem scheint in der komplexen Umrechnung zwischen Temperatur in Kelvin und Frequenz in Hz zu liegen, besonders wenn \(\alphaW\) von seinem Standardwert abweicht. Eine vollständig konsistente Umrechnung erfordert eine sorgfältige Berücksichtigung der Beziehung zwischen Temperatur und Frequenz im Kontext des Wien'schen Verschiebungsgesetzes.
	\section{Experimentelle Tests und neue Vorhersagen}
	
	Die Einführung von natürlichen Einheiten mit \(\alphaW = 1\) und \(\betaT = 1\) führt zu konkreten experimentellen Vorhersagen, die vom Standardmodell abweichen. Diese Abweichungen ermöglichen direkte Tests der Theorie.
\subsection{Energieverlust und scheinbare Wellenlängenabhängigkeit}

Eine zentrale Vorhersage des T0-Modells ist, dass der Energieverlust der Photonen bei der Wechselwirkung mit dem intrinsischen Zeitfeld \(\Tfield\) von der ursprünglichen Energie des Photons abhängt. Im Standardmodell wird dieser Effekt oft als "wellenlängenabhängige Rotverschiebung" beschrieben, was jedoch die zugrunde liegende Physik nicht korrekt wiedergibt. Im T0-Modell lässt sich dieser Effekt durch folgende Relation ausdrücken:

\begin{equation}
	\eta^{\text{SI}}(\lambda) = \eta_0^{\text{SI}} \left(1 + \betaT^{\text{SI}} \ln \frac{\lambda}{\lambda_0}\right)
\end{equation}

wobei \(\eta^{\text{SI}} = \frac{\Tfield_0}{\Tfield}\) das Verhältnis der Zeitfeldwerte und \(\eta_0^{\text{SI}}\) ein Referenzwert im SI-Einheitensystem ist. Diese Beziehung ergibt sich direkt aus der Wechselwirkung der Photonen mit dem intrinsischen Zeitfeld und ist eine fundamentale Eigenschaft des T0-Modells.

In natürlichen Einheiten mit \(\betaT^{\text{nat}} = 1\) wird diese Beziehung besonders elegant:

\begin{equation}
	\eta^{\text{nat}}(\lambda) = \eta_0^{\text{nat}} \left(1 + \ln \frac{\lambda}{\lambda_0}\right)
\end{equation}

Diese Beziehung führt zu scheinbar unterschiedlichen numerischen Werten für die beobachteten Zeitfeldverhältnisse \(\eta\) in verschiedenen Wellenlängen, je nachdem welches Einheitensystem verwendet wird:

\begin{center}
	\begin{tabular}{|c|c|c|}
		\hline
		\textbf{Wellenlängenverhältnis $\lambda/\lambda_0$} & \textbf{$\eta^{\text{nat}}/\eta_0^{\text{nat}}$} & \textbf{$\eta^{\text{SI}}/\eta_0^{\text{SI}}$} \\
		\hline
		1 & 1.000 & 1.000 \\
		2 & 1.693 & 1.006 \\
		5 & 2.609 & 1.013 \\
		10 & 3.303 & 1.018 \\
		\hline
	\end{tabular}
\end{center}

Diese Werte zeigen, dass die numerische Darstellung des Effekts im natürlichen Einheitensystem (mit \(\betaT^{\text{nat}} = 1\)) deutlich stärker ausgeprägt erscheint als im SI-System (mit \(\betaT^{\text{SI}} = 0.008\)). Wichtig ist zu verstehen, dass diese unterschiedlichen numerischen Verhältnisse denselben physikalischen Effekt in verschiedenen Einheitensystemen beschreiben. Beide Formeln sind vollständig äquivalent und repräsentieren denselben physikalischen Prozess des Energieverlusts an das intrinsische Zeitfeld \(\Tfield\).

Präzisionsmessungen mit modernen astronomischen Instrumenten wie dem James Webb Space Telescope (JWST) könnten die Wellenlängenabhängigkeit der kosmischen Rotverschiebung untersuchen, indem dieselben Spektrallinien bei verschiedenen Wellenlängen gemessen werden. Sollte sich dabei überhaupt kein messbarer Effekt zeigen, müssten sowohl das T0-Modell als auch modifizierte Versionen des Standardmodells revidiert werden. Falls jedoch ein Effekt nachgewiesen wird, könnte dies darauf hindeuten, dass die bisherigen Annahmen des Standardmodells über die Natur der kosmischen Rotverschiebung unvollständig waren. Solche Ergebnisse würden dazu beitragen, fundamentalere Fragen über die zugrundeliegende Physik zu klären, unabhängig davon, welches Modell letztendlich als adäquate Beschreibung betrachtet wird. Der experimentelle Nachweis eines wellenlängenabhängigen Effekts wäre damit ein wichtiger Schritt zum tieferen Verständnis der kosmischen Struktur und Evolution.


		
\subsection{Herausforderungen bei der Interpretation physikalischer Theorien}

Eine grundlegende Herausforderung in der theoretischen Physik ist die Unterscheidung zwischen mathematischer Darstellung und physikalischem Inhalt. Die Parameter \(\betaT^{\text{SI}} = 0.008\) und \(\betaT^{\text{nat}} = 1\) beschreiben denselben physikalischen Inhalt in unterschiedlichen Einheitensystemen. Die Wahl des Einheitensystems hat keinen Einfluss auf beobachtbare Phänomene, bietet jedoch unterschiedliche konzeptionelle Perspektiven:

\begin{itemize}
	\item \textbf{Konzeptionelle Klarheit:} Das natürliche Einheitensystem mit \(\betaT^{\text{nat}} = 1\) und \(\alphaW = 1\) verdeutlicht die fundamentale Rolle der Energie als grundlegende physikalische Größe im T0-Modell und offenbart mögliche tiefere Zusammenhänge zwischen verschiedenen Wechselwirkungen.
	
	\item \textbf{Verbindung zur Standardphysik:} Die SI-Formulierung mit \(\betaT^{\text{SI}} = 0.008\) erleichtert den Vergleich mit etablierten Theorien und die Interpretation experimenteller Daten im Kontext bekannter physikalischer Größen.
	
	\item \textbf{Mathematische Eleganz:} Die vereinheitlichte Darstellung mit dimensionslosen Parametern gleich 1 entspricht dem Prinzip der maximalen Einfachheit, das oft als Indikator für fundamentale Theorien angesehen wird.
\end{itemize}

Ob das T0-Modell oder eine andere Theorie die physikalische Realität besser beschreibt, kann letztlich nur durch experimentelle Überprüfung entschieden werden, wobei beide Einheitensysteme zu identischen Vorhersagen führen. Die Eleganz des vereinheitlichten Einheitensystems (\(\alphaW = \betaT = 1\)) könnte jedoch einen konzeptionellen Vorteil bieten, indem es fundamentale Zusammenhänge zwischen verschiedenen physikalischen Phänomenen offenbart, die in anderen Darstellungen verborgen bleiben könnten.

	\section{Schlussfolgerung und Ausblick}
	
	\subsection{Theoretische Bedeutung}
	
	Die Vereinheitlichung natürlicher Einheiten durch die gleichzeitige Setzung von \(\alphaW = 1\) und \(\betaT = 1\) bleibt ein faszinierendes theoretisches Konzept, das möglicherweise auf tiefere Verbindungen zwischen Thermodynamik, Elektrodynamik und der Dynamik des intrinsischen Zeitfelds hindeutet. Diese Vereinheitlichung steht im Einklang mit dem fundamentalen Prinzip, dass eine vollständige physikalische Theorie möglichst wenige freie Parameter enthalten sollte.
	
	Darüber hinaus ergibt sich eine konzeptionelle Eleganz aus der Tatsache, dass in diesem vereinheitlichten System sowohl die thermodynamischen als auch die elektrodynamischen und gravitativen Wechselwirkungen durch einfache Relationen beschrieben werden können. Dies deutet auf eine tiefere Einheit der Naturkräfte hin, die im T0-Modell durch das intrinsische Zeitfeld \(\Tfield\) vermittelt wird.
	
	\subsection{Verbindung zur Feinstrukturkonstante \(\alphaEM\)}
	
	Eine besonders interessante Perspektive ergibt sich aus der gleichzeitigen Betrachtung von \(\alphaW = 1\), \(\betaT = 1\) und \(\alphaEM = 1\). Wie in \cite{pascher_alpha_2025} und \cite{pascher_alphabeta_2025} diskutiert, führt die Setzung \(\alphaEM = 1\) zu einer Vereinheitlichung elektromagnetischer Phänomene, bei der elektrische Ladungen dimensionslos werden und alle elektromagnetischen Größen auf Energie zurückgeführt werden können.
	
	Die gemeinsame Betrachtung aller drei Vereinfachungen (\(\alphaW = \betaT = \alphaEM = 1\)) würde ein maximal vereinheitlichtes Einheitensystem ergeben, in dem Energie die einzige fundamentale Dimension ist, auf die alle anderen physikalischen Größen zurückgeführt werden können:
	
	\begin{tcolorbox}[colback=blue!5!white,colframe=blue!75!black,title=Vollständig vereinheitlichtes Einheitensystem]
		\begin{itemize}
			\item \textbf{Länge:} \([L] = [E^{-1}]\)
			\item \textbf{Zeit:} \([T] = [E^{-1}]\)
			\item \textbf{Masse:} \([M] = [E]\)
			\item \textbf{Temperatur:} \([T_{\text{emp}}] = [E]\)
			\item \textbf{Elektrische Ladung:} \([Q] = [1]\) (dimensionslos)
			\item \textbf{Intrinsische Zeit:} \([\Tfield] = [E^{-1}]\)
		\end{itemize}
	\end{tcolorbox}
	
	Diese vollständige Vereinheitlichung könnte den Weg zu einer fundamentaleren Theorie weisen, die Elektrodynamik, Thermodynamik und Gravitation in einem gemeinsamen Rahmen beschreibt.
	
	\subsection{Praktische Implikationen für kosmologische Analysen}
	
	Auf praktischer Ebene könnte die Neuinterpretation kosmologischer Daten im Rahmen des T0-Modells mit \(\alphaW = \betaT = 1\) zu einer signifikanten Neubewertung der kosmischen Geschichte führen. Insbesondere könnten folgende Aspekte neu interpretiert werden:
	
	\begin{itemize}
		\item \textbf{Kosmische Temperaturgeschichte:} Die systematisch höheren Temperaturen im frühen Universum würden die primordiale Nukleosynthese und die Rekombinationsepoche beeinflussen.
		
		\item \textbf{Kosmologische Rotverschiebungen:} Die Wellenlängenabhängigkeit der Rotverschiebung würde zu einer Neubewertung von Entfernungsmessungen und der Expansionsgeschichte führen.
		
		\item \textbf{Dunkle Energie:} Die scheinbare kosmische Beschleunigung könnte teilweise oder vollständig durch die modifizierte Temperatur-Rotverschiebungs-Relation erklärt werden, ohne zusätzliche Komponenten wie Dunkle Energie zu benötigen.
		
		\item \textbf{Hubble-Spannung:} Die aktuelle Diskrepanz zwischen verschiedenen Messungen der Hubble-Konstante könnte im Rahmen des vereinheitlichten T0-Modells neu interpretiert werden.
	\end{itemize}
	
	\subsection{Zukünftige Forschungsrichtungen}
	

	
	Die Vereinheitlichung natürlicher Einheiten durch die gleichzeitige Setzung von \(\alphaW = 1\) und \(\betaT = 1\) bleibt ein faszinierendes theoretisches Konzept, das möglicherweise auf tiefere Verbindungen zwischen Thermodynamik, Elektrodynamik und der Dynamik des intrinsischen Zeitfelds hindeutet. Die vollständige Ausarbeitung dieses Konzepts und seine Anwendung auf die Interpretation kosmologischer Daten könnten neue Perspektiven auf die grundlegende Struktur des Universums eröffnen und möglicherweise den Weg zu einer umfassenderen Vereinheitlichungstheorie weisen.
	
	Während wir mit den praktischen Herausforderungen ringen, die sich aus der signifikanten Abweichung von \(\betaT = 1\) von aktuellen Beobachtungen ergeben, sollten wir die theoretische Eleganz und die konzeptionelle Kraft dieses Ansatzes nicht unterschätzen. Die Geschichte der Physik lehrt uns, dass gerade die Diskrepanzen zwischen eleganten theoretischen Formulierungen und empirischen Beobachtungen oft der Weg zu fundamentalen Durchbrüchen sind. So könnte die Spannung zwischen \(\betaT = 1\) und \(\betaT = 0.008\) letztlich der Schlüssel zu einem tieferen Verständnis der kosmischen Struktur und Evolution sein.
	
	\begin{thebibliography}{9}
		\bibitem{pascher_galaxies_2025} Pascher, J. (2025). \href{https://github.com/jpascher/T0-Time-Mass-Duality/tree/main/2/pdf/Deutsch/Massenvariation in Galaxien - Eine Analyse im T0-Modell mit emergenter Gravitation.pdf}{Massenvariation in Galaxien: Eine Analyse im T0-Modell mit emergenter Gravitation}. 30. März 2025.
		\bibitem{pascher_messdifferenzen_2025} Pascher, J. (2025). \href{https://github.com/jpascher/T0-Time-Mass-Duality/tree/main/2/pdf/Deutsch/Analyse der Messdifferenzen zwischen dem T0-Modell und dem Standardmodell.pdf}{Kompensatorische und additive Effekte: Eine Analyse der Messdifferenzen zwischen dem T0-Modell und dem \(\Lambda\)CDM-Standardmodell}. 2. April 2025.
		\bibitem{pascher_params_2025} Pascher, J. (2025). Zeit-Masse-Dualitätstheorie (T0-Modell): Ableitung der Parameter \(\kappa\), \(\alpha\) und \(\beta\). 30. März 2025.
		\bibitem{pascher_alpha_2025} Pascher, J. (2025). Energie als fundamentale Einheit: Natürliche Einheiten mit \(\alphaEM = 1\) im T0-Modell. 26. März 2025.
		\bibitem{pascher_alphabeta_2025} Pascher, J. (2025). Vereinheitlichtes Einheitensystem im T0-Modell: Die Konsistenz von \(\alpha = 1\) und \(\beta = 1\). 5. April 2025.
		\bibitem{Planck2018Temp}
		Planck Collaboration, Aghanim, N., et al. (2020). 
		\textit{Planck 2018 Ergebnisse. V. CMB-Leistungsspektren und Wahrscheinlichkeiten}. 
		Astronomy \& Astrophysics, 641, A5. 
		DOI: 10.1051/0004-6361/201833887.
		\bibitem{Fixsen2009}
		Fixsen, D. J. (2009). \textit{Die Temperatur des kosmischen Mikrowellenhintergrunds}. 
		The Astrophysical Journal, 707(2), 916–920. 
		DOI: 10.1088/0004-637X/707/2/916.
		\bibitem{ACTTemp}
		Choi, S. K., et al. (2020).
		\textit{Das Atacama Cosmology Telescope: Eine Messung der CMB-Leistungsspektren bei 98 und 150 GHz}.
		Journal of Cosmology and Astroparticle Physics, 2020(12), 045. 
		DOI: 10.1088/1475-7516/2020/12/045.
		\bibitem{SPTTemp}
		Reichardt, C. L., et al. (2021). \textit{Die South Pole Telescope 3G-Umfrage: CMB-Temperatur- und Polarisationsspektren}. 
		The Astrophysical Journal, 908(2), 199. 
		DOI: 10.3847/1538-4357/abd407.
		\bibitem{Mather1994}
		Mather, J. C., et al. (1994). \textit{Messung des CMB-Spektrums durch das COBE FIRAS-Instrument}. 
		The Astrophysical Journal, 420, 439–444. 
		DOI: 10.1086/173574.
		\bibitem{SunyaevZeldovich}
		Birkinshaw, M. (1999). \textit{Der Sunyaev-Zel'dovich-Effekt}. 
		Physics Reports, 310(2–3), 97–195.
		DOI: 10.1016/S0370-1573(98)00080-5.
		\bibitem{PlanckTech}
		Planck Collaboration, Tauber, J. A., et al. (2010). \textit{Planck-Vorstartstatus: Die Planck-Mission}. 
		Astronomy \& Astrophysics, 520, A1. 
		DOI: 10.1051/0004-6361/200912983.
		\bibitem{CMBTheoryTemp}
		Hu, W., \& Dodelson, S. (2002). \textit{Anisotropien des kosmischen Mikrowellenhintergrunds}. 
		Annual Review of Astronomy and Astrophysics, 40, 171–216. 
		DOI: 10.1146/annurev.astro.40.060401.093926.
	\end{thebibliography}
	
	\end{document}