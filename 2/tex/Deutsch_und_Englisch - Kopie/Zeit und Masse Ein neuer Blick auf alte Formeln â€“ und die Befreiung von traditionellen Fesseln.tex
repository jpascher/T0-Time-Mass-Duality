\documentclass{article}
\usepackage[utf8]{inputenc}
\usepackage{amsmath}
\usepackage{amssymb}
\usepackage[margin=2cm]{geometry}
\usepackage{tikz}
\usepackage[colorlinks=true, linkcolor=blue, citecolor=blue, urlcolor=blue]{hyperref}
\usepackage{tocloft}
\usepackage{xcolor}
\usepackage[ngerman]{babel}

\renewcommand{\cftsecfont}{\color{blue}}
\renewcommand{\cftsubsecfont}{\color{blue}}
\renewcommand{\cftsecpagefont}{\color{blue}}
\renewcommand{\cftsubsecpagefont}{\color{blue}}
\setlength{\cftsecindent}{1cm}
\setlength{\cftsubsecindent}{2cm}

\newcommand{\Tfield}{T(x)}

\title{Zeit und Masse: Ein neuer Blick auf alte Formeln – und Befreiung von traditionellen Fesseln}
\author{Johann Pascher}
\date{25. März 2025}

\begin{document}
	
	\maketitle
	
	\begin{abstract}
		Diese Arbeit stellt eine neue Perspektive auf Zeit und Masse vor, die Zeit-Masse-Dualität, die traditionelle Ansichten in der Quantenmechanik und Relativitätstheorie herausfordert. Durch erweiterte natürliche Einheiten werden physikalische Konstanten als dimensionslose Energieverhältnisse neu interpretiert. Ohne neue Gleichungen zeigt der Ansatz die Unvollständigkeit bestehender Theorien und schlägt eine einheitlichere, intuitivere Beschreibung der Realität vor, mit Implikationen für Quantengravitation, Verschränkung und kosmologische Phänomene.
	\end{abstract}
	
	\tableofcontents
	\newpage
	
	\section{Einführung: Traditionelle Sichten und die verdeckte Perspektive}
	Die Physik hat mit abstrakten Konzepten wie Quantenfeldern und Raumzeitkrümmung bemerkenswerte Erfolge erzielt. Aber sind wir vielleicht zu weit von einer intuitiven, realen Beschreibung der Welt abgekommen? Traditionelle Perspektiven, insbesondere unsere Wahl der Einheiten, könnten ein tieferes, einheitlicheres Verständnis der Natur verdeckt haben. Dieser Ansatz zielt darauf ab, zu den Grundlagen zurückzukehren – und die Physik von unnötigen Fesseln zu befreien.
	
	\section{Natürliche Konstanten und Einheiten: Mehr als willkürliche Zahlen?}
	Einheiten (Meter, Sekunden, Kilogramm) sind historisch bedingt, aber nicht fundamental. Natürliche Konstanten (\( c \), \( \hbar \), \( G \), \( \alpha \)) werden in natürlichen Einheiten oft auf 1 gesetzt. Das T0-Modell sieht sie als aus Energie abgeleitet an.
	
	\section{Die Zeit-Masse-Dualität: Eine alternative Perspektive}
	\begin{itemize}
		\item Standardansicht: \( t' = \gamma t \), \( m_0 = \text{konst.} \)
		\item T0-Modell: \( T_0 = \text{konst.} \), \( m = \gamma m_0 \)
	\end{itemize}
	Intrinsische Zeit:
	\begin{equation}
		\Tfield = \frac{\hbar}{\max(m c^2, \omega)}
	\end{equation}
	Modifizierte Schrödinger-Gleichung:
	\begin{equation}
		i\hbar \Tfield \frac{\partial}{\partial t} \Psi + i\hbar \Psi \frac{\partial \Tfield}{\partial t} = \hat{H} \Psi
	\end{equation}
	
	\section{Alle Konstanten werden natürlich: Energie als einigendes Prinzip}
	Im T0-Modell werden alle physikalischen Konstanten als dimensionslose Verhältnisse einer einzigen fundamentalen Größe – der Energie – ausgedrückt. Traditionelle Konstanten verlieren ihren Status als unabhängige, gegebene Größen und werden zu abgeleiteten Eigenschaften, die aus Energie hervorgehen.
	
	\section{Keine neuen Formeln, aber eine befreite Sicht}
	Dieser Ansatz führt keine völlig neuen Gleichungen ein. Stattdessen untersuchen wir die gleichen fundamentalen Formeln der Quantenmechanik und Relativitätstheorie – jedoch in einem neuen Bezugsrahmen, in dem alle Konstanten dimensionslos oder "natürlich" sind. Dieser scheinbar kleine Wandel hat weitreichende Folgen und zeigt die Grenzen und Lücken bestehender Theorien auf:
	\begin{enumerate}
		\item \textbf{Unvollständigkeit der Quantenmechanik (aus bestehenden Formeln):} Die bekannten Formeln der Quantenmechanik, in dieses neue System übertragen, beschreiben nicht mehr alle Phänomene korrekt. Sie sind unvollständig und erfassen nicht vollständig das dynamische Zusammenspiel von Masse, Zeit und Energie.
		\item \textbf{Erweiterung innerhalb des bestehenden Rahmens:} Die Quantenmechanik muss erweitert werden. Diese Erweiterung ergibt sich jedoch nicht aus willkürlichen neuen Annahmen, sondern aus einer konsistenteren Anwendung bestehender Prinzipien, insbesondere der Energieerhaltung und der untrennbaren Verbindung zwischen Masse und Zeit.
		\item \textbf{Duale Perspektiven als Schlüssel zur Realität:} Die Welle-Teilchen-Dualität und die Zeit-Masse-Dualität sind keine bloßen "Interpretationen". Sie sind Hinweise darauf, dass wir Aspekte der Realität übersehen oder missverstehen, wenn wir an traditionellen, einschränkenden Sichten festhalten. Sie leiten uns zu einer realeren, intuitiveren und einheitlicheren Beschreibung der physikalischen Welt.
	\end{enumerate}
	
	\section{Lagrange-Formulierung}
	Die vollständige Lagrangedichte des T0-Modells lautet:
	\begin{equation}
		\mathcal{L}_{\text{Total}} = \mathcal{L}_{\text{Boson}} + \mathcal{L}_{\text{Fermion}} + \mathcal{L}_{\text{Higgs-T}} + \mathcal{L}_{\text{intrinsic}}, \quad \mathcal{L}_{\text{intrinsic}} = \frac{1}{2} \partial_\mu \Tfield \partial^\mu \Tfield - V(\Tfield)
	\end{equation}
	Dieser Ansatz integriert die Dynamik des intrinsischen Zeitfelds und bietet eine einheitliche Beschreibung der fundamentalen Wechselwirkungen.
	
	\section{Konkrete Implikationen: Hin zu einer umfassenderen Theorie}
	Dieser "befreite" Blick auf die Physik führt zu konkreten Implikationen:
	\begin{itemize}
		\item \textbf{Quantengravitation:} Eine Vereinheitlichung basierend auf einer erweiterten und konsistenteren Quantenmechanik wird greifbarer. Gravitation entsteht als emergente Eigenschaft aus Gradienten des intrinsischen Zeitfelds:
		\begin{equation}
			\nabla \Tfield = -\frac{\hbar}{m^2 c^2} \nabla m
		\end{equation}
		mit dem modifizierten Gravitationspotential:
		\begin{equation}
			\Phi(r) = -\frac{GM}{r} + \kappa r, \quad \kappa \approx 4.8 \times 10^{-11} \, \text{m/s}^2
		\end{equation}
		\item \textbf{Quantenverschränkung:} Die Interpretation über intrinsische Zeit stellt die aktuelle Quantenmechanik infrage und eröffnet neue Perspektiven.
		\item \textbf{Dunkle Energie/Materie:} Neue, konkrete Beziehungen zwischen Masse, Energie und Universumsevolution entstehen, die bestehende Modelle übertreffen. Die Rotverschiebung wird als Energieverlust beschrieben:
		\begin{equation}
			1 + z = e^{\alpha d}, \quad \alpha \approx 2.3 \times 10^{-28} \, \text{m}^{-1}
		\end{equation}
		mit wellenlängenabhängiger Komponente:
		\begin{equation}
			z(\lambda) = z_0 (1 + \beta \ln(\lambda/\lambda_0)), \quad \beta \approx 0.008
		\end{equation}
		\item \textbf{Fundamentale Konstanten:} Ein tieferes Verständnis entsteht, da alle Konstanten auf eine fundamentale Größe (Energie) reduziert werden.
	\end{itemize}
	
	\begin{figure}[h]
		\centering
		\begin{tikzpicture}
			\draw[->] (0,0) -- (6,0) node[right] {Masse \(m\)};
			\draw[->] (0,0) -- (0,4) node[above] {Intrinsische Zeit \(\Tfield\)};
			\draw[scale=0.5, domain=0.1:10, smooth, variable=\x, blue, thick] plot ({\x}, {1/\x});
			\node[blue] at (4.5,2) {\(\Tfield \propto \frac{1}{m}\)};
		\end{tikzpicture}
		\caption{Beziehung zwischen Masse und intrinsischer Zeit: Leichtere Teilchen haben eine langsamere innere Uhr.}
	\end{figure}
	
	\section{Experimentelle Überprüfung und Schlussfolgerung}
	Dieser Ansatz ist nicht nur theoretisch, sondern experimentell überprüfbar. Er macht unterschiedliche Vorhersagen als die aktuelle, unvollständige Quantenmechanik (z. B. mit Präzisionsuhren und verschränkten Teilchen unterschiedlicher Massen). Die Zeit-Masse-Dualität, die "Naturalisierung" aller Konstanten und die resultierende Erweiterung der Quantenmechanik stellen einen radikalen, aber vielversprechenden Weg dar. Sie zeigen, dass wir die Physik grundlegend überdenken müssen – nicht durch das Verwerfen bewährter Formeln, sondern durch deren Befreiung von traditionellen Fesseln und Rückkehr zu einer realeren, intuitiveren und vor allem einheitlicheren Perspektive. Es ist der Beginn einer umfassenderen Theorie, die die größten Rätsel des Universums lösen könnte.
	
	\section{Unsicherheit bei \(\beta\)}
	Der Parameter \( \beta \approx 0.008 \) führt zu Unsicherheit; Werte wie \( \beta = 1 \) würden den Beobachtungen widersprechen. Weitere experimentelle Tests sind erforderlich, um \( \beta \) einzugrenzen.
	
	\begin{thebibliography}{9}
		\bibitem{wesentlicheFormalismen} Pascher, J. (2025). \textit{Wesentliche mathematische Formalismen der Zeit-Masse-Dualitätstheorie mit Lagrange-Dichten}. 29. März 2025.
		\bibitem{einstein} Einstein, A. (1905). \textit{Zur Elektrodynamik bewegter Körper}. Annalen der Physik, 322(10), 891-921.
	\end{thebibliography}
	
\end{document}