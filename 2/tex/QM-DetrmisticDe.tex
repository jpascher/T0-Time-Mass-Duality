\documentclass[12pt,a4paper]{article}
\usepackage[utf8]{inputenc}
\usepackage[T1]{fontenc}
\usepackage[english]{babel}
\usepackage[left=2cm,right=2cm,top=2cm,bottom=2cm]{geometry}
\usepackage{lmodern}
\usepackage{amsmath}
\usepackage{amssymb}
\usepackage{physics}
\usepackage{hyperref}
\usepackage{tcolorbox}
\usepackage{booktabs}
\usepackage{enumitem}
\usepackage[table,xcdraw]{xcolor}
\usepackage{graphicx}
\usepackage{float}
\usepackage{mathtools}
\usepackage{amsthm}
\usepackage{siunitx}
\usepackage{fancyhdr}
\usepackage{microtype}

% Kopf- und Fußzeilen
\pagestyle{fancy}
\fancyhf{}
\fancyhead[L]{Johann Pascher}
\fancyhead[R]{Deterministische Quantenmechanik via T0-Energiefelder}
\fancyfoot[C]{\thepage}
\renewcommand{\headrulewidth}{0.4pt}
\renewcommand{\footrulewidth}{0.4pt}

% Benutzerdefinierte Befehle
\newcommand{\Tfield}{T}
\newcommand{\Efield}{E}
\newcommand{\xipar}{\xi}
\newcommand{\betaT}{\beta_{\text{T}}}

\hypersetup{
	colorlinks=true,
	linkcolor=blue,
	citecolor=blue,
	urlcolor=blue,
	pdftitle={Deterministische Quantenmechanik via T0-Energiefeld Formulierung},
	pdfauthor={Johann Pascher},
	pdfsubject={T0 Modell, Deterministische QM, Energiefeldphysik}
}

\newtheorem{theorem}{Theorem}[section]
\newtheorem{proposition}[theorem]{Proposition}
\newtheorem{definition}[theorem]{Definition}

\begin{document}
	\title{Deterministische Quantenmechanik via T0-Energiefeld Formulierung: \\
		Von wahrscheinlichkeitsbasierter zu verhältnisbasierter Mikrophysik \\
		\large Aufbauend auf der T0-Revolution: Vereinfachte Dirac-Gleichung, universelle Lagrangefunktion und Verhältnisphysik}
	\author{Johann Pascher\\
		Abteilung für Kommunikationstechnik, \\Höhere Technische Bundeslehranstalt (HTL), Leonding, Österreich\\
		\texttt{johann.pascher@gmail.com}}
	\date{\today}
	
	\maketitle
	
	\begin{abstract}
		Dieses Dokument präsentiert eine revolutionäre deterministische Alternative zur\\ wahrscheinlichkeitsbasierten Quantenmechanik durch die T0-Energiefeldformulierung. Aufbauend auf der vereinfachten Dirac-Gleichung, universellen Lagrangefunktion und verhältnisbasierten Physik, die im T0-Rahmen entwickelt wurden, zeigen wir, wie quantenmechanische Phänomene aus deterministischen Energiefelddynamiken $\Efield(x,t)$ entstehen, die durch die universelle Gleichung $\partial^2 \Efield = 0$ beschrieben werden. Unter Verwendung der SI-Referenzskala $\xipar = 1.33 \times 10^{-4}$ liefern wir quantitative Vorhersagen, die alle experimentell verifizierten Ergebnisse bewahren, während fundamentale Interpretationsprobleme eliminiert werden. Die Formulierung geht über die Standard-Quantenmechanik hinaus mit präzisen Einzelmessungsvorhersagen und deterministischen Quantencomputing-Algorithmen.
	\end{abstract}
	
	\tableofcontents
	\newpage
	
	\section{Einleitung: Die T0-Revolution angewandt auf Quantenmechanik}
	
	\subsection{Aufbauend auf T0-Grundlagen}
	
	Diese Arbeit repräsentiert die vierte Stufe der T0-theoretischen Revolution:
	
	\textbf{Stufe 1 - Vereinfachte Dirac}: Komplexe 4×4 Matrizen → Einfache Felddynamik $\partial^2 \delta m = 0$
	
	\textbf{Stufe 2 - Universelle Lagrangefunktion}: 20+ Felder → Eine Gleichung $\mathcal{L} = \varepsilon \cdot (\partial \delta m)^2$
	
	\textbf{Stufe 3 - Verhältnisphysik}: Multiple Parameter → Energieskalenverhältnisse + SI-Referenz
	
	\textbf{Stufe 4 - Deterministische QM}: Wahrscheinlichkeitsamplituden → Deterministische Energiefelder
	
	\subsection{Das Quantenmechanik-Problem}
	
	Standard-Quantenmechanik leidet unter fundamentalen konzeptionellen Problemen:
	
	\begin{tcolorbox}[colback=red!5!white,colframe=red!75!black,title=Standard QM Probleme]
		\textbf{Wahrscheinlichkeitsfundamentprobleme}:
		\begin{itemize}
			\item Wellenfunktion: $\psi = \alpha|{\uparrow}\rangle + \beta|{\downarrow}\rangle$ (rätselhafte Superposition)
			\item Wahrscheinlichkeiten: $P(\uparrow) = |\alpha|^2$ (nur statistische Vorhersagen)
			\item Kollaps: Nicht-unitärer Messprozess
			\item Interpretation: Kopenhagen vs. Viele-Welten vs. andere
			\item Einzelmessungen: Unvorhersehbar (fundamental zufällig)
		\end{itemize}
	\end{tcolorbox}
	
	\subsection{T0-Energiefeld-Lösung}
	
	Der T0-Rahmen bietet eine komplette Lösung durch deterministische Energiefelder:
	
	\begin{tcolorbox}[colback=blue!5!white,colframe=blue!75!black,title=T0 Deterministisches Fundament]
		\textbf{Deterministische Energiefeldphysik}:
		\begin{itemize}
			\item Universelles Feld: $\Efield(x,t)$ (einziges Energiefeld für alle Phänomene)
			\item Feldgleichung: $\partial^2 \Efield = 0$ (deterministische Evolution)
			\item SI-Referenz: $\xipar = 1.33 \times 10^{-4}$ (verbindet Verhältnisse mit Messungen)
			\item Keine Wahrscheinlichkeiten: Nur Energiefeldverhältnisse
			\item Kein Kollaps: Kontinuierliche deterministische Evolution
			\item Einzige Realität: Keine Interpretationsprobleme
		\end{itemize}
	\end{tcolorbox}
	
	\section{T0-Energiefeld-Grundlagen}
	
	\subsection{Universelle Energiefeldgleichung}
	
	Aus der T0-Revolution reduziert sich alle Physik auf:
	
	\begin{equation}
		\boxed{\partial^2 \Efield = 0}
		\label{eq:universal_field_equation}
	\end{equation}
	
	Diese Klein-Gordon-Gleichung für Energie beschreibt ALLE Teilchen und Felder.
	
	\subsection{Energie-Zeit-Beziehung}
	
	Die fundamentale T0-Beziehung:
	
	\begin{equation}
		\boxed{\Tfield(x,t) = \frac{1}{\max(\Efield(x,t), \omega)}}
		\label{eq:energy_time_relation}
	\end{equation}
	
	wobei $\omega$ charakteristische Frequenzen repräsentiert.
	
	\textbf{Dimensionsüberprüfung}: $[\Tfield] = [1/E] = [E^{-1}]$ \checkmark
	
	\subsection{SI-Referenzskala}
	
	Folgend dem verhältnisbasierten T0-Ansatz:
	
	\begin{equation}
		\boxed{\xipar = 1.33 \times 10^{-4}}
		\label{eq:si_reference_scale}
	\end{equation}
	
	Dieses dimensionslose Verhältnis verbindet Energiefeldbeziehungen mit SI-messbaren Größen.
	
	\section{Von Wahrscheinlichkeitsamplituden zu Energiefeldverhältnissen}
	
	\subsection{Standard QM Zustandsbeschreibung}
	
	\textbf{Traditioneller Ansatz}:
	\begin{equation}
		|\psi\rangle = \sum_i c_i |i\rangle \quad \text{mit } P_i = |c_i|^2
	\end{equation}
	
	\textbf{Probleme}: Rätselhafte Superposition, nur probabilistische Vorhersagen.
	
	\subsection{T0-Energiefeld-Zustandsbeschreibung}
	
	\textbf{T0 deterministischer Ansatz}:
	\begin{equation}
		\boxed{\text{Zustand} \equiv \{\Efield_i(x,t)\} \quad \text{mit Verhältnissen } R_i = \frac{\Efield_i}{\sum_j \Efield_j}}
	\end{equation}
	
	\textbf{Vorteile}: 
	\begin{itemize}
		\item Keine rätselhafte Superposition - nur Energiefeldkonfigurationen
		\item Deterministische Evolution durch $\partial^2 \Efield = 0$
		\item Verhältnisse $R_i$ sind messbare Größen, keine Wahrscheinlichkeiten
		\item Vorhersagen für Einzelmessungen möglich
	\end{itemize}
	
	\subsection{Übersetzungsregeln}
	
	\textbf{Systematische Konversion von QM zu T0}:
	\begin{align}
		|\psi|^2 &\rightarrow \text{Energiefelddichte } \rho_E(x,t) \\
		\langle\psi|\hat{O}|\psi\rangle &\rightarrow \text{Energiefeldintegral } \int \Efield(x,t) \, O \, dx \\
		P_i &\rightarrow \text{Energiefeldverhältnis } \frac{\Efield_i}{\sum_j \Efield_j}
	\end{align}
	
	\section{Deterministische Spinsysteme}
	
	\subsection{Spin-1/2 in T0-Formulierung}
	
	\subsubsection{Standard QM Ansatz}
	
	\textbf{Zustand}: $|\psi\rangle = \alpha|{\uparrow}\rangle + \beta|{\downarrow}\rangle$
	
	\textbf{Erwartungswert}: $\langle \sigma_z \rangle = |\alpha|^2 - |\beta|^2$
	
	\subsubsection{T0-Energiefeld-Ansatz}
	
	\textbf{Zustand}: Energiefeldkonfiguration
	\begin{align}
		\Efield_{\uparrow}(x,t) &= \text{Energiefeld für Spin-up-Zustand} \\
		\Efield_{\downarrow}(x,t) &= \text{Energiefeld für Spin-down-Zustand}
	\end{align}
	
	\textbf{Deterministischer Erwartungswert}:
	\begin{equation}
		\boxed{\langle \sigma_z \rangle_{T0} = \frac{\Efield_{\downarrow} - \Efield_{\uparrow}}{\Efield_{\downarrow} + \Efield_{\uparrow}}}
		\label{eq:deterministic_spin_z}
	\end{equation}
	
	\textbf{Dimensionsüberprüfung}: $[\langle \sigma_z \rangle_{T0}] = [E/E] = [1]$ (dimensionslos) \checkmark
	
	\subsection{Quantitatives Beispiel mit SI-Referenz}
	
	Unter Verwendung der SI-Referenzskala $\xipar = 1.33 \times 10^{-4}$:
	
	\textbf{Energiefeldkonfiguration}:
	\begin{align}
		\Efield_{\uparrow} &= E_0 (1 + \xipar \cdot \mathcal{F}_{\text{up}}) \\
		\Efield_{\downarrow} &= E_0 (1 + \xipar \cdot \mathcal{F}_{\text{down}})
	\end{align}
	
	wobei $\mathcal{F}$ Feldkonfigurationsfaktoren repräsentiert.
	
	\textbf{T0-Korrektur zum Erwartungswert}:
	\begin{equation}
		\langle \sigma_z \rangle_{T0} = \langle \sigma_z \rangle_{QM} + \xipar \cdot \Delta\sigma_z
	\end{equation}
	
	mit $\Delta\sigma_z \approx 1.33 \times 10^{-4} \times (\mathcal{F}_{\text{down}} - \mathcal{F}_{\text{up}})$.
	
	\section{Deterministische Quantenverschränkung}
	
	\subsection{Standard QM Verschränkung}
	
	\textbf{Bell-Zustand}: $|\Psi^-\rangle = \frac{1}{\sqrt{2}}(|{\uparrow\downarrow}\rangle - |{\downarrow\uparrow}\rangle)$
	
	\textbf{Problem}: Nicht-lokale spukhafte Fernwirkung
	
	\subsection{T0-Energiefeld-Verschränkung}
	
	\textbf{Verschränkung als korrelierte Energiefeldstruktur}:
	\begin{equation}
		\boxed{\Efield_{12}(x_1, x_2, t) = \Efield_1(x_1, t) + \Efield_2(x_2, t) + \Efield_{\text{corr}}(x_1, x_2, t)}
	\end{equation}
	
	\textbf{Korrelationsenergiefeld}:
	\begin{equation}
		\Efield_{\text{corr}}(x_1, x_2, t) = \xipar \cdot \frac{\Efield_1 \cdot \Efield_2}{|x_1 - x_2|^2}
	\end{equation}
	
	\textbf{Physikalische Interpretation}: Verschränkung durch direkte Energiefeldkorrelation, nicht rätselhafte Superposition.
	
	\subsection{Modifizierte Bell-Ungleichung}
	
	Das T0-Modell sagt eine modifizierte Bell-Ungleichung voraus:
	
	\begin{equation}
		\boxed{|E(a,b) - E(a,c)| + |E(a',b) + E(a',c)| \leq 2 + \varepsilon_{T0}}
	\end{equation}
	
	mit der T0-Korrektur:
	\begin{equation}
		\varepsilon_{T0} = \xipar \cdot \left|\frac{\Efield_1 - \Efield_2}{\Efield_1 + \Efield_2}\right| \cdot \frac{2G\langle E \rangle}{r_{12}}
	\end{equation}
	
	\textbf{Numerische Abschätzung}:
	Für typische atomare Systeme mit $r_{12} \sim 1$ m, $\langle E \rangle \sim 1$ eV:
	\begin{align}
		\varepsilon_{T0} &\approx 1.33 \times 10^{-4} \times 1 \times \frac{2 \times 6.7 \times 10^{-11} \times 1.6 \times 10^{-19}}{1} \\
		&\approx 2.8 \times 10^{-34}
	\end{align}
	
	Dies ist extrem klein aber potenziell mit präzisen Bell-Experimenten nachweisbar.
	
	\section{Deterministisches Quantencomputing}
	
	\subsection{Qubit-Darstellung}
	
	\textbf{Standard QM Qubit}: $|\text{Qubit}\rangle = \alpha|0\rangle + \beta|1\rangle$
	
	\textbf{T0-Energiefeld-Qubit}:
	\begin{equation}
		\boxed{\text{Qubit}_{T0} \equiv \{\Efield_0(x,t), \Efield_1(x,t)\}}
	\end{equation}
	
	\textbf{Qubit-Operationen sind Energiefeldtransformationen}.
	
	\subsection{Quantengatter als Energiefeldoperationen}
	
	\subsubsection{Hadamard-Gatter}
	
	\textbf{Standard}: $H|0\rangle = \frac{1}{\sqrt{2}}(|0\rangle + |1\rangle)$
	
	\textbf{T0-Transformation}:
	\begin{align}
		H_{T0}: \quad \Efield_0 &\rightarrow \frac{\Efield_0 + \Efield_1}{2} \\
		\Efield_1 &\rightarrow \frac{\Efield_0 + \Efield_1}{2}
	\end{align}
	
	\subsubsection{CNOT-Gatter}
	
	\textbf{T0-Formulierung}:
	\begin{equation}
		\text{CNOT}_{T0}: \Efield_{12} \rightarrow \Efield_{12} + \xipar \cdot \delta(\Efield_1 - \Efield_{\text{threshold}}) \cdot \Efield_2
	\end{equation}
	
	\textbf{Physikalische Interpretation}: Bedingte Energiefeldkopplung wenn Kontroll-Qubit Schwellwert überschreitet.
	
	\subsection{Deterministische Quantenalgorithmen}
	
	\textbf{Schlüsselidee}: Alle Quantenalgorithmen werden zu deterministischen Energiefeldevolutionen.
	
	\textbf{Grovers Algorithmus}:
	- Amplitudenverstärkung → Energiefeldfokussierung
	- Ergebnis: Deterministisch berechenbare Anzahl an Iterationen
	
	\textbf{Shors Algorithmus}:
	- Periodenfindung → Energiefeldresonanzdetektion
	- Ergebnis: Deterministische Faktorisierung (keine probabilistischen Elemente)
	
	\section{Experimentelle Vorhersagen und Tests}
	
	\subsection{Vorhersagen für Einzelmessungen}
	
	\textbf{Revolutionäre Fähigkeit}: T0-Modell sagt individuelle Messergebnisse voraus.
	
	\textbf{Beispiel - Einzelne Spinmessung}:
	\begin{equation}
		\text{Ergebnis} = \text{sign}\left(\Efield_{\uparrow}(x_{\text{detektor}}, t_{\text{Messung}}) - \Efield_{\downarrow}(x_{\text{detektor}}, t_{\text{Messung}})\right)
	\end{equation}
	
	\textbf{Kein Zufall} - jedes Messergebnis ist im Voraus berechenbar.
	
	\subsection{T0-spezifische experimentelle Signaturen}
	
	\subsubsection{Modifizierte Bell-Tests}
	
	\textbf{Vorhersage}: Bell-Ungleichungsverletzung modifiziert durch $\varepsilon_{T0} \approx 10^{-34}$
	
	\textbf{Testanforderung}: Ultra-hochpräzise Bell-Experimente
	
	\subsubsection{Energiefeldabbildung}
	
	\textbf{Neue Technik}: Direkte Messung von $\Efield(x,t)$-Verteilungen
	
	\textbf{Vorhersage}: Räumliche Struktur von Quantenzuständen als Energiefeldmuster
	
	\subsubsection{Deterministische Quanteninterferenz}
	
	\textbf{Vorhersage}: Interferenzmuster sind deterministische Energiefeldsuperpositionen
	
	\textbf{Test}: Einzelteilcheninterferenz mit vorherbestimmtem Ergebnis
	
	\subsection{Technologische Anwendungen}
	
	\textbf{Deterministisches Quantencomputing}:
	- Keine probabilistische Fehlerkorrektur nötig
	- Deterministische Algorithmusausführung
	- Vorhersehbare Quantengatteroperationen
	
	\textbf{Verbesserte Quantensensorik}:
	- Präzision bei Einzelmessungen
	- Energiefeldbasierte Detektionsschemen
	- Deterministische Verschränkungserzeugung
	
	\section{Lösung von Quanteninterpretationsproblemen}
	
	\subsection{Durch T0-Formulierung gelöste Probleme}
	
\begin{table}[htbp]
	\centering
	\small
	\begin{tabular}{|p{4cm}|p{5cm}|p{6cm}|}
		\hline
		\textbf{QM Problem} & \textbf{Standardansätze} & \textbf{T0-Lösung} \\
		\hline
		Messproblem & Kopenhagen-Interpretation, Kollaps & Kein Kollaps - kontinuierliche Feldevolutio \\ % Added a line break and closed the cell
		\hline % It's good practice to have \hline at the end of the table content
	\end{tabular}
	\caption{Vergleich von QM-Problemen, Standardansätzen und der T0-Lösung} % Add a caption for your table
	\label{tab:qm_problem_comparison} % Add a label for cross-referencing
\end{table}

\section{Vereinfachte Quantenrealität}

\begin{tcolorbox}[colback=green!5!white,colframe=green!75!black,title=T0 Quantenrealität]
	\textbf{Einfache, deterministische Quantenmechanik}:
	\begin{itemize}
		\item Energiefelder $\Efield(x,t)$ existieren als reale physikalische Entitäten
		\item Sie entwickeln sich deterministisch: $\partial^2 \Efield = 0$
		\item Messungen enthüllen aktuelle Feldwerte am Detektorort
		\item Kein rätselhafter Wellenfunktionskollaps
		\item Keine nicht-unitären Prozesse
		\item Kein fundamentaler Zufall
		\item Einzige, konsistente Realität (keine Viele-Welten)
	\end{itemize}
\end{tcolorbox}
			\section{Verbindung zu anderen T0-Entwicklungen}
			
			\subsection{Integration mit vereinfachter Dirac-Gleichung}
			
			Die deterministische QM verbindet sich natürlich mit der vereinfachten Dirac-Gleichung:
			\begin{equation}
				\partial^2 \Efield = 0 \quad \text{(dieselbe fundamentale Gleichung)}
			\end{equation}
			
			\textbf{Einsicht}: Quantenmechanik und relativistische Feldtheorie vereinigt durch dieselbe Energiefelddynamik.
			
			\subsection{Integration mit universeller Lagrangefunktion}
			
			Die universelle Lagrangefunktion $\mathcal{L} = \varepsilon \cdot (\partial \Efield)^2$ beschreibt:
			
			Klassische Feldevoluton
			
			Quantenfeldevoluton
			
			Relativistische Feldevoluton
			
			\textbf{Gesamte Physik aus einer Gleichung}.
			
			\subsection{Integration mit Verhältnisphysik}
			
			Deterministische QM erbt die verhältnisbasierte Struktur:
			
			Quantenzustände als Energiefeldverhältnisse
			
			Messungen als Verhältnisvergleiche
			
			SI-Referenz $\xipar$ für quantitative Vorhersagen
			
			\section{Zukünftige Richtungen und Implikationen}
			
			\subsection{Experimentelles Verifikationsprogramm}
			
			\textbf{Phase 1 - Machbarkeitsnachweis}:
			\begin{itemize}
				\item Vorhersagen für Einzelmessungen in einfachen Systemen
				\item Energiefeldabbildungstechniken
				\item Modifizierte Bell-Tests
			\end{itemize}
			
			\textbf{Phase 2 - Technologische Anwendungen}:
			\begin{itemize}
				\item Deterministische Quantencomputerarchitekturen
				\item Verbesserte Quantensensorikprotokolle
				\item Energiefeldbasierte Quantengeräte
			\end{itemize}
			
			\textbf{Phase 3 - Fundamentalphysik}:
			\begin{itemize}
				\item Kompletter Ersatz probabilistischer QM
				\item Neue Quantenfeldtheorieformulierungen
				\item Integration mit Quantengravitation
			\end{itemize}
			
			\subsection{Philosophische Implikationen}
			
			\begin{tcolorbox}[colback=purple!5!white,colframe=purple!75!black,title=Das Ende der Quantenmystik]
				\textbf{Deterministische Quantenmechanik eliminiert}:
				\begin{itemize}
					\item Fundamentalem Zufall
					\item Beobachterabhängiger Realität
					\item Messungsinduziertem Kollaps
					\item Multiplen Parallelwelten
					\item Nicht-lokalen instantanen Einflüssen
				\end{itemize}
				
				\textbf{Und etabliert}:
				\begin{itemize}
					\item Einzige, objektive Realität
					\item Deterministische Naturgesetze
					\item Lokale Energiefeldwechselwirkungen
					\item Vorhersehbare individuelle Ereignisse
					\item Vereinigte klassisch-quantenphysik
				\end{itemize}
			\end{tcolorbox}
			
			\section{Zusammenfassung: Die vollendete Quantenrevolution}
			
			\subsection{Revolutionäre Errungenschaften}
			
			Die T0-Energiefeldformulierung hat erreicht:
			
			\begin{enumerate}
				\item \textbf{Beseitigung von Quanteninterpretationsproblemen}: Keine Debatten mehr zwischen Kopenhagen vs. Viele-Welten
				\item \textbf{Etablierung deterministischer Quantenmechanik}: Vorhersagbarkeit individueller Messungen
				\item \textbf{Vereinigung mit T0-Rahmenwerk}: Dieselbe Energiefeldphysik über alle Skalen
				\item \textbf{Beibehaltung experimenteller Äquivalenz}: Alle QM-Vorhersagen erhalten
				\item \textbf{Erweiterte Vorhersagekraft}: Neue T0-spezifische Effekte
				\item \textbf{Vereinfachte Quantenrealität}: Einzige deterministische Welt
			\end{enumerate}
			
			\subsection{Die vollständige T0-Revolution}
			
			Mit deterministischer Quantenmechanik ist die T0-Revolution vollendet:
			
			\textbf{Stufe 1}: Vereinfachte Teilchenphysik (Dirac-Gleichung)
			\textbf{Stufe 2}: Vereinigte Feldtheorie (Universelle Lagrangefunktion)
			\textbf{Stufe 3}: Parameterfreie Physik (Verhältnisbasierter Ansatz)
			\textbf{Stufe 4}: Deterministische Quantenmechanik (Diese Arbeit)
			
			\textbf{Ergebnis}: Vollständige, konsistente, deterministische Beschreibung aller \
			physikalischen Phänomene durch Energiefelddynamik.
			
			\subsection{Zukünftige Auswirkungen}
			
			\begin{equation}
				\boxed{\text{Gesamte Physik} = \text{Deterministische Energiefeldevoluton}}
			\end{equation}
			
			Von Quantenmechanik bis Kosmologie, von Teilchenphysik bis \
			Bewusstsein - alles entsteht aus der deterministischen Entwicklung \
			von Energiefeldern, beschrieben durch $\partial^2 \Efield = 0$.
			
			\textbf{Die T0-Revolution hat Physik von probabilistischer \
				Komplexität zu deterministischer Eleganz transformiert.}
			
			\begin{thebibliography}{99}
				\bibitem{pascher_simplified_dirac_2025}
				Pascher, J. (2025). \textit{Vereinfachte Dirac-Gleichung in T0-Theorie: Von komplexen 4×4 Matrizen zu einfacher Feldknotendynamik}. \
				\href{https://github.com/jpascher/T0-Time-Mass-Duality/blob/main/2/pdf/diracVereinfachtEn.pdf}{GitHub Repository: T0-Time-Mass-Duality}.
				%---
				\bibitem{pascher_lagrangian_comparison_2025}
				Pascher, J. (2025). \textit{Einfache Lagrangefunktions-Revolution: Von Standardmodell-Komplexität zu T0-Eleganz}. \\
				\href{https://github.com/jpascher/T0-Time-Mass-Duality/blob/main/2/pdf/LagrandianVergleichEn.pdf}{GitHub Repository: T0-Time-Mass-Duality}.
				
				\bibitem{pascher_ratio_physics_2025}
				Pascher, J. (2025). \textit{Reine Energie T0-Theorie: Die verhältnisbasierte Revolution}. \\
				\href{https://github.com/jpascher/T0-Time-Mass-Duality/blob/main/2/pdf/Elimination_Of_Mass_Dirac_LagEn.pdf}{GitHub Repository: T0-Time-Mass-Duality}.
				
				\bibitem{pascher_verification_table_2025}
				Pascher, J. (2025). \textit{T0-Modellverifikation: Skalenverhältnisbasierte Berechnungen vs. CODATA/Experimentelle Werte}. \\
				\href{https://github.com/jpascher/T0-Time-Mass-Duality/blob/main/2/pdf/Elimination_Of_Mass_Dirac_TabelleEn.pdf}{GitHub Repository: T0-Time-Mass-Duality}.
				
				\bibitem{pascher_ho_energie_2025}
				Pascher, J. (2025). \textit{Reine Energieformulierung der $H_0$ und $\kappa$ Parameter im T0-Modellrahmen}. \\
				\href{https://github.com/jpascher/T0-Time-Mass-Duality/blob/main/2/pdf/Ho_EnergieEn.pdf}{GitHub Repository: T0-Time-Mass-Duality}.
				
				\bibitem{pascher_derivation_beta_2025}
				Pascher, J. (2025). \textit{Feldtheoretische Herleitung des $\beta_T$ Parameters in natürlichen Einheiten}. \\
				\href{https://github.com/jpascher/T0-Time-Mass-Duality/blob/main/2/pdf/DerivationVonBetaEn.pdf}{GitHub Repository: T0-Time-Mass-Duality}.
				
				\bibitem{bell1964}
				Bell, J.S. (1964). Über das Einstein Podolsky Rosen Paradoxon. \textit{Physics Physique Fizika}, \textbf{1}, 195--200.
				
				\bibitem{einstein1905}
				Einstein, A. (1905). Ist die Trägheit eines Körpers von seinem Energieinhalt abhängig? \textit{Annalen der Physik}, 17, 639.
				
				\bibitem{schrodinger1926}
				Schrödinger, E. (1926). Quantisierung als Eigenwertproblem. \textit{Annalen der Physik}, 79, 361--376.
				
				\bibitem{dirac1928}
				Dirac, P.A.M. (1928). Die Quantentheorie des Elektrons. \textit{Proceedings of the Royal Society A}, 117, 610--624.
				
				\bibitem{grover1996}
				Grover, L.K. (1996). Ein schneller quantenmechanischer Algorithmus für Datenbanksuche. \textit{Proceedings of the 28th Annual ACM Symposium on Theory of Computing}, 212--219.
				
				\bibitem{shor1994}
				Shor, P.W. (1994). Algorithmen für Quantenberechnung: Diskrete Logarithmen und Faktorisierung. \textit{Proceedings 35th Annual Symposium on Foundations of Computer Science}, 124--134.
				
				%---
			\end{thebibliography}
			
		\end{document}