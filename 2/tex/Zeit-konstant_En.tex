\documentclass[12pt,a4paper]{article}
\usepackage[utf8]{inputenc}
\usepackage[T1]{fontenc}
\usepackage[english]{babel}
\usepackage{amsmath,amsfonts,amssymb}
\usepackage{physics}
\usepackage{siunitx}
\usepackage{booktabs}
\usepackage{longtable}
\usepackage{array}
\usepackage{xcolor}
\usepackage{geometry}
\usepackage{textgreek}
\usepackage{fancyhdr}
\usepackage{hyperref}
\usepackage{tocloft}
\geometry{margin=2.5cm}

% Configuration of header and footer
\pagestyle{fancy}
\fancyhf{}
\fancyhead[L]{\textsc{T0 Model}}
\fancyhead[R]{\textsc{A Reformulation of Physics}}
\fancyfoot[C]{\thepage}
\renewcommand{\headrulewidth}{0.4pt}
\renewcommand{\footrulewidth}{0.4pt}

% Formatting of the table of contents
\renewcommand{\cfttoctitlefont}{\huge\bfseries\color{blue}}
\renewcommand{\cftsecfont}{\color{blue}}
\renewcommand{\cftsubsecfont}{\color{blue}}
\renewcommand{\cftsecpagefont}{\color{blue}}
\renewcommand{\cftsubsecpagefont}{\color{blue}}

% Hyperlink configuration
\hypersetup{
	colorlinks=true,
	linkcolor=blue,
	citecolor=blue,
	urlcolor=blue,
	pdftitle={The T0 Model: Time-Energy Duality and Geometric Rest Mass},
	pdfauthor={Johann Pascher},
	pdfsubject={T0 Model, Time-Energy Duality, Theoretical Physics},
	pdfkeywords={T0 Theory, Geometric Rest Mass, Time-Energy Duality, Cosmology}
}

\title{The T0 Model: Time-Energy Duality and Geometric Rest Mass\\
	\large (Energy-Based Version)}
\author{Johann Pascher\\
	\small Höhere Technische Bundeslehranstalt (HTL), Leonding, Austria\\
	\small \texttt{johann.pascher@gmail.com}}
\date{\today}

\begin{document}
	
	\maketitle
	\tableofcontents
	\newpage
	
	\begin{abstract}
		The T0 model describes the physical properties of our observable space within an eternal, infinite, non-expanding universe without a beginning or end. It is based on a time-energy duality and a geometric definition of rest mass, coupled to the spatial geometry. Time could theoretically be absolute, but is set as variable for practical reasons, as measurements rely on frequency changes. The rest mass serves as a practical fixed point but is theoretically variable in a dynamic space. The cosmic microwave background (CMB) is explained through \(\xi\)-field mechanisms, without assuming a Big Bang. Extrapolations to extreme scenarios such as black holes or the use of dark matter and vacuum energy as energy sources are highly speculative and beyond the scope of the model \cite{pascher_t0_energie_2025}.
	\end{abstract}
	
	\section{Introduction}
	The T0 model is a theoretical framework that describes the physical phenomena of our observable space in an eternal, infinite, non-expanding universe without a beginning or end \cite{pascher_t0_energie_2025}. In contrast to the standard model of cosmology, which postulates a Big Bang and an expanding spacetime, the T0 model assumes a fixed universe where the geometric constant \(\xi_0 = \frac{4}{3} \times 10^{-4}\) defines the spatial structure \cite{Casimir1948}. Mass and energy are different forms of an underlying quantity, and time could theoretically be absolute (\( T = t \)), but is practically set as variable to interpret frequency changes. This document summarizes the key aspects of the model, focusing on observable space and explicitly warning against speculative extrapolations to black holes or the use of dark matter and vacuum energy as energy sources.
	
	\textbf{Note:} The T0 model primarily describes observable space through experiments such as the Casimir effect or spectroscopy. Extrapolations to black holes or speculative energy sources like dark matter are highly speculative and not covered by the model.
	
	\section{Universe in the T0 Model}
	The T0 model assumes an eternal, infinite, non-expanding universe without a beginning or end, in contrast to the standard model of cosmology. The spatial structure is defined by the geometric constant \(\xi_0 = \frac{4}{3} \times 10^{-4}\), which is globally stable but can be locally dynamic \cite{pascher_t0_energie_2025}. The cosmic microwave background (CMB) is interpreted as a static property of the universe, arising through \(\xi\)-field mechanisms without assuming a Big Bang \cite{pascher_t0_cmb_2025}. In such a universe, time could theoretically be absolute (\( T = t \)), but is set as locally variable to account for the time-energy duality and frequency measurements.
	
	\section{CMB in the T0 Model: Static \(\xi\)-Universe}
	The cosmic microwave background (CMB) in the T0 model is not explained by a decoupling at \( z \approx 1100 \), as in the standard model, but through \(\xi\)-field mechanisms in an infinitely old universe \cite{pascher_t0_cmb_2025}.
	
	\textbf{Time-energy duality forbids a Big Bang:} The CMB background radiation has a different origin than in the standard model and is explained by the following mechanisms:
	
	\subsection{\(\xi\)-Field Quantum Fluctuations}
	The omnipresent \(\xi\)-field generates vacuum fluctuations with a characteristic energy scale. The ratio \( \frac{T_{\text{CMB}}}{E_\xi} \approx \xi^2 \) connects the CMB temperature to the geometric scale \(\xi_0\) \cite{pascher_t0_cmb_2025}.
	
	\subsection{Steady-State Thermalization}
	In an infinitely old universe, the background radiation reaches thermodynamic equilibrium at a characteristic \(\xi\)-temperature, harmonizing with the geometric scale \cite{pascher_t0_cmb_2025}.
	
	\section{Time-Energy Duality}
	The time-energy duality is the core principle of the T0 model:
	\begin{equation}
		T(x,t) \cdot E(x,t) = 1, \quad T(x,t) = \frac{1}{\max(E(x,t), \omega)}
	\end{equation}
	Here, \(E(x,t)\) is the local energy density, \(T(x,t)\) is the intrinsic time, and \(\omega\) is a reference energy (e.g., rest frequency or photon frequency). In an eternal, infinite universe, time could be globally absolute (\( T = t \)), but is locally set as variable to account for the duality and frequency changes:
	\begin{equation}
		\Delta \omega = \frac{\Delta E}{\hbar}
	\end{equation}
	
	\section{Geometric Definition of Rest Mass}
	The rest mass is defined by a geometric resonance:
	\begin{equation}
		E_{\text{char},i} = m_i c^2 = \frac{1}{\xi_i}, \quad \xi_i = \xi_0 \cdot r_i, \quad \xi_0 = \frac{4}{3} \times 10^{-4}
	\end{equation}
	where \(r_i\) is a suppression factor \cite{pascher_t0_energie_2025}. For an electron:
	\begin{equation}
		\xi_e = \frac{4}{3} \times 10^{-4}, \quad m_e c^2 = 0.511 \, \text{MeV}
	\end{equation}
	
	\subsection{Practical Fixed Point}
	For measurements, the rest mass is assumed to be a fixed point:
	\begin{equation}
		m_i = \frac{1}{\xi_i c^2}
	\end{equation}
	This allows the interpretation of frequency changes:
	\begin{equation}
		E(x,t) = \gamma m_i c^2, \quad \omega = \frac{E(x,t)}{\hbar}
	\end{equation}
	
	\subsection{Theoretical Variability}
	In a dynamic space, the rest mass is variable:
	\begin{equation}
		\xi_i(x,t) = \xi_0(x,t) \cdot r_i, \quad m_i(x,t) = \frac{1}{\xi_i(x,t) c^2}
	\end{equation}
	Frequency changes reflect kinetic energy and mass variations:
	\begin{equation}
		\omega(x,t) = \frac{\gamma(x,t) m_i(x,t) c^2}{\hbar}
	\end{equation}
	
	\section{Vacuum and Casimir-CMB Ratio}
	The vacuum is the ground state of the energy field:
	\begin{equation}
		E(x,t) \approx |\rho_{\text{Casimir}}| = \frac{\pi^2}{240 \times L_\xi^4}, \quad L_\xi = 10^{-4} \, \text{m}
	\end{equation}
	The Casimir-CMB ratio confirms the geometric scale \cite{Casimir1948, Planck2018}:
	\begin{equation}
		\frac{|\rho_{\text{Casimir}}|}{\rho_{\text{CMB}}} = \frac{\pi^2}{240 \xi} \approx 308
	\end{equation}
	In a dynamic space, \(L_\xi(x,t)\) becomes variable, making the ratio dynamic.
	
	\section{Dynamic Space}
	A dynamic space implies:
	\begin{equation}
		\xi_0(x,t)
	\end{equation}
	This allows a variable rest mass and a globally absolute time:
	\begin{equation}
		m_i(x,t) = \frac{1}{\gamma(x,t) c^2 t}
	\end{equation}
	Frequency changes are not specific enough to directly confirm mass variations.
	
	\section{Stability of the Overall System}
	The model remains stable through the field equation:
	\begin{equation}
		\nabla^2 E(x,t) = 4\pi G \rho(x,t) \cdot E(x,t)
	\end{equation}
	Local variations minimally affect the system.
	
	\section{Limitations and Speculations}
	The T0 model describes observable space. Extrapolations to black holes or cosmological scales are speculative due to:
	\begin{itemize}
		\item The spatial geometry not being covered in extreme scenarios.
		\item Frequency measurements in strong gravitational fields exhibiting additional effects.
		\item Lack of experimental data.
	\end{itemize}
	
	\textbf{Warning to Speculators:} Notions of using dark matter or vacuum energy as energy sources are unrealistic. The usable energy is limited to the amount verified by the Casimir effect (\( |\rho_{\text{Casimir}}| = \frac{\pi^2}{240 \times L_\xi^4} \)), which is experimentally confirmed \cite{Casimir1948}. Larger energy quantities, particularly from dark matter, lack any experimental evidence and are beyond the T0 model \cite{pascher_t0_energie_2025}.
	
	\section{Conclusion}
	The T0 model describes observable space in an eternal, infinite, non-expanding universe. The time-energy duality and geometric rest mass provide a robust description, with time potentially globally absolute but locally set as variable. Frequency changes limit the verification of time dilation or mass variations. The CMB is explained through \(\xi\)-field mechanisms, without a Big Bang. Extrapolations to black holes or speculative energy sources like dark matter are unrealistic \cite{pascher_t0_energie_2025}.
	
	\begin{thebibliography}{9}
		\bibitem{pascher_t0_energie_2025}
		Pascher, J. (2025). \textit{The T0 Model (Planck-Referenced): A Reformulation of Physics}. Available at: \url{https://github.com/jpascher/T0-Time-Mass-Duality/tree/main/2/pdf/T0-Energie_De.pdf}
		
		\bibitem{pascher_t0_cmb_2025}
		Pascher, J. (2025). \textit{CMB in T0-Theory: Static \(\xi\)-Universe}. Available at: \url{https://github.com/jpascher/T0-Time-Mass-Duality/tree/main/2/pdf/TempEinheitenCMBEn.pdf}
		
		\bibitem{Casimir1948}
		H. B. G. Casimir, ``On the attraction between two perfectly conducting plates,'' \emph{Proc. K. Ned. Akad. Wet.}, vol. 51, pp. 793--795, 1948.
		
		\bibitem{Planck2018}
		Planck Collaboration, ``Planck 2018 results. VI. Cosmological parameters,'' \emph{Astron. Astrophys.}, vol. 641, A6, 2020.
	\end{thebibliography}
	
\end{document}