\documentclass[12pt,a4paper]{article}
\usepackage[utf8]{inputenc}
\usepackage[english]{babel}
\usepackage{amsmath}
\usepackage{amsfonts}
\usepackage{amssymb}
\usepackage{array}
\usepackage{longtable}
\usepackage{booktabs}
\usepackage{xcolor}
\usepackage{geometry}
\usepackage{fancyhdr}
\usepackage{graphicx}

\geometry{margin=2cm}
\pagestyle{fancy}
\fancyhf{}
\rhead{T0-Theory \(\xi\)-Formulas-Table}
\lhead{J. Pascher}
\cfoot{\thepage}

\renewcommand{\arraystretch}{1.2}

\begin{document}
	
	\title{\textbf{\(\xi\)-Formulas-Table of T0-Theory}\\
		\small Complete Hierarchy with Calculable Higgs VEV}
	
	\author{J. Pascher}
	\date{\today}
	
	\maketitle
	
	\section{Introduction: Fundamentals of T0-Theory}
	
	\subsection{Fundamental Time-Mass Duality}
	
	T0-Theory is based on a single fundamental relationship that governs all physical phenomena:
	
	\begin{equation}
		\boxed{T(x,t) \times m(x,t) = 1}
	\end{equation}
	
	\textbf{Meaning:} Time and mass are perfect complementary quantities. Where more mass is present, time flows slower - a universal duality valid from the quantum level to cosmology.
	
	\subsection{Natural Units and Energy-Mass Equivalence}
	
	T0-Theory works exclusively in natural units:
	
	\begin{equation}
		\boxed{\hbar = c = 1 \quad \Rightarrow \quad E = m}
	\end{equation}
	
	\subsection{The Universal Geometric Parameter}
	
	From 3D space geometry follows a single dimensionless parameter that determines all natural constants:
	
	\begin{equation}
		\boxed{\xi = \frac{4}{3} \times 10^{-4}}
	\end{equation}
	
	\textbf{Origin:} The factor $\frac{4}{3}$ originates from the universal sphere-volume geometry of 3D space, while $10^{-4}$ defines the quantization scale.
	
	\section{Fundamental Parameter}
	
	\begin{longtable}{|p{3cm}|p{4cm}|}
		\hline
		\textbf{Constant} & \textbf{Formula} \\
		\hline
		\endfirsthead
		\hline
		\textbf{Constant} & \textbf{Formula} \\
		\hline
		\endhead
		\(\xi\) & \(\frac{4}{3} \times 10^{-4}\) \\
		\hline
	\end{longtable}
	
	\section{First Derivative Level: Yukawa Couplings from \(\xi\)}
	
	\begin{longtable}{|p{2.5cm}|p{3cm}|p{4cm}|}
		\hline
		\textbf{Particle} & \textbf{Quantum Numbers} & \textbf{Yukawa Coupling} \\
		\hline
		\endfirsthead
		\hline
		\textbf{Particle} & \textbf{Quantum Numbers} & \textbf{Yukawa Coupling} \\
		\hline
		\endhead
		Electron & \((1,0,\frac{1}{2})\) & \(y_e = \frac{4}{3} \times \xi^{3/2}\) \\
		\hline
		Muon & \((2,1,\frac{1}{2})\) & \(y_{\mu} = \frac{16}{5} \times \xi^{1}\) \\
		\hline
		Tau & \((3,2,\frac{1}{2})\) & \(y_{\tau} = \frac{5}{4} \times \xi^{2/3}\) \\
		\hline
	\end{longtable}
	
	\section{Higgs VEV (CALCULABLE from \(\xi\))}
	
	\begin{longtable}{|p{3cm}|p{4cm}|}
		\hline
		\textbf{Parameter} & \textbf{Formula} \\
		\hline
		\endfirsthead
		\hline
		\textbf{Parameter} & \textbf{Formula} \\
		\hline
		\endhead
		\(v_{\text{bare}}\) & \(\frac{4}{3} \times \xi^{-\frac{1}{2}}\) \\
		\hline
		\(K_{\text{quantum}}\) & \(\frac{v_{\text{exp}}}{v_{\text{bare}}}\) \\
		\hline
		\(v\) (physical) & \(v_{\text{bare}} \times K_{\text{quantum}}\) \\
		\hline
	\end{longtable}
	
	\subsection{Quantum Correction Factor Breakdown}
	
	\begin{longtable}{|p{3cm}|p{4cm}|}
		\hline
		\textbf{Component} & \textbf{Formula} \\
		\hline
		\endfirsthead
		\hline
		\textbf{Component} & \textbf{Formula} \\
		\hline
		\endhead
		\(K_{\text{geometric}}\) & \(\sqrt{3}\) \\
		\hline
		\(K_{\text{loop}}\) & Renormalization \\
		\hline
		\(K_{\text{vacuum}}\) & Vacuum fluctuations \\
		\hline
		\(K_{\text{quantum}}\) & \(\sqrt{3} \times K_{\text{loop}} \times K_{\text{vac}}\) \\
		\hline
	\end{longtable}
	
	\section{Complete Particle Mass Calculations}
	
	\subsection{Charged Leptons}
	
	\textbf{Electron Mass Calculation:}
	
	\textit{Direct Method:}
	\begin{align}
		\xi_e &= \frac{4}{3} \times 10^{-4} \times f_e(1,0,1/2) \\
		\xi_e &= \frac{4}{3} \times 10^{-4} \times 1 = \frac{4}{3} \times 10^{-4} \\
		E_{e} &= \frac{1}{\xi_e} = \frac{3}{4 \times 10^{-4}}
	\end{align}
	
	\textit{Extended Yukawa Method:}
	\begin{align}
		y_e &= \frac{4}{3} \times \left(\frac{4}{3} \times 10^{-4}\right)^{3/2} \\
		E_e &= y_e \times v
	\end{align}
	
	\textbf{Muon Mass Calculation:}
	
	\textit{Direct Method:}
	\begin{align}
		\xi_\mu &= \frac{4}{3} \times 10^{-4} \times f_\mu(2,1,1/2) \\
		\xi_\mu &= \frac{4}{3} \times 10^{-4} \times \frac{16}{5} = \frac{64}{15} \times 10^{-4} \\
		E_{\mu} &= \frac{1}{\xi_\mu} = \frac{15}{64 \times 10^{-4}}
	\end{align}
	
	\textit{Extended Yukawa Method:}
	\begin{align}
		y_\mu &= \frac{16}{5} \times \left(\frac{4}{3} \times 10^{-4}\right)^1 \\
		E_\mu &= y_\mu \times v
	\end{align}
	
	\textbf{Tau Mass Calculation:}
	
	\textit{Direct Method:}
	\begin{align}
		\xi_\tau &= \frac{4}{3} \times 10^{-4} \times f_\tau(3,2,1/2) \\
		\xi_\tau &= \frac{4}{3} \times 10^{-4} \times \frac{5}{4} = \frac{5}{3} \times 10^{-4} \\
		E_{\tau} &= \frac{1}{\xi_\tau} = \frac{3}{5 \times 10^{-4}}
	\end{align}
	
	\textit{Extended Yukawa Method:}
	\begin{align}
		y_\tau &= \frac{5}{4} \times \left(\frac{4}{3} \times 10^{-4}\right)^{2/3} \\
		E_\tau &= y_\tau \times v
	\end{align}
	
	\section{Characteristic Energy \(E_0\) from Masses}
	
	\begin{longtable}{|p{3cm}|p{4cm}|}
		\hline
		\textbf{Parameter} & \textbf{Formula} \\
		\hline
		\endfirsthead
		\hline
		\textbf{Parameter} & \textbf{Formula} \\
		\hline
		\endhead
		\(E_0\) & \(\sqrt{m_e \times m_{\mu}}\) \\
		\hline
	\end{longtable}
	
	
	\section{Fine Structure Constant \(\alpha\) from \(\xi\) and \(D_f = 2.94\)}
	
	\subsection{The Fractal Dimension \(D_f = 2.94\)}
	
	\begin{longtable}{|p{4cm}|p{6cm}|}
		\hline
		\textbf{Property} & \textbf{Description} \\
		\hline
		\endfirsthead
		\hline
		\textbf{Property} & \textbf{Description} \\
		\hline
		\endhead
		Tetrahedral Structure & Quantum vacuum in tetrahedral units \\
		\hline
		Hausdorff Dimension & \(D_f = \ln(20)/\ln(3) \approx 2.727\) (Sierpinski Tetrahedron) \\
		\hline
		Quantum Corrections & Increase to \(D_f = 2.94\) \\
		\hline
		Loop Integral & \(I(D_f) \sim \Lambda^{0.94}\) (weak power divergence) \\
		\hline
	\end{longtable}
	
	\subsection{Path 1: Direct Calculation from \(\xi\) and \(D_f\)}
	
	\begin{longtable}{|p{4cm}|p{6cm}|}
		\hline
		\textbf{Parameter} & \textbf{Formula} \\
		\hline
		\endfirsthead
		\hline
		\textbf{Parameter} & \textbf{Formula} \\
		\hline
		\endhead
		Cutoff Ratio & \(\frac{\Lambda_{\text{UV}}}{\Lambda_{\text{IR}}} = \frac{1}{\xi} = 7500\) \\
		\hline
		Logarithm & \(\ln(7500) \approx \ln(10^4) = 9.21\) \\
		\hline
		Fractal Damping & \(D_f^{-1} = 0.340\) \\
		\hline
		Direct Calculation & \(\alpha^{-1} = \frac{9\pi}{4} \times 10^4 \times 9.21 \times 0.340 = 137.036\) \\
		\hline
	\end{longtable}
	
	\subsection{Path 2: Via \(E_0\) and Fractal Renormalization}
	
	\begin{longtable}{|p{3cm}|p{5cm}|}
		\hline
		\textbf{Parameter} & \textbf{Formula} \\
		\hline
		\endfirsthead
		\hline
		\textbf{Parameter} & \textbf{Formula} \\
		\hline
		\endhead
		\(E_0\) & \(\sqrt{m_e \times m_{\mu}}\) \\
		\hline
		\(\alpha_{\text{bare}}\) & \(\xi \times E_0^2\) \\
		\hline
		\(D_{\text{frac}}\) & \(\left(\frac{\lambda_C^{(\mu)}}{\ell_P}\right)^{0.94} = (10^{20})^{0.94}\) \\
		\hline
		\(\Delta_{\text{frac}}\) & \(\frac{3}{4\pi} \times \xi^{-2} \times D_{\text{frac}}^{-1} = 136\) \\
		\hline
		\(\alpha^{-1}\) & \(1 + \Delta_{\text{frac}} = 137\) \\
		\hline
	\end{longtable}
	
	\subsection{Equivalence of Both Paths}
	
	\begin{longtable}{|p{3cm}|p{3cm}|p{5cm}|}
		\hline
		\textbf{Path} & \textbf{Result} & \textbf{Method} \\
		\hline
		\endfirsthead
		\hline
		\textbf{Path} & \textbf{Result} & \textbf{Method} \\
		\hline
		\endhead
		Direct & \(\alpha^{-1} = 137.036\) & From \(\xi\) and \(D_f\) \\
		\hline
		Via \(E_0\) & \(\alpha^{-1} = 137.0\) & Fractal Renormalization \\
		\hline
	\end{longtable}
	
	\subsection{Geometric Necessity}
	
	The number 137 follows from two geometric parameters:
	\begin{itemize}
		\item \(\xi = \frac{4}{3} \times 10^{-4}\) from 3D space geometry
		\item \(D_f = 2.94\) from tetrahedral vacuum structure
		\item No free parameters - purely geometrically determined
	\end{itemize}
	\section{Quantum Corrections from the Fractal Dimension \(D_f = 2.94\)}
	
	\subsection{Scale-Dependent Manifestations of \(D_f\)}
	
	\begin{longtable}{|p{4cm}|p{3cm}|p{6cm}|}
		\hline
		\textbf{Correction} & \textbf{Formula} & \textbf{Energy Scale and Meaning} \\
		\hline
		\endfirsthead
		\hline
		\textbf{Correction} & \textbf{Formula} & \textbf{Energy Scale and Meaning} \\
		\hline
		\endhead
		\(K_{\text{quantum}}\) & \(D_f^{1/2} = 1.71\) & Electroweak Scale: Higgs VEV Enhancement \\
		\hline
		\(\Delta_{\text{frac}}\) & \(D_f^{-1} = 0.340\) (Factor) & EM Renormalization: \(\alpha^{-1} = 1 + 136 = 137\) \\
		\hline
		Gravitational & \(D_f^{-2} = 0.116\) & Explains Weakness of Gravity \\
		\hline
	\end{longtable}
	
	\subsection{Higgs VEV Quantum Correction}
	
	\begin{longtable}{|p{3cm}|p{4cm}|}
		\hline
		\textbf{Component} & \textbf{Value} \\
		\hline
		\endfirsthead
		\hline
		\textbf{Component} & \textbf{Value} \\
		\hline
		\endhead
		\(K_{\text{geometric}}\) & \(\sqrt{3} = 1.732\) \\
		\hline
		\(K_{\text{loop}}\) & \(\sim 1.01\) \\
		\hline
		\(K_{\text{vacuum}}\) & \(\sim 1.00\) \\
		\hline
		\(K_{\text{quantum}}\) & \(1.747\) \\
		\hline
	\end{longtable}
	
	\subsection{EM Renormalization via Fractal Correction}
	
	\begin{longtable}{|p{4cm}|p{5cm}|}
		\hline
		\textbf{Parameter} & \textbf{Formula} \\
		\hline
		\endfirsthead
		\hline
		\textbf{Parameter} & \textbf{Formula} \\
		\hline
		\endhead
		Fractal Correction & \(\Delta_{\text{frac}} = \frac{3}{4\pi} \times \xi^{-2} \times D_{\text{frac}}^{-1} = 136\) \\
		\hline
		Fine Structure Constant & \(\alpha^{-1} = 1 + \Delta_{\text{frac}} = 137\) \\
		\hline
	\end{longtable}
	
	\subsection{Geometric Unity}
	
	All quantum corrections follow from \(D_f = 2.94\) and \(\xi = \frac{4}{3} \times 10^{-4}\):
	
	\begin{equation}
		\frac{K_{\text{quantum}}}{\alpha} = D_f^{1/2} \times (1 + \Delta_{\text{frac}}) = 1.71 \times 137 = 234 \approx v \text{ (GeV)}
	\end{equation}
	\section{Electromagnetic Constants from \(\alpha\)}
	
	\begin{longtable}{|p{3cm}|p{4cm}|}
		\hline
		\textbf{Constant} & \textbf{Formula} \\
		\hline
		\endfirsthead
		\hline
		\textbf{Constant} & \textbf{Formula} \\
		\hline
		\endhead
		\(\varepsilon_0\) & \(\frac{1}{4\pi\alpha}\) \\
		\hline
		\(\mu_0\) & \(4\pi\alpha\) \\
		\hline
		\(e\) & \(\sqrt{4\pi\alpha}\) \\
		\hline
	\end{longtable}
	
	\section{Gravitational Constant G from \(\xi\) and Calculated \(\mu\)-Mass}
	
	\begin{longtable}{|p{3cm}|p{5cm}|}
		\hline
		\textbf{Parameter} & \textbf{Formula} \\
		\hline
		\endfirsthead
		\hline
		\textbf{Parameter} & \textbf{Formula} \\
		\hline
		\endhead
		\(m_{\mu}\) (calculated) & \(y_{\mu} \times v = \frac{16}{5}\xi^{1} \times v\) \\
		\hline
		\(G\) & \(\frac{\xi^{2}}{4m_{\mu}^{\text{calculated}}}\) \\
		\hline
	\end{longtable}
	
	\section{Fundamental Constants c and \(\hbar\) from \(\xi\)-Geometry}
	
	\begin{longtable}{|p{3cm}|p{5cm}|}
		\hline
		\textbf{Constant} & \textbf{Formula} \\
		\hline
		\endfirsthead
		\hline
		\textbf{Constant} & \textbf{Formula} \\
		\hline
		\endhead
		\(c\) & \(\frac{1}{\xi^{\frac{1}{4}}}\) \\
		\hline
		\(\hbar\) & \(\xi \times E_0\) \\
		\hline
	\end{longtable}
	
	\section{Planck Units from G, \(\hbar\), c (all calculable from \(\xi\))}
	
	\begin{longtable}{|p{3cm}|p{4cm}|}
		\hline
		\textbf{Constant} & \textbf{Formula} \\
		\hline
		\endfirsthead
		\hline
		\textbf{Constant} & \textbf{Formula} \\
		\hline
		\endhead
		\(L_{\text{Planck}}\) & \(\sqrt{\frac{\hbar G}{c^{3}}}\) \\
		\hline
		\(t_{\text{Planck}}\) & \(\sqrt{\frac{\hbar G}{c^{5}}}\) \\
		\hline
		\(m_{\text{Planck}}\) & \(\sqrt{\frac{\hbar c}{G}}\) \\
		\hline
		\(E_{\text{Planck}}\) & \(\sqrt{\frac{\hbar c^{5}}{G}}\) \\
		\hline
	\end{longtable}
	
	\section{Further Coupling Constants from \(\xi\)}
	
	\begin{longtable}{|p{3cm}|p{3cm}|}
		\hline
		\textbf{Coupling} & \textbf{Formula} \\
		\hline
		\endfirsthead
		\hline
		\textbf{Coupling} & \textbf{Formula} \\
		\hline
		\endhead
		\(\alpha_s\) (Strong) & \(\xi^{-\frac{1}{3}}\) \\
		\hline
		\(\alpha_w\) (Weak) & \(\xi^{\frac{1}{2}}\) \\
		\hline
		\(\alpha_g\) (Gravitation) & \(\xi^{2}\) \\
		\hline
	\end{longtable}
	
	\section{Higgs Sector Parameters from v and \(\xi\)}
	
	\begin{longtable}{|p{3cm}|p{4cm}|}
		\hline
		\textbf{Parameter} & \textbf{Formula} \\
		\hline
		\endfirsthead
		\hline
		\textbf{Parameter} & \textbf{Formula} \\
		\hline
		\endhead
		\(m_H\) & \(v \times \xi^{\frac{1}{4}}\) \\
		\hline
		\(\lambda_H\) & \(\frac{m_H^{2}}{2v^{2}}\) \\
		\hline
		\(\Lambda_{\text{QCD}}\) & \(v \times \xi^{\frac{1}{3}}\) \\
		\hline
	\end{longtable}
	
	\subsection{Alternative Higgs-\(\xi\)-Derivation}
	
	\begin{longtable}{|p{3cm}|p{5cm}|}
		\hline
		\textbf{Parameter} & \textbf{Formula} \\
		\hline
		\endfirsthead
		\hline
		\textbf{Parameter} & \textbf{Formula} \\
		\hline
		\endhead
		\(\xi\) (from Higgs) & \(\frac{\lambda_h^{2}v^{2}}{16\pi^{3}m_h^{2}}\) \\
		\hline
		\(\xi\) (geometric) & \(\frac{4}{3} \times 10^{-4}\) \\
		\hline
	\end{longtable}
	
	\section{Magnetic Moment Anomaly from Masses}
	
	\begin{longtable}{|p{2.5cm}|p{4.5cm}|}
		\hline
		\textbf{Particle} & \textbf{Final Formula} \\
		\hline
		\endfirsthead
		\hline
		\textbf{Particle} & \textbf{Final Formula} \\
		\hline
		\endhead
		Muon & \(\Delta a_{\mu} = 251 \times 10^{-11} \times \left(\frac{m_{\mu}}{m_{\mu}}\right)^{2}\) \\
		\hline
		Electron & \(\Delta a_{e} = 251 \times 10^{-11} \times \left(\frac{m_{e}}{m_{\mu}}\right)^{2}\) \\
		\hline
		Tau & \(\Delta a_{\tau} = 251 \times 10^{-11} \times \left(\frac{m_{\tau}}{m_{\mu}}\right)^{2}\) \\
		\hline
	\end{longtable}
	
	\section{Neutrino Masses (with double \(\xi\)-suppression)}
	
	\begin{longtable}{|p{3cm}|p{4cm}|}
		\hline
		\textbf{Particle} & \textbf{Formula} \\
		\hline
		\endfirsthead
		\hline
		\textbf{Particle} & \textbf{Formula} \\
		\hline
		\endhead
		\(\nu_e\) & \(m_{\nu e} = y_{\nu e} \times v \times \xi\) \\
		\hline
		\(\nu_{\mu}\) & \(m_{\nu \mu} = y_{\nu \mu} \times v \times \xi\) \\
		\hline
		\(\nu_{\tau}\) & \(m_{\nu \tau} = y_{\nu \tau} \times v \times \xi\) \\
		\hline
	\end{longtable}
	
	\section{Quark Masses from Yukawa Couplings}
	
	\subsection{Light Quarks}
	
	\textbf{Up-Quark:}
	\begin{align}
		\xi_u &= \frac{4}{3} \times 10^{-4} \times f_u(1,0,1/2) \times C_{\text{Color}} \\
		\xi_u &= \frac{4}{3} \times 10^{-4} \times 1 \times 6 = 8.0 \times 10^{-4} \\
		E_u &= \frac{1}{\xi_u}
	\end{align}
	
	\textbf{Down-Quark:}
	\begin{align}
		\xi_d &= \frac{4}{3} \times 10^{-4} \times f_d(1,0,1/2) \times C_{\text{Color}} \times C_{\text{Isospin}} \\
		\xi_d &= \frac{4}{3} \times 10^{-4} \times 1 \times \frac{25}{2} = \frac{50}{3} \times 10^{-4} \\
		E_d &= \frac{1}{\xi_d}
	\end{align}
	
	\subsection{Heavy Quarks}
	
	\textbf{Charm-Quark:}
	\begin{align}
		y_c &= \frac{8}{9} \times \left(\frac{4}{3} \times 10^{-4}\right)^{2/3} \\
		E_c &= y_c \times v
	\end{align}
	
	\textbf{Bottom-Quark:}
	\begin{align}
		y_b &= \frac{3}{2} \times \left(\frac{4}{3} \times 10^{-4}\right)^{1/2} \\
		E_b &= y_b \times v
	\end{align}
	
	\textbf{Top-Quark:}
	\begin{align}
		y_t &= \frac{1}{28} \times \left(\frac{4}{3} \times 10^{-4}\right)^{-1/3} \\
		E_t &= y_t \times v
	\end{align}
	
	\textbf{Strange-Quark:}
	\begin{align}
		y_s &= \frac{26}{9} \times \left(\frac{4}{3} \times 10^{-4}\right)^{1} \\
		E_s &= y_s \times v
	\end{align}
	
	\section{Length Scale Hierarchy}
	
	\begin{longtable}{|p{3cm}|p{4cm}|}
		\hline
		\textbf{Scale} & \textbf{Formula} \\
		\hline
		\endfirsthead
		\hline
		\textbf{Scale} & \textbf{Formula} \\
		\hline
		\endhead
		\(L_0\) & \(\xi \times L_{\text{Planck}}\) \\
		\hline
		\(L_{\xi}\) & \(\xi\) (nat.) \\
		\hline
		\(L_{\text{Casimir}}\) & \(\sim 100\) \(\mu\)m \\
		\hline
	\end{longtable}
	
	\section{Cosmological Parameters from \(\xi\)}
	
	\begin{longtable}{|p{3cm}|p{4cm}|}
		\hline
		\textbf{Parameter} & \textbf{Formula} \\
		\hline
		\endfirsthead
		\hline
		\textbf{Parameter} & \textbf{Formula} \\
		\hline
		\endhead
		\(T_{\text{CMB}}\) & \(\frac{16}{9}\xi^{2} \times E_{\xi}\) \\
		\hline
		\(H_0\) & \(\xi^{2} \times E_{\text{typ}}\) \\
		\hline
		\(\rho_{\text{vac}}\) & \(\frac{\xi\hbar c}{L_{\xi}^{4}}\) \\
		\hline
	\end{longtable}
	
	\section{Gravitation Theory: Time Field Lagrangian}
	
	\begin{longtable}{|p{4cm}|p{5cm}|}
		\hline
		\textbf{Term} & \textbf{Formula} \\
		\hline
		\endfirsthead
		\hline
		\textbf{Term} & \textbf{Formula} \\
		\hline
		\endhead
		Intrinsic Time Field & \(\mathcal{L}_{\text{grav}} = \frac{1}{2}\partial_{\mu}T\partial^{\mu}T - \frac{1}{2}T^{2} - \frac{\rho}{T}\) \\
		\hline
		Gravitational Potential & \(\Phi(r) = -\frac{GM}{r} + \kappa r\) \\
		\hline
		\(\kappa\)-Parameter & \(\kappa = \frac{\sqrt{2}}{4G^{2}m_{\mu}}\) \\
		\hline
	\end{longtable}
	
	\section{COMPLETELY CORRECTED Derivation Chain}
	
	\begin{center}
		\(\xi\) (3D-Geometry) \(\rightarrow\) \(v_{\text{bare}}\) \(\rightarrow\) \(K_{\text{quantum}}\) \(\rightarrow\) \(v\) \(\rightarrow\) Yukawa \(\rightarrow\) Particle Masses \(\rightarrow\) \(E_0\) \(\rightarrow\) \(\alpha\) \(\rightarrow\) \(\varepsilon_0, \mu_0, e\) \(\rightarrow\) \(c, \hbar\) \(\rightarrow\) \(G\) \(\rightarrow\) Planck Units \(\rightarrow\) Further Physics
	\end{center}
	
	\section{Revolutionary Insight}
	
	ALL natural constants (\(c\), \(\hbar\), \(G\), \(\alpha\), \(\varepsilon_0\), \(\mu_0\), \(e\)) are completely calculable from the single geometric parameter \(\xi = \frac{4}{3} \times 10^{-4}\)!
	
	\subsection{Geometric Origin of All Constants}
	
	\begin{longtable}{|p{3cm}|p{5cm}|}
		\hline
		\textbf{Constant} & \textbf{T0-Origin} \\
		\hline
		\endfirsthead
		\hline
		\textbf{Constant} & \textbf{T0-Origin} \\
		\hline
		\endhead
		\(c\) & Maximum Field Propagation \\
		\hline
		\(\hbar\) & Energy-Frequency Ratio \\
		\hline
		\(G\) & \(\xi^{2}\)-Scaling Effect \\
		\hline
		\(\alpha\) & Geometric EM Coupling \\
		\hline
		\(v\) & Quantum Geometry + Corrections \\
		\hline
	\end{longtable}
	
	The T0-Model is a true Theory of Everything with ZERO free parameters!
	
	\section{IMPORTANT NOTES ON CONVERSIONS AND CORRECTIONS}
	
	\subsection{T0-Foundation: Natural Units}
	
	\textbf{FUNDAMENTAL T0-EQUIVALENCE:}
	\begin{center}
		\(\hbar = c = 1 \rightarrow E = m\) (Energy = Mass)
	\end{center}
	
	\subsection{Unit Conversions}
	
	\begin{longtable}{|p{3cm}|p{3cm}|}
		\hline
		\textbf{Conversion} & \textbf{Factor} \\
		\hline
		\endfirsthead
		\hline
		\textbf{Conversion} & \textbf{Factor} \\
		\hline
		\endhead
		Energy \(\rightarrow\) Mass & \(/c^{2}\) \\
		\hline
		Energy \(\rightarrow\) Frequency & \(/\hbar\) \\
		\hline
		Length \(\rightarrow\) Time & \(\times c\) \\
		\hline
	\end{longtable}
	
	\subsection{Fractal Corrections}
	
	\begin{longtable}{|p{4cm}|p{4cm}|p{5cm}|}
		\hline
		\textbf{Parameter} & \textbf{Fractal Correction} & \textbf{Application} \\
		\hline
		\endfirsthead
		\hline
		\textbf{Parameter} & \textbf{Fractal Correction} & \textbf{Application} \\
		\hline
		\endhead
		\(\alpha\) (Fine Structure) & \(K_{\text{frak}} = 0.9862\) & \(\alpha_{\text{phys}} = \alpha_{\text{bare}} \times K_{\text{frak}}\) \\
		\hline
		Particle Masses & \(K_{\text{geom}} \approx 1.00-1.05\) & Geometric Quantization \\
		\hline
		Coupling Constants & \(K_{\text{topo}}\) & Topological Corrections \\
		\hline
	\end{longtable}
	
	\subsection{Dimensional Consistency}
	
	ALWAYS CHECK:
	\begin{itemize}
		\item All formulas in natural units: \([\xi] = [1]\), \([E] = [m] = [L^{-1}] = [t^{-1}]\)
		\item SI conversions: Correct powers of \(c\) and \(\hbar\)
		\item Dimensional analysis: [Left Side] = [Right Side]
	\end{itemize}
	
	\subsection{Numerical Precision}
	
	\begin{itemize}
		\item \textbf{\(\xi\) exact:} \(\frac{4}{30000}\) (rational form for highest precision)
		\item \textbf{Avoid rounding errors:} Use full decimal expansion
		\item \textbf{Experimental values:} Use current PDG/CODATA references
	\end{itemize}
	
	\section{Complete Project Documentation}
	
	\textbf{GitHub Repository:}\\
	\texttt{https://github.com/jpascher/T0-Time-Mass-Duality}
	
	\subsection{Available PDF Documents}
	
	\begin{itemize}
		\item \textbf{\(\xi\)-Hierarchy Derivation:} \texttt{hierarchy\_En.pdf}
		\item \textbf{Experimental Verification:} \texttt{Elimination\_Of\_Mass\_Dirac\_TableEn.pdf}
		\item \textbf{Muon g-2 Analysis:} \texttt{CompleteMuon\_g-2\_AnalysisEn.pdf}
		\item \textbf{Gravitational Constant:} \texttt{gravitational\_constant\_En.pdf}
		\item \textbf{QFT-Basics:} \texttt{QFT\_En.pdf}
		\item \textbf{Mathematical Structure:} \texttt{Mathematical\_structure\_En.pdf}
		\item \textbf{Time Field Lagrangian:} \texttt{MathTimeMassLagrangeEn.pdf}
		\item \textbf{Summary:} \texttt{Summary\_En.pdf}
	\end{itemize}
	
	\subsection{German Documentation}
	
	\begin{itemize}
		\item \textbf{German (De):} Complete original version with detailed derivations
	\end{itemize}
	
	This table is only an overview - for complete mathematical derivations, detailed proofs and numerical calculations see the PDF documents in the GitHub repository!
	
	\textbf{References:} CODATA 2018, PDG 2022, Fermilab Muon g-2 Collaboration
	
\end{document}