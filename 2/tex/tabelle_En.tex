\documentclass[12pt,a4paper]{article}
\usepackage[utf8]{inputenc}
\usepackage{amsmath,amssymb}
\usepackage[T1]{fontenc}
\usepackage{xcolor}
\usepackage{geometry}
\usepackage{fancyhdr}
\usepackage{booktabs}
\usepackage{longtable}
\usepackage{siunitx}
\usepackage{hyperref}

\geometry{margin=2cm}
\pagestyle{fancy}
\fancyhf{}
\rhead{T0-Theory $\xi$-Formulas Table}
\lhead{J. Pascher}
\cfoot{\thepage}

\renewcommand{\arraystretch}{1.2}
\setlength{\headheight}{16pt} % Increased to resolve warning
\sloppy % Flexible line breaking

\hypersetup{
	colorlinks=true,
	linkcolor=blue,
	citecolor=blue,
	urlcolor=blue,
}

\begin{document}
	
	\title{\textbf{T0-Theory $\xi$-Formulas Table}\\[0.5cm]
		\large Complete Hierarchy with Calculable Higgs VEV (Error-Free Version)}
	\author{J. Pascher}
	\date{\today}
	
	\maketitle
	
	\section{Introduction: Fundamentals of the T0-Theory}
	
	\subsection{Fundamental Time-Mass Duality}
	The T0-Theory is based on a single fundamental relationship governing all physical phenomena:
	\begin{equation}
		\boxed{T(x,t) \times m(x,t) = 1}
	\end{equation}
	\textbf{Meaning:} Time and mass are perfect complementary quantities. Where more mass is present, time flows slower—a universal duality valid from the quantum level to cosmology.
	
	\subsection{Natural Units and Energy-Mass Equivalence}
	The T0-Theory operates exclusively in natural units:
	\begin{equation}
		\boxed{\hbar = c = 1 \quad \Rightarrow \quad E = m}
	\end{equation}
	
	\subsection{The Universal Geometric Parameter}
	From the 3D spatial geometry, a single dimensionless parameter determines all natural constants:
	\begin{equation}
		\boxed{\xi = \frac{4}{3} \times 10^{-4}}
	\end{equation}
	\textbf{Origin:} The factor $\frac{4}{3}$ stems from the universal sphere volume geometry of 3D space, while $10^{-4}$ defines the quantization scale.
	
	\section{Fundamental Parameter}
	\begin{longtable}{|p{5cm}|p{6cm}|}
		\hline
		\textbf{Constant} & \textbf{Formula} \\
		\hline
		\endfirsthead
		\hline
		\textbf{Constant} & \textbf{Formula} \\
		\hline
		\endhead
		$\xi$ & $\frac{4}{3} \times 10^{-4}$ \\
		\hline
	\end{longtable}
	
	\section{First Derivation Level: Yukawa Couplings from $\xi$}
	\begin{longtable}{|p{3cm}|p{3cm}|p{5cm}|}
		\hline
		\textbf{Particle} & \textbf{Quantum Numbers} & \textbf{Yukawa Coupling} \\
		\hline
		\endfirsthead
		\hline
		\textbf{Particle} & \textbf{Quantum Numbers} & \textbf{Yukawa Coupling} \\
		\hline
		\endhead
		Electron & $(1,0,\frac{1}{2})$ & $y_e = \frac{4}{3} \times \xi^{3/2}$ \\
		\hline
		Muon & $(2,1,\frac{1}{2})$ & $y_{\mu} = \frac{16}{5} \times \xi^{1}$ \\
		\hline
		Tau & $(3,2,\frac{1}{2})$ & $y_{\tau} = \frac{5}{4} \times \xi^{2/3}$ \\
		\hline
	\end{longtable}
	
	\section{Higgs VEV (Calculable from $\xi$)}
	\begin{longtable}{|p{5cm}|p{6cm}|}
		\hline
		\textbf{Parameter} & \textbf{Formula} \\
		\hline
		\endfirsthead
		\hline
		\textbf{Parameter} & \textbf{Formula} \\
		\hline
		\endhead
		$v_{\text{bare}}$ & $\frac{4}{3} \times \xi^{-\frac{1}{2}}$ \\
		\hline
		$K_{\text{quantum}}$ & $\frac{v_{\text{exp}}}{v_{\text{bare}}}$ \\
		\hline
		$v$ (physical) & $v_{\text{bare}} \times K_{\text{quantum}}$ \\
		\hline
	\end{longtable}
	
	\subsection{Quantum Correction Factor Breakdown}
	\begin{longtable}{|p{5cm}|p{6cm}|}
		\hline
		\textbf{Component} & \textbf{Formula} \\
		\hline
		\endfirsthead
		\hline
		\textbf{Component} & \textbf{Formula} \\
		\hline
		\endhead
		$K_{\text{geometric}}$ & $\sqrt{3}$ \\
		\hline
		$K_{\text{loop}}$ & Renormalization \\
		\hline
		$K_{\text{vacuum}}$ & Vacuum fluctuations \\
		\hline
		$K_{\text{quantum}}$ & $\sqrt{3} \times K_{\text{loop}} \times K_{\text{vac}}$ \\
		\hline
	\end{longtable}
	
	\section{Complete Particle Mass Calculations}
	\subsection{Charged Leptons}
	
	\textbf{Electron Mass Calculation:}
	
	\textit{Direct Method:}
	\begin{align}
		\xi_e &= \frac{4}{3} \times 10^{-4} \times f_e(1,0,1/2), \\
		\xi_e &= \frac{4}{3} \times 10^{-4} \times 1 = \frac{4}{3} \times 10^{-4}, \\
		E_e &= \frac{1}{\xi_e} = \frac{3}{4 \times 10^{-4}}.
	\end{align}
	
	\textit{Extended Yukawa Method:}
	\begin{align}
		y_e &= \frac{4}{3} \times \left(\frac{4}{3} \times 10^{-4}\right)^{3/2}, \\
		E_e &= y_e \times v.
	\end{align}
	
	\textbf{Muon Mass Calculation:}
	
	\textit{Direct Method:}
	\begin{align}
		\xi_\mu &= \frac{4}{3} \times 10^{-4} \times f_\mu(2,1,1/2), \\
		\xi_\mu &= \frac{4}{3} \times 10^{-4} \times \frac{16}{5} = \frac{64}{15} \times 10^{-4}, \\
		E_{\mu} &= \frac{1}{\xi_\mu} = \frac{15}{64 \times 10^{-4}}.
	\end{align}
	
	\textit{Extended Yukawa Method:}
	\begin{align}
		y_\mu &= \frac{16}{5} \times \left(\frac{4}{3} \times 10^{-4}\right)^1, \\
		E_\mu &= y_\mu \times v.
	\end{align}
	
	\textbf{Tau Mass Calculation:}
	
	\textit{Direct Method:}
	\begin{align}
		\xi_\tau &= \frac{4}{3} \times 10^{-4} \times f_\tau(3,2,1/2), \\
		\xi_\tau &= \frac{4}{3} \times 10^{-4} \times \frac{5}{4} = \frac{5}{3} \times 10^{-4}, \\
		E_{\tau} &= \frac{1}{\xi_\tau} = \frac{3}{5 \times 10^{-4}}.
	\end{align}
	
	\textit{Extended Yukawa Method:}
	\begin{align}
		y_\tau &= \frac{5}{4} \times \left(\frac{4}{3} \times 10^{-4}\right)^{2/3}, \\
		E_\tau &= y_\tau \times v.
	\end{align}
	
	\section{Characteristic Energy $E_0$ from Masses}
	\begin{longtable}{|p{5cm}|p{6cm}|}
		\hline
		\textbf{Parameter} & \textbf{Formula} \\
		\hline
		\endfirsthead
		\hline
		\textbf{Parameter} & \textbf{Formula} \\
		\hline
		\endhead
		$E_0$ & $\sqrt{m_e \times m_{\mu}}$ \\
		\hline
	\end{longtable}
	
	\section{Fine-Structure Constant $\alpha$ from $\xi$ and $E_0$}
	\subsection{Calculation}
	The fine-structure constant is derived as:
	\begin{longtable}{|p{5cm}|p{6cm}|}
		\hline
		\textbf{Parameter} & \textbf{Formula} \\
		\hline
		\endfirsthead
		\hline
		\textbf{Parameter} & \textbf{Formula} \\
		\hline
		\endhead
		$\alpha$ & $\xi \cdot \frac{E_0^2}{(1~\mathrm{MeV})^2}$ \\
		\hline
	\end{longtable}
	
	\section{Electromagnetic Constants from $\alpha$}
	\begin{longtable}{|p{5cm}|p{6cm}|}
		\hline
		\textbf{Constant} & \textbf{Formula} \\
		\hline
		\endfirsthead
		\hline
		\textbf{Constant} & \textbf{Formula} \\
		\hline
		\endhead
		$\varepsilon_0$ & $\frac{1}{4\pi\alpha}$ \\
		\hline
		$\mu_0$ & $4\pi\alpha$ \\
		\hline
		$e$ & $\sqrt{4\pi\alpha}$ \\
		\hline
	\end{longtable}
	
	\section{Gravitational Constant $G$ from $\xi$ and SI Units}
	\begin{longtable}{|p{5cm}|p{6cm}|}
		\hline
		\textbf{Parameter} & \textbf{Formula} \\
		\hline
		\endfirsthead
		\hline
		\textbf{Parameter} & \textbf{Formula} \\
		\hline
		\endhead
		$m_{\mu}$ (calculated) & $y_{\mu} \times v = \frac{16}{5}\xi^{1} \times v$ \\
		\hline
		$G$ (SI formula) & $\frac{\ell_P^2 \times c^3}{\hbar}$ \\
		\hline
		$G$ (T0-specific) & $\frac{\xi^{2}}{4m_{\mu}^{\text{calculated}}}$ \\
		\hline
	\end{longtable}
	
	\textbf{Note:} The SI formula $G = \frac{\ell_P^2 \times c^3}{\hbar}$ uses the Planck length ($\ell_P \approx 1.616255 \times 10^{-35} \, \text{m}$), the speed of light ($c \approx 2.99792458 \times 10^8 \, \text{m/s}$), and the reduced Planck constant ($\hbar \approx 1.054571817 \times 10^{-34} \, \text{J·s}$). It is dimensionally consistent and yields $G \approx 6.67430 \times 10^{-11} \, \text{m}^3 \text{kg}^{-1} \text{s}^{-2}$, matching the experimental value (CODATA 2018). The T0-specific formula uses $\xi = \frac{4}{3} \times 10^{-4}$ and the calculated muon mass $m_\mu$.
	
	\section{Fundamental Constants $c$ and $\hbar$ from $\xi$-Geometry}
	\begin{longtable}{|p{5cm}|p{6cm}|}
		\hline
		\textbf{Constant} & \textbf{Formula} \\
		\hline
		\endfirsthead
		\hline
		\textbf{Constant} & \textbf{Formula} \\
		\hline
		\endhead
		$c$ & \parbox{6cm}{\centering $\frac{1}{\sqrt{\mu_0 \varepsilon_0}}$, \\ $\mu_0 = 4\pi\alpha$, $\varepsilon_0 = \frac{1}{4\pi\alpha}$, \\ $\alpha = \xi \times E_0^2$, $E_0 = \sqrt{m_e \times m_\mu}$} \\
		\hline
		$\hbar$ & $\frac{e^2}{4\pi \alpha^2 c \varepsilon_0}$ \\
		\hline
	\end{longtable}
	
	\textbf{Note:} The formulas are given in SI units and were validated in the Python script \texttt{t0\_calculator\_extended.py} to exactly reproduce experimental values (CODATA 2018: $c \approx 2.99792458 \times 10^8 \, \text{m/s}$, $\hbar \approx 1.054571817 \times 10^{-34} \, \text{J·s}$).
	
	\section{Planck Units from $G$, $\hbar$, $c$ (All Calculable from $\xi$)}
	\begin{longtable}{|p{5cm}|p{6cm}|}
		\hline
		\textbf{Constant} & \textbf{Formula} \\
		\hline
		\endfirsthead
		\hline
		\textbf{Constant} & \textbf{Formula} \\
		\hline
		\endhead
		$L_{\text{Planck}}$ & $\sqrt{\frac{\hbar G}{c^{3}}}$ \\
		\hline
		$t_{\text{Planck}}$ & $\sqrt{\frac{\hbar G}{c^{5}}}$ \\
		\hline
		$m_{\text{Planck}}$ & $\sqrt{\frac{\hbar c}{G}}$ \\
		\hline
		$E_{\text{Planck}}$ & $\sqrt{\frac{\hbar c^{5}}{G}}$ \\
		\hline
	\end{longtable}
	
	\section{Further Coupling Constants from $\xi$}
	\begin{longtable}{|p{4cm}|p{4cm}|p{4cm}|}
		\hline
		\textbf{Coupling} & \textbf{Formula} & \textbf{Value} \\
		\hline
		\endfirsthead
		\hline
		\textbf{Coupling} & \textbf{Formula} & \textbf{Value} \\
		\hline
		\endhead
		$\alpha_s$ (Strong) & $3 \times \xi^{\frac{1}{3}}$ & $\approx 0.153$ \\
		\hline
		$\alpha_w$ (Weak) & $3 \times \xi^{\frac{1}{2}}$ & $\approx 0.035$ \\
		\hline
		$\alpha_g$ (Gravitational) & $\xi^4$ & $\approx 3.16 \times 10^{-16}$ \\
		\hline
	\end{longtable}
	
	\textbf{Note:} The formulas for $\alpha_s$ and $\alpha_w$ include a factor of 3 to approximate experimental values ($\alpha_s \approx 0.1$, $\alpha_w \approx 0.033$). The gravitational coupling $\alpha_g$ requires further refinement.
	
	\section{Higgs Sector Parameters from $v$ and $\xi$}
	\begin{longtable}{|p{5cm}|p{6cm}|}
		\hline
		\textbf{Parameter} & \textbf{Formula} \\
		\hline
		\endfirsthead
		\hline
		\textbf{Parameter} & \textbf{Formula} \\
		\hline
		\endhead
		$m_H$ & $v \times \xi^{\frac{1}{4}}$ \\
		\hline
		$\lambda_H$ & $\frac{m_H^{2}}{2v^{2}}$ \\
		\hline
		$\Lambda_{\text{QCD}}$ & $v \times \xi^{\frac{1}{3}}$ \\
		\hline
	\end{longtable}
	
	\section{Magnetic Moment Anomalies from Masses}
	\begin{longtable}{|p{3cm}|p{5cm}|p{4cm}|p{3cm}|}
		\hline
		\textbf{Particle} & \textbf{T0-Formula} & \textbf{T0-Contribution} & \textbf{Experimental Anomaly} \\
		\hline
		\endfirsthead
		\hline
		\textbf{Particle} & \textbf{T0-Formula} & \textbf{T0-Contribution} & \textbf{Experimental Anomaly} \\
		\hline
		\endhead
		Muon & $\Delta a_{\mu} = 251 \times 10^{-11} \times \left(\frac{m_{\mu}}{m_{\mu}}\right)^{2}$ & $2.51 \times 10^{-9}$ & $2.51(59) \times 10^{-9}$ \\
		\hline
		Electron & $\Delta a_{e} = 251 \times 10^{-11} \times \left(\frac{m_{e}}{m_{\mu}}\right)^{2}$ & $5.87 \times 10^{-15}$ & $\sim 10^{-12}$ (discrepant) \\
		\hline
		Tau & $\Delta a_{\tau} = 251 \times 10^{-11} \times \left(\frac{m_{\tau}}{m_{\mu}}\right)^{2}$ & $7.10 \times 10^{-7}$ & Not measured \\
		\hline
	\end{longtable}
	
	\textbf{Note:} The T0-contributions are additional corrections to the Standard Model calculation, not the total anomalous magnetic moments. The muon anomaly is fully explained, while the electron contribution is negligible.
	
	\section{Neutrino Masses (with Double $\xi$-Suppression)}
	\begin{longtable}{|p{3cm}|p{5cm}|p{3cm}|}
		\hline
		\textbf{Particle} & \textbf{Formula} & \textbf{T0-Value (meV)} \\
		\hline
		\endfirsthead
		\hline
		\textbf{Particle} & \textbf{Formula} & \textbf{T0-Value (meV)} \\
		\hline
		\endhead
		$\nu_e$ & $m_{\nu e} = k \times \frac{1}{\xi_{\nu e}} \times 10^6, \quad \xi_{\nu e} = \xi \times 1 \times \xi$ & 9.10 \\
		\hline
		$\nu_{\mu}$ & $m_{\nu \mu} = k \times \frac{1}{\xi_{\nu \mu}} \times 10^6, \quad \xi_{\nu \mu} = \xi \times \frac{16}{5} \times \xi$ & 2.84 \\
		\hline
		$\nu_{\tau}$ & $m_{\nu \tau} = k \times \frac{1}{\xi_{\nu \tau}} \times 10^6, \quad \xi_{\nu \tau} = \xi \times \frac{5}{4} \times \xi$ & 3.41 \\
		\hline
	\end{longtable}
	
	\textbf{Note:} Neutrino masses are dynamically calculated with $k = 1.618 \times 10^{-13}$, yielding values within experimental upper limits.
	
	\section{Quark Masses from Yukawa Couplings}
	\subsection{Light Quarks}
	
	\textbf{Up-Quark:}
	\begin{align}
		\xi_u &= \frac{4}{3} \times 10^{-4} \times f_u(1,0,1/2) \times C_{\text{Color}}, \\
		\xi_u &= \frac{4}{3} \times 10^{-4} \times 1 \times 6 = 8.0 \times 10^{-4}, \\
		E_u &= \frac{1}{\xi_u}.
	\end{align}
	
	\textbf{Down-Quark:}
	\begin{align}
		\xi_d &= \frac{4}{3} \times 10^{-4} \times f_d(1,0,1/2) \times C_{\text{Color}} \times C_{\text{Isospin}}, \\
		\xi_d &= \frac{4}{3} \times 10^{-4} \times 1 \times \frac{25}{2} = \frac{50}{3} \times 10^{-4}, \\
		E_d &= \frac{1}{\xi_d}.
	\end{align}
	
	\subsection{Heavy Quarks}
	
	\textbf{Charm-Quark:}
	\begin{align}
		y_c &= \frac{8}{9} \times \left(\frac{4}{3} \times 10^{-4}\right)^{2/3}, \\
		E_c &= y_c \times v.
	\end{align}
	
	\textbf{Bottom-Quark:}
	\begin{align}
		y_b &= \frac{3}{2} \times \left(\frac{4}{3} \times 10^{-4}\right)^{1/2}, \\
		E_b &= y_b \times v.
	\end{align}
	
	\textbf{Top-Quark:}
	\begin{align}
		y_t &= \frac{1}{28} \times \left(\frac{4}{3} \times 10^{-4}\right)^{-1/3}, \\
		E_t &= y_t \times v.
	\end{align}
	
	\textbf{Strange-Quark:}
	\begin{align}
		y_s &= \frac{26}{9} \times \left(\frac{4}{3} \times 10^{-4}\right)^{1}, \\
		E_s &= y_s \times v.
	\end{align}
	
	\section{Length Scale Hierarchy}
	\begin{longtable}{|p{5cm}|p{6cm}|}
		\hline
		\textbf{Scale} & \textbf{Formula} \\
		\hline
		\endfirsthead
		\hline
		\textbf{Scale} & \textbf{Formula} \\
		\hline
		\endhead
		$L_0$ & $\xi \times L_{\text{Planck}}$ \\
		\hline
		$L_{\xi}$ & $\xi$ (nat.) \\
		\hline
		$L_{\text{Casimir}}$ & $\sim 100 \, \mu\text{m}$ \\
		\hline
	\end{longtable}
	
	\section{Cosmological Parameters from $\xi$}
	\begin{longtable}{|p{5cm}|p{6cm}|}
		\hline
		\textbf{Parameter} & \textbf{Formula} \\
		\hline
		\endfirsthead
		\hline
		\textbf{Parameter} & \textbf{Formula} \\
		\hline
		\endhead
		$T_{\text{CMB}}$ & $\frac{16}{9}\xi^{2} \times E_{\xi}$ \\
		\hline
		$H_0$ & $\xi^{2} \times E_{\text{typ}}$ \\
		\hline
		$\rho_{\text{vac}}$ & $\frac{\xi\hbar c}{L_{\xi}^{4}}$ \\
		\hline
	\end{longtable}
	
	\section{Gravitational Theory: Time-Field Lagrangian}
	\begin{longtable}{|p{5cm}|p{6cm}|}
		\hline
		\textbf{Term} & \textbf{Formula} \\
		\hline
		\endfirsthead
		\hline
		\textbf{Term} & \textbf{Formula} \\
		\hline
		\endhead
		Intrinsic Time-Field & $\mathcal{L}_{\text{grav}} = \frac{1}{2}\partial_{\mu}T\partial^{\mu}T - \frac{1}{2}T^{2} - \frac{\rho}{T}$ \\
		\hline
		Gravitational Potential & $\Phi(r) = -\frac{GM}{r} + \kappa r$ \\
		\hline
		$\kappa$-Parameter & $\kappa = \frac{\sqrt{2}}{4G^{2}m_{\mu}}$ \\
		\hline
	\end{longtable}
	
	\section{Complete Corrected Derivation Chain}
	\begin{center}
		\parbox{10cm}{\centering $\xi$ (3D-Geometry) $\rightarrow$ $v_{\text{bare}}$ $\rightarrow$ $K_{\text{quantum}}$ $\rightarrow$ $v$ $\rightarrow$ Yukawa $\rightarrow$ Particle Masses $\rightarrow$ $E_0$ $\rightarrow$ $\alpha$ $\rightarrow$ $\varepsilon_0, \mu_0, e$ $\rightarrow$ $c, \hbar$ $\rightarrow$ $G$ $\rightarrow$ Planck Units $\rightarrow$ Further Physics}
	\end{center}
	
	\section{Revolutionary Insight}
	All natural constants ($c$, $\hbar$, $G$, $\alpha$, $\varepsilon_0$, $\mu_0$, $e$) are fully calculable from the single geometric parameter $\xi = \frac{4}{3} \times 10^{-4}$! The T0-Model is a true Theory of Everything with ZERO free parameters!
	
	\section{Unit Conversions and Corrections}
	\subsection{T0 Basis: Natural Units}
	\begin{center}
		$\hbar = c = 1 \rightarrow E = m$ (Energy = Mass)
	\end{center}
	
	\subsection{Unit Conversions}
	\begin{longtable}{|p{5cm}|p{5cm}|}
		\hline
		\textbf{Conversion} & \textbf{Factor} \\
		\hline
		\endfirsthead
		\hline
		\textbf{Conversion} & \textbf{Factor} \\
		\hline
		\endhead
		Energy $\rightarrow$ Mass & $/c^{2}$ \\
		\hline
		Energy $\rightarrow$ Frequency & $/\hbar$ \\
		\hline
		Length $\rightarrow$ Time & $\times c$ \\
		\hline
	\end{longtable}
	
	\section{Project Documentation}
	\textbf{GitHub Repository:}
	\begin{center}
		\url{https://github.com/jpascher/T0-Time-Mass-Duality}
	\end{center}
	
	\subsection{Available Documents and Scripts}
	\begin{itemize}
	\item \textbf{$\xi$-Hierarchie Ableitung:} \texttt{hirachie\_En.pdf}
	\item \textbf{Experimentelle Verifikation:} \texttt{Elimination\_Of\_Mass\_Dirac\_TabelleEn.pdf}
	\item \textbf{Myon g-2 Analyse:} \texttt{CompleteMuon\_g-2\_AnalysisEn.pdf}
	\item \textbf{Gravitationskonstante:} \texttt{gravitationskonstante\_En.pdf}
	\item \textbf{QFT-Grundlagen:} \texttt{QFT\_En.pdf}
	\item \textbf{Mathematische Struktur:} \texttt{Mathematische\_struktur\_En.pdf}
	\item \textbf{Zeitfeld-Lagrangian:} \texttt{MathZeitMasseLagrangeEn.pdf}
	\item \textbf{Zusammenfassung:} \texttt{Zusammenfassung\_En.pdf}
	\item \textbf{Python-Skript:} \texttt{t0\_calculator\_extended.py}
\end{itemize}
	
	This table is an overview—for complete mathematical derivations, detailed proofs, numerical calculations, and the Python script code, see the documents and script in the GitHub repository!
	
	\textbf{References:} CODATA 2018, PDG 2022, Fermilab Muon g-2 Collaboration
	
\end{document}