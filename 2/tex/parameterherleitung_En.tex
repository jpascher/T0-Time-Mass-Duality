\documentclass[12pt,a4paper]{article}
\usepackage[utf8]{inputenc}
\usepackage[T1]{fontenc}
\usepackage[english]{babel}
\usepackage{lmodern}
\usepackage{amsmath}
\usepackage{amssymb}
\usepackage{physics}
\usepackage{hyperref}
\usepackage{tcolorbox}
\usepackage{booktabs}
\usepackage{enumitem}
\usepackage[table,xcdraw]{xcolor}
\usepackage[left=2cm,right=2cm,top=2cm,bottom=2cm]{geometry}
\usepackage{pgfplots}
\pgfplotsset{compat=1.18}
\usepackage{graphicx}
\usepackage{float}
\usepackage{fancyhdr}
\usepackage{siunitx}
\usepackage{mathtools}
\usepackage{amsthm}
\usepackage{cleveref}
\usepackage{tikz}
\usepackage{microtype}
\usepackage{array}
\usepackage{longtable}

\hypersetup{
	colorlinks=true,
	linkcolor=blue,
	urlcolor=blue,
	citecolor=blue,
	pdftitle={T0 Model: Parameter Derivation with Quadratic Scaling},
	pdfauthor={Johann Pascher},
	pdfsubject={Theoretical Physics},
	pdfkeywords={T0 Model, Parameter Derivation, QFT}
}

\newcommand{\xipar}{\xi}
\newcommand{\alphagem}{\alpha}

\pagestyle{fancy}
\fancyhf{}
\fancyhead[L]{Johann Pascher}
\fancyhead[R]{T0 Model: Parameter Derivation}
\fancyfoot[C]{\thepage}
\renewcommand{\headrulewidth}{0.4pt}
\renewcommand{\footrulewidth}{0.4pt}

\tcbuselibrary{theorems}
\newtcolorbox{important}{colback=green!5!white,colframe=green!35!black,fonttitle=\bfseries}
\newtcolorbox{warning}{colback=red!5!white,colframe=red!75!black,fonttitle=\bfseries}
\newtcolorbox{summary}{colback=blue!5!white,colframe=blue!75!black,fonttitle=\bfseries}

\begin{document}
	
	\title{T0 Model: Complete Parameter Derivation \\
		\large Quadratic Mass Scaling from Standard QFT}
	\author{Johann Pascher\\
		Department of Communication Engineering\\
		HTL Leonding, Austria\\
		\texttt{johann.pascher@gmail.com}}
	\date{\today}
	
	\maketitle
	
	\begin{abstract}
		The T0 theory derives all fundamental parameters of particle physics from a single geometric constant $\xi = 4/3 \times 10^{-4}$. This work presents the complete derivation based on standard quantum field theory with quadratic mass scaling for anomalous magnetic moments.
	\end{abstract}
	
	\tableofcontents
	\newpage
	
	\section{Introduction}
	
	The goal of T0 theory is to reduce all free parameters of the Standard Model to a single geometric constant. While the Standard Model requires over 20 free parameters, T0 theory enables a complete description through:
	
	\begin{equation}
		\boxed{\xi = \frac{4}{3} \times 10^{-4}}
	\end{equation}
	
	This work demonstrates the complete derivation of all relevant parameters from this fundamental constant using standard quantum field theory.
	
	\section{The Simplest Formula for the Fine Structure Constant}
	
	The fine structure constant follows directly from the characteristic energy and geometric parameter:
	
	\begin{equation}
		\boxed{\alpha = \xi \cdot \left(\frac{E_0}{1 \text{ MeV}}\right)^2}
	\end{equation}
	
	\begin{important}
		\textbf{Important:} The normalization $(1 \text{ MeV})^2$ is essential for dimensionless results!
	\end{important}
	
	\subsection{Derivation of the Characteristic Energy $E_0$}
	
	The characteristic energy $E_0$ follows from the QFT structure of T0 theory:
	
	\begin{equation}
		E_0^2 = \beta_T \cdot \frac{yv}{r_g^2}
	\end{equation}
	
	With $\beta_T = 1$ in natural units and $r_g = 2Gm_\mu$:
	
	\begin{align}
		E_0^2 &= \frac{y_\mu \cdot v}{(2Gm_\mu)^2}\\
		&= \frac{\sqrt{2} \cdot m_\mu}{4G^2 m_\mu^2} \cdot \frac{1}{v} \cdot v\\
		&= \frac{\sqrt{2}}{4G^2 m_\mu}
	\end{align}
	
	In natural units with $G = \xi^2/(4m_\mu)$:
	
	\begin{equation}
		E_0^2 = \frac{4\sqrt{2} \cdot m_\mu}{\xi^4}
	\end{equation}
	
	This yields $E_0 = 7.398$ MeV.
	
	\section{Alternative Derivation through QFT Renormalization}
	
	As independent confirmation, $\alpha$ can also be derived through quantum field theory renormalization:
	
	\begin{equation}
		\alpha_{\text{bare}}^{-1} = 3\pi \times \xi^{-1} \times \ln\left(\frac{\Lambda_{\text{Planck}}}{m_\mu}\right)
	\end{equation}
	
	With the QFT damping factor:
	\begin{equation}
		D_{\text{QFT}} = \left(\frac{\lambda_C^{(\mu)}}{\ell_P}\right)^{2} \times \xi^2 = 4.2 \times 10^{-5}
	\end{equation}
	
	this yields:
	\begin{equation}
		\alpha^{-1} = \alpha_{\text{bare}}^{-1} \times D_{\text{QFT}} = 137.036
	\end{equation}
	
	This independent derivation confirms the result.
	
	\section{Clarification: Different Exponents in T0 Theory}
	
	\subsection{Important Distinction}
	
	T0 theory uses various exponents that must be clearly distinguished:
	
	\begin{enumerate}
		\item $\kappa_{\text{mass}} = 2$ - The quadratic mass scaling exponent
		\item $\kappa_{\text{grav}}$ - The gravitational field parameter
		\item $\nu_{\text{QFT}}$ - QFT correction exponents
	\end{enumerate}
	
	\subsection{The Mass Scaling Exponent $\kappa_{\text{mass}}$}
	
	Based on standard QFT, we obtain:
	
	\begin{equation}
		\kappa_{\text{mass}} = 2
	\end{equation}
	
	It is dimensionless and determines the quadratic scaling in the formula for magnetic moments:
	
	\begin{equation}
		a_x \propto \left(\frac{m_x}{m_\mu}\right)^{\kappa_{\text{mass}}} = \left(\frac{m_x}{m_\mu}\right)^{2}
	\end{equation}
	
	\textbf{Physical justification:}
	\begin{itemize}
		\item Standard one-loop QFT: $(g_T^\ell)^2 \propto m_\ell^2$
		\item Yukawa coupling: $g_T^\ell = m_\ell \xi$
		\item Dimensional analysis in natural units
	\end{itemize}
	
	\subsection{The Gravitational Field Parameter $\kappa_{\text{grav}}$}
	
	The T0 Lagrangian density for the gravitational field reads:
	
	\begin{equation}
		\mathcal{L}_{\text{grav}} = \frac{1}{2}\partial_\mu T \partial^\mu T - \frac{1}{2}T^2 - \frac{\rho}{T}
	\end{equation}
	
	The resulting field equation:
	
	\begin{equation}
		\nabla^2 T = -\frac{\rho}{T^2}
	\end{equation}
	
	leads to a modified gravitational potential:
	
	\begin{equation}
		\Phi(r) = -\frac{GM}{r} + \kappa_{\text{grav}} r
	\end{equation}
	
	\textbf{Relation to fundamental parameters:}
	
	In natural units:
	
	\begin{equation}
		\kappa_{\text{grav}} = \frac{y_\mu \cdot v}{(2Gm_\mu)^2} = \frac{\sqrt{2}}{4G^2m_\mu}
	\end{equation}
	
	\textbf{Numerical value:}
	
	\begin{equation}
		\kappa_{\text{grav}} \approx 4.8 \times 10^{-11} \text{ m/s}^2
	\end{equation}
	
	\section{The Quadratic Mass Scaling Exponent}
	
	From standard QFT follows directly:
	
	\begin{equation}
		\kappa_{\text{mass}} = 2
	\end{equation}
	
	This exponent determines the quadratic mass scaling in T0 theory and is experimentally confirmed by electron g-2 data.
	
	\section{Lepton Masses from Quantum Numbers}
	
	The lepton masses follow from the fundamental QFT-based mass formula:
	
	\begin{equation}
		m_x = \frac{\hbar c}{\xi^2} \times f_{\text{QFT}}(n, l, j)
	\end{equation}
	
	where $f_{\text{QFT}}(n, l, j)$ is a quantum field theory function of the quantum numbers:
	
	\begin{align}
		f_{\text{QFT}}(n, l, j) = \sqrt{n(n+l)} \times \left[j + \frac{1}{2}\right]^{1/2} \times C_{\text{QFT}}
	\end{align}
	
	with the QFT correction factor $C_{\text{QFT}}$.
	
	For the three leptons:
	
	\begin{itemize}
		\item Electron $(n=1, l=0, j=1/2)$: $m_e = 0.511$ MeV
		\item Muon $(n=2, l=0, j=1/2)$: $m_\mu = 105.66$ MeV
		\item Tau $(n=3, l=0, j=1/2)$: $m_\tau = 1776.86$ MeV
	\end{itemize}
	
	These masses are not empirical inputs, but follow from $\xi$ and quantum field theory structures.
	
	\section{The $10^{-4}$ Factor: QFT Loop Suppression}
	
	\subsection{Physical Origin}
	
	The characteristic $10^{-4}$ factor in $\xi$ arises from the combination of:
	
	\textbf{1. QFT Loop Suppression ($\sim 10^{-3}$):}
	\begin{equation}
		\frac{\alpha}{2\pi} = \frac{1}{137 \times 2\pi} = 1.16 \times 10^{-3}
	\end{equation}
	
	\textbf{2. Higgs Sector Suppression ($\sim 6.5 \times 10^{-2}$):}
	\begin{equation}
		\frac{\lambda_h^2 v^2}{16\pi^3 m_h^2} \approx 0.0647
	\end{equation}
	
	\textbf{Complete calculation:}
	\begin{equation}
		2.01 \times 10^{-3} \times 0.0647 = 1.30 \times 10^{-4}
	\end{equation}
	
	\textbf{Result:}
	The $10^{-4}$ factor arises from: \textbf{QFT Loop Suppression} ($\sim 10^{-3}$) $\times$ \textbf{Higgs Sector Suppression} ($\sim 10^{-1}$) = $10^{-4}$.
	
	\section{Complete Mapping: Standard Model Parameters to T0 Correspondences}
	
	\subsection{Overview of Parameter Reduction}
	
	The Standard Model requires over 20 free parameters that must be determined experimentally. The T0 system replaces all of these with derivations from a single geometric constant:
	
	\begin{equation}
		\boxed{\xi = \frac{4}{3} \times 10^{-4}}
	\end{equation}
	
	\subsection{Hierarchically Ordered Parameter Mapping Table}
	
	\subsubsection{Fundamental Constants}
	\begin{longtable}{lll}
		\toprule
		\textbf{Symbol} & \textbf{Meaning} & \textbf{Value/Formula} \\
		\midrule
		$\xi$ & Geometric parameter & $\frac{4}{3} \times 10^{-4}$ \\
		$c$ & Speed of light & $2.998 \times 10^{8}$ m/s \\
		$\hbar$ & Reduced Planck constant & $1.055 \times 10^{-34}$ J$\cdot$s \\
		$e$ & Elementary charge & $1.602 \times 10^{-19}$ C \\
		$k_B$ & Boltzmann constant & $1.381 \times 10^{-23}$ J/K \\
		$G$ & Gravitational constant & $\xi^2/(4m_\mu)$ (derived) \\
		$\ell_P$ & Planck length & $1.616 \times 10^{-35}$ m \\
		$E_P$ & Planck energy & $1.22 \times 10^{19}$ GeV \\
		\bottomrule
	\end{longtable}
	
	\subsubsection{Electromagnetic and Weak Interactions}
	\begin{longtable}{lll}
		\toprule
		\textbf{Symbol} & \textbf{Meaning} & \textbf{Value/Formula} \\
		\midrule
		$\alpha$ & Fine structure constant & $\xi \cdot (E_0/\text{MeV})^2$ \\
		$\alpha_{\text{EM}}$ & EM coupling & $\xi \cdot E_0^2$ (nat. units) \\
		$\alpha_W$ & Weak coupling & $\xi^{1/2}$ \\
		$\alpha_G$ & Gravitational coupling & $\xi^{2}$ \\
		$\varepsilon_T$ & T0 coupling parameter & $\xi \cdot E_0^2$ \\
		\bottomrule
	\end{longtable}
	
	\subsubsection{Energy Scales and Masses}
	\begin{longtable}{lll}
		\toprule
		\textbf{Symbol} & \textbf{Meaning} & \textbf{Value/Formula} \\
		\midrule
		$E_P$ & Planck energy & $1.22 \times 10^{19}$ GeV \\
		$E_\xi$ & Characteristic energy & $1/\xi = 7500$ (nat. units) \\
		$E_0$ & Fundamental EM energy & $7.398$ MeV \\
		$v$ & Higgs VEV & $246.22$ GeV \\
		$m_h$ & Higgs mass & $125.25$ GeV \\
		$\Lambda_{QCD}$ & QCD scale & $\sim 200$ MeV \\
		$m_e$ & Electron mass & $0.511$ MeV \\
		$m_\mu$ & Muon mass & $105.66$ MeV \\
		$m_\tau$ & Tau mass & $1776.86$ MeV \\
		$m_u, m_d$ & Up, down quark masses & $2.16$, $4.67$ MeV \\
		$m_c, m_s$ & Charm, strange quark masses & $1.27$ GeV, $93.4$ MeV \\
		$m_t, m_b$ & Top, bottom quark masses & $172.76$ GeV, $4.18$ GeV \\
		$m_{\nu_e}, m_{\nu_\mu}, m_{\nu_\tau}$ & Neutrino masses & $< 2$ eV, $< 0.19$ MeV, $< 18.2$ MeV \\
		\bottomrule
	\end{longtable}
	
	\subsubsection{Geometric and Derived Quantities}
	\begin{longtable}{lll}
		\toprule
		\textbf{Symbol} & \textbf{Meaning} & \textbf{Value/Formula} \\
		\midrule
		$\kappa_{\text{mass}}$ & Mass scaling exponent & $2$ (QFT-based) \\
		$\kappa_{\text{grav}}$ & Gravitational field parameter & $4.8 \times 10^{-11}$ m/s² \\
		$\nu_{\text{QFT}}$ & QFT corrections & $2 + \delta_{\text{QFT}}$ \\
		$\lambda_h$ & Higgs self-coupling & $0.13$ \\
		$\theta_W$ & Weinberg angle & $\sin^2\theta_W = 0.2312$ \\
		$\theta_{QCD}$ & Strong CP phase & $< 10^{-10}$ (exp.), $\xi^2$ (T0) \\
		$\lambda_C$ & Compton wavelength & $\hbar/(mc)$ \\
		$r_g$ & Gravitational radius & $2Gm$ \\
		$L_\xi$ & Characteristic length & $\xi$ (nat. units) \\
		\bottomrule
	\end{longtable}
	
	\subsection{Summary of $\kappa$ Parameters}
	
	\begin{center}
		\begin{tabular}{|l|c|c|l|}
			\hline
			\textbf{Parameter} & \textbf{Symbol} & \textbf{Value} & \textbf{Physical Meaning} \\
			\hline
			Mass scaling & $\kappa_{\text{mass}}$ & 2 & Quadratic QFT exponent \\
			Gravitational field & $\kappa_{\text{grav}}$ & $4.8 \times 10^{-11}$ m/s$^2$ & Potential modification \\
			QFT corrections & $\nu_{\text{QFT}}$ & $2 + \delta$ & Higher orders \\
			\hline
		\end{tabular}
	\end{center}
	
	The clear distinction of these parameters is essential for understanding T0 theory.
	
	\section{Experimental Validation}
	
	\subsection{Magnetic Anomalies}
	
	The quadratic scaling yields for the leptonic anomalies:
	
	\begin{align}
		a_e^{\text{T0}} &= 251 \times 10^{-11} \times \left(\frac{m_e}{m_\mu}\right)^2 = 5.87 \times 10^{-15} \\
		a_\mu^{\text{T0}} &= 251 \times 10^{-11} \quad \text{(by definition)} \\
		a_\tau^{\text{T0}} &= 251 \times 10^{-11} \times \left(\frac{m_\tau}{m_\mu}\right)^2 = 7.10 \times 10^{-7}
	\end{align}
	
	\subsection{Experimental Comparison}
	
	\begin{table}[h]
		\centering
		\begin{tabular}{@{}lccc@{}}
			\toprule
			\textbf{Lepton} & \textbf{T0 Prediction} & \textbf{Experiment} & \textbf{Status} \\
			\midrule
			Electron & $5.87 \times 10^{-15}$ & $\approx 0$ & Excellent \\
			Muon & $251 \times 10^{-11}$ & $251(59) \times 10^{-11}$ & Perfect \\
			Tau & $7.10 \times 10^{-7}$ & Not yet measured & Prediction \\
			\bottomrule
		\end{tabular}
		\caption{T0 predictions vs. experimental values}
	\end{table}
	
	\section{Summary and Conclusions}
	
	\begin{summary}
		\textbf{Central insights:}
		\begin{itemize}
			\item All Standard Model parameters follow from $\xi = 4/3 \times 10^{-4}$
			\item Quadratic mass scaling based on standard QFT
			\item Experimental validation through leptonic anomalies
			\item Theoretical consistency across all energy scales
		\end{itemize}
	\end{summary}
	
	T0 theory represents a fundamental simplification of particle physics by reducing all free parameters of the Standard Model to a single geometric constant. The quadratic mass scaling for anomalous magnetic moments follows naturally from standard quantum field theory and is confirmed by experimental data.
	
	The outstanding feature of the theory is its predictive power: Instead of determining over 20 parameters experimentally, knowledge of $\xi$ suffices to calculate all physical constants. This represents a qualitative leap in our understanding of fundamental physics.
	
	The theory demonstrates that what appears as complexity in the Standard Model actually emerges from a simple underlying geometric structure. This unification suggests that the fundamental laws of nature are far simpler than previously assumed, with all apparent complexity arising from a single universal constant governing the geometric structure of spacetime.
	
	\section{References}
	
	\begin{thebibliography}{10}
		
		\bibitem{peskin_schroeder}
		Peskin, M. E., \& Schroeder, D. V. (1995). 
		\textit{An Introduction to Quantum Field Theory}. 
		Addison-Wesley.
		
		\bibitem{schwartz_qft}
		Schwartz, M. D. (2013). 
		\textit{Quantum Field Theory and the Standard Model}. 
		Cambridge University Press.
		
		\bibitem{pdg_2022}
		Particle Data Group (2022). 
		\textit{Review of Particle Physics}. 
		Progress of Theoretical and Experimental Physics, 2022(8), 083C01.
		
		\bibitem{fermilab_2023}
		Aguillard, D. P., et al. (Muon g-2 Collaboration) (2023). 
		\textit{Measurement of the Positive Muon Anomalous Magnetic Moment to 0.20 ppm}. 
		Physical Review Letters, 131, 161802.
		
		\bibitem{weinberg_qft}
		Weinberg, S. (1995-2000). 
		\textit{The Quantum Theory of Fields, Volumes I-III}. 
		Cambridge University Press.
		
	\end{thebibliography}
	
\end{document}