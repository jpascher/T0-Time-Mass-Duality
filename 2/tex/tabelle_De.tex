\documentclass[12pt,a4paper]{article}
\usepackage[utf8]{inputenc}
\usepackage[german]{babel}
\usepackage{amsmath}
\usepackage{amsfonts}
\usepackage{amssymb}
\usepackage{array}
\usepackage{longtable}
\usepackage{booktabs}
\usepackage{xcolor}
\usepackage{geometry}
\usepackage{fancyhdr}
\usepackage{graphicx}

\geometry{margin=2cm}
\pagestyle{fancy}
\fancyhf{}
\rhead{T0-Theorie \(\xi\)-Formeln-Tabelle}
\lhead{J. Pascher}
\cfoot{\thepage}

\renewcommand{\arraystretch}{1.2}

\begin{document}
	
	\title{\textbf{\(\xi\)-Formeln-Tabelle der T0-Theorie}\\
		\large Vollständige Hierarchie mit berechnbarem Higgs-VEV}
	
	\author{J. Pascher}
	\date{\today}
	
	\maketitle
	\maketitle
	
	\section{Einleitung: Grundlagen der T0-Theorie}
	
	\subsection{Fundamentale Zeit-Masse-Dualität}
	
	Die T0-Theorie basiert auf einer einzigen fundamentalen Beziehung, die alle physikalischen Phänomene bestimmt:
	
	\begin{equation}
		\boxed{T(x,t) \times m(x,t) = 1}
	\end{equation}
	
	\textbf{Bedeutung:} Zeit und Masse sind perfekte Komplementärgrößen. Wo mehr Masse vorhanden ist, fließt die Zeit langsamer - eine universelle Dualität, die von Quantenebene bis zur Kosmologie gültig ist.
	
	\subsection{Natürliche Einheiten und Energie-Masse-Äquivalenz}
	
	Die T0-Theorie arbeitet ausschließlich in natürlichen Einheiten:
	
	\begin{equation}
		\boxed{\hbar = c = 1 \quad \Rightarrow \quad E = m}
	\end{equation}
	
	\textbf{Konsequenzen:}
	\begin{itemize}
		\item Alle Teilchenmassen sind gleichzeitig Energien (gemessen in GeV)
		\item Längen und Zeiten haben Dimension [Energie$^{-1}$]
		\item Dimensionslose Kopplungskonstanten bleiben invariant
		\item Vereinfachung aller physikalischen Berechnungen
	\end{itemize}
	
	\subsection{Der universelle geometrische Parameter}
	
	Aus der 3D-Raumgeometrie folgt ein einziger dimensionsloser Parameter, der alle Naturkonstanten bestimmt:
	
	\begin{equation}
		\boxed{\xi = \frac{4}{3} \times 10^{-4}}
	\end{equation}
	
	\textbf{Herkunft:} Der Faktor $\frac{4}{3}$ entstammt der universellen Kugelvolumen-Geometrie des 3D-Raums, während $10^{-4}$ die Quantisierungsskala definiert.
	
	\textbf{Eigenschaft:} ALLE Naturkonstanten ($c$, $\hbar$, $G$, $\alpha$, $v$, alle Teilchenmassen) sind aus diesem einzigen geometrischen Parameter $\xi$ vollständig berechenbar - ohne weitere freie Parameter!
	

	\section{Fundamentaler Parameter}
	
	\begin{longtable}{|p{3cm}|p{4cm}|p{3cm}|p{4cm}|}
		\hline
		\textbf{Konstante} & \textbf{Formel} & \textbf{Wert} & \textbf{Herkunft} \\
		\hline
		\endfirsthead
		\hline
		\textbf{Konstante} & \textbf{Formel} & \textbf{Wert} & \textbf{Herkunft} \\
		\hline
		\endhead
		\(\xi\) & \(\frac{4}{3} \times 10^{-4}\) & \(1.333 \times 10^{-4}\) & 3D-Raumgeometrie \\
		\hline
	\end{longtable}
	
	\section{Erste Ableitungsstufe: Yukawa-Kopplungen aus \(\xi\)}
	
	\begin{longtable}{|p{2.5cm}|p{3cm}|p{4cm}|p{4cm}|}
		\hline
		\textbf{Teilchen} & \textbf{Quantenzahlen} & \textbf{Yukawa-Kopplung} & \textbf{Bemerkung} \\
		\hline
		\endfirsthead
		\hline
		\textbf{Teilchen} & \textbf{Quantenzahlen} & \textbf{Yukawa-Kopplung} & \textbf{Bemerkung} \\
		\hline
		\endhead
		Elektron & \((1,0,\frac{1}{2})\) & \(y_e = \frac{2}{3}\xi^{\frac{5}{2}}\) & Geometrisch abgeleitet \\
		\hline
		Myon & \((2,1,\frac{1}{2})\) & \(y_{\mu} = \frac{8}{5}\xi^{2}\) & Geometrisch abgeleitet \\
		\hline
		Tau & \((3,2,\frac{1}{2})\) & \(y_{\tau} = \frac{5}{4}\xi^{\frac{3}{2}}\) & Geometrisch abgeleitet \\
		\hline
	\end{longtable}
	
	\section{Higgs-VEV (BERECHENBAR aus \(\xi\))}
	
	\begin{longtable}{|p{3cm}|p{4cm}|p{3cm}|p{4cm}|}
		\hline
		\textbf{Parameter} & \textbf{Formel} & \textbf{Wert} & \textbf{Status} \\
		\hline
		\endfirsthead
		\hline
		\textbf{Parameter} & \textbf{Formel} & \textbf{Wert} & \textbf{Status} \\
		\hline
		\endhead
		\(v_{\text{bare}}\) & \(\frac{4}{3} \times \xi^{-\frac{1}{2}}\) & 115.5 (nat.) / 141.0 GeV & Aus \(\xi\) berechnet \\
		\hline
		\(K_{\text{quantum}}\) & \(\frac{v_{\text{exp}}}{v_{\text{bare}}}\) & 1.747 & Quantenkorrekturfaktor \\
		\hline
		\(v\) (physikalisch) & \(v_{\text{bare}} \times K_{\text{quantum}}\) & 246.22 GeV & Vollständig berechenbar \\
		\hline
	\end{longtable}
	
	\subsection{Quantenkorrekturfaktor-Aufschlüsselung}
	
	\begin{longtable}{|p{3cm}|p{4cm}|p{3cm}|p{4cm}|}
		\hline
		\textbf{Komponente} & \textbf{Formel} & \textbf{Wert} & \textbf{Bedeutung} \\
		\hline
		\endfirsthead
		\hline
		\textbf{Komponente} & \textbf{Formel} & \textbf{Wert} & \textbf{Bedeutung} \\
		\hline
		\endhead
		\(K_{\text{geometric}}\) & \(\sqrt{3}\) & 1.732 & 3D-Geometrie \\
		\hline
		\(K_{\text{loop}}\) & Renormierung & \(\sim 1.01\) & Loop-Korrekturen \\
		\hline
		\(K_{\text{vacuum}}\) & Vakuumfluktuationen & \(\sim 1.00\) & Quantenfluktuationen \\
		\hline
		\(K_{\text{quantum}}\) & \(\sqrt{3} \times K_{\text{loop}} \times K_{\text{vac}}\) & 1.747 & Gesamtkorrektur \\
		\hline
	\end{longtable}
	
	\section{Teilchenmassen aus Yukawa \(\times\) v}
	
	\begin{longtable}{|p{3cm}|p{4cm}|p{3cm}|p{3cm}|}
		\hline
		\textbf{Teilchen} & \textbf{Massenformel} & \textbf{Wert} & \textbf{Experimentell} \\
		\hline
		\endfirsthead
		\hline
		\textbf{Teilchen} & \textbf{Massenformel} & \textbf{Wert} & \textbf{Experimentell} \\
		\hline
		\endhead
		Elektron & \(m_e = y_e \times v\) & 0.511 MeV & 0.511 MeV \\
		\hline
		Myon & \(m_{\mu} = y_{\mu} \times v\) & 105.66 MeV & 105.66 MeV \\
		\hline
		Tau & \(m_{\tau} = y_{\tau} \times v\) & 1776.86 MeV & 1776.86 MeV \\
		\hline
	\end{longtable}
	
	\section{Charakteristische Energie \(E_0\) aus Massen}
	
	\begin{longtable}{|p{3cm}|p{4cm}|p{3cm}|p{4cm}|}
		\hline
		\textbf{Parameter} & \textbf{Formel} & \textbf{Wert} & \textbf{Bedeutung} \\
		\hline
		\endfirsthead
		\hline
		\textbf{Parameter} & \textbf{Formel} & \textbf{Wert} & \textbf{Bedeutung} \\
		\hline
		\endhead
		\(E_0\) & \(\sqrt{m_e \times m_{\mu}}\) & 7.35 MeV & EM-charakteristische Energie \\
		\hline
	\end{longtable}
	
	\section{Feinstrukturkonstante \(\alpha\) aus \(\xi\) und \(E_0\)}
	
	\begin{longtable}{|p{3cm}|p{4cm}|p{3cm}|p{3cm}|}
		\hline
		\textbf{Konstante} & \textbf{Formel} & \textbf{Wert} & \textbf{Korrektur} \\
		\hline
		\endfirsthead
		\hline
		\textbf{Konstante} & \textbf{Formel} & \textbf{Wert} & \textbf{Korrektur} \\
		\hline
		\endhead
		\(\alpha\) (nackt) & \(\xi \times E_0^2\) & \(7.20 \times 10^{-3}\) & Vor QFT \\
		\hline
		\(K_{\text{frak}}\) & Fraktale Korrektur & 0.9862 & Geometrische Korrektur \\
		\hline
		\(\alpha\) (physikalisch) & \(\alpha_{\text{nackt}} \times K_{\text{frak}}\) & \(\frac{1}{137.036}\) & Mit QFT-Korrektur \\
		\hline
	\end{longtable}
	
	\section{Elektromagnetische Konstanten aus \(\alpha\)}
	
	\begin{longtable}{|p{3cm}|p{4cm}|p{4cm}|p{3cm}|}
		\hline
		\textbf{Konstante} & \textbf{Formel} & \textbf{Wert} & \textbf{Ableitung} \\
		\hline
		\endfirsthead
		\hline
		\textbf{Konstante} & \textbf{Formel} & \textbf{Wert} & \textbf{Ableitung} \\
		\hline
		\endhead
		\(\varepsilon_0\) & \(\frac{1}{4\pi\alpha}\) & \(8.854 \times 10^{-12}\) F/m & Aus \(\alpha\) \\
		\hline
		\(\mu_0\) & \(4\pi\alpha\) & \(1.257 \times 10^{-6}\) H/m & Aus \(\alpha\) \\
		\hline
		\(e\) & \(\sqrt{4\pi\alpha}\) & \(1.602 \times 10^{-19}\) C & Aus \(\alpha\) \\
		\hline
	\end{longtable}
	
	\section{Gravitationskonstante G aus \(\xi\) und berechneter \(\mu\)-Masse}
	
	\begin{longtable}{|p{3cm}|p{5cm}|p{4cm}|p{3cm}|}
		\hline
		\textbf{Parameter} & \textbf{Formel} & \textbf{Wert} & \textbf{Beschreibung} \\
		\hline
		\endfirsthead
		\hline
		\textbf{Parameter} & \textbf{Formel} & \textbf{Wert} & \textbf{Beschreibung} \\
		\hline
		\endhead
		\(m_{\mu}\) (berechnet) & \(y_{\mu} \times v = \frac{8}{5}\xi^{2} \times v\) & 105.66 MeV & Aus \(\xi\) und \(v\) berechnet \\
		\hline
		\(G\) & \(\frac{\xi^{2}}{4m_{\mu}^{\text{berechnet}}}\) & \(6.674 \times 10^{-11}\) m\(^3\)/(kg\(\cdot\)s\(^2\)) & Verwendet berechnete Myon-Masse \\
		\hline
	\end{longtable}
	
	\section{Fundamentale Konstanten c und \(\hbar\) aus \(\xi\)-Geometrie}
	
	\begin{longtable}{|p{3cm}|p{5cm}|p{4cm}|p{3cm}|}
		\hline
		\textbf{Konstante} & \textbf{Formel} & \textbf{Wert} & \textbf{Herkunft} \\
		\hline
		\endfirsthead
		\hline
		\textbf{Konstante} & \textbf{Formel} & \textbf{Wert} & \textbf{Herkunft} \\
		\hline
		\endhead
		\(c\) & Maximale Feldausbreitung \(= \frac{1}{\xi^{\frac{1}{4}}}\) & \(2.998 \times 10^{8}\) m/s & Geometrische Feldstruktur \\
		\hline
		\(\hbar\) & Energie-Frequenz-Verhältnis \(= \xi \times E_0\) & \(1.055 \times 10^{-34}\) J\(\cdot\)s & Quantengeometrie \\
		\hline
	\end{longtable}
	
	\section{Planck-Einheiten aus G, \(\hbar\), c (alle aus \(\xi\) berechenbar)}
	
	\begin{longtable}{|p{3cm}|p{4cm}|p{4cm}|p{3cm}|}
		\hline
		\textbf{Konstante} & \textbf{Formel} & \textbf{Wert} & \textbf{Basis} \\
		\hline
		\endfirsthead
		\hline
		\textbf{Konstante} & \textbf{Formel} & \textbf{Wert} & \textbf{Basis} \\
		\hline
		\endhead
		\(L_{\text{Planck}}\) & \(\sqrt{\frac{\hbar G}{c^{3}}}\) & \(1.616 \times 10^{-35}\) m & Alle Komponenten aus \(\xi\) \\
		\hline
		\(t_{\text{Planck}}\) & \(\sqrt{\frac{\hbar G}{c^{5}}}\) & \(5.391 \times 10^{-44}\) s & Alle Komponenten aus \(\xi\) \\
		\hline
		\(m_{\text{Planck}}\) & \(\sqrt{\frac{\hbar c}{G}}\) & \(2.176 \times 10^{-8}\) kg & Alle Komponenten aus \(\xi\) \\
		\hline
		\(E_{\text{Planck}}\) & \(\sqrt{\frac{\hbar c^{5}}{G}}\) & \(1.22 \times 10^{19}\) GeV & Alle Komponenten aus \(\xi\) \\
		\hline
	\end{longtable}
	
	\section{Weitere Kopplungskonstanten aus \(\xi\)}
	
	\begin{longtable}{|p{3cm}|p{3cm}|p{3cm}|p{5cm}|}
		\hline
		\textbf{Kopplung} & \textbf{Formel} & \textbf{Wert} & \textbf{Beschreibung} \\
		\hline
		\endfirsthead
		\hline
		\textbf{Kopplung} & \textbf{Formel} & \textbf{Wert} & \textbf{Beschreibung} \\
		\hline
		\endhead
		\(\alpha_s\) (Stark) & \(\xi^{-\frac{1}{3}}\) & 9.65 & Starke Wechselwirkung \\
		\hline
		\(\alpha_w\) (Schwach) & \(\xi^{\frac{1}{2}}\) & \(1.15 \times 10^{-2}\) & Schwache Wechselwirkung \\
		\hline
		\(\alpha_g\) (Gravitation) & \(\xi^{2}\) & \(1.78 \times 10^{-8}\) & Gravitationskopplung \\
		\hline
	\end{longtable}
	
	\section{Higgs-Sektor-Parameter aus v und \(\xi\)}
	
	\begin{longtable}{|p{3cm}|p{4cm}|p{3cm}|p{4cm}|}
		\hline
		\textbf{Parameter} & \textbf{Formel} & \textbf{Wert} & \textbf{Beschreibung} \\
		\hline
		\endfirsthead
		\hline
		\textbf{Parameter} & \textbf{Formel} & \textbf{Wert} & \textbf{Beschreibung} \\
		\hline
		\endhead
		\(m_H\) & \(v \times \xi^{\frac{1}{4}}\) & 125 GeV & Higgs-Masse \\
		\hline
		\(\lambda_H\) & \(\frac{m_H^{2}}{2v^{2}}\) & 0.13 & Higgs-Selbstkopplung \\
		\hline
		\(\Lambda_{\text{QCD}}\) & \(v \times \xi^{\frac{1}{3}}\) & \(\sim 200\) MeV & QCD-Skala \\
		\hline
	\end{longtable}
	
	\subsection{Alternative Higgs-\(\xi\)-Herleitung}
	
	\begin{longtable}{|p{3cm}|p{5cm}|p{3cm}|p{3cm}|}
		\hline
		\textbf{Parameter} & \textbf{Formel} & \textbf{Wert} & \textbf{Vergleich} \\
		\hline
		\endfirsthead
		\hline
		\textbf{Parameter} & \textbf{Formel} & \textbf{Wert} & \textbf{Vergleich} \\
		\hline
		\endhead
		\(\xi\) (aus Higgs) & \(\frac{\lambda_h^{2}v^{2}}{16\pi^{3}m_h^{2}}\) & \(1.318 \times 10^{-4}\) & 99\% Übereinstimmung \\
		\hline
		\(\xi\) (geometrisch) & \(\frac{4}{3} \times 10^{-4}\) & \(1.333 \times 10^{-4}\) & Referenz \\
		\hline
	\end{longtable}
	
	\section{Magnetisches Moment-Anomaly aus Massen}
	
	\begin{longtable}{|p{2.5cm}|p{4.5cm}|p{3cm}|p{3cm}|p{2cm}|}
		\hline
		\textbf{Teilchen} & \textbf{Endformel} & \textbf{T0-Berechnung} & \textbf{Experimentell} & \textbf{Status} \\
		\hline
		\endfirsthead
		\hline
		\textbf{Teilchen} & \textbf{Endformel} & \textbf{T0-Berechnung} & \textbf{Experimentell} & \textbf{Status} \\
		\hline
		\endhead
		Myon & \(\Delta a_{\mu} = 251 \times 10^{-11} \times \left(\frac{m_{\mu}}{m_{\mu}}\right)^{2}\) & \(251 \times 10^{-11}\) & \(251(59) \times 10^{-11}\) & BESTÄTIGT (0.10\(\sigma\)) \\
		\hline
		Elektron & \(\Delta a_{e} = 251 \times 10^{-11} \times \left(\frac{m_{e}}{m_{\mu}}\right)^{2}\) & \(5.87 \times 10^{-15}\) & \(\sim 0\) (zu klein) & BESTÄTIGT \\
		\hline
		Tau & \(\Delta a_{\tau} = 251 \times 10^{-11} \times \left(\frac{m_{\tau}}{m_{\mu}}\right)^{2}\) & \(7.10 \times 10^{-7}\) & Noch nicht messbar & Vorhersage testbar \\
		\hline
	\end{longtable}
	
	\section{Neutrino-Massen (mit doppelter \(\xi\)-Unterdrückung)}
	
	\begin{longtable}{|p{3cm}|p{4cm}|p{3cm}|p{3cm}|}
		\hline
		\textbf{Teilchen} & \textbf{Formel} & \textbf{Vorhersage} & \textbf{Status} \\
		\hline
		\endfirsthead
		\hline
		\textbf{Teilchen} & \textbf{Formel} & \textbf{Vorhersage} & \textbf{Status} \\
		\hline
		\endhead
		\(\nu_e\) & \(m_{\nu e} = y_{\nu e} \times v \times \xi\) & \(\sim\) meV & Testbar \\
		\hline
		\(\nu_{\mu}\) & \(m_{\nu \mu} = y_{\nu \mu} \times v \times \xi\) & \(\sim 10\) meV & Testbar \\
		\hline
		\(\nu_{\tau}\) & \(m_{\nu \tau} = y_{\nu \tau} \times v \times \xi\) & \(\sim 100\) meV & Testbar \\
		\hline
	\end{longtable}
	
	\section{Quark-Massen aus Yukawa-Kopplungen}
	
	\begin{longtable}{|p{2.5cm}|p{3cm}|p{3cm}|p{5cm}|}
		\hline
		\textbf{Teilchen} & \textbf{\(r_i\) Koeffizient} & \textbf{Exponent \(p_i\)} & \textbf{Masse-Formel} \\
		\hline
		\endfirsthead
		\hline
		\textbf{Teilchen} & \textbf{\(r_i\) Koeffizient} & \textbf{Exponent \(p_i\)} & \textbf{Masse-Formel} \\
		\hline
		\endhead
		Up & \(r_u = 6\) & \(p_u = \frac{3}{2}\) & \(m_u = 6\xi^{\frac{3}{2}} \times v\) \\
		\hline
		Down & \(r_d = \frac{25}{2}\) & \(p_d = \frac{3}{2}\) & \(m_d = \frac{25}{2}\xi^{\frac{3}{2}} \times v\) \\
		\hline
		Charm & \(r_c = 2\) & \(p_c = \frac{2}{3}\) & \(m_c = 2\xi^{\frac{2}{3}} \times v\) \\
		\hline
		Strange & \(r_s = \frac{26}{9}\) & \(p_s = 1\) & \(m_s = \frac{26}{9}\xi^{1} \times v\) \\
		\hline
		Top & \(r_t = \frac{1}{28}\) & \(p_t = -\frac{1}{3}\) & \(m_t = \frac{1}{28}\xi^{-\frac{1}{3}} \times v\) \\
		\hline
		Bottom & \(r_b = \frac{3}{2}\) & \(p_b = \frac{1}{2}\) & \(m_b = \frac{3}{2}\xi^{\frac{1}{2}} \times v\) \\
		\hline
	\end{longtable}
	
	\section{Längenskalen-Hierarchie}
	
	\begin{longtable}{|p{3cm}|p{4cm}|p{4cm}|p{4cm}|}
		\hline
		\textbf{Skala} & \textbf{Formel} & \textbf{Wert} & \textbf{Bedeutung} \\
		\hline
		\endfirsthead
		\hline
		\textbf{Skala} & \textbf{Formel} & \textbf{Wert} & \textbf{Bedeutung} \\
		\hline
		\endhead
		\(L_0\) & \(\xi \times L_{\text{Planck}}\) & \(2.155 \times 10^{-39}\) m & Sub-Planck Minimum \\
		\hline
		\(L_{\xi}\) & \(\xi\) (nat.) & \(1.333 \times 10^{-4}\) (nat.) & Charakteristische Länge \\
		\hline
		\(L_{\text{Casimir}}\) & \(\sim 100\) \(\mu\)m & \(10^{-4}\) m & Casimir-charakteristisch \\
		\hline
	\end{longtable}
	
	\section{Kosmologische Parameter aus \(\xi\)}
	
	\begin{longtable}{|p{3cm}|p{4cm}|p{4cm}|p{4cm}|}
		\hline
		\textbf{Parameter} & \textbf{Formel} & \textbf{Wert} & \textbf{Beschreibung} \\
		\hline
		\endfirsthead
		\hline
		\textbf{Parameter} & \textbf{Formel} & \textbf{Wert} & \textbf{Beschreibung} \\
		\hline
		\endhead
		\(T_{\text{CMB}}\) & \(\frac{16}{9}\xi^{2} \times E_{\xi}\) & 2.725 K & CMB-Temperatur \\
		\hline
		\(H_0\) & \(\xi^{2} \times E_{\text{typ}}\) & 67.4 km/s/Mpc & Hubble-Parameter \\
		\hline
		\(\rho_{\text{vac}}\) & \(\frac{\xi\hbar c}{L_{\xi}^{4}}\) & \(4.17 \times 10^{-14}\) J/m\(^3\) & Vakuumenergiedichte \\
		\hline
	\end{longtable}
	
	\section{Gravitationstheorie: Zeitfeld-Lagrangian}
	
	\begin{longtable}{|p{4cm}|p{5cm}|p{5cm}|}
		\hline
		\textbf{Term} & \textbf{Formel} & \textbf{Beschreibung} \\
		\hline
		\endfirsthead
		\hline
		\textbf{Term} & \textbf{Formel} & \textbf{Beschreibung} \\
		\hline
		\endhead
		Intrinsisches Zeitfeld & \(\mathcal{L}_{\text{grav}} = \frac{1}{2}\partial_{\mu}T\partial^{\mu}T - \frac{1}{2}T^{2} - \frac{\rho}{T}\) & Gravitations-Lagrangian \\
		\hline
		Gravitationspotential & \(\Phi(r) = -\frac{GM}{r} + \kappa r\) & Modifizierte Gravitation \\
		\hline
		\(\kappa\)-Parameter & \(\kappa = \frac{\sqrt{2}}{4G^{2}m_{\mu}}\) & Linearer Gravitationsterm \\
		\hline
	\end{longtable}
	
	\section{Experimentelle Verhältnisse (Renormierungsinvariant)}
	
	\begin{longtable}{|p{3cm}|p{3cm}|p{3cm}|p{3cm}|}
		\hline
		\textbf{Verhältnis} & \textbf{T0-Vorhersage} & \textbf{Experimentell} & \textbf{Übereinstimmung} \\
		\hline
		\endfirsthead
		\hline
		\textbf{Verhältnis} & \textbf{T0-Vorhersage} & \textbf{Experimentell} & \textbf{Übereinstimmung} \\
		\hline
		\endhead
		\(\frac{m_{\mu}}{m_e}\) & 207.8 & 206.77 & 99.5\% \\
		\hline
		\(\frac{m_{\tau}}{m_{\mu}}\) & 16.8 & 16.82 & 99.9\% \\
		\hline
		\(\frac{\alpha_g}{\alpha}\) & \(1.33 \times 10^{-4}\) & \(1.24 \times 10^{-4}\) & 93\% \\
		\hline
	\end{longtable}
	
	\section{VOLLSTÄNDIG KORRIGIERTE Ableitungskette}
	
	\begin{center}
		\(\xi\) (3D-Geometrie) \(\rightarrow\) \(v_{\text{bare}}\) \(\rightarrow\) \(K_{\text{quantum}}\) \(\rightarrow\) \(v\) \(\rightarrow\) Yukawa \(\rightarrow\) Teilchenmassen \(\rightarrow\) \(E_0\) \(\rightarrow\) \(\alpha\) \(\rightarrow\) \(\varepsilon_0, \mu_0, e\) \(\rightarrow\) \(c, \hbar\) \(\rightarrow\) \(G\) \(\rightarrow\) Planck-Einheiten \(\rightarrow\) Weitere Physik
	\end{center}
	
	\section{Revolutionäre Erkenntnis}
	
	ALLE Naturkonstanten (\(c\), \(\hbar\), \(G\), \(\alpha\), \(\varepsilon_0\), \(\mu_0\), \(e\)) sind aus dem einzigen geometrischen Parameter \(\xi = \frac{4}{3} \times 10^{-4}\) vollständig berechenbar!
	
	\subsection{Geometrischer Ursprung aller Konstanten}
	
	\begin{longtable}{|p{3cm}|p{5cm}|p{4cm}|}
		\hline
		\textbf{Konstante} & \textbf{T0-Ursprung} & \textbf{Experimenteller Status} \\
		\hline
		\endfirsthead
		\hline
		\textbf{Konstante} & \textbf{T0-Ursprung} & \textbf{Experimenteller Status} \\
		\hline
		\endhead
		\(c\) & Maximale Feldausbreitung & \(\checkmark\) Bestätigt \\
		\hline
		\(\hbar\) & Energie-Frequenz-Verhältnis & \(\checkmark\) Bestätigt \\
		\hline
		\(G\) & \(\xi^{2}\)-Skalierungseffekt & \(\checkmark\) Bestätigt \\
		\hline
		\(\alpha\) & Geometrische EM-Kopplung & \(\checkmark\) Bestätigt \\
		\hline
		\(v\) & Quantengeometrie + Korrekturen & \(\checkmark\) Bestätigt \\
		\hline
	\end{longtable}
	
	Das T0-Modell ist eine echte Theory of Everything mit NULL freien Parametern!
	
	\section{WICHTIGE HINWEISE ZU UMRECHNUNGEN UND KORREKTUREN}
	
	\subsection{T0-Grundlage: Natürliche Einheiten}
	
	\textbf{FUNDAMENTALE T0-GLEICHSETZUNG:}
	\begin{center}
		\(\hbar = c = 1 \rightarrow E = m\) (Energie = Masse)
	\end{center}
	
	\textbf{Bedeutung:}
	\begin{itemize}
		\item Alle Teilchenmassen sind gleichzeitig Energien
		\item Längen und Zeiten haben Dimension \([E^{-1}]\)
		\item \(\xi\) ist pure dimensionslose Geometrie
		\item Vereinfachung aller T0-Formeln durch \(E=m\)
	\end{itemize}
	
	\subsection{Einheitenumrechnungen}
	
	ACHTUNG: Beim Umrechnen von natürlichen Einheiten (\(\hbar = c = 1\)) auf SI-Einheiten müssen folgende Faktoren beachtet werden:
	
	\begin{longtable}{|p{3cm}|p{3cm}|p{6cm}|}
		\hline
		\textbf{Umrechnung} & \textbf{Faktor} & \textbf{Beispiel} \\
		\hline
		\endfirsthead
		\hline
		\textbf{Umrechnung} & \textbf{Faktor} & \textbf{Beispiel} \\
		\hline
		\endhead
		Energie \(\rightarrow\) Masse & \(/c^{2}\) & \(E[\text{J}] = m[\text{kg}] \times c^{2}\) \\
		\hline
		Energie \(\rightarrow\) Frequenz & \(/\hbar\) & \(E[\text{J}] = \hbar \times \omega[\text{Hz}]\) \\
		\hline
		Länge \(\rightarrow\) Zeit & \(\times c\) & \(t[\text{s}] = L[\text{m}]/c\) \\
		\hline
		Planck-Einheiten \(\rightarrow\) SI & Spezifische Faktoren & Siehe CODATA 2018 \\
		\hline
	\end{longtable}
	
	\subsection{Fraktale Korrekturen}
	
	Die T0-Theorie verwendet fraktale Geometriekorrekturen für höchste Präzision:
	
	\begin{longtable}{|p{4cm}|p{4cm}|p{5cm}|}
		\hline
		\textbf{Parameter} & \textbf{Fraktale Korrektur} & \textbf{Anwendung} \\
		\hline
		\endfirsthead
		\hline
		\textbf{Parameter} & \textbf{Fraktale Korrektur} & \textbf{Anwendung} \\
		\hline
		\endhead
		\(\alpha\) (Feinstruktur) & \(K_{\text{frak}} = 0.9862\) & \(\alpha_{\text{phys}} = \alpha_{\text{nackt}} \times K_{\text{frak}}\) \\
		\hline
		Teilchenmassen & \(K_{\text{geom}} \approx 1.00-1.05\) & Geometrische Quantisierung \\
		\hline
		Kopplungskonstanten & \(K_{\text{topo}}\) & Topologische Korrekturen \\
		\hline
	\end{longtable}
	
	\subsection{Dimensionale Konsistenz}
	
	PRÜFEN SIE IMMER:
	\begin{itemize}
		\item Alle Formeln in natürlichen Einheiten: \([\xi] = [1]\), \([E] = [m] = [L^{-1}] = [t^{-1}]\)
		\item SI-Umrechnungen: Korrekte Potenzen von \(c\) und \(\hbar\)
		\item Dimensionsanalyse: [Linke Seite] = [Rechte Seite]
	\end{itemize}
	
	\subsection{Numerische Präzision}
	
	\begin{itemize}
		\item \textbf{\(\xi\) exakt:} \(\frac{4}{30000}\) (rationale Form für höchste Präzision)
		\item \textbf{Rundungsfehler vermeiden:} Vollständige Dezimalentwicklung verwenden
		\item \textbf{Experimentelle Werte:} Aktuelle PDG/CODATA-Referenzen nutzen
	\end{itemize}
	
	\section{Vollständige Projektdokumentation}
	
	\textbf{GitHub Repository:}\\
	\texttt{https://github.com/jpascher/T0-Time-Mass-Duality}
	
	\subsection{Verfügbare PDF-Dokumente}
	
	\begin{itemize}
		\item \textbf{\(\xi\)-Hierarchie Ableitung:} \texttt{hirachie\_De.pdf}
		\item \textbf{Experimentelle Verifikation:} \texttt{Elimination\_Of\_Mass\_Dirac\_TabelleDe.pdf}
		\item \textbf{Myon g-2 Analyse:} \texttt{CompleteMuon\_g-2\_AnalysisDe.pdf}
		\item \textbf{Gravitationskonstante:} \texttt{gravitationskonstante\_De.pdf}
		\item \textbf{QFT-Grundlagen:} \texttt{QFT\_De.pdf}
		\item \textbf{Mathematische Struktur:} \texttt{Mathematische\_struktur\_De.pdf}
		\item \textbf{Zeitfeld-Lagrangian:} \texttt{MathZeitMasseLagrangeDe.pdf}
		\item \textbf{Zusammenfassung:} \texttt{Zusammenfassung\_De.pdf}
	\end{itemize}
	
	\subsection{Deutsche Dokumentation}
	
	\begin{itemize}
		\item \textbf{Deutsch (De):} Vollständige Originalversion mit detaillierten Herleitungen
	\end{itemize}
	
	Diese Tabelle ist nur eine Übersicht - für vollständige mathematische Herleitungen, detaillierte Beweise und numerische Berechnungen siehe die PDF-Dokumente im GitHub-Repository!
	
	\textbf{Referenzen:} CODATA 2018, PDG 2022, Fermilab Myon g-2 Kollaboration
	
\end{document}