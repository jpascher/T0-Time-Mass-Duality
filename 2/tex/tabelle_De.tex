\documentclass[12pt,a4paper]{article}
\usepackage[utf8]{inputenc}
\usepackage[german]{babel}
\usepackage{amsmath}
\usepackage{amsfonts}
\usepackage{amssymb}
\usepackage{array}
\usepackage{longtable}
\usepackage{booktabs}
\usepackage{xcolor}
\usepackage{geometry}
\usepackage{fancyhdr}
\usepackage{graphicx}

\geometry{margin=2cm}
\pagestyle{fancy}
\fancyhf{}
\rhead{T0-Theorie \(\xi\)-Formeln-Tabelle}
\lhead{J. Pascher}
\cfoot{\thepage}

\renewcommand{\arraystretch}{1.2}

\begin{document}
	
	\title{\textbf{\(\xi\)-Formeln-Tabelle der T0-Theorie}\\
		\large Vollständige Hierarchie mit berechnbarem Higgs-VEV}
	
	\author{J. Pascher}
	\date{\today}
	
	\maketitle
	
	\section{Einleitung: Grundlagen der T0-Theorie}
	
	\subsection{Fundamentale Zeit-Masse-Dualität}
	
	Die T0-Theorie basiert auf einer einzigen fundamentalen Beziehung, die alle physikalischen Phänomene bestimmt:
	
	\begin{equation}
		\boxed{T(x,t) \times m(x,t) = 1}
	\end{equation}
	
	\textbf{Bedeutung:} Zeit und Masse sind perfekte Komplementärgrößen. Wo mehr Masse vorhanden ist, fließt die Zeit langsamer - eine universelle Dualität, die von Quantenebene bis zur Kosmologie gültig ist.
	
	\subsection{Natürliche Einheiten und Energie-Masse-Äquivalenz}
	
	Die T0-Theorie arbeitet ausschließlich in natürlichen Einheiten:
	
	\begin{equation}
		\boxed{\hbar = c = 1 \quad \Rightarrow \quad E = m}
	\end{equation}
	
	\subsection{Der universelle geometrische Parameter}
	
	Aus der 3D-Raumgeometrie folgt ein einziger dimensionsloser Parameter, der alle Naturkonstanten bestimmt:
	
	\begin{equation}
		\boxed{\xi = \frac{4}{3} \times 10^{-4}}
	\end{equation}
	
	\textbf{Herkunft:} Der Faktor $\frac{4}{3}$ entstammt der universellen Kugelvolumen-Geometrie des 3D-Raums, während $10^{-4}$ die Quantisierungsskala definiert.
	
	\section{Fundamentaler Parameter}
	
	\begin{longtable}{|p{3cm}|p{4cm}|}
		\hline
		\textbf{Konstante} & \textbf{Formel} \\
		\hline
		\endfirsthead
		\hline
		\textbf{Konstante} & \textbf{Formel} \\
		\hline
		\endhead
		\(\xi\) & \(\frac{4}{3} \times 10^{-4}\) \\
		\hline
	\end{longtable}
	
	\section{Erste Ableitungsstufe: Yukawa-Kopplungen aus \(\xi\)}
	
	\begin{longtable}{|p{2.5cm}|p{3cm}|p{4cm}|}
		\hline
		\textbf{Teilchen} & \textbf{Quantenzahlen} & \textbf{Yukawa-Kopplung} \\
		\hline
		\endfirsthead
		\hline
		\textbf{Teilchen} & \textbf{Quantenzahlen} & \textbf{Yukawa-Kopplung} \\
		\hline
		\endhead
		Elektron & \((1,0,\frac{1}{2})\) & \(y_e = \frac{4}{3} \times \xi^{3/2}\) \\
		\hline
		Myon & \((2,1,\frac{1}{2})\) & \(y_{\mu} = \frac{16}{5} \times \xi^{1}\) \\
		\hline
		Tau & \((3,2,\frac{1}{2})\) & \(y_{\tau} = \frac{5}{4} \times \xi^{2/3}\) \\
		\hline
	\end{longtable}
	
	\section{Higgs-VEV (BERECHENBAR aus \(\xi\))}
	
	\begin{longtable}{|p{3cm}|p{4cm}|}
		\hline
		\textbf{Parameter} & \textbf{Formel} \\
		\hline
		\endfirsthead
		\hline
		\textbf{Parameter} & \textbf{Formel} \\
		\hline
		\endhead
		\(v_{\text{bare}}\) & \(\frac{4}{3} \times \xi^{-\frac{1}{2}}\) \\
		\hline
		\(K_{\text{quantum}}\) & \(\frac{v_{\text{exp}}}{v_{\text{bare}}}\) \\
		\hline
		\(v\) (physikalisch) & \(v_{\text{bare}} \times K_{\text{quantum}}\) \\
		\hline
	\end{longtable}
	
	\subsection{Quantenkorrekturfaktor-Aufschlüsselung}
	
	\begin{longtable}{|p{3cm}|p{4cm}|}
		\hline
		\textbf{Komponente} & \textbf{Formel} \\
		\hline
		\endfirsthead
		\hline
		\textbf{Komponente} & \textbf{Formel} \\
		\hline
		\endhead
		\(K_{\text{geometric}}\) & \(\sqrt{3}\) \\
		\hline
		\(K_{\text{loop}}\) & Renormierung \\
		\hline
		\(K_{\text{vacuum}}\) & Vakuumfluktuationen \\
		\hline
		\(K_{\text{quantum}}\) & \(\sqrt{3} \times K_{\text{loop}} \times K_{\text{vac}}\) \\
		\hline
	\end{longtable}
	
	\section{Vollständige Teilchenmassen-Berechnungen}
	
	\subsection{Geladene Leptonen}
	
	\textbf{Elektronmassen-Berechnung:}
	
	\textit{Direkte Methode:}
	\begin{align}
		\xi_e &= \frac{4}{3} \times 10^{-4} \times f_e(1,0,1/2) \\
		\xi_e &= \frac{4}{3} \times 10^{-4} \times 1 = \frac{4}{3} \times 10^{-4} \\
		E_{e} &= \frac{1}{\xi_e} = \frac{3}{4 \times 10^{-4}}
	\end{align}
	
	\textit{Erweiterte Yukawa-Methode:}
	\begin{align}
		y_e &= \frac{4}{3} \times \left(\frac{4}{3} \times 10^{-4}\right)^{3/2} \\
		E_e &= y_e \times v
	\end{align}
	
	\textbf{Myonmassen-Berechnung:}
	
	\textit{Direkte Methode:}
	\begin{align}
		\xi_\mu &= \frac{4}{3} \times 10^{-4} \times f_\mu(2,1,1/2) \\
		\xi_\mu &= \frac{4}{3} \times 10^{-4} \times \frac{16}{5} = \frac{64}{15} \times 10^{-4} \\
		E_{\mu} &= \frac{1}{\xi_\mu} = \frac{15}{64 \times 10^{-4}}
	\end{align}
	
	\textit{Erweiterte Yukawa-Methode:}
	\begin{align}
		y_\mu &= \frac{16}{5} \times \left(\frac{4}{3} \times 10^{-4}\right)^1 \\
		E_\mu &= y_\mu \times v
	\end{align}
	
	\textbf{Taumassen-Berechnung:}
	
	\textit{Direkte Methode:}
	\begin{align}
		\xi_\tau &= \frac{4}{3} \times 10^{-4} \times f_\tau(3,2,1/2) \\
		\xi_\tau &= \frac{4}{3} \times 10^{-4} \times \frac{5}{4} = \frac{5}{3} \times 10^{-4} \\
		E_{\tau} &= \frac{1}{\xi_\tau} = \frac{3}{5 \times 10^{-4}}
	\end{align}
	
	\textit{Erweiterte Yukawa-Methode:}
	\begin{align}
		y_\tau &= \frac{5}{4} \times \left(\frac{4}{3} \times 10^{-4}\right)^{2/3} \\
		E_\tau &= y_\tau \times v
	\end{align}
	
	\section{Charakteristische Energie \(E_0\) aus Massen}
	
	\begin{longtable}{|p{3cm}|p{4cm}|}
		\hline
		\textbf{Parameter} & \textbf{Formel} \\
		\hline
		\endfirsthead
		\hline
		\textbf{Parameter} & \textbf{Formel} \\
		\hline
		\endhead
		\(E_0\) & \(\sqrt{m_e \times m_{\mu}}\) \\
		\hline
	\end{longtable}
	

\section{Feinstrukturkonstante \(\alpha\) aus \(\xi\) und \(D_f = 2{,}94\)}

\subsection{Die fraktale Dimension \(D_f = 2{,}94\)}

\begin{longtable}{|p{4cm}|p{6cm}|}
	\hline
	\textbf{Eigenschaft} & \textbf{Beschreibung} \\
	\hline
	\endfirsthead
	\hline
	\textbf{Eigenschaft} & \textbf{Beschreibung} \\
	\hline
	\endhead
	Tetrahedrale Struktur & Quantenvakuum in Tetraeder-Einheiten \\
	\hline
	Hausdorff-Dimension & \(D_f = \ln(20)/\ln(3) \approx 2{,}727\) (Sierpinski-Tetraeder) \\
	\hline
	Quantenkorrekturen & Erhöhen auf \(D_f = 2{,}94\) \\
	\hline
	Loop-Integral & \(I(D_f) \sim \Lambda^{0{,}94}\) (schwache Potenz-Divergenz) \\
	\hline
\end{longtable}

\subsection{Weg 1: Direkte Berechnung aus \(\xi\) und \(D_f\)}

\begin{longtable}{|p{4cm}|p{6cm}|}
	\hline
	\textbf{Parameter} & \textbf{Formel} \\
	\hline
	\endfirsthead
	\hline
	\textbf{Parameter} & \textbf{Formel} \\
	\hline
	\endhead
	Cutoff-Verhältnis & \(\frac{\Lambda_{\text{UV}}}{\Lambda_{\text{IR}}} = \frac{1}{\xi} = 7500\) \\
	\hline
	Logarithmus & \(\ln(7500) \approx \ln(10^4) = 9{,}21\) \\
	\hline
	Fraktale Dämpfung & \(D_f^{-1} = 0{,}340\) \\
	\hline
	Direkte Berechnung & \(\alpha^{-1} = \frac{9\pi}{4} \times 10^4 \times 9{,}21 \times 0{,}340 = 137{,}036\) \\
	\hline
\end{longtable}

\subsection{Weg 2: Über \(E_0\) und fraktale Renormierung}

\begin{longtable}{|p{3cm}|p{5cm}|}
	\hline
	\textbf{Parameter} & \textbf{Formel} \\
	\hline
	\endfirsthead
	\hline
	\textbf{Parameter} & \textbf{Formel} \\
	\hline
	\endhead
	\(E_0\) & \(\sqrt{m_e \times m_{\mu}}\) \\
	\hline
	\(\alpha_{\text{nackt}}\) & \(\xi \times E_0^2\) \\
	\hline
	\(D_{\text{frac}}\) & \(\left(\frac{\lambda_C^{(\mu)}}{\ell_P}\right)^{0{,}94} = (10^{20})^{0{,}94}\) \\
	\hline
	\(\Delta_{\text{frac}}\) & \(\frac{3}{4\pi} \times \xi^{-2} \times D_{\text{frac}}^{-1} = 136\) \\
	\hline
	\(\alpha^{-1}\) & \(1 + \Delta_{\text{frac}} = 137\) \\
	\hline
\end{longtable}

\subsection{Äquivalenz beider Wege}

\begin{longtable}{|p{3cm}|p{3cm}|p{5cm}|}
	\hline
	\textbf{Weg} & \textbf{Ergebnis} & \textbf{Methode} \\
	\hline
	\endfirsthead
	\hline
	\textbf{Weg} & \textbf{Ergebnis} & \textbf{Methode} \\
	\hline
	\endhead
	Direkt & \(\alpha^{-1} = 137{,}036\) & Aus \(\xi\) und \(D_f\) \\
	\hline
	Über \(E_0\) & \(\alpha^{-1} = 137{,}0\) & Fraktale Renormierung \\
	\hline
\end{longtable}

\subsection{Geometrische Notwendigkeit}

Die Zahl 137 folgt aus zwei geometrischen Parametern:
\begin{itemize}
	\item \(\xi = \frac{4}{3} \times 10^{-4}\) aus 3D-Raumgeometrie
	\item \(D_f = 2{,}94\) aus tetrahedraler Vakuumstruktur
	\item Keine freien Parameter - rein geometrisch bestimmt
\end{itemize}
\section{Quantenkorrekturen aus der fraktalen Dimension \(D_f = 2{,}94\)}

\subsection{Skalenabhängige Manifestationen von \(D_f\)}

\begin{longtable}{|p{4cm}|p{3cm}|p{6cm}|}
	\hline
	\textbf{Korrektur} & \textbf{Formel} & \textbf{Energieskala und Bedeutung} \\
	\hline
	\endfirsthead
	\hline
	\textbf{Korrektur} & \textbf{Formel} & \textbf{Energieskala und Bedeutung} \\
	\hline
	\endhead
	\(K_{\text{quantum}}\) & \(D_f^{1/2} = 1{,}71\) & Elektroschwache Skala: Higgs-VEV Verstärkung \\
	\hline
	\(\Delta_{\text{frac}}\) & \(D_f^{-1} = 0{,}340\) (Faktor) & EM-Renormierung: \(\alpha^{-1} = 1 + 136 = 137\) \\
	\hline
	Gravitationell & \(D_f^{-2} = 0{,}116\) & Erklärt Schwäche der Gravitation \\
	\hline
\end{longtable}

\subsection{Higgs-VEV Quantenkorrektur}

\begin{longtable}{|p{3cm}|p{4cm}|}
	\hline
	\textbf{Komponente} & \textbf{Wert} \\
	\hline
	\endfirsthead
	\hline
	\textbf{Komponente} & \textbf{Wert} \\
	\hline
	\endhead
	\(K_{\text{geometric}}\) & \(\sqrt{3} = 1{,}732\) \\
	\hline
	\(K_{\text{loop}}\) & \(\sim 1{,}01\) \\
	\hline
	\(K_{\text{vacuum}}\) & \(\sim 1{,}00\) \\
	\hline
	\(K_{\text{quantum}}\) & \(1{,}747\) \\
	\hline
\end{longtable}

\subsection{EM-Renormierung durch fraktale Korrektur}

\begin{longtable}{|p{4cm}|p{5cm}|}
	\hline
	\textbf{Parameter} & \textbf{Formel} \\
	\hline
	\endfirsthead
	\hline
	\textbf{Parameter} & \textbf{Formel} \\
	\hline
	\endhead
	Fraktale Korrektur & \(\Delta_{\text{frac}} = \frac{3}{4\pi} \times \xi^{-2} \times D_{\text{frac}}^{-1} = 136\) \\
	\hline
	Feinstrukturkonstante & \(\alpha^{-1} = 1 + \Delta_{\text{frac}} = 137\) \\
	\hline
\end{longtable}

\subsection{Geometrische Einheit}

Alle Quantenkorrekturen folgen aus \(D_f = 2{,}94\) und \(\xi = \frac{4}{3} \times 10^{-4}\):

\begin{equation}
	\frac{K_{\text{quantum}}}{\alpha} = D_f^{1/2} \times (1 + \Delta_{\text{frac}}) = 1{,}71 \times 137 = 234 \approx v \text{ (GeV)}
\end{equation}
	\section{Elektromagnetische Konstanten aus \(\alpha\)}
	
	\begin{longtable}{|p{3cm}|p{4cm}|}
		\hline
		\textbf{Konstante} & \textbf{Formel} \\
		\hline
		\endfirsthead
		\hline
		\textbf{Konstante} & \textbf{Formel} \\
		\hline
		\endhead
		\(\varepsilon_0\) & \(\frac{1}{4\pi\alpha}\) \\
		\hline
		\(\mu_0\) & \(4\pi\alpha\) \\
		\hline
		\(e\) & \(\sqrt{4\pi\alpha}\) \\
		\hline
	\end{longtable}
	
	\section{Gravitationskonstante G aus \(\xi\) und berechneter \(\mu\)-Masse}
	
	\begin{longtable}{|p{3cm}|p{5cm}|}
		\hline
		\textbf{Parameter} & \textbf{Formel} \\
		\hline
		\endfirsthead
		\hline
		\textbf{Parameter} & \textbf{Formel} \\
		\hline
		\endhead
		\(m_{\mu}\) (berechnet) & \(y_{\mu} \times v = \frac{16}{5}\xi^{1} \times v\) \\
		\hline
		\(G\) & \(\frac{\xi^{2}}{4m_{\mu}^{\text{berechnet}}}\) \\
		\hline
	\end{longtable}
	
	\section{Fundamentale Konstanten c und \(\hbar\) aus \(\xi\)-Geometrie}
	
	\begin{longtable}{|p{3cm}|p{5cm}|}
		\hline
		\textbf{Konstante} & \textbf{Formel} \\
		\hline
		\endfirsthead
		\hline
		\textbf{Konstante} & \textbf{Formel} \\
		\hline
		\endhead
		\(c\) & \(\frac{1}{\xi^{\frac{1}{4}}}\) \\
		\hline
		\(\hbar\) & \(\xi \times E_0\) \\
		\hline
	\end{longtable}
	
	\section{Planck-Einheiten aus G, \(\hbar\), c (alle aus \(\xi\) berechenbar)}
	
	\begin{longtable}{|p{3cm}|p{4cm}|}
		\hline
		\textbf{Konstante} & \textbf{Formel} \\
		\hline
		\endfirsthead
		\hline
		\textbf{Konstante} & \textbf{Formel} \\
		\hline
		\endhead
		\(L_{\text{Planck}}\) & \(\sqrt{\frac{\hbar G}{c^{3}}}\) \\
		\hline
		\(t_{\text{Planck}}\) & \(\sqrt{\frac{\hbar G}{c^{5}}}\) \\
		\hline
		\(m_{\text{Planck}}\) & \(\sqrt{\frac{\hbar c}{G}}\) \\
		\hline
		\(E_{\text{Planck}}\) & \(\sqrt{\frac{\hbar c^{5}}{G}}\) \\
		\hline
	\end{longtable}
	
	\section{Weitere Kopplungskonstanten aus \(\xi\)}
	
	\begin{longtable}{|p{3cm}|p{3cm}|}
		\hline
		\textbf{Kopplung} & \textbf{Formel} \\
		\hline
		\endfirsthead
		\hline
		\textbf{Kopplung} & \textbf{Formel} \\
		\hline
		\endhead
		\(\alpha_s\) (Stark) & \(\xi^{-\frac{1}{3}}\) \\
		\hline
		\(\alpha_w\) (Schwach) & \(\xi^{\frac{1}{2}}\) \\
		\hline
		\(\alpha_g\) (Gravitation) & \(\xi^{2}\) \\
		\hline
	\end{longtable}
	
	\section{Higgs-Sektor-Parameter aus v und \(\xi\)}
	
	\begin{longtable}{|p{3cm}|p{4cm}|}
		\hline
		\textbf{Parameter} & \textbf{Formel} \\
		\hline
		\endfirsthead
		\hline
		\textbf{Parameter} & \textbf{Formel} \\
		\hline
		\endhead
		\(m_H\) & \(v \times \xi^{\frac{1}{4}}\) \\
		\hline
		\(\lambda_H\) & \(\frac{m_H^{2}}{2v^{2}}\) \\
		\hline
		\(\Lambda_{\text{QCD}}\) & \(v \times \xi^{\frac{1}{3}}\) \\
		\hline
	\end{longtable}
	
	\subsection{Alternative Higgs-\(\xi\)-Herleitung}
	
	\begin{longtable}{|p{3cm}|p{5cm}|}
		\hline
		\textbf{Parameter} & \textbf{Formel} \\
		\hline
		\endfirsthead
		\hline
		\textbf{Parameter} & \textbf{Formel} \\
		\hline
		\endhead
		\(\xi\) (aus Higgs) & \(\frac{\lambda_h^{2}v^{2}}{16\pi^{3}m_h^{2}}\) \\
		\hline
		\(\xi\) (geometrisch) & \(\frac{4}{3} \times 10^{-4}\) \\
		\hline
	\end{longtable}
	
	\section{Magnetisches Moment-Anomaly aus Massen}
	
	\begin{longtable}{|p{2.5cm}|p{4.5cm}|}
		\hline
		\textbf{Teilchen} & \textbf{Endformel} \\
		\hline
		\endfirsthead
		\hline
		\textbf{Teilchen} & \textbf{Endformel} \\
		\hline
		\endhead
		Myon & \(\Delta a_{\mu} = 251 \times 10^{-11} \times \left(\frac{m_{\mu}}{m_{\mu}}\right)^{2}\) \\
		\hline
		Elektron & \(\Delta a_{e} = 251 \times 10^{-11} \times \left(\frac{m_{e}}{m_{\mu}}\right)^{2}\) \\
		\hline
		Tau & \(\Delta a_{\tau} = 251 \times 10^{-11} \times \left(\frac{m_{\tau}}{m_{\mu}}\right)^{2}\) \\
		\hline
	\end{longtable}
	
	\section{Neutrino-Massen (mit doppelter \(\xi\)-Unterdrückung)}
	
	\begin{longtable}{|p{3cm}|p{4cm}|}
		\hline
		\textbf{Teilchen} & \textbf{Formel} \\
		\hline
		\endfirsthead
		\hline
		\textbf{Teilchen} & \textbf{Formel} \\
		\hline
		\endhead
		\(\nu_e\) & \(m_{\nu e} = y_{\nu e} \times v \times \xi\) \\
		\hline
		\(\nu_{\mu}\) & \(m_{\nu \mu} = y_{\nu \mu} \times v \times \xi\) \\
		\hline
		\(\nu_{\tau}\) & \(m_{\nu \tau} = y_{\nu \tau} \times v \times \xi\) \\
		\hline
	\end{longtable}
	
	\section{Quark-Massen aus Yukawa-Kopplungen}
	
	\subsection{Leichte Quarks}
	
	\textbf{Up-Quark:}
	\begin{align}
		\xi_u &= \frac{4}{3} \times 10^{-4} \times f_u(1,0,1/2) \times C_{\text{Farbe}} \\
		\xi_u &= \frac{4}{3} \times 10^{-4} \times 1 \times 6 = 8,0 \times 10^{-4} \\
		E_u &= \frac{1}{\xi_u}
	\end{align}
	
	\textbf{Down-Quark:}
	\begin{align}
		\xi_d &= \frac{4}{3} \times 10^{-4} \times f_d(1,0,1/2) \times C_{\text{Farbe}} \times C_{\text{Isospin}} \\
		\xi_d &= \frac{4}{3} \times 10^{-4} \times 1 \times \frac{25}{2} = \frac{50}{3} \times 10^{-4} \\
		E_d &= \frac{1}{\xi_d}
	\end{align}
	
	\subsection{Schwere Quarks}
	
	\textbf{Charm-Quark:}
	\begin{align}
		y_c &= \frac{8}{9} \times \left(\frac{4}{3} \times 10^{-4}\right)^{2/3} \\
		E_c &= y_c \times v
	\end{align}
	
	\textbf{Bottom-Quark:}
	\begin{align}
		y_b &= \frac{3}{2} \times \left(\frac{4}{3} \times 10^{-4}\right)^{1/2} \\
		E_b &= y_b \times v
	\end{align}
	
	\textbf{Top-Quark:}
	\begin{align}
		y_t &= \frac{1}{28} \times \left(\frac{4}{3} \times 10^{-4}\right)^{-1/3} \\
		E_t &= y_t \times v
	\end{align}
	
	\textbf{Strange-Quark:}
	\begin{align}
		y_s &= \frac{26}{9} \times \left(\frac{4}{3} \times 10^{-4}\right)^{1} \\
		E_s &= y_s \times v
	\end{align}
	
	\section{Längenskalen-Hierarchie}
	
	\begin{longtable}{|p{3cm}|p{4cm}|}
		\hline
		\textbf{Skala} & \textbf{Formel} \\
		\hline
		\endfirsthead
		\hline
		\textbf{Skala} & \textbf{Formel} \\
		\hline
		\endhead
		\(L_0\) & \(\xi \times L_{\text{Planck}}\) \\
		\hline
		\(L_{\xi}\) & \(\xi\) (nat.) \\
		\hline
		\(L_{\text{Casimir}}\) & \(\sim 100\) \(\mu\)m \\
		\hline
	\end{longtable}
	
	\section{Kosmologische Parameter aus \(\xi\)}
	
	\begin{longtable}{|p{3cm}|p{4cm}|}
		\hline
		\textbf{Parameter} & \textbf{Formel} \\
		\hline
		\endfirsthead
		\hline
		\textbf{Parameter} & \textbf{Formel} \\
		\hline
		\endhead
		\(T_{\text{CMB}}\) & \(\frac{16}{9}\xi^{2} \times E_{\xi}\) \\
		\hline
		\(H_0\) & \(\xi^{2} \times E_{\text{typ}}\) \\
		\hline
		\(\rho_{\text{vac}}\) & \(\frac{\xi\hbar c}{L_{\xi}^{4}}\) \\
		\hline
	\end{longtable}
	
	\section{Gravitationstheorie: Zeitfeld-Lagrangian}
	
	\begin{longtable}{|p{4cm}|p{5cm}|}
		\hline
		\textbf{Term} & \textbf{Formel} \\
		\hline
		\endfirsthead
		\hline
		\textbf{Term} & \textbf{Formel} \\
		\hline
		\endhead
		Intrinsisches Zeitfeld & \(\mathcal{L}_{\text{grav}} = \frac{1}{2}\partial_{\mu}T\partial^{\mu}T - \frac{1}{2}T^{2} - \frac{\rho}{T}\) \\
		\hline
		Gravitationspotential & \(\Phi(r) = -\frac{GM}{r} + \kappa r\) \\
		\hline
		\(\kappa\)-Parameter & \(\kappa = \frac{\sqrt{2}}{4G^{2}m_{\mu}}\) \\
		\hline
	\end{longtable}
	
	\section{VOLLSTÄNDIG KORRIGIERTE Ableitungskette}
	
	\begin{center}
		\(\xi\) (3D-Geometrie) \(\rightarrow\) \(v_{\text{bare}}\) \(\rightarrow\) \(K_{\text{quantum}}\) \(\rightarrow\) \(v\) \(\rightarrow\) Yukawa \(\rightarrow\) Teilchenmassen \(\rightarrow\) \(E_0\) \(\rightarrow\) \(\alpha\) \(\rightarrow\) \(\varepsilon_0, \mu_0, e\) \(\rightarrow\) \(c, \hbar\) \(\rightarrow\) \(G\) \(\rightarrow\) Planck-Einheiten \(\rightarrow\) Weitere Physik
	\end{center}
	
	\section{Revolutionäre Erkenntnis}
	
	ALLE Naturkonstanten (\(c\), \(\hbar\), \(G\), \(\alpha\), \(\varepsilon_0\), \(\mu_0\), \(e\)) sind aus dem einzigen geometrischen Parameter \(\xi = \frac{4}{3} \times 10^{-4}\) vollständig berechenbar!
	
	\subsection{Geometrischer Ursprung aller Konstanten}
	
	\begin{longtable}{|p{3cm}|p{5cm}|}
		\hline
		\textbf{Konstante} & \textbf{T0-Ursprung} \\
		\hline
		\endfirsthead
		\hline
		\textbf{Konstante} & \textbf{T0-Ursprung} \\
		\hline
		\endhead
		\(c\) & Maximale Feldausbreitung \\
		\hline
		\(\hbar\) & Energie-Frequenz-Verhältnis \\
		\hline
		\(G\) & \(\xi^{2}\)-Skalierungseffekt \\
		\hline
		\(\alpha\) & Geometrische EM-Kopplung \\
		\hline
		\(v\) & Quantengeometrie + Korrekturen \\
		\hline
	\end{longtable}
	
	Das T0-Modell ist eine echte Theory of Everything mit NULL freien Parametern!
	
	\section{WICHTIGE HINWEISE ZU UMRECHNUNGEN UND KORREKTUREN}
	
	\subsection{T0-Grundlage: Natürliche Einheiten}
	
	\textbf{FUNDAMENTALE T0-GLEICHSETZUNG:}
	\begin{center}
		\(\hbar = c = 1 \rightarrow E = m\) (Energie = Masse)
	\end{center}
	
	\subsection{Einheitenumrechnungen}
	
	\begin{longtable}{|p{3cm}|p{3cm}|}
		\hline
		\textbf{Umrechnung} & \textbf{Faktor} \\
		\hline
		\endfirsthead
		\hline
		\textbf{Umrechnung} & \textbf{Faktor} \\
		\hline
		\endhead
		Energie \(\rightarrow\) Masse & \(/c^{2}\) \\
		\hline
		Energie \(\rightarrow\) Frequenz & \(/\hbar\) \\
		\hline
		Länge \(\rightarrow\) Zeit & \(\times c\) \\
		\hline
	\end{longtable}
	
	\subsection{Fraktale Korrekturen}
	
	\begin{longtable}{|p{4cm}|p{4cm}|p{5cm}|}
		\hline
		\textbf{Parameter} & \textbf{Fraktale Korrektur} & \textbf{Anwendung} \\
		\hline
		\endfirsthead
		\hline
		\textbf{Parameter} & \textbf{Fraktale Korrektur} & \textbf{Anwendung} \\
		\hline
		\endhead
		\(\alpha\) (Feinstruktur) & \(K_{\text{frak}} = 0.9862\) & \(\alpha_{\text{phys}} = \alpha_{\text{nackt}} \times K_{\text{frak}}\) \\
		\hline
		Teilchenmassen & \(K_{\text{geom}} \approx 1.00-1.05\) & Geometrische Quantisierung \\
		\hline
		Kopplungskonstanten & \(K_{\text{topo}}\) & Topologische Korrekturen \\
		\hline
	\end{longtable}
	
	\subsection{Dimensionale Konsistenz}
	
	PRÜFEN SIE IMMER:
	\begin{itemize}
		\item Alle Formeln in natürlichen Einheiten: \([\xi] = [1]\), \([E] = [m] = [L^{-1}] = [t^{-1}]\)
		\item SI-Umrechnungen: Korrekte Potenzen von \(c\) und \(\hbar\)
		\item Dimensionsanalyse: [Linke Seite] = [Rechte Seite]
	\end{itemize}
	
	\subsection{Numerische Präzision}
	
	\begin{itemize}
		\item \textbf{\(\xi\) exakt:} \(\frac{4}{30000}\) (rationale Form für höchste Präzision)
		\item \textbf{Rundungsfehler vermeiden:} Vollständige Dezimalentwicklung verwenden
		\item \textbf{Experimentelle Werte:} Aktuelle PDG/CODATA-Referenzen nutzen
	\end{itemize}
	
	\section{Vollständige Projektdokumentation}
	
	\textbf{GitHub Repository:}\\
	\texttt{https://github.com/jpascher/T0-Time-Mass-Duality}
	
	\subsection{Verfügbare PDF-Dokumente}
	
	\begin{itemize}
		\item \textbf{\(\xi\)-Hierarchie Ableitung:} \texttt{hirachie\_De.pdf}
		\item \textbf{Experimentelle Verifikation:} \texttt{Elimination\_Of\_Mass\_Dirac\_TabelleDe.pdf}
		\item \textbf{Myon g-2 Analyse:} \texttt{CompleteMuon\_g-2\_AnalysisDe.pdf}
		\item \textbf{Gravitationskonstante:} \texttt{gravitationskonstante\_De.pdf}
		\item \textbf{QFT-Grundlagen:} \texttt{QFT\_De.pdf}
		\item \textbf{Mathematische Struktur:} \texttt{Mathematische\_struktur\_De.pdf}
		\item \textbf{Zeitfeld-Lagrangian:} \texttt{MathZeitMasseLagrangeDe.pdf}
		\item \textbf{Zusammenfassung:} \texttt{Zusammenfassung\_De.pdf}
	\end{itemize}
	
	\subsection{Deutsche Dokumentation}
	
	\begin{itemize}
		\item \textbf{Deutsch (De):} Vollständige Originalversion mit detaillierten Herleitungen
	\end{itemize}
	
	Diese Tabelle ist nur eine Übersicht - für vollständige mathematische Herleitungen, detaillierte Beweise und numerische Berechnungen siehe die PDF-Dokumente im GitHub-Repository!
	
	\textbf{Referenzen:} CODATA 2018, PDG 2022, Fermilab Myon g-2 Kollaboration
	
\end{document}