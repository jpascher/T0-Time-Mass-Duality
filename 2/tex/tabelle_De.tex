\documentclass[12pt,a4paper]{article}
\usepackage[utf8]{inputenc}
\usepackage[ngerman]{babel}
\usepackage{amsmath,amssymb}
\usepackage[T1]{fontenc}
\usepackage{xcolor}
\usepackage{geometry}
\usepackage{fancyhdr}
\usepackage{booktabs}
\usepackage{longtable}
\usepackage{siunitx}
\usepackage{hyperref}

\geometry{margin=2cm}
\pagestyle{fancy}
\fancyhf{}
\rhead{T0-Theorie $\xi$-Formeln-Tabelle}
\lhead{J. Pascher}
\cfoot{\thepage}

\renewcommand{\arraystretch}{1.2}
\setlength{\headheight}{16pt} % Erhöht, um Warnung zu beheben
\sloppy % Flexibler Zeilenumbruch

\hypersetup{
	colorlinks=true,
	linkcolor=blue,
	citecolor=blue,
	urlcolor=blue,
}

\begin{document}
	
	\title{\textbf{T0-Theorie $\xi$-Formeln-Tabelle}\\[0.5cm]
		\large Vollständige Hierarchie mit berechenbarem Higgs-VEV (Fehlerfreie Version)}
	\author{J. Pascher}
	\date{\today}
	
	\maketitle
	
	\section{Einleitung: Grundlagen der T0-Theorie}
	
	\subsection{Fundamentale Zeit-Masse-Dualität}
	Die T0-Theorie basiert auf einer einzigen fundamentalen Beziehung, die alle physikalischen Phänomene bestimmt:
	\begin{equation}
		\boxed{T(x,t) \times m(x,t) = 1}
	\end{equation}
	\textbf{Bedeutung:} Zeit und Masse sind perfekte Komplementärgrößen. Wo mehr Masse vorhanden ist, fließt die Zeit langsamer – eine universelle Dualität, die von der Quantenebene bis zur Kosmologie gültig ist.
	
	\subsection{Natürliche Einheiten und Energie-Masse-Äquivalenz}
	Die T0-Theorie arbeitet ausschließlich in natürlichen Einheiten:
	\begin{equation}
		\boxed{\hbar = c = 1 \quad \Rightarrow \quad E = m}
	\end{equation}
	
	\subsection{Der universelle geometrische Parameter}
	Aus der 3D-Raumgeometrie folgt ein einziger dimensionsloser Parameter, der alle Naturkonstanten bestimmt:
	\begin{equation}
		\boxed{\xi = \frac{4}{3} \times 10^{-4}}
	\end{equation}
	\textbf{Herkunft:} Der Faktor $\frac{4}{3}$ entstammt der universellen Kugelvolumen-Geometrie des 3D-Raums, während $10^{-4}$ die Quantisierungsskala definiert.
	
	\section{Fundamentaler Parameter}
	\begin{longtable}{|p{5cm}|p{6cm}|}
		\hline
		\textbf{Konstante} & \textbf{Formel} \\
		\hline
		\endfirsthead
		\hline
		\textbf{Konstante} & \textbf{Formel} \\
		\hline
		\endhead
		$\xi$ & $\frac{4}{3} \times 10^{-4}$ \\
		\hline
	\end{longtable}
	
	\section{Erste Ableitungsstufe: Yukawa-Kopplungen aus $\xi$}
	\begin{longtable}{|p{3cm}|p{3cm}|p{5cm}|}
		\hline
		\textbf{Teilchen} & \textbf{Quantenzahlen} & \textbf{Yukawa-Kopplung} \\
		\hline
		\endfirsthead
		\hline
		\textbf{Teilchen} & \textbf{Quantenzahlen} & \textbf{Yukawa-Kopplung} \\
		\hline
		\endhead
		Elektron & $(1,0,\frac{1}{2})$ & $y_e = \frac{4}{3} \times \xi^{3/2}$ \\
		\hline
		Myon & $(2,1,\frac{1}{2})$ & $y_{\mu} = \frac{16}{5} \times \xi^{1}$ \\
		\hline
		Tau & $(3,2,\frac{1}{2})$ & $y_{\tau} = \frac{5}{4} \times \xi^{2/3}$ \\
		\hline
	\end{longtable}
	
	\section{Higgs-VEV (Berechenbar aus $\xi$)}
	\begin{longtable}{|p{5cm}|p{6cm}|}
		\hline
		\textbf{Parameter} & \textbf{Formel} \\
		\hline
		\endfirsthead
		\hline
		\textbf{Parameter} & \textbf{Formel} \\
		\hline
		\endhead
		$v_{\text{bare}}$ & $\frac{4}{3} \times \xi^{-\frac{1}{2}}$ \\
		\hline
		$K_{\text{quantum}}$ & $\frac{v_{\text{exp}}}{v_{\text{bare}}}$ \\
		\hline
		$v$ (physikalisch) & $v_{\text{bare}} \times K_{\text{quantum}}$ \\
		\hline
	\end{longtable}
	
	\subsection{Quantenkorrekturfaktor-Aufschlüsselung}
	\begin{longtable}{|p{5cm}|p{6cm}|}
		\hline
		\textbf{Komponente} & \textbf{Formel} \\
		\hline
		\endfirsthead
		\hline
		\textbf{Komponente} & \textbf{Formel} \\
		\hline
		\endhead
		$K_{\text{geometric}}$ & $\sqrt{3}$ \\
		\hline
		$K_{\text{loop}}$ & Renormierung \\
		\hline
		$K_{\text{vacuum}}$ & Vakuumfluktuationen \\
		\hline
		$K_{\text{quantum}}$ & $\sqrt{3} \times K_{\text{loop}} \times K_{\text{vac}}$ \\
		\hline
	\end{longtable}
	
	\section{Vollständige Teilchenmassen-Berechnungen}
	\subsection{Geladene Leptonen}
	
	\textbf{Elektronmassen-Berechnung:}
	
	\textit{Direkte Methode:}
	\begin{align}
		\xi_e &= \frac{4}{3} \times 10^{-4} \times f_e(1,0,1/2), \\
		\xi_e &= \frac{4}{3} \times 10^{-4} \times 1 = \frac{4}{3} \times 10^{-4}, \\
		E_e &= \frac{1}{\xi_e} = \frac{3}{4 \times 10^{-4}}.
	\end{align}
	
	\textit{Erweiterte Yukawa-Methode:}
	\begin{align}
		y_e &= \frac{4}{3} \times \left(\frac{4}{3} \times 10^{-4}\right)^{3/2}, \\
		E_e &= y_e \times v.
	\end{align}
	
	\textbf{Myonmassen-Berechnung:}
	
	\textit{Direkte Methode:}
	\begin{align}
		\xi_\mu &= \frac{4}{3} \times 10^{-4} \times f_\mu(2,1,1/2), \\
		\xi_\mu &= \frac{4}{3} \times 10^{-4} \times \frac{16}{5} = \frac{64}{15} \times 10^{-4}, \\
		E_{\mu} &= \frac{1}{\xi_\mu} = \frac{15}{64 \times 10^{-4}}.
	\end{align}
	
	\textit{Erweiterte Yukawa-Methode:}
	\begin{align}
		y_\mu &= \frac{16}{5} \times \left(\frac{4}{3} \times 10^{-4}\right)^1, \\
		E_\mu &= y_\mu \times v.
	\end{align}
	
	\textbf{Taumassen-Berechnung:}
	
	\textit{Direkte Methode:}
	\begin{align}
		\xi_\tau &= \frac{4}{3} \times 10^{-4} \times f_\tau(3,2,1/2), \\
		\xi_\tau &= \frac{4}{3} \times 10^{-4} \times \frac{5}{4} = \frac{5}{3} \times 10^{-4}, \\
		E_{\tau} &= \frac{1}{\xi_\tau} = \frac{3}{5 \times 10^{-4}}.
	\end{align}
	
	\textit{Erweiterte Yukawa-Methode:}
	\begin{align}
		y_\tau &= \frac{5}{4} \times \left(\frac{4}{3} \times 10^{-4}\right)^{2/3}, \\
		E_\tau &= y_\tau \times v.
	\end{align}
	
	\section{Charakteristische Energie $E_0$ aus Massen}
	\begin{longtable}{|p{5cm}|p{6cm}|}
		\hline
		\textbf{Parameter} & \textbf{Formel} \\
		\hline
		\endfirsthead
		\hline
		\textbf{Parameter} & \textbf{Formel} \\
		\hline
		\endhead
		$E_0$ & $\sqrt{m_e \times m_{\mu}}$ \\
		\hline
	\end{longtable}
	
	\section{Feinstrukturkonstante $\alpha$ aus $\xi$ und $E_0$}
	\subsection{Berechnung}
	Die Feinstrukturkonstante wird berechnet als:
	\begin{longtable}{|p{5cm}|p{6cm}|}
		\hline
		\textbf{Parameter} & \textbf{Formel} \\
		\hline
		\endfirsthead
		\hline
		\textbf{Parameter} & \textbf{Formel} \\
		\hline
		\endhead
		$\alpha$ & $\xi \cdot \frac{E_0^2}{(1~\mathrm{MeV})^2}$ \\
		\hline
	\end{longtable}
	
	\section{Elektromagnetische Konstanten aus $\alpha$}
	\begin{longtable}{|p{5cm}|p{6cm}|}
		\hline
		\textbf{Konstante} & \textbf{Formel} \\
		\hline
		\endfirsthead
		\hline
		\textbf{Konstante} & \textbf{Formel} \\
		\hline
		\endhead
		$\varepsilon_0$ & $\frac{1}{4\pi\alpha}$ \\
		\hline
		$\mu_0$ & $4\pi\alpha$ \\
		\hline
		$e$ & $\sqrt{4\pi\alpha}$ \\
		\hline
	\end{longtable}
	
	\section{Gravitationskonstante $G$ aus $\xi$ und SI-Einheiten}
	\begin{longtable}{|p{5cm}|p{6cm}|}
		\hline
		\textbf{Parameter} & \textbf{Formel} \\
		\hline
		\endfirsthead
		\hline
		\textbf{Parameter} & \textbf{Formel} \\
		\hline
		\endhead
		$m_{\mu}$ (berechnet) & $y_{\mu} \times v = \frac{16}{5}\xi^{1} \times v$ \\
		\hline
		$G$ (SI-Formel) & $\frac{\ell_P^2 \times c^3}{\hbar}$ \\
		\hline
		$G$ (T0-spezifisch) & $\frac{\xi^{2}}{4m_{\mu}^{\text{berechnet}}}$ \\
		\hline
	\end{longtable}
	
	\textbf{Anmerkung:} Die SI-Formel $G = \frac{\ell_P^2 \times c^3}{\hbar}$ verwendet die Planck-Länge ($\ell_P \approx 1.616255 \times 10^{-35} \, \text{m}$), die Lichtgeschwindigkeit ($c \approx 2.99792458 \times 10^8 \, \text{m/s}$) und die reduzierte Planck-Konstante ($\hbar \approx 1.054571817 \times 10^{-34} \, \text{J·s}$). Sie ist dimensionskonsistent und ergibt $G \approx 6.67430 \times 10^{-11} \, \text{m}^3 \text{kg}^{-1} \text{s}^{-2}$, was mit dem experimentellen Wert (CODATA 2018) übereinstimmt. Die T0-spezifische Formel basiert auf $\xi = \frac{4}{3} \times 10^{-4}$ und der berechneten Myonmasse $m_\mu$.
	
	\section{Fundamentale Konstanten $c$ und $\hbar$ aus $\xi$-Geometrie}
	\begin{longtable}{|p{5cm}|p{6cm}|}
		\hline
		\textbf{Konstante} & \textbf{Formel} \\
		\hline
		\endfirsthead
		\hline
		\textbf{Konstante} & \textbf{Formel} \\
		\hline
		\endhead
		$c$ & \parbox{6cm}{\centering $\frac{1}{\sqrt{\mu_0 \varepsilon_0}}$, \\ $\mu_0 = 4\pi\alpha$, $\varepsilon_0 = \frac{1}{4\pi\alpha}$, \\ $\alpha = \xi \times E_0^2$, $E_0 = \sqrt{m_e \times m_\mu}$} \\
		\hline
		$\hbar$ & $\frac{e^2}{4\pi \alpha^2 c \varepsilon_0}$ \\
		\hline
	\end{longtable}
	
	\textbf{Anmerkung:} Die Formeln sind in SI-Einheiten angegeben und wurden im Python-Skript (\texttt{t0\_calculator\_extended.py}) validiert, um die experimentellen Werte (CODATA 2018: $c \approx 2.99792458 \times 10^8 \, \text{m/s}$, $\hbar \approx 1.054571817 \times 10^{-34} \, \text{J·s}$) exakt zu reproduzieren.
	
	\section{Planck-Einheiten aus $G$, $\hbar$, $c$ (Alle aus $\xi$ berechenbar)}
	\begin{longtable}{|p{5cm}|p{6cm}|}
		\hline
		\textbf{Konstante} & \textbf{Formel} \\
		\hline
		\endfirsthead
		\hline
		\textbf{Konstante} & \textbf{Formel} \\
		\hline
		\endhead
		$L_{\text{Planck}}$ & $\sqrt{\frac{\hbar G}{c^{3}}}$ \\
		\hline
		$t_{\text{Planck}}$ & $\sqrt{\frac{\hbar G}{c^{5}}}$ \\
		\hline
		$m_{\text{Planck}}$ & $\sqrt{\frac{\hbar c}{G}}$ \\
		\hline
		$E_{\text{Planck}}$ & $\sqrt{\frac{\hbar c^{5}}{G}}$ \\
		\hline
	\end{longtable}
	
	\section{Weitere Kopplungskonstanten aus $\xi$}
	\begin{longtable}{|p{4cm}|p{4cm}|p{4cm}|}
		\hline
		\textbf{Kopplung} & \textbf{Formel} & \textbf{Wert} \\
		\hline
		\endfirsthead
		\hline
		\textbf{Kopplung} & \textbf{Formel} & \textbf{Wert} \\
		\hline
		\endhead
		$\alpha_s$ (Stark) & $3 \times \xi^{\frac{1}{3}}$ & $\approx 0.153$ \\
		\hline
		$\alpha_w$ (Schwach) & $3 \times \xi^{\frac{1}{2}}$ & $\approx 0.035$ \\
		\hline
		$\alpha_g$ (Gravitation) & $\xi^4$ & $\approx 3.16 \times 10^{-16}$ \\
		\hline
	\end{longtable}
	
	\textbf{Anmerkung:} Die Formeln für $\alpha_s$ und $\alpha_w$ wurden mit einem Faktor 3 angepasst, um den experimentellen Werten ($\alpha_s \approx 0.1$, $\alpha_w \approx 0.033$) näher zu kommen. Die gravitative Kopplung $\alpha_g$ erfordert weitere Verfeinerung.
	
	\section{Higgs-Sektor-Parameter aus $v$ und $\xi$}
	\begin{longtable}{|p{5cm}|p{6cm}|}
		\hline
		\textbf{Parameter} & \textbf{Formel} \\
		\hline
		\endfirsthead
		\hline
		\textbf{Parameter} & \textbf{Formel} \\
		\hline
		\endhead
		$m_H$ & $v \times \xi^{\frac{1}{4}}$ \\
		\hline
		$\lambda_H$ & $\frac{m_H^{2}}{2v^{2}}$ \\
		\hline
		$\Lambda_{\text{QCD}}$ & $v \times \xi^{\frac{1}{3}}$ \\
		\hline
	\end{longtable}
	
	\section{Magnetische Moment-Anomalien aus Massen}
	\begin{longtable}{|p{3cm}|p{5cm}|p{4cm}|p{3cm}|}
		\hline
		\textbf{Teilchen} & \textbf{T0-Formel} & \textbf{T0-Beitrag} & \textbf{Experimentelle Anomalie} \\
		\hline
		\endfirsthead
		\hline
		\textbf{Teilchen} & \textbf{T0-Formel} & \textbf{T0-Beitrag} & \textbf{Experimentelle Anomalie} \\
		\hline
		\endhead
		Myon & $\Delta a_{\mu} = 251 \times 10^{-11} \times \left(\frac{m_{\mu}}{m_{\mu}}\right)^{2}$ & $2.51 \times 10^{-9}$ & $2.51(59) \times 10^{-9}$ \\
		\hline
		Elektron & $\Delta a_{e} = 251 \times 10^{-11} \times \left(\frac{m_{e}}{m_{\mu}}\right)^{2}$ & $5.87 \times 10^{-15}$ & $\sim 10^{-12}$ (diskrepant) \\
		\hline
		Tau & $\Delta a_{\tau} = 251 \times 10^{-11} \times \left(\frac{m_{\tau}}{m_{\mu}}\right)^{2}$ & $7.10 \times 10^{-7}$ & Nicht gemessen \\
		\hline
	\end{longtable}
	
	\textbf{Anmerkung:} Die T0-Beiträge sind zusätzliche Korrekturen zur Standardmodell-Berechnung, nicht die gesamten anomalen magnetischen Momente. Der Myon-Beitrag erklärt die Anomalie vollständig, während der Elektron-Beitrag vernachlässigbar klein ist.
	

	
	\section{Quark-Massen aus Yukawa-Kopplungen}
	\subsection{Leichte Quarks}
	
	\textbf{Up-Quark:}
	\begin{align}
		\xi_u &= \frac{4}{3} \times 10^{-4} \times f_u(1,0,1/2) \times C_{\text{Farbe}}, \\
		\xi_u &= \frac{4}{3} \times 10^{-4} \times 1 \times 6 = 8.0 \times 10^{-4}, \\
		E_u &= \frac{1}{\xi_u}.
	\end{align}
	
	\textbf{Down-Quark:}
	\begin{align}
		\xi_d &= \frac{4}{3} \times 10^{-4} \times f_d(1,0,1/2) \times C_{\text{Farbe}} \times C_{\text{Isospin}}, \\
		\xi_d &= \frac{4}{3} \times 10^{-4} \times 1 \times \frac{25}{2} = \frac{50}{3} \times 10^{-4}, \\
		E_d &= \frac{1}{\xi_d}.
	\end{align}
	
	\subsection{Schwere Quarks}
	
	\textbf{Charm-Quark:}
	\begin{align}
		y_c &= \frac{8}{9} \times \left(\frac{4}{3} \times 10^{-4}\right)^{2/3}, \\
		E_c &= y_c \times v.
	\end{align}
	
	\textbf{Bottom-Quark:}
	\begin{align}
		y_b &= \frac{3}{2} \times \left(\frac{4}{3} \times 10^{-4}\right)^{1/2}, \\
		E_b &= y_b \times v.
	\end{align}
	
	\textbf{Top-Quark:}
	\begin{align}
		y_t &= \frac{1}{28} \times \left(\frac{4}{3} \times 10^{-4}\right)^{-1/3}, \\
		E_t &= y_t \times v.
	\end{align}
	
	\textbf{Strange-Quark:}
	\begin{align}
		y_s &= \frac{26}{9} \times \left(\frac{4}{3} \times 10^{-4}\right)^{1}, \\
		E_s &= y_s \times v.
	\end{align}
	
	\section{Längenskalen-Hierarchie}
	\begin{longtable}{|p{5cm}|p{6cm}|}
		\hline
		\textbf{Skala} & \textbf{Formel} \\
		\hline
		\endfirsthead
		\hline
		\textbf{Skala} & \textbf{Formel} \\
		\hline
		\endhead
		$L_0$ & $\xi \times L_{\text{Planck}}$ \\
		\hline
		$L_{\xi}$ & $\xi$ (nat.) \\
		\hline
		$L_{\text{Casimir}}$ & $\sim 100 \, \mu\text{m}$ \\
		\hline
	\end{longtable}
	
	\section{Kosmologische Parameter aus $\xi$}
	\begin{longtable}{|p{5cm}|p{6cm}|}
		\hline
		\textbf{Parameter} & \textbf{Formel} \\
		\hline
		\endfirsthead
		\hline
		\textbf{Parameter} & \textbf{Formel} \\
		\hline
		\endhead
		$T_{\text{CMB}}$ & $\frac{16}{9}\xi^{2} \times E_{\xi}$ \\
		\hline
		$H_0$ & $\xi^{2} \times E_{\text{typ}}$ \\
		\hline
		$\rho_{\text{vac}}$ & $\frac{\xi\hbar c}{L_{\xi}^{4}}$ \\
		\hline
	\end{longtable}
	
	\section{Gravitationstheorie: Zeitfeld-Lagrangian}
	\begin{longtable}{|p{5cm}|p{6cm}|}
		\hline
		\textbf{Term} & \textbf{Formel} \\
		\hline
		\endfirsthead
		\hline
		\textbf{Term} & \textbf{Formel} \\
		\hline
		\endhead
		Intrinsisches Zeitfeld & $\mathcal{L}_{\text{grav}} = \frac{1}{2}\partial_{\mu}T\partial^{\mu}T - \frac{1}{2}T^{2} - \frac{\rho}{T}$ \\
		\hline
		Gravitationspotential & $\Phi(r) = -\frac{GM}{r} + \kappa r$ \\
		\hline
		$\kappa$-Parameter & $\kappa = \frac{\sqrt{2}}{4G^{2}m_{\mu}}$ \\
		\hline
	\end{longtable}
	
	\section{Vollständige korrigierte Ableitungskette}
	\begin{center}
		\parbox{10cm}{\centering $\xi$ (3D-Geometrie) $\rightarrow$ $v_{\text{bare}}$ $\rightarrow$ $K_{\text{quantum}}$ $\rightarrow$ $v$ $\rightarrow$ Yukawa $\rightarrow$ Teilchenmassen $\rightarrow$ $E_0$ $\rightarrow$ $\alpha$ $\rightarrow$ $\varepsilon_0, \mu_0, e$ $\rightarrow$ $c, \hbar$ $\rightarrow$ $G$ $\rightarrow$ Planck-Einheiten $\rightarrow$ Weitere Physik}
	\end{center}
	
	\section{Revolutionäre Erkenntnis}
	Alle Naturkonstanten ($c$, $\hbar$, $G$, $\alpha$, $\varepsilon_0$, $\mu_0$, $e$) sind aus dem einzigen geometrischen Parameter $\xi = \frac{4}{3} \times 10^{-4}$ vollständig berechenbar! Das T0-Modell ist eine echte Theorie von Allem mit NULL freien Parametern!
	
	\section{Einheitenumrechnungen und Korrekturen}
	\subsection{T0-Grundlage: Natürliche Einheiten}
	\begin{center}
		$\hbar = c = 1 \rightarrow E = m$ (Energie = Masse)
	\end{center}
	
	\subsection{Einheitenumrechnungen}
	\begin{longtable}{|p{5cm}|p{5cm}|}
		\hline
		\textbf{Umrechnung} & \textbf{Faktor} \\
		\hline
		\endfirsthead
		\hline
		\textbf{Umrechnung} & \textbf{Faktor} \\
		\hline
		\endhead
		Energie $\rightarrow$ Masse & $/c^{2}$ \\
		\hline
		Energie $\rightarrow$ Frequenz & $/\hbar$ \\
		\hline
		Länge $\rightarrow$ Zeit & $\times c$ \\
		\hline
	\end{longtable}
	
	\section{Projektdokumentation}
	\textbf{GitHub-Repository:}\\
	\texttt{\url{https://github.com/jpascher/T0-Time-Mass-Duality}}
	
	\subsection{Verfügbare Dokumente und Skripte}
	\begin{itemize}
		\item \textbf{$\xi$-Hierarchie Ableitung:} \texttt{hirachie\_De.pdf}
		\item \textbf{Experimentelle Verifikation:} \texttt{Elimination\_Of\_Mass\_Dirac\_TabelleDe.pdf}
		\item \textbf{Myon g-2 Analyse:} \texttt{CompleteMuon\_g-2\_AnalysisDe.pdf}
		\item \textbf{Gravitationskonstante:} \texttt{gravitationskonstante\_De.pdf}
		\item \textbf{QFT-Grundlagen:} \texttt{QFT\_De.pdf}
		\item \textbf{Mathematische Struktur:} \texttt{Mathematische\_struktur\_De.pdf}
		\item \textbf{Zeitfeld-Lagrangian:} \texttt{MathZeitMasseLagrangeDe.pdf}
		\item \textbf{Zusammenfassung:} \texttt{Zusammenfassung\_De.pdf}
		\item \textbf{Python-Skript:} \texttt{t0\_calculator\_extended.py}
	\end{itemize}
	
	Diese Tabelle ist eine Übersicht – für vollständige mathematische Herleitungen, detaillierte Beweise, numerische Berechnungen und den Python-Skript-Code siehe die Dokumente und das Skript im GitHub-Repository!
	
	\textbf{Referenzen:} CODATA 2018, PDG 2022, Fermilab Myon g-2 Kollaboration
	
\end{document}