% Chapter file: 036_T0_peratt_En_ch.tex
% Source: 036_T0_peratt_En.tex
% No preamble, no headers/footers, no page numbers
	\hfuzz=200pt
	
	\chapter{\textbf{Mathematical Constructs of Alternative CMB Models: Unnikrishnan and Peratt in Harmony with the T0 Theory}}
	
	\thispagestyle{fancy}
	
	\begin{abstract}
		Based on the video ``The CMB Power Spectrum – Cosmology's Untouchable Curve?'' we analyze the mathematical foundations of the alternative models by C. S. Unnikrishnan (cosmic relativity) and Anthony L. Peratt (plasma cosmology) in detail. Unnikrishnan's field equations extend special relativity to include universal gravitational effects in a static space, while Peratt's Maxwell-based plasma model derives synchrotron radiation as the origin of the CMB. We show how both constructs are compatible with the T0 theory: The $\xiT$-field ($\xiT = \frac{4}{3} \times 10^{-4}$) serves as a universal parameter that unifies resonance modes (Unnikrishnan) and filament dynamics (Peratt). The synthesis yields a coherent, expansion-free cosmology that explains the CMB power spectrum as an emergent $\xiT$-harmony.
	\end{abstract}
	
	\newpage
	
	\section{Introduction: From Surface to Mathematical Analysis}
	
	The video \cite{036_video2025} highlights the circular nature of the $\Lambda$CDM model and contrasts it with radical alternatives: Unnikrishnan's static resonance and Peratt's plasma-based radiation. A superficial consideration is insufficient; we delve into the field equations and derivations based on primary sources \cite{036_unnikrishnan2004, 036_peratt1992}. Objective: A synthesis with T0, where the $\xiT$-field connects the duality of time-mass ($T \cdot m = 1$) and fractal geometry. This resolves open problems such as the high Q-factor or spectral precision.
	
	\section{Mathematical Constructs of Cosmic Relativity (Unnikrishnan)}
	
	Unnikrishnan's theory \cite{036_unnikrishnan2004} reformulates relativity as ``cosmic relativity'': Relativistic effects are gravitational gradients of a homogeneous, static universe. No expansion; CMB peaks as standing waves in a cosmic field.
	
	\subsection{Fundamental Field Equations}
	The core idea: The Lorentz transformations $\Lorentz_{v,t}$ become gravitational effects:
	\begin{equation}
		\Lorentz_{v,t} = \exp\left( -\frac{\nabla \Phi}{c^2} \right),
	\end{equation}
	where $\Phi$ is the cosmic gravitational potential ($\Phi = -GM/r$ for a homogeneous universe, $M$ the total mass). Time dilation and length contraction emerge as:
	\begin{equation}
		\frac{\Delta t}{t} = 1 + \frac{\Phi}{c^2}, \quad \frac{\Delta l}{l} = 1 - \frac{\Phi}{c^2}.
	\end{equation}
	The field equation extends Einstein's equations to a ``cosmic metric'':
	\begin{equation}
		\Riem = 8\pi G (T_{\mu\nu} - \frac{1}{2} g_{\mu\nu} T) + \Lambda g_{\mu\nu} + \xiT \nabla_\mu \nabla_\nu \Phi,
	\end{equation}
	with $\xiT$ as the coupling constant (analogous to T0 here). The Weyl part $\Weyl$ represents anisotropic cosmic gradients.
	
	\subsection{CMB Derivation: Standing Waves}
	CMB as resonance modes in a static field: The wave equation in the cosmic frame:
	\begin{equation}
		\square \psi + \frac{\nabla \Phi}{c^2} \partial_t \psi = 0.
	\end{equation}
	This leads to standing waves $\psi = \sum_k A_k \sin(k \cdot x - \omega t + \phi_k)$, with peaks at $k_n = n \pi / L_{\text{cosmic}}$ ($L$ = cosmic size). Q-factor $Q = \omega / \Delta \omega \approx 10^6$ due to gravitational damping. Polarization: $\Weyl$-induced phase shifts.
	
	The video (11:46) describes this as ``living resonance'' – mathematically: Harmonic oscillators in $\Phi$-gradients.
	
	\section{Mathematical Constructs of Plasma Cosmology (Peratt)}
	
	Peratt's model \cite{036_peratt1992} derives the CMB from plasma dynamics: Synchrotron radiation in Birkeland filaments produces a blackbody spectrum through collective emission/absorption.
	
	\subsection{Fundamental Field Equations}
	Based on Maxwell's equations in plasmas:
	\begin{equation}
		\nabla \times \mathbf{B} = \mu_0 \mathbf{J} + \mu_0 \epsilon_0 \frac{\partial \mathbf{E}}{\partial t}, \quad \nabla \cdot \mathbf{B} = 0,
	\end{equation}
	with Lorentz force $\mathbf{F} = q(\mathbf{E} + \mathbf{v} \times \mathbf{B})$. For filaments: Z-pinch equation
	\begin{equation}
		\ZPinch.
	\end{equation}
	where $\mathbf{J}$ is current density ($10^{18}$ A in galactic filaments). Synchrotron power:
	\begin{equation}
		\SynchPower = \frac{2}{3} r_e^2 \gamma^4 \beta^2 c B_\perp^2 \sin^2 \theta,
	\end{equation}
	with $r_e$ classical electron radius, $\gamma$ Lorentz factor.
	
	\subsection{CMB Derivation: Spectrum and Power Spectrum}
	Collective radiation: Integrated spectrum over $N$ filaments:
	\begin{equation}
		I(\nu) = \int N(\mathbf{r}) P_{\text{synch}}(\nu, B(\mathbf{r})) e^{-\tau(\nu)} d\mathbf{r},
	\end{equation}
	where $\tau(\nu)$ is optical depth (self-absorption). For CMB fit: $T \approx 2.7$ K at $\nu \approx 160$ GHz; peaks as interference:
	\begin{equation}
		C_\ell = \frac{1}{2\ell + 1} \sum_m |a_{\ell m}|^2, \quad a_{\ell m} \propto \int Y_{\ell m}^*(\theta, \phi) e^{i \mathbf{k} \cdot \mathbf{r}} d\Omega,
	\end{equation}
	with $\mathbf{k}$ wave vector in filament magnetic fields. BAO: Fractal scales $r_n = r_0 \phi^n$ ($\phi$ golden ratio).
	
	The video (13:46) emphasizes ``pure electrodynamics'' – Peratt's simulations match SED to 1\%.
	
	\section{Synthesis: Harmony with the T0 Theory}
	
	T0 unifies both through the $\xiT$-field: Static universe with fractal geometry, where redshift $z \approx d \cdot C \cdot \xiT$.
	
	\subsection{Unnikrishnan in T0}
	$\xiT$ as cosmic coupling parameter: Replaces $\nabla \Phi / c^2$ with $\xiT \nabla \ln \rho_\xi$, where $\rho_\xi$ is $\xiT$-density. Extended equation:
	\begin{equation}
		\Riem = 8\pi G T_{\mu\nu} + \xiT \nabla_\mu \nabla_\nu \ln \rho_\xi.
	\end{equation}
	Resonance modes: $\square \psi + \xiT \mathcal{F}[\psi] = 0$ (T0 field equation), peaks at $\omega_n = n c / L \cdot (1 - 100 \xiT)$. Q-factor: $Q \approx 1 / (1 - K_{\text{frak}}) \approx 10^4 / \xiT$.
	
	\subsection{Peratt in T0}
	Filaments as $\xiT$-induced currents: $\mathbf{J} = \sigma \mathbf{E} + \xiT \nabla \times \mathbf{B}$. Synchrotron:
	\begin{equation}
		\SynchPower = \frac{2}{3} r_e^2 \gamma^4 \beta^2 c (B_\perp + \xiT \partial_t B)^2.
	\end{equation}
	Power spectrum: Fractal hierarchy $C_\ell \propto \sum_n \xiT^n \sin(\ell \theta_n)$, with $\theta_n = \pi (1 - 100 \xiT)^n$. BAO: $r_{\text{BAO}} \approx 150$ Mpc as $\xiT$-scaled filament length.
	
	\subsection{Unified T0 Equation}
	Combined field equation:
	\begin{equation}
		\square A_\mu + \xiT \left( \nabla^\nu F_{\nu\mu} + \mathcal{F}[A_\mu] \right) = J_\mu,
	\end{equation}
	where $A_\mu$ is the vector potential (Peratt), $\mathcal{F}$ the fractal operator (Unnikrishnan/T0). This generates CMB as $\xiT$-resonance in a static plasma field.
	
	\section{Conclusion}
	
	The mathematical constructs of Unnikrishnan (gravitational Lorentz transformations) and Peratt (Maxwell-synchrotron in filaments) are coherent but isolated. T0 brings them into harmony: $\xiT$ as a bridge between resonance and plasma dynamics. The CMB power spectrum emerges as $\xiT$-harmony – precise, without patches. Future simulations (e.g., FEniCS for $\xiT$-fields) will test this.
	
	\begin{thebibliography}{9}
		\bibitem{036_unnikrishnan2004}
		C. S. Unnikrishnan, \textit{Cosmic Relativity: The Fundamental Theory of Relativity, its Implications, and Experimental Tests},
		arXiv:gr-qc/0406023, 2004.
		\url{https://arxiv.org/abs/gr-qc/0406023}.
		
		\bibitem{036_peratt1992}
		A. L. Peratt, \textit{Physics of the Plasma Universe},
		Springer-Verlag, 1992.
		\url{https://ia600804.us.archive.org/12/items/AnthonyPerattPhysicsOfThePlasmaUniverse_201901/Anthony-Peratt--Physics-of-the-Plasma-Universe.pdf}.
		
		\bibitem{036_peratt1986}
		A. L. Peratt, \textit{Evolution of the Plasma Universe: I. Double Radio Galaxies, Quasars, and Extragalactic Jets},
		IEEE Transactions on Plasma Science, 14(6), 639–660, 1986.
		
		\bibitem{036_pascher:t0_foundations}
		J. Pascher, \textit{T0 Theory: Summary of Insights},
		T0 Document Series, Nov. 2025.
		
		\bibitem{036_video2025}
		See the Pattern, \textit{A Test Only $\Lambda$CDM Can Pass, Because It Wrote the Rules},
		YouTube Video, URL: \url{https://www.youtube.com/watch?v=g7_JZJzVuqs},
		November 16, 2025.
		
	\end{thebibliography}
	
