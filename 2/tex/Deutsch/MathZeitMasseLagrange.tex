\documentclass[12pt,a4paper]{article}
\usepackage[margin=2cm]{geometry}
\usepackage[utf8]{inputenc}
\usepackage[T1]{fontenc}
\usepackage{lmodern}
\usepackage[ngerman]{babel}
\usepackage{amsmath,amssymb,physics,graphicx,xcolor,amsthm}
\usepackage{hyperref}
\usepackage{booktabs}
\usepackage{siunitx}
\usepackage{cleveref}
\usepackage{pgfplots}
\pgfplotsset{compat=1.18}
\usepackage{tikz}
\usetikzlibrary{intersections}
\usepgfplotslibrary{fillbetween}
\usepackage{fancyhdr}

% Custom commands
\newcommand{\Tfield}{T(x)}
\newcommand{\betaT}{\beta_{\text{T}}}
\newcommand{\alphaEM}{\alpha_{\text{EM}}}
\newcommand{\Tzero}{T_0}
\newcommand{\DcovT}[1]{\partial_\mu #1 + #1 \partial_\mu \Tfield}
\newcommand{\DhiggsT}{\Tfield (\partial_\mu + ig A_\mu) \Phi + \Phi \partial_\mu \Tfield}
\newcommand{\gammaf}{\gamma_{\text{Lorentz}}}

% Theorem styles
\newtheorem{theorem}{Satz}[section]
\newtheorem{proposition}[theorem]{Proposition}
\newtheorem{corollary}[theorem]{Korollar}
\newtheorem{lemma}[theorem]{Lemma}
\theoremstyle{definition}
\newtheorem{definition}[theorem]{Definition}
\newtheorem{example}[theorem]{Beispiel}
\theoremstyle{remark}
\newtheorem{remark}[theorem]{Bemerkung}

% Hyperref configuration
\hypersetup{
	colorlinks=true,
	linkcolor=blue,
	urlcolor=blue,
	citecolor=blue,
	pdftitle={Von Zeitdilatation zur Massenvariation: Mathematische Kernformulierungen der Zeit-Masse-Dualitätstheorie},
	pdfauthor={Johann Pascher},
	pdfsubject={Theoretische Physik},
	pdfkeywords={T0-Modell, Zeit-Masse-Dualität, Emergente Gravitation}
}

% Header and Footer Configuration
\pagestyle{fancy}
\fancyhf{}
\fancyhead[L]{Johann Pascher}
\fancyhead[R]{Von Zeitdilatation zur Massenvariation}
\fancyfoot[C]{\thepage}
\renewcommand{\headrulewidth}{0.4pt}
\renewcommand{\footrulewidth}{0.4pt}

\title{Von Zeitdilatation zur Massenvariation: \\ Mathematische Kernformulierungen der Zeit-Masse-Dualitätstheorie}
\author{Johann Pascher}
\date{29. März 2025}

\begin{document}
	
	\maketitle
	
	\begin{abstract}
		Diese Arbeit präsentiert die wesentlichen mathematischen Formulierungen der Zeit-Masse-Dualitätstheorie und konzentriert sich auf die grundlegenden Gleichungen und ihre physikalischen Interpretationen. Die Theorie etabliert eine Dualität zwischen zwei komplementären Beschreibungen der Realität: der Standardsicht mit Zeitdilatation und konstanter Ruhemasse und dem T0-Modell mit absoluter Zeit und variabler Masse. Zentral für diesen Rahmen ist die intrinsische Zeit \( \Tfield = \frac{\hbar}{\max(m c^2, \omega)} \), die eine einheitliche Behandlung von massiven Teilchen und Photonen ermöglicht. Die mathematischen Formulierungen umfassen modifizierte Lagrange-Dichten, die emergente Gravitation und energieverlustbedingte Rotverschiebung in einem statischen Universum hervorheben. Aufbauend auf der konzeptionellen Grundlage aus \cite{pascher_zeit_masse_2025} liefert diese Arbeit die rigorose mathematische Struktur, die für Anwendungen auf spezifische physikalische Szenarien erforderlich ist.
	\end{abstract}
	
	\tableofcontents
	\newpage
	
	\section{Einführung in die Zeit-Masse-Dualität}
	Die Zeit-Masse-Dualitätstheorie schlägt einen alternativen Rahmen zur konventionellen relativistischen Perspektive vor. Während beide Rahmen mathematisch äquivalent sind, bieten sie komplementäre Einblicke in physikalische Phänomene:
	\begin{enumerate}
		\item Standardsicht (Relativistisch): \( t' = \gammaf t \), \( m_0 = \text{const.} \)
		\item T0-Modell: \( \Tzero = \text{const.} \), \( m = \gammaf m_0 \)
	\end{enumerate}
	
	Diese Dualität ist analog zur Welle-Teilchen-Dualität in der Quantenmechanik, wie in \cite{pascher_planck_2025} diskutiert, und bietet komplementäre Perspektiven statt widersprüchlicher Beschreibungen derselben Realität. Das Konzept wurde erstmals in \cite{pascher_zeit_masse_2025} eingeführt und in \cite{pascher_zeit_2025} weiter ausgearbeitet.
	
	\subsection{Beziehung zum Standardmodell}
	Das T0-Modell erweitert das Standardmodell um folgende Schlüsselkomponenten:
	\begin{enumerate}
		\item Intrinsisches Zeitfeld: \( \Tfield = \frac{\hbar}{\max(m c^2, \omega)} \)
		\item Higgs-Feld: \( \Phi \) mit dynamischer Massenkopplung, wie in \cite{pascher_higgs_2025} detailliert beschrieben
		\item Fermionenfelder: \( \psi \) mit Yukawa-Kopplung
		\item Eichbosonenfelder: \( A_\mu \) mit \( \Tfield \)-Wechselwirkung
	\end{enumerate}
	
	Diese Erweiterung ermöglicht eine einheitliche Behandlung von massiven Teilchen und Photonen und adressiert einige der grundlegenden Herausforderungen in der aktuellen Physik, wie in \cite{pascher_erweiterung_2025} diskutiert.
	
	\section{Emergente Gravitation aus dem intrinsischen Zeitfeld}
	Eine der bedeutendsten Implikationen des T0-Modells ist, dass Gravitation auf natürliche Weise aus der Dynamik des intrinsischen Zeitfeldes hervorgeht, wodurch die Notwendigkeit einer separaten fundamentalen Kraft entfällt.
	
	\begin{theorem}[Emergenz der Gravitation]
		Gravitation entsteht aus Gradienten des intrinsischen Zeitfeldes:
		\begin{equation}
			\nabla \Tfield = -\frac{\hbar}{m^2 c^2} \nabla m
		\end{equation}
		mit dem modifizierten Potential:
		\begin{equation}
			\Phi(r) = -\frac{GM}{r} + \kappa r, \quad \kappa^{\text{SI}} \approx 4,8 \times 10^{-11} \, \text{m/s}^2
		\end{equation}
		
		Diese Formulierung wird in \cite{pascher_emergente_gravitation_2025} detailliert dargestellt und wurde auf die Galaxiendynamik in \cite{pascher_galaxies_2025} angewendet.
	\end{theorem}
	
	\begin{proof}
		Aus \( \Tfield = \frac{\hbar}{m c^2} \) für massive Teilchen erhalten wir:
		\begin{equation}
			\nabla \Tfield = -\frac{\hbar}{m^2 c^2} \nabla m
		\end{equation}
		Mit der Massenvariation in einem Gravitationsfeld, gegeben durch \( m(\vec{r}) = m_0 (1 + \frac{\Phi_g}{c^2}) \), können wir schreiben:
		\begin{equation}
			\nabla m = \frac{m_0}{c^2} \nabla \Phi_g
		\end{equation}
		Daraus folgt:
		\begin{equation}
			\nabla \Tfield \approx -\frac{\hbar}{m_0 c^4} \nabla \Phi_g
		\end{equation}
		
		Die vollständige Herleitung mit allen Zwischenschritten findet sich in \cite{pascher_emergente_gravitation_2025}.
	\end{proof}
	
	\section{Mathematische Grundlagen: Intrinsische Zeit}
	Das Konzept der intrinsischen Zeit ist zentral für das T0-Modell und dient als fundamentale Brücke zwischen Quantenmechanik und Relativitätstheorie.
	
	\begin{theorem}[Intrinsische Zeit]
		Die intrinsische Zeit für ein Teilchen mit Masse \(m\) oder ein Photon mit Energie \(\hbar\omega\) ist definiert als:
		\begin{equation}
			\Tfield = \frac{\hbar}{\max(m c^2, \omega)}
		\end{equation}
		
		In natürlichen Einheiten, wo \(\hbar = c = 1\), vereinfacht sich dies zu \(\Tfield = \frac{1}{m}\) für massive Teilchen und \(\Tfield = \frac{1}{\omega}\) für Photonen, wie in \cite{pascher_alpha_2025} gezeigt.
	\end{theorem}
	
	Diese Formulierung ermöglicht eine einheitliche Behandlung sowohl von massiven Teilchen als auch von Photonen und adressiert damit eine der zentralen Herausforderungen in der Quantenfeldtheorie, wie in \cite{pascher_photons_2025} diskutiert.
	
	\section{Modifizierte Ableitungsoperatoren}
	Um das intrinsische Zeitfeld in den Standardformalismus der Quantenfeldtheorie zu integrieren, müssen wir die konventionellen Ableitungsoperatoren modifizieren.
	
	\begin{definition}[Modifizierte Ableitung]
		Die modifizierte kovariante Ableitung im T0-Modell ist definiert als:
		\begin{equation}
			\DcovT{\Psi} = \partial_\mu \Psi + \Psi \partial_\mu \Tfield
		\end{equation}
		wobei \(\Psi\) ein beliebiges Quantenfeld und \(\Tfield\) das intrinsische Zeitfeld ist.
		
		Diese Modifikation ist eine zentrale Komponente des mathematischen Rahmens des T0-Modells, wie in \cite{pascher_feldtheorie_2025} detailliert dargestellt.
	\end{definition}
	
	\section{Modifizierte Feldgleichungen}
	Die Standardgleichungen der Quantenmechanik müssen reformuliert werden, um das intrinsische Zeitfeld zu berücksichtigen. Die grundlegendste davon ist die Schrödinger-Gleichung.
	
	\begin{theorem}[Modifizierte Schrödinger-Gleichung]
		Im T0-Modell nimmt die Schrödinger-Gleichung die Form an:
		\begin{equation}
			i\hbar \Tfield \frac{\partial}{\partial t} \Psi + i\hbar \Psi \frac{\partial \Tfield}{\partial t} = \hat{H} \Psi
		\end{equation}
		
		Diese Gleichung spiegelt die Kopplung zwischen der Quantenwellenfunktion und dem intrinsischen Zeitfeld wider, wie in \cite{pascher_zeit_2025} diskutiert.
	\end{theorem}
	
	\section{Modifizierte Lagrange-Dichte für das Higgs-Feld}
	Der Higgs-Mechanismus spielt eine zentrale Rolle im T0-Modell und dient als Brücke zwischen dem intrinsischen Zeitfeld und den Teilchenmassen.
	
	\begin{theorem}[Higgs-Lagrange-Dichte]
		Die Lagrange-Dichte des Higgs-Feldes mit Kopplung an \(\Tfield\) lautet:
		\begin{multline}
			\mathcal{L}_{\text{Higgs-T}} = |\DhiggsT|^2 + \frac{1}{2} \partial_\mu \Tfield \partial^\mu \Tfield - V(\Tfield, \Phi), \quad \\
			\DhiggsT = \Tfield (\partial_\mu + ig A_\mu) \Phi + \Phi \partial_\mu \Tfield
		\end{multline}
		
		Die detaillierte Herleitung und Implikationen dieser Lagrange-Dichte werden in \cite{pascher_higgs_2025} präsentiert.
	\end{theorem}
	
	\section{Modifizierte Lagrange-Dichte für Fermionen}
	Die Fermionenfelder müssen ebenfalls reformuliert werden, um ihre Kopplung an das intrinsische Zeitfeld zu berücksichtigen.
	
	\begin{theorem}[Fermionen-Lagrange-Dichte]
		Die Lagrange-Dichte für Fermionen im T0-Modell ist:
		\begin{equation}
			\mathcal{L}_{\text{Fermion}} = \bar{\psi} i \gamma^\mu (\partial_\mu \psi + \psi \partial_\mu \Tfield) - y \bar{\psi} \Phi \psi
		\end{equation}
		
		Diese Formulierung zeigt, wie Fermionenmassen aus der Yukawa-Kopplung an das Higgs-Feld entstehen, während sie auch mit dem intrinsischen Zeitfeld wechselwirken, wie in \cite{pascher_higgs_2025} detailliert dargestellt.
	\end{theorem}
	
	\section{Modifizierte Lagrange-Dichte für Eichbosonen}
	Die Eichfelder im T0-Modell erfordern ebenfalls eine Reformulierung, um ihre Wechselwirkung mit dem intrinsischen Zeitfeld zu berücksichtigen.
	
	\begin{theorem}[Eichbosonen-Lagrange-Dichte]
		Die Lagrange-Dichte für Eichbosonen im T0-Modell ist:
		\begin{equation}
			\mathcal{L}_{\text{Boson}} = -\frac{1}{4} F_{\mu\nu} F^{\mu\nu} + \frac{1}{2} \partial_\mu \Tfield \partial^\mu \Tfield
		\end{equation}
		
		Diese Formulierung erhält die Eichinvarianz, während sie die Dynamik des intrinsischen Zeitfeldes einbezieht, wie in \cite{pascher_feldtheorie_2025} diskutiert.
	\end{theorem}
%------	

\section{Vollständige Gesamt-Lagrange-Dichte}
Durch Kombination aller Komponenten können wir die vollständige Lagrange-Dichte für das T0-Modell formulieren.

\begin{theorem}[Gesamt-Lagrange-Dichte]
	Die vollständige Lagrange-Dichte ist:
	\begin{equation}
		\mathcal{L}_{\text{Gesamt}} = \mathcal{L}_{\text{Boson}} + \mathcal{L}_{\text{Fermion}} + \mathcal{L}_{\text{Higgs-T}} + \mathcal{L}_{\text{intrinsisch}},
	\end{equation}
	wobei die vollständige intrinsische Lagrange-Dichte freie Felddynamik und Materiewechselwirkungen kombiniert:
	\begin{equation}
		\mathcal{L}_{\text{intrinsisch}}^{\text{vollständig}} = \underbrace{\frac{1}{2} \partial_\mu \Tfield \partial^\mu \Tfield - \frac{1}{2}\Tfield^2}_{\text{Freie Felddynamik}} + \underbrace{\bar{\psi} \left( i\hbar \gamma^0 \frac{\partial}{\partial (t/\Tfield)} - i\hbar \gamma^0 \frac{\partial}{\partial t} \right) \psi}_{\text{Wechselwirkung mit Materie}}
	\end{equation}
	
	In Anwendungen, die sich ausschließlich auf Feldausbreitung konzentrieren, wird häufig die freie Feldkomponente verwendet:
	\begin{equation}
		\mathcal{L}_{\text{intrinsisch}}^{\text{Feld}} = \frac{1}{2} \partial_\mu \Tfield \partial^\mu \Tfield - \frac{1}{2} \Tfield^2.
	\end{equation}
\end{theorem}
%------	
	\section{Kosmologische Implikationen}
	Das T0-Modell hat tiefgreifende Implikationen für die Kosmologie und bietet alternative Erklärungen für mehrere beobachtete Phänomene, ohne dunkle Energie oder kosmische Inflation zu benötigen.
	
	Die wichtigsten kosmologischen Vorhersagen des T0-Modells umfassen:
	\begin{itemize}
		\item Modifiziertes Gravitationspotential: \( \Phi(r) = -\frac{GM}{r} + \kappa r \), mit \( \kappa^{\text{SI}} \approx 4,8 \times 10^{-11} \, \text{m/s}^2 \), wie in \cite{pascher_emergente_gravitation_2025} hergeleitet
		\item Kosmische Rotverschiebung: \( 1 + z = e^{\alpha d} \), mit \( \alpha^{\text{SI}} \approx 2,3 \times 10^{-18} \, \text{m}^{-1} \), wie in \cite{pascher_galaxies_2025} erklärt
		\item Wellenlängenabhängigkeit: \( z(\lambda) = z_0 (1 + \betaT^{\text{SI}} \ln(\lambda/\lambda_0)) \), mit \( \betaT^{\text{SI}} \approx 0,008 \) in SI-Einheiten, wie in \cite{pascher_messdifferenzen_2025} detailliert dargestellt
	\end{itemize}
	
	Diese kosmologischen Implikationen bieten testbare Vorhersagen, die das T0-Modell vom kosmologischen Standardmodell unterscheiden und Möglichkeiten zur experimentellen Verifizierung bieten, wie in \cite{pascher_messdifferenzen_2025} diskutiert.
	
	\section{Herleitung von \(\betaT\) im T0-Modell}
	Der Parameter \(\betaT\) spielt eine entscheidende Rolle im T0-Modell und beschreibt die Kopplung des intrinsischen Zeitfeldes \(\Tfield\) an physikalische Phänomene wie die wellenlängenabhängige Rotverschiebung. Im T0-Modell wird \(\betaT\) präzise hergeleitet als:
	\begin{equation}
		\betaT^{\text{nat}} = \frac{\lambda_h^2 v^2}{16\pi^3 m_h^2 \xi}{16\pi^3} \cdot \frac{1}{m_h^2} \cdot \frac{1}{\xi}
	\end{equation}
	wobei \(\lambda_h\) die Higgs-Selbstkopplung, \(v\) der Higgs-Vakuum-Erwartungswert, \(m_h\) die Higgs-Masse und \(\xi \approx 1,33 \times 10^{-4}\) ein dimensionsloser Parameter ist, der die charakteristische Längenskala \(r_0 = \xi \cdot l_P\) (\(l_P\): Planck-Länge) definiert.
	
	In natürlichen Einheiten gilt exakt \(\betaT^{\text{nat}} = 1\), was eine theoretische Vorhersage darstellt, die direkt aus den Modellparametern abgeleitet wird. Mit \(\betaT^{\text{nat}} = 1\) können wir den Wert von \(\xi\) bestimmen:
	\begin{equation}
		\xi = \frac{\lambda_h^2 v^2}{16\pi^3 m_h^2} \approx 1,33 \times 10^{-4}
	\end{equation}
	
	Dieser Wert von \(\xi\) verknüpft die charakteristische T0-Länge \(r_0\) mit der Planck-Länge durch \(r_0 = \xi \cdot l_P\). In SI-Einheiten konvertiert sich der Parameterwert zu \(\betaT^{\text{SI}} \approx 0,008\), wie in \cite{pascher_alphabeta_2025} und \cite{pascher_params_2025} detailliert dargestellt.
	
	\section{Schlussfolgerung und Ausblick}
	Die in dieser Arbeit vorgestellten mathematischen Formulierungen bilden eine rigorose Grundlage für die Zeit-Masse-Dualitätstheorie und das T0-Modell. Durch die Reformulierung der konventionellen Gleichungen der Quantenfeldtheorie zur Einbeziehung des intrinsischen Zeitfeldes \(\Tfield\) etablieren wir einen Rahmen, der neue Perspektiven auf die fundamentale Physik und Kosmologie bietet.
	
	Zu den wichtigsten Errungenschaften dieser Formulierung gehören:
	\begin{itemize}
		\item Eine einheitliche Behandlung von massiven Teilchen und Photonen durch das Konzept des intrinsischen Zeitfeldes
		\item Die Emergenz der Gravitation aus der Dynamik des intrinsischen Zeitfeldes, wodurch die Notwendigkeit einer separaten fundamentalen Kraft entfällt
		\item Eine konsistente Erklärung für kosmologische Beobachtungen ohne die Notwendigkeit dunkler Energie oder kosmischer Inflation
		\item Testbare Vorhersagen, die das T0-Modell vom Standardmodell der Physik unterscheiden
	\end{itemize}
	
	Zukünftige Arbeiten werden sich auf die weitere Verfeinerung dieser mathematischen Formulierungen und die Entwicklung spezifischer Anwendungen zur Bewältigung aktueller Herausforderungen in der theoretischen Physik konzentrieren, wie in \cite{pascher_galaxies_2025} und \cite{pascher_feldtheorie_2025} dargelegt.
	
	\begin{thebibliography}{99}
		\bibitem{pascher_zeit_2025} Pascher, J. (2025). \href{https://github.com/jpascher/T0-Time-Mass-Duality/tree/main/2/pdf/Deutsch/ZeitEmergentQM.pdf}{Zeit als emergente Eigenschaft in der Quantenmechanik: Eine Verbindung zwischen Relativität, Feinstrukturkonstante und Quantendynamik}. 23. März 2025.
		\bibitem{pascher_lagrange_2025} Pascher, J. (2025). \href{https://github.com/jpascher/T0-Time-Mass-Duality/tree/main/2/pdf/Deutsch/MathZeitMasseLagrange.pdf}{Von Zeitdilatation zur Massenvariation: Mathematische Kernformulierungen der Zeit-Masse-Dualitätstheorie}. 29. März 2025.
		\bibitem{pascher_photons_2025} Pascher, J. (2025). \href{https://github.com/jpascher/T0-Time-Mass-Duality/tree/main/2/pdf/Deutsch/DynMassePhotonenNichtlokal.pdf}{Dynamische Masse von Photonen und ihre Implikationen für Nichtlokalität im T0-Modell}. 25. März 2025.
		\bibitem{pascher_erweiterung_2025} Pascher, J. (2025). \href{https://github.com/jpascher/T0-Time-Mass-Duality/tree/main/2/pdf/Deutsch/NotwendigkeitQMErweiterung.pdf}{Die Notwendigkeit der Erweiterung der Standard-Quantenmechanik und Quantenfeldtheorie}. 27. März 2025.
		\bibitem{pascher_galaxies_2025} Pascher, J. (2025). \href{https://github.com/jpascher/T0-Time-Mass-Duality/tree/main/2/pdf/Deutsch/MassVarGalaxien.pdf}{MassenVariation in Galaxien: Eine Analyse im T0-Modell mit emergenter Gravitation}. 30. März 2025.
		\bibitem{pascher_higgs_2025} Pascher, J. (2025). \href{https://github.com/jpascher/T0-Time-Mass-Duality/tree/main/2/pdf/Deutsch/MathHiggsZeitMasse.pdf}{Mathematische Formulierung des Higgs-Mechanismus in der Zeit-Masse-Dualität}. 28. März 2025.
		\bibitem{pascher_feldtheorie_2025} Pascher, J. (2025). \href{https://github.com/jpascher/T0-Time-Mass-Duality/tree/main/2/pdf/Deutsch/FeldtheorieQuanten.pdf}{Feldtheorie und Quantenkorrelationen: Eine neue Perspektive auf Instantaneität}. 28. März 2025.
		\bibitem{pascher_messdifferenzen_2025} Pascher, J. (2025). \href{https://github.com/jpascher/T0-Time-Mass-Duality/tree/main/2/pdf/Deutsch/MessdifferenzenT0Standard.pdf}{Kompensatorische und additive Effekte: Eine Analyse der Messunterschiede zwischen dem T0-Modell und dem \(\Lambda\)CDM-Standardmodell}. 2. April 2025.
		\bibitem{pascher_planck_2025} Pascher, J. (2025). \href{https://github.com/jpascher/T0-Time-Mass-Duality/tree/main/2/pdf/Deutsch/JenseitsPlanck.pdf}{Reale Konsequenzen der Neuformulierung von Zeit und Masse in der Physik: Jenseits der Planck-Skala}. 24. März 2025.
		\bibitem{pascher_alpha_2025} Pascher, J. (2025). \href{https://github.com/jpascher/T0-Time-Mass-Duality/tree/main/2/pdf/Deutsch/NatEinheitenAlpha1.pdf}{Energie als fundamentale Einheit: Natürliche Einheiten mit \(\alphaEM = 1\) im T0-Modell}. 26. März 2025.
		\bibitem{pascher_alphabeta_2025} Pascher, J. (2025). \href{https://github.com/jpascher/T0-Time-Mass-Duality/tree/main/2/pdf/Deutsch/Alpha1Beta1Konsistenz.pdf}{Einheitliches Einheitensystem im T0-Modell: Die Konsistenz von \(\alpha = 1\) und \(\beta = 1\)}. 5. April 2025.
		\bibitem{pascher_temp_2025} Pascher, J. (2025). \href{https://github.com/jpascher/T0-Time-Mass-Duality/tree/main/2/pdf/Deutsch/TempEinheitenCMB.pdf}{Anpassung der Temperatureinheiten in natürlichen Einheiten und CMB-Messungen}. 2. April 2025.
		\bibitem{pascher_params_2025} Pascher, J. (2025). \href{https://github.com/jpascher/T0-Time-Mass-Duality/tree/main/2/pdf/Deutsch/ZeitMasseT0Params.pdf}{Zeit-Masse-Dualitätstheorie (T0-Modell): Ableitung der Parameter \(\kappa\), \(\alpha\) und \(\beta\)}. 4. April 2025.
		\bibitem{pascher_emergente_gravitation_2025} Pascher, J. (2025). \href{https://github.com/jpascher/T0-Time-Mass-Duality/tree/main/2/pdf/Deutsch/EmergentGravT0.pdf}{Emergente Gravitation im T0-Modell: Eine umfassende Ableitung}. 1. April 2025.
		\bibitem{pascher_zeit_masse_2025} Pascher, J. (2025). \href{https://github.com/jpascher/T0-Time-Mass-Duality/tree/main/2/pdf/Deutsch/ZeitMasseNeuerBlick.pdf}{Zeit und Masse: Ein neuer Blick auf alte Formeln – und Befreiung von traditionellen Zwängen}. 22. März 2025.
		\bibitem{einstein1905} Einstein, A. (1905). Zur Elektrodynamik bewegter Körper. \textit{Annalen der Physik}, 322(10), 891-921.
		\bibitem{higgs1964} Higgs, P. W. (1964). Broken Symmetries and the Masses of Gauge Bosons. \textit{Physical Review Letters}, 13(16), 508-509.
		\bibitem{bohr1928} Bohr, N. (1928). The Quantum Postulate and the Recent Development of Atomic Theory. \textit{Nature}, 121(3050), 580-590.
	\end{thebibliography}
	
\end{document}