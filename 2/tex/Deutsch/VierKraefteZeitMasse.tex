\documentclass[a4paper,12pt]{article}
\usepackage[utf8]{inputenc}
\usepackage[T1]{fontenc}
\usepackage{lmodern}
\usepackage[ngerman]{babel}
\usepackage{amsmath, amssymb, amsthm}
\usepackage{geometry}
\usepackage{xcolor}
\usepackage{tocloft}
\usepackage{siunitx}
\DeclareSIUnit{\year}{yr}
\DeclareSIUnit{\parsec}{pc}
\usepackage{fancyhdr}

\usepackage{hyperref}
\hypersetup{
	colorlinks=true,
	linkcolor=blue,
	filecolor=blue,
	citecolor=blue,
	urlcolor=blue,
	bookmarks=true,
	bookmarksopen=true,
	pdftitle={Vereinfachte Beschreibung fundamentaler Kräfte mit Zeit-Masse-Dualität},
	pdfauthor={Johann Pascher},
}

\usepackage{cleveref}

\geometry{a4paper, margin=2cm}

% Kopf- und Fußzeilen
\pagestyle{fancy}
\fancyhf{}
\fancyhead[L]{Johann Pascher}
\fancyhead[R]{Zeit-Masse-Dualität}
\fancyfoot[C]{\thepage}
\renewcommand{\headrulewidth}{0.4pt}
\renewcommand{\footrulewidth}{0.4pt}

\renewcommand{\cftsecfont}{\color{blue}}
\renewcommand{\cftsubsecfont}{\color{blue}}
\renewcommand{\cftsecpagefont}{\color{blue}}
\renewcommand{\cftsubsecpagefont}{\color{blue}}
\setlength{\cftsecindent}{1cm}
\setlength{\cftsubsecindent}{2cm}

% Benutzerdefinierte Befehle
\newcommand{\Tfield}{T(x)}
\newcommand{\DcovT}[1]{\Tfield D_\mu #1 + #1 \partial_\mu \Tfield}
\newcommand{\DhiggsT}{\Tfield (\partial_\mu + ig A_\mu) \Phi + \Phi \partial_\mu \Tfield}
\newcommand{\betaT}{\beta_{\text{T}}}
\newcommand{\alphaEM}{\alpha_{\text{EM}}}
\newcommand{\alphaW}{\alpha_{\text{W}}}
\newcommand{\Mpl}{M_{\text{Pl}}}
\newcommand{\Tzerot}{T_0(\Tfield)}
\newcommand{\Tzero}{T_0}
\newcommand{\vecx}{\vec{x}}
\newcommand{\gammaf}{\gamma_{\text{Lorentz}}}

\title{Vereinfachte Beschreibung fundamentaler Kräfte mit Zeit-Masse-Dualität}
\author{Johann Pascher}
\date{27. März 2025}

\begin{document}
	
	\maketitle
	
	\tableofcontents
	\newpage
	
	\section{Vereinheitlichte Lagrange-Dichte mit dualem \\Zeit-Masse-Konzept}
	
	Die Physik beschreibt die Welt durch vier fundamentale Kräfte – starke, schwache, elektromagnetische und Gravitationskraft – die traditionell getrennt betrachtet werden. Im T0-Modell, das auf der Zeit-Masse-Dualität basiert, können diese Kräfte jedoch in einer einzigen Lagrange-Dichte vereint werden, die natürlicherweise sowohl bekannte Wechselwirkungen als auch die Gravitation umfasst. Diese Dichte ist gegeben durch:
	
	\begin{equation}
		\mathcal{L}_\text{total} = \mathcal{L}_\text{SM} + \mathcal{L}_\text{Higgs} + \mathcal{L}_\text{intrinsisch}
	\end{equation}
	
	Hier repräsentiert \(\mathcal{L}_\text{SM}\) die Wechselwirkungen des Standardmodells – die starke, elektromagnetische und schwache Kraft –, \(\mathcal{L}_\text{Higgs}\) beschreibt die Dynamik des Higgs-Feldes, und \(\mathcal{L}_\text{intrinsisch}\) führt das Konzept der intrinsischen Zeit ein, das die Zeit-Masse-Dualität widerspiegelt. Bemerkenswert ist, dass die Gravitation nicht als separate Kraft hinzugefügt wird, sondern aus der Dynamik des intrinsischen Zeitfeldes hervorgeht, wie in ''Mathematische Kernformulierungen'' \cite{pascher_lagrange_2025} und ''Emergente Gravitation im T0-Modell'' \cite{pascher_emergente_gravitation_2025} detailliert beschrieben.
	
	\subsection{Standardmodell}
	
	Das Standardmodell bildet die Grundlage für die Beschreibung der drei Kräfte, die das Teilchenverhalten auf atomarer Ebene bestimmen. Seine Lagrange-Dichte setzt sich zusammen aus:
	
	\begin{equation}
		\mathcal{L}_\text{SM} = \mathcal{L}_\text{stark} + \mathcal{L}_\text{em} + \mathcal{L}_\text{schwach}
	\end{equation}
	
	Hier steht \(\mathcal{L}_\text{stark} = -\frac{1}{4} F_{\mu\nu}^a F^{a\mu\nu} + \bar{\psi}(i \gamma^\mu D_\mu - m_\psi(\phi))\psi\) für die starke Kernkraft, die Quarks zu Protonen und Neutronen bindet; \(\mathcal{L}_\text{em} = -\frac{1}{4} F_{\mu\nu} F^{\mu\nu} + \bar{\psi}(i \gamma^\mu D_\mu - m_\psi(\phi))\psi\) repräsentiert die elektromagnetische Kraft, die Elektronen an Kerne koppelt; und \(\mathcal{L}_\text{schwach} = -\frac{1}{4} W_{\mu\nu}^a W^{a\mu\nu} + \bar{\psi}(i \gamma^\mu D_\mu - m_\psi(\phi))\psi\) repräsentiert die schwache Kraft, die Prozesse wie den radioaktiven Zerfall steuert. Diese konventionelle Beschreibung folgt der standardmäßigen Quantenfeldtheorie-Formulierung \cite{weinberg1995quantum}.
	
	Im T0-Modell wird diese Beschreibung angepasst, indem die Zeitdilatation durch Massenvariation ersetzt wird, was zu einer dualen Formulierung führt:
	
	\begin{equation}
		\mathcal{L}_\text{SM-T} = \mathcal{L}_\text{stark-T} + \mathcal{L}_\text{em-T} + \mathcal{L}_\text{schwach-T}
	\end{equation}
	
	Hier ist die Zeitableitung an die intrinsische Zeit \(\Tfield\) gebunden, sodass \(\partial_t \rightarrow \partial_{t/T}\), eine Anpassung, die die Dynamik unter absoluter Zeit neu interpretiert. Diese Modifikation ist grundlegend für den Ansatz des T0-Modells und steht in direkter Verbindung zum Konzept der Zeit als emergente Eigenschaft, wie in \cite{pascher_zeit_2025} beschrieben.
	
	\subsection{Higgs-Feld}
	
	Das Higgs-Feld, verantwortlich für die Massenerzeugung, wird im Standardmodell beschrieben durch:
	
	\begin{equation}
		\mathcal{L}_\text{Higgs} = (D_\mu \phi)^\dagger (D^\mu \phi) - V(\phi)
	\end{equation}
	
	wobei \(\phi\) das Higgs-Feld und \(V(\phi) = \mu^2 \phi^\dagger \phi + \lambda (\phi^\dagger \phi)^2\) das Potential ist. Diese Formulierung folgt den ursprünglichen Arbeiten von Higgs, Englert und Brout \cite{higgs1964broken, englert1964broken}.
	
	Im T0-Modell wird diese Formel erweitert, um die intrinsische Zeit einzubeziehen:
	
	\begin{equation}
		\mathcal{L}_\text{Higgs-T} = (D_{T\mu} \phi_T)^\dagger (D_T^\mu \phi_T) - V_T(\phi_T)
	\end{equation}
	
	Die kovariante Ableitung \(D_{T\mu}\) berücksichtigt die Zeit-Masse-Dualität und betont die Rolle des Higgs-Feldes als Medium für Masse und Zeit, wie in ''Mathematische Formulierung des Higgs-Mechanismus'' \cite{pascher_higgs_2025} ausgeführt. Diese Modifikation offenbart die tiefere Verbindung zwischen Massenerzeugung und der intrinsischen Zeitskala von Teilchen, eine Schlüsselerkenntnis des T0-Modells, die über die Standardinterpretation des Higgs-Mechanismus hinausgeht.
	%-----	
	\subsection{Lagrange-Dichte für intrinsische Zeit}
	
	Die zentrale Innovation des T0-Modells ist die vollständige Lagrange-Dichte für intrinsische Zeit, die in ihrer kompletten Form wie folgt ausgedrückt werden kann:
	
	\begin{equation}
		\mathcal{L}_{\text{intrinsisch}}^{\text{vollständig}} = \underbrace{\frac{1}{2} \partial_\mu \Tfield \partial^\mu \Tfield - \frac{1}{2}\Tfield^2}_{\text{Freie Felddynamik}} + \underbrace{\bar{\psi} \left( i\hbar \gamma^0 \frac{\partial}{\partial (t/\Tfield)} - i\hbar \gamma^0 \frac{\partial}{\partial t} \right) \psi}_{\text{Wechselwirkung mit Materie}}
	\end{equation}
	
	Diese Formulierung umfasst sowohl die freie Felddynamik von \(\Tfield\) als auch dessen Wechselwirkung mit Materie. In früheren grundlegenden Arbeiten wurde eine vereinfachte Version vorgestellt, die sich nur auf den Materiewechselwirkungsterm konzentrierte:
	
	\begin{equation}
		\mathcal{L}_\text{intrinsisch}^{\text{vereinfacht}} = \bar{\psi} \left( i\hbar \gamma^0 \frac{\partial}{\partial (t/T)} - i\hbar \gamma^0 \frac{\partial}{\partial t} \right) \psi
	\end{equation}
	
	Die vollständige Formulierung offenbart wichtige Konsequenzen: Bei Anwendung des Variationsprinzips erhalten wir die Feldgleichung mit Quellterm:
	
	\begin{equation}
		\nabla^2 \Tfield + \Tfield = -\kappa\rho(x)\Tfield^2
	\end{equation}
	
	Diese Gleichung zeigt, wie die Materiedichte als Quelle für das intrinsische Zeitfeld fungiert und die emergenten Gravitationseffekte erzeugt, die für das T0-Modell zentral sind. Die freie Feldkomponente ermöglicht eine wellenartige Ausbreitung des \(\Tfield\)-Feldes, während die Materiewechselwirkungskomponente dieses Feld an Teilchen koppelt und masseabhängige Dynamik erzeugt.
	
	Hier ist \(\Tfield = \frac{\hbar}{m c^2}\) die intrinsische Zeit, abhängig von der Masse. Diese erweiterte Formulierung, entwickelt in ''Die Notwendigkeit der Erweiterung der Standardquantenmechanik'' \cite{pascher_erweiterung_2025}, verbindet die Teilchendynamik mit ihren individuellen Zeitskalen und ermöglicht eine vereinheitlichte Beschreibung aller Kräfte. Die intrinsische Zeit hat die Dimension \([E^{-1}]\) in natürlichen Einheiten, was die Dimensionskonsistenz im gesamten Formalismus aufrechterhält, wie in \cite{pascher_alpha_2025} und \cite{pascher_alphabeta_2025} diskutiert.
	%--------	
	\section{Vereinfachte Beschreibung von Massentermen mit Zeit-Masse-Dualität}
	
	Im Standardmodell wird die Masse eines Teilchens durch seine Kopplung an das Higgs-Feld definiert: \(m_\psi(\phi) = y_\psi \phi\), wobei die Masse konstant bleibt und die Zeit variabel ist. Im T0-Modell wird diese Sichtweise umgekehrt: Die Zeit bleibt absolut, und die Masse variiert mit dem Lorentz-Faktor \(\gamma\):
	
	\begin{equation}
		m_\psi(\phi_T) = y_\psi \phi_T \cdot \gamma, \quad \gamma = \frac{1}{\sqrt{1 - v^2/c^2}}
	\end{equation}
	
	Diese duale Beschreibung, abgeleitet in ''Zeit-Masse-Dualitätstheorie'' \cite{pascher_params_2025}, erklärt die gleichen Phänomene wie die Zeitdilatation, bietet aber eine neue Perspektive auf die Rolle der Masse. Diese Neuformulierung der Effekte der speziellen Relativitätstheorie bewahrt die experimentellen Vorhersagen von Einsteins Theorie \cite{einstein1905}, bietet jedoch einen konzeptionell anderen Rahmen, der mit der fundamentalen Zeit-Masse-Dualität übereinstimmt, wie in \cite{pascher_zeit_masse_2025} untersucht.
	
	\section{Das Higgs-Feld als universelles Medium mit intrinsischer Zeit}
	
	Das Higgs-Feld ist mehr als ein Mechanismus zur Massenerzeugung – im T0-Modell bestimmt es auch die intrinsischen Zeitskalen der Teilchen. Diese Beziehung wird ausgedrückt als:
	
	\begin{equation}
		\Tfield = \frac{\hbar}{m(\phi) c^2} = \frac{\hbar}{y_\psi \phi \cdot c^2}
	\end{equation}
	
	Die intrinsische Zeit eines Teilchens ist also umgekehrt proportional zu seiner Masse, die durch das Higgs-Feld erzeugt wird. Diese Perspektive erweitert die Rolle des Higgs-Feldes als universelles Medium, das alle Wechselwirkungen beeinflusst, wie in ''Higgs-Mechanismus'' \cite{pascher_higgs_2025} untersucht. Die einzigartige Position des Higgs-Bosons im Teilchenspektrum gewinnt in diesem Rahmen neue Bedeutung, da es nicht nur die Masse, sondern auch die fundamentalen zeitlichen Eigenschaften aller anderen Teilchen vermittelt.
	
	In natürlichen Einheiten, wo \(\hbar = c = 1\), vereinfacht sich diese Beziehung zu \(\Tfield = \frac{1}{m}\), was die fundamentale Dualität zwischen Zeit und Masse hervorhebt. Wenn zusätzlich \(\alphaEM = \betaT = 1\) im vereinheitlichten natürlichen Einheitensystem gesetzt wird, offenbart diese Beziehung weitere Eleganz und Einfachheit, wie in \cite{pascher_alphabeta_2025} diskutiert.
	
	\section{Das Higgs-Feld und das Vakuum: Eine komplexe Beziehung mit intrinsischer Zeit}
	
	Die Vakuumenergie, ein zentrales Problem der modernen Physik, wird im T0-Modell neu interpretiert. Statt einer Summe von Nullpunktenergien könnte sie beschrieben werden als:
	
	\begin{equation}
		E_\text{Vakuum} = \sum_i \frac{\hbar}{2 T_i}
	\end{equation}
	
	wobei \(T_i\) die intrinsische Zeit der Quantenfluktuationen ist. Diese Formulierung verbindet die Vakuumenergie mit der Dynamik des Higgs-Feldes und der Zeit-Masse-Dualität und bietet neue Einblicke in das Problem der kosmologischen Konstante \cite{weinberg1989cosmological}. Indem die Vakuumenergie direkt mit den intrinsischen Zeitskalen der Quantenfluktuationen verbunden wird, bietet das T0-Modell einen potenziellen Weg zur Lösung der enormen Diskrepanz zwischen den Vorhersagen der Quantenfeldtheorie und astronomischen Beobachtungen der Vakuumenergie.
	
	\section{Quantenverschränkung und Nichtlokalität in der Zeit-Masse-Dualität}
	
	Die scheinbare Unmittelbarkeit der Quantenverschränkung wird im T0-Modell durch intrinsische Zeit neu betrachtet. Im \(T_0\)-Modell entstehen Korrelationen nicht instantan, sondern durch Massenvariationen. Bei verschränkten Teilchen mit unterschiedlichen Massen variiert die Zeitentwicklung mit ihren intrinsischen Zeiten. Für Photonen wird dies definiert als:
	
	\begin{equation}
		\Tfield = \frac{\hbar}{E_{\gamma}} e^{\alpha^{\text{SI}} x}, \quad \alpha^{\text{SI}} = \frac{H_0}{c} \approx \SI{2,3e-18}{\per\meter}
	\end{equation}
	
	was den Energieverlust über Entfernungen widerspiegelt, wie in ''Dynamische Masse von Photonen'' \cite{pascher_photons_2025} beschrieben. Dieser Ansatz bietet eine neuartige Perspektive auf die Paradoxa der Quantennichtlokalität \cite{bell1964}, indem die scheinbare instantane Wirkung über Entfernungen in Bezug auf die variierenden intrinsischen Zeitskalen von Quantensystemen neu interpretiert wird. Weitere Implikationen für Quantenkorrelationen und das Bell-Theorem werden in \cite{pascher_feldtheorie_2025} untersucht.
	
	\section{Kosmologische Implikationen der Zeit-Masse-Dualität}
	
	Das T0-Modell bietet natürliche Erklärungen für kosmologische Phänomene durch drei Schlüsselparameter: \(\alpha^{\text{SI}} \approx \SI{2,3e-18}{\per\meter}\) beschreibt den Energieverlust von Photonen über kosmische Entfernungen, \(\kappa^{\text{SI}} \approx \SI{4,8e-11}{\meter\per\second\squared}\) charakterisiert die Stärke des Dunkle-Energie-Feldes in der galaktischen Dynamik, und \(\betaT^{\text{SI}} \approx 0,008\) quantifiziert die Kopplung an baryonische Materie. Das Gravitationspotential wird:
	
	\begin{equation}
		\Phi(r) = -\frac{G M}{r} + \kappa r
	\end{equation}
	
	wobei \(\kappa\) die Dimension \([E]\) in natürlichen Einheiten hat. Diese Parameter, abgeleitet in ''Massenvariation in Galaxien'' \cite{pascher_galaxies_2025} und ''Messdifferenzen'' \cite{pascher_messdifferenzen_2025}, erklären flache Rotationskurven und Rotverschiebung ohne dunkle Materie oder kosmische Expansion zu benötigen.
	
	Dieses modifizierte Gravitationspotential stimmt mit den Beobachtungsdaten von Galaxien-Rotationskurven \cite{rubin1980} und der radialen Beschleunigungsrelation \cite{McGaugh2016} überein und bietet gleichzeitig eine sparsamere Erklärung als Modelle mit dunkler Materie. Der lineare Term \(\kappa r\) im Potential führt zu einer zusätzlichen konstanten Kraftkomponente, die bei großen Entfernungen dominiert und die Abflachung der Galaxien-Rotationskurven auf natürliche Weise erklärt.
	
	Darüber hinaus bietet die Interpretation der kosmischen Rotverschiebung im T0-Modell als Energieverlust statt als Expansion eine Alternative zur Standard-\(\Lambda\)CDM-Kosmologie \cite{Planck2018} und löst potenziell Spannungen in Hubble-Konstanten-Messungen, ohne dunkle Energie zu benötigen.
	
	\section{Zusammenfassung der vereinheitlichten Theorie}
	
	Die vereinheitlichte Theorie wird durch die Wirkung beschrieben:
	
	\begin{equation}
		S_\text{vereinheitlicht} = \int \left( \mathcal{L}_\text{standard} + \mathcal{L}_\text{komplementär} + \mathcal{L}_\text{kopplung} \right) d^4x
	\end{equation}
	
	wobei \(\mathcal{L}_\text{standard}\) das Standardmodell, \(\mathcal{L}_\text{komplementär}\) die duale Formulierung und \(\mathcal{L}_\text{kopplung}\) die Zeit-Masse-Wechselwirkung ist. Dieser Ansatz überbrückt Quantenmechanik und Gravitation, bietet neue Einblicke in Verschränkung und kosmologische Phänomene und ist experimentell überprüfbar.
	
	Das T0-Modell benötigt keine exotischen Komponenten wie dunkle Materie oder dunkle Energie, sondern erklärt diese Phänomene durch die fundamentalen Eigenschaften des intrinsischen Zeitfeldes. Diese Vereinheitlichung scheinbar disparater physikalischer Phänomene durch einen einzigen konzeptionellen Rahmen stellt einen bedeutenden Schritt in Richtung eines kohärenteren Verständnisses der Natur dar.
	
	\section{Experimentelle Überprüfbarkeit}
	
	Das T0-Modell macht mehrere spezifische, experimentell überprüfbare Vorhersagen, die es vom Standardmodell und der konventionellen Kosmologie unterscheiden:
	
	\begin{itemize}
		\item \textbf{Photonen-Energieverlust:} Der Parameter \(\alpha^{\text{SI}} \approx \SI{2,3e-18}{\per\meter}\) sagt einen entfernungsabhängigen Energieverlust für Photonen voraus, der durch Präzisionsspektroskopie entfernter Quellen getestet werden könnte.
		
		\item \textbf{Modifiziertes Gravitationspotential:} Der Parameter \(\kappa^{\text{SI}} \approx \SI{4,8e-11}{\meter\per\second\squared}\) führt zu Abweichungen von der newtonschen Gravitation bei großen Entfernungen, die durch sorgfältige Messungen der Galaxiendynamik oder der Ephemeriden des Sonnensystems erkannt werden könnten.
		
		\item \textbf{Wellenlängenabhängige Rotverschiebung:} Das Modell sagt eine logarithmische Abhängigkeit der Rotverschiebung von der Wellenlänge voraus, charakterisiert durch \(\betaT^{\text{SI}} \approx 0,008\), was mit Multiwellenlängen-Beobachtungen entfernter Galaxien getestet werden könnte.
		
		\item \textbf{Massenabhängige Quantenkohärenz:} Die Kopplung des intrinsischen Zeitfeldes sagt voraus, dass Quantenkohärenzzeiten von der Masse abhängen sollten, was durch Interferenzexperimente mit Teilchen unterschiedlicher Masse verifiziert werden könnte.
	\end{itemize}
	
	Diese Vorhersagen sind in ''Parameterableitungen'' \cite{pascher_params_2025} und ''Messdifferenzen'' \cite{pascher_messdifferenzen_2025} detailliert beschrieben und bieten einen klaren Weg zur experimentellen Verifizierung oder Falsifizierung des T0-Modells.
	
	\section{Verweise auf weitere Arbeiten}
	
	Diese Theorie baut auf meinen früheren Arbeiten auf, die in der Bibliographie aufgeführt sind und verschiedene Aspekte der Zeit-Masse-Dualität eingehend untersuchen. Zusammen bilden diese Arbeiten einen umfassenden Rahmen, der fundamentale Fragen der Physik aus einer neuartigen Perspektive behandelt und potenzielle Lösungen für lang bestehende Probleme in der Quantengravitation, Kosmologie und der Vereinheitlichung fundamentaler Kräfte bietet.
	
	\begin{thebibliography}{99}
		\bibitem{pascher_zeit_2025} Pascher, J. (2025). \href{https://github.com/jpascher/T0-Time-Mass-Duality/tree/main/2/pdf/Deutsch/ZeitEmergentQM.pdf}{Zeit als emergente Eigenschaft in der Quantenmechanik: Eine Verbindung zwischen Relativitätstheorie, Feinstrukturkonstante und Quantendynamik}. 23. März 2025.
		\bibitem{pascher_zeit_masse_2025} Pascher, J. (2025). \href{https://github.com/jpascher/T0-Time-Mass-Duality/tree/main/2/pdf/Deutsch/ZeitMasseNeuerBlick.pdf}{Zeit und Masse: Ein neuer Blick auf alte Formeln – und Befreiung von traditionellen Einschränkungen}. 22. März 2025.
		\bibitem{pascher_params_2025} Pascher, J. (2025). \href{https://github.com/jpascher/T0-Time-Mass-Duality/tree/main/2/pdf/Deutsch/ZeitMasseT0Params.pdf}{Zeit-Masse-Dualitätstheorie (T0-Modell): Ableitung der Parameter \(\kappa\), \(\alpha\) und \(\beta\)}. 4. April 2025.
		\bibitem{pascher_galaxies_2025} Pascher, J. (2025). \href{https://github.com/jpascher/T0-Time-Mass-Duality/tree/main/2/pdf/Deutsch/MassVarGalaxien.pdf}{Massenvariation in Galaxien: Eine Analyse im T0-Modell mit emergenter Gravitation}. 30. März 2025.
		\bibitem{pascher_messdifferenzen_2025} Pascher, J. (2025). \href{https://github.com/jpascher/T0-Time-Mass-Duality/tree/main/2/pdf/Deutsch/MessdifferenzenT0Standard.pdf}{Kompensatorische und additive Effekte: Eine Analyse der Messdifferenzen zwischen dem T0-Modell und dem \(\Lambda\)CDM-Standardmodell}. 2. April 2025.
		\bibitem{pascher_lagrange_2025} Pascher, J. (2025). \href{https://github.com/jpascher/T0-Time-Mass-Duality/tree/main/2/pdf/Deutsch/MathZeitMasseLagrange.pdf}{Von der Zeitdilatation zur Massenvariation: Mathematische Kernformulierungen der Zeit-Masse-Dualitätstheorie}. 29. März 2025.
		\bibitem{pascher_photons_2025} Pascher, J. (2025). \href{https://github.com/jpascher/T0-Time-Mass-Duality/tree/main/2/pdf/Deutsch/DynMassePhotonenNichtlokal.pdf}{Dynamische Masse von Photonen und ihre Implikationen für Nichtlokalität im T0-Modell}. 25. März 2025.
		\bibitem{pascher_erweiterung_2025} Pascher, J. (2025). \href{https://github.com/jpascher/T0-Time-Mass-Duality/tree/main/2/pdf/Deutsch/NotwendigkeitQMErweiterung.pdf}{Die Notwendigkeit der Erweiterung der Standardquantenmechanik und Quantenfeldtheorie}. 27. März 2025.
		\bibitem{pascher_higgs_2025} Pascher, J. (2025). \href{https://github.com/jpascher/T0-Time-Mass-Duality/tree/main/2/pdf/Deutsch/MathHiggsZeitMasse.pdf}{Mathematische Formulierung des Higgs-Mechanismus in der Zeit-Masse-Dualität}. 28. März 2025.
		\bibitem{pascher_emergente_gravitation_2025} Pascher, J. (2025). \href{https://github.com/jpascher/T0-Time-Mass-Duality/tree/main/2/pdf/Deutsch/EmergentGravT0.pdf}{Emergente Gravitation im T0-Modell: Eine umfassende Ableitung}. 1. April 2025.
		\bibitem{pascher_alpha_2025} Pascher, J. (2025). \href{https://github.com/jpascher/T0-Time-Mass-Duality/tree/main/2/pdf/Deutsch/NatEinheitenAlpha1.pdf}{Energie als fundamentale Einheit: Natürliche Einheiten mit \(\alphaEM = 1\) im T0-Modell}. 26. März 2025.
		\bibitem{pascher_alphabeta_2025} Pascher, J. (2025). \href{https://github.com/jpascher/T0-Time-Mass-Duality/tree/main/2/pdf/Deutsch/Alpha1Beta1Konsistenz.pdf}{Einheitliches Einheitensystem im T0-Modell: Die Konsistenz von \(\alphaEM = 1\) und \(\betaT = 1\)}. 5. April 2025.
		\bibitem{pascher_feldtheorie_2025} Pascher, J. (2025). \href{https://github.com/jpascher/T0-Time-Mass-Duality/tree/main/2/pdf/Deutsch/FeldtheorieQuanten.pdf}{Feldtheorie und Quantenkorrelationen: Eine neue Perspektive auf Instantanität}. 28. März 2025.
		\bibitem{pascher_temp_2025} Pascher, J. (2025). \href{https://github.com/jpascher/T0-Time-Mass-Duality/tree/main/2/pdf/Deutsch/TempEinheitenCMB.pdf}{Anpassung der Temperatureinheiten in natürlichen Einheiten und CMB-Messungen}. 2. April 2025.
		\bibitem{einstein1905} Einstein, A. (1905). Zur Elektrodynamik bewegter Körper. \textit{Annalen der Physik}, 322(10), 891-921.
		\bibitem{higgs1964broken} Higgs, P. W. (1964). Broken Symmetries and the Masses of Gauge Bosons. \textit{Physical Review Letters}, 13(16), 508-509.
		\bibitem{englert1964broken} Englert, F., \& Brout, R. (1964). Broken Symmetry and the Mass of Gauge Vector Mesons. \textit{Physical Review Letters}, 13(9), 321-323.
		\bibitem{weinberg1995quantum} Weinberg, S. (1995). \textit{The Quantum Theory of Fields, Volume I: Foundations}. Cambridge University Press.
		\bibitem{weinberg1989cosmological} Weinberg, S. (1989). The Cosmological Constant Problem. \textit{Reviews of Modern Physics}, 61(1), 1-23.
		\bibitem{bell1964} Bell, J. S. (1964). On the Einstein Podolsky Rosen Paradox. \textit{Physics}, 1(3), 195-200.
		\bibitem{rubin1980} Rubin, V. C., Ford, W. K., \& Thonnard, N. (1980). Rotational Properties of 21 SC Galaxies with a Large Range of Luminosities and Radii. \textit{The Astrophysical Journal}, 238, 471-487.
		\bibitem{McGaugh2016} McGaugh, S. S., Lelli, F., \& Schombert, J. M. (2016). Radial Acceleration Relation in Rotationally Supported Galaxies. \textit{Physical Review Letters}, 117(20), 201101.
		\bibitem{Planck2018} Planck Collaboration. (2020). Planck 2018 Results. VI. Cosmological Parameters. \textit{Astronomy \& Astrophysics}, 641, A6.
	\end{thebibliography}
	
\end{document}