\documentclass[twocolumn,aps,prl]{revtex4-2}
\usepackage[utf8]{inputenc}
\usepackage[T1]{fontenc}
\usepackage[ngerman]{babel}
\usepackage{lmodern}
\usepackage{amsmath}
\usepackage{amssymb}
\usepackage{physics}
\usepackage{hyperref}
\usepackage{booktabs}
\usepackage{enumitem}
\usepackage[table,xcdraw]{xcolor}
\usepackage{pgfplots}
\pgfplotsset{compat=1.18}
\usepackage{graphicx}
\usepackage{siunitx}
\usepackage{array} % Added for custom column types

% Custom commands
\newcommand{\Tfield}{T(x)}
\newcommand{\alphaEM}{\alpha_{\text{EM}}}
\newcommand{\alphaW}{\alpha_{\text{W}}}
\newcommand{\betaT}{\beta_{\text{T}}}
\newcommand{\Mpl}{M_{\text{Pl}}}
\newcommand{\Tzerot}{T_0(\Tfield)}
\newcommand{\Tzero}{T_0}
\newcommand{\vecx}{\vec{x}}
\newcommand{\gammaf}{\gamma_{\text{Lorentz}}}
\newcommand{\DhiggsT}{\Tfield (\partial_\mu + ig A_\mu) \Phi + \Phi \partial_\mu \Tfield}
\newcommand{\LCDM}{\Lambda\text{CDM}}
\newcommand{\DTmu}{D_{T,\mu}}
\newcommand{\calL}{\mathcal{L}}
\newcommand{\deq}{\displaystyle}
\newcommand{\e}{\mathrm{e}}

\hypersetup{
	colorlinks=true,
	linkcolor=blue,
	citecolor=blue,
	urlcolor=blue,
	pdftitle={Brückenschlag zwischen Quantenmechanik und Relativitätstheorie durch Zeit-Masse-Dualität: Teil II},
	pdfauthor={Johann Pascher},
	pdfsubject={Theoretische Physik},
	pdfkeywords={T0-Modell, natürliche Einheiten, Zeit-Masse-Dualität, Kosmologie}
}

\begin{document}
	
	\title{Brückenschlag zwischen Quantenmechanik und Relativitätstheorie durch Zeit-Masse-Dualität: Ein einheitliches Rahmenwerk mit natürlichen Einheiten \(\alpha = \beta = 1\) \\ Teil II: Kosmologische Implikationen und experimentelle Validierung}
	\author{Johann Pascher}
	\affiliation{Department für Kommunikationstechnik, Höhere Technische Bundeslehranstalt (HTL), Leonding, Österreich}
	\email{johann.pascher@gmail.com}
	\date{7. April 2025}
	
	\begin{abstract}
		Diese Arbeit erweitert das in Teil I vorgestellte T0-Modell in die Bereiche der Kosmologie und experimentellen Validierung, aufbauend auf einem vereinheitlichten natürlichen Einheitensystem, in dem \(\hbar = c = G = k_B = \alphaEM = \alphaW = \betaT = 1\). Im Gegensatz zum expandierenden Universum des \(\LCDM\)-Modells postulieren wir einen statischen Kosmos, in dem Rotverschiebung durch Photonenenergieverlust entsteht, vermittelt durch das intrinsische Zeitfeld \(\Tfield\). Dieses Rahmenwerk reinterpretiert dunkle Materie und dunkle Energie durch emergente Gravitationseffekte und erweitert das Standardmodell (SM) um eine konsistente Gravitationstheorie. Zentrale Vorhersagen umfassen eine wellenlängenabhängige Rotverschiebung mit einer Variation von etwa \(2,3\%\) pro Dekade, eine kosmische Mikrowellenhintergrundtemperatur (CMB) von \(24000 \, \text{K}\) bei \(z = 1100\) und spekulative Erweiterungen jenseits der Lichtgeschwindigkeit. Diese Vorhersagen sind mit Instrumenten wie dem James Webb Space Telescope (JWST) und zukünftigen CMB-Missionen überprüfbar. Wir behandeln Messprobleme wie die frequenzabhängigen Verzerrungen bei GPS-Präzision und kosmologischen Beobachtungen, die Unterschiede zwischen Massenvariation und Zeitdilatation verschleiern, und bieten eine philosophisch kohärente Alternative zum \(\LCDM\), die theoretische Eleganz mit empirischer Strenge verbindet. Zusätzlich zeigen wir, dass das Standardmodell durch einen krümmungsbasierten Ansatz für Rotverschiebung erweitert werden kann, um mathematische Äquivalenz mit dem T0-Modell zu erreichen und eine duale Beschreibung derselben physikalischen Realität zu bieten.
	\end{abstract}
	
	\maketitle
	
	\section{Einleitung}
	\label{sec:introduction}
	
	In Teil I (\textit{Brückenschlag zwischen Quantenmechanik und Relativitätstheorie durch Zeit-Masse-Dualität: Teil I}, \cite{pascher_part1_2025}) haben wir das T0-Modell als einheitliches Rahmenwerk für Quantenmechanik (QM) und Relativitätstheorie (RT) etabliert, basierend auf dem intrinsischen Zeitfeld \(\Tfield = \frac{\hbar}{\max(mc^2, \omega)}\) innerhalb eines natürlichen Einheitensystems (\(\hbar = c = G = k_B = \alphaEM = \alphaW = \betaT = 1\)). Dieses System, detailliert in Teil I, Abschnitt 2 ''Vereinheitlichung von Konstanten mit natürlichen Einheiten'' \href{https://github.com/jpascher/T0-Time-Mass-Duality/tree/main/2/pdf/Deutsch/QMRelTimeMassPart1.pdf}{[Teil I]}, eliminiert empirisch bestimmte Konstanten und erreicht Konsistenz mit gemessenen Werten (z.B. \(c \approx 3 \times 10^8 \, \text{m/s}\), \(\alphaEM \approx 1/137,036\)) mit Abweichungen unter \(10^{-6}\). Es ermöglichte eine massenabhängige Schrödinger-Gleichung (Teil I, Gleichung (4.5) \href{https://github.com/jpascher/T0-Time-Mass-Duality/tree/main/2/pdf/Deutsch/QMRelTimeMassPart1.pdf}{[Teil I]}) und emergente Gravitation (Teil I, Abschnitt 5 ''Emergente Gravitation'' \href{https://github.com/jpascher/T0-Time-Mass-Duality/tree/main/2/pdf/Deutsch/QMRelTimeMassPart1.pdf}{[Teil I]}), wodurch mikro- und makroskopische Physik verbunden werden.
	
	Teil II erweitert diese Grundlagen in die Bereiche Kosmologie und experimentelle Validierung, im Kontrast zum \(\LCDM\)-Modell mit seinem expandierenden Universum, das vor etwa 13,8 Milliarden Jahren aus einem Urknall entstanden sein soll \cite{Planck2020}. Im \(\LCDM\) ist die kosmische Rotverschiebung ein kinematischer Effekt (\(z \approx H_0 d / c\)), der Inflation und dunkle Energie erfordert \cite{Riess1998,Perlmutter1999}. Das T0-Modell postuliert ein statisches, unendliches und ewiges Universum, in dem Rotverschiebung aus Photonenenergieverlust über \(\Tfield\) resultiert, und erweitert das Standardmodell (SM) um eine konsistente Gravitationstheorie, während der Kern der Teilchenphysik erhalten bleibt.
	
	Zentrale Vorhersagen umfassen:
	- Wellenlängenabhängige Rotverschiebung (\(\sim 2,3\%\) pro Dekade),
	- CMB-Temperatur von \(24000 \, \text{K}\) bei \(z = 1100\),
	- Spekulative überlichtschnelle Erweiterungen.
	
	Diese sind mit JWST-Spektroskopie und CMB-Verzerrungsmessungen überprüfbar, obwohl frequenzbasierte Methoden (z.B. GPS, Rotverschiebung) Massenvariation und Zeitdilatation vermischen, was eine sorgfältige Neubewertung erfordert \cite{pascher_quantum_2025}. Philosophisch vermeidet T0 Singularitäten und bietet einen kohärenten ewigen Kosmos \cite{pascher_perspective_2025}.
	
	Wir zeigen zudem, dass das Standardmodell selbst durch spezifische Modifikationen erweitert werden kann, um mathematische Äquivalenz mit dem T0-Modell zu erreichen, was duale Interpretationen derselben physikalischen Realität ermöglicht. Dieses Konzept der ontologischen Äquivalenz bereichert unser Verständnis davon, wie unterschiedliche theoretische Rahmenwerke identische Beobachtungen beschreiben können.
	
	Diese Arbeit ist wie folgt strukturiert:
	- Abschnitt 2: Statisches Universum und Rotverschiebungsmechanismus.
	- Abschnitt 3: Kosmologische Phänomene und Vorhersagen.
	- Abschnitt 4: Quantitative Vorhersagen.
	- Abschnitt 5: Experimentelle Tests und Messherausforderungen.
	- Abschnitt 6: Implikationen von \(\betaT = 1\).
	- Abschnitt 7: Integration mit T0-Prinzipien.
	- Abschnitt 8: Erweitertes Standardmodell-Rahmenwerk.
	- Abschnitt 9: Ontologische Äquivalenz der Rahmenwerke.
	- Abschnitt 10: Spekulative Erweiterungen und Philosophie.
	
	\section{Statisches Universum-Modell}
	\label{sec:static_universe}
	
	\subsection{Konzept eines statischen Universums}
	\label{subsec:static_concept}
	
	Das T0-Modell konzipiert ein statisches Universum, unendlich im Raum und ewig in der Zeit, im Gegensatz zum expandierenden Kosmos des \(\LCDM\) mit Ursprung im Urknall. Im \(\LCDM\) spiegelt die Rotverschiebung (\(z \approx H_0 d / c\)) die Expansion wider (\(H_0 \approx 70 \, \text{km/s/Mpc}\)) \cite{Planck2020}, was Inflation für Einheitlichkeit und dunkle Energie für Beschleunigung (\(\Omega_{\Lambda} \approx 0,7\)) \cite{Riess1998} erfordert. T0 eliminiert diese, indem es einen stabilen Kosmos postuliert, in dem \(\Tfield\) die Dynamik ohne Expansion steuert.
	
	Vorteile umfassen:
	- \textbf{Horizontproblem:} Unendliche Zeit gewährleistet thermisches Gleichgewicht über alle Skalen \cite{pascher_messdifferenzen_2025}.
	- \textbf{Flachheit:} Keine Expansion eliminiert Krümmungsanpassung.
	- \textbf{Singularitätsfrei:} Ewige Existenz vermeidet unendliche Dichte \cite{pascher_perspective_2025}.
	
	Dies ergänzt die SM-Teilchenphysik mit einem statischen Gravitationsrahmen, abgeleitet in Teil I, Abschnitt 5 ''Emergente Gravitation'' \href{https://github.com/jpascher/T0-Time-Mass-Duality/tree/main/2/pdf/Deutsch/QMRelTimeMassPart1.pdf}{[Teil I]}.
	
	\subsection{Rotverschiebung durch Energieverlust}
	\label{subsec:redshift_energy_loss}
	
	Rotverschiebung im T0-Modell ist:
	\begin{equation}
		1 + z = e^{\alpha d},
		\label{eq:redshift_distance}
	\end{equation}
	mit \(\alpha = H_0 / c \approx 2,3 \times 10^{-18} \, \text{m}^{-1}\) (SI) oder 1 (natürliche Einheiten). Bei niedrigem \(z\):
	\begin{equation}
		z \approx \alpha d,
		\label{eq:hubble_approx}
	\end{equation}
	was lokal mit \(\LCDM\) übereinstimmt. Der Mechanismus ist Photonenenergieverlust:
	\begin{equation}
		\frac{dE}{dx} = -\alpha E,
		\label{eq:energy_loss_rate}
	\end{equation}
	was zu \(E = E_0 e^{-\alpha d}\) führt und damit zu \(1 + z = e^{\alpha d}\), abgeleitet aus \(\Tfield\)-Eigenschaften in Teil I, Abschnitt 3.1 ''Definition und physikalische Grundlage'' \href{https://github.com/jpascher/T0-Time-Mass-Duality/tree/main/2/pdf/Deutsch/QMRelTimeMassPart1.pdf}{[Teil I]} \cite{pascher_messdifferenzen_2025}.
	
	\section{Kosmologische Phänomene}
	\label{sec:cosmological_phenomena}
	
	\subsection{Temperatur-Rotverschiebungs-Relation und CMB}
	\label{subsec:cmb_temp}
	
	\(\LCDM\)s \(T(z) = T_0 (1 + z)\) ergibt \(T \approx 3000 \, \text{K}\) bei \(z = 1100\) (\(T_0 = 2,725 \, \text{K}\)) \cite{Fixsen2009}. T0 sagt voraus:
	\begin{equation}
		T(z) = T_0 (1 + z) (1 + \ln(1 + z)),
		\label{eq:temperature_redshift_simplified}
	\end{equation}
	sodass \(T(1100) \approx 24000 \, \text{K}\), was erhöhten Energieverlust widerspiegelt (Gleichung \ref{eq:energy_loss_rate}) \cite{pascher_temp_2025}. Dies beeinflusst Nukleosynthese und Rekombination, testbar über CMB-Verzerrungen (Abschnitt 5.2).
	
	\subsection{Wellenlängenabhängige Rotverschiebung}
	\label{subsec:wavelength_redshift}
	
	T0 sagt voraus:
	\begin{equation}
		z(\lambda) = z_0 \left(1 + \ln\left(\frac{\lambda}{\lambda_0}\right)\right),
		\label{eq:wavelength_redshift}
	\end{equation}
	mit \(\Delta z / z_0 \approx 3,85\%\) über 0,6-28 \(\si{\micro\meter}\) (JWST-Bereich), aufgrund von:
	\begin{equation}
		\frac{dE}{dx} = -\alpha E \left(1 + \ln\left(\frac{\lambda}{\lambda_0}\right)\right),
		\label{eq:wavelength_energy_loss}
	\end{equation}
	im Gegensatz zur Gleichförmigkeit des \(\LCDM\) \cite{pascher_params_2025}.
	
	\subsection{Reinterpretation von dunkler Materie und dunkler Energie}
	\label{subsec:dark_reinterpretation}
	
	Das Potential:
	\begin{equation}
		\Phi(r) = -\frac{M}{r} + \kappa r,
		\label{eq:grav_potential_t0}
	\end{equation}
	(\(\kappa \approx 4,8 \times 10^{-11} \, \text{m/s}^2\)) reinterpretiert:
	- \textbf{Dunkle Materie:} \(v(r) = \sqrt{\frac{M}{r} + \kappa r}\), wie in Teil I, Abschnitt 5.1 ''Ableitung aus \(\Tfield\)'' \href{https://github.com/jpascher/T0-Time-Mass-Duality/tree/main/2/pdf/Deutsch/QMRelTimeMassPart1.pdf}{[Teil I]} abgeleitet.
	- \textbf{Dunkle Energie:} \(\rho_{\text{DE}} \approx \frac{\kappa}{r^2}\) \cite{pascher_galaxies_2025}.
	
	\subsection{Einfluss auf Galaxiendynamik}
	\label{subsec:galaxy_dynamics}
	
	Das T0-Modell prägt die Galaxiendynamik durch \(\Tfield\) und bietet eine Alternative zu \(\LCDM\), indem es Gravitationseffekte ohne dunkle Materie oder Expansion neu interpretiert.
	
	\subsubsection{Rotationskurven}
	Das Potential (Gleichung \ref{eq:grav_potential_t0}) ergibt:
	\begin{equation}
		v(r) = \sqrt{\frac{M}{r} + \kappa r},
		\label{eq:rotation_velocity}
	\end{equation}
	was flache Rotationskurven reproduziert (z.B. Milchstraße: \(v(30 \, \text{kpc}) \approx 211 \, \text{km/s}\)) \cite{pascher_galaxies_2025}.
	
	\begin{figure}[ht]
		\centering
		\begin{tikzpicture}
			\begin{axis}[
				xlabel={Radius [kpc]},
				ylabel={Rotationsgeschwindigkeit [km/s]},
				xlabel style={font=\large},
				ylabel style={font=\large},
				tick label style={font=\normalsize},
				xmin=0, xmax=30,
				ymin=0, ymax=300,
				legend pos=south east,
				legend style={font=\large},
				grid=both,
				minor tick num=4,
				major grid style={line width=0.8pt, gray!50},
				minor grid style={line width=0.4pt, gray!20},
				width=\columnwidth
				]
				\addplot[blue, ultra thick, domain=0.1:30, samples=100] {220*sqrt(10/x)};
				\addplot[red, dashed, ultra thick, domain=0.1:30, samples=100] {sqrt(220^2*10/x + 4.8*x^2)};
				\legend{Newtonsche Vorhersage, T0-Modell}
			\end{axis}
		\end{tikzpicture}
		\caption{Rotationskurven im Vergleich: Newtonsche Vorhersage (blau) und T0-Modell-Vorhersage (rot) für eine Galaxie mit \(M = 10^{11} M_{\odot}\), \(\kappa_{\text{SI}} = 4,8 \times 10^{-11} \, \text{m/s}^2\).}
		\label{fig:rotation_curves}
	\end{figure}
	
	\subsubsection{Galaxienbildung und -entwicklung}
	Im \(\LCDM\) basiert die Galaxienbildung auf dem gravitativen Kollaps von Halos aus dunkler Materie über 13,8 Milliarden Jahre \cite{Planck2020}. T0 schlägt eine graduelle baryonische Aggregation in einem zeitlich unendlichen Universum vor, angetrieben durch \(\Tfield\) und das emergente Potential (Gleichung \ref{eq:grav_potential_t0}), was die SM-Dynamik ohne dunkle Materie verstärkt \cite{pascher_galaxies_2025}.
	
	\subsubsection{Clusterdynamik und Großraumstruktur}
	Für Cluster wie den Bullet Cluster reduziert T0s \(\kappa r\)-Term Massendiskrepanzen:
	\begin{equation}
		v_{\text{cluster}}(r) = \sqrt{\frac{M_{\text{total}}}{r} + \kappa r},
		\label{eq:cluster_velocity}
	\end{equation}
	was Lensing- und dynamische Massenschätzungen ohne dunkle Materie in Einklang bringt, testbar mit präzisen Durchmusterungen \cite{pascher_emergente_2025}. Großraumstrukturen entstehen aus \(\Tfield\)-Dynamik mit unendlicher Zeit (Teil I, Abschnitt 5 \href{https://github.com/jpascher/T0-Time-Mass-Duality/tree/main/2/pdf/Deutsch/QMRelTimeMassPart1.pdf}{[Teil I]}).
	
	\begin{table}[ht]
		\centering
		\caption{Vergleich der Vorhersagen von \(\LCDM\) und T0-Modell für Galaxiendynamik}
		\label{tab:galaxy_dynamics_comparison}
		\small
		\begin{tabular}{p{0.35\columnwidth} p{0.3\columnwidth} p{0.25\columnwidth}}
			\hline
			\textbf{Phänomen} & \textbf{\(\LCDM\)} & \textbf{T0-Modell} \\
			\hline
			Rotationskurve & Halo aus dunkler Materie & \(\kappa r\)-Term \\
			Galaxienbildung & Kollaps dunkler Materie & Baryonische Aggregation \\
			Clustermasse & Dominanz dunkler Materie & Baryonisch + \(\Tfield\) \\
			Großraumstruktur & Expansionsgetrieben & \(\Tfield\)-getrieben \\
			\hline
		\end{tabular}
	\end{table}
	
	\section{Quantitative Vorhersagen}
	\label{sec:predictions}
	
	\subsection{CMB-Temperaturvorhersage}
	\label{subsec:cmb_temp_prediction}
	
	T0 sagt eine CMB-Temperatur bei \(z = 1100\) von:
	\begin{equation}
		T(1100) \approx 24000 \, \text{K},
		\label{eq:cmb_temp_t0}
	\end{equation}
	im Vergleich zu \(\LCDM\)s \(3000 \, \text{K}\) voraus, aufgrund der logarithmischen Verstärkung durch \(\Tfield\) (Gleichung \ref{eq:temperature_redshift_simplified}).
	
	\subsection{Wellenlängenabhängige Rotverschiebungsvariation}
	\label{subsec:wavelength_redshift_prediction}
	
	Über den JWST-Bereich (0,6-28 \(\si{\micro\meter}\)):
	\begin{equation}
		\Delta z / z_0 \approx 3,85\%,
		\label{eq:wavelength_variation}
	\end{equation}
	oder \(2,3\%\) pro Dekade, testbar über Quasar-Emissionslinien \cite{pascher_params_2025}.
	
	\subsection{Galaxie-Rotationsgeschwindigkeiten}
	\label{subsec:rotation_velocity_prediction}
	
	Für die Milchstraße:
	\begin{equation}
		v(r) = \sqrt{\frac{M}{r} + \kappa r},
		\label{eq:rotation_velocity_repeat}
	\end{equation}
	z.B. \(v(30 \, \text{kpc}) \approx 211 \, \text{km/s}\), konsistent mit Beobachtungen \cite{McGaugh2016}.
	
	\begin{figure}[ht]
		\centering
		\begin{tikzpicture}
			\begin{axis}[
				xlabel={Rotverschiebung \(z\)},
				ylabel={Entfernungsmodul \(\mu\)},
				xmin=0, xmax=2,
				ymin=30, ymax=50,
				legend pos=north west,
				grid=both,
				width=\columnwidth,
				height=6cm,
				samples=100
				]
				\addplot[blue, thick, domain=0.01:2] {5*log10(3e8/70e3*ln(1+x)*(1+x)*0.1) + 25};
				\addplot[red, dashed, domain=0.01:2] {5*log10(3e8/70e3*(1+x)*(2-(1/(1+x)))*1) + 25};
				\legend{T0-Modell, \(\LCDM\)}
			\end{axis}
		\end{tikzpicture}
		\caption{Entfernungsmodul vs. Rotverschiebung im Vergleich: T0-Modell (blau) und \(\LCDM\) (rot).}
		\label{fig:distance_modulus}
	\end{figure}
	
	\section{Experimentelle Tests}
	\label{sec:tests}
	
	\subsection{JWST-Spektroskopie}
	\label{subsec:jwst_test}
	
	\(\Delta z / z \approx 3,85\%\) bei \(z = 7\) ist mit der Präzision des JWST detektierbar und kann T0 von \(\LCDM\) unterscheiden \cite{pascher_params_2025}.
	
	\subsection{CMB-Verzerrungen}
	\label{subsec:cmb_distortions_test}
	
	T0 sagt voraus:
	\begin{equation}
		\mu \approx 1,4 \times 10^{-5}, \quad y \approx 1,6 \times 10^{-6},
		\label{eq:distortion_parameters}
	\end{equation}
	im Gegensatz zu \(\LCDM\)s \(\mu \approx 2 \times 10^{-8}\), \(y \approx 4 \times 10^{-9}\), messbar mit PIXIE \cite{pascher_temp_2025}.
	
	\subsection{Messproblem: GPS und Uhrenpräzision}
	\label{subsec:gps_clock_problem}
	
	GPS-Uhren zeigen eine Verschiebung von \(\Delta t \approx 38 \, \mu\text{s/Tag}\), interpretiert als Massenvariation in T0, mit aktuellen Methoden ununterscheidbar von GRs Zeitdilatation \cite{pascher_quantum_2025}.
	
	\subsection{Messproblem: Kosmologische Beobachtungen}
	\label{subsec:cosmological_measurement_problem}
	
	Rotverschiebung in T0 ist Energieverlust (Gleichung \ref{eq:redshift_distance}), nicht Doppler, was nicht-frequenzbasierte Tests wie Zerfallsraten erfordert \cite{pascher_alphabeta_2025}.
	
	\subsection{Neubewertung von Messungen}
	\label{subsec:reassessment_measurements}
	
	\(\betaT = 1\) bringt kosmologische Daten mit T0s statischem Rahmen in Einklang und löst Spannungen wie \(H_0\)-Diskrepanzen \cite{DiValentino2021}.
	
	\section{Konsequenzen der Festlegung \(\beta = 1\)}
	\label{sec:consequences_beta}
	
	\subsection{Theoretische Eleganz}
	\label{subsec:theoretical_elegance}
	
	\(\betaT = 1\) vereinheitlicht Konstanten, vereinfacht Wechselwirkungen (Teil I, Abschnitt 4.1 \href{https://github.com/jpascher/T0-Time-Mass-Duality/tree/main/2/pdf/Deutsch/QMRelTimeMassPart1.pdf}{[Teil I]}) und verstärkt die Kohärenz \cite{pascher_alphabeta_2025}.
	
	\subsection{Umrechnung in SI-Einheiten}
	\label{subsec:conversion_si}
	
	\begin{equation}
		\betaT^{\text{SI}} = \betaT^{\text{nat}} \cdot \frac{\xi \cdot l_{P,\text{SI}}}{r_{0,\text{SI}}},
		\label{eq:beta_conversion}
	\end{equation}
	ergibt \(\betaT^{\text{SI}} \approx 0,008\), konsistent mit Beobachtungen \cite{pascher_alphabeta_2025}.
	
	\section{Integration in die Zeit-Masse-Dualitätstheorie}
	\label{sec:integration_t0}
	
	\subsection{Konsistenz mit Grundprinzipien}
	\label{subsec:consistency_principles}
	
	\(\betaT = 1\) unterstützt absolute Zeit, Massenvariation und emergente Gravitation (Teil I, Abschnitt 5.1 \href{https://github.com/jpascher/T0-Time-Mass-Duality/tree/main/2/pdf/Deutsch/QMRelTimeMassPart1.pdf}{[Teil I]}) \cite{pascher_lagrange_2025}.
	
	\section{Erweitertes Standardmodell-Rahmenwerk}
	\label{sec:extended_standard_model}
	
	\subsection{Motivation für die Erweiterung des Standardmodells}
	\label{subsec:extended_sm_motivation}
	
	Während das T0-Modell eine radikale Neuinterpretation der Physik mit absoluter Zeit und variabler Masse vorschlägt, besteht ein alternativer Ansatz darin, das Standardmodell zu erweitern, um mathematische Äquivalenz mit dem T0-Modell zu erreichen, während relative Zeit und konstante Ruhemasse beibehalten werden. Diese Erweiterung demonstriert die ontologische Komplementarität zwischen scheinbar unterschiedlichen Rahmenwerken.
	
	\subsection{Krümmungsbasierte Rotverschiebung}
	\label{subsec:curvature_redshift}
	
	Der Standardansatz zur kosmischen Rotverschiebung interpretiert diese als Folge universaler Expansion. Ein erweitertes Standardmodell kann jedoch dieselben Vorhersagen durch eine krümmungsbasierte Interpretation erreichen. Der Schlüssel ist die Modifikation des Gravitationspotentials mit einem linearen Term:
	
	\begin{equation}
		\Phi(r) = -\frac{GM}{r} + \kappa r
		\label{eq:modified_potential_sm}
	\end{equation}
	
	mit \(\kappa \approx 4,8 \times 10^{-11}\) m/s² in SI-Einheiten. Dieses modifizierte Potential erzeugt eine statische, gekrümmte Raumzeit mit der Metrik:
	
	\begin{equation}
		ds^2 = (1 - \frac{2GM}{r} + 2\kappa r)dt^2 - (1 + \frac{2GM}{r} - 2\kappa r)dr^2 - r^2d\Omega^2
		\label{eq:modified_metric}
	\end{equation}
	
	In der allgemeinen Relativitätstheorie ist die Gravitationsrotverschiebung zwischen zwei Punkten:
	
	\begin{equation}
		1 + z = \sqrt{\frac{g_{00}(\text{Emission})}{g_{00}(\text{Beobachtung})}}
		\label{eq:gravitational_redshift}
	\end{equation}
	
	Mit unserer modifizierten Metrik, \(g_{00} = (1 - \frac{2GM}{r} + 2\kappa r)\), und unter Berücksichtigung von Pfaden, bei denen der \(\kappa r\)-Term dominiert, ergibt sich:
	
	\begin{equation}
		1 + z = e^{\alpha d}(1 + \beta \ln(\lambda/\lambda_0))
		\label{eq:extended_sm_redshift}
	\end{equation}
	
	mit \(\alpha \approx 2,3 \times 10^{-18}\) m\(^{-1}\) und \(\beta \approx 0,008\) in SI-Einheiten. Dies ist mathematisch äquivalent zur Rotverschiebungsformel des T0-Modells (Gleichung \ref{eq:wavelength_redshift}). Die Wellenlängenabhängigkeit entsteht natürlich aus den dispersiven Eigenschaften der gekrümmten Raumzeit, wobei \(\kappa\) eine Frequenzabhängigkeit aufweist:
	
	\begin{equation}
		\kappa(\lambda) = \kappa_0(1 + \beta \ln(\lambda/\lambda_0))
		\label{eq:kappa_wavelength}
	\end{equation}
	
	\subsection{Erweiterte Einstein-Feldgleichungen}
	\label{subsec:extended_einstein}
	
	Die Standard-Einstein-Feldgleichungen können erweitert werden zu:
	
	\begin{equation}
		G_{\mu\nu} + \kappa g_{\mu\nu} = 8\pi G T_{\mu\nu} + \nabla_{\mu}\Theta\nabla_{\nu}\Theta - \frac{1}{2}g_{\mu\nu}(\nabla_{\sigma}\Theta\nabla^{\sigma}\Theta)
		\label{eq:extended_einstein}
	\end{equation}
	
	wobei \(\Theta\) ein Skalarfeld ist, das Effekte berücksichtigt, die das T0-Modell dem intrinsischen Zeitfeld zuschreibt. Der Term \(\kappa g_{\mu\nu}\) ist analog zum kosmologischen Konstante-Term, hat aber eine andere physikalische Interpretation, verbunden mit dem linearen Term im modifizierten Gravitationspotential.
	
	\subsection{Modifizierte Quantenevolution}
	\label{subsec:quantum_evolution}
	
	Die Standard-Schrödinger-Gleichung kann mit einer massenabhängigen Zeitevolutionskorrektur erweitert werden:
	
	\begin{equation}
		i\hbar\frac{\partial\Psi}{\partial t} = [\hat{H} + \hat{H}_{\Theta}]\Psi
		\label{eq:extended_schrodinger}
	\end{equation}
	
	wobei:
	
	\begin{equation}
		\hat{H}_{\Theta} = -i\hbar\frac{\partial\Theta}{\partial t}\Psi
		\label{eq:h_theta}
	\end{equation}
	
	Dies bewahrt den Standard-Schrödinger-Rahmen und berücksichtigt gleichzeitig Effekte, die das T0-Modell der variablen Masse zuschreibt, wodurch äquivalente Vorhersagen erzeugt werden, während die Zeitdilatations-Interpretation beibehalten wird.
	
	\subsection{Vereinigung von Quanten- und kosmologischen Phänomenen}
	\label{subsec:quantum_cosmological_unification}
	
	Das erweiterte Standardmodell verbindet Quanten- und kosmologische Skalen durch das Skalarfeld \(\Theta\):
	
	\begin{enumerate}
		\item Auf Quantenskalen modifiziert das \(\Theta\)-Feld die Quantenevolution durch Gleichung \ref{eq:extended_schrodinger}.
		\item Auf galaktischen Skalen erzeugt es flache Rotationskurven über das modifizierte Potential (Gleichung \ref{eq:modified_potential_sm}).
		\item Auf kosmologischen Skalen erzeugt es Rotverschiebung durch Raumzeitkrümmung (Gleichung \ref{eq:extended_sm_redshift}).
	\end{enumerate}
	
	Diese Vereinheitlichung erreicht dieselbe Erklärungskraft wie das T0-Modell, bewahrt aber die geometrische Interpretation der Gravitation und relative Zeit.
	
	\section{Ontologische Äquivalenz der Rahmenwerke}
	\label{sec:ontological_equivalence}
	
	\subsection{Mathematische Abbildung zwischen Rahmenwerken}
	\label{subsec:mathematical_mapping}
	
	Das erweiterte Standardmodell mit krümmungsbasierter Rotverschiebung erreicht mathematische Äquivalenz mit dem T0-Modell durch eine wohldefinierte Transformation:
	
	\begin{equation}
		\begin{aligned}
			g_{\mu\nu} &= \eta_{\mu\nu} + 2\ln(T(x)/T_0)\delta_{\mu\nu} \\
			m &= m_0T_0/T(x) \\
			F &= -\nabla\ln(T(x)/T_0)
		\end{aligned}
		\label{eq:framework_transformation}
	\end{equation}
	
	Diese Transformation bildet Bewegungsgleichungen in einem Rahmenwerk auf ihre Gegenstücke im anderen ab und zeigt, dass das T0-Modell und das erweiterte Standardmodell duale Formulierungen derselben Physik sind.
	
	\subsection{Komplementäre Beschreibungen der Realität}
	\label{subsec:complementary_descriptions}
	
	Die Beziehung zwischen dem T0-Modell und dem erweiterten Standardmodell veranschaulicht ontologische Komplementarität:
	
	\begin{enumerate}
		\item \textbf{Beobachtungsäquivalenz:} Beide Rahmenwerke sagen identische experimentelle Ergebnisse voraus, wenn sie richtig formuliert sind.
		\item \textbf{Transformationsabbildung:} Eine wohldefinierte mathematische Transformation verbindet die Rahmenwerke (Gleichung \ref{eq:framework_transformation}).
		\item \textbf{Erklärungskraft:} Jedes bietet einzigartige Erklärungsvorteile in bestimmten Bereichen.
		\item \textbf{Minimale Verpflichtung:} Beide streben nach der einfachsten Erklärung für Phänomene.
		\item \textbf{Domänenangemessenheit:} Jedes kann für verschiedene Untersuchungsbereiche besser geeignet sein.
	\end{enumerate}
	
	Diese Komplementarität ähnelt der Welle-Teilchen-Dualität in der Quantenmechanik, wo scheinbar widersprüchliche Beschreibungen beide gültig und notwendig für ein vollständiges Verständnis sind.
	
	\begin{table}[ht]
		\centering
		\caption{Vergleich von T0-Modell und erweitertem Standardmodell}
		\label{tab:model_comparison}
		\small
		\begin{tabular}{p{0.4\columnwidth} p{0.25\columnwidth} p{0.25\columnwidth}}
			\hline
			\textbf{Aspekt} & \textbf{T0-Modell} & \textbf{Erweitertes SM} \\
			\hline
			Zeitnatur & Absolut & Relativ \\
			Masse & Variabel & Konstant \\
			Raumzeit & Statisch & Gekrümmt \\
			Gravitationsquelle & \(\Tfield\)-Gradienten & Metrische Krümmung \\
			Rotverschiebungsmechanismus & Energieverlust & Krümmungseffekt \\
			Philosophische Basis & Feldmechanik & Geometrie \\
			\hline
		\end{tabular}
	\end{table}
	
	\subsection{Experimentelle Ansätze zur Modellunterscheidung}
	\label{subsec:model_discrimination}
	
	Obwohl mathematisch äquivalent, könnten diese Rahmenwerke unterschieden werden durch:
	
	\begin{enumerate}
		\item \textbf{Quantengravitationstests:} Die Rahmenwerke könnten unterschiedliches Verhalten auf der Planck-Skala vorhersagen.
		\item \textbf{Neuartige Teilchenwechselwirkungen:} Das intrinsische Zeitfeld könnte sich in Hochenergie-Experimenten anders manifestieren als das Skalarfeld \(\Theta\).
		\item \textbf{Philosophie der Messung:} Wie wir Messung selbst interpretieren, könnte ein Rahmenwerk gegenüber dem anderen begünstigen.
	\end{enumerate}
	
	Diese Tests untersuchen nicht nur die mathematischen Strukturen, sondern die ontologischen Grundlagen der Rahmenwerke.
	
	\section{Jenseits der Grenzen}
	\label{sec:beyond_limits}
	
	\subsection{Spekulative Erweiterungen}
	\label{subsec:speculative_extensions}
	
	\(\Tfield = \frac{\hbar}{m c^2}\) deutet auf langsamere Dynamik unterhalb \(m_P\) hin, testbar in der Nähe von schwarzen Löchern \cite{pascher_planck_2025}.
	
	\begin{figure}[ht]
		\centering
		\begin{tikzpicture}
			\draw[->] (0,0) -- (6,0) node[right] {Masse \(m\)};
			\draw[->] (0,0) -- (0,4) node[above] {Zeit \(T\)};
			\draw[scale=0.5, domain=0.1:10, smooth, variable=\x, blue, thick] plot ({\x}, {1/\x});
			\draw[dotted, red] (1.5,0) -- (1.5,1.5) -- (0,1.5);
			\node at (1.5,-0.3) {\(m_P\)};
			\node at (-0.3,1.5) {\(t_P\)};
		\end{tikzpicture}
		\caption{Masse vs. intrinsische Zeit nahe der Planck-Skala.}
		\label{fig:mass_time}
	\end{figure}
	
	\subsection{Philosophische Reflexionen}
	\label{subsec:philosophical_reflections}
	
	Der statische, ewige Kosmos des T0-Modells weicht fundamental vom endlichen, expandierenden Universum des \(\Lambda\)CDM ab und bietet tiefgreifende philosophische Implikationen. Durch die Vermeidung von Singularitäten und unendlicher Dichte präsentiert T0 eine vereinheitlichte Realität, in der Zeit eine intrinsische Eigenschaft (\(\Tfield\)) statt eine relativistische Variable ist, und Masse sich dynamisch an lokale Bedingungen anpasst. Dies steht im Kontrast zur fragmentierten Ontologie des \(\Lambda\)CDM – mit Urknall-Ursprung, dunklen Komponenten und ungewissem Schicksal – indem es einen kohärenten, unendlichen Rahmen vorschlägt, der mit intuitiven Vorstellungen von Existenz ohne Anfang oder Ende übereinstimmt.
	
	Die Eliminierung von Expansion und dunklen Entitäten vereinfacht die Kosmologie bei gleichzeitiger Wahrung empirischer Konsistenz (Abschnitte 4, 5) und deutet darauf hin, dass die scheinbare Komplexität des Universums aus Fehlinterpretationen frequenzbasierter Messungen stammen könnte (Abschnitt 5.4). Philosophisch resoniert T0 mit einer holistischen Sicht der Natur, in der Quanten- und relativistische Phänomene aus einem einzigen Prinzip – Zeit-Masse-Dualität – entstehen, was die Erklärungskraft des SM über alle Skalen hinweg verstärkt \cite{pascher_perspective_2025}.
	
	\subsection{Ontologische Reflexionen über äquivalente Beschreibungen}
	\label{subsec:ontological_reflections}
	
	Die dualen Formulierungen des T0-Modells und des erweiterten Standardmodells werfen tiefgreifende Fragen zur Natur der physikalischen Realität und unserer theoretischen Beschreibungen auf. Sie zeigen, dass:
	
	\begin{itemize}
		\item Physikalische Theorien möglicherweise nicht ''was die Realität ist'' im absoluten Sinne offenbaren, sondern eher konsistente Rahmenwerke für die Organisation von Beobachtungen bieten.
		\item Die Wahl zwischen ontologisch unterschiedlichen, aber mathematisch äquivalenten Theorien kann von Erklärungseffizienz, konzeptueller Kohärenz, Vorhersagekraft und theoretischer Eleganz abhängen.
		\item Unser Verständnis grundlegender Konzepte wie Zeit, Raum und Masse könnte rahmenwerksabhängig sein, statt absolute Wahrheiten über die Natur widerzuspiegeln.
	\end{itemize}
	
	Diese philosophische Position stimmt mit dem Instrumentalismus in der Wissenschaft überein, behält aber den Realismus bezüglich der zugrundeliegenden Phänomene bei.
	
	\section{Schlussfolgerung}
	\label{sec:conclusion}
	
	Das T0-Modell bietet eine umfassende Alternative zu aktuellen kosmologischen Paradigmen durch Zeit-Masse-Dualität, bewahrt den Kern der Teilchenphysik des Standardmodells und interpretiert Gravitationsphänomene neu. Sein statisches Universum mit intrinsischer zeitvermittelter Rotverschiebung eliminiert die Notwendigkeit für Expansion, dunkle Materie und dunkle Energie und vereint theoretische Einfachheit mit Beobachtungskonsistenz.
	
	Zentrale Beiträge umfassen:
	\begin{itemize}
		\item Ein vereinheitlichtes natürliches Einheitensystem mit \(\hbar = c = G = k_B = \alphaEM = \alphaW = \betaT = 1\)
		\item Charakteristische Vorhersagen für wellenlängenabhängige Rotverschiebung und CMB-Temperatur
		\item Erklärung der Galaxiendynamik ohne dunkle Materie über den \(\kappa r\)-Potentialterm
		\item Demonstration, dass Standardphysik erweitert werden kann, um mathematische Äquivalenz mit T0 zu erreichen, während relative Zeit beibehalten wird
	\end{itemize}
	
	Die dualen Formulierungen – T0-Modell und erweitertes Standardmodell – veranschaulichen ontologische Komplementarität und zeigen, wie scheinbar unterschiedliche Rahmenwerke identische physikalische Realität beschreiben können. Diese Komplementarität bereichert unser Verständnis theoretischer Physik und eröffnet neue Erkundungswege.
	
	Zukünftige Forschung wird sich auf die Prüfung wellenlängenabhängiger Rotverschiebung mit JWST, die Analyse von CMB-Spektralverzerrungen und die Erforschung der Implikationen von \(\betaT = 1\) für eine vereinheitlichte Physik konzentrieren. Die Neukalibrierung kosmologischer Parameter innerhalb des T0-Rahmenwerks, insbesondere die Überprüfung der Rotverschiebung der Rekombination (potenziell näher an z $\approx$ 950 als z $\approx$ 1100), stellt eine bedeutende bevorstehende Herausforderung dar.
	
	Sei es durch die Linse der absoluten Zeit mit variabler Masse oder der relativen Zeit mit geometrischer Gravitation, das T0-Modell und seine Standardmodellerweiterung zeigen, dass unser Verständnis des Kosmos durch fundamentale Neuinterpretation etablierter Prinzipien dramatisch transformiert werden kann.
	
	\begin{acknowledgments}
		Dank gilt Reinsprecht Martin Dipl.-Ing. Dr. für kritisches Feedback.
	\end{acknowledgments}
	
	\begin{thebibliography}{99}
		\bibitem{pascher_emergente_2025} J. Pascher, „Emergente Gravitation im T0-Modell: Eine umfassende Herleitung,'' Preprint (2025), \url{https://github.com/jpascher/T0-Time-Mass-Duality/tree/main/2/pdf/Deutsch/EmergentGravT0.pdf}.
		\bibitem{pascher_part1_2025} J. Pascher, „Brückenschlag zwischen Quantenmechanik und Relativitätstheorie durch Zeit-Masse-Dualität: Ein einheitliches Rahmenwerk mit natürlichen Einheiten $\alpha = \beta = 1$ Teil I: Theoretische Grundlagen,'' Preprint (2025), \url{https://github.com/jpascher/T0-Time-Mass-Duality/tree/main/2/pdf/Deutsch/QMRelTimeMassPart1.pdf}.
		\bibitem{pascher_lagrange_2025} J. Pascher, „Von der Zeitdilatation zur Massenvariation: Mathematische Kernformulierungen der Zeit-Masse-Dualitätstheorie,'' Preprint (2025), \url{https://github.com/jpascher/T0-Time-Mass-Duality/tree/main/2/pdf/Deutsch/MathZeitMasseLagrange.pdf}.
		\bibitem{pascher_quantum_2025} J. Pascher, „Die Notwendigkeit der Erweiterung der Standard-Quantenmechanik und Quantenfeldtheorie,'' Preprint (2025), \url{https://github.com/jpascher/T0-Time-Mass-Duality/tree/main/2/pdf/Deutsch/NotwendigkeitQMErweiterung.pdf}.
		\bibitem{pascher_photons_2025} J. Pascher, „Dynamische Masse von Photonen und ihre Implikationen für Nichtlokalität im T0-Modell,'' Preprint (2025), \url{https://github.com/jpascher/T0-Time-Mass-Duality/tree/main/2/pdf/Deutsch/DynMassePhotonenNichtlokal.pdf}.
		\bibitem{pascher_alphabeta_2025} J. Pascher, „Vereinheitlichtes Einheitensystem im T0-Modell: Die Konsistenz von $\alpha = 1$ und $\beta = 1$,'' Preprint (2025), \url{https://github.com/jpascher/T0-Time-Mass-Duality/tree/main/2/pdf/Deutsch/Alpha1Beta1Konsistenz.pdf}.
		\bibitem{pascher_perspective_2025} J. Pascher, „Eine neue Perspektive auf Zeit und Raum: Johann Paschers revolutionäre Ideen,'' Preprint (2025), \url{https://github.com/jpascher/T0-Time-Mass-Duality/tree/main/2/pdf/Deutsch/ZeitRaumPascher.pdf}.
		\bibitem{pascher_messdifferenzen_2025} J. Pascher, „Kompensatorische und additive Effekte: Eine Analyse der Messdifferenzen zwischen dem T0-Modell und dem $\Lambda$CDM-Standardmodell,'' Preprint (2025), \url{https://github.com/jpascher/T0-Time-Mass-Duality/tree/main/2/pdf/Deutsch/MessdifferenzenT0Standard.pdf}.
		\bibitem{pascher_temp_2025} J. Pascher, „Anpassung der Temperatureinheiten und CMB-Messungen,'' Preprint (2025), \url{https://github.com/jpascher/T0-Time-Mass-Duality/tree/main/2/pdf/Deutsch/TempEinheitenCMB.pdf}.
		\bibitem{pascher_params_2025} J. Pascher, „Herleitung der Parameter $\kappa$, $\alpha$ und $\beta$,'' Preprint (2025), \url{https://github.com/jpascher/T0-Time-Mass-Duality/tree/main/2/pdf/Deutsch/ZeitMasseT0Params.pdf}.
		\bibitem{pascher_galaxies_2025} J. Pascher, „Massenvariation in Galaxien,'' Preprint (2025), \url{https://github.com/jpascher/T0-Time-Mass-Duality/tree/main/2/pdf/Deutsch/MassVarGalaxien.pdf}.
		\bibitem{pascher_planck_2025} J. Pascher, „Jenseits der Planck-Skala,'' Preprint (2025), \url{https://github.com/jpascher/T0-Time-Mass-Duality/tree/main/2/pdf/Deutsch/JenseitsPlanck.pdf}.
		\bibitem{Planck2020} Planck Collaboration, Astron. Astrophys. \textbf{641}, A6 (2020).
		\bibitem{Riess1998} A. G. Riess et al., Astron. J. \textbf{116}, 1009 (1998).
		\bibitem{Perlmutter1999} S. Perlmutter et al., Astrophys. J. \textbf{517}, 565 (1999).
		\bibitem{Fixsen2009} D. J. Fixsen, Astrophys. J. \textbf{707}, 916 (2009).
		\bibitem{McGaugh2016} S. S. McGaugh et al., Phys. Rev. Lett. \textbf{117}, 201101 (2016).
		\bibitem{DiValentino2021} E. Di Valentino et al., Class. Quantum Grav. \textbf{38}, 153001 (2021).
		\bibitem{einstein1905} A. Einstein, Ann. Phys. \textbf{322}, 891 (1905).
		\bibitem{einstein1915} A. Einstein, Sitzungsber. Preuss. Akad. Wiss. \textbf{1915}, 844 (1915).
		\bibitem{einstein1917} A. Einstein, Sitzungsber. Preuss. Akad. Wiss. \textbf{1917}, 142 (1917).
		\bibitem{tHooft1980} G. 't Hooft, NATO Sci. Ser. B. \textbf{59}, 135 (1980).
		\bibitem{weinberg1989} S. Weinberg, Rev. Mod. Phys. \textbf{61}, 1 (1989).
		\bibitem{hawking1975} S. W. Hawking, Commun. Math. Phys. \textbf{43}, 199 (1975).
		\bibitem{will2014} C. M. Will, Living Rev. Relativ. \textbf{17}, 4 (2014).
	\end{thebibliography}
	
\end{document}