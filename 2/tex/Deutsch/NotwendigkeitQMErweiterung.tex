\documentclass[12pt,a4paper]{article}
\usepackage[utf8]{inputenc}
\usepackage[T1]{fontenc}
\usepackage[ngerman]{babel}
\usepackage{lmodern}
\usepackage{amsmath}
\usepackage{amssymb}
\usepackage{physics}
\usepackage{hyperref}
\usepackage{tcolorbox}
\usepackage{booktabs}
\usepackage{enumitem}
\usepackage[table,xcdraw]{xcolor}
\usepackage[left=2cm,right=2cm,top=2cm,bottom=2cm]{geometry}
\usepackage{pgfplots}
\pgfplotsset{compat=1.18}
\usepackage{graphicx}
\usepackage{float}
\usepackage{fancyhdr}

% Headers and Footers
\pagestyle{fancy}
\fancyhf{}
\fancyhead[L]{Johann Pascher}
\fancyhead[R]{Zeit-Masse-Dualität}
\fancyfoot[C]{\thepage}
\renewcommand{\headrulewidth}{0.4pt}
\renewcommand{\footrulewidth}{0.4pt}

% Custom Commands
\newcommand{\Tfield}{T(x)}
\newcommand{\alphaEM}{\alpha_{\text{EM}}}
\newcommand{\alphaW}{\alpha_{\text{W}}}
\newcommand{\betaT}{\beta_{\text{T}}}
\newcommand{\Mpl}{M_{\text{Pl}}}
\newcommand{\Tzerot}{T_0(\Tfield)}
\newcommand{\Tzero}{T_0}
\newcommand{\vecx}{\vec{x}}
\newcommand{\gammaf}{\gamma_{\text{Lorentz}}}
\newcommand{\DhiggsT}{\Tfield (\partial_\mu + ig A_\mu) \Phi + \Phi \partial_\mu \Tfield}

\hypersetup{
	colorlinks=true,
	linkcolor=blue,
	citecolor=blue,
	urlcolor=blue,
	pdftitle={Die Notwendigkeit der Erweiterung der Standard-Quantenmechanik und Quantenfeldtheorie},
	pdfauthor={Johann Pascher},
	pdfsubject={Theoretische Physik},
	pdfkeywords={T0-Modell, Natürliche Einheiten, Feinstrukturkonstante, Einheitliches Einheitensystem, Zeit-Masse-Dualität}
}

\title{Die Notwendigkeit der Erweiterung der Standard-Quantenmechanik und Quantenfeldtheorie}
\author{Johann Pascher}
\date{27. März 2025}

\begin{document}
	
	\maketitle
	
	\begin{abstract}
		Diese Arbeit untersucht grundlegende Einschränkungen der Standard-Quantenmechanik (QM) und Quantenfeldtheorie (QFT) und argumentiert für notwendige Erweiterungen, die Probleme mit Zeit, Kausalität und Messung adressieren. Wir führen ein Modell mit intrinsischer Zeit $\Tfield = \frac{\hbar}{mc^2}$ als physikalisches Feld statt als externen Parameter ein, was zu modifizierten Versionen der Schrödinger-Gleichung und der Lagrange-Formulierung führt. Dieser Ansatz löst Probleme wie den beobachterabhängigen Wellenfunktionskollaps und die Kompatibilität von Quanteneffekten mit der Relativitätstheorie. Das vorgeschlagene Rahmenwerk zeigt, wie die Quantenmechanik aus einem deterministischen System entstehen kann, wenn Messung als intrinsischer physikalischer Prozess verstanden wird. Experimentelle Vorhersagen, einschließlich massenabhängiger Dekohärenzraten und modifizierter Verschränkungsdynamik, werden als überprüfbare Konsequenzen des Modells vorgeschlagen.
	\end{abstract}
	
	\tableofcontents
	\newpage
	
	\section{Einleitung}
	\label{sec:introduction}
	
	Die Standard-Quantenmechanik, eine Theorie, die Anfang des 20. Jahrhunderts durch die Arbeiten von Schrödinger \cite{Schrodinger1926}, Heisenberg \cite{Heisenberg1925}, Born \cite{Born1926} und anderen entwickelt wurde, hat beispiellosen Erfolg bei der Vorhersage mikroskopischer Phänomene mit außerordentlicher Präzision erzielt. Ebenso bietet die Quantenfeldtheorie, entwickelt durch die Bemühungen von Dirac \cite{Dirac1927}, Feynman \cite{Feynman1949}, Schwinger \cite{Schwinger1948}, Tomonaga \cite{Tomonaga1946} und Dyson \cite{Dyson1949}, unsere genaueste Beschreibung von Teilchenwechselwirkungen.
	
	Trotz dieser Erfolge enthalten diese Rahmenbedingungen anhaltende konzeptionelle Probleme, die darauf hindeuten, dass sie möglicherweise unvollständig sind:
	
	\begin{enumerate}
		\item \textbf{Das Messproblem}: Der unerklärte Übergang von einer deterministischen Wellenevolution zu einem probabilistischen Ergebnis bei der Messung \cite{vonNeumann1932, Wheeler1983}.
		
		\item \textbf{Zeit als Parameter}: Die Behandlung der Zeit als externen Parameter anstatt als Observable, wodurch eine Asymmetrie zwischen Raum und Zeit entsteht \cite{Pauli1980, Peres1980}.
		
		\item \textbf{Nichtlokalität und Kausalität}: Die scheinbare Spannung zwischen Quantenverschränkung und relativistischer Kausalität \cite{Bell1964, EPR1935, Aspect1982}.
		
		\item \textbf{Quanten-Klassische Grenze}: Die unklare Abgrenzung zwischen Quanten- und klassischen Regimen \cite{Joos1985, Zurek2003, Leggett2002}.
		
		\item \textbf{Quantengravitation}: Die Herausforderung, Quantenprinzipien mit der Gravitation zu integrieren \cite{Rovelli2004, Kiefer2007, Oriti2009}.
	\end{enumerate}
	
	Diese Arbeit schlägt vor, dass diese Probleme nicht nur philosophische Bedenken sind, sondern Symptome erforderlicher theoretischer Erweiterungen der Quantenmechanik, ähnlich denen, die von den frühen Quantenpionieren wie Einstein, Bohr und Heisenberg erkannt wurden. Wir argumentieren, dass die Einführung eines intrinsischen Zeitfeldes $\Tfield = \frac{\hbar}{mc^2}$ die Grundlage für eine umfassende Erweiterung bildet, die diese Herausforderungen angeht und gleichzeitig den empirischen Erfolg der Standardquantentheorie beibehält \cite{pascher_zeit_2025}.
	
	Insbesondere zeigt diese Arbeit, wie ein modifizierter Rahmen zu Folgendem führt:
	\begin{itemize}
		\item Eine massenabhängige intrinsische Zeit, die messbare Variationen im Quantenverhalten verursacht.
		\item Eine Lösung des Wellenfunktionskollapses als physikalischer, deterministischer Prozess.
		\item Eine Neuinterpretation der Nichtlokalität, die Konsistenz mit relativistischen Prinzipien aufrechterhält.
		\item Ein natürliches Entstehen des klassischen Verhaltens aus Quantengrundlagen.
		\item Ein Weg zur Quantengravitation durch das intrinsische Zeitfeld.
	\end{itemize}
	
	Durch diese Analyse wollen wir zeigen, dass, ähnlich wie bei früheren großen theoretischen Übergängen in der Physik, die Quantenmechanik und die Quantenfeldtheorie möglicherweise vor einer bedeutenden Erweiterung stehen—einer, die sowohl ihre Grundlagen klärt als auch ihren Vorhersagebereich erweitert \cite{pascher_higgs_2025, pascher_photons_2025}.
	
	\section{Beschränkungen der Standard-QM}
	\label{sec:limitations}
	
	\subsection{Der externe Zeitparameter}
	\label{subsec:time_parameter}
	
	Die Standard-Quantenmechanik behandelt Zeit anders als Raum, mit der Schrödinger-Gleichung \cite{Schrodinger1926}:
	
	\begin{equation}
		i\hbar \frac{\partial}{\partial t} \Psi = \hat{H} \Psi,
		\label{eq:schrodinger}
	\end{equation}
	
	Die Zeit erscheint als externer Parameter, nicht als Observable mit einem entsprechenden Operator. Diese Asymmetrie schafft mehrere Probleme:
	
	\begin{enumerate}
		\item \textbf{Kein Zeitoperator}: Während die Position einen Operator $\hat{x}$ mit Eigenzuständen hat, existiert kein Zeitoperator $\hat{t}$ mit vergleichbaren Eigenschaften \cite{Pauli1980}.
		
		\item \textbf{Annahme einer universellen Uhr}: Die Theorie nimmt implizit an, dass alle Systeme eine universelle externe Uhr teilen, unabhängig von ihrem Zustand oder ihrer Interaktionsgeschichte \cite{Page1983}.
		
		\item \textbf{Quanten-Relativistische Inkonsistenz}: Die Behandlung der Zeit steht im Widerspruch zur Behandlung der Zeit in der Relativitätstheorie als Koordinate, nicht als Parameter \cite{Busch1994, Peres1980}.
	\end{enumerate}
	
	Diese Asymmetrie zwischen Raum und Zeit hat tiefe Konsequenzen. Während die räumliche Dynamik quantisiert und zustandsabhängig ist, wird die zeitliche Entwicklung extern und einheitlich auferlegt, was eine konzeptionelle Spannung schafft, die besonders deutlich wird, wenn man versucht, die Quantenmechanik mit der Relativitätstheorie in Einklang zu bringen.
	
	\subsection{Beobachterabhängigkeit und Messung}
	\label{subsec:observer_dependence}
	
	Das Messproblem, erstmals formal von von Neumann \cite{vonNeumann1932} beschrieben, umfasst den unerklärten Übergang von der deterministischen Wellenevolution zu probabilistischen Ergebnissen:
	
	\begin{equation}
		\Psi = \sum_n c_n \psi_n \xrightarrow{\text{Messung}} \psi_m \text{ mit Wahrscheinlichkeit } |c_m|^2,
		\label{eq:collapse}
	\end{equation}
	
	Dies schafft konzeptionelle Probleme:
	
	\begin{enumerate}
		\item \textbf{Beobachterabhängigkeit}: Die Standardformulierung erfordert einen undefinierten "Beobachter" außerhalb des Systems \cite{Wheeler1983, Wigner1967}.
		
		\item \textbf{Messapparatur-Paradoxon}: Das Messgerät, das aus Teilchen besteht, die den Quantenregeln folgen, sollte selbst in Superposition sein \cite{vonNeumann1932, Wigner1963}.
		
		\item \textbf{Willkürliche Trennung}: Die Quanten/Klassik-Grenze wird willkürlich gesetzt \cite{deBroglie1930, Bohm1952, Bell1987}.
	\end{enumerate}
	
	Die Beobachterabhängigkeit in der Quantenmessung wurde durch verschiedene Gedankenexperimente hervorgehoben, am bekanntesten Wigners Freund \cite{Wigner1967}, der illustriert, wie verschiedene Beobachter unterschiedliche Quantenzustände demselben System zuordnen können, was Fragen zur Objektivität von Quantenzuständen aufwirft.
	
	\subsection{Nichtlokalität und Kausalitätsspannungen}
	\label{subsec:nonlocality}
	
	Quantenverschränkung und die experimentellen Verletzungen der Bell'schen Ungleichungen \cite{Bell1964, Aspect1982, Hensen2015} demonstrieren nichtlokale Korrelationen, die unser Verständnis von Kausalität herausfordern. Die Wellenfunktion eines verschränkten Zustands:
	
	\begin{equation}
		|\Psi\rangle = \frac{1}{\sqrt{2}}(|0\rangle_A |1\rangle_B - |1\rangle_A |0\rangle_B),
		\label{eq:entangled_state}
	\end{equation}
	
	führt zu Korrelationen, die schneller-als-Licht-Einflüsse oder retrospektive Bestimmung von Eigenschaften zu erfordern scheinen \cite{EPR1935}.
	
	Dies schafft Spannung mit relativistischen Prinzipien:
	
	\begin{enumerate}
		\item \textbf{Scheinbare Fernwirkung}: Messung an einem Teilchen scheint ein anderes beliebig entferntes Teilchen sofort zu beeinflussen.
		
		\item \textbf{Nicht-Kommunikationstheorem}: Trotz dieser Korrelationen kann keine Information schneller als Licht übertragen werden \cite{Eberhard1978, Ghirardi1980}.
		
		\item \textbf{Bezugssystemabhängigkeit}: Die Gleichzeitigkeit von Messungen hängt in der Relativitätstheorie vom Bezugssystem ab, der Quantenkollaps erscheint jedoch in allen Bezugssystemen instantan \cite{Aharonov1980, Aharonov1981}.
	\end{enumerate}
	
	Diese Probleme deuten darauf hin, dass die Standard-Quantenmechanik möglicherweise eine Erweiterung erfordert, die besser mit relativistischen Prinzipien integriert ist und eine umfassendere Darstellung raumzeitlicher Beziehungen in Quantensystemen bietet.
	
	\section{Das Konzept der intrinsischen Zeit}
	\label{sec:intrinsic_time}
	
	\subsection{Definition und physikalische Grundlage}
	\label{subsec:intrinsic_definition}
	
	Das Konzept der intrinsischen Zeit schlägt vor, dass Zeit nicht nur ein externer Parameter ist, sondern eine physikalische Eigenschaft von Systemen, die mit Masse und Energie variiert, definiert als:
	
	\begin{equation}
		\Tfield = \frac{\hbar}{mc^2},
		\label{eq:intrinsic_time}
	\end{equation}
	
	Für massive Teilchen definiert dies die charakteristische Zeitskala der Quantenevolution des Systems. Für masselose Entitäten wie Photonen verallgemeinert sich dies zu:
	
	\begin{equation}
		\Tfield = \frac{\hbar}{\max(mc^2, \omega)},
		\label{eq:intrinsic_time_general}
	\end{equation}
	
	wobei $\omega$ die Photonenenergie/Frequenz ist.
	
	Die physikalische Grundlage für diese Formulierung umfasst:
	
	\begin{enumerate}
		\item \textbf{Energie-Zeit-Beziehung}: Aus Heisenbergs Unschärfeprinzip \cite{Heisenberg1927, Mandelstam1945}, $\Delta E \Delta t \geq \frac{\hbar}{2}$, was eine energieabhängige Zeitskala nahelegt.
		
		\item \textbf{Compton-Zeit}: Die Compton-Zeit $T_C = \frac{\hbar}{mc^2}$ repräsentiert die Zeit, die Licht benötigt, um die Compton-Wellenlänge eines Teilchens zu durchqueren, eine natürliche Quantenzeitskala \cite{MacGibbon1987, Caldirola1953}.
		
		\item \textbf{Zitterbewegung}: Die intrinsische Zitterbewegung, die von der Dirac-Gleichung vorhergesagt wird, mit der Frequenz $\omega_Z = \frac{2mc^2}{\hbar}$, entsprechend einer Periode $T_Z = \frac{\pi\hbar}{mc^2}$ \cite{Dirac1928, Schrodinger1930, Hestenes1990}.
	\end{enumerate}
	
	Diese intrinsische Zeit bietet eine natürliche Skala für Quantenprozesse und legt nahe, dass schwerere Teilchen eine schnellere interne Zeitentwicklung erleben als leichtere, ein Konzept mit Implikationen für Dekohärenz, Messung und den Quanten-Klassik-Übergang.
	
	\subsection{Beziehung zu de Broglies Uhr}
	\label{subsec:debroglie_clock}
	
	Das Konzept der intrinsischen Zeit ist eng mit de Broglies Idee eines "periodischen Phänomens" oder einer internen Uhr verbunden, die mit Teilchen assoziiert ist \cite{deBroglie1923, deBroglie1924}. De Broglie schlug vor, dass jedes Teilchen eine assoziierte Welle mit der Frequenz hat:
	
	\begin{equation}
		\nu = \frac{mc^2}{h},
		\label{eq:debroglie_frequency}
	\end{equation}
	
	Woraus sich die Periode oder charakteristische Zeit ergibt:
	
	\begin{equation}
		T = \frac{1}{\nu} = \frac{h}{mc^2} = \frac{2\pi\hbar}{mc^2} = 2\pi \Tfield,
		\label{eq:debroglie_period}
	\end{equation}
	
	Somit ist unsere intrinsische Zeit $\Tfield$ direkt mit de Broglies interner Uhrperiode durch einen Faktor von $2\pi$ verknüpft. Diese Verbindung verknüpft unseren Ansatz mit frühen Grundlagen der Quantentheorie und bietet eine physikalische Interpretation der intrinsischen Zeit als Periode der fundamentalen Oszillation, die mit der Ruheenergie eines Teilchens verbunden ist.
	
	De Broglies Versuch, eine kausale Interpretation der Quantenmechanik zu entwickeln \cite{deBroglie1927, deBroglie1930}, stieß in der Ära der Kopenhagener Interpretation auf Hindernisse, aber das Konzept der intrinsischen Zeit könnte Aspekte seines Programms innerhalb eines modernen theoretischen Rahmens wiederbeleben, der sowohl kausal als auch mit der Relativitätstheorie kompatibel ist.
	
	\subsection{Geometrische Interpretation in der Raumzeit}
	\label{subsec:geometric_interpretation}
	
	Das intrinsische Zeitfeld $\Tfield$ kann geometrisch innerhalb eines Raumzeitrahmens interpretiert werden. Im Gegensatz zu Standardansätzen, bei denen die Zeit eine externe Koordinate ist, erzeugt das intrinsische Zeitfeld eine massenabhängige Eigenzeit-Skala, die in der gesamten Raumzeit basierend auf der Verteilung von Materie und Energie variiert.
	
	In dieser Interpretation:
	
	\begin{enumerate}
		\item \textbf{Raumzeit-Metrik}: Das Standard-Raumzeit-Intervall $ds^2 = c^2dt^2 - dx^2 - dy^2 - dz^2$ wird modifiziert, um das intrinsische Zeitfeld einzubeziehen, und wird effektiv massenabhängig.
		
		\item \textbf{Lokale Zeitdilatation}: Anstelle einer koordinatenbasierten Zeitdilatation in der Relativitätstheorie erzeugt das intrinsische Zeitfeld einen lokalen, massenabhängigen "Zeitfluss" für Quantenprozesse.
		
		\item \textbf{Gekrümmter Konfigurationsraum}: Das Vorhandensein des intrinsischen Zeitfeldes krümmt effektiv den Konfigurationsraum von Quantensystemen und beeinflusst ihre Entwicklung in einer Weise, die analog dazu ist, wie Masse die Raumzeit in der Allgemeinen Relativitätstheorie krümmt.
	\end{enumerate}
	
	Diese geometrische Perspektive zeigt, wie der intrinsische Zeitansatz eine Brücke zwischen Quantenmechanik und relativistischen Prinzipien schafft, wobei die Quantenevolution durch ein Feld bestimmt wird, das im gesamten Raum variiert und von der Masse-Energie-Verteilung beeinflusst wird.
	
	\section{Erweiterung der Quantenmechanik}
	\label{sec:qm_extension}
	
	\subsection{Modifizierte Schrödinger-Gleichung}
	\label{subsec:modified_schrodinger}
	
	Die zentrale Erweiterung der Quantenmechanik beinhaltet die Modifikation der Schrödinger-Gleichung, um das intrinsische Zeitfeld $\Tfield$ einzubeziehen:
	
	\begin{equation}
		i\hbar \Tfield \frac{\partial}{\partial t} \Psi + i\hbar \Psi \frac{\partial \Tfield}{\partial t} = \hat{H} \Psi,
		\label{eq:modified_schrodinger}
	\end{equation}
	
	Diese Erweiterung hat mehrere wichtige Eigenschaften:
	
	\begin{enumerate}
		\item \textbf{Massenabhängige Evolution}: Systeme mit unterschiedlichen Massen entwickeln sich mit unterschiedlichen Raten basierend auf ihrem intrinsischen Zeitwert, $\Tfield = \frac{\hbar}{mc^2}$.
		
		\item \textbf{Kopplungsterm}: Der zweite Term, $i\hbar \Psi \frac{\partial \Tfield}{\partial t}$, koppelt die Wellenfunktion an Änderungen im intrinsischen Zeitfeld und schafft einen Feedback-Mechanismus.
		
		\item \textbf{Erhaltungsform}: Die Gleichung behält die korrekte Wahrscheinlichkeitserhaltung bei, während sie die Massenabhängigkeit der Evolution einführt.
	\end{enumerate}
	
	Für einen zeitunabhängigen Hamiltonian und ein statisches intrinsisches Zeitfeld nehmen die Lösungen die Form an:
	
	\begin{equation}
		\Psi(x,t) = \sum_n c_n \psi_n(x) e^{-iE_n t / \hbar \Tfield},
		\label{eq:modified_solution}
	\end{equation}
	
	was zeigt, dass die Zeitentwicklung durch $\Tfield$ skaliert wird, wobei sich schwerere Systeme (kleineres $\Tfield$) schneller in absoluter Zeit entwickeln.
	
	Für mehrere Teilchen, jedes mit seinem eigenen intrinsischen Zeitwert, wird die Evolution zu:
	
	\begin{equation}
		\Psi(x_1,...x_N,t) = \sum_n c_n \psi_n(x_1,...x_N) \exp\left(-i\frac{E_n t}{\hbar \sum_i \frac{1}{T_i}}\right),
		\label{eq:multi_particle}
	\end{equation}
	
	wobei die effektive Zeitskala durch eine Kombination der individuellen intrinsischen Zeiten bestimmt wird.
	
	\subsection{Quantenmessung als physikalischer Prozess}
	\label{subsec:quantum_measurement}
	
	Innerhalb dieses erweiterten Rahmens wird die Quantenmessung als physikalischer Prozess neu interpretiert, der die Wechselwirkung zwischen Systemen mit unterschiedlichen intrinsischen Zeitskalen beinhaltet. Die Hauptmerkmale dieses Ansatzes umfassen:
	
	\begin{enumerate}
		\item \textbf{Intrinsische Zeitdisparität}: Wenn ein Quantensystem mit einem Messapparat mit viel größerer Masse interagiert, unterscheiden sich ihre intrinsischen Zeiten signifikant: $\Tfield_{\text{System}} \gg \Tfield_{\text{Apparat}}$.
		
		\item \textbf{Dekohärenzmechanismus}: Diese Zeitskalen-Diskrepanz führt natürlicherweise zu einer schnellen Dekohärenz von kohärenten Überlagerungen, mit einer Rate proportional zum Massenverhältnis:
		
		\begin{equation}
			\Gamma_{\text{Dekohärenz}} \approx \Gamma_0 \frac{m_{\text{Apparat}}}{m_{\text{System}}},
			\label{eq:decoherence_rate}
		\end{equation}
		
		wobei $\Gamma_0$ eine Kopplungskonstante ist.
		
		\item \textbf{Emergenter Kollaps}: Was als Wellenfunktionskollaps erscheint, ist tatsächlich ein schneller Dekohärenzprozess, der für makroskopische Messgeräte effektiv instantan wird.
	\end{enumerate}
	
	Dieser Ansatz ähnelt der Dekohärenztheorie \cite{Joos1985, Zurek2003}, bietet jedoch einen spezifischen physikalischen Mechanismus durch intrinsische Zeitdisparität und sagt quantitativ die Dekohärenzrate basierend auf Massenverhältnissen voraus.
	
	Der Hauptunterschied zu Standard-Dekohärenzansätzen besteht darin, dass hier der Mechanismus intrinsisch für die beteiligten Systeme ist und nicht von Umgebungsfreiheitsgraden oder willkürlichen Grenzwerten abhängt. Das kollapsähnliche Verhalten entsteht natürlich aus der massenabhängigen Zeitentwicklung.
	
	\subsection{Lösung von Nichtlokalitätsparadoxa}
	\label{subsec:nonlocality_resolution}
	
	Der intrinsische Zeitrahmen bietet eine neuartige Perspektive auf die Quantennichtlokalität, die potenziell die Spannung mit der relativistischen Kausalität löst. Für verschränkte Teilchen:
	
	\begin{equation}
		|\Psi\rangle = \frac{1}{\sqrt{2}}(|0\rangle_A |1\rangle_B - |1\rangle_A |0\rangle_B),
		\label{eq:entangled_state_resolution}
	\end{equation}
	
	Traditionelle Darstellungen deuten auf einen instantanen Kollaps bei der Messung eines Teilchens hin. Im intrinsischen Zeitrahmen:
	
	\begin{enumerate}
		\item \textbf{Intrinsische Zeitverbindung}: Verschränkte Teilchen teilen eine effektive intrinsische Zeit, die ihre Evolution bestimmt, unabhängig von der räumlichen Trennung.
		
		\item \textbf{Keine tatsächliche Übertragung}: Messergebnisse sind Manifestationen vorexistierender Bedingungen, die in der gemeinsamen intrinsischen Zeitstruktur kodiert sind, nicht Signale, die zwischen Teilchen übertragen werden.
		
		\item \textbf{Bezugssystem-invariante Korrelation}: Das intrinsische Zeitfeld bietet einen bezugssystem-unabhängigen Mechanismus für Korrelationen, der in der Raumzeit nichtlokal erscheint, aber in einem erweiterten Konfigurationsraum lokal ist.
	\end{enumerate}
	
	Mathematisch entwickelt sich ein verschränkter Zustand gemäß seiner kombinierten intrinsischen Zeitstruktur:
	
	\begin{equation}
		|\Psi(t)\rangle = \frac{1}{\sqrt{2}} \Big( |0(t/T_A)\rangle_A |1(t/T_B)\rangle_B - |1(t/T_A)\rangle_A |0(t/T_B)\rangle_B \Big),
		\label{eq:time_entangled_evolution}
	\end{equation}
	
	wobei $T_A = \frac{\hbar}{m_A c^2}$ und $T_B = \frac{\hbar}{m_B c^2}$ die jeweiligen intrinsischen Zeiten der Teilchen sind.
	
	Dieser Ansatz widerspricht nicht Bells Theorem \cite{Bell1964, Bell1987}, sondern interpretiert seine Implikationen neu: Die Korrelationen werden durch die intrinsische Zeitstruktur aufrechterhalten, die von verschränkten Teilchen geteilt wird, nicht durch Signale, die sich durch die Raumzeit ausbreiten.
	
	\section{Erweiterung der Quantenfeldtheorie}
	\label{sec:qft_extension}
	
	\subsection{Intrinsisches Zeitfeld in der Lagrange-Formulierung}
	\label{subsec:intrinsic_lagrangian}
	
	Die Erweiterung der Quantenfeldtheorie erfordert die Einbeziehung des intrinsischen Zeitfeldes $\Tfield$ in die Lagrange-Formulierung. Die modifizierte Lagrange-Dichte für das Skalarfeld wird:
	
	\begin{equation}
		\mathcal{L}_{\text{Skalar}} = \frac{1}{2} \Tfield^2 \partial_\mu\phi \partial^\mu\phi - \frac{1}{2}m^2\Tfield^2\phi^2 + \frac{1}{2}\partial_\mu\Tfield\partial^\mu\Tfield - V(\Tfield),
		\label{eq:scalar_lagrangian}
	\end{equation}
	
	wobei der Potentialterm $V(\Tfield)$ die Selbstwechselwirkung des intrinsischen Zeitfeldes darstellt. Für Fermionen lautet die modifizierte Dirac-Lagrange-Dichte:
	
	\begin{equation}
		\mathcal{L}_{\text{Fermion}} = \bar{\psi} i \gamma^\mu \Tfield \partial_\mu \psi + \bar{\psi} i \gamma^\mu \psi \partial_\mu \Tfield - m\Tfield\bar{\psi}\psi,
		\label{eq:fermion_lagrangian}
	\end{equation}
	
	Das intrinsische Zeitfeld selbst hat eine Lagrange-Dichte:
	
	\begin{equation}
		\mathcal{L}_{\text{intrinsisch}} = \frac{1}{2} \partial_\mu \Tfield \partial^\mu \Tfield - \frac{1}{2}\Tfield^2 - \frac{\rho}{\Tfield},
		\label{eq:intrinsic_lagrangian}
	\end{equation}
	
	wobei $\rho$ die Masse-Energie-Dichte darstellt, die als Quelle für das intrinsische Zeitfeld dient.
	
	Diese Modifikationen führen zu Feldgleichungen, die die Standardfelder an das intrinsische Zeitfeld koppeln und einen einheitlichen Rahmen schaffen, in dem der Fortschritt der Quantenprozesse von der Masse-Energie-Verteilung abhängt.
	
	\subsection{Feldquantisierung mit variabler intrinsischer Zeit}
	\label{subsec:field_quantization}
	
	Die Quantisierung von Feldern in Gegenwart eines variablen intrinsischen Zeitfeldes erfordert eine sorgfältige Betrachtung der modifizierten kanonischen Kommutationsrelationen. Das erweiterte Quantisierungsverfahren umfasst:
	
	\begin{enumerate}
		\item \textbf{Modifizierte kanonische Impulse}: Für ein Skalarfeld wird der kanonische Impuls:
		
		\begin{equation}
			\pi_\phi = \frac{\partial \mathcal{L}}{\partial(\partial_0 \phi)} = \Tfield^2 \partial_0 \phi,
			\label{eq:modified_momentum}
		\end{equation}
		
		was zu positionsabhängigen Kommutationsrelationen führt.
		
		\item \textbf{Reskalierte Feldoperatoren}: Feldoperatoren werden mit Faktoren des intrinsischen Zeitfeldes reskaliert, um korrekte Kommutationsrelationen aufrechtzuerhalten:
		
		\begin{equation}
			[\phi(x), \pi_\phi(y)] = i\hbar \delta^3(x-y) \Rightarrow [\phi(x), \Tfield^2(y)\partial_0 \phi(y)] = i\hbar \delta^3(x-y),
			\label{eq:rescaled_commutators}
		\end{equation}
		
		\item \textbf{Modifizierte Propagatoren}: Green-Funktionen und Propagatoren beziehen das intrinsische Zeitfeld ein und modifizieren ihre Raumzeitabhängigkeit:
		
		\begin{equation}
			G(x,y) \sim \int \frac{d^4k}{(2\pi)^4} \frac{e^{-ik(x-y)}}{k^2 - m^2 + i\epsilon} \rightarrow \int \frac{d^4k}{(2\pi)^4} \frac{e^{-ik(x-y)}}{k_\mu \Tfield^2(x) k^\mu - m^2 + i\epsilon},
			\label{eq:modified_propagator}
		\end{equation}
	\end{enumerate}
	
	Diese Modifikationen schaffen eine Quantenfeldtheorie, bei der Ausbreitungs- und Wechselwirkungseigenschaften vom intrinsischen Zeitfeld beeinflusst werden, was potenziell Probleme in der Renormierung und im Hochenergieverhalten der Standard-QFT löst.
	
	\subsection{Higgs-Mechanismus und intrinsische Zeit}
	\label{subsec:higgs_intrinsic}
	
	Das intrinsische Zeitfeld hat eine besondere Beziehung zum Higgs-Mechanismus, der Teilchenmassen im Standardmodell erzeugt. Im erweiterten Rahmen:
	
	\begin{enumerate}
		\item \textbf{Zeit-Higgs-Kopplung}: Das intrinsische Zeitfeld koppelt an das Higgs-Feld durch die Beziehung $\Tfield = \frac{\hbar}{mc^2} = \frac{\hbar}{y v c^2}$, wobei $y$ die Yukawa-Kopplung und $v$ der Vakuumerwartungswert des Higgs-Feldes ist.
		
		\item \textbf{Modifizierte Higgs-Lagrange-Dichte}: Die Higgs-Lagrange-Dichte wird erweitert, um die direkte Kopplung an das intrinsische Zeitfeld einzubeziehen:
		
		\begin{equation}
			\mathcal{L}_{\text{Higgs-T}} = |\DhiggsT|^2 - \lambda(\Phi^* \Phi - v^2)^2 + \frac{1}{2}\partial_\mu \Tfield \partial^\mu \Tfield - V(\Tfield, \Phi),
			\label{eq:higgs_lagrangian}
		\end{equation}
		
		wobei $\DhiggsT = \Tfield (\partial_\mu + ig A_\mu) \Phi + \Phi \partial_\mu \Tfield$ die modifizierte kovariante Ableitung ist.
		
		\item \textbf{Symmetriebrechung}: Die spontane Symmetriebrechung im Higgs-Sektor beeinflusst das intrinsische Zeitfeld und schafft eine gekoppelte Evolution, bei der das Verhalten des Higgs-Feldes die Struktur des intrinsischen Zeitfeldes bestimmt.
	\end{enumerate}
	
	Diese Kopplung zwischen dem Higgs-Mechanismus und dem intrinsischen Zeitfeld bietet eine natürliche Erklärung für die Massenerzeugung und die damit verbundenen Quanten-Zeitskalen und verbindet das Standardmodell mit unserem erweiterten Quantenrahmen.
	
	\section{Experimentelle Vorhersagen und Tests}
	\label{sec:experimental}
	
	\subsection{Massenabhängige Dekohärenzraten}
	\label{subsec:mass_dependent_decoherence}
	
	Eine Schlüsselvorhersage des intrinsischen Zeitrahmens ist, dass Dekohärenzraten direkt von Teilchenmassen abhängen sollten:
	
	\begin{equation}
		\Gamma_{\text{Dekohärenz}} \propto \frac{m c^2}{\hbar},
		\label{eq:decoherence_mass}
	\end{equation}
	
	Dies führt zu testbaren Vorhersagen:
	
	\begin{enumerate}
		\item \textbf{Isotopen-Effekt}: Verschiedene Isotope desselben Elements sollten messbar unterschiedliche Dekohärenzraten in Quanteninterferenzexperimenten zeigen:
		
		\begin{equation}
			\frac{\Gamma_{\text{Isotop 1}}}{\Gamma_{\text{Isotop 2}}} = \frac{m_1}{m_2},
			\label{eq:isotope_ratio}
		\end{equation}
		
		\item \textbf{Skalierungsgesetz}: Die Quantenkohärenzzeit sollte über verschiedene Teilchenarten und Moleküle hinweg umgekehrt proportional zur Masse skalieren:
		
		\begin{equation}
			\tau_{\text{Kohärenz}} \propto \frac{1}{m},
			\label{eq:coherence_scaling}
		\end{equation}
		
		\item \textbf{Temperaturunabhängigkeit}: Im Gegensatz zur umgebungsinduzierten Dekohärenz sollte dieser intrinsische Mechanismus auch bei extrem niedrigen Temperaturen bestehen bleiben, was eine Möglichkeit bietet, ihn von thermischen Effekten zu unterscheiden.
	\end{enumerate}
	
	Experimente mit Materieinterferometern unter Verwendung verschiedener Massenteilchen \cite{Arndt1999, Hornberger2012, Fein2019} oder Präzisionsmessungen von Kohärenzzeiten in Quantensystemen unterschiedlicher Masse könnten diese Vorhersagen testen.
	
	\subsection{Modifizierte Verschränkungsdynamik}
	\label{subsec:entanglement_dynamics}
	
	Der intrinsische Zeitrahmen sagt eine modifizierte Dynamik für verschränkte Systeme voraus:
	
	\begin{enumerate}
		\item \textbf{Massenabhängiger Verschränkungszerfall}: Die Verschränkung zwischen Teilchen unterschiedlicher Massen sollte mit einer Rate zerfallen, die durch ihr Massenverhältnis bestimmt wird:
		
		\begin{equation}
			\frac{dC(t)}{dt} \propto \left|\frac{m_1 - m_2}{m_1 + m_2}\right|,
			\label{eq:entanglement_decay}
		\end{equation}
		
		wobei $C(t)$ ein Maß für die Verschränkung wie die Concurrence ist.
		
		\item \textbf{Frequenzabhängige Photonenkorrelationen}: Für verschränkte Photonen unterschiedlicher Frequenzen sollten Korrelationsmessungen leichte frequenzabhängige Verzögerungen zeigen, die von der Standard-Quantenmechanik nicht vorhergesagt werden:
		
		\begin{equation}
			\Delta t_{\text{Korrelation}} \propto \left|\frac{1}{\omega_1} - \frac{1}{\omega_2}\right|,
			\label{eq:frequency_delay}
		\end{equation}
		
		\item \textbf{Hybrid-System-Asymmetrie}: Hybride verschränkte Systeme (z.B. Atom-Photon-Verschränkung) sollten ein asymmetrisches Verhalten zeigen, das ihre unterschiedlichen intrinsischen Zeitskalen widerspiegelt.
	\end{enumerate}
	
	Diese Vorhersagen könnten in Quantenoptik-Experimenten mit frequenzdiversen verschränkten Photonen oder in hybriden Quantensystemen getestet werden, in denen Teilchen unterschiedlicher Massen verschränkt sind.
	
	\subsection{Gravitationsimplikationen}
	\label{subsec:gravitational_implications}
	
	Der intrinsische Zeitrahmen bietet eine potenzielle Brücke zur Quantengravitation:
	
	\begin{enumerate}
		\item \textbf{Modifiziertes Gravitationspotential}: Der Rahmen sagt ein modifiziertes Gravitationspotential voraus:
		
		\begin{equation}
			\Phi(r) = -\frac{G M}{r} + \kappa r,
			\label{eq:modified_potential}
		\end{equation}
		
		wobei der zusätzliche Term $\kappa r$ eine kleine Korrektur darstellt, die mit intrinsischen Zeitgradienten verknüpft ist.
		
		\item \textbf{Emergente Gravitationseffekte}: Quanteninterferenzexperimente in starken Gravitationsgradienten sollten Effekte zeigen, die über die hinausgehen, die von der Standard-Quantenmechanik gekoppelt an die Newtonsche Gravitation vorhergesagt werden.
		
		\item \textbf{Gravitationsinduzierte Dekohärenz}: Eine spezifische Vorhersage für die gravitationsinduzierte Dekohärenz ergibt sich:
		
		\begin{equation}
			\Gamma_{\text{grav}} \propto G \frac{M m}{\hbar r}
			\label{eq:grav_decoherence}
		\end{equation}
	\end{enumerate}
	
	Tests der Quantenkohärenz in starken Gravitationsfeldern oder Präzisionsmessungen von Gravitationseffekten auf Quantensysteme könnten potenziell diese Signaturen erkennen.
	
	\section{Schlussfolgerungen}
	\label{sec:conclusions}
	
	Diese Arbeit hat einen Rahmen zur Erweiterung der Standard-Quantenmechanik und Quantenfeldtheorie durch die Einführung eines intrinsischen Zeitfeldes $\Tfield = \frac{\hbar}{mc^2}$ vorgestellt. Diese Erweiterung adressiert mehrere grundlegende Beschränkungen der konventionellen Rahmenbedingungen:
	
	\begin{enumerate}
		\item Die modifizierte Schrödinger-Gleichung beinhaltet eine massenabhängige Zeitevolution und entfernt den privilegierten Status der Zeit als externen Parameter.
		
		\item Quantenmessung wird als physikalischer Prozess neu interpretiert, der aus der Wechselwirkung von Systemen mit unterschiedlichen intrinsischen Zeitskalen entsteht und eine deterministische Darstellung des scheinbaren Wellenfunktionskollapses bietet.
		
		\item Das Nichtlokalitätsparadoxon wird durch einen Rahmen adressiert, in dem Quantenkorrelationen aus gemeinsamen intrinsischen Zeitstrukturen entstehen anstatt aus Einflüssen, die schneller als Licht sind.
		
		\item Die Quantenfeldtheorie-Erweiterung bietet eine einheitliche Behandlung von Feldern, Teilchen und ihrer Wechselwirkung mit dem intrinsischen Zeitfeld, mit potenziellen Implikationen für die Renormierung und das Hochenergieverhalten.
	\end{enumerate}
	
	Der intrinsische Zeitansatz macht spezifische experimentelle Vorhersagen, einschließlich massenabhängiger Dekohärenzraten, modifizierter Verschränkungsdynamik und emergenter Gravitationseffekte. Diese Vorhersagen bieten konkrete Möglichkeiten, den Rahmen zu testen und ihn von der Standard-Quantenmechanik zu unterscheiden.
	
	Wenn validiert, könnte diese Erweiterung einen bedeutenden Fortschritt in unserem Verständnis von Quantenphänomenen darstellen, indem sie langjährige konzeptionelle Probleme löst und gleichzeitig den empirischen Erfolg des Standardrahmens beibehält. Das intrinsische Zeitfeld bietet eine natürliche Brücke zwischen Quantenmechanik und Relativitätstheorie und eröffnet einen Weg zu einer einheitlicheren Beschreibung der Natur auf allen Skalen.
	
	Zukünftige Arbeiten werden sich auf die Entwicklung detaillierterer quantitativer Vorhersagen für spezifische experimentelle Szenarien, die Verfeinerung des mathematischen Formalismus und die weitere Erforschung der Implikationen für die Kosmologie und die Grundlagen der Physik konzentrieren.
	
	\begin{thebibliography}{99}
		\bibitem{Schrodinger1926} E. Schrödinger, An Undulatory Theory of the Mechanics of Atoms and Molecules, Physical Review \textbf{28}, 1049 (1926).
		\bibitem{Heisenberg1925} W. Heisenberg, Quantum-Theoretical Re-interpretation of Kinematic and Mechanical Relations, Zeitschrift für Physik \textbf{33}, 879 (1925).
		\bibitem{Born1926} M. Born, Zur Quantenmechanik der Stoßvorgänge, Zeitschrift für Physik \textbf{37}, 863 (1926).
		\bibitem{Dirac1927} P. A. M. Dirac, The Quantum Theory of the Emission and Absorption of Radiation, Proceedings of the Royal Society A \textbf{114}, 243 (1927).
		\bibitem{Feynman1949} R. P. Feynman, Space-Time Approach to Quantum Electrodynamics, Physical Review \textbf{76}, 769 (1949).
		\bibitem{Schwinger1948} J. Schwinger, Quantum Electrodynamics. I. A Covariant Formulation, Physical Review \textbf{74}, 1439 (1948).
		\bibitem{Tomonaga1946} S. Tomonaga, On a Relativistically Invariant Formulation of the Quantum Theory of Wave Fields, Progress of Theoretical Physics \textbf{1}, 27 (1946).
		\bibitem{Dyson1949} F. J. Dyson, The Radiation Theories of Tomonaga, Schwinger, and Feynman, Physical Review \textbf{75}, 486 (1949).
		\bibitem{vonNeumann1932} J. von Neumann, \textit{Mathematical Foundations of Quantum Mechanics} (Princeton University Press, 1955), original auf Deutsch veröffentlicht 1932.
		\bibitem{Wheeler1983} J. A. Wheeler and W. H. Zurek, eds., \textit{Quantum Theory and Measurement} (Princeton University Press, 1983).
		\bibitem{Pauli1980} W. Pauli, \textit{General Principles of Quantum Mechanics} (Springer, 1980).
		\bibitem{Peres1980} A. Peres, Measurement of Time by Quantum Clocks, American Journal of Physics \textbf{48}, 552 (1980).
		\bibitem{Bell1964} J. S. Bell, On the Einstein Podolsky Rosen Paradox, Physics \textbf{1}, 195 (1964).
		\bibitem{EPR1935} A. Einstein, B. Podolsky, and N. Rosen, Can Quantum-Mechanical Description of Physical Reality Be Considered Complete?, Physical Review \textbf{47}, 777 (1935).
		\bibitem{Aspect1982} A. Aspect, P. Grangier, and G. Roger, Experimental Realization of Einstein-Podolsky-Rosen-Bohm Gedankenexperiment: A New Violation of Bell's Inequalities, Physical Review Letters \textbf{49}, 91 (1982).
		\bibitem{Joos1985} E. Joos and H. D. Zeh, The Emergence of Classical Properties Through Interaction with the Environment, Zeitschrift für Physik B \textbf{59}, 223 (1985).
		\bibitem{Zurek2003} W. H. Zurek, Decoherence, Einselection, and the Quantum Origins of the Classical, Reviews of Modern Physics \textbf{75}, 715 (2003).
		\bibitem{Leggett2002} A. J. Leggett, Testing the Limits of Quantum Mechanics: Motivation, State of Play, Prospects, Journal of Physics: Condensed Matter \textbf{14}, R415 (2002).
		\bibitem{Rovelli2004} C. Rovelli, \textit{Quantum Gravity} (Cambridge University Press, 2004).
		\bibitem{Kiefer2007} C. Kiefer, \textit{Quantum Gravity}, 2nd ed. (Oxford University Press, 2007).
		\bibitem{Oriti2009} D. Oriti, ed., \textit{Approaches to Quantum Gravity: Toward a New Understanding of Space, Time and Matter} (Cambridge University Press, 2009).
		\bibitem{pascher_zeit_2025} J. Pascher, Zeit als emergente Eigenschaft in der Quantenmechanik: Eine Verbindung zwischen Relativität, Feinstrukturkonstante und Quantendynamik, (2025).
		\bibitem{pascher_higgs_2025} J. Pascher, Mathematische Formulierung des Higgs-Mechanismus in der Zeit-Masse-Dualität, (2025).
		\bibitem{pascher_photons_2025} J. Pascher, Dynamische Masse von Photonen und ihre Implikationen für Nichtlokalität im T0-Modell, (2025).
		\bibitem{Page1983} D. N. Page and W. K. Wootters, Evolution Without Evolution: Dynamics Described by Stationary Observables, Physical Review D \textbf{27}, 2885 (1983).
		\bibitem{Busch1994} P. Busch, M. Grabowski, and P. J. Lahti, Time Observables in Quantum Theory, Physics Letters A \textbf{191}, 357 (1994).
		\bibitem{Wigner1967} E. P. Wigner, Remarks on the Mind-Body Question, in \textit{Symmetries and Reflections} (Indiana University Press, 1967), pp. 171-184.
		\bibitem{Wigner1963} E. P. Wigner, The Problem of Measurement, American Journal of Physics \textbf{31}, 6 (1963).
		\bibitem{deBroglie1930} L. de Broglie, \textit{An Introduction to the Study of Wave Mechanics} (E. P. Dutton and Company, 1930).
		\bibitem{Bohm1952} D. Bohm, A Suggested Interpretation of the Quantum Theory in Terms of "Hidden" Variables. I, Physical Review \textbf{85}, 166 (1952).
		\bibitem{Bell1987} J. S. Bell, \textit{Speakable and Unspeakable in Quantum Mechanics} (Cambridge University Press, 1987).
		\bibitem{Hensen2015} B. Hensen et al., Loophole-free Bell Inequality Violation Using Electron Spins Separated by 1.3 Kilometres, Nature \textbf{526}, 682 (2015).
		\bibitem{Eberhard1978} P. H. Eberhard, Bell's Theorem and the Different Concepts of Locality, Il Nuovo Cimento B \textbf{46}, 392 (1978).
		\bibitem{Ghirardi1980} G. C. Ghirardi, A. Rimini, and T. Weber, A General Argument Against Superluminal Transmission Through the Quantum Mechanical Measurement Process, Lettere al Nuovo Cimento \textbf{27}, 293 (1980).
		\bibitem{Aharonov1980} Y. Aharonov and D. Z. Albert, States and Observables in Relativistic Quantum Field Theories, Physical Review D \textbf{21}, 3316 (1980).
		\bibitem{Aharonov1981} Y. Aharonov and D. Z. Albert, Can We Make Sense Out of the Measurement Process in Relativistic Quantum Mechanics?, Physical Review D \textbf{24}, 359 (1981).
		\bibitem{Heisenberg1927} W. Heisenberg, Über den anschaulichen Inhalt der quantentheoretischen Kinematik und Mechanik, Zeitschrift für Physik \textbf{43}, 172 (1927).
		\bibitem{Mandelstam1945} L. Mandelstam and I. Tamm, The Uncertainty Relation Between Energy and Time in Non-relativistic Quantum Mechanics, Journal of Physics (USSR) \textbf{9}, 249 (1945).
		\bibitem{MacGibbon1987} J. H. MacGibbon, Can Planck-Mass Relics of Evaporating Black Holes Close the Universe?, Nature \textbf{329}, 308 (1987).
		\bibitem{Caldirola1953} P. Caldirola, Forze Non Conservative Nella Meccanica Quantistica, Il Nuovo Cimento \textbf{10}, 1747 (1953).
		\bibitem{Dirac1928} P. A. M. Dirac, The Quantum Theory of the Electron, Proceedings of the Royal Society A \textbf{117}, 610 (1928).
		\bibitem{Schrodinger1930} E. Schrödinger, Über die kräftefreie Bewegung in der relativistischen Quantenmechanik, Sitzungsberichte der Preussischen Akademie der Wissenschaften, Physikalisch-Mathematische Klasse \textbf{24}, 418 (1930).
		\bibitem{Hestenes1990} D. Hestenes, The Zitterbewegung Interpretation of Quantum Mechanics, Foundations of Physics \textbf{20}, 1213 (1990).
		\bibitem{deBroglie1923} L. de Broglie, Waves and Quanta, Nature \textbf{112}, 540 (1923).
		\bibitem{deBroglie1924} L. de Broglie, Recherches sur la Théorie des Quanta, Ph.D. thesis, Paris (1924).
		\bibitem{deBroglie1927} L. de Broglie, La Mécanique Ondulatoire et la Structure Atomique de la Matière et du Rayonnement, Journal de Physique et le Radium \textbf{8}, 225 (1927).
		\bibitem{Arndt1999} M. Arndt et al., Wave-Particle Duality of C60 Molecules, Nature \textbf{401}, 680 (1999).
		\bibitem{Hornberger2012} K. Hornberger, S. Gerlich, P. Haslinger, S. Nimmrichter, and M. Arndt, Colloquium: Quantum Interference of Clusters and Molecules, Reviews of Modern Physics \textbf{84}, 157 (2012).
		\bibitem{Fein2019} Y. Y. Fein et al., Quantum Superposition of Molecules Beyond 25 kDa, Nature Physics \textbf{15}, 1242 (2019).
	\end{thebibliography}
	
\end{document}