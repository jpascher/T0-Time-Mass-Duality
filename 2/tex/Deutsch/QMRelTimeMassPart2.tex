\documentclass[12pt,a4paper]{article}
\usepackage[utf8]{inputenc}
\usepackage[T1]{fontenc}
\usepackage[ngerman]{babel}
\usepackage{lmodern}
\usepackage{amsmath}
\usepackage{amssymb}
\usepackage{physics}
\usepackage{hyperref}
\usepackage{tcolorbox}
\usepackage{booktabs}
\usepackage{enumitem}
\usepackage[table,xcdraw]{xcolor}
\usepackage[left=2cm,right=2cm,top=2cm,bottom=2cm]{geometry}
\usepackage{pgfplots}
\pgfplotsset{compat=1.18}
\usepackage{graphicx}
\usepackage{float}
\usepackage{fancyhdr}
\usepackage{siunitx}
\usepackage{array}
\usepackage{cleveref}

% Headers and Footers
\pagestyle{fancy}
\fancyhf{}
\fancyhead[L]{Johann Pascher}
\fancyhead[R]{Zeit-Masse-Dualität}
\fancyfoot[C]{\thepage}
\renewcommand{\headrulewidth}{0.4pt}
\renewcommand{\footrulewidth}{0.4pt}

% Define acknowledgments environment
\newenvironment{acknowledgments}{\section*{Danksagung}}{\vspace{1em}}

% Custom commands
\newcommand{\Tfield}{T(x)}
\newcommand{\alphaEM}{\alpha_{\text{EM}}}
\newcommand{\alphaW}{\alpha_{\text{W}}}
\newcommand{\betaT}{\beta_{\text{T}}}
\newcommand{\Mpl}{M_{\text{Pl}}}
\newcommand{\Tzerot}{T_0(\Tfield)}
\newcommand{\Tzero}{T_0}
\newcommand{\vecx}{\vec{x}}
\newcommand{\gammaf}{\gamma_{\text{Lorentz}}}
\newcommand{\DhiggsT}{\Tfield (\partial_\mu + ig A_\mu) \Phi + \Phi \partial_\mu \Tfield}
\newcommand{\LCDM}{\Lambda\text{CDM}}
\newcommand{\DTmu}{D_{T,\mu}}
\newcommand{\calL}{\mathcal{L}}
\newcommand{\deq}{\displaystyle}
\newcommand{\e}{\mathrm{e}}

\hypersetup{
	colorlinks=true,
	linkcolor=blue,
	citecolor=blue,
	urlcolor=blue,
	pdftitle={Überbrückung von Quantenmechanik und Relativität durch Zeit-Masse-Dualität: Teil II},
	pdfauthor={Johann Pascher},
	pdfsubject={Theoretische Physik},
	pdfkeywords={T0-Modell, natürliche Einheiten, Zeit-Masse-Dualität, Kosmologie}
}

\begin{document}
	
	\title{Überbrückung von Quantenmechanik und Relativität durch Zeit-Masse-Dualität: Ein einheitlicher Rahmen mit natürlichen Einheiten \(\alpha = \beta = 1\) \\ Teil II: Kosmologische Implikationen und experimentelle Validierung}
	\author{Johann Pascher\\
		Abteilung für Kommunikationstechnik, \\Höhere Technische Bundeslehranstalt (HTL), Leonding, Österreich\\
		\texttt{johann.pascher@gmail.com}}
	\date{\today}
	
	\maketitle
	
	\begin{abstract}
		Diese Arbeit erweitert das in Teil I vorgestellte T0-Modell auf die Bereiche Kosmologie und experimentelle Validierung, aufbauend auf einem einheitlichen natürlichen Einheitensystem, in dem \(\hbar = c = G = k_B = \alphaEM = \alphaW = \betaT = 1\). Im Gegensatz zum expandierenden Universum des \(\LCDM\)-Modells schlagen wir einen statischen Kosmos vor, in dem Rotverschiebung durch Photonenergieverlust entsteht, der durch das intrinsische Zeitfeld \(\Tfield\) vermittelt wird. Dieser Rahmen reinterpretiert dunkle Materie und dunkle Energie durch emergente Gravitationseffekte und erweitert das Standardmodell um eine konsistente Gravitationstheorie. Zu den wichtigsten Vorhersagen gehören eine wellenlängenabhängige Rotverschiebung mit einer Variation von etwa \(2,3\%\) pro Dekade, eine kosmische Mikrowellenhintergrundtemperatur (CMB) von \(24000 \, \text{K}\) bei \(z = 1100\) (im Vergleich zu \(3000 \, \text{K}\) im Standardmodell) und spekulative Erweiterungen jenseits der Lichtgeschwindigkeit. Wir zeigen, dass das T0-Modell als komplementär zu einem erweiterten Standardmodell-Rahmen betrachtet werden kann, der einen krümmungsbasierten Rotverschiebungsmechanismus aufweist, obwohl sie sich in ihren grundlegenden Interpretationen von Zeit und Masse unterscheiden. Beide Ansätze sagen identische Beobachtungsergebnisse voraus, gehen jedoch von unterschiedlichen ontologischen Ausgangspunkten aus. Diese Vorhersagen können mit Instrumenten wie dem James Webb Space Telescope und zukünftigen CMB-Missionen getestet werden. Wir befassen uns mit Messfragen, wie den frequenzabhängigen Verzerrungen in Präzisionsmessungen und der Notwendigkeit einer Neukalibrierung kosmologischer Parameter, und bieten eine philosophisch kohärente Alternative zu \(\LCDM\), die theoretische Eleganz mit empirischer Strenge in Einklang bringt.
	\end{abstract}
	\newpage		
	\tableofcontents
	\newpage
	\section{Einleitung}
	In Teil I (\textit{Überbrückung von Quantenmechanik und Relativität durch Zeit-Masse-Dualität: Teil I}, \cite{pascher_part1_2025}) haben wir das T0-Modell als einheitlichen Rahmen für Quantenmechanik (QM) und Relativitätstheorie (RT) etabliert, der das intrinsische Zeitfeld \(\Tfield = \frac{\hbar}{\max(mc^2, \omega)}\) innerhalb eines natürlichen Einheitensystems (\(\hbar = c = G = k_B = \alphaEM = \alphaW = \betaT = 1\)) nutzt. Dieses System, das in Teil I, Abschnitt 2 ''Vereinheitlichung von Konstanten mit natürlichen Einheiten'' \cite{pascher_part1_2025} detailliert beschrieben wird, eliminiert empirisch bestimmte Konstanten und erreicht Konsistenz mit gemessenen Werten (z.B. \(c \approx 3 \times 10^8 \, \text{m/s}\), \(\alphaEM \approx 1/137,036\)) mit Abweichungen unter \(10^{-6}\). Es ermöglichte eine massenabhängige Schrödinger-Gleichung und emergente Gravitation, die mikro- und makroskopische Skalen überbrückt.
	
	Teil II erweitert diese Grundlagen auf Kosmologie und experimentelle Validierung und kontrastiert mit dem expandierenden Universum des \(\LCDM\)-Modells, das vor etwa 13,8 Milliarden Jahren aus einem Urknall entstanden ist \cite{Planck2020}. Im \(\LCDM\) ist die kosmische Rotverschiebung ein kinematischer Effekt (\(z \approx H_0 d / c\)), der Inflation und dunkle Energie erfordert \cite{Riess1998,Perlmutter1999}. Das T0-Modell schlägt ein statisches, unendliches und ewiges Universum vor, in dem Rotverschiebung aus Photonenergieverlust über \(\Tfield\) resultiert und das Standardmodell (SM) mit einer konsistenten Gravitationstheorie erweitert, während es seinen Teilchenphysik-Kern beibehält.
	
	Diese Arbeit untersucht auch die Perspektive, dass das T0-Modell und ein angemessen erweitertes Standardmodell komplementäre Beschreibungen derselben physikalischen Realität darstellen könnten. Wie in Abschnitt \ref{sec:extended_standard_model} ausgeführt, können wir durch die Einführung spezifischer Modifikationen am Standardmodell – insbesondere einer krümmungsbasierten Interpretation der Rotverschiebung, die auf dem linearen Term \(\kappa r\) im Gravitationspotential basiert – mathematische Äquivalenz zwischen beiden Rahmenwerken erreichen, während ihre unterschiedlichen ontologischen Grundlagen erhalten bleiben. Dies ist konzeptionell vergleichbar mit der Welle-Teilchen-Dualität in der Quantenmechanik, wo verschiedene mathematische Rahmenwerke dieselben experimentellen Ergebnisse von unterschiedlichen Ausgangspunkten aus beschreiben.
	
	Zu den wichtigsten Vorhersagen gehören:
	\begin{itemize}
		\item Wellenlängenabhängige Rotverschiebung (\(\sim 2,3\%\) pro Dekade)
		\item CMB-Temperatur von \(24000 \, \text{K}\) bei \(z = 1100\) (im Vergleich zu \(3000 \, \text{K}\) im Standardmodell)
		\item Ein modifiziertes Gravitationspotential \(\Phi(r) = -\frac{GM}{r} + \kappa r\), das Galaxienrotationskurven ohne dunkle Materie und kosmische Beschleunigung ohne dunkle Energie erklärt
		\item Spekulative superluminale Erweiterungen jenseits der konventionellen Physik
	\end{itemize}
	
	Diese sind mit JWST-Spektroskopie und CMB-Verzerrungsmessungen testbar, obwohl frequenzbasierte Methoden (z.B. GPS, Rotverschiebung) Massenvariation und Zeitdilatation vermischen, was eine sorgfältige Neubewertung erfordert \cite{pascher_quantum_2025}. Philosophisch vermeidet T0 Singularitäten und bietet einen kohärenten ewigen Kosmos \cite{pascher_perspective_2025}.
	
	Diese Arbeit ist wie folgt strukturiert:
	\begin{itemize}
		\item Abschnitt 2: Statisches Universumsmodell und Rotverschiebungsmechanismus
		\item Abschnitt 3: Kosmologische Phänomene und Vorhersagen
		\item Abschnitt 4: Quantitative Vorhersagen und Beobachtungssignaturen
		\item Abschnitt 5: Experimentelle Tests und Messherausforderungen
		\item Abschnitt 6: Erweitertes Standardmodell als komplementärer Rahmen
		\item Abschnitt 7: Implikationen von \(\betaT = 1\) und Neukalibrierung von Parametern
		\item Abschnitt 8: Spekulative Erweiterungen und philosophische Implikationen
		\item Abschnitt 9: Schlussfolgerung und Ausblick
	\end{itemize}
	
	\section{Statisches Universumsmodell}
	Das T0-Modell stellt sich ein statisches Universum vor, unendlich im Raum und ewig in der Zeit, im Gegensatz zum expandierenden Kosmos des \(\LCDM\) aus einem Urknall. Im \(\LCDM\) spiegelt die Rotverschiebung (\(z \approx H_0 d / c\)) die Expansion (\(H_0 \approx 70 \, \text{km/s/Mpc}\)) wider \cite{Planck2020}, was Inflation für Gleichförmigkeit und dunkle Energie für Beschleunigung (\(\Omega_{\Lambda} \approx 0,7\)) erfordert \cite{Riess1998}. T0 eliminiert diese, indem es einen stabilen Kosmos postuliert, in dem \(\Tfield\) die Dynamik ohne Expansion bestimmt.
	
	Vorteile umfassen:
	\begin{itemize}
		\item \textbf{Horizontproblem:} Unendliche Zeit gewährleistet thermisches Gleichgewicht über verschiedene Skalen \cite{pascher_messdifferenzen_2025}.
		\item \textbf{Flachheit:} Keine Expansion eliminiert Krümmungsabstimmung.
		\item \textbf{Singularitätsfrei:} Ewige Existenz vermeidet unendliche Dichte \cite{pascher_perspective_2025}.
		\item \textbf{Keine freien Parameter:} Die mathematische Struktur des T0-Modells setzt natürlicherweise \(\betaT = 1\) im entsprechenden Einheitensystem, wodurch Feinabstimmung überflüssig wird \cite{pascher_alphabeta_2025}.
	\end{itemize}
	
	Dies ergänzt die SM-Teilchenphysik mit einem statischen Gravitationsrahmen, der in Teil I abgeleitet wurde \cite{pascher_part1_2025}.
	
	Rotverschiebung im T0 ist:
	\begin{equation}
		1 + z = e^{\alpha d},
		\label{eq:redshift_distance}
	\end{equation}
	mit \(\alpha = H_0 / c \approx 2,3 \times 10^{-18} \, \text{m}^{-1}\) (SI) oder 1 (natürliche Einheiten). Bei niedrigem \(z\):
	\begin{equation}
		z \approx \alpha d,
		\label{eq:hubble_approx}
	\end{equation}
	übereinstimmend mit \(\LCDM\) lokal. Der Mechanismus ist Photonenergieverlust:
	\begin{equation}
		\frac{dE}{dx} = -\alpha E,
		\label{eq:energy_loss_rate}
	\end{equation}
	was zu \(E = E_0 e^{-\alpha d}\) und somit \(1 + z = e^{\alpha d}\) führt, wie von den \(\Tfield\)-Eigenschaften in Teil I abgeleitet \cite{pascher_part1_2025}.
	
	Dieser Energieverlustmechanismus führt wellenlängenabhängige Effekte ein, die natürlich aus der Wechselwirkung zwischen elektromagnetischer Strahlung und dem intrinsischen Zeitfeld entstehen:
	\begin{equation}
		\frac{dE}{dx} = -\alpha E \left(1 + \betaT \ln\left(\frac{\lambda}{\lambda_0}\right)\right),
		\label{eq:wavelength_energy_loss}
	\end{equation}
	
	Mit \(\betaT = 1\) in natürlichen Einheiten (\(\betaT^{\text{SI}} \approx 0,008\)) führt dies zur beobachtbaren wellenlängenabhängigen Rotverschiebung, die in Abschnitt \ref{subsec:wavelength_redshift} diskutiert wird \cite{pascher_params_2025}.
	
	\section{Kosmologische Phänomene}
	\(\LCDM\)s \(T(z) = T_0 (1 + z)\) ergibt \(T \approx 3000 \, \text{K}\) bei \(z = 1100\) (\(T_0 = 2,725 \, \text{K}\)) \cite{Fixsen2009}. T0 sagt voraus:
	\begin{equation}
		T(z) = T_0 (1 + z) (1 + \betaT \ln(1 + z)),
		\label{eq:temperature_redshift}
	\end{equation}
	
	Mit \(\betaT^{\text{SI}} \approx 0,008\) ergibt dies \(T(1100) \approx 4,4 \, \text{K}\) anstatt \(3,0 \, \text{K}\) zur Zeit der Rekombination. In natürlichen Einheiten mit \(\betaT = 1\) wird dies zu:
	\begin{equation}
		T(z) = T_0 (1 + z) (1 + \ln(1 + z)),
		\label{eq:temperature_redshift_natural}
	\end{equation}
	
	Dies entspricht \(T(1100) \approx 24000 \, \text{K}\) nach Umrechnung in SI-Einheiten, was den verstärkten Energieverlust (Gleichung \ref{eq:wavelength_energy_loss}) widerspiegelt \cite{pascher_temp_2025}.
	
	\paragraph{Neukalibrierung der Rekombinations-Rotverschiebung} Es ist wichtig zu beachten, dass der konventionell akzeptierte Wert von \(z = 1100\) für die Rekombinationsepoche innerhalb des \(\LCDM\)-Rahmens abgeleitet wurde. Eine grundlegende Neukalibrierung innerhalb des T0-Modellrahmens könnte einen überarbeiteten Wert näher bei \(z \approx 950\) ergeben, der immer noch die physikalische Bedingung von \(T \approx 3000 \, \text{K}\) erfüllt, die für die Wasserstoffrekombination erforderlich ist, während die modifizierte Temperatur-Rotverschiebungs-Beziehung verwendet wird \cite{pascher_temp_2025}.
	
	\subsection{Wellenlängenabhängige Rotverschiebung}
	\label{subsec:wavelength_redshift}
	
	T0 sagt voraus:
	\begin{equation}
		z(\lambda) = z_0 \left(1 + \betaT \ln\left(\frac{\lambda}{\lambda_0}\right)\right),
		\label{eq:wavelength_redshift}
	\end{equation}
	
	Mit \(\betaT^{\text{SI}} \approx 0,008\) ergibt dies \(\Delta z / z_0 \approx 3,85\%\) über den JWST-Beobachtungsbereich (0,6-28 \(\mu\text{m}\)), oder etwa 2,3\% pro Wellenlängendekade \cite{pascher_params_2025}. Dies steht im Gegensatz zur wellenlängenunabhängigen Rotverschiebung des \(\LCDM\) und bietet einen entscheidenden Beobachtungstest.
	
	\subsection{Neuinterpretation von dunkler Materie und dunkler Energie}
	\label{subsec:dark_reinterpretation}
	
	Das modifizierte Gravitationspotential des T0-Modells:
	\begin{equation}
		\Phi(r) = -\frac{GM}{r} + \kappa r,
		\label{eq:grav_potential_t0}
	\end{equation}
	
	mit \(\kappa \approx 4,8 \times 10^{-11} \, \text{m/s}^2\) reinterpretiert:
	\begin{itemize}
		\item \textbf{Dunkle Materie:} Galaxienrotationskurven folgen \(v(r) = \sqrt{\frac{GM}{r} + \kappa r}\), übereinstimmend mit Beobachtungen ohne dunkle Materie \cite{McGaugh2016}.
		\item \textbf{Dunkle Energie:} Der lineare Term \(\kappa r\) erzeugt eine effektive Vakuumenergiedichte \(\rho_{\text{DE}} \approx \frac{\kappa}{r^2}\), was die kosmische Beschleunigung erklärt \cite{pascher_galaxies_2025}.
	\end{itemize}
	
	Der \(\kappa\)-Parameter ist nicht willkürlich, sondern von fundamentalen physikalischen Parametern abgeleitet:
	\begin{equation}
		\kappa^{\text{nat}} = \betaT^{\text{nat}} \cdot \frac{yv}{r_g^2},
		\label{eq:kappa_derivation}
	\end{equation}
	
	wobei \(y\) die Yukawa-Kopplung, \(v\) der Higgs-Vakuumerwartungswert und \(r_g\) eine charakteristische Längenskala ist \cite{pascher_params_2025}.
	
	\subsection{Galaxien- und Clusterdynamik}
	\label{subsec:galaxy_dynamics}
	
	Das modifizierte Potential wirkt sich direkt auf Galaxienrotationskurven aus:
	\begin{equation}
		v(r) = \sqrt{\frac{GM}{r} + \kappa r},
		\label{eq:rotation_velocity}
	\end{equation}
	
	und erzeugt flache Rotationskurven ohne dunkle Materie, wie in Abbildung \ref{fig:rotation_curves} gezeigt.
	
	\begin{figure}[H]
		\centering
		\begin{tikzpicture}[scale=0.85]
			\begin{axis}[
				xlabel={Radius [kpc]},
				ylabel={Rotationsgeschwindigkeit [km/s]},
				xmin=0, xmax=30,
				ymin=0, ymax=300,
				legend pos=south east,
				grid=both,
				width=0.7\textwidth
				]
				\addplot[blue, dashed, thick, domain=0.1:30, samples=100] {220*sqrt(10/x)};
				\addplot[red, thick, domain=0.1:30, samples=100] {sqrt(220^2*10/x + 4.8*x^2)};
				\legend{Newtonsch, T0-Modell}
			\end{axis}
		\end{tikzpicture}
		\caption{Rotationskurven im Vergleich: Newtonsche Vorhersage (blau gestrichelt) und T0-Modell-Vorhersage (rot durchgezogen) für eine Galaxie mit \(M = 10^{11} M_{\odot}\), \(\kappa = 4,8 \times 10^{-11} \, \text{m/s}^2\).}
		\label{fig:rotation_curves}
	\end{figure}
	
	Für Galaxienhaufen macht das T0-Modell eindeutige Vorhersagen:
	\begin{equation}
		v_{\text{cluster}}(r) = \sqrt{\frac{GM_{\text{total}}}{r} + \kappa r},
		\label{eq:cluster_velocity}
	\end{equation}
	
	Der \(\kappa r\)-Term wird auf größeren Skalen bedeutender und erklärt beobachtete Massendiskrepanzen in Haufen wie dem Bullet Cluster ohne dunkle Materie \cite{pascher_emergente_2025}. Dies bietet eine einheitliche Erklärung für Phänomene über Galaxien- und Clusterskalen hinweg.
	
	\begin{table}[H]
		\centering
		\caption{Vergleich der Vorhersagen von \(\LCDM\) und T0-Modell für die Galaxiendynamik}
		\label{tab:galaxy_dynamics_comparison}
		\begin{tabular}{p{0.3\columnwidth} p{0.3\columnwidth} p{0.3\columnwidth}}
			\hline
			\textbf{Phänomen} & \textbf{\(\LCDM\)} & \textbf{T0-Modell} \\
			\hline
			Rotationskurve & Dunkle-Materie-Halo & \(\kappa r\)-Term \\
			Galaxienbildung & Dunkle-Materie-Kollaps & Baryonische Aggregation \\
			Clustermasse & Dunkle Materie dominant & Baryonisch + \(\Tfield\) \\
			Großskalige Struktur & Expansionsgetrieben & \(\Tfield\)-getrieben \\
			\hline
		\end{tabular}
	\end{table}
	
	\section{Quantitative Vorhersagen}
	\label{sec:predictions}
	
	\subsection{CMB-Temperaturvorhersage}
	\label{subsec:cmb_temp_prediction}
	
	Mit Gleichung \ref{eq:temperature_redshift_natural} und \(\betaT = 1\) in natürlichen Einheiten sagt das T0-Modell eine CMB-Temperatur zur konventionell definierten Rekombinationsepoche (\(z = 1100\)) von:
	\begin{equation}
		T(1100) \approx 24000 \, \text{K},
		\label{eq:cmb_temp_t0}
	\end{equation}
	
	im Vergleich zu \(\LCDM\)s \(3000 \, \text{K}\), aufgrund der logarithmischen Verstärkung von \(\Tfield\) (Gleichung \ref{eq:temperature_redshift_natural}).
	
	Diese höhere Temperatur beeinflusst direkt unser Verständnis der primordialen Nukleosynthese und der Strukturbildungsraten. Wie jedoch in Abschnitt \ref{subsec:cmb_temp} angemerkt, wäre eine Neukalibrierung der Rekombinations-Rotverschiebung innerhalb des T0-Rahmens notwendig. Die physikalische Bedingung für die Rekombination (Wasserstoff-Ionisationsgleichgewicht bei \(T \approx 3000 \, \text{K}\)) bleibt unverändert, aber die entsprechende Rotverschiebung wäre im T0-Modell niedriger (\(z \approx 950\)) \cite{pascher_temp_2025}.
	
	\subsection{Wellenlängenabhängige Rotverschiebungsvariation}
	\label{subsec:wavelength_redshift_prediction}
	
	Mit Gleichung \ref{eq:wavelength_redshift} und \(\betaT^{\text{SI}} \approx 0,008\) über den JWST-Beobachtungsbereich (0,6-28 \(\mu\text{m}\)):
	\begin{equation}
		\Delta z / z_0 \approx 3,85\%,
		\label{eq:wavelength_variation}
	\end{equation}
	
	oder \(2,3\%\) pro Wellenlängendekade, testbar über Quasar-Emissionslinien \cite{pascher_params_2025}.
	
	Diese Vorhersage steht im starken Kontrast zur wellenlängenunabhängigen Rotverschiebung des Standardmodells und bietet einen klaren, testbaren Unterschied zwischen den beiden Rahmenwerken.
	
	\subsection{Distanzmodul-Vorhersagen}
	\label{subsec:distance_modulus}
	
	Die modifizierte Rotverschiebungs-Distanz-Beziehung des T0-Modells beeinflusst das Distanzmodul:
	\begin{equation}
		\mu = 5\log_{10}\left(\frac{d_L}{10\text{ pc}}\right),
		\label{eq:distance_modulus}
	\end{equation}
	
	wobei die Leuchtkraftdistanz \(d_L\) von der Rotverschiebungs-Distanz-Beziehung (Gleichung \ref{eq:redshift_distance}) abgeleitet wird.
	
	Abbildung \ref{fig:distance_modulus} vergleicht die Vorhersagen von T0 und \(\LCDM\):
	
	\begin{figure}[H]
		\centering
		\begin{tikzpicture}[scale=0.85]
			\begin{axis}[
				xlabel={Rotverschiebung \(z\)},
				ylabel={Distanzmodul \(\mu\)},
				xmin=0, xmax=2,
				ymin=30, ymax=50,
				legend pos=north west,
				grid=both,
				width=0.7\textwidth,
				height=6cm,
				]
				\addplot[blue, thick, domain=0.01:2] {5*log10(3e8/70e3*ln(1+x)*(1+x)*0.1) + 25};
				\addplot[red, dashed, domain=0.01:2] {5*log10(3e8/70e3*(1+x)*(2-(1/(1+x)))*1) + 25};
				\legend{T0-Modell, \(\LCDM\)}
			\end{axis}
		\end{tikzpicture}
		\caption{Distanzmodul vs. Rotverschiebung im Vergleich: T0-Modell (blau durchgezogen) und \(\LCDM\) (rot gestrichelt). Beide Modelle können die beobachteten Supernova-Ia-Daten erklären, aber mit grundlegend unterschiedlichen physikalischen Interpretationen.}
		\label{fig:distance_modulus}
	\end{figure}
	
	Die Ähnlichkeit zwischen den Kurven erklärt, warum Supernova-Ia-Daten, die ursprünglich als Beweis für beschleunigte Expansion interpretiert wurden, auch durch das T0-Modell ohne dunkle Energie erklärt werden können \cite{pascher_galaxies_2025}.
	
	\section{Experimentelle Tests}
	\label{sec:tests}
	
	\subsection{JWST-Spektroskopie und wellenlängenabhängige Rotverschiebung}
	\label{subsec:jwst_test}
	
	Die Vorhersage der wellenlängenabhängigen Rotverschiebung (Gleichung \ref{eq:wavelength_redshift}) bietet einen Schlüsseltest zwischen den Modellen. Für eine Quelle bei \(z \approx 7\) sagt das T0-Modell voraus:
	\begin{equation}
		\Delta z / z \approx 3,85\%,
		\label{eq:redshift_variation_high_z}
	\end{equation}
	
	über den JWST-Beobachtungsbereich (0,6-28 \(\mu\text{m}\)).
	
	Die Präzisionsspektroskopie-Fähigkeiten des JWST sollten in der Lage sein, diese systematische Variation zu erkennen, indem Emissionslinien-Rotverschiebungen über verschiedene Wellenlängen verglichen werden. Das erwartete Signal beträgt etwa das 5-10-fache der spektroskopischen Unsicherheit des JWST, was dies zu einem durchführbaren kritischen Test macht \cite{pascher_params_2025}.
	
	\subsection{CMB-Verzerrungen}
	\label{subsec:cmb_distortions_test}
	
	Die modifizierte Temperatur-Rotverschiebungs-Beziehung sagt deutliche CMB-spektrale Verzerrungen voraus, charakterisiert durch die \(\mu\)- und \(y\)-Parameter:
	\begin{equation}
		\mu \approx 1,4 \times 10^{-5}, \quad y \approx 1,6 \times 10^{-6},
		\label{eq:distortion_parameters}
	\end{equation}
	
	gegenüber \(\LCDM\)s \(\mu \approx 2 \times 10^{-8}\), \(y \approx 4 \times 10^{-9}\).
	
	Diese Verzerrungen, etwa drei Größenordnungen größer als im Standardmodell, sollten mit zukünftigen Missionen wie dem Primordial Inflation Explorer (PIXIE) oder ähnlichen Instrumenten mit hoher spektraler Empfindlichkeit messbar sein \cite{pascher_temp_2025}.
	
	\subsection{Messherausforderungen}
	\label{subsec:measurement_challenges}
	
	\subsubsection{GPS und Uhrenpräzision}
	\label{subsec:gps_clock_problem}
	
	GPS-Uhren zeigen eine Frequenzverschiebung von \(\Delta f/f \approx 4,4 \times 10^{-10}\) pro Tag, traditionell interpretiert als gravitative Zeitdilatation. Im T0-Modell erscheint dies als Massenvariation, mit identischen Beobachtungsvorhersagen, aber einer grundlegend anderen Interpretation \cite{pascher_quantum_2025}.
	
	Um diese Interpretationen zu unterscheiden, wären nicht-frequenzbasierte Messmethoden erforderlich. Ein vielversprechender Ansatz beinhaltet den Vergleich von frequenzbasierten mit ladungs- oder massebasierten Messungen, die unterschiedlich auf die Effekte des intrinsischen Zeitfelds \(\Tfield\) reagieren würden \cite{pascher_quantum_2025}.
	
	\subsubsection{Kosmologische Beobachtungen}
	\label{subsec:cosmological_measurement_problem}
	
	Standard-Rotverschiebungsmessungen beruhen auf Frequenzverschiebungen, die nicht direkt zwischen expansionsbasierten und energieverlustbasierten Rotverschiebungsmechanismen unterscheiden können. Die wellenlängenabhängige Vorhersage des T0-Modells (Abschnitt \ref{subsec:wavelength_redshift}) bietet einen entscheidenden diskriminierenden Test.
	
	Zusätzliche Tests umfassen:
	\begin{itemize}
		\item Direkte Messungen kosmischer Zeitintervalle bei verschiedenen Rotverschiebungen
		\item Beobachtung von Zerfallsprozessen mit unterschiedlichen Zeitsignaturen
		\item Detaillierte Analyse von Gravitationslinsenphänomenen, die systematische Unterschiede zwischen Modellen zeigen sollten \cite{pascher_alphabeta_2025}
	\end{itemize}
	
	\subsubsection{Neukalibrierung kosmologischer Parameter}
	\label{subsec:parameter_recalibration}
	
	Eine grundlegende Herausforderung besteht darin, dass viele kosmologische Parameter (z.B. Rekombinations-Rotverschiebung, Alter des Universums) innerhalb des \(\LCDM\)-Rahmens abgeleitet werden und nicht direkt in das T0-Modell importiert werden können. Eine vollständige Neukalibrierung ist notwendig, insbesondere für:
	
	\begin{itemize}
		\item Rekombinations-Rotverschiebung (\(z \approx 950\) in T0 vs. \(z \approx 1100\) in \(\LCDM\))
		\item Baryonische akustische Oszillationsskala
		\item Winkelleistungsspektrum der CMB-Fluktuationen
		\item Interpretation der großskaligen Strukturbildung
	\end{itemize}
	
	Diese Neukalibrierung könnte potenziell Spannungen im Standardmodell lösen, wie die Hubble-Spannung, indem sie eine andere Interpretation derselben Beobachtungsdaten bietet \cite{DiValentino2021}.
	
	\section{Erweitertes Standardmodell als komplementärer Rahmen}
	\label{sec:extended_standard_model}
	
	\subsection{Ontologische Komplementarität}
	\label{subsec:ontological_complementarity}
	
	Das T0-Modell und das Standardmodell stellen komplementäre Beschreibungen der physikalischen Realität dar, analog zur Welle-Teilchen-Dualität in der Quantenmechanik. Beide können dieselben physikalischen Phänomene mit identischen Vorhersagen beschreiben, aber von unterschiedlichen ontologischen Ausgangspunkten:
	
	\begin{itemize}
		\item \textbf{Standardmodell-Paradigma:} Zeit ist relativ (unterliegt Dilatation), Ruhemasse ist konstant, Raum expandiert und Gravitation ist eine fundamentale Kraft.
		\item \textbf{T0-Modell-Paradigma:} Zeit ist absolut, Masse variiert, Raum ist statisch und Gravitation entsteht aus dem intrinsischen Zeitfeld \(\Tfield\).
	\end{itemize}
	
	Dieses Komplementaritätsprinzip legt nahe, dass unser Verständnis fundamentaler physikalischer Größen eher rahmenwerksabhängig als absolut sein könnte \cite{pascher_komplementaer_2025}.
	
	\subsection{Erweiterung des Standardmodells}
	\label{subsec:extending_standard_model}
	
	Um Kompatibilität mit dem T0-Modell zu erreichen, während das Kernprinzip der Zeitdilatation bewahrt wird, erfordert das Standardmodell spezifische Erweiterungen:
	
	\begin{enumerate}
		\item \textbf{Erweiterte Einstein-Feldgleichungen:}
		\begin{equation}
			G_{\mu\nu} + \kappa g_{\mu\nu} = 8\pi G T_{\mu\nu} + \nabla_{\mu}\Theta\nabla_{\nu}\Theta - \frac{1}{2}g_{\mu\nu}(\nabla_{\sigma}\Theta\nabla^{\sigma}\Theta)
			\label{eq:extended_einstein}
		\end{equation}
		wobei \(\Theta\) ein Skalarfeld ist, das Effekte berücksichtigt, die im T0-Modell \(\Tfield\) zugeschrieben werden.
		
		\item \textbf{Krümmungsbasierte Rotverschiebungsformel:} Die standardmäßige expansionsbasierte Rotverschiebungsinterpretation wird durch einen krümmungsbasierten Mechanismus ersetzt:
		\begin{equation}
			1 + z = e^{\alpha d}(1 + \beta \ln(\lambda/\lambda_0))
			\label{eq:extended_redshift}
		\end{equation}
		Dies entsteht aus der Wirkung des modifizierten Gravitationspotentials auf die Raumzeit (Gleichung \ref{eq:grav_potential_t0}).
		
		\item \textbf{Modifizierte Quantenevolution:}
		\begin{equation}
			i\hbar\frac{\partial\Psi}{\partial t} = [\hat{H} + \hat{H}_{\Theta}]\Psi
			\label{eq:extended_schrodinger}
		\end{equation}
		wobei \(\hat{H}_{\Theta}\) Massenabhängigkeit zur Zeitevolution einführt, mathematisch äquivalent zum Ansatz des T0-Modells.
	\end{enumerate}
	
	Diese Erweiterungen ermöglichen eine alternative Interpretation kosmologischer Phänomene ohne dunkle Energie oder universelle Expansion zu erfordern, während die relativistische Grundlage der Zeitdilatation des Standardmodells erhalten bleibt \cite{pascher_standardmod_2025}.
	
	\subsection{Mathematische Äquivalenz}
	\label{subsec:mathematical_equivalence}
	
	Das erweiterte Standardmodell und das T0-Modell erreichen mathematische Äquivalenz durch klar definierte Transformationen:
	
	\begin{table}[H]
		\centering
		\caption{Transformationsabbildung zwischen Standardmodell- und T0-Modell-Größen}
		\label{tab:transformation}
		\begin{tabular}{p{0.25\columnwidth} p{0.25\columnwidth} p{0.35\columnwidth}}
			\hline
			\textbf{Physikalische Größe} & \textbf{Transformationsbez.} & \textbf{Physikalische Interpretation} \\
			\hline
			Zeit & \(t = \Tzero/\gamma\) & Zeitdilatation vs. absolute Zeit \\
			Masse & \(m = m_0 \Tzero/\Tfield\) & Konstante Masse vs. variable Masse \\
			Energie & \(E = m_0 c^2 \Tzero/\Tfield\) & Bezugssystemabhängig vs. feldabhängig \\
			Gravitationspotential & \(\Phi_{t0} = \Phi_{sm}/c^2\) & Krümmung vs. Feldgradient \\
			\hline
		\end{tabular}
	\end{table}
	
	Diese Transformationen stellen sicher, dass beide Rahmenwerke identische experimentelle Ergebnisse vorhersagen, wenn sie richtig formuliert sind, was das Konzept der ontologischen Komplementarität veranschaulicht \cite{pascher_standardmod_2025}.
	
	\section{Implikationen von \(\betaT = 1\) und Parameterneukalibrierung}
	\label{sec:consequences_beta}
	
	\subsection{Theoretische Grundlage für \(\betaT = 1\)}
	\label{subsec:beta_foundation}
	
	Die Festlegung von \(\betaT = 1\) in natürlichen Einheiten ist nicht willkürlich, sondern leitet sich aus einer rigorosen theoretischen Grundlage ab:
	
	\begin{equation}
		\betaT = \frac{\lambda_h^2 v^2}{16 \pi^3 m_h^2 \xi}
		\label{eq:beta_derivation}
	\end{equation}
	
	wobei \(\lambda_h \approx 0,13\) die Higgs-Selbstkopplung, \(v \approx 246 \, \text{GeV}\) der Higgs-Vakuumerwartungswert, \(m_h \approx 125 \, \text{GeV}\) die Higgs-Masse und \(\xi = r_0/l_P \approx 1,33 \times 10^{-4}\) das Verhältnis der T0-charakter- \\istischen Länge zur Planck-Länge ist.
	
	Die Festlegung von \(\betaT = 1\) in natürlichen Einheiten bestimmt \(\xi\):
	
	\begin{equation}
		\xi = \frac{\lambda_h^2 v^2}{16 \pi^3 m_h^2} \approx 1,33 \times 10^{-4}
		\label{eq:xi_determination}
	\end{equation}
	
	Mit der Beziehung \(m_h^2 = 2\lambda_h v^2\):
	
	\begin{equation}
		\xi = \frac{\lambda_h}{32 \pi^3} \approx 1,31 \times 10^{-4}
		\label{eq:xi_simplified}
	\end{equation}
	
	Die Übereinstimmung zwischen diesen Ableitungen unterstützt \(\betaT = 1\) als natürlichen Fixpunkt im Renormierungsgruppen-Fluss:
	
	\begin{equation}
		\lim_{E \to 0} \betaT(E) = 1
		\label{eq:beta_fixed_point}
	\end{equation}
	
	Dies stellt einen bedeutenden theoretischen Fortschritt dar, der den T0-Parameter direkt mit Standardmodell-Eigenschaften verbindet \cite{pascher_alphabeta_2025}.
	
	\subsection{Umrechnung in SI-Einheiten}
	\label{subsec:conversion_si}
	
	Der SI-Wert von \(\betaT\) kann aus seinem Wert in natürlichen Einheiten abgeleitet werden:
	
	\begin{equation}
		\betaT^{\text{SI}} = \betaT^{\text{nat}} \cdot \frac{\xi \cdot l_{P,\text{SI}}}{r_{0,\text{SI}}},
		\label{eq:beta_conversion}
	\end{equation}
	
	was \(\betaT^{\text{SI}} \approx 0,008\) ergibt, übereinstimmend mit Beobachtungsbeschränkungen \cite{pascher_alphabeta_2025}.
	
	\subsection{Neukalibrierung von Parametern}
	\label{subsec:recalibration}
	
	Die Festlegung von \(\betaT = 1\) erfordert eine umfassende Neukalibrierung kosmologischer Parameter:
	
	\begin{enumerate}
		\item \textbf{Rekombinations-Rotverschiebung:} Die Bedingung \(T \approx 3000 \, \text{K}\) für die Wasserstoffrekombination tritt bei \(z \approx 950\) statt bei \(z \approx 1100\) auf, wenn Gleichung \ref{eq:temperature_redshift_natural} verwendet wird.
		
		\item \textbf{CMB-Winkelskala:} Die scheinbare 1°-Skala der CMB-Fluktuationen muss ohne Annahme einer Expansion neu interpretiert werden, stattdessen im Zusammenhang mit der intrinsischen Zeitskala bei der Rekombination \cite{pascher_vereinheitlichung_2025}.
		
		\item \textbf{Hubble-Parameter:} Im T0-Modell wird \(H_0\) als \(\alpha = H_0/c\) reinterpretiert und repräsentiert den Energiedämpfungskoeffizienten statt einer Expansionsrate, was potenziell die Hubble-Spannung löst \cite{pascher_params_2025}.
		
		\item \textbf{Alter des Universums:} Das Konzept wird in einem ewigen, statischen Universum bedeutungslos und eliminiert Beschränkungen durch Sternalter und kosmologische Zeitskalen \cite{pascher_perspective_2025}.
	\end{enumerate}
	
	Diese Neukalibrierung stellt eine grundlegende Perspektivenverschiebung dar, die eine sorgfältige Neuanalyse aller kosmologischen Daten erfordert.
	
	\section{Spekulative Erweiterungen und philosophische Implikationen}
	\label{sec:beyond_limits}
	
	\subsection{Jenseits der Planck-Skala}
	\label{subsec:beyond_planck}
	
	Das T0-Modell eröffnet Möglichkeiten für Physik jenseits der konventionellen Planck-Skala. Da \(\Tfield = \frac{\hbar}{mc^2}\), bei ultrahöhen Massen \(m > m_P\) (Planck-Masse) wird die intrinsische Zeit \(\Tfield < t_P\) (Planck-Zeit), was auf mögliche Dynamiken auf sub-Planck'schen Zeitskalen hindeutet. Umgekehrt, für ultraleichte Teilchen \(m < m_P\), entstehen langsamere Dynamiken mit \(\Tfield > t_P\), die potenziell mit kosmischen Skalen verbunden sind \cite{pascher_planck_2025}.
	
	\begin{figure}[H]
		\centering
		\begin{tikzpicture}
			\draw[->] (0,0) -- (6,0) node[right] {Masse \(m\)};
			\draw[->] (0,0) -- (0,4) node[above] {Zeit \(T\)};
			\draw[scale=0.5, domain=0.1:10, smooth, variable=\x, blue, thick] plot ({\x}, {1/\x});
			\draw[dotted, red] (1.5,0) -- (1.5,1.5) -- (0,1.5);
			\node at (1.5,-0.3) {\(m_P\)};
			\node at (-0.3,1.5) {\(t_P\)};
		\end{tikzpicture}
		\caption{Beziehung zwischen Masse und intrinsischer Zeit nahe der Planck-Skala, die inverse Proportionalität zeigt. Für Massen oberhalb von \(m_P\) tritt die intrinsische Zeit in den sub-Planck'schen Bereich ein.}
		\label{fig:mass_time}
	\end{figure}
	
	Dieser intrinsische-Zeit-Ansatz könnte eine natürliche Regularisierungsmethode für Quantengravitation bieten und die Unendlichkeiten vermeiden, die konventionelle Ansätze plagen \cite{pascher_planck_2025}.
	
	\subsection{Philosophische Implikationen}
	\label{subsec:philosophical_reflections}
	
	Das T0-Modell bietet tiefgreifende philosophische Implikationen für unser Verständnis des Kosmos:
	
	\begin{enumerate}
		\item \textbf{Ontologischer Status der Zeit:} Indem es Zeit als absolut statt relativ behandelt, fordert das T0-Modell das vorherrschende relativistische Paradigma heraus und deutet an, dass Zeit fundamentaler sein könnte als derzeit angenommen \cite{pascher_perspective_2025}.
		
		\item \textbf{Natur physikalischer Gesetze:} Der vereinheitlichte Rahmen mit \(\hbar = c = G = k_B = \alphaEM = \alphaW = \betaT = 1\) legt nahe, dass physikalische Gesetze einfacher und einheitlicher sein könnten als derzeit verstanden, wobei die scheinbare Komplexität aus unseren Messkonventionen entsteht \cite{pascher_alpha_2025}.
		
		\item \textbf{Ewiges Universum:} Durch die Postulierung eines statischen, ewigen Kosmos ohne Anfang oder Ende nähert sich das T0-Modell stärker an intuitive Vorstellungen von Existenz an und vermeidet die metaphysischen Komplexitäten der Schöpfung ex nihilo, die durch die Urknall-Kosmologie impliziert werden \cite{pascher_perspective_2025}.
		
		\item \textbf{Leib-Seele-Problem:} Ein absoluter Zeitrahmen könnte neue Perspektiven auf Bewusstsein und das Leib-Seele-Problem bieten, da er einen universellen Bezugsrahmen für kausale Prozesse liefert, die dem Bewusstsein zugrunde liegen \cite{pascher_dualismus_2025}.
	\end{enumerate}
	
	Diese philosophischen Dimensionen gehen über rein wissenschaftliche Überlegungen hinaus und könnten potenziell die Wissenschaft mit breiteren Fragen über die Natur der Realität verbinden.
	
	\section{Schlussfolgerung}
	\label{sec:conclusion}
	
	Diese erweiterte Erkundung des T0-Modells demonstriert sein Potenzial als umfassende Alternative zum standardkosmologischen Paradigma. Durch die Postulierung von absoluter Zeit, variabler Masse und dem intrinsischen Zeitfeld \(\Tfield\) als fundamental erreichen wir eine Vereinheitlichung von Quanten- und relativistischen Phänomenen, ohne Inflation, dunkle Materie oder dunkle Energie zu benötigen. Das Modell macht eindeutige experimentelle Vorhersagen, insbesondere bezüglich wellenlängenabhängiger Rotverschiebung und CMB-Temperatur, die mit aktueller und zukünftiger Technologie getestet werden können.
	
	Wir haben auch festgestellt, dass das T0-Modell als komplementär zu einem angemessen erweiterten Standardmodell-Rahmen betrachtet werden kann, wobei beide Ansätze identische Beobachtungsvorhersagen aus unterschiedlichen ontologischen Ausgangspunkten liefern. Diese Komplementarität legt nahe, dass das, was wir in der Physik als fundamental betrachten – Zeit, Masse, Raum – eher rahmenwerksabhängig als absolut sein könnte.
	
	Die Festlegung von \(\betaT = 1\) in natürlichen Einheiten, abgeleitet von Standardmodell-\\Parametern, stellt einen bedeutenden theoretischen Fortschritt dar, der zuvor getrennte Physikbereiche durch einen vereinheitlichten Rahmen verbindet, in dem Energie als einzige fundamentale Dimension dient. Diese Vereinheitlichung bringt beträchtliche Erklärungskraft mit sich, erfordert jedoch eine fundamentale Neukalibrierung kosmologischer Parameter und Interpretationen.
	
	Zukünftige Arbeiten sollten sich konzentrieren auf:
	\begin{itemize}
		\item Durchführung von Präzisionstests der wellenlängenabhängigen Rotverschiebung mit JWST
		\item Entwicklung detaillierter Simulationen der Strukturbildung im T0-Rahmen
		\item Erforschung nicht-frequenzbasierter Messansätze, um zwischen Zeitdilatation und Massenvariation zu unterscheiden
		\item Weitere Verfeinerung der theoretischen Verbindungen zwischen dem Standardmodell und dem T0-Modell
		\item Untersuchung möglicher experimenteller Signaturen sub-Planck'scher Physik, wie vom Modell vorgeschlagen
	\end{itemize}
	
	Zusammenfassend bietet das T0-Modell der Zeit-Masse-Dualität eine theoretisch elegante und empirisch testbare Alternative zur konventionellen Kosmologie, die potenziell langbestehende Probleme löst und gleichzeitig neue Einblicke in die fundamentale Natur der Realität bietet.
	
	\begin{acknowledgments}
		Dank an Reinsprecht Martin Dipl.-Ing. Dr. für kritisches Feedback und Diskussionen, die dazu beigetragen haben, diese Ideen zu verfeinern.
	\end{acknowledgments}
	
	\clearpage  % Force all figures to be processed before bibliography
	
	\begin{thebibliography}{99}
		\bibitem{pascher_part1_2025} J. Pascher, \href{https://github.com/jpascher/T0-Time-Mass-Duality/tree/main/2/pdf/Deutsch/QMRelTimeMassPart1Z.pdf}{Überbrückung von Quantenmechanik und Relativität durch Zeit-Masse-Dualität: Teil I}, 7. April 2025.
		\bibitem{pascher_lagrange_2025} J. Pascher, \href{https://github.com/jpascher/T0-Time-Mass-Duality/tree/main/2/pdf/Deutsch/MathZeitMasseLagrange.pdf}{Von der Zeitdilatation zur Massenvariation}, 29. März 2025.
		\bibitem{pascher_messdifferenzen_2025} J. Pascher, \href{https://github.com/jpascher/T0-Time-Mass-Duality/tree/main/2/pdf/Deutsch/MessdifferenzenT0Standard.pdf}{Analyse der Messdifferenzen zwischen T0 und \(\LCDM\)}, 2. April 2025.
		\bibitem{pascher_temp_2025} J. Pascher, \href{https://github.com/jpascher/T0-Time-Mass-Duality/tree/main/2/pdf/Deutsch/TempEinheitenCMB.pdf}{Anpassung der Temperatureinheiten und CMB-Messungen}, 2. April 2025.
		\bibitem{pascher_params_2025} J. Pascher, \href{https://github.com/jpascher/T0-Time-Mass-Duality/tree/main/2/pdf/Deutsch/ZeitMasseT0Params.pdf}{Herleitung der Parameter \(\kappa\), \(\alpha\) und \(\beta\)}, 4. April 2025.
		\bibitem{pascher_galaxies_2025} J. Pascher, \href{https://github.com/jpascher/T0-Time-Mass-Duality/tree/main/2/pdf/Deutsch/MassVarGalaxien.pdf}{Massenvariation in Galaxien}, 30. März 2025.
		\bibitem{pascher_quantum_2025} J. Pascher, \href{https://github.com/jpascher/T0-Time-Mass-Duality/tree/main/2/pdf/Deutsch/NotwendigkeitQMErweiterung.pdf}{Erweiterung der Quantenmechanik und QFT}, 27. März 2025.
		\bibitem{pascher_planck_2025} J. Pascher, \href{https://github.com/jpascher/T0-Time-Mass-Duality/tree/main/2/pdf/Deutsch/JenseitsPlanck.pdf}{Jenseits der Planck-Skala}, 24. März 2025.
		\bibitem{pascher_perspective_2025} J. Pascher, \href{https://github.com/jpascher/T0-Time-Mass-Duality/tree/main/2/pdf/Deutsch/ZeitRaumPascher.pdf}{Eine neue Perspektive auf Zeit und Raum}, 25. März 2025.
		\bibitem{pascher_alphabeta_2025} J. Pascher, \href{https://github.com/jpascher/T0-Time-Mass-Duality/tree/main/2/pdf/Deutsch/Alpha1Beta1Konsistenz.pdf}{Konsistenz von \(\alpha = 1\) und \(\beta = 1\)}, 5. April 2025.
		\bibitem{pascher_emergente_2025} J. Pascher, \href{https://github.com/jpascher/T0-Time-Mass-Duality/tree/main/2/pdf/Deutsch/EmergentGravT0.pdf}{Emergente Gravitation im T0-Modell}, 1. April 2025.
		\bibitem{pascher_qft_2025} J. Pascher, \href{https://github.com/jpascher/T0-Time-Mass-Duality/tree/main/2/pdf/Deutsch/QFTIntrinsischesZeitT0.pdf}{Quantenfeldtheoretische Behandlung des intrinsischen Zeitfelds im T0-Modell}, 8. April 2025.
		\bibitem{pascher_alpha_2025} J. Pascher, \href{https://github.com/jpascher/T0-Time-Mass-Duality/tree/main/2/pdf/Deutsch/NatEinheitenAlpha1.pdf}{Energie als fundamentale Einheit: Natürliche Einheiten mit \(\alpha = 1\) im T0-Modell}, 26. März 2025.
		\bibitem{pascher_komplementaer_2025} J. Pascher, \href{https://github.com/jpascher/T0-Time-Mass-Duality/tree/main/2/pdf/Deutsch/KomplementPhysikZeit.pdf}{Komplementäre Erweiterungen der Physik: Absolute Zeit und intrinsische Zeit}, 24. März 2025.
		\bibitem{pascher_dualismus_2025} J. Pascher, \href{https://github.com/jpascher/T0-Time-Mass-Duality/tree/main/2/pdf/Deutsch/KurzKomplementDualPhysik.pdf}{Komplementäre Dualität in der Physik}, 26. März 2025.
		\bibitem{pascher_vereinheitlichung_2025} J. Pascher, \href{https://github.com/jpascher/T0-Time-Mass-Duality/tree/main/2/pdf/Deutsch/T0VereinheitlichungDEGal.pdf}{Vereinheitlichung des T0-Modells: Grundlagen, dunkle Energie und Galaxiendynamik}, 4. April 2025.
		\bibitem{pascher_standardmod_2025} J. Pascher, \href{https://github.com/jpascher/T0-Time-Mass-Duality/tree/main/2/pdf/Deutsch/StandardModKruemmungRotv.pdf}{Vervollständigung des Standardmodells: Eine Erweiterung kompatibel mit dem T0-Modell der Zeit-Masse-Dualität}, 17. April 2025.
		\bibitem{Planck2020} Planck Collaboration, Astron. Astrophys. \textbf{641}, A6 (2020).
		\bibitem{Riess1998} A. G. Riess et al., Astron. J. \textbf{116}, 1009 (1998).
		\bibitem{Perlmutter1999} S. Perlmutter et al., Astrophys. J. \textbf{517}, 565 (1999).
		\bibitem{Fixsen2009} D. J. Fixsen, Astrophys. J. \textbf{707}, 916 (2009).
		\bibitem{McGaugh2016} S. S. McGaugh et al., Phys. Rev. Lett. \textbf{117}, 201101 (2016).
		\bibitem{DiValentino2021} E. Di Valentino et al., Class. Quantum Grav. \textbf{38}, 153001 (2021).
	\end{thebibliography}
	
\end{document}