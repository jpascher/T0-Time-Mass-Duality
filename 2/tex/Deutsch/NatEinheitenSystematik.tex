\documentclass[12pt,a4paper]{article}
\usepackage[utf8]{inputenc}
\usepackage[T1]{fontenc}
\usepackage[ngerman]{babel}
\usepackage{lmodern}
\usepackage{amsmath}
\usepackage{amssymb}
\usepackage{physics}
\usepackage{hyperref}
\usepackage{tcolorbox}
\usepackage{booktabs}
\usepackage{enumitem}
\usepackage[table,xcdraw]{xcolor}
\usepackage[left=2cm,right=2cm,top=2cm,bottom=2cm]{geometry}
\usepackage{pgfplots}
\pgfplotsset{compat=1.18}
\usepackage{graphicx}
\usepackage{float}
\usepackage{fancyhdr}
\usepackage{siunitx}
\usepackage{tikz}
\usepackage{adjustbox}
\usetikzlibrary{shapes.geometric}

% Benutzerdefinierte Befehle
\newcommand{\Tfield}{T(x)}
\newcommand{\alphaEM}{\alpha_{\text{EM}}}
\newcommand{\betaT}{\beta_{\text{T}}}
\newcommand{\Mpl}{M_{\text{Pl}}}
\newcommand{\Tzerot}{T_0(\Tfield)}
\newcommand{\e}{\mathrm{e}}
\newcommand{\alphaEMSI}{\alpha_{\text{EM,SI}}}

% Definiere globalen Skalierungsfaktor für Tabellen
\newcommand{\tablescale}{0.9}

% Konfiguration von Kopf- und Fußzeilen
\pagestyle{fancy}
\fancyhf{}
\fancyhead[L]{Johann Pascher}
\fancyhead[R]{Systematische Zusammenstellung natürlicher Einheiten}
\fancyfoot[C]{\thepage}
\renewcommand{\headrulewidth}{0.4pt}
\renewcommand{\footrulewidth}{0.4pt}

\hypersetup{
	colorlinks=true,
	linkcolor=blue,
	citecolor=blue,
	urlcolor=blue,
	pdftitle={Systematische Zusammenstellung natürlicher Einheiten mit Energie als Grundeinheit},
	pdfauthor={Johann Pascher},
	pdfsubject={Theoretische Physik},
	pdfkeywords={T0-Modell, natürliche Einheiten, Feinstrukturkonstante, vereinheitlichtes Einheitensystem, Zeit-Masse-Dualität}
}

\begin{document}
	
	\title{Systematische Zusammenstellung natürlicher Einheiten im T0-Modell mit Energie als Grundeinheit}
	\author{Johann Pascher}
	\date{13. April 2025}
	
	\maketitle
	
	\begin{abstract}
		Diese Arbeit präsentiert eine umfassende systematische Zusammenstellung natürlicher Einheiten im Rahmen des T0-Modells der Zeit-Masse-Dualität, wobei Energie als fundamentale Einheit verwendet wird. Durch die Normierung dimensionaler Konstanten (\(\hbar = c = G = k_B = 1\)) und dimensionsloser Kopplungskonstanten (\(\alpha_{\text{EM}} = \alpha_W = \beta_T = 1\)) auf Eins entsteht ein vereinheitlichtes Rahmenwerk, das quantenmechanische, relativistische und kosmologische Phänomene integriert. Die Zusammenstellung beschreibt detailliert die Hierarchie der Konstanten, quantisierte Längenskalen vom sub-Planck-Bereich bis zum kosmischen Regime sowie das überraschende Auftreten biologischer Strukturen in verbotenen Zonen. Elektromagnetische, thermodynamische und quantenmechanische Konstanten werden aus der Energieskala abgeleitet, wobei vereinfachte Feldgleichungen die innere Einheit der Naturgesetze offenbaren. Die Einstein-Hilbert-Wirkung bildet die Grundlage für emergente Gravitation, während präzise Umrechnungen in SI-Einheiten, philosophische Implikationen und experimentelle Perspektiven den Diskurs bereichern. Unterstützt durch umfangreiche theoretische Ableitungen und Visualisierungen bietet diese Arbeit eine robuste Grundlage für das T0-Modell und trägt potenziell zur Vereinheitlichung der Physik bei \cite{pascher_alphabeta_2025}.
	\end{abstract}
	
	\tableofcontents
	\newpage
	
	\section{Einführung}
	\label{sec:introduction}
	
	Natürliche Einheiten in der theoretischen Physik vereinfachen die Beschreibung physikalischer Gesetze, indem sie unabhängige Dimensionen reduzieren und fundamentale Konstanten auf Eins setzen, wodurch die innere Einfachheit komplexer Phänomene sichtbar wird. Traditionelle Systeme wie die Planck-Einheiten, bei denen \(\hbar = c = G = 1\) gesetzt wird, dienen seit langem als Eckpfeiler für theoretische Untersuchungen, da sie willkürliche dimensionale Parameter eliminieren und den Fokus auf die Essenz physikalischer Wechselwirkungen legen \cite{Planck1899}. Das T0-Modell der Zeit-Masse-Dualität erweitert dieses Paradigma jedoch, indem es ein vollständig vereinheitlichtes Einheitensystem vorschlägt, in dem nicht nur dimensionale Konstanten (\(\hbar = c = G = k_B = 1\)), sondern auch dimensionslose Kopplungskonstanten – die Feinstrukturkonstante \(\alpha_{\text{EM}}\), die Wien-Konstante \(\alpha_W\) und der modell-spezifische T0-Parameter \(\beta_T\) – auf Eins normiert werden. Diese Normierung ist keine bloße mathematische Vereinfachung, sondern eine tiefgreifende theoretische Notwendigkeit, die die Prämisse des Modells widerspiegelt, dass alle physikalischen Gesetze in ein einziges, energiebasierstes Rahmenwerk konvergieren \cite{pascher_zeit_2025}.
	
	Im Kern definiert das T0-Modell die fundamentale Beziehung zwischen Zeit und Masse neu und stellt konventionelle Annahmen sowohl der Relativitätstheorie als auch der Quantenmechanik in Frage. Im Gegensatz zur relativen Zeit der speziellen Relativitätstheorie oder der Behandlung der Zeit als bloßer Parameter in der Quantenmechanik setzt das T0-Modell Zeit als absolute Entität voraus, wobei die Masse dynamisch in Abhängigkeit vom Zustand des Systems variiert. Diese konzeptionelle Umkehrung wird durch das intrinsische Zeitfeld vermittelt, definiert als:
	\[
	\Tfield = \frac{\hbar}{\max(m c^2, \omega)}
	\]
	Dieses Skalarfeld kapselt die Wechselwirkung zwischen Masse-Energie und Frequenz ein und dient als verbindende Brücke zwischen dem mikroskopischen Bereich der Quantenmechanik und dem makroskopischen Bereich der Relativitätstheorie. Indem gravitative Effekte als emergente Phänomene interpretiert werden, die aus Gradienten in \(\Tfield\) resultieren, eliminiert das Modell die Notwendigkeit einer fundamentalen gravitativen Wechselwirkung, was mit modernen Theorien emergenter Gravitation übereinstimmt und eine neue Perspektive auf kosmische Dynamiken bietet \cite{Verlinde2011, pascher_emergente_2025}.
	
	Die Wahl der Energie als Grundeinheit im T0-Modell ist sowohl intuitiv als auch revolutionär. Energie als gemeinsame Währung physikalischer Wechselwirkungen ermöglicht es, alle Größen – Länge, Zeit, Masse, Temperatur – in Bezug auf \([E]\) oder dessen Inverse \([E^{-1}]\) auszudrücken, wie in Abschnitt \ref{sec:conversions} detailliert beschrieben. Diese Vereinheitlichung vereinfacht Feldgleichungen, wie in Abschnitt \ref{sec:field_equations} gezeigt, und offenbart hierarchische Beziehungen zwischen Konstanten und Skalen, die in den Abschnitten \ref{sec:hierarchy} und \ref{sec:length_scales} vorgestellt werden. Die Fähigkeit des Modells, Phänomene über Skalen hinweg zu erklären – von Quantenverschränkung bis hin zu kosmologischer Rotverschiebung und dunkler Energie – ohne ad-hoc-Konstrukte wie Inflation oder dunkle Materie zu bemühen, unterstreicht sein Potenzial, unser Verständnis des Universums neu zu gestalten \cite{pascher_energiedynamik_2025}.
	
	Diese Zusammenstellung zielt darauf ab, die natürlichen Einheiten des T0-Modells systematisch darzustellen und deren Definitionen, Werte und Verknüpfungen zu betonen. Sie untersucht die theoretischen Grundlagen für die Festlegung von \(\alpha_{\text{EM}} = \beta_T = 1\) (Abschnitt \ref{sec:derivations}), charakterisiert Längenskalen, die 97 Größenordnungen umfassen (Abschnitt \ref{sec:length_scales}), und hebt das überraschende Auftreten biologischer Strukturen in verbotenen Zonen hervor (Abschnitt \ref{sec:bio_anomalies}). Die Arbeit leitet weiterhin elektromagnetische, thermodynamische und quantenmechanische Konstanten aus der Energieskala ab und präsentiert vereinfachte Feldgleichungen, die die Einheit der Naturgesetze verdeutlichen (Abschnitt \ref{sec:field_equations}). Die Einstein-Hilbert-Wirkung bildet die Grundlage für emergente Gravitation (Abschnitt \ref{sec:gravitation}), während Umrechnungen in SI-Einheiten, philosophische Implikationen und experimentelle Perspektiven den Diskurs bereichern (Abschnitte \ref{sec:conversions}, \ref{sec:philosophy}). Visualisierungen wie Abbildungen \ref{fig:length_hierarchy} und \ref{fig:quantity_network} erhöhen die Klarheit und Verknüpfung.
	
	Die Ziele dieser Arbeit sind vielschichtig:
	\begin{itemize}
		\item Die hierarchische Struktur fundamentaler Konstanten und ihrer Werte zu umreißen, wie in Abschnitt \ref{sec:hierarchy} gezeigt.
		\item Strenge Ableitungen für die Normierung von \(\alpha_{\text{EM}}\) und \(\beta_T\) bereitzustellen, detailliert in Abschnitt \ref{sec:derivations}.
		\item Physikalische Längenskalen und deren Quantisierung zu charakterisieren, einschließlich biologischer Anomalien, wie in Abschnitten \ref{sec:length_scales} und \ref{sec:bio_anomalies} untersucht.
		\item Präzise Umrechnungsformeln zwischen natürlichen und SI-Einheiten anzubieten, wie in Abschnitt \ref{sec:conversions} dargestellt.
		\item Vereinfachte Feldgleichungen für Elektromagnetismus, das T0-Modell und die Quantenmechanik abzuleiten, analysiert in Abschnitt \ref{sec:field_equations}.
		\item Die emergente Gravitation über die Einstein-Hilbert-Wirkung zu erläutern, verglichen mit anderen Theorien in Abschnitt \ref{sec:gravitation}.
		\item Philosophische Implikationen, einschließlich ontologischer Diskretheit und emergenter Raumzeit, zu diskutieren, wie in Abschnitt \ref{sec:philosophy}.
		\item Experimentelle Tests zur Validierung der Vorhersagen des Modells vorzuschlagen, wie in Abschnitt \ref{sec:outlook} beschrieben \cite{pascher_vereinheitlichung_2025}.
	\end{itemize}
	
	Durch das Zusammenführen dieser Elemente strebt dieses Dokument an, eine robuste theoretische Grundlage für das T0-Modell zu schaffen, ein tieferes Verständnis der fundamentalen Struktur des Universums zu fördern und den Weg für zukünftige Forschungen zu ebnen.
	
	\section{Teil 1: Überblick über Einheiten und Skalen}
	\label{sec:hierarchy}
	
	\subsection{Stufe 1: Primäre dimensionale Konstanten}
	\label{subsec:level1}
	
	Das natürliche Einheitensystem des T0-Modells ist durch dimensionale Konstanten verankert, die auf Eins gesetzt sind, und legt die grundlegenden Skalen der Physik fest:
	\begin{itemize}
		\item \textbf{Reduzierte Planck-Konstante} (\(\hbar = 1\)): Definiert die Quantenskala, steuert die Energiequantisierung, entscheidend für Abschnitt \ref{subsec:quantum} \cite{Planck1899}.
		\item \textbf{Lichtgeschwindigkeit} (\(c = 1\)): Legt die relativistische Skala fest, vereinigt Raum und Zeit, wesentlich für Abschnitt \ref{sec:gravitation} \cite{Einstein1905}.
		\item \textbf{Gravitationskonstante} (\(G = 1\)): Etabliert die gravitative Skala, verknüpft mit emergenter Gravitation in Abschnitt \ref{sec:gravitation} \cite{Einstein1915}.
		\item \textbf{Boltzmann-Konstante} (\(k_B = 1\)): Definiert die thermodynamische Skala, verbindet Energie mit Temperatur, unterstützend für Abschnitt \ref{subsec:t0_equations}.
	\end{itemize}
	
	Dimensionslose Kopplungskonstanten, auf Eins gesetzt, bestimmen die Stärke der Wechselwirkungen:
	\begin{itemize}
		\item \textbf{Feinstrukturkonstante} (\(\alpha_{\text{EM}} = 1\)): SI-Wert \(\approx 1/137.036\), vereinfacht elektromagnetische Gleichungen (Abschnitt \ref{subsec:maxwell}) \cite{pascher_alpha_2025}.
		\item \textbf{Wien-Konstante} (\(\alpha_W = 1\)): SI-Wert \(\approx 2.82\), vereinheitlicht die Thermodynamik.
		\item \textbf{T0-Parameter} (\(\beta_T = 1\)): SI-Wert \(\approx 0.008\), zentral für die \(\Tfield\)-Dynamik (Abschnitt \ref{subsec:beta_derivation}) \cite{pascher_beta_2025}.
	\end{itemize}
	
	Diese Konstanten sind in Tabelle \ref{tab:fund_const} zusammengefasst und spiegeln ihre Rolle in der Hierarchie des T0-Modells wider.
	
	\begin{table}[htbp]
		\centering
		\begin{adjustbox}{width=\tablescale\textwidth}
			\begin{tabular}{llllll}
				\toprule
				\textbf{Konstante} & \textbf{Symbol} & \textbf{SI-Wert} & \textbf{Natürlicher Wert} & \textbf{Ableitung} & \textbf{Hierarchieebene} \\
				\midrule
				Reduzierte Planck-Konstante & \(\hbar\) & \(\SI{1.055e-34}{\joule\second}\) & 1 & Primär & Stufe 1 \\
				Lichtgeschwindigkeit & \(c\) & \(\SI{3e8}{\meter\per\second}\) & 1 & Primär & Stufe 1 \\
				Gravitationskonstante & \(G\) & \(\SI{6.674e-11}{\meter\cubed\per\kilogram\per\second\squared}\) & 1 & Primär & Stufe 1 \\
				Boltzmann-Konstante & \(k_B\) & \(\SI{1.381e-23}{\joule\per\kelvin}\) & 1 & Primär & Stufe 1 \\
				Feinstrukturkonstante & \(\alpha_{\text{EM}}\) & 1/137.036 & 1 & Sekundär & Stufe 2 \\
				Wien-Konstante & \(\alpha_W\) & 2.82 & 1 & Sekundär & Stufe 2 \\
				T0-Parameter & \(\beta_T\) & 0.008 & 1 & Sekundär & Stufe 2 \\
				\bottomrule
			\end{tabular}
		\end{adjustbox}
		\caption{Fundamentale Konstanten im T0-Modell, verknüpft mit Abschnitten \ref{subsec:alpha_derivation} und \ref{subsec:beta_derivation}}
		\label{tab:fund_const}
	\end{table}
	
	\subsection{Stufe 2.5: Abgeleitete elektromagnetische und gravitative Konstanten}
	\label{subsec:level2.5}
	
	Abgeleitete Konstanten entstehen natürlich mit vereinfachten Werten, was die Kohärenz des Systems widerspiegelt:
	\begin{itemize}
		\item \textbf{Vakuummagnetische Permeabilität} (\(\mu_0 = 1\)): Abgeleitet aus \(\mu_0 = 1/(\varepsilon_0 c^2)\), vereinfacht elektromagnetische Gleichungen (Abschnitt \ref{subsec:maxwell}).
		\item \textbf{Vakuumdielektrizitätskonstante} (\(\varepsilon_0 = 1\)): Aus \(\varepsilon_0 = 1/(\mu_0 c^2)\).
		\item \textbf{Vakuumimpedanz} (\(Z_0 = 1\)): Aus \(Z_0 = \sqrt{\mu_0/\varepsilon_0}\).
		\item \textbf{Elementarladung} (\(e = \sqrt{4\pi} \approx 3.544\)): Aus \(\alpha_{\text{EM}} = e^2/(4\pi \varepsilon_0 \hbar c) = 1\), wodurch die Ladung dimensionslos wird, wie in Abschnitt \ref{subsec:alpha_derivation} diskutiert \cite{Dirac1928}.
		\item \textbf{Planck-Druck} (\(p_P = 1\)): Aus \(p_P = c^7/(\hbar G^2)\).
		\item \textbf{Planck-Kraft} (\(F_P = 1\)): Aus \(F_P = c^4/G\).
		\item \textbf{Einstein-Hilbert-Wirkung}:
		\[
		S_{\text{EH}} = \frac{1}{16\pi} \int R \sqrt{-g} \, d^4x
		\]
		Zentral für die emergente Gravitation, wie in Abschnitt \ref{sec:gravitation} ausgeführt \cite{pascher_emergente_2025}.
	\end{itemize}
	
	Diese sind in Tabelle \ref{tab:em_const} zusammengefasst und verdeutlichen ihre Rolle bei der Vereinheitlichung physikalischer Wechselwirkungen.
	
	\begin{table}[htbp]
		\centering
		\begin{adjustbox}{width=\tablescale\textwidth}
			\begin{tabular}{llllll}
				\toprule
				\textbf{Konstante} & \textbf{Symbol} & \textbf{SI-Wert} & \textbf{Natürlicher Wert} & \textbf{Ableitung} & \textbf{Hierarchieebene} \\
				\midrule
				Vakuummagnetische Permeabilität & \(\mu_0\) & \(\SI{1.256637061e-6}{\henry\per\meter}\) (\(4\pi \times 10^{-7}\)) & 1 & \(\mu_0 = 1/(\varepsilon_0 c^2)\) & Stufe 2.5 \\
				Vakuumdielektrizitätskonstante & \(\varepsilon_0\) & \(\SI{8.85e-12}{\farad\per\meter}\) & 1 & \(\varepsilon_0 = 1/(\mu_0 c^2)\) & Stufe 2.5 \\
				Vakuumimpedanz & \(Z_0\) & \(\SI{376.73}{\ohm}\) & 1 & \(Z_0 = \sqrt{\mu_0/\varepsilon_0}\) & Stufe 2.5 \\
				Elementarladung & \(e\) & \(\SI{1.602e-19}{\coulomb}\) & \(\sqrt{4\pi} \approx 3.544\) & \(e = \sqrt{4\pi \varepsilon_0 \hbar c}\) & Stufe 2.5 \\
				Planck-Druck & \(p_P\) & \(\SI{4.63e113}{\pascal}\) & 1 & \(p_P = c^7/(\hbar G^2)\) & Stufe 2.5 \\
				Planck-Kraft & \(F_P\) & \(\SI{1.21e44}{\newton}\) & 1 & \(F_P = c^4/G\) & Stufe 2.5 \\
				\bottomrule
			\end{tabular}
		\end{adjustbox}
		\caption{Abgeleitete elektromagnetische und gravitative Konstanten im T0-Modell, verknüpft mit Abschnitten \ref{subsec:alpha_derivation}, \ref{subsec:maxwell} und \ref{sec:gravitation}}
		\label{tab:em_const}
	\end{table}
	
	\subsection{Planck-Einheiten im T0-Modell}
	\label{subsec:planck_units}
	
	Planck-Einheiten, abgeleitet aus \(\hbar\), \(c\) und \(G\), sind auf 1 normiert, wie in Tabelle \ref{tab:planck_units} gezeigt. Sie dienen als Referenzpunkte für alle Skalen, erleichtern Umrechnungen (Abschnitt \ref{sec:conversions}) und verankern die Hierarchie in Abbildung \ref{fig:length_hierarchy} \cite{pascher_planck_2025}.
	
	\begin{table}[htbp]
		\centering
		\begin{adjustbox}{width=\tablescale\textwidth}
			\begin{tabular}{lcccl}
				\toprule
				\textbf{Planck-Einheit} & \textbf{Symbol} & \textbf{Definition} & \textbf{SI-Wert} & \textbf{Bedeutung} \\
				\midrule
				Planck-Länge & \(l_P\) & \(\sqrt{\hbar G/c^3}\) & \(\SI{1.616e-35}{\meter}\) & Längeneinheit \\
				Planck-Zeit & \(t_P\) & \(\sqrt{\hbar G/c^5}\) & \(\SI{5.391e-44}{\second}\) & Zeiteinheit \\
				Planck-Masse & \(m_P\) & \(\sqrt{\hbar c/G}\) & \(\SI{2.176e-8}{\kilogram}\) & Masseneinheit \\
				Planck-Energie & \(E_P\) & \(\sqrt{\hbar c^5/G}\) & \(\SI{1.956e9}{\joule}\) & Energieeinheit \\
				Planck-Temperatur & \(T_P\) & \(\sqrt{\hbar c^5/G}/k_B\) & \(\SI{1.417e32}{\kelvin}\) & Temperatureinheit \\
				Planck-Ladung & \(q_P\) & \(\sqrt{4\pi \varepsilon_0 \hbar c}\) & \(\SI{1.875e-18}{\coulomb}\) & Ladungseinheit \\
				Planck-Druck & \(p_P\) & \(c^7/(\hbar G^2)\) & \(\SI{4.633e113}{\pascal}\) & Druckeinheit \\
				Planck-Dichte & \(\rho_P\) & \(c^5/(\hbar G^2)\) & \(\SI{5.155e96}{\kilogram\per\meter\cubed}\) & Dichteeinheit \\
				\bottomrule
			\end{tabular}
		\end{adjustbox}
		\caption{Planck-Einheiten im T0-Modell}
		\label{tab:planck_units}
	\end{table}
	
	\subsection{Karakteristische Längenskalen}
	\label{sec:length_scales}
	
	Das T0-Modell organisiert Längenskalen hierarchisch, wie in Tabelle \ref{tab:length_scales} detailliert beschrieben, und umfasst den Bereich von der Planck-Länge (\(l_P\)) bis zur kosmologischen Korrelationslänge (\(L_T\)). Diese Hierarchie, visualisiert in Abbildung \ref{fig:length_hierarchy}, deckt 97 Größenordnungen ab, wobei jede Skala mit spezifischen physikalischen Phänomenen verknüpft ist, wie in Abschnitt \ref{subsec:quantization} diskutiert.
	
	\begin{table}[H]
		\centering
		\begin{adjustbox}{width=\tablescale\textwidth}
			\begin{tabular}{lccc}
				\toprule
				\textbf{Struktur} & \textbf{Mit \(l_P = 1\)} & \textbf{Mit \(r_0 = 1\)} & \textbf{Beziehung} \\
				\midrule
				Planck-Länge (\(l_P\)) & 1 & \(1/\xi \approx 7519\) & Grundeinheit \\
				T0-Länge (\(r_0\)) & \(\xi \approx 1.33 \times 10^{-4}\) & 1 & \(\xi \cdot l_P\) \\
				Starke Skala & \(\sim 10^{-19}\) & \(\sim 10^{-15}\) & \(\sim \alpha_s \cdot \lambda_{C,h}\) \\
				Higgs-Länge (\(\lambda_{C,h}\)) & \(\sim 1.6 \times 10^{-20}\) & \(\sim 1.2 \times 10^{-16}\) & \(m_P/m_h \cdot l_P\) \\
				Protonenradius & \(\sim 5.2 \times 10^{-20}\) & \(\sim 3.9 \times 10^{-16}\) & \(\sim \alpha_s/(2\pi) \cdot \lambda_{C,p}\) \\
				Elektronenradius (\(r_e\)) & \(\sim 2.4 \times 10^{-23}\) & \(\sim 1.8 \times 10^{-19}\) & \(\alphaEMSI/(2\pi) \cdot \lambda_{C,e}\) \\
				Compton-Länge (\(\lambda_{C,e}\)) & \(\sim 2.1 \times 10^{-23}\) & \(\sim 1.6 \times 10^{-19}\) & \(m_P/m_e \cdot l_P\) \\
				Bohr-Radius (\(a_0\)) & \(\sim 2.9 \times 10^{-21}\) & \(\sim 2.2 \times 10^{-17}\) & \(\lambda_{C,e}/\alphaEMSI\) \\
				DNA-Breite & \(\sim 1.2 \times 10^{-26}\) & \(\sim 9.0 \times 10^{-23}\) & \(\sim \lambda_{C,e} \cdot m_e/m_{\text{DNA}}\) \\
				Zelle & \(\sim 6.2 \times 10^{-30}\) & \(\sim 4.7 \times 10^{-26}\) & \(\sim 10^7 \cdot \text{DNA}\) \\
				Mensch & \(\sim 6.2 \times 10^{-35}\) & \(\sim 4.7 \times 10^{-31}\) & \(\sim 10^5 \cdot \text{Zelle}\) \\
				Erd radius & \(\sim 3.9 \times 10^{-41}\) & \(\sim 2.9 \times 10^{-37}\) & \(\sim (m_P/m_{\text{Erde}})^2 \cdot l_P\) \\
				Sonnenradius & \(\sim 4.3 \times 10^{-43}\) & \(\sim 3.2 \times 10^{-39}\) & \(\sim (m_P/m_{\text{Sonne}})^2 \cdot l_P\) \\
				Sonnensystem & \(\sim 6.2 \times 10^{-47}\) & \(\sim 4.7 \times 10^{-43}\) & \(\sim \alpha_G^{-1/2} \cdot \text{Sonne}\) \\
				Galaxie & \(\sim 6.2 \times 10^{-56}\) & \(\sim 4.7 \times 10^{-52}\) & \(\sim (m_P/m_{\text{Galaxie}})^2 \cdot l_P\) \\
				Cluster & \(\sim 6.2 \times 10^{-58}\) & \(\sim 4.7 \times 10^{-54}\) & \(\sim 10^2 \cdot \text{Galaxie}\) \\
				Horizont (\(d_H\)) & \(\sim 5.4 \times 10^{61}\) & \(\sim 4.1 \times 10^{65}\) & \(\sim 1/H_0\) \\
				Korrelationslänge (\(L_T\)) & \(\sim 3.9 \times 10^{62}\) & \(\sim 2.9 \times 10^{66}\) & \(\sim \betaT^{-1/4} \cdot \xi^{-1/2} \cdot l_P\) \\
				\bottomrule
			\end{tabular}
		\end{adjustbox}
		\caption{Längenskalen}
		\label{tab:length_scales}
	\end{table}
	
	\begin{figure}[htbp]
		\centering
		\begin{tikzpicture}
			\small
			\draw[thick,->] (0,0) -- (12,0) node[right] {$\log(L/l_P)$};
			\draw[thick,->] (0,0) -- (0,5) node[above] {Energieskala};
			\draw (0,0.2) -- (0,-0.2) node[below] {$0$};
			\draw (1,0.2) -- (1,-0.2) node[below] {$r_0$};
			\draw (5,0.2) -- (5,-0.2) node[below] {$\lambda_{C,e}$};
			\draw (6,0.2) -- (6,-0.2) node[below] {$a_0$};
			\draw (11,0.2) -- (11,-0.2) node[below] {$L_T$};
			\node[blue] at (0,1.2) {$l_P = 1$};
			\node[blue] at (1,1.7) {$r_0 \approx 10^{-4}$};
			\node[blue] at (5,1.7) {$\lambda_{C,e} \approx 10^{-23}$};
			\node[blue] at (6,1.2) {$a_0 \approx 2.9 \times 10^{-21}$};
			\node[blue] at (11,1.2) {$L_T \approx 10^{62}$};
			\draw [decorate,decoration={brace,amplitude=10pt,mirror},xshift=0pt,yshift=-20pt]
			(0,0) -- (1,0) node [black,midway,yshift=-20pt] {Ursprungsskala};
			\draw [decorate,decoration={brace,amplitude=10pt,mirror},xshift=0pt,yshift=-20pt]
			(4.8,0) -- (6.2,0) node [black,midway,yshift=-20pt] {Quantenskala};
			\draw [decorate,decoration={brace,amplitude=10pt,mirror},xshift=0pt,yshift=-20pt]
			(10.8,0) -- (12,0) node [black,midway,yshift=-20pt] {Kosmologische Skala};
			\draw[dashed, red] (1,0.5) -- (5,0.5) node[midway, above] {$\propto m_h/m_e$};
			\draw[dashed, red] (6,0.5) -- (11,0.5) node[midway, above] {$\propto 1/H_0$};
		\end{tikzpicture}
		\caption{Hierarchie der Längenskalen im T0-Modell, von \(r_0\) bis \(L_T\), umfassend 66 Größenordnungen, wie in Abschnitt \ref{subsec:quantization} detailliert beschrieben.}
		\label{fig:length_hierarchy}
	\end{figure}
	
	\subsection{Biologische Anomalien in verbotenen Zonen}
	\label{sec:bio_anomalies}
	
	Eine bemerkenswerte Eigenschaft des T0-Modells ist das Auftreten biologischer Strukturen in „verbotenen Zonen“ zwischen quantisierten Skalen, wie in Tabelle \ref{tab:bio_anomalies} und Abbildung \ref{fig:stability_zones} gezeigt. Diese Zonen, diskutiert in Abschnitt \ref{subsec:quantization}, weisen keine stabilen physikalischen Strukturen auf, dennoch gedeihen biologische Systeme aufgrund einzigartiger Stabilisierungsmechanismen:
	\[
	\nabla^2 \Tfield_{\text{bio}} \approx -\frac{\rho}{\Tfield^2} + \delta_{\text{bio}}(x,t)
	\]
	Der Term \(\delta_{\text{bio}}\) berücksichtigt informationsbasierte, topologische und dynamische Stabilisierung, die Leben von unbelebter Materie unterscheidet \cite{pascher_dualismus_2025}.
	
	Die verbotenen Zonen, die etwa 19 und 3 Größenordnungen umfassen, repräsentieren Bereiche, in denen stabile physikalische Strukturen aufgrund der Quantisierung von Längenskalen fehlen. Die ~19-Größenordnungen-Lücke zwischen \( r_0 \approx 1.33 \times 10^{-4} l_P \) und \( \lambda_{C,e} \approx 2.1 \times 10^{-23} l_P \) ergibt sich aus dem Massenverhältnis \( m_h / m_e \approx 2.45 \times 10^{17} \), was zu einer logarithmischen Trennung von \( \log(m_h / m_e) \approx 19.39 \) führt. Die ~3-Größenordnungen-Lücke zwischen \( \lambda_{C,e} \approx 2.1 \times 10^{-23} l_P \) und \( a_0 \approx 2.9 \times 10^{-21} l_P \) entspricht der Feinstrukturkonstanten \( \alpha_{\text{EM,SI}} \approx 1/137.036 \), mit \( \log(1 / \alpha_{\text{EM,SI}}) \approx 2.14 \), zur Vereinfachung als ~3 angenähert. Diese Lücken werden berechnet als:
	\[
	\Delta \log(L / l_P) = \log\left(\frac{L_2}{L_1}\right)
	\]
	wobei \( L_1 \) und \( L_2 \) benachbarte quantisierte Skalen sind. Biologische Strukturen wie DNA und Zellen besetzen diese Zonen aufgrund dynamischer Stabilisierungsmechanismen, wie in Abschnitt \ref{subsec:quantization} diskutiert.
	
	\begin{table}[htbp]
		\centering
		\begin{adjustbox}{width=\tablescale\textwidth}
			\begin{tabular}{lccc}
				\toprule
				\textbf{Struktur} & \textbf{Größe} & \textbf{Verhältnis zu \(l_P\)} & \textbf{Position} \\
				\midrule
				DNA-Durchmesser & \(\SI{2e-9}{\meter}\) & \(\sim 10^{-26}\) & Verbotene Zone \\
				Protein & \(\SI{1e-8}{\meter}\) & \(\sim 10^{-27}\) & Verbotene Zone \\
				Bakterium & \(\SI{1e-6}{\meter}\) & \(\sim 10^{-29}\) & Verbotene Zone \\
				Zelle & \(\SI{1e-5}{\meter}\) & \(\sim 10^{-30}\) & Verbotene Zone \\
				Organismus & \(\SIrange{1e-3}{1}{\meter}\) & \(\sim 10^{-32} - 10^{-35}\) & Verbotene Zone \\
				\bottomrule
			\end{tabular}
		\end{adjustbox}
		\caption{Biologische Strukturen in verbotenen Zonen, visualisiert in Abbildung \ref{fig:stability_zones}}
		\label{tab:bio_anomalies}
	\end{table}
	
	\section{Teil 2: Detaillierte Erklärungen und Ableitungen}
	\label{sec:derivations}
	
	\subsection{Fundamentale Konzepte des T0-Modells}
	\label{subsec:concepts}
	
	Das T0-Modell definiert die Wechselwirkung zwischen Zeit und Masse neu, indem es Zeit als absolute Entität postuliert und Masse als variable Größe betrachtet, was die Paradigmen der Relativitätstheorie (relative Zeit, konstante Masse) und der Quantenmechanik (parametrische Zeit) herausfordert. Diese Verschiebung wird durch das intrinsische Zeitfeld ermöglicht:
	\[
	\Tfield = \frac{\hbar}{\max(m c^2, \omega)}
	\]
	Dieses Skalarfeld fasst die dynamische Beziehung zwischen Masse-Energie und Frequenz zusammen und wirkt als Vermittler, der quantenmechanische und relativistische Phänomene vereinheitlicht. Durch die Behandlung der Zeit als absolut interpretiert das Modell relativistische Effekte – wie Zeitdilatation – als Variationen der Masse, was eine neue Perspektive auf Phänomene wie gravitative Rotverschiebung und Teilchenwechselwirkungen bietet \cite{pascher_zeit_2025}.
	
	Die Normierung der Konstanten (\(\hbar = c = G = k_B = \alpha_{\text{EM}} = \alpha_W = \beta_T = 1\)) ist eine theoretische Notwendigkeit, die die Prämisse des Modells widerspiegelt, dass physikalische Gesetze inhärent vereinheitlicht sind. Energie als Grundeinheit ermöglicht es, alle Größen in einem konsistenten dimensionalen Rahmen auszudrücken, wie in Tabelle \ref{tab:practical_notation} gezeigt. Gravitation entsteht aus \(\Tfield\)-Gradienten, wodurch die Notwendigkeit einer fundamentalen gravitativen Kraft entfällt, wie in Abschnitt \ref{sec:gravitation} detailliert beschrieben. Diese Vereinheitlichung überbrückt Mikro- und Makroskalen und erklärt Phänomene von Quantenverschränkung bis zur kosmischen Expansion ohne ad-hoc-Annahmen \cite{pascher_emergente_2025}.
	
	\subsection{Ableitung von \(\beta_T = 1\)}
	\label{subsec:beta_derivation}
	
	Der T0-Parameter \(\beta_T\), der die Kopplung von \(\Tfield\) steuert, wird durch eine rigorose Ableitung, die an Standardmodell-Parameter gebunden ist, auf 1 normiert:
	\[
	\betaT = \frac{\lambda_h^2 v^2}{16 \pi^3} \cdot \frac{1}{m_h^2} \cdot \frac{1}{\xi}
	\]
	wobei:
	\begin{itemize}
		\item \(\lambda_h \approx 0.13\): Higgs-Selbstkopplung.
		\item \(v \approx \SI{246}{\giga\electronvolt}\): Higgs-Vakuumerwartungswert.
		\item \(m_h \approx \SI{125}{\giga\electronvolt}\): Higgs-Masse.
		\item \(\xi = r_0/l_P\): Verhältnis der T0-Länge zur Planck-Länge, wie in Tabelle \ref{tab:length_scales} gezeigt.
	\end{itemize}
	
	Durch Setzen von \(\betaT = 1\):
	\[
	\xi = \frac{\lambda_h^2 v^2}{16 \pi^3 m_h^2} \approx 1.33 \times 10^{-4}
	\]
	Dies ergibt \(r_0 \approx 1.33 \times 10^{-4} \cdot l_P\). Mit \(m_h^2 = 2 \lambda_h v^2\):
	\[
	\xi = \frac{\lambda_h}{32 \pi^3} \approx 1.31 \times 10^{-4}
	\]
	Die Konsistenz dieser Werte validiert die Ableitung, wie in Abbildung \ref{fig:energy_hierarchy} visualisiert. \(\beta_T = 1\) wirkt als Renormierungsfestpunkt:
	\[
	\lim_{E \to 0} \betaT(E) = 1
	\]
	Der SI-Wert \(\betaT \approx 0.008\) spiegelt endliche Energieeffekte wider und verstärkt die Kohärenz des Modells \cite{pascher_beta_2025}.
	
	\subsection{Ableitung von \(\alpha_{\text{EM}} = 1\)}
	\label{subsec:alpha_derivation}
	
	Die Normierung der Feinstrukturkonstanten ist entscheidend für den Elektromagnetismus:
	\[
	\alphaEM = \frac{e^2}{4 \pi \varepsilon_0 \hbar c} \approx \frac{1}{137.036}
	\]
	Mit \(\hbar = c = \varepsilon_0 = 1\), durch Setzen von \(\alphaEM = 1\):
	\[
	e^2 = 4 \pi \implies e = \sqrt{4 \pi} \approx 3.544
	\]
	Dies macht die Ladung dimensionslos und vereinfacht die Gleichungen in Abschnitt \ref{subsec:maxwell}. Alternativ, unter Verwendung des klassischen Elektronenradius \(r_e = e^2/(4 \pi \varepsilon_0 m_e c^2)\) und der Compton-Wellenlänge \(\lambda_C = h/(m_e c)\):
	\[
	\alphaEM = \frac{2 \pi r_e}{\lambda_C}
	\]
	Mit \(h = 2 \pi \hbar\) bestätigt dies die Standarddefinition \cite{pascher_alpha_2025}. Die Kopplung von \(\mu_0\) und \(\varepsilon_0\):
	\[
	\mu_0 \varepsilon_0 = \frac{1}{c^2} = 1
	\]
	vereinheitlicht elektromagnetische Wechselwirkungen, wie in Tabelle \ref{tab:em_const} gezeigt \cite{pascher_alphabeta_2025}. Der Bohr-Radius wird konsistent definiert als:
	\[
	a_0 = \frac{\lambda_{C,e}}{\alpha_{\text{EM,SI}}} \approx 2.9 \times 10^{-21} l_P
	\]
	
	\subsection{Verbindung zu Higgs-Parametern}
	\label{subsec:higgs}
	
	Die T0-Länge \(r_0\) ist direkt mit Standardmodell-Parametern verknüpft:
	\[
	r_0 = \xi \cdot l_P = \frac{\lambda_h^2 v^2}{16 \pi^3 m_h^2} \cdot l_P \approx 1.33 \times 10^{-4} \cdot l_P
	\]
	Mit \(m_h^2 = 2 \lambda_h v^2\):
	\[
	\xi = \frac{\lambda_h}{32 \pi^3} \approx 1.31 \times 10^{-4}
	\]
	Diese Verbindung, visualisiert in Abbildung \ref{fig:energy_hierarchy}, überbrückt die Quantenfeldtheorie und die emergente Gravitation und verstärkt die Kohärenz des Modells über Skalen hinweg, wie in Abschnitt \ref{sec:length_scales} diskutiert \cite{pascher_higgs_2025}.
	
	\subsection{Quantisierung von Längenskalen}
	\label{subsec:quantization}
	
	Das T0-Modell offenbart eine diskrete Hierarchie von Längenskalen, analog zu atomaren Energieniveaus, wie in Tabelle \ref{tab:length_scales} und Abbildung \ref{fig:length_hierarchy} gezeigt. Diese Quantisierung, ein Eckpfeiler des Modells, folgt:
	\[
	L_n = l_P \times \prod_i \alpha_i^{n_i}
	\]
	wobei \(\alpha_i = \{\alpha_{\text{EM}}, \beta_T, \xi\}\) und \(n_i\) Quantenzahlen sind. Wichtige Skalen umfassen:
	\begin{itemize}
		\item \textbf{Planck-Länge}: \(L_n = l_P\), \(n_i = 0\).
		\item \textbf{T0-Länge}: \(r_0 = \xi \cdot l_P\), \(n_\xi = 1\).
		\item \textbf{Compton-Wellenlänge}: \(\lambda_{C,e} \approx 10^{-23} l_P\), verknüpft mit Elektronendynamik (Abschnitt \ref{subsec:quantum}).
		\item \textbf{Bohr-Radius}: \(a_0 = \lambda_{C,e} / \alpha_{\text{EM,SI}} \approx 2.9 \times 10^{-21} l_P\), definiert atomare Strukturen.
		\item \textbf{Korrelationslänge}: \(L_T \approx 10^{62} l_P\), markiert den kosmischen Horizont \cite{pascher_planck_2025}.
	\end{itemize}
	
	\par % Sicherstellung des äußeren Absatzmodus
	
	\begin{figure}[htbp]
		\centering
		\begin{tikzpicture}
			\small
			\draw[thick,->] (-2,0) -- (12,0) node[right] {$\log(L/l_P)$};
			\draw[thick,->] (0,-0.5) -- (0,4) node[above] {Vorhandensein von Strukturen};
			\filldraw[blue] (0,3) circle (0.1) node[above] {$l_P$};
			\filldraw[blue] (1,2.8) circle (0.1) node[above] {$r_0 \approx 10^{-4}$};
			\filldraw[red] (5,3.2) circle (0.1) node[above] {$\lambda_{C,e} \approx 10^{-23}$};
			\filldraw[red] (5.5,3) circle (0.1) node[above right] {$a_0 \approx 2.9 \times 10^{-21}$};
			\filldraw[green] (7,2.5) circle (0.1) node[above] {DNA $\approx 10^{-26}$};
			\filldraw[green] (8,2.3) circle (0.1) node[above right] {Zelle $\approx 10^{-30}$};
			\filldraw[orange] (10,2.7) circle (0.1) node[above] {Planetar $\approx 10^{-41}$};
			\filldraw[purple] (11,3) circle (0.1) node[above] {$L_T \approx 10^{62}$};
			\draw[thick, dashed, red] (1.7,1.5) -- (4.5,1.5) node[midway, below] {Verbotene Zone ($\sim 19$ Größenordnungen)};
			\draw[thick, dashed, red] (5.8,1) -- (6.8,1) node[midway, below] {Verbotene Zone ($\sim 3$ Größenordnungen)};
			\draw[smooth, thick] (0,3) .. controls (0.5,2.5) and (0.8,2.8) .. (1,2.8)
			.. controls (1.2,2.6) and (1.5,0.5) .. (2,0.5)
			.. controls (4,0.5) and (4.7,2.5) .. (5,3.2)
			.. controls (5.2,3.1) and (5.5,0.5) .. (6,0.5)
			.. controls (6.5,0.5) and (6.8,2.3) .. (7,2.5)
			.. controls (7.2,2.4) and (7.5,2.3) .. (8,2.3)
			.. controls (8.5,2.2) and (9.5,2.5) .. (10,2.7)
			.. controls (10.3,2.8) and (10.8,2.9) .. (11,3);
			\node[align=center, below] at (3,0) {Stabile physikalische Skalen};
			\node[align=center, below] at (7.5,-0.5) {Biologische Anomalien};
		\end{tikzpicture}
		\caption{Stabilitätszentren und verbotene Zonen in der Längenskalenhierarchie des T0-Modells, mit Hervorhebung biologischer Anomalien (DNA bei \(\sim 10^{-26} l_P\), Zelle bei \(\sim 10^{-30} l_P\)). Die verbotenen Zonen, die ~19 und ~3 Größenordnungen umfassen, werden in Abschnitt \ref{sec:bio_anomalies} erklärt. Hinweis: Schematische Skalierung komprimiert \(\log(L/l_P)\) zur Lesbarkeit; tatsächliche Positionen sind DNA bei \(\log(10^{-26}) = -26\), Zelle bei \(\log(10^{-30}) = -30\). Verknüpft mit Tabelle \ref{tab:bio_anomalies}, referenziert in Abschnitt \ref{subsec:quantization} und diskutiert in Abschnitt \ref{sec:bio_anomalies}.}
		\label{fig:stability_zones}
	\end{figure}
	
	Verbotene Zonen, die Größenordnungen umfassen, werden für biologische Strukturen stabilisiert (Abschnitt \ref{sec:bio_anomalies}). Die Quantisierung ähnelt:
	\begin{itemize}
		\item \textbf{Atomare Orbitale}: Diskrete Energieniveaus, wie in Abbildung \ref{fig:orbital_analogy} visualisiert.
		\item \textbf{Bandlücken}: Verbotene Energiezomen in Festkörpern.
		\item \textbf{Resonanzen}: Harmonische Systemfrequenzen \cite{pascher_quantum_2025}.
	\end{itemize}
	
	\begin{figure}[htbp]
		\centering
		\begin{tikzpicture}
			\small
			\draw[thick,->] (0,0) -- (12,0) node[right] {$\log(L/l_P)$};
			\draw[thick,->] (0,-2) -- (0,2) node[above] {Vorhandensein von Strukturen};
			\node at (0,1) [circle, draw, fill=blue!20] {$l_P$};
			\node at (2,0.8) [circle, draw, fill=blue!20] {$r_0 \approx 10^{-4}$};
			\node at (5,1.2) [circle, draw, fill=red!20] {$\lambda_{C,e} \approx 10^{-23}$};
			\node at (5.5,1.1) [circle, draw, fill=red!20] {$a_0 \approx 2.9 \times 10^{-21}$};
			\node at (7,1) [circle, draw, fill=green!20] {DNA $\approx 10^{-26}$};
			\node at (8,0.9) [circle, draw, fill=green!20] {Zelle $\approx 10^{-30}$};
			\node at (11,1.5) [circle, draw, fill=purple!20] {$L_T \approx 10^{62}$};
			\draw[dashed] (0,0) -- (0,1);
			\draw[dashed] (2,0) -- (2,0.8);
			\draw[dashed] (5,0) -- (5,1.2);
			\draw[dashed] (5.5,0) -- (5.5,1.1);
			\draw[dashed] (7,0) -- (7,1);
			\draw[dashed] (8,0) -- (8,0.9);
			\draw[dashed] (11,0) -- (11,1.5);
			\node at (6,-1) {Verbotene Zone};
			\draw[decorate,decoration={brace,amplitude=5pt}] (5.5,0) -- (6.5,0) node[midway,below,yshift=-5pt] {$\sim 3$ Größenordnungen};
		\end{tikzpicture}
		\caption{Analogie zu atomaren Orbitalen, zeigt diskrete Skalen und verbotene Zonen mit biologischen Anomalien (DNA bei \(\sim 10^{-26} l_P\), Zelle bei \(\sim 10^{-30} l_P\)) im T0-Modell. Hinweis: Schematische Skalierung komprimiert \(\log(L/l_P)\); tatsächliche Positionen sind DNA bei \(\log(10^{-26}) = -26\), Zelle bei \(\log(10^{-30}) = -30\). Verknüpft mit Tabelle \ref{tab:bio_anomalies} und Abschnitt \ref{subsec:quantization}.}
		\label{fig:orbital_analogy}
	\end{figure}
	
	Empirische Validierungen umfassen:
	\begin{itemize}
		\item \textbf{Subatomar}: Teilchengrößen stimmen mit vorhergesagten Skalen überein.
		\item \textbf{Atomar}: Ansammlung um den Bohr-Radius.
		\item \textbf{Biologisch}: DNA und Zellen in verbotenen Zonen (Abschnitt \ref{sec:bio_anomalies}).
		\item \textbf{Kosmisch}: Galaxiengrößenkonzentrationen \cite{pascher_galaxies_2025}.
	\end{itemize}
	
	Testbare Vorhersagen sind:
	\begin{itemize}
		\item Keine stabilen Teilchen in verbotenen Zonen.
		\item Galaxiengrößen-Clustering, wie in Abbildung \ref{fig:stability_zones} gezeigt.
		\item Resonanzphänomene bei quantisierten Skalen \cite{pascher_quantum_2025}.
	\end{itemize}
	
	Philosophisch deutet dies an:
	\begin{enumerate}
		\item \textbf{Ontologische Diskretheit}: Die Realität ist geschichtet, nicht kontinuierlich.
		\item \textbf{Emergente Komplexität}: Neue Phänomene auf jeder Skala.
		\item \textbf{Überskalige Einheit}: Konstanten wie \(\xi\) verbinden Ebenen.
		\item \textbf{Deterministische Struktur}: Eine kosmische Ordnung, ähnlich einem „Periodensystem der Skalen“ \cite{pascher_perspective_2025}.
	\end{enumerate}
	
	\subsection{Einstein-Hilbert-Wirkung und emergente Gravitation}
	\label{sec:gravitation}
	
	Das T0-Modell interpretiert Gravitation neu über die Einstein-Hilbert-Wirkung:
	\[
	S_{\text{EH}} = \frac{1}{16 \pi} \int (R - 2 \kappa) \sqrt{-g} \, d^4x
	\]
	Das modifizierte Potential:
	\[
	\Phi(r) = -\frac{M}{r} + \kappa r
	\]
	mit \(\kappa \approx \SI{4.8e-11}{\meter\per\second\squared}\), erklärt dunkle Energie natürlich, verknüpft mit \(\Lambda_{\text{eff}} = \kappa\). Gravitation entsteht aus:
	\[
	\Phi(\vec{x}) = -\ln\left(\frac{\Tfield}{\Tfield_0}\right)
	\]
	Dies vereinheitlicht quantenmechanische und kosmische Skalen, wie in Tabelle \ref{tab:theory_comparison} gezeigt \cite{pascher_emergente_2025}.
	
	\begin{table}[htbp]
		\centering
		\begin{adjustbox}{width=\tablescale\textwidth}
			\begin{tabular}{p{3cm}p{3cm}p{4cm}p{4cm}}
				\toprule
				\textbf{Theorie} & \textbf{Prinzip} & \textbf{Potential} & \textbf{Vergleich mit T0} \\
				\midrule
				Newtonsche & Kraft & \(-\frac{G M}{r}\) & T0-Sonderfall (\(\kappa = 0\)) \\
				Allgemeine Relativität & Krümmung & Schwarzschild & Äquivalent in schwachen Feldern \\
				MOND & Modifizierte Dynamik & \(\mu(\nabla \Phi/a_0)\) & T0 liefert Basis \\
				f(R)-Theorien & Modifizierte Wirkung & Variiert & T0: f(R) = R - 2\(\kappa\) G \\
				T0-Modell & Zeitfeld & \(-\frac{M}{r} + \kappa r\) & Vereinheitlicht QM und Gravitation \\
				\bottomrule
			\end{tabular}
		\end{adjustbox}
		\caption{Vergleich der Gravitationstheorien, verknüpft mit Abschnitt \ref{subsec:t0_equations}}
		\label{tab:theory_comparison}
	\end{table}
	
	\subsection{Feldgleichungen}
	\label{sec:field_equations}
	
	\subsubsection{Maxwell-Gleichungen}
	\label{subsec:maxwell}
	
	Mit \(\alphaEM = 1\), \(\varepsilon_0 = \mu_0 = 1\), vereinfachen sich die Maxwell-Gleichungen, wie in Tabelle \ref{tab:maxwell} gezeigt. Die dimensionslose Ladung \(e = \sqrt{4\pi}\) vereinheitlicht die Feld-Dimensionen, wie in Abschnitt \ref{subsec:alpha_derivation} diskutiert \cite{pascher_alpha_2025}.
	
	\begin{table}[htbp]
		\centering
		\begin{adjustbox}{width=\tablescale\textwidth}
			\begin{tabular}{llll}
				\toprule
				\textbf{Gleichung} & \textbf{Klassische Form} & \textbf{Natürliche Form} & \textbf{Vereinfachung} \\
				\midrule
				Gauß’sches Gesetz & \(\nabla \cdot \vec{E} = \frac{\rho}{\varepsilon_0}\) & \(\nabla \cdot \vec{E} = \rho\) & Direkte Quelle \\
				Ampère’sches Gesetz & \(\nabla \times \vec{B} - \mu_0 \varepsilon_0 \frac{\partial \vec{E}}{\partial t} = \mu_0 \vec{j}\) & \(\nabla \times \vec{B} - \frac{\partial \vec{E}}{\partial t} = \vec{j}\) & Direkte Quelle \\
				Gauß für Magnetismus & \(\nabla \cdot \vec{B} = 0\) & \(\nabla \cdot \vec{B} = 0\) & Unverändert \\
				Faraday’sches Gesetz & \(\nabla \times \vec{E} + \frac{\partial \vec{B}}{\partial t} = 0\) & \(\nabla \times \vec{E} + \frac{\partial \vec{B}}{\partial t} = 0\) & Unverändert \\
				\bottomrule
			\end{tabular}
		\end{adjustbox}
		\caption{Maxwell-Gleichungen in natürlichen Einheiten, verknüpft mit Tabelle \ref{tab:em_const}}
		\label{tab:maxwell}
	\end{table}
	
	\subsubsection{T0-Modell-Gleichungen}
	\label{subsec:t0_equations}
	
	Die Gleichungen des T0-Modells, gezeigt in Tabelle \ref{tab:t0_equations}, spiegeln emergente Gravitation und kosmische Dynamiken wider, verknüpft mit Abschnitt \ref{sec:gravitation} \cite{pascher_emergente_2025}.
	
	\begin{table}[htbp]
		\centering
		\begin{adjustbox}{width=\tablescale\textwidth}
			\begin{tabular}{lll}
				\toprule
				\textbf{Gleichung} & \textbf{Natürliche Form} & \textbf{Bedeutung} \\
				\midrule
				Temperatur-Rotverschiebung & \(T(z) = T_0 (1+z)(1+\ln(1+z))\) & Kosmische Temperatur \\
				Wellenlängen-Rotverschiebung & \(z(\lambda) = z_0 (1+\ln(\lambda/\lambda_0))\) & Frequenzabhängig \\
				Gravitationspotential & \(\Phi(r) = -\frac{M}{r} + r\) & Emergente Gravitation \\
				Intrinsisches Zeitfeld & \(\nabla^2 \Tfield \approx -\frac{\rho}{\Tfield^2}\) & Quellterm \\
				Effektives Potential & \(\Phi(\vec{x}) = -\ln\left(\frac{\Tfield}{\Tfield_0}\right)\) & Gravitationsverknüpfung \\
				Gravitationskraft & \(\vec{F} = -\frac{\nabla \Tfield}{\Tfield}\) & Kraft aus Zeitfeld \\
				\bottomrule
			\end{tabular}
		\end{adjustbox}
		\caption{T0-Modell-Gleichungen, verknüpft mit Abschnitt \ref{sec:gravitation}}
		\label{tab:t0_equations}
	\end{table}
	
	\subsubsection{Modifizierte Quantenmechanik}
	\label{subsec:quantum}
	
	Das T0-Modell modifiziert die Quantenmechanik durch \(\Tfield\), wie in Tabelle \ref{tab:qm_equations} gezeigt. Die Schrödinger-Gleichung:
	\[
	i \Tfield \frac{\partial \Psi}{\partial t} + i \Psi \frac{\partial \Tfield}{\partial t} = \hat{H} \Psi
	\]
	führt massenabhängige Dynamiken ein, erklärt Dekohärenz und Nichtlokalität, wie in Abbildung \ref{fig:orbital_analogy} visualisiert \cite{pascher_quantum_2025}.
	
	\begin{table}[htbp]
		\centering
		\begin{adjustbox}{width=\tablescale\textwidth}
			\begin{tabular}{lll}
				\toprule
				\textbf{Gleichung} & \textbf{Natürliche Form} & \textbf{Standardform} \\
				\midrule
				Schrödinger-Gleichung & \(i \Tfield \frac{\partial \Psi}{\partial t} + i \Psi \frac{\partial \Tfield}{\partial t} = \hat{H} \Psi\) & \(i \hbar \frac{\partial \Psi}{\partial t} = \hat{H} \Psi\) \\
				Dekohärenzrate & \(\Gamma_{\text{dec}} = \Gamma_0 \cdot m\) & \(\Gamma_{\text{dec}} = \Gamma_0 \cdot \frac{m c^2}{\hbar}\) \\
				Welle-Teilchen & \(\lambda = \frac{1}{p}\) & \(\lambda = \frac{h}{p}\) \\
				Unschärfe & \(\Delta E \cdot \Delta t \geq \frac{1}{2}\) & \(\Delta E \cdot \Delta t \geq \frac{\hbar}{2}\) \\
				\bottomrule
			\end{tabular}
		\end{adjustbox}
		\caption{Modifizierte Quantengleichungen, verknüpft mit Abbildung \ref{fig:orbital_analogy}}
		\label{tab:qm_equations}
	\end{table}
	
	\subsection{Fundamentale Beziehungen}
	\label{subsec:relationships}
	
	Die Größen des T0-Modells bilden ein Netzwerk, wie in Abbildung \ref{fig:quantity_network} gezeigt, verknüpft mit Skalen in Abbildung \ref{fig:energy_hierarchy} \cite{pascher_grundkraefte_2025}.
	
	\begin{figure}[htbp]
		\centering
		\begin{tikzpicture}
			\small
			\node[draw, circle, fill=blue!20, minimum size=2cm] (energy) at (0,0) {Energie [E]};
			\node[draw, ellipse, fill=green!20] (mass) at (-3,2) {Masse [E]};
			\node[draw, ellipse, fill=green!20] (temp) at (3,2) {Temperatur [E]};
			\node[draw, ellipse, fill=orange!20] (length) at (-4,-2) {Länge [E\(^{-1}\)]};
			\node[draw, ellipse, fill=orange!20] (time) at (-1,-2) {Zeit [E\(^{-1}\)]};
			\node[draw, ellipse, fill=orange!20] (wavelength) at (2,-2) {Wellenlänge [E\(^{-1}\)]};
			\node[draw, ellipse, fill=orange!20] (frequency) at (5,0) {Frequenz [E]};
			\node[draw, rectangle, fill=red!20] (tfield) at (0,-4) {Zeitfeld \(\Tfield\) [E\(^{-1}\)]};
			\draw[->, thick] (energy) -- (mass) node[midway, above] {$m = E$};
			\draw[->, thick] (energy) -- (temp) node[midway, above] {$T = E$};
			\draw[->, thick] (energy) -- (length) node[midway, above left] {$L = E^{-1}$};
			\draw[->, thick] (energy) -- (time) node[midway, above left] {$t = E^{-1}$};
			\draw[->, thick] (energy) -- (wavelength) node[midway, above right] {$\lambda = E^{-1}$};
			\draw[->, thick] (energy) -- (frequency) node[midway, above] {$\nu = E$};
			\draw[->, thick, dashed] (length) -- (time) node[midway, below] {$c = 1$};
			\draw[->, thick, dashed] (wavelength) -- (frequency) node[midway, below right] {$\lambda \nu = 1$};
			\draw[->, thick, dashed] (mass) to[out=-45, in=135] node[midway, right] {$E = m c^2$} (energy);
			\draw[->, thick, dashed] (temp) to[out=-135, in=45] node[midway, left] {$E = k_B T$} (energy);
			\draw[->, thick, red] (tfield) -- (mass) node[midway, left] {$\Tfield = \frac{1}{m}$};
			\draw[->, thick, red] (tfield) -- (time) node[midway, right] {$\Tfield \sim t$};
			\draw[->, thick, red] (tfield) -- (wavelength) node[midway, right] {$\Tfield = \frac{1}{\omega}$};
		\end{tikzpicture}
		\caption{Netzwerk physikalischer Größen, verknüpft mit Tabelle \ref{tab:practical_notation}}
		\label{fig:quantity_network}
	\end{figure}
	
	\begin{figure}[htbp]
		\centering
		\begin{tikzpicture}
			\footnotesize
			\draw[thick, ->] (0,0) -- (12,0) node[right] {Energieskala};
			\draw[thick, ->] (0,0) -- (0,4) node[above] {Massenbereich};
			\draw[thick] (2,0.2) -- (2,-0.2) node[below] {Planck};
			\draw[thick] (5,0.2) -- (5,-0.2) node[below] {Elektroschwach};
			\draw[thick] (8,0.2) -- (8,-0.2) node[below] {QCD};
			\draw[thick] (11,0.2) -- (11,-0.2) node[below] {Atomar};
			\node[draw, fill=blue!10] at (2,2.5) {$M_P \sim \SI{1e19}{\giga\electronvolt}$};
			\node[draw, fill=blue!10] at (5,2) {$M_W, M_Z, M_H \sim \SI{1e2}{\giga\electronvolt}$};
			\node[draw, fill=blue!10] at (8,1.5) {$\Lambda_{QCD} \sim \SI{0.2}{\giga\electronvolt}$};
			\node[draw, fill=blue!10] at (11,1) {$m_e \sim \SI{0.0005}{\giga\electronvolt}$};
			\draw[thick, red, ->] (2,3.5) -- (5,2.5) node[midway, above] {$\xi = 1.33 \times 10^{-4}$};
			\draw[thick, red, ->] (5,2.5) -- (8,2) node[midway, above] {$\alphaEM = 1$};
			\draw[thick, red, ->] (8,2) -- (11,1.5) node[midway, above] {$\betaT = 1$};
			\node[draw, ellipse, fill=red!10] at (6,3.9) {T0-Modell: $\hbar = c = G = \alphaEM = \betaT = 1$};
		\end{tikzpicture}
		\caption{Hierarchie der Energieskalen, verknüpft mit Abschnitt \ref{subsec:higgs}}
		\label{fig:energy_hierarchy}
	\end{figure}
	
	\subsection{Fundamentale Kräfte}
	\label{subsec:forces}
	
	Das T0-Modell interpretiert Kräfte neu, wie in Tabelle \ref{tab:forces} gezeigt, mit Gravitation, die aus \(\Tfield\) entsteht (Abschnitt \ref{sec:gravitation}) \cite{pascher_grundkraefte_2025}.
	
	\begin{table}[htbp]
		\centering
		\begin{adjustbox}{width=\tablescale\textwidth}
			\begin{tabular}{llll}
				\toprule
				\textbf{Kraft} & \textbf{Dimensionslose Kopplung} & \textbf{Natürlicher Wert} & \textbf{Reichweite} \\
				\midrule
				Elektromagnetisch & \(\alphaEM\) & 1 & \(\infty\) \\
				Stark & \(\alpha_s\) & \(\sim 0.118\) bei \(Q^2 = M_Z^2\) & \(\sim \SI{1e-15}{\meter}\) \\
				Schwach & \(\alpha_W = g^2/(4\pi)\) & \(\sim 1/30\) & \(\sim \SI{1e-18}{\meter}\) \\
				Gravitation & \(\alpha_G = G m^2/\hbar c\) & \(m^2/m_P^2\) & \(\infty\) \\
				\bottomrule
			\end{tabular}
		\end{adjustbox}
		\caption{Fundamentale
			%----
Kräfte in natürlichen Einheiten, verknüpft mit Abschnitt \ref{subsec:maxwell}}
\label{tab:forces}
\end{table}

\subsection{Einheitenumrechnungen}
\label{sec:conversions}

Umrechnungen in SI-Einheiten sind präzise, wie in Tabelle \ref{tab:conversion} gezeigt, verknüpft mit Abbildung \ref{fig:practical_conversion} \cite{pascher_temp_2025}.

\begin{table}[htbp]
\centering
\begin{adjustbox}{width=\tablescale\textwidth}
\begin{tabular}{lcccc}
	\toprule
	\textbf{SI-Einheit} & \textbf{SI-Dimension} & \textbf{T0-Äquivalent} & \textbf{Umrechnung} & \textbf{Genauigkeit} \\
	\midrule
	Meter & \([L]\) & \([E^{-1}]\) & \(\SI{1}{\meter} \leftrightarrow (\SI{197}{\mega\electronvolt})^{-1}\) & \(< 0.001\%\) \\
	Sekunde & \([T]\) & \([E^{-1}]\) & \(\SI{1}{\second} \leftrightarrow (\SI{6.58e-22}{\mega\electronvolt})^{-1}\) & \(< 0.00001\%\) \\
	Kilogramm & \([M]\) & \([E]\) & \(\SI{1}{\kilogram} \leftrightarrow \SI{5.61e26}{\mega\electronvolt}\) & \(< 0.001\%\) \\
	Ampere & \([I]\) & \([E]\) & \(\SI{1}{\ampere} \leftrightarrow [E^2]\) & \(< 0.005\%\) \\
	Kelvin & \([\Theta]\) & \([E]\) & \(\SI{1}{\kelvin} \leftrightarrow \SI{8.62e-5}{\electronvolt}\) & \(< 0.01\%\) \\
	Volt & \([ML^2 T^{-3} I^{-1}]\) & \([E]\) & \(\SI{1}{\volt} \leftrightarrow \SI{1}{\electronvolt}/\sqrt{4\pi}\) & \(< 0.0001\%\) \\
	Coulomb & \([T I]\) & \([1]\) & \(\SI{1}{\coulomb} \leftrightarrow \sqrt{4\pi}/e\) & \(< 0.0001\%\) \\
	\bottomrule
\end{tabular}
\end{adjustbox}
\caption{Umrechnung von SI-Einheiten in T0-Einheiten, verknüpft mit Tabelle \ref{tab:planck_units}}
\label{tab:conversion}
\end{table}

\begin{figure}[htbp]
\centering
\begin{tikzpicture}
\footnotesize
\node[draw, rounded corners, fill=blue!10, minimum width=8cm, minimum height=1.5cm] (conversion) at (0,0) {
	\begin{tabular}{c|c|c}
		Länge & Zeit & Energie/Masse \\
		\hline
		\(\SI{1}{\per\giga\electronvolt} \approx \SI{0.197}{\femto\meter}\) & \(\SI{1}{\per\giga\electronvolt} \approx \SI{6.58e-25}{\second}\) & \(\SI{1}{\electronvolt} \approx \SI{1.78e-36}{\kilogram}\)
	\end{tabular}
};
\node[draw, rounded corners, fill=green!10, minimum width=8cm, minimum height=1.5cm] (practical) at (0,-2.5) {
	\begin{tabular}{c|c|c}
		Protonenmasse & Elektronenmasse & Temperatur \\
		\hline
		\(m_p \approx \SI{0.938}{\giga\electronvolt}\) & \(m_e \approx \SI{0.511}{\mega\electronvolt}\) & \(\SI{1}{\electronvolt} \approx \SI{11.605}{\kelvin}\)
	\end{tabular}
};
\node[above] at (conversion.north) {\textbf{Praktische Umrechnungen}};
\node[above] at (practical.north) {\textbf{Teilchenphysikalische Größen}};
\end{tikzpicture}
\caption{Umrechnungen in natürlichen Einheiten, verknüpft mit Abschnitt \ref{sec:conversions}}
\label{fig:practical_conversion}
\end{figure}

\subsection{Philosophische Implikationen}
\label{sec:philosophy}

Das energiebasierte Rahmenwerk des T0-Modells hat tiefgreifende Implikationen:
\begin{enumerate}
\item \textbf{Ontologische Vereinfachung}: Energie als einzige Entität vereinheitlicht alle Phänomene, im Einklang mit Einsteins Einsichten \cite{Einstein1905, pascher_dualismus_2025}.
\item \textbf{Vereinheitlichte Beschreibung}: Normierte Konstanten offenbaren ein einheitliches Rahmenwerk \cite{pascher_vereinheitlichung_2025}.
\item \textbf{Emergente Raumzeit}: Raumzeit entsteht aus \(\Tfield\), wie in Abbildung \ref{fig:quantity_network} gezeigt \cite{pascher_perspective_2025}.
\item \textbf{Geist-Körper-Problem}: Absolute Zeit bietet eine Grundlage für Bewusstsein \cite{pascher_perspective_2025}.
\end{enumerate}

\section{Zusammenfassung und Ausblick}
\label{sec:outlook}

Das T0-Modell vereinheitlicht die Physik durch:
\begin{enumerate}
\item Hierarchische Konstanten (Abschnitt \ref{sec:hierarchy}).
\item Quantisierte Skalen (Abschnitt \ref{subsec:quantization}).
\item Vereinfachte Gleichungen (Abschnitt \ref{sec:field_equations}).
\item Emergente Gravitation (Abschnitt \ref{sec:gravitation}).
\item Kosmologische Einsichten (Abschnitt \ref{subsec:t0_equations}) \cite{pascher_alphabeta_2025}.
\end{enumerate}

Zukünftige Richtungen umfassen:
\begin{itemize}
\item Testen von Rotverschiebungsvorhersagen.
\item Verifizierung von \(R_\infty = m_e/2\).
\item Quantisierung von \(\Tfield\).
\item Simulation von Galaxiendynamiken \cite{pascher_galaxies_2025}.
\end{itemize}

\bibliographystyle{apsrev4-2}
\begin{thebibliography}{99}
\bibitem{pascher_zeit_2025} J. Pascher, \href{https://github.com/jpascher/T0-Time-Mass-Duality/tree/main/2/pdf/Deutsch/ZeitEmergentQM.pdf}{Zeit als emergente Eigenschaft in der Quantenmechanik: Eine Verbindung zwischen Relativität, Feinstrukturkonstante und Quantendynamik}, 23. März 2025.
\bibitem{pascher_messdifferenzen_2025} J. Pascher, \href{https://github.com/jpascher/T0-Time-Mass-Duality/tree/main/2/pdf/Deutsch/MessdifferenzenT0Standard.pdf}{Kompensatorische und additive Effekte: Eine Analyse von Messdifferenzen zwischen dem T0-Modell und dem $\Lambda$CDM-Standardmodell}, 2. April 2025.
\bibitem{pascher_galaxies_2025} J. Pascher, \href{https://github.com/jpascher/T0-Time-Mass-Duality/tree/main/2/pdf/Deutsch/MassVarGalaxien.pdf}{Massenvariation in Galaxien: Eine Analyse im T0-Modell mit emergenter Gravitation}, 30. März 2025.
\bibitem{pascher_params_2025} J. Pascher, \href{https://github.com/jpascher/T0-Time-Mass-Duality/tree/main/2/pdf/Deutsch/ZeitMasseT0Params.pdf}{Zeit-Masse-Dualitätstheorie (T0-Modell): Ableitung der Parameter $\kappa$, $\alpha$ und $\beta$}, 30. März 2025.
\bibitem{pascher_temp_2025} J. Pascher, \href{https://github.com/jpascher/T0-Time-Mass-Duality/tree/main/2/pdf/Deutsch/NatEinheitenAlpha1.pdf}{Anpassung der Temperatureinheiten in natürlichen Einheiten und CMB-Messungen}, 2. April 2025.
\bibitem{pascher_alpha_2025} J. Pascher, \href{https://github.com/jpascher/T0-Time-Mass-Duality/tree/main/2/pdf/Deutsch/NatEinheitenAlpha1.pdf}{Energie als fundamentale Einheit: Natürliche Einheiten mit $\alphaEM = 1$ im T0-Modell}, 26. März 2025.
\bibitem{pascher_beta_2025} J. Pascher, \href{https://github.com/jpascher/T0-Time-Mass-Duality/tree/main/2/pdf/Deutsch/Alpha1Beta1Konsistenz.pdf}{Dimensionslose Parameter im T0-Modell: Setzen von $\beta = 1$ in natürlichen Einheiten}, 4. April 2025.
\bibitem{pascher_higgs_2025} J. Pascher, \href{https://github.com/jpascher/T0-Time-Mass-Duality/tree/main/2/pdf/Deutsch/MathHiggsZeitMasse.pdf}{Mathematische Formulierung des Higgs-Mechanismus in der Zeit-Masse-Dualität}, 28. März 2025.
\bibitem{pascher_lagrange_2025} J. Pascher, \href{https://github.com/jpascher/T0-Time-Mass-Duality/tree/main/2/pdf/Deutsch/MathZeitMasseLagrange.pdf}{Von Zeitdilatation zur Massenvariation: Mathematische Kernformulierungen der Zeit-Masse-Dualitätstheorie}, 29. März 2025.
\bibitem{pascher_emergente_2025} J. Pascher, \href{https://github.com/jpascher/T0-Time-Mass-Duality/tree/main/2/pdf/Deutsch/EmergentGravT0.pdf}{Emergente Gravitation im T0-Modell: Eine umfassende Ableitung}, 1. April 2025.
\bibitem{pascher_perspective_2025} J. Pascher, \href{https://github.com/jpascher/T0-Time-Mass-Duality/tree/main/2/pdf/Deutsch/ZeitRaumPascher.pdf}{Eine neue Perspektive auf Zeit und Raum: Johann Paschers revolutionäre Ideen}, 25. März 2025.
\bibitem{pascher_dualismus_2025} J. Pascher, \href{https://github.com/jpascher/T0-Time-Mass-Duality/tree/main/2/pdf/Deutsch/KurzKomplementDualPhysik.pdf}{Kurz gefasst - Komplementäre Dualität in der Physik: Vom Welle-Teilchen- zum Zeit-Masse-Konzept}, 26. März 2025.
\bibitem{pascher_grundkraefte_2025} J. Pascher, \href{https://github.com/jpascher/T0-Time-Mass-Duality/tree/main/2/pdf/Deutsch/VierKraefteZeitMasse.pdf}{Vereinfachte Beschreibung der fundamentalen Kräfte mit Zeit-Masse-Dualität}, 27. März 2025.
\bibitem{pascher_zeit_masse_2025} J. Pascher, \href{https://github.com/jpascher/T0-Time-Mass-Duality/tree/main/2/pdf/Deutsch/ZeitMasseNeuerBlick.pdf}{Zeit und Masse: Ein neuer Blick auf alte Formeln – und Befreiung von traditionellen Zwängen}, 22. März 2025.
\bibitem{pascher_quantum_2025} J. Pascher, \href{https://github.com/jpascher/T0-Time-Mass-Duality/tree/main/2/pdf/Deutsch/NotwendigkeitQMErweiterung.pdf}{Die Notwendigkeit einer Erweiterung der Standardquantenmechanik und Quantenfeldtheorie}, 27. März 2025.
\bibitem{pascher_photons_2025} J. Pascher, \href{https://github.com/jpascher/T0-Time-Mass-Duality/tree/main/2/pdf/Deutsch/DynMassePhotonenNichtlokal.pdf}{Dynamische Masse von Photonen und ihre Implikationen für Nichtlokalität im T0-Modell}, 25. März 2025.
\bibitem{pascher_alphabeta_2025} J. Pascher, \href{https://github.com/jpascher/T0-Time-Mass-Duality/tree/main/2/pdf/Deutsch/Alpha1Beta1Konsistenz.pdf}{Vereinheitlichtes Einheitensystem im T0-Modell: Die Konsistenz von $\alpha = 1$ und $\beta = 1$}, 5. April 2025.
\bibitem{pascher_planck_2025} J. Pascher, \href{https://github.com/jpascher/T0-Time-Mass-Duality/tree/main/2/pdf/Deutsch/JenseitsPlanck.pdf}{Reale Konsequenzen der Reformulierung von Zeit und Masse in der Physik: Jenseits der Planck-Skala}, 24. März 2025.
\bibitem{pascher_energiedynamik_2025} J. Pascher, \href{https://github.com/jpascher/T0-Time-Mass-Duality/tree/main/2/pdf/Deutsch/MathEnergiedynamik.pdf}{Dunkle Energie im T0-Modell: Eine mathematische Analyse der Energiedynamik}, 3. April 2025.
\bibitem{pascher_vereinheitlichung_2025} J. Pascher, \href{https://github.com/jpascher/T0-Time-Mass-Duality/tree/main/2/pdf/Deutsch/T0VereinheitlichungDEGal.pdf}{Vereinheitlichung des T0-Modells: Grundlagen, dunkle Energie und Galaxiendynamik}, 4. April 2025.
\bibitem{pascher_formalismen_2025} J. Pascher, \href{https://github.com/jpascher/T0-Time-Mass-Duality/tree/main/2/pdf/Deutsch/MathZeitMasseLagrange.pdf}{Von Zeitdilatation zur Massenvariation: Mathematische Kernformulierungen der Zeit-Masse-Dualitätstheorie}, 5. April 2025.
\bibitem{Planck1899} M. Planck, \textit{Über irreversible Strahlungsprozesse}, Sitzungsberichte der Königlich Preußischen Akademie der Wissenschaften zu Berlin 5, 440–480 (1899).
\bibitem{Dirac1928} P. A. M. Dirac, \textit{Die Quantentheorie des Elektrons}, Proceedings of the Royal Society of London A 117, 610–624 (1928).
\bibitem{Einstein1905} A. Einstein, \textit{Zur Elektrodynamik bewegter Körper}, Annalen der Physik 322, 891–921 (1905).
\bibitem{Einstein1915} A. Einstein, \textit{Die Feldgleichungen der Gravitation}, Sitzungsberichte der Königlich Preußischen Akademie der Wissenschaften, 844–847 (1915).
\bibitem{Sommerfeld1916} A. Sommerfeld, \textit{Zur Quantentheorie der Spektrallinien}, Annalen der Physik 356, 1–94 (1916).
\bibitem{Heisenberg1927} W. Heisenberg, \textit{Über den anschaulichen Inhalt der quantentheoretischen Kinematik und Mechanik}, Zeitschrift für Physik 43, 172–198 (1927).
\bibitem{Schrodinger1926} E. Schrödinger, \textit{Quantisierung als Eigenwertproblem}, Annalen der Physik 384, 361–376 (1926).
\bibitem{Feynman1985} R. P. Feynman, \textit{QED: Die seltsame Theorie des Lichts und der Materie}, Princeton University Press (1985).
\bibitem{Duff2002} M. J. Duff, L. B. Okun \& G. Veneziano, \textit{Trialog über die Anzahl fundamentaler Konstanten}, Journal of High Energy Physics 3, 023 (2002).
\bibitem{Wilczek2008} F. Wilczek, \textit{Die Leichtigkeit des Seins: Masse, Äther und die Vereinheitlichung der Kräfte}, Basic Books (2008).
\bibitem{Verlinde2011} E. Verlinde, \textit{Über den Ursprung der Gravitation und die Gesetze Newtons}, Journal of High Energy Physics 4, 29 (2011).
\bibitem{Greene2020} B. Greene, \textit{Bis zum Ende der Zeit: Geist, Materie und unsere Suche nach Sinn in einem sich entwickelnden Universum}, Alfred A. Knopf (2020).
\bibitem{tHooft1993} G. 't Hooft, \textit{Dimensionale Reduktion in der Quantengravitation}, arXiv:gr-qc/9310026 (1993).
\bibitem{Will2014} C. M. Will, \textit{Die Konfrontation zwischen allgemeiner Relativitätstheorie und Experiment}, Living Reviews in Relativity 17, 4 (2014).
\end{thebibliography}

\end{document}			