\documentclass[12pt,a4paper]{article}
\usepackage[utf8]{inputenc}
\usepackage[T1]{fontenc}
\usepackage[ngerman]{babel}
\usepackage{lmodern}
\usepackage{amsmath}
\usepackage{amssymb}
\usepackage{physics}
\usepackage{hyperref}
\usepackage{tcolorbox}
\usepackage{booktabs}
\usepackage{enumitem}
\usepackage[table,xcdraw]{xcolor}
\usepackage[left=2cm,right=2cm,top=2cm,bottom=2cm]{geometry}
\usepackage{pgfplots}
\pgfplotsset{compat=1.18}
\usepackage{graphicx}
\usepackage{float}
\usepackage{fancyhdr}
\usepackage{siunitx}
\usepackage{tikz}
\usepackage{adjustbox}
\usetikzlibrary{shapes.geometric}

% Custom Commands
\newcommand{\Tfield}{T(x)}
\newcommand{\alphaEM}{\alpha_{\text{EM}}}
\newcommand{\betaT}{\beta_{\text{T}}}
\newcommand{\Mpl}{M_{\text{Pl}}}
\newcommand{\Tzerot}{T_0(\Tfield)}
\newcommand{\e}{\mathrm{e}}
\newcommand{\alphaEMSI}{\alpha_{\text{EM,SI}}}

% Header and Footer Configuration
\pagestyle{fancy}
\fancyhf{}
\fancyhead[L]{Johann Pascher}
\fancyhead[R]{Systematische Zusammenstellung Natürlicher Einheiten}
\fancyfoot[C]{\thepage}
\renewcommand{\headrulewidth}{0.4pt}
\renewcommand{\footrulewidth}{0.4pt}

\hypersetup{
	colorlinks=true,
	linkcolor=blue,
	citecolor=blue,
	urlcolor=blue,
	pdftitle={Systematische Zusammenstellung Natürlicher Einheiten mit Energie als Grundeinheit},
	pdfauthor={Johann Pascher},
	pdfsubject={Theoretische Physik},
	pdfkeywords={T0 Modell, natürliche Einheiten, Feinstrukturkonstante, vereinheitlichtes Einheitensystem, Zeit-Masse-Dualität}
}

\title{Systematische Zusammenstellung Natürlicher Einheiten \\mit Energie als Grundeinheit}
\author{Johann Pascher}
\date{11. April 2025}

\begin{document}
	
	\maketitle
	
	\begin{abstract}
		Diese Arbeit präsentiert eine umfassende systematische Zusammenstellung natürlicher Einheiten im Rahmen des T0-Modells der Zeit-Masse-Dualität. Ausgehend von Energie als fundamentaler Einheit wird eine hierarchische Struktur physikalischer Konstanten entwickelt, wobei alle grundlegenden Konstanten ($\hbar = c = G = k_B = \alphaEM = \alpha_W = \betaT = 1$) auf 1 gesetzt werden. Die daraus abgeleiteten Konstanten und Skalen werden in einem kohärenten Rahmen dargestellt, der Quanten- und relativistische Phänomene vereinheitlicht. Besondere Aufmerksamkeit gilt der Hierarchie von Längenskalen vom subplanckischen bis zum kosmologischen Bereich sowie den Beziehungen zwischen elektromagnetischen, thermodynamischen und quantenmechanischen Konstanten, die alle aus der fundamentalen Energieskala abgeleitet werden.
	\end{abstract}
	
	\tableofcontents
	\newpage
	
	\section{Einleitung}
	
	Die natürlichen Einheiten in der theoretischen Physik erlauben eine fundamentale Vereinfachung und Vereinheitlichung physikalischer Gesetze, indem sie die Anzahl unabhängiger Dimensionen auf ein Minimum reduzieren und fundamentale Naturkonstanten auf den Wert 1 setzen. Während traditionelle natürliche Einheitensysteme wie die Planck-Einheiten ($\hbar = c = G = 1$) seit langem etabliert sind, geht das T0-Modell der Zeit-Masse-Dualität einen Schritt weiter und schlägt ein vollständig vereinheitlichtes natürliches Einheitensystem vor, in dem zusätzlich dimensionslose Kopplungskonstanten wie die Feinstrukturkonstante $\alpha_{\text{EM}}$, die Wien'sche Konstante $\alpha_W$ und der modellspezifische Parameter $\beta_T$ auf 1 gesetzt werden.
	
	Diese Arbeit präsentiert eine systematische Zusammenstellung dieses vereinheitlichten Einheitensystems mit Energie als fundamentaler Grundeinheit. Dabei werden nicht nur die Definitionen und Werte der natürlichen Einheiten präsentiert, sondern auch die hierarchischen Beziehungen zwischen verschiedenen physikalischen Größen, Längenskalen und Konstanten aufgezeigt. Besonderes Augenmerk liegt auf der theoretischen Fundierung der Setzung $\alpha_{\text{EM}} = \beta_T = 1$ und deren Implikationen für die Vereinheitlichung der Physik.
	
	Das T0-Modell geht von der Dualität von Zeit und Masse aus, wobei die Zeit als absolut und die Masse als variabel angenommen wird – im Gegensatz zu den üblichen Annahmen der Relativitätstheorie. Diese konzeptionelle Umkehrung wird durch ein intrinsisches Zeitfeld $\Tfield$ vermittelt, das Quantenmechanik und Relativitätstheorie in einem kohärenten Rahmen verbindet. Das vereinheitlichte natürliche Einheitensystem ist dabei nicht nur eine mathematische Vereinfachung, sondern ein theoretisches Erfordernis des Modells, das eine tiefere Einheit der Naturgesetze reflektiert.
	
	Die vorliegende Zusammenstellung umfasst:
	\begin{itemize}
		\item Die hierarchische Struktur fundamentaler Konstanten und ihre Werte im vereinheitlichten System
		\item Die theoretische Herleitung und Fundierung der Setzungen $\alpha_{\text{EM}} = 1$ und $\beta_T = 1$
		\item Die Charakterisierung physikalischer Längenskalen von subplanckisch bis kosmologisch
		\item Umrechnungsformeln zwischen natürlichen und SI-Einheiten
		\item Die vereinfachten Feldgleichungen in natürlichen Einheiten
		\item Philosophische Implikationen und Ausblick auf experimentelle Tests
	\end{itemize}
	
	Diese systematische Zusammenstellung natürlicher Einheiten mit Energie als Grundeinheit bildet eine solide theoretische Basis für das T0-Modell und könnte den Weg zu einer umfassenderen Vereinheitlichung der Physik ebnen.
	
	\section{Hierarchie der Natürlichen Einheiten}
	
	Die natürlichen Einheiten im T0-Modell bilden eine klare hierarchische Struktur, die in drei Ebenen organisiert werden kann:
	
	\subsection{Dreistufige Hierarchie der Konstanten}
	
	Die Hierarchie lässt sich in drei grundlegende Ebenen gliedern:
	
	\begin{tcolorbox}[colback=blue!5!white,colframe=blue!75!black,title=Hierarchische Ebenen der Konstanten]
		\textbf{Ebene 1: Primäre Dimensionskonstanten}
		\begin{itemize}
			\item \textbf{Planck-Konstante $\hbar = 1$}: Definiert den Quantenmaßstab
			\item \textbf{Lichtgeschwindigkeit $c = 1$}: Definiert den relativistischen Maßstab
			\item \textbf{Gravitationskonstante $G = 1$}: Definiert den Gravitationsmaßstab
			\item \textbf{Boltzmann-Konstante $k_B = 1$}: Definiert den thermodynamischen Maßstab
		\end{itemize}
		
		\textbf{Ebene 2: Dimensionslose Kopplungskonstanten}
		\begin{itemize}
			\item \textbf{Feinstrukturkonstante $\alphaEM = 1$}: Elektromagnetische Wechselwirkungsstärke
			\item \textbf{Wien'sche Konstante $\alpha_W = 1$}: Thermische Strahlungscharakteristik
			\item \textbf{T0-Parameter $\betaT = 1$}: Kopplungsstärke des intrinsischen Zeitfeldes
		\end{itemize}
		
		\textbf{Ebene 3: Abgeleitete Verhältnisse}
		\begin{itemize}
			\item \textbf{$\xi = r_0/l_P = 1,33 \times 10^{-4}$}: Verhältnis der T0-Länge zur Planck-Länge
			\item \textbf{$L_T/l_P = 3,9 \times 10^{62}$}: Verhältnis der kosmologischen Korrelationslänge zur Planck-Länge
			\item \textbf{$r_0/L_T = 3,41 \times 10^{-67}$}: Mikro-zu-Makro-Skalenverhältnis
		\end{itemize}
	\end{tcolorbox}
	
	\subsection{Fundamentale Konzepte des T0-Modells}
	
	Das T0-Modell basiert auf der Dualität von Zeit und Masse, wobei die Zeit als absolut und die Masse als variabel angenommen wird. Dies steht im Gegensatz zu den üblichen Annahmen der Relativitätstheorie (relative Zeit, konstante Masse) und der Quantenmechanik (Parameter-Zeit). Diese konzeptionelle Umkehrung wird durch ein intrinsisches Zeitfeld $\Tfield$ vermittelt, das als skalares Feld definiert ist:
	
	\begin{equation}
		\Tfield = \frac{\hbar}{\max(mc^2, \omega)}
	\end{equation}
	
	Die Einführung eines vereinheitlichten natürlichen Einheitensystems, in dem alle fundamentalen Konstanten auf 1 gesetzt werden, ist keine arbiträre mathematische Vereinfachung, sondern ein theoretisches Erfordernis des Modells, das eine tiefere Einheit der Naturgesetze reflektiert \cite{pascher_alphabeta_2025}.
	
	\subsection{Fundamentale Konstanten mit Wert 1}
	
	Im T0-Modell werden die folgenden Konstanten auf 1 gesetzt, basierend auf theoretischer Notwendigkeit:
	
	\begin{table}[h]
		\centering
		\begin{adjustbox}{scale=0.75}
			\begin{tabular}{lllll}
				\hline
				\textbf{Konstante} & \textbf{Symbol} & \textbf{SI-Wert} & \textbf{Natürlicher Wert} & \textbf{Hierarchieebene} \\
				\hline
				Reduzierte Planck-Konstante & $\hbar$ & $1,055 \times 10^{-34}$ J$\cdot$s & 1 & Primär - Ebene 1 \\
				Lichtgeschwindigkeit & $c$ & $3 \times 10^8$ m/s & 1 & Primär - Ebene 1 \\
				Gravitationskonstante & $G$ & $6,674 \times 10^{-11}$ m$^3$kg$^{-1}$s$^{-2}$ & 1 & Primär - Ebene 1 \\
				Boltzmann-Konstante & $k_B$ & $1,381 \times 10^{-23}$ J/K & 1 & Primär - Ebene 1 \\
				Feinstrukturkonstante & $\alphaEM$ & 1/137,036 & 1 & Sekundär - Ebene 2 \\
				Wien'sche Konstante & $\alpha_W$ & 2,82 & 1 & Sekundär - Ebene 2 \\
				T0-Parameter & $\betaT$ & 0,008 (SI) & 1 & Sekundär - Ebene 2 \\
				\hline
				\multicolumn{4}{c}{} \\
				\hline
			\end{tabular}
		\end{adjustbox}
		\caption{Fundamentale Konstanten im T0-Modell}
		\label{tab:fund_const}
	\end{table}
	
	\subsection{Abgeleitete Elektromagnetische Konstanten}
	
	Mit den primären und sekundären Konstanten (insbesondere $c = 1$ und $\alphaEM = 1$) werden die elektromagnetischen Feldkonstanten auf natürliche Weise normiert:
	
	\begin{table}[h]
		\centering
		\begin{adjustbox}{scale=0.7}
			\begin{tabular}{llllll}
				\hline
				\textbf{Konstante} & \textbf{Symbol} & \textbf{SI-Wert} & \textbf{Natürlicher Wert} & \textbf{Herleitung} & \textbf{Hierarchieebene} \\
				\hline
				Vakuumpermeabilität & $\mu_0$ & $4\pi \times 10^{-7}$ H/m & 1 & $\mu_0 = 1/\varepsilon_0c^2 = 1$ & Abgeleitet - Ebene 2.5 \\
				Vakuumpermittivität & $\varepsilon_0$ & $8,85 \times 10^{-12}$ F/m & 1 & $\varepsilon_0 = 1/\mu_0c^2 = 1$ & Abgeleitet - Ebene 2.5 \\
				Vakuumimpedanz & $Z_0$ & 376,73 $\Omega$ & 1 & $Z_0 = \sqrt{\mu_0/\varepsilon_0} = 1$ & Abgeleitet - Ebene 2.5 \\
				Elementarladung & $e$ & $1,602 \times 10^{-19}$ C & 1 & $e = \sqrt{4\pi\varepsilon_0\hbar c} = 1$ & Abgeleitet - Ebene 2.5 \\
				\hline
				\multicolumn{5}{c}{} \\
				\hline
			\end{tabular}
		\end{adjustbox}
		\caption{Abgeleitete elektromagnetische Konstanten }
		\label{tab:em_const}
	\end{table}
	
	Die Beziehungen zwischen diesen Konstanten sind:
	\begin{itemize}
		\item $\mu_0\varepsilon_0 = 1/c^2 = 1$ (mit $c = 1$)
		\item $Z_0 = \sqrt{\mu_0/\varepsilon_0} = 1$ (mit $\mu_0 = \varepsilon_0 = 1$)
		\item $e^2 = 4\pi\varepsilon_0\hbar c$ (mit $\alphaEM = 1$)
		\item $e = 1$ (mit $\varepsilon_0 = \hbar = c = 1$ und $\alphaEM = 1$)
	\end{itemize}
	
	Diese Normierung der elektromagnetischen Konstanten zeigt, dass elektrische und magnetische Feldstärken in denselben Einheiten gemessen werden können und die Elementarladung dimensionslos wird, was die elektromagnetische Wechselwirkung fundamental vereinfacht \cite{pascher_alpha_2025}.
	
	\subsection{Weitere abgeleitete Konstanten mit Wert 1}
	
	Im vereinheitlichten natürlichen Einheitensystem des T0-Modells können weitere wichtige Konstanten abgeleitet werden, die ebenfalls den natürlichen Wert 1 annehmen oder sich auf einfache Werte reduzieren:
	
	\begin{table}[ht]
		\centering
		\begin{adjustbox}{scale=0.65}
			\begin{tabular}{llllll}
				\hline
				\textbf{Konstante} & \textbf{Symbol} & \textbf{SI-Wert} & \textbf{Natürlicher Wert} & \textbf{Herleitung} & \textbf{Hierarchieebene} \\
				\hline
				Compton-Wellenlänge & $\lambda_{C,e}$ & $2,43 \times 10^{-12}$ m & $1/m_e$ & $\hbar/(m_e\cdot c) = 1/m_e$ & Abgeleitet - Ebene 3 \\
				des Elektrons & & & & & \\
				Rydberg-Konstante & $R_\infty$ & $1,097 \times 10^7$ m$^{-1}$ & $\alphaEM^2\cdot m_e/2 = 1/2$ & $m_e\cdot e^4/(8\varepsilon_0^2h^3c) = 1/2$ & Abgeleitet - Ebene 3 \\
				Josephson-Konstante & $K_J$ & $4,84 \times 10^{14}$ Hz/V & $2e/h = 1/\pi$ & $2e/h = 1/\pi$ & Abgeleitet - Ebene 3 \\
				Von Klitzing-Konstante & $R_K$ & $2,58 \times 10^4$ $\Omega$ & $h/e^2 = 2\pi$ & $h/e^2 = 2\pi$ & Abgeleitet - Ebene 3 \\
				Schwinger-Grenze & $E_S$ & $1,32 \times 10^{18}$ V/m & $m_e^2c^3/e\hbar = m_e^2$ & $m_e^2c^3/e\hbar = m_e^2$ & Abgeleitet - Ebene 3 \\
				Stefan-Boltzmann- & $\sigma$ & $5,67 \times 10^{-8}$ W/(m$^2$K$^4$) & $\pi^2k_B^4/(60\hbar^3c^2) = \pi^2/60$ & $\pi^2k_B^4/(60\hbar^3c^2) = \pi^2/60$ & Abgeleitet - Ebene 3 \\
				Konstante & & & & & \\
				Planck-Druck & $p_P$ & $4,63 \times 10^{113}$ Pa & $c^7/(\hbar G^2) = 1$ & $c^7/(\hbar G^2) = 1$ & Abgeleitet - Ebene 2.5 \\
				Planck-Kraft & $F_P$ & $1,21 \times 10^{44}$ N & $c^4/G = 1$ & $c^4/G = 1$ & Abgeleitet - Ebene 2.5 \\
				Hawking-Temperatur & $T_H$ & $\hbar c^3/(8\pi GMk_B)$ & $1/(8\pi M)$ & $1/(8\pi M)$ & Abgeleitet - Ebene 3 \\
				Bekenstein-Hawking- & $S_{BH}$ & $4\pi GM^2/\hbar c$ & $4\pi M^2$ & $4\pi M^2$ & Abgeleitet - Ebene 3 \\
				Entropie & & & & & \\
				Einstein-Hilbert- & $S_{EH}$ & $c^3/(16\pi G)\int R\sqrt{-g} d^4x$ & $(1/16\pi)\int R\sqrt{-g} d^4x$ & $(1/16\pi)\int R\sqrt{-g} d^4x$ & Abgeleitet - Ebene 2.5 \\
				Wirkung & & & & & \\
				\hline
				\multicolumn{5}{c}{} \\
				\hline
			\end{tabular}
		\end{adjustbox}
		\caption{Weitere abgeleitete Konstanten im T0-Modell}
		\label{tab:derived_const}
	\end{table}
	
	Besonders interessant sind:
	
	\begin{enumerate}
		\item \textbf{Quantenmechanische Konstanten}:
		\begin{itemize}
			\item Die Compton-Wellenlänge wird direkt proportional zur inversen Masse
			\item Die Rydberg-Konstante wird $1/2$, was atomare Energieniveaus vereinfacht
		\end{itemize}
		
		\item \textbf{Quantenmetrologie-Konstanten}:
		\begin{itemize}
			\item Die Josephson-Konstante und von Klitzing-Konstante nehmen einfache Werte an ($1/\pi$ bzw. $2\pi$)
			\item Dies vereinfacht die Definition von elektrischen Maßeinheiten
		\end{itemize}
		
		\item \textbf{Thermodynamische Konstanten}:
		\begin{itemize}
			\item Die Stefan-Boltzmann-Konstante wird $\pi^2/60$, was Strahlungsberechnungen vereinfacht
			\item Die Verbindung zwischen Wärmestrahlung und Quantenphysik wird unmittelbar ersichtlich
		\end{itemize}
		
		\item \textbf{Relativistische Konstanten}:
		\begin{itemize}
			\item Planck-Druck und Planck-Kraft sind 1, vereinheitlichen mechanische Größen
			\item Hawking-Temperatur und Bekenstein-Hawking-Entropie nehmen einfache Formen an
		\end{itemize}
	\end{enumerate}
	
	Dies zeigt die tiefe Vereinheitlichung, die durch das T0-Modell mit Energie als Grundeinheit erreicht wird, wobei viele Naturkonstanten auf 1 oder einfache mathematische Ausdrücke reduziert werden \cite{pascher_params_2025}.
	
	\subsection{Herleitung von $\betaT = 1$}
	
	Die theoretische Konsistenz von $\betaT = 1$ im natürlichen Einheitensystem folgt aus der Definition und den Eigenschaften des T0-Modells:
	
	\begin{enumerate}
		\item \textbf{Definition von $\betaT$}: Im natürlichen Einheitensystem ($\hbar = c = G = 1$) wird $\betaT$ definiert als:
		\begin{equation}
			\betaT = \frac{\lambda_h^2 v^2}{16\pi^3} \cdot \frac{1}{m_h^2} \cdot \frac{1}{\xi}
		\end{equation}
		wobei:
		\begin{itemize}
			\item $\lambda_h \approx 0,13$ (Higgs-Selbstkopplung)
			\item $v \approx 246$ GeV (Higgs-Vakuumerwartungswert)
			\item $m_h \approx 125$ GeV (Higgs-Masse)
			\item $\xi = r_0/l_P$ (Verhältnis der charakteristischen T0-Länge zur Planck-Länge)
		\end{itemize}
		
		\item \textbf{Setzung $\betaT^{nat} = 1$}: Diese Bedingung führt zu:
		\begin{equation}
			\xi = \frac{\lambda_h^2 v^2}{16\pi^3 m_h^2} \approx 1,33 \times 10^{-4}
		\end{equation}
		Dies impliziert $r_0 \approx 1,33 \times 10^{-4} \cdot l_P$, etwa 1/7519 der Planck-Länge.
		
		\item \textbf{Konsistenz mit der Standardmodell-Relation}: Mit $m_h^2 = 2\lambda_h v^2$ erhält man:
		\begin{equation}
			\xi = \frac{\lambda_h}{32\pi^3} \approx \frac{0,13}{32\pi^3} \approx \frac{0,13}{990} \approx 1,31 \times 10^{-4}
		\end{equation}
		was nahezu identisch mit dem vorherigen Wert ist und die Robustheit der Beziehung bestätigt \cite{pascher_alphabeta_2025}.
		
		\item \textbf{Renormierungsgruppeninterpretation}: $\betaT$ kann als Renormierungsgruppen-Fixpunkt im Infrarot-Limit interpretiert werden:
		\begin{equation}
			\lim_{E \to 0} \betaT(E) = 1
		\end{equation}
		wobei der empirische Wert $\betaT^{SI} \approx 0,008$ als Resultat der Renormierungsgruppen-Evolution bei endlichen Energien zu verstehen ist \cite{pascher_alphabeta_2025}.
	\end{enumerate}
	
	Der Parameter $\betaT = 1$ ist somit theoretisch fundiert und kein empirisch angepasster Wert.
	
	\subsection{Herleitung von $\alphaEM = 1$}
	
	Die Setzung der Feinstrukturkonstante $\alphaEM = 1$ hat tiefgreifende Bedeutung im T0-Modell und ist theoretisch fundiert:
	
	\begin{enumerate}
		\item \textbf{Definition der Feinstrukturkonstante}:
		\begin{equation}
			\alphaEM = \frac{e^2}{4\pi\varepsilon_0\hbar c} \approx \frac{1}{137,036}
		\end{equation}
		
		\item \textbf{Implikation von $\alphaEM = 1$}:
		\begin{equation}
			e = \sqrt{4\pi\varepsilon_0\hbar c}
		\end{equation}
		Dies bedeutet, dass die Elementarladung eine dimensionslose Größe wird, die durch fundamentale Konstanten definiert ist.
		
		\item \textbf{Mit $\hbar = c = 1$ vereinfacht sich dies zu}:
		\begin{equation}
			e = \sqrt{4\pi\varepsilon_0}
		\end{equation}
		
		\item \textbf{Physikalische Konsequenz}: 
		Elektrische Ladungen werden dimensionslos, und alle elektromagnetischen Größen können auf Energie reduziert werden \cite{pascher_alpha_2025}.
		
		\item \textbf{Alternative Herleitung über den klassischen Elektronenradius}:
		Der klassische Elektronenradius $r_e = e^2/(4\pi\varepsilon_0m_e c^2)$ und die Compton-Wellenlänge $\lambda_C = h/(m_e c)$ stehen in Beziehung durch:
		\begin{equation}
			\alphaEM = \frac{2\pi r_e}{\lambda_C}
		\end{equation}
		was zur Standarddefinition führt, wenn man $h = 2\pi\hbar$ einsetzt \cite{pascher_alpha_2025}.
		
		\item \textbf{Verbindung mit EM-Konstanten}:
		Die Setzung $\alphaEM = 1$ koppelt die elektromagnetischen Konstanten $\mu_0$ und $\varepsilon_0$ durch $\mu_0\varepsilon_0 = 1/c^2 = 1$ (in natürlichen Einheiten).
	\end{enumerate}
	
	Die Setzung $\alphaEM = 1$ ist somit Teil des konzeptionellen Rahmens, der alle Interaktionen auf energiebasierte Terme reduziert und die intrinsische Einheit der Naturgesetze offenbart \cite{pascher_alphabeta_2025}.
	
	\section{Charakteristische Längenskalen in Natürlichen Einheiten}
	
	\begin{figure}[ht]
		\centering
		\begin{tikzpicture}
			\small % Kleinere Schriftgröße für das gesamte Diagramm
			\draw[thick,->] (0,0) -- (12,0) node[right] {$\log(L/l_P)$};
			\draw[thick,->] (0,0) -- (0,5) node[above] {Energieskala};
			
			% Markierungen auf der Längenskala
			\draw (0,0.2) -- (0,-0.2) node[below] {$0$};
			\draw (1,0.2) -- (1,-0.2) node[below] {$r_0$};
			\draw (5,0.2) -- (5,-0.2) node[below] {$\lambda_{C,e}$};
			\draw (6,0.2) -- (6,-0.2) node[below] {$a_0$};
			\draw (11,0.2) -- (11,-0.2) node[below] {$L_T$};
			
			% Beschriftungen
			\node[blue] at (0,1.2) {$l_P = 1$};
			\node[blue] at (1,1.7) {$r_0 \approx 10^{-4}$};
			\node[blue] at (5,1.7) {$\lambda_{C,e} \approx 10^{-23}$};
			\node[blue] at (6,1.2) {$a_0 \approx 10^{-23}/\alphaEM$};
			\node[blue] at (11,1.2) {$L_T \approx 10^{62}$};
			
			% Hierarchie-Ebenen
			\draw [decorate,decoration={brace,amplitude=10pt,mirror},xshift=0pt,yshift=-20pt]
			(0,0) -- (1,0) node [black,midway,yshift=-20pt] {Primordiale Skala};
			
			\draw [decorate,decoration={brace,amplitude=10pt,mirror},xshift=0pt,yshift=-20pt]
			(4.8,0) -- (6.2,0) node [black,midway,yshift=-20pt] {Quantenmechanische Skala};
			
			\draw [decorate,decoration={brace,amplitude=10pt,mirror},xshift=0pt,yshift=-20pt]
			(10.8,0) -- (12,0) node [black,midway,yshift=-20pt] {Kosmologische Skala};
			
			% Verbindungslinien zwischen den Skalen
			\draw[dashed, red] (1,0.5) -- (5,0.5) node[midway, above] {$\propto m_h/m_e$};
			\draw[dashed, red] (6,0.5) -- (11,0.5) node[midway, above] {$\propto 1/H_0$};
		\end{tikzpicture}
		\caption{Hierarchie der Längenskalen im T0-Modell, mit der Planck-Länge $l_P$ als Referenzeinheit. Die enorme Spannweite von der T0-Charakteristischen Länge $r_0$ bis zur kosmologischen Korrelationslänge $L_T$ umfasst mehr als 66 Größenordnungen, die durch die Setzung von $\hbar = c = G = \alphaEM = \betaT = 1$ in einem einheitlichen Rahmen beschrieben werden können.}
		\label{fig:length_hierarchy}
	\end{figure}
	
	\subsection{Fundamentale Längenskalen}
	
	\begin{table}[ht]
		\centering
		\begin{adjustbox}{scale=0.65}
			\begin{tabular}{llllll}
				\hline
				\textbf{Längenskala} & \textbf{SI-Ausdruck} & \textbf{T0 Nat. Einh.} & \textbf{Prakt. Einheit} & \textbf{Physik. Bedeutung} & \textbf{Verh. zur Planck-L.} \\
				\hline
				Planck-Länge ($l_P$) & $\sqrt{\hbar G/c^3} \approx 1,616 \times 10^{-35}$ m & 1 & 1 (dimensionslos) & Quantengravitationsskala & 1 (Basiseinheit) \\
				T0-Charakt. Länge ($r_0$) & $\xi \cdot l_P$ & $1,33 \times 10^{-4}$ & $1,33 \times 10^{-4} l_P$ & Higgs-bezogene Skala & $1,33 \times 10^{-4}$ \\
				Compton-Wellenlänge des & $\hbar/(m_h \cdot c)$ & $1/m_h$ & $m_h^{-1}$ in E-Einh. & Quantenwellennatur des & $\sim 1,6 \times 10^{-20}$ \\
				Higgs ($\lambda_{C,h}$) & & & & Higgs & \\
				Klassischer Elektronen- & $e^2/(4\pi\varepsilon_0m_e c^2)$ & $\alphaEM \cdot \lambda_{C,e}/(2\pi)$ & $1/(2\pi \cdot m_e)$ & EM-Selbstenergieskala & $\sim 2,4 \times 10^{-23}$ \\
				radius ($r_e$) & & & & & \\
				Compton-Wellenlänge des & $\hbar/(m_e \cdot c)$ & $1/m_e$ & $m_e^{-1}$ in E-Einh. & Quantenwellennatur des & $\sim 2,1 \times 10^{-23}$ \\
				Elektrons ($\lambda_{C,e}$) & & & & Elektrons & \\
				Bohr-Radius ($a_0$) & $4\pi\varepsilon_0\hbar^2/(m_e e^2)$ & $1/(\alphaEM \cdot m_e)$ & $1/m_e$ & Atomare Größenskala & $\sim 4,2 \times 10^{-23}$ \\
				Kosmologischer Horizont & $c/H_0$ & $1/H_0$ & $H_0^{-1}$ & Beobachtbares & $\sim 5,4 \times 10^{61}$ \\
				($d_H$) & & & & Universum & \\
				Kosmologische Korrela- & $(L_T/l_P) \cdot l_P$ & $3,9 \times 10^{62}$ & $3,9 \times 10^{62} l_P$ & Kosmische Struktur- & $3,9 \times 10^{62}$ \\
				tionslänge ($L_T$) & & & & skala & \\
				\hline
				\multicolumn{5}{c}{} \\
				\hline
			\end{tabular}
		\end{adjustbox}
		\caption{Fundamentale Längenskalen im T0-Modell}
		\label{tab:length_scales}
	\end{table}
	
	Diese Tabelle stellt die Größenskalen in hierarchischer Ordnung dar, wobei jede Skala durch ihr Verhältnis zur fundamentalen Planck-Länge charakterisiert wird. Die Verhältnisse zeigen die enorme Spannweite der physikalischen Skalen -- von der subplanckischen T0-Länge ($r_0$) bis zur kosmologischen Korrelationslänge ($L_T$), die über 66 Größenordnungen umfasst \cite{pascher_emergente_2025}.
	
	\subsection{Abgeleitete Längenskalenverhältnisse}
	
	\begin{table}[ht]
		\centering
		\begin{adjustbox}{scale=0.75}
			\begin{tabular}{lllll}
				\hline
				\textbf{Verhältnis} & \textbf{Wert} & \textbf{Formel} & \textbf{Bedeutung} & \textbf{Hierarchie-Ebene} \\
				\hline
				$\xi = r_0/l_P$ & $1,33 \times 10^{-4}$ & $\lambda_h^2v^2/(16\pi^3m_h^2)$ & T0-Planck-Skalen-Verhältnis & 3 - Abgeleitetes Verhältnis \\
				$L_T/l_P$ & $3,9 \times 10^{62}$ & - & Makro-Quanten-Verhältnis & 3 - Abgeleitetes Verhältnis \\
				$r_0/L_T$ & $3,41 \times 10^{-67}$ & $\lambda_h^2v^4/(16\pi^3M_{Pl})$ & Mikro-Makro-Skalen-Verhältnis & 3 - Abgeleitetes Verhältnis \\
				$\lambda_{C,e}/l_P$ & $2,1 \times 10^{-23}$ & $m_P/m_e$ & Elektron-Planck-Massenverhältnis & 3 - Abgeleitetes Verhältnis \\
				$a_0/\lambda_{C,e}$ & $1/(\alphaEM)$ & $1/(\alphaEM)$ & Inverse Feinstrukturkonstante & 2 - Dimensionslose Kopplung \\
				$r_e/\lambda_{C,e}$ & $\alphaEM/(2\pi)$ & $\alphaEM/(2\pi)$ & EM-Selbstenergie-Verhältnis & 2 - Dimensionslose Kopplung \\
				$\lambda_{max} \cdot T$ & $2\pi/\alpha_W$ & $2\pi$ & Wien'sches Verschiebungsgesetz & 2 - Dimensionslose Kopplung \\
				\hline
				 \multicolumn{4}{c}{} \\
				\hline
			\end{tabular}
		\end{adjustbox}
		\caption{Abgeleitete Längenskalenverhältnisse}
		\label{tab:length_ratios}
	\end{table}
	
	Diese Verhältnisse zeigen die hierarchische Struktur der Längenskalen und ihre Beziehungen zu fundamentalen dimensionslosen Konstanten. Sie bilden ein konsistentes Netzwerk von Beziehungen, das die verschiedenen Bereiche der Physik - von der Quantenmechanik über die Elektromagnetik bis zur Kosmologie - miteinander verbindet \cite{pascher_planck_2025}.
	
	\subsection{Verbindung zu Higgs-Parametern}
	
	Die T0-charakteristische Länge $r_0$ ist mit Standardmodell-Parametern durch folgende Beziehung verbunden:
	
	\begin{equation}
		r_0 = \xi \cdot l_P = \frac{\lambda_h^2v^2}{16\pi^3m_h^2} \cdot l_P \approx 1,33 \times 10^{-4} \cdot l_P
	\end{equation}
	
	wobei:
	\begin{itemize}
		\item $\lambda_h \approx 0,13$ (Higgs-Selbstkopplung)
		\item $v \approx 246$ GeV (Higgs-Vakuumerwartungswert)
		\item $m_h \approx 125$ GeV (Higgs-Masse)
	\end{itemize}
	
	Mit der Standardmodell-Relation $m_h^2 = 2\lambda_h v^2$ vereinfacht sich dies zu:
	
	\begin{equation}
		\xi = \frac{\lambda_h}{32\pi^3} \approx 1,31 \times 10^{-4}
	\end{equation}
	
	Diese Verbindung zwischen dem T0-Modell und dem Higgs-Sektor des Standardmodells liefert eine natürliche Brücke zwischen Quantenfeldtheorie und emergenter Gravitation über das intrinsische Zeitfeld $\Tfield$ \cite{pascher_higgs_2025}.
	
	\section{Umrechnung zwischen Natürlichen und SI-Einheiten}
	
	\subsection{Planck-Einheiten und ihre Werte}
	
	Die Planck-Einheiten bilden die Basis des natürlichen Einheitensystems und sind aus den fundamentalen Konstanten $\hbar$, $c$ und $G$ abgeleitet. Im T0-Modell werden sie als Grundlage für ein vereinheitlichtes Einheitensystem verwendet, in dem alle fundamentalen Konstanten auf 1 gesetzt werden:
	
	\begin{table}[ht]
		\centering
		\begin{adjustbox}{scale=0.75}
			\begin{tabular}{lllll}
				\hline
				\textbf{Planck-Einheit} & \textbf{Symbol} & \textbf{Definition} & \textbf{Wert in SI-Einheiten} & \textbf{Bedeutung} \\
				\hline
				Planck-Länge & $l_P$ & $\sqrt{\hbar G/c^3}$ & $1,616 \times 10^{-35}$ m & Fundamentale Längeneinheit \\
				Planck-Zeit & $t_P$ & $\sqrt{\hbar G/c^5}$ & $5,391 \times 10^{-44}$ s & Fundamentale Zeiteinheit \\
				Planck-Masse & $m_P$ & $\sqrt{\hbar c/G}$ & $2,176 \times 10^{-8}$ kg & Fundamentale Masseneinheit \\
				Planck-Energie & $E_P$ & $\sqrt{\hbar c^5/G}$ & $1,956 \times 10^9$ J & Fundamentale Energieeinheit \\
				Planck-Temperatur & $T_P$ & $\sqrt{\hbar c^5/G}/k_B$ & $1,417 \times 10^{32}$ K & Fundamentale Temperatureinheit \\
				Planck-Ladung & $q_P$ & $\sqrt{4\pi\varepsilon_0\hbar c}$ & $1,875 \times 10^{-18}$ C & Fundamentale Ladungseinheit \\
				Planck-Kraft & $F_P$ & $c^4/G$ & $1,210 \times 10^{44}$ N & Fundamentale Krafteinheit \\
				Planck-Druck & $p_P$ & $c^7/(\hbar G^2)$ & $4,633 \times 10^{113}$ Pa & Fundamentale Druckeinheit \\
				Planck-Dichte & $\rho_P$ & $c^5/(\hbar G^2)$ & $5,155 \times 10^{96}$ kg/m$^3$ & Fundamentale Dichteeinheit \\
				\hline
			\multicolumn{4}{c}{} \\
				\hline
			\end{tabular}
		\end{adjustbox}
		\caption{Planck-Einheiten und ihre Werte }
		\label{tab:planck_units}
	\end{table}
	
	Im T0-Modell mit $\hbar = c = G = k_B = \alphaEM = \alpha_W = \betaT = 1$ sind alle diese Planck-Einheiten auf den Wert 1 normiert und dienen als natürliche Referenzeinheiten, aus denen alle anderen physikalischen Größen abgeleitet werden können \cite{pascher_planck_2025}.
	
	\subsection{Umrechnungsformeln zwischen Natürlichen und SI-Einheiten}
	
	Die Umrechnung zwischen natürlichen Einheiten und SI-Einheiten erfolgt durch Multiplikation mit den entsprechenden Planck-Einheiten:
	
	\begin{table}[ht]
		\centering
		\begin{adjustbox}{scale=1}
			\begin{tabular}{lll}
				\hline
				\textbf{Größe} & \textbf{Natürlich $\to$ SI Umrechnung} & \textbf{Praktisches Beispiel} \\
				\hline
				Länge & $L_{\text{SI}} = L_{\text{NE}} \cdot l_{P,\text{SI}}$ & 1 $\to$ $1,616 \times 10^{-35}$ m \\
				Energie & $E_{\text{SI}} = E_{\text{NE}} \cdot E_{P,\text{SI}} = E_{\text{NE}} \cdot \sqrt{\frac{\hbar c^5}{G}}$ & 1 $\to$ $1,956 \times 10^9$ J \\
				Masse & $M_{\text{SI}} = M_{\text{NE}} \cdot M_{P,\text{SI}} = M_{\text{NE}} \cdot \sqrt{\frac{\hbar c}{G}}$ & 1 $\to$ $2,176 \times 10^{-8}$ kg \\
				Zeit & $T_{\text{SI}} = T_{\text{NE}} \cdot t_{P,\text{SI}} = T_{\text{NE}} \cdot \sqrt{\frac{\hbar G}{c^5}}$ & 1 $\to$ $5,391 \times 10^{-44}$ s \\
				Temperatur & $T_{\text{SI}} = T_{\text{NE}} \cdot T_{P,\text{SI}} = T_{\text{NE}} \cdot \frac{M_{P,\text{SI}}\cdot c^2}{k_B}$ & 1 $\to$ $1,417 \times 10^{32}$ K \\
				Elektrische Ladung & $Q_{\text{SI}} = Q_{\text{NE}} \cdot \sqrt{4\pi\varepsilon_0\hbar c}$ & 1 $\to$ $1,875 \times 10^{-18}$ C \\
				\hline
				 \multicolumn{2}{c}{} \\
				\hline
			\end{tabular}
		\end{adjustbox}
		\caption{Umrechnungsformeln zwischen Natürlichen und SI-Einheiten }
		\label{tab:conversion}
	\end{table}
%++++++++++++++++++++++++++++	
\subsection{Fundamentale Längenskalen in Natürlichen Einheiten}

\begin{table}[ht]
	\centering
	\begin{adjustbox}{scale=0.7}
		\begin{tabular}{lllllll}
			\hline
			\textbf{Länge} & \textbf{SI-Wert} & \textbf{T0-Einheiten} & \textbf{Notation} & \textbf{Bedeutung} & \textbf{Verhältnis zu $l_P$} & \textbf{Präzision*} \\
			\hline
			Planck-Länge ($l_P$) & $1.616 \times 10^{-35}$ m & 1 & 1 & QG-Skala & 1 & Ref. \\
			T0-Länge ($r_0$) & - & $1.33 \times 10^{-4}$ & $1.33 \times 10^{-4} l_P$ & Higgs-Skala & $1.33 \times 10^{-4}$ & Theorie \\
			Starke Skala & $\sim 10^{-16}$ m & $\sim 10^{-19}$ & GeV$^{-1}$ & QCD-Skala & $\sim 10^{-19}$ & $10^{-6}$ \\
			Higgs-Länge ($\lambda_{C,h}$) & $1.57 \times 10^{-18}$ m & $1/m_h$ & $m_h^{-1}$ & Higgs-Welle & $\sim 1.6 \times 10^{-20}$ & $10^{-8}$ \\
			Protonradius & $0.84 \times 10^{-15}$ m & $\sim 10^{-20}$ & fm & Hadron-Größe & $\sim 5.2 \times 10^{-20}$ & $10^{-5}$ \\
			Elektronradius ($r_e$) & $2.82 \times 10^{-15}$ m & $1/(2\pi m_e)$ & $\alpha_{EM} \lambda_{C,e}/(2\pi)$ & EM-Energie & $\sim 2.4 \times 10^{-23}$ & $10^{-8}$ \\
			Compton-Länge ($\lambda_{C,e}$) & $2.43 \times 10^{-12}$ m & $1/m_e$ & $m_e^{-1}$ & e$^-$ Welle & $\sim 2.1 \times 10^{-23}$ & $10^{-9}$ \\
			Bohr-Radius ($a_0$) & $5.29 \times 10^{-11}$ m & $1/(\alpha_{EM} m_e)$ & $1/m_e$ & Atomgröße & $\sim 4.2 \times 10^{-23}$ & $10^{-8}$ \\
			DNA-Breite & $2 \times 10^{-9}$ m & $\sim 10^{-26}$ & nm & Gen-Skala & $\sim 1.2 \times 10^{-26}$ & Direkt \\
			Zelle & $\sim 10^{-5}$ m & $\sim 10^{-30}$ & $\mu$m & Lebensskala & $\sim 6.2 \times 10^{-30}$ & Direkt \\
			Mensch & $\sim 1$ m & $\sim 10^{-35}$ & m & Makroskala & $\sim 6.2 \times 10^{-35}$ & Direkt \\
			Erd-Radius & $6.37 \times 10^{6}$ m & $\sim 10^{-41}$ & km & Planetenskala & $\sim 3.9 \times 10^{-41}$ & $10^{-7}$ \\
			Sonnenradius & $6.96 \times 10^{8}$ m & $\sim 10^{-43}$ & R$_{\odot}$ & Sternenskala & $\sim 4.3 \times 10^{-43}$ & $10^{-6}$ \\
			Sonnensystem & $\sim 10^{12}$ m & $\sim 10^{-47}$ & AE & Systemskala & $\sim 6.2 \times 10^{-47}$ & $10^{-6}$ \\
			Galaxie & $\sim 10^{21}$ m & $\sim 10^{-56}$ & kpc & Galaxienskala & $\sim 6.2 \times 10^{-56}$ & $10^{-4}$ \\
			Cluster & $\sim 10^{23}$ m & $\sim 10^{-58}$ & Mpc & Strukturskala & $\sim 6.2 \times 10^{-58}$ & $10^{-3}$ \\
			Horizont ($d_H$) & $\sim 8.8 \times 10^{26}$ m & $1/H_0$ & $H_0^{-1}$ & Beobachtbares U. & $\sim 5.4 \times 10^{61}$ & $10^{-3}$ \\
			Korrelationslänge ($L_T$) & $\sim 6.3 \times 10^{27}$ m & $3.9 \times 10^{62}$ & $3.9 \times 10^{62} l_P$ & Kosmische Skala & $3.9 \times 10^{62}$ & $10^{-2}$ \\
			\hline
			\multicolumn{7}{l}{* Präzision bezieht sich auf die empirische Übereinstimmung zwischen T0-Modell und Messungen nach korrekter Einheitenumrechnung.} \\
			\multicolumn{7}{l}{Abkürzungen: QG - Quantengravitation, QCD - Quantenchromodynamik, EM - Elektromagnetisch} \\
			\hline
		\end{tabular}
	\end{adjustbox}
	\caption{Fundamentale Längenskalen im T0-Modell}
	\label{tab:length_scales}
\end{table}

Diese Tabelle stellt Größenskalen in hierarchischer Reihenfolge dar, wobei jede Skala durch ihr Verhältnis zur fundamentalen Planck-Länge charakterisiert wird. Die Spalte „Präzision“ gibt die Übereinstimmung zwischen den Vorhersagen des T0-Modells und Messungen nach korrekter Einheitenumrechnung an. Dieser umfassende Bereich erstreckt sich von quantenmechanischen bis kosmologischen Skalen und zeigt die Selbstkonsistenz des T0-Modells mit $\alphaEM = \betaT = 1$ und Energie als Grundeinheit.

Die Tabelle überbrückt nun zuvor unberücksichtigte Lücken in der Skalenhierarchie und zeigt, dass das Modell seine Vorhersagekraft über alle beobachtbaren Skalen des Universums hinweg beibehält – von sub-Planck-Distanzen bis hin zu kosmologischen Entfernungen. Diese Kontinuität ist eine zentrale Stärke des vereinheitlichten natürlichen Einheitensystems, das Phänomene über 97 Größenordnungen hinweg elegant beschreibt.

Die bemerkenswerte Übereinstimmung mit experimentellen Messungen bestätigt, dass die Setzung von $\alphaEM = \betaT = 1$ keine Näherung ist, sondern eine mathematisch elegante Neuformulierung, die alle physikalischen Vorhersagen erhält \cite{pascher_emergente_2025, pascher_alphabeta_2025}.
%++++++++++++++++++++++++++++++++	
	\subsubsection{Praktische Verwendung Dimensionsloser Parameter}
	
	Obwohl dimensionslose Parameter keine physikalischen Einheiten haben, werden sie in verschiedenen Kontexten unterschiedlich angegeben:
	
	\begin{itemize}
		\item \textbf{Feinstrukturkonstante $\alphaEM$}: Meist als Bruch (1/137,036) oder Dezimalwert ($\approx$ 0,0073) angegeben
		\item \textbf{T0-Parameter $\betaT$}: In wissenschaftlichen Arbeiten als Dezimalwert (0,008 in SI-Kontext, 1 in natürlichen Einheiten)
		\item \textbf{Wien'sche Konstante $\alpha_W$}: Als Dezimalwert ($\approx$ 2,82) in thermodynamischen Berechnungen
	\end{itemize}
	
	Diese Parameter behalten ihre Zahlenwerte unabhängig vom Einheitensystem bei, wenn sie in dimensionslosen Gleichungen verwendet werden. In Gleichungen mit dimensionsbehafteten Größen müssen sie jedoch ggf. angepasst werden, wenn zwischen natürlichen Einheiten und SI-Einheiten gewechselt wird \cite{pascher_alpha_2025, pascher_beta_2025}.
	
	\section{Feldgleichungen in Natürlichen Einheiten}
	
	\subsection{Maxwell-Gleichungen ($\alphaEM = 1$)}
	
	Mit $\alphaEM = 1$ und den daraus abgeleiteten elektromagnetischen Konstanten $\mu_0 = \varepsilon_0 = 1$ nehmen die Maxwell-Gleichungen eine besonders elegante Form an:
	
	\begin{table}[ht]
		\centering
		\begin{adjustbox}{scale=0.75}
			\begin{tabular}{llll}
				\hline
				\textbf{Gleichung} & \textbf{Klassische Form} & \textbf{Natürliche Form ($\alphaEM = 1$)} & \textbf{Vereinfachung} \\
				\hline
				Gauss'sches Gesetz & $\nabla\cdot\vec{E} = \frac{\rho}{\varepsilon_0}$ & $\nabla\cdot\vec{E} = \rho$ & Ladungsdichte direkt als Feldquelle \\
				Ampère'sches Gesetz & $\nabla\times\vec{B} - \mu_0\varepsilon_0\frac{\partial\vec{E}}{\partial t} = \mu_0\vec{j}$ & $\nabla\times\vec{B} - \frac{\partial\vec{E}}{\partial t} = \vec{j}$ & Stromdichte direkt als Feldquelle \\
				Gauss für Magnetismus & $\nabla\cdot\vec{B} = 0$ & $\nabla\cdot\vec{B} = 0$ & Unveränderlich \\
				Faraday'sches Gesetz & $\nabla\times\vec{E} + \frac{\partial\vec{B}}{\partial t} = 0$ & $\nabla\times\vec{E} + \frac{\partial\vec{B}}{\partial t} = 0$ & Unveränderlich \\
				\hline
			\multicolumn{3}{c}{} \\
				\hline
			\end{tabular}
		\end{adjustbox}
		\caption{Maxwell-Gleichungen in natürlichen Einheiten }
		\label{tab:maxwell}
	\end{table}
	
	In dieser Form zeigt sich die intrinsische Symmetrie zwischen elektrischen und magnetischen Feldern besonders deutlich. Mit der Elementarladung $e = 1$ werden zudem alle elektromagnetischen Größen dimensionslos oder auf Energiedimensionen reduziert:
	
	\begin{table}[ht]
		\centering
		\begin{adjustbox}{scale=0.85}
			\begin{tabular}{llll}
				\hline
				\textbf{Größe} & \textbf{SI-Dimension} & \textbf{Natürliche Dimension} & \textbf{Veranschaulichung} \\
				\hline
				Elektrisches Feld & [V/m] = [ML$^2$T$^{-3}$I$^{-1}$] & [E$^2$] & Energie pro Länge und Ladung \\
				Magnetisches Feld & [T] = [MT$^{-2}$I$^{-1}$] & [E$^2$] & Energie pro Fläche und Ladung \\
				Ladungsdichte & [C/m$^3$] = [L$^{-3}$TI] & [E$^3$] & Ladung pro Volumen \\
				Stromdichte & [A/m$^2$] = [L$^{-2}$I] & [E$^3$] & Ladung pro Fläche und Zeit \\
				\hline
			 \multicolumn{3}{c}{} \\
				\hline
			\end{tabular}
		\end{adjustbox}
		\caption{Dimensionen elektromagnetischer Größen}
		\label{tab:em_dimensions}
	\end{table}
	
	Die Vereinheitlichung der Dimensionen verdeutlicht, dass elektromagnetische Felder fundamentale Manifestationen von Energiegradienten sind - eine direkte Konsequenz des Prinzips, Energie als fundamentale Einheit zu betrachten \cite{pascher_alpha_2025}.
	
	\subsection{T0-Modell-Gleichungen ($\betaT = 1$)}
	
	Im T0-Modell mit $\betaT = 1$ nehmen die grundlegenden Gleichungen besonders elegante Formen an:
	
	\begin{table}[ht]
		\centering
		\begin{adjustbox}{scale=0.70}
			\begin{tabular}{lll}
				\hline
				\textbf{Gleichung} & \textbf{Natürliche Form ($\betaT = 1$)} & \textbf{Physikalische Bedeutung} \\
				\hline
				Temperatur-Rotverschiebungs-Relation & $T(z) = T_0(1+z)(1+\ln(1+z))$ & Erweiterte kosmische Temperaturentwicklung \\
				Wellenlängenabhängige Rotverschiebung & $z(\lambda) = z_0(1+\ln(\lambda/\lambda_0))$ & Frequenzabhängige kosmologische Rotverschiebung \\
				Modifiziertes Gravitationspotential & $\Phi(r) = -\frac{M}{r} + r$ & Emergente Gravitation mit linearterm \\
				Intrinsisches Zeitfeld (statisch) & $\nabla^2\Tfield \approx -\frac{\rho}{\Tfield^2}$ & Quellterm für das intrinsische Zeitfeld \\
				Effektives Gravitationspotential & $\Phi(\vec{x}) = -\ln\left(\frac{\Tfield}{\Tfield_0}\right)$ & Verknüpfung von Gravitation und Zeitfeld \\
				Gravitationskraft & $\vec{F} = -\nabla\Phi = -\frac{\nabla\Tfield}{\Tfield}$ & Kraftgesetz aus Zeitfeldgradienten \\
				\hline
			\multicolumn{2}{c}{} \\
				\hline
			\end{tabular}
		\end{adjustbox}
		\caption{T0-Modell-Gleichungen in natürlichen Einheiten }
		\label{tab:t0_equations}
	\end{table}
	
	Besonders bemerkenswert ist, dass die Gravitation als emergentes Phänomen aus dem intrinsischen Zeitfeld $\Tfield$ hervorgeht, ohne dass eine fundamentale Gravitationswechselwirkung postuliert werden muss. Der lineare Term im modifizierten Gravitationspotential ($+r$) führt zu Effekten, die in der Standardkosmologie der dunklen Energie zugeschrieben werden \cite{pascher_emergente_2025}.
	
	\subsection{Modifizierte Quantenmechanik}
	
	Das T0-Modell modifiziert die Grundgleichungen der Quantenmechanik durch die Einbeziehung des intrinsischen Zeitfeldes $\Tfield$:
	
	\begin{table}[ht]
		\centering
		\begin{adjustbox}{scale=1.1}
			\begin{tabular}{lll}
				\hline
				\textbf{Gleichung} & \textbf{Natürliche Form} & \textbf{Standardform} \\
				\hline
				Modifizierte Schrödinger-Gleichung & $i\Tfield\frac{\partial\Psi}{\partial t} + i\Psi\frac{\partial \Tfield}{\partial t} = \hat{H}\Psi$ & $i\hbar\frac{\partial\Psi}{\partial t} = \hat{H}\Psi$ \\
				Dekohärenzrate & $\Gamma_{\text{dec}} = \Gamma_0 \cdot m$ & $\Gamma_{\text{dec}} = \Gamma_0 \cdot \frac{mc^2}{\hbar}$ \\
				Welle-Teilchen-Relation & $\lambda = \frac{1}{p}$ & $\lambda = \frac{h}{p}$ \\
				Zeit-Energie-Unschärfe & $\Delta E \cdot \Delta t \geq \frac{1}{2}$ & $\Delta E \cdot \Delta t \geq \frac{\hbar}{2}$ \\
				\hline
			\multicolumn{2}{c}{} \\
				\hline
			\end{tabular}
		\end{adjustbox}
		\caption{Modifizierte quantenmechanische Gleichungen}
		\label{tab:qm_equations}
	\end{table}
	
	Die modifizierte Schrödinger-Gleichung verknüpft die Zeitentwicklung des Quantenzustands mit dem intrinsischen Zeitfeld $\Tfield$, was zu einer massenabhängigen Zeitentwicklung führt. Dies ermöglicht eine natürliche Erklärung für:
	
	\begin{itemize}
		\item \textbf{Massenabhängige Dekohärenz:} Schwerere Teilchen dekohärieren schneller, im Einklang mit experimentellen Beobachtungen.
		\item \textbf{Quantenkorrelationen:} Die scheinbare Nichtlokalität in verschränkten Systemen kann durch massenspezifische Zeitskalen erklärt werden.
		\item \textbf{Teilchen-Welle-Dualität:} Durch die Formulierung $\Tfield = \frac{1}{\max(m,\omega)}$ wird die Dualität von Materie und Strahlung vereinheitlicht.
	\end{itemize}
	
	Für verschränkte Zustände nimmt die Zeitentwicklung die Form an:
	\begin{equation}
		|\Psi(t)\rangle = \frac{1}{\sqrt{2}}\left(|0(t/T_1)\rangle_{m_1} \otimes |1(t/T_2)\rangle_{m_2} + |1(t/T_1)\rangle_{m_1} \otimes |0(t/T_2)\rangle_{m_2}\right)
	\end{equation}
	wobei $T_1 = \frac{1}{m_1}$, $T_2 = \frac{1}{m_2}$ die intrinsischen Zeitskalen der beteiligten Teilchen sind \cite{pascher_quantum_2025, pascher_photons_2025}.
	
	\section{Fundamentale Beziehungen zwischen Einheiten im T0-Modell}
	
	\subsection{Netzwerk der Verhältnisse zwischen physikalischen Größen}
	
	Die Hierarchie und Beziehungen zwischen den physikalischen Größen können durch ein Netzwerk von Verhältnissen dargestellt werden:
	
	\begin{figure}[ht]
		\centering
		\begin{tikzpicture}
			\small % Kleinere Schriftgröße für das gesamte Diagramm
			% Zentrale Energieeinheit
			\node[draw, circle, fill=blue!20, minimum size=2cm] (energy) at (0,0) {Energie [E]};
			
			% Obere Abzweigungen
			\node[draw, ellipse, fill=green!20] (mass) at (-3,2) {Masse [E]};
			\node[draw, ellipse, fill=green!20] (temp) at (3,2) {Temperatur [E]};
			
			% Untere Abzweigungen
			\node[draw, ellipse, fill=orange!20] (length) at (-4,-2) {Länge [E$^{-1}$]};
			\node[draw, ellipse, fill=orange!20] (time) at (-1,-2) {Zeit [E$^{-1}$]};
			\node[draw, ellipse, fill=orange!20] (wavelength) at (2,-2) {Wellenlänge [E$^{-1}$]};
			\node[draw, ellipse, fill=orange!20] (frequency) at (5,0) {Frequenz [E]};
			
			% Zentrale Verbindung
			\node[draw, rectangle, fill=red!20] (tfield) at (0,-4) {Intrinsische Zeit $\Tfield$ [E$^{-1}$]};
			
			% Verbindungslinien
			\draw[->, thick] (energy) -- (mass) node[midway, above] {$m = E$};
			\draw[->, thick] (energy) -- (temp) node[midway, above] {$T = E$};
			\draw[->, thick] (energy) -- (length) node[midway, above left] {$L = E^{-1}$};
			\draw[->, thick] (energy) -- (time) node[midway, above left] {$t = E^{-1}$};
			\draw[->, thick] (energy) -- (wavelength) node[midway, above right] {$\lambda = E^{-1}$};
			\draw[->, thick] (energy) -- (frequency) node[midway, above] {$\nu = E$};
			
			\draw[->, thick, dashed] (length) -- (time) node[midway, below] {$c = 1$};
			\draw[->, thick, dashed] (wavelength) -- (frequency) node[midway, below right] {$\lambda\nu = 1$};
			\draw[->, thick, dashed] (mass) to[out=-45, in=135] node[midway, right] {$E = mc^2$} (energy);
			\draw[->, thick, dashed] (temp) to[out=-135, in=45] node[midway, left] {$E = k_B T$} (energy);
			
			\draw[->, thick, red] (tfield) -- (mass) node[midway, left] {$\Tfield = \frac{1}{m}$};
			\draw[->, thick, red] (tfield) -- (time) node[midway, right] {$\Tfield \sim t$};
			\draw[->, thick, red] (tfield) -- (wavelength) node[midway, right] {$\Tfield = \frac{1}{\omega}$};
		\end{tikzpicture}
		\caption{Netzwerk der Verhältnisse zwischen physikalischen Größen im T0-Modell. Alle Größen sind auf Energie [E] als fundamentale Einheit zurückführbar. Die durchgezogenen Linien zeigen direkte Dimensionsbeziehungen, die gestrichelten Linien zeigen physikalische Äquivalenzen durch dimensionslose Konstanten ($c = k_B = 1$). Die roten Linien repräsentieren die vermittelnde Rolle des intrinsischen Zeitfeldes $\Tfield$.}
		\label{fig:quantity_network}
	\end{figure}
	
	\subsection{Quantitative Verhältnisse und Skalenhierarchie}
	
	Die quantitativen Verhältnisse zwischen verschiedenen Skalen ergeben eine hierarchische Struktur:
	
	\begin{figure}[ht]
		\centering
		\begin{tikzpicture}
			\footnotesize % Noch kleinere Schriftgröße für dichtes Diagramm
			% Energieskala-Achse
			\draw[thick, ->] (0,0) -- (12,0) node[right] {Energieskala};
			\draw[thick, ->] (0,0) -- (0,4) node[above] {Massenbereich};
			
			% Skalenmarkierungen
			\draw[thick] (2,0.2) -- (2,-0.2) node[below] {Planck-Skala};
			\draw[thick] (5,0.2) -- (5,-0.2) node[below] {Elektroschwache Skala};
			\draw[thick] (8,0.2) -- (8,-0.2) node[below] {QCD-Skala};
			\draw[thick] (11,0.2) -- (11,-0.2) node[below] {Atomare Skala};
			
			% Massen-/Energiemarkierungen
			\node[draw, fill=blue!10] at (2,2.5) {$M_P \sim 10^{19}$ GeV};
			\node[draw, fill=blue!10] at (5,2) {$M_W, M_Z, M_H \sim 10^2$ GeV};
			\node[draw, fill=blue!10] at (8,1.5) {$\Lambda_{QCD} \sim 0.2$ GeV};
			\node[draw, fill=blue!10] at (11,1) {$m_e \sim 0.0005$ GeV};
			
			% T0-Parameter
			\draw[thick, red, ->] (2,3.5) -- (5,2.5) node[midway, above] {$\xi = 1.33 \times 10^{-4}$};
			\draw[thick, red, ->] (5,2.5) -- (8,2) node[midway, above] {$\alphaEM = 1$};
			\draw[thick, red, ->] (8,2) -- (11,1.5) node[midway, above] {$\betaT = 1$};
			
			% T0-Modell Spezifik
			\node[draw, ellipse, fill=red!10] at (6,3.9) {T0-Modell: $\hbar = c = G = \alphaEM = \betaT = 1$};
		\end{tikzpicture}
		\caption{Hierarchie der Energieskalen im T0-Modell. Die dimensionslosen Konstanten ($\xi$, $\alphaEM$, $\betaT$) verbinden die verschiedenen Energieskalen von der Planck-Skala bis zur atomaren Skala. Im T0-Modell mit $\hbar = c = G = \alphaEM = \betaT = 1$ werden diese Skalen durch reine Zahlenverhältnisse verbunden, wobei Energie die fundamentale Einheit bildet.}
		\label{fig:energy_hierarchy}
	\end{figure}
	
	\subsection{Verhältnisse der Fundamentalen Kräfte in Natürlichen Einheiten}
	
	Im T0-Modell mit vereinheitlichtem natürlichem Einheitensystem können die fundamentalen Wechselwirkungen durch ihre dimensionslosen Kopplungskonstanten charakterisiert werden:
	
	\begin{table}[ht]
		\centering
		\begin{adjustbox}{scale=0.95}
			\begin{tabular}{llll}
				\hline
				\textbf{Kraft} & \textbf{Dimensionslose Kopplung} & \textbf{Natürlicher Wert} & \textbf{Reichweite} \\
				\hline
				Elektromagnetisch & $\alphaEM$ & 1 & $\infty$ \\
				Stark & $\alpha_s$ & $\sim 0,118$ bei $Q^2=M_Z^2$ & $\sim 10^{-15}$ m \\
				Schwach & $\alpha_W = g^2/(4\pi)$ & $\sim 1/30$ & $\sim 10^{-18}$ m \\
				Gravitation & $\alpha_G = Gm^2/\hbar c$ & $m^2/m_P^2$ & $\infty$ \\
				\hline
				\multicolumn{3}{c}{} \\
				\hline
			\end{tabular}
		\end{adjustbox}
		\caption{Fundamentale Kräfte in natürlichen Einheiten}
		\label{tab:forces}
	\end{table}
	
	Im T0-Modell werden diese Verhältnisse neu interpretiert: Die Gravitation ist keine fundamentale Kraft mehr, sondern eine emergente Eigenschaft des intrinsischen Zeitfeldes $\Tfield$, was zu einer natürlichen Vereinheitlichung führt \cite{pascher_grundkraefte_2025}.
	
	\section{Rolle der Energie als Fundamentale Einheit}
	
	Im vereinheitlichten natürlichen Einheitensystem des T0-Modells dient Energie [E] als die fundamentale Einheit, aus der alle anderen physikalischen Größen abgeleitet werden können:
	
	\subsection{Praktische Notationen in natürlichen Einheiten}
	
	\begin{table}[ht]
		\centering
		\begin{adjustbox}{scale=1.2}
			\begin{tabular}{lll}
				\hline
				\textbf{Physikalische Größe} & \textbf{Natürliche Einheit} & \textbf{Praktische Notation} \\
				\hline
				Länge & [E$^{-1}$] & eV$^{-1}$, GeV$^{-1}$, TeV$^{-1}$ \\
				Zeit & [E$^{-1}$] & eV$^{-1}$, GeV$^{-1}$, TeV$^{-1}$ \\
				Masse/Energie & [E] & eV, MeV, GeV, TeV \\
				Temperatur & [E] & eV, MeV \\
				Impuls & [E] & eV, GeV \\
				Wirkungsquerschnitt & [E$^{-2}$] & GeV$^{-2}$, mb, pb, fb \\
				Zerfallsrate & [E] & eV, MeV \\
				\hline
			\multicolumn{2}{c}{} \\
				\hline
			\end{tabular}
		\end{adjustbox}
		\caption{Praktische Notationen physikalischer Größen in natürlichen Einheiten}
		\label{tab:practical_notation}
	\end{table}
	
	\begin{figure}[ht]
		\centering
		\begin{tikzpicture}
			\footnotesize % Kleinere Schriftgröße für das Diagramm
			% Konversionsschema
			\node[draw, rounded corners, fill=blue!10, minimum width=8cm, minimum height=1.5cm] (conversion) at (0,0) {
				\begin{tabular}{c|c|c}
					Länge & Zeit & Energie/Masse \\
					\hline
					1 GeV$^{-1} \approx 0,197$ fm & 1 GeV$^{-1} \approx 6,58 \times 10^{-25}$ s & 1 eV $\approx 1,78 \times 10^{-36}$ kg
				\end{tabular}
			};
			
			\node[draw, rounded corners, fill=green!10, minimum width=8cm, minimum height=1.5cm] (practical) at (0,-2.5) {
				\begin{tabular}{c|c|c}
					Protonen-Masse & Elektron-Masse & Temperatur \\
					\hline
					$m_p \approx 0,938$ GeV & $m_e \approx 0,511$ MeV & 1 eV $\approx 11.605$ K
				\end{tabular}
			};
			
			\node[draw, rounded corners, fill=red!10, minimum width=8cm, minimum height=1.5cm] (fundamental) at (0,-5) {
				\begin{tabular}{c|c|c}
					Planck-Masse & Planck-Länge & Planck-Zeit \\
					\hline
					$m_P = 1$ & $l_P = 1$ & $t_P = 1$
				\end{tabular}
			};
			
			\node[above] at (conversion.north) {\textbf{Praktische Umrechnungen in natürlichen Einheiten}};
			\node[above] at (practical.north) {\textbf{Typische Teilchenphysikalische Größen}};
			\node[above] at (fundamental.north) {\textbf{Fundamentale Einheiten im T0-Modell}};
		\end{tikzpicture}
		\caption{Praktische Umrechnungen zwischen SI-Einheiten und natürlichen Einheiten, sowie typische Größen in der Teilchenphysik. Im T0-Modell mit $\hbar = c = G = \alphaEM = \betaT = 1$ sind alle Planck-Einheiten auf 1 normiert, und alle physikalischen Größen können in Vielfachen oder Bruchteilen dieser Einheiten ausgedrückt werden.}
		\label{fig:practical_conversion}
	\end{figure}
	
	\subsection{Philosophische Implikationen}

Die Verwendung von Energie als fundamentale Einheit im T0-Modell hat tiefgreifende philosophische Implikationen:

\begin{enumerate}
	\item \textbf{Ontologische Vereinfachung:} Energie wird zur fundamentalen Entität, aus der alle anderen physikalischen Größen abgeleitet werden können. Dies steht im Einklang mit der Einsteinäquivalenz von Masse und Energie und erweitert sie auf alle physikalischen Größen.
	
	\item \textbf{Einheitliche Beschreibung der Natur:} Die Verwendung natürlicher Einheiten mit $\hbar = c = G = \alphaEM = \betaT = 1$ ermöglicht eine vereinheitlichte Beschreibung aller bekannten physikalischen Phänomene ohne willkürliche dimensionsbehaftete Konstanten.
	
	\item \textbf{Emergente Raum-Zeit:} Im T0-Modell kann die Raumzeit als emergentes Phänomen betrachtet werden, das aus den Eigenschaften des intrinsischen Zeitfeldes $\Tfield$ hervorgeht. Dies entspricht modernen Ansätzen in der theoretischen Physik, die Raum und Zeit als emergente Eigenschaften eines fundamentaleren Substrats betrachten \cite{pascher_perspective_2025, pascher_zeit_2025}.
	
	\item \textbf{Überwindung des Leib-Seele-Problems:} Die Einführung einer absoluten Zeit im T0-Modell bei gleichzeitiger Reinterpretation relativistischer Effekte als Massenvariation bietet einen neuen Ansatz zum Verständnis des Bewusstseins und seiner Beziehung zur physikalischen Welt \cite{pascher_perspective_2025}.
\end{enumerate}

Die Vereinheitlichung durch Energie als fundamentale Einheit ist nicht nur eine mathematische Vereinfachung, sondern spiegelt die intrinsische Einheit der Naturgesetze wider, wie sie im T0-Modell postuliert wird \cite{pascher_dualismus_2025}.

\section{Zusammenfassung und Ausblick}

Das vereinheitlichte natürliche Einheitensystem des T0-Modells mit $\hbar = c = G = k_B = \alphaEM = \alpha_W = \betaT = 1$ bietet einen eleganten Rahmen für die Vereinheitlichung aller physikalischen Phänomene mit Energie als fundamentaler Einheit. Die wichtigsten Erkenntnisse sind:

\begin{enumerate}
	\item \textbf{Hierarchische Struktur:} Die physikalischen Konstanten und Skalen bilden eine klare hierarchische Struktur, wobei alle Größen auf Energie [E] als fundamentale Einheit zurückgeführt werden können.
	
	\item \textbf{Vereinfachte Feldgleichungen:} Die Grundgleichungen der Physik nehmen in diesem System besonders elegante Formen an, was die intrinsische Einheit der Naturgesetze offenbart.
	
	\item \textbf{Brücke zwischen Quantenmechanik und Relativitätstheorie:} Das intrinsische Zeitfeld $\Tfield$ dient als Mediator zwischen Quantenmechanik und Relativitätstheorie, indem es eine Brücke zwischen Mikro- und Makroskala schlägt.
	
	\item \textbf{Emergente Gravitation:} Die Gravitation wird als emergentes Phänomen aus dem intrinsischen Zeitfeld $\Tfield$ neu interpretiert, ohne dass eine fundamentale Gravitationswechselwirkung angenommen werden muss.
	
	\item \textbf{Natürliche Erklärung kosmologischer Phänomene:} Das T0-Modell bietet natürliche Erklärungen für Phänomene wie Rotverschiebung, kosmische Expansion und dunkle Energie, ohne Ad-hoc-Annahmen wie die Inflationstheorie oder dunkle Materie zu benötigen.
\end{enumerate}

Zukünftige Forschungsrichtungen könnten folgende Aspekte umfassen:

\begin{itemize}
	\item \textbf{Experimentelle Tests der wellenlängenabhängigen Rotverschiebung:} $z(\lambda) = z_0(1+\ln(\lambda/\lambda_0))$, was einen direkten Test des Parameters $\betaT = 1$ ermöglichen würde.
	
	\item \textbf{Präzisionsmessungen atomarer Energieniveaus:} Die Reinterpretation der Rydberg-Konstante als $R_\infty = 1/2$ in natürlichen Einheiten könnte zu neuen experimentellen Tests führen.
	
	\item \textbf{Quantenfeldtheoretische Entwicklung des T0-Modells:} Die vollständige Quantisierung des intrinsischen Zeitfeldes $\Tfield$ und seine Einbettung in einen quantenfeldtheoretischen Rahmen bleibt eine wichtige Herausforderung.
	
	\item \textbf{Numerische Simulationen der kosmologischen Entwicklung:} Mit dem modifizierten Gravitationspotential $\Phi(r) = -\frac{M}{r} + r$ könnten Computersimulationen der Galaxiendynamik durchgeführt werden, um Vorhersagen des T0-Modells mit astronomischen Beobachtungen zu vergleichen.
\end{itemize}

Die vereinheitlichte natürliche Einheitensystematik des T0-Modells mit Energie als fundamentaler Einheit bietet einen vielversprechenden Ansatz zur Überwindung der Trennung zwischen Quantenmechanik und Relativitätstheorie und könnte zu einem tieferen Verständnis der fundamentalen Struktur des Universums führen \cite{pascher_vereinheitlichung_2025}.

\bibliographystyle{apsrev4-2}
\begin{thebibliography}{99}
	\bibitem{pascher_zeit_2025} J. Pascher, \href{https://github.com/jpascher/T0-Time-Mass-Duality/tree/main/2/pdf/English/ZeitEmergentQMEn.pdf}{Time as an Emergent Property in Quantum Mechanics: A Connection Between Relativity, Fine-Structure Constant, and Quantum Dynamics}, March 23, 2025.
	\bibitem{pascher_messdifferenzen_2025} J. Pascher, \href{https://github.com/jpascher/T0-Time-Mass-Duality/tree/main/2/pdf/English/MessdifferenzenT0StandardEn.pdf}{Compensatory and Additive Effects: An Analysis of Measurement Differences Between the T0 Model and the $\Lambda$CDM Standard Model}, April 2, 2025.
	\bibitem{pascher_galaxies_2025} J. Pascher, \href{https://github.com/jpascher/T0-Time-Mass-Duality/tree/main/2/pdf/English/MassVarGalaxienEn.pdf}{Mass Variation in Galaxies: An Analysis in the T0 Model with Emergent Gravitation}, March 30, 2025.
	\bibitem{pascher_params_2025} J. Pascher, \href{https://github.com/jpascher/T0-Time-Mass-Duality/tree/main/2/pdf/English/ZeitMasseT0ParamsEn.pdf}{Time-Mass Duality Theory (T0 Model): Derivation of Parameters $\kappa$, $\alpha$, and $\beta$}, March 30, 2025.
	\bibitem{pascher_temp_2025} J. Pascher, \href{https://github.com/jpascher/T0-Time-Mass-Duality/tree/main/2/pdf/English/NatEinheitenAlpha1En.pdf}{Adjustment of Temperature Units in Natural Units and CMB Measurements}, April 2, 2025.
	\bibitem{pascher_alpha_2025} J. Pascher, \href{https://github.com/jpascher/T0-Time-Mass-Duality/tree/main/2/pdf/English/NatEinheitenAlpha1En.pdf}{Energy as a Fundamental Unit: Natural Units with $\alphaEM = 1$ in the T0 Model}, March 26, 2025.
	\bibitem{pascher_beta_2025} J. Pascher, \href{https://github.com/jpascher/T0-Time-Mass-Duality/tree/main/2/pdf/English/Alpha1Beta1KonsistenzEn.pdf}{Dimensionless Parameters in the T0 Model: Setting $\beta = 1$ in Natural Units}, April 4, 2025.
	\bibitem{pascher_higgs_2025} J. Pascher, \href{https://github.com/jpascher/T0-Time-Mass-Duality/tree/main/2/pdf/English/MathHiggsZeitMasseEn.pdf}{Mathematical Formulation of the Higgs Mechanism in Time-Mass Duality}, March 28, 2025.
	\bibitem{pascher_lagrange_2025} J. Pascher, \href{https://github.com/jpascher/T0-Time-Mass-Duality/tree/main/2/pdf/English/MathZeitMasseLagrangeEn.pdf}{From Time Dilation to Mass Variation: Mathematical Core Formulations of Time-Mass Duality Theory}, March 29, 2025.
	\bibitem{pascher_emergente_2025} J. Pascher, \href{https://github.com/jpascher/T0-Time-Mass-Duality/tree/main/2/pdf/English/EmergentGravT0En.pdf}{Emergent Gravitation in the T0 Model: A Comprehensive Derivation}, April 1, 2025.
	\bibitem{pascher_perspective_2025} J. Pascher, \href{https://github.com/jpascher/T0-Time-Mass-Duality/tree/main/2/pdf/English/ZeitRaumPascherEn.pdf}{A New Perspective on Time and Space: Johann Pascher's Revolutionary Ideas}, March 25, 2025.
	\bibitem{pascher_dualismus_2025} J. Pascher, \href{https://github.com/jpascher/T0-Time-Mass-Duality/tree/main/2/pdf/English/KurzKomplementDualPhysikEn.pdf}{In Brief - Complementary Duality in Physics: From Wave-Particle to Time-Mass Concept}, March 26, 2025.
	\bibitem{pascher_grundkraefte_2025} J. Pascher, \href{https://github.com/jpascher/T0-Time-Mass-Duality/tree/main/2/pdf/English/VierKraefteZeitMasseEn.pdf}{Simplified Description of the Fundamental Forces with Time-Mass Duality}, March 27, 2025.
	\bibitem{pascher_zeit_masse_2025} J. Pascher, \href{https://github.com/jpascher/T0-Time-Mass-Duality/tree/main/2/pdf/English/ZeitMasseNeuerBlickEn.pdf}{Time and Mass: A New Look at Old Formulas – and Liberation from Traditional Constraints}, March 22, 2025.
	\bibitem{pascher_quantum_2025} J. Pascher, \href{https://github.com/jpascher/T0-Time-Mass-Duality/tree/main/2/pdf/English/NotwendigkeitQMErweiterungEn.pdf}{The Necessity of Extending Standard Quantum Mechanics and Quantum Field Theory}, March 27, 2025.
	\bibitem{pascher_photons_2025} J. Pascher, \href{https://github.com/jpascher/T0-Time-Mass-Duality/tree/main/2/pdf/English/DynMassePhotonenNichtlokalEn.pdf}{Dynamic Mass of Photons and Its Implications for Nonlocality in the T0 Model}, March 25, 2025.
	\bibitem{pascher_alphabeta_2025} J. Pascher, \href{https://github.com/jpascher/T0-Time-Mass-Duality/tree/main/2/pdf/English/Alpha1Beta1KonsistenzEn.pdf}{Unified Unit System in the T0 Model: The Consistency of $\alpha = 1$ and $\beta = 1$}, April 5, 2025.
	\bibitem{pascher_planck_2025} J. Pascher, \href{https://github.com/jpascher/T0-Time-Mass-Duality/tree/main/2/pdf/English/JenseitsPlanckEn.pdf}{Real Consequences of Reformulating Time and Mass in Physics: Beyond the Planck Scale}, March 24, 2025.
	\bibitem{pascher_energiedynamik_2025} J. Pascher, \href{https://github.com/jpascher/T0-Time-Mass-Duality/tree/main/2/pdf/English/MathEnergiedynamikEn.pdf}{Dark Energy in the T0 Model: A Mathematical Analysis of Energy Dynamics}, April 3, 2025.
	\bibitem{pascher_vereinheitlichung_2025} J. Pascher, \href{https://github.com/jpascher/T0-Time-Mass-Duality/tree/main/2/pdf/English/T0VereinheitlichungDEGalEn.pdf}{Unification of the T0 Model: Foundations, Dark Energy, and Galaxy Dynamics}, April 4, 2025.
	\bibitem{pascher_formalismen_2025} J. Pascher, \href{https://github.com/jpascher/T0-Time-Mass-Duality/tree/main/2/pdf/English/MathZeitMasseLagrangeEn.pdf}{From Time Dilation to Mass Variation: Mathematical Core Formulations of Time-Mass Duality Theory}, April 5, 2025.
	\bibitem{Planck1899} M. Planck, \textit{Über irreversible Strahlungsvorgänge}, Sitzungsberichte der Königlich Preußischen Akademie der Wissenschaften zu Berlin 5, 440-480 (1899).
	\bibitem{Dirac1928} P. A. M. Dirac, \textit{The Quantum Theory of the Electron}, Proceedings of the Royal Society of London A 117, 610-624 (1928).
	\bibitem{Einstein1905} A. Einstein, \textit{Zur Elektrodynamik bewegter Körper}, Annalen der Physik 322, 891-921 (1905).
	\bibitem{Einstein1915} A. Einstein, \textit{Die Feldgleichungen der Gravitation}, Sitzungsberichte der Königlich Preußischen Akademie der Wissenschaften, 844-847 (1915).
	\bibitem{Sommerfeld1916} A. Sommerfeld, \textit{Zur Quantentheorie der Spektrallinien}, Annalen der Physik 356, 1-94 (1916).
	\bibitem{Heisenberg1927} W. Heisenberg, \textit{Über den anschaulichen Inhalt der quantentheoretischen Kinematik und Mechanik}, Zeitschrift für Physik 43, 172-198 (1927).
	\bibitem{Schrodinger1926} E. Schrödinger, \textit{Quantisierung als Eigenwertproblem}, Annalen der Physik 384, 361-376 (1926).
	\bibitem{Feynman1985} R. P. Feynman, \textit{QED: The Strange Theory of Light and Matter}, Princeton University Press (1985).
	\bibitem{Duff2002} M. J. Duff, L. B. Okun \& G. Veneziano, \textit{Trialogue on the Number of Fundamental Constants}, Journal of High Energy Physics 3, 023 (2002).
	\bibitem{Wilczek2008} F. Wilczek, \textit{The Lightness of Being: Mass, Ether, and the Unification of Forces}, Basic Books (2008).
	\bibitem{Verlinde2011} E. Verlinde, \textit{On the Origin of Gravity and the Laws of Newton}, Journal of High Energy Physics 4, 29 (2011).
	\bibitem{Greene2020} B. Greene, \textit{Until the End of Time: Mind, Matter, and Our Search for Meaning in an Evolving Universe}, Alfred A. Knopf (2020).
	\bibitem{tHooft1993} G. 't Hooft, \textit{Dimensional Reduction in Quantum Gravity}, arXiv:gr-qc/9310026 (1993).
	\bibitem{Will2014} C. M. Will, \textit{The Confrontation between General Relativity and Experiment}, Living Reviews in Relativity 17, 4 (2014).
\end{thebibliography}

\end{document}