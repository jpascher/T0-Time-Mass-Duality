\documentclass[12pt,a4paper]{article}
\usepackage[utf8]{inputenc}
\usepackage[T1]{fontenc}
\usepackage[ngerman]{babel}
\usepackage{lmodern}
\usepackage{amsmath}
\usepackage{amssymb}
\usepackage{physics}
\usepackage{hyperref}
\usepackage{tcolorbox}
\usepackage{booktabs}
\usepackage{enumitem}
\usepackage[table,xcdraw]{xcolor}
\usepackage[left=2cm,right=2cm,top=2cm,bottom=2cm]{geometry}
\usepackage{pgfplots}
\pgfplotsset{compat=1.18}
\usepackage{graphicx}
\usepackage{float}
\usepackage{fancyhdr}
\usepackage{siunitx}

% Danksagungs-Umgebung
\newenvironment{acknowledgments}
{\section*{Danksagungen}}
{\vspace{1em}}

% Benutzerdefinierte Befehle
\newcommand{\Tfield}{T(x)}
\newcommand{\alphaEM}{\alpha_{\text{EM}}}
\newcommand{\alphaW}{\alpha_{\text{W}}}
\newcommand{\betaT}{\beta_{\text{T}}}
\newcommand{\Mpl}{M_{\text{Pl}}}
\newcommand{\Tzerot}{T_0(\Tfield)}
\newcommand{\Tzero}{T_0}
\newcommand{\vecx}{\vec{x}}
\newcommand{\vr}{\vec{r}}
\newcommand{\gammaf}{\gamma_{\text{Lorentz}}}
\newcommand{\DhiggsT}{\Tfield (\partial_\mu + ig A_\mu) \Phi + \Phi \partial_\mu \Tfield}
\newcommand{\LCDM}{\Lambda\text{CDM}}
\newcommand{\DTmu}{D_{T,\mu}}
\newcommand{\calL}{\mathcal{L}}
\newcommand{\deq}{\displaystyle}
\newcommand{\e}{\mathrm{e}}

% Kopf- und Fußzeilen-Konfiguration
\pagestyle{fancy}
\fancyhf{}
\fancyhead[L]{Johann Pascher}
\fancyhead[R]{Zeit-Masse-Dualität: Teil II}
\fancyfoot[C]{\thepage}
\renewcommand{\headrulewidth}{0.4pt}
\renewcommand{\footrulewidth}{0.4pt}

\hypersetup{
	colorlinks=true,
	linkcolor=blue,
	citecolor=blue,
	urlcolor=blue,
	pdftitle={Überbrückung von Quantenmechanik und Relativitätstheorie durch Zeit-Masse-Dualität: Teil II},
	pdfauthor={Johann Pascher},
	pdfsubject={Theoretische Physik},
	pdfkeywords={T0-Modell, natürliche Einheiten, Zeit-Masse-Dualität, Kosmologie}
}

\title{Überbrückung von Quantenmechanik und Relativitätstheorie durch Zeit-Masse-Dualität: \\ Ein einheitlicher Rahmen mit natürlichen Einheiten \(\alpha = \beta = 1\) \\ Teil II: Kosmologische Implikationen und Experimentelle Validierung}
\author{Johann Pascher\\
	Abteilung für Kommunikationstechnik\\
	Höhere Technische Bundeslehranstalt (HTL), Leonding, Österreich\\
	\texttt{johann.pascher@gmail.com}}
\date{7. April 2025}

\begin{document}
	
	\maketitle
	
	\begin{abstract}
		Dieses Papier erweitert das in Teil I eingeführte T0-Modell in die Bereiche Kosmologie und experimentelle Validierung, basierend auf einem einheitlichen System natürlicher Einheiten, in dem \(\hbar = c = G = k_B = \alphaEM = \alphaW = \betaT = 1\). Im Gegensatz zum expandierenden Universum des \(\Lambda\)CDM-Modells schlagen wir ein statisches Kosmos vor, in dem die Rotverschiebung aus dem Energieverlust von Photonen resultiert, vermittelt durch das intrinsische Zeitfeld \(\Tfield\). Dieser Rahmen interpretiert dunkle Materie und dunkle Energie durch emergente Gravitationseffekte neu und erweitert das Standardmodell (SM) mit einer konsistenten Gravitationstheorie. Zu den wichtigsten Vorhersagen gehören eine wellenlängenabhängige Rotverschiebung mit einer Variation von etwa \(2,3\%\) pro Dekade, eine Temperatur des kosmischen Mikrowellenhintergrunds (CMB) von \(24000 \, \text{K}\) bei \(z = 1100\) und spekulative Erweiterungen jenseits der Lichtgeschwindigkeit. Diese Vorhersagen sind mit Instrumenten wie dem James-Webb-Weltraumteleskop (JWST) und zukünftigen CMB-Missionen überprüfbar. Wir behandeln Messprobleme, wie frequenzabhängige Verzerrungen in der GPS-Präzision und kosmologischen Beobachtungen, die Unterschiede zwischen Massenvariation und Zeitdilatation verschleiern, und bieten eine philosophisch kohärente Alternative zum \(\Lambda\)CDM, die theoretische Eleganz mit empirischer Strenge vereint.
	\end{abstract}
	
	\section{Einleitung}
	\label{sec:introduction}
	
	In Teil I (\textit{Überbrückung von Quantenmechanik und Relativitätstheorie durch Zeit-Masse-Dualität: Teil I}, \cite{pascher_part1_2025}) haben wir das T0-Modell als einheitlichen Rahmen für Quantenmechanik (QM) und Relativitätstheorie (RT) etabliert, unter Nutzung des intrinsischen Zeitfelds \(\Tfield = \frac{\hbar}{\max(mc^2, \omega)}\) innerhalb eines Systems natürlicher Einheiten (\(\hbar = c = G = k_B = \alphaEM = \alphaW = \betaT = 1\)). Dieses System, detailliert beschrieben in Teil I, Abschnitt 2 ''Vereinheitlichung der Konstanten mit natürlichen Einheiten'' \href{https://github.com/jpascher/T0-Time-Mass-Duality/tree/main/2/pdf/Deutsch/Bridging Quantum Mechanics and Relativity through Time-Mass Duality Part I Theoretical Foundations.pdf}{[Teil I]}, eliminiert empirisch bestimmte Konstanten und erzielt Übereinstimmung mit gemessenen Werten (z. B. \(c \approx 3 \times 10^8 \, \text{m/s}\), \(\alphaEM \approx 1/137.036\)) mit Abweichungen unter \(10^{-6}\). Es ermöglichte eine massenabhängige Schrödinger-Gleichung (Teil I, Gleichung (4.5) \href{https://github.com/jpascher/T0-Time-Mass-Duality/tree/main/2/pdf/Deutsch/Bridging Quantum Mechanics and Relativity through Time-Mass Duality Part I Theoretical Foundations.pdf}{[Teil I]}) und emergente Gravitation (Teil I, Abschnitt 5 ''Emergente Gravitation'' \href{https://github.com/jpascher/T0-Time-Mass-Duality/tree/main/2/pdf/Deutsch/Bridging Quantum Mechanics and Relativity through Time-Mass Duality Part I Theoretical Foundations.pdf}{[Teil I]}), die mikro- und makroskopische Skalen verbindet.
	
	Teil II erweitert diese Grundlagen in die Kosmologie und experimentelle Validierung, im Gegensatz zum \(\Lambda\)CDM-Modell eines expandierenden Universums, das vor etwa 13,8 Milliarden Jahren mit einem Urknall begann \cite{Planck2020}. Im \(\Lambda\)CDM ist die kosmische Rotverschiebung ein kinematischer Effekt (\(z \approx H_0 d / c\)), der Inflation und dunkle Energie erfordert \cite{Riess1998,Perlmutter1999}. Das T0-Modell schlägt ein statisches, unendliches und ewiges Universum vor, in dem die Rotverschiebung aus dem Energieverlust von Photonen durch \(\Tfield\) resultiert, das Standardmodell (SM) mit einer konsistenten Gravitationstheorie erweitert und dabei seinen Kern der Teilchenphysik beibehält.
	
	Wichtige Vorhersagen umfassen:
	- Wellenlängenabhängige Rotverschiebung (\(\sim 2,3\%\) pro Dekade),
	- CMB-Temperatur von \(24000 \, \text{K}\) bei \(z = 1100\),
	- Spekulative überlichtschnelle Erweiterungen.
	
	Diese sind mit JWST-Spektroskopie und CMB-Verzerrungsmessungen überprüfbar, obwohl frequenzbasierte Methoden (z. B. GPS, Rotverschiebung) Massenvariation und Zeitdilatation vermischen, was eine sorgfältige Neubewertung erfordert \cite{pascher_quantum_2025}. Philosophisch vermeidet T0 Singularitäten und bietet ein kohärentes ewiges Kosmos \cite{pascher_perspective_2025}.
	
	Dieses Papier ist wie folgt strukturiert:
	- Abschnitt 2: Statisches Universum und Rotverschiebungsmechanismus.
	- Abschnitt 3: Kosmologische Phänomene und Vorhersagen.
	- Abschnitt 4: Quantitative Vorhersagen.
	- Abschnitt 5: Experimentelle Tests und Messprobleme.
	- Abschnitt 6: Implikationen von \(\betaT = 1\).
	- Abschnitt 7: Integration in die Zeit-Masse-Dualitätstheorie.
	- Abschnitt 8: Spekulative Erweiterungen und Philosophie.
	
	\section{Statisches Universumsmodell}
	\label{sec:static_universe}
	
	\subsection{Konzept eines statischen Universums}
	\label{subsec:static_concept}
	
	Das T0-Modell stellt sich ein statisches Universum vor, unendlich im Raum und ewig in der Zeit, im Gegensatz zum expandierenden Kosmos des \(\Lambda\)CDM-Modells mit einem Urknall. Im \(\Lambda\)CDM spiegelt die Rotverschiebung (\(z \approx H_0 d / c\)) die Expansion wider (\(H_0 \approx 70 \, \text{km/s/Mpc}\)) \cite{Planck2020}, die Inflation für Gleichförmigkeit und dunkle Energie für Beschleunigung (\(\Omega_{\Lambda} \approx 0.7\)) erfordert \cite{Riess1998}. T0 eliminiert diese und postuliert ein stabiles Kosmos, in dem \(\Tfield\) die Dynamik ohne Expansion steuert.
	
	Vorteile umfassen:
	- **Horizontproblem:** Unendliche Zeit gewährleistet thermisches Gleichgewicht über alle Skalen hinweg \cite{pascher_messdifferenzen_2025}.
	- **Flachheit:** Keine Expansion beseitigt die Notwendigkeit einer Krümmungsanpassung.
	- **Singularitätsfrei:** Ewige Existenz vermeidet unendliche Dichten \cite{pascher_perspective_2025}.
	
	Dies ergänzt die Teilchenphysik des SM mit einem statischen Gravitationsrahmen, der in Teil I, Abschnitt 5 ''Emergente Gravitation'' \href{https://github.com/jpascher/T0-Time-Mass-Duality/tree/main/2/pdf/Deutsch/Bridging Quantum Mechanics and Relativity through Time-Mass Duality Part I Theoretical Foundations.pdf}{[Teil I]} abgeleitet wurde.
	
	\subsection{Rotverschiebung durch Energieverlust}
	\label{subsec:redshift_energy_loss}
	
	Die Rotverschiebung im T0-Modell ist:
	\begin{equation}
		1 + z = e^{\alpha d},
		\label{eq:redshift_distance}
	\end{equation}
	mit \(\alpha = H_0 / c \approx 2.3 \times 10^{-18} \, \text{m}^{-1}\) (SI) oder 1 (natürliche Einheiten). Bei niedrigem \(z\):
	\begin{equation}
		z \approx \alpha d,
		\label{eq:hubble_approx}
	\end{equation}
	was lokal mit \(\Lambda\)CDM übereinstimmt. Der Mechanismus ist der Energieverlust von Photonen:
	\begin{equation}
		\frac{dE}{dx} = -\alpha E,
		\label{eq:energy_loss_rate}
	\end{equation}
	woraus \(E = E_0 e^{-\alpha d}\) folgt, und somit \(1 + z = e^{\alpha d}\), abgeleitet aus den Eigenschaften von \(\Tfield\) in Teil I, Abschnitt 3.1 ''Definition und physikalische Grundlage'' \href{https://github.com/jpascher/T0-Time-Mass-Duality/tree/main/2/pdf/Deutsch/Bridging Quantum Mechanics and Relativity through Time-Mass Duality Part I Theoretical Foundations.pdf}{[Teil I]} \cite{pascher_messdifferenzen_2025}.
	
	\section{Kosmologische Phänomene}
	\label{sec:cosmological_phenomena}
	
	\subsection{Temperatur-Rotverschiebungs-Beziehung und CMB}
	\label{subsec:cmb_temp}
	
	Die Beziehung im \(\Lambda\)CDM, \(T(z) = T_0 (1 + z)\), ergibt \(T \approx 3000 \, \text{K}\) bei \(z = 1100\) (\(T_0 = 2.725 \, \text{K}\)) \cite{Fixsen2009}. T0 sagt voraus:
	\begin{equation}
		T(z) = T_0 (1 + z) (1 + \ln(1 + z)),
		\label{eq:temperature_redshift_simplified}
	\end{equation}
	also \(T(1100) \approx 24000 \, \text{K}\), was einen verstärkten Energieverlust widerspiegelt (Gleichung \ref{eq:energy_loss_rate}) \cite{pascher_temp_2025}. Dies beeinflusst Nukleosynthese und Rekombination und ist durch CMB-Verzerrungen überprüfbar (Abschnitt 5.2).
	
	\subsection{Wellenlängenabhängige Rotverschiebung}
	\label{subsec:wavelength_redshift}
	
	T0 sagt voraus:
	\begin{equation}
		z(\lambda) = z_0 \left(1 + \ln\left(\frac{\lambda}{\lambda_0}\right)\right),
		\label{eq:wavelength_redshift}
	\end{equation}
	mit \(\Delta z / z_0 \approx 3,85\%\) über 0,6-28 \(\mu\text{m}\) (JWST-Bereich), aufgrund von:
	\begin{equation}
		\frac{dE}{dx} = -\alpha E \left(1 + \ln\left(\frac{\lambda}{\lambda_0}\right)\right),
		\label{eq:wavelength_energy_loss}
	\end{equation}
	im Gegensatz zur Uniformität des \(\Lambda\)CDM \cite{pascher_params_2025}.
	
	\subsection{Neuinterpretation von dunkler Materie und dunkler Energie}
	\label{subsec:dark_reinterpretation}
	
	Das Potential:
	\begin{equation}
		\Phi(r) = -\frac{M}{r} + \kappa r,
		\label{eq:grav_potential_t0}
	\end{equation}
	(\(\kappa \approx 4.8 \times 10^{-11} \, \text{m/s}^2\)) interpretiert neu:
	- **Dunkle Materie:** \(v(r) = \sqrt{\frac{M}{r} + \kappa r}\), wie in Teil I, Abschnitt 5.1 ''Ableitung aus \(\Tfield\)'' \href{https://github.com/jpascher/T0-Time-Mass-Duality/tree/main/2/pdf/Deutsch/Bridging Quantum Mechanics and Relativity through Time-Mass Duality Part I Theoretical Foundations.pdf}{[Teil I]} abgeleitet.
	- **Dunkle Energie:** \(\rho_{\text{DE}} \approx \frac{\kappa}{r^2}\) \cite{pascher_galaxies_2025}.
	
	\subsection{Einfluss auf Galaxiendynamik}
	\label{subsec:galaxy_dynamics}
	
	Das T0-Modell prägt die Galaxiendynamik durch \(\Tfield\) und bietet eine Alternative zum \(\Lambda\)CDM, indem es Gravitationseffekte ohne dunkle Materie oder Expansion neu interpretiert.
	
	\subsubsection{Rotationskurven}
	Das Potential (Gleichung \ref{eq:grav_potential_t0}) ergibt:
	\begin{equation}
		v(r) = \sqrt{\frac{M}{r} + \kappa r},
		\label{eq:rotation_velocity}
	\end{equation}
	was flache Rotationskurven reproduziert (z. B. Milchstraße: \(v(30 \, \text{kpc}) \approx 211 \, \text{km/s}\)) \cite{pascher_galaxies_2025}.
	
	\begin{figure}[h]
		\centering
		\begin{tikzpicture}
			\begin{axis}[
				xlabel={Radius [kpc]},
				ylabel={Rotationsgeschwindigkeit [km/s]},
				xlabel style={font=\large},
				ylabel style={font=\large},
				tick label style={font=\normalsize},
				xmin=0, xmax=30,
				ymin=0, ymax=300,
				legend pos=south east,
				legend style={font=\large},
				grid=both,
				minor tick num=4,
				major grid style={line width=0.8pt, gray!50},
				minor grid style={line width=0.4pt, gray!20}
				]
				\addplot[blue, ultra thick, domain=0.1:30, samples=100] {220*sqrt(10/x)};
				\addplot[red, dashed, ultra thick, domain=0.1:30, samples=100] {sqrt(220^2*10/x + 4.8*x^2)};
				\legend{Newtonsche Vorhersage, T0-Modell}
			\end{axis}
		\end{tikzpicture}
		\caption{Rotationskurven im Vergleich zwischen Newton (blau) und T0-Modell (rot) für eine Galaxie mit \(M = 10^{11} M_{\odot}\), \(\kappa_{\text{SI}} = 4.8 \times 10^{-11} \, \text{m/s}^2\). Das T0-Modell erzeugt ein flaches Profil bei großen Radien, konsistent mit Beobachtungen wie der Milchstraße.}
		\label{fig:rotation_curves}
	\end{figure}
	
	\subsubsection{Galaxienbildung und -entwicklung}
	Im \(\Lambda\)CDM hängt die Galaxienbildung vom Gravitationskollaps dunkler Materie-Halos ab, die durch primordiale Fluktuationen und kosmische Expansion über 13,8 Milliarden Jahre verstärkt werden \cite{Planck2020}. Das T0-Modell mit seinem statischen, unendlich alten Universum schlägt einen anderen Mechanismus vor: Galaxien bilden und entwickeln sich durch die allmähliche Aggregation baryonischer Materie unter dem Einfluss der \(\Tfield\)-vermittelten Gravitation, ohne dunkle Materie oder zeitlichen Ursprung. Die unendliche Zeitskala ermöglicht langsame, stetige Prozesse wie Gaskühlung, Sternentstehung und dynamische Relaxation, allein getrieben durch die beobachtbare Masse und das emergente Potential (Gleichung \ref{eq:grav_potential_t0}).
	
	Dieser Prozess erweitert die baryonische Dynamik des SM, indem er einen Gravitationsrahmen bietet, der konsistent über alle Skalen funktioniert. Der \(\kappa r\)-Term führt beispielsweise eine stabilisierende Kraft auf große Reichweite ein, die übermäßige Dispersion von Gaswolken verhindert und die Bildung von Spiralarmen und galaktischen Scheiben über längere Zeiträume erleichtert. Dies steht im Gegensatz zur Abhängigkeit des \(\Lambda\)CDM von dunkler Materie, um die notwendigen Gravitationssenken bereitzustellen, die T0 durch einen rein baryonischen, \(\Tfield\)-getriebenen Mechanismus ersetzt \cite{pascher_galaxies_2025}.
	
	\subsubsection{Cluster-Dynamik und großräumige Struktur}
	Auf größeren Skalen, wie bei Galaxienhaufen (z. B. Bullet Cluster), beruft sich \(\Lambda\)CDM auf dunkle Materie, um Massediskrepanzen zu erklären, die aus Gravitationslinsen und Geschwindigkeitsdispersionen abgeleitet werden \cite{McGaugh2016}. Im T0-Modell modifiziert der \(\kappa r\)-Term die Gravitationswechselwirkungen und reduziert diese Diskrepanzen:
	\begin{equation}
		v_{\text{cluster}}(r) = \sqrt{\frac{M_{\text{total}}}{r} + \kappa r},
		\label{eq:cluster_velocity}
	\end{equation}
	wobei \(M_{\text{total}}\) die gesamte baryonische Masse ist. Für einen Haufen wie den Bullet Cluster (\(M_{\text{total}} \approx 10^{14} M_{\odot}\), \(r \approx 1 \, \text{Mpc}\)) trägt der zusätzliche \(\kappa r\)-Term bei großen Radien erheblich bei, stimmt Linsen- und dynamische Massenschätzungen ohne dunkle Materie ab. Numerische Simulationen deuten darauf hin, dass Abweichungen von \(\Lambda\)CDM-Vorhersagen innerhalb der Beobachtungsfehler liegen (\(\sim 5-10\%\)), überprüfbar mit präzisen Linsensurveys \cite{pascher_emergente_2025}.
	
	Die großräumige Struktur im T0-Modell entsteht aus der unendlich langen Entwicklung baryonischer Materie unter \(\Tfield\), wobei Filamente und Wände natürlich ohne expansionsgetriebenes Wachstum gebildet werden. Dies stimmt mit den Teilchenwechselwirkungen des SM überein, erweitert durch die Gravitationstheorie des T0-Modells (Teil I, Abschnitt 5 ''Emergente Gravitation'' \href{https://github.com/jpascher/T0-Time-Mass-Duality/tree/main/2/pdf/Deutsch/Bridging Quantum Mechanics and Relativity through Time-Mass Duality Part I Theoretical Foundations.pdf}{[Teil I]}).
	
	\begin{table}[ht]
		\centering
		\caption{Vergleich der Vorhersagen von \(\Lambda\)CDM und T0-Modell für Galaxiendynamik}
		\label{tab:galaxy_dynamics_comparison}
		\scalebox{0.8}{
			\begin{tabular}{lcc}
				\hline
				\textbf{Phänomen} & \textbf{\(\Lambda\)CDM} & \textbf{T0-Modell} \\
				\hline
				Rotationskurve & Halo aus dunkler Materie (\(v \propto r^0\)) & \(\kappa r\)-Term (\(v \propto \sqrt{\kappa r}\)) \\
				Galaxienbildung & Kollaps dunkler Materie, 13,8 Mrd. Jahre & Baryonische Aggregation, unendliche Zeit \\
				Clustermasse & Dominiert von dunkler Materie & Baryonisch + \(\Tfield\)-Effekte \\
				Großräumige Struktur & Expansion + Fluktuationen & Statisch \(\Tfield\)-getrieben \\
				\hline
			\end{tabular}
		}
	\end{table}
	
	\section{Quantitative Vorhersagen}
	\label{sec:predictions}
	
	\subsection{CMB-Temperaturvorhersage}
	\label{subsec:cmb_temp_prediction}
	
	T0 sagt eine CMB-Temperatur bei \(z = 1100\) von:
	\begin{equation}
		T(1100) \approx 24000 \, \text{K},
		\label{eq:cmb_temp_t0}
	\end{equation}
	im Vergleich zu \(\Lambda\)CDM’s \(3000 \, \text{K}\), ein Faktor-8-Unterschied aufgrund der logarithmischen Verstärkung durch \(\Tfield\) (Gleichung \ref{eq:temperature_redshift_simplified}), voraus.
	
	\subsection{Variation der wellenlängenabhängigen Rotverschiebung}
	\label{subsec:wavelength_redshift_prediction}
	
	Über den JWST-Bereich (0,6-28 \(\mu\text{m}\)):
	\begin{equation}
		\Delta z / z_0 \approx 3,85\%,
		\label{eq:wavelength_variation}
	\end{equation}
	oder \(2,3\%\) pro Dekade, ein direkter Test von \(\betaT = 1\) (Gleichung \ref{eq:wavelength_redshift}).
	
	\subsection{Galaxienrotationsgeschwindigkeiten}
	\label{subsec:rotation_velocity_prediction}
	
	Für eine Galaxie wie die Milchstraße:
	\begin{equation}
		v(r) = \sqrt{\frac{M}{r} + \kappa r},
		\label{eq:rotation_velocity_repeat}
	\end{equation}
	z. B. \(v(30 \, \text{kpc}) \approx 211 \, \text{km/s}\), konsistent mit beobachteten flachen Kurven \cite{McGaugh2016}.
	
	\begin{figure}[ht]
		\centering
		\begin{tikzpicture}
			\begin{axis}[
				xlabel={Rotverschiebung \(z\)},
				ylabel={Distanzmodul \(\mu = m - M\)},
				xmin=0,
				xmax=2,
				ymin=30,
				ymax=50,
				legend pos=north west,
				grid=both,
				width=\textwidth,
				height=6cm,
				samples=100
				]
				\addplot[blue, thick, domain=0.01:2] {5*log10(3e8/70e3*ln(1+x)*(1+x)*0.1) + 25};
				\addplot[red, dashed, domain=0.01:2] {5*log10(3e8/70e3*(1+x)*(2-(1/(1+x)))*1) + 25};
				\legend{T0-Modell, \(\Lambda\)CDM (\(\Omega_m=0.3\), \(\Omega_{\Lambda}=0.7\))}
			\end{axis}
		\end{tikzpicture}
		\caption{Distanzmodul vs. Rotverschiebung im Vergleich der Vorhersagen von T0 (blau) und \(\Lambda\)CDM (rot), mit \(H_0 = 70 \, \text{km/s/Mpc}\), zeigt unterschiedliche Verhaltensweisen, die mit Supernovae-Daten überprüfbar sind.}
		\label{fig:distance_modulus}
	\end{figure}
	
	\section{Experimentelle Tests}
	\label{sec:tests}
	
	\subsection{JWST-Spektroskopie}
	\label{subsec:jwst_test}
	
	Die vorhergesagte \(\Delta z / z \approx 3,85\%\) bei \(z = 7\) (Gleichung \ref{eq:wavelength_variation}) ist mit der \(0,1\%\)-Präzision des JWST nachweisbar und testet \(\betaT\)-induzierte Variationen über Quasar-Emissionslinien \cite{pascher_params_2025}. Eine Stichprobe hoch-\(z\)-Quasare könnte diesen Effekt bestätigen und T0 von der einheitlichen Rotverschiebung des \(\Lambda\)CDM unterscheiden.
	
	\subsection{CMB-Verzerrungen}
	\label{subsec:cmb_distortions_test}
	
	T0 sagt voraus:
	\begin{equation}
		\mu \approx 1.4 \times 10^{-5}, \quad y \approx 1.6 \times 10^{-6},
		\label{eq:distortion_parameters}
	\end{equation}
	im Vergleich zu \(\Lambda\)CDM’s \(\mu \approx 2 \times 10^{-8}\), \(y \approx 4 \times 10^{-9}\), messbar mit der Empfindlichkeit von PIXIE (\(\sim 10^{-8}\)) \cite{pascher_temp_2025}. Dies testet die Hypothese eines heißeren CMB (Abschnitt 3.1).
	
	\subsection{Messproblem: GPS und Uhrenpräzision}
	\label{subsec:gps_clock_problem}
	
	GPS-Uhren (Cäsium, \(9,19 \, \text{GHz}\)) zeigen eine relativistische Verschiebung von \(\Delta t \approx 38 \, \mu\text{s/Tag}\), traditionell der Zeitdilatation in der GR zugeschrieben. T0 interpretiert dies als Massenvariation (\(m = \gamma m_0\)), da beide die Frequenz (\(f = \frac{c}{\lambda}\)) identisch beeinflussen. Aktuelle frequenzbasierte Methoden können diese nicht unterscheiden, da sie Oszillationsraten statt absoluter Zeit messen, was eine Herausforderung darstellt, T0 lokal von GR zu differenzieren \cite{pascher_quantum_2025}. Alternative Ansätze, wie direkte Tests der Massenvariation (z. B. Vergleich von Teilchenzerfallsraten in Gravitationsfeldern), könnten diese Mehrdeutigkeit lösen, sind jedoch mit der aktuellen Technologie noch nicht machbar.
	
	\subsection{Messproblem: Kosmologische Beobachtungen}
	\label{subsec:cosmological_measurement_problem}
	
	Die kosmische Rotverschiebung (\(z = \frac{\Delta \lambda}{\lambda_0}\)) wird im \(\Lambda\)CDM als Doppler-Effekt interpretiert, während T0 sie dem Energieverlust zuschreibt (Gleichung \ref{eq:redshift_distance}). Frequenzbasierte Spektroskopie, die auf Verschiebungen der Photonenwellenlänge beruht, kann \(\Tfield\)-Variationen nicht von Expansionseffekten isolieren, da beide sich ähnlich in beobachteten Spektren manifestieren. Diese methodische Einschränkung bevorzugt Interpretationen zugunsten des \(\Lambda\)CDM und erfordert nicht-frequenzbasierte Metriken – wie radioaktive Zerfallsraten über kosmische Distanzen –, die derzeit nicht verfügbar sind, aber einen definitiven Test bieten könnten \cite{pascher_alphabeta_2025}. Zum Beispiel könnte der Vergleich von Zerfallsraten von Isotopen in fernen Supernovae Massenvariation unabhängig von Frequenzverschiebungen aufzeigen.
	
	\subsection{Neuinterpretation von Messungen}
	\label{subsec:reassessment_measurements}
	
	Das Setzen von \(\betaT = 1\) in natürlichen Einheiten (Teil I, Abschnitt 2.2 ''Definition des einheitlichen Systems natürlicher Einheiten'' \href{https://github.com/jpascher/T0-Time-Mass-Duality/tree/main/2/pdf/Deutsch/Bridging Quantum Mechanics and Relativity through Time-Mass Duality Part I Theoretical Foundations.pdf}{[Teil I]}) führt zu scheinbaren Diskrepanzen mit SI-Werten (z. B. \(\betaT^{\text{SI}} \approx 0.008\)), die Verzerrungen in den Kalibrierungsmethoden des \(\Lambda\)CDM widerspiegeln, die Expansion annehmen. Eine Neuinterpretation kosmologischer Daten durch den statischen Rahmen des T0-Modells könnte Spannungen lösen, wie die Diskrepanz des Hubble-Parameters (\(H_0\) von 67 bis 73 km/s/Mpc) \cite{DiValentino2021}, indem Variationen dem Energieverlust statt Expansionsraten zugeschrieben werden. Diese Neubewertung erfordert eine Neuanalyse bestehender Datensätze (z. B. Supernovae, CMB) mit den Vorhersagen des T0-Modells, eine Aufgabe, die durch die mathematische Einfachheit des Modells (Gleichungen \ref{eq:redshift_distance}, \ref{eq:temperature_redshift_simplified}) erleichtert wird \cite{pascher_alphabeta_2025}.
	
	\section{Folgen des Setzens von \(\beta = 1\)}
	\label{sec:consequences_beta}
	
	\subsection{Theoretische Eleganz}
	\label{subsec:theoretical_elegance}
	
	Das Setzen von \(\betaT = 1\) in natürlichen Einheiten (Teil I, Abschnitt 2.2 ''Definition des einheitlichen Systems natürlicher Einheiten'' \href{https://github.com/jpascher/T0-Time-Mass-Duality/tree/main/2/pdf/Deutsch/Bridging Quantum Mechanics and Relativity through Time-Mass Duality Part I Theoretical Foundations.pdf}{[Teil I]}) erhöht die theoretische Kohärenz des T0-Modells:
	- **Vereinheitlichte Zeit-Energie-Beziehung:** Die Temperatur-Rotverschiebungs-Beziehung (Gleichung \ref{eq:temperature_redshift_simplified}) integriert logarithmischen Energieverlust nahtlos mit linearen Rotverschiebungstermen und reflektiert ein konsistentes physikalisches Prinzip über Skalen hinweg.
	- **Dimensionslose Konsistenz:** Alle fundamentalen Konstanten (\(\hbar, c, G, k_B, \alphaEM, \alphaW, \betaT\)) sind auf 1 vereinheitlicht, wodurch willkürliche Skalierungsfaktoren eliminiert und mit dem energie-zentrierten Rahmen übereinstimmen, der in Teil I, Abschnitt 2 ''Vereinheitlichung der Konstanten mit natürlichen Einheiten'' \href{https://github.com/jpascher/T0-Time-Mass-Duality/tree/main/2/pdf/Deutsch/Bridging Quantum Mechanics and Relativity through Time-Mass Duality Part I Theoretical Foundations.pdf}{[Teil I]} etabliert wurde.
	- **Vereinfachung der Wechselwirkungen:** Die Kopplung von \(\Tfield\) an Materie und Felder (Teil I, Abschnitt 4.1 ''Lagrange-Dichten'' \href{https://github.com/jpascher/T0-Time-Mass-Duality/tree/main/2/pdf/Deutsch/Bridging Quantum Mechanics and Relativity through Time-Mass Duality Part I Theoretical Foundations.pdf}{[Teil I]}) wird parameterfrei, was die Eleganz des SM durch Reduzierung der Komplexität in gravitativen und quantenmechanischen Wechselwirkungen erhöht \cite{pascher_alphabeta_2025}.
	
	Diese Eleganz steht im Gegensatz zur Abhängigkeit des \(\Lambda\)CDM von mehreren freien Parametern (z. B. \(\Omega_m, \Omega_{\Lambda}\)) und bietet eine vereinheitlichere theoretische Struktur.
	
	\subsection{Umrechnung in SI-Einheiten}
	\label{subsec:conversion_si}
	
	In natürlichen Einheiten ist \(\betaT = 1\), aber die Umrechnung in SI-Einheiten erfordert:
	\begin{equation}
		\betaT^{\text{SI}} = \betaT^{\text{nat}} \cdot \frac{\xi \cdot l_{P,\text{SI}}}{r_{0,\text{SI}}},
		\label{eq:beta_conversion}
	\end{equation}
	wobei \(\xi \approx 1.33 \times 10^{-4}\) (Teil I, Abschnitt 2.3 ''Längenskalen und entsprechende Konstanten'' \href{https://github.com/jpascher/T0-Time-Mass-Duality/tree/main/2/pdf/Deutsch/Bridging Quantum Mechanics and Relativity through Time-Mass Duality Part I Theoretical Foundations.pdf}{[Teil I]}), \(l_{P,\text{SI}} = 1.616 \times 10^{-35} \, \text{m}\) und \(r_{0,\text{SI}}\) die charakteristische T0-Längenskala ist, etwa \(10^{-39} \, \text{m}\), basierend auf Higgs-Parametern \cite{pascher_alphabeta_2025}. Dies ergibt \(\betaT^{\text{SI}} \approx 0.008\), konsistent mit empirischen Schätzungen, und überbrückt theoretische Reinheit mit experimenteller Anwendbarkeit, analog zu \(c = 1\), das in SI-Einheiten \(3 \times 10^8 \, \text{m/s}\) entspricht. Diese Umrechnung stellt sicher, dass die Vorhersagen des T0-Modells (z. B. Gleichung \ref{eq:wavelength_redshift}) mit beobachtbaren Daten übereinstimmen, während seine grundlegende Eleganz erhalten bleibt.
	
	\section{Integration in die Zeit-Masse-Dualitätstheorie}
	\label{sec:integration_t0}
	
	\subsection{Konsistenz mit Grundprinzipien}
	\label{subsec:consistency_principles}
	
	Die Wahl von \(\betaT = 1\) fügt sich nahtlos in die grundlegenden Axiome des T0-Modells ein, die in Teil I, Abschnitt 3 ''Intrinsisches Zeitfeld \(\Tfield\)'' \href{https://github.com/jpascher/T0-Time-Mass-Duality/tree/main/2/pdf/Deutsch/Bridging Quantum Mechanics and Relativity through Time-Mass Duality Part I Theoretical Foundations.pdf}{[Teil I]} etabliert wurden:
	- **Absolute Zeit:** \(\Tfield\) definiert eine universelle, intrinsische Zeitskala für alle Teilchen, unabhängig von der Bewegung des Beobachters, im Gegensatz zur relativen Zeit der GR (Teil I, Abschnitt 3.3 ''Physikalische Interpretation'' \href{https://github.com/jpascher/T0-Time-Mass-Duality/tree/main/2/pdf/Deutsch/Bridging Quantum Mechanics and Relativity through Time-Mass Duality Part I Theoretical Foundations.pdf}{[Teil I]}).
	- **Massenvariation:** Masse wird dynamisch als \(m = \frac{\hbar}{\Tfield c^2}\) bestimmt, vermittelt durch den Higgs-Mechanismus (Teil I, Abschnitt 4.1 ''Lagrange-Dichten'' \href{https://github.com/jpascher/T0-Time-Mass-Duality/tree/main/2/pdf/Deutsch/Bridging Quantum Mechanics and Relativity through Time-Mass Duality Part I Theoretical Foundations.pdf}{[Teil I]}), und bietet eine quanten-relativistische Verbindung, die mit der Teilchenphysik des SM konsistent ist.
	- **Emergente Gravitation:** Gravitationseffekte entstehen aus \(\Tfield\)-Gradienten (Teil I, Abschnitt 5.1 ''Ableitung aus \(\Tfield\)'' \href{https://github.com/jpascher/T0-Time-Mass-Duality/tree/main/2/pdf/Deutsch/Bridging Quantum Mechanics and Relativity through Time-Mass Duality Part I Theoretical Foundations.pdf}{[Teil I]}), vereinheitlichen QM und RT ohne Raumzeitkrümmung, wie durch Galaxiendynamik validiert (Abschnitt 3.4) \cite{pascher_lagrange_2025}.
	
	Diese Integration erweitert das SM, indem die geometrische Gravitation der GR durch einen feldtheoretischen Ansatz ersetzt wird, der Konsistenz über mikro- und makroskopische Phänomene hinweg bewahrt, von der Quantenverschränkung (Teil I, Abschnitt 4.2 ''Erweiterung der Quantenmechanik'' \href{https://github.com/jpascher/T0-Time-Mass-Duality/tree/main/2/pdf/Deutsch/Bridging Quantum Mechanics and Relativity through Time-Mass Duality Part I Theoretical Foundations.pdf}{[Teil I]}) bis hin zu kosmologischen Skalen (Abschnitt 2).
	
	\section{Jenseits der Grenzen}
	\label{sec:beyond_limits}
	
	\subsection{Spekulative Erweiterungen}
	\label{subsec:speculative_extensions}
	
	Die Definition von \(\Tfield = \frac{\hbar}{m c^2}\) im T0-Modell (Teil I, Abschnitt 3.1 ''Definition und physikalische Grundlage'' \href{https://github.com/jpascher/T0-Time-Mass-Duality/tree/main/2/pdf/Deutsch/Bridging Quantum Mechanics and Relativity through Time-Mass Duality Part I Theoretical Foundations.pdf}{[Teil I]}) deutet auf faszinierende Möglichkeiten jenseits konventioneller Grenzen hin, insbesondere nahe der Planck-Skala (\(m_P \approx 2.176 \times 10^{-8} \, \text{kg}\), \(t_P \approx 5.391 \times 10^{-44} \, \text{s}\)). Für Massen unter \(m_P\) gilt \(\Tfield > t_P\), was langsamere intrinsische Dynamiken impliziert:
	\begin{equation}
		T = \frac{\hbar}{m c^2} \propto \frac{1}{m},
		\label{eq:intrinsic_time_repeat}
	\end{equation}
	wobei \(m < m_P\) zu \(T > t_P\) führt. Diese Verlangsamung könnte Zustände nahe Singularitäten stabilisieren und auf endliche physikalische Bedingungen statt unendlicher Dichten der GR (z. B. im Inneren von Schwarzen Löchern) hindeuten.
	
	\begin{figure}[h]
		\centering
		\begin{tikzpicture}
			\draw[->] (0,0) -- (6,0) node[right] {Masse \(m\)};
			\draw[->] (0,0) -- (0,4) node[above] {Intrinsische Zeit \(T\)};
			\draw[scale=0.5, domain=0.1:10, smooth, variable=\x, blue, thick] plot ({\x}, {1/\x});
			\draw[dotted, red] (1.5,0) -- (1.5,1.5) -- (0,1.5);
			\node at (1.5,-0.3) {\(m_P\)};
			\node at (-0.3,1.5) {\(t_P\)};
			\node[blue] at (4.5,2) {\(T = \frac{\hbar}{m c^2}\)};
		\end{tikzpicture}
		\caption{Masse vs. intrinsische Zeit, zeigt, wie \(\Tfield\) mit abnehmender Masse unter der Planck-Skala (\(m_P\)) zunimmt und potenziell Dynamiken nahe Singularitäten stabilisiert.}
		\label{fig:mass_time}
	\end{figure}
	
	Diese spekulative Erweiterung stellt die Lichtgeschwindigkeitsbarriere (\(c = 1\)) infrage, indem sie \(\Tfield\) erlaubt, das Teilchenverhalten jenseits konventioneller Einschränkungen zu beeinflussen, eine Hypothese, die mit zukünftigen Hochenergieexperimenten oder astrophysikalischen Beobachtungen unter extremen Bedingungen (z. B. Ereignishorizonte Schwarzer Löcher) überprüfbar ist \cite{pascher_planck_2025}.
	
	\subsection{Philosophische Reflexionen}
	\label{subsec:philosophical_reflections}
	
	Das statische, ewige Kosmos des T0-Modells weicht grundlegend vom endlichen, expandierenden Universum des \(\Lambda\)CDM ab und bietet tiefgreifende philosophische Implikationen. Indem Singularitäten und unendliche Dichten vermieden werden, präsentiert T0 eine vereinheitlichte Realität, in der Zeit eine intrinsische Eigenschaft (\(\Tfield\)) ist statt einer relativistischen Variable, und Masse sich dynamisch an lokale Bedingungen anpasst. Dies steht im Gegensatz zur fragmentierten Ontologie des \(\Lambda\)CDM – mit einem Urknall-Ursprung, dunklen Komponenten und einem ungewissen Schicksal – und schlägt einen kohärenten, unendlichen Rahmen vor, der mit intuitiven Vorstellungen von Existenz ohne Anfang oder Ende übereinstimmt.
	
	Die Eliminierung von Expansion und dunklen Entitäten vereinfacht die Kosmologie, während die empirische Konsistenz erhalten bleibt (Abschnitte 4, 5), was darauf hindeutet, dass die scheinbare Komplexität des Universums aus Fehlinterpretationen frequenzbasierter Messungen stammen könnte (Abschnitt 5.4). Philosophisch gesehen schwingt T0 mit einer holistischen Sicht der Natur mit, in der quantenmechanische und relativistische Phänomene aus einem einzigen Prinzip – der Zeit-Masse-Dualität – hervorgehen und die Erklärungskraft des SM über alle Skalen hinweg erhöht \cite{pascher_perspective_2025}.
	
	\section{Schlussfolgerung}
	\label{sec:conclusion}
	
	Teil II zeigt, dass das T0-Modell das SM mit einer statischen, testbaren Kosmologie erweitert, die Rotverschiebung, dunkle Materie und dunkle Energie durch \(\Tfield\)-vermittelte Effekte neu interpretiert. Seine Vorhersagen – wellenlängenabhängige Rotverschiebung, ein heißerer CMB und Galaxiendynamik ohne dunkle Materie – bieten empirische Wege, es vom \(\Lambda\)CDM zu unterscheiden, während seine philosophische Kohärenz eine überzeugende Alternative zum Standardparadigma darstellt. Zukünftige Arbeiten werden experimentelle Tests verfeinern und spekulative Erweiterungen erforschen, um die Rolle des T0-Modells als vereinheitlicher Rahmen, der QM und RT verbindet, zu festigen \cite{pascher_perspective_2025}.
	
	\begin{acknowledgments}
		Dank an Reinsprecht Martin Dipl.-Ing. Dr. für kritisches Feedback.
	\end{acknowledgments}
	
	\bibliographystyle{apsrev4-2}
	\begin{thebibliography}{99}
		\bibitem{pascher_part1_2025} J. Pascher, \href{https://github.com/jpascher/T0-Time-Mass-Duality/tree/main/2/pdf/Deutsch/Bridging Quantum Mechanics and Relativity through Time-Mass Duality Part I Theoretical Foundations.pdf}{Überbrückung von Quantenmechanik und Relativitätstheorie durch Zeit-Masse-Dualität: Ein einheitlicher Rahmen mit natürlichen Einheiten \(\alpha = \beta = 1\) Teil I: Theoretische Grundlagen}, 7. April 2025.
		\bibitem{pascher_lagrange_2025} J. Pascher, \href{https://github.com/jpascher/T0-Time-Mass-Duality/tree/main/2/pdf/Deutsch/Mathematische Formulierungen der Zeit-Masse-Dualitätstheorie mit Lagrange.pdf}{Von Zeitdilatation zu Massenvariation: Mathematische Kernformulierungen der Zeit-Masse-Dualitätstheorie}, 29. März 2025.
		\bibitem{pascher_messdifferenzen_2025} J. Pascher, \href{https://github.com/jpascher/T0-Time-Mass-Duality/tree/main/2/pdf/Deutsch/Analyse der Messdifferenzen zwischen dem T0-Modell und dem Standardmodell.pdf}{Kompensatorische und additive Effekte: Eine Analyse der Messdifferenzen zwischen dem T0-Modell und dem \(\Lambda\)CDM-Standardmodell}, 2. April 2025.
		\bibitem{pascher_temp_2025} J. Pascher, \href{https://github.com/jpascher/T0-Time-Mass-Duality/tree/main/2/pdf/Deutsch/Anpassung von Temperatureinheiten in natürlichen Einheiten und CMB-Messungen.pdf}{Anpassung von Temperatureinheiten in natürlichen Einheiten und CMB-Messungen}, 2. April 2025.
		\bibitem{pascher_params_2025} J. Pascher, \href{https://github.com/jpascher/T0-Time-Mass-Duality/tree/main/2/pdf/Deutsch/Zeit-Masse-Dualitätstheorie (T0-Modell) Herleitung der Parameter kappa, alpha und beta.pdf}{Zeit-Masse-Dualitätstheorie (T0-Modell): Herleitung der Parameter \(\kappa\), \(\alpha\) und \(\beta\)}, 4. April 2025.
		\bibitem{pascher_galaxies_2025} J. Pascher, \href{https://github.com/jpascher/T0-Time-Mass-Duality/tree/main/2/pdf/Deutsch/Massenvariation in Galaxien.pdf}{Massenvariation in Galaxien: Eine Analyse im T0-Modell mit emergenter Gravitation}, 30. März 2025.
		\bibitem{pascher_quantum_2025} J. Pascher, \href{https://github.com/jpascher/T0-Time-Mass-Duality/tree/main/2/pdf/Deutsch/Die Notwendigkeit einer Erweiterung der Standard-Quantenmechanik und Quantenfeldtheorie.pdf}{Die Notwendigkeit der Erweiterung der Standard-Quantenmechanik und Quantenfeldtheorie}, 27. März 2025.
		\bibitem{pascher_planck_2025} J. Pascher, \href{https://github.com/jpascher/T0-Time-Mass-Duality/tree/main/2/pdf/Deutsch/Jenseits der Planck-Skala.pdf}{Reale Konsequenzen der Umformulierung von Zeit und Masse in der Physik: Jenseits der Planck-Skala}, 24. März 2025.
		\bibitem{pascher_perspective_2025} J. Pascher, \href{https://github.com/jpascher/T0-Time-Mass-Duality/tree/main/2/pdf/Deutsch/Eine neue Perspektive auf Zeit und Raum Johann Paschers revolutionäre Ideen.pdf}{Eine neue Perspektive auf Zeit und Raum: Johann Paschers revolutionäre Ideen}, 25. März 2025.
		\bibitem{pascher_alphabeta_2025} J. Pascher, \href{https://github.com/jpascher/T0-Time-Mass-Duality/tree/main/2/pdf/Deutsch/Die Konsistenz von alpha = 1 und beta = 1.pdf}{Vereinheitlichtes Einheitensystem im T0-Modell: Die Konsistenz von \(\alpha = 1\) und \(\beta = 1\)}, 5. April 2025.
		\bibitem{pascher_emergente_2025} J. Pascher, \href{https://github.com/jpascher/T0-Time-Mass-Duality/tree/main/2/pdf/Deutsch/Emergente Gravitation im T0-Modell Eine formale Herleitung.pdf}{Emergente Gravitation im T0-Modell: Eine umfassende Herleitung}, 1. April 2025.
		\bibitem{Planck2020} Planck Collaboration, Astron. Astrophys. \textbf{641}, A6 (2020).
		\bibitem{Riess1998} A. G. Riess et al., Astron. J. \textbf{116}, 1009 (1998).
		\bibitem{Perlmutter1999} S. Perlmutter et al., Astrophys. J. \textbf{517}, 565 (1999).
		\bibitem{Fixsen2009} D. J. Fixsen, Astrophys. J. \textbf{707}, 916 (2009).
		\bibitem{McGaugh2016} S. S. McGaugh et al., Phys. Rev. Lett. \textbf{117}, 201101 (2016).
		\bibitem{Will2014} C. M. Will, Living Rev. Relativ. \textbf{17}, 4 (2014).
		\bibitem{DiValentino2021} E. Di Valentino et al., Class. Quantum Grav. \textbf{38}, 153001 (2021).
		\bibitem{pascher_qft_2025} J. Pascher, \href{https://github.com/jpascher/T0-Time-Mass-Duality/tree/main/2/pdf/Deutsch/Quantenfeldtheoretische Behandlung des intrinsischen Zeitfelds im T0-Modell.pdf}{Quantenfeldtheoretische Behandlung des intrinsischen Zeitfelds im T0-Modell}, 8. April 2025.
	\end{thebibliography}
	
\end{document}