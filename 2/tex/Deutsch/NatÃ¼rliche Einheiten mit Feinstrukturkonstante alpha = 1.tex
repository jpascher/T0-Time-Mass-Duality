\documentclass{article}
\usepackage[utf8]{inputenc}
\usepackage[T1]{fontenc}
\usepackage{lmodern}
\usepackage[ngerman]{babel}
\usepackage{amsmath,amssymb,physics}
\usepackage{graphicx,tikz,pgfplots}
\pgfplotsset{compat=1.18}
\usepackage{geometry}
\usepackage[colorlinks=true, linkcolor=blue, citecolor=blue, urlcolor=blue]{hyperref}
\usepackage{booktabs}
\usepackage{siunitx}
\usepackage{cleveref}
\usepackage{amsthm}

\geometry{a4paper, margin=2.5cm}

% Theorem styles
\newtheorem{theorem}{Theorem}[section]
\newtheorem{proposition}[theorem]{Proposition}

\title{Energie als fundamentale Einheit: \\ Natürliche Einheiten mit \(\alpha = 1\)}
\author{Johann Pascher}
\date{26. März 2025}

\begin{document}
	
	\maketitle
	
	\begin{abstract}
		Diese Arbeit untersucht die Konsequenzen der Annahme, dass die Feinstrukturkonstante \(\alpha = 1\) in einem System natürlicher Einheiten (\(\hbar = c = 1\)) gesetzt wird. Dabei wird Energie als fundamentale Einheit identifiziert, auf die alle physikalischen Größen zurückgeführt werden können. Die Analyse umfasst dimensionale Umformulierungen, vereinfachte Grundgleichungen und kosmologische Implikationen.
	\end{abstract}
	
	\tableofcontents
	\newpage
	
	\section{Einleitung}
	In der theoretischen Physik werden üblicherweise \(c\) und \(\hbar\) auf eins gesetzt, wie von Planck eingeführt \cite{Planck1899}. Diese Arbeit untersucht die Konsequenzen, wenn zusätzlich \(\alpha = 1\) gesetzt wird.
	
	\section{Natürliche Einheiten und \(\alpha = 1\)}
	\begin{theorem}[Definition von \(\alpha = 1\)]
		Die Feinstrukturkonstante ist \cite{Feynman1985}:
		\begin{equation}
			\alpha = \frac{e^2}{4\pi\varepsilon_0 \hbar c} \approx \frac{1}{137.036}
		\end{equation}
		Mit \(\alpha = 1\), \(\hbar = c = 1\):
		\begin{equation}
			e = \sqrt{4\pi\varepsilon_0}
		\end{equation}
	\end{theorem}
	
	\section{Energie als fundamentale Einheit}
	\begin{theorem}[Energie als Basis]
		Alle Größen lassen sich auf Energie zurückführen \cite{Duff2002}:
		\begin{itemize}
			\item Länge: \([L] = [E^{-1}]\)
			\item Zeit: \([T] = [E^{-1}]\)
			\item Masse: \([M] = [E]\)
			\item Ladung: \([Q] = [\sqrt{4\pi}]\) (dimensionslos)
		\end{itemize}
	\end{theorem}
	
	\section{Vereinfachte Grundgleichungen}
	\begin{itemize}
		\item Maxwell-Gleichungen \cite{Feynman1985}:
		\begin{align}
			\nabla \cdot \vec{E} &= \rho \\
			\nabla \times \vec{B} - \frac{\partial \vec{E}}{\partial t} &= \vec{j}
		\end{align}
		\item Schrödinger-Gleichung:
		\begin{equation}
			i \frac{\partial \psi}{\partial t} = -\frac{1}{2m} \nabla^2 \psi + V \psi
		\end{equation}
	\end{itemize}
	
	\section{Tabelle der umgeformten Größen}
	\begin{center}
		\begin{tabular}{|l|c|c|}
			\hline
			\textbf{Physikalische Größe} & \textbf{SI-Einheiten} & \textbf{\(\hbar = c = \alpha = 1\)} \\
			\hline
			Länge & m & \(\text{eV}^{-1}\) \\
			Zeit & s & \(\text{eV}^{-1}\) \\
			Masse & kg & eV \\
			Energie & J & eV \\
			Ladung & C & dimensionslos \\
			El. Feld & V/m & \(\text{eV}^2\) \\
			Mag. Feld & T & \(\text{eV}^2\) \\
			\hline
		\end{tabular}
	\end{center}
	
	\section{Kosmologische Implikationen}
	Die Annahme \(\alpha = 1\) könnte \cite{Verlinde2011}:
	\begin{itemize}
		\item Elektromagnetische Wechselwirkungen stärker mit Gravitation verbinden.
		\item Eine einheitliche Energiebeschreibung von Raumzeit und Materie ermöglichen.
	\end{itemize}
	
	\begin{figure}[h]
		\centering
		\begin{tikzpicture}
			\begin{axis}[
				xlabel={Energie [eV]},
				ylabel={Länge [eV\(^{-1}\)]},
				xlabel style={font=\large},
				ylabel style={font=\large},
				tick label style={font=\normalsize},
				xmin=0, xmax=10,
				ymin=0, ymax=10,
				legend pos=north east,
				legend style={font=\large},
				grid=both,
				minor tick num=1
				]
				\addplot[blue, ultra thick, domain=0.1:10, samples=100] {1/x};
				\legend{\(L = E^{-1}\)}
			\end{axis}
		\end{tikzpicture}
		\caption{Beziehung zwischen Energie und Länge im \(\alpha = 1\)-System.}
	\end{figure}
	
	\section{Philosophische Implikationen}
	\begin{itemize}
		\item Energie als fundamentalste Eigenschaft der Realität \cite{Wilczek2008}.
		\item Raum und Zeit als emergente Eigenschaften eines Energiefeldes \cite{Verlinde2011}.
	\end{itemize}
	
	\section{Zusammenfassung}
	Durch \(\alpha = 1\) wird Energie zur fundamentalen Einheit, die eine tiefere Einheit der Natur offenbart.
	
	\begin{thebibliography}{10}
		\bibitem{Planck1899}
		Planck, M. (1899). \textit{Über irreversible Strahlungsvorgänge}. Sitzungsberichte der Preußischen Akademie der Wissenschaften.
		\bibitem{Einstein1905}
		Einstein, A. (1905). \textit{Zur Elektrodynamik bewegter Körper}. Annalen der Physik, 322(10), 891-921.
		\bibitem{Duff2002}
		Duff, M. J., Okun, L. B., \& Veneziano, G. (2002). \textit{Trialogue on the number of fundamental constants}. Journal of High Energy Physics, 2002(03), 023.
		\bibitem{Feynman1985}
		Feynman, R. P. (1985). \textit{QED: The Strange Theory of Light and Matter}. Princeton University Press.
		\bibitem{Verlinde2011}
		Verlinde, E. (2011). \textit{On the origin of gravity and the laws of Newton}. Journal of High Energy Physics, 2011(4), 29.
		\bibitem{Mohr2016}
		Mohr, P. J., Newell, D. B., \& Taylor, B. N. (2016). \textit{CODATA Recommended Values of the Fundamental Physical Constants: 2014}. Reviews of Modern Physics, 88(3), 035009.
		\bibitem{Barrow2002}
		Barrow, J. D. (2002). \textit{The Constants of Nature: The Numbers That Encode the Deepest Secrets of the Universe}. Pantheon Books.
		\bibitem{Schwinger1958}
		Schwinger, J. (1958). \textit{Selected Papers on Quantum Electrodynamics}. Dover Publications.
		\bibitem{Adler2010}
		Adler, R. J. (2010). \textit{Six Easy Pieces: Essentials of Physics Explained by Its Most Brilliant Teacher}. Basic Books.
		\bibitem{Wilczek2008}
		Wilczek, F. (2008). \textit{The Lightness of Being: Mass, Ether, and the Unification of Forces}. Basic Books.
	\end{thebibliography}
	
\end{document}