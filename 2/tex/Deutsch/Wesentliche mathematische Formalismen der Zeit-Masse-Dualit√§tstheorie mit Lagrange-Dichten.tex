\documentclass{article}
\usepackage[utf8]{inputenc}
\usepackage[T1]{fontenc}
\usepackage{lmodern}
\usepackage[ngerman]{babel}
\usepackage{amsmath,amssymb,physics,graphicx,xcolor,amsthm}
\usepackage{hyperref}
\usepackage{booktabs}
\usepackage{siunitx}
\usepackage{cleveref}
\usepackage{pgfplots}
\pgfplotsset{compat=1.18}
\usepackage{tikz}
\usetikzlibrary{intersections}
\usepgfplotslibrary{fillbetween}

% Benutzerdefinierte Befehle
\newcommand{\Tfield}{T(x)}
\newcommand{\DcovT}[1]{\Tfield D_\mu #1 + #1 \partial_\mu \Tfield}
\newcommand{\DhiggsT}{\Tfield (\partial_\mu + igA_\mu)\Phi + \Phi \partial_\mu \Tfield}
\newcommand{\gammaf}{\gamma_{\text{Lorentz}}}

% Theorem-Stile
\newtheorem{theorem}{Theorem}[section]
\newtheorem{proposition}[theorem]{Proposition}
\newtheorem{corollary}[theorem]{Korollar}
\newtheorem{lemma}[theorem]{Lemma}
\theoremstyle{definition}
\newtheorem{definition}[theorem]{Definition}
\newtheorem{example}[theorem]{Beispiel}
\theoremstyle{remark}
\newtheorem{remark}[theorem]{Bemerkung}
\renewcommand{\proofname}{Beweis}

% Hyperref-Konfiguration
\hypersetup{
	colorlinks=true,
	linkcolor=blue,
	urlcolor=blue,
	citecolor=red,
	pdftitle={Von der Zeitdilatation zur Massenvariation: Mathematische Kernformulierungen der Zeit-Masse-Dualitätstheorie},
	pdfauthor={Johann Pascher}
}

\title{Von der Zeitdilatation zur Massenvariation: \\ Mathematische Kernformulierungen der Zeit-Masse-Dualitätstheorie}
\author{Johann Pascher}
\date{29. März 2025}

\begin{document}
	
	\maketitle
	
	\begin{abstract}
		Diese Arbeit präsentiert die wesentlichen mathematischen Formulierungen der Zeit-Masse-Dualitätstheorie, wobei der Fokus auf den grundlegenden Gleichungen und ihren physikalischen Interpretationen liegt. Die Theorie etabliert eine Dualität zwischen zwei komplementären Beschreibungen der Realität: dem Standardbild mit Zeitdilatation und konstanter Ruhemasse sowie einem alternativen Bild mit absoluter Zeit und variabler Masse. Zentrale Konzepte dieses Rahmens sind die intrinsische Zeit \( T = \hbar/mc^2 \), die eine direkte Verbindung zwischen Masse und Zeitentwicklung in Quantensystemen herstellt. Die mathematischen Formulierungen umfassen modifizierte Lagrange-Dichten für das Higgs-Feld, Fermionen und Eichbosonen, wobei ihre Wechselwirkungen und Invarianzeigenschaften hervorgehoben werden. Dieses Dokument dient als prägnante mathematische Referenz für die Zeit-Masse-Dualitätstheorie.
	\end{abstract}
	
	\tableofcontents
	\newpage
	
	\section{Einführung in die Zeit-Masse-Dualität}
	Die Zeit-Masse-Dualitätstheorie schlägt einen alternativen Rahmen vor:
	\begin{enumerate}
		\item Standardbild: \( t' = \gammaf t \), \( m_0 = \text{const.} \)
		\item T0-Modell: \( T_0 = \text{const.} \), \( m = \gammaf m_0 \)
	\end{enumerate}
	
	\subsection{Beziehung zum Standardmodell}
	Die Theorie erweitert das Standardmodell mit:
	\begin{enumerate}
		\item Intrinsisches Zeitfeld: \( \Tfield \)
		\item Higgs-Feld: \( \Phi \) mit \( \DhiggsT \)
		\item Fermion-Felder: \( \psi \) mit Yukawa-Kopplung
		\item Eichboson-Felder: \( A_\mu \) mit \( \Tfield^2 \)
	\end{enumerate}
	
	\section{Emergente Gravitation aus dem intrinsischen Zeitfeld}
	\begin{theorem}[Gravitationsemergenz]
		Gravitation entsteht aus Gradienten des intrinsischen Zeitfelds:
		\begin{equation}
			\nabla \Tfield = -\frac{\hbar}{m^2c^2} \nabla m \sim \nabla \Phi_g
		\end{equation}
	\end{theorem}
	
	\begin{proof}
		Aus \( \Tfield = \frac{\hbar}{mc^2} \) folgt:
		\begin{equation}
			\nabla \Tfield = -\frac{\hbar}{m^2c^2} \nabla m
		\end{equation}
		Mit \( m(\vec{r}) = m_0 (1 + \frac{\Phi_g}{c^2}) \):
		\begin{equation}
			\nabla m = \frac{m_0}{c^2} \nabla \Phi_g
		\end{equation}
		Also:
		\begin{equation}
			\nabla \Tfield \approx -\frac{\hbar}{m_0 c^4} \nabla \Phi_g
		\end{equation}
	\end{proof}
	
	\section{Mathematische Grundlagen: Intrinsische Zeit}
	\begin{theorem}[Intrinsische Zeit]
		\begin{equation}
			T = \frac{\hbar}{mc^2}
		\end{equation}
	\end{theorem}
	
	\section{Modifizierte Ableitungsoperatoren}
	\begin{definition}[Modifizierte kovariante Ableitung]
		\begin{equation}
			\DcovT{\Psi} = \Tfield D_\mu \Psi + \Psi \partial_\mu \Tfield
		\end{equation}
	\end{definition}
	
	\section{Modifizierte Feldgleichungen}
	\begin{theorem}[Modifizierte Schrödinger-Gleichung]
		\begin{equation}
			i\hbar \Tfield \frac{\partial}{\partial t} \Psi + i\hbar \Psi \frac{\partial \Tfield}{\partial t} = \hat{H} \Psi
		\end{equation}
	\end{theorem}
	
	\section{Modifizierte Lagrange-Dichte für das Higgs-Feld}
	\begin{theorem}[Higgs-Lagrange-Dichte]
		\begin{equation}
			\mathcal{L}_{\text{Higgs-T}} = (\DhiggsT)^\dagger (\DhiggsT) - \lambda(|\Phi|^2 - v^2)^2
		\end{equation}
	\end{theorem}
	
	\section{Modifizierte Lagrange-Dichte für Fermionen}
	\begin{theorem}[Fermion-Lagrange-Dichte]
		\begin{equation}
			\mathcal{L}_{\text{Fermion}} = \bar{\psi} i \gamma^\mu \DcovT{\psi} - y \bar{\psi} \Phi \psi
		\end{equation}
	\end{theorem}
	
	\section{Modifizierte Lagrange-Dichte für Eichbosonen}
	\begin{theorem}[Eichboson-Lagrange-Dichte]
		\begin{equation}
			\mathcal{L}_{\text{Boson}} = -\frac{1}{4} \Tfield^2 F_{\mu\nu} F^{\mu\nu}
		\end{equation}
	\end{theorem}
	
	\section{Vollständige totale Lagrange-Dichte}
	\begin{theorem}[Totale Lagrange-Dichte]
		\begin{equation}
			\mathcal{L}_{\text{Gesamt}} = \mathcal{L}_{\text{Boson}} + \mathcal{L}_{\text{Fermion}} + \mathcal{L}_{\text{Higgs-T}}
		\end{equation}
	\end{theorem}
	
	\section{Kosmologische Implikationen}
	Die Theorie hat folgende Implikationen:
	\begin{itemize}
		\item Modifiziertes Gravitationspotential: \( \Phi(r) = -\frac{GM}{r} + \kappa r \)
		\item Kosmische Rotverschiebung: \( 1 + z = e^{\alpha r} \)
	\end{itemize}

	
\end{document}