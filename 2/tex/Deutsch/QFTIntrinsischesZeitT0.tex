\documentclass[12pt,a4paper]{article}
\usepackage[utf8]{inputenc}
\usepackage[T1]{fontenc}
\usepackage[ngerman]{babel}
\usepackage{lmodern}
\usepackage{amsmath}
\usepackage{amssymb}
\usepackage{physics}
\usepackage{hyperref}
\usepackage{bookmark}
\usepackage{tcolorbox}
\usepackage{booktabs}
\usepackage{enumitem}
\usepackage[table,xcdraw]{xcolor}
\usepackage[left=2cm,right=2cm,top=2cm,bottom=2cm]{geometry}
\usepackage{pgfplots}
\pgfplotsset{compat=1.18}
\usepackage{graphicx}
\usepackage{float}
\usepackage{fancyhdr}
\usepackage{siunitx}
\usepackage{url}
\usepackage{bm}

% Danksagungs-Umgebung
\newenvironment{acknowledgments}
{\section*{Danksagungen}}
{\vspace{1em}}

% Benutzerdefinierte Befehle
\newcommand{\Tfield}{T(x)}
\newcommand{\alphaEM}{\alpha_{\text{EM}}}
\newcommand{\alphaW}{\alpha_{\text{W}}}
\newcommand{\betaT}{\beta_{\text{T}}}
\newcommand{\Mpl}{M_{\text{Pl}}}
\newcommand{\Tzerot}{T_0(\Tfield)}
\newcommand{\Tzero}{T_0}
\newcommand{\vecx}{\vec{x}}
\newcommand{\vr}{\vec{r}}
\newcommand{\gammaf}{\gamma_{\text{Lorentz}}}
\newcommand{\DhiggsT}{\Tfield (\partial_\mu + ig A_\mu) \Phi + \Phi \partial_\mu \Tfield}
\newcommand{\LCDM}{\Lambda\text{CDM}}
\newcommand{\DTmu}{D_{T,\mu}}
\newcommand{\calL}{\mathcal{L}}
\newcommand{\deq}{\displaystyle}

% Kopf- und Fußzeilen-Konfiguration
\pagestyle{fancy}
\fancyhf{}
\fancyhead[L]{Johann Pascher}
\fancyhead[R]{Quantenfeldtheorie des T0-Modells}
\fancyfoot[C]{\thepage}
\renewcommand{\headrulewidth}{0.4pt}
\renewcommand{\footrulewidth}{0.4pt}

\hypersetup{
	colorlinks=true,
	linkcolor=blue,
	citecolor=blue,
	urlcolor=blue,
	pdftitle={Quantenfeldtheoretische Behandlung des intrinsischen Zeitfeldes im T0-Modell},
	pdfauthor={Johann Pascher},
	pdfsubject={Theoretische Physik},
	pdfkeywords={T0-Modell, intrinsisches Zeitfeld, Quantenfeldtheorie, Zeit-Masse-Dualität}
}

\title{Quantenfeldtheoretische Behandlung des intrinsischen Zeitfeldes im T0-Modell}
\author{Johann Pascher\\
	Abteilung für Kommunikationstechnik\\
	Höhere Technische Bundeslehranstalt (HTL), Leonding, Österreich\\
	\texttt{johann.pascher@gmail.com}}
\date{8. April 2025}

\begin{document}
	
	\maketitle
	
	\begin{abstract}
		Diese Arbeit präsentiert eine systematische quantenfeldtheoretische Behandlung des intrinsischen Zeitfeldes $\Tfield$ im T0-Modell. Ausgehend von der klassischen Feldtheorie wird eine vollständige Quantisierung entwickelt, die kanonische Vertauschungsrelationen, den Pfadintegralformalismus und Wechselwirkungsdynamik umfasst. Besondere Aufmerksamkeit wird der Integration des quantisierten Zeitfeldes mit den Standardmodell-Feldern durch modifizierte Propagatoren und erweiterte Feynman-Regeln gewidmet. Die Theorie erfüllt die Anforderungen an Unitarität und Kausalität und bietet gleichzeitig eine natürliche Brücke zwischen Quantenmechanik und Relativitätstheorie gemäß dem Zeit-Masse-Dualitätsprinzip. Die Quantisierung liefert spezifische experimentelle Vorhersagen, einschließlich Quantenkorrekturen zur wellenlängenabhängigen Rotverschiebung und modifizierter Gravitationswellenausbreitung. Diese konsistente Quantentheorie des intrinsischen Zeitfeldes adressiert offene Fragen in der grundlegenden Physik und etabliert das T0-Modell als vielversprechende Alternative zu konventionellen Ansätzen der Quantengravitation und vereinheitlichten Theorien.
	\end{abstract}
	
	\tableofcontents
	\newpage
	
	\section{Quantenfeldtheoretische Behandlung des T0-Modells}
	\label{sec:qft_treatment}
	
	\subsection{Grundlagen des T0-Modells}
	\label{subsec:foundations}
	
	Das T0-Modell bietet einen neuartigen Ansatz für die fundamentale Physik, der auf dem intrinsischen Zeitfeld $T(x)$ und der Zeit-Masse-Dualität basiert. Im T0-Rahmen wird das intrinsische Zeitfeld definiert als:
	\begin{equation}
		T(x) = \frac{\hbar}{\max(mc^2, \omega)}
		\label{eq:intrinsic_time}
	\end{equation}
	
	In natürlichen Einheiten ($\hbar = c = G = k_B = 1$) vereinfacht sich dies zu:
	\begin{equation}
		T(x) = \frac{1}{m}
		\label{eq:natural_units_time}
	\end{equation}
	
	Diese Beziehung etabliert Energie als fundamentale Einheit, wobei $T(x)$ die Dimension $[E^{-1}]$ hat. Das vollständige T0-Modell baut auf vereinheitlichten natürlichen Einheiten mit $\alpha_{\text{EM}} = \beta_T = \alpha_W = 1$ auf, wie in \cite{Pascher2025Alpha1Beta1} entwickelt.
	
	\subsection{Feldgleichungen und Lagrange-Dichte}
	\label{subsec:field_equations}
	
	Die Feldgleichung für das intrinsische Zeitfeld lautet:
	\begin{equation}
		\nabla^2 T(x) = -\kappa\rho(x)T(x)^2
		\label{eq:field_equation}
	\end{equation}
	
	wobei $\kappa$ die Dimension $[E]$ und $\rho$ die Dimension $[E^2]$ in natürlichen Einheiten haben. Wie in \cite{Pascher2025EmergentGrav} gezeigt, führt diese Gleichung zu emergenter Gravitation.
	
	Die gesamte Lagrange-Dichte für das T0-Modell ist:
	\begin{equation}
		\mathcal{L}_{\text{Total}} = \mathcal{L}_{\text{Boson}} + \mathcal{L}_{\text{Fermion}} + \mathcal{L}_{\text{Higgs-T}} + \mathcal{L}_{\text{intrinsisch}}
		\label{eq:total_lagrangian}
	\end{equation}
	
	wobei:
	\begin{align}
		\mathcal{L}_{\text{intrinsisch}} &= \frac{1}{2}\partial_{\mu}T(x)\partial^{\mu}T(x) - \frac{1}{2}T(x)^2 - \frac{\rho}{T(x)} \label{eq:intrinsic_lagrangian} \\
		\mathcal{L}_{\text{Boson}} &= -\frac{1}{4}F_{\mu\nu}F^{\mu\nu} \label{eq:boson_lagrangian} \\
		\mathcal{L}_{\text{Fermion}} &= \bar{\psi}i\gamma^{\mu}D_{T\mu}\psi \label{eq:fermion_lagrangian} \\
		\mathcal{L}_{\text{Higgs-T}} &= |D_{T\mu}\Phi|^2 - V(\Phi) \label{eq:higgs_lagrangian}
	\end{align}
	
	mit den T-modifizierten Ableitungen:
	\begin{align}
		D_{T\mu}\psi &= T(x)D_{\mu}\psi + \psi\partial_{\mu}T(x) \label{eq:t_modified_derivative} \\
		D_{T\mu}\Phi &= T(x)(\partial_{\mu} + igA_{\mu})\Phi + \Phi\partial_{\mu}T(x) \label{eq:higgs_t_derivative}
	\end{align}
	
	Diese Ausdrücke wahren vollständige Konsistenz mit den in \cite{Pascher2025Lagrange} und \cite{Pascher2025Higgs} hergeleiteten Feldgleichungen.
	
	\subsection{Kanonische Quantisierung}
	\label{subsec:canonical_quantization}
	
	Um das intrinsische Zeitfeld zu quantisieren, trennen wir es in einen klassischen Hintergrund und Quantenfluktuationen:
	\begin{equation}
		T(x) = T_c(x) + \hat{T}(x)
		\label{eq:quantum_decomposition}
	\end{equation}
	
	Der klassische Teil erfüllt:
	\begin{equation}
		\nabla^2 T_c(x) = -\kappa\rho(x)T_c(x)^2
		\label{eq:classical_field_equation}
	\end{equation}
	
	Für die Quantenfluktuationen wird die effektive Lagrange-Funktion:
	\begin{equation}
		\mathcal{L}_{\text{quantum}} \approx \frac{1}{2}\partial_{\mu}\hat{T}(x)\partial^{\mu}\hat{T}(x) - \frac{1}{2}\hat{T}(x)^2 - \frac{\rho}{T_c(x)^3}\hat{T}(x)^2
		\label{eq:quantum_lagrangian}
	\end{equation}
	
	Der kanonische Impuls ist:
	\begin{equation}
		\Pi(x) = \frac{\partial\mathcal{L}}{\partial(\partial_0 \hat{T})} = \partial_0 \hat{T}(x)
		\label{eq:canonical_momentum}
	\end{equation}
	
	Wir setzen kanonische Vertauschungsrelationen an:
	\begin{equation}
		[\hat{T}(\vec{x}, t), \Pi(\vec{y}, t)] = i\delta^3(\vec{x} - \vec{y})
		\label{eq:commutation_relation}
	\end{equation}
	\begin{equation}
		[\hat{T}(\vec{x}, t), \hat{T}(\vec{y}, t)] = [\Pi(\vec{x}, t), \Pi(\vec{y}, t)] = 0
		\label{eq:field_commutators}
	\end{equation}
	
	Diese Beziehungen folgen direkt aus den Prinzipien der Quantenfeldtheorie, angewandt auf das intrinsische Zeitfeld, ohne Standardmodell-Annahmen zu importieren.
	
	\subsection{Modenentwicklung und Hamiltonian}
	\label{subsec:mode_expansion}
	
	Die effektive Masse für Quantenfluktuationen ist:
	\begin{equation}
		m_{\text{eff}}^2(x) = 1 + \frac{2\rho}{T_c(x)^3}
		\label{eq:effective_mass}
	\end{equation}
	
	In Regionen mit annähernd gleichförmiger Massendichte können wir $\hat{T}(x)$ mithilfe der Modenentwicklung ausdrücken:
	\begin{equation}
		\hat{T}(x) = \int \frac{d^3k}{(2\pi)^3} \frac{1}{\sqrt{2\omega_{\vec{k}}}} \left(a_{\vec{k}} e^{-ik \cdot x} + a_{\vec{k}}^{\dagger} e^{ik \cdot x}\right)
		\label{eq:mode_expansion}
	\end{equation}
	
	wobei $\omega_{\vec{k}} = \sqrt{\vec{k}^2 + m_{\text{eff}}^2}$ und:
	\begin{equation}
		[a_{\vec{k}}, a_{\vec{k'}}^{\dagger}] = (2\pi)^3 \delta^3(\vec{k} - \vec{k'})
		\label{eq:creation_annihilation}
	\end{equation}
	
	Die Hamiltondichte ist:
	\begin{equation}
		\mathcal{H} = \frac{1}{2}\Pi(x)^2 + \frac{1}{2}(\nabla \hat{T}(x))^2 + \frac{1}{2}\left(1 + \frac{2\rho}{T_c(x)^3}\right)\hat{T}(x)^2
		\label{eq:hamiltonian_density}
	\end{equation}
	
	Nach Normalordnung wird der Hamiltonian:
	\begin{equation}
		H = \int \frac{d^3k}{(2\pi)^3} \omega_{\vec{k}} :a_{\vec{k}}^{\dagger}a_{\vec{k}}: + E_0
		\label{eq:hamiltonian}
	\end{equation}
	
	Die Vakuumenergie $E_0$ hängt von der Materieverteilung durch $m_{\text{eff}}$ ab und bietet einen natürlichen Mechanismus, durch den sich die Vakuumenergie an die Präsenz von Materie anpasst. Dies unterscheidet sich grundlegend vom Standardmodell-Ansatz, der zum kosmologischen Konstantenproblem führt. Im T0-Modell wird dieses Problem auf natürliche Weise durch die Kopplung zwischen dem Zeitfeld und der Materieverteilung adressiert, wie in \cite{Pascher2025Energy} detailliert beschrieben.
	
	\subsection{Pfadintegralformulierung}
	\label{subsec:path_integral}
	
	Das erzeugende Funktional für das Quantenzeitfeld ist:
	\begin{equation}
		Z[J] = \int \mathcal{D}T \exp\left(i\int d^4x (\mathcal{L}_{\text{quantum}} + J(x)\hat{T}(x))\right)
		\label{eq:generating_functional}
	\end{equation}
	
	das ausgewertet werden kann als:
	\begin{equation}
		Z[J] = \exp\left(-\frac{i}{2}\int d^4x d^4y J(x)\Delta_F(x-y)J(y)\right)
		\label{eq:evaluated_functional}
	\end{equation}
	
	wobei $\Delta_F(x-y)$ der Feynman-Propagator ist:
	\begin{equation}
		\Delta_F(x-y) = \int \frac{d^4k}{(2\pi)^4} \frac{i}{k^2 - m_{\text{eff}}^2 + i\epsilon} e^{-ik \cdot (x-y)}
		\label{eq:feynman_propagator}
	\end{equation}
	
	Dieser Propagator enthält die effektive Masse $m_{\text{eff}}$, die die Materiekopplung enthält, ein dem T0-Modell eigenes Merkmal, das im Standardmodell-Ansatz nicht zu finden ist.
	
	\subsection{Modifizierte Teilchen-Propagatoren}
	\label{subsec:modified_propagators}
	
	Das Zeitfeld modifiziert die Propagatoren der Standardmodell-Teilchen. Für Fermionen:
	\begin{equation}
		S_F^T(p) = \frac{i(\slash{p} + m)}{p^2 - m^2 + i\epsilon} \cdot \frac{T_c}{T_0}
		\label{eq:fermion_propagator}
	\end{equation}
	
	Für Eichbosonen:
	\begin{equation}
		D_{\mu\nu}^T(p) = \frac{-ig_{\mu\nu} + \frac{p_{\mu}p_{\nu}}{p^2}}{p^2 + i\epsilon} \cdot \left(\frac{T_c}{T_0}\right)^2
		\label{eq:gauge_propagator}
	\end{equation}
	
	Diese Modifikationen folgen direkt aus der Kopplungsstruktur in der T0-Modell-Lagrange-Funktion, ohne zusätzliche Parameter einzuführen, im Einklang mit dem in \cite{Pascher2025Lagrange} dargestellten Rahmen.
	
	\subsection{Feynman-Regeln}
	\label{subsec:feynman_rules}
	
	Die Feynman-Regeln des T0-Modells umfassen:
	
	\begin{enumerate}
		\item $T(x)$-Propagator: $\frac{i}{p^2 - m_{\text{eff}}^2 + i\epsilon}$
		\item Fermion-$T(x)$-Vertex: $i\gamma^{\mu}p_{\mu}$
		\item Eichboson-$T(x)$-Vertex: $-2ig_{\mu\nu}T_c$
		\item Higgs-$T(x)$-Vertex: $ip_{\mu}\Phi^*\partial^{\mu}\Phi + \text{h.c.}$
	\end{enumerate}
	
	Diese Regeln bewahren die dimensionale Konsistenz mit Energie als fundamentaler Einheit und folgen direkt aus den T0-Modell-Lagrange-Dichten, die in \cite{Pascher2025Fields} hergeleitet wurden.
	
	\subsection{Quantenkorrekturen zur wellenlängenabhängigen Rotverschiebung}
	\label{subsec:quantum_redshift}
	
	Die klassische wellenlängenabhängige Rotverschiebung im T0-Modell ist:
	\begin{equation}
		z(\lambda) = z_0\left(1 + \ln\frac{\lambda}{\lambda_0}\right)
		\label{eq:classical_redshift}
	\end{equation}
	
	Quantenfluktuationen des Zeitfeldes führen zu Korrekturen:
	\begin{equation}
		z(\lambda) = z_0\left(1 + \ln\frac{\lambda}{\lambda_0} + \frac{\langle \hat{T}(x)^2 \rangle}{T_c(x)^2}\right)
		\label{eq:quantum_redshift}
	\end{equation}
	
	Der Quantenkorrekturterm $\frac{\langle \hat{T}(x)^2 \rangle}{T_c(x)^2}$ ist proportional zu $\frac{1}{E_{\text{Pl}}}$, was ihn für typische astronomische Beobachtungen klein macht, aber potenziell in hochpräzisen Messungen nachweisbar, wie in \cite{Pascher2025Measurements} diskutiert.
	
	\subsection{Emergente Raumzeit}
	\label{subsec:emergent_spacetime}
	
	Das Quantenzeitfeld führt zu einer emergenten Raumzeitmetrik:
	\begin{equation}
		g_{\mu\nu}^{\text{eff}} = \eta_{\mu\nu} + 2\kappa\langle T(x) \rangle \partial_{\mu}\partial_{\nu}\langle T(x) \rangle - \kappa\eta_{\mu\nu}\partial_{\alpha}\langle T(x) \rangle \partial^{\alpha}\langle T(x) \rangle
		\label{eq:emergent_metric}
	\end{equation}
	
	Dies verbindet sich direkt mit dem Gravitationsparameter $\kappa^{\text{nat}} = \betaT^{\text{nat}} \cdot \frac{yv}{r_g^2}\betaT^{\text{nat}} \cdot \frac{yv}{r_g^2}$ mit $\beta_T = 1$ in natürlichen Einheiten. Die Metrik entsteht auf natürliche Weise aus der Dynamik des Zeitfeldes, ohne die allgemeine Relativitätstheorie vorauszusetzen, wie in \cite{Pascher2025EmergentGrav} gezeigt.
	
	Gravitationswellen entstehen als Oszillationen im Erwartungswert des Zeitfeldes, mit einer Ausbreitungsgeschwindigkeit von:
	\begin{equation}
		v_{\text{GW}} = c\left(1 - \frac{1}{2}\frac{\langle \hat{T}(x)^2 \rangle}{T_c(x)^2}\right)
		\label{eq:gw_speed}
	\end{equation}
	
	was eine kleine Quantenkorrektur zeigt, die möglicherweise in zukünftigen Gravitationswellenbeobachtungen gemessen werden könnte.
	
	\subsection{Modifizierte Unschärferelationen}
	\label{subsec:uncertainty}
	
	Das Zeitfeld führt zu modifizierten Unschärferelationen:
	\begin{equation}
		\Delta x \Delta p \geq \frac{\hbar}{2}\left(1 + \langle \hat{T}(x) \rangle \Delta V\right)
		\label{eq:uncertainty}
	\end{equation}
	
	Dies bietet eine natürliche Brücke zwischen Quanten- und klassischen Regimen, wobei die Standard-Unschärferelation im Grenzfall schwacher Gravitationsfelder wiederhergestellt wird, wie in \cite{Pascher2025Extensions} untersucht.
	
	\section{Experimentelle Vorhersagen}
	\label{sec:predictions}
	
	Die quantenfeldtheoretische Behandlung des T0-Modells führt zu mehreren einzigartigen experimentellen Vorhersagen, die es vom Standardmodell unterscheiden:
	
	\subsection{Wellenlängenabhängige Rotverschiebung}
	\label{subsec:redshift_prediction}
	
	Das T0-Modell sagt eine spezifische Wellenlängenabhängigkeit der kosmischen Rotverschiebung voraus:
	\begin{equation}
		z(\lambda) = z_0\left(1 + \ln\frac{\lambda}{\lambda_0}\right)
		\label{eq:redshift_prediction}
	\end{equation}
	
	Dies kann durch Multiband-Beobachtungen entfernter Galaxien mit Instrumenten wie dem James-Webb-Weltraumteleskop getestet werden, wie in \cite{Pascher2025Measurements} detailliert beschrieben.
	
	\subsection{Modifiziertes Gravitationspotential}
	\label{subsec:potential_prediction}
	
	Das Gravitationspotential im T0-Modell nimmt die Form an:
	\begin{equation}
		\Phi(r) = -\frac{M}{r} + \kappa r
		\label{eq:grav_potential}
	\end{equation}
	
	wobei $\kappa^{\text{nat}} = \betaT^{\text{nat}} \cdot \frac{yv}{r_g^2}\betaT^{\text{nat}} \cdot \frac{yv}{r_g^2}$ die Dimension $[E]$ in natürlichen Einheiten hat. Dieses modifizierte Potential erklärt Galaxie-Rotationskurven ohne dunkle Materie, wie in \cite{Pascher2025Galaxies} gezeigt.
	
	\subsection{Quantengravitative Effekte}
	\label{subsec:quantum_gravity}
	
	Das T0-Modell sagt quantengravitative Effekte bei Energien von ungefähr:
	\begin{equation}
		E_{\text{QG}} \sim \sqrt{\xi} \cdot M_{\text{Pl}} \approx 10^{-2} M_{\text{Pl}}
		\label{eq:quantum_gravity_scale}
	\end{equation}
	
	voraus, wobei $\xi \approx 1,33 \times 10^{-4}$ die charakteristische Längenskala $r_0$ mit der Planck-Länge in Beziehung setzt: $r_0 = \xi \cdot l_P$. Dies bringt quantengravitative Effekte potenziell in Reichweite zukünftiger Experimente, wie in \cite{Pascher2025Planck} diskutiert.
	
	\section{Schlussfolgerung}
	\label{sec:conclusion}
	
	Die quantenfeldtheoretische Behandlung des T0-Modells bietet einen konsistenten Rahmen, der:
	
	\begin{enumerate}
		\item Energie durchgehend als fundamentale Einheit beibehält
		\item Keine neuen unabhängigen Konstanten über die im T0-Modell angegebenen hinaus erfordert
		\item Eine natürliche Erklärung für die Rotverschiebung ohne kosmische Expansion liefert
		\item Eine elegante Lösung für das Vakuumenergieproblems bietet
		\item Spezifische, testbare Vorhersagen macht, die es vom Standardmodell unterscheiden
	\end{enumerate}
	
	Diese Entwicklung vervollständigt die theoretische Struktur des T0-Modells und etabliert es als eine tragfähige Alternative zu konventionellen Ansätzen der Quantengravitation und vereinheitlichten Theorien.
	
	\begin{thebibliography}{99}
		\bibitem{Pascher2025Alpha1Beta1} Pascher, J. (2025). \href{https://github.com/jpascher/T0-Time-Mass-Duality/tree/main/2/pdf/Deutsch/Alpha1Beta1Konsistenz.pdf}{Einheitliches Einheitensystem im T0-Modell: Die Konsistenz von $\alpha = 1$ und $\beta = 1$}.
		
		\bibitem{Pascher2025EmergentGrav} Pascher, J. (2025). \href{https://github.com/jpascher/T0-Time-Mass-Duality/tree/main/2/pdf/Deutsch/EmergentGravT0.pdf}{Emergente Gravitation im T0-Modell: Eine umfassende Ableitung}.
		
		\bibitem{Pascher2025Lagrange} Pascher, J. (2025). \href{https://github.com/jpascher/T0-Time-Mass-Duality/tree/main/2/pdf/Deutsch/MathZeitMasseLagrange.pdf}{Von der Zeitdilatation zur Massenvariation: Mathematische Kernformulierungen der Zeit-Masse-Dualitätstheorie}.
		
		\bibitem{Pascher2025Higgs} Pascher, J. (2025). \href{https://github.com/jpascher/T0-Time-Mass-Duality/tree/main/2/pdf/Deutsch/MathHiggsZeitMasse.pdf}{Mathematische Formulierung des Higgs-Mechanismus in der Zeit-Masse-Dualität}.
		
		\bibitem{Pascher2025Energy} Pascher, J. (2025). \href{https://github.com/jpascher/T0-Time-Mass-Duality/tree/main/2/pdf/Deutsch/MathEnergiedynamik.pdf}{Dunkle Energie im T0-Modell: Eine mathematische Analyse der Energiedynamik}.
		
		\bibitem{Pascher2025Fields} Pascher, J. (2025). \href{https://github.com/jpascher/T0-Time-Mass-Duality/tree/main/2/pdf/Deutsch/FeldtheorieQuanten.pdf}{Feldtheorie und Quantenkorrelationen: Eine neue Perspektive auf Instantanität}.
		
		\bibitem{Pascher2025Measurements} Pascher, J. (2025). \href{https://github.com/jpascher/T0-Time-Mass-Duality/tree/main/2/pdf/Deutsch/MessdifferenzenT0Standard.pdf}{Kompensatorische und additive Effekte: Eine Analyse der Messdifferenzen zwischen dem T0-Modell und dem $\Lambda$CDM-Standardmodell}.
		
		\bibitem{Pascher2025Galaxies} Pascher, J. (2025). \href{https://github.com/jpascher/T0-Time-Mass-Duality/tree/main/2/pdf/Deutsch/MassVarGalaxien.pdf}{Massenvariation in Galaxien: Eine Analyse im T0-Modell mit emergenter Gravitation}.
		
		\bibitem{Pascher2025Extensions} Pascher, J. (2025). \href{https://github.com/jpascher/T0-Time-Mass-Duality/tree/main/2/pdf/Deutsch/NotwendigkeitQMErweiterung.pdf}{Die Notwendigkeit der Erweiterung der Standardquantenmechanik und Quantenfeldtheorie}.
		
		\bibitem{Pascher2025Planck} Pascher, J. (2025). \href{https://github.com/jpascher/T0-Time-Mass-Duality/tree/main/2/pdf/Deutsch/JenseitsPlanck.pdf}{Reale Konsequenzen der Neuformulierung von Zeit und Masse in der Physik: Jenseits der Planck-Skala}.
		
		\bibitem{Pascher2025Time} Pascher, J. (2025). \href{https://github.com/jpascher/T0-Time-Mass-Duality/tree/main/2/pdf/Deutsch/ZeitEmergentQM.pdf}{Zeit als emergente Eigenschaft in der Quantenmechanik: Eine Verbindung zwischen Relativitätstheorie, Feinstrukturkonstante und Quantendynamik}.
		
		\bibitem{Pascher2025TimeMass} Pascher, J. (2025). \href{https://github.com/jpascher/T0-Time-Mass-Duality/tree/main/2/pdf/Deutsch/ZeitMasseNeuerBlick.pdf}{Zeit und Masse: Ein neuer Blick auf alte Formeln – und Befreiung von traditionellen Einschränkungen}.
		
		\bibitem{Pascher2025Parameters} Pascher, J. (2025). \href{https://github.com/jpascher/T0-Time-Mass-Duality/tree/main/2/pdf/Deutsch/ZeitMasseT0Params.pdf}{Zeit-Masse-Dualitätstheorie (T0-Modell): Ableitung der Parameter $\kappa$, $\alpha$ und $\beta$}.
		
		\bibitem{Pascher2025Photons} Pascher, J. (2025). \href{https://github.com/jpascher/T0-Time-Mass-Duality/tree/main/2/pdf/Deutsch/DynMassePhotonenNichtlokal.pdf}{Dynamische Masse von Photonen und ihre Implikationen für Nichtlokalität im T0-Modell}.
		
		\bibitem{Pascher2025Forces} Pascher, J. (2025). \href{https://github.com/jpascher/T0-Time-Mass-Duality/tree/main/2/pdf/Deutsch/VierKraefteZeitMasse.pdf}{Vereinfachte Beschreibung fundamentaler Kräfte mit Zeit-Masse-Dualität}.
	\end{thebibliography}
	
\end{document}