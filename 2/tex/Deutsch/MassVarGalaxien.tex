\documentclass[12pt,a4paper]{article}
\usepackage[utf8]{inputenc}
\usepackage[T1]{fontenc}
\usepackage[ngerman]{babel}
\usepackage[left=2cm,right=2cm,top=2cm,bottom=2cm]{geometry}
\usepackage{lmodern}
\usepackage{amsmath}
\usepackage{amssymb}
\usepackage{physics}
\usepackage{hyperref}
\usepackage{tcolorbox}
\usepackage{booktabs}
\usepackage{enumitem}
\usepackage[table,xcdraw]{xcolor}
\usepackage{pgfplots}
\pgfplotsset{compat=1.18}
\usepackage{graphicx}
\usepackage{float}
\usepackage{mathtools}
\usepackage{tocloft}
\usepackage{fancyhdr}

\renewcommand{\cftsecfont}{\color{blue}}
\renewcommand{\cftsubsecfont}{\color{blue}}
\renewcommand{\cftsecpagefont}{\color{blue}}
\renewcommand{\cftsubsecpagefont}{\color{blue}}
\setlength{\cftsecindent}{1cm}
\setlength{\cftsubsecindent}{2cm}

\hypersetup{
	colorlinks=true,
	linkcolor=blue,
	citecolor=blue,
	urlcolor=blue,
	pdftitle={Massenvariation in Galaxien: Eine Analyse im T0-Modell mit emergenter Gravitation},
	pdfauthor={Johann Pascher},
	pdfsubject={Theoretische Physik},
	pdfkeywords={T0-Modell, Zeit-Masse-Dualität, Galaxiendynamik, Dunkle Materie}
}

% Benutzerdefinierte Befehle
\newcommand{\Tfield}{T(x)}
\newcommand{\betaT}{\beta_{\text{T}}}
\newcommand{\alphaEM}{\alpha_{\text{EM}}}
\newcommand{\alphaW}{\alpha_{\text{W}}}
\newcommand{\Mpl}{M_{\text{Pl}}}
\newcommand{\Tzerot}{T_0(\Tfield)}
\newcommand{\Tzero}{T_0}
\newcommand{\vecx}{\vec{x}}
\newcommand{\DhiggsT}{\Tfield (\partial_\mu + ig A_\mu) \Phi + \Phi \partial_\mu \Tfield}
\newcommand{\DcovT}[1]{\Tfield D_\mu #1 + #1 \partial_\mu \Tfield}
\newcommand{\HiggsLagr}{\mathcal{L}_{\text{Higgs-T}}}

% Kopf- und Fußzeilen
\pagestyle{fancy}
\fancyhf{}
\fancyhead[L]{Johann Pascher}
\fancyhead[R]{Zeit-Masse-Dualität}
\fancyfoot[C]{\thepage}
\renewcommand{\headrulewidth}{0.4pt}
\renewcommand{\footrulewidth}{0.4pt}

\title{Massenvariation in Galaxien: \\Eine Analyse im T0-Modell mit emergenter Gravitation}
\author{Johann Pascher}
\date{30. März 2025}

\begin{document}
	
	\maketitle
	
	\begin{abstract}
		Diese Arbeit analysiert die Galaxiendynamik im Rahmen des T0-Modells der Zeit-Masse-Dualitätstheorie, in der die Zeit absolut ist und die Masse variiert gemäß \( m = \frac{\hbar}{T c^2} \), wobei \( \Tfield \) ein dynamisches intrinsisches Zeitfeld ist. Gravitation wird nicht als fundamentale Wechselwirkung eingeführt, sondern ergibt sich aus den Gradienten von \( \Tfield \). Wir formulieren eine vollständige Lagrange-Dichte, die Beiträge der vier fundamentalen Felder (Higgs, Fermionen, Eichbosonen) und des intrinsischen Zeitfeldes umfasst, und zeigen, dass flache Rotationskurven durch die Variation von \( \Tfield \) erklärt werden können, ohne dunkle Materie oder separate dunkle Energie zu benötigen. Experimentelle Tests zur Validierung des Modells werden vorgeschlagen, einschließlich kosmologischer Implikationen wie der Interpretation der kosmischen Mikrowellenhintergrundstrahlung.
	\end{abstract}
	
	\tableofcontents
	\newpage
	
	\section{Einleitung}
	Die Rotationskurven von Galaxien zeigen ein Verhalten, das nicht allein durch sichtbare Materie erklärt werden kann. In den äußeren Bereichen von Spiralgalaxien bleibt die Rotationsgeschwindigkeit \( v(r) \) nahezu konstant, anstatt gemäß Keplers Gesetz für isolierte Massen mit \( r^{-1/2} \) abzunehmen. Das Standardkosmologische Modell (\(\Lambda\)CDM) erklärt dieses Phänomen durch die Annahme einer unsichtbaren Komponente, der dunklen Materie, die einen ausgedehnten Halo um Galaxien bildet und die Bewegung sichtbarer Materie durch ihr Gravitationsfeld steuert, ergänzt durch dunkle Energie zur Erklärung der kosmischen Beschleunigung.
	
	Diese Arbeit verfolgt einen alternativen Ansatz, der auf dem T0-Modell der Zeit-Masse-Dualitätstheorie basiert, bei dem die Zeit absolut ist und die Teilchenmasse gemäß \( m = \frac{\hbar}{\Tfield c^2} \) variiert, wobei \( \Tfield \) ein dynamisches intrinsisches Zeitfeld ist. In diesem Rahmen wird dunkle Materie nicht als separate Entität betrachtet; stattdessen entstehen die beobachteten dynamischen Effekte aus emergenter Gravitation, die sich aus den Gradienten von \( \Tfield \) ergibt. Ebenso werden Effekte, die traditionell der dunklen Energie zugeschrieben werden, wie die Rotverschiebung, durch die räumliche Variation von \( \Tfield \) erklärt, wodurch die Notwendigkeit einer separaten dunklen Energie wie im \(\Lambda\)CDM-Modell entfällt. Diese Neuformulierung liefert mathematisch äquivalente Vorhersagen für Rotationskurven und bietet eine grundlegend andere physikalische Interpretation, die weder dunkle Materie noch separate dunkle Energie erfordert. Eine detaillierte Analyse der kosmologischen Implikationen des T0-Modells, insbesondere bezüglich Distanzmessungen, Rotverschiebung und der Interpretation der kosmischen Mikrowellenhintergrundstrahlung, wird in \cite{pascher_messdifferenzen_2025} dargelegt.
	
	\subsection{Rotverschiebung im T0-Modell}
	Im T0-Modell wird die Rotverschiebung \( z \) durch die Variation des intrinsischen Zeitfeldes \( \Tfield \) bestimmt. Die Beziehung zwischen Rotverschiebung und Masse ist gegeben durch:
	\begin{equation}
		1 + z = \frac{\Tfield_0}{\Tfield} = \frac{m}{m_0},
	\end{equation}
	wobei \( \Tfield_0 \) und \( m_0 \) die Werte des intrinsischen Zeitfeldes und der Masse am Ort des Beobachters sind. Diese Interpretation der Rotverschiebung basiert auf intrinsischer Zeit und erfordert keine kosmische Expansion, im Gegensatz zum \(\Lambda\)CDM-Modell, wo die Rotverschiebung durch die Expansion des Universums erklärt wird:
	\begin{equation}
		1 + z = \frac{a(t_0)}{a(t_{\text{emit}})}.
	\end{equation}
	Die räumliche Variation von \( \Tfield \) kann mit der Distanz \( d \) über \( \Tfield = \Tfield_0 e^{-\alpha d} \) verknüpft werden, wobei \( \alpha = H_0/c \), was zu einer äquivalenten Form führt:
	\begin{equation}
		1 + z = e^{\alpha d}.
	\end{equation}
	Diese Formulierung stimmt mit dem Energieverlust von Photonen aufgrund der Dynamik von \( \Tfield \) überein, wie in \cite{pascher_messdifferenzen_2025} detailliert beschrieben. Die Beziehung zwischen Rotverschiebung und Distanz \( d \) im T0-Modell ist somit:
	\begin{equation}
		d = \frac{c \ln(1 + z)}{H_0},
	\end{equation}
	wobei \( H_0 \) die Hubble-Konstante ist, die im T0-Modell als Maß für die räumliche Variationsrate von \( \Tfield \) und nicht als Expansionsrate neu interpretiert wird.
	
	\subsection{Kosmologische Implikationen: Distanzmessungen und \\CMB-Interpretation}
	Das T0-Modell hat weitreichende Implikationen für kosmologische Messungen, wie in \cite{pascher_messdifferenzen_2025} detailliert beschrieben. Insbesondere unterscheiden sich die Distanzmaße im T0-Modell von denen im \(\Lambda\)CDM-Modell:
	
	- \textbf{Physikalische Distanz \( d \):}
	\[
	d = \frac{c \ln(1 + z)}{H_0},
	\]
	im Vergleich zu \(\Lambda\)CDM:
	\[
	d = \frac{c}{H_0} \int_0^z \frac{dz'}{\sqrt{\Omega_m (1 + z')^3 + \Omega_\Lambda}}.
	\]
	
	- \textbf{Leuchtkraftdistanz \( d_L \):}
	\[
	d_L = \frac{c}{H_0} \ln(1 + z) (1 + z),
	\]
	im Vergleich zu \(\Lambda\)CDM:
	\[
	d_L = (1 + z) \cdot \frac{c}{H_0} \int_0^z \frac{dz'}{\sqrt{\Omega_m (1 + z')^3 + \Omega_\Lambda}}.
	\]
	
	- \textbf{Winkeldurchmesserdistanz \( d_A \):}
	\[
	d_A = \frac{c \ln(1 + z)}{H_0 (1 + z)},
	\]
	im Vergleich zu \(\Lambda\)CDM:
	\[
	d_A = \frac{d}{1 + z}.
	\]
	
	Zusätzlich wird die CMB-Temperatur-Rotverschiebungs-Beziehung im T0-Modell aufgrund der Dynamik von \( \Tfield \) modifiziert:
	\begin{equation}
		T(z) = T_0 (1 + z) (1 + \betaT \ln(1 + z)),
	\end{equation}
	mit \( \betaT \approx 0.008 \) in SI-Einheiten, verglichen mit der \(\Lambda\)CDM-Vorhersage \( T(z) = T_0 (1 + z) \). Diese Unterschiede führen zu signifikanten Abweichungen bei hohen Rotverschiebungen, insbesondere für die kosmische Mikrowellenhintergrundstrahlung (CMB) bei \( z = 1100 \). Im T0-Modell ist die Winkeldurchmesserdistanz \( d_A \) ungefähr doppelt so groß wie im \(\Lambda\)CDM-Modell (28,9 Mpc gegenüber 13,5 Mpc), was zu einer Winkelgröße von Strukturen von etwa \( 5,8^\circ \) im T0-Modell im Vergleich zu \( 1^\circ \) im \(\Lambda\)CDM-Modell führt. Diese drastischen Unterschiede bieten eine Möglichkeit, die Modelle experimentell zu testen, wie in \cite{pascher_messdifferenzen_2025} weiter ausgeführt wird.
	
	\begin{thebibliography}{99}
		\bibitem{pascher_zeit_2025} Pascher, J. (2025). \href{https://github.com/jpascher/T0-Time-Mass-Duality/tree/main/2/pdf/Deutsch/ZeitEmergentQM.pdf}{Zeit als emergente Eigenschaft in der Quantenmechanik: Eine Verbindung zwischen Relativitätstheorie, Feinstrukturkonstante und Quantendynamik}. 23. März 2025.
		\bibitem{pascher_messdifferenzen_2025} Pascher, J. (2025). \href{https://github.com/jpascher/T0-Time-Mass-Duality/tree/main/2/pdf/Deutsch/MessdifferenzenT0Standard.pdf}{Kompensatorische und additive Effekte: Eine Analyse der Messdifferenzen zwischen dem T0-Modell und dem \(\Lambda\)CDM-Standardmodell}. 2. April 2025.
		\bibitem{pascher_alpha_2025} Pascher, J. (2025). \href{https://github.com/jpascher/T0-Time-Mass-Duality/tree/main/2/pdf/Deutsch/NatEinheitenAlpha1.pdf}{Energie als fundamentale Einheit: Natürliche Einheiten mit \(\alpha = 1\) im T0-Modell}. 26. März 2025.
		\bibitem{pascher_params_2025} Pascher, J. (2025). \href{https://github.com/jpascher/T0-Time-Mass-Duality/tree/main/2/pdf/Deutsch/ZeitMasseT0Params.pdf}{Zeit-Masse-Dualitätstheorie (T0-Modell): Herleitung der Parameter \(\kappa\), \(\alpha\) und \(\beta\)}. 4. April 2025.
		\bibitem{pascher_higgs_2025} Pascher, J. (2025). \href{https://github.com/jpascher/T0-Time-Mass-Duality/tree/main/2/pdf/Deutsch/MathHiggsZeitMasse.pdf}{Mathematische Formulierung des Higgs-Mechanismus in der Zeit-Masse-Dualität}. 28. März 2025.
		\bibitem{pascher_lagrange_2025} Pascher, J. (2025). \href{https://github.com/jpascher/T0-Time-Mass-Duality/tree/main/2/pdf/Deutsch/MathZeitMasseLagrange.pdf}{Von der Zeitdilatation zur Massenvariation: Mathematische Kernformulierungen der Zeit-Masse-Dualitätstheorie}. 29. März 2025.
		\bibitem{pascher_emergente_gravitation_2025} Pascher, J. (2025). \href{https://github.com/jpascher/T0-Time-Mass-Duality/tree/main/2/pdf/Deutsch/EmergentGravT0.pdf}{Emergente Gravitation im T0-Modell: Eine umfassende Herleitung}. 1. April 2025.
		\bibitem{pascher_galaxies_2025} Pascher, J. (2025). \href{https://github.com/jpascher/T0-Time-Mass-Duality/tree/main/2/pdf/Deutsch/MassVarGalaxien.pdf}{Massenvariation in Galaxien: Eine Analyse im T0-Modell mit emergenter Gravitation}. 30. März 2025.
		\bibitem{pascher_temp_2025} Pascher, J. (2025). \href{https://github.com/jpascher/T0-Time-Mass-Duality/tree/main/2/pdf/Deutsch/NatEinheitenAlpha1.pdf}{Anpassung der Temperatureinheiten in natürlichen Einheiten und CMB-Messungen}. 2. April 2025.
		\bibitem{pascher_alphabeta_2025} Pascher, J. (2025). \href{https://github.com/jpascher/T0-Time-Mass-Duality/tree/main/2/pdf/Deutsch/Alpha1Beta1Konsistenz.pdf}{Einheitliches Einheitensystem im T0-Modell: Die Konsistenz von \(\alpha = 1\) und \(\beta = 1\)}. 5. April 2025.
		\bibitem{pascher_feldtheorie_2025} Pascher, J. (2025). \href{https://github.com/jpascher/T0-Time-Mass-Duality/tree/main/2/pdf/Deutsch/FeldtheorieQuanten.pdf}{Feldtheorie und Quantenkorrelationen: Eine neue Perspektive auf Instantanität}. 28. März 2025.
		\bibitem{pascher_planck_2025} Pascher, J. (2025). \href{https://github.com/jpascher/T0-Time-Mass-Duality/tree/main/2/pdf/Deutsch/JenseitsPlanck.pdf}{Reale Konsequenzen der Neuformulierung von Zeit und Masse in der Physik: Jenseits der Planck-Skala}. 24. März 2025.
		\bibitem{pascher_erweiterung_2025} Pascher, J. (2025). \href{https://github.com/jpascher/T0-Time-Mass-Duality/tree/main/2/pdf/Deutsch/NotwendigkeitQMErweiterung.pdf}{Die Notwendigkeit der Erweiterung der Standard-Quantenmechanik und Quantenfeldtheorie}. 27. März 2025.
		\bibitem{rubin1980} Rubin, V. C., Ford Jr, W. K., \& Thonnard, N. (1980). Rotational properties of 21 SC galaxies with a large range of luminosities and radii, from NGC 4605 (R=4kpc) to UGC 2885 (R=122kpc). \textit{The Astrophysical Journal}, 238, 471-487. DOI: 10.1086/158003.
		\bibitem{McGaugh2016} McGaugh, S. S., Lelli, F., \& Schombert, J. M. (2016). Radial acceleration relation in rotationally supported galaxies. \textit{Physical Review Letters}, 117(20), 201101. DOI: 10.1103/PhysRevLett.117.201101.
		\bibitem{Milgrom1983} Milgrom, M. (1983). A modification of the Newtonian dynamics as a possible alternative to the hidden mass hypothesis. \textit{The Astrophysical Journal}, 270, 365-370. DOI: 10.1086/161130.
		\bibitem{Planck2018} Planck Collaboration, Aghanim, N., et al. (2020). Planck 2018 results. VI. Cosmological parameters. \textit{Astronomy \& Astrophysics}, 641, A6. DOI: 10.1051/0004-6361/201833910.
	\end{thebibliography}
	
\end{document}