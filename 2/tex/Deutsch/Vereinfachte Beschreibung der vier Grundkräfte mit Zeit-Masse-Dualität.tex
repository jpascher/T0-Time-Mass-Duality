\documentclass{article}
\usepackage[utf8]{inputenc}
\usepackage[T1]{fontenc}
\usepackage{lmodern}
\usepackage[ngerman]{babel}
\usepackage{amsmath, amssymb, amsthm}
\usepackage{geometry}
\usepackage{xcolor}
\usepackage{tocloft}

% Farbige Links im Inhaltsverzeichnis und im Dokument
\usepackage{hyperref}
\hypersetup{
	colorlinks=true,
	linkcolor=blue,
	filecolor=blue,
	citecolor=blue, 
	urlcolor=blue,
	bookmarks=true,
	bookmarksopen=true,
	pdftitle={Vereinfachte Beschreibung der Grundkräfte mit Zeit-Masse-Dualität},
	pdfauthor={Johann Pascher},
}

% cleveref muss nach hyperref geladen werden
\usepackage{cleveref}

\geometry{a4paper, margin=2cm}

% Komma in deutscher Notation für Zahlen
\newcommand{\dcomma}{,}

\title{Vereinfachte Beschreibung der Grundkräfte mit Zeit-Masse-Dualität}
\author{Johann Pascher}
\date{26. März 2025}

\begin{document}
	\maketitle
	
	\tableofcontents
	\newpage
	
	\section{Vereinheitlichte Lagrange-Dichte mit dualem Zeit-Masse-Konzept}
	
	Die Lagrange-Dichte für die fundamentalen Wechselwirkungen lässt sich in vereinfachter Form darstellen, die die Zeit-Masse-Dualität berücksichtigt:
	
	\begin{equation}
		\mathcal{L}_\text{gesamt} = \mathcal{L}_\text{SM} + \mathcal{L}_\text{Higgs} + \mathcal{L}_\text{intrinsisch},
	\end{equation}
	
	wobei:
	\begin{itemize}
		\item $\mathcal{L}_\text{SM}$ die Lagrange-Dichte des Standardmodells (starke, elektromagnetische und schwache Kräfte) darstellt,
		\item $\mathcal{L}_\text{Higgs}$ die Lagrange-Dichte des Higgs-Feldes ist,
		\item $\mathcal{L}_\text{intrinsisch}$ die neue Lagrange-Dichte für die intrinsische Zeit beschreibt, die auch gravitationelle Effekte implizit enthält.
	\end{itemize}
	
	Die Gravitation wird in diesem Ansatz nicht als separate Kraft hinzugenommen, da sie bereits durch die intrinsische Zeitfelddynamik natürlich emergiert.
	
	\subsection{Standardmodell}
	Die Lagrange-Dichte des Standardmodells umfasst die starke, elektromagnetische und schwache Kraft und kann dual formuliert werden:
	
	\begin{equation}
		\mathcal{L}_\text{SM} = \mathcal{L}_\text{stark} + \mathcal{L}_\text{em} + \mathcal{L}_\text{schwach},
	\end{equation}
	
	wobei:
	\begin{itemize}
		\item $\mathcal{L}_\text{stark} = -\frac{1}{4} F_{\mu\nu}^a F^{a\mu\nu} + \bar{\psi}(i \gamma^\mu D_\mu - m_\psi(\phi))\psi$ die starke Kernkraft beschreibt,
		\item $\mathcal{L}_\text{em} = -\frac{1}{4} F_{\mu\nu} F^{\mu\nu} + \bar{\psi}(i \gamma^\mu D_\mu - m_\psi(\phi))\psi$ die elektromagnetische Kraft beschreibt,
		\item $\mathcal{L}_\text{schwach} = -\frac{1}{4} W_{\mu\nu}^a W^{a\mu\nu} + \bar{\psi}(i \gamma^\mu D_\mu - m_\psi(\phi))\psi$ die schwache Kernkraft beschreibt.
	\end{itemize}
	
	Die komplementäre Formulierung mit intrinsischer Zeit lautet:
	
	\begin{equation}
		\mathcal{L}_\text{SM-T} = \mathcal{L}_\text{stark-T} + \mathcal{L}_\text{em-T} + \mathcal{L}_\text{schwach-T},
	\end{equation}
	
	wobei die Zeitableitung nun bezüglich der intrinsischen Zeit $T$ ist: $\partial_t \rightarrow \partial_{t/T}$.
	
	\subsection{Higgs-Feld}
	Die Lagrange-Dichte des Higgs-Feldes lautet:
	
	\begin{equation}
		\mathcal{L}_\text{Higgs} = (D_\mu \phi)^\dagger (D^\mu \phi) - V(\phi),
	\end{equation}
	
	wobei $\phi$ das Higgs-Feld ist und $V(\phi) = \mu^2 \phi^\dagger \phi + \lambda (\phi^\dagger \phi)^2$ das Higgs-Potential beschreibt.
	
	In der komplementären Formulierung mit intrinsischer Zeit wird dies zu:
	
	\begin{equation}
		\mathcal{L}_\text{Higgs-T} = (D_{T\mu} \phi_T)^\dagger (D_T^\mu \phi_T) - V_T(\phi_T),
	\end{equation}
	
	wobei die kovariante Ableitung $D_{T\mu}$ die intrinsische Zeit berücksichtigt.
	
	\subsection{Lagrange-Dichte für intrinsische Zeit}
	Die neue Komponente, die die Zeit-Masse-Dualität einbezieht, lautet:
	
	\begin{equation}
		\mathcal{L}_\text{intrinsisch} = \bar{\psi}\left(i\hbar\gamma^0 \frac{\partial}{\partial (t/T)} - i\hbar\gamma^0 \frac{\partial}{\partial t}\right)\psi,
	\end{equation}
	
	wobei $T = \frac{\hbar}{mc^2}$ die intrinsische Zeit ist, die von der Masse des betrachteten Teilchens abhängt.
	
	\section{Vereinfachte Beschreibung der Masseterme mit Zeit-Masse-Dualität}
	
	Die Masseterme von Teilchen können nun in dualen Formen dargestellt werden:
	
	\begin{itemize}
		\item Standardmodell (Zeitdilatation): $m_\psi(\phi) = y_\psi \phi$ mit konstanter Masse und variabler Zeit,
		\item Komplementäres Modell (Massenvariation): $m_\psi(\phi_T) = y_\psi \phi_T \cdot \gamma$ mit absoluter Zeit und variabler Masse,
	\end{itemize}
	
	wobei $\gamma = \frac{1}{\sqrt{1-v^2/c^2}}$ der Lorentz-Faktor ist.
	
	\section{Das Higgs-Feld als universelles Medium mit intrinsischer Zeit}
	
	Das Konzept des Higgs-Feldes als Medium, das alle anderen Teilchen und Felder beeinflusst, wird durch die Vorstellung der intrinsischen Zeit erweitert. Das Higgs-Feld könnte nicht nur für die Massenerzeugung verantwortlich sein, sondern auch für die intrinsische Zeitskala von Teilchen:
	
	\begin{equation}
		T = \frac{\hbar}{m(\phi)c^2} = \frac{\hbar}{y_\psi \phi \cdot c^2}
	\end{equation}
	
	Diese Beziehung zeigt, dass die intrinsische Zeit eines Teilchens umgekehrt proportional zu seiner durch das Higgs-Feld erzeugten Masse ist.
	
	\section{Das Higgs-Feld und das Vakuum: Eine komplexe Beziehung mit intrinsischer Zeit}
	
	Die Beziehung zwischen dem Higgs-Feld und dem Vakuum wird mit dem Konzept der intrinsischen Zeit komplexer. Die Vakuumenergie könnte neu interpretiert werden als:
	
	\begin{equation}
		E_\text{Vakuum} = \sum_i \frac{\hbar \omega_i}{2} = \sum_i \frac{\hbar}{2T_i}
	\end{equation}
	
	Diese Formulierung verknüpft die Vakuumenergie direkt mit der intrinsischen Zeit von Quantenfluktuationen.
	
	\section{Quantenverschränkung und Nichtlokalität in der Zeit-Masse-Dualität}
	
	Die scheinbare Instantaneität in der Quantenverschränkung kann durch die Zeit-Masse-Dualität neu interpretiert werden:
	
	\begin{itemize}
		\item Im absoluten Zeitmodell ($T_0$-Modell) treten Korrelationen nicht instantan auf, sondern durch Massenvariation.
		\item Im intrinsischen Zeitmodell würden verschränkte Teilchen unterschiedlicher Massen unterschiedliche Zeitentwicklungen erfahren, die proportional zu ihren intrinsischen Zeitskalen sind.
		\item Für Photonen könnte die intrinsische Zeit definiert werden als $T = \frac{\hbar}{E_{\gamma}} e^{\alpha x}$, wobei $\alpha = \frac{H_0}{c} \approx 2{,}3 \times 10^{-28} \text{ m}^{-1}$ den Energieverlust über die Distanz $x$ berücksichtigt, konsistent mit dem T0-Modell.
	\end{itemize}
	
	\section{Kosmologische Implikationen der Zeit-Masse-Dualität}
	
	Der Zeit-Masse-Dualitätsrahmen bietet natürliche Erklärungen für mehrere kosmologische Phänomene durch folgende Schlüsselparameter:
	
	\begin{itemize}
		\item Der Absorptionskoeffizient $\alpha = \frac{H_0}{c} \approx 2{,}3 \times 10^{-28} \text{ m}^{-1}$ bestimmt die Rate des Energieverlusts von Photonen an das dunkle Energiefeld und erklärt die kosmologische Rotverschiebung jenseits der Standard-Doppler-Interpretation.
		
		\item Der Parameter $\kappa \approx 4{,}8 \times 10^{-7} \text{ GeV/cm}\cdot\text{s}^{-2}$ charakterisiert die Stärke des dunklen Energiefeldes in der galaktischen Dynamik und liefert ein modifiziertes Gravitationspotential, das flache Rotationskurven ohne dunkle Materie erklären kann:
		\[
		\Phi(r) = -\frac{GM}{r} + \kappa r
		\]
		
		\item Die dimensionslose Kopplungskonstante $\beta \approx 10^{-3}$ beschreibt die Wechselwirkungsstärke zwischen dem dunklen Energiefeld und baryonischer Materie. Diese Parameter hängen zusammen durch:
		\[
		\kappa = \frac{\beta^2 H_0^2 M_{\text{Pl}}^2}{c^2 \rho_0}
		\]
		wobei $\rho_0$ die kritische Dichte des Universums ist.
	\end{itemize}
	
	Dies führt zur Vorhersage, dass Bell-Tests mit Teilchen unterschiedlicher Massen oder Photonen unterschiedlicher Frequenzen messbare Verzögerungen in den Korrelationen aufdecken könnten, proportional zum Massenverhältnis $\frac{m_1}{m_2}$ oder Energieverhältnis $\frac{E_1}{E_2}$.
	
	\section{Zusammenfassung der vereinheitlichten Theorie}
	
	Die vollständige vereinheitlichte Theorie kann durch folgende Wirkung beschrieben werden:
	
	\begin{equation}
		S_\text{vereinheitlicht} = \int \left( \mathcal{L}_\text{standard} + \mathcal{L}_\text{komplementär} + \mathcal{L}_\text{Kopplung} \right) d^4x
	\end{equation}
	
	wobei:
	\begin{align}
		\mathcal{L}_\text{standard} &= \mathcal{L}_\text{SM} + (D_\mu \phi)^\dagger (D^\mu \phi) - V(\phi) \\
		\mathcal{L}_\text{komplementär} &= \mathcal{L}_\text{SM-T} + (D_{T\mu} \phi_T)^\dagger (D_T^\mu \phi_T) - V_T(\phi_T) \\
		\mathcal{L}_\text{Kopplung} &= \int \mathcal{D}[\Psi] \, \Psi^* \left( i\hbar \frac{\partial}{\partial t} - i\hbar \frac{\partial}{\partial (t/T)} \right) \Psi
	\end{align}
	
	Diese vereinheitlichte Theorie bietet mehrere bedeutende Vorteile:
	\begin{itemize}
		\item Sie überbrückt Lücken zwischen Quantenmechanik und Quantenfeldtheorie.
		\item Sie bietet eine neue Perspektive auf Quantenverschränkung und Nichtlokalität.
		\item Sie eröffnet neue Wege für die Quantengravitation.
		\item Sie ermöglicht tiefere Einblicke in das Higgs-Feld und das Vakuum.
		\item Sie führt zu experimentell überprüfbaren Vorhersagen.
	\end{itemize}
	
	\section{Experimentelle Überprüfbarkeit}
	
	Die vorgeschlagene vereinheitlichte Theorie mit Zeit-Masse-Dualität führt zu mehreren experimentell überprüfbaren Vorhersagen:
	
	\begin{enumerate}
		\item Messung des Photonenenergieverlusts konsistent mit $\alpha = \frac{H_0}{c}$ bei kosmologischen Distanzen
		\item Nachweis modifizierter Gravitationspotentiale in Galaxien charakterisiert durch $\kappa \approx 4{,}8 \times 10^{-7} \text{ GeV/cm}\cdot\text{s}^{-2}$
		\item Präzisionstests der Materie-dunkle-Energie-Kopplungskonstante $\beta \approx 10^{-3}$
		\item Massenabhängige Zeitentwicklung in Quantensystemen, messbar als unterschiedliche Kohärenzzeiten.
		\item Unterschiede in der Verschränkungsgeschwindigkeit für Teilchen unterschiedlicher Massen.
		\item Skalenabhängige Gravitationskonstante korreliert mit intrinsischer Zeit.
		\item Modifizierte Energie-Impuls-Beziehung für sehr massive Teilchen.
		\item Messbare Abweichungen in Hochpräzisionsexperimenten, die typischerweise durch Zeitdilatation erklärt werden.
	\end{enumerate}
	
	\section{Verweise auf weitere Arbeiten}
	
	Die hier vorgestellte vereinheitlichte Theorie baut auf einer Reihe detaillierter Studien auf, die verschiedene Aspekte der Zeit-Masse-Dualität und ihre Anwendungen behandeln.
	
	\section{Literaturverzeichnis}
	
	\begin{thebibliography}{99}
		
		\bibitem{pascher1} Pascher, J. (2025). Complementary Extensions of Physics: Absolute Time and Intrinsic Time.
		
		\bibitem{pascher2} Pascher, J. (2025). A Model with Absolute Time and Variable Energy: A Comprehensive Investigation of the Foundations.
		
		\bibitem{pascher3} Pascher, J. (2025). Extensions of Quantum Mechanics through Intrinsic Time.
		
		\bibitem{pascher4} Pascher, J. (2025). Integration of Time-Mass Duality into Quantum Field Theory.
		
		\bibitem{pascher5} Pascher, J. (2025). Dynamic Mass of Photons and Their Implications for Nonlocality.
		
		\bibitem{pascher6} Pascher, J. (2025). Fundamental Constants and Their Derivation from Natural Units.
		
		\bibitem{pascher7} Pascher, J. (2025). Real Consequences of Reformulating Time and Mass in Physics: Beyond the Planck Scale.
		
		\bibitem{rotation} Rubin, V. C., Ford, W. K. (1970). Rotation of the Andromeda Nebula from a Spectroscopic Survey of Emission Regions. The Astrophysical Journal, 159, 379.
		
		\bibitem{nfw} Navarro, J. F., Frenk, C. S., White, S. D. M. (1996). The Structure of Cold Dark Matter Halos. The Astrophysical Journal, 462, 563.
		
		\bibitem{tully} Tully, R. B., Fisher, J. R. (1977). A new method of determining distances to galaxies. Astronomy and Astrophysics, 54, 661.
		
		\bibitem{bullet} Clowe, D., Bradač, M., Gonzalez, A. H., et al. (2006). A Direct Empirical Proof of the Existence of Dark Matter. The Astrophysical Journal, 648, L109.
		
		\bibitem{supernova} Perlmutter, S., et al. (1999). Measurements of $\Omega$ and $\Lambda$ from 42 High-Redshift Supernovae. The Astrophysical Journal, 517, 565.
		
		\bibitem{riess} Riess, A. G., et al. (1998). Observational Evidence from Supernovae for an Accelerating Universe and a Cosmological Constant. The Astronomical Journal, 116, 1009.
		
		\bibitem{planck} Planck Collaboration. (2020). Planck 2018 results. VI. Cosmological parameters. Astronomy \& Astrophysics, 641, A6.
		
		\bibitem{cmb} Bennett, C. L., et al. (2013). Nine-year Wilkinson Microwave Anisotropy Probe (WMAP) Observations: Final Maps and Results. The Astrophysical Journal Supplement Series, 208, 20.
		
		\bibitem{bao} Eisenstein, D. J., et al. (2005). Detection of the Baryon Acoustic Peak in the Large-Scale Correlation Function of SDSS Luminous Red Galaxies. The Astrophysical Journal, 633, 560.
		
		\bibitem{quintessence} Caldwell, R. R., Dave, R., Steinhardt, P. J. (1998). Cosmological Imprint of an Energy Component with General Equation of State. Physical Review Letters, 80, 1582.
		
		\bibitem{euclid} Laureijs, R., et al. (2011). Euclid Definition Study Report. ESA/SRE(2011)12.
		
		\bibitem{tired} Zwicky, F. (1929). On the Red Shift of Spectral Lines through Interstellar Space. Proceedings of the National Academy of Sciences, 15, 773.
		
		\bibitem{alfa} Webb, J. K., et al. (2011). Indications of a Spatial Variation of the Fine Structure Constant. Physical Review Letters, 107, 191101.
		
		\bibitem{vacuum} Weinberg, S. (1989). The Cosmological Constant Problem. Reviews of Modern Physics, 61, 1.
		
		\bibitem{scalar} Fujii, Y., Maeda, K. (2003). The Scalar-Tensor Theory of Gravitation. Cambridge University Press.
		
		\bibitem{lambda} Carroll, S. M. (2001). The Cosmological Constant. Living Reviews in Relativity, 4, 1.
	\end{thebibliography}
	
\end{document}