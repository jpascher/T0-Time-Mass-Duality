\documentclass[12pt,a4paper]{article}
\usepackage[utf8]{inputenc}
\usepackage[T1]{fontenc}
\usepackage[ngerman]{babel}
\usepackage{lmodern}
\usepackage{csquotes}
\usepackage{amsmath}
\usepackage{amssymb}
\usepackage{physics}
\usepackage{geometry}
\usepackage{tocloft}
\usepackage{xcolor}
\usepackage{graphicx,tikz,pgfplots}
\pgfplotsset{compat=1.18}
\usepackage{booktabs}
\usepackage{siunitx}
\usepackage{amsthm}
\usepackage[colorlinks=true, linkcolor=blue, citecolor=blue, urlcolor=blue]{hyperref}
\usepackage{cleveref}
\usepackage{fancyhdr}

\geometry{a4paper, margin=2cm}

% Headers and Footers
\pagestyle{fancy}
\fancyhf{}
\fancyhead[L]{Johann Pascher}
\fancyhead[R]{Zeit-Masse-Dualität}
\fancyfoot[C]{\thepage}
\renewcommand{\headrulewidth}{0.4pt}
\renewcommand{\footrulewidth}{0.4pt}

% Table of Contents Styling
\renewcommand{\cftsecfont}{\color{blue}}
\renewcommand{\cftsubsecfont}{\color{blue}}
\renewcommand{\cftsecpagefont}{\color{blue}}
\renewcommand{\cftsubsecpagefont}{\color{blue}}
\setlength{\cftsecindent}{1cm}
\setlength{\cftsubsecindent}{2cm}

% Custom commands (consistent with other documents)
\newcommand{\Tfield}{T(x)}
\newcommand{\betaT}{\beta_{\text{T}}}
\newcommand{\alphaEM}{\alpha_{\text{EM}}}
\newcommand{\alphaW}{\alpha_{\text{W}}}
\newcommand{\Mpl}{M_{\text{Pl}}}
\newcommand{\Tzerot}{T_0(\Tfield)}
\newcommand{\Tzero}{T_0}
\newcommand{\vecx}{\vec{x}}
\newcommand{\gammaf}{\gamma_{\text{Lorentz}}}
\newcommand{\DhiggsT}{\Tfield (\partial_\mu + ig A_\mu) \Phi + \Phi \partial_\mu \Tfield}

\newtheorem{theorem}{Satz}[section]
\newtheorem{proposition}[theorem]{Proposition}

\title{Dynamische Masse von Photonen und ihre Implikationen für Nichtlokalität im T0-Modell}
\author{Johann Pascher}
\date{25. März 2025}

\begin{document}
	
	\maketitle
	
	\begin{abstract}
		Diese Arbeit untersucht die Implikationen der Zuweisung einer dynamischen, frequenzabhängigen effektiven Masse zu Photonen im Rahmen des T0-Modells der Zeit-Masse-Dualität, das absolute Zeit und variable Masse postuliert. Durch die Annahme \(m_\gamma = \omega\) in natürlichen Einheiten wird eine energieabhängige intrinsische Zeit eingeführt, die Nichtlokalität und Kausalität beeinflusst. Die Theorie baut auf der fundamentalen Beziehung des T0-Modells \(\Tfield = \frac{\hbar}{\max(m c^2, \omega)}\) mit der Dimension \([E^{-1}]\) auf und wird durch experimentelle Vorhersagen bezüglich wellenlängenabhängiger Rotverschiebung und energieabhängiger Quantenkorrelationen unterstützt.
	\end{abstract}
	
	\tableofcontents
	\newpage
	
	\section{Einführung}
	Diese Arbeit analysiert die Implikationen einer dynamischen, frequenzabhängigen effektiven Masse für Photonen im T0-Modell der Quantenmechanik, das absolute Zeit und variable Masse annimmt \cite{pascher_zeit_masse_2025, pascher_zeit_2025}. Während Photonen konventionell als masselose Teilchen betrachtet werden, schlägt das T0-Modell vor, ihnen eine effektive Masse zuzuweisen, die direkt proportional zu ihrer Frequenz ist. Dieses Konzept erweitert den intrinsischen Zeitrahmen des Modells, um Nichtlokalität und Kausalität in Quantensystemen zu untersuchen.
	
	Die Idee, dass Photonen eine effektive Masse besitzen könnten, ist nicht völlig neu – verschiedene theoretische Rahmenwerke haben diese Möglichkeit in Betracht gezogen, insbesondere im Kontext modifizierter Elektrodynamik \cite{de_broglie1940, proca1936}. Das T0-Modell nähert sich diesem jedoch aus einer einzigartigen Perspektive, indem es die effektive Masse des Photons als natürliche Konsequenz des fundamentalen Prinzips der Zeit-Masse-Dualität betrachtet.
	
	Durch die Untersuchung, wie diese effektive Photonmasse mit dem intrinsischen Zeitfeld \(\Tfield\) interagiert, gewinnen wir neue Erkenntnisse über Quantenverschränkung und die scheinbare Nichtlokalität von Quantenmessungen. Diese Perspektive bietet eine potenzielle Lösung für die Spannung zwischen Quantenmechanik und spezieller Relativitätstheorie bezüglich instantaner Wirkung auf Distanz, wie in Bells Theorem erforscht \cite{bell}.
	
	\section{Natürliche Einheiten als Grundlage}
	\subsection{Definition natürlicher Einheiten}
	\begin{theorem}[Natürliche Einheiten]
		In einem System mit \(\hbar = c = G = 1\) gelten die folgenden dimensionalen Beziehungen für physikalische Größen:
		\begin{align}
			[L] &= [E^{-1}] \\
			[T] &= [E^{-1}] \\
			[M] &= [E]
		\end{align}
		wobei \([L]\) Länge, \([T]\) Zeit, \([M]\) Masse und \([E]\) Energie repräsentiert.
	\end{theorem}
	
	Diese dimensionale Analyse ist entscheidend, um die Konsistenz der Formulierung des T0-Modells aufrechtzuerhalten. Mit Energie als fundamentaler Dimension können alle anderen physikalischen Größen in Bezug auf Energie ausgedrückt werden, was einen einheitlichen dimensionalen Rahmen schafft \cite{pascher_alpha_2025}.
	
	\subsection{Bedeutung für die Masse-Energie-Äquivalenz}
	Im T0-Modell ist Masse keine statische Eigenschaft, sondern eine dynamische Größe, die durch \(\Tfield = \frac{\hbar}{m c^2}\) mit dem intrinsischen Zeitfeld verknüpft ist \cite{pascher_lagrange_2025}. Für Photonen schlagen wir eine effektive Masse vor, die direkt proportional zu ihrer Energie ist:
	\begin{equation}
		m_\gamma = \omega
	\end{equation}
	wobei \(\omega\) die Winkelgeschwindigkeit des Photons ist. Diese Beziehung ist dimensional konsistent mit \(E = \hbar \omega\) bei Verwendung natürlicher Einheiten (\(\hbar = 1\)), da Energie und Masse dieselbe Dimension \([E]\) teilen.
	
	Dieses Konzept der effektiven Masse ermöglicht es, Photonen und massive Teilchen innerhalb eines einheitlichen Rahmens zu behandeln, in dem beide der gleichen fundamentalen Beziehung zwischen Masse und intrinsischer Zeit folgen. Das Konzept baut auf Einsteins Masse-Energie-Äquivalenz auf \cite{einstein}, erweitert diese jedoch in eine neuartige Richtung, indem es Photonen eine effektive Masse zuweist, die ihre intrinsischen zeitlichen Eigenschaften direkt beeinflusst.
	
	\section{Zeitmodelle in der Quantenmechanik}
	\subsection{Einschränkungen des Standardmodells}
	Die standardmäßige Schrödinger-Gleichung nimmt einen universellen Zeitparameter an, der gleichmäßig auf alle Quantensysteme angewendet wird:
	\begin{equation}
		i\hbar\frac{\partial\psi}{\partial t} = H\psi
	\end{equation}
	
	Dieser Ansatz behandelt Zeit als externen Parameter anstelle eines Operators, was eine Asymmetrie zwischen Raum und Zeit in der Quantenmechanik erzeugt. Diese Asymmetrie wurde als konzeptionelle Einschränkung der standardmäßigen Quantentheorie bemerkt \cite{feynman}, und das T0-Modell adressiert dies durch die Einführung des intrinsischen Zeitfeldes \(\Tfield\) \cite{pascher_erweiterung_2025}.
	
	\subsection{Das T0-Modell mit absoluter Zeit}
	Im T0-Modell ist Energie fundamental mit einer konstanten intrinsischen Zeit \(T_0\) durch die Beziehung verknüpft:
	\begin{equation}
		E = \frac{\hbar}{T_0}
	\end{equation}
	
	Für massive Teilchen ist das intrinsische Zeitfeld durch \(\Tfield = \frac{\hbar}{m c^2}\) gegeben, das invers mit der Masse variiert. Diese Beziehung ist zentral für das Konzept der Zeit-Masse-Dualität und führt zu einer modifizierten Schrödinger-Gleichung, wie in \cite{pascher_lagrange_2025} abgeleitet:
	
	\begin{equation}
		i\hbar \Tfield \frac{\partial\psi}{\partial t} + i\hbar \psi \frac{\partial \Tfield}{\partial t} = H\psi
	\end{equation}
	
	Diese modifizierte Gleichung berücksichtigt die teilchenspezifische intrinsische Zeitskala und ermöglicht eine nuanciertere Behandlung der Quantendynamik.
	
	\subsection{Erweiterung für Photonen}
	Für Photonen erstreckt sich das Konzept der intrinsischen Zeit natürlich auf eine energieabhängige intrinsische Zeit:
	\begin{equation}
		\Tfield = \frac{\hbar}{m_\gamma c^2} = \frac{\hbar}{\omega c^2} = \frac{\hbar}{\omega} = \frac{1}{\omega}
	\end{equation}
	wobei die letzte Gleichheit in natürlichen Einheiten mit \(\hbar = c = 1\) gilt. Dies bleibt konsistent mit der vorgeschlagenen effektiven Masse \(m_\gamma = \omega\) und stimmt mit der Behandlung massiver Teilchen überein.
	
	Diese Formulierung impliziert, dass Photonen mit höherer Energie kürzere intrinsische Zeitskalen haben, eine Vorhersage, die beobachtbare Konsequenzen für hochenergetische astrophysikalische Phänomene und Experimente in der Quantenoptik haben könnte \cite{pascher_emergente_gravitation_2025}.
	
	\section{Vereinheitlichung im T0-Modell}
	Um die Behandlung massiver Teilchen und Photonen innerhalb eines einzigen Rahmens zu vereinheitlichen, definieren wir das intrinsische Zeitfeld als:
	\begin{equation}
		\Tfield = \frac{\hbar}{\max(m c^2, \omega)}
	\end{equation}
	
	Für massive Teilchen in Ruhe oder bei niedrigen Geschwindigkeiten dominiert \(m c^2\), und wir erhalten die standardmäßige T0-Modell-Beziehung. Für Photonen oder ultrarelativistische Teilchen wird \(\omega\) zum bestimmenden Faktor. Dieser einheitliche Ausdruck erhält die dimensionale Konsistenz mit \(\Tfield\), das die Dimension \([E^{-1}]\) in natürlichen Einheiten hat.
	
	Diese Vereinheitlichung bietet einen nahtlosen Übergang zwischen der Behandlung massiver und masseloser Teilchen und adressiert eine langjährige konzeptionelle Trennung in der Quantentheorie. Sie bietet auch neue Perspektiven auf die Quantenfeldtheorie, in der Teilchen als Anregungen von Feldern mit zugehörigen intrinsischen Zeitskalen betrachtet werden können \cite{pascher_feldtheorie_2025}.
	
	\section{Implikationen für Nichtlokalität und Verschränkung}
	\subsection{Energieabhängige Korrelationen}
	Die energieabhängige intrinsische Zeit \(\Tfield\) für Photonen führt zu interessanten Implikationen für die Quantenverschränkung. In einem System mit verschränkten Photonen unterschiedlicher Frequenzen \(\omega_1\) und \(\omega_2\) wäre die Differenz in ihren intrinsischen Zeiten:
	\begin{equation}
		\Delta \Tfield = \left|\frac{1}{\omega_1} - \frac{1}{\omega_2}\right|
	\end{equation}
	
	Diese Zeitdifferenz legt nahe, dass Quantenkorrelationen zwischen verschränkten Photonen möglicherweise nicht wirklich instantan sind, sondern eine leichte Verzögerung im Zusammenhang mit ihrem Energiedifferenz erfahren könnten. Obwohl extrem klein für typische Laborenergien, könnte dieser Effekt für verschränkte Systeme mit hochenergetischen Photonen oder über kosmologische Distanzen signifikant werden.
	
	Diese Perspektive legt nahe, dass Quanten-Nichtlokalität aus intrinsischen Zeitdifferenzen entstehen könnte, anstatt eine echte Wirkung auf Distanz darzustellen, was möglicherweise die Spannung mit der speziellen Relativitätstheorie löst. Der Mechanismus weist konzeptionelle Ähnlichkeit mit dem Energieverlustmechanismus auf, der für die Rotverschiebung im T0-Modell verantwortlich ist \cite{pascher_messdifferenzen_2025}.
	
	\subsection{\(\betaT\) im T0-Modell}
	Im T0-Modell wird die wellenlängenabhängige Komponente der Rotverschiebung durch den Parameter \(\betaT\) charakterisiert, mit \(\betaT^{\text{SI}} \approx 0,008\) in SI-Einheiten und \(\betaT^{\text{nat}} = 1\) in natürlichen Einheiten \cite{pascher_params_2025}. Diese Werte sind mathematisch äquivalent und repräsentieren dieselbe physikalische Realität, ausgedrückt in verschiedenen Einheitensystemen.
	
	Die Umwandlung zwischen diesen Werten ist gegeben durch:
	\begin{equation}
		\betaT^{\text{SI}} = \betaT^{\text{nat}} \cdot \frac{\xi \cdot l_{P,\text{SI}}}{r_{0,\text{SI}}}
	\end{equation}
	
	wobei \(\xi \approx 1,33 \times 10^{-4}\) ein dimensionsloser Parameter ist, der die charakteristische Längenskala \(r_0 = \xi \cdot l_P\) definiert \cite{pascher_temp_2025, pascher_alphabeta_2025}.
	
	Die Ableitung von \(\betaT\) ist im T0-Modell durch die Beziehung gut etabliert:
	\begin{equation}
		\betaT^{\text{nat}} = \frac{\lambda_h^2 v^2}{16\pi^3 m_h^2 \xi}{16\pi^3 m_h^2 \xi}
	\end{equation}
	
	Mit \(\xi = \frac{\lambda_h^2 v^2}{16\pi^3 m_h^2}\) erhalten wir natürlich \(\betaT^{\text{nat}} = 1\). Die Wahl zwischen der Verwendung von \(\betaT^{\text{SI}}\) und \(\betaT^{\text{nat}}\) hängt ausschließlich vom verwendeten Einheitensystem ab und spiegelt keine Unsicherheit in der theoretischen Grundlage wider.
	
	Dieser Parameter erscheint in der Formel für die wellenlängenabhängige Rotverschiebung:
	\begin{equation}
		z(\lambda) = z_0 \left(1 + \betaT \ln\frac{\lambda}{\lambda_0}\right)
	\end{equation}
	
	die eine charakteristische Vorhersage des T0-Modells darstellt, die durch hochpräzise spektroskopische Beobachtungen getestet werden könnte \cite{pascher_messdifferenzen_2025}.
	
	\begin{figure}[h]
		\centering
		\begin{tikzpicture}
			\begin{axis}[
				xlabel={Energie [eV]},
				ylabel={Zeit [eV\(^{-1}\)]},
				xlabel style={font=\large},
				ylabel style={font=\large},
				tick label style={font=\normalsize},
				xmin=0, xmax=10,
				ymin=0, ymax=10,
				legend pos=north east,
				legend style={font=\large},
				grid=both,
				minor tick num=1
				]
				\addplot[blue, ultra thick, domain=0.1:10, samples=100] {1/x};
				\legend{\(T = E^{-1}\)}
			\end{axis}
		\end{tikzpicture}
		\caption{Energieabhängige intrinsische Zeit für Photonen im T0-Modell, die die inverse Beziehung zwischen Energie und intrinsischer Zeit zeigt.}
		\label{fig:energy_time}
	\end{figure}
	
	\section{Experimentelle Überprüfung}
	Das Konzept der dynamischen Masse für Photonen im T0-Modell führt zu mehreren experimentell überprüfbaren Vorhersagen:
	
	\begin{itemize}
		\item \textbf{Frequenzabhängige Bell-Tests:} Experimente könnten entworfen werden, um potenzielle Zeitverzögerungen in Quantenkorrelationen zwischen verschränkten Photonen unterschiedlicher Frequenzen zu messen. Die vorhergesagte Verzögerung \(\Delta \Tfield = \left|\frac{1}{\omega_1} - \frac{1}{\omega_2}\right|\) wäre extrem klein, könnte aber mit ultraprecisen Zeitmessungen in der Quantenoptik nachweisbar sein.
		
		\item \textbf{Wellenlängenabhängige Rotverschiebung:} Die Formel \(z(\lambda) = z_0 \left(1 + \betaT \ln\frac{\lambda}{\lambda_0}\right)\) sagt eine charakteristische Wellenlängenabhängigkeit der kosmologischen Rotverschiebung voraus, die durch hochpräzise spektroskopische Beobachtungen entfernter Quellen über mehrere Wellenlängenbänder getestet werden könnte, wie in \cite{pascher_messdifferenzen_2025} diskutiert.
		
		\item \textbf{Ausbreitung hochenergetischer Photonen:} Die energieabhängige intrinsische Zeit könnte zu subtilen energieabhängigen Ausbreitungseffekten für hochenergetische Gammastrahlen führen, die über kosmologische Distanzen reisen, potenziell nachweisbar mit Gammastrahlenteleskopen, die entfernte energetische Ereignisse wie Gammastrahlenausbrüche beobachten \cite{pascher_galaxies_2025}.
	\end{itemize}
	
	Diese experimentellen Tests würden entscheidende Validierung für die Behandlung von Photonen im T0-Modell und ihre Implikationen für die Quanten-Nichtlokalität bieten. Die spezifischen Vorhersagen unterscheiden sich quantitativ sowohl von der standardmäßigen Quantenmechanik als auch von konventionellen Quantengravitationsansätzen und bieten klare Unterscheidungskriterien.
	
	\section{Physik jenseits der Lichtgeschwindigkeit}
	Das T0-Modell mit dynamischer Photonmasse legt die Möglichkeit einer modifizierten Dispersionsrelation für Photonen nahe:
	\begin{equation}
		E^2 = (m_\gamma c^2)^2 + (p c)^2 + \alpha_c \frac{p^4 c^2}{E_P^2}
	\end{equation}
	
	wobei \(\alpha_c\) eine dimensionslose Kopplungskonstante und \(E_P\) die Planck-Energie ist. Der zusätzliche Term repräsentiert eine Quantengravitationskorrektur, die nur bei sehr hohen Energien signifikant wird.
	
	Diese modifizierte Relation könnte potenzielle Anomalien in der Ausbreitung ultrahochenergetischer kosmischer Strahlen und Gammastrahlen erklären. Sie könnte durch präzise Zeitbeobachtungen von Photonen aus entfernten Gammastrahlenausbrüchen über verschiedene Energiebänder getestet werden, da höherenergetische Photonen aufgrund des energieabhängigen Terms leicht unterschiedliche Ausbreitungszeiten erfahren würden.
	
	Dergleichen Modifikationen der Dispersionsrelationen wurden in verschiedenen Quantengravitationsansätzen betrachtet, aber das T0-Modell bietet eine einzigartige Perspektive, indem es sie direkt mit dem Konzept des intrinsischen Zeitfeldes verbindet. Weitere theoretische Entwicklung dieser Ideen wird in \cite{pascher_planck_2025} präsentiert, das Physik jenseits der Planck-Skala untersucht.
	
	\section{Schlussfolgerung}
	Die dynamische effektive Masse von Photonen im T0-Modell bietet eine neuartige Perspektive auf Quanten-Nichtlokalität als ein emergentes Phänomen, das durch energieabhängige intrinsische Zeit angetrieben wird. Durch die Zuweisung einer frequenzabhängigen effektiven Masse \(m_\gamma = \omega\) etablieren wir einen einheitlichen Rahmen zur Behandlung sowohl massiver als auch masseloser Teilchen durch das intrinsische Zeitfeld \(\Tfield = \frac{\hbar}{\max(m c^2, \omega)}\).
	
	Dieser Ansatz legt nahe, dass Quantenkorrelationen in verschränkten Systemen möglicherweise nicht wirklich instantan sind, sondern subtile energieabhängige Verzögerungen aufweisen könnten, was potenziell die Spannung zwischen Quanten-Nichtlokalität und relativistischer Kausalität löst. Die wellenlängenabhängige Rotverschiebungsformel \(z(\lambda) = z_0 \left(1 + \betaT \ln\frac{\lambda}{\lambda_0}\right)\) liefert eine charakteristische experimentelle Signatur dieses Rahmens.
	
	Die Behandlung von Photonen im T0-Modell erhöht seine erklärende Kraft und schafft einen einheitlicheren theoretischen Rahmen, der Quantenmechanik, Elektrodynamik und Gravitation verbindet. Zukünftige experimentelle Tests, insbesondere hochpräzise Messungen der wellenlängenabhängigen Rotverschiebung und energieabhängiger Quantenkorrelationen, werden entscheidend sein, um diese theoretischen Erkenntnisse zu validieren.
	
	\begin{thebibliography}{99}
		\bibitem{pascher_zeit_2025} Pascher, J. (2025). \href{https://github.com/jpascher/T0-Time-Mass-Duality/tree/main/2/pdf/Deutsch/ZeitEmergentQM.pdf}{Zeit als emergente Eigenschaft in der Quantenmechanik: Eine Verbindung zwischen Relativität, Feinstrukturkonstante und Quantendynamik}. 23. März 2025.
		\bibitem{pascher_zeit_masse_2025} Pascher, J. (2025). \href{https://github.com/jpascher/T0-Time-Mass-Duality/tree/main/2/pdf/Deutsch/ZeitMasseNeuerBlick.pdf}{Zeit und Masse: Ein neuer Blick auf alte Formeln – und Befreiung von traditionellen Zwängen}. 22. März 2025.
		\bibitem{pascher_galaxies_2025} Pascher, J. (2025). \href{https://github.com/jpascher/T0-Time-Mass-Duality/tree/main/2/pdf/Deutsch/MassVarGalaxien.pdf}{MassenVariation in Galaxien: Eine Analyse im T0-Modell mit emergenter Gravitation}. 30. März 2025.
		\bibitem{pascher_messdifferenzen_2025} Pascher, J. (2025). \href{https://github.com/jpascher/T0-Time-Mass-Duality/tree/main/2/pdf/Deutsch/MessdifferenzenT0Standard.pdf}{Kompensatorische und additive Effekte: Eine Analyse der Messunterschiede zwischen dem T0-Modell und dem \(\Lambda\)CDM-Standardmodell}. 2. April 2025.
		\bibitem{pascher_params_2025} Pascher, J. (2025). \href{https://github.com/jpascher/T0-Time-Mass-Duality/tree/main/2/pdf/Deutsch/ZeitMasseT0Params.pdf}{Zeit-Masse-Dualitätstheorie (T0-Modell): Ableitung der Parameter \(\kappa\), \(\alpha\) und \(\beta\)}. 4. April 2025.
		\bibitem{pascher_temp_2025} Pascher, J. (2025). \href{https://github.com/jpascher/T0-Time-Mass-Duality/tree/main/2/pdf/Deutsch/TempEinheitenCMB.pdf}{Anpassung der Temperatureinheiten in natürlichen Einheiten und CMB-Messungen}. 2. April 2025.
		\bibitem{pascher_alpha_2025} Pascher, J. (2025). \href{https://github.com/jpascher/T0-Time-Mass-Duality/tree/main/2/pdf/Deutsch/NatEinheitenAlpha1.pdf}{Energie als fundamentale Einheit: Natürliche Einheiten mit \(\alphaEM = 1\) im T0-Modell}. 26. März 2025.
		\bibitem{pascher_alphabeta_2025} Pascher, J. (2025). \href{https://github.com/jpascher/T0-Time-Mass-Duality/tree/main/2/pdf/Deutsch/Alpha1Beta1Konsistenz.pdf}{Einheitliches Einheitensystem im T0-Modell: Die Konsistenz von \(\alphaEM = 1\) und \(\betaT = 1\)}. 5. April 2025.
		\bibitem{pascher_lagrange_2025} Pascher, J. (2025). \href{https://github.com/jpascher/T0-Time-Mass-Duality/tree/main/2/pdf/Deutsch/MathZeitMasseLagrange.pdf}{Von Zeitdilatation zur Massenvariation: Mathematische Kernformulierungen der Zeit-Masse-Dualitätstheorie}. 29. März 2025.
		\bibitem{pascher_erweiterung_2025} Pascher, J. (2025). \href{https://github.com/jpascher/T0-Time-Mass-Duality/tree/main/2/pdf/Deutsch/NotwendigkeitQMErweiterung.pdf}{Die Notwendigkeit der Erweiterung der Standard-Quantenmechanik und Quantenfeldtheorie}. 27. März 2025.
		\bibitem{pascher_feldtheorie_2025} Pascher, J. (2025). \href{https://github.com/jpascher/T0-Time-Mass-Duality/tree/main/2/pdf/Deutsch/FeldtheorieQuanten.pdf}{Feldtheorie und Quantenkorrelationen: Eine neue Perspektive auf Instantaneität}. 28. März 2025.
		\bibitem{pascher_emergente_gravitation_2025} Pascher, J. (2025). \href{https://github.com/jpascher/T0-Time-Mass-Duality/tree/main/2/pdf/Deutsch/EmergentGravT0.pdf}{Emergente Gravitation im T0-Modell: Eine umfassende Ableitung}. 1. April 2025.
		\bibitem{pascher_planck_2025} Pascher, J. (2025). \href{https://github.com/jpascher/T0-Time-Mass-Duality/tree/main/2/pdf/Deutsch/JenseitsPlanck.pdf}{Reale Konsequenzen der Neuformulierung von Zeit und Masse in der Physik: Jenseits der Planck-Skala}. 24. März 2025.
		\bibitem{einstein} Einstein, A. (1905). \textit{Zur Elektrodynamik bewegter Körper}. \textit{Annalen der Physik}, 322(10), 891-921.
		\bibitem{planck} Planck, M. (1901). \textit{Über das Gesetz der Energieverteilung im Normalspektrum}. \textit{Annalen der Physik}, 309(3), 553-563.
		\bibitem{bell} Bell, J. S. (1964). \textit{Zum Einstein-Podolsky-Rosen-Paradoxon}. \textit{Physics}, 1(3), 195-200.
		\bibitem{feynman} Feynman, R. P. (1985). \textit{QED: Die seltsame Theorie des Lichts und der Materie}. Princeton University Press.
		\bibitem{de_broglie1940} de Broglie, L. (1940). \textit{Die Wellenmechanik des Photons: Eine neue Theorie des Lichts}. Hermann \& Cie.
		\bibitem{proca1936} Proca, A. (1936). \textit{Zur Wellentheorie der positiven und negativen Elektronen}. \textit{Journal de Physique et le Radium}, 7(8), 347-353.
	\end{thebibliography}
	
\end{document}