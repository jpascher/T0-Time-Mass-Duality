\documentclass[a4paper,12pt]{article}
\usepackage[utf8]{inputenc}
\usepackage[T1]{fontenc}
\usepackage[english]{babel}
\usepackage{lmodern}
\usepackage{amsmath}
\usepackage{amssymb}
\usepackage{geometry}
\usepackage{tikz}
\usepackage{pgfplots}
\pgfplotsset{compat=1.18}
\usepackage{xcolor}
\usepackage{tocloft}
\usepackage{amsthm}
\usepackage[colorlinks=true, linkcolor=blue, filecolor=blue, citecolor=blue, urlcolor=blue, bookmarks=true, bookmarksopen=true, pdftitle={Time and Mass: A New Look at Old Formulas – and Liberation from Traditional Shackles}, pdfauthor={Johann Pascher}]{hyperref}
\usepackage{cleveref}

\geometry{a4paper, margin={2.5cm}}

\renewcommand{\cftsecfont}{\color{blue}}
\renewcommand{\cftsubsecfont}{\color{blue}}
\renewcommand{\cftsecpagefont}{\color{blue}}
\renewcommand{\cftsubsecpagefont}{\color{blue}}
\setlength{\cftsecindent}{1cm}
\setlength{\cftsubsecindent}{2cm}

\newtheorem{theorem}{Theorem}[section]
\newtheorem{proposition}[theorem]{Proposition}

\title{Time and Mass: A New Look at Old Formulas – and Liberation from Traditional Shackles}
\author{Johann Pascher}
\date{March 25, 2025}

\begin{document}
	
	\maketitle
	
	\begin{abstract}
		This work presents a new perspective on time and mass, the time-mass dualism, challenging traditional views in quantum mechanics and relativity. By extending the use of natural units, physical constants are reinterpreted as dimensionless ratios of energy. Without introducing new equations, the approach reveals the incompleteness of existing theories and proposes a more unified, intuitive description of reality, with implications for quantum gravity, entanglement, and cosmological phenomena.
	\end{abstract}
	
	\tableofcontents
	\newpage
	
	\section{Introduction: Traditional Views and the Obscured Perspective}
	Physics has achieved remarkable success with abstract concepts like quantum fields and spacetime curvature. But have we perhaps strayed too far from an \emph{intuitive}, \emph{real} description of the world? Traditional perspectives, particularly our choice of units, may have obscured a deeper, \emph{more unified} understanding of nature. This approach seeks to step back to the fundamentals—and free physics from unnecessary shackles.
	
	\section{Natural Constants and Units: More Than Arbitrary Numbers?}
	Our units (meters, seconds, kilograms) are historically derived and practical for everyday use, but are they truly \emph{fundamental}? Natural laws feature \emph{natural constants} (e.g., the speed of light \(c\), reduced Planck’s constant \(\hbar\), gravitational constant \(G\), fine-structure constant \(\alpha\)). Physicists often set \(c = 1\) and \(\hbar = 1\) ("natural units") to simplify equations. Yet, the traditional view often treats these constants as \emph{independent}, \emph{given} quantities. Is this truly the case? Or do they conceal a deeper connection?
	
	\section{The Time-Mass Dualism: An Alternative Perspective}
	\begin{theorem}[Time-Mass Dualism]
		The time-mass dualism proposes:
		\begin{itemize}
			\item Standard view: Constant rest mass, variable time (time dilation).
			\item Alternative: Absolute time, variable mass.
		\end{itemize}
	\end{theorem}
	The \emph{time-mass dualism} offers a new perspective that challenges this traditional view:
	
	*   \textbf{Standard View (Relativity):} An object’s \emph{rest mass} is constant, while \emph{time} is relative (time dilation).
	*   \textbf{Alternative Perspective:} What if \emph{time} is absolute, but \emph{mass} is variable?
	
	Imagine an "internal clock" (\emph{intrinsic time}) for each particle. This clock ticks faster the \emph{heavier} the particle. Lighter particles have a slower internal clock.
	
	\section{All Constants Become Natural: Energy as the Unifying Principle}
	The decisive step is this: The time-mass dualism, combined with an \emph{extended} use of natural units, allows us to express *all* physical constants as \emph{dimensionless numbers}. They become \emph{ratios} of a single fundamental quantity—and that quantity is \emph{energy}. Traditional constants lose their status as independent, given entities; they become \emph{derived} quantities emerging from energy.
	
	\section{No New Formulas, But a Liberated View of Old Ones}
	This approach does \emph{not} introduce entirely new equations. We examine the \emph{same} fundamental formulas of quantum mechanics and relativity—but in a \emph{new reference frame} where all constants are dimensionless, or "natural." This seemingly minor shift has far-reaching consequences, revealing the \emph{limits} and \emph{gaps} in existing theories:
	
	1.  \textbf{Incompleteness of Quantum Mechanics (from Existing Formulas):} The \emph{known} quantum mechanics formulas, transposed into this new system, no longer describe \emph{all} phenomena correctly. They are \emph{incomplete}, failing to fully capture the dynamic interplay of mass, time, and \emph{energy}.
	2.  \textbf{Extension Within the Existing Framework:} Quantum mechanics \emph{must} be extended. Yet, this extension arises not from arbitrary new assumptions but from a \emph{more consistent} application of \emph{existing} principles, particularly energy conservation and the inseparable link between mass and time.
	3.  \textbf{Dual Perspectives as Keys to Reality:} The wave-particle dualism and time-mass dualism are not mere "interpretations." They are \emph{clues} that we overlook or misinterpret aspects of reality when clinging to traditional, restrictive views. They guide us toward a \emph{more real}, \emph{intuitive}, and \emph{unified} description of the physical world.
	
	\section{Concrete Implications: Toward a More Comprehensive Theory}
	This "liberated" view of physics yields concrete implications:
	
	*   \textbf{Quantum Gravity:} Unification based on an \emph{extended} and \emph{more consistent} QM becomes more tangible.
	*   \textbf{Quantum Entanglement:} The interpretation via intrinsic time challenges \emph{current} QM and opens new perspectives.
	*   \textbf{Dark Energy/Matter:} New, \emph{concrete} relationships emerge between mass, energy, and universe expansion, surpassing existing models.
	*   \textbf{Fundamental Constants:} A \emph{deeper} understanding emerges as all constants reduce to \emph{one} fundamental quantity (energy).
	
	\begin{figure}[h]
		\centering
		\begin{tikzpicture}
			\draw[->] (0,0) -- (6,0) node[right] {Mass \(m\)};
			\draw[->] (0,0) -- (0,4) node[above] {Intrinsic Time \(T\)};
			\draw[scale=0.5, domain=0.1:10, smooth, variable=\x, blue, thick] plot ({\x}, {1/\x});
			\node[blue] at (4.5,2) {\(T \propto \frac{1}{m}\)};
		\end{tikzpicture}
		\caption{Relationship between mass and intrinsic time: Lighter particles have a slower internal clock.}
	\end{figure}
	
	\section{Experimental Verification and Conclusion: A New Dawn}
	This approach is not merely theoretical but \emph{experimentally testable}. It makes *different* predictions than the *current*, incomplete QM (e.g., with precision clocks and entangled particles of different masses).
	
	The time-mass dualism, the "naturalization" of all constants, and the resulting extension of quantum mechanics represent a radical yet promising path. They demonstrate that we must \emph{fundamentally} rethink physics—not by discarding proven formulas but by \emph{liberating} them from traditional shackles and returning to a \emph{more real}, \emph{intuitive}, and above all \emph{unified} perspective. It is a dawn of a more comprehensive theory that could unravel the universe’s greatest mysteries.
	
	\bibliographystyle{plain}
	\begin{thebibliography}{2}
		\bibitem{pascher2025} Pascher, J. (2025). \textit{Essential Mathematical Formalisms of the Time-Mass Duality Theory with Lagrangian Densities}. March 29, 2025.
		\bibitem{einstein1905} Einstein, A. (1905). \textit{On the Electrodynamics of Moving Bodies}. Annalen der Physik, 322(10), 891-921.
	\end{thebibliography}
	
\end{document}