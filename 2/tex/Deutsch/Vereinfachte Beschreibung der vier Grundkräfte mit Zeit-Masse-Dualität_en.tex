\documentclass{article}
\usepackage[utf8]{inputenc}
\usepackage[T1]{fontenc}
\usepackage{lmodern}
\usepackage[english]{babel}
\usepackage{amsmath, amssymb, amsthm}
\usepackage{geometry}
\usepackage{xcolor}
\usepackage{tocloft}

% Colored links in table of contents and document
\usepackage{hyperref}
\hypersetup{
	colorlinks=true,
	linkcolor=blue,
	filecolor=blue,
	citecolor=blue, 
	urlcolor=blue,
	bookmarks=true,
	bookmarksopen=true,
	pdftitle={Simplified Description of Fundamental Forces with Time-Mass Duality},
	pdfauthor={Johann Pascher},
}

% cleveref must be loaded after hyperref
\usepackage{cleveref}

\geometry{a4paper, margin=2cm}

% Decimal point in English notation
\newcommand{\dcomma}{.}

\title{Simplified Description of Fundamental Forces with Time-Mass Duality}
\author{Johann Pascher}
\date{March 26, 2025}

\begin{document}
	\maketitle
	
	\tableofcontents
	\newpage
	
	\section{Unified Lagrangian Density with Dual Time-Mass Concept}
	
	The Lagrangian density for fundamental interactions can be presented in simplified form that accounts for time-mass duality:
	
	\begin{equation}
		\mathcal{L}_\text{total} = \mathcal{L}_\text{SM} + \mathcal{L}_\text{Higgs} + \mathcal{L}_\text{intrinsic},
	\end{equation}
	
	where:
	\begin{itemize}
		\item $\mathcal{L}_\text{SM}$ represents the Standard Model Lagrangian density (strong, electromagnetic and weak forces),
		\item $\mathcal{L}_\text{Higgs}$ is the Lagrangian density of the Higgs field,
		\item $\mathcal{L}_\text{intrinsic}$ describes the new Lagrangian density for intrinsic time, which implicitly contains gravitational effects.
	\end{itemize}
	
	Gravity is not added as a separate force in this approach since it naturally emerges through intrinsic time field dynamics.
	
	\subsection{Standard Model}
	The Standard Model Lagrangian density encompasses the strong, electromagnetic and weak forces and can be formulated dually:
	
	\begin{equation}
		\mathcal{L}_\text{SM} = \mathcal{L}_\text{strong} + \mathcal{L}_\text{em} + \mathcal{L}_\text{weak},
	\end{equation}
	
	where:
	\begin{itemize}
		\item $\mathcal{L}_\text{strong} = -\frac{1}{4} F_{\mu\nu}^a F^{a\mu\nu} + \bar{\psi}(i \gamma^\mu D_\mu - m_\psi(\phi))\psi$ describes the strong nuclear force,
		\item $\mathcal{L}_\text{em} = -\frac{1}{4} F_{\mu\nu} F^{\mu\nu} + \bar{\psi}(i \gamma^\mu D_\mu - m_\psi(\phi))\psi$ describes the electromagnetic force,
		\item $\mathcal{L}_\text{weak} = -\frac{1}{4} W_{\mu\nu}^a W^{a\mu\nu} + \bar{\psi}(i \gamma^\mu D_\mu - m_\psi(\phi))\psi$ describes the weak nuclear force.
	\end{itemize}
	
	The complementary formulation with intrinsic time is:
	
	\begin{equation}
		\mathcal{L}_\text{SM-T} = \mathcal{L}_\text{strong-T} + \mathcal{L}_\text{em-T} + \mathcal{L}_\text{weak-T},
	\end{equation}
	
	where the time derivative is now with respect to intrinsic time $T$: $\partial_t \rightarrow \partial_{t/T}$.
	
	\subsection{Higgs Field}
	The Lagrangian density of the Higgs field is:
	
	\begin{equation}
		\mathcal{L}_\text{Higgs} = (D_\mu \phi)^\dagger (D^\mu \phi) - V(\phi),
	\end{equation}
	
	where $\phi$ is the Higgs field and $V(\phi) = \mu^2 \phi^\dagger \phi + \lambda (\phi^\dagger \phi)^2$ describes the Higgs potential.
	
	In the complementary formulation with intrinsic time this becomes:
	
	\begin{equation}
		\mathcal{L}_\text{Higgs-T} = (D_{T\mu} \phi_T)^\dagger (D_T^\mu \phi_T) - V_T(\phi_T),
	\end{equation}
	
	where the covariant derivative $D_{T\mu}$ accounts for intrinsic time.
	
	\subsection{Lagrangian Density for Intrinsic Time}
	The new component incorporating time-mass duality is:
	
	\begin{equation}
		\mathcal{L}_\text{intrinsic} = \bar{\psi}\left(i\hbar\gamma^0 \frac{\partial}{\partial (t/T)} - i\hbar\gamma^0 \frac{\partial}{\partial t}\right)\psi,
	\end{equation}
	
	where $T = \frac{\hbar}{mc^2}$ is the intrinsic time that depends on the mass of the particle under consideration.
	
	\section{Simplified Description of Mass Terms with Time-Mass Duality}
	
	The mass terms of particles can now be presented in dual forms:
	
	\begin{itemize}
		\item Standard Model (time dilation): $m_\psi(\phi) = y_\psi \phi$ with constant mass and variable time,
		\item Complementary Model (mass variation): $m_\psi(\phi_T) = y_\psi \phi_T \cdot \gamma$ with absolute time and variable mass,
	\end{itemize}
	
	where $\gamma = \frac{1}{\sqrt{1-v^2/c^2}}$ is the Lorentz factor.
	
	\section{The Higgs Field as Universal Medium with Intrinsic Time}
	
	The concept of the Higgs field as a medium influencing all other particles and fields is extended by the notion of intrinsic time. The Higgs field might not only be responsible for mass generation but also for the intrinsic time scale of particles:
	
	\begin{equation}
		T = \frac{\hbar}{m(\phi)c^2} = \frac{\hbar}{y_\psi \phi \cdot c^2}
	\end{equation}
	
	This relationship shows that a particle's intrinsic time is inversely proportional to its mass generated by the Higgs field.
	
	\section{The Higgs Field and the Vacuum: A Complex Relationship with Intrinsic Time}
	
	The relationship between the Higgs field and the vacuum becomes more complex with the concept of intrinsic time. The vacuum energy could be reinterpreted as:
	
	\begin{equation}
		E_\text{vacuum} = \sum_i \frac{\hbar \omega_i}{2} = \sum_i \frac{\hbar}{2T_i}
	\end{equation}
	
	This formulation directly links vacuum energy to the intrinsic time of quantum fluctuations.
	
	\section{Quantum Entanglement and Nonlocality in Time-Mass Duality}
	
	The apparent instantaneity in quantum entanglement can be reinterpreted through time-mass duality:
	
	\begin{itemize}
		\item In the absolute time model ($T_0$-model), correlations don't occur instantaneously but through mass variation.
		\item In the intrinsic time model, entangled particles of different masses would experience different time evolutions proportional to their intrinsic time scales.
		\item For photons, intrinsic time could be defined as $T = \frac{\hbar}{E_{\gamma}} e^{\alpha x}$, where $\alpha = \frac{H_0}{c} \approx 2.3 \times 10^{-28} \text{ m}^{-1}$ accounts for energy loss over distance $x$, consistent with the T0-model.
	\end{itemize}
	
	\section{Cosmological Implications of Time-Mass Duality}
	
	The time-mass duality framework provides natural explanations for several cosmological phenomena through the following key parameters:
	
	\begin{itemize}
		\item The absorption coefficient $\alpha = \frac{H_0}{c} \approx 2.3 \times 10^{-28} \text{ m}^{-1}$ determines the rate of photon energy loss to the dark energy field and explains cosmological redshift beyond the standard Doppler interpretation.
		
		\item The parameter $\kappa \approx 4.8 \times 10^{-7} \text{ GeV/cm}\cdot\text{s}^{-2}$ characterizes the strength of the dark energy field in galactic dynamics and provides a modified gravitational potential that can explain flat rotation curves without dark matter:
		\[
		\Phi(r) = -\frac{GM}{r} + \kappa r
		\]
		
		\item The dimensionless coupling constant $\beta \approx 10^{-3}$ describes the interaction strength between the dark energy field and baryonic matter. These parameters are related by:
		\[
		\kappa = \frac{\beta^2 H_0^2 M_{\text{Pl}}^2}{c^2 \rho_0}
		\]
		where $\rho_0$ is the critical density of the universe.
	\end{itemize}
	
	This leads to the prediction that Bell tests with particles of different masses or photons of different frequencies might reveal measurable delays in correlations, proportional to the mass ratio $\frac{m_1}{m_2}$ or energy ratio $\frac{E_1}{E_2}$.
	
	\section{Summary of the Unified Theory}
	
	The complete unified theory can be described by the following action:
	
	\begin{equation}
		S_\text{unified} = \int \left( \mathcal{L}_\text{standard} + \mathcal{L}_\text{complementary} + \mathcal{L}_\text{coupling} \right) d^4x
	\end{equation}
	
	where:
	\begin{align}
		\mathcal{L}_\text{standard} &= \mathcal{L}_\text{SM} + (D_\mu \phi)^\dagger (D^\mu \phi) - V(\phi) \\
		\mathcal{L}_\text{complementary} &= \mathcal{L}_\text{SM-T} + (D_{T\mu} \phi_T)^\dagger (D_T^\mu \phi_T) - V_T(\phi_T) \\
		\mathcal{L}_\text{coupling} &= \int \mathcal{D}[\Psi] \, \Psi^* \left( i\hbar \frac{\partial}{\partial t} - i\hbar \frac{\partial}{\partial (t/T)} \right) \Psi
	\end{align}
	
	This unified theory offers several significant advantages:
	\begin{itemize}
		\item It bridges gaps between quantum mechanics and quantum field theory.
		\item It provides a new perspective on quantum entanglement and nonlocality.
		\item It opens new pathways for quantum gravity.
		\item It enables deeper insights into the Higgs field and the vacuum.
		\item It leads to experimentally testable predictions.
	\end{itemize}
	
	\section{Experimental Verifiability}
	
	The proposed unified theory with time-mass duality leads to several experimentally testable predictions:
	
	\begin{enumerate}
		\item Measurement of photon energy loss consistent with $\alpha = \frac{H_0}{c}$ at cosmological distances
		\item Detection of modified gravitational potentials in galaxies characterized by $\kappa \approx 4.8 \times 10^{-7} \text{ GeV/cm}\cdot\text{s}^{-2}$
		\item Precision tests of the matter-dark energy coupling constant $\beta \approx 10^{-3}$
		\item Mass-dependent time evolution in quantum systems, measurable as different coherence times.
		\item Differences in entanglement speed for particles of different masses.
		\item Scale-dependent gravitational constant correlated with intrinsic time.
		\item Modified energy-momentum relation for very massive particles.
		\item Measurable deviations in high-precision experiments typically explained by time dilation.
	\end{enumerate}
	
	\section{References to Further Works}
	
	The unified theory presented here builds upon a series of detailed studies addressing various aspects of time-mass duality and its applications.
	
	\section{Bibliography}
	
	\begin{thebibliography}{99}
		
		\bibitem{pascher1} Pascher, J. (2025). Complementary Extensions of Physics: Absolute Time and Intrinsic Time.
		
		\bibitem{pascher2} Pascher, J. (2025). A Model with Absolute Time and Variable Energy: A Comprehensive Investigation of the Foundations.
		
		\bibitem{pascher3} Pascher, J. (2025). Extensions of Quantum Mechanics through Intrinsic Time.
		
		\bibitem{pascher4} Pascher, J. (2025). Integration of Time-Mass Duality into Quantum Field Theory.
		
		\bibitem{pascher5} Pascher, J. (2025). Dynamic Mass of Photons and Their Implications for Nonlocality.
		
		\bibitem{pascher6} Pascher, J. (2025). Fundamental Constants and Their Derivation from Natural Units.
		
		\bibitem{pascher7} Pascher, J. (2025). Real Consequences of Reformulating Time and Mass in Physics: Beyond the Planck Scale.
		
		\bibitem{rotation} Rubin, V. C., Ford, W. K. (1970). Rotation of the Andromeda Nebula from a Spectroscopic Survey of Emission Regions. The Astrophysical Journal, 159, 379.
		
		\bibitem{nfw} Navarro, J. F., Frenk, C. S., White, S. D. M. (1996). The Structure of Cold Dark Matter Halos. The Astrophysical Journal, 462, 563.
		
		\bibitem{tully} Tully, R. B., Fisher, J. R. (1977). A new method of determining distances to galaxies. Astronomy and Astrophysics, 54, 661.
		
		\bibitem{bullet} Clowe, D., Bradač, M., Gonzalez, A. H., et al. (2006). A Direct Empirical Proof of the Existence of Dark Matter. The Astrophysical Journal, 648, L109.
		
		\bibitem{supernova} Perlmutter, S., et al. (1999). Measurements of $\Omega$ and $\Lambda$ from 42 High-Redshift Supernovae. The Astrophysical Journal, 517, 565.
		
		\bibitem{riess} Riess, A. G., et al. (1998). Observational Evidence from Supernovae for an Accelerating Universe and a Cosmological Constant. The Astronomical Journal, 116, 1009.
		
		\bibitem{planck} Planck Collaboration. (2020). Planck 2018 results. VI. Cosmological parameters. Astronomy \& Astrophysics, 641, A6.
		
		\bibitem{cmb} Bennett, C. L., et al. (2013). Nine-year Wilkinson Microwave Anisotropy Probe (WMAP) Observations: Final Maps and Results. The Astrophysical Journal Supplement Series, 208, 20.
		
		\bibitem{bao} Eisenstein, D. J., et al. (2005). Detection of the Baryon Acoustic Peak in the Large-Scale Correlation Function of SDSS Luminous Red Galaxies. The Astrophysical Journal, 633, 560.
		
		\bibitem{quintessence} Caldwell, R. R., Dave, R., Steinhardt, P. J. (1998). Cosmological Imprint of an Energy Component with General Equation of State. Physical Review Letters, 80, 1582.
		
		\bibitem{euclid} Laureijs, R., et al. (2011). Euclid Definition Study Report. ESA/SRE(2011)12.
		
		\bibitem{tired} Zwicky, F. (1929). On the Red Shift of Spectral Lines through Interstellar Space. Proceedings of the National Academy of Sciences, 15, 773.
		
		\bibitem{alfa} Webb, J. K., et al. (2011). Indications of a Spatial Variation of the Fine Structure Constant. Physical Review Letters, 107, 191101.
		
		\bibitem{vacuum} Weinberg, S. (1989). The Cosmological Constant Problem. Reviews of Modern Physics, 61, 1.
		
		\bibitem{scalar} Fujii, Y., Maeda, K. (2003). The Scalar-Tensor Theory of Gravitation. Cambridge University Press.
		
		\bibitem{lambda} Carroll, S. M. (2001). The Cosmological Constant. Living Reviews in Relativity, 4, 1.
	\end{thebibliography}
	
\end{document}