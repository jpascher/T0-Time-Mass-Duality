\documentclass[12pt,a4paper]{article}
\usepackage[utf8]{inputenc}
\usepackage[T1]{fontenc}
\usepackage[ngerman]{babel}
\usepackage{lmodern}
\usepackage{amsmath}
\usepackage{amssymb}
\usepackage{physics}
\usepackage{hyperref}
\usepackage{tcolorbox}
\usepackage{booktabs}
\usepackage{enumitem}
\usepackage[table,xcdraw]{xcolor}
\usepackage[left=2cm,right=2cm,top=2cm,bottom=2cm]{geometry}
\usepackage{pgfplots}
\pgfplotsset{compat=1.18}
\usepackage{graphicx}
\usepackage{float}
\usepackage{fancyhdr}
\usepackage{siunitx}
\usepackage{array}
\usepackage{cleveref}

% Kopf- und Fußzeilen
\pagestyle{fancy}
\fancyhf{}
\fancyhead[L]{Johann Pascher}
\fancyhead[R]{T0 vs. Erweitertes SM}
\fancyfoot[C]{\thepage}
\renewcommand{\headrulewidth}{0.4pt}
\renewcommand{\footrulewidth}{0.4pt}

% Benutzerdefinierte Befehle
\newcommand{\Tfield}{T(x)}
\newcommand{\alphaEM}{\alpha_{\text{EM}}}
\newcommand{\alphaW}{\alpha_{\text{W}}}
\newcommand{\betaT}{\beta_{\text{T}}}
\newcommand{\Mpl}{M_{\text{Pl}}}
\newcommand{\Tzerot}{T_0(\Tfield)}
\newcommand{\Tzero}{T_0}
\newcommand{\vecx}{\vec{x}}
\newcommand{\gammaf}{\gamma_{\text{Lorentz}}}
\newcommand{\DhiggsT}{\Tfield (\partial_\mu + ig A_\mu) \Phi + \Phi \partial_\mu \Tfield}
\newcommand{\LCDM}{\Lambda\text{CDM}}
\newcommand{\DTmu}{D_{T,\mu}}
\newcommand{\calL}{\mathcal{L}}
\newcommand{\deq}{\displaystyle}
\newcommand{\e}{\mathrm{e}}

\hypersetup{
	colorlinks=true,
	linkcolor=blue,
	citecolor=blue,
	urlcolor=blue,
	pdftitle={Konzeptueller Vergleich von T0-Modell und Erweitertem Standardmodell},
	pdfauthor={Johann Pascher},
	pdfsubject={Theoretische Physik},
	pdfkeywords={T0-Modell, Erweitertes Standardmodell, Skalarfeld, Intrinsisches Zeitfeld}
}

\begin{document}
	
	\title{Konzeptueller Vergleich von T0-Modell und Erweitertem Standardmodell: \\Feldtheoretische vs. Dimensionale Ansätze}
	\author{Johann Pascher\\
		Abteilung für Kommunikationstechnik, \\Höhere Technische Bundeslehranstalt (HTL), Leonding, Österreich\\
		\texttt{johann.pascher@gmail.com}}
	\date{\today}
	
	\maketitle
	
	\begin{abstract}
		Diese Arbeit präsentiert einen detaillierten konzeptuellen Vergleich zwischen dem T0-Modell und dem Erweiterten Standardmodell, mit Fokus auf ihre jeweilige Behandlung des intrinsischen Zeitfeldes und Skalarfeldes. Obwohl mathematisch äquivalent, stellen diese Rahmenwerke grundlegend verschiedene konzeptuelle Ansätze zur Vereinigung von Quantenmechanik und allgemeiner Relativitätstheorie dar. Wir analysieren den ontologischen Status, die physikalische Interpretation und die mathematische Formulierung beider Modelle, mit besonderer Aufmerksamkeit auf ihre gravitationellen Aspekte. Wir zeigen, dass der feldtheoretische Ansatz des T0-Modells im Vergleich zu den dimensionalen Erweiterungen des Erweiterten Standardmodells eine größere konzeptuelle Einfachheit und intuitive Klarheit bietet. Dieser Vergleich offenbart, dass, obwohl beide Rahmenwerke identische experimentelle Vorhersagen liefern, einschließlich eines statischen Universums ohne Expansion, in dem Rotverschiebung durch Energieabschwächung statt durch kosmische Expansion auftritt, das T0-Modell eine elegantere und konzeptuell kohärentere Beschreibung der physikalischen Realität bietet, indem es ein 3D-durchdringendes intrinsisches Zeitfeld anstelle einer fünfdimensionalen Interpretation verwendet. Die Implikationen für unser Verständnis von Quantengravitation und Kosmologie werden diskutiert.
	\end{abstract}
	\newpage
	\tableofcontents
	\newpage
	\section{Einleitung}
	\label{sec:introduction}
	
	Das Streben nach einer vereinheitlichten Theorie, die sowohl Quantenmechanik als auch allgemeine Relativitätstheorie kohärent beschreibt, bleibt eine der bedeutendsten Herausforderungen in der theoretischen Physik. Traditionelle Ansätze zu diesem Problem beinhalten typischerweise die Erweiterung des Standardmodells (SM) der Teilchenphysik oder die Modifikation der allgemeinen Relativitätstheorie, was oft zu komplexen mathematischen Strukturen führt, die zwar formal korrekt sind, aber konzeptuelle Klarheit und intuitive Zugänglichkeit vermissen lassen. Unter den neuartigen theoretischen Rahmenwerken, die entwickelt wurden, um diese Herausforderung anzugehen, stechen das T0-Modell und das Erweiterte Standardmodell (ESM) als zwei mathematisch äquivalente, aber konzeptuell unterschiedliche Ansätze hervor.
	
	Beide Rahmenwerke zielen darauf ab, Phänomene zu erklären, die derzeit der dunklen Materie und dunklen Energie zugeschrieben werden, während sie eine konsistente Behandlung von Quanten- und Gravitationswechselwirkungen bieten. Sie unterscheiden sich jedoch grundlegend in ihren konzeptuellen Grundlagen. Das T0-Modell postuliert absolute Zeit und variable Masse, mit einem durchdringenden intrinsischen Zeitfeld \(\Tfield\), das gravitationelle Effekte vermittelt. Im Gegensatz dazu behält das Erweiterte Standardmodell die konventionellen Vorstellungen von relativer Zeit und konstanter Masse bei, während es ein Skalarfeld \(\Theta\) einführt, das die Einstein-Feldgleichungen modifiziert.
	
	Diese Arbeit untersucht die konzeptuellen Unterschiede zwischen diesen Rahmenwerken, mit besonderem Fokus auf:
	
	\begin{itemize}
		\item Den ontologischen Status und die physikalische Interpretation der jeweiligen Felder
		\item Die mathematische Formulierung von Gravitationswechselwirkungen
		\item Die potenzielle Interpretation dieser Felder in Bezug auf zusätzliche Dimensionen
		\item Die relative konzeptuelle Klarheit und Eleganz jedes Ansatzes
	\end{itemize}
	
	Unsere Analyse zeigt, dass, während das Erweiterte Standardmodell eine mathematisch gültige Formulierung darstellt, das T0-Modell überlegene konzeptuelle Klarheit bietet, indem es einen feldtheoretischen Ansatz verwendet, anstatt abstrakte dimensionale Erweiterungen einzuführen. Dieser Unterschied ist nicht nur ästhetischer Natur, sondern hat tiefgreifende Implikationen dafür, wie wir fundamentale physikalische Phänomene interpretieren und verstehen.
	
	\section{Mathematische Äquivalenz der beiden Rahmenwerke}
	\label{sec:mathematical_equivalence}
	
	Bevor wir in die konzeptuellen Unterschiede eintauchen, ist es wichtig, die mathematische Äquivalenz des T0-Modells und des Erweiterten Standardmodells festzustellen. Diese Äquivalenz stellt sicher, dass jeder Unterschied zwischen ihnen rein konzeptueller Natur ist und nicht empirisch, da beide Rahmenwerke identische experimentelle Vorhersagen liefern.
	
	\subsection{Transformation zwischen Rahmenwerken}
	\label{subsec:transformation}
	
	Die mathematische Äquivalenz zwischen den beiden Rahmenwerken kann durch eine wohldefinierte Transformation demonstriert werden. Das Skalarfeld \(\Theta\) im Erweiterten Standardmodell und das intrinsische Zeitfeld \(\Tfield\) im T0-Modell stehen in folgender Beziehung:
	
	\begin{equation}
		\Theta(\vecx) \propto \ln\left(\frac{\Tfield}{\Tzero}\right)
	\end{equation}
	
	wobei \(\Tzero\) ein Referenzwert des Zeitfeldes ist. Diese Transformation ermöglicht es uns, jede Gleichung in einem Rahmenwerk in ihr Äquivalent im anderen zu konvertieren.
	
	\subsection{Gravitationspotential}
	\label{subsec:gravitational_potential}
	
	Beide Rahmenwerke sagen ein identisches modifiziertes Gravitationspotential voraus:
	
	\begin{equation}
		\Phi(r) = -\frac{GM}{r} + \kappa r
	\end{equation}
	
	wobei \(\kappa \approx 4,8 \times 10^{-11} \, \text{m/s}^2\) in SI-Einheiten. Dieses Potential erklärt auf natürliche Weise galaktische Rotationskurven ohne dunkle Materie und kosmische Beschleunigung ohne dunkle Energie, wenn auch durch unterschiedliche konzeptuelle Mechanismen in jedem Rahmenwerk.
	
	\subsection{Feldgleichungen}
	\label{subsec:field_equations}
	
	Im T0-Modell ist die Feldgleichung für das intrinsische Zeitfeld unter statischen Bedingungen:
	
	\begin{equation}
		\nabla^2\Tfield \approx -\frac{\rho}{\Tfield^2}
	\end{equation}
	
	Im Erweiterten Standardmodell sind die modifizierten Einstein-Feldgleichungen:
	
	\begin{equation}
		G_{\mu\nu} + \kappa g_{\mu\nu} = 8\pi G T_{\mu\nu} + \nabla_{\mu}\Theta\nabla_{\nu}\Theta - \frac{1}{2}g_{\mu\nu}(\nabla_{\sigma}\Theta\nabla^{\sigma}\Theta)
	\end{equation}
	
	Diese Gleichungen führen, obwohl unterschiedlich formuliert, zu den gleichen physikalischen Vorhersagen, wenn die Transformation zwischen \(\Tfield\) und \(\Theta\) angewendet wird.
	
	\section{Das intrinsische Zeitfeld des T0-Modells}
	\label{sec:t0_time_field}
	
	Das T0-Modell stellt eine revolutionäre Neukonzeptualisierung der fundamentalen Physik dar, indem es die traditionelle Beziehung zwischen Zeit und Masse umkehrt. Dieser Abschnitt untersucht die Natur und Eigenschaften des intrinsischen Zeitfeldes \(\Tfield\), das als Eckpfeiler dieses Rahmenwerks dient.
	
	\subsection{Definition und physikalische Grundlage}
	\label{subsec:time_field_definition}
	
	Das intrinsische Zeitfeld ist definiert als:
	
	\begin{equation}
		\Tfield = \frac{\hbar}{\max(m(\vecx,t)c^2, \omega(\vecx,t))}
	\end{equation}
	
	wobei:
	\begin{itemize}
		\item Für massive Teilchen: \(\Tfield = \frac{\hbar}{m(\vecx,t)c^2}\), wobei \(m(\vecx,t)\) die positions- und zeitabhängige Masse ist
		\item Für Photonen: \(\Tfield = \frac{\hbar}{\omega(\vecx,t)}\), wobei \(\omega(\vecx,t)\) die positions- und zeitabhängige Photonenenergie/Frequenz ist
	\end{itemize}
	
	Dieses Feld repräsentiert eine intrinsische temporale Eigenschaft von Materie und Energie, die kontinuierlich über Raum und Zeit variiert. Die dynamische Natur dieser Definition ist fundamental für das T0-Modell, bei dem schwere Teilchen schnellere interne Uhren (kleineres \(\Tfield\)) und leichte Teilchen langsamere interne Uhren (größeres \(\Tfield\)) haben. Entscheidend ist, dass die Masse \(m(\vecx,t)\) und die Frequenz \(\omega(\vecx,t)\) keine statischen Werte sind, sondern gemäß den lokalen Feldbedingungen mit Position und Zeit variieren.
	
	\subsection{Feldtheoretische Natur}
	\label{subsec:field_theoretic_nature}
	
	Das intrinsische Zeitfeld \(\Tfield\) wird als Skalarfeld konzeptualisiert, das den dreidimensionalen Raum durchdringt und ausfüllt. An jedem Punkt in Raum und Zeit definiert das Feld eine lokale zeitliche Qualität, die beeinflusst, wie sich Materie und Energie entwickeln. Dieser von Natur aus dynamische feldtheoretische Ansatz bietet mehrere Schlüsselvorteile:
	
	\begin{itemize}
		\item Er behält die vertraute dreidimensionale räumliche Struktur bei, ohne zusätzliche räumliche Dimensionen zu benötigen
		\item Er erlaubt lokale Variationen temporaler Eigenschaften, während ein einheitlicher, absoluter Zeitrahmen erhalten bleibt
		\item Er bietet einen natürlichen Mechanismus für Gravitationswechselwirkungen durch Feldgradienten
		\item Er integriert sich nahtlos mit der Quantenfeldtheorie durch modifizierte Lagrange-Funktionen
		\item Er berücksichtigt die inhärente dynamische Natur der Massenvariation über Raum und Zeit
	\end{itemize}
	
	Die vollständige Lagrange-Funktion für das intrinsische Zeitfeld umfasst Terme für sowohl kinetische als auch potentielle Energie, die seine dynamische, feldtheoretische Natur repräsentieren:
	
	\begin{equation}
		\mathcal{L}_{\text{intrinsisch}} = \frac{1}{2}\partial_{\mu}\Tfield\partial^{\mu}\Tfield - \frac{1}{2}\Tfield^2 - \frac{\rho(\vecx,t)}{\Tfield}
	\end{equation}
	
	Diese Lagrange-Funktion bestimmt, wie sich das Feld ausbreitet und mit Materie und Energie interagiert, wobei der Schwerpunkt auf seinem zeitlich entwickelnden Charakter liegt und nicht auf der Darstellung statischer Eigenschaften.
	
	\subsection{Gravitationelle Emergenz}
	\label{subsec:gravitational_emergence_t0}
	
	Eines der elegantesten Merkmale des T0-Modells ist, wie Gravitation natürlich aus dem intrinsischen Zeitfeld entsteht. Das Gravitationspotential entsteht aus dem Logarithmus des Zeitfeldverhältnisses:
	
	\begin{equation}
		\Phi(\vecx,t) = -\ln\left(\frac{\Tfield(\vecx,t)}{\Tzero}\right)
	\end{equation}
	
	wobei \(\Tzero\) ein Referenzwert des Zeitfeldes ist. Diese dynamische Formulierung führt zu dem modifizierten Gravitationspotential:
	
	\begin{equation}
		\Phi(r) = -\frac{GM}{r} + \kappa r
	\end{equation}
	
	Der lineare Term \(\kappa r\) entsteht natürlich aus den Eigenschaften des intrinsischen Zeitfeldes und bietet eine einheitliche Erklärung für Phänomene, die traditionell sowohl der dunklen Materie als auch der dunklen Energie zugeschrieben werden.
	
	Für eine Punktmasse ist die Lösung des Zeitfeldes:
	
	\begin{equation}
		\Tfield(r) = \Tzero\left(1 - \frac{M}{r} + \kappa r\right)
	\end{equation}
	
	Diese Lösung demonstriert die dynamische Natur des Feldes, das kontinuierlich mit der Entfernung von der Masse variiert und auf Änderungen in der Massenverteilung über die Zeit reagiert. Die Gravitationskraft entsteht direkt aus dem Gradienten dieses Feldes:
	
	\begin{equation}
		\vec{F}(\vecx,t) = -\frac{\nabla\Tfield(\vecx,t)}{\Tfield(\vecx,t)}
	\end{equation}
	
	was deutlich macht, dass gravitationelle Effekte von Natur aus dynamische Eigenschaften sind, die aus den raumzeitlichen Variationen im intrinsischen Zeitfeld entstehen.
	
	\section{Das Skalarfeld des Erweiterten Standardmodells}
	\label{sec:esm_scalar_field}
	
	Das Erweiterte Standardmodell (ESM) verfolgt einen anderen Ansatz zur Vereinheitlichung, indem es die konventionellen Vorstellungen von relativer Zeit und konstanter Masse beibehält, während es ein Skalarfeld \(\Theta\) einführt, das den Gravitationssektor modifiziert. Dieser Abschnitt untersucht die Natur und Eigenschaften dieses Skalarfeldes.
	
	\subsection{Definition und Rolle}
	\label{subsec:scalar_field_definition}
	
	Das Skalarfeld \(\Theta\) im Erweiterten Standardmodell ist ein zusätzliches Feld, das an den Gravitationssektor koppelt. Im Gegensatz zum intrinsischen Zeitfeld hat es keine direkte physikalische Interpretation in Bezug auf temporale Eigenschaften. Stattdessen dient es als mathematisches Konstrukt, das die Einstein-Feldgleichungen modifiziert:
	
	\begin{equation}
		G_{\mu\nu} + \kappa g_{\mu\nu} = 8\pi G T_{\mu\nu} + \nabla_{\mu}\Theta\nabla_{\nu}\Theta - \frac{1}{2}g_{\mu\nu}(\nabla_{\sigma}\Theta\nabla^{\sigma}\Theta)
	\end{equation}
	
	Das Feld \(\Theta\) wird durch eine modifizierte Klein-Gordon-Gleichung gesteuert, ähnlich anderen Skalarfeldern in der Quantenfeldtheorie.
	
	\subsection{Geometrische Interpretation}
	\label{subsec:geometrical_interpretation}
	
	Eine potenzielle Interpretation des Skalarfeldes \(\Theta\) ist als Komponente einer höherdimensionalen Geometrie. Diese Interpretation zieht Parallelen zu:
	
	\begin{itemize}
		\item Kaluza-Klein-Theorie, die Gravitation und Elektromagnetismus durch eine fünfte Dimension vereinheitlicht
		\item Bran-Modellen in der Stringtheorie, bei denen unsere vierdimensionale Raumzeit in einen höherdimensionalen Bulk eingebettet ist
		\item Skalar-Tensor-Theorien der Gravitation, bei denen ein Skalarfeld an den metrischen Tensor koppelt
	\end{itemize}
	
	In dieser Sichtweise könnte das Skalarfeld \(\Theta\) die Manifestation einer fünften Dimension darstellen, wobei Feldwerte Positionen in dieser zusätzlichen Dimension entsprechen.
	
	\subsection{Gravitationelle Modifikation}
	\label{subsec:gravitational_modification_esm}
	Das Skalarfeld \(\Theta\) modifiziert die Gravitation durch zusätzliche Terme in den Einstein-Feldgleichungen. Diese Modifikationen führen zu demselben modifizierten Gravitationspotential wie das T0-Modell:
	
	\begin{equation}
		\Phi(r) = -\frac{GM}{r} + \kappa r
	\end{equation}
	
	Die konzeptuelle Interpretation unterscheidet sich jedoch. Im ESM entsteht dieses Potential aus modifizierter Raumzeitkrümmung aufgrund des Einflusses des Skalarfeldes auf die Metrik, anstatt aus intrinsischen temporalen Eigenschaften der Materie.
	
	\section{Konzeptueller Vergleich und Analyse}
	\label{sec:conceptual_comparison}
	
	Nachdem wir die Schlüsselmerkmale beider Rahmenwerke festgestellt haben, führen wir nun einen direkten Vergleich ihrer konzeptuellen Grundlagen und Implikationen durch.
	
	\subsection{Umfassende mathematische Formulierung}
	\label{subsec:comprehensive_math}
	
	Um ein vollständigeres Verständnis beider Rahmenwerke zu bieten, präsentieren wir die detaillierten mathematischen Formulierungen des intrinsischen Zeitfeldes \(\Tfield\) im T0-Modell und des Skalarfeldes \(\Theta\) im Erweiterten Standardmodell.
	
	\subsubsection{Vollständige Formulierung des intrinsischen Zeitfeldes im T0-Modell}
	\label{subsubsec:t0_complete}
	
	Die grundlegende Definition des intrinsischen Zeitfeldes \(\Tfield\) ist:
	
	\begin{equation}
		\Tfield = \frac{\hbar}{\max(mc^2, \omega)}
	\end{equation}
	
	wobei:
	\begin{itemize}
		\item Für massive Teilchen: \(\Tfield = \frac{\hbar}{m(\vecx,t)c^2}\), wobei \(m(\vecx,t)\) die positions- und zeitabhängige Masse ist
		\item Für Photonen: \(\Tfield = \frac{\hbar}{\omega(\vecx,t)}\), wobei \(\omega(\vecx,t)\) die lokale Frequenz/Energie des Photons ist
	\end{itemize}
	
	Es ist entscheidend zu betonen, dass die Werte von Masse und Frequenz in diesen Ausdrücken keine statischen Konstanten sind, sondern dynamische Größen, die mit Position und Zeit variieren. Dies ist ein fundamentaler Aspekt des T0-Modells, der die intrinsische Variabilität der Masse in verschiedenen Gravitationsumgebungen und die sich ändernde Frequenz von Photonen bei der Interaktion mit dem Zeitfeld widerspiegelt.
	
	Die Feldgleichung für \(\Tfield\) unter allgemeinen Bedingungen ist:
	
	\begin{equation}
		\partial_{\mu}\partial^{\mu}\Tfield + \Tfield + \frac{\rho(\vecx,t)}{\Tfield^2} = 0
	\end{equation}
	
	die sich in der statischen Näherung vereinfacht zu:
	
	\begin{equation}
		\nabla^2 \Tfield \approx -\frac{\rho(\vecx)}{\Tfield^2}
	\end{equation}
	
	wobei \(\rho(\vecx)\) die positionsabhängige Massendichte ist.
	
	Die vollständige Lagrange-Funktion für das intrinsische Zeitfeld ist:
	
	\begin{equation}
		\mathcal{L}_{\text{intrinsisch}} = \frac{1}{2}\partial_{\mu}\Tfield\partial^{\mu}\Tfield - \frac{1}{2}\Tfield^2 - \frac{\rho}{\Tfield}
	\end{equation}
	
	Das Gravitationspotential wird abgeleitet als:
	
	\begin{equation}
		\Phi(\vecx) = -\ln\left(\frac{\Tfield}{\Tzero}\right)
	\end{equation}
	
	Für eine Punktmasse ergibt dies:
	
	\begin{equation}
		\Tfield(r) = \Tzero\left(1 - \frac{M}{r} + \kappa r\right)
	\end{equation}
	
	Das resultierende modifizierte Gravitationspotential:
	
	\begin{equation}
		\Phi(r) = -\frac{GM}{r} + \kappa r
	\end{equation}
	
	\subsubsection{Vollständige Formulierung des Skalarfeldes im Erweiterten Standardmodell}
	\label{subsubsec:esm_complete}
	
	Im Gegensatz zum intrinsischen Zeitfeld wird das Skalarfeld \(\Theta(\vecx,t)\) im Erweiterten Standardmodell nicht direkt als Funktion von Masse oder Energie definiert, sondern durch seinen Effekt auf die Einstein-Feldgleichungen:
	
	\begin{equation}
		G_{\mu\nu} + \kappa g_{\mu\nu} = 8\pi G T_{\mu\nu} + \nabla_{\mu}\Theta(\vecx,t)\nabla_{\nu}\Theta(\vecx,t) - \frac{1}{2}g_{\mu\nu}(\nabla_{\sigma}\Theta(\vecx,t)\nabla^{\sigma}\Theta(\vecx,t))
	\end{equation}
	
	Diese Definition betont, dass \(\Theta\) ebenfalls ein dynamisches Feld ist, das mit Position und Zeit variiert. Die Beziehung zwischen dem Skalarfeld \(\Theta\) und dem \(\Tfield\)-Feld des T0-Modells ist:
	
	\begin{equation}
		\Theta(\vecx,t) \propto \ln\left(\frac{\Tfield(\vecx,t)}{\Tzero}\right)
	\end{equation}
	
	Diese logarithmische Beziehung bedeutet, dass die extremen Zustände, die durch die \(\max\)-Funktion im T0-Modell elegant erfasst werden, im ESM durch komplexere mathematische Maschinerie behandelt werden müssen. Der Übergang zwischen reinen Energie-(Wellen-)Zuständen und maximalen Massenzuständen hat nicht die direkte, intuitive Formulierung, die im T0-Modell vorhanden ist. Stattdessen sind diese Übergänge indirekt in der Kopplung zwischen dem Skalarfeld und dem Energie-Impuls-Tensor \(T_{\mu\nu}\) kodiert.
	
	Die Feldgleichung für \(\Theta\) ähnelt einer modifizierten Klein-Gordon-Gleichung:
	
	\begin{equation}
		\partial_{\mu}\partial^{\mu}\Theta(\vecx,t) - \frac{\partial V(\Theta)}{\partial \Theta} = 8\pi G \rho(\vecx,t)
	\end{equation}
	
	die sich unter statischen Bedingungen vereinfacht zu:
	
	\begin{equation}
		\nabla^2 \Theta(\vecx) - \frac{\partial V(\Theta)}{\partial \Theta} = 8\pi G \rho(\vecx)
	\end{equation}
	
	wobei \(V(\Theta)\) das Potential des Skalarfeldes und \(\rho(\vecx,t)\) die positions- und zeitabhängige Massendichte ist.
	
	Die Lagrange-Funktion für das Skalarfeld \(\Theta\) im Erweiterten Standardmodell:
	
	\begin{equation}
		\mathcal{L}_{\Theta} = \frac{1}{2}\partial_{\mu}\Theta(\vecx,t)\partial^{\mu}\Theta(\vecx,t) - V(\Theta) - \Theta(\vecx,t) \cdot \mathcal{R}
	\end{equation}
	
	wobei \(\mathcal{R}\) der Ricci-Skalar ist, der die Kopplung an die Raumzeitkrümmung beschreibt.
	
	Das resultierende modifizierte Gravitationspotential ist identisch mit dem des T0-Modells:
	
	\begin{equation}
		\Phi(r) = -\frac{GM}{r} + \kappa r
	\end{equation}
	
	Es ist jedoch wichtig zu beachten, dass dieses Potential im ESM aus grundlegend anderen Dynamiken entsteht als im T0-Modell, trotz ihrer mathematischen Äquivalenz. Während das T0-Modell eine direkte, intuitive Formulierung der extremen Zustände durch die \(\max\)-Funktion bietet, erfordert das ESM einen komplexeren mathematischen Apparat, um dieselben physikalischen Phänomene zu beschreiben, was die zugrundeliegende physikalische Intuition verschleiert.
	
	\subsubsection{Schlüsselunterschiede in der mathematischen Formulierung}
	\label{subsubsec:math_differences}
	
	Trotz der Tatsache, dass sie zu denselben beobachtbaren Vorhersagen führen, zeigen die mathematischen Formulierungen signifikante konzeptuelle Unterschiede:
	
	\begin{enumerate}
		\item \textbf{Direkte vs. Indirekte Darstellung extremer Zustände:} Das T0-Modell erfasst die beiden extremen physikalischen Zustände (reine Energie und maximale Masse) direkt durch die elegante \(\max\)-Funktion in seiner Definition von \(\Tfield\). Im Gegensatz dazu muss das ESM diese Übergänge durch komplexe Kopplungen zwischen dem Skalarfeld und dem Energie-Impuls-Tensor behandeln, was die zugrundeliegende Physik verschleiert.
		
		\item \textbf{Intuitive physikalische Interpretation:} Das \(\Tfield\)-Feld hat eine direkte physikalische Interpretation als intrinsische Zeitskala, während \(\Theta\) ein abstraktes Skalarfeld ohne klare physikalische Bedeutung ist. Dies macht das T0-Modell intuitiv zugänglicher und konzeptuell klarer.
		
		\item \textbf{Definitorische Klarheit:} \(\Tfield\) wird direkt als Funktion von Masse oder Energie definiert, was eine transparente Verbindung zu physikalischen Größen schafft. Im Gegensatz dazu wird \(\Theta\) indirekt durch seinen Effekt auf die Einstein-Feldgleichungen definiert, was eine Schicht mathematischer Abstraktion hinzufügt.
		
		\item \textbf{Strukturelle Einfachheit:} Die Feldgleichungen für \(\Tfield\) sind einfacher und haben eine klarere physikalische Interpretation als die für \(\Theta\). Die logarithmische Beziehung zwischen den Feldern (\(\Theta \propto \ln(\Tfield/\Tzero)\)) bedeutet, dass das ESM komplexere mathematische Maschinerie benötigt, um dieselben Phänomene zu beschreiben.
		
		\item \textbf{Identische Vorhersagen:} Trotz dieser fundamentalen Unterschiede im Ansatz führen beide zu demselben modifizierten Gravitationspotential \(\Phi(r) = -\frac{GM}{r} + \kappa r\) und identischen beobachtbaren Vorhersagen, aber über völlig unterschiedliche konzeptuelle Wege.
	\end{enumerate}
	
	Diese Unterschiede unterstreichen, warum das T0-Modell überlegene konzeptuelle Klarheit bietet, trotz mathematischer Äquivalenz zum ESM. Die direkte Behandlung extremer Zustände durch die \(\max\)-Funktion ist besonders bedeutsam, da sie einen eleganten Mechanismus zur Beschreibung des kontinuierlichen Spektrums von Energiemanifestationen in der Natur bietet, von reinen Wellen bis zu konzentrierter Masse.
	
	\subsection{Ontologischer Status der Felder}
	\label{subsec:ontological_status}
	
	\begin{table}[ht]
		\centering
		\caption{Ontologischer Vergleich der Felder in T0 und ESM}
		\label{tab:ontological_comparison}
		\begin{tabular}{p{0.45\textwidth}|p{0.45\textwidth}}
			\hline
			\textbf{Intrinsisches Zeitfeld \(\Tfield\) (T0)} & \textbf{Skalarfeld \(\Theta\) (ESM)} \\
			\hline
			Fundamentales Feld, das intrinsische temporale Eigenschaften repräsentiert & Hilfsfeld, das die Standardgravitationstheorie modifiziert \\
			\hline
			Direkte physikalische Interpretation als temporale Qualität & Abstraktes mathematisches Konstrukt ohne klare physikalische Bedeutung \\
			\hline
			Durchdringt 3D-Raum als Feldeigenschaft & Potenziell als fünfte Dimension interpretiert \\
			\hline
			Entsteht natürlich aus dem Zeit-Masse-Dualitätsprinzip & Zur Theorie hinzugefügt ohne klare konzeptuelle Motivation \\
			\hline
		\end{tabular}
	\end{table}
	
	Das T0-Modell weist dem intrinsischen Zeitfeld einen klaren ontologischen Status als fundamentale Eigenschaft der Realität zu, die bestimmt, wie sich Materie und Energie zeitlich entwickeln. Im Gegensatz dazu fehlt dem Skalarfeld im ESM eine klare ontologische Grundlage und dient primär als mathematisches Instrument, um dieselben Vorhersagen zu reproduzieren.
	
	\subsection{Physikalische Interpretation und intuitive Klarheit}
	\label{subsec:physical_interpretation}
	
	Die physikalische Interpretation des intrinsischen Zeitfeldes im T0-Modell bietet überlegene intuitive Klarheit. Es liefert eine direkte, physikalisch bedeutsame Erklärung für:
	
	\begin{itemize}
		\item Warum schwerere Teilchen schnellere interne Dynamiken haben (kleineres \(\Tfield\))
		\item Wie gravitationelle Effekte aus Gradienten in temporalen Eigenschaften entstehen
		\item Warum Quantendekoherenz mit der Masse skaliert
		\item Wie Verschränkung durch geteilte Zeitfeldhistorien funktioniert
	\end{itemize}
	
	Im Gegensatz dazu fehlen dem Skalarfeld \(\Theta\) im ESM diese intuitiven Verbindungen zu physikalischen Phänomenen. Seine Rolle ist primär mathematischer statt konzeptueller Natur, was es schwieriger macht, physikalische Intuition darüber zu entwickeln, wie es gravitationelle und Quanteneffekte vermittelt.
	
	\subsection{Fünfte Dimension vs. feldtheoretische Perspektive}
	\label{subsec:fifth_dimension_vs_field}
	
	Eine potenzielle Interpretation des ESM ist, dass das Skalarfeld \(\Theta\) eine fünfte Dimension jenseits der üblichen vier Dimensionen der Raumzeit darstellt. Diese Interpretation steht jedoch vor mehreren konzeptuellen Herausforderungen:
	
	\begin{itemize}
		\item Wenn \(\Theta\) eine fünfte Dimension darstellt, müsste es dennoch als ein Feld quantifiziert werden, das unseren dreidimensionalen Raum durchdringt
		\item Dies führt die feldtheoretische Perspektive des T0-Modells wieder ein, aber mit zusätzlicher konzeptueller Komplexität
		\item Die dimensionale Interpretation fügt mathematische Abstraktion hinzu, ohne die physikalische Intuition zu verbessern
		\item Ein Feld, das den dreidimensionalen Raum durchdringt, ist konzeptuell einfacher als die Annahme einer zusätzlichen räumlichen Dimension
	\end{itemize}
	
	Das T0-Modell vermeidet diese Komplikationen, indem es direkt einen feldtheoretischen Ansatz innerhalb des vertrauten dreidimensionalen räumlichen Rahmens verwendet. Dieser Ansatz behält konzeptuelle Klarheit bei, während er dieselben mathematischen Ergebnisse erzielt.
	
	\subsection{Theoretische Eleganz und Ökonomie}
	\label{subsec:theoretical_elegance}
	
	Das Prinzip der theoretischen Eleganz—dass unter äquivalenten Theorien die einfachere und konzeptuell kohärentere bevorzugt werden sollte—bevorzugt stark das T0-Modell. Das T0-Modell demonstriert überlegene theoretische Eleganz durch:
	
	\begin{itemize}
		\item Konzeptuelle Einheit: Ein einziges Prinzip (Zeit-Masse-Dualität) liegt allen Modifikationen zugrunde
		\item Ontologische Ökonomie: Keine zusätzlichen Dimensionen sind erforderlich
		\item Interpretative Klarheit: Klare physikalische Bedeutung für alle Komponenten der Theorie
		\item Strukturelle Einfachheit: Feldtheoretischer Ansatz innerhalb des Standard-3D-Raums
	\end{itemize}
	
	Das ESM, obwohl mathematisch äquivalent, erreicht diese Äquivalenz auf Kosten konzeptueller Klarheit. Es bewahrt konventionelle Vorstellungen von relativer Zeit und konstanter Masse, muss aber zusätzliche mathematische Strukturen einführen, denen eine klare physikalische Interpretation fehlt.
	
	\section{Implikationen für Quantengravitation und Kosmologie}
	\label{sec:implications}
	
	Die konzeptuellen Unterschiede zwischen dem T0-Modell und dem Erweiterten Standardmodell haben tiefgreifende Implikationen für unser Verständnis von Quantengravitation und Kosmologie.
	
	\subsection{Ansatz zur Quantengravitation}
	\label{subsec:quantum_gravity}
	
	Der feldtheoretische Ansatz des T0-Modells bietet einen vielversprechenden Weg in Richtung Quantengravitation:
	
	\begin{itemize}
		\item Das intrinsische Zeitfeld \(\Tfield\) kann mit Standard-Quantenfeldtheorie-Techniken quantisiert werden
		\item Gravitationelle Effekte entstehen aus den Quanteneigenschaften des Zeitfeldes
		\item Es entsteht keine fundamentale Unvereinbarkeit zwischen Quantenmechanik und Gravitation
		\item Die Theorie berücksichtigt natürlicherweise sowohl Quanten- als auch Gravitationsphänomene, ohne separate Rahmenwerke zu benötigen
	\end{itemize}
	
	Das ESM, obwohl potenziell kompatibel mit Quantengravitation, führt zusätzliche Komplexität durch sein Skalarfeld und modifizierte Einstein-Gleichungen ein. Diese Komplexität könnte die Entwicklung einer Quantentheorie der Gravitation eher behindern als erleichtern.
	
	\subsection{Kosmologische Interpretation}
	\label{subsec:cosmological_interpretation}
	
	Beide Rahmenwerke sagen ein statisches, ewiges Universum voraus, anstatt ein expandierendes, wobei kosmische Rotverschiebung durch ähnliche Mechanismen erklärt wird:
	
	\begin{itemize}
		\item T0-Modell: Kosmische Rotverschiebung entsteht durch Photonenenergieverlust aufgrund der Interaktion mit dem intrinsischen Zeitfeld während der Ausbreitung
		\item ESM: Kosmische Rotverschiebung entsteht ähnlicherweise durch Photonenenergieverlust, erklärt durch den krümmungsbasierten Einfluss des Skalarfeldes \(\Theta\) auf die Lichtausbreitung
	\end{itemize}
	
	Es ist entscheidend zu betonen, dass keines der Modelle kosmische Expansion beinhaltet. Beide Rahmenwerke beschreiben konsistent ein statisches Universum, in dem Rotverschiebung aufgrund von Lichtenergie-Abschwächung während der Ausbreitung auftritt, nicht durch die Expansion des Raums. Beide Ansätze eliminieren die Notwendigkeit für dunkle Energie und dunkle Materie, aber das T0-Modell bietet einen kohärenteren konzeptuellen Rahmen für das Verständnis dieser Phänomene als emergente Eigenschaften des intrinsischen Zeitfeldes, während das ESM abstraktere mathematische Konstrukte benötigt, um dieselbe Erklärungskraft zu erreichen.
	
	\subsection{Experimentelle Unterscheidung}
	\label{subsec:experimental_discrimination}
	
	Obwohl das T0-Modell und ESM mathematisch äquivalent sind, stellen sie fundamental unterschiedliche Konzeptionen der physikalischen Realität dar. Dies wirft einen wichtigen philosophischen Punkt auf: Mathematisch äquivalente Theorien mit unterschiedlichen ontologischen Verpflichtungen können nicht durch Experimente unterschieden werden.
	
	Jedoch können beide Rahmenwerke kollektiv vom Standard-Kosmologiemodell durch mehrere Schlüsselvorhersagen unterschieden werden:
	
	\begin{itemize}
		\item Wellenlängenabhängige Rotverschiebung: \(z(\lambda) = z_0 (1 + \betaT \ln(\lambda/\lambda_0))\)
		\item Modifizierte CMB-Temperatur-Rotverschiebungs-Relation: \(T(z) = T_0 (1+z)(1+\ln(1+z))\)
		\item Abwesenheit dunkler Materie in galaktischen Rotationskurven
		\item Statisches Universummodell ohne Expansion (beide Rahmenwerke sagen Energieabschwächung statt expandierenden Raum voraus)
		\item Modifiziertes Gravitationspotential, das kosmische Beobachtungen ohne dunkle Energie erklärt
	\end{itemize}
	
	Diese Vorhersagen bieten klare experimentelle Wege, um beide Rahmenwerke gegen das Standardkosmologiemodell zu validieren, auch wenn sie nicht zwischen den Rahmenwerken selbst unterscheiden können. Sowohl das T0-Modell als auch das ESM machen identische Vorhersagen für diese Phänomene, wenn auch durch unterschiedliche konzeptuelle Mechanismen.
	
	\section{Verbindung zu etablierten Beobachtungen}
	\label{sec:established_observations}
	
	Während das T0-Modell und das Erweiterte Standardmodell signifikante Abweichungen von der Standardkosmologie darstellen, stimmen sie mit bestimmten etablierten Beobachtungsphänomenen überein, die bereits Hinweise auf Energieabschwächung und Ablenkungsprozesse zeigen.
	
	\subsection{Sonnensystem-Ablenkung und Energieverlust}
	\label{subsec:solar_deflection}
	
	Interessanterweise demonstriert die Ablenkung elektromagnetischer Wellen in der Nähe massiver Körper wie der Sonne bereits Eigenschaften, die mit beiden Rahmenwerken konsistent sind. In der konventionellen allgemeinen Relativitätstheorie wird die Lichtbeugung nahe der Sonne traditionell durch Raumzeitkrümmung erklärt. Detaillierte Beobachtungen zeigen jedoch Energieverlusteffekte, die diese Ablenkung im Nahfeldbereich begleiten.
	
	Diese etablierten Beobachtungen umfassen:
	
	\begin{itemize}
		\item Gravitationelle Rotverschiebung von Licht, das nahe massiver Körper vorbeigeht, erstmals beobachtet im Sonnenspektrum von Adams im Jahr 1925 \cite{Adams1925} und später durch das Pound-Rebka-Experiment verifiziert \cite{Pound1960}
		\item Frequenzverschiebungen in Raumfahrzeugkommunikationen während Sonnenkonjunktionen, beobachtet in mehreren Missionen einschließlich Viking und Cassini \cite{Bertotti2003}
		\item Shapiro-Zeitverzögerungsmessungen, die Ausbreitungseffekte zeigen, die sowohl mit Pfadverlängerung als auch mit Energiemodifikationen konsistent sind \cite{Shapiro1971}
	\end{itemize}
	
	Die Shapiro-Zeitverzögerung wird insbesondere in der Standardrelativitätstheorie typischerweise rein als Pfadlängeneffekt interpretiert, aber der mathematische Formalismus ist konsistent mit einer Komponente, die intrinsischen Frequenzverschiebungen zugeschrieben werden könnte \cite{Moyer2000, Will2014}. Die im Cassini-Experiment beobachtete Frequenzverschiebung \cite{Bertotti2003} demonstrierte ein Präzisionsniveau (ungefähr \(2,3 \times 10^{-5}\)), das potenziell zwischen reinen geometrischen Effekten und solchen, die Energieübertragung beinhalten, unterscheiden könnte.
	
	Sowohl das T0-Modell als auch das Erweiterte Standardmodell sagen voraus, dass Licht, das gravitationelle Ablenkung erfährt, auch Energieverlust erfahren sollte, obwohl sie dies durch unterschiedliche Mechanismen erklären:
	
	\begin{itemize}
		\item T0-Modell: Der intrinsische Zeitfeldgradient nahe massiver Körper verursacht gleichzeitig Ablenkung und Energieverlust durch seinen Effekt auf die Photonenausbreitung
		\item Erweitertes SM: Das Skalarfeld \(\Theta\) beeinflusst sowohl den Pfad als auch die Energie von Photonen durch seine Modifikation effektiver Raumzeiteigenschaften
	\end{itemize}
	
	Diese Verbindung zu etablierten Sonnensystembeobachtungen bietet einen wichtigen empirischen Anker für beide Rahmenwerke und deutet darauf hin, dass der Energieabschwächungsmechanismus, den sie für kosmische Rotverschiebung vorschlagen, bereits in lokalen Gravitationssystemen beobachtbar sein könnte, wenn auch in einem viel kleineren Maßstab.
	
	\subsection{Überbrückung lokaler und kosmischer Phänomene}
	\label{subsec:bridging_phenomena}
	
	Die Vereinheitlichung lokaler Ablenkungsphänomene mit kosmischer Rotverschiebung stellt eine konzeptuelle Stärke beider Rahmenwerke dar. Während die Standardkosmologie gravitationelle Lichtbeugung und kosmische Rotverschiebung als fundamental unterschiedliche Phänomene behandelt (eines aufgrund von Raumzeitkrümmung, das andere aufgrund von Expansion), bieten sowohl das T0-Modell als auch das Erweiterte SM eine vereinheitlichte Erklärung:
	
	\begin{itemize}
		\item Lokale Ablenkung mit Energieverlust nahe massiver Körper ist eine kleinmaßstäbliche Manifestation desselben Mechanismus, der kosmische Rotverschiebung über große Entfernungen produziert
		\item Die kontinuierliche Natur des intrinsischen Zeitfeldes \(\Tfield\) oder Skalarfeldes \(\Theta\) bietet einen sanften Übergang zwischen lokalen und kosmischen Skalen
		\item Beide sagen eine sanfte Skalenbeziehung zwischen Ablenkungswinkel, zurückgelegter Distanz und Energieverlust voraus
	\end{itemize}
	
	Diese Verbindung zwischen Sonnensystembeobachtungen und kosmologischen Phänomenen bietet eine potenziell testbare Vorhersage: Detaillierte Analyse der Energieverlustkomponente in der Sonnensystemablenkung sollte Muster zeigen, die mit der Gravitationspotentialformel \(\Phi(r) = -\frac{GM}{r} + \kappa r\) konsistent sind, wenn auch mit dem linearen Term, der erst in größeren Skalen signifikant wird.
	
	Jüngste Messungen des Gravitationslinseneffekts, insbesondere die detaillierten Studien der Quasar-Lichtablenkung durch Vordergrundgalaxien \cite{Bolton2008, Suyu2017}, bieten Möglichkeiten, diese Vorhersagen auf mittleren Skalen zu testen. Bei diesen Entfernungen würde der \(\kappa r\)-Term beginnen, messbare Effekte zu zeigen, wenn die T0/ESM-Rahmenwerke korrekt sind. Diese Gravitationslinseneffektstudien haben Präzisionsniveaus erreicht, die potenziell die subtilen Energieverlustkomponenten erkennen könnten, die von beiden Rahmenwerken vorhergesagt werden.
	
	Das Konzept eines kontinuierlichen Mechanismus, der lokale und kosmische Skalen verbindet, stellt einen philosophisch ansprechenden Aspekt beider Modelle dar. In der Standardkosmologie sind unterschiedliche Erklärungen für Phänomene auf unterschiedlichen Skalen erforderlich (z.B. gekrümmte Raumzeit für Sonnensystemeffekte versus expandierender Raum für kosmische Rotverschiebung). Die T0- und Erweiterten SM-Rahmenwerke bieten eine einheitlichere Perspektive, bei der derselbe fundamentale Mechanismus—ob als intrinsisches Zeitfeld oder als Skalarfeld beschrieben, das Gravitationsgleichungen modifiziert—konsistent über alle Skalen operiert und eine kohärentere und elegantere Beschreibung der Realität bietet.
	
	\section{Schlussfolgerung}
	\label{sec:conclusion}
	
	Unsere Analyse hat gezeigt, dass, während das T0-Modell und das Erweiterte Standardmodell mathematisch äquivalente Rahmenwerke sind, sie sich signifikant in ihren konzeptuellen Grundlagen und Klarheit unterscheiden. Das T0-Modell, mit seinem feldtheoretischen Ansatz basierend auf einem intrinsischen Zeitfeld, das den dreidimensionalen Raum durchdringt, bietet überlegene konzeptuelle Eleganz und intuitive Zugänglichkeit im Vergleich zum Skalarfeld des Erweiterten Standardmodells, dem eine klare physikalische Interpretation fehlt und das potenziell unnötige dimensionale Komplexität einführt.
	
	Die feldtheoretische Perspektive des T0-Modells bietet einen natürlicheren und konzeptuell kohärenteren Rahmen für das Verständnis gravitationeller und Quantenphänomene. Das intrinsische Zeitfeld \(\Tfield\) bietet eine direkte physikalische Interpretation für eine breite Palette von Phänomenen, von Quantendekoherenz bis zu galaktischen Rotationskurven, ohne abstrakte höherdimensionale Konstrukte zu benötigen.
	
	Während eine fünfdimensionale Interpretation des Skalarfeldes \(\Theta\) im ESM mathematisch möglich ist, reduziert sie sich letztendlich auf eine feldtheoretische Beschreibung ähnlich dem T0-Modell, aber mit zusätzlichem konzeptuellem Ballast. Wie von philosophischen Prinzipien der Theorieauswahl festgestellt, sollte bei mathematisch äquivalenten Theorien diejenige mit größerer konzeptueller Klarheit und Eleganz bevorzugt werden—in diesem Fall das T0-Modell.
	
	Besonders bedeutsam ist die Verbindung, die wir zwischen etablierten Sonnensystembeobachtungen der Lichtablenkung mit Energieverlust und den breiteren kosmologischen Vorhersagen beider Rahmenwerke hervorgehoben haben. Diese Verbindungen deuten darauf hin, dass die Mechanismen, die für kosmische Rotverschiebung vorgeschlagen werden, bereits in lokalen Gravitationssystemen beobachtbar sein könnten, was einen potenziellen Weg für empirische Validierung bietet.
	
	Dieser Vergleich unterstreicht eine wichtige Lektion in der theoretischen Physik: Mathematische Äquivalenz impliziert nicht konzeptuelle Äquivalenz. Die Art, wie wir die physikalische Realität konzeptualisieren, beeinflusst tiefgreifend unser Verständnis der Natur, selbst wenn unterschiedliche Konzeptualisierungen identische Vorhersagen liefern. Der feldtheoretische Ansatz des T0-Modells stellt nicht nur eine alternative mathematische Formulierung dar, sondern eine fundamental unterschiedliche und potenziell erhellendere Art, die tiefsten Strukturen der physikalischen Realität zu verstehen.
	
	\begin{thebibliography}{99}
		\bibitem{pascher_part1_2025} J. Pascher, \href{https://github.com/jpascher/T0-Time-Mass-Duality/tree/main/2/pdf/Deutsch/QMRelZeitMasseTeil1.pdf}{Überbrückung von Quantenmechanik und Relativitätstheorie durch Zeit-Masse-Dualität: Teil I: Theoretische Grundlagen}, 7. April 2025.
		\bibitem{pascher_part2_2025} J. Pascher, \href{https://github.com/jpascher/T0-Time-Mass-Duality/tree/main/2/pdf/Deutsch/QMRelZeitMasseTeil2.pdf}{Überbrückung von Quantenmechanik und Relativitätstheorie durch Zeit-Masse-Dualität: Teil II: Kosmologische Implikationen und experimentelle Validierung}, 7. April 2025.
		\bibitem{pascher_emergente_2025} J. Pascher, \href{https://github.com/jpascher/T0-Time-Mass-Duality/tree/main/2/pdf/Deutsch/EmergentGravT0.pdf}{Emergente Gravitation im T0-Modell: Eine umfassende Herleitung}, 1. April 2025.
		\bibitem{pascher_standardmod_2025} J. Pascher, \href{https://github.com/jpascher/T0-Time-Mass-Duality/tree/main/2/pdf/Deutsch/StandardModKruemmungRotv.pdf}{Vervollständigung des Standardmodells: Eine Erweiterung kompatibel mit dem T0-Modell der Zeit-Masse-Dualität}, 17. April 2025.
		\bibitem{pascher_vereinheitlichung_2025} J. Pascher, \href{https://github.com/jpascher/T0-Time-Mass-Duality/tree/main/2/pdf/Deutsch/T0VereinheitlichungDEGal.pdf}{Vereinheitlichung des T0-Modells: Grundlagen, Dunkle Energie und Galaxiendynamik}, 4. April 2025.
		\bibitem{pascher_alphabeta_2025} J. Pascher, \href{https://github.com/jpascher/T0-Time-Mass-Duality/tree/main/2/pdf/Deutsch/Alpha1Beta1Konsistenz.pdf}{Einheitliches Einheitensystem im T0-Modell: Die Konsistenz von \(\alpha = 1\) und \(\beta = 1\)}, 5. April 2025.
		\bibitem{Will2014} C. M. Will, \textit{The Confrontation between General Relativity and Experiment}, Living Rev. Rel. \textbf{17}, 4 (2014).
		\bibitem{Verlinde2011} E. Verlinde, \textit{On the Origin of Gravity and the Laws of Newton}, J. High Energy Phys. \textbf{2011}, 29 (2011).
		\bibitem{Bekenstein2004} J. D. Bekenstein, \textit{Relativistic gravitation theory for the modified Newtonian dynamics paradigm}, Phys. Rev. D \textbf{70}, 083509 (2004).
		\bibitem{Kaluza1921} T. Kaluza, \textit{Zum Unitätsproblem der Physik}, Sitzungsber. Preuss. Akad. Wiss. Berlin. (Math. Phys.) \textbf{1921}, 966–972 (1921).
		\bibitem{Klein1926} O. Klein, \textit{Quantentheorie und fünfdimensionale Relativitätstheorie}, Z. Phys. \textbf{37}, 895–906 (1926).
		\bibitem{Brans1961} C. Brans und R. H. Dicke, \textit{Mach's Principle and a Relativistic Theory of Gravitation}, Phys. Rev. \textbf{124}, 925 (1961).
		\bibitem{Weinberg1989} S. Weinberg, \textit{The Cosmological Constant Problem}, Rev. Mod. Phys. \textbf{61}, 1 (1989).
		\bibitem{McGaugh2016} S. S. McGaugh, F. Lelli und J. M. Schombert, \textit{Radial Acceleration Relation in Rotationally Supported Galaxies}, Phys. Rev. Lett. \textbf{117}, 201101 (2016).
		\bibitem{Riess1998} A. G. Riess et al., \textit{Observational Evidence from Supernovae for an Accelerating Universe and a Cosmological Constant}, Astron. J. \textbf{116}, 1009 (1998).
		\bibitem{Kuhn1962} T. S. Kuhn, \textit{The Structure of Scientific Revolutions}, University of Chicago Press (1962).
		\bibitem{Adams1925} W. S. Adams, \textit{The Relativity Displacement of the Spectral Lines in the Companion of Sirius}, Proc. Natl. Acad. Sci. \textbf{11}, 382-387 (1925).
		\bibitem{Pound1960} R. V. Pound und G. A. Rebka Jr., \textit{Apparent Weight of Photons}, Phys. Rev. Lett. \textbf{4}, 337-341 (1960).
		\bibitem{Bertotti2003} B. Bertotti, L. Iess und P. Tortora, \textit{A test of general relativity using radio links with the Cassini spacecraft}, Nature \textbf{425}, 374-376 (2003).
		\bibitem{Shapiro1971} I. I. Shapiro, M. E. Ash, R. P. Ingalls, W. B. Smith, D. B. Campbell, R. B. Dyce, R. F. Jurgens und G. H. Pettengill, \textit{Fourth Test of General Relativity: New Radar Result}, Phys. Rev. Lett. \textbf{26}, 1132-1135 (1971).
		\bibitem{Moyer2000} T. D. Moyer, \textit{Formulation for Observed and Computed Values of Deep Space Network Data Types for Navigation}, JPL Publication \textbf{00-7} (2000).
		\bibitem{Bolton2008} A. S. Bolton, S. Burles, L. V. E. Koopmans, T. Treu und L. A. Moustakas, \textit{The Sloan Lens ACS Survey. V. The Full ACS Strong-Lens Sample}, Astrophys. J. \textbf{682}, 964-984 (2008).
		\bibitem{Suyu2017} S. H. Suyu, V. Bonvin, F. Courbin, et al., \textit{H0LiCOW - I. H0 Lenses in COSMOGRAIL's Wellspring: program overview}, Mon. Not. Roy. Astron. Soc. \textbf{468}, 2590-2604 (2017).
	\end{thebibliography}
	
\end{document}