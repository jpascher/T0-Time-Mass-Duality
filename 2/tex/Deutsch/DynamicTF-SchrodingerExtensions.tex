\documentclass[12pt,a4paper]{article}
\usepackage[utf8]{inputenc}
\usepackage[T1]{fontenc}
\usepackage[ngerman]{babel}
\usepackage{lmodern}
\usepackage{amsmath}
\usepackage{amssymb}
\usepackage{physics}
\usepackage{hyperref}
\usepackage{tcolorbox}
\usepackage{booktabs}
\usepackage{enumitem}
\usepackage[table,xcdraw]{xcolor}
\usepackage[left=2cm,right=2cm,top=2cm,bottom=2cm]{geometry}
\usepackage{pgfplots}
\pgfplotsset{compat=1.18}
\usepackage{graphicx}
\usepackage{float}
\usepackage{fancyhdr}
\usepackage{siunitx}
\usepackage{array}
\usepackage{cleveref}

% Kopf- und Fußzeilen
\pagestyle{fancy}
\fancyhf{}
\fancyhead[L]{Johann Pascher}
\fancyhead[R]{Dynamische Erweiterung des Zeitfeldes}
\fancyfoot[C]{\thepage}
\renewcommand{\headrulewidth}{0.4pt}
\renewcommand{\footrulewidth}{0.4pt}

% Benutzerdefinierte Befehle
\newcommand{\Tfield}{T(x)}
\newcommand{\Tfieldt}{T(x,t)}
\newcommand{\alphaEM}{\alpha_{\text{EM}}}
\newcommand{\alphaW}{\alpha_{\text{W}}}
\newcommand{\betaT}{\beta_{\text{T}}}
\newcommand{\Mpl}{M_{\text{Pl}}}
\newcommand{\Tzerot}{T_0(\Tfield)}
\newcommand{\Tzero}{T_0}
\newcommand{\vecx}{\vec{x}}
\newcommand{\gammaf}{\gamma_{\text{Lorentz}}}
\newcommand{\DhiggsT}{\Tfield (\partial_\mu + ig A_\mu) \Phi + \Phi \partial_\mu \Tfield}
\newcommand{\DhiggsTt}{\Tfieldt (\partial_\mu + ig A_\mu) \Phi + \Phi \partial_\mu \Tfieldt}
\newcommand{\LCDM}{\Lambda\text{CDM}}
\newcommand{\DTmu}{D_{T,\mu}}
\newcommand{\calL}{\mathcal{L}}
\newcommand{\deq}{\displaystyle}
\newcommand{\e}{\mathrm{e}}
\newcommand{\dTdt}{\frac{d\Tfieldt}{dt}}
\newcommand{\pdTdt}{\frac{\partial\Tfieldt}{\partial t}}
\newcommand{\pdTdx}{\nabla\Tfieldt}

\hypersetup{
	colorlinks=true,
	linkcolor=blue,
	citecolor=blue,
	urlcolor=blue,
	pdftitle={Dynamische Erweiterung des intrinsischen Zeitfeldes im T0-Modell},
	pdfauthor={Johann Pascher},
	pdfsubject={Theoretische Physik},
	pdfkeywords={T0-Modell, Intrinsisches Zeitfeld, Dynamische Feldtheorie, Quantenmechanik, Erweiterte Schrödinger-Gleichung}
}

\begin{document}
	
	\title{Dynamische Erweiterung des intrinsischen Zeitfeldes im T0-Modell: \\Vollständige feldtheoretische Behandlung und Implikationen für die Quantenevolution}
	\author{Johann Pascher\\
		Abteilung für Kommunikationstechnik, \\Höhere Technische Bundeslehranstalt (HTL), Leonding, Österreich\\
		\texttt{johann.pascher@gmail.com}}
	\date{\today}
	
	\maketitle
	
	\begin{abstract}
		Diese Arbeit präsentiert eine bedeutende konzeptionelle Erweiterung des T0-Modells durch die vollständige Entwicklung der dynamischen Natur des intrinsischen Zeitfeldes. Anstatt das Zeitfeld als statische räumliche Konfiguration $\Tfield$ zu behandeln, schreiten wir zu einem vollständigen raum-zeit-abhängigen Feld $\Tfieldt$ fort, wobei sowohl Masse als auch Frequenz als Variablen betrachtet werden, die von Position und Zeit abhängen. Diese Erweiterung vertieft die elegante Vereinheitlichung der physikalischen Randbedingungen – Wellenausbreitung und konzentrierte Masse – durch die $\max$-Funktion und erfordert eine entsprechende Erweiterung der modifizierten Schrödinger-Gleichung. Wir formulieren eine vollständige Lagrange-Dichte, die die gesamte Dynamik des Feldes einbezieht, und leiten die Implikationen für die Quantenevolution ab, einschließlich des Auftretens der totalen Zeitableitung in der Quantenevolutionsgleichung. Dieser Fortschritt stellt eine natürliche theoretische Weiterentwicklung dar, die die Fähigkeit des T0-Modells verbessert, Quantensysteme in variierenden Gravitationsumgebungen und Übergänge zwischen teilchen- und wellendominierten Zuständen zu beschreiben.
	\end{abstract}
	\newpage
	\tableofcontents
	\newpage
	\section{Einleitung}
	\label{sec:introduction}
	
	Das T0-Modell der Zeit-Masse-Dualität stellt einen neuartigen Ansatz zur Vereinheitlichung der Quantenmechanik und der Relativitätstheorie durch die Einführung des intrinsischen Zeitfeldes $\Tfield$ und die Umkehrung der traditionellen Beziehung zwischen Zeit und Masse dar \cite{pascher_part1_2025,pascher_part2_2025}. In früheren Formulierungen wurde dieses Feld hauptsächlich als räumliche Konfiguration mit Positionsabhängigkeit behandelt, definiert als $\Tfield = \frac{\hbar}{\max(mc^2, \omega)}$. Diese Definition erfasst elegant beide Extreme der physikalischen Realität – massendominierte Systeme, bei denen $T = \frac{\hbar}{mc^2}$, und wellendominierte Systeme, bei denen $T = \frac{\hbar}{\omega}$ – unter Verwendung eines einzigen mathematischen Konstrukts \cite{pascher_quantum_2025}.
	
	Während frühere Arbeiten die potenzielle Zeitabhängigkeit des Feldes anerkannt haben, insbesondere im Kontext der erweiterten Schrödinger-Gleichung \cite{pascher_quantum_2025}, wurde eine umfassende Behandlung des Feldes als vollständig dynamische Entität $\Tfieldt$ noch nicht vollständig entwickelt. Diese Arbeit adressiert diese konzeptionelle Erweiterung und untersucht die Implikationen der Behandlung sowohl der Masse $m(\vecx,t)$ als auch der Frequenz $\omega(\vecx,t)$ als Variablen, die von Position und Zeit abhängen. Dieser Ansatz entspricht besser den feldtheoretischen Prinzipien, bei denen Felder von Natur aus dynamische Entitäten sind, die sich sowohl im Raum als auch in der Zeit entwickeln.
	
	Diese Erweiterung ist nicht nur eine mathematische Verfeinerung, sondern stellt einen fundamentalen Fortschritt in der konzeptionellen Grundlage des T0-Modells dar. Durch die Behandlung des intrinsischen Zeitfeldes als vollständig dynamisch verbessern wir seine Fähigkeit, physikalische Phänomene über verschiedene Skalen hinweg zu beschreiben, von der Quantendekoheränz in variierenden Gravitationsumgebungen bis zur Evolution verschränkter Zustände in dynamischen Raumzeiten. Darüber hinaus bietet diese Erweiterung einen natürlicheren Rahmen zum Verständnis von Übergängen zwischen teilchenartigen und wellenartigen Verhaltensweisen in Quantensystemen.
	
	Die Arbeit ist wie folgt gegliedert: Abschnitt \ref{sec:dynamic_time_field} präsentiert die erweiterte Definition und Eigenschaften des dynamischen intrinsischen Zeitfeldes. Abschnitt \ref{sec:lagrangian_formulation} entwickelt die vollständige Lagrange-Dichte für dieses Feld. Abschnitt \ref{sec:quantum_evolution} untersucht die Implikationen für die Quantenevolution durch eine erweiterte Schrödinger-Gleichung. Abschnitt \ref{sec:boundary_conditions} untersucht, wie diese dynamische Feldbehandlung die Vereinheitlichung von Wellen- und Teilchen-Randbedingungen verbessert. Schließlich fasst Abschnitt \ref{sec:conclusion} die konzeptionellen Fortschritte zusammen und diskutiert zukünftige Forschungsrichtungen.
	
	\section{Das dynamische intrinsische Zeitfeld}
	\label{sec:dynamic_time_field}
	
	\subsection{Erweiterte Definition}
	\label{subsec:extended_definition}
	
	Die dynamische Erweiterung des intrinsischen Zeitfeldes wird definiert als:
	
	\begin{equation}
		\Tfieldt = \frac{\hbar}{\max(m(\vecx,t)c^2, \omega(\vecx,t))}
		\label{eq:dynamic_time_field}
	\end{equation}
	
	wobei:
	\begin{itemize}
		\item $m(\vecx,t)$ die positions- und zeitabhängige Masse ist
		\item $\omega(\vecx,t)$ die positions- und zeitabhängige Frequenz/Energie ist
	\end{itemize}
	
	Diese Erweiterung erkennt explizit an, dass sowohl Masse als auch Frequenz keine statischen Konstanten sind, sondern dynamische Größen, die über Raum und Zeit variieren. Für massive Teilchen ergibt sich:
	
	\begin{equation}
		\Tfieldt = \frac{\hbar}{m(\vecx,t)c^2}
		\label{eq:massive_dynamic}
	\end{equation}
	
	und für Photonen oder wellendominierte Systeme:
	
	\begin{equation}
		\Tfieldt = \frac{\hbar}{\omega(\vecx,t)}
		\label{eq:photon_dynamic}
	\end{equation}
	
	Die Variation der Masse über Raum und Zeit spiegelt die grundlegende Prämisse des T0-Modells wider – dass Masse eine variable Größe ist, die von der Gravitationsumgebung beeinflusst wird, und nicht eine konstante Eigenschaft, wie in der traditionellen Relativitätstheorie angenommen. Ebenso ändert sich die Frequenz von Photonen, während sie mit dem Zeitfeld während der Ausbreitung interagieren, was einen natürlichen Mechanismus für Phänomene wie die Rotverschiebung bietet, ohne kosmische Expansion zu erfordern \cite{pascher_part2_2025}.
	
	\subsection{Physikalische Bedeutung}
	\label{subsec:physical_significance}
	
	Die dynamische Natur des Zeitfeldes hat tiefgreifende physikalische Bedeutung:
	
	\begin{enumerate}
		\item \textbf{Gravitationsreaktion}: Das Feld $\Tfieldt$ reagiert aktiv auf Veränderungen in Gravitationsumgebungen, wobei die Masse in stärkeren Gravitationsfeldern zunimmt (und somit $\Tfieldt$ abnimmt).
		
		\item \textbf{Energieerhaltung}: Während Photonen sich durch den Raum ausbreiten, verursacht ihre Interaktion mit dem Zeitfeld eine Energieabschwächung, die sich über große Entfernungen als kosmische Rotverschiebung manifestiert \cite{pascher_galaxies_2025}.
		
		\item \textbf{Quantendekoheränz}: Die Rate der Quantendekoheränz ist mit dem lokalen Wert und der Änderungsrate von $\Tfieldt$ verbunden, was erklärt, warum makroskopische Objekte (mit kleinerem $\Tfieldt$) schneller dekohärieren als mikroskopische Quantensysteme \cite{pascher_quantum_2025}.
		
		\item \textbf{Feldausbreitung}: Änderungen im Zeitfeld breiten sich mit endlicher Geschwindigkeit aus, was die Kausalität in den Vorhersagen des Frameworks sicherstellt.
	\end{enumerate}
	
	Diese dynamische Behandlung ermöglicht auch auf natürliche Weise Übergangszustände, bei denen Systeme zwischen wellendominiertem und teilchendominiertem Verhalten wechseln können. Zum Beispiel während der Teilchenerzeugung und -vernichtung geht der relevante Term in der $\max$-Funktion fließend über und bietet eine kontinuierliche Beschreibung solcher Prozesse.
	
	\section{Vollständige Lagrange-Formulierung}
	\label{sec:lagrangian_formulation}
	
	\subsection{Dynamische Feld-Lagrange-Dichte}
	\label{subsec:dynamic_lagrangian}
	
	Die vollständige Lagrange-Dichte für das dynamische intrinsische Zeitfeld ist:
	
	\begin{equation}
		\mathcal{L}_{\text{intrinsisch}} = \frac{1}{2}\partial_{\mu}\Tfieldt\partial^{\mu}\Tfieldt - \frac{1}{2}\Tfieldt^2 - \frac{\rho(\vecx,t)}{\Tfieldt}
		\label{eq:intrinsic_lagrangian}
	\end{equation}
	
	wobei $\rho(\vecx,t)$ die positions- und zeitabhängige Masse-Energie-Dichte ist. Diese Lagrange-Dichte umfasst:
	
	\begin{itemize}
		\item Einen kinetischen Term $\frac{1}{2}\partial_{\mu}\Tfieldt\partial^{\mu}\Tfieldt$, der die Raum-Zeit-Dynamik des Feldes repräsentiert
		\item Einen Potentialterm $\frac{1}{2}\Tfieldt^2$, der die Selbstwechselwirkung des Feldes widerspiegelt
		\item Einen Kopplungsterm $\frac{\rho(\vecx,t)}{\Tfieldt}$, der die Wechselwirkung zwischen dem Feld und Materie-/Energieverteilungen darstellt
	\end{itemize}
	
	Diese Formulierung erweitert frühere Arbeiten \cite{pascher_lagrange_2025}, indem sie die Zeitabhängigkeit des Feldes explizit einbezieht und sicherstellt, dass die Masse-Energie-Dichte als dynamische Größe behandelt wird und nicht als statische Verteilung.
	
	\subsection{Feldgleichungen}
	\label{subsec:field_equations}
	
	Die aus der Lagrange-Dichte abgeleitete Euler-Lagrange-Gleichung ergibt die Feldgleichung:
	
	\begin{equation}
		\partial_{\mu}\partial^{\mu}\Tfieldt + \Tfieldt + \frac{\rho(\vecx,t)}{\Tfieldt^2} = 0
		\label{eq:field_equation}
	\end{equation}
	
	Im statischen Grenzfall, bei dem Zeitableitungen verschwinden, reduziert sich dies auf:
	
	\begin{equation}
		\nabla^2 \Tfieldt \approx -\frac{\rho(\vecx,t)}{\Tfieldt^2}
		\label{eq:static_approximation}
	\end{equation}
	
	was mit früheren Formulierungen \cite{pascher_emergente_2025} übereinstimmt, aber die potenzielle Zeitabhängigkeit der Masse-Energie-Dichte beibehält.
	
	Die dynamische Feldgleichung beschreibt, wie sich das intrinsische Zeitfeld als Reaktion auf Änderungen in der Masse-Energie-Verteilung entwickelt und bietet einen natürlichen Rahmen zum Verständnis von Phänomenen wie Gravitationswellen und dynamischer kosmologischer Evolution innerhalb des T0-Modells.
	
	\section{Implikationen für die Quantenevolution}
	\label{sec:quantum_evolution}
	
	\subsection{Erweiterte Schrödinger-Gleichung}
	\label{subsec:extended_schrodinger}
	
	Die dynamische Erweiterung des Zeitfeldes erfordert eine entsprechende Erweiterung der modifizierten Schrödinger-Gleichung. Die zuvor etablierte Form \cite{pascher_quantum_2025}:
	
	\begin{equation}
		i\hbar \Tfield \frac{\partial\Psi}{\partial t} + i\hbar \Psi \frac{\partial \Tfield}{\partial t} = \hat{H} \Psi
		\label{eq:original_schrodinger}
	\end{equation}
	
	muss nun erweitert werden, um die vollständige Raum-Zeit-Dynamik des Feldes zu berücksichtigen:
	
	\begin{equation}
		i\hbar \Tfieldt \frac{\partial\Psi}{\partial t} + i\hbar \Psi \left[\frac{\partial \Tfieldt}{\partial t} + \vec{v}\cdot\nabla\Tfieldt\right] = \hat{H} \Psi
		\label{eq:dynamic_schrodinger}
	\end{equation}
	
	wobei $\vec{v}$ die Geschwindigkeit des Quantensystems ist. Der Term in eckigen Klammern repräsentiert die totale Zeitableitung des Feldes, wie sie vom bewegten Quantensystem erfahren wird:
	
	\begin{equation}
		\frac{d\Tfieldt}{dt} = \frac{\partial \Tfieldt}{\partial t} + \vec{v}\cdot\nabla\Tfieldt
		\label{eq:total_derivative}
	\end{equation}
	
	Diese totale Ableitung hat tiefgreifende physikalische Bedeutung. Der erste Term $\pdTdt$ repräsentiert, wie sich das Zeitfeld explizit mit der Zeit an einem festen Punkt im Raum ändert, während der zweite Term $\vec{v}\cdot\pdTdx$ berücksichtigt, wie sich das Feld entlang der Trajektorie des bewegten Quantensystems ändert.
	
	\subsection{Physikalische Interpretation}
	\label{subsec:quantum_interpretation}
	
	Die erweiterte Schrödinger-Gleichung bietet mehrere wichtige Erkenntnisse:
	
	\begin{enumerate}
		\item \textbf{Pfadabhängigkeit}: Die Quantenevolution hängt nicht nur vom lokalen Wert des Zeitfeldes ab, sondern auch davon, wie es sich entlang des Pfades des Systems ändert, was ein Element der Geschichtsabhängigkeit in die Quantendynamik einführt.
		
		\item \textbf{Geschwindigkeitskopplung}: Das explizite Auftreten der Geschwindigkeit in der Quantenevolutionsgleichung schafft eine direkte Kopplung zwischen der Bewegung eines Teilchens und seiner internen Quantenevolution.
		
		\item \textbf{Gravitationseinfluss}: Da das Zeitfeld mit der Gravitation verbunden ist, bietet diese Formulierung einen natürlichen Mechanismus dafür, wie Gravitationsumgebungen das Quantenverhalten beeinflussen.
		
		\item \textbf{Zeitdilatationseffekte}: Die Gleichung integriert auf natürliche Weise Effekte, die der Zeitdilatation in der Relativitätstheorie analog sind, jedoch durch den Mechanismus der Massenvariation und nicht durch relativistische Zeiteffekte.
	\end{enumerate}
	
	Dieser Rahmen bietet eine natürliche Erklärung für gravitationsinduzierte Phasenverschiebungen in Quanteninterferenzexperimenten, wie z.B. Neutroneninterferometrie in Gravitationsfeldern \cite{Colella1975}, ohne separate Behandlungen für Quanten- und Gravitationsphänomene zu erfordern.
	
	\section{Vereinheitlichung der Randbedingungen}
	\label{sec:boundary_conditions}
	
	\subsection{Welle-Teilchen-Übergänge}
	\label{subsec:wave_particle}
	
	Eine besonders elegante Eigenschaft des dynamischen Zeitfeldes ist, wie es die beiden physikalischen Randbedingungen – Wellenausbreitung und konzentrierte Masse – durch die $\max$-Funktion in seiner Definition handhabt. Dieser Mechanismus verdient besondere Aufmerksamkeit, da er einen natürlichen Rahmen zum Verständnis der Welle-Teilchen-Dualität bietet.
	
	Für ein Quantensystem, das zwischen wellenähnlichem und teilchenähnlichem Verhalten übergeht, passt sich das Zeitfeld fließend an, je nachdem, welcher Term in dem Ausdruck dominiert:
	
	\begin{equation}
		\Tfieldt = \frac{\hbar}{\max(m(\vecx,t)c^2, \omega(\vecx,t))}
		\label{eq:boundary_transition}
	\end{equation}
	
	Während Messprozessen oder Wechselwirkungen, die ein Quantensystem lokalisieren, kann der Term $m(\vecx,t)c^2$ dominieren und das System in Richtung teilchenähnliches Verhalten verschieben. Umgekehrt kann während der freien Evolution der Term $\omega(\vecx,t)$ dominieren, was zu wellenähnlichem Verhalten führt.
	
	Dieser fließende Übergang bietet eine natürlichere Beschreibung der Quantenmessung und des Wellenfunktionskollaps als die konventionelle Quantenmechanik, bei der der Übergang zwischen Wellen- und Teilchenbeschreibungen oft künstlich auferlegt erscheint, anstatt aus der zugrundeliegenden Physik zu entstehen.
	
	\subsection{Vergleich mit dem Erweiterten Standardmodell}
	\label{subsec:comparison_esm}
	
	Die direkte Handhabung dieser Randbedingungen durch die $\max$-Funktion kontrastiert vorteilhaft mit alternativen Ansätzen wie dem Erweiterten Standardmodell (ESM) \cite{pascher_standardmod_2025}. Im ESM steht das Skalarfeld $\Theta$ in einer logarithmischen Beziehung zum Zeitfeld:
	
	\begin{equation}
		\Theta(\vecx,t) \propto \ln\left(\frac{\Tfieldt}{\Tzero}\right)
		\label{eq:theta_relation}
	\end{equation}
	
	Diese Beziehung verschleiert die elegante Handhabung der Randbedingungen des T0-Modells. Die Extremzustände, die direkt durch die $\max$-Funktion in der Zeitfelddefinition erfasst werden, müssen durch komplexere mathematische Maschinerie im ESM gehandhabt werden, was die physikalische Interpretation weniger intuitiv und direkt macht.
	
	Der Übergang zwischen reinen Energie- (Wellen-) Zuständen und maximalen Masse- (Teilchen-) Zuständen entbehrt der direkten, intuitiven Formulierung, die im T0-Modell vorhanden ist. Stattdessen werden diese Übergänge indirekt in der Kopplung zwischen dem Skalarfeld und dem Energie-Impuls-Tensor kodiert, was eine Schicht mathematischer Abstraktion hinzufügt, die die Theorie von ihrer physikalischen Interpretation entfernt.
	
	\section{Fazit und Ausblick}
	\label{sec:conclusion}
	
	\subsection{Konzeptionelle Fortschritte}
	\label{subsec:advancements}
	
	Die dynamische Erweiterung des intrinsischen Zeitfeldes stellt einen bedeutenden Fortschritt in der konzeptionellen Grundlage des T0-Modells dar. Durch die Behandlung des Zeitfeldes als vollständig dynamische Entität $\Tfieldt$ mit räumlicher und zeitlicher Abhängigkeit haben wir:
	
	\begin{itemize}
		\item Die feldtheoretische Behandlung des T0-Modells vertieft und es enger an etablierte Quantenfeldtheorieprinzipien angeglichen
		\item Die Fähigkeit des Modells verbessert, Quantensysteme in variierenden Gravitationsumgebungen zu beschreiben
		\item Einen natürlicheren Rahmen zum Verständnis der Welle-Teilchen-Dualität und Übergängen zwischen diesen Zuständen bereitgestellt
		\item Eine umfassendere Quantenevolutionsgleichung entwickelt, die die totale Zeitableitung des Feldes berücksichtigt
	\end{itemize}
	
	Diese Fortschritte halten die Kernprinzipien des T0-Modells – absolute Zeit, variable Masse und emergente Gravitation – aufrecht und erweitern gleichzeitig seine Erklärungskraft und theoretische Konsistenz.
	
	\subsection{Zukünftige Forschungsrichtungen}
	\label{subsec:future_research}
	
	Diese dynamische Erweiterung eröffnet mehrere vielversprechende Wege für zukünftige Forschung:
	
	\begin{enumerate}
		\item \textbf{Numerische Simulationen}: Entwicklung numerischer Simulationen der Quantenevolution in variierenden Gravitationsumgebungen unter Verwendung der erweiterten Schrödinger-Gleichung.
		
		\item \textbf{Quanten-Gravitations-Experimente}: Gestaltung experimenteller Tests, die zwischen den Vorhersagen des dynamischen T0-Modells und der konventionellen Quantenmechanik in Gravitationskontexten unterscheiden könnten.
		
		\item \textbf{Kosmologische Anwendungen}: Erweiterung der kosmologischen Vorhersagen des Modells, um zeitlich variierende Gravitationsumgebungen im frühen Universum zu berücksichtigen.
		
		\item \textbf{Welle-Teilchen-Übergänge}: Weitere Erforschung, wie die Handhabung der Welle-Teilchen-Dualität des Modells neue Einblicke in Quantenmessung und Dekoheränz liefern könnte.
		
		\item \textbf{Feldquantisierung}: Entwicklung einer vollständig quantisierten Version des dynamischen Zeitfeldes, die potenziell neue Ansätze zur Quantengravitation eröffnet.
	\end{enumerate}
	
	Die dynamische Erweiterung des intrinsischen Zeitfeldes stellt nicht nur eine mathematische Verfeinerung dar, sondern einen bedeutenden konzeptionellen Fortschritt, der die Fähigkeit des T0-Modells verbessert, eine vereinheitlichte Beschreibung von Quanten- und Gravitationsphänomenen zu liefern. Durch die Behandlung des Zeitfeldes als vollständig dynamische Entität kommen wir einem wirklich vereinheitlichten Rahmen zum Verständnis der fundamentalen Natur der Realität näher.
	
	\bibliographystyle{apsrev4-2}
	\begin{thebibliography}{99}
		\bibitem{pascher_part1_2025} J. Pascher, \href{https://github.com/jpascher/T0-Time-Mass-Duality/tree/main/2/pdf/English/QMRelTimeMassPart1En.pdf}{Brückenschlag zwischen Quantenmechanik und Relativitätstheorie durch Zeit-Masse-Dualität: Teil I: Theoretische Grundlagen}, 7. April 2025.
		\bibitem{pascher_part2_2025} J. Pascher, \href{https://github.com/jpascher/T0-Time-Mass-Duality/tree/main/2/pdf/English/QMRelTimeMassPart2En.pdf}{Brückenschlag zwischen Quantenmechanik und Relativitätstheorie durch Zeit-Masse-Dualität: Teil II: Kosmologische Implikationen und experimentelle Validierung}, 7. April 2025.
		\bibitem{pascher_quantum_2025} J. Pascher, \href{https://github.com/jpascher/T0-Time-Mass-Duality/tree/main/2/pdf/English/NotwendigkeitQMErweiterungEn.pdf}{Die Notwendigkeit der Erweiterung der Standardquantenmechanik und Quantenfeldtheorie}, 27. März 2025.
		\bibitem{pascher_lagrange_2025} J. Pascher, \href{https://github.com/jpascher/T0-Time-Mass-Duality/tree/main/2/pdf/English/MathZeitMasseLagrangeEn.pdf}{Von der Zeitdilatation zur Massenvariation: Mathematische Kernformulierungen der Zeit-Masse-Dualitätstheorie}, 29. März 2025.
		\bibitem{pascher_emergente_2025} J. Pascher, \href{https://github.com/jpascher/T0-Time-Mass-Duality/tree/main/2/pdf/English/EmergentGravT0En.pdf}{Emergente Gravitation im T0-Modell: Eine umfassende Herleitung}, 1. April 2025.
		\bibitem{pascher_galaxies_2025} J. Pascher, \href{https://github.com/jpascher/T0-Time-Mass-Duality/tree/main/2/pdf/English/MassVarGalaxienEn.pdf}{Massenvariation in Galaxien: Eine Analyse im T0-Modell mit emergenter Gravitation}, 30. März 2025.
		\bibitem{pascher_standardmod_2025} J. Pascher, \href{https://github.com/jpascher/T0-Time-Mass-Duality/tree/main/2/pdf/English/StandardModKruemmungRotvEn.pdf}{Vervollständigung des Standardmodells: Eine Erweiterung kompatibel mit dem T0-Modell der Zeit-Masse-Dualität}, 17. April 2025.
		\bibitem{pascher_esm_comparison_2025} J. Pascher, \href{https://github.com/jpascher/T0-Time-Mass-Duality/tree/main/2/pdf/English/T0vsESM_ConceptualAnalysisEn.pdf}{Konzeptioneller Vergleich von T0-Modell und Erweitertem Standardmodell: Feldtheoretische vs. Dimensionale Ansätze}, 25. April 2025.
		\bibitem{Colella1975} R. Colella, A. W. Overhauser und S. A. Werner, \textit{Beobachtung gravitationsinduzierter Quanteninterferenz}, Phys. Rev. Lett. \textbf{34}, 1472 (1975).
		\bibitem{Will2014} C. M. Will, \textit{Die Konfrontation zwischen Allgemeiner Relativitätstheorie und Experiment}, Living Rev. Rel. \textbf{17}, 4 (2014).
		\bibitem{Dirac1928} P. A. M. Dirac, \textit{Die Quantentheorie des Elektrons}, Proc. Roy. Soc. London A \textbf{117}, 610--624 (1928).
		\bibitem{Einstein1915} A. Einstein, \textit{Die Feldgleichungen der Gravitation}, Sitzungsber. Preuss. Akad. Wiss., 844--847 (1915).
		\bibitem{schrodinger1926} E. Schrödinger, \textit{Eine Undulationstheorie der Mechanik von Atomen und Molekülen}, Phys. Rev. \textbf{28}, 1049 (1926).
	\end{thebibliography}
	
\end{document}