\documentclass[12pt,a4paper]{article}
\usepackage[utf8]{inputenc}
\usepackage[T1]{fontenc}
\usepackage[ngerman]{babel}
\usepackage{lmodern}
\usepackage{amsmath}
\usepackage{amssymb}
\usepackage{physics}
\usepackage{hyperref}
\usepackage{bookmark}
\usepackage{tcolorbox}
\usepackage{booktabs}
\usepackage{enumitem}
\usepackage[table,xcdraw]{xcolor}
\usepackage[left=2cm,right=2cm,top=2cm,bottom=2cm]{geometry}
\usepackage{pgfplots}
\pgfplotsset{compat=1.18}
\usepackage{graphicx}
\usepackage{float}
\usepackage{fancyhdr}
\usepackage{siunitx}
\usepackage{url}
\usepackage{bm}
\usepackage{cleveref}

% Acknowledgments environment
\newenvironment{acknowledgments}
{\section*{Danksagung}}
{\vspace{1em}}

% Custom commands
\newcommand{\Tfield}{T(x)}
\newcommand{\alphaEM}{\alpha_{\text{EM}}}
\newcommand{\alphaW}{\alpha_{\text{W}}}
\newcommand{\betaT}{\beta_{\text{T}}}
\newcommand{\Mpl}{M_{\text{Pl}}}
\newcommand{\Tzerot}{T_0(\Tfield)}
\newcommand{\Tzero}{T_0}
\newcommand{\vecx}{\vec{x}}
\newcommand{\vr}{\vec{r}}
\newcommand{\gammaf}{\gamma_{\text{Lorentz}}}
\newcommand{\DhiggsT}{\Tfield (\partial_\mu + ig A_\mu) \Phi + \Phi \partial_\mu \Tfield}
\newcommand{\LCDM}{\Lambda\text{CDM}}
\newcommand{\DTmu}{D_{T,\mu}}
\newcommand{\calL}{\mathcal{L}}
\newcommand{\deq}{\displaystyle}
\newcommand{\e}{\mathrm{e}}

% Header and Footer Configuration
\pagestyle{fancy}
\fancyhf{}
\fancyhead[L]{Johann Pascher}
\fancyhead[R]{Erweitertes Standardmodell-Framework}
\fancyfoot[C]{\thepage}
\renewcommand{\headrulewidth}{0.4pt}
\renewcommand{\footrulewidth}{0.4pt}

\hypersetup{
	colorlinks=true,
	linkcolor=blue,
	citecolor=blue,
	urlcolor=blue,
	pdftitle={Vervollständigung des Standardmodells: Eine Erweiterung kompatibel mit dem T0-Modell der Zeit-Masse-Dualität},
	pdfauthor={Johann Pascher},
	pdfsubject={Theoretische Physik},
	pdfkeywords={Standardmodell, Erweiterung, krümmungsbasierte Rotverschiebung, T0-Modell-Kompatibilität}
}

\title{Vervollständigung des Standardmodells: Eine Erweiterung kompatibel mit\\dem T0-Modell der Zeit-Masse-Dualität}
\author{Johann Pascher\\
	Fachbereich Kommunikationstechnik\\
	Höhere Technische Bundeslehranstalt (HTL), Leonding, Österreich\\
	\texttt{johann.pascher@gmail.com}}
\date{17. April 2025}

\begin{document}
	
	\maketitle
	
	\begin{abstract}
		Diese Arbeit präsentiert eine systematische Erweiterung des Standardmodells der Physik, um dessen Unvollständigkeit anzugehen, insbesondere in den Bereichen Quantengravitation, dunkle Energie und kosmische Strukturbildung. Durch die Einführung spezifischer Modifikationen – bei gleichzeitiger Bewahrung des Kernprinzips der Zeitdilatation – wird dieses erweiterte Framework vollständig kompatibel mit den Vorhersagen des alternativen T0-Modells, das auf Zeit-Masse-Dualität basiert\cite{pascher_zeit_masse_2025}. Wir zeigen, dass eine krümmungsbasierte Interpretation der Rotverschiebung, die auf der Einführung eines linearen Terms $\kappa r$ im Gravitationspotential basiert, die Notwendigkeit einer universellen Expansion eliminiert und gleichzeitig die geometrische Grundlage der allgemeinen Relativitätstheorie beibehält. Sowohl das erweiterte Standardmodell als auch das T0-Modell liefern mathematisch äquivalente Beschreibungen der physikalischen Realität, unterscheiden sich jedoch grundlegend in ihrer philosophischen Interpretation von Zeit und Masse. Wir skizzieren entscheidende Experimente, die helfen könnten zu bestimmen, welcher Rahmen die physikalische Realität genauer widerspiegelt.
	\end{abstract}
	
	\tableofcontents
	\newpage
	
	\section{Einführung: Grenzen des aktuellen Standardmodells}
	\label{sec:introduction}
	
	Das Standardmodell der Physik weist trotz bemerkenswerter Erfolge in vielen Bereichen mehrere theoretische und beobachtungsbedingte Inkonsistenzen auf:
	
	\begin{enumerate}
		\item Die Unvereinbarkeit zwischen allgemeiner Relativitätstheorie\cite{einstein1915} und Quantenmechanik\cite{schrodinger1926}
		\item Die unerklärte Beschleunigung der kosmischen Expansion (dunkle Energie)\cite{perlmutter1999}
		\item Anomale galaktische Rotationskurven, die dunkle Materie erfordern\cite{rubin1980}
		\item Spannungen bei Messungen der Hubble-Konstante\cite{riess2019}
		\item Das Hierarchieproblem zwischen elektroschwacher und Planck-Skala\cite{thooft1980}
		\item Das Problem der kosmologischen Konstante\cite{weinberg1989}
	\end{enumerate}
	
	Diese Einschränkungen deuten darauf hin, dass das Standardmodell eher unvollständig als inkorrekt ist. Währenddessen bietet das alternative T0-Modell von Pascher eine andere Grundlage, die auf absoluter Zeit und variabler Masse basiert\cite{pascher_zeit_2025} und potenziell diese Probleme mit seinem intrinsischen Zeitfeld $\Tfield = \frac{\hbar}{\max(mc^2, \omega)}$ adressiert\cite{pascher_lagrange_2025}. Eine vollständige Aufgabe des Standardrahmens ist jedoch nicht notwendig, wenn entsprechende Erweiterungen eine gleichwertige Vorhersagekraft erreichen können.
	
	Eine Schlüsselinnovation, die wir vorschlagen, ist die Einführung eines modifizierten Gravitationspotentials mit einem linearen Term:
	
	\begin{equation}
		\label{eq:modified_potential}
		\Phi(r) = -\frac{GM}{r} + \kappa r
	\end{equation}
	
	Mit $\kappa \approx 4,8\times10^{-11}$ m/s²\cite{pascher_params_2025}. Diese einzelne Modifikation kaskadiert durch die Theorie und ermöglicht eine krümmungsbasierte Erklärung für die Rotverschiebung, die die Notwendigkeit einer universellen Expansion eliminiert, während der geometrische Rahmen der allgemeinen Relativitätstheorie erhalten bleibt. Die Implikationen dieser Modifikation werden in den folgenden Abschnitten ausführlich untersucht (siehe \cref{sec:extended_gravity}).
	
	\section{Das T0-Modell: Grundlegende Konzepte}
	\label{sec:t0_model_fundamentals}
	
	Bevor wir unsere Erweiterungen des Standardmodells einführen, müssen wir die Schlüsselelemente des T0-Modells verstehen, mit denen unsere Erweiterung kompatibel sein wird. Das T0-Modell basiert auf dem Konzept der Zeit-Masse-Dualität\cite{pascher_zeit_masse_2025} und weist mehrere Kernprinzipien auf:
	
	\subsection{Intrinsisches Zeitfeld}
	\label{subsec:intrinsic_time}
	
	Das zentrale Konzept des T0-Modells ist das intrinsische Zeitfeld:
	
	\begin{equation}
		\label{eq:intrinsic_time}
		\Tfield = \frac{\hbar}{\max(mc^2, \omega)}
	\end{equation}
	
	Dieses Feld hat die Dimension $[E^{-1}]$ in natürlichen Einheiten\cite{pascher_alpha_2025} und repräsentiert die charakteristische Zeitskala, die mit jedem massiven Teilchen oder Energiequantum verbunden ist. Für massive Teilchen vereinfacht es sich zu $\Tfield = \frac{\hbar}{mc^2}$, während es für masselose Teilchen wie Photonen zu $\Tfield = \frac{\hbar}{\omega}$ wird.
	
	\subsection{Variable Masse vs. Zeitdilatation}
	\label{subsec:variable_mass}
	
	Das T0-Modell schlägt zwei äquivalente Bilder der Realität vor\cite{pascher_zeit_2025}:
	
	\begin{enumerate}
		\item Das \textbf{Standardbild}: Zeit ist relativ (zeigt Dilatation), während die Ruhemasse konstant bleibt. Dies entspricht dem konventionellen relativistischen Ansatz\cite{einstein1905}.
		\item Das \textbf{T0-Bild}: Zeit ist absolut, aber die Masse variiert mit der Geschwindigkeit gemäß $m = \gammaf m_0$, wobei $\gammaf$ der Lorentz-Faktor ist.
	\end{enumerate}
	
	Diese Bilder sind mathematisch äquivalent, führen jedoch zu unterschiedlichen konzeptionellen Interpretationen physikalischer Phänomene.
	
	\subsection{Natürliches Einheitensystem}
	\label{subsec:natural_units}
	
	Das T0-Modell verwendet ein hierarchisches natürliches Einheitensystem\cite{pascher_alphabeta_2025}, bei dem:
	
	\begin{equation}
		\label{eq:natural_units}
		\hbar = c = G = k_B = 1
	\end{equation}
	
	Zusätzlich werden dimensionslose Kopplungskonstanten auf Eins normiert:
	
	\begin{equation}
		\label{eq:dimensionless_constants}
		\alphaEM = \alphaW = \betaT = 1
	\end{equation}
	
	In SI-Einheiten haben diese Konstanten die gemessenen Werte:
	\begin{align}
		\alphaEM^{\text{SI}} &\approx 1/137,036 \quad \text{(Feinstrukturkonstante)}\cite{mohr2018} \\
		\alphaW^{\text{SI}} &\approx 2,82 \quad \text{(Wiensche Konstante)}\cite{wien1896} \\
		\betaT^{\text{SI}} &\approx 0,008 \quad \text{(T0-Parameter)}\cite{pascher_params_2025}
	\end{align}
	
	Diese Normierung offenbart eine tiefere Einheit der physikalischen Gesetze, indem alle physikalischen Größen in Bezug auf Energie ausgedrückt werden\cite{pascher_alpha_2025}.
	
	\subsection{Modifiziertes Gravitationspotential}
	\label{subsec:modified_gravity}
	
	Im T0-Modell entsteht Gravitation aus Gradienten im intrinsischen Zeitfeld\cite{pascher_emergente_gravitation_2025}, was zum modifizierten Gravitationspotential führt:
	
	\begin{equation}
		\label{eq:t0_gravity}
		\Phi(r) = -\frac{GM}{r} + \kappa r
	\end{equation}
	
	Dieses Potential erklärt sowohl galaktische Rotationskurven ohne dunkle Materie\cite{pascher_galaxies_2025} als auch kosmologische Rotverschiebung ohne Expansion\cite{pascher_messdifferenzen_2025}.
	
	\subsection{Wellenlängenabhängige Rotverschiebung}
	\label{subsec:wavelength_redshift}
	
	Das T0-Modell sagt eine wellenlängenabhängige Rotverschiebung voraus\cite{pascher_messdifferenzen_2025}:
	
	\begin{equation}
		\label{eq:t0_redshift}
		z(\lambda) = z_0\left(1 + \betaT^{\text{SI}} \ln\frac{\lambda}{\lambda_0}\right)
	\end{equation}
	
	Dies ist eine direkte Folge der Wechselwirkung zwischen elektromagnetischen Wellen und dem intrinsischen Zeitfeld\cite{pascher_photons_2025}.
	
	\subsection{Modifizierte Temperatur-Rotverschiebungs-Beziehung}
	\label{subsec:temp_redshift}
	
	Anstelle der Standardbeziehung $T(z) = T_0(1+z)$ sagt das T0-Modell voraus\cite{pascher_temp_2025}:
	
	\begin{equation}
		\label{eq:t0_temperature}
		T(z) = T_0 (1+z)(1+\betaT^{\text{SI}} \ln(1+z))
	\end{equation}
	
	Dies führt zu systematisch höheren Temperaturen bei hohen Rotverschiebungen im Vergleich zum Standardmodell\cite{pascher_temp_2025}, mit einem Faktor von etwa 1,6 bei $z = 1101$ (Rekombinationsepoche).
	
	\section{Mathematische Erweiterungen des Standardmodells}
	\label{sec:mathematical_extensions}
	
	Aufbauend auf unserem Verständnis des T0-Modells erweitern wir nun systematisch das Standardmodell, um mathematische Äquivalenz mit dem T0-Modell zu erreichen, während das Kernprinzip der Zeitdilatation bewahrt wird.
	
	\subsection{Erweiterte Einstein-Feldgleichungen}
	\label{subsec:extended_einstein}
	
	Die Standard-Einstein-Feldgleichungen\cite{einstein1915} können erweitert werden zu:
	
	\begin{equation}
		\label{eq:extended_einstein}
		G_{\mu\nu} + \kappa g_{\mu\nu} = 8\pi G T_{\mu\nu} + \nabla_{\mu}\Theta\nabla_{\nu}\Theta - \frac{1}{2}g_{\mu\nu}(\nabla_{\sigma}\Theta\nabla^{\sigma}\Theta)
	\end{equation}
	
	Wobei $\Theta$ ein Skalarfeld ist, das die Effekte berücksichtigt, die im T0-Modell dem intrinsischen Zeitfeld zugeschrieben werden\cite{pascher_lagrange_2025}. Diese Erweiterung behält die geometrische Interpretation der Gravitation bei, während sie Effekte einführt, die den Ansatz der variablen Masse des T0-Modells nachahmen. Der Term $\kappa g_{\mu\nu}$ ist analog zum Term der kosmologischen Konstante, hat aber eine andere physikalische Interpretation\cite{pascher_emergente_gravitation_2025}.
	
	\subsection{Krümmungsbasierte Rotverschiebungsformel}
	\label{subsec:curvature_redshift}
	
	Das Standardmodell interpretiert kosmische Rotverschiebung als Beleg für universelle Expansion. Wir schlagen eine alternative Interpretation vor, die vollständig auf der Raumkrümmung basiert, die durch das modifizierte Gravitationspotential erzeugt wird (siehe \cref{eq:modified_potential}):
	
	\begin{equation}
		\Phi(r) = -\frac{GM}{r} + \kappa r
	\end{equation}
	
	Dies erzeugt eine statische, gekrümmte Raumzeit mit der Metrik\cite{pascher_galaxies_2025}:
	
	\begin{equation}
		\label{eq:modified_metric}
		ds^2 = (1 - \frac{2GM}{r} + 2\kappa r)dt^2 - (1 + \frac{2GM}{r} - 2\kappa r)dr^2 - r^2d\Omega^2
	\end{equation}
	
	In der allgemeinen Relativitätstheorie wird die Gravitationsrotverschiebung zwischen zwei Punkten gegeben durch\cite{weinberg1972}:
	
	\begin{equation}
		\label{eq:grav_redshift}
		1 + z = \sqrt{\frac{g_{00}(\text{Emission})}{g_{00}(\text{Beobachtung})}}
	\end{equation}
	
	Mit unserer modifizierten Metrik mit $g_{00} = (1 - \frac{2GM}{r} + 2\kappa r)$ und unter Berücksichtigung von Pfaden, bei denen der $\kappa r$-Term dominiert, entsteht die Rotverschiebungsformel\cite{pascher_messdifferenzen_2025}:
	
	\begin{equation}
		\label{eq:extended_redshift}
		1 + z = e^{\alpha d}(1 + \beta \ln(\lambda/\lambda_0))
	\end{equation}
	
	mit $\alpha \approx 2,3\times10^{-18}$ m$^{-1}$ und $\beta \approx 0,008$ in SI-Einheiten. Diese Formel ist äquivalent zum Ausdruck des T0-Modells, wobei $\beta = \betaT^{\text{SI}}$ (siehe \cref{eq:t0_redshift}). Die Wellenlängenabhängigkeit ergibt sich natürlich aus den dispersiven Eigenschaften der gekrümmten Raumzeit, wobei $\kappa$ eine Frequenzabhängigkeit aufweist\cite{pascher_temp_2025}:
	
	\begin{equation}
		\label{eq:kappa_wavelength}
		\kappa(\lambda) = \kappa_0(1 + \beta \ln(\lambda/\lambda_0))
	\end{equation}
	
	Dieser Ansatz behält die geometrische Grundlage der allgemeinen Relativitätstheorie bei, während er die Notwendigkeit einer universellen Expansion vollständig eliminiert.
	
	\subsection{Erweitertes Gravitationspotential}
	\label{sec:extended_gravity}
	
	Das Newtonsche Gravitationspotential wird erweitert, um einen linearen Term einzuschließen\cite{pascher_galaxies_2025}:
	
	\begin{equation}
		\label{eq:extended_potential}
		\Phi(r) = -\frac{GM}{r} + \kappa r
	\end{equation}
	
	mit $\kappa \approx 4,8\times10^{-11}$ m/s². Diese Modifikation erklärt beobachtete galaktische Dynamiken ohne dunkle Materie zu benötigen\cite{mcgaugh2016} und bleibt dabei kompatibel mit einer relativistischen Interpretation. Dieser Parameter $\kappa$ hat die Dimension $[E]$ in natürlichen Einheiten\cite{pascher_params_2025} und ist direkt mit dem T0-Parameter $\betaT$ durch die Beziehung verbunden:
	
	\begin{equation}
		\label{eq:kappa_beta}
		\kappa^{\text{nat}} = \betaT^{\text{nat}} \cdot \frac{yv}{r_g^2}\betaT^{\text{nat}} \cdot \frac{yv}{r_g^2}
	\end{equation}
	
	wobei $y$ die Yukawa-Kopplung, $v$ der Higgs-Vakuumerwartungswert und $r_g$ eine charakteristische Längenskala ist\cite{pascher_lagrange_2025}.
	
	\subsection{Modifizierte Quantenevolution}
	\label{subsec:quantum_evolution}
	
	Die Standard-Quantenmechanik\cite{schrodinger1926} kann mit einer massenabhängigen Zeitevolutionskorrektur erweitert werden:
	
	\begin{equation}
		\label{eq:extended_schrodinger}
		i\hbar\frac{\partial\Psi}{\partial t} = [\hat{H} + \hat{H}_{\Theta}]\Psi
	\end{equation}
	
	wobei $\hat{H}_{\Theta}$ Massenabhängigkeit in die Zeitentwicklung einführt:
	
	\begin{equation}
		\label{eq:h_theta}
		\hat{H}_{\Theta} = -i\hbar\frac{\partial\Theta}{\partial t}\Psi
	\end{equation}
	
	Dies bewahrt den Standard-Schrödinger-Rahmen und berücksichtigt gleichzeitig Effekte, die das T0-Modell der variablen Masse zuschreibt\cite{pascher_zeit_2025}. Dieser Ansatz ist mathematisch äquivalent zur modifizierten Schrödinger-Gleichung im T0-Modell\cite{pascher_lagrange_2025}:
	
	\begin{equation}
		\label{eq:t0_schrodinger}
		i\hbar \Tfield \frac{\partial\Psi}{\partial t} + i\hbar \Psi \frac{\partial \Tfield}{\partial t} = \hat{H} \Psi
	\end{equation}
	
	\subsection{Erweiterte Lagrange-Dichte}
	\label{subsec:extended_lagrangian}
	
	Die Standardmodell-Lagrange-Dichte kann erweitert werden zu\cite{pascher_lagrange_2025}:
	
	\begin{equation}
		\label{eq:total_lagrangian}
		\mathcal{L}_{\text{Total}} = \mathcal{L}_{\text{SM}} + \mathcal{L}_{\Theta}
	\end{equation}
	
	wobei $\mathcal{L}_{\text{SM}}$ die Standardmodell-Lagrange-Dichte ist und:
	
	\begin{equation}
		\label{eq:theta_lagrangian}
		\mathcal{L}_{\Theta} = \frac{1}{2}\partial_{\mu}\Theta\partial^{\mu}\Theta - V(\Theta) + f(\Theta)F_{\mu\nu}F^{\mu\nu} + g(\Theta)\bar{\psi}\gamma^{\mu}\partial_{\mu}\psi
	\end{equation}
	
	mit Kopplungsfunktionen $f(\Theta)$ und $g(\Theta)$, die Wechselwirkungen von $\Theta$ mit elektromagnetischen Feldern und Fermionen berücksichtigen. Diese Erweiterung ist mathematisch äquivalent zur Lagrange-Dichte des intrinsischen Zeitfeldes im T0-Modell\cite{pascher_qft_2025}:
	
	\begin{equation}
		\label{eq:intrinsic_lagrangian}
		\mathcal{L}_{\text{intrinsic}} = \frac{1}{2} \partial_\mu \Tfield \partial^\mu \Tfield - \frac{1}{2}\Tfield^2 - \frac{\rho}{\Tfield}
	\end{equation}
	
	\subsection{Neuinterpretation natürlicher Einheiten}
	\label{subsec:reinterpretation}
	
	Wir schlagen vor, dass fundamentale Konstanten sich bei geeigneten Energieskalen einfachen Werten annähern\cite{pascher_alpha_2025}:
	
	\begin{equation}
		\label{eq:alpha_running}
		\alpha_{\text{EM}}(\mu) = \alpha_{\text{EM}}(0)[1 + \eta \ln(\mu/\mu_0)]
	\end{equation}
	
	Dies deutet darauf hin, dass die scheinbare Komplexität dimensionaler Konstanten eher ein Artefakt unserer Messsysteme als ein fundamentales Merkmal der Natur ist\cite{duff2002}. In den natürlichen Einheiten des T0-Modells\cite{pascher_alpha_2025, pascher_alphabeta_2025} werden dimensionslose Kopplungskonstanten auf Eins gesetzt:
	
	\begin{equation}
		\label{eq:dimensionless_unity}
		\alphaEM = \alphaW = \betaT = 1
	\end{equation}
	
	Während in SI-Einheiten diese Parameter die empirischen Werte haben\cite{mohr2018, pascher_params_2025}:
	\begin{align}
		\alphaEM^{\text{SI}} &\approx 1/137,036 \\
		\alphaW^{\text{SI}} &\approx 2,82 \\
		\betaT^{\text{SI}} &\approx 0,008
	\end{align}
	
	\section{Integration krümmungsbasierter Kosmologie mit dem erweiterten Standardmodell}
	\label{sec:integration}
	
	Die Einführung krümmungsbasierter Rotverschiebung schafft einen einheitlichen Rahmen, der sich nahtlos in die anderen Erweiterungen des Standardmodells integriert:
	
	\subsection{Verbindung zum Skalarfeld $\Theta$}
	\label{subsec:theta_connection}
	
	Das Skalarfeld $\Theta$ in unseren erweiterten Einstein-Feldgleichungen (siehe \cref{eq:extended_einstein}):
	
	\begin{equation}
		G_{\mu\nu} + \kappa g_{\mu\nu} = 8\pi G T_{\mu\nu} + \nabla_{\mu}\Theta\nabla_{\nu}\Theta - \frac{1}{2}g_{\mu\nu}(\nabla_{\sigma}\Theta\nabla^{\sigma}\Theta)
	\end{equation}
	
	erzeugt genau die Raumzeitkrümmung, die benötigt wird, um sowohl das modifizierte Gravitationspotential als auch den resultierenden Rotverschiebungseffekt zu erzeugen\cite{pascher_emergente_gravitation_2025}. Die Feldgleichung:
	
	\begin{equation}
		\label{eq:theta_field_eq}
		\nabla^2\Theta + V'(\Theta) = 0
	\end{equation}
	
	erzeugt bei Lösung den $\kappa r$-Term im physikalischen Raum. Diese Feldgleichung ist äquivalent zur T0-Modell-Feldgleichung für das intrinsische Zeitfeld\cite{pascher_lagrange_2025}:
	
	\begin{equation}
		\label{eq:t_field_eq}
		\nabla^2 \Tfield = -\kappa \rho(x) \Tfield^2
	\end{equation}
	
	\subsection{Vereinheitlichung von Quanten- und kosmologischen Phänomenen}
	\label{subsec:quantum_cosmo}
	
	Dieser Ansatz überbrückt Quanten- und kosmologische Skalen\cite{pascher_part1_2025, pascher_part2_2025}:
	
	\begin{enumerate}
		\item Auf Quantenskalen modifiziert das $\Theta$-Feld die Quantenevolution durch (siehe \cref{eq:extended_schrodinger}):
		\begin{equation}
			i\hbar\frac{\partial\Psi}{\partial t} = [\hat{H} + \hat{H}_{\Theta}]\Psi
		\end{equation}
		
		\item Auf galaktischen Skalen erzeugt es flache Rotationskurven über das modifizierte Potential (siehe \cref{eq:extended_potential}):
		\begin{equation}
			\Phi(r) = -\frac{GM}{r} + \kappa r
		\end{equation}
		
		\item Auf kosmologischen Skalen erzeugt es Rotverschiebung durch Raumzeitkrümmung (siehe \cref{eq:extended_redshift}):
		\begin{equation}
			1 + z = e^{\alpha d}(1 + \beta \ln(\lambda/\lambda_0))
		\end{equation}
	\end{enumerate}
	
	\subsection{Äquivalenz mit dem T0-Modell}
	\label{subsec:model_equivalence}
	
	Das erweiterte Standardmodell mit krümmungsbasierter Rotverschiebung erreicht mathematische Äquivalenz mit dem T0-Modell\cite{pascher_messdifferenzen_2025, pascher_feldtheorie_2025}:
	
	\begin{enumerate}
		\item Das Skalarfeld $\Theta$ entspricht funktional dem intrinsischen Zeitfeld $\Tfield = \frac{\hbar}{\max(mc^2, \omega)}$ im T0-Modell\cite{pascher_lagrange_2025}
		\item Das modifizierte Gravitationspotential erzeugt identische Vorhersagen für galaktische Rotationskurven\cite{pascher_galaxies_2025}
		\item Die krümmungsbasierte Rotverschiebung erzeugt die gleiche wellenlängenabhängige Rotverschiebungsformel wie das T0-Modell\cite{pascher_messdifferenzen_2025}, spezifisch (siehe \cref{eq:t0_redshift}):
		\begin{equation}
			z(\lambda) = z_0(1 + \betaT \ln(\lambda/\lambda_0))
		\end{equation}
		\item Die Quantenevolutionskorrekturen reproduzieren die Effekte der modifizierten Schrödinger-Gleichung des T0-Modells\cite{pascher_zeit_2025}
		\item Die Temperatur-Rotverschiebungs-Beziehung in unserem erweiterten Modell (siehe \cref{eq:t0_temperature}):
		\begin{equation}
			T(z) = T_0 (1+z)(1+\beta \ln(1+z))
		\end{equation}
		ist identisch mit der T0-Modell-Formulierung mit $\beta = \betaT^{\text{SI}} \approx 0,008$\cite{pascher_temp_2025}
	\end{enumerate}
	
	Der fundamentale Unterschied bleibt philosophisch: Das erweiterte Standardmodell erklärt diese Phänomene durch Raumzeitgeometrie unter Beibehaltung der Zeitdilatation mit konstanter Ruhemasse, während das T0-Modell absolute Zeit mit variabler Masse annimmt\cite{pascher_zeit_masse_2025}.
	
	\section{Experimentelle Tests zur Unterscheidung der Modelle}
	\label{sec:experimental_tests}
	
	Obwohl mathematisch äquivalent in ihren Vorhersagen, unterscheiden sich das erweiterte Standardmodell und das T0-Modell in ihren fundamentalen Interpretationen. Die folgenden Experimente könnten helfen zu bestimmen, welcher Rahmen die physikalische Realität besser widerspiegelt:
	
	\subsection{Präzisionstests der wellenlängenabhängigen Rotverschiebung}
	\label{subsec:redshift_tests}
	
	Beide Modelle sagen wellenlängenabhängige Rotverschiebung voraus, aber aus unterschiedlichen Mechanismen\cite{pascher_messdifferenzen_2025}. Präzise spektroskopische Messungen über mehrere Wellenlängenbänder für Objekte in verschiedenen Entfernungen könnten zwischen expansionsbasierter und energieverlustbasierter Rotverschiebung unterscheiden. Basierend auf unseren Berechnungen sollte der Unterschied in der Rotverschiebung zwischen den Wellenlängen $\lambda_1$ und $\lambda_2$ sein:
	
	\begin{equation}
		\label{eq:delta_z}
		\Delta z = z_0 \cdot \beta \ln(\lambda_2/\lambda_1)
	\end{equation}
	
	Beispielsweise erwarten wir über den JWST-Beobachtungsbereich eine Variation von etwa $\Delta z / z \approx 3,85\%$\cite{pascher_messdifferenzen_2025}.
	
	\subsection{Gravitationswellenausbreitung}
	\label{subsec:grav_wave_tests}
	
	Die beiden Modelle könnten subtil unterschiedliches Verhalten für die Ausbreitung von Gravitationswellen vorhersagen\cite{pascher_emergente_gravitation_2025}, insbesondere hinsichtlich Dispersion und Energieverlustmechanismen über kosmologische Distanzen\cite{abbott2016}.
	
	\subsection{Schwarzloch-Horizontphysik}
	\label{subsec:black_hole_tests}
	
	In der Nähe des Ereignishorizonts von Schwarzen Löchern könnten die beiden Interpretationen zu unterschiedlichen Vorhersagen bezüglich des Verhaltens von Quantenfeldern und des Informationsparadoxons führen\cite{hawking1975, pascher_emergente_gravitation_2025}.
	
	\subsection{Hochpräzisionstests des Äquivalenzprinzips}
	\label{subsec:equivalence_tests}
	
	Das erweiterte Standardmodell und das T0-Modell könnten unterschiedliche Korrekturen höherer Ordnung zum Äquivalenzprinzip vorhersagen, wenn Quanteneffekte berücksichtigt werden\cite{pascher_feldtheorie_2025, will2014}.
	
	\subsection{Frühes Universum Physik}
	\label{subsec:early_universe_tests}
	
	Die kosmologischen Vorhersagen der Modelle divergieren signifikant hinsichtlich des frühen Universums, wobei das Standardmodell Inflation\cite{guth1981} erfordert, während das T0-Modell ein statisches, ewiges Universum vorschlägt\cite{pascher_part2_2025}.
	
	\subsection{CMB-Temperaturberechnungen}
	\label{subsec:cmb_tests}
	
	Ein kritischer Test betrifft die Temperatur der kosmischen Mikrowellenhintergrundstrahlung bei hohen Rotverschiebungen\cite{fixsen2009}. Die beiden Modelle sagen unterschiedliche Temperatur-Rotverschiebungs-Beziehungen voraus\cite{pascher_temp_2025}, wobei die Temperaturvorhersage des T0-Modells etwa 1,6-mal höher ist als die des Standardmodells bei Rotverschiebung $z = 1101$ (Rekombinationsepoche). Dieser Faktor ergibt sich aus der vollständigen Temperatur-Rotverschiebungs-Beziehung (siehe \cref{eq:t0_temperature}):
	
	\begin{equation}
		\label{eq:detailed_temp}
		T(z) = T_0 (1+z)(1 + \betaT^{\text{SI}} \ln(1+z))
	\end{equation}
	
	Dieser Unterschied sollte in Präzisionsmessungen der CMB nachweisbar sein und könnte als entscheidender Unterscheidungstest zwischen den Modellen dienen\cite{pascher_temp_2025}.
	
	\section{Implikationen für die theoretische Physik}
	\label{sec:implications}
	
	\subsection{Paradigmenwechsel in der Kosmologie}
	\label{subsec:paradigm_shift}
	
	Die krümmungsbasierte Rotverschiebungsinterpretation stellt einen fundamentalen Wandel in der Kosmologie dar\cite{pascher_vereinheitlichung_2025}:
	
	\begin{enumerate}
		\item \textbf{Statisches Universum}: Das Universum kann statisch sein, anstatt zu expandieren\cite{einstein1917}, wodurch die Notwendigkeit für Inflation\cite{guth1981}, Urknall\cite{hubble1929} und dunkle Energie\cite{perlmutter1999} entfällt
		\item \textbf{Geometrische Rotverschiebung}: Rotverschiebung wird zu einem rein geometrischen Phänomen anstatt eines Doppler-Effekts\cite{pascher_messdifferenzen_2025}
		\item \textbf{Vereinheitlichte Erklärung}: Galaktische Rotationskurven\cite{mcgaugh2016} und kosmische Rotverschiebung teilen einen gemeinsamen Ursprung im $\kappa r$-Term\cite{pascher_emergente_gravitation_2025}
		\item \textbf{Kein Anfang}: Dieses Modell deutet auf ein ewiges Universum ohne Anfang oder Ende hin\cite{pascher_part2_2025}
	\end{enumerate}
	
	\subsection{Breitere theoretische Implikationen}
	\label{subsec:theoretical_implications}
	
	Der vollständige Rahmen hat tiefgreifende Implikationen für die theoretische Physik\cite{pascher_zeit_masse_2025}:
	
	\begin{enumerate}
		\item Er deutet darauf hin, dass unser Verständnis von Zeit und Masse rahmenabhängig statt fundamental sein könnte\cite{pascher_komplementaer_2025}
		\item Er zeigt, dass dunkle Materie und dunkle Energie möglicherweise Artefakte eines unvollständigen Standardmodells sind, anstatt notwendige Komponenten der Realität\cite{pascher_galaxies_2025}
		\item Er impliziert, dass Quantengravitation potenziell durch Erweiterungen bestehender Rahmen erreicht werden könnte, anstatt völlig neue Ansätze zu erfordern\cite{pascher_emergente_gravitation_2025}
		\item Er demonstriert, dass natürliche Einheiten mit allen Konstanten auf Eins gesetzt möglicherweise eine tiefere Realität widerspiegeln als unsere konventionellen Messsysteme\cite{pascher_alphabeta_2025}
	\end{enumerate}
	
	\subsection{Die Rolle des $\kappa r$-Terms}
	\label{subsec:kappa_r_role}
	
	Der lineare Term $\kappa r$ im Gravitationspotential erweist sich als Eckpfeiler des erweiterten Modells\cite{pascher_galaxies_2025}:
	
	\begin{enumerate}
		\item Er modifiziert galaktische Dynamiken und eliminiert die Notwendigkeit für dunkle Materie\cite{mcgaugh2016, pascher_galaxies_2025}
		\item Er erzeugt die Raumzeitkrümmung, die Rotverschiebung produziert, und eliminiert die Notwendigkeit für Expansion\cite{pascher_messdifferenzen_2025}
		\item Er verbindet sich mit dem Skalarfeld $\Theta$, das die Quantenevolution modifiziert\cite{pascher_lagrange_2025}
		\item Er führt eine fundamentale Längenskala ein, die sich auf das natürliche Einheitensystem bezieht\cite{pascher_params_2025}
	\end{enumerate}
	
	Dieser einzelne Term, mit $\kappa \approx 4,8\times10^{-11}$ m/s², löst mehrere ausstehende Probleme in der Physik, während er die geometrische Grundlage der allgemeinen Relativitätstheorie beibehält.
	
	\section{Mathematische Formulierung der krümmungsbasierten Rotverschiebung}
	\label{sec:math_formulation}
	
	Die mathematische Herleitung des krümmungsbasierten Rotverschiebungsmechanismus folgt direkt aus den geometrischen Prinzipien der allgemeinen Relativitätstheorie\cite{weinberg1972}:
	
	Ausgehend von der modifizierten Metrik aus unserem Gravitationspotential $\Phi(r) = -\frac{GM}{r} + \kappa r$ (siehe \cref{eq:modified_metric}):
	
	\begin{equation}
		ds^2 = (1 - \frac{2GM}{r} + 2\kappa r)dt^2 - (1 + \frac{2GM}{r} - 2\kappa r)dr^2 - r^2d\Omega^2
	\end{equation}
	
	Für einen Lichtweg durch diese gekrümmte Raumzeit ist die Rotverschiebung (siehe \cref{eq:grav_redshift}):
	
	\begin{equation}
		1 + z = \sqrt{\frac{g_{00}(r_1)}{g_{00}(r_2)}} = \sqrt{\frac{1 - \frac{2GM}{r_1} + 2\kappa r_1}{1 - \frac{2GM}{r_2} + 2\kappa r_2}}
	\end{equation}
	
	Für große Entfernungen, bei denen $\kappa r$ dominiert, und unter Berücksichtigung der Pfadintegration\cite{pascher_messdifferenzen_2025}:
	
	\begin{equation}
		\label{eq:base_redshift}
		1 + z \approx e^{\alpha d}
	\end{equation}
	
	Wobei $\alpha$ mit dem Gradienten der Metrik entlang des Pfades zusammenhängt.
	
	Die Wellenlängenabhängigkeit entsteht dadurch, dass $\kappa$ frequenzabhängig gemacht wird (siehe \cref{eq:kappa_wavelength}):
	
	\begin{equation}
		\kappa(\lambda) = \kappa_0(1 + \beta \ln(\lambda/\lambda_0))
	\end{equation}
	
	Dies ergibt unsere vollständige Rotverschiebungsformel (siehe \cref{eq:extended_redshift}):
	
	\begin{equation}
		1 + z = e^{\alpha d}(1 + \beta \ln(\lambda/\lambda_0))
	\end{equation}
	
	Mit $\alpha \approx 2,3\times10^{-18}$ m$^{-1}$ und $\beta \approx 0,008$ entstehen diese Werte natürlich aus den Konstanten der Theorie und sind keine angepassten Parameter\cite{pascher_params_2025}. In natürlichen Einheiten mit $\betaT^{\text{nat}} = 1$ vereinfacht sich diese Beziehung zu\cite{pascher_alphabeta_2025}:
	
	\begin{equation}
		\label{eq:natural_redshift}
		1 + z = e^{\alpha d}(1 + \ln(\lambda/\lambda_0))
	\end{equation}
	
	\section{Unterstützende Forschung für krümmungsbasierte Rotverschiebung}
	\label{sec:supporting_research}
	
	Mehrere etablierte Arbeiten untersuchen Konzepte, die der krümmungsbasierten Rotverschiebung ähnlich sind und Aspekte dieses theoretischen Rahmens unterstützen:
	
	\subsection{Alternative Rotverschiebungsmechanismen}
	\label{subsec:alternative_redshift}
	
	\begin{enumerate}
		\item \textbf{Ari Brynjolfssons "Redshift of photons penetrating a hot plasma"}\cite{brynjolfsson2004} (2004) - Diskutiert, wie Photonen, die durch Plasma reisen, aufgrund der Wechselwirkung mit dem Medium Rotverschiebung erfahren können, anstatt durch Expansion.
		
		\item \textbf{Jean-Pierre Vigiers Arbeiten über "Non-Doppler redshift of some galactic objects"}\cite{vigier1990} - Untersuchte alternative Erklärungen für Rotverschiebung, basierend auf Lichtausbreitung durch verschiedene Medien.
		
		\item \textbf{Halton Arps "Quasars, Redshifts and Controversies"}\cite{arp1987} (1987) - Präsentierte Beobachtungsergebnisse, die die Standardinterpretation der Rotverschiebung als rein expansionsbasiert in Frage stellten.
		
		\item \textbf{Die "Tired Light"-Theorien, ursprünglich von Fritz Zwicky vorgeschlagen}\cite{zwicky1929} (1929) - Obwohl weitgehend abgelehnt, schlugen diese frühen Theorien vor, dass Photonen während der Langstreckenausbreitung Energie verlieren könnten.
		
		\item \textbf{Geoffrey Burbidge, Fred Hoyle und Jayant Narlikars "A Different Approach to Cosmology"}\cite{burbidge2000} (2000) - Präsentierten alternative kosmologische Modelle, die sich nicht allein auf Expansion stützen, um Rotverschiebung zu erklären.
	\end{enumerate}
	
	\subsection{Modifizierte Gravitationsansätze}
	\label{subsec:modified_gravity_lit}
	
	\begin{enumerate}
		\item \textbf{Paul Marmets Arbeit über "Non-Doppler Redshift of Some Galactic Objects"}\cite{marmet1988} - Untersuchte alternative Mechanismen für Rotverschiebung, basierend auf Wechselwirkungen mit dem intergalaktischen Medium.
		
		\item \textbf{Aspekte von Moffats Modifizierter Gravitation (MOG)}\cite{moffat2006} - Obwohl im Ansatz unterschiedlich, führt auch Modifikationen der Gravitationspotentiale ein, die die Raumzeitkrümmung beeinflussen.
		
		\item \textbf{Modifizierte Newtonsche Dynamik (MOND)}\cite{milgrom1983} - Bietet einen alternativen Ansatz zur Erklärung galaktischer Rotationskurven ohne dunkle Materie, konzeptionell verwandt mit unserem $\kappa r$-Term.
	\end{enumerate}
	
	\subsection{Wellenlängenabhängige Effekte}
	\label{subsec:wavelength_effects}
	
	\begin{enumerate}
		\item \textbf{Dispersive Lichtausbreitung im Quantenvakuum}\cite{drummond1980} - Forschung in der Quantenelektrodynamik, die zeigt, wie Vakuum frequenzabhängige Effekte auf sich ausbreitendes Licht haben kann.
		
		\item \textbf{Robert Dickes Arbeiten über das "Äquivalenzprinzip und die schwachen Wechselwirkungen"}\cite{dicke1957} - Frühe Arbeit, die Verbindungen zwischen Gravitation und Elektromagnetismus untersuchte, die die Lichtausbreitung beeinflussen könnten.
		
		\item \textbf{Quantengravitationseffekte auf Photonenausbreitung}\cite{amelino2009} - Theoretische Untersuchungen dazu, wie Quantengravitation wellenlängenabhängige Effekte in der Lichtausbreitung durch die Raumzeit einführen könnte.
	\end{enumerate}
	
	Obwohl keiner dieser Ansätze mit unserer vorgeschlagenen krümmungsbasierten Rotverschiebung mit dem $\kappa r$-Term identisch ist, zeigen sie kollektiv, dass Alternativen zur expansionsbasierten Rotverschiebung in der wissenschaftlichen Literatur ernsthaft in Betracht gezogen wurden, was die Glaubwürdigkeit unserer vorgeschlagenen Erweiterung des Standardmodells unterstützt.
	
	\section{Natürliches Einheitensystem im Kontext}
	\label{sec:natural_units_context}
	
	Das T0-Modell verwendet ein hierarchisches natürliches Einheitensystem, in dem alle fundamentalen Konstanten und dimensionslosen Kopplungskonstanten auf Eins gesetzt werden\cite{pascher_alphabeta_2025}. Dieser Ansatz offenbart die zugrunde liegende Einfachheit physikalischer Gesetze:
	
	\begin{enumerate}
		\item \textbf{Stufe 1 - Fundamentale Konstanten}: $\hbar = c = G = k_B = 1$\cite{planck1899}
		\item \textbf{Stufe 2 - Dimensionslose Kopplungskonstanten}: $\alphaEM = \alphaW = \betaT = 1$\cite{pascher_alpha_2025}
		\item \textbf{Stufe 3 - Abgeleitete Konstanten}: Alle anderen physikalischen Konstanten können aus diesen grundlegenden Normierungsbedingungen abgeleitet werden\cite{pascher_alphabeta_2025}
	\end{enumerate}
	
	In diesem vereinheitlichten natürlichen Einheitensystem können alle physikalischen Größen auf Potenzen der Energie $[E]$ reduziert werden\cite{pascher_alpha_2025}:
	\begin{itemize}
		\item Länge und Zeit: $[L] = [T] = [E^{-1}]$
		\item Masse und Temperatur: $[M] = [T_{\text{emp}}] = [E]$
		\item Elektrische Ladung: $[Q] = [1]$ (dimensionslos)
	\end{itemize}
	
	Die Entsprechung zwischen unserem erweiterten Standardmodell und diesem natürlichen Einheitensystem stellt eine tiefe Verbindung zwischen beiden Frameworks her und steht im Einklang mit neueren theoretischen Perspektiven zur fundamentalen Natur dimensionsloser Konstanten\cite{duff2002}.
	
	\section{Schlussfolgerung}
	\label{sec:conclusion}
	
	Die Standardinterpretation der Physik, basierend auf Zeitdilatation und konstanter Ruhemasse, kann erweitert werden, um vollständige Kompatibilität mit dem alternativen T0-Modell zu erreichen, indem eine krümmungsbasierte Interpretation der Rotverschiebung einbezogen wird. Dieser Ansatz konzentriert sich auf die Schlüsselergänzung des $\kappa r$-Terms zum Gravitationspotential, der eine Kaskade von Effekten auslöst, die wichtige ausstehende Probleme in der Physik adressieren.
	
	Dieses erweiterte Modell behält die geometrische Grundlage der allgemeinen Relativitätstheorie\cite{einstein1915} bei, während es kosmische Phänomene vollständig neu interpretiert:
	\begin{itemize}
		\item Rotverschiebung wird zu einer Konsequenz der Lichtausbreitung durch gekrümmte Raumzeit anstatt der Expansion\cite{pascher_messdifferenzen_2025}
		\item Galaktische Dynamiken werden durch die gleiche Krümmung erklärt, die Rotverschiebung erzeugt\cite{pascher_galaxies_2025}
		\item Quantenphänomene verbinden sich mit kosmologischen Effekten durch das Skalarfeld $\Theta$\cite{pascher_lagrange_2025}
		\item Die Temperatur-Rotverschiebungs-Beziehung folgt der modifizierten Form $T(z) = T_0 (1+z)(1+\beta \ln(1+z))$\cite{pascher_temp_2025}
	\end{itemize}
	
	Das Ergebnis ist ein statisches, ewiges Universumsmodell, das mathematisch äquivalente Vorhersagen zum T0-Modell liefert, während es die Standardinterpretation der Zeitdilatation beibehält. Diese Äquivalenz deutet darauf hin, dass das aktuelle Standardmodell eher unvollständig als inkorrekt ist.
	
	Zukünftige Experimente, die auf wellenlängenabhängige Rotverschiebung abzielen (siehe \cref{subsec:redshift_tests}) und Tests der Effekte des $\kappa r$-Terms werden entscheidend sein, um zu bestimmen, welcher Rahmen – das erweiterte Standardmodell mit seiner geometrischen Interpretation oder das T0-Modell mit seinem Ansatz variabler Masse – die fundamentale Natur der Realität genauer widerspiegelt. Bis solche definitiven Beweise vorliegen, verdienen beide Ansätze gleiche Berücksichtigung in der theoretischen Physik.
	
	\begin{thebibliography}{99}
		% Fundamental T0 Model References
		\bibitem{pascher_zeit_masse_2025} Pascher, J. (2025). \href{https://github.com/jpascher/T0-Time-Mass-Duality/tree/main/2/pdf/Deutsch/ZeitMasseNeuerBlick.pdf}{Zeit und Masse: Ein neuer Blick auf alte Formeln – und Befreiung von traditionellen Zwängen}. 22. März 2025.
		
		\bibitem{pascher_zeit_2025} Pascher, J. (2025). \href{https://github.com/jpascher/T0-Time-Mass-Duality/tree/main/2/pdf/Deutsch/ZeitEmergentQM.pdf}{Zeit als emergente Eigenschaft in der Quantenmechanik: Eine Verbindung zwischen Relativität, Feinstrukturkonstante und Quantendynamik}. 23. März 2025.
		
		\bibitem{pascher_params_2025} Pascher, J. (2025). \href{https://github.com/jpascher/T0-Time-Mass-Duality/tree/main/2/pdf/Deutsch/ZeitMasseT0Params.pdf}{Zeit-Masse-Dualitätstheorie (T0-Modell): Ableitung der Parameter $\kappa$, $\alpha$ und $\beta$}. 4. April 2025.
		
		\bibitem{pascher_komplementaer_2025} Pascher, J. (2025). \href{https://github.com/jpascher/T0-Time-Mass-Duality/tree/main/2/pdf/Deutsch/KomplementPhysikZeit.pdf}{Komplementäre Erweiterungen der Physik: Absolute Zeit und Intrinsische Zeit}. 24. März 2025.
		
		% Key Applications of T0 Model
		\bibitem{pascher_galaxies_2025} Pascher, J. (2025). \href{https://github.com/jpascher/T0-Time-Mass-Duality/tree/main/2/pdf/Deutsch/MassVarGalaxien.pdf}{Massenvariation in Galaxien: Eine Analyse im T0-Modell mit emergenter Gravitation}. 30. März 2025.
		
		\bibitem{pascher_messdifferenzen_2025} Pascher, J. (2025). \href{https://github.com/jpascher/T0-Time-Mass-Duality/tree/main/2/pdf/Deutsch/MessdifferenzenT0Standard.pdf}{Kompensatorische und additive Effekte: Eine Analyse der Messunterschiede zwischen dem T0-Modell und dem $\Lambda$CDM-Standardmodell}. 2. April 2025.
		
		\bibitem{pascher_temp_2025} Pascher, J. (2025). \href{https://github.com/jpascher/T0-Time-Mass-Duality/tree/main/2/pdf/Deutsch/TempEinheitenCMB.pdf}{Anpassung der Temperatureinheiten in natürlichen Einheiten und CMB-Messungen}. 2. April 2025.
		
		\bibitem{pascher_emergente_gravitation_2025} Pascher, J. (2025). \href{https://github.com/jpascher/T0-Time-Mass-Duality/tree/main/2/pdf/Deutsch/EmergentGravT0.pdf}{Emergente Gravitation im T0-Modell: Eine umfassende Ableitung}. 1. April 2025.
		
		\bibitem{pascher_photons_2025} Pascher, J. (2025). \href{https://github.com/jpascher/T0-Time-Mass-Duality/tree/main/2/pdf/Deutsch/DynMassePhotonenNichtlokal.pdf}{Dynamische Masse von Photonen und ihre Implikationen für Nichtlokalität im T0-Modell}. 25. März 2025.
		
		% Mathematical Foundations
		\bibitem{pascher_lagrange_2025} Pascher, J. (2025). \href{https://github.com/jpascher/T0-Time-Mass-Duality/tree/main/2/pdf/Deutsch/MathZeitMasseLagrange.pdf}{Von der Zeitdilatation zur Massenvariation: Mathematische Kernformulierungen der Zeit-Masse-Dualitätstheorie}. 29. März 2025.
		
		\bibitem{pascher_qft_2025} Pascher, J. (2025). \href{https://github.com/jpascher/T0-Time-Mass-Duality/tree/main/2/pdf/Deutsch/QFTIntrinsischesZeitT0.pdf}{Quantenfeldtheoretische Behandlung des intrinsischen Zeitfeldes im T0-Modell}. 8. April 2025.
		
		\bibitem{pascher_part1_2025} Pascher, J. (2025). \href{https://github.com/jpascher/T0-Time-Mass-Duality/tree/main/2/pdf/Deutsch/QMRelTimeMassPart1Z.pdf}{Brückenschlag zwischen Quantenmechanik und Relativitätstheorie durch Zeit-Masse-Dualität: Ein einheitlicher Rahmen mit natürlichen Einheiten $\alpha = \beta = 1$ Teil I: Theoretische Grundlagen}. 7. April 2025.
		
		\bibitem{pascher_part2_2025} Pascher, J. (2025). \href{https://github.com/jpascher/T0-Time-Mass-Duality/tree/main/2/pdf/Deutsch/QMRelTimeMassPart2Z.pdf}{Brückenschlag zwischen Quantenmechanik und Relativitätstheorie durch Zeit-Masse-Dualität: Ein einheitlicher Rahmen mit natürlichen Einheiten $\alpha = \beta = 1$ Teil II: Kosmologische Implikationen und experimentelle Validierung}. 7. April 2025.
		
		% Natural Units System
		\bibitem{pascher_alpha_2025} Pascher, J. (2025). \href{https://github.com/jpascher/T0-Time-Mass-Duality/tree/main/2/pdf/Deutsch/NatEinheitenAlpha1.pdf}{Energie als fundamentale Einheit: Natürliche Einheiten mit $\alpha = 1$ im T0-Modell}. 26. März 2025.
		
		\bibitem{pascher_alphabeta_2025} Pascher, J. (2025). \href{https://github.com/jpascher/T0-Time-Mass-Duality/tree/main/2/pdf/Deutsch/Alpha1Beta1Konsistenz.pdf}{Einheitliches Einheitensystem im T0-Modell: Die Konsistenz von $\alpha = 1$ und $\beta = 1$}. 5. April 2025.
		
		\bibitem{pascher_vereinheitlichung_2025} Pascher, J. (2025). \href{https://github.com/jpascher/T0-Time-Mass-Duality/tree/main/2/pdf/Deutsch/T0VereinheitlichungDEGal.pdf}{Vereinheitlichung des T0-Modells: Grundlagen, Dunkle Energie und Galaxiendynamik}. 4. April 2025.
		
		\bibitem{pascher_feldtheorie_2025} Pascher, J. (2025). \href{https://github.com/jpascher/T0-Time-Mass-Duality/tree/main/2/pdf/Deutsch/FeldtheorieQuanten.pdf}{Feldtheorie und Quantenkorrelationen: Eine neue Perspektive auf Instantaneität}. 28. März 2025.
		
		% Standard Physics References
		\bibitem{einstein1905} Einstein, A. (1905). Zur Elektrodynamik bewegter Körper. \textit{Annalen der Physik}, 322(10), 891-921.
		\bibitem{einstein1915} Einstein, A. (1915). Die Feldgleichungen der Gravitation. \textit{Sitzungsberichte der Preußischen Akademie der Wissenschaften zu Berlin}, 844-847.
		\bibitem{einstein1917} Einstein, A. (1917). Kosmologische Betrachtungen zur allgemeinen Relativitätstheorie. \textit{Sitzungsberichte der Preußischen Akademie der Wissenschaften}, 142-152.
		\bibitem{schrodinger1926} Schrödinger, E. (1926). Quantisierung als Eigenwertproblem. \textit{Annalen der Physik}, 384(4), 361-376.
		\bibitem{hubble1929} Hubble, E. (1929). A relation between distance and radial velocity among extra-galactic nebulae. \textit{Proceedings of the National Academy of Sciences}, 15(3), 168-173.
		\bibitem{guth1981} Guth, A. H. (1981). Inflationary universe: A possible solution to the horizon and flatness problems. \textit{Physical Review D}, 23(2), 347.
		\bibitem{weinberg1972} Weinberg, S. (1972). \textit{Gravitation and Cosmology: Principles and Applications of the General Theory of Relativity}. Wiley.
		\bibitem{weinberg1989} Weinberg, S. (1989). The cosmological constant problem. \textit{Reviews of Modern Physics}, 61(1), 1.
		\bibitem{hawking1975} Hawking, S. W. (1975). Particle creation by black holes. \textit{Communications in Mathematical Physics}, 43(3), 199-220.
		\bibitem{thooft1980} 't Hooft, G. (1980). Naturalness, chiral symmetry, and spontaneous chiral symmetry breaking. \textit{NATO Advanced Study Institutes Series B: Physics}, 59, 135-157.
		\bibitem{will2014} Will, C. M. (2014). The confrontation between general relativity and experiment. \textit{Living Reviews in Relativity}, 17(1), 4.
		\bibitem{mohr2018} Mohr, P. J., Newell, D. B., Taylor, B. N. (2018). CODATA recommended values of the fundamental physical constants: 2018. \textit{Reviews of Modern Physics}, 93(2), 025010.
		\bibitem{perlmutter1999} Perlmutter, S., et al. (1999). Measurements of $\Omega$ and $\Lambda$ from 42 high-redshift supernovae. \textit{The Astrophysical Journal}, 517(2), 565.
		\bibitem{riess2019} Riess, A. G., Casertano, S., Yuan, W., Macri, L. M., Scolnic, D. (2019). Large Magellanic Cloud Cepheid standards provide a 1\% foundation for the determination of the Hubble constant and stronger evidence for physics beyond $\Lambda$CDM. \textit{The Astrophysical Journal}, 876(1), 85.
		\bibitem{abbott2016} Abbott, B. P., et al. (2016). Observation of gravitational waves from a binary black hole merger. \textit{Physical Review Letters}, 116(6), 061102.
		\bibitem{rubin1980} Rubin, V. C., Ford Jr, W. K. (1980). Rotation of the Andromeda nebula from a spectroscopic survey of emission regions. \textit{The Astrophysical Journal}, 159, 379.
		\bibitem{mcgaugh2016} McGaugh, S. S., Lelli, F., Schombert, J. M. (2016). Radial acceleration relation in rotationally supported galaxies. \textit{Physical Review Letters}, 117(20), 201101.
		\bibitem{fixsen2009} Fixsen, D. J. (2009). The temperature of the cosmic microwave background. \textit{The Astrophysical Journal}, 707(2), 916.
		\bibitem{planck1899} Planck, M. (1899). Über irreversible Strahlungsvorgänge. \textit{Sitzungsberichte der Königlich Preußischen Akademie der Wissenschaften zu Berlin}, 5, 440-480.
		\bibitem{wien1896} Wien, W. (1896). Über die Energieverteilung im Emissionsspektrum eines schwarzen Körpers. \textit{Annalen der Physik}, 294(8), 662-669.
		\bibitem{duff2002} Duff, M. J., Okun, L. B., Veneziano, G. (2002). Trialogue on the number of fundamental constants. \textit{Journal of High Energy Physics}, 2002(03), 023.
		
		% Alternative Models References
		\bibitem{brynjolfsson2004} Brynjolfsson, A. (2004). \textit{Redshift of photons penetrating a hot plasma}. arXiv:astro-ph/0401420.
		\bibitem{arp1987} Arp, H. (1987). \textit{Quasars, Redshifts and Controversies}. Interstellar Media, Berkeley.
		\bibitem{zwicky1929} Zwicky, F. (1929). On the Red Shift of Spectral Lines through Interstellar Space. \textit{Proceedings of the National Academy of Sciences}, 15(10), 773-779.
		\bibitem{burbidge2000} Burbidge, G., Hoyle, F., Narlikar, J. V. (2000). \textit{A Different Approach to Cosmology: From a Static Universe through the Big Bang towards Reality}. Cambridge University Press.
		\bibitem{marmet1988} Marmet, P. (1988). A New Non-Doppler Redshift. \textit{Physics Essays}, 1(1), 24-32.
		\bibitem{moffat2006} Moffat, J. W. (2006). Scalar tensor vector gravity theory. \textit{Journal of Cosmology and Astroparticle Physics}, 2006(03), 004.
		\bibitem{dicke1957} Dicke, R. H. (1957). Principle of Equivalence and the Weak Interactions. \textit{Reviews of Modern Physics}, 29(3), 355.
		\bibitem{vigier1990} Vigier, J.-P. (1990). Evidence for nonzero mass photons associated with a vacuum-induced dissipative red-shift mechanism. \textit{IEEE Transactions on Plasma Science}, 18(1), 64-72.
		\bibitem{milgrom1983} Milgrom, M. (1983). A modification of the Newtonian dynamics as a possible alternative to the hidden mass hypothesis. \textit{The Astrophysical Journal}, 270, 365-370.
		\bibitem{drummond1980} Drummond, I. T., Hathrell, S. J. (1980). QED vacuum polarization in a background gravitational field and its effect on the velocity of photons. \textit{Physical Review D}, 22(2), 343.
		\bibitem{amelino2009} Amelino-Camelia, G., et al. (2009). Tests of quantum gravity from observations of $\gamma$-ray bursts. \textit{Nature}, 393(6687), 763-765.
	\end{thebibliography}
	
\end{document}