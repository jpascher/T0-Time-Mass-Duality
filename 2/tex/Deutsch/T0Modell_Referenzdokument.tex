\documentclass[12pt,a4paper]{article}
\usepackage[utf8]{inputenc}
\usepackage[T1]{fontenc}
\usepackage[english,ngerman]{babel}
\usepackage{lmodern}
\usepackage{amsmath,amssymb,physics}
\usepackage{hyperref}
\usepackage{tcolorbox}
\usepackage{booktabs}
\usepackage{enumitem}
\usepackage[table,xcdraw]{xcolor}
\usepackage[left=2cm,right=2cm,top=2cm,bottom=2cm]{geometry}
\usepackage{fancyhdr}

% Headers and Footers
\pagestyle{fancy}
\fancyhf{}
\fancyhead[L]{T0-Modell-Dokumentation}
\fancyhead[R]{Konsistenzhinweise}
\fancyfoot[C]{\thepage}
\renewcommand{\headrulewidth}{0.4pt}
\renewcommand{\footrulewidth}{0.4pt}

% Custom Commands
\newcommand{\Tfield}{T(x)}
\newcommand{\alphaEM}{\alpha_{\text{EM}}}
\newcommand{\alphaW}{\alpha_{\text{W}}}
\newcommand{\betaT}{\beta_{\text{T}}}
\newcommand{\Mpl}{M_{\text{Pl}}}
\newcommand{\Tzerot}{T_0(\Tfield)}
\newcommand{\Tzero}{T_0}
\newcommand{\vecx}{\vec{x}}
\newcommand{\gammaf}{\gamma_{\text{Lorentz}}}
\newcommand{\DhiggsT}{\Tfield (\partial_\mu + ig A_\mu) \Phi + \Phi \partial_\mu \Tfield}

\hypersetup{
	colorlinks=true,
	linkcolor=blue,
	citecolor=blue,
	urlcolor=blue,
	pdftitle={Konsistenzhinweise zum T0-Modell},
	pdfauthor={Johann Pascher},
	pdfsubject={Theoretische Physik},
	pdfkeywords={T0-Modell, Konsistenz, Notation, Dokumentation}
}

\title{Konsistenzhinweise und Formelverzeichnis zum T0-Modell}
\author{Johann Pascher}
\date{\today}

\begin{document}
	
	\maketitle
	
	\begin{abstract}
		Dieses Dokument enthält Hinweise zu bekannten Konsistenzproblemen in der T0-Modell-Dokumentation sowie ein Verzeichnis der verwendeten Formelzeichen. Es dient als Referenz für Leser der Dokumentation und als Anleitung für zukünftige Überarbeitungen. Die identifizierten Inkonsistenzen betreffen hauptsächlich die Darstellung der Feldgleichungen, Lagrange-Dichten und die Kennzeichnung von Näherungen.
	\end{abstract}
	
	\tableofcontents
	\newpage
	
	\section{Bekannte Konsistenzprobleme}
	
	\begin{tcolorbox}[colback=yellow!5!white,colframe=yellow!75!black,title=Hinweis für Leser]
		Die folgende Auflistung enthält bekannte Konsistenzprobleme in der aktuellen T0-Modell-Dokumentation. Diese werden in zukünftigen Überarbeitungen behoben. Bis dahin dient dieser Abschnitt als Orientierungshilfe, um die verschiedenen Darstellungen derselben Konzepte in den verschiedenen Dokumenten korrekt zu interpretieren.
	\end{tcolorbox}
	
	\subsection{Uneinheitliche Kennzeichnung von Näherungen}
	
	\begin{enumerate}[label=\textbf{P\arabic*.}]
		\item \textbf{Feldgleichung für $\Tfield$}: Die Grundgleichung $\nabla^2\Tfield \approx -\frac{\rho}{\Tfield^2}$ wird in manchen Dokumenten ohne das Approximationszeichen ($\approx$) dargestellt. Dies ist inkonsistent und sollte überall als Näherung gekennzeichnet werden.
		
		\textbf{Betroffene Dokumente:}
		\begin{itemize}
			\item MathZeitMasseLagrange.tex (Abschnitt 2)
			\item EmergentGravT0En.tex (Abschnitt 3.2)
			\item MassVarGalaxienEn.tex (Abschnitt 2.1)
			\item QMRelTimeMassPart1ZEn.tex (Abschnitt 5.1)
			\item NatEinheitenSystematikZEn.tex (Abschnitt 15.1)
		\end{itemize}
	\end{enumerate}
	
	\subsection{Varianten der Lagrange-Dichte}
	
	\begin{enumerate}[label=\textbf{P\arabic*.},resume]
		\item \textbf{Intrinsische Lagrange-Dichte}: Es werden verschiedene Versionen präsentiert ohne klare Abgrenzung oder Erklärung der Beziehungen zwischen den Varianten:
		
		\textbf{Einfache Version (freie Felddynamik):}
		\begin{equation}
			\mathcal{L}_{\text{intrinsic}} = \frac{1}{2} \partial_\mu \Tfield \partial^\mu \Tfield - \frac{1}{2}\Tfield^2
		\end{equation}
		
		\textbf{Mit Materiekopplung:}
		\begin{equation}
			\mathcal{L}_{\text{intrinsic}} = \frac{1}{2} \partial_\mu \Tfield \partial^\mu \Tfield - \frac{1}{2}\Tfield^2 - \frac{\rho}{\Tfield}
		\end{equation}
		
		\textbf{Vollständige Form:}
		\begin{equation}
			\mathcal{L}_{\text{intrinsic}}^{\text{complete}} = \underbrace{\frac{1}{2} \partial_\mu \Tfield \partial^\mu \Tfield - \frac{1}{2}\Tfield^2}_{\text{Freie Felddynamik}} + \underbrace{\bar{\psi} \left( i\hbar \gamma^0 \frac{\partial}{\partial (t/\Tfield)} - i\hbar \gamma^0 \frac{\partial}{\partial t} \right) \psi}_{\text{Wechselwirkung mit Materie}}
		\end{equation}
		
		\textbf{Betroffene Dokumente:}
		\begin{itemize}
			\item MathZeitMasseLagrange.tex
			\item NotwendigkeitQMErweiterungEn.tex (Abschnitt 5.1)
			\item EmergentGravT0En.tex (Abschnitt 2.3)
			\item QMRelTimeMassPart1ZEn.tex (Abschnitt 4.1)
		\end{itemize}
	\end{enumerate}
	
	\subsection{Parameter und Konstanten}
	
	\begin{enumerate}[label=\textbf{P\arabic*.},resume]
		\item \textbf{$\kappa$-Parameter}: Die Herleitung und Dimension des $\kappa$-Parameters wird unterschiedlich dargestellt:
		
		\textbf{Variante 1:} $\kappa = \beta_T \cdot \frac{yv}{r_g^2}$ mit Dimension $[E]$
		
		\textbf{Variante 2:} $\kappa^{\text{SI}} \approx 4.8 \times 10^{-11} \, \text{m/s}^2$ ohne explizite Herleitung
		
		\textbf{Betroffene Dokumente:}
		\begin{itemize}
			\item MassVarGalaxienEn.tex (Abschnitt 2.2)
			\item MessdifferenzenT0StandardEn.tex (Abschnitt 3.1)
			\item T0VereinheitlichungDEGal.tex (Abschnitt 4.2)
		\end{itemize}
	\end{enumerate}
	
	\subsection{Referenzen und URL-Struktur}
	
	\begin{enumerate}[label=\textbf{P\arabic*.},resume]
		\item \textbf{Uneinheitliche URL-Struktur}: Die Verweise auf andere Dokumente der Reihe folgen nicht immer dem gleichen Muster:
		
		\textbf{Korrekte Struktur:}
		\begin{itemize}
			\item Deutsch: \texttt{/pdf/Deutsch/Dateiname.pdf}
			\item Englisch: \texttt{/pdf/English/DateinameEn.pdf}
		\end{itemize}
		
		\textbf{Betroffene Dokumente:} Verschiedene Dokumente mit inkonsistenter Verlinkungsstruktur
	\end{enumerate}
	
	\subsection{Prioritäten für zukünftige Überarbeitungen}
	
	Die folgenden Maßnahmen sollten bei zukünftigen Überarbeitungen vorrangig umgesetzt werden:
	
	\begin{enumerate}[label=\textbf{M\arabic*.}]
		\item Einheitliche Kennzeichnung aller Näherungen mit dem Symbol $\approx$ in allen Feldgleichungen.
		
		\item Einführung einer konsistenten Darstellung der Lagrange-Dichte mit klaren Angaben, welche Version in welchem Kontext verwendet wird:
		\begin{itemize}
			\item Freie Felddynamik für Ausbreitungsstudien
			\item Mit Materiekopplung für gravitationsrelevante Anwendungen
			\item Vollständige Form für umfassende theoretische Darstellung
		\end{itemize}
		
		\item Vereinheitlichung der Herleitung und Dimension des $\kappa$-Parameters mit expliziter Verbindung zwischen theoretischer Form und SI-Wert.
		
		\item Standardisierung der URL-Struktur in allen Dokumenten entsprechend der korrekten Struktur.
	\end{enumerate}
	
	\section{Formelzeichenverzeichnis}
	
	\begin{tcolorbox}[colback=blue!5!white,colframe=blue!75!black,title=Formelzeichen des T0-Modells]
		Dieses Verzeichnis enthält die wichtigsten Formelzeichen des T0-Modells mit kurzen Erklärungen.
	\end{tcolorbox}
	
	\begin{table}[h]
		\centering
		\begin{tabular}{lp{12cm}}
			\toprule
			\textbf{Symbol} & \textbf{Bedeutung} \\
			\midrule
			$\Tfield$ & Intrinsisches Zeitfeld; fundamentales Feld mit Dimension $[E^{-1}]$ \\
			$\Tzero$ & Konstante intrinsische Zeit bei Ruhemasse \\
			$\hbar$ & Reduziertes Plancksches Wirkungsquantum; in natürlichen Einheiten $\hbar = 1$ \\
			$c$ & Lichtgeschwindigkeit; in natürlichen Einheiten $c = 1$ \\
			$G$ & Gravitationskonstante; in natürlichen Einheiten $G = 1$ \\
			$k_B$ & Boltzmann-Konstante; in natürlichen Einheiten $k_B = 1$ \\
			$\alphaEM$ & Feinstrukturkonstante; in natürlichen Einheiten $\alphaEM = 1$ \\
			$\alphaW$ & Wien-Konstante; in natürlichen Einheiten $\alphaW = 1$ \\
			$\betaT$ & T0-Parameter; in natürlichen Einheiten $\betaT = 1$ \\
			$\betaT^{\text{SI}}$ & T0-Parameter in SI-Einheiten; $\betaT^{\text{SI}} \approx 0,008$ \\
			$\gammaf$ & Lorentzfaktor; $\gammaf = 1/\sqrt{1-v^2/c^2}$ \\
			$\xi$ & Verhältnis T0-Länge zu Planck-Länge; $\xi = r_0/l_P \approx 1,33 \times 10^{-4}$ \\
			$r_0$ & Charakteristische T0-Länge; $r_0 = \xi \cdot l_P$ \\
			$l_P$ & Planck-Länge; $l_P = \sqrt{\hbar G/c^3}$ \\
			$\Phi(r)$ & Modifiziertes Gravitationspotential; $\Phi(r) = -GM/r + \kappa r$ \\
			$\kappa$ & Linearer Term im modifizierten Gravitationspotential; $\kappa \approx 4,8 \times 10^{-11} \, \text{m/s}^2$ \\
			$\lambda_h$ & Higgs-Selbstkopplung; $\lambda_h \approx 0,13$ \\
			$v$ & Higgs-Vakuumerwartungswert; $v \approx 246 \, \text{GeV}$ \\
			$m_h$ & Higgs-Masse; $m_h \approx 125 \, \text{GeV}$ \\
			$\omega$ & Kreisfrequenz oder Photonenergie \\
			$\rho$ & Massendichte oder allgemein Energiedichte \\
			$\DhiggsT$ & Modifizierte kovariante Ableitung für Higgs-Feld \\
			$\mathcal{L}$ & Lagrange-Dichte \\
			$z$ & Rotverschiebung \\
			$z_0$ & Referenzrotverschiebung \\
			$\lambda$ & Wellenlänge \\
			$\lambda_0$ & Referenzwellenlänge \\
			\bottomrule
		\end{tabular}
		\caption{Wichtige Formelzeichen des T0-Modells}
	\end{table}
	
	\section{Dimensionen in natürlichen Einheiten}
	
	\begin{table}[h]
		\centering
		\begin{tabular}{lll}
			\toprule
			\textbf{Physikalische Größe} & \textbf{SI-Einheit} & \textbf{Dimension in natürlichen Einheiten} \\
			\midrule
			Länge & m & $[E^{-1}]$ \\
			Zeit & s & $[E^{-1}]$ \\
			Masse & kg & $[E]$ \\
			Energie & J & $[E]$ \\
			Temperatur & K & $[E]$ \\
			Elektrische Ladung & C & $[1]$ (dimensionslos) \\
			Elektrisches Feld & V/m & $[E^2]$ \\
			Magnetisches Feld & T & $[E^2]$ \\
			Kraft & N & $[E^2]$ \\
			Druck & Pa & $[E^4]$ \\
			Vakuumpermittivität $\varepsilon_0$ & F/m & $[1]$ (in nat. Einheiten $\varepsilon_0 = 1$) \\
			Vakuumpermeabilität $\mu_0$ & H/m & $[1]$ (in nat. Einheiten $\mu_0 = 1$) \\
			\bottomrule
		\end{tabular}
		\caption{Dimensionen physikalischer Größen in natürlichen Einheiten}
	\end{table}
	
\end{document}