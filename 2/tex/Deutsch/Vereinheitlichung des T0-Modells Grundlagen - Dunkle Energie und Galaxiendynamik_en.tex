\documentclass[a4paper,12pt]{article}
\usepackage[utf8]{inputenc}
\usepackage[T1]{fontenc}
\usepackage{lmodern}
\usepackage[english]{babel}
\usepackage{amsmath, amssymb, amsthm, physics}
\usepackage{graphicx}
\usepackage{xcolor}
\usepackage{tikz}
\usepackage{setspace}
\usepackage{tcolorbox}
\usepackage{booktabs}

% Colored links in the table of contents and document
\usepackage{hyperref}
\hypersetup{
	colorlinks=true,
	linkcolor=blue,
	filecolor=blue,
	citecolor=blue, 
	urlcolor=blue,
	bookmarks=true,
	bookmarksopen=true,
	pdftitle={Unification of the T0 Model: Foundations, Dark Energy, and Galaxy Dynamics},
	pdfauthor={Johann Pascher},
}

% cleveref must be loaded after hyperref
\usepackage{cleveref}

% Theorem styles
\newtheorem{theorem}{Theorem}
\newtheorem{lemma}[theorem]{Lemma}
\newtheorem{proposition}[theorem]{Proposition}
\newtheorem{corollary}[theorem]{Corollary}

\theoremstyle{definition}
\newtheorem{definition}{Definition}

\theoremstyle{remark}
\newtheorem{remark}{Remark}

\begin{document}
	
	\title{Unification of the T0 Model: \\Foundations, Dark Energy, and Galaxy Dynamics}
	\author{Johann Pascher}
	\date{March 27, 2025}
	\maketitle
	
	\begin{abstract}
		This work presents a consistent unification of the T0 model and its applications to cosmological and astrophysical phenomena. The T0 model is based on the assumption of absolute time and variable mass, in contrast to relativity theory with relative time and constant mass. This fundamental reinterpretation leads to alternative explanations for cosmic redshift (through energy loss rather than expansion), dark energy (as a medium for energy exchange), and galaxy dynamics (through mass variation instead of dark matter). This paper ensures mathematical consistency across these various applications and provides a comprehensive theoretical framework that makes experimentally testable predictions.
	\end{abstract}
	
	\tableofcontents
	\newpage
	
	\section{Introduction to the T0 Model: Fundamental Concepts}
	
	\subsection{Basic Assumptions of the T0 Model}
	
	The T0 model is based on the following central assumptions:
	
	\begin{tcolorbox}[colback=blue!5!white,colframe=blue!75!black,title=Basic Assumptions of the T0 Model]
		\begin{align}
			&\text{1. Time $T_0$ is absolute and universally constant.} \\
			&\text{2. Mass varies according to $m = \gamma m_0$, where $\gamma = \frac{1}{\sqrt{1-v^2/c_0^2}}$.} \\
			&\text{3. Total energy is expressed as $E = \frac{\hbar}{T_0}$.} \\
			&\text{4. Redshift arises due to energy loss: $E_2 = E_1(1+z)^{-1}$.}
		\end{align}
	\end{tcolorbox}
	
	These basic assumptions lead to a complementary formulation of physics that provides mathematically equivalent predictions but conceptually deviates from standard physics.
	
	\subsection{Intrinsic Time and Time-Mass Duality}
	
	An important concept of the extended T0 model is intrinsic time:
	
	\begin{itemize}
		\item The intrinsic time of a particle is defined as $T = \frac{\hbar}{mc^2}$. It is inversely proportional to the particle's mass.
		\item This intrinsic time leads to a duality in the description of physical phenomena:
		\begin{itemize}
			\item \textbf{Standard Model}: Time is relative (time dilation), mass is constant
			\item \textbf{Complementary T0 Model}: Time is absolute, mass varies
		\end{itemize}
	\end{itemize}
	
	The time-mass duality enables an alternative interpretation of many phenomena traditionally explained by time dilation.
	
	\subsection{Unified Lagrangian Density}
	
	The Lagrangian density for the unified T0 model is:
	
	\begin{equation}
		\mathcal{L}_\text{total} = \mathcal{L}_\text{Gravitation} + \mathcal{L}_\text{SM} + \mathcal{L}_\text{Higgs} + \mathcal{L}_\text{intrinsic}
	\end{equation}
	
	where:
	\begin{itemize}
		\item $\mathcal{L}_\text{Gravitation}$ describes the Lagrangian density of gravitation,
		\item $\mathcal{L}_\text{SM}$ represents the Lagrangian density of the Standard Model (strong, electromagnetic, and weak forces),
		\item $\mathcal{L}_\text{Higgs}$ is the Lagrangian density of the Higgs field,
		\item $\mathcal{L}_\text{intrinsic}$ is the new Lagrangian density that accounts for intrinsic time.
	\end{itemize}
	
	Gravitation can be expressed in two complementary forms:
	
	\begin{equation}
		\mathcal{L}_\text{Gravitation} = -\frac{1}{16\pi G} \sqrt{-g} R
	\end{equation}
	
	in the Standard Model (with time dilation), and:
	
	\begin{equation}
		\mathcal{L}_\text{Gravitation-T} = -\frac{1}{16\pi G_T} \sqrt{-g_T} R_T
	\end{equation}
	
	in the complementary model (with absolute time and mass variation), where $G_T = G \cdot \frac{T_0}{T}$ is a modified Newton constant that depends on the intrinsic time $T = \frac{\hbar}{mc^2}$, and $T_0$ is a reference timescale (e.g., the Planck time).
	
	This unified Lagrangian density forms the mathematical foundation for applying the T0 model to cosmological and astrophysical phenomena.
	
	% Add missing command definition
	\newcommand{\Tfield}{T(x)} % Intrinsic time as a field
	
	\subsection{The Role of Gravitation in the T0 Model}
	
	In this document, gravitation still appears as a separate term in the Lagrangian density:
	
	\begin{equation}
		\mathcal{L}_\text{total} = \mathcal{L}_\text{Gravitation} + \mathcal{L}_\text{SM} + \mathcal{L}_\text{Higgs} + \mathcal{L}_\text{intrinsic}
	\end{equation}
	
	However, this representation does not reflect the latest development of the T0 model. As elaborated in the reference document "Essential Mathematical Formalisms of the Time-Mass Duality Theory with Lagrangian Densities," gravitation can indeed be understood as an emergent effect from the dynamics of the intrinsic time field, without requiring a separate term:
	
	\begin{theorem}[Gravitational Emergence]
		In the T0 model, gravitational effects arise from the spatial and temporal gradients of the intrinsic time field $\Tfield$, establishing a natural connection between quantum physics and gravitational phenomena through:
		\begin{equation}
			\nabla \Tfield = \nabla \left(\frac{\hbar}{mc^2}\right) = -\frac{\hbar}{m^2c^2}\nabla m \sim \nabla \Phi_g
		\end{equation}
		where $\Phi_g$ is the gravitational potential.
	\end{theorem}
	
	This emergence of gravitation is one of the conceptually most elegant aspects of the T0 model. It implies that the complete unified Lagrangian density should more precisely be formulated as:
	
	\begin{equation}
		\mathcal{L}_\text{total} = \mathcal{L}_\text{SM} + \mathcal{L}_\text{Higgs} + \mathcal{L}_\text{intrinsic}
	\end{equation}
	
	In this formulation, gravitation is no longer added as a separate force but arises naturally from the dynamics of the intrinsic time field. This aligns with the core idea of the T0 model—to reinterpret physical phenomena through complementary perspectives—and significantly simplifies the theoretical framework.
	
	\section{Dark Energy in the T0 Model}
	
	\subsection{Reinterpretation of Dark Energy}
	
	In the T0 model, dark energy is fundamentally reinterpreted differently from the standard cosmological model ($\Lambda$CDM):
	
	\begin{itemize}
		\item \textbf{Standard Model ($\Lambda$CDM)}: Dark energy is a cosmological constant with negative pressure that drives the accelerated expansion of the universe.
		\item \textbf{T0 Model}: Dark energy is a dynamic medium for energy exchange in a static universe.
	\end{itemize}
	
	Dark energy is modeled as a scalar field $\phi_{DE}$ that interacts with matter and radiation. The energy density of this field exhibits a spatial structure:
	
	\begin{equation}
		\rho_{DE}(r) = \frac{\kappa}{r^2}
	\end{equation}
	
	where $\kappa$ is a constant and $r$ denotes the radial distance. This $1/r^2$ profile differs from the constant energy density $\rho_\Lambda$ of the cosmological constant in the standard model.
	
	\subsection{Field-Theoretical Description}
	
	The complete Lagrangian density for the dark energy field is:
	
	\begin{equation}
		\mathcal{L}_{DE} = -\frac{1}{2}\partial_\mu \phi_{DE} \partial^\mu \phi_{DE} - V(\phi_{DE}) - \frac{\beta}{M_{Pl}} \phi_{DE} T^{\mu}_{\mu} - \frac{1}{2}\xi \phi_{DE}^2 R
	\end{equation}
	
	where:
	\begin{itemize}
		\item $\partial_\mu \phi_{DE} \partial^\mu \phi_{DE}$ is the kinetic term,
		\item $V(\phi_{DE})$ is the self-interaction potential of the field,
		\item $\frac{\beta}{M_{Pl}} \phi_{DE} T^{\mu}_{\mu}$ represents the coupling to matter and radiation,
		\item $\frac{1}{2}\xi \phi_{DE}^2 R$ is a non-minimal coupling to spacetime curvature $R$.
	\end{itemize}
	
	The field equation for the dark energy field is:
	
	\begin{equation}
		\Box\phi_{DE} - \frac{dV}{d\phi_{DE}} - \frac{\beta}{M_{Pl}}T^{\mu}_{\mu} - \xi \phi_{DE} R = 0
	\end{equation}
	
	For a massless field ($V(\phi_{DE}) = 0$) and negligible curvature ($\xi R \approx 0$), the field equation simplifies to:
	
	\begin{equation}
		\frac{1}{r^2}\frac{d}{dr}\left(r^2\frac{d\phi_{DE}}{dr}\right) = \frac{\beta}{M_{Pl}}T^{\mu}_{\mu}
	\end{equation}
	
	\subsection{Energy Exchange and Redshift}
	
	A central aspect of the T0 model is the interpretation of cosmic redshift as a result of energy loss of photons to dark energy. The energy change of a photon is described by:
	
	\begin{equation}
		\frac{dE_{\gamma}}{dx} = -\alpha E_{\gamma}
	\end{equation}
	
	with the solution:
	
	\begin{equation}
		E_{\gamma}(x) = E_{\gamma,0} e^{-\alpha x}
	\end{equation}
	
	The redshift $z$ is defined as:
	
	\begin{equation}
		1 + z = \frac{E_0}{E} = \frac{\lambda_{\text{observed}}}{\lambda_{\text{emitted}}} = e^{\alpha d}
	\end{equation}
	
	To ensure consistency with the observed Hubble relation $z \approx H_0 d/c$, it must hold:
	
	\begin{equation}
		\alpha = \frac{H_0}{c} \approx 2.3 \times 10^{-28} \text{ m}^{-1}
	\end{equation}
	
	This extremely small absorption rate explains why the energy loss of photons to dark energy is not measurable in laboratory experiments but becomes significant over cosmological distances.
	
	\section{Galaxy Dynamics in the T0 Model}
	
	\subsection{Flat Rotation Curves without Dark Matter}
	
	In the T0 model, the flat rotation curves of galaxies are explained not by dark matter but by an effective mass variation resulting from interaction with dark energy.
	
	The effective mass of a particle in the T0 model can be regarded as a dynamic quantity that interacts with the dark energy field:
	
	\begin{equation}
		m_{\text{eff}}(r) = m_0 \cdot f(\phi_{DE}(r))
	\end{equation}
	
	This coupling can be modeled by a Yukawa-like term in the Lagrangian density:
	
	\begin{equation}
		\mathcal{L}_{\text{int}} = -g \phi_{DE} \bar{\psi}\psi
	\end{equation}
	
	In Newtonian mechanics, the rotational velocity $v(r)$ of an object in a circular orbit around a mass $M$ is given by:
	
	\begin{equation}
		v^2(r) = \frac{GM(r)}{r}
	\end{equation}
	
	In the T0 model, the rotational velocity is described by the modified equation:
	
	\begin{equation}
		\frac{G \cdot m_{\text{eff}}(r) \cdot M(r)}{r^2} = \frac{v^2(r)}{r} \cdot m_{\text{eff}}(r)
	\end{equation}
	
	where $m_{\text{eff}}(r)$ is the effective mass of a test particle (e.g., a star) at position $r$.
	
	\subsection{Effective Gravitational Constant}
	
	An alternative approach is to introduce an effective gravitational constant that depends on the dark energy field:
	
	\begin{equation}
		G_{\text{eff}}(r) = G\left(1 + \beta\phi_{DE}(r)\right) = G\left(1 - \beta\frac{g\rho_0 r_0^2}{r}\right)
	\end{equation}
	
	The rotational velocity then becomes:
	
	\begin{equation}
		v^2(r) = \frac{G_{\text{eff}}(r)M_{\text{baryon}}(r)}{r}
	\end{equation}
	
	With a dark energy field whose density is proportional to $1/r^2$ for large $r$:
	
	\begin{equation}
		\rho_{DE}(r) = \frac{\kappa}{r^2}
	\end{equation}
	
	the rotational velocity becomes:
	
	\begin{equation}
		v^2(r) \approx \frac{GM_{\text{baryon}}}{r} + \frac{\kappa}{\rho_0}
	\end{equation}
	
	For large $r$, the second term dominates, and we obtain:
	
	\begin{equation}
		v^2(r) \approx \frac{\kappa}{\rho_0} = \text{constant}
	\end{equation}
	
	This exactly matches the observed behavior of flat rotation curves.
	
	\subsection{Parameter Values from Observations}
	
	For a typical spiral galaxy like the Milky Way with a rotational velocity of about $v \approx 220$ km/s, we get:
	
	\begin{equation}
		\kappa = v^2 \rho_0 \approx (220 \text{ km/s})^2 \cdot \rho_0 \approx 4.8 \times 10^{-7} \text{ GeV/cm} \cdot \text{s}^{-2}
	\end{equation}
	
	The dimensionless coupling constant is approximately:
	
	\begin{equation}
		\hat{\beta} \approx 10^{-3}
	\end{equation}
	
	These values are consistent with the parameters of the dark energy field derived from cosmological observations.
	
	\section{Unified Mathematical Formulation}
	
	\subsection{Common Field Equations}
	
	The complete unified theory can be described by the following action:
	
	\begin{equation}
		S_\text{unified} = \int \left( \mathcal{L}_\text{standard} + \mathcal{L}_\text{complementary} + \mathcal{L}_\text{coupling} \right) d^4x
	\end{equation}
	
	where:
	\begin{align}
		\mathcal{L}_\text{standard} &= -\frac{1}{16\pi G} \sqrt{-g} R + \mathcal{L}_\text{SM} + (D_\mu \phi)^\dagger (D^\mu \phi) - V(\phi) \\
		\mathcal{L}_\text{complementary} &= -\frac{1}{16\pi G_T} \sqrt{-g_T} R_T + \mathcal{L}_\text{SM-T} + (D_{T\mu} \phi_T)^\dagger (D_T^\mu \phi_T) - V_T(\phi_T) \\
		\mathcal{L}_\text{coupling} &= \int \mathcal{D}[\Psi] \, \Psi^* \left( i\hbar \frac{\partial}{\partial t} - i\hbar \frac{\partial}{\partial (t/T)} \right) \Psi
	\end{align}
	
	This unified formulation connects the foundations of the T0 model with its applications to dark energy and galaxy dynamics.
	
	On cosmological scales, the Friedmann equations in the T0 model can be reinterpreted. In the standard model, they describe the expansion of the universe:
	
	\begin{align}
		\left(\frac{\dot{a}}{a}\right)^2 &= \frac{8\pi G}{3}\rho \\
		\frac{\ddot{a}}{a} &= -\frac{4\pi G}{3}(\rho + 3p)
	\end{align}
	
	In the T0 model, they instead describe an effective mass variation in a static universe:
	
	\begin{align}
		\left(\frac{\dot{m}}{m}\right)^2 &= \frac{8\pi G}{3}\rho_{\text{eff}} \\
		\frac{\ddot{m}}{m} &= -\frac{4\pi G}{3}(\rho_{\text{eff}} + 3p_{\text{eff}})
	\end{align}
	
	\subsection{Consistent Parameterization}
	
	To ensure consistency across the various applications of the T0 model, the different parameters can be related as follows:
	
	\begin{itemize}
		\item The absorption coefficient $\alpha = \frac{H_{0}}{c} \approx 2.3 \times 10^{-28}$ m$^{-1}$ determines the rate of energy loss of photons.
		\item The parameter $\kappa \approx 4.8 \times 10^{-7}$ GeV/cm$\cdot$s$^{-2}$ determines the strength of the dark energy field for galaxy dynamics.
		\item The dimensionless coupling constant $\beta \approx 10^{-3}$ characterizes the interaction between the dark energy field and baryonic matter.
	\end{itemize}
	
	These parameters are related through the following relationship:
	
	\begin{equation}
		\kappa = \frac{\beta^2 H_0^2 M_{\text{Pl}}^2}{c^2 \rho_0}
	\end{equation}
	
	where $M_{\text{Pl}}$ is the Planck mass and $\rho_0$ is a reference density.
	
	For photons, the intrinsic time can be defined as:
	
	\begin{equation}
		T = \frac{\hbar}{E_{\gamma}} e^{\alpha x}
	\end{equation}
	
	where $\alpha = \frac{H_0}{c} \approx 2.3 \times 10^{-28}$ m$^{-1}$ accounts for the energy loss over distance $x$, consistent with the T0 model.
	
	\section{Experimental Tests of the T0 Model}
	
	\subsection{Common Predictions}
	
	The T0 model leads to several experimentally verifiable predictions that could distinguish it from the standard model:
	
	\begin{enumerate}
		\item \textbf{Mass-dependent time evolution in quantum systems}, measurable as different coherence times.
		\item \textbf{Differences in entanglement speed} for particles of different masses.
		\item \textbf{Scale-dependent gravitational constant}, correlated with intrinsic time.
		\item \textbf{Modified energy-momentum relationship} for very massive particles.
		\item \textbf{Measurable deviations in high-precision experiments}, typically explained by time dilation.
	\end{enumerate}
	
	This leads to the prediction that Bell tests with particles of different masses or photons of different frequencies could show measurable delays in correlations, proportional to the mass ratio $\frac{m_1}{m_2}$ or energy ratio $\frac{E_1}{E_2}$.
	
	\subsection{Tests in a Cosmological Context}
	
	Specific tests of the T0 model in a cosmological context include:
	
	\begin{enumerate}
		\item \textbf{Temporal variation of the fine-structure constant}:
		\begin{equation}
			\frac{d\alpha_{\text{fs}}}{dt} \approx \alpha_{\text{fs}} \cdot \alpha \cdot c \approx 10^{-18} \text{ year}^{-1}
		\end{equation}
		
		\item \textbf{Environment-dependent redshift}:
		\begin{equation}
			\frac{z_{\text{cluster}}}{z_{\text{void}}} \approx 1 + \delta\frac{\rho_{\text{cluster}} - \rho_{\text{void}}}{\rho_0}
		\end{equation}
		
		\item \textbf{Anomalous light propagation in strong gravitational fields} with an effective refractive index:
		\begin{equation}
			n_{\text{eff}}(r) = 1 + \epsilon \frac{\phi_{DE}(r)}{M_{\text{Pl}}}
		\end{equation}
		
		\item \textbf{Differential redshift}:
		\begin{equation}
			\frac{z(\lambda_1)}{z(\lambda_2)} \approx 1 + \eta\frac{\lambda_1 - \lambda_2}{\lambda_0}
		\end{equation}
	\end{enumerate}
	
	\subsection{Tests for Galaxy Dynamics}
	
	Specific tests of the T0 model in the field of galaxy dynamics include:
	
	\begin{enumerate}
		\item \textbf{Modification of the Tully-Fisher relation}:
		\begin{equation}
			L \propto v_{\text{max}}^{4+\epsilon}
		\end{equation}
		where $\epsilon$ is a small correction term dependent on the coupling constant $\beta$:
		\begin{equation}
			\epsilon \approx \frac{\beta^2 \rho_0 r_0^2}{m_0 G}
		\end{equation}
		
		\item \textbf{Mass-dependent gravitational lensing effects}:
		\begin{equation}
			\alpha_{\text{lens}} \propto \int \nabla(\Phi_{\text{Newton}} + \beta\phi_{DE}) dz
		\end{equation}
		
		\item \textbf{Differences between gas-rich and star-rich galaxies}:
		\begin{equation}
			\frac{v^2_{\text{gas-rich}}(r)}{v^2_{\text{gas-poor}}(r)} = 1 + \delta(r)
		\end{equation}
		
		\item \textbf{Dwarf galaxy dynamics} with systematically lower velocity dispersion:
		\begin{equation}
			\sigma_{v,T_0} \approx \sigma_{v,\Lambda CDM} \times \left(1 - \gamma \frac{M_{\text{gas}}}{M_{\text{star}}}\right)
		\end{equation}
	\end{enumerate}
	
	\section{Comparison with the $\Lambda$CDM Standard Model}
	
	\begin{tcolorbox}[colback=yellow!5!white,colframe=yellow!75!black,title=Comparison of Models]
		\begin{tabular}{p{0.45\textwidth}|p{0.45\textwidth}}
			\toprule
			\textbf{$\Lambda$CDM Model} & \textbf{T0 Model} \\
			\midrule
			Dark matter as a separate particle species & No separate dark matter, but effective mass variation \\
			\midrule
			NFW density profile: $\rho_{\text{DM}}(r) = \frac{\rho_0}{\frac{r}{r_s}(1 + \frac{r}{r_s})^2}$ & Effective density profile: $\rho_{\text{eff}}(r) \approx \rho_{\text{baryon}}(r) + \frac{\kappa}{r^2}$ \\
			\midrule
			Time is relative (time dilation), rest mass constant & Time is absolute, mass varies with energy \\
			\midrule
			Dark energy as a driver of cosmic expansion & Dark energy as a medium for energy exchange \\
			\midrule
			Redshift due to expansion & Redshift due to energy loss \\
			\midrule
			Expanding universe & Static universe \\
			\bottomrule
		\end{tabular}
	\end{tcolorbox}
	
	In the $\Lambda$CDM model, the constant energy density of dark energy leads to an accelerated expansion that becomes ever faster. The future of the universe is a "Big Rip" or eternal expansion, depending on the exact equation of state of dark energy.
	
	In the T0 model, there is no true expansion of the universe, but a continuous conversion of matter and radiation energy into dark energy. The energy densities evolve according to:
	
	\begin{align}
		\rho_{\text{matter}}(t) &= \rho_{\text{matter},0} e^{-\alpha c t} \\
		\rho_{\gamma}(t) &= \rho_{\gamma,0} e^{-\alpha c t} \\
		\rho_{\text{DE}}(t) &= \rho_{\text{DE},0} + (\rho_{\text{matter},0} + \rho_{\gamma,0})(1 - e^{-\alpha c t})
	\end{align}
	
	In the long term, the universe approaches a state where all energy exists in the form of dark energy—a "thermal death," but without spatial expansion.
	
	\section{Summary}
	
	The T0 model provides a comprehensive alternative theoretical framework to standard physics, based on the assumption of absolute time and variable mass. This reinterpretation leads to:
	
	\begin{enumerate}
		\item An alternative explanation for cosmic redshift through energy loss rather than expansion
		\item A redefinition of dark energy as a dynamic medium for energy exchange
		\item An explanation for flat rotation curves in galaxies without dark matter
	\end{enumerate}
	
	Mathematical consistency between the fundamental principles of the T0 model, its application to dark energy, and the explanation of galaxy dynamics is ensured by a unified field theory. This theory encompasses both the standard and complementary formulations of physics and offers specific experimental predictions that could distinguish between the T0 model and the standard model.
	
	The T0 model provides a conceptually elegant alternative to the standard cosmological model by reinterpreting fundamental assumptions about time and mass. The proposed tests, particularly the analysis of galaxies with different gas-to-star ratios and detailed measurement of gravitational lensing profiles, offer promising opportunities to differentiate between the models.
	
	Regardless of the outcome of these tests, the mathematical formulation of the T0 model contributes to a deeper understanding of the fundamental concepts of time, mass, and energy in modern physics and opens new perspectives for interpreting cosmic phenomena.
	
	\section{References}
	
	\begin{thebibliography}{99}
		
		\bibitem{pascher1} Pascher, J. (2025). Complementary Extensions of Physics: Absolute Time and Intrinsic Time.
		
		\bibitem{pascher2} Pascher, J. (2025). A Model with Absolute Time and Variable Energy: A Comprehensive Investigation of the Foundations.
		
		\bibitem{pascher3} Pascher, J. (2025). Extensions of Quantum Mechanics through Intrinsic Time.
		
		\bibitem{pascher4} Pascher, J. (2025). Integration of Time-Mass Duality into Quantum Field Theory.
		
		\bibitem{pascher5} Pascher, J. (2025). Dynamic Mass of Photons and Their Implications for Nonlocality.
		
		\bibitem{pascher6} Pascher, J. (2025). Fundamental Constants and Their Derivation from Natural Units.
		
		\bibitem{pascher7} Pascher, J. (2025). Real Consequences of Reformulating Time and Mass in Physics: Beyond the Planck Scale.
		
		\bibitem{rotation} Rubin, V. C., Ford, W. K. (1970). Rotation of the Andromeda Nebula from a Spectroscopic Survey of Emission Regions. The Astrophysical Journal, 159, 379.
		
		\bibitem{nfw} Navarro, J. F., Frenk, C. S., White, S. D. M. (1996). The Structure of Cold Dark Matter Halos. The Astrophysical Journal, 462, 563.
		
		\bibitem{tully} Tully, R. B., Fisher, J. R. (1977). A new method of determining distances to galaxies. Astronomy and Astrophysics, 54, 661.
		
		\bibitem{bullet} Clowe, D., Bradač, M., Gonzalez, A. H., et al. (2006). A Direct Empirical Proof of the Existence of Dark Matter. The Astrophysical Journal, 648, L109.
		
		\bibitem{supernova} Perlmutter, S., et al. (1999). Measurements of $\Omega$ and $\Lambda$ from 42 High-Redshift Supernovae. The Astrophysical Journal, 517, 565.
		
		\bibitem{riess} Riess, A. G., et al. (1998). Observational Evidence from Supernovae for an Accelerating Universe and a Cosmological Constant. The Astronomical Journal, 116, 1009.
		
		\bibitem{planck} Planck Collaboration. (2020). Planck 2018 results. VI. Cosmological parameters. Astronomy \& Astrophysics, 641, A6.
		
		\bibitem{cmb} Bennett, C. L., et al. (2013). Nine-year Wilkinson Microwave Anisotropy Probe (WMAP) Observations: Final Maps and Results. The Astrophysical Journal Supplement Series, 208, 20.
		
		\bibitem{bao} Eisenstein, D. J., et al. (2005). Detection of the Baryon Acoustic Peak in the Large-Scale Correlation Function of SDSS Luminous Red Galaxies. The Astrophysical Journal, 633, 560.
		
		\bibitem{quintessence} Caldwell, R. R., Dave, R., Steinhardt, P. J. (1998). Cosmological Imprint of an Energy Component with General Equation of State. Physical Review Letters, 80, 1582.
		
		\bibitem{euclid} Laureijs, R., et al. (2011). Euclid Definition Study Report. ESA/SRE(2011)12.
		
		\bibitem{tired} Zwicky, F. (1929). On the Red Shift of Spectral Lines through Interstellar Space. Proceedings of the National Academy of Sciences, 15, 773.
		
		\bibitem{alfa} Webb, J. K., et al. (2011). Indications of a Spatial Variation of the Fine Structure Constant. Physical Review Letters, 107, 191101.
		
		\bibitem{vacuum} Weinberg, S. (1989). The Cosmological Constant Problem. Reviews of Modern Physics, 61, 1.
		
		\bibitem{scalar} Fujii, Y., Maeda, K. (2003). The Scalar-Tensor Theory of Gravitation. Cambridge University Press.
		
		\bibitem{lambda} Carroll, S. M. (2001). The Cosmological Constant. Living Reviews in Relativity, 4, 1.
	\end{thebibliography}
	
\end{document}