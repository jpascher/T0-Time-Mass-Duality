\documentclass[a4paper,12pt]{article}
\usepackage[utf8]{inputenc}
\usepackage[english]{babel} % For English language and hyphenation
\usepackage{amsmath, amssymb}
\usepackage{physics}
\usepackage{hyperref}
\usepackage{geometry}
\geometry{a4paper, margin=2.5cm}

\begin{document}
	
	\title{Abstract - Complementary Dualism in Physics: From Wave-Particle to Time-Mass Concepts}
	\author{Johann Pascher}
	\date{March 25, 2025}
	\maketitle
	
	\section{Introduction: Dualism in Modern Physics}
	
	Modern physics is based on dualistic concepts. The wave-particle duality is one of the fundamental principles describing how objects like electrons or photons can exhibit both wave-like and particle-like properties. These seemingly contradictory descriptions are nevertheless both correct and complement each other.
	
	However, the two main pillars of modern physics - quantum mechanics (QM) and quantum field theory (QFT) - themselves represent a kind of dualism. While QM emphasizes the discrete, particle-like nature of matter, QFT focuses on field concepts and continuous aspects. Both theories are incomplete:
	
	\begin{itemize}
		\item \textbf{Quantum mechanics} describes quantum phenomena but cannot fully incorporate relativistic effects
		\item \textbf{Quantum field theory} combines quantum effects with special relativity but reaches its limits when dealing with gravitational theory
	\end{itemize}
	
	Building upon this already established dualism in physics, I present in my work "Complementary Extensions of Physics: Absolute Time and Intrinsic Time" a new, analogous dualism: the time-mass dualism. This could help bridge some of the existing gaps between established theories.
	
	\section{From Particles and Waves to Time and Mass}
	
	\subsection{The Classical Wave-Particle Duality}
	
	In quantum mechanics, we have two complementary descriptions of the same phenomenon:
	
	\begin{itemize}
		\item The \textbf{particle description} focuses on localized objects with defined position and mass
		\item The \textbf{wave description} considers the phenomenon as a spatially extended wave function
	\end{itemize}
	
	Mathematically, these descriptions are connected through the Fourier transform:
	\begin{align}
		\Psi(\vec{x}) &= \frac{1}{(2\pi\hbar)^{3/2}} \int \phi(\vec{p}) e^{i\vec{p}\cdot\vec{x}/\hbar} d^3p \\
		\phi(\vec{p}) &= \frac{1}{(2\pi\hbar)^{3/2}} \int \Psi(\vec{x}) e^{-i\vec{p}\cdot\vec{x}/\hbar} d^3x
	\end{align}
	
	\subsection{The New Time-Mass Dualism}
	
	Analogously, I propose that we can consider two complementary descriptions for relativistic phenomena:
	
	\begin{itemize}
		\item The \textbf{time dilation description} (standard model): Time is variable ($t' = \gamma t$), while rest mass remains constant
		\item The \textbf{mass variation description} (complementary model): Time is absolute ($T_0 = \text{const.}$), while mass is variable ($m = \gamma m_0$)
	\end{itemize}
	
	Mathematically, these descriptions are also connected by a transformation I call the modified Lorentz transformation.
	
	\section{The Concept of Intrinsic Time}
	
	From the complementary model emerges a remarkable concept: intrinsic time. This is defined as:
	\begin{equation}
		T = \frac{\hbar}{mc^2}
	\end{equation}
	
	Intrinsic time is a fundamental property of every object, dependent on its mass. It leads to a modified Schrödinger equation:
	\begin{equation}
		i\hbar \frac{\partial}{\partial (t/T)} \Psi = \hat{H} \Psi
	\end{equation}
	
	This means that heavier objects experience faster internal time evolution than lighter objects - a kind of "proper time" in the quantum mechanical sense.
	
	\section{Parallels Between the Dualisms}
	
	The parallels between wave-particle duality and time-mass dualism are profound:
	
	\begin{enumerate}
		\item \textbf{Complementarity:} Just as position and momentum are complementary observables, time and energy/mass are complementary quantities
		
		\item \textbf{Uncertainty relations:} Corresponding to $\Delta x \Delta p \geq \frac{\hbar}{2}$ of wave-particle duality is $\Delta t \Delta E \geq \frac{\hbar}{2}$ or $\Delta T \Delta m \geq \frac{\hbar}{2c^2}$ in time-mass dualism
		
		\item \textbf{Transformations:} Both dualisms are connected through mathematical transformations
	\end{enumerate}
	
	\section{Necessary Extensions of QM and QFT}
	
	Based on the time-mass dualism, I propose concrete extensions for existing theories:
	
	\subsection{Extension of Quantum Mechanics}
	
	The classical Schrödinger equation must be extended to account for intrinsic time:
	
	\begin{equation}
		i\hbar \frac{\partial}{\partial (t/T)} \Psi = \hat{H} \Psi
	\end{equation}
	
	This modification leads to:
	\begin{itemize}
		\item A mass-dependent time evolution of quantum systems
		\item A natural explanation for different decay rates and coherence times
		\item A new perspective on the measurement problem through the connection between mass and time evolution
	\end{itemize}
	
	\subsection{Extension of Quantum Field Theory}
	
	QFT must be extended to incorporate absolute time or mass-dependent intrinsic time:
	
	\begin{itemize}
		\item Field operators must be reformulated with respect to intrinsic time $T = \frac{\hbar}{mc^2}$
		\item Renormalization can be reinterpreted through mass-dependent time scales
		\item Virtual particles could be understood as manifestations of different intrinsic time scales
	\end{itemize}
	
	These extensions could be particularly fruitful for:
	\begin{itemize}
		\item The integration of gravitation into quantum field theory
		\item The resolution of infinities in quantum field theories
		\item A deeper understanding of vacuum energy and the cosmological constant
	\end{itemize}
	
	\section{The Reality of Time Dilation versus Mass Variation}
	
	A central objection against the concept of absolute time is that we can directly measure time dilation - for example in GPS corrections or muon decay. However, in my work I show that all these measurements are fundamentally based on frequency measurements:
	\begin{equation}
		f = \frac{E}{h} = \frac{mc^2}{h}
	\end{equation}
	
	These measurements can therefore be equally interpreted as time dilation or as mass variation. The experimental data are identical - only our interpretation changes.
	
	\section{Implications for Instantaneity and Nonlocality}
	
	Nonlocality in quantum physics, especially with entangled particles, is often understood as instantaneous action at arbitrary distances. My models offer an alternative interpretation of this apparent instantaneity:
	
	\begin{itemize}
		\item In the $T_0$ model with absolute time, quantum correlations could be explained through mass variation rather than temporal effects. Since time remains absolute, correlations would not be instantaneous but would arise through dynamic mass adjustment ($m = \gamma m_0$).
		
		\item In the intrinsic time model, entangled particles of different mass would experience different time evolutions. A lighter particle with larger $T$ would respond more slowly to state changes than a heavier particle with smaller $T$.
		
		\item For photons, intrinsic time could be defined as $T = \frac{1}{E} = \frac{1}{p}$, corresponding to wavelength. A more energetic (shorter wavelength) photon would thus experience faster time evolution than a less energetic one.
	\end{itemize}
	
	This perspective replaces the counterintuitive instantaneous action with a systematic, mass-dependent dynamics that could be empirically verified. Specifically, Bell tests with particles of different mass or photons of different frequency might reveal measurable delays in correlations, proportional to the mass ratio $\frac{m_1}{m_2}$ or energy ratio $\frac{E_1}{E_2}$.
	
	\section{Consequences and Outlook}
	
	The proposed time-mass dualism offers new perspectives that could go beyond the incompleteness of existing theories:
	
	\begin{itemize}
		\item An alternative conceptual framework for problems of quantum gravity that could address the incompleteness of QFT regarding gravitation
		\item New interpretation of nonlocality through mass-dependent time evolution that might resolve the apparent contradiction between quantum entanglement and relativity theory
		\item A natural connection between discrete quantum phenomena (QM) and continuous fields (QFT) through the concept of intrinsic time
		\item Possible experimental tests that could distinguish between the models
	\end{itemize}
	
	Just as wave-particle duality revolutionized quantum mechanics, time-mass dualism could provide new insights for a more complete theory. While QM and QFT each represent parts of the puzzle, time-mass dualism might offer a unifying framework that could bridge the existing gaps between these theories.
	
	The complete mathematical derivations and detailed implications are thoroughly presented in my works:
	\begin{itemize}
		\item "Complementary Extensions of Physics: Absolute Time and Intrinsic Time" (03/24/2025)
		\item "A Model with Absolute Time and Variable Energy: A Detailed Examination of the Foundations" (03/24/2025)
		\item "Dynamic Mass of Photons and Their Implications for Nonlocality" (03/25/2025)
		\item "Fundamental Constants and Their Derivation from Natural Units" (03/25/2025)
		\item "Real Consequences of the Reformulation of Time and Mass in Physics: Beyond the Planck Scale" (03/24/2025)
	\end{itemize}
	
\end{document}