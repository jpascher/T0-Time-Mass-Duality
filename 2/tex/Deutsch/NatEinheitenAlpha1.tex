\documentclass[12pt,a4paper]{article}
\usepackage[utf8]{inputenc}
\usepackage[T1]{fontenc}
\usepackage[ngerman]{babel}
\usepackage{lmodern}
\usepackage{amsmath}
\usepackage{amssymb}
\usepackage{physics}
\usepackage{hyperref}
\usepackage{tcolorbox}
\usepackage{booktabs}
\usepackage{enumitem}
\usepackage[table,xcdraw]{xcolor}
\usepackage[left=2cm,right=2cm,top=2cm,bottom=2cm]{geometry}
\usepackage{pgfplots}
\pgfplotsset{compat=1.18}
\usepackage{graphicx}
\usepackage{float}
\usepackage{fancyhdr}
\usepackage{siunitx}
\usepackage{mathtools}
\usepackage{amsthm}
\usepackage{cleveref}
\usepackage{tocloft}

% Danksagungs-Umgebung
\newenvironment{acknowledgments}
{\section*{Danksagungen}}
{\vspace{1em}}

% Benutzerdefinierte Befehle
\newcommand{\Tfield}{T(x)}
\newcommand{\alphaEM}{\alpha_{\text{EM}}}
\newcommand{\alphaW}{\alpha_{\text{W}}}
\newcommand{\betaT}{\beta_{\text{T}}}
\newcommand{\Mpl}{M_{\text{Pl}}}
\newcommand{\Tzerot}{T_0(\Tfield)}
\newcommand{\Tzero}{T_0}
\newcommand{\vecx}{\vec{x}}
\newcommand{\vr}{\vec{r}}
\newcommand{\gammaf}{\gamma_{\text{Lorentz}}}
\newcommand{\DhiggsT}{\Tfield (\partial_\mu + ig A_\mu) \Phi + \Phi \partial_\mu \Tfield}
\newcommand{\LCDM}{\Lambda\text{CDM}}
\newcommand{\DTmu}{D_{T,\mu}}
\newcommand{\calL}{\mathcal{L}}
\newcommand{\deq}{\displaystyle}
\newcommand{\e}{\mathrm{e}}

% Kopf- und Fußzeilen-Konfiguration
\pagestyle{fancy}
\fancyhf{}
\fancyhead[L]{Johann Pascher}
\fancyhead[R]{Zeit-Masse-Dualität}
\fancyfoot[C]{\thepage}
\renewcommand{\headrulewidth}{0.4pt}
\renewcommand{\footrulewidth}{0.4pt}

% Inhaltsverzeichnis-Formatierung
\renewcommand{\cftsecfont}{\color{blue}}
\renewcommand{\cftsubsecfont}{\color{blue}}
\renewcommand{\cftsecpagefont}{\color{blue}}
\renewcommand{\cftsubsecpagefont}{\color{blue}}
\setlength{\cftsecindent}{1cm}
\setlength{\cftsubsecindent}{2cm}

\hypersetup{
	colorlinks=true,
	linkcolor=blue,
	citecolor=blue,
	urlcolor=blue,
	pdftitle={Energie als fundamentale Einheit: Natürliche Einheiten mit alphaEM = 1 im T0-Modell},
	pdfauthor={Johann Pascher},
	pdfsubject={Theoretische Physik},
	pdfkeywords={T0-Modell, natürliche Einheiten, Feinstrukturkonstante, vereinheitlichtes Einheitensystem, Zeit-Masse-Dualität}
}

% Theorem-Umgebungen
\newtheorem{theorem}{Theorem}[section]
\newtheorem{proposition}[theorem]{Proposition}

\begin{document}
	
	\title{Energie als fundamentale Einheit: \\ Natürliche Einheiten mit \(\alphaEM = 1\) im T0-Modell}
	\author{Johann Pascher}
	\date{25. März 2025}
	
	\maketitle
	\tableofcontents
	\newpage
	
	\section{Einführung in das vereinheitlichte Einheitensystem}
	\label{sec:intro}
	
	\subsection{Von natürlichen Einheiten zu einem vollständig vereinheitlichten System}
	\label{subsec:natural_units}
	
	In der theoretischen Physik werden verschiedene Systeme natürlicher Einheiten verwendet, um die mathematische Formulierung physikalischer Gesetze zu vereinfachen. Die bekanntesten umfassen:
	
	\begin{itemize}
		\item \textbf{Natürliche Einheiten:} \(\hbar = c = 1\)
		\item \textbf{Planck-Einheiten:} \(\hbar = c = G = 1\)
		\item \textbf{Elektrodynamische natürliche Einheiten:} \(\hbar = c = \alphaEM = 1\)
		\item \textbf{Thermodynamische natürliche Einheiten:} \(\hbar = c = k_B = \alphaW = 1\)
	\end{itemize}
	
	Das T0-Modell führt ein vollständig vereinheitlichtes Einheitensystem ein, in dem zusätzlich:
	\begin{equation}
		\label{eq:unified_system}
		\betaT = \alphaEM = \alphaW = 1
	\end{equation}
	gesetzt wird. In diesem System werden alle physikalischen Größen auf die Dimension der Energie reduziert:
	
	\begin{tcolorbox}[colback=blue!5!white,colframe=blue!75!black,title=Dimensionen im vereinheitlichten Einheitensystem]
		\begin{itemize}
			\item Länge: \([L] = [E^{-1}]\)
			\item Zeit: \([T] = [E^{-1}]\)
			\item Masse: \([M] = [E]\)
			\item Temperatur: \([T_{\text{emp}}] = [E]\)
			\item Elektrische Ladung: \([Q] = [1]\) (dimensionslos)
			\item Intrinsische Zeit: \([\Tfield] = [E^{-1}]\)
		\end{itemize}
	\end{tcolorbox}
	
	Dieses vereinheitlichte System offenbart fundamentale Beziehungen zwischen scheinbar völlig verschiedenen physikalischen Phänomenen und ermöglicht eine elegantere mathematische Formulierung des T0-Modells, wie in \cite{pascher_lagrange_2025} und \cite{pascher_alphabeta_2025} gezeigt.
	
	\subsection{Konzept der Energie als fundamentale Einheit}
	\label{subsec:energy_concept}
	
	Diese Arbeit untersucht auch die Konsequenzen der Annahme, dass die Feinstrukturkonstante \(\alphaEM = 1\) in einem System natürlicher Einheiten (\(\hbar = c = 1\)) auf das T0-Modell der Zeit-Masse-Dualität angewendet wird. Hier wird Energie als die fundamentale Einheit identifiziert, auf die alle physikalischen Größen reduziert werden können. Die Analyse umfasst dimensionale Neuformulierungen, vereinfachte Grundgleichungen und kosmologische Implikationen im Kontext des T0-Modells, das absolute Zeit und variable Masse postuliert, wie in \cite{pascher_zeit_masse_2025} beschrieben.
	
	\section{Extrapolation der Physik über bekannte Grenzen hinaus}
	\label{sec:beyond_limits}
	
	\subsection{Physik jenseits der Lichtgeschwindigkeit}
	\label{subsec:beyond_lightspeed}
	
	Die Lichtgeschwindigkeit \(c\) gilt als absolute Grenze für Materie und Signalübertragung in der Standardphysik, eine direkte Konsequenz der Lorentz-Transformation und der Relativitätstheorie. Innerhalb dieses Rahmens wurden alle Fundamentalkonstanten und die Planck-Skala definiert. Jedoch könnte diese Grenze nur innerhalb unseres gegenwärtigen theoretischen Modells gültig sein. Im T0-Modell mit seiner fundamentalen Zeit-Masse-Dualität könnte eine alternative Interpretation möglich sein:
	
	\begin{itemize}
		\item \textbf{Neuinterpretation der Massenvariation:} Im T0-Modell wird die Masse \(m = \frac{\hbar}{\Tfield c^2}\) durch das intrinsische Zeitfeld bestimmt. Die relativistische Massenänderung\\ \(m = m_0/\sqrt{1-v^2/c^2}\) kann als Variation von \(\Tfield\) interpretiert werden, wie in \cite{pascher_zeit_2025} erklärt.
		\item \textbf{Modifizierte Transformationsgesetze:} Im vereinheitlichten Einheitensystem mit \(c = 1\) könnten erweiterte Transformationen Bereiche mit \(v > 1\) beschreiben, ohne Kausalitätsverletzungen, während die fundamentale Beziehung \(m = \frac{\hbar}{\Tfield c^2}\) erhalten bleibt.
		\item \textbf{Erweiterte Konstanten:} Mit \(\alphaEM = \betaT = \alphaW = 1\) entsteht ein konsistenter Rahmen, der möglicherweise über die Lichtgeschwindigkeit hinaus gültig bleibt.
	\end{itemize}
	
	Diese spekulativen Überlegungen werden in \cref{sec:speculative} weiter erforscht.
	
	\subsection{Konsequenzen für Kausalität und Information}
	\label{subsec:causality}
	
	Im T0-Modell mit seiner Zeit-Masse-Dualität könnte Kausalität neu interpretiert werden, wie in \cite{pascher_feldtheorie_2025} ausgearbeitet:
	\begin{itemize}
		\item \textbf{Zeitfeld-basierte Kausalität:} Kausale Beziehungen könnten durch die Geometrie des Zeitfeldes \(\Tfield\) bestimmt werden, nicht durch Lichtkegel-Strukturen.
		\item \textbf{Nicht-lokaler Informationstransfer:} Die scheinbare Nicht-Lokalität der Quantenmechanik könnte durch die intrinsische Zeitfeldstruktur erklärt werden, ohne überlichtschnelle Signalübertragung zu erfordern.
		\item \textbf{Massenabhängige Kausalstruktur:} Da \(m = \frac{\hbar}{\Tfield c^2}\), könnten kausale Beziehungen von der Masse abhängen, was möglicherweise eine natürliche Erklärung für Quantenkorrelationen bietet.
	\end{itemize}
	
	\section{Einführung in die Feinstrukturkonstante \(\alphaEM\)}
	\label{sec:alpha_em}
	
	Die Feinstrukturkonstante \(\alphaEM\) beschreibt die Stärke der elektromagnetischen Wechselwirkung zwischen Elementarteilchen und ist zentral für die Quantenelektrodynamik. Sie ist definiert als:
	\begin{equation}
		\label{eq:alpha_em_def}
		\alphaEM = \frac{e^2}{4\pi \varepsilon_0 \hbar c} \approx \frac{1}{137.035999}.
	\end{equation}
	
	Im vereinheitlichten Einheitensystem setzen wir \(\alphaEM = 1\), was bedeutet, dass die elektrische Ladung \(e\) dimensionslos wird und ihren Wert direkt aus den elektromagnetischen Vakuumkonstanten ableitet:
	\begin{equation}
		\label{eq:charge_relation}
		e = \sqrt{4\pi \varepsilon_0 \hbar c}
	\end{equation}
	
	Diese Festlegung führt zu einer erheblichen Vereinfachung der elektromagnetischen Gleichungen und offenbart die fundamentale Natur der elektromagnetischen Wechselwirkung als Teil des vereinheitlichten Rahmens.
	
	\subsection{Natürliche Einheiten mit \(\alphaEM = 1\)}
	\label{subsec:alpha_one}
	
	In der theoretischen Physik werden \(c\) und \(\hbar\) üblicherweise gleich Eins gesetzt, wie von Planck \cite{planck1899} eingeführt. Hier untersuchen wir die Konsequenzen der zusätzlichen Festlegung der Feinstrukturkonstante \(\alphaEM = 1\).
	
	\begin{theorem}[Definition von \(\alphaEM = 1\)]
		Die Feinstrukturkonstante ist \cite{Feynman1985}:
		\begin{equation}
			\alphaEM = \frac{e^2}{4\pi\varepsilon_0 \hbar c} \approx \frac{1}{137.036}
		\end{equation}
		Mit \(\alphaEM = 1\), \(\hbar = c = 1\):
		\begin{equation}
			e = \sqrt{4\pi\varepsilon_0}
		\end{equation}
	\end{theorem}
	
	\textbf{Hinweis}: Hier bezeichnet \(\alphaEM\) die Feinstrukturkonstante, nicht die Wien-Konstante \(\alpha_W \approx 2.82\), wie in \cite{pascher_temp_2025} untersucht.
	
	\subsection{Energie als fundamentale Einheit}
	\label{subsec:energy_fundamental}
	
	\begin{theorem}[Energie als Basis]
		Alle Größen können auf Energie reduziert werden \cite{Duff2002}:
		\begin{itemize}
			\item Länge: \([L] = [E^{-1}]\)
			\item Zeit: \([T] = [E^{-1}]\)
			\item Masse: \([M] = [E]\)
			\item Ladung: \([Q] = [\sqrt{4\pi}]\) (dimensionslos)
		\end{itemize}
	\end{theorem}
	
	Im T0-Modell wird dies durch \(\Tfield = \frac{\hbar}{m c^2}\) ergänzt, wobei \(m\) variabel ist, und Energie spielt eine zentrale Rolle, wie in \cite{pascher_zeit_masse_2025} detailliert beschrieben.
	
	\subsection{Vereinfachte Grundgleichungen}
	\label{subsec:simplified_equations}
	
	\begin{itemize}
		\item Maxwell-Gleichungen \cite{Feynman1985}:
		\begin{align}
			\nabla \cdot \vec{E} &= \rho \\
			\nabla \times \vec{B} - \frac{\partial \vec{E}}{\partial t} &= \vec{j}
		\end{align}
		\item Schrödinger-Gleichung:
		\begin{equation}
			i \frac{\partial \psi}{\partial t} = -\frac{1}{2m} \nabla^2 \psi + V \psi
		\end{equation}
	\end{itemize}
	
	Diese vereinfachten Formen entstehen natürlich aus der Lagrange-Formulierung des T0-Modells, wie in \cite{pascher_lagrange_2025} gezeigt.
	
	\subsection{Tabelle der neuformulierten Größen}
	\label{subsec:reformulated_quantities}
	
	\begin{center}
		\begin{tabular}{|l|c|c|}
			\hline
			\textbf{Physikalische Größe} & \textbf{SI-Einheiten} & \textbf{\(\hbar = c = \alphaEM = 1\)} \\
			\hline
			Länge & m & \(\text{eV}^{-1}\) \\
			Zeit & s & \(\text{eV}^{-1}\) \\
			Masse & kg & eV \\
			Energie & J & eV \\
			Ladung & C & dimensionslos \\
			El. Feld & V/m & \(\text{eV}^2\) \\
			Mag. Feld & T & \(\text{eV}^2\) \\
			\hline
		\end{tabular}
	\end{center}
	
	\subsection{Kosmologische Implikationen}
	\label{subsec:cosmological_implications}
	
	Die Annahme \(\alphaEM = 1\) könnte im T0-Modell \cite{pascher_galaxies_2025}:
	\begin{itemize}
		\item Elektromagnetische Wechselwirkungen stärker mit der Gravitation verbinden, da \(\Tfield\) emergent die Gravitation erklärt, wie in \cite{pascher_emergente_gravitation_2025} gezeigt.
		\item Eine einheitliche Energiebeschreibung ermöglichen, die mit der Rotverschiebung aufgrund von Energieverlust an \(\Tfield\) konsistent ist \cite{pascher_messdifferenzen_2025}.
	\end{itemize}
	
	Im T0-Modell wird wellenlängenabhängige Rotverschiebung durch den Parameter \(\betaT^{\text{SI}} \approx 0.008\) in SI-Einheiten beschrieben, während in natürlichen Einheiten \(\betaT = 1\) gilt \cite{pascher_params_2025}. Dies ist konsistent mit:
	\begin{equation}
		\label{eq:wavelength_redshift}
		z(\lambda) = z_0 (1 + \betaT \ln(\lambda/\lambda_0))
	\end{equation}
	
	Wenn sowohl \(\alphaEM = 1\) als auch \(\betaT^{\text{nat}} = 1\) gleichzeitig gesetzt werden, entstehen erhebliche Abweichungen von Standard-Modell-Vorhersagen (z.B. \(z(\lambda) \approx 3.3\) für \(\lambda/\lambda_0 = 10\)). Diese Abweichungen sollten nicht als "unphysikalisch" betrachtet werden, sondern könnten auf eine Standard-Modell-Verzerrung in der Interpretation kosmologischer Daten hinweisen \cite{pascher_alphabeta_2025}.
	
	\begin{figure}[h]
		\centering
		\begin{tikzpicture}
			\begin{axis}[
				xlabel={Energie [eV]},
				ylabel={Länge [eV\(^{-1}\)]},
				xlabel style={font=\large},
				ylabel style={font=\large},
				tick label style={font=\normalsize},
				xmin=0, xmax=10,
				ymin=0, ymax=10,
				legend pos=north east,
				legend style={font=\large},
				grid=both,
				minor tick num=1
				]
				\addplot[blue, ultra thick, domain=0.1:10, samples=100] {1/x};
				\legend{\(L = E^{-1}\)}
			\end{axis}
		\end{tikzpicture}
		\caption{Beziehung zwischen Energie und Länge im \(\alphaEM = 1\)-System.}
		\label{fig:energy_length}
	\end{figure}
	
	\section{Herleitung von Plancks Wirkungsquantum}
	\label{sec:planck_quantum}
	
	Plancks Wirkungsquantum \(h\) bildet eine fundamentale Verbindung zwischen Quantenmechanik und Elektrodynamik. Im T0-Modell kann eine tiefere Beziehung zwischen \(h\) und den elektromagnetischen Vakuumkonstanten hergestellt werden.
	
	\subsection{Verbindung zu elektromagnetischen Konstanten}
	\label{subsec:electromagnetic_constants}
	
	Die Lichtgeschwindigkeit im Vakuum ist gegeben durch:
	\begin{equation}
		\label{eq:speed_of_light}
		c = \frac{1}{\sqrt{\mu_0 \varepsilon_0}}
	\end{equation}
	
	Im vereinheitlichten Einheitensystem mit \(c = 1\):
	\begin{equation}
		\label{eq:mu_epsilon}
		\mu_0 \varepsilon_0 = 1
	\end{equation}
	
	Eine dimensional konsistente Beziehung zwischen Plancks Wirkungsquantum und elektromagnetischen Konstanten kann über die Vakuumimpedanz formuliert werden:
	\begin{equation}
		\label{eq:vacuum_impedance}
		Z_0 = \sqrt{\frac{\mu_0}{\varepsilon_0}} \approx 376.73 \, \Omega
	\end{equation}
	
	Wir können nun eine fundamentale Länge \(\lambda_0\) definieren als:
	\begin{equation}
		\lambda_0 = \frac{c}{2\pi \nu_0}
	\end{equation}
	wobei \(\nu_0\) eine charakteristische Frequenz ist. Wenn wir \(\lambda_0\) als Compton-Wellenlänge eines Elementarteilchens interpretieren, dann:
	\begin{equation}
		\lambda_0 = \frac{h}{m_0 c}
	\end{equation}
	
	Durch Kombination dieser Beziehungen und unter Verwendung der Vakuumimpedanz erhalten wir:
	\begin{equation}
		h = 2\pi m_0 c \lambda_0 = \frac{2\pi m_0 c^2}{\nu_0} = \frac{2\pi E_0}{\nu_0}
	\end{equation}
	
	wobei \(E_0 = m_0 c^2\) die Ruheenergie des Elementarteilchens ist.
	
	Durch Einführung einer fundamentalen Kopplungskonstante \(\kappa_h\), die das Verhältnis zwischen der Vakuumimpedanz und einer charakteristischen Quantenimpedanz darstellt:
	\begin{equation}
		\kappa_h = \frac{Z_0}{Z_Q} = \frac{Z_0}{h/e^2} = \frac{e^2 Z_0}{h}
	\end{equation}
	
	können wir schreiben:
	\begin{equation}
		h = \frac{e^2 Z_0}{\kappa_h} = \frac{e^2}{\kappa_h} \sqrt{\frac{\mu_0}{\varepsilon_0}}
	\end{equation}
	
	Im vereinheitlichten Einheitensystem mit \(\alphaEM = 1\), wobei \(e^2 = 4\pi\varepsilon_0\hbar c\), wird dies zu:
	\begin{equation}
		h = \frac{4\pi\varepsilon_0\hbar c}{\kappa_h} \sqrt{\frac{\mu_0}{\varepsilon_0}} = \frac{4\pi\hbar}{\kappa_h} \sqrt{\mu_0\varepsilon_0} \cdot c^2 \sqrt{\frac{\mu_0}{\varepsilon_0}} = \frac{4\pi\hbar}{\kappa_h} \mu_0 c^2
	\end{equation}
	
	Mit \(\kappa_h = 2\) erhalten wir die einfache Form:
	\begin{equation}
		h = 2\pi\hbar \mu_0 c^2
	\end{equation}
	
	Diese Beziehung zeigt eine tiefe Verbindung zwischen Plancks Wirkungsquantum, elektromagnetischen Vakuumkonstanten und der Struktur der Raumzeit. Im vereinheitlichten Einheitensystem mit \(\hbar = c = \mu_0 = 1\) erhalten wir \(h = 2\pi\), wie erwartet.
	
	\subsection{Alternative Herleitungsansätze}
	\label{subsec:alternative_derivations}
	
	Die Verbindung zwischen \(h\) und elektromagnetischen Konstanten kann auf verschiedene Weise hergestellt werden, alle konsistent mit dem vereinheitlichten Einheitensystem:
	
	\begin{enumerate}
		\item \textbf{De-Broglie-Wellenlänge:} \(\lambda = \frac{h}{p}\) führt mit \(p = \frac{\hbar}{c \cdot \Tfield}\) für masselose Teilchen zu einer Beziehung zwischen dem intrinsischen Zeitfeld und der Wellenlänge.
		\item \textbf{Compton-Streuung:} Die Compton-Wellenlänge \(\lambda_C = \frac{h}{mc}\) ist über \(\lambda_C = \frac{h \cdot \Tfield}{c}\) mit der intrinsischen Zeit verknüpft.
		\item \textbf{Unschärferelation:} Die Energie-Zeit-Unschärfe \(\Delta E \Delta t \geq \frac{\hbar}{2}\) gewinnt im T0-Modell tiefere Bedeutung, da Zeit und Energie durch die fundamentale Beziehung \(E = \frac{\hbar}{\Tfield}\) verbunden sind.
	\end{enumerate}
	
	Alle diese Ansätze bestätigen die fundamentale Rolle von \(h = 2\pi\) im vereinheitlichten Einheitensystem und offenbaren die tiefgreifende Verbindung zwischen Quantenmechanik, Elektrodynamik und der Zeit-Masse-Dualität des T0-Modells, wie in \cite{pascher_zeit_2025} ausgearbeitet.
	
	\section{Zusammenfassung und Ausblick}
	\label{sec:summary}
	
	Das vereinheitlichte Einheitensystem des T0-Modells mit \(\hbar = c = G = \alphaEM = \betaT = \alphaW = 1\) bietet einen eleganten Rahmen für die Vereinheitlichung fundamentaler Wechselwirkungen. Die Herleitung fundamentaler Konstanten aus diesem System legt nahe, dass sie möglicherweise nicht wirklich "fundamental" sind, sondern vielmehr Artefakte unserer gewählten Einheitensysteme. Durch das Setzen von \(\alphaEM = 1\) wird Energie zur fundamentalen Einheit, was im T0-Modell eine tiefere Einheit von Zeit, Masse und Gravitation offenbart.
	
	Die Zeit-Masse-Dualität \(m = \frac{\hbar}{\Tfield c^2}\) deckt tiefgreifende Verbindungen zwischen scheinbar verschiedenen physikalischen Phänomenen auf und könnte der Schlüssel zu einem umfassenderen Verständnis der Natur über bekannte Grenzen hinaus sein.
	
	Zukünftige Forschung sollte sich auf experimentelle Tests der spezifischen Vorhersagen des T0-Modells konzentrieren, insbesondere:
	
	\begin{itemize}
		\item Wellenlängenabhängige Rotverschiebung \(z(\lambda) = z_0 (1 + \betaT \ln(\lambda/\lambda_0))\), wie in \cite{pascher_messdifferenzen_2025} ausgearbeitet.
		\item Emergenz der Gravitation aus dem Zeitfeld, wie in \cite{pascher_emergente_gravitation_2025} hergeleitet.
		\item Verbindung zwischen Quantenkorrelationen und Zeitfeld-Geometrie, wie in \cite{pascher_feldtheorie_2025} erforscht.
	\end{itemize}
	
	Diese Tests könnten unser fundamentales Verständnis von Raum, Zeit und Materie revolutionieren und den Weg für eine vollständige Vereinheitlichung der Physik ebnen.
	
	\begin{thebibliography}{99}
		\bibitem{pascher_zeit_2025} Pascher, J. (2025). \href{https://github.com/jpascher/T0-Time-Mass-Duality/tree/main/2/pdf/Deutsch/ZeitEmergentQM.pdf}{Zeit als emergente Eigenschaft in der Quantenmechanik}. 23. März 2025.
		
		\bibitem{pascher_galaxies_2025} Pascher, J. (2025). \href{https://github.com/jpascher/T0-Time-Mass-Duality/tree/main/2/pdf/Deutsch/MassVarGalaxien.pdf}{Massenvariation in Galaxien: Eine Analyse im T0-Modell}. 30. März 2025.
		
		\bibitem{pascher_messdifferenzen_2025} Pascher, J. (2025). \href{https://github.com/jpascher/T0-Time-Mass-Duality/tree/main/2/pdf/Deutsch/MessdifferenzenT0Standard.pdf}{Messunterschiede zwischen dem T0-Modell und dem $\Lambda$CDM-Standardmodell}. 2. April 2025.
		
		\bibitem{pascher_params_2025} Pascher, J. (2025). \href{https://github.com/jpascher/T0-Time-Mass-Duality/tree/main/2/pdf/Deutsch/ZeitMasseT0Params.pdf}{Zeit-Masse-Dualitätstheorie: Herleitung der Parameter}. 4. April 2025.
		
		\bibitem{pascher_alphabeta_2025} Pascher, J. (2025). \href{https://github.com/jpascher/T0-Time-Mass-Duality/tree/main/2/pdf/Deutsch/Alpha1Beta1Konsistenz.pdf}{Vereinheitlichtes Einheitensystem im T0-Modell}. 5. April 2025.
		
		\bibitem{pascher_temp_2025} Pascher, J. (2025). \href{https://github.com/jpascher/T0-Time-Mass-Duality/tree/main/2/pdf/Deutsch/TempEinheitenCMB.pdf}{Anpassung der Temperatureinheiten in natürlichen Einheiten}. 2. April 2025.
		
		\bibitem{pascher_emergente_gravitation_2025} Pascher, J. (2025). \href{https://github.com/jpascher/T0-Time-Mass-Duality/tree/main/2/pdf/Deutsch/EmergentGravT0.pdf}{Emergente Gravitation im T0-Modell}. 1. April 2025.
		
		\bibitem{pascher_lagrange_2025} Pascher, J. (2025). \href{https://github.com/jpascher/T0-Time-Mass-Duality/tree/main/2/pdf/Deutsch/MathZeitMasseLagrange.pdf}{Mathematische Kernformulierungen der Zeit-Masse-Dualitätstheorie}. 29. März 2025.
		
		\bibitem{pascher_zeit_masse_2025} Pascher, J. (2025). \href{https://github.com/jpascher/T0-Time-Mass-Duality/tree/main/2/pdf/Deutsch/ZeitMasseNeuerBlick.pdf}{Zeit und Masse: Ein neuer Blick auf alte Formeln}. 22. März 2025.
		
		\bibitem{pascher_feldtheorie_2025} Pascher, J. (2025). \href{https://github.com/jpascher/T0-Time-Mass-Duality/tree/main/2/pdf/Deutsch/FeldtheorieQuanten.pdf}{Feldtheorie und Quantenkorrelationen}. 28. März 2025.
		
		\bibitem{planck1899} Planck, M. (1899). Über irreversible Strahlungsvorgänge. \textit{Sitzungsberichte der Königlich Preußischen Akademie der Wissenschaften zu Berlin}, 5, 440-480.
		
		\bibitem{Feynman1985} Feynman, R. P. (1985). \textit{QED: Die seltsame Theorie des Lichts und der Materie}. Princeton University Press.
		
		\bibitem{Duff2002} Duff, M. J., Okun, L. B., \& Veneziano, G. (2002). \textit{Trialog über die Anzahl fundamentaler Konstanten}. \textit{Journal of High Energy Physics}, 2002(03), 023.
		
		\bibitem{Wilczek2008} Wilczek, F. (2008). \textit{Die Leichtigkeit des Seins: Masse, Äther und die Vereinigung der Kräfte}. Basic Books.
	\end{thebibliography}
	
\end{document}