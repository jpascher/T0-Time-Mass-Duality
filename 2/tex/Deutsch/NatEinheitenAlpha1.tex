\documentclass[12pt,a4paper]{article}
\usepackage[utf8]{inputenc}
\usepackage[T1]{fontenc}
\usepackage[ngerman]{babel}
\usepackage[left=2cm,right=2cm,top=2cm,bottom=2cm]{geometry}
\usepackage{lmodern}
\usepackage{amsmath}
\usepackage{amssymb}
\usepackage{physics}  % Already includes \grad, \dv, \pdv, \e, \ii, \vev
\usepackage{hyperref}
\usepackage{tcolorbox}
\usepackage{booktabs}
\usepackage{enumitem}
\usepackage[table,xcdraw]{xcolor}
\usepackage{pgfplots}
\pgfplotsset{compat=1.18}
\usepackage{graphicx}
\usepackage{float}
\usepackage{mathtools}
\usepackage{amsthm}
\usepackage{cleveref}
\usepackage{siunitx}
\usepackage{fancyhdr} % For headers and footers
\usepackage{tocloft}  % For table of contents styling

% Headers and Footers
\pagestyle{fancy}
\fancyhf{}
\fancyhead[L]{Johann Pascher}
\fancyhead[R]{Zeit-Masse-Dualität}
\fancyfoot[C]{\thepage}
\renewcommand{\headrulewidth}{0.4pt}
\renewcommand{\footrulewidth}{0.4pt}

% Table of Contents Styling
\renewcommand{\cftsecfont}{\color{blue}}
\renewcommand{\cftsubsecfont}{\color{blue}}
\renewcommand{\cftsecpagefont}{\color{blue}}
\renewcommand{\cftsubsecpagefont}{\color{blue}}
\setlength{\cftsecindent}{1cm}
\setlength{\cftsubsecindent}{2cm}

\hypersetup{
	colorlinks=true,
	linkcolor=blue,
	citecolor=blue,
	urlcolor=blue,
	pdftitle={Energie als fundamentale Einheit: Natürliche Einheiten mit alphaEM = 1 im T0-Modell},
	pdfauthor={Johann Pascher},
	pdfsubject={Theoretische Physik},
	pdfkeywords={T0-Modell, natürliche Einheiten, Feinstrukturkonstante, einheitliches Einheitensystem, Zeit-Masse-Dualität}
}

% Custom Commands (consistent)
\newcommand{\Tfield}{T(x)}
\newcommand{\betaT}{\beta_{\text{T}}}
\newcommand{\alphaEM}{\alpha_{\text{EM}}}
\newcommand{\alphaW}{\alpha_{\text{W}}}
\newcommand{\Mpl}{M_{\text{Pl}}}
\newcommand{\Tzerot}{T_0(\Tfield)}
\newcommand{\Tzero}{T_0}
\newcommand{\vecx}{\vec{x}}
\newcommand{\gammaf}{\gamma_{\text{Lorentz}}}
\newcommand{\DhiggsT}{\Tfield (\partial_\mu + ig A_\mu) \Phi + \Phi \partial_\mu \Tfield} % Consistent Definition

\newtheorem{theorem}{Satz}[section]
\newtheorem{proposition}[theorem]{Proposition}

\begin{document}
	
	\title{Energie als fundamentale Einheit: \\ Natürliche Einheiten mit \(\alphaEM = 1\) im T0-Modell}
	\author{Johann Pascher}
	\date{25. März 2025}
	
	\maketitle
	\tableofcontents
	\newpage
	
	\section{Einführung in das einheitliche Einheitensystem}
	\label{sec:intro}
	
	\subsection{Von natürlichen Einheiten zu einem vollständig einheitlichen System}
	\label{subsec:natural_units}
	
	In der theoretischen Physik werden verschiedene Systeme natürlicher Einheiten verwendet, um die mathematische Formulierung physikalischer Gesetze zu vereinfachen. Die bekanntesten umfassen:
	
	\begin{itemize}
		\item \textbf{Natürliche Einheiten:} \(\hbar = c = 1\)
		\item \textbf{Planck-Einheiten:} \(\hbar = c = G = 1\)
		\item \textbf{Elektrodynamische natürliche Einheiten:} \(\hbar = c = \alphaEM = 1\)
		\item \textbf{Thermodynamische natürliche Einheiten:} \(\hbar = c = k_B = \alphaW = 1\)
	\end{itemize}
	
	Das T0-Modell führt ein vollständig einheitliches Einheitensystem ein, in dem zusätzlich:
	\begin{equation}
		\label{eq:unified_system}
		\betaT = \alphaEM = \alphaW = 1
	\end{equation}
	gesetzt wird. In diesem System werden alle physikalischen Größen auf die Dimension der Energie reduziert:
	
	\begin{tcolorbox}[colback=blue!5!white,colframe=blue!75!black,title=Dimensionen im einheitlichen Einheitensystem]
		\begin{itemize}
			\item Länge: \([L] = [E^{-1}]\)
			\item Zeit: \([T] = [E^{-1}]\)
			\item Masse: \([M] = [E]\)
			\item Temperatur: \([T_{\text{emp}}] = [E]\)
			\item Elektrische Ladung: \([Q] = [1]\) (dimensionslos)
			\item Intrinsische Zeit: \([\Tfield] = [E^{-1}]\)
		\end{itemize}
	\end{tcolorbox}
	
	Dieses einheitliche System offenbart fundamentale Beziehungen zwischen scheinbar unterschiedlichen physikalischen Phänomenen und ermöglicht eine elegantere mathematische Formulierung des T0-Modells, wie in \cite{pascher_lagrange_2025} und \cite{pascher_alphabeta_2025} gezeigt.
	
	\subsection{Konzept der Energie als fundamentale Einheit}
	\label{subsec:energy_concept}
	
	Diese Arbeit untersucht die Konsequenzen der Annahme, dass die Feinstrukturkonstante \(\alphaEM = 1\) in einem System natürlicher Einheiten (\(\hbar = c = 1\)) auf das T0-Modell der Zeit-Masse-Dualität angewendet wird. Hier wird Energie als fundamentale Einheit identifiziert, auf die alle physikalischen Größen reduziert werden können. Die Analyse umfasst dimensionale Reformulierungen, vereinfachte fundamentale Gleichungen und kosmologische Implikationen im Kontext des T0-Modells, das absolute Zeit und variable Masse postuliert, wie in \cite{pascher_zeit_masse_2025} beschrieben.
	
	\section{Extrapolation der Physik jenseits bekannter Grenzen}
	\label{sec:beyond_limits}
	
	\subsection{Physik jenseits der Lichtgeschwindigkeit}
	\label{subsec:beyond_lightspeed}
	
	Die Lichtgeschwindigkeit \(c\) gilt in der Standardphysik als absolute Grenze für Materie und Signalübertragung, eine direkte Konsequenz der Lorentz-Transformation und der Relativitätstheorie. Innerhalb dieses Rahmens wurden alle fundamentalen Konstanten und die Planck-Skala definiert. Diese Grenze könnte jedoch nur innerhalb unseres aktuellen theoretischen Modells gelten. Im T0-Modell mit seiner fundamentalen Zeit-Masse-Dualität ist eine alternative Interpretation möglich:
	
	\begin{itemize}
		\item \textbf{Neuinterpretation der Massenvariation:} Im T0-Modell wird die Masse \(m = \frac{\hbar}{\Tfield c^2}\) durch das intrinsische Zeitfeld bestimmt. Die relativistische Massenänderung \(m = m_0/\sqrt{1-v^2/c^2}\) kann als Variation von \(\Tfield\) interpretiert werden, wie in \cite{pascher_zeit_2025} erläutert.
		\item \textbf{Modifizierte Transformationsgesetze:} Im einheitlichen Einheitensystem mit \(c = 1\) könnten erweiterte Transformationen Bereiche mit \(v > 1\) ohne Verletzung der Kausalität beschreiben, während die fundamentale Beziehung \(m = \frac{\hbar}{\Tfield c^2}\) erhalten bleibt.
		\item \textbf{Erweiterte Konstanten:} Mit \(\alphaEM = \betaT = \alphaW = 1\) entsteht ein konsistenter Rahmen, der möglicherweise jenseits der Lichtgeschwindigkeit gültig bleibt.
	\end{itemize}
	
	Diese spekulativen Überlegungen werden in \cref{sec:speculative} weiter vertieft.
	
	\subsection{Konsequenzen für Kausalität und Information}
	\label{subsec:causality}
	
	Im T0-Modell mit seiner Zeit-Masse-Dualität könnte Kausalität neu interpretiert werden, wie in \cite{pascher_feldtheorie_2025} ausgeführt:
	\begin{itemize}
		\item \textbf{Zeitfeldbasierte Kausalität:} Kausale Beziehungen könnten durch die Geometrie des Zeitfeldes \(\Tfield\) bestimmt werden, nicht durch Lichtkegelstrukturen.
		\item \textbf{Nichtlokale Informationsübertragung:} Die scheinbare Nichtlokalität der Quantenmechanik könnte durch die Struktur des intrinsischen Zeitfeldes erklärt werden, ohne dass überlichtschnelle Signalübertragung erforderlich ist.
		\item \textbf{Massenabhängige Kausalstruktur:} Da \(m = \frac{\hbar}{\Tfield c^2}\), könnten kausale Beziehungen massenabhängig sein, was möglicherweise eine natürliche Erklärung für Quantenkorrelationen bietet.
	\end{itemize}
	
	\section{Einführung in die Feinstrukturkonstante \(\alphaEM\)}
	\label{sec:alpha_em}
	
	Die Feinstrukturkonstante \(\alphaEM\) beschreibt die Stärke der elektromagnetischen Wechselwirkung zwischen Elementarteilchen und ist zentral für die Quantenelektrodynamik. Sie ist definiert als:
	\begin{equation}
		\label{eq:alpha_em_def}
		\alphaEM = \frac{e^2}{4\pi \varepsilon_0 \hbar c} \approx \frac{1}{137,035999}.
	\end{equation}
	
	Im einheitlichen Einheitensystem setzen wir \(\alphaEM = 1\), was bedeutet, dass die elektrische Ladung \(e\) dimensionslos wird und ihren Wert direkt aus den elektromagnetischen Vakuumkonstanten ableitet:
	\begin{equation}
		\label{eq:charge_relation}
		e = \sqrt{4\pi \varepsilon_0 \hbar c}
	\end{equation}
	
	Diese Festlegung führt zu einer erheblichen Vereinfachung elektromagnetischer Gleichungen und offenbart die fundamentale Natur der elektromagnetischen Wechselwirkung als Teil des einheitlichen Rahmens.
	
	\subsection{Natürliche Einheiten mit \(\alphaEM = 1\)}
	\label{subsec:alpha_one}
	
	In der theoretischen Physik werden \(c\) und \(\hbar\) üblicherweise auf eins gesetzt, wie von Planck eingeführt \cite{planck1899}. Hier untersuchen wir die Konsequenzen des zusätzlichen Setzens der Feinstrukturkonstanten \(\alphaEM = 1\).
	
	\begin{theorem}[Definition von \(\alphaEM = 1\)]
		Die Feinstrukturkonstante ist \cite{Feynman1985}:
		\begin{equation}
			\alphaEM = \frac{e^2}{4\pi\varepsilon_0 \hbar c} \approx \frac{1}{137,036}
		\end{equation}
		Mit \(\alphaEM = 1\), \(\hbar = c = 1\):
		\begin{equation}
			e = \sqrt{4\pi\varepsilon_0}
		\end{equation}
	\end{theorem}
	
	\textbf{Anmerkung}: Hier bezeichnet \(\alphaEM\) die Feinstrukturkonstante, nicht die Wien-Konstante \(\alpha_W \approx 2,82\), wie in \cite{pascher_temp_2025} untersucht.
	
	\subsection{Energie als fundamentale Einheit}
	\label{subsec:energy_fundamental}
	
	\begin{theorem}[Energie als Grundlage]
		Alle Größen können auf Energie reduziert werden \cite{Duff2002}:
		\begin{itemize}
			\item Länge: \([L] = [E^{-1}]\)
			\item Zeit: \([T] = [E^{-1}]\)
			\item Masse: \([M] = [E]\)
			\item Ladung: \([Q] = [\sqrt{4\pi}]\) (dimensionslos)
		\end{itemize}
	\end{theorem}
	
	Im T0-Modell wird dies ergänzt durch \(\Tfield = \frac{\hbar}{m c^2}\), wobei \(m\) variabel ist, und Energie eine zentrale Rolle spielt, wie in \cite{pascher_zeit_masse_2025} detailliert beschrieben.
	
	\subsection{Vereinfachte fundamentale Gleichungen}
	\label{subsec:simplified_equations}
	
	\begin{itemize}
		\item Maxwell-Gleichungen \cite{Feynman1985}:
		\begin{align}
			\nabla \cdot \vec{E} &= \rho \\
			\nabla \times \vec{B} - \frac{\partial \vec{E}}{\partial t} &= \vec{j}
		\end{align}
		\item Schrödinger-Gleichung:
		\begin{equation}
			i \frac{\partial \psi}{\partial t} = -\frac{1}{2m} \nabla^2 \psi + V \psi
		\end{equation}
	\end{itemize}
	
	Diese vereinfachten Formen ergeben sich natürlich aus der Lagrange-Formulierung des T0-Modells, wie in \cite{pascher_lagrange_2025} gezeigt.
	
	\subsection{Tabelle reformulierter Größen}
	\label{subsec:reformulated_quantities}
	
	\begin{center}
		\begin{tabular}{|l|c|c|}
			\hline
			\textbf{Physikalische Größe} & \textbf{SI-Einheiten} & \textbf{\(\hbar = c = \alphaEM = 1\)} \\
			\hline
			Länge & m & \(\text{eV}^{-1}\) \\
			Zeit & s & \(\text{eV}^{-1}\) \\
			Masse & kg & eV \\
			Energie & J & eV \\
			Ladung & C & dimensionslos \\
			Elektrisches Feld & V/m & \(\text{eV}^2\) \\
			Magnetisches Feld & T & \(\text{eV}^2\) \\
			\hline
		\end{tabular}
	\end{center}
	
	\subsection{Kosmologische Implikationen}
	\label{subsec:cosmological_implications}
	
	Die Annahme \(\alphaEM = 1\) könnte im T0-Modell \cite{pascher_galaxies_2025}:
	\begin{itemize}
		\item Elektromagnetische Wechselwirkungen stärker mit Gravitation verbinden, da \(\Tfield\) Gravitation emergent erklärt, wie in \cite{pascher_emergente_gravitation_2025} demonstriert.
		\item Eine einheitliche Energiedarstellung ermöglichen, die mit Rotverschiebung durch Energieverlust an \(\Tfield\) konsistent ist \cite{pascher_messdifferenzen_2025}.
	\end{itemize}
	
	Im T0-Modell wird die wellenlängenabhängige Rotverschiebung durch den Parameter \(\betaT^{\text{SI}} \approx 0,008\) in SI-Einheiten beschrieben, während in natürlichen Einheiten \(\betaT = 1\) gilt \cite{pascher_params_2025}. Dies ist konsistent mit:
	\begin{equation}
		\label{eq:wavelength_redshift}
		z(\lambda) = z_0 (1 + \betaT \ln(\lambda/\lambda_0))
	\end{equation}
	
	Wenn sowohl \(\alphaEM = 1\) als auch \(\betaT^{\text{nat}} = 1\) gleichzeitig gesetzt werden, ergeben sich signifikante Abweichungen von den Vorhersagen des Standardmodells (z. B. \(z(\lambda) \approx 3,3\) für \(\lambda/\lambda_0 = 10\)). Diese Abweichungen sollten nicht als „unphysikalisch“ betrachtet werden, sondern könnten auf eine Voreingenommenheit des Standardmodells bei der Interpretation kosmologischer Daten hinweisen \cite{pascher_alphabeta_2025}.
	
	\begin{figure}[h]
		\centering
		\begin{tikzpicture}
			\begin{axis}[
				xlabel={Energie [eV]},
				ylabel={Länge [eV\(^{-1}\)]},
				xlabel style={font=\large},
				ylabel style={font=\large},
				tick label style={font=\normalsize},
				xmin=0, xmax=10,
				ymin=0, ymax=10,
				legend pos=north east,
				legend style={font=\large},
				grid=both,
				minor tick num=1
				]
				\addplot[blue, ultra thick, domain=0.1:10, samples=100] {1/x};
				\legend{\(L = E^{-1}\)}
			\end{axis}
		\end{tikzpicture}
		\caption{Beziehung zwischen Energie und Länge im \(\alphaEM = 1\)-System.}
		\label{fig:energy_length}
	\end{figure}
	
	\section{Ableitung des Planck'schen Wirkungsquantums}
	\label{sec:planck_quantum}
	
	Das Planck'sche Wirkungsquantum \(h\) bildet eine fundamentale Verbindung zwischen Quantenmechanik und Elektrodynamik. Im T0-Modell kann eine tiefere Beziehung zwischen \(h\) und den elektromagnetischen Vakuumkonstanten hergestellt werden.
	
	\subsection{Verbindung zu elektromagnetischen Konstanten}
	\label{subsec:electromagnetic_constants}
	
	Die Lichtgeschwindigkeit im Vakuum ist gegeben durch:
	\begin{equation}
		\label{eq:speed_of_light}
		c = \frac{1}{\sqrt{\mu_0 \varepsilon_0}}
	\end{equation}
	
	Im einheitlichen Einheitensystem mit \(c = 1\):
	\begin{equation}
		\label{eq:mu_epsilon}
		\mu_0 \varepsilon_0 = 1
	\end{equation}
	
	Eine dimensionskonsistente Beziehung zwischen dem Planck'schen Wirkungsquantum und elektromagnetischen Konstanten kann über die Vakuumimpedanz formuliert werden:
	\begin{equation}
		\label{eq:vacuum_impedance}
		Z_0 = \sqrt{\frac{\mu_0}{\varepsilon_0}} \approx 376,73 \, \Omega
	\end{equation}
	
	Wir können nun eine fundamentale Länge \(\lambda_0\) definieren als:
	\begin{equation}
		\lambda_0 = \frac{c}{2\pi \nu_0}
	\end{equation}
	wobei \(\nu_0\) eine charakteristische Frequenz ist. Wenn wir \(\lambda_0\) als Compton-Wellenlänge eines Elementarteilchens interpretieren, dann:
	\begin{equation}
		\lambda_0 = \frac{h}{m_0 c}
	\end{equation}
	
	Durch Kombination dieser Beziehungen und unter Verwendung der Vakuumimpedanz erhalten wir:
	\begin{equation}
		h = 2\pi m_0 c \lambda_0 = \frac{2\pi m_0 c^2}{\nu_0} = \frac{2\pi E_0}{\nu_0}
	\end{equation}
	
	wobei \(E_0 = m_0 c^2\) die Ruheenergie des Elementarteilchens ist.
	
	Durch Einführung einer fundamentalen Kopplungskonstanten \(\kappa_h\), die das Verhältnis zwischen der Vakuumimpedanz und einer charakteristischen Quantenimpedanz darstellt:
	\begin{equation}
		\kappa_h = \frac{Z_0}{Z_Q} = \frac{Z_0}{h/e^2} = \frac{e^2 Z_0}{h}
	\end{equation}
	
	können wir schreiben:
	\begin{equation}
		h = \frac{e^2 Z_0}{\kappa_h} = \frac{e^2}{\kappa_h} \sqrt{\frac{\mu_0}{\varepsilon_0}}
	\end{equation}
	
	Im einheitlichen Einheitensystem mit \(\alphaEM = 1\), wobei \(e^2 = 4\pi\varepsilon_0\hbar c\), wird dies zu:
	\begin{equation}
		h = \frac{4\pi\varepsilon_0\hbar c}{\kappa_h} \sqrt{\frac{\mu_0}{\varepsilon_0}} = \frac{4\pi\hbar}{\kappa_h} \sqrt{\mu_0\varepsilon_0} \cdot c^2 \sqrt{\frac{\mu_0}{\varepsilon_0}} = \frac{4\pi\hbar}{\kappa_h} \mu_0 c^2
	\end{equation}
	
	Mit \(\kappa_h = 2\) erhalten wir die einfache Form:
	\begin{equation}
		h = 2\pi\hbar \mu_0 c^2
	\end{equation}
	
	Diese Beziehung zeigt eine tiefe Verbindung zwischen dem Planck'schen Wirkungsquantum, elektromagnetischen Vakuumkonstanten und der Struktur der Raumzeit. Im einheitlichen Einheitensystem mit \(\hbar = c = \mu_0 = 1\) erhalten wir \(h = 2\pi\), wie erwartet.
	
	\subsection{Alternative Ableitungsansätze}
	\label{subsec:alternative_derivations}
	
	Die Verbindung zwischen \(h\) und elektromagnetischen Konstanten kann auf verschiedene Weisen hergestellt werden, alle konsistent mit dem einheitlichen Einheitensystem:
	
	\begin{enumerate}
		\item \textbf{De-Broglie-Wellenlänge:} \(\lambda = \frac{h}{p}\) führt, mit \(p = \frac{\hbar}{c \cdot \Tfield}\) für masselose Teilchen, zu einer Beziehung zwischen dem intrinsischen Zeitfeld und der Wellenlänge.
		\item \textbf{Compton-Streuung:} Die Compton-Wellenlänge \(\lambda_C = \frac{h}{mc}\) ist mit der intrinsischen Zeit über \(\lambda_C = \frac{h \cdot \Tfield}{c}\) verknüpft.
		\item \textbf{Unschärferelation:} Die Energie-Zeit-Unschärfe \(\Delta E \Delta t \geq \frac{\hbar}{2}\) gewinnt im T0-Modell tiefere Bedeutung, da Zeit und Energie durch die fundamentale Beziehung \(E = \frac{\hbar}{\Tfield}\) verbunden sind.
	\end{enumerate}
	
	Alle diese Ansätze bestätigen die fundamentale Rolle von \(h = 2\pi\) im einheitlichen Einheitensystem und offenbaren die tiefe Verbindung zwischen Quantenmechanik, Elektrodynamik und der Zeit-Masse-Dualität des T0-Modells, wie in \cite{pascher_zeit_2025} ausgeführt.
	
	\section{Alternative Formulierungen der Feinstrukturkonstanten}
	\label{sec:alternative_alpha}
	
	\subsection{Standarddefinition der Feinstrukturkonstanten}
	\label{subsec:standard_alpha}
	
	Die Feinstrukturkonstante \(\alphaEM\) ist definiert als:
	\begin{equation}
		\alphaEM = \frac{e^2}{4\pi \varepsilon_0 \hbar c} \approx \frac{1}{137,035999}
	\end{equation}
	
	Diese dimensionslose Konstante charakterisiert die Stärke der elektromagnetischen Wechselwirkung.
	
	\subsection{Verwendung des klassischen Elektronenradius}
	\label{subsec:electron_radius}
	
	Der klassische Elektronenradius ist definiert als:
	\begin{equation}
		r_e = \frac{e^2}{4\pi \varepsilon_0 m_e c^2}
	\end{equation}
	
	Die Compton-Wellenlänge des Elektrons ist:
	\begin{equation}
		\lambda_C = \frac{h}{m_e c} = \frac{2\pi\hbar}{m_e c}
	\end{equation}
	
	Die Feinstrukturkonstante kann als Verhältnis dieser charakteristischen Längen ausgedrückt werden:
	\begin{equation}
		\alphaEM = \frac{r_e}{\lambda_C/2\pi} = \frac{2\pi r_e}{\lambda_C}
	\end{equation}
	
	Einsetzen der Definitionen ergibt:
	\begin{equation}
		\alphaEM = \frac{2\pi \cdot \frac{e^2}{4\pi \varepsilon_0 m_e c^2}}{\frac{h}{m_e c}} = \frac{e^2}{2\varepsilon_0 h c}
	\end{equation}
	
	Mit \(h = 2\pi\hbar\) ergibt sich wieder die Standarddefinition:
	\begin{equation}
		\alphaEM = \frac{e^2}{4\pi \varepsilon_0 \hbar c}
	\end{equation}
	
	\section{Wiens Konstante \(\alphaW\) im einheitlichen Einheitensystem}
	\label{sec:wien_constant}
	
	Wiens Konstante \(\alphaW\) bestimmt die Beziehung zwischen der Frequenz des Strahlungsmaximums und der Temperatur bei Schwarzkörperstrahlung:
	
	\begin{equation}
		\label{eq:wien_law}
		\nu_{\text{max}} = \alphaW \cdot \frac{k_B T}{h}
	\end{equation}
	
	mit \(\alphaW \approx 2,821439\).
	
	Im einheitlichen Einheitensystem setzen wir \(k_B = 1\) und \(\hbar = 1\) (also \(h = 2\pi\)), was zu folgender Beziehung führt:
	
	\begin{equation}
		\nu_{\text{max}} = \alphaW \cdot \frac{T}{2\pi}
	\end{equation}
	
	Mit \(\alphaW = 1\) wird dies zu:
	
	\begin{equation}
		\nu_{\text{max}} = \frac{T}{2\pi}
	\end{equation}
	
	Diese Beziehung ist konsistent mit der Diskussion in \cite{pascher_temp_2025} und zeigt die direkte Proportionalität zwischen Temperatur und der Frequenz des Strahlungsmaximums im einheitlichen Einheitensystem.
	
	\section{Der T0-Parameter \(\betaT\) im einheitlichen Einheitensystem}
	\label{sec:beta_t}
	
	Der T0-Parameter \(\betaT\) ist ein fundamentaler dimensionsloser Parameter im T0-Modell, der die Kopplung zwischen dem intrinsischen Zeitfeld \(\Tfield\) und anderen physikalischen Feldern beschreibt. Er tritt in verschiedenen Kontexten auf und verbindet scheinbar unterschiedliche physikalische Phänomene.
	
	\subsection{Ableitung aus fundamentalen Parametern}
	\label{subsec:beta_derivation}
	
	Der Parameter \(\betaT\) kann aus zugrunde liegenden physikalischen Konstanten abgeleitet werden:
	
	\begin{equation}
		\label{eq:beta_fundamental}
		\betaT^{\text{nat}} = \frac{\lambda_h^2 v^2}{16\pi^3 m_h^2 \xi}{16\pi^3 m_h^2 \xi} \cdot \frac{1}{m_h^2} \cdot \frac{1}{\xi}
	\end{equation}
	
	wobei \(\lambda_h\) die Higgs-Selbstkopplung, \(v\) der Higgs-Vakuum-Erwartungswert, \(m_h\) die Higgs-Masse und \(\xi\) ein dimensionsloser Parameter ist, der die charakteristische Längenskala \(r_0 = \xi \cdot l_P\) definiert, mit \(l_P\) als Planck-Länge.
	
	In SI-Einheiten wurde \(\betaT \approx 0,008\) aus kosmologischen Beobachtungen und perturbativen Berechnungen abgeleitet \cite{pascher_params_2025}. Im einheitlichen Einheitensystem setzen wir jedoch \(\betaT = 1\), was zu einer eleganten Vereinfachung vieler Formeln führt.
	
	\subsection{Physikalische Manifestationen von \(\betaT\)}
	\label{subsec:beta_manifestations}
	
	Der Parameter \(\betaT\) manifestiert sich in verschiedenen physikalischen Kontexten:
	
	\begin{enumerate}
		\item \textbf{Temperatur-Rotverschiebungs-Beziehung:} 
		\begin{equation}
			\label{eq:temp_redshift}
			T(z) = T_0 (1 + z) (1 + \betaT \ln(1 + z))
		\end{equation}
		Im einheitlichen Einheitensystem mit \(\betaT = 1\) vereinfacht sich dies zu:
		\begin{equation}
			T(z) = T_0 (1 + z) (1 + \ln(1 + z))
		\end{equation}
		\item \textbf{Wellenlängenabhängige Rotverschiebung:} 
		\begin{equation}
			z(\lambda) = z_0 (1 + \betaT \ln(\lambda/\lambda_0))
		\end{equation}
		Mit \(\betaT = 1\) in natürlichen Einheiten:
		\begin{equation}
			z(\lambda) = z_0 (1 + \ln(\lambda/\lambda_0))
		\end{equation}
		\item \textbf{Kopplung Zeitfeld-Higgs:} Der Parameter \(\betaT\) beschreibt die Kopplung zwischen dem intrinsischen Zeitfeld \(\Tfield\) und dem Higgs-Feld \(\Phi\):
		\begin{equation}
			\Tfield = \frac{\hbar}{y \langle \Phi \rangle c^2}
		\end{equation}
		wobei \(y\) die Yukawa-Kopplung ist.
	\end{enumerate}
	
	Diese Manifestationen werden in \cite{pascher_params_2025} und \cite{pascher_alphabeta_2025} detailliert untersucht.
	
	\subsection{Verbindung zu anderen dimensionslosen Konstanten}
	\label{subsec:connection_constants}
	
	Im einheitlichen Einheitensystem mit \(\alphaEM = \betaT = \alphaW = 1\) werden fundamentale Beziehungen zwischen scheinbar unterschiedlichen physikalischen Phänomenen deutlich. Die Beziehung zwischen \(\betaT\) und \(\alphaEM\) kann ausgedrückt werden als:
	
	\begin{equation}
		\betaT \cdot \alphaEM \approx \frac{\xi \cdot \lambda_h^2 v^2}{16\pi^3 m_h^2} \cdot \frac{1}{\xi} \cdot \frac{e^2}{4\pi\varepsilon_0\hbar c} = \frac{\lambda_h^2 v^2 e^2}{64\pi^4\varepsilon_0\hbar c m_h^2}
	\end{equation}
	
	Im einheitlichen Einheitensystem mit \(\betaT = \alphaEM = 1\) wird dies zu:
	
	\begin{equation}
		\frac{\lambda_h^2 v^2 e^2}{64\pi^4\varepsilon_0\hbar c m_h^2} = 1
	\end{equation}
	
	Diese Beziehung deutet auf eine tiefere Einheit zwischen elektromagnetischen und Higgs-vermittelten Wechselwirkungen hin, die im T0-Modell durch das intrinsische Zeitfeld \(\Tfield\) verbunden sind, wie in \cite{pascher_higgs_2025} ausgeführt.
	
	\section{Ableitung der Gravitationskonstanten \(G\)}
	\label{sec:gravitational_constant}
	
	In Planck-Einheiten:
	\begin{equation}
		G = \frac{\hbar c}{m_P^2}
	\end{equation}
	
	Im T0-Modell entsteht Gravitation aus den Gradienten des intrinsischen Zeitfeldes \(\Tfield\). Im einheitlichen Einheitensystem mit \(G = 1\) wird das Gravitationspotential abgeleitet, wie in \cite{pascher_emergente_gravitation_2025}:
	
	\begin{equation}
		\label{eq:grav_potential}
		\Phi(r) = -\frac{M}{r} + \kappa r
	\end{equation}
	
	wobei der erste Term dem Newtonschen Potential entspricht und der zweite Term aus der globalen Variation des Zeitfeldes resultiert. Dieser lineare Term erklärt Effekte, die im Standardmodell dunkler Energie zugeschrieben werden, wie in \cite{pascher_galaxies_2025} detailliert erläutert.
	
	\section{Spekulative Erweiterungen des T0-Modells}
	\label{sec:speculative}
	
	Das T0-Modell mit seiner Zeit-Masse-Dualität eröffnet Möglichkeiten für spekulative Erweiterungen, die über die bekannten Grenzen der Standardphysik hinausgehen. Obwohl noch nicht experimentell bestätigt, bieten diese Erweiterungen konzeptionell faszinierende Perspektiven.
	
	\subsection{Modifizierte Energie-Impuls-Beziehung}
	\label{subsec:modified_energy_momentum}
	
	Im einheitlichen Einheitensystem kann die relativistische Energie-Impuls-Beziehung durch einen zusätzlichen Term erweitert werden:
	
	\begin{equation}
		\label{eq:modified_energy_momentum}
		E^2 = m^2 + p^2 + \frac{\alpha_c p^4}{E_P^2}
	\end{equation}
	
	wobei \(\alpha_c\) eine dimensionslose Konstante ist, die die Stärke der Modifikation charakterisiert, und \(E_P\) die Planck-Energie ist. Diese Modifikation wird durch verschiedene Ansätze motiviert:
	
	\begin{enumerate}
		\item \textbf{Zeitfeldfluktuationen:} Im T0-Modell könnten Fluktuationen des intrinsischen Zeitfeldes \(\Tfield\) bei hohen Energien zu Abweichungen von der Standard-Dispersionsrelation führen. Der Parameter \(\alpha_c\) quantifiziert die Kopplung zwischen diesen Fluktuationen und der Teilchendynamik.
		\item \textbf{Quantengravitationseffekte:} Nahe der Planck-Skala werden Abweichungen durch Quantengravitationseffekte erwartet. Die \(p^4\)-Abhängigkeit entspricht einer natürlichen Erweiterung der Standardbeziehung durch Korrekturen auf Planck-Skala.
		\item \textbf{Konsistenz mit Zeit-Masse-Dualität:} Die Modifikation kann als natürliche Konsequenz der fundamentalen Beziehung \(m = \frac{\hbar}{\Tfield c^2}\) bei hohen Energien verstanden werden, wo die Struktur des Zeitfeldes von seiner makroskopischen Beschreibung abweichen könnte.
	\end{enumerate}
	
	Basierend auf theoretischen Überlegungen und Konsistenz mit aktuellen experimentellen Grenzen erwarten wir \(|\alpha_c| \lesssim 10^{-2}\).
	
	\subsection{Physik jenseits der Lichtgeschwindigkeit}
	\label{subsec:beyond_c}
	
	Im T0-Modell könnte die konventionelle Lichtgeschwindigkeitsbarriere (\(v \leq c\)) eine emergente Eigenschaft sein, die aus der makroskopischen Struktur des Zeitfeldes resultiert, anstatt eine fundamentale Grenze zu sein. Unter bestimmten Bedingungen könnte \(v > c\) möglich sein, ohne Kausalität zu verletzen:
	
	\begin{enumerate}
		\item \textbf{Modifizierte kausale Struktur:} Kausalität im T0-Modell wird durch die Struktur des Zeitfeldes \(\Tfield\) bestimmt, nicht allein durch die Lichtkegelstruktur. Bei starken Gradienten von \(\Tfield\) könnte die effektive kausale Struktur verändert werden.
		\item \textbf{Massenabhängige Maximalgeschwindigkeit:} Da \(m = \frac{\hbar}{\Tfield c^2}\), könnte die effektive Maximalgeschwindigkeit massenabhängig sein, gegeben durch \(v_{\text{max}}(m) = c \cdot f(m \cdot \Tzero)\), wobei \(f\) eine zu bestimmende Funktion ist.
		\item \textbf{Tunneleffekte im Zeitfeld:} Quantenmechanisches Tunneln durch „Zeitbarrieren“ könnte scheinbar überlichtschnelle Phänomene ermöglichen, ähnlich wie quantenmechanisches Tunneln durch Energiebarrieren.
	\end{enumerate}
	
	Diese Möglichkeiten widersprechen der Relativitätstheorie nicht, sondern erweitern sie innerhalb eines neuen konzeptionellen Rahmens, in dem Lorentz-Invarianz eine emergente Eigenschaft ist, kein fundamentales Prinzip, wie in \cite{pascher_zeit_masse_2025} diskutiert.
	
	\subsection{Experimentelle Signaturen}
	\label{subsec:experimental_signatures}
	
	Obwohl spekulativ, könnten diese Erweiterungen experimentelle Signaturen hinterlassen:
	
	\begin{enumerate}
		\item \textbf{Energieabhängige Lichtgeschwindigkeit:} Die modifizierte Energie-Impuls-Beziehung führt zu einer energieabhängigen Lichtgeschwindigkeit \(c(E) \approx c (1 - \alpha_c \frac{E^2}{2E_P^2})\), testbar durch Beobachtungen hochenergetischer kosmischer Strahlen oder Gammastrahlenausbrüche.
		\item \textbf{Modifizierte Compton-Streuung:} Die Streuung hochenergetischer Photonen an Elektronen könnte Abweichungen vom Standardverhalten zeigen, nachweisbar durch Präzisionsmessungen.
		\item \textbf{Unerwartete Quantenkorrelationen:} Wenn die kausale Struktur durch das Zeitfeld modifiziert wird, könnten Quantenkorrelationen Muster aufweisen, die über die Vorhersagen der Standard-Quantenmechanik hinausgehen.
	\end{enumerate}
	
	Diese spekulativen Erweiterungen bieten konzeptionell faszinierende Möglichkeiten zur Lösung fundamentaler Probleme in der theoretischen Physik, erfordern jedoch experimentelle Beweise, um als Teil des etablierten wissenschaftlichen Rahmens akzeptiert zu werden.
	
	\section{Dimensionsanalyse mit SI-Einheiten}
	\label{sec:dimensional_analysis}
	
	\subsection{Überprüfung der dimensionalen Konsistenz}
	\label{subsec:dimensional_consistency}
	
	Die Dimensionen der abgeleiteten Größen werden überprüft:
	
	\begin{center}
		\begin{tabular}{lcc}
			\toprule
			\textbf{Größe} & \textbf{SI-Einheiten} & \textbf{Natürliche Einheiten} \\
			\midrule
			Länge \(L\) & \si{\meter} & \(\text{Energie}^{-1}\) \\
			Zeit \(T\) & \si{\second} & \(\text{Energie}^{-1}\) \\
			Masse \(M\) & \si{\kilo\gram} & \(\text{Energie}\) \\
			Ladung \(e\) & \si{\coulomb} & \(\sqrt{\alphaEM}\) \\
			\(G\) & \si{\meter^3\kilo\gram^{-1}\second^{-2}} & \(\text{Energie}^{-2}\) \\
			\(\varepsilon_0\) & \si{\farad\per\meter} & \(\text{Energie}^{-2}\) \\
			\(\mu_0\) & \si{\henry\per\meter} & \(\text{Energie}^{-2}\) \\
			\(h\) & \(\SI{6.62607015e-34}{\joule\second} = \si{\kilo\gram \meter\squared\per\second}\) & \(2\pi\) (dimensionslos) \\
			\bottomrule
		\end{tabular}
	\end{center}
	
	Diese Dimensionsanalyse bestätigt die Konsistenz des einheitlichen Einheitensystems mit SI-Einheiten, wie in \cite{pascher_alphabeta_2025} ausgeführt.
	
	\subsection{Übereinstimmung empirischer und theoretischer Werte}
	\label{subsec:empirical_theoretical}
	
	Die theoretische Lichtgeschwindigkeit
	\begin{equation}
		c_{theor} = \frac{1}{\sqrt{\mu_0 \varepsilon_0}}
	\end{equation}
	stimmt mit \(c = \SI{299792458}{\meter\per\second}\) überein, wenn \(\mu_0 = 4\pi \times 10^{-7} \, \si{\henry\per\meter}\) und \(\varepsilon_0 = 8,8541878128 \times 10^{-12} \, \si{\farad\per\meter}\) verwendet werden.
	
	\section{Darstellung als Planck-Größen}
	\label{sec:planck_quantities}
	
	Im einheitlichen Einheitensystem werden physikalische Größen dimensionslos:
	\begin{align}
		\tilde{m} &= \frac{m}{m_P}, \\
		\tilde{L} &= \frac{L}{l_P}, \\
		\tilde{t} &= \frac{t}{t_P}.
	\end{align}
	
	Diese Darstellung ist besonders nützlich für Überlegungen zur Quantengravitation, wie in \cite{pascher_planck_2025} untersucht.
	
	\section{Implikationen für Photonen im T0-Modell}
	\label{sec:photons}
	
	In natürlichen Einheiten (\(c = 1\), \(\hbar = 1\)) gilt für die Photonenergie:
	\begin{equation}
		E = \omega,
	\end{equation}
	und mit \(E = m\) folgt eine frequenzabhängige Masse:
	\begin{equation}
		m_{\gamma} = \omega.
	\end{equation}
	
	Im T0-Modell ist die intrinsische Zeit eines Photons:
	\begin{equation}
		\Tfield_{\gamma} = \frac{\hbar}{\omega c^2}
	\end{equation}
	was die fundamentale Zeit-Masse-Dualität \(m = \frac{\hbar}{\Tfield c^2}\) für Photonen bestätigt. Dieses Konzept wird in \cite{pascher_photons_2025} weiter untersucht.
	
	\section{Überlegungen jenseits der Planck-Skala}
	\label{sec:beyond_planck}
	
	\subsection{Absolute Zeit und intrinsische Zeit im T0-Modell}
	\label{subsec:absolute_intrinsic}
	
	Das T0-Modell vereinheitlicht zwei komplementäre Perspektiven:
	\begin{itemize}
		\item \textbf{Absolute Zeit-Perspektive:} Zeit \(\Tzero\) ist absolut und konstant, während die Masse variiert: \(m = \frac{\hbar}{\Tfield c^2}\).
		\item \textbf{Intrinsische Zeit-Perspektive:} \(\Tfield = \frac{\hbar}{m c^2}\) führt zur modifizierten Schrödinger-Gleichung: \(i\hbar \Tfield \frac{\partial \psi}{\partial t} = \hat{H} \psi\).
	\end{itemize}
	
	Diese Dualität ist konzeptionell verwandt mit der Welle-Teilchen-Dualität und offenbart tiefere Verbindungen zwischen Zeit, Masse und Energie, wie in \cite{pascher_zeit_2025} erläutert.
	
	\subsection{Verbindung zu Planck-Einheiten}
	\label{subsec:planck_connection}
	
	Für Massen nahe der Planck-Masse (\(m \approx m_P\)) nähert sich die intrinsische Zeit \(\Tfield\) der Planck-Zeit \(t_P\) an:
	\begin{equation}
		\Tfield = \frac{\hbar}{m c^2} \approx \frac{\hbar}{m_P c^2} = \frac{\hbar}{\sqrt{\hbar c/G} \cdot c^2} = \sqrt{\frac{\hbar G}{c^5}} = t_P
	\end{equation}
	
	Diese Beziehung legt nahe, dass die Planck-Zeit eine natürliche Schwelle für intrinsische Zeit sein könnte, was möglicherweise neue Einsichten in die Quantengravitation bietet, wie in \cite{pascher_planck_2025} detailliert beschrieben.
	
	\section{Konsequenzen eines vollständig einheitlichen Einheitensystems}
	\label{sec:fully_unified}
	
	Das Setzen von \(\alphaEM = \betaT = \alphaW = 1\) im einheitlichen Einheitensystem führt zu tiefgreifenden konzeptionellen Vereinfachungen:
	
	\begin{enumerate}
		\item \textbf{Elektrodynamik:} Elektrische Ladungen werden dimensionslos, und elektromagnetische Gleichungen nehmen eine elegantere Form an.
		\item \textbf{Thermodynamik:} Temperatur wird direkt proportional zur Frequenz, was eine einheitliche Beschreibung thermischer und quantenmechanischer Phänomene ermöglicht.
		\item \textbf{Gravitation:} Gravitation entsteht natürlich aus den Gradienten des Zeitfeldes ohne zusätzliche Kopplungskonstanten, wie in \cite{pascher_emergente_gravitation_2025} gezeigt.
		\item \textbf{Emergente Raumzeit:} Raumzeitgeometrie kann als emergentes Phänomen aus dem fundamentaleren Zeitfeld verstanden werden.
	\end{enumerate}
	
	\subsection{Philosophische Implikationen}
	\label{subsec:philosophical}
	
	\begin{itemize}
		\item Energie als fundamentalste Eigenschaft der Realität \cite{Wilczek2008}, unterstützt im T0-Modell durch absolute Zeit und variable Masse.
		\item Raum und Zeit als emergente Eigenschaften eines Energiefeldes \cite{Verlinde2011}, kompatibel mit \(\Tfield\) als grundlegendem Feld.
	\end{itemize}
	
	\section{Zusammenfassung und Ausblick}
	\label{sec:summary}
	
	Das einheitliche Einheitensystem des T0-Modells mit \(\hbar = c = G = \alphaEM = \betaT = \alphaW = 1\) bietet einen eleganten Rahmen zur Vereinheitlichung fundamentaler Wechselwirkungen. Die Ableitung fundamentaler Konstanten aus diesem System legt nahe, dass diese möglicherweise nicht wirklich „fundamental“ sind, sondern Artefakte unserer gewählten Einheitensysteme. Durch das Setzen von \(\alphaEM = 1\) wird Energie zur fundamentalen Einheit, was im T0-Modell eine tiefere Einheit von Zeit, Masse und Gravitation offenbart. Diese Vereinfachung steht im Einklang mit dem allgemeinen Prinzip, dass fundamentale dimensionslose Parameter in einer vollständig natürlichen Formulierung einfache Werte annehmen sollten. Ähnlich wie das Setzen von \(\betaT^{\text{nat}} = 1\) führt \(\alphaEM = 1\) zu einer konzeptionell klareren Theorie, in der die Dimensionen aller physikalischen Größen auf eine einzige fundamentale Dimension (Energie) reduziert werden können. Für eine umfassende Analyse der Konsistenz beider Vereinfachungen siehe \cite{pascher_alphabeta_2025}.
	
	Die Zeit-Masse-Dualität \(m = \frac{\hbar}{\Tfield c^2}\) deckt tiefgreifende Verbindungen zwischen scheinbar unterschiedlichen physikalischen Phänomenen auf und könnte der Schlüssel zu einem umfassenderen Verständnis der Natur jenseits bekannter Grenzen sein.
	
	Zukünftige Forschungen sollten sich auf experimentelle Tests der spezifischen Vorhersagen des T0-Modells konzentrieren, insbesondere:
	
	\begin{itemize}
		\item Wellenlängenabhängige Rotverschiebung \(z(\lambda) = z_0 (1 + \betaT \ln(\lambda/\lambda_0))\), wie in \cite{pascher_messdifferenzen_2025} ausgeführt.
		\item Modifizierte Temperatur-Rotverschiebungs-Beziehung aus \cref{eq:temp_redshift}, wie in \cite{pascher_temp_2025} diskutiert.
		\item Entstehung der Gravitation aus dem Zeitfeld, wie in \cite{pascher_emergente_gravitation_2025} abgeleitet.
		\item Verbindung zwischen Quantenkorrelationen und Zeitfeldgeometrie, wie in \cite{pascher_feldtheorie_2025} untersucht.
	\end{itemize}
	
	Diese Tests könnten unser fundamentales Verständnis von Raum, Zeit und Materie revolutionieren und den Weg für eine vollständige Vereinheitlichung der Physik ebnen.
	
	\begin{thebibliography}{99}
		\bibitem{pascher_zeit_2025} Pascher, J. (2025). \href{https://github.com/jpascher/T0-Time-Mass-Duality/tree/main/2/pdf/Deutsch/ZeitEmergentQM.pdf}{Zeit als emergente Eigenschaft in der Quantenmechanik: Eine Verbindung zwischen Relativität, Feinstrukturkonstante und Quantendynamik}. 23. März 2025. \textit{Siehe \cref{subsec:beyond_lightspeed,subsec:alternative_derivations,subsec:absolute_intrinsic} für zentrale Konzepte zur intrinsischen Zeit.}
		
		\bibitem{pascher_galaxies_2025} Pascher, J. (2025). \href{https://github.com/jpascher/T0-Time-Mass-Duality/tree/main/2/pdf/Deutsch/MassVarGalaxien.pdf}{MassenVariation in Galaxien: Eine Analyse im T0-Modell mit emergenter Gravitation}. 30. März 2025. \textit{Referenziert in \cref{subsec:cosmological_implications,sec:gravitational_constant} für Galaxiendynamiken.}
		
		\bibitem{pascher_messdifferenzen_2025} Pascher, J. (2025). \href{https://github.com/jpascher/T0-Time-Mass-Duality/tree/main/2/pdf/Deutsch/MessdifferenzenT0Standard.pdf}{Kompensatorische und additive Effekte: Eine Analyse der Messunterschiede zwischen dem T0-Modell und dem \(\Lambda\)CDM-Standardmodell}. 2. April 2025. \textit{Referenziert in \cref{subsec:cosmological_implications,sec:summary} für kosmologische Implikationen.}
		
		\bibitem{pascher_params_2025} Pascher, J. (2025). \href{https://github.com/jpascher/T0-Time-Mass-Duality/tree/main/2/pdf/Deutsch/ZeitMasseT0Params.pdf}{Zeit-Masse-Dualitätstheorie (T0-Modell): Ableitung der Parameter \(\kappa\), \(\alpha\) und \(\beta\)}. 4. April 2025. \textit{Siehe \cref{subsec:cosmological_implications,subsec:beta_derivation,subsec:beta_manifestations} für Parameterableitungen.}
		
		\bibitem{pascher_higgs_2025} Pascher, J. (2025). \href{https://github.com/jpascher/T0-Time-Mass-Duality/tree/main/2/pdf/Deutsch/MathHiggsZeitMasse.pdf}{Mathematische Formulierung des Higgs-Mechanismus in der Zeit-Masse-Dualität}. 28. März 2025. \textit{Relevant für \cref{subsec:connection_constants} zu Higgs-Zeitfeld-Wechselwirkungen.}
		
		\bibitem{pascher_alphabeta_2025} Pascher, J. (2025). \href{https://github.com/jpascher/T0-Time-Mass-Duality/tree/main/2/pdf/Deutsch/Alpha1Beta1Konsistenz.pdf}{Einheitliches Einheitensystem im T0-Modell: Die Konsistenz von \(\alpha = 1\) und \(\beta = 1\)}. 5. April 2025. \textit{Wesentlich für \cref{subsec:natural_units,subsec:cosmological_implications,subsec:beta_manifestations,subsec:dimensional_consistency,sec:summary} zum einheitlichen Einheitensystem.}
		
		\bibitem{pascher_temp_2025} Pascher, J. (2025). \href{https://github.com/jpascher/T0-Time-Mass-Duality/tree/main/2/pdf/Deutsch/TempEinheitenCMB.pdf}{Anpassung der Temperatureinheiten in natürlichen Einheiten und CMB-Messungen}. 2. April 2025. \textit{Referenziert in \cref{subsec:alpha_one,sec:wien_constant,sec:summary} für Temperaturskalierung.}
		
		\bibitem{pascher_emergente_gravitation_2025} Pascher, J. (2025). \href{https://github.com/jpascher/T0-Time-Mass-Duality/tree/main/2/pdf/Deutsch/EmergentGravT0.pdf}{Emergente Gravitation im T0-Modell: Eine umfassende Ableitung}. 1. April 2025. \textit{Schlüsselreferenz für \cref{subsec:cosmological_implications,sec:gravitational_constant,sec:fully_unified,sec:summary} zu emergenter Gravitation.}
		
		\bibitem{pascher_planck_2025} Pascher, J. (2025). \href{https://github.com/jpascher/T0-Time-Mass-Duality/tree/main/2/pdf/Deutsch/JenseitsPlanck.pdf}{Reale Konsequenzen der Neuformulierung von Zeit und Masse in der Physik: Jenseits der Planck-Skala}. 24. März 2025. \textit{Referenziert in \cref{sec:planck_quantities,subsec:planck_connection} für Implikationen der Planck-Skala.}
		
		\bibitem{pascher_lagrange_2025} Pascher, J. (2025). \href{https://github.com/jpascher/T0-Time-Mass-Duality/tree/main/2/pdf/Deutsch/MathZeitMasseLagrange.pdf}{Von Zeitdilatation zur Massenvariation: Mathematische Kernformulierungen der Zeit-Masse-Dualitätstheorie}. 29. März 2025. \textit{Wichtig für \cref{subsec:natural_units,subsec:simplified_equations} zu mathematischen Formulierungen.}
		
		\bibitem{pascher_zeit_masse_2025} Pascher, J. (2025). \href{https://github.com/jpascher/T0-Time-Mass-Duality/tree/main/2/pdf/Deutsch/ZeitMasseNeuerBlick.pdf}{Zeit und Masse: Ein neuer Blick auf alte Formeln – und Befreiung von traditionellen Zwängen}. 22. März 2025. \textit{Wesentlicher Hintergrund für \cref{subsec:energy_concept,subsec:energy_fundamental,subsec:beyond_c} zur Zeit-Masse-Dualität.}
		
		\bibitem{pascher_photons_2025} Pascher, J. (2025). \href{https://github.com/jpascher/T0-Time-Mass-Duality/tree/main/2/pdf/Deutsch/DynMassePhotonenNichtlokal.pdf}{Dynamische Masse von Photonen und ihre Implikationen für Nichtlokalität im T0-Modell}. 25. März 2025. \textit{Referenziert in \cref{sec:photons} für Photon-Implikationen.}
		
		\bibitem{pascher_feldtheorie_2025} Pascher, J. (2025). \href{https://github.com/jpascher/T0-Time-Mass-Duality/tree/main/2/pdf/Deutsch/FeldtheorieQuanten.pdf}{Feldtheorie und Quantenkorrelationen: Eine neue Perspektive auf Instantaneität}. 28. März 2025. \textit{Relevant für \cref{subsec:causality,sec:summary} zu Quantenkorrelationen.}
		
		\bibitem{einstein1905} Einstein, A. (1905). Hängt die Trägheit eines Körpers von seinem Energiegehalt ab? \textit{Annalen der Physik}, 323(13), 639-641.
		
		\bibitem{planck1899} Planck, M. (1899). Über irreversible Strahlungsvorgänge. \textit{Sitzungsberichte der Königlich Preußischen Akademie der Wissenschaften zu Berlin}, 5, 440-480.
		
		\bibitem{Feynman1985} Feynman, R. P. (1985). \textit{QED: Die seltsame Theorie des Lichts und der Materie}. Princeton University Press.
		
		\bibitem{Duff2002} Duff, M. J., Okun, L. B., \& Veneziano, G. (2002). \textit{Trialog über die Anzahl fundamentaler Konstanten}. \textit{Journal of High Energy Physics}, 2002(03), 023.
		
		\bibitem{Verlinde2011} Verlinde, E. (2011). \textit{Über den Ursprung der Gravitation und die Gesetze Newtons}. \textit{Journal of High Energy Physics}, 2011(4), 29.
		
		\bibitem{Wilczek2008} Wilczek, F. (2008). \textit{Die Leichtigkeit des Seins: Masse, Äther und die Vereinheitlichung der Kräfte}. Basic Books.
		
		\bibitem{Rubin1980} Rubin, V. C., \& Ford Jr, W. K. (1980). Rotation der Andromeda-Nebel aus einer spektroskopischen Untersuchung von Emissionsregionen. \textit{The Astrophysical Journal}, 159, 379.
		
		\bibitem{McGaugh2016} McGaugh, S. S., Lelli, F., \& Schombert, J. M. (2016). Radiale Beschleunigungsbeziehung in rotationsgestützten Galaxien. \textit{Physical Review Letters}, 117(20), 201101.
	\end{thebibliography}
	
\end{document}