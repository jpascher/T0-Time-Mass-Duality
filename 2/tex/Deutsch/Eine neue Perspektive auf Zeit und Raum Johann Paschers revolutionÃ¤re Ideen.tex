\documentclass[a4paper,12pt]{article}
\usepackage[utf8]{inputenc}
\usepackage[T1]{fontenc}
\usepackage[german]{babel}
\usepackage{amsmath}
\usepackage{amssymb}
\usepackage{geometry}
\usepackage{lmodern}
\usepackage{graphicx}
\usepackage{tikz}
\usepackage{pgfplots}
\pgfplotsset{compat=1.18}
\usepackage{xcolor}
\usepackage{tocloft}
\usepackage{amsthm}
\usepackage[colorlinks=true, linkcolor=blue, filecolor=blue, citecolor=blue, urlcolor=blue]{hyperref}
\usepackage{cleveref}

\geometry{a4paper, margin={2.5cm}}

\renewcommand{\cftsecfont}{\color{blue}}
\renewcommand{\cftsubsecfont}{\color{blue}}
\renewcommand{\cftsecpagefont}{\color{blue}}
\renewcommand{\cftsubsecpagefont}{\color{blue}}
\setlength{\cftsecindent}{1cm}
\setlength{\cftsubsecindent}{2cm}

\newtheorem{theorem}{Theorem}[section]
\newtheorem{proposition}[theorem]{Proposition}

\title{Eine neue Perspektive auf Zeit und Raum: Johann Paschers revolutionäre Ideen}
\author{Johann Pascher}
\date{25. März 2025}

\begin{document}
	
	\maketitle
	
	\begin{abstract}
		Diese Arbeit präsentiert Johann Paschers revolutionäres Modell, das die traditionelle Sicht auf Zeit und Raum umkehrt. Im Gegensatz zu Einsteins Relativitätstheorie, die Zeit als variabel und Masse als konstant behandelt, postuliert Pascher eine absolute Zeit und variable Masse. Jedes Teilchen besitzt eine intrinsische Zeit, die umgekehrt proportional zu seiner Masse ist. Diese Perspektive bietet neue Erklärungen für Quantenverschränkung, kosmische Rotverschiebung und die Struktur von Schwarzen Löchern, während sie Energie als fundamentale Größe etabliert. Überprüfbare Vorhersagen werden vorgeschlagen, um das Modell zu testen.
	\end{abstract}
	
	\tableofcontents
	\newpage
	
	Stellen Sie sich vor, Sie betrachten ein vertrautes Gemälde, eines, das Sie schon hundertmal gesehen haben. Dann neigt jemand es leicht, und plötzlich bemerken Sie Details und Muster, die Ihnen nie aufgefallen sind. Genau das macht der Johann Pascher mit unserem Verständnis des Universums.
	
	Seit über einem Jahrhundert dominieren Einsteins Theorien unsere Sicht auf Zeit und Raum. Wir haben akzeptiert, dass Zeit dehnbar ist - sie verlangsamt sich, wenn man sich schnell bewegt oder in ein starkes Gravitationsfeld gerät. Währenddessen glauben wir, dass die Ruhemasse eines Objekts unveränderlich ist, eine feste Eigenschaft der Materie. Dieses Verständnis hat uns gute Dienste geleistet und erklärt alles von GPS-Satelliten bis zur Beugung des Sternenlichts um die Sonne.
	
	Aber Pascher dreht diese Sichtweise um: In seinem alternativen Modell ist die Zeit absolut und fließt konstant, während es die Masse ist, die variiert. Dies ist keine bloße Spekulation, sondern ein vollständig ausgearbeitetes Modell mit eigenen mathematischen Formulierungen, die genau die gleichen experimentellen Beobachtungen erklären, die wir mit dem herkömmlichen Modell verbinden.
	
	\section{Die Uhr in jedem Teilchen}
	In Paschers Modell hat jedes Teilchen im Universum - jedes Elektron, jedes Proton - seine eigene charakteristische Zeitskala, die als „intrinsische Zeit“ bezeichnet wird. Diese Zeit ist umgekehrt proportional zur Masse des Teilchens. Schwerere Teilchen haben schneller tickende Uhren; leichtere Teilchen haben langsamere.
	
	Nehmen wir ein Myon (ähnlich einem Elektron, aber etwa 200-mal schwerer). Im Standardmodell erklären wir seine verlängerte Lebensdauer während der Reise durch unsere Atmosphäre durch Zeitdilatation. In Paschers Modell verändert sich stattdessen die Masse des Myons, während die Zeit konstant weiterläuft. Mathematisch sind diese beiden Beschreibungen äquivalent - sie führen zu identischen messbaren Ergebnissen, bieten aber völlig unterschiedliche Perspektiven auf die zugrundeliegende Realität.
	
	Diese „intrinsische Zeit“ ist nicht nur ein theoretisches Konstrukt, sondern eine mathematisch präzise definierte Größe, die neue Einsichten in Quantenphänomene ermöglicht.
	
	\section{Wenn entfernte Teilchen verbunden sind}
	Die Quantenverschränkung, bei der zwei Teilchen über beliebige Entfernungen verbunden scheinen, erhält in Paschers Rahmenwerk eine neue Interpretation. Während die herkömmliche Quantenmechanik das Phänomen beschreibt, ohne es wirklich zu erklären (Einstein nannte es „spukhafte Fernwirkung“), bietet Paschers Modell einen konkreten Mechanismus.
	
	In seinem Modell ist die Verbindung nicht instantan, sondern hängt von der Masse der beteiligten Teilchen ab. Zwei verschränkte Teilchen unterschiedlicher Masse entwickeln sich mit unterschiedlichen intrinsischen Zeitraten. Was als simultane Korrelation erscheint, hat tatsächlich eine massenabhängige Verzögerung, die proportional zum Verhältnis der Massen ist. Diese Verzögerung ist messbar und stellt eine klare, überprüfbare Vorhersage dar, die das Standardmodell der Quantenmechanik nicht macht.
	
	\begin{theorem}[Massenabhängige Verzögerung]
		Bei verschränkten Teilchen mit Massen $m_1$ und $m_2$ beträgt die Verzögerung in der Korrelation $\Delta t \propto \frac{m_1}{m_2}$, wobei die intrinsische Zeit $T \propto \frac{1}{m}$ ist.
	\end{theorem}
	
	\section{Neudenken von Anfang und Ende}
	Auch unsere Vorstellung vom Universum wird umgekehrt. Die herkömmliche Kosmologie beschreibt einen expandierenden Raum, in dem sich Galaxien voneinander entfernen, was wir als Rotverschiebung des Lichts beobachten. In Paschers alternativer Sicht ist der Raum statisch, und die Rotverschiebung resultiert aus einem Energieverlust des Lichts über die Zeit, ausgedrückt als Massenvariation.
	
	Der Urknall ist nicht der Beginn von Zeit und Raum, sondern ein Zustand extrem hoher Energie und Masse, der sich über die konstante Zeit entwickelt. Diese Sichtweise löst das Horizontproblem der Kosmologie eleganter als die Inflationstheorie und vermeidet die mathematischen Singularitäten, die die Standardtheorie plagen.
	
	Schwarze Löcher behalten in Paschers Modell eine endliche Struktur, ohne die problematische zentrale Singularität des Standardmodells. Der Ereignishorizont markiert eine Grenze extremer Massenvariation, nicht einen Punkt, an dem die Zeit endet. Dies steht im Einklang mit der Thermodynamik und vermeidet das Informationsparadoxon, das in der konventionellen Theorie auftritt.
	
	\section{Ein fundamentaler Baustein: Energie}
	In Paschers erweitertem Modell werden alle fundamentalen Konstanten der Natur - die Lichtgeschwindigkeit, das Plancksche Wirkungsquantum, die Gravitationskonstante - auf eine einzige fundamentale Größe zurückgeführt: Energie. Diese Vereinheitlichung ist nicht spekulativ, sondern mathematisch präzise formuliert und zeigt, dass die scheinbar unabhängigen Konstanten verschiedene Aspekte derselben zugrundeliegenden Realität sind.
	
	Während das Standardmodell der Physik diese Konstanten als gegeben annimmt, zeigt Paschers Ansatz, dass sie aus einfacheren Prinzipien herleitbar sind. Dies ist eine tiefgreifende Vereinfachung unserer Beschreibung der Naturgesetze, vergleichbar mit dem Übergang von der ptolemäischen zur kopernikanischen Astronomie.

	
	\section{Es auf die Probe stellen}
	Paschers erweiterte Quantenmechanik und Quantenfeldtheorie macht klare, überprüfbare Vorhersagen, die sich von denen des Standardmodells unterscheiden:
	
	\begin{itemize}
		\item Bell-Tests mit Teilchen unterschiedlicher Masse werden messbare Verzögerungen in den Korrelationen zeigen, proportional zum Massenverhältnis.
		\item In Systemen mit quantenmechanischer Kohärenz variieren die Kohärenzzeiten mit der Masse, was in Quanteninformationsexperimenten nachweisbar ist.
		\item Die modifizierte Version der Schrödinger-Gleichung mit intrinsischer Zeit führt zu unterschiedlichen Dispersionsrelationen für Materiewellen.
	\end{itemize}
	
	Diese Vorhersagen sind präzise formuliert und bieten klare Tests zwischen den Modellen, die mit heutiger oder naher zukünftiger Technologie durchführbar sind.
	
	\section{Eine neue Linse, ein klareres Bild}
	Paschers Ansatz invertiert unsere übliche Sichtweise, ohne dabei die experimentell bestätigten Gesetze der Physik zu ändern. Die mathematischen Gleichungen bleiben im Kern erhalten, werden aber in einem neuen Rahmen interpretiert und erweitert.
	
	Diese Inversion ist ähnlich dem Wechsel vom geozentrischen zum heliozentrischen Weltbild: Die beobachteten Bewegungen der Himmelskörper bleiben gleich, aber die zugrundeliegende Erklärung wird eleganter und tiefgründiger.
	
	Während die Standardtheorie der Physik weiterhin bemüht ist, Quantenmechanik und Gravitation zu vereinen, bietet Paschers Ansatz einen direkten Weg zu dieser Vereinheitlichung durch die konsistente Behandlung von Zeit und Masse.
	
	Die heutige Physik steht vor großen ungelösten Rätseln - dunkle Materie, dunkle Energie, das Informationsparadoxon schwarzer Löcher. Sowohl das Standardmodell als auch Paschers Theorie haben hier offene Fragen. Doch während das Standardmodell oft auf zusätzliche Annahmen und Korrekturen angewiesen ist, löst Paschers Ansatz viele dieser Probleme direkt durch seine fundamentalere Behandlung von Zeit, Masse und Energie.
	
	Die Geschichte der Wissenschaft lehrt uns, dass die tiefgreifendsten Fortschritte oft nicht durch mehr Daten, sondern durch neue Perspektiven entstehen. Paschers Arbeit erinnert uns daran, dass manchmal die wichtigsten Entdeckungen nicht aus neuen Beobachtungen kommen, sondern daraus, bekannte Fakten auf eine völlig neue Weise zu betrachten.
	
	\begin{thebibliography}{2}
		\bibitem{pascher2025} Pascher, J. (2025). \textit{Wesentliche mathematische Formalismen der Zeit-Masse-Dualitätstheorie mit Lagrange-Dichten}. 29. März 2025.
		\bibitem{einstein1905} Einstein, A. (1905). \textit{Zur Elektrodynamik bewegter Körper}. Annalen der Physik, 322(10), 891-921.
	\end{thebibliography}
	
\end{document}