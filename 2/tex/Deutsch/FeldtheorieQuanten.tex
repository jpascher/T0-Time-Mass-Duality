\documentclass[a4paper,12pt]{article}
\usepackage[utf8]{inputenc}
\usepackage[T1]{fontenc}
\usepackage{lmodern}
\usepackage[ngerman]{babel}
\usepackage{csquotes}
\usepackage{amsmath}
\usepackage{amsfonts}
\usepackage{amssymb}
\usepackage{physics}
\usepackage{geometry}
\usepackage{tocloft}
\usepackage{xcolor}
\usepackage{graphicx,tikz,pgfplots}
\pgfplotsset{compat=1.18}
\usepackage{booktabs}
\usepackage{array}
\usepackage{tabularx}
\usepackage{braket}
\usepackage{siunitx}
\DeclareSIUnit{\year}{yr}
\DeclareSIUnit{\parsec}{pc}
\usepackage{amsthm}
\usepackage[colorlinks=true, linkcolor=blue, citecolor=blue, urlcolor=blue]{hyperref}
\usepackage{cleveref}
\usepackage{fancyhdr}

\geometry{a4paper, margin=2cm}

\hypersetup{
	pdftitle={Feldtheorie und Quantenkorrelationen: Eine neue Perspektive auf Instantaneität},
	pdfauthor={Johann Pascher},
	pdfcreator={LaTeX}
}

% Kopf- und Fußzeilen
\pagestyle{fancy}
\fancyhf{}
\fancyhead[L]{Johann Pascher}
\fancyhead[R]{Zeit-Masse-Dualität}
\fancyfoot[C]{\thepage}
\renewcommand{\headrulewidth}{0.4pt}
\renewcommand{\footrulewidth}{0.4pt}

\renewcommand{\cftsecfont}{\color{blue}}
\renewcommand{\cftsubsecfont}{\color{blue}}
\renewcommand{\cftsecpagefont}{\color{blue}}
\renewcommand{\cftsubsecpagefont}{\color{blue}}
\setlength{\cftsecindent}{1cm}
\setlength{\cftsubsecindent}{2cm}

% Benutzerdefinierte Befehle
\newcommand{\Tfield}{T(x)}
\newcommand{\DcovT}[1]{\Tfield D_\mu #1 + #1 \partial_\mu \Tfield}
\newcommand{\DhiggsT}{\Tfield (\partial_\mu + ig A_\mu) \Phi + \Phi \partial_\mu \Tfield}
\newcommand{\betaT}{\beta_{\text{T}}}
\newcommand{\alphaEM}{\alpha_{\text{EM}}}
\newcommand{\Mpl}{M_{\text{Pl}}}
\newcommand{\Tzerot}{T_0(\Tfield)}
\newcommand{\Tzero}{T_0}
\newcommand{\vecx}{\vec{x}}
\newcommand{\gammaf}{\gamma_{\text{Lorentz}}}

\newtheorem{theorem}{Theorem}[section]
\newtheorem{proposition}[theorem]{Proposition}
\theoremstyle{definition}
\newtheorem{definition}{Definition}[theorem]
\theoremstyle{remark}
\newtheorem{remark}{Bemerkung}

\title{Feldtheorie und Quantenkorrelationen: \\Eine neue Perspektive auf Instantaneität}
\author{Johann Pascher}
\date{28. März 2025}

\begin{document}
	
	\maketitle
	
	\begin{abstract}
		Diese Arbeit entwickelt eine neue Perspektive auf Quantenkorrelationen und ihre scheinbare Instantaneität im Rahmen des T0-Modells. Durch einen vereinheitlichten Feldansatz wird gezeigt, wie die nichtlokalen Eigenschaften der Quantenmechanik als natürliche Folge einer zugrunde liegenden Feldstruktur verstanden werden können. Besonderer Wert wird auf die Rolle des Quantenhintergrunds und die Interpretation moderner Bell-Experimente gelegt. Diese Sichtweise ergänzt die Zeit-Masse-Dualitätstheorie und bietet einen kohärenten Rahmen zur Erklärung quantenmechanischer Phänomene, ohne eine ''spukhafte Fernwirkung'' anzunehmen.
	\end{abstract}
	
	\tableofcontents
	\newpage
	
	\section{Einleitung}
	
	Die Quantenmechanik stellt uns seit langem vor Rätsel, insbesondere hinsichtlich der Natur von Quantenkorrelationen. Die Vorstellung, dass zwei Teilchen, die durch große Entfernungen getrennt sind, augenblicklich verbunden sein könnten, fasziniert Wissenschaftler seit Einsteins berühmter Kritik an der ''spukhaften Fernwirkung''. Moderne Experimente, wie die lückenfreien Bell-Tests ab 2015, haben gezeigt, dass diese Korrelationen real sind und die klassischen Vorstellungen von Lokalität und Kausalität übertreffen. Aber wie können wir dieses Phänomen verstehen, ohne die Grundpfeiler der Physik aufzugeben?
	
	In dieser Arbeit schlage ich vor, die Nichtlokalität der Quantenwelt nicht als mysteriöse Instantaneität zu betrachten, sondern als natürliche Eigenschaft eines vereinheitlichten Quantenfelds, das im Rahmen des T0-Modells entwickelt wurde. Dieses Modell, das auf der Zeit-Masse-Dualität basiert und die Zeit als absolute Größe mit variabler Masse behandelt, bietet eine frische Perspektive. Anstatt Teilchen als isolierte Objekte zu sehen, interpretiere ich sie als Knoten oder Anregungen eines fundamentalen Feldes, dessen Kohärenz die beobachteten Korrelationen erklärt. Diese Sichtweise baut auf der Zeit-Masse-Dualitätsarbeit auf, die in ''Zeit-Masse-Dualitätstheorie: Herleitung der Parameter'' \cite{pascher_params_2025} detailliert beschrieben wird, und erweitert sie mit einer feldtheoretischen Grundlage. Ein zentrales Element ist das intrinsische Zeitfeld \(\Tfield = \frac{\hbar}{\max(m c^2, \omega)}\), das die Zeitskala der Feldknoten bestimmt und Quantenmechanik und Kosmologie überbrückt.
	
	Mein Ansatz beginnt mit der Idee, dass das Vakuum kein leerer Raum ist, sondern ein aktiver Quantenhintergrund mit definierten Eigenschaften, ausgedrückt durch die elektrischen und magnetischen Feldkonstanten \(\varepsilon_0\) und \(\mu_0\). Teilchen sind keine eigenständigen Entitäten, sondern stabile Muster dieses Feldes, während Quantenkorrelationen die inhärente Kohärenz des Feldzustands widerspiegeln. Moderne Experimente, wie die Wiener Tests von 2015 \cite{Giustina2015} oder der ''Big Bell Test'' von 2018 \cite{BigBellTest2018}, unterstützen diese Interpretation, indem sie die Nichtlokalität als Tatsache bestätigen, ohne eine instantane Kommunikation zwischen Teilchen zu erfordern. Um diese Perspektive mathematisch zu untermauern, führe ich eine fundamentale Feldgleichung ein, die die Dynamik dieses vereinheitlichten Feldes beschreibt und es mit der modifizierten Quantenmechanik des T0-Modells verbindet, wie sie in ''Die Notwendigkeit der Erweiterung der Standardquantenmechanik'' \cite{pascher_quantum_2025} entwickelt wurde. Diese Arbeit zielt darauf ab, eine kohärente Darstellung zu schaffen, die sowohl experimentelle Befunde als auch die theoretischen Grundlagen des T0-Modells vereint.
	
	\section{Das Vakuum als Quantenhintergrund}
	
	Das Vakuum, wie es in der modernen Physik verstanden wird, ist weit mehr als ein leerer Raum. Es ist ein dynamisches Medium, das durch fundamentale physikalische Eigenschaften wie die elektrische Feldkonstante \(\varepsilon_0\) und die magnetische Feldkonstante \(\mu_0\) charakterisiert wird. Diese Konstanten sind nicht bloße mathematische Werkzeuge, sondern Ausdruck einer tiefen Struktur, die die Lichtgeschwindigkeit definiert und die Wechselwirkungen aller Felder im Universum ermöglicht. Im T0-Modell wird dieses Vakuum als aktiver Quantenhintergrund betrachtet, der die Grundlage für alle physikalischen Phänomene bildet, einschließlich Quantenkorrelationen. Eine zentrale Beziehung, die diese Rolle unterstreicht, ist:
	
	\begin{equation}
		c = \frac{1}{\sqrt{\varepsilon_0 \mu_0}}
	\end{equation}
	
	Dieser Hintergrund ist keine passive Bühne, sondern ein Trägermedium, das elektromagnetische Wellen und andere fundamentale Felder ermöglicht. Seine Homogenität stellt sicher, dass die Lichtgeschwindigkeit konstant bleibt, wie es die spezielle Relativitätstheorie erfordert. Im T0-Modell geht die Bedeutung des Vakuums jedoch noch weiter: Es beeinflusst direkt das intrinsische Zeitfeld \(\Tfield\), das die Zeitskala jedes Teilchens bestimmt, wie in ''Zeit-Masse-Dualitätstheorie'' \cite{pascher_params_2025} gezeigt wird. Das Vakuum wird somit zum Schlüssel für das Verständnis der nichtlokalen Eigenschaften der Quantenwelt, indem es als kohärentes Medium wirkt, das Korrelationen über weite Entfernungen aufrechterhält.
	
	\section{Quantenkorrelationen im Feldmodell}
	
	Wenn wir von verschränkten Teilchen sprechen, denken wir oft an Photonen, deren Polarisation durch einen gemeinsamen Zustand beschrieben wird. Ein typischer verschränkter Zustand ist:
	
	\begin{equation}
		|\psi\rangle = \frac{1}{\sqrt{2}} (|H\rangle_A |H\rangle_B + |V\rangle_A |V\rangle_B)
	\end{equation}
	
	In der traditionellen Sichtweise erscheinen diese Teilchen als separate Objekte, deren Zustände sich bei der Messung instantan korrelieren. Im Feldansatz des T0-Modells ändert sich diese Perspektive jedoch grundlegend. Verschränkte Zustände sind keine Eigenschaften isolierter Teilchen, sondern kohärente Muster eines vereinheitlichten Quantenfeldes, das den Raum durchdringt, wie in ''Dynamische Masse von Photonen'' \cite{pascher_photons_2025} beschrieben.
	
	Die Bell-Ungleichungen, formuliert von John Bell im Jahr 1964, zeigen deutlich, dass lokal-realistische Theorien die beobachteten Korrelationen nicht erklären können. Mathematisch ausgedrückt lautet eine solche Ungleichung:
	
	\begin{equation}
		|E(a,b) - E(a,c)| \leq 1 + E(b,c)
	\end{equation}
	
	In Experimenten wird diese Ungleichung durchweg verletzt, wie von Alain Aspect 1982 und in nachfolgenden Tests \cite{Aspect1982} demonstriert. Im Feldmodell ist diese Verletzung nicht überraschend: Das Quantenfeld ist inhärent nichtlokal, weil es eine globale Struktur besitzt, die lokale Messungen miteinander verbindet, ohne dass eine Signalübertragung erforderlich ist.
	
	Die Wiener Experimente von 2015, geleitet von Anton Zeilinger, markierten einen Meilenstein, indem sie alle klassischen Schlupflöcher – wie Detektionseffizienz oder räumliche Trennung \cite{Giustina2015} – schlossen. Mit einer Signifikanz von mehr als 11 Standardabweichungen bestätigten sie die Nichtlokalität. Ebenso beeindruckend war der ''Big Bell Test'' von 2018, bei dem über 100.000 Menschen weltweit die Messeinstellungen kontrollierten, um die Wahlfreiheits-Schlupfloch anzusprechen \cite{BigBellTest2018}. Diese Experimente zeigen, dass Quantenkorrelationen real sind, und im T0-Feldmodell finden sie eine natürliche Erklärung als Eigenschaften eines kohärenten Quantenfeldes.
	
	\section{Feldtheorie und Instantaneität}
	
	Um das Konzept der Instantaneität greifbarer zu machen, ziehe ich eine Analogie zu Schallwellen. Wenn eine Schallwelle durch einen Raum wandert, ist sie überall präsent, doch ein Mikrofon misst nur die lokale Vibration. Zwei Mikrofone, die dieselbe Welle aufnehmen, zeigen eine Korrelation, nicht aufgrund einer instantanen Kommunikation zwischen ihnen, sondern aufgrund der gemeinsamen Struktur der Welle. Ähnlich sind im Quantenfeldmodell verschränkte Teilchen Knoten eines globalen Feldes, und ihre Korrelationen sind dem Feldzustand inhärent, bevor eine Messung stattfindet.
	
	Diese Analogie löst das Nichtlokalitätsparadoxon. Es gibt keine ''Wirkung aus der Ferne'', sondern eine inhärente Kohärenz des Feldes, die durch das intrinsische Zeitfeld \(\Tfield\) kontrolliert wird, wie in ''Dynamische Masse von Photonen'' \cite{pascher_photons_2025} beschrieben. Das Higgs-Feld spielt hier eine zentrale Rolle, indem es die Masse und damit die Zeitskala der Feldknoten definiert und die beobachteten Korrelationen mit der Zeit-Masse-Dualität verbindet.
	
	\section{Feldgleichungen in dualer Formulierung}
	
	Die Quantenmechanik im T0-Modell wird durch eine modifizierte Schrödinger-Gleichung beschrieben, die variable Masse einbezieht. Im Gegensatz zur klassischen Form \(i\hbar \frac{\partial}{\partial t} \Psi = \hat{H} \Psi\) wird sie hier ausgedrückt als:
	
	\begin{equation}
		i\hbar \Tfield \frac{\partial}{\partial t} \Psi + i\hbar \Psi \frac{\partial \Tfield}{\partial t} = \hat{H} \Psi
	\end{equation}
	
	Dieser Ansatz, der in ''Die Notwendigkeit der Erweiterung der Standardquantenmechanik'' \cite{pascher_quantum_2025} entwickelt wurde, spiegelt die Zeit-Masse-Dualität wider und integriert sie in die Feldtheorie. Die gesamte Lagrange-Dichte des Modells ist:
	\begin{equation}
		\mathcal{L}_{\text{Total}} = \mathcal{L}_{\text{Boson}} + \mathcal{L}_{\text{Fermion}} + \mathcal{L}_{\text{Higgs-T}} + \mathcal{L}_{\text{intrinsisch}}
	\end{equation}
	wobei \(\mathcal{L}_{\text{intrinsisch}} = \frac{1}{2} \partial_\mu \Tfield \partial^\mu \Tfield - V(\Tfield)\) die Dynamik des Zeitfeldes beschreibt, wie in ''Mathematische Kernformulierungen'' \cite{pascher_lagrange_2025} ausgeführt.
	
	\section{Kosmologische Implikationen}
	
	Das T0-Modell hat weitreichende Implikationen für die Kosmologie, die mit der Feldperspektive übereinstimmen. Das Gravitationspotential wird zu \(\Phi(r) = -\frac{G M}{r} + \kappa r\) modifiziert, wobei \(\kappa \approx \SI{4.8e-11}{\meter\per\second\squared}\) aus der \(\Tfield\)-Dynamik hervorgeht, wie in ''Massenvariation in Galaxien'' \cite{pascher_galaxies_2025} gezeigt. Die kosmische Rotverschiebung wird als Energieverlust beschrieben: \(1 + z = e^{\alpha d}\), mit \(\alpha \approx \SI{2.3e-18}{\per\meter}\), wie in ''Messdifferenzen'' \cite{pascher_messdifferenzen_2025} abgeleitet. Eine wellenlängenabhängige Rotverschiebung entsteht mit \(z(\lambda) = z_0 (1 + \betaT \ln(\lambda/\lambda_0))\), wobei \(\betaT^{\text{SI}} \approx 0.008\) und \(\betaT^{\text{nat}} = 1\), wie in ''Parameterableitungen'' \cite{pascher_params_2025} festgelegt. Diese Effekte zeigen, wie sich das Quantenfeldmodell nahtlos in die kosmologischen Aspekte des T0-Modells integriert.
	
	\section{Schlussfolgerung}
	
	Das T0-Modell bietet eine neue Perspektive auf Quantenkorrelationen, indem es sie als natürliche Eigenschaften eines vereinheitlichten Quantenfeldes erklärt, das durch das intrinsische Zeitfeld \(\Tfield\) kontrolliert wird. Diese Sichtweise löst das Instantanitätsparadoxon, indem sie Nichtlokalität als inhärente Kohärenz des Feldes interpretiert, unterstützt durch moderne Bell-Experimente wie die von Zeilinger und den ''Big Bell Test''. Durch die Integration mit der Zeit-Masse-Dualität, wie sie in den Arbeiten zur Quantenmechanik und Kosmologie des T0-Modells entwickelt wurde, entsteht ein kohärenter Rahmen, der die Grenzen zwischen Quantenphysik und Feldtheorie überwindet.
	
	\begin{thebibliography}{99}
		\bibitem{pascher_params_2025} Pascher, J. (2025). \href{https://github.com/jpascher/T0-Time-Mass-Duality/tree/main/2/pdf/Deutsch/ZeitMasseT0Params.pdf}{Zeit-Masse-Dualitätstheorie (T0-Modell): Herleitung der Parameter \(\kappa\), \(\alpha\) und \(\beta\)}. 4. April 2025.
		\bibitem{pascher_galaxies_2025} Pascher, J. (2025). \href{https://github.com/jpascher/T0-Time-Mass-Duality/tree/main/2/pdf/Deutsch/MassVarGalaxien.pdf}{Massenvariation in Galaxien: Eine Analyse im T0-Modell mit emergenter Gravitation}. 30. März 2025.
		\bibitem{pascher_messdifferenzen_2025} Pascher, J. (2025). \href{https://github.com/jpascher/T0-Time-Mass-Duality/tree/main/2/pdf/Deutsch/MessdifferenzenT0Standard.pdf}{Kompensatorische und additive Effekte: Eine Analyse der Messdifferenzen zwischen dem T0-Modell und dem \(\Lambda\)CDM-Standardmodell}. 2. April 2025.
		\bibitem{pascher_lagrange_2025} Pascher, J. (2025). \href{https://github.com/jpascher/T0-Time-Mass-Duality/tree/main/2/pdf/Deutsch/MathZeitMasseLagrange.pdf}{Von der Zeitdilatation zur Massenvariation: Mathematische Kernformulierungen der Zeit-Masse-Dualitätstheorie}. 29. März 2025.
		\bibitem{pascher_photons_2025} Pascher, J. (2025). \href{https://github.com/jpascher/T0-Time-Mass-Duality/tree/main/2/pdf/Deutsch/DynMassePhotonenNichtlokal.pdf}{Dynamische Masse von Photonen und ihre Implikationen für Nichtlokalität im T0-Modell}. 25. März 2025.
		\bibitem{pascher_quantum_2025} Pascher, J. (2025). \href{https://github.com/jpascher/T0-Time-Mass-Duality/tree/main/2/pdf/Deutsch/NotwendigkeitQMErweiterung.pdf}{Die Notwendigkeit der Erweiterung der Standardquantenmechanik und Quantenfeldtheorie}. 27. März 2025.
		\bibitem{Hensen2015} Hensen, B., et al. (2015). \textit{Schlupflochfreie Verletzung der Bell-Ungleichung mit 1,3 Kilometer voneinander entfernten Elektronenspins}. Nature, 526, 682-686.
		\bibitem{Giustina2015} Giustina, M., et al. (2015). \textit{Signifikanter schlupflochfreier Test von Bells Theorem mit verschränkten Photonen}. Physical Review Letters, 115, 250401.
		\bibitem{BigBellTest2018} The BIG Bell Test Collaboration. (2018). \textit{Herausforderung des lokalen Realismus durch menschliche Entscheidungen}. Nature, 557, 212-216.
		\bibitem{Bell1964} Bell, J. S. (1964). \textit{Über das Einstein-Podolsky-Rosen-Paradoxon}. Physics, 1(3), 195-200.
		\bibitem{Aspect1982} Aspect, A., et al. (1982). \textit{Experimenteller Test der Bell-Ungleichungen mit zeitvariierenden Analysatoren}. Physical Review Letters, 49, 1804-1807.
		\bibitem{Wilczek2008} Wilczek, F. (2008). \textit{Die Leichtigkeit des Seins: Masse, Äther und die Vereinheitlichung der Kräfte}. Basic Books.
		\bibitem{Milonni1994} Milonni, P. W. (1994). \textit{Das Quantenvakuum: Eine Einführung in die Quantenelektrodynamik}. Academic Press.
		\bibitem{Aitchison2004} Aitchison, I. J. R. (2004). \textit{Eine informelle Einführung in Eichfeldtheorien}. Cambridge University Press.
		\bibitem{Weinberg1995} Weinberg, S. (1995). \textit{Die Quantenfeldtheorie}. Cambridge University Press.
		\bibitem{Fox2006} Fox, M. (2006). \textit{Quantenoptik: Eine Einführung}. Oxford University Press.
		\bibitem{Zeilinger2010} Zeilinger, A. (2010). \textit{Tanz der Photonen: Von Einstein zur Quantenteleportation}. Farrar, Straus and Giroux.
		\bibitem{Bohm1980} Bohm, D. (1980). \textit{Ganzheit und implizite Ordnung}. Routledge.
	\end{thebibliography}
	
\end{document}