\documentclass{article}
\usepackage[utf8]{inputenc}
\usepackage[T1]{fontenc}
\usepackage{lmodern}
\usepackage[english]{babel}
\usepackage{amsmath,amssymb,physics,graphicx,xcolor,amsthm}
\usepackage{hyperref}
\usepackage{booktabs}
\usepackage{siunitx}
\usepackage{cleveref}
\usepackage{pgfplots}
\pgfplotsset{compat=1.18}
\usepackage{tikz}
\usetikzlibrary{intersections}
\usepgfplotslibrary{fillbetween}

% Custom commands
\newcommand{\Tfield}{T(x)}
\newcommand{\DcovT}[1]{\Tfield D_\mu #1 + #1 \partial_\mu \Tfield}
\newcommand{\DhiggsT}{\Tfield (\partial_\mu + igA_\mu)\Phi + \Phi \partial_\mu \Tfield}
\newcommand{\gammaf}{\gamma_{\text{Lorentz}}}

% Theorem styles
\newtheorem{theorem}{Theorem}[section]
\newtheorem{proposition}[theorem]{Proposition}
\newtheorem{corollary}[theorem]{Corollary}
\newtheorem{lemma}[theorem]{Lemma}
\theoremstyle{definition}
\newtheorem{definition}[theorem]{Definition}
\newtheorem{example}[theorem]{Example}
\theoremstyle{remark}
\newtheorem{remark}[theorem]{Remark}

% Hyperref configuration
\hypersetup{
	colorlinks=true,
	linkcolor=blue,
	urlcolor=blue,
	citecolor=red,
	pdftitle={From Time Dilation to Mass Variation: Mathematical Core Formulations of the Time-Mass Duality Theory},
	pdfauthor={Johann Pascher}
}

\title{From Time Dilation to Mass Variation: \\ Mathematical Core Formulations of the Time-Mass Duality Theory}
\author{Johann Pascher}
\date{March 29, 2025}

\begin{document}
	
	\maketitle
	
	\begin{abstract}
		This work presents the essential mathematical formulations of the time-mass duality theory, focusing on the fundamental equations and their physical interpretations. The theory establishes a duality between two complementary descriptions of reality: the standard picture with time dilation and constant rest mass, and an alternative picture with absolute time and variable mass. Central to this framework is the intrinsic time \( T = \hbar/mc^2 \), which establishes a direct link between mass and temporal evolution in quantum systems. The mathematical formulations include modified Lagrangian densities for the Higgs field, fermions, and gauge bosons, emphasizing their interactions and invariance properties. This document serves as a concise mathematical reference for the time-mass duality theory.
	\end{abstract}
	
	\tableofcontents
	\newpage
	
	\section{Introduction to Time-Mass Duality}
	The time-mass duality theory proposes an alternative framework:
	\begin{enumerate}
		\item Standard Picture: \( t' = \gammaf t \), \( m_0 = \text{const.} \)
		\item T0 Model: \( T_0 = \text{const.} \), \( m = \gammaf m_0 \)
	\end{enumerate}
	
	\subsection{Relation to the Standard Model}
	The theory extends the Standard Model with:
	\begin{enumerate}
		\item Intrinsic Time Field: \( \Tfield \)
		\item Higgs Field: \( \Phi \) with \( \DhiggsT \)
		\item Fermion Fields: \( \psi \) with Yukawa coupling
		\item Gauge Boson Fields: \( A_\mu \) with \( \Tfield^2 \)
	\end{enumerate}
	
	\section{Emergent Gravity from the Intrinsic Time Field}
	\begin{theorem}[Gravitational Emergence]
		Gravity arises from gradients of the intrinsic time field:
		\begin{equation}
			\nabla \Tfield = -\frac{\hbar}{m^2c^2} \nabla m \sim \nabla \Phi_g
		\end{equation}
	\end{theorem}
	
	\begin{proof}
		From \( \Tfield = \frac{\hbar}{mc^2} \):
		\begin{equation}
			\nabla \Tfield = -\frac{\hbar}{m^2c^2} \nabla m
		\end{equation}
		With \( m(\vec{r}) = m_0 (1 + \frac{\Phi_g}{c^2}) \):
		\begin{equation}
			\nabla m = \frac{m_0}{c^2} \nabla \Phi_g
		\end{equation}
		Thus:
		\begin{equation}
			\nabla \Tfield \approx -\frac{\hbar}{m_0 c^4} \nabla \Phi_g
		\end{equation}
	\end{proof}
	
	\section{Mathematical Foundations: Intrinsic Time}
	\begin{theorem}[Intrinsic Time]
		\begin{equation}
			T = \frac{\hbar}{mc^2}
		\end{equation}
	\end{theorem}
	
	\section{Modified Derivative Operators}
	\begin{definition}[Modified Covariant Derivative]
		\begin{equation}
			\DcovT{\Psi} = \Tfield D_\mu \Psi + \Psi \partial_\mu \Tfield
		\end{equation}
	\end{definition}
	
	\section{Modified Field Equations}
	\begin{theorem}[Modified Schrödinger Equation]
		\begin{equation}
			i\hbar \Tfield \frac{\partial}{\partial t} \Psi + i\hbar \Psi \frac{\partial \Tfield}{\partial t} = \hat{H} \Psi
		\end{equation}
	\end{theorem}
	
	\section{Modified Lagrangian Density for the Higgs Field}
	\begin{theorem}[Higgs Lagrangian Density]
		\begin{equation}
			\mathcal{L}_{\text{Higgs-T}} = (\DhiggsT)^\dagger (\DhiggsT) - \lambda(|\Phi|^2 - v^2)^2
		\end{equation}
	\end{theorem}
	
	\section{Modified Lagrangian Density for Fermions}
	\begin{theorem}[Fermion Lagrangian Density]
		\begin{equation}
			\mathcal{L}_{\text{Fermion}} = \bar{\psi} i \gamma^\mu \DcovT{\psi} - y \bar{\psi} \Phi \psi
		\end{equation}
	\end{theorem}
	
	\section{Modified Lagrangian Density for Gauge Bosons}
	\begin{theorem}[Gauge Boson Lagrangian Density]
		\begin{equation}
			\mathcal{L}_{\text{Boson}} = -\frac{1}{4} \Tfield^2 F_{\mu\nu} F^{\mu\nu}
		\end{equation}
	\end{theorem}
	
	\section{Complete Total Lagrangian Density}
	\begin{theorem}[Total Lagrangian Density]
		\begin{equation}
			\mathcal{L}_{\text{Total}} = \mathcal{L}_{\text{Boson}} + \mathcal{L}_{\text{Fermion}} + \mathcal{L}_{\text{Higgs-T}}
		\end{equation}
	\end{theorem}
	
	\section{Cosmological Implications}
	The theory has the following implications:
	\begin{itemize}
		\item Modified Gravitational Potential: \( \Phi(r) = -\frac{GM}{r} + \kappa r \)
		\item Cosmic Redshift: \( 1 + z = e^{\alpha r} \)
	\end{itemize}
	

	
\end{document}