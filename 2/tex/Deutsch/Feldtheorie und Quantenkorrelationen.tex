\documentclass[12pt,a4paper]{article}
\usepackage[utf8]{inputenc}
\usepackage[T1]{fontenc}
\usepackage[ngerman]{babel}
\usepackage{lmodern}
\usepackage{csquotes}
\usepackage{amsmath}
\usepackage{amsfonts}
\usepackage{amssymb}
\usepackage{physics}
\usepackage{geometry}
\usepackage{tocloft}
\usepackage{xcolor}
\usepackage{graphicx,tikz,pgfplots}
\pgfplotsset{compat=1.18}
\usepackage{booktabs}
\usepackage{array}
\usepackage{tabularx}
\usepackage{fancyhdr}
\usepackage{braket}
\usepackage{siunitx}
\usepackage{amsthm}
\usepackage[colorlinks=true, linkcolor=blue, citecolor=blue, urlcolor=blue]{hyperref}
\usepackage{cleveref}

\geometry{a4paper, margin=2.5cm}

\hypersetup{
	colorlinks=true,
	linkcolor=blue,
	filecolor=magenta,
	urlcolor=blue,
	pdftitle={Feldtheorie und Quantenkorrelationen: Eine neue Perspektive auf Instantanität},
	pdfauthor={Johann Pascher},
	pdfcreator={LaTeX}
}

\renewcommand{\cftsecfont}{\color{blue}}
\renewcommand{\cftsubsecfont}{\color{blue}}
\renewcommand{\cftsecpagefont}{\color{blue}}
\renewcommand{\cftsubsecpagefont}{\color{blue}}
\setlength{\cftsecindent}{1cm}
\setlength{\cftsubsecindent}{2cm}

\newcommand{\DhiggsT}{T(x) (\partial_\mu + igA_\mu)\Phi + \Phi \partial_\mu T(x)}
\newcommand{\Tfield}{T(x)}

\newtheorem{theorem}{Theorem}[section]
\newtheorem{proposition}[theorem]{Proposition}

\title{Feldtheorie und Quantenkorrelationen: \\ Eine neue Perspektive auf Instantanität}
\author{Johann Pascher}
\date{28. März 2025}

\begin{document}
	
	\maketitle
	
	\begin{abstract}
		Diese Arbeit entwickelt eine neue Perspektive auf das Phänomen der Quantenkorrelationen und deren scheinbare Instantanität. Durch die Einführung eines fundamentalen Feldansatzes wird gezeigt, wie die nicht-lokalen Eigenschaften der Quantenmechanik als natürliche Konsequenz einer zugrundeliegenden Feldstruktur verstanden werden können. Besondere Aufmerksamkeit wird dabei der Rolle des Quantenhintergrunds und der Interpretation moderner Bell-Experimente gewidmet. Diese Betrachtung ergänzt die Zeit-Masse-Dualitätstheorie und bietet einen konsistenten Rahmen für das Verständnis von Quantenphänomenen innerhalb eines umfassenden Feldkonzepts.
	\end{abstract}
	
	\tableofcontents
	\newpage
	
	\section{Einleitung}
	Die moderne Quantenphysik steht vor einer fundamentalen Herausforderung: Die scheinbare Instantanität von Quantenkorrelationen scheint unserer klassischen Vorstellung von Lokalität und Kausalität zu widersprechen. Seit den bahnbrechenden Bell-Experimenten, insbesondere den schlupflochfreien Tests seit 2015 \cite{Hensen2015}, wissen wir mit Sicherheit, dass die Quantenwelt nicht-lokale Eigenschaften aufweist. Dennoch bleibt die Frage nach der \textit{Natur} dieser Nicht-Lokalität und ihrer Vereinbarkeit mit der Relativitätstheorie offen.
	
	\subsection{Ein neuer Ansatz}
	Diese Arbeit entwickelt eine alternative Perspektive auf das Problem der Quantenkorrelationen, indem sie einen fundamentalen Feldansatz vorschlägt. Statt separater Quantenfelder wird ein einheitliches Grundfeld postuliert, in dem Teilchen als Feldknoten und Quantenkorrelationen als Feldeigenschaften erscheinen \cite{Wilczek2008}. Diese Sichtweise ermöglicht es, die scheinbare `spukhafte Fernwirkung' als natürliche Konsequenz der Feldstruktur zu verstehen.
	
	\subsection{Theoretische Grundlagen}
	\begin{theorem}[Feldkonzept]
		Der vorgeschlagene Ansatz basiert auf:
		\begin{itemize}
			\item Das Vakuum als aktiver Quantenhintergrund mit definierten Eigenschaften (\(\varepsilon_0\), \(\mu_0\)).
			\item Teilchen als stabile Knoten oder Anregungsmuster im fundamentalen Feld.
			\item Quantenkorrelationen als inhärente Eigenschaften der Feldkohärenz.
		\end{itemize}
	\end{theorem}
	Diese Konzepte stehen in direkter Verbindung mit der Zeit-Masse-Dualitätstheorie \cite{Pascher2024}, bei der das intrinsische Zeitfeld \(T(x) = \hbar/mc^2\) als fundamentale Größe betrachtet wird. Die Eigenschaften des Quantenhintergrunds bestimmen dabei sowohl die Zeitentwicklung des Systems als auch die Struktur der Feldknoten, die als Materie wahrgenommen werden.
	
	\subsection{Experimentelle Evidenz}
	Die Theorie wird durch moderne Experimente gestützt, insbesondere:
	\begin{itemize}
		\item Die Wien-Experimente von 2015, die alle klassischen Schlupflöcher schlossen \cite{Giustina2015}.
		\item Den `Big Bell Test' von 2018 mit seiner einzigartigen Methodik \cite{BigBellTest2018}.
		\item Verschiedene Analogien zu klassischen Feldphänomenen.
	\end{itemize}
	
	\subsection{Mathematischer Rahmen}
	Die grundlegende Feldgleichung kann geschrieben werden als:
	\begin{equation}
		\Box \Psi + V(\Psi) = 0
	\end{equation}
	wobei \(\Box = \frac{\partial^2}{\partial t^2} - c^2 \nabla^2\) der d'Alembert-Operator ist und \(V(\Psi)\) ein Potentialterm, der die Stabilität der Feldknoten gewährleistet. In Verbindung mit der Zeit-Masse-Dualität lässt sich diese Gleichung umformulieren zu:
	\begin{equation}
		\left(\frac{\partial^2}{\partial(t/T(x))^2} - \nabla^2 + m_H^2\right) h_T(x) = 0
	\end{equation}
	für Skalarfelder wie das Higgs-Feld, und für Fermionen gemäß der modifizierten Dirac-Gleichung:
	\begin{equation}
		\left(i\gamma^0\frac{\partial}{\partial(t/T(x))} + i\gamma^i\partial_i - m_f\right) \psi_T(x) = 0
	\end{equation}
	wobei die modifizierte Zeitableitung \(\partial_{t/T} = \frac{\partial}{\partial(t/T(x))} = T(x)\frac{\partial}{\partial t}\) verwendet wird \cite{Pascher2024}. Dies führt zu einer modifizierten Dispersionsrelation:
	\begin{equation}
		\omega_T^2 = \mathbf{k}^2 + \frac{m_H^2 c^4}{\hbar^2} \cdot T(x)^2
	\end{equation}
	
	\section{Das Vakuum als Quantenhintergrund}
	Das Vakuum ist nicht einfach 'nichts', sondern ein aktiver Quantenhintergrund mit definierten physikalischen Eigenschaften \cite{Milonni1994}.
	
	\subsection{Fundamentale Konstanten des Vakuums}
	Die elektrische Feldkonstante (\(\varepsilon_0\)) und die magnetische Feldkonstante (\(\mu_0\)) charakterisieren die fundamentalen Eigenschaften des Vakuums als Quantenhintergrund. Sie bestimmen die Wechselwirkungen im elektromagnetischen Feld und stehen in direkter Beziehung zur Lichtgeschwindigkeit:
	\begin{equation}
		c = \frac{1}{\sqrt{\varepsilon_0 \mu_0}}
	\end{equation}
	Diese Konstanten sind nicht nur mathematische Größen, sondern Ausdruck der physikalischen Struktur des Quantenhintergrunds \cite{Aitchison2004}. In der Zeit-Masse-Dualitätstheorie beeinflussen sie direkt das intrinsische Zeitfeld \(T(x)\) und damit die Energieskala der Feldknoten \cite{Pascher2024}.
	
	\subsection{Das Vakuum als Träger des Feldes}
	Der Quantenhintergrund fungiert als Trägermedium für das elektromagnetische Feld und alle anderen fundamentalen Felder. Diese Betrachtungsweise ermöglicht es:
	\begin{itemize}
		\item Die Ausbreitung von Wellen im 'leeren' Raum zu erklären.
		\item Nicht-lokale Korrelationen als feldinhärente Eigenschaften zu verstehen.
		\item Die Grenzen klassischer Teilchenvorstellungen zu überwinden.
	\end{itemize}
	Die Homogenität des Vakuums und seiner Eigenschaften (\(\varepsilon_0\), \(\mu_0\)) ist dabei entscheidend für die Konstanz der Lichtgeschwindigkeit und damit für die Gültigkeit der speziellen Relativitätstheorie \cite{Weinberg1995}.
	
	\section{Quantenkorrelationen im Feldmodell}
	\subsection{Polarisation und Verschränkung}
	Die Polarisation eines Photons kann als Superposition von horizontaler (H) und vertikaler (V) Polarisation beschrieben werden \cite{Fox2006}:
	\begin{equation}
		|\psi\rangle = \alpha |H\rangle + \beta e^{i\phi} |V\rangle
	\end{equation}
	Bei verschränkten Photonenpaaren entsteht ein gemeinsamer Zustand wie:
	\begin{equation}
		|\psi\rangle = \frac{1}{\sqrt{2}} (|H\rangle_A |H\rangle_B + |V\rangle_A |V\rangle_B)
	\end{equation}
	Im Feldmodell werden diese Zustände nicht als isolierte Teilcheneigenschaften betrachtet, sondern als kohärente Feldmuster, die über den gesamten Raum ausgedehnt sind \cite{Zeilinger2010}. Die Korrelation zwischen den Messungen an Teilchen A und B ist eine inhärente Eigenschaft dieses Feldmusters, nicht das Ergebnis einer instantanen 'Kommunikation' zwischen den Teilchen.
	
	\subsection{Bell'sche Ungleichungen und lokaler Realismus}
	Die Bell'schen Ungleichungen und ihre experimentelle Verletzung zeigen die Grenzen lokaler realistischer Theorien \cite{Bell1964}. Im Feld-Modell lässt sich dies verstehen als:
	\begin{equation}
		|E(a,b) - E(a,c)| \leq 1 + E(b,c)
	\end{equation}
	Die experimentell beobachtete Verletzung dieser Ungleichung (für bestimmte Winkel a, b, c) demonstriert, dass die Quantenwelt nicht durch lokale versteckte Variablen beschrieben werden kann \cite{Aspect1982}. Das Feld-Modell liefert hierfür eine natürliche Erklärung: Das fundamentale Feld ist in seiner Natur nicht-lokal, da es den gesamten Raum durchdringt.
	
	\subsection{Das Wien-Experiment von 2015}
	Das Experiment der Gruppe um Anton Zeilinger in Wien 2015 war eines der ersten wirklich schlupflochfreien Tests von Bell's Theorem \cite{Giustina2015}. Es kombinierte:
	\begin{itemize}
		\item Sehr hohe Detektionseffizienz (>97\% durch SNSPDs).
		\item Ausreichende räumliche Trennung der Messungen.
		\item Schnelle, unabhängige Quantenzufallsgeneratoren.
	\end{itemize}
	Die beobachtete Verletzung der Bell-Ungleichung mit einer statistischen Signifikanz von 11.5 Standardabweichungen bestätigt die Nicht-Lokalität der Quantenwelt. Im Feldmodell ist diese Nicht-Lokalität eine natürliche Eigenschaft des fundamentalen Feldes und seiner kohärenten Struktur \cite{Zeilinger2010}.
	
	\subsection{Der "Big Bell Test" von 2018}
	Der "Big Bell Test" nutzte die Entscheidungen von über 100.000 Menschen weltweit, um die Messeinstellungen in 13 verschiedenen Laboren zu steuern \cite{BigBellTest2018}. Diese menschliche Komponente adressierte das Freie-Wahl-Schlupfloch auf eine neue Art. Die Ergebnisse zeigten eine Verletzung der Bell-Ungleichungen mit statistischen Signifikanzen von bis zu 70 Standardabweichungen.
	Diese Experimente bestätigen die nicht-lokale Natur der Quantenwelt, was im Feldmodell als Ausdruck der kohärenten Feldstruktur verstanden werden kann.
	
	\section{Feldtheorie und Instantanität}
	\subsection{Schallwellen als Analogie}
	Schallwellen bieten eine hilfreiche Analogie zum Verständnis des Feldkonzepts \cite{Bohm1980}:
	\begin{itemize}
		\item Schall existiert als Druckwelle, die den gesamten Raum durchdringt.
		\item Ein Mikrofon misst lokal die Schwingung, aber die Welle selbst ist global präsent.
		\item Die Gleichzeitigkeit der Messung an verschiedenen Mikrofonen ergibt sich aus der kohärenten Struktur der Schallwelle.
	\end{itemize}
	Im Feldmodell sind verschränkte Teilchen wie Knoten in einem globalen Quantenfeld. Die Korrelationen zwischen ihnen sind keine 'Fernwirkung', sondern vorhandene Eigenschaften des Feldes, die bei der Messung nur lokal abgetastet werden. Das Higgs-Feld spielt dabei eine besondere Rolle als universelles Medium, das nicht nur Masse vermittelt, sondern auch die intrinsische Zeitskala aller Teilchen bestimmt, wie in der Zeit-Masse-Dualitätstheorie beschrieben \cite{Pascher2024}.
	
	\subsection{Warum ist diese Analogie wichtig?}
	\subsubsection{Auflösung des Paradoxons}
	Die Nicht-Lokalität erscheint nur paradox, wenn man Teilchen als getrennte Objekte betrachtet. Im Feldmodell sind sie Teile eines Ganzen – wie Schallwellenpunkte in einem Raum \cite{Bohm1980}.
	
	\subsubsection{Realität des Feldes}
	Das Quantenfeld ist keine Abstraktion, sondern die fundamentale Entität \cite{Weinberg1995}. Seine Eigenschaften (Kohärenz, Nicht-Lokalität) sind so real wie die einer Schallwelle.
	
	\subsubsection{Experimentelle Konsequenz}
	Wenn Alice und Bob verschränkte Photonen messen, 'hören' sie quasi zwei Mikrofone ab, die dieselbe Schallwelle abtasten. Die Korrelationen sind im Feld bereits enthalten, nicht erst bei der Messung erzeugt \cite{Zeilinger2010}.
	
	Im Rahmen der Zeit-Masse-Dualitätstheorie erhalten diese Korrelationen eine zusätzliche zeitliche Dimension: Das intrinsische Zeitfeld \(\Tfield = \hbar/mc^2\) bestimmt die Zeitskala der Feldkorrelationen und liefert eine natürliche Erklärung für die beobachteten Kohärenzzeiten und ihre Abhängigkeit von der Masse \cite{Pascher2024}. Die modifizierte Higgs-Yukawa-Kopplung
	\begin{equation}
		\mathcal{L}_{\text{Yukawa-T}} = -y_f \bar{\psi}_L \Phi \psi_R + \text{h.c.}
	\end{equation}
	definiert dabei nicht nur die Masse der Teilchen, sondern gleichzeitig auch ihre intrinsische Zeitskala gemäß
	\begin{equation}
		\Tfield = \frac{\hbar}{m(x) c^2} = \frac{\hbar \sqrt{2}}{y_f v c^2}
	\end{equation}
	Diese Beziehung stellt eine fundamentale Verbindung zwischen dem Higgs-Mechanismus und der Quantenkohärenz her.
	
	\section{Feldgleichungen in dualer Formulierung}
	\subsection{Modifizierte Quantenmechanik mit variabler Masse}
	Im Gegensatz zur konventionellen Schrödinger-Gleichung:
	\begin{equation}
		i\hbar \frac{\partial}{\partial t}\Psi(x,t) = \hat{H}\Psi(x,t)
	\end{equation}
	in der die Zeit (\(t\)) als externer, klassischer Parameter behandelt wird und die Masse als konstant gilt, führt die Zeit-Masse-Dualität zu einer fundamentalen Modifikation:
	\begin{equation}
		i\hbar \frac{\partial}{\partial (t/\Tfield)}\Psi = \hat{H}\Psi
	\end{equation}
	Diese modifizierte Gleichung verwendet das intrinsische Zeitfeld \(\Tfield = \hbar/mc^2\) und berücksichtigt, dass die Zeitentwicklung nicht mehr einheitlich für alle Objekte ist, sondern von deren Masse abhängt. Für Systeme mit variabler Masse kann dies umgeschrieben werden als:
	\begin{equation}
		i\hbar \frac{\partial}{\partial t}\Psi(x,t) = \hat{H}(m(t))\Psi(x,t)
	\end{equation}
	Bei Mehrteilchensystemen mit unterschiedlichen Massen, wie sie bei verschränkten Systemen auftreten, nimmt die modifizierte Gleichung folgende Form an:
	\begin{equation}
		i (m_1 + m_2) c^2 \frac{\partial}{\partial t} \Psi(x_1, x_2, t) = \hat{H} \Psi(x_1, x_2, t)
	\end{equation}
	Diese Erweiterung hat tiefgreifende Auswirkungen auf unser Verständnis von Quantenkorrelationen und der scheinbaren Instantanität bei verschränkten Zuständen \cite{Pascher2024}.
	
	\subsection{Klein-Gordon-Gleichung und Higgs-Feld}
	Die Standard-Klein-Gordon-Gleichung für das Higgs-Boson lautet:
	\begin{equation}
		(\Box + m_H^2) h(x) = 0
	\end{equation}
	Im Zeit-Masse-Dualitätsbild wird sie zu:
	\begin{equation}
		\left(\frac{\partial^2}{\partial(t/\Tfield)^2} - \nabla^2 + m_H^2\right) h_T(x) = 0
	\end{equation}
	Dies führt zu einer modifizierten Dispersionsrelation:
	\begin{equation}
		\omega_T^2 = \mathbf{k}^2 + \frac{m_H^2 c^4}{\hbar^2} \cdot T(x)^2
	\end{equation}
	
	\subsection{Dirac-Gleichung für Fermionen}
	Die Dirac-Gleichung für Fermionen im Standardmodell:
	\begin{equation}
		(i\gamma^\mu\partial_\mu - m_f) \psi(x) = 0
	\end{equation}
	wird im Zeit-Masse-Dualitätsbild zu:
	\begin{equation}
		\left(i\gamma^0\frac{\partial}{\partial(t/\Tfield)} + i\gamma^i\partial_i - m_f\right) \psi_T(x) = 0
	\end{equation}
	
	\subsection{Variable Masse als verborgene Variable}
	Eine besonders faszinierende Konsequenz dieser Betrachtung ist, dass die variable Masse als fundamentale verborgene Variable dienen könnte, die den scheinbaren Indeterminismus der Quantenmechanik erklären kann. Im Gegensatz zu klassischen verborgenen Variablen-Theorien, die durch Bell'sche Experimente weitgehend ausgeschlossen wurden, ist die variable Masse in der Zeit-Masse-Dualität eine fundamentale, bereits in der Physik verankerte Größe.
	Die modifizierte Lagrange-Dichte, die diesen Ansatz formalisiert, lautet:
	\begin{equation}
		\mathcal{L}_\text{gesamt} = \mathcal{L}_\text{Standard} + \mathcal{L}_\text{intrinsisch}
	\end{equation}
	wobei der zusätzliche Term das intrinsische Zeitfeld berücksichtigt:
	\begin{equation}
		\mathcal{L}_\text{intrinsisch} = \bar{\psi}\left(i\hbar\gamma^0 \frac{\partial}{\partial (t/\Tfield)} - i\hbar\gamma^0 \frac{\partial}{\partial t}\right)\psi
	\end{equation}
	Diese Erweiterung könnte den Weg zu einer deterministischen Quantentheorie ebnen, die Einsteins Intuition bestätigt, dass 'Gott nicht würfelt' \cite{Pascher2024}.
	
	\section{Feldtheorie und Lokalität}
	\subsection{Lokale-realistische Modelle und versteckte Variablen}
	In einer lokal-realistischen Theorie müssten die Messergebnisse durch lokale versteckte Variablen \(\lambda\) bestimmt sein \cite{Bell1964}. Die Korrelationsfunktion wäre dann:
	\begin{equation}
		E(a,b) = \int A(a,\lambda)B(b,\lambda)\rho(\lambda)d\lambda
	\end{equation}
	Die Bell'schen Experimente zeigen jedoch, dass diese Annahme die beobachteten Korrelationen nicht erklären kann \cite{Aspect1982}.
	
	\subsection{Das Wellenfeld-Modell als Alternative}
	Das Wellenfeld-Modell bietet eine Alternative, die die Quantenkorrelationen ohne Verletzung der Lokalität erklären kann \cite{Bohm1980}:
	\begin{itemize}
		\item Das verschränkte System wird als ein einheitliches, kohärentes Feld betrachtet.
		\item Die Messungen an verschiedenen Orten sind lokale Abtastungen dieses globalen Feldes.
		\item Die Korrelationen sind inhärente Eigenschaften des Feldes, nicht das Ergebnis einer instantanen Kommunikation.
	\end{itemize}
	Dieses Modell ist konsistent mit der Zeit-Masse-Dualitätstheorie, da beide Ansätze das fundamentale Feld und seine inhärenten Korrelationen als primäre Realität betrachten \cite{Pascher2024}. Das Higgs-Feld kann in diesem Zusammenhang als universelles Medium verstanden werden, das durch die modifizierte Lagrange-Dichte
	\begin{equation}
		\mathcal{L}_{\text{Higgs-T}} = (\DhiggsT)^\dagger (\DhiggsT) - \lambda(|\Phi|^2 - v^2)^2
	\end{equation}
	beschrieben wird. Die modifizierte kovariante Ableitung
	\begin{equation}
		\DhiggsT = T(x) (\partial_\mu + igA_\mu)\Phi + \Phi \partial_\mu T(x)
	\end{equation}
	enthält dabei die zeitliche Modifikation bezüglich des intrinsischen Zeitfelds, wodurch das Higgs-Feld zum zentralen Vermittler zwischen Raum, Zeit und Masse wird.
	
	\subsection{Feldkohärenz und Nicht-Lokalität}
	Die Nicht-Lokalität wird im Feldmodell nicht als 'spukhafte Fernwirkung' interpretiert, sondern als Ausdruck der Kohärenz eines ausgedehnten Systems \cite{Zeilinger2010}:
	\begin{itemize}
		\item Das verschränkte Wellenfeld verbindet die Messorte direkt durch seine kohärenten Eigenschaften.
		\item Die scheinbare 'Kommunikation' ist in diesem Modell keine Übertragung von Informationen, sondern das Ergebnis eines gemeinsamen Feldzustands.
	\end{itemize}
	Die Zeit-Masse-Dualitätstheorie ergänzt dieses Bild, indem sie das intrinsische Zeitfeld \(\Tfield\) als charakteristische Zeitskala der Feldkorrelationen identifiziert. Dies erklärt die beobachteten Kohärenzzeiten und ihre Abhängigkeit von der Masse \cite{Pascher2024}. Für Quantensysteme unterschiedlicher Masse sollten die Kohärenzzeiten \(\tau_1\) und \(\tau_2\) zweier ansonsten identischer Quantensysteme mit Massen \(m_1\) und \(m_2\) dem Verhältnis folgen:
	\begin{equation}
		\frac{\tau_1}{\tau_2} = \frac{m_2}{m_1}
	\end{equation}
	Diese Vorhersage könnte durch Präzisionsexperimente mit verschränkten Teilchen unterschiedlicher Masse überprüft werden und würde eine direkte Verbindung zwischen der klassischen Nichtlokalität und der masseabhängigen Zeitskala des Quantensystems herstellen.
	
	\section{Feldtheorie und Relativitätstheorie}
	\subsection{Das Wellenfeld-Modell und die Relativitätstheorie}
	Das Wellenfeld-Modell lässt sich in Einklang mit der Relativitätstheorie bringen \cite{Maudlin2011}:
	\begin{itemize}
		\item Keine echte Informationsübertragung: Die Korrelationen zwischen verschränkten Teilchen entstehen nicht durch Signalübertragung.
		\item Lokalität der Messergebnisse: Jeder Messwert wird lokal bestimmt, auch wenn die Ergebnisse global korreliert sind.
		\item Ausgedehntes Feld: Das Feld erstreckt sich über den gesamten Raum, aber es ist durch die Lichtkegelstruktur der Raumzeit begrenzt.
	\end{itemize}
	
	\subsection{Relativistische Quantenfeldtheorie und variable Masse}
	Das Wellenfeld-Modell ist auch mit der relativistischen Quantenfeldtheorie kompatibel \cite{Weinberg1995}:
	\begin{itemize}
		\item Verschränkung in der Raumzeit: Die Quantenkorrelationen sind in der relativistischen Quantenfeldtheorie vollständig beschreibbar.
		\item Keine absolute Referenzzeit: Die Relativitätstheorie verlangt, dass es keine bevorzugte Zeitkoordinate gibt. Das Wellenfeld-Modell erfordert keine solche Annahme.
	\end{itemize}
	Diese Betrachtungen zeigen, dass die Konzepte von Lokalität und Realismus neu interpretiert werden müssen, ohne fundamentale Prinzipien wie die Relativitätstheorie zu verletzen. Die Zeit-Masse-Dualitätstheorie bietet hier einen vielversprechenden Ansatz, indem sie zwei komplementäre Sichtweisen verbindet:
	\begin{itemize}
		\item Das Standardmodell mit konstanter Masse und Zeitdilatation (der übliche relativistische Rahmen).
		\item Das komplementäre Modell mit absoluter Zeit und variabler Masse.
	\end{itemize}
	Diese Dualität ermöglicht eine neue Interpretation relativistischer Phänomene und quantenmechanischer Prozesse, wobei das Higgs-Feld als Vermittler zwischen beiden Beschreibungen fungiert. Die modifizierte Dispersionsrelation
	\begin{equation}
		\omega_T^2 = \mathbf{k}^2 + \frac{m_H^2 c^4}{\hbar^2} \cdot T(x)^2
	\end{equation}
	zeigt, wie sich diese Dualität in der Wellenausbreitung manifestiert \cite{Pascher2024}.
	
	\subsection{Das Paradoxon der Instantanität und die Rolle der variablen Masse}
	Die scheinbare Instantanität von Quantenkorrelationen bleibt eines der hartnäckigsten Paradoxa der Quantenmechanik. Die Idee, dass Messungen an verschränkten Teilchen, unabhängig von ihrer räumlichen Trennung, instantan korrelierte Ergebnisse liefern, scheint der Relativitätstheorie zu widersprechen, die keine überlichtschnelle Informationsübertragung erlaubt.
	Die Zeit-Masse-Dualitätstheorie bietet einen eleganten Lösungsansatz für dieses Paradoxon:
	\begin{itemize}
		\item In der konventionellen Betrachtung erscheinen Quantenkorrelationen als instantan, weil wir Zeit als universellen, konstanten Parameter betrachten.
		\item Im Rahmen der Zeit-Masse-Dualität mit variabler Masse wird jedoch erkennbar, dass das masseabhängige intrinsische Zeitfeld \(\Tfield = \hbar/mc^2\) eine fundamentalere Zeitskala darstellt als die externe Laborzeit.
		\item Die Korrelationen erfolgen nicht instantan in der intrinsischen Zeit des Systems, sondern folgen einer Dynamik, die durch die modifizierte Schrödinger-Gleichung
		\begin{equation}
			i\hbar \frac{\partial}{\partial (t/\Tfield)}\Psi = \hat{H}\Psi
		\end{equation}
		beschrieben wird.
	\end{itemize}
	Diese Reformulierung löst das Paradoxon auf natürliche Weise: Was als instantane Wirkung in der Laborzeit erscheint, ist tatsächlich ein Prozess, der in der intrinsischen Zeit des Quantensystems abläuft, wobei die variable Masse als verborgene Variable fungiert \cite{Pascher2024}. Die modifizierte Lagrange-Dichte
	\begin{equation}
		\mathcal{L}_\text{intrinsisch} = \bar{\psi}\left(i\hbar\gamma^0 \frac{\partial}{\partial (t/\Tfield)} - i\hbar\gamma^0 \frac{\partial}{\partial t}\right)\psi
	\end{equation}
	erfasst formal die Diskrepanz zwischen der Zeitentwicklung in der absoluten Zeit und der masseabhängigen intrinsischen Zeit.
	
	\begin{figure}[h]
		\centering
		\begin{tikzpicture}
			\begin{axis}[
				xlabel={Masse [eV]},
				ylabel={Kohärenzzeit [eV\(^{-1}\)]},
				xlabel style={font=\large},
				ylabel style={font=\large},
				tick label style={font=\normalsize},
				xmin=0, xmax=1000,
				ymin=0, ymax=0.01,
				legend pos=north east,
				legend style={font=\large},
				grid=both,
				minor tick num=1
				]
				\addplot[blue, ultra thick, domain=1:1000, samples=100] {1/x};
				\legend{\(\tau \propto 1/m\)}
			\end{axis}
		\end{tikzpicture}
		\caption{Masseabhängige Kohärenzzeit im Feldmodell.}
	\end{figure}
	
	\bibliographystyle{plainnat}
	\begin{thebibliography}{99}
		\bibitem{Aitchison2004} Aitchison, I. J. R. (2004). \textit{An informal introduction to gauge field theories}. Cambridge University Press.
		\bibitem{Aspect1982} Aspect, A., Grangier, P., \& Roger, G. (1982). \textit{Experimental realization of Einstein-Podolsky-Rosen-Bohm Gedankenexperiment: A new violation of Bell's inequalities}. Physical Review Letters, 49(2), 91-94.
		\bibitem{Bell1964} Bell, J. S. (1964). \textit{On the Einstein Podolsky Rosen paradox}. Physics Physique Fizika, 1(3), 195-200.
		\bibitem{BigBellTest2018} BIG Bell Test Collaboration. (2018). \textit{Challenging local realism with human choices}. Nature, 557(7704), 212-216.
		\bibitem{Bohm1980} Bohm, D. (1980). \textit{Wholeness and the Implicate Order}. Routledge.
		\bibitem{Feynman1965} Feynman, R. P. (1965). \textit{The Character of Physical Law}. MIT Press.
		\bibitem{Fox2006} Fox, M. (2006). \textit{Quantum Optics: An Introduction}. Oxford University Press.
		\bibitem{Giustina2015} Giustina, M., et al. (2015). \textit{Significant-loophole-free test of Bell's theorem with entangled photons}. Physical Review Letters, 115(25), 250401.
		\bibitem{Handsteiner2017} Handsteiner, J., et al. (2017). \textit{Cosmic Bell test: measurement settings from Milky Way stars}. Physical Review Letters, 118(6), 060401.
		\bibitem{Hensen2015} Hensen, B., et al. (2015). \textit{Loophole-free Bell inequality violation using electron spins separated by 1.3 kilometres}. Nature, 526(7575), 682-686.
		\bibitem{Jozsa2000} Jozsa, R., et al. (2000). \textit{Quantum clock synchronization based on shared prior entanglement}. Physical Review Letters, 85(9), 2010-2013.
		\bibitem{Maudlin2011} Maudlin, T. (2011). \textit{Quantum Non-Locality and Relativity: Metaphysical Intimations of Modern Physics}. John Wiley \& Sons.
		\bibitem{Milonni1994} Milonni, P. W. (1994). \textit{The Quantum Vacuum: An Introduction to Quantum Electrodynamics}. Academic Press.
		\bibitem{Pascher2024a} Pascher, J. (2024). \textit{Zeit-Masse-Dualität: Ein neuer Ansatz zur Vereinheitlichung fundamentaler Kräfte}.
		\bibitem{Pascher2024b} Pascher, J. (2024). \textit{Mathematische Formulierung des Higgs-Mechanismus in der Zeit-Masse-Dualität}.
		\bibitem{Pascher2024c} Pascher, J. (2024). \textit{Die Notwendigkeit einer Erweiterung der Standard-Quantenmechanik und Quantenfeldtheorie}.
		\bibitem{Pascher2024} Pascher, J. (2024). \textit{Wesentliche mathematische Formalismen der Zeit-Masse-Dualitätstheorie mit Lagrange-Dichten}. 29. März 2025.
		\bibitem{Schlosshauer2013} Schlosshauer, M. (2013). \textit{Elegance and Enigma: The Quantum Interviews}. Springer.
		\bibitem{Shimony2017} Shimony, A. (2017). \textit{Bell's theorem}. In The Stanford Encyclopedia of Philosophy (Fall 2017 Edition), Edward N. Zalta (ed.).
		\bibitem{Wallace2012} Wallace, D. (2012). \textit{The Emergent Multiverse: Quantum Theory according to the Everett Interpretation}. Oxford University Press.
		\bibitem{Weinberg1995} Weinberg, S. (1995). \textit{The Quantum Theory of Fields, Volume 1: Foundations}. Cambridge University Press.
		\bibitem{Wilczek2008} Wilczek, F. (2008). \textit{The Lightness of Being: Mass, Ether, and the Unification of Forces}. Basic Books.
		\bibitem{Yin2017} Yin, J., et al. (2017). \textit{Satellite-based entanglement distribution over 1200 kilometers}. Science, 356(6343), 1140-1144.
		\bibitem{Zeilinger2010} Zeilinger, A. (2010). \textit{Dance of the Photons: From Einstein to Quantum Teleportation}. Farrar, Straus and Giroux.
	\end{thebibliography}
	
\end{document}