\documentclass[12pt,a4paper]{article}
\usepackage[utf8]{inputenc}
\usepackage[T1,T2A]{fontenc} % Preserve T2A for compatibility, though not used
\usepackage[ngerman]{babel}
\usepackage[left=2cm,right=2cm,top=2cm,bottom=2cm]{geometry}
\usepackage{lmodern}
\usepackage{amsmath}
\usepackage{amssymb}
\usepackage{physics}
\usepackage{hyperref}
\usepackage{tocloft}
\usepackage{tcolorbox}
\usepackage{booktabs}
\usepackage{enumitem}
\usepackage[table,xcdraw]{xcolor}
\usepackage{pgfplots}
\pgfplotsset{compat=1.18}
\usepackage{graphicx}
\usepackage{float}
\usepackage{mathtools}
\usepackage{fancyhdr}

\renewcommand{\cftsecfont}{\color{blue}}
\renewcommand{\cftsubsecfont}{\color{blue}}
\renewcommand{\cftsecpagefont}{\color{blue}}
\renewcommand{\cftsubsecpagefont}{\color{blue}}
\setlength{\cftsecindent}{1cm}
\setlength{\cftsubsecindent}{2cm}

\hypersetup{
	colorlinks=true,
	linkcolor=blue,
	citecolor=blue,
	urlcolor=blue,
	pdftitle={Zeit als emergente Eigenschaft in der Quantenmechanik},
	pdfauthor={Johann Pascher},
	pdfsubject={Theoretische Physik},
	pdfkeywords={T0-Modell, Zeit-Masse-Dualität, Quantenmechanik, Feinstrukturkonstante}
}

% Custom Commands
\newcommand{\Tfield}{T(x)}
\newcommand{\betaT}{\beta_{\text{T}}}
\newcommand{\alphaEM}{\alpha_{\text{EM}}}
\newcommand{\alphaW}{\alpha_{\text{W}}}
\newcommand{\Mpl}{M_{\text{Pl}}}
\newcommand{\Tzerot}{T_0(\Tfield)}
\newcommand{\Tzero}{T_0}
\newcommand{\vecx}{\vec{x}}
\newcommand{\gammaf}{\gamma_{\text{Lorentz}}}
\newcommand{\DhiggsT}{\Tfield (\partial_\mu + ig A_\mu) \Phi + \Phi \partial_\mu \Tfield}
\newcommand{\DcovT}[1]{\Tfield D_\mu #1 + #1 \partial_\mu \Tfield}
\newcommand{\HiggsLagr}{\mathcal{L}_{\text{Higgs-T}}}

% Headers and Footers
\pagestyle{fancy}
\fancyhf{}
\fancyhead[L]{Johann Pascher}
\fancyhead[R]{Zeit-Masse-Dualität}
\fancyfoot[C]{\thepage}
\renewcommand{\headrulewidth}{0.4pt}
\renewcommand{\footrulewidth}{0.4pt}

\title{Zeit als emergente Eigenschaft in der Quantenmechanik: \\Eine Verbindung zwischen Relativität, Feinstrukturkonstante und Quantendynamik}
\author{Johann Pascher}
\date{23. März 2025}

\begin{document}
	
	\maketitle
	
	\tableofcontents
	\newpage
	
	\section{Einführung}
	In der modernen Physik werden Zeit und Raum unterschiedlich behandelt. Während Raumkoordinaten in der Quantenmechanik durch Operatoren dargestellt werden, erscheint Zeit primär als Parameter. Diese asymmetrische Behandlung wirft fundamentale Fragen zur Natur der Zeit auf. Diese Arbeit untersucht, inwieweit Zeit als emergente Eigenschaft verstanden werden kann, verknüpft mit fundamentalen Konstanten und der Masse des betrachteten Systems.
	
	Diese Untersuchung ist Teil eines umfassenderen konzeptionellen Rahmens, der detailliert in der begleitenden Arbeit \href{https://github.com/jpascher/T0-Time-Mass-Duality/tree/main/2/pdf/Deutsch/KomplementPhysikZeit.pdf}{\textit{Komplementäre Erweiterungen der Physik: Absolute Zeit und intrinsische Zeit}} \cite{pascher_komplementaer_2025} (24. März 2025) diskutiert wird. Dort wird die hier eingeführte intrinsische Zeit mit einem komplementären Modell absoluter Zeit (dem \href{https://github.com/jpascher/T0-Time-Mass-Duality/tree/main/2/pdf/Deutsch/ZeitMasseT0Params.pdf}{T0-Modell} \cite{pascher_params_2025}) verbunden, was zu einer \href{https://github.com/jpascher/T0-Time-Mass-Duality/tree/main/2/pdf/Deutsch/MathZeitMasseLagrange.pdf}{Zeit-Masse-Dualität} \cite{pascher_lagrange_2025} führt, die konzeptionell der bekannten Welle-Teilchen-Dualität ähnelt. Die philosophischen Implikationen dieser Dualität werden in \cite{pascher_zeit_masse_2025} weiter erforscht.
	
	\section{Zeit in der speziellen Relativitätstheorie}
	Einsteins berühmte Formel \( E = mc^2 \) verbindet Energie und Masse über die Lichtgeschwindigkeit \cite{einstein}. Um eine Verbindung zur Zeit herzustellen, führen wir ein:
	\begin{align}
		E &= mc^2 \\
		E &= h\nu = \frac{h}{T}
	\end{align}
	wobei \( h \) die Planck-Konstante, \( \nu \) die Frequenz und \( T \) die Periode ist. Gleichsetzen ergibt:
	\begin{align}
		mc^2 &= \frac{h}{T} \\
		T &= \frac{h}{mc^2}
	\end{align}
	Diese Zeit \( T \) kann als charakteristische oder intrinsische Zeitskala interpretiert werden, die mit der Masse \( m \) verknüpft ist. Diese fundamentale Beziehung bildet die Grundlage des Konzepts des intrinsischen Zeitfeldes im T0-Modell, wie in \cite{pascher_lagrange_2025} ausgeführt.
	
	\section{Verbindung zur Feinstrukturkonstanten}
	Die Feinstrukturkonstante \( \alphaEM \) ist eine dimensionslose physikalische Konstante, die die Stärke der elektromagnetischen Wechselwirkung beschreibt:
	\begin{equation}
		\alphaEM = \frac{e^2}{4\pi\varepsilon_0\hbar c} \approx \frac{1}{137,035999}
	\end{equation}
	
	\subsection{Ableitung über elektromagnetische Konstanten}
	Die Planck-Konstante kann durch elektromagnetische Vakuumkonstanten ausgedrückt werden:
	\begin{equation}
		h = \frac{1}{2\pi\sqrt{\mu_0\varepsilon_0}}
	\end{equation}
	Somit kann die intrinsische Zeit \( T \) umgeschrieben werden als:
	\begin{align}
		T &= \frac{h}{mc^2} \\
		&= \frac{1}{2\pi\sqrt{\mu_0\varepsilon_0}} \cdot \frac{1}{mc^2}
	\end{align}
	Da \( c = \frac{1}{\sqrt{\mu_0\varepsilon_0}} \), erhalten wir:
	\begin{align}
		T &= \frac{1}{2\pi\sqrt{\mu_0\varepsilon_0}} \cdot \frac{1}{m \cdot \frac{1}{\mu_0\varepsilon_0}} \\
		&= \frac{\mu_0\varepsilon_0}{2\pi m c}
	\end{align}
	
	Diese Beziehung offenbart eine tiefe Verbindung zwischen intrinsischer Zeit, elektromagnetischen Konstanten und Masse, die in \cite{pascher_alpha_2025} weiter untersucht wird, wo ein natürliches Einheitensystem mit \(\alphaEM = 1\) vorgeschlagen wird.
	
	\section{Zeit in der Quantenmechanik}
	\subsection{Standardbehandlung der Zeit}
	In der konventionellen Quantenmechanik erscheint Zeit als Parameter in der Schrödinger-Gleichung:
	\begin{equation}
		i\hbar \frac{\partial}{\partial t}\Psi(x,t) = \hat{H}\Psi(x,t)
	\end{equation}
	Im Gegensatz zu Raum- oder Impulskoordinaten gibt es keinen Zeitoperator. Zeit wird als kontinuierlicher Parameter behandelt, entlang dessen Quantenzustände sich entwickeln. Diese asymmetrische Behandlung von Zeit gegenüber Raum ist ein langjähriges Problem in der Quantentheorie \cite{pascher_erweiterung_2025}.
	
	\subsection{Eine neue Perspektive: Intrinsische Zeit}
	Betrachten wir die charakteristische Zeit \( \Tfield = \frac{\hbar}{\max(m c^2, \omega)} \) als die „intrinsische Zeit“ eines Quantenobjekts, die sowohl massive Teilchen als auch Photonen umfasst. Diese Zeit hängt von der Masse oder Energie des Objekts ab und könnte als minimale Zeitskala interpretiert werden, auf der das Objekt quantenmechanische Änderungen erfährt. Die Schrödinger-Gleichung wird modifiziert zu:
	\begin{equation}
		i\hbar \Tfield \frac{\partial}{\partial t} \Psi + i\hbar \Psi \frac{\partial \Tfield}{\partial t} = \hat{H} \Psi
	\end{equation}
	Dies impliziert, dass die Zeitentwicklung nicht mehr für alle Objekte einheitlich ist, sondern von ihren Eigenschaften abhängt. Für Photonen ist die intrinsische Zeit energieabhängig, mit wichtigen Konsequenzen für Nichtlokalität, wie in \cite{pascher_photons_2025} detailliert beschrieben.
	
	\section{Erweiterte Beziehungen zwischen Zeit, Masse und fundamentalen Konstanten}
	\subsection{Eine einheitliche Beziehung zur Feinstrukturkonstanten}
	Unter Verwendung von \( T = \frac{\hbar}{mc^2} \) für massive Teilchen und Erweiterung:
	\begin{align}
		T &= \frac{\hbar}{mc^2} \cdot \frac{4\pi\varepsilon_0\hbar c}{e^2} \cdot \alphaEM
	\end{align}
	Dies offenbart eine tiefe Verbindung zwischen Zeitentwicklung und elektromagnetischen Wechselwirkungen. Wenn \(\alphaEM = 1\) in natürlichen Einheiten gesetzt wird, wird diese Beziehung besonders elegant, wie in \cite{pascher_alphabeta_2025} diskutiert.
	
	\subsection{Modifizierte Dispersionsrelation}
	In der Standard-Quantenmechanik:
	\begin{equation}
		\omega = \frac{\hbar k^2}{2m}
	\end{equation}
	Mit \( \Tfield = \frac{\hbar}{\max(m c^2, \omega)} \), für massive Teilchen:
	\begin{equation}
		\omega_{\text{eff}} = \frac{\hbar^2 k^2}{2 m^2 c^2}
	\end{equation}
	Dies unterscheidet sich von der Standardrelation \( \omega \propto \frac{1}{m} \). Die modifizierte Dispersionsrelation könnte nachweisbare Konsequenzen für Quantenphänomene bei hohen Energien oder auf kosmologischen Skalen haben, wie in \cite{pascher_emergente_gravitation_2025} untersucht.
	
	\subsection{Behandlung von Mehr-Teilchen-Systemen}
	Für zwei Teilchen (\( m_1 \), \( m_2 \)):
	\begin{equation}
		i (m_1 + m_2) c^2 \frac{\partial}{\partial t} \Psi(x_1, x_2, t) = \hat{H} \Psi(x_1, x_2, t)
	\end{equation}
	Für verschränkte Zustände:
	\begin{equation}
		|\Psi(t)\rangle = \frac{1}{\sqrt{2}}(|0(t/T_1)\rangle_{m_1} \otimes |1(t/T_2)\rangle_{m_2} + |1(t/T_1)\rangle_{m_1} \otimes |0(t/T_2)\rangle_{m_2})
	\end{equation}
	wobei \( T_1 = \frac{\hbar}{m_1 c^2} \), \( T_2 = \frac{\hbar}{m_2 c^2} \). Dieser Formalismus bietet neue Einsichten in Quantenverschränkung und Nichtlokalität, wie in \cite{pascher_feldtheorie_2025} detailliert beschrieben.
	
	\subsection{Massenabhängige Kohärenzzeiten und Instantaneität}
	Dekohärenzrate:
	\begin{equation}
		\Gamma_{\text{dec}} = \Gamma_0 \cdot \frac{m c^2}{\hbar}
	\end{equation}
	Energie-Zeit-Unschärfe:
	\begin{equation}
		\Delta t \gtrsim \frac{\hbar}{mc^2}
	\end{equation}
	
	Diese Beziehungen deuten auf experimentell überprüfbare Vorhersagen hinsichtlich massenabhängiger Quantenkohärenz hin und bieten potenzielle Einsichten in den Übergang vom Quanten- zum Klassischen \cite{pascher_erweiterung_2025}.
%-----	
\section{Lagrange-Formulierung}
Die vollständige mathematische Struktur des T0-Modells wird durch die Gesamt-Lagrange-Dichte erfasst \cite{pascher_lagrange_2025}:
\begin{equation}
	\mathcal{L}_{\text{Gesamt}} = \mathcal{L}_{\text{Boson}} + \mathcal{L}_{\text{Fermion}} + \mathcal{L}_{\text{Higgs-T}} + \mathcal{L}_{\text{intrinsisch}}
\end{equation}

Die Lagrange-Dichte des intrinsischen Zeitfeldes kombiniert in ihrer vollständigen Form sowohl die freie Felddynamik als auch die Materiewechselwirkungen:

\begin{equation}
	\mathcal{L}_{\text{intrinsisch}}^{\text{vollständig}} = \underbrace{\frac{1}{2} \partial_\mu \Tfield \partial^\mu \Tfield - V(\Tfield)}_{\text{Freie Felddynamik}} + \underbrace{\bar{\psi} \left( i\hbar \gamma^0 \frac{\partial}{\partial (t/\Tfield)} - i\hbar \gamma^0 \frac{\partial}{\partial t} \right) \psi}_{\text{Wechselwirkung mit Materie}}
\end{equation}

Je nach Kontext und Analyseschwerpunkt wird entweder die freie Feldkomponente oder die Materiewechselwirkungskomponente betont. Für Untersuchungen zur Feldausbreitung und Potentialenergie wird häufig der erste Term mit $V(\Tfield) = \frac{1}{2}\Tfield^2$ verwendet.
%-----
	
	\section{Implikationen für die Physik}
	\subsection{Eine neue Perspektive auf Zeit}
	Zeit wird nicht als fundamental abgeleitet, sondern als emergente Eigenschaft betrachtet, die durch das intrinsische Zeitfeld \(\Tfield\) mit der Masse verknüpft ist. Diese Perspektive bietet eine radikale Neuinterpretation temporaler Phänomene in der Physik \cite{pascher_zeit_masse_2025}.
	
	\subsection{Verbindung zur Zeitdilatation}
	Das T0-Modell spiegelt relativistische Effekte durch Massenvariation statt Zeitdilatation wider. Obwohl mathematisch äquivalent zur speziellen Relativitätstheorie, bietet dieser Ansatz neue konzeptionelle Einsichten und potenzielle experimentelle Unterscheidungen \cite{pascher_messdifferenzen_2025}.
	
	\subsection{Emergente Gravitation}
	Im T0-Modell emergiert Gravitation als Kraft aus den Gradienten des intrinsischen Zeitfeldes \(\Tfield\), definiert durch \( m = \frac{\hbar}{\Tfield c^2} \) in einem einheitlichen Einheitensystem (\(\hbar = c = G = \alphaEM = \betaT = 1\)). Das Gravitationspotential ergibt sich als:
	\begin{equation}
		\Phi(r) = -\frac{G M}{r} + \kappa r,
	\end{equation}
	wobei \(\kappa\) in natürlichen Einheiten die Dimension \([E]\) hat. Für eine Punktmasse \(M\) bei kurzen Distanzen approximiert das Potential zu \(\Phi(r) \approx -\frac{GM}{r}\). Die resultierende Kraft ist:
	\begin{equation}
		\vec{F} = -\nabla \Phi \approx -\frac{M}{r^2} \hat{r},
	\end{equation}
	und reproduziert die Newtonsche Gravitation. Auf größeren Skalen erklärt der Term \(\kappa r\) flache Rotationskurven ohne die Notwendigkeit dunkler Materie, wie in Galaxien beobachtet \cite{rubin1980, McGaugh2016}. Eine detaillierte Ableitung ist in \cite{pascher_emergente_gravitation_2025, pascher_alphabeta_2025} gegeben.
	
	\section{Kosmologische Implikationen in SI-Einheiten}
	\begin{itemize}
		\item \textbf{Rotverschiebung:} Im T0-Modell wird die kosmische Rotverschiebung \(z\) durch die Variation des intrinsischen Zeitfeldes \(\Tfield\) bestimmt, mit der Beziehung \(1 + z = \frac{\Tfield_0}{\Tfield}\), wobei \(\Tfield_0\) der lokale Wert des Zeitfeldes ist. In SI-Einheiten wird dies zu \(1 + z = e^{\alpha^{\text{SI}} d}\), mit \(\alpha^{\text{SI}} \approx 2,3 \times 10^{-18} \, \text{m}^{-1}\), wobei \(\alpha = H_0/c\) die räumliche Variationsrate von \(\Tfield\) beschreibt \cite{pascher_galaxies_2025, pascher_emergente_gravitation_2025}. Dieser Ansatz bietet eine alternative Erklärung zur kosmischen Expansion für die beobachtete Rotverschiebung entfernter Galaxien.
		
		\item \textbf{Gravitationspotential:} Das emergente Gravitationspotential im T0-Modell ist \(\Phi(r) = -\frac{GM}{r} + \kappa r\), mit \(\kappa^{\text{nat}} = \betaT^{\text{nat}} \cdot \frac{yv}{r_g^2}\frac{y v}{r_g^2}\) in natürlichen Einheiten, wobei \(\kappa\) die Dimension \([E]\) hat \cite{pascher_emergente_gravitation_2025}. Dieses modifizierte Potential erklärt erfolgreich Galaxienrotationskurven ohne dunkle Materie und bietet eine sparsamere Alternative zu MOND \cite{Milgrom1983} und \(\Lambda\)CDM-Modellen \cite{Planck2018}.
		
		\item \textbf{Wellenlängenabhängigkeit:} Die Rotverschiebung zeigt eine wellenlängenabhängige Komponente, beschrieben durch \(z(\lambda) = z_0 \left(1 + \betaT^{\text{SI}} \ln\left(\frac{\lambda}{\lambda_0}\right)\right)\), mit \(\betaT^{\text{SI}} \approx 0,008\). In natürlichen Einheiten mit \(\betaT^{\text{nat}} = 1\) wird dies zu \(z(\lambda) = z_0 \left(1 + \ln\left(\frac{\lambda}{\lambda_0}\right)\right)\) \cite{pascher_temp_2025, pascher_alphabeta_2025}. Diese charakteristische Vorhersage bietet einen klaren experimentellen Test, um das T0-Modell von der Standardkosmologie zu unterscheiden.
	\end{itemize}
	
	\section{Rolle von \(\betaT\)}
	Der Parameter \(\betaT^{\text{SI}} \approx 0,008\) in SI-Einheiten dient als Faktor in der natürlichen Ableitung der wellenlängenabhängigen Rotverschiebung \( z(\lambda) = z_0 \left(1 + \betaT^{\text{SI}} \ln\left(\frac{\lambda}{\lambda_0}\right)\right) \) im T0-Modell \cite{pascher_alphabeta_2025}. In einem einheitlichen Einheitensystem wird \(\betaT^{\text{nat}} = 1\) gesetzt, was die charakteristische Längenskala \( r_0 \approx 1,33 \times 10^{-4} \cdot l_P \) definiert und die Konsistenz mit \(\alphaEM = 1\) unterstützt. Die genaue Ableitung dieses Parameters und seine Beziehung zum Higgs-Mechanismus ist in \cite{pascher_params_2025, pascher_higgs_2025} detailliert beschrieben.
	
	\section{Möglichkeiten der experimentellen Verifikation}
	\begin{itemize}
		\item \textbf{Unterschiede in Kohärenzzeiten:} Messung zeitlicher Abweichungen in interferometrischen Experimenten mit Teilchen unterschiedlicher Massen könnte die massenabhängige Natur der Quantenentwicklung offenbaren \cite{pascher_erweiterung_2025}.
		
		\item \textbf{Massenabhängige Phasenverschiebungen:} Untersuchung von Phasenunterschieden abhängig von Teilchenmassen in Quanteninterferenzexperimenten könnte direkte Beweise für das Konzept des intrinsischen Zeitfeldes liefern \cite{pascher_feldtheorie_2025}.
		
		\item \textbf{Spektroskopische Signaturen:} Nachweis wellenlängenabhängiger Rotverschiebung mit hochpräziser Spektroskopie unter Verwendung von Instrumenten wie dem James-Webb-Weltraumteleskop würde einen definitiven Test der kosmologischen Vorhersagen des T0-Modells bieten \cite{pascher_messdifferenzen_2025}.
		
		\item \textbf{Test der modifizierten Dispersionsrelation:} \( \omega_{\text{eff}} \propto \frac{1}{m^2} \), überprüfbar durch Lichtausbreitungsexperimente bei unterschiedlichen Energien, könnte das T0-Modell von der Standard-Quantentheorie unterscheiden \cite{pascher_photons_2025}.
	\end{itemize}
	
	\section{Effekte auf instantane Kohärenz in der Quantenmechanik}
	\subsection{Problem der instantanen Kohärenz}
	Quantenkorrelationen erscheinen instantan und stellen die Kausalität in der Standardphysik in Frage \cite{bell}. Das T0-Modell bietet eine neue Perspektive auf dieses langjährige Problem.
	
	\subsection{Massenabhängige Kohärenzzeiten}
	Schwerere Teilchen dekohärieren in Labortzeit schneller aufgrund der Massenvariation \( m = \frac{\hbar}{T c^2} \) im T0-Modell \cite{pascher_galaxies_2025, pascher_feldtheorie_2025}. Diese Massenabhängigkeit könnte scheinbare Nichtlokalität erklären, ohne instantane Wirkung über Distanzen zu erfordern.
	
	\subsection{Mathematische Formulierung für Mehr-Teilchen-Systeme}
	Die Dynamik wird durch die Kopplung an \(\Tfield\) über die modifizierte Schrödinger-Gleichung und Lagrange-Dichte beschrieben (siehe detaillierte Gleichungen in \cite{pascher_lagrange_2025}). Dieser Formalismus bietet eine kohärente Beschreibung der Quantenverschränkung und ihrer zeitlichen Aspekte.
	
	\subsection{Effekte auf verschränkte Zustände}
	Massenunterschiede beeinflussen die Kohärenz verschränkter Zustände über die intrinsische Zeit \(\Tfield\). Das T0-Modell sagt voraus, dass verschränkte Teilchen mit unterschiedlichen Massen charakteristische zeitliche Verhaltensweisen zeigen würden, die experimentell überprüfbar sind \cite{pascher_feldtheorie_2025}.
	
	\subsection{Neue Interpretation für EPR und Bell}
	Nichtlokalität könnte massenabhängige Zeitskalen widerspiegeln, überprüfbar durch modifizierte Bell-Experimente mit variablen Massen \cite{bell, pascher_photons_2025}. Dieser Ansatz bietet eine mögliche Lösung für die Spannungen zwischen Quantenmechanik und Relativitätstheorie, ohne verborgene Variablen oder überlichtschnelle Signalübertragung zu erfordern.
	
	\section{Schlussfolgerungen und Ausblick}
	Das T0-Modell betrachtet Zeit als emergent und bietet einen einheitlichen, experimentell überprüfbaren Rahmen, der Relativitätstheorie, Quantenmechanik und die Feinstrukturkonstante verbindet, wobei \(\betaT\) eine zentrale Rolle bei der Ableitung kosmologischer und quantenmechanischer Effekte spielt \cite{pascher_galaxies_2025, pascher_alphabeta_2025}.
	
	Zukünftige Arbeiten werden sich auf die Entwicklung präziserer experimenteller Tests, die Verfeinerung des mathematischen Formalismus und die Erforschung der Implikationen für Quantengravitation und Vereinheitlichungstheorie konzentrieren. Die umfassende Integration fundamentaler Kräfte innerhalb des T0-Rahmens, wie in \cite{pascher_grundkraefte_2025} skizziert, weist auf einen potenziell transformativen Ansatz in der theoretischen Physik hin, der weitere Untersuchungen verdient.
	
	Die sparsame Erklärung des T0-Modells für Phänomene, die traditionell dunkler Materie und dunkler Energie zugeschrieben werden, kombiniert mit seinen neuartigen Vorhersagen bezüglich wellenlängenabhängiger Rotverschiebung und massenabhängiger Quantenentwicklung, bietet eine überzeugende Alternative zu standardkosmologischen und quantenmechanischen Theorien, die ernsthafte Berücksichtigung durch die wissenschaftliche Gemeinschaft verdient.
	
	\begin{thebibliography}{99}
		\bibitem{pascher_komplementaer_2025} Pascher, J. (2025). \href{https://github.com/jpascher/T0-Time-Mass-Duality/tree/main/2/pdf/Deutsch/KomplementPhysikZeit.pdf}{Komplementäre Erweiterungen der Physik: Absolute Zeit und intrinsische Zeit}. 24. März 2025.
		\bibitem{pascher_zeit_masse_2025} Pascher, J. (2025). \href{https://github.com/jpascher/T0-Time-Mass-Duality/tree/main/2/pdf/Deutsch/ZeitMasseNeuerBlick.pdf}{Zeit und Masse: Ein neuer Blick auf alte Formeln – und Befreiung von traditionellen Zwängen}. 22. März 2025.
		\bibitem{pascher_messdifferenzen_2025} Pascher, J. (2025). \href{https://github.com/jpascher/T0-Time-Mass-Duality/tree/main/2/pdf/Deutsch/MessdifferenzenT0Standard.pdf}{Kompensatorische und additive Effekte: Eine Analyse der Messunterschiede zwischen dem T0-Modell und dem \(\Lambda\)CDM-Standardmodell}. 2. April 2025.
		\bibitem{pascher_alpha_2025} Pascher, J. (2025). \href{https://github.com/jpascher/T0-Time-Mass-Duality/tree/main/2/pdf/Deutsch/NatEinheitenAlpha1.pdf}{Energie als fundamentale Einheit: Natürliche Einheiten mit \(\alphaEM = 1\) im T0-Modell}. 26. März 2025.
		\bibitem{pascher_params_2025} Pascher, J. (2025). \href{https://github.com/jpascher/T0-Time-Mass-Duality/tree/main/2/pdf/Deutsch/ZeitMasseT0Params.pdf}{Zeit-Masse-Dualitätstheorie (T0-Modell): Ableitung der Parameter \(\kappa\), \(\alpha\) und \(\beta\)}. 4. April 2025.
		\bibitem{pascher_higgs_2025} Pascher, J. (2025). \href{https://github.com/jpascher/T0-Time-Mass-Duality/tree/main/2/pdf/Deutsch/MathHiggsZeitMasse.pdf}{Mathematische Formulierung des Higgs-Mechanismus in der Zeit-Masse-Dualität}. 28. März 2025.
		\bibitem{pascher_lagrange_2025} Pascher, J. (2025). \href{https://github.com/jpascher/T0-Time-Mass-Duality/tree/main/2/pdf/Deutsch/MathZeitMasseLagrange.pdf}{Von Zeitdilatation zur Massenvariation: Mathematische Kernformulierungen der Zeit-Masse-Dualitätstheorie}. 29. März 2025.
		\bibitem{pascher_emergente_gravitation_2025} Pascher, J. (2025). \href{https://github.com/jpascher/T0-Time-Mass-Duality/tree/main/2/pdf/Deutsch/EmergentGravT0.pdf}{Emergente Gravitation im T0-Modell: Eine umfassende Ableitung}. 1. April 2025.
		\bibitem{pascher_galaxies_2025} Pascher, J. (2025). \href{https://github.com/jpascher/T0-Time-Mass-Duality/tree/main/2/pdf/Deutsch/MassVarGalaxien.pdf}{MassenVariation in Galaxien: Eine Analyse im T0-Modell mit emergenter Gravitation}. 30. März 2025.
		\bibitem{pascher_alphabeta_2025} Pascher, J. (2025). \href{https://github.com/jpascher/T0-Time-Mass-Duality/tree/main/2/pdf/Deutsch/Alpha1Beta1Konsistenz.pdf}{Einheitliches Einheitensystem im T0-Modell: Die Konsistenz von \(\alphaEM = 1\) und \(\betaT = 1\)}. 5. April 2025.
		\bibitem{pascher_temp_2025} Pascher, J. (2025). \href{https://github.com/jpascher/T0-Time-Mass-Duality/tree/main/2/pdf/Deutsch/TempEinheitenCMB.pdf}{Anpassung der Temperatureinheiten in natürlichen Einheiten und CMB-Messungen}. 2. April 2025.
		\bibitem{pascher_erweiterung_2025} Pascher, J. (2025). \href{https://github.com/jpascher/T0-Time-Mass-Duality/tree/main/2/pdf/Deutsch/NotwendigkeitQMErweiterung.pdf}{Die Notwendigkeit der Erweiterung der Standard-Quantenmechanik und Quantenfeldtheorie}. 27. März 2025.
		\bibitem{pascher_feldtheorie_2025} Pascher, J. (2025). \href{https://github.com/jpascher/T0-Time-Mass-Duality/tree/main/2/pdf/Deutsch/FeldtheorieQuanten.pdf}{Feldtheorie und Quantenkorrelationen: Eine neue Perspektive auf Instantaneität}. 28. März 2025.
		\bibitem{pascher_photons_2025} Pascher, J. (2025). \href{https://github.com/jpascher/T0-Time-Mass-Duality/tree/main/2/pdf/Deutsch/DynMassePhotonenNichtlokal.pdf}{Dynamische Masse von Photonen und ihre Implikationen für Nichtlokalität im T0-Modell}. 25. März 2025.
		\bibitem{pascher_grundkraefte_2025} Pascher, J. (2025). \href{https://github.com/jpascher/T0-Time-Mass-Duality/tree/main/2/pdf/Deutsch/VierKraefteZeitMasse.pdf}{Vereinfachte Beschreibung fundamentaler Kräfte mit Zeit-Masse-Dualität}. 27. März 2025.
		\bibitem{pascher_planck_2025} Pascher, J. (2025). \href{https://github.com/jpascher/T0-Time-Mass-Duality/tree/main/2/pdf/Deutsch/JenseitsPlanck.pdf}{Reale Konsequenzen der Neuformulierung von Zeit und Masse in der Physik: Jenseits der Planck-Skala}. 24. März 2025.
		\bibitem{einstein} Einstein, A. (1905). \textit{Zur Elektrodynamik bewegter Körper}. Annalen der Physik, 322(10), 891-921. DOI: 10.1002/andp.19053221004.
		\bibitem{bell} Bell, J. S. (1964). \textit{Zum Einstein-Podolsky-Rosen-Paradoxon}. Physics Physique {\fontencoding{T2A}\selectfont Физика}, 1(3), 195-200. DOI: \href{https://doi.org/10.1103/Physics.1.195}{10.1103/Physics.1.195}
		\bibitem{rubin1980} Rubin, V. C., Ford Jr, W. K., \& Thonnard, N. (1980). Rotationsbewegung von 21 SC-Galaxien mit einem großen Bereich von Leuchtkraft und Radien. \textit{The Astrophysical Journal}, 238, 471-487. DOI: 10.1086/158003.
		\bibitem{McGaugh2016} McGaugh, S. S., Lelli, F., \& Schombert, J. M. (2016). Radiale Beschleunigungsbeziehung in rotationsgestützten Galaxien. \textit{Physical Review Letters}, 117(20), 201101. DOI: 10.1103/PhysRevLett.117.201101.
		\bibitem{Milgrom1983} Milgrom, M. (1983). Eine Modifikation der Newtonschen Dynamik. \textit{The Astrophysical Journal}, 270, 365-370. DOI: 10.1086/161130.
		\bibitem{Planck2018} Planck Collaboration (2020). Planck 2018 Ergebnisse. VI. Kosmologische Parameter. \textit{Astronomy \& Astrophysics}, 641, A6. DOI: 10.1051/0004-6361/201833910.
		\bibitem{planck1899} Planck, M. (1899). Über irreversible Strahlungsvorgänge. \textit{Sitzungsberichte der Preußischen Akademie der Wissenschaften}, 5, 440-480.
		\bibitem{higgs1964} Higgs, P. W. (1964). Gebrochene Symmetrien und die Massen von Eichbosonen. \textit{Physical Review Letters}, 13(16), 508-509. DOI: 10.1103/PhysRevLett.13.508.
		\bibitem{dirac1928} Dirac, P. A. M. (1928). Die Quantentheorie des Elektrons. \textit{Proceedings of the Royal Society A}, 117(778), 610-624. DOI: 10.1098/rspa.1928.0023.
	\end{thebibliography}
	
\end{document}