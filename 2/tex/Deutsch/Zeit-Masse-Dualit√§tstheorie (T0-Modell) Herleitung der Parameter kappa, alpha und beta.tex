\documentclass{article}
\usepackage[utf8]{inputenc}
\usepackage[T1]{fontenc}
\usepackage{lmodern}
\usepackage[ngerman]{babel}
\usepackage{amsmath,amssymb,physics}
\usepackage{graphicx,tikz,pgfplots}
\pgfplotsset{compat=1.18}
\usepackage{tikz-feynman}
\usepackage{microtype}
\usepackage{hyperref}
\usepackage{mathtools,textalpha,upgreek}
\usepackage{booktabs}
\usepackage{siunitx}
\usepackage{cleveref}
\usepackage{amsthm}

% Custom commands
\newcommand{\Tfield}{T(x)}
\newcommand{\DcovT}[1]{\Tfield D_\mu #1 + #1 \partial_\mu \Tfield}
\newcommand{\HiggsLagr}{\mathcal{L}_{\text{Higgs-T}}}
\newcommand{\FermionLagr}{\mathcal{L}_{\text{Fermion-T}}}
\newcommand{\BosonLagr}{\mathcal{L}_{\text{Boson-T}}}
\newcommand{\Mpl}{M_{\text{Pl}}}

% Theorem styles
\newtheorem{theorem}{Theorem}[section]
\newtheorem{proposition}[theorem]{Proposition}

% Hyperref configuration
\hypersetup{
	colorlinks=true,
	linkcolor=blue,
	urlcolor=blue,
	citecolor=red,
	pdftitle={Zeit-Masse-Dualitätstheorie (T0-Modell) Herleitung der Parameter kappa, alpha und beta},
	pdfauthor={Johann Pascher}
}

\title{Zeit-Masse-Dualitätstheorie (T0-Modell) \\ Herleitung der Parameter \(\kappa\), \(\alpha\) und \(\beta\)}
\author{Johann Pascher}
\date{30. März 2025}

\begin{document}
	
	\maketitle
	
	\begin{abstract}
		Dieses Dokument präsentiert eine vollständige theoretische Analyse der zentralen Parameter des T0-Modells:
		\begin{enumerate}
			\item Fundamentale Herleitungen in natürlichen Einheiten (\(\hbar = c = G = 1\))
			\item Konvertierung in SI-Einheiten für experimentelle Vorhersagen
			\item Mikroskopische Begründung der Korrelationslänge \(L_T\)
			\item Störungstheoretische Ableitung von \(\beta\) via Feynman-Diagrammen
		\end{enumerate}
	\end{abstract}
	
	\tableofcontents
	\newpage
	
	\section{Einleitung}
	Das T0-Modell postuliert eine Dualität zwischen zeitlicher und massenbezogener Beschreibung physikalischer Prozesse. Zentrale Parameter sind:
	\begin{itemize}
		\item \(\kappa\): Modifikation des Gravitationspotentials \(\Phi(r) = -\frac{GM}{r} + \kappa r\)
		\item \(\alpha\): Photonen-Energieverlustrate (\(1 + z = e^{\alpha r}\))
		\item \(\beta\): Wellenlängenabhängigkeit der Rotverschiebung (\(z(\lambda) = z_0 (1 + \beta \ln(\lambda/\lambda_0))\))
	\end{itemize}
	
	\section{Herleitung von \(\kappa\)}
	\begin{theorem}[Herleitung von \(\kappa\)]
		In natürlichen Einheiten (\(\hbar = c = G = 1\)):
		\begin{equation}
			\kappa = \beta \frac{y v}{r_g}, \quad r_g = \sqrt{\frac{M}{a_0}}
		\end{equation}
		In SI-Einheiten:
		\begin{equation}
			\kappa_{\text{SI}} = \beta \frac{y v c^2}{r_g^2} \approx 4.8 \times 10^{-11} \text{ m/s}^2
		\end{equation}
	\end{theorem}
	
	\section{Herleitung von \(\alpha\)}
	\begin{theorem}[Herleitung von \(\alpha\)]
		In natürlichen Einheiten (\(\hbar = c = G = 1\)):
		\begin{equation}
			\alpha = \frac{\lambda_h^2 v}{L_T}, \quad L_T \sim \frac{\Mpl}{m_h^2 v}
		\end{equation}
		In SI-Einheiten:
		\begin{equation}
			\alpha_{\text{SI}} = \frac{\lambda_h^2 v c^2}{L_T} \approx 2.3 \times 10^{-18} \text{ m}^{-1}
		\end{equation}
	\end{theorem}
	
	\section{Herleitung von \(\beta\)}
	\begin{theorem}[Herleitung von \(\beta\)]
		In natürlichen Einheiten (\(\hbar = c = G = 1\)):
		\begin{equation}
			\beta = \frac{\lambda_h^2 v^2}{4\pi^2 \lambda_0 \alpha_0}
		\end{equation}
		Störungstheoretisches Ergebnis:
		\begin{equation}
			\beta = \frac{(2\pi)^4 m_h^2}{16 \pi^2 v^4 y^2 \Mpl^2 \lambda_0^4 \alpha_0} \approx 0.008
		\end{equation}
	\end{theorem}
	
	\subsection{Feynman-Diagramm-Analyse}
	\begin{center}
		\feynmandiagram [horizontal=a to b] {
			a [particle=\(\gamma\)] -- [photon] b -- [photon] f [particle=\(\gamma\)],
			b -- [scalar, half left] c -- [scalar, half left] b,
			c -- [photon] d,
		};
	\end{center}
	
	\subsection{Experimentelle Konsequenzen}
	\begin{equation}
		z(\lambda) = z_0 \left(1 + 0.008 \ln \frac{\lambda}{\lambda_0}\right)
	\end{equation}
	
	\section{Kosmologische Implikationen}
	\begin{itemize}
		\item \(\kappa\) erklärt Rotationskurven ohne Dunkle Materie.
		\item \(\alpha\) beschreibt kosmische Expansion ohne Dunkle Energie.
		\item \(\beta\) führt zu wellenlängenabhängiger Rotverschiebung, testbar mit JWST.
	\end{itemize}
	
\begin{figure}[h]
	\centering
	\begin{tikzpicture}
		\begin{axis}[
			xlabel={Radius [kpc]},
			ylabel={Rotationsgeschwindigkeit [km/s]},
			xmin=0, xmax=30,
			ymin=0, ymax=300,
			legend pos=south east,
			grid=both,
			minor tick num=1
			]
			\addplot[blue, thick, domain=0.1:30, samples=100] {220*sqrt(10/x)};
			\addplot[red, dashed, thick, domain=0.1:30, samples=100] {sqrt(220^2*10/x + 4.8*x^2)};
			\legend{Newtonsche Vorhersage, T0-Modell}
		\end{axis}
	\end{tikzpicture}
	\caption{Rotationskurven im T0-Modell.}
\end{figure}
	
	\section{Zusammenfassung}
	\begin{table}[h]
		\centering
		\begin{tabular}{lcc}
			Parameter & Natürliche Form & SI-Wert \\
			\hline
			\(\kappa\) & \(\beta \frac{y v}{r_g}\) & \(4.8 \times 10^{-11}\) m/s\(^2\) \\
			\(\alpha\) & \(\frac{\lambda_h^2 v}{L_T}\) & \(2.3 \times 10^{-18}\) m\(^{-1}\) \\
			\(\beta\) & \(\frac{(2\pi)^4 m_h^2}{16\pi^2 v^4 y^2 \Mpl^2 \lambda_0^4 \alpha_0}\) & 0.008 \\
		\end{tabular}
	\end{table}
	
	\section*{Anhang: Vertiefende Erklärungen}
	\subsection{Mikroskopische Begründung von \(L_T\)}
	\begin{itemize}
		\item Higgs-Fluktuationen:
		\begin{equation}
			\langle \delta\Phi(x) \delta\Phi(0) \rangle \sim \frac{m_h}{16\pi^2 \Mpl} e^{-m_h |x|}
		\end{equation}
		\item Kosmische Skala:
		\begin{equation}
			L_T \sim \frac{\Mpl}{m_h^2 v} \approx 6.3 \times 10^{27} \text{ m}
		\end{equation}
	\end{itemize}
	
	\begin{thebibliography}{9}
		\bibitem{example} Beispielreferenz.
	\end{thebibliography}
	
\end{document}