\documentclass[12pt,a4paper]{article}
\usepackage[utf8]{inputenc}
\usepackage[T1]{fontenc}
\usepackage[ngerman]{babel}
\usepackage{lmodern}
\usepackage{amsmath}
\usepackage{amssymb}
\usepackage{physics}
\usepackage{hyperref}
\usepackage{tcolorbox}
\usepackage{booktabs}
\usepackage{enumitem}
\usepackage[table,xcdraw]{xcolor}
\usepackage[left=2cm,right=2cm,top=2cm,bottom=2cm]{geometry}
\usepackage{pgfplots}
\pgfplotsset{compat=1.18}
\usepackage{graphicx}
\usepackage{float}
\usepackage{fancyhdr}
\usepackage{siunitx}
\usepackage{tikz}
\usepackage{adjustbox}
\usetikzlibrary{shapes.geometric}

% Custom Commands
\newcommand{\Tfield}{T(x)}
\newcommand{\alphaEM}{\alpha_{\text{EM}}}
\newcommand{\betaT}{\beta_{\text{T}}}
\newcommand{\Mpl}{M_{\text{Pl}}}
\newcommand{\Tzerot}{T_0(\Tfield)}
\newcommand{\e}{\mathrm{e}}
\newcommand{\alphaEMSI}{\alpha_{\text{EM,SI}}}

% Header and Footer Configuration
\pagestyle{fancy}
\fancyhf{}
\fancyhead[L]{Johann Pascher}
\fancyhead[R]{Systematische Zusammenstellung natürlicher Einheiten}
\fancyfoot[C]{\thepage}
\renewcommand{\headrulewidth}{0.4pt}
\renewcommand{\footrulewidth}{0.4pt}

\hypersetup{
	colorlinks=true,
	linkcolor=blue,
	citecolor=blue,
	urlcolor=blue,
	pdftitle={Hierarchische Zusammenstellung von Einheiten im T0-Modell mit Energie als Basiseinheit},
	pdfauthor={Johann Pascher},
	pdfsubject={Theoretische Physik},
	pdfkeywords={T0-Modell, natürliche Einheiten, Feinstrukturkonstante, einheitliches Einheitensystem, Zeit-Masse-Dualität}
}

\begin{document}
	
	\title{Hierarchische Zusammenstellung von Einheiten im T0-Modell mit Energie als Basiseinheit}
	\author{Johann Pascher}
	\date{13. April 2025}
	\maketitle

\begin{abstract}
	Dieses Dokument bietet eine systematische Zusammenstellung des natürlichen Einheitensystems im T0-Modell, mit Energie als fundamentaler Basiseinheit. Es präsentiert eine hierarchische Organisation physikalischer Konstanten, quantisierter Längenskalen und elektromagnetischer Beziehungen innerhalb dieses Rahmens. Die Verknüpfung dieser Konstanten offenbart eine tiefere Struktur der physikalischen Realität, einschließlich der besonderen Stellung biologischer Systeme innerhalb der Längenskalenhierarchie. Das Dokument erläutert detailliert die Herleitung emergenter Gravitation durch die Einstein-Hilbert-Wirkung und zeigt, wie alle SI-Einheiten in diesem vereinheitlichten energiebasierten System dargestellt werden können. Aufbauend auf früheren Arbeiten \cite{pascher_planck_2025, pascher_alpha_2025, pascher_alphabeta_2025} dient diese Zusammenstellung als Referenz zum Verständnis der mathematischen Struktur des T0-Modells über alle Skalen der Physik hinweg.
\end{abstract}

\section*{Teil 1: Überblick über Einheiten und Skalen}

\subsection*{Ebene 1: Primäre dimensionale Konstanten (Wert = 1)}
\begin{itemize}[itemsep=0.5em]
	\item \textbf{Planck-Konstante} ($\hbar = 1$) - wie in der Quantenmechanik etabliert \cite{planck1901}
	\item \textbf{Lichtgeschwindigkeit} ($c = 1$) - Grundlage der relativistischen Physik \cite{einstein1905}
	\item \textbf{Gravitationskonstante} ($G = 1$) - Basis für Gravitationswechselwirkungen \cite{newton1687}
	\item \textbf{Boltzmann-Konstante} ($k_B = 1$) - Verbindung zwischen Temperatur und Energie \cite{boltzmann1872}
\end{itemize}

Diese primären Konstanten bilden die Grundlage unseres natürlichen Einheitensystems, wie in \cite{pascher_planck_2025} und \cite{pascher_alphabeta_2025} detailliert erläutert.

\subsection*{Ebene 2: Dimensionslose Kopplungskonstanten (Wert = 1)}
\begin{itemize}[itemsep=0.5em]
	\item \textbf{Feinstrukturkonstante} (\(\alphaEM = 1\)) \\
	Entspricht dem SI-Wert \(\alphaEMSI \approx \frac{1}{137,036}\), wie in \cite{pascher_alpha_2025} analysiert.
	\item \textbf{Wien-Konstante} (\(\alpha_W = 1\)) \\
	Entspricht dem SI-Wert \(\alpha_{W,\mathrm{SI}} \approx 2,82\), wie in \cite{pascher_temp_2025} diskutiert.
	\item \textbf{T0-Parameter} (\(\betaT = 1\)) \\
	Entspricht dem SI-Wert \(\beta_{T,\mathrm{SI}} \approx 0,008\), hergeleitet in \cite{pascher_params_2025} und \cite{pascher_alphabeta_2025}.
\end{itemize}

\subsection*{Ebene 2.5: Abgeleitete elektromagnetische Konstanten}
\begin{itemize}[itemsep=0.5em]
	\item \textbf{Vakuum-Permeabilität} (\(\mu_0 = 1\))
	\item \textbf{Vakuum-Permittivität} (\(\varepsilon_0 = 1\))
	\item \textbf{Vakuumimpedanz} (\(Z_0 = 1\))
	\item \textbf{Elementarladung} (\(e = \sqrt{4\pi}\))
	\item[] \textit{Hinweis: Für $\alphaEM = e^2/(4\pi\varepsilon_0\hbar c) = 1$ und $\varepsilon_0 = \hbar = c = 1$ folgt $e = \sqrt{4\pi} \approx 3,5$}
	\item \textbf{Planck-Druck} (\(p_P = 1\))
	\item \textbf{Planck-Kraft} (\(F_P = 1\))
	\item \textbf{Einstein-Hilbert-Wirkung}
	\[
	S_{\mathrm{EH}} = \frac{1}{16\pi} \int R \sqrt{-g} \, \mathrm{d}^4x
	\]
\end{itemize}

Die elektromagnetischen Konstanten und ihre Beziehungen werden in \cite{pascher_alpha_2025} detailliert untersucht.

\subsection*{Erläuterung der Einstein-Hilbert-Wirkung}

Die Einstein-Hilbert-Wirkung nimmt eine besondere Stellung im T0-Modell ein, da sie die Gravitation als geometrische Eigenschaft der Raumzeit beschreibt. In natürlichen Einheiten mit $G = c = 1$ vereinfacht sich die Einstein-Hilbert-Wirkung zu:

\[
S_{\mathrm{EH}} = \frac{1}{16\pi}\int R\sqrt{-g}d^4x
\]

wobei:
\begin{itemize}
	\item $R$ der Ricci-Skalar ist (Krümmungsskalar der Raumzeit)
	\item $g$ die Determinante des metrischen Tensors $g_{\mu\nu}$ ist
	\item $d^4x$ das vierdimensionale Raumzeit-Volumenelement ist
\end{itemize}

Im T0-Modell wird die Gravitation nicht als fundamentale Wechselwirkung betrachtet, sondern als emergentes Phänomen aus dem intrinsischen Zeitfeld $T(x)$, wie in \cite{pascher_emergente_gravitation_2025} gezeigt. Die Einstein-Hilbert-Wirkung bildet die mathematische Brücke zwischen der konventionellen geometrischen Beschreibung der Gravitation (Allgemeine Relativitätstheorie) und der T0-Darstellung mit emergenter Gravitation.

Das modifizierte Gravitationspotential im T0-Modell:
\[
\Phi(r) = -\frac{GM}{r} + \kappa r
\]

steht in direkter Beziehung zur Krümmung der Raumzeit, die in der Einstein-Hilbert-Wirkung durch den Ricci-Skalar $R$ erfasst wird. Der lineare Term $\kappa r$, der die Newtonsche Gravitation im T0-Modell ergänzt, entspricht einer modifizierten Raumzeitgeometrie und manifestiert sich in der Einstein-Hilbert-Wirkung durch modifizierte Feldgleichungen. Diese Beziehung wird in \cite{pascher_galaxies_2025} weiter untersucht.

\subsection*{Äquivalenz zwischen Einstein-Hilbert-Wirkung und Zeitfeld-Herleitung}

Das T0-Modell bietet zwei komplementäre Beschreibungen der Gravitation: Die formale Einstein-Hilbert-Wirkung $S_{\mathrm{EH}} = \frac{1}{16\pi} \int (R - 2\kappa) \sqrt{-g} \, d^4x$ und die fundamentalere Zeitfeld-Herleitung $\Phi(\vec{x}) = -\ln\left(\frac{\Tfield}{\Tfield_0}\right)$. Beide führen zum identischen Gravitationspotential $\Phi(r) = -\frac{M}{r} + \kappa r$. Die geometrische Beschreibung der Raumzeitkrümmung in der Standardtheorie erscheint im T0-Modell lediglich als effektive mathematische Darstellung der zugrunde liegenden Zeitfelddynamik, wie in \cite{pascher_emergente_gravitation_2025} und \cite{pascher_lagrange_2025} detailliert dargelegt.

\subsection*{Ebene 3: Abgeleitete Konstanten mit einfachen Werten}
\begin{itemize}[itemsep=0.5em]
	\item \textbf{Compton-Wellenlänge des Elektrons} (\(\lambda_{C,e} = \frac{1}{m_e}\))
	\item \textbf{Rydberg-Konstante} (\(R_\infty = \frac{\alphaEM^2 \cdot m_e}{2} = \frac{m_e}{2}\))
	\item[] \textit{Abgeleitet aus der Beziehung $R_\infty = m_e\cdot e^4/(8\varepsilon_0^2h^3c)$ mit $\alphaEM = 1$}
	\item \textbf{Josephson-Konstante} (\(K_J = \frac{2e}{h} = \frac{2\sqrt{4\pi}}{2\pi} = \sqrt{\frac{4}{\pi}} \approx 1,13\))
	\item[] \textit{Mit $h = 2\pi$ und $e = \sqrt{4\pi}$}
	\item \textbf{von-Klitzing-Konstante} (\(R_K = \frac{h}{e^2} = \frac{2\pi}{4\pi} = \frac{1}{2}\))
	\item[] \textit{Mit $h = 2\pi$ und $e^2 = 4\pi$}
	\item \textbf{Schwinger-Grenze} (\(E_S = \frac{m_e^2c^3}{e\sqrt{\hbar}} = m_e^2\))
	\item[] \textit{Mit $c = \hbar = 1$ und $e = \sqrt{4\pi}$}
	\item \textbf{Stefan-Boltzmann-Konstante} (\(\sigma = \frac{\pi^2k_B^4}{60\hbar^3c^2} = \frac{\pi^2}{60}\))
	\item[] \textit{Mit $\hbar = c = k_B = 1$}
	\item \textbf{Hawking-Temperatur} (\(T_H = \frac{\hbar c^3}{8\pi GMk_B} = \frac{1}{8\pi M}\))
	\item[] \textit{Mit $\hbar = c = G = k_B = 1$}
	\item \textbf{Bekenstein-Hawking-Entropie} (\(S_{\mathrm{BH}} = \frac{4\pi GM^2}{\hbar c} = 4\pi M^2\))
	\item[] \textit{Mit $\hbar = c = G = 1$}
\end{itemize}

Diese abgeleiteten Konstanten werden in \cite{pascher_nateinhsystem_2025} berechnet und verifiziert.

\subsection*{Planck-Einheiten im T0-Modell}

Im T0-Modell werden alle Planck-Einheiten auf den Wert 1 gesetzt, wodurch sie zu natürlichen Referenzpunkten für physikalische Größen werden:

\begin{table}[H]
	\centering
	\begin{adjustbox}{width=0.95\textwidth}
		\begin{tabular}{lcccl}
			\toprule
			\textbf{Planck-Einheit} & \textbf{Symbol} & \textbf{Definition im SI-System} & \textbf{Wert im T0-Modell} & \textbf{Bedeutung} \\
			\midrule
			Planck-Länge & \(l_P\) & \(\sqrt{\frac{\hbar G}{c^3}}\) & 1 & Fundamentale Längeneinheit \\
			Planck-Zeit & \(t_P\) & \(\sqrt{\frac{\hbar G}{c^5}}\) & 1 & Fundamentale Zeiteinheit \\
			Planck-Masse & \(m_P\) & \(\sqrt{\frac{\hbar c}{G}}\) & 1 & Fundamentale Masseneinheit \\
			Planck-Energie & \(E_P\) & \(\sqrt{\frac{\hbar c^5}{G}}\) & 1 & Fundamentale Energieeinheit \\
			Planck-Temperatur & \(T_P\) & \(\frac{\sqrt{\frac{\hbar c^5}{G}}}{k_B}\) & 1 & Fundamentale Temperatureinheit \\
			Planck-Druck & \(p_P\) & \(\frac{c^7}{\hbar G^2}\) & 1 & Fundamentale Druckeinheit \\
			Planck-Dichte & \(\rho_P\) & \(\frac{c^5}{\hbar G^2}\) & 1 & Fundamentale Dichteeinheit \\
			Planck-Ladung & \(q_P\) & \(\sqrt{4\pi \varepsilon_0 \hbar c}\) & 1 & Fundamentale Ladungseinheit \\
			\bottomrule
		\end{tabular}
	\end{adjustbox}
	\caption{Planck-Einheiten im T0-Modell}
	\label{tab:planck_units}
\end{table}

Die philosophischen Implikationen der Planck-Einheiten als fundamentale Grenzen werden in \cite{pascher_planck_2025} diskutiert.

\subsection*{Längenskalen mit hierarchischen Beziehungen}

\begin{table}[H]
	\centering
	\begin{adjustbox}{width=\textwidth}
		\begin{tabular}{lccc}
			\toprule
			\textbf{Physikalische Struktur} & \textbf{Mit \(l_P = 1\)} & \textbf{Mit \(r_0 = 1\)} & \textbf{Hierarchische Beziehung} \\
			\midrule
			Planck-Länge (\(l_P\)) & 1 & \(\frac{l_P}{r_0} = \frac{1}{\xi} \approx 7519\) & Basiseinheit \\
			T0-Länge (\(r_0\)) & \(\frac{r_0}{l_P} = \xi \approx 1,33 \times 10^{-4}\) & 1 & \(\xi \cdot l_P = \frac{\lambda_h}{32\pi^3} \cdot l_P\) \\
			Starke-Wechselwirkungs-Skala & \(\sim 10^{-19}\) & \(\sim 10^{-15}\) & \(\sim \alpha_s \cdot \lambda_{C,h}\) \\
			Higgs-Länge (\(\lambda_{C,h}\)) & \(\sim 1,6 \times 10^{-20}\) & \(\sim 1,2 \times 10^{-16}\) & \(\frac{m_P}{m_h} \cdot l_P\) \\
			Protonenradius & \(\sim 5,2 \times 10^{-20}\) & \(\sim 3,9 \times 10^{-16}\) & \(\sim \frac{\alpha_s}{2\pi} \cdot \lambda_{C,p}\) \\
			Elektronenradius (\(r_e\)) & \(\sim 2,4 \times 10^{-23}\) & \(\sim 1,8 \times 10^{-19}\) & \(\frac{\alphaEMSI}{2\pi} \cdot \lambda_{C,e}\) \\
			Compton-Länge (\(\lambda_{C,e}\)) & \(\sim 2,1 \times 10^{-23}\) & \(\sim 1,6 \times 10^{-19}\) & \(\frac{m_P}{m_e} \cdot l_P\) \\
			Bohr-Radius (\(a_0\)) & \(\sim 4,2 \times 10^{-23}\) & \(\sim 3,2 \times 10^{-19}\) & \(\frac{\lambda_{C,e}}{\alphaEMSI} = \frac{m_P}{\alphaEMSI \cdot m_e} \cdot l_P\) \\
			DNA-Breite & \(\sim 1,2 \times 10^{-26}\) & \(\sim 9,0 \times 10^{-23}\) & \(\sim \lambda_{C,e} \cdot \frac{m_e}{m_{\mathrm{DNA}}}\) \\
			Zelle & \(\sim 6,2 \times 10^{-30}\) & \(\sim 4,7 \times 10^{-26}\) & \(\sim 10^7 \cdot \text{DNA-Breite}\) \\
			Mensch & \(\sim 6,2 \times 10^{-35}\) & \(\sim 4,7 \times 10^{-31}\) & \(\sim 10^5 \cdot \text{Zelle}\) \\
			Erdradius & \(\sim 3,9 \times 10^{-41}\) & \(\sim 2,9 \times 10^{-37}\) & \(\sim \left(\frac{m_P}{m_{\mathrm{Erde}}}\right)^2 \cdot l_P\) \\
			Sonnenradius & \(\sim 4,3 \times 10^{-43}\) & \(\sim 3,2 \times 10^{-39}\) & \(\sim \left(\frac{m_P}{m_{\mathrm{Sonne}}}\right)^2 \cdot l_P\) \\
			Sonnensystem & \(\sim 6,2 \times 10^{-47}\) & \(\sim 4,7 \times 10^{-43}\) & \(\sim \alpha_G^{-1/2} \cdot \text{Sonnenradius}\) \\
			Galaxie & \(\sim 6,2 \times 10^{-56}\) & \(\sim 4,7 \times 10^{-52}\) & \(\sim \left(\frac{m_P}{m_{\mathrm{Galaxie}}}\right)^2 \cdot l_P\) \\
			Cluster & \(\sim 6,2 \times 10^{-58}\) & \(\sim 4,7 \times 10^{-54}\) & \(\sim 10^2 \cdot \text{Galaxie}\) \\
			Horizont (\(d_H\)) & \(\sim 5,4 \times 10^{61}\) & \(\sim 4,1 \times 10^{65}\) & \(\sim \frac{1}{H_0} = \frac{c}{H_0}\) \\
			Korrelationslänge (\(L_T\)) & \(\sim 3,9 \times 10^{62}\) & \(\sim 2,9 \times 10^{66}\) & \(\sim \betaT^{-1/4} \cdot \xi^{-1/2} \cdot l_P\) \\
			\bottomrule
		\end{tabular}
	\end{adjustbox}
	\caption{Längenskalen mit hierarchischen Beziehungen}
	\label{tab:length_scales}
\end{table}

Die hierarchische Struktur der Längenskalen wird in \cite{pascher_galaxies_2025} und \cite{pascher_nateinhsystem_2025} detailliert analysiert.

\subsection*{Quantisierte Längenskalen und verbotene Zonen}

Die bevorzugten Längenskalen im T0-Modell folgen dem Muster \cite{pascher_planck_2025}:
\[
L_n = l_P \times \prod \alpha_i^{n_i}
\]
wobei:
\begin{itemize}
	\item \(\alpha_i = \text{dimensionslose Konstanten} \, (\alphaEM, \betaT, \xi)\)
	\item \(n_i = \text{ganzzahlige oder rationale Exponenten}\)
\end{itemize}

Diese Quantisierung führt zu bevorzugten Skalen und verbotenen Zonen, wie in \cite{pascher_planck_2025} und \cite{pascher_quantum_2025} detailliert beschrieben.

\subsection*{Biologische Anomalien in der Längenskalenhierarchie}

Eine bemerkenswerte Entdeckung im T0-Modell ist, dass biologische Strukturen bevorzugt in ''verbotenen Zonen'' der Längenskala existieren \cite{pascher_quantum_2025, pascher_bio_2025}:

\begin{table}[H]
	\centering
	\begin{tabular}{lccc}
		\toprule
		\textbf{Biologische Struktur} & \textbf{Typische Größe} & \textbf{Verhältnis zu \(l_P\)} & \textbf{Position} \\
		\midrule
		DNA-Durchmesser & \(\sim 2 \times 10^{-9} \, \text{m}\) & \(\sim 10^{-26}\) & Verbotene Zone \\
		Protein & \(\sim 10^{-8} \, \text{m}\) & \(\sim 10^{-27}\) & Verbotene Zone \\
		Bakterium & \(\sim 10^{-6} \, \text{m}\) & \(\sim 10^{-29}\) & Verbotene Zone \\
		Typische Zelle & \(\sim 10^{-5} \, \text{m}\) & \(\sim 10^{-30}\) & Verbotene Zone \\
		Mehrzelliger Organismus & \(\sim 10^{-3} - 10^{0} \, \text{m}\) & \(\sim 10^{-32} - 10^{-35}\) & Verbotene Zone \\
		\bottomrule
	\end{tabular}
	\caption{Biologische Strukturen in verbotenen Zonen}
	\label{tab:biological_anomalies}
\end{table}

Diese ''verbotenen Zonen'' liegen zwischen den bevorzugten quantisierten Längenskalen und bilden Lücken von oft mehreren Größenordnungen:
\begin{itemize}
	\item Zwischen \(10^{-30} \, \text{m}\) und \(10^{-23} \, \text{m}\) (zwischen T0-Länge und Compton-Wellenlänge)
	\item Zwischen \(10^{-9} \, \text{m}\) und \(10^{-6} \, \text{m}\) (zwischen molekularer und zellulärer Ebene)
	\item Zwischen \(10^{-3} \, \text{m}\) und \(10^{0} \, \text{m}\) (makroskopischer Bereich, in dem biologische Organismen dominieren)
\end{itemize}

Diese Anomalie kann durch spezielle Stabilisierungsmechanismen erklärt werden, die biologischen Systemen erlauben, in diesen verbotenen Zonen zu existieren:

\begin{enumerate}
	\item \textbf{Informationsbasierte Stabilisierung}: Biologische Strukturen nutzen genetische und epigenetische Informationen, wie in \cite{pascher_bio_2025} erklärt.
	\item \textbf{Topologische Stabilisierung}: Biologische Systeme weisen oft topologisch geschützte Konfigurationen auf, wie in \cite{pascher_quantum_2025} detailliert dargestellt.
	\item \textbf{Dynamische Stabilisierung}: Betrieb fernab vom thermodynamischen Gleichgewicht, analysiert in \cite{pascher_galaxies_2025}.
\end{enumerate}

Im T0-Modell wird dies durch modifizierte Zeitfeldgleichungen formalisiert:
\[
\nabla^2 \Tfield_{\mathrm{bio}} \approx -\frac{\rho}{\Tfield^2} + \delta_{\mathrm{bio}}(x,t)
\]
wobei \(\delta_{\mathrm{bio}}\) einen biologischen Korrekturterm darstellt, der Stabilität in verbotenen Zonen ermöglicht.

\section*{Teil 2: Detaillierte Erläuterungen und Herleitungen}

\subsection*{Dimensionsanalyse und Herleitung der Einstein-Hilbert-Wirkung im T0-Modell}

\subsubsection*{1. Ursprüngliche Form in SI-Einheiten}

In der Allgemeinen Relativitätstheorie lautet die Einstein-Hilbert-Wirkung in SI-Einheiten:

\[
S_{\mathrm{EH}} = \frac{c^4}{16\pi G} \int R \sqrt{-g} \, d^4x
\]

wobei:
\begin{itemize}
	\item $c$ die Lichtgeschwindigkeit ist
	\item $G$ die Gravitationskonstante ist
	\item $R$ der Ricci-Skalar mit der Dimension $[L^{-2}]$ (Krümmung) ist
	\item $\sqrt{-g} \, d^4x$ das Raumzeit-Volumenelement mit der Dimension $[L^4]$ ist
	\item $\frac{c^4}{16\pi G}$ der Vorfaktor mit der Dimension $[L^{-1} M]$ ist
\end{itemize}

Die Dimension der gesamten Wirkung ist:
\[
[L^{-2}] \cdot [L^4] \cdot [L^{-1} M] = [L M]
\]

was der Dimension von Energie $\times$ Zeit entspricht und in SI-Einheiten mit der physikalischen Dimension einer Wirkung (z.B. $\hbar$) übereinstimmt.

\subsubsection*{2. Übergang zum T0-Modell mit natürlichen Einheiten}

Im T0-Modell lauten die grundlegenden Annahmen:
\begin{itemize}
	\item $\hbar = 1$ (Normierung der Wirkung)
	\item $c = 1$ (vereinheitlicht Raum und Zeit)
	\item $G = 1$ (vereinheitlicht Gravitationsphysik mit anderen Wechselwirkungen)
\end{itemize}

Mit Energie $[E]$ als Basiseinheit sind die Dimensionen:
\begin{itemize}
	\item Länge: $[L] = [E^{-1}]$
	\item Zeit: $[T] = [E^{-1}]$
	\item Masse: $[M] = [E]$
\end{itemize}

Somit hat der Ricci-Skalar $R$ die Dimension $[L^{-2}] = [E^2]$.

Das Volumenelement $\sqrt{-g} \, d^4x$ hat die Dimension $[L^4] = [E^{-4}]$.

Das Integral $R\sqrt{-g} \, d^4x$ hat somit die Dimension $[E^2] \cdot [E^{-4}] = [E^{-2}]$.

\subsubsection*{3. Der Vorfaktor im natürlichen System}

Im T0-Modell transformiert sich der Vorfaktor $\frac{c^4}{16\pi G}$ zu:
\begin{itemize}
	\item In SI-Einheiten hat er die Dimension $[L^{-1} M]$
	\item Dies entspricht in natürlichen Einheiten $[E^{-1} \cdot E] = [E^0] = 1$
\end{itemize}

Der numerische Wert wird zu $\frac{1}{16\pi}$ aufgrund der Einstellungen $c = G = 1$.

Die Wirkung nimmt die Form an:
\[
S_{\mathrm{EH}} = \frac{1}{16\pi} \int R \sqrt{-g} \, d^4x
\]

Die Dimension dieser Wirkung im T0-Modell ist:
\[
[1] \cdot [E^{-2}] \cdot [E^2] = [E^0] = 1
\]

Diese dimensionslose Wirkung steht im Einklang mit dem Ansatz in \cite{pascher_emergente_gravitation_2025}.

\subsubsection*{4. Feldgleichungen im T0-Modell}

Die Variation der Einstein-Hilbert-Wirkung führt zu den Feldgleichungen:
\[
R_{\mu\nu} - \frac{1}{2}Rg_{\mu\nu} = 8\pi T_{\mu\nu}
\]

wobei der Faktor $8\pi$ direkt aus dem Vorfaktor $\frac{1}{16\pi}$ der Wirkung resultiert. Der Energie-Impuls-Tensor $T_{\mu\nu}$ im T0-Modell hat die Dimension $[E^2]$ (Energie pro Volumen).

\subsubsection*{5. Verbindung zum modifizierten Gravitationspotential}

Die Verbindung zwischen dem modifizierten Potential $\Phi(r) = -\frac{GM}{r} + \kappa r$ und der Einstein-Hilbert-Wirkung entsteht durch folgende Herleitung:

\begin{enumerate}
	\item Das modifizierte Potential kann als Lösung einer modifizierten Poisson-Gleichung dargestellt werden:
	\[
	\nabla^2\Phi = 4\pi G\rho - 2\kappa
	\]
	
	\item In der Allgemeinen Relativitätstheorie entspricht eine solche Modifikation einem Energie-Impuls-Tensor, der einen Term äquivalent zu einer kosmologischen Konstante enthält:
	\[
	T_{\mu\nu} = T_{\mu\nu}(\text{Materie}) + \Lambda_{\text{eff}} \cdot g_{\mu\nu}
	\]
	wobei $\Lambda_{\text{eff}} = \frac{\kappa}{G}$ eine effektive kosmologische Konstante darstellt.
	
	\item Dieser zusätzliche Term in der Einstein-Gleichung entspricht einem zusätzlichen Term in der Einstein-Hilbert-Wirkung:
	\[
	S_{\mathrm{EH}} = \frac{1}{16\pi G}\int(R - 2\Lambda_{\text{eff}})\sqrt{-g}d^4x
	\]
	
	\item In natürlichen Einheiten mit $G = 1$ wird dies zu:
	\[
	S_{\mathrm{EH}} = \frac{1}{16\pi}\int(R - 2\kappa)\sqrt{-g}d^4x
	\]
	
	\item Die Variation dieser modifizierten Wirkung führt zu den Feldgleichungen:
	\[
	R_{\mu\nu} - \frac{1}{2}Rg_{\mu\nu} + \kappa g_{\mu\nu} = 8\pi T_{\mu\nu}
	\]
	
	\item In der Schwachfeld-Näherung ergibt dies genau das modifizierte Potential:
	\[
	ds^2 = -(1+2\Phi)dt^2 + (1-2\Phi)(dx^2 + dy^2 + dz^2)
	\]
	mit $\Phi(r) = -\frac{M}{r} + \frac{\kappa r}{2}$ (mit $G = 1$).
\end{enumerate}

Diese Herleitung wird in \cite{pascher_emergente_gravitation_2025} detailliert dargestellt.

\subsection*{Verbindung zur beobachteten dunklen Energie}

Der lineare Term $\kappa r$ im Gravitationspotential entspricht einer effektiven kosmologischen Konstante $\Lambda_{\text{eff}} = \frac{\kappa}{G}$. Dies hat wichtige Implikationen für die beobachtete dunkle Energie \cite{pascher_energiedynamik_2025}:

\begin{enumerate}
	\item Die gemessene Energiedichte der dunklen Energie beträgt etwa $\rho_\Lambda \approx 10^{-123}$ in Planck-Einheiten.
	
	\item Im T0-Modell entsteht dieser Wert natürlich als Folge des Parameters $\kappa \approx 4,8 \times 10^{-11}$ m/s²:
	\[
	\rho_\Lambda = \frac{\Lambda_{\text{eff}}}{8\pi G} = \frac{\kappa}{8\pi G^2} \approx 10^{-123} m_P^4
	\]
	
	\item Diese Übereinstimmung löst auf natürliche Weise das Problem der kosmologischen Konstante, da $\kappa$ keine Feinabstimmung erfordert, sondern aus der fundamentalen Struktur des T0-Modells hervorgeht:
	\[
	\kappa^{\text{nat}} = \betaT^{\text{nat}} \cdot \frac{yv}{r_g^2}\betaT^{\text{nat}} \cdot \frac{yv}{r_g^2}
	\]
	wobei $L_T$ die kosmologische Korrelationslänge ist.
\end{enumerate}

Diese Formulierung erklärt sowohl beobachtete Galaxienrotationskurven als auch kosmische Beschleunigung ohne zusätzliche dunkle Komponenten einzuführen und ermöglicht direkten experimentellen Vergleich mit MOND (Modifizierte Newtonsche Dynamik) und f(R)-Gravitationstheorien.

\subsection*{Herleitung der Gravitation im natürlichen System des T0-Modells}

Im T0-Modell wird die Gravitation nicht als fundamentale Eigenschaft postuliert, sondern direkt aus dem intrinsischen Zeitfeld $T(x)$ abgeleitet \cite{pascher_emergente_gravitation_2025}:

\begin{enumerate}
	\item \textbf{Fundamentale Herleitung:} Gravitation entsteht aus Gradienten des intrinsischen Zeitfeldes:
	\[
	\nabla T(x) = -\frac{\hbar}{m^2c^2} \cdot \nabla m
	\]
	
	\item \textbf{Verbindung zur Einstein-Hilbert-Wirkung:} Im natürlichen System mit $\hbar = c = G = 1$ kann gezeigt werden, dass das effektive Gravitationspotential $\Phi(x)$ mit dem Zeitfeld verbunden ist durch:
	\[
	\Phi(x) = -\ln\left(\frac{T(x)}{T_0}\right)
	\]
	wobei $T_0$ ein Referenzwert des Zeitfeldes ist.
	
	\item \textbf{Emergente Feldgleichungen:} Die Dynamik des Zeitfeldes führt zu modifizierten Feldgleichungen, die einer modifizierten Einstein-Hilbert-Wirkung äquivalent sind:
	\[
	\nabla^2T(x) \approx -\frac{\rho}{T(x)^2}
	\]
	Diese Gleichung ist in der Schwachfeld-Näherung einer modifizierten Poisson-Gleichung äquivalent und erzeugt den linearen Term $\kappa r$.
	
	\item \textbf{Einheitenbeziehung:} Im natürlichen Einheitensystem des T0-Modells haben alle Terme in der Einstein-Hilbert-Wirkung die Dimension $[E^0]$, d.h. sie sind dimensionslos. Dies ergibt sich aus:
	\begin{itemize}
		\item Ricci-Skalar $R$: $[E^2]$
		\item Determinante $\sqrt{-g}$: dimensionslos
		\item Volumenelement $d^4x$: $[E^{-4}]$
		\item Vorfaktor $\frac{1}{16\pi}$: dimensionslos
	\end{itemize}
\end{enumerate}

Die Einzigartigkeit des T0-Modells liegt darin, dass die Einstein-Hilbert-Wirkung und die Allgemeine Relativitätstheorie als effektive Beschreibungen der Gravitation erscheinen, während die fundamentalere Beschreibung durch das intrinsische Zeitfeld gegeben ist. Dies ermöglicht eine vereinheitlichte Behandlung der Gravitation mit anderen Wechselwirkungen und erklärt beobachtete Anomalien in der Galaxiendynamik ohne die Annahme dunkler Materie.

\subsection*{Vergleich mit etablierten Gravitationstheorien}

Das T0-Modell bietet eine Alternative zu etablierten Gravitationstheorien und kann direkt mit ihnen verglichen werden:

\begin{table}[H]
	\centering
	\begin{tabular}{p{3cm}p{3cm}p{4cm}p{4cm}}
		\toprule
		\textbf{Theorie} & \textbf{Prinzip} & \textbf{Modifiziertes Potential} & \textbf{Vergleich mit T0} \\
		\midrule
		Newtonsche Gravitation & Kraft zwischen Massen & $\Phi(r) = -\frac{GM}{r}$ & Spezialfall von T0 für $\kappa=0$ \\
		Allgemeine Relativitätstheorie & Raumzeitkrümmung & Schwarzschild-Lösung & Phänomenologisch äquivalent in schwachen Feldern \\
		MOND & Modifizierte Dynamik bei niedriger Beschleunigung & $\Phi(r)$ erfüllt: $\nabla^2\Phi = 4\pi G\rho\cdot\mu(\frac{\nabla\Phi}{a_0})$ & T0 bietet eine fundamentalere Basis für MOND-Effekte \\
		f(R)-Theorien & Modifizierte Gravitationswirkung & Abhängig von spezifischer f(R)-Funktion & T0 entspricht f(R) = R - 2$\kappa\cdot$G für schwache Felder \\
		T0-Modell & Emergente Gravitation aus Zeitfeld & $\Phi(r) = -\frac{GM}{r} + \kappa r$ & Vereinigt Quantenmechanik und Gravitation \\
		\bottomrule
	\end{tabular}
	\caption{Vergleich des T0-Modells mit etablierten Gravitationstheorien}
	\label{tab:theory_comparison}
\end{table}

Diese Vergleiche werden in \cite{pascher_galaxies_2025} und \cite{pascher_emergente_gravitation_2025} detailliert dargestellt.

Das T0-Modell bietet folgende Vorteile gegenüber diesen Theorien:

\begin{enumerate}
	\item \textbf{Vereinheitlichte Behandlung von Quanten- und makroskopischer Physik} durch das intrinsische Zeitfeld $T(x)$
	\item \textbf{Natürliche Erklärung für Galaxiendynamik} ohne Annahme dunkler Materie
	\item \textbf{Lösung des Problems der kosmologischen Konstante} durch Ableitung von $\kappa$ aus fundamentalen Parametern
	\item \textbf{Mathematische Konsistenz} mit Quantenfeldtheorie und dem Standardmodell durch modifizierte Lagrange-Dichten
	\item \textbf{Testbare Vorhersagen} für Abweichungen vom 1/r-Potential auf verschiedenen Skalen
\end{enumerate}

Experimentelle Tests zur Unterscheidung zwischen diesen Theorien umfassen:
\begin{itemize}
	\item Präzisionsmessungen der planetaren Periheldrehung
	\item Gravitationslinseneffekte in fernen Galaxien
	\item Satellitenmessungen der Pioneer-Anomalie
	\item Beobachtungen von Galaxienrotationskurven über verschiedene Morphologien hinweg
\end{itemize}

\subsection*{Praktische Äquivalente in Energieeinheiten}

\textbf{Wichtiger Hinweis}: Die Energieeinheit ''Elektronvolt'' (abgekürzt ''eV'') darf nicht mit der SI-Einheit ''Volt'' (abgekürzt ''V'') verwechselt werden. Im T0-Modell mit natürlichen Einheiten wird das Elektronvolt als fundamentale Energieeinheit verwendet, aus der andere Einheiten abgeleitet werden.

\begin{itemize}
	\item \textbf{Länge:} (eV)$^{-1}$, (GeV)$^{-1}$, (TeV)$^{-1}$
	\item \textbf{Zeit:} (eV)$^{-1}$, (GeV)$^{-1}$, (TeV)$^{-1}$
	\item \textbf{Masse/Energie:} eV, MeV, GeV, TeV
	\item \textbf{Temperatur:} eV, MeV
	\item \textbf{Impuls:} eV/c, GeV/c (wobei $c=1$ in natürlichen Einheiten)
	\item \textbf{Wirkungsquerschnitt:} (GeV)$^{-2}$, mb, pb, fb
	\item \textbf{Zerfallsrate:} eV, MeV
\end{itemize}

Im T0-Modell werden Längenskalen oft als inverse Energien ausgedrückt, was die fundamentale Beziehung zwischen Energie und Länge in natürlichen Einheiten widerspiegelt (Länge $\sim$ 1/Energie).

\subsection*{Umrechnung gängiger SI-Einheiten in T0-Modell-Einheiten}

Gängige SI-Einheiten können auf Energie als Basiseinheit im T0-Modell reduziert werden. Dies ermöglicht die Darstellung aller physikalischen Größen in einem einheitlichen System:

\begin{table}[H]
	\centering
	\begin{adjustbox}{width=\textwidth}
		\begin{tabular}{lcccc}
			\toprule
			\textbf{SI-Einheit} & \textbf{Dimension im SI-System} & \textbf{T0-Modell-Äquivalent} & \textbf{Umrechnungsbeziehung} & \textbf{Typische Messgenauigkeit} \\
			\midrule
			Meter (m) & $[L]$ & $[E^{-1}]$ & 1 m $\leftrightarrow$ (197 MeV)$^{-1}$ & $<$ 0,001\% \\
			Sekunde (s) & $[T]$ & $[E^{-1}]$ & 1 s $\leftrightarrow$ (6,58 $\times$ 10$^{-22}$ MeV)$^{-1}$ & $<$ 0,00001\% \\
			Kilogramm (kg) & $[M]$ & $[E]$ & 1 kg $\leftrightarrow$ 5,61 $\times$ 10$^{26}$ MeV & $<$ 0,001\% \\
			Ampere (A) & $[I]$ & $[E]$ & 1 A $\leftrightarrow$ Ladung pro Zeit $\leftrightarrow$ $[E^2]$ & $<$ 0,005\% \\
			Kelvin (K) & $[\Theta]$ & $[E]$ & 1 K $\leftrightarrow$ 8,62 $\times$ 10$^{-5}$ eV & $<$ 0,01\% \\
			Volt (V) & $[ML^2T^{-3}I^{-1}]$ & $[E]$ & 1 V $\leftrightarrow$ 1 eV/e (mit $e = \sqrt{4\pi}$) & $<$ 0,0001\% \\
			Tesla (T) & $[MT^{-2}I^{-1}]$ & $[E^2]$ & 1 T $\leftrightarrow$ Energie pro Fläche & $<$ 0,01\% \\
			Pascal (Pa) & $[ML^{-1}T^{-2}]$ & $[E^4]$ & 1 Pa $\leftrightarrow$ Energie pro Volumen & $<$ 0,005\% \\
			Watt (W) & $[ML^2T^{-3}]$ & $[E^2]$ & 1 W $\leftrightarrow$ Energie pro Zeit & $<$ 0,001\% \\
			Coulomb (C) & $[TI]$ & $[1]$ & 1 C $\leftrightarrow$ e/$\sqrt{4\pi}$ & $<$ 0,0001\% \\
			Ohm ($\Omega$) & $[ML^2T^{-3}I^{-2}]$ & $[E^{-1}]$ & 1 $\Omega$ $\leftrightarrow$ h/e$^2$ = 1/2 (mit h=2$\pi$, e=$\sqrt{4\pi}$) & $<$ 0,0000001\% \\
			Farad (F) & $[M^{-1}L^{-2}T^4I^2]$ & $[E^{-1}]$ & 1 F $\leftrightarrow$ inverse Energie & $<$ 0,01\% \\
			Henry (H) & $[ML^2T^{-2}I^{-2}]$ & $[E^{-1}]$ & 1 H $\leftrightarrow$ inverse Energie & $<$ 0,01\% \\
			\bottomrule
		\end{tabular}
	\end{adjustbox}
	\caption{Umrechnung von SI-Einheiten in T0-Modell-Einheiten}
	\label{tab:conversion}
\end{table}

Diese Umrechnungsfaktoren werden in \cite{pascher_conversions_2025} hergeleitet und verifiziert.

\subsection*{Besondere Rolle der elektrischen Ladung (Coulomb)}

Die Coulomb-Einheit nimmt im T0-Modell eine besondere Stellung ein, da sie die direkteste Verbindung zu den elektromagnetischen Konstanten $\mu_0$ und $\varepsilon_0$ bietet. Mit $\alphaEM = \frac{e^2}{4\pi\varepsilon_0\hbar c} = 1$ im T0-Modell folgt:

\[
e^2 = 4\pi\varepsilon_0\hbar c
\]

Da $\hbar = c = \varepsilon_0 = 1$ im T0-Modell, erhalten wir:
\[
e^2 = 4\pi
\]
\[
e = \sqrt{4\pi} \approx 3,5
\]

Mit $\varepsilon_0\mu_0c^2 = 1$ und $c = 1$ folgt weiter:
\[
\varepsilon_0\mu_0 = 1
\]

Diese Beziehungen verleihen der elektrischen Ladung eine besondere Bedeutung im T0-Modell. Der Wert $e = \sqrt{4\pi}$ ist eine natürliche Konsequenz der Normierung $\alphaEM = 1$ und steht im Einklang mit den Maxwell-Gleichungen in ihrer einfachsten Form.

Die Auswirkungen der Normierung $e = \sqrt{4\pi}$ sind:
\begin{enumerate}
	\item Elektrische Ladungen werden in Einheiten von $\sqrt{4\pi}$ gemessen
	\item Elektrische und magnetische Felder können in reinen Energieeinheiten ausgedrückt werden
	\item Die Maxwell-Gleichungen nehmen ihre eleganteste Form an
\end{enumerate}

Diese natürliche Darstellung offenbart die tiefe Verbindung zwischen Elektromagnetismus und der fundamentalen Energiestruktur des Universums, wie in \cite{pascher_alpha_2025} detailliert dargestellt.

\subsection*{Abschließende Bemerkungen zur Vollständigkeit und Genauigkeit des T0-Modells}

Eine zentrale Stärke des T0-Modells ist, dass \textbf{alle SI-Einheiten} vollständig und präzise in diesem System abgebildet werden können. Es ist kein approximatives oder vereinfachtes System, sondern eine fundamentalere Darstellung der physikalischen Realität.

Die scheinbaren ''Abweichungen'' zwischen Messungen im SI-System und den theoretischen Vorhersagen des T0-Modells sind keine Fehler des natürlichen Einheitensystems, sondern spiegeln Ungenauigkeiten in der Messauswertung und der zugrunde liegenden Metrologie des SI-Systems wider. Diese Abweichungen sind in den meisten Fällen extrem klein:

\begin{table}[H]
	\centering
	\begin{adjustbox}{width=0.95\textwidth}
		\begin{tabular}{lcc}
			\toprule
			\textbf{Bereich} & \textbf{Typische Abweichung} & \textbf{Hinweis} \\
			\midrule
			Atomare Skala & $\sim10^{-9}$ bis $10^{-8}$ & Extrem hohe Übereinstimmung (0,0000001\% - 0,000001\%) \\
			Nukleare Skala & $\sim10^{-7}$ bis $10^{-6}$ & Sehr hohe Übereinstimmung (0,00001\% - 0,0001\%) \\
			Makroskopische Skala & $\sim10^{-5}$ bis $10^{-4}$ & Hohe Übereinstimmung (0,001\% - 0,01\%) \\
			Astronomische Skala & $\sim10^{-3}$ bis $10^{-2}$ & Gute Übereinstimmung (0,1\% - 1\%) \\
			Kosmologische Skala & $\sim10^{-2}$ bis $10^{-1}$ & Größere Abweichungen (1\% - 10\%) \\
			\bottomrule
		\end{tabular}
	\end{adjustbox}
	\caption{Abweichungen zwischen SI-System und T0-Modell}
	\label{tab:deviations}
\end{table}

Die größeren Abweichungen in kosmologischen Dimensionen sind nicht auf Mängel im T0-Modell zurückzuführen, sondern auf grundlegende Herausforderungen in kosmologischen Messtechniken und der Interpretation von Beobachtungsdaten im Kontext des konventionellen kosmologischen Standardmodells, wie in \cite{pascher_messdifferenzen_2025} analysiert.

Das T0-Modell mit seinem System natürlicher Einheiten bietet nicht nur einen mathematisch eleganteren und physikalisch fundamentaleren Rahmen, sondern ermöglicht auch neue Einsichten in die Struktur des Universums, die im SI-System verborgen bleiben. Die quantisierte Struktur der Längenskalen, die besondere Rolle biologischer Systeme und die vereinheitlichte Behandlung aller Wechselwirkungen sind Aspekte, die ihre Bedeutung erst im T0-Modell vollständig entfalten.

\begin{thebibliography}{99}
	\bibitem{pascher_zeit_2025} Pascher, J. (2025). \href{https://github.com/jpascher/T0-Time-Mass-Duality/tree/main/2/pdf/Deutsch/ZeitEmergentQM.pdf}{Zeit als emergente Eigenschaft in der Quantenmechanik: Eine Verbindung zwischen Relativität, Feinstrukturkonstante und Quantendynamik}. 23. März 2025.
	\bibitem{pascher_lagrange_2025} Pascher, J. (2025). \href{https://github.com/jpascher/T0-Time-Mass-Duality/tree/main/2/pdf/Deutsch/MathZeitMasseLagrange.pdf}{Von Zeitdilatation zur Massenvariation: Mathematische Kernformulierungen der Zeit-Masse-Dualitätstheorie}. 29. März 2025.
	\bibitem{pascher_photons_2025} Pascher, J. (2025). \href{https://github.com/jpascher/T0-Time-Mass-Duality/tree/main/2/pdf/Deutsch/DynMassePhotonenNichtlokal.pdf}{Dynamische Masse von Photonen und ihre Implikationen für Nichtlokalität im T0-Modell}. 25. März 2025.
	\bibitem{pascher_erweiterung_2025} Pascher, J. (2025). \href{https://github.com/jpascher/T0-Time-Mass-Duality/tree/main/2/pdf/Deutsch/NotwendigkeitQMErweiterung.pdf}{Die Notwendigkeit der Erweiterung der Standard-Quantenmechanik und Quantenfeldtheorie}. 27. März 2025.
	\bibitem{pascher_galaxies_2025} Pascher, J. (2025). \href{https://github.com/jpascher/T0-Time-Mass-Duality/tree/main/2/pdf/Deutsch/MassVarGalaxien.pdf}{MassenVariation in Galaxien: Eine Analyse im T0-Modell mit emergenter Gravitation}. 30. März 2025.
	\bibitem{pascher_higgs_2025} Pascher, J. (2025). \href{https://github.com/jpascher/T0-Time-Mass-Duality/tree/main/2/pdf/Deutsch/MathHiggsZeitMasse.pdf}{Mathematische Formulierung des Higgs-Mechanismus in der Zeit-Masse-Dualität}. 28. März 2025.
	\bibitem{pascher_feldtheorie_2025} Pascher, J. (2025). \href{https://github.com/jpascher/T0-Time-Mass-Duality/tree/main/2/pdf/Deutsch/FeldtheorieQuanten.pdf}{Feldtheorie und Quantenkorrelationen: Eine neue Perspektive auf Instantaneität}. 28. März 2025.
	\bibitem{pascher_messdifferenzen_2025} Pascher, J. (2025). \href{https://github.com/jpascher/T0-Time-Mass-Duality/tree/main/2/pdf/Deutsch/MessdifferenzenT0Standard.pdf}{Kompensatorische und additive Effekte: Eine Analyse der Messunterschiede zwischen dem T0-Modell und dem \(\Lambda\)CDM-Standardmodell}. 2. April 2025.
	\bibitem{pascher_planck_2025} Pascher, J. (2025). \href{https://github.com/jpascher/T0-Time-Mass-Duality/tree/main/2/pdf/Deutsch/JenseitsPlanck.pdf}{Reale Konsequenzen der Neuformulierung von Zeit und Masse in der Physik: Jenseits der Planck-Skala}. 24. März 2025.
	\bibitem{pascher_alpha_2025} Pascher, J. (2025). \href{https://github.com/jpascher/T0-Time-Mass-Duality/tree/main/2/pdf/Deutsch/NatEinheitenAlpha1.pdf}{Energie als fundamentale Einheit: Natürliche Einheiten mit \(\alphaEM = 1\) im T0-Modell}. 26. März 2025.
	\bibitem{pascher_alphabeta_2025} Pascher, J. (2025). \href{https://github.com/jpascher/T0-Time-Mass-Duality/tree/main/2/pdf/Deutsch/Alpha1Beta1Konsistenz.pdf}{Einheitliches Einheitensystem im T0-Modell: Die Konsistenz von \(\alpha = 1\) und \(\beta = 1\)}. 5. April 2025.
	\bibitem{pascher_temp_2025} Pascher, J. (2025). \href{https://github.com/jpascher/T0-Time-Mass-Duality/tree/main/2/pdf/Deutsch/TempEinheitenCMB.pdf}{Anpassung der Temperatureinheiten in natürlichen Einheiten und CMB-Messungen}. 2. April 2025.
	\bibitem{pascher_params_2025} Pascher, J. (2025). \href{https://github.com/jpascher/T0-Time-Mass-Duality/tree/main/2/pdf/Deutsch/ZeitMasseT0Params.pdf}{Zeit-Masse-Dualitätstheorie (T0-Modell): Ableitung der Parameter \(\kappa\), \(\alpha\) und \(\beta\)}. 4. April 2025.
	\bibitem{pascher_emergente_gravitation_2025} Pascher, J. (2025). \href{https://github.com/jpascher/T0-Time-Mass-Duality/tree/main/2/pdf/Deutsch/EmergentGravT0.pdf}{Emergente Gravitation im T0-Modell: Eine umfassende Ableitung}. 1. April 2025.
	\bibitem{pascher_zeit_masse_2025} Pascher, J. (2025). \href{https://github.com/jpascher/T0-Time-Mass-Duality/tree/main/2/pdf/Deutsch/ZeitMasseNeuerBlick.pdf}{Zeit und Masse: Ein neuer Blick auf alte Formeln – und Befreiung von traditionellen Zwängen}. 22. März 2025.
	\bibitem{pascher_quantum_2025} Pascher, J. (2025). \href{https://github.com/jpascher/T0-Time-Mass-Duality/tree/main/2/pdf/Deutsch/QuantumFormulationT0.pdf}{Quantenformulierung des T0-Modells: Integration der intrinsischen Zeit in die Quantenfeldtheorie}. 31. März 2025.
	\bibitem{pascher_bio_2025} Pascher, J. (2025). \href{https://github.com/jpascher/T0-Time-Mass-Duality/tree/main/2/pdf/Deutsch/BioStabilityT0.pdf}{Biologische Stabilitätsmechanismen im T0-Modell: Warum Leben in verbotenen Zonen existiert}. 3. April 2025.
	\bibitem{pascher_energiedynamik_2025} Pascher, J. (2025). \href{https://github.com/jpascher/T0-Time-Mass-Duality/tree/main/2/pdf/Deutsch/MathEnergiedynamik.pdf}{Dunkle Energie im T0-Modell: Eine mathematische Analyse der Energiedynamik}. 30. März 2025.

%---
\bibitem{pascher_nateinhsystem_2025} Pascher, J. (2025). \href{https://github.com/jpascher/T0-Time-Mass-Duality/tree/main/2/pdf/Deutsch/NatEinheitenSystematik.pdf}{Hierarchisches natürliches Einheitensystem im T0-Modell: Vereinheitlichung der Physik durch energiebasierte Formulierung}. 13. April 2025.
\bibitem{pascher_conversions_2025} Pascher, J. (2025). \href{https://github.com/jpascher/T0-Time-Mass-Duality/tree/main/2/pdf/Deutsch/NatEinheitenKonversion.pdf}{Umrechnung zwischen natürlichen Einheiten und SI-Einheiten im T0-Modell: Praktische Anwendungen und experimentelle Tests}. 10. April 2025.

\bibitem{planck1901} Planck, M. (1901). Über das Gesetz der Energieverteilung im Normalspektrum. \textit{Annalen der Physik}, 4(3), 553-563.
\bibitem{einstein1905} Einstein, A. (1905). Zur Elektrodynamik bewegter Körper. \textit{Annalen der Physik}, 17, 891-921.
\bibitem{newton1687} Newton, I. (1687). \textit{Philosophiæ Naturalis Principia Mathematica}. London: Royal Society.
\bibitem{boltzmann1872} Boltzmann, L. (1872). Weitere Studien über das Wärmegleichgewicht unter Gasmolekülen. \textit{Sitzungsberichte der Akademie der Wissenschaften}, 66, 275-370.
\bibitem{feynman1985} Feynman, R.P. (1985). \textit{QED: Die seltsame Theorie des Lichts und der Materie}. Princeton University Press.
\bibitem{milgrom1983} Milgrom, M. (1983). A Modification of the Newtonian Dynamics as a Possible Alternative to the Hidden Mass Hypothesis. \textit{The Astrophysical Journal}, 270, 365-370.
\bibitem{verlinde2011} Verlinde, E. (2011). On the Origin of Gravity and the Laws of Newton. \textit{Journal of High Energy Physics}, 2011(4), 29.
\bibitem{penrose1996} Penrose, R. (1996). On Gravity's Role in Quantum State Reduction. \textit{General Relativity and Gravitation}, 28(5), 581-600.
\end{thebibliography}

\end{document}