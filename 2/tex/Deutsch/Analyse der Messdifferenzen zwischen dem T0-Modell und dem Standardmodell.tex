\documentclass[a4paper,12pt]{article}
% Pakete
\usepackage[utf8]{inputenc}
\usepackage[T1]{fontenc}
\usepackage{lmodern}
\usepackage[german]{babel} % Sprache auf Deutsch geändert
\usepackage{amsmath, amssymb}
\usepackage{graphicx}
\usepackage{xcolor}
\usepackage{setspace}
\usepackage{booktabs}
\usepackage{siunitx}
\usepackage{array}
\usepackage{float}
\usepackage[section]{placeins}
\usepackage{tcolorbox}
\usepackage{enumitem}
\usepackage[left=2cm,right=2cm,top=2cm,bottom=2cm]{geometry}
\usepackage{fancyhdr}
\usepackage{parskip}
\usepackage{tocloft}
\usepackage{tikz} % Für Diagramme
\usepackage{pgfplots} % Für Diagramme
\usepackage{hyperref}
\pgfplotsset{compat=1.18}
    ALT: \newcommand{\repobase}{https://github.com/jpascher/T0-Time-Mass-Duality/tree/main/2/pdf/Deutsch} % Verzeichnis auf Deutsch geändert
    NEU: \newcommand{\repobase}{https://github.com/jpascher/T0-Time-Mass-Duality/tree/main/2/pdf/Deutsch} % Verzeichnis auf Deutsch ge84ndert
\newcommand{\repobase}{https://github.com/jpascher/T0-Time-Mass-Duality/tree/main/2/pdf/Deutsch} % Verzeichnis auf Deutsch ge84ndert
% Benutzerdefinierte Befehle
\newcommand{\Tfield}{\ensuremath{T(x)}}
\newcommand{\alphaEM}{\ensuremath{\alpha_{\text{EM}}}}
\newcommand{\alphaW}{\ensuremath{\alpha_{\text{W}}}}
\newcommand{\betaT}{\ensuremath{\beta_{\text{T}}}}
\newcommand{\Mpl}{\ensuremath{M_{\text{Pl}}}}
\newcommand{\Tzerot}{\ensuremath{T_0(T(x))}}
\newcommand{\Tzero}{\ensuremath{T_0}}
\newcommand{\LCDM}{\ensuremath{\Lambda}CDM}
% Kopf- und Fußzeile
\pagestyle{fancy}
\fancyhf{}
\fancyhead[L]{Johann Pascher}
\fancyhead[R]{Korrigierte Analyse T0--\LCDM}
\fancyfoot[C]{\thepage}
\renewcommand{\headrulewidth}{0.4pt}
\renewcommand{\footrulewidth}{0.4pt}
% Hyperref-Einstellungen
\hypersetup{
	colorlinks=true,
	linkcolor=blue,
	filecolor=blue,
	citecolor=blue,
	urlcolor=blue,
	bookmarks=true,
	bookmarksopen=true,
	pdftitle={Kompensatorische und additive Effekte: Eine korrigierte Analyse der Messdifferenzen zwischen dem T0-Modell und dem \(\Lambda\)CDM-Standardmodell},
	pdfauthor={Johann Pascher},
}
\begin{document}
	
	\title{Kompensatorische und additive Effekte: Eine korrigierte Analyse der Messdifferenzen zwischen dem T0-Modell und dem \LCDM-Standardmodell}
	\author{Johann Pascher}
	\date{10. April 2025}
	\maketitle
	
	\begin{abstract}
		Dieses Dokument präsentiert eine korrigierte Analyse der Unterschiede in kosmologischen Messungen zwischen dem Standardmodell \cite{Planck2018} und dem alternativen T0-Modell. Der Fokus liegt auf der konsistenten Interpretation von Winkeldurchmesserabständen und Winkelgrößen, insbesondere in Bezug auf den kosmischen Mikrowellenhintergrund (CMB). Die Analyse zeigt, dass scheinbare Inkonsistenzen in den Berechnungen tatsächlich fundamentale Unterschiede in der theoretischen Interpretation kosmologischer Beobachtungen widerspiegeln. Bei der Neuinterpretation von Daten, die ursprünglich im Rahmen des Standardmodells \cite{Planck2018} erfasst wurden, müssen modellabhängige Annahmen berücksichtigt werden, um Fehlinterpretationen zu vermeiden. Die korrigierte Analyse bietet einen klareren Vergleich zwischen den Modellen und betont die Notwendigkeit modellunabhängiger Validierungstests.
	\end{abstract}
	
	\tableofcontents
	\newpage
	
	\section{Einführung}
	
	Das kosmologische Standardmodell \cite{Planck2018} und das alternative T0-Modell bieten grundlegend unterschiedliche Erklärungen für dieselben astronomischen Beobachtungen. Während das Standardmodell \cite{Planck2018} auf einem expandierenden Universum basiert, postuliert das T0-Modell ein statisches Universum mit absoluter Zeit und variabler Masse. Diese Arbeit untersucht, wie diese unterschiedlichen theoretischen Grundlagen kosmologische Messungen beeinflussen und wie diese Effekte sich entweder verstärken oder kompensieren.
	
	In einer früheren Analyse \cite{Pascher2025a} wurden quantitative Vergleiche zwischen den Modellen präsentiert, insbesondere in Bezug auf Abstandsmessungen, Rotverschiebungen und die Interpretation des kosmischen Mikrowellenhintergrunds (CMB). Bei genauerer Betrachtung wurden jedoch potenzielle Inkonsistenzen in den Berechnungen der Winkelgrößen bei hohen Rotverschiebungen festgestellt, die eine Korrektur und Klärung erfordern.
	
	Diese überarbeitete Analyse beleuchtet die Herausforderungen beim Vergleich alternativer kosmologischer Modelle und betont die Wichtigkeit, zwischen modellinternen Berechnungen und modellübergreifenden Vergleichen klar zu unterscheiden. Besondere Aufmerksamkeit wird der Interpretation des CMB gewidmet, wo die Unterschiede zwischen den Modellen am deutlichsten hervortreten.
	
	\section{Grundkonzepte der Modelle}
	
	\subsection{Das \LCDM-Standardmodell}
	
	Im Standardmodell \cite{Planck2018} wird die beobachtete Rotverschiebung durch die kosmische Expansion erklärt. Die Friedmann-Gleichungen beschreiben die zeitliche Entwicklung des Universums, und die Hubble-Konstante \( H_0 \) repräsentiert die aktuelle Expansionsrate. Die kosmische Rotverschiebung \( z \) ist mit dem Skalenfaktor \( a(t) \) verknüpft durch:
	
	\begin{equation}
		1 + z = \frac{a(t_0)}{a(t_{\text{emit}})}
	\end{equation}
	
	Für kleine Rotverschiebungen gilt die Näherung:
	
	\begin{equation}
		z \approx \frac{H_0 d}{c}
	\end{equation}
	
	\subsection{Das T0-Modell}
	
	Im T0-Modell \cite{Pascher2025e} ist die Zeit absolut, während die Masse gemäß der Relation variiert:
	\begin{equation}
		m = \frac{\hbar}{T(x) c^2}
	\end{equation}
	
	Hier bezeichnet \( T(x) \) das intrinsische Zeitfeld, das durch folgende Beziehung mit dem Higgs-Feld gekoppelt ist:
	\begin{equation}
		T(x) = \frac{\hbar}{y \langle \Phi \rangle c^2}
	\end{equation}
	\cite{Pascher2025b}. Die Rotverschiebung entsteht durch die räumliche Variation von \( T(x) \), was einen Energieverlust der Photonen verursacht:
	
	\begin{equation}
		1 + z = \frac{T(x)}{T_0}
	\end{equation}
	
	wobei \( T_0 \) der Wert am Ort des Beobachters ist. Diese Variation kann ausgedrückt werden als:
	\begin{equation}
		T(x) = T_0 e^{-\alpha d}
	\end{equation}
	mit \( \alpha = H_0 / c \), was zu folgendem führt:
	\begin{equation}
		1 + z = e^{\alpha d}
	\end{equation}
	
	wobei \( d \) der physikalische Abstand und \( H_0 \) die Hubble-Konstante ist, neu interpretiert als die Rate der räumlichen Änderung von \( T(x) \) anstelle eines Expansionsparameters.
	
	Die natürlichen Einheiten im T0-Modell setzen:
	\[
	\hbar = c = G = k_B = \alpha_{\text{EM}} = \alpha_{\text{W}} = \beta_{\text{T}} = 1
	\]
	was bedeutet, dass alle Konstanten vereinheitlicht sind \cite{Pascher2025c}. In SI-Einheiten wird jedoch ein Wert von:
	\[
	\beta_{\text{T}}^{\text{SI}} \approx 0.008
	\]
	abgeleitet \cite{Pascher2025d}.
	
	\section{Grafische Darstellung der korrigierten Ergebnisse}
	
	\begin{figure}[h]
		\centering
		\begin{tikzpicture}
			\begin{axis}[
				width=14cm, height=8cm,
				xlabel={Rotverschiebung $z$},
				ylabel={Winkeldurchmesserabstand $d_A$ [Mpc]},
				xmin=0, xmax=1200,
				ymin=0, ymax=30,
				grid=both,
				legend pos=north east,
				legend style={fill=white, fill opacity=0.7},
				xtick={0, 200, 400, 600, 800, 1000, 1200},
				extra y ticks={13.5, 28.9},
				extra y tick labels={13.5, 28.9},
				extra y tick style={grid=major, grid style={dashed, red}}
				]
				% Standardmodell-Wert bei z=1100
				\addplot[color=blue, mark=*, mark size=4pt] coordinates {(1100, 13.5)};
				\addlegendentry{Standardmodell \cite{Planck2018}}
				% T0-Modell mit H0=70 Wert bei z=1100
				\addplot[color=red, mark=square*, mark size=4pt] coordinates {(1100, 28.9)};
				\addlegendentry{T0 ($H_0=70$)}
				% T0-Modell mit H0=73 Wert bei z=1100
				\addplot[color=orange, mark=diamond*, mark size=4pt] coordinates {(1100, 27.7)};
				\addlegendentry{T0 ($H_0=73$)}
				% Beispielkurven für beide Modelle (vereinfacht)
				\addplot[color=blue, thick, dashed, domain=0:1200, samples=100] {13.5/1100*x/(1+0.0004*x)};
				\addplot[color=red, thick, dashed, domain=0:1200, samples=100] {28.9/1100*x/(1+0.0004*x)};
			\end{axis}
		\end{tikzpicture}
		\caption{Winkeldurchmesserabstand $d_A$ für die CMB-Strahlung ($z=1100$). Der dramatische Unterschied zwischen den Modellen ist hier offensichtlich: Das T0-Modell sagt einen Wert für $d_A$ (28,9 Mpc vs. 13,5 Mpc) voraus, der mehr als doppelt so groß ist. Bei gleicher physikalischer Skala würde dies zu einem entsprechend kleineren Winkel führen, was jedoch der ursprünglichen Interpretation widerspricht.}
		\label{fig:cmb_angular_distance_corrected}
	\end{figure}
	
	\begin{figure}[h]
		\centering
		\begin{tikzpicture}
			\begin{axis}[
				width=14cm, height=8cm,
				xlabel={Physikalische Skala [Mpc]},
				ylabel={Beobachteter Winkel [Grad]},
				xmin=0, xmax=3,
				ymin=0, ymax=6,
				grid=both,
				legend pos=north west,
				legend style={fill=white, fill opacity=0.7}
				]
				% Standardmodell
				\addplot[color=blue, thick, domain=0:3, samples=100] {x/13.5*180/pi};
				\addlegendentry{Standardmodell \cite{Planck2018} ($d_A = 13,5$ Mpc)}
				% T0-Modell
				\addplot[color=red, thick, domain=0:3, samples=100] {x/28.9*180/pi};
				\addlegendentry{T0 ($d_A = 28,9$ Mpc)}
				% Markierte Punkte für r_LCDM und r_T0
				\addplot[color=blue, mark=*, mark size=4pt] coordinates {(0.24, 1)};
				\addplot[color=red, mark=square*, mark size=4pt] coordinates {(2.93, 5.8)};
				% Beschriftungen
				\node[blue] at (0.3, 1.3) {$r_{\LCDM} \approx 0,24$ Mpc};
				\node[red] at (2.4, 5.3) {$r_{T0} \approx 2,93$ Mpc};
			\end{axis}
		\end{tikzpicture}
		\caption{Beziehung zwischen physikalischer Skala und beobachtetem Winkel im CMB für beide Modelle. Die Linien zeigen die theoretische Beziehung für jedes Modell, während die Punkte die impliziten physikalischen Skalen markieren, die aus den beobachteten Winkeln ($1^\circ$ für das Standardmodell \cite{Planck2018} und $5,8^\circ$ für T0) abgeleitet wurden. Die deutlich unterschiedlichen physikalischen Skalen verdeutlichen die fundamental unterschiedliche Interpretation der CMB-Anisotropien in den beiden Modellen.}
		\label{fig:angle_scale_relationship}
	\end{figure}
	
	\section{Fazit}
	
	Die korrigierte Analyse der Messdifferenzen zwischen dem T0-Modell und dem Standardmodell \cite{Planck2018} zeigt, dass das T0-Modell eine tragfähige Alternative mit überprüfbaren Vorhersagen bietet.
	
	\begin{thebibliography}{99}
		\bibitem{Pascher2025a} Pascher, J. (2025). \href{https://github.com/jpascher/T0-Time-Mass-Duality/tree/main/2/pdf/Deutsch/Analyse\%20der\%20Messdifferenzen\%20zwischen\%20dem\%20T0-Modell\%20und\%20dem\%20Standardmodell.pdf}{Kompensatorische und additive Effekte: Eine Analyse der Messdifferenzen zwischen dem T0-Modell und dem \(\Lambda\)CDM-Standardmodell}. 2. April 2025.
    ALT: 		\bibitem{Pascher2025b} Pascher, J. (2025). \href{https://github.com/jpascher/T0-Time-Mass-Duality/tree/main/2/pdf/Deutsch/Massenvariation\%20in\%20Galaxien.pdf}{Massenvariation in Galaxien: Eine Analyse im T0-Modell mit emergenter Gravitation}. 30. März 2025.
    NEU: 		\bibitem{Pascher2025b} Pascher, J. (2025). \href{https://github.com/jpascher/T0-Time-Mass-Duality/tree/main/2/pdf/Deutsch/Massenvariation\%20in\%20Galaxien.pdf}{Massenvariation in Galaxien: Eine Analyse im T0-Modell mit emergenter Gravitation}. 30. M84rz 2025.
		\bibitem{Pascher2025b} Pascher, J. (2025). \href{https://github.com/jpascher/T0-Time-Mass-Duality/tree/main/2/pdf/Deutsch/Massenvariation\%20in\%20Galaxien.pdf}{Massenvariation in Galaxien: Eine Analyse im T0-Modell mit emergenter Gravitation}. 30. M84rz 2025.
		\bibitem{Pascher2025c} Pascher, J. (2025). \href{https://github.com/jpascher/T0-Time-Mass-Duality/tree/main/2/pdf/Deutsch/Die\%20Konsistenz\%20von\%20alpha\%20=\%201\%20und\%20beta\%20=\%201.pdf}{Vereinheitlichtes Einheitensystem im T0-Modell: Die Konsistenz von \(\alpha = 1\) und \(\beta = 1\)}. 5. April 2025.
    ALT: 		\bibitem{Pascher2025d} Pascher, J. (2025). \href{https://github.com/jpascher/T0-Time-Mass-Duality/tree/main/2/pdf/Deutsch/Zeit-Masse-Dualit\%C3\%A4tstheorie\%20(T0-Modell)\%20Herleitung\%20der\%20Parameter\%20kappa,\%20alpha\%20und\%20beta.pdf}{Zeit-Masse-Dualitätstheorie (T0-Modell): Ableitung der Parameter \(\kappa\), \(\alpha\) und \(\beta\)}. 4. April 2025.
    NEU: 		\bibitem{Pascher2025d} Pascher, J. (2025). \href{https://github.com/jpascher/T0-Time-Mass-Duality/tree/main/2/pdf/Deutsch/Zeit-Masse-Dualit\%C3\%A4tstheorie\%20(T0-Modell)\%20Herleitung\%20der\%20Parameter\%20kappa,\%20alpha\%20und\%20beta.pdf}{Zeit-Masse-Dualit84tstheorie (T0-Modell): Ableitung der Parameter \(\kappa\), \(\alpha\) und \(\beta\)}. 4. April 2025.
		\bibitem{Pascher2025d} Pascher, J. (2025). \href{https://github.com/jpascher/T0-Time-Mass-Duality/tree/main/2/pdf/Deutsch/Zeit-Masse-Dualit\%C3\%A4tstheorie\%20(T0-Modell)\%20Herleitung\%20der\%20Parameter\%20kappa,\%20alpha\%20und\%20beta.pdf}{Zeit-Masse-Dualit84tstheorie (T0-Modell): Ableitung der Parameter \(\kappa\), \(\alpha\) und \(\beta\)}. 4. April 2025.
    ALT: 		\bibitem{Pascher2025e} Pascher, J. (2025). \href{https://github.com/jpascher/T0-Time-Mass-Duality/tree/main/2/pdf/Deutsch/Mathematische\%20Formulierungen\%20der\%20Zeit-Masse-Dualit\%C3\%A4tstheorie\%20mit\%20Lagrange.pdf}{Von Zeitdilatation zu Massenvariation: Mathematische Kernformulierungen der Zeit-Masse-Dualitätstheorie}. 29. März 2025.
    NEU: 		\bibitem{Pascher2025e} Pascher, J. (2025). \href{https://github.com/jpascher/T0-Time-Mass-Duality/tree/main/2/pdf/Deutsch/Mathematische\%20Formulierungen\%20der\%20Zeit-Masse-Dualit\%C3\%A4tstheorie\%20mit\%20Lagrange.pdf}{Von Zeitdilatation zu Massenvariation: Mathematische Kernformulierungen der Zeit-Masse-Dualit84tstheorie}. 29. M84rz 2025.
		\bibitem{Pascher2025e} Pascher, J. (2025). \href{https://github.com/jpascher/T0-Time-Mass-Duality/tree/main/2/pdf/Deutsch/Mathematische\%20Formulierungen\%20der\%20Zeit-Masse-Dualit\%C3\%A4tstheorie\%20mit\%20Lagrange.pdf}{Von Zeitdilatation zu Massenvariation: Mathematische Kernformulierungen der Zeit-Masse-Dualit84tstheorie}. 29. M84rz 2025.
		\bibitem{Planck2018} Planck Collaboration, Aghanim, N., et al. (2020). \href{https://doi.org/10.1051/0004-6361/201833910}{Planck 2018 Results. VI. Cosmological Parameters}. Astronomy \& Astrophysics, 641, A6.
	\end{thebibliography}
	
\end{document}
