\documentclass[a4paper,12pt]{article}
\usepackage[utf8]{inputenc}
\usepackage[T1]{fontenc}
\usepackage{lmodern}
\usepackage[ngerman]{babel}
\usepackage{amsmath, amssymb, amsthm, physics}
\usepackage{graphicx}
\usepackage{xcolor}
\usepackage{tikz}
\usepackage{setspace}
\usepackage{tcolorbox}
\usepackage{booktabs}

% Farbige Links im Inhaltsverzeichnis und im Dokument
\usepackage{hyperref}
\hypersetup{
	colorlinks=true,
	linkcolor=blue,
	filecolor=blue,
	citecolor=blue, 
	urlcolor=blue,
	bookmarks=true,
	bookmarksopen=true,
	pdftitle={Vereinheitlichung des T0-Modells: Grundlagen, Dunkle Energie und Galaxiendynamik},
	pdfauthor={Johann Pascher},
}

% cleveref muss nach hyperref geladen werden
\usepackage{cleveref}

% Theorem-Stile
\newtheorem{theorem}{Theorem}
\newtheorem{lemma}[theorem]{Lemma}
\newtheorem{proposition}[theorem]{Proposition}
\newtheorem{corollary}[theorem]{Korollar}

\theoremstyle{definition}
\newtheorem{definition}{Definition}

\theoremstyle{remark}
\newtheorem{remark}{Bemerkung}

\begin{document}
	
	\title{Vereinheitlichung des T0-Modells: \\Grundlagen, Dunkle Energie und Galaxiendynamik}
	\author{Johann Pascher}
	\date{27. März 2025}
	\maketitle
	
	\begin{abstract}
		Diese Arbeit präsentiert eine konsistente Vereinheitlichung des T0-Modells und seiner Anwendungen auf kosmologische und astrophysikalische Phänomene. Das T0-Modell basiert auf der Annahme einer absoluten Zeit und variabler Masse, im Gegensatz zur Relativitätstheorie mit relativer Zeit und konstanter Masse. Diese fundamentale Neuinterpretation führt zu alternativen Erklärungen für die kosmische Rotverschiebung (durch Energieverlust statt Expansion), die dunkle Energie (als Medium für Energieaustausch) und die Galaxiendynamik (durch Massenvariation statt dunkler Materie). Die vorliegende Arbeit stellt die mathematische Konsistenz zwischen diesen verschiedenen Anwendungen sicher und bietet ein umfassendes theoretisches Rahmenwerk, das experimentell testbare Vorhersagen macht.
	\end{abstract}
	
	\tableofcontents
	\newpage
	
	\section{Einführung in das T0-Modell: Fundamentale Konzepte}
	
	\subsection{Grundannahmen des T0-Modells}
	
	Das T0-Modell basiert auf folgenden zentralen Annahmen:
	
	\begin{tcolorbox}[colback=blue!5!white,colframe=blue!75!black,title=Grundannahmen des T0-Modells]
		\begin{align}
			&\text{1. Die Zeit $T_0$ ist absolut und universell konstant.} \\
			&\text{2. Die Masse variiert entsprechend $m = \gamma m_0$, wobei $\gamma = \frac{1}{\sqrt{1-v^2/c_0^2}}$.} \\
			&\text{3. Die Gesamtenergie wird durch $E = \frac{\hbar}{T_0}$ ausgedrückt.} \\
			&\text{4. Die Rotverschiebung entsteht durch Energieverlust: $E_2 = E_1(1+z)^{-1}$.}
		\end{align}
	\end{tcolorbox}
	
	Diese Grundannahmen führen zu einer komplementären Formulierung der Physik, die mathematisch äquivalente Vorhersagen liefert, aber konzeptionell von der Standardphysik abweicht.
	
	\subsection{Intrinsische Zeit und Zeit-Masse-Dualität}
	
	Ein wichtiges Konzept des erweiterten T0-Modells ist die intrinsische Zeit:
	
	\begin{itemize}
		\item Die intrinsische Zeit eines Teilchens ist definiert als $T = \frac{\hbar}{mc^2}$. Sie ist umgekehrt proportional zur Masse des Teilchens.
		\item Diese intrinsische Zeit führt zu einer Dualität in der Beschreibung physikalischer Phänomene:
		\begin{itemize}
			\item \textbf{Standardmodell}: Zeit ist relativ (Zeitdilatation), Masse ist konstant
			\item \textbf{Komplementäres T0-Modell}: Zeit ist absolut, Masse variiert
		\end{itemize}
	\end{itemize}
	
	Die Zeit-Masse-Dualität ermöglicht eine alternative Interpretation vieler Phänomene, die traditionell durch Zeitdilatation erklärt werden.
	
	\subsection{Vereinheitlichte Lagrange-Dichte}
	
	Die Lagrange-Dichte für das vereinheitlichte T0-Modell lautet:
	
	\begin{equation}
		\mathcal{L}_\text{gesamt} = \mathcal{L}_\text{Gravitation} + \mathcal{L}_\text{SM} + \mathcal{L}_\text{Higgs} + \mathcal{L}_\text{intrinsisch}
	\end{equation}
	
	wobei:
	\begin{itemize}
		\item $\mathcal{L}_\text{Gravitation}$ die Lagrange-Dichte der Gravitation beschreibt,
		\item $\mathcal{L}_\text{SM}$ die Lagrange-Dichte des Standardmodells (starke, elektromagnetische und schwache Kräfte) repräsentiert,
		\item $\mathcal{L}_\text{Higgs}$ die Lagrange-Dichte des Higgs-Feldes ist,
		\item $\mathcal{L}_\text{intrinsisch}$ die neue Lagrange-Dichte ist, die die intrinsische Zeit berücksichtigt.
	\end{itemize}
	
	Die Gravitation kann in zwei komplementären Formen ausgedrückt werden:
	
	\begin{equation}
		\mathcal{L}_\text{Gravitation} = -\frac{1}{16\pi G} \sqrt{-g} R
	\end{equation}
	
	im Standardmodell (mit Zeitdilatation), und:
	
	\begin{equation}
		\mathcal{L}_\text{Gravitation-T} = -\frac{1}{16\pi G_T} \sqrt{-g_T} R_T
	\end{equation}
	
	im komplementären Modell (mit absoluter Zeit und Massenvariation), wobei $G_T = G \cdot \frac{T_0}{T}$ eine modifizierte Newton-Konstante ist, die von der intrinsischen Zeit $T = \frac{\hbar}{mc^2}$ abhängt, und $T_0$ eine Referenzzeitskala (z.B. die Planck-Zeit) ist.
	
	Diese vereinheitlichte Lagrange-Dichte bildet die mathematische Grundlage für die Anwendung des T0-Modells auf kosmologische und astrophysikalische Phänomene.
	
	% Fehlende Befehlsdefinition hinzufügen
	\newcommand{\Tfield}{T(x)} % Intrinsische Zeit als Feld
	
	\subsection{Die Rolle der Gravitation im T0-Modell}
	
	Im vorliegenden Dokument erscheint die Gravitation noch als separater Term in der Lagrange-Dichte:
	
	\begin{equation}
		\mathcal{L}_\text{gesamt} = \mathcal{L}_\text{Gravitation} + \mathcal{L}_\text{SM} + \mathcal{L}_\text{Higgs} + \mathcal{L}_\text{intrinsisch}
	\end{equation}
	
	Diese Darstellung spiegelt jedoch nicht die aktuellste Entwicklung des T0-Modells wider. Wie im Referenzdokument „Wesentliche mathematische Formalismen der Zeit-Masse-Dualitätstheorie mit Lagrange-Dichten" ausgeführt, kann die Gravitation tatsächlich als emergenter Effekt aus der Dynamik des intrinsischen Zeitfelds verstanden werden, ohne dass ein separater Term notwendig ist:
	
	\begin{theorem}[Gravitationsemergenz]
		Im T0-Modell entstehen Gravitationseffekte aus den räumlichen und zeitlichen Gradienten des intrinsischen Zeitfelds $\Tfield$, was eine natürliche Verbindung zwischen Quantenphysik und Gravitationsphänomenen herstellt durch:
		\begin{equation}
			\nabla \Tfield = \nabla \left(\frac{\hbar}{mc^2}\right) = -\frac{\hbar}{m^2c^2}\nabla m \sim \nabla \Phi_g
		\end{equation}
		wobei $\Phi_g$ das Gravitationspotential ist.
	\end{theorem}
	
	Diese Emergenz der Gravitation ist einer der konzeptionell elegantesten Aspekte des T0-Modells. Sie bedeutet, dass die vollständige vereinheitlichte Lagrange-Dichte präziser wie folgt formuliert werden sollte:
	
	\begin{equation}
		\mathcal{L}_\text{gesamt} = \mathcal{L}_\text{SM} + \mathcal{L}_\text{Higgs} + \mathcal{L}_\text{intrinsisch}
	\end{equation}
	
	Bei dieser Formulierung wird die Gravitation nicht mehr als separate Kraft hinzugenommen, sondern ergibt sich natürlich aus der intrinsischen Zeitfelddynamik. Dies steht im Einklang mit der grundlegenden Idee des T0-Modells, physikalische Phänomene durch komplementäre Sichtweisen neu zu interpretieren und vereinfacht den theoretischen Rahmen erheblich.
	
	\section{Dunkle Energie im T0-Modell}
	
	\subsection{Neuinterpretation der dunklen Energie}
	
	Im T0-Modell wird die dunkle Energie fundamental anders interpretiert als im Standardmodell der Kosmologie ($\Lambda$CDM):
	
	\begin{itemize}
		\item \textbf{Standardmodell ($\Lambda$CDM)}: Dunkle Energie ist eine kosmologische Konstante mit negativem Druck, die die beschleunigte Expansion des Universums antreibt.
		\item \textbf{T0-Modell}: Dunkle Energie ist ein dynamisches Medium für Energieaustausch in einem statischen Universum.
	\end{itemize}
	
	Die dunkle Energie wird als Skalarfeld $\phi_{DE}$ modelliert, das mit Materie und Strahlung wechselwirkt. Die Energiedichte dieses Feldes weist eine räumliche Struktur auf:
	
	\begin{equation}
		\rho_{DE}(r) = \frac{\kappa}{r^2}
	\end{equation}
	
	wobei $\kappa$ eine Konstante ist und $r$ den radialen Abstand bezeichnet. Dieses $1/r^2$-Profil unterscheidet sich von der konstanten Energiedichte $\rho_\Lambda$ der kosmologischen Konstante im Standardmodell.
	
	\subsection{Feldtheoretische Beschreibung}
	
	Die vollständige Lagrange-Dichte für das dunkle Energiefeld lautet:
	
	\begin{equation}
		\mathcal{L}_{DE} = -\frac{1}{2}\partial_\mu \phi_{DE} \partial^\mu \phi_{DE} - V(\phi_{DE}) - \frac{\beta}{M_{Pl}} \phi_{DE} T^{\mu}_{\mu} - \frac{1}{2}\xi \phi_{DE}^2 R
	\end{equation}
	
	wobei:
	\begin{itemize}
		\item $\partial_\mu \phi_{DE} \partial^\mu \phi_{DE}$ der kinetische Term ist,
		\item $V(\phi_{DE})$ das Selbstwechselwirkungspotential des Feldes,
		\item $\frac{\beta}{M_{Pl}} \phi_{DE} T^{\mu}_{\mu}$ die Kopplung an Materie und Strahlung,
		\item $\frac{1}{2}\xi \phi_{DE}^2 R$ eine nicht-minimale Kopplung an die Raumzeitkrümmung $R$.
	\end{itemize}
	
	Die Feldgleichung für das dunkle Energiefeld lautet:
	
	\begin{equation}
		\Box\phi_{DE} - \frac{dV}{d\phi_{DE}} - \frac{\beta}{M_{Pl}}T^{\mu}_{\mu} - \xi \phi_{DE} R = 0
	\end{equation}
	
	Für ein masseloses Feld ($V(\phi_{DE}) = 0$) und vernachlässigbare Krümmung ($\xi R \approx 0$) vereinfacht sich die Feldgleichung zu:
	
	\begin{equation}
		\frac{1}{r^2}\frac{d}{dr}\left(r^2\frac{d\phi_{DE}}{dr}\right) = \frac{\beta}{M_{Pl}}T^{\mu}_{\mu}
	\end{equation}
	
	\subsection{Energieaustausch und Rotverschiebung}
	
	Ein zentraler Aspekt des T0-Modells ist die Interpretation der kosmischen Rotverschiebung als Folge eines Energieverlusts von Photonen an die dunkle Energie. Die Energieänderung eines Photons wird beschrieben durch:
	
	\begin{equation}
		\frac{dE_{\gamma}}{dx} = -\alpha E_{\gamma}
	\end{equation}
	
	mit der Lösung:
	
	\begin{equation}
		E_{\gamma}(x) = E_{\gamma,0} e^{-\alpha x}
	\end{equation}
	
	Die Rotverschiebung $z$ ist definiert als:
	
	\begin{equation}
		1 + z = \frac{E_0}{E} = \frac{\lambda_{\text{beob}}}{\lambda_{\text{emit}}} = e^{\alpha d}
	\end{equation}
	
	Um Konsistenz mit der beobachteten Hubble-Relation $z \approx H_0 d/c$ zu gewährleisten, muss gelten:
	
	\begin{equation}
		\alpha = \frac{H_0}{c} \approx 2{,}3 \times 10^{-28} \text{ m}^{-1}
	\end{equation}
	
	Diese extrem kleine Absorptionsrate erklärt, warum der Energieverlust von Photonen an die dunkle Energie in Laborexperimenten nicht messbar ist, aber über kosmologische Distanzen signifikant wird.
	
	\section{Galaxiendynamik im T0-Modell}
	
	\subsection{Flache Rotationskurven ohne dunkle Materie}
	
	Im T0-Modell werden die flachen Rotationskurven von Galaxien nicht durch dunkle Materie, sondern durch eine effektive Massenvariation erklärt, die durch Wechselwirkung mit der dunklen Energie entsteht. 
	
	Die effektive Masse eines Teilchens im T0-Modell kann als dynamische Größe betrachtet werden, die mit dem dunklen Energiefeld wechselwirkt:
	
	\begin{equation}
		m_{\text{eff}}(r) = m_0 \cdot f(\phi_{DE}(r))
	\end{equation}
	
	Diese Kopplung kann durch einen Yukawa-artigen Term in der Lagrange-Dichte modelliert werden:
	
	\begin{equation}
		\mathcal{L}_{\text{int}} = -g \phi_{DE} \bar{\psi}\psi
	\end{equation}
	
	In der Newtonschen Mechanik ist die Rotationsgeschwindigkeit $v(r)$ eines Objekts in einer kreisförmigen Umlaufbahn um eine Masse $M$ gegeben durch:
	
	\begin{equation}
		v^2(r) = \frac{GM(r)}{r}
	\end{equation}
	
	Im T0-Modell wird die Rotationsgeschwindigkeit durch die modifizierte Gleichung:
	
	\begin{equation}
		\frac{G \cdot m_{\text{eff}}(r) \cdot M(r)}{r^2} = \frac{v^2(r)}{r} \cdot m_{\text{eff}}(r)
	\end{equation}
	
	beschrieben, wobei $m_{\text{eff}}(r)$ die effektive Masse eines Testpartikels (z.B. eines Sterns) an der Position $r$ ist.
	
	\subsection{Effektive Gravitationskonstante}
	
	Ein alternativer Ansatz besteht darin, eine effektive Gravitationskonstante einzuführen, die vom dunklen Energiefeld abhängt:
	
	\begin{equation}
		G_{\text{eff}}(r) = G\left(1 + \beta\phi_{DE}(r)\right) = G\left(1 - \beta\frac{g\rho_0 r_0^2}{r}\right)
	\end{equation}
	
	Die Rotationsgeschwindigkeit wird dann:
	
	\begin{equation}
		v^2(r) = \frac{G_{\text{eff}}(r)M_{\text{baryon}}(r)}{r}
	\end{equation}
	
	Mit einem dunklen Energiefeld mit einer Dichte, die für große $r$ proportional zu $1/r^2$ ist:
	
	\begin{equation}
		\rho_{DE}(r) = \frac{\kappa}{r^2}
	\end{equation}
	
	wird die Rotationsgeschwindigkeit:
	
	\begin{equation}
		v^2(r) \approx \frac{GM_{\text{baryon}}}{r} + \frac{\kappa}{\rho_0}
	\end{equation}
	
	Für große $r$ dominiert der zweite Term, und wir erhalten:
	
	\begin{equation}
		v^2(r) \approx \frac{\kappa}{\rho_0} = \text{konstant}
	\end{equation}
	
	Dies entspricht genau dem beobachteten Verhalten flacher Rotationskurven.
	
	\subsection{Parameterwerte aus Beobachtungen}
	
	Für eine typische Spiralgalaxie wie die Milchstraße mit einer Rotationsgeschwindigkeit von etwa $v \approx 220$ km/s ergibt sich:
	
	\begin{equation}
		\kappa = v^2 \rho_0 \approx (220 \text{ km/s})^2 \cdot \rho_0 \approx 4{,}8 \times 10^{-7} \text{ GeV/cm} \cdot \text{s}^{-2}
	\end{equation}
	
	Die dimensionslose Kopplungskonstante beträgt etwa:
	
	\begin{equation}
		\hat{\beta} \approx 10^{-3}
	\end{equation}
	
	Diese Werte sind konsistent mit den aus kosmologischen Beobachtungen abgeleiteten Parametern des dunklen Energiefeldes.
	
	\section{Vereinheitlichte Mathematische Formulierung}
	
	\subsection{Gemeinsame Feldgleichungen}
	
	Die vollständige vereinheitlichte Theorie kann durch die folgende Wirkung beschrieben werden:
	
	\begin{equation}
		S_\text{vereinheitlicht} = \int \left( \mathcal{L}_\text{standard} + \mathcal{L}_\text{komplementär} + \mathcal{L}_\text{Kopplung} \right) d^4x
	\end{equation}
	
	wobei:
	\begin{align}
		\mathcal{L}_\text{standard} &= -\frac{1}{16\pi G} \sqrt{-g} R + \mathcal{L}_\text{SM} + (D_\mu \phi)^\dagger (D^\mu \phi) - V(\phi) \\
		\mathcal{L}_\text{komplementär} &= -\frac{1}{16\pi G_T} \sqrt{-g_T} R_T + \mathcal{L}_\text{SM-T} + (D_{T\mu} \phi_T)^\dagger (D_T^\mu \phi_T) - V_T(\phi_T) \\
		\mathcal{L}_\text{Kopplung} &= \int \mathcal{D}[\Psi] \, \Psi^* \left( i\hbar \frac{\partial}{\partial t} - i\hbar \frac{\partial}{\partial (t/T)} \right) \Psi
	\end{align}
	
	Diese vereinheitlichte Formulierung verbindet die Grundlagen des T0-Modells mit seinen Anwendungen auf dunkle Energie und Galaxiendynamik.
	
	Auf kosmologischen Skalen können die Friedmann-Gleichungen im T0-Modell neu interpretiert werden. Im Standardmodell beschreiben sie die Expansion des Universums:
	
	\begin{align}
		\left(\frac{\dot{a}}{a}\right)^2 &= \frac{8\pi G}{3}\rho \\
		\frac{\ddot{a}}{a} &= -\frac{4\pi G}{3}(\rho + 3p)
	\end{align}
	
	Im T0-Modell beschreiben sie stattdessen eine effektive Massenvariation in einem statischen Universum:
	
	\begin{align}
		\left(\frac{\dot{m}}{m}\right)^2 &= \frac{8\pi G}{3}\rho_{\text{eff}} \\
		\frac{\ddot{m}}{m} &= -\frac{4\pi G}{3}(\rho_{\text{eff}} + 3p_{\text{eff}})
	\end{align}
	
	\subsection{Konsistente Parametrisierung}
	
	Um die Konsistenz zwischen den verschiedenen Anwendungen des T0-Modells zu gewährleisten, lassen sich die verschiedenen Parameter wie folgt in Beziehung setzen:
	
	\begin{itemize}
		\item Der Absorptionskoeffizient $\alpha = \frac{H_{0}}{c} \approx 2{,}3 \times 10^{-28}$ m$^{-1}$ bestimmt die Rate des Energieverlusts von Photonen.
		\item Der Parameter $\kappa \approx 4{,}8 \times 10^{-7}$ GeV/cm$\cdot$s$^{-2}$ bestimmt die Stärke des dunklen Energiefeldes für die Galaxiendynamik.
		\item Die dimensionslose Kopplungskonstante $\beta \approx 10^{-3}$ charakterisiert die Wechselwirkung zwischen dem dunklen Energiefeld und der baryonischen Materie.
	\end{itemize}
	
	Diese Parameter sind über folgende Beziehung miteinander verbunden:
	
	\begin{equation}
		\kappa = \frac{\beta^2 H_0^2 M_{\text{Pl}}^2}{c^2 \rho_0}
	\end{equation}
	
	wobei $M_{\text{Pl}}$ die Planck-Masse und $\rho_0$ eine Referenzdichte ist.
	
	Für Photonen kann die intrinsische Zeit definiert werden als:
	
	\begin{equation}
		T = \frac{\hbar}{E_{\gamma}} e^{\alpha x}
	\end{equation}
	
	wobei $\alpha = \frac{H_0}{c} \approx 2{,}3 \times 10^{-28}$ m$^{-1}$ den Energieverlust über die Distanz $x$ berücksichtigt, in Übereinstimmung mit dem T0-Modell.
	
	\section{Experimentelle Tests des T0-Modells}
	
	\subsection{Gemeinsame Vorhersagen}
	
	Das T0-Modell führt zu mehreren experimentell überprüfbaren Vorhersagen, die es vom Standardmodell unterscheiden könnten:
	
	\begin{enumerate}
		\item \textbf{Massenabhängige Zeitentwicklung in Quantensystemen}, messbar als unterschiedliche Kohärenzzeiten.
		\item \textbf{Unterschiede in der Verschränkungsgeschwindigkeit} für Teilchen unterschiedlicher Masse.
		\item \textbf{Skalenabhängige Gravitationskonstante}, korreliert mit der intrinsischen Zeit.
		\item \textbf{Modifizierte Energie-Impuls-Beziehung} für sehr massive Teilchen.
		\item \textbf{Messbare Abweichungen in Hochpräzisionsexperimenten}, die typischerweise durch Zeitdilatation erklärt werden.
	\end{enumerate}
	
	Dies führt zu der Vorhersage, dass Bell-Tests mit Teilchen unterschiedlicher Massen oder Photonen unterschiedlicher Frequenzen messbare Verzögerungen in den Korrelationen zeigen könnten, proportional zum Massenverhältnis $\frac{m_1}{m_2}$ oder Energieverhältnis $\frac{E_1}{E_2}$.
	
	\subsection{Tests im kosmologischen Kontext}
	
	Spezifische Tests des T0-Modells im kosmologischen Kontext umfassen:
	
	\begin{enumerate}
		\item \textbf{Zeitliche Variation der Feinstrukturkonstante}:
		\begin{equation}
			\frac{d\alpha_{\text{fs}}}{dt} \approx \alpha_{\text{fs}} \cdot \alpha \cdot c \approx 10^{-18} \text{ Jahr}^{-1}
		\end{equation}
		
		\item \textbf{Umgebungsabhängigkeit der Rotverschiebung}:
		\begin{equation}
			\frac{z_{\text{cluster}}}{z_{\text{void}}} \approx 1 + \delta\frac{\rho_{\text{cluster}} - \rho_{\text{void}}}{\rho_0}
		\end{equation}
		
		\item \textbf{Anomale Lichtausbreitung in starken Gravitationsfeldern} mit einer effektiven Brechungszahl:
		\begin{equation}
			n_{\text{eff}}(r) = 1 + \epsilon \frac{\phi_{DE}(r)}{M_{\text{Pl}}}
		\end{equation}
		
		\item \textbf{Differenzielle Rotverschiebung}:
		\begin{equation}
			\frac{z(\lambda_1)}{z(\lambda_2)} \approx 1 + \eta\frac{\lambda_1 - \lambda_2}{\lambda_0}
		\end{equation}
	\end{enumerate}
	
	\subsection{Tests für die Galaxiendynamik}
	
	Spezifische Tests des T0-Modells im Bereich der Galaxiendynamik umfassen:
	
	\begin{enumerate}
		\item \textbf{Modifikation der Tully-Fisher-Beziehung}:
		\begin{equation}
			L \propto v_{\text{max}}^{4+\epsilon}
		\end{equation}
		wobei $\epsilon$ ein kleiner Korrekturterm ist, der von der Kopplungskonstante $\beta$ abhängt:
		\begin{equation}
			\epsilon \approx \frac{\beta^2 \rho_0 r_0^2}{m_0 G}
		\end{equation}
		
		\item \textbf{Massenabhängige Gravitationslinseneffekte}:
		\begin{equation}
			\alpha_{\text{lens}} \propto \int \nabla(\Phi_{\text{Newton}} + \beta\phi_{DE}) dz
		\end{equation}
		
		\item \textbf{Unterschiede zwischen gas- und sternreichen Galaxien}:
		\begin{equation}
			\frac{v^2_{\text{gas-reich}}(r)}{v^2_{\text{gas-arm}}(r)} = 1 + \delta(r)
		\end{equation}
		
		\item \textbf{Zwerggalaxien-Dynamik} mit systematisch niedrigerer Geschwindigkeitsdispersion:
		\begin{equation}
			\sigma_{v,T_0} \approx \sigma_{v,\Lambda CDM} \times \left(1 - \gamma \frac{M_{\text{gas}}}{M_{\text{star}}}\right)
		\end{equation}
	\end{enumerate}
	
	\section{Vergleich mit dem $\Lambda$CDM-Standardmodell}
	
	\begin{tcolorbox}[colback=yellow!5!white,colframe=yellow!75!black,title=Vergleich der Modelle]
		\begin{tabular}{p{0.45\textwidth}|p{0.45\textwidth}}
			\toprule
			\textbf{$\Lambda$CDM-Modell} & \textbf{T0-Modell} \\
			\midrule
			Dunkle Materie als separate Teilchenspezies & Keine separate dunkle Materie, sondern effektive Massenvariation \\
			\midrule
			NFW-Dichteprofil: $\rho_{\text{DM}}(r) = \frac{\rho_0}{\frac{r}{r_s}(1 + \frac{r}{r_s})^2}$ & Effektives Dichteprofil: $\rho_{\text{eff}}(r) \approx \rho_{\text{baryon}}(r) + \frac{\kappa}{r^2}$ \\
			\midrule
			Zeit ist relativ (Zeitdilatation), Ruhemasse konstant & Zeit ist absolut, Masse variiert mit der Energie \\
			\midrule
			Dunkle Energie als Antrieb der kosmischen Expansion & Dunkle Energie als Medium für Energieaustausch \\
			\midrule
			Rotverschiebung durch Expansion & Rotverschiebung durch Energieverlust \\
			\midrule
			Expandierendes Universum & Statisches Universum \\
			\bottomrule
		\end{tabular}
	\end{tcolorbox}
	
	Im $\Lambda$CDM-Modell führt die konstante Energiedichte der dunklen Energie zu einer beschleunigten Expansion, die immer schneller wird. Die Zukunft des Universums ist ein "Großer Riss" (Big Rip) oder ewige Expansion, je nach der genauen Zustandsgleichung der dunklen Energie.
	
	Im T0-Modell gibt es keine echte Expansion des Universums, sondern eine kontinuierliche Umwandlung von Materie- und Strahlungsenergie in dunkle Energie. Die Energiedichten entwickeln sich gemäß:
	
	\begin{align}
		\rho_{\text{matter}}(t) &= \rho_{\text{matter},0} e^{-\alpha c t} \\
		\rho_{\gamma}(t) &= \rho_{\gamma,0} e^{-\alpha c t} \\
		\rho_{\text{DE}}(t) &= \rho_{\text{DE},0} + (\rho_{\text{matter},0} + \rho_{\gamma,0})(1 - e^{-\alpha c t})
	\end{align}
	
	Langfristig nähert sich das Universum einem Zustand, in dem alle Energie in Form von dunkler Energie vorliegt - ein "thermischer Tod", aber ohne Expansion des Raumes.

	\section{Zusammenfassung}
	
	Das T0-Modell stellt einen umfassenden alternativen theoretischen Rahmen zur Standardphysik dar, der auf der Annahme einer absoluten Zeit und variabler Masse basiert. Diese Neuinterpretation führt zu:
	
	\begin{enumerate}
		\item Einer alternativen Erklärung für die kosmische Rotverschiebung durch Energieverlust statt Expansion
		\item Einer Neudeutung der dunklen Energie als dynamisches Medium für Energieaustausch
		\item Einer Erklärung für flache Rotationskurven in Galaxien ohne dunkle Materie
	\end{enumerate}
	
	Die mathematische Konsistenz zwischen den fundamentalen Prinzipien des T0-Modells, seiner Anwendung auf die dunkle Energie und der Erklärung der Galaxiendynamik ist durch eine vereinheitlichte Feldtheorie gewährleistet. Diese Theorie umfasst sowohl die standardmäßige als auch die komplementäre Formulierung der Physik und bietet spezifische experimentelle Vorhersagen, die zwischen dem T0-Modell und dem Standardmodell unterscheiden könnten.
	
	Das T0-Modell bietet eine konzeptionell elegante Alternative zum Standardmodell der Kosmologie, indem es fundamentale Annahmen über Zeit und Masse neu interpretiert. Die vorgeschlagenen Tests, insbesondere die Analyse von Galaxien mit unterschiedlichen Gas-zu-Stern-Verhältnissen und die detaillierte Messung von Gravitationslinsenprofilen, bieten vielversprechende Möglichkeiten, zwischen den Modellen zu unterscheiden.
	
	Unabhängig vom Ausgang dieser Tests trägt die mathematische Formulierung des T0-Modells zu einem tieferen Verständnis der fundamentalen Konzepte von Zeit, Masse und Energie in der modernen Physik bei und eröffnet neue Perspektiven für die Interpretation kosmischer Phänomene.
	
	\section{Literaturverzeichnis}
	
	\begin{thebibliography}{99}
		
		\bibitem{pascher1} Pascher, J. (2025). Complementary Extensions of Physics: Absolute Time and Intrinsic Time.
		
		\bibitem{pascher2} Pascher, J. (2025). A Model with Absolute Time and Variable Energy: A Comprehensive Investigation of the Foundations.
		
		\bibitem{pascher3} Pascher, J. (2025). Extensions of Quantum Mechanics through Intrinsic Time.
		
		\bibitem{pascher4} Pascher, J. (2025). Integration of Time-Mass Duality into Quantum Field Theory.
		
		\bibitem{pascher5} Pascher, J. (2025). Dynamic Mass of Photons and Their Implications for Nonlocality.
		
		\bibitem{pascher6} Pascher, J. (2025). Fundamental Constants and Their Derivation from Natural Units.
		
		\bibitem{pascher7} Pascher, J. (2025). Real Consequences of Reformulating Time and Mass in Physics: Beyond the Planck Scale.
		
		\bibitem{rotation} Rubin, V. C., Ford, W. K. (1970). Rotation of the Andromeda Nebula from a Spectroscopic Survey of Emission Regions. The Astrophysical Journal, 159, 379.
		
		\bibitem{nfw} Navarro, J. F., Frenk, C. S., White, S. D. M. (1996). The Structure of Cold Dark Matter Halos. The Astrophysical Journal, 462, 563.
		
		\bibitem{tully} Tully, R. B., Fisher, J. R. (1977). A new method of determining distances to galaxies. Astronomy and Astrophysics, 54, 661.
		
		\bibitem{bullet} Clowe, D., Bradač, M., Gonzalez, A. H., et al. (2006). A Direct Empirical Proof of the Existence of Dark Matter. The Astrophysical Journal, 648, L109.
		
		\bibitem{supernova} Perlmutter, S., et al. (1999). Measurements of $\Omega$ and $\Lambda$ from 42 High-Redshift Supernovae. The Astrophysical Journal, 517, 565.
		
		\bibitem{riess} Riess, A. G., et al. (1998). Observational Evidence from Supernovae for an Accelerating Universe and a Cosmological Constant. The Astronomical Journal, 116, 1009.
		
		\bibitem{planck} Planck Collaboration. (2020). Planck 2018 results. VI. Cosmological parameters. Astronomy \& Astrophysics, 641, A6.
		
		\bibitem{cmb} Bennett, C. L., et al. (2013). Nine-year Wilkinson Microwave Anisotropy Probe (WMAP) Observations: Final Maps and Results. The Astrophysical Journal Supplement Series, 208, 20.
		
		\bibitem{bao} Eisenstein, D. J., et al. (2005). Detection of the Baryon Acoustic Peak in the Large-Scale Correlation Function of SDSS Luminous Red Galaxies. The Astrophysical Journal, 633, 560.
		
		\bibitem{quintessence} Caldwell, R. R., Dave, R., Steinhardt, P. J. (1998). Cosmological Imprint of an Energy Component with General Equation of State. Physical Review Letters, 80, 1582.
		
		\bibitem{euclid} Laureijs, R., et al. (2011). Euclid Definition Study Report. ESA/SRE(2011)12.
		
		\bibitem{tired} Zwicky, F. (1929). On the Red Shift of Spectral Lines through Interstellar Space. Proceedings of the National Academy of Sciences, 15, 773.
		
		\bibitem{alfa} Webb, J. K., et al. (2011). Indications of a Spatial Variation of the Fine Structure Constant. Physical Review Letters, 107, 191101.
		
		\bibitem{vacuum} Weinberg, S. (1989). The Cosmological Constant Problem. Reviews of Modern Physics, 61, 1.
		
		\bibitem{scalar} Fujii, Y., Maeda, K. (2003). The Scalar-Tensor Theory of Gravitation. Cambridge University Press.
		
		\bibitem{lambda} Carroll, S. M. (2001). The Cosmological Constant. Living Reviews in Relativity, 4, 1.
	\end{thebibliography}
	
	\end{document}