\documentclass[12pt,a4paper]{article}
\usepackage[utf8]{inputenc}
\usepackage[T1]{fontenc}
\usepackage[ngerman]{babel}
\usepackage{lmodern}
\usepackage{csquotes}
\usepackage{amsmath}
\usepackage{amssymb}
\usepackage{physics}
\usepackage{geometry}
\usepackage{tocloft}
\usepackage{xcolor}
\usepackage{graphicx,tikz,pgfplots}
\pgfplotsset{compat=1.18}
\usepackage{booktabs}
\usepackage{siunitx}
\usepackage{amsthm}
\usepackage[colorlinks=true, linkcolor=blue, citecolor=blue, urlcolor=blue]{hyperref}
\usepackage{cleveref}
\usepackage{fancyhdr} % Für Kopf- und Fußzeilen

\geometry{a4paper, margin=2cm}

% Kopf- und Fußzeilen
\pagestyle{fancy}
\fancyhf{}
\fancyhead[L]{Johann Pascher}
\fancyhead[R]{Zeit-Masse-Dualität}
\fancyfoot[C]{\thepage}
\renewcommand{\headrulewidth}{0.4pt}
\renewcommand{\footrulewidth}{0.4pt}

% Inhaltsverzeichnis-Styling
\renewcommand{\cftsecfont}{\color{blue}}
\renewcommand{\cftsubsecfont}{\color{blue}}
\renewcommand{\cftsecpagefont}{\color{blue}}
\renewcommand{\cftsubsecpagefont}{\color{blue}}
\setlength{\cftsecindent}{1cm}
\setlength{\cftsubsecindent}{2cm}

% Benutzerdefinierte Befehle (konsistent mit anderen Dokumenten)
\newcommand{\Tfield}{T(x)}
\newcommand{\betaT}{\beta_{\text{T}}}
\newcommand{\alphaEM}{\alpha_{\text{EM}}}
\newcommand{\alphaW}{\alpha_{\text{W}}}
\newcommand{\Mpl}{M_{\text{Pl}}}
\newcommand{\Tzerot}{T_0(\Tfield)}
\newcommand{\Tzero}{T_0}
\newcommand{\vecx}{\vec{x}}
\newcommand{\gammaf}{\gamma_{\text{Lorentz}}}
\newcommand{\DhiggsT}{\Tfield (\partial_\mu + ig A_\mu) \Phi + \Phi \partial_\mu \Tfield}

\newtheorem{theorem}{Satz}[section]
\newtheorem{proposition}[theorem]{Proposition}

\title{Dynamische Masse von Photonen und ihre Auswirkungen auf Nichtlokalität im T0-Modell}
\author{Johann Pascher}
\date{25. März 2025}

\begin{document}
	
	\maketitle
	
	\begin{abstract}
		Diese Arbeit untersucht die Auswirkungen der Zuweisung einer dynamischen, frequenzabhängigen effektiven Masse zu Photonen im Rahmen des T0-Modells der Zeit-Masse-Dualität, das absolute Zeit und variable Masse postuliert. Durch die Annahme \(m_\gamma = \omega\) in natürlichen Einheiten wird eine energieabhängige intrinsische Zeit eingeführt, die Nichtlokalität und Kausalität beeinflusst. Die Theorie baut auf dem Rahmen des T0-Modells auf und wird durch experimentelle Vorhersagen gestützt, die mit seinen Prinzipien übereinstimmen.
	\end{abstract}
	
	\tableofcontents
	\newpage
	
	\section{Einführung}
	Diese Arbeit analysiert die Auswirkungen einer dynamischen, frequenzabhängigen effektiven Masse für Photonen im T0-Modell der Quantenmechanik, das absolute Zeit und variable Masse voraussetzt \cite{pascher_galaxies_2025}. Das Konzept erweitert den Rahmen der intrinsischen Zeit des Modells, um Nichtlokalität und Kausalität zu erforschen.
	
	\section{Natürliche Einheiten als Grundlage}
	\subsection{Definition der natürlichen Einheiten}
	\begin{theorem}[Natürliche Einheiten]
		Mit \(\hbar = c = G = 1\):
		\begin{align}
			[L] &= [E^{-1}] \\
			[T] &= [E^{-1}] \\
			[M] &= [E]
		\end{align}
	\end{theorem}
	
	\subsection{Bedeutung für die Masse-Energie-Äquivalenz}
	Im T0-Modell ist die Masse dynamisch (\(\Tfield = \frac{\hbar}{m c^2}\)). Für Photonen wird eine effektive Masse vorgeschlagen:
	\begin{equation}
		m_\gamma = \omega
	\end{equation}
	wobei \(\omega\) die Winkelkreisfrequenz ist, in Übereinstimmung mit \(E = \hbar \omega\) in natürlichen Einheiten (\(\hbar = 1\)).
	
	\section{Zeitmodelle in der Quantenmechanik}
	\subsection{Einschränkungen des Standardmodells}
	Die Standard-Schrödinger-Gleichung nimmt eine universelle Zeit an:
	\begin{equation}
		i\hbar\frac{\partial\psi}{\partial t} = H\psi
	\end{equation}
	
	\subsection{Das T0-Modell mit absoluter Zeit}
	Im T0-Modell ist die Energie mit einer konstanten intrinsischen Zeit \(T_0\) verknüpft:
	\begin{equation}
		E = \frac{\hbar}{T_0}
	\end{equation}
	Für massereiche Teilchen gilt \(\Tfield = \frac{\hbar}{m c^2}\).
	
	\subsection{Erweiterung für Photonen}
	Für Photonen wird dies zu einer energieabhängigen intrinsischen Zeit erweitert:
	\begin{equation}
		\Tfield = \frac{\hbar}{m_\gamma c^2} = \frac{1}{\omega}
	\end{equation}
	Dies bleibt konsistent mit \(m_\gamma = \omega\) (da \(\hbar = c = 1\)).
	
	\section{Vereinheitlichung im T0-Modell}
	Zur Vereinheitlichung von massereichen Teilchen und Photonen:
	\begin{equation}
		\Tfield = \frac{\hbar}{\max(m c^2, \omega)}
	\end{equation}
	Für massereiche Teilchen dominiert \(m c^2\), für Photonen \(\omega\).
	
	\section{Auswirkungen auf Nichtlokalität und Verschränkung}
	\subsection{Energieabhängige Korrelationen}
	Die energieabhängige \(\Tfield\) führt zu Zeitverzögerungen in verschränkten Systemen:
	\begin{itemize}
		\item Verzögerung: \(\left|\frac{1}{\omega_1} - \frac{1}{\omega_2}\right|\)
	\end{itemize}
	Dies deutet darauf hin, dass Nichtlokalität aus intrinsischen Zeitunterschieden emergiert, ähnlich wie der Energieverlustmechanismus der Rotverschiebung im T0-Modell \cite{pascher_messdifferenzen_2025}.
	
	\subsection{\(\betaT\) im T0-Modell}
	Im T0-Modell wird die wellenlängenabhängige Rotverschiebung durch den Parameter \(\betaT\) beschrieben, wobei \(\betaT^{\text{SI}} \approx 0.008\) in SI-Einheiten und \(\betaT^{\text{nat}} = 1\) in natürlichen Einheiten gilt \cite{pascher_params_2025}. Diese Werte sind äquivalent und spiegeln dieselbe physikalische Realität wider, wobei die Umrechnung durch die charakteristische Längenskala \(r_0\) erfolgt \cite{pascher_temp_2025}. Die Ableitung von \(\betaT\) ist im T0-Modell geklärt, und die Wahl zwischen \(\betaT^{\text{SI}}\) und \(\betaT^{\text{nat}}\) hängt allein vom Einheitensystem ab, ohne dass eine Unsicherheit über die theoretische Grundlage besteht.
	
	\begin{figure}[h]
		\centering
		\begin{tikzpicture}
			\begin{axis}[
				xlabel={Energie [eV]},
				ylabel={Zeit [eV\(^{-1}\)]},
				xlabel style={font=\large},
				ylabel style={font=\large},
				tick label style={font=\normalsize},
				xmin=0, xmax=10,
				ymin=0, ymax=10,
				legend pos=north east,
				legend style={font=\large},
				grid=both,
				minor tick num=1
				]
				\addplot[blue, ultra thick, domain=0.1:10, samples=100] {1/x};
				\legend{\(T = E^{-1}\)}
			\end{axis}
		\end{tikzpicture}
		\caption{Energieabhängige intrinsische Zeit für Photonen im T0-Modell.}
	\end{figure}
	
	\section{Experimentelle Überprüfung}
	\begin{itemize}
		\item Frequenzabhängige Bell-Tests zur Messung von Zeitverzögerungen in der Verschränkung.
		\item Spektroskopische Rotverschiebungsmessungen zur Validierung der wellenlängenabhängigen Rotverschiebung mit \(\betaT\).
	\end{itemize}
	
	\section{Physik jenseits der Lichtgeschwindigkeit}
	Eine hypothetische modifizierte Dispersionsrelation im T0-Modell:
	\begin{equation}
		E^2 = (m_\gamma c^2)^2 + (p c)^2 + \alpha_c p^4 c^2 / E_P^2
	\end{equation}
	wobei \(\alpha_c\) eine Kopplungskonstante und \(E_P\) die Planck-Energie ist, könnte das Verhalten hochenergetischer Photonen erklären und durch kosmische Strahlenmessungen überprüft werden.
	
	\section{Schlussfolgerung}
	Die dynamische effektive Masse von Photonen im T0-Modell bietet eine neuartige Sicht auf Nichtlokalität als emergentes Phänomen, das durch energieabhängige intrinsische Zeit angetrieben wird, und erweitert die Erklärungskraft des Modells.
	
	\begin{thebibliography}{99}
		\bibitem{pascher_galaxies_2025} Pascher, J. (2025). \href{https://github.com/jpascher/T0-Time-Mass-Duality/tree/main/2/pdf/Deutsch/Massenvariation in Galaxien.pdf}{Massenvariation in Galaxien: Eine Analyse im T0-Modell mit emergenter Gravitation}. 30. März 2025.
		\bibitem{pascher_messdifferenzen_2025} Pascher, J. (2025). \href{https://github.com/jpascher/T0-Time-Mass-Duality/tree/main/2/pdf/Deutsch/Analyse der Messdifferenzen zwischen dem T0-Modell und dem Standardmodell.pdf}{Kompensatorische und additive Effekte: Eine Analyse der Messdifferenzen zwischen dem T0-Modell und dem \(\Lambda\)CDM-Standardmodell}. 2. April 2025.
		\bibitem{pascher_params_2025} Pascher, J. (2025). \href{https://github.com/jpascher/T0-Time-Mass-Duality/tree/main/2/pdf/Deutsch/Zeit-Masse-Dualitätstheorie (T0-Modell) Herleitung der Parameter kappa, alpha und beta.pdf}{Zeit-Masse-Dualitätstheorie (T0-Modell): Ableitung der Parameter \(\kappa\), \(\alpha\) und \(\beta\)}. 4. April 2025.
		\bibitem{pascher_temp_2025} Pascher, J. (2025). \href{https://github.com/jpascher/T0-Time-Mass-Duality/tree/main/2/pdf/Deutsch/Anpassung von Temperatureinheiten in natürlichen Einheiten und CMB-Messungen.pdf}{Anpassung der Temperatureinheiten in natürlichen Einheiten und CMB-Messungen}. 2. April 2025.
		\bibitem{einstein} Einstein, A. (1905). \textit{Zur Elektrodynamik bewegter Körper}. \textit{Annalen der Physik}, 322(10), 891-921.
		\bibitem{planck} Planck, M. (1901). \textit{Über das Gesetz der Energieverteilung im Normalspektrum}. \textit{Annalen der Physik}, 309(3), 553-563.
		\bibitem{bell} Bell, J. S. (1964). \textit{Zum Einstein-Podolsky-Rosen-Paradoxon}. \textit{Physics}, 1(3), 195-200.
		\bibitem{feynman} Feynman, R. P. (1985). \textit{QED: Die seltsame Theorie des Lichts und der Materie}. Princeton University Press.
	\end{thebibliography}
	
\end{document}