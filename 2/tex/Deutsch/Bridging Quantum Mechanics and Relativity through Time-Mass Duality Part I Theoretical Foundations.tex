\documentclass[12pt,a4paper]{article}
\usepackage[utf8]{inputenc}
\usepackage[T1]{fontenc}
\usepackage[ngerman]{babel}
\usepackage{lmodern}
\usepackage{amsmath}
\usepackage{amssymb}
\usepackage{physics}
\usepackage{hyperref}
\usepackage{tcolorbox}
\usepackage{booktabs}
\usepackage{enumitem}
\usepackage[table,xcdraw]{xcolor}
\usepackage[left=2cm,right=2cm,top=2cm,bottom=2cm]{geometry}
\usepackage{pgfplots}
\pgfplotsset{compat=1.18}
\usepackage{graphicx}
\usepackage{float}
\usepackage{fancyhdr}
\usepackage{siunitx}

% Danksagungs-Umgebung
\newenvironment{acknowledgments}
{\section*{Danksagungen}}
{\vspace{1em}}

% Benutzerdefinierte Befehle
\newcommand{\Tfield}{T(x)}
\newcommand{\alphaEM}{\alpha_{\text{EM}}}
\newcommand{\alphaW}{\alpha_{\text{W}}}
\newcommand{\betaT}{\beta_{\text{T}}}
\newcommand{\Mpl}{M_{\text{Pl}}}
\newcommand{\Tzerot}{T_0(\Tfield)}
\newcommand{\Tzero}{T_0}
\newcommand{\vecx}{\vec{x}}
\newcommand{\vr}{\vec{r}}
\newcommand{\gammaf}{\gamma_{\text{Lorentz}}}
\newcommand{\DhiggsT}{\Tfield (\partial_\mu + ig A_\mu) \Phi + \Phi \partial_\mu \Tfield}
\newcommand{\LCDM}{\Lambda\text{CDM}}
\newcommand{\DTmu}{D_{T,\mu}}
\newcommand{\calL}{\mathcal{L}}
\newcommand{\deq}{\displaystyle}
\newcommand{\e}{\mathrm{e}}

% Kopf- und Fußzeilen-Konfiguration
\pagestyle{fancy}
\fancyhf{}
\fancyhead[L]{Johann Pascher}
\fancyhead[R]{Zeit-Masse-Dualität: Teil I}
\fancyfoot[C]{\thepage}
\renewcommand{\headrulewidth}{0.4pt}
\renewcommand{\footrulewidth}{0.4pt}

\hypersetup{
	colorlinks=true,
	linkcolor=blue,
	citecolor=blue,
	urlcolor=blue,
	pdftitle={Überbrückung von Quantenmechanik und Relativitätstheorie durch Zeit-Masse-Dualität: Teil I},
	pdfauthor={Johann Pascher},
	pdfsubject={Theoretische Physik},
	pdfkeywords={T0-Modell, natürliche Einheiten, Zeit-Masse-Dualität}
}

\title{Überbrückung von Quantenmechanik und Relativitätstheorie durch Zeit-Masse-Dualität: \\ Ein einheitlicher Rahmen mit natürlichen Einheiten \(\alpha = \beta = 1\) \\ Teil I: Theoretische Grundlagen}
\author{Johann Pascher\\
	Abteilung für Kommunikationstechnik\\
	Höhere Technische Bundeslehranstalt (HTL), Leonding, Österreich\\
	\texttt{johann.pascher@gmail.com}}
\date{7. April 2025}

\begin{document}
	
	\maketitle
	
	\begin{abstract}
		Dieses Papier stellt das T0-Modell der Zeit-Masse-Dualität vor, einen neuartigen theoretischen Rahmen, der Quantenmechanik (QM) und Relativitätstheorie (RT) vereint, indem ihre grundlegenden Konzepte durch absolute Zeit und variable Masse neu definiert werden. Wir etablieren ein einheitliches System natürlicher Einheiten, in dem \(\hbar = c = G = k_B = \alphaEM = \alphaW = \betaT = 1\) gilt, wodurch empirisch bestimmte Konstanten eliminiert werden, während eine bemerkenswerte Übereinstimmung mit experimentellen Messungen erzielt wird, mit Abweichungen unter \(10^{-6}\). Das intrinsische Zeitfeld \(\Tfield = \frac{\hbar}{\max(mc^2, \omega)}\) dient als Eckpfeiler, erweitert die QM mit einer massenabhängigen Schrödinger-Gleichung und interpretiert die Gravitationseffekte der RT als emergent aus der Felddynamik. Teil I konzentriert sich auf diese theoretischen Grundlagen – Vereinheitlichung der Konstanten, Definition von \(\Tfield\), feldtheoretische Formulierung und emergente Gravitation – und verbindet mikro- und makroskopische Physik. Teil II wird die kosmologischen Implikationen und experimentelle Validierung untersuchen, aufbauend auf diesem Fundament.
	\end{abstract}
	
	\section{Einleitung}
	\label{sec:introduction}
	
	Die Vereinheitlichung von Quantenmechanik (QM) und Relativitätstheorie (RT) ist seit über einem Jahrhundert eine zentrale Herausforderung der theoretischen Physik, getrieben durch ihre grundlegend unterschiedlichen Behandlungen von Zeit, Raum und Masse. Die QM, verwurzelt in Schrödingers Wellenmechanik, behandelt Zeit als einen einheitlichen Parameter ohne Operatorstatus (\(i\hbar \frac{\partial}{\partial t}\Psi = \hat{H}\Psi\)) \cite{schrodinger1926} und beschreibt hervorragend mikroskopische Phänomene wie Teilchenverhalten und Verschränkung. Im Gegensatz dazu definiert die RT, die Einsteins spezielle und allgemeine Theorien umfasst, Zeit als relative Dimension (\(t' = \gammaf t\)), verwoben mit dem Raum, mit Masse als Konstante, und regelt makroskopische Phänomene wie Gravitation und Raumzeitkrümmung \cite{einstein1905,einstein1915}. Diese Unterschiede haben eine kohärente Theorie erschwert und Probleme wie Quantengravitation, Erklärungen der Nichtlokalität \cite{bell1964} und kosmologische Modelle wie \(\Lambda\)CDM \cite{Planck2020} kompliziert.
	
	Das T0-Modell der Zeit-Masse-Dualität bietet ein neues Paradigma zur Versöhnung dieser Rahmenwerke, indem es ihre traditionellen Annahmen umkehrt: Zeit ist absolut, und Masse variiert, vermittelt durch ein intrinsisches Zeitfeld \(\Tfield\). Dieser Ansatz basiert auf einem einheitlichen System natürlicher Einheiten, in dem alle fundamentalen Konstanten (\(\hbar = c = G = k_B = \alphaEM = \alphaW = \betaT = 1\)) auf Eins gesetzt werden, nicht als empirische Anpassungen, sondern als theoretische Notwendigkeit, wodurch alle physikalischen Größen auf Energie reduziert werden. Bemerkenswerterweise stimmt dieses System mit gemessenen Werten überein (z. B. \(c \approx 3 \times 10^8 \, \text{m/s}\), \(\alphaEM \approx 1/137.036\)) mit Abweichungen unter \(10^{-6}\), validiert über Skalen von quantenmechanischen bis kosmologischen Phänomenen (siehe Teil II, Abschnitt 4 ''Quantitative Vorhersagen'' \href{https://github.com/jpascher/T0-Time-Mass-Duality/tree/main/2/pdf/Deutsch/Bridging Quantum Mechanics and Relativity through Time-Mass Duality Part II Theoretical Foundations.pdf}{[Teil II]}).
	
	Durch die Erweiterung der QM mit einer massenabhängigen Zeitentwicklung (Abschnitt 4.2 ''Erweiterung der Quantenmechanik'') und die Neuinterpretation der Gravitationseffekte der RT als emergent aus \(\Tfield\)-Gradienten (Abschnitt 5.1 ''Ableitung aus \(\Tfield\)'') verbindet T0 mikro- und makroskopische Physik ohne zusätzliche Dimensionen oder quantisierte Raumzeit, wie in der Stringtheorie oder Schleifenquantengravitation \cite{Greene2020,tHooft1993}. Teil I legt diese theoretischen Grundlagen fest, während Teil II deren kosmologische und experimentelle Implikationen untersucht.
	
	Dieses Papier ist wie folgt strukturiert:
	- Abschnitt 2: Vereinheitlichung der Konstanten mit natürlichen Einheiten.
	- Abschnitt 3: Definition und Eigenschaften von \(\Tfield\).
	- Abschnitt 4: Feldtheoretische Formulierung zur Erweiterung von QM und RT.
	- Abschnitt 5: Emergente Gravitation als Neuinterpretation der RT.
	- Abschnitt 6: Diskussion von Implikationen und Herausforderungen.
	- Abschnitt 7: Schlussfolgerung und Ausblick.
	
	\section{Vereinheitlichung der Konstanten mit natürlichen Einheiten}
	\label{sec:unified_units}
	
	\subsection{Motivation für natürliche Einheiten}
	\label{subsec:motivation_units}
	
	Physikalische Konstanten wie die Lichtgeschwindigkeit \(c\), die reduzierte Planck-Konstante \(\hbar\), die Gravitationskonstante \(G\) und die Feinstrukturkonstante \(\alphaEM\) werden traditionell als empirisch bestimmt angesehen und spiegeln die Skalen der Natur in menschlich definierten Einheiten wie Metern und Sekunden wider. In konventionellen Systemen natürlicher Einheiten (z. B. \(\hbar = c = 1\)) werden diese Konstanten auf Eins gesetzt, um mathematische Formulierungen zu vereinfachen und intrinsische physikalische Beziehungen offenzulegen \cite{Planck1899,Duff2002}. Zum Beispiel vereinheitlicht \(c = 1\) Raum- und Zeitdimensionen (\([L] = [T]\)), während \(\hbar = 1\) Energie und inverse Zeit gleichsetzt (\([E] = [T]^{-1}\)), was Gleichungen in QM und RT vereinfacht.
	
	Das T0-Modell geht bei dieser Vereinheitlichung einen Schritt weiter, indem es postuliert, dass alle fundamentalen Konstanten – nicht nur dimensionale wie \(\hbar\) und \(c\), sondern auch dimensionslose Kopplungen wie \(\alphaEM\) und \(\betaT\) – auf 1 vereinheitlicht werden sollten, nicht aus Bequemlichkeit, sondern als Reflexion einer tieferen, intrinsischen Einheit in der Natur. Dieser Ansatz ist motiviert durch die Beobachtung, dass traditionelle SI-Einheiten künstliche Komplexität einführen. Beispielsweise definieren die elektromagnetischen Konstanten \(\mu_0\) (Permeabilität) und \(\varepsilon_0\) (Permittivität) die Lichtgeschwindigkeit als \(c = \frac{1}{\sqrt{\mu_0\varepsilon_0}}\), doch ihre spezifischen Werte (\(\mu_0 = 4\pi \times 10^{-7} \, \text{H/m}\), \(\varepsilon_0 = 8.854 \times 10^{-12} \, \text{F/m}\)) sind empirisch festgelegt statt theoretisch abgeleitet. Das T0-Modell behauptet, dass \(c = 1\) als fundamentale Eigenschaft gesetzt wird, wodurch solche Willkürlichkeit eliminiert wird und elektromagnetische Eigenschaften inhärent an Zeit- und Energieskalen gebunden sind, eine Verbindung, die später durch das intrinsische Zeitfeld \(\Tfield\) formalisiert wird (Abschnitt 3.1 ''Definition und physikalische Grundlage'').
	
	Diese Vereinheitlichung ist nicht nur eine mathematische Vereinfachung, sondern eine philosophische Haltung: Physikalische Konstanten sind keine unabhängigen Parameter, die experimentelle Anpassung erfordern, sondern Manifestationen eines einzigen zugrunde liegenden Prinzips – Energie als universelles Maß. Durch die Beseitigung empirischer Abhängigkeiten zielt das T0-Modell darauf ab, einen selbstkonsistenten Rahmen zu konstruieren, der natürlich mit beobachteten Phänomenen übereinstimmt, wie durch seine Vorhersagekraft validiert (siehe Teil II, Abschnitt 4 ''Quantitative Vorhersagen'' \href{https://github.com/jpascher/T0-Time-Mass-Duality/tree/main/2/pdf/Deutsch/Bridging Quantum Mechanics and Relativity through Time-Mass Duality Part II Theoretical Foundations.pdf}{[Teil II]}).
	
	\subsection{Definition des einheitlichen Systems natürlicher Einheiten}
	\label{subsec:unified_system}
	
	Das T0-Modell übernimmt ein einheitliches System natürlicher Einheiten, definiert durch:
	\begin{align}
		\hbar &= c = G = k_B = \alphaEM = \alphaW = \betaT = 1,
		\label{eq:unit_system}
	\end{align}
	wobei jede Konstante aufgrund theoretischer Notwendigkeit auf Eins gesetzt wird, nicht als empirische Anpassung. Diese Konstanten repräsentieren:
	- \(\hbar = 1\): Quantenwirkungsskala, traditionell \(1.055 \times 10^{-34} \, \text{Js}\) in SI-Einheiten, die die Skala quantenmechanischer Phänomene bestimmt.
	- \(c = 1\): Raumzeit-Vereinheitlichung, traditionell \(3 \times 10^8 \, \text{m/s}\), die räumliche und zeitliche Dimensionen verbindet.
	- \(G = 1\): Gravitationskopplungsstärke, traditionell \(6.674 \times 10^{-11} \, \text{m}^3\text{kg}^{-1}\text{s}^{-2}\), die makroskopische Wechselwirkungen definiert.
	- \(k_B = 1\): Boltzmann-Konstante, traditionell \(1.381 \times 10^{-23} \, \text{J/K}\), die thermische Energie mit Temperatur verknüpft.
	- \(\alphaEM = \frac{e^2}{4\pi\varepsilon_0\hbar c} = 1\): Feinstrukturkonstante, traditionell \(\approx 1/137.036\), die elektromagnetische Wechselwirkungen vereinheitlicht und die Ladung dimensionslos macht (\(e = \sqrt{4\pi\varepsilon_0}\)).
	- \(\alphaW = 1\): Wien’sche Verschiebungskonstante, traditionell \(\approx 2.821439\), die die Frequenz thermischer Strahlung mit der Temperatur ausrichtet (\(\nu_{\text{max}} = \frac{k_B T}{h}\)).
	- \(\betaT = 1\): T0-Kopplungsparameter, traditionell \(\approx 0.008\) in SI-Einheiten, der die Wechselwirkungsstärke von \(\Tfield\) mit Materie und Feldern normalisiert.
	
	Im Gegensatz zu konventionellen Systemen natürlicher Einheiten (z. B. Planck-Einheiten), bei denen Konstanten wie \(\hbar, c, G\) auf 1 gesetzt werden aus Messkomfort und andere (z. B. \(\alphaEM\)) variabel bleiben, vereinheitlicht das T0-Modell alle Konstanten – einschließlich dimensionsloser – auf theoretischer Basis. Dieses System passt sich nicht an experimentelle Daten an, sondern sagt sie voraus und erzielt bemerkenswerte Übereinstimmung mit gemessenen Werten (z. B. \(c = 3 \times 10^8 \, \text{m/s}\) wird in natürlichen Einheiten zu 1 mit \(< 10^{-6}\) Abweichung bei Rückumrechnung) \cite{pascher_alphabeta_2025}.
	
	\subsubsection{Dimensionszuweisungen}
	In diesem System werden alle physikalischen Größen in Bezug auf Energie (\([E]\)) ausgedrückt, wodurch unabhängige Dimensionen für Länge, Zeit und Masse entfallen:
	\begin{table}[ht]
		\centering
		\caption{Dimensionszuweisungen im T0-einheitlichen System natürlicher Einheiten.}
		\label{tab:dimensions}
		\scalebox{0.8}{
			\begin{tabular}{ll}
				\hline
				\textbf{Physikalische Größe} & \textbf{Dimension in T0-Einheiten} \\
				\hline
				Länge & \([E^{-1}]\) \\
				Zeit & \([E^{-1}]\) \\
				Masse & \([E]\) \\
				Energie & \([E]\) \\
				Temperatur & \([E]\) \\
				Elektrische Ladung & \([1]\) (dimensionslos) \\
				Intrinsische Zeit (\(\Tfield\)) & \([E^{-1}]\) \\
				\hline
			\end{tabular}
		}
	\end{table}
	
	Zum Beispiel teilen Länge und Zeit die Dimension \([E^{-1}]\), da \(c = 1\) impliziert \([L] = [T]\) und \(\hbar = 1\) Zeit mit inverser Energie verknüpft (\([T] = [E^{-1}]\)). Masse und Energie sind äquivalent (\([M] = [E]\)) aufgrund von \(c = 1\), und Temperatur stimmt mit Energie über \(k_B = 1\) überein. Ladung wird dimensionslos mit \(\alphaEM = 1\), was elektromagnetische Wechselwirkungen vereinfacht.
	
	\subsubsection{Rolle der elektromagnetischen Konstanten}
	Die Lichtgeschwindigkeit in SI-Einheiten ist definiert als \(c = \frac{1}{\sqrt{\mu_0\varepsilon_0}}\), wobei \(\mu_0 = 4\pi \times 10^{-7} \, \text{H/m}\) und \(\varepsilon_0 = 8.854 \times 10^{-12} \, \text{F/m}\) empirisch bestimmte Konstanten sind, die \(c \approx 3 \times 10^8 \, \text{m/s}\) ergeben. Im T0-System impliziert das theoretische Setzen von \(c = 1\) \(\mu_0\varepsilon_0 = 1\), wodurch diese als unabhängige Parameter entfallen. Ähnlich wird die Feinstrukturkonstante \(\alphaEM = \frac{e^2}{4\pi\varepsilon_0\hbar c}\) zu 1, wodurch die Rolle von \(\varepsilon_0\) angepasst wird und die Ladung \(e\) eine abgeleitete Größe wird (\(e = \sqrt{4\pi\varepsilon_0}\)). Die Planck-Konstante verbindet sich mit diesem Rahmen über:
	\begin{equation}
		h = 2\pi\hbar = \frac{1}{\sqrt{\mu_0\varepsilon_0}} \cdot \text{(Skalierungsfaktor)},
		\label{eq:planck_em}
	\end{equation}
	was darauf hindeutet, dass Zeitskalen (\(T = \frac{h}{E}\)) inhärent mit elektromagnetischen Eigenschaften verknüpft sind, ein Vorläufer der Definition von \(\Tfield\) (Abschnitt 3.1 ''Definition und physikalische Grundlage''). Diese Vereinheitlichung reduziert die Komplexität elektromagnetischer Wechselwirkungen auf energiebasierte Begriffe, im Einklang mit dem Kernprinzip des T0-Modells.
	
	\subsubsection{Längenskalen und entsprechende Konstanten}
	\label{subsec:length_scales}
	
	Das einheitliche System des T0-Modells definiert Längenskalen in Bezug auf Energie und verbindet sie mit fundamentalen Konstanten und ihren Verhältnissen. Tabelle \ref{tab:length_scales} fasst die wichtigsten Längenskalen, ihre Ausdrücke in SI- und natürlichen Einheiten sowie die repräsentierten Konstanten zusammen und bietet eine Brücke zwischen theoretischen Konstrukten und beobachtbaren Phänomenen:
	
	\begin{table}[ht]
		\centering
		\caption{Längenskalen im T0-Modell und ihre entsprechenden Konstanten.}
		\label{tab:length_scales}
		\scalebox{0.8}{
			\begin{tabular}{lccc}
				\hline
				\textbf{Längenskala} & \textbf{SI-Ausdruck} & \textbf{T0 natürliche Einheiten} & \textbf{Repräsentierte Konstanten} \\
				\hline
				Planck-Länge (\(l_P\)) & \(\sqrt{\frac{\hbar G}{c^3}}\) & 1 & \(\hbar, G, c\) \\
				Compton-Wellenlänge (\(\lambda_C\)) & \(\frac{\hbar}{m c}\) & \(\frac{1}{m}\) & \(\hbar, c, m\) \\
				T0-Charakteristische Länge (\(r_0\)) & \(\xi l_P\) & \(1.33 \times 10^{-4}\) & \(\hbar, G, c, \lambda_h, v, m_h\) \\
				Kosmologische Korrelationslänge (\(L_T\)) & \(\frac{L_T}{l_P} \cdot l_P\) & \(3.9 \times 10^{62}\) & \(\hbar, G, c, \betaT\) \\
				\hline
			\end{tabular}
		}
	\end{table}
	
	- **Planck-Länge (\(l_P\)):** Definiert als \(\sqrt{\frac{\hbar G}{c^3}} \approx 1.616 \times 10^{-35} \, \text{m}\) in SI-Einheiten, wird sie zur fundamentalen Längeneinheit (\(l_P = 1\)) in T0-natürlichen Einheiten und repräsentiert die Skala, bei der \(\hbar, G,\) und \(c\) konvergieren.
	- **Compton-Wellenlänge (\(\lambda_C\)):** Gegeben durch \(\frac{\hbar}{m c}\), skaliert sie umgekehrt mit der Masse (\(\lambda_C = \frac{1}{m}\)) in natürlichen Einheiten, gebunden an \(\hbar\) und \(c\), und spiegelt die quantenmechanische Skala der Wellennatur eines Teilchens wider.
	- **T0-Charakteristische Länge (\(r_0\)):** Abgeleitet als \(\xi l_P\), wobei \(\xi = \frac{\lambda_h^2 v^2}{16\pi^3 m_h^2} \approx 1.33 \times 10^{-4}\), verbindet sie Higgs-Parameter (\(\lambda_h\): Selbstkopplung, \(v\): Vakuum-Erwartungswert, \(m_h\): Higgs-Masse) mit der Planck-Skala und repräsentiert den mikroskopischen Anker des T0-Modells.
	- **Kosmologische Korrelationslänge (\(L_T\)):** Definiert über das Verhältnis \(L_T/l_P \approx 3.9 \times 10^{62}\), entsteht sie aus \(\Tfield\)-Dynamiken und \(\betaT\) und repräsentiert die makroskopische Skala kosmischer Strukturen (siehe Teil II, Abschnitt 2 ''Statisches Universumsmodell'' \href{https://github.com/jpascher/T0-Time-Mass-Duality/tree/main/2/pdf/Deutsch/Bridging Quantum Mechanics and Relativity through Time-Mass Duality Part II Theoretical Foundations.pdf}{[Teil II]}).
	
	Diese Längenskalen zeigen, wie das T0-Modell mikro- und makroskopische Physik durch energiebasierte Einheiten und die Konstanten \(\hbar, c, G\), erweitert durch Higgs- und T0-spezifische Parameter, integriert. Die Verhältnisse (z. B. \(\xi, L_T/l_P\)) sind theoretisch abgeleitet, nicht empirisch angepasst, und ihre Übereinstimmung mit Beobachtungen (z. B. \(l_P\) als Quantengravitationsskala, \(L_T\) als kosmische Skala) validiert das einheitliche System \cite{pascher_alphabeta_2025}.
	
	\subsection{Hierarchie der Einheiten und abgeleitete Konstanten}
	\label{subsec:hierarchy}
	
	Das einheitliche System etabliert eine Hierarchie von Skalen:
	- **Basiseinheiten:** \(\hbar = c = G = k_B = 1\) definieren Energie als primäre Dimension und legen das Fundament für alle physikalischen Größen.
	- **Kopplungskonstanten:** \(\alphaEM = \alphaW = \betaT = 1\) vereinheitlichen die Wechselwirkungsstärken über elektromagnetische, thermische und T0-spezifische Domänen hinweg und eliminieren freie Parameter.
	- **Abgeleitete Skalen:** Schlüsselverhältnisse ergeben sich aus dieser Einheit, wie in Tabelle \ref{tab:derived_constants} gezeigt:
	\begin{table}[ht]
		\centering
		\caption{Abgeleitete Konstanten im T0-Modell, die Skalenhierarchien repräsentieren.}
		\label{tab:derived_constants}
		\scalebox{0.8}{
			\begin{tabular}{llr}
				\hline
				\textbf{Abgeleitete Konstante} & \textbf{Wert} & \textbf{Physikalische Bedeutung} \\
				\hline
				\(\xi = r_0/l_P\) & \(1.33 \times 10^{-4}\) & Verhältnis T0-Länge zu Planck-Länge \\
				\(L_T/l_P\) & \(3.9 \times 10^{62}\) & Kosmologische Korrelationslänge \\
				\(r_0/L_T\) & \(3.41 \times 10^{-67}\) & Verhältnis Mikro- zu Makroskala \\
				\hline
			\end{tabular}
		}
	\end{table}
	
	Der Parameter \(\xi = \frac{\lambda_h^2 v^2}{16\pi^3 m_h^2}\) verbindet den Higgs-Sektor (\(\lambda_h \approx 0.13\), \(v \approx 246 \, \text{GeV}\), \(m_h \approx 125 \, \text{GeV}\)) mit der Planck-Skala, während \(L_T\) \(\Tfield\)-Dynamiken mit kosmischen Skalen verknüpft (Teil II, Abschnitt 2 ''Statisches Universumsmodell'' \href{https://github.com/jpascher/T0-Time-Mass-Duality/tree/main/2/pdf/Deutsch/Bridging Quantum Mechanics and Relativity through Time-Mass Duality Part II Theoretical Foundations.pdf}{[Teil II]}). Diese Verhältnisse, aus ersten Prinzipien abgeleitet, erstrecken sich von quantenmechanischen bis kosmologischen Bereichen und untermauern die Universalität des T0-Modells \cite{pascher_alphabeta_2025}.
	
	\subsection{Vergleich mit anderen Einheitssystemen}
	\label{subsec:unit_comparison}
	
	Das T0-einheitliche System unterscheidet sich von traditionellen Rahmenwerken durch seine umfassende Vereinheitlichung:
	\begin{table*}
		\centering
		\caption{Vergleich von Einheitssystemen, einschließlich SI-Werten (annähernd) und Varianten natürlicher Einheiten.}
		\label{tab:unit_comparison}
		\scalebox{0.8}{
			\begin{tabular}{lccccccc}
				\hline
				\textbf{Einheitssystem} & \(\hbar\) & \(c\) & \(G\) & \(k_B\) & \(\alphaEM\) & \(\alphaW\) & \(\betaT\) \\
				\hline
				SI-Einheiten & \(1.055 \times 10^{-34}\) & \(3 \times 10^8\) & \(6.674 \times 10^{-11}\) & \(1.381 \times 10^{-23}\) & \(\sim 1/137\) & \(\sim 2.82\) & \(\sim 0.008\) \\
				Planck-Einheiten & 1 & 1 & 1 & 1 & \(\sim 1/137\) & \(\sim 2.82\) & variabel \\
				Elektrodynamisches NE & 1 & 1 & variabel & variabel & 1 & \(\sim 2.82\) & variabel \\
				Thermodynamisches NE & 1 & 1 & variabel & 1 & \(\sim 1/137\) & 1 & variabel \\
				T0-Einheitlich (Diese Arbeit) & 1 & 1 & 1 & 1 & 1 & 1 & 1 \\
				\hline
			\end{tabular}
		}
	\end{table*}
	
	Im Gegensatz zu Planck-Einheiten, die empirische Kopplungen beibehalten (z. B. \(\alphaEM\)), oder spezialisierten Systemen, die Teilbereiche fixieren (z. B. elektrodynamisches NE), vereinheitlicht T0 alle Konstanten theoretisch und sagt empirische Werte mit hoher Präzision voraus (z. B. \(\alphaEM = 1\) vs. \(1/137.036\), Abweichung \(< 10^{-6}\)) \cite{Duff2002,pascher_alphabeta_2025}.
	
	\subsection{Implikationen für die Physik}
	\label{subsec:unit_implications}
	
	Diese Vereinheitlichung hat tiefgreifende Implikationen:
	- **Eliminierung empirischer Konstanten:** Durch das theoretische Setzen von \(\hbar, c, G, k_B, \alphaEM, \alphaW, \betaT = 1\) entfällt die Notwendigkeit experimenteller Anpassung, wobei SI-Werte als emergente Eigenschaften vorhergesagt werden (z. B. \(c = 3 \times 10^8 \, \text{m/s}\) in SI stimmt mit \(c = 1\) in natürlichen Einheiten überein).
	- **Energie als universelles Maß:** Alle Phänomene – von Quantenübergängen bis zu Gravitationswechselwirkungen – werden in Energietermen ausgedrückt, was theoretische Konstrukte vereinfacht (Abschnitte 4 ''Feldtheoretische Formulierung'', 5 ''Emergente Gravitation'').
	- **Übereinstimmung mit Messungen:** Die Vorhersagen des Systems stimmen mit Beobachtungen überein (z. B. \(\betaT^{\text{SI}} \approx 0.008\)), was seine grundlegende Einheit validiert \cite{pascher_alphabeta_2025}.
	
	\begin{figure}[ht]
		\centering
		\begin{tikzpicture}
			\draw[->, thick] (0,0) -- (6,0) node[right] {\([E]\)};
			\draw[->, thick] (0,0) -- (0,6) node[above] {\([E^{-1}]\)};
			\node[blue, above right] at (2,5) {Länge, Zeit};
			\node[red, above right] at (5,2) {Masse, Energie};
			\node[green!60!black, above right] at (3,3.5) {\(\Tfield\)};
			\draw[dashed] (0,0) -- (5,5);
			\node[right] at (5,5) {Dualitätslinie};
		\end{tikzpicture}
		\caption{Dimensionale Beziehungen im T0-einheitlichen System, mit \(\Tfield\) als Vermittler zwischen Energie- und inversen Energieskalen, die die Dualität zwischen Masse und Zeit widerspiegelt.}
		\label{fig:dimensions}
	\end{figure}
	
	Dies bereitet die Einführung von \(\Tfield\) als vereinheitlichender Vermittler vor (Abschnitt 3 ''Intrinsisches Zeitfeld \(\Tfield\)'').
	
	\section{Intrinsisches Zeitfeld \(\Tfield\)}
	\label{sec:intrinsic_time}
	
	\subsection{Definition und physikalische Grundlage}
	\label{subsec:time_definition}
	
	Das intrinsische Zeitfeld ist der Eckpfeiler des T0-Modells, definiert als:
	\begin{equation}
		\Tfield = \frac{\hbar}{\max(mc^2, \omega)},
		\label{eq:intrinsic_time}
	\end{equation}
	wobei:
	- Für massive Teilchen: \(\Tfield = \frac{\hbar}{mc^2}\), mit Ruhezustand \(\Tzero = \frac{\hbar}{m_0 c^2}\),
	- Für Photonen: \(\Tfield = \frac{\hbar}{\omega}\), wobei \(\omega\) die Photonenergie/Frequenz ist.
	
	Diese Definition ergibt sich aus dem einheitlichen System natürlicher Einheiten (Abschnitt 2.2 ''Definition des einheitlichen Systems natürlicher Einheiten''). In SI-Einheiten gilt \(c = \frac{1}{\sqrt{\mu_0\varepsilon_0}}\), und die Energie \(E = mc^2\) legt nahe:
	\begin{equation}
		T = \frac{\hbar}{mc^2} = \frac{\hbar}{m} \cdot \mu_0\varepsilon_0,
		\label{eq:time_em}
	\end{equation}
	was mit \(\hbar = c = 1\) und \(\mu_0\varepsilon_0 = 1\) zu \(\Tfield = \frac{1}{m}\) für massive Teilchen in natürlichen Einheiten vereinfacht wird. Für Photonen gilt \(\omega = \frac{h}{\lambda} = \frac{2\pi\hbar c}{\lambda}\), und mit \(c = 1\) wird \(\Tfield = \frac{\hbar}{\omega}\), was Universalität über Teilchentypen hinweg sicherstellt. Dies bindet \(\Tfield\) an den energie-basierten Rahmen, wobei \(\hbar\) und \(c\) intrinsische Zeitskalen diktieren \cite{pascher_lagrange_2025}.
	
	Die physikalische Grundlage von \(\Tfield\) ist die Hypothese, dass jedes Teilchen eine inhärente temporale Skala besitzt, die umgekehrt proportional zu seiner Energie ist, und die relative Zeit der RT durch eine absolute, teilchenspezifische Eigenschaft ersetzt. Dieser Wandel interpretiert relativistische Effekte (z. B. Zeitdilatation) als Massenvariationen neu (Abschnitt 3.2 ''Transformationseigenschaften und Kovarianz'') und stimmt den Zeitparameter der QM mit den dynamischen Skalen der RT ab.
	
	\subsection{Transformationseigenschaften und Kovarianz}
	\label{subsec:transformations}
	
	Unter Lorentz-Transformationen transformiert \(\Tfield\) wie folgt:
	\begin{equation}
		\Tfield = \frac{\Tzero}{\gammaf}, \quad m = \gammaf m_0,
		\label{eq:transform}
	\end{equation}
	wobei \(\gammaf = \frac{1}{\sqrt{1 - v^2/c^2}}\) (mit \(c = 1\)) das Produkt erhält:
	\begin{equation}
		\Tfield \cdot m c^2 = \Tzero \cdot m_0 c^2 = \hbar.
		\label{eq:invariant_product}
	\end{equation}
	Das Transformationsgesetz lautet:
	\begin{equation}
		\delta\Tfield = -x^{\nu}\partial_{\mu}\Tfield\omega_{\nu}^{\mu},
		\label{eq:lorentz_transform}
	\end{equation}
	mit der kovarianten Ableitung, die Invarianz sicherstellt:
	\begin{equation}
		D_{\mu}\Tfield = \partial_{\mu}\Tfield + \Gamma_{\mu\nu}^{\rho}\Tfield,
		\label{eq:covariant_derivative}
	\end{equation}
	wobei \(\Gamma_{\mu\nu}^{\rho}\) Christoffel-Symbole sind, angepasst an die skalare Natur von \(\Tfield\). Diese Kovarianz bewahrt die Konsistenz mit den phänomenologischen Vorhersagen der RT (z. B. Lichtablenkung), während ihre Herkunft als Massenvariation statt Raumzeitkrümmung neu interpretiert wird \cite{pascher_lagrange_2025}.
	
	\subsection{Physikalische Interpretation}
	\label{subsec:time_interpretation}
	
	\(\Tfield\) repräsentiert die intrinsische „Uhr“ eines Teilchens, umgekehrt proportional zu seiner Energie:
	- **Schwere Teilchen:** Hohe \(m\), kurze \(\Tfield\), schnelle Dynamik.
	- **Leichte Teilchen/Photonen:** Niedrige \(m\) oder \(\omega\), lange \(\Tfield\), langsamere Dynamik.
	
	Dieses Skalarfeld durchdringt die Raumzeit, variiert mit lokalen Massen-Energie-Verteilungen und dient als Vermittler, der die Zeitentwicklung der QM mit den Gravitationseffekten der RT vereinigt. Zum Beispiel wird die verlängerte Lebensdauer eines bewegten Muons (traditionell Zeitdilatation) zu einer Massenzunahme (\(m = \gamma m_0\)), wobei \(\Tfield\) entsprechend angepasst wird und die beobachtbare Äquivalenz erhalten bleibt \cite{pascher_quantum_2025}.
	
	\section{Feldtheoretische Formulierung}
	\label{sec:field_theory}
	
	\subsection{Lagrange-Dichten}
	\label{subsec:lagrangian}
	
	Die Dynamik des T0-Modells ist in einer Gesamt-Lagrange-Dichte gekapselt:
	\begin{equation}
		\calL_{\text{Total}} = \calL_{\text{Boson}} + \calL_{\text{Fermion}} + \calL_{\text{Higgs-T}} + \calL_{\text{intrinsic}},
		\label{eq:total_lagrangian}
	\end{equation}
	mit Komponenten:
	- **Eichbosonen:** \(\calL_{\text{Boson}} = -\frac{1}{4}\Tfield^2 F_{\mu\nu}F^{\mu\nu}\), koppelt \(\Tfield\) an elektromagnetische Felder.
	- **Fermionen:** \(\calL_{\text{Fermion}} = \bar{\psi}i\gamma^{\mu}\DTmu\psi - y\bar{\psi}\Phi\psi\), wobei \(\DTmu\psi = \Tfield D_{\mu}\psi + \psi\partial_{\mu}\Tfield\) die kovariante Ableitung modifiziert.
	- **Higgs-Feld:** \(\calL_{\text{Higgs-T}} = |\DhiggsT|^2 - \lambda(|\Phi|^2 - v^2)^2\), integriert \(\Tfield\) mit Higgs-Wechselwirkungen.
	- **Intrinsische Zeit:** \(\calL_{\text{intrinsic}} = \frac{1}{2}\partial_{\mu}\Tfield\partial^{\mu}\Tfield - \frac{1}{2}\Tfield^2\), definiert \(\Tfield\) als Skalarfeld.
	
	Diese Begriffe stellen die universelle Rolle von \(\Tfield\) sicher und erweitern SM-Wechselwirkungen \cite{pascher_lagrange_2025}.
	
	\subsection{Erweiterung der Quantenmechanik}
	\label{subsec:qm_extension}
	
	Die standardmäßige Schrödinger-Gleichung:
	\begin{equation}
		i\hbar \frac{\partial}{\partial t} \Psi = \hat{H} \Psi,
		\label{eq:standard_schrodinger}
	\end{equation}
	nimmt eine einheitliche Zeit an. T0 modifiziert dies zu:
	\begin{equation}
		i\hbar \Tfield \frac{\partial}{\partial t} \Psi + i\hbar \Psi \frac{\partial \Tfield}{\partial t} = \hat{H} \Psi,
		\label{eq:modified_schrodinger}
	\end{equation}
	und führt eine massenabhängige Entwicklung ein. Die Dekohärenzrate wird:
	\begin{equation}
		\Gamma_{\text{dec}} = \Gamma_0 \cdot \frac{m c^2}{\hbar},
		\label{eq:decoherence}
	\end{equation}
	wobei schwerere Teilchen schneller dekohärieren. Für verschränkte Zustände:
	\begin{equation}
		|\Psi(t)\rangle = \frac{1}{\sqrt{2}}(|0(t/T_1)\rangle_{m_1} \otimes |1(t/T_2)\rangle_{m_2} + |1(t/T_1)\rangle_{m_1} \otimes |0(t/T_2)\rangle_{m_2}),
		\label{eq:entangled_state}
	\end{equation}
	wobei \(T_1 = \frac{\hbar}{m_1 c^2}\), \(T_2 = \frac{\hbar}{m_2 c^2}\), löst Nichtlokalität über massenspezifische Zeitskalen \cite{pascher_photons_2025}.
	
	\subsection{Anpassung der Quantenfeldtheorie}
	\label{subsec:qft_extension}
	
	\(\Tfield\) wird als Skalarfeld quantisiert mit der Gleichung:
	\begin{equation}
		\partial_{\mu}\partial^{\mu}\Tfield + \Tfield + \frac{\rho}{\Tfield^2} = 0,
		\label{eq:field_eq}
	\end{equation}
	wobei \(\rho\) die Massen-Energie-Dichte ist. Dies passt die QFT an, um relativistische Massenvariationen einzubeziehen und QM und RT auf Feldebene zu verbinden \cite{pascher_lagrange_2025}.
	
	\section{Emergente Gravitation}
	\label{sec:emergent_grav}
	
	\subsection{Ableitung aus \(\Tfield\)}
	\label{subsec:grav_derivation}
	
	Gravitation entsteht aus \(\Tfield\)-Gradienten. Unter statischen Bedingungen:
	\begin{equation}
		\nabla^2\Tfield \approx -\frac{\rho}{\Tfield^2},
		\label{eq:static_field}
	\end{equation}
	abgeleitet aus Gleichung \ref{eq:field_eq}. Das effektive Potential ist:
	\begin{equation}
		\Phi(\vecx) = -\ln\left(\frac{\Tfield}{\Tzero}\right),
		\label{eq:grav_potential_def}
	\end{equation}
	woraus die Kraft folgt:
	\begin{equation}
		\vec{F} = -\nabla\Phi = -\frac{\nabla\Tfield}{\Tfield}.
		\label{eq:force_from_potential}
	\end{equation}
	Für eine Punktmasse \(M\):
	\begin{equation}
		\Tfield(r) = \Tzero\left(1 - \frac{M}{r}\right),
		\label{eq:time_field_point_mass}
	\end{equation}
	also:
	\begin{equation}
		\vec{F} = -\frac{M}{r^2} \hat{r},
		\label{eq:newton_law}
	\end{equation}
	womit Newtons Gesetz ohne Raumzeitkrümmung reproduziert wird \cite{pascher_emergente_2025}.
	
	\subsection{Neuinterpretation der Relativitätstheorie}
	\label{subsec:rt_reinterpretation}
	
	Die Raumzeitkrümmung der RT wird durch \(\Tfield\)-Dynamiken ersetzt. Post-Newtonsche Tests (z. B. Lichtablenkung \(\delta\phi = \frac{4M}{b}\), Perihelpräzession \(\delta\omega = \frac{6\pi M}{a(1-e^2)}\)) stimmen mit GR überein mit Parametern \(\beta = \gamma = \zeta = 1\), was die beobachtungsmäßige Konsistenz sicherstellt \cite{Will2014}.
	
	\section{Diskussion}
	\label{sec:discussion}
	
	\subsection{Theoretische Vorteile}
	- **QM-RT-Vereinheitlichung:** \(\Tfield\) verbindet mikro- und makroskopische Physik.
	- **Einfachheit:** Energiebasierte Einheit reduziert Komplexität.
	- **Quantengravitation:** Emergente Gravitation stimmt mit QFT überein.
	
	\subsection{Herausforderungen und Lösungen}
	\label{subsec:challenges}
	
	Während das T0-Modell erhebliche theoretische Stärken zeigt, erforderte seine vollständige Realisierung die Bewältigung zentraler Herausforderungen, insbesondere die Quantisierung des intrinsischen Zeitfelds \(\Tfield\). Jüngste Fortschritte, detailliert in einer umfassenden quantenfeldtheoretischen (QFT) Behandlung \cite{pascher_qft_2025}, lösen diese Herausforderungen und verbessern die Kohärenz des Modells.
	
	\begin{enumerate}
		\item \textbf{Quantisierung des intrinsischen Zeitfelds:} Die klassische feldtheoretische Formulierung von \(\Tfield\), etabliert durch die Lagrange-Dichte \(\calL_{\text{intrinsic}} = \frac{1}{2}\partial_{\mu}\Tfield\partial^{\mu}\Tfield - \frac{1}{2}\Tfield^2\) \cite{pascher_lagrange_2025}, wurde nun auf einen vollständigen QFT-Rahmen erweitert. Dies umfasst kanonische Quantisierung, Pfadintegralformulierung, Renormierung und Unitaritätsanalyse, die eine Integration mit der Quantenmechanik und Konsistenz bei hohen Energien gewährleisten. Vorläufige Hinweise, dass \(\Tfield\) quantisiert werden könnte, mit \(\betaT\) als Renormierungsgruppen-Fixpunkt im Infrarot-Limit (\(\lim_{E \to 0} \betaT(E) = 1\)) \cite{pascher_alphabeta_2025}, wurden bestätigt, wodurch eine kritische Lücke geschlossen und T0 mit standardmäßigen QFT-Prinzipien abgeglichen wird.
		
		\item \textbf{Beobachtungsvalidierung von \(\betaT = 1\):} Im einheitlichen System natürlicher Einheiten definiert \(\betaT = 1\) die charakteristische Längenskala \(r_0 = \xi \cdot l_P\), mit \(\xi \approx 1.33 \times 10^{-4}\), abgeleitet aus Higgs-Parametern \cite{pascher_params_2025, pascher_alphabeta_2025}, im Gegensatz zum empirisch geschätzten \(\betaT^{\text{SI}} \approx 0.008\) aus kosmologischen Beobachtungen wie wellenlängenabhängiger Rotverschiebung \cite{pascher_messdifferenzen_2025}. Die QFT-Behandlung unterstützt \(\betaT = 1\) mathematisch innerhalb des natürlichen Einheitenrahmens, ohne Feinabstimmung, da es natürlich aus der Struktur des Modells hervorgeht. Die Validierung gegenüber hochpräzisen Daten (z. B. CMB-Temperaturskalierung, Galaxiendynamik) wird durch experimentelle Tests in Abschnitt 4 ''Quantitative Vorhersagen'' angegangen, wobei Quantenkorrekturen die Vorhersagepräzision erhöhen.
	\end{enumerate}
	
	Diese Fortschritte adressieren frühere Herausforderungen und verwandeln sie in Stärken. Das quantisierte \(\Tfield\) löst Probleme wie das Hierarchieproblem und die Vakuumenergiedichte, indem es Skalen natürlich verknüpft und kosmologische Phänomene ohne dunkle Komponenten neu interpretiert (Abschnitt 3), wodurch T0 zu einem überzeugenden, testbaren Rahmen wird.
	
	\section{Schlussfolgerung}
	\label{sec:conclusion}
	
	Teil II erweitert T0 in eine statische, testbare Kosmologie, interpretiert Rotverschiebung und Gravitationseffekte neu, mit einer robusten QFT-Grundlage, die ihre Tragfähigkeit erhöht \cite{pascher_perspective_2025}.
	
	\begin{acknowledgments}
		Dank an Reinsprecht Martin Dipl.-Ing. Dr. für kritisches Feedback.
	\end{acknowledgments}
	
	\bibliographystyle{apsrev4-2}
	\begin{thebibliography}{99}
		\bibitem{pascher_part1_2025} J. Pascher, \href{https://github.com/jpascher/T0-Time-Mass-Duality/tree/main/2/pdf/Deutsch/Bridging Quantum Mechanics and Relativity through Time-Mass Duality Part I Theoretical Foundations.pdf}{Überbrückung von Quantenmechanik und Relativitätstheorie durch Zeit-Masse-Dualität: Ein einheitlicher Rahmen mit natürlichen Einheiten \(\alpha = \beta = 1\) Teil I: Theoretische Grundlagen}, 7. April 2025.
		\bibitem{pascher_lagrange_2025} J. Pascher, \href{https://github.com/jpascher/T0-Time-Mass-Duality/tree/main/2/pdf/Deutsch/Mathematische Formulierungen der Zeit-Masse-Dualit\%C3\%A4tstheorie mit Lagrange.pdf}{Von Zeitdilatation zu Massenvariation: Mathematische Kernformulierungen der Zeit-Masse-Dualitätstheorie}, 29. März 2025.
		\bibitem{pascher_quantum_2025} J. Pascher, \href{https://github.com/jpascher/T0-Time-Mass-Duality/tree/main/2/pdf/Deutsch/Die Notwendigkeit einer Erweiterung der Standard-Quantenmechanik und Quantenfeldtheorie.pdf}{Die Notwendigkeit der Erweiterung der Standard-Quantenmechanik und Quantenfeldtheorie}, 27. März 2025.
		\bibitem{pascher_photons_2025} J. Pascher, \href{https://github.com/jpascher/T0-Time-Mass-Duality/tree/main/2/pdf/Deutsch/Dynamische Masse von Photonen und ihre Implikationen für Nichtlokalit\%C3\%A4t.pdf}{Dynamische Masse von Photonen und ihre Implikationen für Nichtlokalität im T0-Modell}, 25. März 2025.
		\bibitem{pascher_alphabeta_2025} J. Pascher, \href{https://github.com/jpascher/T0-Time-Mass-Duality/tree/main/2/pdf/Deutsch/Die Konsistenz von alpha = 1 und beta = 1.pdf}{Vereinheitlichtes Einheitensystem im T0-Modell: Die Konsistenz von \(\alpha = 1\) und \(\beta = 1\)}, 5. April 2025.
		\bibitem{pascher_emergente_2025} J. Pascher, \href{https://github.com/jpascher/T0-Time-Mass-Duality/tree/main/2/pdf/Deutsch/Emergente Gravitation im T0-Modell Eine formale Herleitung.pdf}{Emergente Gravitation im T0-Modell: Eine umfassende Herleitung}, 1. April 2025.
		\bibitem{pascher_perspective_2025} J. Pascher, \href{https://github.com/jpascher/T0-Time-Mass-Duality/tree/main/2/pdf/Deutsch/Eine neue Perspektive auf Zeit und Raum Johann Paschers revolution\%C3\%A4re Ideen.pdf}{Eine neue Perspektive auf Zeit und Raum: Johann Paschers revolutionäre Ideen}, 25. März 2025.
		\bibitem{schrodinger1926} E. Schrödinger, Phys. Rev. \textbf{28}, 1049 (1926).
		\bibitem{einstein1905} A. Einstein, Ann. Phys. \textbf{322}, 891 (1905).
		\bibitem{einstein1915} A. Einstein, Sitzungsber. Preuss. Akad. Wiss. \textbf{1915}, 844 (1915).
		\bibitem{bell1964} J. S. Bell, Physics \textbf{1}, 195 (1964).
		\bibitem{Planck2020} Planck Collaboration, Astron. Astrophys. \textbf{641}, A6 (2020).
		\bibitem{Riess1998} A. G. Riess et al., Astron. J. \textbf{116}, 1009 (1998).
		\bibitem{Planck1899} M. Planck, Proc. Prussian Acad. Sci. \textbf{5}, 440 (1899).
		\bibitem{Duff2002} M. J. Duff et al., J. High Energy Phys. \textbf{2002}, 023 (2002).
		\bibitem{Greene2020} B. Greene, \textit{Bis zum Ende der Zeit}, Knopf, New York (2020).
		\bibitem{tHooft1993} G. 't Hooft, arXiv:gr-qc/9310026 (1993).
		\bibitem{Will2014} C. M. Will, Living Rev. Relativ. \textbf{17}, 4 (2014).
		\bibitem{pascher_params_2025} J. Pascher, \href{https://github.com/jpascher/T0-Time-Mass-Duality/tree/main/2/pdf/Deutsch/Zeit-Masse-Dualit\%C3\%A4tstheorie (T0-Modell) Herleitung der Parameter kappa, alpha und beta.pdf}{Zeit-Masse-Dualitätstheorie (T0-Modell): Herleitung der Parameter \(\kappa\), \(\alpha\) und \(\beta\)}, 4. April 2025.
		\bibitem{pascher_messdifferenzen_2025} J. Pascher, \href{https://github.com/jpascher/T0-Time-Mass-Duality/tree/main/2/pdf/Deutsch/Analyse der Messdifferenzen zwischen dem T0-Modell und dem Standardmodell.pdf}{Kompensatorische und additive Effekte: Eine Analyse der Messdifferenzen zwischen dem T0-Modell und dem \(\Lambda\)CDM-Standardmodell}, 2. April 2025.
		\bibitem{pascher_qft_2025} J. Pascher, \href{https://github.com/jpascher/T0-Time-Mass-Duality/tree/main/2/pdf/Deutsch/Quantenfeldtheoretische Behandlung des intrinsischen Zeitfelds im T0-Modell.pdf}{Quantenfeldtheoretische Behandlung des intrinsischen Zeitfelds im T0-Modell}, 8. April 2025.
	\end{thebibliography}
	
\end{document}