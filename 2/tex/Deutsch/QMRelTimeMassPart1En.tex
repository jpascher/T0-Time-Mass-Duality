\documentclass[12pt,a4paper]{article}
\usepackage[utf8]{inputenc}
\usepackage[T1]{fontenc}
\usepackage[ngerman]{babel}
\usepackage{lmodern}
\usepackage{amsmath}
\usepackage{amssymb}
\usepackage{physics}
\usepackage{hyperref}
\usepackage{tcolorbox}
\usepackage{booktabs}
\usepackage{enumitem}
\usepackage[table,xcdraw]{xcolor}
\usepackage[left=2cm,right=2cm,top=2cm,bottom=2cm]{geometry}
\usepackage{pgfplots}
\pgfplotsset{compat=1.18}
\usepackage{graphicx}
\usepackage{float}
\usepackage{fancyhdr}
\usepackage{siunitx}

% Danksagungs-Umgebung
\newenvironment{acknowledgments}
{\section*{Danksagungen}}
{\vspace{1em}}

% Benutzerdefinierte Befehle
\newcommand{\Tfield}{T(x)}
\newcommand{\alphaEM}{\alpha_{\text{EM}}}
\newcommand{\alphaW}{\alpha_{\text{W}}}
\newcommand{\betaT}{\beta_{\text{T}}}
\newcommand{\Mpl}{M_{\text{Pl}}}
\newcommand{\Tzerot}{T_0(\Tfield)}
\newcommand{\Tzero}{T_0}
\newcommand{\vecx}{\vec{x}}
\newcommand{\vr}{\vec{r}}
\newcommand{\gammaf}{\gamma_{\text{Lorentz}}}
\newcommand{\DhiggsT}{\Tfield (\partial_\mu + ig A_\mu) \Phi + \Phi \partial_\mu \Tfield}
\newcommand{\LCDM}{\Lambda\text{CDM}}
\newcommand{\DTmu}{D_{T,\mu}}
\newcommand{\calL}{\mathcal{L}}
\newcommand{\deq}{\displaystyle}
\newcommand{\e}{\mathrm{e}}

% Kopf- und Fußzeilen-Konfiguration
\pagestyle{fancy}
\fancyhf{}
\fancyhead[L]{Johann Pascher}
\fancyhead[R]{Zeit-Masse-Dualität: Teil I}
\fancyfoot[C]{\thepage}
\renewcommand{\headrulewidth}{0.4pt}
\renewcommand{\footrulewidth}{0.4pt}

\hypersetup{
	colorlinks=true,
	linkcolor=blue,
	citecolor=blue,
	urlcolor=blue,
	pdftitle={Überbrückung von Quantenmechanik und Relativitätstheorie durch Zeit-Masse-Dualität: Teil I},
	pdfauthor={Johann Pascher},
	pdfsubject={Theoretische Physik},
	pdfkeywords={T0-Modell, natürliche Einheiten, Zeit-Masse-Dualität}
}

\title{Überbrückung von Quantenmechanik und Relativitätstheorie durch Zeit-Masse-Dualität: \\ Ein einheitlicher Rahmen mit natürlichen Einheiten \(\alpha = \beta = 1\) \\ Teil I: Theoretische Grundlagen}
\author{Johann Pascher\\
	Abteilung für Kommunikationstechnik\\
	Höhere Technische Bundeslehranstalt (HTL), Leonding, Österreich\\
	\texttt{johann.pascher@gmail.com}}
\date{7. April 2025}

\begin{document}
	
	\maketitle
	
	\begin{abstract}
		Diese Arbeit stellt das T0-Modell der Zeit-Masse-Dualität vor, einen neuartigen theoretischen Rahmen, der Quantenmechanik (QM) und Relativitätstheorie (RT) vereinheitlicht, indem ihre grundlegenden Konzepte durch absolute Zeit und variable Masse neu definiert werden. Wir etablieren ein einheitliches natürliches Einheitensystem, in dem \(\hbar = c = G = k_B = \alphaEM = \alphaW = \betaT = 1\) gilt, wodurch empirisch bestimmte Konstanten eliminiert werden, während eine bemerkenswerte Konsistenz mit experimentellen Messungen erreicht wird, mit Abweichungen unter \(10^{-6}\). Das intrinsische Zeitfeld \(\Tfield = \frac{\hbar}{\max(mc^2, \omega)}\) dient als Grundstein, erweitert die QM mit einer masseabhängigen Schrödinger-Gleichung und interpretiert die Gravitationseffekte der RT als emergent aus Felddynamiken. Teil I konzentriert sich auf diese theoretischen Grundlagen — Vereinheitlichung von Konstanten, Definition von \(\Tfield\), feldtheoretische Formulierung und emergente Gravitation — und überbrückt damit mikro- und makroskopische Physik. Teil II wird kosmologische Implikationen und experimentelle Validierung untersuchen und auf diesen Grundlagen aufbauen.
	\end{abstract}
	\newpage
	\tableofcontents
	\newpage
	\section{Einleitung}
	\label{sec:introduction}
	
	Die Vereinheitlichung von Quantenmechanik (QM) und Relativitätstheorie (RT) ist seit über einem Jahrhundert eine zentrale Herausforderung in der theoretischen Physik, bedingt durch ihre grundlegend unterschiedlichen Behandlungen von Zeit, Raum und Masse. Die QM, verwurzelt in Schrödingers Wellenmechanik, behandelt Zeit als einheitlichen Parameter ohne Operatorstatus (\(i\hbar \frac{\partial}{\partial t}\Psi = \hat{H}\Psi\)) \cite{schrodinger1926} und beschreibt hervorragend mikroskopische Phänomene wie Teilchenverhalten und Verschränkung. Im Gegensatz dazu definiert die RT, einschließlich Einsteins spezieller und allgemeiner Theorien, Zeit als relative Dimension (\(t' = \gammaf t\)), die mit dem Raum verflochten ist, wobei die Masse als Konstante behandelt wird, die makroskopische Phänomene wie Gravitation und Raumzeitkrümmung regelt \cite{einstein1905,einstein1915}. Diese Unterschiede haben eine zusammenhängende Theorie behindert und erschweren die Erklärung von Quantengravitation, Nichtlokalität \cite{bell1964} und kosmologischen Modellen wie \(\Lambda\)CDM \cite{Planck2020}.
	
	Das T0-Modell der Zeit-Masse-Dualität bietet ein neuartiges Paradigma, um diese Rahmenbedingungen zu vereinen, indem es ihre traditionellen Annahmen umkehrt: Zeit ist absolut und Masse variiert, vermittelt durch ein intrinsisches Zeitfeld \(\Tfield\). Dieser Ansatz basiert auf einem einheitlichen natürlichen Einheitensystem, in dem alle fundamentalen Konstanten (\(\hbar = c = G = k_B = \alphaEM = \alphaW = \betaT = 1\)) auf Eins gesetzt werden, nicht als empirische Anpassung, sondern als theoretische Notwendigkeit, wodurch alle physikalischen Größen auf Energie reduziert werden. Bemerkenswerterweise stimmt dieses System mit den gemessenen Werten (z.B. \(c \approx 3 \times 10^8 \, \text{m/s}\), \(\alphaEM \approx 1/137.036\)) mit Abweichungen unter \(10^{-6}\) überein, was über verschiedene Skalen von Quanten- bis zu kosmologischen Phänomenen validiert wurde (siehe Teil II, Abschnitt 4 "Quantitative Vorhersagen" \href{https://github.com/jpascher/T0-Time-Mass-Duality/tree/main/2/pdf/Deutsch/QMRelTimeMassPart2.pdf}{[Teil II]}).
	
	Durch die Erweiterung der QM mit einer masseabhängigen Zeitentwicklung (Abschnitt 4.2 "Erweiterung der Quantenmechanik") und die Neuinterpretation der Gravitationseffekte der RT als emergent aus \(\Tfield\)-Gradienten (Abschnitt 5.1 "Ableitung aus \(\Tfield\)") überbrückt T0 mikro- und makroskopische Physik ohne zusätzliche Dimensionen oder quantisierte Raumzeit, wie in der Stringtheorie oder Schleifenquantengravitation \cite{Greene2020,tHooft1993}. Teil I etabliert diese theoretischen Grundlagen, während Teil II deren kosmologische und experimentelle Implikationen erforschen wird.
	
	Diese Arbeit ist wie folgt strukturiert:
	- Abschnitt 2: Vereinheitlichung von Konstanten mit natürlichen Einheiten.
	- Abschnitt 3: Definition und Eigenschaften von \(\Tfield\).
	- Abschnitt 4: Feldtheoretische Formulierung zur Erweiterung von QM und RT.
	- Abschnitt 5: Emergente Gravitation zur Neuinterpretation der RT.
	- Abschnitt 6: Diskussion der Implikationen und Herausforderungen.
	- Abschnitt 7: Schlussfolgerung und Ausblick.
	
	\section{Vereinheitlichung von Konstanten mit natürlichen Einheiten}
	\label{sec:unified_units}
	
	\subsection{Motivation für natürliche Einheiten}
	\label{subsec:motivation_units}
	
	Physikalische Konstanten wie die Lichtgeschwindigkeit \(c\), das reduzierte Plancksche Wirkungsquantum \(\hbar\), die Gravitationskonstante \(G\) und die Feinstrukturkonstante \(\alphaEM\) werden traditionell als empirisch bestimmt betrachtet und spiegeln die Skalen der Natur in von Menschen definierten Einheiten wie Meter und Sekunden wider. In konventionellen natürlichen Einheitensystemen (z.B. \(\hbar = c = 1\)) werden diese Konstanten auf Eins gesetzt, um mathematische Formulierungen zu vereinfachen und intrinsische physikalische Beziehungen zu enthüllen \cite{Planck1899,Duff2002}. Zum Beispiel vereinheitlicht die Festlegung von \(c = 1\) Raum- und Zeitdimensionen (\([L] = [T]\)), während \(\hbar = 1\) Energie und inverse Zeit gleichsetzt (\([E] = [T]^{-1}\)) und so Gleichungen in QM und RT vereinfacht.
	
	Das T0-Modell treibt diese Vereinheitlichung einen Schritt weiter, indem es postuliert, dass alle fundamentalen Konstanten — über dimensionsbehaftete wie \(\hbar\) und \(c\) hinaus, auch dimensionslose Kopplungen wie \(\alphaEM\) und \(\betaT\) — bei 1 vereinheitlicht werden sollten, nicht als Bequemlichkeit, sondern als Ausdruck einer tieferen, intrinsischen Einheit in der Natur. Dieser Ansatz wird durch die Beobachtung motiviert, dass traditionelle SI-Einheiten künstliche Komplexität einführen. Zum Beispiel definieren die elektromagnetischen Konstanten \(\mu_0\) (Permeabilität) und \(\varepsilon_0\) (Permittivität) die Lichtgeschwindigkeit als \(c = \frac{1}{\sqrt{\mu_0\varepsilon_0}}\), doch ihre spezifischen Werte (\(\mu_0 = 4\pi \times 10^{-7} \, \text{H/m}\), \(\varepsilon_0 = 8.854 \times 10^{-12} \, \text{F/m}\)) werden empirisch festgelegt und nicht theoretisch abgeleitet. Das T0-Modell behauptet, dass die Festlegung von \(c = 1\) als grundlegende Eigenschaft, und nicht als gemessenes Ergebnis, diese Willkürlichkeit beseitigt und darauf hindeutet, dass elektromagnetische Eigenschaften von Natur aus mit Zeit- und Energieskalen verbunden sind, eine Verbindung, die später durch das intrinsische Zeitfeld \(\Tfield\) (Abschnitt 3.1 "Definition und physikalische Grundlage") formalisiert wird.
	
	Diese Vereinheitlichung ist nicht nur eine mathematische Vereinfachung, sondern eine philosophische Haltung: Physikalische Konstanten sind keine unabhängigen Parameter, die experimentell abgestimmt werden müssen, sondern Manifestationen eines einzigen zugrundeliegenden Prinzips — Energie als universelles Maß. Durch die Eliminierung empirischer Abhängigkeiten zielt das T0-Modell darauf ab, einen in sich konsistenten Rahmen zu konstruieren, der natürlich mit beobachteten Phänomenen übereinstimmt, wie durch seine Vorhersagekraft validiert (siehe Teil II, Abschnitt 4 "Quantitative Vorhersagen" \href{https://github.com/jpascher/T0-Time-Mass-Duality/tree/main/2/pdf/Deutsch/QMRelTimeMassPart2.pdf}{[Teil II]}).
	
	\subsection{Definition des einheitlichen natürlichen Einheitensystems}
	\label{subsec:unified_system}
	
	Das T0-Modell übernimmt ein einheitliches natürliches Einheitensystem, definiert durch:
	\begin{align}
		\hbar &= c = G = k_B = \alphaEM = \alphaW = \betaT = 1,
		\label{eq:unit_system}
	\end{align}
	wobei jede Konstante aufgrund theoretischer Notwendigkeit und nicht empirischer Anpassung auf Eins gesetzt wird. Diese Konstanten repräsentieren:
	- \(\hbar = 1\): Quantenwirkungsskala, traditionell \(1.055 \times 10^{-34} \, \text{Js}\) in SI-Einheiten, die die Skala der Quantenphänomene bestimmt.
	- \(c = 1\): Raumzeit-Vereinheitlichung, traditionell \(3 \times 10^8 \, \text{m/s}\), verbindet räumliche und zeitliche Dimensionen.
	- \(G = 1\): Gravitationskopplungsstärke, traditionell \(6.674 \times 10^{-11} \, \text{m}^3\text{kg}^{-1}\text{s}^{-2}\), definiert makroskopische Wechselwirkungen.
	- \(k_B = 1\): Boltzmann-Konstante, traditionell \(1.381 \times 10^{-23} \, \text{J/K}\), setzt thermische Energie in Beziehung zur Temperatur.
	- \(\alphaEM = \frac{e^2}{4\pi\varepsilon_0\hbar c} = 1\): Feinstrukturkonstante, traditionell \(\approx 1/137.036\), vereinheitlicht elektromagnetische Wechselwirkungen und macht Ladung dimensionslos (\(e = \sqrt{4\pi\varepsilon_0}\)).
	- \(\alphaW = 1\): Wiensche Verschiebungskonstante, traditionell \(\approx 2.821439\), richtet thermische Strahlungsfrequenz an der Temperatur aus (\(\nu_{\text{max}} = \frac{k_B T}{h}\)).
	- \(\betaT = 1\): T0-Kopplungsparameter, traditionell \(\approx 0.008\) in SI-Einheiten, normalisiert die Wechselwirkungsstärke von \(\Tfield\) mit Materie und Feldern.
	
	Im Gegensatz zu konventionellen natürlichen Einheitensystemen (z.B. Planck-Einheiten), bei denen Konstanten wie \(\hbar, c, G\) aus Messbequemlichkeit auf 1 gesetzt werden und andere (z.B. \(\alphaEM\)) variabel bleiben, vereinheitlicht das T0-Modell alle Konstanten — einschließlich dimensionsloser — auf theoretischer Basis. Dieses System passt sich nicht an, um experimentelle Daten zu ergänzen, sondern sagt sie voraus, wobei eine bemerkenswerte Konsistenz mit gemessenen Werten (z.B. \(c = 3 \times 10^8 \, \text{m/s}\) entspricht 1 in natürlichen Einheiten mit \(< 10^{-6}\) Abweichung bei Rückumrechnung) erreicht wird \cite{pascher_alphabeta_2025}.
	
	\subsubsection{Dimensionale Zuordnungen}
	In diesem System werden alle physikalischen Größen in Bezug auf Energie (\([E]\)) ausgedrückt, wobei unabhängige Dimensionen für Länge, Zeit und Masse eliminiert werden:
	\begin{table}[ht]
		\centering
		\caption{Dimensionale Zuordnungen im T0-einheitlichen natürlichen Einheitensystem.}
		\label{tab:dimensions}
		\scalebox{0.8}{
			\begin{tabular}{ll}
				\hline
				\textbf{Physikalische Größe} & \textbf{Dimension in T0-Einheiten} \\
				\hline
				Länge & \([E^{-1}]\) \\
				Zeit & \([E^{-1}]\) \\
				Masse & \([E]\) \\
				Energie & \([E]\) \\
				Temperatur & \([E]\) \\
				Elektrische Ladung & \([1]\) (dimensionslos) \\
				Intrinsische Zeit (\(\Tfield\)) & \([E^{-1}]\) \\
				\hline
			\end{tabular}
		}
	\end{table}
	
	Zum Beispiel teilen Länge und Zeit die Dimension \([E^{-1}]\), weil \(c = 1\) impliziert \([L] = [T]\), und \(\hbar = 1\) verbindet Zeit mit inverser Energie (\([T] = [E^{-1}]\)). Masse und Energie sind äquivalent (\([M] = [E]\)) aufgrund von \(c = 1\), und Temperatur stimmt mit Energie über \(k_B = 1\) überein. Ladung wird dimensionslos mit \(\alphaEM = 1\), was elektromagnetische Wechselwirkungen vereinfacht.
	
	\subsubsection{Rolle elektromagnetischer Konstanten}
	Die Lichtgeschwindigkeit in SI-Einheiten ist definiert als \(c = \frac{1}{\sqrt{\mu_0\varepsilon_0}}\), wobei \(\mu_0 = 4\pi \times 10^{-7} \, \text{H/m}\) und \(\varepsilon_0 = 8.854 \times 10^{-12} \, \text{F/m}\) empirisch bestimmte Konstanten sind, die \(c \approx 3 \times 10^8 \, \text{m/s}\) ergeben. Im T0-System impliziert die Festlegung von \(c = 1\) theoretisch \(\mu_0\varepsilon_0 = 1\) und eliminiert diese als unabhängige Parameter. Ebenso wird die Feinstrukturkonstante \(\alphaEM = \frac{e^2}{4\pi\varepsilon_0\hbar c}\) zu 1, passt die Rolle von \(\varepsilon_0\) an und macht die Ladung \(e\) zu einer abgeleiteten Größe (\(e = \sqrt{4\pi\varepsilon_0}\)). Die Plancksche Konstante verbindet sich über diesen Rahmen mit:
	\begin{equation}
		h = 2\pi\hbar = \frac{1}{\sqrt{\mu_0\varepsilon_0}} \cdot \text{(Skalierungsfaktor)},
		\label{eq:planck_em}
	\end{equation}
	was darauf hindeutet, dass Zeitskalen (\(T = \frac{h}{E}\)) von Natur aus mit elektromagnetischen Eigenschaften verbunden sind, ein Vorläufer der Definition von \(\Tfield\) (Abschnitt 3.1 "Definition und physikalische Grundlage"). Diese Vereinheitlichung reduziert die Komplexität elektromagnetischer Wechselwirkungen auf energiebasierte Terme und stimmt mit dem Kernprinzip des T0-Modells überein.
	
	\subsubsection{Längenskalen und entsprechende Konstanten}
	\label{subsec:length_scales}
	
	Das einheitliche System des T0-Modells definiert Längenskalen in Bezug auf Energie neu und verbindet sie mit fundamentalen Konstanten und deren Verhältnissen. Tabelle \ref{tab:length_scales} fasst wichtige Längenskalen, ihre Ausdrücke in SI- und natürlichen Einheiten sowie die Konstanten, die sie repräsentieren, zusammen und bietet eine Brücke zwischen theoretischen Konstrukten und beobachtbaren Phänomenen:
	
	\begin{table}[ht]
		\centering
		\caption{Längenskalen im T0-Modell und ihre entsprechenden Konstanten.}
		\label{tab:length_scales}
		\scalebox{0.8}{
			\begin{tabular}{lccc}
				\hline
				\textbf{Längenskala} & \textbf{SI-Ausdruck} & \textbf{T0-Natürliche Einheiten} & \textbf{Repräsentierte Konstanten} \\
				\hline
				Planck-Länge (\(l_P\)) & \(\sqrt{\frac{\hbar G}{c^3}}\) & 1 & \(\hbar, G, c\) \\
				Compton-Wellenlänge (\(\lambda_C\)) & \(\frac{\hbar}{m c}\) & \(\frac{1}{m}\) & \(\hbar, c, m\) \\
				T0-Charakteristische Länge (\(r_0\)) & \(\xi l_P\) & \(1.33 \times 10^{-4}\) & \(\hbar, G, c, \lambda_h, v, m_h\) \\
				Kosmologische Korrelationslänge (\(L_T\)) & \(\frac{L_T}{l_P} \cdot l_P\) & \(3.9 \times 10^{62}\) & \(\hbar, G, c, \betaT\) \\
				\hline
			\end{tabular}
		}
	\end{table}
	
	- **Planck-Länge (\(l_P\)):** Definiert als \(\sqrt{\frac{\hbar G}{c^3}} \approx 1.616 \times 10^{-35} \, \text{m}\) in SI-Einheiten, wird sie zur grundlegenden Längeneinheit (\(l_P = 1\)) in T0-natürlichen Einheiten und repräsentiert die Skala, auf der \(\hbar, G,\) und \(c\) konvergieren.
	- **Compton-Wellenlänge (\(\lambda_C\)):** Gegeben durch \(\frac{\hbar}{m c}\), skaliert sie invers mit der Masse (\(\lambda_C = \frac{1}{m}\)) in natürlichen Einheiten, ist mit \(\hbar\) und \(c\) verbunden und spiegelt die Quantenskala der Wellennatur eines Teilchens wider.
	- **T0-Charakteristische Länge (\(r_0\)):** Abgeleitet als \(\xi l_P\), wobei \(\xi = \frac{\lambda_h^2 v^2}{16\pi^3 m_h^2} \approx 1.33 \times 10^{-4}\), verbindet sie Higgs-Parameter (\(\lambda_h\): Selbstkopplung, \(v\): Vakuumerwartungswert, \(m_h\): Higgs-Masse) mit der Planck-Skala und repräsentiert den Mikroskala-Anker des T0-Modells.
	- **Kosmologische Korrelationslänge (\(L_T\)):** Definiert über das Verhältnis \(L_T/l_P \approx 3.9 \times 10^{62}\), sie entsteht aus der Dynamik von \(\Tfield\) und \(\betaT\) und repräsentiert die makroskopische Skala der kosmischen Struktur (siehe Teil II, Abschnitt 2 "Statisches Universumsmodell" \href{https://github.com/jpascher/T0-Time-Mass-Duality/tree/main/2/pdf/Deutsch/QMRelTimeMassPart2.pdf}{[Teil II]}).
	
	Diese Längenskalen veranschaulichen, wie das T0-Modell mikro- und makroskopische Physik durch energiebasierte Einheiten und die Konstanten \(\hbar, c, G\), erweitert durch Higgs- und T0-spezifische Parameter, integriert. Die Verhältnisse (z.B. \(\xi, L_T/l_P\)) werden theoretisch abgeleitet, nicht empirisch angepasst, und ihre Konsistenz mit Beobachtungen (z.B. \(l_P\) als Quantengravitationsskala, \(L_T\) als kosmische Skala) validiert das einheitliche System \cite{pascher_alphabeta_2025}.
	
	\subsection{Hierarchie der Einheiten und abgeleitete Konstanten}
	\label{subsec:hierarchy}
	
	Das einheitliche System etabliert eine Hierarchie von Skalen:
	- **Basiseinheiten:** \(\hbar = c = G = k_B = 1\) definieren Energie als primäre Dimension und legen die Grundlage für alle physikalischen Größen.
	- **Kopplungskonstanten:** \(\alphaEM = \alphaW = \betaT = 1\) vereinheitlichen Wechselwirkungsstärken über elektromagnetische, thermische und T0-spezifische Domänen hinweg und eliminieren freie Parameter.
	- **Abgeleitete Skalen:** Schlüsselverhältnisse entstehen aus dieser Einheit, wie in Tabelle \ref{tab:derived_constants} gezeigt:
	\begin{table}[ht]
		\centering
		\caption{Abgeleitete Konstanten im T0-Modell, die Skalenhierarchien repräsentieren.}
		\label{tab:derived_constants}
		\scalebox{0.8}{
			\begin{tabular}{llr}
				\hline
				\textbf{Abgeleitete Konstante} & \textbf{Wert} & \textbf{Physikalische Bedeutung} \\
				\hline
				\(\xi = r_0/l_P\) & \(1.33 \times 10^{-4}\) & Verhältnis T0-Länge zu Planck-Länge \\
				\(L_T/l_P\) & \(3.9 \times 10^{62}\) & Kosmologische Korrelationslänge \\
				\(r_0/L_T\) & \(3.41 \times 10^{-67}\) & Mikro-zu-Makro-Skalenbeziehung \\
				\hline
			\end{tabular}
		}
	\end{table}
	
	Der Parameter \(\xi = \frac{\lambda_h^2 v^2}{16\pi^3 m_h^2}\) verbindet den Higgs-Sektor (\(\lambda_h \approx 0.13\), \(v \approx 246 \, \text{GeV}\), \(m_h \approx 125 \, \text{GeV}\)) mit der Planck-Skala, während \(L_T\) die Dynamik von \(\Tfield\) mit kosmischen Skalen verbindet (Teil II, Abschnitt 2 "Statisches Universumsmodell" \href{https://github.com/jpascher/T0-Time-Mass-Duality/tree/main/2/pdf/Deutsch/QMRelTimeMassPart2.pdf}{[Teil II]}). Diese Verhältnisse, aus ersten Prinzipien abgeleitet, erstrecken sich von Quanten- bis zu kosmologischen Bereichen und verstärken die Universalität des T0-Modells \cite{pascher_alphabeta_2025}.
	
	\subsection{Vergleich mit anderen Einheitensystemen}
	\label{subsec:unit_comparison}
	
	Das T0-einheitliche System unterscheidet sich von traditionellen Rahmenwerken durch seine umfassende Vereinheitlichung:
	\begin{table*}
		\centering
		\caption{Vergleich von Einheitensystemen, einschließlich SI-Werte (circa) und Varianten natürlicher Einheiten.}
		\label{tab:unit_comparison}
		\scalebox{0.8}{
			\begin{tabular}{lccccccc}
				\hline
				\textbf{Einheitensystem} & \(\hbar\) & \(c\) & \(G\) & \(k_B\) & \(\alphaEM\) & \(\alphaW\) & \(\betaT\) \\
				\hline
				SI-Einheiten & \(1.055 \times 10^{-34}\) & \(3 \times 10^8\) & \(6.674 \times 10^{-11}\) & \(1.381 \times 10^{-23}\) & \(\sim 1/137\) & \(\sim 2.82\) & \(\sim 0.008\) \\
				Planck-Einheiten & 1 & 1 & 1 & 1 & \(\sim 1/137\) & \(\sim 2.82\) & variabel \\
				Elektrodynamische NE & 1 & 1 & variabel & variabel & 1 & \(\sim 2.82\) & variabel \\
				Thermodynamische NE & 1 & 1 & variabel & 1 & \(\sim 1/137\) & 1 & variabel \\
				T0-Einheitlich (Diese Arbeit) & 1 & 1 & 1 & 1 & 1 & 1 & 1 \\
				\hline
			\end{tabular}
		}
	\end{table*}
	
	Im Gegensatz zu Planck-Einheiten, die empirische Kopplungen beibehalten (z.B. \(\alphaEM\)), oder spezialisierte Systeme, die Teilmengen fixieren (z.B. elektrodynamische NE), vereinheitlicht T0 alle Konstanten theoretisch und sagt empirische Werte mit hoher Präzision voraus (z.B. \(\alphaEM = 1\) vs. \(1/137.036\), Abweichung \(< 10^{-6}\)) \cite{Duff2002,pascher_alphabeta_2025}.
	
	\subsection{Implikationen für die Physik}
	\label{subsec:unit_implications}
	
	Diese Vereinheitlichung hat tiefgreifende Implikationen:
	- **Eliminierung empirischer Konstanten:** Durch die theoretische Festlegung von \(\hbar, c, G, k_B, \alphaEM, \alphaW, \betaT = 1\) entfernt T0 die Notwendigkeit experimenteller Abstimmung und sagt SI-Werte als emergente Eigenschaften voraus (z.B. stimmt \(c = 3 \times 10^8 \, \text{m/s}\) in SI mit \(c = 1\) in natürlichen Einheiten überein).
	- **Energie als universelles Maß:** Alle Phänomene — von Quantenübergängen bis zu Gravitationswechselwirkungen — werden in Energiebegriffen ausgedrückt, was theoretische Konstrukte vereinfacht (Abschnitte 4 "Feldtheoretische Formulierung", 5 "Emergente Gravitation").
	- **Konsistenz mit Messungen:** Die Vorhersagen des Systems stimmen mit Beobachtungen überein (z.B. \(\betaT^{\text{SI}} \approx 0.008\)), was seine grundlegende Einheit validiert \cite{pascher_alphabeta_2025}.
	
	\begin{figure}[ht]
		\centering
		\begin{tikzpicture}
			\draw[->, thick] (0,0) -- (6,0) node[right] {\([E]\)};
			\draw[->, thick] (0,0) -- (0,6) node[above] {\([E^{-1}]\)};
			\node[blue, above right] at (2,5) {Länge, Zeit};
			\node[red, above right] at (5,2) {Masse, Energie};
			\node[green!60!black, above right] at (3,3.5) {\(\Tfield\)};
			\draw[dashed] (0,0) -- (5,5);
			\node[right] at (5,5) {Dualitätslinie};
		\end{tikzpicture}
		\caption{Dimensionale Beziehungen im T0-einheitlichen System, wobei \(\Tfield\) Energie- und inverse Energieskalen vermittelt und die Dualität zwischen Masse und Zeit widerspiegelt.}
		\label{fig:dimensions}
	\end{figure}
	
	Dies bereitet die Einführung von \(\Tfield\) als vereinheitlichenden Vermittler vor (Abschnitt 3 "Intrinsisches Zeitfeld \(\Tfield\)").
	
	\section{Intrinsisches Zeitfeld \(\Tfield\)}
	\label{sec:intrinsic_time}
	
	\subsection{Definition und physikalische Grundlage}
	\label{subsec:time_definition}
	
	Das intrinsische Zeitfeld ist der Grundstein des T0-Modells, definiert als:
	\begin{equation}
		\Tfield = \frac{\hbar}{\max(mc^2, \omega)},
		\label{eq:intrinsic_time}
	\end{equation}
	wobei:
	- Für massive Teilchen: \(\Tfield = \frac{\hbar}{mc^2}\), mit Ruhezustand \(\Tzero = \frac{\hbar}{m_0 c^2}\),
	- Für Photonen: \(\Tfield = \frac{\hbar}{\omega}\), wobei \(\omega\) die Photonenenergie/Frequenz ist.
	
	Diese Definition geht aus dem einheitlichen Einheitensystem hervor (Abschnitt 2.2 "Definition des einheitlichen natürlichen Einheitensystems"). In SI-Einheiten gilt \(c = \frac{1}{\sqrt{\mu_0\varepsilon_0}}\), und Energie \(E = mc^2\) legt nahe:
	\begin{equation}
		T = \frac{\hbar}{mc^2} = \frac{\hbar}{m} \cdot \mu_0\varepsilon_0,
		\label{eq:time_em}
	\end{equation}
	was sich mit \(\hbar = c = 1\) und \(\mu_0\varepsilon_0 = 1\) für massive Teilchen in natürlichen Einheiten zu \(\Tfield = \frac{1}{m}\) vereinfacht. Für Photonen gilt \(\omega = \frac{h}{\lambda} = \frac{2\pi\hbar c}{\lambda}\), und mit \(c = 1\) wird \(\Tfield = \frac{\hbar}{\omega}\), was die Universalität über Teilchentypen hinweg sicherstellt. Dies verbindet \(\Tfield\) mit dem energiebasierten Rahmenwerk, in dem \(\hbar\) und \(c\) intrinsische Zeitskalen diktieren \cite{pascher_lagrange_2025}.
	
	Die physikalische Grundlage von \(\Tfield\) ist die Hypothese, dass jedes Teilchen eine inhärente zeitliche Skala besitzt, die umgekehrt proportional zu seiner Energie ist, wodurch die relative Zeit der RT durch eine absolute, teilchenspezifische Eigenschaft ersetzt wird. Diese Verschiebung interpretiert relativistische Effekte (z.B. Zeitdilatation) als Massenvariationen (Abschnitt 3.2 "Transformationseigenschaften und Kovarianz") neu und bringt den Zeitparameter der QM mit den dynamischen Skalen der RT in Einklang.
	
	\subsection{Transformationseigenschaften und Kovarianz}
	\label{subsec:transformations}
	
	Unter Lorentz-Transformationen transformiert sich \(\Tfield\) als:
	\begin{equation}
		\Tfield = \frac{\Tzero}{\gammaf}, \quad m = \gammaf m_0,
		\label{eq:transform}
	\end{equation}
	wobei \(\gammaf = \frac{1}{\sqrt{1 - v^2/c^2}}\) (mit \(c = 1\)), das Produkt erhaltend:
	\begin{equation}
		\Tfield \cdot m c^2 = \Tzero \cdot m_0 c^2 = \hbar.
		\label{eq:invariant_product}
	\end{equation}
	Das Transformationsgesetz ist:
	\begin{equation}
		\delta\Tfield = -x^{\nu}\partial_{\mu}\Tfield\omega_{\nu}^{\mu},
		\label{eq:lorentz_transform}
	\end{equation}
	wobei die kovariante Ableitung die Invarianz sicherstellt:
	\begin{equation}
		D_{\mu}\Tfield = \partial_{\mu}\Tfield + \Gamma_{\mu\nu}^{\rho}\Tfield,
		\label{eq:covariant_derivative}
	\end{equation}
	wobei \(\Gamma_{\mu\nu}^{\rho}\) Christoffel-Symbole sind, die an die skalare Natur von \(\Tfield\) angepasst sind. Diese Kovarianz erhält die Konsistenz mit den phänomenologischen Vorhersagen der RT (z.B. Lichtablenkung), während ihr Ursprung als Massenvariation statt Raumzeitkrümmung neu interpretiert wird \cite{pascher_lagrange_2025}.
	
	\subsection{Physikalische Interpretation}
	\label{subsec:time_interpretation}
	
	\(\Tfield\) repräsentiert die intrinsische "Uhr" eines Teilchens, umgekehrt proportional zu seiner Energie:
	- **Schwere Teilchen:** Hohes \(m\), kurzes \(\Tfield\), schnelle Dynamik.
	- **Leichte Teilchen/Photonen:** Niedriges \(m\) oder \(\omega\), langes \(\Tfield\), langsamere Dynamik.
	
	Dieses Skalarfeld durchdringt die Raumzeit, variiert mit lokalen Massenergieverteilungen und dient als Vermittler, der die Zeitentwicklung der QM mit den Gravitationseffekten der RT vereint. Zum Beispiel wird die verlängerte Lebensdauer eines Myons im Flug (traditionell Zeitdilatation) zu einer Massenzunahme (\(m = \gamma m_0\)), wobei \(\Tfield\) entsprechend angepasst wird und die beobachtbare Äquivalenz bewahrt \cite{pascher_quantum_2025}.
	
	\section{Feldtheoretische Formulierung}
	\label{sec:field_theory}
	
	\subsection{Lagrange-Dichten}
	\label{subsec:lagrangian}
	
	Die Dynamik des T0-Modells ist in einer Gesamt-Lagrange-Dichte erfasst:
	\begin{equation}
		\calL_{\text{Total}} = \calL_{\text{Boson}} + \calL_{\text{Fermion}} + \calL_{\text{Higgs-T}} + \calL_{\text{intrinsic}},
		\label{eq:total_lagrangian}
	\end{equation}
	mit Komponenten:
	- **Eichbosonen:** \(\calL_{\text{Boson}} = -\frac{1}{4}\Tfield^2 F_{\mu\nu}F^{\mu\nu}\), die \(\Tfield\) mit elektromagnetischen Feldern koppeln.
	- **Fermionen:** \(\calL_{\text{Fermion}} = \bar{\psi}i\gamma^{\mu}\DTmu\psi - y\bar{\psi}\Phi\psi\), wobei \(\DTmu\psi = \Tfield D_{\mu}\psi + \psi\partial_{\mu}\Tfield\) die kovariante Ableitung modifiziert.
	- **Higgs-Feld:** \(\calL_{\text{Higgs-T}} = |\DhiggsT|^2 - \lambda(|\Phi|^2 - v^2)^2\), integriert \(\Tfield\) mit Higgs-Wechselwirkungen.
	- **Intrinsische Zeit:** \(\calL_{\text{intrinsic}} = \frac{1}{2}\partial_{\mu}\Tfield\partial^{\mu}\Tfield - \frac{1}{2}\Tfield^2\), definiert \(\Tfield\) als Skalarfeld.
	
	Diese Terme stellen die universelle Rolle von \(\Tfield\) sicher und erweitern SM-Wechselwirkungen \cite{pascher_lagrange_2025}.
	
	\subsection{Erweiterung der Quantenmechanik}
	\label{subsec:qm_extension}
	
	Die Standard-Schrödinger-Gleichung:
	\begin{equation}
		i\hbar \frac{\partial}{\partial t} \Psi = \hat{H} \Psi,
		\label{eq:standard_schrodinger}
	\end{equation}
	nimmt uniforme Zeit an. T0 modifiziert dies zu:
	\begin{equation}
		i\hbar \Tfield \frac{\partial}{\partial t} \Psi + i\hbar \Psi \frac{\partial \Tfield}{\partial t} = \hat{H} \Psi,
		\label{eq:modified_schrodinger}
	\end{equation}
	und führt eine masseabhängige Evolution ein. Die Dekohärenzrate wird:
	\begin{equation}
		\Gamma_{\text{dec}} = \Gamma_0 \cdot \frac{m c^2}{\hbar},
		\label{eq:decoherence}
	\end{equation}
	wobei schwerere Teilchen schneller dekohärieren. Für verschränkte Zustände:
	\begin{equation}
		|\Psi(t)\rangle = \frac{1}{\sqrt{2}}(|0(t/T_1)\rangle_{m_1} \otimes |1(t/T_2)\rangle_{m_2} + |1(t/T_1)\rangle_{m_1} \otimes |0(t/T_2)\rangle_{m_2}),
		\label{eq:entangled_state}
	\end{equation}
	wobei \(T_1 = \frac{\hbar}{m_1 c^2}\), \(T_2 = \frac{\hbar}{m_2 c^2}\), was Nichtlokalität über massenspezifische Zeitskalen auflöst \cite{pascher_photons_2025}.
	
	\subsection{Anpassung der Quantenfeldtheorie}
	\label{subsec:qft_extension}
	
	\(\Tfield\) wird als Skalarfeld mit der Gleichung quantisiert:
	\begin{equation}
		\partial_{\mu}\partial^{\mu}\Tfield + \Tfield + \frac{\rho}{\Tfield^2} = 0,
		\label{eq:field_eq}
	\end{equation}
	wobei \(\rho\) die Massenergiedichte ist. Dies passt die QFT an, um relativistische Massenvariation einzubeziehen, und überbrückt QM und RT auf Feldebene \cite{pascher_lagrange_2025}.
	
	\section{Emergente Gravitation}
	\label{sec:emergent_grav}
	
	\subsection{Ableitung aus \(\Tfield\)}
	\label{subsec:grav_derivation}
	
	Gravitation entsteht aus \(\Tfield\)-Gradienten. Unter statischen Bedingungen:
	\begin{equation}
		\nabla^2\Tfield \approx -\frac{\rho}{\Tfield^2},
		\label{eq:static_field}
	\end{equation}
	abgeleitet aus Gleichung \ref{eq:field_eq}. Das effektive Potenzial ist:
	\begin{equation}
		\Phi(\vecx) = -\ln\left(\frac{\Tfield}{\Tzero}\right),
		\label{eq:grav_potential_def}
	\end{equation}
	was die Kraft ergibt:
	\begin{equation}
		\vec{F} = -\nabla\Phi = -\frac{\nabla\Tfield}{\Tfield}.
		\label{eq:force_from_potential}
	\end{equation}
	Für eine Punktmasse \(M\):
	\begin{equation}
		\Tfield(r) = \Tzero\left(1 - \frac{M}{r}\right),
		\label{eq:time_field_point_mass}
	\end{equation}
	also:
	\begin{equation}
		\vec{F} = -\frac{M}{r^2} \hat{r},
		\label{eq:newton_law}
	\end{equation}
	reproduziert das Newtonsche Gesetz ohne Raumzeitkrümmung \cite{pascher_emergente_2025}.
	
	\subsection{Neuinterpretation der Relativitätstheorie}
	\label{subsec:rt_reinterpretation}
	
	Die Raumzeitkrümmung der RT wird durch die Dynamik von \(\Tfield\) ersetzt. Post-Newtonsche Tests (z.B. Lichtablenkung \(\delta\phi = \frac{4M}{b}\), Periheldrehung \(\delta\omega = \frac{6\pi M}{a(1-e^2)}\)) stimmen mit der ART mit Parametern \(\beta = \gamma = \zeta = 1\) überein, was beobachtungsmäßige Konsistenz sicherstellt \cite{Will2014}.
	
	\subsection{Komplementarität der Gravitationsbeschreibungen}
	\label{subsec:grav_complementarity}
	
	Es ist wichtig zu betonen, dass die beiden im T0-Modell vorgestellten Beschreibungen der Gravitation — durch die modifizierte Einstein-Hilbert-Wirkung und durch direkte Ableitung aus dem intrinsischen Zeitfeld $\Tfield$ — keine alternativen oder widersprüchlichen Ansätze darstellen, sondern komplementäre Perspektiven desselben physikalischen Prinzips \cite{pascher_emergente_2025}.
	
	Die Einstein-Hilbert-Formulierung:
	\begin{equation}
		S_{\text{EH}} = \frac{1}{16\pi} \int (R - 2\kappa) \sqrt{-g} \, d^4x
	\end{equation}
	liefert eine geometrische Beschreibung, die mit der Relativitätstheorie auf makroskopischer Ebene kompatibel ist.
	
	Die Zeitfeld-Ableitung:
	\begin{equation}
		\Phi(\vecx) = -\ln\left(\frac{\Tfield}{\Tzero}\right)
	\end{equation}
	zeigt den grundlegenderen Mechanismus, durch den Gravitation als Phänomen aus dem Zeitfeld entsteht.
	
	Beide Formulierungen führen zum gleichen modifizierten Gravitationspotenzial und sind im Limit schwacher Felder mathematisch äquivalent, was die Kohärenz des T0-Modells unterstreicht und zeigt, wie konventionelle Raumzeitgeometrie als Manifestation der Zeitfelddynamik verstanden werden kann.
	\section{Diskussion}
	\label{sec:discussion}
	
	\subsection{Theoretische Vorteile}
	- **QM-RT-Vereinheitlichung:** \(\Tfield\) überbrückt mikro- und makroskopische Physik.
	- **Einfachheit:** Energiebasierte Einheit reduziert Komplexität.
	- **Quantengravitation:** Emergente Gravitation stimmt mit QFT überein.
	
	\subsection{Herausforderungen und Lösungen}
	\label{subsec:challenges}
	
	Während das T0-Modell signifikante theoretische Stärken demonstriert, erforderte seine vollständige Realisierung die Bewältigung wichtiger Herausforderungen, insbesondere die Quantisierung des intrinsischen Zeitfeldes \(\Tfield\). Jüngste Fortschritte, detailliert in einer umfassenden quantenfeldtheoretischen (QFT) Behandlung \cite{pascher_qft_2025}, lösen diese Herausforderungen und verbessern die Kohärenz des Modells.
	
	\begin{enumerate}
		\item \textbf{Quantisierung des intrinsischen Zeitfeldes:} Die klassische feldtheoretische Formulierung von \(\Tfield\), etabliert über die Lagrange-Dichte \(\calL_{\text{intrinsic}} = \frac{1}{2}\partial_{\mu}\Tfield\partial^{\mu}\Tfield - \frac{1}{2}\Tfield^2\) \cite{pascher_lagrange_2025}, wurde nun zu einem vollständigen QFT-Rahmen erweitert. Dies umfasst kanonische Quantisierung, Pfadintegralformulierung, Renormierung und Unitaritätsanalyse, was die Integration mit der Quantenmechanik und Konsistenz bei hohen Energien sicherstellt. Vorläufige Hinweise, dass \(\Tfield\) quantisiert werden könnte, mit \(\betaT\) als Renormierungsgruppen-Fixpunkt im Infrarotlimit (\(\lim_{E \to 0} \betaT(E) = 1\)) \cite{pascher_alphabeta_2025}, wurden bestätigt, was eine kritische Lücke schließt und T0 mit Standard-QFT-Prinzipien in Einklang bringt.
		
		\item \textbf{Beobachtungsvalidierung von \(\betaT = 1\):} Im einheitlichen natürlichen Einheitensystem definiert \(\betaT = 1\) die charakteristische Längenskala \(r_0 = \xi \cdot l_P\), mit \(\xi \approx 1.33 \times 10^{-4}\), abgeleitet von Higgs-Parametern \cite{pascher_params_2025, pascher_alphabeta_2025}, im Kontrast zum empirisch geschätzten \(\betaT^{\text{SI}} \approx 0.008\) aus kosmologischen Beobachtungen wie wellenlängenabhängiger Rotverschiebung \cite{pascher_messdifferenzen_2025}. Die QFT-Behandlung unterstützt \(\betaT = 1\) mathematisch innerhalb des natürlichen Einheitenrahmens, erfordert keine Feinabstimmung, da es natürlich aus der Struktur des Modells hervorgeht. Die Validierung gegen hochpräzise Daten (z.B. CMB-Temperaturskalierung, Galaxiendynamik) wird durch experimentelle Tests adressiert, die in Abschnitt 4 "Quantitative Vorhersagen" skizziert werden, wobei Quantenkorrekturen zur Verbesserung der Vorhersagepräzision genutzt werden.
	\end{enumerate}
	
	Diese Fortschritte adressieren frühere Herausforderungen und transformieren sie in Stärken. Das quantisierte \(\Tfield\) löst Probleme wie das Hierarchieproblem und die Vakuumenergiedichte, indem es Skalen natürlich verbindet und kosmologische Phänomene ohne dunkle Komponenten neu interpretiert (Abschnitt 3), was T0 zu einem überzeugenden, testbaren Rahmen macht.
	
	\section{Schlussfolgerung}
	\label{sec:conclusion}
	
	Teil II erweitert T0 zu einer statischen, testbaren Kosmologie, interpretiert Rotverschiebung und Gravitationseffekte neu, mit einer robusten QFT-Grundlage, die seine Durchführbarkeit verbessert \cite{pascher_perspective_2025}.
	
	\begin{acknowledgments}
		Dank an Reinsprecht Martin Dipl.-Ing. Dr. für kritisches Feedback.
	\end{acknowledgments}
	
	\bibliographystyle{apsrev4-2}
	\begin{thebibliography}{99}
		\bibitem{pascher_emergente_2025} J. Pascher, \href{https://github.com/jpascher/T0-Time-Mass-Duality/tree/main/2/pdf/Deutsch/EmergentGravT0.pdf}{Emergente Gravitation im T0-Modell: Eine umfassende Ableitung}, 1. April 2025.
		\bibitem{pascher_part1_2025} J. Pascher, \href{https://github.com/jpascher/T0-Time-Mass-Duality/tree/main/2/pdf/Deutsch/QMRelTimeMassPart1.pdf}{Überbrückung von Quantenmechanik und Relativitätstheorie durch Zeit-Masse-Dualität: Ein einheitlicher Rahmen mit natürlichen Einheiten \(\alpha = \beta = 1\) Teil I: Theoretische Grundlagen}, 7. April 2025.
		\bibitem{pascher_lagrange_2025} J. Pascher, \href{https://github.com/jpascher/T0-Time-Mass-Duality/tree/main/2/pdf/Deutsch/MathZeitMasseLagrange.pdf}{Von Zeitdilatation zu Massenvariation: Mathematische Kernformulierungen der Zeit-Masse-Dualitätstheorie}, 29. März 2025.
		\bibitem{pascher_quantum_2025} J. Pascher, \href{https://github.com/jpascher/T0-Time-Mass-Duality/tree/main/2/pdf/Deutsch/NotwendigkeitQMErweiterung.pdf}{Die Notwendigkeit der Erweiterung der Standard-Quantenmechanik und Quantenfeldtheorie}, 27. März 2025.
		\bibitem{pascher_photons_2025} J. Pascher, \href{https://github.com/jpascher/T0-Time-Mass-Duality/tree/main/2/pdf/Deutsch/DynMassePhotonenNichtlokal.pdf}{Dynamische Masse von Photonen und ihre Implikationen für Nichtlokalität im T0-Modell}, 25. März 2025.
		\bibitem{pascher_alphabeta_2025} J. Pascher, \href{https://github.com/jpascher/T0-Time-Mass-Duality/tree/main/2/pdf/Deutsch/Alpha1Beta1Konsistenz.pdf}{Einheitliches Einheitensystem im T0-Modell: Die Konsistenz von \(\alpha = 1\) und \(\beta = 1\)}, 5. April 2025.
		\bibitem{pascher_emergente_2025} J. Pascher, \href{https://github.com/jpascher/T0-Time-Mass-Duality/tree/main/2/pdf/Deutsch/EmergentGravT0.pdf}{Emergente Gravitation im T0-Modell: Eine umfassende Ableitung}, 1. April 2025.
		\bibitem{pascher_perspective_2025} J. Pascher, \href{https://github.com/jpascher/T0-Time-Mass-Duality/tree/main/2/pdf/Deutsch/ZeitRaumPascher.pdf}{Eine neue Perspektive auf Zeit und Raum: Johann Paschers revolutionäre Ideen}, 25. März 2025.
		\bibitem{schrodinger1926} E. Schrödinger, Phys. Rev. \textbf{28}, 1049 (1926).
		\bibitem{einstein1905} A. Einstein, Ann. Phys. \textbf{322}, 891 (1905).
		\bibitem{einstein1915} A. Einstein, Sitzungsber. Preuss. Akad. Wiss. \textbf{1915}, 844 (1915).
		\bibitem{bell1964} J. S. Bell, Physics \textbf{1}, 195 (1964).
		\bibitem{Planck2020} Planck Collaboration, Astron. Astrophys. \textbf{641}, A6 (2020).
		\bibitem{Riess1998} A. G. Riess et al., Astron. J. \textbf{116}, 1009 (1998).
		\bibitem{Planck1899} M. Planck, Proc. Prussian Acad. Sci. \textbf{5}, 440 (1899).
		\bibitem{Duff2002} M. J. Duff et al., J. High Energy Phys. \textbf{2002}, 023 (2002).
		\bibitem{Greene2020} B. Greene, \textit{Until the End of Time}, Knopf, New York (2020).
		\bibitem{tHooft1993} G. 't Hooft, arXiv:gr-qc/9310026 (1993).
		\bibitem{Will2014} C. M. Will, Living Rev. Relativ. \textbf{17}, 4 (2014).
		\bibitem{pascher_params_2025} J. Pascher, \href{https://github.com/jpascher/T0-Time-Mass-Duality/tree/main/2/pdf/Deutsch/ZeitMasseT0Params.pdf}{Zeit-Masse-Dualitätstheorie (T0-Modell): Ableitung der Parameter \(\kappa\), \(\alpha\) und \(\beta\)}, 4. April 2025.
		\bibitem{pascher_messdifferenzen_2025} J. Pascher, \href{https://github.com/jpascher/T0-Time-Mass-Duality/tree/main/2/pdf/Deutsch/MessdifferenzenT0Standard.pdf}{Kompensatorische und additive Effekte: Eine Analyse der Messdifferenzen zwischen dem