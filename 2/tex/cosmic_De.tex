\documentclass[12pt,a4paper]{article}
\usepackage[utf8]{inputenc}
\usepackage[T1]{fontenc}
\usepackage[german]{babel}
\usepackage[left=2cm,right=2cm,top=2cm,bottom=2cm]{geometry}
\usepackage{lmodern}
\usepackage{amsmath}
\usepackage{amssymb}
\usepackage{physics}
\usepackage{hyperref}
\usepackage{tcolorbox}
\usepackage{booktabs}
\usepackage{enumitem}
\usepackage[table,xcdraw]{xcolor}
\usepackage{longtable}
\usepackage{siunitx}
\usepackage{fancyhdr}
\usepackage{textgreek}

% Header and Footer
\pagestyle{fancy}
\fancyhf{}
\fancyhead[L]{T0-Theorie: Kosmische Beziehungen}
\fancyhead[R]{\thepage}
\fancyfoot[C]{\textit{Von der universellen ξ-Konstante zu kosmischen Strukturen}}
\renewcommand{\headrulewidth}{0.4pt}
\renewcommand{\footrulewidth}{0.4pt}

\hypersetup{
	colorlinks=true,
	linkcolor=blue,
	citecolor=blue,
	urlcolor=blue,
	pdftitle={T0-Theorie: Kosmische Beziehungen und universelle $\xi$-Konstante},
	pdfauthor={T0-Theorie Projekt},
	pdfsubject={Kosmologie, $\xi$-Feld, Gravitation, CMB, Casimir-Effekt}
}

% Custom environments
\newtcolorbox{important}[1][]{colback=yellow!10!white,colframe=yellow!50!black,fonttitle=\bfseries,title=Wichtiger Hinweis,#1}
\newtcolorbox{formula}[1][]{colback=blue!5!white,colframe=blue!75!black,fonttitle=\bfseries,title=Schlüsselformel,#1}
\newtcolorbox{revolutionary}[1][]{colback=red!5!white,colframe=red!75!black,fonttitle=\bfseries,title=Revolutionäre Erkenntnis,#1}
\newtcolorbox{experiment}[1][]{colback=green!5!white,colframe=green!75!black,fonttitle=\bfseries,title=Experimenteller Test,#1}
\newtcolorbox{sibox}[1][]{colback=orange!10!white,colframe=orange!75!black,fonttitle=\bfseries,title=SI-Einheiten (nur zur Referenz),#1}

\title{\Huge\textbf{T0-Theorie: Kosmische Beziehungen}\\
	\Large Die universelle $\xi$-Konstante als Schlüssel \\
	zu Gravitation, CMB und kosmischen Strukturen}
	\author{Johann Pascher\\
	Abteilung für Kommunikationstechnik, \\Höhere Technische Bundeslehranstalt (HTL), Leonding, Österreich\\
	\texttt{johann.pascher@gmail.com}}

\date{\today}

\begin{document}
	
	\maketitle
	
	\begin{abstract}
		Die T0-Theorie demonstriert, wie eine einzige universelle Konstante $\xi = \frac{4}{3} \times 10^{-4}$ s\"amtliche kosmische Ph\"anomene bestimmt. Dieses Dokument pr\"asentiert die fundamentalen Beziehungen zwischen der Gravitationskonstante, der kosmischen Mikrowellenhintergrundstrahlung (CMB), dem Casimir-Effekt und kosmischen Strukturen im Rahmen eines statischen, ewig existierenden Universums. Alle Herleitungen erfolgen in nat\"urlichen Einheiten ($\hbar = c = k_B = 1$) und respektieren die Zeit-Energie-Dualit\"at als fundamentales Prinzip der Quantenmechanik.
	\end{abstract}
	
	\tableofcontents
	\newpage
	
	\section{Einf\"uhrung: Die universelle $\xi$-Konstante}
	
\subsection{Grundlagen der T0-Theorie}

\begin{important}
	Die T0-Theorie basiert auf der universellen dimensionslosen Konstante $\xi = \frac{4}{3} \times 10^{-4}$, die alle physikalischen Phänomene vom subatomaren bis zum kosmischen Bereich bestimmt.
\end{important}

Die T0-Theorie revolutioniert unser Verständnis des Universums durch die Einführung einer einzigen fundamentalen Konstante. Diese Konstante bildet die Grundlage für alle physikalischen Berechnungen und Vorhersagen der Theorie:

\begin{equation}
	\xi = \frac{4}{3} \times 10^{-4} = 1.333333... \times 10^{-4}
\end{equation}

Diese dimensionslose Konstante verbindet Quanten- und Gravitationsphänomene und ermöglicht eine einheitliche Beschreibung aller fundamentalen Wechselwirkungen.

\begin{tcolorbox}[colback=yellow!10!white,colframe=yellow!50!black,title=Hinweis zur Herleitung]
	Für die detaillierte Herleitung und physikalische Begründung dieser fundamentalen Konstante siehe das Dokument "Parameterherleitung" (verfügbar unter: \url{https://github.com/jpascher/T0-Time-Mass-Duality/2/pdf/parameterherleitung_De.pdf}).
\end{tcolorbox}

	\subsection{Zeit-Energie-Dualität als Fundament}
	
	\begin{revolutionary}
		Heisenbergs Unschärferelation $\Delta E \times \Delta t \geq \hbar/2 = 1/2$ (natürliche Einheiten) beweist unwiderlegbar, dass ein Urknall physikalisch unmöglich ist.
	\end{revolutionary}
	
	Die Heisenbergsche Unschärferelation zwischen Energie und Zeit stellt das fundamentale Prinzip der T0-Theorie dar:
	
	\begin{equation}
		\Delta E \times \Delta t \geq \frac{1}{2} \quad \text{(natürliche Einheiten)}
	\end{equation}
	
	Diese Relation hat weitreichende kosmologische Konsequenzen:
	\begin{itemize}
		\item Ein zeitlicher Anfang (Urknall) würde $\Delta t$ = endlich bedeuten
		\item Dies führt zu $\Delta E \to \infty$ - physikalisch inkonsistent
		\item Daher muss das Universum ewig existiert haben: $\Delta t = \infty$
		\item Das Universum ist statisch, ohne expandierenden Raum
	\end{itemize}
	

	\section{Kosmische Mikrowellenhintergrundstrahlung (CMB)}
	
	\subsection{CMB ohne Urknall: $\xi$-Feld-Mechanismen}
	
	\begin{revolutionary}
		Da die Zeit-Energie-Dualität einen Urknall verbietet, muss die CMB einen anderen Ursprung haben als die z=1100-Entkopplung der Standardkosmologie.
	\end{revolutionary}
	
	Die T0-Theorie erklärt die CMB durch $\xi$-Feld-Quantenfluktuationen:
	
	\begin{equation}
		\frac{T_{\text{CMB}}}{E_\xi} = \frac{16}{9} \xi^2
	\end{equation}
	
	Mit $E_\xi = \frac{1}{\xi} = \frac{3}{4} \times 10^4$ (natürliche Einheiten) und $\xi = \frac{4}{3} \times 10^{-4}$ ergibt sich:
	
	\begin{equation}
		T_{\text{CMB}} = \frac{16}{9} \xi^2 \times E_\xi = \frac{16}{9} \times 1{,}78 \times 10^{-8} \times 7500 = 2{,}35 \times 10^{-4}
	\end{equation}
	
	\textbf{Umrechnung in SI-Einheiten:}
	\begin{equation}
		T_{\text{CMB}} = 2{,}725 \text{ K}
	\end{equation}
	
	Dies stimmt perfekt mit den Beobachtungen überein!
	
	\subsection{CMB-Energiedichte und $\xi$-Längenskala}
	
	Die CMB-Energiedichte in natürlichen Einheiten beträgt:
	\begin{equation}
		\rho_{\text{CMB}} = 4{,}87 \times 10^{41} \quad \text{(natürliche Einheiten, Dimension } [E^4] \text{)}
	\end{equation}
	
	Diese Energiedichte definiert eine charakteristische $\xi$-Längenskala:
	\begin{equation}
		L_\xi = \left(\frac{\xi}{\rho_{\text{CMB}}}\right)^{1/4}
	\end{equation}
	
	\begin{formula}
		Fundamentale Beziehung der CMB-Energiedichte:
		\begin{equation}
			\rho_{\text{CMB}} = \frac{\xi}{L_\xi^4} = \frac{\frac{4}{3} \times 10^{-4}}{(L_\xi)^4}
		\end{equation}
	\end{formula}
	
	\section{Casimir-Effekt und $\xi$-Feld-Verbindung}
	
	\subsection{Casimir-CMB-Verhältnis als experimentelle Bestätigung}
	
	\begin{experiment}
		Das Verhältnis zwischen Casimir-Energiedichte und CMB-Energiedichte bestätigt die charakteristische $\xi$-Längenskala von $L_\xi = 10^{-4}$ m.
	\end{experiment}
	
	Die Casimir-Energiedichte bei Plattenabstand $d = L_\xi$ beträgt:
	\begin{equation}
		|\rho_{\text{Casimir}}| = \frac{\pi^2}{240 \times L_\xi^4} \quad \text{(natürliche Einheiten)}
	\end{equation}
	
	Das experimentelle Verhältnis ergibt:
	\begin{equation}
		\frac{|\rho_{\text{Casimir}}|}{\rho_{\text{CMB}}} = \frac{\pi^2}{240 \xi} = \frac{\pi^2 \times 10^4}{320} \approx 308
	\end{equation}
	
	\textbf{Experimentelle Bestätigung:}
	Mit $L_\xi = 10^{-4}$ m ergibt die direkte Berechnung:
	\begin{align}
		|\rho_{\text{Casimir}}| &= \frac{\hbar c \pi^2}{240 \times (10^{-4})^4} = 1{,}3 \times 10^{-11} \text{ J/m}^3 \\
		\rho_{\text{CMB}} &= 4{,}17 \times 10^{-14} \text{ J/m}^3 \\
		\text{Verhältnis} &= \frac{1{,}3 \times 10^{-11}}{4{,}17 \times 10^{-14}} = 312
	\end{align}
	
	Die Übereinstimmung zwischen theoretischer Vorhersage (308) und experimentellem Wert (312) beträgt 1{,}3\% - eine hervorragende Bestätigung!
	
	\subsection{$\xi$-Feld als universelles Vakuum}
	
	\begin{important}
		Das $\xi$-Feld manifestiert sich sowohl in der freien CMB-Strahlung als auch im geometrisch beschränkten Casimir-Vakuum. Dies beweist die fundamentale Realität des $\xi$-Feldes.
	\end{important}
	
	Die charakteristische $\xi$-Längenskala $L_\xi$ ist der Punkt, wo CMB-Vakuum-Energiedichte und Casimir-Energiedichte vergleichbare Größenordnungen erreichen:
	
	\begin{align}
		\text{Freies Vakuum:} \quad &\rho_{\text{CMB}} = +4{,}87 \times 10^{41} \\
		\text{Beschränktes Vakuum:} \quad &|\rho_{\text{Casimir}}| = \frac{\pi^2}{240 d^4}
	\end{align}
	
	\section{Kosmische Rotverschiebung ohne Expansion}
	
	\subsection{$\xi$-Feld-Energieverlust-Mechanismus}
	
	\begin{revolutionary}
		Die beobachtete kosmische Rotverschiebung entsteht nicht durch räumliche Expansion, sondern durch Energieverlust der Photonen im omnipräsenten $\xi$-Feld.
	\end{revolutionary}
	
	Photonen verlieren Energie durch Wechselwirkung mit dem $\xi$-Feld:
	\begin{equation}
		\frac{dE}{dx} = -\xi \cdot f\left(\frac{E}{E_\xi}\right) \cdot E
	\end{equation}
	
	Für den linearen Fall $f\left(\frac{E}{E_\xi}\right) = \frac{E}{E_\xi}$ ergibt sich:
	\begin{equation}
		\frac{dE}{dx} = -\frac{\xi E^2}{E_\xi}
	\end{equation}
	
	\subsection{Wellenlängenabhängige Rotverschiebung}
	
	Die Integration der Energieverlustgleichung führt zur wellenlängenabhängigen Rotverschiebung:
	
	\begin{formula}
		Wellenlängenabhängige Rotverschiebung:
		\begin{equation}
			z(\lambda_0) = \frac{\xi x}{E_\xi} \cdot \lambda_0
		\end{equation}
		wobei $\lambda_0$ die emittierte Wellenlänge und $x$ die zurückgelegte Strecke ist.
	\end{formula}
	
	Diese Formel sagt vorher:
	\begin{itemize}
		\item Kurzwelligeres Licht (UV) zeigt größere Rotverschiebung
		\item Langwelliges Licht (Radio) zeigt kleinere Rotverschiebung
		\item Das Verhältnis ist $z_1/z_2 = \lambda_1/\lambda_2$
	\end{itemize}
	
	\begin{experiment}
		Experimenteller Test: Vergleich von Radio- und optischen Rotverschiebungen
		\begin{itemize}
			\item 21cm-Wasserstofflinie: $\nu = 1420$ MHz
			\item Optische H$\alpha$-Linie: $\nu = 457$ THz
			\item Vorhergesagtes Verhältnis: $z_{21\text{cm}}/z_{\text{H}\alpha} = 3{,}1 \times 10^{-6}$
		\end{itemize}
	\end{experiment}
	
	\section{Strukturbildung im statischen $\xi$-Universum}
	
	\subsection{Kontinuierliche Strukturentwicklung}
	
	Im statischen T0-Universum erfolgt Strukturbildung kontinuierlich ohne Urknall-Beschränkungen:
	
	\begin{equation}
		\frac{d\rho}{dt} = -\nabla \cdot (\rho \mathbf{v}) + S_\xi(\rho, T, \xi)
	\end{equation}
	
	wobei $S_\xi$ der $\xi$-Feld-Quellterm für kontinuierliche Materie/Energie-Transformation ist.
	
	\subsection{$\xi$-unterstützte kontinuierliche Schöpfung}
	
	Das $\xi$-Feld ermöglicht kontinuierliche Materie/Energie-Transformation:
	
	\begin{align}
		\text{Quantenvakuum} &\xrightarrow{\xi} \text{Virtuelle Teilchen} \\
		\text{Virtuelle Teilchen} &\xrightarrow{\xi^2} \text{Reale Teilchen} \\
		\text{Reale Teilchen} &\xrightarrow{\xi^3} \text{Atomkerne} \\
		\text{Atomkerne} &\xrightarrow{\text{Zeit}} \text{Sterne, Galaxien}
	\end{align}
	
	Die Energiebilanz wird aufrechterhalten durch:
	\begin{equation}
		\rho_{\text{gesamt}} = \rho_{\text{Materie}} + \rho_{\xi\text{-Feld}} = \text{konstant}
	\end{equation}
	
	\section{Dimensionslose $\xi$-Hierarchie}
	
	\subsection{Energieskalenverhältnisse}
	
	Alle $\xi$-Beziehungen reduzieren sich auf exakte mathematische Verhältnisse:
	
	\begin{longtable}{lcc}
		\caption{Dimensionslose $\xi$-Verhältnisse} \\
		\toprule
		\textbf{Verhältnis} & \textbf{Ausdruck} & \textbf{Wert} \\
		\midrule
		\endfirsthead
		\multicolumn{3}{c}{\tablename\ \thetable{} -- Fortsetzung} \\
		\toprule
		\textbf{Verhältnis} & \textbf{Ausdruck} & \textbf{Wert} \\
		\midrule
		\endhead
		Temperatur & $\frac{T_{\text{CMB}}}{E_\xi}$ & $3{,}13 \times 10^{-8}$ \\
		Theorie & $\frac{16}{9}\xi^2$ & $3{,}16 \times 10^{-8}$ \\
		Länge & $\frac{\ell_{\xi}}{L_\xi}$ & $\xi^{-1/4}$ \\
		Casimir-CMB & $\frac{|\rho_{\text{Casimir}}|}{\rho_{\text{CMB}}}$ & $\frac{\pi^2 \times 10^4}{320}$ \\
		\bottomrule
	\end{longtable}
	
	\begin{important}
		Alle $\xi$-Beziehungen bestehen aus exakten mathematischen Verhältnissen:
		\begin{itemize}
			\item Brüche: $\frac{4}{3}$, $\frac{3}{4}$, $\frac{16}{9}$
			\item Zehnerpotenzen: $10^{-4}$, $10^3$, $10^4$
			\item Mathematische Konstanten: $\pi^2$
		\end{itemize}
		KEINE willkürlichen Dezimalzahlen! Alles folgt aus der $\xi$-Geometrie.
	\end{important}
	
	\section{Experimentelle Vorhersagen und Tests}
	
	\subsection{Präzisionsmessungen der Gravitationskonstante}
	
	Die T0-Theorie sagt vorher:
	\begin{equation}
		G_{\text{T0}} = 6{,}67430000... \times 10^{-11} \text{ m}^3/(\text{kg} \cdot \text{s}^2)
	\end{equation}
	
	Diese theoretisch exakte Vorhersage kann durch zukünftige Präzisionsmessungen getestet werden.
	
	\subsection{Casimir-Kraft-Anomalien}
	
	\begin{experiment}
		Vorhersage: Casimir-Kraft-Anomalien bei charakteristischer $\xi$-Längenskala
		\begin{itemize}
			\item Standard-Casimir-Gesetz: $F \propto d^{-4}$
			\item $\xi$-Feld-Modifikationen bei $d = L_\xi = 10^{-4}$ m
			\item Messbare Abweichungen durch $\xi$-Vakuum-Kopplung
		\end{itemize}
	\end{experiment}
	
	\subsection{Elektromagnetische Resonanz}
	
	Maximale $\xi$-Feld-Photon-Kopplung bei charakteristischer Frequenz:
	\begin{equation}
		\nu_\xi = \frac{1}{L_\xi} = 10^{4} \text{ Hz} = 10 \text{ kHz}
	\end{equation}
	
	Bei dieser Frequenz sollten elektromagnetische Anomalien auftreten.
	
	\section{Kosmologische Konsequenzen}
	
	\subsection{Lösung der kosmologischen Probleme}
	
	Das T0-Modell löst alle Feinabstimmungsprobleme der Standardkosmologie:
	
	\begin{longtable}{lcc}
		\caption{Kosmologische Probleme: Standard vs. T0} \\
		\toprule
		\textbf{Problem} & \textbf{$\Lambda$CDM} & \textbf{T0-Lösung} \\
		\midrule
		\endfirsthead
		\multicolumn{3}{c}{\tablename\ \thetable{} -- Fortsetzung} \\
		\toprule
		\textbf{Problem} & \textbf{$\Lambda$CDM} & \textbf{T0-Lösung} \\
		\midrule
		\endhead
		Horizontproblem & Inflation erforderlich & Unendliche kausale Konnektivität \\
		Flachheitsproblem & Feinabstimmung & Geometrie stabilisiert über unendliche Zeit \\
		Monopolproblem & Topologische Defekte & Defekte dissipieren über unendliche Zeit \\
		Lithiumproblem & Nukleosynthese-Diskrepanz & Nukleosynthese über unbegrenzte Zeit \\
		Altersproblem & Objekte älter als Universum & Objekte können beliebig alt sein \\
		$H_0$-Spannung & 9\% Diskrepanz & Kein $H_0$ im statischen Universum \\
		Dunkle Energie & 69\% der Energiedichte & Nicht erforderlich \\
		\bottomrule
	\end{longtable}
	
	\subsection{Parameterreduktion}
	
	\begin{revolutionary}
		Revolutionäre Parameterreduktion: Von 25+ Parametern zu einem einzigen!
		\begin{itemize}
			\item Standardmodell der Teilchenphysik: 19+ Parameter
			\item $\Lambda$CDM-Kosmologie: 6 Parameter
			\item T0-Theorie: 1 Parameter ($\xi$)
		\end{itemize}
		Reduktion um 96\%!
	\end{revolutionary}
	
	\section{Schlussfolgerungen}
	

	\subsection{Das Vakuum ist das $\xi$-Feld}
	
	\begin{important}
		Fundamentale Erkenntnis der T0-Theorie:
		\begin{itemize}
			\item Das Vakuum ist identisch mit dem $\xi$-Feld
			\item Die CMB ist die Strahlung dieses Vakuums bei charakteristischer Temperatur
			\item Die Casimir-Kraft entsteht durch geometrische Beschränkung desselben Vakuums
			\item Gravitation folgt aus der $\xi$-Geometrie
			\item Kosmische Rotverschiebung entsteht durch $\xi$-Energieverlust
		\end{itemize}
	\end{important}
	
	\subsection{Mathematische Eleganz}
	
	Die T0-Theorie etabliert:
	\begin{enumerate}
		\item \textbf{Universelle $\xi$-Skalierung}: Alle Phänomene folgen aus $\xi = \frac{4}{3} \times 10^{-4}$
		\item \textbf{Statisches Paradigma}: Kein Urknall, keine Expansion, ewige Existenz
		\item \textbf{Zeit-Energie-Konsistenz}: Respektiert fundamentale Quantenmechanik
		\item \textbf{Dimensionale Konsistenz}: Vollständig in natürlichen Einheiten formuliert
		\item \textbf{Einheitenunabhängige Physik}: Exakte mathematische Verhältnisse
	\end{enumerate}
	
	\begin{revolutionary}
		Die T0-Theorie bietet eine mathematisch konsistente, in natürlichen Einheiten formulierte Alternative zur expansionsbasierten Kosmologie und erklärt alle kosmischen Phänomene mit einer einzigen fundamentalen Konstante in einem statischen, ewig existierenden Universum.
	\end{revolutionary}
	
	Die Übereinstimmungen zwischen theoretischen Vorhersagen und experimentellen Beobachtungen - von der exakten Gravitationskonstante über die CMB-Temperatur bis zum Casimir-CMB-Verhältnis - demonstrieren die innere Konsistenz und prädiktive Kraft der T0-Theorie.
	
	\section{Literaturverzeichnis}
	
	\begin{thebibliography}{20}
		
		\bibitem{t0_lagrangian_de}
		Pascher, Johann (2025). 
		\textit{Vereinfachte Lagrange-Dichte und Zeit-Massen-Dualit\"at in der T0-Theorie}. 
		T0-Theorie Projekt. 
		\url{https://jpascher.github.io/T0-Time-Mass-Duality/2/pdf/lagrandian-einfachDe.pdf}
		
		\bibitem{t0_lagrangian_en}
		Pascher, Johann (2025). 
		\textit{Simplified Lagrangian Density and Time-Mass Duality in T0-Theory}. 
		T0-Theory Project. 
		\url{https://jpascher.github.io/T0-Time-Mass-Duality/2/pdf/lagrandian-einfachEn.pdf}
		
		\bibitem{t0_cosmos_de}
		Pascher, Johann (2025). 
		\textit{T0-Modell: Ein vereinheitlichtes, statisches, zyklisches, dunkle-Materie-freies und dunkle-Energie-freies Universum}. 
		T0-Theorie Projekt. 
		\url{https://jpascher.github.io/T0-Time-Mass-Duality/2/pdf/cos_De.pdf}
		
		\bibitem{t0_cosmos_en}
		Pascher, Johann (2025). 
		\textit{T0-Model: A unified, static, cyclic, dark-matter-free and dark-energy-free universe}. 
		T0-Theory Project. 
		\url{https://jpascher.github.io/T0-Time-Mass-Duality/2/pdf/cos_En.pdf}
		
		\bibitem{t0_cmb_de}
		Pascher, Johann (2025). 
		\textit{Temperatureinheiten in nat\"urlichen Einheiten: T0-Theorie und statisches Universum}. 
		T0-Theorie Projekt. 
		\url{https://jpascher.github.io/T0-Time-Mass-Duality/2/pdf/TempEinheitenCMBDe.pdf}
		
		\bibitem{t0_cmb_en}
		Pascher, Johann (2025). 
		\textit{Temperature Units in Natural Units: T0-Theory and Static Universe}. 
		T0-Theory Project. 
		\url{https://jpascher.github.io/T0-Time-Mass-Duality/2/pdf/TempEinheitenCMBEn.pdf}
		
		\bibitem{t0_gravitation_en}
		Pascher, Johann (2025). 
		\textit{Geometric Determination of the Gravitational Constant: From the T0-Model}. 
		T0-Theory Project. 
		\url{https://jpascher.github.io/T0-Time-Mass-Duality/2/pdf/gravitationskonstnte_En.pdf}
		
		\bibitem{t0_redshift_de}
		Pascher, Johann (2025). 
		\textit{T0-Theorie: Wellenlängenabhängige Rotverschiebung ohne Distanzannahmen}. 
		T0-Theorie Projekt. 
		\url{https://jpascher.github.io/T0-Time-Mass-Duality/2/pdf/redshift_deflection_De.pdf}
		
		\bibitem{t0_redshift_en}
		Pascher, Johann (2025). 
		\textit{T0-Theory: Wavelength-Dependent Redshift without Distance Assumptions}. 
		T0-Theory Project. 
		\url{https://jpascher.github.io/T0-Time-Mass-Duality/2/pdf/redshift_deflection_En.pdf}
		
		\bibitem{heisenberg1927}
		Heisenberg, W. (1927). 
		\textit{\"Uber den anschaulichen Inhalt der quantentheoretischen Kinematik und Mechanik}. 
		Zeitschrift f\"ur Physik, 43(3-4), 172--198.
		
		\bibitem{planck2020}
		Planck Collaboration (2020). 
		\textit{Planck 2018 results. VI. Cosmological parameters}. 
		Astronomy \& Astrophysics, 641, A6. 
		\url{https://doi.org/10.1051/0004-6361/201833910}
		
		\bibitem{codata2018}
		CODATA (2018). 
		\textit{The 2018 CODATA Recommended Values of the Fundamental Physical Constants}. 
		National Institute of Standards and Technology. 
		\url{https://physics.nist.gov/cuu/Constants/}
		
		\bibitem{casimir1948}
		Casimir, H. B. G. (1948). 
		\textit{On the attraction between two perfectly conducting plates}. 
		Proceedings of the Royal Netherlands Academy of Arts and Sciences, 51(7), 793--795.
		
		\bibitem{muon_g2_2021}
		Muon g-2 Collaboration (2021). 
		\textit{Measurement of the Positive Muon Anomalous Magnetic Moment to 0.46 ppm}. 
		Physical Review Letters, 126(14), 141801. 
		\url{https://doi.org/10.1103/PhysRevLett.126.141801}
		
		\bibitem{riess2022}
		Riess, A. G., et al. (2022). 
		\textit{A Comprehensive Measurement of the Local Value of the Hubble Constant with 1 km s$^{-1}$ Mpc$^{-1}$ Uncertainty from the Hubble Space Telescope and the SH0ES Team}. 
		The Astrophysical Journal Letters, 934(1), L7. 
		\url{https://doi.org/10.3847/2041-8213/ac5c5b}
		
		\bibitem{jwst_early}
		Naidu, R. P., et al. (2022). 
		\textit{Two Remarkably Luminous Galaxy Candidates at z $\approx$ 11--13 Revealed by JWST}. 
		The Astrophysical Journal Letters, 940(1), L14. 
		\url{https://doi.org/10.3847/2041-8213/ac9b22}
		
		\bibitem{cobe1992}
		COBE Collaboration (1992). 
		\textit{Structure in the COBE differential microwave radiometer first-year maps}. 
		The Astrophysical Journal Letters, 396, L1--L5. 
		\url{https://doi.org/10.1086/186504}
		
		\bibitem{sparnaay1958}
		Sparnaay, M. J. (1958). 
		\textit{Measurements of attractive forces between flat plates}. 
		Physica, 24(6-10), 751--764. 
		\url{https://doi.org/10.1016/S0031-8914(58)80090-7}
		
		\bibitem{lamoreaux1997}
		Lamoreaux, S. K. (1997). 
		\textit{Demonstration of the Casimir force in the 0.6 to 6 $\mu$m range}. 
		Physical Review Letters, 78(1), 5--8. 
		\url{https://doi.org/10.1103/PhysRevLett.78.5}
		
		\bibitem{einstein1915}
		Einstein, A. (1915). 
		\textit{Die Feldgleichungen der Gravitation}. 
		Sitzungsberichte der Preußischen Akademie der Wissenschaften, 844--847.
		
	\end{thebibliography}
	
\end{document}