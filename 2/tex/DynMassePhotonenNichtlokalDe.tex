\documentclass[12pt,a4paper]{article}
\usepackage[utf8]{inputenc}
\usepackage[T1]{fontenc}
\usepackage[ngerman]{babel}
\usepackage{lmodern}
\usepackage{csquotes}
\usepackage{amsmath}
\usepackage{amssymb}
\usepackage{physics}
\usepackage{geometry}
\usepackage{tocloft}
\usepackage{xcolor}
\usepackage{graphicx,tikz,pgfplots}
\pgfplotsset{compat=1.18}
\usepackage{booktabs}
\usepackage{siunitx}
\usepackage{amsthm}
\usepackage[colorlinks=true, linkcolor=blue, citecolor=blue, urlcolor=blue]{hyperref}
\usepackage{cleveref}
\usepackage{fancyhdr}
\usepackage{tcolorbox}
\usepackage{mathtools}

\geometry{a4paper, margin=2cm}

% Headers and Footers
\pagestyle{fancy}
\fancyhf{}
\fancyhead[L]{Johann Pascher}
\fancyhead[R]{Dynamische Masse von Photonen im T0-Modell - Aktualisiertes Rahmenwerk}
\fancyfoot[C]{\thepage}
\renewcommand{\headrulewidth}{0.4pt}
\renewcommand{\footrulewidth}{0.4pt}

% Table of Contents Styling
\renewcommand{\cftsecfont}{\color{blue}}
\renewcommand{\cftsubsecfont}{\color{blue}}
\renewcommand{\cftsecpagefont}{\color{blue}}
\renewcommand{\cftsubsecpagefont}{\color{blue}}
\setlength{\cftsecindent}{1cm}
\setlength{\cftsubsecindent}{2cm}

% Custom commands (consistent with T0 model reference)
\newcommand{\Tfield}{T(x,t)}
\newcommand{\betaT}{\beta_{\text{T}}}
\newcommand{\alphaEM}{\alpha_{\text{EM}}}
\newcommand{\alphaW}{\alpha_{\text{W}}}
\newcommand{\Mpl}{M_{\text{Pl}}}
\newcommand{\Tzerot}{T_0(\Tfield)}
\newcommand{\Tzero}{T_0}
\newcommand{\vecx}{\vec{x}}
\newcommand{\gammaf}{\gamma_{\text{Lorentz}}}
\newcommand{\DhiggsT}{\Tfield (\partial_\mu + ig A_\mu) \Phi + \Phi \partial_\mu \Tfield}
\newcommand{\xipar}{\xi}
\newcommand{\lP}{\ell_{\text{P}}}

\newtheorem{theorem}{Theorem}[section]
\newtheorem{proposition}[theorem]{Proposition}
\newtheorem{definition}[theorem]{Definition}

\title{Dynamische Masse von Photonen und ihre Implikationen für Nichtlokalität \\ im T0-Modell: Aktualisiertes Rahmenwerk mit \\ vollständigen geometrischen Grundlagen}
\author{Johann Pascher}
\date{\today}

\begin{document}
	
	\maketitle
	
	\begin{abstract}
		Diese aktualisierte Arbeit untersucht die Implikationen der Zuweisung einer dynamischen, frequenzabhängigen effektiven Masse zu Photonen innerhalb des umfassenden Rahmenwerks des T0-Modells, aufbauend auf der vollständigen feldtheoretischen Herleitung und dem natürlichen Einheitensystem, in dem $\hbar = c = \alpha_{\text{EM}} = \beta_{\text{T}} = 1$ gilt. Die Theorie etabliert die fundamentale Beziehung $\Tfield = \frac{1}{\max(m, \omega)}$ mit der Dimension $[E^{-1}]$ und bietet eine einheitliche Behandlung massiver Teilchen und Photonen durch die drei fundamentalen Feldgeometrien. Die dynamische Photonenmasse $m_\gamma = \omega$ führt energieabhängige Nichtlokalitätseffekte ein, mit testbaren Vorhersagen.  Alle Formulierungen bewahren strikte dimensionale Konsistenz mit den festen T0-Parametern $\beta = 2Gm/r$, $\xi = 2\sqrt{G} \cdot m$ und dem kosmischen Abschirmfaktor $\xi_{\text{eff}} = \xi/2$ für unendliche Felder.
	\end{abstract}
	
	\tableofcontents
	\newpage
	
	\section{Einführung: T0-Modell-Grundlage für Photonendynamik}
	
	Diese aktualisierte Analyse baut auf dem umfassenden T0-Modell-Rahmenwerk auf, das in der feldtheoretischen Herleitung etabliert wurde, und integriert die vollständigen geometrischen Grundlagen und das natürliche Einheitensystem. Das Konzept der dynamischen effektiven Masse für Photonen entsteht natürlich aus dem fundamentalen Zeit-Masse-Dualitätsprinzip des T0-Modells.
	
	\subsection{Fundamentales T0-Modell-Rahmenwerk}
	
	Das T0-Modell basiert auf der intrinsischen Zeitfelddefinition:
	
	\begin{equation}
		\boxed{\Tfield = \frac{1}{\max(m(\vec{x},t), \omega)}}
		\label{eq:intrinsic_time_field}
	\end{equation}
	
	\textbf{Dimensionale Verifikation}: $[\Tfield] = [1/E] = [E^{-1}]$ in natürlichen Einheiten \checkmark
	
	Dieses Feld erfüllt die fundamentale Feldgleichung:
	\begin{equation}
		\nabla^2 m(\vec{x},t) = 4\pi G \rho(\vec{x},t) \cdot m(\vec{x},t)
		\label{eq:field_equation}
	\end{equation}
	
	Daraus ergeben sich die Schlüsselparameter:
	
	\begin{tcolorbox}[colback=blue!5!white,colframe=blue!75!black,title=T0-Modell-Parameter für Photonenanalyse]
		\begin{align}
			\beta &= \frac{2Gm}{r} \quad [1] \text{ (dimensionslos)} \\
			\xi &= 2\sqrt{G} \cdot m \quad [1] \text{ (dimensionslos)} \\
			\beta_T &= 1 \quad [1] \text{ (natürliche Einheiten)} \\
			\alpha_{\text{EM}} &= 1 \quad [1] \text{ (natürliche Einheiten)}
		\end{align}
	\end{tcolorbox}
	
	\subsection{Photonenintegration in der Zeit-Masse-Dualität}
	
	Für Photonen weist das T0-Modell eine effektive Masse zu:
	\begin{equation}
		m_\gamma = \omega
		\label{eq:photon_effective_mass}
	\end{equation}
	
	\textbf{Dimensionale Verifikation}: $[m_\gamma] = [\omega] = [E]$ in natürlichen Einheiten \checkmark
	
	Dies ergibt das intrinsische Zeitfeld des Photons:
	\begin{equation}
		\Tfield_\gamma = \frac{1}{\omega}
		\label{eq:photon_time_field}
	\end{equation}
	
	\begin{tcolorbox}[colback=yellow!5!white,colframe=orange!75!black,title=Praktische Vereinfachung]
		\textbf{Vereinfachung:} Da alle Messungen in unserem endlichen, beobachtbaren Universum lokal erfolgen, wird nur die \textbf{lokalisierte Feldgeometrie} verwendet:
		
		$\xi = 2\sqrt{G} \cdot m$ und $\beta = \frac{2Gm}{r}$ für alle Anwendungen.
		
		Der kosmische Abschirmfaktor $\xi_{\text{eff}} = \xi/2$ entfällt.
	\end{tcolorbox}	
	\textbf{Physikalische Interpretation}: Höherenergetische Photonen haben kürzere intrinsische Zeitskalen, was energieabhängige zeitliche Dynamik schafft.
	
	\section{Energieabhängige Nichtlokalität und Quantenkorrelationen}
	
	\subsection{Verschränkte Photonensysteme}
	
	Für verschränkte Photonen mit Energien $\omega_1$ und $\omega_2$ ist die Zeitfelddifferenz:
	\begin{equation}
		\Delta T_\gamma = \left|\frac{1}{\omega_1} - \frac{1}{\omega_2}\right|
		\label{eq:time_field_difference}
	\end{equation}
	
	\textbf{Physikalische Konsequenz}: Quantenkorrelationen erfahren energieabhängige Verzögerungen.
	
	\subsection{Modifizierte Bell-Ungleichung}
	
	Die energieabhängigen Zeitfelder führen zu einer modifizierten Bell-Ungleichung:
	\begin{equation}
		|E(a,b) - E(a,c)| + |E(a',b) + E(a',c)| \leq 2 + \epsilon(\omega_1, \omega_2)
		\label{eq:modified_bell_inequality}
	\end{equation}
	
	wobei:
	\begin{equation}
		\epsilon(\omega_1, \omega_2) = \alpha_{\text{corr}} \left|\frac{1}{\omega_1} - \frac{1}{\omega_2}\right| \frac{2G\langle m \rangle}{r}
		\label{eq:bell_correction}
	\end{equation}
	
	mit $\alpha_{\text{corr}}$ als Korrelationskopplungskonstante und $\langle m \rangle$ als durchschnittliche Masse im experimentellen Aufbau.
	

	\section{Experimentelle Vorhersagen und Tests}
	
	\subsection{Hochpräzisions-Quantenoptik-Tests}
	
	\subsubsection{Energieabhängige Bell-Tests}
	
	Vorhergesagte Zeitverzögerung zwischen verschränkten Photonen:
	\begin{equation}
		\Delta t_{\text{corr}} = \frac{G\langle m \rangle}{r} \left|\frac{1}{\omega_1} - \frac{1}{\omega_2}\right|
		\label{eq:correlation_time_delay}
	\end{equation}
	
	Für Laborbedingungen mit $\langle m \rangle \sim 10^{-3}$ kg, $r \sim 10$ m und $\omega_1,\omega_2 \sim 1$ eV:
	\begin{equation}
		\Delta t_{\text{corr}} \sim 10^{-21} \text{ s}
		\label{eq:laboratory_delay}
	\end{equation}
	

	\section{Dimensionale Konsistenz-Verifikation}
	
	\begin{table}[htbp]
		\centering
		\begin{tabular}{lccl}
			\toprule
			\textbf{Gleichung} & \textbf{Linke Seite} & \textbf{Rechte Seite} & \textbf{Status} \\
			\midrule
			Photonen-effektive Masse & $[m_\gamma] = [E]$ & $[\omega] = [E]$ & \checkmark \\
			Photonen-Zeitfeld & $[T_\gamma] = [E^{-1}]$ & $[1/\omega] = [E^{-1}]$ & \checkmark \\
			Energieverlustrate & $[d\omega/dr] = [E^2]$ & $[g_T \omega^2 2G/r^2] = [E^2]$ & \checkmark \\
			Zeitfelddifferenz & $[\Delta T_\gamma] = [E^{-1}]$ & $[|1/\omega_1 - 1/\omega_2|] = [E^{-1}]$ & \checkmark \\
			Bell-Korrektur & $[\epsilon] = [1]$ & $[\alpha_{\text{corr}} \Delta T_\gamma \beta] = [1]$ & \checkmark \\
			\bottomrule
		\end{tabular}
		\caption{Dimensionale Konsistenz-Verifikation für Photonendynamik im T0-Modell}
	\end{table}
	
	\section{Schlussfolgerungen}
	
	\subsection{Zusammenfassung der Schlüsselergebnisse}
	
	Diese aktualisierte Analyse zeigt, dass das Konzept der dynamischen Photonenmasse nahtlos in das umfassende T0-Modell-Rahmenwerk integriert:
	
	\begin{enumerate}
		\item \textbf{Einheitliche Behandlung}: Photonen und massive Teilchen folgen derselben fundamentalen Beziehung $T = 1/\max(m,\omega)$
		\item \textbf{Energieabhängige Effekte}: Photonendynamik hängt von der Frequenz durch das intrinsische Zeitfeld ab
		\item \textbf{Modifizierte Nichtlokalität}: Quantenkorrelationen erfahren energieabhängige Verzögerungen
		\item \textbf{Testbare Vorhersagen}: Spezifische experimentelle Signaturen unterscheiden T0 von der Standardtheorie
		\item \textbf{Dimensionale Konsistenz}: Alle Gleichungen im natürlichen Einheitenrahmen verifiziert
		\item \textbf{Parameterfreie Theorie}: Alle Effekte durch fundamentale T0-Parameter bestimmt
	\end{enumerate}
	

\end{document}