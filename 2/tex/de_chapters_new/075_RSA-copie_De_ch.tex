% Chapter file: 075_RSA-copie_De_ch.tex
% Source: 075_RSA-copie_De.tex

\chapter{Mathematische Analyse des T0-Shor Algorithmus: Theoretischer Rahmen und Berechnungskomplexität Eine rigorose Untersuchung des T0-Energiefeld-Ansatzes zur Ganzzahlfaktorisierung}

\section*{Abstract}
		Diese Arbeit präsentiert eine mathematische Analyse des T0-Shor Algorithmus basierend auf einer Energiefeld-Formulierung. Wir untersuchen die theoretischen Grundlagen der Zeit-Masse-Dualität $T(x,t) \cdot m(x,t) = 1$ und deren Anwendung auf die Ganzzahlfaktorisierung. Die Analyse umfasst Feldgleichungen, wellenartiges Verhalten ähnlich der akustischen Ausbreitung und materialabhängige Parameter abgeleitet aus der Vakuumphysik. Wir leiten Skalierungsbeziehungen für verschiedene Raumdimensionen ab und untersuchen die Rolle der Rechengenauigkeit für die Algorithmusleistung. Das mathematische Framework wird auf Konsistenz überprüft und praktische Limitationen werden identifiziert.
	
	
	\section{Einleitung}
	
	Der T0-Shor Algorithmus stellt eine theoretische Erweiterung von Shors Faktorisierungsalgorithmus dar, basierend auf Energiefelddynamik anstelle quantenmechanischer Superposition. Diese Arbeit untersucht die mathematischen Grundlagen dieses Ansatzes ohne Behauptungen über praktische Implementierbarkeit oder Überlegenheit gegenüber bestehenden Methoden.
	
	\subsection{Theoretisches Framework}
	
	Das T0-Modell führt folgende fundamentale mathematische Strukturen ein:
	
	\begin{align}
		\text{Zeit-Masse-Dualität}: \quad &T(x,t) \cdot m(x,t) = 1 \label{eq:duality}\\
		\text{Feldgleichung}: \quad &\nabla^2 T(x) = -\frac{\rho(x)}{T(x)^2} \label{eq:field}\\
		\text{Energieentwicklung}: \quad &\frac{\partial^2 E}{\partial t^2} = -\omega^2 E \label{eq:evolution}
	\end{align}
	
	Der Kopplungsparameter $\xipar$ wird theoretisch aus Higgs-Feld-Wechselwirkungen abgeleitet:
	\begin{equation}
		\xipar = g_H \cdot \frac{\langle\phi\rangle}{v_{EW}} \label{eq:xi_higgs}
	\end{equation}
	wobei $g_H$ die Higgs-Kopplungskonstante, $\langle\phi\rangle$ der Vakuumerwartungswert und $v_{EW} = 246$ GeV die elektroschwache Skala ist.
	
	\section{Mathematische Grundlagen}
	
	\subsection{Wellenartiges Verhalten von T0-Feldern}
	
	Das T0-Feld zeigt wellenartige Ausbreitungscharakteristika analog zu akustischen Wellen in Medien. Die fundamentale Wellengleichung für T0-Felder lautet:
	
	\begin{equation}
		\nabla^2 T - \frac{1}{c_{T0}^2} \frac{\partial^2 T}{\partial t^2} = -\frac{\rho(x,t)}{T(x,t)^2} \label{eq:wave_equation}
	\end{equation}
	
	wobei $c_{T0}$ die T0-Feld-Ausbreitungsgeschwindigkeit im Medium ist, analog zur Schallgeschwindigkeit.
	
	\subsection{Mediumabhängige Eigenschaften}
	
	Ähnlich wie akustische Wellen hängt die T0-Feld-Ausbreitung kritisch von den Mediumeigenschaften ab:
	
	\textbf{T0-Feld-Geschwindigkeit in verschiedenen Medien}:
	\begin{align}
		c_{T0,vacuum} &= c \sqrt{\frac{\xipar_0}{\xipar_{vacuum}}} \\
		c_{T0,metal} &= c \sqrt{\frac{\xipar_0 \epsilon_r}{\xipar_{vacuum}}} \\
		c_{T0,dielectric} &= \frac{c}{\sqrt{\epsilon_r \mu_r}} \sqrt{\frac{\xipar_0}{\xipar_{vacuum}}} \\
		c_{T0,plasma} &= c \sqrt{1 - \frac{\omega_p^2}{\omega^2}} \sqrt{\frac{\xipar_0}{\xipar_{vacuum}}}
	\end{align}
	
	wobei $\omega_p$ die Plasmafrequenz und $\epsilon_r$, $\mu_r$ die relative Permittivität und Permeabilität sind.
	
	\subsection{Randbedingungen und Reflexionen}
	
	An Grenzflächen zwischen verschiedenen Medien erfüllen T0-Felder Randbedingungen ähnlich elektromagnetischen Wellen:
	
	\textbf{Kontinuitätsbedingungen}:
	\begin{align}
		T_1|_{interface} &= T_2|_{interface} \quad \text{(Feldkontinuität)} \\
		\frac{1}{m_1} \frac{\partial T_1}{\partial n}\bigg|_{interface} &= \frac{1}{m_2} \frac{\partial T_2}{\partial n}\bigg|_{interface} \quad \text{(Flusskontinuität)}
	\end{align}
	
	\textbf{Reflexions- und Transmissionskoeffizienten}:
	\begin{align}
		r &= \frac{Z_1 - Z_2}{Z_1 + Z_2} \quad \text{(Reflexionskoeffizient)} \\
		t &= \frac{2Z_1}{Z_1 + Z_2} \quad \text{(Transmissionskoeffizient)}
	\end{align}
	
	wobei $Z_i = \sqrt{m_i/T_i}$ die T0-Feld-Impedanz in Medium $i$ ist.
	
	\subsection{Hyperbolische Geometrie im Dualitätsraum}
	
	Die Zeit-Masse-Dualität (Gl.~\ref{eq:duality}) definiert eine hyperbolische Metrik im $(T,m)$ Parameterraum:
	
	\begin{equation}
		ds^2 = \frac{dT \cdot dm}{T \cdot m} = \frac{d(\ln T) \cdot d(\ln m)}{T \cdot m}
	\end{equation}
	
	Diese Geometrie ist charakterisiert durch:
	\begin{itemize}
		\item Konstante negative Krümmung: $K = -1$
		\item Invariantes Maß: $d\mu = \frac{dT \, dm}{T \cdot m}$
		\item Isometriegruppe: $PSL(2,\mathbb{R})$
	\end{itemize}
	
	\subsection{Atomskalige T0-Feld-Parameter}
	
	Da die Vakuumbedingungen bekannt sind, kann das atomare T0-Feld-Verhalten aus Fundamentalkonstanten abgeleitet werden:
	
	\textbf{Vakuum T0-Feld-Basislinie}:
	\begin{align}
		c_{T0,vacuum} &= c = 2,998 \times 10^8 \text{ m/s} \\
		\xipar_{vacuum} &= \xipar_0 = \frac{g_H \langle\phi\rangle}{v_{EW}} \\
		Z_{vacuum} &= Z_0 = \sqrt{\frac{\mu_0}{\epsilon_0}} = 376,73 \text{ $\Omega$}
	\end{align}
	
	\textbf{Atomskalige Ableitungen}:
	
	Für das Wasserstoffatom (Fundamentalfall):
	\begin{align}
		a_0 &= \frac{4\pi\epsilon_0\hbar^2}{m_e e^2} = 5,292 \times 10^{-11} \text{ m} \quad \text{(Bohr-Radius)} \\
		\alpha &= \frac{e^2}{4\pi\epsilon_0\hbar c} = 7,297 \times 10^{-3} \quad \text{(Feinstrukturkonstante)} \\
		r_{e} &= \frac{e^2}{4\pi\epsilon_0 m_e c^2} = 2,818 \times 10^{-15} \text{ m} \quad \text{(klassischer Elektronenradius)}
	\end{align}
	
	\textbf{T0-Feld-Atomparameter}:
	\begin{align}
		c_{T0,atom} &= c \cdot \alpha = 2,19 \times 10^6 \text{ m/s} \\
		\xipar_{atom} &= \xipar_0 \cdot \frac{E_{Rydberg}}{m_e c^2} = \xipar_0 \cdot \frac{\alpha^2}{2} \\
		\lambda_{T0,atom} &= \frac{2\pi a_0}{\alpha} = 2,426 \times 10^{-9} \text{ m}
	\end{align}
	
	\textbf{Skalierung für verschiedene Atome}:
	
	Für Atom mit Kernladung $Z$ und Massenzahl $A$:
	\begin{align}
		c_{T0,Z} &= c_{T0,atom} \cdot Z^{2/3} \quad \text{(Geschwindigkeitsskalierung)} \\
		\xipar_{Z} &= \xipar_{atom} \cdot \frac{Z^4}{A} \quad \text{(Kopplungsskalierung)} \\
		a_{Z} &= \frac{a_0}{Z} \quad \text{(Größenskalierung)} \\
		E_{binding,Z} &= 13,6 \text{ eV} \cdot Z^2 \quad \text{(Energieskalierung)}
	\end{align}
	
	\section{T0-Shor Algorithmus-Formulierung}
	
	\subsection{Geometrisches Hohlraum-Design für Periodenfindung}
	
	Der T0-Shor Algorithmus nutzt geometrische Resonanzhohlräume zur Periodendetektion, analog zu akustischen Resonatoren:
	
	\textbf{Resonanzhohlraum-Dimensionen} für Periode $r$:
	\begin{equation}
		L_{cavity} = n \cdot \frac{\lambda_{T0}}{2} = n \cdot \frac{c_{T0} \cdot r}{2f_0}
	\end{equation}
	
	wobei $f_0$ die fundamentale Antriebsfrequenz und $n$ die Modenzahl ist.
	
	\textbf{Gütefaktor} der Resonanz:
	\begin{equation}
		Q = \frac{f_r}{\Delta f} = \frac{\pi}{\xipar} \cdot \frac{L_{cavity}}{\lambda_{T0}}
	\end{equation}
	
	Höhere $Q$-Werte bieten schärfere Periodendetektion, erfordern aber längere Beobachtungszeiten.
	
	\subsection{Multi-Moden-Resonanzanalyse}
	
	Anstelle der Quanten-Fourier-Transformation verwendet der T0-Shor Algorithmus Multi-Moden-Hohlraumanalyse:
	
	\begin{align}
		\text{Modenspektrum}: \quad &T(x,y,z,t) = \sum_{mnp} A_{mnp}(t) \psi_{mnp}(x,y,z) \\
		\text{Periodendetektion}: \quad &r = \frac{c_{T0}}{2f_{resonance}} \cdot \frac{geometry\_factor}{mode\_number}
	\end{align}
	
	\section{Selbstverstärkende $\xipar$-Optimierung: Die Fehlerreduktions-Rückkopplungsschleife}
	
	\subsection{Die fundamentale Entdeckung: Rechenfehler verschlechtern $\xipar$}
	
	Eine kritische Erkenntnis ergibt sich: Rechengenauigkeit beeinflusst direkt $\xipar$-Parameter-Werte und erschafft einen selbstverstärkenden Optimierungszyklus:
	
	\textbf{Fehlerabhängige $\xipar$-Verschlechterung}:
	\begin{equation}
		\xipar_{effective} = \xipar_{ideal} \cdot \exp\left(-\alpha \sum_{i} p_{error,i} \cdot w_i\right)
	\end{equation}
	
	wobei $p_{error,i}$ Fehlerwahrscheinlichkeiten und $w_i$ Kritikalitätsgewichte sind.
	
	\textbf{Die selbstverstärkende Beziehung}:
	\begin{equation}
		\text{Weniger Fehler} \rightarrow \text{Höheres } \xipar \rightarrow \text{Bessere Feldkohärenz} \rightarrow \text{Noch weniger Fehler}
	\end{equation}
	
	\subsection{Mathematisches Modell der Rückkopplungsschleife}
	
	\textbf{Differentialgleichung für $\xipar$-Entwicklung}:
	\begin{equation}
		\frac{d\xipar}{dt} = \beta \xipar \left(1 - \frac{R_{error}}{R_{threshold}}\right) - \gamma \xipar \frac{R_{error}}{R_{reference}}
	\end{equation}
	
	Kritische Erkenntnis: Wenn $R_{error} < R_{threshold}$, wächst $\xipar$ exponentiell.
	
	\textbf{Typische Schwellenwerte}:
	\begin{align}
		R_{critical} &\approx 10^{-12} \text{ Fehler pro Operation} \\
		R_{64bit} &\approx 10^{-16} \text{ (bereits unter Schwellenwert)} \\
		R_{32bit} &\approx 10^{-7} \text{ (über Schwellenwert)}
	\end{align}
	
	Standard 64-Bit Arithmetik ist bereits im $\xipar$-Verstärkungsbereich.
	
	\section{Vakuum-abgeleitete Atomparameter: Keine freien Parameter}
	
	\subsection{Fundamentale Parameter-Ableitung}
	
	Da Vakuumbedingungen bekannt sind, können alle atomaren T0-Parameter aus Fundamentalkonstanten abgeleitet werden:
	
	\textbf{Vakuum-Basislinie}:
	\begin{align}
		c_{T0,vacuum} &= c = 2,998 \times 10^8 \text{ m/s} \\
		\xipar_{vacuum} &= \xipar_0 = \frac{g_H \langle\phi\rangle}{v_{EW}} \quad \text{(Higgs-abgeleitet)} \\
		Z_{vacuum} &= Z_0 = 376,73 \text{ $\Omega$}
	\end{align}
	
	\textbf{Materialspezifische Vorhersagen}:
	
	Keine freien Parameter - alle $\xipar$-Werte sind berechenbar:
	\begin{align}
		\xipar_{Si} &= \xipar_0 \cdot 0,98 \cdot \frac{E_g}{k_B T} = 43,7 \xipar_0 \quad \text{(bei 300K)} \\
		\xipar_{metal} &= \xipar_0 \sqrt{\frac{n e^2}{\epsilon_0 m_e \omega^2}} \approx (10^{-4} \text{ bis } 10^{-3}) \xipar_0 \\
		\xipar_{SC} &= \xipar_0 \cdot \frac{\Delta}{k_B T_c} \cdot \tanh\left(\frac{\Delta}{2k_B T}\right)
	\end{align}
	
	\textbf{Experimentell testbare Vorhersagen}:
	\begin{align}
		\text{Temperaturskalierung}: \quad &\xipar(T_2)/\xipar(T_1) = T_1/T_2 \\
		\text{Isotopeffekt}: \quad &\xipar(^{13}C)/\xipar(^{12}C) = \sqrt{12/13} = 0,962 \\
		\text{Druckabhängigkeit}: \quad &\xipar(p) = \xipar_0 \left(1 + \kappa \frac{\Delta p}{p_0}\right)
	\end{align}
	
	\section{$\xipar$ als multifunktionaler Parameter: Jenseits einfacher Kopplung}
	
	\subsection{Multiple versteckte Funktionen von $\xipar$}
	
	$\xipar$ erfüllt mehrere fundamentale Rollen jenseits einfacher Feld-Materie-Kopplung:
	
	\begin{align}
		\text{1. Kopplungsstärke}: \quad &\xipar_{coupling} = \text{Feld-Materie-Wechselwirkung} \\
		\text{2. Asymmetrie-Generator}: \quad &\xipar_{asymmetry} = \text{Richtungspräferenz} \\
		\text{3. Skalen-Setzer}: \quad &\xipar_{scale} = \text{charakteristische Länge/Zeit} \\
		\text{4. Informations-Kodierer}: \quad &\xipar_{info} = \text{Berechnungskomplexitäts-Modifikator} \\
		\text{5. Symmetriebrecher}: \quad &\xipar_{symmetry} = \text{spontane Ordnung}
	\end{align}
	
	\subsection{$\xipar$-induzierte Berechnungsasymmetrien}
	
	\textbf{Berechnungschiralität}:
	
	Auch in mathematisch symmetrischen Operationen erschafft $\xipar$ Berechnungspräferenzen:
	\begin{align}
		\text{Vorwärtsberechnung}: \quad &\xipar_{forward} = \xipar_0 \\
		\text{Umkehrberechnung}: \quad &\xipar_{inverse} = \xipar_0 / \alpha \quad (\alpha > 1) \\
		\text{Verifikation}: \quad &\xipar_{verify} = \xipar_0 \cdot \beta \quad (\beta > 1)
	\end{align}
	
	Dies erschafft Berechnungschiralität wo Verifikation einfacher ist als Berechnung.
	
	\subsection{$\xipar$-Gedächtnis und Geschichtsabhängigkeit}
	
	\textbf{$\xipar$ behält Berechnungsgeschichte}:
	\begin{equation}
		\xipar(t) = \xipar_0 + \int_0^t K(t-\tau) \cdot f(\text{computation}(\tau)) \, d\tau
	\end{equation}
	
	wobei $K(t-\tau)$ ein Gedächtniskern ist.
	
	\section{Dimensionale Skalierung: Fundamentale Unterschiede zwischen 2D und 3D}
	
	\subsection{Wellenausbreitungs-Skalierungsgesetze}
	
	Der fundamentale Unterschied zwischen 2D und 3D Raum beeinflusst T0-Feld-Verhalten tiefgreifend:
	
	\textbf{Dimensionale Feldgleichungen}:
	\begin{align}
		\text{2D}: \quad &\frac{1}{r} \frac{\partial}{\partial r}\left(r \frac{\partial T}{\partial r}\right) = -\frac{\rho(r)}{T(r)^2} \\
		\text{3D}: \quad &\frac{1}{r^2} \frac{\partial}{\partial r}\left(r^2 \frac{\partial T}{\partial r}\right) = -\frac{\rho(r)}{T(r)^2}
	\end{align}
	
	\textbf{Green-Funktions-Unterschiede}:
	\begin{align}
		G_{2D}(r) &= -\frac{1}{2\pi} \ln(r) \quad \text{(logarithmischer Abfall)} \\
		G_{3D}(r) &= \frac{1}{4\pi r} \quad \text{(Potenzgesetz-Abfall)}
	\end{align}
	
	\subsection{Kritische Dimensionsschwellenwerte}
	
	\textbf{Untere kritische Dimension}: $d_c^{lower} = 2$
	
	Unter 2D können T0-Felder nicht konventionell propagieren:
	\begin{equation}
		\text{1D}: \quad T(x) = T_0 + A|x| \quad \text{(lineares Wachstum, unphysikalisch)}
	\end{equation}
	
	\textbf{Obere kritische Dimension}: $d_c^{upper} = 4$
	
	Über 4D wird die Molekularfeld-Theorie exakt:
	\begin{equation}
		\text{4D+}: \quad \xipar_{eff} = \xipar_0 \quad \text{(dimensionsunabhängig)}
	\end{equation}
	
	\subsection{Algorithmische Leistungsskalierung}
	
	\textbf{Dimensionale Skalierung beeinflusst T0-Shor Leistung}:
	\begin{align}
		\text{2D Implementierung}: \quad F_{2D} &= \sqrt{\ln(N)} \quad \text{(logarithmisch)} \\
		\text{3D Implementierung}: \quad F_{3D} &= N^{1/3} \quad \text{(Potenzgesetz)}
	\end{align}
	
	\textbf{Optimale Geometrien nach Dimension}:
	\begin{align}
		\text{2D}: \quad &\text{Lange, dünne Strukturen bevorzugt} \\
		&Q \propto L/\lambda_{T0} \\
		\text{3D}: \quad &\text{Kompakte, sphärische Geometrien optimal} \\
		&Q \propto (V/\lambda_{T0}^3)^{1/3}
	\end{align}
	
	\section{Die fundamentale Natur von Zahlen und Primstruktur}
	
	\subsection{Primzahlen als das Gerüst der Mathematik}
	
	Der Grund warum alle Periodenfindungsalgorithmen funktionieren (FFT, Quanten-Shor, T0-Shor) liegt in der fundamentalen Struktur unseres Zahlensystems:
	
	\textbf{Primzahlen als mathematische Atome}:
	\begin{equation}
		\text{Jede Ganzzahl } n > 1: \quad n = p_1^{a_1} \cdot p_2^{a_2} \cdot ... \cdot p_k^{a_k} \quad \text{(eindeutig)}
	\end{equation}
	
	Primzahlen bilden das fundamentale Gerüst - jede Zahl ist vollständig durch Primzahlen bestimmt.
	
	\textbf{Warum Periodizität aus Primstruktur entsteht}:
	\begin{align}
		\text{Euler-Theorem}: \quad &a^{\phi(N)} \equiv 1 \pmod{N} \\
		\text{Periodizität}: \quad &f(x) = a^x \bmod N \text{ ist inhärent periodisch} \\
		\text{Universelles Prinzip}: \quad &\text{Primstruktur} \rightarrow \text{Periodizität} \rightarrow \text{Fourier-Detektion}
	\end{align}
	
	\textbf{Warum Periode Faktorisierungsinformation enthält}:
	\begin{equation}
		a^r \equiv 1 \pmod{N} \Rightarrow a^r - 1 = (a^{r/2} - 1)(a^{r/2} + 1) \equiv 0 \pmod{N}
	\end{equation}
	
	Die Periode $r$ kodiert die Primfaktoren durch diese algebraische Beziehung.
	
	\section{Kritische Bewertung: Warum T0-Shor nur für kleine Zahlen funktioniert}
	
	\subsection{Die Präzisionsbarriere}
	
	Trotz der theoretischen Eleganz steht T0-Shor vor einer fundamentalen Präzisionslimitierung die seine praktische Anwendbarkeit einschränkt:
	
	\textbf{Erforderliche Resonanzpräzision für Periode r}:
	\begin{equation}
		\Delta f_{required} = \frac{f_0}{r} - \frac{f_0}{r+1} = \frac{f_0}{r(r+1)} \approx \frac{f_0}{r^2}
	\end{equation}
	
	Für kryptographisch relevante Zahlen wo $r \approx N$:
	\begin{equation}
		\Delta f_{required} \approx \frac{f_0}{N^2}
	\end{equation}
	
	\textbf{Rechenpräzisionsgrenzen}:
	\begin{align}
		\text{64-Bit Präzision}: \quad &\epsilon \approx 10^{-16} \rightarrow N_{max} \approx 10^8 \text{ (27 Bits)} \\
		\text{128-Bit Präzision}: \quad &\epsilon \approx 10^{-34} \rightarrow N_{max} \approx 10^{17} \text{ (56 Bits)} \\
		\text{1024-Bit RSA erfordert}: \quad &\epsilon \approx 10^{-617} \text{ (unmöglich)}
	\end{align}
	
	\subsection{Die Präzisionsbarriere und Skalierungslimitationen}
	
	Wichtige Klarstellung: T0-Shor funktioniert theoretisch für große Zahlen. Die Limitationen sind praktisch, nicht theoretisch:
	
	\textbf{Fundamentale Skalierungsherausforderungen}:
	\begin{align}
		\text{Speicheranforderungen}: \quad &M(N) = O(N) \text{ Feldpunkte} \\
		\text{Rechenpräzision}: \quad &\epsilon_{required} = O(1/N^2) \\
		\text{Feldauflösung}: \quad &\Delta r = O(1/N) \text{ für Periodendetektion} \\
		\text{Operationszahl}: \quad &\text{Immer noch } O(\log N) \text{ pro erfolgreicher Vorhersage}
	\end{align}
	
	\subsection{Vergleich mit bestehenden Methoden}
	
	\begin{table}[htbp]
		\centering
		\resizebox{\textwidth}{!}{%
		\begin{tabular}{lcccc}
			\toprule
			\textbf{Methode} & \textbf{Operationen (kleine N)} & \textbf{Operationen (große N)} & \textbf{Erfolgsrate} & \textbf{Hardware} \\
			\midrule
			Triviale Faktorisierung & $O(\sqrt{N})$ & $O(\sqrt{N})$ & 100\% & Standard \\
			Klassische FFT & $O(N \log N)$ & $O(N \log N)$ & 100\% & Standard \\
			Quanten-Shor & $O((\log N)^3)$ & $O((\log N)^3)$ & $\approx$50\% & Quantum \\
			T0-Shor (Vorhersage-Treffer) & $O(\log N)$ & $O(\log N)$ & Variabel & Standard \\
			T0-Shor (keine Vorhersage) & $O(N \log N)$ & Durch Präzision begrenzt & Variabel & Standard \\
			\bottomrule
		\end{tabular}}
		\caption{Realistische Vergleich von Faktorisierungsmethoden}
		\label{tab:method_comparison_realistic}
	\end{table}
	
	\textbf{Quantencomputer und das I/O-Engpass}:
	
	Quantencomputer mit elektronenbasiertem Speicher haben einen theoretischen Speichervorteil, stehen aber vor denselben fundamentalen I/O-Limitationen:
	
	\begin{table}[htbp]
		\centering
		\resizebox{\textwidth}{!}{%
		\begin{tabular}{lcccc}
			\toprule
			\textbf{System} & \textbf{Speicher} & \textbf{Eingabe-Abbildung} & \textbf{Ausgabe-Auslesen} & \textbf{Engpass} \\
			\midrule
			T0-Shor & RAM-Limitierung & Direkt & Direkt & Speicherskalierung \\
			QC & Elektronenzustände & Exponentielle Kodierung & Messkollaps & I/O-Komplexität \\
			T0 + QC & Elektronenzustände & Selbes QC-Problem & Selbes QC-Problem & I/O-Komplexität \\
			\bottomrule
		\end{tabular}}
		\caption{Speichersysteme und ihre fundamentalen Engpässe}
		\label{tab:memory_bottlenecks}
	\end{table}
	
	\section{Schlussfolgerungen}
	
	\subsection{Zentrale Erkenntnisse}
	
	Die Zeit-Masse-Dualität führt zu einer mathematisch konsistenten Erweiterung des Shor-Algorithmus mit folgenden Eigenschaften:
	
	\begin{enumerate}
		\item Theoretischer Rahmen: Hyperbolische Geometrie im Dualitätsraum
		\item Wellencharakteristik: T0-Felder verhalten sich ähnlich akustischen Wellen
		\item Vakuum-Ableitung: Alle Parameter aus Fundamentalkonstanten berechenbar
		\item Selbstverstärkung: Fehlerreduktion verbessert $\xipar$-Parameter
		\item Multifunktionalität: $\xipar$ hat Rollen jenseits einfacher Kopplung
		\item Dimensionale Effekte: 2D und 3D verhalten sich fundamental unterschiedlich
		\item Praktische Grenzen: Präzisions- und Speicheranforderungen begrenzen Anwendbarkeit
	\end{enumerate}
	
	\subsection{Offene mathematische Fragen}
	
	Mehrere mathematische Aspekte erfordern weitere Untersuchung:
	
	\begin{enumerate}
		\item Rigoroser Konvergenzbeweis für Feldentwicklungsgleichungen
		\item Analyse nicht-sphärisch symmetrischer Konfigurationen
		\item Untersuchung chaotischer Dynamik in Massenfeld-Evolution
		\item Verbindung zwischen $\xipar$-Parameter und experimentell messbaren Größen
	\end{enumerate}
	
	Der T0-Shor Algorithmus stellt eine interessante theoretische Konstruktion dar, die Konzepte aus Differentialgeometrie, Feldtheorie und Berechnungskomplexität verbindet. Seine praktischen Vorteile gegenüber bestehenden Methoden bleiben jedoch abhängig von mehreren unbewiesenen Annahmen über die physikalische Realisierbarkeit des zugrundeliegenden mathematischen Frameworks.
	
	\begin{thebibliography}{99}
		\bibitem{shor1994}
		Shor, P. W. (1994). Algorithms for quantum computation: discrete logarithms and factoring. \textit{Proceedings 35th Annual Symposium on Foundations of Computer Science}, 124--134.
		
		\bibitem{higgs1964}
		Higgs, P. W. (1964). Broken symmetries and the masses of gauge bosons. \textit{Physical Review Letters}, 13(16), 508--509.
		
		\bibitem{weinberg1967}
		Weinberg, S. (1967). A model of leptons. \textit{Physical Review Letters}, 19(21), 1264--1266.
		
		\bibitem{gelfand1963}
		Gelfand, I. M., \& Fomin, S. V. (1963). \textit{Calculus of variations}. Prentice-Hall.
		
		\bibitem{arnold1989}
		Arnold, V. I. (1989). \textit{Mathematical methods of classical mechanics}. Springer-Verlag.
		
		\bibitem{evans2010}
		Evans, L. C. (2010). \textit{Partial differential equations}. American Mathematical Society.
		
		\bibitem{shannon1948}
		Shannon, C. E. (1948). A mathematical theory of communication. \textit{Bell System Technical Journal}, 27(3), 379--423.
		
		\bibitem{pollard1975}
		Pollard, J. M. (1975). A Monte Carlo method for factorization. \textit{BIT Numerical Mathematics}, 15(3), 331--334.
		
		\bibitem{lenstra1993}
		Lenstra, A. K., \& Lenstra Jr, H. W. (Eds.). (1993). \textit{The development of the number field sieve}. Springer-Verlag.
		
		\bibitem{nielsen_chuang2010}
		Nielsen, M. A., \& Chuang, I. L. (2010). \textit{Quantum computation and quantum information}. Cambridge University Press.
		
		\bibitem{riemannian_geometry}
		Lee, J. M. (2018). \textit{Introduction to Riemannian manifolds}. Springer.
		
		\bibitem{variational_calculus}
		Kot, M. (2014). \textit{A first course in the calculus of variations}. American Mathematical Society.
		
		\bibitem{pde_stability}
		Strikwerda, J. C. (2004). \textit{Finite difference schemes and partial differential equations}. SIAM.
		
		\bibitem{computational_complexity}
		Sipser, M. (2012). \textit{Introduction to the theory of computation}. Cengage Learning.
		
		\bibitem{information_theory}
		Cover, T. M., \& Thomas, J. A. (2012). \textit{Elements of information theory}. John Wiley \& Sons.
	\end{thebibliography}
