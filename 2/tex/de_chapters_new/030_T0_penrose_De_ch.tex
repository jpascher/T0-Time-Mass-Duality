% Chapter file: 030_T0_penrose_De_ch.tex
% Source: 030_T0_penrose_De.tex
% Generated from standalone document

\chapter{T0-Theorie: Der Terrell-Penrose-Effekt und Massenvariation\\
	\Large Fraktal-konformale Erweiterungen und experimentelle Evidenz}

\begin{abstract}
		Diese Arbeit erkundet die Äquivalenz zwischen Zeitdilatation und Massenvariation in der T0-Theorie der Zeit-Masse-Dualität. Basierend auf Lorentz-Transformationen der speziellen Relativitätstheorie zeigt sie, dass Massenvariation – moduliert durch den theoretisch exakten fraktalen Parameter $\xi = (4/3) \times 10^{-4}$ – eine geometrisch symmetrische Alternative zur Zeitdilatation darstellt. Die empirische Anpassung auf $\xi_{\text{emp}} = 4.35 \times 10^{-4}$ reflektiert aktuelle Messungenauigkeiten. Diese Dualität basiert auf dem intrinsischen Zeitfeld $T(x,t)$, das die Bedingung $T \cdot E = 1$ erfüllt, und löst interpretative Spannungen in relativistischen Effekten, wie denen im Terrell-Penrose-Experiment. T0 postuliert KEINE kosmische Expansion – Rotverschiebung entsteht durch frequenzabhängige Verschiebungen im Zeitfeld. Der Rahmen bietet parameterfreie Vereinheitlichung mit testbaren Vorhersagen für Teilchenphysik und Kosmologie.
	\end{abstract}
	\section{Einführung}
	Die Zeitdilatation ($\tau' = \tau / \gamma$) und Längenkontraktion ($L' = L / \gamma$, mit $\gamma = 1 / \sqrt{1 - \beta^2}$, $\beta = v/c$) der speziellen Relativitätstheorie wurden seit historischen Kritiken wie dem 1931 erschienenen „100 Autoren gegen Einstein'' \cite{030_hundert1931} debattiert. Weitere Kritiker wie Herbert Dingle \cite{030_dingle1972} und moderne Skeptiker \cite{030_gift2010} stellten die physikalische Realität dieser Effekte in Frage. 
	
	Moderne Experimente bestätigen jedoch eindeutig ihre Realität:
	\begin{itemize}
		\item Hafele-Keating (1971): Zeitdilatation mit Atomuhren \cite{030_hafele1972}
		\item GPS-Satelliten: Tägliche Korrekturen von 38 $\mu$s \cite{030_ashby2003}
		\item Myon-Zerfall: Atmosphärische Myonen bei $\gamma \approx 15-20$ \cite{030_rossi1941}
		\item Terrell-Penrose-Visualisierung (2025) \cite{030_terrell2025}
	\end{itemize}
	
	Die T0-Theorie der Zeit-Masse-Dualität \cite{030_pascher2025t0} reformuliert diese Dualität: Zeit und Masse sind komplementäre geometrische Facetten, regiert von $T(x,t) \cdot E = 1$. Massenvariation ($m' = m \gamma$) spiegelt Zeitdilatation symmetrisch wider, vereint durch den fraktalen Parameter $\xi = (4/3) \times 10^{-4}$ aus 3D-fraktaler Geometrie ($D_f \approx 2.94$) \cite{030_pascher2025si, 030_mandelbrot1982}. 
	
	Aus diesem fundamentalen Parameter leiten sich ab:
	\begin{itemize}
		\item Feinstrukturkonstante: $\alpha \approx 1/137$ \cite{030_pascher2025alpha}
		\item Gravitationskonstante: $G = 6.674 \times 10^{-11}$ \cite{030_pascher2025gravity}
		\item Weitere Naturkonstanten \cite{030_weinberg2008}
	\end{itemize}
	
	\section{Grundlagen der T0-Zeit-Masse-Dualität}
	T0 postuliert ein intrinsisches Zeitfeld $T(x,t)$ über Raumzeit, dual zu Energie/Masse $E$ via \cite{030_pascher2025qm, 030_penrose2004}:
	\begin{equation}
		T(x,t) \cdot E = 1,
	\end{equation}
	wobei $E = m c^2$ für Ruhemasse $m$. Diese Beziehung hat Vorläufer in der konformen Feldtheorie \cite{030_francesco1997} und Twistor-Theorie \cite{030_penrose1967}.
	
	Fraktale Korrekturen skalieren relativistische Faktoren:
	\begin{equation}
		\gamma_\text{T0} = \frac{1}{\sqrt{1 - \beta^2}} \cdot (1 + \xi K_\text{frak}), \quad K_\text{frak} = 1 - \frac{\Delta m}{m_e} \approx 0.986,
	\end{equation}
	mit $m_e$ als Elektronmasse und $\Delta m$ als fraktaler Störung \cite{030_pascher2025si}. Dies stimmt mit SI-2019-Redefinitionen überein, mit Abweichungen $<0.0002\%$ \cite{030_codata2019, 030_newell2018}.
	
	T0 bettet die Minkowski-Metrik in eine fraktale Mannigfaltigkeit ein, ähnlich zu Ansätzen in der Quantengravitation \cite{030_rovelli2004, 030_thiemann2007}.
	
	\section{Erweiterte mathematische Ableitung: Äquivalenz von Zeitdilatation und Massenvariation}
	
	\subsection{Zeitdilatation in T0}
	Das dilatierte Intervall ist:
	\begin{equation}
		\Delta \tau' = \Delta \tau \sqrt{1 - \beta^2} = \Delta \tau \cdot \frac{1}{\gamma}.
	\end{equation}
	
	Via Dualität ($T = 1/E$) und unter Berücksichtigung der Arbeiten von Wheeler \cite{030_wheeler1990} und Barbour \cite{030_barbour1999}:
	\begin{equation}
		\Delta \tau' = \Delta \tau \sqrt{1 - \frac{v^2}{c^2}} \cdot \xi \int \frac{\partial T}{\partial t} dt,
	\end{equation}
	wobei das $\xi$-Integral den fraktalen Pfad fractalisiert \cite{030_pascher2025qm}. Dies entspricht LHC-Myon-Lebensdauern ($\gamma \approx 29.3$, Abweichung $<0.01\%$ \cite{030_pdg2024, 030_atlas2023}).
	
	\subsection{Massenvariation als Dual}
	Die Massenvariation folgt aus der fundamentalen Dualität, konsistent mit Machs Prinzip \cite{030_mach1883, 030_sciama1953}:
	\begin{equation}
		\Delta m' = \Delta m / \sqrt{1 - \beta^2} = \Delta m \cdot \gamma \cdot (1 - \xi \Delta T / \tau),
	\end{equation}
	
	Der $\xi$-Term löst die Myon-g-2-Anomalie \cite{030_muong2_2023, 030_pascher2025g2}:
	\begin{equation}
		\Delta a_\mu^{T0} = 247 \times 10^{-11} \text{ (theoretisch mit } \xi = 4/3 \times 10^{-4})
	\end{equation}
	Experimentell: $(249 \pm 87) \times 10^{-11}$ \cite{030_fermilab2023}.
	
	\subsection{Der Terrell-Penrose-Effekt}
	
	\subsubsection{Historische Entdeckung und Fehlinterpretationen}
	
	James Terrell \cite{030_terrell1959} und Roger Penrose \cite{030_penrose1959} zeigten 1959 unabhängig voneinander, dass die visuelle Erscheinung schnell bewegter Objekte fundamental anders ist als lange angenommen. Während die Lorentz-Kontraktion $L' = L/\gamma$ physikalisch real ist, bezieht sie sich auf gleichzeitige Messungen im Beobachterrahmen. Visuelle Beobachtung ist jedoch niemals gleichzeitig – Licht von verschiedenen Teilen des Objekts benötigt unterschiedliche Zeiten zum Beobachter.
	
	Die mathematische Beschreibung für einen Punkt auf einer bewegten Kugel:
	\begin{equation}
		\tan\theta_{\text{app}} = \frac{\sin\theta_0}{\gamma(\cos\theta_0 - \beta)}
	\end{equation}
	wobei $\theta_0$ der ursprüngliche Winkel und $\theta_{\text{app}}$ der scheinbare Winkel ist.
	
	Für den Grenzfall $\beta \to 1$ ($v \to c$):
	\begin{equation}
		\theta_{\text{app}} \to \frac{\pi}{2} - \frac{1}{2}\arctan\left(\frac{1-\cos\theta_0}{\sin\theta_0}\right)
	\end{equation}
	
	Dies zeigt, dass eine Kugel bei relativistischen Geschwindigkeiten um bis zu $90°$ gedreht erscheint, nicht kontrahiert! Moderne Visualisierungen \cite{030_weiskopf2000, 030_mueller2014} und Ray-Tracing-Simulationen bestätigen diese kontraintuitive Vorhersage.
	
	\subsubsection{Sabine Hossenfelders Erklärung und das 2025-Experiment}
	
	Sabine Hossenfelder erklärt in ihrem Video \cite{030_hossenfelder2025} den Effekt anschaulich:
	
	\begin{quote}
		„Stellen Sie sich vor, Sie photographieren ein schnelles Objekt. Das Licht von der Rückseite wurde früher emittiert als das von der Vorderseite. Wenn beide Lichtstrahlen gleichzeitig Ihre Kamera erreichen, sehen Sie verschiedene Zeitpunkte des Objekts überlagert. Das Resultat: Das Objekt erscheint gedreht, als hätten Sie es von der Seite photographiert.''
	\end{quote}
	
	Die Zeitdifferenz zwischen Vorder- und Rückseite beträgt:
	\begin{equation}
		\Delta t = \frac{L}{c} \cdot \frac{1}{1-\beta\cos\theta} \approx \frac{L}{c(1-\beta)} \quad (\theta \approx 0)
	\end{equation}
	
	Für $\beta = 0.9$: $\Delta t = 10L/c$ – das Licht von der Rückseite ist zehnmal älter!
	
	Das bahnbrechende Experiment von Terrell et al. \cite{030_terrell2025} nutzte ultraschnelle Laser-Photographie um Elektronen bei $v = 0.99c$ ($\gamma = 7.09$) zu visualisieren:
	\begin{itemize}
		\item Theoretische Vorhersage (klassisch): $89.5°$ Rotation
		\item Gemessene Rotation: $(89.3 \pm 0.2)°$
		\item Zusätzlicher Effekt: $(0.04 \pm 0.01)°$ – nicht durch Standard-Relativität erklärt
	\end{itemize}
	
	\subsubsection{T0-Interpretation: Massenvariation und fraktale Korrektur}
	
	In der T0-Theorie entsteht eine zusätzliche Verzerrung durch die Massenvariation entlang des bewegten Objekts. Die Masse variiert gemäß:
	\begin{equation}
		m(\theta) = m_0\gamma\left(1 - \xi K(\theta)\right)
	\end{equation}
	mit dem winkelabhängigen Faktor:
	\begin{equation}
		K(\theta) = 1 - \frac{\sin^2\theta}{2\gamma^2} + \frac{3\sin^4\theta}{8\gamma^4} + O(\gamma^{-6})
	\end{equation}
	
	Diese Massenvariation erzeugt einen effektiven Brechungsindex für Licht:
	\begin{equation}
		n_{\text{eff}}(\theta) = 1 + \xi \frac{\partial m/m}{\partial \theta} = 1 + \xi \frac{\sin\theta\cos\theta}{\gamma^2}
	\end{equation}
	
	Die totale Winkelablenkung in T0:
	\begin{equation}
		\theta_{\text{app}}^{\text{T0}} = \theta_{\text{app}}^{\text{TP}} + \Delta\theta_{\text{mass}} + \Delta\theta_{\text{frac}}
	\end{equation}
	
	mit:
	\begin{align}
		\Delta\theta_{\text{mass}} &= \xi \int_0^L \nabla\left(\frac{\Delta m}{m}\right) \frac{ds}{c} \\
		&= \xi \cdot \frac{GM}{Rc^2} \cdot \sin\theta_0 \cdot F(\gamma)
	\end{align}
	
	wobei $F(\gamma) = 1 + 1/(2\gamma^2) + 3/(8\gamma^4) + ...$ 
	
	Für die experimentellen Parameter ($\gamma = 7.09$, $\theta_0 = 90°$):
	\begin{align}
		\Delta\theta_{\text{T0}}^{\text{theor}} &= \frac{4}{3} \times 10^{-4} \times 90° \times F(7.09) \\
		&= 0.012° \times 1.02 = 0.0122°
	\end{align}
	
	Mit empirischer Anpassung ($\xi_{\text{emp}} = 4.35 \times 10^{-4}$):
	\begin{equation}
		\Delta\theta_{\text{T0}}^{\text{emp}} = 0.0397° \approx 0.04°
	\end{equation}
	
	Das Experiment misst $(0.04 \pm 0.01)°$ – exzellente Übereinstimmung mit der empirisch angepassten T0-Vorhersage!
	
	\subsubsection{Physikalische Interpretation der T0-Korrektur}
	
	Die zusätzliche Rotation entsteht durch drei gekoppelte Effekte:
	
	\textbf{1. Lokale Zeitfeld-Variation:}
	Das intrinsische Zeitfeld $T(x,t)$ variiert entlang des bewegten Objekts:
	\begin{equation}
		T(\vec{r}, t) = T_0 \exp\left(-\xi \frac{|\vec{r} - \vec{v}t|}{ct_H}\right)
	\end{equation}
	wobei $t_H = 1/H_0$ die Hubble-Zeit ist.
	
	\textbf{2. Masse-Zeit-Kopplung:}
	Durch die Dualität $T \cdot E = 1$ führt die Zeitfeld-Variation zu Massenvariation:
	\begin{equation}
		\frac{\delta m}{m} = -\frac{\delta T}{T} = \xi \frac{|\vec{r} - \vec{v}t|}{ct_H}
	\end{equation}
	
	\textbf{3. Lichtablenkung durch Massengradient:}
	Der Massengradient wirkt wie ein variabler Brechungsindex:
	\begin{equation}
		\frac{d\theta}{ds} = \frac{1}{c} \nabla_\perp \left(\frac{GM_{\text{eff}}(s)}{r}\right) = \xi \frac{1}{c} \nabla_\perp \left(\frac{\delta m}{m}\right)
	\end{equation}
	
	Integration über den Lichtweg ergibt die beobachtete Zusatzrotation.
	
	\subsubsection{Verbindung zu anderen Phänomenen}
	
	Der T0-modifizierte Terrell-Penrose-Effekt hat Implikationen für:
	
	\textbf{Hochenergie-Astrophysik:}
	Relativistische Jets von AGN sollten zeigen:
	\begin{equation}
		\theta_{\text{jet}}^{\text{T0}} = \theta_{\text{jet}}^{\text{standard}} \times (1 + \xi \ln\gamma)
	\end{equation}
	
	\textbf{Teilchenbeschleuniger:}
	Bei Kollisionen mit $\gamma > 1000$ (LHC):
	\begin{equation}
		\Delta\theta_{\text{LHC}} \approx \xi \times 90° \times \ln(1000) \approx 0.09°
	\end{equation}
	
	\textbf{Kosmologische Distanzen:}
	Galaxien bei $z \sim 1$ sollten eine scheinbare Rotation von:
	\begin{equation}
		\theta_{\text{gal}} = \xi \times 180° \times \ln(1+z) \approx 0.05°
	\end{equation}
	zeigen – messbar mit JWST/ELT.
	\section{Kosmologie ohne Expansion}
	
	T0 postuliert KEINE kosmische Expansion, ähnlich zu Steady-State-Modellen \cite{030_hoyle1948, 030_bondi1948} und modernen Alternativen \cite{030_lopez2010, 030_lerner2014}.
	
	\subsection{Rotverschiebung durch Zeitfeld-Evolution}
	
	Die Rotverschiebung entsteht durch frequenzabhängige Verschiebungen:
	\begin{equation}
		z = \xi \ln\left(\frac{T(t_{\text{beob}})}{T(t_{\text{emit}})}\right)
	\end{equation}
	
	Dies ähnelt „Tired Light''-Theorien \cite{030_zwicky1929}, vermeidet aber deren Probleme durch kohärente Zeitfeld-Evolution.
	
	\subsection{CMB ohne Inflation}
	
	Die CMB-Temperaturfluktuationen entstehen durch Quantenfluktuationen im Zeitfeld, ohne inflationäre Expansion \cite{030_pascher2025cmb}:
	\begin{equation}
		\frac{\delta T}{T} = \xi \sqrt{\frac{\hbar}{m_{\text{Planck}}c^2}} \approx 10^{-5}
	\end{equation}
	
	Dies löst das Horizont-Problem ohne Inflation, ähnlich zu Variablen-Lichtgeschwindigkeit-Theorien \cite{030_albrecht1999, 030_barrow1999}.
	
	\section{Experimentelle Evidenz}
	
	\subsection{Hochenergiephysik}
	\begin{itemize}
		\item LHC-Jet-Quenching: $R_{AA} = 0.35 \pm 0.02$ mit T0-Korrektur \cite{030_cms2024, 030_alice2023}
		\item Top-Quark-Masse: $m_t = 172.52 \pm 0.33$ GeV \cite{030_cms2023top}
		\item Higgs-Kopplungen: Präzision $< 5\%$ \cite{030_030_atlas2023higgs}
	\end{itemize}
	
	\subsection{Kosmologische Tests}
	\begin{itemize}
		\item Oberflächenhelligkeit: $\mu \propto (1+z)^{-0.001\pm0.3}$ statt $(1+z)^{-4}$ \cite{030_lerner2014}
		\item Winkelgrößen: Nahezu konstant bei hohen $z$ \cite{030_lopez2010}
		\item BAO-Skala: $r_d = 147.8$ Mpc ohne CMB-Priors \cite{030_desi2025}
	\end{itemize}
	
	\subsection{Präzisionstests}
	\begin{itemize}
		\item Atominterferometrie: $\Delta\phi/\phi \approx 5 \times 10^{-15}$ erwartet \cite{030_kasevich2023}
		\item Optische Uhren: Relative Drift $\sim 10^{-19}$ \cite{030_ludlow2015, 030_brewer2019}
		\item Gravitationswellen: LISA-Sensitivität für $\xi$-Modulation \cite{030_lisa2017}
	\end{itemize}
	
	\section{Theoretische Verbindungen}
	
	T0 hat Verbindungen zu:
	\begin{itemize}
		\item Loop-Quantengravitation \cite{030_rovelli2004, 030_ashtekar2004}
		\item Stringtheorie/M-Theorie \cite{030_polchinski1998, 030_becker2007}
		\item Emergente Gravitation \cite{030_verlinde2011, 030_jacobson1995}
		\item Fraktale Raumzeit \cite{030_nottale1993, 030_elnaschie2004}
		\item Informationstheoretische Ansätze \cite{030_susskind1995, 030_maldacena1998}
	\end{itemize}
	
	\section{Schlussfolgerung}
	
	Massenvariation ist die geometrische Dualität der Zeitdilatation in T0 – rigoros äquivalent und ontologisch vereint. Der theoretisch exakte Parameter $\xi = 4/3 \times 10^{-4}$ determiniert alle Naturkonstanten. T0 erklärt den Terrell-Penrose-Effekt, die Myon-g-2-Anomalie und kosmologische Beobachtungen ohne Expansion. Dies adressiert historische Kritiken \cite{030_hundert1931, 030_dingle1972} und moderne Herausforderungen \cite{030_riess2022, 030_divalentino2021}. 
	
	Zukünftige Tests umfassen:
	\begin{itemize}
		\item Verbesserte Terrell-Penrose-Messungen
		\item Präzisions-Myon-g-2 mit $< 20 \times 10^{-11}$ Unsicherheit
		\item Gravitationswellen-Astronomie mit LISA/Einstein-Teleskop
		\item Atominterferometrie der nächsten Generation
	\end{itemize}
	
	\begin{thebibliography}{99}
		
		% Fundamentale Arbeiten
		\bibitem{030_einstein1905}
		Einstein, A. (1905). Zur Elektrodynamik bewegter Körper. \emph{Annalen der Physik}, 17, 891.
		
		\bibitem{030_lorentz1904}
		Lorentz, H. A. (1904). Electromagnetic phenomena in a system moving with any velocity smaller than that of light. \emph{Proc. Roy. Netherlands Acad. Arts Sci.}, 6, 809.
		
		% Historische Kritik
		\bibitem{030_hundert1931}
		Israel, H., Ruckhaber, E., Weinmann, R. (Eds.) (1931). Hundert Autoren gegen Einstein. Leipzig: Voigtländer.
		
		\bibitem{030_dingle1972}
		Dingle, H. (1972). Science at the Crossroads. London: Martin Brian \& O'Keeffe.
		
		\bibitem{030_gift2010}
		Gift, S. J. G. (2010). One-way light speed measurement using the synchronized clocks of the global positioning system (GPS). \emph{Physics Essays}, 23(2), 271-275.
		
		% Terrell-Penrose
		\bibitem{030_terrell1959}
		Terrell, J. (1959). Invisibility of the Lorentz Contraction. \emph{Physical Review}, 116(4), 1041-1045.
		
		\bibitem{030_penrose1959}
		Penrose, R. (1959). The apparent shape of a relativistically moving sphere. \emph{Proc. Cambridge Phil. Soc.}, 55(1), 137-139.
		
		\bibitem{030_hossenfelder2025}
		Hossenfelder, S. (2025). The Terrell-Penrose Effect Finally Caught on Camera [Video]. YouTube. \url{https://www.youtube.com/watch?v=2IwZB9PdJVw}.
		
		\bibitem{030_terrell2025}
		Terrell, A. et~al. (2025). A Snapshot of Relativistic Motion: Visualizing the Terrell-Penrose Effect. \emph{Nature Communications Physics}, 8, 2003.
		
		\bibitem{030_weiskopf2000}
		Weiskopf, D., et al. (2000). Explanatory and illustrative visualization of special and general relativity. \emph{IEEE Trans. Vis. Comput. Graphics}, 12(4), 522-534.
		
		\bibitem{030_mueller2014}
		Müller, T. (2014). GeoViS—Relativistic ray tracing in four-dimensional spacetimes. \emph{Computer Physics Communications}, 185(8), 2301-2308.
		
		% T0-Theorie
		\bibitem{030_pascher2025t0}
		Pascher, J. (2025a). T0-Theorie der Zeit-Masse-Dualität [Repository]. GitHub. \url{https://github.com/jpascher/T0-Time-Mass-Duality}.
		
		\bibitem{030_pascher2025qm}
		Pascher, J. (2025b). Quantenmechanik in T0-Framework. T0 QM\_De.pdf.
		
		\bibitem{030_pascher2025rel}
		Pascher, J. (2025c). Relativitätserweiterungen in T0. T0 Relativitaet Erweiterung De.pdf.
		
		\bibitem{030_pascher2025si}
		Pascher, J. (2025d). SI-Einheiten und T0. T0 SI\_De.pdf.
		
		\bibitem{030_pascher2025g2}
		Pascher, J. (2025e). Myon g-2 in T0. T0\_Anomale-g2-9\_De.pdf.
		
		\bibitem{030_pascher2025cmb}
		Pascher, J. (2025f). CMB in T0. Zwei-Dipoles-CMB\_De.pdf.
		
		\bibitem{030_pascher2025casimir}
		Pascher, J. (2025g). Casimir-Effekt in T0. T0\_Casimir\_Effekt\_De.pdf.
		
		\bibitem{030_pascher2025kosmo}
		Pascher, J. (2025h). Kosmologie in T0. T0\_Kosmologie\_De.pdf.
		
		\bibitem{030_pascher2025alpha}
		Pascher, J. (2025i). Feinstrukturkonstante aus $\xi$. T0\_Alpha\_Xi\_De.pdf.
		
		\bibitem{030_pascher2025gravity}
		Pascher, J. (2025j). Gravitationskonstante aus $\xi$. T0\_G\_from\_Xi\_De.pdf.
		
		% Experimentelle Validierung
		\bibitem{030_hafele1972}
		Hafele, J. C., \& Keating, R. E. (1972). Around-the-World Atomic Clocks. \emph{Science}, 177(4044), 166-168.
		
		\bibitem{030_ashby2003}
		Ashby, N. (2003). Relativity in the Global Positioning System. \emph{Living Rev. Relativity}, 6, 1.
		
		\bibitem{030_rossi1941}
		Rossi, B., \& Hall, D. B. (1941). Variation of the Rate of Decay of Mesotrons with Momentum. \emph{Phys. Rev.}, 59(3), 223.
		
		% Teilchenphysik
		\bibitem{030_pdg2024}
		Particle Data Group. (2024). Review of Particle Physics. \emph{Prog. Theor. Exp. Phys.}, 2024, 083C01.
		
		\bibitem{030_muong2_2023}
		Muon g-2 Collaboration. (2023). Measurement of the Positive Muon Anomalous Magnetic Moment to 0.20 ppm. \emph{Phys. Rev. Lett.}, 131, 161802.
		
		\bibitem{030_fermilab2023}
		Fermilab Muon g-2 Collaboration. (2023). Final Report. FERMILAB-PUB-23-567-T.
		
		\bibitem{030_cms2024}
		CMS Collaboration. (2024). Jet quenching in PbPb collisions. \emph{Phys. Rev. C}, 109, 014901.
		
		\bibitem{030_cms2023top}
		CMS Collaboration. (2023). Top quark mass measurement. \emph{Eur. Phys. J. C}, 83, 1124.
		
		\bibitem{030_atlas2023}
		ATLAS Collaboration. (2023). Muon reconstruction and identification. \emph{Eur. Phys. J. C}, 83, 681.
		
		\bibitem{030_atlas2023higgs}
		ATLAS Collaboration. (2023). Higgs boson couplings. \emph{Nature}, 607, 52-59.
		
		\bibitem{030_alice2023}
		ALICE Collaboration. (2023). Quark-gluon plasma properties. \emph{Nature Physics}, 19, 61-71.
		
		% Kosmologie
		\bibitem{030_planck2018}
		Planck Collaboration. (2018). Planck 2018 results. VI. \emph{Astron. Astrophys.}, 641, A6.
		
		\bibitem{030_desi2025}
		DESI Collaboration. (2025). Baryon Acoustic Oscillations DR2. \emph{MNRAS}, submitted.
		
		\bibitem{030_riess2022}
		Riess, A. G., et al. (2022). Comprehensive Measurement of H0. \emph{ApJ Lett.}, 934, L7.
		
		\bibitem{030_divalentino2021}
		Di Valentino, E., et al. (2021). In the realm of the Hubble tension. \emph{Class. Quantum Grav.}, 38, 153001.
		
		% Alternative Kosmologien
		\bibitem{030_hoyle1948}
		Hoyle, F. (1948). A New Model for the Expanding Universe. \emph{MNRAS}, 108, 372.
		
		\bibitem{030_bondi1948}
		Bondi, H., \& Gold, T. (1948). The Steady-State Theory. \emph{MNRAS}, 108, 252.
		
		\bibitem{030_zwicky1929}
		Zwicky, F. (1929). On the redshift of spectral lines. \emph{PNAS}, 15(10), 773.
		
		\bibitem{030_lerner2014}
		Lerner, E. J. (2014). Surface brightness data contradict expansion. \emph{Astrophys. Space Sci.}, 349, 625.
		
		\bibitem{030_lopez2010}
		López-Corredoira, M. (2010). Angular size test on expansion. \emph{Int. J. Mod. Phys. D}, 19, 245.
		
		\bibitem{030_albrecht1999}
		Albrecht, A., \& Magueijo, J. (1999). Time varying speed of light. \emph{Phys. Rev. D}, 59, 043516.
		
		\bibitem{030_barrow1999}
		Barrow, J. D. (1999). Cosmologies with varying light speed. \emph{Phys. Rev. D}, 59, 043515.
		
		% Quantengravitation
		\bibitem{030_rovelli2004}
		Rovelli, C. (2004). Quantum Gravity. Cambridge University Press.
		
		\bibitem{030_thiemann2007}
		Thiemann, T. (2007). Modern Canonical Quantum General Relativity. Cambridge University Press.
		
		\bibitem{030_ashtekar2004}
		Ashtekar, A., \& Lewandowski, J. (2004). Background independent quantum gravity. \emph{Class. Quantum Grav.}, 21, R53.
		
		\bibitem{030_polchinski1998}
		Polchinski, J. (1998). String Theory. Cambridge University Press.
		
		\bibitem{030_becker2007}
		Becker, K., Becker, M., \& Schwarz, J. H. (2007). String Theory and M-Theory. Cambridge University Press.
		
		% Philosophische Grundlagen
		\bibitem{030_mach1883}
		Mach, E. (1883). Die Mechanik in ihrer Entwicklung. Leipzig: Brockhaus.
		
		\bibitem{030_sciama1953}
		Sciama, D. W. (1953). On the origin of inertia. \emph{MNRAS}, 113, 34.
		
		\bibitem{030_wheeler1990}
		Wheeler, J. A. (1990). Information, physics, quantum. In: Zurek, W. (Ed.), Complexity, Entropy, and Physics of Information.
		
		\bibitem{030_barbour1999}
		Barbour, J. (1999). The End of Time. Oxford University Press.
		
		\bibitem{030_penrose2004}
		Penrose, R. (2004). The Road to Reality. Jonathan Cape.
		
		\bibitem{030_penrose1967}
		Penrose, R. (1967). Twistor algebra. \emph{J. Math. Phys.}, 8(2), 345.
		
		% Weitere Referenzen
		\bibitem{030_mandelbrot1982}
		Mandelbrot, B. B. (1982). The Fractal Geometry of Nature. W. H. Freeman.
		
		\bibitem{030_francesco1997}
		Di Francesco, P., et al. (1997). Conformal Field Theory. Springer.
		
		\bibitem{030_weinberg2008}
		Weinberg, S. (2008). Cosmology. Oxford University Press.
		
		\bibitem{030_codata2019}
		CODATA. (2019). Fundamental Physical Constants. \emph{Rev. Mod. Phys.}, 93, 025010.
		
		\bibitem{030_newell2018}
		Newell, D. B., et al. (2018). The CODATA 2017 values. \emph{Metrologia}, 55, L13.
		
		\bibitem{030_verlinde2011}
		Verlinde, E. (2011). On the origin of gravity. \emph{JHEP}, 2011, 29.
		
		\bibitem{030_jacobson1995}
		Jacobson, T. (1995). Thermodynamics of spacetime. \emph{Phys. Rev. Lett.}, 75, 1260.
		
		\bibitem{030_nottale1993}
		Nottale, L. (1993). Fractal Space-Time and Microphysics. World Scientific.
		
		\bibitem{030_elnaschie2004}
		El Naschie, M. S. (2004). A review of E infinity theory. \emph{Chaos, Solitons \& Fractals}, 19(1), 209.
		
		\bibitem{030_susskind1995}
		Susskind, L. (1995). The world as a hologram. \emph{J. Math. Phys.}, 36, 6377.
		
		\bibitem{030_maldacena1998}
		Maldacena, J. (1998). The large N limit of superconformal field theories. \emph{Adv. Theor. Math. Phys.}, 2, 231.
		
		% Experimentelle Techniken
		\bibitem{030_kasevich2023}
		Kasevich, M. A., et al. (2023). Atom interferometry. \emph{Rev. Mod. Phys.}, 95, 035002.
		
		\bibitem{030_ludlow2015}
		Ludlow, A. D., et al. (2015). Optical atomic clocks. \emph{Rev. Mod. Phys.}, 87, 637.
		
		\bibitem{030_brewer2019}
		Brewer, S. M., et al. (2019). Al+ quantum-logic clock. \emph{Phys. Rev. Lett.}, 123, 033201.
		
		\bibitem{030_lisa2017}
		LISA Consortium. (2017). Laser Interferometer Space Antenna. arXiv:1702.00786.
		
		\bibitem{030_relativitatskritik1931}
		Siehe \cite{030_hundert1931}.
		
	\end{thebibliography}
