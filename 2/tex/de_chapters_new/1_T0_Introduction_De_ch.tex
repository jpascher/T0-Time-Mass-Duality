% Chapter file generated from 1_T0_Introduction_De.tex
\chapter{1_T0_Introduction_De}

\hfuzz=200pt
	
	\chapter*{Einleitung}
	\addcontentsline{toc}{chapter}{Einleitung}
	
	Dieses Buch stellt den aktuellen Stand des T0-Zeit--Masse-Dualitätsrahmens und seine Anwendungen auf
	Teilchenmassen, fundamentale Konstanten, Quantenmechanik, Gravitation und Kosmologie dar.
	
	Der Hauptteil des Buches besteht aus einer Reihe von Kern-Dokumenten zu T0. Diese Kapitel spiegeln das
	gegenwärtige Verständnis der Theorie und ihrer quantitativen Konsequenzen wider. Soweit möglich wurde
	das Material umorganisiert und vereinheitlicht, damit die Struktur der Theorie so transparent
	wie möglich wird.
	
	Am Ende des Buches sind mehrere ältere Dokumente in einem Anhang enthalten. Diese Texte repräsentieren
	frühere Stadien der Entwicklung des T0-Rahmens. Sie wurden nicht entfernt, weil sie die
	Evolution der Ideen und die Verfeinerung der Formeln sichtbar machen. In vielen Fällen kann man
	sehen, wie Approximationen verbessert, spezielle Fälle verallgemeinert und wie neue empirische Daten
	geholfen haben, frühere Argumente zu schärfen oder zu korrigieren.
	
	Die „live“-Version der Theorie wird in einem öffentlichen GitHub-Repository gepflegt:
	
	\begin{center}
	\end{center}
	
	Die LaTeX-Quellen der Kapitel in diesem Buch stammen aus diesem Repository. Wenn konzeptionelle oder
	numerische Fehler gefunden werden, werden sie dort zuerst korrigiert. Das bedeutet, dass die PDF-Version des
	Buches, die Sie lesen, ein Snapshot eines kontinuierlich evolvierenden Projekts ist. Für die aktuellste Version
	der Dokumente, einschließlich neuer Anhänge oder Korrekturen, sollte das GitHub-Repository immer als
	primäre Referenz betrachtet werden.
	
	Die Absicht dieser Kompilation ist zweifach:
	\begin{itemize}
		\item eine kohärente, lesbare Pfad durch die Kernideen und -ergebnisse des T0-Rahmens bereitzustellen;
		\item die historische Entwicklung dieser Ideen im Anhang zu dokumentieren, einschließlich falscher
		Starts, intermediärer Formulierungen und früher Anpassungen an experimentelle Daten.
	\end{itemize}
	
	Leser, die hauptsächlich an der aktuellen Formulierung der Theorie interessiert sind, können sich auf die Kern-
	kapitel konzentrieren. Leser, die auch am Denken und Trial--and--Error-Prozess hinter der Theorie interessiert
	sind, werden eingeladen, das Anhangmaterial parallel zu studieren.
