% Chapter file generated from 071_QM-Detrmistic_De.tex
\chapter{Deterministische Quantenmechanik via T0-Energiefeld-Formulierung: \\
		Von wahrscheinlichkeitsbasierter zu verhaeltnisbasierter Mikrophysik \\
		\large Aufbauend auf der T0-Revolution: Vereinfachte Dirac-Gleichung, universelle Lagrange-Dichte und Verhaeltnis-Physik\\
		\textbf{}

}
	}
	

	\section*{Abstract}
		Diese Arbeit praesentiert eine revolutionaere deterministische Alternative zur wahrscheinlichkeitsbasierten Quantenmechanik durch die T0-Energiefeld-Formulierung. Aufbauend auf der vereinfachten Dirac-Gleichung, universellen Lagrange-Dichte und verhaeltnisbasierten Physik des T0-Rahmenwerks zeigen wir, wie quantenmechanische Phaenomene aus deterministischer Energiefeld-Dynamik entstehen, die durch die modifizierte Schroedinger-Gleichung regiert wird. Mit dem empirisch bestimmten Parameter $\xipar = 4/3 \times 10^{-4}$ liefern wir quantitative Vorhersagen, die alle experimentell verifizierten Ergebnisse bewahren und gleichzeitig fundamentale Interpretationsprobleme eliminieren.
	

	\section{Einleitung: Die auf die Quantenmechanik angewandte T0-Revolution}
	
	\subsection{Aufbauend auf T0-Grundlagen}
	
	Diese Arbeit repraesentiert die vierte Stufe der theoretischen T0-Revolution:
	
	\textbf{Stufe 1 - Vereinfachte Dirac-Gleichung}: Komplexe $4 \times 4$-Matrizen zu einfacher Felddynamik
	
	\textbf{Stufe 2 - Universelle Lagrange-Dichte}: Mehr als 20 Felder zu einer Gleichung
	
	\textbf{Stufe 3 - Verhaeltnis-Physik}: Mehrere Parameter zu Energieskala-Verhaeltnissen
	
	\textbf{Stufe 4 - Deterministische QM}: Wahrscheinlichkeitsamplituden zu deterministischen Energiefeldern
	
	\subsection{Das Quantenmechanik-Problem}
	
	Die Standard-Quantenmechanik leidet unter fundamentalen konzeptionellen Problemen:
	
	\begin{tcolorbox}[colback=red!5!white,colframe=red!75!black,title=Standard-QM-Probleme]
		\textbf{Wahrscheinlichkeits-Fundament-Probleme}:
		\begin{itemize}
			\item Wellenfunktion: mysterioese Superposition
			\item Wahrscheinlichkeiten: nur statistische Vorhersagen
			\item Kollaps: Nicht-unitaerer Messprozess
			\item Interpretation: Kopenhagen vs. Viele-Welten vs. andere
			\item Einzelmessungen: Unvorhersagbar (fundamental zufaellig)
		\end{itemize}
	\end{tcolorbox}
	
	\subsection{T0-Energiefeld-Loesung}
	
	Das T0-Rahmenwerk bietet eine vollstaendige Loesung durch deterministische Energiefelder:
	
	\begin{tcolorbox}[colback=blue!5!white,colframe=blue!75!black,title=T0-Deterministisches Fundament]
		\textbf{Deterministische Energiefeld-Physik}:
		\begin{itemize}
			\item Universelles Feld: einzelnes Energiefeld fuer alle Phaenomene
			\item Modifizierte Schroedinger-Gleichung mit Zeit-Energie-Dualitaet
			\item Empirischer Parameter: $\xipar = 4/3 \times 10^{-4}$ aus Myon-Anomalie
			\item Messbare Abweichungen von Standard-QM
			\item Kontinuierliche Evolution: Kein Kollaps, nur Felddynamik
			\item Einzige Realitaet: Keine Interpretationsprobleme
		\end{itemize}
	\end{tcolorbox}
	
	\section{T0-Energiefeld-Grundlagen}
	
	\subsection{Modifizierte Schroedinger-Gleichung}
	
	Aus der T0-Revolution wird die Quantenmechanik regiert durch:
	
	\begin{equation}
		\boxed{i \cdot T(x,t) \frac{\partial\psi}{\partial t} = H_0 \psi + V_{\mathrm{T0}} \psi}
		\label{071_eq:modifizierte_schroedinger}
	\end{equation}
	
	wobei:
	\begin{align}
		H_0 &= -\frac{\hbar^2}{2m} \nabla^2 \\
		V_{\mathrm{T0}} &= \hbar^2 \cdot \delta E(x,t)
	\end{align}
	
	\subsection{Energie-Zeit-Dualitaet}
	
	Die fundamentale T0-Beziehung:
	
	\begin{equation}
		\boxed{T(x,t) \cdot E(x,t) = 1}
		\label{071_eq:energie_zeit_dualitaet}
	\end{equation}
	
	\textbf{Dimensionale Verifikation}: $[T][E] = 1$ in natuerlichen Einheiten.
	
	\subsection{Empirischer Parameter}
	
	Folgend den Praezisionsmessungen des anomalen magnetischen Moments des Myons:
	
	\begin{equation}
		\boxed{\xipar = \frac{4}{3} \times 10^{-4} \approx 1{,}333 \times 10^{-4}}
		\label{071_eq:empirischer_parameter}
	\end{equation}
	
	\section{Von Wahrscheinlichkeitsamplituden zu Energiefeld-Verhaeltnissen}
	
	\subsection{Standard-QM-Zustandsbeschreibung}
	
	\textbf{Traditioneller Ansatz}:
	\begin{equation}
		|\psi\rangle = \sum_i c_i |i\rangle \quad \text{mit } P_i = |c_i|^2
	\end{equation}
	
	\textbf{Probleme}: Mysterioese Superposition, nur wahrscheinlichkeitsbasierte Vorhersagen.
	
	\subsection{T0-Energiefeld-Zustandsbeschreibung}
	
	\textbf{T0-feldtheoretischer Ansatz}:
	\begin{equation}
		\boxed{\psi(x,t) = \sqrt{\frac{\delta E(x,t)}{E_0 V_0}} \cdot e^{i\phi(x,t)}}
		\label{071_eq:wellenfunktion_feld}
	\end{equation}
	
	mit Wahrscheinlichkeitsdichte:
	\begin{equation}
		\boxed{|\psi(x,t)|^2 = \frac{\delta E(x,t)}{E_0 V_0}}
		\label{071_eq:wahrscheinlichkeitsdichte}
	\end{equation}
	
	\textbf{Vorteile}: 
	\begin{itemize}
		\item Direkte Verbindung zu messbarer Energiefeld-Dichte
		\item Deterministische Feld-Evolution durch modifizierte Schroedinger-Gleichung
		\item Erhaltung der wahrscheinlichkeitsbasierten Interpretation mit T0-Korrekturen
		\item Feldtheoretisches Fundament fuer Quantenmechanik
	\end{itemize}
	
	\section{Deterministische Spin-Systeme}
	
	\subsection{Spin-1/2 in T0-Formulierung}
	
	\subsubsection{Standard-QM-Ansatz}
	
	\textbf{Zustand}: Superposition von Spin-up und Spin-down
	
	\textbf{Erwartungswert}: Wahrscheinlichkeitsbasiert
	
	\subsubsection{T0-Energiefeld-Ansatz}
	
	\textbf{Zustand}: Energiefeld-Konfiguration mit separaten Feldern fuer beide Spin-Zustaende
	
	\textbf{T0-korrigierter Erwartungswert}:
	\begin{equation}
		\boxed{\langle \sigma_z \rangle_{\mathrm{T0}} = \langle \sigma_z \rangle_{\mathrm{QM}} + \xipar \cdot \frac{\delta E(x,t)}{E_0}}
		\label{071_eq:korrigierter_spin_z}
	\end{equation}
	
	\subsection{Quantitatives Beispiel}
	
	Mit dem empirischen Parameter $\xipar = 4/3 \times 10^{-4}$:
	
	\textbf{T0-Korrektur zum Erwartungswert}:
	\begin{equation}
		\langle \sigma_z \rangle_{\mathrm{T0}} = \langle \sigma_z \rangle_{\mathrm{QM}} + \frac{4}{3} \times 10^{-4} \times \delta\sigma_z
	\end{equation}
	
	\section{Deterministische Quantenverschraenkung}
	
	\subsection{Standard-QM-Verschraenkung}
	
	\textbf{Bell-Zustand}: Antisymmetrische Superposition
	
	\textbf{Problem}: Nicht-lokale spukhafte Fernwirkung
	
	\subsection{T0-Energiefeld-Verschraenkung}
	
	\textbf{Verschraenkung als korrelierte Energiefeld-Struktur}:
	\begin{equation}
		\boxed{E_{12}(x_1, x_2, t) = E_1(x_1, t) + E_2(x_2, t) + E_{\mathrm{korr}}(x_1, x_2, t)}
	\end{equation}
	
	\textbf{Korrelations-Energiefeld}:
	\begin{equation}
		\boxed{E_{\mathrm{korr}}(x_1, x_2, t) = \frac{\xipar}{|x_1 - x_2|} \cos(\phi_1(t) - \phi_2(t) - \pi)}
		\label{071_eq:korrelationsfeld}
	\end{equation}
	
	\subsection{Modifizierte Bell-Ungleichung}
	
	Das T0-Modell sagt eine modifizierte Bell-Ungleichung vorher:
	
	\begin{equation}
		\boxed{|E(a,b) - E(a,c)| + |E(a',b) + E(a',c)| \leq 2 + \varepsilon_{\mathrm{T0}}}
	\end{equation}
	
	mit dem T0-Term:
	\begin{equation}
		\boxed{\varepsilon_{\mathrm{T0}} = \xipar \cdot \frac{2\langle E \rangle \ell_P}{r_{12}}}
		\label{071_eq:bell_korrektur}
	\end{equation}
	
	\textbf{Numerische Abschaetzung}:
	Fuer typische atomare Systeme mit $r_{12} \sim 1$ m:
	\begin{equation}
		\varepsilon_{\mathrm{T0}} \approx 10^{-34}
	\end{equation}
	
	\section{Deterministisches Quantencomputing}
	
	\subsection{Qubit-Darstellung}
	
	\textbf{T0-Energiefeld-Qubit}:
	\begin{equation}
		\boxed{\text{qubit}_{\mathrm{T0}} \equiv \{E_0(x,t), E_1(x,t)\}}
	\end{equation}
	
	mit feldtheoretischen Amplituden:
	\begin{align}
		\alpha_{\mathrm{T0}} &= \sqrt{\frac{E_0}{E_0 + E_1}} \\
		\beta_{\mathrm{T0}} &= \sqrt{\frac{E_1}{E_0 + E_1}}
	\end{align}
	
	\subsection{Quantengatter als Energiefeld-Operationen}
	
	\subsubsection{Hadamard-Gatter}
	
	\textbf{Korrigierte T0-Transformation}:
	\begin{align}
		H_{\mathrm{T0}}: \quad E_0 &\rightarrow \frac{E_0 + E_1}{\sqrt{2}} \\
		E_1 &\rightarrow \frac{E_0 - E_1}{\sqrt{2}}
	\end{align}
	
	\subsubsection{Kontrolliertes-NICHT-Gatter}
	
	\textbf{T0-Formulierung}:
	\begin{equation}
		\text{CNOT}_{\mathrm{T0}}: E_{12} \rightarrow E_{12} + \xipar \cdot \Theta(E_1 - E_{\mathrm{Schwelle}}) \cdot \sigma_x E_2
	\end{equation}
	
	\subsection{Erweiterte Quanten-Algorithmen}
	
	\textbf{Erweiterter Grover-Algorithmus}:
	\begin{itemize}
		\item Standard-Iterationen: $\sim \pi/(4\sqrt{N})$
		\item T0-erweitert: Modifikation durch Energiefeld-Korrekturen
	\end{itemize}
	
	\section{Experimentelle Vorhersagen und Tests}
	
	\subsection{Erweiterte Einzelmessungs-Vorhersagen}
	
	\textbf{Beispiel - Erweiterte Spin-Messung}:
	\begin{equation}
		\boxed{P(\uparrow) = P_{\mathrm{QM}}(\uparrow) \cdot \left(1 + \xipar \frac{E_{\uparrow}(x_{\mathrm{det}}, t) - \langle E \rangle}{E_0}\right)}
		\label{071_eq:erweiterte_messung}
	\end{equation}
	
	\subsection{T0-spezifische experimentelle Signaturen}
	
	\subsubsection{Modifizierte Bell-Tests}
	
	\textbf{Vorhersage}: Bell-Ungleichungs-Verletzung modifiziert um $\varepsilon_{\mathrm{T0}} \approx 10^{-34}$
	
	\subsubsection{Energiefeld-Spektroskopie}
	
	\textbf{Vorhersage}: 
	\begin{equation}
		\Delta E = \xipar \cdot E_n \cdot \frac{\langle \delta E \rangle}{E_0}
	\end{equation}
	
	\subsubsection{Phasen-Akkumulation in Interferometrie}
	
	\textbf{Vorhersage}:
	\begin{equation}
		\phi_{\mathrm{gesamt}} = \phi_0 + \xipar \int_0^t \frac{E(x(t'), t')}{E_0} dt'
	\end{equation}
	
	\section{Aufloesung der Quanten-Interpretations-Probleme}
	
	\subsection{Durch T0-Formulierung adressierte Probleme}
	
	\begin{table}[htbp]
		\centering
		\small
		\begin{tabular}{|p{4cm}|p{5cm}|p{6cm}|}
			\hline
			\textbf{QM-Problem} & \textbf{Standard-Ansaetze} & \textbf{T0-Loesung} \\
			\hline
			Messproblem & Kopenhagener Interpretation & Kontinuierliche Feld-Evolution \\
			\hline
			Schroedingers Katze & Superpositions-Paradox & Definite Feld-Zustaende \\
			\hline
			Viele-Welten vs. Kopenhagen & Multiple Interpretationen & Einzige Realitaet \\
			\hline
			Welle-Teilchen-Dualitaet & Komplementaritaets-Prinzip & Energiefeld-Muster \\
			\hline
			Quanten-Spruenge & Zufaellige Uebergaenge & Feld-vermittelte Uebergaenge \\
			\hline
			Bell-Nichtlokalitaet & Spukhafte Fernwirkung & Feld-Korrelationen \\
			\hline
		\end{tabular}
		\caption{Durch T0-Formulierung adressierte Probleme}
	\end{table}
	
	\subsection{Erweiterte Quanten-Realitaet}
	
	\begin{tcolorbox}[colback=green!5!white,colframe=green!75!black,title=T0-Erweiterte Quanten-Realitaet]
		\textbf{Feldtheoretische Quantenmechanik mit T0-Korrekturen}:
		\begin{itemize}
			\item Energiefelder als physikalische Basis von Wellenfunktionen
			\item Modifizierte Schroedinger-Evolution mit Zeit-Energie-Dualitaet
			\item Messungen offenbaren Feld-Konfigurationen mit T0-Modulationen
			\item Kontinuierliche unitaere Evolution ohne Kollaps
			\item Kleine aber messbare Abweichungen von Standard-QM
			\item Empirisch begruendet durch Myon-Anomalie-Parameter
		\end{itemize}
	\end{tcolorbox}
	
	\section{Verbindung zu anderen T0-Entwicklungen}
	
	\subsection{Integration mit vereinfachter Dirac-Gleichung}
	
	Die erweiterte QM verbindet sich natuerlich mit der vereinfachten Dirac-Gleichung durch die Zeit-Energie-Dualitaet.
	
	\subsection{Integration mit universeller Lagrange-Dichte}
	
	Die universelle Lagrange-Dichte beschreibt:
	\begin{itemize}
		\item Klassische Feld-Evolution
		\item Quanten-Feld-Evolution mit T0-Korrekturen
		\item Relativistische Feld-Evolution
	\end{itemize}
	
	\section{Zukunftige Richtungen und Implikationen}
	
	\subsection{Experimentelles Verifikations-Programm}
	
	\textbf{Phase 1 - Praezisions-Tests}:
	\begin{itemize}
		\item Ultra-hohe Praezisions-Bell-Ungleichungs-Messungen
		\item Atom-Spektroskopie mit T0-Korrekturen
		\item Quanten-Interferometrie-Phasen-Messungen
	\end{itemize}
	
	\textbf{Phase 2 - Technologische Verbesserung}:
	\begin{itemize}
		\item T0-korrigierte Quantencomputing-Architekturen
		\item Erweiterte Quanten-Sensor-Protokolle
		\item Feld-korrelationsbasierte Quanten-Geraete
	\end{itemize}
	
	\subsection{Philosophische Implikationen}
	
	\begin{tcolorbox}[colback=purple!5!white,colframe=purple!75!black,title=Jenseits der Quanten-Mystik]
		\textbf{T0-erweiterte Quantenmechanik bietet}:
		\begin{itemize}
			\item Physikalisches Fundament durch Energiefeld-Theorie
			\item Messbare Abweichungen von reiner Zufaelligkeit
			\item Feldtheoretische Erklaerung von Quanten-Phaenomenen
			\item Empirische Begruendung durch Praezisions-Messungen
		\end{itemize}
		
		\textbf{Waehrend bewahrt wird}:
		\begin{itemize}
			\item Alle erfolgreichen Vorhersagen der Standard-QM
			\item Experimentelle Kontinuitaet mit etablierten Ergebnissen
			\item Mathematische Strenge und Konsistenz
		\end{itemize}
	\end{tcolorbox}
	
	\section{Schlussfolgerung: Die erweiterte Quanten-Revolution}
	
	\subsection{Revolutionaere Errungenschaften}
	
	Die T0-erweiterte Quanten-Formulierung hat erreicht:
	
	\begin{enumerate}
		\item \textbf{Physikalisches Fundament}: Energiefelder als Basis fuer Quantenmechanik
		\item \textbf{Experimentelle Konsistenz}: Alle Standard-QM-Vorhersagen erhalten
		\item \textbf{Messbare Korrekturen}: T0-spezifische Abweichungen fuer Tests
		\item \textbf{T0-Rahmenwerk Integration}: Konsistent mit anderen T0-Entwicklungen
		\item \textbf{Empirische Begruendung}: Parameter aus Praezisions-Messungen
		\item \textbf{Erweiterte Vorhersagekraft}: Neue testbare Effekte
	\end{enumerate}
	
	\subsection{Zukunftiger Einfluss}
	
	\begin{equation}
		\boxed{\text{Erweiterte QM} = \text{Standard-QM} + \text{T0-Feld-Korrekturen}}
	\end{equation}
	
	Die T0-Revolution erweitert die Quantenmechanik mit feldtheoretischen Fundamenten waehrend experimenteller Erfolg bewahrt wird.
	
	\begin{thebibliography}{99}
		\bibitem{pascher_dirac_2025}
		Pascher, J. (2025). \textit{Vereinfachte Dirac-Gleichung in der T0-Theorie}. GitHub Repository: T0-Time-Mass-Duality.
		
		\bibitem{bell1964}
		Bell, J.S. (1964). On the Einstein Podolsky Rosen Paradox. \textit{Physics Physique Fizika}, \textbf{1}, 195--200.
		
		\bibitem{myon_g2_2021}
		Muon g-2 Collaboration (2021). Measurement of the Positive Muon Anomalous Magnetic Moment to 0.46 ppm. \textit{Physical Review Letters}, \textbf{126}, 141801.
	\end{thebibliography}
