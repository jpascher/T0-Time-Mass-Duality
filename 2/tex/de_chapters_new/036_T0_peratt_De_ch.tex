% Chapter file: 036_T0_peratt_De_ch.tex
% Source: 036_T0_peratt_De.tex

\chapter{{Mathematische Konstrukte alternativer CMB-Modelle: Unnikrishnan und Peratt im Einklang mit der T0-Theorie}

\thispagestyle{fancy}
	\section*{Abstract}
		Basierend auf dem Video ``The CMB Power Spectrum -- Cosmology's Untouchable Curve?'' analysieren wir die mathematischen Grundlagen der alternativen Modelle von C. S. Unnikrishnan (kosmische Relativit\"atstheorie) und Anthony L. Peratt (Plasma-Kosmologie) detailliert. Unnikrishnans Feldgleichungen erweitern die Spezielle Relativit\"atstheorie um universelle Gravitationseffekte in einem statischen Raum, w\"ahrend Peratts Maxwell-basiertes Plasma-Modell Synchrotron-Strahlung als CMB-Ursprung ableitet. Wir zeigen, wie beide Konstrukte mit der T0-Theorie vereinbar sind: Das $\xi$-Feld ($\xi = \frac{4}{3} \times 10^{-4}$) dient als universeller Parameter, der Resonanzmoden (Unnikrishnan) und Filament-Dynamiken (Peratt) vereinheitlicht. Die Synthese ergibt eine koh\"arente, expansionsfreie Kosmologie, die das CMB-Power-Spektrum als emergente $\xi$-Harmonie erkl\"art.
	
	\section{Einleitung: Von der Oberfl\"achen- zur mathematischen Analyse}
	Das Video \cite{video2025} hebt die zirkul\"are Natur des $\Lambda$CDM-Modells hervor und kontrastiert es mit radikalen Alternativen: Unnikrishnans statische Resonanz und Peratts plasmabasierte Strahlung. Eine oberfl\"achliche Betrachtung reicht nicht; wir tauchen in die Feldgleichungen und Ableitungen ein, basierend auf Prim\"arquellen \cite{unnikrishnan2004, peratt1992}. Ziel: Eine Synthese mit T0, wo das $\xi$-Feld die Dualit\"at Zeit-Masse ($T \cdot m = 1$) und fraktale Geometrie verbindet. Dies l\"ost offene Probleme wie den hohen Q-Faktor oder Spektral-Pr\"azision.
	\section{Mathematische Konstrukte der kosmischen Relativit\"at (Unnikrishnan)}
	Unnikrishnans Theorie \cite{unnikrishnan2004} reformuliert die Relativit\"at als ``kosmische Relativit\"at'': Relativistische Effekte sind Gravitationsgradienten eines homogenen, statischen Universums. Keine Expansion; CMB-Peaks als stehende Wellen in einem kosmischen Feld.
	\subsection{Fundamentale Feldgleichungen}
	Die Kernidee: Die Lorentz-Transformationen $L(v,t)$ werden zu gravitativen Effekten:
	\begin{equation}
		L(v,t) = \exp\left( -\frac{\nabla \Phi}{c^2} \right),
	\end{equation}
	wobei $\Phi$ das kosmische Gravitationspotential ist ($\Phi = -GM/r$ f\"ur ein homogenes Universum, $M$ die Gesamtmasse). Zeitdilatation und L\"angenkontraktion emergieren als:
	\begin{equation}
		\frac{\Delta t}{t} = 1 + \frac{\Phi}{c^2}, \quad \frac{\Delta l}{l} = 1 - \frac{\Phi}{c^2}.
	\end{equation}
	Die Feldgleichung erweitert Einsteins Gleichungen zu einer ``kosmischen Metrik'':
	\begin{equation}
		R_{\mu\nu} = 8\pi G \left(T_{\mu\nu} - \frac{1}{2} g_{\mu\nu} T\right) + \Lambda g_{\mu\nu} + \xi \nabla_\mu \nabla_\nu \Phi,
	\end{equation}
	mit $\xi$ als Kopplungskonstante (hier analog zu T0). Der Weyl-Teil $W_{\mu\nu\rho\sigma}$ repr\"asentiert anisotrope kosmische Gradienten.
	\subsection{CMB-Ableitung: Stehende Wellen}
	CMB als Resonanzmoden in statischem Feld: Die Wellengleichung im kosmischen Rahmen:
	\begin{equation}
		\square \psi + \frac{\nabla \Phi}{c^2} \partial_t \psi = 0,
	\end{equation}
	f\"uhrt zu stehenden Wellen $\psi = \sum_k A_k \sin(k \cdot x - \omega t + \phi_k)$, wobei Peaks bei $k_n = n \pi / L_{\text{cosmic}}$ (L = Kosmos-Gr\"o\ss e) entstehen. Q-Faktor $Q = \omega / \Delta \omega \approx 10^6$ durch Gravitationsd\"ampfung. Polarisation: $W$-induzierte Phasenverschiebungen.
	Das Video (11:46) beschreibt dies als ``lebendige Resonanz'' -- mathematisch: Harmonische Oszillatoren in $\Phi$-Gradienten.
	\section{Mathematische Konstrukte der Plasma-Kosmologie (Peratt)}
	Peratts Modell \cite{peratt1992} leitet CMB aus Plasma-Dynamik ab: Synchrotron-Strahlung in Birkeland-Filamenten erzeugt Blackbody-Spektrum durch kollektive Emission/Absorption.
	\subsection{Fundamentale Feldgleichungen}
	Basierend auf Maxwell-Gleichungen in Plasmen:
	\begin{equation}
		\nabla \times \mathbf{B} = \mu_0 \mathbf{J} + \mu_0 \epsilon_0 \frac{\partial \mathbf{E}}{\partial t}, \quad \nabla \cdot \mathbf{B} = 0,
	\end{equation}
	mit Lorentz-Kraft $\mathbf{F} = q(\mathbf{E} + \mathbf{v} \times \mathbf{B})$. F\"ur Filamente: Z-Pinch-Gleichung
	\begin{equation}
		\frac{dp}{dt} = \mathbf{J} \times \mathbf{B},
	\end{equation}
	wo $\mathbf{J}$ Stromdichte ist ($10^{18}$ A in galaktischen Filamenten). Synchrotron-Leistung:
	\begin{equation}
		P_{\text{synch}} = \frac{2}{3} r_e^2 \gamma^4 \beta^2 c B_\perp^2 \sin^2 \theta,
	\end{equation}
	mit $r_e$ klassischer Elektronenradius, $\gamma$ Lorentz-Faktor.
	\subsection{CMB-Ableitung: Spektrum und Power-Spektrum}
	Kollektive Strahlung: Integriertes Spektrum \"uber $N$ Filamente:
	\begin{equation}
		I(\nu) = \int N(\mathbf{r}) P_{\text{synch}}(\nu, B(\mathbf{r})) e^{-\tau(\nu)} d\mathbf{r},
	\end{equation}
	wobei $\tau(\nu)$ optische Tiefe (Selbstabsorption) ist. F\"ur CMB-Fit: $T \approx 2.7$ K bei $\nu \approx 160$ GHz; Peaks als Interferenz:
	\begin{equation}
		C_\ell = \frac{1}{2\ell + 1} \sum_m |a_{\ell m}|^2, \quad a_{\ell m} \propto \int Y_{\ell m}^*(\theta, \phi) e^{i \mathbf{k} \cdot \mathbf{r}} d\Omega,
	\end{equation}
	mit $\mathbf{k}$ Wellenvektor in Filament-Magnetfeldern. BAO: Fraktale Skalen $r_n = r_0 \phi^n$ ($\phi$ Goldener Schnitt).
	Das Video (13:46) betont ``reine Elektrodynamik'' -- Peratts Simulationen matchen SED zu 1\%.
	\section{Synthese: Einklang mit der T0-Theorie}
	T0 vereinheitlicht beide durch das $\xi$-Feld: Statisches Universum mit fraktaler Geometrie, wo Rotverschiebung $z \approx d \cdot C \cdot \xi$ ist.
	\subsection{Unnikrishnan in T0}
	$\xi$ als kosmischer Kopplungsparameter: Ersetzt $\nabla \Phi / c^2$ durch $\xi \nabla \ln \rho_\xi$, wobei $\rho_\xi$ $\xi$-Dichte. Erweiterte Gleichung:
	\begin{equation}
		R_{\mu\nu} = 8\pi G T_{\mu\nu} + \xi \nabla_\mu \nabla_\nu \ln \rho_\xi.
	\end{equation}
	Resonanzmoden: $\square \psi + \xi \mathcal{F}[\psi] = 0$ (T0-Feldgleichung), Peaks bei $\omega_n = n c / L \cdot (1 - 100 \xi)$. Q-Faktor: $Q \approx 1 / (1 - K_{\text{frak}}) \approx 10^4 / \xi$.
	\subsection{Peratt in T0}
	Filamente als $\xi$-induzierte Str\"ome: $\mathbf{J} = \sigma \mathbf{E} + \xi \nabla \times \mathbf{B}$. Synchrotron:
	\begin{equation}
		P_{\text{synch}} = \frac{2}{3} r_e^2 \gamma^4 \beta^2 c (B_\perp + \xi \partial_t B)^2.
	\end{equation}
	Power-Spektrum: Fraktale Hierarchie $C_\ell \propto \sum_n \xi^n \sin(\ell \theta_n)$, mit $\theta_n = \pi (1 - 100 \xi)^n$. BAO: $r_{\text{BAO}} \approx 150$ Mpc als $\xi$-skalierte Filament-L\"ange.
	\subsection{Vereinheitlichte T0-Gleichung}
	Kombinierte Feldgleichung:
	\begin{equation}
		\square A_\mu + \xi \left( \nabla^\nu F_{\nu\mu} + \mathcal{F}[A_\mu] \right) = J_\mu,
	\end{equation}
	wo $A_\mu$ Vektorpotential (Peratt), $\mathcal{F}$ fraktaler Operator (Unnikrishnan/T0). Dies erzeugt CMB als $\xi$-Resonanz in statischem Plasma-Feld.
	\section{Schlussfolgerung}
	Die mathematischen Konstrukte von Unnikrishnan (gravitative Lorentz-Transformationen) und Peratt (Maxwell-Synchrotron in Filamenten) sind koh\"arent, aber isoliert. T0 bringt sie in Einklang: $\xi$ als Br\"ucke zwischen Resonanz und Plasma-Dynamik. Das CMB-Power-Spektrum emergiert als $\xi$-Harmonie -- pr\"azise, ohne Patches. Zuk\"unftige Simulationen (z. B. FEniCS f\"ur $\xi$-Felder) werden dies testen.
	\begin{thebibliography}{9}
		\bibitem{unnikrishnan2004}
		C. S. Unnikrishnan, \textit{Cosmic Relativity: The Fundamental Theory of Relativity, its Implications, and Experimental Tests},
		arXiv:gr-qc/0406023, 2004.
		\url{https://arxiv.org/abs/gr-qc/0406023}.
		\bibitem{peratt1992}
		A. L. Peratt, \textit{Physics of the Plasma Universe},
		Springer-Verlag, 1992.
		\url{https://ia600804.us.archive.org/12/items/AnthonyPerattPhysicsOfThePlasmaUniverse_201901/Anthony-Peratt--Physics-of-the-Plasma-Universe.pdf}.
		\bibitem{peratt1986}
		A. L. Peratt, \textit{Evolution of the Plasma Universe: I. Double Radio Galaxies, Quasars, and Extragalactic Jets},
		IEEE Transactions on Plasma Science, 14(6), 639--660, 1986.
		\bibitem{pascher:t0_foundations}
		J. Pascher, \textit{T0-Theorie: Zusammenfassung der Erkenntnisse},
		T0-Dokumentenserie, Nov. 2025.
		\bibitem{video2025}
		See the Pattern, \textit{A Test Only $\Lambda$CDM Can Pass, Because It Wrote the Rules},
		YouTube-Video, URL: \url{https://www.youtube.com/watch?v=g7_JZJzVuqs},
		16. November 2025.
	\end{thebibliography}
