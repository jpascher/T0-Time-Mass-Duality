\documentclass[twocolumn,aps,prl]{revtex4-2}
\usepackage[utf8]{inputenc}
\usepackage[T1]{fontenc}
\usepackage[english]{babel}
\usepackage{lmodern}
\usepackage{amsmath}
\usepackage{amssymb}
\usepackage{physics}
\usepackage{hyperref}
\usepackage{booktabs}
\usepackage{enumitem}
\usepackage[table,xcdraw]{xcolor}
\usepackage{pgfplots}
\pgfplotsset{compat=1.18}
\usepackage{graphicx}
\usepackage{siunitx}
\usepackage{array}
\usepackage{tcolorbox}
\usepackage{mathtools}

% Custom commands - CORRECTED for consistency
\newcommand{\Tfield}{T(x,t)}
\newcommand{\Tfieldt}{T(x,t)}
\newcommand{\alphaEM}{\alpha_{\text{EM}}}
\newcommand{\alphaW}{\alpha_{\text{W}}}
\newcommand{\betaT}{\beta_{\text{T}}}
\newcommand{\Mpl}{M_{\text{Pl}}}
\newcommand{\Tzerot}{T_0(\Tfieldt)}
\newcommand{\Tzero}{T_0}
\newcommand{\vecx}{\vec{x}}
\newcommand{\gammaf}{\gamma_{\text{Lorentz}}}
\newcommand{\DhiggsT}{\Tfieldt (\partial_\mu + ig A_\mu) \Phi + \Phi \partial_\mu \Tfieldt}
\newcommand{\LCDM}{\Lambda\text{CDM}}
\newcommand{\DTmu}{D_{T,\mu}}
\newcommand{\calL}{\mathcal{L}}
\newcommand{\deq}{\displaystyle}
\newcommand{\e}{\mathrm{e}}
\newcommand{\xipar}{\xi}
\newcommand{\lP}{\ell_{\text{P}}}

\hypersetup{
	colorlinks=true,
	linkcolor=blue,
	citecolor=blue,
	urlcolor=blue,
	pdftitle={Bridging Quantum Mechanics and Relativity through Time-Mass Duality: Updated Framework},
	pdfauthor={Johann Pascher},
	pdfsubject={Theoretical Physics},
	pdfkeywords={T0 Model, Time-Mass Duality, Natural Units, Field Theory}
}

\begin{document}
	
	\title{Bridging Quantum Mechanics and Relativity through Time-Mass Duality: A Unified Framework with Natural Units \(\alpha_{\text{EM}} = \beta_{\text{T}} = 1\) \\ Part I: Updated Theoretical Foundations with Complete Geometric Framework}
	\author{Johann Pascher}
	\affiliation{Department of Communications Engineering, Höhere Technische Bundeslehranstalt (HTL), Leonding, Austria}
	\email{johann.pascher@gmail.com}
	\date{\today}
	
	\begin{abstract}
		This updated paper establishes the theoretical foundations of the T0 model based on the comprehensive field-theoretic derivation and complete geometric framework. The theory utilizes the unified natural unit system where \(\hbar = c = \alpha_{\text{EM}} = \beta_{\text{T}} = 1\), reducing all physical quantities to powers of energy. We demonstrate how the intrinsic time field \(\Tfieldt = \frac{1}{\max(m(x,t), \omega(x,t))}\) provides a unified treatment of quantum mechanics and relativity through the three fundamental field geometries: localized spherical, localized non-spherical, and infinite homogeneous. The framework incorporates the derived parameters \(\beta = 2Gm/r\), \(\xi = 2\sqrt{G} \cdot m\), and the cosmic screening factor \(\xi_{\text{eff}} = \xi/2\) for infinite fields. All formulations maintain strict dimensional consistency and contain no adjustable parameters, with connections to Higgs physics through \(\beta_T = \lambda_h^2 v^2/(16\pi^3 m_h^2 \xi) = 1\). This framework bridges quantum mechanics and general relativity through emergent gravitation, offering testable predictions including wavelength-dependent redshift and energy-dependent quantum correlations.
	\end{abstract}
	
	\maketitle
	
	\begin{center}
		\rule{\linewidth}{0.5pt}
		{\bfseries\large Fundamental Principle of the T0 Theory Framework}
		\rule{\linewidth}{0.5pt}
	\end{center}
	
	A defining characteristic of the T0 model presented here is the complete absence of freely adjustable parameters. All constants and values—both dimensional and dimensionless—are rigorously derived from first principles through the field-theoretic framework established in the complete geometric derivation. This parameter-free formulation represents a substantially more stringent test than traditional theories, as all predictions follow directly from the foundational equations without room for post-hoc adjustments to match experimental observations.
	
	The theory builds upon the comprehensive natural units system where \(\hbar = c = \alpha_{\text{EM}} = \beta_{\text{T}} = 1\), unified through connections to Higgs physics and geometric field theory. All parameters including \(\beta = 2Gm/r\), \(\xi = 2\sqrt{G} \cdot m\), and the cosmic screening modifications emerge from the fundamental field equation \(\nabla^2 m = 4\pi G \rho m\) and its solutions across the three fundamental geometries.
	
	\begin{center}
		\rule{\linewidth}{0.5pt}
	\end{center}
	
	\section{Introduction}
	\label{sec:introduction}
	
	This updated work presents the theoretical foundations of the T0 model, building upon the comprehensive field-theoretic derivation and complete geometric framework. The theory establishes a unified treatment of quantum mechanics and relativity through the intrinsic time field \(\Tfieldt\), operating within natural units where fundamental constants achieve unity values through deep theoretical connections.
	
	The T0 model addresses the unification challenge through a novel approach: rather than modifying existing frameworks, it establishes a new foundation based on time-mass duality. This principle inverts the traditional relationship between time and mass, leading to emergent gravitational effects and modified quantum dynamics while maintaining complete mathematical consistency.
	
	Key innovations include: (1) A unified natural unit system based on field-theoretic connections; (2) Three fundamental field geometries with specific parameter modifications; (3) Complete dimensional consistency throughout all formulations; (4) Parameter-free predictions derived from first principles; (5) Testable experimental signatures distinguishing the T0 approach from conventional theories.
	
	\section{Complete Natural Units Framework}
	\label{sec:complete_natural_units}
	
	\subsection{Field-Theoretic Foundation}
	\label{subsec:field_theoretic_foundation}
	
	The natural units system where \(\hbar = c = \alpha_{\text{EM}} = \beta_{\text{T}} = 1\) emerges from the complete field-theoretic derivation:
	
	\begin{equation}
		\beta_{\text{T}} = \frac{\lambda_h^2 v^2}{16\pi^3 m_h^2 \xi} = 1
		\label{eq:beta_t_derivation}
	\end{equation}
	
	where \(\lambda_h \approx 0.13\), \(v \approx 246\) GeV, \(m_h \approx 125\) GeV, and \(\xi = 2\sqrt{G} \cdot m\).
	
	The electromagnetic coupling unity follows from the unified coupling structure:
	\begin{equation}
		\alpha_{\text{EM}} = \beta_{\text{T}} = 1
		\label{eq:coupling_unity}
	\end{equation}
	
	\subsection{Dimensional Structure in Natural Units}
	\label{subsec:dimensional_structure}
	
	All quantities reduce to energy dimensions:
	
	\begin{tcolorbox}[colback=blue!5!white,colframe=blue!75!black,title=T0 Natural Units Dimensions]
		\begin{align}
			\text{Length:} \quad [L] &= [E^{-1}] \\
			\text{Time:} \quad [T] &= [E^{-1}] \\
			\text{Mass:} \quad [M] &= [E] \\
			\text{Charge:} \quad [Q] &= [1] \text{ (dimensionless)} \\
			\text{Temperature:} \quad [\Theta] &= [E]
		\end{align}
	\end{tcolorbox}
	
	\subsection{Planck Scale Connection}
	\label{subsec:planck_connection}
	
	The Planck length in natural units becomes:
	\begin{equation}
		\lP = \sqrt{G}
		\label{eq:planck_length_natural}
	\end{equation}
	
	This connects to the T0 scale parameter:
	\begin{equation}
		\xi = \frac{r_0}{\lP} = \frac{2Gm}{\sqrt{G}} = 2\sqrt{G} \cdot m
		\label{eq:xi_definition}
	\end{equation}
	
	\section{Dual Framework: T0 Model and Extended Standard Model}
	\label{sec:dual_framework}
	
	This work presents two mathematically equivalent but conceptually distinct approaches that bridge quantum mechanics and relativity.
	
	\subsection{T0 Model Framework}
	\label{subsec:t0_framework}
	
	The T0 model posits absolute time and variable mass, based on the field equation:
	
	\begin{equation}
		\nabla^2 m(\vec{x},t) = 4\pi G \rho(\vec{x},t) \cdot m(\vec{x},t)
		\label{eq:t0_field_equation}
	\end{equation}
	
	\textbf{Dimensional verification}: \([\nabla^2 m] = [E^2][E] = [E^3]\) and \([4\pi G \rho m] = [1][E^{-2}][E^4][E] = [E^3]\) \checkmark
	
	\subsection{Extended Standard Model Framework}
	\label{subsec:esm_framework}
	
	The ESM maintains relative time and constant rest mass while introducing a scalar field \(\Theta\) that modifies Einstein's equations:
	
	\begin{equation}
		G_{\mu\nu} + \kappa g_{\mu\nu} = 8\pi G T_{\mu\nu} + \nabla_{\mu}\Theta\nabla_{\nu}\Theta - \frac{1}{2}g_{\mu\nu}(\nabla_{\sigma}\Theta\nabla^{\sigma}\Theta)
		\label{eq:modified_einstein}
	\end{equation}
	
	\subsection{Mathematical Equivalence Between Frameworks}
	\label{subsec:mathematical_equivalence}
	
	The frameworks are connected through the logarithmic transformation:
	
	\begin{equation}
		\Theta(\vec{x},t) = \ln\left(\frac{\Tfieldt}{\Tzero}\right)
		\label{eq:theta_transformation}
	\end{equation}
	
	This ensures that both approaches yield identical predictions while maintaining distinct conceptual foundations:
	
	\begin{align}
		\text{T0:} \quad t &= \text{absolute}, \quad m = \text{variable} \\
		\text{ESM:} \quad t &= \text{relative}, \quad m_0 = \text{constant}
		\label{eq:framework_comparison}
	\end{align}
	
	\subsection{Unified Quantum Evolution}
	\label{subsec:unified_quantum_evolution}
	
	Both frameworks modify quantum mechanics:
	
	\textbf{T0 Model}:
	\begin{equation}
		i \Tfieldt \frac{\partial\Psi}{\partial t} + i \Psi \left[\frac{\partial \Tfieldt}{\partial t} + \vec{v} \cdot \nabla \Tfieldt\right] = \hat{H} \Psi
		\label{eq:t0_schrodinger}
	\end{equation}
	
	\textbf{ESM}:
	\begin{equation}
		i\frac{\partial\Psi}{\partial t} = \hat{H}\Psi - i\left[\frac{\partial\Theta}{\partial t} + \vec{v} \cdot \nabla\Theta\right]\Psi
		\label{eq:esm_schrodinger}
	\end{equation}
	
	Under the transformation \(\Theta = \ln(\Tfieldt/\Tzero)\), both equations yield identical results.
	
	\subsection{Three Fundamental Field Geometries}
	\label{subsec:three_geometries}
	
	The complete T0 framework recognizes three distinct field geometries:
	
	\subsubsection{Localized Spherical Fields}
	\label{subsubsec:localized_spherical}
	
	For spherically symmetric sources, the solution is:
	\begin{equation}
		m(r) = m_0\left(1 + \frac{2Gm}{r}\right) = m_0(1 + \beta)
		\label{eq:spherical_solution}
	\end{equation}
	
	where \(\beta = 2Gm/r\) and \(\xi = 2\sqrt{G} \cdot m\).
	
	\subsubsection{Localized Non-Spherical Fields}
	\label{subsubsec:nonsphere_fields}
	
	For non-spherical mass distributions:
	\begin{align}
		\beta_{ij} &= \frac{r_{0ij}}{r} \quad \text{(tensorial)} \\
		\xi_{ij} &= 2\sqrt{G} \cdot I_{ij} \quad \text{(inertia tensor)}
		\label{eq:tensorial_parameters}
	\end{align}
	
	\subsubsection{Infinite Homogeneous Fields}
	\label{subsubsec:infinite_fields}
	
	For infinite, homogeneous systems, the field equation requires modification:
	\begin{equation}
		\nabla^2 m = 4\pi G \rho_0 m + \Lambda_T m
		\label{eq:infinite_field_equation}
	\end{equation}
	
	with \(\Lambda_T = -4\pi G \rho_0\) and cosmic screening:
	\begin{equation}
		\xi_{\text{eff}} = \sqrt{G} \cdot m = \frac{\xi}{2}
		\label{eq:cosmic_screening}
	\end{equation}
	
	\section{Modified Quantum Mechanics}
	\label{sec:modified_quantum_mechanics}
	
	\subsection{Time Field-Modified Schrödinger Equation}
	\label{subsec:modified_schrodinger}
	
	The quantum evolution equation incorporates the time field:
	
	\begin{equation}
		i \Tfieldt \frac{\partial\Psi}{\partial t} + i \Psi \left[\frac{\partial \Tfieldt}{\partial t} + \vec{v} \cdot \nabla \Tfieldt\right] = \hat{H} \Psi
		\label{eq:t0_schrodinger_modified}
	\end{equation}
	
	\textbf{Dimensional verification}:
	\begin{itemize}
		\item \([i \Tfieldt \partial_t \Psi] = [E^{-1}][E][\Psi] = [\Psi]\)
		\item \([\hat{H} \Psi] = [E][\Psi] = [\Psi]\) \checkmark
	\end{itemize}
	
	\subsection{Energy-Dependent Quantum Dynamics}
	\label{subsec:energy_dependent_dynamics}
	
	For particles with different energies, the time field creates energy-dependent evolution rates:
	\begin{equation}
		\Tfieldt(E) = \frac{1}{E} \quad \text{(for } E \gg m\text{)}
		\label{eq:energy_dependent_time}
	\end{equation}
	
	This leads to energy-dependent quantum correlations and modified Bell inequality tests.
	
	\section{Emergent Gravitation}
	\label{sec:emergent_gravitation}
	
	\subsection{Gravitational Force from Time Field Gradients}
	\label{subsec:gravitational_force}
	
	Gravitational effects emerge from time field gradients:
	\begin{equation}
		\vec{F}_{\text{grav}} = -m \frac{\nabla \Tfieldt}{\Tfieldt} = m \frac{\nabla m}{m^2}
		\label{eq:gravitational_force}
	\end{equation}
	
	\subsection{Modified Gravitational Potential}
	\label{subsec:modified_potential}
	
	The T0 model predicts a modified potential:
	\begin{equation}
		\Phi(r) = -\frac{GM}{r} + \kappa r
		\label{eq:modified_gravitational_potential}
	\end{equation}
	
	where \(\kappa\) depends on the field geometry:
	\begin{align}
		\text{Localized:} \quad \kappa &= \alpha_\kappa H_0 \xi \\
		\text{Cosmic:} \quad \kappa &= H_0
		\label{eq:kappa_values}
	\end{align}
	
	\subsection{Connection to General Relativity}
	\label{subsec:gr_connection}
	
	The T0 model reduces to General Relativity in appropriate limits while providing corrections:
	\begin{itemize}
		\item Weak field limit: Standard post-Newtonian behavior
		\item Strong field regime: Modified through \(\beta\) parameter
		\item Cosmological scales: Natural dark energy replacement
	\end{itemize}
	
	\section{Energy Loss and Cosmological Redshift}
	\label{sec:energy_loss_redshift}
	
	\subsection{Photon Energy Loss Mechanism}
	\label{subsec:photon_energy_loss}
	
	Photons lose energy to time field gradients:
	\begin{equation}
		\frac{d\omega}{dr} = -g_T \omega^2 \frac{2G}{r^2}
		\label{eq:energy_loss_rate}
	\end{equation}
	
	where \(g_T = \alpha_{\text{EM}} = 1\) in natural units.
	
	\textbf{Dimensional verification}: \([d\omega/dr] = [E^2]\) and \([g_T \omega^2 2G/r^2] = [1][E^2][E^{-2}][E^2] = [E^2]\) \checkmark
	
	\subsection{Wavelength-Dependent Redshift}
	\label{subsec:wavelength_redshift}
	
	Integration of energy loss yields:
	\begin{equation}
		z(\lambda) = z_0\left(1 + \beta_T \ln\frac{\lambda}{\lambda_0}\right)
		\label{eq:wavelength_dependent_redshift}
	\end{equation}
	
	with \(\beta_T = 1\) providing a distinctive T0 signature.
	
	\subsection{Static Universe Interpretation}
	\label{subsec:static_universe}
	
	The T0 model explains cosmological observations without spatial expansion:
	\begin{itemize}
		\item Redshift: Energy loss to time field gradients
		\item CMB: Equilibrium radiation in static universe
		\item Structure formation: Modified gravitational dynamics
		\item Cosmic acceleration: Linear \(\kappa r\) term
	\end{itemize}
	
	\section{Connection to Standard Model Physics}
	\label{sec:standard_model_connection}
	
	\subsection{Higgs Mechanism Integration}
	\label{subsec:higgs_integration}
	
	The time field couples to the Higgs field:
	\begin{equation}
		\Tfieldt = \frac{1}{y\langle\Phi\rangle}
		\label{eq:time_higgs_connection}
	\end{equation}
	
	where \(y\) is the Yukawa coupling and \(\langle\Phi\rangle\) is the Higgs VEV.
	
	\subsection{Electromagnetic Coupling Unification}
	\label{subsec:em_coupling_unification}
	
	The unity \(\alpha_{\text{EM}} = \beta_T = 1\) emerges from the shared vacuum structure through Higgs physics, demonstrating the deep connection between electromagnetic and gravitational phenomena.
	
	\subsection{Gauge Field Modifications}
	\label{subsec:gauge_modifications}
	
	Gauge fields couple to the time field through modified Lagrangians:
	\begin{equation}
		\mathcal{L}_{\text{gauge}} = -\frac{1}{4}\Tfieldt^2 F_{\mu\nu}F^{\mu\nu}
		\label{eq:gauge_lagrangian}
	\end{equation}
	
	\section{Experimental Predictions and Tests}
	\label{sec:experimental_predictions}
	
	\subsection{Distinctive T0 Signatures}
	\label{subsec:distinctive_signatures}
	
	The T0 model makes specific testable predictions:
	
	\begin{enumerate}
		\item \textbf{Wavelength-dependent redshift}:
		\begin{equation}
			\frac{\partial z}{\partial \ln\lambda} = z_0 \beta_T = z_0
			\label{eq:redshift_derivative}
		\end{equation}
		
		\item \textbf{Energy-dependent quantum correlations}:
		\begin{equation}
			\Delta t_{\text{corr}} = \frac{G\langle m \rangle}{r} \left|\frac{1}{\omega_1} - \frac{1}{\omega_2}\right|
			\label{eq:correlation_delay}
		\end{equation}
		
		\item \textbf{Modified gravitational dynamics}:
		\begin{equation}
			v^2(r) = \frac{GM}{r} + \kappa r^2
			\label{eq:rotation_curve}
		\end{equation}
	\end{enumerate}
	
	\subsection{Precision Tests}
	\label{subsec:precision_tests}
	
	The parameter-free nature enables stringent tests:
	\begin{itemize}
		\item Multi-wavelength spectroscopy for redshift dependence
		\item High-precision quantum optics for correlation delays
		\item Gravitational wave astronomy for modified dynamics
		\item Atomic clock networks for time field gradients
	\end{itemize}
	
	\section{Dimensional Consistency Verification}
	\label{sec:dimensional_verification}
	
	\begin{table}[h!]
		\centering
		\small
		\begin{tabular}{lcc}
			\toprule
			\textbf{Equation} & \textbf{Left} & \textbf{Right} \\
			\midrule
			Time field & \([T] = [E^{-1}]\) & \([1/m] = [E^{-1}]\) \\
			Field equation & \([\nabla^2 m] = [E^3]\) & \([G\rho m] = [E^3]\) \\
			\(\beta\) parameter & \([\beta] = [1]\) & \([2Gm/r] = [1]\) \\
			\(\xi\) parameter & \([\xi] = [1]\) & \([2\sqrt{G}m] = [1]\) \\
			Energy loss & \([d\omega/dr] = [E^2]\) & \([g_T\omega^2 G/r^2] = [E^2]\) \\
			\bottomrule
		\end{tabular}
		\caption{Dimensional consistency verification}
	\end{table}
	
	All equations maintain perfect dimensional consistency in the natural units framework.
	
	\section{Philosophical Implications}
	\label{sec:philosophical_implications}
	
	\subsection{Parameter-Free Theory}
	\label{subsec:parameter_free_theory}
	
	The absence of adjustable parameters represents a fundamental shift in theoretical physics. Unlike conventional theories that achieve agreement with observations through parameter fitting, the T0 model's predictions follow directly from first principles, creating a more stringent empirical test.
	
	\subsection{Time-Mass Duality}
	\label{subsec:time_mass_duality_philosophy}
	
	The duality between absolute time with variable mass versus relative time with constant mass demonstrates that fundamentally different ontological frameworks can yield identical physical predictions. This principle of complementarity enriches our understanding of the relationship between mathematical formalism and physical reality.
	
	\subsection{Emergent vs. Fundamental Forces}
	\label{subsec:emergent_fundamental}
	
	The T0 model's treatment of gravitation as an emergent phenomenon rather than a fundamental force represents a conceptual shift with implications for unification efforts. This approach suggests that apparent forces may emerge from deeper field-theoretic structures.
	
	\section{Conclusions}
	\label{sec:conclusions}
	
	This updated framework establishes the T0 model as a comprehensive theory bridging quantum mechanics and relativity through the intrinsic time field \(\Tfieldt\). Key achievements include:
	
	\begin{itemize}
		\item Complete field-theoretic foundation with three fundamental geometries
		\item Parameter-free formulation with all constants derived from first principles
		\item Unified natural units system based on deep theoretical connections
		\item Emergent gravitation without requiring a separate fundamental force
		\item Testable predictions distinguishing T0 from conventional theories
		\item Integration with Standard Model physics through Higgs connections
	\end{itemize}
	
	The T0 model demonstrates that alternative foundational principles can lead to a unified description of quantum and gravitational phenomena while maintaining complete mathematical consistency and dimensional accuracy. The framework's parameter-free nature and distinctive experimental signatures provide opportunities for definitive empirical tests.
	
	Future work will explore cosmological applications, detailed experimental proposals, and extensions to non-Abelian gauge theories, further developing this unified approach to fundamental physics.
	
	\begin{acknowledgments}
		The author acknowledges the foundational importance of the complete field-theoretic derivation in establishing this unified framework.
	\end{acknowledgments}
	
	\begin{thebibliography}{99}
		
		\bibitem{pascher_derivation_beta_2025} 
		Pascher, J. (2025). \href{https://github.com/jpascher/T0-Time-Mass-Duality/blob/main/2/pdf/DerivationVonBetaEn.pdf}{\textit{Field-Theoretic Derivation of the $\beta_T$ Parameter in Natural Units ($\hbar = c = 1$)}}. GitHub Repository: T0-Time-Mass-Duality.
		
		\bibitem{pascher_dirac_integration_2025}
		Pascher, J. (2025). \href{https://github.com/jpascher/T0-Time-Mass-Duality/tree/main/2/pdf/diracEn.pdf}{\textit{Integration of the Dirac Equation in the T0 Model: Updated Framework}}. GitHub Repository: T0-Time-Mass-Duality.
		
		\bibitem{pascher_math_zeit_masse_2025}
		Pascher, J. (2025). \href{https://github.com/jpascher/T0-Time-Mass-Duality/tree/main/2/pdf/MathZeitMasseLagrangeEn.pdf}{\textit{Mathematical Core Formulations of Time-Mass Duality Theory: Updated Framework}}. GitHub Repository: T0-Time-Mass-Duality.
		
		\bibitem{pascher_dynamic_photons_2025}
		Pascher, J. (2025). \href{https://github.com/jpascher/T0-Time-Mass-Duality/tree/main/2/pdf/DynMassePhotonenNichtlokalEn.pdf}{\textit{Dynamic Mass of Photons and Nonlocality in T0 Model: Updated Framework}}. GitHub Repository: T0-Time-Mass-Duality.
		
		\bibitem{einstein1905} 
		A. Einstein, ``Zur Elektrodynamik bewegter Körper,'' Ann. Phys. \textbf{17}, 891 (1905).
		
		\bibitem{einstein1915} 
		A. Einstein, ``Die Feldgleichungen der Gravitation,'' Sitzungsber. Preuss. Akad. Wiss. Berlin, 844 (1915).
		
		\bibitem{planck1900} 
		M. Planck, ``Zur Theorie des Gesetzes der Energieverteilung im Normalspektrum,'' Verh. Dtsch. Phys. Ges. \textbf{2}, 237 (1900).
		
		\bibitem{dirac1928} 
		P. A. M. Dirac, ``The Quantum Theory of the Electron,'' Proc. R. Soc. London A \textbf{117}, 610 (1928).
		
		\bibitem{higgs1964} 
		P. W. Higgs, ``Broken Symmetries and the Masses of Gauge Bosons,'' Phys. Rev. Lett. \textbf{13}, 508 (1964).
		
		\bibitem{weinberg1967} 
		S. Weinberg, ``A Model of Leptons,'' Phys. Rev. Lett. \textbf{19}, 1264 (1967).
		
		\bibitem{yang1954} 
		C. N. Yang and R. L. Mills, ``Conservation of Isotopic Spin and Isotopic Gauge Invariance,'' Phys. Rev. \textbf{96}, 191 (1954).
		
		\bibitem{noether1918} 
		E. Noether, ``Invariante Variationsprobleme,'' Nachr. Ges. Wiss. Göttingen, Math.-Phys. Kl., 235 (1918).
		
		\bibitem{weinberg1995} 
		S. Weinberg, The Quantum Theory of Fields, Vol. 1: Foundations. Cambridge: Cambridge University Press, 1995.
		
	\end{thebibliography}
	
\end{document}