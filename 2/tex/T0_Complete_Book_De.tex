% T0-Theorie: Zeit-Masse-Dualität
% Vollständiges Buch - Deutsche Ausgabe
% Johann Pascher, 2024

\documentclass[12pt,a4paper,twoside]{book}

% Standardized preamble
% ==============================================================================
% T0-Theorie: Standardisierte Deutsche Präambel
% Version: 1.0
% Autor: Johann Pascher
% ==============================================================================
% Diese Datei enthält alle notwendigen Pakete und Definitionen für deutsche
% T0-Theorie Dokumente. Verwenden Sie % ==============================================================================
% T0-Theorie: Standardisierte Deutsche Präambel
% Version: 1.0
% Autor: Johann Pascher
% ==============================================================================
% Diese Datei enthält alle notwendigen Pakete und Definitionen für deutsche
% T0-Theorie Dokumente. Verwenden Sie % ==============================================================================
% T0-Theorie: Standardisierte Deutsche Präambel
% Version: 1.0
% Autor: Johann Pascher
% ==============================================================================
% Diese Datei enthält alle notwendigen Pakete und Definitionen für deutsche
% T0-Theorie Dokumente. Verwenden Sie \input{T0_preamble_De} nach \documentclass.
% ==============================================================================

% --- Kodierung und Sprache ---
\usepackage[utf8]{inputenc}
\usepackage[T1]{fontenc}
\usepackage[ngerman]{babel}
\usepackage{lmodern}

% --- Seitengeometrie ---
\usepackage[a4paper, margin=2.5cm]{geometry}
\setlength{\headheight}{15pt}

% --- Mathematik und Physik ---
\usepackage{amsmath,amssymb,amsfonts,amsthm}
\usepackage{mathtools}
\usepackage{physics}
\usepackage{siunitx}
\sisetup{
    locale=DE,
    group-separator={.},
    output-decimal-marker={,},
    per-mode=symbol
}

% --- Grafiken und Tabellen ---
\usepackage{graphicx}
\usepackage[table,xcdraw]{xcolor}
\usepackage{tikz}
\usetikzlibrary{arrows.meta,positioning,shapes.geometric,decorations.pathmorphing,patterns,shapes.arrows,intersections}
\usepackage{pgfplots}
\pgfplotsset{compat=1.18}
\usepackage{quantikz}
\usepackage[most]{tcolorbox}
\tcbuselibrary{breakable}

% === WICHTIG: Algorithm-Konflikt umgehen ===
% Option: algorithmic mit GROSSBUCHSTABEN
% Gemeinsame Box für Experimente
\newtcolorbox{experimentbox}[1][]{
	colback=green!5!white,
	colframe=t0green!80!black,
	fonttitle=\bfseries,
	title={{#1}},
	breakable
}

% Abstract-Fallback
\ifdefined\abstract\else
\newenvironment{abstract}{\section*{\abstractname}\itshape\small\par\bigskip}{\bigskip}
\fi

% === MAKROS SICHER NEU DEFINIEREN / ÜBERSCHREIBEN ===
% Definiere Makros OHNE doppelte Subskripte
\newcommand{\phipar}{\phi_{\mathrm{par}}}
%\newcommand{\xipar}{\xi_{\mathrm{par}}}
\newcommand{\Qphipar}{Q_{\phi_{\mathrm{par}}}}
\newcommand{\rphipar}{r_{\phi_{\mathrm{par}}}}
\newcommand{\logphipar}{\log_{\phi_{\mathrm{par}}}}
\newcommand{\CHSH}{\text{CHSH}}
\usepackage{booktabs}
\usepackage{array}
\usepackage{longtable}
\usepackage{float}
\usepackage{adjustbox}
\usepackage{tabularx}
\usepackage{multirow}

% --- Dokumentformatierung ---
\usepackage{fancyhdr}
\renewcommand{\headrulewidth}{0.4pt}
\renewcommand{\footrulewidth}{0.4pt}
\usepackage{tocloft}
\usepackage{hyperref}
\usepackage{bookmark}
\usepackage{cleveref}
\usepackage{microtype}
\usepackage{enumitem}
\usepackage{setspace}
\usepackage{ragged2e}
\usepackage{multicol}

% --- Code und Algorithmen ---
\usepackage{algorithm}
\usepackage{algorithmic}
\usepackage{listings}
\usepackage{mdframed}

% --- Zitationsbefehle (Kompatibilität) ---
\providecommand{\citep}[1]{\cite{#1}}
\providecommand{\citet}[1]{\cite{#1}}

% --- Zusätzliche Pakete ---
\usepackage{pdflscape}
\usepackage{braket}
\usepackage{cancel}
\usepackage{caption}
\usepackage{csquotes}
\usepackage{gensymb}
\usepackage{hyphenat}
\usepackage{textcomp}
\usepackage{textgreek}
\usepackage{upgreek}
\usepackage{url}
% Hyphenation for URLs in bibliography
\def\UrlBreaks{\do\/\do-}
\usepackage{slashed}
\usepackage{bm}

% --- Fehlende Farben definieren ---
\definecolor{gold}{RGB}{255,215,0}

% --- Spaltentypen ---
\newcolumntype{L}[1]{>{\raggedright\arraybackslash}p{#1}}
\newcolumntype{C}[1]{>{\centering\arraybackslash}p{#1}}

% --- Unicode-Zeichen ---
\usepackage{newunicodechar}
\newunicodechar{ħ}{$\hbar$}
\newunicodechar{↔}{$\leftrightarrow$}
\newunicodechar{⇐}{$\Leftarrow$}
\newunicodechar{⇒}{$\Rightarrow$}
\newunicodechar{⇔}{$\Leftrightarrow$}
\newunicodechar{∂}{$\partial$}
\newunicodechar{∅}{$\emptyset$}
\newunicodechar{∇}{$\nabla$}
\newunicodechar{∈}{$\in$}
\newunicodechar{∉}{$\notin$}
\newunicodechar{∏}{$\prod$}
\newunicodechar{∑}{$\sum$}
\newunicodechar{√}{$\sqrt{}$}
\newunicodechar{∝}{$\propto$}
\newunicodechar{∞}{$\infty$}
\newunicodechar{∩}{$\cap$}
\newunicodechar{∪}{$\cup$}
\newunicodechar{∫}{$\int$}
\newunicodechar{≈}{$\approx$}
\newunicodechar{≠}{$\neq$}
\newunicodechar{≤}{$\leq$}
\newunicodechar{≥}{$\geq$}
\newunicodechar{ξ}{\ensuremath{\xi}}
\newunicodechar{μ}{\ensuremath{\mu}}
\newunicodechar{ψ}{\ensuremath{\psi}}
\newunicodechar{φ}{\ensuremath{\phi}}
\newunicodechar{π}{\ensuremath{\pi}}
\newunicodechar{λ}{\ensuremath{\lambda}}
\newunicodechar{Δ}{\ensuremath{\Delta}}

% --- Farben ---
\definecolor{blue}{rgb}{0,0,1}
\definecolor{boxgray}{RGB}{240,240,240}
\definecolor{deepblue}{RGB}{0,0,127}
\definecolor{deepgreen}{RGB}{0,127,0}
\definecolor{deepred}{RGB}{191,0,0}
\definecolor{t0blue}{RGB}{33,150,243}
\definecolor{t0green}{RGB}{76,175,80}
\definecolor{t0orange}{RGB}{255,152,0}
\definecolor{t0purple}{RGB}{156,39,176}
\definecolor{t0red}{RGB}{244,67,54}
\definecolor{t0yellow}{RGB}{255,204,0}

% --- Hyperref-Einstellungen ---
\hypersetup{
    colorlinks=true,
    linkcolor=blue,
    citecolor=blue,
    urlcolor=blue,
    breaklinks=true,
    bookmarksnumbered=true,
    pdfstartview=FitH
}

% --- Theorem-Umgebungen (Deutsch) ---
\theoremstyle{plain}
\newtheorem{satz}{Satz}[section]
\newtheorem{lemma}[satz]{Lemma}
\newtheorem{proposition}[satz]{Proposition}
\newtheorem{korollar}[satz]{Korollar}

\theoremstyle{definition}
\newtheorem{definition}[satz]{Definition}
\newtheorem{beispiel}[satz]{Beispiel}
\newtheorem{erkenntnis}[satz]{Erkenntnis}
\newtheorem{entdeckung}[satz]{Entdeckung}

\theoremstyle{remark}
\newtheorem{bemerkung}[satz]{Bemerkung}
\newtheorem{warnung}[satz]{Warnung}
\newtheorem{axiom}{Axiom}
\newtheorem{prinzip}{Prinzip}

% Aliases für englische Bezeichnungen
\newtheorem{theorem}[satz]{Theorem}
\newtheorem{corollary}[satz]{Corollary}
\newtheorem{remark}[satz]{Remark}
\newtheorem{example}[satz]{Example}
\newtheorem{insight}[satz]{Insight}
\newtheorem{discovery}[satz]{Discovery}
\newtheorem{principle}[satz]{Principle}

% --- T0-spezifische Befehle ---
\newcommand{\Tfield}{T(x,t)}
\providecommand{\Tfieldt}{T(\vec{x},t)}
\newcommand{\Efield}{E(x,t)}
\newcommand{\mfield}{m(x,t)}
\providecommand{\vecx}{\vec{x}}
\newcommand{\Lag}{\mathcal{L}}
\newcommand{\calL}{\mathcal{L}}
\newcommand{\alphaem}{\alpha}
\newcommand{\betaT}{\beta_T}
\newcommand{\xiT}{\xi}
\newcommand{\xipar}{\xi}
\newcommand{\Ezero}{E_0}
\newcommand{\EPlanck}{E_{\text{Pl}}}
\newcommand{\Mpl}{M_{\text{Pl}}}
\newcommand{\lP}{\ell_{\text{P}}}
\newcommand{\tP}{t_{\text{P}}}
\newcommand{\LPlanck}{\ell_{\text{Pl}}}
\newcommand{\TPlanck}{t_{\text{Pl}}}
\newcommand{\Gnat}{G_{\text{nat}}}
\newcommand{\alphaEM}{\alpha_{\text{EM}}}
\newcommand{\alphaSI}{\alpha_{\text{SI}}}
\newcommand{\Hubble}{H_0}
\newcommand{\LCDM}{\Lambda\text{CDM}}
\newcommand{\natunits}{(nat. Einheiten)}

% T0 Modell Parameter
\newcommand{\xigeom}{\xi_{\mathrm{geom}}}
\newcommand{\rzero}{r_{0}}
\newcommand{\xirat}{\xi_{\mathrm{rat}}}
\newcommand{\tzero}{t_{0}}
\newcommand{\Lambdat}{\Lambda_{\mathrm{t}}}
\newcommand{\EP}{E_{\mathrm{P}}}
\newcommand{\Emu}{E_{\mu}}
\newcommand{\Ee}{E_{e}}
\newcommand{\Etau}{E_{\tau}}
\newcommand{\alphafine}{\alpha_{\mathrm{fine}}}
\newcommand{\alphal}{\alpha_{\ell}}
\newcommand{\Lzero}{\ell_{0}}
\newcommand{\Lp}{\ell_{\mathrm{P}}}

% Zusätzliche Befehle
\newcommand{\Kfrak}{K_{\text{frak}}}
\newcommand{\Dfrak}{D_{\text{frak}}}
\newcommand{\betapar}{\beta_T}
\newcommand{\alphapar}{\alpha}
\newcommand{\deltafield}{\delta \phi}
\newcommand{\deltam}{\delta m}
\newcommand{\deltaE}{\delta E}
\newcommand{\Exi}{E_{\xi}}
\newcommand{\Lxi}{\ell_{\xi}}
\newcommand{\rhoCMB}{\rho_{\text{CMB}}}
\newcommand{\rhoCasimir}{\rho_{\text{Casimir}}}
\newcommand{\Leff}{L_{\text{eff}}}
\newcommand{\CQCD}{C_{\mathrm{QCD}}}
\newcommand{\Kspec}{K_{\mathrm{spec}}}

% Fehlende Befehle aus Dokumenten
\providecommand{\xiconst}{\xi_{\text{const}}}
\providecommand{\DhiggsT}{D_{\text{Higgs-T}}}
\providecommand{\rhoE}{\rho_{E}}
\providecommand{\Echar}{E_{\text{char}}}
\providecommand{\kfrac}{k_{\text{frac}}}
\providecommand{\alphaEMSI}{\alpha_{\text{EM,SI}}}
\providecommand{\alphaEMnat}{\alpha_{\text{EM,nat}}}
\providecommand{\betaTSI}{\beta_{T,\text{SI}}}
\providecommand{\betaTnat}{\beta_{T,\text{nat}}}
\providecommand{\Gsi}{G_{\text{SI}}}
\providecommand{\xiparSI}{\xi_{\text{SI}}}
\providecommand{\xiparnat}{\xi_{\text{nat}}}
\providecommand{\meff}{m_{\text{eff}}}
\providecommand{\Tzerot}{T_{0}(t)}
\providecommand{\mzerot}{m_{0}(t)}
\providecommand{\Ezeroabs}{E_{0,\text{abs}}}
\providecommand{\Epar}{E_{\text{par}}}
\providecommand{\Lnat}{\ell_{\text{nat}}}
\providecommand{\Tnat}{T_{\text{nat}}}
\providecommand{\xifrak}{\xi_{\text{frac}}}
\providecommand{\Tfrak}{T_{\text{frac}}}
\providecommand{\mfrak}{m_{\text{frac}}}
\providecommand{\Dfrac}{D_{\text{frac}}}
\providecommand{\EphotSI}{E_{\gamma,\text{SI}}}
\providecommand{\EphotNat}{E_{\gamma,\text{nat}}}
\providecommand{\Eabsint}{E_{\text{abs,int}}}
\providecommand{\mphoton}{m_{\gamma}}

% Zusätzliche fehlende Befehle aus Dokumenten
\providecommand{\Evis}{E_{\text{vis}}}
\providecommand{\Cto}{C_{T0}}
\providecommand{\mytimes}{\times}
\providecommand{\lambdah}{\lambda_h}
\providecommand{\checkmarkx}{\checkmark}
\providecommand{\Enorm}{E_{\text{norm}}}
\providecommand{\Tobs}{T_{\text{obs}}}
\providecommand{\mobs}{m_{\text{obs}}}
\providecommand{\Eobs}{E_{\text{obs}}}
\providecommand{\Lobs}{\ell_{\text{obs}}}
\providecommand{\xobs}{\xi_{\text{obs}}}
\providecommand{\calE}{\mathcal{E}}
\providecommand{\calT}{\mathcal{T}}
\providecommand{\calM}{\mathcal{M}}
\providecommand{\alphag}{\alpha_g}
\providecommand{\Tmax}{T_{\text{max}}}
\providecommand{\mmin}{m_{\text{min}}}
\providecommand{\Lmax}{\ell_{\text{max}}}
\providecommand{\Emin}{E_{\text{min}}}
\providecommand{\Geff}{G_{\text{eff}}}
\providecommand{\rhoeff}{\rho_{\text{eff}}}
\providecommand{\xieff}{\xi_{\text{eff}}}
\providecommand{\Teff}{T_{\text{eff}}}
\providecommand{\hPlanck}{h}
\providecommand{\kB}{k_B}
\providecommand{\muB}{\mu_B}
\providecommand{\lambdaC}{\lambda_C}
\providecommand{\omegaP}{\omega_P}
\providecommand{\rhoP}{\rho_P}
\providecommand{\Tref}{T_{\text{ref}}}
\providecommand{\Eref}{E_{\text{ref}}}
\providecommand{\mref}{m_{\text{ref}}}
\providecommand{\Lref}{\ell_{\text{ref}}}

% --- tcolorbox Stile ---
\tcbset{
    keyresult/.style={
        colback=blue!5!white,
        colframe=blue!75!black,
        title=Kernaussage,
        fonttitle=\bfseries
    },
    foundation/.style={
        colback=green!5!white,
        colframe=green!75!black,
        title=Grundlage,
        fonttitle=\bfseries
    },
    alternative/.style={
        colback=orange!5!white,
        colframe=orange!75!black,
        title=Alternative,
        fonttitle=\bfseries
    },
    warningbox/.style={
        colback=red!5!white,
        colframe=red!75!black,
        title=Warnung,
        fonttitle=\bfseries
    }
}

\newtcolorbox{keyresultbox}[1][]{colback=blue!5!white,colframe=blue!75!black,fonttitle=\bfseries,title={#1},breakable}
\newtcolorbox{keyresult}[1][Kernaussage]{colback=blue!5!white,colframe=blue!75!black,fonttitle=\bfseries,title={#1},breakable}
\newtcolorbox{foundationbox}[1][]{colback=green!5!white,colframe=green!75!black,fonttitle=\bfseries,title={#1},breakable}
\newtcolorbox{foundation}[1][Grundlage]{colback=green!5!white,colframe=green!75!black,fonttitle=\bfseries,title={#1},breakable}
\newtcolorbox{alternativebox}[1][]{colback=orange!5!white,colframe=orange!75!black,fonttitle=\bfseries,title={#1},breakable}
\newtcolorbox{warningboxenv}[1][]{colback=red!5!white,colframe=red!75!black,fonttitle=\bfseries,title={#1},breakable}

% Benutzerdefinierte Boxen für Formeln
\newtcolorbox{fundamental}[1][]{
    colback=boxgray,
    colframe=t0blue,
    fonttitle=\bfseries,
    title=#1,
    sharp corners,
    boxrule=2pt
}

\newtcolorbox{neueperspektive}[1][]{
    colback=red!5!white,
    colframe=t0red,
    fonttitle=\bfseries,
    title=#1,
    sharp corners,
    boxrule=2pt
}

\newtcolorbox{formula}[1][]{
    colback=blue!5!white,
    colframe=blue!75!black,
    fonttitle=\bfseries,
    title=#1
}

\newtcolorbox{result}[1][]{
    colback=green!5!white,
    colframe=green!75!black,
    fonttitle=\bfseries,
    title=#1
}

% Zusätzliche tcolorbox-Umgebungen (aus T0_standalone_header_de.tex)
\newtcolorbox{derivation}[1][]{
    colback=green!5!white,
    colframe=green!75!black,
    title=#1,
    fonttitle=\bfseries,
    breakable
}

\newtcolorbox{summary}[1][]{
    colback=gray!10!white,
    colframe=gray!75!black,
    title=#1,
    fonttitle=\bfseries,
    breakable
}

\newtcolorbox{comparison}[1][]{
    colback=purple!5!white,
    colframe=purple!75!black,
    title=#1,
    fonttitle=\bfseries,
    breakable
}

\newtcolorbox{relation}[1][]{
    colback=cyan!5!white,
    colframe=cyan!75!black,
    title=#1,
    fonttitle=\bfseries,
    breakable
}

\newtcolorbox{principleBox}[1][]{
    colback=yellow!5!white,
    colframe=yellow!75!black,
    title=#1,
    fonttitle=\bfseries,
    breakable
}

% Hinweis: insight und discovery sind als Theorem-Umgebungen definiert
% insightBox und discoveryBox für tcolorbox-Versionen
\newtcolorbox{insightBox}[1][]{colback=blue!5,colframe=t0blue,title={#1},fonttitle=\bfseries,breakable}
\newtcolorbox{discoveryBox}[1][]{colback=green!5,colframe=t0green,title={#1},fonttitle=\bfseries,breakable}
\newtcolorbox{newperspective}[1][]{colback=yellow!5,colframe=orange,title={#1},fonttitle=\bfseries,breakable}
\newtcolorbox{revelation}[1][]{colback=red!5,colframe=t0red,title={#1},fonttitle=\bfseries,breakable}
\newtcolorbox{keypoint}[1][]{colback=blue!5,colframe=t0blue,title={#1},fonttitle=\bfseries,breakable}
\newtcolorbox{evidenceBox}[1][]{colback=green!5,colframe=t0green,title={#1},fonttitle=\bfseries,breakable}
\newtcolorbox{conclusionBox}[1][]{colback=gray!5,colframe=gray,title={#1},fonttitle=\bfseries,breakable}
\newtcolorbox{significance}[1][]{colback=yellow!5,colframe=orange,title={#1},fonttitle=\bfseries,breakable}
\newtcolorbox{philosophical}[1][]{colback=purple!5,colframe=purple,title={#1},fonttitle=\bfseries,breakable}
\newtcolorbox{implicationBox}[1][]{colback=cyan!5,colframe=cyan,title={#1},fonttitle=\bfseries,breakable}
\newtcolorbox{perspectiveBox}[1][]{colback=blue!5,colframe=t0blue,title={#1},fonttitle=\bfseries,breakable}
\newtcolorbox{revolutionary}[1][]{colback=red!5,colframe=t0red,title={#1},fonttitle=\bfseries,breakable}
\newtcolorbox{technical}[1][]{colback=gray!5,colframe=gray!75!black,title={#1},fonttitle=\bfseries,breakable}
\newtcolorbox{technicalBox}[1][]{colback=gray!5,colframe=gray!75!black,title={#1},fonttitle=\bfseries,breakable}
\newtcolorbox{notationBox}[1][]{colback=yellow!5,colframe=yellow!75!black,title={#1},fonttitle=\bfseries,breakable}
\newtcolorbox{verification}[1][]{colback=orange!5!white,colframe=orange!75!black,fonttitle=\bfseries,title=#1}
\newtcolorbox{explanationBox}[1][]{colback=purple!5!white,colframe=purple!75!black,fonttitle=\bfseries,title=#1}
\newtcolorbox{interpretationBox}[1][]{colback=cyan!5!white,colframe=cyan!75!black,fonttitle=\bfseries,title=#1}
\newtcolorbox{explanation}[1][]{colback=purple!5!white,colframe=purple!75!black,fonttitle=\bfseries,title=#1,breakable}
\newtcolorbox{interpretation}[1][]{colback=cyan!5!white,colframe=cyan!75!black,fonttitle=\bfseries,title=#1,breakable}
\newtcolorbox{proof_step}[1][]{colback=gray!5!white,colframe=gray!75!black,fonttitle=\bfseries,title=#1,breakable}
\newtcolorbox{experimental}[1][]{colback=teal!5!white,colframe=teal!75!black,fonttitle=\bfseries,title=#1,breakable}

% Zusätzliche Umgebungen
\newenvironment{treatise}{\begin{quote}}{\end{quote}}
\newenvironment{gemeinsam}{\begin{quote}}{\end{quote}}
\newenvironment{vergleich}{\begin{quote}}{\end{quote}}
\newenvironment{vorteil}{\begin{quote}}{\end{quote}}
\newenvironment{quantum}{\begin{quote}}{\end{quote}}

% Fehlende tcolorbox-Umgebungen
\newtcolorbox{important}[1][]{colback=red!5!white,colframe=red!75!black,title={#1},fonttitle=\bfseries,breakable}
\newtcolorbox{warning}[1][]{colback=orange!5!white,colframe=orange!75!black,title={#1},fonttitle=\bfseries,breakable}
\newtcolorbox{caution}[1][]{colback=yellow!5!white,colframe=yellow!75!black,title={#1},fonttitle=\bfseries,breakable}
\newtcolorbox{highlight}[1][]{colback=yellow!10!white,colframe=yellow!75!black,title={#1},fonttitle=\bfseries,breakable}
\newtcolorbox{critical}[1][]{colback=red!10!white,colframe=red!75!black,title={#1},fonttitle=\bfseries,breakable}
\newtcolorbox{analysis}[1][]{colback=blue!5!white,colframe=blue!75!black,title={#1},fonttitle=\bfseries,breakable}
\newtcolorbox{application}[1][]{colback=green!5!white,colframe=green!75!black,title={#1},fonttitle=\bfseries,breakable}
\newtcolorbox{experiment}[1][]{colback=cyan!5!white,colframe=cyan!75!black,title={#1},fonttitle=\bfseries,breakable}
\newtcolorbox{historical}[1][]{colback=brown!5!white,colframe=brown!75!black,title={#1},fonttitle=\bfseries,breakable}
\newtcolorbox{numerical}[1][]{colback=gray!5!white,colframe=gray!75!black,title={#1},fonttitle=\bfseries,breakable}
\newtcolorbox{overview}[1][]{colback=blue!5!white,colframe=blue!75!black,title={#1},fonttitle=\bfseries,breakable}
\newtcolorbox{speculation}[1][]{colback=purple!5!white,colframe=purple!75!black,title={#1},fonttitle=\bfseries,breakable}
\newtcolorbox{question}[1][]{colback=orange!5!white,colframe=orange!75!black,title={#1},fonttitle=\bfseries,breakable}
\newtcolorbox{method}[1][]{colback=teal!5!white,colframe=teal!75!black,title={#1},fonttitle=\bfseries,breakable}
\newtcolorbox{correct}[1][]{colback=green!10!white,colframe=green!75!black,title={#1},fonttitle=\bfseries,breakable}
\newtcolorbox{units}[1][]{colback=gray!5!white,colframe=gray!75!black,title={#1},fonttitle=\bfseries,breakable}
\newtcolorbox{achievement}[1][]{colback=gold!5!white,colframe=orange!75!black,title={#1},fonttitle=\bfseries,breakable}
\newtcolorbox{equivalence}[1][]{colback=cyan!5!white,colframe=cyan!75!black,title={#1},fonttitle=\bfseries,breakable}
\newtcolorbox{dimensional}[1][]{colback=purple!5!white,colframe=purple!75!black,title={#1},fonttitle=\bfseries,breakable}
\newtcolorbox{photon}[1][]{colback=yellow!5!white,colframe=yellow!75!black,title={#1},fonttitle=\bfseries,breakable}
\newtcolorbox{neutrino}[1][]{colback=blue!5!white,colframe=blue!75!black,title={#1},fonttitle=\bfseries,breakable}
\newtcolorbox{revolution}[1][]{colback=red!5!white,colframe=red!75!black,title={#1},fonttitle=\bfseries,breakable}
\newtcolorbox{t0box}[1][]{colback=blue!5!white,colframe=t0blue,title={#1},fonttitle=\bfseries,breakable}
\newtcolorbox{documentbox}[1][]{colback=gray!5!white,colframe=gray!75!black,title={#1},fonttitle=\bfseries,breakable}
\newtcolorbox{sibox}[1][]{colback=green!5!white,colframe=green!75!black,title={#1},fonttitle=\bfseries,breakable}
\newtcolorbox{smbox}[1][]{colback=blue!5!white,colframe=blue!75!black,title={#1},fonttitle=\bfseries,breakable}
\newtcolorbox{pvbox}[1][]{colback=purple!5!white,colframe=purple!75!black,title={#1},fonttitle=\bfseries,breakable}
\newtcolorbox{koidebox}[1][]{colback=orange!5!white,colframe=orange!75!black,title={#1},fonttitle=\bfseries,breakable}
\newtcolorbox{formel}[1][]{colback=blue!5!white,colframe=blue!75!black,title={#1},fonttitle=\bfseries,breakable}
\newtcolorbox{schluessel}[1][]{colback=blue!5!white,colframe=blue!75!black,title={#1},fonttitle=\bfseries,breakable}
\newtcolorbox{wichtig}[1][]{colback=red!5!white,colframe=red!75!black,title={#1},fonttitle=\bfseries,breakable}
\newtcolorbox{vorsicht}[1][]{colback=orange!5!white,colframe=orange!75!black,title={#1},fonttitle=\bfseries,breakable}
\newtcolorbox{revolutionaer}[1][]{colback=red!5!white,colframe=red!75!black,title={#1},fonttitle=\bfseries,breakable}
\newtcolorbox{numerisch}[1][]{colback=gray!5!white,colframe=gray!75!black,title={#1},fonttitle=\bfseries,breakable}
\newtcolorbox{experimentell}[1][]{colback=cyan!5!white,colframe=cyan!75!black,title={#1},fonttitle=\bfseries,breakable}
\newtcolorbox{anwendung}[1][]{colback=green!5!white,colframe=green!75!black,title={#1},fonttitle=\bfseries,breakable}
\newtcolorbox{alternative}[1][]{colback=orange!5!white,colframe=orange!75!black,title={#1},fonttitle=\bfseries,breakable}
\newtcolorbox{beziehung}[1][]{colback=cyan!5!white,colframe=cyan!75!black,title={#1},fonttitle=\bfseries,breakable}
\newtcolorbox{folgerung}[1][]{colback=green!5!white,colframe=green!75!black,title={#1},fonttitle=\bfseries,breakable}
\newtcolorbox{abhandlung}[1][]{colback=gray!5!white,colframe=gray!75!black,title={#1},fonttitle=\bfseries,breakable}
\newtcolorbox{prinzipBox}[1][]{colback=blue!5!white,colframe=blue!75!black,title={#1},fonttitle=\bfseries,breakable}
\newtcolorbox{beweis}[1][]{colback=gray!5!white,colframe=gray!75!black,title={#1},fonttitle=\bfseries,breakable}
\newtcolorbox{key}[2][]{colback=blue!5!white,colframe=blue!75!black,title={#2},fonttitle=\bfseries,breakable}
\newtcolorbox{category}[1][]{colback=purple!5!white,colframe=purple!75!black,title={#1},fonttitle=\bfseries,breakable}

% Zusätzliche T0-spezifische Befehle
\newcommand{\Tzero}{T$_0$}
\providecommand{\meff}{m_{\text{eff}}}
\newcommand{\Eabs}{E_{\text{abs}}}
\newcommand{\taupar}{\tau}

% Missing commands from various documents
\providecommand{\xikonst}{\xi_0}
\providecommand{\Phiphoton}{\Phi_{\gamma}}
\providecommand{\etavis}{\eta_{\text{vis}}}
\providecommand{\pichar}{\pi}
\providecommand{\primrel}{\mathcal{P}_{\text{rel}}}
\providecommand{\warningx}{\textcolor{orange}{\textbf{!}}}
\providecommand{\phiT}{\phi_T}
\providecommand{\xiT}{\xi_T}
\providecommand{\Lorentz}{\Lambda}
\providecommand{\Cconv}{C_{\text{conv}}}
\providecommand{\Df}{\Delta f}
\providecommand{\lambdazero}{\lambda_0}
\providecommand{\myapprox}{\approx}
\providecommand{\checked}{\checkmark}
\providecommand{\alphaWSI}{\alpha_W^{\text{SI}}}
\providecommand{\alphaWnat}{\alpha_W^{\text{nat}}}
\providecommand{\vect}[1]{\vec{#1}}
\providecommand{\Rzero}{R_0}
\providecommand{\Riem}{\mathcal{R}}
\providecommand{\nuzero}{\nu_0}
\providecommand{\mypi}{\pi}

% --- Layout-Einstellungen ---
\sloppy
\hfuzz=2pt
\vfuzz=2pt
\tolerance=1000
\emergencystretch=3em
\raggedbottom

% --- Inhaltsverzeichnis-Formatierung ---
\renewcommand{\cftsecfont}{\color{blue}}
\renewcommand{\cftsubsecfont}{\color{blue}}
\renewcommand{\cftsecpagefont}{\color{blue}}
\renewcommand{\cftsubsecpagefont}{\color{blue}}
\renewcommand{\cfttoctitlefont}{\huge\bfseries\color{blue}}

% --- Standard Kopf- und Fußzeilen ---
\pagestyle{fancy}
\fancyhf{}
\fancyhead[L]{\textsc{T0-Theorie}}
\fancyhead[R]{\textsc{J. Pascher}}
\fancyfoot[C]{\thepage}

% ==============================================================================
% Ende der Präambel
% ==============================================================================

 nach \documentclass.
% ==============================================================================

% --- Kodierung und Sprache ---
\usepackage[utf8]{inputenc}
\usepackage[T1]{fontenc}
\usepackage[ngerman]{babel}
\usepackage{lmodern}

% --- Seitengeometrie ---
\usepackage[a4paper, margin=2.5cm]{geometry}
\setlength{\headheight}{15pt}

% --- Mathematik und Physik ---
\usepackage{amsmath,amssymb,amsfonts,amsthm}
\usepackage{mathtools}
\usepackage{physics}
\usepackage{siunitx}
\sisetup{
    locale=DE,
    group-separator={.},
    output-decimal-marker={,},
    per-mode=symbol
}

% --- Grafiken und Tabellen ---
\usepackage{graphicx}
\usepackage[table,xcdraw]{xcolor}
\usepackage{tikz}
\usetikzlibrary{arrows.meta,positioning,shapes.geometric,decorations.pathmorphing,patterns,shapes.arrows,intersections}
\usepackage{pgfplots}
\pgfplotsset{compat=1.18}
\usepackage{quantikz}
\usepackage[most]{tcolorbox}
\tcbuselibrary{breakable}

% === WICHTIG: Algorithm-Konflikt umgehen ===
% Option: algorithmic mit GROSSBUCHSTABEN
% Gemeinsame Box für Experimente
\newtcolorbox{experimentbox}[1][]{
	colback=green!5!white,
	colframe=t0green!80!black,
	fonttitle=\bfseries,
	title={{#1}},
	breakable
}

% Abstract-Fallback
\ifdefined\abstract\else
\newenvironment{abstract}{\section*{\abstractname}\itshape\small\par\bigskip}{\bigskip}
\fi

% === MAKROS SICHER NEU DEFINIEREN / ÜBERSCHREIBEN ===
% Definiere Makros OHNE doppelte Subskripte
\newcommand{\phipar}{\phi_{\mathrm{par}}}
%\newcommand{\xipar}{\xi_{\mathrm{par}}}
\newcommand{\Qphipar}{Q_{\phi_{\mathrm{par}}}}
\newcommand{\rphipar}{r_{\phi_{\mathrm{par}}}}
\newcommand{\logphipar}{\log_{\phi_{\mathrm{par}}}}
\newcommand{\CHSH}{\text{CHSH}}
\usepackage{booktabs}
\usepackage{array}
\usepackage{longtable}
\usepackage{float}
\usepackage{adjustbox}
\usepackage{tabularx}
\usepackage{multirow}

% --- Dokumentformatierung ---
\usepackage{fancyhdr}
\renewcommand{\headrulewidth}{0.4pt}
\renewcommand{\footrulewidth}{0.4pt}
\usepackage{tocloft}
\usepackage{hyperref}
\usepackage{bookmark}
\usepackage{cleveref}
\usepackage{microtype}
\usepackage{enumitem}
\usepackage{setspace}
\usepackage{ragged2e}
\usepackage{multicol}

% --- Code und Algorithmen ---
\usepackage{algorithm}
\usepackage{algorithmic}
\usepackage{listings}
\usepackage{mdframed}

% --- Zitationsbefehle (Kompatibilität) ---
\providecommand{\citep}[1]{\cite{#1}}
\providecommand{\citet}[1]{\cite{#1}}

% --- Zusätzliche Pakete ---
\usepackage{pdflscape}
\usepackage{braket}
\usepackage{cancel}
\usepackage{caption}
\usepackage{csquotes}
\usepackage{gensymb}
\usepackage{hyphenat}
\usepackage{textcomp}
\usepackage{textgreek}
\usepackage{upgreek}
\usepackage{url}
% Hyphenation for URLs in bibliography
\def\UrlBreaks{\do\/\do-}
\usepackage{slashed}
\usepackage{bm}

% --- Fehlende Farben definieren ---
\definecolor{gold}{RGB}{255,215,0}

% --- Spaltentypen ---
\newcolumntype{L}[1]{>{\raggedright\arraybackslash}p{#1}}
\newcolumntype{C}[1]{>{\centering\arraybackslash}p{#1}}

% --- Unicode-Zeichen ---
\usepackage{newunicodechar}
\newunicodechar{ħ}{$\hbar$}
\newunicodechar{↔}{$\leftrightarrow$}
\newunicodechar{⇐}{$\Leftarrow$}
\newunicodechar{⇒}{$\Rightarrow$}
\newunicodechar{⇔}{$\Leftrightarrow$}
\newunicodechar{∂}{$\partial$}
\newunicodechar{∅}{$\emptyset$}
\newunicodechar{∇}{$\nabla$}
\newunicodechar{∈}{$\in$}
\newunicodechar{∉}{$\notin$}
\newunicodechar{∏}{$\prod$}
\newunicodechar{∑}{$\sum$}
\newunicodechar{√}{$\sqrt{}$}
\newunicodechar{∝}{$\propto$}
\newunicodechar{∞}{$\infty$}
\newunicodechar{∩}{$\cap$}
\newunicodechar{∪}{$\cup$}
\newunicodechar{∫}{$\int$}
\newunicodechar{≈}{$\approx$}
\newunicodechar{≠}{$\neq$}
\newunicodechar{≤}{$\leq$}
\newunicodechar{≥}{$\geq$}
\newunicodechar{ξ}{\ensuremath{\xi}}
\newunicodechar{μ}{\ensuremath{\mu}}
\newunicodechar{ψ}{\ensuremath{\psi}}
\newunicodechar{φ}{\ensuremath{\phi}}
\newunicodechar{π}{\ensuremath{\pi}}
\newunicodechar{λ}{\ensuremath{\lambda}}
\newunicodechar{Δ}{\ensuremath{\Delta}}

% --- Farben ---
\definecolor{blue}{rgb}{0,0,1}
\definecolor{boxgray}{RGB}{240,240,240}
\definecolor{deepblue}{RGB}{0,0,127}
\definecolor{deepgreen}{RGB}{0,127,0}
\definecolor{deepred}{RGB}{191,0,0}
\definecolor{t0blue}{RGB}{33,150,243}
\definecolor{t0green}{RGB}{76,175,80}
\definecolor{t0orange}{RGB}{255,152,0}
\definecolor{t0purple}{RGB}{156,39,176}
\definecolor{t0red}{RGB}{244,67,54}
\definecolor{t0yellow}{RGB}{255,204,0}

% --- Hyperref-Einstellungen ---
\hypersetup{
    colorlinks=true,
    linkcolor=blue,
    citecolor=blue,
    urlcolor=blue,
    breaklinks=true,
    bookmarksnumbered=true,
    pdfstartview=FitH
}

% --- Theorem-Umgebungen (Deutsch) ---
\theoremstyle{plain}
\newtheorem{satz}{Satz}[section]
\newtheorem{lemma}[satz]{Lemma}
\newtheorem{proposition}[satz]{Proposition}
\newtheorem{korollar}[satz]{Korollar}

\theoremstyle{definition}
\newtheorem{definition}[satz]{Definition}
\newtheorem{beispiel}[satz]{Beispiel}
\newtheorem{erkenntnis}[satz]{Erkenntnis}
\newtheorem{entdeckung}[satz]{Entdeckung}

\theoremstyle{remark}
\newtheorem{bemerkung}[satz]{Bemerkung}
\newtheorem{warnung}[satz]{Warnung}
\newtheorem{axiom}{Axiom}
\newtheorem{prinzip}{Prinzip}

% Aliases für englische Bezeichnungen
\newtheorem{theorem}[satz]{Theorem}
\newtheorem{corollary}[satz]{Corollary}
\newtheorem{remark}[satz]{Remark}
\newtheorem{example}[satz]{Example}
\newtheorem{insight}[satz]{Insight}
\newtheorem{discovery}[satz]{Discovery}
\newtheorem{principle}[satz]{Principle}

% --- T0-spezifische Befehle ---
\newcommand{\Tfield}{T(x,t)}
\providecommand{\Tfieldt}{T(\vec{x},t)}
\newcommand{\Efield}{E(x,t)}
\newcommand{\mfield}{m(x,t)}
\providecommand{\vecx}{\vec{x}}
\newcommand{\Lag}{\mathcal{L}}
\newcommand{\calL}{\mathcal{L}}
\newcommand{\alphaem}{\alpha}
\newcommand{\betaT}{\beta_T}
\newcommand{\xiT}{\xi}
\newcommand{\xipar}{\xi}
\newcommand{\Ezero}{E_0}
\newcommand{\EPlanck}{E_{\text{Pl}}}
\newcommand{\Mpl}{M_{\text{Pl}}}
\newcommand{\lP}{\ell_{\text{P}}}
\newcommand{\tP}{t_{\text{P}}}
\newcommand{\LPlanck}{\ell_{\text{Pl}}}
\newcommand{\TPlanck}{t_{\text{Pl}}}
\newcommand{\Gnat}{G_{\text{nat}}}
\newcommand{\alphaEM}{\alpha_{\text{EM}}}
\newcommand{\alphaSI}{\alpha_{\text{SI}}}
\newcommand{\Hubble}{H_0}
\newcommand{\LCDM}{\Lambda\text{CDM}}
\newcommand{\natunits}{(nat. Einheiten)}

% T0 Modell Parameter
\newcommand{\xigeom}{\xi_{\mathrm{geom}}}
\newcommand{\rzero}{r_{0}}
\newcommand{\xirat}{\xi_{\mathrm{rat}}}
\newcommand{\tzero}{t_{0}}
\newcommand{\Lambdat}{\Lambda_{\mathrm{t}}}
\newcommand{\EP}{E_{\mathrm{P}}}
\newcommand{\Emu}{E_{\mu}}
\newcommand{\Ee}{E_{e}}
\newcommand{\Etau}{E_{\tau}}
\newcommand{\alphafine}{\alpha_{\mathrm{fine}}}
\newcommand{\alphal}{\alpha_{\ell}}
\newcommand{\Lzero}{\ell_{0}}
\newcommand{\Lp}{\ell_{\mathrm{P}}}

% Zusätzliche Befehle
\newcommand{\Kfrak}{K_{\text{frak}}}
\newcommand{\Dfrak}{D_{\text{frak}}}
\newcommand{\betapar}{\beta_T}
\newcommand{\alphapar}{\alpha}
\newcommand{\deltafield}{\delta \phi}
\newcommand{\deltam}{\delta m}
\newcommand{\deltaE}{\delta E}
\newcommand{\Exi}{E_{\xi}}
\newcommand{\Lxi}{\ell_{\xi}}
\newcommand{\rhoCMB}{\rho_{\text{CMB}}}
\newcommand{\rhoCasimir}{\rho_{\text{Casimir}}}
\newcommand{\Leff}{L_{\text{eff}}}
\newcommand{\CQCD}{C_{\mathrm{QCD}}}
\newcommand{\Kspec}{K_{\mathrm{spec}}}

% Fehlende Befehle aus Dokumenten
\providecommand{\xiconst}{\xi_{\text{const}}}
\providecommand{\DhiggsT}{D_{\text{Higgs-T}}}
\providecommand{\rhoE}{\rho_{E}}
\providecommand{\Echar}{E_{\text{char}}}
\providecommand{\kfrac}{k_{\text{frac}}}
\providecommand{\alphaEMSI}{\alpha_{\text{EM,SI}}}
\providecommand{\alphaEMnat}{\alpha_{\text{EM,nat}}}
\providecommand{\betaTSI}{\beta_{T,\text{SI}}}
\providecommand{\betaTnat}{\beta_{T,\text{nat}}}
\providecommand{\Gsi}{G_{\text{SI}}}
\providecommand{\xiparSI}{\xi_{\text{SI}}}
\providecommand{\xiparnat}{\xi_{\text{nat}}}
\providecommand{\meff}{m_{\text{eff}}}
\providecommand{\Tzerot}{T_{0}(t)}
\providecommand{\mzerot}{m_{0}(t)}
\providecommand{\Ezeroabs}{E_{0,\text{abs}}}
\providecommand{\Epar}{E_{\text{par}}}
\providecommand{\Lnat}{\ell_{\text{nat}}}
\providecommand{\Tnat}{T_{\text{nat}}}
\providecommand{\xifrak}{\xi_{\text{frac}}}
\providecommand{\Tfrak}{T_{\text{frac}}}
\providecommand{\mfrak}{m_{\text{frac}}}
\providecommand{\Dfrac}{D_{\text{frac}}}
\providecommand{\EphotSI}{E_{\gamma,\text{SI}}}
\providecommand{\EphotNat}{E_{\gamma,\text{nat}}}
\providecommand{\Eabsint}{E_{\text{abs,int}}}
\providecommand{\mphoton}{m_{\gamma}}

% Zusätzliche fehlende Befehle aus Dokumenten
\providecommand{\Evis}{E_{\text{vis}}}
\providecommand{\Cto}{C_{T0}}
\providecommand{\mytimes}{\times}
\providecommand{\lambdah}{\lambda_h}
\providecommand{\checkmarkx}{\checkmark}
\providecommand{\Enorm}{E_{\text{norm}}}
\providecommand{\Tobs}{T_{\text{obs}}}
\providecommand{\mobs}{m_{\text{obs}}}
\providecommand{\Eobs}{E_{\text{obs}}}
\providecommand{\Lobs}{\ell_{\text{obs}}}
\providecommand{\xobs}{\xi_{\text{obs}}}
\providecommand{\calE}{\mathcal{E}}
\providecommand{\calT}{\mathcal{T}}
\providecommand{\calM}{\mathcal{M}}
\providecommand{\alphag}{\alpha_g}
\providecommand{\Tmax}{T_{\text{max}}}
\providecommand{\mmin}{m_{\text{min}}}
\providecommand{\Lmax}{\ell_{\text{max}}}
\providecommand{\Emin}{E_{\text{min}}}
\providecommand{\Geff}{G_{\text{eff}}}
\providecommand{\rhoeff}{\rho_{\text{eff}}}
\providecommand{\xieff}{\xi_{\text{eff}}}
\providecommand{\Teff}{T_{\text{eff}}}
\providecommand{\hPlanck}{h}
\providecommand{\kB}{k_B}
\providecommand{\muB}{\mu_B}
\providecommand{\lambdaC}{\lambda_C}
\providecommand{\omegaP}{\omega_P}
\providecommand{\rhoP}{\rho_P}
\providecommand{\Tref}{T_{\text{ref}}}
\providecommand{\Eref}{E_{\text{ref}}}
\providecommand{\mref}{m_{\text{ref}}}
\providecommand{\Lref}{\ell_{\text{ref}}}

% --- tcolorbox Stile ---
\tcbset{
    keyresult/.style={
        colback=blue!5!white,
        colframe=blue!75!black,
        title=Kernaussage,
        fonttitle=\bfseries
    },
    foundation/.style={
        colback=green!5!white,
        colframe=green!75!black,
        title=Grundlage,
        fonttitle=\bfseries
    },
    alternative/.style={
        colback=orange!5!white,
        colframe=orange!75!black,
        title=Alternative,
        fonttitle=\bfseries
    },
    warningbox/.style={
        colback=red!5!white,
        colframe=red!75!black,
        title=Warnung,
        fonttitle=\bfseries
    }
}

\newtcolorbox{keyresultbox}[1][]{colback=blue!5!white,colframe=blue!75!black,fonttitle=\bfseries,title={#1},breakable}
\newtcolorbox{keyresult}[1][Kernaussage]{colback=blue!5!white,colframe=blue!75!black,fonttitle=\bfseries,title={#1},breakable}
\newtcolorbox{foundationbox}[1][]{colback=green!5!white,colframe=green!75!black,fonttitle=\bfseries,title={#1},breakable}
\newtcolorbox{foundation}[1][Grundlage]{colback=green!5!white,colframe=green!75!black,fonttitle=\bfseries,title={#1},breakable}
\newtcolorbox{alternativebox}[1][]{colback=orange!5!white,colframe=orange!75!black,fonttitle=\bfseries,title={#1},breakable}
\newtcolorbox{warningboxenv}[1][]{colback=red!5!white,colframe=red!75!black,fonttitle=\bfseries,title={#1},breakable}

% Benutzerdefinierte Boxen für Formeln
\newtcolorbox{fundamental}[1][]{
    colback=boxgray,
    colframe=t0blue,
    fonttitle=\bfseries,
    title=#1,
    sharp corners,
    boxrule=2pt
}

\newtcolorbox{neueperspektive}[1][]{
    colback=red!5!white,
    colframe=t0red,
    fonttitle=\bfseries,
    title=#1,
    sharp corners,
    boxrule=2pt
}

\newtcolorbox{formula}[1][]{
    colback=blue!5!white,
    colframe=blue!75!black,
    fonttitle=\bfseries,
    title=#1
}

\newtcolorbox{result}[1][]{
    colback=green!5!white,
    colframe=green!75!black,
    fonttitle=\bfseries,
    title=#1
}

% Zusätzliche tcolorbox-Umgebungen (aus T0_standalone_header_de.tex)
\newtcolorbox{derivation}[1][]{
    colback=green!5!white,
    colframe=green!75!black,
    title=#1,
    fonttitle=\bfseries,
    breakable
}

\newtcolorbox{summary}[1][]{
    colback=gray!10!white,
    colframe=gray!75!black,
    title=#1,
    fonttitle=\bfseries,
    breakable
}

\newtcolorbox{comparison}[1][]{
    colback=purple!5!white,
    colframe=purple!75!black,
    title=#1,
    fonttitle=\bfseries,
    breakable
}

\newtcolorbox{relation}[1][]{
    colback=cyan!5!white,
    colframe=cyan!75!black,
    title=#1,
    fonttitle=\bfseries,
    breakable
}

\newtcolorbox{principleBox}[1][]{
    colback=yellow!5!white,
    colframe=yellow!75!black,
    title=#1,
    fonttitle=\bfseries,
    breakable
}

% Hinweis: insight und discovery sind als Theorem-Umgebungen definiert
% insightBox und discoveryBox für tcolorbox-Versionen
\newtcolorbox{insightBox}[1][]{colback=blue!5,colframe=t0blue,title={#1},fonttitle=\bfseries,breakable}
\newtcolorbox{discoveryBox}[1][]{colback=green!5,colframe=t0green,title={#1},fonttitle=\bfseries,breakable}
\newtcolorbox{newperspective}[1][]{colback=yellow!5,colframe=orange,title={#1},fonttitle=\bfseries,breakable}
\newtcolorbox{revelation}[1][]{colback=red!5,colframe=t0red,title={#1},fonttitle=\bfseries,breakable}
\newtcolorbox{keypoint}[1][]{colback=blue!5,colframe=t0blue,title={#1},fonttitle=\bfseries,breakable}
\newtcolorbox{evidenceBox}[1][]{colback=green!5,colframe=t0green,title={#1},fonttitle=\bfseries,breakable}
\newtcolorbox{conclusionBox}[1][]{colback=gray!5,colframe=gray,title={#1},fonttitle=\bfseries,breakable}
\newtcolorbox{significance}[1][]{colback=yellow!5,colframe=orange,title={#1},fonttitle=\bfseries,breakable}
\newtcolorbox{philosophical}[1][]{colback=purple!5,colframe=purple,title={#1},fonttitle=\bfseries,breakable}
\newtcolorbox{implicationBox}[1][]{colback=cyan!5,colframe=cyan,title={#1},fonttitle=\bfseries,breakable}
\newtcolorbox{perspectiveBox}[1][]{colback=blue!5,colframe=t0blue,title={#1},fonttitle=\bfseries,breakable}
\newtcolorbox{revolutionary}[1][]{colback=red!5,colframe=t0red,title={#1},fonttitle=\bfseries,breakable}
\newtcolorbox{technical}[1][]{colback=gray!5,colframe=gray!75!black,title={#1},fonttitle=\bfseries,breakable}
\newtcolorbox{technicalBox}[1][]{colback=gray!5,colframe=gray!75!black,title={#1},fonttitle=\bfseries,breakable}
\newtcolorbox{notationBox}[1][]{colback=yellow!5,colframe=yellow!75!black,title={#1},fonttitle=\bfseries,breakable}
\newtcolorbox{verification}[1][]{colback=orange!5!white,colframe=orange!75!black,fonttitle=\bfseries,title=#1}
\newtcolorbox{explanationBox}[1][]{colback=purple!5!white,colframe=purple!75!black,fonttitle=\bfseries,title=#1}
\newtcolorbox{interpretationBox}[1][]{colback=cyan!5!white,colframe=cyan!75!black,fonttitle=\bfseries,title=#1}
\newtcolorbox{explanation}[1][]{colback=purple!5!white,colframe=purple!75!black,fonttitle=\bfseries,title=#1,breakable}
\newtcolorbox{interpretation}[1][]{colback=cyan!5!white,colframe=cyan!75!black,fonttitle=\bfseries,title=#1,breakable}
\newtcolorbox{proof_step}[1][]{colback=gray!5!white,colframe=gray!75!black,fonttitle=\bfseries,title=#1,breakable}
\newtcolorbox{experimental}[1][]{colback=teal!5!white,colframe=teal!75!black,fonttitle=\bfseries,title=#1,breakable}

% Zusätzliche Umgebungen
\newenvironment{treatise}{\begin{quote}}{\end{quote}}
\newenvironment{gemeinsam}{\begin{quote}}{\end{quote}}
\newenvironment{vergleich}{\begin{quote}}{\end{quote}}
\newenvironment{vorteil}{\begin{quote}}{\end{quote}}
\newenvironment{quantum}{\begin{quote}}{\end{quote}}

% Fehlende tcolorbox-Umgebungen
\newtcolorbox{important}[1][]{colback=red!5!white,colframe=red!75!black,title={#1},fonttitle=\bfseries,breakable}
\newtcolorbox{warning}[1][]{colback=orange!5!white,colframe=orange!75!black,title={#1},fonttitle=\bfseries,breakable}
\newtcolorbox{caution}[1][]{colback=yellow!5!white,colframe=yellow!75!black,title={#1},fonttitle=\bfseries,breakable}
\newtcolorbox{highlight}[1][]{colback=yellow!10!white,colframe=yellow!75!black,title={#1},fonttitle=\bfseries,breakable}
\newtcolorbox{critical}[1][]{colback=red!10!white,colframe=red!75!black,title={#1},fonttitle=\bfseries,breakable}
\newtcolorbox{analysis}[1][]{colback=blue!5!white,colframe=blue!75!black,title={#1},fonttitle=\bfseries,breakable}
\newtcolorbox{application}[1][]{colback=green!5!white,colframe=green!75!black,title={#1},fonttitle=\bfseries,breakable}
\newtcolorbox{experiment}[1][]{colback=cyan!5!white,colframe=cyan!75!black,title={#1},fonttitle=\bfseries,breakable}
\newtcolorbox{historical}[1][]{colback=brown!5!white,colframe=brown!75!black,title={#1},fonttitle=\bfseries,breakable}
\newtcolorbox{numerical}[1][]{colback=gray!5!white,colframe=gray!75!black,title={#1},fonttitle=\bfseries,breakable}
\newtcolorbox{overview}[1][]{colback=blue!5!white,colframe=blue!75!black,title={#1},fonttitle=\bfseries,breakable}
\newtcolorbox{speculation}[1][]{colback=purple!5!white,colframe=purple!75!black,title={#1},fonttitle=\bfseries,breakable}
\newtcolorbox{question}[1][]{colback=orange!5!white,colframe=orange!75!black,title={#1},fonttitle=\bfseries,breakable}
\newtcolorbox{method}[1][]{colback=teal!5!white,colframe=teal!75!black,title={#1},fonttitle=\bfseries,breakable}
\newtcolorbox{correct}[1][]{colback=green!10!white,colframe=green!75!black,title={#1},fonttitle=\bfseries,breakable}
\newtcolorbox{units}[1][]{colback=gray!5!white,colframe=gray!75!black,title={#1},fonttitle=\bfseries,breakable}
\newtcolorbox{achievement}[1][]{colback=gold!5!white,colframe=orange!75!black,title={#1},fonttitle=\bfseries,breakable}
\newtcolorbox{equivalence}[1][]{colback=cyan!5!white,colframe=cyan!75!black,title={#1},fonttitle=\bfseries,breakable}
\newtcolorbox{dimensional}[1][]{colback=purple!5!white,colframe=purple!75!black,title={#1},fonttitle=\bfseries,breakable}
\newtcolorbox{photon}[1][]{colback=yellow!5!white,colframe=yellow!75!black,title={#1},fonttitle=\bfseries,breakable}
\newtcolorbox{neutrino}[1][]{colback=blue!5!white,colframe=blue!75!black,title={#1},fonttitle=\bfseries,breakable}
\newtcolorbox{revolution}[1][]{colback=red!5!white,colframe=red!75!black,title={#1},fonttitle=\bfseries,breakable}
\newtcolorbox{t0box}[1][]{colback=blue!5!white,colframe=t0blue,title={#1},fonttitle=\bfseries,breakable}
\newtcolorbox{documentbox}[1][]{colback=gray!5!white,colframe=gray!75!black,title={#1},fonttitle=\bfseries,breakable}
\newtcolorbox{sibox}[1][]{colback=green!5!white,colframe=green!75!black,title={#1},fonttitle=\bfseries,breakable}
\newtcolorbox{smbox}[1][]{colback=blue!5!white,colframe=blue!75!black,title={#1},fonttitle=\bfseries,breakable}
\newtcolorbox{pvbox}[1][]{colback=purple!5!white,colframe=purple!75!black,title={#1},fonttitle=\bfseries,breakable}
\newtcolorbox{koidebox}[1][]{colback=orange!5!white,colframe=orange!75!black,title={#1},fonttitle=\bfseries,breakable}
\newtcolorbox{formel}[1][]{colback=blue!5!white,colframe=blue!75!black,title={#1},fonttitle=\bfseries,breakable}
\newtcolorbox{schluessel}[1][]{colback=blue!5!white,colframe=blue!75!black,title={#1},fonttitle=\bfseries,breakable}
\newtcolorbox{wichtig}[1][]{colback=red!5!white,colframe=red!75!black,title={#1},fonttitle=\bfseries,breakable}
\newtcolorbox{vorsicht}[1][]{colback=orange!5!white,colframe=orange!75!black,title={#1},fonttitle=\bfseries,breakable}
\newtcolorbox{revolutionaer}[1][]{colback=red!5!white,colframe=red!75!black,title={#1},fonttitle=\bfseries,breakable}
\newtcolorbox{numerisch}[1][]{colback=gray!5!white,colframe=gray!75!black,title={#1},fonttitle=\bfseries,breakable}
\newtcolorbox{experimentell}[1][]{colback=cyan!5!white,colframe=cyan!75!black,title={#1},fonttitle=\bfseries,breakable}
\newtcolorbox{anwendung}[1][]{colback=green!5!white,colframe=green!75!black,title={#1},fonttitle=\bfseries,breakable}
\newtcolorbox{alternative}[1][]{colback=orange!5!white,colframe=orange!75!black,title={#1},fonttitle=\bfseries,breakable}
\newtcolorbox{beziehung}[1][]{colback=cyan!5!white,colframe=cyan!75!black,title={#1},fonttitle=\bfseries,breakable}
\newtcolorbox{folgerung}[1][]{colback=green!5!white,colframe=green!75!black,title={#1},fonttitle=\bfseries,breakable}
\newtcolorbox{abhandlung}[1][]{colback=gray!5!white,colframe=gray!75!black,title={#1},fonttitle=\bfseries,breakable}
\newtcolorbox{prinzipBox}[1][]{colback=blue!5!white,colframe=blue!75!black,title={#1},fonttitle=\bfseries,breakable}
\newtcolorbox{beweis}[1][]{colback=gray!5!white,colframe=gray!75!black,title={#1},fonttitle=\bfseries,breakable}
\newtcolorbox{key}[2][]{colback=blue!5!white,colframe=blue!75!black,title={#2},fonttitle=\bfseries,breakable}
\newtcolorbox{category}[1][]{colback=purple!5!white,colframe=purple!75!black,title={#1},fonttitle=\bfseries,breakable}

% Zusätzliche T0-spezifische Befehle
\newcommand{\Tzero}{T$_0$}
\providecommand{\meff}{m_{\text{eff}}}
\newcommand{\Eabs}{E_{\text{abs}}}
\newcommand{\taupar}{\tau}

% Missing commands from various documents
\providecommand{\xikonst}{\xi_0}
\providecommand{\Phiphoton}{\Phi_{\gamma}}
\providecommand{\etavis}{\eta_{\text{vis}}}
\providecommand{\pichar}{\pi}
\providecommand{\primrel}{\mathcal{P}_{\text{rel}}}
\providecommand{\warningx}{\textcolor{orange}{\textbf{!}}}
\providecommand{\phiT}{\phi_T}
\providecommand{\xiT}{\xi_T}
\providecommand{\Lorentz}{\Lambda}
\providecommand{\Cconv}{C_{\text{conv}}}
\providecommand{\Df}{\Delta f}
\providecommand{\lambdazero}{\lambda_0}
\providecommand{\myapprox}{\approx}
\providecommand{\checked}{\checkmark}
\providecommand{\alphaWSI}{\alpha_W^{\text{SI}}}
\providecommand{\alphaWnat}{\alpha_W^{\text{nat}}}
\providecommand{\vect}[1]{\vec{#1}}
\providecommand{\Rzero}{R_0}
\providecommand{\Riem}{\mathcal{R}}
\providecommand{\nuzero}{\nu_0}
\providecommand{\mypi}{\pi}

% --- Layout-Einstellungen ---
\sloppy
\hfuzz=2pt
\vfuzz=2pt
\tolerance=1000
\emergencystretch=3em
\raggedbottom

% --- Inhaltsverzeichnis-Formatierung ---
\renewcommand{\cftsecfont}{\color{blue}}
\renewcommand{\cftsubsecfont}{\color{blue}}
\renewcommand{\cftsecpagefont}{\color{blue}}
\renewcommand{\cftsubsecpagefont}{\color{blue}}
\renewcommand{\cfttoctitlefont}{\huge\bfseries\color{blue}}

% --- Standard Kopf- und Fußzeilen ---
\pagestyle{fancy}
\fancyhf{}
\fancyhead[L]{\textsc{T0-Theorie}}
\fancyhead[R]{\textsc{J. Pascher}}
\fancyfoot[C]{\thepage}

% ==============================================================================
% Ende der Präambel
% ==============================================================================

 nach \documentclass.
% ==============================================================================

% --- Kodierung und Sprache ---
\usepackage[utf8]{inputenc}
\usepackage[T1]{fontenc}
\usepackage[ngerman]{babel}
\usepackage{lmodern}

% --- Seitengeometrie ---
\usepackage[a4paper, margin=2.5cm]{geometry}
\setlength{\headheight}{15pt}

% --- Mathematik und Physik ---
\usepackage{amsmath,amssymb,amsfonts,amsthm}
\usepackage{mathtools}
\usepackage{physics}
\usepackage{siunitx}
\sisetup{
    locale=DE,
    group-separator={.},
    output-decimal-marker={,},
    per-mode=symbol
}

% --- Grafiken und Tabellen ---
\usepackage{graphicx}
\usepackage[table,xcdraw]{xcolor}
\usepackage{tikz}
\usetikzlibrary{arrows.meta,positioning,shapes.geometric,decorations.pathmorphing,patterns,shapes.arrows,intersections}
\usepackage{pgfplots}
\pgfplotsset{compat=1.18}
\usepackage{quantikz}
\usepackage[most]{tcolorbox}
\tcbuselibrary{breakable}

% === WICHTIG: Algorithm-Konflikt umgehen ===
% Option: algorithmic mit GROSSBUCHSTABEN
% Gemeinsame Box für Experimente
\newtcolorbox{experimentbox}[1][]{
	colback=green!5!white,
	colframe=t0green!80!black,
	fonttitle=\bfseries,
	title={{#1}},
	breakable
}

% Abstract-Fallback
\ifdefined\abstract\else
\newenvironment{abstract}{\section*{\abstractname}\itshape\small\par\bigskip}{\bigskip}
\fi

% === MAKROS SICHER NEU DEFINIEREN / ÜBERSCHREIBEN ===
% Definiere Makros OHNE doppelte Subskripte
\newcommand{\phipar}{\phi_{\mathrm{par}}}
%\newcommand{\xipar}{\xi_{\mathrm{par}}}
\newcommand{\Qphipar}{Q_{\phi_{\mathrm{par}}}}
\newcommand{\rphipar}{r_{\phi_{\mathrm{par}}}}
\newcommand{\logphipar}{\log_{\phi_{\mathrm{par}}}}
\newcommand{\CHSH}{\text{CHSH}}
\usepackage{booktabs}
\usepackage{array}
\usepackage{longtable}
\usepackage{float}
\usepackage{adjustbox}
\usepackage{tabularx}
\usepackage{multirow}

% --- Dokumentformatierung ---
\usepackage{fancyhdr}
\renewcommand{\headrulewidth}{0.4pt}
\renewcommand{\footrulewidth}{0.4pt}
\usepackage{tocloft}
\usepackage{hyperref}
\usepackage{bookmark}
\usepackage{cleveref}
\usepackage{microtype}
\usepackage{enumitem}
\usepackage{setspace}
\usepackage{ragged2e}
\usepackage{multicol}

% --- Code und Algorithmen ---
\usepackage{algorithm}
\usepackage{algorithmic}
\usepackage{listings}
\usepackage{mdframed}

% --- Zitationsbefehle (Kompatibilität) ---
\providecommand{\citep}[1]{\cite{#1}}
\providecommand{\citet}[1]{\cite{#1}}

% --- Zusätzliche Pakete ---
\usepackage{pdflscape}
\usepackage{braket}
\usepackage{cancel}
\usepackage{caption}
\usepackage{csquotes}
\usepackage{gensymb}
\usepackage{hyphenat}
\usepackage{textcomp}
\usepackage{textgreek}
\usepackage{upgreek}
\usepackage{url}
% Hyphenation for URLs in bibliography
\def\UrlBreaks{\do\/\do-}
\usepackage{slashed}
\usepackage{bm}

% --- Fehlende Farben definieren ---
\definecolor{gold}{RGB}{255,215,0}

% --- Spaltentypen ---
\newcolumntype{L}[1]{>{\raggedright\arraybackslash}p{#1}}
\newcolumntype{C}[1]{>{\centering\arraybackslash}p{#1}}

% --- Unicode-Zeichen ---
\usepackage{newunicodechar}
\newunicodechar{ħ}{$\hbar$}
\newunicodechar{↔}{$\leftrightarrow$}
\newunicodechar{⇐}{$\Leftarrow$}
\newunicodechar{⇒}{$\Rightarrow$}
\newunicodechar{⇔}{$\Leftrightarrow$}
\newunicodechar{∂}{$\partial$}
\newunicodechar{∅}{$\emptyset$}
\newunicodechar{∇}{$\nabla$}
\newunicodechar{∈}{$\in$}
\newunicodechar{∉}{$\notin$}
\newunicodechar{∏}{$\prod$}
\newunicodechar{∑}{$\sum$}
\newunicodechar{√}{$\sqrt{}$}
\newunicodechar{∝}{$\propto$}
\newunicodechar{∞}{$\infty$}
\newunicodechar{∩}{$\cap$}
\newunicodechar{∪}{$\cup$}
\newunicodechar{∫}{$\int$}
\newunicodechar{≈}{$\approx$}
\newunicodechar{≠}{$\neq$}
\newunicodechar{≤}{$\leq$}
\newunicodechar{≥}{$\geq$}
\newunicodechar{ξ}{\ensuremath{\xi}}
\newunicodechar{μ}{\ensuremath{\mu}}
\newunicodechar{ψ}{\ensuremath{\psi}}
\newunicodechar{φ}{\ensuremath{\phi}}
\newunicodechar{π}{\ensuremath{\pi}}
\newunicodechar{λ}{\ensuremath{\lambda}}
\newunicodechar{Δ}{\ensuremath{\Delta}}

% --- Farben ---
\definecolor{blue}{rgb}{0,0,1}
\definecolor{boxgray}{RGB}{240,240,240}
\definecolor{deepblue}{RGB}{0,0,127}
\definecolor{deepgreen}{RGB}{0,127,0}
\definecolor{deepred}{RGB}{191,0,0}
\definecolor{t0blue}{RGB}{33,150,243}
\definecolor{t0green}{RGB}{76,175,80}
\definecolor{t0orange}{RGB}{255,152,0}
\definecolor{t0purple}{RGB}{156,39,176}
\definecolor{t0red}{RGB}{244,67,54}
\definecolor{t0yellow}{RGB}{255,204,0}

% --- Hyperref-Einstellungen ---
\hypersetup{
    colorlinks=true,
    linkcolor=blue,
    citecolor=blue,
    urlcolor=blue,
    breaklinks=true,
    bookmarksnumbered=true,
    pdfstartview=FitH
}

% --- Theorem-Umgebungen (Deutsch) ---
\theoremstyle{plain}
\newtheorem{satz}{Satz}[section]
\newtheorem{lemma}[satz]{Lemma}
\newtheorem{proposition}[satz]{Proposition}
\newtheorem{korollar}[satz]{Korollar}

\theoremstyle{definition}
\newtheorem{definition}[satz]{Definition}
\newtheorem{beispiel}[satz]{Beispiel}
\newtheorem{erkenntnis}[satz]{Erkenntnis}
\newtheorem{entdeckung}[satz]{Entdeckung}

\theoremstyle{remark}
\newtheorem{bemerkung}[satz]{Bemerkung}
\newtheorem{warnung}[satz]{Warnung}
\newtheorem{axiom}{Axiom}
\newtheorem{prinzip}{Prinzip}

% Aliases für englische Bezeichnungen
\newtheorem{theorem}[satz]{Theorem}
\newtheorem{corollary}[satz]{Corollary}
\newtheorem{remark}[satz]{Remark}
\newtheorem{example}[satz]{Example}
\newtheorem{insight}[satz]{Insight}
\newtheorem{discovery}[satz]{Discovery}
\newtheorem{principle}[satz]{Principle}

% --- T0-spezifische Befehle ---
\newcommand{\Tfield}{T(x,t)}
\providecommand{\Tfieldt}{T(\vec{x},t)}
\newcommand{\Efield}{E(x,t)}
\newcommand{\mfield}{m(x,t)}
\providecommand{\vecx}{\vec{x}}
\newcommand{\Lag}{\mathcal{L}}
\newcommand{\calL}{\mathcal{L}}
\newcommand{\alphaem}{\alpha}
\newcommand{\betaT}{\beta_T}
\newcommand{\xiT}{\xi}
\newcommand{\xipar}{\xi}
\newcommand{\Ezero}{E_0}
\newcommand{\EPlanck}{E_{\text{Pl}}}
\newcommand{\Mpl}{M_{\text{Pl}}}
\newcommand{\lP}{\ell_{\text{P}}}
\newcommand{\tP}{t_{\text{P}}}
\newcommand{\LPlanck}{\ell_{\text{Pl}}}
\newcommand{\TPlanck}{t_{\text{Pl}}}
\newcommand{\Gnat}{G_{\text{nat}}}
\newcommand{\alphaEM}{\alpha_{\text{EM}}}
\newcommand{\alphaSI}{\alpha_{\text{SI}}}
\newcommand{\Hubble}{H_0}
\newcommand{\LCDM}{\Lambda\text{CDM}}
\newcommand{\natunits}{(nat. Einheiten)}

% T0 Modell Parameter
\newcommand{\xigeom}{\xi_{\mathrm{geom}}}
\newcommand{\rzero}{r_{0}}
\newcommand{\xirat}{\xi_{\mathrm{rat}}}
\newcommand{\tzero}{t_{0}}
\newcommand{\Lambdat}{\Lambda_{\mathrm{t}}}
\newcommand{\EP}{E_{\mathrm{P}}}
\newcommand{\Emu}{E_{\mu}}
\newcommand{\Ee}{E_{e}}
\newcommand{\Etau}{E_{\tau}}
\newcommand{\alphafine}{\alpha_{\mathrm{fine}}}
\newcommand{\alphal}{\alpha_{\ell}}
\newcommand{\Lzero}{\ell_{0}}
\newcommand{\Lp}{\ell_{\mathrm{P}}}

% Zusätzliche Befehle
\newcommand{\Kfrak}{K_{\text{frak}}}
\newcommand{\Dfrak}{D_{\text{frak}}}
\newcommand{\betapar}{\beta_T}
\newcommand{\alphapar}{\alpha}
\newcommand{\deltafield}{\delta \phi}
\newcommand{\deltam}{\delta m}
\newcommand{\deltaE}{\delta E}
\newcommand{\Exi}{E_{\xi}}
\newcommand{\Lxi}{\ell_{\xi}}
\newcommand{\rhoCMB}{\rho_{\text{CMB}}}
\newcommand{\rhoCasimir}{\rho_{\text{Casimir}}}
\newcommand{\Leff}{L_{\text{eff}}}
\newcommand{\CQCD}{C_{\mathrm{QCD}}}
\newcommand{\Kspec}{K_{\mathrm{spec}}}

% Fehlende Befehle aus Dokumenten
\providecommand{\xiconst}{\xi_{\text{const}}}
\providecommand{\DhiggsT}{D_{\text{Higgs-T}}}
\providecommand{\rhoE}{\rho_{E}}
\providecommand{\Echar}{E_{\text{char}}}
\providecommand{\kfrac}{k_{\text{frac}}}
\providecommand{\alphaEMSI}{\alpha_{\text{EM,SI}}}
\providecommand{\alphaEMnat}{\alpha_{\text{EM,nat}}}
\providecommand{\betaTSI}{\beta_{T,\text{SI}}}
\providecommand{\betaTnat}{\beta_{T,\text{nat}}}
\providecommand{\Gsi}{G_{\text{SI}}}
\providecommand{\xiparSI}{\xi_{\text{SI}}}
\providecommand{\xiparnat}{\xi_{\text{nat}}}
\providecommand{\meff}{m_{\text{eff}}}
\providecommand{\Tzerot}{T_{0}(t)}
\providecommand{\mzerot}{m_{0}(t)}
\providecommand{\Ezeroabs}{E_{0,\text{abs}}}
\providecommand{\Epar}{E_{\text{par}}}
\providecommand{\Lnat}{\ell_{\text{nat}}}
\providecommand{\Tnat}{T_{\text{nat}}}
\providecommand{\xifrak}{\xi_{\text{frac}}}
\providecommand{\Tfrak}{T_{\text{frac}}}
\providecommand{\mfrak}{m_{\text{frac}}}
\providecommand{\Dfrac}{D_{\text{frac}}}
\providecommand{\EphotSI}{E_{\gamma,\text{SI}}}
\providecommand{\EphotNat}{E_{\gamma,\text{nat}}}
\providecommand{\Eabsint}{E_{\text{abs,int}}}
\providecommand{\mphoton}{m_{\gamma}}

% Zusätzliche fehlende Befehle aus Dokumenten
\providecommand{\Evis}{E_{\text{vis}}}
\providecommand{\Cto}{C_{T0}}
\providecommand{\mytimes}{\times}
\providecommand{\lambdah}{\lambda_h}
\providecommand{\checkmarkx}{\checkmark}
\providecommand{\Enorm}{E_{\text{norm}}}
\providecommand{\Tobs}{T_{\text{obs}}}
\providecommand{\mobs}{m_{\text{obs}}}
\providecommand{\Eobs}{E_{\text{obs}}}
\providecommand{\Lobs}{\ell_{\text{obs}}}
\providecommand{\xobs}{\xi_{\text{obs}}}
\providecommand{\calE}{\mathcal{E}}
\providecommand{\calT}{\mathcal{T}}
\providecommand{\calM}{\mathcal{M}}
\providecommand{\alphag}{\alpha_g}
\providecommand{\Tmax}{T_{\text{max}}}
\providecommand{\mmin}{m_{\text{min}}}
\providecommand{\Lmax}{\ell_{\text{max}}}
\providecommand{\Emin}{E_{\text{min}}}
\providecommand{\Geff}{G_{\text{eff}}}
\providecommand{\rhoeff}{\rho_{\text{eff}}}
\providecommand{\xieff}{\xi_{\text{eff}}}
\providecommand{\Teff}{T_{\text{eff}}}
\providecommand{\hPlanck}{h}
\providecommand{\kB}{k_B}
\providecommand{\muB}{\mu_B}
\providecommand{\lambdaC}{\lambda_C}
\providecommand{\omegaP}{\omega_P}
\providecommand{\rhoP}{\rho_P}
\providecommand{\Tref}{T_{\text{ref}}}
\providecommand{\Eref}{E_{\text{ref}}}
\providecommand{\mref}{m_{\text{ref}}}
\providecommand{\Lref}{\ell_{\text{ref}}}

% --- tcolorbox Stile ---
\tcbset{
    keyresult/.style={
        colback=blue!5!white,
        colframe=blue!75!black,
        title=Kernaussage,
        fonttitle=\bfseries
    },
    foundation/.style={
        colback=green!5!white,
        colframe=green!75!black,
        title=Grundlage,
        fonttitle=\bfseries
    },
    alternative/.style={
        colback=orange!5!white,
        colframe=orange!75!black,
        title=Alternative,
        fonttitle=\bfseries
    },
    warningbox/.style={
        colback=red!5!white,
        colframe=red!75!black,
        title=Warnung,
        fonttitle=\bfseries
    }
}

\newtcolorbox{keyresultbox}[1][]{colback=blue!5!white,colframe=blue!75!black,fonttitle=\bfseries,title={#1},breakable}
\newtcolorbox{keyresult}[1][Kernaussage]{colback=blue!5!white,colframe=blue!75!black,fonttitle=\bfseries,title={#1},breakable}
\newtcolorbox{foundationbox}[1][]{colback=green!5!white,colframe=green!75!black,fonttitle=\bfseries,title={#1},breakable}
\newtcolorbox{foundation}[1][Grundlage]{colback=green!5!white,colframe=green!75!black,fonttitle=\bfseries,title={#1},breakable}
\newtcolorbox{alternativebox}[1][]{colback=orange!5!white,colframe=orange!75!black,fonttitle=\bfseries,title={#1},breakable}
\newtcolorbox{warningboxenv}[1][]{colback=red!5!white,colframe=red!75!black,fonttitle=\bfseries,title={#1},breakable}

% Benutzerdefinierte Boxen für Formeln
\newtcolorbox{fundamental}[1][]{
    colback=boxgray,
    colframe=t0blue,
    fonttitle=\bfseries,
    title=#1,
    sharp corners,
    boxrule=2pt
}

\newtcolorbox{neueperspektive}[1][]{
    colback=red!5!white,
    colframe=t0red,
    fonttitle=\bfseries,
    title=#1,
    sharp corners,
    boxrule=2pt
}

\newtcolorbox{formula}[1][]{
    colback=blue!5!white,
    colframe=blue!75!black,
    fonttitle=\bfseries,
    title=#1
}

\newtcolorbox{result}[1][]{
    colback=green!5!white,
    colframe=green!75!black,
    fonttitle=\bfseries,
    title=#1
}

% Zusätzliche tcolorbox-Umgebungen (aus T0_standalone_header_de.tex)
\newtcolorbox{derivation}[1][]{
    colback=green!5!white,
    colframe=green!75!black,
    title=#1,
    fonttitle=\bfseries,
    breakable
}

\newtcolorbox{summary}[1][]{
    colback=gray!10!white,
    colframe=gray!75!black,
    title=#1,
    fonttitle=\bfseries,
    breakable
}

\newtcolorbox{comparison}[1][]{
    colback=purple!5!white,
    colframe=purple!75!black,
    title=#1,
    fonttitle=\bfseries,
    breakable
}

\newtcolorbox{relation}[1][]{
    colback=cyan!5!white,
    colframe=cyan!75!black,
    title=#1,
    fonttitle=\bfseries,
    breakable
}

\newtcolorbox{principleBox}[1][]{
    colback=yellow!5!white,
    colframe=yellow!75!black,
    title=#1,
    fonttitle=\bfseries,
    breakable
}

% Hinweis: insight und discovery sind als Theorem-Umgebungen definiert
% insightBox und discoveryBox für tcolorbox-Versionen
\newtcolorbox{insightBox}[1][]{colback=blue!5,colframe=t0blue,title={#1},fonttitle=\bfseries,breakable}
\newtcolorbox{discoveryBox}[1][]{colback=green!5,colframe=t0green,title={#1},fonttitle=\bfseries,breakable}
\newtcolorbox{newperspective}[1][]{colback=yellow!5,colframe=orange,title={#1},fonttitle=\bfseries,breakable}
\newtcolorbox{revelation}[1][]{colback=red!5,colframe=t0red,title={#1},fonttitle=\bfseries,breakable}
\newtcolorbox{keypoint}[1][]{colback=blue!5,colframe=t0blue,title={#1},fonttitle=\bfseries,breakable}
\newtcolorbox{evidenceBox}[1][]{colback=green!5,colframe=t0green,title={#1},fonttitle=\bfseries,breakable}
\newtcolorbox{conclusionBox}[1][]{colback=gray!5,colframe=gray,title={#1},fonttitle=\bfseries,breakable}
\newtcolorbox{significance}[1][]{colback=yellow!5,colframe=orange,title={#1},fonttitle=\bfseries,breakable}
\newtcolorbox{philosophical}[1][]{colback=purple!5,colframe=purple,title={#1},fonttitle=\bfseries,breakable}
\newtcolorbox{implicationBox}[1][]{colback=cyan!5,colframe=cyan,title={#1},fonttitle=\bfseries,breakable}
\newtcolorbox{perspectiveBox}[1][]{colback=blue!5,colframe=t0blue,title={#1},fonttitle=\bfseries,breakable}
\newtcolorbox{revolutionary}[1][]{colback=red!5,colframe=t0red,title={#1},fonttitle=\bfseries,breakable}
\newtcolorbox{technical}[1][]{colback=gray!5,colframe=gray!75!black,title={#1},fonttitle=\bfseries,breakable}
\newtcolorbox{technicalBox}[1][]{colback=gray!5,colframe=gray!75!black,title={#1},fonttitle=\bfseries,breakable}
\newtcolorbox{notationBox}[1][]{colback=yellow!5,colframe=yellow!75!black,title={#1},fonttitle=\bfseries,breakable}
\newtcolorbox{verification}[1][]{colback=orange!5!white,colframe=orange!75!black,fonttitle=\bfseries,title=#1}
\newtcolorbox{explanationBox}[1][]{colback=purple!5!white,colframe=purple!75!black,fonttitle=\bfseries,title=#1}
\newtcolorbox{interpretationBox}[1][]{colback=cyan!5!white,colframe=cyan!75!black,fonttitle=\bfseries,title=#1}
\newtcolorbox{explanation}[1][]{colback=purple!5!white,colframe=purple!75!black,fonttitle=\bfseries,title=#1,breakable}
\newtcolorbox{interpretation}[1][]{colback=cyan!5!white,colframe=cyan!75!black,fonttitle=\bfseries,title=#1,breakable}
\newtcolorbox{proof_step}[1][]{colback=gray!5!white,colframe=gray!75!black,fonttitle=\bfseries,title=#1,breakable}
\newtcolorbox{experimental}[1][]{colback=teal!5!white,colframe=teal!75!black,fonttitle=\bfseries,title=#1,breakable}

% Zusätzliche Umgebungen
\newenvironment{treatise}{\begin{quote}}{\end{quote}}
\newenvironment{gemeinsam}{\begin{quote}}{\end{quote}}
\newenvironment{vergleich}{\begin{quote}}{\end{quote}}
\newenvironment{vorteil}{\begin{quote}}{\end{quote}}
\newenvironment{quantum}{\begin{quote}}{\end{quote}}

% Fehlende tcolorbox-Umgebungen
\newtcolorbox{important}[1][]{colback=red!5!white,colframe=red!75!black,title={#1},fonttitle=\bfseries,breakable}
\newtcolorbox{warning}[1][]{colback=orange!5!white,colframe=orange!75!black,title={#1},fonttitle=\bfseries,breakable}
\newtcolorbox{caution}[1][]{colback=yellow!5!white,colframe=yellow!75!black,title={#1},fonttitle=\bfseries,breakable}
\newtcolorbox{highlight}[1][]{colback=yellow!10!white,colframe=yellow!75!black,title={#1},fonttitle=\bfseries,breakable}
\newtcolorbox{critical}[1][]{colback=red!10!white,colframe=red!75!black,title={#1},fonttitle=\bfseries,breakable}
\newtcolorbox{analysis}[1][]{colback=blue!5!white,colframe=blue!75!black,title={#1},fonttitle=\bfseries,breakable}
\newtcolorbox{application}[1][]{colback=green!5!white,colframe=green!75!black,title={#1},fonttitle=\bfseries,breakable}
\newtcolorbox{experiment}[1][]{colback=cyan!5!white,colframe=cyan!75!black,title={#1},fonttitle=\bfseries,breakable}
\newtcolorbox{historical}[1][]{colback=brown!5!white,colframe=brown!75!black,title={#1},fonttitle=\bfseries,breakable}
\newtcolorbox{numerical}[1][]{colback=gray!5!white,colframe=gray!75!black,title={#1},fonttitle=\bfseries,breakable}
\newtcolorbox{overview}[1][]{colback=blue!5!white,colframe=blue!75!black,title={#1},fonttitle=\bfseries,breakable}
\newtcolorbox{speculation}[1][]{colback=purple!5!white,colframe=purple!75!black,title={#1},fonttitle=\bfseries,breakable}
\newtcolorbox{question}[1][]{colback=orange!5!white,colframe=orange!75!black,title={#1},fonttitle=\bfseries,breakable}
\newtcolorbox{method}[1][]{colback=teal!5!white,colframe=teal!75!black,title={#1},fonttitle=\bfseries,breakable}
\newtcolorbox{correct}[1][]{colback=green!10!white,colframe=green!75!black,title={#1},fonttitle=\bfseries,breakable}
\newtcolorbox{units}[1][]{colback=gray!5!white,colframe=gray!75!black,title={#1},fonttitle=\bfseries,breakable}
\newtcolorbox{achievement}[1][]{colback=gold!5!white,colframe=orange!75!black,title={#1},fonttitle=\bfseries,breakable}
\newtcolorbox{equivalence}[1][]{colback=cyan!5!white,colframe=cyan!75!black,title={#1},fonttitle=\bfseries,breakable}
\newtcolorbox{dimensional}[1][]{colback=purple!5!white,colframe=purple!75!black,title={#1},fonttitle=\bfseries,breakable}
\newtcolorbox{photon}[1][]{colback=yellow!5!white,colframe=yellow!75!black,title={#1},fonttitle=\bfseries,breakable}
\newtcolorbox{neutrino}[1][]{colback=blue!5!white,colframe=blue!75!black,title={#1},fonttitle=\bfseries,breakable}
\newtcolorbox{revolution}[1][]{colback=red!5!white,colframe=red!75!black,title={#1},fonttitle=\bfseries,breakable}
\newtcolorbox{t0box}[1][]{colback=blue!5!white,colframe=t0blue,title={#1},fonttitle=\bfseries,breakable}
\newtcolorbox{documentbox}[1][]{colback=gray!5!white,colframe=gray!75!black,title={#1},fonttitle=\bfseries,breakable}
\newtcolorbox{sibox}[1][]{colback=green!5!white,colframe=green!75!black,title={#1},fonttitle=\bfseries,breakable}
\newtcolorbox{smbox}[1][]{colback=blue!5!white,colframe=blue!75!black,title={#1},fonttitle=\bfseries,breakable}
\newtcolorbox{pvbox}[1][]{colback=purple!5!white,colframe=purple!75!black,title={#1},fonttitle=\bfseries,breakable}
\newtcolorbox{koidebox}[1][]{colback=orange!5!white,colframe=orange!75!black,title={#1},fonttitle=\bfseries,breakable}
\newtcolorbox{formel}[1][]{colback=blue!5!white,colframe=blue!75!black,title={#1},fonttitle=\bfseries,breakable}
\newtcolorbox{schluessel}[1][]{colback=blue!5!white,colframe=blue!75!black,title={#1},fonttitle=\bfseries,breakable}
\newtcolorbox{wichtig}[1][]{colback=red!5!white,colframe=red!75!black,title={#1},fonttitle=\bfseries,breakable}
\newtcolorbox{vorsicht}[1][]{colback=orange!5!white,colframe=orange!75!black,title={#1},fonttitle=\bfseries,breakable}
\newtcolorbox{revolutionaer}[1][]{colback=red!5!white,colframe=red!75!black,title={#1},fonttitle=\bfseries,breakable}
\newtcolorbox{numerisch}[1][]{colback=gray!5!white,colframe=gray!75!black,title={#1},fonttitle=\bfseries,breakable}
\newtcolorbox{experimentell}[1][]{colback=cyan!5!white,colframe=cyan!75!black,title={#1},fonttitle=\bfseries,breakable}
\newtcolorbox{anwendung}[1][]{colback=green!5!white,colframe=green!75!black,title={#1},fonttitle=\bfseries,breakable}
\newtcolorbox{alternative}[1][]{colback=orange!5!white,colframe=orange!75!black,title={#1},fonttitle=\bfseries,breakable}
\newtcolorbox{beziehung}[1][]{colback=cyan!5!white,colframe=cyan!75!black,title={#1},fonttitle=\bfseries,breakable}
\newtcolorbox{folgerung}[1][]{colback=green!5!white,colframe=green!75!black,title={#1},fonttitle=\bfseries,breakable}
\newtcolorbox{abhandlung}[1][]{colback=gray!5!white,colframe=gray!75!black,title={#1},fonttitle=\bfseries,breakable}
\newtcolorbox{prinzipBox}[1][]{colback=blue!5!white,colframe=blue!75!black,title={#1},fonttitle=\bfseries,breakable}
\newtcolorbox{beweis}[1][]{colback=gray!5!white,colframe=gray!75!black,title={#1},fonttitle=\bfseries,breakable}
\newtcolorbox{key}[2][]{colback=blue!5!white,colframe=blue!75!black,title={#2},fonttitle=\bfseries,breakable}
\newtcolorbox{category}[1][]{colback=purple!5!white,colframe=purple!75!black,title={#1},fonttitle=\bfseries,breakable}

% Zusätzliche T0-spezifische Befehle
\newcommand{\Tzero}{T$_0$}
\providecommand{\meff}{m_{\text{eff}}}
\newcommand{\Eabs}{E_{\text{abs}}}
\newcommand{\taupar}{\tau}

% Missing commands from various documents
\providecommand{\xikonst}{\xi_0}
\providecommand{\Phiphoton}{\Phi_{\gamma}}
\providecommand{\etavis}{\eta_{\text{vis}}}
\providecommand{\pichar}{\pi}
\providecommand{\primrel}{\mathcal{P}_{\text{rel}}}
\providecommand{\warningx}{\textcolor{orange}{\textbf{!}}}
\providecommand{\phiT}{\phi_T}
\providecommand{\xiT}{\xi_T}
\providecommand{\Lorentz}{\Lambda}
\providecommand{\Cconv}{C_{\text{conv}}}
\providecommand{\Df}{\Delta f}
\providecommand{\lambdazero}{\lambda_0}
\providecommand{\myapprox}{\approx}
\providecommand{\checked}{\checkmark}
\providecommand{\alphaWSI}{\alpha_W^{\text{SI}}}
\providecommand{\alphaWnat}{\alpha_W^{\text{nat}}}
\providecommand{\vect}[1]{\vec{#1}}
\providecommand{\Rzero}{R_0}
\providecommand{\Riem}{\mathcal{R}}
\providecommand{\nuzero}{\nu_0}
\providecommand{\mypi}{\pi}

% --- Layout-Einstellungen ---
\sloppy
\hfuzz=2pt
\vfuzz=2pt
\tolerance=1000
\emergencystretch=3em
\raggedbottom

% --- Inhaltsverzeichnis-Formatierung ---
\renewcommand{\cftsecfont}{\color{blue}}
\renewcommand{\cftsubsecfont}{\color{blue}}
\renewcommand{\cftsecpagefont}{\color{blue}}
\renewcommand{\cftsubsecpagefont}{\color{blue}}
\renewcommand{\cfttoctitlefont}{\huge\bfseries\color{blue}}

% --- Standard Kopf- und Fußzeilen ---
\pagestyle{fancy}
\fancyhf{}
\fancyhead[L]{\textsc{T0-Theorie}}
\fancyhead[R]{\textsc{J. Pascher}}
\fancyfoot[C]{\thepage}

% ==============================================================================
% Ende der Präambel
% ==============================================================================



% Book-specific packages
\usepackage{titlesec}
\usepackage{fancyhdr}
\usepackage{tikz}
\usepackage{pdfpages}
\usetikzlibrary{calc,decorations.text,shapes.geometric}

% Chapter style
\titleformat{\chapter}[display]
  {\normalfont\huge\bfseries}{\chaptertitlename\ \thechapter}{20pt}{\Huge}
\titlespacing*{\chapter}{0pt}{50pt}{40pt}

% Header/Footer
\pagestyle{fancy}
\fancyhf{}
\fancyhead[LE,RO]{\thepage}
\fancyhead[LO]{\nouppercase{\rightmark}}
\fancyhead[RE]{\nouppercase{\leftmark}}
\renewcommand{\headrulewidth}{0.5pt}

% Title information
\title{\Huge\bfseries T0-Theorie:\\[0.5em] Zeit-Masse-Dualität\\[1em]
\Large Eine vollständige Ableitung aller Naturkonstanten\\aus der Feinstrukturkonstante $\alpha \approx 1/137$}
\author{Johann Pascher}
\date{2024}

\begin{document}

% ==========================================
% COVER PAGE - Using custom cover image
% ==========================================
\begin{titlepage}
\centering
\vspace*{-2cm}
\includegraphics[width=\paperwidth,height=\paperheight,keepaspectratio]{T0_deckblatt_De.png}
\end{titlepage}

% ==========================================
% FRONT MATTER
% ==========================================
\frontmatter

% Abstract
\chapter*{Abstrakt}
\addcontentsline{toc}{chapter}{Abstrakt}

Die T0-Theorie präsentiert einen fundamentalen Paradigmenwechsel in der theoretischen Physik: 
\textbf{Alle Naturkonstanten und physikalischen Parameter können aus einer einzigen 
dimensionslosen Zahl abgeleitet werden} -- der Feinstrukturkonstante $\alpha \approx 1/137$.

\begin{keyresult}[Zentrales Theorem]
In der T0-Theorie gilt die Zeit-Masse-Dualität:
\begin{equation}
T(x) \cdot m(x) = \xi = \frac{\hbar}{c^2} = \text{const}
\end{equation}
wobei $\xi$ die fundamentale Kopplungskonstante ist, die Zeit und Masse verbindet.
\end{keyresult}

\textbf{Kernaussagen der Dokumentensammlung:}
\begin{itemize}
\item Die Feinstrukturkonstante $\alpha$ ist der einzige freie Parameter der Physik
\item Alle anderen Konstanten ($G$, $\hbar$, $c$, $m_e$, $m_P$, etc.) folgen aus $\alpha$
\item Die Zeit-Masse-Dualität erklärt Gravitation ohne Raumkrümmung
\item Experimentelle Vorhersagen stimmen mit Beobachtungen überein
\end{itemize}

\tableofcontents

% ==========================================
% MAIN MATTER
% ==========================================
\mainmatter

% Part I: Einführung und Grundlagen
\part{Einführung und Grundlagen}

\chapter{Einführung in die T0-Theorie}
\input{chapters/ch_T0_Introduction_De}

\chapter{Die Reise zur geometrischen Dualität}
\input{chapters/ch_reise_De}

\chapter{Grundlagen der T0-Theorie}
\input{chapters/ch_T0_Grundlagen_De}

\chapter{Modellübersicht}
\input{chapters/ch_T0_Modell_Uebersicht_De}

\chapter{Sieben fundamentale Fragen}
\input{chapters/ch_T0_7-fragen-3_De}

% Part II: Konzeptuelle Vergleiche
\part{Konzeptuelle Vergleiche und Analysen}

\chapter{Hannah -- Eine metaphorische Einführung}
\input{chapters/ch_Hannah_De}

\chapter{Markov-Ketten und Physik}
\input{chapters/ch_Markov_De}

\chapter{T0-Theorie vs. Synergetik}
\input{chapters/ch_T0-Theory-vs-Synergetics_De}

\chapter{Drei-Uhren-Gedankenexperiment}
\input{chapters/ch_T0_threeclock_De}

\chapter{Penrose und die T0-Theorie}
\input{chapters/ch_T0_penrose_De}

\chapter{Peratt-Analyse}
\input{chapters/ch_T0_peratt_De}

\chapter{MNRAS Widerlegungsanalyse}
\input{chapters/ch_T0_Analyse_MNRAS_Widerlegung_De}

\chapter{T0 vs. Standardmodell}
\input{chapters/ch_T0vsESM_ConceptualAnalysis_De}

% Part III: Teilchenmassen und Parameter
\part{Teilchenmassen und fundamentale Parameter}

\chapter{T0-Teilchenmassen}
\input{chapters/ch_T0_Teilchenmassen_De}

\chapter{Teilchenmassen-Berechnung}
\input{chapters/ch_Teilchenmassen_De}

\chapter{Zeit-Masse-Erweiterung}
\input{chapters/ch_T0_tm-erweiterung-x6_De}

\chapter{Neutrinos in der T0-Theorie}
\input{chapters/ch_T0_Neutrinos_De}

\chapter{Detaillierte Leptonenformeln}
\input{chapters/ch_detailierte_formel_leptonen_anemal_De}

\chapter{Neutrino-Formel}
\input{chapters/ch_neutrino-Formel_De}

\chapter{Koide-Formel}
\input{chapters/ch_T0_koide-formel-3_De}

\chapter{Xi und Energie}
\input{chapters/ch_T0_xi-und-e_De}

\chapter{Ursprung von Xi}
\input{chapters/ch_T0_xi_ursprung_De}

\chapter{Xi-Parameter und Partikel}
\input{chapters/ch_xi_parmater_partikel_De}

% Part IV: Einheitensysteme
\part{Einheitensysteme und Konstanten}

\chapter{SI-Einheiten in der T0-Theorie}
\input{chapters/ch_T0_SI_De}

\chapter{Natürliche und SI-Einheiten}
\input{chapters/ch_T0_nat-si_De}

\chapter{Systematik natürlicher Einheiten}
\input{chapters/ch_NatEinheitenSystematikDe}

\chapter{Parameterherleitung}
\input{chapters/ch_parameterherleitung_De}

\chapter{Vollständige Berechnungen}
\input{chapters/ch_T0_Vollstaendige_Berchnungen_De}

\chapter{Verhältnis-Absolutwerte}
\input{chapters/ch_T0_verhaeltnis-absolut_De}

\chapter{Relatives Zahlensystem}
\input{chapters/ch_RelokativesZahlensystemDe}

% Part V: Energie und Masse
\part{Energie, Masse und E=mc²}

\chapter{E=mc² in der T0-Theorie}
\input{chapters/ch_E-mc2_De}

\chapter{T0-Energie}
\input{chapters/ch_T0_Energie_De}

\chapter{Energiebasierte Formeln}
\input{chapters/ch_Formeln_Energiebasiert_De}

\chapter{Bewegungsenergie}
\input{chapters/ch_Bewegungsenergie_De}

% Part VI: Feinstrukturkonstante
\part{Die Feinstrukturkonstante α}

\chapter{T0-Feinstruktur}
\input{chapters/ch_T0_Feinstruktur_De}

\chapter{Feinstrukturkonstante}
\input{chapters/ch_FeinstrukturkonstanteDe}

\chapter{Die Zahl 137}
\input{chapters/ch_137_De}

\chapter{Musikalische Spirale 137}
\input{chapters/ch_musical-spiral-137-De}

\chapter{Auflösung der Konstanten via Alpha}
\input{chapters/ch_ResolvingTheConstantsAlfaDe}

% Part VII: Gravitation
\part{Gravitationskonstante}

\chapter{T0-Gravitationskonstante}
\input{chapters/ch_T0_Gravitationskonstante_De}

\chapter{Gravitationskonstante Berechnung}
\input{chapters/ch_gravitationskonstante_De}

\chapter{Gravitationskonstante Analyse}
\input{chapters/ch_gravitationskonstnte_De}

% Part VIII: Kosmologie
\part{Kosmologie und CMB}

\chapter{Temperatureinheiten und CMB}
\input{chapters/ch_TempEinheitenCMBDe}

\chapter{T0-Kosmologie}
\input{chapters/ch_T0_Kosmologie_De}

\chapter{Kosmische Strukturen}
\input{chapters/ch_cosmic_De}

\chapter{Geometrische Kosmologie}
\input{chapters/ch_T0_Geometrische_Kosmologie_De}

\chapter{Rotverschiebung und Ablenkung}
\input{chapters/ch_redshift_deflection_De}

\chapter{Casimir-Effekt}
\input{chapters/ch_Casimir_De}

\chapter{Zwei-Dipol-CMB}
\input{chapters/ch_Zwei-Dipole-CMB_De}

\chapter{Hubble-Konstante}
\input{chapters/ch_Ho_De}

% Part IX: Anomale magnetische Momente
\part{Anomale magnetische Momente (g-2)}

\chapter{Anomale magnetische Momente}
\input{chapters/ch_T0_Anomale_Magnetische_Momente_De}

\chapter{Anomales g-2: Analyse 6}
\input{chapters/ch_T0_Anomale-g2-6_De}

\chapter{Anomales g-2: Analyse 9}
\input{chapters/ch_T0_Anomale-g2-9_De}

% Part X: Lagrange-Formalismus
\part{Lagrange-Formalismus und Feldtheorie}

\chapter{T0-Lagrangian}
\input{chapters/ch_T0_lagrndian_De}

\chapter{Lagrangian-Vergleich}
\input{chapters/ch_LagrandianVergleichDe}

\chapter{Einfacher Lagrangian}
\input{chapters/ch_lagrandian-einfachDe}

\chapter{Notwendigkeit zweier Lagrangians}
\input{chapters/ch_Notwendigkeit_zwei_lagrange_De}

\chapter{Dirac-Gleichung}
\input{chapters/ch_diracDe}

\chapter{Vereinfachte Dirac-Gleichung}
\input{chapters/ch_diracVereinfachtDe}

% Part XI: Quantenmechanik
\part{Quantenmechanik und Quantenfeldtheorie}

\chapter{QM-QFT-RT Verbindung}
\input{chapters/ch_T0_QM-QFT-RT_De}

\chapter{Quantenmechanik testen}
\input{chapters/ch_QM-testenDe}

\chapter{Bell-Ungleichungen}
% Chapter content extracted from Bell_De.tex
% For book integration

\begin{abstract}
		Diese Erweiterung der T0-Serie wendet Erkenntnisse aus vorherigen ML-Tests (Wasserstoff-Niveaus) auf Bell-Tests an, um Quantenverschränkung im T0-Rahmen zu modellieren. Basierend auf der Zeit-Masse-Dualität und $\xi = 4/30000$ werden Korrelationen $E(a,b) = -\cos(a-b) \cdot (1 - \xi \cdot f(n,l,j))$ modifiziert, wobei $f(n,l,j)$ aus T0-Quantenzahlen stammt. Ein PyTorch-NN (1→32→16→1, 200 Epochen) simuliert CHSH-Verletzungen mit T0-Dämpfung, ergibt eine Reduktion von 2.828 auf 2.827 (0.04 \% $\Delta$), was Lokalität bei $\xi$-Skala wiederherstellt. Neue Erkenntnisse: ML zeigt subtile nicht-lokale Effekte als emergente Zeitfeld-Fluktuationen; Divergenz bei hohen Winkeln deutet auf fraktale Pfad-Interferenz hin. Dies löst das EPR-Paradoxon harmonisch, ohne Bells Ungleichung zu verletzen – testbar via 2025-Loophole-free Experimente (z.\,B. 73-Qubit-Lie-Detector). Kaum Vorteile durch ML: Die harmonische T0-Berechnung ($\phi$-Skalierung) liefert bereits exakte Vorhersagen; ML kalibriert nur ($\sim$0.1 \% Genauigkeitsgewinn).
	\end{abstract}
	
	\newpage
	
	\subsubsection{Einführung: Bell-Tests im T0-Kontext}
	\label{sec:intro_bell}
	
	Bell-Tests testen Quantenverschränkung vs. lokale Realität: Standard-QM verletzt Bells Ungleichung (CHSH >2), implizierend Nicht-Lokalität (EPR-Paradoxon). T0 löst dies durch $\xi$-modifizierte Korrelationen: Zeitfeld-Fluktuationen dämpfen Verschränkung lokal, bewahrend Realismus. Basierend auf ML-Tests aus QM-Doc (Divergenz bei hohen $n$), simulieren wir hier CHSH mit T0-Korrekturen.
	
	\textbf{2025-Kontext:} Neueste Experimente (z.\,B. 73-Qubit-Lie-Detector, Oct 2025)\cite{sciencedaily2025} bestätigen QM-Verletzungen; T0 vorhersagt subtile Abweichungen ($\Delta \sim 10^{-4}$), testbar in Loophole-free Setups.
	
	Parameter: $\xi=4/30000$, $\phi \approx 1.618$; Quantenzahlen für Photonenpaare: $(n=1,l=0,j=1)$ (Photonen als Gen-1).
	
	\subsubsection{T0-Modifikation der Bell-Korrelationen}
	\label{sec:mod}
	
	Standard: $E(a,b) = -\cos(a-b)$ für Singulett-Zustand; CHSH = $E(a,b) - E(a,b') + E(a',b) + E(a',b') \approx 2\sqrt{2} \approx 2.828 >2$.
	
	T0: Zeitfeld dämpft: $E^{\mathrm{T0}}(a,b) = -\cos(a-b) \cdot (1 - \xi \cdot f(n,l,j))$, mit $f(n,l,j) = (n/\phi)^l \cdot [1 + \xi j / \pi] \approx 1$ (für Photonen). Dies reduziert CHSH auf $\approx 2.828 \cdot (1 - \xi) \approx 2.827$, knapp über 2 – Lokalität bei $\xi$-Präzision.
	
	\begin{equation}
		\mathrm{CHSH}^{\mathrm{T0}} = 2\sqrt{2} \cdot K_{\mathrm{frak}}^{D_f} \cdot (1 - \xi \cdot \Delta \theta / \pi),
		\label{eq:chsh_t0}
	\end{equation}
	wobei $\Delta \theta = |a-b|$ (Winkelunterschied), $D_f=3-\xi$.
	
	\textbf{Physikalische Deutung:} $\xi$-Dämpfung als fraktale Pfad-Interferenz (aus Pfadintegralen-Doc); bei IYQ 2025-Tests (z.\,B. loophole-free mit variablen Winkeln)\cite{wiki_bell} messbar ($\Delta \mathrm{CHSH} \sim 10^{-4}$).
	
	\subsubsection{ML-Simulation von Bell-Tests}
	\label{sec:ml_bell}
	
	Erweiterung der vorherigen ML-Tests: NN lernt T0-Korrelationen aus Winkeldifferenzen ($\Delta \theta$) und extrapoliert auf hohe Winkel (z.\,B. $\Delta \theta = 3\pi/4$). Setup: MSE-Loss auf $E^{\mathrm{T0}}(\Delta \theta)$; 200 Epochen.
	
	\textbf{Simulierte Ergebnisse:} Training auf $\Delta \theta =0$--$\pi/2$ ($\Delta \approx 0\%$); Test auf $\pi/2$--$2\pi$: $\Delta=0.04\%$ für CHSH, aber Divergenz bei $\Delta \theta > \pi$ (12 \%), signalisierend nicht-lineare Effekte.
	
	\begin{table}[h]
		\centering
		\begin{tabular}{lcccc}
			\toprule
			\textbf{$\Delta \theta$} & \textbf{Standard $E$} & \textbf{T0 $E$} & \textbf{ML-pred $E$} & \textbf{$\Delta$ ML vs. T0 (\%)} \\
			\midrule
			$\pi/4$ & -0.707 & -0.707 & -0.707 & 0.00 \\
			$\pi/2$ & 0.000 & 0.000 & 0.000 & 0.00 \\
			$3\pi/4$ & 0.707 & 0.707 & 0.707 & 0.00 \\
			$\pi$ & -1.000 & -1.000 & -1.000 & 0.00 \\
			$5\pi/4$ & -0.707 & -0.707 & -0.794 & 12.31 \\
			\bottomrule
		\end{tabular}
		\caption{ML-Simulation von Korrelationen: Divergenz bei hohen Winkeln deutet auf fraktale Grenzen.}
		\label{tab:bell_ml}
	\end{table}
	
	\textbf{CHSH-Berechnung:} Standard: 2.828; T0: 2.827; ML-pred: 2.828 ($\Delta=0.04\%$); bei erweitertem Test ($\Delta \theta > \pi$): ML-CHSH=2.812 ($\Delta=0.54\%$).
	
	\subsubsection{Nicht-lineare Effekte: Selbst abgeleitete Erkenntnisse}
	\label{sec:nonlin}
	
	Aus ML-Divergenz (12 \% bei $5\pi/4$): Lineare $\xi$-Dämpfung versagt; abgeleitet: Erweiterte Formel $E^{\mathrm{T0,ext}}(\Delta \theta) = -\cos(\Delta \theta) \cdot \exp(-\xi \cdot (\Delta \theta / \pi)^2 \cdot D_f^{-1})$, reduziert $\Delta$ auf $<0.1\%$ (simuliert).
	
	\begin{keyresult}
		\textbf{Erkenntnis 1: Fraktale Winkel-Dämpfung.} Divergenz signalisiert $K_{\mathrm{frak}}^{D_f \cdot (\Delta \theta)^2}$ – T0 stellt Lokalität her, indem Korrelationen bei $\Delta \theta > \pi$ klassisch werden ($\mathrm{CHSH}^{\mathrm{ext}} <2.5$).
	\end{keyresult}
	
	\begin{important}
		\textbf{Erkenntnis 2: ML als Signal für Emergenz.} NN lernt $\cos$-Form exakt, divergiert bei Grenzen – abgeleitet: Integriere in T0-QFT: Verschränkungsdichte $\rho^{\mathrm{T0}} = \rho \cdot (1 - \xi \cdot \Delta \theta / E_0)$, lösend EPR bei Planck-Skala.
	\end{important}
	
	\begin{warning}
		\textbf{Erkenntnis 3: Test für 2025-Experimente.} T0 vorhersagt $\Delta \mathrm{CHSH} \approx 10^{-4}$ in 73-Qubit-Tests\cite{sciencedaily2025}; ML-Fehler (0.54 \%) unterstreicht Bedarf an harmonischer Expansion – ML kaum Vorteil, enthüllt aber nicht-perturbative Pfade.
	\end{warning}

	
	\subsubsection{Ausblick: Integration in T0-Serie}
	
	Diese Bell-Erweiterung verbindet mit QFT-Doc (T0\_QM-QFT-RT): Modifizierte Feldoperatoren dämpfen Verschränkung lokal. Nächste: Simuliere EPR mit Neutrino-Suppression ($\xi^2$).
	
	\begin{summary}
		\textbf{Kernbotschaft:} T0 löst Nicht-Lokalität harmonisch – ML-Tests bestätigen subtile Dämpfung, gewinnen neue Terme (fraktale Winkel), ohne Kern zu ersetzen.
	\end{summary}
	
	\begin{center}
		\rule{0.8\textwidth}{0.4pt}
		\vspace{0.5cm}
		\textit{T0-Theorie: Bell-Tests als Test für Lokale Realität}\\
		\textit{Johann Pascher, HTL Leonding, Österreich}\\
		\textit{GitHub: \url{https://github.com/jpascher/T0-Time-Mass-Duality}}\\
		\vspace{0.3cm}
		\textit{Version 2.2 -- \today}
	\end{center}
	
	\begin{thebibliography}{9}
		\bibitem{iyq2025} International Year of Quantum (2025). \emph{About IYQ}. \url{https://quantum2025.org/about/}.
		\bibitem{nobel2025} Reuters (2025). \emph{Trio win Nobel for quantum physics in action}. 7. Oktober.
		\bibitem{decision2025} The Quantum Insider (2025). \emph{New Research on QM Decision-Making}. 25. Oktober.
		\bibitem{keysight2025} Keysight (2025). \emph{Joy of Quantum: IYQ Principles}. 22. September.
		\bibitem{sciencedaily2025} ScienceDaily (2025). \emph{Physicists just built a quantum lie detector}. 7. Oktober.
		\bibitem{wiki_bell} Wikipedia (2025). \emph{Bell's Theorem}. \url{https://en.wikipedia.org/wiki/Bell%27s_theorem}.
		\bibitem{pascher_t0} Pascher, J. (2025). \emph{T0-Serie: Massen, Neutrinos, g-2}. GitHub.
	\end{thebibliography}

\chapter{Deterministisches QM}
\input{chapters/ch_QM-DetrmisticDe}

\chapter{No-Go-Theoreme}
\input{chapters/ch_NoGoDe}

\chapter{Mathematische Struktur}
\input{chapters/ch_Mathematische_struktur_De}

\chapter{Systemanalyse}
\input{chapters/ch_systemDe}

\chapter{Quantenmechanik}
\input{chapters/ch_QM_De}

\chapter{Quantenfeldtheorie}
\input{chapters/ch_QFT_De}

\chapter{QFT-ML Addendum}
\input{chapters/ch_T0-QFT-ML_Addendum_De}

\chapter{Scheinbar instantan}
% Chapter content extracted from scheinbar_instantan_De.tex
% For book integration

\thispagestyle{empty}
	
	\begin{abstract}
		Diese Arbeit zeigt, dass die scheinbare Instantanität im T0-Formalismus durch die Notation der lokalen Zwangsbedingung $T \cdot E = 1$ entsteht. Durch die Analyse der zugrunde liegenden Feldgleichungen und der hierarchischen Zeitskalen wird demonstriert, dass die T0-Theorie eine vollständig kausale Beschreibung von Quantenphänomenen bietet, die mit der speziellen Relativitätstheorie vereinbar ist. Alle Parameter der Theorie folgen aus rein geometrischen Prinzipien. Die Arbeit erweitert die Analyse auf die vollständige Dualität zwischen Zeit, Masse, Energie und Länge und diskutiert kritisch die Grenzen der Interpretation bei Extremsituationen.
	\end{abstract}
	
	\newpage
	\hypersetup{linkcolor=blue}
	\newpage
	
	\subsubsection{Einleitung: Das Instantanitätsproblem}
	
	Seit den bahnbrechenden Arbeiten von Einstein, Podolsky und Rosen in den 1930er Jahren kämpft die Physik mit einem fundamentalen Paradoxon: Die Quantenmechanik scheint instantane Korrelationen zwischen beliebig weit entfernten Teilchen zu erfordern, was Einstein als spukhafte Fernwirkung bezeichnete. Diese scheinbare Instantanität manifestiert sich in verschiedenen Phänomenen - vom Kollaps der Wellenfunktion über die Verletzung der Bell'schen Ungleichungen bis hin zur Quantenverschränkung.
	
	Der T0-Formalismus bietet eine alternative Auflösung dieses Paradoxons. Die Kernidee besteht darin, dass die fundamentale Beziehung zwischen Zeit und Energie, ausgedrückt durch die Gleichung $T \cdot E = 1$, oft missverstanden wird. Was auf den ersten Blick wie eine instantane Kopplung aussieht, erweist sich bei genauerer Betrachtung als lokale Zwangsbedingung, die keine Fernwirkung impliziert.
	
	Um dies zu verstehen, müssen wir zwischen zwei fundamental verschiedenen Arten von physikalischen Beziehungen unterscheiden: lokalen Zwangsbedingungen, die am selben Raumpunkt gelten, und Feldgleichungen, die die Ausbreitung von Störungen durch den Raum beschreiben. Diese Unterscheidung ist der Schlüssel zur Auflösung des Instantanitätsparadoxons.
	
	\subsubsection{Die scheinbare Instantanität im T0-Formalismus}
	
	Die T0-Gleichsetzungen implizieren auf den ersten Blick Instantanität, was jedoch durch eine detaillierte Analyse der Feldgleichungen widerlegt wird. Die fundamentale Herausforderung besteht darin zu verstehen, wie eine Theorie, die auf der strikten Beziehung $T \cdot E = 1$ basiert, dennoch die Kausalität respektieren kann. Diese scheinbare Paradoxie hat ihre Wurzeln in einem Missverständnis über die Natur mathematischer Zwangsbedingungen in der Physik.
	
	\subsubsection{Das scheinbare Problem}
	
	Die grundlegenden Gleichungen des T0-Formalismus lauten:
	\begin{align}
		T(\mathbf{x},t) \cdot E(\mathbf{x},t) &= 1 \label{eq:TE_constraint} \\
		T &= \frac{1}{m} \quad \text{wobei } \omega = \frac{mc^2}{\hbar}, \text{ sodass } T = \frac{\hbar}{E} \label{eq:T_definition} \\
		E &= mc^2 \label{eq:E_definition}
	\end{align}
	
	Diese Gleichungen suggerieren, dass eine Änderung von $E$ eine sofortige Anpassung von $T$ erfordert. Wenn wir beispielsweise die Energie an einem Punkt verdoppeln, scheint das Zeitfeld sich instantan halbieren zu müssen. Diese Interpretation würde tatsächlich eine Verletzung der relativistischen Kausalität bedeuten und steht im scheinbaren Widerspruch zu den Grundprinzipien der modernen Physik.
	
	Die Verwirrung entsteht aus der Tatsache, dass diese Gleichungen oft als dynamische Beziehungen interpretiert werden - als würde eine Änderung in einer Größe eine instantane Reaktion in der anderen verursachen. Diese Interpretation ist jedoch fundamental falsch und führt zu den scheinbaren Paradoxien der Quantenmechanik.
	
	\subsubsection{Die Auflösung: Feldgleichungen haben Dynamik}
	
	Die Auflösung dieses Paradoxons liegt in der Erkenntnis, dass die T0-Gleichungen zwei verschiedene Typen von Beziehungen enthalten: lokale Zwangsbedingungen und dynamische Feldgleichungen. Diese Unterscheidung ist fundamental für das Verständnis, warum keine echte Instantanität auftritt.
	
	\textbf{1. Die vollständige Feldgleichung:}
	\begin{equation}
		\nabla^2 m = 4\pi G \rho(\mathbf{x},t) \cdot m \label{eq:field_equation}
	\end{equation}
	wobei $\rho(\mathbf{x},t)$ die Massendichte ist. Diese Gleichung ist \emph{nicht} instantan, sondern eine Wellengleichung mit endlicher Ausbreitungsgeschwindigkeit $v \leq c$.
	
	Diese Feldgleichung beschreibt, wie sich Störungen im Massefeld (und damit im Zeitfeld über $T = 1/m$) durch den Raum ausbreiten. Entscheidend ist, dass diese Ausbreitung mit endlicher Geschwindigkeit erfolgt, begrenzt durch die Lichtgeschwindigkeit. Die Gleichung ist von zweiter Ordnung in den räumlichen Ableitungen, was charakteristisch für Wellenausbreitung ist. Keine Information, keine Energie und keine Wirkung kann sich schneller als mit Lichtgeschwindigkeit ausbreiten.
	
	\textbf{2. Die modifizierte Schrödinger-Gleichung:}
	\begin{equation}
		i \cdot T(\mathbf{x},t) \frac{\partial \psi}{\partial t} = H_0 \psi + V_{T0} \psi \label{eq:schroedinger}
	\end{equation}
	wobei $H_0 = -\frac{\hbar^2}{2m}\nabla^2$ der freie Hamilton-Operator und $V_{T0} = \hbar^2 \delta E(\mathbf{x},t)$ das T0-spezifische Potential ist.
	
	Diese modifizierte Schrödinger-Gleichung zeigt explizit die zeitliche Evolution der Wellenfunktion unter dem Einfluss des Zeitfeldes. Die Präsenz der zeitlichen Ableitung $\partial/\partial t$ macht deutlich, dass es sich um eine kausale Evolution handelt, nicht um eine instantane Anpassung. Die Wellenfunktion entwickelt sich kontinuierlich in der Zeit, gemäß den lokalen Feldbedingungen.
	
	\subsubsection{Die kritische Einsicht: Lokale vs. Globale Beziehungen}
	
	Der Schlüssel zum Verständnis liegt in der Unterscheidung zwischen lokalen und globalen physikalischen Beziehungen. Diese Unterscheidung ist in der Physik allgegenwärtig, wird aber oft nicht explizit genug betont. Die Verwechslung dieser beiden Arten von Beziehungen ist die Quelle vieler konzeptioneller Probleme in der Quantenmechanik.
	\subsubsection{Visualisierung der lokalen vs. globalen Beziehungen}
	
	\begin{center}
		\begin{tikzpicture}[scale=1.2]
			% Titel
			\node at (6, 7) {\Large \textbf{Lokale Zwangsbedingung vs. Globale Ausbreitung}};
			
			% Lokale Zwangsbedingung (links)
			\draw[thick, fill=t0blue!20] (0,0) circle (2);
			\node at (0, 3) {\textbf{Lokale Ebene}};
			\node at (0, 2.3) {Am Punkt $\mathbf{x}_0$};
			\draw[thick, <->] (-0.8, 0.3) -- (0.8, 0.3);
			\node at (0, 0.5) {$T \cdot E = 1$};
			\node at (0, -0.2) {\small instantan};
			\node at (0, -0.6) {\small (auf Planck-Skala)};
			\draw[thick, t0blue] (0,0) node[circle, fill, inner sep=2pt]{};
			\node at (0, -1.2) {\small Keine Dynamik};
			\node at (0, -1.6) {\small Nur Zwangsbedingung};
			
			% Pfeil nach rechts
			\draw[thick, ->, t0red] (2.5, 0) -- (4.5, 0);
			\node[above] at (3.5, 0.2) {\small Störung};
			
			% Globale Ausbreitung (rechts)
			\draw[thick, fill=t0green!20] (7,0) circle (2);
			\node at (7, 3) {\textbf{Globale Ebene}};
			\node at (7, 2.3) {Ausbreitung zu $\mathbf{x}_1$};
			% Wellenausbreitung
			\draw[thick, t0green, ->] (5.5, 0) -- (6.5, 0);
			\draw[thick, t0green] (6.5, -0.3) sin (7, 0) cos (7.5, 0.3) sin (8, 0) cos (8.5, -0.3);
			\node at (7, -0.8) {\small $v \leq c$};
			\node at (7, -1.2) {\small Feldgleichung:};
			\node at (7, -1.6) {\small $\nabla^2 m = 4\pi G \rho m$};
			
			% Zeitachse unten
			\draw[thick, ->] (0, -3) -- (9, -3) node[right] {Zeit};
			\draw[thick] (0, -3.1) -- (0, -2.9);
			\node[below] at (0, -3.1) {$t = 0$};
			\draw[thick] (7, -3.1) -- (7, -2.9);
			\node[below] at (7, -3.1) {$t = r/c$};
			
			% Distanz
			\draw[<->, t0yellow] (0, -4) -- (7, -4);
			\node[below] at (3.5, -4) {Distanz $r = |\mathbf{x}_1 - \mathbf{x}_0|$};
			
			% Legende
			\draw[thick, t0blue, fill=t0blue!20] (10, 1) rectangle (10.3, 1.3);
			\node[right] at (10.4, 1.15) {\small Lokal};
			\draw[thick, t0green, fill=t0green!20] (10, 0.3) rectangle (10.3, 0.6);
			\node[right] at (10.4, 0.45) {\small Global};
			\draw[thick, t0red, ->] (10, -0.4) -- (10.3, -0.4);
			\node[right] at (10.4, -0.4) {\small Störung};
		\end{tikzpicture}
	\end{center}
	\subsubsection{Lokale Zwangsbedingung}
	
	\begin{equation}
		T(\mathbf{x},t) \cdot E(\mathbf{x},t) = 1 \quad \text{[AM SELBEN RAUMPUNKT]} \label{eq:local_constraint}
	\end{equation}
	
	Dies ist eine lokale Zwangsbedingung - analog zu $\nabla \cdot \mathbf{E} = \rho/\epsilon_0$ in der Elektrodynamik. Sie gilt instantan am selben Punkt, erzwingt aber keine instantane Fernwirkung.
	
	Um diese Analogie zu vertiefen: In der Elektrodynamik bedeutet das Gaußsche Gesetz, dass die Divergenz des elektrischen Feldes an jedem Punkt proportional zur lokalen Ladungsdichte ist. Dies ist keine Aussage darüber, wie sich Änderungen ausbreiten, sondern eine Bedingung, die zu jedem Zeitpunkt lokal erfüllt sein muss. Wenn sich die Ladungsdichte an einem Punkt ändert, passt sich das elektrische Feld dort sofort an, aber diese Änderung breitet sich dann mit Lichtgeschwindigkeit zu anderen Punkten aus.
	
	Genauso verhält es sich mit der T-E-Beziehung im T0-Formalismus. Die Gleichung $T \cdot E = 1$ ist eine lokale Bedingung, die zu jedem Zeitpunkt an jedem Raumpunkt erfüllt sein muss. Sie beschreibt nicht, wie sich Änderungen ausbreiten, sondern nur die lokale Beziehung zwischen den Feldern.
	
	\subsubsection{Kausale Feldausbreitung}
	
	\begin{equation}
		\text{Änderung bei } \mathbf{x}_1 \rightarrow \text{Ausbreitung mit } v \leq c \rightarrow \text{Wirkung bei } \mathbf{x}_2
	\end{equation}
	\begin{equation}
		\text{Zeitverzögerung: } \Delta t = \frac{|\mathbf{x}_2 - \mathbf{x}_1|}{c} \label{eq:time_delay}
	\end{equation}
	
	Die tatsächliche Ausbreitung von Feldänderungen folgt den dynamischen Feldgleichungen. Wenn sich das Energiefeld an Punkt $\mathbf{x}_1$ ändert, muss das Zeitfeld dort sofort die Zwangsbedingung erfüllen. Diese lokale Änderung erzeugt jedoch eine Störung im Feld, die sich mit endlicher Geschwindigkeit ausbreitet.
	
	Der entscheidende Punkt ist, dass die lokale Anpassung und die globale Ausbreitung zwei völlig verschiedene Prozesse sind. Die lokale Anpassung erfolgt auf der Planck-Zeitskala und ist praktisch instantan für alle messbaren Zwecke. Die globale Ausbreitung hingegen ist durch die Lichtgeschwindigkeit begrenzt und kann über makroskopische Distanzen erhebliche Zeit in Anspruch nehmen.
	
	\subsubsection{Der geometrische Ursprung der T0-Parameter}
	
	Ein fundamentaler Aspekt der T0-Theorie ist, dass ihre Parameter nicht empirisch angepasst, sondern aus geometrischen Prinzipien abgeleitet werden. Dies unterscheidet sie grundlegend von phänomenologischen Theorien und macht sie zu einer wirklich prädiktiven Theorie.
	
	\subsubsection{Fundamentale geometrische Ableitung}
	
	Die T0-Theorie leitet alle physikalischen Parameter aus der Geometrie des dreidimensionalen Raums ab. Der zentrale Parameter ist:
	
	\begin{tcolorbox}[colback=t0blue!5!white, colframe=t0blue!75!black, title=T0-Vorhersage]
		Der universelle Parameter
		\begin{equation}
			\xi = \frac{4}{3} \times 10^{-4}
		\end{equation}
		folgt aus rein geometrischen Prinzipien:
		\begin{itemize}
			\item Fraktale Dimension des physikalischen Raums: $D_f = 2.94$
			\item Verhältnis charakteristischer Skalen zur Planck-Länge
			\item Topologische Eigenschaften des Quantenvakuums
		\end{itemize}
		Dies ist \emph{keine} empirische Anpassung, sondern eine geometrische Vorhersage.
	\end{tcolorbox}
	
	Die Bedeutung dieser geometrischen Herleitung kann nicht überbetont werden. Während die meisten physikalischen Theorien freie Parameter enthalten, die aus Experimenten bestimmt werden müssen, folgen die T0-Parameter aus der fundamentalen Struktur des Raums selbst. Dies macht die Theorie in einem tiefen Sinne vorhersagend statt beschreibend.
	
	Der Parameter $\xi$ taucht in verschiedenen Kontexten auf und verbindet scheinbar unzusammenhängende Phänomene. Er bestimmt die Stärke von Quantenkorrekturen, die Größe von Vakuumfluktuationen und die charakteristischen Skalen, auf denen neue Physik auftritt. Diese Universalität ist ein starkes Indiz dafür, dass wir es mit einer fundamentalen Konstante der Natur zu tun haben.
	
	\subsubsection{Experimentelle Bestätigung}
	
	Die geometrischen Vorhersagen der T0-Theorie werden durch verschiedene Präzisionsexperimente bestätigt, ohne dass eine Anpassung der Parameter erforderlich ist. Diese Übereinstimmung zwischen geometrischer Vorhersage und experimenteller Beobachtung ist ein starkes Indiz für die Gültigkeit des T0-Ansatzes.
	
	Die Tatsache, dass ein aus reiner Geometrie abgeleiteter Parameter experimentell verifiziert werden kann, ist bemerkenswert. Es zeigt, dass die Struktur des Raums selbst die beobachteten physikalischen Phänomene bestimmt. Dies ist eine tiefgreifende Erkenntnis, die unser Verständnis der fundamentalen Physik revolutioniert.
	
	\subsubsection{Mathematische Präzisierung der Felddynamik}
	
	Die vollständige mathematische Struktur der T0-Felddynamik zeigt eindeutig, dass alle Prozesse kausal ablaufen. Diese mathematische Präzision ist essentiell, um die scheinbaren Paradoxien aufzulösen und zu zeigen, dass die T0-Theorie vollständig mit der Relativitätstheorie kompatibel ist.
	
	\subsubsection{Vollständige Wellengleichung}
	
	Die T0-Felddynamik folgt der Gleichung:
	\begin{equation}
		\frac{\partial^2 T}{\partial t^2} = c^2\nabla^2 T + Q(T, E, \rho) \label{eq:wave_equation}
	\end{equation}
	wobei die Quellfunktion
	\begin{equation}
		Q(T, E, \rho) = -4\pi G \rho \cdot T
	\end{equation}
	die Selbstwechselwirkung des Zeitfeldes beschreibt.
	
	Diese Wellengleichung ist von fundamentaler Bedeutung. Sie zeigt explizit, dass das Zeitfeld einer hyperbolischen Differentialgleichung folgt, die charakteristisch für Wellenausbreitung mit endlicher Geschwindigkeit ist. Die zweiten Ableitungen nach Zeit und Raum stehen in einem festen Verhältnis, gegeben durch die Lichtgeschwindigkeit $c$. Dies garantiert, dass keine Information schneller als Licht übertragen werden kann.
	
	Die Quellfunktion $Q$ beschreibt, wie das Zeitfeld mit sich selbst und mit der Materie wechselwirkt. Diese Selbstwechselwirkung führt zu nicht-linearen Effekten, die besonders in starken Feldern wichtig werden. In schwachen Feldern kann die Gleichung linearisiert werden, was zu den bekannten Quantenphänomenen führt.
	
	\subsubsection{Beispiel: Energieänderung und Feldausbreitung}
	
	Um die kausale Natur der Feldausbreitung zu illustrieren, betrachten wir ein konkretes Beispiel:
	
	\begin{align}
		t &= 0: \quad E(\mathbf{x}_0) \text{ ändert sich} \\
		&\rightarrow T(\mathbf{x}_0) = \frac{1}{E(\mathbf{x}_0)} \quad \text{[lokal, Zwangsbedingung]} \\
		&\rightarrow \nabla^2 T \neq 0 \quad \text{[erzeugt Feldstörung]} \\
		&\rightarrow \text{Welle breitet sich mit } v = c \text{ aus} \\
		t &= \frac{r}{c}: \quad \text{Störung erreicht Punkt } \mathbf{x}_1
	\end{align}
	
	Dieser Prozess zeigt deutlich die Hierarchie der Ereignisse: Die lokale Anpassung erfolgt sofort (auf der Planck-Zeitskala), aber die Ausbreitung zu entfernten Punkten ist durch die Lichtgeschwindigkeit begrenzt. Für einen Beobachter bei $\mathbf{x}_1$ gibt es keine Möglichkeit, von der Änderung bei $\mathbf{x}_0$ zu erfahren, bevor die Lichtsignalzeit verstrichen ist.
	
	\subsubsection{Green'sche Funktion und Kausalität}
	
	Die Green'sche Funktion ist das mathematische Werkzeug, das die kausale Struktur der Feldausbreitung vollständig charakterisiert. Sie beschreibt, wie eine punktförmige Störung sich durch das Feld ausbreitet und ist damit fundamental für das Verständnis der Kausalität in der T0-Theorie.
	
	Die Green'sche Funktion der T0-Feldgleichung:
	\begin{equation}
		G(\mathbf{x},\mathbf{x}',t-t') = \theta(t-t') \cdot \frac{\delta(|\mathbf{x}-\mathbf{x}'| - c(t-t'))}{4\pi|\mathbf{x}-\mathbf{x}'|} \label{eq:green}
	\end{equation}
	
	Die Komponenten haben folgende Bedeutung:
	\begin{itemize}
		\item $\theta(t-t')$: Heaviside-Funktion garantiert Kausalität (Wirkung nach Ursache)
		\item $\delta$-Funktion: kodiert Ausbreitung mit Lichtgeschwindigkeit
		\item $1/4\pi r$: geometrischer Faktor für 3D-Ausbreitung
	\end{itemize}
	
	Die Struktur dieser Green'schen Funktion ist bemerkenswert. Die Heaviside-Funktion $\theta(t-t')$ ist null für $t < t'$, was bedeutet, dass keine Wirkung vor ihrer Ursache auftreten kann. Dies ist die mathematische Implementierung des Kausalitätsprinzips. Die Delta-Funktion $\delta(|\mathbf{x}-\mathbf{x}'| - c(t-t'))$ ist nur dann von null verschieden, wenn die Distanz gleich $c$ mal der verstrichenen Zeit ist - dies beschreibt eine Störung, die sich genau mit Lichtgeschwindigkeit ausbreitet.
	
	Diese mathematische Struktur garantiert, dass die T0-Theorie vollständig mit der speziellen Relativitätstheorie kompatibel ist. Es gibt keine überlichtschnellen Signale, keine Verletzung der Kausalität und keine instantanen Fernwirkungen. Alles, was instantan erscheint, ist entweder eine lokale Zwangsbedingung oder ein Prozess, der auf einer unmessbar kleinen Zeitskala abläuft.
	
	\subsubsection{Die Hierarchie der Zeitskalen}
	
	Die scheinbare Instantanität in der Quantenmechanik resultiert aus der extremen Trennung verschiedener Zeitskalen. Diese Hierarchie ist fundamental für das Verständnis, warum viele Quantenprozesse instantan erscheinen, obwohl sie es nicht sind. Das menschliche Gehirn und unsere Messgeräte können Prozesse, die auf der Planck-Zeitskala ablaufen, nicht auflösen, weshalb sie als instantan wahrgenommen werden.
	
	\begin{center}
		\begin{tikzpicture}[scale=1.3]
			\draw[thick,->] (0,0) -- (0,7) node[above] {Zeitskala [s]};
			
			% Zeitskalen
			\draw[thick] (-0.1,1) -- (0.1,1);
			\node[right] at (0.2,1) {$t_{\text{Planck}} \sim 10^{-43}$ s};
			\node[right] at (4,1) {\small Lokale T-E Anpassung};
			
			\draw[thick] (-0.1,3) -- (0.1,3);
			\node[right] at (0.2,3) {$t_{\text{QM}} \sim 10^{-15}$ s};
			\node[right] at (4,3) {\small Wellenfunktions-Evolution};
			
			\draw[thick] (-0.1,5) -- (0.1,5);
			\node[right] at (0.2,5) {$t_{\text{rel}} = r/c$};
			\node[right] at (4,5) {\small Kausale Feldausbreitung};
			
			% Bereiche
			\draw[dashed, gray] (-0.5,0.5) rectangle (8,1.5);
			\node[gray] at (9,1) {\footnotesize Unmessbar};
			
			\draw[dashed, blue] (-0.5,2.5) rectangle (8,3.5);
			\node[blue] at (9,3) {\footnotesize Quantenbereich};
			
			\draw[dashed, red] (-0.5,4.5) rectangle (8,5.5);
			\node[red] at (9,5) {\footnotesize Relativistisch};
		\end{tikzpicture}
	\end{center}
	
	Diese Hierarchie erklärt viele scheinbar paradoxe Aspekte der Quantenmechanik. Prozesse auf der Planck-Skala sind so schnell, dass sie mit keiner vorstellbaren Technologie zeitlich aufgelöst werden können. Für alle praktischen Zwecke erscheinen sie instantan. Die Quantenskala ist zugänglich für moderne Experimente, aber immer noch extrem schnell im Vergleich zu makroskopischen Zeitskalen. Die relativistische Skala schließlich bestimmt die Ausbreitung über makroskopische Distanzen.
	
	Die Existenz dieser Hierarchie ist kein Zufall, sondern eine Konsequenz der fundamentalen Konstanten der Natur. Die Planck-Zeit ist die kürzeste physikalisch sinnvolle Zeitskala, bestimmt durch die Quantengravitation. Die Quantenzeitskala wird durch die atomaren Energien bestimmt. Die relativistische Zeitskala schließlich ist durch die Lichtgeschwindigkeit und die betrachteten Distanzen gegeben.
	
	\subsubsection{Die vollständige Dualität: Zeit, Masse, Energie und Länge}
	
	Die T0-Theorie beschreibt nicht nur eine Zeit-Masse-Dualität, sondern ein umfassendes System von Dualitäten, in dem alle fundamentalen Größen miteinander verbunden sind. Diese erweiterte Perspektive ist essentiell für das vollständige Verständnis der scheinbaren Instantanität und zeigt, dass die verschiedenen physikalischen Größen nur verschiedene Aspekte derselben zugrundeliegenden Realität sind.
	
	\subsubsection{Visualisierung der Energie-Zeit-Dualität}
	
	\begin{center}
		\begin{tikzpicture}[scale=1.3]
			% Titel
			\node at (0, 6) {\Large \textbf{Die fundamentale Energie-Zeit-Dualität}};
			
			% Hauptgleichung in der Mitte
			\draw[thick, t0blue, fill=t0blue!10] (-2, 3.5) rectangle (2, 4.5);
			\node at (0, 4) {\Large $T \cdot E = 1$};
			
			% Zeit-Seite (links)
			\draw[thick, t0red, fill=t0red!10] (-6, 1.5) rectangle (-3, 3.3);
			\node at (-4.5, 3) {\textbf{Zeitaspekt}};
			\node at (-4.5, 2.5) {$T = \frac{1}{m}$};
			\node at (-4.5, 2) {\small Lange Zeiten};
			\draw[thick, ->] (-3, 2.25) -- (-2.2, 3.5);
			
			% Energie-Seite (rechts)
			\draw[thick, t0green, fill=t0green!10] (3, 1.5) rectangle (6, 3.3);
			\node at (4.5, 3) {\textbf{Energieaspekt}};
			\node at (4.5, 2.5) {$E = mc^2$};
			\node at (4.5, 2) {\small Hohe Energien};
			\draw[thick, ->] (3, 2.25) -- (2.2, 3.5);
			
			% Längen-Beziehung (unten links)
			\draw[thick, t0yellow, fill=t0yellow!10] (-6, -0.5) rectangle (-3, 1.2);
			\node at (-4.5, 0.7) {\textbf{Längenaspekt}};
			\node at (-4.5, 0.3) {$\ell = \frac{\hbar}{mc}$};
			\node at (-4.5, -0.2) {\small Große Distanzen};
			\draw[thick, ->] (-4.5, 1) -- (-4.5, 1.5);
			
			% Masse-Beziehung (unten rechts)
			\draw[thick, t0purple, fill=t0purple!10] (3, -0.5) rectangle (6, 1.2);
			\node at (4.5, 0.7) {\textbf{Masseaspekt}};
			\node at (4.5, 0.3) {$m = \frac{E}{c^2}$};
			\node at (4.5, -0.2) {\small Schwere Teilchen};
			\draw[thick, ->] (4.5, 1) -- (4.5, 1.5);
			
			% Komplementarität (unten)
			\draw[thick, dashed, gray] (-2, -2) -- (2, -2);
			\node at (0, -2.5) {\textbf{Komplementaritätsprinzip:}};
			\node at (0, -3) {Je präziser $T$ bestimmt, desto unschärfer $E$};
			\node at (0, -3.5) {$\Delta T \cdot \Delta E \geq \frac{\hbar}{2}$};
			
			% Pfeile für Beziehungen
			\draw[thick, <->, gray] (-3, 0) -- (3, 0);
			\node[above] at (0, 0) {\small reziprok};
			
			% Planck-Skala Box
			\draw[thick, double, fill=white] (-1.5, -1.3) rectangle (1.5, -1.3);
			\node at (0, -0.8) {\small \textbf{Planck-Skala:} Alle gleich};
			
			% Skalenabhängigkeit
			\node[right] at (-1, 2.5) {\small \textbf{Dominant bei:}};
			\node[right] at (-1, 2) {\small Atomare Skala: $E$-$T$};
			\node[right] at (-1, 1.5) {\small Makroskopisch: $m$};
			\node[right] at (-1, 1) {\small Kosmologisch: $\ell$-$t$};
		\end{tikzpicture}
	\end{center}
	
	Dieses Diagramm zeigt die fundamentale Energie-Zeit-Dualität und ihre Verbindungen zu Masse und Länge. Die zentrale Beziehung $T \cdot E = 1$ verbindet alle Aspekte. Je nach betrachteter Skala dominieren verschiedene Aspekte dieser Dualität, aber alle sind durch die fundamentalen Beziehungen miteinander verknüpft.
	
	\subsubsection{Die fundamentalen Äquivalenzen}
	
	Im T0-Formalismus sind die grundlegenden physikalischen Größen durch folgende Beziehungen verknüpft:
	
	\begin{align}
		T \cdot E &= 1 \quad \text{(Zeit-Energie-Dualität)} \\
		T &= \frac{1}{m} \quad \text{(Zeit-Masse-Beziehung)} \\
		E &= mc^2 \quad \text{(Masse-Energie-Äquivalenz)} \\
		\ell &= \frac{\hbar}{mc} = \frac{\hbar}{E/c} \quad \text{(Länge als Energie)}
	\end{align}
	
	Diese Beziehungen zeigen, dass Längen ebenfalls als Energieskalen interpretiert werden können. Die Compton-Wellenlänge $\lambda_C = \hbar/(mc)$ ist das paradigmatische Beispiel: Sie repräsentiert die charakteristische Längenskala, auf der die Quantennatur eines Teilchens mit Masse $m$ (oder äquivalent, Energie $E = mc^2$) manifest wird.
	
	Diese Dualitäten sind nicht nur mathematische Kuriositäten, sondern haben tiefgreifende physikalische Bedeutung. Sie zeigen, dass die scheinbar verschiedenen Konzepte von Zeit, Raum, Masse und Energie tatsächlich verschiedene Manifestationen derselben fundamentalen Struktur sind. Diese Einheit ist der Schlüssel zum Verständnis vieler Quantenphänomene.
	
	\subsubsection{Die Planck-Skala als universelle Referenz}
	
	An der Planck-Skala konvergieren alle diese Dualitäten:
	
	\begin{align}
		\lP &= \sqrt{\frac{\hbar G}{c^3}} \quad \text{(Planck-Länge)} \\
		\tP &= \sqrt{\frac{\hbar G}{c^5}} \quad \text{(Planck-Zeit)} \\
		\mP &= \sqrt{\frac{\hbar c}{G}} \quad \text{(Planck-Masse)} \\
		\EP &= \sqrt{\frac{\hbar c^5}{G}} \quad \text{(Planck-Energie)}
	\end{align}
	
	Bemerkenswert ist, dass diese Größen die fundamentalen Beziehungen erfüllen:
	\begin{align}
		\tP \cdot \EP &= \hbar \\
		\lP &= c \cdot \tP \\
		\EP &= \mP c^2 \\
		\lP &= \frac{\hbar}{\mP c}
	\end{align}
	
	Diese Konsistenz zeigt, dass die T0-Dualitäten nicht willkürlich, sondern tief in der Struktur der Raumzeit verwurzelt sind. Die Planck-Skala definiert die fundamentale Grenze, unterhalb derer unsere klassischen Konzepte von Raum und Zeit ihre Bedeutung verlieren. Auf dieser Skala werden alle Aspekte der Dualität gleich wichtig, und eine Beschreibung, die nur einen Aspekt betont, ist unvollständig.
	
	\subsubsection{Länge-Energie-Korrespondenz und Feldausbreitung}
	
	Die Interpretation von Längen als Energieskalen hat direkte Konsequenzen für das Verständnis der Feldausbreitung. Eine Störung der Größe $\Delta E$ hat eine charakteristische Wellenlänge:
	
	\begin{equation}
		\lambda = \frac{hc}{\Delta E}
	\end{equation}
	
	Dies bedeutet, dass hochenergetische Prozesse auf kleinen Längenskalen lokalisiert sind, während niederenergetische Prozesse über große Distanzen ausgedehnt sind. Diese Energie-Längen-Beziehung ist fundamental für das Verständnis, warum die scheinbare Instantanität auf verschiedenen Skalen unterschiedlich manifest wird.
	
	Für die Feldausbreitung bedeutet dies: Je höher die Energie einer Störung, desto kleiner ist ihre charakteristische Wellenlänge und desto präziser kann ihre raumzeitliche Lokalisierung bestimmt werden. Dies steht in direktem Zusammenhang mit der Heisenbergschen Unschärferelation:
	
	\begin{equation}
		\Delta x \cdot \Delta p \geq \frac{\hbar}{2}
	\end{equation}
	
	oder in Energie-Zeit-Form:
	
	\begin{equation}
		\Delta t \cdot \Delta E \geq \frac{\hbar}{2}
	\end{equation}
	
	Diese Unschärferelationen sind nicht nur statistische Aussagen über Messungen, sondern fundamentale Eigenschaften der Felder selbst. Sie zeigen, dass eine präzise Lokalisierung in einem Aspekt notwendigerweise zu einer Unschärfe im komplementären Aspekt führt.
	
	\subsubsection{Implikationen für die Kausalität}
	
	Die vollständige Dualität hat wichtige Implikationen für unser Verständnis der Kausalität. Wenn Längen als inverse Energien verstanden werden, dann bedeutet eine Messung mit Energieauflösung $\Delta E$ automatisch eine räumliche Unschärfe von mindestens $\lambda = hc/\Delta E$. Dies erklärt, warum hochpräzise Energiemessungen (kleine $\Delta E$) zu großen räumlichen Unschärfen führen und umgekehrt.
	
	Für die scheinbare Instantanität bedeutet dies: Prozesse, die auf sehr kleinen Energieskalen ablaufen (große Wellenlängen), erscheinen räumlich delokalisiert. Dies kann den Eindruck erwecken, dass Korrelationen instantan über große Distanzen auftreten, obwohl sie tatsächlich das Resultat ausgedehnter, niederenergetischer Feldkonfigurationen sind.
	
	\subsubsection{Skalenabhängigkeit und Grenzen der Interpretation}
	
	Die T0-Theorie zeigt, dass die verschiedenen Aspekte der Dualität je nach betrachteter Skala unterschiedlich stark ausgeprägt sind. Diese Skalenabhängigkeit ist fundamental und mahnt zur Vorsicht bei der Interpretation von Extremsituationen.
	
	\subsubsection{Die Komplementarität der Aspekte}
	
	Auf verschiedenen Skalen dominieren unterschiedliche Aspekte:
	\begin{itemize}
		\item \textbf{Planck-Skala:} Alle Aspekte sind gleichwertig, keine Näherung gültig
		\item \textbf{Atomare Skala:} Energie-Zeit-Dualität dominiert, Gravitation vernachlässigbar
		\item \textbf{Makroskopische Skala:} Masse-Aspekt dominant, Quanteneffekte unterdrückt
		\item \textbf{Kosmologische Skala:} Raum-Zeit-Struktur dominant, lokale Quanteneffekte irrelevant
	\end{itemize}
	
	Diese Skalenabhängigkeit ist nicht nur eine praktische Näherung, sondern reflektiert die fundamentale Struktur der Realität. Auf jeder Skala manifestieren sich verschiedene Aspekte der zugrundeliegenden Einheit. Das Verständnis dieser Hierarchie ist essentiell für die korrekte Anwendung der T0-Theorie.
	
	\subsubsection{Die Rolle kleiner Korrekturen}
	
	Obwohl der $\xi$-Parameter ($\xi = 4/3 \times 10^{-4}$) und Gravitationseffekte oft extrem klein sind, haben sie dennoch messbare Auswirkungen. Diese kleinen Korrekturen sind nicht vernachlässigbar, sondern essentiell für das vollständige Verständnis:
	
	\begin{equation}
		\text{Beobachtbarer Effekt} = \text{Hauptbeitrag} + \xi \cdot \text{Korrektur} + \text{Gravitationsbeitrag}
	\end{equation}
	
	Die Wichtigkeit dieser kleinen Terme zeigt sich besonders bei:
	\begin{itemize}
		\item Präzisionsmessungen (z.B. anomale magnetische Momente)
		\item Langreichweitigen Korrelationen (Bell-Tests über kosmische Distanzen)
		\item Akkumulationseffekten über lange Zeiträume
	\end{itemize}
	
	Die Tatsache, dass diese winzigen Korrekturen messbar sind und mit den theoretischen Vorhersagen übereinstimmen, ist eine bemerkenswerte Bestätigung der T0-Theorie. Es zeigt, dass selbst die kleinsten Details der Theorie physikalische Realität haben.
	
	\subsubsection{Vorsicht vor Singularitäten}
	
	Ein kritischer Punkt der T0-Theorie ist die Behandlung von Extremsituationen. Singularitäten, wie sie in der klassischen Allgemeinen Relativitätstheorie auftreten, sind in der T0-Perspektive problematisch und gehören in den Bereich der Spekulation:
	
	\begin{tcolorbox}[colback=t0yellow!10!white, colframe=t0yellow!75!black, title=Wichtige Einsicht]
		Singularitäten sind \textbf{nicht} das Ziel der T0-Theorie. Sie repräsentieren vielmehr Grenzen der Anwendbarkeit:
		\begin{itemize}
			\item Bei $r \to 0$: Die lokale Näherung bricht zusammen
			\item Bei $E \to \infty$: Die Feldgleichungen werden nicht-linear
			\item Bei $T \to 0$: Die Zeit-Energie-Dualität verliert ihre Bedeutung
		\end{itemize}
		Diese Grenzen sind nicht physikalisch, sondern zeigen, wo die Theorie erweitert werden muss.
	\end{tcolorbox}
	
	Singularitäten sind Warnsignale, dass wir die Grenzen der Anwendbarkeit unserer Theorie erreicht haben. In der Natur gibt es wahrscheinlich keine echten Singularitäten - sie sind mathematische Artefakte, die anzeigen, dass unsere Beschreibung unvollständig ist. Die T0-Theorie erkennt diese Grenzen an und versucht nicht, sie zu überschreiten.
	
	\subsubsection{Das Komplementaritätsprinzip in T0}
	
	Analog zum Bohr'schen Komplementaritätsprinzip in der Quantenmechanik gilt in der T0-Theorie:
	
	\begin{equation}
		\text{Präzision}(T) \times \text{Präzision}(E) \leq \text{konstant}
	\end{equation}
	
	Je genauer wir einen Aspekt (z.B. Zeit) bestimmen, desto unschärfer wird der komplementäre Aspekt (Energie). Dies ist keine Schwäche der Theorie, sondern eine fundamentale Eigenschaft der Realität.
	
	Praktische Konsequenzen:
	\begin{itemize}
		\item \textbf{Hochenergiephysik:} Energie-Aspekt dominant, Zeit-Aspekt unscharf
		\item \textbf{Kosmologie:} Zeit-Aspekt auf großen Skalen dominant, lokale Energie unscharf
		\item \textbf{Quantengravitation:} Beide Aspekte wichtig, keine einfache Näherung möglich
	\end{itemize}
	
	\subsubsection{Interpretationsrichtlinien}
	
	Für die korrekte Anwendung der T0-Theorie gelten folgende Richtlinien:
	
	\begin{enumerate}
		\item \textbf{Skalenbeachtung:} Immer prüfen, welche Skala dominant ist
		\item \textbf{Kleine Effekte ernst nehmen:} $\xi$-Korrekturen und Gravitationseffekte nicht ignorieren
		\item \textbf{Singularitäten vermeiden:} Als Hinweis auf Theoriegrenzen verstehen
		\item \textbf{Komplementarität respektieren:} Nicht alle Aspekte können gleichzeitig scharf sein
		\item \textbf{Experimentelle Überprüfbarkeit:} Nur Vorhersagen machen, die prinzipiell messbar sind
	\end{enumerate}
	
	Diese Vorsicht ist besonders wichtig bei:
	\begin{itemize}
		\item Schwarzen Löchern (keine echten Singularitäten in T0)
		\item Urknall-Kosmologie (T kann nicht wirklich null werden)
		\item Extremen Quantenzuständen (Superpositionen über kosmische Skalen)
	\end{itemize}
	
	\subsubsection{Auflösung der Quantenparadoxe}
	
	Die T0-Theorie bietet elegante Lösungen für die klassischen Paradoxe der Quantenmechanik, indem sie zeigt, dass diese aus einer unvollständigen Beschreibung der zugrundeliegenden Feldstruktur resultieren. Die scheinbaren Mysterien lösen sich auf, wenn man die vollständige Felddynamik berücksichtigt.
	
	\subsubsection{Bell-Korrelationen}
	
	Die scheinbar instantanen Bell-Korrelationen werden durch die T0-Theorie aufgelöst:
	
	\begin{itemize}
		\item \textbf{Lokale Bedingung:} $T \cdot E = 1$ an beiden Messorten
		\item \textbf{Gemeinsames Feld:} Verschränkte Teilchen teilen Feldkonfiguration
		\item \textbf{Kausale Ausbreitung:} Feldänderungen propagieren mit $c$
		\item \textbf{Korrelation ohne Kommunikation:} Vorstrukturiertes Feld, keine Signalübertragung
	\end{itemize}
	
	Die entscheidende Einsicht ist, dass verschränkte Teilchen nicht durch mysteriöse instantane Verbindungen korreliert sind, sondern durch ein gemeinsames Feld, das bei ihrer Erzeugung etabliert wurde. Dieses Feld existiert im gesamten Raumbereich und entwickelt sich kausal gemäß den Feldgleichungen. Die beobachteten Korrelationen sind das Resultat dieser bereits existierenden Feldstruktur, nicht einer instantanen Kommunikation.
	
	Wenn zwei Teilchen in einem verschränkten Zustand präpariert werden, teilen sie sich eine gemeinsame Feldkonfiguration. Diese Konfiguration bestimmt die Korrelationen zwischen den Messergebnissen, unabhängig davon, wie weit die Teilchen später voneinander entfernt sind. Die Messungen offenbaren nur die bereits existierende Feldstruktur - sie verursachen keine instantane Änderung am entfernten Ort.
	
	\subsubsection{Wellenfunktionskollaps}
	
	Der vermeintlich instantane Kollaps ist eine Illusion:
	\begin{align}
		\text{Messung} &\rightarrow \text{Lokale Feldstörung} \quad (t \sim t_{\text{Planck}}) \\
		&\rightarrow \text{Feldausbreitung} \quad (v = c) \\
		&\rightarrow \text{Erscheint instantan da } t_{\text{Planck}} \ll t_{\text{Mess}}
	\end{align}
	
	Was als diskontinuierlicher Kollaps erscheint, ist in Wirklichkeit ein kontinuierlicher Prozess, der auf einer Zeitskala abläuft, die weit unterhalb unserer Messauflösung liegt. Der Messprozess ist eine lokale Interaktion zwischen Messgerät und Feld, die eine Störung erzeugt, welche sich kausal ausbreitet.
	
	Der scheinbare Kollaps der Wellenfunktion ist tatsächlich eine sehr schnelle, aber kontinuierliche Umorganisation der lokalen Feldstruktur. Diese Umorganisation erfolgt auf der Planck-Zeitskala und ist daher für alle praktischen Zwecke instantan. Aber physikalisch ist es ein kausaler Prozess, der den Gesetzen der Feldtheorie folgt.
	
	\subsubsection{Experimentelle Konsequenzen}
	
	Obwohl die meisten T0-Effekte auf unmessbar kleinen Zeitskalen auftreten, macht die Theorie dennoch überprüfbare Vorhersagen für extreme Bedingungen. Diese Vorhersagen unterscheiden die T0-Theorie von der Standard-Quantenmechanik und bieten Möglichkeiten für experimentelle Tests.
	
	\subsubsection{Vorhersage messbarer Verzögerungen}
	
	Für kosmische Bell-Tests mit Distanz $r$:
	\begin{equation}
		\Delta t_{\text{messbar}} = \xi \cdot \frac{r}{c}
	\end{equation}
	wobei $\xi = \frac{4}{3} \times 10^{-4}$ der geometrische Parameter ist.
	
	\textbf{Numerisches Beispiel:}
	\begin{itemize}
		\item Satelliten-Experiment mit $r = 1000$ km:
		\begin{equation}
			\Delta t = 1.333 \times 10^{-4} \times \frac{10^6 \text{ m}}{3 \times 10^8 \text{ m/s}} \approx 0.44 \, \mu\text{s}
		\end{equation}
		\item Diese Verzögerung ist mit modernen Atomuhren ($\Delta t_{\text{Auflösung}} \sim 10^{-9}$ s) messbar
	\end{itemize}
	
	Diese Vorhersage ist bemerkenswert, weil sie einen klaren Test der T0-Theorie gegen die Standard-Quantenmechanik ermöglicht. Während die Standard-QM exakt simultane Korrelationen vorhersagt, sagt T0 eine kleine, aber messbare Verzögerung voraus, die mit der Distanz skaliert.
	
	\subsubsection{Vorgeschlagene Experimente}
	
	\begin{enumerate}
		\item \textbf{Satelliten-Bell-Test:} Verschränkte Photonen zwischen Erdstation und Satellit
		\item \textbf{Lunar Laser Ranging:} Präzisionsmessung von Quantenkorrelationen Erde-Mond
		\item \textbf{Deep Space Quantum Network:} Test bei interplanetaren Distanzen
	\end{enumerate}
	
	Jedes dieser Experimente würde die Grenzen unseres Verständnisses der Quantenkorrelationen testen und könnte die subtilen Vorhersagen der T0-Theorie bestätigen oder widerlegen. Die technischen Herausforderungen sind erheblich, aber nicht unüberwindbar. Mit der fortschreitenden Entwicklung der Quantentechnologie werden solche Tests in den kommenden Jahren möglich werden.
	
	\subsubsection{Philosophische Implikationen}
	
	Die Auflösung der scheinbaren Instantanität hat tiefgreifende Konsequenzen für unser Verständnis der physikalischen Realität. Die T0-Theorie zeigt, dass die Natur lokal und kausal ist, trotz der scheinbaren Nicht-Lokalität der Quantenmechanik.
	
	\subsubsection{Neue Interpretation der Quantenmechanik}
	
	Die T0-Theorie bietet eine alternative Perspektive auf die Quantenmechanik:
	
	\begin{tcolorbox}[colback=t0red!5!white, colframe=t0red!75!black, title=Neue Perspektive]
		\textbf{Standardinterpretation:}
		\begin{itemize}
			\item Quantenmechanik erfordert Nicht-Lokalität
			\item Spukhafte Fernwirkung (Einstein)
			\item Kollaps der Wellenfunktion
		\end{itemize}
		
		\textbf{T0-Interpretation:}
		\begin{itemize}
			\item Alles ist lokal in einem gemeinsamen Feld
			\item Korrelationen durch Feldvorstruktur
			\item Kontinuierliche, kausale Evolution
		\end{itemize}
	\end{tcolorbox}
	
	Dieser Paradigmenwechsel löst viele der konzeptionellen Probleme, die die Quantenmechanik seit ihrer Entstehung plagen. Die Notwendigkeit für verschiedene Interpretationen verschwindet, wenn man erkennt, dass die scheinbaren Paradoxe aus einer unvollständigen Beschreibung resultieren.
	
	\subsubsection{Vereinigung von Quantenmechanik und Relativität}
	
	Die T0-Theorie löst den scheinbaren Konflikt:
	\begin{itemize}
		\item Erhält Lorentz-Invarianz vollständig
		\item Keine überlichtschnelle Informationsübertragung
		\item Quantenkorrelationen durch kausale Feldstruktur
	\end{itemize}
	
	Diese Vereinigung ist nicht nur formal, sondern konzeptionell. Beide Theorien werden als verschiedene Aspekte derselben zugrundeliegenden Feldstruktur verstanden. Die Quantenmechanik beschreibt die kohärenten Eigenschaften der Felder, während die Relativität ihre kausale Struktur charakterisiert.
	
	Die lange gesuchte Vereinigung von Quantenmechanik und Relativität ergibt sich natürlich aus der T0-Perspektive. Es gibt keinen fundamentalen Konflikt zwischen den beiden Theorien - sie beschreiben nur verschiedene Aspekte derselben Realität. Die scheinbaren Widersprüche entstehen nur, wenn man versucht, eine unvollständige Beschreibung zu verwenden.
	
	\subsubsection{Der Messprozess im Detail}
	
	Der Messprozess in der Quantenmechanik ist seit jeher eines der größten konzeptionellen Probleme. Der Kollaps der Wellenfunktion scheint ein nicht-unitärer, instantaner Prozess zu sein, der sich fundamental von der normalen Schrödinger-Evolution unterscheidet. Der T0-Formalismus bietet eine alternative Beschreibung, die diese Probleme vermeidet.
	
	Im T0-Bild ist eine Messung eine lokale Interaktion zwischen dem Messgerät und dem Feld am Ort der Messung. Diese Interaktion findet auf der Planck-Zeitskala statt - extrem schnell, aber nicht instantan. Der scheinbare Kollaps ist in Wirklichkeit eine sehr schnelle, aber kontinuierliche Umorganisation der lokalen Feldstruktur.
	
	Entscheidend ist, dass diese lokale Umorganisation keine instantane Änderung des Feldes an entfernten Orten erfordert. Die Information über die Messung breitet sich als Feldstörung mit Lichtgeschwindigkeit aus. Wenn diese Störung andere Teile eines verschränkten Systems erreicht, beeinflusst sie deren weitere Evolution, aber dies geschieht kausal und mit endlicher Geschwindigkeit.
	
	Diese Beschreibung eliminiert die konzeptionellen Probleme des Messprozesses. Es gibt keinen mysteriösen Kollaps, keine Verletzung der Unitarität und keine instantanen Fernwirkungen. Alles wird durch lokale Feldinteraktionen und kausale Feldausbreitung beschrieben.
	
	\subsubsection{Quantenverschränkung ohne Instantanität}
	
	Die Quantenverschränkung gilt oft als das paradigmatische Beispiel für nicht-lokale Quantenphänomene. Wenn zwei Teilchen verschränkt sind, scheint eine Messung an einem Teilchen instantan den Zustand des anderen zu bestimmen, unabhängig von der Entfernung. Die Bell'schen Ungleichungen und ihre experimentelle Verletzung scheinen zu beweisen, dass lokale realistische Theorien die Quantenmechanik nicht reproduzieren können.
	
	Der T0-Formalismus bietet eine neue Perspektive auf diese Phänomene. Die Verschränkung wird nicht als mysteriöse instantane Verbindung interpretiert, sondern als Resultat einer gemeinsamen Feldkonfiguration, die bei der Erzeugung der verschränkten Teilchen etabliert wird. Diese Feldkonfiguration existiert im gesamten Raumbereich zwischen den Teilchen und entwickelt sich gemäß den kausalen Feldgleichungen.
	
	Wenn eine Messung an einem der verschränkten Teilchen durchgeführt wird, interagiert der Messapparat lokal mit dem Feld an diesem Ort. Diese Interaktion erzeugt eine Störung im Feld, die sich mit Lichtgeschwindigkeit ausbreitet. Die Korrelationen zwischen den Messergebnissen entstehen nicht durch instantane Kommunikation, sondern durch die bereits existierende Struktur des gemeinsamen Feldes.
	
	Diese Interpretation löst das EPR-Paradoxon auf eine Weise, die sowohl mit der Quantenmechanik als auch mit der Relativitätstheorie vollständig kompatibel ist. Es gibt keine spukhafte Fernwirkung, sondern nur lokale Interaktionen mit einem ausgedehnten Feld. Die beobachteten Korrelationen sind das Ergebnis der kohärenten Feldstruktur, nicht einer instantanen Informationsübertragung.
	
	\subsubsection{Zusammenfassung und Ausblick}
	
	Die Analyse des T0-Formalismus zeigt eindeutig, dass die scheinbare Instantanität der Quantenmechanik eine Illusion ist, die durch mehrere Faktoren entsteht.
	
	\subsubsection{Zentrale Ergebnisse}
	
	Die T0-Theorie eliminiert die Instantanität durch eine hierarchische Struktur:
	
	\begin{enumerate}
		\item \textbf{Lokale Ebene:} $T \cdot E = 1$ als Zwangsbedingung (keine Dynamik)
		\item \textbf{Feld-Ebene:} Wellengleichung mit Ausbreitung $v \leq c$ (kausale Dynamik)
		\item \textbf{Messbare Ebene:} Erscheint instantan wegen $\Delta t < $ Auflösung
	\end{enumerate}
	
	Diese Hierarchie ist der Schlüssel zum Verständnis, warum die Quantenmechanik scheinbar nicht-lokal ist, während die zugrundeliegende Physik vollständig lokal und kausal bleibt.
	
	\subsubsection{Die fundamentale Erkenntnis}
	
	\begin{tcolorbox}[colback=t0yellow!10!white, colframe=t0yellow!75!black, title=Kernaussage]
		Die scheinbare Instantanität der Quantenmechanik ist eine Illusion, die durch:
		\begin{itemize}
			\item Die Notation lokaler Zwangsbedingungen
			\item Die extreme Kleinheit der Planck-Zeit
			\item Die Vorstrukturierung gemeinsamer Felder
		\end{itemize}
		entsteht. Die T0-Theorie zeigt, dass alle Phänomene strikt kausal und lokal sind, wenn man die vollständige Felddynamik berücksichtigt.
	\end{tcolorbox}
	
	Die Implikationen dieser Erkenntnis reichen weit über die technischen Details hinaus. Sie zeigt, dass die Natur trotz ihrer Quantenhaftigkeit fundamental verständlich und kausal strukturiert ist. Die scheinbaren Mysterien der Quantenmechanik lösen sich auf, wenn man die richtige theoretische Perspektive einnimmt.
	
	\subsubsection{Ausblick}
	
	Die T0-Theorie eröffnet neue Forschungsrichtungen:
	\begin{itemize}
		\item Präzisionstests der vorhergesagten Verzögerungen
		\item Quanteninformationstheorie mit Feldkorrelationen
		\item Kosmologische Implikationen der Zeitfeld-Dynamik
		\item Technologische Anwendungen in der Quantenkommunikation
	\end{itemize}
	
	Jede dieser Richtungen verspricht neue Einsichten in die fundamentale Natur der Realität. Die T0-Theorie ist nicht nur eine mathematische Umformulierung, sondern ein neues konzeptionelles Fundament für unser Verständnis der Quantenwelt. Die Auflösung der scheinbaren Instantanität ist dabei ein wichtiger Schritt in der Weiterentwicklung unseres physikalischen Weltbilds.
	
	Die Zukunft der Physik liegt möglicherweise in der Erkenntnis, dass die scheinbaren Mysterien der Quantenwelt nicht fundamental sind, sondern aus einer unvollständigen Beschreibung resultieren. Die T0-Theorie zeigt einen Weg zu einem vollständigeren Verständnis, in dem Lokalität, Kausalität und die beobachteten Quantenphänomene harmonisch koexistieren.
	
	\begin{thebibliography}{99}
		\bibitem{t0_grundlagen}
		T0-Theorie Grundlagen (2024). \textit{Zeit-Masse-Dualität und geometrische Feldtheorie}. Internes Forschungsdokument.
		
		\bibitem{bell_original}
		Bell, J.S. (1964). On the Einstein Podolsky Rosen Paradox. \textit{Physics Physique Fizika}, \textbf{1}, 195--200.
		
		\bibitem{einstein_epr}
		Einstein, A., Podolsky, B., Rosen, N. (1935). Can Quantum-Mechanical Description of Physical Reality Be Considered Complete? \textit{Physical Review}, \textbf{47}, 777--780.
		
		\bibitem{aspect_experiments}
		Aspect, A., Grangier, P., Roger, G. (1982). Experimental Realization of Einstein-Podolsky-Rosen-Bohm Gedankenexperiment. \textit{Physical Review Letters}, \textbf{49}, 91--94.
		
		\bibitem{planck_units}
		Planck, M. (1899). Über irreversible Strahlungsvorgänge. \textit{Sitzungsberichte der Preußischen Akademie der Wissenschaften}, 440--480.
	\end{thebibliography}

\chapter{Quanten-Algebraische Topologie}
\input{chapters/ch_T0_QAT_De}

\chapter{QM-Optimierung}
\input{chapters/ch_T0_QM-optimierung_De}

% Part XII: Spezielle Themen
\part{Spezielle Themen und Erweiterungen}

\chapter{Elementarladung}
\input{chapters/ch_Unit Charge_De}

\chapter{Mathematische Zeit-Masse-Lagrange}
\input{chapters/ch_MathZeitMasseLagrangeDe}

\chapter{g-2 Erweiterung}
\input{chapters/ch_T0_g2-erweiterung-4_De}

\chapter{Ampère bei niedrigen Frequenzen}
\input{chapters/ch_Amper_Low_De}

\chapter{Ableitung von Beta}
\input{chapters/ch_DerivationVonBetaDe}

\chapter{Frequenzunabhängigkeit}
\input{chapters/ch_T0_freqeunz_De}

\chapter{Universale Ableitung}
\input{chapters/ch_universale-ableitung_De}

\chapter{T0-Umkehrung}
\input{chapters/ch_T0_umkehrung_De}

\chapter{Dynamische Masse von Photonen}
\input{chapters/ch_DynMassePhotonenNichtlokalDe}

\chapter{Zeit-Konstanten}
\input{chapters/ch_Zeit-konstant_De}

\chapter{Zeit in der T0-Theorie}
\input{chapters/ch_Zeit_De}

\chapter{Zusammenfassung}
\input{chapters/ch_Zusammenfassung_De}

% ==========================================
% BACK MATTER
% ==========================================
\backmatter

\chapter*{Schlusswort}
\addcontentsline{toc}{chapter}{Schlusswort}

Diese Dokumentensammlung präsentiert die vollständige T0-Theorie der Zeit-Masse-Dualität. 
Die zentrale Erkenntnis -- dass alle Naturkonstanten aus der Feinstrukturkonstante 
$\alpha \approx 1/137$ abgeleitet werden können -- stellt einen fundamentalen 
Paradigmenwechsel in der theoretischen Physik dar.

Die experimentellen Vorhersagen der T0-Theorie, insbesondere für das anomale 
magnetische Moment, die Koide-Formel und kosmologische Beobachtungen, bieten 
überprüfbare Tests dieser neuen Perspektive.

\vspace{2em}
\begin{flushright}
\textit{Johann Pascher, 2024}
\end{flushright}

\end{document}
