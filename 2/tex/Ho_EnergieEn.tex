\documentclass[12pt,a4paper]{article}
\usepackage[utf8]{inputenc}
\usepackage[T1]{fontenc}
\usepackage[english]{babel}
\usepackage[left=2cm,right=2cm,top=2cm,bottom=2cm]{geometry}
\usepackage{lmodern}
\usepackage{amsmath}
\usepackage{amssymb}
\usepackage{physics}
\usepackage{booktabs}
\usepackage{tcolorbox}
\usepackage{siunitx}
\usepackage[table,xcdraw]{xcolor}
\usepackage{hyperref}

\title{$H_0$ and $\kappa$ Parameters: T0 Model Reference Document\\
	\large Mathematical Derivations and Experimental Comparisons}
\author{Johann Pascher}
\date{\today}

\begin{document}
	
	\maketitle
	
	\section{Introduction}
	
	The T0 model provides a unified framework for deriving cosmological parameters from fundamental field theory. This document presents the mathematical derivations of the Hubble parameter $H_0$ and the linear potential parameter $\kappa$ along with experimental comparisons. The key insight is that both parameters emerge from geometry-dependent energy field dynamics rather than being empirically determined constants.
	
	\section{T0 Model Framework}
	
	\subsection{Natural Units Convention}
	In T0 model natural units:
	\begin{align}
		\hbar = c = \alpha_{\text{em}} = \beta_t = 1
	\end{align}
	
	\subsection{Fundamental Field Equations}
	The T0 energy field satisfies:
	\begin{align}
		E(x,t) &= \frac{1}{\max(m(x,t), \omega)} \\
		\nabla^2 E &= 4\pi G \rho_E
	\end{align}
	
	where $\omega$ represents the fundamental frequency scale and $\rho_E$ is the energy density.
	
	\section{Geometry-Dependent $\xi$ Parameters}
	
	\subsection{Critical Discovery: $4\pi$ Factor Corrections}
	
	Through systematic analysis, geometry-dependent corrections to the fundamental $\xi$ parameter have been identified:
	
	\begin{tcolorbox}[colback=blue!5!white,colframe=blue!75!black,title=Geometry-Dependent $\xi$ Parameters]
		\textbf{Flat Geometry (Local Physics):}
		\begin{equation}
			\xi_{\text{flat}} = \frac{\lambda_h^2 v^2}{16\pi^3 E_h^2} = 1.3165 \times 10^{-4}
		\end{equation}
		
		\textbf{Spherical Geometry (Cosmological Physics):}
		\begin{equation}
			\xi_{\text{spherical}} = \frac{\lambda_h^2 v^2}{24\pi^{5/2} E_h^2} = 1.557 \times 10^{-4}
		\end{equation}
		
		\textbf{Geometric Correction Factor:}
		\begin{equation}
			\frac{\xi_{\text{spherical}}}{\xi_{\text{flat}}} = \sqrt{\frac{4\pi}{9}} = 1.1827
		\end{equation}
	\end{tcolorbox}
	
	\subsection{Physical Origin}
	The correction factor $\sqrt{4\pi/9}$ arises from:
	\begin{itemize}
		\item $4\pi$ factor: Complete solid angle integration over spherical geometry
		\item Factor $9 = 3^2$: Three-dimensional spatial normalization
		\item Combined effect: Electromagnetic field corrections for spherical vs. flat geometry
	\end{itemize}
	
	\section{$H_0$ Parameter Derivation}
	
	\subsection{T0 Theoretical Prediction}
	The Hubble parameter emerges from the energy field hierarchy:
	\begin{align}
		H_0 &= \xi_{\text{spherical}}^{15.697} \times E_P \\
		&= (1.557 \times 10^{-4})^{15.697} \times 1.2209 \times 10^{19} \text{ GeV} \\
		&= 1.490 \times 10^{-42} \text{ GeV} \\
		&= \boxed{69.9 \text{ km/s/Mpc}}
	\end{align}
	
	where $E_P$ is the Planck energy and the exponent 15.697 emerges from the energy cascade analysis.
	
	\subsection{Unit Conversion}
	From natural units to SI units:
	\begin{align}
		H_0 &= 1.490 \times 10^{-42} \text{ GeV} \times \frac{1.602 \times 10^{-10} \text{ J}}{\text{GeV}} \times \frac{1}{1.055 \times 10^{-34} \text{ J·s}} \\
		&= 2.264 \times 10^{-18} \text{ s}^{-1} \\
		&= 69.9 \text{ km/s/Mpc}
	\end{align}
	
	\section{$\kappa$ Parameter}
	
	\subsection{Energy Loss Mechanism}
	The $\kappa$ parameter emerges from energy loss in field gradients:
	\begin{equation}
		\frac{dE}{dr} = -\xi^2 \omega^2 \frac{2G}{r^2}
	\end{equation}
	
	\subsection{Regime Classification}
	\textbf{Local Regime} ($r \ll H_0^{-1}$):
	\begin{equation}
		\kappa = \alpha_\kappa H_0 \xi_{\text{flat}}^2
	\end{equation}
	
	\textbf{Cosmic Regime} ($r \gg H_0^{-1}$):
	\begin{equation}
		\boxed{\kappa = H_0}
	\end{equation}
	
	\section{Infinite Energy Fields and $\Lambda_E$ Term}
	
	\subsection{Mathematical Consistency Requirement}
	For infinite, homogeneous energy distributions with $\rho_E(x) = \rho_{E0} = \text{constant}$, the standard energy field equation has no bounded solution. This requires introduction of a $\Lambda_E$ term:
	
	\begin{equation}
		\nabla^2 E = 4\pi G \rho_{E0} \cdot E + \Lambda_E \cdot E
	\end{equation}
	
	\subsection{Determination of $\Lambda_E$}
	For a stable homogeneous energy background $E = E_0 = \text{constant}$:
	\begin{equation}
		\Lambda_E = -4\pi G \rho_{E0}
	\end{equation}
	
	Using the Friedmann equation relationship $H_0^2 = \frac{8\pi G \rho_{E0}}{3}$:
	\begin{equation}
		\Lambda_E = -\frac{3H_0^2}{2}
	\end{equation}
	
	\section{Experimental Comparisons}
	
	\subsection{Hubble Parameter Measurements}
	
	\begin{table}[htbp]
		\centering
		\begin{tabular}{lccc}
			\toprule
			\textbf{Source} & \textbf{$H_0$ (km/s/Mpc)} & \textbf{Uncertainty} & \textbf{Method} \\
			\midrule
			\rowcolor{green!20}
			\textbf{T0 Prediction} & \textbf{69.9} & \textbf{Theory} & \textbf{Pure energy theory} \\
			Planck 2018 (CMB) & 67.4 & $\pm$ 0.5 & CMB \\
			SH0ES (Riess et al.) & 74.0 & $\pm$ 1.4 & Cepheids \\
			H0LiCOW & 73.3 & $\pm$ 1.7 & Lensing \\
			DES-SN3YR & 67.8 & $\pm$ 1.3 & Supernovae \\
			\bottomrule
		\end{tabular}
		\caption{T0 prediction vs. experimental measurements of $H_0$}
		\label{tab:h0_comparison}
	\end{table}
	
	\subsection{Agreement Analysis}
	\begin{itemize}
		\item \textbf{T0 vs. Planck}: $69.9$ vs. $67.4$ km/s/Mpc $\rightarrow$ $103.7\%$ agreement
		\item \textbf{T0 vs. SH0ES}: $69.9$ vs. $74.0$ km/s/Mpc $\rightarrow$ $94.4\%$ agreement
		\item \textbf{T0 vs. H0LiCOW}: $69.9$ vs. $73.3$ km/s/Mpc $\rightarrow$ $95.3\%$ agreement
		\item \textbf{T0 vs. Average}: $69.9$ vs. $71.6$ km/s/Mpc $\rightarrow$ $97.6\%$ agreement
	\end{itemize}
	
	\subsection{Hubble Tension Resolution}
	The T0 prediction of $H_0 = 69.9$ km/s/Mpc provides an optimal compromise:
	\begin{itemize}
		\item Only $2.5$ km/s/Mpc from Planck measurement
		\item Only $4.1$ km/s/Mpc from SH0ES measurement
		\item Lies within the range of most experimental uncertainties
	\end{itemize}
	
	\section{Scale Hierarchy Analysis}
	
	\subsection{Energy-Based Scale Relations}
	
	\begin{table}[htbp]
		\centering
		\begin{tabular}{lccc}
			\toprule
			\textbf{Scale} & \textbf{Characteristic Energy} & \textbf{$\xi$ Parameter} & \textbf{Regime} \\
			\midrule
			Planck & $E_P = 1.22 \times 10^{19}$ GeV & $\xi = 2$ & Reference \\
			Higgs (local) & $E_h = 125$ GeV & $\xi_{\text{flat}} = 1.32 \times 10^{-4}$ & Local physics \\
			Higgs (cosmological) & Effective scale & $\xi_{\text{spherical}} = 1.557 \times 10^{-4}$ & Cosmic physics \\
			Proton & $E_p = 0.938$ GeV & $1.54 \times 10^{-19}$ & Local physics \\
			Electron & $E_e = 0.511$ MeV & $8.37 \times 10^{-23}$ & Local physics \\
			\bottomrule
		\end{tabular}
		\caption{Energy scales and corresponding $\xi$ parameters}
		\label{tab:energy_scales}
	\end{table}
	
	\subsection{Transition Scale}
	The transition between local and cosmic regimes occurs at:
	\begin{equation}
		r_{\text{transition}} \sim H_0^{-1} = 1.28 \times 10^{26} \text{ m}
	\end{equation}
	
	This scale marks where electromagnetic geometry corrections become important.
	
	\section{Planck Current Verification}
	
	\subsection{Standard vs. Complete Formulation}
	\textbf{Standard Literature (Incomplete):}
	\begin{equation}
		I_P^{\text{incomplete}} = \sqrt{\frac{c^6\varepsilon_0}{G}} = 9.81 \times 10^{24} \text{ A}
	\end{equation}
	
	\textbf{Geometrically Complete:}
	\begin{equation}
		I_P^{\text{complete}} = \sqrt{\frac{4\pi c^6\varepsilon_0}{G}} = 3.479 \times 10^{25} \text{ A}
	\end{equation}
	
	\textbf{CODATA Reference:} $I_P = 3.479 \times 10^{25}$ A
	
	\textbf{Agreement:} Complete formulation achieves $99.98\%$ accuracy vs. $28.2\%$ for incomplete version.
	
	\section{Mathematical Framework}
	
	\subsection{Energy Field Equation}
	\begin{equation}
		\nabla^2 E = 4\pi G \rho_E(x,t) \cdot E
	\end{equation}
	
	\subsection{Modified Energy Potential}
	\begin{equation}
		\Phi_E(r) = -\frac{GE_{\text{source}}}{r} + \kappa r
	\end{equation}
	
	\subsection{Scale Hierarchy}
	The T0 model connects scales through:
	\begin{equation}
		\text{Planck scale} \xrightarrow{15.697 \text{ steps}} \text{Hubble scale}
	\end{equation}
	
	with each step involving factor $\xi_{\text{spherical}}$ reduction.
	
	\section{Universe Age Calculation}
	
	From the T0 derived $H_0$:
	\begin{align}
		t_{\text{universe}}^{(T0)} &= \frac{1}{H_0} = \frac{1}{2.264 \times 10^{-18} \text{ s}^{-1}} \\
		&= 4.42 \times 10^{17} \text{ s} \\
		&= 14.0 \text{ billion years}
	\end{align}
	
	\textbf{Observational value:} $13.8 \pm 0.2$ billion years
	
	\textbf{Agreement:} $98.6\%$
	
	\section{Key Physical Insights}
	
	\subsection{No Spatial Expansion}
	The T0 model interprets $H_0$ not as expansion rate but as:
	\begin{itemize}
		\item Characteristic energy scale for regime transitions
		\item Energy loss rate to background time field
		\item Threshold for cosmic screening effects
	\end{itemize}
	
	\subsection{Redshift Mechanism}
	\begin{equation}
		z = \frac{\Delta E}{E} = \frac{H_0 \cdot r}{c} \quad \text{(energy loss)}
	\end{equation}
	
	\subsection{Geometry Dependence}
	Different physical regimes require different geometric treatments:
	\begin{itemize}
		\item Local physics: Flat geometry ($\xi_{\text{flat}}$)
		\item Cosmological physics: Spherical geometry ($\xi_{\text{spherical}}$)
		\item Transition at scale $r \sim H_0^{-1}$
	\end{itemize}
	
	\section{Mathematical Consistency}
	
	\subsection{Dimensional Verification}
	All T0 equations maintain dimensional consistency in natural units:
	
	\begin{table}[htbp]
		\centering
		\begin{tabular}{lccc}
			\toprule
			\textbf{Equation} & \textbf{Left Side} & \textbf{Right Side} & \textbf{Status} \\
			\midrule
			Energy field & $[E] = [E]$ & $[1/\max(m,\omega)] = [E^{-1}]$ & \checkmark \\
			Field equation & $[\nabla^2 E] = [E^3]$ & $[4\pi G \rho_E E] = [E^3]$ & \checkmark \\
			Energy loss & $[dE/dr] = [E^2]$ & $[\xi^2 \omega^2 2G/r^2] = [E^2]$ & \checkmark \\
			$\Lambda_E$ term & $[\Lambda_E] = [E^2]$ & $[4\pi G \rho_{E0}] = [E^2]$ & \checkmark \\
			$\kappa$ parameter & $[\kappa] = [E^2]$ & $[H_0 \hbar] = [E^2]$ & \checkmark \\
			\bottomrule
		\end{tabular}
		\caption{Dimensional consistency verification}
		\label{tab:dimensional_check}
	\end{table}
	
	\subsection{Internal Consistency}
	Key relationships satisfied by the T0 model:
	\begin{align}
		\Lambda_E &= -\frac{3H_0^2}{2} \quad \text{(Friedmann relation)} \\
		\kappa &= H_0 \quad \text{(cosmic regime)} \\
		\xi_{\text{spherical}} &= \xi_{\text{flat}} \times \sqrt{\frac{4\pi}{9}} \quad \text{(electromagnetic geometry)} \\
		H_0 &= 69.9 \text{ km/s/Mpc} \quad \text{(theoretical prediction)}
	\end{align}
	
	\section{Conclusions}
	
	The energy-based T0 formulation successfully derives the Hubble parameter $H_0 = 69.9$ km/s/Mpc from first principles, providing optimal resolution of the Hubble tension. The key discoveries include:
	
	\begin{itemize}
		\item Geometry-dependent $\xi$ parameters with $4\pi$ corrections
		\item Direct connection between quantum and cosmological energy scales
		\item Parameter-free derivation achieving greater than 95\% experimental agreement
		\item Alternative interpretation of cosmological observations without spatial expansion
		\item Energy field unification spanning Planck to Hubble scales
	\end{itemize}
	
	The fundamental relationship $\kappa = H_0$ in the cosmic regime establishes a direct bridge between energy field theory and cosmology, suggesting that large-scale cosmic phenomena emerge from the same principles governing quantum energy field interactions.
	
	\begin{thebibliography}{99}
		
		\bibitem{planck2020}
		Planck Collaboration (2020). Planck 2018 results. VI. Cosmological parameters. \textit{Astronomy \& Astrophysics}, 641, A6.
		
		\bibitem{riess2019}
		Riess, A. G., et al. (2019). Large Magellanic Cloud Cepheid Standards Provide a 1\% Foundation for the Determination of the Hubble Constant and Stronger Evidence for Physics beyond $\Lambda$CDM. \textit{The Astrophysical Journal}, 876, 85.
		
		\bibitem{wong2020}
		Wong, K. C., et al. (2020). H0LiCOW -- XIII. A 2.4 per cent measurement of $H_0$ from lensed quasars: 5.3$\sigma$ tension between early- and late-Universe probes. \textit{Monthly Notices of the Royal Astronomical Society}, 498, 1420-1439.
		
		\bibitem{codata2018}
		CODATA (2018). \textit{CODATA Internationally recommended 2018 values of the Fundamental Physical Constants}. NIST.
		
		\bibitem{weinberg2008}
		Weinberg, S. (2008). \textit{Cosmology}. Oxford University Press.
		
		\bibitem{pascher2025}
		Pascher, J. (2025). \textit{Pure Energy Formulation of T0 Theory: Mass-Free Approach to Fundamental Physics}.
		
	\end{thebibliography}
	
\end{document}