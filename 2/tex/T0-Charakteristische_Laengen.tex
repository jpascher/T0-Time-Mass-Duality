\documentclass[12pt,a4paper]{article}
\usepackage[utf8]{inputenc}
\usepackage{amsmath, amssymb, siunitx}
\usepackage[margin=1in]{geometry}
\usepackage[locale=DE]{siunitx} % Für den Dezimalpunkt in Zahlen
\sisetup{per-mode = fraction, output-decimal-marker={,}}

\title{Charakteristische Längenskalen der T0-Theorie und ihre kosmische Bedeutung}
\author{}
\date{}

\begin{document}
	
	\maketitle
	
	\section{Charakteristische Skalen $L_0$, $E_0$, $m_0$, $T_0$}
	
	\subsection{Definition in natürlichen Einheiten ($\hbar = c = 1$)}
	
	Die Theorie postuliert eine dimensionslose Fundamentalkonstante $\xi$:
	\[
	\xi = \frac{4}{3} \times 10^{-4} \approx \num{1,333e-4}
	\]
	
	Aus $\xi$ werden die charakteristischen Skalen der Theorie abgeleitet. In natürlichen Einheiten gelten folgende Definitionen:
	
	\begin{table}[h!]
		\centering
		\begin{tabular}{|c|c|c|}
			\hline
			\textbf{Größe} & \textbf{Wert} & \textbf{Bedeutung} \\
			\hline
			Energie $E_0$ & $E_0 = \xi^{-1} \, \text{GeV}$ & Charakteristische Energie \\
			Masse $m_0$ & $m_0 = \xi^{-1} \, \text{GeV}$ & Charakteristische Masse \\
			Länge $L_0$ & $L_0 = \xi \, \text{GeV}^{-1}$ & Fundamentale ``Korngröße'' der Raumzeit \\
			Temperatur $T_0$ & $T_0 \sim \xi^{-1} \, \text{GeV}$ & Charakteristische Temperatur \\
			\hline
		\end{tabular}
		\caption{T0-Charakteristische Größen in natürlichen Einheiten. Die dimensionslose Zahl $\xi$ skaliert die physikalischen Einheiten.}
	\end{table}
	
	\[
	\Rightarrow \quad E_0 = m_0 \approx \frac{1}{\num{1,333e-4}} \, \text{GeV} = 7500 \, \text{GeV} \quad \Rightarrow \quad L_0 = \num{1,333e-4} \, \text{GeV}^{-1}
	\]
	
	\subsection{Umrechnung in SI-Einheiten}
	
	Der Konversionsfaktor zwischen Länge und Energie ist:
	\[
	1 \, \text{GeV}^{-1} = \hbar c \approx \num{1,973e-16} \, \text{m}
	\]
	
	Die charakteristische Länge in Metern ist somit:
	\[
	L_0 = \xi \cdot \hbar c = \num{1,333e-4} \cdot \num{1,973e-16} \, \text{m} \approx \num{2,63e-20} \, \text{m}
	\]
	
	\section{Kosmische Länge $L_{\rm cosmic}$ und der Hierarchie-Exponent $N$}
	
	\subsection{Definition der kosmischen Länge}
	
	Die charakteristische kosmische Länge wird durch den Hubble-Radius definiert:
	\[
	L_{\rm cosmic} \sim L_H = \frac{c}{H_0} \approx \num{1,4e26} \, \text{m}
	\]
	
	\subsection{Herleitung der Hierarchie über $\xi$}
	
	Die fundamentale Beobachtung der T0-Theorie ist, dass sich das Verhältnis zwischen der kosmischen und der mikroskopischen Länge durch eine einfache Potenz der Fundamentalkonstante $\xi$ ausdrücken lässt:
	\[
	\frac{L_{\rm cosmic}}{L_0} \sim \xi^{-N}
	\]
	Durch Einsetzen der Zahlenwerte kann der Hierarchie-Exponent $N$ bestimmt werden:
	\[
	\frac{L_{\rm cosmic}}{L_0} \approx \frac{\num{1,4e26}}{\num{2,63e-20}} \approx \num{5,32e45}
	\]
	\[
	\xi^{-N} = (\num{1,333e-4})^{-N} = \num{5,32e45}
	\]
	Logarithmieren zur Basis 10 liefert:
	\[
	-N \cdot \log_{10}(\num{1,333e-4}) = \log_{10}(\num{5,32e45})
	\]
	\[
	-N \cdot (\log_{10}(\num{1,333}) + \log_{10}(10^{-4})) = \log_{10}(\num{5,32}) + 45
	\]
	\[
	-N \cdot (\num{0,1249} - 4) = \num{0,7259} + 45
	\]
	\[
	-N \cdot (-\num{3,8751}) = \num{45,7259}
	\]
	\[
	N \cdot \num{3,8751} = \num{45,7259}
	\]
	\[
	N = \frac{\num{45,7259}}{\num{3,8751}} \approx 11,8
	\]
	
	Die mikroskopische und die kosmische Skala sind thus durch einen Faktor $\xi^{-12}$ verbunden.
	\[
	L_{\rm cosmic} \sim L_0 \cdot \xi^{-12}
	\]
	
	\section{Zusammenfassung und Interpretation}
	
	\begin{itemize}
		\item Die T0-Theorie führt eine dimensionslose Fundamentalkonstante $\xi = \num{1,333e-4}$ ein.
		\item Daraus leiten sich die charakteristischen Skalen ab:
		\begin{align*}
			L_0 &= \xi \, \text{GeV}^{-1} \approx \num{2,63e-20} \, \text{m} \quad \text{(Mikroskopische Länge)} \\
			E_0 &= m_0 = \xi^{-1} \, \text{GeV} \approx 7500 \, \text{GeV}
		\end{align*}
		\item Das beobachtete Universum operiert auf einer Skala von $L_{\rm cosmic} \approx \num{1,4e26} \, \text{m}$.
		\item Der gewaltige Skalenunterschied von $\sim 46$ Größenordnungen wird durch eine Potenz von $\xi$ erklärt:
		\[
		\frac{L_{\rm cosmic}}{L_0} \sim \xi^{-12}
		\]
		\item Die ~4\% Abweichung in der Berechnung von $N$ ($11,8$ vs. $12$) könnte auf dynamische Aspekte des $\xi$-Feldes oder Messungenauigkeiten der kosmologischen Parameter hinweisen und stellt eine potentielle Vorhersage der Theorie dar.
	\end{itemize}
	
	Die Stärke der T0-Theorie liegt in dieser elegante Erklärung der Hierarchie zwischen mikroskopischen und kosmischen Phänomenen durch eine einzige, fundamentale dimensionslose Konstante.
	
\end{document}