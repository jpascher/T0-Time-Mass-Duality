\documentclass[12pt,a4paper]{article}
\usepackage[utf8]{inputenc}
\usepackage{amsmath, amssymb, siunitx}

\title{T0-Charakteristische Längen und kosmische Skalen in der T0-Theorie}
\author{}
\date{}

\begin{document}
	
	\maketitle
	
	\section{Charakteristische Skalen $L_0$, $E_0$, $m_0$, $T_0$}
	
	\subsection{Definition in natürlichen Einheiten ($\hbar = c = 1$)}
	
	\begin{table}[h!]
		\centering
		\begin{tabular}{|c|c|c|}
			\hline
			\textbf{Größe} & \textbf{Dimension} & \textbf{Beziehung} \\
			\hline
			Energie $E_0$ & [E] = GeV & $E_0 = 1/\xi$ \\
			Masse $m_0$ & [m] = GeV & $m_0 = E_0$ \\
			Länge $L_0$ & [L] = GeV$^{-1}$ & $L_0 = 1/E_0 = \xi$ \\
			Temperatur $T_0$ & [E] = GeV & $T_0 \sim E_0$ \\
			\hline
		\end{tabular}
		\caption{T0-Charakteristische Größen in natürlichen Einheiten.}
	\end{table}
	
	\[
	\xi = \frac{4}{3} \times 10^{-4} \quad \Rightarrow \quad E_0 = 1/\xi = 7500 \,\text{GeV} \quad \Rightarrow \quad L_0 = \xi
	\]
	
	\subsection{Umrechnung in SI-Einheiten}
	
	\[
	1 \,\text{GeV}^{-1} = \hbar c = 1.973 \times 10^{-16}\,\text{m}
	\]
	
	\[
	L_0 = \xi \cdot \hbar c = 4/3 \times 10^{-4} \cdot 1.973 \times 10^{-16}\,\text{m} \approx 2.63 \times 10^{-20}\,\text{m}
	\]
	
	\subsection{Physikalische Bedeutung}
	
	\begin{itemize}
		\item $L_0$ ist die fundamentale "Korngröße" der Raumzeit.
		\item $E_0$ und $m_0$ repräsentieren die zugehörigen Energien/Massen.
		\item $T_0$ ist die charakteristische Temperatur des $\xi$-Feldes.
	\end{itemize}
	
	\section{Kosmische Länge $L_{\rm cosmic}$ und CMB-Bezug}
	
	\subsection{Definition}
	
	\[
	L_{\rm cosmic} \sim \frac{c}{H_0} \sim 10^{26}\,\text{m}
	\]
	
	\subsection{CMB-Energiedichte}
	
	\[
	\rho_{\rm CMB} = \frac{\pi^2}{15} (k_B T_{\rm CMB})^4 \approx 4.0 \times 10^{-14}\, \text{J/m}^3
	\]
	
	Die Verbindung zur T0-Länge:
	
	\[
	L_0^{\rm eff} = \left(\frac{\xi}{\rho_{\rm CMB}}\right)^{1/4} \sim 10^{-4}\,\text{m}
	\]
	
	\subsection{Verbindung über $\xi$-Hierarchie}
	
	\[
	\frac{L_{\rm cosmic}}{L_0} \sim \xi^{-N} \quad \Rightarrow \quad L_{\rm cosmic} \sim L_0 \, \xi^{-N}, \quad N \approx 30
	\]
	
	\section{Prozentuale Abweichung von der Hubble-Länge}
	
	\[
	\Delta_{\%} = \frac{L_H - L_{\rm cosmic}}{L_H} \times 100\% \approx 4\%
	\]
	
	\section{Bemerkenswerter Zusammenhang}
	
	\begin{itemize}
		\item Die Zahl $\xi \sim 4/3 \times 10^{-4}$ erscheint sowohl in der T0-Länge als auch in der CMB-Skala.
		\item Die mikroskopische Skala $L_0$ und die kosmische Skala $L_{\rm cosmic}$ sind über Potenzen von $\xi$ verbunden.
		\item Dies legt nahe, dass $\xi$ eine Brücke zwischen mikroskopischen und kosmischen Skalen darstellt.
	\end{itemize}
	
	\section{Zusammenfassung}
	
	\begin{itemize}
		\item T0-Charakteristische Skalen: $L_0 = \xi \approx 2.63 \times 10^{-20}\,\text{m}$, $E_0 = m_0 = 1/\xi$, $T_0 \sim E_0$.
		\item Kosmische Länge $L_{\rm cosmic} \sim 10^{26}\,\text{m}$ über Potenzen von $\xi$ aus T0 ableitbar.
		\item Prozentuale Abweichung zur Hubble-Länge ca. 4\%.
		\item $\xi$ verknüpft mikroskopische und kosmische Skalen hierarchisch.
	\end{itemize}
\section{Zweite Herleitung: Charakteristische Länge $r_0$}

\subsection{Definition von $r_0$ aus der vereinfachten Lagrangedichte}

In manchen Herleitungen der T0-Theorie wird eine charakteristische Länge $r_0$ direkt aus der Lagrangedichte des $\xi$-Feldes definiert:

\begin{equation}
	\mathcal{L} \sim \frac{1}{2} (\partial_\mu \xi)^2 - V(\xi), \quad V(\xi) = \frac{\xi^2}{2 r_0^2} + \dots
\end{equation}

Die Minimierung der Wirkung liefert dann eine natürliche Längenskala:

\begin{equation}
	r_0 = \sqrt{\frac{\langle \xi^2 \rangle}{V(\xi)}} \sim \text{Charakteristische Länge der $\xi$-Fluktuationen}.
\end{equation}

Diese Definition ist unabhängig von kosmologischen Parametern und ergibt eine **mikroskopische Skala**, die der T0-Länge \(L_{T0}\) entspricht, also:

\begin{equation}
	r_0 \sim L_{T0} = \frac{\hbar c}{E_0} \approx 2.63 \cdot 10^{-20}\, \mathrm{m}.
\end{equation}

\subsection{Herleitung von $r_0$ in Bezug auf die Plancklänge}

Alternativ kann $r_0$ über die Plancklänge $L_{\rm Planck}$ hergeleitet werden, wobei $\xi$ als dimensionslose Hierarchie-Konstante dient:

\begin{equation}
	r_0 \sim \sqrt{\xi} \, L_{\rm Planck} \quad \Rightarrow \quad r_0 \sim 10^{-20}\, \mathrm{m}.
\end{equation}

Damit bestätigt sich, dass $r_0$ auf derselben Größenordnung liegt wie $L_{T0}$, jedoch aus einer anderen theoretischen Ausgangslage:  

\begin{itemize}
	\item Erste Kette: $L_{T0}$ direkt aus 3D-Raumgeometrie und $\xi$.
	\item Zweite Kette: $r_0$ aus Lagrangedichte bzw. Plancklänge.
\end{itemize}

\subsection{Zusammenhang zu kosmischen Längen $L_0$}

Auch über $r_0$ lässt sich die Hierarchie zwischen mikroskopischer und kosmischer Skala ausdrücken:

\begin{equation}
	\frac{L_0}{r_0} \sim 10^{46} \sim \xi^{-1} \text{ bis } \xi^{-2} \quad (\text{analog zu } L_0/L_{T0}).
\end{equation}

\textbf{Fazit:} $r_0$ liefert eine konsistente zweite Beweiskette, die unabhängig vom direkten geometrischen Ansatz ist, aber auf dieselben mikroskopischen Längenordnungen wie $L_{T0}$ kommt und die kosmische Hierarchie über $\xi$ reproduziert.
	
\end{document}
