\documentclass[12pt,a4paper]{article}
\usepackage[utf8]{inputenc}
\usepackage{amsmath, amssymb, siunitx}
\usepackage[ngerman]{babel} % Für korrekte Silbentrennung auf Deutsch

\title{T0-Charakteristische Längen und kosmische Skalen in der T0-Theorie}
\author{}
\date{}

\begin{document}
	
	\maketitle
	
	\section{Charakteristische Skalen $L_0$, $E_0$, $m_0$, $T_0$}
	
	\subsection{Definition in natürlichen Einheiten ($\hbar = c = 1$)}
	
	\begin{table}[h!]
		\centering
		\begin{tabular}{|c|c|c|}
			\hline
			\textbf{Größe} & \textbf{Dimension} & \textbf{Beziehung} \\
			\hline
			Energie $E_0$ & [E] = GeV & $E_0 = 1/\xi$ \\
			Masse $m_0$ & [m] = GeV & $m_0 = E_0$ \\
			Länge $L_0$ & [L] = GeV$^{-1}$ & $L_0 = 1/E_0 = \xi$ \\
			Temperatur $T_0$ & [E] = GeV & $T_0 = E_0$ \\
			\hline
		\end{tabular}
		\caption{T0-Charakteristische Größen in natürlichen Einheiten.}
	\end{table}
	
	\[
	\xi = \frac{4}{3} \times 10^{-4} \quad \Rightarrow \quad E_0 = 1/\xi = 7500 \,\text{GeV} \quad \Rightarrow \quad L_0 = \xi
	\]
	
	\subsection{Umrechnung in SI-Einheiten}
	
	\[
	1 \,\text{GeV}^{-1} = \hbar c = \SI{1.973e-16}{\meter}
	\]
	
	\[
	L_0 = \xi \cdot \hbar c = \frac{4}{3} \times 10^{-4} \cdot \SI{1.973e-16}{\meter} \approx \SI{2.63e-20}{\meter}
	\]
	
	\subsection{Physikalische Bedeutung}
	
	\begin{itemize}
		\item $L_0$ ist die fundamentale "Korngröße" der Raumzeit und stellt eine minimale Länge dar.
		\item $E_0$ und $m_0$ repräsentieren die zugehörigen charakteristischen Energie- bzw. Massenskalen.
		\item $T_0$ ist die charakteristische Temperatur des $\xi$-Feldes.
	\end{itemize}
	
	\section{Kosmische Länge $L_{\mathrm{cosmic}}$ und CMB-Bezug}
	
	\subsection{Definition}
	
	\[
	L_{\mathrm{cosmic}} \sim \frac{c}{H_0} \sim \SI{e26}{\meter}
	\]
	
	\subsection{CMB-Energiedichte}
	
	\[
	\rho_{\mathrm{CMB}} = \frac{\pi^2}{15} \frac{(k_B T_{\mathrm{CMB}})^4}{(\hbar c)^3} \approx \SI{4.17e-14}{\joule\per\cubic\meter}
	\]
	
	Die Verbindung zur T0-Länge erfolgt über die charakteristische Vakuumlänge $L_\xi$:
	
	\[
	L_\xi = \left(\frac{\hbar c}{\xi \rho_{\mathrm{CMB}}}\right)^{1/4} \sim \SI{e-4}{\meter}
	\]
	
	\subsection{Verbindung über $\xi$-Hierarchie}
	
	\[
	\frac{L_{\mathrm{cosmic}}}{L_\xi} \sim \xi^{-N} \quad \Rightarrow \quad L_{\mathrm{cosmic}} \sim L_\xi \, \xi^{-N}, \quad N \approx 30
	\]
	
	\section{Prozentuale Abweichung von der Hubble-Länge}
	
	\[
	\Delta_{\%} = \frac{L_H - L_{\mathrm{cosmic}}}{L_H} \times 100\% \approx 4\%
	\]
	
	\section{Bemerkenswerter Zusammenhang}
	
	\begin{itemize}
		\item Die dimensionslose Konstante $\xi \sim 4/3 \times 10^{-4}$ erscheint in verschiedenen physikalischen Kontexten.
		\item Die mikroskopische Skala $L_0$ und die kosmische Skala $L_{\mathrm{cosmic}}$ sind über Potenzen von $\xi$ verbunden.
		\item Die charakteristische Vakuumlänge $L_\xi \sim 0.1$ mm bildet eine Brücke zwischen Quantenphänomenen und kosmologischen Skalen.
	\end{itemize}
	
	\section{Zusammenfassung}
	
	\begin{itemize}
		\item T0-Charakteristische Skalen: $L_0 = \xi \approx \SI{2.63e-20}{\meter}$, $E_0 = m_0 = 1/\xi = 7500$ GeV, $T_0 = E_0$.
		\item Charakteristische Vakuumlänge: $L_\xi \sim \SI{e-4}{\meter}$ aus CMB-Energiedichte ableitbar.
		\item Kosmische Länge $L_{\mathrm{cosmic}} \sim \SI{e26}{\meter}$ über Potenzen von $\xi$ aus $L_\xi$ ableitbar.
		\item Prozentuale Abweichung zur Hubble-Länge ca. 4\%.
		\item $\xi$ verknüpft mikroskopische und kosmische Skalen hierarchisch.
	\end{itemize}
	
	\section{Zweite Herleitung: Charakteristische Länge $r_0$}
	
	\subsection{Definition von $r_0$ aus der vereinfachten Lagrangedichte}
	
	In manchen Herleitungen der T0-Theorie wird eine charakteristische Länge $r_0$ direkt aus der Lagrangedichte des $\xi$-Feldes definiert:
	
	\begin{equation}
		\mathcal{L} \sim \frac{1}{2} (\partial_\mu \xi)^2 - V(\xi), \quad V(\xi) = \frac{\xi^2}{2 r_0^2} + \dots
	\end{equation}
	
	Die Minimierung der Wirkung liefert dann eine natürliche Längenskala:
	
	\begin{equation}
		r_0 \sim \sqrt{\frac{\langle \xi^2 \rangle}{V(\xi)}} \sim \text{Charakteristische Länge der $\xi$-Fluktuationen}.
	\end{equation}
	
	Diese Definition ist unabhängig von kosmologischen Parametern und ergibt eine \textbf{mikroskopische Skala}, die der T0-Länge $L_0$ entspricht, also:
	
	\begin{equation}
		r_0 \sim L_0 = \xi \cdot \hbar c \approx \SI{2.63e-20}{\meter}.
	\end{equation}
	
	\subsection{Herleitung von $r_0$ in Bezug auf die Plancklänge}
	
	Alternativ kann $r_0$ über die Plancklänge $L_{\mathrm{Planck}}$ hergeleitet werden, wobei $\xi$ als dimensionslose Hierarchie-Konstante dient:
	
	\begin{equation}
		r_0 \sim \xi \, L_{\mathrm{Planck}} \quad \Rightarrow \quad r_0 \sim \SI{e-20}{\meter}.
	\end{equation}
	
	Damit bestätigt sich, dass $r_0$ auf derselben Größenordnung liegt wie $L_0$, jedoch aus einer anderen theoretischen Ausgangslage:  
	
	\begin{itemize}
		\item Erste Herleitung: $L_0$ direkt aus der universellen $\xi$-Konstante.
		\item Zweite Herleitung: $r_0$ aus Lagrangedichte bzw. Plancklänge.
	\end{itemize}
	
	\subsection{Zusammenhang zu kosmischen Längen}
	
	Auch über $r_0$ lässt sich die Hierarchie zwischen mikroskopischer und kosmischer Skala ausdrücken:
	
	\begin{equation}
		\frac{L_{\mathrm{cosmic}}}{r_0} \sim 10^{46} \sim \xi^{-N}, \quad N \approx 30
	\end{equation}
	
	\textbf{Fazit:} $r_0$ liefert eine konsistente zweite Beweiskette, die unabhängig vom direkten geometrischen Ansatz ist, aber auf dieselben mikroskopischen Längenordnungen wie $L_0$ kommt und die kosmische Hierarchie über $\xi$ reproduziert.
	
\end{document}