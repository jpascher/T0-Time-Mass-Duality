\documentclass[12pt,a4paper]{article}
\usepackage[utf8]{inputenc}
\usepackage[T1]{fontenc}
\usepackage[english]{babel}
\usepackage[left=2cm,right=2cm,top=2cm,bottom=2cm]{geometry}
\usepackage{lmodern}
\usepackage{amsmath}
\usepackage{amssymb}
\usepackage{physics}
\usepackage{hyperref}
\usepackage{tcolorbox}
\usepackage{booktabs}
\usepackage{enumitem}
\usepackage[table,xcdraw]{xcolor}
\usepackage{graphicx}
\usepackage{float}
\usepackage{mathtools}
\usepackage{amsthm}
\usepackage{siunitx}
\usepackage{fancyhdr}
\usepackage{microtype} % Better text formatting

% Headers and Footers
\pagestyle{fancy}
\fancyhf{}
\fancyhead[L]{Johann Pascher}
\fancyhead[R]{Deterministic Quantum Mechanics via T0-Energy Field}
\fancyfoot[C]{\thepage}
\renewcommand{\headrulewidth}{0.4pt}
\renewcommand{\footrulewidth}{0.4pt}

% Custom Commands - Fixed braket definition
\newcommand{\Tfield}{T(x,t)}
\newcommand{\Efield}{E(x,t)}
\newcommand{\psiket}[1]{|#1\rangle}
\newcommand{\psibra}[1]{\langle#1|}
% Use existing braket from physics package instead of redefining
\newcommand{\xipar}{\xi}
\newcommand{\betaT}{\beta_{\text{T}}}

\hypersetup{
	colorlinks=true,
	linkcolor=blue,
	citecolor=blue,
	urlcolor=blue,
	pdftitle={Deterministic Quantum Mechanics via T0-Energy Field Formulation},
	pdfauthor={Johann Pascher},
	pdfsubject={T0 Model, Deterministic QM, Energy Field Physics}
}

\newtheorem{theorem}{Theorem}[section]
\newtheorem{proposition}[theorem]{Proposition}
\newtheorem{definition}[theorem]{Definition}

\begin{document}
	
	\title{Deterministic Quantum Mechanics via T0-Energy Field Formulation: \\
		From Probability-based to Field-based Microphysics}
	\author{Johann Pascher\\
		Department of Communications Engineering, \\H{\"o}here Technische Bundeslehranstalt (HTL), Leonding, Austria\\
		\texttt{johann.pascher@gmail.com}}
	\date{\today}
	
	\maketitle
	
	\begin{abstract}
		This document presents a revolutionary alternative to probability-based quantum mechanics through the deterministic T0-energy field formulation. Based on the closed, parameter-free T0 model, we demonstrate how quantum mechanical phenomena can be calculated using deterministic energy fields $\Tfield = 1/\max(\Efield, \omega)$, without relying on probability amplitudes and their collapse. The formulation preserves all experimentally verified predictions of standard quantum mechanics while extending these with precise single-measurement predictions and eliminating fundamental interpretation problems. Concrete calculation examples for spin systems, entanglement, and quantum transitions demonstrate the practical applicability of deterministic energy field physics in the microscopic domain.
	\end{abstract}
	
	\tableofcontents
	\newpage
	
	\section{Introduction: The Probability Problem of Quantum Mechanics}
	
	Standard quantum mechanics is based on fundamental probability concepts that lead to a series of interpretation problems:
	
	\subsection{Current Problems of Standard QM}
	
	\textbf{Probability Basis}:
	\begin{equation}
		\psiket{\psi} = \alpha\psiket{\uparrow} + \beta\psiket{\downarrow}
	\end{equation}
	
	with probabilities:
	\begin{align}
		P(\uparrow) &= |\alpha|^2 \\
		P(\downarrow) &= |\beta|^2
	\end{align}
	
	\textbf{Fundamental Problems}:
	\begin{itemize}
		\item Only statistical predictions possible
		\item "Wave function collapse" mysterious and non-unitary
		\item Many-worlds vs.~Copenhagen vs.~other interpretations
		\item Measurement problem: When/how does collapse occur?
		\item No deterministic single-event predictions
	\end{itemize}
	
	\subsection{T0-Energy Field as Alternative}
	
	The closed T0 model offers a fundamental alternative:
	
	\begin{tcolorbox}[colback=blue!5!white,colframe=blue!75!black,title=T0 Fundamental Principle]
		Instead of indeterminate probability amplitudes, the T0 model uses \textbf{deterministic energy fields}:
		
		\begin{equation}
			\boxed{\Tfield = \frac{1}{\max(\Efield, \omega)}}
		\end{equation}
		
		All quantum mechanical phenomena arise from the deterministic evolution of these energy fields.
	\end{tcolorbox}
	
	\section{Foundations of T0-Energy Field Quantum Mechanics}
	
	\subsection{Fundamental Field Equation}
	
	The T0 model is based on the field equation:
	\begin{equation}
		\nabla^2 \Efield = 4\pi G \rho_E(\vec{x},t) \cdot \Efield
		\label{eq:energy_field_equation}
	\end{equation}
	
	where $\rho_E(\vec{x},t)$ is the energy density and $\Efield$ is the fundamental energy field.
	
	\textbf{Dimensional verification}: $[\nabla^2 \Efield] = [E^2][E] = [E^3]$ and $[4\pi G \rho_E \Efield] = [1][E^{-2}][E^4][E] = [E^3]$ \checkmark
	
	\subsection{Time Field Definition}
	
	The intrinsic time field follows from:
	\begin{equation}
		\boxed{\Tfield = \frac{1}{\max(\Efield, \omega)}}
		\label{eq:time_field_definition}
	\end{equation}
	
	\textbf{Dimensional verification}: $[\Tfield] = [1/E] = [E^{-1}]$ \checkmark
	
	\textbf{Physical meaning}: The time field is inversely proportional to the characteristic energy scale and represents the local "time texture" of space.
	
	\subsection{E = m Identity in Natural Units}
	
	In the T0 model, fundamentally:
	\begin{equation}
		\boxed{E = m \text{ (in natural units)}}
	\end{equation}
	
	This is not a conversion, but an \textbf{identity} - different names for the same physical quantity.
	
	\section{Elimination of Probability Interpretation}
	
	\subsection{Standard QM State Description}
	
	\textbf{Standard approach}:
	\begin{equation}
		\psiket{\psi} = \sum_i c_i \psiket{i}
	\end{equation}
	
	with $P_i = |c_i|^2$ as probabilities.
	
	\subsection{T0-Energy Field State Description}
	
	\textbf{T0 alternative}:
	\begin{equation}
		\boxed{\text{State} \equiv \{\Tfield(\vec{x},t), \Efield(\vec{x},t)\}}
	\end{equation}
	
	\textbf{No probabilities} - only deterministic field distributions.
	
	\textbf{Measurement values} result directly from field values:
	\begin{equation}
		\text{Measurement} = f(\Tfield, \Efield) \quad \text{(deterministic)}
	\end{equation}
	
	\section{Concrete Calculation Formulations}
	
	\subsection{Spin-1/2 Systems}
	
	\subsubsection{Standard QM Formulation}
	
	\textbf{State}:
	\begin{equation}
		\psiket{\psi} = \alpha\psiket{\uparrow} + \beta\psiket{\downarrow}
	\end{equation}
	
	\textbf{Expectation value}:
	\begin{equation}
		\langle \sigma_z \rangle = \braket{\psi}{\sigma_z \psi} = |\alpha|^2 - |\beta|^2
	\end{equation}
	
	\subsubsection{T0-Energy Field Formulation}
	
	\textbf{State}: Energy field configuration
	\begin{align}
		T_{\uparrow}(\vec{x}) &= \frac{1}{E_{\uparrow}(\vec{x})} \\
		T_{\downarrow}(\vec{x}) &= \frac{1}{E_{\downarrow}(\vec{x})}
	\end{align}
	
	\textbf{Deterministic expectation value}:
	\begin{equation}
		\boxed{\langle \sigma_z \rangle_{T0} = \frac{T_{\downarrow} - T_{\uparrow}}{T_{\downarrow} + T_{\uparrow}}}
	\end{equation}
	
	\textbf{Dimensional verification}: $[\langle \sigma_z \rangle_{T0}] = [T/T] = [1]$ (dimensionless) \checkmark
	
	\subsubsection{Equivalence Proof}
	
	For $|\alpha|^2 = T_{\downarrow}/(T_{\downarrow} + T_{\uparrow})$ and $|\beta|^2 = T_{\uparrow}/(T_{\downarrow} + T_{\uparrow})$:
	
	\begin{align}
		\langle \sigma_z \rangle_{QM} &= |\alpha|^2 - |\beta|^2 \\
		&= \frac{T_{\downarrow}}{T_{\downarrow} + T_{\uparrow}} - \frac{T_{\uparrow}}{T_{\downarrow} + T_{\uparrow}} \\
		&= \frac{T_{\downarrow} - T_{\uparrow}}{T_{\downarrow} + T_{\uparrow}} = \langle \sigma_z \rangle_{T0}
	\end{align}
	
	\textbf{Identical predictions}, but deterministically calculable!
	
	\subsection{Electron Spin in Magnetic Field}
	
	\subsubsection{Standard QM with Statistical Predictions}
	
	\textbf{Hamiltonian}:
	\begin{equation}
		H = -\vec{\mu} \cdot \vec{B} = -\gamma(\sigma_x B_x + \sigma_y B_y + \sigma_z B_z)
	\end{equation}
	
	\textbf{Thermal expectation value}:
	\begin{equation}
		\langle \sigma_z \rangle = \tanh\left(\frac{\mu B}{k_B T}\right)
	\end{equation}
	
	\subsubsection{T0-Deterministic Formulation}
	
	\textbf{Energy field configuration in B-field}:
	\begin{align}
		T_{\uparrow} &= \frac{1}{E_0 + \mu B} \\
		T_{\downarrow} &= \frac{1}{E_0 - \mu B}
	\end{align}
	
	\textbf{Deterministic expectation value}:
	\begin{equation}
		\boxed{\langle \sigma_z \rangle_{T0} = \frac{T_{\downarrow} - T_{\uparrow}}{T_{\downarrow} + T_{\uparrow}} = \frac{2\mu B}{2E_0} = \frac{\mu B}{E_0}}
	\end{equation}
	
	\textbf{For $E_0 = k_B T$}: $\langle \sigma_z \rangle_{T0} = \frac{\mu B}{k_B T} \approx \tanh\left(\frac{\mu B}{k_B T}\right)$ for small fields.
	
	\section{Entanglement without Probability Superposition}
	
	\subsection{Standard QM Entanglement}
	
	\textbf{Bell state}:
	\begin{equation}
		\psiket{\Psi^-} = \frac{1}{\sqrt{2}}(\psiket{\uparrow\downarrow} - \psiket{\downarrow\uparrow})
	\end{equation}
	
	\textbf{Problem}: Mysterious "superposition" of non-classical states.
	
	\subsection{T0-Energy Field Entanglement}
	
	\textbf{Entanglement as correlated energy field structure}:
	\begin{equation}
		\boxed{T_{12}(\vec{x}_1, \vec{x}_2) = \frac{1}{E_1(\vec{x}_1) + E_2(\vec{x}_2)}}
	\end{equation}
	
	\textbf{Correlation condition}:
	\begin{equation}
		E_1(\vec{x}_1) + E_2(\vec{x}_2) = \text{const}
	\end{equation}
	
	\textbf{Physical meaning}: Entanglement arises through \textbf{energetic correlations} in field structure, not through mysterious superposition.
	
	\subsection{Bell Inequality in T0 Formulation}
	
	\textbf{Modified Bell inequality}:
	\begin{equation}
		\boxed{|E(a,b) - E(a,c)| + |E(a',b) + E(a',c)| \leq 2 + \varepsilon(E_1, E_2)}
	\end{equation}
	
	with the T0 correction term:
	\begin{equation}
		\varepsilon(E_1, E_2) = \alpha_{\text{corr}} \left|\frac{1}{E_1} - \frac{1}{E_2}\right| \frac{2G\langle E \rangle}{r}
	\end{equation}
	
	\textbf{Dimensional verification}: $[\varepsilon] = [1][E^{-1}][E^{-2}][E][E] = [1]$ (dimensionless) \checkmark
	
	\section{Deterministic Quantum Transitions}
	
	\subsection{Standard QM: Random Jumps}
	
	\textbf{Transition probability}:
	\begin{equation}
		P_{i \to j} = \frac{|\braket{j}{H_{\text{int}} i}|^2}{\hbar^2} \frac{\sin^2(\omega t/2)}{(\omega/2)^2}
	\end{equation}
	
	\textbf{Problem}: When exactly does the transition occur? (Random)
	
	\subsection{T0: Resonance-based Transitions}
	
	\textbf{Transition condition}:
	\begin{equation}
		\boxed{T(E_i) = T(E_j) \Rightarrow \frac{1}{E_i} = \frac{1}{E_j} \Rightarrow E_i = E_j}
	\end{equation}
	
	\textbf{Transition occurs at energy field resonance}, not randomly!
	
	\textbf{Time deterministically calculable}:
	\begin{equation}
		t_{\text{transition}} = t \text{ for which } E_i(t) = E_j(t)
	\end{equation}
	
	\section{Two-Level System: Deterministic Rabi Oscillations}
	
	\subsection{Standard QM Formulation}
	
	\textbf{Time evolution}:
	\begin{equation}
		\psiket{\psi(t)} = e^{-iHt/\hbar}\psiket{\psi(0)}
	\end{equation}
	
	\textbf{Occupation probabilities}:
	\begin{align}
		P_1(t) &= |\braket{1}{\psi(t)}|^2 = \cos^2(\Omega t/2) \\
		P_2(t) &= |\braket{2}{\psi(t)}|^2 = \sin^2(\Omega t/2)
	\end{align}
	
	\subsection{T0-Deterministic Formulation}
	
	\textbf{Energy field evolution}:
	\begin{align}
		T_1(t) &= T_1(0) \cos^2(\Omega t/2) \\
		T_2(t) &= T_2(0) \sin^2(\Omega t/2)
	\end{align}
	
	with normalization condition:
	\begin{equation}
		T_1(t) + T_2(t) = T_{\text{total}} = \text{const}
	\end{equation}
	
	\textbf{Identical predictions as standard QM}, but without probability interpretation!
	
	\section{Quantum Computing with T0-Energy Fields}
	
	\subsection{Qubit Representation}
	
	\textbf{Standard QM qubit}:
	\begin{equation}
		\psiket{\text{qubit}} = \alpha\psiket{0} + \beta\psiket{1}
	\end{equation}
	
	\textbf{T0-energy field qubit}:
	\begin{equation}
		\boxed{\text{qubit}_{T0} \equiv \{T_0(\vec{x}), T_1(\vec{x})\}}
	\end{equation}
	
	with $T_0 = 1/E_0$ and $T_1 = 1/E_1$.
	
	\subsection{Quantum Gates as Energy Field Transformations}
	
	\subsubsection{Hadamard Gate}
	
	\textbf{Standard}: $H\psiket{0} = \frac{1}{\sqrt{2}}(\psiket{0} + \psiket{1})$
	
	\textbf{T0}: $H_{T0}: T_0 \to T'_0 = \frac{T_0 + T_1}{2}$, $T_1 \to T'_1 = \frac{T_0 + T_1}{2}$
	
	\subsubsection{CNOT Gate}
	
	\textbf{T0 formulation}:
	\begin{equation}
		\text{CNOT}_{T0}: T_{12} \to T'_{12} = f(T_1, T_2)
	\end{equation}
	
	with a deterministic function $f$ that establishes energy field correlation.
	
	\subsection{Quantum Algorithms Deterministically}
	
	\textbf{Shor's algorithm}: Period finding through energy field resonances
	
	\textbf{Grover's search}: Amplitude amplification = energy field focusing
	
	All algorithms run \textbf{deterministically} - the result is uniquely predictable for given initial conditions.
	
	\section{Experimental Consequences and New Measurement Techniques}
	
	\subsection{New Measurement Device Concepts}
	
	\subsubsection{Time Field Detectors}
	
	Devices for direct measurement of $\Tfield(\vec{x})$ instead of statistical frequencies.
	
	\subsubsection{Energy Field Mappers}
	
	Spatial mapping of energy field distribution $\Efield(\vec{x})$.
	
	\subsubsection{T0 Interferometers}
	
	Interference between different energy field configurations.
	
	\subsection{Single-Measurement Predictions}
	
	\textbf{Standard QM}: Only statistical predictions possible
	
	\textbf{T0 model}: \textbf{Precise prediction for each individual measurement}
	
	\begin{equation}
		\text{Single result} = f(\Tfield(\vec{x}_{\text{detector}}, t_{\text{measurement}}))
	\end{equation}
	
	\subsection{T0-Specific Corrections}
	
	The T0 model makes specific predictions that distinguish it from standard QM:
	
	\begin{itemize}
		\item Energy-dependent Bell corrections
		\item Gravitationally coupled quantum correlations
		\item Deterministic collapse-free measurements
	\end{itemize}
	
	\section{Elimination of QM Interpretation Problems}
	
	\subsection{Solved Problems}
	
	\textbf{Measurement problem}: Eliminated - no "collapse", only continuous field evolution
	
	\textbf{Schrödinger's cat}: Eliminated - deterministic field evolution, no superposition
	
	\textbf{Many-worlds vs.~Copenhagen}: Eliminated - single, deterministic reality
	
	\textbf{Quantum jump problem}: Eliminated - continuous transitions at resonances
	
	\subsection{Simplified Quantum Reality}
	
	\begin{tcolorbox}[colback=green!5!white,colframe=green!75!black,title=T0 Quantum Reality]
		\textbf{Simple, deterministic description}:
		\begin{itemize}
			\item Energy fields $\{\Tfield, \Efield\}$ exist as real entities
			\item They evolve according to deterministic field equations
			\item Measurements reveal current field values
			\item No mysterious probability amplitudes
			\item No non-unitary collapse processes
		\end{itemize}
	\end{tcolorbox}
	
	\section{Practical Implementation}
	
	\subsection{Rewriting Existing QM Formulas}
	
	\textbf{Systematic translation}:
	\begin{align}
		|\psi|^2 &\to \text{T-field density} \\
		\braket{\psi}{\hat{O} \psi} &\to \text{Energy field average} \\
		P_i &\to T_i/T_{\text{total}}
	\end{align}
	
	\subsection{New Calculation Methods}
	
	\subsubsection{Field Differential Equations}
	
	Instead of Schrödinger equation:
	\begin{equation}
		\frac{\partial \Tfield}{\partial t} = f(\nabla^2 \Tfield, \Efield, \ldots)
	\end{equation}
	
	\subsubsection{Energy Field Simulations}
	
	Numerical integration of T0 field equations for complex systems.
	
	\subsubsection{Deterministic Monte Carlo}
	
	Statistical methods for deterministic energy field transport.
	
	\section{Comparison: Standard QM vs.~T0-Deterministic QM}
	
	\begin{table}[htbp]
		\centering
		\small
		\begin{tabular}{|p{4.5cm}|p{5.5cm}|p{6cm}|}
			\hline
			\textbf{Aspect} & \textbf{Standard QM} & \textbf{T0-Deterministic QM} \\
			\hline
			Foundation & Probability amplitudes & Deterministic energy fields \\
			\hline
			State description & $|\psi\rangle = \sum_i c_i |i\rangle$ & $\{T(x,t), E(x,t)\}$ \\
			\hline
			Measurement predictions & Only statistical: $P_i = |c_i|^2$ & Deterministic: $f(T, E)$ \\
			\hline
			Time evolution & Unitary + non-unitary collapse & Continuously deterministic \\
			\hline
			Interpretation problems & Many (collapse, many-worlds, ...) & None (single deterministic reality) \\
			\hline
			Single measurements & Unpredictable (random) & Precisely predictable \\
			\hline
			Entanglement & Mysterious superposition & Energy field correlations \\
			\hline
			Bell inequality & Standard form & Modified with T0 correction \\
			\hline
			Experimental equivalence & - & Yes (+ new T0 effects) \\
			\hline
		\end{tabular}
		\caption{Comparison Standard QM vs.~T0-Deterministic QM}
	\end{table}
	
	\section{Conclusions and Outlook}
	
	\subsection{Revolutionary Possibilities}
	
	The T0-energy field formulation opens fundamental new possibilities:
	
	\begin{enumerate}
		\item \textbf{Deterministic quantum mechanics} without probability mysticism
		\item \textbf{Precise single-measurement predictions} instead of only statistics
		\item \textbf{Elimination of all QM interpretation problems}
		\item \textbf{New experimental techniques} (time field detectors, energy field mappers)
		\item \textbf{Deterministic quantum computer algorithms}
		\item \textbf{Simplified quantum reality} (single, deterministic world)
	\end{enumerate}
	
	\subsection{Experimental Verification}
	
	The T0 model makes specific, testable predictions:
	
	\begin{itemize}
		\item Energy-dependent Bell corrections: $\varepsilon(E_1, E_2)$
		\item Gravitationally coupled quantum correlations
		\item Deterministic transitions at calculable times
		\item Direct energy field measurements
	\end{itemize}
	
	\subsection{Paradigm Shift in Quantum Physics}
	
	\begin{tcolorbox}[colback=red!5!white,colframe=red!75!black,title=Quantum Physics Revolution]
		\textbf{From probabilistic to deterministic microphysics}:
		
		The T0-energy field model shows that quantum mechanics without probability amplitudes, without mysterious collapse, and without interpretation problems is possible.
		
		\textbf{All quantum mechanical phenomena arise from the deterministic evolution of energy fields.}
	\end{tcolorbox}
	
	\subsection{Future Research Directions}
	
	\begin{itemize}
		\item Development of time field measurement devices
		\item Experimental verification of T0-Bell corrections
		\item Deterministic quantum computer architectures
		\item Energy field-based QM textbooks
		\item Philosophical reconsideration of quantum reality
	\end{itemize}
	
	\textbf{The T0 model could change quantum mechanics as fundamentally as Einstein revolutionized classical physics.}
	
	\begin{thebibliography}{99}
		\bibitem{pascher_ho_energie_2025}
		Pascher, J. (2025). \href{https://github.com/jpascher/T0-Time-Mass-Duality/blob/main/2/pdf/Ho_EnergieEn.pdf}{\textit{Pure Energy Formulation of $H_0$ and $\kappa$ Parameters in the T0 Model Framework: From Energy Field Theory to Cosmological Scale Relations}}. GitHub Repository: T0-Time-Mass-Duality.
		
		\bibitem{pascher_dyn_masse_2025}
		Pascher, J. (2025). \href{https://github.com/jpascher/T0-Time-Mass-Duality/blob/main/2/pdf/DynMassePhotonenNichtlokalEn.pdf}{\textit{Dynamic Mass of Photons and Its Implications for Nonlocality in the T0 Model: Updated Framework with Complete Geometric Foundations}}. GitHub Repository: T0-Time-Mass-Duality.
		
		\bibitem{pascher_derivation_beta_2025}
		Pascher, J. (2025). \href{https://github.com/jpascher/T0-Time-Mass-Duality/blob/main/2/pdf/DerivationVonBetaEn.pdf}{\textit{Field-Theoretic Derivation of the $\beta_T$ Parameter in Natural Units ($\hbar = c = 1$)}}. GitHub Repository: T0-Time-Mass-Duality.
		
		\bibitem{pascher_temp_einheiten_2025}
		Pascher, J. (2025). \href{https://github.com/jpascher/T0-Time-Mass-Duality/blob/main/2/pdf/TempEinheitenCMBEn.pdf}{\textit{Temperature Units in Natural Units: Field-Theoretic Foundations and CMB Analysis}}. GitHub Repository: T0-Time-Mass-Duality.
		
		\bibitem{pascher_elimination_mass_2025}
		Pascher, J. (2025). \href{https://github.com/jpascher/T0-Time-Mass-Duality/blob/main/2/pdf/Elimination_Of_Mass_Dirac_Tabelle.pdf}{\textit{T0 Model Calculation Verification: Scale Ratio-Based Physics}}. GitHub Repository: T0-Time-Mass-Duality.
		
		\bibitem{bell1964}
		Bell, J.S. (1964). On the Einstein Podolsky Rosen Paradox. \textit{Physics Physique Fizika}, \textbf{1}, 195--200.
		
		\bibitem{einstein1905}
		Einstein, A. (1905). Ist die Trägheit eines Körpers von seinem Energieinhalt abhängig? \textit{Annalen der Physik}, 17, 639.
		
		\bibitem{schrodinger1926}
		Schrödinger, E. (1926). Quantisierung als Eigenwertproblem. \textit{Annalen der Physik}, 79, 361--376.
		
		\bibitem{dirac1928}
		Dirac, P.A.M. (1928). The Quantum Theory of the Electron. \textit{Proceedings of the Royal Society A}, 117, 610--624.
	\end{thebibliography}
	
\end{document}