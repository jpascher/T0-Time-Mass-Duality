\documentclass[12pt,a4paper]{article}
\usepackage[utf8]{inputenc}
\usepackage[T1]{fontenc}
\usepackage[ngerman]{babel}
\usepackage{amsmath,amsfonts,amssymb}
\usepackage{physics}
\usepackage{siunitx}
\usepackage{booktabs}
\usepackage{longtable}
\usepackage{array}
\usepackage{xcolor}
\usepackage{geometry}
\usepackage{textgreek}
\usepackage{fancyhdr}
\usepackage{hyperref}
\usepackage{tocloft}
\geometry{margin=2.5cm}
% Konfiguration von Kopf- und Fu\ss{}zeile
\pagestyle{fancy}
\fancyhf{}
\fancyhead[L]{\textsc{T0-Modell}}
\fancyhead[R]{\textsc{Eine Neuformulierung der Physik}}
\fancyfoot[C]{\thepage}
\renewcommand{\headrulewidth}{0.4pt}
\renewcommand{\footrulewidth}{0.4pt}
% Formatierung des Inhaltsverzeichnisses
\renewcommand{\cfttoctitlefont}{\huge\bfseries\color{blue}}
\renewcommand{\cftsecfont}{\color{blue}}
\renewcommand{\cftsubsecfont}{\color{blue}}
\renewcommand{\cftsecpagefont}{\color{blue}}
\renewcommand{\cftsubsecpagefont}{\color{blue}}
\hypersetup{
	colorlinks=true,
	linkcolor=blue,
	citecolor=blue,
	urlcolor=blue,
	pdftitle={T0-Modell Formelsammlung (Energiebasierte Version)},
	pdfauthor={Johann Pascher},
	pdfsubject={T0-Modell, Zeit-Energie-Dualit\"{a}t, Theoretische Physik},
	pdfkeywords={T0-Theorie, Nat\"{u}rliche Einheiten, Quantenmechanik, Kosmologie}
}
\title{T0-Modell Formelsammlung\\
	\large (Energiebasierte Version)}
\author{Johann Pascher\\
	\small H\"{o}here Technische Bundeslehranstalt (HTL), Leonding, \"{O}sterreich\\
	\small \texttt{johann.pascher@gmail.com}}
\date{\today}

\begin{document}
	
	\maketitle
	\tableofcontents
	\newpage
	
	\section{FUNDAMENTALE PRINZIPIEN}
	
	\subsection{Universeller geometrischer Parameter}
	\begin{itemize}
		\item Der grundlegende Parameter des T0-Modells:
		$$\xi = \frac{4}{3} \times 10^{-4}$$
		
		\item Beziehung zur 3D-Geometrie:
		$$G_3 = \frac{4}{3} \text{ (dreidimensionaler Geometriefaktor)}$$
	\end{itemize}
	
	\subsection{Zeit-Energie-Dualit\"{a}t}
	\begin{itemize}
		\item Grundlegende Dualit\"{a}tsbeziehung:
		$$T_{\text{field}} \cdot E_{\text{field}} = 1$$
		
		\item Charakteristische T0-L\"{a}nge:
		$$r_0 = 2GE$$
		
		\item Charakteristische T0-Zeit:
		$$t_0 = 2GE$$
	\end{itemize}
	
	\subsection{Universelle Wellengleichung}
	\begin{itemize}
		\item D'Alembert-Operator auf Energiefeld:
		$$\square E_{\text{field}} = \left(\nabla^2 - \frac{\partial^2}{\partial t^2}\right) E_{\text{field}} = 0$$
		
		\item Geometriegekoppelte Gleichung:
		$$\square E_{\text{field}} + \frac{G_3}{\ell_P^2} E_{\text{field}} = 0$$
	\end{itemize}
	
	\subsection{Universelle Lagrange-Dichte}
	\begin{itemize}
		\item Fundamentales Wirkungsprinzip:
		$$\boxed{\mathcal{L} = \varepsilon \cdot (\partial E_{\text{field}})^2}$$
		
		\item Kopplungsparameter:
		$$\varepsilon = \frac{\xi}{E_P^2} = \frac{4/3 \times 10^{-4}}{E_P^2}$$
	\end{itemize}
	
	\section{NAT\"{U}RLICHE EINHEITEN UND SKALEN}
	
	\subsection{Nat\"{u}rliche Einheiten}
	\begin{itemize}
		\item Fundamentale Konstanten:
		$$\hbar = c = k_B = 1$$
		
		\item Gravitationskonstante:
		$$G = 1 \text{ numerisch, beh\"{a}lt aber Dimension } [G] = [E^{-2}]$$
	\end{itemize}
	

	\subsection{Planck-Skala als Referenz}
\begin{itemize}
	\item Planck-L\"{a}nge:
	$$\ell_P = \sqrt{G}$$
	
	\item Skalenverh\"{a}ltnis:
	$$\xi_{\text{rat}} = \frac{\ell_P}{r_0}$$
	
	\item Verh\"{a}ltnis zwischen Planck- und T0-Skalen:
	$$\xi = \frac{\ell_P}{r_0} = \frac{\sqrt{G}}{2GE} = \frac{1}{2\sqrt{G} \cdot E}$$
\end{itemize}
	
	\subsection{Energieskalen-Hierarchie}
	\begin{itemize}
		\item Planck-Energie:
		$$E_P = 1 \text{ (Planck-Referenzskala)}$$
		
		\item Elektroschwache Energie:
		$$E_{\text{electroweak}} = \sqrt{\xi} \cdot E_P \approx 0,012 \, E_P$$
		
		\item T0-Energie:
		$$E_{\text{T0}} = \xi \cdot E_P \approx 1,33 \times 10^{-4} \, E_P$$
		
		\item Atomare Energie:
		$$E_{\text{atomic}} = \xi^{3/2} \cdot E_P \approx 1,5 \times 10^{-6} \, E_P$$
	\end{itemize}
	
	\subsection{Universelle Skalierungsgesetze}
	\begin{itemize}
		\item Energieskalenverh\"{a}ltnis:
		$$\frac{E_i}{E_j} = \left(\frac{\xi_i}{\xi_j}\right)^{\alpha_{ij}}$$
		
		\item Wechselwirkungsspezifische Exponenten:
		\begin{align*}
			\alpha_{\text{EM}} &= 1 \quad \text{(lineare elektromagnetische Skalierung)}\\
			\alpha_{\text{weak}} &= 1/2 \quad \text{(Quadratwurzel-schwache Skalierung)}\\
			\alpha_{\text{strong}} &= 1/3 \quad \text{(Kubikwurzel-starke Skalierung)}\\
			\alpha_{\text{grav}} &= 2 \quad \text{(quadratische Gravitationsskalierung)}
		\end{align*}
	\end{itemize}
	
	\section{ELEKTROMAGNETISMUS UND KOPPLUNG}
	
	\subsection{Kopplungskonstanten}
\begin{itemize}
	\item Elektromagnetische Kopplung:
	$$\alpha_{\text{EM}} = 1 \text{ (nat\"{u}rliche Einheiten)}, 1/137,036 \text{ (SI)}$$
	
	\item Gravitationskopplung:
	$$\alpha_G = \xi^2 = 1,78 \times 10^{-8}$$
	
	\item Schwache Kopplung:
	$$\alpha_W = \xi^{1/2} = 1,15 \times 10^{-2}$$
	
	\item Starke Kopplung:
	$$\alpha_S = \xi^{-1/3} = 9,65$$
\end{itemize}

	\subsection{Feinstrukturkonstante}
	\begin{itemize}
		\item Feinstrukturkonstante in SI-Einheiten:
		$$\frac{1}{137,036} = 1 \cdot \frac{\hbar c}{4\pi\varepsilon_0 e^2}$$
		
		\item Beziehung zum T0-Modell:
		$$\alpha_{\text{observed}} = \xi \cdot f_{\text{geometric}} = \frac{4}{3} \times 10^{-4} \cdot f_{\text{EM}}$$
		
		\item Berechnung des geometrischen Faktors:
		$$f_{\text{EM}} = \frac{\alpha_{\text{SI}}}{\xi} = \frac{7,297 \times 10^{-3}}{1,333 \times 10^{-4}} = 54,7$$
		
		\item Geometrische Interpretation:
		$$f_{\text{EM}} = \frac{4\pi^2}{3} \approx 13,16 \times 4,16 \approx 55$$
	\end{itemize}
	
	\subsection{Elektromagnetische Lagrange-Dichte}
	\begin{itemize}
		\item Elektromagnetische Lagrange-Dichte:
		$$\mathcal{L}_{\text{EM}} = -\frac{1}{4}F_{\mu\nu}F^{\mu\nu} + \bar{\psi}(i\gamma^\mu D_\mu - m)\psi$$
		
		\item Kovariante Ableitung:
		$$D_\mu = \partial_\mu + i \alpha_{\text{EM}} A_\mu = \partial_\mu + i A_\mu$$
		(Da $\alpha_{\text{EM}} = 1$ in nat\"{u}rlichen Einheiten)
	\end{itemize}
	
	\section{ANOMALES MAGNETISCHES MOMENT}

\subsection{Fundamentale T0-Formel}
\begin{itemize}
	\item Parameterfreie Vorhersage f\"{u}r das Myon-g-2:
	$$\boxed{a_\mu^{\text{T0}} = \frac{\xi}{2\pi} \left(\frac{E_\mu}{E_e}\right)^2}$$
	
	\item Universelle Leptonenformel:
	$$\boxed{a_\ell^{\text{T0}} = \frac{\xi}{2\pi} \left(\frac{E_\ell}{E_e}\right)^2}$$
\end{itemize}

\subsection{Berechnung f\"{u}r das Myon}
\begin{itemize}
	\item Energieverh\"{a}ltnis f\"{u}r das Myon:
	$$\frac{E_\mu}{E_e} = \frac{105,658 \text{ MeV}}{0,511 \text{ MeV}} = 206,768$$
	
	\item Berechnetes Energieverh\"{a}ltnis zum Quadrat:
	$$\left(\frac{E_\mu}{E_e}\right)^2 = (206,768)^2 = 42.753,2$$
	
	\item Geometrischer Faktor:
	$$\frac{\xi}{2\pi} = \frac{4/3 \times 10^{-4}}{2\pi} = \frac{1,3333 \times 10^{-4}}{6,2832} = 2,122 \times 10^{-5}$$
	
	\item Vollst\"{a}ndige Berechnung:
	$$a_\mu^{\text{T0}} = 2,122 \times 10^{-5} \times 42.753,2 = 9,071 \times 10^{-1}$$
	
	\item Vorhersage in experimentellen Einheiten:
	$$a_\mu^{\text{T0}} = 245(12) \times 10^{-11}$$
\end{itemize}

\subsection{Vorhersagen f\"{u}r andere Leptonen}
\begin{itemize}
	\item Tau-g-2 Vorhersage:
	$$a_\tau^{\text{T0}} = 257(13) \times 10^{-11}$$
	
	\item Elektron-g-2 Vorhersage:
	$$a_e^{\text{T0}} = 1,15 \times 10^{-19}$$
\end{itemize}

\subsection{Experimentelle Vergleiche}
\begin{itemize}
	\item T0-Vorhersage vs. Experiment f\"{u}r Myon-g-2:
	\begin{align*}
		a_\mu^{\text{T0}} &= 245(12) \times 10^{-11}\\
		a_\mu^{\text{exp}} &= 251(59) \times 10^{-11}\\
		\text{Abweichung} &= 0,10\sigma
	\end{align*}
	
	\item Standardmodell vs. Experiment:
	\begin{align*}
		a_\mu^{\text{SM}} &= 181(43) \times 10^{-11}\\
		\text{Abweichung} &= 4,2\sigma
	\end{align*}
	
	\item Statistische Analyse:
	$$\text{T0-Abweichung} = \frac{|a_\mu^{\text{exp}} - a_\mu^{\text{T0}}|}{\sigma_{\text{total}}} = \frac{|251 - 245| \times 10^{-11}}{\sqrt{59^2 + 12^2} \times 10^{-11}} = \frac{6 \times 10^{-11}}{60,2 \times 10^{-11}} = 0,10\sigma$$
\end{itemize}
\section{YUKAWA-KOPPLUNGSSTRUKTUR}

\subsection{Universelles Yukawa-Muster}
\begin{itemize}
	\item Allgemeine Massenformel:
	$$m_i = v \cdot y_i = 246 \text{ GeV} \cdot r_i \cdot \xi^{p_i}$$
	
	\item Vollständige Fermion-Struktur:
	\begin{align*}
		y_e &= \frac{4}{3}\xi^{3/2} = 2{,}04 \times 10^{-6} \quad \text{(Elektron)}\\
		y_\mu &= \frac{16}{5}\xi^1 = 4{,}25 \times 10^{-4} \quad \text{(Myon)}\\
		y_\tau &= \frac{5}{4}\xi^{2/3} = 7{,}31 \times 10^{-3} \quad \text{(Tau)}\\
		y_u &= 6\xi^{3/2} = 9{,}23 \times 10^{-6} \quad \text{(Up-Quark)}\\
		y_d &= \frac{25}{2}\xi^{3/2} = 1{,}92 \times 10^{-5} \quad \text{(Down-Quark)}\\
		y_s &= 3\xi^1 = 3{,}98 \times 10^{-4} \quad \text{(Strange-Quark)}\\
		y_c &= \frac{8}{9}\xi^{2/3} = 5{,}20 \times 10^{-3} \quad \text{(Charm-Quark)}\\
		y_b &= \frac{3}{2}\xi^{1/2} = 1{,}73 \times 10^{-2} \quad \text{(Bottom-Quark)}\\
		y_t &= \frac{1}{28}\xi^{-1/3} = 0{,}694 \quad \text{(Top-Quark)}
	\end{align*}
\end{itemize}

\subsection{Generationen-Hierarchie}
\begin{itemize}
	\item Erste Generation: Exponent $p = 3/2$
	\item Zweite Generation: Exponent $p = 1 \rightarrow 2/3$
	\item Dritte Generation: Exponent $p = 2/3 \rightarrow -1/3$
	
	\item Geometrische Interpretation:
	\begin{align*}
		\text{3D-Packung (Gen. 1)} &\rightarrow \xi^{3/2}\\
		\text{2D-Anordnungen (Gen. 2)} &\rightarrow \xi^1\\
		\text{1D-Strukturen (Gen. 3)} &\rightarrow \xi^{2/3}\\
		\text{Inverse Skalierung (Top)} &\rightarrow \xi^{-1/3}
	\end{align*}
\end{itemize}

\subsection{Experimentelle Validierung der Massen}
\begin{itemize}
	\item Durchschnittliche Abweichung: $< 0{,}5\%$
	\item Elektron: $0{,}0\%$ Abweichung
	\item Myon: $0{,}0\%$ Abweichung  
	\item Top-Quark: $1{,}2\%$ Abweichung
	\item Bemerkenswerte Präzision ohne freie Parameter
\end{itemize}
	\section{QUANTENMECHANIK IM T0-MODELL}
	
	\subsection{Vereinfachte Dirac-Gleichung}
	\begin{itemize}
		\item Die traditionelle Dirac-Gleichung enth\"{a}lt 4×4 Matrizen (64 komplexe Elemente):
		$$\left(i\gamma^\mu \partial_\mu - m\right) \psi = 0$$
		
		\item Modifizierte Dirac-Gleichung mit Zeitfeld-Kopplung:
		$$\boxed{\left[i\gamma^\mu\left(\partial_\mu + \Gamma_\mu^{(T)}\right) - E_{\text{char}}(x,t)\right]\psi = 0}$$
		
		\item Zeitfeld-Verbindung:
		$$\Gamma_\mu^{(T)} = \frac{1}{T_{\text{field}}} \partial_\mu T_{\text{field}} = -\frac{\partial_\mu E_{\text{field}}}{E_{\text{field}}^2}$$
		
		\item Radikale Vereinfachung zur universellen Feldgleichung:
		$$\boxed{\partial^2 \delta E = 0}$$
		
		\item Spinor-zu-Feld-Abbildung:
		$$\psi = \begin{pmatrix} \psi_1 \\ \psi_2 \\ \psi_3 \\ \psi_4 \end{pmatrix} \rightarrow E_{\text{field}} = \sum_{i=1}^4 c_i E_i(x,t)$$
		
		\item Informationskodierung im T0-Modell:
		\begin{align*}
			\text{Spin-Information} &\rightarrow \nabla \times E_{\text{field}}\\
			\text{Ladungs-Information} &\rightarrow \phi(\vec{r}, t)\\
			\text{Massen-Information} &\rightarrow E_0 \text{ und } r_0 = 2GE_0\\
			\text{Antiteilchen-Information} &\rightarrow \pm E_{\text{field}}
		\end{align*}
	\end{itemize}
	
	\subsection{Erweiterte Schr\"{o}dinger-Gleichung}
	\begin{itemize}
		\item Standardform der Schr\"{o}dinger-Gleichung:
		$$i\hbar \frac{\partial \psi}{\partial t} = \hat{H}\psi$$
		
		\item Erweiterte Schr\"{o}dinger-Gleichung mit Zeitfeld-Kopplung:
		$$\boxed{i\hbar \frac{\partial\psi}{\partial t} + i\psi\left[\frac{\partial T_{\text{field}}}{\partial t} + \vec{v} \cdot \nabla T_{\text{field}}\right] = \hat{H}\psi}$$
		
		\item Alternative Formulierung mit explizitem Zeitfeld:
		$$\boxed{i T_{\text{field}} \frac{\partial\Psi}{\partial t} + i\Psi\left[\frac{\partial T_{\text{field}}}{\partial t} + \vec{v} \cdot \nabla T_{\text{field}}\right] = \hat{H}\Psi}$$
		
		\item Deterministische L\"{o}sungsstruktur:
		$$\psi(x,t) = \psi_0(x) \exp\left(-\frac{i}{\hbar} \int_0^t \left[E_0 + V_{\text{eff}}(x,t')\right] dt'\right)$$
		
		\item Modifizierte Dispersionsrelationen:
		$$E^2 = p^2 + E_0^2 + \xi \cdot g(T_{\text{field}}(x,t))$$
		
		\item Wellenfunktion als Energiefeld-Darstellung:
		$$\psi(x,t) = \sqrt{\frac{\delta E(x,t)}{E_0 V_0}} \cdot e^{i\phi(x,t)}$$
	\end{itemize}
	
	\subsection{Deterministische Quantenphysik}
	\begin{itemize}
		\item Standard-QM vs. T0-Darstellung:
		
		Standard QM: $$|\psi\rangle = \sum_i c_i |i\rangle \quad \text{mit} \quad P_i = |c_i|^2$$
		
		T0 Deterministisch: $$\text{Zustand} \equiv \{E_i(x,t)\} \quad \text{mit Verh\"{a}ltnissen} \quad R_i = \frac{E_i}{\sum_j E_j}$$
		
		\item Messungs-Wechselwirkungshamiltonian:
		$$H_{\text{int}} = \frac{\xi}{E_P} \int \frac{E_{\text{system}}(x,t) \cdot E_{\text{detector}}(x,t)}{\ell_P^3} d^3x$$
		
		\item Messungsergebnis (deterministisch):
		$$\text{Messungsergebnis} = \arg\max_i\{E_i(x_{\text{detector}}, t_{\text{measurement}})\}$$
	\end{itemize}
	
	\subsection{Verschr\"{a}nkung und Bell-Ungleichungen}
	\begin{itemize}
		\item Verschr\"{a}nkung als Energiefeld-Korrelationen:
		$$E_{12}(x_1,x_2,t) = E_1(x_1,t) + E_2(x_2,t) + E_{\text{corr}}(x_1,x_2,t)$$
		
		\item Singlett-Zustand-Darstellung:
		$$|\psi^-\rangle = \frac{1}{\sqrt{2}}(|01\rangle - |10\rangle) \rightarrow \frac{1}{\sqrt{2}}[E_0(x_1)E_1(x_2) - E_1(x_1)E_0(x_2)]$$
		
		\item Feldkorrelationsfunktion:
		$$C(x_1,x_2) = \langle E(x_1,t) E(x_2,t) \rangle - \langle E(x_1,t) \rangle \langle E(x_2,t) \rangle$$
		
		\item Modifizierte Bell-Ungleichungen:
		$$|E(a,b) - E(a,c)| + |E(a',b) + E(a',c)| \leq 2 + \varepsilon_{T0}$$
		
		\item T0-Korrekturfaktor:
		$$\varepsilon_{T0} = \xi \cdot \frac{2G\langle E \rangle}{r_{12}} \approx 10^{-34}$$
	\end{itemize}
	
	\subsection{Quantengatter und Operationen}
	\begin{itemize}
		\item Pauli-X-Gatter (Bit-Flip):
		$$X: E_0(x,t) \leftrightarrow E_1(x,t)$$
		
		\item Pauli-Y-Gatter:
		$$Y: E_0 \rightarrow iE_1, \quad E_1 \rightarrow -iE_0$$
		
		\item Pauli-Z-Gatter (Phasen-Flip):
		$$Z: E_0 \rightarrow E_0, \quad E_1 \rightarrow -E_1$$
		
		\item Hadamard-Gatter:
		$$H: E_0(x,t) \rightarrow \frac{1}{\sqrt{2}}[E_0(x,t) + E_1(x,t)]$$
		
		\item CNOT-Gatter:
		$$\text{CNOT}: E_{12}(x_1,x_2,t) = E_1(x_1,t) \cdot f_{\text{control}}(E_2(x_2,t))$$
		
		Mit der Kontrollfunktion:
		$$f_{\text{control}}(E_2) = \begin{cases}
			E_2 & \text{wenn } E_1 = E_0 \\
			-E_2 & \text{wenn } E_1 = E_1
		\end{cases}$$
	\end{itemize}
	
	\subsection{Quantenalgorithmen}
	\begin{itemize}
		\item Quanten-Fourier-Transformation:
		$$\text{QFT}: E_j \rightarrow \frac{1}{\sqrt{N}} \sum_{k=0}^{N-1} E_k e^{2\pi i jk/N}$$
		
		\item Resonanzperiode-Detektion:
		$$E_{\text{resonance}}(t) = E_0 \cos\left(\frac{2\pi t}{r \cdot t_0}\right)$$
		
		\item Grover-Algorithmus Oracle-Operation:
		$$O: E_{\text{target}} \rightarrow -E_{\text{target}}, \quad E_{\text{others}} \rightarrow E_{\text{others}}$$
		
		\item Grover-Diffusionsoperation:
		$$D: E_i \rightarrow 2\langle E \rangle - E_i$$
		wobei $\langle E \rangle = \frac{1}{N}\sum_i E_i$ das durchschnittliche Energiefeld ist
		
		\item Amplitudenverst\"{a}rkung nach $k$ Iterationen:
		$$E_{\text{target}}^{(k)} = E_0 \sin\left((2k+1)\arcsin\sqrt{\frac{1}{N}}\right)$$
	\end{itemize}
	
	\section{KOSMOLOGIE IM T0-MODELL}
	
	\subsection{Statisches Universum}
\begin{itemize}
	\item Metrik im statischen Universum:
	$$ds^2 = -dt^2 + a^2(t)[dr^2 + r^2(d\theta^2 + \sin^2\theta d\phi^2)]$$
	Mit: $a(t) = \text{konstant}$ im T0-statischen Modell
	
	\item Teilchenhorizont im statischen Universum:
	$$r_H = \int_0^t c \, dt' = ct$$
\end{itemize}

	\subsection{Rotverschiebung und CMB}
	\begin{itemize}
		\item Rotverschiebungs-Formel mit Wellenl\"{a}ngenabh\"{a}ngigkeit:
		$$z(\lambda) = z_0\left(1 - \alpha \ln\frac{\lambda}{\lambda_0}\right)$$
		
		\item Erwartetes Signal f\"{u}r einen Quasar bei $z_0 = 2$:
		\begin{align*}
			z(\text{blau}) &= 2,0 \times (1 - 0,1 \times \ln(0,5)) = 2,0 \times (1 + 0,069) = 2,14\\
			z(\text{rot}) &= 2,0 \times (1 - 0,1 \times \ln(2,0)) = 2,0 \times (1 - 0,069) = 1,86
		\end{align*}
		
		\item Rotverschiebungsableitung nach Wellenl\"{a}nge:
		$$\frac{dz}{d\ln\lambda} = -\alpha z_0$$
		
		\item CMB-Frequenzabh\"{a}ngigkeit:
		$$\Delta z = \xi \ln\frac{\nu_1}{\nu_2}$$
		
		\item Vorhersage f\"{u}r Planck-Frequenzb\"{a}nder:
		$$\Delta z_{30-353} = \frac{4}{3} \times 10^{-4} \times \ln\frac{353}{30} = 1,33 \times 10^{-4} \times 2,46 = 3,3 \times 10^{-4}$$
		
		\item Modifizierte CMB-Temperatur-Entwicklung:
		$$\boxed{T(z) = T_0(1+z)\left(1 + \beta \ln(1+z)\right)}$$
	\end{itemize}
	
	\subsection{Energieverlustmechanismus f\"{u}r Photonen}
	\begin{itemize}
		\item Energieverlustrate f\"{u}r Photonen:
		$$\frac{dE_\gamma}{dr} = -g_T \omega^2 \frac{2G}{r^2}$$
		
		\item Korrigierte Energieverlustrate mit geometrischem Parameter:
		$$\boxed{\frac{dE_\gamma}{dr} = -\xi \frac{E_\gamma^2}{E_{\text{field}} \cdot r} = -\frac{4}{3} \times 10^{-4} \frac{E_\gamma^2}{E_{\text{field}} \cdot r}}$$
		
		\item Integrierte Energieverlustgleichung:
		$$\frac{1}{E_{\gamma,0}} - \frac{1}{E_\gamma(r)} = \xi \frac{\ln(r/r_0)}{E_{\text{field}}}$$
		
		\item Approximation f\"{u}r kleine Korrekturen ($\xi \ll 1$):
		$$E_\gamma(r) \approx E_{\gamma,0} \left(1 - \xi \frac{E_{\gamma,0}}{E_{\text{field}}} \ln\left(\frac{r}{r_0}\right)\right)$$
	\end{itemize}
	
	\subsection{Hubble-Parameter und Gravitationsdynamik}
	\begin{itemize}
		\item Rotverschiebungsdefinition:
		$$z = \frac{\lambda_{\text{observed}} - \lambda_{\text{emitted}}}{\lambda_{\text{emitted}}} = \frac{E_{\text{emitted}} - E_{\text{observed}}}{E_{\text{observed}}}$$
		
		\item Hubble-\"{a}hnliche Beziehung f\"{u}r kleine Rotverschiebungen:
		$$z \approx \frac{E_{\gamma,0} - E_\gamma(r)}{E_\gamma(r)} \approx \xi \frac{E_{\gamma,0}}{E_{\text{field}}} \ln\left(\frac{r}{r_0}\right)$$
		
		\item F\"{u}r nahe Entfernungen, wo $\ln(r/r_0) \approx r/r_0 - 1$:
		$$z \approx \xi \frac{E_{\gamma,0}}{E_{\text{field}}} \frac{r}{r_0} = H_0 \frac{r}{c}$$
		
		\item Effektiver Hubble-Parameter:
		$$H_0 = \xi \frac{E_{\gamma,0}}{E_{\text{field}}} \frac{c}{r_0}$$
		
		\item Modifizierte Galaxienrotationskurven:
		$$v(r) = \sqrt{\frac{GE_{\text{total}}}{r} + \Omega r^2}$$
		wobei $\Omega$ die Dimension $[E^3]$ hat
		
		\item Beobachtete "Hubble-Parameter" als Artefakte verschiedener Energieverlustmechanismen:
		$$H_0^{\text{apparent}}(z) = H_0^{\text{local}} \cdot f(z, \xi, E_{\text{field}}(z))$$
		
		\item Hubble-Spannung:
		$$\text{Tension} = \frac{|H_0^{\text{SH0ES}} - H_0^{\text{Planck}}|}{\sqrt{\sigma_{\text{SH0ES}}^2 + \sigma_{\text{Planck}}^2}} = \frac{5,6}{\sqrt{1,4^2 + 0,5^2}} = \frac{5,6}{1,49} = 3,8\sigma$$
	\end{itemize}
	
	\section{DIMENSIONSANALYSE UND EINHEITEN}
	
	\subsection{Dimensionen fundamentaler Gr\"{o}\ss{}en}
	\begin{itemize}
		\item Energie: $[E]$ (fundamental)
		\item Masse: $[M] = [E]$
		\item L\"{a}nge: $[L] = [E^{-1}]$
		\item Zeit: $[T] = [E^{-1}]$
		\item Impuls: $[p] = [E]$
		\item Kraft: $[F] = [E^2]$
		\item Ladung: $[q] = [1]$
		\item Wirkung: $[S] = [1]$
		\item Querschnitt: $[\sigma] = [E^{-2}]$
		\item Lagrange-Dichte: $[\mathcal{L}] = [E^4]$
		\item Energiedichte: $[\rho] = [E^4]$
		\item Wellenfunktion: $[\psi] = [E^{3/2}]$
		\item Feldst\"{a}rketensor: $[F_{\mu\nu}] = [E^2]$
		\item Beschleunigung: $[a] = [E^2]$
		\item Stromdichte: $[J^\mu] = [E^3]$
		\item D'Alembert-Operator: $[\square] = [E^2]$
		\item Ricci-Tensor: $[R_{\mu\nu}] = [E^2]$
	\end{itemize}
	
	\subsection{H\"{a}ufig verwendete Kombinationen}
	\begin{itemize}
		\item g-2 Vorfaktor: $\frac{\xi}{2\pi} = 2,122 \times 10^{-5}$
		\item Myon-Elektron-Verh\"{a}ltnis: $\frac{E_\mu}{E_e} = 206,768$
		\item Tau-Elektron-Verh\"{a}ltnis: $\frac{E_\tau}{E_e} = 3477,7$
		\item Gravitationskopplung: $\xi^2 = 1,78 \times 10^{-8}$
		\item Schwache Kopplung: $\xi^{1/2} = 1,15 \times 10^{-2}$
		\item Starke Kopplung: $\xi^{-1/3} = 9,65$
		\item Universelle T0-Skala: $2GE$
		\item Zeit-Energie-Dualit\"{a}t: $T_{\text{field}} \cdot E_{\text{field}} = 1$
	\end{itemize}
	
	\section{GRAVITATIONSEFFEKTE UND VEREINHEITLICHUNG}
	
	\subsection{Energieverlust von Photonen}
	\begin{itemize}
		\item Universelle Energieverlustrate:
		$$\boxed{\frac{dE_\gamma}{dr} = -\xi \frac{E_\gamma^2}{E_{\text{field}} \cdot r}}$$
		
		\item Wellenl\"{a}ngenformulierung:
		$$\frac{d\lambda}{dr} = \xi \frac{\lambda^2 \cdot E_{\text{field}}}{r}$$
		
		\item Integrierte Wellenl\"{a}ngengleichung:
		$$\int_{\lambda_0}^{\lambda(r)} \frac{d\lambda'}{\lambda'^2} = \xi E_{\text{field}} \int_0^r \frac{dr'}{r'}$$
		
		\item Wellenl\"{a}ngenbeziehung nach Integration:
		$$\frac{1}{\lambda_0} - \frac{1}{\lambda(r)} = \xi E_{\text{field}} \ln\left(\frac{r}{r_0}\right)$$
		
		\item Approximation f\"{u}r kleine Verschiebungen:
		$$\lambda(r) \approx \lambda_0 \left(1 + \xi E_{\text{field}} \lambda_0 \ln\left(\frac{r}{r_0}\right)\right)$$
		
		\item Alternativer Ausdruck mit urspr\"{u}nglicher Energieverlustform:
		$$\frac{dE_\gamma}{dr} = -g_T \omega^2 \frac{2G}{r^2}$$
	\end{itemize}
	
	\subsection{Wellenl\"{a}ngenabh\"{a}ngige Rotverschiebung}
\begin{itemize}
	\item Definition der Rotverschiebung:
	$$z = \frac{\lambda_{\text{observed}} - \lambda_{\text{emitted}}}{\lambda_{\text{emitted}}} = \frac{\lambda(r) - \lambda_0}{\lambda_0}$$
	
	\item Universelle Rotverschiebungsformel:
	$$\boxed{z(\lambda) = z_0\left(1 - \alpha \ln\frac{\lambda}{\lambda_0}\right)}$$
	
	\item Rotverschiebungsgradient:
	$$\frac{dz}{d\ln\lambda} = -\alpha z_0$$
	
	\item Beispiel f\"{u}r Rotverschiebungsvariationen bei einem Quasar mit $z_0 = 2$:
	\begin{align*}
		z(\text{blau}) &= 2,0 \times (1 - 0,1 \times \ln(0,5)) = 2,0 \times (1 + 0,069) = 2,14\\
		z(\text{rot}) &= 2,0 \times (1 - 0,1 \times \ln(2,0)) = 2,0 \times (1 - 0,069) = 1,86
	\end{align*}
	
	\item Beziehung zwischen Rotverschiebung und Energieverlust:
	$$z \approx \xi E_{\text{field}} \lambda_0 \ln\left(\frac{r}{r_0}\right) \approx \frac{E_{\gamma,0} - E_\gamma(r)}{E_\gamma(r)}$$
\end{itemize}

	\subsection{Energieabh\"{a}ngige Lichtablenkung}
	\begin{itemize}
		\item Modifizierte Ablenkungsformel:
		$$\boxed{\theta = \frac{4GM}{bc^2}\left(1 + \xi \frac{E_\gamma}{E_0}\right)}$$
		
		\item Verh\"{a}ltnis der Ablenkungswinkel f\"{u}r verschiedene Photonenenergien:
		$$\frac{\theta(E_1)}{\theta(E_2)} = \frac{1 + \xi \frac{E_1}{E_0}}{1 + \xi \frac{E_2}{E_0}}$$
		
		\item Approximation f\"{u}r $\xi \frac{E}{E_0} \ll 1$:
		$$\frac{\theta(E_1)}{\theta(E_2)} \approx 1 + \xi \frac{E_1 - E_2}{E_0}$$
		
		\item Modifizierter Einstein-Ring-Radius:
		$$\theta_E(\lambda) = \theta_{E,0} \sqrt{1 + \xi \frac{hc}{\lambda E_0}}$$
		
		\item Beispiel f\"{u}r R\"{o}ntgen (10 keV) und optische (2 eV) Photonen bei Sonnenablenkung:
		$$\frac{\theta_{\text{X-ray}}}{\theta_{\text{optical}}} \approx 1 + \frac{4}{3} \times 10^{-4} \cdot \frac{10^4 \text{ eV} - 2 \text{ eV}}{511 \times 10^3 \text{ eV}} \approx 1 + 2,6 \times 10^{-6}$$
	\end{itemize}
	
	\subsection{Universelle Geod\"{a}tengleichung}
	\begin{itemize}
		\item Vereinheitlichte Geod\"{a}tengleichung:
		$$\boxed{\frac{d^2 x^\mu}{d\lambda^2} + \Gamma^\mu_{\alpha\beta}\frac{dx^\alpha}{d\lambda}\frac{dx^\beta}{d\lambda} = \xi \cdot \partial^\mu \ln(E_{\text{field}})}$$
		
		\item Modifizierte Christoffel-Symbole:
		$$\Gamma^\lambda_{\mu\nu} = \Gamma^\lambda_{\mu\nu|0} + \frac{\xi}{2} \left(\delta^\lambda_\mu \partial_\nu T_{\text{field}} + \delta^\lambda_\nu \partial_\mu T_{\text{field}} - g_{\mu\nu} \partial^\lambda T_{\text{field}}\right)$$
		
		\item Korrelation zwischen Rotverschiebung und Lichtablenkung:
		$$\frac{\Delta z}{\Delta \theta} = \frac{\xi E_{\gamma,0}}{E_{\text{field}}} \cdot \frac{bc^2}{4GM} \cdot \frac{1}{\ln\left(\frac{r}{r_0}\right)} \cdot \frac{1}{\xi \frac{E_\gamma}{E_0}}$$
	\end{itemize}
	
	\subsection{Experimentelle Vorhersagen}
\begin{itemize}
	\item Wellenl\"{a}ngenabh\"{a}ngige Rotverschiebung f\"{u}r Quasare:
	$$z(450\text{ nm}) - z(700\text{ nm}) \approx 0,138 \times z_0$$
	
	\item Energieabh\"{a}ngige Lichtablenkung am Sonnenrand:
	$$\frac{\theta_{10\text{ keV}}}{\theta_{2\text{ eV}}} \approx 1 + 2,6 \times 10^{-6}$$
	
	\item CMB-Temperaturvariation mit Rotverschiebung:
	$$T(z) = T_0(1+z)\left(1 + \beta \ln(1+z)\right)$$
	
	\item CMB-Frequenzabh\"{a}ngigkeit:
	$$\Delta z = \xi \ln\frac{\nu_1}{\nu_2}$$
	
	\item Vorhersage f\"{u}r Planck-Frequenzb\"{a}nder:
	$$\Delta z_{30-353} = \frac{4}{3} \times 10^{-4} \times \ln\frac{353}{30} = 1,33 \times 10^{-4} \times 2,46 = 3,3 \times 10^{-4}$$
\end{itemize}

	\subsection{Einstein-Varianten der Masse-Energie-Beziehung}
	\begin{itemize}
		\item Die vier Einstein-Formen der Masse-Energie-Beziehung illustrieren die fundamentale \"{A}quivalenz:
		
		$$\text{Form 1 (Standard):} \quad \boxed{E = mc^2}$$
		
		$$\text{Form 2 (Variable Masse):} \quad \boxed{E = m(x,t) \cdot c^2}$$
		
		$$\text{Form 3 (Variable Lichtgeschwindigkeit):} \quad \boxed{E = m \cdot c^2(x,t)}$$
		
		$$\text{Form 4 (T0-Modell):} \quad \boxed{E = m(x,t) \cdot c^2(x,t)}$$
		
		\item Das T0-Modell verwendet die allgemeinste Darstellung mit der zeitfeldabh\"{a}ngigen Lichtgeschwindigkeit:
		$$c(x,t) = c_0 \cdot \frac{T_0}{T(x,t)}$$
		
		\item Experimentelle Ununterscheidbarkeit:
		\begin{itemize}
			\item Alle vier Formulierungen sind mathematisch konsistent und f\"{u}hren zu identischen experimentellen Vorhersagen
			\item Messger\"{a}te erfassen immer nur das Produkt aus effektiver Masse und effektiver Lichtgeschwindigkeit
			\item Nur die allgemeinste Form (Form 4) ist mit dem T0-Modell vollst\"{a}ndig kompatibel und beschreibt korrekt die Energiefeld-Wechselwirkungen
		\end{itemize}
		
		\item Zeit-Energie-Dualit\"{a}t im Kontext der Masse-Energie-\"{A}quivalenz:
		$$E = m(x,t) \cdot c^2(x,t) = m_0 \cdot c_0^2 \cdot \frac{T_0}{T(x,t)}$$
	\end{itemize}
	
	\section{$\xi$-HARMONISCHE THEORIE UND FAKTORISIERUNG}
	
	\subsection{$\xi$-Parameter als Unsch\"{a}rfe-Parameter}
	\begin{itemize}
		\item Heisenbergsche Unsch\"{a}rferelation:
		$$\Delta\omega \times \Delta t \geq \xi/2$$
		
		\item $\xi$ als Resonanz-Fenster:
		$$\text{Resonance}(\omega, \omega_{\text{target}}, \xi) = \exp\left(-\frac{(\omega-\omega_{\text{target}})^2}{4\xi}\right)$$
		
		\item Optimaler Parameter:
		$$\xi = 1/10 \text{ (f\"{u}r mittlere Selektivit\"{a}t)}$$
		
		\item Akzeptanz-Radius:
		$$r_{\text{accept}} = \sqrt{4\xi} \approx 0,63 \text{ (f\"{u}r } \xi = 1/10)$$
	\end{itemize}
	
	\subsection{Spektrale Dirac-Darstellung}
	\begin{itemize}
		\item Dirac-Darstellung einer Zahl $n = p \times q$:
		$$\delta_n(f) = A_1\delta(f - f_1) + A_2\delta(f - f_2)$$
		
		\item $\xi$-verbreiterte Dirac-Funktion:
		$$\delta_\xi(\omega - \omega_0) = \frac{1}{\sqrt{4\pi\xi}} \times \exp\left(-\frac{(\omega-\omega_0)^2}{4\xi}\right)$$
		
		\item Vollst\"{a}ndige Dirac-Zahlen-Funktion:
		$$\Psi_n(\omega,\xi) = \sum_i A_i \times \frac{1}{\sqrt{4\pi\xi}} \times \exp\left(-\frac{(\omega-\omega_i)^2}{4\xi}\right)$$
	\end{itemize}
	
	\subsection{Faktorisierung durch FFT-Spektraltheorie}
	\begin{itemize}
		\item Grundfrequenzen im Spektrum entsprechen Primfaktoren:
		$$n = p \times q \rightarrow \{f_1 = f_0 \times p, f_2 = f_0 \times q\}$$
		
		\item Spektrales Verh\"{a}ltnis (muss immer als Verh\"{a}ltnis betrachtet werden):
		$$R(n) = \frac{q}{p} = \frac{\max(p,q)}{\min(p,q)}$$
		
		\item Oktaven-Reduktion zur Vermeidung von Rundungsfehlern:
		$$R_{\text{oct}}(n) = \frac{R(n)}{2^{\lfloor\log_2(R(n))\rfloor}}$$
		
		\item Beatfrequenz (Differenzfrequenz):
		$$f_{\text{beat}} = |f_2 - f_1| = f_0 \times |q - p|$$
	\end{itemize}
	
	\subsection{Harmonische Hierarchie f\"{u}r Faktorisierungen}
	\begin{itemize}
		\item Basis (1,0 - 1,4): Klassische Harmonien
		$$\text{z.B. } \frac{3}{2} = 1,5 \text{ (Quinte), } \frac{5}{4} = 1,25 \text{ (Gro\ss{}e Terz)}$$
		
		\item Erweitert (1,4 - 1,6): Jazz/moderne Harmonien
		$$\text{z.B. } \frac{11}{8} = 1,375, \frac{13}{8} = 1,625$$
		
		\item Komplex (1,6 - 1,85): Mikrotonale Spektren
		$$\text{z.B. } \frac{29}{16} = 1,8125, \frac{31}{16} = 1,9375$$
		
		\item Ultra (1,85+): Xenharmonische Spektren
		$$\text{z.B. } \frac{61}{32} = 1,90625, \frac{37}{32} = 1,15625$$
	\end{itemize}
	
	\subsection{Resonanz-Score f\"{u}r Faktorisierungen}
	\begin{itemize}
		\item Optimaler Resonanzparameter:
		$$\xi = \frac{1}{10}$$
		
		\item Kreisfrequenz f\"{u}r Periode $r$:
		$$\omega = \frac{2\pi}{r}$$
		
		\item Resonanz-Score:
		$$\text{Res}(r,\xi) = \frac{1}{1 + \frac{|(\omega-\pi)^2|}{4\xi}}$$
	\end{itemize}
	
	\subsection{Verh\"{a}ltnisbasierte Berechnung zur Vermeidung von Rundungsfehlern}
	\begin{itemize}
		\item Statt absoluter Werte sollten Verh\"{a}ltnisse verwendet werden:
		$$\frac{f_1}{f_0} = p, \quad \frac{f_2}{f_0} = q, \quad \frac{f_2}{f_1} = \frac{q}{p}$$
		
		\item Harmonische Distanz (in Cent):
		$$d_{\text{harm}}(n,h) = 1200 \times \left|\log_2\left(\frac{R_{\text{oct}}(n)}{h}\right)\right|$$
		
		\item \"{U}bereinstimmungskriterium:
		$$\text{Match}(n, \text{harmonic\_ratio}) = \text{TRUE wenn } |R_{\text{oct}}(n) - \text{harmonic\_ratio}|^2 < 4\xi$$
	\end{itemize}
	
	\section{FORMELZEICHENERKL\"{A}RUNGEN}
	
	\subsection{Allgemeine Symbole}
	\begin{itemize}
		\item $\xi$ = Universeller geometrischer Parameter (4/3 × 10$^{-4}$)
		\item $G$ = Gravitationskonstante
		\item $c$ = Lichtgeschwindigkeit
		\item $\hbar$ = Reduziertes Plancksches Wirkungsquantum
		\item $k_B$ = Boltzmann-Konstante
		\item $E_P$ = Planck-Energie
		\item $\ell_P$ = Planck-L\"{a}nge
		\item $T_0$ = Referenz-Zeitfeldwert
		\item $E_0$ = Referenz-Energiefeldwert
	\end{itemize}
	
	\subsection{Feldtheorie-Symbole}
	\begin{itemize}
		\item $E_{\text{field}}$ = Energiefeld
		\item $T_{\text{field}}$ = Zeitfeld
		\item $\delta E$ = Energiefeldfluktuation
		\item $\mathcal{L}$ = Lagrange-Dichte
		\item $\square$ = D'Alembert-Operator
		\item $\Gamma_\mu^{(T)}$ = Zeitfeld-Verbindung
		\item $\nabla$ = Nabla-Operator
		\item $\partial_\mu$ = Partielle Ableitung nach Koordinate $\mu$
	\end{itemize}
	
	\subsection{Quantenmechanische Symbole}
	\begin{itemize}
		\item $\psi$ = Wellenfunktion
		\item $\gamma^\mu$ = Dirac-Matrizen
		\item $\hat{H}$ = Hamilton-Operator
		\item $|\psi\rangle$ = Zustandsvektor
		\item $\langle A \rangle$ = Erwartungswert der Observable $A$
		\item $a_\mu$ = Anomales magnetisches Moment des Myons
		\item $a_\ell$ = Anomales magnetisches Moment eines Leptons
	\end{itemize}
	
	\subsection{Teilchenphysik-Symbole}
	\begin{itemize}
		\item $\alpha_{\text{EM}}$ = Elektromagnetische Kopplungskonstante
		\item $\alpha_G$ = Gravitationskopplung
		\item $\alpha_W$ = Schwache Kopplung
		\item $\alpha_S$ = Starke Kopplung
		\item $E_\mu$ = Myon-Energie/Masse
		\item $E_e$ = Elektron-Energie/Masse
		\item $E_\tau$ = Tau-Energie/Masse
	\end{itemize}
	
	\subsection{Kosmologische Symbole}
	\begin{itemize}
		\item $z$ = Rotverschiebung
		\item $\lambda$ = Wellenl\"{a}nge
		\item $\nu$ = Frequenz
		\item $H_0$ = Hubble-Parameter
		\item $\theta$ = Ablenkungswinkel
		\item $ds^2$ = Linienelement
		\item $a(t)$ = Skalenfaktor
	\end{itemize}
	
	\subsection{Spektralanalyse und Faktorisierung}
	\begin{itemize}
		\item $R(n)$ = Spektrales Verh\"{a}ltnis einer Zahl $n$
		\item $R_{\text{oct}}(n)$ = Oktavenreduziertes spektrales Verh\"{a}ltnis
		\item $f_{\text{beat}}$ = Beatfrequenz
		\item $\delta_\xi$ = $\xi$-verbreiterte Dirac-Funktion
		\item $\Psi_n$ = Spektrale Wellenfunktion einer Zahl
		\item $\omega$ = Kreisfrequenz
		\item $d_{\text{harm}}$ = Harmonische Distanz
	\end{itemize}
	
\end{document}