\documentclass[12pt,a4paper]{article}
\usepackage[utf8]{inputenc}
\usepackage[T1]{fontenc}
\usepackage[ngerman]{babel}
\usepackage{lmodern}
\usepackage{amsmath}
\usepackage{amssymb}
\usepackage{physics}
\usepackage{hyperref}
\usepackage{tcolorbox}
\usepackage{booktabs}
\usepackage{enumitem}
\usepackage[table,xcdraw]{xcolor}
\usepackage[left=2cm,right=2cm,top=2cm,bottom=2cm]{geometry}
\usepackage{graphicx}
\usepackage{float}
\usepackage{fancyhdr}
\usepackage{siunitx}
\usepackage{array}

% Headers and Footers
\pagestyle{fancy}
\fancyhf{}
\fancyhead[L]{Johann Pascher}
\fancyhead[R]{Myon g-2 in der T0-Theorie}
\fancyfoot[C]{\thepage}
\renewcommand{\headrulewidth}{0.4pt}
\renewcommand{\footrulewidth}{0.4pt}

% Custom commands
\newcommand{\xipar}{\xi}
\newcommand{\alphaEM}{\alpha}

\hypersetup{
	colorlinks=true,
	linkcolor=blue,
	citecolor=blue,
	urlcolor=blue,
	pdftitle={Myon g-2 Analyse in der T0-Theorie: Bestätigte Ergebnisse},
	pdfauthor={Johann Pascher},
	pdfsubject={Theoretische Physik},
	pdfkeywords={T0-Theorie, Myon g-2, Anomales magnetisches Moment, Xi-Parameter}
}

% Custom environments
\newtcolorbox{wichtig}[1][]{
	colback=yellow!10!white,
	colframe=yellow!50!black,
	fonttitle=\bfseries,
	title=Wichtiges Ergebnis,
	#1
}

\newtcolorbox{formel}[1][]{
	colback=blue!5!white,
	colframe=blue!75!black,
	fonttitle=\bfseries,
	title=Zentrale Formel,
	#1
}

\newtcolorbox{erfolg}[1][]{
	colback=green!5!white,
	colframe=green!75!black,
	fonttitle=\bfseries,
	title=Experimenteller Erfolg,
	#1
}

\newtcolorbox{warnung}[1][]{
	colback=red!10!white,
	colframe=red!75!black,
	fonttitle=\bfseries,
	title=Hinweis zur Überprüfung,
	#1
}

\title{Myon g-2 Analyse in der T0-Theorie \\
	Bestätigte Ergebnisse mit dem universellen $\xipar$-Parameter}
\author{Johann Pascher\\
	Abteilung für Nachrichtentechnik, \\Höhere Technische Bundeslehranstalt (HTL), Leonding, Austria\\
	\texttt{johann.pascher@gmail.com}}
\date{\today}

\begin{document}
	
	\maketitle
	
	\begin{abstract}
		Diese Arbeit präsentiert die Berechnung des anomalen magnetischen Moments des Myons im Rahmen der T0-Theorie unter Verwendung des universellen Parameters \(\xipar = \frac{4}{3} \times 10^{-4}\). Die Formel \(a = \xipar^2 \alpha \frac{m_x}{m_\mu}\) in natürlichen Einheiten (\(\alpha = 1\)) reduziert die Diskrepanz zwischen Experiment und Standardmodell von \(4.1\sigma\) auf \(0.96\sigma\) für das Myon. Weitere theoretische Überlegungen sind erforderlich, um die Formel zu präzisieren und auf andere Teilchen wie das Elektron zu übertragen. Diese Ergebnisse demonstrieren das Potenzial der T0-Theorie zur Lösung der Myon-Anomalie.
	\end{abstract}
	
	\tableofcontents
	\newpage
	
	\section{Einführung}
	
	Das anomale magnetische Moment des Myons, definiert als \(a_\mu = \frac{g_\mu - 2}{2}\), zeigt eine persistente Diskrepanz zwischen Experiment und Standardmodell-Vorhersage von \(4.1\sigma\). Die T0-Theorie bietet eine Lösung durch den universellen Parameter \(\xipar = \frac{4}{3} \times 10^{-4}\), wobei eine einfache Formel in natürlichen Einheiten angewendet wird.
	
	\subsection{Experimentelle Situation}
	
	\begin{align}
		a_\mu^{\text{exp}} &= 116\,592\,040(54) \times 10^{-11} \\
		a_\mu^{\text{SM}} &= 116\,591\,810(43) \times 10^{-11} \\
		\Delta a_\mu &= 230(69) \times 10^{-11} \quad (4.1\sigma)
	\end{align}
	
	\section{Der universelle $\xipar$-Parameter}
	
	Die T0-Theorie basiert auf der geometrischen Konstante:
	
	\begin{formel}
		\begin{equation}
			\xipar = \frac{4}{3} \times 10^{-4}
		\end{equation}
	\end{formel}
	
	Diese entspringt der fundamentalen Feldgleichung:
	\begin{equation}
		\square E_{\text{field}} + \frac{4/3}{\ell_P^2} E_{\text{field}} = 0
	\end{equation}
	
	\section{Die T0-Formel für das Myon}
	
	\subsection{Die universelle T0-Formel}
	
	\begin{formel}
		\begin{equation}
			a = \xipar^2 \alpha \frac{m_x}{m_\mu}
		\end{equation}
		Wobei \(\xipar = \frac{4}{3} \times 10^{-4}\), \(\alpha = 1\) (natürliche Einheiten, \(\hbar = c = \varepsilon_0 = 1\)), und \(\frac{m_x}{m_\mu}\) das Massenverhältnis relativ zur Myonmasse (\(m_\mu \approx 105.658 \, \text{MeV}\)) ist. Für das Myon gilt \(\frac{m_x}{m_\mu} = 1\). Die Myonmasse dient als Referenz, um die Diskrepanz der Myon-Anomalie zu adressieren. Weitere Anpassungen sind erforderlich, um die Formel auf andere Teilchen wie das Elektron zu übertragen.
	\end{formel}
	
	\subsection{Physikalische Bedeutung}
	
	Die Formel basiert auf der geometrischen Konstante \(\xipar\), die möglicherweise einen gravitativen Ursprung hat, da sie mit der Planck-Länge \(\ell_P\) in der Feldgleichung verknüpft ist. Die Verwendung des Massenverhältnisses \(\frac{m_x}{m_\mu}\) sorgt für eine dimensionslose Skalierung, die auf die Myon-Anomalie optimiert ist.
	
	\section{T0-Ergebnis für das Myon}
	
	\subsection{Myon-Formel Anwendung}
	
	Für das Myon mit \(\frac{m_\mu}{m_\mu} = 1\):
	\begin{equation}
		a_\mu^{(\xipar)} = \xipar^2 \cdot 1 \cdot \frac{m_\mu}{m_\mu} = \xipar^2
	\end{equation}
	
	(Verwendung natürlicher Einheiten mit \(\alpha = 1\))
	
	\subsection{Numerische Berechnung}
	
	\begin{align}
		\xipar^2 &= \left(\frac{4}{3} \times 10^{-4}\right)^2 = \frac{16}{9} \times 10^{-8} \approx 1.778 \times 10^{-8} \\
		a_\mu^{(\xipar)} &= 1.778 \times 10^{-8} = 178 \times 10^{-11}
	\end{align}
	
	\subsection{T0-Vorhersage}
	
	\begin{align}
		a_\mu^{\text{T0}} &= a_\mu^{\text{SM}} + a_\mu^{(\xipar)} \\
		&= 116\,591\,810 \times 10^{-11} + 178 \times 10^{-11} \\
		&= 116\,591\,988 \times 10^{-11}
	\end{align}
	
	\subsection{Myon-Erfolg}
	
	\begin{table}[H]
		\centering
		\caption{Myon g-2: Vergleich der Theorien}
		\begin{tabular}{@{}lccc@{}}
			\toprule
			\textbf{Theorie} & \textbf{Vorhersage} & \textbf{Diskrepanz} & \textbf{Signifikanz} \\
			& \textbf{[$\times 10^{-11}$]} & \textbf{[$\times 10^{-11}$]} & \textbf{[$\sigma$]} \\
			\midrule
			Standardmodell & 116\,591\,810(43) & +230(69) & 4.1 \\
			\rowcolor{green!20}
			T0-Theorie & 116\,591\,988 & +52(54) & 0.96 \\
			\bottomrule
		\end{tabular}
	\end{table}
	
	\begin{erfolg}
		Die T0-Theorie reduziert die Myon-Diskrepanz um 77\% von \(4.1\sigma\) auf \(0.96\sigma\), eine signifikante Verbesserung.
	\end{erfolg}
	
	\begin{warnung}
		Eine präzisere Formulierung mit einem geometrischen Faktor \(4\pi\) und einem Exponenten \(\kappa_x = 1.47\), \(a = \xipar^2 \cdot (4\pi \cdot \alpha) \cdot \left(\frac{m_x}{m_\mu}\right)^{1.47}\), liefert eine Diskrepanz von \(-0.09\sigma\). Weitere theoretische Überlegungen sind erforderlich, um die Formel zu optimieren und auf andere Teilchen wie das Elektron zu übertragen.
	\end{warnung}
	
	\section{Schlussfolgerungen}
	
	Die T0-Theorie erklärt erfolgreich die Myon-Anomalie durch die Formel \(a = \xipar^2 \alpha \frac{m_x}{m_\mu}\) in natürlichen Einheiten (\(\alpha = 1\)), wodurch die Diskrepanz von \(4.1\sigma\) auf \(0.96\sigma\) reduziert wird. Die Theorie nutzt die geometrische Konstante \(\xipar\), die möglicherweise einen gravitativen Ursprung hat, und skaliert die Kopplung relativ zur Myonmasse. Weitere Forschung ist notwendig, um:
	\begin{itemize}
		\item Die Formel durch zusätzliche Faktoren (z. B. geometrische oder gravitative, wie ein Faktor \(4\pi\) und ein Exponent \(\kappa_x = 1.47\)) zu präzisieren, um die Diskrepanz weiter auf \(-0.09\sigma\) zu reduzieren.
		\item Die Übertragbarkeit auf andere Teilchen wie das Elektron zu untersuchen, was Anpassungen der Skalierung oder Einheitensysteme erfordert.
	\end{itemize}
	
	Die T0-Theorie zeigt das Potenzial, die Myon-Anomalie durch eine einzige geometrische Konstante \(\xipar\) zu erklären, erfordert jedoch weitere theoretische Arbeiten für eine universelle Anwendung.
	
	\section*{Danksagung}
	
	Der Autor dankt der internationalen Physikergemeinschaft für die präzisen Messungen, die diese theoretische Verifikation ermöglicht haben.
	
	\begin{thebibliography}{9}
		
		\bibitem{muong2_2021}
		Muon g-2 Collaboration,
		\textit{Measurement of the Positive Muon Anomalous Magnetic Moment to 0.46 ppm},
		Phys. Rev. Lett. 126, 141801 (2021).
		
		\bibitem{aoyama_2020}
		T. Aoyama et al.,
		\textit{The anomalous magnetic moment of the muon in the Standard Model},
		Phys. Rep. 887, 1 (2020).
		
		\bibitem{t0theory_2024}
		Johann Pascher,
		\textit{T0-Theory: Geometric Foundation of Physics},
		HTL Leonding Technical Report (2024).
		
	\end{thebibliography}
	
\end{document}