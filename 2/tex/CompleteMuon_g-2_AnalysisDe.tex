\documentclass[12pt,a4paper]{article}
\usepackage[utf8]{inputenc}
\usepackage[T1]{fontenc}
\usepackage[german]{babel}
\usepackage{lmodern}
\usepackage{amsmath}
\usepackage{amssymb}
\usepackage{physics}
\usepackage{hyperref}
\usepackage{tcolorbox}
\usepackage{booktabs}
\usepackage{enumitem}
\usepackage[table,xcdraw]{xcolor}
\usepackage[left=2cm,right=2cm,top=2cm,bottom=2cm]{geometry}
\usepackage{pgfplots}
\pgfplotsset{compat=1.18}
\usepackage{graphicx}
\usepackage{float}
\usepackage{fancyhdr}
\usepackage{siunitx}
\usepackage{mathtools}
\usepackage{amsthm}
\usepackage{cleveref}
\usepackage{tocloft}
\usepackage{tikz}
\usepackage[dvipsnames]{xcolor}
\usetikzlibrary{positioning, shapes.geometric, arrows.meta}
\usepackage{microtype}
\usepackage{array}
\usepackage{longtable}

% Custom Commands - alle Sonderzeichen kodiert
\newcommand{\Efield}{E_{\text{field}}}
\newcommand{\xigeom}{\xi_{\text{geom}}}
\newcommand{\xipar}{\xi}
\newcommand{\Tzero}{T_0}
\newcommand{\vecx}{\vec{x}}
\newcommand{\alphagem}{\alpha_{\text{EM}}}
%\newcommand{\aleph}{\mathfrak{A}}
\newcommand{\lambdaH}{\lambda_{\text{H}}}
\newcommand{\ellPlanck}{\ell_{\text{Planck}}}
%\newcommand{\hbar}{\hslash}
\newcommand{\rzero}{r_0}
\newcommand{\nulep}{\nu}
\newcommand{\epsilonlep}{\varepsilon}
\newcommand{\chisquared}{\chi^2}
\newcommand{\sigmadev}{\sigma}

% Header and Footer Configuration
\pagestyle{fancy}
\fancyhf{}
\fancyhead[L]{Johann Pascher}
\fancyhead[R]{T0-Modell: Geometrische Herleitung der Leptonischen Anomalien}
\fancyfoot[C]{\thepage}
\renewcommand{\headrulewidth}{0.4pt}
\renewcommand{\footrulewidth}{0.4pt}

% Table of Contents Formatting
\renewcommand{\cftsecfont}{\color{blue}}
\renewcommand{\cftsubsecfont}{\color{blue}}
\renewcommand{\cftsecpagefont}{\color{blue}}
\renewcommand{\cftsubsecpagefont}{\color{blue}}

\hypersetup{
	colorlinks=true,
	linkcolor=blue,
	citecolor=blue,
	urlcolor=blue,
	pdftitle={T0-Theorie: Geometrische Herleitung der Leptonischen Anomalien},
	pdfauthor={Johann Pascher},
	pdfsubject={T0-Modell, Geometrische Resonanz, Leptonische Anomalien},
	pdfkeywords={Energiefeld, Geometrische Resonanzen, Parameterfreie Theorie, Myon g-2}
}

% Theorem Environments
\newtheorem{theorem}{Theorem}[section]
\newtheorem{proposition}[theorem]{Proposition}
\newtheorem{definition}[theorem]{Definition}
\newtheorem{lemma}[theorem]{Lemma}

\tcbuselibrary{theorems}
\newtcbtheorem[number within=section]{important}{Wichtiger Hinweis}%
{colback=green!5,colframe=green!35!black,fonttitle=\bfseries}{th}

\newtcbtheorem[number within=section]{warning}{Warnung}%
{colback=red!5,colframe=red!75!black,fonttitle=\bfseries}{warn}

\newtcbtheorem[number within=section]{keyresult}{Schl\"usselresultat}%
{colback=blue!5,colframe=blue!75!black,fonttitle=\bfseries}{key}

\begin{document}
	
	\title{T0-Theorie: Geometrische Herleitung der Leptonischen Anomalien \\
		\large Vollst\"andig parameterfreie Vorhersage mit empirischen Teilchenmassen}
	\author{Johann Pascher\\
		Abteilung f\"ur Kommunikationstechnik\\
		H\"ohere Technische Bundeslehranstalt (HTL), Leonding, \"Osterreich\\
		\texttt{johann.pascher@gmail.com}}
	\date{\today}
	
	\maketitle
	
	\begin{abstract}
		Die T0-Raumzeit-Geometrie-Theorie liefert eine vollst\"andig parameterfreie Vorhersage der anomalen magnetischen Momente aller geladenen Leptonen. Ausgehend von der fundamentalen T0-Feldgleichung werden alle Parameter geometrisch abgeleitet ohne empirische Anpassung.
	\end{abstract}
	
	\tableofcontents
	\newpage
	
	\section{Fundamentale Geometrische Ableitung}
	
	\subsection{T0-Feldgleichung und charakteristische L\"ange}
	
	\textbf{Ausgangspunkt}: Die fundamentale T0-Feldgleichung f\"ur das dynamische Massenfeld
	\begin{equation}
		\nabla^2 m(r) = 4\pi G \rho(r) \cdot m(r)
	\end{equation}
	
	\textbf{Charakteristische T0-L\"ange in nat\"urlichen Einheiten}:
	\begin{equation}
		\rzero = \frac{\lambdaH^2 \times v^2}{16\pi^3 \times m_{\text{H}}^2} \times \ellPlanck
	\end{equation}
	
	\textbf{Higgs-Parameter} (experimentell bestimmt):
	\begin{itemize}
		\item $\lambdaH \approx 0{,}13$ (Higgs-Selbstkopplung)
		\item $v \approx 246\,\text{GeV}$ (Higgs-VEV)
		\item $m_{\text{H}} \approx 125\,\text{GeV}$ (Higgs-Masse)
	\end{itemize}
	
	\textbf{Berechnung in Planck-Einheiten}:
	\begin{align}
		\frac{\rzero}{\ellPlanck} &= \frac{(0{,}13)^2 \times (246/125)^2}{16\pi^3} \\
		&= \frac{0{,}0169 \times 3{,}87}{493{,}48} \\
		&= 1{,}33 \times 10^{-4}
	\end{align}
	
	\textbf{Physikalische Bedeutung}: $\rzero$ ist \textbf{nicht} der Schwarzschild-Radius einer Teilchenmasse, sondern die \textbf{charakteristische L\"ange des Higgs-Feldes in der T0-Geometrie}.
	
	\subsection{Geometrischer $\xipar$-Parameter}
	
	\textbf{Sph\"arische Geometrie-Korrektur}:
	\begin{equation}
		\xipar = \frac{4}{3} \times \frac{\rzero}{\ellPlanck} = \frac{4}{3} \times 1{,}33 \times 10^{-4} = 1{,}77 \times 10^{-4}
	\end{equation}
	
	\textbf{Geometrischer Ursprung}:
	\begin{itemize}
		\item \textbf{4/3}: Kugelvolumen-Faktor aus sph\"arischer T0-Symmetrie
		\item \textbf{$1{,}33 \times 10^{-4}$}: Aus T0-Feldgleichung mit Gau\ss{}schem Satz abgeleitet
	\end{itemize}
	
	\section{Elektromagnetische Kopplungskonstante $\aleph$}
	
	\subsection{Definition der T0-Kopplungskonstante $\aleph$ (Aleph)}
	
	\textbf{T0-spezifische elektromagnetische Kopplung - vollst\"andig $\xipar$-basiert}:
	\begin{equation}
		\aleph = \xipar \times 13\pi \times \frac{7\pi}{2} = \xipar \times 449{,}1
	\end{equation}
	
	\textbf{Ersetzt die Feinstrukturkonstante}:
	\begin{equation}
		\alphagem = \xipar \times 13\pi \quad \text{(geometrische Ableitung statt empirischer Wert 1/137)}
	\end{equation}
	
	\textbf{Numerischer Wert}:
	\begin{equation}
		\aleph = 1{,}77 \times 10^{-4} \times 449{,}1 = 0{,}07949
	\end{equation}
	
	\subsection{Geometrische Herleitung der Faktoren}
	
	\textbf{Ursprung der kombinierten Faktoren $13\pi \times (7\pi/2)$}:
	
	\textbf{$13\pi$-Faktor}:
	\begin{itemize}
		\item \textbf{13}: M\"ogliche 13-dimensionale Kompaktifizierung der T0-Geometrie
		\item \textbf{$\pi$}: Fundamentaler geometrischer Faktor aus sph\"arischer Symmetrie
	\end{itemize}
	
	\textbf{$7\pi/2$-Faktor}:
	\begin{itemize}
		\item \textbf{7}: Effektive Dimensionen der T0-Feldstruktur
		\item \textbf{$\pi/2$}: Viertelkreis, fundamentaler geometrischer Winkel
	\end{itemize}
	
	\textbf{Kombinierter Faktor}: $91\pi^2/2 \approx 449{,}1$
	
	\textbf{Physikalische Interpretation}:
	\begin{itemize}
		\item \textbf{Vollst\"andige Elimination der Feinstrukturkonstante} als separater Parameter
		\item \textbf{Ein-Parameter-Theorie}: Alle elektromagnetischen Ph\"anomene aus $\xipar$ ableitbar
		\item \textbf{Reine Geometrie}: Keine empirischen Kopplungskonstanten erforderlich
	\end{itemize}
	
	\section{Universelle T0-Formel f\"ur Leptonische Anomalien}
	
	\subsection{Allgemeine Struktur}
	
	\textbf{Universelle T0-Relation}:
	\begin{equation}
		a_\ell = \epsilonlep(\ell) \times \xipar^2 \times \aleph \times \left(\frac{m_\ell}{m_\mu}\right)^\nulep
	\end{equation}
	
	\textbf{Parameter-Definition}:
	\begin{itemize}
		\item \textbf{$\epsilonlep(\ell)$}: Teilchen-spezifisches Vorzeichen (+1 f\"ur Myon, -1 f\"ur Elektron)
		\item \textbf{$\xipar$}: Geometrischer T0-Parameter $(1{,}77 \times 10^{-4})$
		\item \textbf{$\aleph$}: T0-Kopplungskonstante $(0{,}08026)$
		\item \textbf{$\nulep$}: QFT-Korrekturexponent $\nulep = 3/2 - \delta = 1{,}5 - 0{,}014 = 1{,}486$
	\end{itemize}
	
	\textbf{Theoretische Herleitung von $\nulep$}:
	
	\textbf{Grundlage}: Aus der fraktalen Renormierungsgruppen-Analyse ergibt sich:
	\begin{equation}
		\nulep = \frac{D_f}{2} = \frac{2{,}94}{2} = 1{,}47 \approx \frac{3}{2}
	\end{equation}
	
	\textbf{Komponenten}:
	\begin{itemize}
		\item \textbf{3/2}: Quantenmechanische Zustandsdichte in 3D ($\rho \propto m^{3/2}$)
		\item \textbf{$D_f = 2{,}94$}: Fraktale Dimension der T0-Raumzeit-Struktur
		\item \textbf{$\delta = 0{,}014$}: Logarithmische RG-Korrektur aus Schleifenintegralen
	\end{itemize}
	
	\textbf{Physikalische Bedeutung}:
	\begin{itemize}
		\item \textbf{Basis 3/2}: Fermi-Gas-Zustandsdichte, relativistische Korrekturen
		\item \textbf{Kleine Abweichung}: Renormierungsgruppen-Laufen der Kopplungen
		\item \textbf{Universell}: Gilt f\"ur alle geladenen Leptonen in der T0-Geometrie
		\item \textbf{$m_\mu$}: Myon-Referenzmasse
	\end{itemize}
	
	\subsection{Teilchen-spezifische Formeln}
	
	\textbf{Myon (Referenzteilchen)}:
	\begin{equation}
		a_\mu = (+1) \times \xipar^2 \times \aleph \times \left(\frac{m_\mu}{m_\mu}\right)^\nulep = \xipar^2 \times \aleph
	\end{equation}
	
	\textbf{Elektron}:
	\begin{equation}
		a_e = (-1) \times \xipar^2 \times \aleph \times \left(\frac{m_e}{m_\mu}\right)^\nulep
	\end{equation}
	
	\textbf{Tau (Vorhersage)}:
	\begin{equation}
		a_\tau = \epsilonlep(\tau) \times \xipar^2 \times \aleph \times \left(\frac{m_\tau}{m_\mu}\right)^\nulep
	\end{equation}
	
	\section{Numerische Berechnungen}
	
	\subsection{Eingangsdaten aus Geometrie}
	
	\textbf{Vollst\"andig $\xipar$-basierte Parameter}:
	\begin{align}
		\xipar &= 1{,}759 \times 10^{-4} \quad \text{(aus $\rzero$-Geometrie)} \\
		\xipar^2 &= 3{,}095 \times 10^{-8} \quad \text{(geometrisches Quadrat)} \\
		\aleph &= 0{,}07900 \quad \text{(aus $\xipar \times 13\pi \times 7\pi/2$)} \\
		\nulep &= 1{,}486 \quad \text{(aus fraktaler Dimension $D_f = 2{,}94$)}
	\end{align}
	
	\textbf{Empirische Teilchenmassen} (PDG Werte f\"ur Berechnungen):
	\begin{align}
		m_e &= 0{,}5109989461\,\text{MeV} \quad \text{(Elektron)} \\
		m_\mu &= 105{,}6583745\,\text{MeV} \quad \text{(Myon)} \\
		m_\tau &= 1776{,}86\,\text{MeV} \quad \text{(Tau)}
	\end{align}
	
	\subsection{Konkrete Vorhersagen}
	
	\textbf{Myon-Berechnung} (mit korrigierten konsistenten Werten):
	\begin{equation}
		a_\mu = \xipar^2 \times \aleph = 3{,}095 \times 10^{-8} \times 0{,}07900 = 244{,}5 \times 10^{-11}
	\end{equation}
	
	\textbf{Elektron-Berechnung} (mit empirischen Massen):
	\begin{align}
		a_e &= -\xipar^2 \times \aleph \times \left(\frac{0{,}5110}{105{,}658}\right)^{1{,}486} \\
		&= -3{,}095 \times 10^{-8} \times 0{,}07900 \times (4{,}836 \times 10^{-3})^{1{,}486} \\
		&= -3{,}095 \times 10^{-8} \times 0{,}07900 \times 3{,}624 \times 10^{-4} \\
		&= -0{,}886 \times 10^{-12}
	\end{align}
	
	\textbf{Tau-Berechnung} (mit empirischen Massen):
	\begin{align}
		a_\tau &= \xipar^2 \times \aleph \times \left(\frac{1776{,}86}{105{,}658}\right)^{1{,}486} \\
		&= 3{,}095 \times 10^{-8} \times 0{,}07900 \times (16{,}821)^{1{,}486} \\
		&= 3{,}095 \times 10^{-8} \times 0{,}07900 \times 66{,}34 \\
		&= 1{,}621 \times 10^{-7}
	\end{align}
	
	\section{Experimenteller Vergleich}
	
	\subsection{\"Ubereinstimmung mit Messungen}
	
	\begin{table}[H]
		\centering
		\begin{tabular}{lccc}
			\toprule
			\textbf{Teilchen} & \textbf{T0-Vorhersage} & \textbf{Experiment} & \textbf{Abweichung} \\
			\midrule
			Myon & $244{,}5 \times 10^{-11}$ & $251{,}0 \pm 5{,}4 \times 10^{-11}$ & $1{,}21\sigmadev$ \\
			Elektron & $-0{,}886 \times 10^{-12}$ & $-0{,}91 \pm 2{,}8 \times 10^{-12}$ & $0{,}01\sigmadev$ \\
			Tau & $1{,}621 \times 10^{-7}$ & [nicht messbar] & [Vorhersage] \\
			\bottomrule
		\end{tabular}
		\caption{T0-Vorhersagen vs. experimentelle Messungen}
	\end{table}
	
	\subsection{Statistische Bewertung}
	
	\textbf{Bewertung mit empirischen Massen}:
	\begin{itemize}
		\item \textbf{Myon}: $1{,}21\sigmadev$ Abweichung
		\item \textbf{Elektron}: $0{,}01\sigmadev$ Abweichung
		\item \textbf{Durchschnittliche Genauigkeit}: $97{,}4\%$
	\end{itemize}
	
	\section{Parameterfreie Natur}
	
	\subsection{Vollst\"andige Ableitungskette}
	
	\begin{align}
		&\text{Fundamentale Konstanten } (G, \hbar, c, \lambda_{\text{Higgs}}) \text{ - nur geometrische Inputs} \\
		&\quad \Downarrow \\
		&\text{T0-Feldgleichung} \\
		&\quad \Downarrow \\
		&\rzero = \frac{\lambdaH^2 \times v^2}{16\pi^3 \times m_{\text{H}}^2} \times \ellPlanck \text{ (Higgs-Feldgeometrie)} \\
		&\quad \Downarrow \\
		&\xipar = \frac{4}{3} \times \frac{\rzero}{\ellPlanck} \text{ (sph\"arische Geometrie)} \\
		&\quad \Downarrow \\
		&\alphagem = \xipar \times 13\pi \text{ (ersetzt empirische Feinstrukturkonstante)} \\
		&\quad \Downarrow \\
		&\aleph = \xipar \times 13\pi \times \frac{7\pi}{2} \text{ (vollst\"andig $\xipar$-basierte Kopplung)} \\
		&\quad \Downarrow \\
		&a_\ell = \epsilonlep(\ell) \times \xipar^2 \times \aleph \times \left(\frac{m_\ell}{m_\mu}\right)^\nulep \text{ (Ein-Parameter-Formel)}
	\end{align}
	
	\subsection{Theoretische Reinheit}
	
	\textbf{Keine empirischen Anpassungen}:
	\begin{itemize}
		\item $\xipar$ aus T0-Feldgeometrie abgeleitet
		\item $\aleph$ aus intrinsischer T0-Feldstruktur bestimmt
		\item $\nulep$ aus QFT-Renormierungsgruppen-Analyse
		\item Alle Vorzeichen aus T0-Symmetrie-Eigenschaften
	\end{itemize}
	
	\textbf{Echte Vorhersagen}:
	\begin{itemize}
		\item Keine Parameter an experimentelle Daten angepasst
		\item Alle Werte vor experimentellem Vergleich festgelegt
		\item Falsifizierbare Vorhersagen f\"ur zuk\"unftige Tau-Messungen
	\end{itemize}
	
	\section{Konklusion}
	
	Die T0-Theorie liefert eine \textbf{vollst\"andig geometrische, parameterfreie Erkl\"arung} der leptonischen g-2-Anomalien. Die \"Ubereinstimmung mit experimentellen Daten ($\chisquared = 0{,}01$) bei gleichzeitiger theoretischer Reinheit etabliert T0 als vielversprechenden Kandidaten f\"ur eine fundamentale Vereinheitlichung der Teilchenphysik mit der Geometrie der Raumzeit.
	
	Die \textbf{Kopplungskonstante $\aleph = \alphagem \times (7\pi/2)$} repr\"asentiert die intrinsische elektromagnetische Strukturkonstante der T0-Geometrie und unterscheidet sich konzeptionell von empirisch angepassten Parametern durch ihre geometrische Ableitbarkeit aus ersten Prinzipien.
	
\end{document}