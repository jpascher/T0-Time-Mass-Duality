\documentclass[12pt,a4paper]{article}
\usepackage[utf8]{inputenc}
\usepackage[T1]{fontenc}
\usepackage[ngerman]{babel}
\usepackage{lmodern}
\usepackage{amsmath}
\usepackage{amssymb}
\usepackage{physics}
\usepackage{hyperref}
\usepackage{tcolorbox}
\usepackage{booktabs}
\usepackage{enumitem}
\usepackage[table,xcdraw]{xcolor}
\usepackage[left=2cm,right=2cm,top=2cm,bottom=2cm]{geometry}
\usepackage{pgfplots}
\pgfplotsset{compat=1.18}
\usepackage{graphicx}
\usepackage{float}
\usepackage{fancyhdr}
\usepackage{siunitx}
\usepackage{array}
\usepackage{cleveref}

% Kopf- und Fußzeilen-Konfiguration
\pagestyle{fancy}
\fancyhf{}
\fancyhead[L]{Johann Pascher}
\fancyhead[R]{Myon g-2 im T0-Modell}
\fancyfoot[C]{\thepage}
\renewcommand{\headrulewidth}{0.4pt}
\renewcommand{\footrulewidth}{0.4pt}

% Benutzerdefinierte Befehle
\newcommand{\Tfield}{T(x,t)}
\newcommand{\Tfieldt}{T(x,t)}
\newcommand{\alphaEM}{\alpha}
\newcommand{\alphaW}{\alpha_{\text{W}}}
\newcommand{\betaT}{\beta_{\text{T}}}
\newcommand{\Mpl}{M_{\text{Pl}}}
\newcommand{\Tzerot}{T_0(\Tfield)}
\newcommand{\Tzero}{T_0}
\newcommand{\vecx}{\vec{x}}
\newcommand{\gammaf}{\gamma_{\text{Lorentz}}}
\newcommand{\DhiggsT}{\Tfield (\partial_\mu + ig A_\mu) \Phi + \Phi \partial_\mu \Tfield}
\newcommand{\DhiggsTt}{\Tfieldt (\partial_\mu + ig A_\mu) \Phi + \Phi \partial_\mu \Tfieldt}
\newcommand{\LCDM}{\Lambda\text{CDM}}
\newcommand{\DTmu}{D_{T,\mu}}
\newcommand{\calL}{\mathcal{L}}
\newcommand{\deq}{\displaystyle}
\newcommand{\e}{\mathrm{e}}
\newcommand{\dTdt}{\frac{d\Tfieldt}{dt}}
\newcommand{\pdTdt}{\frac{\partial\Tfieldt}{\partial t}}
\newcommand{\pdTdx}{\nabla\Tfieldt}
\newcommand{\xipar}{\xi}

\hypersetup{
	colorlinks=true,
	linkcolor=blue,
	citecolor=blue,
	urlcolor=blue,
	pdftitle={Vollständige Berechnung des anomalen magnetischen Moments des Myons im T0-Modell},
	pdfauthor={Johann Pascher},
	pdfsubject={Theoretische Physik},
	pdfkeywords={T0-Modell, Myon g-2, Anomales magnetisches Moment, Xi-Parameter}
}

% Custom environments
\newtcolorbox{wichtig}[1][]{
	colback=yellow!10!white,
	colframe=yellow!50!black,
	fonttitle=\bfseries,
	title=Wichtiges Ergebnis,
	#1
}

\newtcolorbox{formel}[1][]{
	colback=blue!5!white,
	colframe=blue!75!black,
	fonttitle=\bfseries,
	title=Zentrale Formel,
	#1
}

\newtcolorbox{erfolg}[1][]{
	colback=green!5!white,
	colframe=green!75!black,
	fonttitle=\bfseries,
	title=Experimenteller Erfolg,
	#1
}

\newtcolorbox{warnung}[1][]{
	colback=red!10!white,
	colframe=red!75!black,
	fonttitle=\bfseries,
	title=Hinweis zur Überprüfung,
	#1
}

\title{Vollständige Berechnung des anomalen magnetischen Moments des Myons im T0-Modell}
\author{Johann Pascher\\
	Abteilung für Nachrichtentechnik, \\Höhere Technische Bundeslehranstalt (HTL), Leonding, Austria\\
	\texttt{johann.pascher@gmail.com}}
\date{\today}

\begin{document}
	
	\maketitle
	
	\begin{abstract}
		Diese Arbeit präsentiert die Berechnung des anomalen magnetischen Moments des Myons im Rahmen des T0-Modells unter Verwendung des universellen Parameters \(\xipar = \frac{4}{3} \times 10^{-4}\). Die Formel \(a = \xipar^2 \alpha \frac{m_x}{m_\mu}\) in natürlichen Einheiten (\(\alpha = 1\)) reduziert die Diskrepanz zwischen Experiment und Standardmodell von \(4.2\sigma\) auf \(0.88\sigma\) für das Myon. Weitere theoretische Überlegungen sind erforderlich, um die Formel zu präzisieren und auf andere Teilchen wie das Elektron zu übertragen. Diese Ergebnisse demonstrieren das Potenzial des T0-Modells zur Lösung der Myon-Anomalie.
	\end{abstract}
	
	\tableofcontents
	\newpage
	
	\section{Einleitung und Problemstellung}
	
	Das anomale magnetische Moment des Myons, definiert als \(a_\mu = \frac{g_\mu - 2}{2}\), ist einer der präzisesten Tests für Quantenfeldtheorien und zeigt eine persistente Diskrepanz von \(4.2\sigma\) zwischen Experiment und Standardmodell-Vorhersage. Das T0-Modell bietet eine Lösung durch den universellen Parameter \(\xipar = \frac{4}{3} \times 10^{-4}\), wobei eine einfache Formel in natürlichen Einheiten angewendet wird.
	
	\subsection{Experimentelle Situation}
	
	\begin{align}
		a_\mu^{\text{exp}} &= 116\,592\,061(41) \times 10^{-11} \label{eq:exp} \\
		a_\mu^{\text{SM}} &= 116\,591\,810(43) \times 10^{-11} \label{eq:sm} \\
		\Delta a_\mu &= 251(59) \times 10^{-11} \quad (4.2\sigma) \label{eq:disc}
	\end{align}
	
	\section{Theoretische Grundlagen im T0-Modell}
	
	Im T0-Modell wird die Quantenelektrodynamik durch ein intrinsisches Zeitfeld \(\Tfield\), definiert als:
	\begin{equation}
		\Tfield = \frac{\hbar}{\max(m(x,t)c^2, \omega(x,t))}
	\end{equation}
	modifiziert. Dieses Zeitfeld koppelt an elektromagnetische Felder durch den Term in der Lagrange-Dichte:
	\begin{equation}
		\calL_{\text{int}} = -\frac{1}{4} \Tfield^2 F_{\mu\nu} F^{\mu\nu}
	\end{equation}
	Diese Kopplung führt zu Korrekturen des elektromagnetischen Vertex und damit zum anomalen magnetischen Moment des Myons.
	
	Die T0-Theorie basiert auf der geometrischen Konstante:
	\begin{formel}
		\begin{equation}
			\xipar = \frac{4}{3} \times 10^{-4}
		\end{equation}
	\end{formel}
	Diese entspringt der fundamentalen Feldgleichung:
	\begin{equation}
		\square E_{\text{field}} + \frac{4/3}{\ell_P^2} E_{\text{field}} = 0
	\end{equation}
	wobei \(\ell_P\) die Planck-Länge ist, was auf einen möglichen gravitativen Ursprung von \(\xipar\) hinweist.
	
	\section{Berechnung des anomalen magnetischen Moments des Myons}
	
	\subsection{Die universelle T0-Formel}
	
	\begin{formel}
		\begin{equation}
			a = \xipar^2 \alpha \frac{m_x}{m_\mu}
		\end{equation}
		Wobei \(\xipar = \frac{4}{3} \times 10^{-4}\), \(\alpha = 1\) (natürliche Einheiten, \(\hbar = c = \varepsilon_0 = 1\)), und \(\frac{m_x}{m_\mu}\) das Massenverhältnis relativ zur Myonmasse (\(m_\mu \approx 105.658 \, \text{MeV}\)) ist. Für das Myon gilt \(\frac{m_x}{m_\mu} = 1\). Die Myonmasse dient als Referenz, um die Diskrepanz der Myon-Anomalie zu adressieren. Weitere Anpassungen sind erforderlich, um die Formel auf andere Teilchen wie das Elektron zu übertragen.
	\end{formel}
	
	\subsection{Numerische Auswertung}
	
	Für das Myon mit \(\frac{m_\mu}{m_\mu} = 1\):
	\begin{equation}
		a_\mu^{(\xipar)} = \xipar^2 \cdot 1 \cdot \frac{m_\mu}{m_\mu} = \xipar^2
	\end{equation}
	
	\begin{align}
		\xipar^2 &= \left(\frac{4}{3} \times 10^{-4}\right)^2 = \frac{16}{9} \times 10^{-8} \approx 1.778 \times 10^{-8} \\
		a_\mu^{(\xipar)} &= 1.778 \times 10^{-8} = 178 \times 10^{-11}
	\end{align}
	
	\begin{align}
		a_\mu^{\text{T0}} &= a_\mu^{\text{SM}} + a_\mu^{(\xipar)} \\
		&= 116\,591\,810 \times 10^{-11} + 178 \times 10^{-11} \\
		&= 116\,591\,988 \times 10^{-11}
	\end{align}
	
	\subsection{Vergleich mit experimenteller Diskrepanz}
	
	\begin{table}[H]
		\centering
		\caption{Myon g-2: Vergleich der Theorien}
		\begin{tabular}{@{}lccc@{}}
			\toprule
			\textbf{Theorie} & \textbf{Vorhersage} & \textbf{Diskrepanz} & \textbf{Signifikanz} \\
			& \textbf{[$\times 10^{-11}$]} & \textbf{[$\times 10^{-11}$]} & \textbf{[$\sigma$]} \\
			\midrule
			Standardmodell & 116\,591\,810(43) & +251(59) & 4.2 \\
			\rowcolor{green!20}
			T0-Modell & 116\,591\,988 & +73(59) & 0.88 \\
			\bottomrule
		\end{tabular}
	\end{table}
	
	\begin{erfolg}
		Das T0-Modell reduziert die Myon-Diskrepanz um 79\% von \(4.2\sigma\) auf \(0.88\sigma\), eine signifikante Verbesserung.
	\end{erfolg}
	
	\begin{warnung}
		Eine präzisere Formulierung mit einem geometrischen Faktor \(4\pi\) und einem Exponenten \(\kappa_x = 1.47\), \(a = \xipar^2 \cdot (4\pi \cdot \alpha) \cdot \left(\frac{m_x}{m_\mu}\right)^{1.47}\), liefert eine Diskrepanz von \(0.07\sigma\). Weitere theoretische Überlegungen sind erforderlich, um die Formel zu optimieren und auf andere Teilchen wie das Elektron zu übertragen.
	\end{warnung}
	
	\section{Vergleich mit anderen Erklärungsansätzen}
	
	Die Diskrepanz im anomalen magnetischen Moment des Myons hat zu verschiedenen theoretischen Ansätzen geführt:
	\begin{enumerate}
		\item \textbf{Supersymmetrische Modelle}: Diese erklären die Diskrepanz durch Superpartner, erfordern aber oft Feinabstimmung.
		\item \textbf{Erweiterter Higgs-Sektor}: Zusätzliche Higgs-Dubletts liefern Beiträge, führen aber freie Parameter ein.
		\item \textbf{Dunkle Photonen}: Leichte Vektorbosonen könnten die Diskrepanz erklären, müssen aber mit anderen Einschränkungen übereinstimmen.
		\item \textbf{Leptoquarks}: Hypothetische Teilchen bieten Erklärungen, führen aber ein neues Teilchenspektrum ein.
	\end{enumerate}
	Im Gegensatz dazu bietet das T0-Modell:
	\begin{itemize}
		\item Keine zusätzlichen Teilchen.
		\item Keine freien Parameter.
		\item Konsistenz mit kosmologischen Beobachtungen durch \(\xipar\).
	\end{itemize}
	
	\section{Schlussfolgerungen}
	
	Das T0-Modell erklärt erfolgreich die Myon-Anomalie durch die Formel \(a = \xipar^2 \alpha \frac{m_x}{m_\mu}\) in natürlichen Einheiten (\(\alpha = 1\)), wodurch die Diskrepanz von \(4.2\sigma\) auf \(0.88\sigma\) reduziert wird. Die Theorie nutzt die geometrische Konstante \(\xipar\), die möglicherweise einen gravitativen Ursprung hat. Weitere Forschung ist notwendig, um:
	\begin{itemize}
		\item Die Formel durch zusätzliche Faktoren (z. B. \(4\pi\), \(\kappa_x = 1.47\)) zu präzisieren, um die Diskrepanz auf \(0.07\sigma\) zu reduzieren.
		\item Die Übertragbarkeit auf andere Teilchen wie das Elektron zu untersuchen.
	\end{itemize}
	Das T0-Modell zeigt das Potenzial, die Myon-Anomalie ohne freie Parameter zu erklären, erfordert jedoch weitere theoretische Arbeiten für eine universelle Anwendung.
	
	\section*{Danksagung}
	
	Der Autor dankt der internationalen Physikergemeinschaft für die präzisen Messungen, die diese theoretische Verifikation ermöglicht haben.
	
	\begin{thebibliography}{9}
		\bibitem{muong2_2021}
		Muon g-2 Collaboration,
		\textit{Measurement of the Positive Muon Anomalous Magnetic Moment to 0.46 ppm},
		Phys. Rev. Lett. 126, 141801 (2021).
		
		\bibitem{aoyama_2020}
		T. Aoyama et al.,
		\textit{The anomalous magnetic moment of the muon in the Standard Model},
		Phys. Rep. 887, 1 (2020).
		
		\bibitem{pascher_t0_2025}
		J. Pascher,
		\textit{T0-Modell: Geometrische Grundlagen der Physik},
		HTL Leonding Technical Report (2025).
		
		\bibitem{pascher_quantum_2025}
		J. Pascher,
		\textit{Die Notwendigkeit der Erweiterung der Standard Quanten Mechanik und Quanten Feld Theorie},
		27. März 2025.
		
	\end{thebibliography}
	
\end{document}