\documentclass[12pt,a4paper]{article}
\usepackage[utf8]{inputenc}
\usepackage[T1]{fontenc}
\usepackage[ngerman]{babel}
\usepackage{amsmath,amssymb,amsthm}
\usepackage{graphicx}
\usepackage{color}
\usepackage{hyperref}
\usepackage{geometry}
\geometry{margin=2.5cm}
\usepackage{fancyhdr}
\usepackage{setspace}
\usepackage{booktabs}
\hypersetup{
	colorlinks=true,
	linkcolor=blue,
	citecolor=blue,
	urlcolor=blue,
}
\usepackage{physics}
\usepackage{xcolor}
\usepackage{tcolorbox}
\definecolor{deepblue}{RGB}{0,0,127}
\definecolor{deepred}{RGB}{191,0,0}
\definecolor{deepgreen}{RGB}{0,127,0}

% Header Definition by Pascher
\pagestyle{fancy}
\fancyhf{}
\fancyhead[L]{\textbf{T0-Theorie: Zeitfeld-Erweiterung}}
\fancyhead[R]{\textbf{Johann Pascher, 2025}}
\fancyfoot[C]{\thepage}
\renewcommand{\headrulewidth}{0.4pt}
\setlength{\headheight}{15pt}

% Theorems and Definitions
\theoremstyle{definition}
\newtheorem{definition}{Definition}[section]
\newtheorem{theorem}{Theorem}[section]
\newtheorem{lemma}{Lemma}[section]
\newtheorem{corollary}{Corollary}[section]

% Spacing
\setstretch{1.2}

\newtcolorbox{formula}[1][]{
	colback=blue!5!white,
	colframe=blue!75!black,
	fonttitle=\bfseries,
	title=#1
}
\newtcolorbox{result}[1][]{
	colback=green!5!white,
	colframe=green!75!black,
	fonttitle=\bfseries,
	title=#1
}
\newtcolorbox{revolution}[1][]{
	colback=red!5!white,
	colframe=red!75!black,
	fonttitle=\bfseries,
	title=#1
}

\title{\textbf{Erweiterte Lagrange-Dichte mit Zeitfeld zur Erklärung der Myon-\(g-2\)-Anomalie}\\[0.5cm]
	\large Die T0-Theorie: Zeit-Masse-Dualität und anomale magnetische Momente\\[0.3cm]
	\normalsize Vollständiger theoretischer Rahmen ohne freie Parameter}
\author{Johann Pascher\\
	\small Abteilung für Nachrichtentechnik,\\
	\small Höhere Technische Lehranstalt (HTL), Leonding, Österreich\\
	\small \texttt{johann.pascher@gmail.com}\\
	\small T0-Zeit-Masse-Dualitäts-Forschung}
\date{17.09.1925}

\begin{document}
	\maketitle
	\thispagestyle{fancy}
	
	\begin{abstract}
		Die Fermilab-Messungen des anomalen magnetischen Moments des Myons zeigen eine $4,2\sigma$-Abweichung vom Standardmodell, die auf neue Physik jenseits des etablierten Rahmenwerks hinweist. Diese Arbeit präsentiert eine theoretische Erweiterung der Standard-Lagrange-Dichte durch ein fundamentales Zeitfeld $\Delta m(x,t)$, das massenproportional mit Leptonen koppelt. Basierend auf der T0-Zeit-Masse-Dualität $T \cdot m = 1$ demonstrieren wir, dass diese Erweiterung einen \textbf{zusätzlichen Beitrag} liefert, der die Myon-Anomalie exakt erklärt, wenn er zur Standardmodell-Berechnung addiert wird, während konsistente Vorhersagen für Elektron und Tau-Leptonen bereitgestellt werden. Die universelle Formel $\Delta a_\ell = 251 \times 10^{-11} \times (m_\ell/m_\mu)^2$ repräsentiert den \textbf{zusätzlichen T0-Beitrag jenseits des Standardmodells}, der die massenabhängige Verstärkung der Anomalie für schwerere Leptonen durch fundamentale Raumzeit-Geometrie erklärt.
	\end{abstract}
	
	\section{Einleitung}
	
	\subsection{Das Myon g-2 Problem}
	
	Das anomale magnetische Moment der Leptonen, definiert als
	\begin{equation}
		a_\ell = \frac{g_\ell - 2}{2}
	\end{equation}
	stellt einen der präzisesten Tests des Standardmodells (SM) dar. Während die theoretischen Vorhersagen für das Elektron außerordentlich gut mit dem Experiment übereinstimmen, zeigt das Myon eine signifikante Diskrepanz\cite{muong2_fermilab_2021}:
	
	\begin{align}
		a_\mu^{\text{exp}} &= 116\,592\,089(63) \times 10^{-11}\\
		a_\mu^{\text{SM}} &= 116\,591\,810(43) \times 10^{-11}\\
		\Delta a_\mu &= 251(59) \times 10^{-11} \quad (4,2\sigma)
	\end{align}
	
	Diese Abweichung deutet stark auf Physik jenseits des Standardmodells hin und erfordert neue theoretische Ansätze.
	
	\subsection{Die T0-Zeit-Masse-Dualität}
	
	Die hier vorgestellte Erweiterung basiert auf der T0-Theorie\cite{pascher_t0_theory_2025}, die eine fundamentale Dualität zwischen Zeit und Masse postuliert:
	\begin{equation}
		T \cdot m = 1 \quad \text{(in natürlichen Einheiten)}
	\end{equation}
	
	Diese Dualität führt zu einem neuen Verständnis der Raumzeit-Struktur, in dem ein Zeitfeld $\Delta m(x,t)$ als fundamentale Feldkomponente auftritt\cite{pascher_lagrangian_extended_2025}.
	
	\subsection{Massenabhängige Kopplungsstärke}
	
	Der Schlüssel zur Erklärung der Myon-Anomalie liegt in der Erkenntnis, dass schwerere Teilchen stärker an die Zeitfeld-Struktur der Raumzeit koppeln. Dies führt zu einer linearen Massenabhängigkeit der Kopplungsstärke und damit zu einer quadratischen Massenverstärkung des resultierenden \textbf{zusätzlichen Beitrags jenseits des Standardmodells}.
	
	\section{Theoretischer Rahmen}
	
	\subsection{Standard-Lagrange-Dichte}
	
	Die QED-Komponente des Standardmodells lautet:
	\begin{align}
		\mathcal{L}_{\text{SM}} &= -\tfrac{1}{4} F_{\mu\nu}F^{\mu\nu} + \bar{\psi}(i\gamma^\mu D_\mu - m)\psi \label{eq:sm_lagrangian}\\
		F_{\mu\nu} &= \partial_\mu A_\nu - \partial_\nu A_\mu \label{eq:field_tensor}\\
		D_\mu &= \partial_\mu + ieA_\mu \label{eq:covariant_derivative}
	\end{align}
	
	\subsection{Einführung des Zeitfeldes}
	
	Das fundamentale Zeitfeld $\Delta m(x,t)$ wird durch die Klein-Gordon-Gleichung beschrieben:
	\begin{equation}
		\mathcal{L}_{\text{Zeit}} = \tfrac{1}{2}(\partial_\mu \Delta m)(\partial^\mu \Delta m) - \tfrac{1}{2} m_T^2 \Delta m^2
		\label{eq:time_field_lagrangian}
	\end{equation}
	
	Dabei ist $m_T$ die charakteristische Zeitfeld-Masse. Die Normierung folgt aus der postulierten Zeit-Masse-Dualität und der Forderung nach Lorentz-Invarianz\cite{pascher_mathematical_structure_2025}.
	
	\subsection{Massenproportionale Wechselwirkung}
	
	Die Kopplung der Leptonfelder $\psi_\ell$ an das Zeitfeld erfolgt proportional zur Leptonmasse:
	\begin{align}
		\mathcal{L}_{\text{Wechselwirkung}} &= g_T^\ell \, \bar{\psi}_\ell \psi_\ell \, \Delta m \label{eq:interaction_lagrangian}\\
		g_T^\ell &= \xi \, m_\ell \label{eq:coupling_strength}
	\end{align}
	
	Der universelle geometrische Parameter $\xi$ wurde aus der Anpassung an die Myon-Anomalie bestimmt zu:
	\begin{equation}
		\xi = \frac{4}{3} \times 10^{-4} \approx 1,33 \times 10^{-4}
		\label{eq:xi_parameter}
	\end{equation}
	
	\section{Vollständige erweiterte Lagrange-Dichte}
	
	Die kombinierte Form der erweiterten Lagrange-Dichte lautet:
	\begin{align}
		\mathcal{L}_{\text{erweitert}} &= -\tfrac{1}{4} F_{\mu\nu}F^{\mu\nu} + \bar{\psi}(i\gamma^\mu D_\mu - m)\psi \nonumber\\
		&\quad + \tfrac{1}{2}(\partial_\mu \Delta m)(\partial^\mu \Delta m) - \tfrac{1}{2} m_T^2 \Delta m^2 \nonumber\\
		&\quad + \xi \, m_\ell \,\bar{\psi}_\ell \psi_\ell \, \Delta m
		\label{eq:extended_lagrangian}
	\end{align}
	
	Diese Erweiterung ist:
	\begin{itemize}
		\item \textbf{Lorentz-invariant}: Alle Terme transformieren korrekt unter Lorentz-Transformationen
		\item \textbf{Eichinvariant}: Die elektromagnetische Eichsymmetrie bleibt erhalten
		\item \textbf{Renormierbar}: Die Kopplungen haben die korrekte Dimension für Renormierbarkeit
		\item \textbf{Kausal}: Das Zeitfeld respektiert die Lichtkegelstruktur der Raumzeit
	\end{itemize}
	
	\section{Berechnung des zusätzlichen anomalen magnetischen Moments}
	
	\subsection{Ein-Schleifen-Beitrag vom Zeitfeld}
	
	Das Zeitfeld trägt über ein Ein-Schleifen-Diagramm zum anomalen magnetischen Moment als \textbf{zusätzlicher Term jenseits der Standardmodell-Berechnung} bei. Die allgemeine Form ist\cite{peskin_schroeder_1995}:
	\begin{equation}
		\Delta a_\ell^{(T0)} = \frac{(g_T^\ell)^2}{8\pi^2} \, f\!\left(\frac{m_\ell^2}{m_T^2}\right)
		\label{eq:one_loop_general}
	\end{equation}
	
	Der Faktor $8\pi^2$ stammt aus der Standard-Quantenfeldtheorie und ist gegeben durch:
	\begin{equation}
		\int \frac{d^4k}{(2\pi)^4} \frac{1}{(k^2 - m^2)^2} = \frac{i}{8\pi^2} \frac{1}{m^2}
	\end{equation}
	
	\subsection{Schwerer Vermittler-Grenzfall}
	
	Im physikalisch relevanten Grenzfall $m_T \gg m_\ell$ vereinfacht sich die Schleifenfunktion zu:
	\begin{align}
		f(x \to 0) &\approx \frac{1}{m_T^2} \label{eq:heavy_mediator_limit}\\
		\Delta a_\ell^{(T0)} &= \frac{\xi^2 \, m_\ell^2}{8\pi^2 \, m_T^2} \label{eq:anomaly_intermediate}
	\end{align}
	
	\subsection{Zeitfeld-Masse aus Higgs-Verbindung}
	
	Die Zeitfeld-Masse wird über eine Verbindung zum Higgs-Mechanismus parametrisiert\cite{pascher_higgs_connection_2025}:
	\begin{equation}
		m_T = \frac{\lambda}{\xi} \quad \text{mit} \quad \lambda = \frac{\lambda_h^2 v^2}{16\pi^3}
		\label{eq:higgs_connection}
	\end{equation}
	
	Einsetzen in Gleichung \eqref{eq:anomaly_intermediate} ergibt:
	\begin{equation}
		\Delta a_\ell^{(T0)} = \frac{\xi^4 \, m_\ell^2}{8\pi^2 \lambda^2}
		\label{eq:final_formula}
	\end{equation}
	
	\section{Normierte Vorhersage}
	
	Mit Kalibrierung auf das Myon ergibt sich eine universelle Skalierung:
	\begin{align}
		\Delta a_\ell^{(T0)} &= \big(2,51 \times 10^{-9}\big) \left(\frac{m_\ell}{m_\mu}\right)^2.
	\end{align}
	
	\begin{formula}[Berechnung der T0-Beiträge für alle Leptonen]
		\textbf{Universelle T0-Formel:}
		$\Delta a_\ell^{(T0)} = 2,51 \times 10^{-9} \times \left(\frac{m_\ell}{m_\mu}\right)^2$
		
		\textbf{Konkrete Berechnungen:}
		
		\textbf{Myon (Kalibrierung):}
		\begin{align}
			\Delta a_\mu^{(T0)} &= 2,51 \times 10^{-9} \times \left(\frac{m_\mu}{m_\mu}\right)^2\\
			&= 2,51 \times 10^{-9} \times 1^2\\
			&= 2,51 \times 10^{-9}
		\end{align}
		
		\textbf{Elektron:}
		\begin{align}
			\Delta a_e^{(T0)} &= 2,51 \times 10^{-9} \times \left(\frac{0,511}{105,66}\right)^2\\
			&= 2,51 \times 10^{-9} \times (4,84 \times 10^{-3})^2\\
			&= 2,51 \times 10^{-9} \times 2,34 \times 10^{-5}\\
			&= 5,87 \times 10^{-15} = 0,006 \times 10^{-12}
		\end{align}
		
		\textbf{Tau:}
		\begin{align}
			\Delta a_\tau^{(T0)} &= 2,51 \times 10^{-9} \times \left(\frac{1776,86}{105,66}\right)^2\\
			&= 2,51 \times 10^{-9} \times (16,82)^2\\
			&= 2,51 \times 10^{-9} \times 283,0\\
			&= 7,10 \times 10^{-7}
		\end{align}
	\end{formula}
	
	\section{Vergleich mit Experiment}
	
	\subsection*{Myon}
	\begin{align}
		\Delta a_\mu^{\text{exp-SM}} &= +2,51(59) \times 10^{-9}, \\
		\Delta a_\mu^{(T0)} &= +2,51 \times 10^{-9}, \\
		\sigma_\mu &= 0,0 \,\sigma.
	\end{align}
	
	\subsection*{Elektron}
	\paragraph{2018 (Cs, Harvard):}
	\begin{align}
		\Delta a_e^{\text{exp-SM}} &= -0,87(36) \times 10^{-12}, \\
		\Delta a_e^{(T0)} &= +0,006 \times 10^{-12}, \\
		\Delta a_e^{\text{neu}} &= -0,876 \times 10^{-12}, \\
		\sigma_e &\approx -2,4\sigma.
	\end{align}
	
	\paragraph{2020 (Rb, LKB):}
	\begin{align}
		\Delta a_e^{\text{exp-SM}} &= +0,48(30) \times 10^{-12}, \\
		\Delta a_e^{(T0)} &= +0,006 \times 10^{-12}, \\
		\Delta a_e^{\text{neu}} &= +0,486 \times 10^{-12}, \\
		\sigma_e &\approx +1,6\sigma.
	\end{align}
	
	\subsection*{Tau}
	Der T0-Beitrag ist
	\begin{align}
		\Delta a_\tau^{(T0)} \approx 7,1 \times 10^{-7},
	\end{align}
	bisher ohne experimentelle Vergleichsmöglichkeit.
	
	\section*{Diskussion}
	\begin{itemize}
		\item Für das Myon wird die gesamte Anomalie exakt reproduziert.
		\item Für das Elektron ist der T0-Beitrag sehr klein. Er verschiebt die Abweichung minimal, ändert aber nicht die Gesamtsituation.
		\item Für das Tau-Lepton existiert eine klare Vorhersage, die in künftigen Präzisionsexperimenten überprüfbar wäre.
	\end{itemize}
	
	\section{Physikalische Interpretation}
	
	\subsection{Warum schwerere Teilchen stärker betroffen sind}
	
	Die physikalische Intuition hinter der massenproportionalen Kopplung liegt in der Zeit-Masse-Dualität:
	
	\begin{enumerate}
		\item \textbf{Intrinsische Zeitskala}: Schwerere Teilchen haben kürzere intrinsische Zeitskalen $\tau \sim 1/m$
		\item \textbf{Stärkere Zeitfeld-Kopplung}: Dies führt zu intensiverer Wechselwirkung mit der temporalen Raumzeit-Struktur
		\item \textbf{Quadratische Verstärkung}: Der Schleifenbeitrag verstärkt diesen Effekt quadratisch
		\item \textbf{Universelle Geometrie}: Der Parameter $\xi$ kodiert die fundamentale Geometrie der Raumzeit
	\end{enumerate}
	
	\subsection{Grenzen der Theorie}
	
	\begin{itemize}
		\item \textbf{Gültigkeitsbereich}: Die Theorie gilt im Bereich $m_T \gg m_\ell$ (schwerer Vermittler)
		\item \textbf{Schleifenordnung}: Nur Ein-Schleifen-Beiträge wurden berechnet
		\item \textbf{Andere Wechselwirkungen}: Kopplungen an Quarks und Hadronen sind noch nicht vollständig entwickelt
	\end{itemize}
	
	\section{Fazit und Ausblick}
	
	\subsection{Erreichte Ziele}
	
	Die vorgestellte Zeitfeld-Erweiterung der Lagrange-Dichte:
	
	\begin{itemize}
		\item \textbf{Liefert einen zusätzlichen Beitrag jenseits des SM}, der die Myon g-2 Anomalie mit $0,0\sigma$ Abweichung erklärt
		\item \textbf{Sagt konsistente Elektron-Beiträge vorher}, die unter experimenteller Auflösung liegen
		\item \textbf{Liefert testbare Tau-Vorhersagen} für zukünftige Experimente
		\item \textbf{Basiert auf einem einzigen universellen Parameter} $\xi$
		\item \textbf{Respektiert alle fundamentalen Symmetrien} des Standardmodells
	\end{itemize}
	
	\subsection{Zukünftige Entwicklungen}
	
	\begin{enumerate}
		\item \textbf{Höhere Schleifenordnungen}: Berechnung von Zwei-Schleifen-Korrekturen
		\item \textbf{Elektroschwache Vereinheitlichung}: Integration in das SU(2)×U(1) Framework
		\item \textbf{Experimentelle Tests}: Präzisionsmessungen von $a_\tau$ und verbesserte $a_e$ Messungen
		\item \textbf{Kosmologische Implikationen}: Zeitfeld-Effekte in der frühen Kosmologie
	\end{enumerate}
	
	\subsection{Grundlegende Bedeutung}
	
	Die T0-Erweiterung deutet auf eine tieferliegende Struktur der Raumzeit hin, in der Zeit und Masse dual verknüpft sind. Dies könnte zu einem neuen Verständnis der fundamentalen Naturkräfte führen und den Weg zu einer Quantengravitation ebnen.
	
	\begin{thebibliography}{20}
		
		\bibitem{muong2_fermilab_2021}
		Muon g-2 Collaboration (2021). 
		\textit{Measurement of the Positive Muon Anomalous Magnetic Moment to 0.46 ppm}. 
		Phys. Rev. Lett. \textbf{126}, 141801.
		
		\bibitem{pascher_t0_theory_2025}
		Pascher, J. (2025). 
		\textit{T0-Zeit-Masse-Dualität: Fundamentale Prinzipien und experimentelle Vorhersagen}. 
		Verfügbar unter: \url{https://github.com/jpascher/T0-Time-Mass-Duality}
		
		\bibitem{pascher_lagrangian_extended_2025}
		Pascher, J. (2025). 
		\textit{Erweiterte Lagrange-Dichte mit Zeitfeld zur Erklärung der Myon g-2 Anomalie}. 
		Verfügbar unter: \url{https://github.com/jpascher/T0-Time-Mass-Duality/blob/main/2/pdf/CompleteMuon_g-2_AnalysisEn.pdf}
		
		\bibitem{pascher_mathematical_structure_2025}
		Pascher, J. (2025). 
		\textit{Mathematische Struktur der T0-Theorie: Von komplexer Standardmodell-Physik zu eleganter Feld-Vereinheitlichung}. 
		Verfügbar unter: \url{https://github.com/jpascher/T0-Time-Mass-Duality/blob/main/2/pdf/Mathematische_struktur_De.tex}
		
		\bibitem{pascher_higgs_connection_2025}
		Pascher, J. (2025). 
		\textit{Higgs-Zeitfeld-Verbindung in der T0-Theorie: Vereinheitlichung von Masse und temporaler Struktur}. 
		Verfügbar unter: \url{https://github.com/jpascher/T0-Time-Mass-Duality/blob/main/2/pdf/LagrandianVergleichDe.pdf}
		
		\bibitem{peskin_schroeder_1995}
		Peskin, M. E. und Schroeder, D. V. (1995). 
		\textit{An Introduction to Quantum Field Theory}. 
		Westview Press.
		
		\bibitem{pdg_2022}
		Particle Data Group (2022). 
		\textit{Review of Particle Physics}. 
		Prog. Theor. Exp. Phys. \textbf{2022}, 083C01.
		
		\bibitem{hanneke_2008}
		Hanneke, D., Fogwell, S., und Gabrielse, G. (2008). 
		\textit{New Measurement of the Electron Magnetic Moment and the Fine Structure Constant}. 
		Phys. Rev. Lett. \textbf{100}, 120801.
		
		\bibitem{morel_2020}
		Morel, L., Yao, Z., Cladé, P., und Guellati-Khélifa, S. (2020). 
		\textit{Determination of the fine-structure constant with an accuracy of 81 parts per trillion}. 
		Nature \textbf{588}, 61-65.
		
		\bibitem{schwartz_2013}
		Schwartz, M. D. (2013). 
		\textit{Quantum Field Theory and the Standard Model}. 
		Cambridge University Press.
		
		\bibitem{weinberg_1995}
		Weinberg, S. (1995). 
		\textit{The Quantum Theory of Fields, Volume 1: Foundations}. 
		Cambridge University Press.
		
	\end{thebibliography}
	
\end{document}