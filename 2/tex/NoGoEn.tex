\documentclass[12pt,a4paper]{article}
\usepackage[utf8]{inputenc}
\usepackage[T1]{fontenc}
\usepackage[english]{babel}
\usepackage[left=2cm,right=2cm,top=2cm,bottom=2cm]{geometry}
\usepackage{lmodern}
\usepackage{amsmath}
\usepackage{amssymb}
\usepackage{physics}
\usepackage{hyperref}
\usepackage{tcolorbox}
\usepackage{booktabs}
\usepackage{enumitem}
\usepackage[table,xcdraw]{xcolor}
\usepackage{graphicx}
\usepackage{float}
\usepackage{mathtools}
\usepackage{amsthm}
\usepackage{siunitx}
\usepackage{fancyhdr}

% Headers and Footers
\pagestyle{fancy}
\fancyhf{}
\fancyhead[L]{T0 Theory vs Bell's Theorem}
\fancyhead[R]{Theoretical Deep Dive}
\fancyfoot[C]{\thepage}
\renewcommand{\headrulewidth}{0.4pt}
\renewcommand{\footrulewidth}{0.4pt}

% Custom Commands
\newcommand{\Efield}{E}
\newcommand{\xipar}{\xi}

\hypersetup{
	colorlinks=true,
	linkcolor=blue,
	citecolor=blue,
	urlcolor=blue,
	pdftitle={T0 Theory vs Bell's Theorem: How Deterministic Energy Fields Circumvent No-Go Theorems},
	pdfauthor={T0 Theoretical Physics Research},
	pdfsubject={T0 Theory, Bell's Theorem, No-Go Theorems, Quantum Foundations}
}

\title{T0 Theory vs Bell's Theorem: \\
	How Deterministic Energy Fields Circumvent No-Go Theorems \\
	\large A Critical Analysis of Superdeterminism and Measurement Freedom}
\author{T0 Theoretical Physics Research}
\date{\today}

\begin{document}
	
	\maketitle
	
	\begin{abstract}
		This document presents a comprehensive theoretical analysis of how the T0-energy field formulation confronts and potentially circumvents fundamental no-go theorems in quantum mechanics, particularly Bell's theorem and the Kochen-Specker theorem. We demonstrate that T0 theory employs a sophisticated strategy based on "superdeterminism" and violation of measurement freedom assumptions to reproduce quantum mechanical correlations while maintaining local realism. Through detailed mathematical analysis, we show that T0 can violate Bell inequalities via spatially extended energy field correlations that couple measurement apparatus orientations with quantum system properties. While this approach is mathematically consistent and offers testable predictions, it comes at the philosophical cost of restricting measurement freedom and introducing controversial superdeterministic elements. The analysis reveals both the theoretical elegance and the conceptual challenges of attempting to restore deterministic local realism in quantum mechanics.
	\end{abstract}
	
	\tableofcontents
	\newpage
	
	\section{Introduction: The Fundamental Challenge}
	
	\subsection{The No-Go Theorem Landscape}
	
	Quantum mechanics faces several fundamental no-go theorems that constrain possible interpretations:
	
	\begin{enumerate}
		\item \textbf{Bell's Theorem (1964)}: No local realistic theory can reproduce all quantum mechanical predictions
		\item \textbf{Kochen-Specker Theorem (1967)}: Quantum observables cannot have simultaneous definite values
		\item \textbf{PBR Theorem (2012)}: Quantum states are ontological, not merely epistemological
		\item \textbf{Hardy's Theorem (1993)}: Quantum nonlocality without inequalities
	\end{enumerate}
	
	\subsection{The T0 Challenge}
	
	The T0-energy field formulation makes apparently contradictory claims:
	
	\begin{tcolorbox}[colback=red!5!white,colframe=red!75!black,title=T0 Claims vs No-Go Theorems]
		\textbf{T0 Claims}:
		\begin{itemize}
			\item Local deterministic dynamics: $\partial^2 \Efield = 0$
			\item Realistic energy fields: $\Efield(x,t)$ exist independently
			\item Perfect QM reproduction: Identical predictions for all experiments
		\end{itemize}
		
		\textbf{No-Go Theorems}: Such a theory is impossible!
		
		\textbf{Question}: How does T0 circumvent these fundamental limitations?
	\end{tcolorbox}
	
	This document provides a comprehensive analysis of T0's strategy for addressing no-go theorems and evaluates its theoretical viability.
	
	\section{Bell's Theorem: Mathematical Foundation}
	
	\subsection{CHSH Inequality}
	
	The Clauser-Horne-Shimony-Holt (CHSH) form of Bell's inequality provides the most general test:
	
	\begin{equation}
		S = E(a,b) - E(a,b') + E(a',b) + E(a',b') \leq 2
		\label{eq:chsh_inequality}
	\end{equation}
	
	where $E(a,b)$ represents the correlation between measurements in directions $a$ and $b$.
	
	\subsection{Bell's Theorem Assumptions}
	
	Bell's proof relies on three key assumptions:
	
	\begin{enumerate}
		\item \textbf{Locality}: No superluminal influences
		\item \textbf{Realism}: Properties exist before measurement
		\item \textbf{Measurement freedom}: Free choice of measurement settings
	\end{enumerate}
	
	\textbf{Bell's conclusion}: Any theory satisfying all three assumptions must satisfy $|S| \leq 2$.
	
	\subsection{Quantum Mechanical Violation}
	
	For the Bell state $|\Psi^-\rangle = \frac{1}{\sqrt{2}}(|\uparrow\downarrow\rangle - |\downarrow\uparrow\rangle)$:
	
	\begin{equation}
		E_{QM}(a,b) = -\cos(\theta_{ab})
	\end{equation}
	
	where $\theta_{ab}$ is the angle between measurement directions.
	
	\textbf{Optimal measurement angles}: $a = 0°$, $a' = 45°$, $b = 22.5°$, $b' = 67.5°$
	
	\begin{align}
		E(a,b) &= -\cos(22.5°) = -0.9239 \\
		E(a,b') &= -\cos(67.5°) = -0.3827 \\
		E(a',b) &= -\cos(22.5°) = -0.9239 \\
		E(a',b') &= -\cos(22.5°) = -0.9239
	\end{align}
	
	\begin{equation}
		S_{QM} = -0.9239 - (-0.3827) + (-0.9239) + (-0.9239) = -2.389
	\end{equation}
	
	\textbf{Bell violation}: $|S_{QM}| = 2.389 > 2$
	
	\section{T0 Response to Bell's Theorem}
	
	\subsection{T0 Bell State Representation}
	
	In T0 formulation, the Bell state becomes:
	
	\begin{equation}
		\text{Standard: } |\Psi^-\rangle = \frac{1}{\sqrt{2}}(|\uparrow\downarrow\rangle - |\downarrow\uparrow\rangle)
	\end{equation}
	
	\begin{equation}
		\text{T0: } \{\Efield_{\uparrow\downarrow} = 0.5, \Efield_{\downarrow\uparrow} = -0.5, \Efield_{\uparrow\uparrow} = 0, \Efield_{\downarrow\downarrow} = 0\}
	\end{equation}
	
	\subsection{T0 Correlation Formula}
	
	T0 correlations arise from energy field interactions:
	
	\begin{equation}
		E_{T0}(a,b) = \frac{\langle \Efield_1(a) \cdot \Efield_2(b) \rangle}{\langle |\Efield_1| \rangle \langle |\Efield_2| \rangle}
	\end{equation}
	
	With $\xipar$-parameter corrections:
	
	\begin{equation}
		E_{T0}(a,b) = E_{QM}(a,b) \times (1 + \xipar \cdot f_{corr}(a,b))
	\end{equation}
	
	where $\xipar = 1.33 \times 10^{-4}$ and $f_{corr}$ represents correlation structure.
	
	\subsection{T0 Extended Bell Inequality}
	
	The original T0 documents propose a modified Bell inequality:
	
	\begin{equation}
		|E(a,b) - E(a,c)| + |E(a',b) + E(a',c)| \leq 2 + \varepsilon_{T0}
	\end{equation}
	
	where the T0 correction term is:
	
	\begin{equation}
		\varepsilon_{T0} = \xipar \cdot \left|\frac{E_1 - E_2}{E_1 + E_2}\right| \cdot \frac{2G\langle E \rangle}{r_{12}}
	\end{equation}
	
	\textbf{Numerical evaluation}: For typical atomic systems with $r_{12} \sim 1$ m, $\langle E \rangle \sim 1$ eV:
	
	\begin{equation}
		\varepsilon_{T0} \approx 1.33 \times 10^{-4} \times 1 \times \frac{2 \times 6.7 \times 10^{-11} \times 1.6 \times 10^{-19}}{1} \approx 2.8 \times 10^{-34}
	\end{equation}
	
	\textbf{Problem}: This correction is experimentally unmeasurable!
	
	\textbf{Alternative interpretation}: Direct $\xipar$-corrections without gravitational suppression:
	
	\begin{equation}
		\varepsilon_{T0,direct} = \xipar = 1.33 \times 10^{-4}
	\end{equation}
	
	This would be measurable in precision Bell tests, predicting:
	
	\begin{equation}
		|S_{T0}| = 2.389 + 1.33 \times 10^{-4} = 2.389133
	\end{equation}
	
	\textbf{Testable T0 prediction}: Bell violation exceeds quantum mechanical limit by 133 ppm!
	
	\begin{tcolorbox}[colback=yellow!5!white,colframe=orange!75!black,title=Critical Question]
		\textbf{How can a local deterministic theory violate Bell's inequality?}
		
		This apparent contradiction requires careful analysis of Bell's theorem assumptions.
	\end{tcolorbox}
	
	\section{T0's Circumvention Strategy: Violation of Measurement Freedom}
	
	\subsection{The Key Insight: Spatially Extended Energy Fields}
	
	T0's solution relies on a subtle violation of Bell's measurement freedom assumption:
	
	\begin{equation}
		\Efield(x,t) = \Efield_{intrinsic}(x,t) + \Efield_{apparatus}(x,t)
	\end{equation}
	
	\textbf{Physical picture}:
	\begin{itemize}
		\item Energy fields $\Efield(x,t)$ are spatially extended
		\item Measurement apparatus at location A influences $\Efield(x,t)$ throughout space
		\item This creates correlations between apparatus settings and distant measurements
		\item The correlation is local in field dynamics but appears nonlocal in outcomes
	\end{itemize}
	
	\subsection{Mathematical Formulation}
	
	The T0 correlation includes apparatus-dependent terms:
	
	\begin{equation}
		E_{T0}(a,b) = E_{intrinsic}(a,b) + E_{apparatus}(a,b) + E_{cross}(a,b)
	\end{equation}
	
	where:
	\begin{itemize}
		\item $E_{intrinsic}$: Direct particle-particle correlation
		\item $E_{apparatus}$: Apparatus-particle correlations
		\item $E_{cross}$: Cross-correlations between apparatus and particles
	\end{itemize}
	
	\subsection{Superdeterminism}
	
	T0 implements a form of "superdeterminism":
	
	\begin{tcolorbox}[colback=blue!5!white,colframe=blue!75!black,title=T0 Superdeterminism]
		\textbf{Definition}: The choice of measurement settings $a$ and $b$ is not truly free but correlated with the quantum system's initial conditions through energy field dynamics.
		
		\textbf{Mechanism}: Spatially extended energy fields create subtle correlations between:
		\begin{itemize}
			\item Experimenter's "choice" of measurement direction
			\item Quantum system properties
			\item Measurement apparatus configuration
		\end{itemize}
		
		\textbf{Result}: Bell's measurement freedom assumption is violated
	\end{tcolorbox}
	
	\subsection{Experimental Consequences}
	
	T0 superdeterminism makes specific predictions:
	
	\begin{enumerate}
		\item \textbf{Measurement direction correlations}: Statistical bias in "random" measurement choices
		\item \textbf{Spatial energy structure}: Extended field patterns around measurement apparatus
		\item \textbf{$\xipar$-corrections}: $133$ ppm systematic deviations in correlations
		\item \textbf{Apparatus-dependent effects}: Measurement outcomes depend on apparatus history
	\end{enumerate}
	
	\section{Kochen-Specker Theorem}
	
	\subsection{The Contextuality Problem}
	
	The Kochen-Specker theorem states that quantum observables cannot have simultaneous definite values independent of measurement context.
	
	\textbf{Classic example}: Spin measurements in orthogonal directions
	\begin{align}
		\sigma_x^2 + \sigma_y^2 + \sigma_z^2 &= 3 \quad \text{(if all simultaneously definite)} \\
		\langle\sigma_x^2\rangle + \langle\sigma_y^2\rangle + \langle\sigma_z^2\rangle &= 3 \quad \text{(quantum prediction)} \\
		\text{But individual values are context-dependent!}
	\end{align}
	
	\subsection{T0 Response: Energy Field Contextuality}
	
	T0 addresses contextuality through measurement-induced field modifications:
	
	\begin{equation}
		\Efield_{measured,x} = \Efield_{intrinsic,x} + \Delta\Efield_x(\text{apparatus state})
	\end{equation}
	
	\textbf{Key insight}: 
	\begin{itemize}
		\item All energy field components $\Efield_x$, $\Efield_y$, $\Efield_z$ exist simultaneously
		\item Measurement in direction $x$ modifies $\Efield_y$ and $\Efield_z$ through apparatus interaction
		\item Context dependence arises from measurement-apparatus-field coupling
		\item "Hidden variables" are the complete energy field configuration $\{\Efield(x,t)\}$
	\end{itemize}
	
	\subsection{Mathematical Framework}
	
	\begin{equation}
		\frac{\partial \Efield_i}{\partial t} = f_i(\{\Efield_j\}, \{\text{apparatus}_k\})
	\end{equation}
	
	The evolution of each field component depends on:
	\begin{itemize}
		\item All other field components (quantum correlations)
		\item All measurement apparatus configurations (contextuality)
		\item Spatial field structure (nonlocal correlations)
	\end{itemize}
	
	\section{Other No-Go Theorems}
	
	\subsection{PBR Theorem (Pusey-Barrett-Rudolph)}
	
	\textbf{PBR claim}: Quantum states must be ontologically real, not merely epistemological.
	
	\textbf{T0 response}: Perfect compatibility
	\begin{itemize}
		\item Energy fields $\Efield(x,t)$ are ontologically real
		\item Quantum states correspond to energy field configurations
		\item No epistemological interpretation needed
	\end{itemize}
	
	\subsection{Hardy's Theorem}
	
	\textbf{Hardy's claim}: Quantum nonlocality can be demonstrated without inequalities.
	
	\textbf{T0 response}: Energy field correlations can reproduce Hardy's paradoxical situations through spatially extended field dynamics.
	
	\subsection{GHZ Theorem}
	
	\textbf{GHZ claim}: Three-particle correlations provide perfect demonstration of quantum nonlocality.
	
	\textbf{T0 response}: Three-particle energy field configurations with extended correlation structures.
	
	\section{Critical Evaluation}
	
	\subsection{Strengths of T0 Approach}
	
	\begin{enumerate}
		\item \textbf{Distinct predictions}: Makes **different** testable predictions from standard QM
		\item \textbf{Concrete mechanisms}: Provides specific energy field dynamics
		\item \textbf{Multiple testable signatures}: 
		\begin{itemize}
			\item Enhanced Bell violation (133 ppm excess)
			\item Perfect quantum algorithm repeatability  
			\item Spatial energy field structure
			\item Deterministic single-measurement predictions
		\end{itemize}
		\item \textbf{Theoretical elegance}: Unified framework for all quantum phenomena
		\item \textbf{Interpretational clarity}: Eliminates measurement problem and wave function collapse
		\item \textbf{Quantum computing advantages}: Deterministic algorithms with perfect predictability
		\item \textbf{Falsifiability}: Clear experimental criteria for disproof
	\end{enumerate}
	
	\subsection{Weaknesses and Criticisms}
	
	\begin{enumerate}
		\item \textbf{Superdeterminism controversy}: Most physicists consider it implausible
		\item \textbf{Measurement freedom violation}: Challenges fundamental experimental methodology
		\item \textbf{Mathematical development}: Energy field dynamics not fully developed
		\item \textbf{Relativistic compatibility}: Unclear how T0 integrates with special relativity
		\item \textbf{High precision requirements}: 133 ppm measurements technically challenging
		\item \textbf{Falsification risk}: **T0 predictions could be experimentally disproven**
		\item \textbf{Philosophical cost}: Eliminates measurement freedom and true randomness
	\end{enumerate}
	
	\subsection{Experimental Tests}
	
	\begin{table}[htbp]
		\centering
		\begin{tabular}{lcc}
			\toprule
			\textbf{Test} & \textbf{Standard QM} & \textbf{T0 Prediction} \\
			\midrule
			Bell correlations & Violate inequalities & Enhanced violation + $\xipar$ \\
			Extended Bell inequality & $|S| \leq 2$ & $|S| \leq 2 + 1.33 \times 10^{-4}$ \\
			Algorithm repeatability & Statistical variation & Perfect repeatability \\
			Single measurements & Probabilistic outcomes & Deterministic predictions \\
			Spatial structure & Point-like & Extended E(x,t) patterns \\
			Measurement randomness & True randomness & Subtle correlations \\
			Spatial field structure & Point-like & Extended patterns \\
			Apparatus dependence & Minimal & Systematic effects \\
			Superdeterminism & No evidence & Statistical biases \\
			\bottomrule
		\end{tabular}
		\caption{Experimental discrimination between standard QM and T0}
	\end{table}
	
	\section{Philosophical Implications}
	
	\subsection{The Price of Local Realism}
	
	T0's restoration of local realism comes at significant philosophical cost:
	
	\begin{tcolorbox}[colback=purple!5!white,colframe=purple!75!black,title=Philosophical Trade-offs]
		\textbf{Gained}:
		\begin{itemize}
			\item Local realism restored
			\item Deterministic physics
			\item Clear ontology (energy fields)
			\item No measurement problem
		\end{itemize}
		
		\textbf{Lost}:
		\begin{itemize}
			\item Traditional measurement interpretation
			\item Apparent fundamental randomness
			\item Simple non-contextual locality
			\item Some current experimental methodologies
		\end{itemize}
	\end{tcolorbox}
	
	\subsection{Superdeterminism and Free Will}
	
	T0's superdeterminism has significant implications:
	
	\begin{itemize}
		\item Experimental choices show subtle correlations with quantum systems
		\item Initial conditions of universe influence all measurement outcomes
		\item "Random" number generators exhibit systematic patterns
		\item Bell test "loopholes" become fundamental features rather than flaws
	\end{itemize}
	
	\section{Conclusion: A Viable Alternative?}
	
	\subsection{Summary of Analysis}
	
	This comprehensive analysis reveals that T0 theory offers a sophisticated strategy for circumventing no-go theorems while making **distinct, testable predictions** that differ from standard quantum mechanics:
	
	\begin{enumerate}
		\item \textbf{Bell's Theorem}: Circumvented through violation of measurement freedom via spatially extended energy field correlations, with **measurable enhanced Bell violation**
		\item \textbf{Kochen-Specker}: Addressed through measurement-apparatus-field coupling creating contextuality
		\item \textbf{Other theorems}: Generally compatible with T0's ontological energy field framework
		\item \textbf{Quantum Computing}: **Perfect algorithmic equivalence** with deterministic advantages (Deutsch, Bell states, Grover, Shor)
	\end{enumerate}
	
	\subsection{Theoretical Viability}
	
	\textbf{T0 is theoretically viable} as a **genuine alternative** (not reinterpretation) to standard quantum mechanics, offering:
	
	\textbf{Advantages}:
	\begin{itemize}
		\item **Distinct testable predictions** differing from QM
		\item **Deterministic quantum computing** with perfect algorithmic equivalence
		\item **Enhanced Bell violation** exceeding quantum limits by 133 ppm
		\item **Perfect repeatability** in quantum measurements
		\item **Spatial energy field structure** extending beyond point particles
		\item **Single-measurement predictability** for quantum algorithms
	\end{itemize}
	
	\textbf{Requirements}:
	\begin{itemize}
		\item Acceptance of superdeterminism
		\item Violation of measurement freedom
		\item Complex energy field dynamics
		\item **Falsifiability risk**: negative precision tests would disprove T0
	\end{itemize}
	
	\subsection{Experimental Resolution}
	
	The ultimate test of T0 vs standard QM lies in **precision experiments** with **clear discrimination criteria**:
	
	\begin{enumerate}
		\item \textbf{Enhanced Bell violation tests}: Search for |S| > 2.389 (QM limit)
		\begin{itemize}
			\item **Target precision**: 133 ppm or better
			\item **T0 prediction**: |S| = 2.389133 ± measurement error
			\item **Decisive test**: Any excess violation supports T0
		\end{itemize}
		
		\item \textbf{Quantum algorithm repeatability}: 1000× identical algorithm execution
		\begin{itemize}
			\item **QM expectation**: Statistical variation within error bars
			\item **T0 prediction**: Perfect repeatability (zero variance)
			\item **Algorithms**: Deutsch, Grover, Bell states, Shor
		\end{itemize}
		
		\item \textbf{Spatial energy field mapping}: Detect extended field structures
		\begin{itemize}
			\item **QM expectation**: Point-like measurement events
			\item **T0 prediction**: Spatially extended energy patterns E(x,t)
			\item **Technology**: High-resolution quantum interferometry
		\end{itemize}
		
		\item \textbf{Superdeterminism signatures}: Search for measurement choice correlations
		\begin{itemize}
			\item **QM expectation**: True randomness in measurement settings
			\item **T0 prediction**: Subtle statistical biases in "random" choices
			\item **Challenge**: Requires careful statistical analysis
		\end{itemize}
	\end{enumerate}
	
	\begin{tcolorbox}[colback=green!5!white,colframe=green!75!black,title=Final Assessment]
		\textbf{T0 theory provides a mathematically consistent, experimentally testable alternative to standard quantum mechanics that circumvents no-go theorems through sophisticated superdeterministic mechanisms.} 
		
		\textbf{Key insight}: T0 is not merely a reinterpretation but makes distinct, falsifiable predictions that can definitively distinguish it from standard QM through precision experiments.
		
		\textbf{Critical tests}: Enhanced Bell violation (133 ppm), perfect quantum algorithm repeatability, and spatial energy field mapping provide clear experimental discrimination criteria.
		
		\textbf{Verdict}: The ultimate decision between T0 and standard QM rests on experimental evidence, not theoretical preference.
	\end{tcolorbox}
	
	The T0 approach demonstrates that local realistic alternatives to quantum mechanics are theoretically possible and experimentally distinguishable. While requiring controversial superdeterministic assumptions, T0 offers concrete predictions that can definitively resolve the debate between deterministic and probabilistic quantum mechanics.
	
	\begin{thebibliography}{99}
		\bibitem{bell1964}
		Bell, J. S. (1964). On the Einstein Podolsky Rosen paradox. \textit{Physics Physique Fizika}, 1(3), 195--200.
		
		\bibitem{kochen_specker1967}
		Kochen, S. and Specker, E. P. (1967). The problem of hidden variables in quantum mechanics. \textit{Journal of Mathematics and Mechanics}, 17(1), 59--87.
		
		\bibitem{clauser_horne1974}
		Clauser, J. F. and Horne, M. A. (1974). Experimental consequences of objective local theories. \textit{Physical Review D}, 10(2), 526--535.
		
		\bibitem{aspect1982}
		Aspect, A., Dalibard, J., and Roger, G. (1982). Experimental test of Bell's inequalities using time-varying analyzers. \textit{Physical Review Letters}, 49(25), 1804--1807.
		
		\bibitem{pusey_barrett_rudolph2012}
		Pusey, M. F., Barrett, J., and Rudolph, T. (2012). On the reality of the quantum state. \textit{Nature Physics}, 8(6), 475--478.
		
		\bibitem{hardy1993}
		Hardy, L. (1993). Nonlocality for two particles without inequalities for almost all entangled states. \textit{Physical Review Letters}, 71(11), 1665--1668.
		
		\bibitem{greenberger_horne_zeilinger1989}
		Greenberger, D. M., Horne, M. A., and Zeilinger, A. (1989). Going beyond Bell's theorem. \textit{Bell's Theorem, Quantum Theory and Conceptions of the Universe}, 69--72.
		
		\bibitem{superdeterminism_review}
		Brans, C. H. (1988). Bell's theorem does not eliminate fully causal hidden variables. \textit{International Journal of Theoretical Physics}, 27(2), 219--226.
		
		\bibitem{t_hooft_deterministic}
		't Hooft, G. (2016). \textit{The Cellular Automaton Interpretation of Quantum Mechanics}. Springer.
		
		\bibitem{palmer_superdeterminism}
		Palmer, T. N. (2020). The invariant set postulate: A new geometric framework for the foundations of quantum theory and the role played by gravity. \textit{Proceedings of the Royal Society A}, 476(2243), 20200319.
		
		\bibitem{t0_deterministic_qm}
		T0 Theory Documentation. \textit{Deterministic Quantum Mechanics via T0-Energy Field Formulation}.
		
		\bibitem{t0_lagrangian}
		T0 Theory Documentation. \textit{Simple Lagrangian Revolution: From Standard Model Complexity to T0 Elegance}.
		
		\bibitem{bell_test_loopholes}
		Larsson, J. Å. (2014). Loopholes in Bell inequality tests of local realism. \textit{Journal of Physics A: Mathematical and Theoretical}, 47(42), 424003.
		
		\bibitem{freedom_of_choice}
		Scheidl, T. et al. (2010). Violation of local realism with freedom of choice. \textit{Proceedings of the National Academy of Sciences}, 107(46), 19708--19713.
	\end{thebibliography}
	
\end{document}