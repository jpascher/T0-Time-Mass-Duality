\documentclass[12pt,a4paper]{article}
\usepackage[utf8]{inputenc}
\usepackage[T1]{fontenc}
\usepackage[english]{babel}
\usepackage{lmodern}
\usepackage{csquotes}
\usepackage{amsmath}
\usepackage{amssymb}
\usepackage{physics}
\usepackage{geometry}
\usepackage{tocloft}
\usepackage{xcolor}
\usepackage{graphicx,tikz,pgfplots}
\pgfplotsset{compat=1.18}
\usepackage{booktabs}
\usepackage{siunitx}
\usepackage{amsthm}
\usepackage[colorlinks=true, linkcolor=blue, citecolor=blue, urlcolor=blue]{hyperref}
\usepackage{cleveref}
\usepackage{fancyhdr}
\usepackage{tcolorbox}
\usepackage{mathtools}

\geometry{a4paper, margin=2cm}

% Headers and Footers
\pagestyle{fancy}
\fancyhf{}
\fancyhead[L]{Johann Pascher}
\fancyhead[R]{Dynamic Mass of Photons in T0 Model - Updated Framework}
\fancyfoot[C]{\thepage}
\renewcommand{\headrulewidth}{0.4pt}
\renewcommand{\footrulewidth}{0.4pt}

% Table of Contents Styling
\renewcommand{\cftsecfont}{\color{blue}}
\renewcommand{\cftsubsecfont}{\color{blue}}
\renewcommand{\cftsecpagefont}{\color{blue}}
\renewcommand{\cftsubsecpagefont}{\color{blue}}
\setlength{\cftsecindent}{1cm}
\setlength{\cftsubsecindent}{2cm}

% Custom commands (consistent with T0 model reference)
\newcommand{\Tfield}{T(x,t)}
\newcommand{\betaT}{\beta_{\text{T}}}
\newcommand{\alphaEM}{\alpha_{\text{EM}}}
\newcommand{\alphaW}{\alpha_{\text{W}}}
\newcommand{\Mpl}{M_{\text{Pl}}}
\newcommand{\Tzerot}{T_0(\Tfield)}
\newcommand{\Tzero}{T_0}
\newcommand{\vecx}{\vec{x}}
\newcommand{\gammaf}{\gamma_{\text{Lorentz}}}
\newcommand{\DhiggsT}{\Tfield (\partial_\mu + ig A_\mu) \Phi + \Phi \partial_\mu \Tfield}
\newcommand{\xipar}{\xi}
\newcommand{\lP}{\ell_{\text{P}}}

\newtheorem{theorem}{Theorem}[section]
\newtheorem{proposition}[theorem]{Proposition}
\newtheorem{definition}[theorem]{Definition}

\title{Dynamic Mass of Photons and Its Implications for Nonlocality \\ in the T0 Model: Updated Framework with \\ Complete Geometric Foundations}
\author{Johann Pascher}
\date{\today}

\begin{document}
	
	\maketitle
	
	\begin{abstract}
		This updated work examines the implications of assigning a dynamic, frequency-dependent effective mass to photons within the comprehensive framework of the T0 model, building upon the complete field-theoretic derivation and natural units system where $\hbar = c = \alpha_{\text{EM}} = \beta_{\text{T}} = 1$. The theory establishes the fundamental relationship $\Tfield = \frac{1}{\max(m, \omega)}$ with dimension $[E^{-1}]$, providing a unified treatment of massive particles and photons through the three fundamental field geometries. The dynamic photon mass $m_\gamma = \omega$ introduces energy-dependent nonlocality effects, with testable predictions. All formulations maintain strict dimensional consistency with the fixed T0 parameters $\beta = 2Gm/r$, $\xi = 2\sqrt{G} \cdot m$, and the cosmic screening factor $\xi_{\text{eff}} = \xi/2$ for infinite fields.
	\end{abstract}
	
	\tableofcontents
	\newpage
	
	\section{Introduction: T0 Model Foundation for Photon Dynamics}
	
	This updated analysis builds upon the comprehensive T0 model framework established in the field-theoretic derivation, incorporating the complete geometric foundations and natural units system. The dynamic effective mass concept for photons emerges naturally from the T0 model's fundamental time-mass duality principle.
	
	\subsection{Fundamental T0 Model Framework}
	
	The T0 model is based on the intrinsic time field definition:
	
	\begin{equation}
		\boxed{\Tfield = \frac{1}{\max(m(\vec{x},t), \omega)}}
		\label{eq:intrinsic_time_field}
	\end{equation}
	
	\textbf{Dimensional verification}: $[\Tfield] = [1/E] = [E^{-1}]$ in natural units \checkmark
	
	This field satisfies the fundamental field equation:
	\begin{equation}
		\nabla^2 m(\vec{x},t) = 4\pi G \rho(\vec{x},t) \cdot m(\vec{x},t)
		\label{eq:field_equation}
	\end{equation}
	
	From this foundation emerge the key parameters:
	
	\begin{tcolorbox}[colback=blue!5!white,colframe=blue!75!black,title=T0 Model Parameters for Photon Analysis]
		\begin{align}
			\beta &= \frac{2Gm}{r} \quad [1] \text{ (dimensionless)} \\
			\xi &= 2\sqrt{G} \cdot m \quad [1] \text{ (dimensionless)} \\
			\beta_T &= 1 \quad [1] \text{ (natural units)} \\
			\alpha_{\text{EM}} &= 1 \quad [1] \text{ (natural units)}
		\end{align}
	\end{tcolorbox}
	
	\subsection{Photon Integration in Time-Mass Duality}
	
	For photons, the T0 model assigns an effective mass:
	\begin{equation}
		m_\gamma = \omega
		\label{eq:photon_effective_mass}
	\end{equation}
	
	\textbf{Dimensional verification}: $[m_\gamma] = [\omega] = [E]$ in natural units \checkmark
	
	This gives the photon's intrinsic time field:
	\begin{equation}
		\Tfield_\gamma = \frac{1}{\omega}
		\label{eq:photon_time_field}
	\end{equation}
\begin{tcolorbox}[colback=yellow!5!white,colframe=orange!75!black,title=Praktische Vereinfachung]
	\textbf{Vereinfachung:} Da alle Messungen in unserem endlichen, beobachtbaren Universum lokal erfolgen, wird nur die \textbf{lokalisierte Feldgeometrie} verwendet:
	
	$\xi = 2\sqrt{G} \cdot m$ und $\beta = \frac{2Gm}{r}$ für alle Anwendungen.
	
	Der kosmische Abschirmfaktor $\xi_{\text{eff}} = \xi/2$ entfällt.
\end{tcolorbox}	
	\textbf{Physical interpretation}: Higher-energy photons have shorter intrinsic time scales, creating energy-dependent temporal dynamics.
	
	\section{Energy-Dependent Nonlocality and Quantum Correlations}
	
	\subsection{Entangled Photon Systems}
	
	For entangled photons with energies $\omega_1$ and $\omega_2$, the time field difference is:
	\begin{equation}
		\Delta T_\gamma = \left|\frac{1}{\omega_1} - \frac{1}{\omega_2}\right|
		\label{eq:time_field_difference}
	\end{equation}
	
	\textbf{Physical consequence}: Quantum correlations experience energy-dependent delays.
	
	\subsection{Modified Bell Inequality}
	
	The energy-dependent time fields lead to a modified Bell inequality:
	\begin{equation}
		|E(a,b) - E(a,c)| + |E(a',b) + E(a',c)| \leq 2 + \epsilon(\omega_1, \omega_2)
		\label{eq:modified_bell_inequality}
	\end{equation}
	
	where:
	\begin{equation}
		\epsilon(\omega_1, \omega_2) = \alpha_{\text{corr}} \left|\frac{1}{\omega_1} - \frac{1}{\omega_2}\right| \frac{2G\langle m \rangle}{r}
		\label{eq:bell_correction}
	\end{equation}
	
	with $\alpha_{\text{corr}}$ being a correlation coupling constant and $\langle m \rangle$ the average mass in the experimental setup.
	

	\section{Experimental Predictions and Tests}
	
	\subsection{High-Precision Quantum Optics Tests}
	
	\subsubsection{Energy-Dependent Bell Tests}
	
	Predicted time delay between entangled photons:
	\begin{equation}
		\Delta t_{\text{corr}} = \frac{G\langle m \rangle}{r} \left|\frac{1}{\omega_1} - \frac{1}{\omega_2}\right|
		\label{eq:correlation_time_delay}
	\end{equation}
	
	For laboratory conditions with $\langle m \rangle \sim 10^{-3}$ kg, $r \sim 10$ m, and $\omega_1,\omega_2 \sim 1$ eV:
	\begin{equation}
		\Delta t_{\text{corr}} \sim 10^{-21} \text{ s}
		\label{eq:laboratory_delay}
	\end{equation}
	

	\section{Dimensional Consistency Verification}
	
	\begin{table}[htbp]
		\centering
		\begin{tabular}{lccl}
			\toprule
			\textbf{Equation} & \textbf{Left Side} & \textbf{Right Side} & \textbf{Status} \\
			\midrule
			Photon effective mass & $[m_\gamma] = [E]$ & $[\omega] = [E]$ & \checkmark \\
			Photon time field & $[T_\gamma] = [E^{-1}]$ & $[1/\omega] = [E^{-1}]$ & \checkmark \\
			Energy loss rate & $[d\omega/dr] = [E^2]$ & $[g_T \omega^2 2G/r^2] = [E^2]$ & \checkmark \\
			Time field difference & $[\Delta T_\gamma] = [E^{-1}]$ & $[|1/\omega_1 - 1/\omega_2|] = [E^{-1}]$ & \checkmark \\
			Bell correction & $[\epsilon] = [1]$ & $[\alpha_{\text{corr}} \Delta T_\gamma \beta] = [1]$ & \checkmark \\
			\bottomrule
		\end{tabular}
		\caption{Dimensional consistency verification for photon dynamics in T0 model}
	\end{table}
	
	\section{Conclusions}
	
	\subsection{Summary of Key Results}
	
	This updated analysis demonstrates that the dynamic photon mass concept integrates seamlessly into the comprehensive T0 model framework:
	
	\begin{enumerate}
		\item \textbf{Unified treatment}: Photons and massive particles follow the same fundamental relationship $T = 1/\max(m,\omega)$
		\item \textbf{Energy-dependent effects}: Photon dynamics depend on frequency through the intrinsic time field
		\item \textbf{Modified nonlocality}: Quantum correlations experience energy-dependent delays
		\item \textbf{Testable predictions}: Specific experimental signatures distinguish T0 from standard theory
		\item \textbf{Dimensional consistency}: All equations verified in natural units framework
		\item \textbf{Parameter-free theory}: All effects determined by fundamental T0 parameters
	\end{enumerate}
	

\end{document}