=== T0_Kosmologie_De.tex.preamble ===

\documentclass[12pt,a4paper]{article}
\usepackage[utf8]{inputenc}
\usepackage[T1]{fontenc}
\usepackage[ngerman]{babel}
\usepackage{lmodern}
\usepackage{amsmath,amssymb,amsthm}
\usepackage{geometry}
\usepackage{booktabs}
\usepackage{array}
\usepackage{xcolor}
\usepackage{tcolorbox}
\usepackage{fancyhdr}
\usepackage{hyperref}
\usepackage{physics}
\usepackage{siunitx}
\usepackage{longtable}

\definecolor{deepblue}{RGB}{0,0,127}
\definecolor{deepred}{RGB}{191,0,0}
\definecolor{deepgreen}{RGB}{0,127,0}

\geometry{a4paper, margin=2.5cm}
\setlength{\headheight}{15pt}

% Header- und Footer-Konfiguration
\pagestyle{fancy}
\fancyhf{}
\fancyhead[L]{\textsc{T0-Theorie: Kosmologie}}
\fancyhead[R]{\textsc{J. Pascher}}
\fancyfoot[C]{\thepage}
\renewcommand{\headrulewidth}{0.4pt}
\renewcommand{\footrulewidth}{0.4pt}

% Hyperref-Einstellungen
\hypersetup{
	colorlinks=true,
	linkcolor=blue,
	citecolor=blue,
	urlcolor=blue,
	pdftitle={T0-Theorie: Kosmologie},
	pdfauthor={Johann Pascher},
	pdfsubject={T0-Theorie, Statisches Universum, CMB, Casimir-Effekt}
}

% Benutzerdefinierte Befehle
\newcommand{\xipar}{\xi}
\newcommand{\Lxi}{L_\xi}
\newcommand{\Exi}{E_\xi}
\newcommand{\rhoCMB}{\rho_{\text{CMB}}}
\newcommand{\rhoCasimir}{\rho_{\text{Casimir}}}

% Umgebung für Schlüsselergebnisse
\newtcolorbox{keyresult}{colback=blue!5, colframe=blue!75!black, title=Schlüsselergebnis}
\newtcolorbox{warning}{colback=red!5, colframe=red!75!black, title=Wichtiger Hinweis}
\newtcolorbox{revolutionary}{colback=green!5, colframe=green!75!black, title=Revolutionäre Erkenntnis}
\newtcolorbox{formula}{colback=yellow!5, colframe=orange!75!black, title=Zentrale Formel}
\newtcolorbox{experiment}{colback=purple!5, colframe=purple!75!black, title=Experimenteller Test}
\newtcolorbox{alternative}{colback=gray!10!white, colframe=gray!75!black, title=Alternative Interpretation}

\title{\textbf{T0-Theorie: Kosmologie}\\[0.5cm]
	\large Statisches Universum und $\xi$-Feld-Manifestationen\\[0.3cm]
	\normalsize Dokument 6 der T0-Serie}
\author{Johann Pascher\\
	Abteilung für Kommunikationstechnologie\\
	Höhere Technische Lehranstalt (HTL), Leonding, Österreich\\
	\texttt{johann.pascher@gmail.com}}
\date{\today}



==================================================

=== T0_Kosmologie_En.tex.preamble ===

\documentclass[12pt,a4paper]{article}
\usepackage[utf8]{inputenc}
\usepackage[T1]{fontenc}
\usepackage[english]{babel}
\usepackage{lmodern}
\usepackage{amsmath,amssymb,amsthm}
\usepackage{geometry}
\usepackage{booktabs}
\usepackage{array}
\usepackage{xcolor}
\usepackage{tcolorbox}
\usepackage{fancyhdr}
\usepackage{hyperref}
\usepackage{physics}
\usepackage{siunitx}
\usepackage{longtable}

\definecolor{deepblue}{RGB}{0,0,127}
\definecolor{deepred}{RGB}{191,0,0}
\definecolor{deepgreen}{RGB}{0,127,0}

\geometry{a4paper, margin=2.5cm}
\setlength{\headheight}{15pt}

% Header and Footer Configuration
\pagestyle{fancy}
\fancyhf{}
\fancyhead[L]{\textsc{T0-Theory: Cosmology}}
\fancyhead[R]{\textsc{J. Pascher}}
\fancyfoot[C]{\thepage}
\renewcommand{\headrulewidth}{0.4pt}
\renewcommand{\footrulewidth}{0.4pt}

% Hyperref Settings
\hypersetup{
	colorlinks=true,
	linkcolor=blue,
	citecolor=blue,
	urlcolor=blue,
	pdftitle={T0-Theory: Cosmology},
	pdfauthor={Johann Pascher},
	pdfsubject={T0-Theory, Static Universe, CMB, Casimir Effect}
}

% User-defined Commands
\newcommand{\xipar}{\xi}
\newcommand{\Lxi}{L_\xi}
\newcommand{\Exi}{E_\xi}
\newcommand{\rhoCMB}{\rho_{\text{CMB}}}
\newcommand{\rhoCasimir}{\rho_{\text{Casimir}}}

% Environment for Key Results
\newtcolorbox{keyresult}{colback=blue!5, colframe=blue!75!black, title=Key Result}
\newtcolorbox{warning}{colback=red!5, colframe=red!75!black, title=Important Note}
\newtcolorbox{revolutionary}{colback=green!5, colframe=green!75!black, title=Revolutionary Insight}
\newtcolorbox{formula}{colback=yellow!5, colframe=orange!75!black, title=Central Formula}
\newtcolorbox{experiment}{colback=purple!5, colframe=purple!75!black, title=Experimental Test}
\newtcolorbox{alternative}{colback=gray!10!white, colframe=gray!75!black, title=Alternative Interpretation}

\title{\textbf{T0-Theory: Cosmology}\\[0.5cm]
	\large Static Universe and $\xi$-Field Manifestations\\[0.3cm]
	\normalsize Document 6 of the T0 Series}
\author{Johann Pascher\\
	Department of Communication Technology\\
	Higher Technical College (HTL), Leonding, Austria\\
	\texttt{johann.pascher@gmail.com}}
\date{\today}



==================================================

=== T0_Modell_Uebersicht_De.tex.preamble ===

\documentclass[12pt,a4paper]{article}
\usepackage[utf8]{inputenc}
\usepackage[T1]{fontenc}
\usepackage[ngerman]{babel}
\usepackage[left=2.5cm,right=2.5cm,top=2.5cm,bottom=2.5cm]{geometry}
\usepackage{lmodern}
\usepackage{amsmath}
\usepackage{amssymb}
\usepackage{physics}
\usepackage{hyperref}
\usepackage{tcolorbox}
\usepackage{booktabs}
\usepackage{enumitem}
\usepackage[table,xcdraw]{xcolor}
\usepackage{graphicx}
\usepackage{float}
\usepackage{mathtools}
\usepackage{amsthm}
\usepackage{siunitx}
\usepackage{fancyhdr}
\usepackage{tocloft}
\usepackage{longtable}
\usepackage{array}
\usepackage{microtype}

% Kopf- und Fußzeilen
\pagestyle{fancy}
\fancyhf{}
\fancyhead[L]{T0-Modell: Vollständige Dokumentenanalyse}
\fancyhead[R]{Strukturierte Zusammenfassung}
\fancyfoot[C]{\thepage}
\renewcommand{\headrulewidth}{0.4pt}
\renewcommand{\footrulewidth}{0.4pt}

% Inhaltsverzeichnis-Stil
\renewcommand{\cfttoctitlefont}{\huge\bfseries\color{blue}}
\renewcommand{\cftsecfont}{\large\bfseries\color{blue}}
\renewcommand{\cftsubsecfont}{\color{blue}}
\renewcommand{\cftsubsubsecfont}{\color{blue}}
\renewcommand{\cftsecpagefont}{\large\bfseries\color{blue}}
\renewcommand{\cftsubsecpagefont}{\color{blue}}
\renewcommand{\cftsubsubsecpagefont}{\color{blue}}

\hypersetup{
	colorlinks=true,
	linkcolor=blue,
	citecolor=blue,
	urlcolor=blue,
	pdftitle={T0-Modell: Vollständige Dokumentenanalyse und strukturierte Zusammenfassung},
	pdfauthor={Basierend auf Johann Pascher},
	pdfsubject={T0-Theorie, Vereinheitlichung, Geometrische Konstante},
	pdfkeywords={T0-Modell, ξ-Konstante, Energiefeld, Quantenmechanik}
}

% Benutzerdefinierte Befehle
\newcommand{\xipar}{\xi}
\newcommand{\Efield}{E_{\text{Feld}}}

\title{{\Huge T0-Modell: Vollständige Dokumentenanalyse}\\
	{\LARGE und strukturierte Zusammenfassung}\\
	\vspace{1cm}
	{\Large Von der geometrischen Konstante zur Vereinheitlichung der Physik}}

\author{Basierend auf den Arbeiten von Johann Pascher\\
	HTL Leonding, Österreich\\
	\texttt{johann.pascher@gmail.com}}

\date{\today}



==================================================

=== T0_Modell_Uebersicht_En.tex.preamble ===

\documentclass[12pt,a4paper]{article}
\usepackage[utf8]{inputenc}
\usepackage[T1]{fontenc}
\usepackage[english]{babel}
\usepackage[left=2.5cm,right=2.5cm,top=2.5cm,bottom=2.5cm]{geometry}
\usepackage{lmodern}
\usepackage{amsmath}
\usepackage{amssymb}
\usepackage{physics}
\usepackage{hyperref}
\usepackage{tcolorbox}
\usepackage{booktabs}
\usepackage{enumitem}
\usepackage[table,xcdraw]{xcolor}
\usepackage{graphicx}
\usepackage{float}
\usepackage{mathtools}
\usepackage{amsthm}
\usepackage{siunitx}
\usepackage{fancyhdr}
\usepackage{tocloft}
\usepackage{longtable}
\usepackage{array}
\usepackage{microtype}

% Headers and footers
\pagestyle{fancy}
\fancyhf{}
\fancyhead[L]{T0-Model: Complete Document Analysis}
\fancyhead[R]{Structured Summary}
\fancyfoot[C]{\thepage}
\renewcommand{\headrulewidth}{0.4pt}
\renewcommand{\footrulewidth}{0.4pt}

% Table of contents style
\renewcommand{\cfttoctitlefont}{\huge\bfseries\color{blue}}
\renewcommand{\cftsecfont}{\large\bfseries\color{blue}}
\renewcommand{\cftsubsecfont}{\color{blue}}
\renewcommand{\cftsubsubsecfont}{\color{blue}}
\renewcommand{\cftsecpagefont}{\large\bfseries\color{blue}}
\renewcommand{\cftsubsecpagefont}{\color{blue}}
\renewcommand{\cftsubsubsecpagefont}{\color{blue}}

\hypersetup{
	colorlinks=true,
	linkcolor=blue,
	citecolor=blue,
	urlcolor=blue,
	pdftitle={T0-Model: Complete Document Analysis and Structured Summary},
	pdfauthor={Based on Johann Pascher},
	pdfsubject={T0-Theory, Unification, Geometric Constant},
	pdfkeywords={T0-Model, xi-constant, Energy Field, Quantum Mechanics}
}

% Custom commands for Greek letters
\newcommand{\xipar}{\xi}
\newcommand{\alphapar}{\alpha}
\newcommand{\betapar}{\beta}
\newcommand{\Efield}{E_{\text{field}}}

\title{{\Huge T0-Model: Complete Document Analysis}\\
	{\LARGE and Structured Summary}\\
	\vspace{1cm}
	{\Large From Geometric Constant to Physics Unification}}

\author{Based on the work of Johann Pascher\\
	HTL Leonding, Austria\\
	\texttt{johann.pascher@gmail.com}}

\date{\today}



==================================================

=== T0_Neutrinos_De.tex.preamble ===

\documentclass[12pt,a4paper]{article}
\usepackage[utf8]{inputenc}
\usepackage[T1]{fontenc}
\usepackage[english]{babel}
\usepackage{lmodern}
\usepackage{amsmath,amssymb,amsthm}
\usepackage{geometry}
\usepackage{booktabs}
\usepackage{array}
\usepackage{xcolor}
\usepackage{tcolorbox}
\usepackage{fancyhdr}
\usepackage{tocloft}
\usepackage{hyperref}
\usepackage{tikz}
\usepackage{physics}
\usepackage{siunitx}
\usepackage{longtable}
\usepackage{caption}

\definecolor{deepblue}{RGB}{0,0,127}
\definecolor{deepred}{RGB}{191,0,0}
\definecolor{deepgreen}{RGB}{0,127,0}

\geometry{a4paper, margin=2.5cm}

\usetikzlibrary{positioning, arrows.meta}

% Header and Footer Configuration
\pagestyle{fancy}
\fancyhf{}
\fancyhead[L]{\textsc{T0-Theory: Neutrinos}}
\fancyhead[R]{\textsc{J. Pascher}}
\fancyfoot[C]{\thepage}
\renewcommand{\headrulewidth}{0.4pt}
\renewcommand{\footrulewidth}{0.4pt}
\setlength{\headheight}{15pt}

% Table of Contents Style - Blue
\renewcommand{\cfttoctitlefont}{\huge\bfseries\color{blue}}
\renewcommand{\cftsecfont}{\color{blue}}
\renewcommand{\cftsubsecfont}{\color{blue}}
\renewcommand{\cftsecpagefont}{\color{blue}}
\renewcommand{\cftsubsecpagefont}{\color{blue}}
\setlength{\cftsecindent}{0pt}
\setlength{\cftsubsecindent}{0pt}

% Hyperref Settings
\hypersetup{
colorlinks=true,
linkcolor=blue,
citecolor=blue,
urlcolor=blue,
pdftitle={T0-Theory: Neutrinos},
pdfauthor={Johann Pascher},
pdfsubject={T0-Theory, Neutrinos, Photon Analogy, Geometric Oscillations, Koide Extension}
}

% User-Defined Commands
\newcommand{\xipar}{\xi_0}
\newcommand{\Kfrak}{K_{\text{frak}}}

% Unicode symbols
\newcommand{\checkmarkx}{\checkmark}
\newcommand{\warningx}{\color{red}\textbf{!}}

% Environment for Key Results
\newtcolorbox{keyresult}{colback=blue!5, colframe=blue!75!black, title={Key Result}}
\newtcolorbox{warning}{colback=red!5, colframe=red!75!black, title={Scientific Warning}}
\newtcolorbox{speculation}{colback=purple!5, colframe=purple!75!black, title={Speculative Hypothesis}}
\newtcolorbox{photon}{colback=yellow!5, colframe=orange!75!black, title={Photon Analogy}}
\newtcolorbox{experimental}{colback=green!5, colframe=green!75!black, title={Experimental Assessment}}
\newtcolorbox{koidebox}{colback=cyan!5, colframe=cyan!75!black, title={Koide Integration}}

\title{\textbf{T0-Theory: Neutrinos}\\[0.5cm]
\large The Photon Analogy, Geometric Oscillations, and Koide Extension\\[0.3cm]
\normalsize Document 5 of the T0 Series}
\author{Johann Pascher\\
Department of Communication Technology\\
Higher Technical College (HTL), Leonding, Austria\\
\texttt{johann.pascher@gmail.com}}
\date{\today}



==================================================

=== T0_Neutrinos_En.tex.preamble ===

\documentclass[12pt,a4paper]{article}
\usepackage[utf8]{inputenc}
\usepackage[T1]{fontenc}
\usepackage[english]{babel}
\usepackage{lmodern}
\usepackage{amsmath,amssymb,amsthm}
\usepackage{geometry}
\usepackage{booktabs}
\usepackage{array}
\usepackage{xcolor}
\usepackage{tcolorbox}
\usepackage{fancyhdr}
\usepackage{tocloft}
\usepackage{hyperref}
\usepackage{tikz}
\usepackage{physics}
\usepackage{siunitx}
\usepackage{longtable}
\usepackage{caption}

\definecolor{deepblue}{RGB}{0,0,127}
\definecolor{deepred}{RGB}{191,0,0}
\definecolor{deepgreen}{RGB}{0,127,0}

\geometry{a4paper, margin=2.5cm}

\usetikzlibrary{positioning, arrows.meta}

% Header and Footer Configuration
\pagestyle{fancy}
\fancyhf{}
\fancyhead[L]{\textsc{T0-Theory: Neutrinos}}
\fancyhead[R]{\textsc{J. Pascher}}
\fancyfoot[C]{\thepage}
\renewcommand{\headrulewidth}{0.4pt}
\renewcommand{\footrulewidth}{0.4pt}
\setlength{\headheight}{15pt}

% Table of Contents Style - Blue
\renewcommand{\cfttoctitlefont}{\huge\bfseries\color{blue}}
\renewcommand{\cftsecfont}{\color{blue}}
\renewcommand{\cftsubsecfont}{\color{blue}}
\renewcommand{\cftsecpagefont}{\color{blue}}
\renewcommand{\cftsubsecpagefont}{\color{blue}}
\setlength{\cftsecindent}{0pt}
\setlength{\cftsubsecindent}{0pt}

% Hyperref Settings
\hypersetup{
	colorlinks=true,
	linkcolor=blue,
	citecolor=blue,
	urlcolor=blue,
	pdftitle={T0-Theory: Neutrinos},
	pdfauthor={Johann Pascher},
	pdfsubject={T0-Theory, Neutrinos, Photon Analogy, Geometric Oscillations, Koide Extension}
}

% User-Defined Commands
\newcommand{\xipar}{\xi_0}
\newcommand{\Kfrak}{K_{\text{frak}}}

% Unicode symbols
\newcommand{\checkmarkx}{\checkmark}
\newcommand{\warningx}{\color{red}\textbf{!}}

% Environment for Key Results
\newtcolorbox{keyresult}{colback=blue!5, colframe=blue!75!black, title={Key Result}}
\newtcolorbox{warning}{colback=red!5, colframe=red!75!black, title={Scientific Warning}}
\newtcolorbox{speculation}{colback=purple!5, colframe=purple!75!black, title={Speculative Hypothesis}}
\newtcolorbox{photon}{colback=yellow!5, colframe=orange!75!black, title={Photon Analogy}}
\newtcolorbox{experimental}{colback=green!5, colframe=green!75!black, title={Experimental Assessment}}
\newtcolorbox{koidebox}{colback=cyan!5, colframe=cyan!75!black, title={Koide Integration}}

\title{\textbf{T0-Theory: Neutrinos}\\[0.5cm]
	\large The Photon Analogy, Geometric Oscillations, and Koide Extension\\[0.3cm]
	\normalsize Document 5 of the T0 Series}
\author{Johann Pascher\\
	Department of Communication Technology\\
	Higher Technical College (HTL), Leonding, Austria\\
	\texttt{johann.pascher@gmail.com}}
\date{\today}



==================================================

=== T0_QAT_De.tex.preamble ===

\documentclass[12pt,a4paper]{article}
\usepackage[utf8]{inputenc}
\usepackage[T1]{fontenc}
\usepackage[ngerman]{babel}
\usepackage[left=2.5cm,right=2.5cm,top=2.5cm,bottom=2.5cm]{geometry}
\usepackage{amsmath}
\usepackage{amssymb}
\usepackage{hyperref}
\usepackage{booktabs}
\usepackage{siunitx}
\usepackage[table]{xcolor}
\usepackage{physics}
\usepackage{fancyhdr}
\usepackage{tocloft}
\usepackage{breakurl}
\usepackage{float}
\usepackage{tikz}

% Kopf- und Fußzeilen
\pagestyle{fancy}
\fancyhf{}
\fancyhead[L]{Johann Pascher}
\fancyhead[R]{T0-QAT: $\xi$-Aware Quantization}
\fancyfoot[C]{\thepage}
\renewcommand{\headrulewidth}{0.4pt}
\renewcommand{\footrulewidth}{0.4pt}
\setlength{\headheight}{15pt}

% Farbdefinitionen
\definecolor{blue}{rgb}{0,0,1}
\renewcommand{\cftsecfont}{\color{blue}}
\renewcommand{\cftsecpagefont}{\color{blue}}

% Hyperlink Setup
\hypersetup{
	colorlinks=true,
	linkcolor=blue,
	citecolor=blue,
	urlcolor=blue,
	pdftitle={T0-QAT: $\xi$-Aware Quantization-Aware Training},
	pdfauthor={Johann Pascher}
}

% Titel
\title{\textbf{T0-QAT: $\xi$-Aware Quantization-Aware Training}\\[0.5cm]
	\large Experimental Validation of Noise-Resilient AI Training\\[0.3cm]
	\normalsize Based on T0 Time-Mass Duality Theory}
\author{Johann Pascher\\
	Department for Communication Technology\\
	Higher Technical College (HTL), Leonding, Austria\\
	\texttt{github.com/jpascher}}
\date{\today}



==================================================

=== T0_QAT_En.tex.preamble ===

\documentclass[12pt,a4paper]{article}
\usepackage[utf8]{inputenc}
\usepackage[T1]{fontenc}
\usepackage[english]{babel}
\usepackage[left=2.5cm,right=2.5cm,top=2.5cm,bottom=2.5cm]{geometry}
\usepackage{amsmath}
\usepackage{amssymb}
\usepackage{hyperref}
\usepackage{booktabs}
\usepackage{siunitx}
\usepackage[table]{xcolor}
\usepackage{physics}
\usepackage{fancyhdr}
\usepackage{tocloft}
\usepackage{breakurl}
\usepackage{float}
\usepackage{tikz}

% Headers and footers
\pagestyle{fancy}
\fancyhf{}
\fancyhead[L]{Johann Pascher}
\fancyhead[R]{T0-QAT: $\xi$-Aware Quantization}
\fancyfoot[C]{\thepage}
\renewcommand{\headrulewidth}{0.4pt}
\renewcommand{\footrulewidth}{0.4pt}
\setlength{\headheight}{15pt}

% Color definitions
\definecolor{blue}{rgb}{0,0,1}
\renewcommand{\cftsecfont}{\color{blue}}
\renewcommand{\cftsecpagefont}{\color{blue}}

% Hyperlink Setup
\hypersetup{
	colorlinks=true,
	linkcolor=blue,
	citecolor=blue,
	urlcolor=blue,
	pdftitle={T0-QAT: $\xi$-Aware Quantization-Aware Training},
	pdfauthor={Johann Pascher}
}

% Title
\title{\textbf{T0-QAT: $\xi$-Aware Quantization-Aware Training}\\[0.5cm]
	\large Experimental Validation of Noise-Resilient AI Training\\[0.3cm]
	\normalsize Based on T0 Time-Mass Duality Theory}
\author{Johann Pascher\\
	Department for Communication Technology\\
	Higher Technical College (HTL), Leonding, Austria\\
	\texttt{github.com/jpascher}}
\date{\today}



==================================================

=== T0_QM-QFT-RT_De.tex.preamble ===

\documentclass[12pt,a4paper]{article}
\usepackage[utf8]{inputenc}
\usepackage[T1]{fontenc}
\usepackage[ngerman]{babel}
\usepackage[left=2.5cm,right=2.5cm,top=2.5cm,bottom=2.5cm]{geometry}
\usepackage{lmodern}
\usepackage{amsmath}
\usepackage{amssymb}
\usepackage{physics}
\usepackage{hyperref}
\usepackage{tcolorbox}
\usepackage{booktabs}
\usepackage{enumitem}
\usepackage[table]{xcolor}
\usepackage{graphicx}
\usepackage{float}
\usepackage{mathtools}
\usepackage{amsthm}
\usepackage{siunitx}
\usepackage{fancyhdr}
\usepackage{longtable}

% Headers and Footers
\pagestyle{fancy}
\fancyhf{}
\fancyhead[L]{T0 Quantenfeldtheorie: Vollständige Erweiterung}
\fancyhead[R]{QFT, QM \& Quantencomputer}
\fancyfoot[C]{\thepage}
\renewcommand{\headrulewidth}{0.4pt}
\renewcommand{\footrulewidth}{0.4pt}
\setlength{\headheight}{15pt}

% Custom Commands
\newcommand{\Efield}{E_{\text{field}}}
\newcommand{\xipar}{\xi}
\newcommand{\Tfield}{T_{\text{field}}}
\newcommand{\deltaE}{\delta E}
\newcommand{\EPlanck}{E_{\text{Pl}}}
\newcommand{\LPlanck}{\ell_{\text{Pl}}}
\newcommand{\TPlanck}{t_{\text{Pl}}}

\hypersetup{
	colorlinks=true,
	linkcolor=blue,
	citecolor=blue,
	urlcolor=blue,
	pdftitle={T0 Quantenfeldtheorie: QFT, QM und Quantencomputer},
	pdfauthor={Johann Pascher},
	pdfsubject={T0-Theorie, Quantenfeldtheorie, Quantenmechanik, Quantencomputer}
}

\title{T0 Quantenfeldtheorie: Vollständige Erweiterung \\
	QFT, Quantenmechanik und Quantencomputer im T0-Framework \\
	\large Von fundamentalen Gleichungen zu technologischen Anwendungen}
\author{Johann Pascher \\
	T0-Theorie Forschungsgruppe}
\date{\today}



==================================================

=== T0_QM-QFT-RT_En.tex.preamble ===

\documentclass[12pt,a4paper]{article}
\usepackage[utf8]{inputenc}
\usepackage[T1]{fontenc}
\usepackage[english]{babel}
\usepackage[left=2.5cm,right=2.5cm,top=2.5cm,bottom=2.5cm]{geometry}
\usepackage{lmodern}
\usepackage{amsmath}
\usepackage{amssymb}
\usepackage{physics}
\usepackage{hyperref}
\usepackage{tcolorbox}
\usepackage{booktabs}
\usepackage{enumitem}
\usepackage[table]{xcolor}
\usepackage{graphicx}
\usepackage{float}
\usepackage{mathtools}
\usepackage{amsthm}
\usepackage{siunitx}
\usepackage{fancyhdr}
\usepackage{longtable}

% Headers and Footers
\pagestyle{fancy}
\fancyhf{}
\fancyhead[L]{T0 Quantum Field Theory: Complete Extension}
\fancyhead[R]{QFT, QM \& Quantum Computers}
\fancyfoot[C]{\thepage}
\renewcommand{\headrulewidth}{0.4pt}
\renewcommand{\footrulewidth}{0.4pt}
\setlength{\headheight}{15pt}

% Custom Commands
\newcommand{\Efield}{E_{\text{field}}}
\newcommand{\xipar}{\xi}
\newcommand{\Tfield}{T_{\text{field}}}
\newcommand{\deltaE}{\delta E}
\newcommand{\EPlanck}{E_{\text{Pl}}}
\newcommand{\LPlanck}{\ell_{\text{Pl}}}
\newcommand{\TPlanck}{t_{\text{Pl}}}

\hypersetup{
	colorlinks=true,
	linkcolor=blue,
	citecolor=blue,
	urlcolor=blue,
	pdftitle={T0 Quantum Field Theory: QFT, QM and Quantum Computers},
	pdfauthor={Johann Pascher},
	pdfsubject={T0-Theory, Quantum Field Theory, Quantum Mechanics, Quantum Computers}
}

\title{T0 Quantum Field Theory: Complete Extension \\
	QFT, Quantum Mechanics and Quantum Computers in the T0-Framework \\
	\large From fundamental equations to technological applications}
\author{Johann Pascher \\
	T0-Theory Research Group}
\date{\today}



==================================================

=== T0_QM-optimierung_De.tex.preamble ===

\documentclass[12pt,a4paper]{article}

% --- Grundlegende Pakete ---
\usepackage[utf8]{inputenc}
\usepackage[T1]{fontenc}
\usepackage[ngerman]{babel}
\usepackage{lmodern}
\usepackage{amsmath,amssymb,amsthm}
\usepackage{physics}
\usepackage{siunitx}
\usepackage{listings}

% --- Seitenlayout und Design ---
\usepackage[margin=2.5cm]{geometry}
\usepackage{fancyhdr}
\usepackage{hyperref}
\usepackage{graphicx}
\usepackage{booktabs}
\usepackage{enumitem}

% --- Hyperref-Konfiguration ---
\hypersetup{
	colorlinks=true,
	linkcolor=blue,
	citecolor=blue,
	urlcolor=blue,
	pdftitle={Der geometrische Formalismus der T0-Quantenmechanik und seine Anwendung auf Quantencomputer},
	pdfauthor={Johann Pascher},
	pdfsubject={T0-Theorie, Quantencomputing, Geometrische Quantenmechanik}
}

% --- Kopf- und Fußzeile ---
\pagestyle{fancy}
\fancyhf{}
\fancyhead[L]{\textsc{T0-Theorie: Geometrische Quantenmechanik}}
\fancyhead[R]{\textsc{J. Pascher}}
\fancyfoot[C]{\thepage}
\renewcommand{\headrulewidth}{0.4pt}
\setlength{\headheight}{15pt}

% --- Mathematische Befehle ---
\newcommand{\xiT}{\xi}
\newcommand{\Df}{D_f}
\newcommand{\Kfrak}{K_{\text{frak}}}
\newcommand{\phiT}{\phi}
\newcommand{\Ezero}{E_0}

% --- Titel-Informationen ---
\title{\textbf{Der geometrische Formalismus der T0-Quantenmechanik und seine Anwendung auf Quantencomputer}\\[0.5cm]
	\large Ein Rahmenwerk für eine neue Generation physik-bewusster Quantenprozessoren}
\author{Johann Pascher}
\date{2025-11-09 15:17:13 UTC}



==================================================

=== T0_QM-optimierung_En.tex.preamble ===

\documentclass[12pt,a4paper]{article}

% --- Basic Packages ---
\usepackage[utf8]{inputenc}
\usepackage[T1]{fontenc}
\usepackage[english]{babel}
\usepackage{lmodern}
\usepackage{amsmath,amssymb,amsthm}
\usepackage{physics}
\usepackage{siunitx}
\usepackage{listings}

% --- Page Layout and Design ---
\usepackage[margin=2.5cm]{geometry}
\usepackage{fancyhdr}
\usepackage{hyperref}
\usepackage{graphicx}
\usepackage{booktabs}
\usepackage{enumitem}

% --- Hyperref Configuration ---
\hypersetup{
	colorlinks=true,
	linkcolor=blue,
	citecolor=blue,
	urlcolor=blue,
	pdftitle={The Geometric Formalism of T0 Quantum Mechanics and its Application to Quantum Computing},
	pdfauthor={Johann Pascher},
	pdfsubject={T0-Theory, Quantum Computing, Geometric Quantum Mechanics}
}

% --- Header and Footer ---
\pagestyle{fancy}
\fancyhf{}
\fancyhead[L]{\textsc{T0-Theory: Geometric Quantum Mechanics}}
\fancyhead[R]{\textsc{J. Pascher}}
\fancyfoot[C]{\thepage}
\renewcommand{\headrulewidth}{0.4pt}
\setlength{\headheight}{15pt}

% --- Mathematical Commands ---
\newcommand{\xiT}{\xi}
\newcommand{\Df}{D_f}
\newcommand{\Kfrak}{K_{\text{frak}}}
\newcommand{\phiT}{\phi}
\newcommand{\Ezero}{E_0}

% --- Title Information ---
\title{\textbf{The Geometric Formalism of T0 Quantum Mechanics and its Application to Quantum Computing}\\[0.5cm]
	\large A Framework for a New Generation of Physics-Aware Quantum Processors}
\author{Johann Pascher}
\date{2025-11-09 15:13:37 UTC}



==================================================

=== T0_SI_De.tex.preamble ===

\documentclass[12pt,a4paper]{article}
\usepackage[utf8]{inputenc}
\usepackage[T1]{fontenc}
\usepackage[ngerman]{babel}
\usepackage{amsmath,amssymb,amsthm}
\usepackage{geometry}
\usepackage{xcolor}
\usepackage{tcolorbox}
\usepackage{booktabs}
\usepackage{array}
\usepackage{hyperref}
\usepackage{tocloft}
\usepackage{fancyhdr}
\usepackage{graphicx}

\geometry{a4paper, margin=2.5cm}

% Farbdefinitionen
\definecolor{deepblue}{RGB}{0,0,127}
\definecolor{deepred}{RGB}{191,0,0}
\definecolor{deepgreen}{RGB}{0,127,0}

% Kopf- und Fu{\ss}zeile
\pagestyle{fancy}
\fancyhf{}
\fancyhead[L]{\textsc{T0-Theorie: Vollst{\"a}ndiger Abschluss}}
\fancyhead[R]{\textsc{J. Pascher}}
\fancyfoot[C]{\thepage}
\renewcommand{\headrulewidth}{0.4pt}
\renewcommand{\footrulewidth}{0.4pt}
\setlength{\headheight}{14.5pt}

% Blaue Formatierung f{\"u}r Inhaltsverzeichnis
\renewcommand{\cfttoctitlefont}{\huge\bfseries\color{blue}}
\renewcommand{\cftsecfont}{\color{blue}\bfseries}
\renewcommand{\cftsecpagefont}{\color{blue}\bfseries}
\renewcommand{\cftsubsecfont}{\color{blue!80!black}}
\renewcommand{\cftsubsecpagefont}{\color{blue!80!black}}
\renewcommand{\cftsubsubsecfont}{\color{blue!60!black}}
\renewcommand{\cftsubsubsecpagefont}{\color{blue!60!black}}
\setlength{\cftsecindent}{0pt}
\setlength{\cftsubsecindent}{0pt}

% Hyperref-Einstellungen
\hypersetup{
	colorlinks=true,
	linkcolor=blue,
	citecolor=blue,
	urlcolor=blue,
	pdftitle={T0-Theorie: Vollstaendiger Abschluss},
	pdfauthor={Johann Pascher},
	pdfsubject={T0-Theorie, SI-Reform 2019, Geometrische Physik}
}

% Benutzerdefinierte Boxen
\newtcolorbox{keyresult}{colback=blue!5, colframe=blue!75!black, title=Schl{\"u}sselergebnis}
\newtcolorbox{warning}{colback=red!5, colframe=red!75!black, title=Wichtiger Hinweis}
\newtcolorbox{derivation}{colback=green!5, colframe=green!75!black, title=Herleitung}
\newtcolorbox{insight}{colback=yellow!5, colframe=orange!75!black, title=Fundamentale Einsicht}
\newtcolorbox{historical}{colback=orange!5, colframe=orange!75!black, title=Historischer Kontext}

\title{\textbf{Der vollst{\"a}ndige Abschluss der T0-Theorie}\\[0.5cm]
	\large Von $\xi$ zur SI-Reform 2019:\\
	Warum das moderne SI-System die fundamentale Geometrie des Universums widerspiegelt\\[0.3cm]
	\normalsize Dokument {\"u}ber die vollst{\"a}ndige Parameterfreiheit der T0-Reihe}

\author{Johann Pascher\\
	Abteilung Kommunikationstechnik\\
	H{\"o}here Technische Lehranstalt (HTL), Leonding, {\"O}sterreich\\
	\texttt{johann.pascher@gmail.com}}

\date{\today}



==================================================

=== T0_SI_En.tex.preamble ===

\documentclass[12pt,a4paper]{article}
\usepackage[utf8]{inputenc}
\usepackage[T1]{fontenc}
\usepackage[english]{babel}
\usepackage{amsmath,amssymb,amsthm}
\usepackage{geometry}
\usepackage{xcolor}
\usepackage{tcolorbox}
\usepackage{booktabs}
\usepackage{array}
\usepackage{hyperref}
\usepackage{tocloft}
\usepackage{fancyhdr}
\usepackage{graphicx}

\geometry{a4paper, margin=2.5cm}

% Color definitions
\definecolor{deepblue}{RGB}{0,0,127}
\definecolor{deepred}{RGB}{191,0,0}
\definecolor{deepgreen}{RGB}{0,127,0}

% Header and footer
\pagestyle{fancy}
\fancyhf{}
\fancyhead[L]{\textsc{T0-Theory: Complete Closure}}
\fancyhead[R]{\textsc{J. Pascher}}
\fancyfoot[C]{\thepage}
\renewcommand{\headrulewidth}{0.4pt}
\renewcommand{\footrulewidth}{0.4pt}
\setlength{\headheight}{14.5pt}

% Blue formatting for table of contents
\renewcommand{\cfttoctitlefont}{\huge\bfseries\color{blue}}
\renewcommand{\cftsecfont}{\color{blue}\bfseries}
\renewcommand{\cftsecpagefont}{\color{blue}\bfseries}
\renewcommand{\cftsubsecfont}{\color{blue!80!black}}
\renewcommand{\cftsubsecpagefont}{\color{blue!80!black}}
\renewcommand{\cftsubsubsecfont}{\color{blue!60!black}}
\renewcommand{\cftsubsubsecpagefont}{\color{blue!60!black}}
\setlength{\cftsecindent}{0pt}
\setlength{\cftsubsecindent}{0pt}

% Hyperref settings
\hypersetup{
	colorlinks=true,
	linkcolor=blue,
	citecolor=blue,
	urlcolor=blue,
	pdftitle={T0-Theory: Complete Closure},
	pdfauthor={Johann Pascher},
	pdfsubject={T0-Theory, SI Reform 2019, Geometric Physics}
}

% Custom boxes
\newtcolorbox{keyresult}{colback=blue!5, colframe=blue!75!black, title=Key Result}
\newtcolorbox{warning}{colback=red!5, colframe=red!75!black, title=Important Note}
\newtcolorbox{derivation}{colback=green!5, colframe=green!75!black, title=Derivation}
\newtcolorbox{insight}{colback=yellow!5, colframe=orange!75!black, title=Fundamental Insight}
\newtcolorbox{historical}{colback=orange!5, colframe=orange!75!black, title=Historical Context}

\title{\textbf{The Complete Closure of T0-Theory}\\[0.5cm]
	\large From $\xi$ to the SI Reform 2019:\\
	Why the Modern SI System Reflects the Fundamental Geometry of the Universe\\[0.3cm]
	\normalsize Document on the Complete Parameter Freedom of the T0 Series}

\author{Johann Pascher\\
	Department of Communication Engineering\\
	Higher Technical Institute (HTL), Leonding, Austria\\
	\texttt{johann.pascher@gmail.com}}

\date{\today}



==================================================

=== T0_Teilchenmassen_De.tex.preamble ===

\documentclass[12pt,a4paper]{article}
\usepackage[utf8]{inputenc}
\usepackage[T1]{fontenc}
\usepackage[ngerman]{babel}
\usepackage{lmodern}
\usepackage{amsmath,amssymb,amsthm}
\usepackage{geometry}
\usepackage{booktabs}
\usepackage{array}
\usepackage{xcolor}
\usepackage{tcolorbox}
\usepackage{fancyhdr}
\usepackage{tocloft}
\usepackage{hyperref}
\usepackage{tikz}
\usepackage{physics}
\usepackage{siunitx}
\usepackage{longtable}
\usepackage{caption}

\definecolor{deepblue}{RGB}{0,0,127}
\definecolor{deepred}{RGB}{191,0,0}
\definecolor{deepgreen}{RGB}{0,127,0}

\geometry{a4paper, margin=2.5cm}
\setlength{\headheight}{15pt}

\usetikzlibrary{positioning, arrows.meta}

% Header- und Footer-Konfiguration
\pagestyle{fancy}
\fancyhf{}
\fancyhead[L]{\textsc{T0-Theorie: Teilchenmassen}}
\fancyhead[R]{\textsc{J. Pascher}}
\fancyfoot[C]{\thepage}
\renewcommand{\headrulewidth}{0.4pt}
\renewcommand{\footrulewidth}{0.4pt}

% Inhaltsverzeichnis-Stil - Blau
\renewcommand{\cfttoctitlefont}{\huge\bfseries\color{blue}}
\renewcommand{\cftsecfont}{\color{blue}}
\renewcommand{\cftsubsecfont}{\color{blue}}
\renewcommand{\cftsecpagefont}{\color{blue}}
\renewcommand{\cftsubsecpagefont}{\color{blue}}
\setlength{\cftsecindent}{0pt}
\setlength{\cftsubsecindent}{0pt}

% Hyperref-Einstellungen
\hypersetup{
	colorlinks=true,
	linkcolor=blue,
	citecolor=blue,
	urlcolor=blue,
	pdftitle={T0-Theorie: Teilchenmassen},
	pdfauthor={Johann Pascher},
	pdfsubject={T0-Theorie, Teilchenmassen, Parameterfreie Berechnung}
}

% Benutzerdefinierte Befehle
\newcommand{\xipar}{\xi_0}
\newcommand{\Kfrak}{K_{\text{frak}}}
\newcommand{\Cconv}{C_{\text{conv}}}
\newcommand{\checkmarkx}{\checkmark}
\newcommand{\warningx}{{\color{red}\textbf{!}}}

% Umgebung für Schlüsselergebnisse
\newtcolorbox{keyresult}{colback=blue!5, colframe=blue!75!black, title=Schlüsselergebnis}
\newtcolorbox{warning}{colback=red!5, colframe=red!75!black, title=Wichtiger Hinweis}
\newtcolorbox{method}{colback=green!5, colframe=green!75!black, title=Berechnungsmethode}
\newtcolorbox{equivalence}{colback=purple!5, colframe=purple!75!black, title=Äquivalenznachweis}
\newtcolorbox{experimental}{colback=orange!5, colframe=orange!75!black, title=Experimenteller Vergleich}

\title{\textbf{T0-Theorie: Teilchenmassen}\\[0.5cm]
	\large Parameterfreie Berechnung aller Fermionmassen\\[0.3cm]
	\normalsize Dokument 4 der T0-Serie}
\author{Johann Pascher\\
	Abteilung für Kommunikationstechnologie\\
	Höhere Technische Lehranstalt (HTL), Leonding, Österreich\\
	\texttt{johann.pascher@gmail.com}}
\date{\today}



==================================================

=== T0_Teilchenmassen_En.tex.preamble ===

\documentclass[12pt,a4paper]{article}
\usepackage[utf8]{inputenc}
\usepackage[T1]{fontenc}
\usepackage[english]{babel}
\usepackage{lmodern}
\usepackage{amsmath,amssymb,amsthm}
\usepackage{geometry}
\usepackage{booktabs}
\usepackage{array}
\usepackage{xcolor}
\usepackage{tcolorbox}
\usepackage{fancyhdr}
\usepackage{tocloft}
\usepackage{hyperref}
\usepackage{tikz}
\usepackage{physics}
\usepackage{siunitx}
\usepackage{longtable}
\usepackage{caption}

\definecolor{deepblue}{RGB}{0,0,127}
\definecolor{deepred}{RGB}{191,0,0}
\definecolor{deepgreen}{RGB}{0,127,0}

\geometry{a4paper, margin=2.5cm}
\setlength{\headheight}{15pt}

\usetikzlibrary{positioning, arrows.meta}

% Header and Footer Configuration
\pagestyle{fancy}
\fancyhf{}
\fancyhead[L]{\textsc{T0-Theory: Particle Masses}}
\fancyhead[R]{\textsc{J. Pascher}}
\fancyfoot[C]{\thepage}
\renewcommand{\headrulewidth}{0.4pt}
\renewcommand{\footrulewidth}{0.4pt}

% Table of Contents Style - Blue
\renewcommand{\cfttoctitlefont}{\huge\bfseries\color{blue}}
\renewcommand{\cftsecfont}{\color{blue}}
\renewcommand{\cftsubsecfont}{\color{blue}}
\renewcommand{\cftsecpagefont}{\color{blue}}
\renewcommand{\cftsubsecpagefont}{\color{blue}}
\setlength{\cftsecindent}{0pt}
\setlength{\cftsubsecindent}{0pt}

% Hyperref Settings
\hypersetup{
	colorlinks=true,
	linkcolor=blue,
	citecolor=blue,
	urlcolor=blue,
	pdftitle={T0-Theory: Particle Masses},
	pdfauthor={Johann Pascher},
	pdfsubject={T0-Theory, Particle Masses, Parameter-Free Calculation}
}

% User-Defined Commands
\newcommand{\xipar}{\xi_0}
\newcommand{\Kfrak}{K_{\text{frak}}}
\newcommand{\Cconv}{C_{\text{conv}}}
\newcommand{\checkmarkx}{\checkmark}
\newcommand{\warningx}{{\color{red}\textbf{!}}}

% Environment for Key Results
\newtcolorbox{keyresult}{colback=blue!5, colframe=blue!75!black, title={Key Result}}
\newtcolorbox{warning}{colback=red!5, colframe=red!75!black, title={Important Note}}
\newtcolorbox{method}{colback=green!5, colframe=green!75!black, title={Calculation Method}}
\newtcolorbox{equivalence}{colback=purple!5, colframe=purple!75!black, title={Equivalence Proof}}
\newtcolorbox{experimental}{colback=orange!5, colframe=orange!75!black, title={Experimental Comparison}}

\title{\textbf{T0-Theory: Particle Masses}\\[0.5cm]
	\large Parameter-Free Calculation of All Fermion Masses\\[0.3cm]
	\normalsize Document 4 of the T0 Series}
\author{Johann Pascher\\
	Department of Communication Technology\\
	Higher Technical College (HTL), Leonding, Austria\\
	\texttt{johann.pascher@gmail.com}}
\date{\today}



==================================================

=== T0_Vollstaendige_Berchnungen_De.tex.preamble ===

\documentclass[11pt,a4paper]{article}
\usepackage[utf8]{inputenc}
\usepackage[ngerman]{babel}
\usepackage{amsmath,amsfonts,amssymb,physics}
\usepackage{booktabs,array,longtable,multirow}
\usepackage{geometry,fancyhdr}
\usepackage{siunitx,xcolor,graphicx}
\usepackage{hyperref,url}
\usepackage{listings,enumerate}

\geometry{margin=2cm}
\sisetup{locale = DE, group-separator = {.}, output-decimal-marker = {,}}
\setlength{\headheight}{13.6pt}
% Inhaltsverzeichnis-Styling


% Hyperlink-Setup
\hypersetup{
	colorlinks=true,
	linkcolor=blue,
	citecolor=blue,
	urlcolor=blue,
	pdftitle={Das T0-Modell (Planck-Referenziert): Eine Neuformulierung der Physik},
	pdfauthor={Johann Pascher},
	pdfsubject={T0-Modell, Planck-Referenzierte Physik, Theoretische Physik, Natürliche Einheiten},
	pdfkeywords={T0 Theorie, Planck-Skala, Quantenmechanik, Feldtheorie, Vereinheitlichte Physik}
}

% Farben definieren
\definecolor{t0blue}{RGB}{33,150,243}
\definecolor{t0green}{RGB}{76,175,80}
\definecolor{t0orange}{RGB}{255,152,0}
\definecolor{t0red}{RGB}{244,67,54}

% Header/Footer
\pagestyle{fancy}
\fancyhf{}
\fancyhead[L]{T0-Theorie: v3.2}
\fancyhead[R]{\today}
\fancyfoot[C]{\thepage}

\title{\textbf{T0-Theorie: Berechnung von Teilchenmassen und physikalischen Konstanten}\\
	\large Vereinigte Berechnung von Teilchenmassen und physikalischen Konstanten per Skript\\
	\large Version 3.2}

\author{Johann Pascher\\
	HTL Leonding, Österreich\\
	\texttt{v3.2}}

\date{\today}



==================================================

=== T0_Vollstaendige_Berchnungen_En.tex.preamble ===

\documentclass[11pt,a4paper]{article}
\usepackage[utf8]{inputenc}
\usepackage[english]{babel}
\usepackage{amsmath,amsfonts,amssymb,physics}
\usepackage{booktabs,array,longtable,multirow}
\usepackage{geometry,fancyhdr}
\usepackage{siunitx,xcolor,graphicx}
\usepackage{hyperref,url}
\usepackage{listings,enumerate}

\geometry{margin=2cm}
\sisetup{locale = DE, group-separator = {.}, output-decimal-marker = {,}}
\setlength{\headheight}{13.6pt}
% Table of Contents Styling


% Hyperlink Setup
\hypersetup{
	colorlinks=true,
	linkcolor=blue,
	citecolor=blue,
	urlcolor=blue,
	pdftitle={The T0 Model (Planck-Referenced): A Reformulation of Physics},
	pdfauthor={Johann Pascher},
	pdfsubject={T0 Model, Planck-Referenced Physics, Theoretical Physics, Natural Units},
	pdfkeywords={T0 Theory, Planck Scale, Quantum Mechanics, Field Theory, Unified Physics}
}

% Define colors
\definecolor{t0blue}{RGB}{33,150,243}
\definecolor{t0green}{RGB}{76,175,80}
\definecolor{t0orange}{RGB}{255,152,0}
\definecolor{t0red}{RGB}{244,67,54}

% Header/Footer
\pagestyle{fancy}
\fancyhf{}
\fancyhead[L]{T0 Theory: v3.2}
\fancyhead[R]{\today}
\fancyfoot[C]{\thepage}

\title{\textbf{T0 Theory: Calculation of Particle Masses and Physical Constants}\\
	\large Unified Calculation of Particle Masses and Physical Constants  with script\\
	\large Version 3.2}

\author{Johann Pascher\\
	HTL Leonding, Austria\\
	\texttt{v3.2}}

\date{\today}



==================================================

=== T0_g2-erweiterung-4_De.tex.preamble ===

\documentclass[12pt,a4paper]{article}
\usepackage[utf8]{inputenc}
\usepackage[T1]{fontenc}
\usepackage[german]{babel}
\usepackage{lmodern}
\usepackage{amsmath}
\usepackage{amssymb}
\usepackage{physics}
\usepackage{hyperref}
\usepackage{tcolorbox}
\usepackage{booktabs}
\usepackage{enumitem}
\usepackage[table]{xcolor}
\usepackage[left=2cm,right=2cm,top=2cm,bottom=2cm]{geometry}
\usepackage{pgfplots}
\pgfplotsset{compat=1.18}
\usepackage{graphicx}
\usepackage{float}
\usepackage{fancyhdr}
\usepackage{siunitx}
\usepackage{mathtools}
\usepackage{amsthm}
\usepackage{cleveref}
\usepackage{tocloft}
\usepackage{tikz}
\usepackage[dvipsnames]{xcolor}
\usetikzlibrary{positioning, shapes.geometric, arrows.meta}
\usepackage{microtype}
\usepackage{array}
\usepackage{longtable}
\usepackage{url}

% Kopfzeilen-Höhe anpassen
\setlength{\headheight}{14.5pt}

% Custom Commands
\newcommand{\Efield}{E_{\text{Feld}}}
\newcommand{\xigeom}{\xi_{\text{geom}}}
\newcommand{\Tzero}{T_0}
\newcommand{\vecx}{\vec{x}}
\newcommand{\xipar}{\xi}
\newcommand{\Kfrak}{K_{\text{frak}}}
\newcommand{\CQCD}{C_{\text{QCD}}}
\newcommand{\Kspec}{K_{\text{spec}}}

% Header and Footer Configuration
\pagestyle{fancy}
\fancyhf{}
\fancyhead[L]{Johann Pascher}
\fancyhead[R]{T0-Theorie: Finale Erweiterung auf Hadronen}
\fancyfoot[C]{\thepage}
\renewcommand{\headrulewidth}{0.4pt}
\renewcommand{\footrulewidth}{0.4pt}

% Table of Contents Formatting - BLAU
\renewcommand{\cftsecfont}{\color{blue}}
\renewcommand{\cftsubsecfont}{\color{blue}}
\renewcommand{\cftsecpagefont}{\color{blue}}
\renewcommand{\cftsubsecpagefont}{\color{blue}}

\hypersetup{
	colorlinks=true,
	linkcolor=blue,
	citecolor=blue,
	urlcolor=blue,
	pdftitle={T0-Theorie: Finale Erweiterung auf Hadronen - Physikalisch abgeleitete Korrekturen},
	pdfauthor={Johann Pascher},
	pdfsubject={T0-Theorie, Hadronen g-2, Fraktale QCD, Parameterfreie Berechnung},
	pdfkeywords={T0-Theorie, Anomalous Magnetic Moment, Hadronen, Quarks, Parameterfrei}
}

% Theorem Environments
\newtheorem{theorem}{Theorem}[section]
\newtheorem{proposition}[theorem]{Proposition}
\newtheorem{definition}[theorem]{Definition}
\newtheorem{lemma}[theorem]{Lemma}

\tcbuselibrary{theorems}
\newtcbtheorem[number within=section]{important}{Wichtige Erkenntnis}%
{colback=green!5,colframe=green!35!black,fonttitle=\bfseries}{th}
\newtcbtheorem[number within=section]{schluessel}{Schlüssel}%
{colback=blue!5,colframe=blue!75!black,fonttitle=\bfseries}{key}
\newtcbtheorem[number within=section]{result}{Ergebnis}%
{colback=green!5,colframe=green!75!black,fonttitle=\bfseries}{res}
\newtcbtheorem[number within=section]{keyresult}{Schlüsselergebnis}%
{colback=blue!5,colframe=blue!75!black,fonttitle=\bfseries}{keyres}

\title{T0-Time-Mass-Dualitäts-Theorie: Finale Erweiterung auf Hadronen \\
	\large Physikalisch abgeleitete Korrekturfaktoren für exakte Übereinstimmung}
\author{Johann Pascher\\
	Abteilung für Kommunikationstechnologie\\
	Höhere Technische Bundeslehranstalt (HTL), Leonding, Österreich\\
	\texttt{johann.pascher@gmail.com}}
\date{1. November 2025}



==================================================

=== T0_g2-erweiterung-4_En.tex.preamble ===

\documentclass[12pt,a4paper]{article}
\usepackage[utf8]{inputenc}
\usepackage[T1]{fontenc}
\usepackage[english]{babel}
\usepackage{lmodern}
\usepackage{amsmath}
\usepackage{amssymb}
\usepackage{physics}
\usepackage{hyperref}
\usepackage{tcolorbox}
\usepackage{booktabs}
\usepackage{enumitem}
\usepackage[table]{xcolor}
\usepackage[left=2cm,right=2cm,top=2cm,bottom=2cm]{geometry}
\usepackage{pgfplots}
\pgfplotsset{compat=1.18}
\usepackage{graphicx}
\usepackage{float}
\usepackage{fancyhdr}
\usepackage{siunitx}
\usepackage{mathtools}
\usepackage{amsthm}
\usepackage{cleveref}
\usepackage{tocloft}
\usepackage{tikz}
\usepackage[dvipsnames]{xcolor}
\usetikzlibrary{positioning, shapes.geometric, arrows.meta}
\usepackage{microtype}
\usepackage{array}
\usepackage{longtable}
\usepackage{url}

% Adjust header height
\setlength{\headheight}{14.5pt}

% Custom Commands
\newcommand{\Efield}{E_{\text{field}}}
\newcommand{\xigeom}{\xi_{\text{geom}}}
\newcommand{\Tzero}{T_0}
\newcommand{\vecx}{\vec{x}}
\newcommand{\xipar}{\xi}
\newcommand{\Kfrak}{K_{\text{frac}}}
\newcommand{\CQCD}{C_{\text{QCD}}}
\newcommand{\Kspec}{K_{\text{spec}}}

% Header and Footer Configuration
\pagestyle{fancy}
\fancyhf{}
\fancyhead[L]{Johann Pascher}
\fancyhead[R]{T0 Theory: Final Extension to Hadrons}
\fancyfoot[C]{\thepage}
\renewcommand{\headrulewidth}{0.4pt}
\renewcommand{\footrulewidth}{0.4pt}

% Table of Contents Formatting - BLUE
\renewcommand{\cftsecfont}{\color{blue}}
\renewcommand{\cftsubsecfont}{\color{blue}}
\renewcommand{\cftsecpagefont}{\color{blue}}
\renewcommand{\cftsubsecpagefont}{\color{blue}}

\hypersetup{
	colorlinks=true,
	linkcolor=blue,
	citecolor=blue,
	urlcolor=blue,
	pdftitle={T0 Theory: Final Extension to Hadrons - Physically Derived Corrections},
	pdfauthor={Johann Pascher},
	pdfsubject={T0 Theory, Hadron g-2, Fractal QCD, Parameter-Free Calculation},
	pdfkeywords={T0 Theory, Anomalous Magnetic Moment, Hadrons, Quarks, Parameter-Free}
}

% Theorem Environments
\newtheorem{theorem}{Theorem}[section]
\newtheorem{proposition}[theorem]{Proposition}
\newtheorem{definition}[theorem]{Definition}
\newtheorem{lemma}[theorem]{Lemma}

\tcbuselibrary{theorems}
\newtcbtheorem[number within=section]{important}{Important Insight}%
{colback=green!5,colframe=green!35!black,fonttitle=\bfseries}{th}
\newtcbtheorem[number within=section]{keypoint}{Key Point}%
{colback=blue!5,colframe=blue!75!black,fonttitle=\bfseries}{key}
\newtcbtheorem[number within=section]{result}{Result}%
{colback=green!5,colframe=green!75!black,fonttitle=\bfseries}{res}
\newtcbtheorem[number within=section]{keyresult}{Key Result}%
{colback=blue!5,colframe=blue!75!black,fonttitle=\bfseries}{keyres}

\title{T0-Time-Mass-Duality Theory: Final Extension to Hadrons \\
	\large Physically Derived Correction Factors for Exact Agreement}
\author{Johann Pascher\\
	Department of Communication Technology\\
	Higher Technical Federal Institute (HTL), Leonding, Austria\\
	\texttt{johann.pascher@gmail.com}}
\date{November 1, 2025}



==================================================

=== T0_koide-formel-3_De.tex.preamble ===

\documentclass[12pt,a4paper]{article}
\usepackage[utf8]{inputenc}
\usepackage[T1]{fontenc}
\usepackage[ngerman]{babel}
\usepackage[left=2.5cm,right=2.5cm,top=2.5cm,bottom=2.5cm]{geometry}
\usepackage{amsmath}
\usepackage{amssymb}
\usepackage{hyperref}
\usepackage{booktabs}
\usepackage{tcolorbox}
\usepackage{physics}
\usepackage{fancyhdr}
\usepackage{tocloft} % Für Anpassungen am Inhaltsverzeichnis
\usepackage[table]{xcolor} % Für Farben

% Header and Footer Configuration
\pagestyle{fancy}
\fancyhf{}
\fancyhead[L]{Johann Pascher}
\fancyhead[R]{T0-Theorie: Koide-Formel und $\xi$}
\fancyfoot[C]{\thepage}
\renewcommand{\headrulewidth}{0.4pt}
\renewcommand{\footrulewidth}{0.4pt}
\setlength{\headheight}{15pt} % Fix for headheight warning

\tcbuselibrary{theorems}
\newtcolorbox{theorem}{colback=blue!5!white,colframe=blue!75!black,fonttitle=\bfseries,title=Haupttheorem}
\newtcolorbox{beweis}{colback=green!5!white,colframe=green!35!black,fonttitle=\bfseries,title=Beweis}
\newtcolorbox{folgerung}{colback=yellow!5!white,colframe=orange!75!black,fonttitle=\bfseries,title=Folgerung}

% Blau für das Inhaltsverzeichnis
\definecolor{blue}{rgb}{0,0,1}
\renewcommand{\cftsecfont}{\color{blue}}
\renewcommand{\cftsecpagefont}{\color{blue}}
\renewcommand{\cftsubsecfont}{\color{blue}}
\renewcommand{\cftsubsubsecfont}{\color{blue}}
% Removed invalid \cftsecentryfont
\renewcommand{\cftsubsecpagefont}{\color{blue}}
\renewcommand{\cftsubsubsecpagefont}{\color{blue}}

\hypersetup{
	colorlinks=true,
	linkcolor=blue,
	citecolor=blue,
	urlcolor=blue,
	pdftitle={Beweis: Die Koide-Formel enthält implizit $\xi$},
	pdfauthor={Johann Pascher},
	pdfsubject={T0-Theorie, Koide-Formel, Leptonmassen}
}

\title{\textbf{Beweis: Die Koide-Formel enthält implizit $\xi$}\\[0.5cm]
	\large Geometrische Herleitung der Leptonmassen-Symmetrie\\[0.3cm]
	\normalsize aus der T0-Theorie}
\author{Johann Pascher\\
	HTL Leonding, Österreich\\
	\texttt{johann.pascher@gmail.com}}
\date{\today}



==================================================

=== T0_koide-formel-3_En.tex.preamble ===

\documentclass[12pt,a4paper]{article}
\usepackage[utf8]{inputenc}
\usepackage[T1]{fontenc}
\usepackage[english]{babel}
\usepackage[left=2.5cm,right=2.5cm,top=2.5cm,bottom=2.5cm]{geometry}
\usepackage{amsmath}
\usepackage{amssymb}
\usepackage{hyperref}
\usepackage{booktabs}
\usepackage{tcolorbox}
\usepackage{physics}
\usepackage{fancyhdr}
\usepackage{tocloft} % For adjustments to the table of contents
\usepackage[table]{xcolor} % For colors

% Header and Footer Configuration
\pagestyle{fancy}
\fancyhf{}
\fancyhead[L]{Johann Pascher}
\fancyhead[R]{T0 Theory: Koide Formula and $\xi$}
\fancyfoot[C]{\thepage}
\renewcommand{\headrulewidth}{0.4pt}
\renewcommand{\footrulewidth}{0.4pt}
\setlength{\headheight}{15pt} % Fix for headheight warning

\tcbuselibrary{theorems}
\newtcolorbox{theorem}{colback=blue!5!white,colframe=blue!75!black,fonttitle=\bfseries,title=Main Theorem}
\newtcolorbox{beweis}{colback=green!5!white,colframe=green!35!black,fonttitle=\bfseries,title=Proof}
\newtcolorbox{folgerung}{colback=yellow!5!white,colframe=orange!75!black,fonttitle=\bfseries,title=Corollary}

% Blue for the table of contents
\definecolor{blue}{rgb}{0,0,1}
\renewcommand{\cftsecfont}{\color{blue}}
\renewcommand{\cftsecpagefont}{\color{blue}}
\renewcommand{\cftsubsecfont}{\color{blue}}
\renewcommand{\cftsubsubsecfont}{\color{blue}}
% Removed invalid \cftsecentryfont
\renewcommand{\cftsubsecpagefont}{\color{blue}}
\renewcommand{\cftsubsubsecpagefont}{\color{blue}}

\hypersetup{
	colorlinks=true,
	linkcolor=blue,
	citecolor=blue,
	urlcolor=blue,
	pdftitle={Proof: The Koide Formula Implicitly Contains $\xi$},
	pdfauthor={Johann Pascher},
	pdfsubject={T0 Theory, Koide Formula, Lepton Masses}
}

\title{\textbf{Proof: The Koide Formula Implicitly Contains $\xi$}\\[0.5cm]
	\large Geometric Derivation of Lepton Mass Symmetry\\[0.3cm]
	\normalsize from the T0 Theory}
\author{Johann Pascher\\
	HTL Leonding, Austria\\
	\texttt{johann.pascher@gmail.com}}
\date{\today}



==================================================

=== T0_lagrndian_De.tex.preamble ===

\documentclass[12pt,a4paper]{article}
\usepackage[utf8]{inputenc}
\usepackage[T1]{fontenc}
\usepackage[ngerman]{babel}
\usepackage{lmodern}
\usepackage{amsmath,amssymb,amsthm}
\usepackage{geometry}
\usepackage{booktabs}
\usepackage{array}
\usepackage{xcolor}
\usepackage{tcolorbox}
\usepackage{fancyhdr}
\usepackage{tocloft}
\usepackage{hyperref}
\usepackage{enumitem}
\usepackage{physics}

\definecolor{deepblue}{RGB}{0,0,127}
\definecolor{deepred}{RGB}{191,0,0}
\definecolor{deepgreen}{RGB}{0,127,0}

\geometry{a4paper, margin=2.5cm}

% Header und Footer Konfiguration
\pagestyle{fancy}
\fancyhf{}
\fancyhead[L]{\textsc{T0-Theorie: Die T0-Zeit-Masse-Dualität}}
\fancyhead[R]{\textsc{Johann Pascher}}
\fancyfoot[C]{\thepage}
\renewcommand{\headrulewidth}{0.4pt}
\renewcommand{\footrulewidth}{0.4pt}

% Fix head height warning
\setlength{\headheight}{14.5pt}

% Inhaltsverzeichnis Stil
\renewcommand{\cfttoctitlefont}{\huge\bfseries\color{blue}}
\renewcommand{\cftsecfont}{\color{blue}}
\renewcommand{\cftsubsecfont}{\color{blue}}
\renewcommand{\cftsecpagefont}{\color{blue}}
\renewcommand{\cftsubsecpagefont}{\color{blue}}
\setlength{\cftsecindent}{0pt}
\setlength{\cftsubsecindent}{0pt}

% Hyperref Einstellungen
\hypersetup{
	colorlinks=true,
	linkcolor=blue,
	citecolor=blue,
	urlcolor=blue,
	pdftitle={T0-Theorie: Die T0-Zeit-Masse-Dualität},
	pdfauthor={Johann Pascher},
	pdfsubject={T0-Theorie, Zeit-Masse-Dualität, Lagrangian-Formulierungen}
}

% Benutzerdefinierte Befehle
\newcommand{\deltaE}{\delta E}
\newcommand{\Echar}{E_{\text{char}}}
\newcommand{\MPl}{M_{\text{Pl}}}

% Umgebungen für Schlüsselergebnisse
\newtcolorbox{keyresult}{colback=blue!5, colframe=blue!75!black, title=Schlüsselergebnis}
\newtcolorbox{warning}{colback=red!5, colframe=red!75!black, title=Wichtiger Hinweis}
\newtcolorbox{derivation}{colback=green!5, colframe=green!75!black, title=Ableitung}
\newtcolorbox{dimensional}{colback=yellow!5, colframe=orange!75!black, title=Dimensionsanalyse}
\newtcolorbox{verification}{colback=purple!5, colframe=purple!75!black, title=Experimentelle Verifikation}
\newtcolorbox{explanation}{colback=orange!5, colframe=orange!75!black, title=T0 Erklärung}

% Bessere Tabellenformatierung
\newcolumntype{L}[1]{>{\raggedright\arraybackslash}p{#1}}
\newcolumntype{C}[1]{>{\centering\arraybackslash}p{#1}}
\newcolumntype{R}[1]{>{\raggedleft\arraybackslash}p{#1}}

% Silbentrennung für lange Wörter
\hyphenation{Prä-zi-si-ons-ex-pe-ri-men-te theo-re-ti-sche For-mu-lie-run-gen}

\title{\textbf{T0-Theorie: Die T0-Zeit-Masse-Dualität}\\[0.5cm]
	\large Vollständige theoretische Formulierung und experimentelle Vorhersagen\\[0.3cm]
	\normalsize Dokument der T0-Serie}
\author{Johann Pascher\\
	Department für Kommunikationstechnik\\
	Höhere Technische Lehranstalt (HTL), Leonding, Österreich\\
	\texttt{johann.pascher@gmail.com}}
\date{26. Oktober 2025}



==================================================

=== T0_lagrndian_En.tex.preamble ===

\documentclass[12pt,a4paper]{article}
\usepackage[utf8]{inputenc}
\usepackage[T1]{fontenc}
\usepackage[english]{babel}
\usepackage{lmodern}
\usepackage{amsmath,amssymb,amsthm}
\usepackage{geometry}
\usepackage{booktabs}
\usepackage{array}
\usepackage{xcolor}
\usepackage{tcolorbox}
\usepackage{fancyhdr}
\usepackage{tocloft}
\usepackage{hyperref}
\usepackage{tikz}
\usepackage{siunitx}
\usepackage{graphicx}
\usepackage{enumitem}
\usepackage{physics}

\definecolor{deepblue}{RGB}{0,0,127}
\definecolor{deepred}{RGB}{191,0,0}
\definecolor{deepgreen}{RGB}{0,127,0}

\geometry{a4paper, margin=2.5cm}

% Header and Footer Configuration
\pagestyle{fancy}
\fancyhf{}
\fancyhead[L]{\textsc{T0-Theory: The T0-Time-Mass Duality}}
\fancyhead[R]{\textsc{Johann Pascher}}
\fancyfoot[C]{\thepage}
\renewcommand{\headrulewidth}{0.4pt}
\renewcommand{\footrulewidth}{0.4pt}

% Fix head height warning
\setlength{\headheight}{14.5pt}

% Table of Contents Style
\renewcommand{\cfttoctitlefont}{\huge\bfseries\color{blue}}
\renewcommand{\cftsecfont}{\color{blue}}
\renewcommand{\cftsubsecfont}{\color{blue}}
\renewcommand{\cftsecpagefont}{\color{blue}}
\renewcommand{\cftsubsecpagefont}{\color{blue}}
\setlength{\cftsecindent}{0pt}
\setlength{\cftsubsecindent}{0pt}

% Hyperref Settings
\hypersetup{
	colorlinks=true,
	linkcolor=blue,
	citecolor=blue,
	urlcolor=blue,
	pdftitle={T0-Theory: The T0-Time-Mass Duality},
	pdfauthor={Johann Pascher},
	pdfsubject={T0-Theory, Time-Mass Duality, Lagrangian Formulations}
}

% Custom Commands
\newcommand{\deltaE}{\delta E}
\newcommand{\Echar}{E_{\text{char}}}
\newcommand{\MPl}{M_{\text{Pl}}}

% Environments for Key Results
\newtcolorbox{keyresult}{colback=blue!5, colframe=blue!75!black, title=Key Result}
\newtcolorbox{warning}{colback=red!5, colframe=red!75!black, title=Important Note}
\newtcolorbox{derivation}{colback=green!5, colframe=green!75!black, title=Derivation}
\newtcolorbox{dimensional}{colback=yellow!5, colframe=orange!75!black, title=Dimensions Analysis}
\newtcolorbox{verification}{colback=purple!5, colframe=purple!75!black, title=Experimental Verification}
\newtcolorbox{explanation}{colback=orange!5, colframe=orange!75!black, title=T0 Explanation}

% Better table formatting
\newcolumntype{L}[1]{>{\raggedright\arraybackslash}p{#1}}
\newcolumntype{C}[1]{>{\centering\arraybackslash}p{#1}}
\newcolumntype{R}[1]{>{\raggedleft\arraybackslash}p{#1}}

% Hyphenation for long words
\hyphenation{pre-ci-sion-ex-per-i-ments theo-ret-i-cal form-u-la-tions}

\title{\textbf{T0-Theory: The T0-Time-Mass Duality}\\[0.5cm]
	\large Complete Theoretical Formulation and Experimental Predictions\\[0.3cm]
	\normalsize Document of the T0-Series}
\author{Johann Pascher\\
	Department for Communication Technology\\
	Higher Technical College (HTL), Leonding, Austria\\
	\texttt{johann.pascher@gmail.com}}
\date{October 26, 2025}



==================================================

=== T0_nat-si_De.tex.preamble ===

\documentclass[12pt,a4paper]{article}
\usepackage[utf8]{inputenc}
\usepackage[T1]{fontenc}
\usepackage[german]{babel}
\usepackage{geometry}
\usepackage{lmodern}
\usepackage{amsmath}
\usepackage{amssymb}
\usepackage{hyperref}
\usepackage{booktabs}
\usepackage{enumitem}
\usepackage[table,xcdraw]{xcolor}
\usepackage{newunicodechar}

% Unicode setups for Greek letters
\newunicodechar{ξ}{\ensuremath{\xi}}
\newunicodechar{μ}{\ensuremath{\mu}}

\geometry{left=2cm,right=2cm,top=2cm,bottom=2cm}

\hypersetup{
	colorlinks=true,
	linkcolor=blue,
	citecolor=blue,
	urlcolor=blue,
	pdftitle={Natürliche Einheiten in der theoretischen Physik: Eine Abhandlung im Kontext der T0-Theorie},
	pdfauthor={Johann Pascher},
	pdfsubject={Theoretische Physik, T0-Theorie, Natürliche Einheiten}
}

\title{Natürliche Einheiten in der theoretischen Physik: Eine Abhandlung im Kontext der T0-Theorie}
\author{Johann Pascher\\
	Abteilung für Nachrichtentechnik\\
	Höhere Technische Lehranstalt, Leonding, Österreich\\
	\texttt{johann.pascher@gmail.com}}
\date{\today}



==================================================

=== T0_nat-si_En.tex.preamble ===

\documentclass[12pt,a4paper]{article}
\usepackage[utf8]{inputenc}
\usepackage[T1]{fontenc}
\usepackage{geometry}
\usepackage{lmodern}
\usepackage{amsmath}
\usepackage{amssymb}
\usepackage{hyperref}
\usepackage{booktabs}
\usepackage{enumitem}
\usepackage[table,xcdraw]{xcolor}
\usepackage{newunicodechar}

% Unicode setups for Greek letters
\newunicodechar{ξ}{\ensuremath{\xi}}
\newunicodechar{μ}{\ensuremath{\mu}}

\geometry{left=2cm,right=2cm,top=2cm,bottom=2cm}

\hypersetup{
	colorlinks=true,
	linkcolor=blue,
	citecolor=blue,
	urlcolor=blue,
	pdftitle={Natural Units in Theoretical Physics: A Treatise in the Context of T0 Theory},
	pdfauthor={Johann Pascher},
	pdfsubject={Theoretical Physics, T0 Theory, Natural Units}
}

\title{Natural Units in Theoretical Physics: A Treatise in the Context of T0 Theory}
\author{Johann Pascher\\
	Department of Communications Engineering\\
	Higher Technical Institute, Leonding, Austria\\
	\texttt{johann.pascher@gmail.com}}
\date{\today}



==================================================

=== T0_netze_De.tex.preamble ===

\documentclass[12pt,a4paper]{article}
\usepackage[utf8]{inputenc}
\usepackage[T1]{fontenc}
\usepackage[german]{babel}
\usepackage[left=2cm,right=2cm,top=2cm,bottom=2cm]{geometry}
\usepackage{lmodern}
\usepackage{amsmath}
\usepackage{amssymb}
\usepackage{physics}
\usepackage{hyperref}
\usepackage{tcolorbox}
\usepackage{booktabs}
\usepackage{enumitem}
\usepackage[table,xcdraw]{xcolor}
\usepackage{longtable}
\usepackage{siunitx}
\usepackage{fancyhdr}
\usepackage{textgreek}
\usepackage{array}  % Beibehalten aus Original für Tabellen

% Header and Footer
\pagestyle{fancy}
\fancyhf{}
\fancyhead[L]{T0-Theorie: Netzwerkdarstellung und Dimensionsanalyse}
\fancyhead[R]{\thepage}
\fancyfoot[C]{\textit{Von der universellen $\xi$-Konstante zu multidimensionalen Netzwerken und Faktorisierung}}
\renewcommand{\headrulewidth}{0.4pt}
\renewcommand{\footrulewidth}{0.4pt}
\setlength{\headheight}{15pt}  % Fix für headheight-Warnung

\hypersetup{
	colorlinks=true,
	linkcolor=blue,
	citecolor=blue,
	urlcolor=blue,
	pdftitle={T0-Theorie: Netzwerkdarstellung und Dimensionsanalyse in der T0-Theorie},
	pdfauthor={Johann Pascher},
	pdfsubject={T0-Theorie, Netzwerke, Dimensionseffekte, Faktorisierung, $\xi$-Parameter, Neuronale Netzwerke}
}

% Custom environments (angepasst und erweitert aus Beispiel)
\newtcolorbox{important}[1][]{colback=yellow!10!white,colframe=yellow!50!black,fonttitle=\bfseries,title=Wichtiger Hinweis,#1}
\newtcolorbox{formula}[1][]{colback=blue!5!white,colframe=blue!75!black,fonttitle=\bfseries,title=Schlüsselformel,#1}
\newtcolorbox{revolutionary}[1][]{colback=red!5!white,colframe=red!75!black,fonttitle=\bfseries,title=Revolutionäre Erkenntnis,#1}
\newtcolorbox{experiment}[1][]{colback=green!5!white,colframe=green!75!black,fonttitle=\bfseries,title=Experimenteller Test,#1}
\newtcolorbox{sibox}[1][]{colback=orange!10!white,colframe=orange!75!black,fonttitle=\bfseries,title=SI-Einheiten (nur zur Referenz),#1}

% Definiere gängige mathematische Symbole für konsistente Verwendung (aus Original beibehalten)
\newcommand{\xipar}{\ensuremath{\xi}}
\newcommand{\deltafield}{\ensuremath{\delta m}}
\newcommand{\partialop}{\ensuremath{\partial}}
\newcommand{\lambdah}{\ensuremath{\lambda_h}}
\newcommand{\betaT}{\ensuremath{\beta_T}}
\newcommand{\alphaEM}{\ensuremath{\alpha_{\text{EM}}}}
\newcommand{\rhofield}{\ensuremath{\rho}}
\newcommand{\mypi}{\ensuremath{\pi}}
\newcommand{\myphi}{\ensuremath{\phi}}
\newcommand{\myomega}{\ensuremath{\omega}}
\newcommand{\mytimes}{\ensuremath{\times}}
\newcommand{\myapprox}{\ensuremath{\approx}}
\newcommand{\myrightarrow}{\ensuremath{\rightarrow}}
\newcommand{\myRightarrow}{\ensuremath{\Rightarrow}}
\newcommand{\mypropto}{\ensuremath{\propto}}
\newcommand{\mysim}{\ensuremath{\sim}}
\newcommand{\mysqrt}{\ensuremath{\sqrt}}

\title{\Huge\textbf{T0-Theorie: Netzwerkdarstellung und Dimensionsanalyse}\\
	\Large Mathematischer Rahmen, Dimensionseffekte und Faktorisierungsanwendungen}
\author{Johann Pascher\\
	Abteilung für Kommunikationstechnik, \\Höhere Technische Bundeslehranstalt (HTL), Leonding, Österreich\\
	\texttt{johann.pascher@gmail.com}}
\date{\today}



==================================================

=== T0_netze_En.tex.preamble ===

\documentclass[12pt,a4paper]{article}
\usepackage[utf8]{inputenc}
\usepackage[T1]{fontenc}
\usepackage[english]{babel}
\usepackage[left=2cm,right=2cm,top=2cm,bottom=2cm]{geometry}
\usepackage{lmodern}
\usepackage{amsmath}
\usepackage{amssymb}
\usepackage{physics}
\usepackage{hyperref}
\usepackage{tcolorbox}
\usepackage{booktabs}
\usepackage{enumitem}
\usepackage[table,xcdraw]{xcolor}
\usepackage{longtable}
\usepackage{siunitx}
\usepackage{fancyhdr}
\usepackage{textgreek}
\usepackage{array}  % Retained from original for tables

% Header and Footer
\pagestyle{fancy}
\fancyhf{}
\fancyhead[L]{T0-Theory: Network Representation and Dimensional Analysis}
\fancyhead[R]{\thepage}
\fancyfoot[C]{\textit{From the Universal $\xi$-Constant to Multidimensional Networks and Factorization}}
\renewcommand{\headrulewidth}{0.4pt}
\renewcommand{\footrulewidth}{0.4pt}
\setlength{\headheight}{15pt}  % Fix for headheight warning

\hypersetup{
	colorlinks=true,
	linkcolor=blue,
	citecolor=blue,
	urlcolor=blue,
	pdftitle={T0-Theory: Network Representation and Dimensional Analysis in the T0-Theory},
	pdfauthor={Johann Pascher},
	pdfsubject={T0-Theory, Networks, Dimensional Effects, Factorization, $\xi$-Parameter, Neural Networks}
}

% Custom environments (adapted and extended from example)
\newtcolorbox{important}[1][]{colback=yellow!10!white,colframe=yellow!50!black,fonttitle=\bfseries,title=Important Note,#1}
\newtcolorbox{formula}[1][]{colback=blue!5!white,colframe=blue!75!black,fonttitle=\bfseries,title=Key Formula,#1}
\newtcolorbox{revolutionary}[1][]{colback=red!5!white,colframe=red!75!black,fonttitle=\bfseries,title=Revolutionary Insight,#1}
\newtcolorbox{experiment}[1][]{colback=green!5!white,colframe=green!75!black,fonttitle=\bfseries,title=Experimental Test,#1}
\newtcolorbox{sibox}[1][]{colback=orange!10!white,colframe=orange!75!black,fonttitle=\bfseries,title=SI-Units (for reference only),#1}

% Define common mathematical symbols for consistent usage (retained from original)
\newcommand{\xipar}{\ensuremath{\xi}}
\newcommand{\deltafield}{\ensuremath{\delta m}}
\newcommand{\partialop}{\ensuremath{\partial}}
\newcommand{\lambdah}{\ensuremath{\lambda_h}}
\newcommand{\betaT}{\ensuremath{\beta_T}}
\newcommand{\alphaEM}{\ensuremath{\alpha_{\text{EM}}}}
\newcommand{\rhofield}{\ensuremath{\rho}}
\newcommand{\mypi}{\ensuremath{\pi}}
\newcommand{\myphi}{\ensuremath{\phi}}
\newcommand{\myomega}{\ensuremath{\omega}}
\newcommand{\mytimes}{\ensuremath{\times}}
\newcommand{\myapprox}{\ensuremath{\approx}}
\newcommand{\myrightarrow}{\ensuremath{\rightarrow}}
\newcommand{\myRightarrow}{\ensuremath{\Rightarrow}}
\newcommand{\mypropto}{\ensuremath{\propto}}
\newcommand{\mysim}{\ensuremath{\sim}}
\newcommand{\mysqrt}{\ensuremath{\sqrt}}

\title{\Huge\textbf{T0-Theory: Network Representation and Dimensional Analysis}\\
	\Large Mathematical Framework, Dimensional Effects, and Factorization Applications}
\author{Johann Pascher\\
	Department of Communication Engineering, \\Higher Technical Federal Teaching and Research Institute (HTL), Leonding, Austria\\
	\texttt{johann.pascher@gmail.com}}
\date{\today}



==================================================

=== T0_penrose_De.tex.preamble ===

\documentclass[12pt,a4paper]{article}
\usepackage[utf8]{inputenc}
\usepackage[T1]{fontenc}
\usepackage[ngerman]{babel}
\usepackage[left=2.5cm,right=2.5cm,top=2.5cm,bottom=2.5cm]{geometry}
\usepackage{amsmath}
\usepackage{amssymb}
\usepackage{hyperref}
\usepackage{booktabs}
\usepackage{siunitx}
\usepackage[table]{xcolor}
\usepackage{physics}
\usepackage{fancyhdr}
\usepackage{tocloft}
\usepackage{breakurl}
\usepackage{float}
\usepackage{tikz}
% Header and Footer Configuration
\pagestyle{fancy}
\fancyhf{}
\fancyhead[L]{Johann Pascher}
\fancyhead[R]{T0-Theorie: Der Terrell-Penrose-Effekt und Massenvariation}
\fancyfoot[C]{\thepage}
\renewcommand{\headrulewidth}{0.4pt}
\renewcommand{\footrulewidth}{0.4pt}
\setlength{\headheight}{15pt}
\definecolor{blue}{rgb}{0,0,1}
\renewcommand{\cftsecfont}{\color{blue}}
\renewcommand{\cftsecpagefont}{\color{blue}}
\hypersetup{
	colorlinks=true,
	linkcolor=blue,
	citecolor=blue,
	urlcolor=blue,
	pdftitle={T0-Theorie: Der Terrell-Penrose-Effekt und Massenvariation},
	pdfauthor={Johann Pascher}
}
% Fix for Unicode ξ in text mode
\DeclareUnicodeCharacter{03BE}{$\xi$}
\DeclareUnicodeCharacter{03B1}{$\alpha$}
\title{\textbf{T0-Theorie: Der Terrell-Penrose-Effekt und Massenvariation}\\
	\Large Fraktal-konformale Erweiterungen und experimentelle Evidenz}
\author{Johann Pascher\\
	Department for Communication Technology\\
	Higher Technical College (HTL), Leonding, Austria\\
	\texttt{johann.pascher@gmail.com}}
\date{\today}


==================================================

=== T0_penrose_En.tex.preamble ===

\documentclass[12pt,a4paper]{article}
\usepackage[utf8]{inputenc}
\usepackage[T1]{fontenc}
\usepackage[english]{babel}
\usepackage[left=2.5cm,right=2.5cm,top=2.5cm,bottom=2.5cm]{geometry}
\usepackage{amsmath}
\usepackage{amssymb}
\usepackage{hyperref}
\usepackage{booktabs}
\usepackage{siunitx}
\usepackage[table]{xcolor}
\usepackage{physics}
\usepackage{fancyhdr}
\usepackage{tocloft}
\usepackage{breakurl}
\usepackage{float}
\usepackage{tikz}
% Header and Footer Configuration
\pagestyle{fancy}
\fancyhf{}
\fancyhead[L]{Johann Pascher}
\fancyhead[R]{T0-Theory: Mass Variation as an Equivalent to Time Dilation}
\fancyfoot[C]{\thepage}
\renewcommand{\headrulewidth}{0.4pt}
\renewcommand{\footrulewidth}{0.4pt}
\setlength{\headheight}{15pt}
\definecolor{blue}{rgb}{0,0,1}
\renewcommand{\cftsecfont}{\color{blue}}
\renewcommand{\cftsecpagefont}{\color{blue}}
\hypersetup{
	colorlinks=true,
	linkcolor=blue,
	citecolor=blue,
	urlcolor=blue,
	pdftitle={T0-Theory: Mass Variation as an Equivalent to Time Dilation},
	pdfauthor={Johann Pascher}
}
% Fix for Unicode ξ in text mode
\DeclareUnicodeCharacter{03BE}{$\xi$}
\title{\textbf{T0-Theory: Mass Variation as an Equivalent to Time Dilation}\\
	\Large Fractal-Conformal Extensions and Experimental Evidence}
\author{Johann Pascher\\
	Department for Communication Technology\\
	Higher Technical College (HTL), Leonding, Austria\\
	\texttt{johann.pascher@gmail.com}}
\date{\today}


==================================================

=== T0_peratt_De.tex.preamble ===

\documentclass[12pt,a4paper]{article}

% --- Grundlegende Pakete ---
\usepackage[utf8]{inputenc}
\usepackage[T1]{fontenc}
\usepackage[ngerman]{babel}
\usepackage{lmodern}
\usepackage{amsmath,amssymb,amsthm}
\usepackage{physics}
\usepackage{siunitx}
\usepackage{listings}
\usepackage{xcolor}

% --- Seitenlayout und Design ---
\usepackage[margin=2.5cm]{geometry}
\usepackage{fancyhdr}
\usepackage{hyperref}
\usepackage{graphicx}
\usepackage{booktabs}
\usepackage{enumitem}

% --- Hyperref-Konfiguration ---
\hypersetup{
	colorlinks=true,
	linkcolor=blue,
	citecolor=blue,
	urlcolor=blue,
	pdftitle={Mathematische Konstrukte alternativer CMB-Modelle: Unnikrishnan und Peratt im Einklang mit der T0-Theorie},
	pdfauthor={Johann Pascher},
	pdfsubject={T0-Theorie, Kosmische Relativität, Plasma-Kosmologie, CMB-Power-Spektrum}
}

% --- Kopf- und Fußzeile ---
\pagestyle{fancy}
\fancyhf{}
\fancyhead[L]{\textsc{T0-Theorie: Mathematische Integration alternativer Modelle}}
\fancyhead[R]{\textsc{J. Pascher}}
\fancyfoot[C]{\thepage}
\renewcommand{\headrulewidth}{0.4pt}
\setlength{\headheight}{15pt}

% --- Mathematische Befehle ---
\newcommand{\xiT}{\xi}
\newcommand{\Riem}{R^\rho{}_{\sigma\mu\nu}}
\newcommand{\Weyl}{C^\rho{}_{\sigma\mu\nu}}
\newcommand{\Ricci}{R_{\mu\nu}}
\newcommand{\Scal}{R}
\newcommand{\Lorentz}[2]{{\Lambda^\mu{}_\nu}(#1,#2)}
\newcommand{\SynchPower}{P_{\text{synch}}}
\newcommand{\ZPinch}{J \times B = \nabla p}

% --- Titel-Informationen ---
\title{\textbf{Mathematische Konstrukte alternativer CMB-Modelle: Unnikrishnan und Peratt im Einklang mit der T0-Theorie}\\[0.5cm]
	\large Eine detaillierte Analyse der Feldgleichungen und ihre Synthese mit dem $\xi$-Feld}
\author{Johann Pascher}
\date{17. November 2025}



==================================================

=== T0_peratt_En.tex.preamble ===

\documentclass[12pt,a4paper]{article}

% --- Basic Packages ---
\usepackage[utf8]{inputenc}
\usepackage[T1]{fontenc}
\usepackage[english]{babel}
\usepackage{lmodern}
\usepackage{amsmath,amssymb,amsthm}
\usepackage{physics}
\usepackage{siunitx}
\usepackage{listings}
\usepackage{xcolor}

% --- Page Layout and Design ---
\usepackage[margin=2.5cm]{geometry}
\usepackage{fancyhdr}
\usepackage{hyperref}
\usepackage{graphicx}
\usepackage{booktabs}
\usepackage{enumitem}

% --- Hyperref Configuration ---
\hypersetup{
	colorlinks=true,
	linkcolor=blue,
	citecolor=blue,
	urlcolor=blue,
	pdftitle={Mathematical Constructs of Alternative CMB Models: Unnikrishnan and Peratt in Harmony with the T0 Theory},
	pdfauthor={Johann Pascher},
	pdfsubject={T0 Theory, Cosmic Relativity, Plasma Cosmology, CMB Power Spectrum}
}

% --- Header and Footer ---
\pagestyle{fancy}
\fancyhf{}
\fancyhead[L]{\textsc{T0 Theory: Mathematical Integration of Alternative Models}}
\fancyhead[R]{\textsc{J. Pascher}}
\fancyfoot[C]{\thepage}
\renewcommand{\headrulewidth}{0.4pt}
\setlength{\headheight}{15pt}

% --- Mathematical Commands ---
\newcommand{\xiT}{\xi}
\newcommand{\Riem}{R^\rho{}_{\sigma\mu\nu}}
\newcommand{\Weyl}{C^\rho{}_{\sigma\mu\nu}}
\newcommand{\Ricci}{R_{\mu\nu}}
\newcommand{\Scal}{R}
\newcommand{\Lorentz}[2]{{\Lambda^\mu{}_\nu}(#1,#2)}
\newcommand{\SynchPower}{P_{\text{synch}}}
\newcommand{\ZPinch}{J \times B = \nabla p}

% --- Title Information ---
\title{\textbf{Mathematical Constructs of Alternative CMB Models: Unnikrishnan and Peratt in Harmony with the T0 Theory}\\[0.5cm]
	\large A Detailed Analysis of the Field Equations and Their Synthesis with the $\xi$-Field}
\author{Johann Pascher}
\date{November 17, 2025}



==================================================

=== T0_photonenchip-china_De.tex.preamble ===

\documentclass[12pt,a4paper]{article}
\usepackage[utf8]{inputenc}
\usepackage[T1]{fontenc}
\usepackage[ngerman]{babel}
\usepackage[left=2cm,right=2cm,top=2cm,bottom=2cm]{geometry}
\usepackage{lmodern}
\usepackage{amsmath}
\usepackage{amssymb}
\usepackage{amsfonts}
\usepackage{physics}
\usepackage{hyperref}
\usepackage{tcolorbox} % Für tcolorbox-Umgebungen
\usepackage{booktabs}
\usepackage{enumitem}
\usepackage[table,xcdraw]{xcolor}
\usepackage{longtable}
\usepackage{siunitx}
\usepackage{fancyhdr}
\usepackage{textgreek}
\usepackage{graphicx} % Für Abbildungen
\usepackage{listings} % Für Code-Snippets
\usepackage{caption} % Für Robustheit

% Header und Footer
\pagestyle{fancy}
\fancyhf{}
\fancyhead[L]{T0-Theorie: Photonische Quantenchips}
\fancyhead[R]{\thepage}
\fancyfoot[C]{\textit{Integration der T0-Dualität in hybride Quanten-AI-Hardware}}
\renewcommand{\headrulewidth}{0.4pt}
\renewcommand{\footrulewidth}{0.4pt}
\setlength{\headheight}{15pt}

\hypersetup{
	colorlinks=true,
	linkcolor=blue,
	citecolor=blue,
	urlcolor=blue,
	pdftitle={Chinas Photonischer Quantenchip: 1000x-Speedup und T0-Integration},
	pdfauthor={Johann Pascher},
	pdfsubject={T0-Theorie, Photonische Quantencomputing, AI-Beschleunigung, $\xi$-Dualität}
}

% Custom environments (T0-Stil)
\tcbuselibrary{skins,breakable}
\newtcolorbox{important}[1][]{colback=yellow!10!white,colframe=yellow!50!black,fonttitle=\bfseries,title=Wichtiger Hinweis,#1}
\newtcolorbox{formula}[1][]{colback=blue!5!white,colframe=blue!75!black,fonttitle=\bfseries,title=Schlüsselformel,#1}
\newtcolorbox{experiment}[1][]{colback=green!5!white,colframe=green!75!black,fonttitle=\bfseries,title=Experimenteller Test,#1}

% Gängige Symbole (T0-Stil)
\newcommand{\xipar}{\ensuremath{\xi}}
\newcommand{\deltafield}{\ensuremath{\delta m}}
\newcommand{\partialop}{\ensuremath{\partial}}
\newcommand{\lambdah}{\ensuremath{\lambda_h}}
\newcommand{\betaT}{\ensuremath{\beta_T}}
\newcommand{\alphaEM}{\ensuremath{\alpha_{\text{EM}}}}
\newcommand{\rhofield}{\ensuremath{\rho}}
\newcommand{\mypi}{\ensuremath{\pi}}
\newcommand{\myphi}{\ensuremath{\phi}}
\newcommand{\myomega}{\ensuremath{\omega}}
\newcommand{\mytimes}{\ensuremath{\times}}
\newcommand{\myapprox}{\ensuremath{\approx}}
\newcommand{\myrightarrow}{\ensuremath{\rightarrow}}
\newcommand{\myRightarrow}{\ensuremath{\Rightarrow}}
\newcommand{\mypropto}{\ensuremath{\propto}}
\newcommand{\mysim}{\ensuremath{\sim}}
\newcommand{\mysqrt}{\ensuremath{\sqrt}}
\newcommand{\Kfrak}{K_{\text{frak}}}

% Listings für Code
\lstset{
	language=Python,
	basicstyle=\ttfamily\small,
	keywordstyle=\color{blue},
	stringstyle=\color{red},
	commentstyle=\color{green},
	breaklines=true,
	frame=single
}

\title{\Huge\textbf{T0-Theorie: Chinas Photonischer Quantenchip – 1000x-Speedup für AI}\\
	\large Integration fraktaler Dualität in hybride Quanten-Hardware}
\author{Johann Pascher\\
	Abteilung für Kommunikationstechnik, \\Höhere Technische Bundeslehr- und Versuchsanstalt (HTL), Leonding, Österreich\\
	\texttt{johann.pascher@gmail.com}}
\date{17. November 2025}



==================================================

=== T0_photonenchip-china_En.tex.preamble ===

\documentclass[12pt,a4paper]{article}
\usepackage[utf8]{inputenc}
\usepackage[T1]{fontenc}
\usepackage[ngerman]{babel}
\usepackage[left=2cm,right=2cm,top=2cm,bottom=2cm]{geometry}
\usepackage{lmodern}
\usepackage{amsmath}
\usepackage{amssymb}
\usepackage{amsfonts}
\usepackage{physics}
\usepackage{hyperref}
\usepackage{tcolorbox} % Für tcolorbox-Umgebungen
\usepackage{booktabs}
\usepackage{enumitem}
\usepackage[table,xcdraw]{xcolor}
\usepackage{longtable}
\usepackage{siunitx}
\usepackage{fancyhdr}
\usepackage{textgreek}
\usepackage{graphicx} % Für Abbildungen
\usepackage{listings} % Für Code-Snippets
\usepackage{caption} % Für Robustheit

% Header und Footer
\pagestyle{fancy}
\fancyhf{}
\fancyhead[L]{T0-Theorie: Photonische Quantenchips}
\fancyhead[R]{\thepage}
\fancyfoot[C]{\textit{Integration der T0-Dualität in hybride Quanten-AI-Hardware}}
\renewcommand{\headrulewidth}{0.4pt}
\renewcommand{\footrulewidth}{0.4pt}
\setlength{\headheight}{15pt}

\hypersetup{
	colorlinks=true,
	linkcolor=blue,
	citecolor=blue,
	urlcolor=blue,
	pdftitle={Chinas Photonischer Quantenchip: 1000x-Speedup und T0-Integration},
	pdfauthor={Johann Pascher},
	pdfsubject={T0-Theorie, Photonische Quantencomputing, AI-Beschleunigung, $\xi$-Dualität}
}

% Custom environments (T0-Stil)
\tcbuselibrary{skins,breakable}
\newtcolorbox{important}[1][]{colback=yellow!10!white,colframe=yellow!50!black,fonttitle=\bfseries,title=Wichtiger Hinweis,#1}
\newtcolorbox{formula}[1][]{colback=blue!5!white,colframe=blue!75!black,fonttitle=\bfseries,title=Schlüsselformel,#1}
\newtcolorbox{experiment}[1][]{colback=green!5!white,colframe=green!75!black,fonttitle=\bfseries,title=Experimenteller Test,#1}

% Gängige Symbole (T0-Stil)
\newcommand{\xipar}{\ensuremath{\xi}}
\newcommand{\deltafield}{\ensuremath{\delta m}}
\newcommand{\partialop}{\ensuremath{\partial}}
\newcommand{\lambdah}{\ensuremath{\lambda_h}}
\newcommand{\betaT}{\ensuremath{\beta_T}}
\newcommand{\alphaEM}{\ensuremath{\alpha_{\text{EM}}}}
\newcommand{\rhofield}{\ensuremath{\rho}}
\newcommand{\mypi}{\ensuremath{\pi}}
\newcommand{\myphi}{\ensuremath{\phi}}
\newcommand{\myomega}{\ensuremath{\omega}}
\newcommand{\mytimes}{\ensuremath{\times}}
\newcommand{\myapprox}{\ensuremath{\approx}}
\newcommand{\myrightarrow}{\ensuremath{\rightarrow}}
\newcommand{\myRightarrow}{\ensuremath{\Rightarrow}}
\newcommand{\mypropto}{\ensuremath{\propto}}
\newcommand{\mysim}{\ensuremath{\sim}}
\newcommand{\mysqrt}{\ensuremath{\sqrt}}
\newcommand{\Kfrak}{K_{\text{frak}}}

% Listings für Code
\lstset{
	language=Python,
	basicstyle=\ttfamily\small,
	keywordstyle=\color{blue},
	stringstyle=\color{red},
	commentstyle=\color{green},
	breaklines=true,
	frame=single
}

\title{\Huge\textbf{T0-Theorie: Chinas Photonischer Quantenchip – 1000x-Speedup für AI}\\
	\large Integration fraktaler Dualität in hybride Quanten-Hardware}
\author{Johann Pascher\\
	Abteilung für Kommunikationstechnik, \\Höhere Technische Bundeslehr- und Versuchsanstalt (HTL), Leonding, Österreich\\
	\texttt{johann.pascher@gmail.com}}
\date{17. November 2025}



==================================================

=== T0_photonenchip-einführung_De.tex.preamble ===

\documentclass[12pt,a4paper]{article}
\usepackage[utf8]{inputenc}
\usepackage[T1]{fontenc}
\usepackage[ngerman]{babel}
\usepackage[left=2cm,right=2cm,top=2cm,bottom=2cm]{geometry}
\usepackage{lmodern}
\usepackage{amsmath}
\usepackage{amssymb}
\usepackage{amsfonts}
\usepackage{physics}
\usepackage{hyperref}
\usepackage{booktabs}
\usepackage{enumitem}
\usepackage[table,xcdraw]{xcolor}
\usepackage{longtable}
\usepackage{siunitx}
\usepackage{fancyhdr}
\usepackage{textgreek}
\usepackage{graphicx} % Für Abbildungen (falls erweitert)
\usepackage{listings} % Für Code-Snippets (optional)
\usepackage{caption} % Für Robustheit

% Header und Footer
\pagestyle{fancy}
\fancyhf{}
\fancyhead[L]{Einführung: Photonische Quantenchips}
\fancyhead[R]{\thepage}
\fancyfoot[C]{\textit{Für Nachrichtentechniker – Latenzarme Signalverarbeitung}}
\renewcommand{\headrulewidth}{0.4pt}
\renewcommand{\footrulewidth}{0.4pt}
\setlength{\headheight}{15pt}

\hypersetup{
	colorlinks=true,
	linkcolor=blue,
	citecolor=blue,
	urlcolor=blue,
	pdftitle={Einführung in photonische Quantenchips für Nachrichtentechniker},
	pdfauthor={Johann Pascher},
	pdfsubject={Photonik, Quantencomputing, 6G, Signalverarbeitung}
}

% Custom environments (einfach gehalten)
\newenvironment{important}{\begin{quote}\textbf{Wichtiger Hinweis:}}{\end{quote}}
\newenvironment{formula}{\begin{quote}\textbf{Schlüsselformel:}}{\end{quote}}

% Listings (falls benötigt)
\lstset{
	language=Python,
	basicstyle=\ttfamily\small,
	keywordstyle=\color{blue},
	stringstyle=\color{red},
	commentstyle=\color{green},
	breaklines=true,
	frame=single
}

\title{\Huge\textbf{Einführung in photonische Quantenchips für Nachrichtentechniker}\\
	\large Analoge Realisierungen und Operationen für 6G-Signalverarbeitung}
\author{Johann Pascher\\
	Abteilung für Kommunikationstechnik, \\Höhere Technische Bundeslehr- und Versuchsanstalt (HTL), Leonding, Österreich\\
	\texttt{johann.pascher@gmail.com}}
\date{17. November 2025}



==================================================

=== T0_photonenchip-einführung_En.tex.preamble ===

\documentclass[12pt,a4paper]{article}
\usepackage[utf8]{inputenc}
\usepackage[T1]{fontenc}
\usepackage[english]{babel}
\usepackage[left=2cm,right=2cm,top=2cm,bottom=2cm]{geometry}
\usepackage{lmodern}
\usepackage{amsmath}
\usepackage{amssymb}
\usepackage{amsfonts}
\usepackage{physics}
\usepackage{hyperref}
\usepackage{booktabs}
\usepackage{enumitem}
\usepackage[table,xcdraw]{xcolor}
\usepackage{longtable}
\usepackage{siunitx}
\usepackage{fancyhdr}
\usepackage{textgreek}
\usepackage{graphicx} % For figures (if extended)
\usepackage{listings} % For code snippets (optional)
\usepackage{caption} % For robustness

% Header and Footer
\pagestyle{fancy}
\fancyhf{}
\fancyhead[L]{Introduction: Photonic Quantum Chips}
\fancyhead[R]{\thepage}
\fancyfoot[C]{\textit{For Communication Engineers – Low-Latency Signal Processing}}
\renewcommand{\headrulewidth}{0.4pt}
\renewcommand{\footrulewidth}{0.4pt}
\setlength{\headheight}{15pt}

\hypersetup{
	colorlinks=true,
	linkcolor=blue,
	citecolor=blue,
	urlcolor=blue,
	pdftitle={Introduction to Photonic Quantum Chips for Communication Engineers},
	pdfauthor={Johann Pascher},
	pdfsubject={Photonics, Quantum Computing, 6G, Signal Processing}
}

% Custom environments (kept simple)
\newenvironment{important}{\begin{quote}\textbf{Important Note:}}{\end{quote}}
\newenvironment{formula}{\begin{quote}\textbf{Key Formula:}}{\end{quote}}

% Listings (if needed)
\lstset{
	language=Python,
	basicstyle=\ttfamily\small,
	keywordstyle=\color{blue},
	stringstyle=\color{red},
	commentstyle=\color{green},
	breaklines=true,
	frame=single
}

\title{\Huge\textbf{Introduction to Photonic Quantum Chips for Communication Engineers}\\
	\large Analog Realizations and Operations for 6G Signal Processing}
\author{Johann Pascher\\
	Department of Communication Technology, \\Higher Technical Federal Teaching and Research Institute (HTL), Leonding, Austria\\
	\texttt{johann.pascher@gmail.com}}
\date{November 17, 2025}



==================================================

=== T0_photonenchip-umsetzung_De.tex.preamble ===

\documentclass[12pt,a4paper]{article}
\usepackage[utf8]{inputenc}
\usepackage[T1]{fontenc}
\usepackage[ngerman]{babel}
\usepackage[left=2cm,right=2cm,top=2cm,bottom=2cm]{geometry}
\usepackage{lmodern}
\usepackage{amsmath}
\usepackage{amssymb}
\usepackage{amsfonts}
\usepackage{physics}
\usepackage{hyperref}
\usepackage{booktabs}
\usepackage{enumitem}
\usepackage[table,xcdraw]{xcolor}
\usepackage{longtable}
\usepackage{siunitx}
\usepackage{fancyhdr}
\usepackage{textgreek}
\usepackage{graphicx} % Für Abbildungen (falls erweitert)
\usepackage{listings} % Für Code-Snippets (optional)
\usepackage{caption} % Für Robustheit

% Header und Footer
\pagestyle{fancy}
\fancyhf{}
\fancyhead[L]{Einführung: Photonische Wafer-Fabrikation}
\fancyhead[R]{\thepage}
\fancyfoot[C]{\textit{Für Nachrichtentechniker – Bauteil-Integration auf Wafern}}
\renewcommand{\headrulewidth}{0.4pt}
\renewcommand{\footrulewidth}{0.4pt}
\setlength{\headheight}{15pt}

\hypersetup{
	colorlinks=true,
	linkcolor=blue,
	citecolor=blue,
	urlcolor=blue,
	pdftitle={Einführung in die Umsetzung photonischer Bauteile auf Wafern für Nachrichtentechniker},
	pdfauthor={Johann Pascher},
	pdfsubject={Photonische Integration, Wafer-Fabrikation, 6G, Signalverarbeitung}
}

% Custom environments (einfach gehalten)
\newenvironment{important}{\begin{quote}\textbf{Wichtiger Hinweis:}}{\end{quote}}
\newenvironment{formula}{\begin{quote}\textbf{Schlüsselformel:}}{\end{quote}}

% Listings (falls benötigt)
\lstset{
	language=Python,
	basicstyle=\ttfamily\small,
	keywordstyle=\color{blue},
	stringstyle=\color{red},
	commentstyle=\color{green},
	breaklines=true,
	frame=single
}

\title{\Huge\textbf{Einführung in die Umsetzung photonischer Bauteile auf Wafern}\\
	\large Für Nachrichtentechniker: Von TFLN-Wafern bis 6G-Integration (2024–2025)}
\author{Johann Pascher\\
	Abteilung für Kommunikationstechnik, \\Höhere Technische Bundeslehr- und Versuchsanstalt (HTL), Leonding, Österreich\\
	\texttt{johann.pascher@gmail.com}}
\date{17. November 2025}



==================================================

=== T0_photonenchip-umsetzung_En.tex.preamble ===


\documentclass[12pt,a4paper]{article}
\usepackage[utf8]{inputenc}
\usepackage[T1]{fontenc}
\usepackage[english]{babel}
\usepackage[left=2cm,right=2cm,top=2cm,bottom=2cm]{geometry}
\usepackage{lmodern}
\usepackage{amsmath}
\usepackage{amssymb}
\usepackage{amsfonts}
\usepackage{physics}
\usepackage{hyperref}
\usepackage{booktabs}
\usepackage{enumitem}
\usepackage[table,xcdraw]{xcolor}
\usepackage{longtable}
\usepackage{siunitx}
\usepackage{fancyhdr}
\usepackage{textgreek}
\usepackage{graphicx} % For figures (if extended)
\usepackage{listings} % For code snippets (optional)
\usepackage{caption} % For robustness

% Header and Footer
\pagestyle{fancy}
\fancyhf{}
\fancyhead[L]{Introduction: Photonic Wafer Fabrication}
\fancyhead[R]{\thepage}
\fancyfoot[C]{\textit{For Communication Engineers – Component Integration on Wafers}}
\renewcommand{\headrulewidth}{0.4pt}
\renewcommand{\footrulewidth}{0.4pt}
\setlength{\headheight}{15pt}

\hypersetup{
	colorlinks=true,
	linkcolor=blue,
	citecolor=blue,
	urlcolor=blue,
	pdftitle={Introduction to the Implementation of Photonic Components on Wafers for Communication Engineers},
	pdfauthor={Johann Pascher},
	pdfsubject={Photonic Integration, Wafer Fabrication, 6G, Signal Processing}
}

% Custom environments (kept simple)
\newenvironment{important}{\begin{quote}\textbf{Important Note:}}{\end{quote}}
\newenvironment{formula}{\begin{quote}\textbf{Key Formula:}}{\end{quote}}

% Listings (if needed)
\lstset{
	language=Python,
	basicstyle=\ttfamily\small,
	keywordstyle=\color{blue},
	stringstyle=\color{red},
	commentstyle=\color{green},
	breaklines=true,
	frame=single
}

\title{\Huge\textbf{Introduction to the Implementation of Photonic Components on Wafers}\\
	\large For Communication Engineers: From TFLN Wafers to 6G Integration (2024–2025)}
\author{Johann Pascher\\
	Department of Communication Technology, \\Higher Technical Federal Teaching and Research Institute (HTL), Leonding, Austria\\
	\texttt{johann.pascher@gmail.com}}
\date{November 17, 2025}



==================================================

=== T0_threeclock_De.tex.preamble ===

\documentclass[12pt,a4paper]{article}
\usepackage[utf8]{inputenc}
\usepackage[T1]{fontenc}
\usepackage[ngerman]{babel}
\usepackage{amsmath,amssymb,amsthm}
\usepackage{geometry}
\usepackage{booktabs}
\usepackage{array}
\usepackage{hyperref}
\usepackage{tocloft}
\usepackage{fancyhdr}
\usepackage{graphicx}
\usepackage{physics}
\usepackage{siunitx}
\usepackage{enumitem}

\geometry{a4paper, margin=2.5cm}

\pagestyle{fancy}
\fancyhf{}
\fancyhead[L]{T0: Ein-Uhr-Metrologie}
\fancyhead[R]{Scientific Reports 2024}
\fancyfoot[C]{\thepage}
\renewcommand{\headrulewidth}{0.4pt}
\renewcommand{\footrulewidth}{0.4pt}
\setlength{\headheight}{14.5pt}

\renewcommand{\cfttoctitlefont}{\huge\bfseries}
\renewcommand{\cftsecfont}{\bfseries}
\renewcommand{\cftsecpagefont}{\bfseries}
\setlength{\cftsecindent}{0pt}
\setlength{\cftsubsecindent}{0pt}

\hypersetup{
  colorlinks=true,
  linkcolor=blue,
  citecolor=blue,
  urlcolor=blue,
  pdftitle={T0-Theorie: Ein-Uhr-Metrologie und Drei-Uhren-Experiment},
  pdfauthor={Johann Pascher},
  pdfsubject={T0-Theorie, SI-Einheiten, Drei-Uhren-Experiment, Ein-Uhr-Metrologie}
}

\title{\textbf{Ein-Uhr-Metrologie und Drei-Uhren-Experiment}\\[0.5cm]
  \large Das Scientific-Reports-Paper zur Ein-Uhr-Metrologie\\
  und seine Übereinstimmung mit der T0-Zeit-Masse-Dualität}
\author{Johann Pascher\\
  HTL Leonding, Österreich\\
  \texttt{johann.pascher@gmail.com}}
\date{\today}



==================================================

=== T0_threeclock_En.tex.preamble ===

\documentclass[12pt,a4paper]{article}
\usepackage[utf8]{inputenc}
\usepackage[T1]{fontenc}
\usepackage[english]{babel}
\usepackage{amsmath,amssymb,amsthm}
\usepackage{geometry}
\usepackage{booktabs}
\usepackage{array}
\usepackage{hyperref}
\usepackage{tocloft}
\usepackage{fancyhdr}
\usepackage{graphicx}
\usepackage{physics}
\usepackage{siunitx}
\usepackage{enumitem}

\geometry{a4paper, margin=2.5cm}

\pagestyle{fancy}
\fancyhf{}
\fancyhead[L]{T0: Single-Clock Metrology}
\fancyhead[R]{Scientific Reports 2024}
\fancyfoot[C]{\thepage}
\renewcommand{\headrulewidth}{0.4pt}
\renewcommand{\footrulewidth}{0.4pt}
\setlength{\headheight}{14.5pt}

\renewcommand{\cfttoctitlefont}{\huge\bfseries}
\renewcommand{\cftsecfont}{\bfseries}
\renewcommand{\cftsecpagefont}{\bfseries}
\setlength{\cftsecindent}{0pt}
\setlength{\cftsubsecindent}{0pt}

\hypersetup{
  colorlinks=true,
  linkcolor=blue,
  citecolor=blue,
  urlcolor=blue,
  pdftitle={T0-Theory: Single-Clock Metrology and Three-Clock Experiment},
  pdfauthor={Johann Pascher},
  pdfsubject={T0-Theory, SI units, Three-Clock Experiment, Single-Clock Metrology}
}

\title{\textbf{Single-Clock Metrology and the Three-Clock Experiment}\\[0.5cm]
  \large The Scientific Reports paper on single-clock metrology\\
  and its agreement with the T0 time–mass duality}
\author{Johann Pascher\\
  HTL Leonding, Austria\\
  \texttt{johann.pascher@gmail.com}}
\date{\today}



==================================================

=== T0_tm-erweiterung-x6_De.tex.preamble ===

\documentclass[12pt,a4paper]{article}
\usepackage[utf8]{inputenc}
\usepackage[T1]{fontenc}
\usepackage[ngerman]{babel}
\usepackage{lmodern}
\usepackage{amsmath}
\usepackage{amssymb}
\usepackage{hyperref}
\usepackage{booktabs}
\usepackage{enumitem}
\usepackage[table,xcdraw]{xcolor}
\usepackage{newunicodechar}
\usepackage{fancyhdr}
\usepackage{siunitx}
\usepackage{physics}
\usepackage{tcolorbox}
\usepackage{geometry}
\usepackage{graphicx}
\usepackage{float}
\usepackage{mathtools}
\usepackage{amsthm}
\usepackage{microtype}
\usepackage{array}

% Unicode setups for Greek letters
\newunicodechar{ξ}{\ensuremath{\xi}}
\newunicodechar{μ}{\ensuremath{\mu}}

\geometry{left=2.5cm,right=2.5cm,top=2.5cm,bottom=2.5cm}

\hypersetup{
	colorlinks=true,
	linkcolor=blue,
	citecolor=blue,
	urlcolor=blue,
	pdftitle={T0-Theorie: Finale Fraktale Massenformeln (November 2025)},
	pdfauthor={Johann Pascher},
	pdfsubject={Theoretische Physik, T0 Theorie, Fraktale Massenformeln}
}

% Header- und Footer-Konfiguration
\pagestyle{fancy}
\fancyhf{}
\fancyhead[L]{Johann Pascher}
\fancyhead[R]{T0-Theorie: Finale Fraktale Massenformeln}
\fancyfoot[C]{\thepage}
\renewcommand{\headrulewidth}{0.4pt}
\renewcommand{\footrulewidth}{0.4pt}

% Tcolorbox-Stile
\tcbuselibrary{theorems}
\newtcolorbox{units}{colback=blue!5!white,colframe=blue!75!black,fonttitle=\bfseries}
\newtcolorbox{important}{colback=green!5!white,colframe=green!35!black,fonttitle=\bfseries}
\newtcolorbox{summary}{colback=yellow!5!white,colframe=orange!75!black,fonttitle=\bfseries}

\title{\textbf{T0-Theorie: Finale Fraktale Massenformeln (November 2025, $<$3\% $\Delta$)}\\[0.5cm]
	\large Zwei komplementäre Methoden zur parameterfreien Massenberechnung\\[0.3cm]
	\normalsize Erweiterte Dokumentation der T0-Massentheorie}
\author{Johann Pascher\\
	Abteilung für Kommunikationstechnologie\\
	Höhere Technische Lehranstalt (HTL), Leonding, Österreich\\
	\texttt{johann.pascher@gmail.com}}
\date{\today}



==================================================

=== T0_tm-erweiterung-x6_En.tex.preamble ===

\documentclass[12pt,a4paper]{article}
\usepackage[utf8]{inputenc}
\usepackage[T1]{fontenc}
\usepackage[english]{babel}
\usepackage{lmodern}
\usepackage{amsmath}
\usepackage{amssymb}
\usepackage{hyperref}
\usepackage{booktabs}
\usepackage{enumitem}
\usepackage[table,xcdraw]{xcolor}
\usepackage{newunicodechar}
\usepackage{fancyhdr}
\usepackage{siunitx}
\usepackage{physics}
\usepackage{tcolorbox}
\usepackage{geometry}
\usepackage{graphicx}
\usepackage{float}
\usepackage{mathtools}
\usepackage{amsthm}
\usepackage{microtype}
\usepackage{array}

% Unicode setups for Greek letters
\newunicodechar{ξ}{\ensuremath{\xi}}
\newunicodechar{μ}{\ensuremath{\mu}}

\geometry{left=2.5cm,right=2.5cm,top=2.5cm,bottom=2.5cm}

\hypersetup{
	colorlinks=true,
	linkcolor=blue,
	citecolor=blue,
	urlcolor=blue,
	pdftitle={T0-Theory: Final Fractal Mass Formulas (November 2025)},
	pdfauthor={Johann Pascher},
	pdfsubject={Theoretical Physics, T0 Theory, Fractal Mass Formulas}
}

% Header and Footer Configuration
\pagestyle{fancy}
\fancyhf{}
\fancyhead[L]{Johann Pascher}
\fancyhead[R]{T0-Theory: Final Fractal Mass Formulas}
\fancyfoot[C]{\thepage}
\renewcommand{\headrulewidth}{0.4pt}
\renewcommand{\footrulewidth}{0.4pt}

% Tcolorbox Styles
\tcbuselibrary{theorems}
\newtcolorbox{units}{colback=blue!5!white,colframe=blue!75!black,fonttitle=\bfseries}
\newtcolorbox{important}{colback=green!5!white,colframe=green!35!black,fonttitle=\bfseries}
\newtcolorbox{summary}{colback=yellow!5!white,colframe=orange!75!black,fonttitle=\bfseries}

\title{\textbf{T0-Theory: Final Fractal Mass Formulas (November 2025, $<$3\% $\Delta$)}\\[0.5cm]
	\large Two Complementary Methods for Parameter-Free Mass Calculation\\[0.3cm]
	\normalsize Extended Documentation of the T0 Mass Theory}
\author{Johann Pascher\\
	Department for Communication Technology\\
	Higher Technical College (HTL), Leonding, Austria\\
	\texttt{johann.pascher@gmail.com}}
\date{\today}



==================================================

=== T0_umkehrung_De.tex.preamble ===

\documentclass[12pt,a4paper]{article}
\usepackage[utf8]{inputenc}
\usepackage[T1]{fontenc}
\usepackage[german]{babel}
\usepackage{lmodern}
\usepackage{amsmath}
\usepackage{amssymb}
\usepackage{hyperref}
\usepackage{tcolorbox}
\usepackage{booktabs}
\usepackage{enumitem}
\usepackage{xcolor}
\usepackage[left=2cm,right=2cm,top=2cm,bottom=2cm]{geometry}
\usepackage{graphicx}
\usepackage{float}
\usepackage{fancyhdr}
\usepackage{siunitx}
\usepackage{mathtools}
\usepackage{amsthm}
\usepackage{cleveref}
\usepackage{tocloft}
\usepackage{microtype}
\usepackage{array}

% Kopfhöhe korrigieren
\setlength{\headheight}{14.5pt}

% Custom Commands für Fraktaldimension
\newcommand{\Efield}{E_{\text{Feld}}}
\newcommand{\xigeom}{\xi_{\text{geom}}}
\newcommand{\Tzero}{T_0}
\newcommand{\vecx}{\vec{x}}
\newcommand{\xipar}{\xi}
\newcommand{\Kfrak}{K_{\text{frak}}}

% Header and Footer Configuration
\pagestyle{fancy}
\fancyhf{}
\fancyhead[L]{Johann Pascher}
\fancyhead[R]{T0-Theorie: Fraktaldimension $D_f$ aus Lepton-Massenverhältnis}
\fancyfoot[C]{\thepage}
\renewcommand{\headrulewidth}{0.4pt}
\renewcommand{\footrulewidth}{0.4pt}

% Table of Contents Formatting
\renewcommand{\cftsecfont}{\color{blue}}
\renewcommand{\cftsubsecfont}{\color{blue}}
\renewcommand{\cftsecpagefont}{\color{blue}}
\renewcommand{\cftsubsecpagefont}{\color{blue}}

\hypersetup{
	colorlinks=true,
	linkcolor=blue,
	citecolor=blue,
	urlcolor=blue,
	pdftitle={T0-Theorie: Fraktaldimension aus Lepton-Massenverhältnis},
	pdfauthor={Johann Pascher},
	pdfsubject={T0-Theorie, Fraktaldimension, Lepton-Massen, Geometrische Ableitung},
	pdfkeywords={Fraktaldimension, Lepton-Massenverhältnis, Geometrische Theorie, Parameterfrei}
}

% Theorem Environments
\newtheorem{theorem}{Theorem}[section]
\newtheorem{proposition}[theorem]{Proposition}
\newtheorem{definition}[theorem]{Definition}
\newtheorem{lemma}[theorem]{Lemma}

\tcbuselibrary{theorems}
\newtcbtheorem[number within=section]{important}{Wichtige Erkenntnis}%
{colback=green!5,colframe=green!35!black,fonttitle=\bfseries}{th}
\newtcbtheorem[number within=section]{schluessel}{Schlüssel}%
{colback=blue!5,colframe=blue!75!black,fonttitle=\bfseries}{key}
\newtcbtheorem[number within=section]{result}{Ergebnis}%
{colback=green!5,colframe=green!75!black,fonttitle=\bfseries}{res}
\newtcbtheorem[number within=section]{keyresult}{Schlüsselergebnis}%
{colback=blue!5,colframe=blue!75!black,fonttitle=\bfseries}{key}

\title{T0-Time-Mass-Dualitäts-Theorie: Zwingende Ableitung der Fraktaldimension $D_f$ aus dem Lepton-Massenverhältnis \\
	\large Validierung der geometrischen Grundlagen - Komplementär zu Teilchenmassen\_De.pdf}
\author{Johann Pascher\\
	Abteilung für Kommunikationstechnologie\\
	Höhere Technische Bundeslehranstalt (HTL), Leonding, Österreich\\
	\texttt{johann.pascher@gmail.com}}
\date{31. Oktober 2025}



==================================================

=== T0_umkehrung_En.tex.preamble ===

\documentclass[12pt,a4paper]{article}
\usepackage[utf8]{inputenc}
\usepackage[T1]{fontenc}
\usepackage[english]{babel}
\usepackage{lmodern}
\usepackage{amsmath}
\usepackage{amssymb}
\usepackage{hyperref}
\usepackage{tcolorbox}
\usepackage{booktabs}
\usepackage{enumitem}
\usepackage{xcolor}
\usepackage[left=2cm,right=2cm,top=2cm,bottom=2cm]{geometry}
\usepackage{graphicx}
\usepackage{float}
\usepackage{fancyhdr}
\usepackage{siunitx}
\usepackage{mathtools}
\usepackage{amsthm}
\usepackage{cleveref}
\usepackage{tocloft}
\usepackage{microtype}
\usepackage{array}

% Header height correction
\setlength{\headheight}{14.5pt}

% Custom Commands for fractal dimension
\newcommand{\Efield}{E_{\text{Field}}}
\newcommand{\xigeom}{\xi_{\text{geom}}}
\newcommand{\Tzero}{T_0}
\newcommand{\vecx}{\vec{x}}
\newcommand{\xipar}{\xi}
\newcommand{\Kfrak}{K_{\text{frac}}}

% Header and Footer Configuration
\pagestyle{fancy}
\fancyhf{}
\fancyhead[L]{Johann Pascher}
\fancyhead[R]{T0-Theory: Fractal Dimension $D_f$ from Lepton Mass Ratio}
\fancyfoot[C]{\thepage}
\renewcommand{\headrulewidth}{0.4pt}
\renewcommand{\footrulewidth}{0.4pt}

% Table of Contents Formatting
\renewcommand{\cftsecfont}{\color{blue}}
\renewcommand{\cftsubsecfont}{\color{blue}}
\renewcommand{\cftsecpagefont}{\color{blue}}
\renewcommand{\cftsubsecpagefont}{\color{blue}}

\hypersetup{
	colorlinks=true,
	linkcolor=blue,
	citecolor=blue,
	urlcolor=blue,
	pdftitle={T0-Theory: Fractal Dimension from Lepton Mass Ratio},
	pdfauthor={Johann Pascher},
	pdfsubject={T0-Theory, Fractal Dimension, Lepton Masses, Geometric Derivation},
	pdfkeywords={Fractal Dimension, Lepton Mass Ratio, Geometric Theory, Parameter-Free}
}

% Theorem Environments
\newtheorem{theorem}{Theorem}[section]
\newtheorem{proposition}[theorem]{Proposition}
\newtheorem{definition}[theorem]{Definition}
\newtheorem{lemma}[theorem]{Lemma}

\tcbuselibrary{theorems}
\newtcbtheorem[number within=section]{important}{Important Insight}%
{colback=green!5,colframe=green!35!black,fonttitle=\bfseries}{th}
\newtcbtheorem[number within=section]{schluessel}{Key Point}%
{colback=blue!5,colframe=blue!75!black,fonttitle=\bfseries}{key}
\newtcbtheorem[number within=section]{result}{Result}%
{colback=green!5,colframe=green!75!black,fonttitle=\bfseries}{res}
\newtcbtheorem[number within=section]{keyresult}{Key Result}%
{colback=blue!5,colframe=blue!75!black,fonttitle=\bfseries}{key}

\title{T0-Time-Mass-Duality Theory: Compelling Derivation of Fractal Dimension $D_f$ from Lepton Mass Ratio \\
	\large Validation of Geometric Foundations - Complementary to ParticleMasses\_En.pdf}
\author{Johann Pascher\\
	Department of Communication Technology\\
	Federal Higher Technical College (HTL), Leonding, Austria\\
	\texttt{johann.pascher@gmail.com}}
\date{October 31, 2025}



==================================================

=== T0_unified_report.tex.preamble ===

\documentclass[11pt,a4paper]{article}
\usepackage[utf8]{inputenc}
\usepackage[english]{babel}
\usepackage{amsmath}
\usepackage{amsfonts}
\usepackage{amssymb}
\usepackage{booktabs}
\usepackage{longtable}
\usepackage{geometry}
\usepackage{siunitx}
\usepackage{hyperref}
\geometry{margin=2cm}

\title{T0-Theory: Unified Calculator Results\\
\large Masses and Physical Constants from Geometric Principles}
\author{Johann Pascher\\HTL Leonding, Austria\\
\texttt{Automatically generated by the T0 Unified Calculator v3.0}}
\date{\today}




==================================================

=== T0_vereinigter_bericht.tex.preamble ===

\documentclass[11pt,a4paper]{article}
\usepackage[utf8]{inputenc}
\usepackage[ngerman]{babel}
\usepackage{amsmath}
\usepackage{amsfonts}
\usepackage{amssymb}
\usepackage{booktabs}
\usepackage{longtable}
\usepackage{geometry}
\usepackage{siunitx}
\usepackage{hyperref}
\geometry{margin=2cm}

\title{T0-Theorie: Vereinigter Rechner Ergebnisse\\
\large Massen und physikalische Konstanten aus geometrischen Prinzipien}
\author{Johann Pascher\\HTL Leonding, Österreich\\
\texttt{Automatisch generiert vom T0-Vereinigten Rechner v3.0}}
\date{\today}



==================================================

=== T0_verhaeltnis-absolut_De.tex.preamble ===

\documentclass[12pt,a4paper]{article}
\usepackage[utf8]{inputenc}
\usepackage[T1]{fontenc}
\usepackage[german]{babel}
\usepackage{geometry}
\usepackage{lmodern}
\usepackage{amsmath}
\usepackage{amssymb}
\usepackage{hyperref}
\usepackage{booktabs}
\usepackage{enumitem}
\usepackage[table,xcdraw]{xcolor}
\usepackage{newunicodechar}

% Unicode setups for Greek letters
\newunicodechar{ξ}{\ensuremath{\xi}}
\newunicodechar{μ}{\ensuremath{\mu}}

\geometry{left=2cm,right=2cm,top=2cm,bottom=2cm}

\hypersetup{
	colorlinks=true,
	linkcolor=blue,
	citecolor=blue,
	urlcolor=blue,
	pdftitle={Verhältnisbasiert vs. Absolut: Die Rolle der fraktalen Korrektur in der T0-Theorie},
	pdfauthor={Johann Pascher},
	pdfsubject={T0-Theorie, Fraktale Korrektur, Theoretische Physik}
}

\title{Verhältnisbasiert vs. Absolut: \\ Die Rolle der fraktalen Korrektur in der T0-Theorie \\ \large Mit Implikationen für fundamentale Konstanten}
\author{Johann Pascher\\
	Abteilung für Nachrichtentechnik\\
	Höhere Technische Lehranstalt, Leonding, Österreich\\
	\texttt{johann.pascher@gmail.com}}
\date{\today}



==================================================

=== T0_verhaeltnis-absolut_En.tex.preamble ===

\documentclass[12pt,a4paper]{article}
\usepackage[utf8]{inputenc}
\usepackage[T1]{fontenc}
\usepackage[english]{babel}
\usepackage{geometry}
\usepackage{lmodern}
\usepackage{amsmath}
\usepackage{amssymb}
\usepackage{hyperref}
\usepackage{booktabs}
\usepackage{enumitem}
\usepackage[table,xcdraw]{xcolor}
\usepackage{newunicodechar}

% Unicode setups for Greek letters
\newunicodechar{ξ}{\ensuremath{\xi}}
\newunicodechar{μ}{\ensuremath{\mu}}

\geometry{left=2cm,right=2cm,top=2cm,bottom=2cm}

\hypersetup{
	colorlinks=true,
	linkcolor=blue,
	citecolor=blue,
	urlcolor=blue,
	pdftitle={Ratio-Based vs. Absolute: The Role of Fractal Correction in T0 Theory},
	pdfauthor={Johann Pascher},
	pdfsubject={T0 Theory, Fractal Correction, Theoretical Physics}
}

\title{Ratio-Based vs. Absolute: \\ The Role of Fractal Correction in T0 Theory \\ \large With Implications for Fundamental Constants}
\author{Johann Pascher\\
	Department of Communications Engineering\\
	Higher Technical Institute, Leonding, Austria\\
	\texttt{johann.pascher@gmail.com}}
\date{\today}



==================================================

=== T0_xi-und-e_De.tex.preamble ===

\documentclass[12pt,a4paper]{article}
\usepackage[utf8]{inputenc}
\usepackage[T1]{fontenc}
\usepackage[ngerman]{babel}
\usepackage{lmodern}
\usepackage{amsmath,amssymb,amsthm}
\usepackage{geometry}
\usepackage{booktabs}
\usepackage{xcolor}
\usepackage{tcolorbox}
\usepackage{fancyhdr}
\usepackage{hyperref}
\usepackage{tikz}
\usepackage{physics}
\usepackage{siunitx}
\usepackage{multicol}

\definecolor{t0blue}{RGB}{33,150,243}
\definecolor{t0green}{RGB}{76,175,80}
\definecolor{t0orange}{RGB}{255,152,0}
\definecolor{t0red}{RGB}{244,67,54}

\geometry{a4paper, margin=2.5cm}
\setlength{\headheight}{15pt}

\pagestyle{fancy}
\fancyhf{}
\fancyhead[L]{\textsc{T0: $\xi$ und $e$ - Fundamentale Verbindung}}
\fancyhead[R]{\textsc{Mathematische Analyse}}
\fancyfoot[C]{\thepage}

\hypersetup{
	colorlinks=true,
	linkcolor=t0blue,
	citecolor=t0blue,
	urlcolor=t0blue,
}

\newcommand{\xipar}{\xi}
\newcommand{\inftytext}{$\infty$}

\newtcolorbox{erkenntnis}{colback=t0blue!5, colframe=t0blue!75!black, title={Fundamentale Erkenntnis}}
\newtcolorbox{beziehung}{colback=t0green!5, colframe=t0green!75!black, title={Mathematische Beziehung}}
\newtcolorbox{anwendung}{colback=t0orange!5, colframe=t0orange!75!black, title={Physikalische Anwendung}}
\newtcolorbox{abhandlung}{colback=t0red!5, colframe=t0red!75!black, title={Theoretische Abhandlung}}

\title{\textbf{T0-Theorie: $\xi$ und $e$}\\[0.5cm]
	\large Die fundamentale Verbindung zwischen geometrischem Parameter\\
	und natürlichem Exponential\\[0.3cm]
	\normalsize Eine umfassende mathematische und physikalische Analyse}
\author{T0-Theory: Time-Mass Duality Framework\\ \small Basierend auf \url{https://github.com/jpascher/T0-Time-Mass-Duality/}}
\date{\today}



==================================================

=== T0_xi-und-e_En.tex.preamble ===

\documentclass[12pt,a4paper]{article}
\usepackage[utf8]{inputenc}
\usepackage[T1]{fontenc}
\usepackage[english]{babel}
\usepackage{lmodern}
\usepackage{amsmath,amssymb,amsthm}
\usepackage{geometry}
\usepackage{booktabs}
\usepackage{xcolor}
\usepackage{tcolorbox}
\usepackage{fancyhdr}
\usepackage{hyperref}
\usepackage{tikz}
\usepackage{physics}
\usepackage{siunitx}
\usepackage{multicol}

\definecolor{t0blue}{RGB}{33,150,243}
\definecolor{t0green}{RGB}{76,175,80}
\definecolor{t0orange}{RGB}{255,152,0}
\definecolor{t0red}{RGB}{244,67,54}

\geometry{a4paper, margin=2.5cm}
\setlength{\headheight}{15pt}

\pagestyle{fancy}
\fancyhf{}
\fancyhead[L]{\textsc{T0: $\xi$ and $e$ - Fundamental Connection}}
\fancyhead[R]{\textsc{Mathematical Analysis}}
\fancyfoot[C]{\thepage}

\hypersetup{
	colorlinks=true,
	linkcolor=t0blue,
	citecolor=t0blue,
	urlcolor=t0blue,
}

\newcommand{\xipar}{\xi}
\newcommand{\inftytext}{$\infty$}

\newtcolorbox{insight}{colback=t0blue!5, colframe=t0blue!75!black, title={Fundamental Insight}}
\newtcolorbox{relation}{colback=t0green!5, colframe=t0green!75!black, title={Mathematical Relation}}
\newtcolorbox{application}{colback=t0orange!5, colframe=t0orange!75!black, title={Physical Application}}
\newtcolorbox{treatise}{colback=t0red!5, colframe=t0red!75!black, title={Theoretical Treatise}}

\title{\textbf{T0-Theory: $\xi$ and $e$}\\[0.5cm]
	\large The Fundamental Connection Between Geometric Parameter\\
	and Natural Exponential\\[0.3cm]
	\normalsize A Comprehensive Mathematical and Physical Analysis}
\author{T0-Theory: Time-Mass Duality Framework\\ \small Based on \url{https://github.com/jpascher/T0-Time-Mass-Duality/}}
\date{\today}



==================================================

=== T0_xi_ursprung_De.tex.preamble ===

\documentclass[12pt,a4paper]{article}
\usepackage[utf8]{inputenc}
\usepackage[T1]{fontenc}
\usepackage[ngerman]{babel}
\usepackage[left=2.5cm,right=2.5cm,top=2.5cm,bottom=2.5cm]{geometry}
\usepackage{lmodern}
\usepackage{amsmath}
\usepackage{amssymb}
\usepackage{physics}
\usepackage{hyperref}
\usepackage{tcolorbox}
\usepackage{booktabs}
\usepackage{enumitem}
\usepackage[table]{xcolor}
\usepackage{graphicx}
\usepackage{float}
\usepackage{mathtools}
\usepackage{fancyhdr}
\usepackage{array}
\usepackage{multirow}

\pagestyle{fancy}
\fancyhf{}
\fancyhead[L]{T0-Theorie: Der Massenskalierungsexponent $\kappa$}
\fancyhead[R]{Echte Herleitung ohne Zirkularität}
\fancyfoot[C]{\thepage}
\renewcommand{\headrulewidth}{0.4pt}
\renewcommand{\footrulewidth}{0.4pt}
\setlength{\headheight}{15pt}

\hypersetup{
	colorlinks=true,
	linkcolor=blue,
	citecolor=blue,
	urlcolor=blue,
	pdftitle={Der Massenskalierungsexponent κ in der T0-Theorie},
	pdfauthor={Johann Pascher},
	pdfsubject={T0-Theorie, Massenhierarchie, Fundamentale Konstanten}
}

\title{\textbf{Der Massenskalierungsexponent $\kappa$}\\[0.5cm]
	\large Echte Herleitung aus dem e-p-$\mu$-System ohne Zirkularität\\[0.3cm]
	\normalsize Die fundamentale Begründung für $\xi = \frac{4}{30000}$}
\author{Johann Pascher\\
	T0-Theorie Forschungsgruppe\\
	HTL Leonding, Österreich\\
	\texttt{johann.pascher@gmail.com}}
\date{\today}



==================================================

=== T0_xi_ursprung_En.tex.preamble ===

\documentclass[12pt,a4paper]{article}
\usepackage[utf8]{inputenc}
\usepackage[T1]{fontenc}
\usepackage[english]{babel}
\usepackage[left=2.5cm,right=2.5cm,top=2.5cm,bottom=2.5cm]{geometry}
\usepackage{lmodern}
\usepackage{amsmath}
\usepackage{amssymb}
\usepackage{physics}
\usepackage{hyperref}
\usepackage{tcolorbox}
\usepackage{booktabs}
\usepackage{enumitem}
\usepackage[table]{xcolor}
\usepackage{graphicx}
\usepackage{float}
\usepackage{mathtools}
\usepackage{fancyhdr}
\usepackage{array}
\usepackage{multirow}

\pagestyle{fancy}
\fancyhf{}
\fancyhead[L]{T0 Theory: The Mass Scaling Exponent $\kappa$}
\fancyhead[R]{Genuine Derivation Without Circularity}
\fancyfoot[C]{\thepage}
\renewcommand{\headrulewidth}{0.4pt}
\renewcommand{\footrulewidth}{0.4pt}
\setlength{\headheight}{15pt}

\hypersetup{
	colorlinks=true,
	linkcolor=blue,
	citecolor=blue,
	urlcolor=blue,
	pdftitle={The Mass Scaling Exponent κ in T0 Theory},
	pdfauthor={Johann Pascher},
	pdfsubject={T0 Theory, Mass Hierarchy, Fundamental Constants}
}

\title{\textbf{The Mass Scaling Exponent $\kappa$}\\[0.5cm]
	\large Genuine Derivation from the e-p-$\mu$ System Without Circularity\\[0.3cm]
	\normalsize The Fundamental Justification for $\xi = \frac{4}{30000}$}
\author{Johann Pascher\\
	T0 Theory Research Group\\
	HTL Leonding, Austria\\
	\texttt{johann.pascher@gmail.com}}
\date{\today}



==================================================

=== T0vsESM_ConceptualAnalysis_De.tex.preamble ===

\documentclass[12pt,a4paper]{article}
\usepackage[utf8]{inputenc}
\usepackage[T1]{fontenc}
\usepackage[ngerman]{babel}
\usepackage{lmodern}
\usepackage{amsmath}
\usepackage{amssymb}
\usepackage{physics}
\usepackage{hyperref}
\usepackage{tcolorbox}
\usepackage{booktabs}
\usepackage{enumitem}
\usepackage[table,xcdraw]{xcolor}
\usepackage[left=2cm,right=2cm,top=2cm,bottom=2cm]{geometry}
\usepackage{pgfplots}
\pgfplotsset{compat=1.18}
\usepackage{graphicx}
\usepackage{float}
\usepackage{fancyhdr}
\usepackage{siunitx}
\usepackage{mathtools}
\usepackage{amsthm}
\usepackage{cleveref}
\usepackage{tocloft}
\usepackage{array}
\usepackage{microtype}
\usepackage{pdflscape}
\usepackage{newunicodechar}

% Unicode-Sterne beheben
\newunicodechar{★}{\ensuremath{\star}}

% Bessere Abstände und Zeilenumbrüche
\emergencystretch 3em
\tolerance 9999
\hbadness 9999

% Kopf- und Fußzeilen
\pagestyle{fancy}
\fancyhf{}
\fancyhead[L]{Johann Pascher}
\fancyhead[R]{Einheitliche Natürliche Einheiten vs. Erweitertes Standardmodell}
\fancyfoot[C]{\thepage}
\renewcommand{\headrulewidth}{0.4pt}
\renewcommand{\footrulewidth}{0.4pt}

% Inhaltsverzeichnis-Formatierung
\renewcommand{\cfttoctitlefont}{\huge\bfseries\color{blue}}
\renewcommand{\cftsecfont}{\color{blue}}
\renewcommand{\cftsubsecfont}{\color{blue}}
\renewcommand{\cftsecpagefont}{\color{blue}}
\renewcommand{\cftsubsecpagefont}{\color{blue}}

\hypersetup{
	colorlinks=true,
	linkcolor=blue,
	citecolor=blue,
	urlcolor=blue,
	pdftitle={Konzeptioneller Vergleich von Einheitlichen Natürlichen Einheiten und Erweitertem Standardmodell},
	pdfauthor={Johann Pascher},
	pdfsubject={Theoretische Physik},
	pdfkeywords={Einheitliche Natürliche Einheiten, Erweitertes Standardmodell, Alpha=1, Beta=1, Intrinsisches Zeitfeld}
}

% Benutzerdefinierte Befehle (abgestimmt mit Referenzdokument)
\newcommand{\Tfield}{T(x)}
\newcommand{\Tfieldt}{T(x,t)}
\newcommand{\alphaEM}{\alpha_{\text{EM}}}
\newcommand{\betaT}{\beta_{\text{T}}}
\newcommand{\Mpl}{M_{\text{Pl}}}
\newcommand{\Tzero}{T_0}
\newcommand{\vecx}{\vec{x}}
\newcommand{\lP}{\ell_{\text{P}}}
\newcommand{\xipar}{\xi}
\newcommand{\LCDM}{\Lambda\text{CDM}}

% Theorem-Umgebungen
\newtheorem{principle}{Grundlegendes Prinzip}[section]
\newtheorem{insight}{Wichtige Einsicht}[section]



==================================================

=== T0vsESM_ConceptualAnalysis_En.tex.preamble ===

\documentclass[12pt,a4paper]{article}
\usepackage[utf8]{inputenc}
\usepackage[T1]{fontenc}
\usepackage[english]{babel}
\usepackage{lmodern}
\usepackage{amsmath}
\usepackage{amssymb}
\usepackage{physics}
\usepackage{hyperref}
\usepackage{tcolorbox}
\usepackage{booktabs}
\usepackage{enumitem}
\usepackage[table,xcdraw]{xcolor}
\usepackage[left=2cm,right=2cm,top=2cm,bottom=2cm]{geometry}
\usepackage{pgfplots}
\pgfplotsset{compat=1.18}
\usepackage{graphicx}
\usepackage{float}
\usepackage{fancyhdr}
\usepackage{siunitx}
\usepackage{array}
\usepackage{cleveref}

% Headers and Footers
\pagestyle{fancy}
\fancyhf{}
\fancyhead[L]{Johann Pascher}
\fancyhead[R]{Unified Natural Units vs. Extended SM}
\fancyfoot[C]{\thepage}
\renewcommand{\headrulewidth}{0.4pt}
\renewcommand{\footrulewidth}{0.4pt}

% Custom commands (aligned with reference document)
\newcommand{\Tfield}{T(x)}
\newcommand{\Tfieldt}{T(x,t)}
\newcommand{\alphaEM}{\alpha_{\text{EM}}}
\newcommand{\betaT}{\beta_{\text{T}}}
\newcommand{\Mpl}{M_{\text{Pl}}}
\newcommand{\Tzero}{T_0}
\newcommand{\vecx}{\vec{x}}
\newcommand{\lP}{\ell_{\text{P}}}
\newcommand{\xipar}{\xi}
\newcommand{\LCDM}{\Lambda\text{CDM}}

\hypersetup{
	colorlinks=true,
	linkcolor=blue,
	citecolor=blue,
	urlcolor=blue,
	pdftitle={Conceptual Comparison of Unified Natural Units and Extended Standard Model},
	pdfauthor={Johann Pascher},
	pdfsubject={Theoretical Physics},
	pdfkeywords={Unified Natural Units, Extended Standard Model, Alpha=1, Beta=1, Intrinsic Time Field}
}



==================================================

=== Teilchenmassen_De.tex.preamble ===

\documentclass[12pt,a4paper]{article}
\usepackage[utf8]{inputenc}
\usepackage[T1]{fontenc}
\usepackage[german]{babel}
\usepackage{lmodern}
\usepackage{amsmath}
\usepackage{amssymb}
\usepackage{physics}
\usepackage{hyperref}
\usepackage{tcolorbox}
\usepackage{booktabs}
\usepackage{enumitem}
\usepackage[table,xcdraw]{xcolor}
\usepackage[left=2cm,right=2cm,top=2cm,bottom=2cm]{geometry}
\usepackage{pgfplots}
\pgfplotsset{compat=1.18}
\usepackage{graphicx}
\usepackage{float}
\usepackage{fancyhdr}
\usepackage{siunitx}
\usepackage{mathtools}
\usepackage{amsthm}
\usepackage{cleveref}
\usepackage{tocloft}
\usepackage{tikz}
\usepackage[dvipsnames]{xcolor}
\usetikzlibrary{positioning, shapes.geometric, arrows.meta}
\usepackage{microtype}
\usepackage{array}
\usepackage{longtable}

% Custom Commands
\newcommand{\Efield}{E_{\text{Feld}}}
\newcommand{\xigeom}{\xi_{\text{geom}}}
\newcommand{\Tzero}{T_0}
\newcommand{\vecx}{\vec{x}}
\newcommand{\xipar}{\xi}

% Header and Footer Configuration
\pagestyle{fancy}
\fancyhf{}
\fancyhead[L]{Johann Pascher}
\fancyhead[R]{T0-Modell: Vollständige parameterfreie Teilchenmassen-Berechnung}
\fancyfoot[C]{\thepage}
\renewcommand{\headrulewidth}{0.4pt}
\renewcommand{\footrulewidth}{0.4pt}

% Table of Contents Formatting
\renewcommand{\cftsecfont}{\color{blue}}
\renewcommand{\cftsubsecfont}{\color{blue}}
\renewcommand{\cftsecpagefont}{\color{blue}}
\renewcommand{\cftsubsecpagefont}{\color{blue}}

\hypersetup{
	colorlinks=true,
	linkcolor=blue,
	citecolor=blue,
	urlcolor=blue,
	pdftitle={T0-Modell: Vollständige parameterfreie Teilchenmassen-Berechnung},
	pdfauthor={Johann Pascher},
	pdfsubject={T0-Modell, Geometrische Resonanz, Yukawa-Methode, Vollständige Neutrino-Behandlung},
	pdfkeywords={Energiefeld, Geometrische Resonanzen, Yukawa-Kopplungen, Parameterfreie Theorie, Neutrino-Massen}
}

% Theorem Environments
\newtheorem{theorem}{Theorem}[section]
\newtheorem{proposition}[theorem]{Proposition}
\newtheorem{definition}[theorem]{Definition}
\newtheorem{lemma}[theorem]{Lemma}

\tcbuselibrary{theorems}
\newtcbtheorem[number within=section]{important}{Wichtige Erkenntnis}%
{colback=green!5,colframe=green!35!black,fonttitle=\bfseries}{th}
\newtcbtheorem[number within=section]{schluessel}{Schlüssel}%
{colback=blue!5,colframe=blue!75!black,fonttitle=\bfseries}{key}
\newtcbtheorem[number within=section]{warning}{Warnung}%
{colback=red!5,colframe=red!75!black,fonttitle=\bfseries}{warn}
\newtcbtheorem[number within=section]{keyresult}{Schlüsselergebnis}%
{colback=blue!5,colframe=blue!75!black,fonttitle=\bfseries}{key}
\newtcbtheorem[number within=section]{ratiomethod}{Verhältnismethode}%
{colback=orange!5,colframe=orange!75!black,fonttitle=\bfseries}{ratio}
\newtcbtheorem[number within=section]{neutrino}{Neutrino-Behandlung}%
{colback=purple!5,colframe=purple!75!black,fonttitle=\bfseries}{nu}

\title{T0-Modell: Vollständige parameterfreie Teilchenmassen-Berechnung \\
	\large Direkte geometrische Methode vs. Erweiterte Yukawa-Methode \\
	\large Mit vollständiger Neutrino-Quantenzahlen-Analyse und QFT-Herleitung}
\author{Johann Pascher\\
	Abteilung für Kommunikationstechnologie\\
	Höhere Technische Bundeslehranstalt (HTL), Leonding, Österreich\\
	\texttt{johann.pascher@gmail.com}}
\date{\today}



==================================================

=== Teilchenmassen_En.tex.preamble ===

\documentclass[12pt,a4paper]{article}
\usepackage[utf8]{inputenc}
\usepackage[T1]{fontenc}
\usepackage{lmodern}
\usepackage{amsmath}
\usepackage{amssymb}
\usepackage{physics}
\usepackage{hyperref}
\usepackage{tcolorbox}
\usepackage{booktabs}
\usepackage{enumitem}
\usepackage[table,xcdraw]{xcolor}
\usepackage[left=2cm,right=2cm,top=2cm,bottom=2cm]{geometry}
\usepackage{pgfplots}
\pgfplotsset{compat=1.18}
\usepackage{graphicx}
\usepackage{float}
\usepackage{fancyhdr}
\usepackage{siunitx}
\usepackage{mathtools}
\usepackage{amsthm}
\usepackage{cleveref}
\usepackage{tocloft}
\usepackage{tikz}
\usepackage[dvipsnames]{xcolor}
\usetikzlibrary{positioning, shapes.geometric, arrows.meta}
\usepackage{microtype}
\usepackage{array}
\usepackage{longtable}

% Custom Commands
\newcommand{\Efield}{E_{\text{Field}}}
\newcommand{\xigeom}{\xi_{\text{geom}}}
\newcommand{\Tzero}{T_0}
\newcommand{\vecx}{\vec{x}}
\newcommand{\xipar}{\xi}

% Header and Footer Configuration
\pagestyle{fancy}
\fancyhf{}
\fancyhead[L]{Johann Pascher}
\fancyhead[R]{T0 Model: Complete Parameter-Free Particle Mass Calculation}
\fancyfoot[C]{\thepage}
\renewcommand{\headrulewidth}{0.4pt}
\renewcommand{\footrulewidth}{0.4pt}

% Table of Contents Formatting
\renewcommand{\cftsecfont}{\color{blue}}
\renewcommand{\cftsubsecfont}{\color{blue}}
\renewcommand{\cftsecpagefont}{\color{blue}}
\renewcommand{\cftsubsecpagefont}{\color{blue}}

\hypersetup{
	colorlinks=true,
	linkcolor=blue,
	citecolor=blue,
	urlcolor=blue,
	pdftitle={T0 Model: Complete Parameter-Free Particle Mass Calculation},
	pdfauthor={Johann Pascher},
	pdfsubject={T0 Model, Geometric Resonance, Yukawa Method, Complete Neutrino Treatment},
	pdfkeywords={Energy Field, Geometric Resonances, Yukawa Couplings, Parameter-Free Theory, Neutrino Masses}
}

% Theorem Environments
\newtheorem{theorem}{Theorem}[section]
\newtheorem{proposition}[theorem]{Proposition}
\newtheorem{definition}[theorem]{Definition}
\newtheorem{lemma}[theorem]{Lemma}

\tcbuselibrary{theorems}
\newtcbtheorem[number within=section]{important}{Important Insight}%
{colback=green!5,colframe=green!35!black,fonttitle=\bfseries}{th}
\newtcbtheorem[number within=section]{key}{Key Point}%
{colback=blue!5,colframe=blue!75!black,fonttitle=\bfseries}{key}
\newtcbtheorem[number within=section]{warning}{Warning}%
{colback=red!5,colframe=red!75!black,fonttitle=\bfseries}{warn}
\newtcbtheorem[number within=section]{keyresult}{Key Result}%
{colback=blue!5,colframe=blue!75!black,fonttitle=\bfseries}{keyres}
\newtcbtheorem[number within=section]{ratiomethod}{Ratio Method}%
{colback=orange!5,colframe=orange!75!black,fonttitle=\bfseries}{ratio}
\newtcbtheorem[number within=section]{neutrino}{Neutrino Treatment}%
{colback=purple!5,colframe=purple!75!black,fonttitle=\bfseries}{nu}

\title{T0 Model: Complete Parameter-Free Particle Mass Calculation \\
	\large Direct Geometric Method vs. Extended Yukawa Method \\
	\large With Complete Neutrino Quantum Number Analysis and QFT Derivation}
\author{Johann Pascher\\
	Department of Communication Technology\\
	Higher Technical Federal Institute (HTL), Leonding, Austria\\
	\texttt{johann.pascher@gmail.com}}
\date{\today}



==================================================

=== TempEinheitenCMBDe.tex.preamble ===

\documentclass[12pt,a4paper]{article}
\usepackage[utf8]{inputenc}
\usepackage[T1]{fontenc}
\usepackage[ngerman]{babel}
\usepackage[left=2cm,right=2cm,top=2cm,bottom=2cm]{geometry}
\usepackage{lmodern}
\usepackage{amsmath}
\usepackage{amssymb}
\usepackage{physics}
\usepackage{hyperref}
\usepackage{tcolorbox}
\usepackage{booktabs}
\usepackage{array}
\usepackage{tabularx}
\usepackage{braket}
\usepackage{siunitx}
\usepackage{amsthm}
\usepackage{cleveref}
\usepackage{enumitem}
\usepackage[table,xcdraw]{xcolor}
\usepackage{longtable}
\usepackage{fancyhdr}
\usepackage{listings}
\usepackage{tikz,pgfplots}
\pgfplotsset{compat=1.18}

% Kopf- und Fu\ss{}zeilen
\pagestyle{fancy}
\fancyhf{}
\fancyhead[L]{Johann Pascher}
\fancyhead[R]{Temperatureinheiten in der T0-Theorie}
\fancyfoot[C]{\thepage}
\renewcommand{\headrulewidth}{0.4pt}
\renewcommand{\footrulewidth}{0.4pt}

\hypersetup{
	colorlinks=true,
	linkcolor=blue,
	citecolor=blue,
	urlcolor=blue,
	pdftitle={Temperatureinheiten in nat\"urlichen Einheiten: T0-Theorie},
	pdfauthor={Johann Pascher},
	pdfsubject={T0-Modell, xi-Konstante},
	pdfkeywords={xi-Feld, Nat\"urliche Einheiten, Temperatur, T0-Theorie}
}

% Benutzerdefinierte Umgebungen
\newtcolorbox{important}[1][]{colback=yellow!10!white,colframe=yellow!50!black,fonttitle=\bfseries,title=Wichtiger Hinweis,#1}
\newtcolorbox{formula}[1][]{colback=blue!5!white,colframe=blue!75!black,fonttitle=\bfseries,title=Schl\"usselformel,#1}
\newtcolorbox{revolutionary}[1][]{colback=red!5!white,colframe=red!75!black,fonttitle=\bfseries,title=Revolution\"are Einsicht,#1}
\newtcolorbox{sibox}[1][]{colback=orange!10!white,colframe=orange!75!black,fonttitle=\bfseries,title=SI-Einheiten (nur zur Referenz),#1}

% Benutzerdefinierte Befehle aus CMB-Dokument
\newcommand{\Tfield}{T(x)}
\newcommand{\xipar}{\xi}
\newcommand{\Tzero}{T_0}

% Theorem-Umgebungen
\newtheorem{theorem}{Theorem}[section]
\newtheorem{lemma}[theorem]{Lemma}
\newtheorem{proposition}[theorem]{Proposition}
\theoremstyle{definition}
\newtheorem{definition}[theorem]{Definition}
\theoremstyle{remark}
\newtheorem{remark}[theorem]{Bemerkung}

% Code-Listing-Stil
\lstset{
	language=Python,
	basicstyle=\small\ttfamily,
	keywordstyle=\color{blue},
	commentstyle=\color{gray},
	stringstyle=\color{red},
	showstringspaces=false,
	breaklines=true,
	frame=single,
	framerule=0.5pt,
	frameround=tttt,
	backgroundcolor=\color{gray!10}
}



==================================================

=== TempEinheitenCMBEn.tex.preamble ===

\documentclass[12pt,a4paper]{article}
\usepackage[utf8]{inputenc}
\usepackage[T1]{fontenc}
\usepackage[english]{babel}
\usepackage[left=2cm,right=2cm,top=2cm,bottom=2cm]{geometry}
\usepackage{lmodern}
\usepackage{amsmath}
\usepackage{amssymb}
\usepackage{physics}
\usepackage{hyperref}
\usepackage{tcolorbox}
\usepackage{booktabs}
\usepackage{array}
\usepackage{tabularx}
\usepackage{braket}
\usepackage{siunitx}
\usepackage{amsthm}
\usepackage{cleveref}
\usepackage{enumitem}
\usepackage[table,xcdraw]{xcolor}
\usepackage{longtable}
\usepackage{fancyhdr}
\usepackage{listings}
\usepackage{tikz,pgfplots}
\pgfplotsset{compat=1.18}

% Header and Footer
\pagestyle{fancy}
\fancyhf{}
\fancyhead[L]{Johann Pascher}
\fancyhead[R]{Temperature Units in T0-Theory}
\fancyfoot[C]{\thepage}
\renewcommand{\headrulewidth}{0.4pt}
\renewcommand{\footrulewidth}{0.4pt}

\hypersetup{
	colorlinks=true,
	linkcolor=blue,
	citecolor=blue,
	urlcolor=blue,
	pdftitle={Temperature Units in Natural Units: T0-Theory},
	pdfauthor={Johann Pascher},
	pdfsubject={T0 Model, xi-constant},
	pdfkeywords={xi-field, Natural Units, Temperature, T0-Theory}
}

% Custom environments
\newtcolorbox{important}[1][]{colback=yellow!10!white,colframe=yellow!50!black,fonttitle=\bfseries,title=Important Note,#1}
\newtcolorbox{formula}[1][]{colback=blue!5!white,colframe=blue!75!black,fonttitle=\bfseries,title=Key Formula,#1}
\newtcolorbox{revolutionary}[1][]{colback=red!5!white,colframe=red!75!black,fonttitle=\bfseries,title=Revolutionary Insight,#1}
\newtcolorbox{sibox}[1][]{colback=orange!10!white,colframe=orange!75!black,fonttitle=\bfseries,title=SI Units (for reference only),#1}

% Custom commands from CMB document
\newcommand{\Tfield}{T(x)}
\newcommand{\xipar}{\xi}
\newcommand{\Tzero}{T_0}

% Theorem environments
\newtheorem{theorem}{Theorem}[section]
\newtheorem{lemma}[theorem]{Lemma}
\newtheorem{proposition}[theorem]{Proposition}
\theoremstyle{definition}
\newtheorem{definition}[theorem]{Definition}
\theoremstyle{remark}
\newtheorem{remark}[theorem]{Remark}

% Code listing style
\lstset{
	language=Python,
	basicstyle=\small\ttfamily,
	keywordstyle=\color{blue},
	commentstyle=\color{gray},
	stringstyle=\color{red},
	showstringspaces=false,
	breaklines=true,
	frame=single,
	framerule=0.5pt,
	frameround=tttt,
	backgroundcolor=\color{gray!10}
}



==================================================

=== Unit Charge_De.tex.preamble ===

\documentclass[12pt,a4paper]{article}
\usepackage[utf8]{inputenc}
\usepackage[T1]{fontenc}
\usepackage{geometry}
\usepackage{lmodern}
\usepackage{amsmath}
\usepackage{amssymb}
\usepackage{amsfonts} % Für bessere Mathe-Symbole
\usepackage{hyperref}
\usepackage{booktabs}
\usepackage{enumitem}
\usepackage[table,xcdraw]{xcolor}
\usepackage{newunicodechar}
\usepackage[ngerman]{babel} % Für deutsche Trennregeln und Sprache

% Unicode setups for Greek letters and symbols
\newunicodechar{ξ}{\ensuremath{\xi}}
\newunicodechar{μ}{\ensuremath{\mu}}
\newunicodechar{π}{\ensuremath{\pi}}

\geometry{left=2cm,right=2cm,top=2cm,bottom=2cm}

\hypersetup{
	colorlinks=true,
	linkcolor=blue,
	citecolor=blue,
	urlcolor=blue,
	pdftitle={Die Elektroneneinheitsladung in der T0-Theorie: Jenseits von Punkt-Singularitäten},
	pdfauthor={Johann Pascher},
	pdfsubject={T0-Theorie, Elektronenladung, Singularitäten, Elektrodynamik}
}

\title{Die Elektroneneinheitsladung in der T0-Theorie:\\Jenseits von Punkt-Singularitäten}
\author{Johann Pascher\\
	Abteilung für Kommunikationstechnik\\
	Höhere Technische Lehranstalt Leonding, Österreich\\
	\texttt{johann.pascher@gmail.com}}
\date{21. Oktober 2025}



==================================================

=== Unit Charge_En.tex.preamble ===

\documentclass[12pt,a4paper]{article}
\usepackage[utf8]{inputenc}
\usepackage[T1]{fontenc}
\usepackage{geometry}
\usepackage{lmodern}
\usepackage{amsmath}
\usepackage{amssymb}
\usepackage{amsfonts} % Für bessere Mathe-Symbole
\usepackage{hyperref}
\usepackage{booktabs}
\usepackage{enumitem}
\usepackage[table,xcdraw]{xcolor}
\usepackage{newunicodechar}

% Unicode setups for Greek letters and symbols (falls benötigt, aber wir verwenden LaTeX-Makros)
\newunicodechar{ξ}{\ensuremath{\xi}}
\newunicodechar{μ}{\ensuremath{\mu}}
\newunicodechar{π}{\ensuremath{\pi}}

\geometry{left=2cm,right=2cm,top=2cm,bottom=2cm}

\hypersetup{
	colorlinks=true,
	linkcolor=blue,
	citecolor=blue,
	urlcolor=blue,
	pdftitle={The Electron Unit Charge in T0 Theory: Beyond Point Singularities},
	pdfauthor={Johann Pascher},
	pdfsubject={T0 Theory, Electron Charge, Singularities, Electrodynamics}
}

\title{The Electron Unit Charge in T0 Theory:\\Beyond Point Singularities}
\author{Johann Pascher\\
	Department of Communications Engineering\\
	Higher Technical Institute Leonding, Austria\\
	\texttt{johann.pascher@gmail.com}}
\date{October 21, 2025}



==================================================

=== Zeit-konstant_De.tex.preamble ===

\documentclass[12pt,a4paper]{article}
\usepackage[utf8]{inputenc}
\usepackage[T1]{fontenc}
\usepackage[ngerman]{babel}
\usepackage{amsmath,amsfonts,amssymb}
\usepackage{physics}
\usepackage{siunitx}
\usepackage{booktabs}
\usepackage{longtable}
\usepackage{array}
\usepackage{xcolor}
\usepackage{geometry}
\usepackage{textgreek}
\usepackage{fancyhdr}
\usepackage{hyperref}
\usepackage{tocloft}
\geometry{margin=2.5cm}

% Konfiguration von Kopf- und Fußzeile
\pagestyle{fancy}
\fancyhf{}
\fancyhead[L]{\textsc{T0-Modell}}
\fancyhead[R]{\textsc{Eine Neuformulierung der Physik}}
\fancyfoot[C]{\thepage}
\renewcommand{\headrulewidth}{0.4pt}
\renewcommand{\footrulewidth}{0.4pt}

% Formatierung des Inhaltsverzeichnisses
\renewcommand{\cfttoctitlefont}{\huge\bfseries\color{blue}}
\renewcommand{\cftsecfont}{\color{blue}}
\renewcommand{\cftsubsecfont}{\color{blue}}
\renewcommand{\cftsecpagefont}{\color{blue}}
\renewcommand{\cftsubsecpagefont}{\color{blue}}

% Hyperlink-Konfiguration
\hypersetup{
	colorlinks=true,
	linkcolor=blue,
	citecolor=blue,
	urlcolor=blue,
	pdftitle={Das T0-Modell: Zeit-Energie-Dualität und geometrische Ruhemasse},
	pdfauthor={Johann Pascher},
	pdfsubject={T0-Modell, Zeit-Energie-Dualität, Theoretische Physik},
	pdfkeywords={T0-Theorie, Geometrische Ruhemasse, Zeit-Energie-Dualität, Kosmologie}
}

\title{Das T0-Modell: Zeit-Energie-Dualität und geometrische Ruhemasse\\
	\large (Energiebasierte Version)}
\author{Johann Pascher\\
	\small Höhere Technische Bundeslehranstalt (HTL), Leonding, Österreich\\
	\small \texttt{johann.pascher@gmail.com}}
\date{\today}



==================================================

=== Zeit-konstant_En.tex.preamble ===

\documentclass[12pt,a4paper]{article}
\usepackage[utf8]{inputenc}
\usepackage[T1]{fontenc}
\usepackage[english]{babel}
\usepackage{amsmath,amsfonts,amssymb}
\usepackage{physics}
\usepackage{siunitx}
\usepackage{booktabs}
\usepackage{longtable}
\usepackage{array}
\usepackage{xcolor}
\usepackage{geometry}
\usepackage{textgreek}
\usepackage{fancyhdr}
\usepackage{hyperref}
\usepackage{tocloft}
\geometry{margin=2.5cm}

% Configuration of header and footer
\pagestyle{fancy}
\fancyhf{}
\fancyhead[L]{\textsc{T0 Model}}
\fancyhead[R]{\textsc{A Reformulation of Physics}}
\fancyfoot[C]{\thepage}
\renewcommand{\headrulewidth}{0.4pt}
\renewcommand{\footrulewidth}{0.4pt}

% Formatting of the table of contents
\renewcommand{\cfttoctitlefont}{\huge\bfseries\color{blue}}
\renewcommand{\cftsecfont}{\color{blue}}
\renewcommand{\cftsubsecfont}{\color{blue}}
\renewcommand{\cftsecpagefont}{\color{blue}}
\renewcommand{\cftsubsecpagefont}{\color{blue}}

% Hyperlink configuration
\hypersetup{
	colorlinks=true,
	linkcolor=blue,
	citecolor=blue,
	urlcolor=blue,
	pdftitle={The T0 Model: Time-Energy Duality and Geometric Rest Mass},
	pdfauthor={Johann Pascher},
	pdfsubject={T0 Model, Time-Energy Duality, Theoretical Physics},
	pdfkeywords={T0 Theory, Geometric Rest Mass, Time-Energy Duality, Cosmology}
}

\title{The T0 Model: Time-Energy Duality and Geometric Rest Mass\\
	\large (Energy-Based Version)}
\author{Johann Pascher\\
	\small Höhere Technische Bundeslehranstalt (HTL), Leonding, Austria\\
	\small \texttt{johann.pascher@gmail.com}}
\date{\today}



==================================================

=== Zeit_De.tex.preamble ===

\documentclass[12pt,a4paper]{article}
\usepackage[margin=2cm]{geometry}
\usepackage[utf8]{inputenc}
\usepackage[T1]{fontenc}
\usepackage{lmodern}
\usepackage[ngerman]{babel}
\usepackage{amsmath,amssymb,physics,graphicx,xcolor,amsthm}
\usepackage{hyperref}
\usepackage{booktabs}
\usepackage{siunitx}
\usepackage{cleveref}
\usepackage{fancyhdr}
\usepackage{tcolorbox}
\usepackage{mathtools}
\usepackage{textcomp}

% Benutzerdefinierte Befehle
\newcommand{\Tfield}{T(x,t)}
\newcommand{\mfield}{m(x,t)}
\newcommand{\xipar}{\xi}
\newcommand{\Lzero}{L_0}
\newcommand{\Lp}{L_{\text{P}}}
\newcommand{\mikrometer}{\ensuremath{\mu}\text{m}}
\DeclareUnicodeCharacter{03BC}{\ensuremath{\mu}}

% Theorem-Stile
\newtheorem{theorem}{Theorem}[section]
\newtheorem{proposition}[theorem]{Proposition}
\newtheorem{corollary}[theorem]{Korollar}
\newtheorem{lemma}[theorem]{Lemma}
\theoremstyle{definition}
\newtheorem{definition}[theorem]{Definition}
\newtheorem{example}[theorem]{Beispiel}
\theoremstyle{remark}
\newtheorem{remark}[theorem]{Bemerkung}

% Hyperref-Konfiguration
\hypersetup{
	colorlinks=true,
	linkcolor=blue,
	urlcolor=blue,
	citecolor=blue,
	pdftitle={T0-Modell: Granulation, Limits und fundamentale Asymmetrie},
	pdfauthor={Johann Pascher},
	pdfsubject={Theoretische Physik},
	pdfkeywords={T0-Modell, Granulation, Asymmetrie, Zeit-Masse-Dualitaet}
}

% Kopf- und Fusszeilen-Konfiguration
\pagestyle{fancy}
\fancyhf{}
\fancyhead[L]{Johann Pascher}
\fancyhead[R]{T0-Modell: Granulation, Limits und fundamentale Asymmetrie}
\fancyfoot[C]{\thepage}
\renewcommand{\headrulewidth}{0.4pt}
\renewcommand{\footrulewidth}{0.4pt}

\title{T0-Modell: Granulation, Limits und fundamentale Asymmetrie}
\author{Johann Pascher}
\date{\today}



==================================================

=== Zeit_En.tex.preamble ===

\documentclass[12pt,a4paper]{article}
\usepackage[margin=2cm]{geometry}
\usepackage[utf8]{inputenc}
\usepackage[T1]{fontenc}
\usepackage{lmodern}
\usepackage[english]{babel}
\usepackage{amsmath,amssymb,physics,graphicx,xcolor,amsthm}
\usepackage{hyperref}
\usepackage{booktabs}
\usepackage{siunitx}
\usepackage{cleveref}
\usepackage{fancyhdr}
\usepackage{tcolorbox}
\usepackage{mathtools}
\usepackage{textcomp}

% Custom Commands
\newcommand{\Tfield}{T(x,t)}
\newcommand{\mfield}{m(x,t)}
\newcommand{\xipar}{\xi}
\newcommand{\Lzero}{L_0}
\newcommand{\Lp}{L_{\text{P}}}
\newcommand{\micrometer}{\ensuremath{\mu}\text{m}}
\DeclareUnicodeCharacter{03BC}{\ensuremath{\mu}}

% Theorem Styles
\newtheorem{theorem}{Theorem}[section]
\newtheorem{proposition}[theorem]{Proposition}
\newtheorem{corollary}[theorem]{Corollary}
\newtheorem{lemma}[theorem]{Lemma}
\theoremstyle{definition}
\newtheorem{definition}[theorem]{Definition}
\newtheorem{example}[theorem]{Example}
\theoremstyle{remark}
\newtheorem{remark}[theorem]{Remark}

% Hyperref Configuration
\hypersetup{
	colorlinks=true,
	linkcolor=blue,
	urlcolor=blue,
	citecolor=blue,
	pdftitle={T0 Model: Granulation, Limits and Fundamental Asymmetry},
	pdfauthor={Johann Pascher},
	pdfsubject={Theoretical Physics},
	pdfkeywords={T0 Model, Granulation, Asymmetry, Time-Mass Duality}
}

% Header and Footer Configuration
\pagestyle{fancy}
\fancyhf{}
\fancyhead[L]{Johann Pascher}
\fancyhead[R]{T0 Model: Granulation, Limits and Fundamental Asymmetry}
\fancyfoot[C]{\thepage}
\renewcommand{\headrulewidth}{0.4pt}
\renewcommand{\footrulewidth}{0.4pt}

\title{T0 Model: Granulation, Limits and Fundamental Asymmetry}
\author{Johann Pascher}
\date{\today}



==================================================

=== Zusammenfassung_De.tex.preamble ===

\documentclass[12pt,a4paper]{article}
\usepackage[utf8]{inputenc}
\usepackage[T1]{fontenc}
\usepackage[ngerman]{babel}
\usepackage[left=2cm,right=2cm,top=2cm,bottom=2cm]{geometry}
\usepackage{lmodern}
\usepackage{amsmath}
\usepackage{amssymb}
\usepackage{physics}
\usepackage{booktabs}
\usepackage{tcolorbox}
\usepackage{siunitx}
\usepackage[table,xcdraw]{xcolor}
\usepackage{hyperref}
\usepackage{array}
\usepackage{textgreek}
\usepackage{fancyhdr}
\usepackage{enumitem}
\usepackage{graphicx}
\usepackage{float}

% Farbdefinitionen
\definecolor{t0blue}{RGB}{0,102,204}
\definecolor{t0green}{RGB}{0,153,76}
\definecolor{t0red}{RGB}{204,0,51}
\definecolor{t0yellow}{RGB}{255,204,0}
\definecolor{t0purple}{RGB}{102,0,204}

% Header und Footer
\pagestyle{fancy}
\fancyhf{}
\fancyhead[L]{\color{t0blue}\textsc{T0-Modell: Vollständige Zusammenfassung}}
\fancyhead[R]{\color{t0blue}\textsc{Johann Pascher}}
\fancyfoot[C]{\thepage}
\renewcommand{\headrulewidth}{0.4pt}
\renewcommand{\footrulewidth}{0.4pt}

% Custom-Befehle
\newcommand{\xipar}{\ensuremath{\xi}}
\newcommand{\Efield}{E_\text{field}}
\newcommand{\lP}{\ell_{\text{P}}}
\newcommand{\Mpl}{M_{\text{Pl}}}
\newcommand{\EP}{E_{\text{P}}}
\newcommand{\alphaEM}{\alpha_{\text{EM}}}
\newcommand{\betaT}{\beta_{\text{T}}}

% Tcolorbox-Umgebungen
\newtcolorbox{important}[1][]{
	colback=t0yellow!10!white,
	colframe=t0yellow!75!black,
	fonttitle=\bfseries,
	title=Wichtige Einsicht,
	#1
}

\newtcolorbox{formula}[1][]{
	colback=t0blue!5!white,
	colframe=t0blue!75!black,
	fonttitle=\bfseries,
	title=T0-Vorhersage,
	#1
}

\newtcolorbox{revolutionary}[1][]{
	colback=t0red!5!white,
	colframe=t0red!75!black,
	fonttitle=\bfseries,
	title=Neue Perspektive,
	#1
}

\newtcolorbox{experimental}[1][]{
	colback=t0green!5!white,
	colframe=t0green!75!black,
	fonttitle=\bfseries,
	title=Experimentelle Überprüfung,
	#1
}

\newtcolorbox{quantum}[1][]{
	colback=t0purple!5!white,
	colframe=t0purple!75!black,
	fonttitle=\bfseries,
	title=Quantenmechanik,
	#1
}

% Hyperref-Einstellungen
\hypersetup{
	colorlinks=true,
	linkcolor=t0blue,
	citecolor=t0blue,
	urlcolor=t0blue,
	pdftitle={T0-Modell: Vollständige theoretische Zusammenfassung},
	pdfauthor={Johann Pascher},
	pdfsubject={Theoretische Physik, T0-Theorie, Vereinheitlichung},
	pdfkeywords={T0-Modell, Geometrische Konstante, Quantenmechanik, Vereinheitlichte Feldtheorie}
}

\title{{\Huge \color{t0blue}Das T0-Modell}\\
	{\LARGE Eine umfassende theoretische Abhandlung}\\
	\vspace{1cm}
	{\Large Von der geometrischen Konstante zur Vereinheitlichung der Physik}}

\author{{\Large Johann Pascher}\\
	Abteilung für Kommunikationstechnik\\
	HTL Leonding, Österreich\\
	\texttt{johann.pascher@gmail.com}}

\date{\today}



==================================================

=== Zusammenfassung_En.tex.preamble ===

\documentclass[12pt,a4paper]{article}
\usepackage[utf8]{inputenc}
\usepackage[T1]{fontenc}
\usepackage[english]{babel}
\usepackage[left=2cm,right=2cm,top=2cm,bottom=2cm]{geometry}
\usepackage{lmodern}
\usepackage{amsmath}
\usepackage{amssymb}
\usepackage{physics}
\usepackage{booktabs}
\usepackage{tcolorbox}
\usepackage{siunitx}
\usepackage[table,xcdraw]{xcolor}
\usepackage{hyperref}
\usepackage{array}
\usepackage{textgreek}
\usepackage{fancyhdr}
\usepackage{enumitem}
\usepackage{graphicx}
\usepackage{float}

% Color definitions
\definecolor{t0blue}{RGB}{0,102,204}
\definecolor{t0green}{RGB}{0,153,76}
\definecolor{t0red}{RGB}{204,0,51}
\definecolor{t0yellow}{RGB}{255,204,0}
\definecolor{t0purple}{RGB}{102,0,204}

% Header and Footer
\pagestyle{fancy}
\fancyhf{}
\fancyhead[L]{\color{t0blue}\textsc{T0 Model: Complete Summary}}
\fancyhead[R]{\color{t0blue}\textsc{Johann Pascher}}
\fancyfoot[C]{\thepage}
\renewcommand{\headrulewidth}{0.4pt}
\renewcommand{\footrulewidth}{0.4pt}

% Custom commands
\newcommand{\xipar}{\ensuremath{\xi}}
\newcommand{\Efield}{E_\text{field}}
\newcommand{\lP}{\ell_{\text{P}}}
\newcommand{\Mpl}{M_{\text{Pl}}}
\newcommand{\EP}{E_{\text{P}}}
\newcommand{\alphaEM}{\alpha_{\text{EM}}}
\newcommand{\betaT}{\beta_{\text{T}}}

% Tcolorbox environments
\newtcolorbox{important}[1][]{
	colback=t0yellow!10!white,
	colframe=t0yellow!75!black,
	fonttitle=\bfseries,
	title=Important Insight,
	#1
}

\newtcolorbox{formula}[1][]{
	colback=t0blue!5!white,
	colframe=t0blue!75!black,
	fonttitle=\bfseries,
	title=T0 Prediction,
	#1
}

\newtcolorbox{revolutionary}[1][]{
	colback=t0red!5!white,
	colframe=t0red!75!black,
	fonttitle=\bfseries,
	title=New Perspective,
	#1
}

\newtcolorbox{experimental}[1][]{
	colback=t0green!5!white,
	colframe=t0green!75!black,
	fonttitle=\bfseries,
	title=Experimental Verification,
	#1
}

\newtcolorbox{quantum}[1][]{
	colback=t0purple!5!white,
	colframe=t0purple!75!black,
	fonttitle=\bfseries,
	title=Quantum Mechanics,
	#1
}

% Hyperref settings
\hypersetup{
	colorlinks=true,
	linkcolor=t0blue,
	citecolor=t0blue,
	urlcolor=t0blue,
	pdftitle={T0 Model: Complete Theoretical Summary},
	pdfauthor={Johann Pascher},
	pdfsubject={Theoretical Physics, T0 Theory, Unification},
	pdfkeywords={T0 Model, Geometric Constant, Quantum Mechanics, Unified Field Theory}
}

\title{{\Huge \color{t0blue}The T0 Model}\\
	{\LARGE A Comprehensive Theoretical Treatise}\\
	\vspace{1cm}
	{\Large From Geometric Constant to Physics Unification}}

\author{{\Large Johann Pascher}\\
	Department of Communications Engineering\\
	HTL Leonding, Austria\\
	\texttt{johann.pascher@gmail.com}}

\date{\today}



==================================================

=== Zwei-Dipole-CMB_De.tex.preamble ===

\documentclass{article}
\usepackage[utf8]{inputenc}
\usepackage[ngerman]{babel}
\usepackage{amsmath}
\usepackage{amsfonts}
\usepackage{array}
\usepackage{booktabs}
\usepackage[margin=1in]{geometry}
\usepackage[breaklinks=true]{hyperref}

\title{Kommentar: CMB- und Quasar-Dipol-Anomalie -- Eine dramatische Bestätigung der T0-Vorhersagen!}
\author{}
\date{}



==================================================

=== Zwei-Dipole-CMB_En.tex.preamble ===

\documentclass{article}
\usepackage[utf8]{inputenc}
\usepackage[english]{babel}
\usepackage{amsmath}
\usepackage{amsfonts}
\usepackage{array}
\usepackage{booktabs}
\usepackage[margin=1in]{geometry}
\usepackage[breaklinks=true]{hyperref}

\title{Commentary: CMB and Quasar Dipole Anomaly -- A Dramatic Confirmation of T0 Predictions!}
\author{}
\date{}



==================================================

=== _test_all_chapters.tex.preamble ===

\documentclass[11pt,a4paper]{book}
\usepackage[utf8]{inputenc}
\usepackage[english]{babel}
\usepackage{lmodern}
\usepackage{amsmath,amssymb,amsthm}
\usepackage{graphicx}
\usepackage[unicode,pdfencoding=auto]{hyperref}
\usepackage{booktabs}
\usepackage{longtable}
\usepackage{siunitx}
\usepackage{fancyhdr}
\usepackage{float}
\usepackage{tikz}
\usepackage[most]{tcolorbox}
\usepackage{xcolor}
\setlength{\parindent}{0pt}
\setlength{\parskip}{6pt}
\pagestyle{fancy}
\definecolor{t0blue}{RGB}{0,102,204}
\definecolor{t0red}{RGB}{192,0,0}
\definecolor{t0green}{RGB}{0,128,64}
\definecolor{t0orange}{RGB}{255,140,0}
\tikzset{
  t0blue/.style={draw=t0blue, fill=t0blue!20},
  t0red/.style={draw=t0red, fill=t0red!20},
  t0green/.style={draw=t0green, fill=t0green!20},
  t0orange/.style={draw=t0orange, fill=t0orange!20},
}
\newcommand{\checkmarkxa}{\checkmark}
\newcommand{\warningxa}{\textbf{!}}
\newcommand\{Cconv}{\newcommand\Cconv{C_{\text{conv}}
\newcommand\{Dfrak}{\mathcal{D}}
\newcommand\{EP}{\newcommand\EP{E_{\text{P}}
\newcommand\{EPratio}{\newcommand\EPratio[1]{\frac{#1}}
\newcommand\{EW}{\newcommand\EW{E_W}}
\newcommand\{EZ}{\newcommand\EZ{E_Z}}
\newcommand\{Ee}{\newcommand\Ee{E_e}}
\newcommand\{Efield}{\newcommand\Efield{E(x,t)}}
\newcommand\{Egamma}{\newcommand\Egamma{E_\gamma}}
\newcommand\{Eh}{\newcommand\Eh{E_h}}
\newcommand\{Emu}{\newcommand\Emu{E_\mu}}
\newcommand\{En}{\newcommand\En{E_n}}
\newcommand\{Enorm}{\newcommand\Enorm[1]{E_{\text{norm}}
\newcommand\{Ep}{\newcommand\Ep{E_p}}
\newcommand\{Eratio}{\newcommand\Eratio[2]{\frac{E_{#1}}
\newcommand\{Etau}{\newcommand\Etau{E_\tau}}
\newcommand\{Gnat}{\newcommand\Gnat{G_{\text{nat}}
\newcommand\{Gsi}{\newcommand\Gsi{G_{\text{SI}}
\newcommand\{Kfrak}{\newcommand\Kfrak{K_{\text{frak}}
\newcommand\{Lag}{\newcommand\Lag{\mathcal{L}}
\newcommand\{Lambdat}{\newcommand\Lambdat{\Lambda_T}}
\newcommand\{Rzero}{\newcommand\Rzero{R_\infty}}
\newcommand\{Tfield}{\newcommand\Tfield{T(x,t)}}
\newcommand\{Tfieldt}{\newcommand\Tfieldt{T(\vec{x}}
\newcommand\{Tzero}{\newcommand\Tzero{T_0}}
\newcommand\{alphaEM}{\newcommand\alphaEM{\alpha_{\text{EM}}
\newcommand\{alphaQCD}{\newcommand\alphaQCD{\alpha_s}}
\newcommand\{alphaQED}{\newcommand\alphaQED{\alpha_{\text{QED}}
\newcommand\{alphaT}{\newcommand\alphaT{\alpha_{\text{T}}
\newcommand\{alphaW}{\newcommand\alphaW{\alpha_{\text{W}}
\newcommand\{alphafine}{\newcommand\alphafine{\alpha}}
\newcommand\{alphapar}{\alpha}
\newcommand\{betaT}{\newcommand\betaT{\beta_{T}}
\newcommand\{betapar}{\beta}
\newcommand\{cftchapfont}{\renewcommand\cftchapfont{\large\bfseries\color{blue}}
\newcommand\{cftchappagefont}{\renewcommand\cftchappagefont{\large\bfseries\color{blue}}
\newcommand\{cftsecfont}{\renewcommand\cftsecfont{\color{blue}}
\newcommand\{cftsecpagefont}{\renewcommand\cftsecpagefont{\color{blue}}
\newcommand\{cftsubsecfont}{\renewcommand\cftsubsecfont{\color{blue!80!black}}
\newcommand\{cftsubsecpagefont}{\renewcommand\cftsubsecpagefont{\color{blue!80!black}}
\newcommand\{cftsubsubsecfont}{\renewcommand\cftsubsubsecfont{\color{blue!60!black}}
\newcommand\{cftsubsubsecpagefont}{\renewcommand\cftsubsubsecpagefont{\color{blue!60!black}}
\newcommand\{cfttoctitlefont}{\renewcommand\cfttoctitlefont{\huge\bfseries\color{blue}}
\newcommand\{deltaE}{\newcommand\deltaE{\delta E}}
\newcommand\{deltafield}{\delta\phi}
\newcommand\{footrulewidth}{\renewcommand\footrulewidth{0.4pt}}
\newcommand\{gW}{\newcommand\gW{g_W}}
\newcommand\{gs}{\newcommand\gs{g_s}}
\newcommand\{headrulewidth}{\renewcommand\headrulewidth{0.4pt}}
\newcommand\{lP}{\newcommand\lP{\ell_{\text{P}}
\newcommand\{lambdah}{\newcommand\lambdah{\lambda_h}}
\newcommand\{rzero}{\newcommand\rzero{r_0}}
\newcommand\{tP}{\newcommand\tP{t_{\text{P}}
\newcommand\{tzero}{\newcommand\tzero{t_0}}
\newcommand\{vecx}{\newcommand\vecx{\vec{x}}
\newcommand\{xigeom}{\newcommand\xigeom{\xi_{\text{geom}}
\newcommand\{xipar}{\newcommand\xipar{\xi_0}}
\newcommand\{xiparticle}{\newcommand\xiparticle{\xi_{\text{particle}}
\newcommand\{xirat}{\newcommand\xirat{\xi_{\text{ratio}}
\newenvironment{abstract}{
  \begin{center}\bfseries Abstract\end{center}\small
}{\par}


==================================================

=== bell-myon.tex.preamble ===

\documentclass[12pt,a4paper]{article}
\usepackage[utf8]{inputenc}
\usepackage[T1]{fontenc}
\usepackage[german]{babel}
\usepackage{amsmath}
\usepackage{amssymb}
\usepackage{booktabs}
\usepackage{xcolor}
\usepackage{tcolorbox}
\usepackage[left=2.5cm,right=2.5cm,top=2.5cm,bottom=2.5cm]{geometry}
\usepackage{hyperref}
\usepackage{enumitem}

\title{Die Rolle der Bell-Tests in Verbindung mit der Myon-Anomalie}
\author{Analyse der T0-Quantenkorrelationen}
\date{\today}

\newtcolorbox{question}{
	colback=blue!5!white,
	colframe=blue!75!black,
	title=Frage
}

\newtcolorbox{answer}{
	colback=green!5!white,
	colframe=green!75!black,
	title=Antwort
}

\newtcolorbox{technical}{
	colback=purple!5!white,
	colframe=purple!75!black,
	title=Technische Details
}

\newtcolorbox{critical}{
	colback=red!5!white,
	colframe=red!75!black,
	title=Kritische Analyse
}



==================================================

=== cosmic_De.tex.preamble ===

\documentclass[12pt,a4paper]{article}
\usepackage[utf8]{inputenc}
\usepackage[T1]{fontenc}
\usepackage[german]{babel}
\usepackage[left=2cm,right=2cm,top=2cm,bottom=2cm]{geometry}
\usepackage{lmodern}
\usepackage{amsmath}
\usepackage{amssymb}
\usepackage{physics}
\usepackage{hyperref}
\usepackage{tcolorbox}
\usepackage{booktabs}
\usepackage{enumitem}
\usepackage[table,xcdraw]{xcolor}
\usepackage{longtable}
\usepackage{siunitx}
\usepackage{fancyhdr}
\usepackage{textgreek}

% Header and Footer
\pagestyle{fancy}
\fancyhf{}
\fancyhead[L]{T0-Theorie: Kosmische Beziehungen}
\fancyhead[R]{\thepage}
\fancyfoot[C]{\textit{Von der universellen ξ-Konstante zu kosmischen Strukturen}}
\renewcommand{\headrulewidth}{0.4pt}
\renewcommand{\footrulewidth}{0.4pt}

\hypersetup{
	colorlinks=true,
	linkcolor=blue,
	citecolor=blue,
	urlcolor=blue,
	pdftitle={T0-Theorie: Kosmische Beziehungen und universelle $\xi$-Konstante},
	pdfauthor={T0-Theorie Projekt},
	pdfsubject={Kosmologie, $\xi$-Feld, Gravitation, CMB, Casimir-Effekt}
}

% Custom environments
\newtcolorbox{important}[1][]{colback=yellow!10!white,colframe=yellow!50!black,fonttitle=\bfseries,title=Wichtiger Hinweis,#1}
\newtcolorbox{formula}[1][]{colback=blue!5!white,colframe=blue!75!black,fonttitle=\bfseries,title=Schlüsselformel,#1}
\newtcolorbox{revolutionary}[1][]{colback=red!5!white,colframe=red!75!black,fonttitle=\bfseries,title=Revolutionäre Erkenntnis,#1}
\newtcolorbox{experiment}[1][]{colback=green!5!white,colframe=green!75!black,fonttitle=\bfseries,title=Experimenteller Test,#1}
\newtcolorbox{sibox}[1][]{colback=orange!10!white,colframe=orange!75!black,fonttitle=\bfseries,title=SI-Einheiten (nur zur Referenz),#1}

\title{\Huge\textbf{T0-Theorie: Kosmische Beziehungen}\\
	\Large Die universelle $\xi$-Konstante als Schlüssel \\
	zu Gravitation, CMB und kosmischen Strukturen}
	\author{Johann Pascher\\
	Abteilung für Kommunikationstechnik, \\Höhere Technische Bundeslehranstalt (HTL), Leonding, Österreich\\
	\texttt{johann.pascher@gmail.com}}

\date{\today}



==================================================

=== cosmic_En.tex.preamble ===

\documentclass[12pt,a4paper]{article}
\usepackage[utf8]{inputenc}
\usepackage[T1]{fontenc}
\usepackage[english]{babel}
\usepackage[left=2cm,right=2cm,top=2cm,bottom=2cm]{geometry}
\usepackage{lmodern}
\usepackage{amsmath}
\usepackage{amssymb}
\usepackage{physics}
\usepackage{hyperref}
\usepackage{tcolorbox}
\usepackage{booktabs}
\usepackage{enumitem}
\usepackage[table,xcdraw]{xcolor}
\usepackage{longtable}
\usepackage{siunitx}
\usepackage{fancyhdr}
\usepackage{textgreek}

% Header and Footer
\pagestyle{fancy}
\fancyhf{}
\fancyhead[L]{T0-Theory: Cosmic Relations}
\fancyhead[R]{\thepage}
\fancyfoot[C]{\textit{From the universal $\xi$-constant to cosmic structures}}
\renewcommand{\headrulewidth}{0.4pt}
\renewcommand{\footrulewidth}{0.4pt}

\hypersetup{
	colorlinks=true,
	linkcolor=blue,
	citecolor=blue,
	urlcolor=blue,
	pdftitle={T0-Theory: Cosmic Relations and universal $\xi$-constant},
	pdfauthor={T0-Theory Project},
	pdfsubject={Cosmology, $\xi$-field, Gravitation, CMB, Casimir Effect}
}

% Custom environments
\newtcolorbox{important}[1][]{colback=yellow!10!white,colframe=yellow!50!black,fonttitle=\bfseries,title=Important Note,#1}
\newtcolorbox{formula}[1][]{colback=blue!5!white,colframe=blue!75!black,fonttitle=\bfseries,title=Key Formula,#1}
\newtcolorbox{revolutionary}[1][]{colback=red!5!white,colframe=red!75!black,fonttitle=\bfseries,title=Revolutionary Insight,#1}
\newtcolorbox{experiment}[1][]{colback=green!5!white,colframe=green!75!black,fonttitle=\bfseries,title=Experimental Test,#1}
\newtcolorbox{sibox}[1][]{colback=orange!10!white,colframe=orange!75!black,fonttitle=\bfseries,title=SI Units (for reference only),#1}

\title{\Huge\textbf{T0-Theory: Cosmic Relations}\\
	\Large The universal $\xi$-constant as key \\
	to gravitation, CMB and cosmic structures}
	\author{\Large Johann Pascher\\
	Department of Communications Engineering,\\
	Higher Technical Federal Institute (HTL), Leonding, Austria\\
	\texttt{johann.pascher@gmail.com}}

\date{\today}



==================================================

=== detailierte_formel_leptonen_anemal_De.tex.preamble ===

\documentclass[12pt,a4paper]{article}
\usepackage[utf8]{inputenc}
\usepackage[T1]{fontenc}
\usepackage[german]{babel}
\usepackage{lmodern}
\usepackage{amsmath}
\usepackage{amssymb}
\usepackage{physics}
\usepackage{hyperref}
\usepackage{tcolorbox}
\usepackage{booktabs}
\usepackage{enumitem}
\usepackage[table,xcdraw]{xcolor}
\usepackage[left=2cm,right=2cm,top=2cm,bottom=2cm]{geometry}
\usepackage{pgfplots}
\pgfplotsset{compat=1.18}
\usepackage{graphicx}
\usepackage{float}
\usepackage{fancyhdr}
\usepackage{siunitx}
\usepackage{mathtools}
\usepackage{amsthm}
\usepackage{cleveref}
\usepackage{tikz}
\usepackage{microtype}
\usepackage{array}

\hypersetup{
	colorlinks=true,
	linkcolor=blue,
	urlcolor=blue,
	citecolor=blue,
	pdftitle={T0 Model: Detaillierte Formel für leptonische Anomalien},
	pdfauthor={Johann Pascher},
	pdfsubject={Theoretical Physics},
	pdfkeywords={T0 Model, Lepton Anomalies, Magnetic Moments}
}

\newcommand{\xipar}{\xi}
\newcommand{\nulep}{\nu}
\newcommand{\alphagem}{\alpha}
%\newcommand{\aleph}{\aleph}

\pagestyle{fancy}
\fancyhf{}
\fancyhead[L]{Johann Pascher}
\fancyhead[R]{T0-Modell: Detaillierte leptonische Anomalien}
\fancyfoot[C]{\thepage}
\renewcommand{\headrulewidth}{0.4pt}
\renewcommand{\footrulewidth}{0.4pt}

\tcbuselibrary{theorems}
\newtcolorbox{units}{colback=blue!5!white,colframe=blue!75!black,fonttitle=\bfseries}
\newtcolorbox{important}{colback=green!5!white,colframe=green!35!black,fonttitle=\bfseries}
\newtcolorbox{summary}{colback=yellow!5!white,colframe=orange!75!black,fonttitle=\bfseries}



==================================================

=== detailierte_formel_leptonen_anemal_En.tex.preamble ===

\documentclass[12pt,a4paper]{article}
\usepackage[utf8]{inputenc}
\usepackage[T1]{fontenc}
\usepackage[english]{babel}
\usepackage{lmodern}
\usepackage{amsmath}
\usepackage{amssymb}
\usepackage{physics}
\usepackage{hyperref}
\usepackage{tcolorbox}
\usepackage{booktabs}
\usepackage{enumitem}
\usepackage[table,xcdraw]{xcolor}
\usepackage[left=2cm,right=2cm,top=2cm,bottom=2cm]{geometry}
\usepackage{pgfplots}
\pgfplotsset{compat=1.18}
\usepackage{graphicx}
\usepackage{float}
\usepackage{fancyhdr}
\usepackage{siunitx}
\usepackage{mathtools}
\usepackage{amsthm}
\usepackage{cleveref}
\usepackage{tikz}
\usepackage{microtype}
\usepackage{array}

\hypersetup{
	colorlinks=true,
	linkcolor=blue,
	urlcolor=blue,
	citecolor=blue,
	pdftitle={T0 Model: Detailed Formula for Leptonic Anomalies},
	pdfauthor={Johann Pascher},
	pdfsubject={Theoretical Physics},
	pdfkeywords={T0 Model, Lepton Anomalies, Magnetic Moments}
}

\newcommand{\xipar}{\xi}
\newcommand{\nulep}{\nu}
\newcommand{\alphagem}{\alpha}
%\newcommand{\aleph}{\aleph}

\pagestyle{fancy}
\fancyhf{}
\fancyhead[L]{Johann Pascher}
\fancyhead[R]{T0 Model: Detailed Leptonic Anomalies}
\fancyfoot[C]{\thepage}
\renewcommand{\headrulewidth}{0.4pt}
\renewcommand{\footrulewidth}{0.4pt}

\tcbuselibrary{theorems}
\newtcolorbox{units}{colback=blue!5!white,colframe=blue!75!black,fonttitle=\bfseries}
\newtcolorbox{important}{colback=green!5!white,colframe=green!35!black,fonttitle=\bfseries}
\newtcolorbox{summary}{colback=yellow!5!white,colframe=orange!75!black,fonttitle=\bfseries}



==================================================

=== diracDe.tex.preamble ===

\documentclass[12pt,a4paper]{article}
\usepackage[utf8]{inputenc}
\usepackage[T1]{fontenc}
\usepackage[ngerman]{babel}
\usepackage{lmodern}
\usepackage{amsmath}
\usepackage{amssymb}
\usepackage{physics}
\usepackage{hyperref}
\usepackage{tcolorbox}
\usepackage{booktabs}
\usepackage{enumitem}
\usepackage[table,xcdraw]{xcolor}
\usepackage[left=2cm,right=2cm,top=2cm,bottom=2cm]{geometry}
\usepackage{pgfplots}
\pgfplotsset{compat=1.18}
\usepackage{graphicx}
\usepackage{float}
\usepackage{fancyhdr}
\usepackage{siunitx}
\usepackage{array}
\usepackage{cleveref}
\usepackage{mathtools}
\usepackage{amsthm}

% Kopf- und Fußzeilen
\pagestyle{fancy}
\fancyhf{}
\fancyhead[L]{Johann Pascher}
\fancyhead[R]{Dirac-Gleichung im T0-Modell-Rahmenwerk}
\fancyfoot[C]{\thepage}
\renewcommand{\headrulewidth}{0.4pt}
\renewcommand{\footrulewidth}{0.4pt}

% Benutzerdefinierte Befehle im Einklang mit der T0-Modell-Referenz
\newcommand{\Tfield}{T(x)}
\newcommand{\Tfieldt}{T(\vec{x},t)}
\newcommand{\Tzero}{T_0}
\newcommand{\alphaEM}{\alpha_{\text{EM}}}
\newcommand{\alphaW}{\alpha_{\text{W}}}
\newcommand{\betaT}{\beta_{\text{T}}}
\newcommand{\Mpl}{M_{\text{Pl}}}
\newcommand{\vecx}{\vec{x}}
\newcommand{\gammaf}{\gamma_{\text{Lorentz}}}
\newcommand{\LCDM}{\Lambda\text{CDM}}
\newcommand{\DTmu}{D_{T,\mu}}
\newcommand{\calL}{\mathcal{L}}
\newcommand{\deq}{\displaystyle}
\newcommand{\e}{\mathrm{e}}
\newcommand{\dTdt}{\frac{d\Tfieldt}{dt}}
\newcommand{\pdTdt}{\frac{\partial\Tfieldt}{\partial t}}
\newcommand{\pdTdx}{\nabla\Tfieldt}
\newcommand{\lP}{\ell_{\text{P}}}
\newcommand{\xipar}{\xi}

\hypersetup{
	colorlinks=true,
	linkcolor=blue,
	citecolor=blue,
	urlcolor=blue,
	pdftitle={Integration der Dirac-Gleichung im T0-Modell: Natürliche-Einheiten-Rahmenwerk},
	pdfauthor={Johann Pascher},
	pdfsubject={Theoretische Physik},
	pdfkeywords={T0-Modell, Dirac-Gleichung, Natürliche Einheiten, Quantenfeldtheorie, Zeit-Masse-Dualität}
}

\newtheorem{theorem}{Theorem}[section]
\newtheorem{proposition}[theorem]{Proposition} 
\newtheorem{definition}[theorem]{Definition}



==================================================

=== diracEn.tex.preamble ===

\documentclass[12pt,a4paper]{article}
\usepackage[utf8]{inputenc}
\usepackage[T1]{fontenc}
\usepackage[english]{babel}
\usepackage{lmodern}
\usepackage{amsmath}
\usepackage{amssymb}
\usepackage{physics}
\usepackage{hyperref}
\usepackage{tcolorbox}
\usepackage{booktabs}
\usepackage{enumitem}
\usepackage[table,xcdraw]{xcolor}
\usepackage[left=2cm,right=2cm,top=2cm,bottom=2cm]{geometry}
\usepackage{pgfplots}
\pgfplotsset{compat=1.18}
\usepackage{graphicx}
\usepackage{float}
\usepackage{fancyhdr}
\usepackage{siunitx}
\usepackage{array}
\usepackage{cleveref}
\usepackage{mathtools}
\usepackage{amsthm}

% Headers and Footers
\pagestyle{fancy}
\fancyhf{}
\fancyhead[L]{Johann Pascher}
\fancyhead[R]{Dirac Equation in the T0 Model Framework}
\fancyfoot[C]{\thepage}
\renewcommand{\headrulewidth}{0.4pt}
\renewcommand{\footrulewidth}{0.4pt}

% Custom commands aligned with T0 model reference
\newcommand{\Tfield}{T(x)}
\newcommand{\Tfieldt}{T(\vec{x},t)}
\newcommand{\Tzero}{T_0}
\newcommand{\alphaEM}{\alpha_{\text{EM}}}
\newcommand{\alphaW}{\alpha_{\text{W}}}
\newcommand{\betaT}{\beta_{\text{T}}}
\newcommand{\Mpl}{M_{\text{Pl}}}
\newcommand{\vecx}{\vec{x}}
\newcommand{\gammaf}{\gamma_{\text{Lorentz}}}
\newcommand{\LCDM}{\Lambda\text{CDM}}
\newcommand{\DTmu}{D_{T,\mu}}
\newcommand{\calL}{\mathcal{L}}
\newcommand{\deq}{\displaystyle}
\newcommand{\e}{\mathrm{e}}
\newcommand{\dTdt}{\frac{d\Tfieldt}{dt}}
\newcommand{\pdTdt}{\frac{\partial\Tfieldt}{\partial t}}
\newcommand{\pdTdx}{\nabla\Tfieldt}
\newcommand{\lP}{\ell_{\text{P}}}
\newcommand{\xipar}{\xi}

\hypersetup{
	colorlinks=true,
	linkcolor=blue,
	citecolor=blue,
	urlcolor=blue,
	pdftitle={Integration of the Dirac Equation in the T0 Model: Natural Units Framework},
	pdfauthor={Johann Pascher},
	pdfsubject={Theoretical Physics},
	pdfkeywords={T0 Model, Dirac Equation, Natural Units, Quantum Field Theory, Time-Mass Duality}
}

\newtheorem{theorem}{Theorem}[section]
\newtheorem{proposition}[theorem]{Proposition} 
\newtheorem{definition}[theorem]{Definition}



==================================================

=== diracVereinfachtDe.tex.preamble ===

\documentclass[12pt,a4paper]{article}
\usepackage[utf8]{inputenc} % Für Umlaute und Sonderzeichen
\usepackage[T1]{fontenc}    % Für korrekte Silbentrennung und Font-Kodierung
\usepackage[ngerman]{babel} % Wichtig für die Neue Deutsche Rechtschreibung
\usepackage{lmodern}
\usepackage{amsmath}
\usepackage{amssymb}
\usepackage{physics}
\usepackage{hyperref}
\usepackage{tcolorbox}
\usepackage{booktabs}
\usepackage{enumitem}
\usepackage[table,xcdraw]{xcolor}
\usepackage[left=2cm,right=2cm,top=2cm,bottom=2cm]{geometry}
\usepackage{pgfplots}
\pgfplotsset{compat=1.18}
\usepackage{graphicx}
\usepackage{float}
\usepackage{fancyhdr}
\usepackage{siunitx}
\usepackage{array}
\usepackage{cleveref}


\usepackage{textcomp}
\usepackage{mathtools}
\usepackage{amsthm}

% Kopf- und Fußzeilen
\pagestyle{fancy}
\fancyhf{}
\fancyhead[L]{Johann Pascher}
\fancyhead[R]{Vereinfachte Dirac-Gleichung in der T0-Theorie}
\fancyfoot[C]{\thepage}
\renewcommand{\headrulewidth}{0.4pt}
\renewcommand{\footrulewidth}{0.4pt}
\setlength{\headheight}{15pt}

% Benutzerdefinierte Befehle
\newcommand{\Lag}{\mathcal{L}}
\newcommand{\deltam}{\delta m}
\newcommand{\xipar}{\xi}

% Theorem-Umgebungen
\newtheorem{theorem}{Theorem}[section]
\newtheorem{proposition}[theorem]{Proposition}
\newtheorem{corollary}[theorem]{Korollar}
\newtheorem{lemma}[theorem]{Lemma}
\theoremstyle{definition}
\newtheorem{definition}[theorem]{Definition}
\newtheorem{example}[theorem]{Beispiel}
\theoremstyle{remark}
\newtheorem{remark}[theorem]{Bemerkung}

\hypersetup{
	colorlinks=true,
	linkcolor=blue,
	citecolor=blue,
	urlcolor=blue,
	pdftitle={Vereinfachte Dirac-Gleichung in der T0-Theorie: Feldknoten-Ansatz},
	pdfauthor={Johann Pascher},
	pdfsubject={Theoretische Physik},
	pdfkeywords={T0-Theorie, Dirac-Gleichung, Feldknoten, Vereinfachte Lagrangedichte}
}

\title{Vereinfachte Dirac-Gleichung in der T0-Theorie: \\
	Von komplexen 4×4-Matrizen zu einfacher Feldknotendynamik \\
	\large Die revolutionäre Vereinheitlichung von Quantenmechanik und Feldtheorie}
\author{Johann Pascher\\
	Abteilung für Kommunikationstechnik, \\Höhere Technische Bundeslehranstalt (HTL), Leonding, Österreich\\
	\texttt{johann.pascher@gmail.com}}
\date{\today}



==================================================

=== diracVereinfachtEn.tex.preamble ===

\documentclass[12pt,a4paper]{article}
\usepackage[utf8]{inputenc}
\usepackage[T1]{fontenc}
\usepackage[english]{babel}
\usepackage{textcomp}
\usepackage{lmodern}
\usepackage{amsmath}
\usepackage{amssymb}
\usepackage{physics}
\usepackage{hyperref}
\usepackage{tcolorbox}
\usepackage{booktabs}
\usepackage{enumitem}
\usepackage[table,xcdraw]{xcolor}
\usepackage[left=2cm,right=2cm,top=2cm,bottom=2cm]{geometry}
\usepackage{graphicx}
\usepackage{float}
\usepackage{fancyhdr}
\usepackage{siunitx}
\usepackage{array}
\usepackage{cleveref}
\usepackage{mathtools}
\usepackage{amsthm}

% Headers and Footers
\pagestyle{fancy}
\fancyhf{}
\fancyhead[L]{Johann Pascher}
\fancyhead[R]{Simplified Dirac Equation in T0 Theory}
\fancyfoot[C]{\thepage}
\renewcommand{\headrulewidth}{0.4pt}
\renewcommand{\footrulewidth}{0.4pt}
\setlength{\headheight}{15pt}

% Custom commands
\newcommand{\Lag}{\mathcal{L}}
\newcommand{\deltam}{\delta m}
\newcommand{\xipar}{\xi}

% Theorem environments
\newtheorem{theorem}{Theorem}[section]
\newtheorem{proposition}[theorem]{Proposition}
\newtheorem{corollary}[theorem]{Corollary}
\newtheorem{lemma}[theorem]{Lemma}
\theoremstyle{definition}
\newtheorem{definition}[theorem]{Definition}
\newtheorem{example}[theorem]{Example}
\theoremstyle{remark}
\newtheorem{remark}[theorem]{Remark}

\hypersetup{
	colorlinks=true,
	linkcolor=blue,
	citecolor=blue,
	urlcolor=blue,
	pdftitle={Simplified Dirac Equation in T0 Theory: Field Node Approach},
	pdfauthor={Johann Pascher},
	pdfsubject={Theoretical Physics},
	pdfkeywords={T0 Theory, Dirac Equation, Field Nodes, Simplified Lagrangian}
}

\title{Simplified Dirac Equation in T0 Theory: \\
	From Complex 4×4 Matrices to Simple Field Node Dynamics \\
	\large The Revolutionary Unification of Quantum Mechanics and Field Theory}
\author{Johann Pascher\\
	Department of Communications Engineering, \\H\"ohere Technische Bundeslehranstalt (HTL), Leonding, Austria\\
	\texttt{johann.pascher@gmail.com}}
\date{\today}



==================================================

=== gravitationskonstante_De.tex.preamble ===

\documentclass[12pt,a4paper]{article}
\usepackage[utf8]{inputenc}
\usepackage[T1]{fontenc}
\usepackage[ngerman]{babel}
\usepackage{lmodern}
\usepackage{amsmath,amssymb,amsthm}
\usepackage{geometry}
\usepackage{booktabs}
\usepackage{array}
\usepackage{xcolor}
\usepackage{tcolorbox}
\usepackage{fancyhdr}
\usepackage{tocloft}
\usepackage{hyperref}
\usepackage{tikz}
\usepackage{physics}
\usepackage{siunitx}

\definecolor{deepblue}{RGB}{0,0,127}
\definecolor{deepred}{RGB}{191,0,0}
\definecolor{deepgreen}{RGB}{0,127,0}

\geometry{a4paper, margin=2.5cm}

\usetikzlibrary{positioning, arrows.meta}

% Header- und Footer-Konfiguration
\pagestyle{fancy}
\fancyhf{}
\fancyhead[L]{\textsc{T0-Gravitationskonstante}}
\fancyhead[R]{\textsc{J. Pascher}}
\fancyfoot[C]{\thepage}
\renewcommand{\headrulewidth}{0.4pt}
\renewcommand{\footrulewidth}{0.4pt}

% Inhaltsverzeichnis-Stil - Blau
\renewcommand{\cfttoctitlefont}{\huge\bfseries\color{blue}}
\renewcommand{\cftsecfont}{\color{blue}}
\renewcommand{\cftsubsecfont}{\color{blue}}
\renewcommand{\cftsecpagefont}{\color{blue}}
\renewcommand{\cftsubsecpagefont}{\color{blue}}
\setlength{\cftsecindent}{0pt}
\setlength{\cftsubsecindent}{0pt}

% Hyperref-Einstellungen
\hypersetup{
	colorlinks=true,
	linkcolor=blue,
	citecolor=blue,
	urlcolor=blue,
	pdftitle={T0-Theorie: Herleitung der Gravitationskonstanten},
	pdfauthor={Johann Pascher},
	pdfsubject={T0-Theorie, Dimensionsanalyse, Umrechnungsfaktoren}
}

% Benutzerdefinierte Befehle
\newcommand{\xipar}{\xi}
\newcommand{\Kfrak}{K_{\text{frak}}}
\newcommand{\Cconv}{C_{\text{conv}}}

% Umgebung für Schlüsselergebnisse
\newtcolorbox{keyresult}[1][]{
	colback=blue!5, 
	colframe=blue!75!black, 
	fonttitle=\bfseries,
	title=#1
}
\newtcolorbox{correct}[1][]{
	colback=green!5, 
	colframe=green!75!black, 
	fonttitle=\bfseries,
	title=#1
}
\newtcolorbox{analysis}[1][]{
	colback=yellow!5, 
	colframe=orange!75!black, 
	fonttitle=\bfseries,
	title=#1
}

\title{\textbf{T0-Theorie: Herleitung der Gravitationskonstanten}\\
	\large Dimensionsanalytisch konsistente Formel mit expliziten Umrechnungsfaktoren\\[0.3cm]
	\normalsize Systematische Ableitung aus fundamentalen T0-Prinzipien}
\author{Johann Pascher\\
	Abteilung für Kommunikationstechnologie\\
	Höhere Technische Lehranstalt (HTL), Leonding, Österreich\\
	\texttt{johann.pascher@gmail.com}}
\date{\today}



==================================================

=== gravitationskonstante_En.tex.preamble ===

\documentclass[12pt,a4paper]{article}
\usepackage[utf8]{inputenc}
\usepackage[T1]{fontenc}
\usepackage[english]{babel}
\usepackage{lmodern}
\usepackage{amsmath,amssymb,amsthm}
\usepackage{geometry}
\usepackage{booktabs}
\usepackage{array}
\usepackage{xcolor}
\usepackage{tcolorbox}
\usepackage{fancyhdr}
\usepackage{tocloft}
\usepackage{hyperref}
\usepackage{tikz}
\usepackage{physics}
\usepackage{siunitx}

\definecolor{deepblue}{RGB}{0,0,127}
\definecolor{deepred}{RGB}{191,0,0}
\definecolor{deepgreen}{RGB}{0,127,0}

\geometry{a4paper, margin=2.5cm}

\usetikzlibrary{positioning, arrows.meta}

% Header and Footer Configuration
\pagestyle{fancy}
\fancyhf{}
\fancyhead[L]{\textsc{T0-Gravitational Constant}}
\fancyhead[R]{\textsc{J. Pascher}}
\fancyfoot[C]{\thepage}
\renewcommand{\headrulewidth}{0.4pt}
\renewcommand{\footrulewidth}{0.4pt}

% Table of Contents Style - Blue
\renewcommand{\cfttoctitlefont}{\huge\bfseries\color{blue}}
\renewcommand{\cftsecfont}{\color{blue}}
\renewcommand{\cftsubsecfont}{\color{blue}}
\renewcommand{\cftsecpagefont}{\color{blue}}
\renewcommand{\cftsubsecpagefont}{\color{blue}}
\setlength{\cftsecindent}{0pt}
\setlength{\cftsubsecindent}{0pt}

% Hyperref Settings
\hypersetup{
	colorlinks=true,
	linkcolor=blue,
	citecolor=blue,
	urlcolor=blue,
	pdftitle={T0-Theory: Derivation of the Gravitational Constant},
	pdfauthor={Johann Pascher},
	pdfsubject={T0-Theory, Dimensional Analysis, Conversion Factors}
}

% Custom Commands
\newcommand{\xipar}{\xi}
\newcommand{\Kfrak}{K_{\text{frak}}}
\newcommand{\Cconv}{C_{\text{conv}}}

% Environment for Key Results
\newtcolorbox{keyresult}[1][]{
	colback=blue!5, 
	colframe=blue!75!black, 
	fonttitle=\bfseries,
	title=#1
}
\newtcolorbox{correct}[1][]{
	colback=green!5, 
	colframe=green!75!black, 
	fonttitle=\bfseries,
	title=#1
}
\newtcolorbox{analysis}[1][]{
	colback=yellow!5, 
	colframe=orange!75!black, 
	fonttitle=\bfseries,
	title=#1
}

\title{\textbf{T0-Theory: Derivation of the Gravitational Constant}\\
	\large Dimensionally Consistent Formula with Explicit Conversion Factors\\[0.3cm]
	\normalsize Systematic Derivation from Fundamental T0 Principles}
\author{Johann Pascher\\
	Department of Communication Technology\\
	Higher Technical Federal Institute (HTL), Leonding, Austria\\
	\texttt{johann.pascher@gmail.com}}
\date{\today}



==================================================

=== gravitationskonstnte_De.tex.preamble ===

\documentclass[12pt,a4paper]{article}
\usepackage[utf8]{inputenc}
\usepackage[ngerman]{babel}
\usepackage{amsmath,amssymb,amsthm}
\usepackage{graphicx}
\usepackage{color}
\usepackage{hyperref}
\usepackage{geometry}
\geometry{margin=2.5cm}
\usepackage{fancyhdr}
\usepackage{setspace}
\usepackage{booktabs}
\hypersetup{
	colorlinks=true,
	linkcolor=blue,
	citecolor=blue,
	urlcolor=blue,
}
\usepackage{physics}
\usepackage{xcolor}
\usepackage{tcolorbox}
\definecolor{deepblue}{RGB}{0,0,127}
\definecolor{deepred}{RGB}{191,0,0}
\definecolor{deepgreen}{RGB}{0,127,0}

% Header Definition nach Pascher
\pagestyle{fancy}
\fancyhf{}
\fancyhead[L]{\textbf{T0-Theorie: G aus SI-Konstanten}}
\fancyhead[R]{\textbf{Johann Pascher, 2025}}
\fancyfoot[C]{\thepage}
\renewcommand{\headrulewidth}{0.4pt}
\setlength{\headheight}{15pt}

% Theoreme und Definitionen
\theoremstyle{definition}
\newtheorem{definition}{Definition}[section]
\newtheorem{theorem}{Theorem}[section]
\newtheorem{lemma}{Lemma}[section]
\newtheorem{corollary}{Korollar}[section]

% Abstände
\setstretch{1.2}

\newtcolorbox{formula}[1][]{
	colback=blue!5!white,
	colframe=blue!75!black,
	fonttitle=\bfseries,
	title=#1
}

\newtcolorbox{result}[1][]{
	colback=green!5!white,
	colframe=green!75!black,
	fonttitle=\bfseries,
	title=#1
}

\newtcolorbox{revolution}[1][]{
	colback=red!5!white,
	colframe=red!75!black,
	fonttitle=\bfseries,
	title=#1
}

\title{\textbf{Berechnung der Gravitationskonstanten aus SI-Konstanten}\\[0.5cm]
	\large Die T0-Theorie: Emergenz von G aus der Raumzeit-Geometrie\\[0.3cm]
	\normalsize Vollständige Herleitung ohne experimentelle Eingangswerte}
\author{Johann Pascher\\
	\small Abteilung Kommunikationstechnik,\\
	\small Höhere Technische Lehranstalt (HTL), Leonding, Österreich\\
	\small \texttt{johann.pascher@gmail.com}}
\date{Dezember 2025}



==================================================

=== gravitationskonstnte_En.tex.preamble ===

\documentclass[12pt,a4paper]{article}
\usepackage[utf8]{inputenc}
\usepackage[english]{babel}
\usepackage{amsmath,amssymb,amsthm}
\usepackage{graphicx}
\usepackage{color}
\usepackage{hyperref}
\usepackage{geometry}
\geometry{margin=2.5cm}
\usepackage{fancyhdr}
\usepackage{setspace}
\usepackage{booktabs}
\hypersetup{
	colorlinks=true,
	linkcolor=blue,
	citecolor=blue,
	urlcolor=blue,
}
\usepackage{physics}
\usepackage{xcolor}
\usepackage{tcolorbox}
\definecolor{deepblue}{RGB}{0,0,127}
\definecolor{deepred}{RGB}{191,0,0}
\definecolor{deepgreen}{RGB}{0,127,0}

% Header Definition by Pascher
\pagestyle{fancy}
\fancyhf{}
\fancyhead[L]{\textbf{T0-Theory: G from SI Constants}}
\fancyhead[R]{\textbf{Johann Pascher, 2025}}
\fancyfoot[C]{\thepage}
\renewcommand{\headrulewidth}{0.4pt}
\setlength{\headheight}{15pt}

% Theorems and Definitions
\theoremstyle{definition}
\newtheorem{definition}{Definition}[section]
\newtheorem{theorem}{Theorem}[section]
\newtheorem{lemma}{Lemma}[section]
\newtheorem{corollary}{Corollary}[section]

% Spacing
\setstretch{1.2}

\newtcolorbox{formula}[1][]{
	colback=blue!5!white,
	colframe=blue!75!black,
	fonttitle=\bfseries,
	title=#1
}

\newtcolorbox{result}[1][]{
	colback=green!5!white,
	colframe=green!75!black,
	fonttitle=\bfseries,
	title=#1
}

\newtcolorbox{revolution}[1][]{
	colback=red!5!white,
	colframe=red!75!black,
	fonttitle=\bfseries,
	title=#1
}

\title{\textbf{Calculation of the Gravitational Constant from SI Constants}\\[0.5cm]
	\large The T0-Theory: Emergence of G from Spacetime Geometry\\[0.3cm]
	\normalsize Complete derivation without experimental input values}
\author{Johann Pascher\\
	\small Department of Communication Engineering,\\
	\small Higher Technical Institute (HTL), Leonding, Austria\\
	\small \texttt{johann.pascher@gmail.com}}
\date{December 2025}



==================================================

=== lagrandian-einfachDe.tex.preamble ===

\documentclass[12pt,a4paper]{article}
\usepackage[utf8]{inputenc}
\usepackage[T1]{fontenc}
\usepackage[ngerman]{babel}
\usepackage{textcomp}
\usepackage{lmodern}
\usepackage{amsmath}
\usepackage{amssymb}
\usepackage{physics}
\usepackage{hyperref}
\usepackage{tcolorbox}
\usepackage{booktabs}
\usepackage{enumitem}
\usepackage[table,xcdraw]{xcolor}
\usepackage[left=2cm,right=2cm,top=2cm,bottom=2cm]{geometry}
\usepackage{pgfplots}
\pgfplotsset{compat=1.18}
\usepackage{graphicx}
\usepackage{float}
\usepackage{fancyhdr}
\usepackage{siunitx}
\usepackage{array}
\usepackage{cleveref}
\usepackage{mathtools}
\usepackage{amsthm}

% Kopf- und Fußzeilen
\pagestyle{fancy}
\fancyhf{}
\fancyhead[L]{Johann Pascher}
\fancyhead[R]{Vereinfachte T0-Theorie}
\fancyfoot[C]{\thepage}
\renewcommand{\headrulewidth}{0.4pt}
\renewcommand{\footrulewidth}{0.4pt}
\setlength{\headheight}{15pt}

% Benutzerdefinierte Befehle
\newcommand{\Tfield}{T(x,t)}
\newcommand{\mfield}{m(x,t)}
\newcommand{\deltam}{\delta m}
\newcommand{\Lag}{\mathcal{L}}
\newcommand{\xipar}{\xi}

% Theorem-Umgebungen
\newtheorem{theorem}{Theorem}[section]
\newtheorem{proposition}[theorem]{Proposition}
\newtheorem{corollary}[theorem]{Korollar}
\newtheorem{lemma}[theorem]{Lemma}
\theoremstyle{definition}
\newtheorem{definition}[theorem]{Definition}
\newtheorem{example}[theorem]{Beispiel}
\theoremstyle{remark}
\newtheorem{remark}[theorem]{Bemerkung}

\hypersetup{
	colorlinks=true,
	linkcolor=blue,
	citecolor=blue,
	urlcolor=blue,
	pdftitle={Vereinfachte T0-Theorie: Elegante Lagrange-Dichte für Zeit-Masse-Dualität},
	pdfauthor={Johann Pascher},
	pdfsubject={Theoretische Physik},
	pdfkeywords={T0-Modell, Vereinfachte Lagrange-Funktion, Antiteilchen, Zeit-Masse-Dualität, Natürliche Einheiten}
}

\title{Vereinfachte T0-Theorie: \\
	Elegante Lagrange-Dichte für Zeit-Masse-Dualität \\
	\large Von der Komplexität zur fundamentalen Einfachheit}
\author{Johann Pascher\\
	Abteilung für Nachrichtentechnik, \\Höhere Technische Bundeslehranstalt (HTL), Leonding, Österreich\\
	\texttt{johann.pascher@gmail.com}}
\date{\today}



==================================================

=== lagrandian-einfachEn.tex.preamble ===

\documentclass[12pt,a4paper]{article}
\usepackage[utf8]{inputenc}
\usepackage[T1]{fontenc}
\usepackage[english]{babel}
\usepackage{textcomp}
\usepackage{lmodern}
\usepackage{amsmath}
\usepackage{amssymb}
\usepackage{physics}
\usepackage{hyperref}
\usepackage{tcolorbox}
\usepackage{booktabs}
\usepackage{enumitem}
\usepackage[table,xcdraw]{xcolor}
\usepackage[left=2cm,right=2cm,top=2cm,bottom=2cm]{geometry}
\usepackage{pgfplots}
\pgfplotsset{compat=1.18}
\usepackage{graphicx}
\usepackage{float}
\usepackage{fancyhdr}
\usepackage{siunitx}
\usepackage{array}
\usepackage{cleveref}
\usepackage{mathtools}
\usepackage{amsthm}

% Headers and Footers
\pagestyle{fancy}
\fancyhf{}
\fancyhead[L]{Johann Pascher}
\fancyhead[R]{Simplified T0 Theory}
\fancyfoot[C]{\thepage}
\renewcommand{\headrulewidth}{0.4pt}
\renewcommand{\footrulewidth}{0.4pt}

% Custom commands
\newcommand{\Tfield}{T(x,t)}
\newcommand{\mfield}{m(x,t)}
\newcommand{\deltam}{\delta m}
\newcommand{\Lag}{\mathcal{L}}
\newcommand{\xipar}{\xi}

% Theorem environments
\newtheorem{theorem}{Theorem}[section]
\newtheorem{proposition}[theorem]{Proposition}
\newtheorem{corollary}[theorem]{Corollary}
\newtheorem{lemma}[theorem]{Lemma}
\theoremstyle{definition}
\newtheorem{definition}[theorem]{Definition}
\newtheorem{example}[theorem]{Example}
\theoremstyle{remark}
\newtheorem{remark}[theorem]{Remark}

\hypersetup{
	colorlinks=true,
	linkcolor=blue,
	citecolor=blue,
	urlcolor=blue,
	pdftitle={Simplified T0 Theory: Elegant Lagrangian Density for Time-Mass Duality},
	pdfauthor={Johann Pascher},
	pdfsubject={Theoretical Physics},
	pdfkeywords={T0 Model, Simplified Lagrangian, Time-Mass Duality, Natural Units}
}

\title{Simplified T0 Theory: \\
	Elegant Lagrangian Density for Time-Mass Duality \\
	\large From Complexity to Fundamental Simplicity}
\author{Johann Pascher\\
	Department of Communications Engineering, \\Höhere Technische Bundeslehranstalt (HTL), Leonding, Austria\\
	\texttt{johann.pascher@gmail.com}}
\date{\today}



==================================================

=== musical-spiral-137-De.tex.preamble ===

\documentclass[12pt,a4paper]{article}
\usepackage[utf8]{inputenc}
\usepackage[T1]{fontenc}
\usepackage[german]{babel}
\usepackage[left=2.5cm,right=2.5cm,top=2.5cm,bottom=2.5cm]{geometry}
\usepackage{lmodern}
\usepackage{amsmath}
\usepackage{amssymb}
\usepackage{physics}
\usepackage{hyperref}
\usepackage{tcolorbox}
\usepackage{booktabs}
\usepackage{enumitem}
\usepackage{graphicx}
\usepackage{float}
\usepackage{fancyhdr}
\usepackage{siunitx}
\usepackage{array}
\usepackage{cleveref}
\usepackage{mathtools}
\usepackage{bm}
\usepackage{tikz}
\usepackage{pgfplots}
\pgfplotsset{compat=1.18}
\usepackage{longtable}

% Kopf- und Fußzeilen
\pagestyle{fancy}
\fancyhf{}
\fancyhead[L]{Johann Pascher}
\fancyhead[R]{Die Musikalische Spirale und die 137}
\fancyfoot[C]{\thepage}
\renewcommand{\headrulewidth}{0.4pt}
\renewcommand{\footrulewidth}{0.4pt}

% Benutzerdefinierte Befehle
\newcommand{\xipar}{\xi}
\newcommand{\Df}{D_f}

\hypersetup{
	colorlinks=true,
	linkcolor=blue,
	citecolor=blue,
	urlcolor=blue,
	pdftitle={Die Musikalische Spirale und die 137: Die mathematische Entdeckung der kosmischen Verstimmung},
	pdfauthor={Johann Pascher},
	pdfsubject={Theoretische Physik},
	pdfkeywords={T0-Theorie, Musikalische Spirale, Feinstrukturkonstante, 137, Analog-Digital Hybrid}
}

\title{Die Musikalische Spirale und die 137:\\
	Die mathematische Entdeckung der kosmischen Verstimmung}
\author{Johann Pascher\\
	Fachbereich Kommunikationstechnik,\\
	Höhere Technische Lehranstalt (HTL), Leonding, Österreich\\
	\texttt{johann.pascher@gmail.com}}
\date{\today}



==================================================

=== musical-spiral-137-En.tex.preamble ===

\documentclass[12pt,a4paper]{article}
\usepackage[utf8]{inputenc}
\usepackage[T1]{fontenc}
\usepackage[english]{babel}
\usepackage[left=2.5cm,right=2.5cm,top=2.5cm,bottom=2.5cm]{geometry}
\usepackage{lmodern}
\usepackage{amsmath}
\usepackage{amssymb}
\usepackage{physics}
\usepackage{hyperref}
\usepackage{tcolorbox}
\usepackage{booktabs}
\usepackage{enumitem}
\usepackage{graphicx}
\usepackage{float}
\usepackage{fancyhdr}
\usepackage{siunitx}
\usepackage{array}
\usepackage{cleveref}
\usepackage{mathtools}
\usepackage{bm}
\usepackage{tikz}
\usepackage{pgfplots}
\pgfplotsset{compat=1.18}
\usepackage{longtable}

% Headers and Footers
\pagestyle{fancy}
\fancyhf{}
\fancyhead[L]{Johann Pascher}
\fancyhead[R]{The Musical Spiral and 137}
\fancyfoot[C]{\thepage}
\renewcommand{\headrulewidth}{0.4pt}
\renewcommand{\footrulewidth}{0.4pt}

% Custom commands
\newcommand{\xipar}{\xi}
\newcommand{\Df}{D_f}

\hypersetup{
	colorlinks=true,
	linkcolor=blue,
	citecolor=blue,
	urlcolor=blue,
	pdftitle={The Musical Spiral and 137: The Mathematical Discovery of Cosmic Detuning},
	pdfauthor={Johann Pascher},
	pdfsubject={Theoretical Physics},
	pdfkeywords={T0 Theory, Musical Spiral, Fine Structure Constant, 137, Analog-Digital Hybrid}
}

\title{The Musical Spiral and 137:\\
	The Mathematical Discovery of Cosmic Detuning}
\author{Johann Pascher\\
	Department of Communication Technology,\\
	Higher Technical College (HTL), Leonding, Austria\\
	\texttt{johann.pascher@gmail.com}}
\date{\today}



==================================================

=== neutrino-Formel_De.tex.preamble ===

\documentclass[12pt,a4paper]{article}
\usepackage[utf8]{inputenc}
\usepackage[T1]{fontenc}
\usepackage[german]{babel}
\usepackage[left=2cm,right=2cm,top=2cm,bottom=2cm]{geometry}
\usepackage{lmodern}
\usepackage{amsmath}
\usepackage{amssymb}
\usepackage{physics}
\usepackage{hyperref}
\usepackage{tcolorbox}
\usepackage{booktabs}
\usepackage{enumitem}
\usepackage[table,xcdraw]{xcolor}
\usepackage{longtable}
\usepackage{array}
\usepackage{multirow}
\usepackage{siunitx}
\usepackage{mathtools}
\usepackage{amsthm}
\usepackage{fancyhdr}
\usepackage{microtype}
\usepackage{float}
\usepackage{graphicx}

% Erweiterte Typographische Einstellungen
\emergencystretch 3em
\tolerance 9999
\hbadness 9999
\setlength{\hfuzz}{15pt}

% Header and Footer Configuration
\pagestyle{fancy}
\fancyhf{}
\fancyhead[L]{Johann Pascher}
\fancyhead[R]{T0-Modell: Einheitliche Neutrino-Formel-Struktur}
\fancyfoot[C]{\thepage}
\renewcommand{\headrulewidth}{0.4pt}
\renewcommand{\footrulewidth}{0.4pt}

% Custom Commands
\newcommand{\Efield}{E_{\text{Feld}}}
\newcommand{\xipar}{\xi}
\newcommand{\Tzero}{T_0}
\newcommand{\vecx}{\vec{x}}

% Hyperlink-Setup
\hypersetup{
	colorlinks=true,
	linkcolor=blue,
	citecolor=blue,
	urlcolor=blue,
	pdftitle={T0-Modell: Einheitliche Neutrino-Formel-Struktur},
	pdfauthor={Johann Pascher},
	pdfsubject={T0-Modell, Neutrino-Massen, Spekulative Extrapolationen},
	pdfkeywords={Neutrinos, Geometrische Harmonien, Mathematische Konsistenz, Spekulative Physik, Neutrino-Oszillation, Geometrische Phasen}
}

% Custom environments
\newtcolorbox{important}[1][]{
	colback=yellow!10!white,
	colframe=yellow!50!black,
	fonttitle=\bfseries,
	title=Wichtiger Hinweis,
	#1
}

\newtcolorbox{formula}[1][]{
	colback=blue!5!white,
	colframe=blue!75!black,
	fonttitle=\bfseries,
	title=Mathematische Formel,
	#1
}

\newtcolorbox{warning}[1][]{
	colback=red!5!white,
	colframe=red!75!black,
	fonttitle=\bfseries,
	title=Wissenschaftliche Warnung,
	#1
}

\newtcolorbox{experimental}[1][]{
	colback=green!5!white,
	colframe=green!75!black,
	fonttitle=\bfseries,
	title=Experimenteller Vergleich,
	#1
}

\newtcolorbox{speculation}[1][]{
	colback=purple!5!white,
	colframe=purple!75!black,
	fonttitle=\bfseries,
	title=Spekulative Hypothese,
	#1
}

\title{\Huge\textbf{T0-Modell: Einheitliche Neutrino-Formel-Struktur}\\
	\Large Mathematisch konsistente Extrapolationen \\
	bei spekulativer physikalischer Basis}

\author{Johann Pascher\\
	Abteilung für Kommunikationstechnik, \\
	Höhere Technische Bundeslehranstalt (HTL), Leonding, Österreich\\
	\texttt{johann.pascher@gmail.com}}

\date{\today}



==================================================

=== neutrino-Formel_En.tex.preamble ===

\documentclass[12pt,a4paper]{article}
\usepackage[utf8]{inputenc}
\usepackage[T1]{fontenc}
\usepackage[left=2cm,right=2cm,top=2cm,bottom=2cm]{geometry}
\usepackage{lmodern}
\usepackage{amsmath}
\usepackage{amssymb}
\usepackage{physics}
\usepackage{hyperref}
\usepackage{tcolorbox}
\usepackage{booktabs}
\usepackage{enumitem}
\usepackage[table,xcdraw]{xcolor}
\usepackage{longtable}
\usepackage{array}
\usepackage{multirow}
\usepackage{siunitx}
\usepackage{mathtools}
\usepackage{amsthm}
\usepackage{fancyhdr}
\usepackage{microtype}
\usepackage{float}
\usepackage{graphicx}

% Enhanced Typographic Settings
\emergencystretch 3em
\tolerance 9999
\hbadness 9999
\setlength{\hfuzz}{15pt}

% Header and Footer Configuration
\pagestyle{fancy}
\fancyhf{}
\fancyhead[L]{Johann Pascher}
\fancyhead[R]{T0 Model: Unified Neutrino Formula Structure}
\fancyfoot[C]{\thepage}
\renewcommand{\headrulewidth}{0.4pt}
\renewcommand{\footrulewidth}{0.4pt}

% Custom Commands
\newcommand{\Efield}{E_{\text{Field}}}
\newcommand{\xipar}{\xi}
\newcommand{\Tzero}{T_0}
\newcommand{\vecx}{\vec{x}}

% Hyperlink Setup
\hypersetup{
	colorlinks=true,
	linkcolor=blue,
	citecolor=blue,
	urlcolor=blue,
	pdftitle={T0 Model: Unified Neutrino Formula Structure},
	pdfauthor={Johann Pascher},
	pdfsubject={T0 Model, Neutrino Masses, Speculative Extrapolations},
	pdfkeywords={Neutrinos, Geometric Harmonies, Mathematical Consistency, Speculative Physics, Neutrino Oscillation, Geometric Phases}
}

% Custom environments
\newtcolorbox{important}[1][]{
	colback=yellow!10!white,
	colframe=yellow!50!black,
	fonttitle=\bfseries,
	title=Important Note,
	#1
}

\newtcolorbox{formula}[1][]{
	colback=blue!5!white,
	colframe=blue!75!black,
	fonttitle=\bfseries,
	title=Mathematical Formula,
	#1
}

\newtcolorbox{warning}[1][]{
	colback=red!5!white,
	colframe=red!75!black,
	fonttitle=\bfseries,
	title=Scientific Warning,
	#1
}

\newtcolorbox{experimental}[1][]{
	colback=green!5!white,
	colframe=green!75!black,
	fonttitle=\bfseries,
	title=Experimental Comparison,
	#1
}

\newtcolorbox{speculation}[1][]{
	colback=purple!5!white,
	colframe=purple!75!black,
	fonttitle=\bfseries,
	title=Speculative Hypothesis,
	#1
}

\title{\Huge\textbf{T0 Model: Unified Neutrino Formula Structure}\\
	\Large Mathematically Consistent Extrapolations \\
	with Speculative Physical Basis}

\author{Johann Pascher\\
	Department of Communications Engineering, \\
	Higher Technical Federal Institute (HTL), Leonding, Austria\\
	\texttt{johann.pascher@gmail.com}}

\date{\today}



==================================================

=== parameterherleitung_De.tex.preamble ===

\documentclass[12pt,a4paper]{article}
\usepackage[utf8]{inputenc}
\usepackage[T1]{fontenc}
\usepackage[german]{babel}
\usepackage[left=2cm,right=2cm,top=2cm,bottom=2cm]{geometry}
\usepackage{lmodern}
\usepackage{amsmath}
\usepackage{amssymb}
\usepackage{physics}
\usepackage{hyperref}
\usepackage{tcolorbox}
\usepackage{booktabs}
\usepackage{enumitem}
\usepackage[table,xcdraw]{xcolor}
\usepackage{graphicx}
\usepackage{float}
\usepackage{mathtools}
\usepackage{amsthm}
\usepackage{siunitx}
\usepackage{fancyhdr}
\usepackage{longtable}

% === NEUE PAKETE FÜR KOSMOLOGISCHE TABELLEN ===
\usepackage{array}        % Erweiterte Tabellenfunktionen (p{} Spalten)
\usepackage{multirow}     % Für Zellen über mehrere Zeilen
\usepackage{tikz}         % Für Hierarchie-Diagramme
\usetikzlibrary{positioning, shapes.geometric, arrows.meta}
\usepackage{microtype}    % Bessere Typografie

% === UNICODE-ZEICHEN BEHANDLUNG ===
\usepackage{newunicodechar}
% Griechische Buchstaben
\newunicodechar{Λ}{\ensuremath{\Lambda}}
\newunicodechar{λ}{\ensuremath{\lambda}}
\newunicodechar{ξ}{\ensuremath{\xi}}
\newunicodechar{Ω}{\ensuremath{\Omega}}
\newunicodechar{σ}{\ensuremath{\sigma}}
\newunicodechar{τ}{\ensuremath{\tau}}
\newunicodechar{θ}{\ensuremath{\theta}}
\newunicodechar{α}{\ensuremath{\alpha}}
\newunicodechar{β}{\ensuremath{\beta}}
\newunicodechar{ρ}{\ensuremath{\rho}}
\newunicodechar{δ}{\ensuremath{\delta}}
\newunicodechar{π}{\ensuremath{\pi}}

% === TCOLORBOX BIBLIOTHEKEN ===
\tcbuselibrary{theorems,skins,breakable}  % Für erweiterte Boxen

% === LONGTABLE EINSTELLUNGEN ===
\setlength{\LTpre}{6pt}
\setlength{\LTpost}{6pt}
\setlength{\LTcapwidth}{\textwidth}

% Headers and Footers
\pagestyle{fancy}
\fancyhf{}
\fancyhead[L]{T0 Deterministic Quantum Computing}
\fancyhead[R]{Complete Algorithm Analysis}
\fancyfoot[C]{\thepage}
\renewcommand{\headrulewidth}{0.4pt}
\renewcommand{\footrulewidth}{0.4pt}

% Custom Commands
\newcommand{\Efield}{E}

% === NEUE CUSTOM COMMANDS FÜR KOSMOLOGIE ===
\newcommand{\xipar}{\xi}                    % Xi-Parameter
\newcommand{\LCDM}{\Lambda\text{CDM}}       % Lambda-CDM
\newcommand{\OmegaLambda}{\Omega_{\Lambda}} % Omega Lambda
\newcommand{\OmegaDM}{\Omega_{\text{DM}}}   % Omega Dark Matter
\newcommand{\Omegab}{\Omega_b}              % Omega Baryonen
\newcommand{\natunits}{\text{(nat. Einh.)}} % Natürliche Einheiten
\newcommand{\GeV}{\,\text{GeV}}             % GeV Einheit
\newcommand{\MeV}{\,\text{MeV}}             % MeV Einheit
\newcommand{\eV}{\,\text{eV}}               % eV Einheit

% === THEOREMUMGEBUNGEN (falls benötigt) ===
\theoremstyle{definition}
\newtheorem{definition}{Definition}[section]
\newtheorem{theorem}{Theorem}[section]

\hypersetup{
	colorlinks=true,
	linkcolor=blue,
	citecolor=blue,
	urlcolor=blue,
	pdftitle={T0 Deterministic Quantum Computing: Complete Analysis of Important Algorithms},
	pdfauthor={T0 Quantum Computing Research},
	pdfsubject={T0-Theory, Deterministic Quantum Computing, Algorithm Analysis}
}

\title{T0-Theorie: Vollst\"andige Herleitung aller Parameter ohne Zirkularit\"at}
\author{Johann Pascher\\
	Abteilung f\"ur Nachrichtentechnik\\
	H\"ohere Technische Lehranstalt, Leonding, \"Osterreich\\
	\texttt{johann.pascher@gmail.com}}
\date{\today}




==================================================

=== parameterherleitung_En.tex.preamble ===

\documentclass[12pt,a4paper]{article}
\usepackage[utf8]{inputenc}
\usepackage[T1]{fontenc}
\usepackage[english]{babel}
\usepackage{lmodern}
\usepackage{amsmath}
\usepackage{amssymb}
\usepackage{physics}
\usepackage{hyperref}
\usepackage{geometry}
\usepackage{tcolorbox}
\usepackage{booktabs}
\usepackage{enumitem}
\usepackage[table,xcdraw]{xcolor}
\usepackage{graphicx}
\usepackage{float}
\usepackage{mathtools}
\usepackage{amsthm}
\usepackage{siunitx}
\usepackage{fancyhdr}
\usepackage{longtable}

% Geometry settings
\geometry{left=2.5cm,right=2.5cm,top=2.5cm,bottom=2.5cm}

% Headers and Footers
\pagestyle{fancy}
\fancyhf{}
\fancyhead[L]{T0-Theory}
\fancyhead[R]{Parameter Derivation}
\fancyfoot[C]{\thepage}
\renewcommand{\headrulewidth}{0.4pt}
\renewcommand{\footrulewidth}{0.4pt}

% Custom Commands
\newcommand{\Efield}{E}

\hypersetup{
	colorlinks=true,
	linkcolor=blue,
	citecolor=blue,
	urlcolor=blue,
	pdftitle={T0-Theory: Complete Derivation of All Parameters Without Circularity},
	pdfauthor={Johann Pascher},
	pdfsubject={T0-Theory, Geometric Derivation, Fine Structure Constant}
}

\title{T0-Theory: Complete Derivation of All Parameters Without Circularity}
\author{Johann Pascher\\
	Department of Communication Technology\\
	Higher Technical College, Leonding, Austria\\
	\texttt{johann.pascher@gmail.com}}
\date{\today}



==================================================

=== redshift_deflection_De.tex.preamble ===

\documentclass[12pt,a4paper]{article}
\usepackage[utf8]{inputenc}
\usepackage[T1]{fontenc}
\usepackage[ngerman]{babel}
\usepackage{amsmath,amssymb,amsfonts,amsthm}
\usepackage{physics}
\usepackage{siunitx}
\usepackage{geometry}
\usepackage{fancyhdr}
\usepackage{enumitem}
\usepackage{booktabs}
\usepackage{longtable}
\usepackage{array}
\usepackage{xcolor}
\usepackage{tcolorbox}
\usepackage{mdframed}
\usepackage{graphicx}
\usepackage{hyperref}

\geometry{margin=2.5cm}
\pagestyle{fancy}
\fancyhf{}
\fancyhead[L]{T0-Theorie: Rotverschiebungsmechanismus}
\fancyhead[R]{\thepage}
\fancyfoot[C]{\textit{Wellenl\"angenabh\"angige Rotverschiebung ohne Entfernungsannahmen}}

\hypersetup{
	colorlinks=true,
	linkcolor=blue,
	filecolor=magenta,
	urlcolor=cyan,
}

% Benutzerdefinierte Befehle
\newcommand{\xiconst}{\xi = \frac{4}{3} \times 10^{-4}}
\newcommand{\Exi}{E_\xi}
\newcommand{\xicoupling}{f(E/\Exi)}
\newcommand{\lambdazero}{\lambda_0}
\newcommand{\nuzero}{\nu_0}

% Benutzerdefinierte Umgebungen
\newtcolorbox{important}[1][]{colback=yellow!10!white,colframe=yellow!50!black,fonttitle=\bfseries,title=Schl\"usseleinsicht,#1}
\newtcolorbox{formula}[1][]{colback=blue!5!white,colframe=blue!75!black,fonttitle=\bfseries,title=T0-Vorhersage,#1}
\newtcolorbox{experiment}[1][]{colback=green!5!white,colframe=green!75!black,fonttitle=\bfseries,title=Experimenteller Test,#1}
\newtcolorbox{revolutionary}[1][]{colback=red!5!white,colframe=red!75!black,fonttitle=\bfseries,title=Paradigmenwechsel,#1}

\theoremstyle{definition}
\newtheorem{principle}{Prinzip}

\title{\Huge\textbf{T0-Theorie: Rotverschiebungsmechanismus}\\
	\Large Wellenl\"angenabh\"angige Rotverschiebung \\
	ohne Entfernungsannahmen oder r\"aumliche Expansion}

\author{Basierend auf dem T0-Theorie-Rahmenwerk\\
	Spektroskopische Tests unter Verwendung kosmischer Objektmassen}

\date{\today}



==================================================

=== redshift_deflection_En.tex.preamble ===

\documentclass[12pt,a4paper]{article}
\usepackage[utf8]{inputenc}
\usepackage[T1]{fontenc}
\usepackage[english]{babel}
\usepackage{amsmath,amssymb,amsfonts,amsthm}
\usepackage{physics}
\usepackage{siunitx}
\usepackage{geometry}
\usepackage{fancyhdr}
\usepackage{enumitem}
\usepackage{booktabs}
\usepackage{longtable}
\usepackage{array}
\usepackage{xcolor}
\usepackage{tcolorbox}
\usepackage{mdframed}
\usepackage{graphicx}
\usepackage{hyperref}

\geometry{margin=2.5cm}
\pagestyle{fancy}
\fancyhf{}
\fancyhead[L]{T0-Theory: Redshift Mechanism}
\fancyhead[R]{\thepage}
\fancyfoot[C]{\textit{Wavelength-Dependent Redshift without Distance Assumptions}}

\hypersetup{
	colorlinks=true,
	linkcolor=blue,
	filecolor=magenta,
	urlcolor=cyan,
}

% Custom commands
\newcommand{\xiconst}{\xi = \frac{4}{3} \times 10^{-4}}
\newcommand{\Exi}{E_\xi}
\newcommand{\xicoupling}{f(E/\Exi)}
\newcommand{\lambdazero}{\lambda_0}
\newcommand{\nuzero}{\nu_0}
\newcommand{\betaT}{\beta_T}
\newcommand{\rzero}{r_0}

% Custom environments
\newtcolorbox{important}[1][]{colback=yellow!10!white,colframe=yellow!50!black,fonttitle=\bfseries,title=Key Insight,#1}
\newtcolorbox{formula}[1][]{colback=blue!5!white,colframe=blue!75!black,fonttitle=\bfseries,title=T0 Prediction,#1}
\newtcolorbox{experiment}[1][]{colback=green!5!white,colframe=green!75!black,fonttitle=\bfseries,title=Experimental Test,#1}
\newtcolorbox{revolutionary}[1][]{colback=red!5!white,colframe=red!75!black,fonttitle=\bfseries,title=Paradigm Shift,#1}

\theoremstyle{definition}
\newtheorem{principle}{Principle}
\newtheorem{theorem}{Theorem}
\newtheorem{proposition}{Proposition}

\title{\Huge\textbf{T0-Theory: Redshift Mechanism}\\
	\Large Wavelength-Dependent Redshift \\
	without Distance Assumptions or Spatial Expansion}

\author{Based on the T0-Theory Framework\\
	Spectroscopic Tests Using Cosmic Object Masses}

\date{\today}



==================================================

=== scheinbar_instantan_De.tex.preamble ===

\documentclass[12pt,a4paper]{article}
\usepackage[utf8]{inputenc}
\usepackage[T1]{fontenc}
\usepackage[ngerman]{babel}
\usepackage[left=2cm,right=2cm,top=2cm,bottom=2cm]{geometry}
\usepackage{lmodern}
\usepackage{amsmath}
\usepackage{amssymb}
\usepackage{physics}
\usepackage{booktabs}
\usepackage{tcolorbox}
\usepackage{siunitx}
\usepackage[table,xcdraw]{xcolor}
\usepackage{hyperref}
\usepackage{array}
\usepackage{textgreek}
\usepackage{fancyhdr}
\usepackage{enumitem}
\usepackage{graphicx}
\usepackage{float}
\usepackage{tikz}
\usepackage{braket}

% Farbdefinitionen
\definecolor{t0blue}{RGB}{0,102,204}
\definecolor{t0green}{RGB}{0,153,76}
\definecolor{t0red}{RGB}{204,0,51}
\definecolor{t0yellow}{RGB}{255,204,0}
\definecolor{t0purple}{RGB}{102,0,204}

% Header und Footer
\pagestyle{fancy}
\fancyhf{}
\fancyhead[L]{\color{t0blue}\textsc{T0-Theorie: Auflösung der Instantanität}}
\fancyhead[R]{\color{t0blue}\textsc{Johann Pascher}}
\fancyfoot[C]{\thepage}
\renewcommand{\headrulewidth}{0.4pt}
\renewcommand{\footrulewidth}{0.4pt}

% Custom-Befehle
\newcommand{\xipar}{\ensuremath{\xi}}
\newcommand{\Tfield}{\mathcal{T}}
\newcommand{\lP}{\ell_{\text{P}}}
\newcommand{\mP}{m_{\text{P}}}
\newcommand{\EP}{E_{\text{P}}}
\newcommand{\tP}{t_{\text{P}}}

% Hyperref-Einstellungen
\hypersetup{
	colorlinks=true,
	linkcolor=t0blue,
	citecolor=t0blue,
	urlcolor=t0blue,
	pdftitle={T0-Theorie: Auflösung der scheinbaren Instantanität},
	pdfauthor={Johann Pascher},
	pdfsubject={Theoretische Physik, T0-Theorie, Quantenmechanik},
	pdfkeywords={T0-Modell, Instantanität, Quantenverschränkung, Feldtheorie}
}

\title{{\Huge \color{t0blue}T0-Formalismus}\\
	{\LARGE Vollständige Auflösung der scheinbaren Instantanität}\\
	\vspace{1cm}
	{\Large Eine feldtheoretische Analyse der Kausalität in der Quantenmechanik}}
\author{{\Large Johann Pascher}\\
	Abteilung für Kommunikationstechnik\\
	HTL Leonding, Österreich\\
	\texttt{johann.pascher@gmail.com}}
\date{\today}



==================================================

=== scheinbar_instantan_En.tex.preamble ===

\documentclass[12pt,a4paper]{article}
\usepackage[utf8]{inputenc}
\usepackage[T1]{fontenc}
\usepackage[english]{babel}
\usepackage[left=2cm,right=2cm,top=2cm,bottom=2cm]{geometry}
\usepackage{lmodern}
\usepackage{amsmath}
\usepackage{amssymb}
\usepackage{physics}
\usepackage{booktabs}
\usepackage{tcolorbox}
\usepackage{siunitx}
\usepackage[table,xcdraw]{xcolor}
\usepackage{hyperref}
\usepackage{array}
\usepackage{textgreek}
\usepackage{fancyhdr}
\usepackage{enumitem}
\usepackage{graphicx}
\usepackage{float}
\usepackage{tikz}
\usepackage{braket}

% Color definitions
\definecolor{t0blue}{RGB}{0,102,204}
\definecolor{t0green}{RGB}{0,153,76}
\definecolor{t0red}{RGB}{204,0,51}
\definecolor{t0yellow}{RGB}{255,204,0}
\definecolor{t0purple}{RGB}{102,0,204}

% Header and Footer
\pagestyle{fancy}
\fancyhf{}
\fancyhead[L]{\color{t0blue}\textsc{T0 Theory: Resolution of Instantaneity}}
\fancyhead[R]{\color{t0blue}\textsc{Johann Pascher}}
\fancyfoot[C]{\thepage}
\renewcommand{\headrulewidth}{0.4pt}
\renewcommand{\footrulewidth}{0.4pt}

% Custom commands
\newcommand{\xipar}{\ensuremath{\xi}}
\newcommand{\Tfield}{\mathcal{T}}
\newcommand{\lP}{\ell_{\text{P}}}
\newcommand{\mP}{m_{\text{P}}}
\newcommand{\EP}{E_{\text{P}}}
\newcommand{\tP}{t_{\text{P}}}

% Hyperref settings
\hypersetup{
	colorlinks=true,
	linkcolor=t0blue,
	citecolor=t0blue,
	urlcolor=t0blue,
	pdftitle={T0 Theory: Resolution of Apparent Instantaneity},
	pdfauthor={Johann Pascher},
	pdfsubject={Theoretical Physics, T0 Theory, Quantum Mechanics},
	pdfkeywords={T0 Model, Instantaneity, Quantum Entanglement, Field Theory}
}

\title{{\Huge \color{t0blue}T0 Formalism}\\
	{\LARGE Complete Resolution of Apparent Instantaneity}\\
	\vspace{1cm}
	{\Large A Field-Theoretic Analysis of Causality in Quantum Mechanics}}
\author{{\Large Johann Pascher}\\
	Department of Communication Technology\\
	HTL Leonding, Austria\\
	\texttt{johann.pascher@gmail.com}}
\date{\today}



==================================================

=== systemDe.tex.preamble ===

\documentclass[12pt,a4paper]{article}
\usepackage[utf8]{inputenc}
\usepackage[T1]{fontenc}
\usepackage[ngerman]{babel}
\usepackage[left=2cm,right=2cm,top=2cm,bottom=2cm]{geometry}
\usepackage{lmodern}
\usepackage{amsmath}
\usepackage{amssymb}
\usepackage{physics}
\usepackage{hyperref}
\usepackage{tcolorbox}
\usepackage{booktabs}
\usepackage{enumitem}
\usepackage[table,xcdraw]{xcolor}
\usepackage{graphicx}
\usepackage{float}
\usepackage{mathtools}
\usepackage{amsthm}
\usepackage{siunitx}
\usepackage{fancyhdr}
\usepackage{longtable}
\usepackage{array}
\usepackage{multirow}
\usepackage{url}
\usepackage{textcomp}

% Prevent page overflow warnings
\setlength{\extrarowheight}{2pt}
\setlength{\headheight}{14.49998pt}

% Headers and Footers
\pagestyle{fancy}
\fancyhf{}
\fancyhead[L]{Johann Pascher}
\fancyhead[R]{Vollständiges Teilchenspektrum: Standard-Modell vs T0}
\fancyfoot[C]{\thepage}
\renewcommand{\headrulewidth}{0.4pt}
\renewcommand{\footrulewidth}{0.4pt}

% Custom Commands
\newcommand{\deltam}{\delta m}
\newcommand{\xipar}{\xi}
\newcommand{\Lag}{\mathcal{L}}

\hypersetup{
	colorlinks=true,
	linkcolor=blue,
	citecolor=blue,
	urlcolor=blue,
	pdftitle={Vollständiges Teilchenspektrum: Standard-Modell vs T0-Theorie},
	pdfauthor={Johann Pascher},
	pdfsubject={T0-Theorie, Vollständiges Teilchenspektrum, Feldvereinheitlichung}
}

\newtheorem{theorem}{Theorem}[section]
\newtheorem{proposition}[theorem]{Proposition}
\newtheorem{definition}[theorem]{Definition}



==================================================

=== systemEn.tex.preamble ===

\documentclass[12pt,a4paper]{article}
\usepackage[utf8]{inputenc}
\usepackage[T1]{fontenc}
\usepackage[english]{babel}
\usepackage[left=2cm,right=2cm,top=2cm,bottom=2cm]{geometry}
\usepackage{lmodern}
\usepackage{amsmath}
\usepackage{amssymb}
\usepackage{physics}
\usepackage{hyperref}
\usepackage{tcolorbox}
\usepackage{booktabs}
\usepackage{enumitem}
\usepackage[table,xcdraw]{xcolor}
\usepackage{graphicx}
\usepackage{float}
\usepackage{mathtools}
\usepackage{amsthm}
\usepackage{siunitx}
\usepackage{fancyhdr}
\usepackage{longtable}
\usepackage{array}
\usepackage{multirow}  % FIXED: Added missing package
\usepackage{url}       % For URL formatting in bibliography
\usepackage{textcomp}  % For additional text symbols

% FIXED: Prevent page overflow warnings
\setlength{\extrarowheight}{2pt}
\setlength{\headheight}{14.49998pt} % FIXED: Set proper head height

% Headers and Footers
\pagestyle{fancy}
\fancyhf{}
\fancyhead[L]{Johann Pascher}
\fancyhead[R]{Complete Particle Spectrum: Standard Model vs T0}
\fancyfoot[C]{\thepage}
\renewcommand{\headrulewidth}{0.4pt}
\renewcommand{\footrulewidth}{0.4pt}

% Custom Commands
\newcommand{\deltam}{\delta m}
\newcommand{\xipar}{\xi}
\newcommand{\Lag}{\mathcal{L}}

\hypersetup{
	colorlinks=true,
	linkcolor=blue,
	citecolor=blue,
	urlcolor=blue,
	pdftitle={Complete Particle Spectrum: Standard Model vs T0 Theory},
	pdfauthor={Johann Pascher},
	pdfsubject={T0 Theory, Complete Particle Spectrum, Field Unification}
}

\newtheorem{theorem}{Theorem}[section]
\newtheorem{proposition}[theorem]{Proposition}
\newtheorem{definition}[theorem]{Definition}



==================================================

=== universale-ableitung_De.tex.preamble ===

\documentclass[12pt,a4paper]{article}
\usepackage[utf8]{inputenc}
\usepackage[T1]{fontenc}
\usepackage[ngerman]{babel}
\usepackage{amsmath}
\usepackage{amsfonts}
\usepackage{amssymb}
\usepackage{amsthm}
\usepackage{siunitx}
\usepackage{booktabs}
\usepackage{xcolor}
\usepackage{tcolorbox}
\usepackage{geometry}
\usepackage{fancyhdr}
\usepackage{setspace}
\usepackage{hyperref}

\geometry{margin=2.5cm}

% Farben definieren
\definecolor{deepblue}{RGB}{0,0,127}
\definecolor{deepred}{RGB}{191,0,0}
\definecolor{deepgreen}{RGB}{0,127,0}

% Hyperref Setup
\hypersetup{
	colorlinks=true,
	linkcolor=blue,
	citecolor=blue,
	urlcolor=blue,
}

% Header Definition nach Pascher-Stil
\pagestyle{fancy}
\fancyhf{}
\fancyhead[L]{\textbf{T0-Theorie: Universelle Ableitung}}
\fancyhead[R]{\textbf{Johann Pascher, 2025}}
\fancyfoot[C]{\thepage}
\renewcommand{\headrulewidth}{0.4pt}
\setlength{\headheight}{15pt}

% Theoreme und Definitionen
\theoremstyle{definition}
\newtheorem{definition}{Definition}[section]
\newtheorem{theorem}{Theorem}[section]
\newtheorem{lemma}{Lemma}[section]
\newtheorem{corollary}{Korollar}[section]

% Abstände
\setstretch{1.2}

\title{\textbf{Universelle Ableitung aller physikalischen Konstanten\\aus der Feinstrukturkonstante und Planck-L\"ange}\\[0.5cm]
	\large Mit Klarstellung der charakteristischen Energie E\_0\\und Entkr\"aftung der Zirkularit\"ats-Einw\"ande\\[0.3cm]
	\normalsize T0-Modell: Systematische Herleitung in SI-Einheiten}
\author{Johann Pascher\\
	\small Abteilung Kommunikationstechnik,\\
	\small H\"ohere Technische Lehranstalt (HTL), Leonding, \"Osterreich\\
	\small \texttt{johann.pascher@gmail.com}}
\date{September 2025}



==================================================

=== universale-ableitung_En.tex.preamble ===

\documentclass[12pt,a4paper]{article}
\usepackage[utf8]{inputenc}
\usepackage[T1]{fontenc}
\usepackage[english]{babel}
\usepackage{amsmath}
\usepackage{amsfonts}
\usepackage{amssymb}
\usepackage{amsthm}
\usepackage{siunitx}
\usepackage{booktabs}
\usepackage{xcolor}
\usepackage{tcolorbox}
\usepackage{geometry}
\usepackage{fancyhdr}
\usepackage{setspace}
\usepackage{hyperref}

\geometry{margin=2.5cm}

% Color definitions
\definecolor{deepblue}{RGB}{0,0,127}
\definecolor{deepred}{RGB}{191,0,0}
\definecolor{deepgreen}{RGB}{0,127,0}

% Hyperref Setup
\hypersetup{
	colorlinks=true,
	linkcolor=blue,
	citecolor=blue,
	urlcolor=blue,
}

% Header Definition in Pascher style
\pagestyle{fancy}
\fancyhf{}
\fancyhead[L]{\textbf{T0-Theory: Universal Derivation}}
\fancyhead[R]{\textbf{Johann Pascher, 2025}}
\fancyfoot[C]{\thepage}
\renewcommand{\headrulewidth}{0.4pt}
\setlength{\headheight}{15pt}

% Theorems and Definitions
\theoremstyle{definition}
\newtheorem{definition}{Definition}[section]
\newtheorem{theorem}{Theorem}[section]
\newtheorem{lemma}{Lemma}[section]
\newtheorem{corollary}{Corollary}[section]

% Spacing
\setstretch{1.2}

\title{\textbf{Universal Derivation of All Physical Constants\\from the Fine-Structure Constant and Planck Length}\\[0.5cm]
	\large With Clarification of the Characteristic Energy E\_0\\and Refutation of Circularity Objections\\[0.3cm]
	\normalsize T0-Model: Systematic Derivation in SI Units}
\author{Johann Pascher\\
	\small Department of Communication Technology,\\
	\small Higher Technical Institute (HTL), Leonding, Austria\\
	\small \texttt{johann.pascher@gmail.com}}
\date{September 2025}



==================================================

=== xi_parmater_partikel_De.tex.preamble ===

\documentclass[12pt,a4paper]{article}
\usepackage[utf8]{inputenc}
\usepackage[T1]{fontenc}
\usepackage[ngerman]{babel}
\usepackage[left=2.5cm,right=2.5cm,top=2.5cm,bottom=2.5cm]{geometry}
\usepackage{lmodern}
\usepackage{amsmath}
\usepackage{amssymb}
\usepackage{physics}
\usepackage{hyperref}
\usepackage{tcolorbox}
\usepackage{booktabs}
\usepackage{enumitem}
\usepackage[table,xcdraw]{xcolor}
\usepackage{graphicx}
\usepackage{float}
\usepackage{mathtools}
\usepackage{amsthm}
\usepackage{siunitx}
\usepackage{fancyhdr}
\usepackage{longtable}
\usepackage{multirow}
\usepackage{array}
\usepackage{textgreek}

% Headers and Footers
\pagestyle{fancy}
\fancyhf{}
\fancyhead[L]{Der $\xi$ Parameter und Teilchendifferenzierung in der T0-Theorie}
\fancyhead[R]{Mathematische Analyse}
\fancyfoot[C]{\thepage}
\renewcommand{\headrulewidth}{0.4pt}
\renewcommand{\footrulewidth}{0.4pt}

% Define common mathematical symbols for consistent usage
\newcommand{\xipar}{\ensuremath{\xi}}
\newcommand{\deltafield}{\ensuremath{\delta m}}
\newcommand{\partialop}{\ensuremath{\partial}}
\newcommand{\lambdah}{\ensuremath{\lambda_h}}
\newcommand{\betaT}{\ensuremath{\beta_T}}
\newcommand{\alphaEM}{\ensuremath{\alpha_{\text{EM}}}}
\newcommand{\rhofield}{\ensuremath{\rho}}
\newcommand{\mypi}{\ensuremath{\pi}}
\newcommand{\myphi}{\ensuremath{\phi}}
\newcommand{\myomega}{\ensuremath{\omega}}
\newcommand{\mytimes}{\ensuremath{\times}}
\newcommand{\myapprox}{\ensuremath{\approx}}
\newcommand{\myrightarrow}{\ensuremath{\rightarrow}}
\newcommand{\myRightarrow}{\ensuremath{\Rightarrow}}
\newcommand{\mypropto}{\ensuremath{\propto}}
\newcommand{\mysim}{\ensuremath{\sim}}

\hypersetup{
	colorlinks=true,
	linkcolor=blue,
	citecolor=blue,
	urlcolor=blue,
	pdftitle={Der xi Parameter und Teilchendifferenzierung in der T0-Theorie},
	pdfauthor={Johann Pascher},
	pdfsubject={T0-Theorie, xi Parameter, Teilchenphysik, Mathematische Analyse}
}

\title{Der $\xi$ Parameter und Teilchendifferenzierung in der T0-Theorie: \\
	Mathematische Analyse, Geometrische Interpretation und Universelle Feldmuster \\
	\large Eine umfassende Untersuchung der geometrischen Grundlagen und Vereinheitlichung}
\author{Johann Pascher \\
	T0-Theorie Analyse-Framework}
\date{7. Juni 2025}



==================================================

=== xi_parmater_partikel_En.tex.preamble ===

\documentclass[12pt,a4paper]{article}
\usepackage[utf8]{inputenc}
\usepackage[T1]{fontenc}
\usepackage[english]{babel}
\usepackage[left=2cm,right=2cm,top=2cm,bottom=2cm]{geometry}
\usepackage{lmodern}
\usepackage{amsmath}
\usepackage{amssymb}
\usepackage{physics}
\usepackage{booktabs}
\usepackage{tcolorbox}
\usepackage{siunitx}
\usepackage[table,xcdraw]{xcolor}
\usepackage{hyperref}
\usepackage{array}
\usepackage{textgreek}
\usepackage{fancyhdr}
\pagestyle{fancy}
\fancyhf{}
\fancyhead[L]{The $\xi$ Parameter and Particle Differentiation in T0 Theory}
\fancyhead[R]{Mathematical Analysis}
\fancyfoot[C]{\thepage}

% Define common mathematical symbols for consistent usage
\newcommand{\xipar}{\ensuremath{\xi}}
\newcommand{\deltafield}{\ensuremath{\delta m}}
\newcommand{\partialop}{\ensuremath{\partial}}
\newcommand{\lambdah}{\ensuremath{\lambda_h}}
\newcommand{\betaT}{\ensuremath{\beta_T}}
\newcommand{\alphaEM}{\ensuremath{\alpha_{\text{EM}}}}
\newcommand{\rhofield}{\ensuremath{\rho}}
\newcommand{\mypi}{\ensuremath{\pi}}
\newcommand{\myphi}{\ensuremath{\phi}}
\newcommand{\myomega}{\ensuremath{\omega}}
\newcommand{\mytimes}{\ensuremath{\times}}
\newcommand{\myapprox}{\ensuremath{\approx}}
\newcommand{\myrightarrow}{\ensuremath{\rightarrow}}
\newcommand{\myRightarrow}{\ensuremath{\Rightarrow}}
\newcommand{\mypropto}{\ensuremath{\propto}}
\newcommand{\mysim}{\ensuremath{\sim}}
\newcommand{\mysqrt}{\ensuremath{\sqrt}}
\hypersetup{
	colorlinks=true,
	linkcolor=blue,
	citecolor=blue,
	urlcolor=blue,
	pdftitle={T0-Theory: Complete Derivation of All Parameters Without Circularity},
	pdfauthor={Johann Pascher},
	pdfsubject={T0-Theory, Geometric Derivation, Fine Structure Constant}
}
\title{The $\xi$ Parameter and Particle Differentiation in T0 Theory:\\
	\large Mathematical Analysis, Geometric Interpretation, and Universal Field Patterns}
\author{Johann Pascher}
\date{\today}



==================================================

