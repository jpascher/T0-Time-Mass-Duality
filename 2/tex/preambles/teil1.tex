=== 137_De.tex.preamble ===

\documentclass[12pt,a4paper]{article}
\usepackage[utf8]{inputenc}
\usepackage[T1]{fontenc}
\usepackage[ngerman]{babel}
\usepackage{lmodern}
\usepackage{amsmath,amssymb,amsthm}
\usepackage{physics}
\usepackage{graphicx}
\usepackage{xcolor}
\usepackage{tcolorbox}
\usepackage{hyperref}
\usepackage[left=2.5cm,right=2.5cm,top=2.5cm,bottom=2.5cm]{geometry}
\usepackage{booktabs}
\usepackage{siunitx}
\usepackage{tikz}
\usepackage{fancyhdr}
\usetikzlibrary{arrows.meta,positioning,shapes.geometric}

% Farben definieren
\definecolor{t0blue}{RGB}{0,102,204}
\definecolor{t0red}{RGB}{204,0,0}
\definecolor{t0green}{RGB}{0,153,0}
\definecolor{boxgray}{RGB}{240,240,240}

% Theorem-Umgebungen definieren
\theoremstyle{definition}
\newtheorem{erkenntnis}{Erkenntnis}[section]
\newtheorem{entdeckung}{Entdeckung}[section]

% Benutzerdefinierte Boxen
\newtcolorbox{fundamental}[1][]{
	colback=boxgray,
	colframe=t0blue,
	fonttitle=\bfseries,
	title=#1,
	sharp corners,
	boxrule=2pt
}

\newtcolorbox{neueperspektive}[1][]{
	colback=red!5!white,
	colframe=t0red,
	fonttitle=\bfseries,
	title=#1,
	sharp corners,
	boxrule=2pt
}

% Kopf- und Fußzeilen
\pagestyle{fancy}
\fancyhf{}
\fancyhead[L]{Johann Pascher}
\fancyhead[R]{Das verborgene Geheimnis von 1/137}
\fancyfoot[C]{\thepage}
\renewcommand{\headrulewidth}{0.4pt}
\renewcommand{\footrulewidth}{0.4pt}

% Dokument-Metadaten
\hypersetup{
	colorlinks=true,
	linkcolor=t0blue,
	citecolor=t0green,
	urlcolor=t0blue,
	pdftitle={Das verborgene Geheimnis von 1/137},
	pdfauthor={Johann Pascher}
}

\title{
	\textbf{Das verborgene Geheimnis von 1/137}\\
	\vspace{0.5cm}
	\Large Die neue Umkehrung der Perspektive in der Fundamentalphysik
}

\author{Johann Pascher\\
	Fachbereich Kommunikationstechnik\\
	Höhere Technische Bundeslehranstalt (HTL), Leonding, Österreich\\
	\texttt{johann.pascher@gmail.com}}
\date{\today}



==================================================

=== 137_En.tex.preamble ===

\documentclass[12pt,a4paper]{article}
\usepackage[utf8]{inputenc}
\usepackage[T1]{fontenc}
\usepackage[english]{babel}
\usepackage{lmodern}
\usepackage{amsmath,amssymb,amsthm}
\usepackage{physics}
\usepackage{graphicx}
\usepackage{xcolor}
\usepackage{tcolorbox}
\usepackage{hyperref}
\usepackage[left=2.5cm,right=2.5cm,top=2.5cm,bottom=2.5cm]{geometry}
\usepackage{booktabs}
\usepackage{siunitx}
\usepackage{tikz}
\usepackage{fancyhdr}
\usetikzlibrary{arrows.meta,positioning,shapes.geometric}

% Define colors
\definecolor{t0blue}{RGB}{0,102,204}
\definecolor{t0red}{RGB}{204,0,0}
\definecolor{t0green}{RGB}{0,153,0}
\definecolor{boxgray}{RGB}{240,240,240}

% Theorem environments
\theoremstyle{definition}
\newtheorem{insight}{Insight}[section]
\newtheorem{discovery}{Discovery}[section]

% Custom boxes
\newtcolorbox{fundamental}[1][]{
	colback=boxgray,
	colframe=t0blue,
	fonttitle=\bfseries,
	title=#1,
	sharp corners,
	boxrule=2pt
}

\newtcolorbox{newperspective}[1][]{
	colback=red!5!white,
	colframe=t0red,
	fonttitle=\bfseries,
	title=#1,
	sharp corners,
	boxrule=2pt
}

% Headers and footers
\pagestyle{fancy}
\fancyhf{}
\fancyhead[L]{Johann Pascher}
\fancyhead[R]{The Hidden Secret of 1/137}
\fancyfoot[C]{\thepage}
\renewcommand{\headrulewidth}{0.4pt}
\renewcommand{\footrulewidth}{0.4pt}

% Document metadata
\hypersetup{
	colorlinks=true,
	linkcolor=t0blue,
	citecolor=t0green,
	urlcolor=t0blue,
	pdftitle={The Hidden Secret of 1/137},
	pdfauthor={Johann Pascher}
}

\title{
	\textbf{The Hidden Secret of 1/137}\\
	\vspace{0.5cm}
	\Large The New Reversal of Perspective in Fundamental Physics
}

\author{Johann Pascher\\
	Department of Communication Engineering\\
	Higher Technical Federal College (HTL), Leonding, Austria\\
	\texttt{johann.pascher@gmail.com}}
\date{\today}



==================================================

=== Amper_Low_De.tex.preamble ===

\documentclass[10pt, a4paper]{article}
\usepackage[utf8]{inputenc}
\usepackage[german]{babel}
\usepackage[T1]{fontenc}
\usepackage{amsmath}
\usepackage{amsfonts}
\usepackage{amssymb}
\usepackage{physics}
\usepackage{graphicx}
\usepackage{float}
\usepackage{tikz}
\usetikzlibrary{decorations.pathmorphing, patterns, shapes.arrows}
\usepackage{hyperref}
\usepackage{cleveref}
\usepackage{geometry}
\geometry{a4paper, left=25mm, right=25mm, top=25mm, bottom=25mm}

\title{Das T0-Modell: Eine kausale Theorie der konjugierten Basisgrößen mit Anwendungen auf die Ampère-Kraft, longitudinale Moden und geometrieabhängige Skalierung}
\author{Johann Pascher}
\date{\today}



==================================================

=== Amper_Low_En.tex.preamble ===

\documentclass[10pt, a4paper]{article}
\usepackage[utf8]{inputenc}
\usepackage[english]{babel}
\usepackage[T1]{fontenc}
\usepackage{amsmath}
\usepackage{amsfonts}
\usepackage{amssymb}
\usepackage{physics}
\usepackage{graphicx}
\usepackage{float}
\usepackage{tikz}
\usetikzlibrary{decorations.pathmorphing, patterns, shapes.arrows}
\usepackage{hyperref}
\usepackage{cleveref}
\usepackage{geometry}
\geometry{a4paper, left=25mm, right=25mm, top=25mm, bottom=25mm}

\title{The T0 Model: A Causal Theory of Conjugate Base Quantities with Applications to the Ampère Force, Longitudinal Modes, and Geometry-Dependent Scaling}
\author{Johann Pascher}
\date{\today}



==================================================

=== Bell_De.tex.preamble ===

\documentclass[12pt,a4paper]{article}
\usepackage[utf8]{inputenc}
\usepackage[T1]{fontenc}
\usepackage[ngerman]{babel}
\usepackage[left=2.5cm,right=2.5cm,top=2.5cm,bottom=2.5cm]{geometry}
\usepackage{lmodern}
\usepackage{amsmath}
\usepackage{amssymb}
\usepackage{hyperref}
\usepackage{booktabs}
\usepackage{enumitem}
\usepackage[table,xcdraw]{xcolor}
\usepackage{newunicodechar}
\usepackage{fancyhdr}
\usepackage{siunitx}
\usepackage{physics}
\usepackage{tcolorbox}
\usepackage{graphicx}
\usepackage{float}
\usepackage{mathtools}
\usepackage{amsthm}
\usepackage{microtype}
\usepackage{array}

% Unicode setups for Greek letters and symbols
\newunicodechar{ξ}{\ensuremath{\xi}}
\newunicodechar{μ}{\ensuremath{\mu}}
\newunicodechar{ψ}{\ensuremath{\psi}}
\newunicodechar{∝}{\ensuremath{\propto}}
\newunicodechar{ħ}{\ensuremath{\hbar}}
\newunicodechar{φ}{\ensuremath{\phi}}
\newunicodechar{≈}{\ensuremath{\approx}}
\newunicodechar{π}{\ensuremath{\pi}}
\newunicodechar{λ}{\ensuremath{\lambda}}
\newunicodechar{∫}{\ensuremath{\int}}
\newunicodechar{Δ}{\ensuremath{\Delta}}

\geometry{left=2.5cm,right=2.5cm,top=2.5cm,bottom=2.5cm}

\hypersetup{
	colorlinks=true,
	linkcolor=blue,
	citecolor=blue,
	urlcolor=blue,
	pdftitle={T0-Theorie: Erweiterung auf Bell-Tests – ML-Simulationen (November 2025)},
	pdfauthor={Johann Pascher},
	pdfsubject={Theoretische Physik, T0-Theorie, Bell-Tests, Quantenverschränkung}
}

% Header and Footer Configuration
\pagestyle{fancy}
\fancyhf{}
\fancyhead[L]{Johann Pascher}
\fancyhead[R]{T0-Theorie: Erweiterung auf Bell-Tests}
\fancyfoot[C]{\thepage}
\renewcommand{\headrulewidth}{0.4pt}
\renewcommand{\footrulewidth}{0.4pt}

% Tcolorbox Styles
\tcbuselibrary{theorems}
\newtcolorbox{units}{colback=blue!5!white,colframe=blue!75!black,fonttitle=\bfseries}
\newtcolorbox{important}{colback=green!5!white,colframe=green!35!black,fonttitle=\bfseries}
\newtcolorbox{summary}{colback=yellow!5!white,colframe=orange!75!black,fonttitle=\bfseries}
\newtcolorbox{keyresult}{colback=blue!5,colframe=blue!75!black,fonttitle=\bfseries}
\newtcolorbox{warning}{colback=red!5,colframe=red!75!black,fonttitle=\bfseries}

\title{\textbf{T0-Theorie: Erweiterung auf Bell-Tests}\\[0.5cm]
	\large ML-Simulationen und neue Erkenntnisse zur Verschränkung\\[0.3cm]
	\normalsize Erweiterung der T0-Serie: Lokale Realität durch $\xi$-Modifikationen}
\author{Johann Pascher\\
	Abteilung für Kommunikationstechnologie\\
	Höhere Technische Lehranstalt (HTL), Leonding, Österreich\\
	\texttt{johann.pascher@gmail.com}}
\date{\today}



==================================================

=== Bell_En.tex.preamble ===

\documentclass[12pt,a4paper]{article}
\usepackage[utf8]{inputenc}
\usepackage[T1]{fontenc}
\usepackage[english]{babel}
\usepackage[left=2.5cm,right=2.5cm,top=2.5cm,bottom=2.5cm]{geometry}
\usepackage{lmodern}
\usepackage{amsmath}
\usepackage{amssymb}
\usepackage{hyperref}
\usepackage{booktabs}
\usepackage{enumitem}
\usepackage[table,xcdraw]{xcolor}
\usepackage{newunicodechar}
\usepackage{fancyhdr}
\usepackage{siunitx}
\usepackage{physics}
\usepackage{tcolorbox}
\usepackage{graphicx}
\usepackage{float}
\usepackage{mathtools}
\usepackage{amsthm}
\usepackage{microtype}
\usepackage{array}

% Unicode setups for Greek letters and symbols
\newunicodechar{ξ}{\ensuremath{\xi}}
\newunicodechar{μ}{\ensuremath{\mu}}
\newunicodechar{ψ}{\ensuremath{\psi}}
\newunicodechar{∝}{\ensuremath{\propto}}
\newunicodechar{ħ}{\ensuremath{\hbar}}
\newunicodechar{φ}{\ensuremath{\phi}}
\newunicodechar{≈}{\ensuremath{\approx}}
\newunicodechar{π}{\ensuremath{\pi}}
\newunicodechar{λ}{\ensuremath{\lambda}}
\newunicodechar{∫}{\ensuremath{\int}}
\newunicodechar{Δ}{\ensuremath{\Delta}}

\geometry{left=2.5cm,right=2.5cm,top=2.5cm,bottom=2.5cm}

\hypersetup{
	colorlinks=true,
	linkcolor=blue,
	citecolor=blue,
	urlcolor=blue,
	pdftitle={T0-Theory: Extension to Bell Tests – ML Simulations (November 2025)},
	pdfauthor={Johann Pascher},
	pdfsubject={Theoretical Physics, T0 Theory, Bell Tests, Quantum Entanglement}
}

% Header and Footer Configuration
\pagestyle{fancy}
\fancyhf{}
\fancyhead[L]{Johann Pascher}
\fancyhead[R]{T0 Theory: Extension to Bell Tests}
\fancyfoot[C]{\thepage}
\renewcommand{\headrulewidth}{0.4pt}
\renewcommand{\footrulewidth}{0.4pt}

% Tcolorbox Styles
\tcbuselibrary{theorems}
\newtcolorbox{units}{colback=blue!5!white,colframe=blue!75!black,fonttitle=\bfseries}
\newtcolorbox{important}{colback=green!5!white,colframe=green!35!black,fonttitle=\bfseries}
\newtcolorbox{summary}{colback=yellow!5!white,colframe=orange!75!black,fonttitle=\bfseries}
\newtcolorbox{keyresult}{colback=blue!5,colframe=blue!75!black,fonttitle=\bfseries}
\newtcolorbox{warning}{colback=red!5,colframe=red!75!black,fonttitle=\bfseries}

\title{\textbf{T0 Theory: Extension to Bell Tests}\\[0.5cm]
	\large ML Simulations and New Insights on Entanglement\\[0.3cm]
	\normalsize Extension of the T0 Series: Local Reality through $\xi$-Modifications}
\author{Johann Pascher\\
	Department of Communication Technology\\
	Higher Technical Institute (HTL), Leonding, Austria\\
	\texttt{johann.pascher@gmail.com}}
\date{\today}



==================================================

=== Bewegungsenergie_De.tex.preamble ===

\documentclass[12pt,a4paper]{article}
\usepackage[utf8]{inputenc}
\usepackage[T1]{fontenc}
\usepackage[german]{babel}
\usepackage{lmodern}
\usepackage{amsmath}
\usepackage{amssymb}
\usepackage{physics}
\usepackage{hyperref}
\usepackage{tcolorbox}
\usepackage{booktabs}
\usepackage{enumitem}
\usepackage[table,xcdraw]{xcolor}
\usepackage[left=2cm,right=2cm,top=2cm,bottom=2cm]{geometry}
\usepackage{siunitx}
\usepackage{mathtools}
\usepackage{amsthm}
\usepackage{cleveref}
\usepackage{tocloft}
\usepackage{microtype}
\usepackage{fancyhdr}

% Custom Commands
\newcommand{\Efield}{E_{\text{Feld}}}
\newcommand{\xigeom}{\xi}
\newcommand{\Tzero}{T_0}
\newcommand{\vecx}{\vec{x}}
\newcommand{\xipar}{\xi}

% Header and Footer Configuration
\pagestyle{fancy}
\fancyhf{}
\fancyhead[L]{Johann Pascher}
\fancyhead[R]{T0-Modell: Bewegungsenergie von Elektronen und Photonen}
\fancyfoot[C]{\thepage}
\renewcommand{\headrulewidth}{0.4pt}
\renewcommand{\footrulewidth}{0.4pt}

% Table of Contents Formatting
\renewcommand{\cftsecfont}{\color{blue}}
\renewcommand{\cftsubsecfont}{\color{blue}}
\renewcommand{\cftsecpagefont}{\color{blue}}
\renewcommand{\cftsubsecpagefont}{\color{blue}}

\hypersetup{
	colorlinks=true,
	linkcolor=blue,
	citecolor=blue,
	urlcolor=blue,
	pdftitle={T0-Modell: Bewegungsenergie von Elektronen und Photonen},
	pdfauthor={Johann Pascher},
	pdfsubject={Zeit-Energie-Dualität, Bewegungsenergie, Elektronen, Photonen},
	pdfkeywords={T0-Modell, Bewegungsenergie, Zeit-Energie-Dualität, Elektronen, Photonen}
}

% Theorem Environments
\newtheorem{theorem}{Theorem}[section]
\newtheorem{proposition}[theorem]{Proposition}
\newtheorem{definition}[theorem]{Definition}
\newtheorem{lemma}[theorem]{Lemma}

\tcbuselibrary{theorems}
\newtcbtheorem[number within=section]{important}{Wichtige Erkenntnis}%
{colback=green!5,colframe=green!35!black,fonttitle=\bfseries}{th}



==================================================

=== Bewegungsenergie_En.tex.preamble ===

\documentclass[12pt,a4paper]{article}
\usepackage[utf8]{inputenc}
\usepackage[T1]{fontenc}
\usepackage[english]{babel}
\usepackage{lmodern}
\usepackage{amsmath}
\usepackage{amssymb}
\usepackage{physics}
\usepackage{hyperref}
\usepackage{tcolorbox}
\usepackage{booktabs}
\usepackage{enumitem}
\usepackage[table,xcdraw]{xcolor}
\usepackage[left=2cm,right=2cm,top=2cm,bottom=2cm]{geometry}
\usepackage{siunitx}
\usepackage{mathtools}
\usepackage{amsthm}
\usepackage{cleveref}
\usepackage{tocloft}
\usepackage{microtype}
\usepackage{fancyhdr}

% Custom Commands
\newcommand{\Efield}{E_{\text{Field}}}
\newcommand{\xigeom}{\xi}
\newcommand{\Tzero}{T_0}
\newcommand{\vecx}{\vec{x}}
\newcommand{\xipar}{\xi}

% Header and Footer Configuration
\pagestyle{fancy}
\fancyhf{}
\fancyhead[L]{Johann Pascher}
\fancyhead[R]{T0-Model: Kinetic Energy of Electrons and Photons}
\fancyfoot[C]{\thepage}
\renewcommand{\headrulewidth}{0.4pt}
\renewcommand{\footrulewidth}{0.4pt}

% Table of Contents Formatting
\renewcommand{\cftsecfont}{\color{blue}}
\renewcommand{\cftsubsecfont}{\color{blue}}
\renewcommand{\cftsecpagefont}{\color{blue}}
\renewcommand{\cftsubsecpagefont}{\color{blue}}

\hypersetup{
	colorlinks=true,
	linkcolor=blue,
	citecolor=blue,
	urlcolor=blue,
	pdftitle={T0-Model: Kinetic Energy of Electrons and Photons},
	pdfauthor={Johann Pascher},
	pdfsubject={Time-Energy Duality, Kinetic Energy, Electrons, Photons},
	pdfkeywords={T0-Model, Kinetic Energy, Time-Energy Duality, Electrons, Photons}
}

% Theorem Environments
\newtheorem{theorem}{Theorem}[section]
\newtheorem{proposition}[theorem]{Proposition}
\newtheorem{definition}[theorem]{Definition}
\newtheorem{lemma}[theorem]{Lemma}

\tcbuselibrary{theorems}
\newtcbtheorem[number within=section]{important}{Key Insight}%
{colback=green!5,colframe=green!35!black,fonttitle=\bfseries}{th}



==================================================

=== Casimir_De.tex.preamble ===

\documentclass[12pt,a4paper]{article}
\usepackage[utf8]{inputenc}
\usepackage[T1]{fontenc}
\usepackage[ngerman]{babel}
%\usepackage{amsmath}
\usepackage{amsfonts}
%\usepackage{amssymb}
\usepackage{booktabs}
%\usepackage{siunitx}
%\usepackage{geometry}
\usepackage{float}

\usepackage{amsmath,amssymb}
\usepackage{graphicx}
\usepackage{caption}
\usepackage{hyperref}
\usepackage{geometry}
\usepackage{amssymb}
\usepackage{booktabs}
\usepackage{siunitx}
\usepackage{graphicx}
\usepackage{caption}
\usepackage{hyperref}
\usepackage{geometry}
\usepackage{float}
\usepackage{longtable}
\usepackage{array}

\geometry{margin=2.5cm}
\sisetup{locale = DE}



==================================================

=== DerivationVonBetaDe.tex.preamble ===

\documentclass[12pt,a4paper]{article}
\usepackage[utf8]{inputenc}
\usepackage[T1]{fontenc}
\usepackage[ngerman]{babel}
\usepackage{lmodern}
\usepackage{amsmath}
\usepackage{amssymb}
\usepackage{physics}
\usepackage{hyperref}
\usepackage{tcolorbox}
\usepackage{booktabs}
\usepackage{enumitem}
\usepackage[table,xcdraw]{xcolor}
\usepackage[left=2cm,right=2cm,top=2cm,bottom=2cm]{geometry}
\usepackage{pgfplots}
\pgfplotsset{compat=1.18}
\usepackage{graphicx}
\usepackage{float}
\usepackage{fancyhdr}
\usepackage{siunitx}
\usepackage{mathtools}
\usepackage{amsthm}
\usepackage{cleveref}
\usepackage{tocloft}
\usepackage{tikz}
\usepackage[dvipsnames]{xcolor}
\usetikzlibrary{positioning, shapes.geometric, arrows.meta}
\usepackage{microtype}
\usepackage{natbib}
\usepackage{doi}

% Erweiterte Cross-Referencing-Konfiguration
\crefname{equation}{Gl.}{Gln.}
\crefname{section}{Abschn.}{Abschn.}
\crefname{subsection}{Abschn.}{Abschn.}
\crefname{table}{Tab.}{Tabs.}
\crefname{figure}{Abb.}{Abbn.}

% Benutzerdefinierte Befehle
\newcommand{\Tfield}{T(x)}
\newcommand{\alphaEM}{\alpha_{\text{EM}}}
\newcommand{\alphaW}{\alpha_{\text{W}}}
\newcommand{\betaT}{\beta_{\text{T}}}
\newcommand{\Mpl}{M_{\text{Pl}}}
\newcommand{\Tzerot}{T_0(\Tfield)}
\newcommand{\Tzero}{T_0}
\newcommand{\vecx}{\vec{x}}
\newcommand{\vr}{\vec{r}}
\newcommand{\gammaf}{\gamma_{\text{Lorentz}}}
\newcommand{\LCDM}{\Lambda\text{CDM}}
\newcommand{\DTmu}{D_{T,\mu}}
\newcommand{\calL}{\mathcal{L}}
\newcommand{\deq}{\displaystyle}
\newcommand{\e}{\mathrm{e}}
\newcommand{\alphaT}{\alpha_{\text{T}}}
\newcommand{\lP}{\ell_{\text{P}}}

% Kopf- und Fußzeilen-Konfiguration
\pagestyle{fancy}
\fancyhf{}
\fancyhead[L]{Johann Pascher}
\fancyhead[R]{Feldtheoretische Herleitung des $\beta$-Parameters}
\fancyfoot[C]{\thepage}
\renewcommand{\headrulewidth}{0.4pt}
\renewcommand{\footrulewidth}{0.4pt}

% Hyperref-Konfiguration
\hypersetup{
	colorlinks=true,
	linkcolor=blue,
	citecolor=red,
	urlcolor=blue,
	bookmarks=true,
	bookmarksnumbered=true,
	pdfstartview=FitH,
	pdftitle={T0-Modell - Feldtheoretische Herleitung des Beta-Parameters},
	pdfauthor={Johann Pascher},
	pdfsubject={T0-Modell, Beta-Parameter, Natürliche Einheiten, Quantenfeldtheorie},
	pdfkeywords={Zeitfeld, Beta-Parameter, Planck-Einheiten, Allgemeine Relativitätstheorie}
}

% Theorem-Umgebungen
\newtheorem{theorem}{Theorem}[section]
\newtheorem{proposition}[theorem]{Proposition}
\newtheorem{definition}[theorem]{Definition}



==================================================

=== DerivationVonBetaEn.tex.preamble ===

\documentclass[12pt,a4paper]{article}
\usepackage[utf8]{inputenc}
\usepackage[T1]{fontenc}
\usepackage{lmodern}
\usepackage{amsmath}
\usepackage{amssymb}
\usepackage{physics}
\usepackage{hyperref}
\usepackage{tcolorbox}
\usepackage{booktabs}
\usepackage{enumitem}
\usepackage[table,xcdraw]{xcolor}
\usepackage[left=2cm,right=2cm,top=2cm,bottom=2cm]{geometry}
\usepackage{pgfplots}
\pgfplotsset{compat=1.18}
\usepackage{graphicx}
\usepackage{float}
\usepackage{fancyhdr}
\usepackage{siunitx}
\usepackage{mathtools}
\usepackage{amsthm}
\usepackage{cleveref}
\usepackage{tocloft}
\usepackage{tikz}
\usepackage[dvipsnames]{xcolor}
\usetikzlibrary{positioning, shapes.geometric, arrows.meta}
\usepackage{microtype}
\usepackage{natbib}
\usepackage{doi}

% Cross-referencing configuration
\crefname{equation}{Eq.}{Eqs.}
\crefname{section}{Sec.}{Secs.}
\crefname{subsection}{Sec.}{Secs.}
\crefname{table}{Tab.}{Tabs.}
\crefname{figure}{Fig.}{Figs.}

% Custom commands
\newcommand{\Tfield}{T(x)}
\newcommand{\alphaEM}{\alpha_{\text{EM}}}
\newcommand{\alphaW}{\alpha_{\text{W}}}
\newcommand{\betaT}{\beta_{\text{T}}}
\newcommand{\Mpl}{M_{\text{Pl}}}
\newcommand{\Tzerot}{T_0(\Tfield)}
\newcommand{\Tzero}{T_0}
\newcommand{\vecx}{\vec{x}}
\newcommand{\vr}{\vec{r}}
\newcommand{\gammaf}{\gamma_{\text{Lorentz}}}
\newcommand{\LCDM}{\Lambda\text{CDM}}
\newcommand{\DTmu}{D_{T,\mu}}
\newcommand{\calL}{\mathcal{L}}
\newcommand{\deq}{\displaystyle}
\newcommand{\e}{\mathrm{e}}
\newcommand{\alphaT}{\alpha_{\text{T}}}
\newcommand{\lP}{\ell_{\text{P}}}

% Header and footer configuration
\pagestyle{fancy}
\fancyhf{}
\fancyhead[L]{Johann Pascher}
\fancyhead[R]{Field-Theoretic Derivation of the $\beta$-Parameter}
\fancyfoot[C]{\thepage}
\renewcommand{\headrulewidth}{0.4pt}
\renewcommand{\footrulewidth}{0.4pt}

% Hyperref configuration
\hypersetup{
	colorlinks=true,
	linkcolor=blue,
	citecolor=red,
	urlcolor=blue,
	bookmarks=true,
	bookmarksnumbered=true,
	pdfstartview=FitH,
	pdftitle={T0 Model - Field-Theoretic Derivation of the Beta Parameter},
	pdfauthor={Johann Pascher},
	pdfsubject={T0 Model, Beta Parameter, Natural Units, Quantum Field Theory},
	pdfkeywords={Time Field, Beta Parameter, Planck Units, General Relativity}
}

% Theorem environments
\newtheorem{theorem}{Theorem}[section]
\newtheorem{proposition}[theorem]{Proposition}
\newtheorem{definition}[theorem]{Definition}



==================================================

=== DynMassePhotonenNichtlokalDe.tex.preamble ===

\documentclass[12pt,a4paper]{article}
\usepackage[utf8]{inputenc}
\usepackage[T1]{fontenc}
\usepackage[ngerman]{babel}
\usepackage{lmodern}
\usepackage{csquotes}
\usepackage{amsmath}
\usepackage{amssymb}
\usepackage{physics}
\usepackage{geometry}
\usepackage{tocloft}
\usepackage{xcolor}
\usepackage{graphicx,tikz,pgfplots}
\pgfplotsset{compat=1.18}
\usepackage{booktabs}
\usepackage{siunitx}
\usepackage{amsthm}
\usepackage[colorlinks=true, linkcolor=blue, citecolor=blue, urlcolor=blue]{hyperref}
\usepackage{cleveref}
\usepackage{fancyhdr}
\usepackage{tcolorbox}
\usepackage{mathtools}

\geometry{a4paper, margin=2cm}

% Headers and Footers
\pagestyle{fancy}
\fancyhf{}
\fancyhead[L]{Johann Pascher}
\fancyhead[R]{Dynamische Masse von Photonen im T0-Modell - Aktualisiertes Rahmenwerk}
\fancyfoot[C]{\thepage}
\renewcommand{\headrulewidth}{0.4pt}
\renewcommand{\footrulewidth}{0.4pt}

% Table of Contents Styling
\renewcommand{\cftsecfont}{\color{blue}}
\renewcommand{\cftsubsecfont}{\color{blue}}
\renewcommand{\cftsecpagefont}{\color{blue}}
\renewcommand{\cftsubsecpagefont}{\color{blue}}
\setlength{\cftsecindent}{1cm}
\setlength{\cftsubsecindent}{2cm}

% Custom commands (consistent with T0 model reference)
\newcommand{\Tfield}{T(x,t)}
\newcommand{\betaT}{\beta_{\text{T}}}
\newcommand{\alphaEM}{\alpha_{\text{EM}}}
\newcommand{\alphaW}{\alpha_{\text{W}}}
\newcommand{\Mpl}{M_{\text{Pl}}}
\newcommand{\Tzerot}{T_0(\Tfield)}
\newcommand{\Tzero}{T_0}
\newcommand{\vecx}{\vec{x}}
\newcommand{\gammaf}{\gamma_{\text{Lorentz}}}
\newcommand{\DhiggsT}{\Tfield (\partial_\mu + ig A_\mu) \Phi + \Phi \partial_\mu \Tfield}
\newcommand{\xipar}{\xi}
\newcommand{\lP}{\ell_{\text{P}}}

\newtheorem{theorem}{Theorem}[section]
\newtheorem{proposition}[theorem]{Proposition}
\newtheorem{definition}[theorem]{Definition}

\title{Dynamische Masse von Photonen und ihre Implikationen für Nichtlokalität \\ im T0-Modell: Aktualisiertes Rahmenwerk mit \\ vollständigen geometrischen Grundlagen}
\author{Johann Pascher}
\date{\today}



==================================================

=== DynMassePhotonenNichtlokalEn.tex.preamble ===

\documentclass[12pt,a4paper]{article}
\usepackage[utf8]{inputenc}
\usepackage[T1]{fontenc}
\usepackage[english]{babel}
\usepackage{lmodern}
\usepackage{csquotes}
\usepackage{amsmath}
\usepackage{amssymb}
\usepackage{physics}
\usepackage{geometry}
\usepackage{tocloft}
\usepackage{xcolor}
\usepackage{graphicx,tikz,pgfplots}
\pgfplotsset{compat=1.18}
\usepackage{booktabs}
\usepackage{siunitx}
\usepackage{amsthm}
\usepackage[colorlinks=true, linkcolor=blue, citecolor=blue, urlcolor=blue]{hyperref}
\usepackage{cleveref}
\usepackage{fancyhdr}
\usepackage{tcolorbox}
\usepackage{mathtools}

\geometry{a4paper, margin=2cm}

% Headers and Footers
\pagestyle{fancy}
\fancyhf{}
\fancyhead[L]{Johann Pascher}
\fancyhead[R]{Dynamic Mass of Photons in T0 Model - Updated Framework}
\fancyfoot[C]{\thepage}
\renewcommand{\headrulewidth}{0.4pt}
\renewcommand{\footrulewidth}{0.4pt}

% Table of Contents Styling
\renewcommand{\cftsecfont}{\color{blue}}
\renewcommand{\cftsubsecfont}{\color{blue}}
\renewcommand{\cftsecpagefont}{\color{blue}}
\renewcommand{\cftsubsecpagefont}{\color{blue}}
\setlength{\cftsecindent}{1cm}
\setlength{\cftsubsecindent}{2cm}

% Custom commands (consistent with T0 model reference)
\newcommand{\Tfield}{T(x,t)}
\newcommand{\betaT}{\beta_{\text{T}}}
\newcommand{\alphaEM}{\alpha_{\text{EM}}}
\newcommand{\alphaW}{\alpha_{\text{W}}}
\newcommand{\Mpl}{M_{\text{Pl}}}
\newcommand{\Tzerot}{T_0(\Tfield)}
\newcommand{\Tzero}{T_0}
\newcommand{\vecx}{\vec{x}}
\newcommand{\gammaf}{\gamma_{\text{Lorentz}}}
\newcommand{\DhiggsT}{\Tfield (\partial_\mu + ig A_\mu) \Phi + \Phi \partial_\mu \Tfield}
\newcommand{\xipar}{\xi}
\newcommand{\lP}{\ell_{\text{P}}}

\newtheorem{theorem}{Theorem}[section]
\newtheorem{proposition}[theorem]{Proposition}
\newtheorem{definition}[theorem]{Definition}

\title{Dynamic Mass of Photons and Its Implications for Nonlocality \\ in the T0 Model: Updated Framework with \\ Complete Geometric Foundations}
\author{Johann Pascher}
\date{\today}



==================================================

=== E-mc2_De.tex.preamble ===

\documentclass[12pt,a4paper]{article}
\usepackage[utf8]{inputenc}
\usepackage[T1]{fontenc}
\usepackage[ngerman]{babel}
\usepackage[left=2cm,right=2cm,top=2cm,bottom=2cm]{geometry}
\usepackage{lmodern}
\usepackage{amsmath}
\usepackage{amssymb}
\usepackage{physics}
\usepackage{hyperref}
\usepackage{tcolorbox}
\usepackage{booktabs}
\usepackage{enumitem}
\usepackage[table,xcdraw]{xcolor}
\usepackage{graphicx}
\usepackage{float}
\usepackage{mathtools}
\usepackage{amsthm}
\usepackage{siunitx}
\usepackage{fancyhdr}
\usepackage{microtype}

% Kopf- und Fußzeilen
\pagestyle{fancy}
\fancyhf{}
\fancyhead[L]{Johann Pascher}
\fancyhead[R]{E=mc² = E=m: Die Konstanten-Illusion entlarvt}
\fancyfoot[C]{\thepage}
\renewcommand{\headrulewidth}{0.4pt}
\renewcommand{\footrulewidth}{0.4pt}

% Benutzerdefinierte Befehle
\newcommand{\Tfield}{T}
\newcommand{\xipar}{\xi}

\hypersetup{
	colorlinks=true,
	linkcolor=blue,
	citecolor=blue,
	urlcolor=blue,
	pdftitle={E=mc² = E=m: Die Konstanten-Illusion entlarvt},
	pdfauthor={Johann Pascher},
	pdfsubject={T0-Modell, Einstein-Kritik, c-Konstante}
}

\newtheorem{theorem}{Theorem}[section]
\newtheorem{proposition}[theorem]{Proposition}
\newtheorem{definition}[theorem]{Definition}



==================================================

=== E-mc2_En.tex.preamble ===

\documentclass[12pt,a4paper]{article}
\usepackage[utf8]{inputenc}
\usepackage[T1]{fontenc}
\usepackage[english]{babel}
\usepackage[left=2cm,right=2cm,top=2cm,bottom=2cm]{geometry}
\usepackage{lmodern}
\usepackage{amsmath}
\usepackage{amssymb}
\usepackage{physics}
\usepackage{hyperref}
\usepackage{tcolorbox}
\usepackage{booktabs}
\usepackage{enumitem}
\usepackage[table,xcdraw]{xcolor}
\usepackage{graphicx}
\usepackage{float}
\usepackage{mathtools}
\usepackage{amsthm}
\usepackage{siunitx}
\usepackage{fancyhdr}
\usepackage{microtype}

% Headers and Footers
\pagestyle{fancy}
\fancyhf{}
\fancyhead[L]{Johann Pascher}
\fancyhead[R]{E=mc² = E=m: The Constants Illusion Exposed}
\fancyfoot[C]{\thepage}
\renewcommand{\headrulewidth}{0.4pt}
\renewcommand{\footrulewidth}{0.4pt}

% Custom Commands
\newcommand{\Tfield}{T}
\newcommand{\xipar}{\xi}

\hypersetup{
	colorlinks=true,
	linkcolor=blue,
	citecolor=blue,
	urlcolor=blue,
	pdftitle={E=mc² = E=m: The Constants Illusion Exposed},
	pdfauthor={Johann Pascher},
	pdfsubject={T0 Model, Einstein Critique, c-constant}
}

\newtheorem{theorem}{Theorem}[section]
\newtheorem{proposition}[theorem]{Proposition}
\newtheorem{definition}[theorem]{Definition}



==================================================

=== EliminationOfMassDe.tex.preamble ===

\documentclass[12pt,a4paper]{article}
\usepackage[utf8]{inputenc}
\usepackage[T1]{fontenc}
\usepackage[ngerman]{babel}
\usepackage[left=2cm,right=2cm,top=2cm,bottom=2cm]{geometry}
\usepackage{lmodern}
\usepackage{amsmath}
\usepackage{amssymb}
\usepackage{physics}
\usepackage{hyperref}
\usepackage{tcolorbox}
\usepackage{booktabs}
\usepackage{enumitem}
\usepackage[table,xcdraw]{xcolor}
\usepackage{pgfplots}
\pgfplotsset{compat=1.18}
\usepackage{graphicx}
\usepackage{float}
\usepackage{mathtools}
\usepackage{amsthm}
\usepackage{cleveref}
\usepackage{siunitx}
\usepackage{fancyhdr}
\usepackage{tocloft}

% Headers and Footers
\pagestyle{fancy}
\fancyhf{}
\fancyhead[L]{Johann Pascher}
\fancyhead[R]{Massenelimination im T0-Modell}
\fancyfoot[C]{\thepage}
\renewcommand{\headrulewidth}{0.4pt}
\renewcommand{\footrulewidth}{0.4pt}

% Table of Contents Styling
\renewcommand{\cftsecfont}{\color{blue}}
\renewcommand{\cftsubsecfont}{\color{blue}}
\renewcommand{\cftsecpagefont}{\color{blue}}
\renewcommand{\cftsubsecpagefont}{\color{blue}}
\setlength{\cftsecindent}{1cm}
\setlength{\cftsubsecindent}{2cm}

\hypersetup{
	colorlinks=true,
	linkcolor=blue,
	citecolor=blue,
	urlcolor=blue,
	pdftitle={Elimination der Masse als dimensionaler Platzhalter im T0-Modell},
	pdfauthor={Johann Pascher},
	pdfsubject={T0-Modell, Massenelimination, Planck-Skala},
	pdfkeywords={Zeitfeld, Natürliche Einheiten, Parameterfreie Theorie, Dimensionsanalyse}
}

% Custom Commands
\newcommand{\Tfield}{T(x)}
\newcommand{\Tfieldt}{T(\vec{x},t)}
\newcommand{\betaT}{\beta_{\text{T}}}
\newcommand{\alphaEM}{\alpha_{\text{EM}}}
\newcommand{\alphaW}{\alpha_{\text{W}}}
\newcommand{\Mpl}{M_{\text{Pl}}}
\newcommand{\Tzero}{T_0}
\newcommand{\vecx}{\vec{x}}
\newcommand{\lP}{\ell_{\text{P}}}
\newcommand{\tP}{t_{\text{P}}}
\newcommand{\EP}{E_{\text{P}}}
\newcommand{\xipar}{\xi}

\newtheorem{theorem}{Theorem}[section]
\newtheorem{proposition}[theorem]{Proposition}
\newtheorem{definition}[theorem]{Definition}



==================================================

=== EliminationOfMassEn.tex.preamble ===

\documentclass[12pt,a4paper]{article}
\usepackage[utf8]{inputenc}
\usepackage[T1]{fontenc}
\usepackage[english]{babel}
\usepackage[left=2cm,right=2cm,top=2cm,bottom=2cm]{geometry}
\usepackage{lmodern}
\usepackage{amsmath}
\usepackage{amssymb}
\usepackage{physics}
\usepackage{hyperref}
\usepackage{tcolorbox}
\usepackage{booktabs}
\usepackage{enumitem}
\usepackage[table,xcdraw]{xcolor}
\usepackage{pgfplots}
\pgfplotsset{compat=1.18}
\usepackage{graphicx}
\usepackage{float}
\usepackage{mathtools}
\usepackage{amsthm}
\usepackage{cleveref}
\usepackage{siunitx}
\usepackage{fancyhdr}
\usepackage{tocloft}

% Headers and Footers
\pagestyle{fancy}
\fancyhf{}
\fancyhead[L]{Johann Pascher}
\fancyhead[R]{Mass Elimination in T0 Model}
\fancyfoot[C]{\thepage}
\renewcommand{\headrulewidth}{0.4pt}
\renewcommand{\footrulewidth}{0.4pt}

% Table of Contents Styling
\renewcommand{\cftsecfont}{\color{blue}}
\renewcommand{\cftsubsecfont}{\color{blue}}
\renewcommand{\cftsecpagefont}{\color{blue}}
\renewcommand{\cftsubsecpagefont}{\color{blue}}
\setlength{\cftsecindent}{1cm}
\setlength{\cftsubsecindent}{2cm}

\hypersetup{
	colorlinks=true,
	linkcolor=blue,
	citecolor=blue,
	urlcolor=blue,
	pdftitle={Elimination of Mass as Dimensional Placeholder in the T0 Model},
	pdfauthor={Johann Pascher},
	pdfsubject={T0 Model, Mass Elimination, Planck Scale},
	pdfkeywords={Time Field, Natural Units, Parameter-Free Theory, Dimensional Analysis}
}

% Custom Commands
\newcommand{\Tfield}{T(x)}
\newcommand{\Tfieldt}{T(\vec{x},t)}
\newcommand{\betaT}{\beta_{\text{T}}}
\newcommand{\alphaEM}{\alpha_{\text{EM}}}
\newcommand{\alphaW}{\alpha_{\text{W}}}
\newcommand{\Mpl}{M_{\text{Pl}}}
\newcommand{\Tzero}{T_0}
\newcommand{\vecx}{\vec{x}}
\newcommand{\lP}{\ell_{\text{P}}}
\newcommand{\tP}{t_{\text{P}}}
\newcommand{\EP}{E_{\text{P}}}
\newcommand{\xipar}{\xi}

\newtheorem{theorem}{Theorem}[section]
\newtheorem{proposition}[theorem]{Proposition}
\newtheorem{definition}[theorem]{Definition}



==================================================

=== Elimination_Of_Mass_Dirac_LagDe.tex.preamble ===

\documentclass[12pt,a4paper]{article}
\usepackage[utf8]{inputenc}
\usepackage[T1]{fontenc}
\usepackage[ngerman]{babel}
\usepackage{lmodern}
\usepackage{amsmath}
\usepackage{amssymb}
\usepackage{physics}
\usepackage{hyperref}
\usepackage{tcolorbox}
\usepackage{booktabs}
\usepackage{enumitem}
\usepackage[table,xcdraw]{xcolor}
\usepackage[left=2cm,right=2cm,top=2cm,bottom=2cm]{geometry}
\usepackage{pgfplots}
\pgfplotsset{compat=1.18}
\usepackage{graphicx}
\usepackage{float}
\usepackage{fancyhdr}
\usepackage{siunitx}
\usepackage{mathtools}
\usepackage{amsthm}
\usepackage{cleveref}
\usepackage{tocloft}

% Kopf- und Fußzeilen
\pagestyle{fancy}
\fancyhf{}
\fancyhead[L]{Johann Pascher}
\fancyhead[R]{Reine Energie T0-Theorie: Verhältnis-basierte Physik}
\fancyfoot[C]{\thepage}
\renewcommand{\headrulewidth}{0.4pt}
\renewcommand{\footrulewidth}{0.4pt}
\setlength{\headheight}{15pt}

% Inhaltsverzeichnis-Formatierung
\renewcommand{\cftsecfont}{\color{blue}}
\renewcommand{\cftsubsecfont}{\color{blue}}
\renewcommand{\cftsecpagefont}{\color{blue}}
\renewcommand{\cftsubsecpagefont}{\color{blue}}
\setlength{\cftsecindent}{1cm}
\setlength{\cftsubsecindent}{2cm}

% Benutzerdefinierte Befehle
\newcommand{\Lag}{\mathcal{L}}
\newcommand{\deltam}{\delta m}
\newcommand{\Efield}{E}
\newcommand{\xipar}{\xi}

% Theorem-Umgebungen
\newtheorem{theorem}{Theorem}[section]
\newtheorem{proposition}[theorem]{Proposition}
\newtheorem{corollary}[theorem]{Korollar}
\newtheorem{lemma}[theorem]{Lemma}
\theoremstyle{definition}
\newtheorem{definition}[theorem]{Definition}
\newtheorem{example}[theorem]{Beispiel}
\theoremstyle{remark}
\newtheorem{remark}[theorem]{Bemerkung}

\hypersetup{
	colorlinks=true,
	linkcolor=blue,
	citecolor=blue,
	urlcolor=blue,
	pdftitle={Reine Energie T0-Theorie: Verhältnis-basierte Physik mit SI-Referenz},
	pdfauthor={Johann Pascher},
	pdfsubject={T0-Theorie, Verhältnis-basierte Physik, Energie-Skalierung},
	pdfkeywords={T0-Theorie, Energie-Verhältnisse, Skalen-Beziehungen, SI-Referenz}
}

\title{Reine Energie T0-Theorie: Die Verhältnis-basierte Revolution \\
	Von Parameter-Physik zu Skalen-Beziehungen \\
	\large Aufbauend auf vereinfachter Dirac- und universeller Lagrange-Grundlage}
\author{Johann Pascher\\
	Abteilung für Nachrichtentechnik, \\Höhere Technische Bundeslehranstalt (HTL), Leonding, Österreich\\
	\texttt{johann.pascher@gmail.com}}
\date{\today}



==================================================

=== Elimination_Of_Mass_Dirac_LagEn.tex.preamble ===

\documentclass[12pt,a4paper]{article}
\usepackage[utf8]{inputenc}
\usepackage[T1]{fontenc}
\usepackage[english]{babel}
\usepackage[left=2cm,right=2cm,top=2cm,bottom=2cm]{geometry}
\usepackage{lmodern}
\usepackage{amsmath}
\usepackage{amssymb}
\usepackage{physics}
\usepackage{hyperref}
\usepackage{tcolorbox}
\usepackage{booktabs}
\usepackage{enumitem}
\usepackage[table,xcdraw]{xcolor}
\usepackage{pgfplots}
\pgfplotsset{compat=1.18}
\usepackage{graphicx}
\usepackage{float}
\usepackage{mathtools}
\usepackage{amsthm}
\usepackage{cleveref}
\usepackage{siunitx}
\usepackage{fancyhdr}

% Headers and Footers
\pagestyle{fancy}
\fancyhf{}
\fancyhead[L]{Johann Pascher}
\fancyhead[R]{Pure Energy T0 Theory: Ratio-Based Physics}
\fancyfoot[C]{\thepage}
\renewcommand{\headrulewidth}{0.4pt}
\renewcommand{\footrulewidth}{0.4pt}
\setlength{\headheight}{15pt}

% Custom commands
\newcommand{\Lag}{\mathcal{L}}
\newcommand{\deltam}{\delta m}
\newcommand{\Efield}{E}
\newcommand{\xipar}{\xi}

% Theorem environments
\newtheorem{theorem}{Theorem}[section]
\newtheorem{proposition}[theorem]{Proposition}
\newtheorem{corollary}[theorem]{Corollary}
\newtheorem{lemma}[theorem]{Lemma}
\theoremstyle{definition}
\newtheorem{definition}[theorem]{Definition}
\newtheorem{example}[theorem]{Example}
\theoremstyle{remark}
\newtheorem{remark}[theorem]{Remark}

\hypersetup{
	colorlinks=true,
	linkcolor=blue,
	citecolor=blue,
	urlcolor=blue,
	pdftitle={Pure Energy T0 Theory: Ratio-Based Physics with SI Reference},
	pdfauthor={Johann Pascher},
	pdfsubject={T0 Theory, Ratio-Based Physics, Energy Scaling},
	pdfkeywords={T0 Theory, Energy Ratios, Scale Relations, SI Reference}
}

\title{Pure Energy T0 Theory: The Ratio-Based Revolution \\
	From Parameter Physics to Scale Relations \\
	\large Building on Simplified Dirac and Universal Lagrangian Foundations}
\author{Johann Pascher\\
	Department of Communications Engineering, \\H\"ohere Technische Bundeslehranstalt (HTL), Leonding, Austria\\
	\texttt{johann.pascher@gmail.com}}
\date{\today}



==================================================

=== Elimination_Of_Mass_Dirac_TabelleDe.tex.preamble ===

\documentclass[12pt,a4paper]{article}
\usepackage[utf8]{inputenc}
\usepackage[T1]{fontenc}
\usepackage[ngerman]{babel}
\usepackage[left=2cm,right=2cm,top=2cm,bottom=2cm]{geometry}
\usepackage{amsmath}
\usepackage{amssymb}
\usepackage{booktabs}
\usepackage{longtable}
\usepackage{array}
\usepackage[table,xcdraw]{xcolor}
\usepackage{siunitx}
\usepackage{pdflscape}
\usepackage{url}
\usepackage{tcolorbox}
\usepackage{hyperref}
\usepackage{fancyhdr}

% Kopf- und Fußzeilen
\pagestyle{fancy}
\fancyhf{}
\fancyhead[L]{Johann Pascher}
\fancyhead[R]{T0-Modell-Verifikation: Skalen-Verhältnis-basierte Berechnungen}
\fancyfoot[C]{\thepage}
\renewcommand{\headrulewidth}{0.4pt}
\renewcommand{\footrulewidth}{0.4pt}

\hypersetup{
	colorlinks=true,
	linkcolor=blue,
	citecolor=blue,
	urlcolor=blue,
	pdftitle={T0-Modell-Verifikation: Skalen-Verhältnis-basierte Berechnungen},
	pdfauthor={Johann Pascher},
	pdfsubject={T0-Modell, Skalen-Verhältnisse, Verifikation},
	pdfkeywords={T0-Modell, Energie-Verhältnisse, CODATA, Experimentelle Werte}
}

\title{T0-Modell-Verifikation: Skalen-Verhältnis-basierte Berechnungen}
\author{T0-Modell-Analyse}
\date{\today}



==================================================

=== Elimination_Of_Mass_Dirac_TabelleEn.tex.preamble ===

\documentclass[12pt,a4paper]{article}
\usepackage[utf8]{inputenc}
\usepackage[T1]{fontenc}
\usepackage[english]{babel}
\usepackage[left=2cm,right=2cm,top=2cm,bottom=2cm]{geometry}
\usepackage{amsmath}
\usepackage{amssymb}
\usepackage{booktabs}
\usepackage{longtable}
\usepackage{array}
\usepackage[table,xcdraw]{xcolor}
\usepackage{siunitx}
\usepackage{pdflscape}
\usepackage{url}
\usepackage{tcolorbox}

\title{T0 Model Verification: Scale Ratio-Based Calculations}
\author{T0 Model Analysis}
\date{\today}



==================================================

=== FeinstrukturkonstanteDe.tex.preamble ===

\documentclass[12pt,a4paper]{article}
\usepackage[utf8]{inputenc}
\usepackage[T1]{fontenc}
\usepackage[ngerman]{babel}
\usepackage[left=2cm,right=2cm,top=2cm,bottom=2cm]{geometry}
\usepackage{lmodern}
\usepackage{amssymb}
\usepackage{physics}
\usepackage{hyperref}
\usepackage{tcolorbox}
\usepackage{booktabs}
\usepackage{enumitem}
\usepackage[table,xcdraw]{xcolor}
\usepackage{pgfplots}
\pgfplotsset{compat=1.18}
\usepackage{graphicx}
\usepackage{float}
\usepackage{mathtools}
\usepackage{amsthm}
\usepackage{cleveref}
\usepackage{siunitx}
\usepackage{fancyhdr}
\usepackage{tocloft}
\usepackage{tikz}
\usepackage[dvipsnames]{xcolor}
\usetikzlibrary{positioning, shapes.geometric, arrows.meta}
\usepackage{microtype}
\usepackage{forest}
\usepackage{amsmath}

% Headers and Footers
\pagestyle{fancy}
\fancyhf{}
\fancyhead[L]{Johann Pascher}
\fancyhead[R]{Die Feinstrukturkonstante: Verschiedene Darstellungen und Beziehungen}
\fancyfoot[C]{\thepage}
\renewcommand{\headrulewidth}{0.4pt}
\renewcommand{\footrulewidth}{0.4pt}

% Table of Contents Styling
\renewcommand{\cftsecfont}{\color{blue}}
\renewcommand{\cftsubsecfont}{\color{blue}}
\renewcommand{\cftsecpagefont}{\color{blue}}
\renewcommand{\cftsubsecpagefont}{\color{blue}}
\setlength{\cftsecindent}{1cm}
\setlength{\cftsubsecindent}{2cm}

\hypersetup{
	colorlinks=true,
	linkcolor=blue,
	citecolor=blue,
	urlcolor=blue,
	pdftitle={Die Feinstrukturkonstante: Verschiedene Darstellungen und Beziehungen},
	pdfauthor={Johann Pascher},
	pdfsubject={Feinstrukturkonstante, Natürliche Einheiten, Fundamentale Physik},
	pdfkeywords={Feinstrukturkonstante, Alpha, Natürliche Einheiten, Quantenelektrodynamik}
}

% Custom Commands
\newcommand{\alphaem}{\alpha_{EM}}
\newcommand{\betaT}{\beta_{\text{T}}}
\newcommand{\alphaT}{\alpha_{\text{T}}}

\newtheorem{theorem}{Theorem}[section]
\newtheorem{proposition}[theorem]{Proposition}
\newtheorem{definition}[theorem]{Definition}



==================================================

=== FeinstrukturkonstanteEn.tex.preamble ===

\documentclass[12pt,a4paper]{article}
\usepackage[utf8]{inputenc}
\usepackage[T1]{fontenc}
\usepackage[english]{babel}
\usepackage[left=2cm,right=2cm,top=2cm,bottom=2cm]{geometry}
\usepackage{lmodern}
\usepackage{amssymb}
\usepackage{physics}
\usepackage{hyperref}
\usepackage{tcolorbox}
\usepackage{booktabs}
\usepackage{enumitem}
\usepackage[table,xcdraw]{xcolor}
\usepackage{pgfplots}
\pgfplotsset{compat=1.18}
\usepackage{graphicx}
\usepackage{float}
\usepackage{mathtools}
\usepackage{amsthm}
\usepackage{cleveref}
\usepackage{siunitx}
\usepackage{fancyhdr}
\usepackage{tocloft}
\usepackage{tikz}
\usepackage[dvipsnames]{xcolor}
\usetikzlibrary{positioning, shapes.geometric, arrows.meta}
\usepackage{microtype}
\usepackage{forest}
\usepackage{amsmath}

% Headers and Footers
\pagestyle{fancy}
\fancyhf{}
\fancyhead[L]{Johann Pascher}
\fancyhead[R]{The Fine Structure Constant: Various Representations and Relationships}
\fancyfoot[C]{\thepage}
\renewcommand{\headrulewidth}{0.4pt}
\renewcommand{\footrulewidth}{0.4pt}

% Table of Contents Styling
\renewcommand{\cftsecfont}{\color{blue}}
\renewcommand{\cftsubsecfont}{\color{blue}}
\renewcommand{\cftsecpagefont}{\color{blue}}
\renewcommand{\cftsubsecpagefont}{\color{blue}}
\setlength{\cftsecindent}{1cm}
\setlength{\cftsubsecindent}{2cm}

\hypersetup{
	colorlinks=true,
	linkcolor=blue,
	citecolor=blue,
	urlcolor=blue,
	pdftitle={The Fine Structure Constant: Various Representations and Relationships},
	pdfauthor={Johann Pascher},
	pdfsubject={Fine Structure Constant, Natural Units, Fundamental Physics},
	pdfkeywords={Fine Structure Constant, Alpha, Natural Units, Quantum Electrodynamics}
}

% Custom Commands
\newcommand{\alphaem}{\alpha_{EM}}
\newcommand{\betaT}{\beta_{\text{T}}}
\newcommand{\alphaT}{\alpha_{\text{T}}}

\newtheorem{theorem}{Theorem}[section]
\newtheorem{proposition}[theorem]{Proposition}
\newtheorem{definition}[theorem]{Definition}



==================================================

=== Formeln_Energiebasiert_De.tex.preamble ===

\documentclass[12pt,a4paper]{article}
\usepackage[utf8]{inputenc}
\usepackage[T1]{fontenc}
\usepackage[german]{babel}
\usepackage{lmodern}
\usepackage{amsmath}
\usepackage{amssymb}
\usepackage{physics}
\usepackage{hyperref}
\usepackage{tcolorbox}
\usepackage{booktabs}
\usepackage{enumitem}
\usepackage[table,xcdraw]{xcolor}
\usepackage[left=2cm,right=2cm,top=2cm,bottom=2cm]{geometry}
\usepackage{pgfplots}
\pgfplotsset{compat=1.18}
\usepackage{graphicx}
\usepackage{float}
\usepackage{fancyhdr}
\usepackage{siunitx}
\usepackage{mathtools}
\usepackage{amsthm}
\usepackage{cleveref}
\usepackage{tikz}
\usepackage{microtype}
\usepackage{array}

\hypersetup{
	colorlinks=true,
	linkcolor=blue,
	urlcolor=blue,
	citecolor=blue,
	pdftitle={T0-Modell: Energiebasierte Formeln mit quadratischer Skalierung},
	pdfauthor={Johann Pascher},
	pdfsubject={Theoretical Physics},
	pdfkeywords={T0 Model, Energy-based Formulas, QFT}
}

\newcommand{\xipar}{\xi}
\newcommand{\alphagem}{\alpha}
\newcommand{\epsilonT}{\varepsilon_T}

\pagestyle{fancy}
\fancyhf{}
\fancyhead[L]{Johann Pascher}
\fancyhead[R]{T0-Modell: Energiebasierte Formeln}
\fancyfoot[C]{\thepage}
\renewcommand{\headrulewidth}{0.4pt}
\renewcommand{\footrulewidth}{0.4pt}

\tcbuselibrary{theorems}
\newtcolorbox{important}{colback=green!5!white,colframe=green!35!black,fonttitle=\bfseries}
\newtcolorbox{warning}{colback=red!5!white,colframe=red!75!black,fonttitle=\bfseries}
\newtcolorbox{highlight}{colback=blue!5!white,colframe=blue!75!black,fonttitle=\bfseries}



==================================================

=== Formeln_Energiebasiert_En.tex.preamble ===

\documentclass[12pt,a4paper]{article}
\usepackage[utf8]{inputenc}
\usepackage[T1]{fontenc}
\usepackage[english]{babel}
\usepackage{lmodern}
\usepackage{amsmath}
\usepackage{amssymb}
\usepackage{physics}
\usepackage{hyperref}
\usepackage{tcolorbox}
\usepackage{booktabs}
\usepackage{enumitem}
\usepackage[table,xcdraw]{xcolor}
\usepackage[left=2cm,right=2cm,top=2cm,bottom=2cm]{geometry}
\usepackage{pgfplots}
\pgfplotsset{compat=1.18}
\usepackage{graphicx}
\usepackage{float}
\usepackage{fancyhdr}
\usepackage{siunitx}
\usepackage{mathtools}
\usepackage{amsthm}
\usepackage{cleveref}
\usepackage{tikz}
\usepackage{microtype}
\usepackage{array}

\hypersetup{
	colorlinks=true,
	linkcolor=blue,
	urlcolor=blue,
	citecolor=blue,
	pdftitle={T0 Model: Energy-based Formulas with Quadratic Scaling},
	pdfauthor={Johann Pascher},
	pdfsubject={Theoretical Physics},
	pdfkeywords={T0 Model, Energy-based Formulas, QFT}
}

\newcommand{\xipar}{\xi}
\newcommand{\alphagem}{\alpha}
\newcommand{\epsilonT}{\varepsilon_T}

\pagestyle{fancy}
\fancyhf{}
\fancyhead[L]{Johann Pascher}
\fancyhead[R]{T0 Model: Energy-based Formulas}
\fancyfoot[C]{\thepage}
\renewcommand{\headrulewidth}{0.4pt}
\renewcommand{\footrulewidth}{0.4pt}

\tcbuselibrary{theorems}
\newtcolorbox{important}{colback=green!5!white,colframe=green!35!black,fonttitle=\bfseries}
\newtcolorbox{warning}{colback=red!5!white,colframe=red!75!black,fonttitle=\bfseries}
\newtcolorbox{highlight}{colback=blue!5!white,colframe=blue!75!black,fonttitle=\bfseries}



==================================================

=== Hannah_De.tex.preamble ===

\documentclass[11pt,a4paper]{article}
\usepackage[utf8]{inputenc}
\usepackage[T1]{fontenc}
\usepackage[german]{babel}
\usepackage[left=2cm,right=2cm,top=2cm,bottom=2cm]{geometry}
\usepackage{lmodern}
\usepackage{amsmath}
\usepackage{amssymb}
\usepackage{amsfonts}
\usepackage{physics}
\usepackage{hyperref}
\usepackage{booktabs}
\usepackage{enumitem}
\usepackage[table,xcdraw]{xcolor}
\usepackage{longtable}
\usepackage{siunitx}
\usepackage{fancyhdr}
\usepackage{textgreek}
\usepackage{graphicx} % Zur Aufnahme von Abbildungen

% Header und Footer
\pagestyle{fancy}
\fancyhf{}
\fancyhead[L]{T0-Theorie: Verbindungen zum Mizohata-Takeuchi-Gegenbeispiel}
\fancyhead[R]{\thepage}
\fancyfoot[C]{\textit{Fraktale Dualität in dispersiven partiellen Differentialgleichungen}}
\renewcommand{\headrulewidth}{0.4pt}
\renewcommand{\footrulewidth}{0.4pt}
\setlength{\headheight}{15pt}  % Behebung der headheight-Warnung

\hypersetup{
	colorlinks=true,
	linkcolor=blue,
	citecolor=blue,
	urlcolor=blue,
	pdftitle={Verbindungen zwischen dem Mizohata-Takeuchi-Gegenbeispiel und der T0-Zeit-Masse-Dualitätstheorie},
	pdfauthor={Johann Pascher},
	pdfsubject={T0-Theorie, dispersive PDEs, Fraktale Geometrie, Mizohata-Takeuchi-Vermutung, Quantenphysik}
}

% Definition gängiger mathematischer Symbole für konsistente Verwendung (T0-Stil)
\newcommand{\xipar}{\ensuremath{\xi}}
\newcommand{\deltafield}{\ensuremath{\delta m}}
\newcommand{\partialop}{\ensuremath{\partial}}
\newcommand{\lambdah}{\ensuremath{\lambda_h}}
\newcommand{\betaT}{\ensuremath{\beta_T}}
\newcommand{\alphaEM}{\ensuremath{\alpha_{\text{EM}}}}
\newcommand{\rhofield}{\ensuremath{\rho}}
\newcommand{\mypi}{\ensuremath{\pi}}
\newcommand{\myphi}{\ensuremath{\phi}}
\newcommand{\myomega}{\ensuremath{\omega}}
\newcommand{\mytimes}{\ensuremath{\times}}
\newcommand{\myapprox}{\ensuremath{\approx}}
\newcommand{\myrightarrow}{\ensuremath{\rightarrow}}
\newcommand{\myRightarrow}{\ensuremath{\Rightarrow}}
\newcommand{\mypropto}{\ensuremath{\propto}}
\newcommand{\mysim}{\ensuremath{\sim}}
\newcommand{\mysqrt}{\ensuremath{\sqrt}}

\title{\Huge\textbf{T0-Theorie: Verbindungen zum Mizohata-Takeuchi-Gegenbeispiel}\\
	\Large Analyse der fraktalen Dualität in dispersiven partiellen Differentialgleichungen}
\author{Johann Pascher\\
	Abteilung für Kommunikationstechnik, \\Höhere Technische Bundeslehr- und Versuchsanstalt (HTL), Leonding, Österreich\\
	\texttt{johann.pascher@gmail.com}}
\date{November 2025}



==================================================

=== Hannah_En.tex.preamble ===

\documentclass[11pt,a4paper]{article}
\usepackage[utf8]{inputenc}
\usepackage[T1]{fontenc}
\usepackage[english]{babel}
\usepackage[left=2cm,right=2cm,top=2cm,bottom=2cm]{geometry}
\usepackage{lmodern}
\usepackage{amsmath}
\usepackage{amssymb}
\usepackage{amsfonts}
\usepackage{physics}
\usepackage{hyperref}
\usepackage{booktabs}
\usepackage{enumitem}
\usepackage[table,xcdraw]{xcolor}
\usepackage{longtable}
\usepackage{siunitx}
\usepackage{fancyhdr}
\usepackage{textgreek}
\usepackage{graphicx} % Retained for figures

% Header and Footer
\pagestyle{fancy}
\fancyhf{}
\fancyhead[L]{T0-Theory: Connections to Mizohata-Takeuchi Counterexample}
\fancyhead[R]{\thepage}
\fancyfoot[C]{\textit{Fractal Duality in Dispersive PDEs}}
\renewcommand{\headrulewidth}{0.4pt}
\renewcommand{\footrulewidth}{0.4pt}
\setlength{\headheight}{15pt}  % Fix for headheight warning

\hypersetup{
	colorlinks=true,
	linkcolor=blue,
	citecolor=blue,
	urlcolor=blue,
	pdftitle={Connections between the Mizohata-Takeuchi Counterexample and the T0 Time-Mass Duality Theory},
	pdfauthor={Johann Pascher},
	pdfsubject={T0-Theory, Dispersive PDEs, Fractal Geometry, Mizohata-Takeuchi Conjecture, Quantum Physics}
}

% Define common mathematical symbols for consistent usage (T0-style)
\newcommand{\xipar}{\ensuremath{\xi}}
\newcommand{\deltafield}{\ensuremath{\delta m}}
\newcommand{\partialop}{\ensuremath{\partial}}
\newcommand{\lambdah}{\ensuremath{\lambda_h}}
\newcommand{\betaT}{\ensuremath{\beta_T}}
\newcommand{\alphaEM}{\ensuremath{\alpha_{\text{EM}}}}
\newcommand{\rhofield}{\ensuremath{\rho}}
\newcommand{\mypi}{\ensuremath{\pi}}
\newcommand{\myphi}{\ensuremath{\phi}}
\newcommand{\myomega}{\ensuremath{\omega}}
\newcommand{\mytimes}{\ensuremath{\times}}
\newcommand{\myapprox}{\ensuremath{\approx}}
\newcommand{\myrightarrow}{\ensuremath{\rightarrow}}
\newcommand{\myRightarrow}{\ensuremath{\Rightarrow}}
\newcommand{\mypropto}{\ensuremath{\propto}}
\newcommand{\mysim}{\ensuremath{\sim}}
\newcommand{\mysqrt}{\ensuremath{\sqrt}}

\title{\Huge\textbf{T0-Theory: Connections to Mizohata-Takeuchi Counterexample}\\
	\Large Analysis of Fractal Duality in Dispersive Partial Differential Equations}
\author{Johann Pascher\\
	Department of Communication Engineering, \\Higher Technical Federal Teaching and Research Institute (HTL), Leonding, Austria\\
	\texttt{johann.pascher@gmail.com}}
\date{November 2025}



==================================================

=== HdokumentDe.tex.preamble ===

% T0 Modell Master Dokument - Deutsche Version mit vollständiger Unicode-Kodierung
\documentclass[12pt,a4paper]{report}
\usepackage[utf8]{inputenc}
\usepackage[T1]{fontenc}
\usepackage[ngerman]{babel}
\usepackage[left=2.5cm,right=2.5cm,top=3cm,bottom=3cm]{geometry}
\usepackage{lmodern}
\usepackage{amsmath}
\usepackage{amssymb}
\usepackage{physics}
\usepackage{hyperref}
\usepackage{tcolorbox}
\usepackage{booktabs}
\usepackage{enumitem}
\usepackage[table]{xcolor}
\usepackage{graphicx}
\usepackage{float}
\usepackage{mathtools}
\usepackage{amsthm}
\usepackage{cleveref}
\usepackage{siunitx}
\usepackage{fancyhdr}
\usepackage{tocloft}
\usepackage{longtable}
\usepackage{array}
\usepackage{microtype}
\usepackage{pdflscape}
\usepackage{newunicodechar}

% Vollständige Unicode-Kodierung für alle griechischen Buchstaben
% Kleine griechische Buchstaben
\newunicodechar{α}{\ensuremath{\alpha}}
\newunicodechar{β}{\ensuremath{\beta}}
\newunicodechar{γ}{\ensuremath{\gamma}}
\newunicodechar{δ}{\ensuremath{\delta}}
\newunicodechar{ε}{\ensuremath{\varepsilon}}
\newunicodechar{ζ}{\ensuremath{\zeta}}
\newunicodechar{η}{\ensuremath{\eta}}
\newunicodechar{θ}{\ensuremath{\theta}}
\newunicodechar{ι}{\ensuremath{\iota}}
\newunicodechar{κ}{\ensuremath{\kappa}}
\newunicodechar{λ}{\ensuremath{\lambda}}
\newunicodechar{μ}{\ensuremath{\mu}}
\newunicodechar{ν}{\ensuremath{\nu}}
\newunicodechar{ξ}{\ensuremath{\xi}}
\newunicodechar{ο}{\ensuremath{o}}
\newunicodechar{π}{\ensuremath{\pi}}
\newunicodechar{ρ}{\ensuremath{\rho}}
\newunicodechar{σ}{\ensuremath{\sigma}}
\newunicodechar{τ}{\ensuremath{\tau}}
\newunicodechar{υ}{\ensuremath{\upsilon}}
\newunicodechar{φ}{\ensuremath{\varphi}}
\newunicodechar{χ}{\ensuremath{\chi}}
\newunicodechar{ψ}{\ensuremath{\psi}}
\newunicodechar{ω}{\ensuremath{\omega}}

% Große griechische Buchstaben
\newunicodechar{Α}{\ensuremath{A}}
\newunicodechar{Β}{\ensuremath{B}}
\newunicodechar{Γ}{\ensuremath{\Gamma}}
\newunicodechar{Δ}{\ensuremath{\Delta}}
\newunicodechar{Ε}{\ensuremath{E}}
\newunicodechar{Ζ}{\ensuremath{Z}}
\newunicodechar{Η}{\ensuremath{H}}
\newunicodechar{Θ}{\ensuremath{\Theta}}
\newunicodechar{Ι}{\ensuremath{I}}
\newunicodechar{Κ}{\ensuremath{K}}
\newunicodechar{Λ}{\ensuremath{\Lambda}}
\newunicodechar{Μ}{\ensuremath{M}}
\newunicodechar{Ν}{\ensuremath{N}}
\newunicodechar{Ξ}{\ensuremath{\Xi}}
\newunicodechar{Ο}{\ensuremath{O}}
\newunicodechar{Π}{\ensuremath{\Pi}}
\newunicodechar{Ρ}{\ensuremath{P}}
\newunicodechar{Σ}{\ensuremath{\Sigma}}
\newunicodechar{Τ}{\ensuremath{T}}
\newunicodechar{Υ}{\ensuremath{\Upsilon}}
\newunicodechar{Φ}{\ensuremath{\Phi}}
\newunicodechar{Χ}{\ensuremath{X}}
\newunicodechar{Ψ}{\ensuremath{\Psi}}
\newunicodechar{Ω}{\ensuremath{\Omega}}

% Weitere mathematische Symbole
\newunicodechar{∞}{\ensuremath{\infty}}
\newunicodechar{∂}{\ensuremath{\partial}}
\newunicodechar{∇}{\ensuremath{\nabla}}
\newunicodechar{√}{\ensuremath{\sqrt}}
\newunicodechar{±}{\ensuremath{\pm}}
\newunicodechar{×}{\ensuremath{\times}}
\newunicodechar{÷}{\ensuremath{\div}}
\newunicodechar{≈}{\ensuremath{\approx}}
\newunicodechar{≠}{\ensuremath{\neq}}
\newunicodechar{≤}{\ensuremath{\leq}}
\newunicodechar{≥}{\ensuremath{\geq}}
\newunicodechar{→}{\ensuremath{\rightarrow}}
\newunicodechar{←}{\ensuremath{\leftarrow}}
\newunicodechar{↔}{\ensuremath{\leftrightarrow}}
\newunicodechar{⇒}{\ensuremath{\Rightarrow}}
\newunicodechar{⇐}{\ensuremath{\Leftarrow}}
\newunicodechar{⇔}{\ensuremath{\Leftrightarrow}}
\newunicodechar{∈}{\ensuremath{\in}}
\newunicodechar{∉}{\ensuremath{\notin}}
\newunicodechar{∩}{\ensuremath{\cap}}
\newunicodechar{∪}{\ensuremath{\cup}}
\newunicodechar{∅}{\ensuremath{\emptyset}}
\newunicodechar{∑}{\ensuremath{\sum}}
\newunicodechar{∏}{\ensuremath{\prod}}
\newunicodechar{∫}{\ensuremath{\int}}
\newunicodechar{★}{\ensuremath{\star}}
\newunicodechar{✓}{\checkmark}

% Bessere Abstände und Zeilenumbrüche
\emergencystretch 3em
\tolerance 9999
\hbadness 9999
\setlength{\hfuzz}{15pt}

% Kopf- und Fußzeilen
\pagestyle{fancy}
\fancyhf{}
\fancyhead[L]{T0 Modell Vollständiges Framework}
\fancyhead[R]{Universelle Energiefeld-Theorie}
\fancyfoot[C]{\thepage}
\renewcommand{\headrulewidth}{0.4pt}
\renewcommand{\footrulewidth}{0.4pt}

% Inhaltsverzeichnis-Stil
\renewcommand{\cfttoctitlefont}{\huge\bfseries\color{blue}}
\renewcommand{\cftchapfont}{\large\bfseries\color{blue}}
\renewcommand{\cftsecfont}{\color{blue}}
\renewcommand{\cftsubsecfont}{\color{blue}}
\renewcommand{\cftchappagefont}{\large\bfseries\color{blue}}
\renewcommand{\cftsecpagefont}{\color{blue}}
\renewcommand{\cftsubsecpagefont}{\color{blue}}

\hypersetup{
	colorlinks=true,
	linkcolor=blue,
	citecolor=blue,
	urlcolor=blue,
	pdftitle={T0 Modell: Vollständiges Framework - Von Zeit-Energie-Dualität zu universellen Konstanten},
	pdfauthor={Johann Pascher},
	pdfsubject={T0 Modell, Zeit-Energie-Dualität, Universelle Konstanten, Natürliche Einheiten},
	pdfkeywords={T0 Theorie, $\xi$-Konstante, Natürliche Einheiten, Universelles Energiefeld, Parameterfreie Physik}
}

% Benutzerdefinierte Befehle - Deutsche Version
\newcommand{\Efield}{E_{\text{Feld}}}
\newcommand{\xikonst}{\xi = \frac{4}{3} \times 10^{-4}}
\newcommand{\xipar}{\xi}
\newcommand{\Exi}{E_\xi}
\newcommand{\EP}{E_{\text{P}}}
\newcommand{\lP}{\ell_{\text{P}}}
\newcommand{\rzero}{r_0}
\newcommand{\tzero}{t_0}
\newcommand{\Gnat}{G_{\text{nat}}}

% Erweiterte Umgebungen
\newtcolorbox{wichtig}[1][]{colback=yellow!10!white,colframe=yellow!50!black,fonttitle=\bfseries,title=Wichtige Erkenntnis,#1}
\newtcolorbox{formel}[1][]{colback=blue!5!white,colframe=blue!75!black,fonttitle=\bfseries,title=T0 Vorhersage,#1}
\newtcolorbox{revolutionaer}[1][]{colback=red!5!white,colframe=red!75!black,fonttitle=\bfseries,title=Revolutionäre Entdeckung,#1}
\newtcolorbox{experimentell}[1][]{colback=green!5!white,colframe=green!75!black,fonttitle=\bfseries,title=Experimentelle Überlegung,#1}
\newtcolorbox{vorsicht}[1][]{colback=orange!5!white,colframe=orange!75!black,fonttitle=\bfseries,title=Experimentelle Vorsicht,#1}

% Theorem-Umgebungen
\newtheorem{prinzip}{Grundprinzip}[chapter]
\newtheorem{erkenntnis}{Schlüsselerkenntnis}[chapter]
\newtheorem{entdeckung}{Revolutionäre Entdeckung}[chapter]



==================================================

=== HdokumentEn.tex.preamble ===

% T0 Model Master Document - English Version with Complete Unicode Encoding
\documentclass[12pt,a4paper]{report}
\usepackage[utf8]{inputenc}
\usepackage[T1]{fontenc}
\usepackage[english]{babel}
\usepackage[left=2.5cm,right=2.5cm,top=3cm,bottom=3cm]{geometry}
\usepackage{lmodern}
\usepackage{amsmath}
\usepackage{amssymb}
\usepackage{physics}
\usepackage{hyperref}
\usepackage{tcolorbox}
\usepackage{booktabs}
\usepackage{enumitem}
\usepackage[table]{xcolor}
\usepackage{graphicx}
\usepackage{float}
\usepackage{mathtools}
\usepackage{amsthm}
\usepackage{cleveref}
\usepackage{siunitx}
\usepackage{fancyhdr}
\usepackage{tocloft}
\usepackage{longtable}
\usepackage{array}
\usepackage{microtype}
\usepackage{pdflscape}
\usepackage{newunicodechar}

% Complete Unicode encoding for all Greek letters
% Small Greek letters
\newunicodechar{α}{\ensuremath{\alpha}}
\newunicodechar{β}{\ensuremath{\beta}}
\newunicodechar{γ}{\ensuremath{\gamma}}
\newunicodechar{δ}{\ensuremath{\delta}}
\newunicodechar{ε}{\ensuremath{\varepsilon}}
\newunicodechar{ζ}{\ensuremath{\zeta}}
\newunicodechar{η}{\ensuremath{\eta}}
\newunicodechar{θ}{\ensuremath{\theta}}
\newunicodechar{ι}{\ensuremath{\iota}}
\newunicodechar{κ}{\ensuremath{\kappa}}
\newunicodechar{λ}{\ensuremath{\lambda}}
\newunicodechar{μ}{\ensuremath{\mu}}
\newunicodechar{ν}{\ensuremath{\nu}}
\newunicodechar{ξ}{\ensuremath{\xi}}
\newunicodechar{ο}{\ensuremath{o}}
\newunicodechar{π}{\ensuremath{\pi}}
\newunicodechar{ρ}{\ensuremath{\rho}}
\newunicodechar{σ}{\ensuremath{\sigma}}
\newunicodechar{τ}{\ensuremath{\tau}}
\newunicodechar{υ}{\ensuremath{\upsilon}}
\newunicodechar{φ}{\ensuremath{\varphi}}
\newunicodechar{χ}{\ensuremath{\chi}}
\newunicodechar{ψ}{\ensuremath{\psi}}
\newunicodechar{ω}{\ensuremath{\omega}}

% Capital Greek letters
\newunicodechar{Α}{\ensuremath{A}}
\newunicodechar{Β}{\ensuremath{B}}
\newunicodechar{Γ}{\ensuremath{\Gamma}}
\newunicodechar{Δ}{\ensuremath{\Delta}}
\newunicodechar{Ε}{\ensuremath{E}}
\newunicodechar{Ζ}{\ensuremath{Z}}
\newunicodechar{Η}{\ensuremath{H}}
\newunicodechar{Θ}{\ensuremath{\Theta}}
\newunicodechar{Ι}{\ensuremath{I}}
\newunicodechar{Κ}{\ensuremath{K}}
\newunicodechar{Λ}{\ensuremath{\Lambda}}
\newunicodechar{Μ}{\ensuremath{M}}
\newunicodechar{Ν}{\ensuremath{N}}
\newunicodechar{Ξ}{\ensuremath{\Xi}}
\newunicodechar{Ο}{\ensuremath{O}}
\newunicodechar{Π}{\ensuremath{\Pi}}
\newunicodechar{Ρ}{\ensuremath{P}}
\newunicodechar{Σ}{\ensuremath{\Sigma}}
\newunicodechar{Τ}{\ensuremath{T}}
\newunicodechar{Υ}{\ensuremath{\Upsilon}}
\newunicodechar{Φ}{\ensuremath{\Phi}}
\newunicodechar{Χ}{\ensuremath{X}}
\newunicodechar{Ψ}{\ensuremath{\Psi}}
\newunicodechar{Ω}{\ensuremath{\Omega}}

% Additional mathematical symbols
\newunicodechar{∞}{\ensuremath{\infty}}
\newunicodechar{∂}{\ensuremath{\partial}}
\newunicodechar{∇}{\ensuremath{\nabla}}
\newunicodechar{√}{\ensuremath{\sqrt}}
\newunicodechar{±}{\ensuremath{\pm}}
\newunicodechar{×}{\ensuremath{\times}}
\newunicodechar{÷}{\ensuremath{\div}}
\newunicodechar{≈}{\ensuremath{\approx}}
\newunicodechar{≠}{\ensuremath{\neq}}
\newunicodechar{≤}{\ensuremath{\leq}}
\newunicodechar{≥}{\ensuremath{\geq}}
\newunicodechar{→}{\ensuremath{\rightarrow}}
\newunicodechar{←}{\ensuremath{\leftarrow}}
\newunicodechar{↔}{\ensuremath{\leftrightarrow}}
\newunicodechar{⇒}{\ensuremath{\Rightarrow}}
\newunicodechar{⇐}{\ensuremath{\Leftarrow}}
\newunicodechar{⇔}{\ensuremath{\Leftrightarrow}}
\newunicodechar{∈}{\ensuremath{\in}}
\newunicodechar{∉}{\ensuremath{\notin}}
\newunicodechar{∩}{\ensuremath{\cap}}
\newunicodechar{∪}{\ensuremath{\cup}}
\newunicodechar{∅}{\ensuremath{\emptyset}}
\newunicodechar{∑}{\ensuremath{\sum}}
\newunicodechar{∏}{\ensuremath{\prod}}
\newunicodechar{∫}{\ensuremath{\int}}
\newunicodechar{★}{\ensuremath{\star}}
\newunicodechar{✓}{\checkmark}

% Better spacing and line breaks
\emergencystretch 3em
\tolerance 9999
\hbadness 9999
\setlength{\hfuzz}{15pt}

% Headers and footers
\pagestyle{fancy}
\fancyhf{}
\fancyhead[L]{T0 Model Complete Framework}
\fancyhead[R]{Universal Energy Field Theory}
\fancyfoot[C]{\thepage}
\renewcommand{\headrulewidth}{0.4pt}
\renewcommand{\footrulewidth}{0.4pt}

% Table of contents style
\renewcommand{\cfttoctitlefont}{\huge\bfseries\color{blue}}
\renewcommand{\cftchapfont}{\large\bfseries\color{blue}}
\renewcommand{\cftsecfont}{\color{blue}}
\renewcommand{\cftsubsecfont}{\color{blue}}
\renewcommand{\cftchappagefont}{\large\bfseries\color{blue}}
\renewcommand{\cftsecpagefont}{\color{blue}}
\renewcommand{\cftsubsecpagefont}{\color{blue}}

\hypersetup{
	colorlinks=true,
	linkcolor=blue,
	citecolor=blue,
	urlcolor=blue,
	pdftitle={T0 Model: Complete Framework - From Time-Energy Duality to Universal Constants},
	pdfauthor={Johann Pascher},
	pdfsubject={T0 Model, Time-Energy Duality, Universal Constants, Natural Units},
	pdfkeywords={T0 Theory, $\xi$-constant, Natural Units, Universal Energy Field, Parameter-free Physics}
}

% Custom commands - English version
\newcommand{\Efield}{E_{\text{field}}}
\newcommand{\xiconst}{\xi = \frac{4}{3} \times 10^{-4}}
\newcommand{\xipar}{\xi}
\newcommand{\Exi}{E_\xi}
\newcommand{\EP}{E_{\text{P}}}
\newcommand{\lP}{\ell_{\text{P}}}
\newcommand{\rzero}{r_0}
\newcommand{\tzero}{t_0}
\newcommand{\Gnat}{G_{\text{nat}}}

% Extended environments
\newtcolorbox{important}[1][]{colback=yellow!10!white,colframe=yellow!50!black,fonttitle=\bfseries,title=Important Insight,#1}
\newtcolorbox{formula}[1][]{colback=blue!5!white,colframe=blue!75!black,fonttitle=\bfseries,title=T0 Prediction,#1}
\newtcolorbox{revolutionary}[1][]{colback=red!5!white,colframe=red!75!black,fonttitle=\bfseries,title=Revolutionary Discovery,#1}
\newtcolorbox{experimental}[1][]{colback=green!5!white,colframe=green!75!black,fonttitle=\bfseries,title=Experimental Consideration,#1}
\newtcolorbox{caution}[1][]{colback=orange!5!white,colframe=orange!75!black,fonttitle=\bfseries,title=Experimental Caution,#1}

% Theorem environments
\newtheorem{principle}{Fundamental Principle}[chapter]
\newtheorem{insight}{Key Insight}[chapter]
\newtheorem{discovery}{Revolutionary Discovery}[chapter]



==================================================

=== Ho_De.tex.preamble ===

\documentclass[12pt,a4paper]{article}
\usepackage[utf8]{inputenc}
\usepackage[T1]{fontenc}
\usepackage[ngerman]{babel}
\usepackage[left=2cm,right=2cm,top=2cm,bottom=2cm]{geometry}
\usepackage{lmodern}
\usepackage{amsmath}
\usepackage{amssymb}
\usepackage{physics}
\usepackage{hyperref}
\usepackage{tcolorbox}
\usepackage{booktabs}
\usepackage{enumitem}
\usepackage[table,xcdraw]{xcolor}
\usepackage{longtable}
\usepackage{siunitx}
\usepackage{fancyhdr}

% Kopf- und Fußzeilen
\pagestyle{fancy}
\fancyhf{}
\fancyhead[L]{Johann Pascher}
\fancyhead[R]{T0-Modell: Die Hubble-Konstante im statischen Universum}
\fancyfoot[C]{\thepage}
\renewcommand{\headrulewidth}{0.4pt}
\renewcommand{\footrulewidth}{0.4pt}

\hypersetup{
	colorlinks=true,
	linkcolor=blue,
	citecolor=blue,
	urlcolor=blue,
	pdftitle={T0-Modell: Die Hubble-Konstante im statischen Universum},
	pdfauthor={Johann Pascher},
	pdfsubject={T0-Modell, Statisches Universum, Hubble-Parameter},
	pdfkeywords={xi-Feld, Energieverlust, T0-Theorie}
}

% Benutzerdefinierte Umgebungen
\newtcolorbox{important}[1][]{colback=yellow!10!white,colframe=yellow!50!black,fonttitle=\bfseries,title=Wichtiger Hinweis,#1}
\newtcolorbox{formula}[1][]{colback=blue!5!white,colframe=blue!75!black,fonttitle=\bfseries,title=Zentrale Formel,#1}
\newtcolorbox{revolutionary}[1][]{colback=red!5!white,colframe=red!75!black,fonttitle=\bfseries,title=Revolutionäre Erkenntnis,#1}
\newtcolorbox{experimental}[1][]{colback=green!5!white,colframe=green!75!black,fonttitle=\bfseries,title=Experimentelle Analyse,#1}



==================================================

=== Ho_En.tex.preamble ===

\documentclass[12pt,a4paper]{article}
\usepackage[utf8]{inputenc}
\usepackage[T1]{fontenc}
\usepackage[english]{babel}
\usepackage[left=2cm,right=2cm,top=2cm,bottom=2cm]{geometry}
\usepackage{lmodern}
\usepackage{amsmath}
\usepackage{amssymb}
\usepackage{physics}
\usepackage{hyperref}
\usepackage{tcolorbox}
\usepackage{booktabs}
\usepackage{enumitem}
\usepackage[table,xcdraw]{xcolor}
\usepackage{longtable}
\usepackage{siunitx}
\usepackage{fancyhdr}

% Header and Footer
\pagestyle{fancy}
\fancyhf{}
\fancyhead[L]{Johann Pascher}
\fancyhead[R]{T0-Model: The Hubble Parameter in Static Universe}
\fancyfoot[C]{\thepage}
\renewcommand{\headrulewidth}{0.4pt}
\renewcommand{\footrulewidth}{0.4pt}

\hypersetup{
	colorlinks=true,
	linkcolor=blue,
	citecolor=blue,
	urlcolor=blue,
	pdftitle={T0-Model: The Hubble Parameter in Static Universe},
	pdfauthor={Johann Pascher},
	pdfsubject={T0-Model, Static Universe, Hubble Parameter},
	pdfkeywords={xi-field, Energy Loss, T0-Theory}
}

% Custom environments
\newtcolorbox{important}[1][]{colback=yellow!10!white,colframe=yellow!50!black,fonttitle=\bfseries,title=Important Note,#1}
\newtcolorbox{formula}[1][]{colback=blue!5!white,colframe=blue!75!black,fonttitle=\bfseries,title=Key Formula,#1}
\newtcolorbox{revolutionary}[1][]{colback=red!5!white,colframe=red!75!black,fonttitle=\bfseries,title=Revolutionary Insight,#1}
\newtcolorbox{experimental}[1][]{colback=green!5!white,colframe=green!75!black,fonttitle=\bfseries,title=Experimental Analysis,#1}



==================================================

=== LagrandianVergleichDe.tex.preamble ===

\documentclass[12pt,a4paper]{article}
\usepackage[utf8]{inputenc}
\usepackage[T1]{fontenc}
\usepackage[ngerman]{babel}
\usepackage{textcomp}
\usepackage{lmodern}
\usepackage{amsmath}
\usepackage{amssymb}
\usepackage{physics}
\usepackage{hyperref}
\usepackage{tcolorbox}
\usepackage{booktabs}
\usepackage{enumitem}
\usepackage[table,xcdraw]{xcolor}
\usepackage[left=2cm,right=2cm,top=2cm,bottom=2cm]{geometry}
\usepackage{pgfplots}
\pgfplotsset{compat=1.18}
\usepackage{graphicx}
\usepackage{float}
\usepackage{fancyhdr}
\usepackage{siunitx}
\usepackage{array}
\usepackage{cleveref}
\usepackage{mathtools}
\usepackage{amsthm}

% Kopf- und Fußzeilen
\pagestyle{fancy}
\fancyhf{}
\fancyhead[L]{Johann Pascher}
\fancyhead[R]{Einfache Lagrange-Funktion vs. Standardmodell}
\fancyfoot[C]{\thepage}
\renewcommand{\headrulewidth}{0.4pt}
\renewcommand{\footrulewidth}{0.4pt}
\setlength{\headheight}{15pt}

% Benutzerdefinierte Befehle
\newcommand{\Lag}{\mathcal{L}}
\newcommand{\deltam}{\delta m}
\newcommand{\xipar}{\xi}

% Theorem-Umgebungen
\newtheorem{theorem}{Theorem}[section]
\newtheorem{proposition}[theorem]{Proposition}
\newtheorem{corollary}[theorem]{Korollar}
\newtheorem{lemma}[theorem]{Lemma}
\theoremstyle{definition}
\newtheorem{definition}[theorem]{Definition}
\newtheorem{example}[theorem]{Beispiel}
\theoremstyle{remark}
\newtheorem{remark}[theorem]{Bemerkung}

\hypersetup{
	colorlinks=true,
	linkcolor=blue,
	citecolor=blue,
	urlcolor=blue,
	pdftitle={Einfache Lagrange-Revolution: Von der Standardmodell-Komplexität zur T0-Eleganz},
	pdfauthor={Johann Pascher},
	pdfsubject={Theoretische Physik},
	pdfkeywords={Standardmodell, Vereinfachte Lagrange-Funktion, Antiteilchen, T0-Theorie}
}

\title{Einfache Lagrange-Revolution: \\
	Von der Standardmodell-Komplexität zur T0-Eleganz \\
	\large Wie eine Gleichung 20+ Felder ersetzt und Antiteilchen erklärt}
\author{Johann Pascher\\
	Abteilung für Nachrichtentechnik, \\Höhere Technische Bundeslehranstalt (HTL), Leonding, Österreich\\
	\texttt{johann.pascher@gmail.com}}
\date{\today}



==================================================

=== LagrandianVergleichEn.tex.preamble ===

\documentclass[12pt,a4paper]{article}
\usepackage[utf8]{inputenc}
\usepackage[T1]{fontenc}
\usepackage[english]{babel}
\usepackage{textcomp}
\usepackage{lmodern}
\usepackage{amsmath}
\usepackage{amssymb}
\usepackage{physics}
\usepackage{hyperref}
\usepackage{tcolorbox}
\usepackage{booktabs}
\usepackage{enumitem}
\usepackage[table,xcdraw]{xcolor}
\usepackage[left=2cm,right=2cm,top=2cm,bottom=2cm]{geometry}
\usepackage{pgfplots}
\pgfplotsset{compat=1.18}
\usepackage{graphicx}
\usepackage{float}
\usepackage{fancyhdr}
\usepackage{siunitx}
\usepackage{array}
\usepackage{cleveref}
\usepackage{mathtools}
\usepackage{amsthm}

% Headers and Footers
\pagestyle{fancy}
\fancyhf{}
\fancyhead[L]{Johann Pascher}
\fancyhead[R]{Simple Lagrangian vs. Standard Model}
\fancyfoot[C]{\thepage}
\renewcommand{\headrulewidth}{0.4pt}
\renewcommand{\footrulewidth}{0.4pt}
\setlength{\headheight}{15pt}

% Custom commands
\newcommand{\Lag}{\mathcal{L}}
\newcommand{\deltam}{\delta m}
\newcommand{\xipar}{\xi}

% Theorem environments
\newtheorem{theorem}{Theorem}[section]
\newtheorem{proposition}[theorem]{Proposition}
\newtheorem{corollary}[theorem]{Corollary}
\newtheorem{lemma}[theorem]{Lemma}
\theoremstyle{definition}
\newtheorem{definition}[theorem]{Definition}
\newtheorem{example}[theorem]{Example}
\theoremstyle{remark}
\newtheorem{remark}[theorem]{Remark}

\hypersetup{
	colorlinks=true,
	linkcolor=blue,
	citecolor=blue,
	urlcolor=blue,
	pdftitle={Simple Lagrangian Revolution: From Standard Model Complexity to T0 Elegance},
	pdfauthor={Johann Pascher},
	pdfsubject={Theoretical Physics},
	pdfkeywords={Standard Model, Simplified Lagrangian, Antiparticles, T0 Theory}
}

\title{Simple Lagrangian Revolution: \\
	From Standard Model Complexity to T0 Elegance \\
	\large How One Equation Replaces 20+ Fields and Explains Antiparticles}
\author{Johann Pascher\\
	Department of Communications Engineering, \\H\"ohere Technische Bundeslehranstalt (HTL), Leonding, Austria\\
	\texttt{johann.pascher@gmail.com}}
\date{\today}



==================================================

=== Markov_De.tex.preamble ===

\documentclass[12pt,a4paper]{article}
\usepackage[utf8]{inputenc}
\usepackage[T1]{fontenc}
\usepackage{geometry}
\usepackage{lmodern}
\usepackage{amsmath}
\usepackage{amssymb}
\usepackage{hyperref}
\usepackage{booktabs}
\usepackage{enumitem}
\usepackage[table,xcdraw]{xcolor}
\usepackage{newunicodechar}
\usepackage[ngerman]{babel} % Für deutsche Trennregeln und Sprache

% Unicode setups for Greek letters and symbols
\newunicodechar{ξ}{\ensuremath{\xi}}
\newunicodechar{μ}{\ensuremath{\mu}}
\newunicodechar{π}{\ensuremath{\pi}}

\geometry{left=2cm,right=2cm,top=2cm,bottom=2cm}

\hypersetup{
	colorlinks=true,
	linkcolor=blue,
	citecolor=blue,
	urlcolor=blue,
	pdftitle={Markov-Ketten im Kontext der T0-Theorie: Deterministisch oder stochastisch? Ein Traktat zu Mustern, Voraussetzungen und Unsicherheit},
	pdfauthor={Johann Pascher},
	pdfsubject={Stochastische Prozesse, Markov-Ketten, T0-Theorie, Determinismus vs. Stochastik}
}

\title{Markov-Ketten im Kontext der T0-Theorie:\\Deterministisch oder stochastisch?\\Ein Traktat zu Mustern, Voraussetzungen und Unsicherheit}
\author{Johann Pascher\\
	Abteilung für Kommunikationstechnik\\
	Höhere Technische Lehranstalt Leonding, Österreich\\
	\texttt{johann.pascher@gmail.com}}
\date{20. Oktober 2025}



==================================================

=== Markov_En.tex.preamble ===

\documentclass[12pt,a4paper]{article}
\usepackage[utf8]{inputenc}
\usepackage[T1]{fontenc}
\usepackage{geometry}
\usepackage{lmodern}
\usepackage{amsmath}
\usepackage{amssymb}
\usepackage{hyperref}
\usepackage{booktabs}
\usepackage{enumitem}
\usepackage[table,xcdraw]{xcolor}
\usepackage{newunicodechar}

% Unicode setups for Greek letters and symbols
\newunicodechar{ξ}{\ensuremath{\xi}}
\newunicodechar{μ}{\ensuremath{\mu}}
\newunicodechar{π}{\ensuremath{\pi}}

\geometry{left=2cm,right=2cm,top=2cm,bottom=2cm}

\hypersetup{
	colorlinks=true,
	linkcolor=blue,
	citecolor=blue,
	urlcolor=blue,
	pdftitle={Markov Chains in the Context of T0 Theory: Deterministic or Stochastic? A Treatise on Patterns, Preconditions, and Uncertainty},
	pdfauthor={Johann Pascher},
	pdfsubject={Stochastic Processes, Markov Chains, T0 Theory, Determinism vs. Stochastics}
}

\title{Markov Chains in the Context of T0 Theory:\\Deterministic or Stochastic?\\A Treatise on Patterns, Preconditions, and Uncertainty}
\author{Johann Pascher\\
	Department of Communications Engineering\\
	Higher Technical Institute, Leonding, Austria\\
	\texttt{johann.pascher@gmail.com}}
\date{20. Oktober 2025}



==================================================

=== MathZeitMasseLagrangeDe.tex.preamble ===

\documentclass[12pt,a4paper]{article}
\usepackage[margin=2cm]{geometry}
\usepackage[utf8]{inputenc}
\usepackage[T1]{fontenc}
\usepackage{lmodern}
\usepackage[ngerman]{babel}
\usepackage{amsmath,amssymb,physics,graphicx,xcolor,amsthm}
\usepackage{hyperref}
\usepackage{booktabs}
\usepackage{siunitx}
\usepackage{cleveref}
\usepackage{pgfplots}
\pgfplotsset{compat=1.18}
\usepackage{tikz}
\usetikzlibrary{intersections}
\usepgfplotslibrary{fillbetween}
\usepackage{fancyhdr}
\usepackage{tcolorbox}
\usepackage{mathtools}

% Benutzerdefinierte Befehle - Aktualisiert für Konsistenz mit T0-Modell-Referenz
\newcommand{\Tfield}{T(x,t)}
\newcommand{\mfield}{m(x,t)}
\newcommand{\betaT}{\beta_{\text{T}}}
\newcommand{\alphaEM}{\alpha_{\text{EM}}}
\newcommand{\Tzero}{T_0}
\newcommand{\DcovT}[1]{\partial_\mu #1 + #1 \partial_\mu \Tfield}
\newcommand{\DhiggsT}{\Tfield (\partial_\mu + ig A_\mu) \Phi + \Phi \partial_\mu \Tfield}
\newcommand{\gammaf}{\gamma_{\text{Lorentz}}}
\newcommand{\xipar}{\xi}
\newcommand{\Lambdat}{\Lambda_T}
\newcommand{\lP}{\ell_{\text{P}}}
\newcommand{\Mpl}{M_{\text{Pl}}}

% Theorem-Stile
\newtheorem{theorem}{Theorem}[section]
\newtheorem{proposition}[theorem]{Proposition}
\newtheorem{corollary}[theorem]{Korollar}
\newtheorem{lemma}[theorem]{Lemma}
\theoremstyle{definition}
\newtheorem{definition}[theorem]{Definition}
\newtheorem{example}[theorem]{Beispiel}
\theoremstyle{remark}
\newtheorem{remark}[theorem]{Bemerkung}

% Hyperref-Konfiguration
\hypersetup{
	colorlinks=true,
	linkcolor=blue,
	urlcolor=blue,
	citecolor=blue,
	pdftitle={Von Zeitdilatation zu Massenvariation: Mathematische Kernformulierungen der Zeit-Masse-Dualitätstheorie - Aktualisiertes Framework},
	pdfauthor={Johann Pascher},
	pdfsubject={Theoretische Physik},
	pdfkeywords={T0-Modell, Zeit-Masse-Dualität, Natürliche Einheiten, Feldtheorie}
}

% Kopf- und Fußzeilen-Konfiguration
\pagestyle{fancy}
\fancyhf{}
\fancyhead[L]{Johann Pascher}
\fancyhead[R]{Mathematische Kernformulierungen - T0-Modell}
\fancyfoot[C]{\thepage}
\renewcommand{\headrulewidth}{0.4pt}
\renewcommand{\footrulewidth}{0.4pt}

\title{Von Zeitdilatation zu Massenvariation: \\ Mathematische Kernformulierungen der Zeit-Masse-Dualitätstheorie \\ \large Aktualisiertes Framework mit vollständigen geometrischen Grundlagen}
\author{Johann Pascher}
\date{\today}



==================================================

=== MathZeitMasseLagrangeEn.tex.preamble ===

\documentclass[12pt,a4paper]{article}
\usepackage[margin=2cm]{geometry}
\usepackage[utf8]{inputenc}
\usepackage[T1]{fontenc}
\usepackage{lmodern}
\usepackage[english]{babel}
\usepackage{amsmath,amssymb,physics,graphicx,xcolor,amsthm}
\usepackage{hyperref}
\usepackage{booktabs}
\usepackage{siunitx}
\usepackage{cleveref}
\usepackage{pgfplots}
\pgfplotsset{compat=1.18}
\usepackage{tikz}
\usetikzlibrary{intersections}
\usepgfplotslibrary{fillbetween}
\usepackage{fancyhdr}
\usepackage{tcolorbox}
\usepackage{mathtools}

% Custom commands - Updated for consistency with T0 model reference
\newcommand{\Tfield}{T(x,t)}
\newcommand{\mfield}{m(x,t)}
\newcommand{\betaT}{\beta_{\text{T}}}
\newcommand{\alphaEM}{\alpha_{\text{EM}}}
\newcommand{\Tzero}{T_0}
\newcommand{\DcovT}[1]{\partial_\mu #1 + #1 \partial_\mu \Tfield}
\newcommand{\DhiggsT}{\Tfield (\partial_\mu + ig A_\mu) \Phi + \Phi \partial_\mu \Tfield}
\newcommand{\gammaf}{\gamma_{\text{Lorentz}}}
\newcommand{\xipar}{\xi}
\newcommand{\Lambdat}{\Lambda_T}
\newcommand{\lP}{\ell_{\text{P}}}
\newcommand{\Mpl}{M_{\text{Pl}}}

% Theorem styles
\newtheorem{theorem}{Theorem}[section]
\newtheorem{proposition}[theorem]{Proposition}
\newtheorem{corollary}[theorem]{Corollary}
\newtheorem{lemma}[theorem]{Lemma}
\theoremstyle{definition}
\newtheorem{definition}[theorem]{Definition}
\newtheorem{example}[theorem]{Example}
\theoremstyle{remark}
\newtheorem{remark}[theorem]{Remark}

% Hyperref configuration
\hypersetup{
	colorlinks=true,
	linkcolor=blue,
	urlcolor=blue,
	citecolor=blue,
	pdftitle={From Time Dilation to Mass Variation: Mathematical Core Formulations of Time-Mass Duality Theory - Updated Framework},
	pdfauthor={Johann Pascher},
	pdfsubject={Theoretical Physics},
	pdfkeywords={T0 Model, Time-Mass Duality, Natural Units, Field Theory}
}

% Header and Footer Configuration
\pagestyle{fancy}
\fancyhf{}
\fancyhead[L]{Johann Pascher}
\fancyhead[R]{Mathematical Core Formulations - T0 Model}
\fancyfoot[C]{\thepage}
\renewcommand{\headrulewidth}{0.4pt}
\renewcommand{\footrulewidth}{0.4pt}

\title{From Time Dilation to Mass Variation: \\ Mathematical Core Formulations of Time-Mass Duality Theory \\ \large Updated Framework with Complete Geometric Foundations}
\author{Johann Pascher}
\date{\today}



==================================================

=== Mathematische_struktur_De.tex.preamble ===

\documentclass[12pt,a4paper]{article}
\usepackage[utf8]{inputenc}
\usepackage[T1]{fontenc}
\usepackage[ngerman]{babel}
\usepackage{lmodern}
\usepackage{amsmath,amssymb,amsthm}
\usepackage{geometry}
\usepackage{booktabs}
\usepackage{array}
\usepackage{xcolor}
\usepackage{tcolorbox}
\usepackage{fancyhdr}
\usepackage{tocloft}
\usepackage{hyperref}
\usepackage{tikz}
\usepackage{physics}

\definecolor{deepblue}{RGB}{0,0,127}
\definecolor{deepred}{RGB}{191,0,0}
\definecolor{deepgreen}{RGB}{0,127,0}



\geometry{a4paper, margin=1in}



\usetikzlibrary{positioning, arrows}
\geometry{a4paper, margin=2.5cm}

% Kopf- und Fußzeilenkonfiguration
\pagestyle{fancy}
\fancyhf{}
\fancyhead[L]{\textsc{Warum Zahlenverhältnisse nicht gekürzt werden dürfen}}
\fancyhead[R]{\textsc{J. Pascher}}
\fancyfoot[C]{\thepage}
\renewcommand{\headrulewidth}{0.4pt}
\renewcommand{\footrulewidth}{0.4pt}

% Inhaltsverzeichnis-Stil - Blau
\renewcommand{\cfttoctitlefont}{\huge\bfseries\color{blue}}
\renewcommand{\cftsecfont}{\color{blue}}
\renewcommand{\cftsubsecfont}{\color{blue}}
\renewcommand{\cftsecpagefont}{\color{blue}}
\renewcommand{\cftsubsecpagefont}{\color{blue}}
\setlength{\cftsecindent}{0.5cm}
\setlength{\cftsubsecindent}{1cm}

% Hyperref-Einstellungen
\hypersetup{
	colorlinks=true,
	linkcolor=blue,
	citecolor=blue,
	urlcolor=blue,
	pdftitle={Warum Zahlenverhältnisse nicht direkt gekürzt werden dürfen},
	pdfauthor={Johann Pascher},
	pdfsubject={T0-Theorie, Geometrische Physik, Fundamentale Konstanten}
}

% Benutzerdefinierte Befehle
\newcommand{\lP}{l_P}
\newcommand{\EP}{E_P}
\newcommand{\tP}{t_P}
\newcommand{\rzero}{r_0}
\newcommand{\tzero}{t_0}
\newcommand{\Ezero}{E_0}
\newcommand{\xipar}{\xi}
\newcommand{\kfrac}{K_{\text{frak}}}

% Umgebung für Schlüsselergebnisse
\newtcolorbox{keyresult}{colback=blue!5, colframe=blue!75!black, title=Schlüsselergebnis}

% Titel
\title{\textbf{Zur mathematischen Struktur der T0-Theorie: \ Warum Zahlenverhältnisse nicht direkt gekürzt werden dürfen}\\[0.5cm]
	\large Aufbau der physikalischen Realität aus reiner Geometrie\\[0.3cm]
	\normalsize Ohne empirische Eingaben}
\author{Johann Pascher\\
	Abteilung für Kommunikationstechnologie\\
	Höhere Technische Lehranstalt (HTL), Leonding, Österreich\\
	\texttt{johann.pascher@gmail.com}}
\date{\today}


==================================================

=== Mathematische_struktur_En.tex.preamble ===

\documentclass[12pt,a4paper]{article}
\usepackage[utf8]{inputenc}
\usepackage[T1]{fontenc}
\usepackage[english]{babel}
\usepackage{lmodern}
\usepackage{amsmath,amssymb,amsthm}
\usepackage{geometry}
\usepackage{booktabs}
\usepackage{array}
\usepackage{xcolor}
\usepackage{tcolorbox}
\usepackage{fancyhdr}
\usepackage{tocloft}
\usepackage{hyperref}
\usepackage{tikz}
\usepackage{physics}

\definecolor{deepblue}{RGB}{0,0,127}
\definecolor{deepred}{RGB}{191,0,0}
\definecolor{deepgreen}{RGB}{0,127,0}

\geometry{a4paper, margin=1in}

\usetikzlibrary{positioning, arrows}
\geometry{a4paper, margin=2.5cm}

% Header and Footer Configuration
\pagestyle{fancy}
\fancyhf{}
\fancyhead[L]{\textsc{Why Numerical Ratios Must Not Be Simplified}}
\fancyhead[R]{\textsc{J. Pascher}}
\fancyfoot[C]{\thepage}
\renewcommand{\headrulewidth}{0.4pt}
\renewcommand{\footrulewidth}{0.4pt}

% Table of Contents Style - Blue
\renewcommand{\cfttoctitlefont}{\huge\bfseries\color{blue}}
\renewcommand{\cftsecfont}{\color{blue}}
\renewcommand{\cftsubsecfont}{\color{blue}}
\renewcommand{\cftsecpagefont}{\color{blue}}
\renewcommand{\cftsubsecpagefont}{\color{blue}}
\setlength{\cftsecindent}{0.5cm}
\setlength{\cftsubsecindent}{1cm}

% Hyperref Settings
\hypersetup{
	colorlinks=true,
	linkcolor=blue,
	citecolor=blue,
	urlcolor=blue,
	pdftitle={Why Numerical Ratios Must Not Be Directly Simplified},
	pdfauthor={Johann Pascher},
	pdfsubject={T0-Theory, Geometric Physics, Fundamental Constants}
}

% Custom Commands
\newcommand{\lP}{l_P}
\newcommand{\EP}{E_P}
\newcommand{\tP}{t_P}
\newcommand{\rzero}{r_0}
\newcommand{\tzero}{t_0}
\newcommand{\Ezero}{E_0}
\newcommand{\xipar}{\xi}
\newcommand{\kfrac}{K_{\text{frak}}}

% Environment for Key Results
\newtcolorbox{keyresult}{colback=blue!5, colframe=blue!75!black, title=Key Result}

% Title
\title{\textbf{On the Mathematical Structure of the T0-Theory: \ Why Numerical Ratios Must Not Be Directly Simplified}\\[0.5cm]
	\large Construction of Physical Reality from Pure Geometry\\[0.3cm]
	\normalsize Without Empirical Inputs}
\author{Johann Pascher\\
	Department of Communications Technology\\
	Higher Technical Institute (HTL), Leonding, Austria\\
	\texttt{johann.pascher@gmail.com}}
\date{\today}



==================================================

=== Moll_CandelaDe.tex.preamble ===

\documentclass[12pt,a4paper]{article}
\usepackage[utf8]{inputenc}
\usepackage[T1]{fontenc}
\usepackage[ngerman]{babel}
\usepackage[left=2cm,right=2cm,top=2cm,bottom=2cm]{geometry}
\usepackage{lmodern}
\usepackage{amsmath}
\usepackage{amssymb}
\usepackage{physics}
\usepackage{tcolorbox}
\usepackage{booktabs}
\usepackage{enumitem}
\usepackage[table,xcdraw]{xcolor}
\usepackage{graphicx}
\usepackage{float}
\usepackage{mathtools}
\usepackage{amsthm}
\usepackage{siunitx}
\usepackage{fancyhdr}
\usepackage{tocloft}
\usepackage{hyperref}
\usepackage{cleveref}

% Kopf- und Fußzeilen
\pagestyle{fancy}
\fancyhf{}
\fancyhead[L]{T0-Modell}
\fancyhead[R]{Universelle Energiebeziehungen: Mol \& Candela}
\fancyfoot[C]{\thepage}
\renewcommand{\headrulewidth}{0.4pt}
\renewcommand{\footrulewidth}{0.4pt}

% Benutzerdefinierte Befehle für T0-Notation
\newcommand{\xipar}{\xi}
\newcommand{\EP}{E_{\text{P}}}
\newcommand{\lP}{\ell_{\text{P}}}
\newcommand{\tP}{t_{\text{P}}}
\newcommand{\Tfield}{T(x,t)}
\newcommand{\rhoE}{\rho_E}
\newcommand{\Echar}{E_{\text{char}}}
\newcommand{\Evis}{E_{\text{vis}}}
\newcommand{\Cto}{C_{\text{T0}}}
\newcommand{\etavis}{\eta_{\text{visual}}}
\newcommand{\Phiphoton}{\Phi_{\text{photon}}}
\newcommand{\checked}{\checkmark}

\hypersetup{
	colorlinks=true,
	linkcolor=blue,
	citecolor=blue,
	urlcolor=blue,
	pdftitle={T0-Modell: Universelle Energiebeziehungen für Mol- und Candela-Einheiten},
	pdfauthor={T0-Modell-Analyse},
	pdfsubject={Energiebasierte Einheitenherleitungen},
	pdfkeywords={T0-Modell, Mol, Candela, Energiebeziehungen, Universelle Skalierung}
}



==================================================

=== Moll_CandelaEn.tex.preamble ===

\documentclass[12pt,a4paper]{article}
\usepackage[utf8]{inputenc}
\usepackage[T1]{fontenc}
\usepackage[english]{babel}
\usepackage[left=2cm,right=2cm,top=2cm,bottom=2cm]{geometry}
\usepackage{lmodern}
\usepackage{amsmath}
\usepackage{amssymb}
\usepackage{physics}
\usepackage{tcolorbox}
\usepackage{booktabs}
\usepackage{enumitem}
\usepackage[table,xcdraw]{xcolor}
\usepackage{graphicx}
\usepackage{float}
\usepackage{mathtools}
\usepackage{amsthm}
\usepackage{siunitx}
\usepackage{fancyhdr}
\usepackage{tocloft}
\usepackage{hyperref}
\usepackage{cleveref}

% Headers and Footers
\pagestyle{fancy}
\fancyhf{}
\fancyhead[L]{T0 Model}
\fancyhead[R]{Universal Energy Relations: Mol \& Candela}
\fancyfoot[C]{\thepage}
\renewcommand{\headrulewidth}{0.4pt}
\renewcommand{\footrulewidth}{0.4pt}

% Custom Commands for T0 notation
\newcommand{\xipar}{\xi}
\newcommand{\EP}{E_{\text{P}}}
\newcommand{\lP}{\ell_{\text{P}}}
\newcommand{\tP}{t_{\text{P}}}
\newcommand{\Tfield}{T(x,t)}
\newcommand{\rhoE}{\rho_E}
\newcommand{\Echar}{E_{\text{char}}}
\newcommand{\Evis}{E_{\text{vis}}}
\newcommand{\Cto}{C_{\text{T0}}}
\newcommand{\etavis}{\eta_{\text{visual}}}
\newcommand{\Phiphoton}{\Phi_{\text{photon}}}
\newcommand{\checked}{\checkmark}

\hypersetup{
	colorlinks=true,
	linkcolor=blue,
	citecolor=blue,
	urlcolor=blue,
	pdftitle={T0 Model: Universal Energy Relations for Mol and Candela Units},
	pdfauthor={T0 Model Analysis},
	pdfsubject={Energy-Based Unit Derivations},
	pdfkeywords={T0 Model, Mol, Candela, Energy Relations, Universal Scaling}
}



==================================================

=== NatEinheitenSystematikDe.tex.preamble ===

\documentclass[11pt,a4paper]{article}
\usepackage[utf8]{inputenc}
\usepackage[T1]{fontenc}
\usepackage[ngerman]{babel}
\usepackage{lmodern}
\usepackage{amsmath}
\usepackage{amssymb}
\usepackage{physics}
\usepackage{hyperref}
\usepackage{booktabs}
\usepackage{array}
\usepackage[left=2cm,right=2cm,top=2cm,bottom=2cm]{geometry}
\usepackage{fancyhdr}
\usepackage{siunitx}
\usepackage{amsthm}
\usepackage{adjustbox}

% Benutzerdefinierte Befehle
\newcommand{\tablescale}{0.9}

% Theorem-Umgebungen
\newtheorem{definition}{Definition}[section]

% Kopf- und Fußzeile
\pagestyle{fancy}
\fancyhf{}
\fancyhead[L]{Johann Pascher}
\fancyhead[R]{Natürliche Einheitensysteme}
\fancyfoot[C]{\thepage}
\renewcommand{\headrulewidth}{0.4pt}
\renewcommand{\footrulewidth}{0.4pt}

\hypersetup{
	colorlinks=true,
	linkcolor=blue,
	citecolor=blue,
	urlcolor=blue,
	pdftitle={Natürliche Einheitensysteme: Universelle Energieumwandlung und fundamentale Längenskala-Hierarchie},
	pdfauthor={Johann Pascher}
}

\title{Natürliche Einheitensysteme:\\
	Universelle Energieumwandlung und\\
	fundamentale Längenskala-Hierarchie}
\author{Johann Pascher\\
	Abteilung für Kommunikationstechnik\\
	Höhere Technische Bundeslehranstalt (HTL), Leonding, Österreich\\
	\texttt{johann.pascher@gmail.com}}
\date{13. April 2025}



==================================================

=== NatEinheitenSystematikEn.tex.preamble ===

\documentclass[11pt,a4paper]{article}
\usepackage[utf8]{inputenc}
\usepackage[T1]{fontenc}
\usepackage[english]{babel}
\usepackage{lmodern}
\usepackage{amsmath}
\usepackage{amssymb}
\usepackage{physics}
\usepackage{hyperref}
\usepackage{booktabs}
\usepackage{array}
\usepackage[left=2cm,right=2cm,top=2cm,bottom=2cm]{geometry}
\usepackage{fancyhdr}
\usepackage{siunitx}
\usepackage{amsthm}
\usepackage{adjustbox}

% Custom Commands
\newcommand{\tablescale}{0.9}

% Theorem environments
\newtheorem{definition}{Definition}[section]

% Header and footer
\pagestyle{fancy}
\fancyhf{}
\fancyhead[L]{Johann Pascher}
\fancyhead[R]{Natural Unit Systems}
\fancyfoot[C]{\thepage}
\renewcommand{\headrulewidth}{0.4pt}
\renewcommand{\footrulewidth}{0.4pt}

\hypersetup{
	colorlinks=true,
	linkcolor=blue,
	citecolor=blue,
	urlcolor=blue,
	pdftitle={Natural Unit Systems: Universal Energy Conversion and Fundamental Length Scale Hierarchy},
	pdfauthor={Johann Pascher}
}

\title{Natural Unit Systems:\\
	Universal Energy Conversion and\\
	Fundamental Length Scale Hierarchy}
\author{Johann Pascher\\
	Department of Communication Technology\\
	Higher Technical Federal Institute (HTL), Leonding, Austria\\
	\texttt{johann.pascher@gmail.com}}
\date{April 13, 2025}



==================================================

=== NoGoDe.tex.preamble ===

\documentclass[12pt,a4paper]{article}
\usepackage[utf8]{inputenc}
\usepackage[T1]{fontenc}
\usepackage[ngerman]{babel}
\usepackage[left=2cm,right=2cm,top=2cm,bottom=2cm]{geometry}
\usepackage{lmodern}
\usepackage{amsmath}
\usepackage{amssymb}
\usepackage{physics}
\usepackage{hyperref}
\usepackage{tcolorbox}
\usepackage{booktabs}
\usepackage{enumitem}
\usepackage[table,xcdraw]{xcolor}
\usepackage{graphicx}
\usepackage{float}
\usepackage{mathtools}
\usepackage{amsthm}
\usepackage{siunitx}
\usepackage{fancyhdr}

% Headers and Footers
\pagestyle{fancy}
\fancyhf{}
\fancyhead[L]{T0-Theorie vs Bells Theorem}
\fancyhead[R]{Theoretische Tiefenanalyse}
\fancyfoot[C]{\thepage}
\renewcommand{\headrulewidth}{0.4pt}
\renewcommand{\footrulewidth}{0.4pt}

% Custom Commands
\newcommand{\Efield}{E}
\newcommand{\xipar}{\xi}

\hypersetup{
	colorlinks=true,
	linkcolor=blue,
	citecolor=blue,
	urlcolor=blue,
	pdftitle={T0-Theorie vs Bells Theorem: Wie deterministische Energiefelder No-Go-Theoreme umgehen},
	pdfauthor={T0 Theoretische Physik Forschung},
	pdfsubject={T0-Theorie, Bells Theorem, No-Go-Theoreme, Quantengrundlagen}
}

\title{T0-Theorie vs Bells Theorem: \\
	Wie deterministische Energiefelder No-Go-Theoreme umgehen \\
	\large Eine kritische Analyse von Superdeterminismus und Messfreiheit}
\author{T0 Theoretische Physik Forschung}
\date{\today}



==================================================

=== NoGoEn.tex.preamble ===

\documentclass[12pt,a4paper]{article}
\usepackage[utf8]{inputenc}
\usepackage[T1]{fontenc}
\usepackage[english]{babel}
\usepackage[left=2cm,right=2cm,top=2cm,bottom=2cm]{geometry}
\usepackage{lmodern}
\usepackage{amsmath}
\usepackage{amssymb}
\usepackage{physics}
\usepackage{hyperref}
\usepackage{tcolorbox}
\usepackage{booktabs}
\usepackage{enumitem}
\usepackage[table,xcdraw]{xcolor}
\usepackage{graphicx}
\usepackage{float}
\usepackage{mathtools}
\usepackage{amsthm}
\usepackage{siunitx}
\usepackage{fancyhdr}

% Headers and Footers
\pagestyle{fancy}
\fancyhf{}
\fancyhead[L]{T0 Theory vs Bell's Theorem}
\fancyhead[R]{Theoretical Deep Dive}
\fancyfoot[C]{\thepage}
\renewcommand{\headrulewidth}{0.4pt}
\renewcommand{\footrulewidth}{0.4pt}

% Custom Commands
\newcommand{\Efield}{E}
\newcommand{\xipar}{\xi}

\hypersetup{
	colorlinks=true,
	linkcolor=blue,
	citecolor=blue,
	urlcolor=blue,
	pdftitle={T0 Theory vs Bell's Theorem: How Deterministic Energy Fields Circumvent No-Go Theorems},
	pdfauthor={T0 Theoretical Physics Research},
	pdfsubject={T0 Theory, Bell's Theorem, No-Go Theorems, Quantum Foundations}
}

\title{T0 Theory vs Bell's Theorem: \\
	How Deterministic Energy Fields Circumvent No-Go Theorems \\
	\large A Critical Analysis of Superdeterminism and Measurement Freedom}
\author{T0 Theoretical Physics Research}
\date{\today}



==================================================

=== Notwendigkeit_zwei_lagrange_De.tex.preamble ===

\documentclass[12pt,a4paper]{article}
\usepackage[utf8]{inputenc}
\usepackage[ngerman]{babel}
\usepackage{amsmath}
\usepackage{amssymb}
\usepackage{physics}
\usepackage{geometry}
\geometry{margin=2.5cm}
\usepackage{tcolorbox}
\tcbuselibrary{skins,breakable}

% Farbboxen für verschiedene Konzepte
\newtcolorbox{t0box}[1][]{
	colback=blue!5!white,
	colframe=blue!75!black,
	title=#1,
	breakable
}

\newtcolorbox{smbox}[1][]{
	colback=green!5!white,
	colframe=green!75!black,
	title=#1,
	breakable
}

\title{Die Notwendigkeit zweier Lagrange-Formulierungen:\\
	Vereinfachte T0-Theorie und erweiterte Standard-Modell Darstellungen\\
	\large Mit dem universellen Zeitfeld und $\xi$-Parameter}
\author{Basierend auf den T0-Modell Entwicklungen von Johann Pascher}
\date{\today}



==================================================

=== Notwendigkeit_zwei_lagrange_En.tex.preamble ===

\documentclass[12pt,a4paper]{article}
\usepackage[utf8]{inputenc}
\usepackage[english]{babel}
\usepackage{amsmath}
\usepackage{amssymb}
\usepackage{physics}
\usepackage{geometry}
\geometry{margin=2.5cm}
\usepackage{tcolorbox}
\tcbuselibrary{skins,breakable}

% Color boxes for different concepts
\newtcolorbox{t0box}[1][]{
	colback=blue!5!white,
	colframe=blue!75!black,
	title=#1,
	breakable
}

\newtcolorbox{smbox}[1][]{
	colback=green!5!white,
	colframe=green!75!black,
	title=#1,
	breakable
}

\title{The Necessity of Two Lagrangian Formulations:\\
	Simplified T0-Theory and Extended Standard Model Descriptions\\
	\large With Universal Time Field and $\xi$-Parameter}
\author{Based on T0-Model Developments by Johann Pascher}
\date{\today}



==================================================

=== ParameterSystemdipendentDe.tex.preamble ===

\documentclass[12pt,a4paper]{article}
\usepackage[utf8]{inputenc}
\usepackage[T1]{fontenc}
\usepackage[ngerman]{babel}
\usepackage[left=2cm,right=2cm,top=2cm,bottom=2cm]{geometry}
\usepackage{lmodern}
\usepackage{amsmath}
\usepackage{amssymb}
\usepackage{physics}
\usepackage{hyperref}
\usepackage{tcolorbox}
\usepackage{booktabs}
\usepackage{enumitem}
\usepackage[table,xcdraw]{xcolor}
\usepackage{pgfplots}
\pgfplotsset{compat=1.18}
\usepackage{graphicx}
\usepackage{float}
\usepackage{mathtools}
\usepackage{amsthm}
\usepackage{cleveref}
\usepackage{siunitx}
\usepackage{fancyhdr}
\usepackage{tocloft}

% Kopf- und Fußzeilen
\pagestyle{fancy}
\fancyhf{}
\fancyhead[L]{Johann Pascher}
\fancyhead[R]{Parameter-Systemabhängigkeit T0-Modell}
\fancyfoot[C]{\thepage}
\renewcommand{\headrulewidth}{0.4pt}
\renewcommand{\footrulewidth}{0.4pt}

% Inhaltsverzeichnis-Formatierung
\renewcommand{\cftsecfont}{\color{blue}}
\renewcommand{\cftsubsecfont}{\color{blue}}
\renewcommand{\cftsecpagefont}{\color{blue}}
\renewcommand{\cftsubsecpagefont}{\color{blue}}
\setlength{\cftsecindent}{1cm}
\setlength{\cftsubsecindent}{2cm}

\hypersetup{
	colorlinks=true,
	linkcolor=blue,
	citecolor=blue,
	urlcolor=blue,
	pdftitle={Parameter-Systemabhängigkeit im T0-Modell: SI- vs. natürliche Einheiten},
	pdfauthor={Johann Pascher},
	pdfsubject={T0-Modell, Einheitensysteme, Parameter-Transformation},
	pdfkeywords={Natürliche Einheiten, SI-Einheiten, Parameter-Abhängigkeit, T0-Modell}
}

% Benutzerdefinierte Befehle - KORRIGIERT um doppelte Indizes zu vermeiden
\newcommand{\xipar}{\xi}
\newcommand{\lP}{\ell_{\text{P}}}
\newcommand{\tP}{t_{\text{P}}}
\newcommand{\EP}{E_{\text{P}}}
\newcommand{\lambdah}{\lambda_h}
\newcommand{\epsilonzero}{\varepsilon_0}
\newcommand{\Rzero}{R_\infty}
\newcommand{\pichar}{\pi}

% Spezifische systemabhängige Befehle zur Vermeidung von Verwirrung
\newcommand{\alphaEMSI}{\alpha_{\text{EM,SI}}}
\newcommand{\alphaEMnat}{\alpha_{\text{EM,nat}}}
\newcommand{\betaTSI}{\beta_{\text{T,SI}}}
\newcommand{\betaTnat}{\beta_{\text{T,nat}}}
\newcommand{\alphaWSI}{\alpha_{\text{W,SI}}}
\newcommand{\alphaWnat}{\alpha_{\text{W,nat}}}

\newtheorem{theorem}{Theorem}[section]
\newtheorem{proposition}[theorem]{Proposition}
\newtheorem{definition}[theorem]{Definition}
\newtheorem{warning}[theorem]{Warnung}



==================================================

=== ParameterSystemdipendentEn.tex.preamble ===

\documentclass[12pt,a4paper]{article}
\usepackage[utf8]{inputenc}
\usepackage[T1]{fontenc}
\usepackage[english]{babel}
\usepackage[left=2cm,right=2cm,top=2cm,bottom=2cm]{geometry}
\usepackage{lmodern}
\usepackage{amsmath}
\usepackage{amssymb}
\usepackage{physics}
\usepackage{hyperref}
\usepackage{tcolorbox}
\usepackage{booktabs}
\usepackage{enumitem}
\usepackage[table,xcdraw]{xcolor}
\usepackage{pgfplots}
\pgfplotsset{compat=1.18}
\usepackage{graphicx}
\usepackage{float}
\usepackage{mathtools}
\usepackage{amsthm}
\usepackage{cleveref}
\usepackage{siunitx}
\usepackage{fancyhdr}
\usepackage{tocloft}

% Headers and Footers
\pagestyle{fancy}
\fancyhf{}
\fancyhead[L]{Johann Pascher}
\fancyhead[R]{Parameter System-Dependency T0-Model}
\fancyfoot[C]{\thepage}
\renewcommand{\headrulewidth}{0.4pt}
\renewcommand{\footrulewidth}{0.4pt}

% Table of Contents Styling
\renewcommand{\cftsecfont}{\color{blue}}
\renewcommand{\cftsubsecfont}{\color{blue}}
\renewcommand{\cftsecpagefont}{\color{blue}}
\renewcommand{\cftsubsecpagefont}{\color{blue}}
\setlength{\cftsecindent}{1cm}
\setlength{\cftsubsecindent}{2cm}

\hypersetup{
	colorlinks=true,
	linkcolor=blue,
	citecolor=blue,
	urlcolor=blue,
	pdftitle={Parameter System-Dependency in T0-Model: SI vs. Natural Units},
	pdfauthor={Johann Pascher},
	pdfsubject={T0 Model, Unit Systems, Parameter Transformation},
	pdfkeywords={Natural Units, SI Units, Parameter Dependency, T0 Model}
}

% Custom Commands - CORRECTED to avoid double subscript errors
\newcommand{\xipar}{\xi}
\newcommand{\lP}{\ell_{\text{P}}}
\newcommand{\tP}{t_{\text{P}}}
\newcommand{\EP}{E_{\text{P}}}
\newcommand{\lambdah}{\lambda_h}
\newcommand{\epsilonzero}{\varepsilon_0}
\newcommand{\Rzero}{R_\infty}
\newcommand{\pichar}{\pi}

% Specific system-dependent commands to avoid confusion
\newcommand{\alphaEMSI}{\alpha_{\text{EM,SI}}}
\newcommand{\alphaEMnat}{\alpha_{\text{EM,nat}}}
\newcommand{\betaTSI}{\beta_{\text{T,SI}}}
\newcommand{\betaTnat}{\beta_{\text{T,nat}}}
\newcommand{\alphaWSI}{\alpha_{\text{W,SI}}}
\newcommand{\alphaWnat}{\alpha_{\text{W,nat}}}

\newtheorem{theorem}{Theorem}[section]
\newtheorem{proposition}[theorem]{Proposition}
\newtheorem{definition}[theorem]{Definition}
\newtheorem{warning}[theorem]{Warning}



==================================================

=== QFT_De.tex.preamble ===

\documentclass[12pt,a4paper]{article}
\usepackage[utf8]{inputenc}
\usepackage[T1]{fontenc}
\usepackage[german]{babel}
\usepackage[left=2cm,right=2cm,top=2cm,bottom=2cm]{geometry}
\usepackage{lmodern}
\usepackage{amsmath}
\usepackage{amssymb}
\usepackage{physics}
\usepackage{cancel}
\usepackage{slashed}
\usepackage{hyperref}
\usepackage{tcolorbox}
\usepackage{booktabs}
\usepackage{enumitem}
\usepackage[table,xcdraw]{xcolor}
\usepackage{graphicx}
\usepackage{float}
\usepackage{mathtools}
\usepackage{amsthm}
\usepackage{siunitx}
\usepackage{fancyhdr}
\usepackage{longtable}
\usepackage{array}
\usepackage{multirow}
\usepackage{tikz}
\usetikzlibrary{positioning, shapes.geometric, arrows.meta}
\usepackage{microtype}

% TCOLORBOX Bibliotheken
\tcbuselibrary{theorems,skins,breakable}

% Korrekte Header-Höhe setzen
\setlength{\headheight}{14.49998pt}

% Headers and Footers
\pagestyle{fancy}
\fancyhf{}
\fancyhead[L]{T0 Deterministic Quantum Computing}
\fancyhead[R]{Complete Algorithm Analysis}
\fancyfoot[C]{\thepage}
\renewcommand{\headrulewidth}{0.4pt}
\renewcommand{\footrulewidth}{0.4pt}

% Custom Commands
\newcommand{\Efield}{E}
\newcommand{\xipar}{\xi}
\newcommand{\LCDM}{\Lambda\text{CDM}}
\newcommand{\OmegaLambda}{\Omega_{\Lambda}}
\newcommand{\OmegaDM}{\Omega_{\text{DM}}}
\newcommand{\Omegab}{\Omega_b}
\newcommand{\natunits}{\text{(nat. Einh.)}}
\newcommand{\GeV}{\,\text{GeV}}
\newcommand{\MeV}{\,\text{MeV}}
\newcommand{\eV}{\,\text{eV}}
\newcommand{\mh}{m_h}
\newcommand{\vh}{v}
\newcommand{\lambdah}{\lambda_h}
\newcommand{\gammamu}{\gamma^\mu}
\newcommand{\slashp}{\cancel{p}}
\newcommand{\slashk}{\cancel{k}}
\newcommand{\slashq}{\cancel{q}}
\newcommand{\Czero}{C_0}
\newcommand{\Bzero}{B_0}
\newcommand{\MSbar}{\overline{\text{MS}}}

% Theorem-Umgebungen
\theoremstyle{definition}
\newtheorem{definition}{Definition}[section]
\newtheorem{theorem}{Theorem}[section]

% ALLE ERFORDERLICHEN TCOLORBOX-UMGEBUNGEN
\newtcolorbox{wichtig}[1][]{
	colback=yellow!10!white,
	colframe=yellow!50!black,
	fonttitle=\bfseries,
	title=Wichtige Erkenntnis,
	breakable,
	#1
}

\newtcolorbox{formel}[1][]{
	colback=blue!5!white,
	colframe=blue!75!black,
	fonttitle=\bfseries,
	title=Zentrale Formel,
	breakable,
	#1
}

\newtcolorbox{pvbox}[1][]{
	colback=green!5!white,
	colframe=green!75!black,
	fonttitle=\bfseries,
	title=Passarino-Veltman Zerlegung,
	breakable,
	#1
}

\newtcolorbox{numerisch}[1][]{
	colback=orange!5!white,
	colframe=orange!75!black,
	fonttitle=\bfseries,
	title=Numerische Auswertung,
	breakable,
	#1
}

\hypersetup{
	colorlinks=true,
	linkcolor=blue,
	citecolor=blue,
	urlcolor=blue,
	pdftitle={Vollständige Herleitung der Higgs-Masse und Wilson-Koeffizienten},
	pdfauthor={Johann Pascher},
	pdfsubject={T0-Model, Higgs Physics, Wilson Coefficients},
	pdfkeywords={Higgs mass, Wilson coefficients, T0 theory, EFT matching}
}

\title{Vollständige Herleitung der Higgs-Masse und Wilson-Koeffizienten:\\Von fundamentalen Loop-Integralen zu experimentell testbaren Vorhersagen\\
	\large Systematische Quantenfeldtheorie}
\author{Johann Pascher\\
	Department of Communication Technology\\
	Higher Technical Federal Institute (HTL), Leonding, Austria\\
	\texttt{johann.pascher@gmail.com}}
\date{\today}



==================================================

=== QFT_En.tex.preamble ===

\documentclass[12pt,a4paper]{article}
\usepackage[utf8]{inputenc}
\usepackage[T1]{fontenc}
\usepackage[english]{babel}
\usepackage[left=2cm,right=2cm,top=2cm,bottom=2cm]{geometry}
\usepackage{lmodern}
\usepackage{amsmath}
\usepackage{amssymb}
\usepackage{physics}
\usepackage{cancel}
\usepackage{slashed}
\usepackage{hyperref}
\usepackage{tcolorbox}
\usepackage{booktabs}
\usepackage{enumitem}
\usepackage[table,xcdraw]{xcolor}
\usepackage{graphicx}
\usepackage{float}
\usepackage{mathtools}
\usepackage{amsthm}
\usepackage{siunitx}
\usepackage{fancyhdr}
\usepackage{longtable}
\usepackage{array}
\usepackage{multirow}
\usepackage{tikz}
\usetikzlibrary{positioning, shapes.geometric, arrows.meta}
\usepackage{microtype}

% TCOLORBOX Libraries
\tcbuselibrary{theorems,skins,breakable}

% Correct header height setting
\setlength{\headheight}{14.49998pt}

% Headers and Footers
\pagestyle{fancy}
\fancyhf{}
\fancyhead[L]{T0 Deterministic Quantum Computing}
\fancyhead[R]{Complete Algorithm Analysis}
\fancyfoot[C]{\thepage}
\renewcommand{\headrulewidth}{0.4pt}
\renewcommand{\footrulewidth}{0.4pt}

% Custom Commands
\newcommand{\Efield}{E}
\newcommand{\xipar}{\xi}
\newcommand{\LCDM}{\Lambda\text{CDM}}
\newcommand{\OmegaLambda}{\Omega_{\Lambda}}
\newcommand{\OmegaDM}{\Omega_{\text{DM}}}
\newcommand{\Omegab}{\Omega_b}
\newcommand{\natunits}{\text{(nat. units)}}
\newcommand{\GeV}{\,\text{GeV}}
\newcommand{\MeV}{\,\text{MeV}}
\newcommand{\eV}{\,\text{eV}}
\newcommand{\mh}{m_h}
\newcommand{\vh}{v}
\newcommand{\lambdah}{\lambda_h}
\newcommand{\gammamu}{\gamma^\mu}
\newcommand{\slashp}{\cancel{p}}
\newcommand{\slashk}{\cancel{k}}
\newcommand{\slashq}{\cancel{q}}
\newcommand{\Czero}{C_0}
\newcommand{\Bzero}{B_0}
\newcommand{\MSbar}{\overline{\text{MS}}}

% Theorem Environments
\theoremstyle{definition}
\newtheorem{definition}{Definition}[section]
\newtheorem{theorem}{Theorem}[section]

% ALL REQUIRED TCOLORBOX ENVIRONMENTS
\newtcolorbox{important}[1][]{
	colback=yellow!10!white,
	colframe=yellow!50!black,
	fonttitle=\bfseries,
	title=Important Insight,
	breakable,
	#1
}

\newtcolorbox{formula}[1][]{
	colback=blue!5!white,
	colframe=blue!75!black,
	fonttitle=\bfseries,
	title=Central Formula,
	breakable,
	#1
}

\newtcolorbox{pvbox}[1][]{
	colback=green!5!white,
	colframe=green!75!black,
	fonttitle=\bfseries,
	title=Passarino-Veltman Decomposition,
	breakable,
	#1
}

\newtcolorbox{numerical}[1][]{
	colback=orange!5!white,
	colframe=orange!75!black,
	fonttitle=\bfseries,
	title=Numerical Evaluation,
	breakable,
	#1
}

\hypersetup{
	colorlinks=true,
	linkcolor=blue,
	citecolor=blue,
	urlcolor=blue,
	pdftitle={Complete Derivation of Higgs Mass and Wilson Coefficients},
	pdfauthor={Johann Pascher},
	pdfsubject={T0-Model, Higgs Physics, Wilson Coefficients},
	pdfkeywords={Higgs mass, Wilson coefficients, T0 theory, EFT matching}
}

\title{Complete Derivation of Higgs Mass and Wilson Coefficients:\\From Fundamental Loop Integrals to Experimentally Testable Predictions\\
	\large Systematic Quantum Field Theory}
\author{Johann Pascher\\
	Department of Communication Technology\\
	Higher Technical Federal Institute (HTL), Leonding, Austria\\
	\texttt{johann.pascher@gmail.com}}
\date{\today}



==================================================

=== QM-DetrmisticDe.tex.preamble ===

\documentclass[12pt,a4paper]{article}
\usepackage[utf8]{inputenc}
\usepackage[T1]{fontenc}
\usepackage[ngerman]{babel}
\usepackage[left=2cm,right=2cm,top=2cm,bottom=2cm]{geometry}
\usepackage{lmodern}
\usepackage{amsmath}
\usepackage{amssymb}
\usepackage{physics}
\usepackage{hyperref}
\usepackage{tcolorbox}
\usepackage{booktabs}
\usepackage{enumitem}
\usepackage[table,xcdraw]{xcolor}
\usepackage{graphicx}
\usepackage{float}
\usepackage{mathtools}
\usepackage{amsthm}
\usepackage{siunitx}
\usepackage{fancyhdr}
\usepackage{microtype}

% Kopf- und Fusszeilen
\pagestyle{fancy}
\fancyhf{}
\fancyhead[L]{Johann Pascher}
\fancyhead[R]{Deterministische Quantenmechanik via T0-Energiefelder}
\fancyfoot[C]{\thepage}
\renewcommand{\headrulewidth}{0.4pt}
\renewcommand{\footrulewidth}{0.4pt}

% Einfache Befehle ohne Subscripts
\newcommand{\xipar}{\xi}

\hypersetup{
	colorlinks=true,
	linkcolor=blue,
	citecolor=blue,
	urlcolor=blue,
	pdftitle={Deterministische Quantenmechanik via T0-Energiefeld-Formulierung},
	pdfauthor={Johann Pascher},
	pdfsubject={T0-Modell, Deterministische QM, Energiefeld-Physik}
}

\newtheorem{theorem}{Theorem}[section]
\newtheorem{proposition}[theorem]{Proposition}
\newtheorem{definition}[theorem]{Definition}



==================================================

=== QM-DetrmisticEn.tex.preamble ===

\documentclass[12pt,a4paper]{article}
\usepackage[utf8]{inputenc}
\usepackage[T1]{fontenc}
\usepackage[english]{babel}
\usepackage[left=2cm,right=2cm,top=2cm,bottom=2cm]{geometry}
\usepackage{lmodern}
\usepackage{amsmath}
\usepackage{amssymb}
\usepackage{physics}
\usepackage{hyperref}
\usepackage{tcolorbox}
\usepackage{booktabs}
\usepackage{enumitem}
\usepackage[table,xcdraw]{xcolor}
\usepackage{graphicx}
\usepackage{float}
\usepackage{mathtools}
\usepackage{amsthm}
\usepackage{siunitx}
\usepackage{fancyhdr}
\usepackage{microtype}

% Headers and footers
\pagestyle{fancy}
\fancyhf{}
\fancyhead[L]{Johann Pascher}
\fancyhead[R]{Deterministic Quantum Mechanics via T0-Energy Fields}
\fancyfoot[C]{\thepage}
\renewcommand{\headrulewidth}{0.4pt}
\renewcommand{\footrulewidth}{0.4pt}

% Simple commands without subscripts
\newcommand{\xipar}{\xi}

\hypersetup{
	colorlinks=true,
	linkcolor=blue,
	citecolor=blue,
	urlcolor=blue,
	pdftitle={Deterministic Quantum Mechanics via T0-Energy Field Formulation},
	pdfauthor={Johann Pascher},
	pdfsubject={T0 Model, Deterministic QM, Energy Field Physics}
}

\newtheorem{theorem}{Theorem}[section]
\newtheorem{proposition}[theorem]{Proposition}
\newtheorem{definition}[theorem]{Definition}



==================================================

=== QM-Detrmistic_p_De.tex.preamble ===

\documentclass[12pt,a4paper]{article}
\usepackage[utf8]{inputenc}
\usepackage[T1]{fontenc}
\usepackage[ngerman]{babel}
\usepackage{textcomp}
\usepackage{lmodern}
\usepackage{amsmath}
\usepackage{amssymb}
\usepackage{physics}
\usepackage{hyperref}
\usepackage{tcolorbox}
\usepackage{booktabs}
\usepackage{enumitem}
\usepackage[table,xcdraw]{xcolor}
\usepackage[left=2cm,right=2cm,top=2cm,bottom=2cm]{geometry}
\usepackage{pgfplots}
\pgfplotsset{compat=1.18}
\usepackage{graphicx}
\usepackage{float}
\usepackage{fancyhdr}
\usepackage{siunitx}
\usepackage{array}
\usepackage{cleveref}
\usepackage{mathtools}
\usepackage{amsthm}

% Kopf- und Fu{\ss}zeilen
\pagestyle{fancy}
\fancyhf{}
\fancyhead[L]{Johann Pascher}
\fancyhead[R]{T0 Quantenmechanik}
\fancyfoot[C]{\thepage}
\renewcommand{\headrulewidth}{0.4pt}
\renewcommand{\footrulewidth}{0.4pt}

% Benutzerdefinierte Befehle
\newcommand{\Tfield}{T_{\text{field}}(x,t)}
\newcommand{\Efield}{E_{\text{field}}(x,t)}
\newcommand{\deltaE}{\delta E}
\newcommand{\Lag}{\mathcal{L}}
\newcommand{\xipar}{\xi}
\newcommand{\EPlanck}{E_P}

% Theorem-Umgebungen
\newtheorem{theorem}{Theorem}[section]
\newtheorem{proposition}[theorem]{Proposition}
\newtheorem{corollary}[theorem]{Korollar}
\newtheorem{lemma}[theorem]{Lemma}
\theoremstyle{definition}
\newtheorem{definition}[theorem]{Definition}
\newtheorem{example}[theorem]{Beispiel}
\theoremstyle{remark}
\newtheorem{remark}[theorem]{Bemerkung}

\hypersetup{
colorlinks=true,
linkcolor=blue,
citecolor=blue,
urlcolor=blue,
pdftitle={Quantenmechanik im T0-Modell: Feldtheoretische Grundlagen},
pdfauthor={Johann Pascher},
pdfsubject={Quantenmechanik, T0-Theorie},
pdfkeywords={T0-Modell, Quantenmechanik, Feldtheorie, Schr{\"o}dinger-Gleichung, Dirac-Gleichung}
}

\title{Quantenmechanik im T0-Modell: \\
Feldtheoretische Grundlagen \\
\large Von der Standard-QM zu dynamischen Zeit-Energie-Feldern}
\author{Johann Pascher\\
Abteilung f{\"u}r Nachrichtentechnik, \\H{\"o}here Technische Bundeslehranstalt (HTL), Leonding, {\"O}sterreich\\
\texttt{johann.pascher@gmail.com}}
\date{\today}



==================================================

=== QM-Detrmistic_p_En.tex.preamble ===

\documentclass[12pt,a4paper]{article}
\usepackage[utf8]{inputenc}
\usepackage[T1]{fontenc}
\usepackage[english]{babel}
\usepackage{textcomp}
\usepackage{lmodern}
\usepackage{amsmath}
\usepackage{amssymb}
\usepackage{physics}
\usepackage{hyperref}
\usepackage{tcolorbox}
\usepackage{booktabs}
\usepackage{enumitem}
\usepackage[table,xcdraw]{xcolor}
\usepackage[left=2cm,right=2cm,top=2cm,bottom=2cm]{geometry}
\usepackage{pgfplots}
\pgfplotsset{compat=1.18}
\usepackage{graphicx}
\usepackage{float}
\usepackage{fancyhdr}
\usepackage{siunitx}
\usepackage{array}
\usepackage{cleveref}
\usepackage{mathtools}
\usepackage{amsthm}

% Headers and footers
\pagestyle{fancy}
\fancyhf{}
\fancyhead[L]{Johann Pascher}
\fancyhead[R]{T0 Quantum Mechanics}
\fancyfoot[C]{\thepage}
\renewcommand{\headrulewidth}{0.4pt}
\renewcommand{\footrulewidth}{0.4pt}

% Custom commands
\newcommand{\Tfield}{T_{\text{field}}(x,t)}
\newcommand{\Efield}{E_{\text{field}}(x,t)}
\newcommand{\deltaE}{\delta E}
\newcommand{\Lag}{\mathcal{L}}
\newcommand{\xipar}{\xi}
\newcommand{\EPlanck}{E_P}

% Theorem environments
\newtheorem{theorem}{Theorem}[section]
\newtheorem{proposition}[theorem]{Proposition}
\newtheorem{corollary}[theorem]{Corollary}
\newtheorem{lemma}[theorem]{Lemma}
\theoremstyle{definition}
\newtheorem{definition}[theorem]{Definition}
\newtheorem{example}[theorem]{Example}
\theoremstyle{remark}
\newtheorem{remark}[theorem]{Remark}

\hypersetup{
	colorlinks=true,
	linkcolor=blue,
	citecolor=blue,
	urlcolor=blue,
	pdftitle={Quantum Mechanics in the T0 Model: Field-Theoretic Foundations},
	pdfauthor={Johann Pascher},
	pdfsubject={Quantum Mechanics, T0 Theory},
	pdfkeywords={T0 Model, Quantum Mechanics, Field Theory, Schrödinger Equation, Dirac Equation}
}

\title{Quantum Mechanics in the T0 Model: \\
	Field-Theoretic Foundations \\
	\large From Standard QM to Dynamic Time-Energy Fields}
\author{Johann Pascher\\
	Department of Communications Engineering, \\Higher Technical Federal Institute (HTL), Leonding, Austria\\
	\texttt{johann.pascher@gmail.com}}
\date{\today}



==================================================

=== QM-testenDe.tex.preamble ===

\documentclass[12pt,a4paper]{article}
\usepackage[utf8]{inputenc}
\usepackage[T1]{fontenc}
\usepackage[ngerman]{babel}
\usepackage[left=2cm,right=2cm,top=2cm,bottom=2cm]{geometry}
\usepackage{lmodern}
\usepackage{amsmath}
\usepackage{amssymb}
\usepackage{physics}
\usepackage{hyperref}
\usepackage{tcolorbox}
\usepackage{booktabs}
\usepackage{enumitem}
\usepackage[table,xcdraw]{xcolor}
\usepackage{graphicx}
\usepackage{float}
\usepackage{mathtools}
\usepackage{amsthm}
\usepackage{siunitx}
\usepackage{fancyhdr}
\usepackage{longtable}

% Headers and Footers
\pagestyle{fancy}
\fancyhf{}
\fancyhead[L]{T0 Deterministisches Quantencomputing}
\fancyhead[R]{Vollständige Algorithmusanalyse}
\fancyfoot[C]{\thepage}
\renewcommand{\headrulewidth}{0.4pt}
\renewcommand{\footrulewidth}{0.4pt}

% Custom Commands
\newcommand{\Efield}{E}

\hypersetup{
	colorlinks=true,
	linkcolor=blue,
	citecolor=blue,
	urlcolor=blue,
	pdftitle={T0 Deterministisches Quantencomputing: Vollständige Analyse wichtiger Algorithmen},
	pdfauthor={T0 Quantencomputing Forschung},
	pdfsubject={T0-Theorie, Deterministisches Quantencomputing, Algorithmusanalyse}
}

\title{T0 Deterministisches Quantencomputing: \\
	Vollständige Analyse wichtiger Algorithmen \\
	\large Von Deutsch bis Shor: Energiefeld-Formulierung vs. Standard-QM \\
	\textbf{Aktualisiert mit Higgs-$\xi$-Kopplungsanalyse}}
\author{T0 Quantencomputing Forschung}
\date{\today}



==================================================

=== QM-testenEn.tex.preamble ===

\documentclass[12pt,a4paper]{article}
\usepackage[utf8]{inputenc}
\usepackage[T1]{fontenc}
\usepackage[english]{babel}
\usepackage[left=2cm,right=2cm,top=2cm,bottom=2cm]{geometry}
\usepackage{lmodern}
\usepackage{amsmath}
\usepackage{amssymb}
\usepackage{physics}
\usepackage{hyperref}
\usepackage{tcolorbox}
\usepackage{booktabs}
\usepackage{enumitem}
\usepackage[table,xcdraw]{xcolor}
\usepackage{graphicx}
\usepackage{float}
\usepackage{mathtools}
\usepackage{amsthm}
\usepackage{siunitx}
\usepackage{fancyhdr}
\usepackage{longtable}

% Headers and Footers
\pagestyle{fancy}
\fancyhf{}
\fancyhead[L]{T0 Deterministic Quantum Computing}
\fancyhead[R]{Complete Algorithm Analysis}
\fancyfoot[C]{\thepage}
\renewcommand{\headrulewidth}{0.4pt}
\renewcommand{\footrulewidth}{0.4pt}

% Custom Commands
\newcommand{\Efield}{E}

\hypersetup{
	colorlinks=true,
	linkcolor=blue,
	citecolor=blue,
	urlcolor=blue,
	pdftitle={T0 Deterministic Quantum Computing: Complete Analysis of Important Algorithms},
	pdfauthor={T0 Quantum Computing Research},
	pdfsubject={T0-Theory, Deterministic Quantum Computing, Algorithm Analysis}
}

\title{T0 Deterministic Quantum Computing: \\
	Complete Analysis of Important Algorithms \\
	\large From Deutsch to Shor: Energy Field Formulation vs. Standard QM \\
	\textbf{Updated with Higgs-$\xi$ Coupling Analysis}}
\author{T0 Quantum Computing Research}
\date{\today}



==================================================

=== QM_De.tex.preamble ===

\documentclass{article}
\usepackage[utf8]{inputenc}
\usepackage[german]{babel}
\usepackage{amsmath}
\usepackage{amssymb}
\usepackage{booktabs}
\usepackage{geometry}
\geometry{margin=1in}
\usepackage{hyperref}
\usepackage{array}

\title{T0-Theorie: Zusammenfassung der Erkenntnisse (Stand: November 03, 2025)}
\author{}
\date{}



==================================================

=== QM_En.tex.preamble ===

\documentclass{article}
\usepackage[utf8]{inputenc}
\usepackage[english]{babel}
\usepackage{amsmath}
\usepackage{amssymb}
\usepackage{booktabs}
\usepackage{geometry}
\geometry{margin=1in}
\usepackage{hyperref}
\usepackage{array}

\title{T0 Theory: Summary of Findings (Status: November 03, 2025)}
\author{}
\date{}



==================================================

=== RSA_De.tex.preamble ===

\documentclass[12pt,a4paper]{article}
\usepackage[utf8]{inputenc}
\usepackage[T1]{fontenc}
\usepackage[ngerman]{babel}
\usepackage[left=2.5cm,right=2.5cm,top=2.5cm,bottom=2.5cm]{geometry}
\usepackage{lmodern}
\usepackage{amsmath}
\usepackage{amssymb}
\usepackage{physics}
\usepackage{hyperref}
\usepackage{tcolorbox}
\usepackage{booktabs}
\usepackage{enumitem}
\usepackage[table,xcdraw]{xcolor}
\usepackage{graphicx}
\usepackage{float}
\usepackage{mathtools}
\usepackage{amsthm}
\usepackage{siunitx}
\usepackage{fancyhdr}
\usepackage{longtable}
\usepackage{multirow}
\usepackage{array}

% Headers and Footers
\pagestyle{fancy}
\fancyhf{}
\fancyhead[L]{Mathematische Analyse: T0-Shor Algorithmus}
\fancyhead[R]{Theoretischer Rahmen}
\fancyfoot[C]{\thepage}
\renewcommand{\headrulewidth}{0.4pt}
\renewcommand{\footrulewidth}{0.4pt}

% Custom Commands
\newcommand{\Efield}{E}
\newcommand{\xipar}{\xi}
\newcommand{\Tfield}{T}
\newcommand{\mfield}{m}

\hypersetup{
	colorlinks=true,
	linkcolor=blue,
	citecolor=blue,
	urlcolor=blue,
	pdftitle={Mathematische Analyse des T0-Shor Algorithmus: Theoretischer Rahmen und Berechnungskomplexität},
	pdfauthor={Johann Pascher},
	pdfsubject={T0-Theorie, RSA-Algorithmen, Mathematische Analyse}
}

\title{Mathematische Analyse des T0-Shor Algorithmus: \\
	Theoretischer Rahmen und Berechnungskomplexität \\
	\large Eine rigorose Untersuchung des T0-Energiefeld-Ansatzes zur Ganzzahlfaktorisierung}
\author{Johann Pascher \\
	T0-Theorie Analyse-Framework}
\date{7. Juni 2025}



==================================================

=== RSA_En.tex.preamble ===

\documentclass[12pt,a4paper]{article}
\usepackage[utf8]{inputenc}
\usepackage[T1]{fontenc}
\usepackage[english]{babel}
\usepackage[left=2.5cm,right=2.5cm,top=2.5cm,bottom=2.5cm]{geometry}
\usepackage{lmodern}
\usepackage{amsmath}
\usepackage{amssymb}
\usepackage{physics}
\usepackage{hyperref}
\usepackage{tcolorbox}
\usepackage{booktabs}
\usepackage{enumitem}
\usepackage[table,xcdraw]{xcolor}
\usepackage{graphicx}
\usepackage{float}
\usepackage{mathtools}
\usepackage{amsthm}
\usepackage{siunitx}
\usepackage{fancyhdr}
\usepackage{longtable}
\usepackage{multirow}
\usepackage{array}

% Headers and Footers
\pagestyle{fancy}
\fancyhf{}
\fancyhead[L]{Mathematical Analysis: T0-Shor Algorithm}
\fancyhead[R]{Theoretical Framework}
\fancyfoot[C]{\thepage}
\renewcommand{\headrulewidth}{0.4pt}
\renewcommand{\footrulewidth}{0.4pt}

% Custom Commands
\newcommand{\Efield}{E}
\newcommand{\xipar}{\xi}
\newcommand{\Tfield}{T}
\newcommand{\mfield}{m}

\hypersetup{
	colorlinks=true,
	linkcolor=blue,
	citecolor=blue,
	urlcolor=blue,
	pdftitle={Mathematical Analysis of T0-Shor Algorithm: Theoretical Framework and Computational Complexity},
	pdfauthor={Johann Pascher},
	pdfsubject={T0-Theory, RSA-Algorithms, Mathematical Analysis}
}

\title{Mathematical Analysis of T0-Shor Algorithm: \\
	Theoretical Framework and Computational Complexity \\
	\large A Rigorous Examination of the T0-Energy Field Approach to Integer Factorization}
\author{Johann Pascher \\
	T0-Theory Analysis Framework}
\date{June 7, 2025}



==================================================

=== RSAtest_De.tex.preamble ===

\documentclass[12pt,a4paper]{article}
\usepackage[utf8]{inputenc}
\usepackage[T1]{fontenc}
\usepackage[ngerman]{babel}
\usepackage[left=2.5cm,right=2.5cm,top=2.5cm,bottom=2.5cm]{geometry}
\usepackage{lmodern}
\usepackage{amsmath}
\usepackage{amssymb}
\usepackage{physics}
\usepackage{hyperref}
\usepackage{tcolorbox}
\usepackage{booktabs}
\usepackage{enumitem}
\usepackage[table,xcdraw]{xcolor}
\usepackage{graphicx}
\usepackage{float}
\usepackage{mathtools}
\usepackage{amsthm}
\usepackage{siunitx}
\usepackage{fancyhdr}
\usepackage{longtable}
\usepackage{multirow}
\usepackage{array}
\usepackage{algorithm}
\usepackage{algorithmic}

% Kopf- und Fußzeilen
\pagestyle{fancy}
\fancyhf{}
\fancyhead[L]{Empirische Faktorisierungsmethoden}
\fancyhead[R]{Experimentelle Ergebnisse}
\fancyfoot[C]{\thepage}
\renewcommand{\headrulewidth}{0.4pt}
\renewcommand{\footrulewidth}{0.4pt}

\hypersetup{
	colorlinks=true,
	linkcolor=blue,
	citecolor=blue,
	urlcolor=blue,
	pdftitle={Empirische Analyse deterministischer Faktorisierungsmethoden},
	pdfauthor={Johann Pascher},
	pdfsubject={Faktorisierungsalgorithmen, Empirische Ergebnisse, Deterministische Methoden}
}

\title{Empirische Analyse deterministischer Faktorisierungsmethoden \\
	\large Systematische Bewertung klassischer und alternativer Ansätze}
\author{Johann Pascher}
\date{9. Juni 2025}



==================================================

=== RSAtest_En.tex.preamble ===

\documentclass[12pt,a4paper]{article}
\usepackage[utf8]{inputenc}
\usepackage[T1]{fontenc}
\usepackage[english]{babel}
\usepackage[left=2.5cm,right=2.5cm,top=2.5cm,bottom=2.5cm]{geometry}
\usepackage{lmodern}
\usepackage{amsmath}
\usepackage{amssymb}
\usepackage{physics}
\usepackage{hyperref}
\usepackage{tcolorbox}
\usepackage{booktabs}
\usepackage{enumitem}
\usepackage[table,xcdraw]{xcolor}
\usepackage{graphicx}
\usepackage{float}
\usepackage{mathtools}
\usepackage{amsthm}
\usepackage{siunitx}
\usepackage{fancyhdr}
\usepackage{longtable}
\usepackage{multirow}
\usepackage{array}
\usepackage{algorithm}
\usepackage{algorithmic}

% Headers and Footers
\pagestyle{fancy}
\fancyhf{}
\fancyhead[L]{Empirical Factorization Methods}
\fancyhead[R]{Experimental Results}
\fancyfoot[C]{\thepage}
\renewcommand{\headrulewidth}{0.4pt}
\renewcommand{\footrulewidth}{0.4pt}

\hypersetup{
	colorlinks=true,
	linkcolor=blue,
	citecolor=blue,
	urlcolor=blue,
	pdftitle={Empirical Analysis of Deterministic Factorization Methods},
	pdfauthor={Johann Pascher},
	pdfsubject={Factorization Algorithms, Empirical Results, Deterministic Methods}
}

\title{Empirical Analysis of Deterministic Factorization Methods \\
	\large Systematic Evaluation of Classical and Alternative Approaches}
\author{Johann Pascher}
\date{June 9, 2025}



==================================================

=== RelokativesZahlensystemDe.tex.preamble ===

\documentclass[11pt,a4paper]{article}
\usepackage[utf8]{inputenc}
\usepackage[T1]{fontenc}
\usepackage[ngerman]{babel}
\usepackage{lmodern}
\usepackage{amsmath}
\usepackage{amssymb}
\usepackage{textgreek}  % Für griechische Buchstaben im Text
\usepackage{physics}
\usepackage{hyperref}
\usepackage{booktabs}
\usepackage{array}
\usepackage[left=2cm,right=2cm,top=2cm,bottom=2cm]{geometry}
\usepackage{fancyhdr}
\usepackage{siunitx}
\usepackage{amsthm}
\usepackage{adjustbox}
\usepackage{tikz}
\usepackage{pgfplots}
\usepackage{algorithm}
\usepackage{algorithmic}
\pgfplotsset{compat=1.17}
\usetikzlibrary{positioning,shapes,arrows}

% Benutzerdefinierte Befehle
\newcommand{\tablescale}{0.9}
\newcommand{\primrel}[1]{\mathbf{#1}}
\newcommand{\interval}[2]{#1:#2}
\newcommand{\vect}[1]{\boldsymbol{#1}}

% Theorem-Umgebungen
\newtheorem{definition}{Definition}[section]
\newtheorem{theorem}{Theorem}[section]
\newtheorem{example}{Beispiel}[section]
\newtheorem{axiom}{Axiom}[section]
\newtheorem{category}{Kategorientheoretische Basis}[section]

% Kopf- und Fußzeile
\pagestyle{fancy}
\fancyhf{}
\fancyhead[L]{Das Relationale Zahlensystem}
\fancyhead[R]{Primzahlen als Verhältnisse}
\fancyfoot[C]{\thepage}
\renewcommand{\headrulewidth}{0.4pt}
\renewcommand{\footrulewidth}{0.4pt}

\hypersetup{
	colorlinks=true,
	linkcolor=blue,
	citecolor=blue,
	urlcolor=blue,
	pdftitle={Das Relationale Zahlensystem: Primzahlen als fundamentale Verhältnisse},
	pdfauthor={Relationale Mathematik}
}

\title{Das Relationale Zahlensystem:\\
	Primzahlen als fundamentale Verhältnisse}
\author{Johann Pascher\\
	Grundlagen eines alternativen Zahlensystems\\
	\texttt{johann.pascher@gmail.com}}
\date{\today}



==================================================

=== RelokativesZahlensystemEn.tex.preamble ===

\documentclass[11pt,a4paper]{article}
\usepackage[utf8]{inputenc}
\usepackage[T1]{fontenc}
\usepackage[english]{babel}
\usepackage{lmodern}
\usepackage{amsmath}
\usepackage{amssymb}
\usepackage{textgreek}  % For Greek letters in text
\usepackage{physics}
\usepackage{hyperref}
\usepackage{booktabs}
\usepackage{array}
\usepackage[left=2cm,right=2cm,top=2cm,bottom=2cm]{geometry}
\usepackage{fancyhdr}
\usepackage{siunitx}
\usepackage{amsthm}
\usepackage{adjustbox}
\usepackage{tikz}
\usepackage{pgfplots}
\usepackage{algorithm}
\usepackage{algorithmic}
\pgfplotsset{compat=1.17}
\usetikzlibrary{positioning,shapes,arrows}

% Custom commands
\newcommand{\tablescale}{0.9}
\newcommand{\primrel}[1]{\mathbf{#1}}
\newcommand{\interval}[2]{#1:#2}
\newcommand{\vect}[1]{\boldsymbol{#1}}

% Theorem environments
\newtheorem{definition}{Definition}[section]
\newtheorem{theorem}{Theorem}[section]
\newtheorem{example}{Example}[section]
\newtheorem{axiom}{Axiom}[section]
\newtheorem{category}{Category-Theoretic Basis}[section]

% Header and footer
\pagestyle{fancy}
\fancyhf{}
\fancyhead[L]{The Relational Number System}
\fancyhead[R]{Prime Numbers as Ratios}
\fancyfoot[C]{\thepage}
\renewcommand{\headrulewidth}{0.4pt}
\renewcommand{\footrulewidth}{0.4pt}

\hypersetup{
	colorlinks=true,
	linkcolor=blue,
	citecolor=blue,
	urlcolor=blue,
	pdftitle={The Relational Number System: Prime Numbers as Fundamental Ratios},
	pdfauthor={Relational Mathematics}
}

\title{The Relational Number System:\\
	Prime Numbers as Fundamental Ratios}
\author{Johann Pascher\\
	Foundations of an Alternative Number System\\
	\texttt{johann.pascher@gmail.com}}
\date{\today}



==================================================

=== ResolvingTheConstantsAlfaDe.tex.preamble ===

\documentclass[12pt,a4paper]{article}
\usepackage[utf8]{inputenc}
\usepackage[T1]{fontenc}
\usepackage[ngerman]{babel}
\usepackage[left=2.5cm,right=2.5cm,top=2.5cm,bottom=2.5cm]{geometry}
\usepackage{lmodern}
\usepackage{amsmath}
\usepackage{amssymb}
\usepackage{physics}
\usepackage{hyperref}
\usepackage{tcolorbox}
\usepackage{booktabs}
\usepackage{enumitem}
\usepackage[table,xcdraw]{xcolor}
\usepackage{graphicx}
\usepackage{float}
\usepackage{mathtools}
\usepackage{amsthm}
\usepackage{cleveref}
\usepackage{siunitx}
\usepackage{fancyhdr}
\usepackage{textcomp}
\usepackage{gensymb}

% Kopf- und Fußzeilen
\pagestyle{fancy}
\fancyhf{}
\fancyhead[L]{Johann Pascher}
\fancyhead[R]{Beweis: $\alpha = 1$ in natürlichen Einheiten}
\fancyfoot[C]{\thepage}
\renewcommand{\headrulewidth}{0.4pt}
\renewcommand{\footrulewidth}{0.4pt}

\hypersetup{
	colorlinks=true,
	linkcolor=blue,
	citecolor=blue,
	urlcolor=blue,
	pdftitle={Beweis: Die Feinstrukturkonstante α = 1 in natürlichen Einheiten},
	pdfauthor={Johann Pascher},
	pdfsubject={Feinstrukturkonstante, Natürliche Einheiten, Fundamentalphysik},
	pdfkeywords={Feinstrukturkonstante, Natürliche Einheiten, Elektromagnetische Dualität}
}

% Benutzerdefinierte Befehle
\newcommand{\alphaem}{\alpha}
\newcommand{\alphanat}{\alpha_{\text{nat}}}
\newcommand{\alphaSI}{\alpha_{\text{SI}}}

\newtheorem{theorem}{Theorem}[section]
\newtheorem{lemma}[theorem]{Lemma}
\newtheorem{proposition}[theorem]{Proposition}
\newtheorem{definition}[theorem]{Definition}
\newtheorem{proof_step}{Beweisschritt}[section]



==================================================

=== ResolvingTheConstantsAlfaEn.tex.preamble ===

\documentclass[12pt,a4paper]{article}
\usepackage[utf8]{inputenc}
\usepackage[T1]{fontenc}
\usepackage[german,english]{babel}
\usepackage[left=2.5cm,right=2.5cm,top=2.5cm,bottom=2.5cm]{geometry}
\usepackage{lmodern}
\usepackage{amsmath}
\usepackage{amssymb}
\usepackage{physics}
\usepackage{hyperref}
\usepackage{tcolorbox}
\usepackage{booktabs}
\usepackage{enumitem}
\usepackage[table,xcdraw]{xcolor}
\usepackage{graphicx}
\usepackage{float}
\usepackage{mathtools}
\usepackage{amsthm}
\usepackage{cleveref}
\usepackage{siunitx}
\usepackage{fancyhdr}
\usepackage{textcomp}
\usepackage{gensymb}

% Headers and Footers
\pagestyle{fancy}
\fancyhf{}
\fancyhead[L]{Johann Pascher}
\fancyhead[R]{Proof: $\alpha = 1$ in Natural Units}
\fancyfoot[C]{\thepage}
\renewcommand{\headrulewidth}{0.4pt}
\renewcommand{\footrulewidth}{0.4pt}

\hypersetup{
	colorlinks=true,
	linkcolor=blue,
	citecolor=blue,
	urlcolor=blue,
	pdftitle={Proof: The Fine Structure Constant α = 1 in Natural Units},
	pdfauthor={Johann Pascher},
	pdfsubject={Fine Structure Constant, Natural Units, Fundamental Physics},
	pdfkeywords={Fine Structure Constant, Natural Units, Electromagnetic Duality}
}

% Custom Commands
\newcommand{\alphaem}{\alpha}
\newcommand{\alphanat}{\alpha_{\text{nat}}}
\newcommand{\alphaSI}{\alpha_{\text{SI}}}

\newtheorem{theorem}{Theorem}[section]
\newtheorem{lemma}[theorem]{Lemma}
\newtheorem{proposition}[theorem]{Proposition}
\newtheorem{definition}[theorem]{Definition}
\newtheorem{proof_step}{Proof Step}[section]



==================================================

=== T0-QFT-ML_Addendum_De.tex.preamble ===

\documentclass[12pt,a4paper]{article}
\usepackage[utf8]{inputenc}
\usepackage[T1]{fontenc}
\usepackage[ngerman]{babel}
\usepackage[left=2.5cm,right=2.5cm,top=2.5cm,bottom=2.5cm]{geometry}
\usepackage{lmodern}
\usepackage{amssymb}
\usepackage{physics}
\usepackage{hyperref}
\usepackage{tcolorbox}
\usepackage{enumitem}
\usepackage[table]{xcolor}
\usepackage{graphicx}
\usepackage{float}
\usepackage{mathtools}
\usepackage{amsthm}
\usepackage{siunitx}
\usepackage{fancyhdr}
\usepackage{longtable}
\usepackage{booktabs}
\usepackage{amsmath}

% Kopf- und Fußzeilen
\pagestyle{fancy}
\fancyhf{}
\fancyhead[L]{T0-QFT: ML-abgeleitete Erweiterungen}
\fancyhead[R]{Maschinelle Lern-Erkenntnisse}
\fancyfoot[C]{\thepage}
\renewcommand{\headrulewidth}{0.4pt}
\renewcommand{\footrulewidth}{0.4pt}
\setlength{\headheight}{15pt}

\hypersetup{
	colorlinks=true,
	linkcolor=blue,
	citecolor=blue,
	urlcolor=blue,
	pdftitle={T0-QFT ML-Addendum: Maschinelle Lern-abgeleitete Erweiterungen},
	pdfauthor={Johann Pascher},
	pdfsubject={T0-Theorie, Maschinelles Lernen, QFT-Erweiterungen}
}

\title{\textbf{T0 Quantenfeldtheorie: ML-abgeleitete Erweiterungen}\\[0.5cm]
	\large Addendum zum vollständigen QFT-QM-QC Framework\\[0.3cm]
	\normalsize Maschinelle Lern-Erkenntnisse und emergente nicht-störungstheoretische Terme}
\author{Johann Pascher\\
	T0-Theorie Forschungsgruppe\\
	HTL Leonding, Österreich\\
	\texttt{johann.pascher@gmail.com}}
\date{\today}



==================================================

=== T0-QFT-ML_Addendum_En.tex.preamble ===

\documentclass[12pt,a4paper]{article}
\usepackage[utf8]{inputenc}
\usepackage[T1]{fontenc}
\usepackage[english]{babel}
\usepackage[left=2.5cm,right=2.5cm,top=2.5cm,bottom=2.5cm]{geometry}
\usepackage{lmodern}
%\usepackage{amsmath}
\usepackage{amssymb}
\usepackage{physics}
\usepackage{hyperref}
\usepackage{tcolorbox}
%\usepackage{booktabs}
\usepackage{enumitem}
\usepackage[table]{xcolor}
\usepackage{graphicx}
\usepackage{float}
\usepackage{mathtools}
\usepackage{amsthm}
\usepackage{siunitx}
\usepackage{fancyhdr}
\usepackage{longtable}
\usepackage{booktabs}
\usepackage{amsmath}  % for \text command
% Headers and Footers
\pagestyle{fancy}
\fancyhf{}
\fancyhead[L]{T0-QFT: ML-Derived Extensions}
\fancyhead[R]{Machine Learning Insights}
\fancyfoot[C]{\thepage}
\renewcommand{\headrulewidth}{0.4pt}
\renewcommand{\footrulewidth}{0.4pt}
\setlength{\headheight}{15pt}

\hypersetup{
	colorlinks=true,
	linkcolor=blue,
	citecolor=blue,
	urlcolor=blue,
	pdftitle={T0-QFT ML Addendum: Machine Learning Derived Extensions},
	pdfauthor={Johann Pascher},
	pdfsubject={T0-Theory, Machine Learning, QFT Extensions}
}

\title{\textbf{T0 Quantum Field Theory: ML-Derived Extensions}\\[0.5cm]
	\large Addendum to the Complete QFT-QM-QC Framework\\[0.3cm]
	\normalsize Machine Learning Insights and Emergent Non-Perturbative Terms}
\author{Johann Pascher\\
	T0-Theory Research Group\\
	HTL Leonding, Austria\\
	\texttt{johann.pascher@gmail.com}}
\date{\today}



==================================================

=== T0-Theory-vs-Synergetics_De.tex.preamble ===

\documentclass[12pt,a4paper]{article}
\usepackage[utf8]{inputenc}
\usepackage[T1]{fontenc}
\usepackage[ngerman]{babel}
\usepackage{lmodern}
\usepackage{amsmath,amssymb,amsthm}
\usepackage{geometry}
\usepackage{booktabs}
\usepackage{array}
\usepackage{xcolor}
\usepackage{tcolorbox}
\usepackage{fancyhdr}
\usepackage{hyperref}
\usepackage{tikz}
\usepackage{physics}

\definecolor{t0blue}{RGB}{33,150,243}
\definecolor{t0green}{RGB}{76,175,80}
\definecolor{t0orange}{RGB}{255,152,0}
\definecolor{t0red}{RGB}{244,67,54}

\geometry{a4paper, margin=2.5cm}
\setlength{\headheight}{15pt}

\pagestyle{fancy}
\fancyhf{}
\fancyhead[L]{\textsc{T0-Theorie: Die elegantere Lösung}}
\fancyhead[R]{\textsc{Vergleichsanalyse}}
\fancyfoot[C]{\thepage}

\hypersetup{
	colorlinks=true,
	linkcolor=t0blue,
	citecolor=t0blue,
	urlcolor=t0blue,
	pdftitle={T0 vs Synergetics: Vereinfachung durch natürliche Einheiten},
	pdfauthor={Vergleichsanalyse}
}

\newcommand{\xipar}{\xi}

\newtcolorbox{vergleich}{colback=t0blue!5, colframe=t0blue!75!black, title={Direkter Vergleich}}
\newtcolorbox{vorteil}{colback=t0green!5, colframe=t0green!75!black, title={T0-Vorteil}}
\newtcolorbox{gemeinsam}{colback=t0orange!5, colframe=t0orange!75!black, title={Gemeinsame Grundlage}}

\title{\textbf{T0-Theorie vs. Synergetics-Ansatz}\\[0.5cm]
	\large Wie natürliche Einheiten die geometrische Physik vereinfachen\\[0.3cm]
	\normalsize Eine detaillierte Vergleichsanalyse zweier konvergenter Ansätze}
\author{Vergleichende Analyse der geometrischen Reformulierung der Physik}
\date{\today}



==================================================

=== T0-Theory-vs-Synergetics_En.tex.preamble ===

\documentclass[12pt,a4paper]{article}
\usepackage[utf8]{inputenc}
\usepackage[T1]{fontenc}
\usepackage[ngerman]{babel}
\usepackage{lmodern}
\usepackage{amsmath,amssymb,amsthm}
\usepackage{geometry}
\usepackage{booktabs}
\usepackage{array}
\usepackage{xcolor}
\usepackage{tcolorbox}
\usepackage{fancyhdr}
\usepackage{hyperref}
\usepackage{tikz}
\usepackage{physics}

\definecolor{t0blue}{RGB}{33,150,243}
\definecolor{t0green}{RGB}{76,175,80}
\definecolor{t0orange}{RGB}{255,152,0}
\definecolor{t0red}{RGB}{244,67,54}

\geometry{a4paper, margin=2.5cm}
\setlength{\headheight}{15pt}

\pagestyle{fancy}
\fancyhf{}
\fancyhead[L]{\textsc{T0-Theorie: Die elegantere Lösung}}
\fancyhead[R]{\textsc{Vergleichsanalyse}}
\fancyfoot[C]{\thepage}

\hypersetup{
	colorlinks=true,
	linkcolor=t0blue,
	citecolor=t0blue,
	urlcolor=t0blue,
	pdftitle={T0 vs Synergetics: Vereinfachung durch natürliche Einheiten},
	pdfauthor={Vergleichsanalyse}
}

\newcommand{\xipar}{\xi}

\newtcolorbox{vergleich}{colback=t0blue!5, colframe=t0blue!75!black, title={Direkter Vergleich}}
\newtcolorbox{vorteil}{colback=t0green!5, colframe=t0green!75!black, title={T0-Vorteil}}
\newtcolorbox{gemeinsam}{colback=t0orange!5, colframe=t0orange!75!black, title={Gemeinsame Grundlage}}

\title{\textbf{T0-Theorie vs. Synergetics-Ansatz}\\[0.5cm]
	\large Wie natürliche Einheiten die geometrische Physik vereinfachen\\[0.3cm]
	\normalsize Eine detaillierte Vergleichsanalyse zweier konvergenter Ansätze}
\author{Vergleichende Analyse der geometrischen Reformulierung der Physik}
\date{\today}



==================================================

=== T0_7-fragen-3_De.tex.preamble ===

\documentclass[12pt,a4paper]{article}
\usepackage[utf8]{inputenc}
\usepackage[T1]{fontenc}
\usepackage[ngerman]{babel}
\usepackage[left=2.5cm,right=2.5cm,top=2.5cm,bottom=2.5cm]{geometry}
\usepackage{amsmath}
\usepackage{amssymb}
\usepackage{hyperref}
\usepackage{booktabs}
\usepackage{siunitx}
\usepackage[table]{xcolor}
\usepackage{physics}
\usepackage{fancyhdr}
\usepackage{tocloft}
\usepackage{breakurl}
\usepackage{float}
\usepackage{tikz}
% Header and Footer Configuration
\pagestyle{fancy}
\fancyhf{}
\fancyhead[L]{Johann Pascher}
\fancyhead[R]{T0-Theorie: Die sieben Rätsel}
\fancyfoot[C]{\thepage}
\renewcommand{\headrulewidth}{0.4pt}
\renewcommand{\footrulewidth}{0.4pt}
\setlength{\headheight}{15pt}
\definecolor{blue}{rgb}{0,0,1}
\renewcommand{\cftsecfont}{\color{blue}}
\renewcommand{\cftsecpagefont}{\color{blue}}
\hypersetup{
	colorlinks=true,
	linkcolor=blue,
	citecolor=blue,
	urlcolor=blue,
	pdftitle={T0-Theorie: Die sieben Rätsel},
	pdfauthor={Johann Pascher}
}
% Fix for Unicode ξ in text mode
\DeclareUnicodeCharacter{03BE}{$\xi$}
\title{\textbf{T0-Theorie: Die sieben Rätsel der Physik}\\[0.5cm]
	\large Vollständige Lösung durch fundamentale $\xi$-Geometrie\\[0.3cm]
	\normalsize Mathematisch exakte Herleitung aller Phänomene – Integration kosmologischer Aspekte}
\author{Johann Pascher\\
	Department for Communication Technology\\
	Higher Technical College (HTL), Leonding, Austria\\
	\texttt{johann.pascher@gmail.com}}
\date{\today}


==================================================

=== T0_7-fragen-3_En.tex.preamble ===

\documentclass[12pt,a4paper]{article}
\usepackage[utf8]{inputenc}
\usepackage[T1]{fontenc}
\usepackage[english]{babel}
\usepackage[left=2.5cm,right=2.5cm,top=2.5cm,bottom=2.5cm]{geometry}
\usepackage{amsmath}
\usepackage{amssymb}
\usepackage{hyperref}
\usepackage{booktabs}
\usepackage{siunitx}
\usepackage[table]{xcolor}
\usepackage{physics}
\usepackage{fancyhdr}
\usepackage{tocloft}
\usepackage{breakurl}
\usepackage{float}
\usepackage{tikz}
% Header and Footer Configuration
\pagestyle{fancy}
\fancyhf{}
\fancyhead[L]{Johann Pascher}
\fancyhead[R]{T0-Theory: The Seven Riddles}
\fancyfoot[C]{\thepage}
\renewcommand{\headrulewidth}{0.4pt}
\renewcommand{\footrulewidth}{0.4pt}
\setlength{\headheight}{15pt}
\definecolor{blue}{rgb}{0,0,1}
\renewcommand{\cftsecfont}{\color{blue}}
\renewcommand{\cftsecpagefont}{\color{blue}}
\hypersetup{
	colorlinks=true,
	linkcolor=blue,
	citecolor=blue,
	urlcolor=blue,
	pdftitle={T0-Theory: The Seven Riddles},
	pdfauthor={Johann Pascher}
}
% Fix for Unicode ξ in text mode
\DeclareUnicodeCharacter{03BE}{$\xi$}
\title{\textbf{T0-Theory: The Seven Riddles of Physics}\\[0.5cm]
	\large Complete Solution through Fundamental $\xi$-Geometry\\[0.3cm]
	\normalsize Mathematically Exact Derivation of All Phenomena – Integration of Cosmological Aspects}
\author{Johann Pascher\\
	Department for Communication Technology\\
	Higher Technical College (HTL), Leonding, Austria\\
	\texttt{johann.pascher@gmail.com}}
\date{\today}


==================================================

=== T0_Analyse_MNRAS_Widerlegung_De.tex.preamble ===

\documentclass[12pt,a4paper]{article}

% --- Grundlegende Pakete ---
\usepackage[utf8]{inputenc}
\usepackage[T1]{fontenc}
\usepackage[ngerman]{babel}
\usepackage{lmodern}
\usepackage{amsmath,amssymb,amsthm}
\usepackage{physics}
\usepackage{siunitx}
\usepackage{listings}
\usepackage{xcolor}

% --- Seitenlayout und Design ---
\usepackage[margin=2.5cm]{geometry}
\usepackage{fancyhdr}
\usepackage{hyperref}
\usepackage{graphicx}
\usepackage{booktabs}
\usepackage{enumitem}

% --- Hyperref-Konfiguration ---
\hypersetup{
    colorlinks=true,
    linkcolor=blue,
    citecolor=blue,
    urlcolor=blue,
    pdftitle={Analyse und Implikationen des MNRAS-Papiers 544 für die T0-Theorie},
    pdfauthor={Johann Pascher},
    pdfsubject={T0-Theorie, Modifizierte Gravitation, Hubble-Spannung}
}

% --- Kopf- und Fußzeile ---
\pagestyle{fancy}
\fancyhf{}
\fancyhead[L]{\textsc{T0-Theorie: Analyse externer Evidenz}}
\fancyhead[R]{\textsc{J. Pascher}}
\fancyfoot[C]{\thepage}
\renewcommand{\headrulewidth}{0.4pt}
\setlength{\headheight}{15pt}

% --- Mathematische Befehle ---
\newcommand{\xiT}{\xi}
\newcommand{\Hubble}{H_0}

% --- Titel-Informationen ---
\title{\textbf{Analyse des MNRAS-Papiers 544: Eine Falsifizierung modifizierter Gravitationsmodelle und eine indirekte Bestätigung der T0-Theorie}\\[0.5cm]
\large Wie die Nicht-Beobachtung von Anomalien im Sonnensystem die T0-Kosmologie stützt}
\author{Johann Pascher}
\date{2025-11-10 08:09:53 UTC}



==================================================

=== T0_Analyse_MNRAS_Widerlegung_En.tex.preamble ===

\documentclass[12pt,a4paper]{article}

% --- Basic Packages ---
\usepackage[utf8]{inputenc}
\usepackage[T1]{fontenc}
\usepackage[english]{babel}
\usepackage{lmodern}
\usepackage{amsmath,amssymb,amsthm}
\usepackage{physics}
\usepackage{siunitx}
\usepackage{listings}
\usepackage{xcolor}

% --- Page Layout and Design ---
\usepackage[margin=2.5cm]{geometry}
\usepackage{fancyhdr}
\usepackage{hyperref}
\usepackage{graphicx}
\usepackage{booktabs}
\usepackage{enumitem}

% --- Hyperref Configuration ---
\hypersetup{
	colorlinks=true,
	linkcolor=blue,
	citecolor=blue,
	urlcolor=blue,
	pdftitle={Analysis and Implications of MNRAS Paper 544 for the T0-Theory},
	pdfauthor={Johann Pascher},
	pdfsubject={T0-Theory, Modified Gravity, Hubble Tension}
}

% --- Header and Footer ---
\pagestyle{fancy}
\fancyhf{}
\fancyhead[L]{\textsc{T0-Theory: Analysis of External Evidence}}
\fancyhead[R]{\textsc{J. Pascher}}
\fancyfoot[C]{\thepage}
\renewcommand{\headrulewidth}{0.4pt}
\setlength{\headheight}{15pt}

% --- Mathematical Commands ---
\newcommand{\xiT}{\xi}
\newcommand{\Hubble}{H_0}

% --- Title Information ---
\title{\textbf{Analysis of MNRAS Paper 544: A Refutation of Modified Gravity Models and an Indirect Confirmation of the T0-Theory}\\[0.5cm]
	\large How the Non-Observation of Solar System Anomalies Supports T0 Cosmology}
\author{Johann Pascher}
\date{2025-11-10 08:09:53 UTC}



==================================================

=== T0_Anomale-g2-6_De.tex.preamble ===

\documentclass[12pt,a4paper]{article}
\usepackage[utf8]{inputenc}
\usepackage[T1]{fontenc}
\usepackage[german]{babel}
\usepackage{amsmath,amssymb,amsthm}
\usepackage{graphicx}
\usepackage{xcolor}
\usepackage{hyperref}
\usepackage{geometry}
\geometry{margin=2.5cm}
\usepackage{fancyhdr}
\usepackage{setspace}
\usepackage{booktabs}
\usepackage{enumitem}
\usepackage{siunitx}
\let\qty\relax
\usepackage{url}
\usepackage{longtable}
\usepackage{array}
\usepackage{colortbl}
\usepackage{adjustbox}
\usepackage{physics}
\usepackage{tcolorbox}
\sloppy

\hypersetup{
	colorlinks=true,
	linkcolor=blue,
	citecolor=blue,
	urlcolor=blue,
}

\definecolor{deepblue}{RGB}{0,0,127}
\definecolor{deepred}{RGB}{191,0,0}
\definecolor{deepgreen}{RGB}{0,127,0}

% Header Definition
\pagestyle{fancy}
\fancyhf{}
\fancyhead[L]{\textbf{T0-Theorie: Vereinheitlichte g-2-Berechnung (Rev. 6)}}
\fancyhead[R]{\textbf{Johann Pascher, 2025}}
\fancyfoot[C]{\thepage}
\renewcommand{\headrulewidth}{0.4pt}
\setlength{\headheight}{15pt}

% Line spacing
\setstretch{1.2}
\raggedbottom

% Colored boxes
\newtcolorbox{formula}[1][]{
	colback=blue!5!white,
	colframe=blue!75!black,
	fonttitle=\bfseries,
	title=#1
}
\newtcolorbox{result}[1][]{
	colback=green!5!white,
	colframe=green!75!black,
	fonttitle=\bfseries,
	title=#1
}
\newtcolorbox{verification}[1][]{
	colback=orange!5!white,
	colframe=orange!75!black,
	fonttitle=\bfseries,
	title=#1
}
\newtcolorbox{derivation}[1][]{
	colback=gray!5!white,
	colframe=gray!75!black,
	fonttitle=\bfseries,
	title=#1
}
\newtcolorbox{explanation}[1][]{
	colback=purple!5!white,
	colframe=purple!75!black,
	fonttitle=\bfseries,
	title=#1
}
\newtcolorbox{interpretation}[1][]{
	colback=cyan!5!white,
	colframe=cyan!75!black,
	fonttitle=\bfseries,
	title=#1
}

\title{\textbf{Vereinheitlichte Berechnung des anomalen magnetischen Moments in der T0-Theorie (Rev. 6)}\\[0.5cm]
	\large Vollständiger Beitrag von $\xi$ mit Torsion-Erweiterung -- Parameterfreie geometrische Lösung\\[0.3cm]
	\normalsize Erweiterte Ableitung mit SymPy-verifizierten Schleifenintegralen, Lagrangedichte und GitHub-Validierung (November 2025)}
\author{Johann Pascher\\
	\small Department of Communication Engineering,\\
	\small Higher Technical College (HTL), Leonding, Austria\\
	\small \texttt{johann.pascher@gmail.com}\\
	\small T0 Time-Mass Duality Research}
\date{1. November 2025}



==================================================

=== T0_Anomale-g2-6_En.tex.preamble ===

\documentclass[12pt,a4paper]{article}
\usepackage[utf8]{inputenc}
\usepackage[T1]{fontenc}
\usepackage[english]{babel}
\usepackage{amsmath,amssymb,amsthm}
\usepackage{graphicx}
\usepackage{xcolor}
\usepackage{hyperref}
\usepackage{geometry}
\geometry{margin=2.5cm}
\usepackage{fancyhdr}
\usepackage{setspace}
\usepackage{booktabs}
\usepackage{enumitem}
\usepackage{siunitx}  % Moved after \let\qty\relax to suppress warning
\let\qty\relax  % Suppress siunitx qty redefinition warning
\usepackage{url}
\usepackage{longtable}
\usepackage{array}
\usepackage{colortbl}
\usepackage{adjustbox}
\usepackage{physics}
\usepackage{tcolorbox}
\sloppy

\hypersetup{
	colorlinks=true,
	linkcolor=blue,
	citecolor=blue,
	urlcolor=blue,
}

\definecolor{deepblue}{RGB}{0,0,127}
\definecolor{deepred}{RGB}{191,0,0}
\definecolor{deepgreen}{RGB}{0,127,0}

% Header Definition
\pagestyle{fancy}
\fancyhf{}
\fancyhead[L]{\textbf{T0 Theory: Unified g-2 Calculation (Rev. 6)}}
\fancyhead[R]{\textbf{Johann Pascher, 2025}}
\fancyfoot[C]{\thepage}
\renewcommand{\headrulewidth}{0.4pt}
\setlength{\headheight}{15pt}

% Line spacing
\setstretch{1.2}
\raggedbottom

% Colored boxes
\newtcolorbox{formula}[1][]{
	colback=blue!5!white,
	colframe=blue!75!black,
	fonttitle=\bfseries,
	title=#1
}
\newtcolorbox{result}[1][]{
	colback=green!5!white,
	colframe=green!75!black,
	fonttitle=\bfseries,
	title=#1
}
\newtcolorbox{verification}[1][]{
	colback=orange!5!white,
	colframe=orange!75!black,
	fonttitle=\bfseries,
	title=#1
}
\newtcolorbox{derivation}[1][]{
	colback=gray!5!white,
	colframe=gray!75!black,
	fonttitle=\bfseries,
	title=#1
}
\newtcolorbox{explanation}[1][]{
	colback=purple!5!white,
	colframe=purple!75!black,
	fonttitle=\bfseries,
	title=#1
}
\newtcolorbox{interpretation}[1][]{
	colback=cyan!5!white,
	colframe=cyan!75!black,
	fonttitle=\bfseries,
	title=#1
}

\title{\textbf{Unified Calculation of the Anomalous Magnetic Moment in the T0 Theory (Rev. 6)}\\[0.5cm]
	\large Complete Contribution from $\xi$ with Torsion Extension -- Parameter-Free Geometric Solution\\[0.3cm]
	\normalsize Extended Derivation with SymPy-Verified Loop Integrals, Lagrangian Density, and GitHub Validation (November 2025)}
\author{Johann Pascher\\
	\small Department of Communication Engineering,\\
	\small Higher Technical College (HTL), Leonding, Austria\\
	\small \texttt{johann.pascher@gmail.com}\\
	\small T0 Time-Mass Duality Research}
\date{November 1, 2025}



==================================================

=== T0_Anomale-g2-9_De.tex.preamble ===

\documentclass[12pt,a4paper]{article}
\usepackage[utf8]{inputenc}
\usepackage[T1]{fontenc}
\usepackage[ngerman]{babel}
\usepackage{amsmath,amssymb,amsthm}
\usepackage{graphicx}
\usepackage{xcolor}
\usepackage{hyperref}
\usepackage{geometry}
\geometry{margin=2.5cm}
\usepackage{fancyhdr}
\usepackage{setspace}
\usepackage{booktabs}
\usepackage{enumitem}
\usepackage{siunitx}
\usepackage{url}
\usepackage{longtable}
\usepackage{array}
\usepackage{colortbl}
\usepackage{adjustbox}
\usepackage{physics}
\usepackage{tcolorbox}
\sloppy

\hypersetup{
	colorlinks=true,
	linkcolor=blue,
	citecolor=blue,
	urlcolor=blue,
}

\definecolor{deepblue}{RGB}{0,0,127}
\definecolor{deepred}{RGB}{191,0,0}
\definecolor{deepgreen}{RGB}{0,127,0}

% Header Definition
\pagestyle{fancy}
\fancyhf{}
\fancyhead[L]{\textbf{T0-Theorie: Vereinheitlichte g-2-Berechnung (Rev. 9 -- Überarbeitet, Brücke zu Sept.-Prototyp)}}
\fancyhead[R]{\textbf{Johann Pascher, 2025}}
\fancyfoot[C]{\thepage}
\renewcommand{\headrulewidth}{0.4pt}
\setlength{\headheight}{15.1pt}

% Line spacing
\setstretch{1.2}
\raggedbottom

% Colored boxes
\newtcolorbox{formula}[1][]{
	colback=blue!5!white,
	colframe=blue!75!black,
	fonttitle=\bfseries,
	title=#1
}
\newtcolorbox{result}[1][]{
	colback=green!5!white,
	colframe=green!75!black,
	fonttitle=\bfseries,
	title=#1
}
\newtcolorbox{verification}[1][]{
	colback=orange!5!white,
	colframe=orange!75!black,
	fonttitle=\bfseries,
	title=#1
}
\newtcolorbox{derivation}[1][]{
	colback=gray!5!white,
	colframe=gray!75!black,
	fonttitle=\bfseries,
	title=#1
}
\newtcolorbox{explanation}[1][]{
	colback=purple!5!white,
	colframe=purple!75!black,
	fonttitle=\bfseries,
	title=#1
}
\newtcolorbox{interpretation}[1][]{
	colback=cyan!5!white,
	colframe=cyan!75!black,
	fonttitle=\bfseries,
	title=#1
}

% Theorem styles from original
\theoremstyle{definition}
\newtheorem{definition}{Definition}[section]
\newtheorem{theorem}{Theorem}[section]
\newtheorem{lemma}{Lemma}[section]
\newtheorem{corollary}{Korollar}[section]

\title{\textbf{Vereinheitlichte Berechnung des anomalen magnetischen Moments in der T0-Theorie (Rev. 9 -- Überarbeitet)}\\[0.5cm]
	\large Vollständiger Beitrag von $\xi$ mit Torsionserweiterung -- Parameterfreie geometrische Lösung\\[0.3cm]
	\normalsize Erweiterte Ableitung mit SymPy-verifizierten Schleifenintegralen, Lagrangedichte und GitHub-Validierung (November 2025) -- Mit RG-Dualitätskorrektur und Integration des Sept.-Prototyps}
\author{Johann Pascher\\
	\small Abteilung für Kommunikationstechnik,\\
	\small Höhere Technische Lehranstalt (HTL), Leonding, Österreich\\
	\small \texttt{johann.pascher@gmail.com}\\
	\small T0 Zeit-Masse-Dualitätsforschung}
\date{1. November 2025}



==================================================

=== T0_Anomale-g2-9_En.tex.preamble ===

\documentclass[12pt,a4paper]{article}
\usepackage[utf8]{inputenc}
\usepackage[T1]{fontenc}
\usepackage[english]{babel}
\usepackage{amsmath,amssymb,amsthm}
\usepackage{graphicx}
\usepackage{xcolor}
\usepackage{hyperref}
\usepackage{geometry}
\geometry{margin=2.5cm}
\usepackage{fancyhdr}
\usepackage{setspace}
\usepackage{booktabs}
\usepackage{enumitem}
\usepackage{siunitx}
\usepackage{url}
\usepackage{longtable}
\usepackage{array}
\usepackage{colortbl}
\usepackage{adjustbox}
\usepackage{physics}
\usepackage{tcolorbox}
\sloppy

\hypersetup{
	colorlinks=true,
	linkcolor=blue,
	citecolor=blue,
	urlcolor=blue,
}

\definecolor{deepblue}{RGB}{0,0,127}
\definecolor{deepred}{RGB}{191,0,0}
\definecolor{deepgreen}{RGB}{0,127,0}

% Header Definition
\pagestyle{fancy}
\fancyhf{}
\fancyhead[L]{\textbf{T0 Theory: Unified g-2 Calculation (Rev. 9 -- Revised, Bridge to Sept. Prototype)}}
\fancyhead[R]{\textbf{Johann Pascher, 2025}}
\fancyfoot[C]{\thepage}
\renewcommand{\headrulewidth}{0.4pt}
\setlength{\headheight}{15.1pt}

% Line spacing
\setstretch{1.2}
\raggedbottom

% Colored boxes
\newtcolorbox{formula}[1][]{
	colback=blue!5!white,
	colframe=blue!75!black,
	fonttitle=\bfseries,
	title=#1
}
\newtcolorbox{result}[1][]{
	colback=green!5!white,
	colframe=green!75!black,
	fonttitle=\bfseries,
	title=#1
}
\newtcolorbox{verification}[1][]{
	colback=orange!5!white,
	colframe=orange!75!black,
	fonttitle=\bfseries,
	title=#1
}
\newtcolorbox{derivation}[1][]{
	colback=gray!5!white,
	colframe=gray!75!black,
	fonttitle=\bfseries,
	title=#1
}
\newtcolorbox{explanation}[1][]{
	colback=purple!5!white,
	colframe=purple!75!black,
	fonttitle=\bfseries,
	title=#1
}
\newtcolorbox{interpretation}[1][]{
	colback=cyan!5!white,
	colframe=cyan!75!black,
	fonttitle=\bfseries,
	title=#1
}

% Theorem styles from original
\theoremstyle{definition}
\newtheorem{definition}{Definition}[section]
\newtheorem{theorem}{Theorem}[section]
\newtheorem{lemma}{Lemma}[section]
\newtheorem{corollary}{Corollary}[section]

\title{\textbf{Unified Calculation of the Anomalous Magnetic Moment in the T0 Theory (Rev. 9 -- Revised)}\\[0.5cm]
	\large Complete Contribution from $\xi$ with Torsion Extension -- Parameter-Free Geometric Solution\\[0.3cm]
	\normalsize Extended Derivation with SymPy-Verified Loop Integrals, Lagrangian Density, and GitHub Validation (November 2025) -- With RG-Duality Correction and Integration of the Sept. Prototype}
\author{Johann Pascher\\
	\small Department of Communication Engineering,\\
	\small Higher Technical College (HTL), Leonding, Austria\\
	\small \texttt{johann.pascher@gmail.com}\\
	\small T0 Time-Mass Duality Research}
\date{November 1, 2025}



==================================================

=== T0_Anomale_Magnetische_Momente_De.tex.preamble ===

\documentclass[12pt,a4paper]{article}
\usepackage[utf8]{inputenc}
\usepackage[T1]{fontenc}
\usepackage[ngerman]{babel}
\usepackage{amsmath,amssymb,amsthm}
\usepackage{graphicx}
\usepackage{color}
\usepackage{hyperref}
\usepackage{geometry}
\geometry{margin=2.5cm}
\usepackage{fancyhdr}
\usepackage{setspace}
\usepackage{booktabs}
\hypersetup{
	colorlinks=true,
	linkcolor=blue,
	citecolor=blue,
	urlcolor=blue,
}
\usepackage{physics}
\usepackage{xcolor}
\usepackage{tcolorbox}
\definecolor{deepblue}{RGB}{0,0,127}
\definecolor{deepred}{RGB}{191,0,0}
\definecolor{deepgreen}{RGB}{0,127,0}

% Header Definition von Pascher
\pagestyle{fancy}
\fancyhf{}
\fancyhead[L]{\textbf{T0-Theorie: Zeitfeld-Erweiterung}}
\fancyhead[R]{\textbf{Johann Pascher, 2025}}
\fancyfoot[C]{\thepage}
\renewcommand{\headrulewidth}{0.4pt}
\setlength{\headheight}{15pt}

% Theoreme und Definitionen
\theoremstyle{definition}
\newtheorem{definition}{Definition}[section]
\newtheorem{theorem}{Theorem}[section]
\newtheorem{lemma}{Lemma}[section]
\newtheorem{corollary}{Korollar}[section]

% Zeilenabstand
\setstretch{1.2}

\newtcolorbox{formula}[1][]{
	colback=blue!5!white,
	colframe=blue!75!black,
	fonttitle=\bfseries,
	title=#1
}
\newtcolorbox{result}[1][]{
	colback=green!5!white,
	colframe=green!75!black,
	fonttitle=\bfseries,
	title=#1
}
\newtcolorbox{revolution}[1][]{
	colback=red!5!white,
	colframe=red!75!black,
	fonttitle=\bfseries,
	title=#1
}
\newtcolorbox{verification}[1][]{
	colback=orange!5!white,
	colframe=orange!75!black,
	fonttitle=\bfseries,
	title=#1
}
\newtcolorbox{derivation}[1][]{
	colback=gray!5!white,
	colframe=gray!75!black,
	fonttitle=\bfseries,
	title=#1
}
\newtcolorbox{explanation}[1][]{
	colback=purple!5!white,
	colframe=purple!75!black,
	fonttitle=\bfseries,
	title=#1
}

\title{\textbf{Erweiterte Lagrange-Dichte mit Zeitfeld zur Erklärung des Myon \(g-2\)-Anomalie}\\[0.5cm]
	\large Die T0-Theorie: Zeit-Masse-Dualität und anomale magnetische Momente\\[0.3cm]
	\normalsize Vollständige theoretische Ableitung ohne freie Parameter}
\author{Johann Pascher\\
	\small Department für Kommunikationstechnik,\\
	\small Höhere Technische Lehranstalt (HTL), Leonding, Österreich\\
	\small \texttt{johann.pascher@gmail.com}\\
	\small T0-Zeit-Masse-Dualitätsforschung}
\date{17. September 2025}



==================================================

=== T0_Anomale_Magnetische_Momente_En.tex.preamble ===

\documentclass[12pt,a4paper]{article}
\usepackage[utf8]{inputenc}
\usepackage[T1]{fontenc}
\usepackage[english]{babel}
\usepackage{amsmath,amssymb,amsthm}
\usepackage{graphicx}
\usepackage{color}
\usepackage{hyperref}
\usepackage{geometry}
\geometry{margin=2.5cm}
\usepackage{fancyhdr}
\usepackage{setspace}
\usepackage{booktabs}
\hypersetup{
	colorlinks=true,
	linkcolor=blue,
	citecolor=blue,
	urlcolor=blue,
}
\usepackage{physics}
\usepackage{xcolor}
\usepackage{tcolorbox}
\definecolor{deepblue}{RGB}{0,0,127}
\definecolor{deepred}{RGB}{191,0,0}
\definecolor{deepgreen}{RGB}{0,127,0}

% Header Definition by Pascher
\pagestyle{fancy}
\fancyhf{}
\fancyhead[L]{\textbf{T0-Theory: Time Field Extension}}
\fancyhead[R]{\textbf{Johann Pascher, 2025}}
\fancyfoot[C]{\thepage}
\renewcommand{\headrulewidth}{0.4pt}
\setlength{\headheight}{15pt}

% Theorems and Definitions
\theoremstyle{definition}
\newtheorem{definition}{Definition}[section]
\newtheorem{theorem}{Theorem}[section]
\newtheorem{lemma}{Lemma}[section]
\newtheorem{corollary}{Corollary}[section]

% Spacing
\setstretch{1.2}

\newtcolorbox{formula}[1][]{
	colback=blue!5!white,
	colframe=blue!75!black,
	fonttitle=\bfseries,
	title=#1
}
\newtcolorbox{result}[1][]{
	colback=green!5!white,
	colframe=green!75!black,
	fonttitle=\bfseries,
	title=#1
}
\newtcolorbox{revolution}[1][]{
	colback=red!5!white,
	colframe=red!75!black,
	fonttitle=\bfseries,
	title=#1
}
\newtcolorbox{verification}[1][]{
	colback=orange!5!white,
	colframe=orange!75!black,
	fonttitle=\bfseries,
	title=#1
}
\newtcolorbox{derivation}[1][]{
	colback=gray!5!white,
	colframe=gray!75!black,
	fonttitle=\bfseries,
	title=#1
}
\newtcolorbox{explanation}[1][]{
	colback=purple!5!white,
	colframe=purple!75!black,
	fonttitle=\bfseries,
	title=#1
}

\title{\textbf{Extended Lagrangian Density with Time Field for Explaining the Muon \(g-2\) Anomaly}\\[0.5cm]
	\large The T0-Theory: Time-Mass Duality and Anomalous Magnetic Moments\\[0.3cm]
	\normalsize Complete Theoretical Derivation Without Free Parameters}
\author{Johann Pascher\\
	\small Department of Communication Engineering,\\
	\small Higher Technical Institute (HTL), Leonding, Austria\\
	\small \texttt{johann.pascher@gmail.com}\\
	\small T0-Time-Mass-Duality Research}
\date{17 September 2025}



==================================================

=== T0_Bibliography_De.tex.preamble ===

\documentclass{article}

% ===== KORRIGIERTER PREAMBLE =====
\usepackage[utf8]{inputenc}
\usepackage[T1]{fontenc}
\usepackage{textcomp}
\usepackage{amsmath,amssymb}
\usepackage{upgreek}          % for Greek letters in math mode
\usepackage{microtype}        % improves typography
\usepackage{ragged2e}         % better justification
\usepackage{float}            % for [H] float placement
\usepackage{url}              % for URL formatting
\usepackage{hyphenat}         % better hyphenation
\usepackage{hyperref}         % for better link handling
\usepackage[german]{babel}    % German language support

% Scalable Fonts für Auto-Expansion
\usepackage{lmodern}          % Latin Modern Font (skalierbar)
\usepackage{bm}               % for bold math symbols

% Improve line breaking - weniger strikte Einstellungen
\emergencystretch=2em
\tolerance=2000
\hyphenpenalty=500
\exhyphenpenalty=500

% Fix URL formatting in bibliography
\urlstyle{sf}                 % Sans-Serif für bessere Lesbarkeit

% Deaktiviere Font Expansion falls nötig
\pdfminorversion=5
\pdfobjcompresslevel=0



==================================================

=== T0_Bibliography_En.tex.preamble ===

\documentclass{article}

% ===== KORRIGIERTER PREAMBLE =====
\usepackage[utf8]{inputenc}
\usepackage[T1]{fontenc}
\usepackage{textcomp}
\usepackage{amsmath,amssymb}
\usepackage{upgreek}          % for Greek letters in math mode
\usepackage{microtype}        % improves typography
\usepackage{ragged2e}         % better justification
\usepackage{float}            % for [H] float placement
\usepackage{url}              % for URL formatting
\usepackage{hyphenat}         % better hyphenation
\usepackage{hyperref}         % for better link handling

% Scalable Fonts für Auto-Expansion
\usepackage{lmodern}          % Latin Modern Font (skalierbar)
\usepackage{bm}               % for bold math symbols

% Improve line breaking - weniger strikte Einstellungen
\emergencystretch=2em
\tolerance=2000
\hyphenpenalty=500
\exhyphenpenalty=500

% Fix URL formatting in bibliography
\urlstyle{sf}                 % Sans-Serif für bessere Lesbarkeit

% Deaktiviere Font Expansion falls nötig
\pdfminorversion=5
\pdfobjcompresslevel=0



==================================================

=== T0_Book_En (2).tex.preamble ===

This is pdfTeX, Version 3.141592653-2.6-1.40.25 (MiKTeX 24.1) (preloaded format=pdflatex 2025.1.3)  23 NOV 2025 17:16
entering extended mode
 restricted \write18 enabled.
 %&-line parsing enabled.
**./T0_Book_En.tex
(T0_Book_En.tex
LaTeX2e <2024-06-01> patch level 2
L3 programming layer <2024-08-16>
(C:\Users\johann\AppData\Local\Programs\MiKTeX\tex/latex/base\book.cls
Document Class: book 2024/02/08 v1.4n Standard LaTeX document class
(C:\Users\johann\AppData\Local\Programs\MiKTeX\tex/latex/base\bk11.clo
File: bk11.clo 2024/02/08 v1.4n Standard LaTeX file (size option)
)
\c@part=\count194
\c@chapter=\count195
\c@section=\count196
\c@subsection=\count197
\c@subsubsection=\count198
\c@paragraph=\count199
\c@subparagraph=\count266
\c@figure=\count267
\c@table=\count268
\abovecaptionskip=\skip49
\belowcaptionskip=\skip50
\bibindent=\dimen141
)
(C:\Users\johann\AppData\Local\Programs\MiKTeX\tex/latex/geometry\geometry.sty
Package: geometry 2020/01/02 v5.9 Page Geometry

(C:\Users\johann\AppData\Local\Programs\MiKTeX\tex/latex/graphics\keyval.sty
Package: keyval 2022/05/29 v1.15 key=value parser (DPC)
\KV@toks@=\toks17
)
(C:\Users\johann\AppData\Local\Programs\MiKTeX\tex/generic/iftex\ifvtex.sty
Package: ifvtex 2019/10/25 v1.7 ifvtex legacy package. Use iftex instead.

(C:\Users\johann\AppData\Local\Programs\MiKTeX\tex/generic/iftex\iftex.sty
Package: iftex 2022/02/03 v1.0f TeX engine tests
))
\Gm@cnth=\count269
\Gm@cntv=\count270
\c@Gm@tempcnt=\count271
\Gm@bindingoffset=\dimen142
\Gm@wd@mp=\dimen143
\Gm@odd@mp=\dimen144
\Gm@even@mp=\dimen145
\Gm@layoutwidth=\dimen146
\Gm@layoutheight=\dimen147
\Gm@layouthoffset=\dimen148
\Gm@layoutvoffset=\dimen149
\Gm@dimlist=\toks18

(C:\Users\johann\AppData\Local\Programs\MiKTeX\tex/latex/geometry\geometry.cfg)
) (C:\Users\johann\AppData\Local\Programs\MiKTeX\tex/latex/base\inputenc.sty
Package: inputenc 2024/02/08 v1.3d Input encoding file
\inpenc@prehook=\toks19
\inpenc@posthook=\toks20
)
(C:\Users\johann\AppData\Local\Programs\MiKTeX\tex/generic/babel\babel.sty
Package: babel 2024/08/29 v24.9 The Babel package
\babel@savecnt=\count272
\U@D=\dimen150
\l@unhyphenated=\language79

(C:\Users\johann\AppData\Local\Programs\MiKTeX\tex/generic/babel\txtbabel.def)
\bbl@readstream=\read2
\bbl@dirlevel=\count273

*************************************
* Local config file bblopts.cfg used
*
(C:\Users\johann\AppData\Local\Programs\MiKTeX\tex/latex/arabi\bblopts.cfg
File: bblopts.cfg 2005/09/08 v0.1 add Arabic and Farsi to "declared" options of
 babel
)
(C:\Users\johann\AppData\Local\Programs\MiKTeX\tex/latex/babel-english\english.
ldf
Language: english 2017/06/06 v3.3r English support from the babel system
Package babel Info: Hyphen rules for 'canadian' set to \l@english
(babel)             (\language0). Reported on input line 102.
Package babel Info: Hyphen rules for 'australian' set to \l@ukenglish
(babel)             (\language73). Reported on input line 105.
Package babel Info: Hyphen rules for 'newzealand' set to \l@ukenglish
(babel)             (\language73). Reported on input line 108.
))
(C:\Users\johann\AppData\Local\Programs\MiKTeX\tex/generic/babel/locale/en\babe
l-english.tex
Package babel Info: Importing font and identification data for english
(babel)             from babel-en.ini. Reported on input line 11.
)
(C:\Users\johann\AppData\Local\Programs\MiKTeX\tex/latex/lm\lmodern.sty
Package: lmodern 2015/05/01 v1.6.1 Latin Modern Fonts
LaTeX Font Info:    Overwriting symbol font `operators' in version `normal'
(Font)                  OT1/cmr/m/n --> OT1/lmr/m/n on input line 22.
LaTeX Font Info:    Overwriting symbol font `letters' in version `normal'
(Font)                  OML/cmm/m/it --> OML/lmm/m/it on input line 23.
LaTeX Font Info:    Overwriting symbol font `symbols' in version `normal'
(Font)                  OMS/cmsy/m/n --> OMS/lmsy/m/n on input line 24.
LaTeX Font Info:    Overwriting symbol font `largesymbols' in version `normal'
(Font)                  OMX/cmex/m/n --> OMX/lmex/m/n on input line 25.
LaTeX Font Info:    Overwriting symbol font `operators' in version `bold'
(Font)                  OT1/cmr/bx/n --> OT1/lmr/bx/n on input line 26.
LaTeX Font Info:    Overwriting symbol font `letters' in version `bold'
(Font)                  OML/cmm/b/it --> OML/lmm/b/it on input line 27.
LaTeX Font Info:    Overwriting symbol font `symbols' in version `bold'
(Font)                  OMS/cmsy/b/n --> OMS/lmsy/b/n on input line 28.
LaTeX Font Info:    Overwriting symbol font `largesymbols' in version `bold'
(Font)                  OMX/cmex/m/n --> OMX/lmex/m/n on input line 29.
LaTeX Font Info:    Overwriting math alphabet `\mathbf' in version `normal'
(Font)                  OT1/cmr/bx/n --> OT1/lmr/bx/n on input line 31.
LaTeX Font Info:    Overwriting math alphabet `\mathsf' in version `normal'
(Font)                  OT1/cmss/m/n --> OT1/lmss/m/n on input line 32.
LaTeX Font Info:    Overwriting math alphabet `\mathit' in version `normal'
(Font)                  OT1/cmr/m/it --> OT1/lmr/m/it on input line 33.
LaTeX Font Info:    Overwriting math alphabet `\mathtt' in version `normal'
(Font)                  OT1/cmtt/m/n --> OT1/lmtt/m/n on input line 34.
LaTeX Font Info:    Overwriting math alphabet `\mathbf' in version `bold'
(Font)                  OT1/cmr/bx/n --> OT1/lmr/bx/n on input line 35.
LaTeX Font Info:    Overwriting math alphabet `\mathsf' in version `bold'
(Font)                  OT1/cmss/bx/n --> OT1/lmss/bx/n on input line 36.
LaTeX Font Info:    Overwriting math alphabet `\mathit' in version `bold'
(Font)                  OT1/cmr/bx/it --> OT1/lmr/bx/it on input line 37.
LaTeX Font Info:    Overwriting math alphabet `\mathtt' in version `bold'
(Font)                  OT1/cmtt/m/n --> OT1/lmtt/m/n on input line 38.
)
(C:\Users\johann\AppData\Local\Programs\MiKTeX\tex/latex/amsmath\amsmath.sty
Package: amsmath 2024/05/23 v2.17q AMS math features
\@mathmargin=\skip51

For additional information on amsmath, use the `?' option.
(C:\Users\johann\AppData\Local\Programs\MiKTeX\tex/latex/amsmath\amstext.sty
Package: amstext 2021/08/26 v2.01 AMS text

(C:\Users\johann\AppData\Local\Programs\MiKTeX\tex/latex/amsmath\amsgen.sty
File: amsgen.sty 1999/11/30 v2.0 generic functions
\@emptytoks=\toks21
\ex@=\dimen151
))
(C:\Users\johann\AppData\Local\Programs\MiKTeX\tex/latex/amsmath\amsbsy.sty
Package: amsbsy 1999/11/29 v1.2d Bold Symbols
\pmbraise@=\dimen152
)
(C:\Users\johann\AppData\Local\Programs\MiKTeX\tex/latex/amsmath\amsopn.sty
Package: amsopn 2022/04/08 v2.04 operator names
)
\inf@bad=\count274
LaTeX Info: Redefining \frac on input line 233.
\uproot@=\count275
\leftroot@=\count276
LaTeX Info: Redefining \overline on input line 398.
LaTeX Info: Redefining \colon on input line 409.
\classnum@=\count277
\DOTSCASE@=\count278
LaTeX Info: Redefining \ldots on input line 495.
LaTeX Info: Redefining \dots on input line 498.
LaTeX Info: Redefining \cdots on input line 619.
\Mathstrutbox@=\box52
\strutbox@=\box53
LaTeX Info: Redefining \big on input line 721.
LaTeX Info: Redefining \Big on input line 722.
LaTeX Info: Redefining \bigg on input line 723.
LaTeX Info: Redefining \Bigg on input line 724.
\big@size=\dimen153
LaTeX Font Info:    Redeclaring font encoding OML on input line 742.
LaTeX Font Info:    Redeclaring font encoding OMS on input line 743.
\macc@depth=\count279
LaTeX Info: Redefining \bmod on input line 904.
LaTeX Info: Redefining \pmod on input line 909.
LaTeX Info: Redefining \smash on input line 939.
LaTeX Info: Redefining \relbar on input line 969.
LaTeX Info: Redefining \Relbar on input line 970.
\c@MaxMatrixCols=\count280
\dotsspace@=\muskip17
\c@parentequation=\count281
\dspbrk@lvl=\count282
\tag@help=\toks22
\row@=\count283
\column@=\count284
\maxfields@=\count285
\andhelp@=\toks23
\eqnshift@=\dimen154
\alignsep@=\dimen155
\tagshift@=\dimen156
\tagwidth@=\dimen157
\totwidth@=\dimen158
\lineht@=\dimen159
\@envbody=\toks24
\multlinegap=\skip52
\multlinetaggap=\skip53
\mathdisplay@stack=\toks25
LaTeX Info: Redefining \[ on input line 2953.
LaTeX Info: Redefining \] on input line 2954.
)
(C:\Users\johann\AppData\Local\Programs\MiKTeX\tex/latex/amsfonts\amssymb.sty
Package: amssymb 2013/01/14 v3.01 AMS font symbols

(C:\Users\johann\AppData\Local\Programs\MiKTeX\tex/latex/amsfonts\amsfonts.sty
Package: amsfonts 2013/01/14 v3.01 Basic AMSFonts support
\symAMSa=\mathgroup4
\symAMSb=\mathgroup5
LaTeX Font Info:    Redeclaring math symbol \hbar on input line 98.
LaTeX Font Info:    Overwriting math alphabet `\mathfrak' in version `bold'
(Font)                  U/euf/m/n --> U/euf/b/n on input line 106.
)) (C:\Program Files\MiKTeX\tex/latex/amscls\amsthm.sty
Package: amsthm 2020/05/29 v2.20.6
\thm@style=\toks26
\thm@bodyfont=\toks27
\thm@headfont=\toks28
\thm@notefont=\toks29
\thm@headpunct=\toks30
\thm@preskip=\skip54
\thm@postskip=\skip55
\thm@headsep=\skip56
\dth@everypar=\toks31
)
(C:\Users\johann\AppData\Local\Programs\MiKTeX\tex/latex/graphics\graphicx.sty
Package: graphicx 2021/09/16 v1.2d Enhanced LaTeX Graphics (DPC,SPQR)

(C:\Users\johann\AppData\Local\Programs\MiKTeX\tex/latex/graphics\graphics.sty
Package: graphics 2024/05/23 v1.4g Standard LaTeX Graphics (DPC,SPQR)

(C:\Users\johann\AppData\Local\Programs\MiKTeX\tex/latex/graphics\trig.sty
Package: trig 2023/12/02 v1.11 sin cos tan (DPC)
)
(C:\Users\johann\AppData\Local\Programs\MiKTeX\tex/latex/graphics-cfg\graphics.
cfg
File: graphics.cfg 2016/06/04 v1.11 sample graphics configuration
)
Package graphics Info: Driver file: pdftex.def on input line 106.

(C:\Users\johann\AppData\Local\Programs\MiKTeX\tex/latex/graphics-def\pdftex.de
f
File: pdftex.def 2024/04/13 v1.2c Graphics/color driver for pdftex
))
\Gin@req@height=\dimen160
\Gin@req@width=\dimen161
)
(C:\Users\johann\AppData\Local\Programs\MiKTeX\tex/latex/hyperref\hyperref.sty
Package: hyperref 2024-07-10 v7.01j Hypertext links for LaTeX

(C:\Users\johann\AppData\Local\Programs\MiKTeX\tex/latex/kvsetkeys\kvsetkeys.st
y
Package: kvsetkeys 2022-10-05 v1.19 Key value parser (HO)
)
(C:\Users\johann\AppData\Local\Programs\MiKTeX\tex/generic/kvdefinekeys\kvdefin
ekeys.sty
Package: kvdefinekeys 2019-12-19 v1.6 Define keys (HO)
)
(C:\Users\johann\AppData\Local\Programs\MiKTeX\tex/generic/pdfescape\pdfescape.
sty
Package: pdfescape 2019/12/09 v1.15 Implements pdfTeX's escape features (HO)

(C:\Users\johann\AppData\Local\Programs\MiKTeX\tex/generic/ltxcmds\ltxcmds.sty
Package: ltxcmds 2023-12-04 v1.26 LaTeX kernel commands for general use (HO)
)
(C:\Users\johann\AppData\Local\Programs\MiKTeX\tex/generic/pdftexcmds\pdftexcmd
s.sty
Package: pdftexcmds 2020-06-27 v0.33 Utility functions of pdfTeX for LuaTeX (HO
)

(C:\Users\johann\AppData\Local\Programs\MiKTeX\tex/generic/infwarerr\infwarerr.
sty
Package: infwarerr 2019/12/03 v1.5 Providing info/warning/error messages (HO)
)
Package pdftexcmds Info: \pdf@primitive is available.
Package pdftexcmds Info: \pdf@ifprimitive is available.
Package pdftexcmds Info: \pdfdraftmode found.
))
(C:\Users\johann\AppData\Local\Programs\MiKTeX\tex/latex/hycolor\hycolor.sty
Package: hycolor 2020-01-27 v1.10 Color options for hyperref/bookmark (HO)
)
(C:\Users\johann\AppData\Local\Programs\MiKTeX\tex/latex/hyperref\nameref.sty
Package: nameref 2023-11-26 v2.56 Cross-referencing by name of section

(C:\Users\johann\AppData\Local\Programs\MiKTeX\tex/latex/refcount\refcount.sty
Package: refcount 2019/12/15 v3.6 Data extraction from label references (HO)
)
(C:\Users\johann\AppData\Local\Programs\MiKTeX\tex/generic/gettitlestring\getti
tlestring.sty
Package: gettitlestring 2019/12/15 v1.6 Cleanup title references (HO)

(C:\Users\johann\AppData\Local\Programs\MiKTeX\tex/latex/kvoptions\kvoptions.st
y
Package: kvoptions 2022-06-15 v3.15 Key value format for package options (HO)
))
\c@section@level=\count286
)
(C:\Users\johann\AppData\Local\Programs\MiKTeX\tex/latex/etoolbox\etoolbox.sty
Package: etoolbox 2020/10/05 v2.5k e-TeX tools for LaTeX (JAW)
\etb@tempcnta=\count287
)
(C:\Users\johann\AppData\Local\Programs\MiKTeX\tex/generic/stringenc\stringenc.
sty
Package: stringenc 2019/11/29 v1.12 Convert strings between diff. encodings (HO
)
)
\@linkdim=\dimen162
\Hy@linkcounter=\count288
\Hy@pagecounter=\count289

(C:\Users\johann\AppData\Local\Programs\MiKTeX\tex/latex/hyperref\pd1enc.def
File: pd1enc.def 2024-07-10 v7.01j Hyperref: PDFDocEncoding definition (HO)
Now handling font encoding PD1 ...
... no UTF-8 mapping file for font encoding PD1
)
(C:\Users\johann\AppData\Local\Programs\MiKTeX\tex/generic/intcalc\intcalc.sty
Package: intcalc 2019/12/15 v1.3 Expandable calculations with integers (HO)
)
\Hy@SavedSpaceFactor=\count290
(C:\Users\johann\AppData\Local\Programs\MiKTeX\tex/latex/hyperref\puenc.def
File: puenc.def 2024-07-10 v7.01j Hyperref: PDF Unicode definition (HO)
Now handling font encoding PU ...
... no UTF-8 mapping file for font encoding PU
)
Package hyperref Info: Option `unicode' set `true' on input line 4040.
Package hyperref Info: Option `unicode' set `true' on input line 4040.
Package hyperref Info: Hyper figures OFF on input line 4157.
Package hyperref Info: Link nesting OFF on input line 4162.
Package hyperref Info: Hyper index ON on input line 4165.
Package hyperref Info: Plain pages OFF on input line 4172.
Package hyperref Info: Backreferencing OFF on input line 4177.
Package hyperref Info: Implicit mode ON; LaTeX internals redefined.
Package hyperref Info: Bookmarks ON on input line 4424.
\c@Hy@tempcnt=\count291

(C:\Users\johann\AppData\Local\Programs\MiKTeX\tex/latex/url\url.sty
\Urlmuskip=\muskip18
Package: url 2013/09/16  ver 3.4  Verb mode for urls, etc.
)
LaTeX Info: Redefining \url on input line 4763.
\XeTeXLinkMargin=\dimen163

(C:\Users\johann\AppData\Local\Programs\MiKTeX\tex/generic/bitset\bitset.sty
Package: bitset 2019/12/09 v1.3 Handle bit-vector datatype (HO)

(C:\Users\johann\AppData\Local\Programs\MiKTeX\tex/generic/bigintcalc\bigintcal
c.sty
Package: bigintcalc 2019/12/15 v1.5 Expandable calculations on big integers (HO
)
))
\Fld@menulength=\count292
\Field@Width=\dimen164
\Fld@charsize=\dimen165
Package hyperref Info: Hyper figures OFF on input line 6042.
Package hyperref Info: Link nesting OFF on input line 6047.
Package hyperref Info: Hyper index ON on input line 6050.
Package hyperref Info: backreferencing OFF on input line 6057.
Package hyperref Info: Link coloring OFF on input line 6062.
Package hyperref Info: Link coloring with OCG OFF on input line 6067.
Package hyperref Info: PDF/A mode OFF on input line 6072.

(C:\Users\johann\AppData\Local\Programs\MiKTeX\tex/latex/base\atbegshi-ltx.sty
Package: atbegshi-ltx 2021/01/10 v1.0c Emulation of the original atbegshi
package with kernel methods
)
\Hy@abspage=\count293
\c@Item=\count294
\c@Hfootnote=\count295
)
Package hyperref Info: Driver (autodetected): hpdftex.
 (C:\Users\johann\AppData\Local\Programs\MiKTeX\tex/latex/hyperref\hpdftex.def
File: hpdftex.def 2024-07-10 v7.01j Hyperref driver for pdfTeX

(C:\Users\johann\AppData\Local\Programs\MiKTeX\tex/latex/base\atveryend-ltx.sty
Package: atveryend-ltx 2020/08/19 v1.0a Emulation of the original atveryend pac
kage
with kernel methods
)
\Fld@listcount=\count296
\c@bookmark@seq@number=\count297

(C:\Users\johann\AppData\Local\Programs\MiKTeX\tex/latex/rerunfilecheck\rerunfi
lecheck.sty
Package: rerunfilecheck 2022-07-10 v1.10 Rerun checks for auxiliary files (HO)

(C:\Users\johann\AppData\Local\Programs\MiKTeX\tex/generic/uniquecounter\unique
counter.sty
Package: uniquecounter 2019/12/15 v1.4 Provide unlimited unique counter (HO)
)
Package uniquecounter Info: New unique counter `rerunfilecheck' on input line 2
85.
)
\Hy@SectionHShift=\skip57
)
(C:\Users\johann\AppData\Local\Programs\MiKTeX\tex/latex/booktabs\booktabs.sty
Package: booktabs 2020/01/12 v1.61803398 Publication quality tables
\heavyrulewidth=\dimen166
\lightrulewidth=\dimen167
\cmidrulewidth=\dimen168
\belowrulesep=\dimen169
\belowbottomsep=\dimen170
\aboverulesep=\dimen171
\abovetopsep=\dimen172
\cmidrulesep=\dimen173
\cmidrulekern=\dimen174
\defaultaddspace=\dimen175
\@cmidla=\count298
\@cmidlb=\count299
\@aboverulesep=\dimen176
\@belowrulesep=\dimen177
\@thisruleclass=\count300
\@lastruleclass=\count301
\@thisrulewidth=\dimen178
) (C:\Users\johann\AppData\Local\Programs\MiKTeX\tex/latex/tools\longtable.sty
Package: longtable 2024-04-26 v4.20 Multi-page Table package (DPC)
\LTleft=\skip58
\LTright=\skip59
\LTpre=\skip60
\LTpost=\skip61
\LTchunksize=\count302
\LTcapwidth=\dimen179
\LT@head=\box54
\LT@firsthead=\box55
\LT@foot=\box56
\LT@lastfoot=\box57
\LT@gbox=\box58
\LT@cols=\count303
\LT@rows=\count304
\c@LT@tables=\count305
\c@LT@chunks=\count306
\LT@p@ftn=\toks32
)
(C:\Users\johann\AppData\Local\Programs\MiKTeX\tex/latex/siunitx\siunitx.sty
Package: siunitx 2024-12-06 v3.4.0 A comprehensive (SI) units package
\l__siunitx_number_uncert_offset_int=\count307
\l__siunitx_number_exponent_fixed_int=\count308
\l__siunitx_number_min_decimal_int=\count309
\l__siunitx_number_min_integer_int=\count310
\l__siunitx_number_round_precision_int=\count311
\l__siunitx_number_lower_threshold_int=\count312
\l__siunitx_number_upper_threshold_int=\count313
\l__siunitx_number_group_first_int=\count314
\l__siunitx_number_group_size_int=\count315
\l__siunitx_number_group_minimum_int=\count316
\l__siunitx_angle_tmp_dim=\dimen180
\l__siunitx_angle_marker_box=\box59
\l__siunitx_angle_unit_box=\box60
\l__siunitx_compound_count_int=\count317

(C:\Users\johann\AppData\Local\Programs\MiKTeX\tex/latex/translations\translati
ons.sty
Package: translations 2022/02/05 v1.12 internationalization of LaTeX2e packages
 (CN)
)
\l__siunitx_table_tmp_box=\box61
\l__siunitx_table_tmp_dim=\dimen181
\l__siunitx_table_column_width_dim=\dimen182
\l__siunitx_table_integer_box=\box62
\l__siunitx_table_decimal_box=\box63
\l__siunitx_table_uncert_box=\box64
\l__siunitx_table_before_box=\box65
\l__siunitx_table_after_box=\box66
\l__siunitx_table_before_dim=\dimen183
\l__siunitx_table_carry_dim=\dimen184
\l__siunitx_unit_tmp_int=\count318
\l__siunitx_unit_position_int=\count319
\l__siunitx_unit_total_int=\count320

(C:\Users\johann\AppData\Local\Programs\MiKTeX\tex/latex/tools\array.sty
Package: array 2024/06/14 v2.6d Tabular extension package (FMi)
\col@sep=\dimen185
\ar@mcellbox=\box67
\extrarowheight=\dimen186
\NC@list=\toks33
\extratabsurround=\skip62
\backup@length=\skip63
\ar@cellbox=\box68
))
(C:\Program Files\MiKTeX\tex/latex/fancyhdr\fancyhdr.sty
Package: fancyhdr 2025/02/07 v5.2 Extensive control of page headers and footers

\f@nch@headwidth=\skip64
\f@nch@offset@elh=\skip65
\f@nch@offset@erh=\skip66
\f@nch@offset@olh=\skip67
\f@nch@offset@orh=\skip68
\f@nch@offset@elf=\skip69
\f@nch@offset@erf=\skip70
\f@nch@offset@olf=\skip71
\f@nch@offset@orf=\skip72
\f@nch@height=\skip73
\f@nch@footalignment=\skip74
\f@nch@widthL=\skip75
\f@nch@widthC=\skip76
\f@nch@widthR=\skip77
\@temptokenb=\toks34
)
(C:\Users\johann\AppData\Local\Programs\MiKTeX\tex/latex/float\float.sty
Package: float 2001/11/08 v1.3d Float enhancements (AL)
\c@float@type=\count321
\float@exts=\toks35
\float@box=\box69
\@float@everytoks=\toks36
\@floatcapt=\box70
)
(C:\Users\johann\AppData\Local\Programs\MiKTeX\tex/latex/pgf/frontendlayer\tikz
.sty
(C:\Users\johann\AppData\Local\Programs\MiKTeX\tex/latex/pgf/basiclayer\pgf.sty

(C:\Users\johann\AppData\Local\Programs\MiKTeX\tex/latex/pgf/utilities\pgfrcs.s
ty
(C:\Users\johann\AppData\Local\Programs\MiKTeX\tex/generic/pgf/utilities\pgfuti
l-common.tex
\pgfutil@everybye=\toks37
\pgfutil@tempdima=\dimen187
\pgfutil@tempdimb=\dimen188
)
(C:\Users\johann\AppData\Local\Programs\MiKTeX\tex/generic/pgf/utilities\pgfuti
l-latex.def
\pgfutil@abb=\box71
)
(C:\Users\johann\AppData\Local\Programs\MiKTeX\tex/generic/pgf/utilities\pgfrcs
.code.tex
(C:\Users\johann\AppData\Local\Programs\MiKTeX\tex/generic/pgf\pgf.revision.tex
)
Package: pgfrcs 2023-01-15 v3.1.10 (3.1.10)
))
Package: pgf 2023-01-15 v3.1.10 (3.1.10)

(C:\Users\johann\AppData\Local\Programs\MiKTeX\tex/latex/pgf/basiclayer\pgfcore
.sty
(C:\Users\johann\AppData\Local\Programs\MiKTeX\tex/latex/pgf/systemlayer\pgfsys
.sty
(C:\Users\johann\AppData\Local\Programs\MiKTeX\tex/generic/pgf/systemlayer\pgfs
ys.code.tex
Package: pgfsys 2023-01-15 v3.1.10 (3.1.10)

(C:\Users\johann\AppData\Local\Programs\MiKTeX\tex/generic/pgf/utilities\pgfkey
s.code.tex
\pgfkeys@pathtoks=\toks38
\pgfkeys@temptoks=\toks39

(C:\Users\johann\AppData\Local\Programs\MiKTeX\tex/generic/pgf/utilities\pgfkey
slibraryfiltered.code.tex
\pgfkeys@tmptoks=\toks40
))
\pgf@x=\dimen189
\pgf@y=\dimen190
\pgf@xa=\dimen191
\pgf@ya=\dimen192
\pgf@xb=\dimen193
\pgf@yb=\dimen194
\pgf@xc=\dimen195
\pgf@yc=\dimen196
\pgf@xd=\dimen197
\pgf@yd=\dimen198
\w@pgf@writea=\write3
\r@pgf@reada=\read3
\c@pgf@counta=\count322
\c@pgf@countb=\count323
\c@pgf@countc=\count324
\c@pgf@countd=\count325
\t@pgf@toka=\toks41
\t@pgf@tokb=\toks42
\t@pgf@tokc=\toks43
\pgf@sys@id@count=\count326

(C:\Users\johann\AppData\Local\Programs\MiKTeX\tex/generic/pgf/systemlayer\pgf.
cfg
File: pgf.cfg 2023-01-15 v3.1.10 (3.1.10)
)
Driver file for pgf: pgfsys-pdftex.def

(C:\Users\johann\AppData\Local\Programs\MiKTeX\tex/generic/pgf/systemlayer\pgfs
ys-pdftex.def
File: pgfsys-pdftex.def 2023-01-15 v3.1.10 (3.1.10)

(C:\Users\johann\AppData\Local\Programs\MiKTeX\tex/generic/pgf/systemlayer\pgfs
ys-common-pdf.def
File: pgfsys-common-pdf.def 2023-01-15 v3.1.10 (3.1.10)
)))
(C:\Users\johann\AppData\Local\Programs\MiKTeX\tex/generic/pgf/systemlayer\pgfs
yssoftpath.code.tex
File: pgfsyssoftpath.code.tex 2023-01-15 v3.1.10 (3.1.10)
\pgfsyssoftpath@smallbuffer@items=\count327
\pgfsyssoftpath@bigbuffer@items=\count328
)
(C:\Users\johann\AppData\Local\Programs\MiKTeX\tex/generic/pgf/systemlayer\pgfs
ysprotocol.code.tex
File: pgfsysprotocol.code.tex 2023-01-15 v3.1.10 (3.1.10)
))
(C:\Users\johann\AppData\Local\Programs\MiKTeX\tex/latex/xcolor\xcolor.sty
Package: xcolor 2023/11/15 v3.01 LaTeX color extensions (UK)

(C:\Users\johann\AppData\Local\Programs\MiKTeX\tex/latex/graphics-cfg\color.cfg
File: color.cfg 2016/01/02 v1.6 sample color configuration
)
Package xcolor Info: Driver file: pdftex.def on input line 274.

(C:\Users\johann\AppData\Local\Programs\MiKTeX\tex/latex/graphics\mathcolor.ltx
)
Package xcolor Info: Model `cmy' substituted by `cmy0' on input line 1350.
Package xcolor Info: Model `hsb' substituted by `rgb' on input line 1354.
Package xcolor Info: Model `RGB' extended on input line 1366.
Package xcolor Info: Model `HTML' substituted by `rgb' on input line 1368.
Package xcolor Info: Model `Hsb' substituted by `hsb' on input line 1369.
Package xcolor Info: Model `tHsb' substituted by `hsb' on input line 1370.
Package xcolor Info: Model `HSB' substituted by `hsb' on input line 1371.
Package xcolor Info: Model `Gray' substituted by `gray' on input line 1372.
Package xcolor Info: Model `wave' substituted by `hsb' on input line 1373.
)
(C:\Users\johann\AppData\Local\Programs\MiKTeX\tex/generic/pgf/basiclayer\pgfco
re.code.tex
Package: pgfcore 2023-01-15 v3.1.10 (3.1.10)

(C:\Users\johann\AppData\Local\Programs\MiKTeX\tex/generic/pgf/math\pgfmath.cod
e.tex
(C:\Users\johann\AppData\Local\Programs\MiKTeX\tex/generic/pgf/math\pgfmathutil
.code.tex)
(C:\Users\johann\AppData\Local\Programs\MiKTeX\tex/generic/pgf/math\pgfmathpars
er.code.tex
\pgfmath@dimen=\dimen199
\pgfmath@count=\count329
\pgfmath@box=\box72
\pgfmath@toks=\toks44
\pgfmath@stack@operand=\toks45
\pgfmath@stack@operation=\toks46
)
(C:\Users\johann\AppData\Local\Programs\MiKTeX\tex/generic/pgf/math\pgfmathfunc
tions.code.tex)
(C:\Users\johann\AppData\Local\Programs\MiKTeX\tex/generic/pgf/math\pgfmathfunc
tions.basic.code.tex)
(C:\Users\johann\AppData\Local\Programs\MiKTeX\tex/generic/pgf/math\pgfmathfunc
tions.trigonometric.code.tex)
(C:\Users\johann\AppData\Local\Programs\MiKTeX\tex/generic/pgf/math\pgfmathfunc
tions.random.code.tex)
(C:\Users\johann\AppData\Local\Programs\MiKTeX\tex/generic/pgf/math\pgfmathfunc
tions.comparison.code.tex)
(C:\Users\johann\AppData\Local\Programs\MiKTeX\tex/generic/pgf/math\pgfmathfunc
tions.base.code.tex)
(C:\Users\johann\AppData\Local\Programs\MiKTeX\tex/generic/pgf/math\pgfmathfunc
tions.round.code.tex)
(C:\Users\johann\AppData\Local\Programs\MiKTeX\tex/generic/pgf/math\pgfmathfunc
tions.misc.code.tex)
(C:\Users\johann\AppData\Local\Programs\MiKTeX\tex/generic/pgf/math\pgfmathfunc
tions.integerarithmetics.code.tex)
(C:\Users\johann\AppData\Local\Programs\MiKTeX\tex/generic/pgf/math\pgfmathcalc
.code.tex)
(C:\Users\johann\AppData\Local\Programs\MiKTeX\tex/generic/pgf/math\pgfmathfloa
t.code.tex
\c@pgfmathroundto@lastzeros=\count330
))
(C:\Users\johann\AppData\Local\Programs\MiKTeX\tex/generic/pgf/math\pgfint.code
.tex)
(C:\Users\johann\AppData\Local\Programs\MiKTeX\tex/generic/pgf/basiclayer\pgfco
repoints.code.tex
File: pgfcorepoints.code.tex 2023-01-15 v3.1.10 (3.1.10)
\pgf@picminx=\dimen256
\pgf@picmaxx=\dimen257
\pgf@picminy=\dimen258
\pgf@picmaxy=\dimen259
\pgf@pathminx=\dimen260
\pgf@pathmaxx=\dimen261
\pgf@pathminy=\dimen262
\pgf@pathmaxy=\dimen263
\pgf@xx=\dimen264
\pgf@xy=\dimen265
\pgf@yx=\dimen266
\pgf@yy=\dimen267
\pgf@zx=\dimen268
\pgf@zy=\dimen269
)
(C:\Users\johann\AppData\Local\Programs\MiKTeX\tex/generic/pgf/basiclayer\pgfco
repathconstruct.code.tex
File: pgfcorepathconstruct.code.tex 2023-01-15 v3.1.10 (3.1.10)
\pgf@path@lastx=\dimen270
\pgf@path@lasty=\dimen271
)
(C:\Users\johann\AppData\Local\Programs\MiKTeX\tex/generic/pgf/basiclayer\pgfco
repathusage.code.tex
File: pgfcorepathusage.code.tex 2023-01-15 v3.1.10 (3.1.10)
\pgf@shorten@end@additional=\dimen272
\pgf@shorten@start@additional=\dimen273
)
(C:\Users\johann\AppData\Local\Programs\MiKTeX\tex/generic/pgf/basiclayer\pgfco
rescopes.code.tex
File: pgfcorescopes.code.tex 2023-01-15 v3.1.10 (3.1.10)
\pgfpic=\box73
\pgf@hbox=\box74
\pgf@layerbox@main=\box75
\pgf@picture@serial@count=\count331
)
(C:\Users\johann\AppData\Local\Programs\MiKTeX\tex/generic/pgf/basiclayer\pgfco
regraphicstate.code.tex
File: pgfcoregraphicstate.code.tex 2023-01-15 v3.1.10 (3.1.10)
\pgflinewidth=\dimen274
)
(C:\Users\johann\AppData\Local\Programs\MiKTeX\tex/generic/pgf/basiclayer\pgfco
retransformations.code.tex
File: pgfcoretransformations.code.tex 2023-01-15 v3.1.10 (3.1.10)
\pgf@pt@x=\dimen275
\pgf@pt@y=\dimen276
\pgf@pt@temp=\dimen277
)
(C:\Users\johann\AppData\Local\Programs\MiKTeX\tex/generic/pgf/basiclayer\pgfco
requick.code.tex
File: pgfcorequick.code.tex 2023-01-15 v3.1.10 (3.1.10)
)
(C:\Users\johann\AppData\Local\Programs\MiKTeX\tex/generic/pgf/basiclayer\pgfco
reobjects.code.tex
File: pgfcoreobjects.code.tex 2023-01-15 v3.1.10 (3.1.10)
)
(C:\Users\johann\AppData\Local\Programs\MiKTeX\tex/generic/pgf/basiclayer\pgfco
repathprocessing.code.tex
File: pgfcorepathprocessing.code.tex 2023-01-15 v3.1.10 (3.1.10)
)
(C:\Users\johann\AppData\Local\Programs\MiKTeX\tex/generic/pgf/basiclayer\pgfco
rearrows.code.tex
File: pgfcorearrows.code.tex 2023-01-15 v3.1.10 (3.1.10)
\pgfarrowsep=\dimen278
)
(C:\Users\johann\AppData\Local\Programs\MiKTeX\tex/generic/pgf/basiclayer\pgfco
reshade.code.tex
File: pgfcoreshade.code.tex 2023-01-15 v3.1.10 (3.1.10)
\pgf@max=\dimen279
\pgf@sys@shading@range@num=\count332
\pgf@shadingcount=\count333
)
(C:\Users\johann\AppData\Local\Programs\MiKTeX\tex/generic/pgf/basiclayer\pgfco
reimage.code.tex
File: pgfcoreimage.code.tex 2023-01-15 v3.1.10 (3.1.10)
)
(C:\Users\johann\AppData\Local\Programs\MiKTeX\tex/generic/pgf/basiclayer\pgfco
reexternal.code.tex
File: pgfcoreexternal.code.tex 2023-01-15 v3.1.10 (3.1.10)
\pgfexternal@startupbox=\box76
)
(C:\Users\johann\AppData\Local\Programs\MiKTeX\tex/generic/pgf/basiclayer\pgfco
relayers.code.tex
File: pgfcorelayers.code.tex 2023-01-15 v3.1.10 (3.1.10)
)
(C:\Users\johann\AppData\Local\Programs\MiKTeX\tex/generic/pgf/basiclayer\pgfco
retransparency.code.tex
File: pgfcoretransparency.code.tex 2023-01-15 v3.1.10 (3.1.10)
)
(C:\Users\johann\AppData\Local\Programs\MiKTeX\tex/generic/pgf/basiclayer\pgfco
repatterns.code.tex
File: pgfcorepatterns.code.tex 2023-01-15 v3.1.10 (3.1.10)
)
(C:\Users\johann\AppData\Local\Programs\MiKTeX\tex/generic/pgf/basiclayer\pgfco
rerdf.code.tex
File: pgfcorerdf.code.tex 2023-01-15 v3.1.10 (3.1.10)
)))
(C:\Users\johann\AppData\Local\Programs\MiKTeX\tex/generic/pgf/modules\pgfmodul
eshapes.code.tex
File: pgfmoduleshapes.code.tex 2023-01-15 v3.1.10 (3.1.10)
\pgfnodeparttextbox=\box77
)
(C:\Users\johann\AppData\Local\Programs\MiKTeX\tex/generic/pgf/modules\pgfmodul
eplot.code.tex
File: pgfmoduleplot.code.tex 2023-01-15 v3.1.10 (3.1.10)
)
(C:\Users\johann\AppData\Local\Programs\MiKTeX\tex/latex/pgf/compatibility\pgfc
omp-version-0-65.sty
Package: pgfcomp-version-0-65 2023-01-15 v3.1.10 (3.1.10)
\pgf@nodesepstart=\dimen280
\pgf@nodesepend=\dimen281
)
(C:\Users\johann\AppData\Local\Programs\MiKTeX\tex/latex/pgf/compatibility\pgfc
omp-version-1-18.sty
Package: pgfcomp-version-1-18 2023-01-15 v3.1.10 (3.1.10)
))
(C:\Users\johann\AppData\Local\Programs\MiKTeX\tex/latex/pgf/utilities\pgffor.s
ty
(C:\Users\johann\AppData\Local\Programs\MiKTeX\tex/latex/pgf/utilities\pgfkeys.
sty
(C:\Users\johann\AppData\Local\Programs\MiKTeX\tex/generic/pgf/utilities\pgfkey
s.code.tex))
(C:\Users\johann\AppData\Local\Programs\MiKTeX\tex/latex/pgf/math\pgfmath.sty
(C:\Users\johann\AppData\Local\Programs\MiKTeX\tex/generic/pgf/math\pgfmath.cod
e.tex))
(C:\Users\johann\AppData\Local\Programs\MiKTeX\tex/generic/pgf/utilities\pgffor
.code.tex
Package: pgffor 2023-01-15 v3.1.10 (3.1.10)
\pgffor@iter=\dimen282
\pgffor@skip=\dimen283
\pgffor@stack=\toks47
\pgffor@toks=\toks48
))
(C:\Users\johann\AppData\Local\Programs\MiKTeX\tex/generic/pgf/frontendlayer/ti
kz\tikz.code.tex
Package: tikz 2023-01-15 v3.1.10 (3.1.10)

(C:\Users\johann\AppData\Local\Programs\MiKTeX\tex/generic/pgf/libraries\pgflib
raryplothandlers.code.tex
File: pgflibraryplothandlers.code.tex 2023-01-15 v3.1.10 (3.1.10)
\pgf@plot@mark@count=\count334
\pgfplotmarksize=\dimen284
)
\tikz@lastx=\dimen285
\tikz@lasty=\dimen286
\tikz@lastxsaved=\dimen287
\tikz@lastysaved=\dimen288
\tikz@lastmovetox=\dimen289
\tikz@lastmovetoy=\dimen290
\tikzleveldistance=\dimen291
\tikzsiblingdistance=\dimen292
\tikz@figbox=\box78
\tikz@figbox@bg=\box79
\tikz@tempbox=\box80
\tikz@tempbox@bg=\box81
\tikztreelevel=\count335
\tikznumberofchildren=\count336
\tikznumberofcurrentchild=\count337
\tikz@fig@count=\count338

(C:\Users\johann\AppData\Local\Programs\MiKTeX\tex/generic/pgf/modules\pgfmodul
ematrix.code.tex
File: pgfmodulematrix.code.tex 2023-01-15 v3.1.10 (3.1.10)
\pgfmatrixcurrentrow=\count339
\pgfmatrixcurrentcolumn=\count340
\pgf@matrix@numberofcolumns=\count341
)
\tikz@expandcount=\count342

(C:\Users\johann\AppData\Local\Programs\MiKTeX\tex/generic/pgf/frontendlayer/ti
kz/libraries\tikzlibrarytopaths.code.tex
File: tikzlibrarytopaths.code.tex 2023-01-15 v3.1.10 (3.1.10)
)))
(C:\Program Files\MiKTeX\tex/latex/setspace\setspace.sty
Package: setspace 2022/12/04 v6.7b set line spacing
)
(C:\Program Files\MiKTeX\tex/latex/enumitem\enumitem.sty
Package: enumitem 2019/06/20 v3.9 Customized lists
\labelindent=\skip78
\enit@outerparindent=\dimen293
\enit@toks=\toks49
\enit@inbox=\box82
\enit@count@id=\count343
\enitdp@description=\count344
)
(C:\Program Files\MiKTeX\tex/latex/adjustbox\adjustbox.sty
Package: adjustbox 2025/02/26 v1.3c Adjusting TeX boxes (trim, clip, ...)

(C:\Users\johann\AppData\Local\Programs\MiKTeX\tex/latex/xkeyval\xkeyval.sty
Package: xkeyval 2022/06/16 v2.9 package option processing (HA)

(C:\Users\johann\AppData\Local\Programs\MiKTeX\tex/generic/xkeyval\xkeyval.tex
(C:\Users\johann\AppData\Local\Programs\MiKTeX\tex/generic/xkeyval\xkvutils.tex
\XKV@toks=\toks50
\XKV@tempa@toks=\toks51
)
\XKV@depth=\count345
File: xkeyval.tex 2014/12/03 v2.7a key=value parser (HA)
)) (C:\Program Files\MiKTeX\tex/latex/adjustbox\adjcalc.sty
Package: adjcalc 2012/05/16 v1.1 Provides advanced setlength with multiple back
-ends (calc, etex, pgfmath)
)
(C:\Program Files\MiKTeX\tex/latex/adjustbox\trimclip.sty
Package: trimclip 2025/02/21 v1.2a Trim and clip general TeX material

(C:\Program Files\MiKTeX\tex/latex/collectbox\collectbox.sty
Package: collectbox 2022/10/17 v0.4c Collect macro arguments as boxes
\collectedbox=\box83
)
\tc@llx=\dimen294
\tc@lly=\dimen295
\tc@urx=\dimen296
\tc@ury=\dimen297
Package trimclip Info: Using driver 'tc-pdftex.def'.

(C:\Program Files\MiKTeX\tex/latex/adjustbox\tc-pdftex.def
File: tc-pdftex.def 2025/02/26 v2.3 Clipping driver for pdftex
))
\adjbox@Width=\dimen298
\adjbox@Height=\dimen299
\adjbox@Depth=\dimen300
\adjbox@Totalheight=\dimen301
\adjbox@pwidth=\dimen302
\adjbox@pheight=\dimen303
\adjbox@pdepth=\dimen304
\adjbox@ptotalheight=\dimen305

(C:\Program Files\MiKTeX\tex/latex/ifoddpage\ifoddpage.sty
Package: ifoddpage 2022/10/18 v1.2 Conditionals for odd/even page detection
\c@checkoddpage=\count346
)
(C:\Users\johann\AppData\Local\Programs\MiKTeX\tex/latex/varwidth\varwidth.sty
Package: varwidth 2009/03/30 ver 0.92;  Variable-width minipages
\@vwid@box=\box84
\sift@deathcycles=\count347
\@vwid@loff=\dimen306
\@vwid@roff=\dimen307
))
Package hyperref Info: Option `colorlinks' set `true' on input line 30.
LaTeX Font Info:    Trying to load font information for OT1+lmss on input line 
42.
 (C:\Users\johann\AppData\Local\Programs\MiKTeX\tex/latex/lm\ot1lmss.fd
File: ot1lmss.fd 2015/05/01 v1.6.1 Font defs for Latin Modern
)
(C:\Users\johann\AppData\Local\Programs\MiKTeX\tex/latex/l3backend\l3backend-pd
ftex.def
File: l3backend-pdftex.def 2024-05-08 L3 backend support: PDF output (pdfTeX)
\l__color_backend_stack_int=\count348
\l__pdf_internal_box=\box85
) (T0_Book_En.aux

LaTeX Warning: Label `sec:introduction' multiply defined.


LaTeX Warning: Label `eq:t0_fundamental' multiply defined.


LaTeX Warning: Label `eq:g_fundamental' multiply defined.


LaTeX Warning: Label `eq:characteristic_mass' multiply defined.


LaTeX Warning: Label `eq:G_complete' multiply defined.


LaTeX Warning: Label `tab:symbols' multiply defined.


LaTeX Warning: Label `tab:results' multiply defined.


LaTeX Warning: Label `sec:introduction' multiply defined.


LaTeX Warning: Label `subsec:future_research' multiply defined.


LaTeX Warning: Label `subsec:key_findings' multiply defined.


LaTeX Warning: Label `sec:introduction' multiply defined.


LaTeX Warning: Label `tab:results' multiply defined.


LaTeX Warning: Label `sec:introduction' multiply defined.


LaTeX Warning: Label `sec:parameters' multiply defined.


LaTeX Warning: Label `eq:xi' multiply defined.


LaTeX Warning: Label `eq:Df' multiply defined.


LaTeX Warning: Label `eq:K' multiply defined.


LaTeX Warning: Label `eq:E0' multiply defined.


LaTeX Warning: Label `eq:mT' multiply defined.


LaTeX Warning: Label `sec:summary' multiply defined.


LaTeX Warning: Label `eq:alpha' multiply defined.


LaTeX Warning: Label `eq:interaction_lagrangian' multiply defined.


LaTeX Warning: Label `eq:coupling_strength' multiply defined.


LaTeX Warning: Label `eq:extended_lagrangian' multiply defined.


LaTeX Warning: Label `eq:t0_fundamental_formula' multiply defined.


LaTeX Warning: Label `eq:final_t0_formula' multiply defined.


LaTeX Warning: Label `sec:introduction' multiply defined.


LaTeX Warning: Label `eq:time_field_lagrangian' multiply defined.


LaTeX Warning: Label `eq:higgs_connection' multiply defined.


LaTeX Warning: Label `sec:conclusions' multiply defined.


LaTeX Warning: Label `eq:time_mass_duality' multiply defined.


LaTeX Warning: Label `sec:introduction' multiply defined.


LaTeX Warning: Label `eq:xi_definition' multiply defined.


LaTeX Warning: Label `subsec:three_geometries' multiply defined.


LaTeX Warning: Label `eq:mass_field_solution' multiply defined.


LaTeX Warning: Label `eq:time_field_lagrangian' multiply defined.


LaTeX Warning: Label `eq:higgs_connection' multiply defined.


LaTeX Warning: Label `sec:experimental_predictions' multiply defined.


LaTeX Warning: Label `subsec:precision_tests' multiply defined.


LaTeX Warning: Label `subsec:modified_dirac' multiply defined.


LaTeX Warning: Label `eq:time_field_connection' multiply defined.


LaTeX Warning: Label `sec:conclusions' multiply defined.


LaTeX Warning: Label `subsec:summary_achievements' multiply defined.


LaTeX Warning: Label `subsec:key_insights' multiply defined.


LaTeX Warning: Label `sec:introduction' multiply defined.


LaTeX Warning: Label `subsec:parameter_problem' multiply defined.


LaTeX Warning: Label `sec:experimental_predictions' multiply defined.


LaTeX Warning: Label `eq:E0_fundamental' multiply defined.


LaTeX Warning: Label `eq:E0_logarithmic' multiply defined.


LaTeX Warning: Label `eq:log_symmetry' multiply defined.


LaTeX Warning: Label `eq:t0_fundamental' multiply defined.


LaTeX Warning: Label `eq:g_fundamental' multiply defined.


LaTeX Warning: Label `sec:sm_t0_mapping' multiply defined.


LaTeX Warning: Label `subsec:philosophical_implications' multiply defined.


LaTeX Warning: Label `eq:yukawa_mass_formula' multiply defined.


LaTeX Warning: Label `sec:introduction' multiply defined.


LaTeX Warning: Label `eq:intrinsic_time_field' multiply defined.


LaTeX Warning: Label `sec:comparison' multiply defined.


LaTeX Warning: Label `sec:summary' multiply defined.


LaTeX Warning: Label `sec:introduction' multiply defined.


LaTeX Warning: Label `sec:experimental_predictions' multiply defined.


LaTeX Warning: Label `subsec:precision_tests' multiply defined.


LaTeX Warning: Label `sec:conclusion' multiply defined.


LaTeX Warning: Label `subsec:key_findings' multiply defined.


LaTeX Warning: Label `subsec:theoretical_significance' multiply defined.


LaTeX Warning: Label `subsec:philosophical_implications' multiply defined.


LaTeX Warning: Label `eq:duality' multiply defined.


LaTeX Warning: Label `sec:introduction' multiply defined.


LaTeX Warning: Label `eq:t0_fundamental' multiply defined.


LaTeX Warning: Label `sec:t0_framework' multiply defined.


LaTeX Warning: Label `eq:t0_field_equation' multiply defined.


LaTeX Warning: Label `eq:t0_coupling_prediction' multiply defined.


LaTeX Warning: Label `sec:theoretical_implications' multiply defined.


LaTeX Warning: Label `sec:conclusions' multiply defined.


LaTeX Warning: Label `subsec:summary_achievements' multiply defined.


LaTeX Warning: Label `subsec:future_research' multiply defined.


LaTeX Warning: Label `sec:introduction' multiply defined.


LaTeX Warning: Label `sec:experimental_implications' multiply defined.


LaTeX Warning: Label `sec:conclusions' multiply defined.


LaTeX Warning: Label `subsec:theoretical_significance' multiply defined.


LaTeX Warning: Label `subsec:future_directions' multiply defined.


LaTeX Warning: Label `sec:final_remarks' multiply defined.


LaTeX Warning: Label `eq:intrinsic_time_field' multiply defined.


LaTeX Warning: Label `eq:field_equation' multiply defined.


LaTeX Warning: Label `eq:photon_time_field' multiply defined.


LaTeX Warning: Label `eq:universal_lagrangian' multiply defined.


LaTeX Warning: Label `eq:modified_schrodinger' multiply defined.


LaTeX Warning: Label `eq:bell_correction' multiply defined.


LaTeX Warning: Label `eq:wavefunction_field' multiply defined.


LaTeX Warning: Label `eq:probability_density' multiply defined.


LaTeX Warning: Label `eq:correlation_field' multiply defined.


LaTeX Warning: Label `eq:wavefunction_field' multiply defined.


LaTeX Warning: Label `eq:probability_density' multiply defined.


LaTeX Warning: Label `eq:wavefunction_field' multiply defined.


LaTeX Warning: Label `eq:probability_density' multiply defined.


LaTeX Warning: Label `eq:t0_schrodinger_general' multiply defined.


LaTeX Warning: Label `eq:t0_schrodinger_energy' multiply defined.


LaTeX Warning: Label `eq:t0_schrodinger_free' multiply defined.


LaTeX Warning: Label `eq:local_time_flow' multiply defined.


LaTeX Warning: Label `eq:plane_wave' multiply defined.


LaTeX Warning: Label `eq:modified_dispersion' multiply defined.


LaTeX Warning: Label `eq:energy_shift' multiply defined.


LaTeX Warning: Label `eq:measurement_interaction' multiply defined.


LaTeX Warning: Label `eq:measurement_probability' multiply defined.


LaTeX Warning: Label `eq:entangled_energy' multiply defined.


LaTeX Warning: Label `eq:singlet_correlation' multiply defined.


LaTeX Warning: Label `eq:position_momentum_correlation' multiply defined.


LaTeX Warning: Label `eq:field_correlation_function' multiply defined.


LaTeX Warning: Label `eq:standard_bell' multiply defined.


LaTeX Warning: Label `eq:modified_bell' multiply defined.


LaTeX Warning: Label `eq:t0_bell_correction' multiply defined.


LaTeX Warning: Label `eq:spectroscopic_shift' multiply defined.


LaTeX Warning: Label `eq:phase_accumulation' multiply defined.


LaTeX Warning: Label `eq:t0_schrodinger_general' multiply defined.


LaTeX Warning: Label `eq:t0_schrodinger_energy' multiply defined.


LaTeX Warning: Label `eq:t0_schrodinger_free' multiply defined.


LaTeX Warning: Label `eq:local_time_flow' multiply defined.


LaTeX Warning: Label `eq:plane_wave' multiply defined.


LaTeX Warning: Label `eq:modified_dispersion' multiply defined.


LaTeX Warning: Label `eq:energy_shift' multiply defined.


LaTeX Warning: Label `eq:measurement_interaction' multiply defined.


LaTeX Warning: Label `eq:measurement_probability' multiply defined.


LaTeX Warning: Label `eq:entangled_energy' multiply defined.


LaTeX Warning: Label `eq:singlet_correlation' multiply defined.


LaTeX Warning: Label `eq:position_momentum_correlation' multiply defined.


LaTeX Warning: Label `eq:field_correlation_function' multiply defined.


LaTeX Warning: Label `eq:standard_bell' multiply defined.


LaTeX Warning: Label `eq:modified_bell' multiply defined.


LaTeX Warning: Label `eq:t0_bell_correction' multiply defined.


LaTeX Warning: Label `eq:spectroscopic_shift' multiply defined.


LaTeX Warning: Label `eq:phase_accumulation' multiply defined.


LaTeX Warning: Label `tab:dimensions' multiply defined.


LaTeX Warning: Label `eq:field_equation' multiply defined.


LaTeX Warning: Label `eq:wave_equation' multiply defined.


LaTeX Warning: Label `eq:time_energy_duality' multiply defined.


LaTeX Warning: Label `eq:intrinsic_time_field' multiply defined.


LaTeX Warning: Label `eq:energy_field_equation' multiply defined.


LaTeX Warning: Label `eq:time_field_solution' multiply defined.


LaTeX Warning: Label `eq:xi_effective' multiply defined.


LaTeX Warning: Label `eq:modified_schrodinger' multiply defined.


LaTeX Warning: Label `eq:universal_lagrangian' multiply defined.


LaTeX Warning: Label `eq:time_energy_duality' multiply defined.


LaTeX Warning: Label `eq:field_equation' multiply defined.


LaTeX Warning: Label `eq:modified_schrodinger' multiply defined.


LaTeX Warning: Label `eq:universal_wave_equation' multiply defined.


LaTeX Warning: Label `sec:experimental_predictions' multiply defined.


LaTeX Warning: Label `sec:mass_elimination' multiply defined.


LaTeX Warning: Label `eq:complete_energy_solution' multiply defined.


LaTeX Warning: Label `subsec:emergent_mass' multiply defined.


LaTeX Warning: Label `sec:energy_fields_to_masses' multiply defined.


LaTeX Warning: Label `subsec:fundamental_challenge' multiply defined.


LaTeX Warning: Label `subsec:energy_based_mass' multiply defined.


LaTeX Warning: Label `eq:mass_to_energy' multiply defined.


LaTeX Warning: Label `sec:two_calculation_methods' multiply defined.


LaTeX Warning: Label `subsec:direct_geometric_method' multiply defined.


LaTeX Warning: Label `eq:geometric_quantization' multiply defined.


LaTeX Warning: Label `eq:resonance_frequencies' multiply defined.


LaTeX Warning: Label `eq:energy_from_frequency' multiply defined.


LaTeX Warning: Label `eq:characteristic_energy_direct' multiply defined.


LaTeX Warning: Label `subsec:extended_yukawa_method' multiply defined.


LaTeX Warning: Label `eq:yukawa_mass_formula' multiply defined.


LaTeX Warning: Label `eq:geometric_yukawa' multiply defined.


LaTeX Warning: Label `sec:physical_interpretation' multiply defined.


LaTeX Warning: Label `subsec:physical_interpretation' multiply defined.


LaTeX Warning: Label `eq:universal_scaling' multiply defined.


LaTeX Warning: Label `sec:physical_interpretation' multiply defined.


LaTeX Warning: Label `sec:experimental_tests' multiply defined.


LaTeX Warning: Label `sec:conclusion' multiply defined.


LaTeX Warning: Label `eq:universal_field_equation' multiply defined.


LaTeX Warning: Label `eq:modified_schrodinger' multiply defined.


LaTeX Warning: Label `sec:experimental_predictions' multiply defined.


LaTeX Warning: Label `subsec:geometric_foundation' multiply defined.


LaTeX Warning: Label `subsec:natural_scale_relationships' multiply defined.


LaTeX Warning: Label `tab:energy_scales_no_xi' multiply defined.


LaTeX Warning: Label `eq:universal_field_equation' multiply defined.


LaTeX Warning: Label `sec:experimental_verification' multiply defined.


LaTeX Warning: Label `sec:philosophical_implications' multiply defined.


LaTeX Warning: Label `eq:universal_wave_equation' multiply defined.


LaTeX Warning: Label `eq:universal_lagrangian' multiply defined.


LaTeX Warning: Label `sec:experimental_predictions' multiply defined.


LaTeX Warning: Label `sec:conclusion' multiply defined.


LaTeX Warning: Label `eq:xi_exact' multiply defined.


LaTeX Warning: Label `sec:experimental_verification' multiply defined.


LaTeX Warning: Label `tab:parameter_comparison' multiply defined.


LaTeX Warning: Label `subsec:geometric_foundation' multiply defined.


LaTeX Warning: Label `sec:philosophical_implications' multiply defined.


LaTeX Warning: Label `subsec:pythagorean_physics' multiply defined.


LaTeX Warning: Label `subsec:observer_independent_reality' multiply defined.


LaTeX Warning: Label `tab:results' multiply defined.


LaTeX Warning: Label `tab:string_comparison' multiply defined.


LaTeX Warning: Label `tab:extended_comparison' multiply defined.


LaTeX Warning: Label `tab:pre2025' multiply defined.


LaTeX Warning: Label `tab:hybrid_pure' multiply defined.


LaTeX Warning: Label `tab:uncertainties' multiply defined.


LaTeX Warning: Label `tab:hybrid_inconsistency' multiply defined.


LaTeX Warning: Label `tab:embedding_electron' multiply defined.


LaTeX Warning: Label `tab:prototype_comparison' multiply defined.


LaTeX Warning: Label `sec:intro' multiply defined.


LaTeX Warning: Label `tab:comparison' multiply defined.


LaTeX Warning: Label `pascher_deterministic_qm_2025' multiply defined.


LaTeX Warning: Label `peskin_schroeder' multiply defined.


LaTeX Warning: Label `pascher_xi_parameter_2025' multiply defined.


LaTeX Warning: Label `casimir1948' multiply defined.


LaTeX Warning: Label `weinberg1989' multiply defined.


LaTeX Warning: Label `planck2018' multiply defined.


LaTeX Warning: Label `heisenberg1927' multiply defined.


LaTeX Warning: Label `planck2020' multiply defined.


LaTeX Warning: Label `casimir1948' multiply defined.


LaTeX Warning: Label `hossenfelder2025' multiply defined.


LaTeX Warning: Label `pdg2024' multiply defined.


LaTeX Warning: Label `planck2018' multiply defined.


LaTeX Warning: Label `riess2022' multiply defined.


LaTeX Warning: Label `pascher_t0_2025' multiply defined.


LaTeX Warning: Label `pdg_2024' multiply defined.


LaTeX Warning: Label `pascher:t0_foundations' multiply defined.


LaTeX Warning: Label `T0_tm_erweiterung' multiply defined.


LaTeX Warning: Label `T0_g2_erweiterung' multiply defined.


LaTeX Warning: Label `sm_g2_2025' multiply defined.


LaTeX Warning: Label `pdg_2024' multiply defined.


LaTeX Warning: Label `particle_data_group_2022' multiply defined.


LaTeX Warning: Label `planck_collaboration_2020' multiply defined.


LaTeX Warning: Label `weinberg_qft1' multiply defined.


LaTeX Warning: Label `peskin_schroeder' multiply defined.


LaTeX Warning: Label `planck2020' multiply defined.


LaTeX Warning: Label `casimir1948' multiply defined.


LaTeX Warning: Label `riess2022' multiply defined.


LaTeX Warning: Label `pascher_cosmic_2025' multiply defined.


LaTeX Warning: Label `pascher_t0_energie_2025' multiply defined.


LaTeX Warning: Label `pascher_redshift_2025' multiply defined.


LaTeX Warning: Label `planck_2020' multiply defined.


LaTeX Warning: Label `pascher_t0_energie_2025' multiply defined.


LaTeX Warning: Label `heisenberg1927' multiply defined.


LaTeX Warning: Label `einstein1915' multiply defined.


LaTeX Warning: Label `einstein1905' multiply defined.


LaTeX Warning: Label `dirac1928' multiply defined.


LaTeX Warning: Label `feynman1949' multiply defined.


LaTeX Warning: Label `higgs1964' multiply defined.


LaTeX Warning: Label `weinberg1967' multiply defined.


LaTeX Warning: Label `weinberg1989' multiply defined.


LaTeX Warning: Label `yang1954' multiply defined.


LaTeX Warning: Label `yukawa1935' multiply defined.


LaTeX Warning: Label `bohr1928' multiply defined.


LaTeX Warning: Label `planck2020' multiply defined.


LaTeX Warning: Label `riess2022' multiply defined.


LaTeX Warning: Label `jwst_early' multiply defined.


LaTeX Warning: Label `cobe1992' multiply defined.


LaTeX Warning: Label `codata2018' multiply defined.


LaTeX Warning: Label `casimir1948' multiply defined.


LaTeX Warning: Label `lamoreaux1997' multiply defined.


LaTeX Warning: Label `muon_g2_2021' multiply defined.


LaTeX Warning: Label `ludlow2015' multiply defined.


LaTeX Warning: Label `ashby2003' multiply defined.


LaTeX Warning: Label `peskin1995' multiply defined.


LaTeX Warning: Label `heisenberg1927' multiply defined.


LaTeX Warning: Label `planck2020' multiply defined.


LaTeX Warning: Label `codata2018' multiply defined.


LaTeX Warning: Label `casimir1948' multiply defined.


LaTeX Warning: Label `muon_g2_2021' multiply defined.


LaTeX Warning: Label `riess2022' multiply defined.


LaTeX Warning: Label `jwst_early' multiply defined.


LaTeX Warning: Label `cobe1992' multiply defined.


LaTeX Warning: Label `sparnaay1958' multiply defined.


LaTeX Warning: Label `lamoreaux1997' multiply defined.


LaTeX Warning: Label `einstein1915' multiply defined.


LaTeX Warning: Label `peskin_schroeder' multiply defined.


LaTeX Warning: Label `pdg_2022' multiply defined.


LaTeX Warning: Label `peskin_schroeder' multiply defined.


LaTeX Warning: Label `pdg_2022' multiply defined.


LaTeX Warning: Label `pascher_beta_derivation_2025' multiply defined.


LaTeX Warning: Label `PDG2020' multiply defined.


LaTeX Warning: Label `Einstein1905' multiply defined.


LaTeX Warning: Label `pascher_simplified_dirac_2025' multiply defined.


LaTeX Warning: Label `pascher_universal_lagrangian_2025' multiply defined.


LaTeX Warning: Label `pascher_mass_elimination_2025' multiply defined.


LaTeX Warning: Label `pascher_deterministic_qm_2025' multiply defined.


LaTeX Warning: Label `particle_data_group_2022' multiply defined.


LaTeX Warning: Label `weinberg_qft1' multiply defined.


LaTeX Warning: Label `peskin_schroeder' multiply defined.


LaTeX Warning: Label `muong2_experiment_2021' multiply defined.


LaTeX Warning: Label `higgs_discovery_atlas' multiply defined.


LaTeX Warning: Label `planck_collaboration_2020' multiply defined.


LaTeX Warning: Label `higgs1964' multiply defined.


LaTeX Warning: Label `weinberg1967' multiply defined.


LaTeX Warning: Label `pollard1975' multiply defined.


LaTeX Warning: Label `Jackson1999' multiply defined.


LaTeX Warning: Label `Weinberg1995' multiply defined.


LaTeX Warning: Label `higgs1964' multiply defined.


LaTeX Warning: Label `codata2018' multiply defined.


LaTeX Warning: Label `nielsen_chuang2010' multiply defined.


LaTeX Warning: Label `bell1964' multiply defined.


LaTeX Warning: Label `aspect1982' multiply defined.


LaTeX Warning: Label `pascher_dirac_2025' multiply defined.


LaTeX Warning: Label `einstein1905' multiply defined.


LaTeX Warning: Label `pascher_derivation_beta_2025' multiply defined.


LaTeX Warning: Label `pascher_simplified_dirac_2025' multiply defined.


LaTeX Warning: Label `pascher_ratio_physics_2025' multiply defined.


LaTeX Warning: Label `planck1900' multiply defined.


LaTeX Warning: Label `lorentz1904' multiply defined.


LaTeX Warning: Label `pascher_beta_derivation_2025' multiply defined.


LaTeX Warning: Label `pascher_simplified_dirac_2025' multiply defined.


LaTeX Warning: Label `pascher_verification_table_2025' multiply defined.


LaTeX Warning: Label `muong2_experiment_2021' multiply defined.


LaTeX Warning: Label `planck_collaboration_2020' multiply defined.


LaTeX Warning: Label `particle_data_group_2022' multiply defined.


LaTeX Warning: Label `pascher_dirac_2025' multiply defined.


LaTeX Warning: Label `bell1964' multiply defined.


LaTeX Warning: Label `muon_g2_2021' multiply defined.


LaTeX Warning: Label `einstein1905' multiply defined.


LaTeX Warning: Label `dirac1928' multiply defined.


LaTeX Warning: Label `shor1994' multiply defined.


LaTeX Warning: Label `quantenjahr25' multiply defined.


LaTeX Warning: Label `tfln_foundry' multiply defined.


LaTeX Warning: Label `phoquant' multiply defined.


LaTeX Warning: Label `quantenjahr25' multiply defined.


LaTeX Warning: Label `thz_epfl' multiply defined.


LaTeX Warning: Label `hhi_6g' multiply defined.


LaTeX Warning: Label `planck_units' multiply defined.


LaTeX Warning: Label `nufit_2024' multiply defined.


LaTeX Warning: Label `t0_neutrinos' multiply defined.


LaTeX Warning: Label `t0_kosmologie' multiply defined.


LaTeX Warning: Label `planck1900' multiply defined.


LaTeX Warning: Label `particle_data_group_2022' multiply defined.


LaTeX Warning: Label `weinberg_qft1' multiply defined.


LaTeX Warning: Label `weinberg_1989' multiply defined.


LaTeX Warning: Label `dirac_principles' multiply defined.


LaTeX Warning: Label `ligo_collaboration_2016' multiply defined.


LaTeX Warning: Label `T0_SI' multiply defined.


LaTeX Warning: Label `QFT_T0' multiply defined.


LaTeX Warning: Label `Fermilab2025' multiply defined.


LaTeX Warning: Label `CODATA2025' multiply defined.


LaTeX Warning: Label `BelleII2025' multiply defined.


LaTeX Warning: Label `T0_Calc' multiply defined.


LaTeX Warning: Label `T0_gravitational_constant' multiply defined.


LaTeX Warning: Label `T0_fine_structure' multiply defined.


LaTeX Warning: Label `T0_ratio_absolute' multiply defined.


LaTeX Warning: Label `Hierarchy' multiply defined.


LaTeX Warning: Label `Fermilab2023' multiply defined.


LaTeX Warning: Label `Hanneke2008' multiply defined.


LaTeX Warning: Label `DELPHI2004' multiply defined.


LaTeX Warning: Label `bell_muon' multiply defined.


LaTeX Warning: Label `CODATA2022' multiply defined.


LaTeX Warning: Label `Feynman1985' multiply defined.


LaTeX Warning: Label `Sommerfeld1916' multiply defined.


LaTeX Warning: Label `Brannen2005' multiply defined.


LaTeX Warning: Label `PhaseVectors2025' multiply defined.

)
\openout1 = `T0_Book_En.aux'.

LaTeX Font Info:    Checking defaults for OML/cmm/m/it on input line 42.
LaTeX Font Info:    ... okay on input line 42.
LaTeX Font Info:    Checking defaults for OMS/cmsy/m/n on input line 42.
LaTeX Font Info:    ... okay on input line 42.
LaTeX Font Info:    Checking defaults for OT1/cmr/m/n on input line 42.
LaTeX Font Info:    ... okay on input line 42.
LaTeX Font Info:    Checking defaults for T1/cmr/m/n on input line 42.
LaTeX Font Info:    ... okay on input line 42.
LaTeX Font Info:    Checking defaults for TS1/cmr/m/n on input line 42.
LaTeX Font Info:    ... okay on input line 42.
LaTeX Font Info:    Checking defaults for OMX/cmex/m/n on input line 42.
LaTeX Font Info:    ... okay on input line 42.
LaTeX Font Info:    Checking defaults for U/cmr/m/n on input line 42.
LaTeX Font Info:    ... okay on input line 42.
LaTeX Font Info:    Checking defaults for PD1/pdf/m/n on input line 42.
LaTeX Font Info:    ... okay on input line 42.
LaTeX Font Info:    Checking defaults for PU/pdf/m/n on input line 42.
LaTeX Font Info:    ... okay on input line 42.

*geometry* driver: auto-detecting
*geometry* detected driver: pdftex
*geometry* verbose mode - [ preamble ] result:
* driver: pdftex
* paper: a4paper
* layout: <same size as paper>
* layoutoffset:(h,v)=(0.0pt,0.0pt)
* modes: twoside 
* h-part:(L,W,R)=(56.9055pt, 483.69687pt, 56.9055pt)
* v-part:(T,H,B)=(56.9055pt, 731.23584pt, 56.9055pt)
* \paperwidth=597.50787pt
* \paperheight=845.04684pt
* \textwidth=483.69687pt
* \textheight=731.23584pt
* \oddsidemargin=-15.36449pt
* \evensidemargin=-15.36449pt
* \topmargin=-47.23828pt
* \headheight=12.0pt
* \headsep=19.8738pt
* \topskip=11.0pt
* \footskip=27.46295pt
* \marginparwidth=106.0pt
* \marginparsep=7.0pt
* \columnsep=10.0pt
* \skip\footins=10.0pt plus 4.0pt minus 2.0pt
* \hoffset=0.0pt
* \voffset=0.0pt
* \mag=1000
* \@twocolumnfalse
* \@twosidetrue
* \@mparswitchtrue
* \@reversemarginfalse
* (1in=72.27pt=25.4mm, 1cm=28.453pt)


(C:\Users\johann\AppData\Local\Programs\MiKTeX\tex/context/base/mkii\supp-pdf.m
kii
[Loading MPS to PDF converter (version 2006.09.02).]
\scratchcounter=\count349
\scratchdimen=\dimen308
\scratchbox=\box86
\nofMPsegments=\count350
\nofMParguments=\count351
\everyMPshowfont=\toks52
\MPscratchCnt=\count352
\MPscratchDim=\dimen309
\MPnumerator=\count353
\makeMPintoPDFobject=\count354
\everyMPtoPDFconversion=\toks53
)
(C:\Users\johann\AppData\Local\Programs\MiKTeX\tex/latex/epstopdf-pkg\epstopdf-
base.sty
Package: epstopdf-base 2020-01-24 v2.11 Base part for package epstopdf
Package epstopdf-base Info: Redefining graphics rule for `.eps' on input line 4
85.

(C:\Users\johann\AppData\Local\Programs\MiKTeX\tex/latex/00miktex\epstopdf-sys.
cfg
File: epstopdf-sys.cfg 2021/03/18 v2.0 Configuration of epstopdf for MiKTeX
))
Package hyperref Info: Link coloring ON on input line 42.
 (T0_Book_En.out) (T0_Book_En.out)
\@outlinefile=\write4
\openout4 = `T0_Book_En.out'.


(C:\Users\johann\AppData\Local\Programs\MiKTeX\tex/latex/translations/dicts\tra
nslations-basic-dictionary-english.trsl
File: translations-basic-dictionary-english.trsl (english translation file `tra
nslations-basic-dictionary')
)
Package translations Info: loading dictionary `translations-basic-dictionary' f
or `english'. on input line 42.
LaTeX Font Info:    Trying to load font information for OT1+lmr on input line 4
5.

(C:\Users\johann\AppData\Local\Programs\MiKTeX\tex/latex/lm\ot1lmr.fd
File: ot1lmr.fd 2015/05/01 v1.6.1 Font defs for Latin Modern
)
LaTeX Font Info:    Trying to load font information for OML+lmm on input line 4
5.

(C:\Users\johann\AppData\Local\Programs\MiKTeX\tex/latex/lm\omllmm.fd
File: omllmm.fd 2015/05/01 v1.6.1 Font defs for Latin Modern
)
LaTeX Font Info:    Trying to load font information for OMS+lmsy on input line 
45.

(C:\Users\johann\AppData\Local\Programs\MiKTeX\tex/latex/lm\omslmsy.fd
File: omslmsy.fd 2015/05/01 v1.6.1 Font defs for Latin Modern
)
LaTeX Font Info:    Trying to load font information for OMX+lmex on input line 
45.

(C:\Users\johann\AppData\Local\Programs\MiKTeX\tex/latex/lm\omxlmex.fd
File: omxlmex.fd 2015/05/01 v1.6.1 Font defs for Latin Modern
)
LaTeX Font Info:    External font `lmex10' loaded for size
(Font)              <12> on input line 45.
LaTeX Font Info:    External font `lmex10' loaded for size
(Font)              <8> on input line 45.
LaTeX Font Info:    External font `lmex10' loaded for size
(Font)              <6> on input line 45.
LaTeX Font Info:    Trying to load font information for U+msa on input line 45.


(C:\Users\johann\AppData\Local\Programs\MiKTeX\tex/latex/amsfonts\umsa.fd
File: umsa.fd 2013/01/14 v3.01 AMS symbols A
)
LaTeX Font Info:    Trying to load font information for U+msb on input line 45.


(C:\Users\johann\AppData\Local\Programs\MiKTeX\tex/latex/amsfonts\umsb.fd
File: umsb.fd 2013/01/14 v3.01 AMS symbols B
)


(C:\Users\johann\AppData\Local\Programs\MiKTeX\tex/generic/stringenc\se-pdfdoc.
def
File: se-pdfdoc.def 2019/11/29 v1.12 stringenc: PDFDocEncoding
) [1


{C:/Users/johann/AppData/Local/MiKTeX/fonts/map/pdftex/pdftex.map}{C:/Users/joh
ann/AppData/Local/Programs/MiKTeX/fonts/enc/dvips/lm/lm-rm.enc}]
(T0_Book_En.toc
LaTeX Font Info:    External font `lmex10' loaded for size
(Font)              <10.95> on input line 3.

Underfull \vbox (badness 1142) has occurred while \output is active []



[2

{C:/Users/johann/AppData/Local/Programs/MiKTeX/fonts/enc/dvips/lm/lm-mathit.enc
}]


Package fancyhdr Warning: \headheight is too small (12.0pt): 
(fancyhdr)                Make it at least 13.59999pt, for example:
(fancyhdr)                \setlength{\headheight}{13.59999pt}.
(fancyhdr)                You might also make \topmargin smaller:
(fancyhdr)                \addtolength{\topmargin}{-1.59999pt}.

[3]


Package fancyhdr Warning: \headheight is too small (12.0pt): 
(fancyhdr)                Make it at least 13.59999pt, for example:
(fancyhdr)                \setlength{\headheight}{13.59999pt}.
(fancyhdr)                You might also make \topmargin smaller:
(fancyhdr)                \addtolength{\topmargin}{-1.59999pt}.

[4]


Package fancyhdr Warning: \headheight is too small (12.0pt): 
(fancyhdr)                Make it at least 13.59999pt, for example:
(fancyhdr)                \setlength{\headheight}{13.59999pt}.
(fancyhdr)                You might also make \topmargin smaller:
(fancyhdr)                \addtolength{\topmargin}{-1.59999pt}.

[5]


Package fancyhdr Warning: \headheight is too small (12.0pt): 
(fancyhdr)                Make it at least 13.59999pt, for example:
(fancyhdr)                \setlength{\headheight}{13.59999pt}.
(fancyhdr)                You might also make \topmargin smaller:
(fancyhdr)                \addtolength{\topmargin}{-1.59999pt}.

[6]


Package fancyhdr Warning: \headheight is too small (12.0pt): 
(fancyhdr)                Make it at least 13.59999pt, for example:
(fancyhdr)                \setlength{\headheight}{13.59999pt}.
(fancyhdr)                You might also make \topmargin smaller:
(fancyhdr)                \addtolength{\topmargin}{-1.59999pt}.

[7]


Package fancyhdr Warning: \headheight is too small (12.0pt): 
(fancyhdr)                Make it at least 13.59999pt, for example:
(fancyhdr)                \setlength{\headheight}{13.59999pt}.
(fancyhdr)                You might also make \topmargin smaller:
(fancyhdr)                \addtolength{\topmargin}{-1.59999pt}.

[8]


Package fancyhdr Warning: \headheight is too small (12.0pt): 
(fancyhdr)                Make it at least 13.59999pt, for example:
(fancyhdr)                \setlength{\headheight}{13.59999pt}.
(fancyhdr)                You might also make \topmargin smaller:
(fancyhdr)                \addtolength{\topmargin}{-1.59999pt}.

[9]


Package fancyhdr Warning: \headheight is too small (12.0pt): 
(fancyhdr)                Make it at least 13.59999pt, for example:
(fancyhdr)                \setlength{\headheight}{13.59999pt}.
(fancyhdr)                You might also make \topmargin smaller:
(fancyhdr)                \addtolength{\topmargin}{-1.59999pt}.

[10{C:/Users/johann/AppData/Local/Programs/MiKTeX/fonts/enc/dvips/lm/lm-mathsy.
enc}]


Package fancyhdr Warning: \headheight is too small (12.0pt): 
(fancyhdr)                Make it at least 13.59999pt, for example:
(fancyhdr)                \setlength{\headheight}{13.59999pt}.
(fancyhdr)                You might also make \topmargin smaller:
(fancyhdr)                \addtolength{\topmargin}{-1.59999pt}.

[11]


Package fancyhdr Warning: \headheight is too small (12.0pt): 
(fancyhdr)                Make it at least 13.59999pt, for example:
(fancyhdr)                \setlength{\headheight}{13.59999pt}.
(fancyhdr)                You might also make \topmargin smaller:
(fancyhdr)                \addtolength{\topmargin}{-1.59999pt}.

[12]


Package fancyhdr Warning: \headheight is too small (12.0pt): 
(fancyhdr)                Make it at least 13.59999pt, for example:
(fancyhdr)                \setlength{\headheight}{13.59999pt}.
(fancyhdr)                You might also make \topmargin smaller:
(fancyhdr)                \addtolength{\topmargin}{-1.59999pt}.

[13]


Package fancyhdr Warning: \headheight is too small (12.0pt): 
(fancyhdr)                Make it at least 13.59999pt, for example:
(fancyhdr)                \setlength{\headheight}{13.59999pt}.
(fancyhdr)                You might also make \topmargin smaller:
(fancyhdr)                \addtolength{\topmargin}{-1.59999pt}.

[14]


Package fancyhdr Warning: \headheight is too small (12.0pt): 
(fancyhdr)                Make it at least 13.59999pt, for example:
(fancyhdr)                \setlength{\headheight}{13.59999pt}.
(fancyhdr)                You might also make \topmargin smaller:
(fancyhdr)                \addtolength{\topmargin}{-1.59999pt}.

[15]


Package fancyhdr Warning: \headheight is too small (12.0pt): 
(fancyhdr)                Make it at least 13.59999pt, for example:
(fancyhdr)                \setlength{\headheight}{13.59999pt}.
(fancyhdr)                You might also make \topmargin smaller:
(fancyhdr)                \addtolength{\topmargin}{-1.59999pt}.

[16]


Package fancyhdr Warning: \headheight is too small (12.0pt): 
(fancyhdr)                Make it at least 13.59999pt, for example:
(fancyhdr)                \setlength{\headheight}{13.59999pt}.
(fancyhdr)                You might also make \topmargin smaller:
(fancyhdr)                \addtolength{\topmargin}{-1.59999pt}.

[17]


Package fancyhdr Warning: \headheight is too small (12.0pt): 
(fancyhdr)                Make it at least 13.59999pt, for example:
(fancyhdr)                \setlength{\headheight}{13.59999pt}.
(fancyhdr)                You might also make \topmargin smaller:
(fancyhdr)                \addtolength{\topmargin}{-1.59999pt}.

[18]


Package fancyhdr Warning: \headheight is too small (12.0pt): 
(fancyhdr)                Make it at least 13.59999pt, for example:
(fancyhdr)                \setlength{\headheight}{13.59999pt}.
(fancyhdr)                You might also make \topmargin smaller:
(fancyhdr)                \addtolength{\topmargin}{-1.59999pt}.

[19]


Package fancyhdr Warning: \headheight is too small (12.0pt): 
(fancyhdr)                Make it at least 13.59999pt, for example:
(fancyhdr)                \setlength{\headheight}{13.59999pt}.
(fancyhdr)                You might also make \topmargin smaller:
(fancyhdr)                \addtolength{\topmargin}{-1.59999pt}.

[20]
Underfull \vbox (badness 2653) has occurred while \output is active []




Package fancyhdr Warning: \headheight is too small (12.0pt): 
(fancyhdr)                Make it at least 13.59999pt, for example:
(fancyhdr)                \setlength{\headheight}{13.59999pt}.
(fancyhdr)                You might also make \topmargin smaller:
(fancyhdr)                \addtolength{\topmargin}{-1.59999pt}.

[21]


Package fancyhdr Warning: \headheight is too small (12.0pt): 
(fancyhdr)                Make it at least 13.59999pt, for example:
(fancyhdr)                \setlength{\headheight}{13.59999pt}.
(fancyhdr)                You might also make \topmargin smaller:
(fancyhdr)                \addtolength{\topmargin}{-1.59999pt}.

[22]


Package fancyhdr Warning: \headheight is too small (12.0pt): 
(fancyhdr)                Make it at least 13.59999pt, for example:
(fancyhdr)                \setlength{\headheight}{13.59999pt}.
(fancyhdr)                You might also make \topmargin smaller:
(fancyhdr)                \addtolength{\topmargin}{-1.59999pt}.

[23]


Package fancyhdr Warning: \headheight is too small (12.0pt): 
(fancyhdr)                Make it at least 13.59999pt, for example:
(fancyhdr)                \setlength{\headheight}{13.59999pt}.
(fancyhdr)                You might also make \topmargin smaller:
(fancyhdr)                \addtolength{\topmargin}{-1.59999pt}.

[24]


Package fancyhdr Warning: \headheight is too small (12.0pt): 
(fancyhdr)                Make it at least 13.59999pt, for example:
(fancyhdr)                \setlength{\headheight}{13.59999pt}.
(fancyhdr)                You might also make \topmargin smaller:
(fancyhdr)                \addtolength{\topmargin}{-1.59999pt}.

[25]


Package fancyhdr Warning: \headheight is too small (12.0pt): 
(fancyhdr)                Make it at least 13.59999pt, for example:
(fancyhdr)                \setlength{\headheight}{13.59999pt}.
(fancyhdr)                You might also make \topmargin smaller:
(fancyhdr)                \addtolength{\topmargin}{-1.59999pt}.

[26]


Package fancyhdr Warning: \headheight is too small (12.0pt): 
(fancyhdr)                Make it at least 13.59999pt, for example:
(fancyhdr)                \setlength{\headheight}{13.59999pt}.
(fancyhdr)                You might also make \topmargin smaller:
(fancyhdr)                \addtolength{\topmargin}{-1.59999pt}.

[27]


Package fancyhdr Warning: \headheight is too small (12.0pt): 
(fancyhdr)                Make it at least 13.59999pt, for example:
(fancyhdr)                \setlength{\headheight}{13.59999pt}.
(fancyhdr)                You might also make \topmargin smaller:
(fancyhdr)                \addtolength{\topmargin}{-1.59999pt}.

[28]


Package fancyhdr Warning: \headheight is too small (12.0pt): 
(fancyhdr)                Make it at least 13.59999pt, for example:
(fancyhdr)                \setlength{\headheight}{13.59999pt}.
(fancyhdr)                You might also make \topmargin smaller:
(fancyhdr)                \addtolength{\topmargin}{-1.59999pt}.

[29]


Package fancyhdr Warning: \headheight is too small (12.0pt): 
(fancyhdr)                Make it at least 13.59999pt, for example:
(fancyhdr)                \setlength{\headheight}{13.59999pt}.
(fancyhdr)                You might also make \topmargin smaller:
(fancyhdr)                \addtolength{\topmargin}{-1.59999pt}.

[30]


Package fancyhdr Warning: \headheight is too small (12.0pt): 
(fancyhdr)                Make it at least 13.59999pt, for example:
(fancyhdr)                \setlength{\headheight}{13.59999pt}.
(fancyhdr)                You might also make \topmargin smaller:
(fancyhdr)                \addtolength{\topmargin}{-1.59999pt}.

[31]


Package fancyhdr Warning: \headheight is too small (12.0pt): 
(fancyhdr)                Make it at least 13.59999pt, for example:
(fancyhdr)                \setlength{\headheight}{13.59999pt}.
(fancyhdr)                You might also make \topmargin smaller:
(fancyhdr)                \addtolength{\topmargin}{-1.59999pt}.

[32]
LaTeX Font Info:    Trying to load font information for TS1+lmss on input line 
1267.
 (C:\Users\johann\AppData\Local\Programs\MiKTeX\tex/latex/lm\ts1lmss.fd
File: ts1lmss.fd 2015/05/01 v1.6.1 Font defs for Latin Modern
)


Package fancyhdr Warning: \headheight is too small (12.0pt): 
(fancyhdr)                Make it at least 13.59999pt, for example:
(fancyhdr)                \setlength{\headheight}{13.59999pt}.
(fancyhdr)                You might also make \topmargin smaller:
(fancyhdr)                \addtolength{\topmargin}{-1.59999pt}.

[33{C:/Users/johann/AppData/Local/Programs/MiKTeX/fonts/enc/dvips/lm/lm-ts1.enc
}]


Package fancyhdr Warning: \headheight is too small (12.0pt): 
(fancyhdr)                Make it at least 13.59999pt, for example:
(fancyhdr)                \setlength{\headheight}{13.59999pt}.
(fancyhdr)                You might also make \topmargin smaller:
(fancyhdr)                \addtolength{\topmargin}{-1.59999pt}.

[34]
Underfull \vbox (badness 2221) has occurred while \output is active []




Package fancyhdr Warning: \headheight is too small (12.0pt): 
(fancyhdr)                Make it at least 13.59999pt, for example:
(fancyhdr)                \setlength{\headheight}{13.59999pt}.
(fancyhdr)                You might also make \topmargin smaller:
(fancyhdr)                \addtolength{\topmargin}{-1.59999pt}.

[35]


Package fancyhdr Warning: \headheight is too small (12.0pt): 
(fancyhdr)                Make it at least 13.59999pt, for example:
(fancyhdr)                \setlength{\headheight}{13.59999pt}.
(fancyhdr)                You might also make \topmargin smaller:
(fancyhdr)                \addtolength{\topmargin}{-1.59999pt}.

[36]


Package fancyhdr Warning: \headheight is too small (12.0pt): 
(fancyhdr)                Make it at least 13.59999pt, for example:
(fancyhdr)                \setlength{\headheight}{13.59999pt}.
(fancyhdr)                You might also make \topmargin smaller:
(fancyhdr)                \addtolength{\topmargin}{-1.59999pt}.

[37]


Package fancyhdr Warning: \headheight is too small (12.0pt): 
(fancyhdr)                Make it at least 13.59999pt, for example:
(fancyhdr)                \setlength{\headheight}{13.59999pt}.
(fancyhdr)                You might also make \topmargin smaller:
(fancyhdr)                \addtolength{\topmargin}{-1.59999pt}.

[38]


Package fancyhdr Warning: \headheight is too small (12.0pt): 
(fancyhdr)                Make it at least 13.59999pt, for example:
(fancyhdr)                \setlength{\headheight}{13.59999pt}.
(fancyhdr)                You might also make \topmargin smaller:
(fancyhdr)                \addtolength{\topmargin}{-1.59999pt}.

[39]


Package fancyhdr Warning: \headheight is too small (12.0pt): 
(fancyhdr)                Make it at least 13.59999pt, for example:
(fancyhdr)                \setlength{\headheight}{13.59999pt}.
(fancyhdr)                You might also make \topmargin smaller:
(fancyhdr)                \addtolength{\topmargin}{-1.59999pt}.

[40]
Underfull \vbox (badness 5711) has occurred while \output is active []




Package fancyhdr Warning: \headheight is too small (12.0pt): 
(fancyhdr)                Make it at least 13.59999pt, for example:
(fancyhdr)                \setlength{\headheight}{13.59999pt}.
(fancyhdr)                You might also make \topmargin smaller:
(fancyhdr)                \addtolength{\topmargin}{-1.59999pt}.

[41]


Package fancyhdr Warning: \headheight is too small (12.0pt): 
(fancyhdr)                Make it at least 13.59999pt, for example:
(fancyhdr)                \setlength{\headheight}{13.59999pt}.
(fancyhdr)                You might also make \topmargin smaller:
(fancyhdr)                \addtolength{\topmargin}{-1.59999pt}.

[42]


Package fancyhdr Warning: \headheight is too small (12.0pt): 
(fancyhdr)                Make it at least 13.59999pt, for example:
(fancyhdr)                \setlength{\headheight}{13.59999pt}.
(fancyhdr)                You might also make \topmargin smaller:
(fancyhdr)                \addtolength{\topmargin}{-1.59999pt}.

[43]


Package fancyhdr Warning: \headheight is too small (12.0pt): 
(fancyhdr)                Make it at least 13.59999pt, for example:
(fancyhdr)                \setlength{\headheight}{13.59999pt}.
(fancyhdr)                You might also make \topmargin smaller:
(fancyhdr)                \addtolength{\topmargin}{-1.59999pt}.

[44]


Package fancyhdr Warning: \headheight is too small (12.0pt): 
(fancyhdr)                Make it at least 13.59999pt, for example:
(fancyhdr)                \setlength{\headheight}{13.59999pt}.
(fancyhdr)                You might also make \topmargin smaller:
(fancyhdr)                \addtolength{\topmargin}{-1.59999pt}.

[45]


Package fancyhdr Warning: \headheight is too small (12.0pt): 
(fancyhdr)                Make it at least 13.59999pt, for example:
(fancyhdr)                \setlength{\headheight}{13.59999pt}.
(fancyhdr)                You might also make \topmargin smaller:
(fancyhdr)                \addtolength{\topmargin}{-1.59999pt}.

[46]


Package fancyhdr Warning: \headheight is too small (12.0pt): 
(fancyhdr)                Make it at least 13.59999pt, for example:
(fancyhdr)                \setlength{\headheight}{13.59999pt}.
(fancyhdr)                You might also make \topmargin smaller:
(fancyhdr)                \addtolength{\topmargin}{-1.59999pt}.

[47]


Package fancyhdr Warning: \headheight is too small (12.0pt): 
(fancyhdr)                Make it at least 13.59999pt, for example:
(fancyhdr)                \setlength{\headheight}{13.59999pt}.
(fancyhdr)                You might also make \topmargin smaller:
(fancyhdr)                \addtolength{\topmargin}{-1.59999pt}.

[48]


Package fancyhdr Warning: \headheight is too small (12.0pt): 
(fancyhdr)                Make it at least 13.59999pt, for example:
(fancyhdr)                \setlength{\headheight}{13.59999pt}.
(fancyhdr)                You might also make \topmargin smaller:
(fancyhdr)                \addtolength{\topmargin}{-1.59999pt}.

[49]


Package fancyhdr Warning: \headheight is too small (12.0pt): 
(fancyhdr)                Make it at least 13.59999pt, for example:
(fancyhdr)                \setlength{\headheight}{13.59999pt}.
(fancyhdr)                You might also make \topmargin smaller:
(fancyhdr)                \addtolength{\topmargin}{-1.59999pt}.

[50]


Package fancyhdr Warning: \headheight is too small (12.0pt): 
(fancyhdr)                Make it at least 13.59999pt, for example:
(fancyhdr)                \setlength{\headheight}{13.59999pt}.
(fancyhdr)                You might also make \topmargin smaller:
(fancyhdr)                \addtolength{\topmargin}{-1.59999pt}.

[51]


Package fancyhdr Warning: \headheight is too small (12.0pt): 
(fancyhdr)                Make it at least 13.59999pt, for example:
(fancyhdr)                \setlength{\headheight}{13.59999pt}.
(fancyhdr)                You might also make \topmargin smaller:
(fancyhdr)                \addtolength{\topmargin}{-1.59999pt}.

[52]


Package fancyhdr Warning: \headheight is too small (12.0pt): 
(fancyhdr)                Make it at least 13.59999pt, for example:
(fancyhdr)                \setlength{\headheight}{13.59999pt}.
(fancyhdr)                You might also make \topmargin smaller:
(fancyhdr)                \addtolength{\topmargin}{-1.59999pt}.

[53]


Package fancyhdr Warning: \headheight is too small (12.0pt): 
(fancyhdr)                Make it at least 13.59999pt, for example:
(fancyhdr)                \setlength{\headheight}{13.59999pt}.
(fancyhdr)                You might also make \topmargin smaller:
(fancyhdr)                \addtolength{\topmargin}{-1.59999pt}.

[54]


Package fancyhdr Warning: \headheight is too small (12.0pt): 
(fancyhdr)                Make it at least 13.59999pt, for example:
(fancyhdr)                \setlength{\headheight}{13.59999pt}.
(fancyhdr)                You might also make \topmargin smaller:
(fancyhdr)                \addtolength{\topmargin}{-1.59999pt}.

[55]


Package fancyhdr Warning: \headheight is too small (12.0pt): 
(fancyhdr)                Make it at least 13.59999pt, for example:
(fancyhdr)                \setlength{\headheight}{13.59999pt}.
(fancyhdr)                You might also make \topmargin smaller:
(fancyhdr)                \addtolength{\topmargin}{-1.59999pt}.

[56]


Package fancyhdr Warning: \headheight is too small (12.0pt): 
(fancyhdr)                Make it at least 13.59999pt, for example:
(fancyhdr)                \setlength{\headheight}{13.59999pt}.
(fancyhdr)                You might also make \topmargin smaller:
(fancyhdr)                \addtolength{\topmargin}{-1.59999pt}.

[57]


Package fancyhdr Warning: \headheight is too small (12.0pt): 
(fancyhdr)                Make it at least 13.59999pt, for example:
(fancyhdr)                \setlength{\headheight}{13.59999pt}.
(fancyhdr)                You might also make \topmargin smaller:
(fancyhdr)                \addtolength{\topmargin}{-1.59999pt}.

[58]


Package fancyhdr Warning: \headheight is too small (12.0pt): 
(fancyhdr)                Make it at least 13.59999pt, for example:
(fancyhdr)                \setlength{\headheight}{13.59999pt}.
(fancyhdr)                You might also make \topmargin smaller:
(fancyhdr)                \addtolength{\topmargin}{-1.59999pt}.

[59]


Package fancyhdr Warning: \headheight is too small (12.0pt): 
(fancyhdr)                Make it at least 13.59999pt, for example:
(fancyhdr)                \setlength{\headheight}{13.59999pt}.
(fancyhdr)                You might also make \topmargin smaller:
(fancyhdr)                \addtolength{\topmargin}{-1.59999pt}.

[60]


Package fancyhdr Warning: \headheight is too small (12.0pt): 
(fancyhdr)                Make it at least 13.59999pt, for example:
(fancyhdr)                \setlength{\headheight}{13.59999pt}.
(fancyhdr)                You might also make \topmargin smaller:
(fancyhdr)                \addtolength{\topmargin}{-1.59999pt}.

[61]
Underfull \hbox (badness 10000) in paragraph at lines 2361--2361
 [] [][][]\OT1/lmss/bx/n/10.95 Introduction: The Uni-ver-sal  
 []

! Extra }, or forgotten $.
<argument> ... The Universal $$-Constant (cosmic)}
                                                  \Hy@toclinkend 
l.2361 ...l $$-Constant (cosmic)}{674}{appendix.R}
                                                  %
I've deleted a group-closing symbol because it seems to be
spurious, as in `$x}$'. But perhaps the } is legitimate and
you forgot something else, as in `\hbox{$x}'. In such cases
the way to recover is to insert both the forgotten and the
deleted material, e.g., by typing `I$}'.

! Missing $ inserted.
<inserted text> 
                $
l.2361 ...l $$-Constant (cosmic)}{674}{appendix.R}
                                                  %
I've inserted something that you may have forgotten.
(See the <inserted text> above.)
With luck, this will get me unwedged. But if you
really didn't forget anything, try typing `2' now; then
my insertion and my current dilemma will both disappear.

! Display math should end with $$.
<to be read again> 
                   \endgroup 
l.2361 ...l $$-Constant (cosmic)}{674}{appendix.R}
                                                  %
The `$' that I just saw supposedly matches a previous `$$'.
So I shall assume that you typed `$$' both times.

! Missing } inserted.
<inserted text> 
                }
l.2361 ...l $$-Constant (cosmic)}{674}{appendix.R}
                                                  %
I've inserted something that you may have forgotten.
(See the <inserted text> above.)
With luck, this will get me unwedged. But if you
really didn't forget anything, try typing `2' now; then
my insertion and my current dilemma will both disappear.




Package fancyhdr Warning: \headheight is too small (12.0pt): 
(fancyhdr)                Make it at least 13.59999pt, for example:
(fancyhdr)                \setlength{\headheight}{13.59999pt}.
(fancyhdr)                You might also make \topmargin smaller:
(fancyhdr)                \addtolength{\topmargin}{-1.59999pt}.

[62]


Package fancyhdr Warning: \headheight is too small (12.0pt): 
(fancyhdr)                Make it at least 13.59999pt, for example:
(fancyhdr)                \setlength{\headheight}{13.59999pt}.
(fancyhdr)                You might also make \topmargin smaller:
(fancyhdr)                \addtolength{\topmargin}{-1.59999pt}.

[63]


Package fancyhdr Warning: \headheight is too small (12.0pt): 
(fancyhdr)                Make it at least 13.59999pt, for example:
(fancyhdr)                \setlength{\headheight}{13.59999pt}.
(fancyhdr)                You might also make \topmargin smaller:
(fancyhdr)                \addtolength{\topmargin}{-1.59999pt}.

[64]


Package fancyhdr Warning: \headheight is too small (12.0pt): 
(fancyhdr)                Make it at least 13.59999pt, for example:
(fancyhdr)                \setlength{\headheight}{13.59999pt}.
(fancyhdr)                You might also make \topmargin smaller:
(fancyhdr)                \addtolength{\topmargin}{-1.59999pt}.

[65]


Package fancyhdr Warning: \headheight is too small (12.0pt): 
(fancyhdr)                Make it at least 13.59999pt, for example:
(fancyhdr)                \setlength{\headheight}{13.59999pt}.
(fancyhdr)                You might also make \topmargin smaller:
(fancyhdr)                \addtolength{\topmargin}{-1.59999pt}.

[66]


Package fancyhdr Warning: \headheight is too small (12.0pt): 
(fancyhdr)                Make it at least 13.59999pt, for example:
(fancyhdr)                \setlength{\headheight}{13.59999pt}.
(fancyhdr)                You might also make \topmargin smaller:
(fancyhdr)                \addtolength{\topmargin}{-1.59999pt}.

[67]


Package fancyhdr Warning: \headheight is too small (12.0pt): 
(fancyhdr)                Make it at least 13.59999pt, for example:
(fancyhdr)                \setlength{\headheight}{13.59999pt}.
(fancyhdr)                You might also make \topmargin smaller:
(fancyhdr)                \addtolength{\topmargin}{-1.59999pt}.

[68]


Package fancyhdr Warning: \headheight is too small (12.0pt): 
(fancyhdr)                Make it at least 13.59999pt, for example:
(fancyhdr)                \setlength{\headheight}{13.59999pt}.
(fancyhdr)                You might also make \topmargin smaller:
(fancyhdr)                \addtolength{\topmargin}{-1.59999pt}.

[69]


Package fancyhdr Warning: \headheight is too small (12.0pt): 
(fancyhdr)                Make it at least 13.59999pt, for example:
(fancyhdr)                \setlength{\headheight}{13.59999pt}.
(fancyhdr)                You might also make \topmargin smaller:
(fancyhdr)                \addtolength{\topmargin}{-1.59999pt}.

[70]


Package fancyhdr Warning: \headheight is too small (12.0pt): 
(fancyhdr)                Make it at least 13.59999pt, for example:
(fancyhdr)                \setlength{\headheight}{13.59999pt}.
(fancyhdr)                You might also make \topmargin smaller:
(fancyhdr)                \addtolength{\topmargin}{-1.59999pt}.

[71]


Package fancyhdr Warning: \headheight is too small (12.0pt): 
(fancyhdr)                Make it at least 13.59999pt, for example:
(fancyhdr)                \setlength{\headheight}{13.59999pt}.
(fancyhdr)                You might also make \topmargin smaller:
(fancyhdr)                \addtolength{\topmargin}{-1.59999pt}.

[72]


Package fancyhdr Warning: \headheight is too small (12.0pt): 
(fancyhdr)                Make it at least 13.59999pt, for example:
(fancyhdr)                \setlength{\headheight}{13.59999pt}.
(fancyhdr)                You might also make \topmargin smaller:
(fancyhdr)                \addtolength{\topmargin}{-1.59999pt}.

[73]


Package fancyhdr Warning: \headheight is too small (12.0pt): 
(fancyhdr)                Make it at least 13.59999pt, for example:
(fancyhdr)                \setlength{\headheight}{13.59999pt}.
(fancyhdr)                You might also make \topmargin smaller:
(fancyhdr)                \addtolength{\topmargin}{-1.59999pt}.

[74]


Package fancyhdr Warning: \headheight is too small (12.0pt): 
(fancyhdr)                Make it at least 13.59999pt, for example:
(fancyhdr)                \setlength{\headheight}{13.59999pt}.
(fancyhdr)                You might also make \topmargin smaller:
(fancyhdr)                \addtolength{\topmargin}{-1.59999pt}.

[75]


Package fancyhdr Warning: \headheight is too small (12.0pt): 
(fancyhdr)                Make it at least 13.59999pt, for example:
(fancyhdr)                \setlength{\headheight}{13.59999pt}.
(fancyhdr)                You might also make \topmargin smaller:
(fancyhdr)                \addtolength{\topmargin}{-1.59999pt}.

[76]


Package fancyhdr Warning: \headheight is too small (12.0pt): 
(fancyhdr)                Make it at least 13.59999pt, for example:
(fancyhdr)                \setlength{\headheight}{13.59999pt}.
(fancyhdr)                You might also make \topmargin smaller:
(fancyhdr)                \addtolength{\topmargin}{-1.59999pt}.

[77]


Package fancyhdr Warning: \headheight is too small (12.0pt): 
(fancyhdr)                Make it at least 13.59999pt, for example:
(fancyhdr)                \setlength{\headheight}{13.59999pt}.
(fancyhdr)                You might also make \topmargin smaller:
(fancyhdr)                \addtolength{\topmargin}{-1.59999pt}.

[78]


Package fancyhdr Warning: \headheight is too small (12.0pt): 
(fancyhdr)                Make it at least 13.59999pt, for example:
(fancyhdr)                \setlength{\headheight}{13.59999pt}.
(fancyhdr)                You might also make \topmargin smaller:
(fancyhdr)                \addtolength{\topmargin}{-1.59999pt}.

[79]


Package fancyhdr Warning: \headheight is too small (12.0pt): 
(fancyhdr)                Make it at least 13.59999pt, for example:
(fancyhdr)                \setlength{\headheight}{13.59999pt}.
(fancyhdr)                You might also make \topmargin smaller:
(fancyhdr)                \addtolength{\topmargin}{-1.59999pt}.

[80]


Package fancyhdr Warning: \headheight is too small (12.0pt): 
(fancyhdr)                Make it at least 13.59999pt, for example:
(fancyhdr)                \setlength{\headheight}{13.59999pt}.
(fancyhdr)                You might also make \topmargin smaller:
(fancyhdr)                \addtolength{\topmargin}{-1.59999pt}.

[81]


Package fancyhdr Warning: \headheight is too small (12.0pt): 
(fancyhdr)                Make it at least 13.59999pt, for example:
(fancyhdr)                \setlength{\headheight}{13.59999pt}.
(fancyhdr)                You might also make \topmargin smaller:
(fancyhdr)                \addtolength{\topmargin}{-1.59999pt}.

[82]


Package fancyhdr Warning: \headheight is too small (12.0pt): 
(fancyhdr)                Make it at least 13.59999pt, for example:
(fancyhdr)                \setlength{\headheight}{13.59999pt}.
(fancyhdr)                You might also make \topmargin smaller:
(fancyhdr)                \addtolength{\topmargin}{-1.59999pt}.

[83]


Package fancyhdr Warning: \headheight is too small (12.0pt): 
(fancyhdr)                Make it at least 13.59999pt, for example:
(fancyhdr)                \setlength{\headheight}{13.59999pt}.
(fancyhdr)                You might also make \topmargin smaller:
(fancyhdr)                \addtolength{\topmargin}{-1.59999pt}.

[84]


Package fancyhdr Warning: \headheight is too small (12.0pt): 
(fancyhdr)                Make it at least 13.59999pt, for example:
(fancyhdr)                \setlength{\headheight}{13.59999pt}.
(fancyhdr)                You might also make \topmargin smaller:
(fancyhdr)                \addtolength{\topmargin}{-1.59999pt}.

[85]


Package fancyhdr Warning: \headheight is too small (12.0pt): 
(fancyhdr)                Make it at least 13.59999pt, for example:
(fancyhdr)                \setlength{\headheight}{13.59999pt}.
(fancyhdr)                You might also make \topmargin smaller:
(fancyhdr)                \addtolength{\topmargin}{-1.59999pt}.

[86]
Overfull \hbox (13.09625pt too wide) in paragraph at lines 3442--3442
 [] [][][]\OT1/lmss/bx/n/10.95 T0 Model Ver-i-fi-ca-tion: Scale Ratio-Based Cal
-cu-la-tions (Elim-i-na-tion Of Mass Dirac TabelleEn)[][] []  
 []




Package fancyhdr Warning: \headheight is too small (12.0pt): 
(fancyhdr)                Make it at least 13.59999pt, for example:
(fancyhdr)                \setlength{\headheight}{13.59999pt}.
(fancyhdr)                You might also make \topmargin smaller:
(fancyhdr)                \addtolength{\topmargin}{-1.59999pt}.

[87]


Package fancyhdr Warning: \headheight is too small (12.0pt): 
(fancyhdr)                Make it at least 13.59999pt, for example:
(fancyhdr)                \setlength{\headheight}{13.59999pt}.
(fancyhdr)                You might also make \topmargin smaller:
(fancyhdr)                \addtolength{\topmargin}{-1.59999pt}.

[88]


Package fancyhdr Warning: \headheight is too small (12.0pt): 
(fancyhdr)                Make it at least 13.59999pt, for example:
(fancyhdr)                \setlength{\headheight}{13.59999pt}.
(fancyhdr)                You might also make \topmargin smaller:
(fancyhdr)                \addtolength{\topmargin}{-1.59999pt}.

[89]


Package fancyhdr Warning: \headheight is too small (12.0pt): 
(fancyhdr)                Make it at least 13.59999pt, for example:
(fancyhdr)                \setlength{\headheight}{13.59999pt}.
(fancyhdr)                You might also make \topmargin smaller:
(fancyhdr)                \addtolength{\topmargin}{-1.59999pt}.

[90]


Package fancyhdr Warning: \headheight is too small (12.0pt): 
(fancyhdr)                Make it at least 13.59999pt, for example:
(fancyhdr)                \setlength{\headheight}{13.59999pt}.
(fancyhdr)                You might also make \topmargin smaller:
(fancyhdr)                \addtolength{\topmargin}{-1.59999pt}.

[91]
Overfull \hbox (4.92744pt too wide) detected at line 3659
 []\OT1/lmss/m/n/10.95 1000
 []


Overfull \hbox (4.92744pt too wide) detected at line 3662
 []\OT1/lmss/m/n/10.95 1000
 []


Overfull \hbox (4.92744pt too wide) detected at line 3665
 []\OT1/lmss/m/n/10.95 1001
 []


Overfull \hbox (4.92744pt too wide) detected at line 3668
 []\OT1/lmss/m/n/10.95 1001
 []


Overfull \hbox (4.92744pt too wide) detected at line 3669
 []\OT1/lmss/m/n/10.95 1001
 []


Overfull \hbox (4.92744pt too wide) detected at line 3672
 []\OT1/lmss/m/n/10.95 1002
 []


Overfull \hbox (4.92744pt too wide) detected at line 3675
 []\OT1/lmss/m/n/10.95 1002
 []


Overfull \hbox (4.92744pt too wide) detected at line 3678
 []\OT1/lmss/m/n/10.95 1003
 []


Overfull \hbox (4.92744pt too wide) detected at line 3679
 []\OT1/lmss/m/n/10.95 1003
 []


Overfull \hbox (4.92744pt too wide) detected at line 3681
 []\OT1/lmss/m/n/10.95 1005
 []


Overfull \hbox (4.92744pt too wide) detected at line 3682
 []\OT1/lmss/m/n/10.95 1005
 []


Overfull \hbox (4.92744pt too wide) detected at line 3683
 []\OT1/lmss/m/n/10.95 1006
 []


Overfull \hbox (4.92744pt too wide) detected at line 3684
 []\OT1/lmss/m/n/10.95 1006
 []


Overfull \hbox (4.92744pt too wide) detected at line 3685
 []\OT1/lmss/m/n/10.95 1006
 []


Overfull \hbox (4.92744pt too wide) detected at line 3686
 []\OT1/lmss/m/n/10.95 1006
 []




Package fancyhdr Warning: \headheight is too small (12.0pt): 
(fancyhdr)                Make it at least 13.59999pt, for example:
(fancyhdr)                \setlength{\headheight}{13.59999pt}.
(fancyhdr)                You might also make \topmargin smaller:
(fancyhdr)                \addtolength{\topmargin}{-1.59999pt}.

[92]
Overfull \hbox (4.92744pt too wide) detected at line 3687
 []\OT1/lmss/m/n/10.95 1007
 []


Overfull \hbox (4.92744pt too wide) detected at line 3688
 []\OT1/lmss/m/n/10.95 1007
 []


Overfull \hbox (4.92744pt too wide) detected at line 3689
 []\OT1/lmss/m/n/10.95 1007
 []


Overfull \hbox (4.92744pt too wide) detected at line 3690
 []\OT1/lmss/m/n/10.95 1007
 []


Overfull \hbox (4.92744pt too wide) detected at line 3691
 []\OT1/lmss/m/n/10.95 1008
 []


Overfull \hbox (4.92744pt too wide) detected at line 3692
 []\OT1/lmss/m/n/10.95 1008
 []


Overfull \hbox (4.92744pt too wide) detected at line 3693
 []\OT1/lmss/m/n/10.95 1008
 []


Overfull \hbox (4.92744pt too wide) detected at line 3694
 []\OT1/lmss/m/n/10.95 1008
 []


Overfull \hbox (4.92744pt too wide) detected at line 3695
 []\OT1/lmss/m/n/10.95 1009
 []


Overfull \hbox (4.92744pt too wide) detected at line 3696
 []\OT1/lmss/m/n/10.95 1009
 []


Overfull \hbox (4.92744pt too wide) detected at line 3697
 []\OT1/lmss/m/n/10.95 1009
 []


Overfull \hbox (4.92744pt too wide) detected at line 3698
 []\OT1/lmss/m/n/10.95 1010
 []


Overfull \hbox (4.92744pt too wide) detected at line 3699
 []\OT1/lmss/m/n/10.95 1010
 []


Overfull \hbox (4.92744pt too wide) detected at line 3700
 []\OT1/lmss/m/n/10.95 1010
 []


Overfull \hbox (4.92744pt too wide) detected at line 3701
 []\OT1/lmss/m/n/10.95 1010
 []


Overfull \hbox (4.92744pt too wide) detected at line 3702
 []\OT1/lmss/m/n/10.95 1011
 []


Overfull \hbox (4.92744pt too wide) detected at line 3703
 []\OT1/lmss/m/n/10.95 1011
 []


Overfull \hbox (4.92744pt too wide) detected at line 3704
 []\OT1/lmss/m/n/10.95 1011
 []


Overfull \hbox (4.92744pt too wide) detected at line 3705
 []\OT1/lmss/m/n/10.95 1011
 []


Overfull \hbox (4.92744pt too wide) detected at line 3706
 []\OT1/lmss/m/n/10.95 1012
 []


Overfull \hbox (4.92744pt too wide) detected at line 3707
 []\OT1/lmss/m/n/10.95 1012
 []


Overfull \hbox (4.92744pt too wide) detected at line 3708
 []\OT1/lmss/m/n/10.95 1012
 []


Overfull \hbox (4.92744pt too wide) detected at line 3709
 []\OT1/lmss/m/n/10.95 1012
 []


Overfull \hbox (4.92744pt too wide) detected at line 3710
 []\OT1/lmss/m/n/10.95 1012
 []


Overfull \hbox (4.92744pt too wide) detected at line 3711
 []\OT1/lmss/m/n/10.95 1013
 []


Overfull \hbox (4.92744pt too wide) detected at line 3712
 []\OT1/lmss/m/n/10.95 1013
 []


Overfull \hbox (4.92744pt too wide) detected at line 3714
 []\OT1/lmss/m/n/10.95 1015
 []


Overfull \hbox (4.92744pt too wide) detected at line 3715
 []\OT1/lmss/m/n/10.95 1015
 []


Overfull \hbox (4.92744pt too wide) detected at line 3717
 []\OT1/lmss/m/n/10.95 1017
 []


Overfull \hbox (4.92744pt too wide) detected at line 3718
 []\OT1/lmss/m/n/10.95 1018
 []


Overfull \hbox (4.92744pt too wide) detected at line 3719
 []\OT1/lmss/m/n/10.95 1018
 []


Overfull \hbox (4.92744pt too wide) detected at line 3720
 []\OT1/lmss/m/n/10.95 1018
 []


Overfull \hbox (4.92744pt too wide) detected at line 3721
 []\OT1/lmss/m/n/10.95 1018
 []


Overfull \hbox (4.92744pt too wide) detected at line 3722
 []\OT1/lmss/m/n/10.95 1018
 []




Package fancyhdr Warning: \headheight is too small (12.0pt): 
(fancyhdr)                Make it at least 13.59999pt, for example:
(fancyhdr)                \setlength{\headheight}{13.59999pt}.
(fancyhdr)                You might also make \topmargin smaller:
(fancyhdr)                \addtolength{\topmargin}{-1.59999pt}.

[93]
Overfull \hbox (4.92744pt too wide) detected at line 3723
 []\OT1/lmss/m/n/10.95 1018
 []


Overfull \hbox (4.92744pt too wide) detected at line 3724
 []\OT1/lmss/m/n/10.95 1018
 []


Overfull \hbox (4.92744pt too wide) detected at line 3725
 []\OT1/lmss/m/n/10.95 1019
 []


Overfull \hbox (4.92744pt too wide) detected at line 3726
 []\OT1/lmss/m/n/10.95 1019
 []


Overfull \hbox (19.31749pt too wide) in paragraph at lines 3727--3727
 [] [][][]\OT1/lmss/bx/n/10.95 Basics: Why Wafer In-te-gra-tion in Com-mu-ni-ca
-tion En-gi-neer-ing? (T0 photonenchip-umsetzung)[][] []  
 []


Overfull \hbox (4.92744pt too wide) detected at line 3728
 []\OT1/lmss/m/n/10.95 1021
 []


Overfull \hbox (4.92744pt too wide) detected at line 3729
 []\OT1/lmss/m/n/10.95 1022
 []


Overfull \hbox (4.92744pt too wide) detected at line 3730
 []\OT1/lmss/m/n/10.95 1022
 []


Overfull \hbox (4.92744pt too wide) detected at line 3731
 []\OT1/lmss/m/n/10.95 1023
 []


Overfull \hbox (4.92744pt too wide) detected at line 3733
 []\OT1/lmss/m/n/10.95 1025
 []


Overfull \hbox (4.92744pt too wide) detected at line 3734
 []\OT1/lmss/m/n/10.95 1026
 []


Overfull \hbox (4.92744pt too wide) detected at line 3735
 []\OT1/lmss/m/n/10.95 1026
 []


Overfull \hbox (4.92744pt too wide) detected at line 3736
 []\OT1/lmss/m/n/10.95 1027
 []


Overfull \hbox (4.92744pt too wide) detected at line 3738
 []\OT1/lmss/m/n/10.95 1028
 []


Overfull \hbox (4.92744pt too wide) detected at line 3739
 []\OT1/lmss/m/n/10.95 1028
 []


Overfull \hbox (4.92744pt too wide) detected at line 3740
 []\OT1/lmss/m/n/10.95 1028
 []


Overfull \hbox (4.92744pt too wide) detected at line 3741
 []\OT1/lmss/m/n/10.95 1029
 []


Overfull \hbox (4.92744pt too wide) detected at line 3742
 []\OT1/lmss/m/n/10.95 1029
 []


Overfull \hbox (4.92744pt too wide) detected at line 3743
 []\OT1/lmss/m/n/10.95 1029
 []


Overfull \hbox (4.92744pt too wide) detected at line 3744
 []\OT1/lmss/m/n/10.95 1029
 []


Overfull \hbox (4.92744pt too wide) detected at line 3745
 []\OT1/lmss/m/n/10.95 1029
 []


Overfull \hbox (4.92744pt too wide) detected at line 3746
 []\OT1/lmss/m/n/10.95 1030
 []


Overfull \hbox (4.92744pt too wide) detected at line 3747
 []\OT1/lmss/m/n/10.95 1030
 []


Overfull \hbox (4.92744pt too wide) detected at line 3748
 []\OT1/lmss/m/n/10.95 1030
 []


Overfull \hbox (4.92744pt too wide) detected at line 3749
 []\OT1/lmss/m/n/10.95 1030
 []


Overfull \hbox (4.92744pt too wide) detected at line 3750
 []\OT1/lmss/m/n/10.95 1030
 []


Overfull \hbox (4.92744pt too wide) detected at line 3751
 []\OT1/lmss/m/n/10.95 1031
 []


Overfull \hbox (4.92744pt too wide) detected at line 3752
 []\OT1/lmss/m/n/10.95 1031
 []


Overfull \hbox (4.92744pt too wide) detected at line 3753
 []\OT1/lmss/m/n/10.95 1031
 []


Overfull \hbox (4.92744pt too wide) detected at line 3754
 []\OT1/lmss/m/n/10.95 1032
 []


Overfull \hbox (4.92744pt too wide) detected at line 3755
 []\OT1/lmss/m/n/10.95 1032
 []


Overfull \hbox (4.92744pt too wide) detected at line 3756
 []\OT1/lmss/m/n/10.95 1032
 []


Overfull \hbox (4.92744pt too wide) detected at line 3757
 []\OT1/lmss/m/n/10.95 1032
 []


Overfull \hbox (4.92744pt too wide) detected at line 3758
 []\OT1/lmss/m/n/10.95 1033
 []




Package fancyhdr Warning: \headheight is too small (12.0pt): 
(fancyhdr)                Make it at least 13.59999pt, for example:
(fancyhdr)                \setlength{\headheight}{13.59999pt}.
(fancyhdr)                You might also make \topmargin smaller:
(fancyhdr)                \addtolength{\topmargin}{-1.59999pt}.

[94]
Overfull \hbox (4.92744pt too wide) detected at line 3759
 []\OT1/lmss/m/n/10.95 1033
 []


Overfull \hbox (4.92744pt too wide) detected at line 3760
 []\OT1/lmss/m/n/10.95 1033
 []


Overfull \hbox (4.92744pt too wide) detected at line 3761
 []\OT1/lmss/m/n/10.95 1033
 []


Overfull \hbox (4.92744pt too wide) detected at line 3762
 []\OT1/lmss/m/n/10.95 1034
 []


Overfull \hbox (4.92744pt too wide) detected at line 3763
 []\OT1/lmss/m/n/10.95 1034
 []


Overfull \hbox (4.92744pt too wide) detected at line 3764
 []\OT1/lmss/m/n/10.95 1034
 []


Overfull \hbox (4.92744pt too wide) detected at line 3765
 []\OT1/lmss/m/n/10.95 1034
 []


Overfull \hbox (4.92744pt too wide) detected at line 3766
 []\OT1/lmss/m/n/10.95 1035
 []


Overfull \hbox (4.92744pt too wide) detected at line 3767
 []\OT1/lmss/m/n/10.95 1035
 []


Overfull \hbox (4.92744pt too wide) detected at line 3768
 []\OT1/lmss/m/n/10.95 1035
 []


Overfull \hbox (4.92744pt too wide) detected at line 3769
 []\OT1/lmss/m/n/10.95 1035
 []


Overfull \hbox (4.92744pt too wide) detected at line 3770
 []\OT1/lmss/m/n/10.95 1035
 []


Overfull \hbox (4.92744pt too wide) detected at line 3771
 []\OT1/lmss/m/n/10.95 1035
 []


Overfull \hbox (4.92744pt too wide) detected at line 3772
 []\OT1/lmss/m/n/10.95 1037
 []


Overfull \hbox (4.92744pt too wide) detected at line 3774
 []\OT1/lmss/m/n/10.95 1039
 []


Overfull \hbox (4.92744pt too wide) detected at line 3775
 []\OT1/lmss/m/n/10.95 1040
 []


Overfull \hbox (4.92744pt too wide) detected at line 3776
 []\OT1/lmss/m/n/10.95 1040
 []


Overfull \hbox (4.92744pt too wide) detected at line 3777
 []\OT1/lmss/m/n/10.95 1040
 []


Overfull \hbox (4.92744pt too wide) detected at line 3778
 []\OT1/lmss/m/n/10.95 1040
 []


Overfull \hbox (4.92744pt too wide) detected at line 3779
 []\OT1/lmss/m/n/10.95 1041
 []


Overfull \hbox (4.92744pt too wide) detected at line 3780
 []\OT1/lmss/m/n/10.95 1041
 []


Overfull \hbox (4.92744pt too wide) detected at line 3781
 []\OT1/lmss/m/n/10.95 1041
 []


Overfull \hbox (4.92744pt too wide) detected at line 3782
 []\OT1/lmss/m/n/10.95 1041
 []


Overfull \hbox (4.92744pt too wide) detected at line 3783
 []\OT1/lmss/m/n/10.95 1041
 []


Overfull \hbox (4.92744pt too wide) detected at line 3784
 []\OT1/lmss/m/n/10.95 1042
 []


Overfull \hbox (4.92744pt too wide) detected at line 3785
 []\OT1/lmss/m/n/10.95 1042
 []


Overfull \hbox (4.92744pt too wide) detected at line 3786
 []\OT1/lmss/m/n/10.95 1042
 []


Overfull \hbox (4.92744pt too wide) detected at line 3787
 []\OT1/lmss/m/n/10.95 1043
 []


Overfull \hbox (4.92744pt too wide) detected at line 3788
 []\OT1/lmss/m/n/10.95 1043
 []


Overfull \hbox (4.92744pt too wide) detected at line 3789
 []\OT1/lmss/m/n/10.95 1044
 []


Overfull \hbox (4.92744pt too wide) detected at line 3790
 []\OT1/lmss/m/n/10.95 1044
 []


Overfull \hbox (4.92744pt too wide) detected at line 3791
 []\OT1/lmss/m/n/10.95 1044
 []


Overfull \hbox (4.92744pt too wide) detected at line 3792
 []\OT1/lmss/m/n/10.95 1044
 []


Overfull \hbox (4.92744pt too wide) detected at line 3793
 []\OT1/lmss/m/n/10.95 1044
 []


Overfull \hbox (4.92744pt too wide) detected at line 3794
 []\OT1/lmss/m/n/10.95 1045
 []




Package fancyhdr Warning: \headheight is too small (12.0pt): 
(fancyhdr)                Make it at least 13.59999pt, for example:
(fancyhdr)                \setlength{\headheight}{13.59999pt}.
(fancyhdr)                You might also make \topmargin smaller:
(fancyhdr)                \addtolength{\topmargin}{-1.59999pt}.

[95]
Overfull \hbox (4.92744pt too wide) detected at line 3795
 []\OT1/lmss/m/n/10.95 1045
 []


Overfull \hbox (4.92744pt too wide) detected at line 3796
 []\OT1/lmss/m/n/10.95 1045
 []


Overfull \hbox (4.92744pt too wide) detected at line 3798
 []\OT1/lmss/m/n/10.95 1047
 []


Overfull \hbox (4.92744pt too wide) detected at line 3799
 []\OT1/lmss/m/n/10.95 1047
 []


Overfull \hbox (4.92744pt too wide) detected at line 3800
 []\OT1/lmss/m/n/10.95 1047
 []


Overfull \hbox (4.92744pt too wide) detected at line 3801
 []\OT1/lmss/m/n/10.95 1048
 []


Overfull \hbox (4.92744pt too wide) detected at line 3802
 []\OT1/lmss/m/n/10.95 1048
 []


Overfull \hbox (4.92744pt too wide) detected at line 3803
 []\OT1/lmss/m/n/10.95 1048
 []


Overfull \hbox (4.92744pt too wide) detected at line 3804
 []\OT1/lmss/m/n/10.95 1048
 []


Overfull \hbox (4.92744pt too wide) detected at line 3805
 []\OT1/lmss/m/n/10.95 1048
 []


Overfull \hbox (4.92744pt too wide) detected at line 3806
 []\OT1/lmss/m/n/10.95 1049
 []


Overfull \hbox (4.92744pt too wide) detected at line 3807
 []\OT1/lmss/m/n/10.95 1049
 []


Overfull \hbox (4.92744pt too wide) detected at line 3808
 []\OT1/lmss/m/n/10.95 1049
 []


Overfull \hbox (4.92744pt too wide) detected at line 3809
 []\OT1/lmss/m/n/10.95 1049
 []


Overfull \hbox (4.92744pt too wide) detected at line 3810
 []\OT1/lmss/m/n/10.95 1050
 []


Overfull \hbox (4.92744pt too wide) detected at line 3811
 []\OT1/lmss/m/n/10.95 1050
 []


Overfull \hbox (4.92744pt too wide) detected at line 3812
 []\OT1/lmss/m/n/10.95 1050
 []


Overfull \hbox (4.92744pt too wide) detected at line 3813
 []\OT1/lmss/m/n/10.95 1050
 []


Overfull \hbox (4.92744pt too wide) detected at line 3814
 []\OT1/lmss/m/n/10.95 1050
 []


Overfull \hbox (4.92744pt too wide) detected at line 3815
 []\OT1/lmss/m/n/10.95 1051
 []


Overfull \hbox (4.92744pt too wide) detected at line 3816
 []\OT1/lmss/m/n/10.95 1051
 []


Overfull \hbox (4.92744pt too wide) detected at line 3817
 []\OT1/lmss/m/n/10.95 1051
 []


Overfull \hbox (4.92744pt too wide) detected at line 3818
 []\OT1/lmss/m/n/10.95 1051
 []


Overfull \hbox (4.92744pt too wide) detected at line 3819
 []\OT1/lmss/m/n/10.95 1051
 []


Overfull \hbox (4.92744pt too wide) detected at line 3820
 []\OT1/lmss/m/n/10.95 1051
 []


Overfull \hbox (4.92744pt too wide) detected at line 3821
 []\OT1/lmss/m/n/10.95 1051
 []


Overfull \hbox (4.92744pt too wide) detected at line 3822
 []\OT1/lmss/m/n/10.95 1051
 []


Overfull \hbox (4.92744pt too wide) detected at line 3823
 []\OT1/lmss/m/n/10.95 1052
 []


Overfull \hbox (4.92744pt too wide) detected at line 3824
 []\OT1/lmss/m/n/10.95 1052
 []


Overfull \hbox (4.92744pt too wide) detected at line 3825
 []\OT1/lmss/m/n/10.95 1052
 []


Overfull \hbox (4.92744pt too wide) detected at line 3826
 []\OT1/lmss/m/n/10.95 1052
 []


Overfull \hbox (4.92744pt too wide) detected at line 3827
 []\OT1/lmss/m/n/10.95 1053
 []


Overfull \hbox (4.92744pt too wide) detected at line 3828
 []\OT1/lmss/m/n/10.95 1053
 []


Overfull \hbox (4.92744pt too wide) detected at line 3829
 []\OT1/lmss/m/n/10.95 1053
 []


Overfull \hbox (4.92744pt too wide) detected at line 3830
 []\OT1/lmss/m/n/10.95 1053
 []


Overfull \hbox (4.92744pt too wide) detected at line 3831
 []\OT1/lmss/m/n/10.95 1054
 []




Package fancyhdr Warning: \headheight is too small (12.0pt): 
(fancyhdr)                Make it at least 13.59999pt, for example:
(fancyhdr)                \setlength{\headheight}{13.59999pt}.
(fancyhdr)                You might also make \topmargin smaller:
(fancyhdr)                \addtolength{\topmargin}{-1.59999pt}.

[96]
Overfull \hbox (4.92744pt too wide) detected at line 3832
 []\OT1/lmss/m/n/10.95 1054
 []


Overfull \hbox (4.92744pt too wide) detected at line 3833
 []\OT1/lmss/m/n/10.95 1054
 []


Overfull \hbox (4.92744pt too wide) detected at line 3834
 []\OT1/lmss/m/n/10.95 1054
 []


Overfull \hbox (4.92744pt too wide) detected at line 3835
 []\OT1/lmss/m/n/10.95 1054
 []


Overfull \hbox (4.92744pt too wide) detected at line 3836
 []\OT1/lmss/m/n/10.95 1054
 []


Overfull \hbox (4.92744pt too wide) detected at line 3837
 []\OT1/lmss/m/n/10.95 1054
 []


Overfull \hbox (4.92744pt too wide) detected at line 3838
 []\OT1/lmss/m/n/10.95 1054
 []


Overfull \hbox (4.92744pt too wide) detected at line 3839
 []\OT1/lmss/m/n/10.95 1054
 []


Overfull \hbox (4.92744pt too wide) detected at line 3840
 []\OT1/lmss/m/n/10.95 1055
 []


Overfull \hbox (4.92744pt too wide) detected at line 3841
 []\OT1/lmss/m/n/10.95 1055
 []


Overfull \hbox (4.92744pt too wide) detected at line 3842
 []\OT1/lmss/m/n/10.95 1055
 []


Overfull \hbox (4.92744pt too wide) detected at line 3843
 []\OT1/lmss/m/n/10.95 1055
 []


Overfull \hbox (4.92744pt too wide) detected at line 3844
 []\OT1/lmss/m/n/10.95 1055
 []


Overfull \hbox (4.92744pt too wide) detected at line 3845
 []\OT1/lmss/m/n/10.95 1056
 []


Overfull \hbox (4.92744pt too wide) detected at line 3846
 []\OT1/lmss/m/n/10.95 1056
 []


Overfull \hbox (4.92744pt too wide) detected at line 3847
 []\OT1/lmss/m/n/10.95 1056
 []


Overfull \hbox (4.92744pt too wide) detected at line 3848
 []\OT1/lmss/m/n/10.95 1056
 []


Overfull \hbox (4.92744pt too wide) detected at line 3849
 []\OT1/lmss/m/n/10.95 1056
 []


Overfull \hbox (4.92744pt too wide) detected at line 3850
 []\OT1/lmss/m/n/10.95 1056
 []


Overfull \hbox (4.92744pt too wide) detected at line 3851
 []\OT1/lmss/m/n/10.95 1057
 []


Overfull \hbox (4.92744pt too wide) detected at line 3852
 []\OT1/lmss/m/n/10.95 1057
 []


Overfull \hbox (4.92744pt too wide) detected at line 3853
 []\OT1/lmss/m/n/10.95 1057
 []


Overfull \hbox (4.92744pt too wide) detected at line 3854
 []\OT1/lmss/m/n/10.95 1057
 []


Overfull \hbox (4.92744pt too wide) detected at line 3855
 []\OT1/lmss/m/n/10.95 1057
 []


Overfull \hbox (4.92744pt too wide) detected at line 3856
 []\OT1/lmss/m/n/10.95 1057
 []


Overfull \hbox (4.92744pt too wide) detected at line 3857
 []\OT1/lmss/m/n/10.95 1058
 []


Overfull \hbox (4.92744pt too wide) detected at line 3858
 []\OT1/lmss/m/n/10.95 1058
 []


Overfull \hbox (4.92744pt too wide) detected at line 3859
 []\OT1/lmss/m/n/10.95 1058
 []


Overfull \hbox (4.92744pt too wide) detected at line 3860
 []\OT1/lmss/m/n/10.95 1059
 []


Overfull \hbox (4.92744pt too wide) detected at line 3861
 []\OT1/lmss/m/n/10.95 1059
 []


Overfull \hbox (4.92744pt too wide) detected at line 3862
 []\OT1/lmss/m/n/10.95 1059
 []


Overfull \hbox (4.92744pt too wide) detected at line 3863
 []\OT1/lmss/m/n/10.95 1059
 []


Overfull \hbox (4.92744pt too wide) detected at line 3864
 []\OT1/lmss/m/n/10.95 1059
 []


Overfull \hbox (4.92744pt too wide) detected at line 3865
 []\OT1/lmss/m/n/10.95 1059
 []


Overfull \hbox (4.92744pt too wide) detected at line 3866
 []\OT1/lmss/m/n/10.95 1059
 []


Overfull \hbox (4.92744pt too wide) detected at line 3867
 []\OT1/lmss/m/n/10.95 1059
 []


Overfull \hbox (4.92744pt too wide) detected at line 3868
 []\OT1/lmss/m/n/10.95 1059
 []




Package fancyhdr Warning: \headheight is too small (12.0pt): 
(fancyhdr)                Make it at least 13.59999pt, for example:
(fancyhdr)                \setlength{\headheight}{13.59999pt}.
(fancyhdr)                You might also make \topmargin smaller:
(fancyhdr)                \addtolength{\topmargin}{-1.59999pt}.

[97]
Overfull \hbox (4.92744pt too wide) detected at line 3869
 []\OT1/lmss/m/n/10.95 1060
 []


Overfull \hbox (4.92744pt too wide) detected at line 3870
 []\OT1/lmss/m/n/10.95 1060
 []


Overfull \hbox (4.92744pt too wide) detected at line 3871
 []\OT1/lmss/m/n/10.95 1060
 []


Overfull \hbox (4.92744pt too wide) detected at line 3872
 []\OT1/lmss/m/n/10.95 1060
 []


Overfull \hbox (4.92744pt too wide) detected at line 3873
 []\OT1/lmss/m/n/10.95 1061
 []


Overfull \hbox (4.92744pt too wide) detected at line 3874
 []\OT1/lmss/m/n/10.95 1061
 []


Overfull \hbox (4.92744pt too wide) detected at line 3876
 []\OT1/lmss/m/n/10.95 1063
 []


Overfull \hbox (4.92744pt too wide) detected at line 3877
 []\OT1/lmss/m/n/10.95 1063
 []


Overfull \hbox (4.92744pt too wide) detected at line 3878
 []\OT1/lmss/m/n/10.95 1063
 []


Overfull \hbox (4.92744pt too wide) detected at line 3879
 []\OT1/lmss/m/n/10.95 1063
 []


Overfull \hbox (4.92744pt too wide) detected at line 3880
 []\OT1/lmss/m/n/10.95 1064
 []


Overfull \hbox (4.92744pt too wide) detected at line 3881
 []\OT1/lmss/m/n/10.95 1064
 []


Overfull \hbox (4.92744pt too wide) detected at line 3882
 []\OT1/lmss/m/n/10.95 1064
 []


Overfull \hbox (4.92744pt too wide) detected at line 3883
 []\OT1/lmss/m/n/10.95 1064
 []


Overfull \hbox (4.92744pt too wide) detected at line 3884
 []\OT1/lmss/m/n/10.95 1064
 []


Overfull \hbox (4.92744pt too wide) detected at line 3885
 []\OT1/lmss/m/n/10.95 1064
 []


Overfull \hbox (4.92744pt too wide) detected at line 3886
 []\OT1/lmss/m/n/10.95 1064
 []


Overfull \hbox (4.92744pt too wide) detected at line 3887
 []\OT1/lmss/m/n/10.95 1064
 []


Overfull \hbox (4.92744pt too wide) detected at line 3888
 []\OT1/lmss/m/n/10.95 1064
 []


Overfull \hbox (4.92744pt too wide) detected at line 3889
 []\OT1/lmss/m/n/10.95 1065
 []


Overfull \hbox (4.92744pt too wide) detected at line 3890
 []\OT1/lmss/m/n/10.95 1065
 []


Overfull \hbox (4.92744pt too wide) detected at line 3891
 []\OT1/lmss/m/n/10.95 1065
 []


Overfull \hbox (4.92744pt too wide) detected at line 3892
 []\OT1/lmss/m/n/10.95 1065
 []


Overfull \hbox (4.92744pt too wide) detected at line 3893
 []\OT1/lmss/m/n/10.95 1065
 []


Overfull \hbox (4.92744pt too wide) detected at line 3894
 []\OT1/lmss/m/n/10.95 1065
 []


Overfull \hbox (4.92744pt too wide) detected at line 3895
 []\OT1/lmss/m/n/10.95 1065
 []


Overfull \hbox (4.92744pt too wide) detected at line 3896
 []\OT1/lmss/m/n/10.95 1066
 []


Overfull \hbox (4.92744pt too wide) detected at line 3897
 []\OT1/lmss/m/n/10.95 1066
 []


Overfull \hbox (4.92744pt too wide) detected at line 3898
 []\OT1/lmss/m/n/10.95 1066
 []


Overfull \hbox (4.92744pt too wide) detected at line 3899
 []\OT1/lmss/m/n/10.95 1066
 []


Overfull \hbox (4.92744pt too wide) detected at line 3900
 []\OT1/lmss/m/n/10.95 1066
 []


Overfull \hbox (4.92744pt too wide) detected at line 3901
 []\OT1/lmss/m/n/10.95 1066
 []


Overfull \hbox (4.92744pt too wide) detected at line 3902
 []\OT1/lmss/m/n/10.95 1066
 []


Overfull \hbox (4.92744pt too wide) detected at line 3903
 []\OT1/lmss/m/n/10.95 1066
 []


Overfull \hbox (4.92744pt too wide) detected at line 3904
 []\OT1/lmss/m/n/10.95 1066
 []


Overfull \hbox (4.92744pt too wide) detected at line 3905
 []\OT1/lmss/m/n/10.95 1067
 []




Package fancyhdr Warning: \headheight is too small (12.0pt): 
(fancyhdr)                Make it at least 13.59999pt, for example:
(fancyhdr)                \setlength{\headheight}{13.59999pt}.
(fancyhdr)                You might also make \topmargin smaller:
(fancyhdr)                \addtolength{\topmargin}{-1.59999pt}.

[98]
Overfull \hbox (4.92744pt too wide) detected at line 3906
 []\OT1/lmss/m/n/10.95 1067
 []


Overfull \hbox (4.92744pt too wide) detected at line 3908
 []\OT1/lmss/m/n/10.95 1068
 []


Overfull \hbox (4.92744pt too wide) detected at line 3909
 []\OT1/lmss/m/n/10.95 1068
 []


Overfull \hbox (4.92744pt too wide) detected at line 3910
 []\OT1/lmss/m/n/10.95 1068
 []


Overfull \hbox (4.92744pt too wide) detected at line 3911
 []\OT1/lmss/m/n/10.95 1069
 []


Overfull \hbox (4.92744pt too wide) detected at line 3912
 []\OT1/lmss/m/n/10.95 1069
 []


Overfull \hbox (4.92744pt too wide) detected at line 3913
 []\OT1/lmss/m/n/10.95 1069
 []


Overfull \hbox (4.92744pt too wide) detected at line 3914
 []\OT1/lmss/m/n/10.95 1069
 []


Overfull \hbox (4.92744pt too wide) detected at line 3915
 []\OT1/lmss/m/n/10.95 1069
 []


Overfull \hbox (4.92744pt too wide) detected at line 3916
 []\OT1/lmss/m/n/10.95 1070
 []


Overfull \hbox (4.92744pt too wide) detected at line 3917
 []\OT1/lmss/m/n/10.95 1070
 []


Overfull \hbox (4.92744pt too wide) detected at line 3918
 []\OT1/lmss/m/n/10.95 1071
 []


Overfull \hbox (4.92744pt too wide) detected at line 3919
 []\OT1/lmss/m/n/10.95 1071
 []


Overfull \hbox (4.92744pt too wide) detected at line 3920
 []\OT1/lmss/m/n/10.95 1071
 []


Overfull \hbox (4.92744pt too wide) detected at line 3921
 []\OT1/lmss/m/n/10.95 1071
 []


Overfull \hbox (4.92744pt too wide) detected at line 3922
 []\OT1/lmss/m/n/10.95 1072
 []


Overfull \hbox (4.92744pt too wide) detected at line 3923
 []\OT1/lmss/m/n/10.95 1072
 []


Overfull \hbox (4.92744pt too wide) detected at line 3924
 []\OT1/lmss/m/n/10.95 1072
 []


Overfull \hbox (4.92744pt too wide) detected at line 3925
 []\OT1/lmss/m/n/10.95 1072
 []


Overfull \hbox (4.92744pt too wide) detected at line 3926
 []\OT1/lmss/m/n/10.95 1072
 []


Overfull \hbox (4.92744pt too wide) detected at line 3927
 []\OT1/lmss/m/n/10.95 1072
 []


Overfull \hbox (4.92744pt too wide) detected at line 3928
 []\OT1/lmss/m/n/10.95 1073
 []


Overfull \hbox (4.92744pt too wide) detected at line 3929
 []\OT1/lmss/m/n/10.95 1073
 []


Overfull \hbox (4.92744pt too wide) detected at line 3930
 []\OT1/lmss/m/n/10.95 1073
 []


Overfull \hbox (4.92744pt too wide) detected at line 3931
 []\OT1/lmss/m/n/10.95 1073
 []


Overfull \hbox (4.92744pt too wide) detected at line 3932
 []\OT1/lmss/m/n/10.95 1074
 []


Overfull \hbox (4.92744pt too wide) detected at line 3933
 []\OT1/lmss/m/n/10.95 1074
 []


Overfull \hbox (4.92744pt too wide) detected at line 3934
 []\OT1/lmss/m/n/10.95 1074
 []


Overfull \hbox (4.92744pt too wide) detected at line 3935
 []\OT1/lmss/m/n/10.95 1074
 []


Overfull \hbox (4.92744pt too wide) detected at line 3936
 []\OT1/lmss/m/n/10.95 1074
 []


Overfull \hbox (4.92744pt too wide) detected at line 3937
 []\OT1/lmss/m/n/10.95 1075
 []


Overfull \hbox (4.92744pt too wide) detected at line 3938
 []\OT1/lmss/m/n/10.95 1075
 []


Overfull \hbox (4.92744pt too wide) detected at line 3939
 []\OT1/lmss/m/n/10.95 1075
 []


Overfull \hbox (4.92744pt too wide) detected at line 3940
 []\OT1/lmss/m/n/10.95 1075
 []


Overfull \hbox (4.92744pt too wide) detected at line 3941
 []\OT1/lmss/m/n/10.95 1076
 []


Overfull \hbox (4.92744pt too wide) detected at line 3942
 []\OT1/lmss/m/n/10.95 1076
 []




Package fancyhdr Warning: \headheight is too small (12.0pt): 
(fancyhdr)                Make it at least 13.59999pt, for example:
(fancyhdr)                \setlength{\headheight}{13.59999pt}.
(fancyhdr)                You might also make \topmargin smaller:
(fancyhdr)                \addtolength{\topmargin}{-1.59999pt}.

[99]
Overfull \hbox (4.92744pt too wide) detected at line 3943
 []\OT1/lmss/m/n/10.95 1076
 []


Overfull \hbox (4.92744pt too wide) detected at line 3945
 []\OT1/lmss/m/n/10.95 1078
 []


Overfull \hbox (4.92744pt too wide) detected at line 3946
 []\OT1/lmss/m/n/10.95 1078
 []


Overfull \hbox (4.92744pt too wide) detected at line 3947
 []\OT1/lmss/m/n/10.95 1079
 []


Overfull \hbox (4.92744pt too wide) detected at line 3948
 []\OT1/lmss/m/n/10.95 1079
 []


Overfull \hbox (4.92744pt too wide) detected at line 3949
 []\OT1/lmss/m/n/10.95 1079
 []


Overfull \hbox (4.92744pt too wide) detected at line 3950
 []\OT1/lmss/m/n/10.95 1079
 []


Overfull \hbox (4.92744pt too wide) detected at line 3951
 []\OT1/lmss/m/n/10.95 1080
 []


Overfull \hbox (4.92744pt too wide) detected at line 3952
 []\OT1/lmss/m/n/10.95 1080
 []


Overfull \hbox (4.92744pt too wide) detected at line 3953
 []\OT1/lmss/m/n/10.95 1080
 []


Overfull \hbox (4.92744pt too wide) detected at line 3954
 []\OT1/lmss/m/n/10.95 1080
 []


Overfull \hbox (4.92744pt too wide) detected at line 3955
 []\OT1/lmss/m/n/10.95 1081
 []


Overfull \hbox (4.92744pt too wide) detected at line 3956
 []\OT1/lmss/m/n/10.95 1081
 []


Overfull \hbox (4.92744pt too wide) detected at line 3957
 []\OT1/lmss/m/n/10.95 1081
 []


Overfull \hbox (4.92744pt too wide) detected at line 3958
 []\OT1/lmss/m/n/10.95 1082
 []


Overfull \hbox (4.92744pt too wide) detected at line 3959
 []\OT1/lmss/m/n/10.95 1082
 []


Overfull \hbox (4.92744pt too wide) detected at line 3960
 []\OT1/lmss/m/n/10.95 1082
 []


Overfull \hbox (4.92744pt too wide) detected at line 3961
 []\OT1/lmss/m/n/10.95 1082
 []


Overfull \hbox (4.92744pt too wide) detected at line 3962
 []\OT1/lmss/m/n/10.95 1083
 []


Overfull \hbox (4.92744pt too wide) detected at line 3963
 []\OT1/lmss/m/n/10.95 1083
 []


Overfull \hbox (4.92744pt too wide) detected at line 3964
 []\OT1/lmss/m/n/10.95 1083
 []


Overfull \hbox (4.92744pt too wide) detected at line 3965
 []\OT1/lmss/m/n/10.95 1083
 []


Overfull \hbox (4.92744pt too wide) detected at line 3966
 []\OT1/lmss/m/n/10.95 1084
 []


Overfull \hbox (4.92744pt too wide) detected at line 3967
 []\OT1/lmss/m/n/10.95 1084
 []


Overfull \hbox (4.92744pt too wide) detected at line 3968
 []\OT1/lmss/m/n/10.95 1084
 []


Overfull \hbox (4.92744pt too wide) detected at line 3969
 []\OT1/lmss/m/n/10.95 1085
 []


Overfull \hbox (4.92744pt too wide) detected at line 3971
 []\OT1/lmss/m/n/10.95 1087
 []


Overfull \hbox (4.92744pt too wide) detected at line 3972
 []\OT1/lmss/m/n/10.95 1087
 []


Overfull \hbox (4.92744pt too wide) detected at line 3973
 []\OT1/lmss/m/n/10.95 1088
 []


Overfull \hbox (4.92744pt too wide) detected at line 3974
 []\OT1/lmss/m/n/10.95 1088
 []


Overfull \hbox (4.92744pt too wide) detected at line 3975
 []\OT1/lmss/m/n/10.95 1088
 []


Overfull \hbox (4.92744pt too wide) detected at line 3976
 []\OT1/lmss/m/n/10.95 1088
 []


Overfull \hbox (4.92744pt too wide) detected at line 3977
 []\OT1/lmss/m/n/10.95 1089
 []


Overfull \hbox (4.92744pt too wide) detected at line 3978
 []\OT1/lmss/m/n/10.95 1089
 []




Package fancyhdr Warning: \headheight is too small (12.0pt): 
(fancyhdr)                Make it at least 13.59999pt, for example:
(fancyhdr)                \setlength{\headheight}{13.59999pt}.
(fancyhdr)                You might also make \topmargin smaller:
(fancyhdr)                \addtolength{\topmargin}{-1.59999pt}.

[100]
Overfull \hbox (4.92744pt too wide) detected at line 3979
 []\OT1/lmss/m/n/10.95 1090
 []


Overfull \hbox (4.92744pt too wide) detected at line 3980
 []\OT1/lmss/m/n/10.95 1090
 []


Overfull \hbox (4.92744pt too wide) detected at line 3981
 []\OT1/lmss/m/n/10.95 1090
 []


Overfull \hbox (4.92744pt too wide) detected at line 3982
 []\OT1/lmss/m/n/10.95 1091
 []


Overfull \hbox (4.92744pt too wide) detected at line 3983
 []\OT1/lmss/m/n/10.95 1091
 []


Overfull \hbox (4.92744pt too wide) detected at line 3984
 []\OT1/lmss/m/n/10.95 1091
 []


Overfull \hbox (4.92744pt too wide) detected at line 3985
 []\OT1/lmss/m/n/10.95 1091
 []


Overfull \hbox (4.92744pt too wide) detected at line 3986
 []\OT1/lmss/m/n/10.95 1092
 []


Overfull \hbox (4.92744pt too wide) detected at line 3987
 []\OT1/lmss/m/n/10.95 1092
 []


Overfull \hbox (4.92744pt too wide) detected at line 3988
 []\OT1/lmss/m/n/10.95 1093
 []


Overfull \hbox (4.92744pt too wide) detected at line 3989
 []\OT1/lmss/m/n/10.95 1093
 []


Overfull \hbox (4.92744pt too wide) detected at line 3990
 []\OT1/lmss/m/n/10.95 1093
 []


Overfull \hbox (4.92744pt too wide) detected at line 3991
 []\OT1/lmss/m/n/10.95 1094
 []


Overfull \hbox (4.92744pt too wide) detected at line 3992
 []\OT1/lmss/m/n/10.95 1094
 []


Overfull \hbox (4.92744pt too wide) detected at line 3993
 []\OT1/lmss/m/n/10.95 1094
 []


Overfull \hbox (4.92744pt too wide) detected at line 3994
 []\OT1/lmss/m/n/10.95 1095
 []


Overfull \hbox (4.92744pt too wide) detected at line 3995
 []\OT1/lmss/m/n/10.95 1095
 []


Overfull \hbox (4.92744pt too wide) detected at line 3996
 []\OT1/lmss/m/n/10.95 1095
 []


Overfull \hbox (4.92744pt too wide) detected at line 3997
 []\OT1/lmss/m/n/10.95 1095
 []


Overfull \hbox (4.92744pt too wide) detected at line 3998
 []\OT1/lmss/m/n/10.95 1096
 []


Overfull \hbox (4.92744pt too wide) detected at line 3999
 []\OT1/lmss/m/n/10.95 1096
 []


Overfull \hbox (4.92744pt too wide) detected at line 4000
 []\OT1/lmss/m/n/10.95 1096
 []


Overfull \hbox (4.92744pt too wide) detected at line 4001
 []\OT1/lmss/m/n/10.95 1096
 []


Overfull \hbox (4.92744pt too wide) detected at line 4002
 []\OT1/lmss/m/n/10.95 1096
 []


Overfull \hbox (4.92744pt too wide) detected at line 4003
 []\OT1/lmss/m/n/10.95 1097
 []


Overfull \hbox (4.92744pt too wide) detected at line 4004
 []\OT1/lmss/m/n/10.95 1097
 []


Overfull \hbox (4.92744pt too wide) detected at line 4005
 []\OT1/lmss/m/n/10.95 1097
 []


Overfull \hbox (4.92744pt too wide) detected at line 4006
 []\OT1/lmss/m/n/10.95 1097
 []


Overfull \hbox (4.92744pt too wide) detected at line 4007
 []\OT1/lmss/m/n/10.95 1097
 []


Overfull \hbox (4.92744pt too wide) detected at line 4008
 []\OT1/lmss/m/n/10.95 1097
 []


Overfull \hbox (4.92744pt too wide) detected at line 4009
 []\OT1/lmss/m/n/10.95 1098
 []


Overfull \hbox (4.92744pt too wide) detected at line 4010
 []\OT1/lmss/m/n/10.95 1098
 []


Overfull \hbox (4.92744pt too wide) detected at line 4011
 []\OT1/lmss/m/n/10.95 1098
 []


Overfull \hbox (4.92744pt too wide) detected at line 4012
 []\OT1/lmss/m/n/10.95 1099
 []


Overfull \hbox (4.92744pt too wide) detected at line 4013
 []\OT1/lmss/m/n/10.95 1099
 []


Overfull \hbox (4.92744pt too wide) detected at line 4014
 []\OT1/lmss/m/n/10.95 1099
 []


Overfull \hbox (4.92744pt too wide) detected at line 4015
 []\OT1/lmss/m/n/10.95 1099
 []




Package fancyhdr Warning: \headheight is too small (12.0pt): 
(fancyhdr)                Make it at least 13.59999pt, for example:
(fancyhdr)                \setlength{\headheight}{13.59999pt}.
(fancyhdr)                You might also make \topmargin smaller:
(fancyhdr)                \addtolength{\topmargin}{-1.59999pt}.

[101]
Overfull \hbox (4.92744pt too wide) detected at line 4017
 []\OT1/lmss/m/n/10.95 1102
 []


Overfull \hbox (4.92744pt too wide) detected at line 4018
 []\OT1/lmss/m/n/10.95 1102
 []


Overfull \hbox (4.92744pt too wide) detected at line 4019
 []\OT1/lmss/m/n/10.95 1103
 []


Overfull \hbox (4.92744pt too wide) detected at line 4020
 []\OT1/lmss/m/n/10.95 1103
 []


Overfull \hbox (4.92744pt too wide) detected at line 4021
 []\OT1/lmss/m/n/10.95 1103
 []


Overfull \hbox (4.92744pt too wide) detected at line 4022
 []\OT1/lmss/m/n/10.95 1103
 []


Overfull \hbox (4.92744pt too wide) detected at line 4023
 []\OT1/lmss/m/n/10.95 1103
 []


Overfull \hbox (4.92744pt too wide) detected at line 4024
 []\OT1/lmss/m/n/10.95 1103
 []


Overfull \hbox (4.92744pt too wide) detected at line 4025
 []\OT1/lmss/m/n/10.95 1104
 []


Overfull \hbox (4.92744pt too wide) detected at line 4026
 []\OT1/lmss/m/n/10.95 1104
 []


Overfull \hbox (4.92744pt too wide) detected at line 4027
 []\OT1/lmss/m/n/10.95 1104
 []


Overfull \hbox (4.92744pt too wide) detected at line 4028
 []\OT1/lmss/m/n/10.95 1104
 []


Overfull \hbox (4.92744pt too wide) detected at line 4029
 []\OT1/lmss/m/n/10.95 1104
 []


Overfull \hbox (4.92744pt too wide) detected at line 4030
 []\OT1/lmss/m/n/10.95 1104
 []


Overfull \hbox (4.92744pt too wide) detected at line 4031
 []\OT1/lmss/m/n/10.95 1104
 []


Overfull \hbox (4.92744pt too wide) detected at line 4032
 []\OT1/lmss/m/n/10.95 1105
 []


Overfull \hbox (4.92744pt too wide) detected at line 4033
 []\OT1/lmss/m/n/10.95 1105
 []


Overfull \hbox (4.92744pt too wide) detected at line 4034
 []\OT1/lmss/m/n/10.95 1105
 []


Overfull \hbox (4.92744pt too wide) detected at line 4035
 []\OT1/lmss/m/n/10.95 1105
 []


Overfull \hbox (4.92744pt too wide) detected at line 4036
 []\OT1/lmss/m/n/10.95 1105
 []


Overfull \hbox (4.92744pt too wide) detected at line 4037
 []\OT1/lmss/m/n/10.95 1105
 []


Overfull \hbox (4.92744pt too wide) detected at line 4038
 []\OT1/lmss/m/n/10.95 1106
 []


Overfull \hbox (4.92744pt too wide) detected at line 4039
 []\OT1/lmss/m/n/10.95 1106
 []


Overfull \hbox (4.92744pt too wide) detected at line 4040
 []\OT1/lmss/m/n/10.95 1106
 []


Overfull \hbox (4.92744pt too wide) detected at line 4041
 []\OT1/lmss/m/n/10.95 1106
 []


Overfull \hbox (4.92744pt too wide) detected at line 4042
 []\OT1/lmss/m/n/10.95 1106
 []


Overfull \hbox (4.92744pt too wide) detected at line 4043
 []\OT1/lmss/m/n/10.95 1106
 []


Overfull \hbox (4.92744pt too wide) detected at line 4044
 []\OT1/lmss/m/n/10.95 1106
 []


Overfull \hbox (4.92744pt too wide) detected at line 4045
 []\OT1/lmss/m/n/10.95 1107
 []


Overfull \hbox (4.92744pt too wide) detected at line 4046
 []\OT1/lmss/m/n/10.95 1107
 []


Overfull \hbox (4.92744pt too wide) detected at line 4047
 []\OT1/lmss/m/n/10.95 1107
 []


Overfull \hbox (4.92744pt too wide) detected at line 4048
 []\OT1/lmss/m/n/10.95 1107
 []


Overfull \hbox (4.92744pt too wide) detected at line 4049
 []\OT1/lmss/m/n/10.95 1107
 []


Overfull \hbox (4.92744pt too wide) detected at line 4050
 []\OT1/lmss/m/n/10.95 1107
 []


Overfull \hbox (4.92744pt too wide) detected at line 4051
 []\OT1/lmss/m/n/10.95 1107
 []


Overfull \hbox (4.92744pt too wide) detected at line 4052
 []\OT1/lmss/m/n/10.95 1108
 []




Package fancyhdr Warning: \headheight is too small (12.0pt): 
(fancyhdr)                Make it at least 13.59999pt, for example:
(fancyhdr)                \setlength{\headheight}{13.59999pt}.
(fancyhdr)                You might also make \topmargin smaller:
(fancyhdr)                \addtolength{\topmargin}{-1.59999pt}.

[102]
Overfull \hbox (4.92744pt too wide) detected at line 4053
 []\OT1/lmss/m/n/10.95 1108
 []


Overfull \hbox (4.92744pt too wide) detected at line 4054
 []\OT1/lmss/m/n/10.95 1108
 []


Overfull \hbox (4.92744pt too wide) detected at line 4055
 []\OT1/lmss/m/n/10.95 1108
 []


Overfull \hbox (4.92744pt too wide) detected at line 4056
 []\OT1/lmss/m/n/10.95 1108
 []


Overfull \hbox (4.92744pt too wide) detected at line 4057
 []\OT1/lmss/m/n/10.95 1108
 []


Overfull \hbox (4.92744pt too wide) detected at line 4058
 []\OT1/lmss/m/n/10.95 1109
 []


Overfull \hbox (4.92744pt too wide) detected at line 4059
 []\OT1/lmss/m/n/10.95 1109
 []


Overfull \hbox (4.92744pt too wide) detected at line 4060
 []\OT1/lmss/m/n/10.95 1109
 []


Overfull \hbox (4.92744pt too wide) detected at line 4061
 []\OT1/lmss/m/n/10.95 1109
 []


Overfull \hbox (4.92744pt too wide) detected at line 4062
 []\OT1/lmss/m/n/10.95 1109
 []


Overfull \hbox (4.92744pt too wide) detected at line 4063
 []\OT1/lmss/m/n/10.95 1110
 []


Overfull \hbox (4.92744pt too wide) detected at line 4064
 []\OT1/lmss/m/n/10.95 1110
 []


Overfull \hbox (4.92744pt too wide) detected at line 4065
 []\OT1/lmss/m/n/10.95 1110
 []


Overfull \hbox (4.92744pt too wide) detected at line 4066
 []\OT1/lmss/m/n/10.95 1110
 []


Overfull \hbox (4.92744pt too wide) detected at line 4067
 []\OT1/lmss/m/n/10.95 1110
 []


Overfull \hbox (4.92744pt too wide) detected at line 4068
 []\OT1/lmss/m/n/10.95 1110
 []


Overfull \hbox (4.92744pt too wide) detected at line 4069
 []\OT1/lmss/m/n/10.95 1110
 []


Overfull \hbox (4.92744pt too wide) detected at line 4070
 []\OT1/lmss/m/n/10.95 1111
 []


Overfull \hbox (4.92744pt too wide) detected at line 4071
 []\OT1/lmss/m/n/10.95 1111
 []


Overfull \hbox (4.92744pt too wide) detected at line 4072
 []\OT1/lmss/m/n/10.95 1111
 []


Overfull \hbox (4.92744pt too wide) detected at line 4073
 []\OT1/lmss/m/n/10.95 1111
 []


Overfull \hbox (4.92744pt too wide) detected at line 4074
 []\OT1/lmss/m/n/10.95 1111
 []


Overfull \hbox (4.92744pt too wide) detected at line 4075
 []\OT1/lmss/m/n/10.95 1112
 []


Overfull \hbox (4.92744pt too wide) detected at line 4076
 []\OT1/lmss/m/n/10.95 1112
 []


Overfull \hbox (4.92744pt too wide) detected at line 4077
 []\OT1/lmss/m/n/10.95 1112
 []


Overfull \hbox (4.92744pt too wide) detected at line 4078
 []\OT1/lmss/m/n/10.95 1112
 []


Overfull \hbox (4.92744pt too wide) detected at line 4079
 []\OT1/lmss/m/n/10.95 1112
 []


Overfull \hbox (4.92744pt too wide) detected at line 4080
 []\OT1/lmss/m/n/10.95 1112
 []


Overfull \hbox (4.92744pt too wide) detected at line 4081
 []\OT1/lmss/m/n/10.95 1113
 []


Overfull \hbox (4.92744pt too wide) detected at line 4082
 []\OT1/lmss/m/n/10.95 1113
 []


Overfull \hbox (4.92744pt too wide) detected at line 4083
 []\OT1/lmss/m/n/10.95 1113
 []


Overfull \hbox (4.92744pt too wide) detected at line 4084
 []\OT1/lmss/m/n/10.95 1113
 []


Overfull \hbox (4.92744pt too wide) detected at line 4085
 []\OT1/lmss/m/n/10.95 1113
 []


Overfull \hbox (4.92744pt too wide) detected at line 4086
 []\OT1/lmss/m/n/10.95 1113
 []


Overfull \hbox (4.92744pt too wide) detected at line 4087
 []\OT1/lmss/m/n/10.95 1113
 []


Overfull \hbox (4.92744pt too wide) detected at line 4088
 []\OT1/lmss/m/n/10.95 1113
 []


Overfull \hbox (4.92744pt too wide) detected at line 4089
 []\OT1/lmss/m/n/10.95 1114
 []




Package fancyhdr Warning: \headheight is too small (12.0pt): 
(fancyhdr)                Make it at least 13.59999pt, for example:
(fancyhdr)                \setlength{\headheight}{13.59999pt}.
(fancyhdr)                You might also make \topmargin smaller:
(fancyhdr)                \addtolength{\topmargin}{-1.59999pt}.

[103]
Overfull \hbox (4.92744pt too wide) detected at line 4090
 []\OT1/lmss/m/n/10.95 1114
 []


Overfull \hbox (4.92744pt too wide) detected at line 4091
 []\OT1/lmss/m/n/10.95 1114
 []


Overfull \hbox (4.92744pt too wide) detected at line 4092
 []\OT1/lmss/m/n/10.95 1114
 []


Overfull \hbox (4.92744pt too wide) detected at line 4093
 []\OT1/lmss/m/n/10.95 1114
 []


Overfull \hbox (4.92744pt too wide) detected at line 4094
 []\OT1/lmss/m/n/10.95 1114
 []


Overfull \hbox (4.92744pt too wide) detected at line 4095
 []\OT1/lmss/m/n/10.95 1115
 []


Overfull \hbox (4.92744pt too wide) detected at line 4096
 []\OT1/lmss/m/n/10.95 1115
 []


Overfull \hbox (4.92744pt too wide) detected at line 4097
 []\OT1/lmss/m/n/10.95 1115
 []


Overfull \hbox (4.92744pt too wide) detected at line 4098
 []\OT1/lmss/m/n/10.95 1115
 []


Overfull \hbox (4.92744pt too wide) detected at line 4099
 []\OT1/lmss/m/n/10.95 1115
 []


Overfull \hbox (4.92744pt too wide) detected at line 4100
 []\OT1/lmss/m/n/10.95 1115
 []


Overfull \hbox (4.92744pt too wide) detected at line 4101
 []\OT1/lmss/m/n/10.95 1115
 []


Overfull \hbox (4.92744pt too wide) detected at line 4102
 []\OT1/lmss/m/n/10.95 1115
 []


Overfull \hbox (4.92744pt too wide) detected at line 4103
 []\OT1/lmss/m/n/10.95 1116
 []


Overfull \hbox (4.92744pt too wide) detected at line 4104
 []\OT1/lmss/m/n/10.95 1116
 []


Overfull \hbox (4.92744pt too wide) detected at line 4105
 []\OT1/lmss/m/n/10.95 1116
 []


Overfull \hbox (4.92744pt too wide) detected at line 4106
 []\OT1/lmss/m/n/10.95 1116
 []


Overfull \hbox (4.92744pt too wide) detected at line 4107
 []\OT1/lmss/m/n/10.95 1116
 []


Overfull \hbox (4.92744pt too wide) detected at line 4108
 []\OT1/lmss/m/n/10.95 1116
 []


Overfull \hbox (4.92744pt too wide) detected at line 4109
 []\OT1/lmss/m/n/10.95 1117
 []


Overfull \hbox (4.92744pt too wide) detected at line 4110
 []\OT1/lmss/m/n/10.95 1117
 []


Overfull \hbox (4.92744pt too wide) detected at line 4111
 []\OT1/lmss/m/n/10.95 1117
 []


Overfull \hbox (4.92744pt too wide) detected at line 4112
 []\OT1/lmss/m/n/10.95 1117
 []


Overfull \hbox (4.92744pt too wide) detected at line 4113
 []\OT1/lmss/m/n/10.95 1117
 []


Overfull \hbox (4.92744pt too wide) detected at line 4114
 []\OT1/lmss/m/n/10.95 1117
 []


Overfull \hbox (4.92744pt too wide) detected at line 4115
 []\OT1/lmss/m/n/10.95 1117
 []


Overfull \hbox (4.92744pt too wide) detected at line 4116
 []\OT1/lmss/m/n/10.95 1118
 []


Overfull \hbox (4.92744pt too wide) detected at line 4117
 []\OT1/lmss/m/n/10.95 1118
 []


Overfull \hbox (4.92744pt too wide) detected at line 4118
 []\OT1/lmss/m/n/10.95 1118
 []


Overfull \hbox (4.92744pt too wide) detected at line 4119
 []\OT1/lmss/m/n/10.95 1118
 []


Overfull \hbox (4.92744pt too wide) detected at line 4120
 []\OT1/lmss/m/n/10.95 1118
 []


Overfull \hbox (4.92744pt too wide) detected at line 4121
 []\OT1/lmss/m/n/10.95 1119
 []


Overfull \hbox (4.92744pt too wide) detected at line 4122
 []\OT1/lmss/m/n/10.95 1119
 []


Overfull \hbox (4.92744pt too wide) detected at line 4123
 []\OT1/lmss/m/n/10.95 1119
 []


Overfull \hbox (4.92744pt too wide) detected at line 4124
 []\OT1/lmss/m/n/10.95 1119
 []


Overfull \hbox (4.92744pt too wide) detected at line 4125
 []\OT1/lmss/m/n/10.95 1119
 []


Overfull \hbox (4.92744pt too wide) detected at line 4126
 []\OT1/lmss/m/n/10.95 1119
 []




Package fancyhdr Warning: \headheight is too small (12.0pt): 
(fancyhdr)                Make it at least 13.59999pt, for example:
(fancyhdr)                \setlength{\headheight}{13.59999pt}.
(fancyhdr)                You might also make \topmargin smaller:
(fancyhdr)                \addtolength{\topmargin}{-1.59999pt}.

[104]
Overfull \hbox (4.92744pt too wide) detected at line 4127
 []\OT1/lmss/m/n/10.95 1120
 []


Overfull \hbox (4.92744pt too wide) detected at line 4128
 []\OT1/lmss/m/n/10.95 1120
 []


Overfull \hbox (4.92744pt too wide) detected at line 4129
 []\OT1/lmss/m/n/10.95 1120
 []


Overfull \hbox (4.92744pt too wide) detected at line 4130
 []\OT1/lmss/m/n/10.95 1120
 []


Overfull \hbox (4.92744pt too wide) detected at line 4131
 []\OT1/lmss/m/n/10.95 1120
 []


Overfull \hbox (4.92744pt too wide) detected at line 4132
 []\OT1/lmss/m/n/10.95 1120
 []


Overfull \hbox (4.92744pt too wide) detected at line 4133
 []\OT1/lmss/m/n/10.95 1120
 []


Overfull \hbox (4.92744pt too wide) detected at line 4134
 []\OT1/lmss/m/n/10.95 1121
 []


Overfull \hbox (4.92744pt too wide) detected at line 4135
 []\OT1/lmss/m/n/10.95 1121
 []


Overfull \hbox (4.92744pt too wide) detected at line 4136
 []\OT1/lmss/m/n/10.95 1121
 []


Overfull \hbox (4.92744pt too wide) detected at line 4137
 []\OT1/lmss/m/n/10.95 1121
 []


Overfull \hbox (4.92744pt too wide) detected at line 4138
 []\OT1/lmss/m/n/10.95 1121
 []


Overfull \hbox (4.92744pt too wide) detected at line 4139
 []\OT1/lmss/m/n/10.95 1122
 []


Overfull \hbox (4.92744pt too wide) detected at line 4140
 []\OT1/lmss/m/n/10.95 1122
 []


Overfull \hbox (4.92744pt too wide) detected at line 4141
 []\OT1/lmss/m/n/10.95 1122
 []


Overfull \hbox (4.92744pt too wide) detected at line 4142
 []\OT1/lmss/m/n/10.95 1122
 []


Overfull \hbox (4.92744pt too wide) detected at line 4143
 []\OT1/lmss/m/n/10.95 1122
 []


Overfull \hbox (4.92744pt too wide) detected at line 4144
 []\OT1/lmss/m/n/10.95 1122
 []


Overfull \hbox (4.92744pt too wide) detected at line 4145
 []\OT1/lmss/m/n/10.95 1123
 []


Overfull \hbox (4.92744pt too wide) detected at line 4146
 []\OT1/lmss/m/n/10.95 1123
 []


Overfull \hbox (4.92744pt too wide) detected at line 4147
 []\OT1/lmss/m/n/10.95 1123
 []


Overfull \hbox (4.92744pt too wide) detected at line 4148
 []\OT1/lmss/m/n/10.95 1123
 []


Overfull \hbox (4.92744pt too wide) detected at line 4149
 []\OT1/lmss/m/n/10.95 1123
 []


Overfull \hbox (4.92744pt too wide) detected at line 4150
 []\OT1/lmss/m/n/10.95 1123
 []


Overfull \hbox (4.92744pt too wide) detected at line 4151
 []\OT1/lmss/m/n/10.95 1124
 []


Overfull \hbox (4.92744pt too wide) detected at line 4152
 []\OT1/lmss/m/n/10.95 1124
 []


Overfull \hbox (4.92744pt too wide) detected at line 4153
 []\OT1/lmss/m/n/10.95 1124
 []


Overfull \hbox (4.92744pt too wide) detected at line 4154
 []\OT1/lmss/m/n/10.95 1124
 []


Overfull \hbox (4.92744pt too wide) detected at line 4155
 []\OT1/lmss/m/n/10.95 1124
 []


Overfull \hbox (4.92744pt too wide) detected at line 4156
 []\OT1/lmss/m/n/10.95 1124
 []


Overfull \hbox (4.92744pt too wide) detected at line 4157
 []\OT1/lmss/m/n/10.95 1125
 []


Overfull \hbox (4.92744pt too wide) detected at line 4158
 []\OT1/lmss/m/n/10.95 1125
 []


Overfull \hbox (4.92744pt too wide) detected at line 4159
 []\OT1/lmss/m/n/10.95 1125
 []


Overfull \hbox (4.92744pt too wide) detected at line 4160
 []\OT1/lmss/m/n/10.95 1125
 []


Overfull \hbox (4.92744pt too wide) detected at line 4161
 []\OT1/lmss/m/n/10.95 1125
 []


Overfull \hbox (4.92744pt too wide) detected at line 4162
 []\OT1/lmss/m/n/10.95 1125
 []


Overfull \hbox (4.92744pt too wide) detected at line 4163
 []\OT1/lmss/m/n/10.95 1126
 []




Package fancyhdr Warning: \headheight is too small (12.0pt): 
(fancyhdr)                Make it at least 13.59999pt, for example:
(fancyhdr)                \setlength{\headheight}{13.59999pt}.
(fancyhdr)                You might also make \topmargin smaller:
(fancyhdr)                \addtolength{\topmargin}{-1.59999pt}.

[105]
Overfull \hbox (4.92744pt too wide) detected at line 4164
 []\OT1/lmss/m/n/10.95 1126
 []


Overfull \hbox (4.92744pt too wide) detected at line 4165
 []\OT1/lmss/m/n/10.95 1126
 []


Overfull \hbox (4.92744pt too wide) detected at line 4166
 []\OT1/lmss/m/n/10.95 1126
 []


Overfull \hbox (4.92744pt too wide) detected at line 4167
 []\OT1/lmss/m/n/10.95 1127
 []


Overfull \hbox (4.92744pt too wide) detected at line 4168
 []\OT1/lmss/m/n/10.95 1127
 []


Overfull \hbox (4.92744pt too wide) detected at line 4169
 []\OT1/lmss/m/n/10.95 1127
 []


Overfull \hbox (4.92744pt too wide) detected at line 4170
 []\OT1/lmss/m/n/10.95 1127
 []


Overfull \hbox (4.92744pt too wide) detected at line 4171
 []\OT1/lmss/m/n/10.95 1127
 []


Overfull \hbox (4.92744pt too wide) detected at line 4172
 []\OT1/lmss/m/n/10.95 1127
 []


Overfull \hbox (4.92744pt too wide) detected at line 4173
 []\OT1/lmss/m/n/10.95 1127
 []


Overfull \hbox (4.92744pt too wide) detected at line 4174
 []\OT1/lmss/m/n/10.95 1128
 []


Overfull \hbox (4.92744pt too wide) detected at line 4175
 []\OT1/lmss/m/n/10.95 1128
 []


Overfull \hbox (4.92744pt too wide) detected at line 4176
 []\OT1/lmss/m/n/10.95 1128
 []


Overfull \hbox (4.92744pt too wide) detected at line 4177
 []\OT1/lmss/m/n/10.95 1128
 []


Overfull \hbox (4.92744pt too wide) detected at line 4178
 []\OT1/lmss/m/n/10.95 1128
 []


Overfull \hbox (4.92744pt too wide) detected at line 4179
 []\OT1/lmss/m/n/10.95 1129
 []


Overfull \hbox (4.92744pt too wide) detected at line 4180
 []\OT1/lmss/m/n/10.95 1129
 []


Overfull \hbox (4.92744pt too wide) detected at line 4181
 []\OT1/lmss/m/n/10.95 1129
 []


Overfull \hbox (4.92744pt too wide) detected at line 4182
 []\OT1/lmss/m/n/10.95 1129
 []


Overfull \hbox (1.58195pt too wide) detected at line 4183
\OT1/lmss/m/n/10.95 V.100 
 []


Overfull \hbox (4.92744pt too wide) detected at line 4183
 []\OT1/lmss/m/n/10.95 1129
 []


Overfull \hbox (0.24365pt too wide) detected at line 4184
\OT1/lmss/m/n/10.95 V.100.1 
 []


Overfull \hbox (4.92744pt too wide) detected at line 4184
 []\OT1/lmss/m/n/10.95 1129
 []


Overfull \hbox (1.58195pt too wide) detected at line 4185
\OT1/lmss/m/n/10.95 V.101 
 []


Overfull \hbox (4.92744pt too wide) detected at line 4185
 []\OT1/lmss/m/n/10.95 1129
 []


Overfull \hbox (1.58195pt too wide) detected at line 4186
\OT1/lmss/m/n/10.95 V.102 
 []


Overfull \hbox (4.92744pt too wide) detected at line 4186
 []\OT1/lmss/m/n/10.95 1130
 []


Overfull \hbox (1.58195pt too wide) detected at line 4187
\OT1/lmss/m/n/10.95 V.103 
 []


Overfull \hbox (4.92744pt too wide) detected at line 4187
 []\OT1/lmss/m/n/10.95 1130
 []


Overfull \hbox (0.24365pt too wide) detected at line 4188
\OT1/lmss/m/n/10.95 V.103.1 
 []


Overfull \hbox (4.92744pt too wide) detected at line 4188
 []\OT1/lmss/m/n/10.95 1130
 []


Overfull \hbox (0.24365pt too wide) detected at line 4189
\OT1/lmss/m/n/10.95 V.103.2 
 []


Overfull \hbox (4.92744pt too wide) detected at line 4189
 []\OT1/lmss/m/n/10.95 1130
 []


Overfull \hbox (1.58195pt too wide) detected at line 4190
\OT1/lmss/m/n/10.95 V.104 
 []


Overfull \hbox (4.92744pt too wide) detected at line 4190
 []\OT1/lmss/m/n/10.95 1130
 []


Overfull \hbox (1.58195pt too wide) detected at line 4191
\OT1/lmss/m/n/10.95 V.105 
 []


Overfull \hbox (4.92744pt too wide) detected at line 4191
 []\OT1/lmss/m/n/10.95 1130
 []


Overfull \hbox (0.24365pt too wide) detected at line 4192
\OT1/lmss/m/n/10.95 V.105.1 
 []


Overfull \hbox (4.92744pt too wide) detected at line 4192
 []\OT1/lmss/m/n/10.95 1130
 []


Overfull \hbox (0.24365pt too wide) detected at line 4193
\OT1/lmss/m/n/10.95 V.105.2 
 []


Overfull \hbox (4.92744pt too wide) detected at line 4193
 []\OT1/lmss/m/n/10.95 1131
 []


Overfull \hbox (0.24365pt too wide) detected at line 4194
\OT1/lmss/m/n/10.95 V.105.3 
 []


Overfull \hbox (4.92744pt too wide) detected at line 4194
 []\OT1/lmss/m/n/10.95 1131
 []


Overfull \hbox (1.58195pt too wide) detected at line 4195
\OT1/lmss/m/n/10.95 V.106 
 []


Overfull \hbox (4.92744pt too wide) detected at line 4195
 []\OT1/lmss/m/n/10.95 1131
 []


Overfull \hbox (0.24365pt too wide) detected at line 4196
\OT1/lmss/m/n/10.95 V.106.1 
 []


Overfull \hbox (4.92744pt too wide) detected at line 4196
 []\OT1/lmss/m/n/10.95 1131
 []


Overfull \hbox (0.24365pt too wide) detected at line 4197
\OT1/lmss/m/n/10.95 V.106.2 
 []


Overfull \hbox (4.92744pt too wide) detected at line 4197
 []\OT1/lmss/m/n/10.95 1131
 []


Overfull \hbox (1.58195pt too wide) detected at line 4198
\OT1/lmss/m/n/10.95 V.107 
 []


Overfull \hbox (4.92744pt too wide) detected at line 4198
 []\OT1/lmss/m/n/10.95 1132
 []


Overfull \hbox (1.58195pt too wide) detected at line 4199
\OT1/lmss/m/n/10.95 V.108 
 []


Overfull \hbox (4.92744pt too wide) detected at line 4199
 []\OT1/lmss/m/n/10.95 1132
 []


Overfull \hbox (1.58195pt too wide) detected at line 4200
\OT1/lmss/m/n/10.95 V.109 
 []


Overfull \hbox (4.92744pt too wide) detected at line 4200
 []\OT1/lmss/m/n/10.95 1132
 []




Package fancyhdr Warning: \headheight is too small (12.0pt): 
(fancyhdr)                Make it at least 13.59999pt, for example:
(fancyhdr)                \setlength{\headheight}{13.59999pt}.
(fancyhdr)                You might also make \topmargin smaller:
(fancyhdr)                \addtolength{\topmargin}{-1.59999pt}.

[106]
Overfull \hbox (1.58195pt too wide) detected at line 4201
\OT1/lmss/m/n/10.95 V.110 
 []


Overfull \hbox (4.92744pt too wide) detected at line 4201
 []\OT1/lmss/m/n/10.95 1132
 []


Overfull \hbox (1.58195pt too wide) detected at line 4202
\OT1/lmss/m/n/10.95 V.111 
 []


Overfull \hbox (4.92744pt too wide) detected at line 4202
 []\OT1/lmss/m/n/10.95 1132
 []


Overfull \hbox (0.24365pt too wide) detected at line 4203
\OT1/lmss/m/n/10.95 V.111.1 
 []


Overfull \hbox (4.92744pt too wide) detected at line 4203
 []\OT1/lmss/m/n/10.95 1132
 []


Overfull \hbox (0.24365pt too wide) detected at line 4204
\OT1/lmss/m/n/10.95 V.111.2 
 []


Overfull \hbox (4.92744pt too wide) detected at line 4204
 []\OT1/lmss/m/n/10.95 1133
 []


Overfull \hbox (1.58195pt too wide) detected at line 4205
\OT1/lmss/m/n/10.95 V.112 
 []


Overfull \hbox (4.92744pt too wide) detected at line 4205
 []\OT1/lmss/m/n/10.95 1133
 []


Overfull \hbox (0.24365pt too wide) detected at line 4206
\OT1/lmss/m/n/10.95 V.112.1 
 []


Overfull \hbox (4.92744pt too wide) detected at line 4206
 []\OT1/lmss/m/n/10.95 1133
 []


Overfull \hbox (1.58195pt too wide) detected at line 4207
\OT1/lmss/m/n/10.95 V.113 
 []


Overfull \hbox (4.92744pt too wide) detected at line 4207
 []\OT1/lmss/m/n/10.95 1133
 []


Overfull \hbox (1.58195pt too wide) detected at line 4208
\OT1/lmss/m/n/10.95 V.114 
 []


Overfull \hbox (4.92744pt too wide) detected at line 4208
 []\OT1/lmss/m/n/10.95 1133
 []


Overfull \hbox (1.58195pt too wide) detected at line 4209
\OT1/lmss/m/n/10.95 V.115 
 []


Overfull \hbox (4.92744pt too wide) detected at line 4209
 []\OT1/lmss/m/n/10.95 1133
 []


Overfull \hbox (1.58195pt too wide) detected at line 4210
\OT1/lmss/m/n/10.95 V.116 
 []


Overfull \hbox (4.92744pt too wide) detected at line 4210
 []\OT1/lmss/m/n/10.95 1133
 []


Overfull \hbox (1.58195pt too wide) detected at line 4211
\OT1/lmss/m/n/10.95 V.117 
 []


Overfull \hbox (4.92744pt too wide) detected at line 4211
 []\OT1/lmss/m/n/10.95 1133
 []


Overfull \hbox (0.24365pt too wide) detected at line 4212
\OT1/lmss/m/n/10.95 V.117.1 
 []


Overfull \hbox (4.92744pt too wide) detected at line 4212
 []\OT1/lmss/m/n/10.95 1133
 []


Overfull \hbox (0.24365pt too wide) detected at line 4213
\OT1/lmss/m/n/10.95 V.117.2 
 []


Overfull \hbox (4.92744pt too wide) detected at line 4213
 []\OT1/lmss/m/n/10.95 1134
 []


Overfull \hbox (1.58195pt too wide) detected at line 4214
\OT1/lmss/m/n/10.95 V.118 
 []


Overfull \hbox (4.92744pt too wide) detected at line 4214
 []\OT1/lmss/m/n/10.95 1134
 []


Overfull \hbox (1.58195pt too wide) detected at line 4215
\OT1/lmss/m/n/10.95 V.119 
 []


Overfull \hbox (4.92744pt too wide) detected at line 4215
 []\OT1/lmss/m/n/10.95 1134
 []


Overfull \hbox (1.58195pt too wide) detected at line 4216
\OT1/lmss/m/n/10.95 V.120 
 []


Overfull \hbox (4.92744pt too wide) detected at line 4216
 []\OT1/lmss/m/n/10.95 1134
 []


Overfull \hbox (1.58195pt too wide) detected at line 4217
\OT1/lmss/m/n/10.95 V.121 
 []


Overfull \hbox (4.92744pt too wide) detected at line 4217
 []\OT1/lmss/m/n/10.95 1134
 []


Overfull \hbox (1.58195pt too wide) detected at line 4218
\OT1/lmss/m/n/10.95 V.122 
 []


Overfull \hbox (4.92744pt too wide) detected at line 4218
 []\OT1/lmss/m/n/10.95 1134
 []


Overfull \hbox (0.24365pt too wide) detected at line 4219
\OT1/lmss/m/n/10.95 V.122.1 
 []


Overfull \hbox (4.92744pt too wide) detected at line 4219
 []\OT1/lmss/m/n/10.95 1134
 []


Overfull \hbox (1.58195pt too wide) detected at line 4220
\OT1/lmss/m/n/10.95 V.123 
 []


Overfull \hbox (4.92744pt too wide) detected at line 4220
 []\OT1/lmss/m/n/10.95 1134
 []


Overfull \hbox (0.24365pt too wide) detected at line 4221
\OT1/lmss/m/n/10.95 V.123.1 
 []


Overfull \hbox (4.92744pt too wide) detected at line 4221
 []\OT1/lmss/m/n/10.95 1135
 []


Overfull \hbox (1.58195pt too wide) detected at line 4222
\OT1/lmss/m/n/10.95 V.124 
 []


Overfull \hbox (4.92744pt too wide) detected at line 4222
 []\OT1/lmss/m/n/10.95 1135
 []


Overfull \hbox (1.58195pt too wide) detected at line 4223
\OT1/lmss/m/n/10.95 V.125 
 []


Overfull \hbox (4.92744pt too wide) detected at line 4223
 []\OT1/lmss/m/n/10.95 1135
 []


Overfull \hbox (0.24365pt too wide) detected at line 4224
\OT1/lmss/m/n/10.95 V.125.1 
 []


Overfull \hbox (4.92744pt too wide) detected at line 4224
 []\OT1/lmss/m/n/10.95 1135
 []


Overfull \hbox (1.58195pt too wide) detected at line 4225
\OT1/lmss/m/n/10.95 V.126 
 []


Overfull \hbox (4.92744pt too wide) detected at line 4225
 []\OT1/lmss/m/n/10.95 1135
 []


Overfull \hbox (1.58195pt too wide) detected at line 4226
\OT1/lmss/m/n/10.95 V.127 
 []


Overfull \hbox (4.92744pt too wide) detected at line 4226
 []\OT1/lmss/m/n/10.95 1135
 []


Overfull \hbox (0.24365pt too wide) detected at line 4227
\OT1/lmss/m/n/10.95 V.127.1 
 []


Overfull \hbox (4.92744pt too wide) detected at line 4227
 []\OT1/lmss/m/n/10.95 1135
 []


Overfull \hbox (1.58195pt too wide) detected at line 4228
\OT1/lmss/m/n/10.95 V.128 
 []


Overfull \hbox (4.92744pt too wide) detected at line 4228
 []\OT1/lmss/m/n/10.95 1135
 []


Overfull \hbox (0.24365pt too wide) detected at line 4229
\OT1/lmss/m/n/10.95 V.128.1 
 []


Overfull \hbox (4.92744pt too wide) detected at line 4229
 []\OT1/lmss/m/n/10.95 1135
 []


Overfull \hbox (1.58195pt too wide) detected at line 4230
\OT1/lmss/m/n/10.95 V.129 
 []


Overfull \hbox (4.92744pt too wide) detected at line 4230
 []\OT1/lmss/m/n/10.95 1136
 []


Overfull \hbox (1.58195pt too wide) detected at line 4231
\OT1/lmss/m/n/10.95 V.130 
 []


Overfull \hbox (4.92744pt too wide) detected at line 4231
 []\OT1/lmss/m/n/10.95 1136
 []


Overfull \hbox (1.58195pt too wide) detected at line 4232
\OT1/lmss/m/n/10.95 V.131 
 []


Overfull \hbox (4.92744pt too wide) detected at line 4232
 []\OT1/lmss/m/n/10.95 1136
 []


Overfull \hbox (0.24365pt too wide) detected at line 4233
\OT1/lmss/m/n/10.95 V.131.1 
 []


Overfull \hbox (4.92744pt too wide) detected at line 4233
 []\OT1/lmss/m/n/10.95 1136
 []


Overfull \hbox (1.58195pt too wide) detected at line 4234
\OT1/lmss/m/n/10.95 V.132 
 []


Overfull \hbox (4.92744pt too wide) detected at line 4234
 []\OT1/lmss/m/n/10.95 1136
 []


Overfull \hbox (0.24365pt too wide) detected at line 4235
\OT1/lmss/m/n/10.95 V.132.1 
 []


Overfull \hbox (4.92744pt too wide) detected at line 4235
 []\OT1/lmss/m/n/10.95 1136
 []


Overfull \hbox (0.24365pt too wide) detected at line 4236
\OT1/lmss/m/n/10.95 V.132.2 
 []


Overfull \hbox (4.92744pt too wide) detected at line 4236
 []\OT1/lmss/m/n/10.95 1136
 []


Overfull \hbox (0.24365pt too wide) detected at line 4237
\OT1/lmss/m/n/10.95 V.132.3 
 []


Overfull \hbox (4.92744pt too wide) detected at line 4237
 []\OT1/lmss/m/n/10.95 1137
 []




Package fancyhdr Warning: \headheight is too small (12.0pt): 
(fancyhdr)                Make it at least 13.59999pt, for example:
(fancyhdr)                \setlength{\headheight}{13.59999pt}.
(fancyhdr)                You might also make \topmargin smaller:
(fancyhdr)                \addtolength{\topmargin}{-1.59999pt}.

[107]
Overfull \hbox (1.58195pt too wide) detected at line 4238
\OT1/lmss/m/n/10.95 V.133 
 []


Overfull \hbox (4.92744pt too wide) detected at line 4238
 []\OT1/lmss/m/n/10.95 1137
 []


Overfull \hbox (0.24365pt too wide) detected at line 4239
\OT1/lmss/m/n/10.95 V.133.1 
 []


Overfull \hbox (4.92744pt too wide) detected at line 4239
 []\OT1/lmss/m/n/10.95 1137
 []


Overfull \hbox (0.24365pt too wide) detected at line 4240
\OT1/lmss/m/n/10.95 V.133.2 
 []


Overfull \hbox (4.92744pt too wide) detected at line 4240
 []\OT1/lmss/m/n/10.95 1137
 []


Overfull \hbox (1.58195pt too wide) detected at line 4241
\OT1/lmss/m/n/10.95 V.134 
 []


Overfull \hbox (4.92744pt too wide) detected at line 4241
 []\OT1/lmss/m/n/10.95 1137
 []


Overfull \hbox (1.58195pt too wide) detected at line 4242
\OT1/lmss/m/n/10.95 V.135 
 []


Overfull \hbox (4.92744pt too wide) detected at line 4242
 []\OT1/lmss/m/n/10.95 1137
 []


Overfull \hbox (1.58195pt too wide) detected at line 4243
\OT1/lmss/m/n/10.95 V.136 
 []


Overfull \hbox (4.92744pt too wide) detected at line 4243
 []\OT1/lmss/m/n/10.95 1137
 []


Overfull \hbox (1.58195pt too wide) detected at line 4244
\OT1/lmss/m/n/10.95 V.137 
 []


Overfull \hbox (4.92744pt too wide) detected at line 4244
 []\OT1/lmss/m/n/10.95 1137
 []


Overfull \hbox (1.58195pt too wide) detected at line 4245
\OT1/lmss/m/n/10.95 V.138 
 []


Overfull \hbox (4.92744pt too wide) detected at line 4245
 []\OT1/lmss/m/n/10.95 1138
 []


Overfull \hbox (1.58195pt too wide) detected at line 4246
\OT1/lmss/m/n/10.95 V.139 
 []


Overfull \hbox (4.92744pt too wide) detected at line 4246
 []\OT1/lmss/m/n/10.95 1138
 []


Overfull \hbox (1.58195pt too wide) detected at line 4247
\OT1/lmss/m/n/10.95 V.140 
 []


Overfull \hbox (4.92744pt too wide) detected at line 4247
 []\OT1/lmss/m/n/10.95 1138
 []


Overfull \hbox (1.58195pt too wide) detected at line 4248
\OT1/lmss/m/n/10.95 V.141 
 []


Overfull \hbox (4.92744pt too wide) detected at line 4248
 []\OT1/lmss/m/n/10.95 1138
 []


Overfull \hbox (0.24365pt too wide) detected at line 4249
\OT1/lmss/m/n/10.95 V.141.1 
 []


Overfull \hbox (4.92744pt too wide) detected at line 4249
 []\OT1/lmss/m/n/10.95 1138
 []


Overfull \hbox (1.58195pt too wide) detected at line 4250
\OT1/lmss/m/n/10.95 V.142 
 []


Overfull \hbox (4.92744pt too wide) detected at line 4250
 []\OT1/lmss/m/n/10.95 1138
 []


Overfull \hbox (0.24365pt too wide) detected at line 4251
\OT1/lmss/m/n/10.95 V.142.1 
 []


Overfull \hbox (4.92744pt too wide) detected at line 4251
 []\OT1/lmss/m/n/10.95 1138
 []


Overfull \hbox (1.58195pt too wide) detected at line 4252
\OT1/lmss/m/n/10.95 V.143 
 []


Overfull \hbox (4.92744pt too wide) detected at line 4252
 []\OT1/lmss/m/n/10.95 1139
 []


Overfull \hbox (1.58195pt too wide) detected at line 4253
\OT1/lmss/m/n/10.95 V.144 
 []


Overfull \hbox (4.92744pt too wide) detected at line 4253
 []\OT1/lmss/m/n/10.95 1139
 []


Overfull \hbox (0.24365pt too wide) detected at line 4254
\OT1/lmss/m/n/10.95 V.144.1 
 []


Overfull \hbox (4.92744pt too wide) detected at line 4254
 []\OT1/lmss/m/n/10.95 1139
 []


Overfull \hbox (0.24365pt too wide) detected at line 4255
\OT1/lmss/m/n/10.95 V.144.2 
 []


Overfull \hbox (4.92744pt too wide) detected at line 4255
 []\OT1/lmss/m/n/10.95 1139
 []


Overfull \hbox (1.58195pt too wide) detected at line 4256
\OT1/lmss/m/n/10.95 V.145 
 []


Overfull \hbox (4.92744pt too wide) detected at line 4256
 []\OT1/lmss/m/n/10.95 1139
 []


Overfull \hbox (0.24365pt too wide) detected at line 4257
\OT1/lmss/m/n/10.95 V.145.1 
 []


Overfull \hbox (4.92744pt too wide) detected at line 4257
 []\OT1/lmss/m/n/10.95 1139
 []


Overfull \hbox (1.58195pt too wide) detected at line 4258
\OT1/lmss/m/n/10.95 V.146 
 []


Overfull \hbox (4.92744pt too wide) detected at line 4258
 []\OT1/lmss/m/n/10.95 1139
 []


Overfull \hbox (1.58195pt too wide) detected at line 4259
\OT1/lmss/m/n/10.95 V.147 
 []


Overfull \hbox (4.92744pt too wide) detected at line 4259
 []\OT1/lmss/m/n/10.95 1140
 []


Overfull \hbox (0.24365pt too wide) detected at line 4260
\OT1/lmss/m/n/10.95 V.147.1 
 []


Overfull \hbox (4.92744pt too wide) detected at line 4260
 []\OT1/lmss/m/n/10.95 1140
 []


Overfull \hbox (1.58195pt too wide) detected at line 4261
\OT1/lmss/m/n/10.95 V.148 
 []


Overfull \hbox (4.92744pt too wide) detected at line 4261
 []\OT1/lmss/m/n/10.95 1140
 []


Overfull \hbox (0.24365pt too wide) detected at line 4262
\OT1/lmss/m/n/10.95 V.148.1 
 []


Overfull \hbox (4.92744pt too wide) detected at line 4262
 []\OT1/lmss/m/n/10.95 1140
 []


Overfull \hbox (0.24365pt too wide) detected at line 4263
\OT1/lmss/m/n/10.95 V.148.2 
 []


Overfull \hbox (4.92744pt too wide) detected at line 4263
 []\OT1/lmss/m/n/10.95 1140
 []


Overfull \hbox (1.58195pt too wide) detected at line 4264
\OT1/lmss/m/n/10.95 V.149 
 []


Overfull \hbox (4.92744pt too wide) detected at line 4264
 []\OT1/lmss/m/n/10.95 1140
 []


Overfull \hbox (0.24365pt too wide) detected at line 4265
\OT1/lmss/m/n/10.95 V.149.1 
 []


Overfull \hbox (4.92744pt too wide) detected at line 4265
 []\OT1/lmss/m/n/10.95 1140
 []


Overfull \hbox (0.24365pt too wide) detected at line 4266
\OT1/lmss/m/n/10.95 V.149.2 
 []


Overfull \hbox (4.92744pt too wide) detected at line 4266
 []\OT1/lmss/m/n/10.95 1141
 []


Overfull \hbox (1.58195pt too wide) detected at line 4267
\OT1/lmss/m/n/10.95 V.150 
 []


Overfull \hbox (4.92744pt too wide) detected at line 4267
 []\OT1/lmss/m/n/10.95 1141
 []


Overfull \hbox (0.24365pt too wide) detected at line 4268
\OT1/lmss/m/n/10.95 V.150.1 
 []


Overfull \hbox (4.92744pt too wide) detected at line 4268
 []\OT1/lmss/m/n/10.95 1141
 []


Overfull \hbox (0.24365pt too wide) detected at line 4269
\OT1/lmss/m/n/10.95 V.150.2 
 []


Overfull \hbox (4.92744pt too wide) detected at line 4269
 []\OT1/lmss/m/n/10.95 1141
 []


Overfull \hbox (1.58195pt too wide) detected at line 4270
\OT1/lmss/m/n/10.95 V.151 
 []


Overfull \hbox (4.92744pt too wide) detected at line 4270
 []\OT1/lmss/m/n/10.95 1141
 []


Overfull \hbox (0.24365pt too wide) detected at line 4271
\OT1/lmss/m/n/10.95 V.151.1 
 []


Overfull \hbox (4.92744pt too wide) detected at line 4271
 []\OT1/lmss/m/n/10.95 1141
 []


Overfull \hbox (0.24365pt too wide) detected at line 4272
\OT1/lmss/m/n/10.95 V.151.2 
 []


Overfull \hbox (4.92744pt too wide) detected at line 4272
 []\OT1/lmss/m/n/10.95 1142
 []


Overfull \hbox (1.58195pt too wide) detected at line 4273
\OT1/lmss/m/n/10.95 V.152 
 []


Overfull \hbox (4.92744pt too wide) detected at line 4273
 []\OT1/lmss/m/n/10.95 1142
 []


Overfull \hbox (1.58195pt too wide) detected at line 4274
\OT1/lmss/m/n/10.95 V.153 
 []


Overfull \hbox (4.92744pt too wide) detected at line 4274
 []\OT1/lmss/m/n/10.95 1142
 []




Package fancyhdr Warning: \headheight is too small (12.0pt): 
(fancyhdr)                Make it at least 13.59999pt, for example:
(fancyhdr)                \setlength{\headheight}{13.59999pt}.
(fancyhdr)                You might also make \topmargin smaller:
(fancyhdr)                \addtolength{\topmargin}{-1.59999pt}.

[108]
Overfull \hbox (1.58195pt too wide) detected at line 4275
\OT1/lmss/m/n/10.95 V.154 
 []


Overfull \hbox (4.92744pt too wide) detected at line 4275
 []\OT1/lmss/m/n/10.95 1142
 []


Overfull \hbox (1.58195pt too wide) detected at line 4276
\OT1/lmss/m/n/10.95 V.155 
 []


Overfull \hbox (4.92744pt too wide) detected at line 4276
 []\OT1/lmss/m/n/10.95 1142
 []


Overfull \hbox (0.24365pt too wide) detected at line 4277
\OT1/lmss/m/n/10.95 V.155.1 
 []


Overfull \hbox (4.92744pt too wide) detected at line 4277
 []\OT1/lmss/m/n/10.95 1142
 []


Overfull \hbox (0.24365pt too wide) detected at line 4278
\OT1/lmss/m/n/10.95 V.155.2 
 []


Overfull \hbox (4.92744pt too wide) detected at line 4278
 []\OT1/lmss/m/n/10.95 1142
 []


Overfull \hbox (1.58195pt too wide) detected at line 4279
\OT1/lmss/m/n/10.95 V.156 
 []


Overfull \hbox (4.92744pt too wide) detected at line 4279
 []\OT1/lmss/m/n/10.95 1142
 []


Overfull \hbox (0.24365pt too wide) detected at line 4280
\OT1/lmss/m/n/10.95 V.156.1 
 []


Overfull \hbox (4.92744pt too wide) detected at line 4280
 []\OT1/lmss/m/n/10.95 1142
 []


Overfull \hbox (0.24365pt too wide) detected at line 4281
\OT1/lmss/m/n/10.95 V.156.2 
 []


Overfull \hbox (4.92744pt too wide) detected at line 4281
 []\OT1/lmss/m/n/10.95 1143
 []


Overfull \hbox (1.58195pt too wide) detected at line 4282
\OT1/lmss/m/n/10.95 V.157 
 []


Overfull \hbox (4.92744pt too wide) detected at line 4282
 []\OT1/lmss/m/n/10.95 1143
 []


Overfull \hbox (0.24365pt too wide) detected at line 4283
\OT1/lmss/m/n/10.95 V.157.1 
 []


Overfull \hbox (4.92744pt too wide) detected at line 4283
 []\OT1/lmss/m/n/10.95 1143
 []


Overfull \hbox (1.58195pt too wide) detected at line 4284
\OT1/lmss/m/n/10.95 V.158 
 []


Overfull \hbox (4.92744pt too wide) detected at line 4284
 []\OT1/lmss/m/n/10.95 1143
 []


Overfull \hbox (1.58195pt too wide) detected at line 4285
\OT1/lmss/m/n/10.95 V.159 
 []


Overfull \hbox (4.92744pt too wide) detected at line 4285
 []\OT1/lmss/m/n/10.95 1143
 []


Overfull \hbox (1.58195pt too wide) detected at line 4286
\OT1/lmss/m/n/10.95 V.160 
 []


Overfull \hbox (4.92744pt too wide) detected at line 4286
 []\OT1/lmss/m/n/10.95 1143
 []


Overfull \hbox (1.58195pt too wide) detected at line 4287
\OT1/lmss/m/n/10.95 V.161 
 []


Overfull \hbox (4.92744pt too wide) detected at line 4287
 []\OT1/lmss/m/n/10.95 1143
 []


Overfull \hbox (0.24365pt too wide) detected at line 4288
\OT1/lmss/m/n/10.95 V.161.1 
 []


Overfull \hbox (4.92744pt too wide) detected at line 4288
 []\OT1/lmss/m/n/10.95 1143
 []


Overfull \hbox (0.24365pt too wide) detected at line 4289
\OT1/lmss/m/n/10.95 V.161.2 
 []


Overfull \hbox (4.92744pt too wide) detected at line 4289
 []\OT1/lmss/m/n/10.95 1144
 []


Overfull \hbox (1.58195pt too wide) detected at line 4290
\OT1/lmss/m/n/10.95 V.162 
 []


Overfull \hbox (4.92744pt too wide) detected at line 4290
 []\OT1/lmss/m/n/10.95 1144
 []


Overfull \hbox (0.24365pt too wide) detected at line 4291
\OT1/lmss/m/n/10.95 V.162.1 
 []


Overfull \hbox (4.92744pt too wide) detected at line 4291
 []\OT1/lmss/m/n/10.95 1144
 []


Overfull \hbox (1.58195pt too wide) detected at line 4292
\OT1/lmss/m/n/10.95 V.163 
 []


Overfull \hbox (4.92744pt too wide) detected at line 4292
 []\OT1/lmss/m/n/10.95 1144
 []


Overfull \hbox (0.24365pt too wide) detected at line 4293
\OT1/lmss/m/n/10.95 V.163.1 
 []


Overfull \hbox (4.92744pt too wide) detected at line 4293
 []\OT1/lmss/m/n/10.95 1144
 []


Overfull \hbox (1.58195pt too wide) detected at line 4294
\OT1/lmss/m/n/10.95 V.164 
 []


Overfull \hbox (4.92744pt too wide) detected at line 4294
 []\OT1/lmss/m/n/10.95 1144
 []


Overfull \hbox (0.24365pt too wide) detected at line 4295
\OT1/lmss/m/n/10.95 V.164.1 
 []


Overfull \hbox (4.92744pt too wide) detected at line 4295
 []\OT1/lmss/m/n/10.95 1145
 []


Overfull \hbox (1.58195pt too wide) detected at line 4296
\OT1/lmss/m/n/10.95 V.165 
 []


Overfull \hbox (4.92744pt too wide) detected at line 4296
 []\OT1/lmss/m/n/10.95 1145
 []


Overfull \hbox (0.24365pt too wide) detected at line 4297
\OT1/lmss/m/n/10.95 V.165.1 
 []


Overfull \hbox (4.92744pt too wide) detected at line 4297
 []\OT1/lmss/m/n/10.95 1145
 []


Overfull \hbox (0.24365pt too wide) detected at line 4298
\OT1/lmss/m/n/10.95 V.165.2 
 []


Overfull \hbox (4.92744pt too wide) detected at line 4298
 []\OT1/lmss/m/n/10.95 1145
 []


Overfull \hbox (1.58195pt too wide) detected at line 4299
\OT1/lmss/m/n/10.95 V.166 
 []


Overfull \hbox (4.92744pt too wide) detected at line 4299
 []\OT1/lmss/m/n/10.95 1146
 []


Overfull \hbox (0.24365pt too wide) detected at line 4300
\OT1/lmss/m/n/10.95 V.166.1 
 []


Overfull \hbox (4.92744pt too wide) detected at line 4300
 []\OT1/lmss/m/n/10.95 1146
 []


Overfull \hbox (0.24365pt too wide) detected at line 4301
\OT1/lmss/m/n/10.95 V.166.2 
 []


Overfull \hbox (4.92744pt too wide) detected at line 4301
 []\OT1/lmss/m/n/10.95 1146
 []


Overfull \hbox (1.58195pt too wide) detected at line 4302
\OT1/lmss/m/n/10.95 V.167 
 []


Overfull \hbox (4.92744pt too wide) detected at line 4302
 []\OT1/lmss/m/n/10.95 1146
 []


Overfull \hbox (0.24365pt too wide) detected at line 4303
\OT1/lmss/m/n/10.95 V.167.1 
 []


Overfull \hbox (4.92744pt too wide) detected at line 4303
 []\OT1/lmss/m/n/10.95 1146
 []


Overfull \hbox (1.58195pt too wide) detected at line 4304
\OT1/lmss/m/n/10.95 V.168 
 []


Overfull \hbox (4.92744pt too wide) detected at line 4304
 []\OT1/lmss/m/n/10.95 1147
 []


Overfull \hbox (0.24365pt too wide) detected at line 4305
\OT1/lmss/m/n/10.95 V.168.1 
 []


Overfull \hbox (4.92744pt too wide) detected at line 4305
 []\OT1/lmss/m/n/10.95 1147
 []


Overfull \hbox (0.24365pt too wide) detected at line 4306
\OT1/lmss/m/n/10.95 V.168.2 
 []


Overfull \hbox (4.92744pt too wide) detected at line 4306
 []\OT1/lmss/m/n/10.95 1147
 []


Overfull \hbox (1.58195pt too wide) detected at line 4307
\OT1/lmss/m/n/10.95 V.169 
 []


Overfull \hbox (4.92744pt too wide) detected at line 4307
 []\OT1/lmss/m/n/10.95 1147
 []


Overfull \hbox (0.24365pt too wide) detected at line 4308
\OT1/lmss/m/n/10.95 V.169.1 
 []


Overfull \hbox (4.92744pt too wide) detected at line 4308
 []\OT1/lmss/m/n/10.95 1147
 []


Overfull \hbox (0.24365pt too wide) detected at line 4309
\OT1/lmss/m/n/10.95 V.169.2 
 []


Overfull \hbox (4.92744pt too wide) detected at line 4309
 []\OT1/lmss/m/n/10.95 1147
 []


Overfull \hbox (1.58195pt too wide) detected at line 4310
\OT1/lmss/m/n/10.95 V.170 
 []


Overfull \hbox (4.92744pt too wide) detected at line 4310
 []\OT1/lmss/m/n/10.95 1147
 []


Overfull \hbox (1.58195pt too wide) detected at line 4311
\OT1/lmss/m/n/10.95 V.171 
 []


Overfull \hbox (4.92744pt too wide) detected at line 4311
 []\OT1/lmss/m/n/10.95 1148
 []




Package fancyhdr Warning: \headheight is too small (12.0pt): 
(fancyhdr)                Make it at least 13.59999pt, for example:
(fancyhdr)                \setlength{\headheight}{13.59999pt}.
(fancyhdr)                You might also make \topmargin smaller:
(fancyhdr)                \addtolength{\topmargin}{-1.59999pt}.

[109]
Overfull \hbox (1.58195pt too wide) detected at line 4312
\OT1/lmss/m/n/10.95 V.172 
 []


Overfull \hbox (4.92744pt too wide) detected at line 4312
 []\OT1/lmss/m/n/10.95 1148
 []


Overfull \hbox (1.58195pt too wide) detected at line 4313
\OT1/lmss/m/n/10.95 V.173 
 []


Overfull \hbox (4.92744pt too wide) detected at line 4313
 []\OT1/lmss/m/n/10.95 1148
 []


Overfull \hbox (0.24365pt too wide) detected at line 4314
\OT1/lmss/m/n/10.95 V.173.1 
 []


Overfull \hbox (4.92744pt too wide) detected at line 4314
 []\OT1/lmss/m/n/10.95 1148
 []


Overfull \hbox (0.24365pt too wide) detected at line 4315
\OT1/lmss/m/n/10.95 V.173.2 
 []


Overfull \hbox (4.92744pt too wide) detected at line 4315
 []\OT1/lmss/m/n/10.95 1148
 []


Overfull \hbox (1.58195pt too wide) detected at line 4316
\OT1/lmss/m/n/10.95 V.174 
 []


Overfull \hbox (4.92744pt too wide) detected at line 4316
 []\OT1/lmss/m/n/10.95 1148
 []


Overfull \hbox (0.24365pt too wide) detected at line 4317
\OT1/lmss/m/n/10.95 V.174.1 
 []


Overfull \hbox (4.92744pt too wide) detected at line 4317
 []\OT1/lmss/m/n/10.95 1148
 []


Overfull \hbox (0.24365pt too wide) detected at line 4318
\OT1/lmss/m/n/10.95 V.174.2 
 []


Overfull \hbox (4.92744pt too wide) detected at line 4318
 []\OT1/lmss/m/n/10.95 1149
 []


Overfull \hbox (0.24365pt too wide) detected at line 4319
\OT1/lmss/m/n/10.95 V.174.3 
 []


Overfull \hbox (4.92744pt too wide) detected at line 4319
 []\OT1/lmss/m/n/10.95 1149
 []


Overfull \hbox (1.58195pt too wide) detected at line 4320
\OT1/lmss/m/n/10.95 V.175 
 []


Overfull \hbox (4.92744pt too wide) detected at line 4320
 []\OT1/lmss/m/n/10.95 1149
 []


Overfull \hbox (1.58195pt too wide) detected at line 4321
\OT1/lmss/m/n/10.95 V.176 
 []


Overfull \hbox (4.92744pt too wide) detected at line 4321
 []\OT1/lmss/m/n/10.95 1149
 []


Overfull \hbox (0.24365pt too wide) detected at line 4322
\OT1/lmss/m/n/10.95 V.176.1 
 []


Overfull \hbox (4.92744pt too wide) detected at line 4322
 []\OT1/lmss/m/n/10.95 1149
 []


Overfull \hbox (0.24365pt too wide) detected at line 4323
\OT1/lmss/m/n/10.95 V.176.2 
 []


Overfull \hbox (4.92744pt too wide) detected at line 4323
 []\OT1/lmss/m/n/10.95 1149
 []


Overfull \hbox (1.58195pt too wide) detected at line 4324
\OT1/lmss/m/n/10.95 V.177 
 []


Overfull \hbox (4.92744pt too wide) detected at line 4324
 []\OT1/lmss/m/n/10.95 1150
 []


Overfull \hbox (1.58195pt too wide) detected at line 4325
\OT1/lmss/m/n/10.95 V.178 
 []


Overfull \hbox (4.92744pt too wide) detected at line 4325
 []\OT1/lmss/m/n/10.95 1150
 []


Overfull \hbox (1.58195pt too wide) detected at line 4326
\OT1/lmss/m/n/10.95 V.179 
 []


Overfull \hbox (4.92744pt too wide) detected at line 4326
 []\OT1/lmss/m/n/10.95 1150
 []


Overfull \hbox (1.58195pt too wide) detected at line 4327
\OT1/lmss/m/n/10.95 V.180 
 []


Overfull \hbox (4.92744pt too wide) detected at line 4327
 []\OT1/lmss/m/n/10.95 1150
 []


Overfull \hbox (0.24365pt too wide) detected at line 4328
\OT1/lmss/m/n/10.95 V.180.1 
 []


Overfull \hbox (4.92744pt too wide) detected at line 4328
 []\OT1/lmss/m/n/10.95 1150
 []


Overfull \hbox (1.58195pt too wide) detected at line 4329
\OT1/lmss/m/n/10.95 V.181 
 []


Overfull \hbox (4.92744pt too wide) detected at line 4329
 []\OT1/lmss/m/n/10.95 1150
 []


Overfull \hbox (0.24365pt too wide) detected at line 4330
\OT1/lmss/m/n/10.95 V.181.1 
 []


Overfull \hbox (4.92744pt too wide) detected at line 4330
 []\OT1/lmss/m/n/10.95 1150
 []


Overfull \hbox (0.24365pt too wide) detected at line 4331
\OT1/lmss/m/n/10.95 V.181.2 
 []


Overfull \hbox (4.92744pt too wide) detected at line 4331
 []\OT1/lmss/m/n/10.95 1151
 []


Overfull \hbox (1.58195pt too wide) detected at line 4332
\OT1/lmss/m/n/10.95 V.182 
 []


Overfull \hbox (4.92744pt too wide) detected at line 4332
 []\OT1/lmss/m/n/10.95 1151
 []


Overfull \hbox (0.24365pt too wide) detected at line 4333
\OT1/lmss/m/n/10.95 V.182.1 
 []


Overfull \hbox (4.92744pt too wide) detected at line 4333
 []\OT1/lmss/m/n/10.95 1151
 []


Overfull \hbox (1.58195pt too wide) detected at line 4334
\OT1/lmss/m/n/10.95 V.183 
 []


Overfull \hbox (4.92744pt too wide) detected at line 4334
 []\OT1/lmss/m/n/10.95 1151
 []


Overfull \hbox (0.24365pt too wide) detected at line 4335
\OT1/lmss/m/n/10.95 V.183.1 
 []


Overfull \hbox (4.92744pt too wide) detected at line 4335
 []\OT1/lmss/m/n/10.95 1151
 []


Overfull \hbox (1.58195pt too wide) detected at line 4336
\OT1/lmss/m/n/10.95 V.184 
 []


Overfull \hbox (4.92744pt too wide) detected at line 4336
 []\OT1/lmss/m/n/10.95 1151
 []


Overfull \hbox (1.58195pt too wide) detected at line 4337
\OT1/lmss/m/n/10.95 V.185 
 []


Overfull \hbox (4.92744pt too wide) detected at line 4337
 []\OT1/lmss/m/n/10.95 1152
 []


Overfull \hbox (1.58195pt too wide) detected at line 4338
\OT1/lmss/m/n/10.95 V.186 
 []


Overfull \hbox (4.92744pt too wide) detected at line 4338
 []\OT1/lmss/m/n/10.95 1152
 []


Overfull \hbox (0.24365pt too wide) detected at line 4339
\OT1/lmss/m/n/10.95 V.186.1 
 []


Overfull \hbox (4.92744pt too wide) detected at line 4339
 []\OT1/lmss/m/n/10.95 1152
 []


Overfull \hbox (0.24365pt too wide) detected at line 4340
\OT1/lmss/m/n/10.95 V.186.2 
 []


Overfull \hbox (4.92744pt too wide) detected at line 4340
 []\OT1/lmss/m/n/10.95 1152
 []


Overfull \hbox (1.58195pt too wide) detected at line 4341
\OT1/lmss/m/n/10.95 V.187 
 []


Overfull \hbox (4.92744pt too wide) detected at line 4341
 []\OT1/lmss/m/n/10.95 1152
 []


Overfull \hbox (0.24365pt too wide) detected at line 4342
\OT1/lmss/m/n/10.95 V.187.1 
 []


Overfull \hbox (4.92744pt too wide) detected at line 4342
 []\OT1/lmss/m/n/10.95 1152
 []


Overfull \hbox (1.58195pt too wide) detected at line 4343
\OT1/lmss/m/n/10.95 V.188 
 []


Overfull \hbox (4.92744pt too wide) detected at line 4343
 []\OT1/lmss/m/n/10.95 1152
 []


Overfull \hbox (1.58195pt too wide) detected at line 4344
\OT1/lmss/m/n/10.95 V.189 
 []


Overfull \hbox (4.92744pt too wide) detected at line 4344
 []\OT1/lmss/m/n/10.95 1152
 []


Overfull \hbox (0.24365pt too wide) detected at line 4345
\OT1/lmss/m/n/10.95 V.189.1 
 []


Overfull \hbox (4.92744pt too wide) detected at line 4345
 []\OT1/lmss/m/n/10.95 1152
 []


Overfull \hbox (1.58195pt too wide) detected at line 4346
\OT1/lmss/m/n/10.95 V.190 
 []


Overfull \hbox (4.92744pt too wide) detected at line 4346
 []\OT1/lmss/m/n/10.95 1153
 []


Overfull \hbox (0.24365pt too wide) detected at line 4347
\OT1/lmss/m/n/10.95 V.190.1 
 []


Overfull \hbox (4.92744pt too wide) detected at line 4347
 []\OT1/lmss/m/n/10.95 1153
 []


Overfull \hbox (1.58195pt too wide) detected at line 4348
\OT1/lmss/m/n/10.95 V.191 
 []


Overfull \hbox (4.92744pt too wide) detected at line 4348
 []\OT1/lmss/m/n/10.95 1153
 []




Package fancyhdr Warning: \headheight is too small (12.0pt): 
(fancyhdr)                Make it at least 13.59999pt, for example:
(fancyhdr)                \setlength{\headheight}{13.59999pt}.
(fancyhdr)                You might also make \topmargin smaller:
(fancyhdr)                \addtolength{\topmargin}{-1.59999pt}.

[110]
Overfull \hbox (1.58195pt too wide) detected at line 4349
\OT1/lmss/m/n/10.95 V.192 
 []


Overfull \hbox (4.92744pt too wide) detected at line 4349
 []\OT1/lmss/m/n/10.95 1153
 []


Overfull \hbox (0.24365pt too wide) detected at line 4350
\OT1/lmss/m/n/10.95 V.192.1 
 []


Overfull \hbox (4.92744pt too wide) detected at line 4350
 []\OT1/lmss/m/n/10.95 1153
 []


Overfull \hbox (0.24365pt too wide) detected at line 4351
\OT1/lmss/m/n/10.95 V.192.2 
 []


Overfull \hbox (4.92744pt too wide) detected at line 4351
 []\OT1/lmss/m/n/10.95 1154
 []


Overfull \hbox (1.58195pt too wide) detected at line 4352
\OT1/lmss/m/n/10.95 V.193 
 []


Overfull \hbox (4.92744pt too wide) detected at line 4352
 []\OT1/lmss/m/n/10.95 1154
 []


Overfull \hbox (1.58195pt too wide) detected at line 4353
\OT1/lmss/m/n/10.95 V.194 
 []


Overfull \hbox (4.92744pt too wide) detected at line 4353
 []\OT1/lmss/m/n/10.95 1154
 []


Overfull \hbox (0.24365pt too wide) detected at line 4354
\OT1/lmss/m/n/10.95 V.194.1 
 []


Overfull \hbox (4.92744pt too wide) detected at line 4354
 []\OT1/lmss/m/n/10.95 1154
 []


Overfull \hbox (1.58195pt too wide) detected at line 4355
\OT1/lmss/m/n/10.95 V.195 
 []


Overfull \hbox (4.92744pt too wide) detected at line 4355
 []\OT1/lmss/m/n/10.95 1154
 []


Overfull \hbox (0.24365pt too wide) detected at line 4356
\OT1/lmss/m/n/10.95 V.195.1 
 []


Overfull \hbox (4.92744pt too wide) detected at line 4356
 []\OT1/lmss/m/n/10.95 1154
 []


Overfull \hbox (0.24365pt too wide) detected at line 4357
\OT1/lmss/m/n/10.95 V.195.2 
 []


Overfull \hbox (4.92744pt too wide) detected at line 4357
 []\OT1/lmss/m/n/10.95 1154
 []


Overfull \hbox (1.58195pt too wide) detected at line 4358
\OT1/lmss/m/n/10.95 V.196 
 []


Overfull \hbox (4.92744pt too wide) detected at line 4358
 []\OT1/lmss/m/n/10.95 1154
 []


Overfull \hbox (1.58195pt too wide) detected at line 4359
\OT1/lmss/m/n/10.95 V.197 
 []


Overfull \hbox (4.92744pt too wide) detected at line 4359
 []\OT1/lmss/m/n/10.95 1155
 []


Overfull \hbox (1.58195pt too wide) detected at line 4360
\OT1/lmss/m/n/10.95 V.198 
 []


Overfull \hbox (4.92744pt too wide) detected at line 4360
 []\OT1/lmss/m/n/10.95 1155
 []


Overfull \hbox (1.58195pt too wide) detected at line 4361
\OT1/lmss/m/n/10.95 V.199 
 []


Overfull \hbox (4.92744pt too wide) detected at line 4361
 []\OT1/lmss/m/n/10.95 1155
 []


Overfull \hbox (0.24365pt too wide) detected at line 4362
\OT1/lmss/m/n/10.95 V.199.1 
 []


Overfull \hbox (4.92744pt too wide) detected at line 4362
 []\OT1/lmss/m/n/10.95 1155
 []


Overfull \hbox (1.58195pt too wide) detected at line 4363
\OT1/lmss/m/n/10.95 V.200 
 []


Overfull \hbox (4.92744pt too wide) detected at line 4363
 []\OT1/lmss/m/n/10.95 1155
 []


Overfull \hbox (0.24365pt too wide) detected at line 4364
\OT1/lmss/m/n/10.95 V.200.1 
 []


Overfull \hbox (4.92744pt too wide) detected at line 4364
 []\OT1/lmss/m/n/10.95 1155
 []


Overfull \hbox (1.58195pt too wide) detected at line 4365
\OT1/lmss/m/n/10.95 V.201 
 []


Overfull \hbox (4.92744pt too wide) detected at line 4365
 []\OT1/lmss/m/n/10.95 1156
 []


Overfull \hbox (1.58195pt too wide) detected at line 4366
\OT1/lmss/m/n/10.95 V.202 
 []


Overfull \hbox (4.92744pt too wide) detected at line 4366
 []\OT1/lmss/m/n/10.95 1156
 []


Overfull \hbox (0.24365pt too wide) detected at line 4367
\OT1/lmss/m/n/10.95 V.202.1 
 []


Overfull \hbox (4.92744pt too wide) detected at line 4367
 []\OT1/lmss/m/n/10.95 1156
 []


Overfull \hbox (1.58195pt too wide) detected at line 4368
\OT1/lmss/m/n/10.95 V.203 
 []


Overfull \hbox (4.92744pt too wide) detected at line 4368
 []\OT1/lmss/m/n/10.95 1156
 []


Overfull \hbox (1.58195pt too wide) detected at line 4369
\OT1/lmss/m/n/10.95 V.204 
 []


Overfull \hbox (4.92744pt too wide) detected at line 4369
 []\OT1/lmss/m/n/10.95 1156
 []


Overfull \hbox (1.58195pt too wide) detected at line 4370
\OT1/lmss/m/n/10.95 V.205 
 []


Overfull \hbox (4.92744pt too wide) detected at line 4370
 []\OT1/lmss/m/n/10.95 1156
 []


Overfull \hbox (0.24365pt too wide) detected at line 4371
\OT1/lmss/m/n/10.95 V.205.1 
 []


Overfull \hbox (4.92744pt too wide) detected at line 4371
 []\OT1/lmss/m/n/10.95 1156
 []


Overfull \hbox (1.58195pt too wide) detected at line 4372
\OT1/lmss/m/n/10.95 V.206 
 []


Overfull \hbox (4.92744pt too wide) detected at line 4372
 []\OT1/lmss/m/n/10.95 1156
 []


Overfull \hbox (1.58195pt too wide) detected at line 4373
\OT1/lmss/m/n/10.95 V.207 
 []


Overfull \hbox (4.92744pt too wide) detected at line 4373
 []\OT1/lmss/m/n/10.95 1157
 []


Overfull \hbox (1.58195pt too wide) detected at line 4374
\OT1/lmss/m/n/10.95 V.208 
 []


Overfull \hbox (4.92744pt too wide) detected at line 4374
 []\OT1/lmss/m/n/10.95 1157
 []


Overfull \hbox (0.24365pt too wide) detected at line 4375
\OT1/lmss/m/n/10.95 V.208.1 
 []


Overfull \hbox (4.92744pt too wide) detected at line 4375
 []\OT1/lmss/m/n/10.95 1157
 []


Overfull \hbox (1.58195pt too wide) detected at line 4376
\OT1/lmss/m/n/10.95 V.209 
 []


Overfull \hbox (4.92744pt too wide) detected at line 4376
 []\OT1/lmss/m/n/10.95 1157
 []


Overfull \hbox (0.24365pt too wide) detected at line 4377
\OT1/lmss/m/n/10.95 V.209.1 
 []


Overfull \hbox (4.92744pt too wide) detected at line 4377
 []\OT1/lmss/m/n/10.95 1157
 []


Overfull \hbox (1.58195pt too wide) detected at line 4378
\OT1/lmss/m/n/10.95 V.210 
 []


Overfull \hbox (4.92744pt too wide) detected at line 4378
 []\OT1/lmss/m/n/10.95 1157
 []


Overfull \hbox (1.58195pt too wide) detected at line 4379
\OT1/lmss/m/n/10.95 V.211 
 []


Overfull \hbox (4.92744pt too wide) detected at line 4379
 []\OT1/lmss/m/n/10.95 1157
 []


Overfull \hbox (1.58195pt too wide) detected at line 4380
\OT1/lmss/m/n/10.95 V.212 
 []


Overfull \hbox (4.92744pt too wide) detected at line 4380
 []\OT1/lmss/m/n/10.95 1157
 []


Overfull \hbox (0.24365pt too wide) detected at line 4381
\OT1/lmss/m/n/10.95 V.212.1 
 []


Overfull \hbox (4.92744pt too wide) detected at line 4381
 []\OT1/lmss/m/n/10.95 1157
 []


Overfull \hbox (1.58195pt too wide) detected at line 4382
\OT1/lmss/m/n/10.95 V.213 
 []


Overfull \hbox (4.92744pt too wide) detected at line 4382
 []\OT1/lmss/m/n/10.95 1158
 []


Overfull \hbox (1.58195pt too wide) detected at line 4383
\OT1/lmss/m/n/10.95 V.214 
 []


Overfull \hbox (4.92744pt too wide) detected at line 4383
 []\OT1/lmss/m/n/10.95 1158
 []


Overfull \hbox (1.58195pt too wide) detected at line 4384
\OT1/lmss/m/n/10.95 V.215 
 []


Overfull \hbox (4.92744pt too wide) detected at line 4384
 []\OT1/lmss/m/n/10.95 1158
 []


Overfull \hbox (1.58195pt too wide) detected at line 4385
\OT1/lmss/m/n/10.95 V.216 
 []


Overfull \hbox (4.92744pt too wide) detected at line 4385
 []\OT1/lmss/m/n/10.95 1158
 []




Package fancyhdr Warning: \headheight is too small (12.0pt): 
(fancyhdr)                Make it at least 13.59999pt, for example:
(fancyhdr)                \setlength{\headheight}{13.59999pt}.
(fancyhdr)                You might also make \topmargin smaller:
(fancyhdr)                \addtolength{\topmargin}{-1.59999pt}.

[111]
Overfull \hbox (0.24365pt too wide) detected at line 4386
\OT1/lmss/m/n/10.95 V.216.1 
 []


Overfull \hbox (4.92744pt too wide) detected at line 4386
 []\OT1/lmss/m/n/10.95 1158
 []


Overfull \hbox (1.58195pt too wide) detected at line 4387
\OT1/lmss/m/n/10.95 V.217 
 []


Overfull \hbox (4.92744pt too wide) detected at line 4387
 []\OT1/lmss/m/n/10.95 1158
 []


Overfull \hbox (0.24365pt too wide) detected at line 4388
\OT1/lmss/m/n/10.95 V.217.1 
 []


Overfull \hbox (4.92744pt too wide) detected at line 4388
 []\OT1/lmss/m/n/10.95 1158
 []


Overfull \hbox (1.58195pt too wide) detected at line 4389
\OT1/lmss/m/n/10.95 V.218 
 []


Overfull \hbox (4.92744pt too wide) detected at line 4389
 []\OT1/lmss/m/n/10.95 1159
 []


Overfull \hbox (1.58195pt too wide) detected at line 4390
\OT1/lmss/m/n/10.95 V.219 
 []


Overfull \hbox (4.92744pt too wide) detected at line 4390
 []\OT1/lmss/m/n/10.95 1159
 []


Overfull \hbox (0.24365pt too wide) detected at line 4391
\OT1/lmss/m/n/10.95 V.219.1 
 []


Overfull \hbox (4.92744pt too wide) detected at line 4391
 []\OT1/lmss/m/n/10.95 1159
 []


Overfull \hbox (1.58195pt too wide) detected at line 4392
\OT1/lmss/m/n/10.95 V.220 
 []


Overfull \hbox (4.92744pt too wide) detected at line 4392
 []\OT1/lmss/m/n/10.95 1159
 []


Overfull \hbox (1.58195pt too wide) detected at line 4393
\OT1/lmss/m/n/10.95 V.221 
 []


Overfull \hbox (4.92744pt too wide) detected at line 4393
 []\OT1/lmss/m/n/10.95 1159
 []


Overfull \hbox (1.58195pt too wide) detected at line 4394
\OT1/lmss/m/n/10.95 V.222 
 []


Overfull \hbox (4.92744pt too wide) detected at line 4394
 []\OT1/lmss/m/n/10.95 1159
 []


Overfull \hbox (1.58195pt too wide) detected at line 4395
\OT1/lmss/m/n/10.95 V.223 
 []


Overfull \hbox (4.92744pt too wide) detected at line 4395
 []\OT1/lmss/m/n/10.95 1159
 []


Overfull \hbox (0.24365pt too wide) detected at line 4396
\OT1/lmss/m/n/10.95 V.223.1 
 []


Overfull \hbox (4.92744pt too wide) detected at line 4396
 []\OT1/lmss/m/n/10.95 1159
 []


Overfull \hbox (1.58195pt too wide) detected at line 4397
\OT1/lmss/m/n/10.95 V.224 
 []


Overfull \hbox (4.92744pt too wide) detected at line 4397
 []\OT1/lmss/m/n/10.95 1160
 []


Overfull \hbox (0.24365pt too wide) detected at line 4398
\OT1/lmss/m/n/10.95 V.224.1 
 []


Overfull \hbox (4.92744pt too wide) detected at line 4398
 []\OT1/lmss/m/n/10.95 1160
 []


Overfull \hbox (1.58195pt too wide) detected at line 4399
\OT1/lmss/m/n/10.95 V.225 
 []


Overfull \hbox (4.92744pt too wide) detected at line 4399
 []\OT1/lmss/m/n/10.95 1160
 []


Overfull \hbox (1.58195pt too wide) detected at line 4400
\OT1/lmss/m/n/10.95 V.226 
 []


Overfull \hbox (4.92744pt too wide) detected at line 4400
 []\OT1/lmss/m/n/10.95 1160
 []


Overfull \hbox (1.58195pt too wide) detected at line 4401
\OT1/lmss/m/n/10.95 V.227 
 []


Overfull \hbox (4.92744pt too wide) detected at line 4401
 []\OT1/lmss/m/n/10.95 1160
 []


Overfull \hbox (1.58195pt too wide) detected at line 4402
\OT1/lmss/m/n/10.95 V.228 
 []


Overfull \hbox (4.92744pt too wide) detected at line 4402
 []\OT1/lmss/m/n/10.95 1160
 []


Overfull \hbox (0.24365pt too wide) detected at line 4403
\OT1/lmss/m/n/10.95 V.228.1 
 []


Overfull \hbox (4.92744pt too wide) detected at line 4403
 []\OT1/lmss/m/n/10.95 1160
 []


Overfull \hbox (1.58195pt too wide) detected at line 4404
\OT1/lmss/m/n/10.95 V.229 
 []


Overfull \hbox (4.92744pt too wide) detected at line 4404
 []\OT1/lmss/m/n/10.95 1160
 []


Overfull \hbox (1.58195pt too wide) detected at line 4405
\OT1/lmss/m/n/10.95 V.230 
 []


Overfull \hbox (4.92744pt too wide) detected at line 4405
 []\OT1/lmss/m/n/10.95 1161
 []


Overfull \hbox (1.58195pt too wide) detected at line 4406
\OT1/lmss/m/n/10.95 V.231 
 []


Overfull \hbox (4.92744pt too wide) detected at line 4406
 []\OT1/lmss/m/n/10.95 1161
 []


Overfull \hbox (1.58195pt too wide) detected at line 4407
\OT1/lmss/m/n/10.95 V.232 
 []


Overfull \hbox (4.92744pt too wide) detected at line 4407
 []\OT1/lmss/m/n/10.95 1161
 []


Overfull \hbox (1.58195pt too wide) detected at line 4408
\OT1/lmss/m/n/10.95 V.233 
 []


Overfull \hbox (4.92744pt too wide) detected at line 4408
 []\OT1/lmss/m/n/10.95 1161
 []


Overfull \hbox (0.24365pt too wide) detected at line 4409
\OT1/lmss/m/n/10.95 V.233.1 
 []


Overfull \hbox (4.92744pt too wide) detected at line 4409
 []\OT1/lmss/m/n/10.95 1161
 []


Overfull \hbox (0.24365pt too wide) detected at line 4410
\OT1/lmss/m/n/10.95 V.233.2 
 []


Overfull \hbox (4.92744pt too wide) detected at line 4410
 []\OT1/lmss/m/n/10.95 1162
 []


Overfull \hbox (0.24365pt too wide) detected at line 4411
\OT1/lmss/m/n/10.95 V.233.3 
 []


Overfull \hbox (4.92744pt too wide) detected at line 4411
 []\OT1/lmss/m/n/10.95 1162
 []


Overfull \hbox (1.58195pt too wide) detected at line 4412
\OT1/lmss/m/n/10.95 V.234 
 []


Overfull \hbox (4.92744pt too wide) detected at line 4412
 []\OT1/lmss/m/n/10.95 1162
 []


Overfull \hbox (1.58195pt too wide) detected at line 4413
\OT1/lmss/m/n/10.95 V.235 
 []


Overfull \hbox (4.92744pt too wide) detected at line 4413
 []\OT1/lmss/m/n/10.95 1162
 []


Overfull \hbox (0.24365pt too wide) detected at line 4414
\OT1/lmss/m/n/10.95 V.235.1 
 []


Overfull \hbox (4.92744pt too wide) detected at line 4414
 []\OT1/lmss/m/n/10.95 1162
 []


Overfull \hbox (0.24365pt too wide) detected at line 4415
\OT1/lmss/m/n/10.95 V.235.2 
 []


Overfull \hbox (4.92744pt too wide) detected at line 4415
 []\OT1/lmss/m/n/10.95 1163
 []


Overfull \hbox (1.58195pt too wide) detected at line 4416
\OT1/lmss/m/n/10.95 V.236 
 []


Overfull \hbox (4.92744pt too wide) detected at line 4416
 []\OT1/lmss/m/n/10.95 1163
 []


Overfull \hbox (1.58195pt too wide) detected at line 4417
\OT1/lmss/m/n/10.95 V.237 
 []


Overfull \hbox (4.92744pt too wide) detected at line 4417
 []\OT1/lmss/m/n/10.95 1163
 []


Overfull \hbox (0.24365pt too wide) detected at line 4418
\OT1/lmss/m/n/10.95 V.237.1 
 []


Overfull \hbox (4.92744pt too wide) detected at line 4418
 []\OT1/lmss/m/n/10.95 1163
 []


Overfull \hbox (1.58195pt too wide) detected at line 4419
\OT1/lmss/m/n/10.95 V.238 
 []


Overfull \hbox (4.92744pt too wide) detected at line 4419
 []\OT1/lmss/m/n/10.95 1163
 []


Overfull \hbox (1.58195pt too wide) detected at line 4420
\OT1/lmss/m/n/10.95 V.239 
 []


Overfull \hbox (4.92744pt too wide) detected at line 4420
 []\OT1/lmss/m/n/10.95 1163
 []


Overfull \hbox (1.58195pt too wide) detected at line 4421
\OT1/lmss/m/n/10.95 V.240 
 []


Overfull \hbox (4.92744pt too wide) detected at line 4421
 []\OT1/lmss/m/n/10.95 1163
 []


Overfull \hbox (1.58195pt too wide) detected at line 4422
\OT1/lmss/m/n/10.95 V.241 
 []


Overfull \hbox (4.92744pt too wide) detected at line 4422
 []\OT1/lmss/m/n/10.95 1164
 []




Package fancyhdr Warning: \headheight is too small (12.0pt): 
(fancyhdr)                Make it at least 13.59999pt, for example:
(fancyhdr)                \setlength{\headheight}{13.59999pt}.
(fancyhdr)                You might also make \topmargin smaller:
(fancyhdr)                \addtolength{\topmargin}{-1.59999pt}.

[112]
Overfull \hbox (0.24365pt too wide) detected at line 4423
\OT1/lmss/m/n/10.95 V.241.1 
 []


Overfull \hbox (4.92744pt too wide) detected at line 4423
 []\OT1/lmss/m/n/10.95 1164
 []


Overfull \hbox (1.58195pt too wide) detected at line 4424
\OT1/lmss/m/n/10.95 V.242 
 []


Overfull \hbox (4.92744pt too wide) detected at line 4424
 []\OT1/lmss/m/n/10.95 1164
 []


Overfull \hbox (1.58195pt too wide) detected at line 4425
\OT1/lmss/m/n/10.95 V.243 
 []


Overfull \hbox (4.92744pt too wide) detected at line 4425
 []\OT1/lmss/m/n/10.95 1164
 []


Overfull \hbox (0.24365pt too wide) detected at line 4426
\OT1/lmss/m/n/10.95 V.243.1 
 []


Overfull \hbox (4.92744pt too wide) detected at line 4426
 []\OT1/lmss/m/n/10.95 1164
 []


Overfull \hbox (1.58195pt too wide) detected at line 4427
\OT1/lmss/m/n/10.95 V.244 
 []


Overfull \hbox (4.92744pt too wide) detected at line 4427
 []\OT1/lmss/m/n/10.95 1164
 []


Overfull \hbox (1.58195pt too wide) detected at line 4428
\OT1/lmss/m/n/10.95 V.245 
 []


Overfull \hbox (4.92744pt too wide) detected at line 4428
 []\OT1/lmss/m/n/10.95 1164
 []


Overfull \hbox (1.58195pt too wide) detected at line 4429
\OT1/lmss/m/n/10.95 V.246 
 []


Overfull \hbox (4.92744pt too wide) detected at line 4429
 []\OT1/lmss/m/n/10.95 1165
 []


Overfull \hbox (4.92744pt too wide) detected at line 4430
 []\OT1/lmss/m/n/10.95 1165
 []


Overfull \hbox (4.92744pt too wide) detected at line 4431
 []\OT1/lmss/m/n/10.95 1165
 []


Overfull \hbox (4.92744pt too wide) detected at line 4432
 []\OT1/lmss/m/n/10.95 1165
 []


Overfull \hbox (4.92744pt too wide) detected at line 4433
 []\OT1/lmss/m/n/10.95 1166
 []


Overfull \hbox (4.92744pt too wide) detected at line 4434
 []\OT1/lmss/m/n/10.95 1166
 []


Overfull \hbox (4.92744pt too wide) detected at line 4435
 []\OT1/lmss/m/n/10.95 1166
 []


Overfull \hbox (4.92744pt too wide) detected at line 4436
 []\OT1/lmss/m/n/10.95 1166
 []


Overfull \hbox (4.92744pt too wide) detected at line 4437
 []\OT1/lmss/m/n/10.95 1166
 []


Overfull \hbox (4.92744pt too wide) detected at line 4438
 []\OT1/lmss/m/n/10.95 1166
 []


Overfull \hbox (4.92744pt too wide) detected at line 4440
 []\OT1/lmss/m/n/10.95 1168
 []


Overfull \hbox (4.92744pt too wide) detected at line 4441
 []\OT1/lmss/m/n/10.95 1168
 []


Overfull \hbox (4.92744pt too wide) detected at line 4442
 []\OT1/lmss/m/n/10.95 1169
 []


Overfull \hbox (4.92744pt too wide) detected at line 4443
 []\OT1/lmss/m/n/10.95 1169
 []


Overfull \hbox (4.92744pt too wide) detected at line 4444
 []\OT1/lmss/m/n/10.95 1169
 []


Overfull \hbox (4.92744pt too wide) detected at line 4446
 []\OT1/lmss/m/n/10.95 1169
 []


Overfull \hbox (4.92744pt too wide) detected at line 4447
 []\OT1/lmss/m/n/10.95 1170
 []


Overfull \hbox (4.92744pt too wide) detected at line 4448
 []\OT1/lmss/m/n/10.95 1170
 []


Overfull \hbox (4.92744pt too wide) detected at line 4451
 []\OT1/lmss/m/n/10.95 1170
 []


Overfull \hbox (4.92744pt too wide) detected at line 4452
 []\OT1/lmss/m/n/10.95 1171
 []


Overfull \hbox (4.92744pt too wide) detected at line 4453
 []\OT1/lmss/m/n/10.95 1171
 []


Overfull \hbox (4.92744pt too wide) detected at line 4454
 []\OT1/lmss/m/n/10.95 1171
 []


Overfull \hbox (4.92744pt too wide) detected at line 4455
 []\OT1/lmss/m/n/10.95 1171
 []


Overfull \hbox (4.92744pt too wide) detected at line 4456
 []\OT1/lmss/m/n/10.95 1171
 []


Overfull \hbox (4.92744pt too wide) detected at line 4457
 []\OT1/lmss/m/n/10.95 1172
 []


Overfull \hbox (4.92744pt too wide) detected at line 4458
 []\OT1/lmss/m/n/10.95 1172
 []


Overfull \hbox (4.92744pt too wide) detected at line 4459
 []\OT1/lmss/m/n/10.95 1172
 []


Overfull \hbox (4.92744pt too wide) detected at line 4460
 []\OT1/lmss/m/n/10.95 1172
 []


Overfull \hbox (4.92744pt too wide) detected at line 4463
 []\OT1/lmss/m/n/10.95 1173
 []


Overfull \hbox (4.92744pt too wide) detected at line 4466
 []\OT1/lmss/m/n/10.95 1173
 []




Package fancyhdr Warning: \headheight is too small (12.0pt): 
(fancyhdr)                Make it at least 13.59999pt, for example:
(fancyhdr)                \setlength{\headheight}{13.59999pt}.
(fancyhdr)                You might also make \topmargin smaller:
(fancyhdr)                \addtolength{\topmargin}{-1.59999pt}.

[113]
Overfull \hbox (4.92744pt too wide) detected at line 4469
 []\OT1/lmss/m/n/10.95 1173
 []


Overfull \hbox (4.92744pt too wide) detected at line 4470
 []\OT1/lmss/m/n/10.95 1173
 []


Overfull \hbox (4.92744pt too wide) detected at line 4471
 []\OT1/lmss/m/n/10.95 1174
 []


Overfull \hbox (4.92744pt too wide) detected at line 4472
 []\OT1/lmss/m/n/10.95 1174
 []


Overfull \hbox (4.92744pt too wide) detected at line 4473
 []\OT1/lmss/m/n/10.95 1175
 []


Overfull \hbox (4.92744pt too wide) detected at line 4474
 []\OT1/lmss/m/n/10.95 1175
 []


Overfull \hbox (4.92744pt too wide) detected at line 4475
 []\OT1/lmss/m/n/10.95 1175
 []


Overfull \hbox (4.92744pt too wide) detected at line 4476
 []\OT1/lmss/m/n/10.95 1175
 []


Overfull \hbox (4.92744pt too wide) detected at line 4477
 []\OT1/lmss/m/n/10.95 1176
 []


Overfull \hbox (4.92744pt too wide) detected at line 4480
 []\OT1/lmss/m/n/10.95 1176
 []


Overfull \hbox (4.92744pt too wide) detected at line 4481
 []\OT1/lmss/m/n/10.95 1176
 []


Overfull \hbox (4.92744pt too wide) detected at line 4482
 []\OT1/lmss/m/n/10.95 1177
 []


Overfull \hbox (4.92744pt too wide) detected at line 4483
 []\OT1/lmss/m/n/10.95 1177
 []


Overfull \hbox (4.92744pt too wide) detected at line 4484
 []\OT1/lmss/m/n/10.95 1177
 []


Overfull \hbox (4.92744pt too wide) detected at line 4485
 []\OT1/lmss/m/n/10.95 1177
 []


Overfull \hbox (4.92744pt too wide) detected at line 4486
 []\OT1/lmss/m/n/10.95 1178
 []


Overfull \hbox (4.92744pt too wide) detected at line 4487
 []\OT1/lmss/m/n/10.95 1178
 []


Overfull \hbox (4.92744pt too wide) detected at line 4488
 []\OT1/lmss/m/n/10.95 1178
 []


Overfull \hbox (4.92744pt too wide) detected at line 4489
 []\OT1/lmss/m/n/10.95 1178
 []


Overfull \hbox (4.92744pt too wide) detected at line 4490
 []\OT1/lmss/m/n/10.95 1178
 []


Overfull \hbox (4.92744pt too wide) detected at line 4491
 []\OT1/lmss/m/n/10.95 1178
 []


Overfull \hbox (4.92744pt too wide) detected at line 4492
 []\OT1/lmss/m/n/10.95 1179
 []


Overfull \hbox (4.92744pt too wide) detected at line 4494
 []\OT1/lmss/m/n/10.95 1181
 []


Overfull \hbox (4.92744pt too wide) detected at line 4495
 []\OT1/lmss/m/n/10.95 1181
 []


Overfull \hbox (4.92744pt too wide) detected at line 4496
 []\OT1/lmss/m/n/10.95 1181
 []


Overfull \hbox (4.92744pt too wide) detected at line 4497
 []\OT1/lmss/m/n/10.95 1181
 []


Overfull \hbox (4.92744pt too wide) detected at line 4498
 []\OT1/lmss/m/n/10.95 1182
 []


Overfull \hbox (4.92744pt too wide) detected at line 4499
 []\OT1/lmss/m/n/10.95 1182
 []


Overfull \hbox (4.92744pt too wide) detected at line 4500
 []\OT1/lmss/m/n/10.95 1182
 []


Overfull \hbox (4.92744pt too wide) detected at line 4501
 []\OT1/lmss/m/n/10.95 1182
 []


Overfull \hbox (4.92744pt too wide) detected at line 4502
 []\OT1/lmss/m/n/10.95 1182
 []


Overfull \hbox (4.92744pt too wide) detected at line 4503
 []\OT1/lmss/m/n/10.95 1182
 []


Overfull \hbox (4.92744pt too wide) detected at line 4504
 []\OT1/lmss/m/n/10.95 1183
 []


Overfull \hbox (4.92744pt too wide) detected at line 4505
 []\OT1/lmss/m/n/10.95 1183
 []


Overfull \hbox (4.92744pt too wide) detected at line 4506
 []\OT1/lmss/m/n/10.95 1183
 []


Overfull \hbox (4.92744pt too wide) detected at line 4507
 []\OT1/lmss/m/n/10.95 1183
 []




Package fancyhdr Warning: \headheight is too small (12.0pt): 
(fancyhdr)                Make it at least 13.59999pt, for example:
(fancyhdr)                \setlength{\headheight}{13.59999pt}.
(fancyhdr)                You might also make \topmargin smaller:
(fancyhdr)                \addtolength{\topmargin}{-1.59999pt}.

[114]
Overfull \hbox (4.92744pt too wide) detected at line 4508
 []\OT1/lmss/m/n/10.95 1183
 []


Overfull \hbox (4.92744pt too wide) detected at line 4510
 []\OT1/lmss/m/n/10.95 1183
 []


Overfull \hbox (4.92744pt too wide) detected at line 4511
 []\OT1/lmss/m/n/10.95 1183
 []


Overfull \hbox (4.92744pt too wide) detected at line 4513
 []\OT1/lmss/m/n/10.95 1183
 []


Overfull \hbox (4.92744pt too wide) detected at line 4514
 []\OT1/lmss/m/n/10.95 1184
 []


Overfull \hbox (4.92744pt too wide) detected at line 4515
 []\OT1/lmss/m/n/10.95 1184
 []


Overfull \hbox (4.92744pt too wide) detected at line 4516
 []\OT1/lmss/m/n/10.95 1184
 []


Overfull \hbox (4.92744pt too wide) detected at line 4517
 []\OT1/lmss/m/n/10.95 1184
 []


Overfull \hbox (4.92744pt too wide) detected at line 4518
 []\OT1/lmss/m/n/10.95 1184
 []


Overfull \hbox (4.92744pt too wide) detected at line 4519
 []\OT1/lmss/m/n/10.95 1184
 []


Overfull \hbox (4.92744pt too wide) detected at line 4520
 []\OT1/lmss/m/n/10.95 1184
 []


Overfull \hbox (4.92744pt too wide) detected at line 4521
 []\OT1/lmss/m/n/10.95 1185
 []


Overfull \hbox (4.92744pt too wide) detected at line 4522
 []\OT1/lmss/m/n/10.95 1185
 []


Overfull \hbox (4.92744pt too wide) detected at line 4523
 []\OT1/lmss/m/n/10.95 1185
 []


Overfull \hbox (4.92744pt too wide) detected at line 4524
 []\OT1/lmss/m/n/10.95 1185
 []


Overfull \hbox (4.92744pt too wide) detected at line 4525
 []\OT1/lmss/m/n/10.95 1185
 []


Overfull \hbox (4.92744pt too wide) detected at line 4526
 []\OT1/lmss/m/n/10.95 1185
 []


Overfull \hbox (4.92744pt too wide) detected at line 4527
 []\OT1/lmss/m/n/10.95 1185
 []


Overfull \hbox (4.92744pt too wide) detected at line 4528
 []\OT1/lmss/m/n/10.95 1186
 []


Overfull \hbox (4.92744pt too wide) detected at line 4529
 []\OT1/lmss/m/n/10.95 1186
 []


Overfull \hbox (4.92744pt too wide) detected at line 4530
 []\OT1/lmss/m/n/10.95 1186
 []


Overfull \hbox (4.92744pt too wide) detected at line 4531
 []\OT1/lmss/m/n/10.95 1186
 []


Overfull \hbox (4.92744pt too wide) detected at line 4532
 []\OT1/lmss/m/n/10.95 1186
 []


Overfull \hbox (4.92744pt too wide) detected at line 4533
 []\OT1/lmss/m/n/10.95 1186
 []


Overfull \hbox (4.92744pt too wide) detected at line 4534
 []\OT1/lmss/m/n/10.95 1186
 []


Overfull \hbox (4.92744pt too wide) detected at line 4535
 []\OT1/lmss/m/n/10.95 1187
 []


Overfull \hbox (4.92744pt too wide) detected at line 4536
 []\OT1/lmss/m/n/10.95 1187
 []


Overfull \hbox (4.92744pt too wide) detected at line 4537
 []\OT1/lmss/m/n/10.95 1187
 []


Overfull \hbox (4.92744pt too wide) detected at line 4538
 []\OT1/lmss/m/n/10.95 1187
 []


Overfull \hbox (4.92744pt too wide) detected at line 4539
 []\OT1/lmss/m/n/10.95 1187
 []


Overfull \hbox (4.92744pt too wide) detected at line 4540
 []\OT1/lmss/m/n/10.95 1187
 []


Overfull \hbox (4.92744pt too wide) detected at line 4541
 []\OT1/lmss/m/n/10.95 1187
 []


Overfull \hbox (4.92744pt too wide) detected at line 4542
 []\OT1/lmss/m/n/10.95 1187
 []


Overfull \hbox (4.92744pt too wide) detected at line 4543
 []\OT1/lmss/m/n/10.95 1188
 []


Overfull \hbox (4.92744pt too wide) detected at line 4544
 []\OT1/lmss/m/n/10.95 1188
 []


Overfull \hbox (4.92744pt too wide) detected at line 4545
 []\OT1/lmss/m/n/10.95 1188
 []


Overfull \hbox (4.92744pt too wide) detected at line 4546
 []\OT1/lmss/m/n/10.95 1188
 []




Package fancyhdr Warning: \headheight is too small (12.0pt): 
(fancyhdr)                Make it at least 13.59999pt, for example:
(fancyhdr)                \setlength{\headheight}{13.59999pt}.
(fancyhdr)                You might also make \topmargin smaller:
(fancyhdr)                \addtolength{\topmargin}{-1.59999pt}.

[115]
Overfull \hbox (4.92744pt too wide) detected at line 4547
 []\OT1/lmss/m/n/10.95 1188
 []


Overfull \hbox (4.92744pt too wide) detected at line 4548
 []\OT1/lmss/m/n/10.95 1188
 []


Overfull \hbox (4.92744pt too wide) detected at line 4549
 []\OT1/lmss/m/n/10.95 1189
 []


Overfull \hbox (4.92744pt too wide) detected at line 4550
 []\OT1/lmss/m/n/10.95 1189
 []


Overfull \hbox (4.92744pt too wide) detected at line 4551
 []\OT1/lmss/m/n/10.95 1189
 []


Overfull \hbox (4.92744pt too wide) detected at line 4552
 []\OT1/lmss/m/n/10.95 1189
 []


Overfull \hbox (4.92744pt too wide) detected at line 4553
 []\OT1/lmss/m/n/10.95 1189
 []


Overfull \hbox (4.92744pt too wide) detected at line 4554
 []\OT1/lmss/m/n/10.95 1190
 []


Overfull \hbox (4.92744pt too wide) detected at line 4555
 []\OT1/lmss/m/n/10.95 1190
 []


Overfull \hbox (4.92744pt too wide) detected at line 4556
 []\OT1/lmss/m/n/10.95 1190
 []


Overfull \hbox (4.92744pt too wide) detected at line 4557
 []\OT1/lmss/m/n/10.95 1190
 []


Overfull \hbox (4.92744pt too wide) detected at line 4558
 []\OT1/lmss/m/n/10.95 1190
 []


Overfull \hbox (4.92744pt too wide) detected at line 4559
 []\OT1/lmss/m/n/10.95 1191
 []


Overfull \hbox (4.92744pt too wide) detected at line 4560
 []\OT1/lmss/m/n/10.95 1191
 []


Overfull \hbox (4.92744pt too wide) detected at line 4561
 []\OT1/lmss/m/n/10.95 1191
 []


Overfull \hbox (4.92744pt too wide) detected at line 4562
 []\OT1/lmss/m/n/10.95 1191
 []


Overfull \hbox (4.92744pt too wide) detected at line 4563
 []\OT1/lmss/m/n/10.95 1191
 []


Overfull \hbox (4.92744pt too wide) detected at line 4564
 []\OT1/lmss/m/n/10.95 1191
 []


Overfull \hbox (4.92744pt too wide) detected at line 4565
 []\OT1/lmss/m/n/10.95 1192
 []


Overfull \hbox (4.92744pt too wide) detected at line 4566
 []\OT1/lmss/m/n/10.95 1192
 []


Overfull \hbox (4.92744pt too wide) detected at line 4567
 []\OT1/lmss/m/n/10.95 1192
 []


Overfull \hbox (4.92744pt too wide) detected at line 4568
 []\OT1/lmss/m/n/10.95 1192
 []


Overfull \hbox (4.92744pt too wide) detected at line 4569
 []\OT1/lmss/m/n/10.95 1192
 []


Overfull \hbox (4.92744pt too wide) detected at line 4570
 []\OT1/lmss/m/n/10.95 1193
 []


Overfull \hbox (4.92744pt too wide) detected at line 4571
 []\OT1/lmss/m/n/10.95 1193
 []


Overfull \hbox (4.92744pt too wide) detected at line 4572
 []\OT1/lmss/m/n/10.95 1193
 []


Overfull \hbox (4.92744pt too wide) detected at line 4573
 []\OT1/lmss/m/n/10.95 1193
 []


Overfull \hbox (4.92744pt too wide) detected at line 4574
 []\OT1/lmss/m/n/10.95 1193
 []


Overfull \hbox (4.92744pt too wide) detected at line 4575
 []\OT1/lmss/m/n/10.95 1194
 []


Overfull \hbox (4.92744pt too wide) detected at line 4576
 []\OT1/lmss/m/n/10.95 1194
 []


Overfull \hbox (4.92744pt too wide) detected at line 4577
 []\OT1/lmss/m/n/10.95 1194
 []


Overfull \hbox (4.92744pt too wide) detected at line 4578
 []\OT1/lmss/m/n/10.95 1194
 []


Overfull \hbox (4.92744pt too wide) detected at line 4579
 []\OT1/lmss/m/n/10.95 1194
 []


Overfull \hbox (4.92744pt too wide) detected at line 4580
 []\OT1/lmss/m/n/10.95 1194
 []


Overfull \hbox (4.92744pt too wide) detected at line 4581
 []\OT1/lmss/m/n/10.95 1194
 []


Overfull \hbox (4.92744pt too wide) detected at line 4582
 []\OT1/lmss/m/n/10.95 1194
 []


Overfull \hbox (4.92744pt too wide) detected at line 4583
 []\OT1/lmss/m/n/10.95 1195
 []




Package fancyhdr Warning: \headheight is too small (12.0pt): 
(fancyhdr)                Make it at least 13.59999pt, for example:
(fancyhdr)                \setlength{\headheight}{13.59999pt}.
(fancyhdr)                You might also make \topmargin smaller:
(fancyhdr)                \addtolength{\topmargin}{-1.59999pt}.

[116]
Overfull \hbox (4.92744pt too wide) detected at line 4584
 []\OT1/lmss/m/n/10.95 1195
 []


Overfull \hbox (4.92744pt too wide) detected at line 4585
 []\OT1/lmss/m/n/10.95 1195
 []


Overfull \hbox (4.92744pt too wide) detected at line 4586
 []\OT1/lmss/m/n/10.95 1195
 []


Overfull \hbox (4.92744pt too wide) detected at line 4587
 []\OT1/lmss/m/n/10.95 1195
 []


Overfull \hbox (4.92744pt too wide) detected at line 4588
 []\OT1/lmss/m/n/10.95 1195
 []


Overfull \hbox (4.92744pt too wide) detected at line 4589
 []\OT1/lmss/m/n/10.95 1196
 []


Overfull \hbox (4.92744pt too wide) detected at line 4590
 []\OT1/lmss/m/n/10.95 1196
 []


Overfull \hbox (4.92744pt too wide) detected at line 4591
 []\OT1/lmss/m/n/10.95 1196
 []


Overfull \hbox (4.92744pt too wide) detected at line 4592
 []\OT1/lmss/m/n/10.95 1197
 []


Overfull \hbox (4.92744pt too wide) detected at line 4593
 []\OT1/lmss/m/n/10.95 1197
 []


Overfull \hbox (4.92744pt too wide) detected at line 4594
 []\OT1/lmss/m/n/10.95 1197
 []


Overfull \hbox (4.92744pt too wide) detected at line 4595
 []\OT1/lmss/m/n/10.95 1197
 []


Overfull \hbox (4.92744pt too wide) detected at line 4596
 []\OT1/lmss/m/n/10.95 1197
 []


Overfull \hbox (4.92744pt too wide) detected at line 4597
 []\OT1/lmss/m/n/10.95 1197
 []


Overfull \hbox (4.92744pt too wide) detected at line 4598
 []\OT1/lmss/m/n/10.95 1198
 []


Overfull \hbox (4.92744pt too wide) detected at line 4599
 []\OT1/lmss/m/n/10.95 1198
 []


Overfull \hbox (4.92744pt too wide) detected at line 4600
 []\OT1/lmss/m/n/10.95 1198
 []


Overfull \hbox (4.92744pt too wide) detected at line 4601
 []\OT1/lmss/m/n/10.95 1198
 []


Overfull \hbox (4.92744pt too wide) detected at line 4602
 []\OT1/lmss/m/n/10.95 1198
 []


Overfull \hbox (4.92744pt too wide) detected at line 4603
 []\OT1/lmss/m/n/10.95 1198
 []


Overfull \hbox (4.92744pt too wide) detected at line 4604
 []\OT1/lmss/m/n/10.95 1198
 []


Overfull \hbox (4.92744pt too wide) detected at line 4605
 []\OT1/lmss/m/n/10.95 1199
 []


Overfull \hbox (4.92744pt too wide) detected at line 4606
 []\OT1/lmss/m/n/10.95 1199
 []


Overfull \hbox (4.92744pt too wide) detected at line 4607
 []\OT1/lmss/m/n/10.95 1199
 []


Overfull \hbox (4.92744pt too wide) detected at line 4608
 []\OT1/lmss/m/n/10.95 1199
 []


Overfull \hbox (4.92744pt too wide) detected at line 4609
 []\OT1/lmss/m/n/10.95 1199
 []


Overfull \hbox (4.92744pt too wide) detected at line 4610
 []\OT1/lmss/m/n/10.95 1199
 []


Overfull \hbox (4.92744pt too wide) detected at line 4611
 []\OT1/lmss/m/n/10.95 1200
 []


Overfull \hbox (4.92744pt too wide) detected at line 4612
 []\OT1/lmss/m/n/10.95 1200
 []


Overfull \hbox (4.92744pt too wide) detected at line 4613
 []\OT1/lmss/m/n/10.95 1200
 []


Overfull \hbox (4.92744pt too wide) detected at line 4614
 []\OT1/lmss/m/n/10.95 1200
 []


Overfull \hbox (4.92744pt too wide) detected at line 4615
 []\OT1/lmss/m/n/10.95 1200
 []


Overfull \hbox (4.92744pt too wide) detected at line 4616
 []\OT1/lmss/m/n/10.95 1201
 []


Overfull \hbox (4.92744pt too wide) detected at line 4617
 []\OT1/lmss/m/n/10.95 1201
 []


Overfull \hbox (4.92744pt too wide) detected at line 4618
 []\OT1/lmss/m/n/10.95 1201
 []


Overfull \hbox (4.92744pt too wide) detected at line 4619
 []\OT1/lmss/m/n/10.95 1201
 []


Overfull \hbox (4.92744pt too wide) detected at line 4620
 []\OT1/lmss/m/n/10.95 1201
 []




Package fancyhdr Warning: \headheight is too small (12.0pt): 
(fancyhdr)                Make it at least 13.59999pt, for example:
(fancyhdr)                \setlength{\headheight}{13.59999pt}.
(fancyhdr)                You might also make \topmargin smaller:
(fancyhdr)                \addtolength{\topmargin}{-1.59999pt}.

[117]
Overfull \hbox (4.92744pt too wide) detected at line 4621
 []\OT1/lmss/m/n/10.95 1201
 []


Overfull \hbox (4.92744pt too wide) detected at line 4622
 []\OT1/lmss/m/n/10.95 1201
 []


Overfull \hbox (4.92744pt too wide) detected at line 4623
 []\OT1/lmss/m/n/10.95 1202
 []


Overfull \hbox (4.92744pt too wide) detected at line 4624
 []\OT1/lmss/m/n/10.95 1202
 []


Overfull \hbox (4.92744pt too wide) detected at line 4625
 []\OT1/lmss/m/n/10.95 1202
 []


Overfull \hbox (4.92744pt too wide) detected at line 4626
 []\OT1/lmss/m/n/10.95 1202
 []


Overfull \hbox (4.92744pt too wide) detected at line 4627
 []\OT1/lmss/m/n/10.95 1202
 []


Overfull \hbox (4.92744pt too wide) detected at line 4628
 []\OT1/lmss/m/n/10.95 1202
 []


Overfull \hbox (4.92744pt too wide) detected at line 4629
 []\OT1/lmss/m/n/10.95 1202
 []


Overfull \hbox (4.92744pt too wide) detected at line 4630
 []\OT1/lmss/m/n/10.95 1203
 []


Overfull \hbox (4.92744pt too wide) detected at line 4631
 []\OT1/lmss/m/n/10.95 1203
 []


Overfull \hbox (4.92744pt too wide) detected at line 4632
 []\OT1/lmss/m/n/10.95 1203
 []


Overfull \hbox (4.92744pt too wide) detected at line 4633
 []\OT1/lmss/m/n/10.95 1203
 []


Overfull \hbox (4.92744pt too wide) detected at line 4634
 []\OT1/lmss/m/n/10.95 1204
 []


Overfull \hbox (4.92744pt too wide) detected at line 4635
 []\OT1/lmss/m/n/10.95 1204
 []


Overfull \hbox (4.92744pt too wide) detected at line 4637
 []\OT1/lmss/m/n/10.95 1206
 []


Overfull \hbox (4.92744pt too wide) detected at line 4638
 []\OT1/lmss/m/n/10.95 1207
 []


Overfull \hbox (4.92744pt too wide) detected at line 4639
 []\OT1/lmss/m/n/10.95 1207
 []


Overfull \hbox (4.92744pt too wide) detected at line 4640
 []\OT1/lmss/m/n/10.95 1207
 []


Overfull \hbox (4.92744pt too wide) detected at line 4641
 []\OT1/lmss/m/n/10.95 1207
 []


Overfull \hbox (4.92744pt too wide) detected at line 4642
 []\OT1/lmss/m/n/10.95 1208
 []


Overfull \hbox (4.92744pt too wide) detected at line 4644
 []\OT1/lmss/m/n/10.95 1208
 []


Overfull \hbox (4.92744pt too wide) detected at line 4645
 []\OT1/lmss/m/n/10.95 1208
 []


Overfull \hbox (4.92744pt too wide) detected at line 4646
 []\OT1/lmss/m/n/10.95 1208
 []


Overfull \hbox (4.92744pt too wide) detected at line 4647
 []\OT1/lmss/m/n/10.95 1209
 []


Overfull \hbox (4.92744pt too wide) detected at line 4648
 []\OT1/lmss/m/n/10.95 1209
 []


Overfull \hbox (4.92744pt too wide) detected at line 4649
 []\OT1/lmss/m/n/10.95 1209
 []


Overfull \hbox (4.92744pt too wide) detected at line 4650
 []\OT1/lmss/m/n/10.95 1209
 []


Overfull \hbox (4.92744pt too wide) detected at line 4651
 []\OT1/lmss/m/n/10.95 1209
 []


Overfull \hbox (4.92744pt too wide) detected at line 4652
 []\OT1/lmss/m/n/10.95 1210
 []


Overfull \hbox (4.92744pt too wide) detected at line 4653
 []\OT1/lmss/m/n/10.95 1211
 []


Overfull \hbox (4.92744pt too wide) detected at line 4654
 []\OT1/lmss/m/n/10.95 1211
 []


Overfull \hbox (4.92744pt too wide) detected at line 4655
 []\OT1/lmss/m/n/10.95 1211
 []


Overfull \hbox (4.92744pt too wide) detected at line 4656
 []\OT1/lmss/m/n/10.95 1212
 []


Overfull \hbox (4.92744pt too wide) detected at line 4657
 []\OT1/lmss/m/n/10.95 1212
 []




Package fancyhdr Warning: \headheight is too small (12.0pt): 
(fancyhdr)                Make it at least 13.59999pt, for example:
(fancyhdr)                \setlength{\headheight}{13.59999pt}.
(fancyhdr)                You might also make \topmargin smaller:
(fancyhdr)                \addtolength{\topmargin}{-1.59999pt}.

[118]
Overfull \hbox (4.92744pt too wide) detected at line 4658
 []\OT1/lmss/m/n/10.95 1213
 []


Overfull \hbox (4.92744pt too wide) detected at line 4659
 []\OT1/lmss/m/n/10.95 1214
 []


Overfull \hbox (4.92744pt too wide) detected at line 4660
 []\OT1/lmss/m/n/10.95 1214
 []


Overfull \hbox (4.92744pt too wide) detected at line 4662
 []\OT1/lmss/m/n/10.95 1215
 []


Overfull \hbox (4.92744pt too wide) detected at line 4663
 []\OT1/lmss/m/n/10.95 1215
 []


Overfull \hbox (4.92744pt too wide) detected at line 4664
 []\OT1/lmss/m/n/10.95 1215
 []


Overfull \hbox (4.92744pt too wide) detected at line 4665
 []\OT1/lmss/m/n/10.95 1215
 []


Overfull \hbox (4.92744pt too wide) detected at line 4667
 []\OT1/lmss/m/n/10.95 1217
 []


Overfull \hbox (4.92744pt too wide) detected at line 4668
 []\OT1/lmss/m/n/10.95 1217
 []


Overfull \hbox (4.92744pt too wide) detected at line 4669
 []\OT1/lmss/m/n/10.95 1217
 []


Overfull \hbox (4.92744pt too wide) detected at line 4670
 []\OT1/lmss/m/n/10.95 1218
 []


Overfull \hbox (4.92744pt too wide) detected at line 4671
 []\OT1/lmss/m/n/10.95 1218
 []


Overfull \hbox (4.92744pt too wide) detected at line 4672
 []\OT1/lmss/m/n/10.95 1218
 []


Overfull \hbox (4.92744pt too wide) detected at line 4673
 []\OT1/lmss/m/n/10.95 1218
 []


Overfull \hbox (4.92744pt too wide) detected at line 4674
 []\OT1/lmss/m/n/10.95 1219
 []


Overfull \hbox (4.92744pt too wide) detected at line 4675
 []\OT1/lmss/m/n/10.95 1219
 []


Overfull \hbox (4.92744pt too wide) detected at line 4676
 []\OT1/lmss/m/n/10.95 1219
 []


Overfull \hbox (4.92744pt too wide) detected at line 4677
 []\OT1/lmss/m/n/10.95 1219
 []


Overfull \hbox (4.92744pt too wide) detected at line 4678
 []\OT1/lmss/m/n/10.95 1219
 []


Overfull \hbox (4.92744pt too wide) detected at line 4679
 []\OT1/lmss/m/n/10.95 1219
 []


Overfull \hbox (4.92744pt too wide) detected at line 4680
 []\OT1/lmss/m/n/10.95 1219
 []


Overfull \hbox (4.92744pt too wide) detected at line 4681
 []\OT1/lmss/m/n/10.95 1220
 []


Overfull \hbox (4.92744pt too wide) detected at line 4682
 []\OT1/lmss/m/n/10.95 1220
 []


Overfull \hbox (4.92744pt too wide) detected at line 4683
 []\OT1/lmss/m/n/10.95 1220
 []


Overfull \hbox (4.92744pt too wide) detected at line 4684
 []\OT1/lmss/m/n/10.95 1221
 []


Overfull \hbox (4.92744pt too wide) detected at line 4685
 []\OT1/lmss/m/n/10.95 1221
 []


Overfull \hbox (4.92744pt too wide) detected at line 4686
 []\OT1/lmss/m/n/10.95 1221
 []


Overfull \hbox (4.92744pt too wide) detected at line 4687
 []\OT1/lmss/m/n/10.95 1221
 []


Overfull \hbox (4.92744pt too wide) detected at line 4688
 []\OT1/lmss/m/n/10.95 1221
 []


Overfull \hbox (4.92744pt too wide) detected at line 4689
 []\OT1/lmss/m/n/10.95 1221
 []


Overfull \hbox (4.92744pt too wide) detected at line 4690
 []\OT1/lmss/m/n/10.95 1222
 []


Overfull \hbox (4.92744pt too wide) detected at line 4691
 []\OT1/lmss/m/n/10.95 1222
 []


Overfull \hbox (4.92744pt too wide) detected at line 4692
 []\OT1/lmss/m/n/10.95 1222
 []


Overfull \hbox (4.92744pt too wide) detected at line 4693
 []\OT1/lmss/m/n/10.95 1222
 []


Overfull \hbox (4.92744pt too wide) detected at line 4694
 []\OT1/lmss/m/n/10.95 1222
 []


Overfull \hbox (4.92744pt too wide) detected at line 4695
 []\OT1/lmss/m/n/10.95 1222
 []




Package fancyhdr Warning: \headheight is too small (12.0pt): 
(fancyhdr)                Make it at least 13.59999pt, for example:
(fancyhdr)                \setlength{\headheight}{13.59999pt}.
(fancyhdr)                You might also make \topmargin smaller:
(fancyhdr)                \addtolength{\topmargin}{-1.59999pt}.

[119]
Overfull \hbox (4.92744pt too wide) detected at line 4696
 []\OT1/lmss/m/n/10.95 1222
 []


Overfull \hbox (4.92744pt too wide) detected at line 4697
 []\OT1/lmss/m/n/10.95 1223
 []


Overfull \hbox (4.92744pt too wide) detected at line 4698
 []\OT1/lmss/m/n/10.95 1223
 []


Overfull \hbox (4.92744pt too wide) detected at line 4699
 []\OT1/lmss/m/n/10.95 1223
 []


Overfull \hbox (4.92744pt too wide) detected at line 4701
 []\OT1/lmss/m/n/10.95 1224
 []


Overfull \hbox (4.92744pt too wide) detected at line 4702
 []\OT1/lmss/m/n/10.95 1224
 []


Overfull \hbox (4.92744pt too wide) detected at line 4703
 []\OT1/lmss/m/n/10.95 1224
 []


Overfull \hbox (4.92744pt too wide) detected at line 4704
 []\OT1/lmss/m/n/10.95 1225
 []


Overfull \hbox (4.92744pt too wide) detected at line 4705
 []\OT1/lmss/m/n/10.95 1225
 []


Overfull \hbox (4.92744pt too wide) detected at line 4706
 []\OT1/lmss/m/n/10.95 1225
 []


Overfull \hbox (4.92744pt too wide) detected at line 4707
 []\OT1/lmss/m/n/10.95 1225
 []


Overfull \hbox (4.92744pt too wide) detected at line 4708
 []\OT1/lmss/m/n/10.95 1226
 []


Overfull \hbox (4.92744pt too wide) detected at line 4709
 []\OT1/lmss/m/n/10.95 1227
 []


Overfull \hbox (4.92744pt too wide) detected at line 4710
 []\OT1/lmss/m/n/10.95 1227
 []


Overfull \hbox (4.92744pt too wide) detected at line 4711
 []\OT1/lmss/m/n/10.95 1227
 []


Overfull \hbox (4.92744pt too wide) detected at line 4713
 []\OT1/lmss/m/n/10.95 1229
 []


Overfull \hbox (4.92744pt too wide) detected at line 4714
 []\OT1/lmss/m/n/10.95 1229
 []


Overfull \hbox (4.92744pt too wide) detected at line 4715
 []\OT1/lmss/m/n/10.95 1230
 []


Overfull \hbox (4.92744pt too wide) detected at line 4716
 []\OT1/lmss/m/n/10.95 1230
 []


Overfull \hbox (4.92744pt too wide) detected at line 4717
 []\OT1/lmss/m/n/10.95 1230
 []


Overfull \hbox (4.92744pt too wide) detected at line 4718
 []\OT1/lmss/m/n/10.95 1230
 []


Overfull \hbox (4.92744pt too wide) detected at line 4719
 []\OT1/lmss/m/n/10.95 1230
 []


Overfull \hbox (4.92744pt too wide) detected at line 4720
 []\OT1/lmss/m/n/10.95 1230
 []


Overfull \hbox (4.92744pt too wide) detected at line 4721
 []\OT1/lmss/m/n/10.95 1231
 []


Overfull \hbox (4.92744pt too wide) detected at line 4722
 []\OT1/lmss/m/n/10.95 1231
 []


Overfull \hbox (4.92744pt too wide) detected at line 4723
 []\OT1/lmss/m/n/10.95 1231
 []


Overfull \hbox (4.92744pt too wide) detected at line 4724
 []\OT1/lmss/m/n/10.95 1231
 []


Overfull \hbox (4.92744pt too wide) detected at line 4725
 []\OT1/lmss/m/n/10.95 1231
 []


Overfull \hbox (4.92744pt too wide) detected at line 4726
 []\OT1/lmss/m/n/10.95 1231
 []


Overfull \hbox (4.92744pt too wide) detected at line 4727
 []\OT1/lmss/m/n/10.95 1231
 []


Overfull \hbox (4.92744pt too wide) detected at line 4728
 []\OT1/lmss/m/n/10.95 1231
 []


Overfull \hbox (4.92744pt too wide) detected at line 4729
 []\OT1/lmss/m/n/10.95 1231
 []


Overfull \hbox (4.92744pt too wide) detected at line 4730
 []\OT1/lmss/m/n/10.95 1231
 []


Underfull \vbox (badness 1038) has occurred while \output is active []




Package fancyhdr Warning: \headheight is too small (12.0pt): 
(fancyhdr)                Make it at least 13.59999pt, for example:
(fancyhdr)                \setlength{\headheight}{13.59999pt}.
(fancyhdr)                You might also make \topmargin smaller:
(fancyhdr)                \addtolength{\topmargin}{-1.59999pt}.

[120]
Overfull \hbox (4.92744pt too wide) detected at line 4731
 []\OT1/lmss/m/n/10.95 1231
 []


Overfull \hbox (4.92744pt too wide) detected at line 4732
 []\OT1/lmss/m/n/10.95 1232
 []


Overfull \hbox (4.92744pt too wide) detected at line 4733
 []\OT1/lmss/m/n/10.95 1232
 []


Overfull \hbox (4.92744pt too wide) detected at line 4734
 []\OT1/lmss/m/n/10.95 1232
 []


Overfull \hbox (4.92744pt too wide) detected at line 4735
 []\OT1/lmss/m/n/10.95 1232
 []


Overfull \hbox (4.92744pt too wide) detected at line 4736
 []\OT1/lmss/m/n/10.95 1232
 []


Overfull \hbox (4.92744pt too wide) detected at line 4737
 []\OT1/lmss/m/n/10.95 1232
 []


Overfull \hbox (4.92744pt too wide) detected at line 4738
 []\OT1/lmss/m/n/10.95 1232
 []


Overfull \hbox (4.92744pt too wide) detected at line 4739
 []\OT1/lmss/m/n/10.95 1232
 []


Overfull \hbox (4.92744pt too wide) detected at line 4740
 []\OT1/lmss/m/n/10.95 1232
 []


Overfull \hbox (4.92744pt too wide) detected at line 4741
 []\OT1/lmss/m/n/10.95 1233
 []


Overfull \hbox (4.92744pt too wide) detected at line 4742
 []\OT1/lmss/m/n/10.95 1233
 []


Overfull \hbox (4.92744pt too wide) detected at line 4743
 []\OT1/lmss/m/n/10.95 1233
 []


Overfull \hbox (4.92744pt too wide) detected at line 4744
 []\OT1/lmss/m/n/10.95 1233
 []


Overfull \hbox (4.92744pt too wide) detected at line 4745
 []\OT1/lmss/m/n/10.95 1233
 []


Overfull \hbox (4.92744pt too wide) detected at line 4746
 []\OT1/lmss/m/n/10.95 1234
 []


Overfull \hbox (4.92744pt too wide) detected at line 4747
 []\OT1/lmss/m/n/10.95 1234
 []


Overfull \hbox (4.92744pt too wide) detected at line 4748
 []\OT1/lmss/m/n/10.95 1234
 []


Overfull \hbox (4.92744pt too wide) detected at line 4749
 []\OT1/lmss/m/n/10.95 1234
 []


Overfull \hbox (4.92744pt too wide) detected at line 4750
 []\OT1/lmss/m/n/10.95 1234
 []


Overfull \hbox (4.92744pt too wide) detected at line 4751
 []\OT1/lmss/m/n/10.95 1234
 []


Overfull \hbox (4.92744pt too wide) detected at line 4752
 []\OT1/lmss/m/n/10.95 1234
 []


Overfull \hbox (4.92744pt too wide) detected at line 4753
 []\OT1/lmss/m/n/10.95 1234
 []


Overfull \hbox (4.92744pt too wide) detected at line 4754
 []\OT1/lmss/m/n/10.95 1235
 []


Overfull \hbox (4.92744pt too wide) detected at line 4755
 []\OT1/lmss/m/n/10.95 1235
 []


Overfull \hbox (4.92744pt too wide) detected at line 4756
 []\OT1/lmss/m/n/10.95 1235
 []


Overfull \hbox (4.92744pt too wide) detected at line 4757
 []\OT1/lmss/m/n/10.95 1235
 []


Overfull \hbox (4.92744pt too wide) detected at line 4758
 []\OT1/lmss/m/n/10.95 1235
 []


Overfull \hbox (4.92744pt too wide) detected at line 4760
 []\OT1/lmss/m/n/10.95 1237
 []


Overfull \hbox (4.92744pt too wide) detected at line 4761
 []\OT1/lmss/m/n/10.95 1237
 []


Overfull \hbox (4.92744pt too wide) detected at line 4762
 []\OT1/lmss/m/n/10.95 1237
 []


Overfull \hbox (4.92744pt too wide) detected at line 4763
 []\OT1/lmss/m/n/10.95 1237
 []


Overfull \hbox (4.92744pt too wide) detected at line 4764
 []\OT1/lmss/m/n/10.95 1238
 []


Overfull \hbox (4.92744pt too wide) detected at line 4765
 []\OT1/lmss/m/n/10.95 1238
 []




Package fancyhdr Warning: \headheight is too small (12.0pt): 
(fancyhdr)                Make it at least 13.59999pt, for example:
(fancyhdr)                \setlength{\headheight}{13.59999pt}.
(fancyhdr)                You might also make \topmargin smaller:
(fancyhdr)                \addtolength{\topmargin}{-1.59999pt}.

[121]
Overfull \hbox (4.92744pt too wide) detected at line 4766
 []\OT1/lmss/m/n/10.95 1238
 []


Overfull \hbox (4.92744pt too wide) detected at line 4767
 []\OT1/lmss/m/n/10.95 1238
 []


Overfull \hbox (4.92744pt too wide) detected at line 4768
 []\OT1/lmss/m/n/10.95 1238
 []


Overfull \hbox (4.92744pt too wide) detected at line 4769
 []\OT1/lmss/m/n/10.95 1239
 []


Overfull \hbox (4.92744pt too wide) detected at line 4770
 []\OT1/lmss/m/n/10.95 1239
 []

)
\tf@toc=\write5
\openout5 = `T0_Book_En.toc'.

 (chapters_en/T0_Grundlagen_En_ch.tex


Package fancyhdr Warning: \headheight is too small (12.0pt): 
(fancyhdr)                Make it at least 13.59999pt, for example:
(fancyhdr)                \setlength{\headheight}{13.59999pt}.
(fancyhdr)                You might also make \topmargin smaller:
(fancyhdr)                \addtolength{\topmargin}{-1.59999pt}.

[122]
Chapter 1.
! Undefined control sequence.
<argument> \cftsecindent 
                         
l.10 \setlength{\cftsecindent}{0pt}
                                   
The control sequence at the end of the top line
of your error message was never \def'ed. If you have
misspelled it (e.g., `\hobx'), type `I' and the correct
spelling (e.g., `I\hbox'). Otherwise just continue,
and I'll forget about whatever was undefined.

! Undefined control sequence.
<argument> \cftsubsecindent 
                            
l.11 \setlength{\cftsubsecindent}{0pt}
                                      
The control sequence at the end of the top line
of your error message was never \def'ed. If you have
misspelled it (e.g., `\hobx'), type `I' and the correct
spelling (e.g., `I\hbox'). Otherwise just continue,
and I'll forget about whatever was undefined.

! Too many }'s.
l.14 }
      
You've closed more groups than you opened.
Such booboos are generally harmless, so keep going.

! Too many }'s.
l.25 	\normalsize Document 1 of the T0 Series}
                                              
You've closed more groups than you opened.
Such booboos are generally harmless, so keep going.



[123

]

Package hyperref Warning: Difference (2) between bookmark levels is greater 
(hyperref)                than one, level fixed on input line 34.


! LaTeX Error: Environment foundation undefined.

See the LaTeX manual or LaTeX Companion for explanation.
Type  H <return>  for immediate help.
 ...                                              
                                                  
l.49 	\begin{foundation}
                        
Your command was ignored.
Type  I <command> <return>  to replace it with another command,
or  <return>  to continue without it.


! LaTeX Error: 

==================================================

=== T0_Book_En.tex.preamble ===

\documentclass[11pt,a4paper,openany]{book}
\usepackage[a4paper,margin=2cm]{geometry}
\usepackage[utf8]{inputenc}
\usepackage[english]{babel}
\usepackage{lmodern}
\renewcommand{\familydefault}{\sfdefault}

\usepackage{amsmath,amssymb,amsthm}
\usepackage{graphicx}
\usepackage[unicode,pdfencoding=auto]{hyperref}
\usepackage{booktabs}
\usepackage{longtable}
\usepackage{siunitx}
\usepackage{fancyhdr}
\usepackage{float}
\usepackage{tikz}
\usepackage[most]{tcolorbox}
\tcbset{colback=white,colframe=black,arc=2pt,boxrule=0.5pt}
\tikzset{
  t0blue/.style={draw=blue,fill=blue!10},
  t0red/.style={draw=red,fill=red!10},
  t0green/.style={draw=green!50!black,fill=green!10},
  t0orange/.style={draw=orange,fill=orange!10},
}
\usepackage{setspace}
\usepackage{enumitem}
\usepackage{adjustbox}
\usepackage{xcolor}

\setlength{\parindent}{0pt}
\setlength{\parskip}{6pt}

\hypersetup{
  colorlinks=true,
  linkcolor=blue,
  citecolor=blue,
  urlcolor=blue
}
\pagestyle{fancy}
\setlength{\headheight}{15pt}

\newcommand{\checkmarkx}{\checkmark}
\newcommand{\warningx}{\textbf{!}}

% Makros aus Einzel-Dokumenten (Fallback-Definitionen)
\newcommand{\mytimes}{\times}
\newcommand{\myapprox}{\approx}
\newcommand{\mysim}{\sim}
\newcommand{\myomega}{\omega}
\newcommand{\mypi}{\pi}
\newcommand{\myrightarrow}{\rightarrow}
\newcommand{\mypropto}{\propto}
\newcommand{\deltafield}{\delta\phi}
\newcommand{\xipar}{\xi}
\newcommand{\lambdah}{\lambda_h}

% Einfache abstract-Umgebung für Kapitel:
\newenvironment{abstract}{%
  \begin{center}\bfseries Abstract\end{center}\small
}{\par}

\title{T0 Time--Mass Duality\\Unified English Book}
\author{J. Pascher}
\date{\today}



==================================================

=== T0_Book_En_ch.tex.preamble ===

\chapter{T0 Book En}


\usepackage[utf8]{inputenc}
\usepackage[english]{babel}
\usepackage{amsmath,amssymb,amsthm}
\usepackage{graphicx}
\usepackage{hyperref}
\usepackage{cite}

\title{T0 Time--Mass Duality\\Unified English Book}
\author{J. Pascher}
\date{\today}



==================================================

=== T0_Book_En_sceleton.tex.preamble ===

\documentclass[11pt,a4paper,openany]{book}
% Skeleton-Buchdatei. Wird später automatisch mit Inhalt gefüllt.

\usepackage[a4paper,margin=2cm]{geometry}
\usepackage[utf8]{inputenc}
\usepackage[english]{babel}
\usepackage{lmodern}
\renewcommand{\familydefault}{\sfdefault}

\usepackage{amsmath,amssymb,amsthm}
\usepackage{graphicx}
\usepackage[unicode,pdfencoding=auto]{hyperref}
\usepackage{booktabs}
\usepackage{longtable}
\usepackage{siunitx}
\usepackage{fancyhdr}
\usepackage{float}
\usepackage{tikz}
\usepackage{setspace}
\usepackage{enumitem}
\usepackage{adjustbox}
\usepackage{xcolor}

\setlength{\parindent}{0pt}
\setlength{\parskip}{6pt}

\hypersetup{
  colorlinks=true,
  linkcolor=blue,
  citecolor=blue,
  urlcolor=blue
}
\pagestyle{fancy}

\newcommand{\checkmarkx}{\checkmark}
\newcommand{\warningx}{\textbf{!}}

\title{T0 Time--Mass Duality\\Unified English Book (Skeleton)}
\author{J. Pascher}
\date{\today}



==================================================

=== T0_Dokumentenübersicht_De.tex.preamble ===

\documentclass[12pt,a4paper]{article}
\usepackage[utf8]{inputenc}
\usepackage[T1]{fontenc}
\usepackage[ngerman]{babel}
\usepackage{lmodern}
\usepackage{amsmath,amssymb,amsthm}
\usepackage{geometry}
\usepackage{booktabs}
\usepackage{array}
\usepackage{xcolor}
\usepackage{tcolorbox}
\usepackage{fancyhdr}
\usepackage{tocloft}
\usepackage{hyperref}
\usepackage{tikz}
\usepackage{physics}
\usepackage{siunitx}
\usepackage{longtable}
\usepackage{graphicx}
\usepackage{multicol}

\definecolor{t0blue}{RGB}{33,150,243}
\definecolor{t0green}{RGB}{76,175,80}
\definecolor{t0orange}{RGB}{255,152,0}
\definecolor{t0red}{RGB}{244,67,54}
\definecolor{t0purple}{RGB}{156,39,176}

\geometry{a4paper, margin=2cm}
\setlength{\headheight}{15pt}

% Header- und Footer-Konfiguration
\pagestyle{fancy}
\fancyhf{}
\fancyhead[L]{\textsc{T0-Theorie: Dokumentenübersicht}}
\fancyhead[R]{\textsc{J. Pascher}}
\fancyfoot[C]{\thepage}
\renewcommand{\headrulewidth}{0.4pt}
\renewcommand{\footrulewidth}{0.4pt}

% Hyperref-Einstellungen
\hypersetup{
	colorlinks=true,
	linkcolor=t0blue,
	citecolor=t0blue,
	urlcolor=t0blue,
	pdftitle={T0-Theorie: Vollständige Dokumentenübersicht},
	pdfauthor={Johann Pascher},
	pdfsubject={T0-Theorie, Geometrische Physik, Übersicht}
}

% Benutzerdefinierte Befehle
\newcommand{\xipar}{\xi}
\newcommand{\Efield}{E_{\text{field}}}
\newcommand{\EP}{E_{\text{P}}}
\newcommand{\docref}[1]{\texttt{#1}}

% Umgebungen für verschiedene Bereiche
\newtcolorbox{overview}{colback=t0blue!5, colframe=t0blue!75!black, title=Gesamtübersicht}
\newtcolorbox{foundation}{colback=t0green!5, colframe=t0green!75!black, title=Fundamentale Erkenntnisse}
\newtcolorbox{achievement}{colback=t0orange!5, colframe=t0orange!75!black, title=Wissenschaftliche Erfolge}
\newtcolorbox{documentbox}{colback=t0purple!5, colframe=t0purple!75!black, title=Dokumentinhalt}

\title{\textbf{T0-Theorie: Dokumentenserieübersicht}\\[0.5cm]
	\large Eine revolutionäre geometrische Reformulierung der Physik\\[0.3cm]
	\normalsize Systematische Darstellung aller 8 Kerndokumente}
\author{Johann Pascher\\
	Abteilung für Kommunikationstechnologie\\
	Höhere Technische Lehranstalt (HTL), Leonding, Österreich\\
	\texttt{johann.pascher@gmail.com}}
\date{\today}



==================================================

=== T0_Dokumentenübersicht_En.tex.preamble ===

\documentclass[12pt,a4paper]{article}
\usepackage[utf8]{inputenc}
\usepackage[T1]{fontenc}
\usepackage[english]{babel}
\usepackage{lmodern}
\usepackage{amsmath,amssymb,amsthm}
\usepackage{geometry}
\usepackage{booktabs}
\usepackage{array}
\usepackage{xcolor}
\usepackage{tcolorbox}
\usepackage{fancyhdr}
\usepackage{tocloft}
\usepackage{hyperref}
\usepackage{tikz}
\usepackage{physics}
\usepackage{siunitx}
\usepackage{longtable}
\usepackage{graphicx}
\usepackage{multicol}

\definecolor{t0blue}{RGB}{33,150,243}
\definecolor{t0green}{RGB}{76,175,80}
\definecolor{t0orange}{RGB}{255,152,0}
\definecolor{t0red}{RGB}{244,67,54}
\definecolor{t0purple}{RGB}{156,39,176}

\geometry{a4paper, margin=2cm}
\setlength{\headheight}{15pt}

% Header and Footer Configuration
\pagestyle{fancy}
\fancyhf{}
\fancyhead[L]{\textsc{T0-Theory: Document Overview}}
\fancyhead[R]{\textsc{J. Pascher}}
\fancyfoot[C]{\thepage}
\renewcommand{\headrulewidth}{0.4pt}
\renewcommand{\footrulewidth}{0.4pt}

% Hyperref Settings
\hypersetup{
	colorlinks=true,
	linkcolor=t0blue,
	citecolor=t0blue,
	urlcolor=t0blue,
	pdftitle={T0-Theory: Complete Document Overview},
	pdfauthor={Johann Pascher},
	pdfsubject={T0-Theory, Geometric Physics, Overview}
}

% Custom Commands
\newcommand{\xipar}{\xi}
\newcommand{\Efield}{E_{\text{field}}}
\newcommand{\EP}{E_{\text{P}}}
\newcommand{\docref}[1]{\texttt{#1}}

% Environments for Different Areas
\newtcolorbox{overview}{colback=t0blue!5, colframe=t0blue!75!black, title={Overall Overview}}
\newtcolorbox{foundation}{colback=t0green!5, colframe=t0green!75!black, title={Fundamental Insights}}
\newtcolorbox{achievement}{colback=t0orange!5, colframe=t0orange!75!black, title={Scientific Achievements}}
\newtcolorbox{documentbox}{colback=t0purple!5, colframe=t0purple!75!black, title={Document Content}}

\title{\textbf{T0-Theory: Document Series Overview}\\[0.5cm]
	\large A Revolutionary Geometric Reformulation of Physics\\[0.3cm]
	\normalsize Systematic Presentation of All 8 Core Documents}
\author{Johann Pascher\\
	Department of Communication Technology\\
	Higher Technical College (HTL), Leonding, Austria\\
	\texttt{johann.pascher@gmail.com}}
\date{\today}



==================================================

=== T0_Energie_De.tex.preamble ===

\documentclass[12pt,a4paper]{report}
\usepackage[utf8]{inputenc}
\usepackage[T1]{fontenc}
\usepackage[ngerman]{babel}
\usepackage[left=2.5cm,right=2.5cm,top=3cm,bottom=3cm]{geometry}
\usepackage{lmodern}
\usepackage{amsmath}
\usepackage{amssymb}
\usepackage{physics}
\usepackage{hyperref}
\usepackage{booktabs}
\usepackage{enumitem}
\usepackage[table]{xcolor}
\usepackage{graphicx}
\usepackage{float}
\usepackage{mathtools}
\usepackage{amsthm}
\usepackage{cleveref}
\usepackage{siunitx}
\usepackage{fancyhdr}
\usepackage{tocloft}
\usepackage{longtable}
\usepackage{array}
\usepackage{microtype}
\usepackage{pdflscape}
\usepackage{newunicodechar}
\usepackage{tikz}
\usepackage{pgfplots}
\usepackage{tcolorbox}

% Setup
\pgfplotsset{compat=1.18}
\usetikzlibrary{positioning,shapes,arrows}

% Erweiterte Typographische Einstellungen
\emergencystretch 3em
\tolerance 9999
\hbadness 9999
\setlength{\hfuzz}{15pt}

% Kopf- und Fußzeilen-Konfiguration
\pagestyle{fancy}
\fancyhf{}
\fancyhead[L]{\textsc{T0-Modell (Planck-Referenziert)}}
\fancyhead[R]{\textsc{Reine Energie-Physik}}
\fancyfoot[C]{\thepage}
\renewcommand{\headrulewidth}{0.4pt}
\renewcommand{\footrulewidth}{0.4pt}

% Inhaltsverzeichnis-Styling
\renewcommand{\cfttoctitlefont}{\huge\bfseries\color{blue}}
\renewcommand{\cftchapfont}{\large\bfseries\color{blue}}
\renewcommand{\cftsecfont}{\color{blue}}
\renewcommand{\cftsubsecfont}{\color{blue}}
\renewcommand{\cftchappagefont}{\large\bfseries\color{blue}}
\renewcommand{\cftsecpagefont}{\color{blue}}
\renewcommand{\cftsubsecpagefont}{\color{blue}}

% Hyperlink-Setup
\hypersetup{
	colorlinks=true,
	linkcolor=blue,
	citecolor=blue,
	urlcolor=blue,
	pdftitle={Das T0-Modell (Planck-Referenziert): Eine Neuformulierung der Physik},
	pdfauthor={Johann Pascher},
	pdfsubject={T0-Modell, Planck-Referenzierte Physik, Theoretische Physik, Natürliche Einheiten},
	pdfkeywords={T0 Theorie, Planck-Skala, Quantenmechanik, Feldtheorie, Vereinheitlichte Physik}
}

% Mathematische Notation - PLANCK-REFERENZIERT
\newcommand{\Tfield}{T(x,t)}              % Intrinsisches Zeitfeld
\newcommand{\Efield}{E(x,t)}              % Dynamisches Energiefeld
\newcommand{\xipar}{\xi}                  % Fundamentaler dimensionsloser Parameter
\newcommand{\betaT}{\beta_{T}}            % Zeitparameter in natürlichen Einheiten = 1
\newcommand{\alphaEM}{\alpha_{\text{EM}}} % Elektromagnetische Kopplungskonstante
\newcommand{\EP}{E_{\text{P}}}            % Planck-Energie
\newcommand{\lP}{\ell_{\text{P}}}         % Planck-Länge (REFERENZ)
\newcommand{\tP}{t_{\text{P}}}            % Planck-Zeit (REFERENZ)
\newcommand{\Tzero}{T_0}                  % Grundzustand des Zeitfeldes
\newcommand{\Lambdat}{\Lambda_T}          % Feldkonstante

% T0-Skalen - PLANCK-REFERENZIERT
\newcommand{\rzero}{r_0}                  % T0-charakteristische Länge: r_0 = 2GE
\newcommand{\tzero}{t_0}                  % T0-charakteristische Zeit: t_0 = r_0/c = 2GE
\newcommand{\xigeom}{\xi_{\text{geom}}}   % Geometrischer Parameter: 4/3 × 10^-4
\newcommand{\xirat}{\xi_{\text{ratio}}}   % Skalenverhältnis: ℓ_P/r_0

% Energie-basierte Teilchen-Notation
\newcommand{\Ee}{E_e}                     % Elektron-charakteristische Energie
\newcommand{\Emu}{E_\mu}                  % Myon-charakteristische Energie  
\newcommand{\Etau}{E_\tau}                % Tau-charakteristische Energie
\newcommand{\Ep}{E_p}                     % Proton-charakteristische Energie
\newcommand{\En}{E_n}                     % Neutron-charakteristische Energie
\newcommand{\Eh}{E_h}                     % Higgs-charakteristische Energie
\newcommand{\EW}{E_W}                     % W-Boson-charakteristische Energie
\newcommand{\EZ}{E_Z}                     % Z-Boson-charakteristische Energie
\newcommand{\Egamma}{E_\gamma}            % Photon-Energie (massenlos)

% Zusätzliche mathematische Befehle
\newcommand{\deltaE}{\delta E}            % Energiefeld-Fluktuation
\newcommand{\Lag}{\mathcal{L}}           % Lagrange-Dichte
\newcommand{\Tfieldt}{T(\vec{x},t)}      % Explizite Raum-Zeit-Abhängigkeit
\newcommand{\vecx}{\vec{x}}              % Positionsvektor
\newcommand{\alphaW}{\alpha_{\text{W}}}  % Schwache Wechselwirkungskonstante
\newcommand{\alphaT}{\alpha_{\text{T}}}  % Zeitfeld-Kopplungskonstante
\newcommand{\Rzero}{R_\infty}            % Rydberg-Konstante
\newcommand{\lambdah}{\lambda_h}         % Higgs-Kopplungskonstante

% Kopplungskonstanten und Verhältnisse
\newcommand{\alphafine}{\alpha}          % Feinstrukturkonstante
\newcommand{\alphaQED}{\alpha_{\text{QED}}} % QED-Kopplung
\newcommand{\alphaQCD}{\alpha_s}         % Starke Kopplung
\newcommand{\gW}{g_W}                    % Schwache Kopplungskonstante
\newcommand{\gs}{g_s}                    % Starke Kopplungskonstante

% Energieverhältnisse und dimensionslose Parameter
\newcommand{\Enorm}[1]{E_{\text{norm}}^{(#1)}} % Normalisierte Energie
\newcommand{\Eratio}[2]{\frac{E_{#1}}{E_{#2}}} % Energieverhältnis
\newcommand{\EPratio}[1]{\frac{#1}{\EP}}        % Planck-Energieverhältnis

% Natürliche Einheiten Erklärung
\newcommand{\natunits}{\hbar = c = G = k_B = 1} % Natürliche Einheiten-Setzung

% Theorem-Umgebungen
\newtheorem{principle}{Fundamentales Prinzip}[chapter]
\newtheorem{insight}{Zentrale Einsicht}[chapter]
\newtheorem{discovery}{Neue Entdeckung}[chapter]
\newtheorem{definition}{Definition}[chapter]
\newtheorem{theorem}{Theorem}[chapter]
\newtheorem{example}{Beispiel}[chapter]
\newtheorem{axiom}{Axiom}[chapter]

% T0-Skalen-Definitionen
\newcommand{\xiparticle}{\xi_{\text{particle}}}   % = 4/3 × 10^{-4}

% Dokument-Titelseite
\title{
	{\Huge Das T0-Modell (Planck-Referenziert)}\\
	{\LARGE Eine Neuformulierung der Physik}\\
	{\Large Von Zeit-Energie-Dualität zu reiner\\energie-basierter Beschreibung der Natur}\\
	\vspace{1cm}
	{\large Eine theoretische Arbeit über die fundamentale\\Vereinfachung physikalischer Konzepte durch\\energie-basierte Formulierungen mit Planck-Skalen-Referenz}
}

\author{
	{\Large Johann Pascher}\\
	Abteilung Kommunikationstechnik\\
	Höhere Technische Bundeslehranstalt (HTL), Leonding, Österreich\\
	\texttt{johann.pascher@gmail.com}
}

\date{\today}



==================================================

=== T0_Energie_En.tex.preamble ===

\documentclass[12pt,a4paper]{report}
\usepackage[utf8]{inputenc}
\usepackage[T1]{fontenc}
\usepackage[english]{babel}
\usepackage[left=2.5cm,right=2.5cm,top=3cm,bottom=3cm]{geometry}
\usepackage{lmodern}
\usepackage{amsmath}
\usepackage{amssymb}
\usepackage{physics}
\usepackage{hyperref}
\usepackage{booktabs}
\usepackage{enumitem}
\usepackage[table]{xcolor}
\usepackage{graphicx}
\usepackage{float}
\usepackage{mathtools}
\usepackage{amsthm}
\usepackage{cleveref}
\usepackage{siunitx}
\usepackage{fancyhdr}
\usepackage{tocloft}
\usepackage{longtable}
\usepackage{array}
\usepackage{microtype}
\usepackage{pdflscape}
\usepackage{newunicodechar}
\usepackage{tikz}
\usepackage{pgfplots}
\usepackage{tcolorbox}

% Setup
\pgfplotsset{compat=1.18}
\usetikzlibrary{positioning,shapes,arrows}

% Enhanced Typographic Settings
\emergencystretch 3em
\tolerance 9999
\hbadness 9999
\setlength{\hfuzz}{15pt}

% Header and Footer Configuration
\pagestyle{fancy}
\fancyhf{}
\fancyhead[L]{\textsc{T0-Model (Planck-Referenced)}}
\fancyhead[R]{\textsc{Pure Energy Physics}}
\fancyfoot[C]{\thepage}
\renewcommand{\headrulewidth}{0.4pt}
\renewcommand{\footrulewidth}{0.4pt}

% Table of Contents Styling
\renewcommand{\cfttoctitlefont}{\huge\bfseries\color{blue}}
\renewcommand{\cftchapfont}{\large\bfseries\color{blue}}
\renewcommand{\cftsecfont}{\color{blue}}
\renewcommand{\cftsubsecfont}{\color{blue}}
\renewcommand{\cftchappagefont}{\large\bfseries\color{blue}}
\renewcommand{\cftsecpagefont}{\color{blue}}
\renewcommand{\cftsubsecpagefont}{\color{blue}}

% Hyperlink Setup
\hypersetup{
	colorlinks=true,
	linkcolor=blue,
	citecolor=blue,
	urlcolor=blue,
	pdftitle={The T0-Model (Planck-Referenced): A Reformulation of Physics},
	pdfauthor={Johann Pascher},
	pdfsubject={T0-Model, Planck-Referenced Physics, Theoretical Physics, Natural Units},
	pdfkeywords={T0 Theory, Planck Scale, Quantum Mechanics, Field Theory, Unified Physics}
}

% Mathematical Notation - PLANCK-REFERENCED
\newcommand{\Tfield}{T(x,t)}              % Intrinsic time field
\newcommand{\Efield}{E(x,t)}              % Dynamic energy field
\newcommand{\xipar}{\xi}                  % Fundamental dimensionless parameter
\newcommand{\betaT}{\beta_{T}}            % Time parameter in natural units = 1
\newcommand{\alphaEM}{\alpha_{\text{EM}}} % Electromagnetic coupling constant
\newcommand{\EP}{E_{\text{P}}}            % Planck energy
\newcommand{\lP}{\ell_{\text{P}}}         % Planck length (REFERENCE)
\newcommand{\tP}{t_{\text{P}}}            % Planck time (REFERENCE)
\newcommand{\Tzero}{T_0}                  % Ground state of time field
\newcommand{\Lambdat}{\Lambda_T}          % Field constant

% T0 Scales - PLANCK-REFERENCED
\newcommand{\rzero}{r_0}                  % T0 characteristic length: r_0 = 2GE
\newcommand{\tzero}{t_0}                  % T0 characteristic time: t_0 = r_0/c = 2GE
\newcommand{\xigeom}{\xi_{\text{geom}}}   % Geometric parameter: 4/3 × 10^-4
\newcommand{\xirat}{\xi_{\text{ratio}}}   % Scale ratio: ℓ_P/r_0

% Energy-Based Particle Notation
\newcommand{\Ee}{E_e}                     % Electron characteristic energy
\newcommand{\Emu}{E_\mu}                  % Muon characteristic energy  
\newcommand{\Etau}{E_\tau}                % Tau characteristic energy
\newcommand{\Ep}{E_p}                     % Proton characteristic energy
\newcommand{\En}{E_n}                     % Neutron characteristic energy
\newcommand{\Eh}{E_h}                     % Higgs characteristic energy
\newcommand{\EW}{E_W}                     % W boson characteristic energy
\newcommand{\EZ}{E_Z}                     % Z boson characteristic energy
\newcommand{\Egamma}{E_\gamma}            % Photon energy (massless)

% Additional Mathematical Commands
\newcommand{\deltaE}{\delta E}            % Energy field fluctuation
\newcommand{\Lag}{\mathcal{L}}           % Lagrangian density
\newcommand{\Tfieldt}{T(\vec{x},t)}      % Explicit space-time dependence
\newcommand{\vecx}{\vec{x}}              % Position vector
\newcommand{\alphaW}{\alpha_{\text{W}}}  % Weak interaction constant
\newcommand{\alphaT}{\alpha_{\text{T}}}  % Time field coupling constant
\newcommand{\Rzero}{R_\infty}            % Rydberg constant
\newcommand{\lambdah}{\lambda_h}         % Higgs coupling constant

% Coupling Constants and Ratios
\newcommand{\alphafine}{\alpha}          % Fine structure constant
\newcommand{\alphaQED}{\alpha_{\text{QED}}} % QED coupling
\newcommand{\alphaQCD}{\alpha_s}         % Strong coupling
\newcommand{\gW}{g_W}                    % Weak coupling constant
\newcommand{\gs}{g_s}                    % Strong coupling constant

% Energy Ratios and Dimensionless Parameters
\newcommand{\Enorm}[1]{E_{\text{norm}}^{(#1)}} % Normalized energy
\newcommand{\Eratio}[2]{\frac{E_{#1}}{E_{#2}}} % Energy ratio
\newcommand{\EPratio}[1]{\frac{#1}{\EP}}        % Planck energy ratio

% Natural Units Explanation
\newcommand{\natunits}{\hbar = c = G = k_B = 1} % Natural units setting

% Theorem Environments
\newtheorem{principle}{Fundamental Principle}[chapter]
\newtheorem{insight}{Central Insight}[chapter]
\newtheorem{discovery}{New Discovery}[chapter]
\newtheorem{definition}{Definition}[chapter]
\newtheorem{theorem}{Theorem}[chapter]
\newtheorem{example}{Example}[chapter]
\newtheorem{axiom}{Axiom}[chapter]

% T0 Scale Definitions
\newcommand{\xiparticle}{\xi_{\text{particle}}}   % = 4/3 × 10^{-4}

% Document Title Page
\title{
	{\Huge The T0-Model (Planck-Referenced)}\\
	{\LARGE A Reformulation of Physics}\\
	{\Large From Time-Energy Duality to Pure\\Energy-Based Description of Nature}\\
	\vspace{1cm}
	{\large A theoretical work on the fundamental\\simplification of physical concepts through\\energy-based formulations with Planck-scale reference}
}

\author{
	{\Large Johann Pascher}\\
	Department of Communication Technology\\
	Higher Technical Federal Institute (HTL), Leonding, Austria\\
	\texttt{johann.pascher@gmail.com}
}

\date{\today}



==================================================

=== T0_Feinstruktur_De.tex.preamble ===

\documentclass[12pt,a4paper]{article}
\usepackage[utf8]{inputenc}
\usepackage[T1]{fontenc}
\usepackage[ngerman]{babel}
\usepackage{lmodern}
\usepackage{amsmath,amssymb,amsthm}
\usepackage{geometry}
\usepackage{booktabs}
\usepackage{array}
\usepackage{xcolor}
\usepackage{tcolorbox}
\usepackage{fancyhdr}
\usepackage{tocloft}
\usepackage{hyperref}
\usepackage{tikz}
\usepackage{physics}
\usepackage{siunitx}
\usepackage{longtable}

\definecolor{deepblue}{RGB}{0,0,127}
\definecolor{deepred}{RGB}{191,0,0}
\definecolor{deepgreen}{RGB}{0,127,0}

\geometry{a4paper, margin=2.5cm}
\setlength{\headheight}{15pt}

\usetikzlibrary{positioning, arrows.meta}

% Header- und Footer-Konfiguration
\pagestyle{fancy}
\fancyhf{}
\fancyhead[L]{\textsc{T0-Theorie: Die Feinstrukturkonstante}}
\fancyhead[R]{\textsc{J. Pascher}}
\fancyfoot[C]{\thepage}
\renewcommand{\headrulewidth}{0.4pt}
\renewcommand{\footrulewidth}{0.4pt}

% Inhaltsverzeichnis-Stil - Blau
\renewcommand{\cfttoctitlefont}{\huge\bfseries\color{blue}}
\renewcommand{\cftsecfont}{\color{blue}}
\renewcommand{\cftsubsecfont}{\color{blue}}
\renewcommand{\cftsecpagefont}{\color{blue}}
\renewcommand{\cftsubsecpagefont}{\color{blue}}
\setlength{\cftsecindent}{0pt}
\setlength{\cftsubsecindent}{0pt}

% Hyperref-Einstellungen
\hypersetup{
	colorlinks=true,
	linkcolor=blue,
	citecolor=blue,
	urlcolor=blue,
	pdftitle={T0-Theorie: Die Feinstrukturkonstante},
	pdfauthor={Johann Pascher},
	pdfsubject={T0-Theorie, Feinstrukturkonstante, Geometrische Ableitung}
}

% Benutzerdefinierte Befehle
% Erklärungen zu den Symbolen (als Kommentare für Klarheit):
% \xipar: Der fundamentale geometrische Parameter ξ₀, der die fraktale Struktur der Raumzeit beschreibt. Wert: 4/3 × 10^{-4}.
% \Kfrak: Die fraktale Korrekturkonstante K_{frak}, die Quanteneffekte in der T0-Theorie berücksichtigt. Wert: ≈0.986.
% \Ezero: Die charakteristische Energie E₀, geometrisches Mittel der Leptonmassen. Wert: 7.398 MeV.
% \alphaem: Die Feinstrukturkonstante α, die die Stärke der elektromagnetischen Wechselwirkung misst. Wert: ≈1/137.
% \Dfrak: Die fraktale Dimension D_f, die die Abweichung von der euklidischen Raumzeit beschreibt. Wert: ≈2.94.
\newcommand{\xipar}{\xi_0}
\newcommand{\Kfrak}{K_{\text{frak}}}
\newcommand{\Ezero}{E_0}
\newcommand{\alphaem}{\alpha}
\newcommand{\Dfrak}{D_f}

% Umgebungen für besondere Inhalte (mit Erklärungen):
% keyresult: Blaue Box für zentrale Ergebnisse und Formeln.
% warning: Rote Box für wichtige Hinweise und Warnungen.
% alternative: Grüne Box für alternative Herleitungen.
% dimensional: Gelbe Box für Dimensionsanalysen (nicht verwendet, aber definiert).
% method: Violette Box für methodische Betrachtungen (nicht verwendet).
% foundation: Gelbe Box für fundamentale Prinzipien.
\newtcolorbox{keyresult}{colback=blue!5, colframe=blue!75!black, title=Schlüsselergebnis}
\newtcolorbox{warning}{colback=red!5, colframe=red!75!black, title=Wichtiger Hinweis}
\newtcolorbox{alternative}{colback=green!5, colframe=green!75!black, title=Alternative Herleitung}
\newtcolorbox{dimensional}{colback=yellow!5, colframe=orange!75!black, title=Dimensionsanalyse}
\newtcolorbox{method}{colback=purple!5, colframe=purple!75!black, title=Methodische Betrachtung}
\newtcolorbox{foundation}{colback=yellow!5, colframe=orange!75!black, title=Fundamentales Prinzip}

\title{\textbf{T0-Theorie: Die Feinstrukturkonstante}\\[0.5cm]
	\large Herleitung von $\alpha$ aus geometrischen Prinzipien\\[0.3cm]
	\normalsize Dokument 2 der T0-Serie}
\author{Johann Pascher\\
	Abteilung für Kommunikationstechnologie\\
	Höhere Technische Lehranstalt (HTL), Leonding, Österreich\\
	\texttt{johann.pascher@gmail.com}}
\date{\today}



==================================================

=== T0_Feinstruktur_En.tex.preamble ===

\documentclass[12pt,a4paper]{article}
\usepackage[utf8]{inputenc}
\usepackage[T1]{fontenc}
\usepackage[english]{babel}
\usepackage{lmodern}
\usepackage{amsmath,amssymb,amsthm}
\usepackage{geometry}
\usepackage{booktabs}
\usepackage{array}
\usepackage{xcolor}
\usepackage{tcolorbox}
\usepackage{fancyhdr}
\usepackage{tocloft}
\usepackage{hyperref}
\usepackage{tikz}
\usepackage{physics}
\usepackage{siunitx}
\usepackage{longtable}

\definecolor{deepblue}{RGB}{0,0,127}
\definecolor{deepred}{RGB}{191,0,0}
\definecolor{deepgreen}{RGB}{0,127,0}

\geometry{a4paper, margin=2.5cm}
\setlength{\headheight}{15pt}

\usetikzlibrary{positioning, arrows.meta}

% Header and Footer Configuration
\pagestyle{fancy}
\fancyhf{}
\fancyhead[L]{\textsc{T0 Theory: The Fine-Structure Constant}}
\fancyhead[R]{\textsc{J. Pascher}}
\fancyfoot[C]{\thepage}
\renewcommand{\headrulewidth}{0.4pt}
\renewcommand{\footrulewidth}{0.4pt}

% Table of Contents Style - Blue
\renewcommand{\cfttoctitlefont}{\huge\bfseries\color{blue}}
\renewcommand{\cftsecfont}{\color{blue}}
\renewcommand{\cftsubsecfont}{\color{blue}}
\renewcommand{\cftsecpagefont}{\color{blue}}
\renewcommand{\cftsubsecpagefont}{\color{blue}}
\setlength{\cftsecindent}{0pt}
\setlength{\cftsubsecindent}{0pt}

% Hyperref Settings
\hypersetup{
	colorlinks=true,
	linkcolor=blue,
	citecolor=blue,
	urlcolor=blue,
	pdftitle={T0 Theory: The Fine-Structure Constant},
	pdfauthor={Johann Pascher},
	pdfsubject={T0 Theory, Fine-Structure Constant, Geometric Derivation}
}

% User-Defined Commands
% Explanations of the symbols (as comments for clarity):
% \xipar: The fundamental geometric parameter ξ₀, which describes the fractal structure of spacetime. Value: 4/3 × 10^{-4}.
% \Kfrak: The fractal correction constant K_{frak}, which accounts for quantum effects in the T0 Theory. Value: ≈0.986.
% \Ezero: The characteristic energy E₀, geometric mean of the lepton masses. Value: 7.398 MeV.
% \alphaem: The fine-structure constant α, which measures the strength of the electromagnetic interaction. Value: ≈1/137.
% \Dfrak: The fractal dimension D_f, which describes the deviation from Euclidean spacetime. Value: ≈2.94.
\newcommand{\xipar}{\xi_0}
\newcommand{\Kfrak}{K_{\text{frak}}}
\newcommand{\Ezero}{E_0}
\newcommand{\alphaem}{\alpha}
\newcommand{\Dfrak}{D_f}

% Environments for special content (with explanations):
% keyresult: Blue box for central results and formulas.
% warning: Red box for important notes and warnings.
% alternative: Green box for alternative derivations.
% dimensional: Yellow box for dimensional analyses (not used, but defined).
% method: Violet box for methodological considerations (not used).
% foundation: Yellow box for fundamental principles.
\newtcolorbox{keyresult}{colback=blue!5, colframe=blue!75!black, title=Key Result}
\newtcolorbox{warning}{colback=red!5, colframe=red!75!black, title=Important Note}
\newtcolorbox{alternative}{colback=green!5, colframe=green!75!black, title=Alternative Derivation}
\newtcolorbox{dimensional}{colback=yellow!5, colframe=orange!75!black, title=Dimensional Analysis}
\newtcolorbox{method}{colback=purple!5, colframe=purple!75!black, title=Methodological Consideration}
\newtcolorbox{foundation}{colback=yellow!5, colframe=orange!75!black, title=Fundamental Principle}

\title{\textbf{T0 Theory: The Fine-Structure Constant}\\[0.5cm]
	\large Derivation of $\alpha$ from Geometric Principles\\[0.3cm]
	\normalsize Document 2 of the T0 Series}
\author{Johann Pascher\\
	Department of Communication Technology\\
	Higher Technical College (HTL), Leonding, Austria\\
	\texttt{johann.pascher@gmail.com}}
\date{\today}



==================================================

=== T0_Geometrische_Kosmologie_De.tex.preamble ===

\documentclass[12pt,a4paper]{article}

% --- Grundlegende Pakete ---
\usepackage[utf8]{inputenc}
\usepackage[T1]{fontenc}
\usepackage[ngerman]{babel}
\usepackage{lmodern}
\usepackage{amsmath,amssymb,amsthm}
\usepackage{physics}
\usepackage{siunitx}
\usepackage{listings}
\usepackage{xcolor} % Benötigt für Farben in Listings

% --- Seitenlayout und Design ---
\usepackage[margin=2.5cm]{geometry}
\usepackage{fancyhdr}
\usepackage{hyperref}
\usepackage{graphicx}
\usepackage{booktabs}
\usepackage{enumitem}

% --- Python Code Listing Konfiguration (FINAL KORRIGIERT) ---
\lstset{
	language=Python,
	basicstyle=\ttfamily\small,
	keywordstyle=\color{blue},
	stringstyle=\color{red},
	commentstyle=\color{green!50!black},
	comment=[l]{\#}, % Explizite Definition für Python-Kommentare
	showstringspaces=false,
	breaklines=true,
	captionpos=b,
	frame=single,
	numbers=left,
	numberstyle=\tiny\color{gray},
	% Sichere Behandlung problematischer Zeichen
	literate={_}{\textunderscore}1 
	{ä}{{\"a}}1 {ö}{{\"o}}1 {ü}{{\"u}}1 {ß}{{\ss}}1
}

% --- Hyperref-Konfiguration ---
\hypersetup{
	colorlinks=true,
	linkcolor=blue,
	citecolor=blue,
	urlcolor=blue,
	pdftitle={T0-Kosmologie: Rotverschiebung als geometrischer Pfad-Effekt im statischen Universum},
	pdfauthor={Johann Pascher},
	pdfsubject={T0-Theorie, Kosmologie, Rotverschiebung, Finite-Elemente-Methode}
}

% --- Kopf- und Fußzeile ---
\pagestyle{fancy}
\fancyhf{}
\fancyhead[L]{\textsc{T0-Theorie: Geometrische Kosmologie}}
\fancyhead[R]{\textsc{J. Pascher}}
\fancyfoot[C]{\thepage}
\renewcommand{\headrulewidth}{0.4pt}
\setlength{\headheight}{15pt}

% --- Mathematische Befehle ---
\newcommand{\xiT}{\xi}
\newcommand{\Df}{D_f}
\newcommand{\Leff}{L_{\text{eff}}}
\newcommand{\Hubble}{H_0}

% --- Titel-Informationen ---
\title{\textbf{T0-Kosmologie: Rotverschiebung als geometrischer Pfad-Effekt in einem statischen Universum}\\[0.5cm]
	\large Eine numerische Herleitung der Hubble-Konstante mittels Finite-Elemente-Simulation des T0-Vakuums}
\author{Johann Pascher}
\date{2025-11-09 16:23:46 UTC}



==================================================

=== T0_Geometrische_Kosmologie_En.tex.preamble ===

\documentclass[12pt,a4paper]{article}

% --- Basic Packages ---
\usepackage[utf8]{inputenc}
\usepackage[T1]{fontenc}
\usepackage[english]{babel}
\usepackage{lmodern}
\usepackage{amsmath,amssymb,amsthm}
\usepackage{physics}
\usepackage{siunitx}
\usepackage{listings}
\usepackage{xcolor} % Required for colors in listings

% --- Page Layout and Design ---
\usepackage[margin=2.5cm]{geometry}
\usepackage{fancyhdr}
\usepackage{hyperref}
\usepackage{graphicx}
\usepackage{booktabs}
\usepackage{enumitem}

% --- Python Code Listing Configuration (CORRECTED) ---
\lstset{
	language=Python,
	basicstyle=\ttfamily\small,
	keywordstyle=\color{blue},
	stringstyle=\color{red},
	commentstyle=\color{green!50!black},
	comment=[l]{\#}, % Explicit definition for Python comments
	showstringspaces=false,
	breaklines=true,
	captionpos=b,
	frame=single,
	numbers=left,
	numberstyle=\tiny\color{gray},
	% Safe handling of problematic characters
	literate={_}{\textunderscore}1
}

% --- Hyperref Configuration ---
\hypersetup{
	colorlinks=true,
	linkcolor=blue,
	citecolor=blue,
	urlcolor=blue,
	pdftitle={T0 Cosmology: Redshift as a Geometric Path Effect in a Static Universe},
	pdfauthor={Johann Pascher},
	pdfsubject={T0-Theory, Cosmology, Redshift, Finite Element Method}
}

% --- Header and Footer ---
\pagestyle{fancy}
\fancyhf{}
\fancyhead[L]{\textsc{T0-Theory: Geometric Cosmology}}
\fancyhead[R]{\textsc{J. Pascher}}
\fancyfoot[C]{\thepage}
\renewcommand{\headrulewidth}{0.4pt}
\setlength{\headheight}{15pt}

% --- Mathematical Commands ---
\newcommand{\xiT}{\xi}
\newcommand{\Df}{D_f}
\newcommand{\Leff}{L_{\text{eff}}}
\newcommand{\Hubble}{H_0}

% --- Title Information ---
\title{\textbf{T0 Cosmology: Redshift as a Geometric Path Effect in a Static Universe}\\[0.5cm]
	\large A Numerical Derivation of the Hubble Constant via Finite Element Simulation of the T0 Vacuum}
\author{Johann Pascher}
\date{2025-11-09 16:23:46 UTC}



==================================================

=== T0_Gravitationskonstante_De.tex.preamble ===

\documentclass[12pt,a4paper]{article}
\usepackage[utf8]{inputenc}
\usepackage[T1]{fontenc}
\usepackage[ngerman]{babel}
\usepackage{lmodern}
\usepackage{amsmath,amssymb,amsthm}
\usepackage{geometry}
\usepackage{booktabs}
\usepackage{array}
\usepackage{xcolor}
\usepackage{tcolorbox}
\usepackage{fancyhdr}
\usepackage{tocloft}
\usepackage{hyperref}
\usepackage{tikz}
\usepackage{physics}
\usepackage{siunitx}

\definecolor{deepblue}{RGB}{0,0,127}
\definecolor{deepred}{RGB}{191,0,0}
\definecolor{deepgreen}{RGB}{0,127,0}

\geometry{a4paper, margin=2.5cm}

\usetikzlibrary{positioning, arrows.meta}

% Header- und Footer-Konfiguration
\pagestyle{fancy}
\fancyhf{}
\fancyhead[L]{\textsc{T0-Theorie: Die Gravitationskonstante}}
\fancyhead[R]{\textsc{J. Pascher}}
\fancyfoot[C]{\thepage}
\renewcommand{\headrulewidth}{0.4pt}
\renewcommand{\footrulewidth}{0.4pt}

% Fix head height warning
\setlength{\headheight}{14.5pt}

% Inhaltsverzeichnis-Stil - Blau
\renewcommand{\cfttoctitlefont}{\huge\bfseries\color{blue}}
\renewcommand{\cftsecfont}{\color{blue}}
\renewcommand{\cftsubsecfont}{\color{blue}}
\renewcommand{\cftsecpagefont}{\color{blue}}
\renewcommand{\cftsubsecpagefont}{\color{blue}}
\setlength{\cftsecindent}{0pt}
\setlength{\cftsubsecindent}{0pt}

% Hyperref-Einstellungen
\hypersetup{
	colorlinks=true,
	linkcolor=blue,
	citecolor=blue,
	urlcolor=blue,
	pdftitle={T0-Theorie: Die Gravitationskonstante},
	pdfauthor={Johann Pascher},
	pdfsubject={T0-Theorie, Gravitationskonstante, Geometrische Ableitung}
}

% Benutzerdefinierte Befehle
\newcommand{\xipar}{\xi_0}
\newcommand{\Kfrak}{K_{\text{frak}}}
\newcommand{\Cconv}{C_{\text{conv}}}
\newcommand{\Gsi}{G_{\text{SI}}}
\newcommand{\Gnat}{G_{\text{nat}}}

% Umgebung für Schlüsselergebnisse
\newtcolorbox{keyresult}{colback=blue!5, colframe=blue!75!black, title=Schlüsselergebnis}
\newtcolorbox{warning}{colback=red!5, colframe=red!75!black, title=Wichtiger Hinweis}
\newtcolorbox{derivation}{colback=green!5, colframe=green!75!black, title=Herleitung}
\newtcolorbox{dimensional}{colback=yellow!5, colframe=orange!75!black, title=Dimensionsanalyse}
\newtcolorbox{verification}{colback=purple!5, colframe=purple!75!black, title=Experimentelle Verifikation}

\title{\textbf{T0-Theorie: Die Gravitationskonstante}\\[0.5cm]
	\large Systematische Herleitung von $G$ aus geometrischen Prinzipien\\[0.3cm]
	\normalsize Dokument 3 der T0-Serie}
\author{Johann Pascher\\
	Abteilung für Kommunikationstechnologie\\
	Höhere Technische Lehranstalt (HTL), Leonding, Österreich\\
	\texttt{johann.pascher@gmail.com}}
\date{\today}



==================================================

=== T0_Gravitationskonstante_En.tex.preamble ===

\documentclass[12pt,a4paper]{article}
\usepackage[utf8]{inputenc}
\usepackage[T1]{fontenc}
\usepackage{amsmath,amssymb,amsthm}
\usepackage{geometry}
\usepackage{booktabs}
\usepackage{array}
\usepackage{xcolor}
\usepackage{tcolorbox}
\usepackage{fancyhdr}
\usepackage{tocloft}
\usepackage{hyperref}
\usepackage{tikz}
\usepackage{physics}
\usepackage{siunitx}

\definecolor{deepblue}{RGB}{0,0,127}
\definecolor{deepred}{RGB}{191,0,0}
\definecolor{deepgreen}{RGB}{0,127,0}

\geometry{a4paper, margin=2.5cm}

\usetikzlibrary{positioning, arrows.meta}

% Header and footer configuration
\pagestyle{fancy}
\fancyhf{}
\fancyhead[L]{\textsc{T0 Theory: The Gravitational Constant}}
\fancyhead[R]{\textsc{J. Pascher}}
\fancyfoot[C]{\thepage}
\renewcommand{\headrulewidth}{0.4pt}
\renewcommand{\footrulewidth}{0.4pt}

% Fix head height warning
\setlength{\headheight}{14.5pt}

% Table of contents style
\renewcommand{\cfttoctitlefont}{\huge\bfseries\color{blue}}
\renewcommand{\cftsecfont}{\color{blue}}
\renewcommand{\cftsubsecfont}{\color{blue}}
\renewcommand{\cftsecpagefont}{\color{blue}}
\renewcommand{\cftsubsecpagefont}{\color{blue}}
\setlength{\cftsecindent}{0pt}
\setlength{\cftsubsecindent}{0pt}

% Hyperref settings
\hypersetup{
	colorlinks=true,
	linkcolor=blue,
	citecolor=blue,
	urlcolor=blue,
	pdftitle={T0 Theory: The Gravitational Constant},
	pdfauthor={Johann Pascher},
	pdfsubject={T0 Theory, Gravitational Constant, Geometric Derivation}
}

% Custom commands
\newcommand{\xipar}{\xi_0}
\newcommand{\Kfrak}{K_{\text{frak}}}
\newcommand{\Cconv}{C_{\text{conv}}}
\newcommand{\Gsi}{G_{\text{SI}}}
\newcommand{\Gnat}{G_{\text{nat}}}

% Environments for key results
\newtcolorbox{keyresult}{colback=blue!5, colframe=blue!75!black, title=Key Result}
\newtcolorbox{warning}{colback=red!5, colframe=red!75!black, title=Important Note}
\newtcolorbox{derivation}{colback=green!5, colframe=green!75!black, title=Derivation}
\newtcolorbox{dimensional}{colback=yellow!5, colframe=orange!75!black, title=Dimensional Analysis}
\newtcolorbox{verification}{colback=purple!5, colframe=purple!75!black, title=Experimental Verification}

\title{\textbf{T0 Theory: The Gravitational Constant}\\[0.5cm]
	\large Systematic Derivation of $G$ from Geometric Principles\\[0.3cm]
	\normalsize Document 3 of the T0 Series}
\author{Johann Pascher\\
	Department of Communication Technology\\
	Higher Technical Institute (HTL), Leonding, Austria\\
	\texttt{johann.pascher@gmail.com}}
\date{\today}



==================================================

=== T0_Grundlagen_De.tex.preamble ===

\documentclass[12pt,a4paper]{article}
\usepackage[utf8]{inputenc}
\usepackage[T1]{fontenc}
\usepackage[ngerman]{babel}
\usepackage{lmodern}
\usepackage{amsmath,amssymb,amsthm}
\usepackage{geometry}
\usepackage{booktabs}
\usepackage{array}
\usepackage{xcolor}
\usepackage{tcolorbox}
\usepackage{fancyhdr}
\usepackage{tocloft}
\usepackage{hyperref}
\usepackage{tikz}
\usepackage{physics}
\usepackage{siunitx}

\definecolor{deepblue}{RGB}{0,0,127}
\definecolor{deepred}{RGB}{191,0,0}
\definecolor{deepgreen}{RGB}{0,127,0}

\geometry{a4paper, margin=2.5cm}
\setlength{\headheight}{15pt}

\usetikzlibrary{positioning, arrows.meta}

% Header- und Footer-Konfiguration
\pagestyle{fancy}
\fancyhf{}
\fancyhead[L]{\textsc{T0-Theorie: Fundamentale Prinzipien}}
\fancyhead[R]{\textsc{J. Pascher}}
\fancyfoot[C]{\thepage}
\renewcommand{\headrulewidth}{0.4pt}
\renewcommand{\footrulewidth}{0.4pt}

% Inhaltsverzeichnis-Stil - Blau
\renewcommand{\cfttoctitlefont}{\huge\bfseries\color{blue}}
\renewcommand{\cftsecfont}{\color{blue}}
\renewcommand{\cftsubsecfont}{\color{blue}}
\renewcommand{\cftsecpagefont}{\color{blue}}
\renewcommand{\cftsubsecpagefont}{\color{blue}}
\setlength{\cftsecindent}{0pt}
\setlength{\cftsubsecindent}{0pt}

% Hyperref-Einstellungen
\hypersetup{
	colorlinks=true,
	linkcolor=blue,
	citecolor=blue,
	urlcolor=blue,
	pdftitle={T0-Theorie: Fundamentale Prinzipien},
	pdfauthor={Johann Pascher},
	pdfsubject={T0-Theorie, Geometrische Physik, Fundamentale Konstanten}
}

% Benutzerdefinierte Befehle
\newcommand{\xipar}{\xi}
\newcommand{\Kfrak}{K_{\text{frak}}}
\newcommand{\Ezero}{E_0}
\newcommand{\Dfrak}{D_f}

% Umgebung für Schlüsselergebnisse
\newtcolorbox{keyresult}{colback=blue!5, colframe=blue!75!black, title=Schlüsselergebnis}
\newtcolorbox{warning}{colback=red!5, colframe=red!75!black, title=Wichtiger Hinweis}
\newtcolorbox{alternative}{colback=green!5, colframe=green!75!black, title=Alternative Sichtweise}
\newtcolorbox{foundation}{colback=yellow!5, colframe=orange!75!black, title=Fundamentales Prinzip}

\title{\textbf{T0-Theorie: Fundamentale Prinzipien}\\[0.5cm]
	\large Die geometrischen Grundlagen der Physik\\[0.3cm]
	\normalsize Dokument 1 der T0-Serie}
\author{Johann Pascher\\
	Abteilung für Kommunikationstechnologie\\
	Höhere Technische Lehranstalt (HTL), Leonding, Österreich\\
	\texttt{johann.pascher@gmail.com}}
\date{\today}



==================================================

=== T0_Grundlagen_En.tex.preamble ===

\documentclass[12pt,a4paper]{article}
\usepackage[utf8]{inputenc}
\usepackage[T1]{fontenc}
\usepackage[english]{babel}
\usepackage{lmodern}
\usepackage{amsmath,amssymb,amsthm}
\usepackage{geometry}
\usepackage{booktabs}
\usepackage{array}
\usepackage{xcolor}
\usepackage{tcolorbox}
\usepackage{fancyhdr}
\usepackage{tocloft}
\usepackage{hyperref}
\usepackage{tikz}
\usepackage{physics}
\usepackage{siunitx}

\definecolor{deepblue}{RGB}{0,0,127}
\definecolor{deepred}{RGB}{191,0,0}
\definecolor{deepgreen}{RGB}{0,127,0}

\geometry{a4paper, margin=2.5cm}
\setlength{\headheight}{15pt}

\usetikzlibrary{positioning, arrows.meta}

% Header and Footer Configuration
\pagestyle{fancy}
\fancyhf{}
\fancyhead[L]{\textsc{T0-Theory: Fundamental Principles}}
\fancyhead[R]{\textsc{J. Pascher}}
\fancyfoot[C]{\thepage}
\renewcommand{\headrulewidth}{0.4pt}
\renewcommand{\footrulewidth}{0.4pt}

% Table of Contents Style - Blue
\renewcommand{\cfttoctitlefont}{\huge\bfseries\color{blue}}
\renewcommand{\cftsecfont}{\color{blue}}
\renewcommand{\cftsubsecfont}{\color{blue}}
\renewcommand{\cftsecpagefont}{\color{blue}}
\renewcommand{\cftsubsecpagefont}{\color{blue}}
\setlength{\cftsecindent}{0pt}
\setlength{\cftsubsecindent}{0pt}

% Hyperref Settings
\hypersetup{
	colorlinks=true,
	linkcolor=blue,
	citecolor=blue,
	urlcolor=blue,
	pdftitle={T0-Theory: Fundamental Principles},
	pdfauthor={Johann Pascher},
	pdfsubject={T0-Theory, Geometric Physics, Fundamental Constants}
}

% User-Defined Commands
\newcommand{\xipar}{\xi}
\newcommand{\Kfrak}{K_{\text{frak}}}
\newcommand{\Ezero}{E_0}
\newcommand{\Dfrak}{D_f}

% Environment for Key Results
\newtcolorbox{keyresult}{colback=blue!5, colframe=blue!75!black, title=Key Result}
\newtcolorbox{warning}{colback=red!5, colframe=red!75!black, title=Important Note}
\newtcolorbox{alternative}{colback=green!5, colframe=green!75!black, title=Alternative Perspective}
\newtcolorbox{foundation}{colback=yellow!5, colframe=orange!75!black, title=Fundamental Principle}

\title{\textbf{T0-Theory: Fundamental Principles}\\[0.5cm]
	\large The Geometric Foundations of Physics\\[0.3cm]
	\normalsize Document 1 of the T0 Series}
\author{Johann Pascher\\
	Department for Communication Technology\\
	Higher Technical College (HTL), Leonding, Austria\\
	\texttt{johann.pascher@gmail.com}}
\date{\today}



==================================================

=== T0_Introduction_En.tex.preamble ===

\documentclass[11pt,a4paper,openany]{book}

\usepackage[a4paper,margin=2cm]{geometry}
\usepackage[utf8]{inputenc}
\usepackage[english]{babel}
\usepackage{lmodern}
\renewcommand{\familydefault}{\sfdefault}

\usepackage{amsmath,amssymb,amsthm}
\usepackage{graphicx}
\usepackage[unicode,pdfencoding=auto]{hyperref}
\usepackage{booktabs}
\usepackage{longtable}
\usepackage{siunitx}
\usepackage{fancyhdr}
\usepackage{float}
\usepackage{tikz}
\usepackage{setspace}
\usepackage{enumitem}
\usepackage{adjustbox}

\setlength{\parindent}{0pt}
\setlength{\parskip}{6pt}

\hypersetup{
	colorlinks=true,
	linkcolor=blue,  % TOC-Links blau
	citecolor=blue,
	urlcolor=blue
}

\pagestyle{fancy}

\title{T0 Time--Mass Duality\\Unified English Book}
\author{J. Pascher}
\date{\today}



==================================================

