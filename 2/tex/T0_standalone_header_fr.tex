% French Standalone Document Header
\documentclass[12pt,a4paper]{article}
\usepackage[utf8]{inputenc}
\usepackage[T1]{fontenc}
\usepackage[french]{babel}
\usepackage{lmodern}
\usepackage{amsmath,amssymb,amsthm}
\usepackage{physics}
\usepackage{siunitx}
\usepackage{geometry}
\geometry{margin=2.5cm}
\usepackage{fancyhdr}
\usepackage{titlesec}
\usepackage{booktabs}
\usepackage{longtable}
\usepackage{graphicx}
\usepackage{tikz}
\usepackage{hyperref}
\usepackage{cleveref}
\usepackage{xcolor}
\usepackage{tcolorbox}

% Custom commands
\newcommand{\Tfield}{T(x,t)}
\newcommand{\xipar}{\xi}

% French quote style
\frenchbsetup{StandardLists=true}

% tcolorbox environments
\newtcolorbox{insight}[1][]{colback=blue!5,colframe=blue!75!black,title=Aperçu,#1}
\newtcolorbox{discovery}[1][]{colback=green!5,colframe=green!75!black,title=Découverte,#1}
\newtcolorbox{keypoint}[1][]{colback=red!5,colframe=red!75!black,title=Point Clé,#1}
\newtcolorbox{conclusion}[1][]{colback=gray!5,colframe=gray!75!black,title=Conclusion,#1}
\newtcolorbox{significance}[1][]{colback=yellow!5,colframe=yellow!75!black,title=Signification,#1}
\newtcolorbox{philosophical}[1][]{colback=purple!5,colframe=purple!75!black,title=Réflexion Philosophique,#1}
\newtcolorbox{implication}[1][]{colback=orange!5,colframe=orange!75!black,title=Implication,#1}
\newtcolorbox{newperspective}[1][]{colback=cyan!5,colframe=cyan!75!black,title=Nouvelle Perspective,#1}
\newtcolorbox{revelation}[1][]{colback=magenta!5,colframe=magenta!75!black,title=Révélation,#1}
\newtcolorbox{evidence}[1][]{colback=teal!5,colframe=teal!75!black,title=Preuve,#1}
\newtcolorbox{perspective}[1][]{colback=lime!5,colframe=lime!75!black,title=Perspective,#1}
\newtcolorbox{revolutionary}[1][]{colback=pink!5,colframe=pink!75!black,title=Révolutionnaire,#1}

% Theorem environments
\newtheorem{theorem}{Théorème}
\newtheorem{lemma}{Lemme}
\newtheorem{proposition}{Proposition}
\newtheorem{corollary}{Corollaire}
\theoremstyle{definition}
\newtheorem{definition}{Définition}
\newtheorem{example}{Exemple}
\theoremstyle{remark}
\newtheorem{remark}{Remarque}
