% T0 Standalone Header für Französisch
\documentclass[12pt,a4paper]{report}

% Encoding und Sprache
\usepackage[utf8]{inputenc}
\usepackage[T1]{fontenc}
\usepackage[french]{babel}

% Mathematik
\usepackage{amsmath,amssymb,amsfonts}
\usepackage{mathtools}
\usepackage{physics}

% Graphik und Farben
\usepackage{graphicx}
\usepackage{xcolor}
\usepackage{tikz}
\usetikzlibrary{shapes,arrows,positioning,calc}

% Tabellen
\usepackage{booktabs}
\usepackage{longtable}
\usepackage{array}
\usepackage{multirow}

% Layout
\usepackage{geometry}
\geometry{margin=2.5cm}
\usepackage{fancyhdr}
\usepackage{setspace}

% Hyperlinks
\usepackage{hyperref}
\hypersetup{
    colorlinks=true,
    linkcolor=blue,
    filecolor=magenta,      
    urlcolor=cyan,
    pdftitle={T0 Theory},
    pdfauthor={Johann Pascher},
}

% Boxen für Hervorhebungen
\usepackage[most]{tcolorbox}

% Französische Anführungszeichen
\newcommand{\fq}[1]{\og{}#1\fg{}}

% Definiere tcolorbox Umgebungen
\newtcolorbox{insight}[1][]{
    colback=blue!5,
    colframe=blue!75!black,
    title=#1,
    fonttitle=\bfseries
}

\newtcolorbox{discovery}[1][]{
    colback=green!5,
    colframe=green!75!black,
    title=#1,
    fonttitle=\bfseries
}

\newtcolorbox{newperspective}[1][]{
    colback=orange!5,
    colframe=orange!75!black,
    title=#1,
    fonttitle=\bfseries
}

\newtcolorbox{revelation}[1][]{
    colback=purple!5,
    colframe=purple!75!black,
    title=#1,
    fonttitle=\bfseries
}

\newtcolorbox{keypoint}[1][]{
    colback=red!5,
    colframe=red!75!black,
    title=#1,
    fonttitle=\bfseries
}

\newtcolorbox{evidence}[1][]{
    colback=teal!5,
    colframe=teal!75!black,
    title=#1,
    fonttitle=\bfseries
}

\newtcolorbox{conclusion}[1][]{
    colback=gray!5,
    colframe=gray!75!black,
    title=#1,
    fonttitle=\bfseries
}

\newtcolorbox{significance}[1][]{
    colback=yellow!5,
    colframe=yellow!75!black,
    title=#1,
    fonttitle=\bfseries
}

\newtcolorbox{philosophical}[1][]{
    colback=brown!5,
    colframe=brown!75!black,
    title=#1,
    fonttitle=\bfseries
}

\newtcolorbox{implication}[1][]{
    colback=cyan!5,
    colframe=cyan!75!black,
    title=#1,
    fonttitle=\bfseries
}

\newtcolorbox{perspective}[1][]{
    colback=magenta!5,
    colframe=magenta!75!black,
    title=#1,
    fonttitle=\bfseries
}

\newtcolorbox{revolutionary}[1][]{
    colback=red!10,
    colframe=red!90!black,
    title=#1,
    fonttitle=\bfseries
}

% Mathematische Befehle
\newcommand{\Etau}{E_\tau}
\newcommand{\mtau}{m_\tau}
\newcommand{\mmu}{m_\mu}
\providecommand{\mel}{m_e}
\renewcommand{\mel}{m_e}

% Resizebox für Tabellen
\usepackage{graphicx}
