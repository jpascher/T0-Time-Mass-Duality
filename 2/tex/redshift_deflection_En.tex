\documentclass[12pt,a4paper]{article}
\usepackage[utf8]{inputenc}
\usepackage{amsmath,amssymb,amsfonts,amsthm}
\usepackage{physics}
\usepackage{siunitx}
\usepackage{geometry}
\usepackage{fancyhdr}
\usepackage{enumitem}
\usepackage{booktabs}
\usepackage{longtable}
\usepackage{array}
\usepackage{xcolor}
\usepackage{tcolorbox}
\usepackage{mdframed}
\usepackage{graphicx}
\usepackage{hyperref}

\geometry{margin=2.5cm}
\pagestyle{fancy}
\fancyhf{}
\fancyhead[L]{T0-Theory: Mathematical Equivalence Formulation}
\fancyhead[R]{\thepage}
\fancyfoot[C]{\textit{Energy Loss, Redshift, and Light Deflection Unified}}

\hypersetup{
	colorlinks=true,
	linkcolor=blue,
	filecolor=magenta,
	urlcolor=cyan,
}

\newcommand{\ts}{\textsuperscript}
\newcommand{\xired}{\xi_{\text{red}}}
\newcommand{\ee}{\text{$\mathrm{e}$}}
\newcommand{\mmu}{\text{$\mu$}}
\newcommand{\ttau}{\text{$\tau$}}
\newcommand{\tfield}{T_{\text{field}}}
\newcommand{\efield}{E_{\text{field}}}
\newcommand{\dfield}{\delta E}
\newcommand{\echar}{E_{\text{char}}}
\newcommand{\eratio}[2]{\frac{E_{#1}}{E_{#2}}}
\newcommand{\T}[1]{\text{#1}}
\newcommand{\vektor}[1]{\vec{#1}}
\newcommand{\dimcheck}[1]{\textcolor{blue}{[#1]}}
\newcommand{\lp}{\ell_{\text{P}}}
\newcommand{\ep}{E_{\text{P}}}
\newcommand{\alphae}{\alpha_{\text{EM}}}
\newcommand{\alphag}{\alpha_{\text{G}}}
\newcommand{\alphaw}{\alpha_{\text{W}}}
\newcommand{\alphas}{\alpha_{\text{S}}}
\newcommand{\xisi}{\xi_{\text{SI}}}
\newcommand{\xit}{\xi_{\text{T0}}}
\newcommand{\epst}{\varepsilon_{\text{T0}}}

\newmdenv[
linecolor=black,
frametitle={Dimensional Analysis:},
frametitlebackgroundcolor=gray!20,
backgroundcolor=gray!5,
]{dimanalysis}

\newtcolorbox{important}[1][]{
	colback=yellow!10!white,
	colframe=yellow!50!black,
	fonttitle=\bfseries,
	title=Important Note,
	#1
}

\newtcolorbox{formula}[1][]{
	colback=blue!5!white,
	colframe=blue!75!black,
	fonttitle=\bfseries,
	title=Key Formula,
	#1
}

\theoremstyle{definition}
\newtheorem{prinzip}{Principle}
\newtheorem{beobachtung}{Observation}

\title{\Huge\textbf{Mathematical Equivalence in T0-Theory}\\\Large Unified Description of Energy Loss, Redshift, and Light Deflection}
\author{Based on the work of Johann Pascher\\
	Department of Communications Engineering, \\Höhere Technische Bundeslehranstalt (HTL), Leonding, Austria}
\date{\today}

\begin{document}
	
	\maketitle
	\tableofcontents
	\thispagestyle{fancy}
	\newpage
	
	\section{Introduction}
	
	This document presents the mathematical equivalence of three phenomena that are treated as separate effects in standard physics but are unified in the T0-model:
	
	\begin{enumerate}
		\item Energy loss of photons during propagation
		\item Cosmological redshift
		\item Gravitational light deflection
	\end{enumerate}
	
	The central insight of T0-Theory is that these phenomena are different manifestations of the same underlying field equation, not separate physical processes. This unification is achieved through a single geometric parameter $\xi = \frac{4}{3} \times 10^{-4}$ that determines the coupling between the energy field and spacetime geometry.
	
	\section{Basic Formulas}
	
	\subsection{Energy Loss of Photons}
	
	\begin{formula}
		Energy loss rate:
		\begin{equation}
			\boxed{\frac{dE_\gamma}{dr} = -\xi \frac{E_\gamma^2}{\efield \cdot r}}
		\end{equation}
		where $\xi = \frac{4}{3} \times 10^{-4}$ is the universal geometric parameter.
	\end{formula}
	
	\begin{dimanalysis}
		$\left[\frac{dE_\gamma}{dr}\right] = \frac{[E]}{[L]} = \frac{[E]}{[E^{-1}]} = [E^2]$\\
		$[\xi] = [1]$ (dimensionless)\\
		$\left[\frac{E_\gamma^2}{\efield \cdot r}\right] = \frac{[E^2]}{[E] \cdot [E^{-1}]} = \frac{[E^2]}{[1]} = [E^2]$ \checkmark
	\end{dimanalysis}
	
	Since $E_\gamma = \frac{hc}{\lambda}$ (or $E_\gamma = \frac{1}{\lambda}$ in natural units), this can be expressed in terms of wavelength:
	
	\begin{equation}
		\frac{d(1/\lambda)}{dr} = -\xi \frac{(1/\lambda)^2}{\efield \cdot r}
	\end{equation}
	
	Rearranging:
	\begin{equation}
		\frac{d\lambda}{dr} = \xi \frac{\lambda^2 \cdot \efield}{r}
	\end{equation}
	
	Integrating the wavelength-dependent energy loss equation:
	\begin{equation}
		\int_{\lambda_0}^{\lambda(r)} \frac{d\lambda'}{\lambda'^2} = \xi \efield \int_0^r \frac{dr'}{r'}
	\end{equation}
	
	This yields:
	\begin{equation}
		\frac{1}{\lambda_0} - \frac{1}{\lambda(r)} = \xi \efield \ln\left(\frac{r}{r_0}\right)
	\end{equation}
	
	For small corrections:
	\begin{equation}
		\lambda(r) \approx \lambda_0 \left(1 + \xi \efield \lambda_0 \ln\left(\frac{r}{r_0}\right)\right)
	\end{equation}
	
	\subsection{Redshift Formulation}
	
	The redshift is defined as:
	\begin{equation}
		z = \frac{\lambda_{\text{observed}} - \lambda_{\text{emitted}}}{\lambda_{\text{emitted}}} = \frac{\lambda(r) - \lambda_0}{\lambda_0}
	\end{equation}
	
	Using the previously derived expression:
	\begin{equation}
		z \approx \xi \efield \lambda_0 \ln\left(\frac{r}{r_0}\right)
	\end{equation}
	
	Since $\lambda_0 \propto \frac{1}{E_{\gamma,0}}$, we can write:
	
	\begin{formula}
		Wavelength-dependent redshift:
		\begin{equation}
			\boxed{z(\lambda) = z_0\left(1 - \alpha \ln\frac{\lambda}{\lambda_0}\right)}
		\end{equation}
		where $z_0$ is the reference redshift and $\alpha$ is a dimensionless parameter related to $\xi$.
	\end{formula}
	
	\begin{dimanalysis}
		$[z(\lambda)] = [1]$\\
		$[z_0] = [1]$\\
		$[\alpha] = [1]$\\
		$\left[\ln\frac{\lambda}{\lambda_0}\right] = \ln\left(\frac{[L]}{[L]}\right) = \ln([1]) = [1]$\\
		$\left[z_0\left(1 - \alpha \ln\frac{\lambda}{\lambda_0}\right)\right] = [1] \cdot ([1] - [1] \cdot [1]) = [1]$ \checkmark
	\end{dimanalysis}
	
	A distinctive feature of this redshift formula is its wavelength dependence, which provides a testable prediction:
	
	\begin{equation}
		\frac{dz}{d\ln\lambda} = -\alpha z_0
	\end{equation}
	
	This distinguishes the T0 model from standard cosmological models that predict no wavelength dependence ($\frac{dz}{d\ln\lambda} = 0$).
	
	\subsection{Gravitational Light Deflection}
	
	\begin{formula}
		Modified gravitational deflection:
		\begin{equation}
			\boxed{\theta = \frac{4GM}{bc^2}\left(1 + \xi \frac{E_\gamma}{E_0}\right)}
		\end{equation}
		where $\theta$ is the deflection angle, $M$ is the mass of the deflecting object, $b$ is the impact parameter, $E_\gamma$ is the photon energy, and $E_0$ is a reference energy.
	\end{formula}
	
	\begin{dimanalysis}
		$[G] = [E^{-2}]$\\
		$[M] = [E]$\\
		$[b] = [E^{-1}]$\\
		$[c^2] = [1]$ (in natural units)\\
		$\left[\frac{4GM}{bc^2}\right] = \frac{[E^{-2}][E]}{[E^{-1}][1]} = [1]$ (dimensionless)\\
		$\left[\xi \frac{E_\gamma}{E_0}\right] = [1] \cdot \frac{[E]}{[E]} = [1]$ (dimensionless)\\
		$[\theta] = [1] \cdot ([1] + [1]) = [1]$ (dimensionless) \checkmark
	\end{dimanalysis}
	
	Unlike General Relativity, which predicts wavelength-independent light deflection, the T0-model introduces an explicit energy dependence. This energy-dependent gravitational lensing leads to a modified Einstein ring radius:
	
	\begin{equation}
		\theta_E(\lambda) = \theta_{E,0} \sqrt{1 + \xi \frac{hc}{\lambda E_0}}
	\end{equation}
	
	For two different photon energies, the ratio of deflection angles is:
	
	\begin{equation}
		\frac{\theta(E_1)}{\theta(E_2)} = \frac{1 + \xi \frac{E_1}{E_0}}{1 + \xi \frac{E_2}{E_0}}
	\end{equation}
	
	For cases where $\xi \frac{E}{E_0} \ll 1$ (typical for astrophysical observations), this can be approximated as:
	
	\begin{equation}
		\frac{\theta(E_1)}{\theta(E_2)} \approx 1 + \xi \frac{E_1 - E_2}{E_0}
	\end{equation}
	
	\section{Unifying Geodesic Equation}
	
	The three phenomena described above (energy loss, redshift, and light deflection) are unified in the T0-model through a single geodesic equation with time field corrections:
	
	\begin{formula}
		Universal geodesic equation:
		\begin{equation}
			\boxed{\frac{d^2 x^\mu}{d\lambda^2} + \Gamma^\mu_{\alpha\beta}\frac{dx^\alpha}{d\lambda}\frac{dx^\beta}{d\lambda} = \xi \cdot \partial^\mu \ln(\efield)}
		\end{equation}
		where $x^\mu$ is the spacetime position, $\lambda$ is an affine parameter along the photon path, $\Gamma^\mu_{\alpha\beta}$ are the Christoffel symbols, and $\efield$ is the local energy field.
	\end{formula}
	
	\begin{dimanalysis}
		$[\Gamma^\mu_{\alpha\beta}] = [E]$ (Christoffel symbols)\\
		$\left[\frac{dx^\alpha}{d\lambda}\right] = \frac{[E^{-1}]}{[E^{-1}]} = [1]$ (dimensionless)\\
		$[\partial^\mu \ln(\efield)] = [E] \cdot [1] = [E]$\\
		$[\xi \cdot \partial^\mu \ln(\efield)] = [1] \cdot [E] = [E]$ \checkmark
	\end{dimanalysis}
	
	The Christoffel symbols themselves acquire time field corrections:
	
	\begin{equation}
		\Gamma^\lambda_{\mu\nu} = \Gamma^\lambda_{\mu\nu|0} + \frac{\xi}{2} \left(\delta^\lambda_\mu \partial_\nu \tfield + \delta^\lambda_\nu \partial_\mu \tfield - g_{\mu\nu} \partial^\lambda \tfield\right)
	\end{equation}
	
	where $\Gamma^\lambda_{\mu\nu|0}$ are the standard Christoffel symbols, $\tfield$ is the time field, $\delta^\lambda_\mu$ is the Kronecker delta, and $g_{\mu\nu}$ is the metric tensor.
	
	\begin{important}
		The mathematical equivalence of these three phenomena means that T0-Theory explains with a single mechanism what the Standard Model explains through different physical processes. Specifically:
		
		\begin{enumerate}
			\item Cosmological redshift is not a consequence of spatial expansion, but of a gradual energy loss of photons
			\item This energy loss follows the same field equation that also describes the gravitational deflection of light
			\item Both phenomena are manifestations of the local variation of the energy field, described by the parameter $\xi$
		\end{enumerate}
		
		This unification is a central conceptual advantage of the T0-model over the Standard Model.
	\end{important}
	
	\section{Experimental Signatures and Tests}
	
	The mathematical equivalence of energy loss, redshift, and light deflection leads to specific experimental predictions that can distinguish the T0-model from standard physics:
	
	\subsection{Wavelength-Dependent Redshift}
	
	For a quasar at redshift $z_0 = 2$, with $\alpha = 0.1$:
	\begin{align}
		z(\text{blue}) &= 2.0 \times (1 - 0.1 \times \ln(0.5)) = 2.0 \times (1 + 0.069) = 2.14 \\
		z(\text{red}) &= 2.0 \times (1 - 0.1 \times \ln(2.0)) = 2.0 \times (1 - 0.069) = 1.86
	\end{align}
	
	This predicts a systematic variation in redshift with wavelength, which could be tested by measuring the redshift of the same astronomical object at different wavelengths.
	
	\subsection{Energy-Dependent Light Deflection}
	
	For X-ray (10 keV) and optical (2 eV) photons in a deflection by the Sun:
	\begin{equation}
		\frac{\theta_{\text{X-ray}}}{\theta_{\text{optical}}} \approx 1 + \frac{4}{3} \times 10^{-4} \cdot \frac{10^4 \text{ eV} - 2 \text{ eV}}{511 \times 10^3 \text{ eV}} \approx 1 + 2.6 \times 10^{-6}
	\end{equation}
	
	This small but potentially measurable difference in deflection angle could be detected with future high-precision observations.
	
	\subsection{Correlation Between Redshift and Light Deflection}
	
	The correlation between redshift and gravitational deflection is described by:
	\begin{equation}
		\frac{\Delta z}{\Delta \theta} = \frac{\xi E_{\gamma,0}}{\efield} \cdot \frac{bc^2}{4GM} \cdot \frac{1}{\ln\left(\frac{r}{r_0}\right)} \cdot \frac{1}{\xi \frac{E_\gamma}{E_0}}
	\end{equation}
	
	When observing gravitational lensing of distant objects, a specific correlation between the degree of light deflection and redshift should be detectable, which differs from the prediction of the Standard Model.
	
	\section{Conclusion}
	
	The T0-Theory unifies the phenomena of energy loss, redshift, and light deflection through a single geodesic equation with time field corrections. This unification is achieved through the universal geometric parameter $\xi = \frac{4}{3} \times 10^{-4}$, which determines the coupling between the energy field and spacetime geometry.
	
	The mathematical equivalence of these phenomena leads to specific experimental predictions that could potentially be tested with high-precision astronomical observations, providing a way to distinguish between the T0-model and standard physics.
	
	This unified approach represents a conceptual advance over the Standard Model, which treats these phenomena as distinct effects requiring separate theoretical frameworks.
	
\end{document}