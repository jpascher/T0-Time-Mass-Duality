\documentclass[12pt,a4paper]{article}
\usepackage[utf8]{inputenc}
\usepackage[T1]{fontenc}
\usepackage[english]{babel}
\usepackage{amsmath,amssymb,amsfonts,amsthm}
\usepackage{physics}
\usepackage{siunitx}
\usepackage{geometry}
\usepackage{fancyhdr}
\usepackage{enumitem}
\usepackage{booktabs}
\usepackage{longtable}
\usepackage{array}
\usepackage{xcolor}
\usepackage{tcolorbox}
\usepackage{mdframed}
\usepackage{graphicx}
\usepackage{hyperref}

\geometry{margin=2.5cm}
\pagestyle{fancy}
\fancyhf{}
\fancyhead[L]{T0-Theory: Redshift Mechanism}
\fancyhead[R]{\thepage}
\fancyfoot[C]{\textit{Wavelength-Dependent Redshift without Distance Assumptions}}

\hypersetup{
	colorlinks=true,
	linkcolor=blue,
	filecolor=magenta,
	urlcolor=cyan,
}

% Custom commands
\newcommand{\xiconst}{\xi = \frac{4}{3} \times 10^{-4}}
\newcommand{\Exi}{E_\xi}
\newcommand{\xicoupling}{f(E/\Exi)}
\newcommand{\lambdazero}{\lambda_0}
\newcommand{\nuzero}{\nu_0}
\newcommand{\betaT}{\beta_T}
\newcommand{\rzero}{r_0}

% Custom environments
\newtcolorbox{important}[1][]{colback=yellow!10!white,colframe=yellow!50!black,fonttitle=\bfseries,title=Key Insight,#1}
\newtcolorbox{formula}[1][]{colback=blue!5!white,colframe=blue!75!black,fonttitle=\bfseries,title=T0 Prediction,#1}
\newtcolorbox{experiment}[1][]{colback=green!5!white,colframe=green!75!black,fonttitle=\bfseries,title=Experimental Test,#1}
\newtcolorbox{revolutionary}[1][]{colback=red!5!white,colframe=red!75!black,fonttitle=\bfseries,title=Paradigm Shift,#1}

\theoremstyle{definition}
\newtheorem{principle}{Principle}
\newtheorem{theorem}{Theorem}
\newtheorem{proposition}{Proposition}

\title{\Huge\textbf{T0-Theory: Redshift Mechanism}\\
	\Large Wavelength-Dependent Redshift \\
	without Distance Assumptions or Spatial Expansion}

\author{Based on the T0-Theory Framework\\
	Spectroscopic Tests Using Cosmic Object Masses}

\date{\today}

\begin{document}
	
	\maketitle
	
	\begin{abstract}
		The T0 model explains cosmological redshift through $\xi$-field energy loss during photon propagation, without requiring spatial expansion or distance measurements. This mechanism predicts a wavelength-dependent redshift $z \propto \lambda$ that can be tested with spectroscopic observations of cosmic objects. Using the universal constant $\xiconst$ and measured masses of astronomical objects, the theory provides model-independent tests distinguishable from standard cosmology. The $\xi$-field also explains the cosmic microwave background temperature ($T_{\text{CMB}} = 2.7255$ K) in a static, eternally existing universe, as detailed in \cite{pascher_cmb_en,pascher_cosmos_en}.
	\end{abstract}
	
	\tableofcontents
	\newpage
	
	\section{Introduction}
	
	\subsection{Universal $\xi$-Constant}
	
	The T0-theory is based on a single fundamental constant \cite{pascher_lagrangian_en}:
	\begin{equation}
		\boxed{\xiconst}
	\end{equation}
	
	This value arises from geometric considerations and determines all fundamental interactions in the universe \cite{pascher_gravitation_en}. The geometric origin stems from the ratio of characteristic scales in the universe, connecting quantum mechanics to cosmology through a single parameter.
	
	\subsection{$\xi$-Field Structure}
	
	The $\xi$-field permeates the entire universe and manifests in three fundamental forms:
	\begin{enumerate}
		\item \textbf{Cosmic Microwave Background (CMB)}: Free $\xi$-field radiation at $T = 2.7255$ K
		\item \textbf{Casimir Vacuum}: Geometrically constrained $\xi$-field between conducting plates
		\item \textbf{Gravitational Interaction}: $\xi$-field coupling to matter determines $G$
	\end{enumerate}
	
	The relationship between these manifestations is given by:
	\begin{equation}
		\frac{|\rho_{\text{Casimir}}|}{\rho_{\text{CMB}}} = \frac{\pi^2}{240 \xi} = \frac{\pi^2 \times 10^4}{320} \approx 308
	\end{equation}
	
	\section{Energy Loss Mechanism}
	
	\subsection{Photon-$\xi$-Field Interaction}
	
	\begin{principle}[$\xi$-Field Energy Loss]
		Photons propagating through the omnipresent $\xi$-field lose energy according to:
		\begin{equation}
			\frac{dE}{dx} = -\xi \cdot \xicoupling \cdot E
		\end{equation}
		where $\xicoupling$ is the energy-dependent coupling function.
	\end{principle}
	
	For the linear coupling case:
	\begin{equation}
		f\left(\frac{E}{\Exi}\right) = \frac{E}{\Exi}
	\end{equation}
	
	This yields the simplified energy loss equation:
	\begin{equation}
		\frac{dE}{dx} = -\frac{\xi E^2}{\Exi}
	\end{equation}
	
	\subsection{Energy-to-Wavelength Conversion}
	
	Since $E = \frac{hc}{\lambda}$ (or $E = \frac{1}{\lambda}$ in natural units, $\hbar = c = 1$), we can express the energy loss in terms of wavelength. Substituting $E = \frac{1}{\lambda}$:
	\begin{equation}
		\frac{d(1/\lambda)}{dx} = -\frac{\xi}{\Exi} \cdot \frac{1}{\lambda^2}
	\end{equation}
	
	Rearranging for wavelength evolution:
	\begin{equation}
		\frac{d\lambda}{dx} = \frac{\xi \lambda^2}{\Exi}
	\end{equation}
	
	\section{Redshift Formula Derivation}
	
	\subsection{Integration for Small $\xi$-Effects}
	
	For the wavelength evolution equation:
	\begin{equation}
		\frac{d\lambda}{dx} = \frac{\xi \lambda^2}{\Exi}
	\end{equation}
	
	Separating variables and integrating:
	\begin{equation}
		\int_{\lambdazero}^{\lambda} \frac{d\lambda'}{\lambda'^2} = \frac{\xi}{\Exi} \int_0^x dx'
	\end{equation}
	
	This yields:
	\begin{equation}
		\frac{1}{\lambdazero} - \frac{1}{\lambda} = \frac{\xi x}{\Exi}
	\end{equation}
	
	Solving for the observed wavelength:
	\begin{equation}
		\lambda = \frac{\lambdazero}{1 - \frac{\xi x \lambdazero}{\Exi}}
	\end{equation}
	
	\subsection{Redshift Definition and Formula}
	
	\begin{formula}
		Redshift definition:
		\begin{equation}
			z = \frac{\lambda_{\text{observed}} - \lambda_{\text{emitted}}}{\lambda_{\text{emitted}}} = \frac{\lambda}{\lambdazero} - 1
		\end{equation}
	\end{formula}
	
	For small $\xi$-effects where $\frac{\xi x \lambdazero}{\Exi} \ll 1$, we can expand:
	\begin{equation}
		z \approx \frac{\xi x \lambdazero}{\Exi} = \frac{\xi x}{\Exi / (\hbar c)} \cdot \lambdazero \quad (\text{in conventional units})
	\end{equation}
	
	\begin{important}
		\textbf{Key T0 Prediction: Wavelength-Dependent Redshift}
		\begin{equation}
			\boxed{z(\lambdazero) = \frac{\xi x}{\Exi} \cdot \lambdazero \quad (\text{natural units, } \hbar = c = 1)}
		\end{equation}
		This works WITHOUT spatial expansion! In conventional units, $\Exi$ scales with $\hbar c \approx 197.3$ MeV$\cdot$fm, so $\Exi \approx 1.5$ GeV corresponds to $\Exi / (\hbar c) \approx 7500$ m$^{-1}$, ensuring dimensional consistency.
	\end{important}
	
	\subsection{Consistency with Observed Redshifts}
	The wavelength-dependent redshift, given by $z \propto \frac{\xi x}{\Exi} \cdot \lambdazero$, explains observed cosmological redshifts in combination with complementary effects such as Doppler shifts, gravitational redshift, and nonlinear $\xi$-field interactions. For high-redshift objects ($z > 10$, e.g., \cite{jwst_early}), the coupling function $f\left(\frac{E}{\Exi}\right)$ may contain higher-order terms ensuring consistency with observations without cosmic expansion. Ongoing spectroscopic tests, as described in Section \ref{sec:experimental_tests}, aim to validate this mechanism.
	
	\section{Frequency-Based Formulation}
	
	\subsection{Frequency Energy Loss}
	
	Since $E = h\nu$, the energy loss equation becomes:
	\begin{equation}
		\frac{d(h\nu)}{dx} = -\frac{\xi (h\nu)^2}{\Exi}
	\end{equation}
	
	Simplifying:
	\begin{equation}
		\frac{d\nu}{dx} = -\frac{\xi h \nu^2}{\Exi}
	\end{equation}
	
	\subsection{Frequency Redshift Formula}
	
	Integrating the frequency evolution:
	\begin{equation}
		\int_{\nuzero}^{\nu} \frac{d\nu'}{\nu'^2} = -\frac{\xi h}{\Exi} \int_0^x dx'
	\end{equation}
	
	This yields:
	\begin{equation}
		\frac{1}{\nu} - \frac{1}{\nuzero} = \frac{\xi h x}{\Exi}
	\end{equation}
	
	Therefore:
	\begin{equation}
		\nu = \frac{\nuzero}{1 + \frac{\xi h x \nuzero}{\Exi}}
	\end{equation}
	
	\begin{formula}
		Frequency redshift:
		\begin{equation}
			z = \frac{\nuzero}{\nu} - 1 \approx \frac{\xi h x \nuzero}{\Exi} \quad (\text{natural units, } h = 1; \text{conventional units, } h = \hbar)
		\end{equation}
	\end{formula}
	
	\begin{important}
		Since $\nu = \frac{c}{\lambda}$, we have $h\nu = \frac{hc}{\lambda}$, confirming:
		\begin{equation}
			z \propto \nu \propto \frac{1}{\lambda}
		\end{equation}
		\textbf{Higher-frequency photons show greater redshift!} In conventional units, $\Exi$ scales with $\hbar c$ to maintain dimensional consistency.
	\end{important}
	
	\section{Observable Predictions without Distance Assumptions}
	
	\subsection{Spectral Line Ratios}
	
	Different atomic transitions should show different redshifts according to their wavelengths:
	\begin{equation}
		\frac{z(\lambda_1)}{z(\lambda_2)} = \frac{\lambda_1}{\lambda_2}
	\end{equation}
	
	\begin{experiment}
		\textbf{Hydrogen Line Test:}
		\begin{itemize}
			\item Lyman-$\alpha$ (121.6 nm) vs. H$\alpha$ (656.3 nm)
			\item Predicted ratio: $\frac{z_{\text{Ly}\alpha}}{z_{\text{H}\alpha}} = \frac{121.6}{656.3} = 0.185$
			\item \textbf{Standard cosmology predicts: 1.000}
		\end{itemize}
	\end{experiment}
	
	\subsection{Frequency-Dependent Effects}
	
	For radio vs. optical observations:
	\begin{itemize}
		\item 21 cm line: $\lambda = 0.21$ m
		\item H$\alpha$ line: $\lambda = 6.563 \times 10^{-7}$ m
		\item Predicted ratio: $z_{21\text{cm}}/z_{\text{H}\alpha} = 3.2 \times 10^5$
	\end{itemize}
	
	\section{Experimental Tests via Spectroscopy}
	\label{sec:experimental_tests}
	
	\subsection{Multi-Wavelength Observations}
	
	\begin{experiment}
		\textbf{Simultaneous Multiband Spectroscopy:}
		\begin{enumerate}
			\item Observe quasar/galaxy simultaneously in UV, optical, IR
			\item Measure redshift from different spectral lines
			\item Test whether $z \propto \lambda$ relationship holds
			\item Compare with standard cosmology prediction ($z = \text{constant}$)
		\end{enumerate}
	\end{experiment}
	
	\subsection{Radio vs. Optical Redshift}
	
	\begin{experiment}
		\textbf{21cm vs. Optical Line Comparison:}
		\begin{itemize}
			\item \textbf{Radio surveys}: ALFALFA, HIPASS (21cm redshifts)
			\item \textbf{Optical surveys}: SDSS, 2dF (H$\alpha$, H$\beta$ redshifts)
			\item \textbf{Method}: Compare objects observed in both surveys
			\item \textbf{Prediction}: $z_{21\text{cm}} \neq z_{\text{optical}}$ (T0) vs. $z_{21\text{cm}} = z_{\text{optical}}$ (Standard)
		\end{itemize}
	\end{experiment}
	
	\subsection{Expected Signal Strength}
	
	For typical cosmic objects with $\xiconst$, the relative difference in redshift between two spectral lines:
	\begin{equation}
		\frac{\Delta z}{z} = \left| \frac{z(\lambda_1) - z(\lambda_2)}{z(\lambda_{\text{mean}})} \right| = \left| \frac{\lambda_1 - \lambda_2}{\lambda_{\text{mean}}} \right| \times \xi \approx 10^{-4} \text{ to } 10^{-5}
	\end{equation}
	
	\begin{important}
		This wavelength effect is at the limit of current spectroscopic precision but potentially detectable with next-generation instruments:
		\begin{itemize}
			\item Extremely Large Telescope (ELT)
			\item James Webb Space Telescope (JWST)
			\item Square Kilometre Array (SKA)
		\end{itemize}
	\end{important}
	
	\section{Advantages over Standard Cosmology}
	
	\subsection{Model-Independent Approach}
	
	\begin{longtable}{lcc}
		\caption{T0-Theory vs. Standard Cosmology} \\
		\toprule
		\textbf{Aspect} & \textbf{T0-Theory} & \textbf{$\Lambda$CDM} \\
		\midrule
		\endfirsthead
		\multicolumn{3}{c}%
		{{\tablename\ \thetable{} -- continued from previous page}} \\
		\toprule
		\textbf{Aspect} & \textbf{T0-Theory} & \textbf{$\Lambda$CDM} \\
		\midrule
		\endhead
		\bottomrule
		\endfoot
		\bottomrule
		\endlastfoot
		Universal constant & $\xi = 4/3 \times 10^{-4}$ & None \\
		Dark energy required & No & Yes (70\%) \\
		Dark matter required & No & Yes (25\%) \\
		Number of parameters & 1 & 6+ \\
		Hubble tension & Resolved & Unresolved \\
		JWST observations & Consistent & Problematic \\
		Big Bang singularity & None & Required \\
		Horizon problem & None & Unresolved \\
		Flatness problem & Natural & Fine-tuning required \\
	\end{longtable}
	
	\subsection{Unified Explanations}
	
	The single $\xi$-constant explains:
	\begin{enumerate}
		\item \textbf{Gravitational constant}: $G = \frac{\xi^2}{4m}$
		\item \textbf{CMB temperature}: $T_{\text{CMB}} = \frac{16}{9} \xi^2 \times E_\xi$
		\item \textbf{Casimir effect}: Related to $\xi$-field vacuum
		\item \textbf{Cosmological redshift}: Energy loss through $\xi$-field
		\item \textbf{Particle masses}: Geometric resonances in $\xi$-field
		\item \textbf{Fine structure constant}: $\alpha = (4/3)^3 \approx 1/137$
		\item \textbf{Muon anomalous magnetic moment}: $a_\mu = \frac{\xi}{2\pi} \left(\frac{E_\mu}{E_e}\right)^2$
	\end{enumerate}
	
	\section{Statistical Analysis Method}
	
	\subsection{Multi-Line Regression}
	
	\begin{experiment}
		\textbf{Wavelength-Redshift Correlation Test:}
		\begin{enumerate}
			\item Collect redshift measurements: $\{z_i, \lambda_i\}$ for each object
			\item Fit linear relationship: $z = \alpha \cdot \lambda + \beta$
			\item Compare slope $\alpha$ with T0 prediction: $\alpha = \frac{\xi x}{\Exi}$
			\item Test against standard cosmology: $\alpha = 0$
		\end{enumerate}
	\end{experiment}
	
	\subsection{Required Precision}
	
	To detect T0 effects with $\xiconst$:
	\begin{itemize}
		\item \textbf{Minimum required precision}: $\frac{\Delta z}{z} \approx 10^{-5}$
		\item \textbf{Current best precision}: $\frac{\Delta z}{z} \approx 10^{-4}$ (barely sufficient)
		\item \textbf{Next generation instruments}: $\frac{\Delta z}{z} \approx 10^{-6}$ (clearly detectable)
	\end{itemize}
	
	\section{Mathematical Equivalence of Space Expansion, Energy Loss, and Diffraction}
	\label{sec:equivalence}
	
	\subsection{Formal Equivalence Proofs}
	\label{subsec:equivalence_proofs}
	
	The three fundamental mechanisms for explaining cosmological redshift can be described by different physical processes but lead to mathematically equivalent results under certain conditions.
	
	\begin{table}[h]
		\centering
		\caption{Comparison of Redshift Mechanisms with Extended Developments}
		\scalebox{0.75}{
			\begin{tabular}{lllc}
				\toprule
				\textbf{Mechanism} & \textbf{Physical Process} & \textbf{Redshift Formula} & \textbf{Taylor Expansion} \\
				\midrule
				Space Expansion ($\Lambda$CDM) & Metric expansion & $1+z = \frac{a(t_0)}{a(t_e)}$ & $z \approx H_0 D + \frac{1}{2}q_0(H_0 D)^2$ \\
				Energy Loss (T0-E) & Photon fatigue & $1+z = \exp\left(\int_0^D \xi \frac{H}{T} dl\right)$ & $z \approx \xi \frac{H_0 D}{T_0} + \frac{1}{2}\xi^2\left(\frac{H_0 D}{T_0}\right)^2$ \\
				Vacuum Diffraction (T0-B) & Refractive index change & $1+z = \frac{n(t_e)}{n(t_0)}$ & $z \approx \xi \ln\left(1+\frac{H_0 D}{c}\right)\left(1+\frac{\xi\lambda_0}{2\lambda_{crit}}\right)$ \\
				\bottomrule
			\end{tabular}
		}
	\end{table}
	
	\subsubsection{Mathematical Equivalence Conditions}
	
	For the equivalence of the three mechanisms, the following conditions must be satisfied:
	
	\begin{equation}
		\boxed{\frac{1}{a}\frac{da}{dt} = -\frac{1}{n}\frac{dn}{dt} = \xi \frac{H}{T_0}}
	\end{equation}
	
	This leads to the relationships:
	\begin{itemize}
		\item \textbf{$\Lambda$CDM $\leftrightarrow$ T0-B}: $n(t) = a^{-1}(t)$
		\item \textbf{$\Lambda$CDM $\leftrightarrow$ T0-E}: $\dot{E}/E = -H(t)$
		\item \textbf{T0-B $\leftrightarrow$ T0-E}: $n(t) \propto E^{-1}(t)$
	\end{itemize}
	
	\subsubsection{Perturbative Development}
	
	The equivalence holds exactly only in first order. Higher-order deviations provide distinguishing signatures:
	
	\begin{equation}
		z_{total} = z_0 + \Delta z_{mechanism} + O(\xi^2)
	\end{equation}
	
	where $\Delta z_{mechanism}$ depends on the specific physical process.
	
	\subsection{Energy Conservation and Thermodynamics}
	\label{subsec:energy_conservation}
	
	\subsubsection{Energy Balance in Different Formalisms}
	
	\textbf{$\Lambda$CDM (apparent energy loss):}
	\begin{equation}
		E_{photon} = \frac{h\nu_0}{1+z} = \frac{h\nu_0 a(t_e)}{a(t_0)}
	\end{equation}
	
	\textbf{T0-Diffraction (energy conservation):}
	\begin{equation}
		E_{photon} = \frac{h\nu}{n(t)} = \frac{h\nu_0}{(1+z)n(t)} = \text{const}
	\end{equation}
	
	\textbf{T0-Energy Loss (real loss):}
	\begin{equation}
		\frac{dE}{dt} = -\xi H E \quad \Rightarrow \quad E(t) = E_0 \exp\left(-\int_0^t \xi H(t') dt'\right)
	\end{equation}
	
	\subsubsection{Thermodynamic Consistency}
	
	The entropy change for the different mechanisms:
	
	\begin{equation}
		\Delta S = \begin{cases}
			0 & \text{($\Lambda$CDM: adiabatic)} \\
			k_B \xi N_{photon} \ln(1+z) & \text{(T0-Energy Loss)} \\
			0 & \text{(T0-Diffraction: reversible)}
		\end{cases}
	\end{equation}
	
	\section{Implications for Cosmology}
	
	\subsection{Static Universe Model}
	
	The T0-theory describes a static, eternally existing universe where:
	\begin{itemize}
		\item Redshift arises from energy loss, not expansion
		\item CMB is equilibrium radiation of the $\xi$-field
		\item No Big Bang singularity required
		\item No dark energy or dark matter needed
		\item Cyclic processes possible within static framework
	\end{itemize}
	
	\subsection{Resolution of Cosmological Tensions}
	
	The T0 model resolves:
	\begin{enumerate}
		\item \textbf{Hubble tension}: Different measurements reconciled through $\xi$-effects
		\item \textbf{JWST early galaxies}: No formation time paradox in static universe
		\item \textbf{Cosmic coincidence}: Natural explanation through $\xi$-geometry
		\item \textbf{Horizon problem}: No horizon in eternal universe
		\item \textbf{Flatness problem}: Natural consequence of static geometry
	\end{enumerate}
	
	\section{Robustness of Core T0 Predictions}
	
	\subsection{Independent of Redshift Mechanism}
	
	Even if spectroscopic tests fail to detect wavelength-dependent redshift, the following T0 predictions remain valid:
	
	\begin{enumerate}
		\item \textbf{Gravitational constant}: $G = \frac{\xi^2 c^3}{16\pi m_p^2} = 6.674 \times 10^{-11}$ m$^3$kg$^{-1}$s$^{-2}$ (accurate to 8 digits) remains valid, independent of cosmological tests
		
		\item \textbf{Geometric constants}: The derivation of $\alpha \approx 1/137$ from $(4/3)^3$ scaling remains
		
		\item \textbf{Mass hierarchy}: $m_e : m_\mu : m_\tau = 1 : 206.768 : 3477.15$ follows from quantum numbers, not redshift
		
		\item \textbf{Hubble tension}: The 4/3 explanation works regardless of specific mechanism
	\end{enumerate}
	
	\subsection{Adaptivity of Theoretical Structure}
	
	The T0-theory has natural adaptation mechanisms:
	
	\begin{equation}
		\xi_{eff}(\text{Scale}) = \xi_0 \times f(\text{Environment}) \times g(\text{Energy})
	\end{equation}
	
	where:
	\begin{itemize}
		\item $f(\text{Environment}) = 4/3$ in galaxy clusters, $= 1$ in intergalactic medium
		\item $g(\text{Energy})$ describes renormalization group running
	\end{itemize}
	
	This flexibility is not an ad-hoc adjustment but follows from the geometric structure of the theory.
	
	\section{Conclusions}
	
	The T0-theory provides a revolutionary alternative to expansion-based cosmology through a single universal constant $\xiconst$. The wavelength-dependent redshift prediction offers a clear experimental test to distinguish between T0 and standard cosmology. While current precision barely reaches the detection threshold, next-generation spectroscopic instruments should definitively test this fundamental prediction.
	
	The unification of gravitational, electromagnetic, and quantum phenomena through the $\xi$-field represents a paradigm shift from complex multi-parameter models to elegant geometric simplicity. The experimental tests proposed here, particularly multi-wavelength spectroscopy of cosmic objects, provide clear pathways to validate or refute the theory.
	
	\begin{revolutionary}
		The T0-theory demonstrates that all cosmic phenomena can be understood through a single geometric constant, eliminating the need for dark matter, dark energy, inflation, and the Big Bang singularity. This represents the most significant simplification in physics since Newton's unification of terrestrial and celestial mechanics.
	\end{revolutionary}
	
	% Bibliography
	\bibliographystyle{unsrt}
	\begin{thebibliography}{99}
		
		% Primary T0-Theory Documents (German and English)
		\bibitem{pascher_lagrangian_de}
		Pascher, Johann (2025). 
		\textit{Vereinfachte Lagrange-Dichte und Zeit-Massen-Dualität in der T0-Theorie}. 
		T0-Theory Project. 
		\url{https://jpascher.github.io/T0-Time-Mass-Duality/2/pdf/lagrandian-einfachDe.pdf}
		
		\bibitem{pascher_lagrangian_en}
		Pascher, Johann (2025). 
		\textit{Simplified Lagrangian Density and Time-Mass Duality in T0-Theory}. 
		T0-Theory Project. 
		\url{https://jpascher.github.io/T0-Time-Mass-Duality/2/pdf/lagrandian-einfachEn.pdf}
		
		\bibitem{pascher_cosmos_de}
		Pascher, Johann (2025). 
		\textit{T0-Modell: Ein vereinheitlichtes, statisches, zyklisches, dunkle-Materie-freies und dunkle-Energie-freies Universum}. 
		T0-Theory Project. 
		\url{https://jpascher.github.io/T0-Time-Mass-Duality/2/pdf/cos_De.pdf}
		
		\bibitem{pascher_cosmos_en}
		Pascher, Johann (2025). 
		\textit{T0-Model: A unified, static, cyclic, dark-matter-free and dark-energy-free universe}. 
		T0-Theory Project. 
		\url{https://jpascher.github.io/T0-Time-Mass-Duality/2/pdf/cos_En.pdf}
		
		\bibitem{pascher_cmb_de}
		Pascher, Johann (2025). 
		\textit{Temperatureinheiten in natürlichen Einheiten: T0-Theorie und statisches Universum}. 
		T0-Theory Project. 
		\url{https://jpascher.github.io/T0-Time-Mass-Duality/2/pdf/TempEinheitenCMBDe.pdf}
		
		\bibitem{pascher_cmb_en}
		Pascher, Johann (2025). 
		\textit{Temperature Units in Natural Units: T0-Theory and Static Universe}. 
		T0-Theory Project. 
		\url{https://jpascher.github.io/T0-Time-Mass-Duality/2/pdf/TempEinheitenCMBEn.pdf}
		
		\bibitem{pascher_gravitation_en}
		Pascher, Johann (2025). 
		\textit{Geometric Determination of the Gravitational Constant: From the T0-Model}. 
		T0-Theory Project. 
		\url{https://jpascher.github.io/T0-Time-Mass-Duality/2/pdf/gravitationskonstnte_En.pdf}
		

		
		\bibitem{pascher_redshift_en}
		Pascher, Johann (2025). 
		\textit{T0-Theory: Wavelength-Dependent Redshift without Distance Assumptions}. 
		T0-Theory Project. 
		\url{https://jpascher.github.io/T0-Time-Mass-Duality/2/pdf/redshift_deflection_En.pdf}
		
		\bibitem{pascher_derivation_beta}
		Pascher, J. (2025). 
		\textit{Field-Theoretic Derivation of the $\beta_T$ Parameter in Natural Units ($\hbar = c = 1$)}. 
		GitHub Repository: T0-Time-Mass-Duality.
		\url{https://github.com/jpascher/T0-Time-Mass-Duality/blob/main/2/pdf/DerivationVonBetaEn.pdf}
		
		\bibitem{pascher_unified}
		Pascher, J. (2025).
		\textit{Mathematical Proof: The Fine Structure Constant $\alpha = 1$ in Natural Units}.
		\url{https://github.com/jpascher/T0-Time-Mass-Duality/blob/main/2/pdf/ResolvingTheConstantsAlfaEn.pdf}
		
		\bibitem{pascher_muon_g2}
		Pascher, J. (2025).
		\textit{Complete Calculation of the Muon's Anomalous Magnetic Moment in the Unified Natural Unit System}.
		\url{https://github.com/jpascher/T0-Time-Mass-Duality/blob/main/2/pdf/CompleteMuon_g-2_AnalysisEn.pdf}
		
		\bibitem{pascher_pragmatic}
		Pascher, J. (2025).
		\textit{Established Calculations in the Unified Natural Unit System: Reinterpretation Rather Than Rejection}.
		\url{https://github.com/jpascher/T0-Time-Mass-Duality/blob/main/2/pdf/PragmaticApproachT0-ModelEn.pdf}
		
		\bibitem{pascher_t0_energie}
		Pascher, J. (2025). 
		\textit{The T0-Model (Planck-Referenced): A Reformulation of Physics}. 
		\url{https://github.com/jpascher/T0-Time-Mass-Duality/tree/main/2/pdf}
		
		\bibitem{pascher_units}
		Pascher, J. (2025). 
		\textit{Natural Unit Systems: Universal Energy Conversion and Fundamental Length Scale Hierarchy}. 
		\url{https://github.com/jpascher/T0-Time-Mass-Duality/blob/main/2/pdf/NatEinheitenSystematikEn.pdf}
		
		% Fundamental Physics References
		\bibitem{heisenberg1927}
		Heisenberg, W. (1927). 
		\textit{On the intuitive content of quantum theoretical kinematics and mechanics}. 
		Zeitschrift für Physik, 43(3-4), 172--198.
		
		\bibitem{einstein1915}
		Einstein, A. (1915). 
		\textit{Die Feldgleichungen der Gravitation}. 
		Sitzungsberichte der Preußischen Akademie der Wissenschaften, 844--847.
		
		\bibitem{einstein1905}
		Einstein, A. (1905). 
		\textit{Ist die Trägheit eines Körpers von seinem Energieinhalt abhängig?} 
		Ann. Phys., 17, 639--641.
		
		\bibitem{dirac1928}
		Dirac, P. A. M. (1928). 
		\textit{The Quantum Theory of the Electron}. 
		Proc. R. Soc. London A, 117, 610.
		
		\bibitem{dirac1958}
		Dirac, P. A. M. (1958). 
		\textit{The Principles of Quantum Mechanics}. 
		4th Edition, Oxford University Press.
		
		\bibitem{feynman1949}
		Feynman, R. P. (1949). 
		\textit{Space-Time Approach to Quantum Electrodynamics}. 
		Phys. Rev., 76, 769.
		
		\bibitem{higgs1964}
		Higgs, P. W. (1964).
		\textit{Broken Symmetries and the Masses of Gauge Bosons}.
		Phys. Rev. Lett., 13, 508.
		
		\bibitem{weinberg1967}
		Weinberg, S. (1967).
		\textit{A Model of Leptons}.
		Phys. Rev. Lett., 19, 1264.
		
		\bibitem{weinberg1979}
		Weinberg, S. (1979). 
		\textit{Phenomenological Lagrangians}. 
		Physica A, 96, 327--340.
		
		\bibitem{weinberg1989}
		Weinberg, S. (1989). 
		\textit{The Cosmological Constant Problem}. 
		Rev. Mod. Phys., 61, 1.
		
		\bibitem{yang1954}
		Yang, C. N. and Mills, R. L. (1954).
		\textit{Conservation of Isotopic Spin and Isotopic Gauge Invariance}.
		Phys. Rev., 96, 191.
		
		\bibitem{yukawa1935}
		Yukawa, H. (1935).
		\textit{On the Interaction of Elementary Particles}.
		Proc. Phys. Math. Soc. Japan, 17, 48.
		
		\bibitem{bohr1928}
		Bohr, N. (1928).
		\textit{The Quantum Postulate and the Recent Development of Atomic Theory}.
		Nature, 121, 580.
		
		\bibitem{maxwell1873}
		Maxwell, J. C. (1873). 
		\textit{A Treatise on Electricity and Magnetism}. 
		Clarendon Press, Oxford.
		
		\bibitem{kaluza1921}
		Kaluza, T. (1921).
		\textit{Zum Unitätsproblem der Physik}.
		Sitzungsber. Preuss. Akad. Wiss. Berlin (Math. Phys.), 966--972.
		
		\bibitem{klein1926}
		Klein, O. (1926).
		\textit{Quantentheorie und fünfdimensionale Relativitätstheorie}.
		Z. Phys., 37, 895--906.
		
		% Cosmological Observations
		\bibitem{planck2020}
		Planck Collaboration (2020). 
		\textit{Planck 2018 results. VI. Cosmological parameters}. 
		Astronomy \& Astrophysics, 641, A6. 
		\url{https://doi.org/10.1051/0004-6361/201833910}
		
		\bibitem{riess1998}
		Riess, A. G., et al. (1998). 
		\textit{Observational Evidence from Supernovae for an Accelerating Universe and a Cosmological Constant}. 
		Astron. J., 116, 1009.
		
		\bibitem{riess2022}
		Riess, A. G., et al. (2022). 
		\textit{A Comprehensive Measurement of the Local Value of the Hubble Constant with 1 km s$^{-1}$ Mpc$^{-1}$ Uncertainty from the Hubble Space Telescope and the SH0ES Team}. 
		The Astrophysical Journal Letters, 934(1), L7. 
		\url{https://doi.org/10.3847/2041-8213/ac5c5b}
		
		\bibitem{jwst_early}
		Naidu, R. P., et al. (2022). 
		\textit{Two Remarkably Luminous Galaxy Candidates at z $\approx$ 11--13 Revealed by JWST}. 
		The Astrophysical Journal Letters, 940(1), L14. 
		\url{https://doi.org/10.3847/2041-8213/ac9b22}
		
		\bibitem{cobe1992}
		COBE Collaboration (1992). 
		\textit{Structure in the COBE differential microwave radiometer first-year maps}. 
		The Astrophysical Journal Letters, 396, L1--L5. 
		\url{https://doi.org/10.1086/186504}
		
		\bibitem{mcgaugh2016}
		McGaugh, S. S., Lelli, F., and Schombert, J. M. (2016). 
		\textit{Radial Acceleration Relation in Rotationally Supported Galaxies}. 
		Phys. Rev. Lett., 117, 201101.
		
		\bibitem{bolton2008}
		Bolton, A. S., Burles, S., Koopmans, L. V. E., Treu, T., and Moustakas, L. A. (2008). 
		\textit{The Sloan Lens ACS Survey. V. The Full ACS Strong-Lens Sample}. 
		Astrophys. J., 682, 964--984.
		
		\bibitem{suyu2017}
		Suyu, S. H., Bonvin, V., Courbin, F., et al. (2017). 
		\textit{H0LiCOW - I. H0 Lenses in COSMOGRAIL's Wellspring: program overview}. 
		Mon. Not. Roy. Astron. Soc., 468, 2590--2604.
		
		% Experimental Physics
		\bibitem{codata2018}
		CODATA (2018). 
		\textit{The 2018 CODATA Recommended Values of the Fundamental Physical Constants}. 
		National Institute of Standards and Technology. 
		\url{https://physics.nist.gov/cuu/Constants/}
		
		\bibitem{casimir1948}
		Casimir, H. B. G. (1948). 
		\textit{On the attraction between two perfectly conducting plates}. 
		Proceedings of the Royal Netherlands Academy of Arts and Sciences, 51(7), 793--795.
		
		\bibitem{sparnaay1958}
		Sparnaay, M. J. (1958). 
		\textit{Measurements of attractive forces between flat plates}. 
		Physica, 24(6-10), 751--764. 
		\url{https://doi.org/10.1016/S0031-8914(58)80090-7}
		
		\bibitem{lamoreaux1997}
		Lamoreaux, S. K. (1997). 
		\textit{Demonstration of the Casimir force in the 0.6 to 6 $\mu$m range}. 
		Physical Review Letters, 78(1), 5--8. 
		\url{https://doi.org/10.1103/PhysRevLett.78.5}
		
		\bibitem{muon_g2_2021}
		Muon g-2 Collaboration (2021). 
		\textit{Measurement of the Positive Muon Anomalous Magnetic Moment to 0.46 ppm}. 
		Physical Review Letters, 126(14), 141801. 
		\url{https://doi.org/10.1103/PhysRevLett.126.141801}
		
		\bibitem{katrin_2024}
		KATRIN Collaboration (2024). 
		\textit{Direct neutrino-mass measurement based on 259 days of KATRIN data}. 
		arXiv:2406.13516.
		
		\bibitem{nufit_2024}
		Esteban, I., et al. (2024). 
		\textit{NuFit-6.0: updated global analysis of three-flavor neutrino oscillations}. 
		J. High Energy Phys., 12, 216.
		
		\bibitem{pound1960}
		Pound, R. V. and Rebka Jr., G. A. (1960).
		\textit{Apparent Weight of Photons}.
		Phys. Rev. Lett., 4, 337--341.
		
		\bibitem{pound1971}
		Pound, R. V. and Snider, J. L. (1971). 
		\textit{Effect of Gravity on Nuclear Resonance}. 
		Phys. Rev. Lett., 26, 1132--1135.
		
		\bibitem{webb2001}
		Webb, J. K., Murphy, M. T., Flambaum, V. V., Dzuba, V. A., Barrow, J. D., Churchill, C. W., Prochaska, J. X., and Wolfe, A. M. (2001). 
		\textit{Further Evidence for Cosmological Evolution of the Fine Structure Constant}. 
		Phys. Rev. Lett., 87, 091301.
		
		\bibitem{ludlow2015}
		Ludlow, A. D., Boyd, M. M., Ye, J., Peik, E., and Schmidt, P. O. (2015). 
		\textit{Optical atomic clocks}. 
		Rev. Mod. Phys., 87, 637--701.
		
		\bibitem{quinn2013}
		Quinn, T., Parks, H., Speake, C., and Davis, R. (2013). 
		\textit{Improved Determination of G Using Two Methods}. 
		Phys. Rev. Lett., 111, 101102.
		
		\bibitem{ashby2003}
		Ashby, N. (2003). 
		\textit{Relativity in the Global Positioning System}. 
		Living Rev. Rel., 6, 1.
		
		% Additional Theoretical References
		\bibitem{peskin1995}
		Peskin, M. E. and Schroeder, D. V. (1995). 
		\textit{An Introduction to Quantum Field Theory}. 
		Addison-Wesley, Reading.
		
		\bibitem{pdg2020}
		Zyla, P. A., et al. (Particle Data Group) (2020). 
		\textit{Review of Particle Physics}. 
		Prog. Theor. Exp. Phys., 2020, 083C01.
		
		\bibitem{bertone2005}
		Bertone, G., Hooper, D., and Silk, J. (2005). 
		\textit{Particle dark matter: evidence, candidates and constraints}. 
		Phys. Rep., 405, 279--390.
		
	\end{thebibliography}
	
\end{document}