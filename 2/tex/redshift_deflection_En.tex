\documentclass[12pt,a4paper]{article}
\usepackage[utf8]{inputenc}
\usepackage[T1]{fontenc}
\usepackage[english]{babel}
\usepackage{amsmath,amssymb,amsfonts,amsthm}
\usepackage{physics}
\usepackage{siunitx}
\usepackage{geometry}
\usepackage{fancyhdr}
\usepackage{enumitem}
\usepackage{booktabs}
\usepackage{longtable}
\usepackage{array}
\usepackage{xcolor}
\usepackage{tcolorbox}
\usepackage{mdframed}
\usepackage{graphicx}
\usepackage{hyperref}

\geometry{margin=2.5cm}
\pagestyle{fancy}
\fancyhf{}
\fancyhead[L]{T0-Theory: Redshift Mechanism}
\fancyhead[R]{\thepage}
\fancyfoot[C]{\textit{Wavelength-Dependent Redshift without Distance Assumptions}}

\hypersetup{
	colorlinks=true,
	linkcolor=blue,
	filecolor=magenta,
	urlcolor=cyan,
}

% Custom commands
\newcommand{\xiconst}{\xi = \frac{4}{3} \times 10^{-4}}
\newcommand{\Exi}{E_\xi}
\newcommand{\xicoupling}{f(E/\Exi)}
\newcommand{\lambdazero}{\lambda_0}
\newcommand{\nuzero}{\nu_0}

% Custom environments
\newtcolorbox{important}[1][]{colback=yellow!10!white,colframe=yellow!50!black,fonttitle=\bfseries,title=Key Insight,#1}
\newtcolorbox{formula}[1][]{colback=blue!5!white,colframe=blue!75!black,fonttitle=\bfseries,title=T0-Prediction,#1}
\newtcolorbox{experiment}[1][]{colback=green!5!white,colframe=green!75!black,fonttitle=\bfseries,title=Experimental Test,#1}

\theoremstyle{definition}
\newtheorem{principle}{Principle}

\title{\Huge\textbf{T0-Theory: Redshift Mechanism}\\
	\Large Wavelength-Dependent Redshift \\
	without Distance Assumptions}

\author{Based on T0-Theory Framework\\
	Spectroscopic Tests using Cosmic Object Masses}

\date{\today}

\begin{document}
	
	\maketitle
	
	\begin{abstract}
		The T0-model explains cosmological redshift through $\xi$-field energy loss during photon propagation, without requiring spatial expansion or distance measurements. This mechanism predicts wavelength-dependent redshift $z \propto \lambda$ that can be tested with spectroscopic observations of cosmic objects. Using the universal constant $\xiconst$ and measured masses of astronomical objects, the theory provides model-independent tests distinguishable from standard cosmology.
	\end{abstract}
	
	\tableofcontents
	\newpage
	
	\section{Fundamental $\xi$-Field Energy Loss}
	
	\subsection{Basic Mechanism}
	
	\begin{principle}[$\xi$-Field Photon Interaction]
		Photons lose energy through interaction with the universal $\xi$-field during propagation:
		\begin{equation}
			\frac{dE}{dx} = -\xi \cdot f\left(\frac{E}{\Exi}\right) \cdot E
		\end{equation}
		where $\xiconst$ is the universal geometric constant and $\Exi = \frac{1}{\xi} = 7500$ (natural units).
	\end{principle}
	
	The coupling function $f(E/\Exi)$ is dimensionless and describes the energy-dependent interaction strength. For the linear coupling case:
	\begin{equation}
		f\left(\frac{E}{\Exi}\right) = \frac{E}{\Exi}
	\end{equation}
	
	This yields the simplified energy loss equation:
	\begin{equation}
		\frac{dE}{dx} = -\frac{\xi E^2}{\Exi}
	\end{equation}
	
	\subsection{Energy-to-Wavelength Conversion}
	
	Since $E = \frac{hc}{\lambda}$ (or $E = \frac{1}{\lambda}$ in natural units), we can express the energy loss in terms of wavelength. Substituting $E = \frac{1}{\lambda}$:
	
	\begin{equation}
		\frac{d(1/\lambda)}{dx} = -\frac{\xi}{\Exi} \cdot \frac{1}{\lambda^2}
	\end{equation}
	
	Rearranging to find the wavelength evolution:
	\begin{equation}
		\frac{d\lambda}{dx} = \frac{\xi \lambda^2}{\Exi}
	\end{equation}
	
	\section{Redshift Formula Derivation}
	
	\subsection{Integration for Small $\xi$-Effects}
	
	For the wavelength evolution equation:
	\begin{equation}
		\frac{d\lambda}{dx} = \frac{\xi \lambda^2}{\Exi}
	\end{equation}
	
	Separating variables and integrating:
	\begin{equation}
		\int_{\lambdazero}^{\lambda} \frac{d\lambda'}{\lambda'^2} = \frac{\xi}{\Exi} \int_0^x dx'
	\end{equation}
	
	This yields:
	\begin{equation}
		\frac{1}{\lambdazero} - \frac{1}{\lambda} = \frac{\xi x}{\Exi}
	\end{equation}
	
	Solving for the observed wavelength:
	\begin{equation}
		\lambda = \frac{\lambdazero}{1 - \frac{\xi x \lambdazero}{\Exi}}
	\end{equation}
	
	\subsection{Redshift Definition and Formula}
	
	\begin{formula}
		Redshift Definition:
		\begin{equation}
			z = \frac{\lambda_{\text{observed}} - \lambda_{\text{emitted}}}{\lambda_{\text{emitted}}} = \frac{\lambda}{\lambdazero} - 1
		\end{equation}
	\end{formula}
	
	For small $\xi$-effects where $\frac{\xi x \lambdazero}{\Exi} \ll 1$, we can expand:
	\begin{equation}
		z \approx \frac{\xi x \lambdazero}{\Exi} = \frac{\xi x}{\Exi} \cdot \lambdazero
	\end{equation}
	
	\begin{important}
		\textbf{Key T0-Prediction: Wavelength-Dependent Redshift}
		\begin{equation}
			\boxed{z(\lambdazero) = \frac{\xi x}{\Exi} \cdot \lambdazero}
		\end{equation}
		
		This is the fundamental prediction of T0-theory: \textbf{Redshift is proportional to the emitted wavelength!}
	\end{important}
	
	\section{Frequency-Based Formulation}
	
	\subsection{Frequency Energy Loss}
	
	Since $E = h\nu$, the energy loss equation becomes:
	\begin{equation}
		\frac{d(h\nu)}{dx} = -\frac{\xi (h\nu)^2}{\Exi}
	\end{equation}
	
	Simplifying:
	\begin{equation}
		\frac{d\nu}{dx} = -\frac{\xi h \nu^2}{\Exi}
	\end{equation}
	
	\subsection{Frequency Redshift Formula}
	
	Integrating the frequency evolution:
	\begin{equation}
		\int_{\nuzero}^{\nu} \frac{d\nu'}{\nu'^2} = -\frac{\xi h}{\Exi} \int_0^x dx'
	\end{equation}
	
	This yields:
	\begin{equation}
		\frac{1}{\nu} - \frac{1}{\nuzero} = \frac{\xi h x}{\Exi}
	\end{equation}
	
	Therefore:
	\begin{equation}
		\nu = \frac{\nuzero}{1 + \frac{\xi h x \nuzero}{\Exi}}
	\end{equation}
	
	\begin{formula}
		Frequency Redshift:
		\begin{equation}
			z = \frac{\nuzero}{\nu} - 1 \approx \frac{\xi h x \nuzero}{\Exi}
		\end{equation}
	\end{formula}
	
	\begin{important}
		Since $\nu = \frac{c}{\lambda}$, we have $h\nu = \frac{hc}{\lambda}$, confirming:
		\begin{equation}
			z \propto \nu \propto \frac{1}{\lambda}
		\end{equation}
		\textbf{Higher frequency photons show larger redshift!}
	\end{important}
	
	\section{Observable Predictions Without Distance Assumptions}
	
	\subsection{Spectral Line Ratios}
	
	Different atomic transitions should show different redshifts according to their wavelengths:
	
	\begin{equation}
		\frac{z(\lambda_1)}{z(\lambda_2)} = \frac{\lambda_1}{\lambda_2}
	\end{equation}
	
	\begin{experiment}
		\textbf{Hydrogen Line Test:}
		\begin{itemize}
			\item Lyman-$\alpha$ (121.6 nm) vs. H$\alpha$ (656.3 nm)
			\item Predicted ratio: $\frac{z_{\text{Ly}\alpha}}{z_{\text{H}\alpha}} = \frac{121.6}{656.3} = 0.185$
			\item \textbf{Standard cosmology predicts: 1.000}
		\end{itemize}
	\end{experiment}
	
	\subsection{Frequency-Dependent Effects}
	
	For radio vs. optical observations of the same object:
	\begin{equation}
		\frac{z_{\text{radio}}}{z_{\text{optical}}} = \frac{\nu_{\text{radio}}}{\nu_{\text{optical}}}
	\end{equation}
	
	\begin{experiment}
		\textbf{21cm vs. H$\alpha$ Test:}
		\begin{itemize}
			\item 21cm hydrogen line: $\nu = 1420$ MHz
			\item Optical H$\alpha$ line: $\nu = 457$ THz
			\item Predicted ratio: $\frac{z_{21\text{cm}}}{z_{\text{H}\alpha}} = \frac{1.42 \times 10^9}{4.57 \times 10^{14}} = 3.1 \times 10^{-6}$
		\end{itemize}
	\end{experiment}
	
	\section{Mass-Based Energy Scale Calibration}
	
	\subsection{Using Known Cosmic Object Masses}
	
	Instead of assuming distances, we use measured masses of cosmic objects to calibrate the energy scale:
	
	\begin{longtable}{lll}
		\caption{Well-Determined Cosmic Masses} \\
		\toprule
		\textbf{Object Type} & \textbf{Example} & \textbf{Mass} \\
		\midrule
		\endfirsthead
		\multicolumn{3}{c}{\tablename\ \thetable{} -- Continued} \\
		\toprule
		\textbf{Object Type} & \textbf{Example} & \textbf{Mass} \\
		\midrule
		\endhead
		\multicolumn{3}{l}{\emph{Stellar Masses (Precise)}} \\
		Sun & Sol & $1.989 \times 10^{30}$ kg \\
		Sirius A & Alpha CMa A & $2.02\,M_\odot$ \\
		Alpha Centauri A & $\alpha$ Cen A & $1.1\,M_\odot$ \\
		\midrule
		\multicolumn{3}{l}{\emph{Galaxy Masses (From Dynamics)}} \\
		Milky Way & Our Galaxy & $10^{12}\,M_\odot$ \\
		Andromeda & M31 & $1.5 \times 10^{12}\,M_\odot$ \\
		Local Group & Total & $\approx 3 \times 10^{12}\,M_\odot$ \\
		\bottomrule
	\end{longtable}
	
	\subsection{Mass-Energy Relation in $\xi$-Field}
	
	The characteristic energy scale is:
	\begin{equation}
		\Exi = \xi^{-1} = \frac{3}{4 \times 10^{-4}} = 7500 \text{ (natural units)}
	\end{equation}
	
	Converting to conventional units:
	\begin{equation}
		\Exi = 7500 \times (\hbar c) \approx 1.5 \text{ GeV}
	\end{equation}
	
	This energy scale is comparable to nuclear binding energies, suggesting the $\xi$-field couples to fundamental mass scales in cosmic structures.
	
	\section{Experimental Tests Using Spectroscopy}
	
	\subsection{Multi-Wavelength Observations}
	
	\begin{experiment}
		\textbf{Simultaneous Multi-Band Spectroscopy:}
		\begin{enumerate}
			\item Observe quasar/galaxy simultaneously in UV, optical, IR
			\item Measure redshift from different spectral lines
			\item Test if $z \propto \lambda$ relationship holds
			\item Compare with standard cosmology prediction ($z = \text{constant}$)
		\end{enumerate}
	\end{experiment}
	
	\subsection{Radio vs. Optical Redshift}
	
	\begin{experiment}
		\textbf{21cm vs. Optical Line Comparison:}
		\begin{itemize}
			\item \textbf{Radio surveys}: ALFALFA, HIPASS (21cm redshifts)
			\item \textbf{Optical surveys}: SDSS, 2dF (H$\alpha$, H$\beta$ redshifts)
			\item \textbf{Method}: Compare objects observed in both surveys
			\item \textbf{Prediction}: $z_{21\text{cm}} \neq z_{\text{optical}}$ (T0) vs. $z_{21\text{cm}} = z_{\text{optical}}$ (Standard)
		\end{itemize}
	\end{experiment}
	
	\subsection{Expected Signal Strength}
	
	For typical cosmic objects with $\xiconst$:
	
	\begin{equation}
		\frac{\Delta z}{z} = \frac{\lambda_1 - \lambda_2}{\lambda_{\text{avg}}} \times \xi \approx 10^{-4} \text{ to } 10^{-5}
	\end{equation}
	
	\begin{important}
		This wavelength effect is at the limit of current spectroscopic precision but potentially detectable with next-generation instruments like:
		\begin{itemize}
			\item Extremely Large Telescope (ELT)
			\item James Webb Space Telescope (JWST)
			\item Square Kilometre Array (SKA)
		\end{itemize}
	\end{important}
	
	\section{Advantages Over Standard Cosmology}
	
	\subsection{Model-Independent Approach}
	
	\begin{longtable}{lcc}
		\caption{T0-Theory vs. Standard Cosmology} \\
		\toprule
		\textbf{Aspect} & \textbf{Standard Cosmology} & \textbf{T0-Theory} \\
		\midrule
		\endfirsthead
		\multicolumn{3}{c}{\tablename\ \thetable{} -- Continued} \\
		\toprule
		\textbf{Aspect} & \textbf{Standard Cosmology} & \textbf{T0-Theory} \\
		\midrule
		\endhead
		Distance Requirement & $z \rightarrow d$ (via Hubble) & Direct spectroscopic test \\
		Wavelength Dependence & $\frac{dz}{d\lambda} = 0$ & $\frac{dz}{d\lambda} \propto \xi$ \\
		Free Parameters & $\Omega_m, \Omega_\Lambda, H_0, \ldots$ & Single parameter $\xi$ \\
		Exotic Components & Dark Energy (69\%) & Only $\xi$-field \\
		Testability & Indirect (via distance ladder) & Direct (spectroscopy) \\
		\bottomrule
	\end{longtable}
	
	\subsection{Testable Predictions}
	
	\begin{formula}
		\textbf{Distinguishing Test:}
		\begin{align}
			\text{Standard:} \quad &z_{\text{blue}} = z_{\text{red}} \\
			\text{T0:} \quad &\frac{z_{\text{blue}}}{z_{\text{red}}} = \frac{\lambda_{\text{blue}}}{\lambda_{\text{red}}} < 1
		\end{align}
	\end{formula}
	
	\section{Observational Strategy}
	
	\subsection{Target Selection}
	
	Focus on objects with:
	\begin{enumerate}
		\item \textbf{Strong spectral lines} across wide wavelength range
		\item \textbf{Well-determined masses} from stellar/galactic dynamics
		\item \textbf{High signal-to-noise} spectra available
	\end{enumerate}
	
	\textbf{Ideal targets:}
	\begin{itemize}
		\item Bright quasars with broad spectral coverage
		\item Nearby galaxies with multiple emission lines
		\item Binary star systems with precise mass determinations
	\end{itemize}
	
	\subsection{Data Analysis Protocol}
	
	\begin{experiment}
		\textbf{Analysis Steps:}
		\begin{enumerate}
			\item Measure redshifts from multiple spectral lines
			\item Plot $z$ vs. $\lambda$ for each object
			\item Fit linear relationship: $z = \alpha \cdot \lambda + \beta$
			\item Compare slope $\alpha$ with T0-prediction: $\alpha = \frac{\xi x}{\Exi}$
			\item Test against standard cosmology: $\alpha = 0$
		\end{enumerate}
	\end{experiment}
	
	\subsection{Required Precision}
	
	To detect T0-effects with $\xiconst$:
	
	\begin{itemize}
		\item \textbf{Minimum precision needed}: $\frac{\Delta z}{z} \approx 10^{-5}$
		\item \textbf{Current best precision}: $\frac{\Delta z}{z} \approx 10^{-4}$ (barely sufficient)
		\item \textbf{Next-generation instruments}: $\frac{\Delta z}{z} \approx 10^{-6}$ (clearly detectable)
	\end{itemize}
	
	\section{Conclusion}
	
	\subsection{Summary of T0-Redshift Mechanism}
	
	The T0-theory provides a \textbf{distance-independent} mechanism for cosmological redshift through $\xi$-field energy loss. The key features are:
	
	\begin{enumerate}
		\item \textbf{Universal constant}: $\xiconst$ determines all redshift effects
		\item \textbf{Wavelength dependence}: $z \propto \lambda$ (fundamental prediction)
		\item \textbf{Mass-based calibration}: Uses measured cosmic object masses
		\item \textbf{Model-independent tests}: Direct spectroscopic verification
	\end{enumerate}
	
	\subsection{Experimental Accessibility}
	
	The theory provides concrete, testable predictions:
	
	\begin{formula}
		\textbf{Key Experimental Signature:}
		\begin{equation}
			\boxed{\frac{z_{\text{blue}}}{z_{\text{red}}} = \frac{\lambda_{\text{blue}}}{\lambda_{\text{red}}} \neq 1}
		\end{equation}
	\end{formula}
	
	This prediction can be tested with:
	\begin{itemize}
		\item Multi-wavelength spectroscopy of the same objects
		\item Radio vs. optical redshift comparisons
		\item High-precision measurements of spectral line ratios
	\end{itemize}
	
	\subsection{Revolutionary Implications}
	
	\begin{important}
		If confirmed, wavelength-dependent redshift would revolutionize our understanding of:
		\begin{itemize}
			\item \textbf{Cosmic redshift origin}: Energy loss vs. spatial expansion
			\item \textbf{Distance measurements}: Model-independent spectroscopic distances
			\item \textbf{Dark energy}: No longer required to explain cosmic acceleration
			\item \textbf{Fundamental physics}: New field interactions on cosmic scales
		\end{itemize}
	\end{important}
	
	The T0-redshift mechanism offers a \textbf{testable alternative} to standard cosmology that can be verified through spectroscopic observations, making it experimentally accessible with current or near-future astronomical instruments.
	
	\bibliographystyle{plain}
	\begin{thebibliography}{9}
		
		\bibitem{pascher2024}
		Pascher, J. (2024). \textit{T0-Theory: Mathematical Equivalence Formulation}. HTL Leonding, Department of Communications Engineering.
		
		\bibitem{planck2020}
		Planck Collaboration (2020). \textit{Planck 2018 results. VI. Cosmological parameters}. Astron. Astrophys. 641, A6.
		
		\bibitem{sdss2020}
		SDSS Collaboration (2020). \textit{The Sloan Digital Sky Survey: Technical Summary}. Astron. J. 120, 1579.
		
		\bibitem{alfalfa2018}
		ALFALFA Team (2018). \textit{The Arecibo Legacy Fast ALFA Survey}. Astrophys. J. Suppl. 232, 21.
		
	\end{thebibliography}
	
\end{document}