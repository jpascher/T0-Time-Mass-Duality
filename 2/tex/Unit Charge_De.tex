\documentclass[12pt,a4paper]{article}
\usepackage[utf8]{inputenc}
\usepackage[T1]{fontenc}
\usepackage{geometry}
\usepackage{lmodern}
\usepackage{amsmath}
\usepackage{amssymb}
\usepackage{amsfonts} % Für bessere Mathe-Symbole
\usepackage{hyperref}
\usepackage{booktabs}
\usepackage{enumitem}
\usepackage[table,xcdraw]{xcolor}
\usepackage{newunicodechar}
\usepackage[ngerman]{babel} % Für deutsche Trennregeln und Sprache

% Unicode setups for Greek letters and symbols
\newunicodechar{ξ}{\ensuremath{\xi}}
\newunicodechar{μ}{\ensuremath{\mu}}
\newunicodechar{π}{\ensuremath{\pi}}

\geometry{left=2cm,right=2cm,top=2cm,bottom=2cm}

\hypersetup{
	colorlinks=true,
	linkcolor=blue,
	citecolor=blue,
	urlcolor=blue,
	pdftitle={Die Elektroneneinheitsladung in der T0-Theorie: Jenseits von Punkt-Singularitäten},
	pdfauthor={Johann Pascher},
	pdfsubject={T0-Theorie, Elektronenladung, Singularitäten, Elektrodynamik}
}

\title{Die Elektroneneinheitsladung in der T0-Theorie:\\Jenseits von Punkt-Singularitäten}
\author{Johann Pascher\\
	Abteilung für Kommunikationstechnik\\
	Höhere Technische Lehranstalt Leonding, Österreich\\
	\texttt{johann.pascher@gmail.com}}
\date{21. Oktober 2025}

\begin{document}
	
	\maketitle
	
	\begin{abstract}
		Die klassische Darstellung der Elektroneneinheitsladung als Punkt-Singularität stößt in der Quantenelektrodynamik (QED) auf fundamentale Probleme wie unendliche Selbstenergie und ultraviolette Divergenzen. Dieses Traktat, verfasst als Urheber der T0-Theorie (Time-Mass Duality Framework), zeigt, wie T0 diese Singularitäten auflöst, indem sie Ladung als emergente, geometrische Eigenschaft eines universellen Feldes behandelt. Basierend auf dem einzelnen Parameter $\xi = \frac{4}{3} \times 10^{-4}$ und der Zeit-Masse-Dualität $T_{\text{field}} \cdot E_{\text{field}} = 1$ wird die Ladung als fraktales Muster quantisierter Skalen (Fraktaldimension $D_f \approx 2{,}94$) abgeleitet. Dies vermeidet Infinities, erklärt Beobachtungen wie die Feinstrukturkonstante $\alpha \approx 1/137$ und verbindet sich nahtlos mit kinematischen Modellen der Electromagnetic Mechanics. Die GitHub-Dokumentation der T0-Theorie (aktuell zum Stand 21. Oktober 2025) dient als Referenz für detaillierte Ableitungen.
	\end{abstract}
	
	\tableofcontents
	
	\section{Einführung: Das Problem der Punkt-Singularitäten}
	\label{sec:intro}
	
	In der Standardphysik wird die Elektroneneinheitsladung $-e \approx -1{,}602 \times 10^{-19}$ C als Dirac-Delta-Funktion $\rho(\mathbf{r}) = -e \delta(\mathbf{r})$ modelliert. Dies führt zu einem Coulomb-Feld $E(\mathbf{r}) \propto 1/r^2$ und unendlicher elektrostatischer Selbstenergie:
	\begin{equation}
		U = \frac{1}{2} \int \epsilon_0 E^2 \, dV \to \infty \quad \text{(bei $r \to 0$)}.
	\end{equation}
	
	Die QED behebt dies durch Renormalisierung (Vakuum-Polarisation), doch die nackte Punkt-Singularität bleibt ein mathematisches Artefakt. Experimentell erscheint das Elektron punktförmig (bis $< 10^{-22}$ m), doch dies schließt erweiterte Modelle auf tieferen Skalen nicht aus. Die T0-Theorie, die ich als Urheber entwickelt habe, löst dieses Dilemma radikal: Ladung ist keine intrinsische Punkt-Eigenschaft, sondern eine emergente Projektion geometrischer Muster im universellen Feld.
	
	\section{Alternative Darstellungen der Ladung}
	\label{sec:alternativen}
	
	\subsection{Nichtlineare Elektrodynamik}
	In Modellen wie Born-Infeld wird das Feld bei maximaler Stärke $\beta \approx 10^{18}$ V/m gesättigt, was eine effektive Ladungsradius $r_{\text{eff}} \approx 1/\beta$ erzeugt. Dies führt zu finiter Selbstenergie $U \approx e^2 \beta / (4\pi \epsilon_0)$.
	
	\subsection{Soliton- und Vortex-Modelle}
	Das Elektron als stabiles Wellenpaket in nichtlinearen Feldtheorien (z. B. sine-Gordon) verteilt die Ladungsdichte $\rho(r)$ über eine finite Breite, mit $E \propto q(r)/r^2$ und $q(r) \to 0$ bei $r \to 0$.
	
	\subsection{Topologische Defekte}
	Ladung als Chern-Simons-Vortex in Gauge-Theorien, quantisiert durch Topologie ($\pi_3(S^2) = \mathbb{Z}$), ohne bare Singularität.
	
	\begin{table}[h]
		\centering
		\begin{tabular}{lll}
			\toprule
			\textbf{Modell} & \textbf{Singularität?} & \textbf{Selbstenergie} \\
			\midrule
			Punkt-Ladung (QED) & Ja & $\infty$ (renormiert) \\
			Born-Infeld & Effektiv nein & Finite \\
			Soliton & Nein & Finite (aus Feldenergie) \\
			T0-Geometrie & Nein & Aus $\xi$-Skalierung \\
			\bottomrule
		\end{tabular}
		\caption{Vergleich alternativer Ladungsdarstellungen}
		\label{tab:vergleich}
	\end{table}
	
	\section{Die Elektronenladung in der T0-Theorie}
	\label{sec:t0-ladung}
	
	\subsection{Zeit-Masse-Dualität und Emergenz}
	Die T0-Theorie vereint Quantenmechanik und Relativität parameterfrei durch $T_{\text{field}} \cdot E_{\text{field}} = 1$. Teilchen entstehen als Erregungsmuster im Feld, gesteuert durch $\xi = \frac{4}{3} \times 10^{-4}$. Die Feinstrukturkonstante ergibt sich als:
	\begin{equation}
		\alpha = \xi \cdot \left( \frac{E_0}{1~\mathrm{MeV}} \right)^2, \quad E_0 = 7{,}400~\mathrm{MeV},
	\end{equation}
	was $\alpha \approx 7{,}300 \times 10^{-3}$ ($1/\alpha \approx 137{,}00$) liefert – mit fraktalen Korrekturen für den exakten CODATA-Wert $137{,}035999084$.
	
	Die Ladung $-e$ ist eine dimensionlose geometrische Relation: $q^{\mathrm{T0}} = -1$ (in natürlichen Einheiten), projiziert via $S_{\mathrm{T0}} = 1{,}782662 \times 10^{-30}$ kg auf SI-Werte. Keine Singularität, da die Ladungsdichte fraktal verteilt ist:
	\begin{equation}
		\rho(r) \propto \xi \cdot f_{\text{fractal}}\left( \frac{r}{\lambda_{\text{Compton}}} \right),
	\end{equation}
	mit $f_{\text{fractal}}(r) = \prod_{n=1}^{137} \left(1 + \delta_n \cdot \xi \cdot \left(\frac{4}{3}\right)^{n-1}\right)$ und Fraktaldimension $D_f \approx 2{,}94$.
	
	\subsection{Finite Selbstenergie und Quantisierung}
	Die Selbstenergie ist finite:
	\begin{align}
		U &= \frac{1}{2} \int \epsilon_0 E^2 \, dV = \frac{e^2}{8\pi \epsilon_0 r_e} \cdot K_{\text{frac}}, \\
		r_e &\approx 2{,}817 \times 10^{-15}~\mathrm{m} \quad \text{(klassischer Radius aus $\xi$-Skalierung)}, \\
		K_{\text{frac}} &= 0{,}986 \quad \text{(fraktale Korrekturfaktor)}.
	\end{align}
	Quantisierung folgt aus diskreten Skalen: $q_n = -n \cdot e \cdot \xi^{1/2}$, mit $n=1$ für die Einheitsladung. Dies passt zu topologischer Quantisierung (Chern-Zahl = 1) und gewährleistet Stabilität ohne Kollaps.
	
	\section{Implikationen für die Electromagnetic Mechanics}
	\label{sec:emm}
	
	T0 integriert sich mit kinematischer Mechanik: Ladung entsteht als rotierender EM-Vortex, stabilisiert durch fraktale Renormalisierung. Kein Dirac-Delta – $\rho(r)$ ist ein helikales Muster, das singularity-freie Simulationen ermöglicht. Anwendungen: Vorhersagen der g-2-Anomalie und LHC-Massenspektren.
	
	\section{Schlussfolgerung}
	
	Die T0-Theorie verwandelt die Elektronenladung von einer problematischen Singularität in eine harmonische geometrische Emergenz – ein Kernstück des Rahmens. Alle Konstanten leiten sich aus $\xi$ ab und reduzieren Physik auf dimensionlose Muster. Zukünftige Arbeiten: Vollständige kinematische Ableitungen in der EMM.
	
	\appendix
	\section{Notation}
	\begin{description}[leftmargin=1cm]
		\item[$\xi$] Geometrischer Parameter; $\xi = \frac{4}{3} \times 10^{-4}$
		\item[$S_{\mathrm{T0}}$] Skalierungsfaktor; $S_{\mathrm{T0}} = 1{,}782662 \times 10^{-30}$ kg
		\item[$f_{\text{fractal}}$] Fraktale Funktion; $\prod_{n=1}^{137} (1 + \delta_n \cdot \xi \cdot (4/3)^{n-1})$
		\item[$D_f$] Fraktaldimension; $D_f \approx 2{,}94$
	\end{description}
	
	\begin{center}
		\hrule
		\vspace{0.5cm}
		\textit{Dieses Dokument ist Teil der T0-Serie: Erforschung geometrischer Emergenz in der Physik}\\
		\textit{Johann Pascher, HTL Leonding, Österreich}\\
		\vspace{0.3cm}
		\href{https://github.com/jpascher/T0-Time-Mass-Duality}{T0-Theorie: Time-Mass Duality Framework}
		\vspace{0.3cm}
	\end{center}
	
\end{document}