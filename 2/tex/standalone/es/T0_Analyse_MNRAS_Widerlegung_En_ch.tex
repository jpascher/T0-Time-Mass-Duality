\documentclass[11pt,a4paper]{article}
\usepackage[a4paper,margin=2cm]{geometry}
\usepackage[utf8]{inputenc}
\usepackage[spanish]{babel}
\usepackage{lmodern}
\usepackage{amsmath,amssymb}
\usepackage[unicode,hypertexnames=false]{hyperref}
\usepackage{enumitem}

% T0-specific macros (comprehensive)
\newcommand{\xiT}{\xi}
\newcommand{\xipar}{\xi}
\newcommand{\phiT}{\phi}
\newcommand{\Tfield}{T}
\newcommand{\Tfieldt}{T}
\newcommand{\Efield}{E}
\providecommand{\lP}{\ell_P}
\providecommand{\tP}{t_P}
\providecommand{\mP}{m_P}
\providecommand{\EP}{E_P}
\providecommand{\EPlanck}{E_P}
\providecommand{\Ezero}{E_0}
\providecommand{\Exi}{E_\xi}
\providecommand{\Ee}{E_e}
\providecommand{\Emu}{E_\mu}
\providecommand{\Echar}{E_{\text{char}}}
\providecommand{\Evis}{E_{\text{vis}}}
\providecommand{\Lag}{\mathcal{L}}
\providecommand{\Leff}{\mathcal{L}_{\text{eff}}}
\providecommand{\Lxi}{L_\xi}
\providecommand{\Lzero}{L_0}
\providecommand{\Lp}{\ell_P}
\providecommand{\Kfrak}{K_{\text{frak}}}
\providecommand{\Dfrak}{D_f}
\providecommand{\Df}{D_f}
\providecommand{\betapar}{\beta}
\providecommand{\alphapar}{\alpha}
\providecommand{\Hubble}{H}
\providecommand{\Lambdat}{\Lambda_t}
\providecommand{\Tzero}{T_0}
\providecommand{\CQCD}{C_{\text{QCD}}}
\providecommand{\Cconv}{C_{\text{conv}}}
\providecommand{\Cto}{C_{\text{T0}}}
\providecommand{\deltam}{\delta m}
\providecommand{\Weyl}{W}
\providecommand{\Riem}{\mathcal{R}}
\providecommand{\Lorentz}{\mathcal{L}}
\providecommand{\SynchPower}{P_{\text{synch}}}
\providecommand{\Phiphoton}{\Phi_{\gamma}}
\providecommand{\DhiggsT}{D_{H,T}}
\providecommand{\xigeom}{\xi_{\text{geom}}}
\providecommand{\rzero}{r_0}


\setlength{\parindent}{0pt}
\setlength{\parskip}{6pt}

\hypersetup{
  colorlinks=true,
  linkcolor=blue,
  citecolor=blue,
  urlcolor=blue
}

\title{T0 Analyse MNRAS Widerlegung En}
\author{J. Pascher}
\date{\today}

\begin{document}
\maketitle

\section*{T0 Analyse Mnras Widerlegung (T0 Analyse MNRAS Widerlegung)}

	\begin{abstract}
		This document analyzes the findings of the influential paper "Does the Hubble tension eclipse the Solar System?" (MNRAS, 544, 1, 2024) \cite{nathan2024} and places them in the context of the T0-Theory. The paper refutes a significant class of modified gravity theories by demonstrating that they would lead to measurable anomalies in Solar System orbits, which are not observed. We argue that this falsification should be considered strong, indirect evidence for the T0-Theory's approach, as T0-Theory is, by definition, consistent with high-precision Solar System data.
	\end{abstract}
	
	
	\section{Summary of the MNRAS Paper}
	
	The "Hubble tension"—the discrepancy between measurements of the universe's expansion rate in the near and distant cosmos—is one of the greatest puzzles in modern cosmology. A popular proposed solution is to modify the theory of General Relativity on cosmological scales.
	
	The paper by Nathan et al. \cite{nathan2024}, published in \textit{Monthly Notices of the Royal Astronomical Society} (MNRAS), applies a rigorous test to this hypothesis:
	\begin{enumerate}
		\item \textbf{Assumption:} The authors assume a class of modified gravity theories designed to resolve the Hubble tension.
		\item \textbf{Solar System Test:} They apply the same theory to our local environment and calculate the theoretically expected effects on the high-precision orbit of the planet Saturn.
		\item \textbf{Result:} The modifications required to explain the Hubble tension would produce significant, easily measurable deviations in Saturn's orbit.
		\item \textbf{Falsification:} High-precision observational data, particularly from the Cassini spacecraft, show no sign of these predicted anomalies. The observed orbit aligns perfectly with the predictions of unmodified General Relativity.
	\end{enumerate}
	
	The paper's conclusion is unequivocal: This specific class of modified gravity theories is incompatible with observations and is therefore refuted as an explanation for the Hubble tension.
	
	\section{Implications for the T0-Theory}
	
	The falsification of a competing model often serves as strong, indirect confirmation for an alternative theory. This is especially true here, as the T0-Theory solves the problem at a more fundamental level and trivially passes the "test" described in the paper.
	
	\subsection{T0-Theory Does Not Modify Gravity}
	The crucial difference is that T0-Theory leaves General Relativity untouched on Solar System scales. It does not postulate any ad-hoc modification of gravity. Instead, it addresses the flawed premise upon which the Hubble tension is based: the assumption of cosmic expansion.
	
	\subsection{Redshift as a Geometric Effect}
	In the T0-Theory, there is no accelerated expansion and, consequently, no "Hubble tension" to explain. The observed cosmological redshift is instead explained as an emergent, geometric effect:
	\begin{itemize}
		\item Light loses energy on its journey through the T0 vacuum via a cumulative interaction with the field's fractal geometry.
		\item This effect manifests as a systematic redshift that is proportional to the distance traveled.
	\end{itemize}
	
	\subsection{Consistency with Solar System Data}
	The mechanism of geometric redshift is absolutely negligible over the comparatively tiny distances of the Solar System (a few light-hours). The cumulative effect only becomes measurable over millions and billions of light-years.
	
	It follows that:
	\begin{center}
\section*{The T0-Theory predicts exactly zero measurable anomalies in the planetary orbits of the Solar System.}
	\end{center}
	It is therefore, by definition, perfectly consistent with the high-precision data from the Cassini mission that refutes the modified gravity models.
	
	\section{Conclusion}
	
	The paper by Nathan et al. \cite{nathan2024} makes an important contribution by closing a speculative and inconsistent avenue for resolving the Hubble tension. Simultaneously, it highlights the strength of a more fundamental approach, such as the one pursued by the T0-Theory.
	
	By addressing the cause (the interpretation of redshift) rather than the symptom (the expansion), the T0-Theory not only resolves the Hubble tension but also remains in full agreement with the most precise observations in our own Solar System. The failure of modified gravity is thus a success for the physical consistency of T0 cosmology.
	
	


\begin{thebibliography}{99}

\bibitem{nathan2024}
Nathan, A. et al. (2024).
	\textit{Quantum computing advances}.
	Nature.

\end{thebibliography}


\end{document}
