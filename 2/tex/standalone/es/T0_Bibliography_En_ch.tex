\documentclass[11pt,a4paper]{article}
\usepackage[a4paper,margin=2cm]{geometry}
\usepackage[utf8]{inputenc}
\usepackage[spanish]{babel}
\usepackage{lmodern}
\usepackage{amsmath,amssymb}
\usepackage[unicode,hypertexnames=false]{hyperref}

% T0-specific macros
\newcommand{\xiT}{\xi}
\newcommand{\phiT}{\phi}
\newcommand{\Tfield}{T}
\providecommand{\lP}{\ell_P}
\providecommand{\tP}{t_P}
\providecommand{\mP}{m_P}
\providecommand{\EP}{E_P}

\setlength{\parindent}{0pt}
\setlength{\parskip}{6pt}

\hypersetup{
  colorlinks=true,
  linkcolor=blue,
  citecolor=blue,
  urlcolor=blue
}

\title{T0 Bibliography En}
\author{J. Pascher}
\date{\today}

\begin{document}
\maketitle

\section*{T0 Bibliography (T0 Bibliography)}

	\begin{abstract}
		This document contains the complete bibliography of the T0 Time-Mass Duality framework, including foundational documents, mathematical foundations, particle physics applications, cosmology, and quantum mechanics developments.
	\end{abstract}
	
	
	\section{Introduction}
	The T0 Framework represents a comprehensive approach to theoretical physics, unifying concepts of time-mass duality through mathematical consistency and empirical validation.
	
	\section{Bibliography}
	
	




\end{document}
