\documentclass[11pt,a4paper]{article}
\usepackage[a4paper,margin=2cm]{geometry}
\usepackage[utf8]{inputenc}
\usepackage[english]{babel}
\usepackage{lmodern}
\usepackage{amsmath,amssymb}
\usepackage[unicode,hypertexnames=false]{hyperref}
\usepackage{enumitem}

% T0-specific macros (comprehensive)
\newcommand{\xiT}{\xi}
\newcommand{\xipar}{\xi}
\newcommand{\phiT}{\phi}
\newcommand{\Tfield}{T}
\newcommand{\Tfieldt}{T}
\newcommand{\Efield}{E}
\providecommand{\lP}{\ell_P}
\providecommand{\tP}{t_P}
\providecommand{\mP}{m_P}
\providecommand{\EP}{E_P}
\providecommand{\EPlanck}{E_P}
\providecommand{\Ezero}{E_0}
\providecommand{\Exi}{E_\xi}
\providecommand{\Ee}{E_e}
\providecommand{\Emu}{E_\mu}
\providecommand{\Echar}{E_{\text{char}}}
\providecommand{\Evis}{E_{\text{vis}}}
\providecommand{\Lag}{\mathcal{L}}
\providecommand{\Leff}{\mathcal{L}_{\text{eff}}}
\providecommand{\Lxi}{L_\xi}
\providecommand{\Lzero}{L_0}
\providecommand{\Lp}{\ell_P}
\providecommand{\Kfrak}{K_{\text{frak}}}
\providecommand{\Dfrak}{D_f}
\providecommand{\Df}{D_f}
\providecommand{\betapar}{\beta}
\providecommand{\alphapar}{\alpha}
\providecommand{\Hubble}{H}
\providecommand{\Lambdat}{\Lambda_t}
\providecommand{\Tzero}{T_0}
\providecommand{\CQCD}{C_{\text{QCD}}}
\providecommand{\Cconv}{C_{\text{conv}}}
\providecommand{\Cto}{C_{\text{T0}}}
\providecommand{\deltam}{\delta m}
\providecommand{\Weyl}{W}
\providecommand{\Riem}{\mathcal{R}}
\providecommand{\Lorentz}{\mathcal{L}}
\providecommand{\SynchPower}{P_{\text{synch}}}
\providecommand{\Phiphoton}{\Phi_{\gamma}}
\providecommand{\DhiggsT}{D_{H,T}}
\providecommand{\xigeom}{\xi_{\text{geom}}}
\providecommand{\rzero}{r_0}


\setlength{\parindent}{0pt}
\setlength{\parskip}{6pt}

\hypersetup{
  colorlinks=true,
  linkcolor=blue,
  citecolor=blue,
  urlcolor=blue
}

\title{T0 Introduction En}
\author{J. Pascher}
\date{\today}

\begin{document}
\maketitle

\section*{T0 Time--Mass Duality\ English Book (T0 Introduction)}

	\section*{Introduction}
	\addcontentsline{toc}{chapter}{Introduction}
	
	This book presents the current state of the T0 time--mass duality framework and its applications to
	particle masses, fundamental constants, quantum mechanics, gravitation, and cosmology.
	
	The main body of the book consists of a set of core T0 documents. These chapters reflect the
	present understanding of the theory and its quantitative consequences. Wherever possible, the
	material has been reorganized and unified so that the structure of the theory becomes as transparent
	as possible.
	
	At the end of the book, several older documents are included in an appendix. These texts represent
	earlier stages of the development of the T0 framework. They were not removed, because they make
	the evolution of the ideas and the refinement of the formulas visible. In many cases, one can see
	how approximations were improved, how special cases were generalized, and how new empirical data
	helped to sharpen or correct earlier arguments.
	
	The ``live'' version of the theory is maintained in a public GitHub repository:
	
	\begin{center}
		\url{https://github.com/jpascher/T0-Time-Mass-Duality}
	\end{center}
	
	The LaTeX sources of the chapters in this book are taken from that repository. If conceptual or
	numerical errors are found, they are corrected there first. This means that the PDF version of the
	book you are reading is a snapshot of a continuously evolving project. For the most recent version
	of the documents, including new appendices or corrections, the GitHub repository should always be
	considered the primary reference.
	
	The intention of this compilation is twofold:
	\begin{itemize}
		\item to provide a coherent, readable path through the core ideas and results of the T0 framework;
		\item to document, in the appendix, the historical development of these ideas, including false
		starts, intermediate formulations, and early fits to experimental data.
	\end{itemize}
	
	Readers who are mainly interested in the current formulation of the theory may focus on the core
	chapters. Readers who are also interested in the reasoning and trial--and--error process behind
	the theory are invited to study the appendix material in parallel.
	




\end{document}
