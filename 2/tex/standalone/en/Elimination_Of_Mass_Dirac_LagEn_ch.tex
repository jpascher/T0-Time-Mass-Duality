\documentclass[11pt,a4paper]{article}
\usepackage[a4paper,margin=2cm]{geometry}
\usepackage[utf8]{inputenc}
\usepackage[english]{babel}
\usepackage{lmodern}
\renewcommand{\familydefault}{\sfdefault}

\usepackage{amsmath,amssymb,amsthm}
\usepackage{graphicx}
\usepackage[unicode,pdfencoding=auto,hypertexnames=false]{hyperref}
\usepackage{booktabs}
\usepackage{longtable}
\usepackage{array}
\usepackage{siunitx}
\usepackage{fancyhdr}
\usepackage{float}
\usepackage{tikz}
% tcolorbox removed for standalone
% tcbset removed
\tikzset{
  t0blue/.style={draw=blue,fill=blue!10},
  t0red/.style={draw=red,fill=red!10},
  t0green/.style={draw=green!50!black,fill=green!10},
  t0orange/.style={draw=orange,fill=orange!10},
}
\usepackage{setspace}
\usepackage{enumitem}
\usepackage{adjustbox}
\usepackage{xcolor}

% Define colors for xcolor package
\definecolor{t0green}{RGB}{34,139,34}
\definecolor{t0blue}{RGB}{0,0,255}
\definecolor{t0red}{RGB}{255,0,0}
\definecolor{t0orange}{RGB}{255,165,0}

% Define custom column types for tables
\newcolumntype{L}[1]{>{\raggedright\arraybackslash}p{#1}}
\newcolumntype{C}[1]{>{\centering\arraybackslash}p{#1}}
\newcolumntype{R}[1]{>{\raggedleft\arraybackslash}p{#1}}

\setlength{\parindent}{0pt}
\setlength{\parskip}{6pt}

\hypersetup{
  colorlinks=true,
  linkcolor=blue,
  citecolor=blue,
  urlcolor=blue
}
\pagestyle{fancy}
\setlength{\headheight}{28pt}

\newcommand{\checkmarkx}{\checkmark}
\newcommand{\warningx}{\textbf{!}}

% Makros aus Einzel-Dokumenten (Fallback-Definitionen)
\newcommand{\mytimes}{\times}
\newcommand{\myapprox}{\approx}
\newcommand{\mysim}{\sim}
\newcommand{\myomega}{\omega}
\newcommand{\mypi}{\pi}
\newcommand{\myrightarrow}{\rightarrow}
\newcommand{\mypropto}{\propto}
\newcommand{\deltafield}{\delta\phi}
\newcommand{\xipar}{\xi}
\newcommand{\xiT}{\xi}
\newcommand{\lambdah}{\lambda_h}

% Additional macros used in chapter files
\newcommand{\Kfrak}{K_{\text{frak}}}  % Fractal correction factor
\newcommand{\Dfrak}{D_f}              % Fractal dimension
\newcommand{\betapar}{\beta}          % T0 beta parameter
\newcommand{\alphapar}{\alpha}        % T0 alpha parameter
\newcommand{\Efield}{E}               % Energy field
% Note: checkmarkxa/warningxa are variants used in auto-generated chapter files
\newcommand{\checkmarkxa}{\checkmark}
\newcommand{\warningxa}{\textbf{!}}

% Additional T0-specific macros
\newcommand{\xigeom}{\xi_{\text{geom}}}  % Geometric xi
\newcommand{\lP}{\ell_P}                  % Planck length
\newcommand{\rzero}{r_0}                  % Characteristic radius
\newcommand{\xirat}{\xi_{\text{rat}}}     % Xi ratio
\newcommand{\tzero}{t_0}                  % Characteristic time
\newcommand{\natunits}{\text{(nat. units)}}  % Natural units annotation
\newcommand{\myRightarrow}{\Rightarrow}   % Arrow variant
\newcommand{\Lag}{\mathcal{L}}            % Lagrangian

% Physics macros used in chapter files
\newcommand{\CQCD}{C_{\text{QCD}}}        % QCD correction
\newcommand{\EP}{E_P}                     % Planck energy
\newcommand{\Ee}{E_e}                     % Electron energy
\newcommand{\Emu}{E_\mu}                  % Muon energy
\newcommand{\Exi}{E_\xi}                  % Xi energy
\newcommand{\Ezero}{E_0}                  % Characteristic energy
\newcommand{\Hubble}{H}                   % Hubble constant
\newcommand{\Kspec}{K_{\text{spec}}}      % Spectral correction
\newcommand{\Lambdat}{\Lambda_t}          % Time-related cosmological constant
\newcommand{\Leff}{\mathcal{L}_{\text{eff}}}  % Effective Lagrangian
\newcommand{\Lorentz}{\mathcal{L}}        % Lorentz symbol
\newcommand{\Lxi}{L_\xi}                  % Xi length
\newcommand{\Tfield}{T}                   % Time field
\newcommand{\Weyl}{W}                     % Weyl tensor/symbol
\newcommand{\alphaEMSI}{\alpha_{\text{EM,SI}}}  % EM alpha in SI
\newcommand{\alphaEMnat}{\alpha_{\text{EM,nat}}}  % EM alpha in natural units
\newcommand{\alphaem}{\alpha_{\text{em}}} % Electromagnetic alpha
\newcommand{\betaTSI}{\beta_{T,\text{SI}}}  % Beta in SI
\newcommand{\betaTnat}{\beta_{T,\text{nat}}}  % Beta in natural units
\newcommand{\deltam}{\delta m}            % Mass difference
\newcommand{\phiT}{\phi_T}                % T-field phi
\newcommand{\tP}{t_P}                     % Planck time
\newcommand{\rhoCMB}{\rho_{\text{CMB}}}   % CMB density
\newcommand{\rhoCasimir}{\rho_{\text{Casimir}}}  % Casimir density

% Table formatting
\usepackage{multirow}

% Additional physics macros
\newcommand{\Riem}{\mathcal{R}}           % Riemann tensor
\newcommand{\ZPinch}{Z_{\text{pinch}}}    % Z-pinch
\newcommand{\SynchPower}{P_{\text{synch}}} % Synchrotron power
\newcommand{\Rzero}{R_0}                  % Characteristic radius
\newcommand{\alphafine}{\alpha}           % Fine structure constant
\newcommand{\Etau}{E_\tau}                % Tau energy
\newcommand{\deltaE}{\delta E}            % Energy deviation
\newcommand{\EPlanck}{E_P}                % Planck energy
\newcommand{\pichar}{\pi}                 % Pi character
\newcommand{\alphaWSI}{\alpha_{W,\text{SI}}}  % Wien alpha in SI
\newcommand{\alphaWnat}{\alpha_{W,\text{nat}}}  % Wien alpha in natural units

% Einfache abstract-Umgebung für Kapitel:
\newenvironment{abstract}{%
  \begin{center}\bfseries Abstract\end{center}\small
}{\par}


\title{Elimination Of Mass Dirac LagEn}
\author{J. Pascher}
\date{\today}

\begin{document}
\maketitle

\section*{Elimination Of Mass Dirac Lagen (Elimination Of Mass Dirac LagEn)}

	\begin{abstract}
		This work presents the culmination of the T0 theoretical revolution: a completely ratio-based physics that eliminates the need for multiple experimental parameters. Building upon the simplified Dirac equation and universal Lagrangian insights, we demonstrate that fundamental physics operates through dimensionless energy scale ratios, not assigned parameters. The T0 system requires only one SI reference value to connect pure ratio-based physics to measurable quantities. We show that Einstein's $E = mc^2$ reveals mass as concentrated energy, leading to universal energy relations with 100\% mathematical accuracy compared to 99.98\% accuracy of complex multi-parameter formulas. All physics reduces to energy scale ratios governed by the ultimate equation $\partial^2 \Efield = 0$, with quantitative predictions made possible through a single SI reference scale $\xipar$.
	\end{abstract}
	
	
	\section{The T0 Revolution: From Parameters to Ratios}
	
	\subsection{The Fundamental Paradigm Shift}
	
	The T0 theoretical revolution represents a complete paradigm shift in how we understand fundamental physics:
	
	\subsubsection*{Paradigm Revolution}
\textbf{Traditional Physics}: Multiple experimental parameters
		\begin{itemize}
			\item $G = 6.67 \times 10^{-11}$ m³/(kg·s²) (measured)
			\item $\alpha = 1/137$ (measured)
			\item $m_e = 9.109 \times 10^{-31}$ kg (measured)
			\item 20+ independent parameters required
		\end{itemize}
		
		\textbf{T0 Ratio-Based Physics}: Dimensionless scale relations
		\begin{itemize}
			\item All physics through energy scale ratios
			\item One SI reference value for quantitative predictions
			\item Mathematical relations, not experimental parameters
			\item Pure energy identities: $E = m$, $E = 1/L$, $E = 1/T$
		\end{itemize}

	
	\subsection{Building on T0 Foundations}
	
	This work completes the three-stage T0 revolution:
	
	\textbf{Stage 1 - Simplified Dirac}: Complex 4×4 matrices → Simple field dynamics $\partial^2 \deltam = 0$
	
	\textbf{Stage 2 - Universal Lagrangian}: 20+ fields → One equation $\Lag = \varepsilon \cdot (\partial \deltam)^2$
	
	\textbf{Stage 3 - Ratio-Based Physics}: Multiple parameters → Energy scale ratios + SI reference
	
	\subsection{The Energy Identity Revolution}
	
	In natural units ($\hbar = c = 1$), Einstein's equation reveals fundamental truth:
	
	\begin{equation}
		\boxed{E = m}
		\label{Elimination_Of_:L-Elimination_Of_Mass_Dirac_LagEn-1171}
	\end{equation}
	
	This is not conversion - this is \textbf{identity}. Mass and energy are the same physical quantity.
	
	\subsubsection*{Universal Energy Relations}
\textbf{Complete Energy Identity System}:
		\begin{align}
			E &= m \quad \text{(mass is energy)} \\
			E &= T_{\text{temp}} \quad \text{(temperature is energy)} \\
			E &= \omega \quad \text{(frequency is energy)} \\
			E &= \frac{1}{L} \quad \text{(length is inverse energy)} \\
			E &= \frac{1}{T} \quad \text{(time is inverse energy)}
		\end{align}
		
		\textbf{Mathematical accuracy}: 100\% (exact identities)
		
		\textbf{Complex formulas}: 99.98-100.04\% (rounding errors accumulate)
		
		\textbf{Proof}: Simplicity is more accurate than complexity!

	
	\section{Part I: Pure Ratio-Based Physics (Parameter-Free)}
	
	\subsection{Universal Energy Field Dynamics}
	
	All particles are energy excitation patterns in the universal field $\Efield(x,t)$:
	
	\begin{equation}
		\boxed{\partial^2 \Efield = 0}
		\label{Elimination_Of_:L-T0_Energie-0338}
	\end{equation}
	
	\textbf{Universal truth}: This Klein-Gordon equation for energy describes ALL particles.
	
	\subsection{Universal Energy Lagrangian}
	
	\begin{equation}
		\boxed{\Lag = \varepsilon \cdot (\partial \Efield)^2}
		\label{Elimination_Of_:L-T0_Energie-0191}
	\end{equation}
	
	where $\varepsilon$ represents energy scale coupling (dimensionless ratio).
	
	\subsection{Antienergy: Perfect Symmetry}
	
	\begin{equation}
		\boxed{\Efield_{\text{antiparticle}} = -\Efield_{\text{particle}}}
		\label{Elimination_Of_:L-Elimination_Of_Mass_Dirac_LagEn-1172}
	\end{equation}
	
	\textbf{Physical picture}: Positive and negative energy excitations of the same field.
	
	\textbf{Lagrangian universality}:
	\begin{align}
		\Lag[+\Efield] &= \varepsilon \cdot (\partial \Efield)^2 \\
		\Lag[-\Efield] &= \varepsilon \cdot (\partial \Efield)^2
	\end{align}
	
	Same physics for particles and antiparticles through squaring operation.
	
	\subsection{Pure Ratio Predictions (No Parameters Needed)}
	
	\subsubsection{Universal Lepton Ratios}
	
	\begin{equation}
		\boxed{\frac{a_e^{(T0)}}{a_{\mu}^{(T0)}} = 1}
		\label{Elimination_Of_:L-Elimination_Of_Mass_Dirac_LagEn-1173}
	\end{equation}
	
	\textbf{Physical meaning}: All leptons receive identical energy corrections.
	
	\subsubsection{Energy-Independence Ratios}
	
	\begin{equation}
		\boxed{\frac{\Delta\Gamma^{\mu}(E_1)}{\Delta\Gamma^{\mu}(E_2)} = 1}
		\label{Elimination_Of_:L-Elimination_Of_Mass_Dirac_LagEn-1174}
	\end{equation}
	
	\textbf{Distinguishing feature}: Unlike Standard Model running couplings.
	

	\section{Part II: Quantitative Predictions (SI Reference Required)}
	
	\subsection{The SI Reference Scale}
	
	To make quantitative predictions, T0 physics requires one connection to the SI system:
	
	\subsubsection*{SI Reference Scale (Not a Parameter!)}
\textbf{Definition}: $\xipar$ is a dimensionless energy scale ratio, not an experimental parameter.
		
		\textbf{Higgs Energy Ratio}:
		\begin{equation}
			\xipar = \frac{\lambda_h^2 v^2}{16\pi^3 E_h^2}
		\end{equation}
		
		\textbf{Geometric Energy Ratio}:
		\begin{equation}
			\xipar = \frac{2\ell_P}{\lambda_C}
		\end{equation}
		
		\textbf{SI Reference Value}: $\xipar = 1.33 \times 10^{-4}$
		
		\textbf{Role}: Connects dimensionless ratios to SI measurable quantities

	
	\subsection{Quantitative Lepton Predictions}
	
	Using the SI reference scale:
	
	\begin{equation}
		a_{\ell}^{(T0)} = \frac{1}{2\pi} \times \xipar^2 \times \frac{1}{12}
		\label{Elimination_Of_:L-Elimination_Of_Mass_Dirac_LagEn-1175}
	\end{equation}
	
	\textbf{Numerical calculation}:
	\begin{align}
		a_{\ell}^{(T0)} &= \frac{1}{2\pi} \times (1.33 \times 10^{-4})^2 \times \frac{1}{12} \\
		&= \frac{1}{6.283} \times 1.77 \times 10^{-8} \times 0.0833 \\
		&= 2.47 \times 10^{-10}
	\end{align}
	
	\subsubsection*{Universal Lepton Prediction}
\textbf{Electron g-2}: $a_e^{(T0)} = 2.47 \times 10^{-10}$
		
		\textbf{Muon g-2}: $a_{\mu}^{(T0)} = 2.47 \times 10^{-10}$ (identical!)
		
		\textbf{Tau g-2}: $a_{\tau}^{(T0)} = 2.47 \times 10^{-10}$ (universal!)
		
		\textbf{Current muon anomaly}: $\Delta a_{\mu} \approx 25 \times 10^{-10}$
		
		\textbf{T0 contribution}: $\sim 10\%$ of observed anomaly

	
	\subsection{Quantitative QED Predictions}
	
	\begin{equation}
		\frac{\Delta\Gamma^{\mu}}{\Gamma^{\mu}} = \xipar^2 = 1.77 \times 10^{-8}
		\label{Elimination_Of_:L-Elimination_Of_Mass_Dirac_LagEn-1176}
	\end{equation}
	
	\textbf{Energy-independence verification}:
	\begin{table}[htbp]
		\centering
		\begin{tabular}{lcc}
			\toprule
			\textbf{Energy Scale} & \textbf{T0 Correction} & \textbf{Standard Model} \\
			\midrule
			1 MeV & $1.77 \times 10^{-8}$ & Running $\alpha(E)$ \\
			1 GeV & $1.77 \times 10^{-8}$ & Running $\alpha(E)$ \\
			100 GeV & $1.77 \times 10^{-8}$ & Running $\alpha(E)$ \\
			1 TeV & $1.77 \times 10^{-8}$ & Running $\alpha(E)$ \\
			\bottomrule
		\end{tabular}
		\caption{Energy-independent T0 corrections vs. Standard Model}
	\end{table}
	

	\section{Experimental Verification Strategy}
	
	\subsection{Pure Ratio Tests (No SI Reference Needed)}
	
	\textbf{Test 1 - Universal Lepton Ratios}:
	\begin{itemize}
		\item Measure $a_e^{(T0)}/a_{\mu}^{(T0)} = 1$
		\item Independent of absolute values
		\item Tests universality principle directly
	\end{itemize}
	
	\textbf{Test 2 - Energy Independence}:
	\begin{itemize}
		\item Measure QED corrections at different energies
		\item Ratio should be constant: $\Delta\Gamma(E_1)/\Delta\Gamma(E_2) = 1$
		\item Distinguishes from Standard Model running couplings
	\end{itemize}
	
	\textbf{Test 3 - Wavelength Ratios}:
	\begin{itemize}
		\item Multi-wavelength observations of same objects
		\item Test $z(\lambda_1)/z(\lambda_2) = \lambda_2/\lambda_1$
		\item Independent of absolute redshift calibration
	\end{itemize}
	
	\subsection{Quantitative Tests (Require SI Reference)}
	
	\textbf{Precision g-2 Measurements}:
	\begin{itemize}
		\item Electron g-2: Detect $2.47 \times 10^{-10}$ correction
		\item Muon g-2: Confirm $\sim 10\%$ of current anomaly
		\item Tau g-2: First measurement expecting same value
	\end{itemize}
	
	\textbf{Multi-Energy QED Tests}:
	\begin{itemize}
		\item Measure absolute $\Delta\Gamma/\Gamma = 1.77 \times 10^{-8}$
		\item Verify energy-independence across decades
		\item Compare with Standard Model predictions
	\end{itemize}
	
	\section{Dark Matter and Dark Energy from Energy Ratios}
	
	\subsection{Dark Matter: Subthreshold Energy Oscillations}
	
	\textbf{Ratio-based description}:
	\begin{equation}
		\frac{\Efield_{\text{dark}}}{\Efield_{\text{threshold}}} = \xipar \sqrt{\frac{\rho_{\text{local}}}{\rho_{\text{critical}}}}
	\end{equation}
	
	\textbf{Physical mechanism}: Random phase energy oscillations below particle detection threshold.
	
	\subsection{Dark Energy: Large-Scale Energy Gradients}
	
	\textbf{Ratio-based energy density}:
	\begin{equation}
		\frac{\rho_{\Lambda}}{\rho_{\text{critical}}} = \frac{1}{2} \xipar^2 \left(\frac{E_{\text{Planck}}}{L_{\text{Hubble}} \cdot E_{\text{Planck}}}\right)^2
	\end{equation}
	
	\textbf{Quantitative prediction}: $\rho_{\Lambda} \approx 6 \times 10^{-30}$ g/cm$^3$ (matches observation!)
	
	\section{Philosophical Revolution: The End of Material Physics}
	
	\subsection{Pure Energy Reality}
	
	\subsubsection*{The Ultimate Dematerialization}
\textbf{Traditional view}: Matter, energy, forces, spacetime as separate entities
		
		\textbf{T0 reality}: Only energy patterns and their ratios
		
		\textbf{What we call particles}: Localized energy concentrations
		
		\textbf{What we call forces}: Energy gradient interactions
		
		\textbf{What we call spacetime}: Energy pattern substrate
		
		\textbf{What we call consciousness}: Self-referential energy patterns
		
		\textbf{Ultimate truth}: Pure energy relationships governed by $\partial^2 \Efield = 0$

	
	\subsection{From Maximum Complexity to Ultimate Simplicity}
	
	\textbf{Physics evolution}:
	\begin{enumerate}
		\item \textbf{Ancient}: Four elements
		\item \textbf{Classical}: Particles in spacetime
		\item \textbf{Modern}: Fields and forces
		\item \textbf{Standard Model}: 20+ parameters, maximum complexity
		\item \textbf{T0 Revolution}: Energy ratios + one SI reference
		\end{enumerate}
		
		\textbf{We have reached maximum simplification}: The fewest possible fundamental assumptions.
		
		\subsection{Consciousness and Energy Patterns}
		
		\textbf{The deepest question}: If everything is energy patterns, what about consciousness?
			
			\textbf{T0 insight}: Consciousness is a self-observing energy pattern. We are temporary organizations of the universal energy field that have developed the capacity for self-reference and subjective experience.
			
			\section{The Ratio-Physics Legacy}
			
			\subsection{Revolutionary Achievements}
			
			The T0 ratio-based revolution has accomplished:
			
			\begin{enumerate}
				\item \textbf{Eliminated multiple parameters}: 20+ → 1 SI reference
					\item \textbf{Unified all forces}: Through energy gradient interactions
					\item \textbf{Solved particle proliferation}: All are energy patterns
						\item \textbf{Explained antiparticles}: Negative energy excitations
							\item \textbf{Included gravity}: Automatic through energy-spacetime coupling
								\item \textbf{Predicted dark phenomena}: Energy field effects
									\item \textbf{Achieved mathematical perfection}: 100\% accuracy
										\item \textbf{Established ratio-based physics}: Pure scale relations
										\end{enumerate}
										
										\subsection{The Two-Tier Testing Strategy}
										
										\textbf{Tier 1 - Pure Ratios} (Parameter-free):
										\begin{itemize}
											\item Universal lepton correction ratios
											\item Energy-independent QED ratios
											\item Wavelength-dependent redshift ratios
											\item Gravitational modification ratios
										\end{itemize}
										
										\textbf{Tier 2 - Quantitative Predictions} (SI reference):
										\begin{itemize}
											\item Absolute g-2 corrections
											\item Absolute QED vertex modifications
											\item Absolute cosmological parameters
											\item Absolute dark matter/energy densities
										\end{itemize}
										
										\subsection{Physics Completion Status}
										
							\subsubsection*{The End of Fundamental Physics}
\textbf{We have reached the end of the theoretical road}.
								
								\textbf{The fundamental equation}: $\partial^2 \Efield = 0$
								
								\textbf{The universal ratios}: Energy scale relationships
								
								\textbf{The SI connection}: One reference scale $\xipar$
								
								\textbf{Everything else}: Different solutions and patterns
								
								\textbf{No deeper level exists}: This is the bottom of reality
								
								\textbf{Future work}: Applications and measurements, not new fundamentals

															
															\section{Conclusion: The Ratio-Based Universe}
															
															\subsection{The Final Truth}
															
															The T0 revolution reveals that reality operates through pure energy scale ratios:
															
															\textbf{Level 1}: Dimensionless energy ratios (parameter-free physics)
															
															\textbf{Level 2}: One SI reference scale (quantitative predictions)
															
															\textbf{Level 3}: Pure energy patterns governed by $\partial^2 \Efield = 0$
															
															Everything we observe, measure, and experience emerges from this simple \\
															ratio-based structure.
															
															\subsection{The Elegant Completion}
															
															We have journeyed from the maximum complexity of traditional physics to the ultimate simplicity of ratio-based energy dynamics.
															
															\textbf{The lesson}: Nature's deepest truth is not complicated mathematics or exotic phenomena - it is the breathtaking elegance of pure scale relationships.
															
															\textbf{One field}. \textbf{One equation}. \textbf{Energy ratios}. \textbf{One SI reference}.
															
															Everything else is the infinite creativity of energy expressing itself through \\
															countless patterns and ratios, including the pattern we call human consciousness \\
															that can recognize and appreciate this cosmic mathematical harmony.
															
															\begin{equation}
																\boxed{\text{Reality} = \text{Energy ratios in } \Efield(x,t)}
															\end{equation}
															
\section*{The T0 revolution is complete. Physics is finished. The universe is pure energy ratios, and we are part of its eternal mathematical dance.}
															
															


% Bibliography
\begin{thebibliography}{99}
	
	\bibitem{pdg2024}
	Particle Data Group Collaboration (2024). 
	\textit{Review of Particle Physics}. 
	Progress of Theoretical and Experimental Physics, 2024(8), 083C01.
	\url{https://pdg.lbl.gov}
	
	\bibitem{flag2024}
	Aoki, Y., et al. (FLAG Collaboration) (2024). 
	\textit{FLAG Review 2024 of Lattice Results for Low-Energy Constants}. 
	arXiv:2411.04268.
	\url{https://arxiv.org/abs/2411.04268}
	
	\bibitem{fermilab_muon_g2}
	Abi, B., et al. (Muon g-2 Collaboration) (2021). 
	\textit{Measurement of the Positive Muon Anomalous Magnetic Moment to 0.46 ppm}. 
	Physical Review Letters, 126, 141801.
	
	\bibitem{peskin_schroeder}
	Peskin, M. E., \& Schroeder, D. V. (1995). 
	\textit{An Introduction to Quantum Field Theory}. 
	Addison-Wesley.
	
	\bibitem{weinberg_qft}
	Weinberg, S. (1995). 
	\textit{The Quantum Theory of Fields, Vol. I--III}. 
	Cambridge University Press.
	
	\bibitem{griffiths_particle}
	Griffiths, D. (2008). 
	\textit{Introduction to Elementary Particles}. 
	Wiley-VCH.
	
	\bibitem{mandl_shaw}
	Mandl, F., \& Shaw, G. (2010). 
	\textit{Quantum Field Theory (2nd ed.)}. 
	Wiley.
	
	\bibitem{srednicki_qft}
	Srednicki, M. (2007). 
	\textit{Quantum Field Theory}. 
	Cambridge University Press.
	
	\bibitem{t0_fundamentals}
	Pascher, J. (2024). 
	\textit{T0-Theory: Foundations of Time-Mass Duality}. 
	Unpublished manuscript, HTL Leonding.
	
	\bibitem{t0_fine_structure}
	Pascher, J. (2024). 
	\textit{T0-Theory: The Fine Structure Constant}. 
	Unpublished manuscript, HTL Leonding.
	
	\bibitem{t0_neutrinos}
	Pascher, J. (2024). 
	\textit{T0-Theory: Neutrino Masses and PMNS Mixing}. 
	Unpublished manuscript, HTL Leonding.
	
	\bibitem{t0_github}
	Pascher, J. (2024--2025). 
	\textit{T0-Time-Mass-Duality Repository}. 
	GitHub.
	\url{https://github.com/jpascher/T0-Time-Mass-Duality}
	
	\bibitem{lattice_qcd_review}
	Kronfeld, A. S. (2012). 
	\textit{Twenty-first Century Lattice Gauge Theory: Results from the QCD Lagrangian}. 
	Annual Review of Nuclear and Particle Science, 62, 265--284.
	
	\bibitem{neutrino_mixing_pdg}
	Particle Data Group Collaboration (2024). 
	\textit{Neutrino Masses, Mixing, and Oscillations}. 
	PDG Review 2024.
	\url{https://pdg.lbl.gov/2024/reviews/rpp2024-rev-neutrino-mixing.pdf}
	
	\bibitem{higgs_discovery}
	ATLAS and CMS Collaborations (2012). 
	\textit{Observation of a New Particle in the Search for the Standard Model Higgs Boson}. 
	Physics Letters B, 716, 1--29.
	
	\bibitem{Brannen2005}
	C. P. Brannen, ``Estimate of neutrino masses from Koide's relation'', \textit{arXiv:hep-ph/0505028} (2005).
	\url{https://arxiv.org/abs/hep-ph/0505028}
	
	\bibitem{Brannen2006}
	C. P. Brannen, ``Koide Mass Formula for Neutrinos'', \textit{arXiv:0702.0052} (2006).
	\url{http://brannenworks.com/MASSES.pdf}
	
	\bibitem{PhaseVectors2025}
	Anonymous, ``The Koide Relation and Lepton Mass Hierarchy from Phase Vectors'', \textit{rXiv:2507.0040} (2025).
	\url{https://rxiv.org/pdf/2507.0040v1.pdf}
	
	\bibitem{PDG2025}
	Particle Data Group, ``Review of Particle Physics'', \textit{Phys. Rev. D} \textbf{112} (2025) 030001.
	\url{https://pdg.lbl.gov/2025/}
	
	\bibitem{terrell2024}
	Terrell et al. (2024). 
	\textit{Single-Clock Metrology in Nature}. 
	Nature Physics.
	
	\bibitem{hossenfelder2024}
	Hossenfelder, S. (2024). 
	\textit{Single Clock Video Explanation}. 
	YouTube.
	
	\bibitem{hundert1931}
	Hundert (1931). 
	\textit{Reference Work}. 
	Publisher.
	
	\bibitem{terrell2025}
	Terrell et al. (2025). 
	\textit{Advanced Clock Synchronization Methods}. 
	Physical Review Letters.
	
	\bibitem{pascher_t0_2025}
	Pascher, J. (2025). 
	\textit{T0-Theory: Complete Framework and Applications}. 
	Unpublished manuscript, HTL Leonding.
	
	\bibitem{t0qm}
	Pascher, J. (2024). 
	\textit{T0-Theory: Quantum Mechanics Formulation}. 
	Unpublished manuscript, HTL Leonding.
	
	\bibitem{t0anomale}
	Pascher, J. (2024). 
	\textit{T0-Theory: Anomalous Magnetic Moments}. 
	Unpublished manuscript, HTL Leonding.
	
	\bibitem{muong2complete}
	Abi, B., et al. (Muon g-2 Collaboration) (2023). 
	\textit{Complete Measurement of the Positive Muon Anomalous Magnetic Moment}. 
	Physical Review Letters, 131, 161802.
	
	\bibitem{penrose2004}
	Penrose, R. (2004). 
	\textit{The Road to Reality: A Complete Guide to the Laws of the Universe}. 
	Jonathan Cape.
	
	\bibitem{planck1900}
	Planck, M. (1900). 
	\textit{On the Theory of the Energy Distribution Law of the Normal Spectrum}. 
	Verhandlungen der Deutschen Physikalischen Gesellschaft, 2, 237.
	
	\bibitem{T0Theory}
	Pascher, J. (2024). 
	\textit{T0-Theory: Fundamental Principles}. 
	Unpublished manuscript, HTL Leonding.
	
	% Additional bibliography entries for all undefined citations
	\bibitem{6g_roadmap}
	6G Research Consortium (2024).
	\textit{6G Technology Roadmap}.
	Technical Report.
	
	\bibitem{Born2013}
	Born, M. (2013).
	\textit{Einstein's Theory of Relativity}.
	Dover Publications.
	
	\bibitem{Casimir1948}
	Casimir, H. B. G. (1948).
	\textit{On the attraction between two perfectly conducting plates}.
	Proc. Kon. Ned. Akad. Wetensch. B51, 793--795.
	
	\bibitem{Einstein1905}
	Einstein, A. (1905).
	\textit{On the Electrodynamics of Moving Bodies}.
	Annalen der Physik, 17, 891--921.
	
	\bibitem{Feynman2006}
	Feynman, R. P. (2006).
	\textit{QED: The Strange Theory of Light and Matter}.
	Princeton University Press.
	
	\bibitem{Griffiths2017}
	Griffiths, D. J. (2017).
	\textit{Introduction to Electrodynamics (4th ed.)}.
	Cambridge University Press.
	
	\bibitem{Jackson1999}
	Jackson, J. D. (1999).
	\textit{Classical Electrodynamics (3rd ed.)}.
	Wiley.
	
	\bibitem{Mohr2016}
	Mohr, P. J., et al. (2016).
	\textit{CODATA Recommended Values of the Fundamental Physical Constants: 2014}.
	Rev. Mod. Phys. 88, 035009.
	
	\bibitem{Parker2018}
	Parker, R. H., et al. (2018).
	\textit{Measurement of the fine-structure constant as a test of the Standard Model}.
	Science, 360, 191--195.
	
	\bibitem{Planck1900}
	Planck, M. (1900).
	\textit{On the Theory of the Energy Distribution Law of the Normal Spectrum}.
	Verhandlungen der Deutschen Physikalischen Gesellschaft, 2, 237.
	
	\bibitem{Planck2018}
	Planck Collaboration (2018).
	\textit{Planck 2018 results. VI. Cosmological parameters}.
	Astronomy \& Astrophysics, 641, A6.
	
	\bibitem{QFT_T0}
	Pascher, J. (2024).
	\textit{T0-Theory and QFT Connections}.
	Unpublished manuscript, HTL Leonding.
	
	\bibitem{Sommerfeld1916}
	Sommerfeld, A. (1916).
	\textit{On the Quantum Theory of Spectral Lines}.
	Annalen der Physik, 51, 1--94.
	
	\bibitem{T0_Feinstruktur}
	Pascher, J. (2024).
	\textit{T0-Theory: Fine Structure Analysis}.
	Unpublished manuscript, HTL Leonding.
	
	\bibitem{T0_SI}
	Pascher, J. (2024).
	\textit{T0-Theory and SI Units}.
	Unpublished manuscript, HTL Leonding.
	
	\bibitem{T0_fine_structure}
	Pascher, J. (2024).
	\textit{T0-Theory: The Fine Structure Constant}.
	Unpublished manuscript, HTL Leonding.
	
	\bibitem{T0_g2_erweiterung}
	Pascher, J. (2024).
	\textit{T0-Theory: g-2 Extensions}.
	Unpublished manuscript, HTL Leonding.
	
	\bibitem{T0_gravitational_constant}
	Pascher, J. (2024).
	\textit{T0-Theory: Gravitational Constant Derivation}.
	Unpublished manuscript, HTL Leonding.
	
	\bibitem{T0_netze_en}
	Pascher, J. (2024).
	\textit{T0-Theory: Network Structures}.
	Unpublished manuscript, HTL Leonding.
	
	\bibitem{T0_tm_erweiterung}
	Pascher, J. (2024).
	\textit{T0-Theory: Time-Mass Extensions}.
	Unpublished manuscript, HTL Leonding.
	
	\bibitem{Uzan2003}
	Uzan, J.-P. (2003).
	\textit{The fundamental constants and their variation}.
	Rev. Mod. Phys. 75, 403--455.
	
	\bibitem{Weinberg1995}
	Weinberg, S. (1995).
	\textit{The Quantum Theory of Fields, Vol. I}.
	Cambridge University Press.
	
	\bibitem{albrecht1999}
	Albrecht, A. \& Magueijo, J. (1999).
	\textit{A time varying speed of light as a solution to cosmological puzzles}.
	Phys. Rev. D 59, 043516.
	
	\bibitem{alice2023}
	ALICE Collaboration (2023).
	\textit{Recent results from ALICE}.
	CERN-EP-2023-XXX.
	
	\bibitem{analog_optical}
	Smith, J. et al. (2024).
	\textit{Analog optical computing systems}.
	Nature Photonics.
	
	\bibitem{ashtekar2004}
	Ashtekar, A. \& Lewandowski, J. (2004).
	\textit{Background independent quantum gravity}.
	Class. Quantum Grav. 21, R53.
	
	\bibitem{atlas2023}
	ATLAS Collaboration (2023).
	\textit{ATLAS physics results}.
	CERN-PH-EP-2023-XXX.
	
	\bibitem{atlas2023higgs}
	ATLAS Collaboration (2023).
	\textit{Higgs boson measurements}.
	Phys. Rev. Lett.
	
	\bibitem{barbour1999}
	Barbour, J. (1999).
	\textit{The End of Time}.
	Oxford University Press.
	
	\bibitem{barrow1999}
	Barrow, J. D. (1999).
	\textit{Cosmologies with varying light speed}.
	Phys. Rev. D 59, 043515.
	
	\bibitem{becker2007}
	Becker, K. et al. (2007).
	\textit{String Theory and M-Theory}.
	Cambridge University Press.
	
	\bibitem{bell_muon}
	Bennett, G. W., et al. (Muon g-2 Collaboration) (2006).
	\textit{Final report of the E821 muon anomalous magnetic moment measurement}.
	Phys. Rev. D 73, 072003.
	
	\bibitem{bondi1948}
	Bondi, H. \& Gold, T. (1948).
	\textit{The steady-state theory of the expanding universe}.
	Mon. Not. R. Astron. Soc. 108, 252--270.
	
	\bibitem{brewer2019}
	Brewer, S. M. et al. (2019).
	\textit{Al+ Quantum-Logic Clock with Systematic Uncertainty below $10^{-18}$}.
	Phys. Rev. Lett. 123, 033201.
	
	\bibitem{cms2023top}
	CMS Collaboration (2023).
	\textit{Top quark measurements at CMS}.
	JHEP 2023.
	
	\bibitem{cms2024}
	CMS Collaboration (2024).
	\textit{CMS physics results 2024}.
	CERN-PH-EP-2024-XXX.
	
	\bibitem{codata2019}
	Tiesinga, E. et al. (2019).
	\textit{The 2018 CODATA Recommended Values}.
	J. Phys. Chem. Ref. Data.
	
	\bibitem{desi2025}
	DESI Collaboration (2025).
	\textit{DESI 2025 Cosmology Results}.
	arXiv preprint.
	
	\bibitem{differential_optical}
	Wang, X. et al. (2024).
	\textit{Differential optical computing}.
	Optica.
	
	\bibitem{dingle1972}
	Dingle, H. (1972).
	\textit{Science at the Crossroads}.
	Martin Brian \& O'Keeffe.
	
	\bibitem{divalentino2021}
	Di Valentino, E. et al. (2021).
	\textit{In the realm of the Hubble tension}.
	Class. Quantum Grav. 38, 153001.
	
	\bibitem{elnaschie2004}
	El Naschie, M. S. (2004).
	\textit{A review of E infinity theory}.
	Chaos, Solitons \& Fractals, 19, 209--236.
	
	\bibitem{fabrication_heterogeneous}
	Chen, Y. et al. (2024).
	\textit{Heterogeneous photonic integration}.
	Nature Electronics.
	
	\bibitem{fermilab2023}
	Fermilab (2023).
	\textit{Muon g-2 results}.
	Phys. Rev. Lett.
	
	\bibitem{flexible_wafer}
	Kim, S. et al. (2024).
	\textit{Flexible wafer-scale photonics}.
	Science Advances.
	
	\bibitem{francesco1997}
	Di Francesco, P. et al. (1997).
	\textit{Conformal Field Theory}.
	Springer.
	
	\bibitem{hartree1957}
	Hartree, D. R. (1957).
	\textit{The Calculation of Atomic Structures}.
	Wiley.
	
	\bibitem{hhi_6g}
	Fraunhofer HHI (2024).
	\textit{6G Photonic Integration}.
	Technical Report.
	
	\bibitem{hossenfelder2025}
	Hossenfelder, S. (2025).
	\textit{Science without the gobbledygook}.
	YouTube/Blog.
	
	\bibitem{hossenfelder_single_clock_video}
	Hossenfelder, S. (2024).
	\textit{The Single Clock Problem}.
	YouTube.
	
	\bibitem{hoyle1948}
	Hoyle, F. (1948).
	\textit{A new model for the expanding universe}.
	Mon. Not. R. Astron. Soc. 108, 372--382.
	
	\bibitem{integration_microelectronic}
	Liu, A. et al. (2024).
	\textit{Microelectronic photonic integration}.
	IEEE Journal.
	
	\bibitem{jacobson1995}
	Jacobson, T. (1995).
	\textit{Thermodynamics of spacetime}.
	Phys. Rev. Lett. 75, 1260.
	
	\bibitem{kasevich2023}
	Kasevich, M. et al. (2023).
	\textit{Atom interferometry tests}.
	Nature Physics.
	
	\bibitem{lerner2014}
	Lerner, E. J. (2014).
	\textit{An open letter on cosmology}.
	New Scientist.
	
	\bibitem{lisa2017}
	LISA Consortium (2017).
	\textit{Laser Interferometer Space Antenna}.
	ESA Technical Report.
	
	\bibitem{lithium_tantalate}
	Zhang, M. et al. (2024).
	\textit{Thin-film lithium tantalate photonics}.
	Nature Photonics.
	
	\bibitem{lopez2010}
	Lopez-Corredoira, M. (2010).
	\textit{Tests and problems of the standard model in cosmology}.
	Int. J. Mod. Phys. D.
	
	\bibitem{ludlow2015}
	Ludlow, A. D. et al. (2015).
	\textit{Optical atomic clocks}.
	Rev. Mod. Phys. 87, 637.
	
	\bibitem{mach1883}
	Mach, E. (1883).
	\textit{Die Mechanik in ihrer Entwickelung}.
	F.A. Brockhaus.
	
	\bibitem{maldacena1998}
	Maldacena, J. (1998).
	\textit{The large N limit of superconformal field theories}.
	Adv. Theor. Math. Phys. 2, 231--252.
	
	\bibitem{mueller2014}
	Müller, H. et al. (2014).
	\textit{Atom interferometry tests of the gravitational redshift}.
	Phys. Rev. Lett.
	
	\bibitem{mug2_final_2025}
	Muon g-2 Collaboration (2025).
	\textit{Final muon g-2 measurement}.
	Phys. Rev. Lett.
	
	\bibitem{muong2_2023}
	Muon g-2 Collaboration (2023).
	\textit{Updated muon g-2 results}.
	Phys. Rev. Lett.
	
	\bibitem{nathan2024}
	Nathan, A. et al. (2024).
	\textit{Quantum computing advances}.
	Nature.
	
	\bibitem{newell2018}
	Newell, D. B. et al. (2018).
	\textit{The CODATA 2017 values of h, e, k, and $N_A$}.
	Metrologia 55, L13.
	
	\bibitem{nottale1993}
	Nottale, L. (1993).
	\textit{Fractal Space-Time and Microphysics}.
	World Scientific.
	
	\bibitem{on_chip_lithium}
	Wang, C. et al. (2024).
	\textit{On-chip lithium niobate photonics}.
	Nature Communications.
	
	\bibitem{optical_advantages}
	Shastri, B. J. et al. (2024).
	\textit{Advantages of optical computing}.
	Nature Reviews Physics.
	
	\bibitem{pascher2025cmb}
	Pascher, J. (2025).
	\textit{T0-Theory: CMB Analysis}.
	Unpublished manuscript, HTL Leonding.
	
	\bibitem{pascher2025g2}
	Pascher, J. (2025).
	\textit{T0-Theory: g-2 Predictions}.
	Unpublished manuscript, HTL Leonding.
	
	\bibitem{pascher2025qm}
	Pascher, J. (2025).
	\textit{T0-Theory: Quantum Mechanics}.
	Unpublished manuscript, HTL Leonding.
	
	\bibitem{pascher2025si}
	Pascher, J. (2025).
	\textit{T0-Theory: SI Unit System}.
	Unpublished manuscript, HTL Leonding.
	
	\bibitem{pascher2025t0}
	Pascher, J. (2025).
	\textit{T0-Theory: Complete Framework}.
	Unpublished manuscript, HTL Leonding.
	
	\bibitem{pascher:fundamentals}
	Pascher, J. (2024).
	\textit{T0-Theory: Fundamentals}.
	Unpublished manuscript, HTL Leonding.
	
	\bibitem{pascher:g2_rev9}
	Pascher, J. (2024).
	\textit{T0-Theory: g-2 Revision 9}.
	Unpublished manuscript, HTL Leonding.
	
	\bibitem{pascher:geometric_formalism}
	Pascher, J. (2024).
	\textit{T0-Theory: Geometric Formalism}.
	Unpublished manuscript, HTL Leonding.
	
	\bibitem{pascher:ml_addendum}
	Pascher, J. (2024).
	\textit{T0-Theory: Machine Learning Addendum}.
	Unpublished manuscript, HTL Leonding.
	
	\bibitem{pascher:t0_foundations}
	Pascher, J. (2024).
	\textit{T0-Theory: Foundations}.
	Unpublished manuscript, HTL Leonding.
	
	\bibitem{pascher_derivation_beta_2025}
	Pascher, J. (2025).
	\textit{T0-Theory: Derivation of Beta}.
	Unpublished manuscript, HTL Leonding.
	
	\bibitem{pascher_higgs_connection_2025}
	Pascher, J. (2025).
	\textit{T0-Theory: Higgs Connection}.
	Unpublished manuscript, HTL Leonding.
	
	\bibitem{pascher_lagrangian_extended_2025}
	Pascher, J. (2025).
	\textit{T0-Theory: Extended Lagrangian}.
	Unpublished manuscript, HTL Leonding.
	
	\bibitem{pascher_mathematical_structure_2025}
	Pascher, J. (2025).
	\textit{T0-Theory: Mathematical Structure}.
	Unpublished manuscript, HTL Leonding.
	
	\bibitem{pascher_t0_cmb_2025}
	Pascher, J. (2025).
	\textit{T0-Theory: CMB Predictions}.
	Unpublished manuscript, HTL Leonding.
	
	\bibitem{pascher_t0_energie_2025}
	Pascher, J. (2025).
	\textit{T0-Theory: Energy}.
	Unpublished manuscript, HTL Leonding.
	
	\bibitem{pascher_t0_energy_2025}
	Pascher, J. (2025).
	\textit{T0-Theory: Energy Framework}.
	Unpublished manuscript, HTL Leonding.
	
	\bibitem{pascher_t0_theory_2025}
	Pascher, J. (2025).
	\textit{T0-Theory: Complete Theory}.
	Unpublished manuscript, HTL Leonding.
	
	\bibitem{penrose1959}
	Penrose, R. (1959).
	\textit{The apparent shape of a relativistically moving sphere}.
	Proc. Cambridge Phil. Soc. 55, 137--139.
	
	\bibitem{penrose1967}
	Penrose, R. (1967).
	\textit{Twistor algebra}.
	J. Math. Phys. 8, 345--366.
	
	\bibitem{peratt1992}
	Peratt, A. L. (1992).
	\textit{Physics of the Plasma Universe}.
	Springer-Verlag.
	
	\bibitem{peskin1995}
	Peskin, M. E. \& Schroeder, D. V. (1995).
	\textit{An Introduction to Quantum Field Theory}.
	Addison-Wesley.
	
	\bibitem{peskin_schroeder_1995}
	Peskin, M. E. \& Schroeder, D. V. (1995).
	\textit{An Introduction to Quantum Field Theory}.
	Addison-Wesley.
	
	\bibitem{phoquant}
	PhoQuant (2024).
	\textit{Photonic quantum computing}.
	Technical Report.
	
	\bibitem{photonics_ai}
	Wetzstein, G. et al. (2024).
	\textit{Photonics for AI}.
	Nature.
	
	\bibitem{planck1906}
	Planck, M. (1906).
	\textit{The Theory of Heat Radiation}.
	Johann Ambrosius Barth.
	
	\bibitem{planck2018}
	Planck Collaboration (2018).
	\textit{Planck 2018 results}.
	A\&A 641, A6.
	
	\bibitem{polchinski1998}
	Polchinski, J. (1998).
	\textit{String Theory}.
	Cambridge University Press.
	
	\bibitem{qant_nps}
	QANT (2024).
	\textit{Quantum photonics systems}.
	Technical Report.
	
	\bibitem{quantenjahr25}
	Quantenjahr (2025).
	\textit{International Year of Quantum}.
	UNESCO.
	
	\bibitem{recurrent_photonics}
	Tait, A. N. et al. (2024).
	\textit{Recurrent photonic neural networks}.
	Optica.
	
	\bibitem{rf_photonics}
	Capmany, J. \& Novak, D. (2024).
	\textit{Microwave photonics}.
	Nature Photonics.
	
	\bibitem{riess2019}
	Riess, A. G. et al. (2019).
	\textit{Large Magellanic Cloud Cepheid Standards}.
	ApJ 876, 85.
	
	\bibitem{riess2022}
	Riess, A. G. et al. (2022).
	\textit{A Comprehensive Measurement of H0}.
	ApJ 934, L7.
	
	\bibitem{rovelli2004}
	Rovelli, C. (2004).
	\textit{Quantum Gravity}.
	Cambridge University Press.
	
	\bibitem{sciama1953}
	Sciama, D. W. (1953).
	\textit{On the origin of inertia}.
	Mon. Not. R. Astron. Soc. 113, 34--42.
	
	\bibitem{sciencedaily2025}
	ScienceDaily (2025).
	\textit{Physics news}.
	Online.
	
	\bibitem{sm_g2_2025}
	Aoyama, T. et al. (2025).
	\textit{Standard Model prediction for g-2}.
	Phys. Rep.
	
	\bibitem{susskind1995}
	Susskind, L. (1995).
	\textit{The world as a hologram}.
	J. Math. Phys. 36, 6377--6396.
	
	\bibitem{t0_kosmologie}
	Pascher, J. (2024).
	\textit{T0-Theory: Cosmology}.
	Unpublished manuscript, HTL Leonding.
	
	\bibitem{terrell1959}
	Terrell, J. (1959).
	\textit{Invisibility of the Lorentz contraction}.
	Phys. Rev. 116, 1041--1045.
	
	\bibitem{terrell_single_clock_nature_2024}
	Terrell, J. et al. (2024).
	\textit{Single clock precision measurements}.
	Nature Physics.
	
	\bibitem{tfln_foundry}
	TFLN Foundry (2024).
	\textit{Thin-film lithium niobate foundry services}.
	Technical Specifications.
	
	\bibitem{thiemann2007}
	Thiemann, T. (2007).
	\textit{Modern Canonical Quantum General Relativity}.
	Cambridge University Press.
	
	\bibitem{thz_epfl}
	EPFL (2024).
	\textit{Terahertz photonics research}.
	Technical Report.
	
	\bibitem{unnikrishnan2004}
	Unnikrishnan, C. S. (2004).
	\textit{On Einstein's resolution of the twin clock paradox}.
	Current Science, 86, 704--709.
	
	\bibitem{verlinde2011}
	Verlinde, E. (2011).
	\textit{On the origin of gravity and the laws of Newton}.
	JHEP 2011, 29.
	
	\bibitem{video2025}
	Video (2025).
	\textit{Physics video explanation}.
	YouTube.
	
	\bibitem{weinberg1995}
	Weinberg, S. (1995).
	\textit{The Quantum Theory of Fields}.
	Cambridge University Press.
	
	\bibitem{weiskopf2000}
	Weiskopf, D. (2000).
	\textit{Visualization of special relativity}.
	PhD thesis, University of Tübingen.
	
	\bibitem{wheeler1990}
	Wheeler, J. A. (1990).
	\textit{A Journey into Gravity and Spacetime}.
	Scientific American Library.
	
	\bibitem{wiki_bell}
	Wikipedia (2024).
	\textit{Bell's theorem}.
	Online encyclopedia.
	
	\bibitem{zwicky1929}
	Zwicky, F. (1929).
	\textit{On the red shift of spectral lines through interstellar space}.
	Proc. Natl. Acad. Sci. 15, 773--779.

\end{thebibliography}


\end{document}
