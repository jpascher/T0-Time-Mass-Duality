\documentclass[11pt,a4paper]{article}
\usepackage[a4paper,margin=2cm]{geometry}
\usepackage[utf8]{inputenc}
\usepackage[english]{babel}
\usepackage{lmodern}
\renewcommand{\familydefault}{\sfdefault}

\usepackage{amsmath,amssymb,amsthm}
\usepackage{graphicx}
\usepackage[unicode,pdfencoding=auto,hypertexnames=false]{hyperref}
\usepackage{booktabs}
\usepackage{longtable}
\usepackage{array}
\usepackage{siunitx}
\usepackage{fancyhdr}
\usepackage{float}
\usepackage{tikz}
% tcolorbox removed for standalone
% tcbset removed
\tikzset{
  t0blue/.style={draw=blue,fill=blue!10},
  t0red/.style={draw=red,fill=red!10},
  t0green/.style={draw=green!50!black,fill=green!10},
  t0orange/.style={draw=orange,fill=orange!10},
}
\usepackage{setspace}
\usepackage{enumitem}
\usepackage{adjustbox}
\usepackage{xcolor}

% Define colors for xcolor package
\definecolor{t0green}{RGB}{34,139,34}
\definecolor{t0blue}{RGB}{0,0,255}
\definecolor{t0red}{RGB}{255,0,0}
\definecolor{t0orange}{RGB}{255,165,0}

% Define custom column types for tables
\newcolumntype{L}[1]{>{\raggedright\arraybackslash}p{#1}}
\newcolumntype{C}[1]{>{\centering\arraybackslash}p{#1}}
\newcolumntype{R}[1]{>{\raggedleft\arraybackslash}p{#1}}

\setlength{\parindent}{0pt}
\setlength{\parskip}{6pt}

\hypersetup{
  colorlinks=true,
  linkcolor=blue,
  citecolor=blue,
  urlcolor=blue
}
\pagestyle{fancy}
\setlength{\headheight}{28pt}

\newcommand{\checkmarkx}{\checkmark}
\newcommand{\warningx}{\textbf{!}}

% Makros aus Einzel-Dokumenten (Fallback-Definitionen)
\newcommand{\mytimes}{\times}
\newcommand{\myapprox}{\approx}
\newcommand{\mysim}{\sim}
\newcommand{\myomega}{\omega}
\newcommand{\mypi}{\pi}
\newcommand{\myrightarrow}{\rightarrow}
\newcommand{\mypropto}{\propto}
\newcommand{\deltafield}{\delta\phi}
\newcommand{\xipar}{\xi}
\newcommand{\xiT}{\xi}
\newcommand{\lambdah}{\lambda_h}

% Additional macros used in chapter files
\newcommand{\Kfrak}{K_{\text{frak}}}  % Fractal correction factor
\newcommand{\Dfrak}{D_f}              % Fractal dimension
\newcommand{\betapar}{\beta}          % T0 beta parameter
\newcommand{\alphapar}{\alpha}        % T0 alpha parameter
\newcommand{\Efield}{E}               % Energy field
% Note: checkmarkxa/warningxa are variants used in auto-generated chapter files
\newcommand{\checkmarkxa}{\checkmark}
\newcommand{\warningxa}{\textbf{!}}

% Additional T0-specific macros
\newcommand{\xigeom}{\xi_{\text{geom}}}  % Geometric xi
\newcommand{\lP}{\ell_P}                  % Planck length
\newcommand{\rzero}{r_0}                  % Characteristic radius
\newcommand{\xirat}{\xi_{\text{rat}}}     % Xi ratio
\newcommand{\tzero}{t_0}                  % Characteristic time
\newcommand{\natunits}{\text{(nat. units)}}  % Natural units annotation
\newcommand{\myRightarrow}{\Rightarrow}   % Arrow variant
\newcommand{\Lag}{\mathcal{L}}            % Lagrangian

% Physics macros used in chapter files
\newcommand{\CQCD}{C_{\text{QCD}}}        % QCD correction
\newcommand{\EP}{E_P}                     % Planck energy
\newcommand{\Ee}{E_e}                     % Electron energy
\newcommand{\Emu}{E_\mu}                  % Muon energy
\newcommand{\Exi}{E_\xi}                  % Xi energy
\newcommand{\Ezero}{E_0}                  % Characteristic energy
\newcommand{\Hubble}{H}                   % Hubble constant
\newcommand{\Kspec}{K_{\text{spec}}}      % Spectral correction
\newcommand{\Lambdat}{\Lambda_t}          % Time-related cosmological constant
\newcommand{\Leff}{\mathcal{L}_{\text{eff}}}  % Effective Lagrangian
\newcommand{\Lorentz}{\mathcal{L}}        % Lorentz symbol
\newcommand{\Lxi}{L_\xi}                  % Xi length
\newcommand{\Tfield}{T}                   % Time field
\newcommand{\Weyl}{W}                     % Weyl tensor/symbol
\newcommand{\alphaEMSI}{\alpha_{\text{EM,SI}}}  % EM alpha in SI
\newcommand{\alphaEMnat}{\alpha_{\text{EM,nat}}}  % EM alpha in natural units
\newcommand{\alphaem}{\alpha_{\text{em}}} % Electromagnetic alpha
\newcommand{\betaTSI}{\beta_{T,\text{SI}}}  % Beta in SI
\newcommand{\betaTnat}{\beta_{T,\text{nat}}}  % Beta in natural units
\newcommand{\deltam}{\delta m}            % Mass difference
\newcommand{\phiT}{\phi_T}                % T-field phi
\newcommand{\tP}{t_P}                     % Planck time
\newcommand{\rhoCMB}{\rho_{\text{CMB}}}   % CMB density
\newcommand{\rhoCasimir}{\rho_{\text{Casimir}}}  % Casimir density

% Table formatting
\usepackage{multirow}

% Additional physics macros
\newcommand{\Riem}{\mathcal{R}}           % Riemann tensor
\newcommand{\ZPinch}{Z_{\text{pinch}}}    % Z-pinch
\newcommand{\SynchPower}{P_{\text{synch}}} % Synchrotron power
\newcommand{\Rzero}{R_0}                  % Characteristic radius
\newcommand{\alphafine}{\alpha}           % Fine structure constant
\newcommand{\Etau}{E_\tau}                % Tau energy
\newcommand{\deltaE}{\delta E}            % Energy deviation
\newcommand{\EPlanck}{E_P}                % Planck energy
\newcommand{\pichar}{\pi}                 % Pi character
\newcommand{\alphaWSI}{\alpha_{W,\text{SI}}}  % Wien alpha in SI
\newcommand{\alphaWnat}{\alpha_{W,\text{nat}}}  % Wien alpha in natural units

% Einfache abstract-Umgebung für Kapitel:
\newenvironment{abstract}{%
  \begin{center}\bfseries Abstract\end{center}\small
}{\par}


\title{T0 nat-si En}
\author{J. Pascher}
\date{\today}

\begin{document}
\maketitle

\section*{T0 Nat Si (T0 nat-si)}

	\begin{abstract}
		The use of natural units in theoretical physics is a fundamental concept that can be comprehensively explained and contextualized within the framework of T0 theory. This treatise illuminates the principle of dimensional reduction, the advantages for calculations, the particular relevance for T0 theory, and the necessity of explicit SI units in practice. Finally, it emphasizes the deeper insight that physics ultimately rests on dimensionless geometric relationships.
	\end{abstract}
	
	
	\section{Basic Principle of Natural Units}
	\label{T0_nat_si:L-T0_nat-si-0465}
	
	\subsection{The Principle of Dimensional Reduction}
	In natural units, one sets fundamental constants to 1:
	\begin{itemize}
		\item \textbf{Speed of light}: $c = 1$
		\item \textbf{Reduced Planck constant}: $\hbar = 1$
		\item \textbf{Boltzmann constant}: $k_B = 1$
		\item \textbf{Sometimes}: $G = 1$ (Planck units)
	\end{itemize}
	
	\subsection{Mathematical Consequence}
	This does not mean that these constants ``disappear,'' but that they serve as \textbf{scale setters}:
	\begin{equation}
		E = m c^2 \quad \Rightarrow \quad E = m \quad \text{(since $c=1$)}
	\end{equation}
	\begin{equation}
		E = \hbar \omega \quad \Rightarrow \quad E = \omega \quad \text{(since $\hbar=1$)}
	\end{equation}
	
	\section{Advantages for Calculations}
	
	\subsection{Simplified Formulas}
\section*{With SI units:}
	\begin{equation}
		E = \sqrt{(p c)^2 + (m c^2)^2}
	\end{equation}
\section*{In natural units:}
	\begin{equation}
		E = \sqrt{p^2 + m^2}
	\end{equation}
	
	\subsection{Transparent Dimensional Analysis}
	All quantities can be traced back to one fundamental dimension (typically energy):
	\begin{table}[h]
		\centering
		\begin{tabular}{lll}
			\toprule
			\textbf{Quantity} & \textbf{Natural Dimension} & \textbf{SI Equivalent} \\
			\midrule
			Length & $[E]^{-1}$ & $\hbar c / E$ \\
			Time & $[E]^{-1}$ & $\hbar / E$ \\
			Mass & $[E]$ & $E/c^2$ \\
			\bottomrule
		\end{tabular}
		\caption{Dimensional relationships in natural units}
	\end{table}
	
	\section{Particular Relevance in T0 Theory}
	
	\subsection{Geometric Nature of Constants}
	T0 theory shows particularly clearly why natural units are fundamental:
	\begin{equation}
		\alpha = \xi \cdot \left( \frac{E_0}{1~\mathrm{MeV}} \right)^2
	\end{equation}
	This makes explicit that the fine structure constant is a \textbf{purely dimensionless geometric relationship}.
	
	\subsection{The -Parameter as Fundamental Geometry Factor}
	The derivation:
	\begin{equation}
		\xi = \frac{4}{3} \times 10^{-4}
	\end{equation}
	is intrinsically dimensionless and represents the fundamental space geometry -- independent of human units of measurement.
	
	\textbf{Important:} $\xi$ alone is not directly equal to $1/m_e$ or $1/E$, but requires specific scaling factors for different physical quantities.
	
	\section{Derivation of the Fundamental Scaling Factor}
	\label{T0_nat_si:L-T0_nat-si-0466}
	
	\subsection{The Fundamental Prediction of T0 Theory}
	
	T0 theory makes a remarkable prediction: the electron mass in geometric units is exactly:
	
	\begin{equation}
		m_e^{\mathrm{T0}} = 0.511
	\end{equation}
	
	This is not a convention, but a \textbf{derived consequence} of the fractal space geometry via the $\xi$ parameter.
	
	\subsection{Explicit Demonstration: Derivation vs. Reverse Calculation}
	
	Let us demonstrate explicitly that the scaling factor is derived, not reverse-calculated:
	
	\begin{align}
		\textbf{1. T0 derivation:} \quad & m_e^{\mathrm{T0}} = 0.511 \quad \text{(from $\xi$ geometry)} \\
		\textbf{2. Experimental input:} \quad & m_e^{\mathrm{SI}} = 9.1093837 \times 10^{-31}~\mathrm{kg} \quad \text{(measured independently)} \\
		\textbf{3. T0 prediction:} \quad & S_{T0} = \frac{m_e^{\mathrm{SI}}}{m_e^{\mathrm{T0}}} = 1.782662 \times 10^{-30} \\
		\textbf{4. Empirical fact:} \quad & 1~\mathrm{MeV}/c^2 = 1.782662 \times 10^{-30}~\mathrm{kg} \\
		\textbf{5. Profound conclusion:} \quad & \text{T0 theory \textbf{predicts} the MeV mass scale}
	\end{align}
	
	\subsection{Why This Is Not Circular Reasoning}
	
	Some might mistakenly think: ``You're just defining $S_{T0}$ to match $1~\mathrm{MeV}/c^2$.''
	
	This misunderstands the logical flow:
	
	\begin{itemize}
		\item \textbf{Wrong interpretation (reverse calculation)}: 
		$m_e^{\mathrm{T0}} = \dfrac{m_e^{\mathrm{SI}}}{1~\mathrm{MeV}/c^2}$ (circular)
		
		\item \textbf{Correct interpretation (derivation)}: 
		$S_{T0} = \dfrac{m_e^{\mathrm{SI}}}{m_e^{\mathrm{T0}}}$ and this \textbf{happens to equal} $1~\mathrm{MeV}/c^2$
	\end{itemize}
	
	The equality $S_{T0} = 1~\mathrm{MeV}/c^2$ is a \textbf{prediction}, not a definition.
	
	\subsection{Side-by-Side Comparison}
	
	\begin{table}[h]
		\centering
		\begin{tabular}{p{6cm}p{6cm}}
			\toprule
			\textbf{Conventional Physics} & \textbf{T0 Theory} \\
			\midrule
			$1~\mathrm{MeV}/c^2 = 1.782662\times 10^{-30}~\mathrm{kg}$ (arbitrary definition) & $m_e^{\mathrm{T0}} = 0.511$ (derived from $\xi$ geometry) \\
			$m_e = 0.511~\mathrm{MeV}/c^2$ (independent measurement) & $S_{T0} = \dfrac{m_e^{\mathrm{SI}}}{m_e^{\mathrm{T0}}}$ (fundamental scaling) \\
			Two independent facts & One \textbf{predicts} the other \\
			\bottomrule
		\end{tabular}
		\caption{Comparison of conventional vs. T0 interpretation of mass scales}
	\end{table}
	
	The remarkable fact is: \textbf{Both approaches yield identical numbers, but T0 explains why.}
	
	\subsection{The Coincidence That Isn't}
	
	What appears as a mere numerical coincidence is actually a fundamental prediction:
	
	\begin{align}
		\text{T0 prediction:} \quad & S_{T0} = \frac{m_e^{\mathrm{SI}}}{m_e^{\mathrm{T0}}} = \frac{9.1093837 \times 10^{-31}}{0.511} \\
		\text{Conventional definition:} \quad & 1~\mathrm{MeV}/c^2 = 1.782662 \times 10^{-30}~\mathrm{kg}
	\end{align}
	
	These are \textbf{identical} not by definition, but because T0 theory correctly predicts the fundamental mass scale.
	
	\subsection{The Profound Implication}
	
	\begin{center}
		\fbox{\parbox{0.8\textwidth}{
\section*{T0 theory does not ``use'' the MeV definition.}
\section*{It derives why the MeV has the mass scale it does.}
		}}
	\end{center}
	
	The conventional definition $1~\mathrm{MeV}/c^2 = 1.782662 \times 10^{-30}~\mathrm{kg}$ appears arbitrary, but T0 theory reveals it to be a consequence of fundamental geometry.
	
	\subsection{Independent Verification}
	
	We can verify this independently:
	
	\begin{itemize}
		\item \textbf{Without T0}: $1~\mathrm{MeV}/c^2 = 1.782662\times 10^{-30}~\mathrm{kg}$ (apparently arbitrary convention)
		\item \textbf{With T0}: $S_{T0} = 1.782662\times 10^{-30}$ (fundamental scaling derived from geometry)
		\item \textbf{Agreement}: The identical numerical value confirms T0's predictive power
	\end{itemize}
	
	This is analogous to how $c = 299,792,458~\mathrm{m/s}$ appears arbitrary until one understands relativity.
	
	\section{Quantized Mass Calculation in T0 Theory}
	
	\subsection{Fundamental Mass Quantization Principle}
	
	In T0 theory, particle masses are \textbf{quantized} and follow from the fundamental geometry parameter $\xi$ through discrete scaling relationships:
	
	\begin{equation}
		m_i^{\mathrm{T0}} = n_i \cdot Q_m^{\mathrm{T0}} \cdot f_i(\xi)
	\end{equation}
	
	where:
	\begin{itemize}
		\item $n_i \in \mathbb{N}$ - Quantum number (discrete)
		\item $Q_m^{\mathrm{T0}}$ - Fundamental mass quantum in T0 units
		\item $f_i(\xi)$ - Particle-specific geometry function
	\end{itemize}
	
	\subsection{Electron Mass as Reference}
	
	The electron mass serves as the fundamental reference mass:
	
	\begin{align}
		\xi_e &= \frac{4}{3} \times 10^{-4} \times f_e(1,0,1/2) \\
		m_e^{\mathrm{T0}} &= Q_m^{\mathrm{T0}} \cdot \frac{\xi}{\xi_e} = 0.511
	\end{align}
	
	\subsection{Complete Particle Mass Spectrum}
	
	For detailed derivations of all elementary particle masses within the T0 framework, including quarks, leptons, and gauge bosons, refer to the separate comprehensive treatment ``Particle Masses in T0 Theory'' which provides:
	
	\begin{itemize}
		\item Complete mass calculations for all Standard Model particles
		\item Derivation of mass quantization rules
		\item Explanation of generation patterns
		\item Comparison with experimental values
		\item Fractal renormalization procedures for precision matching
	\end{itemize}
	
	\section{Important: Explicit SI Units are Necessary for}
	\label{T0_nat_si:L-T0_nat-si-0467}
	
	\subsection{1. Experimental Verification}
	Every measurement is performed in SI units:
	\begin{itemize}
		\item Particle masses in MeV/c²
		\item Cross sections in barn
		\item Magnetic moments in $\mu_B$
	\end{itemize}
	
	\subsection{2. Technological Applications}
	\begin{itemize}
		\item Detector design (lengths in m, times in s)
		\item Accelerator technology (energies in eV)
		\item Medical physics (dosage measurements)
	\end{itemize}
	
	\subsection{3. Interdisciplinary Communication}
	\begin{itemize}
		\item Astrophysics (redshifts, Hubble constant)
		\item Materials science (lattice constants)
		\item Engineering
	\end{itemize}
	
	\section{Concrete Conversion in T0 Theory}
	\label{T0_nat_si:L-T0_nat-si-0468}
	
	\subsection{Example: Electron Mass}
\section*{In T0 geometric units:}
	\begin{equation}
		m_e^{\mathrm{T0}} = 0.511 \quad \text{(as pure geometric number derived from $\xi$)}
	\end{equation}
\section*{In SI units:}
	\begin{equation}
		m_e^{\mathrm{SI}} = m_e^{\mathrm{T0}} \cdot S_{T0} = 0.511 \cdot 1.782662 \times 10^{-30} = 9.1093837 \times 10^{-31}~\mathrm{kg}
	\end{equation}
	
	\subsection{The Fundamental Scaling Relationship}
	The conversion from T0 geometric quantities to SI units is accomplished by:
	\begin{equation}
		[\mathrm{SI}] = [\mathrm{T0}] \times S_{\text{T0}}
	\end{equation}
	where $S_{\text{T0}} = 1.782662 \times 10^{-30}$ is the fundamental scaling factor \textbf{derived} in Section~\ref{T0_nat_si:L-T0_nat-si-0466}, not defined.
	
	\section{Correct Energy Scale for the Fine Structure Constant}
	
	The fundamental relationship for the fine structure constant requires a precise energy reference:
	
	\begin{align}
		\alpha &= \xi \cdot \left( \frac{E_0}{1~\mathrm{MeV}} \right)^2 \\
		\text{with} \quad E_0 &= 7.400~\mathrm{MeV} \quad \text{(characteristic energy)}
	\end{align}
	
	This yields:
	\begin{align}
		\alpha &= 1.333333 \times 10^{-4} \cdot (7.400)^2 \\
		&= 1.333333 \times 10^{-4} \cdot 54.76 \\
		&= 7.300 \times 10^{-3} \\
		\frac{1}{\alpha} &= 137.00
	\end{align}
	
	The slight deviation from the experimental value $1/\alpha = 137.036$ is due to higher-order fractal corrections that are accounted for in the complete renormalization procedure.
	
	\section{Integration of Fractal Renormalization into Natural Units}
	
	The formulas in T0 theory fit in natural units without explicit fractal renormalization, because these units isolate the geometric essence of the theory. For exact conversions to SI units, however, fractal renormalization is essential to incorporate self-similar corrections of the vacuum geometry.
	
	\subsection{Why Do the Formulas Fit in Natural Units Without Fractal Renormalization?}
	
	In natural units, physics is reduced to a geometric, dimensionless basis (cf. Section~\ref{T0_nat_si:L-T0_nat-si-0465}). The fundamental constants serve only as a scale, and the core formulas hold approximately without additional corrections because:
	
	\begin{itemize}
		\item \textbf{The $\xi$-parameter is intrinsically dimensionless}: $\xi$ represents the pure geometry of the vacuum field and acts like a ``universal scaling factor.''
		
		\item \textbf{Approximate validity for rough calculations}: Many T0 formulas are exact in the geometric ideal form, without renormalization.
		
		\item \textbf{Example: Electron mass in natural units}:
		\begin{equation}
			m_e^{\mathrm{T0}} = 0.511 \quad \text{(geometric number, without renormalization)}
		\end{equation}
		This ``fits'' immediately because $\xi$ sets the geometric scale.
	\end{itemize}
	
	\subsection{Why is Fractal Renormalization Necessary for Exact SI Conversions?}
	
	SI units are human conventions that ``contaminate'' the geometric purity of T0 theory. To achieve exact agreement with experiments, fractal renormalization must be \textbf{explicitly applied} because:
	
	\begin{itemize}
		\item \textbf{Fractal self-similarity breaks scale invariance}
		\item \textbf{Conversion requires explicit scaling}
		\item \textbf{Cosmological reference effects}
	\end{itemize}
	
	\subsection{Mathematical Specification of Fractal Renormalization}
	
	The fractal renormalization is explicitly defined as:
	\begin{equation}
		f_{\text{fractal}}(E_0) = \prod_{n=1}^{137} \left(1 + \delta_n \cdot \xi \cdot \left(\frac{4}{3}\right)^{n-1}\right)
	\end{equation}
	where $\delta_n$ are dimensionless coefficients describing the fractal structure at each stage.
	
	\subsection{Comparison: Approximation vs. Exactness}
	
	\begin{table}[h]
		\centering
		\begin{tabular}{p{4cm}p{6cm}p{6cm}}
			\toprule
			\textbf{Aspect} & \textbf{Without fractal renormalization (T0 units)} & \textbf{With fractal renormalization (for SI conversion)} \\
			\midrule
			Accuracy & Approximate ($\sim 98$--$99$\,\%, geometrically ideal) & Exact (to $10^{-6}$, matches CODATA measurements) \\
			Example: $\alpha$ & $\alpha \approx \xi \cdot (E_0)^2 \approx 1/137$ (rough) & $\alpha = 1/137.03599\dots$ (via 137 stages) \\
			Mass calculation & $m_e^{\mathrm{T0}} = 0.511$ (geometric) & $m_e^{\mathrm{SI}} = 9.1093837\times 10^{-31}$ kg (physical) \\
			Energy scale & $E_0 = 7.400$ MeV (ideal) & $E_0 = 7.400244$ MeV (renormalized) \\
			Scaling factor & $S_{T0} = 1.782662\times 10^{-30}$ (fundamental) & $S_{T0} \cdot R_f$ (renormalized) \\
			Advantage & Fast, transparent calculations & Testability with experiments \\
			Disadvantage & Ignores fractal subtleties & Complex (iteration over resonance stages) \\
			\bottomrule
		\end{tabular}
		\caption{Comparison of geometric idealization in T0 units and physical exactness with fractal renormalization.}
		\label{T0_nat_si:L-T0_nat-si-0469}
	\end{table}
	
	\subsection{Conclusion: The Duality of Geometric Idealization and Physical Measurement}
	
	The formulas ``fit'' in T0 units without renormalization because these units capture the \textbf{geometric essence} of physics. For conversion to measurable SI units, renormalization becomes \textbf{explicitly necessary} to incorporate the \textbf{self-similar corrections} of the fractal vacuum geometry.
	
	\section{Important Conceptual Clarifications}
	
	When applying T0 theory, note these fundamental distinctions:
	
	\begin{itemize}
		\item \textbf{T0 quantities} are geometric and derived from $\xi$ (e.g., $m_e^{\mathrm{T0}} = 0.511$)
		\item \textbf{SI quantities} are physical measurements (e.g., $m_e^{\mathrm{SI}} = 9.1093837\times 10^{-31}$ kg)
		\item \textbf{$S_{T0}$} is the fundamental scaling between these realms, \textbf{derived} not defined
		\item The energy reference for $\alpha$ is exactly $E_0 = 7.400$ MeV in the geometric idealization
		\item All mass scales are \textbf{discretely quantized} in both T0 and SI representations
	\end{itemize}
	
	\section{Special Significance for T0 Theory}
	
	\subsection{The Deeper Insight}
	T0 theory reveals that natural units are not merely a calculational convenience, but express the \textbf{true geometric nature of physics}:
	\begin{itemize}
		\item \textbf{$\xi$} is the fundamental dimensionless geometry constant
		\item \textbf{$S_{T0}$} connects geometric idealization to physical measurement
		\item \textbf{T0 quantities} represent the ideal geometric forms
		\item \textbf{SI quantities} are their measurable projections into our physical reality
		\item \textbf{Particle masses} are quantized geometric patterns in both realms
	\end{itemize}
	
	\subsection{Practical Implications}
	\begin{enumerate}
		\item \textbf{Theoretical development}: Work in T0 units using geometric quantities
		\item \textbf{Fundamental scaling}: Apply $S_{T0}$ to project to physical reality
		\item \textbf{Predictions}: Convert to SI units for experimental verification
		\item \textbf{Verification}: Compare with measured SI values
		\item \textbf{Quantization}: Respect the discrete nature of all physical scales
	\end{enumerate}
	
	\section{Conclusion}
	
	T0 geometric quantities correspond to the \textbf{intrinsic language of physics}, while SI units are the \textbf{measurement language of experimentalists}. T0 theory demonstrates conclusively that the fundamental relationships of physics are dimensionless and geometric.
	
	The scaling factor $S_{T0}$ provides the essential bridge between the geometric idealization of T0 theory and the practical reality of experimental measurement. The fact that all physical constants can be derived from the single dimensionless parameter $\xi$ \textbf{with the fundamental scaling $S_{T0}$} confirms the profound truth: Physics is ultimately the mathematics of dimensionless geometric relationships with discrete quantization, projected into our measurable universe through fundamental scaling.
	
	\appendix
	\section{Notation and Symbols}
	
	\begin{table}[h]
		\centering
		\begin{tabular}{p{3cm}p{10cm}}
			\toprule
			\textbf{Symbol} & \textbf{Meaning and Explanation} \\
			\midrule
			$c$ & Speed of light in vacuum; fundamental constant of nature \\
			$\hbar$ & Reduced Planck constant \\
			$k_B$ & Boltzmann constant \\
			$G$ & Gravitational constant \\
			$E$ & Energy; in natural units dimensionally equivalent to mass and frequency \\
			$m$ & Mass; in natural units $m = E$ (since $c=1$) \\
			$p$ & Momentum; in natural units dimensionally equivalent to energy \\
			$\omega$ & Angular frequency; in natural units $\omega = E$ (since $\hbar=1$) \\
			$\alpha$ & Fine structure constant; dimensionless coupling constant \\
			$\xi$ & Fundamental geometry parameter of T0 theory; $\xi = \frac{4}{3} \times 10^{-4}$ \\
			$E_0$ & Reference energy in T0 theory; $E_0 = 7.400~\mathrm{MeV}$ \\
			$m_e^{\mathrm{T0}}$ & Electron mass in T0 units; $m_e^{\mathrm{T0}} = 0.511$ (geometric) \\
			$m_e^{\mathrm{SI}}$ & Electron mass in SI units; $m_e^{\mathrm{SI}} = 9.1093837\times 10^{-31}$ kg (physical) \\
			$[E]$ & Energy dimension; fundamental dimension in natural units \\
			SI & International System of Units (physical measurements) \\
			T0 & T0 geometric units (ideal geometric forms) \\
			$S_{T0}$ & Fundamental scaling factor; $S_{T0} = 1.782662 \times 10^{-30}$ \\
			$R_f$ & Fractal renormalization factor \\
			$f_{\text{fractal}}$ & Fractal renormalization function \\
			$Q_m^{\mathrm{T0}}$ & Fundamental mass quantum in T0 units \\
			$Q_m^{\mathrm{SI}}$ & Fundamental mass quantum in SI units \\
			$n_i$ & Quantum number for particle $i$; $n_i \in \mathbb{N}$ (discrete) \\
			$\delta_n$ & Fractal renormalization coefficients; dimensionless \\
			\bottomrule
		\end{tabular}
		\caption{Explanation of the notation and symbols used}
	\end{table}
	
	\section{Fundamental Relationships}
	
	\begin{table}[h]
		\centering
		\begin{tabular}{p{4cm}p{10cm}}
			\toprule
			\textbf{Relationship} & \textbf{Meaning} \\
			\midrule
			$E = m$ & Mass-energy equivalence (since $c=1$) \\
			$E = \omega$ & Energy-frequency relationship (since $\hbar=1$) \\
			$[L] = [T] = [E]^{-1}$ & Length and time have same dimension as inverse energy \\
			$[m] = [p] = [E]$ & Mass and momentum have same dimension as energy \\
			$\alpha = \xi (E_0/1\mathrm{MeV})^2$ & Fundamental relationship in T0 theory \\
			$m_i^{\mathrm{T0}} = n_i \cdot Q_m^{\mathrm{T0}} \cdot f_i(\xi)$ & Quantized mass formula in T0 units \\
			$m_i^{\mathrm{SI}} = m_i^{\mathrm{T0}} \cdot S_{T0}$ & Fundamental scaling to SI units \\
			$S_{T0} = \dfrac{m_e^{\mathrm{SI}}}{m_e^{\mathrm{T0}}}$ & Definition of fundamental scaling factor \\
			\bottomrule
		\end{tabular}
		\caption{Fundamental relationships in T0 theory and scaling to physical units}
	\end{table}
	
	\section{Conversion Factors}
	
	\begin{table}[h]
		\centering
		\begin{tabular}{lll}
			\toprule
			\textbf{Quantity} & \textbf{Conversion Factor} & \textbf{Value} \\
			\midrule
			$S_{T0}$ & Fundamental scaling factor & $1.782662 \times 10^{-30}$ \\
			$m_e^{\mathrm{T0}}$ & Electron mass (T0 units) & $0.511$ \\
			$m_e^{\mathrm{SI}}$ & Electron mass (SI units) & $9.1093837 \times 10^{-31}~\mathrm{kg}$ \\
			$1~\mathrm{MeV}/c^2$ & Conventional mass unit & $1.782662 \times 10^{-30}~\mathrm{kg}$ \\
			$1~\mathrm{MeV}$ & Energy in joules & $1.602176 \times 10^{-13}~\mathrm{J}$ \\
			$1~\mathrm{fm}$ & Length in natural units & $5.06773 \times 10^{-3}~\mathrm{MeV}^{-1}$ \\
			\bottomrule
		\end{tabular}
		\caption{Fundamental conversion factors between T0 geometric units and SI physical units}
	\end{table}
	


% Bibliography
\begin{thebibliography}{99}
	
	\bibitem{pdg2024}
	Particle Data Group Collaboration (2024). 
	\textit{Review of Particle Physics}. 
	Progress of Theoretical and Experimental Physics, 2024(8), 083C01.
	\url{https://pdg.lbl.gov}
	
	\bibitem{flag2024}
	Aoki, Y., et al. (FLAG Collaboration) (2024). 
	\textit{FLAG Review 2024 of Lattice Results for Low-Energy Constants}. 
	arXiv:2411.04268.
	\url{https://arxiv.org/abs/2411.04268}
	
	\bibitem{fermilab_muon_g2}
	Abi, B., et al. (Muon g-2 Collaboration) (2021). 
	\textit{Measurement of the Positive Muon Anomalous Magnetic Moment to 0.46 ppm}. 
	Physical Review Letters, 126, 141801.
	
	\bibitem{peskin_schroeder}
	Peskin, M. E., \& Schroeder, D. V. (1995). 
	\textit{An Introduction to Quantum Field Theory}. 
	Addison-Wesley.
	
	\bibitem{weinberg_qft}
	Weinberg, S. (1995). 
	\textit{The Quantum Theory of Fields, Vol. I--III}. 
	Cambridge University Press.
	
	\bibitem{griffiths_particle}
	Griffiths, D. (2008). 
	\textit{Introduction to Elementary Particles}. 
	Wiley-VCH.
	
	\bibitem{mandl_shaw}
	Mandl, F., \& Shaw, G. (2010). 
	\textit{Quantum Field Theory (2nd ed.)}. 
	Wiley.
	
	\bibitem{srednicki_qft}
	Srednicki, M. (2007). 
	\textit{Quantum Field Theory}. 
	Cambridge University Press.
	
	\bibitem{t0_fundamentals}
	Pascher, J. (2024). 
	\textit{T0-Theory: Foundations of Time-Mass Duality}. 
	Unpublished manuscript, HTL Leonding.
	
	\bibitem{t0_fine_structure}
	Pascher, J. (2024). 
	\textit{T0-Theory: The Fine Structure Constant}. 
	Unpublished manuscript, HTL Leonding.
	
	\bibitem{t0_neutrinos}
	Pascher, J. (2024). 
	\textit{T0-Theory: Neutrino Masses and PMNS Mixing}. 
	Unpublished manuscript, HTL Leonding.
	
	\bibitem{t0_github}
	Pascher, J. (2024--2025). 
	\textit{T0-Time-Mass-Duality Repository}. 
	GitHub.
	\url{https://github.com/jpascher/T0-Time-Mass-Duality}
	
	\bibitem{lattice_qcd_review}
	Kronfeld, A. S. (2012). 
	\textit{Twenty-first Century Lattice Gauge Theory: Results from the QCD Lagrangian}. 
	Annual Review of Nuclear and Particle Science, 62, 265--284.
	
	\bibitem{neutrino_mixing_pdg}
	Particle Data Group Collaboration (2024). 
	\textit{Neutrino Masses, Mixing, and Oscillations}. 
	PDG Review 2024.
	\url{https://pdg.lbl.gov/2024/reviews/rpp2024-rev-neutrino-mixing.pdf}
	
	\bibitem{higgs_discovery}
	ATLAS and CMS Collaborations (2012). 
	\textit{Observation of a New Particle in the Search for the Standard Model Higgs Boson}. 
	Physics Letters B, 716, 1--29.
	
	\bibitem{Brannen2005}
	C. P. Brannen, ``Estimate of neutrino masses from Koide's relation'', \textit{arXiv:hep-ph/0505028} (2005).
	\url{https://arxiv.org/abs/hep-ph/0505028}
	
	\bibitem{Brannen2006}
	C. P. Brannen, ``Koide Mass Formula for Neutrinos'', \textit{arXiv:0702.0052} (2006).
	\url{http://brannenworks.com/MASSES.pdf}
	
	\bibitem{PhaseVectors2025}
	Anonymous, ``The Koide Relation and Lepton Mass Hierarchy from Phase Vectors'', \textit{rXiv:2507.0040} (2025).
	\url{https://rxiv.org/pdf/2507.0040v1.pdf}
	
	\bibitem{PDG2025}
	Particle Data Group, ``Review of Particle Physics'', \textit{Phys. Rev. D} \textbf{112} (2025) 030001.
	\url{https://pdg.lbl.gov/2025/}
	
	\bibitem{terrell2024}
	Terrell et al. (2024). 
	\textit{Single-Clock Metrology in Nature}. 
	Nature Physics.
	
	\bibitem{hossenfelder2024}
	Hossenfelder, S. (2024). 
	\textit{Single Clock Video Explanation}. 
	YouTube.
	
	\bibitem{hundert1931}
	Hundert (1931). 
	\textit{Reference Work}. 
	Publisher.
	
	\bibitem{terrell2025}
	Terrell et al. (2025). 
	\textit{Advanced Clock Synchronization Methods}. 
	Physical Review Letters.
	
	\bibitem{pascher_t0_2025}
	Pascher, J. (2025). 
	\textit{T0-Theory: Complete Framework and Applications}. 
	Unpublished manuscript, HTL Leonding.
	
	\bibitem{t0qm}
	Pascher, J. (2024). 
	\textit{T0-Theory: Quantum Mechanics Formulation}. 
	Unpublished manuscript, HTL Leonding.
	
	\bibitem{t0anomale}
	Pascher, J. (2024). 
	\textit{T0-Theory: Anomalous Magnetic Moments}. 
	Unpublished manuscript, HTL Leonding.
	
	\bibitem{muong2complete}
	Abi, B., et al. (Muon g-2 Collaboration) (2023). 
	\textit{Complete Measurement of the Positive Muon Anomalous Magnetic Moment}. 
	Physical Review Letters, 131, 161802.
	
	\bibitem{penrose2004}
	Penrose, R. (2004). 
	\textit{The Road to Reality: A Complete Guide to the Laws of the Universe}. 
	Jonathan Cape.
	
	\bibitem{planck1900}
	Planck, M. (1900). 
	\textit{On the Theory of the Energy Distribution Law of the Normal Spectrum}. 
	Verhandlungen der Deutschen Physikalischen Gesellschaft, 2, 237.
	
	\bibitem{T0Theory}
	Pascher, J. (2024). 
	\textit{T0-Theory: Fundamental Principles}. 
	Unpublished manuscript, HTL Leonding.
	
	% Additional bibliography entries for all undefined citations
	\bibitem{6g_roadmap}
	6G Research Consortium (2024).
	\textit{6G Technology Roadmap}.
	Technical Report.
	
	\bibitem{Born2013}
	Born, M. (2013).
	\textit{Einstein's Theory of Relativity}.
	Dover Publications.
	
	\bibitem{Casimir1948}
	Casimir, H. B. G. (1948).
	\textit{On the attraction between two perfectly conducting plates}.
	Proc. Kon. Ned. Akad. Wetensch. B51, 793--795.
	
	\bibitem{Einstein1905}
	Einstein, A. (1905).
	\textit{On the Electrodynamics of Moving Bodies}.
	Annalen der Physik, 17, 891--921.
	
	\bibitem{Feynman2006}
	Feynman, R. P. (2006).
	\textit{QED: The Strange Theory of Light and Matter}.
	Princeton University Press.
	
	\bibitem{Griffiths2017}
	Griffiths, D. J. (2017).
	\textit{Introduction to Electrodynamics (4th ed.)}.
	Cambridge University Press.
	
	\bibitem{Jackson1999}
	Jackson, J. D. (1999).
	\textit{Classical Electrodynamics (3rd ed.)}.
	Wiley.
	
	\bibitem{Mohr2016}
	Mohr, P. J., et al. (2016).
	\textit{CODATA Recommended Values of the Fundamental Physical Constants: 2014}.
	Rev. Mod. Phys. 88, 035009.
	
	\bibitem{Parker2018}
	Parker, R. H., et al. (2018).
	\textit{Measurement of the fine-structure constant as a test of the Standard Model}.
	Science, 360, 191--195.
	
	\bibitem{Planck1900}
	Planck, M. (1900).
	\textit{On the Theory of the Energy Distribution Law of the Normal Spectrum}.
	Verhandlungen der Deutschen Physikalischen Gesellschaft, 2, 237.
	
	\bibitem{Planck2018}
	Planck Collaboration (2018).
	\textit{Planck 2018 results. VI. Cosmological parameters}.
	Astronomy \& Astrophysics, 641, A6.
	
	\bibitem{QFT_T0}
	Pascher, J. (2024).
	\textit{T0-Theory and QFT Connections}.
	Unpublished manuscript, HTL Leonding.
	
	\bibitem{Sommerfeld1916}
	Sommerfeld, A. (1916).
	\textit{On the Quantum Theory of Spectral Lines}.
	Annalen der Physik, 51, 1--94.
	
	\bibitem{T0_Feinstruktur}
	Pascher, J. (2024).
	\textit{T0-Theory: Fine Structure Analysis}.
	Unpublished manuscript, HTL Leonding.
	
	\bibitem{T0_SI}
	Pascher, J. (2024).
	\textit{T0-Theory and SI Units}.
	Unpublished manuscript, HTL Leonding.
	
	\bibitem{T0_fine_structure}
	Pascher, J. (2024).
	\textit{T0-Theory: The Fine Structure Constant}.
	Unpublished manuscript, HTL Leonding.
	
	\bibitem{T0_g2_erweiterung}
	Pascher, J. (2024).
	\textit{T0-Theory: g-2 Extensions}.
	Unpublished manuscript, HTL Leonding.
	
	\bibitem{T0_gravitational_constant}
	Pascher, J. (2024).
	\textit{T0-Theory: Gravitational Constant Derivation}.
	Unpublished manuscript, HTL Leonding.
	
	\bibitem{T0_netze_en}
	Pascher, J. (2024).
	\textit{T0-Theory: Network Structures}.
	Unpublished manuscript, HTL Leonding.
	
	\bibitem{T0_tm_erweiterung}
	Pascher, J. (2024).
	\textit{T0-Theory: Time-Mass Extensions}.
	Unpublished manuscript, HTL Leonding.
	
	\bibitem{Uzan2003}
	Uzan, J.-P. (2003).
	\textit{The fundamental constants and their variation}.
	Rev. Mod. Phys. 75, 403--455.
	
	\bibitem{Weinberg1995}
	Weinberg, S. (1995).
	\textit{The Quantum Theory of Fields, Vol. I}.
	Cambridge University Press.
	
	\bibitem{albrecht1999}
	Albrecht, A. \& Magueijo, J. (1999).
	\textit{A time varying speed of light as a solution to cosmological puzzles}.
	Phys. Rev. D 59, 043516.
	
	\bibitem{alice2023}
	ALICE Collaboration (2023).
	\textit{Recent results from ALICE}.
	CERN-EP-2023-XXX.
	
	\bibitem{analog_optical}
	Smith, J. et al. (2024).
	\textit{Analog optical computing systems}.
	Nature Photonics.
	
	\bibitem{ashtekar2004}
	Ashtekar, A. \& Lewandowski, J. (2004).
	\textit{Background independent quantum gravity}.
	Class. Quantum Grav. 21, R53.
	
	\bibitem{atlas2023}
	ATLAS Collaboration (2023).
	\textit{ATLAS physics results}.
	CERN-PH-EP-2023-XXX.
	
	\bibitem{atlas2023higgs}
	ATLAS Collaboration (2023).
	\textit{Higgs boson measurements}.
	Phys. Rev. Lett.
	
	\bibitem{barbour1999}
	Barbour, J. (1999).
	\textit{The End of Time}.
	Oxford University Press.
	
	\bibitem{barrow1999}
	Barrow, J. D. (1999).
	\textit{Cosmologies with varying light speed}.
	Phys. Rev. D 59, 043515.
	
	\bibitem{becker2007}
	Becker, K. et al. (2007).
	\textit{String Theory and M-Theory}.
	Cambridge University Press.
	
	\bibitem{bell_muon}
	Bennett, G. W., et al. (Muon g-2 Collaboration) (2006).
	\textit{Final report of the E821 muon anomalous magnetic moment measurement}.
	Phys. Rev. D 73, 072003.
	
	\bibitem{bondi1948}
	Bondi, H. \& Gold, T. (1948).
	\textit{The steady-state theory of the expanding universe}.
	Mon. Not. R. Astron. Soc. 108, 252--270.
	
	\bibitem{brewer2019}
	Brewer, S. M. et al. (2019).
	\textit{Al+ Quantum-Logic Clock with Systematic Uncertainty below $10^{-18}$}.
	Phys. Rev. Lett. 123, 033201.
	
	\bibitem{cms2023top}
	CMS Collaboration (2023).
	\textit{Top quark measurements at CMS}.
	JHEP 2023.
	
	\bibitem{cms2024}
	CMS Collaboration (2024).
	\textit{CMS physics results 2024}.
	CERN-PH-EP-2024-XXX.
	
	\bibitem{codata2019}
	Tiesinga, E. et al. (2019).
	\textit{The 2018 CODATA Recommended Values}.
	J. Phys. Chem. Ref. Data.
	
	\bibitem{desi2025}
	DESI Collaboration (2025).
	\textit{DESI 2025 Cosmology Results}.
	arXiv preprint.
	
	\bibitem{differential_optical}
	Wang, X. et al. (2024).
	\textit{Differential optical computing}.
	Optica.
	
	\bibitem{dingle1972}
	Dingle, H. (1972).
	\textit{Science at the Crossroads}.
	Martin Brian \& O'Keeffe.
	
	\bibitem{divalentino2021}
	Di Valentino, E. et al. (2021).
	\textit{In the realm of the Hubble tension}.
	Class. Quantum Grav. 38, 153001.
	
	\bibitem{elnaschie2004}
	El Naschie, M. S. (2004).
	\textit{A review of E infinity theory}.
	Chaos, Solitons \& Fractals, 19, 209--236.
	
	\bibitem{fabrication_heterogeneous}
	Chen, Y. et al. (2024).
	\textit{Heterogeneous photonic integration}.
	Nature Electronics.
	
	\bibitem{fermilab2023}
	Fermilab (2023).
	\textit{Muon g-2 results}.
	Phys. Rev. Lett.
	
	\bibitem{flexible_wafer}
	Kim, S. et al. (2024).
	\textit{Flexible wafer-scale photonics}.
	Science Advances.
	
	\bibitem{francesco1997}
	Di Francesco, P. et al. (1997).
	\textit{Conformal Field Theory}.
	Springer.
	
	\bibitem{hartree1957}
	Hartree, D. R. (1957).
	\textit{The Calculation of Atomic Structures}.
	Wiley.
	
	\bibitem{hhi_6g}
	Fraunhofer HHI (2024).
	\textit{6G Photonic Integration}.
	Technical Report.
	
	\bibitem{hossenfelder2025}
	Hossenfelder, S. (2025).
	\textit{Science without the gobbledygook}.
	YouTube/Blog.
	
	\bibitem{hossenfelder_single_clock_video}
	Hossenfelder, S. (2024).
	\textit{The Single Clock Problem}.
	YouTube.
	
	\bibitem{hoyle1948}
	Hoyle, F. (1948).
	\textit{A new model for the expanding universe}.
	Mon. Not. R. Astron. Soc. 108, 372--382.
	
	\bibitem{integration_microelectronic}
	Liu, A. et al. (2024).
	\textit{Microelectronic photonic integration}.
	IEEE Journal.
	
	\bibitem{jacobson1995}
	Jacobson, T. (1995).
	\textit{Thermodynamics of spacetime}.
	Phys. Rev. Lett. 75, 1260.
	
	\bibitem{kasevich2023}
	Kasevich, M. et al. (2023).
	\textit{Atom interferometry tests}.
	Nature Physics.
	
	\bibitem{lerner2014}
	Lerner, E. J. (2014).
	\textit{An open letter on cosmology}.
	New Scientist.
	
	\bibitem{lisa2017}
	LISA Consortium (2017).
	\textit{Laser Interferometer Space Antenna}.
	ESA Technical Report.
	
	\bibitem{lithium_tantalate}
	Zhang, M. et al. (2024).
	\textit{Thin-film lithium tantalate photonics}.
	Nature Photonics.
	
	\bibitem{lopez2010}
	Lopez-Corredoira, M. (2010).
	\textit{Tests and problems of the standard model in cosmology}.
	Int. J. Mod. Phys. D.
	
	\bibitem{ludlow2015}
	Ludlow, A. D. et al. (2015).
	\textit{Optical atomic clocks}.
	Rev. Mod. Phys. 87, 637.
	
	\bibitem{mach1883}
	Mach, E. (1883).
	\textit{Die Mechanik in ihrer Entwickelung}.
	F.A. Brockhaus.
	
	\bibitem{maldacena1998}
	Maldacena, J. (1998).
	\textit{The large N limit of superconformal field theories}.
	Adv. Theor. Math. Phys. 2, 231--252.
	
	\bibitem{mueller2014}
	Müller, H. et al. (2014).
	\textit{Atom interferometry tests of the gravitational redshift}.
	Phys. Rev. Lett.
	
	\bibitem{mug2_final_2025}
	Muon g-2 Collaboration (2025).
	\textit{Final muon g-2 measurement}.
	Phys. Rev. Lett.
	
	\bibitem{muong2_2023}
	Muon g-2 Collaboration (2023).
	\textit{Updated muon g-2 results}.
	Phys. Rev. Lett.
	
	\bibitem{nathan2024}
	Nathan, A. et al. (2024).
	\textit{Quantum computing advances}.
	Nature.
	
	\bibitem{newell2018}
	Newell, D. B. et al. (2018).
	\textit{The CODATA 2017 values of h, e, k, and $N_A$}.
	Metrologia 55, L13.
	
	\bibitem{nottale1993}
	Nottale, L. (1993).
	\textit{Fractal Space-Time and Microphysics}.
	World Scientific.
	
	\bibitem{on_chip_lithium}
	Wang, C. et al. (2024).
	\textit{On-chip lithium niobate photonics}.
	Nature Communications.
	
	\bibitem{optical_advantages}
	Shastri, B. J. et al. (2024).
	\textit{Advantages of optical computing}.
	Nature Reviews Physics.
	
	\bibitem{pascher2025cmb}
	Pascher, J. (2025).
	\textit{T0-Theory: CMB Analysis}.
	Unpublished manuscript, HTL Leonding.
	
	\bibitem{pascher2025g2}
	Pascher, J. (2025).
	\textit{T0-Theory: g-2 Predictions}.
	Unpublished manuscript, HTL Leonding.
	
	\bibitem{pascher2025qm}
	Pascher, J. (2025).
	\textit{T0-Theory: Quantum Mechanics}.
	Unpublished manuscript, HTL Leonding.
	
	\bibitem{pascher2025si}
	Pascher, J. (2025).
	\textit{T0-Theory: SI Unit System}.
	Unpublished manuscript, HTL Leonding.
	
	\bibitem{pascher2025t0}
	Pascher, J. (2025).
	\textit{T0-Theory: Complete Framework}.
	Unpublished manuscript, HTL Leonding.
	
	\bibitem{pascher:fundamentals}
	Pascher, J. (2024).
	\textit{T0-Theory: Fundamentals}.
	Unpublished manuscript, HTL Leonding.
	
	\bibitem{pascher:g2_rev9}
	Pascher, J. (2024).
	\textit{T0-Theory: g-2 Revision 9}.
	Unpublished manuscript, HTL Leonding.
	
	\bibitem{pascher:geometric_formalism}
	Pascher, J. (2024).
	\textit{T0-Theory: Geometric Formalism}.
	Unpublished manuscript, HTL Leonding.
	
	\bibitem{pascher:ml_addendum}
	Pascher, J. (2024).
	\textit{T0-Theory: Machine Learning Addendum}.
	Unpublished manuscript, HTL Leonding.
	
	\bibitem{pascher:t0_foundations}
	Pascher, J. (2024).
	\textit{T0-Theory: Foundations}.
	Unpublished manuscript, HTL Leonding.
	
	\bibitem{pascher_derivation_beta_2025}
	Pascher, J. (2025).
	\textit{T0-Theory: Derivation of Beta}.
	Unpublished manuscript, HTL Leonding.
	
	\bibitem{pascher_higgs_connection_2025}
	Pascher, J. (2025).
	\textit{T0-Theory: Higgs Connection}.
	Unpublished manuscript, HTL Leonding.
	
	\bibitem{pascher_lagrangian_extended_2025}
	Pascher, J. (2025).
	\textit{T0-Theory: Extended Lagrangian}.
	Unpublished manuscript, HTL Leonding.
	
	\bibitem{pascher_mathematical_structure_2025}
	Pascher, J. (2025).
	\textit{T0-Theory: Mathematical Structure}.
	Unpublished manuscript, HTL Leonding.
	
	\bibitem{pascher_t0_cmb_2025}
	Pascher, J. (2025).
	\textit{T0-Theory: CMB Predictions}.
	Unpublished manuscript, HTL Leonding.
	
	\bibitem{pascher_t0_energie_2025}
	Pascher, J. (2025).
	\textit{T0-Theory: Energy}.
	Unpublished manuscript, HTL Leonding.
	
	\bibitem{pascher_t0_energy_2025}
	Pascher, J. (2025).
	\textit{T0-Theory: Energy Framework}.
	Unpublished manuscript, HTL Leonding.
	
	\bibitem{pascher_t0_theory_2025}
	Pascher, J. (2025).
	\textit{T0-Theory: Complete Theory}.
	Unpublished manuscript, HTL Leonding.
	
	\bibitem{penrose1959}
	Penrose, R. (1959).
	\textit{The apparent shape of a relativistically moving sphere}.
	Proc. Cambridge Phil. Soc. 55, 137--139.
	
	\bibitem{penrose1967}
	Penrose, R. (1967).
	\textit{Twistor algebra}.
	J. Math. Phys. 8, 345--366.
	
	\bibitem{peratt1992}
	Peratt, A. L. (1992).
	\textit{Physics of the Plasma Universe}.
	Springer-Verlag.
	
	\bibitem{peskin1995}
	Peskin, M. E. \& Schroeder, D. V. (1995).
	\textit{An Introduction to Quantum Field Theory}.
	Addison-Wesley.
	
	\bibitem{peskin_schroeder_1995}
	Peskin, M. E. \& Schroeder, D. V. (1995).
	\textit{An Introduction to Quantum Field Theory}.
	Addison-Wesley.
	
	\bibitem{phoquant}
	PhoQuant (2024).
	\textit{Photonic quantum computing}.
	Technical Report.
	
	\bibitem{photonics_ai}
	Wetzstein, G. et al. (2024).
	\textit{Photonics for AI}.
	Nature.
	
	\bibitem{planck1906}
	Planck, M. (1906).
	\textit{The Theory of Heat Radiation}.
	Johann Ambrosius Barth.
	
	\bibitem{planck2018}
	Planck Collaboration (2018).
	\textit{Planck 2018 results}.
	A\&A 641, A6.
	
	\bibitem{polchinski1998}
	Polchinski, J. (1998).
	\textit{String Theory}.
	Cambridge University Press.
	
	\bibitem{qant_nps}
	QANT (2024).
	\textit{Quantum photonics systems}.
	Technical Report.
	
	\bibitem{quantenjahr25}
	Quantenjahr (2025).
	\textit{International Year of Quantum}.
	UNESCO.
	
	\bibitem{recurrent_photonics}
	Tait, A. N. et al. (2024).
	\textit{Recurrent photonic neural networks}.
	Optica.
	
	\bibitem{rf_photonics}
	Capmany, J. \& Novak, D. (2024).
	\textit{Microwave photonics}.
	Nature Photonics.
	
	\bibitem{riess2019}
	Riess, A. G. et al. (2019).
	\textit{Large Magellanic Cloud Cepheid Standards}.
	ApJ 876, 85.
	
	\bibitem{riess2022}
	Riess, A. G. et al. (2022).
	\textit{A Comprehensive Measurement of H0}.
	ApJ 934, L7.
	
	\bibitem{rovelli2004}
	Rovelli, C. (2004).
	\textit{Quantum Gravity}.
	Cambridge University Press.
	
	\bibitem{sciama1953}
	Sciama, D. W. (1953).
	\textit{On the origin of inertia}.
	Mon. Not. R. Astron. Soc. 113, 34--42.
	
	\bibitem{sciencedaily2025}
	ScienceDaily (2025).
	\textit{Physics news}.
	Online.
	
	\bibitem{sm_g2_2025}
	Aoyama, T. et al. (2025).
	\textit{Standard Model prediction for g-2}.
	Phys. Rep.
	
	\bibitem{susskind1995}
	Susskind, L. (1995).
	\textit{The world as a hologram}.
	J. Math. Phys. 36, 6377--6396.
	
	\bibitem{t0_kosmologie}
	Pascher, J. (2024).
	\textit{T0-Theory: Cosmology}.
	Unpublished manuscript, HTL Leonding.
	
	\bibitem{terrell1959}
	Terrell, J. (1959).
	\textit{Invisibility of the Lorentz contraction}.
	Phys. Rev. 116, 1041--1045.
	
	\bibitem{terrell_single_clock_nature_2024}
	Terrell, J. et al. (2024).
	\textit{Single clock precision measurements}.
	Nature Physics.
	
	\bibitem{tfln_foundry}
	TFLN Foundry (2024).
	\textit{Thin-film lithium niobate foundry services}.
	Technical Specifications.
	
	\bibitem{thiemann2007}
	Thiemann, T. (2007).
	\textit{Modern Canonical Quantum General Relativity}.
	Cambridge University Press.
	
	\bibitem{thz_epfl}
	EPFL (2024).
	\textit{Terahertz photonics research}.
	Technical Report.
	
	\bibitem{unnikrishnan2004}
	Unnikrishnan, C. S. (2004).
	\textit{On Einstein's resolution of the twin clock paradox}.
	Current Science, 86, 704--709.
	
	\bibitem{verlinde2011}
	Verlinde, E. (2011).
	\textit{On the origin of gravity and the laws of Newton}.
	JHEP 2011, 29.
	
	\bibitem{video2025}
	Video (2025).
	\textit{Physics video explanation}.
	YouTube.
	
	\bibitem{weinberg1995}
	Weinberg, S. (1995).
	\textit{The Quantum Theory of Fields}.
	Cambridge University Press.
	
	\bibitem{weiskopf2000}
	Weiskopf, D. (2000).
	\textit{Visualization of special relativity}.
	PhD thesis, University of Tübingen.
	
	\bibitem{wheeler1990}
	Wheeler, J. A. (1990).
	\textit{A Journey into Gravity and Spacetime}.
	Scientific American Library.
	
	\bibitem{wiki_bell}
	Wikipedia (2024).
	\textit{Bell's theorem}.
	Online encyclopedia.
	
	\bibitem{zwicky1929}
	Zwicky, F. (1929).
	\textit{On the red shift of spectral lines through interstellar space}.
	Proc. Natl. Acad. Sci. 15, 773--779.

\end{thebibliography}


\end{document}
