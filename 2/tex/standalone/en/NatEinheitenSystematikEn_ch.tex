\documentclass[11pt,a4paper]{article}
\usepackage[a4paper,margin=2cm]{geometry}
\usepackage[utf8]{inputenc}
\usepackage[english]{babel}
\usepackage{lmodern}
\renewcommand{\familydefault}{\sfdefault}

\usepackage{amsmath,amssymb,amsthm}
\usepackage{graphicx}
\usepackage[unicode,pdfencoding=auto,hypertexnames=false]{hyperref}
\usepackage{booktabs}
\usepackage{longtable}
\usepackage{array}
\usepackage{siunitx}
\usepackage{fancyhdr}
\usepackage{float}
\usepackage{tikz}
% tcolorbox removed for standalone
% tcbset removed
\tikzset{
  t0blue/.style={draw=blue,fill=blue!10},
  t0red/.style={draw=red,fill=red!10},
  t0green/.style={draw=green!50!black,fill=green!10},
  t0orange/.style={draw=orange,fill=orange!10},
}
\usepackage{setspace}
\usepackage{enumitem}
\usepackage{adjustbox}
\usepackage{xcolor}

% Define colors for xcolor package
\definecolor{t0green}{RGB}{34,139,34}
\definecolor{t0blue}{RGB}{0,0,255}
\definecolor{t0red}{RGB}{255,0,0}
\definecolor{t0orange}{RGB}{255,165,0}

% Define custom column types for tables
\newcolumntype{L}[1]{>{\raggedright\arraybackslash}p{#1}}
\newcolumntype{C}[1]{>{\centering\arraybackslash}p{#1}}
\newcolumntype{R}[1]{>{\raggedleft\arraybackslash}p{#1}}

\setlength{\parindent}{0pt}
\setlength{\parskip}{6pt}

\hypersetup{
  colorlinks=true,
  linkcolor=blue,
  citecolor=blue,
  urlcolor=blue
}
\pagestyle{fancy}
\setlength{\headheight}{28pt}

\newcommand{\checkmarkx}{\checkmark}
\newcommand{\warningx}{\textbf{!}}

% Makros aus Einzel-Dokumenten (Fallback-Definitionen)
\newcommand{\mytimes}{\times}
\newcommand{\myapprox}{\approx}
\newcommand{\mysim}{\sim}
\newcommand{\myomega}{\omega}
\newcommand{\mypi}{\pi}
\newcommand{\myrightarrow}{\rightarrow}
\newcommand{\mypropto}{\propto}
\newcommand{\deltafield}{\delta\phi}
\newcommand{\xipar}{\xi}
\newcommand{\xiT}{\xi}
\newcommand{\lambdah}{\lambda_h}

% Additional macros used in chapter files
\newcommand{\Kfrak}{K_{\text{frak}}}  % Fractal correction factor
\newcommand{\Dfrak}{D_f}              % Fractal dimension
\newcommand{\betapar}{\beta}          % T0 beta parameter
\newcommand{\alphapar}{\alpha}        % T0 alpha parameter
\newcommand{\Efield}{E}               % Energy field
% Note: checkmarkxa/warningxa are variants used in auto-generated chapter files
\newcommand{\checkmarkxa}{\checkmark}
\newcommand{\warningxa}{\textbf{!}}

% Additional T0-specific macros
\newcommand{\xigeom}{\xi_{\text{geom}}}  % Geometric xi
\newcommand{\lP}{\ell_P}                  % Planck length
\newcommand{\rzero}{r_0}                  % Characteristic radius
\newcommand{\xirat}{\xi_{\text{rat}}}     % Xi ratio
\newcommand{\tzero}{t_0}                  % Characteristic time
\newcommand{\natunits}{\text{(nat. units)}}  % Natural units annotation
\newcommand{\myRightarrow}{\Rightarrow}   % Arrow variant
\newcommand{\Lag}{\mathcal{L}}            % Lagrangian

% Physics macros used in chapter files
\newcommand{\CQCD}{C_{\text{QCD}}}        % QCD correction
\newcommand{\EP}{E_P}                     % Planck energy
\newcommand{\Ee}{E_e}                     % Electron energy
\newcommand{\Emu}{E_\mu}                  % Muon energy
\newcommand{\Exi}{E_\xi}                  % Xi energy
\newcommand{\Ezero}{E_0}                  % Characteristic energy
\newcommand{\Hubble}{H}                   % Hubble constant
\newcommand{\Kspec}{K_{\text{spec}}}      % Spectral correction
\newcommand{\Lambdat}{\Lambda_t}          % Time-related cosmological constant
\newcommand{\Leff}{\mathcal{L}_{\text{eff}}}  % Effective Lagrangian
\newcommand{\Lorentz}{\mathcal{L}}        % Lorentz symbol
\newcommand{\Lxi}{L_\xi}                  % Xi length
\newcommand{\Tfield}{T}                   % Time field
\newcommand{\Weyl}{W}                     % Weyl tensor/symbol
\newcommand{\alphaEMSI}{\alpha_{\text{EM,SI}}}  % EM alpha in SI
\newcommand{\alphaEMnat}{\alpha_{\text{EM,nat}}}  % EM alpha in natural units
\newcommand{\alphaem}{\alpha_{\text{em}}} % Electromagnetic alpha
\newcommand{\betaTSI}{\beta_{T,\text{SI}}}  % Beta in SI
\newcommand{\betaTnat}{\beta_{T,\text{nat}}}  % Beta in natural units
\newcommand{\deltam}{\delta m}            % Mass difference
\newcommand{\phiT}{\phi_T}                % T-field phi
\newcommand{\tP}{t_P}                     % Planck time
\newcommand{\rhoCMB}{\rho_{\text{CMB}}}   % CMB density
\newcommand{\rhoCasimir}{\rho_{\text{Casimir}}}  % Casimir density

% Table formatting
\usepackage{multirow}

% Additional physics macros
\newcommand{\Riem}{\mathcal{R}}           % Riemann tensor
\newcommand{\ZPinch}{Z_{\text{pinch}}}    % Z-pinch
\newcommand{\SynchPower}{P_{\text{synch}}} % Synchrotron power
\newcommand{\Rzero}{R_0}                  % Characteristic radius
\newcommand{\alphafine}{\alpha}           % Fine structure constant
\newcommand{\Etau}{E_\tau}                % Tau energy
\newcommand{\deltaE}{\delta E}            % Energy deviation
\newcommand{\EPlanck}{E_P}                % Planck energy
\newcommand{\pichar}{\pi}                 % Pi character
\newcommand{\alphaWSI}{\alpha_{W,\text{SI}}}  % Wien alpha in SI
\newcommand{\alphaWnat}{\alpha_{W,\text{nat}}}  % Wien alpha in natural units

% Einfache abstract-Umgebung für Kapitel:
\newenvironment{abstract}{%
  \begin{center}\bfseries Abstract\end{center}\small
}{\par}


\title{NatEinheitenSystematikEn}
\author{J. Pascher}
\date{\today}

\begin{document}
\maketitle

\section*{Nateinheitensystematiken (NatEinheitenSystematikEn)}

	\begin{abstract}
		This foundational document establishes the natural unit system used throughout the T0 model framework. By setting fundamental constants to unity and adopting energy as the base dimension, all physical quantities can be expressed as powers of energy. This document serves as the reference for unit conversions and dimensional analysis across all T0 model applications.
	\end{abstract}
	
	
	\section{List of Symbols and Notation}
	
	{\small
		\begin{table}[htbp]
			\centering
			\begin{adjustbox}{width=0.98\textwidth}
				\begin{tabular}{lll}
					\toprule
					\textbf{Symbol} & \textbf{Meaning} & \textbf{Units/Notes} \\
					\midrule
					\multicolumn{3}{c}{\textbf{Fundamental Constants}} \\
					$\hbar$ & Reduced Planck constant & Set to 1 \\
					$c$ & Speed of light & Set to 1 \\
					$G$ & Gravitational constant & Set to 1 \\
					$k_B$ & Boltzmann constant & Set to 1 \\
					$e$ & Elementary charge & $[E^0]$ (dimensionless) \\
					$\varepsilon_0, \mu_0$ & Vacuum permittivity, permeability & Set to 1 in QED units \\
					\midrule
					\multicolumn{3}{c}{\textbf{Units}} \\
					$l_P, t_P, m_P, E_P, T_P$ & Planck length, time, mass, energy, temp. & Natural base units \\
					$m_e, a_0, E_h$ & Electron mass, Bohr radius, Hartree energy & Atomic units \\
					\midrule
					\multicolumn{3}{c}{\textbf{Coupling Constants}} \\
					$\alpha_{\text{EM}}$ & Fine-structure constant & $e^2/(4\pi) = 1$ (nat.), $\approx 1/137$ (SI) \\
					$\alpha_s, \alpha_W, \alpha_G$ & Strong, weak, gravitational coupling & Dimensionless \\
					\midrule
					\multicolumn{3}{c}{\textbf{Physical Quantities}} \\
					$E, m, \Theta$ & Energy, mass, temperature & $[E]$ \\
					$L, r, \lambda, t$ & Length, radius, wavelength, time & $[E^{-1}]$ \\
					$p, \omega, \nu$ & Momentum, angular freq., frequency & $[E]$ \\
					$F$ & Force & $[E^2]$ \\
					$v$ & Velocity & Dimensionless \\
					$q$ & Electric charge & $[E^0]$ (dimensionless) \\
					\midrule
					\multicolumn{3}{c}{\textbf{Special Scales \& Notation}} \\
					$r_0, \xi$ & T0 length, scaling parameter & $\xi l_P, \xi \approx 1.33 \times 10^{-4}$ \\
					$\lambda_{C,e}, r_e$ & Compton wavelength, classical e radius & $\hbar/(m_e c), e^2/(4\pi\varepsilon_0 m_e c^2)$ \\
					$[X], [E^n]$ & Dimension of X, energy dimension & Dimensional analysis \\
					$\sim, \leftrightarrow$ & Approximately, conversion & Order of magnitude, units \\
					\bottomrule
				\end{tabular}
			\end{adjustbox}
			\caption{Symbols and notation}
			\label{NatEinheitenSys:L-T0_tm-erweiterung-x6-0030}
		\end{table}
	
	
	\section{Introduction}
	
	Natural units are unit systems where fundamental physical constants are set to unity to simplify calculations and reveal the underlying mathematical structure of physical laws. The most well-known systems are **Planck units** (for gravitation and quantum physics) and **atomic units** (for quantum chemistry).
	
	This document establishes the complete framework for the natural unit system used in the T0 model, which is based on Planck units with energy as the fundamental dimension. The key insight is that energy $[E]$ serves as the universal dimension from which all other physical quantities derive.
	
	\subsection{Comparison with Other Natural Unit Systems}
	
	\begin{table}[htbp]
		\centering
		\begin{adjustbox}{width=0.95\textwidth}
			\begin{tabular}{lllll}
				\toprule
				\textbf{System} & \textbf{Constants Set to 1} & \textbf{Base Units} & \textbf{Applications} & \textbf{Notes} \\
				\midrule
				Planck Units & $\hbar, c, G, k_B = 1$ & $l_P, t_P, m_P, E_P$ & Quantum gravity, cosmology & Universal significance \\
				Atomic Units & $m_e, e, \hbar, \frac{1}{4\pi\varepsilon_0} = 1$ & $a_0, E_h$ & Quantum chemistry, atoms & Chemistry applications \\
				Particle Physics & $\hbar, c = 1$ & GeV & High energy physics & Practical for colliders \\
				T0 Model & $\hbar, c, G, k_B = 1$ & Energy $[E]$ & Unified physics & Energy as base dimension \\
				\bottomrule
			\end{tabular}
		\end{adjustbox}
		\caption{Comparison of natural unit systems}
		\label{NatEinheitenSys:L-NatEinheitenSystematikEn-0470}
	\end{table}
	
	\section{Fundamentals of Natural Unit Systems}
	
	\subsection{Planck Units}
	
	The Planck units were proposed by Max Planck in 1899 \cite{planck1900,planck1906} and are based on the fundamental natural constants:
	\begin{align}
		G &= 1 \quad \text{(gravitational constant)} \\
		c &= 1 \quad \text{(speed of light)} \\
		\hbar &= 1 \quad \text{(reduced Planck constant)}
	\end{align}
	
	Planck recognized that these units \textit{``retain their meaning for all times and for all, including extraterrestrial and non-human cultures necessarily''} \cite{planck1900}.
	
	\subsection{Atomic Units}
	
	The atomic units, introduced by Hartree in 1927 \cite{hartree1957}, set:
	\begin{align}
		m_e &= 1 \quad \text{(electron mass)} \\
		e &= 1 \quad \text{(elementary charge)} \\
		\hbar &= 1 \\
		\frac{1}{4\pi\varepsilon_0} &= 1 \quad \text{(Coulomb constant)}
	\end{align}
	
	\subsection{Quantum Optical Units}
	
	For quantum field theory applications, quantum optical units are commonly used:
	\begin{align}
		c &= 1 \quad \text{(speed of light)} \\
		\hbar &= 1 \quad \text{(reduced Planck constant)} \\
		\varepsilon_0 &= 1 \quad \text{(permittivity)} \\
		\mu_0 &= 1 \quad \text{(permeability, because } c = 1/\sqrt{\varepsilon_0 \mu_0}\text{)}
	\end{align}
	
	\subsection{Advantages of Natural Units}
	
	Natural units offer several key advantages:
	\begin{itemize}
		\item **Simplified equations** (e.g., $E = m$ instead of $E = mc^2$)
		\item **No superfluous constants** in calculations
		\item **Universal scaling** for fundamental physics
		\item **Reveals fundamental relationships** between physical quantities
		\item **Provides dimensional consistency** checks
		\item **Eliminates arbitrary conversion factors**
		\item **Highlights the universal role** of energy
	\end{itemize}
	
	\section{Mathematical Proof of Energy Equivalence}
	
	\subsection{Fundamental Dimensional Relations}
	
	In natural units, all physical quantities have dimensions that can be expressed as powers of energy $[E]$ \cite{weinberg1995,peskin1995}:
	
	\begin{align}
		[L] &= [E]^{-1} \quad \text{(from } \hbar c = 1\text{)} \\
		[T] &= [E]^{-1} \quad \text{(from } \hbar = 1\text{)} \\
		[M] &= [E] \quad \text{(from } c = 1\text{)}
	\end{align}
	
	\subsection{Conversion of Fundamental Quantities}
	
	\textbf{Length:} From the relation $\hbar c = 1$ it follows:
	\begin{equation}
		[L] = \frac{[\hbar][c]}{[E]} = [E]^{-1}
	\end{equation}
	
	\textbf{Time:} From $\hbar = 1$ and $E = \hbar \omega$ it follows:
	\begin{equation}
		[T] = \frac{[\hbar]}{[E]} = [E]^{-1}
	\end{equation}
	
	\textbf{Mass:} From $E = mc^2$ and $c = 1$ it follows:
	\begin{equation}
		[M] = [E]
	\end{equation}
	
\section*{Velocity:}
	\begin{equation}
		[v] = \frac{[L]}{[T]} = \frac{[E]^{-1}}{[E]^{-1}} = [E]^0 = \text{dimensionless}
	\end{equation}
	
\section*{Momentum:}
	\begin{equation}
		[p] = [M][v] = [E] \cdot [E]^0 = [E]
	\end{equation}
	
\section*{Force:}
	\begin{equation}
		[F] = [M][a] = [E] \cdot [E]^{-1} = [E]^2
	\end{equation}
	
	\textbf{Charge:} In Planck units from $F = \frac{1}{4\pi\varepsilon_0} \frac{q^2}{r^2}$:
	\begin{equation}
		[q] = [E]^{1/2}
	\end{equation}
	
	\subsection{Generalization}
	
	Any physical quantity $G$ can be represented as a product of powers of the fundamental constants:
	\begin{equation}
		G = c^a \cdot \hbar^b \cdot G^c \cdot k_B^d \cdot \ldots
	\end{equation}
	
	In natural units this becomes:
	\begin{equation}
		[G] = [E]^n \quad \text{for a specific } n \in \mathbb{Q}
	\end{equation}
	
	\begin{table}[htbp]
		\centering
		\begin{adjustbox}{width=0.9\textwidth}
			\begin{tabular}{lccc}
				\toprule
				\textbf{Physical Quantity} & \textbf{SI Dimension} & \textbf{Natural Dimension} & \textbf{Derivation} \\
				\midrule
				Energy & $[ML^2T^{-2}]$ & $[E]$ & Base dimension \\
				Mass & $[M]$ & $[E]$ & $E = mc^2, c = 1$ \\
				Temperature & $[\Theta]$ & $[E]$ & $E = k_BT, k_B = 1$ \\
				Length & $[L]$ & $[E^{-1}]$ & $l_P = \sqrt{\hbar G/c^3} = 1$ \\
				Time & $[T]$ & $[E^{-1}]$ & $t_P = \sqrt{\hbar G/c^5} = 1$ \\
				Momentum & $[MLT^{-1}]$ & $[E]$ & $p = mv, v = [E^0]$ \\
				Force & $[MLT^{-2}]$ & $[E^2]$ & $F = ma = [E][E] = [E^2]$ \\
				Power & $[ML^2T^{-3}]$ & $[E^2]$ & $P = E/t = [E]/[E^{-1}] = [E^2]$ \\
						Charge & $[AT]$ & $[E^0]$ & Dimensionless in Planck units \\
				Electric Field & $[MLT^{-3}A^{-1}]$ & $[E^2]$ & $\vec{E} = \vec{F}/q$ \\
				Magnetic Field & $[MT^{-2}A^{-1}]$ & $[E^2]$ & $\vec{B} = \vec{F}/(qv)$ \\
				\bottomrule
			\end{tabular}
		\end{adjustbox}
		\caption{Universal energy dimensions of physical quantities}
		\label{NatEinheitenSys:L-NatEinheitenSystematikEn-0471}
	\end{table}
	
	\subsection{Fundamental Relationships}
	
	The key relationships in natural units become:
	\begin{align}
		E &= m \quad \text{(mass-energy equivalence)} \\
		E &= T \quad \text{(temperature-energy equivalence)} \\
		[L] &= [T] = [E^{-1}] \quad \text{(space-time unity)} \\
		\omega &= E \quad \text{(frequency-energy equivalence)} \\
		p &= E \quad \text{(momentum-energy equivalence for massless particles)}
	\end{align}
	
	\section{Length Scale Hierarchy}
	
	\subsection{Standard Length Scales}
	
	Physical systems organize themselves around characteristic length scales:
	
	\begin{table}[htbp]
		\centering
		\begin{adjustbox}{width=0.95\textwidth}
			\begin{tabular}{lccc}
				\toprule
				\textbf{Scale} & \textbf{Symbol} & \textbf{SI Value (m)} & \textbf{Natural Units ($l_P = 1$)} \\
				\midrule
				Planck Length & $l_P$ & $1.616 \times 10^{-35}$ & $1$ \\
				Compton (electron) & $\lambda_{C,e}$ & $2.426 \times 10^{-12}$ & $1.5 \times 10^{23}$ \\
				Classical electron radius & $r_e$ & $2.818 \times 10^{-15}$ & $1.7 \times 10^{20}$ \\
				Bohr radius & $a_0$ & $5.292 \times 10^{-11}$ & $3.3 \times 10^{24}$ \\
				Nuclear scale & $\sim 10^{-15}$ & $10^{-15}$ & $6.2 \times 10^{19}$ \\
				Atomic scale & $\sim 10^{-10}$ & $10^{-10}$ & $6.2 \times 10^{24}$ \\
				Human scale & $\sim 1$ & $1$ & $6.2 \times 10^{34}$ \\
				Earth radius & $R_\oplus$ & $6.371 \times 10^6$ & $3.9 \times 10^{41}$ \\
				Solar System & $\sim 10^{12}$ & $10^{12}$ & $6.2 \times 10^{46}$ \\
				Galactic scale & $\sim 10^{21}$ & $10^{21}$ & $6.2 \times 10^{55}$ \\
				\bottomrule
			\end{tabular}
		\end{adjustbox}
		\caption{Standard length scales in natural units}
		\label{NatEinheitenSys:L-NatEinheitenSystematikEn-0472}
	\end{table}
	
	\subsection{The T0 Length Scale}
	
	The T0 model introduces a sub-Planckian length scale:
	
\section*{Definition}
		\begin{equation}
			r_0 = \xi \cdot l_P
		\end{equation}
		where $\xi \approx 1.33 \times 10^{-4}$ is a dimensionless parameter.
% end box definition
	
	This gives:
	\begin{align}
		r_0 &= \xi \cdot l_P = 1.33 \times 10^{-4} \times 1.616 \times 10^{-35}\,\text{m} \\
		&= 2.15 \times 10^{-39}\,\text{m}
	\end{align}
	
	In natural units with $l_P = 1$:
	\begin{equation}
		r_0 = \xi \approx 1.33 \times 10^{-4}
	\end{equation}
	
	\section{Unit Conversions}
	
	\subsection{Energy as Reference}
	
	Using the electronvolt (eV) as the practical energy unit:
	
	\begin{table}[htbp]
		\centering
		\begin{adjustbox}{width=0.9\textwidth}
			\begin{tabular}{lll}
				\toprule
				\textbf{Physical Quantity} & \textbf{Conversion to SI} & \textbf{Example (1 GeV)} \\
				\midrule
				Energy & $\SI{1}{\electronvolt} = \SI{1.602e-19}{\joule}$ & $\SI{1.602e-10}{\joule}$ \\
				Mass & $E(\text{eV}) \times \SI{1.783e-36}{\kilogram\per\electronvolt}$ & $\SI{1.783e-27}{\kilogram}$ \\
				Length & $E(\text{eV})^{-1} \times \SI{1.973e-7}{\meter\electronvolt}$ & $\SI{1.973e-16}{\meter}$ \\
				Time & $E(\text{eV})^{-1} \times \SI{6.582e-16}{\second\electronvolt}$ & $\SI{6.582e-25}{\second}$ \\
				Temperature & $E(\text{eV}) \times \SI{1.161e4}{\kelvin\per\electronvolt}$ & $\SI{1.161e13}{\kelvin}$ \\
				\bottomrule
			\end{tabular}
		\end{adjustbox}
		\caption{Conversion factors from natural to SI units}
		\label{NatEinheitenSys:L-NatEinheitenSystematikEn-0473}
	\end{table}
	
	\subsection{Planck Scale Conversions}
	
	Converting between Planck units and SI:
	
	\begin{table}[htbp]
		\centering
		\begin{adjustbox}{width=0.8\textwidth}
			\begin{tabular}{lll}
				\toprule
				\textbf{Planck Unit} & \textbf{Natural Value} & \textbf{SI Value} \\
				\midrule
				Length ($l_P$) & $1$ & $\SI{1.616e-35}{\meter}$ \\
				Time ($t_P$) & $1$ & $\SI{5.391e-44}{\second}$ \\
				Mass ($m_P$) & $1$ & $\SI{2.176e-8}{\kilogram}$ \\
				Energy ($E_P$) & $1$ & $\SI{1.220e19}{\giga\electronvolt}$ \\
				Temperature ($T_P$) & $1$ & $\SI{1.417e32}{\kelvin}$ \\
				\bottomrule
			\end{tabular}
		\end{adjustbox}
		\caption{Planck unit conversions}
		\label{NatEinheitenSys:L-NatEinheitenSystematikEn-0474}
	\end{table}
	
	\section{Mathematical Framework}
	
	\subsection{Simplified Equations}
	
	In natural units, fundamental equations become elegantly simple:
	
	\subsubsection{Quantum Mechanics}
	\begin{align}
		\text{Schrödinger equation:} \quad & i\frac{\partial\psi}{\partial t} = H\psi \\
		\text{Uncertainty principle:} \quad & \Delta E \Delta t \geq \frac{1}{2} \\
		\text{de Broglie relation:} \quad & \lambda = \frac{1}{p}
	\end{align}
	
	\subsubsection{Special Relativity}
	\begin{align}
		\text{Mass-energy:} \quad & E = m \\
		\text{Energy-momentum:} \quad & E^2 = p^2 + m^2 \\
		\text{Lorentz factor:} \quad & \gamma = \frac{1}{\sqrt{1-v^2}}
	\end{align}
	
	\subsubsection{General Relativity}
	\begin{align}
		\text{Einstein equations:} \quad & G_{\mu\nu} = 8\pi T_{\mu\nu} \\
		\text{Schwarzschild radius:} \quad & r_s = 2M
	\end{align}
	
	\subsubsection{Electromagnetism}
	\begin{align}
		\text{Coulomb's law:} \quad & F = \frac{q_1 q_2}{4\pi r^2} \\
		\text{Fine structure constant:} \quad & \alpha = \frac{e^2}{4\pi}
		\text{(with } 4\pi\varepsilon_0 = 1\text{)}
	\end{align}
	
	\subsubsection{Thermodynamics}
	\begin{align}
		\text{Stefan-Boltzmann:} \quad & j = \sigma T^4 \\
		\text{Wien's law:} \quad & \lambda_{max} T = b \\
		\text{Boltzmann distribution:} \quad & P \propto e^{-E/T}
	\end{align}
	
	\section{Advantages and Applications}
	
	\subsection{Advantages of Natural Units}
	\begin{itemize}
		\item **Simplified equations** (e.g., $E = m$ instead of $E = mc^2$)
		\item **No superfluous constants** in calculations
		\item **Universal scaling** for fundamental physics
		\item **Reveals fundamental relationships** between physical quantities
		\item **Provides dimensional consistency** checks
		\item **Eliminates arbitrary conversion factors**
		\item **Highlights the universal role** of energy
	\end{itemize}
	
	\subsection{Disadvantages}
	\begin{itemize}
		\item **Unintuitive for macroscopic applications**
		\item **Conversion to SI requires knowledge** of fundamental constants
		\item **Initial unfamiliarity** for those used to SI units
		\item **Engineering preference** for practical SI units
	\end{itemize}
	
	\subsection{Practical Applications}
	\begin{itemize}
		\item Particle physics calculations
		\item Quantum field theory
		\item General relativity and cosmology
		\item High-energy astrophysics
		\item String theory and quantum gravity
		\item Fundamental constant relationships
	\end{itemize}
	
	\section{Working with Natural Units}
	
	\subsection{Working with Natural Units}
	
	To convert a calculation from SI to natural units:
	\begin{enumerate}
		\item Express all quantities in terms of energy (eV or GeV)
		\item Set $\hbar = c = G = k_B = 1$
		\item Perform the calculation
		\item Convert results back to SI if needed
	\end{enumerate}
	
	\subsection{Dimensional Check}
	
	Always verify dimensional consistency:
	\begin{itemize}
		\item All terms in an equation must have the same energy dimension
		\item Check that exponents are consistent
		\item Use dimensional analysis to verify results
	\end{itemize}
	
	\subsection{Fundamental Forces in Natural Units}
	
	The four fundamental forces can be characterized by their dimensionless coupling constants:
	
	\begin{table}[htbp]
		\centering
		\begin{adjustbox}{width=0.9\textwidth}
			\begin{tabular}{llll}
				\toprule
				\textbf{Force} & \textbf{Dimensionless Coupling} & \textbf{Typical Value} & \textbf{Range} \\
				\midrule
				Electromagnetic & $\alpha_{\text{EM}}$ & $\sim 1/137$ & $\infty$ \\
				Strong & $\alpha_s$ & $\sim 0.118$ at $Q^2 = M_Z^2$ & $\sim \SI{1e-15}{\meter}$ \\
				Weak & $\alpha_W = g^2/(4\pi)$ & $\sim 1/30$ & $\sim \SI{1e-18}{\meter}$ \\
				Gravitation & $\alpha_G = G m^2/(\hbar c)$ & $m^2/m_P^2$ & $\infty$ \\
				\bottomrule
			\end{tabular}
		\end{adjustbox}
		\caption{Fundamental forces characterized by coupling constants}
		\label{NatEinheitenSys:L-NatEinheitenSystematikEn-0475}
	\end{table}
	
	\subsection{Comprehensive Unit Conversions}
	
	\begin{table}[htbp]
		\centering
		\begin{adjustbox}{width=0.95\textwidth}
			\begin{tabular}{lcccc}
				\toprule
				\textbf{SI Unit} & \textbf{SI Dimension} & \textbf{Natural Dimension} & \textbf{Conversion} & \textbf{Accuracy} \\
				\midrule
				Meter & $[L]$ & $[E^{-1}]$ & $\SI{1}{\meter} \leftrightarrow (\SI{197}{\mega\electronvolt})^{-1}$ & $< 0.001\%$ \\
				Second & $[T]$ & $[E^{-1}]$ & $\SI{1}{\second} \leftrightarrow (\SI{6.58e-22}{\mega\electronvolt})^{-1}$ & $< 0.00001\%$ \\
				Kilogram & $[M]$ & $[E]$ & $\SI{1}{\kilogram} \leftrightarrow \SI{5.61e26}{\mega\electronvolt}$ & $< 0.001\%$ \\
				Ampere & $[I]$ & $[E]^{1/2}$ & $\SI{1}{\ampere} \leftrightarrow (\SI{6.24e18}{\electronvolt})^{1/2}/\si{\second}$ & $< 0.005\%$ \\
				Kelvin & $[\Theta]$ & $[E]$ & $\SI{1}{\kelvin} \leftrightarrow \SI{8.62e-5}{\electronvolt}$ & $< 0.01\%$ \\
				Volt & $[ML^2 T^{-3} I^{-1}]$ & $[E]$ & $\SI{1}{\volt} \leftrightarrow \SI{1}{\electronvolt}/e$ & $< 0.0001\%$ \\
				Coulomb & $[T I]$ & $[E^0]$ & $\SI{1}{\coulomb} \leftrightarrow 6.24 \times 10^{18} \, e$ & $< 0.0001\%$ \\
				\bottomrule
			\end{tabular}
		\end{adjustbox}
		\caption{Comprehensive unit conversions from SI to natural units}
		\label{NatEinheitenSys:L-NatEinheitenSystematikEn-0476}
	\end{table}
	
	\section{Conclusion}
	
	This natural unit system provides the foundation for all T0 model calculations. By establishing energy as the universal dimension and setting fundamental constants to unity, we reveal the underlying unity of physical laws across all scales from the sub-Planckian T0 length to cosmological distances.
	
	Key principles:
	\begin{enumerate}
		\item Energy is the fundamental dimension
		\item All physical quantities are powers of energy
		\item The T0 length extends physics below the Planck scale
		\item Natural units simplify fundamental equations
		\item Dimensional consistency is paramount
	\end{enumerate}
	
	This framework serves as the basis for all further developments in the T0 model, providing both computational tools and conceptual insights into the nature of physical reality.
	
	\bibliographystyle{plain}
	


% Bibliography
\begin{thebibliography}{99}
	
	\bibitem{pdg2024}
	Particle Data Group Collaboration (2024). 
	\textit{Review of Particle Physics}. 
	Progress of Theoretical and Experimental Physics, 2024(8), 083C01.
	\url{https://pdg.lbl.gov}
	
	\bibitem{flag2024}
	Aoki, Y., et al. (FLAG Collaboration) (2024). 
	\textit{FLAG Review 2024 of Lattice Results for Low-Energy Constants}. 
	arXiv:2411.04268.
	\url{https://arxiv.org/abs/2411.04268}
	
	\bibitem{fermilab_muon_g2}
	Abi, B., et al. (Muon g-2 Collaboration) (2021). 
	\textit{Measurement of the Positive Muon Anomalous Magnetic Moment to 0.46 ppm}. 
	Physical Review Letters, 126, 141801.
	
	\bibitem{peskin_schroeder}
	Peskin, M. E., \& Schroeder, D. V. (1995). 
	\textit{An Introduction to Quantum Field Theory}. 
	Addison-Wesley.
	
	\bibitem{weinberg_qft}
	Weinberg, S. (1995). 
	\textit{The Quantum Theory of Fields, Vol. I--III}. 
	Cambridge University Press.
	
	\bibitem{griffiths_particle}
	Griffiths, D. (2008). 
	\textit{Introduction to Elementary Particles}. 
	Wiley-VCH.
	
	\bibitem{mandl_shaw}
	Mandl, F., \& Shaw, G. (2010). 
	\textit{Quantum Field Theory (2nd ed.)}. 
	Wiley.
	
	\bibitem{srednicki_qft}
	Srednicki, M. (2007). 
	\textit{Quantum Field Theory}. 
	Cambridge University Press.
	
	\bibitem{t0_fundamentals}
	Pascher, J. (2024). 
	\textit{T0-Theory: Foundations of Time-Mass Duality}. 
	Unpublished manuscript, HTL Leonding.
	
	\bibitem{t0_fine_structure}
	Pascher, J. (2024). 
	\textit{T0-Theory: The Fine Structure Constant}. 
	Unpublished manuscript, HTL Leonding.
	
	\bibitem{t0_neutrinos}
	Pascher, J. (2024). 
	\textit{T0-Theory: Neutrino Masses and PMNS Mixing}. 
	Unpublished manuscript, HTL Leonding.
	
	\bibitem{t0_github}
	Pascher, J. (2024--2025). 
	\textit{T0-Time-Mass-Duality Repository}. 
	GitHub.
	\url{https://github.com/jpascher/T0-Time-Mass-Duality}
	
	\bibitem{lattice_qcd_review}
	Kronfeld, A. S. (2012). 
	\textit{Twenty-first Century Lattice Gauge Theory: Results from the QCD Lagrangian}. 
	Annual Review of Nuclear and Particle Science, 62, 265--284.
	
	\bibitem{neutrino_mixing_pdg}
	Particle Data Group Collaboration (2024). 
	\textit{Neutrino Masses, Mixing, and Oscillations}. 
	PDG Review 2024.
	\url{https://pdg.lbl.gov/2024/reviews/rpp2024-rev-neutrino-mixing.pdf}
	
	\bibitem{higgs_discovery}
	ATLAS and CMS Collaborations (2012). 
	\textit{Observation of a New Particle in the Search for the Standard Model Higgs Boson}. 
	Physics Letters B, 716, 1--29.
	
	\bibitem{Brannen2005}
	C. P. Brannen, ``Estimate of neutrino masses from Koide's relation'', \textit{arXiv:hep-ph/0505028} (2005).
	\url{https://arxiv.org/abs/hep-ph/0505028}
	
	\bibitem{Brannen2006}
	C. P. Brannen, ``Koide Mass Formula for Neutrinos'', \textit{arXiv:0702.0052} (2006).
	\url{http://brannenworks.com/MASSES.pdf}
	
	\bibitem{PhaseVectors2025}
	Anonymous, ``The Koide Relation and Lepton Mass Hierarchy from Phase Vectors'', \textit{rXiv:2507.0040} (2025).
	\url{https://rxiv.org/pdf/2507.0040v1.pdf}
	
	\bibitem{PDG2025}
	Particle Data Group, ``Review of Particle Physics'', \textit{Phys. Rev. D} \textbf{112} (2025) 030001.
	\url{https://pdg.lbl.gov/2025/}
	
	\bibitem{terrell2024}
	Terrell et al. (2024). 
	\textit{Single-Clock Metrology in Nature}. 
	Nature Physics.
	
	\bibitem{hossenfelder2024}
	Hossenfelder, S. (2024). 
	\textit{Single Clock Video Explanation}. 
	YouTube.
	
	\bibitem{hundert1931}
	Hundert (1931). 
	\textit{Reference Work}. 
	Publisher.
	
	\bibitem{terrell2025}
	Terrell et al. (2025). 
	\textit{Advanced Clock Synchronization Methods}. 
	Physical Review Letters.
	
	\bibitem{pascher_t0_2025}
	Pascher, J. (2025). 
	\textit{T0-Theory: Complete Framework and Applications}. 
	Unpublished manuscript, HTL Leonding.
	
	\bibitem{t0qm}
	Pascher, J. (2024). 
	\textit{T0-Theory: Quantum Mechanics Formulation}. 
	Unpublished manuscript, HTL Leonding.
	
	\bibitem{t0anomale}
	Pascher, J. (2024). 
	\textit{T0-Theory: Anomalous Magnetic Moments}. 
	Unpublished manuscript, HTL Leonding.
	
	\bibitem{muong2complete}
	Abi, B., et al. (Muon g-2 Collaboration) (2023). 
	\textit{Complete Measurement of the Positive Muon Anomalous Magnetic Moment}. 
	Physical Review Letters, 131, 161802.
	
	\bibitem{penrose2004}
	Penrose, R. (2004). 
	\textit{The Road to Reality: A Complete Guide to the Laws of the Universe}. 
	Jonathan Cape.
	
	\bibitem{planck1900}
	Planck, M. (1900). 
	\textit{On the Theory of the Energy Distribution Law of the Normal Spectrum}. 
	Verhandlungen der Deutschen Physikalischen Gesellschaft, 2, 237.
	
	\bibitem{T0Theory}
	Pascher, J. (2024). 
	\textit{T0-Theory: Fundamental Principles}. 
	Unpublished manuscript, HTL Leonding.
	
	% Additional bibliography entries for all undefined citations
	\bibitem{6g_roadmap}
	6G Research Consortium (2024).
	\textit{6G Technology Roadmap}.
	Technical Report.
	
	\bibitem{Born2013}
	Born, M. (2013).
	\textit{Einstein's Theory of Relativity}.
	Dover Publications.
	
	\bibitem{Casimir1948}
	Casimir, H. B. G. (1948).
	\textit{On the attraction between two perfectly conducting plates}.
	Proc. Kon. Ned. Akad. Wetensch. B51, 793--795.
	
	\bibitem{Einstein1905}
	Einstein, A. (1905).
	\textit{On the Electrodynamics of Moving Bodies}.
	Annalen der Physik, 17, 891--921.
	
	\bibitem{Feynman2006}
	Feynman, R. P. (2006).
	\textit{QED: The Strange Theory of Light and Matter}.
	Princeton University Press.
	
	\bibitem{Griffiths2017}
	Griffiths, D. J. (2017).
	\textit{Introduction to Electrodynamics (4th ed.)}.
	Cambridge University Press.
	
	\bibitem{Jackson1999}
	Jackson, J. D. (1999).
	\textit{Classical Electrodynamics (3rd ed.)}.
	Wiley.
	
	\bibitem{Mohr2016}
	Mohr, P. J., et al. (2016).
	\textit{CODATA Recommended Values of the Fundamental Physical Constants: 2014}.
	Rev. Mod. Phys. 88, 035009.
	
	\bibitem{Parker2018}
	Parker, R. H., et al. (2018).
	\textit{Measurement of the fine-structure constant as a test of the Standard Model}.
	Science, 360, 191--195.
	
	\bibitem{Planck1900}
	Planck, M. (1900).
	\textit{On the Theory of the Energy Distribution Law of the Normal Spectrum}.
	Verhandlungen der Deutschen Physikalischen Gesellschaft, 2, 237.
	
	\bibitem{Planck2018}
	Planck Collaboration (2018).
	\textit{Planck 2018 results. VI. Cosmological parameters}.
	Astronomy \& Astrophysics, 641, A6.
	
	\bibitem{QFT_T0}
	Pascher, J. (2024).
	\textit{T0-Theory and QFT Connections}.
	Unpublished manuscript, HTL Leonding.
	
	\bibitem{Sommerfeld1916}
	Sommerfeld, A. (1916).
	\textit{On the Quantum Theory of Spectral Lines}.
	Annalen der Physik, 51, 1--94.
	
	\bibitem{T0_Feinstruktur}
	Pascher, J. (2024).
	\textit{T0-Theory: Fine Structure Analysis}.
	Unpublished manuscript, HTL Leonding.
	
	\bibitem{T0_SI}
	Pascher, J. (2024).
	\textit{T0-Theory and SI Units}.
	Unpublished manuscript, HTL Leonding.
	
	\bibitem{T0_fine_structure}
	Pascher, J. (2024).
	\textit{T0-Theory: The Fine Structure Constant}.
	Unpublished manuscript, HTL Leonding.
	
	\bibitem{T0_g2_erweiterung}
	Pascher, J. (2024).
	\textit{T0-Theory: g-2 Extensions}.
	Unpublished manuscript, HTL Leonding.
	
	\bibitem{T0_gravitational_constant}
	Pascher, J. (2024).
	\textit{T0-Theory: Gravitational Constant Derivation}.
	Unpublished manuscript, HTL Leonding.
	
	\bibitem{T0_netze_en}
	Pascher, J. (2024).
	\textit{T0-Theory: Network Structures}.
	Unpublished manuscript, HTL Leonding.
	
	\bibitem{T0_tm_erweiterung}
	Pascher, J. (2024).
	\textit{T0-Theory: Time-Mass Extensions}.
	Unpublished manuscript, HTL Leonding.
	
	\bibitem{Uzan2003}
	Uzan, J.-P. (2003).
	\textit{The fundamental constants and their variation}.
	Rev. Mod. Phys. 75, 403--455.
	
	\bibitem{Weinberg1995}
	Weinberg, S. (1995).
	\textit{The Quantum Theory of Fields, Vol. I}.
	Cambridge University Press.
	
	\bibitem{albrecht1999}
	Albrecht, A. \& Magueijo, J. (1999).
	\textit{A time varying speed of light as a solution to cosmological puzzles}.
	Phys. Rev. D 59, 043516.
	
	\bibitem{alice2023}
	ALICE Collaboration (2023).
	\textit{Recent results from ALICE}.
	CERN-EP-2023-XXX.
	
	\bibitem{analog_optical}
	Smith, J. et al. (2024).
	\textit{Analog optical computing systems}.
	Nature Photonics.
	
	\bibitem{ashtekar2004}
	Ashtekar, A. \& Lewandowski, J. (2004).
	\textit{Background independent quantum gravity}.
	Class. Quantum Grav. 21, R53.
	
	\bibitem{atlas2023}
	ATLAS Collaboration (2023).
	\textit{ATLAS physics results}.
	CERN-PH-EP-2023-XXX.
	
	\bibitem{atlas2023higgs}
	ATLAS Collaboration (2023).
	\textit{Higgs boson measurements}.
	Phys. Rev. Lett.
	
	\bibitem{barbour1999}
	Barbour, J. (1999).
	\textit{The End of Time}.
	Oxford University Press.
	
	\bibitem{barrow1999}
	Barrow, J. D. (1999).
	\textit{Cosmologies with varying light speed}.
	Phys. Rev. D 59, 043515.
	
	\bibitem{becker2007}
	Becker, K. et al. (2007).
	\textit{String Theory and M-Theory}.
	Cambridge University Press.
	
	\bibitem{bell_muon}
	Bennett, G. W., et al. (Muon g-2 Collaboration) (2006).
	\textit{Final report of the E821 muon anomalous magnetic moment measurement}.
	Phys. Rev. D 73, 072003.
	
	\bibitem{bondi1948}
	Bondi, H. \& Gold, T. (1948).
	\textit{The steady-state theory of the expanding universe}.
	Mon. Not. R. Astron. Soc. 108, 252--270.
	
	\bibitem{brewer2019}
	Brewer, S. M. et al. (2019).
	\textit{Al+ Quantum-Logic Clock with Systematic Uncertainty below $10^{-18}$}.
	Phys. Rev. Lett. 123, 033201.
	
	\bibitem{cms2023top}
	CMS Collaboration (2023).
	\textit{Top quark measurements at CMS}.
	JHEP 2023.
	
	\bibitem{cms2024}
	CMS Collaboration (2024).
	\textit{CMS physics results 2024}.
	CERN-PH-EP-2024-XXX.
	
	\bibitem{codata2019}
	Tiesinga, E. et al. (2019).
	\textit{The 2018 CODATA Recommended Values}.
	J. Phys. Chem. Ref. Data.
	
	\bibitem{desi2025}
	DESI Collaboration (2025).
	\textit{DESI 2025 Cosmology Results}.
	arXiv preprint.
	
	\bibitem{differential_optical}
	Wang, X. et al. (2024).
	\textit{Differential optical computing}.
	Optica.
	
	\bibitem{dingle1972}
	Dingle, H. (1972).
	\textit{Science at the Crossroads}.
	Martin Brian \& O'Keeffe.
	
	\bibitem{divalentino2021}
	Di Valentino, E. et al. (2021).
	\textit{In the realm of the Hubble tension}.
	Class. Quantum Grav. 38, 153001.
	
	\bibitem{elnaschie2004}
	El Naschie, M. S. (2004).
	\textit{A review of E infinity theory}.
	Chaos, Solitons \& Fractals, 19, 209--236.
	
	\bibitem{fabrication_heterogeneous}
	Chen, Y. et al. (2024).
	\textit{Heterogeneous photonic integration}.
	Nature Electronics.
	
	\bibitem{fermilab2023}
	Fermilab (2023).
	\textit{Muon g-2 results}.
	Phys. Rev. Lett.
	
	\bibitem{flexible_wafer}
	Kim, S. et al. (2024).
	\textit{Flexible wafer-scale photonics}.
	Science Advances.
	
	\bibitem{francesco1997}
	Di Francesco, P. et al. (1997).
	\textit{Conformal Field Theory}.
	Springer.
	
	\bibitem{hartree1957}
	Hartree, D. R. (1957).
	\textit{The Calculation of Atomic Structures}.
	Wiley.
	
	\bibitem{hhi_6g}
	Fraunhofer HHI (2024).
	\textit{6G Photonic Integration}.
	Technical Report.
	
	\bibitem{hossenfelder2025}
	Hossenfelder, S. (2025).
	\textit{Science without the gobbledygook}.
	YouTube/Blog.
	
	\bibitem{hossenfelder_single_clock_video}
	Hossenfelder, S. (2024).
	\textit{The Single Clock Problem}.
	YouTube.
	
	\bibitem{hoyle1948}
	Hoyle, F. (1948).
	\textit{A new model for the expanding universe}.
	Mon. Not. R. Astron. Soc. 108, 372--382.
	
	\bibitem{integration_microelectronic}
	Liu, A. et al. (2024).
	\textit{Microelectronic photonic integration}.
	IEEE Journal.
	
	\bibitem{jacobson1995}
	Jacobson, T. (1995).
	\textit{Thermodynamics of spacetime}.
	Phys. Rev. Lett. 75, 1260.
	
	\bibitem{kasevich2023}
	Kasevich, M. et al. (2023).
	\textit{Atom interferometry tests}.
	Nature Physics.
	
	\bibitem{lerner2014}
	Lerner, E. J. (2014).
	\textit{An open letter on cosmology}.
	New Scientist.
	
	\bibitem{lisa2017}
	LISA Consortium (2017).
	\textit{Laser Interferometer Space Antenna}.
	ESA Technical Report.
	
	\bibitem{lithium_tantalate}
	Zhang, M. et al. (2024).
	\textit{Thin-film lithium tantalate photonics}.
	Nature Photonics.
	
	\bibitem{lopez2010}
	Lopez-Corredoira, M. (2010).
	\textit{Tests and problems of the standard model in cosmology}.
	Int. J. Mod. Phys. D.
	
	\bibitem{ludlow2015}
	Ludlow, A. D. et al. (2015).
	\textit{Optical atomic clocks}.
	Rev. Mod. Phys. 87, 637.
	
	\bibitem{mach1883}
	Mach, E. (1883).
	\textit{Die Mechanik in ihrer Entwickelung}.
	F.A. Brockhaus.
	
	\bibitem{maldacena1998}
	Maldacena, J. (1998).
	\textit{The large N limit of superconformal field theories}.
	Adv. Theor. Math. Phys. 2, 231--252.
	
	\bibitem{mueller2014}
	Müller, H. et al. (2014).
	\textit{Atom interferometry tests of the gravitational redshift}.
	Phys. Rev. Lett.
	
	\bibitem{mug2_final_2025}
	Muon g-2 Collaboration (2025).
	\textit{Final muon g-2 measurement}.
	Phys. Rev. Lett.
	
	\bibitem{muong2_2023}
	Muon g-2 Collaboration (2023).
	\textit{Updated muon g-2 results}.
	Phys. Rev. Lett.
	
	\bibitem{nathan2024}
	Nathan, A. et al. (2024).
	\textit{Quantum computing advances}.
	Nature.
	
	\bibitem{newell2018}
	Newell, D. B. et al. (2018).
	\textit{The CODATA 2017 values of h, e, k, and $N_A$}.
	Metrologia 55, L13.
	
	\bibitem{nottale1993}
	Nottale, L. (1993).
	\textit{Fractal Space-Time and Microphysics}.
	World Scientific.
	
	\bibitem{on_chip_lithium}
	Wang, C. et al. (2024).
	\textit{On-chip lithium niobate photonics}.
	Nature Communications.
	
	\bibitem{optical_advantages}
	Shastri, B. J. et al. (2024).
	\textit{Advantages of optical computing}.
	Nature Reviews Physics.
	
	\bibitem{pascher2025cmb}
	Pascher, J. (2025).
	\textit{T0-Theory: CMB Analysis}.
	Unpublished manuscript, HTL Leonding.
	
	\bibitem{pascher2025g2}
	Pascher, J. (2025).
	\textit{T0-Theory: g-2 Predictions}.
	Unpublished manuscript, HTL Leonding.
	
	\bibitem{pascher2025qm}
	Pascher, J. (2025).
	\textit{T0-Theory: Quantum Mechanics}.
	Unpublished manuscript, HTL Leonding.
	
	\bibitem{pascher2025si}
	Pascher, J. (2025).
	\textit{T0-Theory: SI Unit System}.
	Unpublished manuscript, HTL Leonding.
	
	\bibitem{pascher2025t0}
	Pascher, J. (2025).
	\textit{T0-Theory: Complete Framework}.
	Unpublished manuscript, HTL Leonding.
	
	\bibitem{pascher:fundamentals}
	Pascher, J. (2024).
	\textit{T0-Theory: Fundamentals}.
	Unpublished manuscript, HTL Leonding.
	
	\bibitem{pascher:g2_rev9}
	Pascher, J. (2024).
	\textit{T0-Theory: g-2 Revision 9}.
	Unpublished manuscript, HTL Leonding.
	
	\bibitem{pascher:geometric_formalism}
	Pascher, J. (2024).
	\textit{T0-Theory: Geometric Formalism}.
	Unpublished manuscript, HTL Leonding.
	
	\bibitem{pascher:ml_addendum}
	Pascher, J. (2024).
	\textit{T0-Theory: Machine Learning Addendum}.
	Unpublished manuscript, HTL Leonding.
	
	\bibitem{pascher:t0_foundations}
	Pascher, J. (2024).
	\textit{T0-Theory: Foundations}.
	Unpublished manuscript, HTL Leonding.
	
	\bibitem{pascher_derivation_beta_2025}
	Pascher, J. (2025).
	\textit{T0-Theory: Derivation of Beta}.
	Unpublished manuscript, HTL Leonding.
	
	\bibitem{pascher_higgs_connection_2025}
	Pascher, J. (2025).
	\textit{T0-Theory: Higgs Connection}.
	Unpublished manuscript, HTL Leonding.
	
	\bibitem{pascher_lagrangian_extended_2025}
	Pascher, J. (2025).
	\textit{T0-Theory: Extended Lagrangian}.
	Unpublished manuscript, HTL Leonding.
	
	\bibitem{pascher_mathematical_structure_2025}
	Pascher, J. (2025).
	\textit{T0-Theory: Mathematical Structure}.
	Unpublished manuscript, HTL Leonding.
	
	\bibitem{pascher_t0_cmb_2025}
	Pascher, J. (2025).
	\textit{T0-Theory: CMB Predictions}.
	Unpublished manuscript, HTL Leonding.
	
	\bibitem{pascher_t0_energie_2025}
	Pascher, J. (2025).
	\textit{T0-Theory: Energy}.
	Unpublished manuscript, HTL Leonding.
	
	\bibitem{pascher_t0_energy_2025}
	Pascher, J. (2025).
	\textit{T0-Theory: Energy Framework}.
	Unpublished manuscript, HTL Leonding.
	
	\bibitem{pascher_t0_theory_2025}
	Pascher, J. (2025).
	\textit{T0-Theory: Complete Theory}.
	Unpublished manuscript, HTL Leonding.
	
	\bibitem{penrose1959}
	Penrose, R. (1959).
	\textit{The apparent shape of a relativistically moving sphere}.
	Proc. Cambridge Phil. Soc. 55, 137--139.
	
	\bibitem{penrose1967}
	Penrose, R. (1967).
	\textit{Twistor algebra}.
	J. Math. Phys. 8, 345--366.
	
	\bibitem{peratt1992}
	Peratt, A. L. (1992).
	\textit{Physics of the Plasma Universe}.
	Springer-Verlag.
	
	\bibitem{peskin1995}
	Peskin, M. E. \& Schroeder, D. V. (1995).
	\textit{An Introduction to Quantum Field Theory}.
	Addison-Wesley.
	
	\bibitem{peskin_schroeder_1995}
	Peskin, M. E. \& Schroeder, D. V. (1995).
	\textit{An Introduction to Quantum Field Theory}.
	Addison-Wesley.
	
	\bibitem{phoquant}
	PhoQuant (2024).
	\textit{Photonic quantum computing}.
	Technical Report.
	
	\bibitem{photonics_ai}
	Wetzstein, G. et al. (2024).
	\textit{Photonics for AI}.
	Nature.
	
	\bibitem{planck1906}
	Planck, M. (1906).
	\textit{The Theory of Heat Radiation}.
	Johann Ambrosius Barth.
	
	\bibitem{planck2018}
	Planck Collaboration (2018).
	\textit{Planck 2018 results}.
	A\&A 641, A6.
	
	\bibitem{polchinski1998}
	Polchinski, J. (1998).
	\textit{String Theory}.
	Cambridge University Press.
	
	\bibitem{qant_nps}
	QANT (2024).
	\textit{Quantum photonics systems}.
	Technical Report.
	
	\bibitem{quantenjahr25}
	Quantenjahr (2025).
	\textit{International Year of Quantum}.
	UNESCO.
	
	\bibitem{recurrent_photonics}
	Tait, A. N. et al. (2024).
	\textit{Recurrent photonic neural networks}.
	Optica.
	
	\bibitem{rf_photonics}
	Capmany, J. \& Novak, D. (2024).
	\textit{Microwave photonics}.
	Nature Photonics.
	
	\bibitem{riess2019}
	Riess, A. G. et al. (2019).
	\textit{Large Magellanic Cloud Cepheid Standards}.
	ApJ 876, 85.
	
	\bibitem{riess2022}
	Riess, A. G. et al. (2022).
	\textit{A Comprehensive Measurement of H0}.
	ApJ 934, L7.
	
	\bibitem{rovelli2004}
	Rovelli, C. (2004).
	\textit{Quantum Gravity}.
	Cambridge University Press.
	
	\bibitem{sciama1953}
	Sciama, D. W. (1953).
	\textit{On the origin of inertia}.
	Mon. Not. R. Astron. Soc. 113, 34--42.
	
	\bibitem{sciencedaily2025}
	ScienceDaily (2025).
	\textit{Physics news}.
	Online.
	
	\bibitem{sm_g2_2025}
	Aoyama, T. et al. (2025).
	\textit{Standard Model prediction for g-2}.
	Phys. Rep.
	
	\bibitem{susskind1995}
	Susskind, L. (1995).
	\textit{The world as a hologram}.
	J. Math. Phys. 36, 6377--6396.
	
	\bibitem{t0_kosmologie}
	Pascher, J. (2024).
	\textit{T0-Theory: Cosmology}.
	Unpublished manuscript, HTL Leonding.
	
	\bibitem{terrell1959}
	Terrell, J. (1959).
	\textit{Invisibility of the Lorentz contraction}.
	Phys. Rev. 116, 1041--1045.
	
	\bibitem{terrell_single_clock_nature_2024}
	Terrell, J. et al. (2024).
	\textit{Single clock precision measurements}.
	Nature Physics.
	
	\bibitem{tfln_foundry}
	TFLN Foundry (2024).
	\textit{Thin-film lithium niobate foundry services}.
	Technical Specifications.
	
	\bibitem{thiemann2007}
	Thiemann, T. (2007).
	\textit{Modern Canonical Quantum General Relativity}.
	Cambridge University Press.
	
	\bibitem{thz_epfl}
	EPFL (2024).
	\textit{Terahertz photonics research}.
	Technical Report.
	
	\bibitem{unnikrishnan2004}
	Unnikrishnan, C. S. (2004).
	\textit{On Einstein's resolution of the twin clock paradox}.
	Current Science, 86, 704--709.
	
	\bibitem{verlinde2011}
	Verlinde, E. (2011).
	\textit{On the origin of gravity and the laws of Newton}.
	JHEP 2011, 29.
	
	\bibitem{video2025}
	Video (2025).
	\textit{Physics video explanation}.
	YouTube.
	
	\bibitem{weinberg1995}
	Weinberg, S. (1995).
	\textit{The Quantum Theory of Fields}.
	Cambridge University Press.
	
	\bibitem{weiskopf2000}
	Weiskopf, D. (2000).
	\textit{Visualization of special relativity}.
	PhD thesis, University of Tübingen.
	
	\bibitem{wheeler1990}
	Wheeler, J. A. (1990).
	\textit{A Journey into Gravity and Spacetime}.
	Scientific American Library.
	
	\bibitem{wiki_bell}
	Wikipedia (2024).
	\textit{Bell's theorem}.
	Online encyclopedia.
	
	\bibitem{zwicky1929}
	Zwicky, F. (1929).
	\textit{On the red shift of spectral lines through interstellar space}.
	Proc. Natl. Acad. Sci. 15, 773--779.

\end{thebibliography}


\end{document}
