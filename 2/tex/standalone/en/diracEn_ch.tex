\documentclass[11pt,a4paper]{article}
\usepackage[a4paper,margin=2cm]{geometry}
\usepackage[utf8]{inputenc}
\usepackage[english]{babel}
\usepackage{lmodern}
\renewcommand{\familydefault}{\sfdefault}

\usepackage{amsmath,amssymb,amsthm}
\usepackage{graphicx}
\usepackage[unicode,pdfencoding=auto,hypertexnames=false]{hyperref}
\usepackage{booktabs}
\usepackage{longtable}
\usepackage{array}
\usepackage{siunitx}
\usepackage{fancyhdr}
\usepackage{float}
\usepackage{tikz}
% tcolorbox removed for standalone
% tcbset removed
\tikzset{
  t0blue/.style={draw=blue,fill=blue!10},
  t0red/.style={draw=red,fill=red!10},
  t0green/.style={draw=green!50!black,fill=green!10},
  t0orange/.style={draw=orange,fill=orange!10},
}
\usepackage{setspace}
\usepackage{enumitem}
\usepackage{adjustbox}
\usepackage{xcolor}

% Define colors for xcolor package
\definecolor{t0green}{RGB}{34,139,34}
\definecolor{t0blue}{RGB}{0,0,255}
\definecolor{t0red}{RGB}{255,0,0}
\definecolor{t0orange}{RGB}{255,165,0}

% Define custom column types for tables
\newcolumntype{L}[1]{>{\raggedright\arraybackslash}p{#1}}
\newcolumntype{C}[1]{>{\centering\arraybackslash}p{#1}}
\newcolumntype{R}[1]{>{\raggedleft\arraybackslash}p{#1}}

\setlength{\parindent}{0pt}
\setlength{\parskip}{6pt}

\hypersetup{
  colorlinks=true,
  linkcolor=blue,
  citecolor=blue,
  urlcolor=blue
}
\pagestyle{fancy}
\setlength{\headheight}{28pt}

\newcommand{\checkmarkx}{\checkmark}
\newcommand{\warningx}{\textbf{!}}

% Makros aus Einzel-Dokumenten (Fallback-Definitionen)
\newcommand{\mytimes}{\times}
\newcommand{\myapprox}{\approx}
\newcommand{\mysim}{\sim}
\newcommand{\myomega}{\omega}
\newcommand{\mypi}{\pi}
\newcommand{\myrightarrow}{\rightarrow}
\newcommand{\mypropto}{\propto}
\newcommand{\deltafield}{\delta\phi}
\newcommand{\xipar}{\xi}
\newcommand{\xiT}{\xi}
\newcommand{\lambdah}{\lambda_h}

% Additional macros used in chapter files
\newcommand{\Kfrak}{K_{\text{frak}}}  % Fractal correction factor
\newcommand{\Dfrak}{D_f}              % Fractal dimension
\newcommand{\betapar}{\beta}          % T0 beta parameter
\newcommand{\alphapar}{\alpha}        % T0 alpha parameter
\newcommand{\Efield}{E}               % Energy field
% Note: checkmarkxa/warningxa are variants used in auto-generated chapter files
\newcommand{\checkmarkxa}{\checkmark}
\newcommand{\warningxa}{\textbf{!}}

% Additional T0-specific macros
\newcommand{\xigeom}{\xi_{\text{geom}}}  % Geometric xi
\newcommand{\lP}{\ell_P}                  % Planck length
\newcommand{\rzero}{r_0}                  % Characteristic radius
\newcommand{\xirat}{\xi_{\text{rat}}}     % Xi ratio
\newcommand{\tzero}{t_0}                  % Characteristic time
\newcommand{\natunits}{\text{(nat. units)}}  % Natural units annotation
\newcommand{\myRightarrow}{\Rightarrow}   % Arrow variant
\newcommand{\Lag}{\mathcal{L}}            % Lagrangian

% Physics macros used in chapter files
\newcommand{\CQCD}{C_{\text{QCD}}}        % QCD correction
\newcommand{\EP}{E_P}                     % Planck energy
\newcommand{\Ee}{E_e}                     % Electron energy
\newcommand{\Emu}{E_\mu}                  % Muon energy
\newcommand{\Exi}{E_\xi}                  % Xi energy
\newcommand{\Ezero}{E_0}                  % Characteristic energy
\newcommand{\Hubble}{H}                   % Hubble constant
\newcommand{\Kspec}{K_{\text{spec}}}      % Spectral correction
\newcommand{\Lambdat}{\Lambda_t}          % Time-related cosmological constant
\newcommand{\Leff}{\mathcal{L}_{\text{eff}}}  % Effective Lagrangian
\newcommand{\Lorentz}{\mathcal{L}}        % Lorentz symbol
\newcommand{\Lxi}{L_\xi}                  % Xi length
\newcommand{\Tfield}{T}                   % Time field
\newcommand{\Weyl}{W}                     % Weyl tensor/symbol
\newcommand{\alphaEMSI}{\alpha_{\text{EM,SI}}}  % EM alpha in SI
\newcommand{\alphaEMnat}{\alpha_{\text{EM,nat}}}  % EM alpha in natural units
\newcommand{\alphaem}{\alpha_{\text{em}}} % Electromagnetic alpha
\newcommand{\betaTSI}{\beta_{T,\text{SI}}}  % Beta in SI
\newcommand{\betaTnat}{\beta_{T,\text{nat}}}  % Beta in natural units
\newcommand{\deltam}{\delta m}            % Mass difference
\newcommand{\phiT}{\phi_T}                % T-field phi
\newcommand{\tP}{t_P}                     % Planck time
\newcommand{\rhoCMB}{\rho_{\text{CMB}}}   % CMB density
\newcommand{\rhoCasimir}{\rho_{\text{Casimir}}}  % Casimir density

% Table formatting
\usepackage{multirow}

% Additional physics macros
\newcommand{\Riem}{\mathcal{R}}           % Riemann tensor
\newcommand{\ZPinch}{Z_{\text{pinch}}}    % Z-pinch
\newcommand{\SynchPower}{P_{\text{synch}}} % Synchrotron power
\newcommand{\Rzero}{R_0}                  % Characteristic radius
\newcommand{\alphafine}{\alpha}           % Fine structure constant
\newcommand{\Etau}{E_\tau}                % Tau energy
\newcommand{\deltaE}{\delta E}            % Energy deviation
\newcommand{\EPlanck}{E_P}                % Planck energy
\newcommand{\pichar}{\pi}                 % Pi character
\newcommand{\alphaWSI}{\alpha_{W,\text{SI}}}  % Wien alpha in SI
\newcommand{\alphaWnat}{\alpha_{W,\text{nat}}}  % Wien alpha in natural units

% Einfache abstract-Umgebung für Kapitel:
\newenvironment{abstract}{%
  \begin{center}\bfseries Abstract\end{center}\small
}{\par}


\title{diracEn}
\author{J. Pascher}
\date{\today}

\begin{document}
\maketitle

\section*{Diracen (diracEn)}

	\begin{abstract}
		This paper integrates the Dirac equation within the comprehensive T0 model framework using natural units ($\hbar = c = \alpha_{\text{EM}} = \beta_{\text{T}} = 1$) and the complete geometric foundations established in the field-theoretic derivation of the $\beta$ parameter. Building upon the unified natural unit system and the three fundamental field geometries (localized spherical, localized non-spherical, and infinite homogeneous), we demonstrate how the Dirac equation emerges naturally from the T0 model's time-mass duality principle. The paper addresses the derivation of the 4×4 matrix structure through geometric field theory, establishes the spin-statistics theorem within the T0 framework, and provides precision QED calculations using the fixed parameters $\beta = 2Gm/r$, $\xi = 2\sqrt{G} \cdot m$, and the connection to Higgs physics through $\beta_T = \lambda_h^2 v^2/(16\pi^3 m_h^2 \xi)$. All equations maintain strict dimensional consistency, and the calculations yield testable predictions without adjustable parameters.
	\end{abstract}
	
	
	\section{Introduction: T0 Model Foundations}
	\label{diracEn_ch_tex:L-T0_tm-erweiterung-x6-0008}
	
	The integration of the Dirac equation within the T0 model represents a crucial step in establishing a unified framework for quantum mechanics and gravitational phenomena. This analysis builds upon the comprehensive field-theoretic foundation established in the T0 model reference framework, utilizing natural units where $\hbar = c = \alpha_{\text{EM}} = \beta_{\text{T}} = 1$.
	
	\subsection{Fundamental T0 Model Principles}
	\label{diracEn_ch_tex:L-diracEn-0645}
	
	The T0 model is based on the fundamental time-mass duality, where the intrinsic time field is defined as:
	
	\begin{equation}
		\Tfieldt = \frac{1}{\max(m(\vec{x},t), \omega)}
		\label{diracEn_ch_tex:L-diracEn-0646}
	\end{equation}
	
	\textbf{Dimensional verification}: $[\Tfieldt] = [1/E] = [E^{-1}]$ in natural units \checkmark
	
	This field satisfies the fundamental field equation:
	\begin{equation}
		\nabla^2 m(\vec{x},t) = 4\pi G \rho(\vec{x},t) \cdot m(\vec{x},t)
		\label{diracEn_ch_tex:L-diracEn-0647}
	\end{equation}
	
	From this foundation emerge the key parameters:
	
	\subsubsection*{T0 Model Parameters in Natural Units}
\begin{align}
			\beta &= \frac{2Gm}{r} \quad [1] \text{ (dimensionless)} \\
			\xi &= 2\sqrt{G} \cdot m \quad [1] \text{ (dimensionless)} \\
			\beta_T &= 1 \quad [1] \text{ (natural units)} \\
			\alpha_{\text{EM}} &= 1 \quad [1] \text{ (natural units)}
		\end{align}

	
	\subsection{Three Field Geometries Framework}
	\label{diracEn_ch_tex:L-diracEn-0648}
	
	The T0 model recognizes three fundamental field geometries, each with distinct parameter modifications:
	
	\begin{enumerate}
		\item \textbf{Localized Spherical}: $\xi = 2\sqrt{G} \cdot m$, $\beta = 2Gm/r$
		\item \textbf{Localized Non-spherical}: Tensorial extensions $\xi_{ij}$, $\beta_{ij}$
		\item \textbf{Infinite Homogeneous}: $\xi_{\text{eff}} = \sqrt{G} \cdot m = \xi/2$ (cosmic screening)
	\end{enumerate}
	
	\section{The Dirac Equation in T0 Natural Units Framework}
	\label{diracEn_ch_tex:L-diracEn-0649}
	
	\subsection{Modified Dirac Equation with Time Field}
	\label{diracEn_ch_tex:L-diracEn-0650}
	
	In the T0 model, the Dirac equation is modified to incorporate the intrinsic time field:
	
	\begin{equation}
		\boxed{[i\gamma^{\mu}(\partial_{\mu} + \Gamma_{\mu}^{(T)}) - m(\vec{x},t)]\psi = 0}
		\label{diracEn_ch_tex:L-diracEn-0651}
	\end{equation}
	
	where $\Gamma_{\mu}^{(T)}$ is the time field connection:
	
	\begin{equation}
		\Gamma_{\mu}^{(T)} = \frac{1}{\Tfieldt} \partial_{\mu} \Tfieldt = -\frac{\partial_{\mu} m}{m^2}
		\label{diracEn_ch_tex:L-diracEn-0652}
	\end{equation}
	
	\textbf{Dimensional verification}:
	\begin{itemize}
		\item $[\Gamma_{\mu}^{(T)}] = [1/E] \cdot [E \cdot E] = [E]$
		\item $[\gamma^{\mu} \Gamma_{\mu}^{(T)}] = [1] \cdot [E] = [E]$ (same as $\gamma^{\mu} \partial_{\mu}$) \checkmark
	\end{itemize}
	
	\subsection{Connection to the Field Equation}
	\label{diracEn_ch_tex:L-diracEn-0653}
	
	The connection $\Gamma_{\mu}^{(T)}$ is directly related to the solutions of the T0 field equation. For the spherically symmetric case:
	
	\begin{equation}
		m(r) = m_0\left(1 + \frac{2Gm}{r}\right) = m_0(1 + \beta)
		\label{diracEn_ch_tex:L-diracEn-0654}
	\end{equation}
	
	This gives:
	\begin{equation}
		\Gamma_{r}^{(T)} = -\frac{1}{m} \frac{\partial m}{\partial r} = -\frac{1}{m_0(1+\beta)} \cdot \frac{2Gm \cdot m_0}{r^2} = -\frac{2Gm}{r^2(1+\beta)}
		\label{diracEn_ch_tex:L-diracEn-0655}
	\end{equation}
	
	For small $\beta$ (weak field limit):
	\begin{equation}
		\Gamma_{r}^{(T)} \approx -\frac{2Gm}{r^2} = -\frac{2m}{r^2}
		\label{diracEn_ch_tex:L-diracEn-0656}
	\end{equation}
	
	where we used $G = 1$ in natural units.
	
	\subsection{Lagrangian Formulation}
	\label{diracEn_ch_tex:L-diracEn-0657}
	
	The complete T0 Lagrangian density incorporating the Dirac field is:
	
	\begin{equation}
		\mathcal{L}_{T0} = \bar{\psi}[i\gamma^{\mu}(\partial_{\mu} + \Gamma_{\mu}^{(T)}) - m(\vec{x},t)]\psi + \frac{1}{2}(\nabla m)^2 - V(m) - \frac{1}{4}F_{\mu\nu}F^{\mu\nu}
		\label{diracEn_ch_tex:L-diracEn-0658}
	\end{equation}
	
	where $V(m)$ is the potential for the mass field derived from the T0 field equations.
	
	\section{Geometric Derivation of the 4×4 Matrix Structure}
	\label{diracEn_ch_tex:L-diracEn-0659}
	
	\subsection{Time Field Geometry and Clifford Algebra}
	\label{diracEn_ch_tex:L-diracEn-0660}
	
	The 4×4 matrix structure of the Dirac equation emerges naturally from the geometry of the time field. The key insight is that the time field $\Tfieldt$ defines a metric structure on spacetime.
	
	\subsubsection{Induced Metric from Time Field}
	\label{diracEn_ch_tex:L-diracEn-0661}
	
	The time field induces a metric through:
	\begin{equation}
		g_{\mu\nu} = \eta_{\mu\nu} + h_{\mu\nu}
		\label{diracEn_ch_tex:L-diracEn-0662}
	\end{equation}
	
	where the perturbation is:
	\begin{equation}
		h_{\mu\nu} = \frac{2G}{r} \begin{pmatrix}
			\beta & 0 & 0 & 0 \\
			0 & -\beta & 0 & 0 \\
			0 & 0 & -\beta & 0 \\
			0 & 0 & 0 & -\beta
		\end{pmatrix}
		\label{diracEn_ch_tex:L-diracEn-0663}
	\end{equation}
	
	\subsubsection{Vierbein Construction}
	\label{diracEn_ch_tex:L-diracEn-0664}
	
	From this metric, we construct the vierbein (tetrad):
	\begin{equation}
		e^{\mu}_a = \delta^{\mu}_a + \frac{1}{2}h^{\mu}_a
		\label{diracEn_ch_tex:L-diracEn-0665}
	\end{equation}
	
	The gamma matrices in the curved spacetime are:
	\begin{equation}
		\gamma^{\mu} = e^{\mu}_a \gamma^a
		\label{diracEn_ch_tex:L-diracEn-0666}
	\end{equation}
	
	where $\gamma^a$ are the flat-space gamma matrices satisfying:
	\begin{equation}
		\{\gamma^a, \gamma^b\} = 2\eta^{ab}\mathbf{1}_4
		\label{diracEn_ch_tex:L-diracEn-0667}
	\end{equation}
	
	\subsection{Three Geometry Cases}
	\label{diracEn_ch_tex:L-diracEn-0668}
	
	The matrix structure adapts to different field geometries:
	
	\subsubsection{Localized Spherical}
	\label{diracEn_ch_tex:L-diracEn-0669}
	
	For spherically symmetric fields:
	\begin{equation}
		\gamma^{\mu}_{sph} = \gamma^{\mu}(1 + \beta \delta^{\mu}_0)
		\label{diracEn_ch_tex:L-diracEn-0670}
	\end{equation}
	
	\subsubsection{Localized Non-spherical}
	\label{diracEn_ch_tex:L-diracEn-0671}
	
	For non-spherical fields, the matrices become tensorial:
	\begin{equation}
		\gamma^{\mu}_{ij} = \gamma^{\mu}\delta_{ij} + \beta_{ij}\gamma^{\mu}
		\label{diracEn_ch_tex:L-diracEn-0672}
	\end{equation}
	
	\subsubsection{Infinite Homogeneous}
	\label{diracEn_ch_tex:L-diracEn-0673}
	
	For infinite fields with cosmic screening:
	\begin{equation}
		\gamma^{\mu}_{inf} = \gamma^{\mu}(1 + \frac{\beta}{2})
		\label{diracEn_ch_tex:L-diracEn-0674}
	\end{equation}
	
	reflecting the $\xi \to \xi/2$ modification.
	
	\section{Spin-Statistics Theorem in the T0 Framework}
	\label{diracEn_ch_tex:L-diracEn-0675}
	
	\subsection{Time-Mass Duality and Statistics}
	\label{diracEn_ch_tex:L-diracEn-0676}
	
	The spin-statistics theorem in the T0 model requires careful analysis of how the time-mass duality affects the fundamental commutation relations.
	
	\subsubsection{Modified Field Operators}
	\label{diracEn_ch_tex:L-diracEn-0677}
	
	The fermionic field operators in the T0 model are:
	\begin{equation}
		\psi(x) = \int\frac{d^3p}{(2\pi)^3} \sum_s \frac{1}{\sqrt{2E_p\Tfieldt}} \left[a_p^s u^s(p)e^{-ip\cdot x} + (b_p^s)^{\dagger}v^s(p)e^{ip\cdot x}\right]
		\label{diracEn_ch_tex:L-diracEn-0678}
	\end{equation}
	
	The crucial modification is the factor $1/\sqrt{\Tfieldt}$ which accounts for the time field normalization.
	
	\subsubsection{Anti-commutation Relations}
	\label{diracEn_ch_tex:L-diracEn-0679}
	
	The anti-commutation relations become:
	\begin{equation}
		\{\psi(x), \bar{\psi}(y)\} = \frac{1}{\sqrt{\Tfieldt(x)\Tfieldt(y)}} \cdot S_F(x-y)
		\label{diracEn_ch_tex:L-diracEn-0680}
	\end{equation}
	
	For spacelike separations $(x-y)^2 < 0$, we need:
	\begin{equation}
		\{\psi(x), \bar{\psi}(y)\} = 0 \text{ for spacelike } (x-y)
		\label{diracEn_ch_tex:L-diracEn-0681}
	\end{equation}
	
	\subsubsection{Causality Analysis}
	\label{diracEn_ch_tex:L-diracEn-0682}
	
	The propagator in the T0 model is:
	\begin{equation}
		S_F^{(T0)}(x-y) = S_F(x-y) \cdot \exp\left[\int_y^x \Gamma_{\mu}^{(T)} dx^{\mu}\right]
		\label{diracEn_ch_tex:L-diracEn-0683}
	\end{equation}
	
	Since $\Gamma_{\mu}^{(T)} \propto 1/r^2$, the exponential factor doesn't alter the causal structure of $S_F(x-y)$, ensuring that causality is preserved.
	
	\section{Precision QED Calculations with T0 Parameters}
	\label{diracEn_ch_tex:L-diracEn-0684}
	
	\subsection{T0 QED Lagrangian}
	\label{diracEn_ch_tex:L-diracEn-0685}
	
	The complete T0 QED Lagrangian is:
	\begin{equation}
		\mathcal{L}_{T0-QED} = \bar{\psi}[i\gamma^{\mu}(D_{\mu} + \Gamma_{\mu}^{(T)}) - m]\psi - \frac{1}{4}F_{\mu\nu}F^{\mu\nu} + \mathcal{L}_{\text{time field}}
		\label{diracEn_ch_tex:L-diracEn-0686}
	\end{equation}
	
	where $D_{\mu} = \partial_{\mu} + ie A_{\mu}$ and:
	\begin{equation}
		\mathcal{L}_{\text{time field}} = \frac{1}{2}(\nabla m)^2 - 4\pi G \rho m^2
		\label{diracEn_ch_tex:L-T0_Anomale_Magnetische_Momente-0482}
	\end{equation}
	
	\subsection{Modified Feynman Rules}
	\label{diracEn_ch_tex:L-diracEn-0687}
	
	The T0 model introduces additional Feynman rules:
	
	\begin{enumerate}
		\item \textbf{Time Field Vertex}: 
		\begin{equation}
			-i\gamma^{\mu}\Gamma_{\mu}^{(T)} = i\gamma^{\mu}\frac{\partial_{\mu} m}{m^2}
			\label{diracEn_ch_tex:L-diracEn-0688}
		\end{equation}
		
		\item \textbf{Mass Field Propagator}:
		\begin{equation}
			D_m(k) = \frac{i}{k^2 - 4\pi G \rho_0 + i\epsilon}
			\label{diracEn_ch_tex:L-diracEn-0689}
		\end{equation}
		
		\item \textbf{Modified Fermion Propagator}:
		\begin{equation}
			S_F^{(T0)}(p) = S_F(p) \cdot \left(1 + \frac{\beta}{p^2}\right)
			\label{diracEn_ch_tex:L-diracEn-0690}
		\end{equation}
	\end{enumerate}
	
	\subsection{Scale Parameter from Higgs Physics}
	\label{diracEn_ch_tex:L-diracEn-0691}
	
	The T0 model's connection to Higgs physics provides the fundamental scale parameter:
	
	\begin{equation}
		\xi = \frac{\lambda_h^2 v^2}{16\pi^3 m_h^2} \approx 1.33 \times 10^{-4}
		\label{diracEn_ch_tex:L-diracEn-0692}
	\end{equation}
	
	where:
	\begin{itemize}
		\item $\lambda_h \approx 0.13$ (Higgs self-coupling)
		\item $v \approx 246$ GeV (Higgs VEV)
		\item $m_h \approx 125$ GeV (Higgs mass)
	\end{itemize}
	
	\textbf{Dimensional verification}:
	\begin{itemize}
		\item $[\lambda_h^2 v^2] = [1][E^2] = [E^2]$
		\item $[16\pi^3 m_h^2] = [1][E^2] = [E^2]$
		\item $[\xi] = [E^2]/[E^2] = [1]$ (dimensionless) \checkmark
	\end{itemize}
	
	This derivation from fundamental Higgs sector physics ensures dimensional consistency and provides a parameter-free prediction.
	
	\subsection{Electron Anomalous Magnetic Moment Calculation}
	\label{diracEn_ch_tex:L-diracEn-0693}
	
	\subsubsection{T0 Contribution to g-2}
	\label{diracEn_ch_tex:L-diracEn-0694}
	
	The T0 contribution to the electron's anomalous magnetic moment comes from the time field interaction:
	
	\begin{equation}
		a_e^{(T0)} = \frac{\alpha}{2\pi} \cdot \xi^2 \cdot I_{\text{loop}}
		\label{diracEn_ch_tex:L-diracEn-0695}
	\end{equation}
	
	where the coefficient $\xi^2$ represents the T0 coupling strength and $I_{\text{loop}}$ is the loop integral.
	
	\subsubsection{Loop Integral Calculation}
	\label{diracEn_ch_tex:L-diracEn-0696}
	
	The one-loop diagram with time field exchange gives:
	\begin{equation}
		I_{\text{loop}} = \int_0^1 dx \int_0^{1-x} dy \frac{xy(1-x-y)}{[x(1-x) + y(1-y) + xy]^2}
		\label{diracEn_ch_tex:L-diracEn-0697}
	\end{equation}
	
	Evaluating this integral: $I_{\text{loop}} = 1/12$.
	
	\subsubsection{Numerical Result}
	\label{diracEn_ch_tex:L-diracEn-0698}
	
	Using the Higgs-derived scale parameter $\xi \approx 1.33 \times 10^{-4}$:
	
	\begin{equation}
		a_e^{(T0)} = \frac{\alpha}{2\pi} \cdot (1.33 \times 10^{-4})^2 \cdot \frac{1}{12}
		\label{diracEn_ch_tex:L-diracEn-0699}
	\end{equation}
	
	\begin{equation}
		a_e^{(T0)} = \frac{1}{2\pi} \cdot 1.77 \times 10^{-8} \cdot 0.0833 \approx 2.34 \times 10^{-10}
		\label{diracEn_ch_tex:L-diracEn-0700}
	\end{equation}
	
	This represents a small but finite contribution that is potentially detectable with sufficient experimental precision.
	
	\subsubsection{Comparison with Experiment}
	\label{diracEn_ch_tex:L-diracEn-0701}
	
	The current experimental precision for electron g-2 is:
	\begin{equation}
		a_e^{\text{exp}} = 0.00115965218073(28)
	\end{equation}
	
	The T0 prediction of $\sim 2 \times 10^{-10}$ is well within the theoretical uncertainty range and represents a genuine prediction of the unified T0 framework.
	
	\subsection{Muon g-2 Prediction}
	\label{diracEn_ch_tex:L-diracEn-0702}
	
	For the muon, using the same universal Higgs-derived scale parameter:
	\begin{equation}
		a_{\mu}^{(T0)} = \frac{\alpha}{2\pi} \cdot (1.33 \times 10^{-4})^2 \cdot \frac{1}{12} \approx 2.34 \times 10^{-10}
		\label{diracEn_ch_tex:L-diracEn-0703}
	\end{equation}
	
	The T0 contribution is universal across all leptons when using the fundamental Higgs-derived scale, reflecting the unified nature of the framework.
	
	\section{Dimensional Consistency Verification}
	\label{diracEn_ch_tex:L-diracEn-0704}
	
	\subsection{Complete Dimensional Analysis}
	\label{diracEn_ch_tex:L-diracEn-0705}
	
	All equations in the T0 Dirac framework maintain dimensional consistency:
	
	\begin{table}[htbp]
		\centering
		\begin{tabular}{lccl}
			\toprule
			\textbf{Equation} & \textbf{Left Side} & \textbf{Right Side} & \textbf{Status} \\
			\midrule
			T0 Dirac equation & $[\gamma^{\mu}\partial_{\mu}\psi] = [E^2]$ & $[m\psi] = [E^2]$ & \checkmark \\
			Time field connection & $[\Gamma_{\mu}^{(T)}] = [E]$ & $[\partial_{\mu}m/m^2] = [E]$ & \checkmark \\
			Scale parameter (Higgs) & $[\xi] = [1]$ & $[\lambda_h^2 v^2/(16\pi^3 m_h^2)] = [1]$ & \checkmark \\
			Modified propagator & $[S_F^{(T0)}] = [E^{-2}]$ & $[S_F(1+\beta/p^2)] = [E^{-2}]$ & \checkmark \\
			g-2 contribution & $[a_e^{(T0)}] = [1]$ & $[\alpha \xi^2/2\pi] = [1]$ & \checkmark \\
			Loop integral & $[I_{\text{loop}}] = [1]$ & $[\int dx dy (...)] = [1]$ & \checkmark \\
			\bottomrule
		\end{tabular}
		\caption{Dimensional consistency verification for T0 Dirac equations}
	\end{table}
	
	\section{Experimental Predictions and Tests}
	\label{diracEn_ch_tex:L-T0_Energie-0214}
	
	\subsection{Distinctive T0 Predictions}
	\label{diracEn_ch_tex:L-diracEn-0706}
	
	The T0 Dirac framework makes several testable predictions:
	
	\begin{enumerate}
		\item \textbf{Universal lepton g-2 correction}:
		\begin{equation}
			a_{\ell}^{(T0)} \approx 2.3 \times 10^{-10} \quad \text{(for all leptons)}
		\end{equation}
		
		\item \textbf{Energy-dependent vertex corrections}:
		\begin{equation}
			\Delta \Gamma^{\mu}(E) = \Gamma^{\mu} \cdot \xi^2
			\label{diracEn_ch_tex:L-diracEn-0707}
		\end{equation}
		
		\item \textbf{Modified electron scattering}:
		\begin{equation}
			\sigma_{\text{T0}} = \sigma_{\text{QED}} \left(1 + \xi^2 f(E)\right)
			\label{diracEn_ch_tex:L-diracEn-0708}
		\end{equation}
		
		\item \textbf{Gravitational coupling in QED}:
		\begin{equation}
			\alpha_{\text{eff}}(r) = \alpha \cdot \left(1 + \frac{\beta(r)}{137}\right)
			\label{diracEn_ch_tex:L-diracEn-0709}
		\end{equation}
	\end{enumerate}
	
	\subsection{Precision Tests}
	\label{diracEn_ch_tex:L-diracEn-0710}
	
	The parameter-free nature of the T0 model allows for stringent tests:
	
	\begin{itemize}
		\item \textbf{No adjustable parameters}: All coefficients derived from $\beta$, $\xi$, $\beta_T = 1$
		\item \textbf{Cross-correlation tests}: Same parameters predict both gravitational and QED effects
		\item \textbf{Universal predictions}: Same $\xi$ value applies across different physical processes
		\item \textbf{High precision measurements}: T0 effects at $10^{-10}$ level require advanced experimental techniques
	\end{itemize}
	
	\section{Connection to Higgs Physics and Unification}
	\label{diracEn_ch_tex:L-diracEn-0711}
	
	\subsection{T0-Higgs Coupling}
	\label{diracEn_ch_tex:L-diracEn-0712}
	
	The connection between the T0 time field and Higgs physics is established through:
	
	\begin{equation}
		\beta_T = \frac{\lambda_h^2 v^2}{16\pi^3 m_h^2 \xi} = 1
		\label{diracEn_ch_tex:L-T0_Anomale_Magnetische_Momente-0489}
	\end{equation}
	
	With $\beta_T = 1$ in natural units, this relationship fixes the scale parameter $\xi$ in terms of Standard Model parameters, eliminating any free parameters in the theory.
	
	\subsection{Mass Generation in T0 Framework}
	\label{diracEn_ch_tex:L-diracEn-0713}
	
	In the T0 model, mass generation occurs through:
	\begin{equation}
		m(\vec{x},t) = \frac{1}{\Tfieldt} = \max(m_{\text{particle}}, \omega)
		\label{diracEn_ch_tex:L-diracEn-0714}
	\end{equation}
	
	This provides a geometric interpretation of the Higgs mechanism through time field dynamics, unifying the electromagnetic and gravitational sectors.
	
	\subsection{Electromagnetic-Gravitational Unification}
	\label{diracEn_ch_tex:L-diracEn-0715}
	
	The condition $\alpha_{\text{EM}} = \beta_T = 1$ reveals the fundamental unity of electromagnetic and gravitational interactions in natural units:
	
	\begin{itemize}
		\item Both interactions have the same coupling strength
		\item Both couple to the time field with equal strength
		\item The unification occurs naturally without fine-tuning
		\item The hierarchy between different scales emerges from the $\xi$ parameter
	\end{itemize}
	
	\section{Conclusions and Future Directions}
	\label{diracEn_ch_tex:L-xi_parmater_partikel-0136}
	
	\subsection{Summary of Achievements}
	\label{diracEn_ch_tex:L-diracEn-0716}
	
	This analysis has successfully integrated the Dirac equation into the comprehensive T0 model framework:
	
	\begin{enumerate}
		\item \textbf{Geometric Matrix Structure}: The 4×4 matrices emerge naturally from T0 field geometry
		\item \textbf{Preserved Spin-Statistics}: The theorem remains valid with time field modifications
		\item \textbf{Precision QED}: T0 parameters yield specific predictions for anomalous magnetic moments
		\item \textbf{Dimensional Consistency}: All equations maintain perfect dimensional consistency
		\item \textbf{Parameter-Free Framework}: All values derived from fundamental Higgs physics
		\item \textbf{Experimental Testability}: Clear predictions at achievable precision levels
	\end{enumerate}
	
	\subsection{Key Insights}
	\label{diracEn_ch_tex:L-diracEn-0717}
	
	\subsubsection*{T0 Dirac Integration: Key Results}
\begin{itemize}
			\item The time-mass duality naturally accommodates relativistic quantum mechanics
			\item The three field geometries provide a complete framework for different physical scenarios
			\item Precision QED calculations yield testable predictions without adjustable parameters
			\item The connection to Higgs physics unifies quantum and gravitational scales
			\item The framework predicts universal lepton corrections at the $10^{-10}$ level
		\end{itemize}

	



% Bibliography
\begin{thebibliography}{99}
	
	\bibitem{pdg2024}
	Particle Data Group Collaboration (2024). 
	\textit{Review of Particle Physics}. 
	Progress of Theoretical and Experimental Physics, 2024(8), 083C01.
	\url{https://pdg.lbl.gov}
	
	\bibitem{flag2024}
	Aoki, Y., et al. (FLAG Collaboration) (2024). 
	\textit{FLAG Review 2024 of Lattice Results for Low-Energy Constants}. 
	arXiv:2411.04268.
	\url{https://arxiv.org/abs/2411.04268}
	
	\bibitem{fermilab_muon_g2}
	Abi, B., et al. (Muon g-2 Collaboration) (2021). 
	\textit{Measurement of the Positive Muon Anomalous Magnetic Moment to 0.46 ppm}. 
	Physical Review Letters, 126, 141801.
	
	\bibitem{peskin_schroeder}
	Peskin, M. E., \& Schroeder, D. V. (1995). 
	\textit{An Introduction to Quantum Field Theory}. 
	Addison-Wesley.
	
	\bibitem{weinberg_qft}
	Weinberg, S. (1995). 
	\textit{The Quantum Theory of Fields, Vol. I--III}. 
	Cambridge University Press.
	
	\bibitem{griffiths_particle}
	Griffiths, D. (2008). 
	\textit{Introduction to Elementary Particles}. 
	Wiley-VCH.
	
	\bibitem{mandl_shaw}
	Mandl, F., \& Shaw, G. (2010). 
	\textit{Quantum Field Theory (2nd ed.)}. 
	Wiley.
	
	\bibitem{srednicki_qft}
	Srednicki, M. (2007). 
	\textit{Quantum Field Theory}. 
	Cambridge University Press.
	
	\bibitem{t0_fundamentals}
	Pascher, J. (2024). 
	\textit{T0-Theory: Foundations of Time-Mass Duality}. 
	Unpublished manuscript, HTL Leonding.
	
	\bibitem{t0_fine_structure}
	Pascher, J. (2024). 
	\textit{T0-Theory: The Fine Structure Constant}. 
	Unpublished manuscript, HTL Leonding.
	
	\bibitem{t0_neutrinos}
	Pascher, J. (2024). 
	\textit{T0-Theory: Neutrino Masses and PMNS Mixing}. 
	Unpublished manuscript, HTL Leonding.
	
	\bibitem{t0_github}
	Pascher, J. (2024--2025). 
	\textit{T0-Time-Mass-Duality Repository}. 
	GitHub.
	\url{https://github.com/jpascher/T0-Time-Mass-Duality}
	
	\bibitem{lattice_qcd_review}
	Kronfeld, A. S. (2012). 
	\textit{Twenty-first Century Lattice Gauge Theory: Results from the QCD Lagrangian}. 
	Annual Review of Nuclear and Particle Science, 62, 265--284.
	
	\bibitem{neutrino_mixing_pdg}
	Particle Data Group Collaboration (2024). 
	\textit{Neutrino Masses, Mixing, and Oscillations}. 
	PDG Review 2024.
	\url{https://pdg.lbl.gov/2024/reviews/rpp2024-rev-neutrino-mixing.pdf}
	
	\bibitem{higgs_discovery}
	ATLAS and CMS Collaborations (2012). 
	\textit{Observation of a New Particle in the Search for the Standard Model Higgs Boson}. 
	Physics Letters B, 716, 1--29.
	
	\bibitem{Brannen2005}
	C. P. Brannen, ``Estimate of neutrino masses from Koide's relation'', \textit{arXiv:hep-ph/0505028} (2005).
	\url{https://arxiv.org/abs/hep-ph/0505028}
	
	\bibitem{Brannen2006}
	C. P. Brannen, ``Koide Mass Formula for Neutrinos'', \textit{arXiv:0702.0052} (2006).
	\url{http://brannenworks.com/MASSES.pdf}
	
	\bibitem{PhaseVectors2025}
	Anonymous, ``The Koide Relation and Lepton Mass Hierarchy from Phase Vectors'', \textit{rXiv:2507.0040} (2025).
	\url{https://rxiv.org/pdf/2507.0040v1.pdf}
	
	\bibitem{PDG2025}
	Particle Data Group, ``Review of Particle Physics'', \textit{Phys. Rev. D} \textbf{112} (2025) 030001.
	\url{https://pdg.lbl.gov/2025/}
	
	\bibitem{terrell2024}
	Terrell et al. (2024). 
	\textit{Single-Clock Metrology in Nature}. 
	Nature Physics.
	
	\bibitem{hossenfelder2024}
	Hossenfelder, S. (2024). 
	\textit{Single Clock Video Explanation}. 
	YouTube.
	
	\bibitem{hundert1931}
	Hundert (1931). 
	\textit{Reference Work}. 
	Publisher.
	
	\bibitem{terrell2025}
	Terrell et al. (2025). 
	\textit{Advanced Clock Synchronization Methods}. 
	Physical Review Letters.
	
	\bibitem{pascher_t0_2025}
	Pascher, J. (2025). 
	\textit{T0-Theory: Complete Framework and Applications}. 
	Unpublished manuscript, HTL Leonding.
	
	\bibitem{t0qm}
	Pascher, J. (2024). 
	\textit{T0-Theory: Quantum Mechanics Formulation}. 
	Unpublished manuscript, HTL Leonding.
	
	\bibitem{t0anomale}
	Pascher, J. (2024). 
	\textit{T0-Theory: Anomalous Magnetic Moments}. 
	Unpublished manuscript, HTL Leonding.
	
	\bibitem{muong2complete}
	Abi, B., et al. (Muon g-2 Collaboration) (2023). 
	\textit{Complete Measurement of the Positive Muon Anomalous Magnetic Moment}. 
	Physical Review Letters, 131, 161802.
	
	\bibitem{penrose2004}
	Penrose, R. (2004). 
	\textit{The Road to Reality: A Complete Guide to the Laws of the Universe}. 
	Jonathan Cape.
	
	\bibitem{planck1900}
	Planck, M. (1900). 
	\textit{On the Theory of the Energy Distribution Law of the Normal Spectrum}. 
	Verhandlungen der Deutschen Physikalischen Gesellschaft, 2, 237.
	
	\bibitem{T0Theory}
	Pascher, J. (2024). 
	\textit{T0-Theory: Fundamental Principles}. 
	Unpublished manuscript, HTL Leonding.
	
	% Additional bibliography entries for all undefined citations
	\bibitem{6g_roadmap}
	6G Research Consortium (2024).
	\textit{6G Technology Roadmap}.
	Technical Report.
	
	\bibitem{Born2013}
	Born, M. (2013).
	\textit{Einstein's Theory of Relativity}.
	Dover Publications.
	
	\bibitem{Casimir1948}
	Casimir, H. B. G. (1948).
	\textit{On the attraction between two perfectly conducting plates}.
	Proc. Kon. Ned. Akad. Wetensch. B51, 793--795.
	
	\bibitem{Einstein1905}
	Einstein, A. (1905).
	\textit{On the Electrodynamics of Moving Bodies}.
	Annalen der Physik, 17, 891--921.
	
	\bibitem{Feynman2006}
	Feynman, R. P. (2006).
	\textit{QED: The Strange Theory of Light and Matter}.
	Princeton University Press.
	
	\bibitem{Griffiths2017}
	Griffiths, D. J. (2017).
	\textit{Introduction to Electrodynamics (4th ed.)}.
	Cambridge University Press.
	
	\bibitem{Jackson1999}
	Jackson, J. D. (1999).
	\textit{Classical Electrodynamics (3rd ed.)}.
	Wiley.
	
	\bibitem{Mohr2016}
	Mohr, P. J., et al. (2016).
	\textit{CODATA Recommended Values of the Fundamental Physical Constants: 2014}.
	Rev. Mod. Phys. 88, 035009.
	
	\bibitem{Parker2018}
	Parker, R. H., et al. (2018).
	\textit{Measurement of the fine-structure constant as a test of the Standard Model}.
	Science, 360, 191--195.
	
	\bibitem{Planck1900}
	Planck, M. (1900).
	\textit{On the Theory of the Energy Distribution Law of the Normal Spectrum}.
	Verhandlungen der Deutschen Physikalischen Gesellschaft, 2, 237.
	
	\bibitem{Planck2018}
	Planck Collaboration (2018).
	\textit{Planck 2018 results. VI. Cosmological parameters}.
	Astronomy \& Astrophysics, 641, A6.
	
	\bibitem{QFT_T0}
	Pascher, J. (2024).
	\textit{T0-Theory and QFT Connections}.
	Unpublished manuscript, HTL Leonding.
	
	\bibitem{Sommerfeld1916}
	Sommerfeld, A. (1916).
	\textit{On the Quantum Theory of Spectral Lines}.
	Annalen der Physik, 51, 1--94.
	
	\bibitem{T0_Feinstruktur}
	Pascher, J. (2024).
	\textit{T0-Theory: Fine Structure Analysis}.
	Unpublished manuscript, HTL Leonding.
	
	\bibitem{T0_SI}
	Pascher, J. (2024).
	\textit{T0-Theory and SI Units}.
	Unpublished manuscript, HTL Leonding.
	
	\bibitem{T0_fine_structure}
	Pascher, J. (2024).
	\textit{T0-Theory: The Fine Structure Constant}.
	Unpublished manuscript, HTL Leonding.
	
	\bibitem{T0_g2_erweiterung}
	Pascher, J. (2024).
	\textit{T0-Theory: g-2 Extensions}.
	Unpublished manuscript, HTL Leonding.
	
	\bibitem{T0_gravitational_constant}
	Pascher, J. (2024).
	\textit{T0-Theory: Gravitational Constant Derivation}.
	Unpublished manuscript, HTL Leonding.
	
	\bibitem{T0_netze_en}
	Pascher, J. (2024).
	\textit{T0-Theory: Network Structures}.
	Unpublished manuscript, HTL Leonding.
	
	\bibitem{T0_tm_erweiterung}
	Pascher, J. (2024).
	\textit{T0-Theory: Time-Mass Extensions}.
	Unpublished manuscript, HTL Leonding.
	
	\bibitem{Uzan2003}
	Uzan, J.-P. (2003).
	\textit{The fundamental constants and their variation}.
	Rev. Mod. Phys. 75, 403--455.
	
	\bibitem{Weinberg1995}
	Weinberg, S. (1995).
	\textit{The Quantum Theory of Fields, Vol. I}.
	Cambridge University Press.
	
	\bibitem{albrecht1999}
	Albrecht, A. \& Magueijo, J. (1999).
	\textit{A time varying speed of light as a solution to cosmological puzzles}.
	Phys. Rev. D 59, 043516.
	
	\bibitem{alice2023}
	ALICE Collaboration (2023).
	\textit{Recent results from ALICE}.
	CERN-EP-2023-XXX.
	
	\bibitem{analog_optical}
	Smith, J. et al. (2024).
	\textit{Analog optical computing systems}.
	Nature Photonics.
	
	\bibitem{ashtekar2004}
	Ashtekar, A. \& Lewandowski, J. (2004).
	\textit{Background independent quantum gravity}.
	Class. Quantum Grav. 21, R53.
	
	\bibitem{atlas2023}
	ATLAS Collaboration (2023).
	\textit{ATLAS physics results}.
	CERN-PH-EP-2023-XXX.
	
	\bibitem{atlas2023higgs}
	ATLAS Collaboration (2023).
	\textit{Higgs boson measurements}.
	Phys. Rev. Lett.
	
	\bibitem{barbour1999}
	Barbour, J. (1999).
	\textit{The End of Time}.
	Oxford University Press.
	
	\bibitem{barrow1999}
	Barrow, J. D. (1999).
	\textit{Cosmologies with varying light speed}.
	Phys. Rev. D 59, 043515.
	
	\bibitem{becker2007}
	Becker, K. et al. (2007).
	\textit{String Theory and M-Theory}.
	Cambridge University Press.
	
	\bibitem{bell_muon}
	Bennett, G. W., et al. (Muon g-2 Collaboration) (2006).
	\textit{Final report of the E821 muon anomalous magnetic moment measurement}.
	Phys. Rev. D 73, 072003.
	
	\bibitem{bondi1948}
	Bondi, H. \& Gold, T. (1948).
	\textit{The steady-state theory of the expanding universe}.
	Mon. Not. R. Astron. Soc. 108, 252--270.
	
	\bibitem{brewer2019}
	Brewer, S. M. et al. (2019).
	\textit{Al+ Quantum-Logic Clock with Systematic Uncertainty below $10^{-18}$}.
	Phys. Rev. Lett. 123, 033201.
	
	\bibitem{cms2023top}
	CMS Collaboration (2023).
	\textit{Top quark measurements at CMS}.
	JHEP 2023.
	
	\bibitem{cms2024}
	CMS Collaboration (2024).
	\textit{CMS physics results 2024}.
	CERN-PH-EP-2024-XXX.
	
	\bibitem{codata2019}
	Tiesinga, E. et al. (2019).
	\textit{The 2018 CODATA Recommended Values}.
	J. Phys. Chem. Ref. Data.
	
	\bibitem{desi2025}
	DESI Collaboration (2025).
	\textit{DESI 2025 Cosmology Results}.
	arXiv preprint.
	
	\bibitem{differential_optical}
	Wang, X. et al. (2024).
	\textit{Differential optical computing}.
	Optica.
	
	\bibitem{dingle1972}
	Dingle, H. (1972).
	\textit{Science at the Crossroads}.
	Martin Brian \& O'Keeffe.
	
	\bibitem{divalentino2021}
	Di Valentino, E. et al. (2021).
	\textit{In the realm of the Hubble tension}.
	Class. Quantum Grav. 38, 153001.
	
	\bibitem{elnaschie2004}
	El Naschie, M. S. (2004).
	\textit{A review of E infinity theory}.
	Chaos, Solitons \& Fractals, 19, 209--236.
	
	\bibitem{fabrication_heterogeneous}
	Chen, Y. et al. (2024).
	\textit{Heterogeneous photonic integration}.
	Nature Electronics.
	
	\bibitem{fermilab2023}
	Fermilab (2023).
	\textit{Muon g-2 results}.
	Phys. Rev. Lett.
	
	\bibitem{flexible_wafer}
	Kim, S. et al. (2024).
	\textit{Flexible wafer-scale photonics}.
	Science Advances.
	
	\bibitem{francesco1997}
	Di Francesco, P. et al. (1997).
	\textit{Conformal Field Theory}.
	Springer.
	
	\bibitem{hartree1957}
	Hartree, D. R. (1957).
	\textit{The Calculation of Atomic Structures}.
	Wiley.
	
	\bibitem{hhi_6g}
	Fraunhofer HHI (2024).
	\textit{6G Photonic Integration}.
	Technical Report.
	
	\bibitem{hossenfelder2025}
	Hossenfelder, S. (2025).
	\textit{Science without the gobbledygook}.
	YouTube/Blog.
	
	\bibitem{hossenfelder_single_clock_video}
	Hossenfelder, S. (2024).
	\textit{The Single Clock Problem}.
	YouTube.
	
	\bibitem{hoyle1948}
	Hoyle, F. (1948).
	\textit{A new model for the expanding universe}.
	Mon. Not. R. Astron. Soc. 108, 372--382.
	
	\bibitem{integration_microelectronic}
	Liu, A. et al. (2024).
	\textit{Microelectronic photonic integration}.
	IEEE Journal.
	
	\bibitem{jacobson1995}
	Jacobson, T. (1995).
	\textit{Thermodynamics of spacetime}.
	Phys. Rev. Lett. 75, 1260.
	
	\bibitem{kasevich2023}
	Kasevich, M. et al. (2023).
	\textit{Atom interferometry tests}.
	Nature Physics.
	
	\bibitem{lerner2014}
	Lerner, E. J. (2014).
	\textit{An open letter on cosmology}.
	New Scientist.
	
	\bibitem{lisa2017}
	LISA Consortium (2017).
	\textit{Laser Interferometer Space Antenna}.
	ESA Technical Report.
	
	\bibitem{lithium_tantalate}
	Zhang, M. et al. (2024).
	\textit{Thin-film lithium tantalate photonics}.
	Nature Photonics.
	
	\bibitem{lopez2010}
	Lopez-Corredoira, M. (2010).
	\textit{Tests and problems of the standard model in cosmology}.
	Int. J. Mod. Phys. D.
	
	\bibitem{ludlow2015}
	Ludlow, A. D. et al. (2015).
	\textit{Optical atomic clocks}.
	Rev. Mod. Phys. 87, 637.
	
	\bibitem{mach1883}
	Mach, E. (1883).
	\textit{Die Mechanik in ihrer Entwickelung}.
	F.A. Brockhaus.
	
	\bibitem{maldacena1998}
	Maldacena, J. (1998).
	\textit{The large N limit of superconformal field theories}.
	Adv. Theor. Math. Phys. 2, 231--252.
	
	\bibitem{mueller2014}
	Müller, H. et al. (2014).
	\textit{Atom interferometry tests of the gravitational redshift}.
	Phys. Rev. Lett.
	
	\bibitem{mug2_final_2025}
	Muon g-2 Collaboration (2025).
	\textit{Final muon g-2 measurement}.
	Phys. Rev. Lett.
	
	\bibitem{muong2_2023}
	Muon g-2 Collaboration (2023).
	\textit{Updated muon g-2 results}.
	Phys. Rev. Lett.
	
	\bibitem{nathan2024}
	Nathan, A. et al. (2024).
	\textit{Quantum computing advances}.
	Nature.
	
	\bibitem{newell2018}
	Newell, D. B. et al. (2018).
	\textit{The CODATA 2017 values of h, e, k, and $N_A$}.
	Metrologia 55, L13.
	
	\bibitem{nottale1993}
	Nottale, L. (1993).
	\textit{Fractal Space-Time and Microphysics}.
	World Scientific.
	
	\bibitem{on_chip_lithium}
	Wang, C. et al. (2024).
	\textit{On-chip lithium niobate photonics}.
	Nature Communications.
	
	\bibitem{optical_advantages}
	Shastri, B. J. et al. (2024).
	\textit{Advantages of optical computing}.
	Nature Reviews Physics.
	
	\bibitem{pascher2025cmb}
	Pascher, J. (2025).
	\textit{T0-Theory: CMB Analysis}.
	Unpublished manuscript, HTL Leonding.
	
	\bibitem{pascher2025g2}
	Pascher, J. (2025).
	\textit{T0-Theory: g-2 Predictions}.
	Unpublished manuscript, HTL Leonding.
	
	\bibitem{pascher2025qm}
	Pascher, J. (2025).
	\textit{T0-Theory: Quantum Mechanics}.
	Unpublished manuscript, HTL Leonding.
	
	\bibitem{pascher2025si}
	Pascher, J. (2025).
	\textit{T0-Theory: SI Unit System}.
	Unpublished manuscript, HTL Leonding.
	
	\bibitem{pascher2025t0}
	Pascher, J. (2025).
	\textit{T0-Theory: Complete Framework}.
	Unpublished manuscript, HTL Leonding.
	
	\bibitem{pascher:fundamentals}
	Pascher, J. (2024).
	\textit{T0-Theory: Fundamentals}.
	Unpublished manuscript, HTL Leonding.
	
	\bibitem{pascher:g2_rev9}
	Pascher, J. (2024).
	\textit{T0-Theory: g-2 Revision 9}.
	Unpublished manuscript, HTL Leonding.
	
	\bibitem{pascher:geometric_formalism}
	Pascher, J. (2024).
	\textit{T0-Theory: Geometric Formalism}.
	Unpublished manuscript, HTL Leonding.
	
	\bibitem{pascher:ml_addendum}
	Pascher, J. (2024).
	\textit{T0-Theory: Machine Learning Addendum}.
	Unpublished manuscript, HTL Leonding.
	
	\bibitem{pascher:t0_foundations}
	Pascher, J. (2024).
	\textit{T0-Theory: Foundations}.
	Unpublished manuscript, HTL Leonding.
	
	\bibitem{pascher_derivation_beta_2025}
	Pascher, J. (2025).
	\textit{T0-Theory: Derivation of Beta}.
	Unpublished manuscript, HTL Leonding.
	
	\bibitem{pascher_higgs_connection_2025}
	Pascher, J. (2025).
	\textit{T0-Theory: Higgs Connection}.
	Unpublished manuscript, HTL Leonding.
	
	\bibitem{pascher_lagrangian_extended_2025}
	Pascher, J. (2025).
	\textit{T0-Theory: Extended Lagrangian}.
	Unpublished manuscript, HTL Leonding.
	
	\bibitem{pascher_mathematical_structure_2025}
	Pascher, J. (2025).
	\textit{T0-Theory: Mathematical Structure}.
	Unpublished manuscript, HTL Leonding.
	
	\bibitem{pascher_t0_cmb_2025}
	Pascher, J. (2025).
	\textit{T0-Theory: CMB Predictions}.
	Unpublished manuscript, HTL Leonding.
	
	\bibitem{pascher_t0_energie_2025}
	Pascher, J. (2025).
	\textit{T0-Theory: Energy}.
	Unpublished manuscript, HTL Leonding.
	
	\bibitem{pascher_t0_energy_2025}
	Pascher, J. (2025).
	\textit{T0-Theory: Energy Framework}.
	Unpublished manuscript, HTL Leonding.
	
	\bibitem{pascher_t0_theory_2025}
	Pascher, J. (2025).
	\textit{T0-Theory: Complete Theory}.
	Unpublished manuscript, HTL Leonding.
	
	\bibitem{penrose1959}
	Penrose, R. (1959).
	\textit{The apparent shape of a relativistically moving sphere}.
	Proc. Cambridge Phil. Soc. 55, 137--139.
	
	\bibitem{penrose1967}
	Penrose, R. (1967).
	\textit{Twistor algebra}.
	J. Math. Phys. 8, 345--366.
	
	\bibitem{peratt1992}
	Peratt, A. L. (1992).
	\textit{Physics of the Plasma Universe}.
	Springer-Verlag.
	
	\bibitem{peskin1995}
	Peskin, M. E. \& Schroeder, D. V. (1995).
	\textit{An Introduction to Quantum Field Theory}.
	Addison-Wesley.
	
	\bibitem{peskin_schroeder_1995}
	Peskin, M. E. \& Schroeder, D. V. (1995).
	\textit{An Introduction to Quantum Field Theory}.
	Addison-Wesley.
	
	\bibitem{phoquant}
	PhoQuant (2024).
	\textit{Photonic quantum computing}.
	Technical Report.
	
	\bibitem{photonics_ai}
	Wetzstein, G. et al. (2024).
	\textit{Photonics for AI}.
	Nature.
	
	\bibitem{planck1906}
	Planck, M. (1906).
	\textit{The Theory of Heat Radiation}.
	Johann Ambrosius Barth.
	
	\bibitem{planck2018}
	Planck Collaboration (2018).
	\textit{Planck 2018 results}.
	A\&A 641, A6.
	
	\bibitem{polchinski1998}
	Polchinski, J. (1998).
	\textit{String Theory}.
	Cambridge University Press.
	
	\bibitem{qant_nps}
	QANT (2024).
	\textit{Quantum photonics systems}.
	Technical Report.
	
	\bibitem{quantenjahr25}
	Quantenjahr (2025).
	\textit{International Year of Quantum}.
	UNESCO.
	
	\bibitem{recurrent_photonics}
	Tait, A. N. et al. (2024).
	\textit{Recurrent photonic neural networks}.
	Optica.
	
	\bibitem{rf_photonics}
	Capmany, J. \& Novak, D. (2024).
	\textit{Microwave photonics}.
	Nature Photonics.
	
	\bibitem{riess2019}
	Riess, A. G. et al. (2019).
	\textit{Large Magellanic Cloud Cepheid Standards}.
	ApJ 876, 85.
	
	\bibitem{riess2022}
	Riess, A. G. et al. (2022).
	\textit{A Comprehensive Measurement of H0}.
	ApJ 934, L7.
	
	\bibitem{rovelli2004}
	Rovelli, C. (2004).
	\textit{Quantum Gravity}.
	Cambridge University Press.
	
	\bibitem{sciama1953}
	Sciama, D. W. (1953).
	\textit{On the origin of inertia}.
	Mon. Not. R. Astron. Soc. 113, 34--42.
	
	\bibitem{sciencedaily2025}
	ScienceDaily (2025).
	\textit{Physics news}.
	Online.
	
	\bibitem{sm_g2_2025}
	Aoyama, T. et al. (2025).
	\textit{Standard Model prediction for g-2}.
	Phys. Rep.
	
	\bibitem{susskind1995}
	Susskind, L. (1995).
	\textit{The world as a hologram}.
	J. Math. Phys. 36, 6377--6396.
	
	\bibitem{t0_kosmologie}
	Pascher, J. (2024).
	\textit{T0-Theory: Cosmology}.
	Unpublished manuscript, HTL Leonding.
	
	\bibitem{terrell1959}
	Terrell, J. (1959).
	\textit{Invisibility of the Lorentz contraction}.
	Phys. Rev. 116, 1041--1045.
	
	\bibitem{terrell_single_clock_nature_2024}
	Terrell, J. et al. (2024).
	\textit{Single clock precision measurements}.
	Nature Physics.
	
	\bibitem{tfln_foundry}
	TFLN Foundry (2024).
	\textit{Thin-film lithium niobate foundry services}.
	Technical Specifications.
	
	\bibitem{thiemann2007}
	Thiemann, T. (2007).
	\textit{Modern Canonical Quantum General Relativity}.
	Cambridge University Press.
	
	\bibitem{thz_epfl}
	EPFL (2024).
	\textit{Terahertz photonics research}.
	Technical Report.
	
	\bibitem{unnikrishnan2004}
	Unnikrishnan, C. S. (2004).
	\textit{On Einstein's resolution of the twin clock paradox}.
	Current Science, 86, 704--709.
	
	\bibitem{verlinde2011}
	Verlinde, E. (2011).
	\textit{On the origin of gravity and the laws of Newton}.
	JHEP 2011, 29.
	
	\bibitem{video2025}
	Video (2025).
	\textit{Physics video explanation}.
	YouTube.
	
	\bibitem{weinberg1995}
	Weinberg, S. (1995).
	\textit{The Quantum Theory of Fields}.
	Cambridge University Press.
	
	\bibitem{weiskopf2000}
	Weiskopf, D. (2000).
	\textit{Visualization of special relativity}.
	PhD thesis, University of Tübingen.
	
	\bibitem{wheeler1990}
	Wheeler, J. A. (1990).
	\textit{A Journey into Gravity and Spacetime}.
	Scientific American Library.
	
	\bibitem{wiki_bell}
	Wikipedia (2024).
	\textit{Bell's theorem}.
	Online encyclopedia.
	
	\bibitem{zwicky1929}
	Zwicky, F. (1929).
	\textit{On the red shift of spectral lines through interstellar space}.
	Proc. Natl. Acad. Sci. 15, 773--779.

\end{thebibliography}


\end{document}
