\documentclass[11pt,a4paper]{article}
\usepackage[a4paper,margin=2cm]{geometry}
\usepackage[utf8]{inputenc}
\usepackage[english]{babel}
\usepackage{lmodern}
\renewcommand{\familydefault}{\sfdefault}

\usepackage{amsmath,amssymb,amsthm}
\usepackage{graphicx}
\usepackage[unicode,pdfencoding=auto,hypertexnames=false]{hyperref}
\usepackage{booktabs}
\usepackage{longtable}
\usepackage{array}
\usepackage{siunitx}
\usepackage{fancyhdr}
\usepackage{float}
\usepackage{tikz}
% tcolorbox removed for standalone
% tcbset removed
\tikzset{
  t0blue/.style={draw=blue,fill=blue!10},
  t0red/.style={draw=red,fill=red!10},
  t0green/.style={draw=green!50!black,fill=green!10},
  t0orange/.style={draw=orange,fill=orange!10},
}
\usepackage{setspace}
\usepackage{enumitem}
\usepackage{adjustbox}
\usepackage{xcolor}

% Define colors for xcolor package
\definecolor{t0green}{RGB}{34,139,34}
\definecolor{t0blue}{RGB}{0,0,255}
\definecolor{t0red}{RGB}{255,0,0}
\definecolor{t0orange}{RGB}{255,165,0}

% Define custom column types for tables
\newcolumntype{L}[1]{>{\raggedright\arraybackslash}p{#1}}
\newcolumntype{C}[1]{>{\centering\arraybackslash}p{#1}}
\newcolumntype{R}[1]{>{\raggedleft\arraybackslash}p{#1}}

\setlength{\parindent}{0pt}
\setlength{\parskip}{6pt}

\hypersetup{
  colorlinks=true,
  linkcolor=blue,
  citecolor=blue,
  urlcolor=blue
}
\pagestyle{fancy}
\setlength{\headheight}{28pt}

\newcommand{\checkmarkx}{\checkmark}
\newcommand{\warningx}{\textbf{!}}

% Makros aus Einzel-Dokumenten (Fallback-Definitionen)
\newcommand{\mytimes}{\times}
\newcommand{\myapprox}{\approx}
\newcommand{\mysim}{\sim}
\newcommand{\myomega}{\omega}
\newcommand{\mypi}{\pi}
\newcommand{\myrightarrow}{\rightarrow}
\newcommand{\mypropto}{\propto}
\newcommand{\deltafield}{\delta\phi}
\newcommand{\xipar}{\xi}
\newcommand{\xiT}{\xi}
\newcommand{\lambdah}{\lambda_h}

% Additional macros used in chapter files
\newcommand{\Kfrak}{K_{\text{frak}}}  % Fractal correction factor
\newcommand{\Dfrak}{D_f}              % Fractal dimension
\newcommand{\betapar}{\beta}          % T0 beta parameter
\newcommand{\alphapar}{\alpha}        % T0 alpha parameter
\newcommand{\Efield}{E}               % Energy field
% Note: checkmarkxa/warningxa are variants used in auto-generated chapter files
\newcommand{\checkmarkxa}{\checkmark}
\newcommand{\warningxa}{\textbf{!}}

% Additional T0-specific macros
\newcommand{\xigeom}{\xi_{\text{geom}}}  % Geometric xi
\newcommand{\lP}{\ell_P}                  % Planck length
\newcommand{\rzero}{r_0}                  % Characteristic radius
\newcommand{\xirat}{\xi_{\text{rat}}}     % Xi ratio
\newcommand{\tzero}{t_0}                  % Characteristic time
\newcommand{\natunits}{\text{(nat. units)}}  % Natural units annotation
\newcommand{\myRightarrow}{\Rightarrow}   % Arrow variant
\newcommand{\Lag}{\mathcal{L}}            % Lagrangian

% Physics macros used in chapter files
\newcommand{\CQCD}{C_{\text{QCD}}}        % QCD correction
\newcommand{\EP}{E_P}                     % Planck energy
\newcommand{\Ee}{E_e}                     % Electron energy
\newcommand{\Emu}{E_\mu}                  % Muon energy
\newcommand{\Exi}{E_\xi}                  % Xi energy
\newcommand{\Ezero}{E_0}                  % Characteristic energy
\newcommand{\Hubble}{H}                   % Hubble constant
\newcommand{\Kspec}{K_{\text{spec}}}      % Spectral correction
\newcommand{\Lambdat}{\Lambda_t}          % Time-related cosmological constant
\newcommand{\Leff}{\mathcal{L}_{\text{eff}}}  % Effective Lagrangian
\newcommand{\Lorentz}{\mathcal{L}}        % Lorentz symbol
\newcommand{\Lxi}{L_\xi}                  % Xi length
\newcommand{\Tfield}{T}                   % Time field
\newcommand{\Weyl}{W}                     % Weyl tensor/symbol
\newcommand{\alphaEMSI}{\alpha_{\text{EM,SI}}}  % EM alpha in SI
\newcommand{\alphaEMnat}{\alpha_{\text{EM,nat}}}  % EM alpha in natural units
\newcommand{\alphaem}{\alpha_{\text{em}}} % Electromagnetic alpha
\newcommand{\betaTSI}{\beta_{T,\text{SI}}}  % Beta in SI
\newcommand{\betaTnat}{\beta_{T,\text{nat}}}  % Beta in natural units
\newcommand{\deltam}{\delta m}            % Mass difference
\newcommand{\phiT}{\phi_T}                % T-field phi
\newcommand{\tP}{t_P}                     % Planck time
\newcommand{\rhoCMB}{\rho_{\text{CMB}}}   % CMB density
\newcommand{\rhoCasimir}{\rho_{\text{Casimir}}}  % Casimir density

% Table formatting
\usepackage{multirow}

% Additional physics macros
\newcommand{\Riem}{\mathcal{R}}           % Riemann tensor
\newcommand{\ZPinch}{Z_{\text{pinch}}}    % Z-pinch
\newcommand{\SynchPower}{P_{\text{synch}}} % Synchrotron power
\newcommand{\Rzero}{R_0}                  % Characteristic radius
\newcommand{\alphafine}{\alpha}           % Fine structure constant
\newcommand{\Etau}{E_\tau}                % Tau energy
\newcommand{\deltaE}{\delta E}            % Energy deviation
\newcommand{\EPlanck}{E_P}                % Planck energy
\newcommand{\pichar}{\pi}                 % Pi character
\newcommand{\alphaWSI}{\alpha_{W,\text{SI}}}  % Wien alpha in SI
\newcommand{\alphaWnat}{\alpha_{W,\text{nat}}}  % Wien alpha in natural units

% Einfache abstract-Umgebung für Kapitel:
\newenvironment{abstract}{%
  \begin{center}\bfseries Abstract\end{center}\small
}{\par}


\title{EliminationOfMassEn}
\author{J. Pascher}
\date{\today}

\begin{document}
\maketitle

\section*{Eliminationofmassen (EliminationOfMassEn)}

	\begin{abstract}
		This paper demonstrates that the mass parameter $m$ appearing throughout the T0 model formulations serves exclusively as a dimensional placeholder and can be systematically eliminated from all equations. Through rigorous dimensional analysis and mathematical reformulation, we show that the apparent dependence on specific particle masses is an artifact of conventional notation rather than fundamental physics. The elimination of $m$ reveals the T0 model as a truly parameter-free theory based solely on the Planck scale, providing universal scaling laws and eliminating systematic biases from empirical mass determinations. This work establishes the mathematical foundation for a complete ab-initio formulation of the T0 model requiring no external experimental inputs beyond the fundamental constants $\hbar$, $c$, $G$, and $k_B$.
	\end{abstract}
	
	
	\section{Introduction}
	\label{EliminationOfMa:L-T0_tm-erweiterung-x6-0008}
	
	\subsection{The Problem of Mass Parameters}
	\label{EliminationOfMa:L-EliminationOfMassEn-1126}
	
	The T0 model, as formulated in previous works, appears to depend critically on specific particle masses such as the electron mass $m_e$, proton mass $m_p$, and Higgs boson mass $m_h$. This apparent dependence has led to concerns about the model's predictive power and its reliance on empirical inputs that may themselves be contaminated by Standard Model assumptions.
	
	A careful analysis reveals, however, that the mass parameter $m$ serves a purely **dimensional function** in the T0 equations. This paper demonstrates that $m$ can be systematically eliminated from all formulations, revealing the T0 model as a fundamentally parameter-free theory based exclusively on Planck-scale physics.
	
	\subsection{Dimensional Analysis Approach}
	\label{EliminationOfMa:L-EliminationOfMassEn-1127}
	
	In natural units where $\hbar = c = G = k_B = 1$, all physical quantities can be expressed as powers of energy $[E]$:
	
	\begin{align}
		\text{Length:} \quad [L] &= [E^{-1}] \\
		\text{Time:} \quad [T] &= [E^{-1}] \\
		\text{Mass:} \quad [M] &= [E] \\
		\text{Temperature:} \quad [\Theta] &= [E]
	\end{align}
	
	This dimensional structure suggests that mass parameters may be replaceable by energy scales, leading to more fundamental formulations.
	
	\section{Systematic Mass Elimination}
	\label{EliminationOfMa:L-T0_Energie-0224}
	
	\subsection{The Intrinsic Time Field}
	\label{EliminationOfMa:L-EliminationOfMassEn-1128}
	
	\subsubsection{Original Formulation}
	
	The intrinsic time field is traditionally defined as:
	
	\begin{equation}
		\Tfieldt = \frac{1}{\max(m(\vecx,t), \omega)}
		\label{EliminationOfMa:L-EliminationOfMassEn-1129}
	\end{equation}
	
\section*{Dimensional analysis:}
	\begin{itemize}
		\item $[\Tfieldt] = [E^{-1}]$ (time field dimension)
		\item $[m] = [E]$ (mass as energy)
		\item $[\omega] = [E]$ (frequency as energy)
		\item $[1/\max(m,\omega)] = [E^{-1}]$ \checkmark
	\end{itemize}
	
	\subsubsection{Mass-Free Reformulation}
	
	The fundamental insight is that only the **ratio** between characteristic energy and frequency matters physically. We reformulate as:
	
	\begin{equation}
		\boxed{\Tfieldt = \tP \cdot g(E_{\text{norm}}(\vecx,t), \omega_{\text{norm}})}
		\label{EliminationOfMa:L-EliminationOfMassEn-1130}
	\end{equation}
	
	where:
	\begin{align}
		\tP &= \sqrt{\frac{\hbar G}{c^5}} \quad \text{(Planck time)} \\
		E_{\text{norm}} &= \frac{E(\vecx,t)}{\EP} \quad \text{(normalized energy)} \\
		\omega_{\text{norm}} &= \frac{\omega}{\EP} \quad \text{(normalized frequency)} \\
		g(E_{\text{norm}}, \omega_{\text{norm}}) &= \frac{1}{\max(E_{\text{norm}}, \omega_{\text{norm}})}
	\end{align}
	
	\textbf{Result:} Mass completely eliminated, only Planck scale and dimensionless ratios remain.
	
	\subsection{Field Equation Reformulation}
	\label{EliminationOfMa:L-EliminationOfMassEn-1131}
	
	\subsubsection{Original Field Equation}
	
	\begin{equation}
		\nabla^2 \Tfield = -4\pi G \rho(\vecx) \Tfield^2
		\label{EliminationOfMa:L-EliminationOfMassEn-1132}
	\end{equation}
	
	with mass density $\rho(\vecx) = m \cdot \delta^3(\vecx)$ for a point source.
	
	\subsubsection{Energy-Based Formulation}
	
	Replacing mass density with energy density:
	
	\begin{equation}
		\boxed{\nabla^2 \Tfield = -4\pi G \frac{E(\vecx)}{\EP} \delta^3(\vecx) \frac{\Tfield^2}{\tP^2}}
		\label{EliminationOfMa:L-EliminationOfMassEn-1133}
	\end{equation}
	
\section*{Dimensional verification:}
	\begin{align}
		[\nabla^2 \Tfield] &= [E^{-1} \cdot E^2] = [E] \\
		[4\pi G E_{\text{norm}} \delta^3(\vecx) \Tfield^2/\tP^2] &= [E^{-2}][1][E^6][E^{-2}]/[E^{-2}] = [E] \quad \checkmark
	\end{align}
	
	\subsection{Point Source Solution: Parameter Separation}
	\label{EliminationOfMa:L-EliminationOfMassEn-1134}
	
	\subsubsection{The Mass Redundancy Problem}
	
	The traditional point source solution exhibits apparent mass redundancy:
	
	\begin{equation}
		\Tfield(r) = \frac{1}{m}\left(1 - \frac{r_0}{r}\right)
		\label{EliminationOfMa:L-EliminationOfMassEn-1135}
	\end{equation}
	
	with $r_0 = 2Gm$. Substituting:
	
	\begin{equation}
		\Tfield(r) = \frac{1}{m}\left(1 - \frac{2Gm}{r}\right) = \frac{1}{m} - \frac{2G}{r}
		\label{EliminationOfMa:L-EliminationOfMassEn-1136}
	\end{equation}
	
	\textbf{Critical observation:} Mass $m$ appears in \textbf{two different roles}:
	\begin{enumerate}
		\item As normalization factor $(1/m)$
		\item As source parameter $(2Gm)$
	\end{enumerate}
	
	This suggests that $m$ masks **two independent physical scales**.
	
	\subsubsection{Parameter Separation Solution}
	
	We reformulate with independent parameters:
	
	\begin{equation}
		\boxed{\Tfield(r) = \Tzero\left(1 - \frac{L_0}{r}\right)}
		\label{EliminationOfMa:L-EliminationOfMassEn-1137}
	\end{equation}
	
	where:
	\begin{itemize}
		\item $\Tzero$: Characteristic time scale $[E^{-1}]$
		\item $L_0$: Characteristic length scale $[E^{-1}]$
	\end{itemize}
	
\section*{Physical interpretation:}
	\begin{itemize}
		\item $\Tzero$ determines the \textbf{amplitude} of the time field
		\item $L_0$ determines the \textbf{range} of the time field
		\item Both derivable from source geometry without specific masses
	\end{itemize}
	
	\subsection{The Parameter: Universal Scaling}
	\label{EliminationOfMa:L-EliminationOfMassEn-1138}
	
	\subsubsection{Traditional Mass-Dependent Definition}
	
	\begin{equation}
		\xipar = 2\sqrt{G} \cdot m
		\label{EliminationOfMa:L-EliminationOfMassEn-1139}
	\end{equation}
	
	\textbf{Problem:} Requires specific particle masses as input.
	
	\subsubsection{Universal Energy-Based Definition}
	
	\begin{equation}
		\boxed{\xipar = 2\sqrt{\frac{E_{\text{characteristic}}}{\EP}}}
		\label{EliminationOfMa:L-EliminationOfMassEn-1140}
	\end{equation}
	
\section*{Universal scaling for different energy scales:}
	\begin{align}
		\text{Planck energy } (E = \EP): \quad &\xipar = 2 \\
		\text{Electroweak scale } (E \sim 100 \text{ GeV}): \quad &\xipar \sim 10^{-8} \\
		\text{QCD scale } (E \sim 1 \text{ GeV}): \quad &\xipar \sim 10^{-9} \\
		\text{Atomic scale } (E \sim 1 \text{ eV}): \quad &\xipar \sim 10^{-28}
	\end{align}
	
\section*{No specific particle masses required!}
	
	\section{Complete Mass-Free T0 Formulation}
	\label{EliminationOfMa:L-EliminationOfMassEn-1141}
	
	\subsection{Fundamental Equations}
	\label{EliminationOfMa:L-EliminationOfMassEn-1142}
	
	The complete mass-free T0 system:
	
	\subsubsection*{Mass-Free T0 Model}
\begin{align}
			\text{Time field:} \quad &\Tfieldt = \tP \cdot f(E_{\text{norm}}(\vecx,t), \omega_{\text{norm}}) \\
			\text{Field equation:} \quad &\nabla^2 \Tfield = -4\pi G \frac{E_{\text{norm}}}{\lP^2} \delta^3(\vecx) \Tfield^2 \\
			\text{Point sources:} \quad &\Tfield(r) = \Tzero\left(1 - \frac{L_0}{r}\right) \\
			\text{Coupling parameter:} \quad &\xipar = 2\sqrt{\frac{E}{\EP}}
		\end{align}

	
	\subsection{Parameter Count Analysis}
	\label{EliminationOfMa:L-EliminationOfMassEn-1143}
	
	\begin{center}
		\begin{tabular}{|l|c|c|}
			\hline
			\textbf{Formulation} & \textbf{Before Mass Elimination} & \textbf{After Mass Elimination} \\
			\hline
			\hline
			Fundamental constants & $\hbar, c, G, k_B$ & $\hbar, c, G, k_B$ \\
			\hline
			Particle-specific masses & $m_e, m_\mu, m_p, m_h, \ldots$ & None \\
			\hline
			Dimensionless ratios & None explicit & $E/\EP$, $L/\lP$, $T/\tP$ \\
			\hline
			Free parameters & $\infty$ (one per particle) & 0 \\
			\hline
			Empirical inputs required & Yes (masses) & No \\
			\hline
		\end{tabular}
	\end{center}
	
	\subsection{Dimensional Consistency Verification}
	\label{EliminationOfMa:L-EliminationOfMassEn-1144}
	
	\begin{table}[htbp]
		\centering
		\begin{tabular}{lccl}
			\toprule
			\textbf{Equation} & \textbf{Left Side} & \textbf{Right Side} & \textbf{Status} \\
			\midrule
			Time field & $[\Tfieldt] = [E^{-1}]$ & $[\tP \cdot f(\cdot)] = [E^{-1}]$ & \checkmark \\
			Field equation & $[\nabla^2 \Tfield] = [E]$ & $[G E_{\text{norm}} \delta^3 \Tfield^2/\lP^2] = [E]$ & \checkmark \\
			Point source & $[\Tfield(r)] = [E^{-1}]$ & $[\Tzero(1-L_0/r)] = [E^{-1}]$ & \checkmark \\
			$\xipar$ parameter & $[\xipar] = [1]$ & $[\sqrt{E/\EP}] = [1]$ & \checkmark \\
			\bottomrule
		\end{tabular}
		\caption{Dimensional consistency of mass-free formulations}
	\end{table}
	
	\section{Experimental Implications}
	\label{EliminationOfMa:L-xi_parmater_partikel-0118}
	
	\subsection{Universal Predictions}
	\label{EliminationOfMa:L-EliminationOfMassEn-1145}
	
	The mass-free T0 model makes universal predictions independent of specific particle properties:
	
	\subsubsection{Scaling Laws}
	
	\begin{equation}
		\xipar(E) = 2\sqrt{\frac{E}{\EP}}
		\label{EliminationOfMa:L-T0_Energie-0313}
	\end{equation}
	
	This relationship must hold for \textbf{all} energy scales, providing a stringent test of the theory.
	
	\subsubsection{QED Anomalies}
	
	The electron anomalous magnetic moment becomes:
	
	\begin{equation}
		a_e^{(\text{T0})} = \frac{\alpha}{2\pi} \cdot C_{\text{T0}} \cdot \left(\frac{E_e}{\EP}\right)
		\label{EliminationOfMa:L-EliminationOfMassEn-1146}
	\end{equation}
	
	where $E_e$ is the characteristic energy scale of the electron, not its rest mass.
	
	\subsubsection{Gravitational Effects}
	
	\begin{equation}
		\Phi(r) = -\frac{G E_{\text{source}}}{\EP} \cdot \frac{\lP}{r}
		\label{EliminationOfMa:L-EliminationOfMassEn-1147}
	\end{equation}
	
	Universal scaling for all gravitational sources.
	
	\subsection{Elimination of Systematic Biases}
	\label{EliminationOfMa:L-EliminationOfMassEn-1148}
	
	\subsubsection{Problems with Mass-Dependent Formulations}
	
	Traditional approaches suffer from:
	\begin{itemize}
		\item \textbf{Circular dependencies:} Using experimentally determined masses to predict the same experiments
		\item \textbf{Standard Model contamination:} All mass measurements assume SM physics
		\item \textbf{Precision illusions:} High apparent precision masking systematic theoretical errors
	\end{itemize}
	
	\subsubsection{Advantages of Mass-Free Approach}
	
	\begin{itemize}
		\item \textbf{Model independence:} No reliance on potentially biased mass determinations
		\item \textbf{Universal tests:} Same scaling laws apply across all energy scales
		\item \textbf{Theoretical purity:} Ab-initio predictions from Planck scale alone
	\end{itemize}
	
	\subsection{Proposed Experimental Tests}
	\label{EliminationOfMa:L-EliminationOfMassEn-1149}
	
	\subsubsection{Multi-Scale Consistency}
	
	Test the universal scaling relation:
	\begin{equation}
		\frac{\xipar(E_1)}{\xipar(E_2)} = \sqrt{\frac{E_1}{E_2}}
		\label{EliminationOfMa:L-EliminationOfMassEn-1150}
	\end{equation}
	
	across different energy scales: atomic, nuclear, electroweak, and cosmological.
	
	\subsubsection{Energy-Dependent Anomalies}
	
	Measure anomalous magnetic moments as functions of energy scale rather than particle identity:
	\begin{equation}
		a(E) = a_{\text{SM}}(E) + a^{(\text{T0})}(E/\EP)
		\label{EliminationOfMa:L-EliminationOfMassEn-1151}
	\end{equation}
	
	\subsubsection{Geometric Independence}
	
	Verify that $\Tzero$ and $L_0$ can be determined independently from source geometry without requiring specific mass values.
	
	\section{Geometric Parameter Determination}
	\label{EliminationOfMa:L-EliminationOfMassEn-1152}
	
	\subsection{Source Geometry Analysis}
	\label{EliminationOfMa:L-EliminationOfMassEn-1153}
	
	\subsubsection{Spherically Symmetric Sources}
	
	For a spherically symmetric energy distribution $E(r)$:
	
	\begin{align}
		\Tzero &= \tP \cdot f\left(\frac{\int E(r) d^3r}{\EP}\right) \\
		L_0 &= \lP \cdot g\left(\frac{R_{\text{characteristic}}}{\lP}\right)
	\end{align}
	
	where $f$ and $g$ are dimensionless functions determined by the field equations.
	
	\subsubsection{Non-Spherical Sources}
	
	For general geometries, the parameters become tensorial:
	
	\begin{align}
		\Tzero^{ij} &= \tP \cdot f_{ij}\left(\frac{I^{ij}}{\EP \lP^2}\right) \\
		L_0^{ij} &= \lP \cdot g_{ij}\left(\frac{I^{ij}}{\lP^2}\right)
	\end{align}
	
	where $I^{ij}$ is the energy moment tensor of the source.
	
	\subsection{Universal Geometric Relations}
	\label{EliminationOfMa:L-EliminationOfMassEn-1154}
	
	The mass-free formulation reveals universal relationships between geometric and energetic properties:
	
	\begin{equation}
		\frac{L_0}{\lP} = h\left(\frac{\Tzero}{\tP}, \text{shape parameters}\right)
		\label{EliminationOfMa:L-EliminationOfMassEn-1155}
	\end{equation}
	
	These relationships are \textbf{independent of specific mass values} and depend only on:
	\begin{itemize}
		\item Energy distribution geometry
		\item Planck-scale ratios
		\item Dimensionless shape parameters
	\end{itemize}
	
	\section{Connection to Fundamental Physics}
	\label{EliminationOfMa:L-EliminationOfMassEn-1156}
	
	\subsection{Emergent Mass Concept}
	\label{EliminationOfMa:L-T0_Energie-0239}
	
	\subsubsection{Mass as Effective Parameter}
	
	In the mass-free formulation, what we traditionally call "mass" emerges as:
	
	\begin{equation}
		m_{\text{effective}} = E_{\text{characteristic}} \cdot f(\text{geometry}, \text{couplings})
		\label{EliminationOfMa:L-EliminationOfMassEn-1157}
	\end{equation}
	
\section*{Different "masses" for different contexts:}
	\begin{itemize}
		\item \textbf{Rest mass:} Intrinsic energy scale of localized excitation
		\item \textbf{Gravitational mass:} Coupling strength to spacetime curvature  
		\item \textbf{Inertial mass:} Resistance to acceleration in external fields
	\end{itemize}
	
	All reducible to \textbf{energy scales and geometric factors}.
	
	\subsubsection{Resolution of Mass Hierarchies}
	
	The apparent hierarchy of particle masses becomes a hierarchy of \textbf{energy scales}:
	
	\begin{align}
		\frac{m_t}{m_e} &\rightarrow \frac{E_{\text{top}}}{E_{\text{electron}}} \\
		\frac{m_W}{m_e} &\rightarrow \frac{E_{\text{electroweak}}}{E_{\text{electron}}} \\
		\frac{m_P}{m_e} &\rightarrow \frac{\EP}{E_{\text{electron}}}
	\end{align}
	
	\textbf{No fundamental mass parameters}, only energy scale ratios.
	
	\subsection{Unification with Planck Scale Physics}
	\label{EliminationOfMa:L-EliminationOfMassEn-1158}
	
	\subsubsection{Natural Scale Emergence}
	
	All physics naturally organizes around Planck scale:
	
	\begin{align}
		\text{Microscopic physics:} \quad &E \ll \EP, \quad L \gg \lP \\
		\text{Macroscopic physics:} \quad &E \ll \EP, \quad L \gg \lP \\
		\text{Quantum gravity:} \quad &E \sim \EP, \quad L \sim \lP
	\end{align}
	
	\subsubsection{Scale-Dependent Effective Theories}
	
	Different energy regimes correspond to different limits of the universal T0 theory:
	
	\begin{align}
		E \ll \EP: \quad &\text{Standard Model limit} \\
		E \sim \text{TeV}: \quad &\text{Electroweak unification} \\
		E \sim \EP: \quad &\text{Quantum gravity unification}
	\end{align}
	
	\section{Philosophical Implications}
	\label{EliminationOfMa:L-EliminationOfMassEn-1159}
	
	\subsection{Reductionism to Planck Scale}
	\label{EliminationOfMa:L-EliminationOfMassEn-1160}
	
	The elimination of mass parameters demonstrates that \textbf{all physics} is reducible to the \textbf{Planck scale}:
	
	\begin{itemize}
		\item No fundamental mass parameters exist
		\item Only energy and length ratios matter
		\item Universal dimensionless couplings emerge naturally
		\item True parameter-free physics achieved
	\end{itemize}
	
	\subsection{Ontological Implications}
	\label{EliminationOfMa:L-EliminationOfMassEn-1161}
	
	\subsubsection{Mass as Human Construct}
	
	The traditional concept of "mass" appears to be a \textbf{human construct} rather than fundamental reality:
	
	\begin{itemize}
		\item Useful for practical calculations
		\item Not present in deepest level of theory
		\item Emergent from more fundamental energy relationships
	\end{itemize}
	
	\subsubsection{Universal Energy Monism}
	
	The mass-free T0 model supports a form of \textbf{energy monism}:
	\begin{itemize}
		\item Energy as the only fundamental quantity
		\item All other quantities as energy relationships
		\item Space and time as energy-derived concepts
		\item Matter as structured energy patterns
	\end{itemize}
	
	\section{Conclusions}
	\label{EliminationOfMa:L-xi_parmater_partikel-0136}
	
	\subsection{Summary of Results}
	\label{EliminationOfMa:L-EliminationOfMassEn-1162}
	
	We have demonstrated that:
	
	\begin{enumerate}
		\item \textbf{Mass $m$ serves only as dimensional placeholder} in T0 formulations
		\item \textbf{All equations can be systematically reformulated} without mass parameters
		\item \textbf{Universal scaling laws emerge} based solely on Planck scale
		\item \textbf{True parameter-free theory} results from mass elimination
		\item \textbf{Experimental predictions become model-independent}
	\end{enumerate}
	
	\subsection{Theoretical Significance}
	\label{EliminationOfMa:L-T0_umkehrung-0610}
	
	The mass elimination reveals the T0 model as:
	
	\subsubsection*{T0 Model: True Nature}
\begin{itemize}
			\item \textbf{Truly fundamental theory} based on Planck scale alone
			\item \textbf{Parameter-free formulation} with universal predictions
			\item \textbf{Unification of all energy scales} through dimensionless ratios
			\item \textbf{Resolution of fine-tuning problems} via scale relationships
		\end{itemize}

	
	\subsection{Experimental Program}
	\label{EliminationOfMa:L-EliminationOfMassEn-1163}
	
	The mass-free formulation enables:
	
	\begin{itemize}
		\item \textbf{Model-independent tests} of universal scaling
		\item \textbf{Elimination of systematic biases} from mass measurements
		\item \textbf{Direct connection} between quantum and gravitational scales
		\item \textbf{Ab-initio predictions} from pure theory
	\end{itemize}
	
	\subsection{Future Directions}
	\label{EliminationOfMa:L-MathZeitMasseLagrangeEn-0830}
	
	\subsubsection{Immediate Research Priorities}
	
	\begin{enumerate}
		\item \textbf{Complete geometric formulation:} Develop full tensor treatment for arbitrary source geometries
		\item \textbf{Quantum field theory extension:} Formulate mass-free QFT on T0 background
		\item \textbf{Cosmological applications:} Apply to large-scale structure without dark matter/energy
		\item \textbf{Experimental design:} Develop tests of universal scaling laws
	\end{enumerate}
	
	\subsubsection{Long-term Goals}
	
	\begin{itemize}
		\item Complete replacement of Standard Model with mass-free T0 theory
		\item Unification of all interactions through energy scale relationships
		\item Resolution of quantum gravity through Planck-scale physics
		\item Experimental verification of parameter-free predictions
	\end{itemize}
	
	\section{Final Remarks}
	\label{EliminationOfMa:L-Moll_CandelaEn-1125}
	
	The elimination of mass as a fundamental parameter represents more than a technical improvement—it reveals the \textbf{true nature of physical reality} as organized around energy relationships and geometric structures. 
	
	The apparent complexity of particle physics, with its multitude of masses and coupling constants, emerges from our limited perspective on more fundamental energy scale relationships. The T0 model, in its mass-free formulation, provides a window into this deeper reality.
	
\section*{Mass was always an illusion—energy and geometry are the fundamental reality.}
	
\section*{Introduction}
\addcontentsline{toc}{chapter}{Introduction}

This book presents the current state of the T0 time--mass duality framework and its applications to
particle masses, fundamental constants, quantum mechanics, gravitation, and cosmology.

The main body of the book consists of a set of core T0 documents. These chapters reflect the
present understanding of the theory and its quantitative consequences. Wherever possible, the
material has been reorganized and unified so that the structure of the theory becomes as transparent
as possible.

At the end of the book, several older documents are included in an appendix. These texts represent
earlier stages of the development of the T0 framework. They were not removed, because they make
the evolution of the ideas and the refinement of the formulas visible. In many cases, one can see
how approximations were improved, how special cases were generalized, and how new empirical data
helped to sharpen or correct earlier arguments.

The “live” version of the theory is maintained in a public GitHub repository:

\begin{center}
	\url{https://github.com/jpascher/T0-Time-Mass-Duality}
\end{center}

The LaTeX sources of the chapters in this book are taken from that repository. If conceptual or
numerical errors are found, they are corrected there first. This means that the PDF version of the
book you are reading is a snapshot of a continuously evolving project. For the most recent version
of the documents, including new appendices or corrections, the GitHub repository should always be
considered the primary reference.

The intention of this compilation is twofold:
\begin{itemize}
	\item to provide a coherent, readable path through the core ideas and results of the T0 framework;
	\item to document, in the appendix, the historical development of these ideas, including false
	starts, intermediate formulations, and early fits to experimental data.
\end{itemize}

Readers who are mainly interested in the current formulation of the theory may focus on the core
chapters. Readers who are also interested in the reasoning and trial--and--error process behind
the theory are invited to study the appendix material in parallel.	


% Bibliography
\begin{thebibliography}{99}
	
	\bibitem{pdg2024}
	Particle Data Group Collaboration (2024). 
	\textit{Review of Particle Physics}. 
	Progress of Theoretical and Experimental Physics, 2024(8), 083C01.
	\url{https://pdg.lbl.gov}
	
	\bibitem{flag2024}
	Aoki, Y., et al. (FLAG Collaboration) (2024). 
	\textit{FLAG Review 2024 of Lattice Results for Low-Energy Constants}. 
	arXiv:2411.04268.
	\url{https://arxiv.org/abs/2411.04268}
	
	\bibitem{fermilab_muon_g2}
	Abi, B., et al. (Muon g-2 Collaboration) (2021). 
	\textit{Measurement of the Positive Muon Anomalous Magnetic Moment to 0.46 ppm}. 
	Physical Review Letters, 126, 141801.
	
	\bibitem{peskin_schroeder}
	Peskin, M. E., \& Schroeder, D. V. (1995). 
	\textit{An Introduction to Quantum Field Theory}. 
	Addison-Wesley.
	
	\bibitem{weinberg_qft}
	Weinberg, S. (1995). 
	\textit{The Quantum Theory of Fields, Vol. I--III}. 
	Cambridge University Press.
	
	\bibitem{griffiths_particle}
	Griffiths, D. (2008). 
	\textit{Introduction to Elementary Particles}. 
	Wiley-VCH.
	
	\bibitem{mandl_shaw}
	Mandl, F., \& Shaw, G. (2010). 
	\textit{Quantum Field Theory (2nd ed.)}. 
	Wiley.
	
	\bibitem{srednicki_qft}
	Srednicki, M. (2007). 
	\textit{Quantum Field Theory}. 
	Cambridge University Press.
	
	\bibitem{t0_fundamentals}
	Pascher, J. (2024). 
	\textit{T0-Theory: Foundations of Time-Mass Duality}. 
	Unpublished manuscript, HTL Leonding.
	
	\bibitem{t0_fine_structure}
	Pascher, J. (2024). 
	\textit{T0-Theory: The Fine Structure Constant}. 
	Unpublished manuscript, HTL Leonding.
	
	\bibitem{t0_neutrinos}
	Pascher, J. (2024). 
	\textit{T0-Theory: Neutrino Masses and PMNS Mixing}. 
	Unpublished manuscript, HTL Leonding.
	
	\bibitem{t0_github}
	Pascher, J. (2024--2025). 
	\textit{T0-Time-Mass-Duality Repository}. 
	GitHub.
	\url{https://github.com/jpascher/T0-Time-Mass-Duality}
	
	\bibitem{lattice_qcd_review}
	Kronfeld, A. S. (2012). 
	\textit{Twenty-first Century Lattice Gauge Theory: Results from the QCD Lagrangian}. 
	Annual Review of Nuclear and Particle Science, 62, 265--284.
	
	\bibitem{neutrino_mixing_pdg}
	Particle Data Group Collaboration (2024). 
	\textit{Neutrino Masses, Mixing, and Oscillations}. 
	PDG Review 2024.
	\url{https://pdg.lbl.gov/2024/reviews/rpp2024-rev-neutrino-mixing.pdf}
	
	\bibitem{higgs_discovery}
	ATLAS and CMS Collaborations (2012). 
	\textit{Observation of a New Particle in the Search for the Standard Model Higgs Boson}. 
	Physics Letters B, 716, 1--29.
	
	\bibitem{Brannen2005}
	C. P. Brannen, ``Estimate of neutrino masses from Koide's relation'', \textit{arXiv:hep-ph/0505028} (2005).
	\url{https://arxiv.org/abs/hep-ph/0505028}
	
	\bibitem{Brannen2006}
	C. P. Brannen, ``Koide Mass Formula for Neutrinos'', \textit{arXiv:0702.0052} (2006).
	\url{http://brannenworks.com/MASSES.pdf}
	
	\bibitem{PhaseVectors2025}
	Anonymous, ``The Koide Relation and Lepton Mass Hierarchy from Phase Vectors'', \textit{rXiv:2507.0040} (2025).
	\url{https://rxiv.org/pdf/2507.0040v1.pdf}
	
	\bibitem{PDG2025}
	Particle Data Group, ``Review of Particle Physics'', \textit{Phys. Rev. D} \textbf{112} (2025) 030001.
	\url{https://pdg.lbl.gov/2025/}
	
	\bibitem{terrell2024}
	Terrell et al. (2024). 
	\textit{Single-Clock Metrology in Nature}. 
	Nature Physics.
	
	\bibitem{hossenfelder2024}
	Hossenfelder, S. (2024). 
	\textit{Single Clock Video Explanation}. 
	YouTube.
	
	\bibitem{hundert1931}
	Hundert (1931). 
	\textit{Reference Work}. 
	Publisher.
	
	\bibitem{terrell2025}
	Terrell et al. (2025). 
	\textit{Advanced Clock Synchronization Methods}. 
	Physical Review Letters.
	
	\bibitem{pascher_t0_2025}
	Pascher, J. (2025). 
	\textit{T0-Theory: Complete Framework and Applications}. 
	Unpublished manuscript, HTL Leonding.
	
	\bibitem{t0qm}
	Pascher, J. (2024). 
	\textit{T0-Theory: Quantum Mechanics Formulation}. 
	Unpublished manuscript, HTL Leonding.
	
	\bibitem{t0anomale}
	Pascher, J. (2024). 
	\textit{T0-Theory: Anomalous Magnetic Moments}. 
	Unpublished manuscript, HTL Leonding.
	
	\bibitem{muong2complete}
	Abi, B., et al. (Muon g-2 Collaboration) (2023). 
	\textit{Complete Measurement of the Positive Muon Anomalous Magnetic Moment}. 
	Physical Review Letters, 131, 161802.
	
	\bibitem{penrose2004}
	Penrose, R. (2004). 
	\textit{The Road to Reality: A Complete Guide to the Laws of the Universe}. 
	Jonathan Cape.
	
	\bibitem{planck1900}
	Planck, M. (1900). 
	\textit{On the Theory of the Energy Distribution Law of the Normal Spectrum}. 
	Verhandlungen der Deutschen Physikalischen Gesellschaft, 2, 237.
	
	\bibitem{T0Theory}
	Pascher, J. (2024). 
	\textit{T0-Theory: Fundamental Principles}. 
	Unpublished manuscript, HTL Leonding.
	
	% Additional bibliography entries for all undefined citations
	\bibitem{6g_roadmap}
	6G Research Consortium (2024).
	\textit{6G Technology Roadmap}.
	Technical Report.
	
	\bibitem{Born2013}
	Born, M. (2013).
	\textit{Einstein's Theory of Relativity}.
	Dover Publications.
	
	\bibitem{Casimir1948}
	Casimir, H. B. G. (1948).
	\textit{On the attraction between two perfectly conducting plates}.
	Proc. Kon. Ned. Akad. Wetensch. B51, 793--795.
	
	\bibitem{Einstein1905}
	Einstein, A. (1905).
	\textit{On the Electrodynamics of Moving Bodies}.
	Annalen der Physik, 17, 891--921.
	
	\bibitem{Feynman2006}
	Feynman, R. P. (2006).
	\textit{QED: The Strange Theory of Light and Matter}.
	Princeton University Press.
	
	\bibitem{Griffiths2017}
	Griffiths, D. J. (2017).
	\textit{Introduction to Electrodynamics (4th ed.)}.
	Cambridge University Press.
	
	\bibitem{Jackson1999}
	Jackson, J. D. (1999).
	\textit{Classical Electrodynamics (3rd ed.)}.
	Wiley.
	
	\bibitem{Mohr2016}
	Mohr, P. J., et al. (2016).
	\textit{CODATA Recommended Values of the Fundamental Physical Constants: 2014}.
	Rev. Mod. Phys. 88, 035009.
	
	\bibitem{Parker2018}
	Parker, R. H., et al. (2018).
	\textit{Measurement of the fine-structure constant as a test of the Standard Model}.
	Science, 360, 191--195.
	
	\bibitem{Planck1900}
	Planck, M. (1900).
	\textit{On the Theory of the Energy Distribution Law of the Normal Spectrum}.
	Verhandlungen der Deutschen Physikalischen Gesellschaft, 2, 237.
	
	\bibitem{Planck2018}
	Planck Collaboration (2018).
	\textit{Planck 2018 results. VI. Cosmological parameters}.
	Astronomy \& Astrophysics, 641, A6.
	
	\bibitem{QFT_T0}
	Pascher, J. (2024).
	\textit{T0-Theory and QFT Connections}.
	Unpublished manuscript, HTL Leonding.
	
	\bibitem{Sommerfeld1916}
	Sommerfeld, A. (1916).
	\textit{On the Quantum Theory of Spectral Lines}.
	Annalen der Physik, 51, 1--94.
	
	\bibitem{T0_Feinstruktur}
	Pascher, J. (2024).
	\textit{T0-Theory: Fine Structure Analysis}.
	Unpublished manuscript, HTL Leonding.
	
	\bibitem{T0_SI}
	Pascher, J. (2024).
	\textit{T0-Theory and SI Units}.
	Unpublished manuscript, HTL Leonding.
	
	\bibitem{T0_fine_structure}
	Pascher, J. (2024).
	\textit{T0-Theory: The Fine Structure Constant}.
	Unpublished manuscript, HTL Leonding.
	
	\bibitem{T0_g2_erweiterung}
	Pascher, J. (2024).
	\textit{T0-Theory: g-2 Extensions}.
	Unpublished manuscript, HTL Leonding.
	
	\bibitem{T0_gravitational_constant}
	Pascher, J. (2024).
	\textit{T0-Theory: Gravitational Constant Derivation}.
	Unpublished manuscript, HTL Leonding.
	
	\bibitem{T0_netze_en}
	Pascher, J. (2024).
	\textit{T0-Theory: Network Structures}.
	Unpublished manuscript, HTL Leonding.
	
	\bibitem{T0_tm_erweiterung}
	Pascher, J. (2024).
	\textit{T0-Theory: Time-Mass Extensions}.
	Unpublished manuscript, HTL Leonding.
	
	\bibitem{Uzan2003}
	Uzan, J.-P. (2003).
	\textit{The fundamental constants and their variation}.
	Rev. Mod. Phys. 75, 403--455.
	
	\bibitem{Weinberg1995}
	Weinberg, S. (1995).
	\textit{The Quantum Theory of Fields, Vol. I}.
	Cambridge University Press.
	
	\bibitem{albrecht1999}
	Albrecht, A. \& Magueijo, J. (1999).
	\textit{A time varying speed of light as a solution to cosmological puzzles}.
	Phys. Rev. D 59, 043516.
	
	\bibitem{alice2023}
	ALICE Collaboration (2023).
	\textit{Recent results from ALICE}.
	CERN-EP-2023-XXX.
	
	\bibitem{analog_optical}
	Smith, J. et al. (2024).
	\textit{Analog optical computing systems}.
	Nature Photonics.
	
	\bibitem{ashtekar2004}
	Ashtekar, A. \& Lewandowski, J. (2004).
	\textit{Background independent quantum gravity}.
	Class. Quantum Grav. 21, R53.
	
	\bibitem{atlas2023}
	ATLAS Collaboration (2023).
	\textit{ATLAS physics results}.
	CERN-PH-EP-2023-XXX.
	
	\bibitem{atlas2023higgs}
	ATLAS Collaboration (2023).
	\textit{Higgs boson measurements}.
	Phys. Rev. Lett.
	
	\bibitem{barbour1999}
	Barbour, J. (1999).
	\textit{The End of Time}.
	Oxford University Press.
	
	\bibitem{barrow1999}
	Barrow, J. D. (1999).
	\textit{Cosmologies with varying light speed}.
	Phys. Rev. D 59, 043515.
	
	\bibitem{becker2007}
	Becker, K. et al. (2007).
	\textit{String Theory and M-Theory}.
	Cambridge University Press.
	
	\bibitem{bell_muon}
	Bennett, G. W., et al. (Muon g-2 Collaboration) (2006).
	\textit{Final report of the E821 muon anomalous magnetic moment measurement}.
	Phys. Rev. D 73, 072003.
	
	\bibitem{bondi1948}
	Bondi, H. \& Gold, T. (1948).
	\textit{The steady-state theory of the expanding universe}.
	Mon. Not. R. Astron. Soc. 108, 252--270.
	
	\bibitem{brewer2019}
	Brewer, S. M. et al. (2019).
	\textit{Al+ Quantum-Logic Clock with Systematic Uncertainty below $10^{-18}$}.
	Phys. Rev. Lett. 123, 033201.
	
	\bibitem{cms2023top}
	CMS Collaboration (2023).
	\textit{Top quark measurements at CMS}.
	JHEP 2023.
	
	\bibitem{cms2024}
	CMS Collaboration (2024).
	\textit{CMS physics results 2024}.
	CERN-PH-EP-2024-XXX.
	
	\bibitem{codata2019}
	Tiesinga, E. et al. (2019).
	\textit{The 2018 CODATA Recommended Values}.
	J. Phys. Chem. Ref. Data.
	
	\bibitem{desi2025}
	DESI Collaboration (2025).
	\textit{DESI 2025 Cosmology Results}.
	arXiv preprint.
	
	\bibitem{differential_optical}
	Wang, X. et al. (2024).
	\textit{Differential optical computing}.
	Optica.
	
	\bibitem{dingle1972}
	Dingle, H. (1972).
	\textit{Science at the Crossroads}.
	Martin Brian \& O'Keeffe.
	
	\bibitem{divalentino2021}
	Di Valentino, E. et al. (2021).
	\textit{In the realm of the Hubble tension}.
	Class. Quantum Grav. 38, 153001.
	
	\bibitem{elnaschie2004}
	El Naschie, M. S. (2004).
	\textit{A review of E infinity theory}.
	Chaos, Solitons \& Fractals, 19, 209--236.
	
	\bibitem{fabrication_heterogeneous}
	Chen, Y. et al. (2024).
	\textit{Heterogeneous photonic integration}.
	Nature Electronics.
	
	\bibitem{fermilab2023}
	Fermilab (2023).
	\textit{Muon g-2 results}.
	Phys. Rev. Lett.
	
	\bibitem{flexible_wafer}
	Kim, S. et al. (2024).
	\textit{Flexible wafer-scale photonics}.
	Science Advances.
	
	\bibitem{francesco1997}
	Di Francesco, P. et al. (1997).
	\textit{Conformal Field Theory}.
	Springer.
	
	\bibitem{hartree1957}
	Hartree, D. R. (1957).
	\textit{The Calculation of Atomic Structures}.
	Wiley.
	
	\bibitem{hhi_6g}
	Fraunhofer HHI (2024).
	\textit{6G Photonic Integration}.
	Technical Report.
	
	\bibitem{hossenfelder2025}
	Hossenfelder, S. (2025).
	\textit{Science without the gobbledygook}.
	YouTube/Blog.
	
	\bibitem{hossenfelder_single_clock_video}
	Hossenfelder, S. (2024).
	\textit{The Single Clock Problem}.
	YouTube.
	
	\bibitem{hoyle1948}
	Hoyle, F. (1948).
	\textit{A new model for the expanding universe}.
	Mon. Not. R. Astron. Soc. 108, 372--382.
	
	\bibitem{integration_microelectronic}
	Liu, A. et al. (2024).
	\textit{Microelectronic photonic integration}.
	IEEE Journal.
	
	\bibitem{jacobson1995}
	Jacobson, T. (1995).
	\textit{Thermodynamics of spacetime}.
	Phys. Rev. Lett. 75, 1260.
	
	\bibitem{kasevich2023}
	Kasevich, M. et al. (2023).
	\textit{Atom interferometry tests}.
	Nature Physics.
	
	\bibitem{lerner2014}
	Lerner, E. J. (2014).
	\textit{An open letter on cosmology}.
	New Scientist.
	
	\bibitem{lisa2017}
	LISA Consortium (2017).
	\textit{Laser Interferometer Space Antenna}.
	ESA Technical Report.
	
	\bibitem{lithium_tantalate}
	Zhang, M. et al. (2024).
	\textit{Thin-film lithium tantalate photonics}.
	Nature Photonics.
	
	\bibitem{lopez2010}
	Lopez-Corredoira, M. (2010).
	\textit{Tests and problems of the standard model in cosmology}.
	Int. J. Mod. Phys. D.
	
	\bibitem{ludlow2015}
	Ludlow, A. D. et al. (2015).
	\textit{Optical atomic clocks}.
	Rev. Mod. Phys. 87, 637.
	
	\bibitem{mach1883}
	Mach, E. (1883).
	\textit{Die Mechanik in ihrer Entwickelung}.
	F.A. Brockhaus.
	
	\bibitem{maldacena1998}
	Maldacena, J. (1998).
	\textit{The large N limit of superconformal field theories}.
	Adv. Theor. Math. Phys. 2, 231--252.
	
	\bibitem{mueller2014}
	Müller, H. et al. (2014).
	\textit{Atom interferometry tests of the gravitational redshift}.
	Phys. Rev. Lett.
	
	\bibitem{mug2_final_2025}
	Muon g-2 Collaboration (2025).
	\textit{Final muon g-2 measurement}.
	Phys. Rev. Lett.
	
	\bibitem{muong2_2023}
	Muon g-2 Collaboration (2023).
	\textit{Updated muon g-2 results}.
	Phys. Rev. Lett.
	
	\bibitem{nathan2024}
	Nathan, A. et al. (2024).
	\textit{Quantum computing advances}.
	Nature.
	
	\bibitem{newell2018}
	Newell, D. B. et al. (2018).
	\textit{The CODATA 2017 values of h, e, k, and $N_A$}.
	Metrologia 55, L13.
	
	\bibitem{nottale1993}
	Nottale, L. (1993).
	\textit{Fractal Space-Time and Microphysics}.
	World Scientific.
	
	\bibitem{on_chip_lithium}
	Wang, C. et al. (2024).
	\textit{On-chip lithium niobate photonics}.
	Nature Communications.
	
	\bibitem{optical_advantages}
	Shastri, B. J. et al. (2024).
	\textit{Advantages of optical computing}.
	Nature Reviews Physics.
	
	\bibitem{pascher2025cmb}
	Pascher, J. (2025).
	\textit{T0-Theory: CMB Analysis}.
	Unpublished manuscript, HTL Leonding.
	
	\bibitem{pascher2025g2}
	Pascher, J. (2025).
	\textit{T0-Theory: g-2 Predictions}.
	Unpublished manuscript, HTL Leonding.
	
	\bibitem{pascher2025qm}
	Pascher, J. (2025).
	\textit{T0-Theory: Quantum Mechanics}.
	Unpublished manuscript, HTL Leonding.
	
	\bibitem{pascher2025si}
	Pascher, J. (2025).
	\textit{T0-Theory: SI Unit System}.
	Unpublished manuscript, HTL Leonding.
	
	\bibitem{pascher2025t0}
	Pascher, J. (2025).
	\textit{T0-Theory: Complete Framework}.
	Unpublished manuscript, HTL Leonding.
	
	\bibitem{pascher:fundamentals}
	Pascher, J. (2024).
	\textit{T0-Theory: Fundamentals}.
	Unpublished manuscript, HTL Leonding.
	
	\bibitem{pascher:g2_rev9}
	Pascher, J. (2024).
	\textit{T0-Theory: g-2 Revision 9}.
	Unpublished manuscript, HTL Leonding.
	
	\bibitem{pascher:geometric_formalism}
	Pascher, J. (2024).
	\textit{T0-Theory: Geometric Formalism}.
	Unpublished manuscript, HTL Leonding.
	
	\bibitem{pascher:ml_addendum}
	Pascher, J. (2024).
	\textit{T0-Theory: Machine Learning Addendum}.
	Unpublished manuscript, HTL Leonding.
	
	\bibitem{pascher:t0_foundations}
	Pascher, J. (2024).
	\textit{T0-Theory: Foundations}.
	Unpublished manuscript, HTL Leonding.
	
	\bibitem{pascher_derivation_beta_2025}
	Pascher, J. (2025).
	\textit{T0-Theory: Derivation of Beta}.
	Unpublished manuscript, HTL Leonding.
	
	\bibitem{pascher_higgs_connection_2025}
	Pascher, J. (2025).
	\textit{T0-Theory: Higgs Connection}.
	Unpublished manuscript, HTL Leonding.
	
	\bibitem{pascher_lagrangian_extended_2025}
	Pascher, J. (2025).
	\textit{T0-Theory: Extended Lagrangian}.
	Unpublished manuscript, HTL Leonding.
	
	\bibitem{pascher_mathematical_structure_2025}
	Pascher, J. (2025).
	\textit{T0-Theory: Mathematical Structure}.
	Unpublished manuscript, HTL Leonding.
	
	\bibitem{pascher_t0_cmb_2025}
	Pascher, J. (2025).
	\textit{T0-Theory: CMB Predictions}.
	Unpublished manuscript, HTL Leonding.
	
	\bibitem{pascher_t0_energie_2025}
	Pascher, J. (2025).
	\textit{T0-Theory: Energy}.
	Unpublished manuscript, HTL Leonding.
	
	\bibitem{pascher_t0_energy_2025}
	Pascher, J. (2025).
	\textit{T0-Theory: Energy Framework}.
	Unpublished manuscript, HTL Leonding.
	
	\bibitem{pascher_t0_theory_2025}
	Pascher, J. (2025).
	\textit{T0-Theory: Complete Theory}.
	Unpublished manuscript, HTL Leonding.
	
	\bibitem{penrose1959}
	Penrose, R. (1959).
	\textit{The apparent shape of a relativistically moving sphere}.
	Proc. Cambridge Phil. Soc. 55, 137--139.
	
	\bibitem{penrose1967}
	Penrose, R. (1967).
	\textit{Twistor algebra}.
	J. Math. Phys. 8, 345--366.
	
	\bibitem{peratt1992}
	Peratt, A. L. (1992).
	\textit{Physics of the Plasma Universe}.
	Springer-Verlag.
	
	\bibitem{peskin1995}
	Peskin, M. E. \& Schroeder, D. V. (1995).
	\textit{An Introduction to Quantum Field Theory}.
	Addison-Wesley.
	
	\bibitem{peskin_schroeder_1995}
	Peskin, M. E. \& Schroeder, D. V. (1995).
	\textit{An Introduction to Quantum Field Theory}.
	Addison-Wesley.
	
	\bibitem{phoquant}
	PhoQuant (2024).
	\textit{Photonic quantum computing}.
	Technical Report.
	
	\bibitem{photonics_ai}
	Wetzstein, G. et al. (2024).
	\textit{Photonics for AI}.
	Nature.
	
	\bibitem{planck1906}
	Planck, M. (1906).
	\textit{The Theory of Heat Radiation}.
	Johann Ambrosius Barth.
	
	\bibitem{planck2018}
	Planck Collaboration (2018).
	\textit{Planck 2018 results}.
	A\&A 641, A6.
	
	\bibitem{polchinski1998}
	Polchinski, J. (1998).
	\textit{String Theory}.
	Cambridge University Press.
	
	\bibitem{qant_nps}
	QANT (2024).
	\textit{Quantum photonics systems}.
	Technical Report.
	
	\bibitem{quantenjahr25}
	Quantenjahr (2025).
	\textit{International Year of Quantum}.
	UNESCO.
	
	\bibitem{recurrent_photonics}
	Tait, A. N. et al. (2024).
	\textit{Recurrent photonic neural networks}.
	Optica.
	
	\bibitem{rf_photonics}
	Capmany, J. \& Novak, D. (2024).
	\textit{Microwave photonics}.
	Nature Photonics.
	
	\bibitem{riess2019}
	Riess, A. G. et al. (2019).
	\textit{Large Magellanic Cloud Cepheid Standards}.
	ApJ 876, 85.
	
	\bibitem{riess2022}
	Riess, A. G. et al. (2022).
	\textit{A Comprehensive Measurement of H0}.
	ApJ 934, L7.
	
	\bibitem{rovelli2004}
	Rovelli, C. (2004).
	\textit{Quantum Gravity}.
	Cambridge University Press.
	
	\bibitem{sciama1953}
	Sciama, D. W. (1953).
	\textit{On the origin of inertia}.
	Mon. Not. R. Astron. Soc. 113, 34--42.
	
	\bibitem{sciencedaily2025}
	ScienceDaily (2025).
	\textit{Physics news}.
	Online.
	
	\bibitem{sm_g2_2025}
	Aoyama, T. et al. (2025).
	\textit{Standard Model prediction for g-2}.
	Phys. Rep.
	
	\bibitem{susskind1995}
	Susskind, L. (1995).
	\textit{The world as a hologram}.
	J. Math. Phys. 36, 6377--6396.
	
	\bibitem{t0_kosmologie}
	Pascher, J. (2024).
	\textit{T0-Theory: Cosmology}.
	Unpublished manuscript, HTL Leonding.
	
	\bibitem{terrell1959}
	Terrell, J. (1959).
	\textit{Invisibility of the Lorentz contraction}.
	Phys. Rev. 116, 1041--1045.
	
	\bibitem{terrell_single_clock_nature_2024}
	Terrell, J. et al. (2024).
	\textit{Single clock precision measurements}.
	Nature Physics.
	
	\bibitem{tfln_foundry}
	TFLN Foundry (2024).
	\textit{Thin-film lithium niobate foundry services}.
	Technical Specifications.
	
	\bibitem{thiemann2007}
	Thiemann, T. (2007).
	\textit{Modern Canonical Quantum General Relativity}.
	Cambridge University Press.
	
	\bibitem{thz_epfl}
	EPFL (2024).
	\textit{Terahertz photonics research}.
	Technical Report.
	
	\bibitem{unnikrishnan2004}
	Unnikrishnan, C. S. (2004).
	\textit{On Einstein's resolution of the twin clock paradox}.
	Current Science, 86, 704--709.
	
	\bibitem{verlinde2011}
	Verlinde, E. (2011).
	\textit{On the origin of gravity and the laws of Newton}.
	JHEP 2011, 29.
	
	\bibitem{video2025}
	Video (2025).
	\textit{Physics video explanation}.
	YouTube.
	
	\bibitem{weinberg1995}
	Weinberg, S. (1995).
	\textit{The Quantum Theory of Fields}.
	Cambridge University Press.
	
	\bibitem{weiskopf2000}
	Weiskopf, D. (2000).
	\textit{Visualization of special relativity}.
	PhD thesis, University of Tübingen.
	
	\bibitem{wheeler1990}
	Wheeler, J. A. (1990).
	\textit{A Journey into Gravity and Spacetime}.
	Scientific American Library.
	
	\bibitem{wiki_bell}
	Wikipedia (2024).
	\textit{Bell's theorem}.
	Online encyclopedia.
	
	\bibitem{zwicky1929}
	Zwicky, F. (1929).
	\textit{On the red shift of spectral lines through interstellar space}.
	Proc. Natl. Acad. Sci. 15, 773--779.

\end{thebibliography}


\end{document}
