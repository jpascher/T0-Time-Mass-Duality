\documentclass[11pt,a4paper]{article}
\usepackage[a4paper,margin=2cm]{geometry}
\usepackage[utf8]{inputenc}
\usepackage[english]{babel}
\usepackage{lmodern}
\renewcommand{\familydefault}{\sfdefault}

\usepackage{amsmath,amssymb,amsthm}
\usepackage{graphicx}
\usepackage[unicode,pdfencoding=auto,hypertexnames=false]{hyperref}
\usepackage{booktabs}
\usepackage{longtable}
\usepackage{array}
\usepackage{siunitx}
\usepackage{fancyhdr}
\usepackage{float}
\usepackage{tikz}
% tcolorbox removed for standalone
% tcbset removed
\tikzset{
  t0blue/.style={draw=blue,fill=blue!10},
  t0red/.style={draw=red,fill=red!10},
  t0green/.style={draw=green!50!black,fill=green!10},
  t0orange/.style={draw=orange,fill=orange!10},
}
\usepackage{setspace}
\usepackage{enumitem}
\usepackage{adjustbox}
\usepackage{xcolor}

% Define colors for xcolor package
\definecolor{t0green}{RGB}{34,139,34}
\definecolor{t0blue}{RGB}{0,0,255}
\definecolor{t0red}{RGB}{255,0,0}
\definecolor{t0orange}{RGB}{255,165,0}

% Define custom column types for tables
\newcolumntype{L}[1]{>{\raggedright\arraybackslash}p{#1}}
\newcolumntype{C}[1]{>{\centering\arraybackslash}p{#1}}
\newcolumntype{R}[1]{>{\raggedleft\arraybackslash}p{#1}}

\setlength{\parindent}{0pt}
\setlength{\parskip}{6pt}

\hypersetup{
  colorlinks=true,
  linkcolor=blue,
  citecolor=blue,
  urlcolor=blue
}
\pagestyle{fancy}
\setlength{\headheight}{28pt}

\newcommand{\checkmarkx}{\checkmark}
\newcommand{\warningx}{\textbf{!}}

% Makros aus Einzel-Dokumenten (Fallback-Definitionen)
\newcommand{\mytimes}{\times}
\newcommand{\myapprox}{\approx}
\newcommand{\mysim}{\sim}
\newcommand{\myomega}{\omega}
\newcommand{\mypi}{\pi}
\newcommand{\myrightarrow}{\rightarrow}
\newcommand{\mypropto}{\propto}
\newcommand{\deltafield}{\delta\phi}
\newcommand{\xipar}{\xi}
\newcommand{\xiT}{\xi}
\newcommand{\lambdah}{\lambda_h}

% Additional macros used in chapter files
\newcommand{\Kfrak}{K_{\text{frak}}}  % Fractal correction factor
\newcommand{\Dfrak}{D_f}              % Fractal dimension
\newcommand{\betapar}{\beta}          % T0 beta parameter
\newcommand{\alphapar}{\alpha}        % T0 alpha parameter
\newcommand{\Efield}{E}               % Energy field
% Note: checkmarkxa/warningxa are variants used in auto-generated chapter files
\newcommand{\checkmarkxa}{\checkmark}
\newcommand{\warningxa}{\textbf{!}}

% Additional T0-specific macros
\newcommand{\xigeom}{\xi_{\text{geom}}}  % Geometric xi
\newcommand{\lP}{\ell_P}                  % Planck length
\newcommand{\rzero}{r_0}                  % Characteristic radius
\newcommand{\xirat}{\xi_{\text{rat}}}     % Xi ratio
\newcommand{\tzero}{t_0}                  % Characteristic time
\newcommand{\natunits}{\text{(nat. units)}}  % Natural units annotation
\newcommand{\myRightarrow}{\Rightarrow}   % Arrow variant
\newcommand{\Lag}{\mathcal{L}}            % Lagrangian

% Physics macros used in chapter files
\newcommand{\CQCD}{C_{\text{QCD}}}        % QCD correction
\newcommand{\EP}{E_P}                     % Planck energy
\newcommand{\Ee}{E_e}                     % Electron energy
\newcommand{\Emu}{E_\mu}                  % Muon energy
\newcommand{\Exi}{E_\xi}                  % Xi energy
\newcommand{\Ezero}{E_0}                  % Characteristic energy
\newcommand{\Hubble}{H}                   % Hubble constant
\newcommand{\Kspec}{K_{\text{spec}}}      % Spectral correction
\newcommand{\Lambdat}{\Lambda_t}          % Time-related cosmological constant
\newcommand{\Leff}{\mathcal{L}_{\text{eff}}}  % Effective Lagrangian
\newcommand{\Lorentz}{\mathcal{L}}        % Lorentz symbol
\newcommand{\Lxi}{L_\xi}                  % Xi length
\newcommand{\Tfield}{T}                   % Time field
\newcommand{\Weyl}{W}                     % Weyl tensor/symbol
\newcommand{\alphaEMSI}{\alpha_{\text{EM,SI}}}  % EM alpha in SI
\newcommand{\alphaEMnat}{\alpha_{\text{EM,nat}}}  % EM alpha in natural units
\newcommand{\alphaem}{\alpha_{\text{em}}} % Electromagnetic alpha
\newcommand{\betaTSI}{\beta_{T,\text{SI}}}  % Beta in SI
\newcommand{\betaTnat}{\beta_{T,\text{nat}}}  % Beta in natural units
\newcommand{\deltam}{\delta m}            % Mass difference
\newcommand{\phiT}{\phi_T}                % T-field phi
\newcommand{\tP}{t_P}                     % Planck time
\newcommand{\rhoCMB}{\rho_{\text{CMB}}}   % CMB density
\newcommand{\rhoCasimir}{\rho_{\text{Casimir}}}  % Casimir density

% Table formatting
\usepackage{multirow}

% Additional physics macros
\newcommand{\Riem}{\mathcal{R}}           % Riemann tensor
\newcommand{\ZPinch}{Z_{\text{pinch}}}    % Z-pinch
\newcommand{\SynchPower}{P_{\text{synch}}} % Synchrotron power
\newcommand{\Rzero}{R_0}                  % Characteristic radius
\newcommand{\alphafine}{\alpha}           % Fine structure constant
\newcommand{\Etau}{E_\tau}                % Tau energy
\newcommand{\deltaE}{\delta E}            % Energy deviation
\newcommand{\EPlanck}{E_P}                % Planck energy
\newcommand{\pichar}{\pi}                 % Pi character
\newcommand{\alphaWSI}{\alpha_{W,\text{SI}}}  % Wien alpha in SI
\newcommand{\alphaWnat}{\alpha_{W,\text{nat}}}  % Wien alpha in natural units

% Einfache abstract-Umgebung für Kapitel:
\newenvironment{abstract}{%
  \begin{center}\bfseries Abstract\end{center}\small
}{\par}


\title{T0 Modell Uebersicht En}
\author{J. Pascher}
\date{\today}

\begin{document}
\maketitle

\section*{T0 Modell Uebersicht (T0 Modell Uebersicht)}

	\begin{abstract}
		Based on the analysis of available PDF documents from the GitHub repository \texttt{jpascher/T0-Time-Mass-Duality}, a comprehensive summary has been created. The documents are available in both German (\texttt{.De.pdf}) and English (\texttt{.En.pdf}) versions. The T0-Model pursues the ambitious goal of reducing all physics from over 20 free parameters of the Standard Model to a single geometric constant $\xipar = \frac{4}{3} \times 10^{-4}$. This treatise presents a complete exposition of theoretical foundations, mathematical structures, and experimental predictions.
	\end{abstract}
	
	\tableofcontents
	\newpage
\section{The T0-Model: A New Perspective for Communications Engineers}

\subsection{The Parameter Problem of Modern Physics}

You know from communications engineering the problem of parameter optimization. In designing a filter, you need to set many coefficients; in an amplifier, you choose different operating points. The more parameters, the more complex the system becomes and the more susceptible to instabilities.

Modern physics has exactly this problem: The Standard Model of particle physics requires over 20 free parameters - masses, coupling constants, mixing angles. These must all be determined experimentally without us understanding why they have precisely these values. It's like having to tune a 20-stage amplifier without understanding the circuit.

The T0-Model proposes a radical simplification: All physics can be reduced to a single dimensionless parameter: $\xi = \frac{4}{3} \times 10^{-4}$.

\subsection{The Universal Constant}

From signal processing, you know that certain ratios always recur. The golden ratio in image processing, the Nyquist frequency in sampling, characteristic impedances in transmission lines. The $\xi$-constant plays a similar universal role.

The value $\xi = \frac{4}{3} \times 10^{-4}$ arises from the geometry of three-dimensional space. The factor $\frac{4}{3}$ you know from the sphere volume $V = \frac{4\pi}{3}r^3$ - it characterizes optimal 3D packing densities. The factor $10^{-4}$ arises from quantum field theory loop suppression factors, similar to damping factors in your control loops.

\subsection{Energy Fields as Foundation}

In communications engineering, you constantly work with fields: electromagnetic fields in antennas, evanescent fields in waveguides, near-fields in capacitive sensors. The T0-Model extends this concept: The entire universe consists of a single universal energy field $E(x,t)$.

This field obeys the d'Alembert equation:
$$\square E = \left(\nabla^2 - \frac{1}{c^2}\frac{\partial^2}{\partial t^2}\right) E = 0$$

This is familiar from electromagnetism - it's the wave equation for electromagnetic fields in vacuum. The difference: In the T0-Model, this one equation describes not only light, but all physical phenomena.

\subsection{Time-Energy Duality and Modulation}

From communications engineering, you know time-frequency dualities. A narrow function in time becomes broad in the frequency domain, and vice versa. The T0-Model introduces a similar duality between time and energy:

$$T(x,t) \cdot E(x,t) = 1$$

This is analogous to the uncertainty relation $\Delta t \cdot \Delta f \geq \frac{1}{4\pi}$ that you use in signal analysis. Where energy is locally concentrated, time passes more slowly - like an energy-dependent clock frequency.

\subsection{Deterministic Quantum Mechanics}

Standard quantum mechanics uses probabilistic descriptions because it has only incomplete information. This is like noise analysis in your systems: When you don't know the exact noise source, you use statistical models.

The T0-Model claims that quantum mechanics is actually deterministic. The apparent randomness arises from very fast changes in the energy field - so fast that they lie below the temporal resolution of our measuring devices. It's like aliasing in signal processing: Changes that are too fast appear as seemingly random artifacts.

The famous Schrödinger equation is extended:
$$i\hbar\frac{\partial\psi}{\partial t} + i\psi\left[\frac{\partial T}{\partial t} + \vec{v} \cdot \nabla T\right] = \hat{H}\psi$$

The additional term $\frac{\partial T}{\partial t} + \vec{v} \cdot \nabla T$ describes coupling to the time field - similar to Doppler terms in moving reference frames.

\subsection{Field Geometries and System Theory}

The T0-Model distinguishes three characteristic field geometries:

\begin{enumerate}
	\item \textbf{Localized spherical fields}: Describe point-like particles. Parameters: $\xi = \frac{\ell_P}{r_0}$, $\beta = \frac{r_0}{r}$.
	\item \textbf{Localized non-spherical fields}: For complex systems with multipole expansion similar to your antenna theory.
	\item \textbf{Extended homogeneous fields}: Cosmological applications with modified $\xi_{\text{eff}} = \xi/2$ due to screening effects.
\end{enumerate}

This classification corresponds to system theory: lumped elements (R, L, C), distributed elements (transmission lines), and continuum systems (fields).

\subsection{Experimental Verification: Muon g-2}

The most convincing argument for the T0-Model comes from precision measurements. The anomalous magnetic moment of the muon shows a 4.2$\sigma$ deviation from the Standard Model - a clear sign of new physics.

The T0-Model makes a parameter-free prediction:
$$\Delta a_\ell = 251 \times 10^{-11} \times \left(\frac{m_\ell}{m_\mu}\right)^2$$

For the muon ($m_\ell = m_\mu$), this yields exactly the experimental value of $251 \times 10^{-11}$. For the electron, a testable prediction of $\Delta a_e = 5.87 \times 10^{-15}$ follows.

This is like a perfect impedance match in a broadband system - strong evidence that the theory correctly describes the underlying physics.

\subsection{Technological Implications}

New physical insights often lead to technological breakthroughs. Quantum mechanics enabled transistors and lasers, relativity theory enabled GPS and particle accelerators.

If the T0-Model is correct, completely new technologies could emerge:
\begin{itemize}
	\item Deterministic quantum computers without decoherence problems
	\item Energy field-based sensors with highest precision
	\item Possibly manipulation of local time rate through energy field control
	\item New materials based on controlled field geometries
\end{itemize}

\subsection{Mathematical Elegance}

What makes the T0-Model particularly attractive is its mathematical simplicity. Instead of complex Lagrangians with dozens of terms, a single universal Lagrangian density suffices:

$$\mathcal{L} = \frac{\xi}{E_P^2} \cdot (\partial E)^2$$

This is analogous to your simplest circuits: one resistor, one capacitor, but with universal validity. All the complexity of physics emerges as an emergent property of this one basic principle - like complex network behavior from simple Kirchhoff rules.

The elegance lies in the fact that a single geometric constant $\xi$ determines all observable phenomena, from subatomic particles to cosmological structures.	
	\section{Overview of Analyzed Documents}
	
	Based on the analysis of available PDF documents from the GitHub repository \texttt{jpascher/T0-Time-Mass-Duality}, a comprehensive summary has been created. The documents are available in both German (\texttt{.De.pdf}) and English (\texttt{.En.pdf}) versions.
	
	\subsection{Main Documents in GitHub Repository}
	
	\textbf{GitHub Path:} \url{https://github.com/jpascher/T0-Time-Mass-Duality/blob/main/2/pdf/}
	
	\begin{enumerate}
		\item \textbf{HdokumentDe.pdf} - Master document of complete T0-Framework
		\item \textbf{Zusammenfassung\_De.pdf} - Comprehensive theoretical treatise
		\item \textbf{T0-Energie\_De.pdf} - Energy-based formulation
		\item \textbf{cosmic\_De.pdf} - Cosmological applications
		\item \textbf{DerivationVonBetaDe.pdf} - Derivation of $\betapar$-parameter
		\item \textbf{xi\_parameter\_partikel\_De.pdf} - Mathematical analysis of $\xipar$-parameter
		\item \textbf{systemDe.pdf} - System-theoretical foundations
		\item \textbf{T0vsESM\_ConceptualAnalysis\_De.pdf} - Comparison with Standard Model
	\end{enumerate}
	
	\section{Foundations of the T0-Model}
	
	\subsection{The Central Vision}
	
	The T0-Model pursues the ambitious goal of reducing all physics from over 20 free parameters of the Standard Model to a single geometric constant:
	
	\begin{equation}
		\xipar = \frac{4}{3} \times 10^{-4} = 1.3333\ldots \times 10^{-4}
	\end{equation}
	
	\textbf{Document Reference:} \textit{HdokumentDe.pdf}, \textit{Zusammenfassung\_De.pdf}
	
	\subsection{The Universal Energy Field}
	
	The core of the T0-Model is a universal energy field $\Efield(x,t)$ described by a single fundamental equation:
	
	\begin{equation}
		\square \Efield = \left(\nabla^2 - \frac{\partial^2}{\partial t^2}\right) \Efield = 0
	\end{equation}
	
	This d'Alembert equation describes:
	\begin{itemize}
		\item All particles as localized energy field excitations
		\item All forces as energy field gradient interactions
		\item All dynamics through deterministic field evolution
	\end{itemize}
	
	\textbf{Document Reference:} \textit{T0-Energie\_De.pdf}, \textit{systemDe.pdf}
	
	\subsection{Time-Energy Duality}
	
	A fundamental insight of the T0-Model is the time-energy duality:
	
	\begin{equation}
		T_{\text{field}}(x,t) \cdot E_{\text{field}}(x,t) = 1
	\end{equation}
	
	This relationship leads to the T0-time scale:
	\begin{equation}
		t_0 = 2GE
	\end{equation}
	
	\textbf{Document Reference:} \textit{T0-Energie\_De.pdf}, \textit{HdokumentDe.pdf}
	
	\section{Mathematical Structure}
	
	\subsection{The $\xi$-Constant as Geometric Parameter}
	
	The dimensionless constant $\xipar = \frac{4}{3} \times 10^{-4}$ arises from:
	
	\begin{enumerate}
		\item Three-dimensional space geometry: Factor $\frac{4}{3}$
		\item Fractal dimension: Scale factor $10^{-4}$
	\end{enumerate}
	
	The geometric derivation:
	\begin{equation}
		\xipar = \frac{4\pi}{3} \cdot \frac{1}{4\pi \times 10^4} = \frac{4}{3} \times 10^{-4}
	\end{equation}
	
	\textbf{Document Reference:} \textit{xi\_parameter\_partikel\_De.pdf}, \textit{DerivationVonBetaDe.pdf}
	
	\subsection{Parameter-free Lagrangian}
	
	The complete T0-system requires no empirical inputs:
	
	\begin{equation}
		\mathcal{L} = \varepsilon \cdot (\partial \Efield)^2
	\end{equation}
	
	where:
	\begin{equation}
		\varepsilon = \frac{\xipar}{E_P^2} = \frac{4/3 \times 10^{-4}}{E_P^2}
	\end{equation}
	
	\textbf{Document Reference:} \textit{T0-Energie\_De.pdf}
	
	\subsection{Three Fundamental Field Geometries}
	
	The T0-Model distinguishes three field geometries:
	
	\begin{enumerate}
		\item Localized spherical energy fields (particles, atoms, nuclei, localized excitations)
		\item Localized non-spherical energy fields (molecular systems, crystal structures, anisotropic field configurations)
		\item Extended homogeneous energy fields (cosmological structures with screening effect)
	\end{enumerate}
	
\section*{Specific Parameters:}
	\begin{itemize}
		\item Spherical: $\xipar = \ell_P/r_0$, $\betapar = r_0/r$, Field equation: $\nabla^2 E = 4\pi G \rho_E E$
		\item Non-spherical: Tensorial parameters $\betapar_{ij}$, $\xipar_{ij}$, multipole expansion
		\item Extended homogeneous: $\xipar_{\text{eff}} = \xipar/2$ (natural screening effect), additional $\Lambda_T$ term
	\end{itemize}
	
	\textbf{Document Reference:} \textit{T0-Energie\_De.pdf}
	
	\section{Experimental Confirmation and Empirical Validation}
	
	\subsection{Already Confirmed Predictions}
	
	\subsubsection{Anomalous Magnetic Moment of the Muon}
	
	The T0-Model uses the universal formula for all leptons:
	
	\begin{equation}
		\Delta a_\ell^{(T0)} = 251 \times 10^{-11} \times \left(\frac{m_\ell}{m_\mu}\right)^2
	\end{equation}
	
\section*{Specific Values:}
	\begin{itemize}
		\item Muon: $\Delta a_\mu = 251 \times 10^{-11} \times 1 = 251 \times 10^{-11}$ \checkmark
		\item Electron: $\Delta a_e = 251 \times 10^{-11} \times (0.511/105.66)^2 = 5.87 \times 10^{-15}$
		\item Tau: $\Delta a_\tau = 251 \times 10^{-11} \times (1777/105.66)^2 = 7.10 \times 10^{-7}$
	\end{itemize}
	
	\textbf{Experimental Success:} Perfect agreement with muon g-2 experiment, parameter-free predictions for electron and tau
	
	\textbf{Document Reference:} \textit{CompleteMuon\_g-2\_AnalysisDe.pdf}, \textit{detailierte\_formel\_leptonen\_anemal\_De.pdf}
	
	\subsubsection{Other Empirically Confirmed Values}
	
	\begin{itemize}
		\item Gravitational constant: $G = 6.67430\ldots \times 10^{-11} \, \text{m}^3 \, \text{kg}^{-1} \, \text{s}^{-2}$ \checkmark
		\item Fine structure constant: $\alphapar^{-1} = 137.036\ldots$ \checkmark
		\item Lepton mass ratios: $m_\mu/m_e = 207.8$ (theory) vs $206.77$ (experiment) \checkmark
		\item Hubble constant: $H_0 = 67.2 \, \text{km/s/Mpc}$ (99.7\% agreement with Planck) \checkmark
	\end{itemize}
	
	\textbf{Document Reference:} \textit{CompleteMuon\_g-2\_AnalysisDe.pdf}, \textit{T0-Theory: Formulas for xi and Gravitational Constant.md}
	
	\subsection{Testable Parameters without New Free Constants}
	
	The T0-Model makes predictions for not yet measured values:
	
	\begin{table}[h]
		\centering
		\begin{tabular}{lccc}
			\toprule
			\textbf{Observable} & \textbf{T0-Prediction} & \textbf{Status} & \textbf{Precision} \\
			\midrule
			Electron g-2 & $5.87 \times 10^{-15}$ & Measurable & $10^{-13}$ \\
			Tau g-2 & $7.10 \times 10^{-7}$ & Future measurable & $10^{-9}$ \\
			\bottomrule
		\end{tabular}
		\caption{Future testable predictions}
	\end{table}
	
	Important distinction: These are not free parameters but follow directly from the already confirmed muon g-2 formula: $\Delta a_\ell = 251 \times 10^{-11} \times (m_\ell/m_\mu)^2$
	
	\subsection{Particle Physics}
	
	\subsubsection{Simplified Dirac Equation}
	
	The T0-Model reduces the complex $4 \times 4$ matrix structure of the Dirac equation to simple field node dynamics.
	
	\textbf{Document Reference:} \textit{systemDe.pdf}
	
	\subsection{Cosmology}
	
	\subsubsection{Static, Cyclic Universe}
	
	The T0-Model proposes a unified, static, cyclic universe that operates without dark matter and dark energy.
	
	\subsubsection{Wavelength-dependent Redshift}
	
	The T0-Model offers alternative mechanisms for redshift:
	
	\begin{equation}
		\frac{dE}{dx} = -\xipar \cdot f(E/E_\xipar) \cdot E
	\end{equation}
	
	The T0-Model proposes several explanations (besides standard space expansion): photon energy loss through $\xipar$-field interaction and diffraction effects. While diffraction effects are theoretically preferred, the energy loss mechanism is mathematically simpler to formulate.
	
	\textbf{Document Reference:} \textit{cosmic\_De.pdf}
	
	\subsection{Quantum Mechanics}
	
	\subsubsection{Deterministic Quantum Mechanics}
	
	The T0-Model develops an alternative deterministic quantum mechanics:
	
\section*{Eliminated Concepts:}
	\begin{itemize}
		\item Wave function collapse dependent on measurement
		\item Observer-dependent reality in quantum mechanics
		\item Probabilistic fundamental laws
		\item Multiple parallel universes
		\item Fundamental randomness
	\end{itemize}
	
\section*{New Concepts:}
	\begin{itemize}
		\item Deterministic field evolution
		\item Objective geometric reality
		\item Universal physical laws
		\item Single, consistent universe
		\item Predictable individual events
	\end{itemize}
	
	\subsubsection{Modified Schrödinger Equation}
	
	\begin{equation}
		i\hbar\frac{\partial\psi}{\partial t} + i\psi\left[\frac{\partial T_{\text{field}}}{\partial t} + \vec{v} \cdot \nabla T_{\text{field}}\right] = \hat{H}\psi
	\end{equation}
	
	\subsubsection{Deterministic Entanglement}
	
	Entanglement arises from correlated energy field structures:
	\begin{equation}
		E_{12}(x_1,x_2,t) = E_1(x_1,t) + E_2(x_2,t) + E_{\text{corr}}(x_1,x_2,t)
	\end{equation}
	
	\subsubsection{Modified Quantum Mechanics}
	
	\begin{itemize}
		\item Continuous energy field evolution instead of collapse
		\item Deterministic individual measurement predictions
		\item Objective, deterministic reality
		\item Local energy field interactions
	\end{itemize}
	
	\textbf{Document Reference:} \textit{QM-Detrmistic\_p\_De.pdf}, \textit{scheinbar\_instantan\_De.pdf}, \textit{QM-testenDe.pdf}, \textit{T0-Energie\_De.pdf}
	
	\section{Theoretical Implications}
	
	\subsection{Elimination of Free Parameters}
	
	The T0-Model successfully eliminates the over 20 free parameters of the Standard Model through:
	
	\begin{itemize}
		\item Reduction to one geometric constant
		\item Universal energy field description
		\item Geometric foundation of all physics
	\end{itemize}
	
	\subsection{Simplification of Physics Hierarchy}
	
\section*{Standard Model Hierarchy:}
	\begin{equation}
		\text{Quarks \& Leptons} \rightarrow \text{Particles} \rightarrow \text{Atoms} \rightarrow \text{???}
	\end{equation}
	
\section*{T0-Geometric Hierarchy:}
	\begin{equation}
		\text{3D$\xi$-Geometry} \rightarrow \text{Energy Fields} \rightarrow \text{Particles} \rightarrow \text{Atoms}
	\end{equation}
	
	\textbf{Document Reference:} \textit{T0-Energie\_De.pdf}, \textit{Zusammenfassung\_De.pdf}
	
	\subsection{Epistemological Considerations}
	
	The T0-Model acknowledges fundamental epistemological limits:
	\begin{itemize}
		\item Theoretical underdetermination
		\item Multiple possible mathematical frameworks
		\item Necessity of empirical distinguishability
	\end{itemize}
	
	\textbf{Document Reference:} \textit{T0-Energie\_De.pdf}
	
	\section{Future Perspectives}
	
	\subsection{Theoretical Development}
	
	Priorities for further research:
	
	\begin{enumerate}
		\item Complete mathematical formalization of the $\xipar$-field
		\item Detailed calculations for all particle masses
		\item Consistency checks with established theories
		\item Alternative derivations of the $\xipar$-constant
	\end{enumerate}
	
	\subsection{Experimental Programs}
	
	Required measurements:
	
	\begin{enumerate}
		\item High-precision spectroscopy at various wavelengths
		\item Improved g-2 measurements for all leptons
		\item Tests of modified Bell inequalities
		\item Search for $\xipar$-field signatures in precision experiments
	\end{enumerate}
	
	\textbf{Document Reference:} \textit{HdokumentDe.pdf}
	
	\section{Final Assessment}
	
	\subsection{Essential Aspects}
	
	The T0-Model demonstrates a novel approach through:
	
	\begin{itemize}
		\item Radical simplification: From 20+ parameters to one geometric framework
		\item Conceptual clarity: Unified description of all physics
		\item Mathematical elegance: Geometric beauty of the reduction
		\item Experimental relevance: Remarkable agreement with muon g-2
	\end{itemize}
	
	\subsection{Central Message}
	
	The T0-Model shows that the search for the theory of everything may possibly lie not in greater complexity, but in radical simplification. The ultimate truth could be extraordinarily simple.
	
	\textbf{Document Reference:} \textit{HdokumentDe.pdf}
	
	\section{References}
	
	All documents are available at: \url{https://github.com/jpascher/T0-Time-Mass-Duality/blob/main/2/pdf/}
	
	\subsection{German Versions}
	
	\begin{itemize}
		\item HdokumentDe.pdf (Master document)
		\item Zusammenfassung\_De.pdf (Theoretical treatise)
		\item T0-Energie\_De.pdf (Energy-based formulation)
		\item cosmic\_De.pdf (Cosmological applications)
		\item DerivationVonBetaDe.pdf ($\betapar$-parameter derivation)
		\item xi\_parameter\_partikel\_De.pdf ($\xipar$-parameter analysis)
		\item systemDe.pdf (System-theoretical foundations)
		\item T0vsESM\_ConceptualAnalysis\_De.pdf (Standard Model comparison)
	\end{itemize}
	
	\subsection{English Versions}
	
	Corresponding \texttt{.En.pdf} versions available
	
	\textbf{Author:} Johann Pascher, HTL Leonding, Austria\\
	\textbf{Email:} johann.pascher@gmail.com
	


% Bibliography
\begin{thebibliography}{99}
	
	\bibitem{pdg2024}
	Particle Data Group Collaboration (2024). 
	\textit{Review of Particle Physics}. 
	Progress of Theoretical and Experimental Physics, 2024(8), 083C01.
	\url{https://pdg.lbl.gov}
	
	\bibitem{flag2024}
	Aoki, Y., et al. (FLAG Collaboration) (2024). 
	\textit{FLAG Review 2024 of Lattice Results for Low-Energy Constants}. 
	arXiv:2411.04268.
	\url{https://arxiv.org/abs/2411.04268}
	
	\bibitem{fermilab_muon_g2}
	Abi, B., et al. (Muon g-2 Collaboration) (2021). 
	\textit{Measurement of the Positive Muon Anomalous Magnetic Moment to 0.46 ppm}. 
	Physical Review Letters, 126, 141801.
	
	\bibitem{peskin_schroeder}
	Peskin, M. E., \& Schroeder, D. V. (1995). 
	\textit{An Introduction to Quantum Field Theory}. 
	Addison-Wesley.
	
	\bibitem{weinberg_qft}
	Weinberg, S. (1995). 
	\textit{The Quantum Theory of Fields, Vol. I--III}. 
	Cambridge University Press.
	
	\bibitem{griffiths_particle}
	Griffiths, D. (2008). 
	\textit{Introduction to Elementary Particles}. 
	Wiley-VCH.
	
	\bibitem{mandl_shaw}
	Mandl, F., \& Shaw, G. (2010). 
	\textit{Quantum Field Theory (2nd ed.)}. 
	Wiley.
	
	\bibitem{srednicki_qft}
	Srednicki, M. (2007). 
	\textit{Quantum Field Theory}. 
	Cambridge University Press.
	
	\bibitem{t0_fundamentals}
	Pascher, J. (2024). 
	\textit{T0-Theory: Foundations of Time-Mass Duality}. 
	Unpublished manuscript, HTL Leonding.
	
	\bibitem{t0_fine_structure}
	Pascher, J. (2024). 
	\textit{T0-Theory: The Fine Structure Constant}. 
	Unpublished manuscript, HTL Leonding.
	
	\bibitem{t0_neutrinos}
	Pascher, J. (2024). 
	\textit{T0-Theory: Neutrino Masses and PMNS Mixing}. 
	Unpublished manuscript, HTL Leonding.
	
	\bibitem{t0_github}
	Pascher, J. (2024--2025). 
	\textit{T0-Time-Mass-Duality Repository}. 
	GitHub.
	\url{https://github.com/jpascher/T0-Time-Mass-Duality}
	
	\bibitem{lattice_qcd_review}
	Kronfeld, A. S. (2012). 
	\textit{Twenty-first Century Lattice Gauge Theory: Results from the QCD Lagrangian}. 
	Annual Review of Nuclear and Particle Science, 62, 265--284.
	
	\bibitem{neutrino_mixing_pdg}
	Particle Data Group Collaboration (2024). 
	\textit{Neutrino Masses, Mixing, and Oscillations}. 
	PDG Review 2024.
	\url{https://pdg.lbl.gov/2024/reviews/rpp2024-rev-neutrino-mixing.pdf}
	
	\bibitem{higgs_discovery}
	ATLAS and CMS Collaborations (2012). 
	\textit{Observation of a New Particle in the Search for the Standard Model Higgs Boson}. 
	Physics Letters B, 716, 1--29.
	
	\bibitem{Brannen2005}
	C. P. Brannen, ``Estimate of neutrino masses from Koide's relation'', \textit{arXiv:hep-ph/0505028} (2005).
	\url{https://arxiv.org/abs/hep-ph/0505028}
	
	\bibitem{Brannen2006}
	C. P. Brannen, ``Koide Mass Formula for Neutrinos'', \textit{arXiv:0702.0052} (2006).
	\url{http://brannenworks.com/MASSES.pdf}
	
	\bibitem{PhaseVectors2025}
	Anonymous, ``The Koide Relation and Lepton Mass Hierarchy from Phase Vectors'', \textit{rXiv:2507.0040} (2025).
	\url{https://rxiv.org/pdf/2507.0040v1.pdf}
	
	\bibitem{PDG2025}
	Particle Data Group, ``Review of Particle Physics'', \textit{Phys. Rev. D} \textbf{112} (2025) 030001.
	\url{https://pdg.lbl.gov/2025/}
	
	\bibitem{terrell2024}
	Terrell et al. (2024). 
	\textit{Single-Clock Metrology in Nature}. 
	Nature Physics.
	
	\bibitem{hossenfelder2024}
	Hossenfelder, S. (2024). 
	\textit{Single Clock Video Explanation}. 
	YouTube.
	
	\bibitem{hundert1931}
	Hundert (1931). 
	\textit{Reference Work}. 
	Publisher.
	
	\bibitem{terrell2025}
	Terrell et al. (2025). 
	\textit{Advanced Clock Synchronization Methods}. 
	Physical Review Letters.
	
	\bibitem{pascher_t0_2025}
	Pascher, J. (2025). 
	\textit{T0-Theory: Complete Framework and Applications}. 
	Unpublished manuscript, HTL Leonding.
	
	\bibitem{t0qm}
	Pascher, J. (2024). 
	\textit{T0-Theory: Quantum Mechanics Formulation}. 
	Unpublished manuscript, HTL Leonding.
	
	\bibitem{t0anomale}
	Pascher, J. (2024). 
	\textit{T0-Theory: Anomalous Magnetic Moments}. 
	Unpublished manuscript, HTL Leonding.
	
	\bibitem{muong2complete}
	Abi, B., et al. (Muon g-2 Collaboration) (2023). 
	\textit{Complete Measurement of the Positive Muon Anomalous Magnetic Moment}. 
	Physical Review Letters, 131, 161802.
	
	\bibitem{penrose2004}
	Penrose, R. (2004). 
	\textit{The Road to Reality: A Complete Guide to the Laws of the Universe}. 
	Jonathan Cape.
	
	\bibitem{planck1900}
	Planck, M. (1900). 
	\textit{On the Theory of the Energy Distribution Law of the Normal Spectrum}. 
	Verhandlungen der Deutschen Physikalischen Gesellschaft, 2, 237.
	
	\bibitem{T0Theory}
	Pascher, J. (2024). 
	\textit{T0-Theory: Fundamental Principles}. 
	Unpublished manuscript, HTL Leonding.
	
	% Additional bibliography entries for all undefined citations
	\bibitem{6g_roadmap}
	6G Research Consortium (2024).
	\textit{6G Technology Roadmap}.
	Technical Report.
	
	\bibitem{Born2013}
	Born, M. (2013).
	\textit{Einstein's Theory of Relativity}.
	Dover Publications.
	
	\bibitem{Casimir1948}
	Casimir, H. B. G. (1948).
	\textit{On the attraction between two perfectly conducting plates}.
	Proc. Kon. Ned. Akad. Wetensch. B51, 793--795.
	
	\bibitem{Einstein1905}
	Einstein, A. (1905).
	\textit{On the Electrodynamics of Moving Bodies}.
	Annalen der Physik, 17, 891--921.
	
	\bibitem{Feynman2006}
	Feynman, R. P. (2006).
	\textit{QED: The Strange Theory of Light and Matter}.
	Princeton University Press.
	
	\bibitem{Griffiths2017}
	Griffiths, D. J. (2017).
	\textit{Introduction to Electrodynamics (4th ed.)}.
	Cambridge University Press.
	
	\bibitem{Jackson1999}
	Jackson, J. D. (1999).
	\textit{Classical Electrodynamics (3rd ed.)}.
	Wiley.
	
	\bibitem{Mohr2016}
	Mohr, P. J., et al. (2016).
	\textit{CODATA Recommended Values of the Fundamental Physical Constants: 2014}.
	Rev. Mod. Phys. 88, 035009.
	
	\bibitem{Parker2018}
	Parker, R. H., et al. (2018).
	\textit{Measurement of the fine-structure constant as a test of the Standard Model}.
	Science, 360, 191--195.
	
	\bibitem{Planck1900}
	Planck, M. (1900).
	\textit{On the Theory of the Energy Distribution Law of the Normal Spectrum}.
	Verhandlungen der Deutschen Physikalischen Gesellschaft, 2, 237.
	
	\bibitem{Planck2018}
	Planck Collaboration (2018).
	\textit{Planck 2018 results. VI. Cosmological parameters}.
	Astronomy \& Astrophysics, 641, A6.
	
	\bibitem{QFT_T0}
	Pascher, J. (2024).
	\textit{T0-Theory and QFT Connections}.
	Unpublished manuscript, HTL Leonding.
	
	\bibitem{Sommerfeld1916}
	Sommerfeld, A. (1916).
	\textit{On the Quantum Theory of Spectral Lines}.
	Annalen der Physik, 51, 1--94.
	
	\bibitem{T0_Feinstruktur}
	Pascher, J. (2024).
	\textit{T0-Theory: Fine Structure Analysis}.
	Unpublished manuscript, HTL Leonding.
	
	\bibitem{T0_SI}
	Pascher, J. (2024).
	\textit{T0-Theory and SI Units}.
	Unpublished manuscript, HTL Leonding.
	
	\bibitem{T0_fine_structure}
	Pascher, J. (2024).
	\textit{T0-Theory: The Fine Structure Constant}.
	Unpublished manuscript, HTL Leonding.
	
	\bibitem{T0_g2_erweiterung}
	Pascher, J. (2024).
	\textit{T0-Theory: g-2 Extensions}.
	Unpublished manuscript, HTL Leonding.
	
	\bibitem{T0_gravitational_constant}
	Pascher, J. (2024).
	\textit{T0-Theory: Gravitational Constant Derivation}.
	Unpublished manuscript, HTL Leonding.
	
	\bibitem{T0_netze_en}
	Pascher, J. (2024).
	\textit{T0-Theory: Network Structures}.
	Unpublished manuscript, HTL Leonding.
	
	\bibitem{T0_tm_erweiterung}
	Pascher, J. (2024).
	\textit{T0-Theory: Time-Mass Extensions}.
	Unpublished manuscript, HTL Leonding.
	
	\bibitem{Uzan2003}
	Uzan, J.-P. (2003).
	\textit{The fundamental constants and their variation}.
	Rev. Mod. Phys. 75, 403--455.
	
	\bibitem{Weinberg1995}
	Weinberg, S. (1995).
	\textit{The Quantum Theory of Fields, Vol. I}.
	Cambridge University Press.
	
	\bibitem{albrecht1999}
	Albrecht, A. \& Magueijo, J. (1999).
	\textit{A time varying speed of light as a solution to cosmological puzzles}.
	Phys. Rev. D 59, 043516.
	
	\bibitem{alice2023}
	ALICE Collaboration (2023).
	\textit{Recent results from ALICE}.
	CERN-EP-2023-XXX.
	
	\bibitem{analog_optical}
	Smith, J. et al. (2024).
	\textit{Analog optical computing systems}.
	Nature Photonics.
	
	\bibitem{ashtekar2004}
	Ashtekar, A. \& Lewandowski, J. (2004).
	\textit{Background independent quantum gravity}.
	Class. Quantum Grav. 21, R53.
	
	\bibitem{atlas2023}
	ATLAS Collaboration (2023).
	\textit{ATLAS physics results}.
	CERN-PH-EP-2023-XXX.
	
	\bibitem{atlas2023higgs}
	ATLAS Collaboration (2023).
	\textit{Higgs boson measurements}.
	Phys. Rev. Lett.
	
	\bibitem{barbour1999}
	Barbour, J. (1999).
	\textit{The End of Time}.
	Oxford University Press.
	
	\bibitem{barrow1999}
	Barrow, J. D. (1999).
	\textit{Cosmologies with varying light speed}.
	Phys. Rev. D 59, 043515.
	
	\bibitem{becker2007}
	Becker, K. et al. (2007).
	\textit{String Theory and M-Theory}.
	Cambridge University Press.
	
	\bibitem{bell_muon}
	Bennett, G. W., et al. (Muon g-2 Collaboration) (2006).
	\textit{Final report of the E821 muon anomalous magnetic moment measurement}.
	Phys. Rev. D 73, 072003.
	
	\bibitem{bondi1948}
	Bondi, H. \& Gold, T. (1948).
	\textit{The steady-state theory of the expanding universe}.
	Mon. Not. R. Astron. Soc. 108, 252--270.
	
	\bibitem{brewer2019}
	Brewer, S. M. et al. (2019).
	\textit{Al+ Quantum-Logic Clock with Systematic Uncertainty below $10^{-18}$}.
	Phys. Rev. Lett. 123, 033201.
	
	\bibitem{cms2023top}
	CMS Collaboration (2023).
	\textit{Top quark measurements at CMS}.
	JHEP 2023.
	
	\bibitem{cms2024}
	CMS Collaboration (2024).
	\textit{CMS physics results 2024}.
	CERN-PH-EP-2024-XXX.
	
	\bibitem{codata2019}
	Tiesinga, E. et al. (2019).
	\textit{The 2018 CODATA Recommended Values}.
	J. Phys. Chem. Ref. Data.
	
	\bibitem{desi2025}
	DESI Collaboration (2025).
	\textit{DESI 2025 Cosmology Results}.
	arXiv preprint.
	
	\bibitem{differential_optical}
	Wang, X. et al. (2024).
	\textit{Differential optical computing}.
	Optica.
	
	\bibitem{dingle1972}
	Dingle, H. (1972).
	\textit{Science at the Crossroads}.
	Martin Brian \& O'Keeffe.
	
	\bibitem{divalentino2021}
	Di Valentino, E. et al. (2021).
	\textit{In the realm of the Hubble tension}.
	Class. Quantum Grav. 38, 153001.
	
	\bibitem{elnaschie2004}
	El Naschie, M. S. (2004).
	\textit{A review of E infinity theory}.
	Chaos, Solitons \& Fractals, 19, 209--236.
	
	\bibitem{fabrication_heterogeneous}
	Chen, Y. et al. (2024).
	\textit{Heterogeneous photonic integration}.
	Nature Electronics.
	
	\bibitem{fermilab2023}
	Fermilab (2023).
	\textit{Muon g-2 results}.
	Phys. Rev. Lett.
	
	\bibitem{flexible_wafer}
	Kim, S. et al. (2024).
	\textit{Flexible wafer-scale photonics}.
	Science Advances.
	
	\bibitem{francesco1997}
	Di Francesco, P. et al. (1997).
	\textit{Conformal Field Theory}.
	Springer.
	
	\bibitem{hartree1957}
	Hartree, D. R. (1957).
	\textit{The Calculation of Atomic Structures}.
	Wiley.
	
	\bibitem{hhi_6g}
	Fraunhofer HHI (2024).
	\textit{6G Photonic Integration}.
	Technical Report.
	
	\bibitem{hossenfelder2025}
	Hossenfelder, S. (2025).
	\textit{Science without the gobbledygook}.
	YouTube/Blog.
	
	\bibitem{hossenfelder_single_clock_video}
	Hossenfelder, S. (2024).
	\textit{The Single Clock Problem}.
	YouTube.
	
	\bibitem{hoyle1948}
	Hoyle, F. (1948).
	\textit{A new model for the expanding universe}.
	Mon. Not. R. Astron. Soc. 108, 372--382.
	
	\bibitem{integration_microelectronic}
	Liu, A. et al. (2024).
	\textit{Microelectronic photonic integration}.
	IEEE Journal.
	
	\bibitem{jacobson1995}
	Jacobson, T. (1995).
	\textit{Thermodynamics of spacetime}.
	Phys. Rev. Lett. 75, 1260.
	
	\bibitem{kasevich2023}
	Kasevich, M. et al. (2023).
	\textit{Atom interferometry tests}.
	Nature Physics.
	
	\bibitem{lerner2014}
	Lerner, E. J. (2014).
	\textit{An open letter on cosmology}.
	New Scientist.
	
	\bibitem{lisa2017}
	LISA Consortium (2017).
	\textit{Laser Interferometer Space Antenna}.
	ESA Technical Report.
	
	\bibitem{lithium_tantalate}
	Zhang, M. et al. (2024).
	\textit{Thin-film lithium tantalate photonics}.
	Nature Photonics.
	
	\bibitem{lopez2010}
	Lopez-Corredoira, M. (2010).
	\textit{Tests and problems of the standard model in cosmology}.
	Int. J. Mod. Phys. D.
	
	\bibitem{ludlow2015}
	Ludlow, A. D. et al. (2015).
	\textit{Optical atomic clocks}.
	Rev. Mod. Phys. 87, 637.
	
	\bibitem{mach1883}
	Mach, E. (1883).
	\textit{Die Mechanik in ihrer Entwickelung}.
	F.A. Brockhaus.
	
	\bibitem{maldacena1998}
	Maldacena, J. (1998).
	\textit{The large N limit of superconformal field theories}.
	Adv. Theor. Math. Phys. 2, 231--252.
	
	\bibitem{mueller2014}
	Müller, H. et al. (2014).
	\textit{Atom interferometry tests of the gravitational redshift}.
	Phys. Rev. Lett.
	
	\bibitem{mug2_final_2025}
	Muon g-2 Collaboration (2025).
	\textit{Final muon g-2 measurement}.
	Phys. Rev. Lett.
	
	\bibitem{muong2_2023}
	Muon g-2 Collaboration (2023).
	\textit{Updated muon g-2 results}.
	Phys. Rev. Lett.
	
	\bibitem{nathan2024}
	Nathan, A. et al. (2024).
	\textit{Quantum computing advances}.
	Nature.
	
	\bibitem{newell2018}
	Newell, D. B. et al. (2018).
	\textit{The CODATA 2017 values of h, e, k, and $N_A$}.
	Metrologia 55, L13.
	
	\bibitem{nottale1993}
	Nottale, L. (1993).
	\textit{Fractal Space-Time and Microphysics}.
	World Scientific.
	
	\bibitem{on_chip_lithium}
	Wang, C. et al. (2024).
	\textit{On-chip lithium niobate photonics}.
	Nature Communications.
	
	\bibitem{optical_advantages}
	Shastri, B. J. et al. (2024).
	\textit{Advantages of optical computing}.
	Nature Reviews Physics.
	
	\bibitem{pascher2025cmb}
	Pascher, J. (2025).
	\textit{T0-Theory: CMB Analysis}.
	Unpublished manuscript, HTL Leonding.
	
	\bibitem{pascher2025g2}
	Pascher, J. (2025).
	\textit{T0-Theory: g-2 Predictions}.
	Unpublished manuscript, HTL Leonding.
	
	\bibitem{pascher2025qm}
	Pascher, J. (2025).
	\textit{T0-Theory: Quantum Mechanics}.
	Unpublished manuscript, HTL Leonding.
	
	\bibitem{pascher2025si}
	Pascher, J. (2025).
	\textit{T0-Theory: SI Unit System}.
	Unpublished manuscript, HTL Leonding.
	
	\bibitem{pascher2025t0}
	Pascher, J. (2025).
	\textit{T0-Theory: Complete Framework}.
	Unpublished manuscript, HTL Leonding.
	
	\bibitem{pascher:fundamentals}
	Pascher, J. (2024).
	\textit{T0-Theory: Fundamentals}.
	Unpublished manuscript, HTL Leonding.
	
	\bibitem{pascher:g2_rev9}
	Pascher, J. (2024).
	\textit{T0-Theory: g-2 Revision 9}.
	Unpublished manuscript, HTL Leonding.
	
	\bibitem{pascher:geometric_formalism}
	Pascher, J. (2024).
	\textit{T0-Theory: Geometric Formalism}.
	Unpublished manuscript, HTL Leonding.
	
	\bibitem{pascher:ml_addendum}
	Pascher, J. (2024).
	\textit{T0-Theory: Machine Learning Addendum}.
	Unpublished manuscript, HTL Leonding.
	
	\bibitem{pascher:t0_foundations}
	Pascher, J. (2024).
	\textit{T0-Theory: Foundations}.
	Unpublished manuscript, HTL Leonding.
	
	\bibitem{pascher_derivation_beta_2025}
	Pascher, J. (2025).
	\textit{T0-Theory: Derivation of Beta}.
	Unpublished manuscript, HTL Leonding.
	
	\bibitem{pascher_higgs_connection_2025}
	Pascher, J. (2025).
	\textit{T0-Theory: Higgs Connection}.
	Unpublished manuscript, HTL Leonding.
	
	\bibitem{pascher_lagrangian_extended_2025}
	Pascher, J. (2025).
	\textit{T0-Theory: Extended Lagrangian}.
	Unpublished manuscript, HTL Leonding.
	
	\bibitem{pascher_mathematical_structure_2025}
	Pascher, J. (2025).
	\textit{T0-Theory: Mathematical Structure}.
	Unpublished manuscript, HTL Leonding.
	
	\bibitem{pascher_t0_cmb_2025}
	Pascher, J. (2025).
	\textit{T0-Theory: CMB Predictions}.
	Unpublished manuscript, HTL Leonding.
	
	\bibitem{pascher_t0_energie_2025}
	Pascher, J. (2025).
	\textit{T0-Theory: Energy}.
	Unpublished manuscript, HTL Leonding.
	
	\bibitem{pascher_t0_energy_2025}
	Pascher, J. (2025).
	\textit{T0-Theory: Energy Framework}.
	Unpublished manuscript, HTL Leonding.
	
	\bibitem{pascher_t0_theory_2025}
	Pascher, J. (2025).
	\textit{T0-Theory: Complete Theory}.
	Unpublished manuscript, HTL Leonding.
	
	\bibitem{penrose1959}
	Penrose, R. (1959).
	\textit{The apparent shape of a relativistically moving sphere}.
	Proc. Cambridge Phil. Soc. 55, 137--139.
	
	\bibitem{penrose1967}
	Penrose, R. (1967).
	\textit{Twistor algebra}.
	J. Math. Phys. 8, 345--366.
	
	\bibitem{peratt1992}
	Peratt, A. L. (1992).
	\textit{Physics of the Plasma Universe}.
	Springer-Verlag.
	
	\bibitem{peskin1995}
	Peskin, M. E. \& Schroeder, D. V. (1995).
	\textit{An Introduction to Quantum Field Theory}.
	Addison-Wesley.
	
	\bibitem{peskin_schroeder_1995}
	Peskin, M. E. \& Schroeder, D. V. (1995).
	\textit{An Introduction to Quantum Field Theory}.
	Addison-Wesley.
	
	\bibitem{phoquant}
	PhoQuant (2024).
	\textit{Photonic quantum computing}.
	Technical Report.
	
	\bibitem{photonics_ai}
	Wetzstein, G. et al. (2024).
	\textit{Photonics for AI}.
	Nature.
	
	\bibitem{planck1906}
	Planck, M. (1906).
	\textit{The Theory of Heat Radiation}.
	Johann Ambrosius Barth.
	
	\bibitem{planck2018}
	Planck Collaboration (2018).
	\textit{Planck 2018 results}.
	A\&A 641, A6.
	
	\bibitem{polchinski1998}
	Polchinski, J. (1998).
	\textit{String Theory}.
	Cambridge University Press.
	
	\bibitem{qant_nps}
	QANT (2024).
	\textit{Quantum photonics systems}.
	Technical Report.
	
	\bibitem{quantenjahr25}
	Quantenjahr (2025).
	\textit{International Year of Quantum}.
	UNESCO.
	
	\bibitem{recurrent_photonics}
	Tait, A. N. et al. (2024).
	\textit{Recurrent photonic neural networks}.
	Optica.
	
	\bibitem{rf_photonics}
	Capmany, J. \& Novak, D. (2024).
	\textit{Microwave photonics}.
	Nature Photonics.
	
	\bibitem{riess2019}
	Riess, A. G. et al. (2019).
	\textit{Large Magellanic Cloud Cepheid Standards}.
	ApJ 876, 85.
	
	\bibitem{riess2022}
	Riess, A. G. et al. (2022).
	\textit{A Comprehensive Measurement of H0}.
	ApJ 934, L7.
	
	\bibitem{rovelli2004}
	Rovelli, C. (2004).
	\textit{Quantum Gravity}.
	Cambridge University Press.
	
	\bibitem{sciama1953}
	Sciama, D. W. (1953).
	\textit{On the origin of inertia}.
	Mon. Not. R. Astron. Soc. 113, 34--42.
	
	\bibitem{sciencedaily2025}
	ScienceDaily (2025).
	\textit{Physics news}.
	Online.
	
	\bibitem{sm_g2_2025}
	Aoyama, T. et al. (2025).
	\textit{Standard Model prediction for g-2}.
	Phys. Rep.
	
	\bibitem{susskind1995}
	Susskind, L. (1995).
	\textit{The world as a hologram}.
	J. Math. Phys. 36, 6377--6396.
	
	\bibitem{t0_kosmologie}
	Pascher, J. (2024).
	\textit{T0-Theory: Cosmology}.
	Unpublished manuscript, HTL Leonding.
	
	\bibitem{terrell1959}
	Terrell, J. (1959).
	\textit{Invisibility of the Lorentz contraction}.
	Phys. Rev. 116, 1041--1045.
	
	\bibitem{terrell_single_clock_nature_2024}
	Terrell, J. et al. (2024).
	\textit{Single clock precision measurements}.
	Nature Physics.
	
	\bibitem{tfln_foundry}
	TFLN Foundry (2024).
	\textit{Thin-film lithium niobate foundry services}.
	Technical Specifications.
	
	\bibitem{thiemann2007}
	Thiemann, T. (2007).
	\textit{Modern Canonical Quantum General Relativity}.
	Cambridge University Press.
	
	\bibitem{thz_epfl}
	EPFL (2024).
	\textit{Terahertz photonics research}.
	Technical Report.
	
	\bibitem{unnikrishnan2004}
	Unnikrishnan, C. S. (2004).
	\textit{On Einstein's resolution of the twin clock paradox}.
	Current Science, 86, 704--709.
	
	\bibitem{verlinde2011}
	Verlinde, E. (2011).
	\textit{On the origin of gravity and the laws of Newton}.
	JHEP 2011, 29.
	
	\bibitem{video2025}
	Video (2025).
	\textit{Physics video explanation}.
	YouTube.
	
	\bibitem{weinberg1995}
	Weinberg, S. (1995).
	\textit{The Quantum Theory of Fields}.
	Cambridge University Press.
	
	\bibitem{weiskopf2000}
	Weiskopf, D. (2000).
	\textit{Visualization of special relativity}.
	PhD thesis, University of Tübingen.
	
	\bibitem{wheeler1990}
	Wheeler, J. A. (1990).
	\textit{A Journey into Gravity and Spacetime}.
	Scientific American Library.
	
	\bibitem{wiki_bell}
	Wikipedia (2024).
	\textit{Bell's theorem}.
	Online encyclopedia.
	
	\bibitem{zwicky1929}
	Zwicky, F. (1929).
	\textit{On the red shift of spectral lines through interstellar space}.
	Proc. Natl. Acad. Sci. 15, 773--779.

\end{thebibliography}


\end{document}
