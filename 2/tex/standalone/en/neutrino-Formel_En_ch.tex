\documentclass[11pt,a4paper]{article}
\usepackage[a4paper,margin=2cm]{geometry}
\usepackage[utf8]{inputenc}
\usepackage[english]{babel}
\usepackage{lmodern}
\renewcommand{\familydefault}{\sfdefault}

\usepackage{amsmath,amssymb,amsthm}
\usepackage{graphicx}
\usepackage[unicode,pdfencoding=auto,hypertexnames=false]{hyperref}
\usepackage{booktabs}
\usepackage{longtable}
\usepackage{array}
\usepackage{siunitx}
\usepackage{fancyhdr}
\usepackage{float}
\usepackage{tikz}
% tcolorbox removed for standalone
% tcbset removed
\tikzset{
  t0blue/.style={draw=blue,fill=blue!10},
  t0red/.style={draw=red,fill=red!10},
  t0green/.style={draw=green!50!black,fill=green!10},
  t0orange/.style={draw=orange,fill=orange!10},
}
\usepackage{setspace}
\usepackage{enumitem}
\usepackage{adjustbox}
\usepackage{xcolor}

% Define colors for xcolor package
\definecolor{t0green}{RGB}{34,139,34}
\definecolor{t0blue}{RGB}{0,0,255}
\definecolor{t0red}{RGB}{255,0,0}
\definecolor{t0orange}{RGB}{255,165,0}

% Define custom column types for tables
\newcolumntype{L}[1]{>{\raggedright\arraybackslash}p{#1}}
\newcolumntype{C}[1]{>{\centering\arraybackslash}p{#1}}
\newcolumntype{R}[1]{>{\raggedleft\arraybackslash}p{#1}}

\setlength{\parindent}{0pt}
\setlength{\parskip}{6pt}

\hypersetup{
  colorlinks=true,
  linkcolor=blue,
  citecolor=blue,
  urlcolor=blue
}
\pagestyle{fancy}
\setlength{\headheight}{28pt}

\newcommand{\checkmarkx}{\checkmark}
\newcommand{\warningx}{\textbf{!}}

% Makros aus Einzel-Dokumenten (Fallback-Definitionen)
\newcommand{\mytimes}{\times}
\newcommand{\myapprox}{\approx}
\newcommand{\mysim}{\sim}
\newcommand{\myomega}{\omega}
\newcommand{\mypi}{\pi}
\newcommand{\myrightarrow}{\rightarrow}
\newcommand{\mypropto}{\propto}
\newcommand{\deltafield}{\delta\phi}
\newcommand{\xipar}{\xi}
\newcommand{\xiT}{\xi}
\newcommand{\lambdah}{\lambda_h}

% Additional macros used in chapter files
\newcommand{\Kfrak}{K_{\text{frak}}}  % Fractal correction factor
\newcommand{\Dfrak}{D_f}              % Fractal dimension
\newcommand{\betapar}{\beta}          % T0 beta parameter
\newcommand{\alphapar}{\alpha}        % T0 alpha parameter
\newcommand{\Efield}{E}               % Energy field
% Note: checkmarkxa/warningxa are variants used in auto-generated chapter files
\newcommand{\checkmarkxa}{\checkmark}
\newcommand{\warningxa}{\textbf{!}}

% Additional T0-specific macros
\newcommand{\xigeom}{\xi_{\text{geom}}}  % Geometric xi
\newcommand{\lP}{\ell_P}                  % Planck length
\newcommand{\rzero}{r_0}                  % Characteristic radius
\newcommand{\xirat}{\xi_{\text{rat}}}     % Xi ratio
\newcommand{\tzero}{t_0}                  % Characteristic time
\newcommand{\natunits}{\text{(nat. units)}}  % Natural units annotation
\newcommand{\myRightarrow}{\Rightarrow}   % Arrow variant
\newcommand{\Lag}{\mathcal{L}}            % Lagrangian

% Physics macros used in chapter files
\newcommand{\CQCD}{C_{\text{QCD}}}        % QCD correction
\newcommand{\EP}{E_P}                     % Planck energy
\newcommand{\Ee}{E_e}                     % Electron energy
\newcommand{\Emu}{E_\mu}                  % Muon energy
\newcommand{\Exi}{E_\xi}                  % Xi energy
\newcommand{\Ezero}{E_0}                  % Characteristic energy
\newcommand{\Hubble}{H}                   % Hubble constant
\newcommand{\Kspec}{K_{\text{spec}}}      % Spectral correction
\newcommand{\Lambdat}{\Lambda_t}          % Time-related cosmological constant
\newcommand{\Leff}{\mathcal{L}_{\text{eff}}}  % Effective Lagrangian
\newcommand{\Lorentz}{\mathcal{L}}        % Lorentz symbol
\newcommand{\Lxi}{L_\xi}                  % Xi length
\newcommand{\Tfield}{T}                   % Time field
\newcommand{\Weyl}{W}                     % Weyl tensor/symbol
\newcommand{\alphaEMSI}{\alpha_{\text{EM,SI}}}  % EM alpha in SI
\newcommand{\alphaEMnat}{\alpha_{\text{EM,nat}}}  % EM alpha in natural units
\newcommand{\alphaem}{\alpha_{\text{em}}} % Electromagnetic alpha
\newcommand{\betaTSI}{\beta_{T,\text{SI}}}  % Beta in SI
\newcommand{\betaTnat}{\beta_{T,\text{nat}}}  % Beta in natural units
\newcommand{\deltam}{\delta m}            % Mass difference
\newcommand{\phiT}{\phi_T}                % T-field phi
\newcommand{\tP}{t_P}                     % Planck time
\newcommand{\rhoCMB}{\rho_{\text{CMB}}}   % CMB density
\newcommand{\rhoCasimir}{\rho_{\text{Casimir}}}  % Casimir density

% Table formatting
\usepackage{multirow}

% Additional physics macros
\newcommand{\Riem}{\mathcal{R}}           % Riemann tensor
\newcommand{\ZPinch}{Z_{\text{pinch}}}    % Z-pinch
\newcommand{\SynchPower}{P_{\text{synch}}} % Synchrotron power
\newcommand{\Rzero}{R_0}                  % Characteristic radius
\newcommand{\alphafine}{\alpha}           % Fine structure constant
\newcommand{\Etau}{E_\tau}                % Tau energy
\newcommand{\deltaE}{\delta E}            % Energy deviation
\newcommand{\EPlanck}{E_P}                % Planck energy
\newcommand{\pichar}{\pi}                 % Pi character
\newcommand{\alphaWSI}{\alpha_{W,\text{SI}}}  % Wien alpha in SI
\newcommand{\alphaWnat}{\alpha_{W,\text{nat}}}  % Wien alpha in natural units

% Einfache abstract-Umgebung für Kapitel:
\newenvironment{abstract}{%
  \begin{center}\bfseries Abstract\end{center}\small
}{\par}


\title{neutrino-Formel En}
\author{J. Pascher}
\date{\today}

\begin{document}
\maketitle

\section*{Neutrino Formel (neutrino-Formel)}

	\begin{abstract}
		This document presents a mathematically consistent formula structure for neutrino calculations within the T0 model, based on the hypothesis of equal masses for all flavor states (\(\nu_e, \nu_\mu, \nu_\tau\)). The neutrino mass is derived from the photon analogy (\(\frac{\xipar^2}{2}\)-suppression), and oscillations are explained by geometric phases based on \( T_x \cdot m_x = 1 \), with quantum numbers (\(n, \ell, j\)) determining phase differences. A plausible target value for the neutrino mass (\(m_\nu = 15 \text{ meV}\)) is derived from empirical data (cosmological constraints). The T0 model is based on speculative geometric harmonies without empirical support and is highly likely to be incomplete or incorrect. Scientific integrity requires a clear distinction between mathematical correctness and physical validity.
	\end{abstract}
	
	
	\section{Preamble: Scientific Integrity}
	
\section*{Warning}
		\textbf{CRITICAL LIMITATION:} The following formulas for neutrino masses are \textbf{speculative extrapolations} based on the untested hypothesis that neutrinos follow geometric harmonies and all flavor states have equal masses. This hypothesis has \textbf{no empirical basis} and is highly likely to be incomplete or incorrect. The mathematical formulas are nonetheless internally consistent and error-free.
		
		\vspace{0.5cm}
\section*{Scientific Integrity Requires:}
		\begin{itemize}
			\item Honesty about the speculative nature of predictions
			\item Mathematical correctness despite physical uncertainty
			\item Clear separation between hypotheses and verified facts
		\end{itemize}
% end box warning
	
	\section{Neutrinos as ''Near-Massless Photons'': The T0 Photon Analogy}
	
\section*{Speculation}
		\textbf{Fundamental T0 Insight:} Neutrinos can be understood as ''damped photons.''
		
		The remarkable similarity between photons and neutrinos suggests a deeper geometric kinship:
		\begin{itemize}
			\item \textbf{Speed:} Both propagate at nearly the speed of light
			\item \textbf{Penetration:} Both have extreme penetration capabilities
			\item \textbf{Mass:} Photon is exactly massless, neutrino is nearly massless
			\item \textbf{Interaction:} Photon interacts electromagnetically, neutrino interacts weakly
		\end{itemize}
% end box speculation
	
	\subsection{Photon-Neutrino Correspondence}
	
\section*{Important}
\section*{Physical Parallels:}
		\begin{align}
			\text{Photon:} \quad &E^2 = (pc)^2 + 0 \quad \text{(perfectly massless)} \\
			\text{Neutrino:} \quad &E^2 = (pc)^2 + \left(\sqrt{\frac{\xipar^2}{2}} m c^2\right)^2 \quad \text{(nearly massless)}
		\end{align}
		
\section*{Speed Comparison:}
		\begin{align}
			v_\gamma &= c \quad \text{(exact)} \\
			v_\nu &= c \times \left(1 - \frac{\xipar^2}{2}\right) \approx 0.9999999911 \times c
		\end{align}
		
		The speed difference is only \(8.89 \times 10^{-9}\) -- practically unmeasurable!
% end box important
	
	\subsection{Double \()-Suppression from Photon Analogy}
	
\section*{Formula}
		\textbf{T0 Hypothesis:} Neutrino = Photon with Geometric Double Damping
		
		If neutrinos are ''near-photons,'' two suppression factors arise:
		\begin{itemize}
			\item \textbf{First \(\xipar\) Factor:} ''Near massless'' (like a photon, but not perfect)
			\item \textbf{Second \(\xipar\) Factor:} ''Weak interaction'' (geometric coupling)
			\item \textbf{Result:} \(m_\nu \propto \frac{\xipar^2}{2}\), consistent with the speed difference \(v_\nu = c \times \left(1 - \frac{\xipar^2}{2}\right)\)
		\end{itemize}
		
\section*{Interaction Strength Comparison:}
		\begin{align}
			\sigma_\gamma &\sim \alpha_{\text{EM}} \approx \frac{1}{137} \\
			\sigma_\nu &\sim \frac{\xipar^2}{2} \times G_F \approx 8.888888 \times 10^{-9}
		\end{align}
		
		The ratio \(\sigma_\nu/\sigma_\gamma \sim \frac{\xipar^2}{2}\) confirms the geometric suppression!
% end box formula
	
	\section{Neutrino Oscillations}
	
\section*{Important}
		\textbf{Neutrino Oscillations:} Neutrinos can change their identity (flavor) during flight -- a phenomenon known as neutrino oscillation. A neutrino produced as an electron neutrino (\(\nu_e\)) can later be detected as a muon neutrino (\(\nu_\mu\)) or tau neutrino (\(\nu_\tau\)) and vice versa.
		
		In standard physics, this behavior is described by the mixing of mass eigenstates (\(\nu_1, \nu_2, \nu_3\)) connected to flavor states (\(\nu_e, \nu_\mu, \nu_\tau\)) via the PMNS matrix (Pontecorvo-Maki-Nakagawa-Sakata):
		\begin{align}
			\begin{pmatrix}
				\nu_e \\ \nu_\mu \\ \nu_\tau
			\end{pmatrix}
			=
			U_{\text{PMNS}}
			\begin{pmatrix}
				\nu_1 \\ \nu_2 \\ \nu_3
			\end{pmatrix},
		\end{align}
		where \(U_{\text{PMNS}}\) is the mixing matrix.
		
		Oscillations depend on mass differences \(\Delta m^2_{ij} = m_i^2 - m_j^2\) and mixing angles. Current experimental data (2025) provide:
		\begin{align}
			\Delta m^2_{21} &\approx 7.53 \times 10^{-5} \text{ eV}^2 \quad \text{[Solar]} \\
			\Delta m^2_{32} &\approx 2.44 \times 10^{-3} \text{ eV}^2 \quad \text{[Atmospheric]} \\
			m_\nu &> 0.06 \text{ eV} \quad \text{[At least one neutrino, 3}\sigma\text{]}
		\end{align}
		
\section*{Implications for T0:}
		\begin{itemize}
			\item The T0 model postulates equal masses for flavor states (\(\nu_e, \nu_\mu, \nu_\tau\)), implying \(\Delta m^2_{ij} = 0\), which is incompatible with standard oscillations.
			\item To explain oscillations, the T0 model uses geometric phases based on \( T_x \cdot m_x = 1 \), with quantum numbers (\(n, \ell, j\)) determining phase differences.
		\end{itemize}
% end box important
	
	\subsection{Geometric Phases as Oscillation Mechanism}
	
\section*{Speculation}
\section*{T0 Hypothesis: Geometric Phases for Oscillations}
		
		To reconcile the hypothesis of equal masses (\(m_{\nu_e} = m_{\nu_\mu} = m_{\nu_\tau} = m_\nu\)) with neutrino oscillations, it is speculated that oscillations in the T0 model are caused by geometric phases rather than mass differences. This is based on the T0 relation:
		\[
		T_x \cdot m_x = 1,
		\]
		where \(m_x = m_\nu = 4.54 \text{ meV}\) is the neutrino mass, and \(T_x\) is a characteristic time or frequency:
		\[
		T_x = \frac{1}{m_\nu} = \frac{1}{4.54 \times 10^{-3} \text{ eV}} \approx 2.2026 \times 10^2 \text{ eV}^{-1} \approx 1.449 \times 10^{-13} \text{ s}.
		\]
		
		The geometric phase is determined by the T0 quantum numbers (\(n, \ell, j\)):
		\[
		\phi_{\text{geo}, i} \propto f(n, \ell, j) \cdot \frac{L}{E} \cdot \frac{1}{T_x},
		\]
		where \(f(n, \ell, j) = \frac{n^6}{\ell^3}\) (or 1 for \(\ell = 0\)) are the geometric factors:
		\begin{align}
			f_{\nu_e} &= 1, \\
			f_{\nu_\mu} &= 64, \\
			f_{\nu_\tau} &= 91.125.
		\end{align}
		
\section*{Calculated Phase Differences:}
		\begin{align}
			\phi_{\nu_e} &\propto 1 \cdot \frac{L}{E} \cdot \frac{1}{T_x}, \\
			\phi_{\nu_\mu} &\propto 64 \cdot \frac{L}{E} \cdot \frac{1}{T_x}, \\
			\phi_{\nu_\tau} &\propto 91.125 \cdot \frac{L}{E} \cdot \frac{1}{T_x}.
		\end{align}
		
		These phase differences could cause oscillations between flavor states without requiring different masses. The exact form of the oscillation probability requires further development but remains highly speculative.
		
		\textbf{WARNING:} This approach is purely hypothetical and lacks empirical confirmation. It contradicts the established theory that oscillations are caused by \(\Delta m^2_{ij} \neq 0\).
% end box speculation
	
	\section{Fundamental Constants and Units}
	
	\subsection{Base Parameters}
	
\section*{Formula}
\section*{T0 Base Constants:}
		\begin{align}
			\xipar &= \frac{4}{3} \times 10^{-4} \approx 1.333333 \times 10^{-4} \quad \text{[dimensionless]} \\
			\frac{\xipar^2}{2} &= \frac{\left(\frac{4}{3} \times 10^{-4}\right)^2}{2} \approx 8.888888 \times 10^{-9} \quad \text{[dimensionless]} \\
			v &= 246.22 \text{ GeV} \quad \text{[Higgs VEV]} \\
			\hbar c &= 0.19733 \text{ GeV·fm} \quad \text{[Conversion constant]} \\
			T_x &= \frac{1}{4.54 \times 10^{-3} \text{ eV}} \approx 2.2026 \times 10^2 \text{ eV}^{-1} \approx 1.449 \times 10^{-13} \text{ s} \quad \text{[T0 Mass]}
		\end{align}
% end box formula
	
	\subsection{Unit Conventions}
	
\section*{Important}
\section*{Consistent Unit Hierarchy:}
		\begin{align}
			\text{Standard:} &\quad \text{GeV} \\
			\text{Submultiples:} &\quad 1 \text{ eV} = 10^{-9} \text{ GeV} \\
			&\quad 1 \text{ meV} = 10^{-12} \text{ GeV} = 10^{-3} \text{ eV} \\
			\text{Masses:} &\quad m[\text{GeV}/c^2] = E[\text{GeV}]/c^2 \approx E[\text{GeV}] \text{ (natural units)} \\
			\text{Time:} &\quad 1 \text{ eV}^{-1} \approx 6.582 \times 10^{-16} \text{ s}
		\end{align}
% end box important
	
	\section{Charged Lepton Reference Masses}
	
	\subsection{Precise Experimental Values (PDG 2024)}
	
\section*{Experimental}
\section*{Verified Particle Masses:}
		\begin{align}
			m_e &= 0.51099895000 \times 10^{-3} \text{ GeV} = 510.99895 \text{ keV} \\
			m_\mu &= 105.6583745 \times 10^{-3} \text{ GeV} = 105.6583745 \text{ MeV} \\
			m_\tau &= 1776.86 \times 10^{-3} \text{ GeV} = 1.77686 \text{ GeV}
		\end{align}
		
\section*{Unit Conversion to eV:}
		\begin{align}
			m_e &= 510998.95 \text{ eV} = 510998950 \text{ meV} \\
			m_\mu &= 105658374.5 \text{ eV} \\
			m_\tau &= 1776860000 \text{ eV}
		\end{align}
% end box experimental
	
	\section{Neutrino Quantum Numbers (T0 Hypothesis)}
	
	\subsection{Postulated Quantum Number Assignment}
	
\section*{Speculation}
\section*{Hypothetical Neutrino Quantum Numbers:}
		\begin{align}
			\nu_e: &\quad n=1, \ell=0, j=1/2 \quad \text{[Ground state neutrino]} \\
			\nu_\mu: &\quad n=2, \ell=1, j=1/2 \quad \text{[First excitation]} \\
			\nu_\tau: &\quad n=3, \ell=2, j=1/2 \quad \text{[Second excitation]}
		\end{align}
		
\section*{Role of Quantum Numbers:}
		The quantum numbers do not affect neutrino masses (since \(m_{\nu_e} = m_{\nu_\mu} = m_{\nu_\tau}\)) but determine the geometric factors \(f(n, \ell, j)\), which govern the oscillation phases.
		
		\textbf{WARNING:} These assignments are purely speculative and lack experimental basis.
% end box speculation
	
	\subsection{Geometric Factors}
	
\section*{Formula}
\section*{T0 Geometric Factors:}
		\begin{align}
			f(n,\ell,j) &= \frac{n^6}{\ell^3} \quad \text{for } \ell > 0 \\
			f(1,0,j) &= 1 \quad \text{for } \ell = 0 \text{ (special case)}
		\end{align}
		
\section*{Calculated Values:}
		\begin{align}
			f_{\nu_e} &= f(1,0,1/2) = 1 \\
			f_{\nu_\mu} &= f(2,1,1/2) = \frac{2^6}{1^3} = 64 \\
			f_{\nu_\tau} &= f(3,2,1/2) = \frac{3^6}{2^3} = \frac{729}{8} = 91.125
		\end{align}
% end box formula
	
	\section{Neutrino Mass Formula}
	
	\subsection{T0 Hypothesis: Equal Masses with Geometric Phases}
	
\section*{Speculation}
\section*{T0 Hypothesis: Equal Neutrino Masses with Geometric Phases}
		
		The T0 model postulates that all flavor states (\(\nu_e, \nu_\mu, \nu_\tau\)) have the same mass:
		\[
		m_{\nu_e} = m_{\nu_\mu} = m_{\nu_\tau} = m_\nu = 4.54 \text{ meV}.
		\]
		The mass is derived from the photon analogy:
		\[
		m_\nu = \frac{\xipar^2}{2} \times m_e = \left(8.888888 \times 10^{-9}\right) \times (0.51099895 \times 10^{-3} \text{ GeV}) = 4.54 \text{ meV}.
		\]
		
		To explain oscillations, a geometric mechanism is postulated based on the T0 relation:
		\[
		T_x \cdot m_x = 1, \quad m_x = 4.54 \text{ meV}, \quad T_x \approx 2.2026 \times 10^2 \text{ eV}^{-1} \approx 1.449 \times 10^{-13} \text{ s}.
		\]
		
		The oscillation phases are determined by geometric factors \(f(n, \ell, j)\):
		\[
		\phi_{\text{geo}, i} \propto f_{\nu_i} \cdot \frac{L}{E} \cdot \frac{1}{T_x},
		\]
		where \(f_{\nu_e} = 1\), \(f_{\nu_\mu} = 64\), \(f_{\nu_\tau} = 91.125\).
		
\section*{Rationale:}
		\begin{itemize}
			\item The mass \(4.54 \text{ meV}\) is consistent with the cosmological constraint (\(\Sigma m_\nu = 0.01362 \text{ eV} < 0.07 \text{ eV}\)).
			\item Geometric phases enable oscillations without mass differences, supporting the equal-mass hypothesis.
			\item This hypothesis is highly speculative and lacks empirical confirmation.
		\end{itemize}
% end box speculation
	
\section*{Formula}
		\textbf{Formula:} \(m_{\nu_i} = 4.54 \text{ meV}\)
		
\section*{Total Mass:}
		\[
		\Sigma m_\nu = 3 \times 4.54 \text{ meV} = 13.62 \text{ meV} = 0.01362 \text{ eV}
		\]
		
\section*{Comparison with Plausible Target Value:}
		\begin{itemize}
			\item \(\nu_e, \nu_\mu, \nu_\tau\): \(4.54 \text{ meV}\) vs. \(15 \text{ meV}\) (Agreement: \(30.3\%\))
			\item \(\Sigma m_\nu\): \(13.62 \text{ meV}\) vs. \(45 \text{ meV}\) (Deviation: Factor \(\approx 3.30\))
		\end{itemize}
% end box formula
	
\section*{Warning}
		\textbf{CRITICAL FINDING:} The hypothesis of equal masses with geometric phases is incompatible with experimental oscillation data (\(\Delta m^2_{21} \approx 7.53 \times 10^{-5} \text{ eV}^2\), \(\Delta m^2_{32} \approx 2.44 \times 10^{-3} \text{ eV}^2\)), as it implies \(\Delta m^2_{ij} = 0\). The geometric approach is purely speculative and requires further theoretical and experimental validation.
% end box warning
	
	\section{Plausible Target Value Based on Empirical Data}
	
	\subsection{Derivation from Measurements}
	
\section*{Experimental}
\section*{Plausible Target Value:}
		The T0 model postulates equal masses for all flavor states (\(\nu_e, \nu_\mu, \nu_\tau\)). Thus, a single target value for the neutrino mass \(m_\nu\) is derived based on empirical data (as of 2025):
		\begin{itemize}
			\item Cosmological Constraint: \(\Sigma m_\nu = 3 m_\nu < 0.07 \text{ eV} \implies m_\nu < 23.33 \text{ meV}\).
			\item Oscillation Data: \(\Delta m^2_{21} \approx 7.53 \times 10^{-5} \text{ eV}^2\), \(\Delta m^2_{32} \approx 2.44 \times 10^{-3} \text{ eV}^2\), typically requiring different masses. The T0 model bypasses this via geometric phases.
			\item Plausible Target Value: \(m_\nu \approx 15 \text{ meV}\), lying between the solar (\(8.68 \text{ meV}\)) and atmospheric scales (\(50.15 \text{ meV}\)) and satisfying the cosmological constraint:
			\[
			\Sigma m_\nu = 3 \times 15 \text{ meV} = 45 \text{ meV} = 0.045 \text{ eV} < 0.07 \text{ eV}.
			\]
		\end{itemize}
		
\section*{Rationale:}
		\begin{itemize}
			\item The target value is consistent with the cosmological constraint and lies within the order of magnitude of oscillation data.
			\item The equal-mass hypothesis is supported by geometric phases, distinguishing the T0 model from standard physics.
			\item The value is plausible but not directly measured, as flavor masses are mixtures of eigenstates.
			\item The T0 mass (\(4.54 \text{ meV}\)) is below the target value (\(30.3\%\)) but also cosmologically consistent.
		\end{itemize}
% end box experimental
	
	\section{Experimental Comparison}
	
	\subsection{Current Experimental Upper Limits (2025)}
	
\section*{Experimental}
\section*{Experimental Limits:}
		\begin{align}
			m_{\nu_e} &< 0.45 \text{ eV} \quad \text{[KATRIN, 90\% CL]} \\
			m_{\nu_\mu} &< 0.17 \text{ MeV} \quad \text{[Muon decay, indirect]} \\
			m_{\nu_\tau} &< 18.2 \text{ MeV} \quad \text{[Tau decay, indirect]} \\
			\Sigma m_\nu &< 0.07 \text{ eV} \quad \text{[DESI+Planck, 95\% CL]} \\
			\Delta m^2_{21} &\approx 7.53 \times 10^{-5} \text{ eV}^2 \quad \text{[Solar]} \\
			\Delta m^2_{32} &\approx 2.44 \times 10^{-3} \text{ eV}^2 \quad \text{[Atmospheric]} \\
			m_\nu &> 0.06 \text{ eV} \quad \text{[At least one neutrino, 3}\sigma\text{]}
		\end{align}
% end box experimental
	
	\subsection{Safety Margins for T0 Hypothesis}
	
	\begin{longtable}[c]{@{}lcc@{}}
		\caption{Safety Margins of the T0 Hypothesis Against Experimental Limits} \\
		\toprule
		\textbf{Parameter} & \textbf{T0 Mass (\(4.54 \text{ meV}\))} & \textbf{Target Value (\(15 \text{ meV}\))} \\
		\midrule
		\endfirsthead
		\toprule
		\textbf{Parameter} & \textbf{T0 Mass (\(4.54 \text{ meV}\))} & \textbf{Target Value (\(15 \text{ meV}\))} \\
		\midrule
		\endhead
		$m_{\nu_e}$ vs 0.45 eV & 99200× & 30× \\
		$m_{\nu_\mu}$ vs 0.17 MeV & 3.74E7× & 11333× \\
		$m_{\nu_\tau}$ vs 18.2 MeV & 4.01E9× & 1.21E6× \\
		\midrule
		$\Sigma m_\nu$ vs 0.07 eV & 5.14× & 1.56× \\
		$\Sigma m_\nu$ vs 0.06 eV & 4.41× & 1.33× \\
		\bottomrule
	\end{longtable}
	
\section*{Important}
\section*{T0 Hypothesis:}
		\begin{itemize}
			\item The T0 mass (\(4.54 \text{ meV}\)) is consistent with cosmological constraints (\(\Sigma m_\nu = 0.01362 \text{ eV} < 0.07 \text{ eV}\)) and lies below the target value (\(15 \text{ meV}\), \(30.3\%\)).
			\item Geometric phases (\(T_x \cdot m_x = 1\)) provide a speculative mechanism for oscillations but are incompatible with standard oscillations.
			\item Physical Rationale: The mass is based on \(\frac{\xipar^2}{2}\)-suppression, consistent with the speed difference \(v_\nu = c \times \left(1 - \frac{\xipar^2}{2}\right)\).
		\end{itemize}
% end box important
	
	\section{Consistency Checks and Validation}
	
	\subsection{Dimensional Analysis}
	
\section*{Formula}
\section*{Dimensional Consistency:}
		\begin{align}
			[\xipar] &= 1 \quad \checkmark \text{ dimensionless} \\
			[m_e] &= \text{GeV} \quad \checkmark \text{ energy/mass} \\
			\left[\frac{\xipar^2}{2} \times m_e\right] &= \text{GeV} \quad \checkmark \text{ energy/mass} \\
			[f_{\nu_i}] &= 1 \quad \checkmark \text{ dimensionless} \\
			[m_\nu] &= \text{eV} \quad \checkmark \text{ (fixed mass)} \\
			[T_x] &= \text{eV}^{-1} \quad \checkmark \text{ (time)}
		\end{align}
		All formulas are dimensionally consistent.
% end box formula
	
	\subsection{Mathematical Consistency}
	
\section*{Important}
\section*{Consistency of the Hypothesis:}
		\begin{itemize}
			\item The formula \(m_\nu = \frac{\xipar^2}{2} \times m_e = 4.54 \text{ meV}\) is physically grounded in the photon analogy and consistent with the speed difference.
			\item Geometric phases based on \(f(n, \ell, j)\) and \(T_x \cdot m_x = 1\) provide a speculative mechanism for oscillations.
			\item No free parameters except \(\xipar\), simplifying the theory.
		\end{itemize}
% end box important
	
	\subsection{Experimental Validation}
	
\section*{Experimental}
\section*{Validation Status (as of 2025):}
		\begin{itemize}
			\item The T0 mass (\(4.54 \text{ meV}\)) satisfies cosmological constraints (\(\Sigma m_\nu = 0.01362 \text{ eV} < 0.07 \text{ eV}\)) and is close to the target value (\(15 \text{ meV}\), \(30.3\%\)).
			\item Incompatible with standard oscillations (\(\Delta m^2_{ij} = 0\)), but geometric phases offer a speculative workaround.
			\item The target value (\(15 \text{ meV}\)) is consistent with cosmological constraints but not directly measured.
		\end{itemize}
% end box experimental
	
	\section{Conclusion}
	
\section*{Important}
\section*{Summary and Outlook:}
		\begin{itemize}
			\item The T0 model postulates equal neutrino masses (\(m_\nu = 4.54 \text{ meV}\)) based on the photon analogy (\(\frac{\xipar^2}{2} \times m_e\)), consistent with the speed difference (\(v_\nu = c \times \left(1 - \frac{\xipar^2}{2}\right)\)).
			\item Geometric phases based on \(T_x \cdot m_x = 1\) and quantum numbers (\(f_{\nu_e} = 1\), \(f_{\nu_\mu} = 64\), \(f_{\nu_\tau} = 91.125\)) speculatively explain oscillations without mass differences.
			\item The plausible target value (\(m_\nu = 15 \text{ meV}\)) is derived from empirical data (cosmological constraint) and lies within the order of magnitude of oscillation data but is not directly measured.
			\item The T0 mass (\(4.54 \text{ meV}\)) is reasonably close to the target value (\(30.3\%\)), satisfies cosmological constraints, but is incompatible with standard oscillations.
			\item The T0 model remains speculative, relying on geometric harmonies without empirical basis.
			\item Future experiments (2025–2030, e.g., KATRIN upgrade, DESI, Euclid) could further test or refute the T0 hypothesis, particularly the geometric oscillation mechanism.
			\item Scientific integrity requires clearly communicating the speculative nature of the T0 model and awaiting further tests.
		\end{itemize}
% end box important
	


% Bibliography
\begin{thebibliography}{99}
	
	\bibitem{pdg2024}
	Particle Data Group Collaboration (2024). 
	\textit{Review of Particle Physics}. 
	Progress of Theoretical and Experimental Physics, 2024(8), 083C01.
	\url{https://pdg.lbl.gov}
	
	\bibitem{flag2024}
	Aoki, Y., et al. (FLAG Collaboration) (2024). 
	\textit{FLAG Review 2024 of Lattice Results for Low-Energy Constants}. 
	arXiv:2411.04268.
	\url{https://arxiv.org/abs/2411.04268}
	
	\bibitem{fermilab_muon_g2}
	Abi, B., et al. (Muon g-2 Collaboration) (2021). 
	\textit{Measurement of the Positive Muon Anomalous Magnetic Moment to 0.46 ppm}. 
	Physical Review Letters, 126, 141801.
	
	\bibitem{peskin_schroeder}
	Peskin, M. E., \& Schroeder, D. V. (1995). 
	\textit{An Introduction to Quantum Field Theory}. 
	Addison-Wesley.
	
	\bibitem{weinberg_qft}
	Weinberg, S. (1995). 
	\textit{The Quantum Theory of Fields, Vol. I--III}. 
	Cambridge University Press.
	
	\bibitem{griffiths_particle}
	Griffiths, D. (2008). 
	\textit{Introduction to Elementary Particles}. 
	Wiley-VCH.
	
	\bibitem{mandl_shaw}
	Mandl, F., \& Shaw, G. (2010). 
	\textit{Quantum Field Theory (2nd ed.)}. 
	Wiley.
	
	\bibitem{srednicki_qft}
	Srednicki, M. (2007). 
	\textit{Quantum Field Theory}. 
	Cambridge University Press.
	
	\bibitem{t0_fundamentals}
	Pascher, J. (2024). 
	\textit{T0-Theory: Foundations of Time-Mass Duality}. 
	Unpublished manuscript, HTL Leonding.
	
	\bibitem{t0_fine_structure}
	Pascher, J. (2024). 
	\textit{T0-Theory: The Fine Structure Constant}. 
	Unpublished manuscript, HTL Leonding.
	
	\bibitem{t0_neutrinos}
	Pascher, J. (2024). 
	\textit{T0-Theory: Neutrino Masses and PMNS Mixing}. 
	Unpublished manuscript, HTL Leonding.
	
	\bibitem{t0_github}
	Pascher, J. (2024--2025). 
	\textit{T0-Time-Mass-Duality Repository}. 
	GitHub.
	\url{https://github.com/jpascher/T0-Time-Mass-Duality}
	
	\bibitem{lattice_qcd_review}
	Kronfeld, A. S. (2012). 
	\textit{Twenty-first Century Lattice Gauge Theory: Results from the QCD Lagrangian}. 
	Annual Review of Nuclear and Particle Science, 62, 265--284.
	
	\bibitem{neutrino_mixing_pdg}
	Particle Data Group Collaboration (2024). 
	\textit{Neutrino Masses, Mixing, and Oscillations}. 
	PDG Review 2024.
	\url{https://pdg.lbl.gov/2024/reviews/rpp2024-rev-neutrino-mixing.pdf}
	
	\bibitem{higgs_discovery}
	ATLAS and CMS Collaborations (2012). 
	\textit{Observation of a New Particle in the Search for the Standard Model Higgs Boson}. 
	Physics Letters B, 716, 1--29.
	
	\bibitem{Brannen2005}
	C. P. Brannen, ``Estimate of neutrino masses from Koide's relation'', \textit{arXiv:hep-ph/0505028} (2005).
	\url{https://arxiv.org/abs/hep-ph/0505028}
	
	\bibitem{Brannen2006}
	C. P. Brannen, ``Koide Mass Formula for Neutrinos'', \textit{arXiv:0702.0052} (2006).
	\url{http://brannenworks.com/MASSES.pdf}
	
	\bibitem{PhaseVectors2025}
	Anonymous, ``The Koide Relation and Lepton Mass Hierarchy from Phase Vectors'', \textit{rXiv:2507.0040} (2025).
	\url{https://rxiv.org/pdf/2507.0040v1.pdf}
	
	\bibitem{PDG2025}
	Particle Data Group, ``Review of Particle Physics'', \textit{Phys. Rev. D} \textbf{112} (2025) 030001.
	\url{https://pdg.lbl.gov/2025/}
	
	\bibitem{terrell2024}
	Terrell et al. (2024). 
	\textit{Single-Clock Metrology in Nature}. 
	Nature Physics.
	
	\bibitem{hossenfelder2024}
	Hossenfelder, S. (2024). 
	\textit{Single Clock Video Explanation}. 
	YouTube.
	
	\bibitem{hundert1931}
	Hundert (1931). 
	\textit{Reference Work}. 
	Publisher.
	
	\bibitem{terrell2025}
	Terrell et al. (2025). 
	\textit{Advanced Clock Synchronization Methods}. 
	Physical Review Letters.
	
	\bibitem{pascher_t0_2025}
	Pascher, J. (2025). 
	\textit{T0-Theory: Complete Framework and Applications}. 
	Unpublished manuscript, HTL Leonding.
	
	\bibitem{t0qm}
	Pascher, J. (2024). 
	\textit{T0-Theory: Quantum Mechanics Formulation}. 
	Unpublished manuscript, HTL Leonding.
	
	\bibitem{t0anomale}
	Pascher, J. (2024). 
	\textit{T0-Theory: Anomalous Magnetic Moments}. 
	Unpublished manuscript, HTL Leonding.
	
	\bibitem{muong2complete}
	Abi, B., et al. (Muon g-2 Collaboration) (2023). 
	\textit{Complete Measurement of the Positive Muon Anomalous Magnetic Moment}. 
	Physical Review Letters, 131, 161802.
	
	\bibitem{penrose2004}
	Penrose, R. (2004). 
	\textit{The Road to Reality: A Complete Guide to the Laws of the Universe}. 
	Jonathan Cape.
	
	\bibitem{planck1900}
	Planck, M. (1900). 
	\textit{On the Theory of the Energy Distribution Law of the Normal Spectrum}. 
	Verhandlungen der Deutschen Physikalischen Gesellschaft, 2, 237.
	
	\bibitem{T0Theory}
	Pascher, J. (2024). 
	\textit{T0-Theory: Fundamental Principles}. 
	Unpublished manuscript, HTL Leonding.
	
	% Additional bibliography entries for all undefined citations
	\bibitem{6g_roadmap}
	6G Research Consortium (2024).
	\textit{6G Technology Roadmap}.
	Technical Report.
	
	\bibitem{Born2013}
	Born, M. (2013).
	\textit{Einstein's Theory of Relativity}.
	Dover Publications.
	
	\bibitem{Casimir1948}
	Casimir, H. B. G. (1948).
	\textit{On the attraction between two perfectly conducting plates}.
	Proc. Kon. Ned. Akad. Wetensch. B51, 793--795.
	
	\bibitem{Einstein1905}
	Einstein, A. (1905).
	\textit{On the Electrodynamics of Moving Bodies}.
	Annalen der Physik, 17, 891--921.
	
	\bibitem{Feynman2006}
	Feynman, R. P. (2006).
	\textit{QED: The Strange Theory of Light and Matter}.
	Princeton University Press.
	
	\bibitem{Griffiths2017}
	Griffiths, D. J. (2017).
	\textit{Introduction to Electrodynamics (4th ed.)}.
	Cambridge University Press.
	
	\bibitem{Jackson1999}
	Jackson, J. D. (1999).
	\textit{Classical Electrodynamics (3rd ed.)}.
	Wiley.
	
	\bibitem{Mohr2016}
	Mohr, P. J., et al. (2016).
	\textit{CODATA Recommended Values of the Fundamental Physical Constants: 2014}.
	Rev. Mod. Phys. 88, 035009.
	
	\bibitem{Parker2018}
	Parker, R. H., et al. (2018).
	\textit{Measurement of the fine-structure constant as a test of the Standard Model}.
	Science, 360, 191--195.
	
	\bibitem{Planck1900}
	Planck, M. (1900).
	\textit{On the Theory of the Energy Distribution Law of the Normal Spectrum}.
	Verhandlungen der Deutschen Physikalischen Gesellschaft, 2, 237.
	
	\bibitem{Planck2018}
	Planck Collaboration (2018).
	\textit{Planck 2018 results. VI. Cosmological parameters}.
	Astronomy \& Astrophysics, 641, A6.
	
	\bibitem{QFT_T0}
	Pascher, J. (2024).
	\textit{T0-Theory and QFT Connections}.
	Unpublished manuscript, HTL Leonding.
	
	\bibitem{Sommerfeld1916}
	Sommerfeld, A. (1916).
	\textit{On the Quantum Theory of Spectral Lines}.
	Annalen der Physik, 51, 1--94.
	
	\bibitem{T0_Feinstruktur}
	Pascher, J. (2024).
	\textit{T0-Theory: Fine Structure Analysis}.
	Unpublished manuscript, HTL Leonding.
	
	\bibitem{T0_SI}
	Pascher, J. (2024).
	\textit{T0-Theory and SI Units}.
	Unpublished manuscript, HTL Leonding.
	
	\bibitem{T0_fine_structure}
	Pascher, J. (2024).
	\textit{T0-Theory: The Fine Structure Constant}.
	Unpublished manuscript, HTL Leonding.
	
	\bibitem{T0_g2_erweiterung}
	Pascher, J. (2024).
	\textit{T0-Theory: g-2 Extensions}.
	Unpublished manuscript, HTL Leonding.
	
	\bibitem{T0_gravitational_constant}
	Pascher, J. (2024).
	\textit{T0-Theory: Gravitational Constant Derivation}.
	Unpublished manuscript, HTL Leonding.
	
	\bibitem{T0_netze_en}
	Pascher, J. (2024).
	\textit{T0-Theory: Network Structures}.
	Unpublished manuscript, HTL Leonding.
	
	\bibitem{T0_tm_erweiterung}
	Pascher, J. (2024).
	\textit{T0-Theory: Time-Mass Extensions}.
	Unpublished manuscript, HTL Leonding.
	
	\bibitem{Uzan2003}
	Uzan, J.-P. (2003).
	\textit{The fundamental constants and their variation}.
	Rev. Mod. Phys. 75, 403--455.
	
	\bibitem{Weinberg1995}
	Weinberg, S. (1995).
	\textit{The Quantum Theory of Fields, Vol. I}.
	Cambridge University Press.
	
	\bibitem{albrecht1999}
	Albrecht, A. \& Magueijo, J. (1999).
	\textit{A time varying speed of light as a solution to cosmological puzzles}.
	Phys. Rev. D 59, 043516.
	
	\bibitem{alice2023}
	ALICE Collaboration (2023).
	\textit{Recent results from ALICE}.
	CERN-EP-2023-XXX.
	
	\bibitem{analog_optical}
	Smith, J. et al. (2024).
	\textit{Analog optical computing systems}.
	Nature Photonics.
	
	\bibitem{ashtekar2004}
	Ashtekar, A. \& Lewandowski, J. (2004).
	\textit{Background independent quantum gravity}.
	Class. Quantum Grav. 21, R53.
	
	\bibitem{atlas2023}
	ATLAS Collaboration (2023).
	\textit{ATLAS physics results}.
	CERN-PH-EP-2023-XXX.
	
	\bibitem{atlas2023higgs}
	ATLAS Collaboration (2023).
	\textit{Higgs boson measurements}.
	Phys. Rev. Lett.
	
	\bibitem{barbour1999}
	Barbour, J. (1999).
	\textit{The End of Time}.
	Oxford University Press.
	
	\bibitem{barrow1999}
	Barrow, J. D. (1999).
	\textit{Cosmologies with varying light speed}.
	Phys. Rev. D 59, 043515.
	
	\bibitem{becker2007}
	Becker, K. et al. (2007).
	\textit{String Theory and M-Theory}.
	Cambridge University Press.
	
	\bibitem{bell_muon}
	Bennett, G. W., et al. (Muon g-2 Collaboration) (2006).
	\textit{Final report of the E821 muon anomalous magnetic moment measurement}.
	Phys. Rev. D 73, 072003.
	
	\bibitem{bondi1948}
	Bondi, H. \& Gold, T. (1948).
	\textit{The steady-state theory of the expanding universe}.
	Mon. Not. R. Astron. Soc. 108, 252--270.
	
	\bibitem{brewer2019}
	Brewer, S. M. et al. (2019).
	\textit{Al+ Quantum-Logic Clock with Systematic Uncertainty below $10^{-18}$}.
	Phys. Rev. Lett. 123, 033201.
	
	\bibitem{cms2023top}
	CMS Collaboration (2023).
	\textit{Top quark measurements at CMS}.
	JHEP 2023.
	
	\bibitem{cms2024}
	CMS Collaboration (2024).
	\textit{CMS physics results 2024}.
	CERN-PH-EP-2024-XXX.
	
	\bibitem{codata2019}
	Tiesinga, E. et al. (2019).
	\textit{The 2018 CODATA Recommended Values}.
	J. Phys. Chem. Ref. Data.
	
	\bibitem{desi2025}
	DESI Collaboration (2025).
	\textit{DESI 2025 Cosmology Results}.
	arXiv preprint.
	
	\bibitem{differential_optical}
	Wang, X. et al. (2024).
	\textit{Differential optical computing}.
	Optica.
	
	\bibitem{dingle1972}
	Dingle, H. (1972).
	\textit{Science at the Crossroads}.
	Martin Brian \& O'Keeffe.
	
	\bibitem{divalentino2021}
	Di Valentino, E. et al. (2021).
	\textit{In the realm of the Hubble tension}.
	Class. Quantum Grav. 38, 153001.
	
	\bibitem{elnaschie2004}
	El Naschie, M. S. (2004).
	\textit{A review of E infinity theory}.
	Chaos, Solitons \& Fractals, 19, 209--236.
	
	\bibitem{fabrication_heterogeneous}
	Chen, Y. et al. (2024).
	\textit{Heterogeneous photonic integration}.
	Nature Electronics.
	
	\bibitem{fermilab2023}
	Fermilab (2023).
	\textit{Muon g-2 results}.
	Phys. Rev. Lett.
	
	\bibitem{flexible_wafer}
	Kim, S. et al. (2024).
	\textit{Flexible wafer-scale photonics}.
	Science Advances.
	
	\bibitem{francesco1997}
	Di Francesco, P. et al. (1997).
	\textit{Conformal Field Theory}.
	Springer.
	
	\bibitem{hartree1957}
	Hartree, D. R. (1957).
	\textit{The Calculation of Atomic Structures}.
	Wiley.
	
	\bibitem{hhi_6g}
	Fraunhofer HHI (2024).
	\textit{6G Photonic Integration}.
	Technical Report.
	
	\bibitem{hossenfelder2025}
	Hossenfelder, S. (2025).
	\textit{Science without the gobbledygook}.
	YouTube/Blog.
	
	\bibitem{hossenfelder_single_clock_video}
	Hossenfelder, S. (2024).
	\textit{The Single Clock Problem}.
	YouTube.
	
	\bibitem{hoyle1948}
	Hoyle, F. (1948).
	\textit{A new model for the expanding universe}.
	Mon. Not. R. Astron. Soc. 108, 372--382.
	
	\bibitem{integration_microelectronic}
	Liu, A. et al. (2024).
	\textit{Microelectronic photonic integration}.
	IEEE Journal.
	
	\bibitem{jacobson1995}
	Jacobson, T. (1995).
	\textit{Thermodynamics of spacetime}.
	Phys. Rev. Lett. 75, 1260.
	
	\bibitem{kasevich2023}
	Kasevich, M. et al. (2023).
	\textit{Atom interferometry tests}.
	Nature Physics.
	
	\bibitem{lerner2014}
	Lerner, E. J. (2014).
	\textit{An open letter on cosmology}.
	New Scientist.
	
	\bibitem{lisa2017}
	LISA Consortium (2017).
	\textit{Laser Interferometer Space Antenna}.
	ESA Technical Report.
	
	\bibitem{lithium_tantalate}
	Zhang, M. et al. (2024).
	\textit{Thin-film lithium tantalate photonics}.
	Nature Photonics.
	
	\bibitem{lopez2010}
	Lopez-Corredoira, M. (2010).
	\textit{Tests and problems of the standard model in cosmology}.
	Int. J. Mod. Phys. D.
	
	\bibitem{ludlow2015}
	Ludlow, A. D. et al. (2015).
	\textit{Optical atomic clocks}.
	Rev. Mod. Phys. 87, 637.
	
	\bibitem{mach1883}
	Mach, E. (1883).
	\textit{Die Mechanik in ihrer Entwickelung}.
	F.A. Brockhaus.
	
	\bibitem{maldacena1998}
	Maldacena, J. (1998).
	\textit{The large N limit of superconformal field theories}.
	Adv. Theor. Math. Phys. 2, 231--252.
	
	\bibitem{mueller2014}
	Müller, H. et al. (2014).
	\textit{Atom interferometry tests of the gravitational redshift}.
	Phys. Rev. Lett.
	
	\bibitem{mug2_final_2025}
	Muon g-2 Collaboration (2025).
	\textit{Final muon g-2 measurement}.
	Phys. Rev. Lett.
	
	\bibitem{muong2_2023}
	Muon g-2 Collaboration (2023).
	\textit{Updated muon g-2 results}.
	Phys. Rev. Lett.
	
	\bibitem{nathan2024}
	Nathan, A. et al. (2024).
	\textit{Quantum computing advances}.
	Nature.
	
	\bibitem{newell2018}
	Newell, D. B. et al. (2018).
	\textit{The CODATA 2017 values of h, e, k, and $N_A$}.
	Metrologia 55, L13.
	
	\bibitem{nottale1993}
	Nottale, L. (1993).
	\textit{Fractal Space-Time and Microphysics}.
	World Scientific.
	
	\bibitem{on_chip_lithium}
	Wang, C. et al. (2024).
	\textit{On-chip lithium niobate photonics}.
	Nature Communications.
	
	\bibitem{optical_advantages}
	Shastri, B. J. et al. (2024).
	\textit{Advantages of optical computing}.
	Nature Reviews Physics.
	
	\bibitem{pascher2025cmb}
	Pascher, J. (2025).
	\textit{T0-Theory: CMB Analysis}.
	Unpublished manuscript, HTL Leonding.
	
	\bibitem{pascher2025g2}
	Pascher, J. (2025).
	\textit{T0-Theory: g-2 Predictions}.
	Unpublished manuscript, HTL Leonding.
	
	\bibitem{pascher2025qm}
	Pascher, J. (2025).
	\textit{T0-Theory: Quantum Mechanics}.
	Unpublished manuscript, HTL Leonding.
	
	\bibitem{pascher2025si}
	Pascher, J. (2025).
	\textit{T0-Theory: SI Unit System}.
	Unpublished manuscript, HTL Leonding.
	
	\bibitem{pascher2025t0}
	Pascher, J. (2025).
	\textit{T0-Theory: Complete Framework}.
	Unpublished manuscript, HTL Leonding.
	
	\bibitem{pascher:fundamentals}
	Pascher, J. (2024).
	\textit{T0-Theory: Fundamentals}.
	Unpublished manuscript, HTL Leonding.
	
	\bibitem{pascher:g2_rev9}
	Pascher, J. (2024).
	\textit{T0-Theory: g-2 Revision 9}.
	Unpublished manuscript, HTL Leonding.
	
	\bibitem{pascher:geometric_formalism}
	Pascher, J. (2024).
	\textit{T0-Theory: Geometric Formalism}.
	Unpublished manuscript, HTL Leonding.
	
	\bibitem{pascher:ml_addendum}
	Pascher, J. (2024).
	\textit{T0-Theory: Machine Learning Addendum}.
	Unpublished manuscript, HTL Leonding.
	
	\bibitem{pascher:t0_foundations}
	Pascher, J. (2024).
	\textit{T0-Theory: Foundations}.
	Unpublished manuscript, HTL Leonding.
	
	\bibitem{pascher_derivation_beta_2025}
	Pascher, J. (2025).
	\textit{T0-Theory: Derivation of Beta}.
	Unpublished manuscript, HTL Leonding.
	
	\bibitem{pascher_higgs_connection_2025}
	Pascher, J. (2025).
	\textit{T0-Theory: Higgs Connection}.
	Unpublished manuscript, HTL Leonding.
	
	\bibitem{pascher_lagrangian_extended_2025}
	Pascher, J. (2025).
	\textit{T0-Theory: Extended Lagrangian}.
	Unpublished manuscript, HTL Leonding.
	
	\bibitem{pascher_mathematical_structure_2025}
	Pascher, J. (2025).
	\textit{T0-Theory: Mathematical Structure}.
	Unpublished manuscript, HTL Leonding.
	
	\bibitem{pascher_t0_cmb_2025}
	Pascher, J. (2025).
	\textit{T0-Theory: CMB Predictions}.
	Unpublished manuscript, HTL Leonding.
	
	\bibitem{pascher_t0_energie_2025}
	Pascher, J. (2025).
	\textit{T0-Theory: Energy}.
	Unpublished manuscript, HTL Leonding.
	
	\bibitem{pascher_t0_energy_2025}
	Pascher, J. (2025).
	\textit{T0-Theory: Energy Framework}.
	Unpublished manuscript, HTL Leonding.
	
	\bibitem{pascher_t0_theory_2025}
	Pascher, J. (2025).
	\textit{T0-Theory: Complete Theory}.
	Unpublished manuscript, HTL Leonding.
	
	\bibitem{penrose1959}
	Penrose, R. (1959).
	\textit{The apparent shape of a relativistically moving sphere}.
	Proc. Cambridge Phil. Soc. 55, 137--139.
	
	\bibitem{penrose1967}
	Penrose, R. (1967).
	\textit{Twistor algebra}.
	J. Math. Phys. 8, 345--366.
	
	\bibitem{peratt1992}
	Peratt, A. L. (1992).
	\textit{Physics of the Plasma Universe}.
	Springer-Verlag.
	
	\bibitem{peskin1995}
	Peskin, M. E. \& Schroeder, D. V. (1995).
	\textit{An Introduction to Quantum Field Theory}.
	Addison-Wesley.
	
	\bibitem{peskin_schroeder_1995}
	Peskin, M. E. \& Schroeder, D. V. (1995).
	\textit{An Introduction to Quantum Field Theory}.
	Addison-Wesley.
	
	\bibitem{phoquant}
	PhoQuant (2024).
	\textit{Photonic quantum computing}.
	Technical Report.
	
	\bibitem{photonics_ai}
	Wetzstein, G. et al. (2024).
	\textit{Photonics for AI}.
	Nature.
	
	\bibitem{planck1906}
	Planck, M. (1906).
	\textit{The Theory of Heat Radiation}.
	Johann Ambrosius Barth.
	
	\bibitem{planck2018}
	Planck Collaboration (2018).
	\textit{Planck 2018 results}.
	A\&A 641, A6.
	
	\bibitem{polchinski1998}
	Polchinski, J. (1998).
	\textit{String Theory}.
	Cambridge University Press.
	
	\bibitem{qant_nps}
	QANT (2024).
	\textit{Quantum photonics systems}.
	Technical Report.
	
	\bibitem{quantenjahr25}
	Quantenjahr (2025).
	\textit{International Year of Quantum}.
	UNESCO.
	
	\bibitem{recurrent_photonics}
	Tait, A. N. et al. (2024).
	\textit{Recurrent photonic neural networks}.
	Optica.
	
	\bibitem{rf_photonics}
	Capmany, J. \& Novak, D. (2024).
	\textit{Microwave photonics}.
	Nature Photonics.
	
	\bibitem{riess2019}
	Riess, A. G. et al. (2019).
	\textit{Large Magellanic Cloud Cepheid Standards}.
	ApJ 876, 85.
	
	\bibitem{riess2022}
	Riess, A. G. et al. (2022).
	\textit{A Comprehensive Measurement of H0}.
	ApJ 934, L7.
	
	\bibitem{rovelli2004}
	Rovelli, C. (2004).
	\textit{Quantum Gravity}.
	Cambridge University Press.
	
	\bibitem{sciama1953}
	Sciama, D. W. (1953).
	\textit{On the origin of inertia}.
	Mon. Not. R. Astron. Soc. 113, 34--42.
	
	\bibitem{sciencedaily2025}
	ScienceDaily (2025).
	\textit{Physics news}.
	Online.
	
	\bibitem{sm_g2_2025}
	Aoyama, T. et al. (2025).
	\textit{Standard Model prediction for g-2}.
	Phys. Rep.
	
	\bibitem{susskind1995}
	Susskind, L. (1995).
	\textit{The world as a hologram}.
	J. Math. Phys. 36, 6377--6396.
	
	\bibitem{t0_kosmologie}
	Pascher, J. (2024).
	\textit{T0-Theory: Cosmology}.
	Unpublished manuscript, HTL Leonding.
	
	\bibitem{terrell1959}
	Terrell, J. (1959).
	\textit{Invisibility of the Lorentz contraction}.
	Phys. Rev. 116, 1041--1045.
	
	\bibitem{terrell_single_clock_nature_2024}
	Terrell, J. et al. (2024).
	\textit{Single clock precision measurements}.
	Nature Physics.
	
	\bibitem{tfln_foundry}
	TFLN Foundry (2024).
	\textit{Thin-film lithium niobate foundry services}.
	Technical Specifications.
	
	\bibitem{thiemann2007}
	Thiemann, T. (2007).
	\textit{Modern Canonical Quantum General Relativity}.
	Cambridge University Press.
	
	\bibitem{thz_epfl}
	EPFL (2024).
	\textit{Terahertz photonics research}.
	Technical Report.
	
	\bibitem{unnikrishnan2004}
	Unnikrishnan, C. S. (2004).
	\textit{On Einstein's resolution of the twin clock paradox}.
	Current Science, 86, 704--709.
	
	\bibitem{verlinde2011}
	Verlinde, E. (2011).
	\textit{On the origin of gravity and the laws of Newton}.
	JHEP 2011, 29.
	
	\bibitem{video2025}
	Video (2025).
	\textit{Physics video explanation}.
	YouTube.
	
	\bibitem{weinberg1995}
	Weinberg, S. (1995).
	\textit{The Quantum Theory of Fields}.
	Cambridge University Press.
	
	\bibitem{weiskopf2000}
	Weiskopf, D. (2000).
	\textit{Visualization of special relativity}.
	PhD thesis, University of Tübingen.
	
	\bibitem{wheeler1990}
	Wheeler, J. A. (1990).
	\textit{A Journey into Gravity and Spacetime}.
	Scientific American Library.
	
	\bibitem{wiki_bell}
	Wikipedia (2024).
	\textit{Bell's theorem}.
	Online encyclopedia.
	
	\bibitem{zwicky1929}
	Zwicky, F. (1929).
	\textit{On the red shift of spectral lines through interstellar space}.
	Proc. Natl. Acad. Sci. 15, 773--779.

\end{thebibliography}


\end{document}
