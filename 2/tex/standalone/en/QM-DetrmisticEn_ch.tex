\documentclass[11pt,a4paper]{article}
\usepackage[a4paper,margin=2cm]{geometry}
\usepackage[utf8]{inputenc}
\usepackage[english]{babel}
\usepackage{lmodern}
\renewcommand{\familydefault}{\sfdefault}

\usepackage{amsmath,amssymb,amsthm}
\usepackage{graphicx}
\usepackage[unicode,pdfencoding=auto,hypertexnames=false]{hyperref}
\usepackage{booktabs}
\usepackage{longtable}
\usepackage{array}
\usepackage{siunitx}
\usepackage{fancyhdr}
\usepackage{float}
\usepackage{tikz}
% tcolorbox removed for standalone
% tcbset removed
\tikzset{
  t0blue/.style={draw=blue,fill=blue!10},
  t0red/.style={draw=red,fill=red!10},
  t0green/.style={draw=green!50!black,fill=green!10},
  t0orange/.style={draw=orange,fill=orange!10},
}
\usepackage{setspace}
\usepackage{enumitem}
\usepackage{adjustbox}
\usepackage{xcolor}

% Define colors for xcolor package
\definecolor{t0green}{RGB}{34,139,34}
\definecolor{t0blue}{RGB}{0,0,255}
\definecolor{t0red}{RGB}{255,0,0}
\definecolor{t0orange}{RGB}{255,165,0}

% Define custom column types for tables
\newcolumntype{L}[1]{>{\raggedright\arraybackslash}p{#1}}
\newcolumntype{C}[1]{>{\centering\arraybackslash}p{#1}}
\newcolumntype{R}[1]{>{\raggedleft\arraybackslash}p{#1}}

\setlength{\parindent}{0pt}
\setlength{\parskip}{6pt}

\hypersetup{
  colorlinks=true,
  linkcolor=blue,
  citecolor=blue,
  urlcolor=blue
}
\pagestyle{fancy}
\setlength{\headheight}{28pt}

\newcommand{\checkmarkx}{\checkmark}
\newcommand{\warningx}{\textbf{!}}

% Makros aus Einzel-Dokumenten (Fallback-Definitionen)
\newcommand{\mytimes}{\times}
\newcommand{\myapprox}{\approx}
\newcommand{\mysim}{\sim}
\newcommand{\myomega}{\omega}
\newcommand{\mypi}{\pi}
\newcommand{\myrightarrow}{\rightarrow}
\newcommand{\mypropto}{\propto}
\newcommand{\deltafield}{\delta\phi}
\newcommand{\xipar}{\xi}
\newcommand{\xiT}{\xi}
\newcommand{\lambdah}{\lambda_h}

% Additional macros used in chapter files
\newcommand{\Kfrak}{K_{\text{frak}}}  % Fractal correction factor
\newcommand{\Dfrak}{D_f}              % Fractal dimension
\newcommand{\betapar}{\beta}          % T0 beta parameter
\newcommand{\alphapar}{\alpha}        % T0 alpha parameter
\newcommand{\Efield}{E}               % Energy field
% Note: checkmarkxa/warningxa are variants used in auto-generated chapter files
\newcommand{\checkmarkxa}{\checkmark}
\newcommand{\warningxa}{\textbf{!}}

% Additional T0-specific macros
\newcommand{\xigeom}{\xi_{\text{geom}}}  % Geometric xi
\newcommand{\lP}{\ell_P}                  % Planck length
\newcommand{\rzero}{r_0}                  % Characteristic radius
\newcommand{\xirat}{\xi_{\text{rat}}}     % Xi ratio
\newcommand{\tzero}{t_0}                  % Characteristic time
\newcommand{\natunits}{\text{(nat. units)}}  % Natural units annotation
\newcommand{\myRightarrow}{\Rightarrow}   % Arrow variant
\newcommand{\Lag}{\mathcal{L}}            % Lagrangian

% Physics macros used in chapter files
\newcommand{\CQCD}{C_{\text{QCD}}}        % QCD correction
\newcommand{\EP}{E_P}                     % Planck energy
\newcommand{\Ee}{E_e}                     % Electron energy
\newcommand{\Emu}{E_\mu}                  % Muon energy
\newcommand{\Exi}{E_\xi}                  % Xi energy
\newcommand{\Ezero}{E_0}                  % Characteristic energy
\newcommand{\Hubble}{H}                   % Hubble constant
\newcommand{\Kspec}{K_{\text{spec}}}      % Spectral correction
\newcommand{\Lambdat}{\Lambda_t}          % Time-related cosmological constant
\newcommand{\Leff}{\mathcal{L}_{\text{eff}}}  % Effective Lagrangian
\newcommand{\Lorentz}{\mathcal{L}}        % Lorentz symbol
\newcommand{\Lxi}{L_\xi}                  % Xi length
\newcommand{\Tfield}{T}                   % Time field
\newcommand{\Weyl}{W}                     % Weyl tensor/symbol
\newcommand{\alphaEMSI}{\alpha_{\text{EM,SI}}}  % EM alpha in SI
\newcommand{\alphaEMnat}{\alpha_{\text{EM,nat}}}  % EM alpha in natural units
\newcommand{\alphaem}{\alpha_{\text{em}}} % Electromagnetic alpha
\newcommand{\betaTSI}{\beta_{T,\text{SI}}}  % Beta in SI
\newcommand{\betaTnat}{\beta_{T,\text{nat}}}  % Beta in natural units
\newcommand{\deltam}{\delta m}            % Mass difference
\newcommand{\phiT}{\phi_T}                % T-field phi
\newcommand{\tP}{t_P}                     % Planck time
\newcommand{\rhoCMB}{\rho_{\text{CMB}}}   % CMB density
\newcommand{\rhoCasimir}{\rho_{\text{Casimir}}}  % Casimir density

% Table formatting
\usepackage{multirow}

% Additional physics macros
\newcommand{\Riem}{\mathcal{R}}           % Riemann tensor
\newcommand{\ZPinch}{Z_{\text{pinch}}}    % Z-pinch
\newcommand{\SynchPower}{P_{\text{synch}}} % Synchrotron power
\newcommand{\Rzero}{R_0}                  % Characteristic radius
\newcommand{\alphafine}{\alpha}           % Fine structure constant
\newcommand{\Etau}{E_\tau}                % Tau energy
\newcommand{\deltaE}{\delta E}            % Energy deviation
\newcommand{\EPlanck}{E_P}                % Planck energy
\newcommand{\pichar}{\pi}                 % Pi character
\newcommand{\alphaWSI}{\alpha_{W,\text{SI}}}  % Wien alpha in SI
\newcommand{\alphaWnat}{\alpha_{W,\text{nat}}}  % Wien alpha in natural units

% Einfache abstract-Umgebung für Kapitel:
\newenvironment{abstract}{%
  \begin{center}\bfseries Abstract\end{center}\small
}{\par}


\title{QM-DetrmisticEn}
\author{J. Pascher}
\date{\today}

\begin{document}
\maketitle

\section*{Qm Detrmisticen (QM-DetrmisticEn)}

	\begin{abstract}
		This work presents a revolutionary deterministic alternative to probability-based quantum mechanics through the T0-energy field formulation. Building upon the simplified Dirac equation, universal Lagrangian, and ratio-based physics of the T0 framework, we demonstrate how quantum mechanical phenomena emerge from deterministic energy field dynamics governed by the modified Schrodinger equation. Using the empirically determined parameter $\xipar = 4/3 \times 10^{-4}$, we provide quantitative predictions that preserve all experimentally verified results while eliminating fundamental interpretation problems.
	\end{abstract}
	
	
	\section{Introduction: The T0 Revolution Applied to Quantum Mechanics}
	
	\subsection{Building on T0 Foundations}
	
	This work represents the fourth stage of the theoretical T0 revolution:
	
	\textbf{Stage 1 - Simplified Dirac Equation}: Complex $4 \times 4$ matrices to simple field dynamics
	
	\textbf{Stage 2 - Universal Lagrangian}: More than 20 fields to one equation
	
	\textbf{Stage 3 - Ratio Physics}: Multiple parameters to energy scale ratios
	
	\textbf{Stage 4 - Deterministic QM}: Probability amplitudes to deterministic energy fields
	
	\subsection{The Quantum Mechanics Problem}
	
	Standard quantum mechanics suffers from fundamental conceptual problems:
	
	\subsubsection*{Standard QM Problems}
\textbf{Probability Foundation Problems}:
		\begin{itemize}
			\item Wave function: mysterious superposition
			\item Probabilities: only statistical predictions
			\item Collapse: non-unitary measurement process
			\item Interpretation: Copenhagen vs. Many-worlds vs. others
			\item Single measurements: unpredictable (fundamentally random)
		\end{itemize}

	
	\subsection{T0-Energy Field Solution}
	
	The T0 framework offers a complete solution through deterministic energy fields:
	
	\subsubsection*{T0 Deterministic Foundation}
\textbf{Deterministic Energy Field Physics}:
		\begin{itemize}
			\item Universal field: single energy field for all phenomena
			\item Modified Schrodinger equation with time-energy duality
			\item Empirical parameter: $\xipar = 4/3 \times 10^{-4}$ from muon anomaly
			\item Measurable deviations from standard QM
			\item Continuous evolution: no collapse, only field dynamics
			\item Single reality: no interpretation problems
		\end{itemize}

	
	\section{T0-Energy Field Foundations}
	
	\subsection{Modified Schrodinger Equation}
	
	From the T0 revolution, quantum mechanics is governed by:
	
	\begin{equation}
		\boxed{i \cdot T(x,t) \frac{\partial\psi}{\partial t} = H_0 \psi + V_{\mathrm{T0}} \psi}
		\label{QM_DetrmisticEn:L-T0_Energie-0188}
	\end{equation}
	
	where:
	\begin{align}
		H_0 &= -\frac{\hbar^2}{2m} \nabla^2 \\
		V_{\mathrm{T0}} &= \hbar^2 \cdot \delta E(x,t)
	\end{align}
	
	\subsection{Energy-Time Duality}
	
	The fundamental T0 relationship:
	
	\begin{equation}
		\boxed{T(x,t) \cdot E(x,t) = 1}
		\label{QM_DetrmisticEn:L-QM-DetrmisticEn-1177}
	\end{equation}
	
	\textbf{Dimensional verification}: $[T][E] = 1$ in natural units.
	
	\subsection{Empirical Parameter}
	
	Following precision measurements of the muon anomalous magnetic moment:
	
	\begin{equation}
		\boxed{\xipar = \frac{4}{3} \times 10^{-4} \approx 1.333 \times 10^{-4}}
		\label{QM_DetrmisticEn:L-QM-DetrmisticEn-1178}
	\end{equation}
	
	\section{From Probability Amplitudes to Energy Field Ratios}
	
	\subsection{Standard QM State Description}
	
	\textbf{Traditional approach}:
	\begin{equation}
		|\psi\rangle = \sum_i c_i |i\rangle \quad \text{with } P_i = |c_i|^2
	\end{equation}
	
	\textbf{Problems}: Mysterious superposition, only probability-based predictions.
	
	\subsection{T0-Energy Field State Description}
	
	\textbf{T0 field-theoretic approach}:
	\begin{equation}
		\boxed{\psi(x,t) = \sqrt{\frac{\delta E(x,t)}{E_0 V_0}} \cdot e^{i\phi(x,t)}}
		\label{QM_DetrmisticEn:L-QM-DetrmisticEn-1179}
	\end{equation}
	
	with probability density:
	\begin{equation}
		\boxed{|\psi(x,t)|^2 = \frac{\delta E(x,t)}{E_0 V_0}}
		\label{QM_DetrmisticEn:L-QM-DetrmisticEn-1180}
	\end{equation}
	
	\textbf{Advantages}: 
	\begin{itemize}
		\item Direct connection to measurable energy field density
		\item Deterministic field evolution through modified Schrodinger equation
		\item Preservation of probabilistic interpretation with T0 corrections
		\item Field-theoretic foundation for quantum mechanics
	\end{itemize}
	
	\section{Deterministic Spin Systems}
	
	\subsection{Spin-1/2 in T0 Formulation}
	
	\subsubsection{Standard QM Approach}
	
	\textbf{State}: Superposition of spin-up and spin-down
	
	\textbf{Expectation value}: Probability-based
	
	\subsubsection{T0-Energy Field Approach}
	
	\textbf{State}: Energy field configuration with separate fields for both spin states
	
	\textbf{T0-corrected expectation value}:
	\begin{equation}
		\boxed{\langle \sigma_z \rangle_{\mathrm{T0}} = \langle \sigma_z \rangle_{\mathrm{QM}} + \xipar \cdot \frac{\delta E(x,t)}{E_0}}
		\label{QM_DetrmisticEn:L-QM-DetrmisticEn-1181}
	\end{equation}
	
	\subsection{Quantitative Example}
	
	With the empirical parameter $\xipar = 4/3 \times 10^{-4}$:
	
	\textbf{T0 correction to expectation value}:
	\begin{equation}
		\langle \sigma_z \rangle_{\mathrm{T0}} = \langle \sigma_z \rangle_{\mathrm{QM}} + \frac{4}{3} \times 10^{-4} \times \delta\sigma_z
	\end{equation}
	
	\section{Deterministic Quantum Entanglement}
	
	\subsection{Standard QM Entanglement}
	
	\textbf{Bell state}: Antisymmetric superposition
	
	\textbf{Problem}: Non-local spooky action at a distance
	
	\subsection{T0-Energy Field Entanglement}
	
	\textbf{Entanglement as correlated energy field structure}:
	\begin{equation}
		\boxed{E_{12}(x_1, x_2, t) = E_1(x_1, t) + E_2(x_2, t) + E_{\mathrm{corr}}(x_1, x_2, t)}
	\end{equation}
	
	\textbf{Correlation energy field}:
	\begin{equation}
		\boxed{E_{\mathrm{corr}}(x_1, x_2, t) = \frac{\xipar}{|x_1 - x_2|} \cos(\phi_1(t) - \phi_2(t) - \pi)}
		\label{QM_DetrmisticEn:L-QM-DetrmisticEn-1182}
	\end{equation}
	
	\subsection{Modified Bell Inequality}
	
	The T0 model predicts a modified Bell inequality:
	
	\begin{equation}
		\boxed{|E(a,b) - E(a,c)| + |E(a',b) + E(a',c)| \leq 2 + \varepsilon_{\mathrm{T0}}}
	\end{equation}
	
	with the T0 term:
	\begin{equation}
		\boxed{\varepsilon_{\mathrm{T0}} = \xipar \cdot \frac{2\langle E \rangle \ell_P}{r_{12}}}
		\label{QM_DetrmisticEn:L-DynMassePhotonenNichtlokalEn-1167}
	\end{equation}
	
	\textbf{Numerical estimate}:
	For typical atomic systems with $r_{12} \sim 1$ m:
	\begin{equation}
		\varepsilon_{\mathrm{T0}} \approx 10^{-34}
	\end{equation}
	
	\section{Deterministic Quantum Computing}
	
	\subsection{Qubit Representation}
	
	\textbf{T0-energy field qubit}:
	\begin{equation}
		\boxed{\text{qubit}_{\mathrm{T0}} \equiv \{E_0(x,t), E_1(x,t)\}}
	\end{equation}
	
	with field-theoretic amplitudes:
	\begin{align}
		\alpha_{\mathrm{T0}} &= \sqrt{\frac{E_0}{E_0 + E_1}} \\
		\beta_{\mathrm{T0}} &= \sqrt{\frac{E_1}{E_0 + E_1}}
	\end{align}
	
	\subsection{Quantum Gates as Energy Field Operations}
	
	\subsubsection{Hadamard Gate}
	
	\textbf{Corrected T0 transformation}:
	\begin{align}
		H_{\mathrm{T0}}: \quad E_0 &\rightarrow \frac{E_0 + E_1}{\sqrt{2}} \\
		E_1 &\rightarrow \frac{E_0 - E_1}{\sqrt{2}}
	\end{align}
	
	\subsubsection{Controlled-NOT Gate}
	
	\textbf{T0 formulation}:
	\begin{equation}
		\text{CNOT}_{\mathrm{T0}}: E_{12} \rightarrow E_{12} + \xipar \cdot \Theta(E_1 - E_{\mathrm{threshold}}) \cdot \sigma_x E_2
	\end{equation}
	
	\subsection{Enhanced Quantum Algorithms}
	
	\textbf{Enhanced Grover Algorithm}:
	\begin{itemize}
		\item Standard iterations: $\sim \pi/(4\sqrt{N})$
		\item T0-enhanced: modification through energy field corrections
	\end{itemize}
	
	\section{Experimental Predictions and Tests}
	
	\subsection{Enhanced Single-Measurement Predictions}
	
	\textbf{Example - Enhanced spin measurement}:
	\begin{equation}
		\boxed{P(\uparrow) = P_{\mathrm{QM}}(\uparrow) \cdot \left(1 + \xipar \frac{E_{\uparrow}(x_{\mathrm{det}}, t) - \langle E \rangle}{E_0}\right)}
		\label{QM_DetrmisticEn:L-QM-DetrmisticEn-1183}
	\end{equation}
	
	\subsection{T0-Specific Experimental Signatures}
	
	\subsubsection{Modified Bell Tests}
	
	\textbf{Prediction}: Bell inequality violation modified by $\varepsilon_{\mathrm{T0}} \approx 10^{-34}$
	
	\subsubsection{Energy Field Spectroscopy}
	
	\textbf{Prediction}: 
	\begin{equation}
		\Delta E = \xipar \cdot E_n \cdot \frac{\langle \delta E \rangle}{E_0}
	\end{equation}
	
	\subsubsection{Phase Accumulation in Interferometry}
	
	\textbf{Prediction}:
	\begin{equation}
		\phi_{\mathrm{total}} = \phi_0 + \xipar \int_0^t \frac{E(x(t'), t')}{E_0} dt'
	\end{equation}
	
	\section{Resolution of Quantum Interpretation Problems}
	
	\subsection{Problems Addressed by T0 Formulation}
	
	\begin{table}[htbp]
		\centering
		\small
		\begin{tabular}{|p{4cm}|p{5cm}|p{6cm}|}
			\hline
			\textbf{QM Problem} & \textbf{Standard Approaches} & \textbf{T0 Solution} \\
			\hline
			Measurement problem & Copenhagen interpretation & Continuous field evolution \\
			\hline
			Schrodinger's cat & Superposition paradox & Definite field states \\
			\hline
			Many-worlds vs. Copenhagen & Multiple interpretations & Single reality \\
			\hline
			Wave-particle duality & Complementarity principle & Energy field patterns \\
			\hline
			Quantum jumps & Random transitions & Field-mediated transitions \\
			\hline
			Bell nonlocality & Spooky action at distance & Field correlations \\
			\hline
		\end{tabular}
		\caption{Problems addressed by T0 formulation}
	\end{table}
	
	\subsection{Enhanced Quantum Reality}
	
	\subsubsection*{T0-Enhanced Quantum Reality}
\textbf{Field-theoretic quantum mechanics with T0 corrections}:
		\begin{itemize}
			\item Energy fields as physical basis of wave functions
			\item Modified Schrodinger evolution with time-energy duality
			\item Measurements reveal field configurations with T0 modulations
			\item Continuous unitary evolution without collapse
			\item Small but measurable deviations from standard QM
			\item Empirically grounded through muon anomaly parameter
		\end{itemize}

	
	\section{Connection to Other T0 Developments}
	
	\subsection{Integration with Simplified Dirac Equation}
	
	The enhanced QM naturally connects with the simplified Dirac equation through the time-energy duality.
	
	\subsection{Integration with Universal Lagrangian}
	
	The universal Lagrangian describes:
	\begin{itemize}
		\item Classical field evolution
		\item Quantum field evolution with T0 corrections
		\item Relativistic field evolution
	\end{itemize}
	
	\section{Future Directions and Implications}
	
	\subsection{Experimental Verification Program}
	
	\textbf{Phase 1 - Precision Tests}:
	\begin{itemize}
		\item Ultra-high precision Bell inequality measurements
		\item Atomic spectroscopy with T0 corrections
		\item Quantum interferometry phase measurements
	\end{itemize}
	
	\textbf{Phase 2 - Technological Enhancement}:
	\begin{itemize}
		\item T0-corrected quantum computing architectures
		\item Enhanced quantum sensor protocols
		\item Field correlation-based quantum devices
	\end{itemize}
	
	\subsection{Philosophical Implications}
	
	\subsubsection*{Beyond Quantum Mysticism}
\textbf{T0-enhanced quantum mechanics provides}:
		\begin{itemize}
			\item Physical foundation through energy field theory
			\item Measurable deviations from pure randomness
			\item Field-theoretic explanation of quantum phenomena
			\item Empirical grounding through precision measurements
		\end{itemize}
		
		\textbf{While preserving}:
		\begin{itemize}
			\item All successful predictions of standard QM
			\item Experimental continuity with established results
			\item Mathematical rigor and consistency
		\end{itemize}

	
	\section{Conclusion: The Enhanced Quantum Revolution}
	
	\subsection{Revolutionary Achievements}
	
	The T0-enhanced quantum formulation has achieved:
	
	\begin{enumerate}
		\item \textbf{Physical foundation}: Energy fields as basis for quantum mechanics
		\item \textbf{Experimental consistency}: All standard QM predictions preserved
		\item \textbf{Measurable corrections}: T0-specific deviations for tests
		\item \textbf{T0 framework integration}: Consistent with other T0 developments
		\item \textbf{Empirical grounding}: Parameter from precision measurements
		\item \textbf{Enhanced predictive power}: New testable effects
	\end{enumerate}
	
	\subsection{Future Impact}
	
	\begin{equation}
		\boxed{\text{Enhanced QM} = \text{Standard QM} + \text{T0 Field Corrections}}
	\end{equation}
	
	The T0 revolution enhances quantum mechanics with field-theoretic foundations while preserving experimental success.
	
	


% Bibliography
\begin{thebibliography}{99}
	
	\bibitem{pdg2024}
	Particle Data Group Collaboration (2024). 
	\textit{Review of Particle Physics}. 
	Progress of Theoretical and Experimental Physics, 2024(8), 083C01.
	\url{https://pdg.lbl.gov}
	
	\bibitem{flag2024}
	Aoki, Y., et al. (FLAG Collaboration) (2024). 
	\textit{FLAG Review 2024 of Lattice Results for Low-Energy Constants}. 
	arXiv:2411.04268.
	\url{https://arxiv.org/abs/2411.04268}
	
	\bibitem{fermilab_muon_g2}
	Abi, B., et al. (Muon g-2 Collaboration) (2021). 
	\textit{Measurement of the Positive Muon Anomalous Magnetic Moment to 0.46 ppm}. 
	Physical Review Letters, 126, 141801.
	
	\bibitem{peskin_schroeder}
	Peskin, M. E., \& Schroeder, D. V. (1995). 
	\textit{An Introduction to Quantum Field Theory}. 
	Addison-Wesley.
	
	\bibitem{weinberg_qft}
	Weinberg, S. (1995). 
	\textit{The Quantum Theory of Fields, Vol. I--III}. 
	Cambridge University Press.
	
	\bibitem{griffiths_particle}
	Griffiths, D. (2008). 
	\textit{Introduction to Elementary Particles}. 
	Wiley-VCH.
	
	\bibitem{mandl_shaw}
	Mandl, F., \& Shaw, G. (2010). 
	\textit{Quantum Field Theory (2nd ed.)}. 
	Wiley.
	
	\bibitem{srednicki_qft}
	Srednicki, M. (2007). 
	\textit{Quantum Field Theory}. 
	Cambridge University Press.
	
	\bibitem{t0_fundamentals}
	Pascher, J. (2024). 
	\textit{T0-Theory: Foundations of Time-Mass Duality}. 
	Unpublished manuscript, HTL Leonding.
	
	\bibitem{t0_fine_structure}
	Pascher, J. (2024). 
	\textit{T0-Theory: The Fine Structure Constant}. 
	Unpublished manuscript, HTL Leonding.
	
	\bibitem{t0_neutrinos}
	Pascher, J. (2024). 
	\textit{T0-Theory: Neutrino Masses and PMNS Mixing}. 
	Unpublished manuscript, HTL Leonding.
	
	\bibitem{t0_github}
	Pascher, J. (2024--2025). 
	\textit{T0-Time-Mass-Duality Repository}. 
	GitHub.
	\url{https://github.com/jpascher/T0-Time-Mass-Duality}
	
	\bibitem{lattice_qcd_review}
	Kronfeld, A. S. (2012). 
	\textit{Twenty-first Century Lattice Gauge Theory: Results from the QCD Lagrangian}. 
	Annual Review of Nuclear and Particle Science, 62, 265--284.
	
	\bibitem{neutrino_mixing_pdg}
	Particle Data Group Collaboration (2024). 
	\textit{Neutrino Masses, Mixing, and Oscillations}. 
	PDG Review 2024.
	\url{https://pdg.lbl.gov/2024/reviews/rpp2024-rev-neutrino-mixing.pdf}
	
	\bibitem{higgs_discovery}
	ATLAS and CMS Collaborations (2012). 
	\textit{Observation of a New Particle in the Search for the Standard Model Higgs Boson}. 
	Physics Letters B, 716, 1--29.
	
	\bibitem{Brannen2005}
	C. P. Brannen, ``Estimate of neutrino masses from Koide's relation'', \textit{arXiv:hep-ph/0505028} (2005).
	\url{https://arxiv.org/abs/hep-ph/0505028}
	
	\bibitem{Brannen2006}
	C. P. Brannen, ``Koide Mass Formula for Neutrinos'', \textit{arXiv:0702.0052} (2006).
	\url{http://brannenworks.com/MASSES.pdf}
	
	\bibitem{PhaseVectors2025}
	Anonymous, ``The Koide Relation and Lepton Mass Hierarchy from Phase Vectors'', \textit{rXiv:2507.0040} (2025).
	\url{https://rxiv.org/pdf/2507.0040v1.pdf}
	
	\bibitem{PDG2025}
	Particle Data Group, ``Review of Particle Physics'', \textit{Phys. Rev. D} \textbf{112} (2025) 030001.
	\url{https://pdg.lbl.gov/2025/}
	
	\bibitem{terrell2024}
	Terrell et al. (2024). 
	\textit{Single-Clock Metrology in Nature}. 
	Nature Physics.
	
	\bibitem{hossenfelder2024}
	Hossenfelder, S. (2024). 
	\textit{Single Clock Video Explanation}. 
	YouTube.
	
	\bibitem{hundert1931}
	Hundert (1931). 
	\textit{Reference Work}. 
	Publisher.
	
	\bibitem{terrell2025}
	Terrell et al. (2025). 
	\textit{Advanced Clock Synchronization Methods}. 
	Physical Review Letters.
	
	\bibitem{pascher_t0_2025}
	Pascher, J. (2025). 
	\textit{T0-Theory: Complete Framework and Applications}. 
	Unpublished manuscript, HTL Leonding.
	
	\bibitem{t0qm}
	Pascher, J. (2024). 
	\textit{T0-Theory: Quantum Mechanics Formulation}. 
	Unpublished manuscript, HTL Leonding.
	
	\bibitem{t0anomale}
	Pascher, J. (2024). 
	\textit{T0-Theory: Anomalous Magnetic Moments}. 
	Unpublished manuscript, HTL Leonding.
	
	\bibitem{muong2complete}
	Abi, B., et al. (Muon g-2 Collaboration) (2023). 
	\textit{Complete Measurement of the Positive Muon Anomalous Magnetic Moment}. 
	Physical Review Letters, 131, 161802.
	
	\bibitem{penrose2004}
	Penrose, R. (2004). 
	\textit{The Road to Reality: A Complete Guide to the Laws of the Universe}. 
	Jonathan Cape.
	
	\bibitem{planck1900}
	Planck, M. (1900). 
	\textit{On the Theory of the Energy Distribution Law of the Normal Spectrum}. 
	Verhandlungen der Deutschen Physikalischen Gesellschaft, 2, 237.
	
	\bibitem{T0Theory}
	Pascher, J. (2024). 
	\textit{T0-Theory: Fundamental Principles}. 
	Unpublished manuscript, HTL Leonding.
	
	% Additional bibliography entries for all undefined citations
	\bibitem{6g_roadmap}
	6G Research Consortium (2024).
	\textit{6G Technology Roadmap}.
	Technical Report.
	
	\bibitem{Born2013}
	Born, M. (2013).
	\textit{Einstein's Theory of Relativity}.
	Dover Publications.
	
	\bibitem{Casimir1948}
	Casimir, H. B. G. (1948).
	\textit{On the attraction between two perfectly conducting plates}.
	Proc. Kon. Ned. Akad. Wetensch. B51, 793--795.
	
	\bibitem{Einstein1905}
	Einstein, A. (1905).
	\textit{On the Electrodynamics of Moving Bodies}.
	Annalen der Physik, 17, 891--921.
	
	\bibitem{Feynman2006}
	Feynman, R. P. (2006).
	\textit{QED: The Strange Theory of Light and Matter}.
	Princeton University Press.
	
	\bibitem{Griffiths2017}
	Griffiths, D. J. (2017).
	\textit{Introduction to Electrodynamics (4th ed.)}.
	Cambridge University Press.
	
	\bibitem{Jackson1999}
	Jackson, J. D. (1999).
	\textit{Classical Electrodynamics (3rd ed.)}.
	Wiley.
	
	\bibitem{Mohr2016}
	Mohr, P. J., et al. (2016).
	\textit{CODATA Recommended Values of the Fundamental Physical Constants: 2014}.
	Rev. Mod. Phys. 88, 035009.
	
	\bibitem{Parker2018}
	Parker, R. H., et al. (2018).
	\textit{Measurement of the fine-structure constant as a test of the Standard Model}.
	Science, 360, 191--195.
	
	\bibitem{Planck1900}
	Planck, M. (1900).
	\textit{On the Theory of the Energy Distribution Law of the Normal Spectrum}.
	Verhandlungen der Deutschen Physikalischen Gesellschaft, 2, 237.
	
	\bibitem{Planck2018}
	Planck Collaboration (2018).
	\textit{Planck 2018 results. VI. Cosmological parameters}.
	Astronomy \& Astrophysics, 641, A6.
	
	\bibitem{QFT_T0}
	Pascher, J. (2024).
	\textit{T0-Theory and QFT Connections}.
	Unpublished manuscript, HTL Leonding.
	
	\bibitem{Sommerfeld1916}
	Sommerfeld, A. (1916).
	\textit{On the Quantum Theory of Spectral Lines}.
	Annalen der Physik, 51, 1--94.
	
	\bibitem{T0_Feinstruktur}
	Pascher, J. (2024).
	\textit{T0-Theory: Fine Structure Analysis}.
	Unpublished manuscript, HTL Leonding.
	
	\bibitem{T0_SI}
	Pascher, J. (2024).
	\textit{T0-Theory and SI Units}.
	Unpublished manuscript, HTL Leonding.
	
	\bibitem{T0_fine_structure}
	Pascher, J. (2024).
	\textit{T0-Theory: The Fine Structure Constant}.
	Unpublished manuscript, HTL Leonding.
	
	\bibitem{T0_g2_erweiterung}
	Pascher, J. (2024).
	\textit{T0-Theory: g-2 Extensions}.
	Unpublished manuscript, HTL Leonding.
	
	\bibitem{T0_gravitational_constant}
	Pascher, J. (2024).
	\textit{T0-Theory: Gravitational Constant Derivation}.
	Unpublished manuscript, HTL Leonding.
	
	\bibitem{T0_netze_en}
	Pascher, J. (2024).
	\textit{T0-Theory: Network Structures}.
	Unpublished manuscript, HTL Leonding.
	
	\bibitem{T0_tm_erweiterung}
	Pascher, J. (2024).
	\textit{T0-Theory: Time-Mass Extensions}.
	Unpublished manuscript, HTL Leonding.
	
	\bibitem{Uzan2003}
	Uzan, J.-P. (2003).
	\textit{The fundamental constants and their variation}.
	Rev. Mod. Phys. 75, 403--455.
	
	\bibitem{Weinberg1995}
	Weinberg, S. (1995).
	\textit{The Quantum Theory of Fields, Vol. I}.
	Cambridge University Press.
	
	\bibitem{albrecht1999}
	Albrecht, A. \& Magueijo, J. (1999).
	\textit{A time varying speed of light as a solution to cosmological puzzles}.
	Phys. Rev. D 59, 043516.
	
	\bibitem{alice2023}
	ALICE Collaboration (2023).
	\textit{Recent results from ALICE}.
	CERN-EP-2023-XXX.
	
	\bibitem{analog_optical}
	Smith, J. et al. (2024).
	\textit{Analog optical computing systems}.
	Nature Photonics.
	
	\bibitem{ashtekar2004}
	Ashtekar, A. \& Lewandowski, J. (2004).
	\textit{Background independent quantum gravity}.
	Class. Quantum Grav. 21, R53.
	
	\bibitem{atlas2023}
	ATLAS Collaboration (2023).
	\textit{ATLAS physics results}.
	CERN-PH-EP-2023-XXX.
	
	\bibitem{atlas2023higgs}
	ATLAS Collaboration (2023).
	\textit{Higgs boson measurements}.
	Phys. Rev. Lett.
	
	\bibitem{barbour1999}
	Barbour, J. (1999).
	\textit{The End of Time}.
	Oxford University Press.
	
	\bibitem{barrow1999}
	Barrow, J. D. (1999).
	\textit{Cosmologies with varying light speed}.
	Phys. Rev. D 59, 043515.
	
	\bibitem{becker2007}
	Becker, K. et al. (2007).
	\textit{String Theory and M-Theory}.
	Cambridge University Press.
	
	\bibitem{bell_muon}
	Bennett, G. W., et al. (Muon g-2 Collaboration) (2006).
	\textit{Final report of the E821 muon anomalous magnetic moment measurement}.
	Phys. Rev. D 73, 072003.
	
	\bibitem{bondi1948}
	Bondi, H. \& Gold, T. (1948).
	\textit{The steady-state theory of the expanding universe}.
	Mon. Not. R. Astron. Soc. 108, 252--270.
	
	\bibitem{brewer2019}
	Brewer, S. M. et al. (2019).
	\textit{Al+ Quantum-Logic Clock with Systematic Uncertainty below $10^{-18}$}.
	Phys. Rev. Lett. 123, 033201.
	
	\bibitem{cms2023top}
	CMS Collaboration (2023).
	\textit{Top quark measurements at CMS}.
	JHEP 2023.
	
	\bibitem{cms2024}
	CMS Collaboration (2024).
	\textit{CMS physics results 2024}.
	CERN-PH-EP-2024-XXX.
	
	\bibitem{codata2019}
	Tiesinga, E. et al. (2019).
	\textit{The 2018 CODATA Recommended Values}.
	J. Phys. Chem. Ref. Data.
	
	\bibitem{desi2025}
	DESI Collaboration (2025).
	\textit{DESI 2025 Cosmology Results}.
	arXiv preprint.
	
	\bibitem{differential_optical}
	Wang, X. et al. (2024).
	\textit{Differential optical computing}.
	Optica.
	
	\bibitem{dingle1972}
	Dingle, H. (1972).
	\textit{Science at the Crossroads}.
	Martin Brian \& O'Keeffe.
	
	\bibitem{divalentino2021}
	Di Valentino, E. et al. (2021).
	\textit{In the realm of the Hubble tension}.
	Class. Quantum Grav. 38, 153001.
	
	\bibitem{elnaschie2004}
	El Naschie, M. S. (2004).
	\textit{A review of E infinity theory}.
	Chaos, Solitons \& Fractals, 19, 209--236.
	
	\bibitem{fabrication_heterogeneous}
	Chen, Y. et al. (2024).
	\textit{Heterogeneous photonic integration}.
	Nature Electronics.
	
	\bibitem{fermilab2023}
	Fermilab (2023).
	\textit{Muon g-2 results}.
	Phys. Rev. Lett.
	
	\bibitem{flexible_wafer}
	Kim, S. et al. (2024).
	\textit{Flexible wafer-scale photonics}.
	Science Advances.
	
	\bibitem{francesco1997}
	Di Francesco, P. et al. (1997).
	\textit{Conformal Field Theory}.
	Springer.
	
	\bibitem{hartree1957}
	Hartree, D. R. (1957).
	\textit{The Calculation of Atomic Structures}.
	Wiley.
	
	\bibitem{hhi_6g}
	Fraunhofer HHI (2024).
	\textit{6G Photonic Integration}.
	Technical Report.
	
	\bibitem{hossenfelder2025}
	Hossenfelder, S. (2025).
	\textit{Science without the gobbledygook}.
	YouTube/Blog.
	
	\bibitem{hossenfelder_single_clock_video}
	Hossenfelder, S. (2024).
	\textit{The Single Clock Problem}.
	YouTube.
	
	\bibitem{hoyle1948}
	Hoyle, F. (1948).
	\textit{A new model for the expanding universe}.
	Mon. Not. R. Astron. Soc. 108, 372--382.
	
	\bibitem{integration_microelectronic}
	Liu, A. et al. (2024).
	\textit{Microelectronic photonic integration}.
	IEEE Journal.
	
	\bibitem{jacobson1995}
	Jacobson, T. (1995).
	\textit{Thermodynamics of spacetime}.
	Phys. Rev. Lett. 75, 1260.
	
	\bibitem{kasevich2023}
	Kasevich, M. et al. (2023).
	\textit{Atom interferometry tests}.
	Nature Physics.
	
	\bibitem{lerner2014}
	Lerner, E. J. (2014).
	\textit{An open letter on cosmology}.
	New Scientist.
	
	\bibitem{lisa2017}
	LISA Consortium (2017).
	\textit{Laser Interferometer Space Antenna}.
	ESA Technical Report.
	
	\bibitem{lithium_tantalate}
	Zhang, M. et al. (2024).
	\textit{Thin-film lithium tantalate photonics}.
	Nature Photonics.
	
	\bibitem{lopez2010}
	Lopez-Corredoira, M. (2010).
	\textit{Tests and problems of the standard model in cosmology}.
	Int. J. Mod. Phys. D.
	
	\bibitem{ludlow2015}
	Ludlow, A. D. et al. (2015).
	\textit{Optical atomic clocks}.
	Rev. Mod. Phys. 87, 637.
	
	\bibitem{mach1883}
	Mach, E. (1883).
	\textit{Die Mechanik in ihrer Entwickelung}.
	F.A. Brockhaus.
	
	\bibitem{maldacena1998}
	Maldacena, J. (1998).
	\textit{The large N limit of superconformal field theories}.
	Adv. Theor. Math. Phys. 2, 231--252.
	
	\bibitem{mueller2014}
	Müller, H. et al. (2014).
	\textit{Atom interferometry tests of the gravitational redshift}.
	Phys. Rev. Lett.
	
	\bibitem{mug2_final_2025}
	Muon g-2 Collaboration (2025).
	\textit{Final muon g-2 measurement}.
	Phys. Rev. Lett.
	
	\bibitem{muong2_2023}
	Muon g-2 Collaboration (2023).
	\textit{Updated muon g-2 results}.
	Phys. Rev. Lett.
	
	\bibitem{nathan2024}
	Nathan, A. et al. (2024).
	\textit{Quantum computing advances}.
	Nature.
	
	\bibitem{newell2018}
	Newell, D. B. et al. (2018).
	\textit{The CODATA 2017 values of h, e, k, and $N_A$}.
	Metrologia 55, L13.
	
	\bibitem{nottale1993}
	Nottale, L. (1993).
	\textit{Fractal Space-Time and Microphysics}.
	World Scientific.
	
	\bibitem{on_chip_lithium}
	Wang, C. et al. (2024).
	\textit{On-chip lithium niobate photonics}.
	Nature Communications.
	
	\bibitem{optical_advantages}
	Shastri, B. J. et al. (2024).
	\textit{Advantages of optical computing}.
	Nature Reviews Physics.
	
	\bibitem{pascher2025cmb}
	Pascher, J. (2025).
	\textit{T0-Theory: CMB Analysis}.
	Unpublished manuscript, HTL Leonding.
	
	\bibitem{pascher2025g2}
	Pascher, J. (2025).
	\textit{T0-Theory: g-2 Predictions}.
	Unpublished manuscript, HTL Leonding.
	
	\bibitem{pascher2025qm}
	Pascher, J. (2025).
	\textit{T0-Theory: Quantum Mechanics}.
	Unpublished manuscript, HTL Leonding.
	
	\bibitem{pascher2025si}
	Pascher, J. (2025).
	\textit{T0-Theory: SI Unit System}.
	Unpublished manuscript, HTL Leonding.
	
	\bibitem{pascher2025t0}
	Pascher, J. (2025).
	\textit{T0-Theory: Complete Framework}.
	Unpublished manuscript, HTL Leonding.
	
	\bibitem{pascher:fundamentals}
	Pascher, J. (2024).
	\textit{T0-Theory: Fundamentals}.
	Unpublished manuscript, HTL Leonding.
	
	\bibitem{pascher:g2_rev9}
	Pascher, J. (2024).
	\textit{T0-Theory: g-2 Revision 9}.
	Unpublished manuscript, HTL Leonding.
	
	\bibitem{pascher:geometric_formalism}
	Pascher, J. (2024).
	\textit{T0-Theory: Geometric Formalism}.
	Unpublished manuscript, HTL Leonding.
	
	\bibitem{pascher:ml_addendum}
	Pascher, J. (2024).
	\textit{T0-Theory: Machine Learning Addendum}.
	Unpublished manuscript, HTL Leonding.
	
	\bibitem{pascher:t0_foundations}
	Pascher, J. (2024).
	\textit{T0-Theory: Foundations}.
	Unpublished manuscript, HTL Leonding.
	
	\bibitem{pascher_derivation_beta_2025}
	Pascher, J. (2025).
	\textit{T0-Theory: Derivation of Beta}.
	Unpublished manuscript, HTL Leonding.
	
	\bibitem{pascher_higgs_connection_2025}
	Pascher, J. (2025).
	\textit{T0-Theory: Higgs Connection}.
	Unpublished manuscript, HTL Leonding.
	
	\bibitem{pascher_lagrangian_extended_2025}
	Pascher, J. (2025).
	\textit{T0-Theory: Extended Lagrangian}.
	Unpublished manuscript, HTL Leonding.
	
	\bibitem{pascher_mathematical_structure_2025}
	Pascher, J. (2025).
	\textit{T0-Theory: Mathematical Structure}.
	Unpublished manuscript, HTL Leonding.
	
	\bibitem{pascher_t0_cmb_2025}
	Pascher, J. (2025).
	\textit{T0-Theory: CMB Predictions}.
	Unpublished manuscript, HTL Leonding.
	
	\bibitem{pascher_t0_energie_2025}
	Pascher, J. (2025).
	\textit{T0-Theory: Energy}.
	Unpublished manuscript, HTL Leonding.
	
	\bibitem{pascher_t0_energy_2025}
	Pascher, J. (2025).
	\textit{T0-Theory: Energy Framework}.
	Unpublished manuscript, HTL Leonding.
	
	\bibitem{pascher_t0_theory_2025}
	Pascher, J. (2025).
	\textit{T0-Theory: Complete Theory}.
	Unpublished manuscript, HTL Leonding.
	
	\bibitem{penrose1959}
	Penrose, R. (1959).
	\textit{The apparent shape of a relativistically moving sphere}.
	Proc. Cambridge Phil. Soc. 55, 137--139.
	
	\bibitem{penrose1967}
	Penrose, R. (1967).
	\textit{Twistor algebra}.
	J. Math. Phys. 8, 345--366.
	
	\bibitem{peratt1992}
	Peratt, A. L. (1992).
	\textit{Physics of the Plasma Universe}.
	Springer-Verlag.
	
	\bibitem{peskin1995}
	Peskin, M. E. \& Schroeder, D. V. (1995).
	\textit{An Introduction to Quantum Field Theory}.
	Addison-Wesley.
	
	\bibitem{peskin_schroeder_1995}
	Peskin, M. E. \& Schroeder, D. V. (1995).
	\textit{An Introduction to Quantum Field Theory}.
	Addison-Wesley.
	
	\bibitem{phoquant}
	PhoQuant (2024).
	\textit{Photonic quantum computing}.
	Technical Report.
	
	\bibitem{photonics_ai}
	Wetzstein, G. et al. (2024).
	\textit{Photonics for AI}.
	Nature.
	
	\bibitem{planck1906}
	Planck, M. (1906).
	\textit{The Theory of Heat Radiation}.
	Johann Ambrosius Barth.
	
	\bibitem{planck2018}
	Planck Collaboration (2018).
	\textit{Planck 2018 results}.
	A\&A 641, A6.
	
	\bibitem{polchinski1998}
	Polchinski, J. (1998).
	\textit{String Theory}.
	Cambridge University Press.
	
	\bibitem{qant_nps}
	QANT (2024).
	\textit{Quantum photonics systems}.
	Technical Report.
	
	\bibitem{quantenjahr25}
	Quantenjahr (2025).
	\textit{International Year of Quantum}.
	UNESCO.
	
	\bibitem{recurrent_photonics}
	Tait, A. N. et al. (2024).
	\textit{Recurrent photonic neural networks}.
	Optica.
	
	\bibitem{rf_photonics}
	Capmany, J. \& Novak, D. (2024).
	\textit{Microwave photonics}.
	Nature Photonics.
	
	\bibitem{riess2019}
	Riess, A. G. et al. (2019).
	\textit{Large Magellanic Cloud Cepheid Standards}.
	ApJ 876, 85.
	
	\bibitem{riess2022}
	Riess, A. G. et al. (2022).
	\textit{A Comprehensive Measurement of H0}.
	ApJ 934, L7.
	
	\bibitem{rovelli2004}
	Rovelli, C. (2004).
	\textit{Quantum Gravity}.
	Cambridge University Press.
	
	\bibitem{sciama1953}
	Sciama, D. W. (1953).
	\textit{On the origin of inertia}.
	Mon. Not. R. Astron. Soc. 113, 34--42.
	
	\bibitem{sciencedaily2025}
	ScienceDaily (2025).
	\textit{Physics news}.
	Online.
	
	\bibitem{sm_g2_2025}
	Aoyama, T. et al. (2025).
	\textit{Standard Model prediction for g-2}.
	Phys. Rep.
	
	\bibitem{susskind1995}
	Susskind, L. (1995).
	\textit{The world as a hologram}.
	J. Math. Phys. 36, 6377--6396.
	
	\bibitem{t0_kosmologie}
	Pascher, J. (2024).
	\textit{T0-Theory: Cosmology}.
	Unpublished manuscript, HTL Leonding.
	
	\bibitem{terrell1959}
	Terrell, J. (1959).
	\textit{Invisibility of the Lorentz contraction}.
	Phys. Rev. 116, 1041--1045.
	
	\bibitem{terrell_single_clock_nature_2024}
	Terrell, J. et al. (2024).
	\textit{Single clock precision measurements}.
	Nature Physics.
	
	\bibitem{tfln_foundry}
	TFLN Foundry (2024).
	\textit{Thin-film lithium niobate foundry services}.
	Technical Specifications.
	
	\bibitem{thiemann2007}
	Thiemann, T. (2007).
	\textit{Modern Canonical Quantum General Relativity}.
	Cambridge University Press.
	
	\bibitem{thz_epfl}
	EPFL (2024).
	\textit{Terahertz photonics research}.
	Technical Report.
	
	\bibitem{unnikrishnan2004}
	Unnikrishnan, C. S. (2004).
	\textit{On Einstein's resolution of the twin clock paradox}.
	Current Science, 86, 704--709.
	
	\bibitem{verlinde2011}
	Verlinde, E. (2011).
	\textit{On the origin of gravity and the laws of Newton}.
	JHEP 2011, 29.
	
	\bibitem{video2025}
	Video (2025).
	\textit{Physics video explanation}.
	YouTube.
	
	\bibitem{weinberg1995}
	Weinberg, S. (1995).
	\textit{The Quantum Theory of Fields}.
	Cambridge University Press.
	
	\bibitem{weiskopf2000}
	Weiskopf, D. (2000).
	\textit{Visualization of special relativity}.
	PhD thesis, University of Tübingen.
	
	\bibitem{wheeler1990}
	Wheeler, J. A. (1990).
	\textit{A Journey into Gravity and Spacetime}.
	Scientific American Library.
	
	\bibitem{wiki_bell}
	Wikipedia (2024).
	\textit{Bell's theorem}.
	Online encyclopedia.
	
	\bibitem{zwicky1929}
	Zwicky, F. (1929).
	\textit{On the red shift of spectral lines through interstellar space}.
	Proc. Natl. Acad. Sci. 15, 773--779.

\end{thebibliography}


\end{document}
