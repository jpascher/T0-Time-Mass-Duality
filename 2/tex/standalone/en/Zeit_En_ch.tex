\documentclass[11pt,a4paper]{article}
\usepackage[a4paper,margin=2cm]{geometry}
\usepackage[utf8]{inputenc}
\usepackage[english]{babel}
\usepackage{lmodern}
\renewcommand{\familydefault}{\sfdefault}

\usepackage{amsmath,amssymb,amsthm}
\usepackage{graphicx}
\usepackage[unicode,pdfencoding=auto,hypertexnames=false]{hyperref}
\usepackage{booktabs}
\usepackage{longtable}
\usepackage{array}
\usepackage{siunitx}
\usepackage{fancyhdr}
\usepackage{float}
\usepackage{tikz}
% tcolorbox removed for standalone
% tcbset removed
\tikzset{
  t0blue/.style={draw=blue,fill=blue!10},
  t0red/.style={draw=red,fill=red!10},
  t0green/.style={draw=green!50!black,fill=green!10},
  t0orange/.style={draw=orange,fill=orange!10},
}
\usepackage{setspace}
\usepackage{enumitem}
\usepackage{adjustbox}
\usepackage{xcolor}

% Define colors for xcolor package
\definecolor{t0green}{RGB}{34,139,34}
\definecolor{t0blue}{RGB}{0,0,255}
\definecolor{t0red}{RGB}{255,0,0}
\definecolor{t0orange}{RGB}{255,165,0}

% Define custom column types for tables
\newcolumntype{L}[1]{>{\raggedright\arraybackslash}p{#1}}
\newcolumntype{C}[1]{>{\centering\arraybackslash}p{#1}}
\newcolumntype{R}[1]{>{\raggedleft\arraybackslash}p{#1}}

\setlength{\parindent}{0pt}
\setlength{\parskip}{6pt}

\hypersetup{
  colorlinks=true,
  linkcolor=blue,
  citecolor=blue,
  urlcolor=blue
}
\pagestyle{fancy}
\setlength{\headheight}{28pt}

\newcommand{\checkmarkx}{\checkmark}
\newcommand{\warningx}{\textbf{!}}

% Makros aus Einzel-Dokumenten (Fallback-Definitionen)
\newcommand{\mytimes}{\times}
\newcommand{\myapprox}{\approx}
\newcommand{\mysim}{\sim}
\newcommand{\myomega}{\omega}
\newcommand{\mypi}{\pi}
\newcommand{\myrightarrow}{\rightarrow}
\newcommand{\mypropto}{\propto}
\newcommand{\deltafield}{\delta\phi}
\newcommand{\xipar}{\xi}
\newcommand{\xiT}{\xi}
\newcommand{\lambdah}{\lambda_h}

% Additional macros used in chapter files
\newcommand{\Kfrak}{K_{\text{frak}}}  % Fractal correction factor
\newcommand{\Dfrak}{D_f}              % Fractal dimension
\newcommand{\betapar}{\beta}          % T0 beta parameter
\newcommand{\alphapar}{\alpha}        % T0 alpha parameter
\newcommand{\Efield}{E}               % Energy field
% Note: checkmarkxa/warningxa are variants used in auto-generated chapter files
\newcommand{\checkmarkxa}{\checkmark}
\newcommand{\warningxa}{\textbf{!}}

% Additional T0-specific macros
\newcommand{\xigeom}{\xi_{\text{geom}}}  % Geometric xi
\newcommand{\lP}{\ell_P}                  % Planck length
\newcommand{\rzero}{r_0}                  % Characteristic radius
\newcommand{\xirat}{\xi_{\text{rat}}}     % Xi ratio
\newcommand{\tzero}{t_0}                  % Characteristic time
\newcommand{\natunits}{\text{(nat. units)}}  % Natural units annotation
\newcommand{\myRightarrow}{\Rightarrow}   % Arrow variant
\newcommand{\Lag}{\mathcal{L}}            % Lagrangian

% Physics macros used in chapter files
\newcommand{\CQCD}{C_{\text{QCD}}}        % QCD correction
\newcommand{\EP}{E_P}                     % Planck energy
\newcommand{\Ee}{E_e}                     % Electron energy
\newcommand{\Emu}{E_\mu}                  % Muon energy
\newcommand{\Exi}{E_\xi}                  % Xi energy
\newcommand{\Ezero}{E_0}                  % Characteristic energy
\newcommand{\Hubble}{H}                   % Hubble constant
\newcommand{\Kspec}{K_{\text{spec}}}      % Spectral correction
\newcommand{\Lambdat}{\Lambda_t}          % Time-related cosmological constant
\newcommand{\Leff}{\mathcal{L}_{\text{eff}}}  % Effective Lagrangian
\newcommand{\Lorentz}{\mathcal{L}}        % Lorentz symbol
\newcommand{\Lxi}{L_\xi}                  % Xi length
\newcommand{\Tfield}{T}                   % Time field
\newcommand{\Weyl}{W}                     % Weyl tensor/symbol
\newcommand{\alphaEMSI}{\alpha_{\text{EM,SI}}}  % EM alpha in SI
\newcommand{\alphaEMnat}{\alpha_{\text{EM,nat}}}  % EM alpha in natural units
\newcommand{\alphaem}{\alpha_{\text{em}}} % Electromagnetic alpha
\newcommand{\betaTSI}{\beta_{T,\text{SI}}}  % Beta in SI
\newcommand{\betaTnat}{\beta_{T,\text{nat}}}  % Beta in natural units
\newcommand{\deltam}{\delta m}            % Mass difference
\newcommand{\phiT}{\phi_T}                % T-field phi
\newcommand{\tP}{t_P}                     % Planck time
\newcommand{\rhoCMB}{\rho_{\text{CMB}}}   % CMB density
\newcommand{\rhoCasimir}{\rho_{\text{Casimir}}}  % Casimir density

% Table formatting
\usepackage{multirow}

% Additional physics macros
\newcommand{\Riem}{\mathcal{R}}           % Riemann tensor
\newcommand{\ZPinch}{Z_{\text{pinch}}}    % Z-pinch
\newcommand{\SynchPower}{P_{\text{synch}}} % Synchrotron power
\newcommand{\Rzero}{R_0}                  % Characteristic radius
\newcommand{\alphafine}{\alpha}           % Fine structure constant
\newcommand{\Etau}{E_\tau}                % Tau energy
\newcommand{\deltaE}{\delta E}            % Energy deviation
\newcommand{\EPlanck}{E_P}                % Planck energy
\newcommand{\pichar}{\pi}                 % Pi character
\newcommand{\alphaWSI}{\alpha_{W,\text{SI}}}  % Wien alpha in SI
\newcommand{\alphaWnat}{\alpha_{W,\text{nat}}}  % Wien alpha in natural units

% Einfache abstract-Umgebung für Kapitel:
\newenvironment{abstract}{%
  \begin{center}\bfseries Abstract\end{center}\small
}{\par}


\title{Zeit En}
\author{J. Pascher}
\date{\today}

\begin{document}
\maketitle

\section*{Zeit (Zeit)}

	\begin{abstract}
		The T0 model describes a fundamental granulation of spacetime at the sub-Planck scale $\Lzero = \xipar \times \Lp$ with $\xipar \approx 1.333 \times 10^{-4}$. This work examines the consequences for scale hierarchies, time continuity, and the mathematical completeness of various gravitational theories. The time-mass duality $T(x,t) \cdot m(x,t) = 1$ requires both fields to be coupled and variable, while the fundamental $\xipar$-asymmetry enables all developmental processes.
	\end{abstract}
	
	
	\section{Granulation as Fundamental Principle of Reality}
	
	\subsection{Minimum Length Scale}
	
	The T0 model introduces a fundamental length scale deeper than the Planck length:
	
	\begin{equation}
		\Lzero = \xipar \times \Lp \approx \frac{4}{3} \times 10^{-4} \times 1.616 \times 10^{-35} \text{ m} \approx 2.155 \times 10^{-39} \text{ m}
	\end{equation}
	
	\textbf{Significance of $\Lzero$}:
	\begin{itemize}
		\item Absolute physical lower limit for spatial structures
		\item Granulated spacetime structure - not continuous
		\item Sub-Planck physics with new fundamental laws
		\item Universal scale for all physical phenomena
	\end{itemize}
	
	\subsection{The Extreme Scale Hierarchy}
	
	From $\Lzero$ to cosmological scales extends a hierarchy of over 60 orders of magnitude:
	
	\begin{align}
		\Lzero &\approx 10^{-39} \text{ m} \quad \text{(Sub-Planck minimum)} \\
		\Lp &\approx 10^{-35} \text{ m} \quad \text{(Planck length)} \\
		L_{\text{Casimir}} &\approx 100 \text{ micrometers} \quad \text{(Casimir scale)} \\
		L_{\text{Atom}} &\approx 10^{-10} \text{ m} \quad \text{(Atomic scale)} \\
		L_{\text{Macro}} &\approx 1 \text{ m} \quad \text{(Human scale)} \\
		L_{\text{Cosmo}} &\approx 10^{26} \text{ m} \quad \text{(Cosmological scale)}
	\end{align}
	
	\subsection{Casimir Scale as Evidence of Granulation}
	
	At the Casimir characteristic scale, first measurable effects appear:
	
	\begin{equation}
		L_{\xipar} \approx \frac{1}{\sqrt{\xipar \times \Lp}} \approx 100 \text{ micrometers}
	\end{equation}
	
	\textbf{Experimental evidence}:
	\begin{itemize}
		\item Deviations from $1/d^4$ law at distances $\approx 10$ nm
		\item $\xipar$-corrections in Casimir force measurements
		\item Limits of continuum physics become visible
	\end{itemize}
	
	\section{Limit Systems and Scale Hierarchies}
	
	\subsection{Three-Scale Hierarchy}
	
	The T0 model organizes all physical scales into three fundamental domains:
	
	\begin{enumerate}
		\item \textbf{$\Lzero$-domain}: Granulated physics, universal laws
		\item \textbf{Planck domain}: Quantum gravity, transition dynamics
		\item \textbf{Macro domain}: Classical physics with $\xipar$-corrections
	\end{enumerate}
	
	\subsection{Relational Number System}
	
	Prime number ratios organize particles into natural generations:
	
	\begin{itemize}
		\item \textbf{3-limit}: u-, d-quarks (1st generation)
		\item \textbf{5-limit}: c-, s-quarks (2nd generation)
		\item \textbf{7-limit}: t-, b-quarks (3rd generation)
	\end{itemize}
	
	The next prime number (11) leads to $\xipar^{11}$-corrections $\approx 10^{-44}$, which lie below the Planck scale.
	
	\subsection{CP Violation from Universal Asymmetry}
	
	The $\xipar$-asymmetry explains:
	\begin{itemize}
		\item CP violation in weak interactions
		\item Matter-antimatter asymmetry in the universe
		\item Chiral symmetry breaking in nature
	\end{itemize}
	
	\section{Fundamental Asymmetry as Motion Principle}
	
	\subsection{The Universal -Constant}
	
	\begin{equation}
		\xipar = \frac{4}{3} \times 10^{-4} \approx 1.333 \times 10^{-4}
	\end{equation}
	
	\textbf{Origin}: Geometric 4/3-constant from optimal 3D space packing
	
	\textbf{Effect}: Universal asymmetry enabling all development
	
	\subsection{Eternal Universe Without Big Bang}
	
	The T0 model describes an eternal, infinite, non-expanding universe:
	
	\begin{itemize}
		\item No beginning, no end - timeless existence
		\item Heisenberg's uncertainty principle forbids Big Bang: $\Delta E \times \Delta t \geq \hbar/2$
		\item Structured development instead of chaotic explosion
		\item Continuous $\xipar$-field dynamics instead of Big Bang
	\end{itemize}
	
	\subsection{Time Exists Only After Field-Asymmetry Excitation}
	
	\textbf{Hierarchy of time emergence}:
	\begin{enumerate}
		\item \textbf{Timeless universe}: Perfect symmetry, no time
		\item \textbf{$\xipar$-asymmetry arises}: Symmetry breaking activates time field
		\item \textbf{Time-energy duality}: $T(x,t) \cdot E(x,t) = 1$ becomes active
		\item \textbf{Manifested time}: Local time emerges through field dynamics
		\item \textbf{Directed time}: Thermodynamic arrow of time stabilizes
	\end{enumerate}
	
	Time is not fundamental but emergent from field asymmetry.
	
	\section{Hierarchical Structure: Universe > Field > Space}
	
	\subsection{The Fundamental Order Hierarchy}
	
	\textbf{Universe (highest order level)}:
	\begin{itemize}
		\item Superordinate structure with eternal, infinite properties
		\item Global organizational principles determine everything below
		\item $\xipar$-asymmetry as universal guiding structure
		\item Thermodynamic overall balance of all processes
	\end{itemize}
	
	\textbf{Field (middle organizational level)}:
	\begin{itemize}
		\item Universal $\xipar$-field as mediator between universe and space
		\item Local dynamics within global constraints
		\item Time-energy duality as field principle
		\item Structure-forming processes through asymmetry
	\end{itemize}
	
	\textbf{Space (manifestation level)}:
	\begin{itemize}
		\item 3D geometry as stage for field manifestations
		\item Granulation at $\Lzero$-scale
		\item Local interactions between field excitations
	\end{itemize}
	
	\subsection{Causal Downward Coupling}
	
	\begin{equation}
		\text{UNIVERSE} \rightarrow \text{FIELD} \rightarrow \text{SPACE} \rightarrow \text{PARTICLES}
	\end{equation}
	
	The universe is not just the sum of its spatial parts. Superordinate properties emerge only at the highest level. The $\xipar$-constant is universal, not a space property.
	
	\section{Continuous Time Beyond Certain Scales}
	
	\subsection{The Crucial Scale Hierarchy of Time}
	
	In the T0 model, different time domains exist with fundamentally different properties. The further we move from $\Lzero$, the more continuous and constant time becomes.
	
	\subsubsection{Granulated Zone (below )}
	
	\begin{equation}
		\Lzero = \xipar \times \Lp \approx 2.155 \times 10^{-39} \text{ m}
	\end{equation}
	
	\begin{itemize}
		\item Time is discretely granulated, not continuous
		\item Chaotic quantum fluctuations dominate
		\item Physics loses classical meaning
		\item All fundamental forces equally strong
	\end{itemize}
	
	\subsubsection{Transition Zone (around )}
	
	\begin{itemize}
		\item Time-mass duality $T \cdot m = 1$ becomes fully active
		\item Intensive interaction of all fields
		\item Transition from granulated to continuous
	\end{itemize}
	
	\subsubsection{Continuous Zone (above )}
	
	\subsubsection*{Central Insight}
\begin{equation}
			\text{Distance to } \Lzero \uparrow \quad \Rightarrow \quad \text{Time continuity} \uparrow \quad \Rightarrow \quad \text{Constant direction} \uparrow
		\end{equation}

	
	\begin{itemize}
		\item Beyond a certain point, time becomes continuous
		\item Constant directed flow direction emerges
		\item The greater the distance to $\Lzero$, the more stable the time direction
		\item Emergent classical physics with $\xipar$-corrections
	\end{itemize}
	
	\subsection{Quantitative Scaling of Time Continuity}
	
	\textbf{Time continuity as function of distance to $\Lzero$}:
	\begin{equation}
		\text{Time continuity} \propto \log\left(\frac{L}{\Lzero}\right) \quad \text{for } L \gg \Lzero
	\end{equation}
	
	\textbf{Practical scales}:
	\begin{align}
		L = 10^{-35}\text{ m (Planck)}: &\quad \text{Still granulated} \\
		L = 10^{-15}\text{ m (Nuclear)}: &\quad \text{Transition to continuity} \\
		L = 10^{-10}\text{ m (Atomic)}: &\quad \text{Practically continuous} \\
		L = 10^{-3}\text{ m (mm)}: &\quad \text{Completely continuous, constant direction} \\
		L = 1\text{ m (Meter)}: &\quad \text{Perfectly linear, directed time}
	\end{align}
	
	\subsection{Thermodynamic Arrow of Time}
	
	\textbf{Scale-dependent entropy}:
	\begin{itemize}
		\item \textbf{Granulated level ($\Lzero$)}: Maximum entropy, perfect symmetry
		\item \textbf{Transition level}: Entropy gradients emerge
		\item \textbf{Continuous level}: Second law becomes active
		\item \textbf{Macroscopic level}: Irreversible time direction
	\end{itemize}
	
	\section{Practical vs. Fundamental Physics}
	
	\subsection{Time is Practically Experienced as Constant}
	
	De facto for us: Time flows constantly in our experience domain
	\begin{itemize}
		\item \textbf{Local scales (m to km)}: Time is practically perfectly linear and constant
		\item \textbf{Measurable variations}: Only under extreme conditions (GPS satellites, particle accelerators)
		\item \textbf{Everyday physics}: Time constancy is a good approximation
	\end{itemize}
	
	\subsection{Speed of Light as Clear Upper Limit}
	
	\textbf{Observed reality}:
	\begin{itemize}
		\item $c = 299,792,458$ m/s is measurable upper limit for information transfer
		\item \textbf{Causality}: No signals faster than $c$ observed
		\item \textbf{Relativistic effects}: Clearly measurable at $v \rightarrow c$
		\item \textbf{Particle accelerators}: Confirm $c$-limit daily
	\end{itemize}
	
	\subsection{Resolution of the Apparent Contradiction}
	
	\textbf{Macroscopic level (our world)}:
	\begin{equation}
		L = 1 \text{ m to } 10^6 \text{ m (km range)}
	\end{equation}
	
	\begin{itemize}
		\item Time flows constantly: $dt/dt_0 \approx 1 + 10^{-16}$ (immeasurable)
		\item $c$ is practically constant: $\Delta c/c \approx 10^{-16}$ (immeasurable)
		\item Einstein physics works perfectly
	\end{itemize}
	
	\textbf{Fundamental level (T0 model)}:
	\begin{equation}
		\Lzero = 10^{-39} \text{ m to } \Lp = 10^{-35} \text{ m}
	\end{equation}
	
	\begin{itemize}
		\item Time-mass duality: $T \cdot m = 1$ is fundamental
		\item $c$ is ratio: $c = L/T$ (must be variable)
		\item Mathematical consistency requires coupled variation
	\end{itemize}
	
\section*{These variations are $10^6$ times smaller than our best measurement precision!}
	
	\section{Gravitation: Mass Variation vs. Space Curvature}
	
	\subsection{Two Equivalent Interpretations}
	
	\textbf{Einstein interpretation}:
	\begin{itemize}
		\item $m = $ constant (fixed mass)
		\item $g_{\mu\nu} = $ variable (curved spacetime)
		\item Mass causes space curvature
	\end{itemize}
	
	\textbf{T0 interpretation}:
	\begin{itemize}
		\item $m(x,t) = $ variable (dynamic mass)
		\item $g_{\mu\nu} = $ fixed (flat Euclidean space)
		\item Mass varies locally through $\xipar$-field
	\end{itemize}
	
	\subsection{Important Insight: We Don't Know!}
	
	\subsubsection*{Attention - Fundamental Point}
We DO NOT KNOW whether mass causes space curvature or whether mass itself varies!
		
		This is an assumption, not a proven fact!

	
	\textbf{Both interpretations are equally valid}:
	
	\textbf{Einstein assumption}:
	\begin{align}
		\text{Mass/energy} &\rightarrow \text{Space curvature} \rightarrow \text{Gravitation} \\
		G_{\mu\nu} &= 8\pi T_{\mu\nu}
	\end{align}
	
	\textbf{T0 alternative}:
	\begin{align}
		\xipar\text{-field} &\rightarrow \text{Mass variation} \rightarrow \text{Gravitational effects} \\
		m(x,t) &= m_0 \cdot (1 + \xipar \cdot \Phi(x,t))
	\end{align}
	
	\subsection{Experimental Indistinguishability}
	
	\textbf{All measurements are frequency-based}:
	\begin{itemize}
		\item \textbf{Clocks}: Hyperfine transition frequencies
		\item \textbf{Scales}: Spring oscillations/resonance frequencies
		\item \textbf{Spectrometers}: Light frequencies and transitions
		\item \textbf{Interferometers}: Phases = frequency integrals
	\end{itemize}
	
	\textbf{Identical frequency shifts}:
	\begin{align}
		\text{Einstein}: \quad \nu' &= \nu_0 \sqrt{1 + 2\Phi/c^2} \approx \nu_0 (1 + \Phi/c^2) \\
		\text{T0}: \quad \nu' &= \nu_0 \cdot \frac{m(x,t)}{T(x,t)} \approx \nu_0 (1 + \Phi/c^2)
	\end{align}
	
	Only frequency ratios are measurable - absolute frequencies are fundamentally inaccessible!
	
	\section{Mathematical Completeness: Both Fields Coupled Variable}
	
	\subsection{The Correct Mathematical Formulation}
	
	\textbf{Mathematically correct in T0 model}:
	\begin{align}
		T(x,t) &= \text{variable} \quad \text{(Time as dynamic field)} \\
		m(x,t) &= \text{variable} \quad \text{(Mass as dynamic field)}
	\end{align}
	
	\textbf{Coupled through fundamental duality}:
	\begin{equation}
		T(x,t) \cdot m(x,t) = 1
	\end{equation}
	
	\textbf{Both fields vary TOGETHER}:
	\begin{align}
		T(x,t) &= T_0 \cdot (1 + \xipar \cdot \Phi(x,t)) \\
		m(x,t) &= m_0 \cdot (1 - \xipar \cdot \Phi(x,t))
	\end{align}
	
	\subsection{Verification of Mathematical Consistency}
	
	\textbf{Duality check}:
	\begin{align}
		T(x,t) \cdot m(x,t) &= T_0 m_0 \cdot (1 + \xipar \Phi)(1 - \xipar \Phi) \\
		&= T_0 m_0 \cdot (1 - \xipar^2 \Phi^2) \\
		&\approx T_0 m_0 = 1 \quad \text{(for } \xipar \Phi \ll 1\text{)}
	\end{align}
	
	Mathematical consistency confirmed!
	
	\subsection{Why Both Fields Must Be Variable}
	
	\textbf{Lagrange formalism requires}:
	\begin{equation}
		\delta S = \int \delta \mathcal{L} \, d^4x = 0
	\end{equation}
	
	\textbf{Complete variation}:
	\begin{equation}
		\delta \mathcal{L} = \frac{\partial \mathcal{L}}{\partial T}\delta T + \frac{\partial \mathcal{L}}{\partial m}\delta m + \frac{\partial \mathcal{L}}{\partial \partial_\mu T}\delta \partial_\mu T + \frac{\partial \mathcal{L}}{\partial \partial_\mu m}\delta \partial_\mu m
	\end{equation}
	
	For mathematical completeness:
	\begin{itemize}
		\item $\delta T \neq 0$ (Time must be variable)
		\item $\delta m \neq 0$ (Mass must be variable)
		\item Both coupled through $T \cdot m = 1$
	\end{itemize}
	
	\subsection{Einstein's Arbitrary Constant Setting}
	
	Einstein arbitrarily sets:
	\begin{equation}
		m_0 = \text{constant} \quad \Rightarrow \quad \delta m = 0
	\end{equation}
	
	\textbf{Mathematical problem}:
	\begin{itemize}
		\item Incomplete variation of the Lagrangian
		\item Violates variation principle of field theory
		\item Arbitrary symmetry breaking without justification
	\end{itemize}
	
	\subsection{Parameter Elegance}
	
	\begin{align}
		\text{Einstein}: \quad &m_0, c, G, \hbar, \Lambda, \alpha_{\text{EM}}, \ldots \quad (\gg 10 \text{ free parameters}) \\
		\text{T0}: \quad &\xipar \quad (1 \text{ universal parameter})
	\end{align}
	
	\section{Pragmatic Preference: Variable Mass with Constant Time}
	
	\subsection{The Pragmatic Alternative for Our Experience Space}
	
	As pragmatists, one can certainly prefer:
	\begin{align}
		\text{Time}: \quad t &= \text{constant} \quad \text{(practical experience)} \\
		\text{Mass}: \quad m(x,t) &= \text{variable} \quad \text{(dynamic adjustment)}
	\end{align}
	
	\textbf{Why this is pragmatically sensible}:
	\begin{itemize}
		\item Time constancy corresponds to our direct experience
		\item Mass variation is conceptually easier to imagine
		\item Practical calculations often become simpler
		\item Intuitive understandability for applications
	\end{itemize}
	
	\subsection{Practical Advantages of Constant Time}
	
	In our experienceable space (m to km):
	\begin{itemize}
		\item Time flows linearly and constantly - our direct experience
		\item Clocks tick uniformly - practical time measurement
		\item Causal sequences are clearly defined
		\item Technical applications (GPS, navigation) function
	\end{itemize}
	
	\textbf{Language convention}:
	\begin{itemize}
		\item Time passes constantly
		\item Mass adapts to the fields
		\item Matter becomes heavier/lighter depending on location
	\end{itemize}
	
	\subsection{Variable Mass as Intuitive Concept}
	
	\textbf{Pragmatic interpretation}:
	\begin{equation}
		m(x) = m_0 \cdot (1 + \xipar \cdot \text{Gravitational field}(x))
	\end{equation}
	
	\textbf{Intuitive conception}:
	\begin{itemize}
		\item Mass increases in strong gravitational fields
		\item Mass decreases in weaker fields
		\item Matter feels the local $\xipar$-field
		\item Dynamic adaptation to environment
	\end{itemize}
	
	\subsection{Scientific Legitimacy of Preference}
	
	\subsubsection*{Important Insight}
Pragmatic preferences are scientifically justified when both approaches are experimentally equivalent!

	
	\textbf{Justification}:
	\begin{itemize}
		\item Scientifically equivalent to Einstein approach
		\item Often practically advantageous for applications
		\item Didactically easier to teach
		\item Technically more efficient to implement
	\end{itemize}
	
	The choice between constant time + variable mass vs. Einstein is a matter of taste - both are scientifically equally justified!
	
	\section{The Eternal Philosophical Boundary}
	
	\subsection{What the T0 Model Explains}
	
	\begin{itemize}
		\item HOW the $\xipar$-asymmetry works
		\item WHAT the consequences are
		\item WHICH laws follow from it
		\item WHEN time and development emerge
	\end{itemize}
	
	\subsection{What the T0 Model CANNOT Explain}
	
	The fundamental questions remain:
	\begin{itemize}
		\item WHY does the $\xipar$-asymmetry exist?
		\item WHERE does the original energy come from?
		\item WHO/WHAT gave the first impulse?
		\item WHY does anything exist at all instead of nothing?
	\end{itemize}
	
	\subsection{Scientific Humility}
	
	\textbf{The eternal boundary}:
	Every explanation needs unexplained axioms. The ultimate reason always remains mysterious. The that of existence is given, the why remains open.
	
	\textbf{The elegant shift}:
	The T0 model shifts the mystery to a deeper, more elegant level - but it cannot resolve the fundamental riddle of existence.
	
	And that is good. Because a universe without mystery would be a boring universe.
	
	\section{Experimental Predictions and Tests}
	
	\subsection{Casimir Effect Modifications}
	
	\begin{itemize}
		\item Deviations from $1/d^4$ law at $d \approx 10$ nm
		\item $\xipar$-corrections in precision measurements
		\item Frequency-dependent Casimir forces
	\end{itemize}
	
	\subsection{Atom Interferometry}
	
	\begin{itemize}
		\item $\xipar$-resonances in quantum interferometers
		\item Mass variations in gravitational fields
		\item Time-mass duality in precision experiments
	\end{itemize}
	
	\subsection{Gravitational Wave Detection}
	
	\begin{itemize}
		\item $\xipar$-corrections in LIGO/Virgo data
		\item Modifications of wave dispersion
		\item Sub-Planck structures in gravitational waves
	\end{itemize}
	
	\section{Conclusion: Asymmetry as Engine of Reality}
	
	The T0 model shows that granulation, limits, and fundamental asymmetry are inseparably connected with the scale-dependent nature of time:
	
	\begin{enumerate}
		\item \textbf{Granulation} at $\Lzero$ defines the base scale of all physics
		\item \textbf{Limit systems} organize particles into natural generations
		\item \textbf{Fundamental asymmetry} generates time, development, and structure formation
		\item \textbf{Hierarchical organization} from universe through field to space
		\item \textbf{Continuous time} emerges beyond certain scales through distance to $\Lzero$
		\item \textbf{Mathematical completeness} requires T0 formulation over Einstein
		\item \textbf{Experimental indistinguishability} of different interpretations
		\item \textbf{Pragmatic preferences} are scientifically justified
		\item \textbf{Philosophical boundaries} remain and preserve the mystery
	\end{enumerate}
	
	The $\xipar$-asymmetry is the engine of reality - without it, the universe would remain in perfect, timeless symmetry. With it emerges the entire diversity and dynamics of our observable world.
	
	The T0 model thus offers a unified explanation for fundamental puzzles of physics - from the granulation of spacetime to the emergence of time itself.
	% Mathematical Proof: The Formula T·m = 1 Excludes Singularities
	% This segment can be inserted into an existing LaTeX document
	
	\section{Mathematical Proof: The Formula Excludes Singularities}
	
	\subsection{Important Clarification: as Oscillation Period}
	
	\textbf{ATTENTION:} In this analysis, $T$ does not mean the experienced, continuously flowing time, but the \textbf{oscillation period} or \textbf{characteristic time constant} of a system. This is a fundamental difference:
	
	\begin{itemize}
		\item $T =$ oscillation period (discrete, characteristic time unit)
		\item Not: $T =$ continuous time coordinate (our everyday experience)
	\end{itemize}
	
	\subsection{The Fundamental Exclusion Property}
	
	The equation $T \cdot m = 1$ is not just a mathematical relationship -- it is an \textbf{exclusion theorem}. Through its algebraic structure, it makes certain states mathematically impossible.
	
	\subsection{Proof 1: Exclusion of Infinite Mass}
	
	\textbf{Assumption:} There exists an infinite mass $m = \infty$
	
\section*{Mathematical consequence:}
	\begin{align}
		T \cdot m &= 1\\
		T \cdot \infty &= 1\\
		T &= \frac{1}{\infty} = 0
	\end{align}
	
	\textbf{Contradiction:} $T = 0$ is not in the domain of the equation $T \cdot m = 1$, since:
	\begin{itemize}
		\item The product $0 \cdot \infty$ is mathematically undefined
		\item The original equation $T \cdot m = 1$ would be violated $(0 \cdot \infty \neq 1)$
	\end{itemize}
	
	\textbf{Conclusion:} $m = \infty$ is excluded by the formula.
	
	\subsection{Proof 2: Exclusion of Infinite Time}
	
	\textbf{Assumption:} There exists an infinite time $T = \infty$
	
\section*{Mathematical consequence:}
	\begin{align}
		T \cdot m &= 1\\
		\infty \cdot m &= 1\\
		m &= \frac{1}{\infty} = 0
	\end{align}
	
	\textbf{Contradiction:} $m = 0$ is not in the domain, since:
	\begin{itemize}
		\item The product $\infty \cdot 0$ is mathematically undefined
		\item The equation $T \cdot m = 1$ would be violated $(\infty \cdot 0 \neq 1)$
	\end{itemize}
	
	\textbf{Conclusion:} $T = \infty$ is excluded by the formula.
	
	\subsection{Proof 3: Exclusion of Zero Values}
	
	\textbf{Assumption:} There exists $T = 0$ or $m = 0$
	
	\textbf{Case 1:} $T = 0$
	\begin{equation}
		T \cdot m = 1 \Rightarrow 0 \cdot m = 1
	\end{equation}
	This is impossible for any finite value of $m$, since $0 \cdot m = 0 \neq 1$.
	
	\textbf{Case 2:} $m = 0$
	\begin{equation}
		T \cdot m = 1 \Rightarrow T \cdot 0 = 1
	\end{equation}
	This is impossible for any finite value of $T$, since $T \cdot 0 = 0 \neq 1$.
	
	\textbf{Conclusion:} Both $T = 0$ and $m = 0$ are excluded by the formula.
	
	\subsection{Proof 4: Exclusion of Mathematical Singularities}
	
	\textbf{Definition of a singularity:} A point where a function becomes undefined or infinite.
	
	\textbf{Analysis of the function} $T = \frac{1}{m}$:
	
\section*{Potential singularities could occur at:}
	\begin{itemize}
		\item $m = 0$ (division by zero)
		\item $T \to \infty$ (infinite function values)
	\end{itemize}
	
	\textbf{Exclusion by the constraint} $T \cdot m = 1$:
	\begin{enumerate}
		\item \textbf{At} $m = 0$: The equation $T \cdot m = 1$ cannot be satisfied
		\item \textbf{At} $T \to \infty$: Would require $m \to 0$, which is already excluded
	\end{enumerate}
	
\section*{Mathematical proof of singularity freedom:}
	
	For every point $(T,m)$ with $T \cdot m = 1$:
	\begin{align}
		T &= \frac{1}{m} \text{ with } m \in (0, +\infty)\\
		m &= \frac{1}{T} \text{ with } T \in (0, +\infty)
	\end{align}
	
	Both functions are on their entire domain:
	\begin{itemize}
		\item \textbf{Continuous}
		\item \textbf{Differentiable}
		\item \textbf{Finite}
\section*{Well-defined}
	\end{itemize}
	
	\subsection{The Algebraic Protection Function}
	
	The equation $T \cdot m = 1$ acts like an \textbf{algebraic protection} against singularities:
	
	\subsubsection{Automatic Correction}
	\begin{align}
		\text{If } m \text{ becomes very small} &\Rightarrow T \text{ automatically becomes very large}\\
		\text{If } T \text{ becomes very small} &\Rightarrow m \text{ automatically becomes very large}\\
		\text{But: } T \cdot m &\text{ always remains exactly } 1
	\end{align}
	
	\subsubsection{Mathematical Stability}
	\begin{align}
		\lim_{m \to 0^+} T &= +\infty, \text{ but } T \cdot m = 1 \text{ remains satisfied}\\
		\lim_{T \to 0^+} m &= +\infty, \text{ but } T \cdot m = 1 \text{ remains satisfied}
	\end{align}
	
	The constraint \textbf{forces} the variables into a finite, well-defined region.
	
	\subsection{Proof 5: Positive Definiteness}
	
	\textbf{Theorem:} All solutions of $T \cdot m = 1$ are positive.
	
\section*{Proof:}
	\begin{equation}
		T \cdot m = 1 > 0
	\end{equation}
	
	Since the product is positive, both factors must have the same sign.
	
\section*{Exclusion of negative values:}
	\begin{itemize}
		\item If $T < 0$ and $m < 0$, then $T \cdot m > 0$, but physically meaningless
		\item If $T > 0$ and $m < 0$, then $T \cdot m < 0 \neq 1$
		\item If $T < 0$ and $m > 0$, then $T \cdot m < 0 \neq 1$
	\end{itemize}
	
	\textbf{Conclusion:} Only $T > 0$ and $m > 0$ satisfy the equation.
	
	\subsection{The Fundamental Insight About Time and Continuity}
	
\section*{Important physical clarification:}
	
	The formula $T \cdot m = 1$ describes \textbf{discrete, characteristic properties} of systems, not the continuous time flow of our experience. This means:
	
	\subsubsection{What does NOT state:}
	\begin{itemize}
		\item \glqq Time stands still\grqq\ $(T = 0)$
		\item \glqq Processes take infinitely long\grqq\ $(T = \infty)$
		\item \glqq The time flow is interrupted\grqq
		\item \glqq Our experienced time disappears\grqq
	\end{itemize}
	
	\subsubsection{What actually describes:}
	\begin{itemize}
		\item \textbf{Oscillation periods} have mathematical limits
		\item \textbf{Characteristic time constants} cannot become arbitrary
		\item \textbf{Discrete time units} stand in fixed relation to mass
		\item \textbf{Periodic processes} follow the constraint $T \cdot m = 1$
	\end{itemize}
	
	\subsubsection{The continuous time flow remains unaffected}
	
	The continuous time coordinate $t$ (our \glqq arrow time\grqq) is \textbf{not affected} by this relationship. $T \cdot m = 1$ regulates only the \textbf{intrinsic time scales} of physical systems, not the superordinate time flow in which these systems exist.
	
\section*{Important insight about our time perception:}
	
	Our continuous time perception could practically be only a \textbf{tiny excerpt} of a much larger period -- an oscillation period so immense that it far exceeds anything humans could ever experience or conceive.
	
\section*{Conceivable orders of magnitude:}
	\begin{itemize}
		\item \textbf{Human life:} $\sim 10^2$ years
		\item \textbf{Human history:} $\sim 10^4$ years
		\item \textbf{Earth age:} $\sim 10^9$ years
		\item \textbf{Universe age:} $\sim 10^{10}$ years
		\textbf{Possible cosmic period:} $10^{50}$, $10^{100}$ or even larger time scales
	\end{itemize}
	
	In such a scenario, our entire observable universe would experience only an \textbf{infinitesimal small fraction} of a fundamental oscillation period. For us, time appears linear and continuous because we perceive only a vanishingly small section of a huge cosmic \glqq oscillation\grqq.
	
	\textbf{Analogy:} Just as a bacterium on a clock hand would perceive the movement as \glqq straight ahead\grqq, although it moves on a circular path, we might experience \glqq linear time\grqq, although we are in a gigantic periodic structure.
	
	This perspective shows that $T \cdot m = 1$ and our time perception can operate on completely different scales without contradicting each other.
	
	\subsection{Cosmological Implications}
	
\section*{This viewpoint opens new possibilities:}
	
	What we observe as cosmic development and change could be only a \textbf{small section} in a much larger cyclic pattern that follows the fundamental relationship $T \cdot m = 1$.
	
\section*{Possible cosmic structure:}
	\begin{itemize}
		\item \textbf{Local time perception:} Linear, continuous (our experience domain)
		\item \textbf{Middle time scales:} Observable cosmic developments
		\item \textbf{Fundamental time scale:} Gigantic period according to $T \cdot m = 1$
	\end{itemize}
	
\section*{Implications:}
	\begin{itemize}
		\item Nature could be organized in \textbf{layered-periodic} fashion
		\item Different time scales follow different regularities
		\item $T \cdot m = 1$ could be the \textbf{master constraint} for the largest scale
		\item Our observable cosmic development would be a fragment of a cyclic system
	\end{itemize}
	
	This interpretation shows how mathematical constraints $(T \cdot m = 1)$ and physical observations (linear time perception) can coexist in a \textbf{hierarchical time model}.
	
	\subsection{Conclusion: Mathematical Certainty}
	
	The formula $T \cdot m = 1$ is not just an equation -- it is an \textbf{existence proof} for singularity-free physics. It proves mathematically that:
	
	\begin{itemize}
		\item \textbf{Infinite masses do not exist}
		\item \textbf{Infinite oscillation periods do not exist}
		\item \textbf{Zero masses are excluded}
		\item \textbf{Zero oscillation periods are excluded}
		\item \textbf{Singularities in characteristic time scales cannot occur}
	\end{itemize}
	
\section*{Mathematics itself protects physics from singularities -- without affecting the continuous time flow.}
	


% Bibliography
\begin{thebibliography}{99}
	
	\bibitem{pdg2024}
	Particle Data Group Collaboration (2024). 
	\textit{Review of Particle Physics}. 
	Progress of Theoretical and Experimental Physics, 2024(8), 083C01.
	\url{https://pdg.lbl.gov}
	
	\bibitem{flag2024}
	Aoki, Y., et al. (FLAG Collaboration) (2024). 
	\textit{FLAG Review 2024 of Lattice Results for Low-Energy Constants}. 
	arXiv:2411.04268.
	\url{https://arxiv.org/abs/2411.04268}
	
	\bibitem{fermilab_muon_g2}
	Abi, B., et al. (Muon g-2 Collaboration) (2021). 
	\textit{Measurement of the Positive Muon Anomalous Magnetic Moment to 0.46 ppm}. 
	Physical Review Letters, 126, 141801.
	
	\bibitem{peskin_schroeder}
	Peskin, M. E., \& Schroeder, D. V. (1995). 
	\textit{An Introduction to Quantum Field Theory}. 
	Addison-Wesley.
	
	\bibitem{weinberg_qft}
	Weinberg, S. (1995). 
	\textit{The Quantum Theory of Fields, Vol. I--III}. 
	Cambridge University Press.
	
	\bibitem{griffiths_particle}
	Griffiths, D. (2008). 
	\textit{Introduction to Elementary Particles}. 
	Wiley-VCH.
	
	\bibitem{mandl_shaw}
	Mandl, F., \& Shaw, G. (2010). 
	\textit{Quantum Field Theory (2nd ed.)}. 
	Wiley.
	
	\bibitem{srednicki_qft}
	Srednicki, M. (2007). 
	\textit{Quantum Field Theory}. 
	Cambridge University Press.
	
	\bibitem{t0_fundamentals}
	Pascher, J. (2024). 
	\textit{T0-Theory: Foundations of Time-Mass Duality}. 
	Unpublished manuscript, HTL Leonding.
	
	\bibitem{t0_fine_structure}
	Pascher, J. (2024). 
	\textit{T0-Theory: The Fine Structure Constant}. 
	Unpublished manuscript, HTL Leonding.
	
	\bibitem{t0_neutrinos}
	Pascher, J. (2024). 
	\textit{T0-Theory: Neutrino Masses and PMNS Mixing}. 
	Unpublished manuscript, HTL Leonding.
	
	\bibitem{t0_github}
	Pascher, J. (2024--2025). 
	\textit{T0-Time-Mass-Duality Repository}. 
	GitHub.
	\url{https://github.com/jpascher/T0-Time-Mass-Duality}
	
	\bibitem{lattice_qcd_review}
	Kronfeld, A. S. (2012). 
	\textit{Twenty-first Century Lattice Gauge Theory: Results from the QCD Lagrangian}. 
	Annual Review of Nuclear and Particle Science, 62, 265--284.
	
	\bibitem{neutrino_mixing_pdg}
	Particle Data Group Collaboration (2024). 
	\textit{Neutrino Masses, Mixing, and Oscillations}. 
	PDG Review 2024.
	\url{https://pdg.lbl.gov/2024/reviews/rpp2024-rev-neutrino-mixing.pdf}
	
	\bibitem{higgs_discovery}
	ATLAS and CMS Collaborations (2012). 
	\textit{Observation of a New Particle in the Search for the Standard Model Higgs Boson}. 
	Physics Letters B, 716, 1--29.
	
	\bibitem{Brannen2005}
	C. P. Brannen, ``Estimate of neutrino masses from Koide's relation'', \textit{arXiv:hep-ph/0505028} (2005).
	\url{https://arxiv.org/abs/hep-ph/0505028}
	
	\bibitem{Brannen2006}
	C. P. Brannen, ``Koide Mass Formula for Neutrinos'', \textit{arXiv:0702.0052} (2006).
	\url{http://brannenworks.com/MASSES.pdf}
	
	\bibitem{PhaseVectors2025}
	Anonymous, ``The Koide Relation and Lepton Mass Hierarchy from Phase Vectors'', \textit{rXiv:2507.0040} (2025).
	\url{https://rxiv.org/pdf/2507.0040v1.pdf}
	
	\bibitem{PDG2025}
	Particle Data Group, ``Review of Particle Physics'', \textit{Phys. Rev. D} \textbf{112} (2025) 030001.
	\url{https://pdg.lbl.gov/2025/}
	
	\bibitem{terrell2024}
	Terrell et al. (2024). 
	\textit{Single-Clock Metrology in Nature}. 
	Nature Physics.
	
	\bibitem{hossenfelder2024}
	Hossenfelder, S. (2024). 
	\textit{Single Clock Video Explanation}. 
	YouTube.
	
	\bibitem{hundert1931}
	Hundert (1931). 
	\textit{Reference Work}. 
	Publisher.
	
	\bibitem{terrell2025}
	Terrell et al. (2025). 
	\textit{Advanced Clock Synchronization Methods}. 
	Physical Review Letters.
	
	\bibitem{pascher_t0_2025}
	Pascher, J. (2025). 
	\textit{T0-Theory: Complete Framework and Applications}. 
	Unpublished manuscript, HTL Leonding.
	
	\bibitem{t0qm}
	Pascher, J. (2024). 
	\textit{T0-Theory: Quantum Mechanics Formulation}. 
	Unpublished manuscript, HTL Leonding.
	
	\bibitem{t0anomale}
	Pascher, J. (2024). 
	\textit{T0-Theory: Anomalous Magnetic Moments}. 
	Unpublished manuscript, HTL Leonding.
	
	\bibitem{muong2complete}
	Abi, B., et al. (Muon g-2 Collaboration) (2023). 
	\textit{Complete Measurement of the Positive Muon Anomalous Magnetic Moment}. 
	Physical Review Letters, 131, 161802.
	
	\bibitem{penrose2004}
	Penrose, R. (2004). 
	\textit{The Road to Reality: A Complete Guide to the Laws of the Universe}. 
	Jonathan Cape.
	
	\bibitem{planck1900}
	Planck, M. (1900). 
	\textit{On the Theory of the Energy Distribution Law of the Normal Spectrum}. 
	Verhandlungen der Deutschen Physikalischen Gesellschaft, 2, 237.
	
	\bibitem{T0Theory}
	Pascher, J. (2024). 
	\textit{T0-Theory: Fundamental Principles}. 
	Unpublished manuscript, HTL Leonding.
	
	% Additional bibliography entries for all undefined citations
	\bibitem{6g_roadmap}
	6G Research Consortium (2024).
	\textit{6G Technology Roadmap}.
	Technical Report.
	
	\bibitem{Born2013}
	Born, M. (2013).
	\textit{Einstein's Theory of Relativity}.
	Dover Publications.
	
	\bibitem{Casimir1948}
	Casimir, H. B. G. (1948).
	\textit{On the attraction between two perfectly conducting plates}.
	Proc. Kon. Ned. Akad. Wetensch. B51, 793--795.
	
	\bibitem{Einstein1905}
	Einstein, A. (1905).
	\textit{On the Electrodynamics of Moving Bodies}.
	Annalen der Physik, 17, 891--921.
	
	\bibitem{Feynman2006}
	Feynman, R. P. (2006).
	\textit{QED: The Strange Theory of Light and Matter}.
	Princeton University Press.
	
	\bibitem{Griffiths2017}
	Griffiths, D. J. (2017).
	\textit{Introduction to Electrodynamics (4th ed.)}.
	Cambridge University Press.
	
	\bibitem{Jackson1999}
	Jackson, J. D. (1999).
	\textit{Classical Electrodynamics (3rd ed.)}.
	Wiley.
	
	\bibitem{Mohr2016}
	Mohr, P. J., et al. (2016).
	\textit{CODATA Recommended Values of the Fundamental Physical Constants: 2014}.
	Rev. Mod. Phys. 88, 035009.
	
	\bibitem{Parker2018}
	Parker, R. H., et al. (2018).
	\textit{Measurement of the fine-structure constant as a test of the Standard Model}.
	Science, 360, 191--195.
	
	\bibitem{Planck1900}
	Planck, M. (1900).
	\textit{On the Theory of the Energy Distribution Law of the Normal Spectrum}.
	Verhandlungen der Deutschen Physikalischen Gesellschaft, 2, 237.
	
	\bibitem{Planck2018}
	Planck Collaboration (2018).
	\textit{Planck 2018 results. VI. Cosmological parameters}.
	Astronomy \& Astrophysics, 641, A6.
	
	\bibitem{QFT_T0}
	Pascher, J. (2024).
	\textit{T0-Theory and QFT Connections}.
	Unpublished manuscript, HTL Leonding.
	
	\bibitem{Sommerfeld1916}
	Sommerfeld, A. (1916).
	\textit{On the Quantum Theory of Spectral Lines}.
	Annalen der Physik, 51, 1--94.
	
	\bibitem{T0_Feinstruktur}
	Pascher, J. (2024).
	\textit{T0-Theory: Fine Structure Analysis}.
	Unpublished manuscript, HTL Leonding.
	
	\bibitem{T0_SI}
	Pascher, J. (2024).
	\textit{T0-Theory and SI Units}.
	Unpublished manuscript, HTL Leonding.
	
	\bibitem{T0_fine_structure}
	Pascher, J. (2024).
	\textit{T0-Theory: The Fine Structure Constant}.
	Unpublished manuscript, HTL Leonding.
	
	\bibitem{T0_g2_erweiterung}
	Pascher, J. (2024).
	\textit{T0-Theory: g-2 Extensions}.
	Unpublished manuscript, HTL Leonding.
	
	\bibitem{T0_gravitational_constant}
	Pascher, J. (2024).
	\textit{T0-Theory: Gravitational Constant Derivation}.
	Unpublished manuscript, HTL Leonding.
	
	\bibitem{T0_netze_en}
	Pascher, J. (2024).
	\textit{T0-Theory: Network Structures}.
	Unpublished manuscript, HTL Leonding.
	
	\bibitem{T0_tm_erweiterung}
	Pascher, J. (2024).
	\textit{T0-Theory: Time-Mass Extensions}.
	Unpublished manuscript, HTL Leonding.
	
	\bibitem{Uzan2003}
	Uzan, J.-P. (2003).
	\textit{The fundamental constants and their variation}.
	Rev. Mod. Phys. 75, 403--455.
	
	\bibitem{Weinberg1995}
	Weinberg, S. (1995).
	\textit{The Quantum Theory of Fields, Vol. I}.
	Cambridge University Press.
	
	\bibitem{albrecht1999}
	Albrecht, A. \& Magueijo, J. (1999).
	\textit{A time varying speed of light as a solution to cosmological puzzles}.
	Phys. Rev. D 59, 043516.
	
	\bibitem{alice2023}
	ALICE Collaboration (2023).
	\textit{Recent results from ALICE}.
	CERN-EP-2023-XXX.
	
	\bibitem{analog_optical}
	Smith, J. et al. (2024).
	\textit{Analog optical computing systems}.
	Nature Photonics.
	
	\bibitem{ashtekar2004}
	Ashtekar, A. \& Lewandowski, J. (2004).
	\textit{Background independent quantum gravity}.
	Class. Quantum Grav. 21, R53.
	
	\bibitem{atlas2023}
	ATLAS Collaboration (2023).
	\textit{ATLAS physics results}.
	CERN-PH-EP-2023-XXX.
	
	\bibitem{atlas2023higgs}
	ATLAS Collaboration (2023).
	\textit{Higgs boson measurements}.
	Phys. Rev. Lett.
	
	\bibitem{barbour1999}
	Barbour, J. (1999).
	\textit{The End of Time}.
	Oxford University Press.
	
	\bibitem{barrow1999}
	Barrow, J. D. (1999).
	\textit{Cosmologies with varying light speed}.
	Phys. Rev. D 59, 043515.
	
	\bibitem{becker2007}
	Becker, K. et al. (2007).
	\textit{String Theory and M-Theory}.
	Cambridge University Press.
	
	\bibitem{bell_muon}
	Bennett, G. W., et al. (Muon g-2 Collaboration) (2006).
	\textit{Final report of the E821 muon anomalous magnetic moment measurement}.
	Phys. Rev. D 73, 072003.
	
	\bibitem{bondi1948}
	Bondi, H. \& Gold, T. (1948).
	\textit{The steady-state theory of the expanding universe}.
	Mon. Not. R. Astron. Soc. 108, 252--270.
	
	\bibitem{brewer2019}
	Brewer, S. M. et al. (2019).
	\textit{Al+ Quantum-Logic Clock with Systematic Uncertainty below $10^{-18}$}.
	Phys. Rev. Lett. 123, 033201.
	
	\bibitem{cms2023top}
	CMS Collaboration (2023).
	\textit{Top quark measurements at CMS}.
	JHEP 2023.
	
	\bibitem{cms2024}
	CMS Collaboration (2024).
	\textit{CMS physics results 2024}.
	CERN-PH-EP-2024-XXX.
	
	\bibitem{codata2019}
	Tiesinga, E. et al. (2019).
	\textit{The 2018 CODATA Recommended Values}.
	J. Phys. Chem. Ref. Data.
	
	\bibitem{desi2025}
	DESI Collaboration (2025).
	\textit{DESI 2025 Cosmology Results}.
	arXiv preprint.
	
	\bibitem{differential_optical}
	Wang, X. et al. (2024).
	\textit{Differential optical computing}.
	Optica.
	
	\bibitem{dingle1972}
	Dingle, H. (1972).
	\textit{Science at the Crossroads}.
	Martin Brian \& O'Keeffe.
	
	\bibitem{divalentino2021}
	Di Valentino, E. et al. (2021).
	\textit{In the realm of the Hubble tension}.
	Class. Quantum Grav. 38, 153001.
	
	\bibitem{elnaschie2004}
	El Naschie, M. S. (2004).
	\textit{A review of E infinity theory}.
	Chaos, Solitons \& Fractals, 19, 209--236.
	
	\bibitem{fabrication_heterogeneous}
	Chen, Y. et al. (2024).
	\textit{Heterogeneous photonic integration}.
	Nature Electronics.
	
	\bibitem{fermilab2023}
	Fermilab (2023).
	\textit{Muon g-2 results}.
	Phys. Rev. Lett.
	
	\bibitem{flexible_wafer}
	Kim, S. et al. (2024).
	\textit{Flexible wafer-scale photonics}.
	Science Advances.
	
	\bibitem{francesco1997}
	Di Francesco, P. et al. (1997).
	\textit{Conformal Field Theory}.
	Springer.
	
	\bibitem{hartree1957}
	Hartree, D. R. (1957).
	\textit{The Calculation of Atomic Structures}.
	Wiley.
	
	\bibitem{hhi_6g}
	Fraunhofer HHI (2024).
	\textit{6G Photonic Integration}.
	Technical Report.
	
	\bibitem{hossenfelder2025}
	Hossenfelder, S. (2025).
	\textit{Science without the gobbledygook}.
	YouTube/Blog.
	
	\bibitem{hossenfelder_single_clock_video}
	Hossenfelder, S. (2024).
	\textit{The Single Clock Problem}.
	YouTube.
	
	\bibitem{hoyle1948}
	Hoyle, F. (1948).
	\textit{A new model for the expanding universe}.
	Mon. Not. R. Astron. Soc. 108, 372--382.
	
	\bibitem{integration_microelectronic}
	Liu, A. et al. (2024).
	\textit{Microelectronic photonic integration}.
	IEEE Journal.
	
	\bibitem{jacobson1995}
	Jacobson, T. (1995).
	\textit{Thermodynamics of spacetime}.
	Phys. Rev. Lett. 75, 1260.
	
	\bibitem{kasevich2023}
	Kasevich, M. et al. (2023).
	\textit{Atom interferometry tests}.
	Nature Physics.
	
	\bibitem{lerner2014}
	Lerner, E. J. (2014).
	\textit{An open letter on cosmology}.
	New Scientist.
	
	\bibitem{lisa2017}
	LISA Consortium (2017).
	\textit{Laser Interferometer Space Antenna}.
	ESA Technical Report.
	
	\bibitem{lithium_tantalate}
	Zhang, M. et al. (2024).
	\textit{Thin-film lithium tantalate photonics}.
	Nature Photonics.
	
	\bibitem{lopez2010}
	Lopez-Corredoira, M. (2010).
	\textit{Tests and problems of the standard model in cosmology}.
	Int. J. Mod. Phys. D.
	
	\bibitem{ludlow2015}
	Ludlow, A. D. et al. (2015).
	\textit{Optical atomic clocks}.
	Rev. Mod. Phys. 87, 637.
	
	\bibitem{mach1883}
	Mach, E. (1883).
	\textit{Die Mechanik in ihrer Entwickelung}.
	F.A. Brockhaus.
	
	\bibitem{maldacena1998}
	Maldacena, J. (1998).
	\textit{The large N limit of superconformal field theories}.
	Adv. Theor. Math. Phys. 2, 231--252.
	
	\bibitem{mueller2014}
	Müller, H. et al. (2014).
	\textit{Atom interferometry tests of the gravitational redshift}.
	Phys. Rev. Lett.
	
	\bibitem{mug2_final_2025}
	Muon g-2 Collaboration (2025).
	\textit{Final muon g-2 measurement}.
	Phys. Rev. Lett.
	
	\bibitem{muong2_2023}
	Muon g-2 Collaboration (2023).
	\textit{Updated muon g-2 results}.
	Phys. Rev. Lett.
	
	\bibitem{nathan2024}
	Nathan, A. et al. (2024).
	\textit{Quantum computing advances}.
	Nature.
	
	\bibitem{newell2018}
	Newell, D. B. et al. (2018).
	\textit{The CODATA 2017 values of h, e, k, and $N_A$}.
	Metrologia 55, L13.
	
	\bibitem{nottale1993}
	Nottale, L. (1993).
	\textit{Fractal Space-Time and Microphysics}.
	World Scientific.
	
	\bibitem{on_chip_lithium}
	Wang, C. et al. (2024).
	\textit{On-chip lithium niobate photonics}.
	Nature Communications.
	
	\bibitem{optical_advantages}
	Shastri, B. J. et al. (2024).
	\textit{Advantages of optical computing}.
	Nature Reviews Physics.
	
	\bibitem{pascher2025cmb}
	Pascher, J. (2025).
	\textit{T0-Theory: CMB Analysis}.
	Unpublished manuscript, HTL Leonding.
	
	\bibitem{pascher2025g2}
	Pascher, J. (2025).
	\textit{T0-Theory: g-2 Predictions}.
	Unpublished manuscript, HTL Leonding.
	
	\bibitem{pascher2025qm}
	Pascher, J. (2025).
	\textit{T0-Theory: Quantum Mechanics}.
	Unpublished manuscript, HTL Leonding.
	
	\bibitem{pascher2025si}
	Pascher, J. (2025).
	\textit{T0-Theory: SI Unit System}.
	Unpublished manuscript, HTL Leonding.
	
	\bibitem{pascher2025t0}
	Pascher, J. (2025).
	\textit{T0-Theory: Complete Framework}.
	Unpublished manuscript, HTL Leonding.
	
	\bibitem{pascher:fundamentals}
	Pascher, J. (2024).
	\textit{T0-Theory: Fundamentals}.
	Unpublished manuscript, HTL Leonding.
	
	\bibitem{pascher:g2_rev9}
	Pascher, J. (2024).
	\textit{T0-Theory: g-2 Revision 9}.
	Unpublished manuscript, HTL Leonding.
	
	\bibitem{pascher:geometric_formalism}
	Pascher, J. (2024).
	\textit{T0-Theory: Geometric Formalism}.
	Unpublished manuscript, HTL Leonding.
	
	\bibitem{pascher:ml_addendum}
	Pascher, J. (2024).
	\textit{T0-Theory: Machine Learning Addendum}.
	Unpublished manuscript, HTL Leonding.
	
	\bibitem{pascher:t0_foundations}
	Pascher, J. (2024).
	\textit{T0-Theory: Foundations}.
	Unpublished manuscript, HTL Leonding.
	
	\bibitem{pascher_derivation_beta_2025}
	Pascher, J. (2025).
	\textit{T0-Theory: Derivation of Beta}.
	Unpublished manuscript, HTL Leonding.
	
	\bibitem{pascher_higgs_connection_2025}
	Pascher, J. (2025).
	\textit{T0-Theory: Higgs Connection}.
	Unpublished manuscript, HTL Leonding.
	
	\bibitem{pascher_lagrangian_extended_2025}
	Pascher, J. (2025).
	\textit{T0-Theory: Extended Lagrangian}.
	Unpublished manuscript, HTL Leonding.
	
	\bibitem{pascher_mathematical_structure_2025}
	Pascher, J. (2025).
	\textit{T0-Theory: Mathematical Structure}.
	Unpublished manuscript, HTL Leonding.
	
	\bibitem{pascher_t0_cmb_2025}
	Pascher, J. (2025).
	\textit{T0-Theory: CMB Predictions}.
	Unpublished manuscript, HTL Leonding.
	
	\bibitem{pascher_t0_energie_2025}
	Pascher, J. (2025).
	\textit{T0-Theory: Energy}.
	Unpublished manuscript, HTL Leonding.
	
	\bibitem{pascher_t0_energy_2025}
	Pascher, J. (2025).
	\textit{T0-Theory: Energy Framework}.
	Unpublished manuscript, HTL Leonding.
	
	\bibitem{pascher_t0_theory_2025}
	Pascher, J. (2025).
	\textit{T0-Theory: Complete Theory}.
	Unpublished manuscript, HTL Leonding.
	
	\bibitem{penrose1959}
	Penrose, R. (1959).
	\textit{The apparent shape of a relativistically moving sphere}.
	Proc. Cambridge Phil. Soc. 55, 137--139.
	
	\bibitem{penrose1967}
	Penrose, R. (1967).
	\textit{Twistor algebra}.
	J. Math. Phys. 8, 345--366.
	
	\bibitem{peratt1992}
	Peratt, A. L. (1992).
	\textit{Physics of the Plasma Universe}.
	Springer-Verlag.
	
	\bibitem{peskin1995}
	Peskin, M. E. \& Schroeder, D. V. (1995).
	\textit{An Introduction to Quantum Field Theory}.
	Addison-Wesley.
	
	\bibitem{peskin_schroeder_1995}
	Peskin, M. E. \& Schroeder, D. V. (1995).
	\textit{An Introduction to Quantum Field Theory}.
	Addison-Wesley.
	
	\bibitem{phoquant}
	PhoQuant (2024).
	\textit{Photonic quantum computing}.
	Technical Report.
	
	\bibitem{photonics_ai}
	Wetzstein, G. et al. (2024).
	\textit{Photonics for AI}.
	Nature.
	
	\bibitem{planck1906}
	Planck, M. (1906).
	\textit{The Theory of Heat Radiation}.
	Johann Ambrosius Barth.
	
	\bibitem{planck2018}
	Planck Collaboration (2018).
	\textit{Planck 2018 results}.
	A\&A 641, A6.
	
	\bibitem{polchinski1998}
	Polchinski, J. (1998).
	\textit{String Theory}.
	Cambridge University Press.
	
	\bibitem{qant_nps}
	QANT (2024).
	\textit{Quantum photonics systems}.
	Technical Report.
	
	\bibitem{quantenjahr25}
	Quantenjahr (2025).
	\textit{International Year of Quantum}.
	UNESCO.
	
	\bibitem{recurrent_photonics}
	Tait, A. N. et al. (2024).
	\textit{Recurrent photonic neural networks}.
	Optica.
	
	\bibitem{rf_photonics}
	Capmany, J. \& Novak, D. (2024).
	\textit{Microwave photonics}.
	Nature Photonics.
	
	\bibitem{riess2019}
	Riess, A. G. et al. (2019).
	\textit{Large Magellanic Cloud Cepheid Standards}.
	ApJ 876, 85.
	
	\bibitem{riess2022}
	Riess, A. G. et al. (2022).
	\textit{A Comprehensive Measurement of H0}.
	ApJ 934, L7.
	
	\bibitem{rovelli2004}
	Rovelli, C. (2004).
	\textit{Quantum Gravity}.
	Cambridge University Press.
	
	\bibitem{sciama1953}
	Sciama, D. W. (1953).
	\textit{On the origin of inertia}.
	Mon. Not. R. Astron. Soc. 113, 34--42.
	
	\bibitem{sciencedaily2025}
	ScienceDaily (2025).
	\textit{Physics news}.
	Online.
	
	\bibitem{sm_g2_2025}
	Aoyama, T. et al. (2025).
	\textit{Standard Model prediction for g-2}.
	Phys. Rep.
	
	\bibitem{susskind1995}
	Susskind, L. (1995).
	\textit{The world as a hologram}.
	J. Math. Phys. 36, 6377--6396.
	
	\bibitem{t0_kosmologie}
	Pascher, J. (2024).
	\textit{T0-Theory: Cosmology}.
	Unpublished manuscript, HTL Leonding.
	
	\bibitem{terrell1959}
	Terrell, J. (1959).
	\textit{Invisibility of the Lorentz contraction}.
	Phys. Rev. 116, 1041--1045.
	
	\bibitem{terrell_single_clock_nature_2024}
	Terrell, J. et al. (2024).
	\textit{Single clock precision measurements}.
	Nature Physics.
	
	\bibitem{tfln_foundry}
	TFLN Foundry (2024).
	\textit{Thin-film lithium niobate foundry services}.
	Technical Specifications.
	
	\bibitem{thiemann2007}
	Thiemann, T. (2007).
	\textit{Modern Canonical Quantum General Relativity}.
	Cambridge University Press.
	
	\bibitem{thz_epfl}
	EPFL (2024).
	\textit{Terahertz photonics research}.
	Technical Report.
	
	\bibitem{unnikrishnan2004}
	Unnikrishnan, C. S. (2004).
	\textit{On Einstein's resolution of the twin clock paradox}.
	Current Science, 86, 704--709.
	
	\bibitem{verlinde2011}
	Verlinde, E. (2011).
	\textit{On the origin of gravity and the laws of Newton}.
	JHEP 2011, 29.
	
	\bibitem{video2025}
	Video (2025).
	\textit{Physics video explanation}.
	YouTube.
	
	\bibitem{weinberg1995}
	Weinberg, S. (1995).
	\textit{The Quantum Theory of Fields}.
	Cambridge University Press.
	
	\bibitem{weiskopf2000}
	Weiskopf, D. (2000).
	\textit{Visualization of special relativity}.
	PhD thesis, University of Tübingen.
	
	\bibitem{wheeler1990}
	Wheeler, J. A. (1990).
	\textit{A Journey into Gravity and Spacetime}.
	Scientific American Library.
	
	\bibitem{wiki_bell}
	Wikipedia (2024).
	\textit{Bell's theorem}.
	Online encyclopedia.
	
	\bibitem{zwicky1929}
	Zwicky, F. (1929).
	\textit{On the red shift of spectral lines through interstellar space}.
	Proc. Natl. Acad. Sci. 15, 773--779.

\end{thebibliography}


\end{document}
