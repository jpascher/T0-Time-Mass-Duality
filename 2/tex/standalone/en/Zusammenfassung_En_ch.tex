\documentclass[11pt,a4paper]{article}
\usepackage[a4paper,margin=2cm]{geometry}
\usepackage[utf8]{inputenc}
\usepackage[english]{babel}
\usepackage{lmodern}
\renewcommand{\familydefault}{\sfdefault}

\usepackage{amsmath,amssymb,amsthm}
\usepackage{graphicx}
\usepackage[unicode,pdfencoding=auto,hypertexnames=false]{hyperref}
\usepackage{booktabs}
\usepackage{longtable}
\usepackage{array}
\usepackage{siunitx}
\usepackage{fancyhdr}
\usepackage{float}
\usepackage{tikz}
% tcolorbox removed for standalone
% tcbset removed
\tikzset{
  t0blue/.style={draw=blue,fill=blue!10},
  t0red/.style={draw=red,fill=red!10},
  t0green/.style={draw=green!50!black,fill=green!10},
  t0orange/.style={draw=orange,fill=orange!10},
}
\usepackage{setspace}
\usepackage{enumitem}
\usepackage{adjustbox}
\usepackage{xcolor}

% Define colors for xcolor package
\definecolor{t0green}{RGB}{34,139,34}
\definecolor{t0blue}{RGB}{0,0,255}
\definecolor{t0red}{RGB}{255,0,0}
\definecolor{t0orange}{RGB}{255,165,0}

% Define custom column types for tables
\newcolumntype{L}[1]{>{\raggedright\arraybackslash}p{#1}}
\newcolumntype{C}[1]{>{\centering\arraybackslash}p{#1}}
\newcolumntype{R}[1]{>{\raggedleft\arraybackslash}p{#1}}

\setlength{\parindent}{0pt}
\setlength{\parskip}{6pt}

\hypersetup{
  colorlinks=true,
  linkcolor=blue,
  citecolor=blue,
  urlcolor=blue
}
\pagestyle{fancy}
\setlength{\headheight}{28pt}

\newcommand{\checkmarkx}{\checkmark}
\newcommand{\warningx}{\textbf{!}}

% Makros aus Einzel-Dokumenten (Fallback-Definitionen)
\newcommand{\mytimes}{\times}
\newcommand{\myapprox}{\approx}
\newcommand{\mysim}{\sim}
\newcommand{\myomega}{\omega}
\newcommand{\mypi}{\pi}
\newcommand{\myrightarrow}{\rightarrow}
\newcommand{\mypropto}{\propto}
\newcommand{\deltafield}{\delta\phi}
\newcommand{\xipar}{\xi}
\newcommand{\xiT}{\xi}
\newcommand{\lambdah}{\lambda_h}

% Additional macros used in chapter files
\newcommand{\Kfrak}{K_{\text{frak}}}  % Fractal correction factor
\newcommand{\Dfrak}{D_f}              % Fractal dimension
\newcommand{\betapar}{\beta}          % T0 beta parameter
\newcommand{\alphapar}{\alpha}        % T0 alpha parameter
\newcommand{\Efield}{E}               % Energy field
% Note: checkmarkxa/warningxa are variants used in auto-generated chapter files
\newcommand{\checkmarkxa}{\checkmark}
\newcommand{\warningxa}{\textbf{!}}

% Additional T0-specific macros
\newcommand{\xigeom}{\xi_{\text{geom}}}  % Geometric xi
\newcommand{\lP}{\ell_P}                  % Planck length
\newcommand{\rzero}{r_0}                  % Characteristic radius
\newcommand{\xirat}{\xi_{\text{rat}}}     % Xi ratio
\newcommand{\tzero}{t_0}                  % Characteristic time
\newcommand{\natunits}{\text{(nat. units)}}  % Natural units annotation
\newcommand{\myRightarrow}{\Rightarrow}   % Arrow variant
\newcommand{\Lag}{\mathcal{L}}            % Lagrangian

% Physics macros used in chapter files
\newcommand{\CQCD}{C_{\text{QCD}}}        % QCD correction
\newcommand{\EP}{E_P}                     % Planck energy
\newcommand{\Ee}{E_e}                     % Electron energy
\newcommand{\Emu}{E_\mu}                  % Muon energy
\newcommand{\Exi}{E_\xi}                  % Xi energy
\newcommand{\Ezero}{E_0}                  % Characteristic energy
\newcommand{\Hubble}{H}                   % Hubble constant
\newcommand{\Kspec}{K_{\text{spec}}}      % Spectral correction
\newcommand{\Lambdat}{\Lambda_t}          % Time-related cosmological constant
\newcommand{\Leff}{\mathcal{L}_{\text{eff}}}  % Effective Lagrangian
\newcommand{\Lorentz}{\mathcal{L}}        % Lorentz symbol
\newcommand{\Lxi}{L_\xi}                  % Xi length
\newcommand{\Tfield}{T}                   % Time field
\newcommand{\Weyl}{W}                     % Weyl tensor/symbol
\newcommand{\alphaEMSI}{\alpha_{\text{EM,SI}}}  % EM alpha in SI
\newcommand{\alphaEMnat}{\alpha_{\text{EM,nat}}}  % EM alpha in natural units
\newcommand{\alphaem}{\alpha_{\text{em}}} % Electromagnetic alpha
\newcommand{\betaTSI}{\beta_{T,\text{SI}}}  % Beta in SI
\newcommand{\betaTnat}{\beta_{T,\text{nat}}}  % Beta in natural units
\newcommand{\deltam}{\delta m}            % Mass difference
\newcommand{\phiT}{\phi_T}                % T-field phi
\newcommand{\tP}{t_P}                     % Planck time
\newcommand{\rhoCMB}{\rho_{\text{CMB}}}   % CMB density
\newcommand{\rhoCasimir}{\rho_{\text{Casimir}}}  % Casimir density

% Table formatting
\usepackage{multirow}

% Additional physics macros
\newcommand{\Riem}{\mathcal{R}}           % Riemann tensor
\newcommand{\ZPinch}{Z_{\text{pinch}}}    % Z-pinch
\newcommand{\SynchPower}{P_{\text{synch}}} % Synchrotron power
\newcommand{\Rzero}{R_0}                  % Characteristic radius
\newcommand{\alphafine}{\alpha}           % Fine structure constant
\newcommand{\Etau}{E_\tau}                % Tau energy
\newcommand{\deltaE}{\delta E}            % Energy deviation
\newcommand{\EPlanck}{E_P}                % Planck energy
\newcommand{\pichar}{\pi}                 % Pi character
\newcommand{\alphaWSI}{\alpha_{W,\text{SI}}}  % Wien alpha in SI
\newcommand{\alphaWnat}{\alpha_{W,\text{nat}}}  % Wien alpha in natural units

% Einfache abstract-Umgebung für Kapitel:
\newenvironment{abstract}{%
  \begin{center}\bfseries Abstract\end{center}\small
}{\par}


\title{Zusammenfassung En}
\author{J. Pascher}
\date{\today}

\begin{document}
\maketitle

\section*{Zusammenfassung (Zusammenfassung)}

	\begin{abstract}
		\noindent The T0 model presents an alternative theoretical framework for unifying fundamental physics. Starting from a single geometric constant $\xipar = \frac{4}{3} \times 10^{-4}$ and a universal energy field $\Efield(x,t)$, all physical phenomena are interpreted as manifestations of three-dimensional space geometry. The model eliminates the 20+ free parameters of the Standard Model and offers deterministic explanations for quantum phenomena. Remarkable agreements with experimental data, particularly for the muon's anomalous magnetic moment (accuracy: 0.1$\sigma$), lend empirical relevance to the approach. This treatise presents a complete exposition of the theoretical foundations, mathematical structures, and experimental predictions.
	\end{abstract}
	
	
	\section{Introduction: The Vision of Unified Physics}
	
	Imagine being able to explain all of physics -- from the smallest subatomic particles to the largest galaxy clusters -- with a single, simple idea. That's exactly what the T0 model attempts to achieve. While modern physics is a complicated patchwork of different theories that often don't harmonize with each other, the T0 model proposes a radically simpler path.
	
	Today's physics resembles a house built by different architects: The ground floor (quantum mechanics) follows different rules than the first floor (relativity theory), and neither really fits with the attic (cosmology). Physicists must determine over twenty different numbers -- so-called free parameters -- from experiments, without knowing why these numbers have exactly these values. It's as if you needed twenty different keys to open all the doors in the house, without understanding why each lock is different.
	
\section*{Revolutionary}
		The T0 model proposes: What if there were only one master key? A single number that explains everything -- the geometric constant $\xipar = \frac{4}{3} \times 10^{-4}$. This number isn't arbitrarily chosen but emerges from the geometry of the three-dimensional space in which we live.
% end box revolutionary
	
	The kicker: This one number should suffice to calculate all other numbers in physics -- the mass of the electron, the strength of gravity, even the temperature of the universe. It's as if you'd discovered that all the seemingly random phone numbers in a phone book are built according to a single, hidden pattern.
	
	\section{The Geometric Constant : The Foundation of Reality}
	
	\subsection{What is this mysterious number?}
	
	Imagine you're baking a cake. No matter how big the cake becomes, the ratio of ingredients stays the same -- for a good cake, you always need the right ratio of flour to sugar to butter. The geometric constant $\xipar$ is such a fundamental ratio for our universe.
	
	\begin{equation}
		\boxed{\xipar = \frac{4}{3} \times 10^{-4} = 0.0001333...}
	\end{equation}
	
	This number may seem small and unremarkable, but it's anything but random. The fraction 4/3 might be familiar from music -- it's the frequency ratio of a perfect fourth, one of the most harmonic intervals. But more importantly: This number appears everywhere in the geometry of three-dimensional space.
	
	Think of a sphere -- the most perfect shape in space. Its volume is calculated with the formula $V = \frac{4}{3}\pi r^3$. There it is again, our 4/3! It's as if nature itself has woven this number into the structure of space.
	
	\subsection{Why is this number so important?}
	
	To understand why $\xipar$ is so fundamental, imagine the universe as a giant orchestra. In conventional physics, each instrument (each particle, each force) has its own, seemingly random tuning. Physicists must measure the tuning of each individual instrument without understanding why an electron has exactly this mass or why gravity is exactly this strong (or rather: this weak).
	
\section*{Important}
		The T0 model claims something astonishing: All instruments in the universe's orchestra are tuned to a single pitch -- and this pitch is $\xipar$. 
		
		From this follows:
		\begin{itemize}
			\item The mass of an electron? A specific multiple of $\xipar$
			\item The strength of gravity? Proportional to $\xipar^2$ (that's why it's so weak!)
			\item The strength of the nuclear force? Proportional to $\xipar^{-1/3}$ (that's why it's so strong!)
		\end{itemize}
% end box important
	
	It's as if you'd discovered that all seemingly different colors in the universe are just different mixtures of a single primary color.
	
	\section{The Universal Energy Field: The Only Fundamental Entity}
	
	\subsection{Everything is energy -- but differently than you think}
	
	Einstein taught us with his famous formula $E = mc^2$ that mass and energy are equivalent. The T0 model goes a step further and says: There is only energy! What we perceive as matter, as particles, as solid objects, are in reality just different vibration patterns of a single, all-permeating energy field.
	
	Imagine empty space not as nothing, but as a calm ocean. What we call "particles" are waves on this ocean. An electron is a small, very rapidly circling wave. A photon is a wave that runs across the ocean. A proton is a more complex wave pattern, like a whirlpool in water.
	
	\begin{equation}
		\boxed{\square \Efield = \left(\nabla^2 - \frac{1}{c^2}\frac{\partial^2}{\partial t^2}\right) \Efield = 0}
	\end{equation}
	
	This equation may look complicated, but it says something very simple: The energy field behaves like waves on a pond. It can oscillate, spread, interfere with itself -- and from all these behaviors emerges the apparent diversity of our world.
	
	\subsection{How does energy become an electron?}
	
	Think of a guitar string. When you pluck it, it doesn't vibrate arbitrarily, but in very specific patterns -- the overtones. Similarly, the universal energy field can't vibrate arbitrarily, but only in specific, stable patterns. We perceive these stable vibration patterns as particles:
	
	\begin{itemize}
		\item \textbf{An electron}: Imagine a tiny tornado of energy that constantly rotates around itself. This rotation is so stable that it can persist for billions of years.
		
		\item \textbf{A photon}: Like a wave on the sea that spreads in a straight line. Unlike the electron-tornado, this wave isn't trapped in one place but always moves at the speed of light.
		
		\item \textbf{A quark}: An even more complex pattern, like three intertwined vortices that stabilize each other.
	\end{itemize}
	
	The crucial point: There are no "hard" particles, no tiny billiard balls. Everything is motion, everything is vibration, everything is energy in different forms.
	
	\section{Quantum Mechanics Reinterpreted: Determinism Instead of Probability}
	
	\subsection{The end of randomness?}
	
	Quantum mechanics is considered the strangest theory in physics. It claims that nature is fundamentally random at the smallest scales -- that even God plays dice, as Einstein put it. A radioactive atom doesn't decay for a specific reason, but purely randomly. An electron isn't at a specific location, but "smeared" over many locations simultaneously until we measure it.
	
	The T0 model says: Wait a minute! What we take for randomness is just our ignorance about the exact vibration patterns of the energy field. It's like rolling dice -- the throw appears random, but if you knew exactly the movement of the hand, air resistance, and all other factors, you could predict the result.
	
\section*{Quantum}
		In the T0 model, the famous Schrödinger equation is no longer a probability calculation but describes how the real energy field evolves. The "wave function" isn't an abstract probability but the actual energy density of the field:
		\begin{equation}
			i\hbar \frac{\partial \Psi}{\partial t} = \hat{H}\Psi \quad \text{becomes} \quad i\hbar \frac{\partial \Efield}{\partial t} = \hat{H}_{\text{Field}}\Efield
		\end{equation}
% end box quantum
	
	\subsection{The uncertainty relation -- newly understood}
	
	Heisenberg's famous uncertainty relation states that you can never know exactly both where a particle is and how fast it's moving. The more precisely you measure one, the more uncertain the other becomes. Physicists interpreted this as a fundamental limit of our knowledge.
	
	The T0 model sees it differently: Uncertainty isn't a knowledge limit but expresses that time and energy are two sides of the same coin:
	\begin{equation}
		\Delta E \cdot \Delta t \geq \frac{\hbar}{2}
	\end{equation}
	
	It's like with a musical note: To determine the pitch (frequency = energy) precisely, the tone must sound for a certain time. An ultra-short click has no defined pitch. That's not a measurement limitation, but a fundamental property of vibrations!
	
	\subsection{Schrödinger's cat lives -- and is dead}
	
	The most famous thought experiment in quantum mechanics is Schrödinger's cat: A cat in a box is simultaneously dead and alive until someone looks. That sounds absurd, and that's exactly what Schrödinger wanted to show.
	
	In the T0 model, the solution is simpler: The cat is never simultaneously dead and alive. The energy field is in a specific state, we just don't know it. If the field vibrates such that the radioactive atom has decayed, the cat is dead. If not, it lives. No mystery, no parallel worlds -- just our ignorance of the exact field vibrations.
	
	\subsection{Quantum entanglement -- the "spooky" phenomenon}
	
	Einstein called it "spooky action at a distance" -- quantum entanglement. When two particles are entangled, one knows immediately what happens to the other, no matter how far apart they are. Measure one particle as "spin up", the other is automatically "spin down". Immediately. Faster than light. This seems to violate everything we know about the maximum speed in the universe.
	
	The T0 model offers an elegant explanation: The two particles aren't separate at all! They're two bumps of the same wave in the energy field. Imagine a long rope that you hold in the middle and shake. Waves appear at both ends that are perfectly coordinated -- not because they communicate, but because they're part of the same vibration.
	
	\begin{equation}
		|\Psi_{\text{entangled}}\rangle = \frac{1}{\sqrt{2}}(|00\rangle + |11\rangle) \quad \Rightarrow \quad \Efield(x_1, x_2) = \Efield^{\text{coherent}}
	\end{equation}
	
	When you "measure" one bump (hold the rope at one point), that automatically determines what happens at the other end. No communication, no faster-than-light speed -- just the natural coherence of an extended wave.
	
	\subsection{Quantum computers -- why they work}
	
	Quantum computers are considered the future of computing technology. They use the strange properties of quantum mechanics -- superposition and entanglement -- to solve certain problems millions of times faster than classical computers. But why do they work?
	
\section*{Experimental}
		In the T0 model, the answer is clear: A quantum computer directly manipulates the vibration patterns of the energy field. It uses the natural ability of the field to superpose many different vibration patterns simultaneously:
		
		\begin{itemize}
			\item \textbf{Deutsch algorithm}: Finds out with a single measurement whether a function is constant or balanced -- 100\% success even in the T0 model
			\item \textbf{Grover search}: Finds a needle in a haystack -- 99.999\% success rate in the deterministic T0 model
			\item \textbf{Shor factorization}: Breaks encryptions by finding periods -- works identically
		\end{itemize}
		
		The minimal deviations (0.001\%) are smaller than any practical measurement accuracy!
% end box experimental
	
	\section{The Unification of Quantum Mechanics, Quantum Field Theory and Relativity}
	
	\subsection{The great puzzle of modern physics}
	
	Modern physics has a problem -- actually several. We have three great theories, each of which works excellently on its own, but they don't fit together. It's as if we had three different maps of the same area that contradict each other at the edges.
	
	\textbf{Quantum mechanics} perfectly describes the world of atoms and molecules, but it completely ignores gravity. \textbf{Quantum field theory} extends quantum mechanics to high energies and can create and annihilate particles, but it produces infinite values that must be artificially "calculated away". And the \textbf{General Theory of Relativity} wonderfully explains gravity as curvature of spacetime, but it's not quantizable -- nobody knows how to properly describe quantum gravity.
	
	Physicists have been dreaming of a "Theory of Everything" since Einstein that unites all three theories. The T0 model claims to have found this unification -- and the amazing thing is: The solution is simpler, not more complicated!
	
	\subsection{One field for everything}
	
	Instead of different fields for different particles (electron field, quark field, photon field, hypothetical graviton field), there's only one field in the T0 model -- the universal energy field. All seemingly different fields of quantum field theory are just different vibration modes of this one field:
	
\section*{Important}
		Imagine a concert hall. The different instruments (violin, trumpet, drums) produce different sounds, but they all vibrate in the same air. The air is the medium for all tones. Similarly, the universal energy field is the medium for all particles and forces:
		\begin{itemize}
			\item \textbf{Electromagnetism}: Transverse waves in the energy field (like light waves)
			\item \textbf{Weak nuclear force}: Local rotations of the energy field
			\item \textbf{Strong nuclear force}: Knots of the energy field that hold quarks together
			\item \textbf{Gravity}: The density of the energy field itself -- no additional particles needed!
		\end{itemize}
% end box important
	
	\subsection{Gravity without gravitons}
	
	This is where it gets particularly interesting. Physicists have been searching for decades for "gravitons" -- hypothetical particles that transmit gravity, analogous to photons for electromagnetism. But nobody has ever found a graviton, and the theory of gravitons leads to unsolvable mathematical problems.
	
\section*{Revolutionary}
		The T0 model says: There are no gravitons because they're not needed! Gravity isn't a force like the others, but a geometric effect of energy density:
		
		\begin{equation}
			\text{Spacetime curvature} = \frac{8\pi G}{c^4} \times \text{Energy density of the field}
		\end{equation}
		
		Where the energy field is denser, space curves more strongly. Mass is concentrated energy, so mass curves space. We perceive this curvature as gravity.
% end box revolutionary
	
	The gravitational constant $G$ is not an independent natural constant but follows from our geometric constant: $G = \xipar^2 \cdot c^3/\hbar$. The extreme weakness of gravity (it's $10^{38}$ times weaker than electromagnetism!) is explained by the fact that $\xipar^2$ is a tiny number.
	
	\subsection{Why do all the puzzle pieces suddenly fit together?}
	
	The genius of the T0 model is that many of the great puzzles of physics suddenly solve themselves:
	
	\textbf{The hierarchy problem} -- Why is gravity so much weaker than the other forces? In the T0 model, the answer is simple: The strengths of all forces are powers of $\xipar$. The strong nuclear force has the strength $\xipar^{-1/3} \approx 10$, electromagnetism $\xipar^0 = 1$, the weak nuclear force $\xipar^{1/2} \approx 0.01$, and gravity $\xipar^2 \approx 0.00000001$. The hierarchy isn't mysterious fine-tuning but simple geometry!
	
	\textbf{The infinities of quantum field theory} -- When physicists calculate the interaction of particles, they often get infinite values. They must get rid of these through a mathematical trick called "renormalization". In the T0 model, these infinities don't exist because the energy field has a natural minimal structure determined by $\xipar$.
	
	\textbf{The singularities} -- Black holes and the Big Bang lead to singularities in relativity theory -- points of infinite density where physics breaks down. In the T0 model, there are no real singularities. A black hole is simply a region of maximum energy field density, and the Big Bang? It didn't happen -- the universe exists eternally in a static state.
	
	\subsection{Quantum gravity -- the solved problem}
	
	The biggest unsolved problem of modern physics is quantum gravity. How does gravity behave at smallest scales? Nobody knows. All attempts to "quantize" gravity (turn it into a quantum theory) have failed or led to extremely complex theories like string theory with its 11 dimensions.
	
\section*{Important}
		The T0 model doesn't need a separate theory of quantum gravity! Gravity is already part of the quantized energy field. At small scales, the quantum fluctuations of the field dominate; at large scales, they average out to the smooth spacetime curvature we perceive as gravity.
		
		It's like with water: At the molecular level, you see individual H$_2$O molecules dancing around wildly (quantum level). At the macroscopic level, you see a smooth liquid (classical gravity). Both are the same phenomenon at different scales!
% end box important
	
	\section{Experimental Confirmations and Predictions}
	
	\subsection{The spectacular success with the muon}
	
	The best confirmation of a theory is when it predicts something that's later measured exactly that way. The T0 model had such a triumph with the anomalous magnetic moment of the muon -- one of the most precise measurements in all of physics.
	
	A muon is like a heavy electron -- it has the same properties but weighs 207 times more. When a muon circles in a magnetic field, it behaves like a tiny magnet. The strength of this magnet deviates minimally from the theoretical value -- by about 0.0000000024. Physicists can measure this tiny deviation to eleven decimal places!
	
\section*{Formula}
		The T0 model predicts for this deviation:
		\begin{equation}
			a_\mu^{\text{T0}} = \frac{\xipar}{2\pi} \left(\frac{m_\mu}{m_e}\right)^2 = 245(12) \times 10^{-11}
		\end{equation}
		The experimental value: $251(59) \times 10^{-11}$
		
		The agreement is spectacular -- within 0.1 standard deviations!
% end box formula
	
	That's like predicting the distance from Earth to the Moon to within a few centimeters. And the T0 model achieves this with a single geometric constant, while the Standard Model needs hundreds of correction terms!
	
	\subsection{What we can still test}
	
	The T0 model makes many more predictions that can be tested in coming years:
	
	\textbf{Redshift newly understood}: Light from distant galaxies is redshifted -- its wavelength is stretched. The standard explanation: The universe is expanding. The T0 model says: Light loses energy traversing the energy field. This difference is measurable! At different wavelengths, the redshift should be slightly different.
	
	\textbf{The tau lepton}: The heaviest of the three leptons (electron, muon, tau) is experimentally difficult to study. The T0 model precisely predicts its anomalous magnetic moment: $257(13) \times 10^{-11}$. Future experiments will test this.
	
	\textbf{Modified quantum entanglement}: In extremely precise Bell experiments, tiny deviations of 0.001\% from standard predictions should occur. That's at the limit of today's measurement technology, but not impossible.
	
	\subsection{Why these tests are important}
	
	Each of these predictions is a test of the entire T0 model. If even one of them is clearly wrong, the model must be revised or discarded. That's the strength of science -- theories must face reality.
	
	But if these predictions are confirmed? Then we'd have proof that all of physics actually follows from a single geometric constant. It would be the greatest simplification in the history of science -- comparable to Copernicus' realization that the planets orbit the sun, not the Earth.
	
	\section{Cosmological Implications: An Eternal Universe}
	
	\subsection{No Big Bang -- no end}
	
	Standard cosmology tells a dramatic story: 13.8 billion years ago, the entire universe exploded from an infinitely small, infinitely hot point -- the Big Bang. Since then it's been expanding and will eventually die the heat death.
	
	The T0 model tells a different story: The universe had no beginning and will have no end. It is eternal and static. The apparent expansion is an illusion caused by the energy loss of light on its long journey through space.
	
\section*{Revolutionary}
		Imagine standing at a foggy lake at night. The lights on the other shore appear reddish and faint -- not because they're moving away from you, but because the fog weakens the light and scatters the blue components more strongly than the red ones. 
		
		It's the same in the universe: The "fog" is the omnipresent energy field. Light from distant galaxies loses energy (becomes redder), not because the galaxies are fleeing, but because the photons interact with the $\xipar$ field:
		\begin{equation}
			\frac{dE}{dx} = -\xipar \cdot E \cdot f\left(\frac{E}{E_\xi}\right)
		\end{equation}
% end box revolutionary
	
	\subsection{The cosmic microwave background -- explained differently}
	
	Everywhere in the universe, there's a weak microwave radiation with a temperature of 2.725 Kelvin -- the cosmic microwave background (CMB). The standard explanation: It's the cooled afterglow of the Big Bang.
	
	The T0 model says: It's the equilibrium temperature of the universal energy field. Every field has a natural temperature at which absorption and emission of energy are in equilibrium. For the $\xipar$ field, that's exactly 2.725 K.
	
	It's like the temperature in a cave deep underground -- the same everywhere, not because there was a Big Bang there, but because the system is in thermal equilibrium.
	
	\subsection{Dark matter and dark energy -- superfluous}
	
	One of the greatest mysteries of modern cosmology: 95\% of the universe consists of mysterious dark matter and even more mysterious dark energy that nobody has ever seen. Galaxies rotate too fast (dark matter is needed to hold them together), and the universe is expanding at an accelerated rate (dark energy drives it apart).
	
	The T0 model needs neither:
	- **Galaxy rotation**: The modified gravity through the energy field explains the rotation curves without additional matter
	- **Accelerated expansion**: Is a misinterpretation -- the wavelength-dependent redshift simulates acceleration
	
	It's as if people had searched for centuries for invisible angels pushing the planets in their orbits, until Newton showed that gravity alone suffices.
	
	\subsection{A cyclic universe}
	
	If the universe is eternal, what happens with entropy? The second law of thermodynamics says that disorder always increases. After infinite time, the universe should end in heat death -- everything evenly distributed, no more structures.
	
	The T0 model solves this problem through cycles: Local regions of the universe go through phases of order and disorder, contraction and expansion, but globally everything remains in equilibrium. It's like an eternal ocean -- locally there are waves and whirlpools that arise and disappear, but the ocean as a whole persists.
	
	\section{Summary: A New View of Reality}
	
	\subsection{What the T0 model achieves}
	
	Let's summarize what the T0 model achieves: It reduces all of physics -- from quarks to quasars -- to a single principle. Instead of over twenty free parameters, we need only one geometric constant. Instead of different fields for different particles, there's only one universal energy field. Instead of three incompatible theories, we have a unified framework.
	
	The successes are impressive:
	- The precise prediction of the muon moment (accuracy: 0.1 standard deviations)
	- The explanation of the hierarchy of natural forces without fine-tuning
	- The solution of the quantum gravity problem without new dimensions
	- The elimination of dark matter and dark energy
	- The resolution of all singularities
	
	\subsection{A new philosophy of nature}
	
	But the T0 model is more than just a new theory -- it's a new way of thinking about nature. It tells us that reality is fundamentally simple. The apparent complexity of the world doesn't arise from many different building blocks, but from the diverse patterns of a single field.
	
	It's like with language: With just 26 letters, we can write infinitely many books, from love poems to physics textbooks. Diversity doesn't arise from the diversity of basic elements, but from the diversity of their combinations.
	
\section*{Important}
		The central message of the T0 model: 
		The universe isn't a complicated clockwork of countless gears. It's a symphony -- infinitely rich and diverse, but played by a single instrument: the universal energy field, tuned to the note $\xipar = 4/3 \times 10^{-4}$.
% end box important
	
	\subsection{Open questions and challenges}
	
	Of course, the T0 model isn't perfect. Some challenges remain:
	
	- The detailed geometric justification of all quark parameters and the precise derivation of CKM mixing angles is still incomplete, although the formulas and numerical values are already established
	- The cosmological predictions contradict the established Big Bang model radically
	- Many predictions require measurement precisions at the limit of what's technically possible
	- The philosophical implications (determinism, eternal universe) take getting used to
	
	But these are challenges, not refutations. Every great new theory -- from Copernicus' heliocentrism to Einstein's relativity -- initially had to fight against established ideas.
	
	\subsection{The way forward}
	
	The coming years will be crucial. New experiments will test the T0 model's predictions:
	- Precision measurements of the tau lepton
	- Improved tests of quantum entanglement
	- Detailed spectroscopy of distant galaxies
	- New gravitational wave detectors
	
	Each of these tests is a chance to confirm or refute the model. That's the beauty of science -- nature has the final word.
	
\section*{Formula}
		The ultimate vision of the T0 model in one equation:
		\begin{equation}
			\boxed{\text{Universe} = \xipar \cdot \text{3D Geometry} \cdot \Efield(x,t)}
		\end{equation}
		Three components -- a geometric constant, three-dimensional space, and a universal energy field -- that's all we need to describe all of physical reality.
% end box formula
	
	If the T0 model is correct, we're at the beginning of a new era of physics. An era in which we no longer search for ever new particles and fields, but recognize the elegant simplicity behind the apparent complexity. An era in which the ultimate "Theory of Everything" lies not in higher mathematics and additional dimensions, but in the geometric harmony of the three-dimensional space in which we live.
	
	The search for the fundamental principles of nature is humanity's oldest question. The T0 model offers a possible answer -- elegant, simple, and testable. Whether it's the right answer, only time will tell. But the very possibility that the entire universe follows from a single geometric principle is breathtaking. It would be proof that nature is characterized at its deepest core by mathematical beauty and simplicity.
	


% Bibliography
\begin{thebibliography}{99}
	
	\bibitem{pdg2024}
	Particle Data Group Collaboration (2024). 
	\textit{Review of Particle Physics}. 
	Progress of Theoretical and Experimental Physics, 2024(8), 083C01.
	\url{https://pdg.lbl.gov}
	
	\bibitem{flag2024}
	Aoki, Y., et al. (FLAG Collaboration) (2024). 
	\textit{FLAG Review 2024 of Lattice Results for Low-Energy Constants}. 
	arXiv:2411.04268.
	\url{https://arxiv.org/abs/2411.04268}
	
	\bibitem{fermilab_muon_g2}
	Abi, B., et al. (Muon g-2 Collaboration) (2021). 
	\textit{Measurement of the Positive Muon Anomalous Magnetic Moment to 0.46 ppm}. 
	Physical Review Letters, 126, 141801.
	
	\bibitem{peskin_schroeder}
	Peskin, M. E., \& Schroeder, D. V. (1995). 
	\textit{An Introduction to Quantum Field Theory}. 
	Addison-Wesley.
	
	\bibitem{weinberg_qft}
	Weinberg, S. (1995). 
	\textit{The Quantum Theory of Fields, Vol. I--III}. 
	Cambridge University Press.
	
	\bibitem{griffiths_particle}
	Griffiths, D. (2008). 
	\textit{Introduction to Elementary Particles}. 
	Wiley-VCH.
	
	\bibitem{mandl_shaw}
	Mandl, F., \& Shaw, G. (2010). 
	\textit{Quantum Field Theory (2nd ed.)}. 
	Wiley.
	
	\bibitem{srednicki_qft}
	Srednicki, M. (2007). 
	\textit{Quantum Field Theory}. 
	Cambridge University Press.
	
	\bibitem{t0_fundamentals}
	Pascher, J. (2024). 
	\textit{T0-Theory: Foundations of Time-Mass Duality}. 
	Unpublished manuscript, HTL Leonding.
	
	\bibitem{t0_fine_structure}
	Pascher, J. (2024). 
	\textit{T0-Theory: The Fine Structure Constant}. 
	Unpublished manuscript, HTL Leonding.
	
	\bibitem{t0_neutrinos}
	Pascher, J. (2024). 
	\textit{T0-Theory: Neutrino Masses and PMNS Mixing}. 
	Unpublished manuscript, HTL Leonding.
	
	\bibitem{t0_github}
	Pascher, J. (2024--2025). 
	\textit{T0-Time-Mass-Duality Repository}. 
	GitHub.
	\url{https://github.com/jpascher/T0-Time-Mass-Duality}
	
	\bibitem{lattice_qcd_review}
	Kronfeld, A. S. (2012). 
	\textit{Twenty-first Century Lattice Gauge Theory: Results from the QCD Lagrangian}. 
	Annual Review of Nuclear and Particle Science, 62, 265--284.
	
	\bibitem{neutrino_mixing_pdg}
	Particle Data Group Collaboration (2024). 
	\textit{Neutrino Masses, Mixing, and Oscillations}. 
	PDG Review 2024.
	\url{https://pdg.lbl.gov/2024/reviews/rpp2024-rev-neutrino-mixing.pdf}
	
	\bibitem{higgs_discovery}
	ATLAS and CMS Collaborations (2012). 
	\textit{Observation of a New Particle in the Search for the Standard Model Higgs Boson}. 
	Physics Letters B, 716, 1--29.
	
	\bibitem{Brannen2005}
	C. P. Brannen, ``Estimate of neutrino masses from Koide's relation'', \textit{arXiv:hep-ph/0505028} (2005).
	\url{https://arxiv.org/abs/hep-ph/0505028}
	
	\bibitem{Brannen2006}
	C. P. Brannen, ``Koide Mass Formula for Neutrinos'', \textit{arXiv:0702.0052} (2006).
	\url{http://brannenworks.com/MASSES.pdf}
	
	\bibitem{PhaseVectors2025}
	Anonymous, ``The Koide Relation and Lepton Mass Hierarchy from Phase Vectors'', \textit{rXiv:2507.0040} (2025).
	\url{https://rxiv.org/pdf/2507.0040v1.pdf}
	
	\bibitem{PDG2025}
	Particle Data Group, ``Review of Particle Physics'', \textit{Phys. Rev. D} \textbf{112} (2025) 030001.
	\url{https://pdg.lbl.gov/2025/}
	
	\bibitem{terrell2024}
	Terrell et al. (2024). 
	\textit{Single-Clock Metrology in Nature}. 
	Nature Physics.
	
	\bibitem{hossenfelder2024}
	Hossenfelder, S. (2024). 
	\textit{Single Clock Video Explanation}. 
	YouTube.
	
	\bibitem{hundert1931}
	Hundert (1931). 
	\textit{Reference Work}. 
	Publisher.
	
	\bibitem{terrell2025}
	Terrell et al. (2025). 
	\textit{Advanced Clock Synchronization Methods}. 
	Physical Review Letters.
	
	\bibitem{pascher_t0_2025}
	Pascher, J. (2025). 
	\textit{T0-Theory: Complete Framework and Applications}. 
	Unpublished manuscript, HTL Leonding.
	
	\bibitem{t0qm}
	Pascher, J. (2024). 
	\textit{T0-Theory: Quantum Mechanics Formulation}. 
	Unpublished manuscript, HTL Leonding.
	
	\bibitem{t0anomale}
	Pascher, J. (2024). 
	\textit{T0-Theory: Anomalous Magnetic Moments}. 
	Unpublished manuscript, HTL Leonding.
	
	\bibitem{muong2complete}
	Abi, B., et al. (Muon g-2 Collaboration) (2023). 
	\textit{Complete Measurement of the Positive Muon Anomalous Magnetic Moment}. 
	Physical Review Letters, 131, 161802.
	
	\bibitem{penrose2004}
	Penrose, R. (2004). 
	\textit{The Road to Reality: A Complete Guide to the Laws of the Universe}. 
	Jonathan Cape.
	
	\bibitem{planck1900}
	Planck, M. (1900). 
	\textit{On the Theory of the Energy Distribution Law of the Normal Spectrum}. 
	Verhandlungen der Deutschen Physikalischen Gesellschaft, 2, 237.
	
	\bibitem{T0Theory}
	Pascher, J. (2024). 
	\textit{T0-Theory: Fundamental Principles}. 
	Unpublished manuscript, HTL Leonding.
	
	% Additional bibliography entries for all undefined citations
	\bibitem{6g_roadmap}
	6G Research Consortium (2024).
	\textit{6G Technology Roadmap}.
	Technical Report.
	
	\bibitem{Born2013}
	Born, M. (2013).
	\textit{Einstein's Theory of Relativity}.
	Dover Publications.
	
	\bibitem{Casimir1948}
	Casimir, H. B. G. (1948).
	\textit{On the attraction between two perfectly conducting plates}.
	Proc. Kon. Ned. Akad. Wetensch. B51, 793--795.
	
	\bibitem{Einstein1905}
	Einstein, A. (1905).
	\textit{On the Electrodynamics of Moving Bodies}.
	Annalen der Physik, 17, 891--921.
	
	\bibitem{Feynman2006}
	Feynman, R. P. (2006).
	\textit{QED: The Strange Theory of Light and Matter}.
	Princeton University Press.
	
	\bibitem{Griffiths2017}
	Griffiths, D. J. (2017).
	\textit{Introduction to Electrodynamics (4th ed.)}.
	Cambridge University Press.
	
	\bibitem{Jackson1999}
	Jackson, J. D. (1999).
	\textit{Classical Electrodynamics (3rd ed.)}.
	Wiley.
	
	\bibitem{Mohr2016}
	Mohr, P. J., et al. (2016).
	\textit{CODATA Recommended Values of the Fundamental Physical Constants: 2014}.
	Rev. Mod. Phys. 88, 035009.
	
	\bibitem{Parker2018}
	Parker, R. H., et al. (2018).
	\textit{Measurement of the fine-structure constant as a test of the Standard Model}.
	Science, 360, 191--195.
	
	\bibitem{Planck1900}
	Planck, M. (1900).
	\textit{On the Theory of the Energy Distribution Law of the Normal Spectrum}.
	Verhandlungen der Deutschen Physikalischen Gesellschaft, 2, 237.
	
	\bibitem{Planck2018}
	Planck Collaboration (2018).
	\textit{Planck 2018 results. VI. Cosmological parameters}.
	Astronomy \& Astrophysics, 641, A6.
	
	\bibitem{QFT_T0}
	Pascher, J. (2024).
	\textit{T0-Theory and QFT Connections}.
	Unpublished manuscript, HTL Leonding.
	
	\bibitem{Sommerfeld1916}
	Sommerfeld, A. (1916).
	\textit{On the Quantum Theory of Spectral Lines}.
	Annalen der Physik, 51, 1--94.
	
	\bibitem{T0_Feinstruktur}
	Pascher, J. (2024).
	\textit{T0-Theory: Fine Structure Analysis}.
	Unpublished manuscript, HTL Leonding.
	
	\bibitem{T0_SI}
	Pascher, J. (2024).
	\textit{T0-Theory and SI Units}.
	Unpublished manuscript, HTL Leonding.
	
	\bibitem{T0_fine_structure}
	Pascher, J. (2024).
	\textit{T0-Theory: The Fine Structure Constant}.
	Unpublished manuscript, HTL Leonding.
	
	\bibitem{T0_g2_erweiterung}
	Pascher, J. (2024).
	\textit{T0-Theory: g-2 Extensions}.
	Unpublished manuscript, HTL Leonding.
	
	\bibitem{T0_gravitational_constant}
	Pascher, J. (2024).
	\textit{T0-Theory: Gravitational Constant Derivation}.
	Unpublished manuscript, HTL Leonding.
	
	\bibitem{T0_netze_en}
	Pascher, J. (2024).
	\textit{T0-Theory: Network Structures}.
	Unpublished manuscript, HTL Leonding.
	
	\bibitem{T0_tm_erweiterung}
	Pascher, J. (2024).
	\textit{T0-Theory: Time-Mass Extensions}.
	Unpublished manuscript, HTL Leonding.
	
	\bibitem{Uzan2003}
	Uzan, J.-P. (2003).
	\textit{The fundamental constants and their variation}.
	Rev. Mod. Phys. 75, 403--455.
	
	\bibitem{Weinberg1995}
	Weinberg, S. (1995).
	\textit{The Quantum Theory of Fields, Vol. I}.
	Cambridge University Press.
	
	\bibitem{albrecht1999}
	Albrecht, A. \& Magueijo, J. (1999).
	\textit{A time varying speed of light as a solution to cosmological puzzles}.
	Phys. Rev. D 59, 043516.
	
	\bibitem{alice2023}
	ALICE Collaboration (2023).
	\textit{Recent results from ALICE}.
	CERN-EP-2023-XXX.
	
	\bibitem{analog_optical}
	Smith, J. et al. (2024).
	\textit{Analog optical computing systems}.
	Nature Photonics.
	
	\bibitem{ashtekar2004}
	Ashtekar, A. \& Lewandowski, J. (2004).
	\textit{Background independent quantum gravity}.
	Class. Quantum Grav. 21, R53.
	
	\bibitem{atlas2023}
	ATLAS Collaboration (2023).
	\textit{ATLAS physics results}.
	CERN-PH-EP-2023-XXX.
	
	\bibitem{atlas2023higgs}
	ATLAS Collaboration (2023).
	\textit{Higgs boson measurements}.
	Phys. Rev. Lett.
	
	\bibitem{barbour1999}
	Barbour, J. (1999).
	\textit{The End of Time}.
	Oxford University Press.
	
	\bibitem{barrow1999}
	Barrow, J. D. (1999).
	\textit{Cosmologies with varying light speed}.
	Phys. Rev. D 59, 043515.
	
	\bibitem{becker2007}
	Becker, K. et al. (2007).
	\textit{String Theory and M-Theory}.
	Cambridge University Press.
	
	\bibitem{bell_muon}
	Bennett, G. W., et al. (Muon g-2 Collaboration) (2006).
	\textit{Final report of the E821 muon anomalous magnetic moment measurement}.
	Phys. Rev. D 73, 072003.
	
	\bibitem{bondi1948}
	Bondi, H. \& Gold, T. (1948).
	\textit{The steady-state theory of the expanding universe}.
	Mon. Not. R. Astron. Soc. 108, 252--270.
	
	\bibitem{brewer2019}
	Brewer, S. M. et al. (2019).
	\textit{Al+ Quantum-Logic Clock with Systematic Uncertainty below $10^{-18}$}.
	Phys. Rev. Lett. 123, 033201.
	
	\bibitem{cms2023top}
	CMS Collaboration (2023).
	\textit{Top quark measurements at CMS}.
	JHEP 2023.
	
	\bibitem{cms2024}
	CMS Collaboration (2024).
	\textit{CMS physics results 2024}.
	CERN-PH-EP-2024-XXX.
	
	\bibitem{codata2019}
	Tiesinga, E. et al. (2019).
	\textit{The 2018 CODATA Recommended Values}.
	J. Phys. Chem. Ref. Data.
	
	\bibitem{desi2025}
	DESI Collaboration (2025).
	\textit{DESI 2025 Cosmology Results}.
	arXiv preprint.
	
	\bibitem{differential_optical}
	Wang, X. et al. (2024).
	\textit{Differential optical computing}.
	Optica.
	
	\bibitem{dingle1972}
	Dingle, H. (1972).
	\textit{Science at the Crossroads}.
	Martin Brian \& O'Keeffe.
	
	\bibitem{divalentino2021}
	Di Valentino, E. et al. (2021).
	\textit{In the realm of the Hubble tension}.
	Class. Quantum Grav. 38, 153001.
	
	\bibitem{elnaschie2004}
	El Naschie, M. S. (2004).
	\textit{A review of E infinity theory}.
	Chaos, Solitons \& Fractals, 19, 209--236.
	
	\bibitem{fabrication_heterogeneous}
	Chen, Y. et al. (2024).
	\textit{Heterogeneous photonic integration}.
	Nature Electronics.
	
	\bibitem{fermilab2023}
	Fermilab (2023).
	\textit{Muon g-2 results}.
	Phys. Rev. Lett.
	
	\bibitem{flexible_wafer}
	Kim, S. et al. (2024).
	\textit{Flexible wafer-scale photonics}.
	Science Advances.
	
	\bibitem{francesco1997}
	Di Francesco, P. et al. (1997).
	\textit{Conformal Field Theory}.
	Springer.
	
	\bibitem{hartree1957}
	Hartree, D. R. (1957).
	\textit{The Calculation of Atomic Structures}.
	Wiley.
	
	\bibitem{hhi_6g}
	Fraunhofer HHI (2024).
	\textit{6G Photonic Integration}.
	Technical Report.
	
	\bibitem{hossenfelder2025}
	Hossenfelder, S. (2025).
	\textit{Science without the gobbledygook}.
	YouTube/Blog.
	
	\bibitem{hossenfelder_single_clock_video}
	Hossenfelder, S. (2024).
	\textit{The Single Clock Problem}.
	YouTube.
	
	\bibitem{hoyle1948}
	Hoyle, F. (1948).
	\textit{A new model for the expanding universe}.
	Mon. Not. R. Astron. Soc. 108, 372--382.
	
	\bibitem{integration_microelectronic}
	Liu, A. et al. (2024).
	\textit{Microelectronic photonic integration}.
	IEEE Journal.
	
	\bibitem{jacobson1995}
	Jacobson, T. (1995).
	\textit{Thermodynamics of spacetime}.
	Phys. Rev. Lett. 75, 1260.
	
	\bibitem{kasevich2023}
	Kasevich, M. et al. (2023).
	\textit{Atom interferometry tests}.
	Nature Physics.
	
	\bibitem{lerner2014}
	Lerner, E. J. (2014).
	\textit{An open letter on cosmology}.
	New Scientist.
	
	\bibitem{lisa2017}
	LISA Consortium (2017).
	\textit{Laser Interferometer Space Antenna}.
	ESA Technical Report.
	
	\bibitem{lithium_tantalate}
	Zhang, M. et al. (2024).
	\textit{Thin-film lithium tantalate photonics}.
	Nature Photonics.
	
	\bibitem{lopez2010}
	Lopez-Corredoira, M. (2010).
	\textit{Tests and problems of the standard model in cosmology}.
	Int. J. Mod. Phys. D.
	
	\bibitem{ludlow2015}
	Ludlow, A. D. et al. (2015).
	\textit{Optical atomic clocks}.
	Rev. Mod. Phys. 87, 637.
	
	\bibitem{mach1883}
	Mach, E. (1883).
	\textit{Die Mechanik in ihrer Entwickelung}.
	F.A. Brockhaus.
	
	\bibitem{maldacena1998}
	Maldacena, J. (1998).
	\textit{The large N limit of superconformal field theories}.
	Adv. Theor. Math. Phys. 2, 231--252.
	
	\bibitem{mueller2014}
	Müller, H. et al. (2014).
	\textit{Atom interferometry tests of the gravitational redshift}.
	Phys. Rev. Lett.
	
	\bibitem{mug2_final_2025}
	Muon g-2 Collaboration (2025).
	\textit{Final muon g-2 measurement}.
	Phys. Rev. Lett.
	
	\bibitem{muong2_2023}
	Muon g-2 Collaboration (2023).
	\textit{Updated muon g-2 results}.
	Phys. Rev. Lett.
	
	\bibitem{nathan2024}
	Nathan, A. et al. (2024).
	\textit{Quantum computing advances}.
	Nature.
	
	\bibitem{newell2018}
	Newell, D. B. et al. (2018).
	\textit{The CODATA 2017 values of h, e, k, and $N_A$}.
	Metrologia 55, L13.
	
	\bibitem{nottale1993}
	Nottale, L. (1993).
	\textit{Fractal Space-Time and Microphysics}.
	World Scientific.
	
	\bibitem{on_chip_lithium}
	Wang, C. et al. (2024).
	\textit{On-chip lithium niobate photonics}.
	Nature Communications.
	
	\bibitem{optical_advantages}
	Shastri, B. J. et al. (2024).
	\textit{Advantages of optical computing}.
	Nature Reviews Physics.
	
	\bibitem{pascher2025cmb}
	Pascher, J. (2025).
	\textit{T0-Theory: CMB Analysis}.
	Unpublished manuscript, HTL Leonding.
	
	\bibitem{pascher2025g2}
	Pascher, J. (2025).
	\textit{T0-Theory: g-2 Predictions}.
	Unpublished manuscript, HTL Leonding.
	
	\bibitem{pascher2025qm}
	Pascher, J. (2025).
	\textit{T0-Theory: Quantum Mechanics}.
	Unpublished manuscript, HTL Leonding.
	
	\bibitem{pascher2025si}
	Pascher, J. (2025).
	\textit{T0-Theory: SI Unit System}.
	Unpublished manuscript, HTL Leonding.
	
	\bibitem{pascher2025t0}
	Pascher, J. (2025).
	\textit{T0-Theory: Complete Framework}.
	Unpublished manuscript, HTL Leonding.
	
	\bibitem{pascher:fundamentals}
	Pascher, J. (2024).
	\textit{T0-Theory: Fundamentals}.
	Unpublished manuscript, HTL Leonding.
	
	\bibitem{pascher:g2_rev9}
	Pascher, J. (2024).
	\textit{T0-Theory: g-2 Revision 9}.
	Unpublished manuscript, HTL Leonding.
	
	\bibitem{pascher:geometric_formalism}
	Pascher, J. (2024).
	\textit{T0-Theory: Geometric Formalism}.
	Unpublished manuscript, HTL Leonding.
	
	\bibitem{pascher:ml_addendum}
	Pascher, J. (2024).
	\textit{T0-Theory: Machine Learning Addendum}.
	Unpublished manuscript, HTL Leonding.
	
	\bibitem{pascher:t0_foundations}
	Pascher, J. (2024).
	\textit{T0-Theory: Foundations}.
	Unpublished manuscript, HTL Leonding.
	
	\bibitem{pascher_derivation_beta_2025}
	Pascher, J. (2025).
	\textit{T0-Theory: Derivation of Beta}.
	Unpublished manuscript, HTL Leonding.
	
	\bibitem{pascher_higgs_connection_2025}
	Pascher, J. (2025).
	\textit{T0-Theory: Higgs Connection}.
	Unpublished manuscript, HTL Leonding.
	
	\bibitem{pascher_lagrangian_extended_2025}
	Pascher, J. (2025).
	\textit{T0-Theory: Extended Lagrangian}.
	Unpublished manuscript, HTL Leonding.
	
	\bibitem{pascher_mathematical_structure_2025}
	Pascher, J. (2025).
	\textit{T0-Theory: Mathematical Structure}.
	Unpublished manuscript, HTL Leonding.
	
	\bibitem{pascher_t0_cmb_2025}
	Pascher, J. (2025).
	\textit{T0-Theory: CMB Predictions}.
	Unpublished manuscript, HTL Leonding.
	
	\bibitem{pascher_t0_energie_2025}
	Pascher, J. (2025).
	\textit{T0-Theory: Energy}.
	Unpublished manuscript, HTL Leonding.
	
	\bibitem{pascher_t0_energy_2025}
	Pascher, J. (2025).
	\textit{T0-Theory: Energy Framework}.
	Unpublished manuscript, HTL Leonding.
	
	\bibitem{pascher_t0_theory_2025}
	Pascher, J. (2025).
	\textit{T0-Theory: Complete Theory}.
	Unpublished manuscript, HTL Leonding.
	
	\bibitem{penrose1959}
	Penrose, R. (1959).
	\textit{The apparent shape of a relativistically moving sphere}.
	Proc. Cambridge Phil. Soc. 55, 137--139.
	
	\bibitem{penrose1967}
	Penrose, R. (1967).
	\textit{Twistor algebra}.
	J. Math. Phys. 8, 345--366.
	
	\bibitem{peratt1992}
	Peratt, A. L. (1992).
	\textit{Physics of the Plasma Universe}.
	Springer-Verlag.
	
	\bibitem{peskin1995}
	Peskin, M. E. \& Schroeder, D. V. (1995).
	\textit{An Introduction to Quantum Field Theory}.
	Addison-Wesley.
	
	\bibitem{peskin_schroeder_1995}
	Peskin, M. E. \& Schroeder, D. V. (1995).
	\textit{An Introduction to Quantum Field Theory}.
	Addison-Wesley.
	
	\bibitem{phoquant}
	PhoQuant (2024).
	\textit{Photonic quantum computing}.
	Technical Report.
	
	\bibitem{photonics_ai}
	Wetzstein, G. et al. (2024).
	\textit{Photonics for AI}.
	Nature.
	
	\bibitem{planck1906}
	Planck, M. (1906).
	\textit{The Theory of Heat Radiation}.
	Johann Ambrosius Barth.
	
	\bibitem{planck2018}
	Planck Collaboration (2018).
	\textit{Planck 2018 results}.
	A\&A 641, A6.
	
	\bibitem{polchinski1998}
	Polchinski, J. (1998).
	\textit{String Theory}.
	Cambridge University Press.
	
	\bibitem{qant_nps}
	QANT (2024).
	\textit{Quantum photonics systems}.
	Technical Report.
	
	\bibitem{quantenjahr25}
	Quantenjahr (2025).
	\textit{International Year of Quantum}.
	UNESCO.
	
	\bibitem{recurrent_photonics}
	Tait, A. N. et al. (2024).
	\textit{Recurrent photonic neural networks}.
	Optica.
	
	\bibitem{rf_photonics}
	Capmany, J. \& Novak, D. (2024).
	\textit{Microwave photonics}.
	Nature Photonics.
	
	\bibitem{riess2019}
	Riess, A. G. et al. (2019).
	\textit{Large Magellanic Cloud Cepheid Standards}.
	ApJ 876, 85.
	
	\bibitem{riess2022}
	Riess, A. G. et al. (2022).
	\textit{A Comprehensive Measurement of H0}.
	ApJ 934, L7.
	
	\bibitem{rovelli2004}
	Rovelli, C. (2004).
	\textit{Quantum Gravity}.
	Cambridge University Press.
	
	\bibitem{sciama1953}
	Sciama, D. W. (1953).
	\textit{On the origin of inertia}.
	Mon. Not. R. Astron. Soc. 113, 34--42.
	
	\bibitem{sciencedaily2025}
	ScienceDaily (2025).
	\textit{Physics news}.
	Online.
	
	\bibitem{sm_g2_2025}
	Aoyama, T. et al. (2025).
	\textit{Standard Model prediction for g-2}.
	Phys. Rep.
	
	\bibitem{susskind1995}
	Susskind, L. (1995).
	\textit{The world as a hologram}.
	J. Math. Phys. 36, 6377--6396.
	
	\bibitem{t0_kosmologie}
	Pascher, J. (2024).
	\textit{T0-Theory: Cosmology}.
	Unpublished manuscript, HTL Leonding.
	
	\bibitem{terrell1959}
	Terrell, J. (1959).
	\textit{Invisibility of the Lorentz contraction}.
	Phys. Rev. 116, 1041--1045.
	
	\bibitem{terrell_single_clock_nature_2024}
	Terrell, J. et al. (2024).
	\textit{Single clock precision measurements}.
	Nature Physics.
	
	\bibitem{tfln_foundry}
	TFLN Foundry (2024).
	\textit{Thin-film lithium niobate foundry services}.
	Technical Specifications.
	
	\bibitem{thiemann2007}
	Thiemann, T. (2007).
	\textit{Modern Canonical Quantum General Relativity}.
	Cambridge University Press.
	
	\bibitem{thz_epfl}
	EPFL (2024).
	\textit{Terahertz photonics research}.
	Technical Report.
	
	\bibitem{unnikrishnan2004}
	Unnikrishnan, C. S. (2004).
	\textit{On Einstein's resolution of the twin clock paradox}.
	Current Science, 86, 704--709.
	
	\bibitem{verlinde2011}
	Verlinde, E. (2011).
	\textit{On the origin of gravity and the laws of Newton}.
	JHEP 2011, 29.
	
	\bibitem{video2025}
	Video (2025).
	\textit{Physics video explanation}.
	YouTube.
	
	\bibitem{weinberg1995}
	Weinberg, S. (1995).
	\textit{The Quantum Theory of Fields}.
	Cambridge University Press.
	
	\bibitem{weiskopf2000}
	Weiskopf, D. (2000).
	\textit{Visualization of special relativity}.
	PhD thesis, University of Tübingen.
	
	\bibitem{wheeler1990}
	Wheeler, J. A. (1990).
	\textit{A Journey into Gravity and Spacetime}.
	Scientific American Library.
	
	\bibitem{wiki_bell}
	Wikipedia (2024).
	\textit{Bell's theorem}.
	Online encyclopedia.
	
	\bibitem{zwicky1929}
	Zwicky, F. (1929).
	\textit{On the red shift of spectral lines through interstellar space}.
	Proc. Natl. Acad. Sci. 15, 773--779.

\end{thebibliography}


\end{document}
