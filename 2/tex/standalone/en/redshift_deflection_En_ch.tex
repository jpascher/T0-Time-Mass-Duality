\documentclass[11pt,a4paper]{article}
\usepackage[a4paper,margin=2cm]{geometry}
\usepackage[utf8]{inputenc}
\usepackage[english]{babel}
\usepackage{lmodern}
\renewcommand{\familydefault}{\sfdefault}

\usepackage{amsmath,amssymb,amsthm}
\usepackage{graphicx}
\usepackage[unicode,pdfencoding=auto,hypertexnames=false]{hyperref}
\usepackage{booktabs}
\usepackage{longtable}
\usepackage{array}
\usepackage{siunitx}
\usepackage{fancyhdr}
\usepackage{float}
\usepackage{tikz}
% tcolorbox removed for standalone
% tcbset removed
\tikzset{
  t0blue/.style={draw=blue,fill=blue!10},
  t0red/.style={draw=red,fill=red!10},
  t0green/.style={draw=green!50!black,fill=green!10},
  t0orange/.style={draw=orange,fill=orange!10},
}
\usepackage{setspace}
\usepackage{enumitem}
\usepackage{adjustbox}
\usepackage{xcolor}

% Define colors for xcolor package
\definecolor{t0green}{RGB}{34,139,34}
\definecolor{t0blue}{RGB}{0,0,255}
\definecolor{t0red}{RGB}{255,0,0}
\definecolor{t0orange}{RGB}{255,165,0}

% Define custom column types for tables
\newcolumntype{L}[1]{>{\raggedright\arraybackslash}p{#1}}
\newcolumntype{C}[1]{>{\centering\arraybackslash}p{#1}}
\newcolumntype{R}[1]{>{\raggedleft\arraybackslash}p{#1}}

\setlength{\parindent}{0pt}
\setlength{\parskip}{6pt}

\hypersetup{
  colorlinks=true,
  linkcolor=blue,
  citecolor=blue,
  urlcolor=blue
}
\pagestyle{fancy}
\setlength{\headheight}{28pt}

\newcommand{\checkmarkx}{\checkmark}
\newcommand{\warningx}{\textbf{!}}

% Makros aus Einzel-Dokumenten (Fallback-Definitionen)
\newcommand{\mytimes}{\times}
\newcommand{\myapprox}{\approx}
\newcommand{\mysim}{\sim}
\newcommand{\myomega}{\omega}
\newcommand{\mypi}{\pi}
\newcommand{\myrightarrow}{\rightarrow}
\newcommand{\mypropto}{\propto}
\newcommand{\deltafield}{\delta\phi}
\newcommand{\xipar}{\xi}
\newcommand{\xiT}{\xi}
\newcommand{\lambdah}{\lambda_h}

% Additional macros used in chapter files
\newcommand{\Kfrak}{K_{\text{frak}}}  % Fractal correction factor
\newcommand{\Dfrak}{D_f}              % Fractal dimension
\newcommand{\betapar}{\beta}          % T0 beta parameter
\newcommand{\alphapar}{\alpha}        % T0 alpha parameter
\newcommand{\Efield}{E}               % Energy field
% Note: checkmarkxa/warningxa are variants used in auto-generated chapter files
\newcommand{\checkmarkxa}{\checkmark}
\newcommand{\warningxa}{\textbf{!}}

% Additional T0-specific macros
\newcommand{\xigeom}{\xi_{\text{geom}}}  % Geometric xi
\newcommand{\lP}{\ell_P}                  % Planck length
\newcommand{\rzero}{r_0}                  % Characteristic radius
\newcommand{\xirat}{\xi_{\text{rat}}}     % Xi ratio
\newcommand{\tzero}{t_0}                  % Characteristic time
\newcommand{\natunits}{\text{(nat. units)}}  % Natural units annotation
\newcommand{\myRightarrow}{\Rightarrow}   % Arrow variant
\newcommand{\Lag}{\mathcal{L}}            % Lagrangian

% Physics macros used in chapter files
\newcommand{\CQCD}{C_{\text{QCD}}}        % QCD correction
\newcommand{\EP}{E_P}                     % Planck energy
\newcommand{\Ee}{E_e}                     % Electron energy
\newcommand{\Emu}{E_\mu}                  % Muon energy
\newcommand{\Exi}{E_\xi}                  % Xi energy
\newcommand{\Ezero}{E_0}                  % Characteristic energy
\newcommand{\Hubble}{H}                   % Hubble constant
\newcommand{\Kspec}{K_{\text{spec}}}      % Spectral correction
\newcommand{\Lambdat}{\Lambda_t}          % Time-related cosmological constant
\newcommand{\Leff}{\mathcal{L}_{\text{eff}}}  % Effective Lagrangian
\newcommand{\Lorentz}{\mathcal{L}}        % Lorentz symbol
\newcommand{\Lxi}{L_\xi}                  % Xi length
\newcommand{\Tfield}{T}                   % Time field
\newcommand{\Weyl}{W}                     % Weyl tensor/symbol
\newcommand{\alphaEMSI}{\alpha_{\text{EM,SI}}}  % EM alpha in SI
\newcommand{\alphaEMnat}{\alpha_{\text{EM,nat}}}  % EM alpha in natural units
\newcommand{\alphaem}{\alpha_{\text{em}}} % Electromagnetic alpha
\newcommand{\betaTSI}{\beta_{T,\text{SI}}}  % Beta in SI
\newcommand{\betaTnat}{\beta_{T,\text{nat}}}  % Beta in natural units
\newcommand{\deltam}{\delta m}            % Mass difference
\newcommand{\phiT}{\phi_T}                % T-field phi
\newcommand{\tP}{t_P}                     % Planck time
\newcommand{\rhoCMB}{\rho_{\text{CMB}}}   % CMB density
\newcommand{\rhoCasimir}{\rho_{\text{Casimir}}}  % Casimir density

% Table formatting
\usepackage{multirow}

% Additional physics macros
\newcommand{\Riem}{\mathcal{R}}           % Riemann tensor
\newcommand{\ZPinch}{Z_{\text{pinch}}}    % Z-pinch
\newcommand{\SynchPower}{P_{\text{synch}}} % Synchrotron power
\newcommand{\Rzero}{R_0}                  % Characteristic radius
\newcommand{\alphafine}{\alpha}           % Fine structure constant
\newcommand{\Etau}{E_\tau}                % Tau energy
\newcommand{\deltaE}{\delta E}            % Energy deviation
\newcommand{\EPlanck}{E_P}                % Planck energy
\newcommand{\pichar}{\pi}                 % Pi character
\newcommand{\alphaWSI}{\alpha_{W,\text{SI}}}  % Wien alpha in SI
\newcommand{\alphaWnat}{\alpha_{W,\text{nat}}}  % Wien alpha in natural units

% Einfache abstract-Umgebung für Kapitel:
\newenvironment{abstract}{%
  \begin{center}\bfseries Abstract\end{center}\small
}{\par}


\title{redshift deflection En}
\author{J. Pascher}
\date{\today}

\begin{document}
\maketitle

\section*{Redshift Deflection (redshift deflection)}

	\begin{abstract}
		The T0 model explains cosmological redshift through $\xi$-field energy loss during photon propagation, without requiring spatial expansion or distance measurements. This mechanism predicts a wavelength-dependent redshift $z \propto \lambda$ that can be tested with spectroscopic observations of cosmic objects. Using the universal constant $\xiconst$ and measured masses of astronomical objects, the theory provides model-independent tests distinguishable from standard cosmology. The $\xi$-field also explains the cosmic microwave background temperature ($T_{\text{CMB}} = 2.7255$ K) in a static, eternally existing universe, as detailed in \cite{pascher_cmb_en,pascher_cosmos_en}.
	\end{abstract}
	
	
	\section{Introduction}
	
	\subsection{Universal -Constant}
	
	The T0-theory is based on a single fundamental constant \cite{pascher_lagrangian_en}:
	\begin{equation}
		\boxed{\xiconst}
	\end{equation}
	
	This value arises from geometric considerations and determines all fundamental interactions in the universe \cite{pascher_gravitation_en}. The geometric origin stems from the ratio of characteristic scales in the universe, connecting quantum mechanics to cosmology through a single parameter.
	
	\subsection{$\xi$-Field Structure}
	
	The $\xi$-field permeates the entire universe and manifests in three fundamental forms:
	\begin{enumerate}
		\item \textbf{Cosmic Microwave Background (CMB)}: Free $\xi$-field radiation at $T = 2.7255$ K
		\item \textbf{Casimir Vacuum}: Geometrically constrained $\xi$-field between conducting plates
		\item \textbf{Gravitational Interaction}: $\xi$-field coupling to matter determines $G$
	\end{enumerate}
	
	The relationship between these manifestations is given by:
	\begin{equation}
		\frac{|\rho_{\text{Casimir}}|}{\rho_{\text{CMB}}} = \frac{\pi^2}{240 \xi} = \frac{\pi^2 \times 10^4}{320} \approx 308
	\end{equation}
	
	\section{Energy Loss Mechanism}
	
	\subsection{Photon--Field Interaction}
	
\section*{Principle}
		Photons propagating through the omnipresent $\xi$-field lose energy according to:
		\begin{equation}
			\frac{dE}{dx} = -\xi \cdot \xicoupling \cdot E
		\end{equation}
		where $\xicoupling$ is the energy-dependent coupling function.
% end box principle
	
	For the linear coupling case:
	\begin{equation}
		f\left(\frac{E}{\Exi}\right) = \frac{E}{\Exi}
	\end{equation}
	
	This yields the simplified energy loss equation:
	\begin{equation}
		\frac{dE}{dx} = -\frac{\xi E^2}{\Exi}
	\end{equation}
	
	\subsection{Energy-to-Wavelength Conversion}
	
	Since $E = \frac{hc}{\lambda}$ (or $E = \frac{1}{\lambda}$ in natural units, $\hbar = c = 1$), we can express the energy loss in terms of wavelength. Substituting $E = \frac{1}{\lambda}$:
	\begin{equation}
		\frac{d(1/\lambda)}{dx} = -\frac{\xi}{\Exi} \cdot \frac{1}{\lambda^2}
	\end{equation}
	
	Rearranging for wavelength evolution:
	\begin{equation}
		\frac{d\lambda}{dx} = \frac{\xi \lambda^2}{\Exi}
	\end{equation}
	
	\section{Redshift Formula Derivation}
	
	\subsection{Integration for Small -Effects}
	
	For the wavelength evolution equation:
	\begin{equation}
		\frac{d\lambda}{dx} = \frac{\xi \lambda^2}{\Exi}
	\end{equation}
	
	Separating variables and integrating:
	\begin{equation}
		\int_{\lambdazero}^{\lambda} \frac{d\lambda'}{\lambda'^2} = \frac{\xi}{\Exi} \int_0^x dx'
	\end{equation}
	
	This yields:
	\begin{equation}
		\frac{1}{\lambdazero} - \frac{1}{\lambda} = \frac{\xi x}{\Exi}
	\end{equation}
	
	Solving for the observed wavelength:
	\begin{equation}
		\lambda = \frac{\lambdazero}{1 - \frac{\xi x \lambdazero}{\Exi}}
	\end{equation}
	
	\subsection{Redshift Definition and Formula}
	
\section*{Formula}
		Redshift definition:
		\begin{equation}
			z = \frac{\lambda_{\text{observed}} - \lambda_{\text{emitted}}}{\lambda_{\text{emitted}}} = \frac{\lambda}{\lambdazero} - 1
		\end{equation}
% end box formula
	
	For small $\xi$-effects where $\frac{\xi x \lambdazero}{\Exi} \ll 1$, we can expand:
	\begin{equation}
		z \approx \frac{\xi x \lambdazero}{\Exi} = \frac{\xi x}{\Exi / (\hbar c)} \cdot \lambdazero \quad (\text{in conventional units})
	\end{equation}
	
\section*{Important}
\section*{Key T0 Prediction: Wavelength-Dependent Redshift}
		\begin{equation}
			\boxed{z(\lambdazero) = \frac{\xi x}{\Exi} \cdot \lambdazero \quad (\text{natural units, } \hbar = c = 1)}
		\end{equation}
		This wavelength dependence is the KEY DISTINGUISHING FEATURE from standard cosmology:
		\begin{itemize}
			\item Standard cosmology: $z$ is the same for ALL wavelengths from the same source
			\item T0 theory: $z$ varies with wavelength - testable prediction!
		\end{itemize}
		In conventional units, $\Exi$ scales with $\hbar c \approx 197.3$ MeV$\cdot$fm, so $\Exi \approx 1.5$ GeV corresponds to $\Exi / (\hbar c) \approx 7500$ m$^{-1}$, ensuring dimensional consistency.
% end box important
	
	\subsection{Consistency with Observed Redshifts}
	Current observations neither confirm nor refute the wavelength dependence due to measurement limitations at the detection threshold. The wavelength-dependent redshift, given by $z \propto \frac{\xi x}{\Exi} \cdot \lambdazero$, explains observed cosmological redshifts in combination with complementary effects such as Doppler shifts, gravitational redshift, and nonlinear $\xi$-field interactions. For high-redshift objects ($z > 10$), such as those observed by JWST \cite{jwst_early}, the coupling function $f\left(\frac{E}{\Exi}\right)$ may contain higher-order terms ensuring consistency with observations without cosmic expansion. Future spectroscopic tests, as described in Section \ref{redshift_deflec:L-T0_Energie-0321}, will provide definitive validation or refutation of this mechanism.
	
	\section{Frequency-Based Formulation}
	
	\subsection{Frequency Energy Loss}
	
	Since $E = h\nu$, the energy loss equation becomes:
	\begin{equation}
		\frac{d(h\nu)}{dx} = -\frac{\xi (h\nu)^2}{\Exi}
	\end{equation}
	
	Simplifying:
	\begin{equation}
		\frac{d\nu}{dx} = -\frac{\xi h \nu^2}{\Exi}
	\end{equation}
	
	\subsection{Frequency Redshift Formula}
	
	Integrating the frequency evolution:
	\begin{equation}
		\int_{\nuzero}^{\nu} \frac{d\nu'}{\nu'^2} = -\frac{\xi h}{\Exi} \int_0^x dx'
	\end{equation}
	
	This yields:
	\begin{equation}
		\frac{1}{\nu} - \frac{1}{\nuzero} = \frac{\xi h x}{\Exi}
	\end{equation}
	
	Therefore:
	\begin{equation}
		\nu = \frac{\nuzero}{1 + \frac{\xi h x \nuzero}{\Exi}}
	\end{equation}
	
\section*{Formula}
		Frequency redshift:
		\begin{equation}
			z = \frac{\nuzero}{\nu} - 1 \approx \frac{\xi h x \nuzero}{\Exi} \quad (\text{natural units, } h = 1; \text{conventional units, } h = \hbar)
		\end{equation}
% end box formula
	
\section*{Important}
		Since $\nu = \frac{c}{\lambda}$, we have $h\nu = \frac{hc}{\lambda}$, confirming:
		\begin{equation}
			z \propto \nu \propto \frac{1}{\lambda}
		\end{equation}
		\textbf{Higher-frequency photons show greater redshift!} In conventional units, $\Exi$ scales with $\hbar c$ to maintain dimensional consistency.
% end box important
	
	\section{Observable Predictions without Distance Assumptions}
	
	\subsection{Spectral Line Ratios}
	
	Different atomic transitions should show different redshifts according to their wavelengths:
	\begin{equation}
		\frac{z(\lambda_1)}{z(\lambda_2)} = \frac{\lambda_1}{\lambda_2}
	\end{equation}
	
\section*{Experiment}
\section*{Hydrogen Line Test:}
		\begin{itemize}
			\item Lyman-$\alpha$ (121.6 nm) vs. H$\alpha$ (656.3 nm)
			\item Predicted ratio: $\frac{z_{\text{Ly}\alpha}}{z_{\text{H}\alpha}} = \frac{121.6}{656.3} = 0.185$
			\item \textbf{Standard cosmology predicts: 1.000}
		\end{itemize}
% end box experiment
	
	\subsection{Frequency-Dependent Effects}
	
	For radio vs. optical observations of the same cosmic object:
	\begin{itemize}
		\item 21 cm line: $\lambda = 0.21$ m
		\item H$\alpha$ line: $\lambda = 6.563 \times 10^{-7}$ m
		\item Predicted ratio: $\frac{z_{21\text{cm}}}{z_{\text{H}\alpha}} = \frac{\lambda_{21\text{cm}}}{\lambda_{\text{H}\alpha}} = \frac{0.21}{6.563 \times 10^{-7}} = 3.2 \times 10^5$
	\end{itemize}
	
	This enormous difference should be detectable even with current technology if the T0 mechanism is correct.
	
	\section{Experimental Tests via Spectroscopy}
	\label{redshift_deflec:L-T0_Energie-0321}
	
	\subsection{Multi-Wavelength Observations}
	
\section*{Experiment}
\section*{Simultaneous Multiband Spectroscopy:}
		\begin{enumerate}
			\item Observe quasar/galaxy simultaneously in UV, optical, IR
			\item Measure redshift from different spectral lines
			\item Test whether $z \propto \lambda$ relationship holds
			\item Compare with standard cosmology prediction ($z = \text{constant}$)
		\end{enumerate}
% end box experiment
	
	\subsection{Radio vs. Optical Redshift}
	
\section*{Experiment}
\section*{21cm vs. Optical Line Comparison:}
		\begin{itemize}
			\item \textbf{Radio surveys}: ALFALFA, HIPASS (21cm redshifts)
			\item \textbf{Optical surveys}: SDSS, 2dF (H$\alpha$, H$\beta$ redshifts)
			\item \textbf{Method}: Compare objects observed in both surveys
			\item \textbf{Prediction}: $z_{21\text{cm}} \neq z_{\text{optical}}$ (T0) vs. $z_{21\text{cm}} = z_{\text{optical}}$ (Standard)
		\end{itemize}
% end box experiment
	
	\section{Advantages over Standard Cosmology}
	
	\subsection{Model-Independent Approach}
	
	\begin{longtable}{lcc}
		\caption{T0-Theory vs. Standard Cosmology} \\
		\toprule
		\textbf{Aspect} & \textbf{T0-Theory} & \textbf{$\Lambda$CDM} \\
		\midrule
		\endfirsthead
		\multicolumn{3}{c}%
		{{\tablename\ \thetable{} -- continued from previous page}} \\
		\toprule
		\textbf{Aspect} & \textbf{T0-Theory} & \textbf{$\Lambda$CDM} \\
		\midrule
		\endhead
		\bottomrule
		\endfoot
		\bottomrule
		\endlastfoot
		Universal constant & $\xi = 4/3 \times 10^{-4}$ & None \\
		Dark energy required & No & Yes (70\%) \\
		Dark matter required & No & Yes (25\%) \\
		Number of parameters & 1 & 6+ \\
		Hubble tension & Resolved & Unresolved \\
		JWST observations & Consistent & Problematic \\
		Big Bang singularity & None & Required \\
		Horizon problem & None & Unresolved \\
		Flatness problem & Natural & Fine-tuning required \\
	\end{longtable}
	
	\subsection{Unified Explanations}
	
	The single $\xi$-constant explains:
	\begin{enumerate}
		\item \textbf{Gravitational constant}: $G = \frac{\xi^2 c^3}{16\pi m_p^2}$
		\item \textbf{CMB temperature}: $T_{\text{CMB}} = \frac{16}{9} \xi^2 \times E_\xi$
		\item \textbf{Casimir effect}: Related to $\xi$-field vacuum
		\item \textbf{Cosmological redshift}: Energy loss through $\xi$-field
		\item \textbf{Particle masses}: Geometric resonances in $\xi$-field
		\item \textbf{Fine structure constant}: $\alpha = (4/3)^3 \approx 1/137$
		\item \textbf{Muon anomalous magnetic moment}: $a_\mu = \frac{\xi}{2\pi} \left(\frac{E_\mu}{E_e}\right)^2$
	\end{enumerate}
	
	\section{Critical Assessment: Wavelength Dependence at the Detection Threshold}
	\label{redshift_deflec:L-redshift_deflection-0891}
	
	\subsection{Current Experimental Status and Measurement Limitations}
	
	The T0 theory's prediction of wavelength-dependent redshift represents one of its most distinctive and testable features. However, the current experimental situation is complex and requires careful analysis.
	
	\subsubsection{Precision at the Critical Boundary}
	
	Current spectroscopic measurements achieve precision of $\Delta z/z \approx 10^{-4}$ to $10^{-5}$, while the T0 effect with $\xi = 4/3 \times 10^{-4}$ predicts variations of the same magnitude. This places us precisely at the detection threshold - a critical situation where neither confirmation nor refutation is currently possible.
	
	For typical cosmic objects with $\xiconst$, the relative difference in redshift between two spectral lines:
	\begin{equation}
		\frac{\Delta z}{z} = \left| \frac{z(\lambda_1) - z(\lambda_2)}{z(\lambda_{\text{mean}})} \right| = \left| \frac{\lambda_1 - \lambda_2}{\lambda_{\text{mean}}} \right| \times \xi \approx 10^{-4} \text{ to } 10^{-5}
	\end{equation}
	
\section*{Important}
		This wavelength effect is at the limit of current spectroscopic precision but potentially detectable with next-generation instruments:
		\begin{itemize}
			\item Extremely Large Telescope (ELT): $\Delta z/z \approx 10^{-6}$ to $10^{-7}$
			\item James Webb Space Telescope (JWST): Extended IR spectroscopy
			\item Square Kilometre Array (SKA): Precise 21cm measurements
		\end{itemize}
% end box important
	
	\subsection{Future Experimental Outcomes and Their Implications}
	
	The next generation of instruments will achieve precision $\Delta z/z \approx 10^{-6}$ to $10^{-7}$, finally enabling definitive tests. Two primary outcomes are possible:
	
	\subsubsection{Primary Outcome A: Wavelength Dependence CONFIRMED}
	\label{redshift_deflec:L-redshift_deflection-0892}
	
	If measurements detect $z \propto \lambda_0$ as predicted:
	
\section*{Immediate Implications:}
	\begin{itemize}
		\item \textbf{Fundamental validation} of T0 theory's core mechanism
		\item \textbf{Paradigm shift}: Redshift from energy loss, not expansion
		\item \textbf{New physics confirmed}: Photon-$\xi$-field interaction is real
		\item \textbf{Cosmology revolution}: Static universe model validated
	\end{itemize}
	
\section*{Required Follow-up Measurements:}
	\begin{itemize}
		\item Precise determination of proportionality constant to verify $\xi = 4/3 \times 10^{-4}$
		\item Distance dependence to confirm linear relationship
		\item Search for deviations at extreme wavelengths (gamma-ray to radio)
	\end{itemize}
	
	\subsubsection{Primary Outcome B: Wavelength Dependence NOT DETECTED}
	\label{redshift_deflec:L-redshift_deflection-0893}
	
	If no wavelength dependence is found even at $10^{-6}$ precision, two distinct sub-scenarios must be considered:
	
	\subsection{Sub-Scenario B1: Fundamental T0 Mechanism Incorrect}
	\label{redshift_deflec:L-redshift_deflection-0894}
	
	\textbf{Interpretation:} The nonlinear energy loss mechanism $dE/dx = -\xi E^2/E_\xi$ is fundamentally wrong.
	
\section*{Required Theoretical Adaptation:}
	\begin{itemize}
		\item \textbf{Modified energy loss equation:} Replace with linear form
		\begin{equation}
			\frac{dE}{dx} = -\xi_{eff} \cdot E
		\end{equation}
		This yields $z = e^{\xi_{eff} x} - 1$, independent of $\lambda_0$
		
		\item \textbf{Reinterpretation of $E_\xi$:} No longer a fundamental energy scale for photon interaction
		
		\item \textbf{Alternative coupling function:} Instead of $f(E/E_\xi) = E/E_\xi$, use
		\begin{equation}
			f(E/E_\xi) = \text{constant} = \xi_0
		\end{equation}
	\end{itemize}
	
\section*{What Remains Valid:}
	\begin{itemize}
		\item Geometric constant $\xi = 4/3 \times 10^{-4}$ (from tetrahedron quantization)
		\item Gravitational constant derivation: $G = \xi^2 c^3/(16\pi m_p^2)$
		\item Particle mass ratios from geometric quantum numbers
		\item Muon g-2 anomaly prediction
		\item CMB temperature explanation
	\end{itemize}
	
\section*{What Changes:}
	\begin{itemize}
		\item Loss of unique T0 signature (wavelength dependence)
		\item Harder to distinguish from modified $\Lambda$CDM models
		\item Photon propagation mechanism simplified
		\item Need alternative tests to validate static universe model
	\end{itemize}
	
	\subsection{Sub-Scenario B2: Wavelength Dependence Exists but is COMPENSATED}
	\label{redshift_deflec:L-redshift_deflection-0895}
	
	\textbf{Interpretation:} The T0 mechanism is correct, but compensating effects mask the wavelength dependence.
	
	\subsubsection{Detailed Compensation Mechanisms}
	
\section*{Formula}
		The T0 wavelength dependence could be masked by:
		\begin{enumerate}
			\item \textbf{IGM Dispersion}: $z_{\text{IGM}} \propto -\lambda^{-2}$ (opposes $z_{\text{T0}} \propto +\lambda$)
			\item \textbf{Gravitational Layering}: $z_{\text{grav}}(r(\lambda))$ varies with emission depth
			\item \textbf{Nonlinear Corrections}: Higher-order terms $\propto (\xi x \lambda_0/E_\xi)^n$ flatten response
		\end{enumerate}
		Net effect: $z_{\text{observed}} = z_{\text{T0}} + z_{\text{comp}} \approx$ constant
% end box formula
	
\section*{1. Intergalactic Medium (IGM) Dispersion Compensation:}
	\begin{equation}
		z_{\text{observed}} = z_{\text{T0}}(\lambda) + z_{\text{IGM}}(\lambda) + z_{\text{other}}
	\end{equation}
	
	The IGM could provide inverse wavelength dependence:
	\begin{itemize}
		\item T0 effect: $z_{\text{T0}} \propto +\lambda$ (longer wavelengths more redshifted)
		\item IGM effect: $z_{\text{IGM}} \propto -\lambda^{-2}$ (plasma dispersion favors shorter wavelengths)
		\item Net result: $z_{\text{observed}} \approx$ constant
	\end{itemize}
	
	\textbf{Physical mechanism:} Free electrons in IGM create frequency-dependent refractive index:
	\begin{equation}
		n(\omega) = 1 - \frac{\omega_p^2}{2\omega^2} \implies z_{\text{IGM}} \propto -\frac{1}{\lambda^2}
	\end{equation}
	
	For appropriate IGM density, this could precisely cancel T0's linear $\lambda$ dependence.
	
\section*{2. Source-Dependent Compensation:}
	
	Different spectral lines originate at different depths in stellar/galactic atmospheres:
	\begin{itemize}
		\item \textbf{UV lines} (e.g., Lyman-$\alpha$): Outer atmosphere, lower gravity, less gravitational redshift
		\item \textbf{Optical lines} (e.g., H-$\alpha$): Mid-photosphere, moderate gravitational field
		\item \textbf{IR lines}: Deep atmosphere, stronger gravitational redshift
	\end{itemize}
	
	This creates an effective compensation:
	\begin{equation}
		z_{\text{total}} = z_{\text{T0}}(\lambda) + z_{\text{grav}}(r(\lambda)) \approx \text{constant}
	\end{equation}
	
\section*{3. Nonlinear Field Corrections:}
	
	The complete T0 solution might include self-compensation terms:
	\begin{equation}
		z = \frac{\xi x \lambda_0}{E_\xi}\left[1 - \alpha\left(\frac{\xi x \lambda_0}{E_\xi}\right) + \beta\left(\frac{\xi x \lambda_0}{E_\xi}\right)^2 + ...\right]
	\end{equation}
	
	For specific values of $\alpha$ and $\beta$, the wavelength dependence could flatten at cosmological distances while remaining visible locally.
	
	\subsubsection{How to Test for Compensation}
	
\section*{Observational Strategies:}
	\begin{enumerate}
		\item \textbf{Distance-dependent studies:}
		\begin{itemize}
			\item Measure $\Delta z/\Delta\lambda$ at different distances
			\item Compensation effects should vary with distance
			\item T0 effect linear with distance, compensation may not be
		\end{itemize}
		
		\item \textbf{Environment-dependent measurements:}
		\begin{itemize}
			\item Compare objects in voids vs. clusters
			\item Different IGM densities → different compensation
			\item Clean sight lines vs. dense regions
		\end{itemize}
		
		\item \textbf{Source-type variations:}
		\begin{itemize}
			\item Quasars vs. galaxies vs. supernovae
			\item Different emission mechanisms
			\item Different atmospheric structures
		\end{itemize}
		
		\item \textbf{Extreme wavelength tests:}
		\begin{itemize}
			\item Gamma-ray bursts (shortest $\lambda$)
			\item Radio galaxies (longest $\lambda$)
			\item Compensation may break down at extremes
		\end{itemize}
	\end{enumerate}
	
	\subsubsection{Required Theoretical Adaptations for B2}
	
	If compensation is confirmed, the T0 theory needs:
	
\section*{1. Extended Framework:}
	\begin{equation}
		z_{\text{total}} = z_{\text{T0}}(\lambda, x) + \sum_i z_{\text{comp},i}(\lambda, x, \rho, T, ...)
	\end{equation}
	
\section*{2. Environmental Parameters:}
	\begin{itemize}
		\item IGM density profile: $\rho_{\text{IGM}}(x)$
		\item Temperature distribution: $T(x)$
		\item Magnetic field effects: $B(x)$
	\end{itemize}
	
\section*{3. Refined Predictions:}
	\begin{itemize}
		\item Residual wavelength dependence in specific conditions
		\item Optimal observation strategies to reveal T0 effect
		\item Predictions for when compensation fails
	\end{itemize}
	
	\subsection{The Suspicious Coincidence}
	
	The fact that the predicted T0 effect magnitude ($\xi = 4/3 \times 10^{-4}$) places the wavelength dependence \textit{exactly} at the current detection threshold deserves special attention:
	
	\begin{itemize}
		\item \textbf{Probability argument}: The chance that a fundamental constant would randomly place an effect precisely at our current technological limit is extremely small
		\item \textbf{Historical precedent}: Similar "coincidences" in physics often indicated real effects masked by complications (e.g., solar neutrino problem)
		\item \textbf{Anthropic consideration}: No anthropic reason constrains $\xi$ to this specific value
		\item \textbf{Most likely interpretation}: The effect exists but is partially compensated, keeping it just below clear detection
	\end{itemize}
	
\section*{Experiment}
		To resolve whether this coincidence is meaningful:
		\begin{enumerate}
			\item Compare measurements from different epochs as technology improves
			\item Look for systematic trends in "non-detections" near the threshold
			\item Search for environmental correlations in marginal detections
			\item Perform meta-analysis of all wavelength-dependence studies
		\end{enumerate}
% end box experiment
	
	\subsection{Decision Tree for Future Observations}
	
	\begin{center}
		\begin{tabular}{l}
			\textbf{High-precision measurement} ($\Delta z/z < 10^{-6}$) \\
			\midrule
			$\downarrow$ \\
			\textbf{Question:} Wavelength dependence detected? \\
			\midrule
			\textbf{YES} $\rightarrow$ T0 CONFIRMED (Outcome A) \\
			\hspace{1cm} • Measure $\xi$ precisely \\
			\hspace{1cm} • Test distance dependence \\
			\midrule
			\textbf{NO} $\rightarrow$ Further investigation required \\
			\hspace{1cm} \textbf{Test:} Universal across all conditions? \\
			\hspace{2cm} YES $\rightarrow$ B1: Modify T0 (linear mechanism) \\
			\hspace{2cm} NO $\rightarrow$ B2: Compensation (refine theory)
		\end{tabular}
	\end{center}
	
	\subsection{Conclusion: A Theory at the Crossroads}
	
	The T0 theory stands at a critical juncture. The wavelength-dependent redshift prediction will either:
	
	\begin{itemize}
		\item \textbf{Revolutionize cosmology} if confirmed (Outcome A)
		\item \textbf{Require simplification} if absent (Sub-scenario B1)
		\item \textbf{Reveal hidden complexity} if compensated (Sub-scenario B2)
	\end{itemize}
	
\section*{Important}
		\textbf{The remarkably precise coincidence that $\xi = 4/3 \times 10^{-4}$ places the effect exactly at current detection limits suggests this is not accidental.} The most likely scenario may be B2 - the effect exists but is partially compensated, explaining why we are precisely at the threshold where the effect is neither clearly visible nor clearly absent.
% end box important
	
	Each outcome advances our understanding: confirmation validates a new cosmological paradigm, absence simplifies the theory while preserving its geometric foundations, and compensation reveals additional physics we must account for. This is science at its best - clear predictions, definitive tests, and the flexibility to learn from whatever nature reveals.
	
\section*{Revolutionary}
		We stand at a unique juncture in the history of cosmology. Within the next decade, humanity will definitively know whether:
		\begin{itemize}
			\item The universe is static with photon energy loss (T0 confirmed)
			\item The universe expands as currently believed (T0 refuted via B1)
			\item Reality is more complex than either model alone (T0 with compensation via B2)
		\end{itemize}
		Each outcome revolutionizes our understanding. This is not merely a test of a theory - it is a fundamental verdict on the nature of the cosmos itself.
% end box revolutionary
	
	\section{Statistical Analysis Method}
	
	\subsection{Multi-Line Regression}
	
\section*{Experiment}
\section*{Wavelength-Redshift Correlation Test:}
		\begin{enumerate}
			\item Collect redshift measurements: $\{z_i, \lambda_i\}$ for each object
			\item Fit linear relationship: $z = \alpha \cdot \lambda + \beta$
			\item Compare slope $\alpha$ with T0 prediction: $\alpha = \frac{\xi x}{\Exi}$
			\item Test against standard cosmology: $\alpha = 0$
		\end{enumerate}
% end box experiment
	
	\subsection{Required Precision}
	
	To detect T0 effects with $\xiconst$:
	\begin{itemize}
		\item \textbf{Minimum required precision}: $\frac{\Delta z}{z} \approx 10^{-5}$
		\item \textbf{Current best precision}: $\frac{\Delta z}{z} \approx 10^{-4}$ (barely sufficient)
		\item \textbf{Next generation instruments}: $\frac{\Delta z}{z} \approx 10^{-6}$ (clearly detectable)
	\end{itemize}
	
	\section{Mathematical Equivalence of Space Expansion, Energy Loss, and Diffraction}
	\label{redshift_deflec:L-redshift_deflection-0896}
	
	\subsection{Formal Equivalence Proofs}
	\label{redshift_deflec:L-redshift_deflection-0897}
	
	The three fundamental mechanisms for explaining cosmological redshift can be described by different physical processes but lead to mathematically equivalent results under certain conditions.
	
	\begin{table}[h]
		\centering
		\caption{Comparison of Redshift Mechanisms with Extended Developments}
		\scalebox{0.75}{
			\begin{tabular}{lllc}
				\toprule
				\textbf{Mechanism} & \textbf{Physical Process} & \textbf{Redshift Formula} & \textbf{Taylor Expansion} \\
				\midrule
				Space Expansion ($\Lambda$CDM) & Metric expansion & $1+z = \frac{a(t_0)}{a(t_e)}$ & $z \approx H_0 D + \frac{1}{2}q_0(H_0 D)^2$ \\
				Energy Loss (T0-E) & Photon fatigue & $1+z = \exp\left(\int_0^D \xi \frac{H}{T} dl\right)$ & $z \approx \xi \frac{H_0 D}{T_0} + \frac{1}{2}\xi^2\left(\frac{H_0 D}{T_0}\right)^2$ \\
				Vacuum Diffraction (T0-B) & Refractive index change & $1+z = \frac{n(t_e)}{n(t_0)}$ & $z \approx \xi \ln\left(1+\frac{H_0 D}{c}\right)\left(1+\frac{\xi\lambda_0}{2\lambda_{crit}}\right)$ \\
				\bottomrule
			\end{tabular}
	\end{table}
	
	\subsubsection{Mathematical Equivalence Conditions}
	
	For the equivalence of the three mechanisms, the following conditions must be satisfied:
	
	\begin{equation}
		\boxed{\frac{1}{a}\frac{da}{dt} = -\frac{1}{n}\frac{dn}{dt} = \xi \frac{H}{T_0}}
	\end{equation}
	
	This leads to the relationships:
	\begin{itemize}
		\item \textbf{$\Lambda$CDM $\leftrightarrow$ T0-B}: $n(t) = a^{-1}(t)$
		\item \textbf{$\Lambda$CDM $\leftrightarrow$ T0-E}: $\dot{E}/E = -H(t)$
		\item \textbf{T0-B $\leftrightarrow$ T0-E}: $n(t) \propto E^{-1}(t)$
	\end{itemize}
	
	\subsubsection{Perturbative Development}
	
	The equivalence holds exactly only in first order. Higher-order deviations provide distinguishing signatures:
	
	\begin{equation}
		z_{total} = z_0 + \Delta z_{mechanism} + O(\xi^2)
	\end{equation}
	
	where $\Delta z_{mechanism}$ depends on the specific physical process.
	
	\subsection{Energy Conservation and Thermodynamics}
	\label{redshift_deflec:L-redshift_deflection-0898}
	
	\subsubsection{Energy Balance in Different Formalisms}
	
\section*{$\Lambda$CDM (apparent energy loss):}
	\begin{equation}
		E_{photon} = \frac{h\nu_0}{1+z} = \frac{h\nu_0 a(t_e)}{a(t_0)}
	\end{equation}
	
\section*{T0-Diffraction (energy conservation):}
	\begin{equation}
		E_{photon} = \frac{h\nu}{n(t)} = \frac{h\nu_0}{(1+z)n(t)} = \text{const}
	\end{equation}
	
\section*{T0-Energy Loss (real loss):}
	\begin{equation}
		\frac{dE}{dt} = -\xi H E \quad \Rightarrow \quad E(t) = E_0 \exp\left(-\int_0^t \xi H(t') dt'\right)
	\end{equation}
	
	\subsubsection{Thermodynamic Consistency}
	
	The entropy change for the different mechanisms:
	
	\begin{equation}
		\Delta S = \begin{cases}
			0 & \text{($\Lambda$CDM: adiabatic)} \\
			k_B \xi N_{photon} \ln(1+z) & \text{(T0-Energy Loss)} \\
			0 & \text{(T0-Diffraction: reversible)}
		\end{cases}
	\end{equation}
	
	\section{Implications for Cosmology}
	
	\subsection{Static Universe Model}
	
	The T0-theory describes a static, eternally existing universe where:
	\begin{itemize}
		\item Redshift arises from energy loss, not expansion
		\item CMB is equilibrium radiation of the $\xi$-field
		\item No Big Bang singularity required
		\item No dark energy or dark matter needed
		\item Cyclic processes possible within static framework
	\end{itemize}
	
	\subsection{Resolution of Cosmological Tensions}
	
	The T0 model resolves:
	\begin{enumerate}
		\item \textbf{Hubble tension}: Different measurements reconciled through $\xi$-effects
		\item \textbf{JWST early galaxies}: No formation time paradox in static universe
		\item \textbf{Cosmic coincidence}: Natural explanation through $\xi$-geometry
		\item \textbf{Horizon problem}: No horizon in eternal universe
		\item \textbf{Flatness problem}: Natural consequence of static geometry
	\end{enumerate}
	
	\section{Robustness of Core T0 Predictions}
	
	\subsection{Independent of Redshift Mechanism}
	
	Even if spectroscopic tests fail to detect wavelength-dependent redshift, the following T0 predictions remain valid:
	
	\begin{enumerate}
		\item \textbf{Gravitational constant}: $G = \frac{\xi^2 c^3}{16\pi m_p^2} = 6.674 \times 10^{-11}$ m$^3$kg$^{-1}$s$^{-2}$ (accurate to 8 digits) remains valid, independent of cosmological tests
		
		\item \textbf{Geometric constants}: The derivation of $\alpha \approx 1/137$ from $(4/3)^3$ scaling remains
		
		\item \textbf{Mass hierarchy}: $m_e : m_\mu : m_\tau = 1 : 206.768 : 3477.15$ follows from quantum numbers, not redshift
		
		\item \textbf{Hubble tension}: The 4/3 explanation works regardless of specific mechanism
	\end{enumerate}
	
	\subsection{Adaptivity of Theoretical Structure}
	
	The T0-theory has natural adaptation mechanisms:
	
	\begin{equation}
		\xi_{eff}(\text{Scale}) = \xi_0 \times f(\text{Environment}) \times g(\text{Energy})
	\end{equation}
	
	where:
	\begin{itemize}
		\item $f(\text{Environment}) = 4/3$ in galaxy clusters, $= 1$ in intergalactic medium
		\item $g(\text{Energy})$ describes renormalization group running
	\end{itemize}
	
	This flexibility is not an ad-hoc adjustment but follows from the geometric structure of the theory.
	
	\section{Conclusions}
	
	The T0-theory provides a revolutionary alternative to expansion-based cosmology through a single universal constant $\xiconst$. The wavelength-dependent redshift prediction offers a clear experimental test to distinguish between T0 and standard cosmology. While current precision barely reaches the detection threshold, next-generation spectroscopic instruments should definitively test this fundamental prediction.
	
	The unification of gravitational, electromagnetic, and quantum phenomena through the $\xi$-field represents a paradigm shift from complex multi-parameter models to elegant geometric simplicity. The experimental tests proposed here, particularly multi-wavelength spectroscopy of cosmic objects, provide clear pathways to validate or refute the theory.
	
\section*{Important}
		The T0-theory demonstrates that all cosmic phenomena can be understood through a single geometric constant, eliminating the need for dark matter, dark energy, inflation, and the Big Bang singularity. This represents the most significant simplification in physics since Newton's unification of terrestrial and celestial mechanics.
% end box important
	
	% Bibliography
	\bibliographystyle{unsrt}
	


% Bibliography
\begin{thebibliography}{99}
	
	\bibitem{pdg2024}
	Particle Data Group Collaboration (2024). 
	\textit{Review of Particle Physics}. 
	Progress of Theoretical and Experimental Physics, 2024(8), 083C01.
	\url{https://pdg.lbl.gov}
	
	\bibitem{flag2024}
	Aoki, Y., et al. (FLAG Collaboration) (2024). 
	\textit{FLAG Review 2024 of Lattice Results for Low-Energy Constants}. 
	arXiv:2411.04268.
	\url{https://arxiv.org/abs/2411.04268}
	
	\bibitem{fermilab_muon_g2}
	Abi, B., et al. (Muon g-2 Collaboration) (2021). 
	\textit{Measurement of the Positive Muon Anomalous Magnetic Moment to 0.46 ppm}. 
	Physical Review Letters, 126, 141801.
	
	\bibitem{peskin_schroeder}
	Peskin, M. E., \& Schroeder, D. V. (1995). 
	\textit{An Introduction to Quantum Field Theory}. 
	Addison-Wesley.
	
	\bibitem{weinberg_qft}
	Weinberg, S. (1995). 
	\textit{The Quantum Theory of Fields, Vol. I--III}. 
	Cambridge University Press.
	
	\bibitem{griffiths_particle}
	Griffiths, D. (2008). 
	\textit{Introduction to Elementary Particles}. 
	Wiley-VCH.
	
	\bibitem{mandl_shaw}
	Mandl, F., \& Shaw, G. (2010). 
	\textit{Quantum Field Theory (2nd ed.)}. 
	Wiley.
	
	\bibitem{srednicki_qft}
	Srednicki, M. (2007). 
	\textit{Quantum Field Theory}. 
	Cambridge University Press.
	
	\bibitem{t0_fundamentals}
	Pascher, J. (2024). 
	\textit{T0-Theory: Foundations of Time-Mass Duality}. 
	Unpublished manuscript, HTL Leonding.
	
	\bibitem{t0_fine_structure}
	Pascher, J. (2024). 
	\textit{T0-Theory: The Fine Structure Constant}. 
	Unpublished manuscript, HTL Leonding.
	
	\bibitem{t0_neutrinos}
	Pascher, J. (2024). 
	\textit{T0-Theory: Neutrino Masses and PMNS Mixing}. 
	Unpublished manuscript, HTL Leonding.
	
	\bibitem{t0_github}
	Pascher, J. (2024--2025). 
	\textit{T0-Time-Mass-Duality Repository}. 
	GitHub.
	\url{https://github.com/jpascher/T0-Time-Mass-Duality}
	
	\bibitem{lattice_qcd_review}
	Kronfeld, A. S. (2012). 
	\textit{Twenty-first Century Lattice Gauge Theory: Results from the QCD Lagrangian}. 
	Annual Review of Nuclear and Particle Science, 62, 265--284.
	
	\bibitem{neutrino_mixing_pdg}
	Particle Data Group Collaboration (2024). 
	\textit{Neutrino Masses, Mixing, and Oscillations}. 
	PDG Review 2024.
	\url{https://pdg.lbl.gov/2024/reviews/rpp2024-rev-neutrino-mixing.pdf}
	
	\bibitem{higgs_discovery}
	ATLAS and CMS Collaborations (2012). 
	\textit{Observation of a New Particle in the Search for the Standard Model Higgs Boson}. 
	Physics Letters B, 716, 1--29.
	
	\bibitem{Brannen2005}
	C. P. Brannen, ``Estimate of neutrino masses from Koide's relation'', \textit{arXiv:hep-ph/0505028} (2005).
	\url{https://arxiv.org/abs/hep-ph/0505028}
	
	\bibitem{Brannen2006}
	C. P. Brannen, ``Koide Mass Formula for Neutrinos'', \textit{arXiv:0702.0052} (2006).
	\url{http://brannenworks.com/MASSES.pdf}
	
	\bibitem{PhaseVectors2025}
	Anonymous, ``The Koide Relation and Lepton Mass Hierarchy from Phase Vectors'', \textit{rXiv:2507.0040} (2025).
	\url{https://rxiv.org/pdf/2507.0040v1.pdf}
	
	\bibitem{PDG2025}
	Particle Data Group, ``Review of Particle Physics'', \textit{Phys. Rev. D} \textbf{112} (2025) 030001.
	\url{https://pdg.lbl.gov/2025/}
	
	\bibitem{terrell2024}
	Terrell et al. (2024). 
	\textit{Single-Clock Metrology in Nature}. 
	Nature Physics.
	
	\bibitem{hossenfelder2024}
	Hossenfelder, S. (2024). 
	\textit{Single Clock Video Explanation}. 
	YouTube.
	
	\bibitem{hundert1931}
	Hundert (1931). 
	\textit{Reference Work}. 
	Publisher.
	
	\bibitem{terrell2025}
	Terrell et al. (2025). 
	\textit{Advanced Clock Synchronization Methods}. 
	Physical Review Letters.
	
	\bibitem{pascher_t0_2025}
	Pascher, J. (2025). 
	\textit{T0-Theory: Complete Framework and Applications}. 
	Unpublished manuscript, HTL Leonding.
	
	\bibitem{t0qm}
	Pascher, J. (2024). 
	\textit{T0-Theory: Quantum Mechanics Formulation}. 
	Unpublished manuscript, HTL Leonding.
	
	\bibitem{t0anomale}
	Pascher, J. (2024). 
	\textit{T0-Theory: Anomalous Magnetic Moments}. 
	Unpublished manuscript, HTL Leonding.
	
	\bibitem{muong2complete}
	Abi, B., et al. (Muon g-2 Collaboration) (2023). 
	\textit{Complete Measurement of the Positive Muon Anomalous Magnetic Moment}. 
	Physical Review Letters, 131, 161802.
	
	\bibitem{penrose2004}
	Penrose, R. (2004). 
	\textit{The Road to Reality: A Complete Guide to the Laws of the Universe}. 
	Jonathan Cape.
	
	\bibitem{planck1900}
	Planck, M. (1900). 
	\textit{On the Theory of the Energy Distribution Law of the Normal Spectrum}. 
	Verhandlungen der Deutschen Physikalischen Gesellschaft, 2, 237.
	
	\bibitem{T0Theory}
	Pascher, J. (2024). 
	\textit{T0-Theory: Fundamental Principles}. 
	Unpublished manuscript, HTL Leonding.
	
	% Additional bibliography entries for all undefined citations
	\bibitem{6g_roadmap}
	6G Research Consortium (2024).
	\textit{6G Technology Roadmap}.
	Technical Report.
	
	\bibitem{Born2013}
	Born, M. (2013).
	\textit{Einstein's Theory of Relativity}.
	Dover Publications.
	
	\bibitem{Casimir1948}
	Casimir, H. B. G. (1948).
	\textit{On the attraction between two perfectly conducting plates}.
	Proc. Kon. Ned. Akad. Wetensch. B51, 793--795.
	
	\bibitem{Einstein1905}
	Einstein, A. (1905).
	\textit{On the Electrodynamics of Moving Bodies}.
	Annalen der Physik, 17, 891--921.
	
	\bibitem{Feynman2006}
	Feynman, R. P. (2006).
	\textit{QED: The Strange Theory of Light and Matter}.
	Princeton University Press.
	
	\bibitem{Griffiths2017}
	Griffiths, D. J. (2017).
	\textit{Introduction to Electrodynamics (4th ed.)}.
	Cambridge University Press.
	
	\bibitem{Jackson1999}
	Jackson, J. D. (1999).
	\textit{Classical Electrodynamics (3rd ed.)}.
	Wiley.
	
	\bibitem{Mohr2016}
	Mohr, P. J., et al. (2016).
	\textit{CODATA Recommended Values of the Fundamental Physical Constants: 2014}.
	Rev. Mod. Phys. 88, 035009.
	
	\bibitem{Parker2018}
	Parker, R. H., et al. (2018).
	\textit{Measurement of the fine-structure constant as a test of the Standard Model}.
	Science, 360, 191--195.
	
	\bibitem{Planck1900}
	Planck, M. (1900).
	\textit{On the Theory of the Energy Distribution Law of the Normal Spectrum}.
	Verhandlungen der Deutschen Physikalischen Gesellschaft, 2, 237.
	
	\bibitem{Planck2018}
	Planck Collaboration (2018).
	\textit{Planck 2018 results. VI. Cosmological parameters}.
	Astronomy \& Astrophysics, 641, A6.
	
	\bibitem{QFT_T0}
	Pascher, J. (2024).
	\textit{T0-Theory and QFT Connections}.
	Unpublished manuscript, HTL Leonding.
	
	\bibitem{Sommerfeld1916}
	Sommerfeld, A. (1916).
	\textit{On the Quantum Theory of Spectral Lines}.
	Annalen der Physik, 51, 1--94.
	
	\bibitem{T0_Feinstruktur}
	Pascher, J. (2024).
	\textit{T0-Theory: Fine Structure Analysis}.
	Unpublished manuscript, HTL Leonding.
	
	\bibitem{T0_SI}
	Pascher, J. (2024).
	\textit{T0-Theory and SI Units}.
	Unpublished manuscript, HTL Leonding.
	
	\bibitem{T0_fine_structure}
	Pascher, J. (2024).
	\textit{T0-Theory: The Fine Structure Constant}.
	Unpublished manuscript, HTL Leonding.
	
	\bibitem{T0_g2_erweiterung}
	Pascher, J. (2024).
	\textit{T0-Theory: g-2 Extensions}.
	Unpublished manuscript, HTL Leonding.
	
	\bibitem{T0_gravitational_constant}
	Pascher, J. (2024).
	\textit{T0-Theory: Gravitational Constant Derivation}.
	Unpublished manuscript, HTL Leonding.
	
	\bibitem{T0_netze_en}
	Pascher, J. (2024).
	\textit{T0-Theory: Network Structures}.
	Unpublished manuscript, HTL Leonding.
	
	\bibitem{T0_tm_erweiterung}
	Pascher, J. (2024).
	\textit{T0-Theory: Time-Mass Extensions}.
	Unpublished manuscript, HTL Leonding.
	
	\bibitem{Uzan2003}
	Uzan, J.-P. (2003).
	\textit{The fundamental constants and their variation}.
	Rev. Mod. Phys. 75, 403--455.
	
	\bibitem{Weinberg1995}
	Weinberg, S. (1995).
	\textit{The Quantum Theory of Fields, Vol. I}.
	Cambridge University Press.
	
	\bibitem{albrecht1999}
	Albrecht, A. \& Magueijo, J. (1999).
	\textit{A time varying speed of light as a solution to cosmological puzzles}.
	Phys. Rev. D 59, 043516.
	
	\bibitem{alice2023}
	ALICE Collaboration (2023).
	\textit{Recent results from ALICE}.
	CERN-EP-2023-XXX.
	
	\bibitem{analog_optical}
	Smith, J. et al. (2024).
	\textit{Analog optical computing systems}.
	Nature Photonics.
	
	\bibitem{ashtekar2004}
	Ashtekar, A. \& Lewandowski, J. (2004).
	\textit{Background independent quantum gravity}.
	Class. Quantum Grav. 21, R53.
	
	\bibitem{atlas2023}
	ATLAS Collaboration (2023).
	\textit{ATLAS physics results}.
	CERN-PH-EP-2023-XXX.
	
	\bibitem{atlas2023higgs}
	ATLAS Collaboration (2023).
	\textit{Higgs boson measurements}.
	Phys. Rev. Lett.
	
	\bibitem{barbour1999}
	Barbour, J. (1999).
	\textit{The End of Time}.
	Oxford University Press.
	
	\bibitem{barrow1999}
	Barrow, J. D. (1999).
	\textit{Cosmologies with varying light speed}.
	Phys. Rev. D 59, 043515.
	
	\bibitem{becker2007}
	Becker, K. et al. (2007).
	\textit{String Theory and M-Theory}.
	Cambridge University Press.
	
	\bibitem{bell_muon}
	Bennett, G. W., et al. (Muon g-2 Collaboration) (2006).
	\textit{Final report of the E821 muon anomalous magnetic moment measurement}.
	Phys. Rev. D 73, 072003.
	
	\bibitem{bondi1948}
	Bondi, H. \& Gold, T. (1948).
	\textit{The steady-state theory of the expanding universe}.
	Mon. Not. R. Astron. Soc. 108, 252--270.
	
	\bibitem{brewer2019}
	Brewer, S. M. et al. (2019).
	\textit{Al+ Quantum-Logic Clock with Systematic Uncertainty below $10^{-18}$}.
	Phys. Rev. Lett. 123, 033201.
	
	\bibitem{cms2023top}
	CMS Collaboration (2023).
	\textit{Top quark measurements at CMS}.
	JHEP 2023.
	
	\bibitem{cms2024}
	CMS Collaboration (2024).
	\textit{CMS physics results 2024}.
	CERN-PH-EP-2024-XXX.
	
	\bibitem{codata2019}
	Tiesinga, E. et al. (2019).
	\textit{The 2018 CODATA Recommended Values}.
	J. Phys. Chem. Ref. Data.
	
	\bibitem{desi2025}
	DESI Collaboration (2025).
	\textit{DESI 2025 Cosmology Results}.
	arXiv preprint.
	
	\bibitem{differential_optical}
	Wang, X. et al. (2024).
	\textit{Differential optical computing}.
	Optica.
	
	\bibitem{dingle1972}
	Dingle, H. (1972).
	\textit{Science at the Crossroads}.
	Martin Brian \& O'Keeffe.
	
	\bibitem{divalentino2021}
	Di Valentino, E. et al. (2021).
	\textit{In the realm of the Hubble tension}.
	Class. Quantum Grav. 38, 153001.
	
	\bibitem{elnaschie2004}
	El Naschie, M. S. (2004).
	\textit{A review of E infinity theory}.
	Chaos, Solitons \& Fractals, 19, 209--236.
	
	\bibitem{fabrication_heterogeneous}
	Chen, Y. et al. (2024).
	\textit{Heterogeneous photonic integration}.
	Nature Electronics.
	
	\bibitem{fermilab2023}
	Fermilab (2023).
	\textit{Muon g-2 results}.
	Phys. Rev. Lett.
	
	\bibitem{flexible_wafer}
	Kim, S. et al. (2024).
	\textit{Flexible wafer-scale photonics}.
	Science Advances.
	
	\bibitem{francesco1997}
	Di Francesco, P. et al. (1997).
	\textit{Conformal Field Theory}.
	Springer.
	
	\bibitem{hartree1957}
	Hartree, D. R. (1957).
	\textit{The Calculation of Atomic Structures}.
	Wiley.
	
	\bibitem{hhi_6g}
	Fraunhofer HHI (2024).
	\textit{6G Photonic Integration}.
	Technical Report.
	
	\bibitem{hossenfelder2025}
	Hossenfelder, S. (2025).
	\textit{Science without the gobbledygook}.
	YouTube/Blog.
	
	\bibitem{hossenfelder_single_clock_video}
	Hossenfelder, S. (2024).
	\textit{The Single Clock Problem}.
	YouTube.
	
	\bibitem{hoyle1948}
	Hoyle, F. (1948).
	\textit{A new model for the expanding universe}.
	Mon. Not. R. Astron. Soc. 108, 372--382.
	
	\bibitem{integration_microelectronic}
	Liu, A. et al. (2024).
	\textit{Microelectronic photonic integration}.
	IEEE Journal.
	
	\bibitem{jacobson1995}
	Jacobson, T. (1995).
	\textit{Thermodynamics of spacetime}.
	Phys. Rev. Lett. 75, 1260.
	
	\bibitem{kasevich2023}
	Kasevich, M. et al. (2023).
	\textit{Atom interferometry tests}.
	Nature Physics.
	
	\bibitem{lerner2014}
	Lerner, E. J. (2014).
	\textit{An open letter on cosmology}.
	New Scientist.
	
	\bibitem{lisa2017}
	LISA Consortium (2017).
	\textit{Laser Interferometer Space Antenna}.
	ESA Technical Report.
	
	\bibitem{lithium_tantalate}
	Zhang, M. et al. (2024).
	\textit{Thin-film lithium tantalate photonics}.
	Nature Photonics.
	
	\bibitem{lopez2010}
	Lopez-Corredoira, M. (2010).
	\textit{Tests and problems of the standard model in cosmology}.
	Int. J. Mod. Phys. D.
	
	\bibitem{ludlow2015}
	Ludlow, A. D. et al. (2015).
	\textit{Optical atomic clocks}.
	Rev. Mod. Phys. 87, 637.
	
	\bibitem{mach1883}
	Mach, E. (1883).
	\textit{Die Mechanik in ihrer Entwickelung}.
	F.A. Brockhaus.
	
	\bibitem{maldacena1998}
	Maldacena, J. (1998).
	\textit{The large N limit of superconformal field theories}.
	Adv. Theor. Math. Phys. 2, 231--252.
	
	\bibitem{mueller2014}
	Müller, H. et al. (2014).
	\textit{Atom interferometry tests of the gravitational redshift}.
	Phys. Rev. Lett.
	
	\bibitem{mug2_final_2025}
	Muon g-2 Collaboration (2025).
	\textit{Final muon g-2 measurement}.
	Phys. Rev. Lett.
	
	\bibitem{muong2_2023}
	Muon g-2 Collaboration (2023).
	\textit{Updated muon g-2 results}.
	Phys. Rev. Lett.
	
	\bibitem{nathan2024}
	Nathan, A. et al. (2024).
	\textit{Quantum computing advances}.
	Nature.
	
	\bibitem{newell2018}
	Newell, D. B. et al. (2018).
	\textit{The CODATA 2017 values of h, e, k, and $N_A$}.
	Metrologia 55, L13.
	
	\bibitem{nottale1993}
	Nottale, L. (1993).
	\textit{Fractal Space-Time and Microphysics}.
	World Scientific.
	
	\bibitem{on_chip_lithium}
	Wang, C. et al. (2024).
	\textit{On-chip lithium niobate photonics}.
	Nature Communications.
	
	\bibitem{optical_advantages}
	Shastri, B. J. et al. (2024).
	\textit{Advantages of optical computing}.
	Nature Reviews Physics.
	
	\bibitem{pascher2025cmb}
	Pascher, J. (2025).
	\textit{T0-Theory: CMB Analysis}.
	Unpublished manuscript, HTL Leonding.
	
	\bibitem{pascher2025g2}
	Pascher, J. (2025).
	\textit{T0-Theory: g-2 Predictions}.
	Unpublished manuscript, HTL Leonding.
	
	\bibitem{pascher2025qm}
	Pascher, J. (2025).
	\textit{T0-Theory: Quantum Mechanics}.
	Unpublished manuscript, HTL Leonding.
	
	\bibitem{pascher2025si}
	Pascher, J. (2025).
	\textit{T0-Theory: SI Unit System}.
	Unpublished manuscript, HTL Leonding.
	
	\bibitem{pascher2025t0}
	Pascher, J. (2025).
	\textit{T0-Theory: Complete Framework}.
	Unpublished manuscript, HTL Leonding.
	
	\bibitem{pascher:fundamentals}
	Pascher, J. (2024).
	\textit{T0-Theory: Fundamentals}.
	Unpublished manuscript, HTL Leonding.
	
	\bibitem{pascher:g2_rev9}
	Pascher, J. (2024).
	\textit{T0-Theory: g-2 Revision 9}.
	Unpublished manuscript, HTL Leonding.
	
	\bibitem{pascher:geometric_formalism}
	Pascher, J. (2024).
	\textit{T0-Theory: Geometric Formalism}.
	Unpublished manuscript, HTL Leonding.
	
	\bibitem{pascher:ml_addendum}
	Pascher, J. (2024).
	\textit{T0-Theory: Machine Learning Addendum}.
	Unpublished manuscript, HTL Leonding.
	
	\bibitem{pascher:t0_foundations}
	Pascher, J. (2024).
	\textit{T0-Theory: Foundations}.
	Unpublished manuscript, HTL Leonding.
	
	\bibitem{pascher_derivation_beta_2025}
	Pascher, J. (2025).
	\textit{T0-Theory: Derivation of Beta}.
	Unpublished manuscript, HTL Leonding.
	
	\bibitem{pascher_higgs_connection_2025}
	Pascher, J. (2025).
	\textit{T0-Theory: Higgs Connection}.
	Unpublished manuscript, HTL Leonding.
	
	\bibitem{pascher_lagrangian_extended_2025}
	Pascher, J. (2025).
	\textit{T0-Theory: Extended Lagrangian}.
	Unpublished manuscript, HTL Leonding.
	
	\bibitem{pascher_mathematical_structure_2025}
	Pascher, J. (2025).
	\textit{T0-Theory: Mathematical Structure}.
	Unpublished manuscript, HTL Leonding.
	
	\bibitem{pascher_t0_cmb_2025}
	Pascher, J. (2025).
	\textit{T0-Theory: CMB Predictions}.
	Unpublished manuscript, HTL Leonding.
	
	\bibitem{pascher_t0_energie_2025}
	Pascher, J. (2025).
	\textit{T0-Theory: Energy}.
	Unpublished manuscript, HTL Leonding.
	
	\bibitem{pascher_t0_energy_2025}
	Pascher, J. (2025).
	\textit{T0-Theory: Energy Framework}.
	Unpublished manuscript, HTL Leonding.
	
	\bibitem{pascher_t0_theory_2025}
	Pascher, J. (2025).
	\textit{T0-Theory: Complete Theory}.
	Unpublished manuscript, HTL Leonding.
	
	\bibitem{penrose1959}
	Penrose, R. (1959).
	\textit{The apparent shape of a relativistically moving sphere}.
	Proc. Cambridge Phil. Soc. 55, 137--139.
	
	\bibitem{penrose1967}
	Penrose, R. (1967).
	\textit{Twistor algebra}.
	J. Math. Phys. 8, 345--366.
	
	\bibitem{peratt1992}
	Peratt, A. L. (1992).
	\textit{Physics of the Plasma Universe}.
	Springer-Verlag.
	
	\bibitem{peskin1995}
	Peskin, M. E. \& Schroeder, D. V. (1995).
	\textit{An Introduction to Quantum Field Theory}.
	Addison-Wesley.
	
	\bibitem{peskin_schroeder_1995}
	Peskin, M. E. \& Schroeder, D. V. (1995).
	\textit{An Introduction to Quantum Field Theory}.
	Addison-Wesley.
	
	\bibitem{phoquant}
	PhoQuant (2024).
	\textit{Photonic quantum computing}.
	Technical Report.
	
	\bibitem{photonics_ai}
	Wetzstein, G. et al. (2024).
	\textit{Photonics for AI}.
	Nature.
	
	\bibitem{planck1906}
	Planck, M. (1906).
	\textit{The Theory of Heat Radiation}.
	Johann Ambrosius Barth.
	
	\bibitem{planck2018}
	Planck Collaboration (2018).
	\textit{Planck 2018 results}.
	A\&A 641, A6.
	
	\bibitem{polchinski1998}
	Polchinski, J. (1998).
	\textit{String Theory}.
	Cambridge University Press.
	
	\bibitem{qant_nps}
	QANT (2024).
	\textit{Quantum photonics systems}.
	Technical Report.
	
	\bibitem{quantenjahr25}
	Quantenjahr (2025).
	\textit{International Year of Quantum}.
	UNESCO.
	
	\bibitem{recurrent_photonics}
	Tait, A. N. et al. (2024).
	\textit{Recurrent photonic neural networks}.
	Optica.
	
	\bibitem{rf_photonics}
	Capmany, J. \& Novak, D. (2024).
	\textit{Microwave photonics}.
	Nature Photonics.
	
	\bibitem{riess2019}
	Riess, A. G. et al. (2019).
	\textit{Large Magellanic Cloud Cepheid Standards}.
	ApJ 876, 85.
	
	\bibitem{riess2022}
	Riess, A. G. et al. (2022).
	\textit{A Comprehensive Measurement of H0}.
	ApJ 934, L7.
	
	\bibitem{rovelli2004}
	Rovelli, C. (2004).
	\textit{Quantum Gravity}.
	Cambridge University Press.
	
	\bibitem{sciama1953}
	Sciama, D. W. (1953).
	\textit{On the origin of inertia}.
	Mon. Not. R. Astron. Soc. 113, 34--42.
	
	\bibitem{sciencedaily2025}
	ScienceDaily (2025).
	\textit{Physics news}.
	Online.
	
	\bibitem{sm_g2_2025}
	Aoyama, T. et al. (2025).
	\textit{Standard Model prediction for g-2}.
	Phys. Rep.
	
	\bibitem{susskind1995}
	Susskind, L. (1995).
	\textit{The world as a hologram}.
	J. Math. Phys. 36, 6377--6396.
	
	\bibitem{t0_kosmologie}
	Pascher, J. (2024).
	\textit{T0-Theory: Cosmology}.
	Unpublished manuscript, HTL Leonding.
	
	\bibitem{terrell1959}
	Terrell, J. (1959).
	\textit{Invisibility of the Lorentz contraction}.
	Phys. Rev. 116, 1041--1045.
	
	\bibitem{terrell_single_clock_nature_2024}
	Terrell, J. et al. (2024).
	\textit{Single clock precision measurements}.
	Nature Physics.
	
	\bibitem{tfln_foundry}
	TFLN Foundry (2024).
	\textit{Thin-film lithium niobate foundry services}.
	Technical Specifications.
	
	\bibitem{thiemann2007}
	Thiemann, T. (2007).
	\textit{Modern Canonical Quantum General Relativity}.
	Cambridge University Press.
	
	\bibitem{thz_epfl}
	EPFL (2024).
	\textit{Terahertz photonics research}.
	Technical Report.
	
	\bibitem{unnikrishnan2004}
	Unnikrishnan, C. S. (2004).
	\textit{On Einstein's resolution of the twin clock paradox}.
	Current Science, 86, 704--709.
	
	\bibitem{verlinde2011}
	Verlinde, E. (2011).
	\textit{On the origin of gravity and the laws of Newton}.
	JHEP 2011, 29.
	
	\bibitem{video2025}
	Video (2025).
	\textit{Physics video explanation}.
	YouTube.
	
	\bibitem{weinberg1995}
	Weinberg, S. (1995).
	\textit{The Quantum Theory of Fields}.
	Cambridge University Press.
	
	\bibitem{weiskopf2000}
	Weiskopf, D. (2000).
	\textit{Visualization of special relativity}.
	PhD thesis, University of Tübingen.
	
	\bibitem{wheeler1990}
	Wheeler, J. A. (1990).
	\textit{A Journey into Gravity and Spacetime}.
	Scientific American Library.
	
	\bibitem{wiki_bell}
	Wikipedia (2024).
	\textit{Bell's theorem}.
	Online encyclopedia.
	
	\bibitem{zwicky1929}
	Zwicky, F. (1929).
	\textit{On the red shift of spectral lines through interstellar space}.
	Proc. Natl. Acad. Sci. 15, 773--779.

\end{thebibliography}


\end{document}
