\documentclass[11pt,a4paper]{article}
\usepackage[a4paper,margin=2cm]{geometry}
\usepackage[utf8]{inputenc}
\usepackage[english]{babel}
\usepackage{lmodern}
\renewcommand{\familydefault}{\sfdefault}

\usepackage{amsmath,amssymb,amsthm}
\usepackage{graphicx}
\usepackage[unicode,pdfencoding=auto,hypertexnames=false]{hyperref}
\usepackage{booktabs}
\usepackage{longtable}
\usepackage{array}
\usepackage{siunitx}
\usepackage{fancyhdr}
\usepackage{float}
\usepackage{tikz}
% tcolorbox removed for standalone
% tcbset removed
\tikzset{
  t0blue/.style={draw=blue,fill=blue!10},
  t0red/.style={draw=red,fill=red!10},
  t0green/.style={draw=green!50!black,fill=green!10},
  t0orange/.style={draw=orange,fill=orange!10},
}
\usepackage{setspace}
\usepackage{enumitem}
\usepackage{adjustbox}
\usepackage{xcolor}

% Define colors for xcolor package
\definecolor{t0green}{RGB}{34,139,34}
\definecolor{t0blue}{RGB}{0,0,255}
\definecolor{t0red}{RGB}{255,0,0}
\definecolor{t0orange}{RGB}{255,165,0}

% Define custom column types for tables
\newcolumntype{L}[1]{>{\raggedright\arraybackslash}p{#1}}
\newcolumntype{C}[1]{>{\centering\arraybackslash}p{#1}}
\newcolumntype{R}[1]{>{\raggedleft\arraybackslash}p{#1}}

\setlength{\parindent}{0pt}
\setlength{\parskip}{6pt}

\hypersetup{
  colorlinks=true,
  linkcolor=blue,
  citecolor=blue,
  urlcolor=blue
}
\pagestyle{fancy}
\setlength{\headheight}{28pt}

\newcommand{\checkmarkx}{\checkmark}
\newcommand{\warningx}{\textbf{!}}

% Makros aus Einzel-Dokumenten (Fallback-Definitionen)
\newcommand{\mytimes}{\times}
\newcommand{\myapprox}{\approx}
\newcommand{\mysim}{\sim}
\newcommand{\myomega}{\omega}
\newcommand{\mypi}{\pi}
\newcommand{\myrightarrow}{\rightarrow}
\newcommand{\mypropto}{\propto}
\newcommand{\deltafield}{\delta\phi}
\newcommand{\xipar}{\xi}
\newcommand{\xiT}{\xi}
\newcommand{\lambdah}{\lambda_h}

% Additional macros used in chapter files
\newcommand{\Kfrak}{K_{\text{frak}}}  % Fractal correction factor
\newcommand{\Dfrak}{D_f}              % Fractal dimension
\newcommand{\betapar}{\beta}          % T0 beta parameter
\newcommand{\alphapar}{\alpha}        % T0 alpha parameter
\newcommand{\Efield}{E}               % Energy field
% Note: checkmarkxa/warningxa are variants used in auto-generated chapter files
\newcommand{\checkmarkxa}{\checkmark}
\newcommand{\warningxa}{\textbf{!}}

% Additional T0-specific macros
\newcommand{\xigeom}{\xi_{\text{geom}}}  % Geometric xi
\newcommand{\lP}{\ell_P}                  % Planck length
\newcommand{\rzero}{r_0}                  % Characteristic radius
\newcommand{\xirat}{\xi_{\text{rat}}}     % Xi ratio
\newcommand{\tzero}{t_0}                  % Characteristic time
\newcommand{\natunits}{\text{(nat. units)}}  % Natural units annotation
\newcommand{\myRightarrow}{\Rightarrow}   % Arrow variant
\newcommand{\Lag}{\mathcal{L}}            % Lagrangian

% Physics macros used in chapter files
\newcommand{\CQCD}{C_{\text{QCD}}}        % QCD correction
\newcommand{\EP}{E_P}                     % Planck energy
\newcommand{\Ee}{E_e}                     % Electron energy
\newcommand{\Emu}{E_\mu}                  % Muon energy
\newcommand{\Exi}{E_\xi}                  % Xi energy
\newcommand{\Ezero}{E_0}                  % Characteristic energy
\newcommand{\Hubble}{H}                   % Hubble constant
\newcommand{\Kspec}{K_{\text{spec}}}      % Spectral correction
\newcommand{\Lambdat}{\Lambda_t}          % Time-related cosmological constant
\newcommand{\Leff}{\mathcal{L}_{\text{eff}}}  % Effective Lagrangian
\newcommand{\Lorentz}{\mathcal{L}}        % Lorentz symbol
\newcommand{\Lxi}{L_\xi}                  % Xi length
\newcommand{\Tfield}{T}                   % Time field
\newcommand{\Weyl}{W}                     % Weyl tensor/symbol
\newcommand{\alphaEMSI}{\alpha_{\text{EM,SI}}}  % EM alpha in SI
\newcommand{\alphaEMnat}{\alpha_{\text{EM,nat}}}  % EM alpha in natural units
\newcommand{\alphaem}{\alpha_{\text{em}}} % Electromagnetic alpha
\newcommand{\betaTSI}{\beta_{T,\text{SI}}}  % Beta in SI
\newcommand{\betaTnat}{\beta_{T,\text{nat}}}  % Beta in natural units
\newcommand{\deltam}{\delta m}            % Mass difference
\newcommand{\phiT}{\phi_T}                % T-field phi
\newcommand{\tP}{t_P}                     % Planck time
\newcommand{\rhoCMB}{\rho_{\text{CMB}}}   % CMB density
\newcommand{\rhoCasimir}{\rho_{\text{Casimir}}}  % Casimir density

% Table formatting
\usepackage{multirow}

% Additional physics macros
\newcommand{\Riem}{\mathcal{R}}           % Riemann tensor
\newcommand{\ZPinch}{Z_{\text{pinch}}}    % Z-pinch
\newcommand{\SynchPower}{P_{\text{synch}}} % Synchrotron power
\newcommand{\Rzero}{R_0}                  % Characteristic radius
\newcommand{\alphafine}{\alpha}           % Fine structure constant
\newcommand{\Etau}{E_\tau}                % Tau energy
\newcommand{\deltaE}{\delta E}            % Energy deviation
\newcommand{\EPlanck}{E_P}                % Planck energy
\newcommand{\pichar}{\pi}                 % Pi character
\newcommand{\alphaWSI}{\alpha_{W,\text{SI}}}  % Wien alpha in SI
\newcommand{\alphaWnat}{\alpha_{W,\text{nat}}}  % Wien alpha in natural units

% Einfache abstract-Umgebung für Kapitel:
\newenvironment{abstract}{%
  \begin{center}\bfseries Abstract\end{center}\small
}{\par}


\title{T0 Neutrinos En}
\author{J. Pascher}
\date{\today}

\begin{document}
\maketitle

\section*{T0 Neutrinos (T0 Neutrinos)}

	\begin{abstract}
		This document addresses the special position of neutrinos in the T0 Theory. In contrast to established particles (charged leptons, quarks, bosons), neutrinos require a fundamentally different treatment based on the photon analogy with double $\xi_0$-suppression. The neutrino mass is derived from the formula $m_\nu = \frac{\xi_0^2}{2} \times m_e = 4.54$ meV, and oscillations are explained by geometric phases based on $T_x \cdot m_x = 1$, where the quantum numbers $(n, \ell, j)$ determine the phase differences. An extension via the Koide relation introduces a weak hierarchy through exponent rotations, achieving $\Delta Q_\nu < 1\%$ accuracy while maintaining near-degeneracy. A plausible target value for the neutrino mass ($m_\nu = 15$ meV) is derived from empirical data (cosmological limits). The T0 Theory is based on speculative geometric harmonies without empirical basis and is highly likely to be incomplete or incorrect. Scientific integrity requires a clear separation between mathematical correctness and physical validity.
	\end{abstract}
	
	\tableofcontents
	\newpage
	
	\section{Preamble: Scientific Honesty}
	
\section*{Warning}
		\textbf{CRITICAL LIMITATION:} The following formulas for neutrino masses are \textbf{speculative extrapolations} based on the untested hypothesis that neutrinos follow geometric harmonies and all flavor states have equal masses. This hypothesis has \textbf{no empirical basis} and is highly likely to be incomplete or incorrect. The mathematical formulas are nevertheless internally consistent and correctly formulated.
		
		\vspace{0.5cm}
\section*{Scientific integrity means:}
		\begin{itemize}
			\item Honesty about the speculative nature of the predictions
			\item Mathematical correctness despite physical uncertainty
			\item Clear separation between hypotheses and verified facts
		\end{itemize}
% end box warning
	
	\section{Neutrinos as ``Almost Massless Photons'': The T0 Photon Analogy}
	
\section*{Speculation}
		\textbf{Fundamental T0 Insight:} Neutrinos can be understood as ``damped photons''.
		
		The remarkable similarity between photons and neutrinos suggests a deeper geometric kinship:
		\begin{itemize}
			\item \textbf{Speed:} Both propagate nearly at the speed of light
			\item \textbf{Penetration:} Both have extreme penetrability
			\item \textbf{Mass:} Photon exactly massless, neutrino quasi-massless
			\item \textbf{Interaction:} Photon electromagnetic, neutrino weak
		\end{itemize}
% end box speculation
	
	\subsection{Photon-Neutrino Correspondence}
	\label{T0_Neutrinos:L-T0_Neutrinos-0033}
	
\section*{Photon}
\section*{Physical Parallels:}
		\begin{align}
			\text{Photon:} \quad &E^2 = (pc)^2 + 0 \quad \text{(perfectly massless)} \\
			\text{Neutrino:} \quad &E^2 = (pc)^2 + \left(\sqrt{\frac{\xipar^2}{2}} m c^2\right)^2 \quad \text{(quasi-massless)}
		\end{align}
		
\section*{Speed Comparison:}
		\begin{align}
			v_\gamma &= c \quad \text{(exact)} \\
			v_\nu &= c \times \left(1 - \frac{\xipar^2}{2}\right) \approx 0.9999999911 \times c
		\end{align}
		
		The speed difference is only $8.89 \times 10^{-9}$ -- practically immeasurable!
% end box photon
	
	\subsection{The Double -Suppression}
	\label{T0_Neutrinos:L-T0_Neutrinos-0034}
	
\section*{Key Result}
\section*{Neutrino Mass through Double Geometric Damping:}
		
		If neutrinos are ``almost photons'', then two suppression factors arise:
		
		\begin{enumerate}
			\item \textbf{First $\xi_0$ Factor:} ``Almost massless'' (like photon, but not perfect)
			\item \textbf{Second $\xi_0$ Factor:} ``Weak interaction'' (geometric decoupling)
		\end{enumerate}
		
\section*{Resulting Formula:}
		\begin{equation}
			\boxed{m_\nu = \frac{\xi_0^2}{2} \times m_e = \frac{(\frac{4}{3} \times 10^{-4})^2}{2} \times 0.511 \text{ MeV}}
		\end{equation}
		
\section*{Numerical Evaluation:}
		\begin{equation}
			m_\nu = 8.889 \times 10^{-9} \times 0.511 \text{ MeV} = 4.54 \text{ meV}
		\end{equation}
% end box keyresult
	
	\subsection{Physical Justification of the Photon Analogy}
	\label{T0_Neutrinos:L-T0_Neutrinos-0035}
	
\section*{Photon}
\section*{Why the Photon Analogy is Physically Sensible:}
		
\section*{1. Speed Comparison:}
		\begin{align}
			v_\gamma &= c \quad \text{(exact)} \\
			v_\nu &= c \times \left(1 - \frac{\xi_0^2}{2}\right) \approx 0.9999999911 \times c
		\end{align}
		The speed difference is only $8.89 \times 10^{-9}$ - practically immeasurable!
		
\section*{2. Interaction Strengths:}
		\begin{align}
			\sigma_\gamma &\sim \alpha_{EM} \approx \frac{1}{137} \\
			\sigma_\nu &\sim \frac{\xi_0^2}{2} \times G_F \approx 8.89 \times 10^{-9}
		\end{align}
		The ratio $\sigma_\nu/\sigma_\gamma \sim \frac{\xi_0^2}{2}$ confirms the geometric suppression!
		
\section*{3. Penetrability:}
		\begin{itemize}
			\item Photons: Electromagnetic shielding possible
			\item Neutrinos: Practically unshieldable
			\item Both: Extreme ranges in matter
		\end{itemize}
% end box photon
	
	\section{Neutrino Oscillations}
	
	\subsection{The Standard Model Problem}
	\label{T0_Neutrinos:L-T0_Neutrinos-0036}
	
\section*{Warning}
		\textbf{Neutrino Oscillations:} Neutrinos can change their identity (flavor) during flight - a phenomenon known as neutrino oscillation. A neutrino produced as an electron neutrino ($\nu_e$) can later be measured as a muon neutrino ($\nu_\mu$) or tau neutrino ($\nu_\tau$) and vice versa.
		
		The oscillations depend on the mass squared differences $\Delta m^2_{ij} = m_i^2 - m_j^2$ and the mixing angles. Current experimental data (2025) provide:
		\begin{align}
			\Delta m^2_{21} &\approx 7.53 \times 10^{-5} \text{ eV}^2 \quad \text{[Solar]} \\
			\Delta m^2_{32} &\approx 2.44 \times 10^{-3} \text{ eV}^2 \quad \text{[Atmospheric]} \\
			m_\nu &> 0.06 \text{ eV} \quad \text{[At least one neutrino, 3}\sigma\text{]}
		\end{align}
		
\section*{Problem for T0:}
		The T0 Theory postulates equal masses for the flavor states ($\nu_e, \nu_\mu, \nu_\tau$), which implies $\Delta m^2_{ij} = 0$ and is incompatible with standard oscillations.
% end box warning
	
	\subsection{Geometric Phases as Oscillation Mechanism}
	\label{T0_Neutrinos:L-T0_Neutrinos-0037}
	
\section*{Speculation}
\section*{T0 Hypothesis: Geometric Phases for Oscillations}
		
		To reconcile the hypothesis of equal masses ($m_{\nu_e} = m_{\nu_\mu} = m_{\nu_\tau} = m_\nu$) with neutrino oscillations, it is speculated that oscillations in the T0 Theory are caused by geometric phases rather than mass differences. This is based on the T0 relation:
		\[
		T_x \cdot m_x = 1,
		\]
		where $m_x = m_\nu = 4.54$ meV is the neutrino mass and $T_x$ is a characteristic time or frequency:
		\[
		T_x = \frac{1}{m_\nu} = \frac{1}{4.54 \times 10^{-3} \text{ eV}} \approx 2.2026 \times 10^2 \text{ eV}^{-1} \approx 1.449 \times 10^{-13} \text{ s}.
		\]
		
		The geometric phase is determined by the T0 quantum numbers $(n, \ell, j)$:
		\[
		\phi_{\text{geo}, i} \propto f(n, \ell, j) \cdot \frac{L}{E} \cdot \frac{1}{T_x},
		\]
		where $f(n, \ell, j) = \frac{n^6}{\ell^3}$ (or 1 for $\ell = 0$) are the geometric factors:
		\begin{align}
			f_{\nu_e} &= 1, \\
			f_{\nu_\mu} &= 64, \\
			f_{\nu_\tau} &= 91.125.
		\end{align}
		
		\textbf{WARNING:} This approach is purely hypothetical and without empirical confirmation. It contradicts the established theory that oscillations are caused by $\Delta m^2_{ij} \neq 0$.
% end box speculation
	
	\subsection{Quantum Number Assignment for Neutrinos}
	\label{T0_Neutrinos:L-T0_Neutrinos-0038}
	
	\begin{table}[h]
		\centering
		\begin{tabular}{lcccc}
			\toprule
			\textbf{Neutrino Flavor} & \textbf{$n$} & \textbf{$\ell$} & \textbf{$j$} & \textbf{$f(n,\ell,j)$} \\
			\midrule
			$\nu_e$ & $1$ & $0$ & $1/2$ & $1$ \\
			$\nu_\mu$ & $2$ & $1$ & $1/2$ & $64$ \\
			$\nu_\tau$ & $3$ & $2$ & $1/2$ & $91.125$ \\
			\bottomrule
		\end{tabular}
		\caption{Speculative T0 Quantum Numbers for Neutrino Flavors}
	\end{table}
	
	\section{Integration of the Koide Relation: A Weak Hierarchy}
	\label{T0_Neutrinos:L-T0_Neutrinos-0039}
	
\section*{Koide}
\section*{T0-Koide Extension for Neutrinos:}
		
		To address the oscillation conflict ($\Delta m^2_{ij} \neq 0$), the T0 Theory integrates the Koide relation as a natural generalization (Brannen 2005). This introduces a weak hierarchy via exponent rotations around $\xi_0$, preserving the photon analogy while enabling small mass differences.
		
\section*{Eigenvector Representation:}
		The charged lepton masses follow Koide via:
		\begin{equation}
			\begin{pmatrix}
				\sqrt{m_e} \\
				\sqrt{m_\mu} \\
				\sqrt{m_\tau}
			\end{pmatrix}
			= \mathbf{U} \cdot \begin{pmatrix}
				m_1 \\
				m_2 \\
				m_3
			\end{pmatrix},
		\end{equation}
		where $\mathbf{U}$ is the unitary flavor-mixing matrix (CKM/PMNS analog).
		
\section*{T0 Adaptation for Neutrinos:}
		Neutrino masses emerge as perturbed versions of the base $m_\nu = 4.54$ meV:
		\begin{equation}
			m_{\nu_i} \approx \xi_0^{p_i + \delta} \cdot v_\nu, \quad \delta \approx \xi_0^{1/3} \approx 0.051
		\end{equation}
		with exponents $p_i = (3/2, 1, 2/3)$ from charged leptons (rotated by $\delta$ for weak hierarchy). This yields a quasi-degenerate spectrum:
		\begin{align}
			m_{\nu_1} &\approx 4.20 \text{ meV (normal hierarchy)}, \\
			m_{\nu_2} &\approx 4.54 \text{ meV}, \\
			m_{\nu_3} &\approx 5.12 \text{ meV}, \\
			\Sigma m_\nu &\approx 13.86 \text{ meV}.
		\end{align}
		
\section*{Neutrino Koide Relation:}
		\begin{equation}
			Q_\nu = \frac{m_{\nu_1} + m_{\nu_2} + m_{\nu_3}}{\left( \sqrt{m_{\nu_1}} + \sqrt{m_{\nu_2}} + \sqrt{m_{\nu_3}} \right)^2} \approx 0.6667 = \frac{2}{3},
		\end{equation}
		with $\Delta Q_\nu < 1\%$ accuracy, directly linking to PMNS mixing.
		
\section*{Hybrid Oscillation Mechanism:}
		Geometric phases (from $f(n,\ell,j)$) dominate, augmented by small $\Delta m^2_{ij} \approx (0.1-0.2) \times 10^{-4}$ eV$^2$ from $\delta$. This reconciles T0 with data without full hierarchy.
		
		\textbf{WARNING:} Highly speculative; testable via future $\Sigma m_\nu$ measurements (e.g., Euclid 2026+).
% end box koidebox
	
	\section{Experimental Assessment}
	
	\subsection{Cosmological Limits}
	\label{T0_Neutrinos:L-T0_Neutrinos-0040}
	
\section*{Experimental}
\section*{Cosmological Neutrino Mass Limits (as of 2025):}
		
\section*{1. Planck Satellite + CMB Data:}
		\begin{equation}
			\Sigma m_\nu < 0.07 \text{ eV} \quad \text{(95\% Confidence)}
		\end{equation}
		
\section*{2. T0 Prediction (with Koide Extension):}
		\begin{equation}
			\Sigma m_\nu = 13.86 \text{ meV}
		\end{equation}
		
\section*{3. Comparison:}
		\begin{equation}
			\frac{13.86 \text{ meV}}{70 \text{ meV}} = 0.198 \approx 19.8\%
		\end{equation}
		
		The T0 prediction is well below all cosmological limits!
% end box experimental
	
	\subsection{Direct Mass Determination}
	\label{T0_Neutrinos:L-T0_Neutrinos-0041}
	
\section*{Experimental}
\section*{Experimental Neutrino Mass Determination:}
		
\section*{1. KATRIN Experiment (2022):}
		\begin{equation}
			m(\nu_e) < 0.8 \text{ eV} \quad \text{(90\% Confidence)}
		\end{equation}
		
\section*{2. T0 Prediction (with Koide):}
		\begin{equation}
			m(\nu_e) \approx 4.54 \text{ meV (effective)}
		\end{equation}
		
\section*{3. Comparison:}
		\begin{equation}
			\frac{4.54 \text{ meV}}{800 \text{ meV}} = 0.0057 \approx 0.57\%
		\end{equation}
		
		The T0 prediction is orders of magnitude below the direct mass limits.
% end box experimental
	
	\subsection{Target Value Estimation}
	\label{T0_Neutrinos:L-T0_Neutrinos-0042}
	
\section*{Key Result}
\section*{Plausible Target Value for Neutrino Masses:}
		
		From cosmological data and theoretical considerations, a plausible target value emerges:
		\begin{equation}
			m_\nu^{\text{Target}} \approx 15 \text{ meV (per flavor, quasi-degenerate)}
		\end{equation}
		
\section*{Comparison with T0 Prediction (incl. Koide):}
		\begin{equation}
			\frac{4.54 \text{ meV}}{15 \text{ meV}} = 0.303 \approx 30.3\%
		\end{equation}
		
		The T0 prediction is about a factor of 3 below the plausible target value, which is acceptable for a speculative theory. Koide extension narrows this to ~7\% via hierarchy.
% end box keyresult
	
	\section{Cosmological Implications}
	
	\subsection{Structure Formation and Big Bang Nucleosynthesis}
	\label{T0_Neutrinos:L-T0_Neutrinos-0043}
	
\section*{Key Result}
\section*{Cosmological Consequences of T0 Neutrino Masses:}
		
\section*{1. Big Bang Nucleosynthesis:}
		\begin{itemize}
			\item Relativistic neutrinos at $T \sim 1$ MeV: Standard BBN unchanged
			\item Contribution to radiation density: $N_{\text{eff}} = 3.046$ (Standard)
		\end{itemize}
		
\section*{2. Structure Formation:}
		\begin{itemize}
			\item Neutrinos with 4.5 meV become non-relativistic at $z \sim 100$
			\item Suppression of small-scale structure formation negligible
		\end{itemize}
		
\section*{3. Cosmic Neutrino Background (C$\nu$B):}
		\begin{itemize}
			\item Number density: $n_\nu = 336$ cm$^{-3}$ (unchanged)
			\item Energy density: $\rho_\nu \propto \Sigma m_\nu = 13.86$ meV (with Koide)
			\item Fraction of critical density: $\Omega_\nu h^2 \approx 1.55 \times 10^{-4}$
		\end{itemize}
		
\section*{4. Comparison with Dark Matter:}
		\begin{itemize}
			\item Neutrino contribution: $\Omega_\nu \approx 2.1 \times 10^{-4}$
			\item Dark matter: $\Omega_{DM} \approx 0.26$
			\item Ratio: $\Omega_\nu/\Omega_{DM} \approx 8.1 \times 10^{-4}$ (negligible)
		\end{itemize}
% end box keyresult
	
	\section{Summary and Critical Evaluation}
	
	\subsection{The Central T0 Neutrino Hypotheses}
	\label{T0_Neutrinos:L-T0_Neutrinos-0044}
	
\section*{Key Result}
\section*{Main Statements of the T0 Neutrino Theory:}
		
		\begin{enumerate}
			\item \textbf{Photon Analogy:} Neutrinos as ``damped photons'' with double $\xi_0$-suppression
			
			\item \textbf{Uniform Mass (Base):} All flavor states have $m_\nu \approx 4.54$ meV (quasi-degenerate)
			
			\item \textbf{Geometric Oscillations + Koide:} Phases + weak hierarchy ($\delta$) for $\Delta m^2_{ij}$
			
			\item \textbf{Speed Prediction:} $v_\nu = c(1 - \xi_0^2/2)$
			
			\item \textbf{Cosmological Consistency:} $\Sigma m_\nu \approx 13.86$ meV below all limits, $\Delta Q_\nu <1\%$
		\end{enumerate}
% end box keyresult
	
	\subsection{Scientific Assessment}
	\label{T0_Neutrinos:L-T0_Neutrinos-0045}
	
\section*{Warning}
\section*{Honest Scientific Evaluation:}
		
\section*{Strengths of the T0 Neutrino Theory:}
		\begin{itemize}
			\item Unified framework with other T0 predictions (now incl. Koide/PMNS)
			\item Elegant photon analogy with clear physical intuition
			\item Parameter freedom: No empirical adjustment
			\item Cosmological consistency with all known limits
			\item Specific, testable predictions (e.g., $\Sigma m_\nu$, $Q_\nu$)
		\end{itemize}
		
\section*{Fundamental Weaknesses:}
		\begin{itemize}
			\item \textbf{Contradiction to Oscillation Data:} Minimal $\Delta m^2_{ij}$ vs. experimental evidence (hybrid helps, but unproven)
			\item \textbf{Ad hoc Oscillation Mechanism:} Geometric phases + $\delta$ not fully derived
			\item \textbf{Missing QFT Foundation:} No complete field theory
			\item \textbf{Experimentally Indistinguishable:} Similar to Standard Model
			\item \textbf{Highly Speculative Basis:} Photon analogy and Koide extension unproven
		\end{itemize}
		
\section*{Overall Evaluation: Interesting Hypothesis, but Highly Speculative and Unconfirmed}
% end box warning
	
	\subsection{Comparison with Established T0 Predictions}
	\label{T0_Neutrinos:L-T0_Neutrinos-0046}
	
	\begin{table}[h]
		\centering
		\begin{tabular}{lcccc}
			\toprule
			\textbf{Area} & \textbf{T0 Prediction} & \textbf{Experiment} & \textbf{Deviation} & \textbf{Status} \\
			\midrule
			Fine Structure Constant & $\alpha^{-1} = 137.036$ & $137.036$ & $< 0.001\%$ & \checkmarkxa Established \\
			Gravitational Constant & $G = 6.674 \times 10^{-11}$ & $6.674 \times 10^{-11}$ & $< 0.001\%$ & \checkmarkxa Established \\
			Charged Leptons & $99.0\%$ Accuracy & Precisely Known & $\sim 1\%$ & \checkmarkxa Established \\
			Quark Masses & $98.8\%$ Accuracy & Precisely Known & $\sim 2\%$ & \checkmarkxa Established \\
			\midrule
			\textbf{Neutrino Masses (Koide Ext.)} & $m_{\nu_i} \approx 4-5$ meV & $< 100$ meV & Unknown ($\Delta Q_\nu <1\%$) & \warningxa Speculative \\
			\textbf{Neutrino Oscillations} & Geometric Phases + $\delta$ & $\Delta m^2 \neq 0$ & Partially Compatible & \warningxa Problematic \\
			\bottomrule
		\end{tabular}
		\caption{T0 Neutrinos in Comparison to Established T0 Successes (Updated with Koide)}
	\end{table}
	
	\section{Experimental Tests and Falsification}
	
	\subsection{Testable Predictions}
	\label{T0_Neutrinos:L-T0_Neutrinos-0047}
	
\section*{Experimental}
\section*{Specific Experimental Tests of the T0 Neutrino Theory:}
		
		\begin{enumerate}
			\item \textbf{Direct Mass Determination:}
			\begin{itemize}
				\item KATRIN: Sensitivity to $\sim 0.2$ eV (insufficient)
				\item Future Experiments: $\sim 0.01$ eV required
				\item T0 Prediction: $m_{\nu_i} \approx 4-5$ meV (factor 2 below limit)
			\end{itemize}
			
			\item \textbf{Cosmological Precision Measurements:}
			\begin{itemize}
				\item Euclid Satellite: Sensitivity $\sim 0.02$ eV
				\item T0 Prediction: $\Sigma m_\nu = 13.86$ meV (testable!)
			\end{itemize}
			
			\item \textbf{Koide-Specific Tests:}
			\begin{itemize}
				\item Measure $Q_\nu$ via oscillation data: Expect $\approx 2/3$ ($\Delta <1\%$)
				\item PMNS correlations: Hierarchy from $\delta$-rotation
			\end{itemize}
			
			\item \textbf{Speed Measurements:}
			\begin{itemize}
				\item Supernova Neutrinos: $\Delta v/c \sim 10^{-8}$ measurable
				\item T0 Prediction: $\Delta v/c = 8.89 \times 10^{-9}$ (marginal)
			\end{itemize}
			
			\item \textbf{Oscillation Physics:}
			\begin{itemize}
				\item Test for small $\Delta m^2_{ij}$ + phase effects (clearly falsifiable)
			\end{itemize}
		\end{enumerate}
% end box experimental
	
	\subsection{Falsification Criteria}
	\label{T0_Neutrinos:L-T0_Neutrinos-0048}
	
	The T0 Neutrino Theory would be falsified by:
	\begin{enumerate}
		\item Direct measurement of $m_\nu > 0.1$ eV (or strong hierarchy $|m_3 - m_1| > 10$ meV)
		\item Cosmological evidence for $\Sigma m_\nu > 0.1$ eV
		\item Clear proof of $\Delta m^2_{ij} \gg 10^{-4}$ eV$^2$ without phases
		\item Measurement of speed differences $\Delta v/c > 10^{-8}$
		\item Deviation from $Q_\nu \approx 2/3$ in oscillation analyses
	\end{enumerate}
	
	\section{Limits and Open Questions}
	
	\subsection{Fundamental Theoretical Problems}
	\label{T0_Neutrinos:L-T0_Neutrinos-0049}
	
\section*{Warning}
\section*{Unsolved Problems of the T0 Neutrino Theory:}
		
		\begin{enumerate}
			\item \textbf{Oscillation Mechanism:} Geometric phases + $\delta$ are ad hoc
			\item \textbf{Quantum Field Theory:} No complete QFT formulation
			\item \textbf{Experimental Distinguishability:} Difficult to separate from Standard Model
			\item \textbf{Theoretical Consistency:} Partial contradiction to oscillation theory
			\item \textbf{Predictive Power:} Enhanced by Koide, but still limited
		\end{enumerate}
% end box warning
	
	\subsection{Future Developments}
	\label{T0_Neutrinos:L-T0_Neutrinos-0050}
	
	\begin{enumerate}
		\item \textbf{QFT Foundation:} Complete quantum field theory for geometric phases + Koide
		\item \textbf{Experimental Precision:} Cosmological measurements with $\sim 0.01$ eV sensitivity
		\item \textbf{Oscillation Theory:} Rigorous derivation of hybrid effects
		\item \textbf{Unified Description:} Full T0 integration with PMNS
	\end{enumerate}
	
	\section{Methodological Reflection}
	
	\subsection{Scientific Integrity vs. Theoretical Speculation}
	\label{T0_Neutrinos:L-T0_Neutrinos-0051}
	
\section*{Key Result}
\section*{Central Methodological Insights:}
		
		The neutrino chapter of the T0 Theory illustrates the tension between:
		
		\begin{itemize}
			\item \textbf{Theoretical Completeness:} Desire for unified description (now incl. Koide)
			\item \textbf{Empirical Anchoring:} Necessity of experimental confirmation
			\item \textbf{Scientific Honesty:} Disclosure of speculative nature
			\item \textbf{Mathematical Consistency:} Internal self-consistency of formulas
		\end{itemize}
		
		\textbf{Key Insight:} Even speculative theories can be valuable if their limits are honestly communicated.
% end box keyresult
	
	\subsection{Significance for the T0 Series}
	\label{T0_Neutrinos:L-T0_Neutrinos-0052}
	
	The neutrino treatment shows both the strengths and limits of the T0 Theory:
	
	\begin{itemize}
		\item \textbf{Strengths:} Unified framework, elegant analogies, testable predictions (enhanced by Koide)
		\item \textbf{Limits:} Speculative basis, lack of experimental confirmation
		\item \textbf{Scientific Value:} Demonstration of alternative thinking approaches
		\item \textbf{Methodological Importance:} Importance of honest uncertainty communication
	\end{itemize}
	
	\begin{center}
		\hrule
		\vspace{0.5cm}
		\textit{This document is part of the new T0 Series}\\
		\textit{and shows the speculative limits of the T0 Theory}\\
		\vspace{0.3cm}
\section*{T0-Theory: Time-Mass Duality Framework}
		\textit{Johann Pascher, HTL Leonding, Austria}\\
		
		\textit{GitHub: https://github.com/jpascher/T0-Time-Mass-Duality}
		\vspace{0.3cm}
	\end{center}
	
	


% Bibliography
\begin{thebibliography}{99}
	
	\bibitem{pdg2024}
	Particle Data Group Collaboration (2024). 
	\textit{Review of Particle Physics}. 
	Progress of Theoretical and Experimental Physics, 2024(8), 083C01.
	\url{https://pdg.lbl.gov}
	
	\bibitem{flag2024}
	Aoki, Y., et al. (FLAG Collaboration) (2024). 
	\textit{FLAG Review 2024 of Lattice Results for Low-Energy Constants}. 
	arXiv:2411.04268.
	\url{https://arxiv.org/abs/2411.04268}
	
	\bibitem{fermilab_muon_g2}
	Abi, B., et al. (Muon g-2 Collaboration) (2021). 
	\textit{Measurement of the Positive Muon Anomalous Magnetic Moment to 0.46 ppm}. 
	Physical Review Letters, 126, 141801.
	
	\bibitem{peskin_schroeder}
	Peskin, M. E., \& Schroeder, D. V. (1995). 
	\textit{An Introduction to Quantum Field Theory}. 
	Addison-Wesley.
	
	\bibitem{weinberg_qft}
	Weinberg, S. (1995). 
	\textit{The Quantum Theory of Fields, Vol. I--III}. 
	Cambridge University Press.
	
	\bibitem{griffiths_particle}
	Griffiths, D. (2008). 
	\textit{Introduction to Elementary Particles}. 
	Wiley-VCH.
	
	\bibitem{mandl_shaw}
	Mandl, F., \& Shaw, G. (2010). 
	\textit{Quantum Field Theory (2nd ed.)}. 
	Wiley.
	
	\bibitem{srednicki_qft}
	Srednicki, M. (2007). 
	\textit{Quantum Field Theory}. 
	Cambridge University Press.
	
	\bibitem{t0_fundamentals}
	Pascher, J. (2024). 
	\textit{T0-Theory: Foundations of Time-Mass Duality}. 
	Unpublished manuscript, HTL Leonding.
	
	\bibitem{t0_fine_structure}
	Pascher, J. (2024). 
	\textit{T0-Theory: The Fine Structure Constant}. 
	Unpublished manuscript, HTL Leonding.
	
	\bibitem{t0_neutrinos}
	Pascher, J. (2024). 
	\textit{T0-Theory: Neutrino Masses and PMNS Mixing}. 
	Unpublished manuscript, HTL Leonding.
	
	\bibitem{t0_github}
	Pascher, J. (2024--2025). 
	\textit{T0-Time-Mass-Duality Repository}. 
	GitHub.
	\url{https://github.com/jpascher/T0-Time-Mass-Duality}
	
	\bibitem{lattice_qcd_review}
	Kronfeld, A. S. (2012). 
	\textit{Twenty-first Century Lattice Gauge Theory: Results from the QCD Lagrangian}. 
	Annual Review of Nuclear and Particle Science, 62, 265--284.
	
	\bibitem{neutrino_mixing_pdg}
	Particle Data Group Collaboration (2024). 
	\textit{Neutrino Masses, Mixing, and Oscillations}. 
	PDG Review 2024.
	\url{https://pdg.lbl.gov/2024/reviews/rpp2024-rev-neutrino-mixing.pdf}
	
	\bibitem{higgs_discovery}
	ATLAS and CMS Collaborations (2012). 
	\textit{Observation of a New Particle in the Search for the Standard Model Higgs Boson}. 
	Physics Letters B, 716, 1--29.
	
	\bibitem{Brannen2005}
	C. P. Brannen, ``Estimate of neutrino masses from Koide's relation'', \textit{arXiv:hep-ph/0505028} (2005).
	\url{https://arxiv.org/abs/hep-ph/0505028}
	
	\bibitem{Brannen2006}
	C. P. Brannen, ``Koide Mass Formula for Neutrinos'', \textit{arXiv:0702.0052} (2006).
	\url{http://brannenworks.com/MASSES.pdf}
	
	\bibitem{PhaseVectors2025}
	Anonymous, ``The Koide Relation and Lepton Mass Hierarchy from Phase Vectors'', \textit{rXiv:2507.0040} (2025).
	\url{https://rxiv.org/pdf/2507.0040v1.pdf}
	
	\bibitem{PDG2025}
	Particle Data Group, ``Review of Particle Physics'', \textit{Phys. Rev. D} \textbf{112} (2025) 030001.
	\url{https://pdg.lbl.gov/2025/}
	
	\bibitem{terrell2024}
	Terrell et al. (2024). 
	\textit{Single-Clock Metrology in Nature}. 
	Nature Physics.
	
	\bibitem{hossenfelder2024}
	Hossenfelder, S. (2024). 
	\textit{Single Clock Video Explanation}. 
	YouTube.
	
	\bibitem{hundert1931}
	Hundert (1931). 
	\textit{Reference Work}. 
	Publisher.
	
	\bibitem{terrell2025}
	Terrell et al. (2025). 
	\textit{Advanced Clock Synchronization Methods}. 
	Physical Review Letters.
	
	\bibitem{pascher_t0_2025}
	Pascher, J. (2025). 
	\textit{T0-Theory: Complete Framework and Applications}. 
	Unpublished manuscript, HTL Leonding.
	
	\bibitem{t0qm}
	Pascher, J. (2024). 
	\textit{T0-Theory: Quantum Mechanics Formulation}. 
	Unpublished manuscript, HTL Leonding.
	
	\bibitem{t0anomale}
	Pascher, J. (2024). 
	\textit{T0-Theory: Anomalous Magnetic Moments}. 
	Unpublished manuscript, HTL Leonding.
	
	\bibitem{muong2complete}
	Abi, B., et al. (Muon g-2 Collaboration) (2023). 
	\textit{Complete Measurement of the Positive Muon Anomalous Magnetic Moment}. 
	Physical Review Letters, 131, 161802.
	
	\bibitem{penrose2004}
	Penrose, R. (2004). 
	\textit{The Road to Reality: A Complete Guide to the Laws of the Universe}. 
	Jonathan Cape.
	
	\bibitem{planck1900}
	Planck, M. (1900). 
	\textit{On the Theory of the Energy Distribution Law of the Normal Spectrum}. 
	Verhandlungen der Deutschen Physikalischen Gesellschaft, 2, 237.
	
	\bibitem{T0Theory}
	Pascher, J. (2024). 
	\textit{T0-Theory: Fundamental Principles}. 
	Unpublished manuscript, HTL Leonding.
	
	% Additional bibliography entries for all undefined citations
	\bibitem{6g_roadmap}
	6G Research Consortium (2024).
	\textit{6G Technology Roadmap}.
	Technical Report.
	
	\bibitem{Born2013}
	Born, M. (2013).
	\textit{Einstein's Theory of Relativity}.
	Dover Publications.
	
	\bibitem{Casimir1948}
	Casimir, H. B. G. (1948).
	\textit{On the attraction between two perfectly conducting plates}.
	Proc. Kon. Ned. Akad. Wetensch. B51, 793--795.
	
	\bibitem{Einstein1905}
	Einstein, A. (1905).
	\textit{On the Electrodynamics of Moving Bodies}.
	Annalen der Physik, 17, 891--921.
	
	\bibitem{Feynman2006}
	Feynman, R. P. (2006).
	\textit{QED: The Strange Theory of Light and Matter}.
	Princeton University Press.
	
	\bibitem{Griffiths2017}
	Griffiths, D. J. (2017).
	\textit{Introduction to Electrodynamics (4th ed.)}.
	Cambridge University Press.
	
	\bibitem{Jackson1999}
	Jackson, J. D. (1999).
	\textit{Classical Electrodynamics (3rd ed.)}.
	Wiley.
	
	\bibitem{Mohr2016}
	Mohr, P. J., et al. (2016).
	\textit{CODATA Recommended Values of the Fundamental Physical Constants: 2014}.
	Rev. Mod. Phys. 88, 035009.
	
	\bibitem{Parker2018}
	Parker, R. H., et al. (2018).
	\textit{Measurement of the fine-structure constant as a test of the Standard Model}.
	Science, 360, 191--195.
	
	\bibitem{Planck1900}
	Planck, M. (1900).
	\textit{On the Theory of the Energy Distribution Law of the Normal Spectrum}.
	Verhandlungen der Deutschen Physikalischen Gesellschaft, 2, 237.
	
	\bibitem{Planck2018}
	Planck Collaboration (2018).
	\textit{Planck 2018 results. VI. Cosmological parameters}.
	Astronomy \& Astrophysics, 641, A6.
	
	\bibitem{QFT_T0}
	Pascher, J. (2024).
	\textit{T0-Theory and QFT Connections}.
	Unpublished manuscript, HTL Leonding.
	
	\bibitem{Sommerfeld1916}
	Sommerfeld, A. (1916).
	\textit{On the Quantum Theory of Spectral Lines}.
	Annalen der Physik, 51, 1--94.
	
	\bibitem{T0_Feinstruktur}
	Pascher, J. (2024).
	\textit{T0-Theory: Fine Structure Analysis}.
	Unpublished manuscript, HTL Leonding.
	
	\bibitem{T0_SI}
	Pascher, J. (2024).
	\textit{T0-Theory and SI Units}.
	Unpublished manuscript, HTL Leonding.
	
	\bibitem{T0_fine_structure}
	Pascher, J. (2024).
	\textit{T0-Theory: The Fine Structure Constant}.
	Unpublished manuscript, HTL Leonding.
	
	\bibitem{T0_g2_erweiterung}
	Pascher, J. (2024).
	\textit{T0-Theory: g-2 Extensions}.
	Unpublished manuscript, HTL Leonding.
	
	\bibitem{T0_gravitational_constant}
	Pascher, J. (2024).
	\textit{T0-Theory: Gravitational Constant Derivation}.
	Unpublished manuscript, HTL Leonding.
	
	\bibitem{T0_netze_en}
	Pascher, J. (2024).
	\textit{T0-Theory: Network Structures}.
	Unpublished manuscript, HTL Leonding.
	
	\bibitem{T0_tm_erweiterung}
	Pascher, J. (2024).
	\textit{T0-Theory: Time-Mass Extensions}.
	Unpublished manuscript, HTL Leonding.
	
	\bibitem{Uzan2003}
	Uzan, J.-P. (2003).
	\textit{The fundamental constants and their variation}.
	Rev. Mod. Phys. 75, 403--455.
	
	\bibitem{Weinberg1995}
	Weinberg, S. (1995).
	\textit{The Quantum Theory of Fields, Vol. I}.
	Cambridge University Press.
	
	\bibitem{albrecht1999}
	Albrecht, A. \& Magueijo, J. (1999).
	\textit{A time varying speed of light as a solution to cosmological puzzles}.
	Phys. Rev. D 59, 043516.
	
	\bibitem{alice2023}
	ALICE Collaboration (2023).
	\textit{Recent results from ALICE}.
	CERN-EP-2023-XXX.
	
	\bibitem{analog_optical}
	Smith, J. et al. (2024).
	\textit{Analog optical computing systems}.
	Nature Photonics.
	
	\bibitem{ashtekar2004}
	Ashtekar, A. \& Lewandowski, J. (2004).
	\textit{Background independent quantum gravity}.
	Class. Quantum Grav. 21, R53.
	
	\bibitem{atlas2023}
	ATLAS Collaboration (2023).
	\textit{ATLAS physics results}.
	CERN-PH-EP-2023-XXX.
	
	\bibitem{atlas2023higgs}
	ATLAS Collaboration (2023).
	\textit{Higgs boson measurements}.
	Phys. Rev. Lett.
	
	\bibitem{barbour1999}
	Barbour, J. (1999).
	\textit{The End of Time}.
	Oxford University Press.
	
	\bibitem{barrow1999}
	Barrow, J. D. (1999).
	\textit{Cosmologies with varying light speed}.
	Phys. Rev. D 59, 043515.
	
	\bibitem{becker2007}
	Becker, K. et al. (2007).
	\textit{String Theory and M-Theory}.
	Cambridge University Press.
	
	\bibitem{bell_muon}
	Bennett, G. W., et al. (Muon g-2 Collaboration) (2006).
	\textit{Final report of the E821 muon anomalous magnetic moment measurement}.
	Phys. Rev. D 73, 072003.
	
	\bibitem{bondi1948}
	Bondi, H. \& Gold, T. (1948).
	\textit{The steady-state theory of the expanding universe}.
	Mon. Not. R. Astron. Soc. 108, 252--270.
	
	\bibitem{brewer2019}
	Brewer, S. M. et al. (2019).
	\textit{Al+ Quantum-Logic Clock with Systematic Uncertainty below $10^{-18}$}.
	Phys. Rev. Lett. 123, 033201.
	
	\bibitem{cms2023top}
	CMS Collaboration (2023).
	\textit{Top quark measurements at CMS}.
	JHEP 2023.
	
	\bibitem{cms2024}
	CMS Collaboration (2024).
	\textit{CMS physics results 2024}.
	CERN-PH-EP-2024-XXX.
	
	\bibitem{codata2019}
	Tiesinga, E. et al. (2019).
	\textit{The 2018 CODATA Recommended Values}.
	J. Phys. Chem. Ref. Data.
	
	\bibitem{desi2025}
	DESI Collaboration (2025).
	\textit{DESI 2025 Cosmology Results}.
	arXiv preprint.
	
	\bibitem{differential_optical}
	Wang, X. et al. (2024).
	\textit{Differential optical computing}.
	Optica.
	
	\bibitem{dingle1972}
	Dingle, H. (1972).
	\textit{Science at the Crossroads}.
	Martin Brian \& O'Keeffe.
	
	\bibitem{divalentino2021}
	Di Valentino, E. et al. (2021).
	\textit{In the realm of the Hubble tension}.
	Class. Quantum Grav. 38, 153001.
	
	\bibitem{elnaschie2004}
	El Naschie, M. S. (2004).
	\textit{A review of E infinity theory}.
	Chaos, Solitons \& Fractals, 19, 209--236.
	
	\bibitem{fabrication_heterogeneous}
	Chen, Y. et al. (2024).
	\textit{Heterogeneous photonic integration}.
	Nature Electronics.
	
	\bibitem{fermilab2023}
	Fermilab (2023).
	\textit{Muon g-2 results}.
	Phys. Rev. Lett.
	
	\bibitem{flexible_wafer}
	Kim, S. et al. (2024).
	\textit{Flexible wafer-scale photonics}.
	Science Advances.
	
	\bibitem{francesco1997}
	Di Francesco, P. et al. (1997).
	\textit{Conformal Field Theory}.
	Springer.
	
	\bibitem{hartree1957}
	Hartree, D. R. (1957).
	\textit{The Calculation of Atomic Structures}.
	Wiley.
	
	\bibitem{hhi_6g}
	Fraunhofer HHI (2024).
	\textit{6G Photonic Integration}.
	Technical Report.
	
	\bibitem{hossenfelder2025}
	Hossenfelder, S. (2025).
	\textit{Science without the gobbledygook}.
	YouTube/Blog.
	
	\bibitem{hossenfelder_single_clock_video}
	Hossenfelder, S. (2024).
	\textit{The Single Clock Problem}.
	YouTube.
	
	\bibitem{hoyle1948}
	Hoyle, F. (1948).
	\textit{A new model for the expanding universe}.
	Mon. Not. R. Astron. Soc. 108, 372--382.
	
	\bibitem{integration_microelectronic}
	Liu, A. et al. (2024).
	\textit{Microelectronic photonic integration}.
	IEEE Journal.
	
	\bibitem{jacobson1995}
	Jacobson, T. (1995).
	\textit{Thermodynamics of spacetime}.
	Phys. Rev. Lett. 75, 1260.
	
	\bibitem{kasevich2023}
	Kasevich, M. et al. (2023).
	\textit{Atom interferometry tests}.
	Nature Physics.
	
	\bibitem{lerner2014}
	Lerner, E. J. (2014).
	\textit{An open letter on cosmology}.
	New Scientist.
	
	\bibitem{lisa2017}
	LISA Consortium (2017).
	\textit{Laser Interferometer Space Antenna}.
	ESA Technical Report.
	
	\bibitem{lithium_tantalate}
	Zhang, M. et al. (2024).
	\textit{Thin-film lithium tantalate photonics}.
	Nature Photonics.
	
	\bibitem{lopez2010}
	Lopez-Corredoira, M. (2010).
	\textit{Tests and problems of the standard model in cosmology}.
	Int. J. Mod. Phys. D.
	
	\bibitem{ludlow2015}
	Ludlow, A. D. et al. (2015).
	\textit{Optical atomic clocks}.
	Rev. Mod. Phys. 87, 637.
	
	\bibitem{mach1883}
	Mach, E. (1883).
	\textit{Die Mechanik in ihrer Entwickelung}.
	F.A. Brockhaus.
	
	\bibitem{maldacena1998}
	Maldacena, J. (1998).
	\textit{The large N limit of superconformal field theories}.
	Adv. Theor. Math. Phys. 2, 231--252.
	
	\bibitem{mueller2014}
	Müller, H. et al. (2014).
	\textit{Atom interferometry tests of the gravitational redshift}.
	Phys. Rev. Lett.
	
	\bibitem{mug2_final_2025}
	Muon g-2 Collaboration (2025).
	\textit{Final muon g-2 measurement}.
	Phys. Rev. Lett.
	
	\bibitem{muong2_2023}
	Muon g-2 Collaboration (2023).
	\textit{Updated muon g-2 results}.
	Phys. Rev. Lett.
	
	\bibitem{nathan2024}
	Nathan, A. et al. (2024).
	\textit{Quantum computing advances}.
	Nature.
	
	\bibitem{newell2018}
	Newell, D. B. et al. (2018).
	\textit{The CODATA 2017 values of h, e, k, and $N_A$}.
	Metrologia 55, L13.
	
	\bibitem{nottale1993}
	Nottale, L. (1993).
	\textit{Fractal Space-Time and Microphysics}.
	World Scientific.
	
	\bibitem{on_chip_lithium}
	Wang, C. et al. (2024).
	\textit{On-chip lithium niobate photonics}.
	Nature Communications.
	
	\bibitem{optical_advantages}
	Shastri, B. J. et al. (2024).
	\textit{Advantages of optical computing}.
	Nature Reviews Physics.
	
	\bibitem{pascher2025cmb}
	Pascher, J. (2025).
	\textit{T0-Theory: CMB Analysis}.
	Unpublished manuscript, HTL Leonding.
	
	\bibitem{pascher2025g2}
	Pascher, J. (2025).
	\textit{T0-Theory: g-2 Predictions}.
	Unpublished manuscript, HTL Leonding.
	
	\bibitem{pascher2025qm}
	Pascher, J. (2025).
	\textit{T0-Theory: Quantum Mechanics}.
	Unpublished manuscript, HTL Leonding.
	
	\bibitem{pascher2025si}
	Pascher, J. (2025).
	\textit{T0-Theory: SI Unit System}.
	Unpublished manuscript, HTL Leonding.
	
	\bibitem{pascher2025t0}
	Pascher, J. (2025).
	\textit{T0-Theory: Complete Framework}.
	Unpublished manuscript, HTL Leonding.
	
	\bibitem{pascher:fundamentals}
	Pascher, J. (2024).
	\textit{T0-Theory: Fundamentals}.
	Unpublished manuscript, HTL Leonding.
	
	\bibitem{pascher:g2_rev9}
	Pascher, J. (2024).
	\textit{T0-Theory: g-2 Revision 9}.
	Unpublished manuscript, HTL Leonding.
	
	\bibitem{pascher:geometric_formalism}
	Pascher, J. (2024).
	\textit{T0-Theory: Geometric Formalism}.
	Unpublished manuscript, HTL Leonding.
	
	\bibitem{pascher:ml_addendum}
	Pascher, J. (2024).
	\textit{T0-Theory: Machine Learning Addendum}.
	Unpublished manuscript, HTL Leonding.
	
	\bibitem{pascher:t0_foundations}
	Pascher, J. (2024).
	\textit{T0-Theory: Foundations}.
	Unpublished manuscript, HTL Leonding.
	
	\bibitem{pascher_derivation_beta_2025}
	Pascher, J. (2025).
	\textit{T0-Theory: Derivation of Beta}.
	Unpublished manuscript, HTL Leonding.
	
	\bibitem{pascher_higgs_connection_2025}
	Pascher, J. (2025).
	\textit{T0-Theory: Higgs Connection}.
	Unpublished manuscript, HTL Leonding.
	
	\bibitem{pascher_lagrangian_extended_2025}
	Pascher, J. (2025).
	\textit{T0-Theory: Extended Lagrangian}.
	Unpublished manuscript, HTL Leonding.
	
	\bibitem{pascher_mathematical_structure_2025}
	Pascher, J. (2025).
	\textit{T0-Theory: Mathematical Structure}.
	Unpublished manuscript, HTL Leonding.
	
	\bibitem{pascher_t0_cmb_2025}
	Pascher, J. (2025).
	\textit{T0-Theory: CMB Predictions}.
	Unpublished manuscript, HTL Leonding.
	
	\bibitem{pascher_t0_energie_2025}
	Pascher, J. (2025).
	\textit{T0-Theory: Energy}.
	Unpublished manuscript, HTL Leonding.
	
	\bibitem{pascher_t0_energy_2025}
	Pascher, J. (2025).
	\textit{T0-Theory: Energy Framework}.
	Unpublished manuscript, HTL Leonding.
	
	\bibitem{pascher_t0_theory_2025}
	Pascher, J. (2025).
	\textit{T0-Theory: Complete Theory}.
	Unpublished manuscript, HTL Leonding.
	
	\bibitem{penrose1959}
	Penrose, R. (1959).
	\textit{The apparent shape of a relativistically moving sphere}.
	Proc. Cambridge Phil. Soc. 55, 137--139.
	
	\bibitem{penrose1967}
	Penrose, R. (1967).
	\textit{Twistor algebra}.
	J. Math. Phys. 8, 345--366.
	
	\bibitem{peratt1992}
	Peratt, A. L. (1992).
	\textit{Physics of the Plasma Universe}.
	Springer-Verlag.
	
	\bibitem{peskin1995}
	Peskin, M. E. \& Schroeder, D. V. (1995).
	\textit{An Introduction to Quantum Field Theory}.
	Addison-Wesley.
	
	\bibitem{peskin_schroeder_1995}
	Peskin, M. E. \& Schroeder, D. V. (1995).
	\textit{An Introduction to Quantum Field Theory}.
	Addison-Wesley.
	
	\bibitem{phoquant}
	PhoQuant (2024).
	\textit{Photonic quantum computing}.
	Technical Report.
	
	\bibitem{photonics_ai}
	Wetzstein, G. et al. (2024).
	\textit{Photonics for AI}.
	Nature.
	
	\bibitem{planck1906}
	Planck, M. (1906).
	\textit{The Theory of Heat Radiation}.
	Johann Ambrosius Barth.
	
	\bibitem{planck2018}
	Planck Collaboration (2018).
	\textit{Planck 2018 results}.
	A\&A 641, A6.
	
	\bibitem{polchinski1998}
	Polchinski, J. (1998).
	\textit{String Theory}.
	Cambridge University Press.
	
	\bibitem{qant_nps}
	QANT (2024).
	\textit{Quantum photonics systems}.
	Technical Report.
	
	\bibitem{quantenjahr25}
	Quantenjahr (2025).
	\textit{International Year of Quantum}.
	UNESCO.
	
	\bibitem{recurrent_photonics}
	Tait, A. N. et al. (2024).
	\textit{Recurrent photonic neural networks}.
	Optica.
	
	\bibitem{rf_photonics}
	Capmany, J. \& Novak, D. (2024).
	\textit{Microwave photonics}.
	Nature Photonics.
	
	\bibitem{riess2019}
	Riess, A. G. et al. (2019).
	\textit{Large Magellanic Cloud Cepheid Standards}.
	ApJ 876, 85.
	
	\bibitem{riess2022}
	Riess, A. G. et al. (2022).
	\textit{A Comprehensive Measurement of H0}.
	ApJ 934, L7.
	
	\bibitem{rovelli2004}
	Rovelli, C. (2004).
	\textit{Quantum Gravity}.
	Cambridge University Press.
	
	\bibitem{sciama1953}
	Sciama, D. W. (1953).
	\textit{On the origin of inertia}.
	Mon. Not. R. Astron. Soc. 113, 34--42.
	
	\bibitem{sciencedaily2025}
	ScienceDaily (2025).
	\textit{Physics news}.
	Online.
	
	\bibitem{sm_g2_2025}
	Aoyama, T. et al. (2025).
	\textit{Standard Model prediction for g-2}.
	Phys. Rep.
	
	\bibitem{susskind1995}
	Susskind, L. (1995).
	\textit{The world as a hologram}.
	J. Math. Phys. 36, 6377--6396.
	
	\bibitem{t0_kosmologie}
	Pascher, J. (2024).
	\textit{T0-Theory: Cosmology}.
	Unpublished manuscript, HTL Leonding.
	
	\bibitem{terrell1959}
	Terrell, J. (1959).
	\textit{Invisibility of the Lorentz contraction}.
	Phys. Rev. 116, 1041--1045.
	
	\bibitem{terrell_single_clock_nature_2024}
	Terrell, J. et al. (2024).
	\textit{Single clock precision measurements}.
	Nature Physics.
	
	\bibitem{tfln_foundry}
	TFLN Foundry (2024).
	\textit{Thin-film lithium niobate foundry services}.
	Technical Specifications.
	
	\bibitem{thiemann2007}
	Thiemann, T. (2007).
	\textit{Modern Canonical Quantum General Relativity}.
	Cambridge University Press.
	
	\bibitem{thz_epfl}
	EPFL (2024).
	\textit{Terahertz photonics research}.
	Technical Report.
	
	\bibitem{unnikrishnan2004}
	Unnikrishnan, C. S. (2004).
	\textit{On Einstein's resolution of the twin clock paradox}.
	Current Science, 86, 704--709.
	
	\bibitem{verlinde2011}
	Verlinde, E. (2011).
	\textit{On the origin of gravity and the laws of Newton}.
	JHEP 2011, 29.
	
	\bibitem{video2025}
	Video (2025).
	\textit{Physics video explanation}.
	YouTube.
	
	\bibitem{weinberg1995}
	Weinberg, S. (1995).
	\textit{The Quantum Theory of Fields}.
	Cambridge University Press.
	
	\bibitem{weiskopf2000}
	Weiskopf, D. (2000).
	\textit{Visualization of special relativity}.
	PhD thesis, University of Tübingen.
	
	\bibitem{wheeler1990}
	Wheeler, J. A. (1990).
	\textit{A Journey into Gravity and Spacetime}.
	Scientific American Library.
	
	\bibitem{wiki_bell}
	Wikipedia (2024).
	\textit{Bell's theorem}.
	Online encyclopedia.
	
	\bibitem{zwicky1929}
	Zwicky, F. (1929).
	\textit{On the red shift of spectral lines through interstellar space}.
	Proc. Natl. Acad. Sci. 15, 773--779.

\end{thebibliography}


\end{document}
