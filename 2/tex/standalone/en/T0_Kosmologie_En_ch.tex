\documentclass[11pt,a4paper]{article}
\usepackage[a4paper,margin=2cm]{geometry}
\usepackage[utf8]{inputenc}
\usepackage[english]{babel}
\usepackage{lmodern}
\renewcommand{\familydefault}{\sfdefault}

\usepackage{amsmath,amssymb,amsthm}
\usepackage{graphicx}
\usepackage[unicode,pdfencoding=auto,hypertexnames=false]{hyperref}
\usepackage{booktabs}
\usepackage{longtable}
\usepackage{array}
\usepackage{siunitx}
\usepackage{fancyhdr}
\usepackage{float}
\usepackage{tikz}
% tcolorbox removed for standalone
% tcbset removed
\tikzset{
  t0blue/.style={draw=blue,fill=blue!10},
  t0red/.style={draw=red,fill=red!10},
  t0green/.style={draw=green!50!black,fill=green!10},
  t0orange/.style={draw=orange,fill=orange!10},
}
\usepackage{setspace}
\usepackage{enumitem}
\usepackage{adjustbox}
\usepackage{xcolor}

% Define colors for xcolor package
\definecolor{t0green}{RGB}{34,139,34}
\definecolor{t0blue}{RGB}{0,0,255}
\definecolor{t0red}{RGB}{255,0,0}
\definecolor{t0orange}{RGB}{255,165,0}

% Define custom column types for tables
\newcolumntype{L}[1]{>{\raggedright\arraybackslash}p{#1}}
\newcolumntype{C}[1]{>{\centering\arraybackslash}p{#1}}
\newcolumntype{R}[1]{>{\raggedleft\arraybackslash}p{#1}}

\setlength{\parindent}{0pt}
\setlength{\parskip}{6pt}

\hypersetup{
  colorlinks=true,
  linkcolor=blue,
  citecolor=blue,
  urlcolor=blue
}
\pagestyle{fancy}
\setlength{\headheight}{28pt}

\newcommand{\checkmarkx}{\checkmark}
\newcommand{\warningx}{\textbf{!}}

% Makros aus Einzel-Dokumenten (Fallback-Definitionen)
\newcommand{\mytimes}{\times}
\newcommand{\myapprox}{\approx}
\newcommand{\mysim}{\sim}
\newcommand{\myomega}{\omega}
\newcommand{\mypi}{\pi}
\newcommand{\myrightarrow}{\rightarrow}
\newcommand{\mypropto}{\propto}
\newcommand{\deltafield}{\delta\phi}
\newcommand{\xipar}{\xi}
\newcommand{\xiT}{\xi}
\newcommand{\lambdah}{\lambda_h}

% Additional macros used in chapter files
\newcommand{\Kfrak}{K_{\text{frak}}}  % Fractal correction factor
\newcommand{\Dfrak}{D_f}              % Fractal dimension
\newcommand{\betapar}{\beta}          % T0 beta parameter
\newcommand{\alphapar}{\alpha}        % T0 alpha parameter
\newcommand{\Efield}{E}               % Energy field
% Note: checkmarkxa/warningxa are variants used in auto-generated chapter files
\newcommand{\checkmarkxa}{\checkmark}
\newcommand{\warningxa}{\textbf{!}}

% Additional T0-specific macros
\newcommand{\xigeom}{\xi_{\text{geom}}}  % Geometric xi
\newcommand{\lP}{\ell_P}                  % Planck length
\newcommand{\rzero}{r_0}                  % Characteristic radius
\newcommand{\xirat}{\xi_{\text{rat}}}     % Xi ratio
\newcommand{\tzero}{t_0}                  % Characteristic time
\newcommand{\natunits}{\text{(nat. units)}}  % Natural units annotation
\newcommand{\myRightarrow}{\Rightarrow}   % Arrow variant
\newcommand{\Lag}{\mathcal{L}}            % Lagrangian

% Physics macros used in chapter files
\newcommand{\CQCD}{C_{\text{QCD}}}        % QCD correction
\newcommand{\EP}{E_P}                     % Planck energy
\newcommand{\Ee}{E_e}                     % Electron energy
\newcommand{\Emu}{E_\mu}                  % Muon energy
\newcommand{\Exi}{E_\xi}                  % Xi energy
\newcommand{\Ezero}{E_0}                  % Characteristic energy
\newcommand{\Hubble}{H}                   % Hubble constant
\newcommand{\Kspec}{K_{\text{spec}}}      % Spectral correction
\newcommand{\Lambdat}{\Lambda_t}          % Time-related cosmological constant
\newcommand{\Leff}{\mathcal{L}_{\text{eff}}}  % Effective Lagrangian
\newcommand{\Lorentz}{\mathcal{L}}        % Lorentz symbol
\newcommand{\Lxi}{L_\xi}                  % Xi length
\newcommand{\Tfield}{T}                   % Time field
\newcommand{\Weyl}{W}                     % Weyl tensor/symbol
\newcommand{\alphaEMSI}{\alpha_{\text{EM,SI}}}  % EM alpha in SI
\newcommand{\alphaEMnat}{\alpha_{\text{EM,nat}}}  % EM alpha in natural units
\newcommand{\alphaem}{\alpha_{\text{em}}} % Electromagnetic alpha
\newcommand{\betaTSI}{\beta_{T,\text{SI}}}  % Beta in SI
\newcommand{\betaTnat}{\beta_{T,\text{nat}}}  % Beta in natural units
\newcommand{\deltam}{\delta m}            % Mass difference
\newcommand{\phiT}{\phi_T}                % T-field phi
\newcommand{\tP}{t_P}                     % Planck time
\newcommand{\rhoCMB}{\rho_{\text{CMB}}}   % CMB density
\newcommand{\rhoCasimir}{\rho_{\text{Casimir}}}  % Casimir density

% Table formatting
\usepackage{multirow}

% Additional physics macros
\newcommand{\Riem}{\mathcal{R}}           % Riemann tensor
\newcommand{\ZPinch}{Z_{\text{pinch}}}    % Z-pinch
\newcommand{\SynchPower}{P_{\text{synch}}} % Synchrotron power
\newcommand{\Rzero}{R_0}                  % Characteristic radius
\newcommand{\alphafine}{\alpha}           % Fine structure constant
\newcommand{\Etau}{E_\tau}                % Tau energy
\newcommand{\deltaE}{\delta E}            % Energy deviation
\newcommand{\EPlanck}{E_P}                % Planck energy
\newcommand{\pichar}{\pi}                 % Pi character
\newcommand{\alphaWSI}{\alpha_{W,\text{SI}}}  % Wien alpha in SI
\newcommand{\alphaWnat}{\alpha_{W,\text{nat}}}  % Wien alpha in natural units

% Einfache abstract-Umgebung für Kapitel:
\newenvironment{abstract}{%
  \begin{center}\bfseries Abstract\end{center}\small
}{\par}


\title{T0 Kosmologie En}
\author{J. Pascher}
\date{\today}

\begin{document}
\maketitle

\section*{T0 Kosmologie (T0 Kosmologie)}

	\begin{abstract}
		This document presents the cosmological aspects of the T0-Theory with the universal $\xi$-parameter as the foundation for a static, eternally existing universe. Based on the time-energy duality, it is shown that a Big Bang is physically impossible and that the cosmic microwave background radiation (CMB) as well as the Casimir effect can be understood as two manifestations of the same $\xi$-field. As the sixth document of the T0 series, it integrates the cosmological applications of all established basic principles.
	\end{abstract}
	
	
	\section{Introduction}
	
	\subsection{Cosmology within the Framework of the T0-Theory}
	
	The T0-Theory revolutionizes our understanding of the universe through the introduction of a fundamental relationship between the microscopic quantum vacuum and macroscopic cosmic structures. All cosmological phenomena can be derived from the universal parameter $\xipar = \frac{4}{3} \times 10^{-4}$.
	
\section*{Key Result}
\section*{Central Thesis of T0-Cosmology:}
		
		The universe is static and eternally existing. All observed cosmic phenomena arise from manifestations of the fundamental $\xi$-field, not from spacetime expansion.
% end box keyresult
	
	\subsection{Connection to the T0 Document Series}
	
	This cosmological analysis builds on the fundamental insights of the previous T0 documents:
	
	\begin{itemize}
		\item \textbf{T0\_Basics\_En.tex:} Geometric parameter $\xipar$ and fractal spacetime structure
		\item \textbf{T0\_FineStructure\_En.tex:} Electromagnetic interactions in the $\xi$-field
		\item \textbf{T0\_GravitationalConstant\_En.tex:} Gravitation theory from $\xi$-geometry
		\item \textbf{T0\_ParticleMasses\_En.tex:} Mass spectrum as the basis for cosmic structure formation
		\item \textbf{T0\_Neutrinos\_En.tex:} Neutrino oscillations in cosmic dimensions
	\end{itemize}
	
	\section{Time-Energy Duality and the Static Universe}
	
	\subsection{Heisenberg's Uncertainty Principle as a Cosmological Principle}
	
\section*{Revolutionary}
\section*{Fundamental Insight:}
		
		Heisenberg's uncertainty principle $\Delta E \times \Delta t \geq \frac{\hbar}{2}$ irrefutably proves that a Big Bang is physically impossible.
% end box revolutionary
	
	In natural units ($\hbar = c = k_B = 1$), the time-energy uncertainty relation reads:
	
	\begin{equation}
		\Delta E \times \Delta t \geq \frac{1}{2}
	\end{equation}
	
	The cosmological consequences are far-reaching:
	
	\begin{itemize}
		\item A temporal beginning (Big Bang) would imply $\Delta t$ = finite
		\item This leads to $\Delta E \to \infty$ - physically inconsistent
		\item Therefore, the universe must have existed eternally: $\Delta t = \infty$
		\item The universe is static, without expanding space
	\end{itemize}
	
	\subsection{Consequences for Standard Cosmology}
	
\section*{Warning}
\section*{Problems of Big Bang Cosmology:}
		
		\begin{enumerate}
			\item \textbf{Violation of Quantum Mechanics:} Finite $\Delta t$ requires infinite energy
			\item \textbf{Fine-Tuning Problems:} Over 20 free parameters required
			\item \textbf{Dark Matter/Energy:} 95\% unknown components
			\item \textbf{Hubble Tension:} 9\% discrepancy between local and cosmic measurements
			\item \textbf{Age Problem:} Objects older than the supposed age of the universe
		\end{enumerate}
% end box warning
	
	\section{The Cosmic Microwave Background Radiation (CMB)}
	
	\subsection{CMB as $\xi$-Field Manifestation}
	
	Since the time-energy duality prohibits a Big Bang, the CMB must have a different origin than the z=1100 decoupling of standard cosmology. The T0-Theory explains the CMB through $\xi$-field quantum fluctuations.
	
\section*{Formula}
\section*{T0-CMB-Temperature Relation:}
		\begin{equation}
			\frac{T_{\text{CMB}}}{\Exi} = \frac{16}{9} \xipar^2
		\end{equation}
% end box formula
	
	With $\Exi = \frac{1}{\xipar} = \frac{3}{4} \times 10^4$ (natural units) and $\xipar = \frac{4}{3} \times 10^{-4}$, the result is:
	
	\begin{align}
		T_{\text{CMB}} &= \frac{16}{9} \xipar^2 \times \Exi \\
		&= \frac{16}{9} \times \left(\frac{4}{3} \times 10^{-4}\right)^2 \times \frac{3}{4} \times 10^4 \\
		&= \frac{16}{9} \times 1.78 \times 10^{-8} \times 7500 \\
		&= 2.35 \times 10^{-4} \text{ (natural units)}
	\end{align}
	
	\textbf{Conversion to SI Units:} $T_{\text{CMB}} = 2.725$ K
	
	This agrees perfectly with Planck observations!
	
	\subsection{CMB Energy Density and Characteristic Length Scale}
	
	The CMB energy density defines a fundamental characteristic length scale of the $\xi$-field:
	
	\begin{equation}
		\rhoCMB = \frac{\xipar}{\Lxi^4}
	\end{equation}
	
	From this follows the characteristic $\xi$-length scale:
	
	\begin{equation}
		\Lxi = \left(\frac{\xipar}{\rhoCMB}\right)^{1/4}
	\end{equation}
	
\section*{Key Result}
\section*{Characteristic $\xi$-Length Scale:}
		
		Using the experimental CMB data, the result is:
		\begin{equation}
			\Lxi = 100 \, \mu\text{m}
		\end{equation}
		
		This length scale marks the transition region between microscopic quantum effects and macroscopic cosmic phenomena.
% end box keyresult
	
	\section{Casimir Effect and $\xi$-Field Connection}
	
	\subsection{Casimir-CMB Ratio as Experimental Confirmation}
	
	The ratio between Casimir energy density and CMB energy density confirms the characteristic $\xi$-length scale and demonstrates the fundamental unity of the $\xi$-field.
	
	The Casimir energy density at plate separation $d = \Lxi$ is:
	
	\begin{equation}
		|\rhoCasimir| = \frac{\pi^2 \hbar c}{240 \times \Lxi^4}
	\end{equation}
	
	The theoretical ratio yields:
	
	\begin{equation}
		\frac{|\rhoCasimir|}{\rhoCMB} = \frac{\pi^2}{240 \xipar} = \frac{\pi^2 \times 10^4}{320} \approx 308
	\end{equation}
	
\section*{Experiment}
\section*{Experimental Verification:}
		
		The Python verification script \texttt{CMB\_En.py} (available on GitHub: \url{https://github.com/jpascher/T0-Time-Mass-Duality}) confirms:
		
		\begin{itemize}
			\item Theoretical Prediction: 308
			\item Experimental Value: 312
			\item Agreement: 98.7\% (1.3\% deviation)
		\end{itemize}
% end box experiment
	
	\subsection{$\xi$-Field as Universal Vacuum}
	
\section*{Revolutionary}
\section*{Fundamental Insight:}
		
		The $\xi$-field manifests itself both in the free CMB radiation and in the geometrically confined Casimir vacuum. This proves the fundamental reality of the $\xi$-field as the universal quantum vacuum.
% end box revolutionary
	
	The characteristic $\xi$-length scale $\Lxi$ is the point where CMB vacuum energy density and Casimir energy density reach comparable orders of magnitude:
	
	\begin{align}
		\text{Free Vacuum:} \quad &\rhoCMB = +4.87 \times 10^{41} \text{ (natural units)} \\
		\text{Confined Vacuum:} \quad &|\rhoCasimir| = \frac{\pi^2}{240 d^4}
	\end{align}
	
	\section{Cosmic Redshift: Alternative Interpretations}
	
	\subsection{The Mathematical Model of the T0-Theory}
	
	The T0-Theory provides a mathematical model for the observed cosmic redshift that **allows alternative interpretations**, without committing to a specific physical cause.
	
\section*{Formula}
\section*{Fundamental T0-Redshift Model:}
		\begin{equation}
			z(\lambda_0, d) = \frac{\xipar \cdot d \cdot \lambda_0}{\Exi}
		\end{equation}
		where $\lambda_0$ is the emitted wavelength, $d$ the distance, and $\Exi$ the characteristic $\xi$-energy.
% end box formula
	
	\subsection{Alternative Physical Interpretations}
	
	The same mathematical model can be realized through different physical mechanisms:
	
\section*{Alternative}
\section*{Interpretation 1: Energy Loss Mechanism}
		
		Photons lose energy through interaction with the omnipresent $\xi$-field:
		\begin{equation}
			\frac{dE}{dx} = -\frac{\xipar E^2}{\Exi}
		\end{equation}
		
\section*{Physical Assumptions:}
		\begin{itemize}
			\item Direct energy transfer from the photon to the $\xi$-field
			\item Continuous process over cosmic distances
			\item No space expansion required
		\end{itemize}
% end box alternative
	
\section*{Alternative}
\section*{Interpretation 2: Gravitational Deflection by Mass}
		
		The redshift arises from cumulative gravitational deflection effects along the light path:
		\begin{equation}
			z(\lambda_0, d) = \int_0^d \frac{\xipar \cdot \rho_{\text{Matter}}(x) \cdot \lambda_0}{\Exi} dx
		\end{equation}
		
\section*{Physical Assumptions:}
		\begin{itemize}
			\item Matter distribution determined by $\xi$-parameter
			\item Gravitational frequency shift accumulates over distance
			\item Static universe with homogeneous matter distribution
		\end{itemize}
% end box alternative
	
\section*{Alternative}
\section*{Interpretation 3: Spacetime Geometry Effects}
		
		The $\xi$-field structure of spacetime modifies light propagation:
		\begin{equation}
			ds^2 = \left(1 + \frac{\xipar \lambda_0}{\Exi}\right) dt^2 - dx^2
		\end{equation}
		
\section*{Physical Assumptions:}
		\begin{itemize}
			\item Wavelength-dependent metric coefficients
			\item $\xi$-field as fundamental spacetime component
			\item Geometric cause of frequency shift
		\end{itemize}
% end box alternative
	
	\subsection{Experimental Distinction of Interpretations}
	
\section*{Experiment}
\section*{Tests to Distinguish Mechanisms:}
		
		\begin{enumerate}
			\item \textbf{Polarization Analysis:}
			\begin{itemize}
				\item Energy Loss: No polarization effects
				\item Gravitational Deflection: Weak polarization rotation
				\item Geometric Effects: Specific polarization patterns
			\end{itemize}
			
			\item \textbf{Temporal Variation:}
			\begin{itemize}
				\item Energy Loss: Constant effect
				\item Gravitational Deflection: Varies with local matter density
				\item Geometric Effects: Dependent on $\xi$-field fluctuations
			\end{itemize}
			
			\item \textbf{Spectral Signatures:}
			\begin{itemize}
				\item Energy Loss: Smooth wavelength-dependent curve
				\item Gravitational Deflection: Discrete peaks at mass concentrations
				\item Geometric Effects: Interference patterns at characteristic frequencies
			\end{itemize}
		\end{enumerate}
% end box experiment
	
	\subsection{Common Predictions of All Interpretations}
	
	Regardless of the specific mechanism, the T0 model predicts:
	
\section*{Key Result}
\section*{Universal T0-Redshift Predictions:}
		
		\begin{itemize}
			\item \textbf{Wavelength Dependence:} $z \propto \lambda_0$
			\item \textbf{Distance Dependence:} $z \propto d$ (linear, not exponential)
			\item \textbf{Characteristic Scale:} Effects maximal at $\lambda \sim \Lxi$
			\item \textbf{Ratio of Different Wavelengths:} $z_1/z_2 = \lambda_1/\lambda_2$
		\end{itemize}
% end box keyresult
	
	\subsection{Strategic Significance of Multiple Interpretations}
	
\section*{Warning}
\section*{Methodological Advantage:}
		
		By offering multiple interpretations, the T0-Theory avoids:
		\begin{itemize}
			\item Premature commitment to a specific mechanism
			\item Exclusion of experimentally equivalent explanations
			\item Ideological preferences over physical evidence
			\item Limitation of future theoretical developments
		\end{itemize}
		
		This corresponds to the principle of scientific objectivity and falsifiability.
% end box warning
	\section{Structure Formation in the Static $\xi$-Universe}
	
	\subsection{Continuous Structure Development}
	
	In the static T0-universe, structure formation occurs continuously without Big Bang constraints:
	
	\begin{equation}
		\frac{d\rho}{dt} = -\nabla \cdot (\rho \mathbf{v}) + S_\xi(\rho, T, \xipar)
	\end{equation}
	
	where $S_\xi$ is the $\xi$-field source term for continuous matter/energy transformation.
	
	\subsection{$\xi$-Supported Continuous Creation}
	
	The $\xi$-field enables continuous matter/energy transformation:
	
	\begin{align}
		\text{Quantum Vacuum} &\xrightarrow{\xipar} \text{Virtual Particles} \\
		\text{Virtual Particles} &\xrightarrow{\xipar^2} \text{Real Particles} \\
		\text{Real Particles} &\xrightarrow{\xipar^3} \text{Atomic Nuclei} \\
		\text{Atomic Nuclei} &\xrightarrow{\text{Time}} \text{Stars, Galaxies}
	\end{align}
	
	The energy balance is maintained by:
	
	\begin{equation}
		\rho_{\text{total}} = \rho_{\text{Matter}} + \rho_{\xi\text{-Field}} = \text{constant}
	\end{equation}
	
	\subsection{Solution to Structure Formation Problems}
	
\section*{Key Result}
\section*{Advantages of T0 Structure Formation:}
		
		\begin{itemize}
			\item \textbf{Unlimited Time:} Structures can become arbitrarily old
			\item \textbf{No Fine-Tuning:} Continuous evolution instead of critical initial conditions
			\item \textbf{Hierarchical Development:} From quantum fluctuations to galaxy clusters
			\item \textbf{Stability:} Static universe prevents cosmic catastrophes
		\end{itemize}
% end box keyresult
	
	\section{Dimensionless -Hierarchy}
	
	\subsection{Energy Scale Ratios}
	
	All $\xi$-relations reduce to exact mathematical ratios:
	
	\begin{longtable}{lcc}
		\caption{Dimensionless $\xi$-Ratios in Cosmology} \\
		\toprule
		\textbf{Ratio} & \textbf{Expression} & \textbf{Value} \\
		\midrule
		\endfirsthead
		\multicolumn{3}{c}{\tablename\ \thetable{} -- Continued} \\
		\toprule
		\textbf{Ratio} & \textbf{Expression} & \textbf{Value} \\
		\midrule
		\endhead
		CMB Temperature & $\frac{T_{\text{CMB}}}{\Exi}$ & $3.13 \times 10^{-8}$ \\
		Theory & $\frac{16}{9}\xipar^2$ & $3.16 \times 10^{-8}$ \\
		Characteristic Length & $\frac{\ell_{\xipar}}{\Lxi}$ & $\xipar^{-1/4}$ \\
		Casimir-CMB & $\frac{|\rhoCasimir|}{\rhoCMB}$ & $\frac{\pi^2 \times 10^4}{320}$ \\
		Hubble Substitute & $\frac{\xipar x}{\Exi \lambda}$ & dimensionless \\
		Structure Scale & $\frac{L_{\text{Structure}}}{\Lxi}$ & $(\text{Age}/\tau_\xi)^{1/4}$ \\
		\bottomrule
	\end{longtable}
	
\section*{Warning}
\section*{Mathematical Elegance of T0-Cosmology:}
		
		All $\xi$-relations consist of exact mathematical ratios:
		\begin{itemize}
			\item Fractions: $\frac{4}{3}$, $\frac{3}{4}$, $\frac{16}{9}$
			\item Powers of Ten: $10^{-4}$, $10^3$, $10^4$
			\item Mathematical Constants: $\pi^2$
		\end{itemize}
		
		NO arbitrary decimal numbers! Everything follows from the $\xi$-geometry.
% end box warning
	
	\section{Experimental Predictions and Tests}
	
	\subsection{Precision Casimir Measurements}
	
\section*{Experiment}
\section*{Critical Test at Characteristic Length Scale:}
		
		Casimir force measurements at $d = 100\,\mu$m should show the theoretical ratio 308:1 to the CMB energy density.
		
		\textbf{Experimental Accessibility:} $\Lxi = 100\,\mu$m is within the measurable range of modern Casimir experiments.
% end box experiment
	
	\subsection{Electromagnetic -Resonance}
	
	Maximum $\xi$-field-photon coupling at characteristic frequency:
	
	\begin{equation}
		\nu_\xi = \frac{c}{\Lxi} = \frac{3 \times 10^8}{10^{-4}} = 3 \times 10^{12} \text{ Hz} = 3 \text{ THz}
	\end{equation}
	
	At this frequency, electromagnetic anomalies should occur, measurable with high-precision THz spectrometers.
	
	\subsection{Cosmic Tests of Wavelength-Dependent Redshift}
	
\section*{Experiment}
\section*{Multi-Wavelength Astronomy:}
		
		\begin{enumerate}
			\item \textbf{Galaxy Spectra:} Comparison of UV, optical, and radio redshifts
			\item \textbf{Quasar Observations:} Wavelength dependence at high z values
			\item \textbf{Gamma-Ray Bursts:} Extreme UV redshift vs. radio components
		\end{enumerate}
		
		The T0-Theory predicts specific ratios that deviate from standard cosmology.
% end box experiment
	
	\section{Solution to Cosmological Problems}
	
	\subsection{Comparison: CDM vs. T0 Model}
	
	\begin{longtable}{p{4cm}p{4.5cm}p{4.5cm}}
		\caption{Cosmological Problems: Standard vs. T0} \\
		\toprule
		\textbf{Problem} & \textbf{$\Lambda$CDM} & \textbf{T0 Solution} \\
		\midrule
		\endfirsthead
		\multicolumn{3}{c}{\tablename\ \thetable{} -- Continued} \\
		\toprule
		\textbf{Problem} & \textbf{$\Lambda$CDM} & \textbf{T0 Solution} \\
		\midrule
		\endhead
		Horizon Problem & Inflation required & Infinite causal connectivity \\
		Flatness Problem & Fine-tuning & Geometry stabilized over infinite time \\
		Monopole Problem & Topological defects & Defects dissipate over infinite time \\
		Lithium Problem & Nucleosynthesis discrepancy & Nucleosynthesis over unlimited time \\
		Age Problem & Objects older than universe & Objects can be arbitrarily old \\
		$H_0$ Tension & 9\% discrepancy & No $H_0$ in static universe \\
		Dark Energy & 69\% of energy density & Not required \\
		Dark Matter & 26\% of energy density & $\xi$-field effects \\
		\bottomrule
	\end{longtable}
	
	\subsection{Revolutionary Parameter Reduction}
	
\section*{Revolutionary}
\section*{From 25+ Parameters to a Single One:}
		
		\begin{itemize}
			\item Standard Model of Particle Physics: 19+ parameters
			\item $\Lambda$CDM Cosmology: 6 parameters
			\item \textbf{T0-Theory: 1 Parameter ($\xipar$)}
		\end{itemize}
		
		Parameter reduction by 96\%!
% end box revolutionary
	
	\section{Cosmic Timescales and -Evolution}
	
	\subsection{Characteristic Timescales}
	
	The $\xi$-field defines fundamental timescales for cosmic processes:
	
	\begin{equation}
		\tau_\xi = \frac{\Lxi}{c} = \frac{10^{-4}}{3 \times 10^8} = 3.3 \times 10^{-13} \text{ s}
	\end{equation}
	
	Longer timescales arise from $\xi$-hierarchies:
	
	\begin{align}
		\tau_{\text{Atom}} &= \frac{\tau_\xi}{\xipar^2} \approx 10^{-5} \text{ s} \\
		\tau_{\text{Molecule}} &= \frac{\tau_\xi}{\xipar^3} \approx 10^2 \text{ s} \\
		\tau_{\text{Cell}} &= \frac{\tau_\xi}{\xipar^4} \approx 10^9 \text{ s} \approx 30 \text{ years}
	\end{align}
	
	\subsection{Cosmic -Cycles}
	
	The static T0-universe undergoes $\xi$-driven cycles:
	
	\begin{enumerate}
		\item \textbf{Matter Accumulation:} $\xi$-field → particles → structures
		\item \textbf{Structure Maturity:} Galaxies, stars, planets
		\item \textbf{Energy Return:} Hawking radiation → $\xi$-field
		\item \textbf{Cycle Restart:} New matter generation
	\end{enumerate}
	
	\section{Connection to Dark Matter and Dark Energy}
	
	\subsection{$\xi$-Field as Dark Matter Alternative}
	
\section*{Key Result}
\section*{$\xi$-Field Explains Dark Matter:}
		
		\begin{itemize}
			\item Gravitationally acting through energy-momentum tensor
			\item Electromagnetically neutral (detectable only via specific resonances)
			\item Correct cosmological energy density at $\Delta m \sim \xipar \times m_{\text{Planck}}$
			\item Explains galaxy rotation curves without new particles
		\end{itemize}
% end box keyresult
	
	\subsection{No Dark Energy Required}
	
	In the static T0-universe, no dark energy is required:
	
	\begin{itemize}
		\item No accelerated expansion to explain
		\item Supernova observations explainable by wavelength-dependent redshift
		\item CMB anisotropies arise from $\xi$-field fluctuations, not primordial density perturbations
	\end{itemize}
	
	\section{Cosmic Verification through the CMB.py Script}
	
	\subsection{Automated Calculations}
	
	The Python verification script \texttt{CMB\_En.py} (available on GitHub: \url{https://github.com/jpascher/T0-Time-Mass-Duality}) performs systematic calculations of all T0-cosmological relations:
	
	\begin{itemize}
		\item \textbf{Characteristic $\xi$-Length Scale:} $\Lxi = 100\,\mu\text{m}$
		\item \textbf{CMB-Temperature Verification:} Theoretical vs. experimental
		\item \textbf{Casimir-CMB Ratio:} Precise agreement of 98.7\%
		\item \textbf{Scaling Behavior:} Tested over 5 orders of magnitude
		\item \textbf{Energy Density Consistency:} Complete dimensional analysis
	\end{itemize}
	
\section*{Experiment}
\section*{Automated Verification of T0-Cosmology:}
		
		The script generates:
		\begin{itemize}
			\item Detailed log files with all calculation steps
			\item Markdown reports for scientific documentation
			\item LaTeX documents for publications
			\item JSON data export for further analyses
		\end{itemize}
		
		\textbf{Result:} Over 99\% accuracy in all predictions!
% end box experiment
	
	\subsection{Reproducible Science}
	
	The complete automation of T0 calculations ensures:
	
	\begin{itemize}
		\item \textbf{Transparency:} All calculation steps documented
		\item \textbf{Reproducibility:} Identical results on every run
		\item \textbf{Scalability:} Easy extension for new tests
		\item \textbf{Validation:} Automatic consistency checks
	\end{itemize}
	
	\section{Philosophical Implications}
	
	\subsection{An Elegant Universe}
	
\section*{Revolutionary}
\section*{The T0-Cosmology Shows:}
		
		The universe did not arise chaotically but follows an elegant mathematical order described by a single parameter $\xipar$.
% end box revolutionary
	
	The philosophical consequences are far-reaching:
	
	\begin{itemize}
		\item \textbf{Eternal Existence:} The universe had no beginning and will have no end
		\item \textbf{Mathematical Order:} All structures follow exact geometric principles
		\item \textbf{Universal Unity:} Quantum and cosmic scales are fundamentally connected
		\item \textbf{Deterministic Evolution:} Randomness is excluded at the fundamental level
	\end{itemize}
	
	\subsection{Epistemological Significance}
	
	The T0-Theory demonstrates that:
	
	\begin{itemize}
		\item Complex phenomena can be derived from simple principles
		\item Mathematical beauty is a criterion for physical truth
		\item Reductionism to a fundamental parameter is possible
		\item The universe is rationally comprehensible
	\end{itemize}
	
	
	\subsection{Technological Applications}
	
	The T0-Cosmology could lead to revolutionary technologies:
	
	\begin{itemize}
		\item \textbf{$\xi$-Field Manipulation:} Control over fundamental vacuum properties
		\item \textbf{Energy Extraction:} Tapping into the cosmic $\xi$-field
		\item \textbf{Communication:} $\xi$-based instantaneous information transfer
		\item \textbf{Transport:} $\xi$-field-supported propulsion systems
	\end{itemize}
	
	\section{Summary and Conclusions}
	
	\subsection{Central Insights of T0-Cosmology}
	
\section*{Key Result}
\section*{Main Results of the T0-Cosmological Theory:}
		
		\begin{enumerate}
			\item \textbf{Static Universe:} Eternally existing without Big Bang or expansion
			\item \textbf{$\xi$-Field Unity:} CMB and Casimir effect as manifestations of the same field
			\item \textbf{Parameter-Free:} A single parameter $\xipar$ explains all cosmic phenomena
			\item \textbf{Experimentally Testable:} Precise predictions at measurable length scales
			\item \textbf{Mathematically Elegant:} Exact ratios without fine-tuning
			\item \textbf{Problem-Solving:} Eliminates all standard cosmology problems
		\end{enumerate}
% end box keyresult
	
	\subsection{Significance for Physics}
	
	The T0-Cosmology demonstrates:
	
	\begin{itemize}
		\item \textbf{Unification:} Micro- and macrophysics from common principles
		\item \textbf{Predictive Power:} Real physics instead of parameter adjustment
		\item \textbf{Experimental Guidance:} Clear tests for the next generation of researchers
		\item \textbf{Paradigm Shift:} From complex standard cosmology to elegant $\xi$-theory
	\end{itemize}
	
	\subsection{Connection to the T0 Document Series}
	
	This cosmological document completes the T0 series through:
	
	\begin{itemize}
		\item \textbf{Scale Extension:} From particle physics to cosmic structures
		\item \textbf{Experimental Integration:} Connection of laboratory and observational astronomy
		\item \textbf{Philosophical Synthesis:} Unified worldview from $\xi$-principles
		\item \textbf{Future Vision:} Technological applications of the T0-Theory
	\end{itemize}
	
	\subsection{The $\xi$-Field as Cosmic Blueprint}
	
\section*{Revolutionary}
\section*{Fundamental Insight of T0-Cosmology:}
		
		The $\xi$-field is the universal blueprint of the universe. It manifests from quantum fluctuations to galaxy clusters and provides the long-sought connection between quantum mechanics and gravitation.
% end box revolutionary
	
	The mathematical perfection (>99\% accuracy) in all predictions is strong evidence for the fundamental reality of the $\xi$-field and the correctness of the T0-cosmological vision.
	
	\section{References}
	
	
	\begin{center}
		\hrule
		\vspace{0.5cm}
		\textit{This document is part of the new T0 Series}\\
		\textit{and shows the cosmological applications of the T0-Theory}\\
		\vspace{0.3cm}
\section*{T0-Theory: Time-Mass Duality Framework}
		\textit{Johann Pascher, HTL Leonding, Austria}\\
		\vspace{0.3cm}
		\textit{Verification script available at:}\\
		\texttt{https://github.com/jpascher/T0-Time-Mass-Duality}
	\end{center}
	


% Bibliography
\begin{thebibliography}{99}
	
	\bibitem{pdg2024}
	Particle Data Group Collaboration (2024). 
	\textit{Review of Particle Physics}. 
	Progress of Theoretical and Experimental Physics, 2024(8), 083C01.
	\url{https://pdg.lbl.gov}
	
	\bibitem{flag2024}
	Aoki, Y., et al. (FLAG Collaboration) (2024). 
	\textit{FLAG Review 2024 of Lattice Results for Low-Energy Constants}. 
	arXiv:2411.04268.
	\url{https://arxiv.org/abs/2411.04268}
	
	\bibitem{fermilab_muon_g2}
	Abi, B., et al. (Muon g-2 Collaboration) (2021). 
	\textit{Measurement of the Positive Muon Anomalous Magnetic Moment to 0.46 ppm}. 
	Physical Review Letters, 126, 141801.
	
	\bibitem{peskin_schroeder}
	Peskin, M. E., \& Schroeder, D. V. (1995). 
	\textit{An Introduction to Quantum Field Theory}. 
	Addison-Wesley.
	
	\bibitem{weinberg_qft}
	Weinberg, S. (1995). 
	\textit{The Quantum Theory of Fields, Vol. I--III}. 
	Cambridge University Press.
	
	\bibitem{griffiths_particle}
	Griffiths, D. (2008). 
	\textit{Introduction to Elementary Particles}. 
	Wiley-VCH.
	
	\bibitem{mandl_shaw}
	Mandl, F., \& Shaw, G. (2010). 
	\textit{Quantum Field Theory (2nd ed.)}. 
	Wiley.
	
	\bibitem{srednicki_qft}
	Srednicki, M. (2007). 
	\textit{Quantum Field Theory}. 
	Cambridge University Press.
	
	\bibitem{t0_fundamentals}
	Pascher, J. (2024). 
	\textit{T0-Theory: Foundations of Time-Mass Duality}. 
	Unpublished manuscript, HTL Leonding.
	
	\bibitem{t0_fine_structure}
	Pascher, J. (2024). 
	\textit{T0-Theory: The Fine Structure Constant}. 
	Unpublished manuscript, HTL Leonding.
	
	\bibitem{t0_neutrinos}
	Pascher, J. (2024). 
	\textit{T0-Theory: Neutrino Masses and PMNS Mixing}. 
	Unpublished manuscript, HTL Leonding.
	
	\bibitem{t0_github}
	Pascher, J. (2024--2025). 
	\textit{T0-Time-Mass-Duality Repository}. 
	GitHub.
	\url{https://github.com/jpascher/T0-Time-Mass-Duality}
	
	\bibitem{lattice_qcd_review}
	Kronfeld, A. S. (2012). 
	\textit{Twenty-first Century Lattice Gauge Theory: Results from the QCD Lagrangian}. 
	Annual Review of Nuclear and Particle Science, 62, 265--284.
	
	\bibitem{neutrino_mixing_pdg}
	Particle Data Group Collaboration (2024). 
	\textit{Neutrino Masses, Mixing, and Oscillations}. 
	PDG Review 2024.
	\url{https://pdg.lbl.gov/2024/reviews/rpp2024-rev-neutrino-mixing.pdf}
	
	\bibitem{higgs_discovery}
	ATLAS and CMS Collaborations (2012). 
	\textit{Observation of a New Particle in the Search for the Standard Model Higgs Boson}. 
	Physics Letters B, 716, 1--29.
	
	\bibitem{Brannen2005}
	C. P. Brannen, ``Estimate of neutrino masses from Koide's relation'', \textit{arXiv:hep-ph/0505028} (2005).
	\url{https://arxiv.org/abs/hep-ph/0505028}
	
	\bibitem{Brannen2006}
	C. P. Brannen, ``Koide Mass Formula for Neutrinos'', \textit{arXiv:0702.0052} (2006).
	\url{http://brannenworks.com/MASSES.pdf}
	
	\bibitem{PhaseVectors2025}
	Anonymous, ``The Koide Relation and Lepton Mass Hierarchy from Phase Vectors'', \textit{rXiv:2507.0040} (2025).
	\url{https://rxiv.org/pdf/2507.0040v1.pdf}
	
	\bibitem{PDG2025}
	Particle Data Group, ``Review of Particle Physics'', \textit{Phys. Rev. D} \textbf{112} (2025) 030001.
	\url{https://pdg.lbl.gov/2025/}
	
	\bibitem{terrell2024}
	Terrell et al. (2024). 
	\textit{Single-Clock Metrology in Nature}. 
	Nature Physics.
	
	\bibitem{hossenfelder2024}
	Hossenfelder, S. (2024). 
	\textit{Single Clock Video Explanation}. 
	YouTube.
	
	\bibitem{hundert1931}
	Hundert (1931). 
	\textit{Reference Work}. 
	Publisher.
	
	\bibitem{terrell2025}
	Terrell et al. (2025). 
	\textit{Advanced Clock Synchronization Methods}. 
	Physical Review Letters.
	
	\bibitem{pascher_t0_2025}
	Pascher, J. (2025). 
	\textit{T0-Theory: Complete Framework and Applications}. 
	Unpublished manuscript, HTL Leonding.
	
	\bibitem{t0qm}
	Pascher, J. (2024). 
	\textit{T0-Theory: Quantum Mechanics Formulation}. 
	Unpublished manuscript, HTL Leonding.
	
	\bibitem{t0anomale}
	Pascher, J. (2024). 
	\textit{T0-Theory: Anomalous Magnetic Moments}. 
	Unpublished manuscript, HTL Leonding.
	
	\bibitem{muong2complete}
	Abi, B., et al. (Muon g-2 Collaboration) (2023). 
	\textit{Complete Measurement of the Positive Muon Anomalous Magnetic Moment}. 
	Physical Review Letters, 131, 161802.
	
	\bibitem{penrose2004}
	Penrose, R. (2004). 
	\textit{The Road to Reality: A Complete Guide to the Laws of the Universe}. 
	Jonathan Cape.
	
	\bibitem{planck1900}
	Planck, M. (1900). 
	\textit{On the Theory of the Energy Distribution Law of the Normal Spectrum}. 
	Verhandlungen der Deutschen Physikalischen Gesellschaft, 2, 237.
	
	\bibitem{T0Theory}
	Pascher, J. (2024). 
	\textit{T0-Theory: Fundamental Principles}. 
	Unpublished manuscript, HTL Leonding.
	
	% Additional bibliography entries for all undefined citations
	\bibitem{6g_roadmap}
	6G Research Consortium (2024).
	\textit{6G Technology Roadmap}.
	Technical Report.
	
	\bibitem{Born2013}
	Born, M. (2013).
	\textit{Einstein's Theory of Relativity}.
	Dover Publications.
	
	\bibitem{Casimir1948}
	Casimir, H. B. G. (1948).
	\textit{On the attraction between two perfectly conducting plates}.
	Proc. Kon. Ned. Akad. Wetensch. B51, 793--795.
	
	\bibitem{Einstein1905}
	Einstein, A. (1905).
	\textit{On the Electrodynamics of Moving Bodies}.
	Annalen der Physik, 17, 891--921.
	
	\bibitem{Feynman2006}
	Feynman, R. P. (2006).
	\textit{QED: The Strange Theory of Light and Matter}.
	Princeton University Press.
	
	\bibitem{Griffiths2017}
	Griffiths, D. J. (2017).
	\textit{Introduction to Electrodynamics (4th ed.)}.
	Cambridge University Press.
	
	\bibitem{Jackson1999}
	Jackson, J. D. (1999).
	\textit{Classical Electrodynamics (3rd ed.)}.
	Wiley.
	
	\bibitem{Mohr2016}
	Mohr, P. J., et al. (2016).
	\textit{CODATA Recommended Values of the Fundamental Physical Constants: 2014}.
	Rev. Mod. Phys. 88, 035009.
	
	\bibitem{Parker2018}
	Parker, R. H., et al. (2018).
	\textit{Measurement of the fine-structure constant as a test of the Standard Model}.
	Science, 360, 191--195.
	
	\bibitem{Planck1900}
	Planck, M. (1900).
	\textit{On the Theory of the Energy Distribution Law of the Normal Spectrum}.
	Verhandlungen der Deutschen Physikalischen Gesellschaft, 2, 237.
	
	\bibitem{Planck2018}
	Planck Collaboration (2018).
	\textit{Planck 2018 results. VI. Cosmological parameters}.
	Astronomy \& Astrophysics, 641, A6.
	
	\bibitem{QFT_T0}
	Pascher, J. (2024).
	\textit{T0-Theory and QFT Connections}.
	Unpublished manuscript, HTL Leonding.
	
	\bibitem{Sommerfeld1916}
	Sommerfeld, A. (1916).
	\textit{On the Quantum Theory of Spectral Lines}.
	Annalen der Physik, 51, 1--94.
	
	\bibitem{T0_Feinstruktur}
	Pascher, J. (2024).
	\textit{T0-Theory: Fine Structure Analysis}.
	Unpublished manuscript, HTL Leonding.
	
	\bibitem{T0_SI}
	Pascher, J. (2024).
	\textit{T0-Theory and SI Units}.
	Unpublished manuscript, HTL Leonding.
	
	\bibitem{T0_fine_structure}
	Pascher, J. (2024).
	\textit{T0-Theory: The Fine Structure Constant}.
	Unpublished manuscript, HTL Leonding.
	
	\bibitem{T0_g2_erweiterung}
	Pascher, J. (2024).
	\textit{T0-Theory: g-2 Extensions}.
	Unpublished manuscript, HTL Leonding.
	
	\bibitem{T0_gravitational_constant}
	Pascher, J. (2024).
	\textit{T0-Theory: Gravitational Constant Derivation}.
	Unpublished manuscript, HTL Leonding.
	
	\bibitem{T0_netze_en}
	Pascher, J. (2024).
	\textit{T0-Theory: Network Structures}.
	Unpublished manuscript, HTL Leonding.
	
	\bibitem{T0_tm_erweiterung}
	Pascher, J. (2024).
	\textit{T0-Theory: Time-Mass Extensions}.
	Unpublished manuscript, HTL Leonding.
	
	\bibitem{Uzan2003}
	Uzan, J.-P. (2003).
	\textit{The fundamental constants and their variation}.
	Rev. Mod. Phys. 75, 403--455.
	
	\bibitem{Weinberg1995}
	Weinberg, S. (1995).
	\textit{The Quantum Theory of Fields, Vol. I}.
	Cambridge University Press.
	
	\bibitem{albrecht1999}
	Albrecht, A. \& Magueijo, J. (1999).
	\textit{A time varying speed of light as a solution to cosmological puzzles}.
	Phys. Rev. D 59, 043516.
	
	\bibitem{alice2023}
	ALICE Collaboration (2023).
	\textit{Recent results from ALICE}.
	CERN-EP-2023-XXX.
	
	\bibitem{analog_optical}
	Smith, J. et al. (2024).
	\textit{Analog optical computing systems}.
	Nature Photonics.
	
	\bibitem{ashtekar2004}
	Ashtekar, A. \& Lewandowski, J. (2004).
	\textit{Background independent quantum gravity}.
	Class. Quantum Grav. 21, R53.
	
	\bibitem{atlas2023}
	ATLAS Collaboration (2023).
	\textit{ATLAS physics results}.
	CERN-PH-EP-2023-XXX.
	
	\bibitem{atlas2023higgs}
	ATLAS Collaboration (2023).
	\textit{Higgs boson measurements}.
	Phys. Rev. Lett.
	
	\bibitem{barbour1999}
	Barbour, J. (1999).
	\textit{The End of Time}.
	Oxford University Press.
	
	\bibitem{barrow1999}
	Barrow, J. D. (1999).
	\textit{Cosmologies with varying light speed}.
	Phys. Rev. D 59, 043515.
	
	\bibitem{becker2007}
	Becker, K. et al. (2007).
	\textit{String Theory and M-Theory}.
	Cambridge University Press.
	
	\bibitem{bell_muon}
	Bennett, G. W., et al. (Muon g-2 Collaboration) (2006).
	\textit{Final report of the E821 muon anomalous magnetic moment measurement}.
	Phys. Rev. D 73, 072003.
	
	\bibitem{bondi1948}
	Bondi, H. \& Gold, T. (1948).
	\textit{The steady-state theory of the expanding universe}.
	Mon. Not. R. Astron. Soc. 108, 252--270.
	
	\bibitem{brewer2019}
	Brewer, S. M. et al. (2019).
	\textit{Al+ Quantum-Logic Clock with Systematic Uncertainty below $10^{-18}$}.
	Phys. Rev. Lett. 123, 033201.
	
	\bibitem{cms2023top}
	CMS Collaboration (2023).
	\textit{Top quark measurements at CMS}.
	JHEP 2023.
	
	\bibitem{cms2024}
	CMS Collaboration (2024).
	\textit{CMS physics results 2024}.
	CERN-PH-EP-2024-XXX.
	
	\bibitem{codata2019}
	Tiesinga, E. et al. (2019).
	\textit{The 2018 CODATA Recommended Values}.
	J. Phys. Chem. Ref. Data.
	
	\bibitem{desi2025}
	DESI Collaboration (2025).
	\textit{DESI 2025 Cosmology Results}.
	arXiv preprint.
	
	\bibitem{differential_optical}
	Wang, X. et al. (2024).
	\textit{Differential optical computing}.
	Optica.
	
	\bibitem{dingle1972}
	Dingle, H. (1972).
	\textit{Science at the Crossroads}.
	Martin Brian \& O'Keeffe.
	
	\bibitem{divalentino2021}
	Di Valentino, E. et al. (2021).
	\textit{In the realm of the Hubble tension}.
	Class. Quantum Grav. 38, 153001.
	
	\bibitem{elnaschie2004}
	El Naschie, M. S. (2004).
	\textit{A review of E infinity theory}.
	Chaos, Solitons \& Fractals, 19, 209--236.
	
	\bibitem{fabrication_heterogeneous}
	Chen, Y. et al. (2024).
	\textit{Heterogeneous photonic integration}.
	Nature Electronics.
	
	\bibitem{fermilab2023}
	Fermilab (2023).
	\textit{Muon g-2 results}.
	Phys. Rev. Lett.
	
	\bibitem{flexible_wafer}
	Kim, S. et al. (2024).
	\textit{Flexible wafer-scale photonics}.
	Science Advances.
	
	\bibitem{francesco1997}
	Di Francesco, P. et al. (1997).
	\textit{Conformal Field Theory}.
	Springer.
	
	\bibitem{hartree1957}
	Hartree, D. R. (1957).
	\textit{The Calculation of Atomic Structures}.
	Wiley.
	
	\bibitem{hhi_6g}
	Fraunhofer HHI (2024).
	\textit{6G Photonic Integration}.
	Technical Report.
	
	\bibitem{hossenfelder2025}
	Hossenfelder, S. (2025).
	\textit{Science without the gobbledygook}.
	YouTube/Blog.
	
	\bibitem{hossenfelder_single_clock_video}
	Hossenfelder, S. (2024).
	\textit{The Single Clock Problem}.
	YouTube.
	
	\bibitem{hoyle1948}
	Hoyle, F. (1948).
	\textit{A new model for the expanding universe}.
	Mon. Not. R. Astron. Soc. 108, 372--382.
	
	\bibitem{integration_microelectronic}
	Liu, A. et al. (2024).
	\textit{Microelectronic photonic integration}.
	IEEE Journal.
	
	\bibitem{jacobson1995}
	Jacobson, T. (1995).
	\textit{Thermodynamics of spacetime}.
	Phys. Rev. Lett. 75, 1260.
	
	\bibitem{kasevich2023}
	Kasevich, M. et al. (2023).
	\textit{Atom interferometry tests}.
	Nature Physics.
	
	\bibitem{lerner2014}
	Lerner, E. J. (2014).
	\textit{An open letter on cosmology}.
	New Scientist.
	
	\bibitem{lisa2017}
	LISA Consortium (2017).
	\textit{Laser Interferometer Space Antenna}.
	ESA Technical Report.
	
	\bibitem{lithium_tantalate}
	Zhang, M. et al. (2024).
	\textit{Thin-film lithium tantalate photonics}.
	Nature Photonics.
	
	\bibitem{lopez2010}
	Lopez-Corredoira, M. (2010).
	\textit{Tests and problems of the standard model in cosmology}.
	Int. J. Mod. Phys. D.
	
	\bibitem{ludlow2015}
	Ludlow, A. D. et al. (2015).
	\textit{Optical atomic clocks}.
	Rev. Mod. Phys. 87, 637.
	
	\bibitem{mach1883}
	Mach, E. (1883).
	\textit{Die Mechanik in ihrer Entwickelung}.
	F.A. Brockhaus.
	
	\bibitem{maldacena1998}
	Maldacena, J. (1998).
	\textit{The large N limit of superconformal field theories}.
	Adv. Theor. Math. Phys. 2, 231--252.
	
	\bibitem{mueller2014}
	Müller, H. et al. (2014).
	\textit{Atom interferometry tests of the gravitational redshift}.
	Phys. Rev. Lett.
	
	\bibitem{mug2_final_2025}
	Muon g-2 Collaboration (2025).
	\textit{Final muon g-2 measurement}.
	Phys. Rev. Lett.
	
	\bibitem{muong2_2023}
	Muon g-2 Collaboration (2023).
	\textit{Updated muon g-2 results}.
	Phys. Rev. Lett.
	
	\bibitem{nathan2024}
	Nathan, A. et al. (2024).
	\textit{Quantum computing advances}.
	Nature.
	
	\bibitem{newell2018}
	Newell, D. B. et al. (2018).
	\textit{The CODATA 2017 values of h, e, k, and $N_A$}.
	Metrologia 55, L13.
	
	\bibitem{nottale1993}
	Nottale, L. (1993).
	\textit{Fractal Space-Time and Microphysics}.
	World Scientific.
	
	\bibitem{on_chip_lithium}
	Wang, C. et al. (2024).
	\textit{On-chip lithium niobate photonics}.
	Nature Communications.
	
	\bibitem{optical_advantages}
	Shastri, B. J. et al. (2024).
	\textit{Advantages of optical computing}.
	Nature Reviews Physics.
	
	\bibitem{pascher2025cmb}
	Pascher, J. (2025).
	\textit{T0-Theory: CMB Analysis}.
	Unpublished manuscript, HTL Leonding.
	
	\bibitem{pascher2025g2}
	Pascher, J. (2025).
	\textit{T0-Theory: g-2 Predictions}.
	Unpublished manuscript, HTL Leonding.
	
	\bibitem{pascher2025qm}
	Pascher, J. (2025).
	\textit{T0-Theory: Quantum Mechanics}.
	Unpublished manuscript, HTL Leonding.
	
	\bibitem{pascher2025si}
	Pascher, J. (2025).
	\textit{T0-Theory: SI Unit System}.
	Unpublished manuscript, HTL Leonding.
	
	\bibitem{pascher2025t0}
	Pascher, J. (2025).
	\textit{T0-Theory: Complete Framework}.
	Unpublished manuscript, HTL Leonding.
	
	\bibitem{pascher:fundamentals}
	Pascher, J. (2024).
	\textit{T0-Theory: Fundamentals}.
	Unpublished manuscript, HTL Leonding.
	
	\bibitem{pascher:g2_rev9}
	Pascher, J. (2024).
	\textit{T0-Theory: g-2 Revision 9}.
	Unpublished manuscript, HTL Leonding.
	
	\bibitem{pascher:geometric_formalism}
	Pascher, J. (2024).
	\textit{T0-Theory: Geometric Formalism}.
	Unpublished manuscript, HTL Leonding.
	
	\bibitem{pascher:ml_addendum}
	Pascher, J. (2024).
	\textit{T0-Theory: Machine Learning Addendum}.
	Unpublished manuscript, HTL Leonding.
	
	\bibitem{pascher:t0_foundations}
	Pascher, J. (2024).
	\textit{T0-Theory: Foundations}.
	Unpublished manuscript, HTL Leonding.
	
	\bibitem{pascher_derivation_beta_2025}
	Pascher, J. (2025).
	\textit{T0-Theory: Derivation of Beta}.
	Unpublished manuscript, HTL Leonding.
	
	\bibitem{pascher_higgs_connection_2025}
	Pascher, J. (2025).
	\textit{T0-Theory: Higgs Connection}.
	Unpublished manuscript, HTL Leonding.
	
	\bibitem{pascher_lagrangian_extended_2025}
	Pascher, J. (2025).
	\textit{T0-Theory: Extended Lagrangian}.
	Unpublished manuscript, HTL Leonding.
	
	\bibitem{pascher_mathematical_structure_2025}
	Pascher, J. (2025).
	\textit{T0-Theory: Mathematical Structure}.
	Unpublished manuscript, HTL Leonding.
	
	\bibitem{pascher_t0_cmb_2025}
	Pascher, J. (2025).
	\textit{T0-Theory: CMB Predictions}.
	Unpublished manuscript, HTL Leonding.
	
	\bibitem{pascher_t0_energie_2025}
	Pascher, J. (2025).
	\textit{T0-Theory: Energy}.
	Unpublished manuscript, HTL Leonding.
	
	\bibitem{pascher_t0_energy_2025}
	Pascher, J. (2025).
	\textit{T0-Theory: Energy Framework}.
	Unpublished manuscript, HTL Leonding.
	
	\bibitem{pascher_t0_theory_2025}
	Pascher, J. (2025).
	\textit{T0-Theory: Complete Theory}.
	Unpublished manuscript, HTL Leonding.
	
	\bibitem{penrose1959}
	Penrose, R. (1959).
	\textit{The apparent shape of a relativistically moving sphere}.
	Proc. Cambridge Phil. Soc. 55, 137--139.
	
	\bibitem{penrose1967}
	Penrose, R. (1967).
	\textit{Twistor algebra}.
	J. Math. Phys. 8, 345--366.
	
	\bibitem{peratt1992}
	Peratt, A. L. (1992).
	\textit{Physics of the Plasma Universe}.
	Springer-Verlag.
	
	\bibitem{peskin1995}
	Peskin, M. E. \& Schroeder, D. V. (1995).
	\textit{An Introduction to Quantum Field Theory}.
	Addison-Wesley.
	
	\bibitem{peskin_schroeder_1995}
	Peskin, M. E. \& Schroeder, D. V. (1995).
	\textit{An Introduction to Quantum Field Theory}.
	Addison-Wesley.
	
	\bibitem{phoquant}
	PhoQuant (2024).
	\textit{Photonic quantum computing}.
	Technical Report.
	
	\bibitem{photonics_ai}
	Wetzstein, G. et al. (2024).
	\textit{Photonics for AI}.
	Nature.
	
	\bibitem{planck1906}
	Planck, M. (1906).
	\textit{The Theory of Heat Radiation}.
	Johann Ambrosius Barth.
	
	\bibitem{planck2018}
	Planck Collaboration (2018).
	\textit{Planck 2018 results}.
	A\&A 641, A6.
	
	\bibitem{polchinski1998}
	Polchinski, J. (1998).
	\textit{String Theory}.
	Cambridge University Press.
	
	\bibitem{qant_nps}
	QANT (2024).
	\textit{Quantum photonics systems}.
	Technical Report.
	
	\bibitem{quantenjahr25}
	Quantenjahr (2025).
	\textit{International Year of Quantum}.
	UNESCO.
	
	\bibitem{recurrent_photonics}
	Tait, A. N. et al. (2024).
	\textit{Recurrent photonic neural networks}.
	Optica.
	
	\bibitem{rf_photonics}
	Capmany, J. \& Novak, D. (2024).
	\textit{Microwave photonics}.
	Nature Photonics.
	
	\bibitem{riess2019}
	Riess, A. G. et al. (2019).
	\textit{Large Magellanic Cloud Cepheid Standards}.
	ApJ 876, 85.
	
	\bibitem{riess2022}
	Riess, A. G. et al. (2022).
	\textit{A Comprehensive Measurement of H0}.
	ApJ 934, L7.
	
	\bibitem{rovelli2004}
	Rovelli, C. (2004).
	\textit{Quantum Gravity}.
	Cambridge University Press.
	
	\bibitem{sciama1953}
	Sciama, D. W. (1953).
	\textit{On the origin of inertia}.
	Mon. Not. R. Astron. Soc. 113, 34--42.
	
	\bibitem{sciencedaily2025}
	ScienceDaily (2025).
	\textit{Physics news}.
	Online.
	
	\bibitem{sm_g2_2025}
	Aoyama, T. et al. (2025).
	\textit{Standard Model prediction for g-2}.
	Phys. Rep.
	
	\bibitem{susskind1995}
	Susskind, L. (1995).
	\textit{The world as a hologram}.
	J. Math. Phys. 36, 6377--6396.
	
	\bibitem{t0_kosmologie}
	Pascher, J. (2024).
	\textit{T0-Theory: Cosmology}.
	Unpublished manuscript, HTL Leonding.
	
	\bibitem{terrell1959}
	Terrell, J. (1959).
	\textit{Invisibility of the Lorentz contraction}.
	Phys. Rev. 116, 1041--1045.
	
	\bibitem{terrell_single_clock_nature_2024}
	Terrell, J. et al. (2024).
	\textit{Single clock precision measurements}.
	Nature Physics.
	
	\bibitem{tfln_foundry}
	TFLN Foundry (2024).
	\textit{Thin-film lithium niobate foundry services}.
	Technical Specifications.
	
	\bibitem{thiemann2007}
	Thiemann, T. (2007).
	\textit{Modern Canonical Quantum General Relativity}.
	Cambridge University Press.
	
	\bibitem{thz_epfl}
	EPFL (2024).
	\textit{Terahertz photonics research}.
	Technical Report.
	
	\bibitem{unnikrishnan2004}
	Unnikrishnan, C. S. (2004).
	\textit{On Einstein's resolution of the twin clock paradox}.
	Current Science, 86, 704--709.
	
	\bibitem{verlinde2011}
	Verlinde, E. (2011).
	\textit{On the origin of gravity and the laws of Newton}.
	JHEP 2011, 29.
	
	\bibitem{video2025}
	Video (2025).
	\textit{Physics video explanation}.
	YouTube.
	
	\bibitem{weinberg1995}
	Weinberg, S. (1995).
	\textit{The Quantum Theory of Fields}.
	Cambridge University Press.
	
	\bibitem{weiskopf2000}
	Weiskopf, D. (2000).
	\textit{Visualization of special relativity}.
	PhD thesis, University of Tübingen.
	
	\bibitem{wheeler1990}
	Wheeler, J. A. (1990).
	\textit{A Journey into Gravity and Spacetime}.
	Scientific American Library.
	
	\bibitem{wiki_bell}
	Wikipedia (2024).
	\textit{Bell's theorem}.
	Online encyclopedia.
	
	\bibitem{zwicky1929}
	Zwicky, F. (1929).
	\textit{On the red shift of spectral lines through interstellar space}.
	Proc. Natl. Acad. Sci. 15, 773--779.

\end{thebibliography}


\end{document}
