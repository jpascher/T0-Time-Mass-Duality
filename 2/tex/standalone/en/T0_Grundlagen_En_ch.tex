\documentclass[11pt,a4paper]{article}
\usepackage[a4paper,margin=2cm]{geometry}
\usepackage[utf8]{inputenc}
\usepackage[english]{babel}
\usepackage{lmodern}
\renewcommand{\familydefault}{\sfdefault}

\usepackage{amsmath,amssymb,amsthm}
\usepackage{graphicx}
\usepackage[unicode,pdfencoding=auto,hypertexnames=false]{hyperref}
\usepackage{booktabs}
\usepackage{longtable}
\usepackage{array}
\usepackage{siunitx}
\usepackage{fancyhdr}
\usepackage{float}
\usepackage{tikz}
% tcolorbox removed for standalone
% tcbset removed
\tikzset{
  t0blue/.style={draw=blue,fill=blue!10},
  t0red/.style={draw=red,fill=red!10},
  t0green/.style={draw=green!50!black,fill=green!10},
  t0orange/.style={draw=orange,fill=orange!10},
}
\usepackage{setspace}
\usepackage{enumitem}
\usepackage{adjustbox}
\usepackage{xcolor}

% Define colors for xcolor package
\definecolor{t0green}{RGB}{34,139,34}
\definecolor{t0blue}{RGB}{0,0,255}
\definecolor{t0red}{RGB}{255,0,0}
\definecolor{t0orange}{RGB}{255,165,0}

% Define custom column types for tables
\newcolumntype{L}[1]{>{\raggedright\arraybackslash}p{#1}}
\newcolumntype{C}[1]{>{\centering\arraybackslash}p{#1}}
\newcolumntype{R}[1]{>{\raggedleft\arraybackslash}p{#1}}

\setlength{\parindent}{0pt}
\setlength{\parskip}{6pt}

\hypersetup{
  colorlinks=true,
  linkcolor=blue,
  citecolor=blue,
  urlcolor=blue
}
\pagestyle{fancy}
\setlength{\headheight}{28pt}

\newcommand{\checkmarkx}{\checkmark}
\newcommand{\warningx}{\textbf{!}}

% Makros aus Einzel-Dokumenten (Fallback-Definitionen)
\newcommand{\mytimes}{\times}
\newcommand{\myapprox}{\approx}
\newcommand{\mysim}{\sim}
\newcommand{\myomega}{\omega}
\newcommand{\mypi}{\pi}
\newcommand{\myrightarrow}{\rightarrow}
\newcommand{\mypropto}{\propto}
\newcommand{\deltafield}{\delta\phi}
\newcommand{\xipar}{\xi}
\newcommand{\xiT}{\xi}
\newcommand{\lambdah}{\lambda_h}

% Additional macros used in chapter files
\newcommand{\Kfrak}{K_{\text{frak}}}  % Fractal correction factor
\newcommand{\Dfrak}{D_f}              % Fractal dimension
\newcommand{\betapar}{\beta}          % T0 beta parameter
\newcommand{\alphapar}{\alpha}        % T0 alpha parameter
\newcommand{\Efield}{E}               % Energy field
% Note: checkmarkxa/warningxa are variants used in auto-generated chapter files
\newcommand{\checkmarkxa}{\checkmark}
\newcommand{\warningxa}{\textbf{!}}

% Additional T0-specific macros
\newcommand{\xigeom}{\xi_{\text{geom}}}  % Geometric xi
\newcommand{\lP}{\ell_P}                  % Planck length
\newcommand{\rzero}{r_0}                  % Characteristic radius
\newcommand{\xirat}{\xi_{\text{rat}}}     % Xi ratio
\newcommand{\tzero}{t_0}                  % Characteristic time
\newcommand{\natunits}{\text{(nat. units)}}  % Natural units annotation
\newcommand{\myRightarrow}{\Rightarrow}   % Arrow variant
\newcommand{\Lag}{\mathcal{L}}            % Lagrangian

% Physics macros used in chapter files
\newcommand{\CQCD}{C_{\text{QCD}}}        % QCD correction
\newcommand{\EP}{E_P}                     % Planck energy
\newcommand{\Ee}{E_e}                     % Electron energy
\newcommand{\Emu}{E_\mu}                  % Muon energy
\newcommand{\Exi}{E_\xi}                  % Xi energy
\newcommand{\Ezero}{E_0}                  % Characteristic energy
\newcommand{\Hubble}{H}                   % Hubble constant
\newcommand{\Kspec}{K_{\text{spec}}}      % Spectral correction
\newcommand{\Lambdat}{\Lambda_t}          % Time-related cosmological constant
\newcommand{\Leff}{\mathcal{L}_{\text{eff}}}  % Effective Lagrangian
\newcommand{\Lorentz}{\mathcal{L}}        % Lorentz symbol
\newcommand{\Lxi}{L_\xi}                  % Xi length
\newcommand{\Tfield}{T}                   % Time field
\newcommand{\Weyl}{W}                     % Weyl tensor/symbol
\newcommand{\alphaEMSI}{\alpha_{\text{EM,SI}}}  % EM alpha in SI
\newcommand{\alphaEMnat}{\alpha_{\text{EM,nat}}}  % EM alpha in natural units
\newcommand{\alphaem}{\alpha_{\text{em}}} % Electromagnetic alpha
\newcommand{\betaTSI}{\beta_{T,\text{SI}}}  % Beta in SI
\newcommand{\betaTnat}{\beta_{T,\text{nat}}}  % Beta in natural units
\newcommand{\deltam}{\delta m}            % Mass difference
\newcommand{\phiT}{\phi_T}                % T-field phi
\newcommand{\tP}{t_P}                     % Planck time
\newcommand{\rhoCMB}{\rho_{\text{CMB}}}   % CMB density
\newcommand{\rhoCasimir}{\rho_{\text{Casimir}}}  % Casimir density

% Table formatting
\usepackage{multirow}

% Additional physics macros
\newcommand{\Riem}{\mathcal{R}}           % Riemann tensor
\newcommand{\ZPinch}{Z_{\text{pinch}}}    % Z-pinch
\newcommand{\SynchPower}{P_{\text{synch}}} % Synchrotron power
\newcommand{\Rzero}{R_0}                  % Characteristic radius
\newcommand{\alphafine}{\alpha}           % Fine structure constant
\newcommand{\Etau}{E_\tau}                % Tau energy
\newcommand{\deltaE}{\delta E}            % Energy deviation
\newcommand{\EPlanck}{E_P}                % Planck energy
\newcommand{\pichar}{\pi}                 % Pi character
\newcommand{\alphaWSI}{\alpha_{W,\text{SI}}}  % Wien alpha in SI
\newcommand{\alphaWnat}{\alpha_{W,\text{nat}}}  % Wien alpha in natural units

% Einfache abstract-Umgebung für Kapitel:
\newenvironment{abstract}{%
  \begin{center}\bfseries Abstract\end{center}\small
}{\par}


\title{T0 Grundlagen En}
\author{J. Pascher}
\date{\today}

\begin{document}
\maketitle

\section*{T0 Grundlagen (T0 Grundlagen)}

	\begin{abstract}
		This document introduces the fundamental principles of the T0-Theory, a geometric reformulation of physics based on a single universal parameter $\xipar = \frac{4}{3} \times 10^{-4}$. The theory demonstrates how all fundamental constants and particle masses can be derived from the three-dimensional space geometry. Various interpretive approaches---harmonic, geometric, and field-theoretic---are presented on an equal footing. The fractal structure of quantum spacetime is systematically accounted for by the correction factor $\Kfrak = 0.986$.
	\end{abstract}
	
	\tableofcontents
	\newpage
	
	\section{Introduction to the T0-Theory}
	\subsection{Time-Mass Duality}
	
	In natural units ($\hbar = c = 1$), the fundamental relation holds:
	\begin{equation}
		T \cdot m = 1
		\label{T0_Grundlagen:L-T0_Grundlagen-0001}
	\end{equation}
	Time and mass are dual to each other: Heavy particles have short characteristic time scales, light particles long ones.
	
	This duality is not merely a mathematical relation but reflects a fundamental property of spacetime. It explains why heavy particles couple more strongly to the temporal structure of spacetime.
	
	\subsection{The Central Hypothesis}
	
	The T0-Theory is based on the revolutionary hypothesis that all physical phenomena can be derived from the geometric structure of three-dimensional space. At its center is a single universal parameter:
	
\section*{Foundation}
\section*{The Fundamental Geometric Parameter:}
		\begin{equation}
			\boxed{\xipar = \frac{4}{3} \times 10^{-4} = 1.333333\dots \times 10^{-4}}
			\label{T0_Grundlagen:L-T0_Grundlagen-0002}
		\end{equation}
		This parameter is dimensionless and contains all the information about the physical structure of the universe.
% end box foundation
	
	\subsection{Paradigm Shift Compared to the Standard Model}
	
	\begin{table}[htbp]
		\centering
		\begin{tabular}{lcc}
			\toprule
			\textbf{Aspect} & \textbf{Standard Model} & \textbf{T0-Theory} \\
			\midrule
			Free Parameters & $> 20$ & $1$ \\
			Theoretical Basis & Empirical Adjustment & Geometric Derivation \\
			Particle Masses & Arbitrary & Computable from Quantum Numbers \\
			Constants & Experimentally Determined & Geometrically Derived \\
			Unification & Separate Theories & Unified Framework \\
			\bottomrule
		\end{tabular}
		\caption{Comparison between Standard Model and T0-Theory}
	\end{table}
	
	\section{The Geometric Parameter}
	
	\subsection{Mathematical Structure}
	
	The parameter $\xipar$ consists of two fundamental components:
	
	\begin{equation}
		\xipar = \underbrace{\frac{4}{3}}_{\text{Harmonic-geometric}} \times \underbrace{10^{-4}}_{\text{Scale Hierarchy}}
		\label{T0_Grundlagen:L-T0_Grundlagen-0003}
	\end{equation}
	
	\subsection{The Harmonic-Geometric Component: 4/3}
	
\section*{Alternative}
\section*{Harmonic Interpretation:}
		
		The factor $\frac{4}{3}$ corresponds to the \textbf{perfect fourth}, one of the fundamental harmonic intervals:
		\begin{itemize}
			\item \textbf{Octave:} 2:1 (always universal)
			\item \textbf{Fifth:} 3:2 (always universal)  
			\item \textbf{Fourth:} 4:3 (always universal!)
		\end{itemize}
		
		These ratios are \textbf{geometric/mathematical}, not material-dependent. Space itself has a harmonic structure, and 4/3 (the fourth) is its fundamental signature.
% end box alternative
	
\section*{Alternative}
\section*{Geometric Interpretation:}
		
		The factor $\frac{4}{3}$ arises from the tetrahedral packing structure of three-dimensional space:
		\begin{itemize}
			\item \textbf{Tetrahedron Volume:} $V = \frac{\sqrt{2}}{12}a^3$
			\item \textbf{Sphere Volume:} $V = \frac{4\pi}{3}r^3$ 
			\item \textbf{Packing Density:} $\eta = \frac{\pi}{3\sqrt{2}} \approx 0.74$
			\item \textbf{Geometric Ratio:} $\frac{4}{3}$ from optimal space division
		\end{itemize}
% end box alternative
	
	\subsection{The Scale Hierarchy:}
	
\section*{Foundation}
		\textbf{Quantum Field Theoretic Derivation of $10^{-4}$:}
		
		The factor $10^{-4}$ arises from the combination of:
		
\section*{1. Loop Suppression (Quantum Field Theory):}
		\begin{equation}
			\frac{1}{16\pi^3} = 2.01 \times 10^{-3}
		\end{equation}
		
\section*{2. T0-Higgs Parameter:}
		\begin{equation}
			(\lambda_h^{(T0)})^2 \frac{(v^{(T0)})^2}{(m_h^{(T0)})^2} = 0.0647
		\end{equation}
		
\section*{3. Complete Calculation:}
		\begin{equation}
			2.01 \times 10^{-3} \times 0.0647 = 1.30 \times 10^{-4}
		\end{equation}
		
		Thus: \textbf{QFT Loop Suppression} ($\sim 10^{-3}$) $\times$ \textbf{T0 Higgs Sector} ($\sim 10^{-1}$) = $10^{-4}$
% end box foundation
	
	\section{Fractal Spacetime Structure}
	
	\subsection{Quantum Spacetime Effects}
	
	The T0-Theory recognizes that spacetime exhibits a fractal structure on Planck scales due to quantum fluctuations:
	
\section*{Key Result}
\section*{Fractal Spacetime Parameters:}
		\begin{align}
			\Dfrak &= 2.94 \quad \text{(effective fractal dimension)} \\
			\Kfrak &= 1 - \frac{\Dfrak - 2}{68} = 1 - \frac{0.94}{68} = 0.986
		\end{align}
		
\section*{Physical Interpretation:}
		\begin{itemize}
			\item $\Dfrak < 3$: Spacetime is ``porous'' on smallest scales
			\item $\Kfrak = 0.986 < 1$: Reduced effective interaction strength
			\item The constant 68 arises from the tetrahedral symmetry of 3D space
			\item Quantum fluctuations and vacuum structure effects
		\end{itemize}
% end box keyresult
	
	\subsection{Origin of the Constant 68}
	
\section*{Alternative}
\section*{Tetrahedron Geometry:}
		
		All tetrahedron combinations yield 72:
		\begin{align}
			6 \times 12 &= 72 \quad \text{(edges $\times$ rotations)} \\
			4 \times 18 &= 72 \quad \text{(faces $\times$ 18)} \\
			24 \times 3 &= 72 \quad \text{(symmetries $\times$ dimensions)}
		\end{align}
		
		The value 68 = 72 - 4 accounts for the 4 vertices of the tetrahedron as exceptions.
% end box alternative
	
	\section{Characteristic Energy Scales}
	
	\subsection{The T0 Energy Hierarchy}
	
	From the parameter $\xipar$, natural energy scales emerge:
	
	\begin{align}
		(E_0)_{\xipar} &= \frac{1}{\xipar} = 7500 \quad \text{(in natural units)} \\
		(E_0)_{\text{EM}} &= 7.398\,\si{\mega\electronvolt} \quad \text{(characteristic EM energy)} \\
		(E_0)_{\text{char}} &= 28.4 \quad \text{(characteristic T0 energy)}
	\end{align}
	
	\subsection{The Characteristic Electromagnetic Energy}
	
\section*{Key Result}
\section*{Gravitational-Geometric Derivation of $E_0$:}
		
		The characteristic energy follows from the coupling relation:
		\begin{equation}
			E_0^2 = \frac{4\sqrt{2} \cdot m_\mu}{\xipar^4}
		\end{equation}
		
		This yields $E_0 = 7.398$ MeV as the fundamental electromagnetic energy scale.
% end box keyresult
	
\section*{Alternative}
\section*{Geometric Mean of Lepton Masses:}
		
		Alternatively, $E_0$ can be defined as the geometric mean:
		\begin{equation}
			E_0 = \sqrt{m_e \cdot m_\mu} = 7.35\,\si{\mega\electronvolt}
		\end{equation}
		
		The difference from 7.398 MeV ($< 1\%$) is explainable by quantum corrections.
% end box alternative
	
	\section{Dimensional Analytic Foundations}
	
	\subsection{Natural Units}
	
	The T0-Theory works in natural units, where:
	
	\begin{align}
		\hbar = c = 1 \quad \text{(convention)}
	\end{align}
	
	In this system, all quantities have energy dimension or are dimensionless:
	
	\begin{align}
		[M] &= [E] \quad \text{(from $E = mc^2$ with $c = 1$)} \\
		[L] &= [E^{-1}] \quad \text{(from $\lambda = \hbar/p$ with $\hbar = 1$)} \\
		[T] &= [E^{-1}] \quad \text{(from $\omega = E/\hbar$ with $\hbar = 1$)}
	\end{align}
	
	\subsection{Conversion Factors}
	
\section*{Warning}
\section*{Critical Importance of Conversion Factors:}
		
		For experimental comparison, conversion factors from natural to SI units are essential:
		\begin{itemize}
			\item These are \textbf{not} arbitrary but follow from fundamental constants
			\item They encode the connection between geometric theory and measurable quantities
			\item Example: $C_{\text{conv}} = 7.783 \times 10^{-3}$ for the gravitational constant $G$ in \si{\cubic\meter\per\cubic\kilo\gram\per\square\second}
		\end{itemize}
% end box warning
	
	\section{The Universal T0 Formula Structure}
	
	\subsection{Basic Pattern of T0 Relations}
	
	All T0 formulas follow the universal pattern:
	
	\begin{equation}
		\boxed{\text{Physical Quantity} = f(\xipar, \text{Quantum Numbers}) \times \text{Conversion Factor}}
		\label{T0_Grundlagen:L-T0_Grundlagen-0004}
	\end{equation}
	
	where:
	\begin{itemize}
		\item $f(\xipar, \text{Quantum Numbers})$ encodes the geometric relation
		\item Quantum numbers $(n,l,j)$ determine the specific configuration
		\item Conversion factors establish the connection to SI units
	\end{itemize}
	
	\subsection{Examples of the Universal Structure}
	
	\begin{align}
		\text{Gravitational Constant:} \quad G &= \frac{\xipar^2}{4m_e} \times C_{\text{conv}} \times \Kfrak \\
		\text{Particle Masses:} \quad m_i &= \frac{\Kfrak}{\xipar \cdot f(n_i,l_i,j_i)} \times C_{\text{conv}} \\
		\text{Fine Structure Constant:} \quad \alpha &= \xipar \times \left(\frac{E_0}{1\,\si{\mega\electronvolt}}\right)^2
	\end{align}
	
	\section{Various Levels of Interpretation}
	
	\subsection{Hierarchy of Levels of Understanding}
	
\section*{Foundation}
\section*{The T0-Theory can be understood on various levels:}
		
\section*{1. Phenomenological Level:}
		\begin{itemize}
			\item Empirical Observation: One constant explains everything
			\item Practical Application: Prediction of new values
		\end{itemize}
		
\section*{2. Geometric Level:}
		\begin{itemize}
			\item Space structure determines physical properties
			\item Tetrahedral packing as basic principle
		\end{itemize}
		
\section*{3. Harmonic Level:}
		\begin{itemize}
			\item Spacetime as a harmonic system
			\item Particles as ``tones'' in cosmic harmony
		\end{itemize}
		
\section*{4. Quantum Field Theoretic Level:}
		\begin{itemize}
			\item Loop suppressions and Higgs mechanism
			\item Fractal corrections as quantum effects
		\end{itemize}
% end box foundation
	
	\subsection{Complementary Perspectives}
	
\section*{Alternative}
\section*{Reductionist vs. Holistic Perspective:}
		
\section*{Reductionist:}
		\begin{itemize}
			\item $\xipar$ as an empirical parameter that ``accidentally'' works
			\item Geometric interpretations as added post hoc
		\end{itemize}
		
\section*{Holistic:}
		\begin{itemize}
			\item Space-Time-Matter as inseparable unity
			\item $\xipar$ as expression of a deeper cosmic order
		\end{itemize}
% end box alternative
	
	
	\section{Basic Calculation Methods}
	
	\subsection{Direct Geometric Method}
	
	The simplest application of the T0-Theory uses direct geometric relations:
	\begin{equation}
		\text{Physical Quantity} = \text{Geometric Factor} \times \xi^n \times \text{Normalization}
		\label{T0_Grundlagen:L-T0_Grundlagen-0005}
	\end{equation}
	
	where the exponent $n$ follows from dimensional analysis and the geometric factor contains rational numbers like $\frac{4}{3}$, $\frac{16}{5}$, etc.
	
	\subsection{Extended Yukawa Method}
	
	For particle masses, the Higgs mechanism is additionally considered:
	\begin{equation}
		m_i = y_i \cdot v
		\label{T0_Grundlagen:L-T0_Grundlagen-0006}
	\end{equation}
	
	where the Yukawa couplings $y_i$ are geometrically calculated from the T0 structure:
	\begin{equation}
		y_i = r_i \times \xi^{p_i}
		\label{T0_Grundlagen:L-T0_Grundlagen-0007}
	\end{equation}
	
	The parameters $r_i$ and $p_i$ are exact rational numbers that follow from the quantum number assignment of the T0 geometry.
	
	\section{Philosophical Implications}
	
	\subsection{The Problem of Naturalness}
	
\section*{Foundation}
\section*{Why is the Universe Mathematically Describable?}
		
		The T0-Theory offers a possible answer: The universe is mathematically describable because it is \textbf{itself} mathematically structured. The parameter $\xipar$ is not just a description of nature---it \textbf{is} nature.
		
		\begin{itemize}
			\item \textbf{Platonic Perspective:} Mathematical structures are fundamental
			\item \textbf{Pythagorean Perspective:} ``Everything is number and harmony''
			\item \textbf{Modern Interpretation:} Geometry as the basis of physics
		\end{itemize}
% end box foundation
	
	\subsection{The Anthropic Principle}
	
\section*{Alternative}
\section*{Weak vs. Strong Anthropic Principle:}
		
\section*{Weak (observation-dependent):}
		\begin{itemize}
			\item We observe $\xipar = \frac{4}{3} \times 10^{-4}$ because only in such a universe can observers exist
			\item Multiverse with different $\xipar$ values
		\end{itemize}
		
\section*{Strong (principled):}
		\begin{itemize}
			\item $\xipar$ has this value \textbf{because} it follows from the logic of spacetime
			\item Only this value is mathematically consistent
		\end{itemize}
% end box alternative
	
	
	
	
	\section{Experimental Confirmation}
	
	\subsection{Successful Predictions}
	
	The T0-Theory has already passed several experimental tests.
	
	\subsection{Testable Predictions}
	
\section*{Key Result}
		The theory makes specific, falsifiable predictions:
		\begin{enumerate}
			\item Neutrino Mass: $m_\nu = 4{,}54$ meV (geometric prediction)
			\item Tau Anomaly: $\Delta a_\tau = 7{,}1 \times 10^{-9}$ (not yet measurable)
			\item Modified Gravity at Characteristic T0 Length Scales
			\item Alternative Cosmological Parameters without Dark Energy
		\end{enumerate}
% end box keyresult
	
	\section{Summary and Outlook}
	
	\subsection{The Central Insights}
	
\section*{Foundation}
\section*{Fundamental T0 Principles:}
		
		\begin{enumerate}
			\item \textbf{Geometric Unity:} One parameter $\xipar = \frac{4}{3} \times 10^{-4}$ determines all physics
			\item \textbf{Fractal Structure:} Quantum spacetime with $D_f = 2.94$ and $K_{\text{frak}} = 0.986$
			\item \textbf{Harmonic Order:} 4/3 as fundamental harmonic ratio
			\item \textbf{Hierarchical Scales:} From Planck to cosmological dimensions
			\item \textbf{Experimental Testability:} Concrete, falsifiable predictions
		\end{enumerate}
% end box foundation
	
	
	\subsection{The Next Steps}
	
	This first document of the T0 Series has established the fundamental principles. The following documents will deepen these foundations in specific applications.
	
	\section{Structure of the T0 Document Series}
	
	This foundational document forms the starting point for a systematic presentation of the T0-Theory. The following documents deepen specific aspects:
	
	\begin{itemize}
		\item \textbf{T0\_FineStructure\_En.tex}: Mathematical Derivation of the Fine Structure Constant
		\item \textbf{T0\_GravitationalConstant\_En.tex}: Detailed Calculation of Gravity
		\item \textbf{T0\_ParticleMasses\_En.tex}: Systematic Mass Calculation of All Fermions
		\item \textbf{T0\_Neutrinos\_En.tex}: Special Treatment of Neutrino Physics
		\item \textbf{T0\_AnomalousMagneticMoments\_En.tex}: Solution to the Muon g-2 Anomaly
		\item \textbf{T0\_Cosmology\_En.tex}: Cosmological Applications of the T0-Theory
		\item \textbf{T0\_QM-QFT-RT\_En.tex}: Complete Quantum Field Theory in the T0 Framework with Quantum Mechanics and Quantum Computing Applications
	\end{itemize}
	
	Each document builds on the principles established here and demonstrates their application in a specific area of physics.
	
	\section{References}
	
	\subsection{Fundamental T0 Documents}
	
	\begin{enumerate}
		\item Pascher, J. (2025). \textit{T0-Theory: Derivation of the Gravitational Constant}. Technical Documentation.
		\item Pascher, J. (2025). \textit{T0-Model: Parameter-Free Particle Mass Calculation with Fractal Corrections}. Scientific Treatise.
		\item Pascher, J. (2025). \textit{T0-Model: Unified Neutrino Formula Structure}. Special Analysis.
	\end{enumerate}
	
	\subsection{Related Works}
	
	\begin{enumerate}
		\item Einstein, A. (1915). \textit{The Field Equations of Gravitation}. Proceedings of the Royal Prussian Academy of Sciences.
		\item Planck, M. (1900). \textit{On the Theory of the Law of Energy Distribution in the Normal Spectrum}. Proceedings of the German Physical Society.
		\item Wheeler, J.A. (1989). \textit{Information, Physics, Quantum: The Search for Links}. Proceedings of the 3rd International Symposium on Foundations of Quantum Mechanics.
	\end{enumerate}
	
	\begin{center}
		\hrule
		\vspace{0.5cm}
		\textit{This document is part of the new T0 Series}\\
		\textit{and replaces the older, inconsistent presentations}\\
		\vspace{0.3cm}
\section*{T0-Theory: Time-Mass Duality Framework}
		\textit{Johann Pascher, HTL Leonding, Austria}\\
	\end{center}
	


% Bibliography
\begin{thebibliography}{99}
	
	\bibitem{pdg2024}
	Particle Data Group Collaboration (2024). 
	\textit{Review of Particle Physics}. 
	Progress of Theoretical and Experimental Physics, 2024(8), 083C01.
	\url{https://pdg.lbl.gov}
	
	\bibitem{flag2024}
	Aoki, Y., et al. (FLAG Collaboration) (2024). 
	\textit{FLAG Review 2024 of Lattice Results for Low-Energy Constants}. 
	arXiv:2411.04268.
	\url{https://arxiv.org/abs/2411.04268}
	
	\bibitem{fermilab_muon_g2}
	Abi, B., et al. (Muon g-2 Collaboration) (2021). 
	\textit{Measurement of the Positive Muon Anomalous Magnetic Moment to 0.46 ppm}. 
	Physical Review Letters, 126, 141801.
	
	\bibitem{peskin_schroeder}
	Peskin, M. E., \& Schroeder, D. V. (1995). 
	\textit{An Introduction to Quantum Field Theory}. 
	Addison-Wesley.
	
	\bibitem{weinberg_qft}
	Weinberg, S. (1995). 
	\textit{The Quantum Theory of Fields, Vol. I--III}. 
	Cambridge University Press.
	
	\bibitem{griffiths_particle}
	Griffiths, D. (2008). 
	\textit{Introduction to Elementary Particles}. 
	Wiley-VCH.
	
	\bibitem{mandl_shaw}
	Mandl, F., \& Shaw, G. (2010). 
	\textit{Quantum Field Theory (2nd ed.)}. 
	Wiley.
	
	\bibitem{srednicki_qft}
	Srednicki, M. (2007). 
	\textit{Quantum Field Theory}. 
	Cambridge University Press.
	
	\bibitem{t0_fundamentals}
	Pascher, J. (2024). 
	\textit{T0-Theory: Foundations of Time-Mass Duality}. 
	Unpublished manuscript, HTL Leonding.
	
	\bibitem{t0_fine_structure}
	Pascher, J. (2024). 
	\textit{T0-Theory: The Fine Structure Constant}. 
	Unpublished manuscript, HTL Leonding.
	
	\bibitem{t0_neutrinos}
	Pascher, J. (2024). 
	\textit{T0-Theory: Neutrino Masses and PMNS Mixing}. 
	Unpublished manuscript, HTL Leonding.
	
	\bibitem{t0_github}
	Pascher, J. (2024--2025). 
	\textit{T0-Time-Mass-Duality Repository}. 
	GitHub.
	\url{https://github.com/jpascher/T0-Time-Mass-Duality}
	
	\bibitem{lattice_qcd_review}
	Kronfeld, A. S. (2012). 
	\textit{Twenty-first Century Lattice Gauge Theory: Results from the QCD Lagrangian}. 
	Annual Review of Nuclear and Particle Science, 62, 265--284.
	
	\bibitem{neutrino_mixing_pdg}
	Particle Data Group Collaboration (2024). 
	\textit{Neutrino Masses, Mixing, and Oscillations}. 
	PDG Review 2024.
	\url{https://pdg.lbl.gov/2024/reviews/rpp2024-rev-neutrino-mixing.pdf}
	
	\bibitem{higgs_discovery}
	ATLAS and CMS Collaborations (2012). 
	\textit{Observation of a New Particle in the Search for the Standard Model Higgs Boson}. 
	Physics Letters B, 716, 1--29.
	
	\bibitem{Brannen2005}
	C. P. Brannen, ``Estimate of neutrino masses from Koide's relation'', \textit{arXiv:hep-ph/0505028} (2005).
	\url{https://arxiv.org/abs/hep-ph/0505028}
	
	\bibitem{Brannen2006}
	C. P. Brannen, ``Koide Mass Formula for Neutrinos'', \textit{arXiv:0702.0052} (2006).
	\url{http://brannenworks.com/MASSES.pdf}
	
	\bibitem{PhaseVectors2025}
	Anonymous, ``The Koide Relation and Lepton Mass Hierarchy from Phase Vectors'', \textit{rXiv:2507.0040} (2025).
	\url{https://rxiv.org/pdf/2507.0040v1.pdf}
	
	\bibitem{PDG2025}
	Particle Data Group, ``Review of Particle Physics'', \textit{Phys. Rev. D} \textbf{112} (2025) 030001.
	\url{https://pdg.lbl.gov/2025/}
	
	\bibitem{terrell2024}
	Terrell et al. (2024). 
	\textit{Single-Clock Metrology in Nature}. 
	Nature Physics.
	
	\bibitem{hossenfelder2024}
	Hossenfelder, S. (2024). 
	\textit{Single Clock Video Explanation}. 
	YouTube.
	
	\bibitem{hundert1931}
	Hundert (1931). 
	\textit{Reference Work}. 
	Publisher.
	
	\bibitem{terrell2025}
	Terrell et al. (2025). 
	\textit{Advanced Clock Synchronization Methods}. 
	Physical Review Letters.
	
	\bibitem{pascher_t0_2025}
	Pascher, J. (2025). 
	\textit{T0-Theory: Complete Framework and Applications}. 
	Unpublished manuscript, HTL Leonding.
	
	\bibitem{t0qm}
	Pascher, J. (2024). 
	\textit{T0-Theory: Quantum Mechanics Formulation}. 
	Unpublished manuscript, HTL Leonding.
	
	\bibitem{t0anomale}
	Pascher, J. (2024). 
	\textit{T0-Theory: Anomalous Magnetic Moments}. 
	Unpublished manuscript, HTL Leonding.
	
	\bibitem{muong2complete}
	Abi, B., et al. (Muon g-2 Collaboration) (2023). 
	\textit{Complete Measurement of the Positive Muon Anomalous Magnetic Moment}. 
	Physical Review Letters, 131, 161802.
	
	\bibitem{penrose2004}
	Penrose, R. (2004). 
	\textit{The Road to Reality: A Complete Guide to the Laws of the Universe}. 
	Jonathan Cape.
	
	\bibitem{planck1900}
	Planck, M. (1900). 
	\textit{On the Theory of the Energy Distribution Law of the Normal Spectrum}. 
	Verhandlungen der Deutschen Physikalischen Gesellschaft, 2, 237.
	
	\bibitem{T0Theory}
	Pascher, J. (2024). 
	\textit{T0-Theory: Fundamental Principles}. 
	Unpublished manuscript, HTL Leonding.
	
	% Additional bibliography entries for all undefined citations
	\bibitem{6g_roadmap}
	6G Research Consortium (2024).
	\textit{6G Technology Roadmap}.
	Technical Report.
	
	\bibitem{Born2013}
	Born, M. (2013).
	\textit{Einstein's Theory of Relativity}.
	Dover Publications.
	
	\bibitem{Casimir1948}
	Casimir, H. B. G. (1948).
	\textit{On the attraction between two perfectly conducting plates}.
	Proc. Kon. Ned. Akad. Wetensch. B51, 793--795.
	
	\bibitem{Einstein1905}
	Einstein, A. (1905).
	\textit{On the Electrodynamics of Moving Bodies}.
	Annalen der Physik, 17, 891--921.
	
	\bibitem{Feynman2006}
	Feynman, R. P. (2006).
	\textit{QED: The Strange Theory of Light and Matter}.
	Princeton University Press.
	
	\bibitem{Griffiths2017}
	Griffiths, D. J. (2017).
	\textit{Introduction to Electrodynamics (4th ed.)}.
	Cambridge University Press.
	
	\bibitem{Jackson1999}
	Jackson, J. D. (1999).
	\textit{Classical Electrodynamics (3rd ed.)}.
	Wiley.
	
	\bibitem{Mohr2016}
	Mohr, P. J., et al. (2016).
	\textit{CODATA Recommended Values of the Fundamental Physical Constants: 2014}.
	Rev. Mod. Phys. 88, 035009.
	
	\bibitem{Parker2018}
	Parker, R. H., et al. (2018).
	\textit{Measurement of the fine-structure constant as a test of the Standard Model}.
	Science, 360, 191--195.
	
	\bibitem{Planck1900}
	Planck, M. (1900).
	\textit{On the Theory of the Energy Distribution Law of the Normal Spectrum}.
	Verhandlungen der Deutschen Physikalischen Gesellschaft, 2, 237.
	
	\bibitem{Planck2018}
	Planck Collaboration (2018).
	\textit{Planck 2018 results. VI. Cosmological parameters}.
	Astronomy \& Astrophysics, 641, A6.
	
	\bibitem{QFT_T0}
	Pascher, J. (2024).
	\textit{T0-Theory and QFT Connections}.
	Unpublished manuscript, HTL Leonding.
	
	\bibitem{Sommerfeld1916}
	Sommerfeld, A. (1916).
	\textit{On the Quantum Theory of Spectral Lines}.
	Annalen der Physik, 51, 1--94.
	
	\bibitem{T0_Feinstruktur}
	Pascher, J. (2024).
	\textit{T0-Theory: Fine Structure Analysis}.
	Unpublished manuscript, HTL Leonding.
	
	\bibitem{T0_SI}
	Pascher, J. (2024).
	\textit{T0-Theory and SI Units}.
	Unpublished manuscript, HTL Leonding.
	
	\bibitem{T0_fine_structure}
	Pascher, J. (2024).
	\textit{T0-Theory: The Fine Structure Constant}.
	Unpublished manuscript, HTL Leonding.
	
	\bibitem{T0_g2_erweiterung}
	Pascher, J. (2024).
	\textit{T0-Theory: g-2 Extensions}.
	Unpublished manuscript, HTL Leonding.
	
	\bibitem{T0_gravitational_constant}
	Pascher, J. (2024).
	\textit{T0-Theory: Gravitational Constant Derivation}.
	Unpublished manuscript, HTL Leonding.
	
	\bibitem{T0_netze_en}
	Pascher, J. (2024).
	\textit{T0-Theory: Network Structures}.
	Unpublished manuscript, HTL Leonding.
	
	\bibitem{T0_tm_erweiterung}
	Pascher, J. (2024).
	\textit{T0-Theory: Time-Mass Extensions}.
	Unpublished manuscript, HTL Leonding.
	
	\bibitem{Uzan2003}
	Uzan, J.-P. (2003).
	\textit{The fundamental constants and their variation}.
	Rev. Mod. Phys. 75, 403--455.
	
	\bibitem{Weinberg1995}
	Weinberg, S. (1995).
	\textit{The Quantum Theory of Fields, Vol. I}.
	Cambridge University Press.
	
	\bibitem{albrecht1999}
	Albrecht, A. \& Magueijo, J. (1999).
	\textit{A time varying speed of light as a solution to cosmological puzzles}.
	Phys. Rev. D 59, 043516.
	
	\bibitem{alice2023}
	ALICE Collaboration (2023).
	\textit{Recent results from ALICE}.
	CERN-EP-2023-XXX.
	
	\bibitem{analog_optical}
	Smith, J. et al. (2024).
	\textit{Analog optical computing systems}.
	Nature Photonics.
	
	\bibitem{ashtekar2004}
	Ashtekar, A. \& Lewandowski, J. (2004).
	\textit{Background independent quantum gravity}.
	Class. Quantum Grav. 21, R53.
	
	\bibitem{atlas2023}
	ATLAS Collaboration (2023).
	\textit{ATLAS physics results}.
	CERN-PH-EP-2023-XXX.
	
	\bibitem{atlas2023higgs}
	ATLAS Collaboration (2023).
	\textit{Higgs boson measurements}.
	Phys. Rev. Lett.
	
	\bibitem{barbour1999}
	Barbour, J. (1999).
	\textit{The End of Time}.
	Oxford University Press.
	
	\bibitem{barrow1999}
	Barrow, J. D. (1999).
	\textit{Cosmologies with varying light speed}.
	Phys. Rev. D 59, 043515.
	
	\bibitem{becker2007}
	Becker, K. et al. (2007).
	\textit{String Theory and M-Theory}.
	Cambridge University Press.
	
	\bibitem{bell_muon}
	Bennett, G. W., et al. (Muon g-2 Collaboration) (2006).
	\textit{Final report of the E821 muon anomalous magnetic moment measurement}.
	Phys. Rev. D 73, 072003.
	
	\bibitem{bondi1948}
	Bondi, H. \& Gold, T. (1948).
	\textit{The steady-state theory of the expanding universe}.
	Mon. Not. R. Astron. Soc. 108, 252--270.
	
	\bibitem{brewer2019}
	Brewer, S. M. et al. (2019).
	\textit{Al+ Quantum-Logic Clock with Systematic Uncertainty below $10^{-18}$}.
	Phys. Rev. Lett. 123, 033201.
	
	\bibitem{cms2023top}
	CMS Collaboration (2023).
	\textit{Top quark measurements at CMS}.
	JHEP 2023.
	
	\bibitem{cms2024}
	CMS Collaboration (2024).
	\textit{CMS physics results 2024}.
	CERN-PH-EP-2024-XXX.
	
	\bibitem{codata2019}
	Tiesinga, E. et al. (2019).
	\textit{The 2018 CODATA Recommended Values}.
	J. Phys. Chem. Ref. Data.
	
	\bibitem{desi2025}
	DESI Collaboration (2025).
	\textit{DESI 2025 Cosmology Results}.
	arXiv preprint.
	
	\bibitem{differential_optical}
	Wang, X. et al. (2024).
	\textit{Differential optical computing}.
	Optica.
	
	\bibitem{dingle1972}
	Dingle, H. (1972).
	\textit{Science at the Crossroads}.
	Martin Brian \& O'Keeffe.
	
	\bibitem{divalentino2021}
	Di Valentino, E. et al. (2021).
	\textit{In the realm of the Hubble tension}.
	Class. Quantum Grav. 38, 153001.
	
	\bibitem{elnaschie2004}
	El Naschie, M. S. (2004).
	\textit{A review of E infinity theory}.
	Chaos, Solitons \& Fractals, 19, 209--236.
	
	\bibitem{fabrication_heterogeneous}
	Chen, Y. et al. (2024).
	\textit{Heterogeneous photonic integration}.
	Nature Electronics.
	
	\bibitem{fermilab2023}
	Fermilab (2023).
	\textit{Muon g-2 results}.
	Phys. Rev. Lett.
	
	\bibitem{flexible_wafer}
	Kim, S. et al. (2024).
	\textit{Flexible wafer-scale photonics}.
	Science Advances.
	
	\bibitem{francesco1997}
	Di Francesco, P. et al. (1997).
	\textit{Conformal Field Theory}.
	Springer.
	
	\bibitem{hartree1957}
	Hartree, D. R. (1957).
	\textit{The Calculation of Atomic Structures}.
	Wiley.
	
	\bibitem{hhi_6g}
	Fraunhofer HHI (2024).
	\textit{6G Photonic Integration}.
	Technical Report.
	
	\bibitem{hossenfelder2025}
	Hossenfelder, S. (2025).
	\textit{Science without the gobbledygook}.
	YouTube/Blog.
	
	\bibitem{hossenfelder_single_clock_video}
	Hossenfelder, S. (2024).
	\textit{The Single Clock Problem}.
	YouTube.
	
	\bibitem{hoyle1948}
	Hoyle, F. (1948).
	\textit{A new model for the expanding universe}.
	Mon. Not. R. Astron. Soc. 108, 372--382.
	
	\bibitem{integration_microelectronic}
	Liu, A. et al. (2024).
	\textit{Microelectronic photonic integration}.
	IEEE Journal.
	
	\bibitem{jacobson1995}
	Jacobson, T. (1995).
	\textit{Thermodynamics of spacetime}.
	Phys. Rev. Lett. 75, 1260.
	
	\bibitem{kasevich2023}
	Kasevich, M. et al. (2023).
	\textit{Atom interferometry tests}.
	Nature Physics.
	
	\bibitem{lerner2014}
	Lerner, E. J. (2014).
	\textit{An open letter on cosmology}.
	New Scientist.
	
	\bibitem{lisa2017}
	LISA Consortium (2017).
	\textit{Laser Interferometer Space Antenna}.
	ESA Technical Report.
	
	\bibitem{lithium_tantalate}
	Zhang, M. et al. (2024).
	\textit{Thin-film lithium tantalate photonics}.
	Nature Photonics.
	
	\bibitem{lopez2010}
	Lopez-Corredoira, M. (2010).
	\textit{Tests and problems of the standard model in cosmology}.
	Int. J. Mod. Phys. D.
	
	\bibitem{ludlow2015}
	Ludlow, A. D. et al. (2015).
	\textit{Optical atomic clocks}.
	Rev. Mod. Phys. 87, 637.
	
	\bibitem{mach1883}
	Mach, E. (1883).
	\textit{Die Mechanik in ihrer Entwickelung}.
	F.A. Brockhaus.
	
	\bibitem{maldacena1998}
	Maldacena, J. (1998).
	\textit{The large N limit of superconformal field theories}.
	Adv. Theor. Math. Phys. 2, 231--252.
	
	\bibitem{mueller2014}
	Müller, H. et al. (2014).
	\textit{Atom interferometry tests of the gravitational redshift}.
	Phys. Rev. Lett.
	
	\bibitem{mug2_final_2025}
	Muon g-2 Collaboration (2025).
	\textit{Final muon g-2 measurement}.
	Phys. Rev. Lett.
	
	\bibitem{muong2_2023}
	Muon g-2 Collaboration (2023).
	\textit{Updated muon g-2 results}.
	Phys. Rev. Lett.
	
	\bibitem{nathan2024}
	Nathan, A. et al. (2024).
	\textit{Quantum computing advances}.
	Nature.
	
	\bibitem{newell2018}
	Newell, D. B. et al. (2018).
	\textit{The CODATA 2017 values of h, e, k, and $N_A$}.
	Metrologia 55, L13.
	
	\bibitem{nottale1993}
	Nottale, L. (1993).
	\textit{Fractal Space-Time and Microphysics}.
	World Scientific.
	
	\bibitem{on_chip_lithium}
	Wang, C. et al. (2024).
	\textit{On-chip lithium niobate photonics}.
	Nature Communications.
	
	\bibitem{optical_advantages}
	Shastri, B. J. et al. (2024).
	\textit{Advantages of optical computing}.
	Nature Reviews Physics.
	
	\bibitem{pascher2025cmb}
	Pascher, J. (2025).
	\textit{T0-Theory: CMB Analysis}.
	Unpublished manuscript, HTL Leonding.
	
	\bibitem{pascher2025g2}
	Pascher, J. (2025).
	\textit{T0-Theory: g-2 Predictions}.
	Unpublished manuscript, HTL Leonding.
	
	\bibitem{pascher2025qm}
	Pascher, J. (2025).
	\textit{T0-Theory: Quantum Mechanics}.
	Unpublished manuscript, HTL Leonding.
	
	\bibitem{pascher2025si}
	Pascher, J. (2025).
	\textit{T0-Theory: SI Unit System}.
	Unpublished manuscript, HTL Leonding.
	
	\bibitem{pascher2025t0}
	Pascher, J. (2025).
	\textit{T0-Theory: Complete Framework}.
	Unpublished manuscript, HTL Leonding.
	
	\bibitem{pascher:fundamentals}
	Pascher, J. (2024).
	\textit{T0-Theory: Fundamentals}.
	Unpublished manuscript, HTL Leonding.
	
	\bibitem{pascher:g2_rev9}
	Pascher, J. (2024).
	\textit{T0-Theory: g-2 Revision 9}.
	Unpublished manuscript, HTL Leonding.
	
	\bibitem{pascher:geometric_formalism}
	Pascher, J. (2024).
	\textit{T0-Theory: Geometric Formalism}.
	Unpublished manuscript, HTL Leonding.
	
	\bibitem{pascher:ml_addendum}
	Pascher, J. (2024).
	\textit{T0-Theory: Machine Learning Addendum}.
	Unpublished manuscript, HTL Leonding.
	
	\bibitem{pascher:t0_foundations}
	Pascher, J. (2024).
	\textit{T0-Theory: Foundations}.
	Unpublished manuscript, HTL Leonding.
	
	\bibitem{pascher_derivation_beta_2025}
	Pascher, J. (2025).
	\textit{T0-Theory: Derivation of Beta}.
	Unpublished manuscript, HTL Leonding.
	
	\bibitem{pascher_higgs_connection_2025}
	Pascher, J. (2025).
	\textit{T0-Theory: Higgs Connection}.
	Unpublished manuscript, HTL Leonding.
	
	\bibitem{pascher_lagrangian_extended_2025}
	Pascher, J. (2025).
	\textit{T0-Theory: Extended Lagrangian}.
	Unpublished manuscript, HTL Leonding.
	
	\bibitem{pascher_mathematical_structure_2025}
	Pascher, J. (2025).
	\textit{T0-Theory: Mathematical Structure}.
	Unpublished manuscript, HTL Leonding.
	
	\bibitem{pascher_t0_cmb_2025}
	Pascher, J. (2025).
	\textit{T0-Theory: CMB Predictions}.
	Unpublished manuscript, HTL Leonding.
	
	\bibitem{pascher_t0_energie_2025}
	Pascher, J. (2025).
	\textit{T0-Theory: Energy}.
	Unpublished manuscript, HTL Leonding.
	
	\bibitem{pascher_t0_energy_2025}
	Pascher, J. (2025).
	\textit{T0-Theory: Energy Framework}.
	Unpublished manuscript, HTL Leonding.
	
	\bibitem{pascher_t0_theory_2025}
	Pascher, J. (2025).
	\textit{T0-Theory: Complete Theory}.
	Unpublished manuscript, HTL Leonding.
	
	\bibitem{penrose1959}
	Penrose, R. (1959).
	\textit{The apparent shape of a relativistically moving sphere}.
	Proc. Cambridge Phil. Soc. 55, 137--139.
	
	\bibitem{penrose1967}
	Penrose, R. (1967).
	\textit{Twistor algebra}.
	J. Math. Phys. 8, 345--366.
	
	\bibitem{peratt1992}
	Peratt, A. L. (1992).
	\textit{Physics of the Plasma Universe}.
	Springer-Verlag.
	
	\bibitem{peskin1995}
	Peskin, M. E. \& Schroeder, D. V. (1995).
	\textit{An Introduction to Quantum Field Theory}.
	Addison-Wesley.
	
	\bibitem{peskin_schroeder_1995}
	Peskin, M. E. \& Schroeder, D. V. (1995).
	\textit{An Introduction to Quantum Field Theory}.
	Addison-Wesley.
	
	\bibitem{phoquant}
	PhoQuant (2024).
	\textit{Photonic quantum computing}.
	Technical Report.
	
	\bibitem{photonics_ai}
	Wetzstein, G. et al. (2024).
	\textit{Photonics for AI}.
	Nature.
	
	\bibitem{planck1906}
	Planck, M. (1906).
	\textit{The Theory of Heat Radiation}.
	Johann Ambrosius Barth.
	
	\bibitem{planck2018}
	Planck Collaboration (2018).
	\textit{Planck 2018 results}.
	A\&A 641, A6.
	
	\bibitem{polchinski1998}
	Polchinski, J. (1998).
	\textit{String Theory}.
	Cambridge University Press.
	
	\bibitem{qant_nps}
	QANT (2024).
	\textit{Quantum photonics systems}.
	Technical Report.
	
	\bibitem{quantenjahr25}
	Quantenjahr (2025).
	\textit{International Year of Quantum}.
	UNESCO.
	
	\bibitem{recurrent_photonics}
	Tait, A. N. et al. (2024).
	\textit{Recurrent photonic neural networks}.
	Optica.
	
	\bibitem{rf_photonics}
	Capmany, J. \& Novak, D. (2024).
	\textit{Microwave photonics}.
	Nature Photonics.
	
	\bibitem{riess2019}
	Riess, A. G. et al. (2019).
	\textit{Large Magellanic Cloud Cepheid Standards}.
	ApJ 876, 85.
	
	\bibitem{riess2022}
	Riess, A. G. et al. (2022).
	\textit{A Comprehensive Measurement of H0}.
	ApJ 934, L7.
	
	\bibitem{rovelli2004}
	Rovelli, C. (2004).
	\textit{Quantum Gravity}.
	Cambridge University Press.
	
	\bibitem{sciama1953}
	Sciama, D. W. (1953).
	\textit{On the origin of inertia}.
	Mon. Not. R. Astron. Soc. 113, 34--42.
	
	\bibitem{sciencedaily2025}
	ScienceDaily (2025).
	\textit{Physics news}.
	Online.
	
	\bibitem{sm_g2_2025}
	Aoyama, T. et al. (2025).
	\textit{Standard Model prediction for g-2}.
	Phys. Rep.
	
	\bibitem{susskind1995}
	Susskind, L. (1995).
	\textit{The world as a hologram}.
	J. Math. Phys. 36, 6377--6396.
	
	\bibitem{t0_kosmologie}
	Pascher, J. (2024).
	\textit{T0-Theory: Cosmology}.
	Unpublished manuscript, HTL Leonding.
	
	\bibitem{terrell1959}
	Terrell, J. (1959).
	\textit{Invisibility of the Lorentz contraction}.
	Phys. Rev. 116, 1041--1045.
	
	\bibitem{terrell_single_clock_nature_2024}
	Terrell, J. et al. (2024).
	\textit{Single clock precision measurements}.
	Nature Physics.
	
	\bibitem{tfln_foundry}
	TFLN Foundry (2024).
	\textit{Thin-film lithium niobate foundry services}.
	Technical Specifications.
	
	\bibitem{thiemann2007}
	Thiemann, T. (2007).
	\textit{Modern Canonical Quantum General Relativity}.
	Cambridge University Press.
	
	\bibitem{thz_epfl}
	EPFL (2024).
	\textit{Terahertz photonics research}.
	Technical Report.
	
	\bibitem{unnikrishnan2004}
	Unnikrishnan, C. S. (2004).
	\textit{On Einstein's resolution of the twin clock paradox}.
	Current Science, 86, 704--709.
	
	\bibitem{verlinde2011}
	Verlinde, E. (2011).
	\textit{On the origin of gravity and the laws of Newton}.
	JHEP 2011, 29.
	
	\bibitem{video2025}
	Video (2025).
	\textit{Physics video explanation}.
	YouTube.
	
	\bibitem{weinberg1995}
	Weinberg, S. (1995).
	\textit{The Quantum Theory of Fields}.
	Cambridge University Press.
	
	\bibitem{weiskopf2000}
	Weiskopf, D. (2000).
	\textit{Visualization of special relativity}.
	PhD thesis, University of Tübingen.
	
	\bibitem{wheeler1990}
	Wheeler, J. A. (1990).
	\textit{A Journey into Gravity and Spacetime}.
	Scientific American Library.
	
	\bibitem{wiki_bell}
	Wikipedia (2024).
	\textit{Bell's theorem}.
	Online encyclopedia.
	
	\bibitem{zwicky1929}
	Zwicky, F. (1929).
	\textit{On the red shift of spectral lines through interstellar space}.
	Proc. Natl. Acad. Sci. 15, 773--779.

\end{thebibliography}


\end{document}
