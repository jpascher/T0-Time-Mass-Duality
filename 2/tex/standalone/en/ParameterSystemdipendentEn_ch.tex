\documentclass[11pt,a4paper]{article}
\usepackage[a4paper,margin=2cm]{geometry}
\usepackage[utf8]{inputenc}
\usepackage[english]{babel}
\usepackage{lmodern}
\renewcommand{\familydefault}{\sfdefault}

\usepackage{amsmath,amssymb,amsthm}
\usepackage{graphicx}
\usepackage[unicode,pdfencoding=auto,hypertexnames=false]{hyperref}
\usepackage{booktabs}
\usepackage{longtable}
\usepackage{array}
\usepackage{siunitx}
\usepackage{fancyhdr}
\usepackage{float}
\usepackage{tikz}
% tcolorbox removed for standalone
% tcbset removed
\tikzset{
  t0blue/.style={draw=blue,fill=blue!10},
  t0red/.style={draw=red,fill=red!10},
  t0green/.style={draw=green!50!black,fill=green!10},
  t0orange/.style={draw=orange,fill=orange!10},
}
\usepackage{setspace}
\usepackage{enumitem}
\usepackage{adjustbox}
\usepackage{xcolor}

% Define colors for xcolor package
\definecolor{t0green}{RGB}{34,139,34}
\definecolor{t0blue}{RGB}{0,0,255}
\definecolor{t0red}{RGB}{255,0,0}
\definecolor{t0orange}{RGB}{255,165,0}

% Define custom column types for tables
\newcolumntype{L}[1]{>{\raggedright\arraybackslash}p{#1}}
\newcolumntype{C}[1]{>{\centering\arraybackslash}p{#1}}
\newcolumntype{R}[1]{>{\raggedleft\arraybackslash}p{#1}}

\setlength{\parindent}{0pt}
\setlength{\parskip}{6pt}

\hypersetup{
  colorlinks=true,
  linkcolor=blue,
  citecolor=blue,
  urlcolor=blue
}
\pagestyle{fancy}
\setlength{\headheight}{28pt}

\newcommand{\checkmarkx}{\checkmark}
\newcommand{\warningx}{\textbf{!}}

% Makros aus Einzel-Dokumenten (Fallback-Definitionen)
\newcommand{\mytimes}{\times}
\newcommand{\myapprox}{\approx}
\newcommand{\mysim}{\sim}
\newcommand{\myomega}{\omega}
\newcommand{\mypi}{\pi}
\newcommand{\myrightarrow}{\rightarrow}
\newcommand{\mypropto}{\propto}
\newcommand{\deltafield}{\delta\phi}
\newcommand{\xipar}{\xi}
\newcommand{\xiT}{\xi}
\newcommand{\lambdah}{\lambda_h}

% Additional macros used in chapter files
\newcommand{\Kfrak}{K_{\text{frak}}}  % Fractal correction factor
\newcommand{\Dfrak}{D_f}              % Fractal dimension
\newcommand{\betapar}{\beta}          % T0 beta parameter
\newcommand{\alphapar}{\alpha}        % T0 alpha parameter
\newcommand{\Efield}{E}               % Energy field
% Note: checkmarkxa/warningxa are variants used in auto-generated chapter files
\newcommand{\checkmarkxa}{\checkmark}
\newcommand{\warningxa}{\textbf{!}}

% Additional T0-specific macros
\newcommand{\xigeom}{\xi_{\text{geom}}}  % Geometric xi
\newcommand{\lP}{\ell_P}                  % Planck length
\newcommand{\rzero}{r_0}                  % Characteristic radius
\newcommand{\xirat}{\xi_{\text{rat}}}     % Xi ratio
\newcommand{\tzero}{t_0}                  % Characteristic time
\newcommand{\natunits}{\text{(nat. units)}}  % Natural units annotation
\newcommand{\myRightarrow}{\Rightarrow}   % Arrow variant
\newcommand{\Lag}{\mathcal{L}}            % Lagrangian

% Physics macros used in chapter files
\newcommand{\CQCD}{C_{\text{QCD}}}        % QCD correction
\newcommand{\EP}{E_P}                     % Planck energy
\newcommand{\Ee}{E_e}                     % Electron energy
\newcommand{\Emu}{E_\mu}                  % Muon energy
\newcommand{\Exi}{E_\xi}                  % Xi energy
\newcommand{\Ezero}{E_0}                  % Characteristic energy
\newcommand{\Hubble}{H}                   % Hubble constant
\newcommand{\Kspec}{K_{\text{spec}}}      % Spectral correction
\newcommand{\Lambdat}{\Lambda_t}          % Time-related cosmological constant
\newcommand{\Leff}{\mathcal{L}_{\text{eff}}}  % Effective Lagrangian
\newcommand{\Lorentz}{\mathcal{L}}        % Lorentz symbol
\newcommand{\Lxi}{L_\xi}                  % Xi length
\newcommand{\Tfield}{T}                   % Time field
\newcommand{\Weyl}{W}                     % Weyl tensor/symbol
\newcommand{\alphaEMSI}{\alpha_{\text{EM,SI}}}  % EM alpha in SI
\newcommand{\alphaEMnat}{\alpha_{\text{EM,nat}}}  % EM alpha in natural units
\newcommand{\alphaem}{\alpha_{\text{em}}} % Electromagnetic alpha
\newcommand{\betaTSI}{\beta_{T,\text{SI}}}  % Beta in SI
\newcommand{\betaTnat}{\beta_{T,\text{nat}}}  % Beta in natural units
\newcommand{\deltam}{\delta m}            % Mass difference
\newcommand{\phiT}{\phi_T}                % T-field phi
\newcommand{\tP}{t_P}                     % Planck time
\newcommand{\rhoCMB}{\rho_{\text{CMB}}}   % CMB density
\newcommand{\rhoCasimir}{\rho_{\text{Casimir}}}  % Casimir density

% Table formatting
\usepackage{multirow}

% Additional physics macros
\newcommand{\Riem}{\mathcal{R}}           % Riemann tensor
\newcommand{\ZPinch}{Z_{\text{pinch}}}    % Z-pinch
\newcommand{\SynchPower}{P_{\text{synch}}} % Synchrotron power
\newcommand{\Rzero}{R_0}                  % Characteristic radius
\newcommand{\alphafine}{\alpha}           % Fine structure constant
\newcommand{\Etau}{E_\tau}                % Tau energy
\newcommand{\deltaE}{\delta E}            % Energy deviation
\newcommand{\EPlanck}{E_P}                % Planck energy
\newcommand{\pichar}{\pi}                 % Pi character
\newcommand{\alphaWSI}{\alpha_{W,\text{SI}}}  % Wien alpha in SI
\newcommand{\alphaWnat}{\alpha_{W,\text{nat}}}  % Wien alpha in natural units

% Einfache abstract-Umgebung für Kapitel:
\newenvironment{abstract}{%
  \begin{center}\bfseries Abstract\end{center}\small
}{\par}


\title{ParameterSystemdipendentEn}
\author{J. Pascher}
\date{\today}

\begin{document}
\maketitle

\section*{Parametersystemdipendenten (ParameterSystemdipendentEn)}

	\begin{abstract}
		This paper systematically analyzes the parameter dependency between SI units and T0-model natural units, revealing that fundamental parameters like $\xipar$, $\alpha_{\text{EM}}$, $\beta_{\text{T}}$, and Yukawa couplings have dramatically different numerical values in different unit systems. Through detailed calculations, we demonstrate that direct transfer of parameter values between systems leads to errors spanning multiple orders of magnitude. The analysis extends beyond specific parameters to establish universal transformation rules and provides critical warnings against naive parameter transfer. This work establishes that the apparent inconsistencies in T0-model parameters are actually systematic unit-system dependencies that require careful transformation protocols for experimental verification.
	\end{abstract}
	
	
	\section{Introduction}
	\label{ParameterSystem:L-T0_tm-erweiterung-x6-0008}
	
	\subsection{The Parameter Transfer Problem}
	\label{ParameterSystem:L-ParameterSystemdipendentEn-0729}
	
	The T0 model, formulated in natural units where $\hbar = c = G = k_B = \alpha_{\text{EM}} = \alpha_{\text{W}} = \beta_{\text{T}} = 1$, presents a fundamental challenge when compared with experimental data expressed in SI units. This paper demonstrates that the apparent inconsistencies between T0-model predictions and experimental observations are not physical contradictions but systematic unit-system dependencies.
	
	The core insight is that parameters such as $\xipar$, $\alpha_{\text{EM}}$, and $\beta_{\text{T}}$ represent fundamentally different quantities when expressed in different unit systems:
	
	$$\xipar_{\text{SI}} \neq \xipar_{\text{nat}}, \quad \alphaEMSI \neq \alphaEMnat, \quad \betaTSI \neq \betaTnat$$
	
	\subsection{Scope and Methodology}
	\label{ParameterSystem:L-ParameterSystemdipendentEn-0730}
	
	This analysis covers:
	\begin{itemize}
		\item Systematic calculation of parameter ratios between SI and T0-natural units
		\item Demonstration of transformation invariance for dimensionless ratios
		\item Extension to variable parameters like $\xipar$ and Yukawa couplings
		\item Universal warnings against direct parameter transfer
		\item Guidelines for correct experimental comparison protocols
	\end{itemize}
	
	\section{The Parameter: Variable Across Mass Scales}
	\label{ParameterSystem:L-ParameterSystemdipendentEn-0731}
	\section{The Universal -Field Framework}
	
	The cornerstone of the T0-model is the universal geometric constant that serves as the fundamental parameter for all physical calculations.
	

		The universal geometric constant:
		\begin{equation}
			\xi = \frac{4}{3} \times 10^{-4} = 1.3333... \times 10^{-4}
		\end{equation}

	
	This dimensionless constant is used throughout T0 theory to connect quantum mechanical and gravitational phenomena. It establishes the characteristic strength of field interactions and provides the foundation for unified field descriptions.
	

		For the detailed derivation and physical justification of this parameter, see the document "Parameter Derivation" (available at: \url{https://github.com/jpascher/T0-Time-Mass-Duality/2/pdf/parameterherleitung_En.pdf}).

	
	This geometric constant determines a characteristic energy scale for the $\xi$-field:
	
	\begin{equation}
		E_\xi = \frac{1}{\xi} = \frac{3}{4 \times 10^{-4}} = 7500 \text{ (natural units)}
	\end{equation}
	\subsection{Definition and Physical Meaning}
	\label{ParameterSystem:L-ParameterSystemdipendentEn-0732}
	
	The parameter $\xipar$ is also the ratio of the Schwarzschild radius to the Planck length:
	
	\begin{equation}
		\xipar = \frac{r_0}{\lP} = \frac{2Gm}{\lP}
		\label{ParameterSystem:L-TempEinheitenCMBEn-0719}
	\end{equation}
	
\textbf{Crucial:} The parameter $\xipar$ scales with the mass of the object under consideration according to $\xipar(m) = 2Gm/\lP$. The Higgs mass defines the fundamental reference scale $\xipar_0 = 1.33 \times 10^{-4}$, to which all other masses are normalized in the T0 model.
	
	\subsection{Connection to Higgs Physics}
	\label{ParameterSystem:L-ParameterSystemdipendentEn-0733}
	
	The T0 model establishes a fundamental connection between $\xipar$ and Higgs sector physics through the relationship derived in the complete field-theoretic framework 
	
	\begin{equation}
		\xipar = \frac{\lambdah^2 v^2}{16\pichar^3 m_h^2} \approx 1.33 \times 10^{-4}
		\label{ParameterSystem:L-ParameterSystemdipendentEn-0734}
	\end{equation}
	
	where:
	\begin{itemize}
		\item $\lambdah \approx 0.13$ (Higgs self-coupling)
		\item $v \approx 246$ GeV (Higgs VEV)
		\item $m_h \approx 125$ GeV (Higgs mass)
	\end{itemize}
	
	This represents the universal scale parameter that emerges from fundamental Standard Model physics, while the mass-dependent form $\xipar = 2Gm/\lP$ applies to specific objects.
	
	\subsection{Values in the SI System}
	\label{ParameterSystem:L-ParameterSystemdipendentEn-0735}
	
	Using SI constants:
	\begin{align}
		G &= 6.674 \times 10^{-11} \text{ m}^3/(\text{kg} \cdot \text{s}^2) \\
		\lP &= 1.616 \times 10^{-35} \text{ m}
	\end{align}
	
	We calculate $\xipar_{\text{SI}}$ for various objects:
	
	\begin{table}[htbp]
		\centering
		\begin{tabular}{lcc}
			\toprule
			\textbf{Object} & \textbf{Mass} & \textbf{$\xipar_{\text{SI}}$} \\
			\midrule
			Electron & $9.109 \times 10^{-31}$ kg & $7.52 \times 10^{-7}$ \\
			Proton & $1.673 \times 10^{-27}$ kg & $1.38 \times 10^{-3}$ \\
			Human (70 kg) & $7.0 \times 10^{1}$ kg & $6.4 \times 10^{6}$ \\
			Earth & $5.972 \times 10^{24}$ kg & $4.1 \times 10^{28}$ \\
			Sun & $1.989 \times 10^{30}$ kg & $1.8 \times 10^{38}$ \\
			Planck mass & $2.176 \times 10^{-8}$ kg & $2.0$ \\
			\bottomrule
		\end{tabular}
		\caption{$\xipar$ values for different objects in SI units}
		\label{ParameterSystem:L-ParameterSystemdipendentEn-0736}
	\end{table}
	
\section*{The parameter $\xipar$ varies over 46 orders of magnitude!}
	
	\subsection{Transformation to T0-Natural Units}
	\label{ParameterSystem:L-ParameterSystemdipendentEn-0737}
	
	Based on the comprehensive transformation analysis, the conversion factor between systems is approximately:
	
	$$\frac{\xipar_{\text{nat}}}{\xipar_{\text{SI}}} \approx 4100$$
	
	This gives T0-natural unit values:
	
	\begin{table}[htbp]
		\centering
		\begin{tabular}{lcc}
			\toprule
			\textbf{Object} & \textbf{$\xipar_{\text{SI}}$} & \textbf{$\xipar_{\text{nat}}$} \\
			\midrule
			Electron & $7.52 \times 10^{-7}$ & $3.1 \times 10^{-3}$ \\
			Proton & $1.38 \times 10^{-3}$ & $5.7$ \\
			Human (70 kg) & $6.4 \times 10^{6}$ & $2.6 \times 10^{10}$ \\
			Sun & $1.8 \times 10^{38}$ & $7.4 \times 10^{41}$ \\
			\bottomrule
		\end{tabular}
		\caption{$\xipar$ transformation between unit systems}
		\label{ParameterSystem:L-ParameterSystemdipendentEn-0738}
	\end{table}
	
	\subsection{Invariance of Ratios}
	\label{ParameterSystem:L-ParameterSystemdipendentEn-0739}
	
	\textbf{Critical verification:} The ratios between different objects remain identical in both systems:
	
	\begin{align}
		\frac{\xipar_{\text{Sun},\text{SI}}}{\xipar_{\text{e},\text{SI}}} &= \frac{1.8 \times 10^{38}}{7.52 \times 10^{-7}} = 2.4 \times 10^{44} \\
		\frac{\xipar_{\text{Sun},\text{nat}}}{\xipar_{\text{e},\text{nat}}} &= \frac{7.4 \times 10^{41}}{3.1 \times 10^{-3}} = 2.4 \times 10^{44}
	\end{align}
	
	\boxed{\text{Ratios are invariant under system transformation!}}
	
\section{The Fine-Structure Constant}
\label{ParameterSystem:L-ParameterSystemdipendentEn-0740}

\subsection{The Mystification of 1/137}
\label{ParameterSystem:L-ParameterSystemdipendentEn-0741}

The fine-structure constant $\alpha_{\text{EM}} \approx 1/137$ has been declared one of the greatest mysteries of physics by prominent physicists:

\begin{itemize}
	\item \textbf{Richard Feynman}: ``It is one of the greatest damn mysteries of physics: a magic number that comes to us with no understanding whatsoever.''
	\item \textbf{Wolfgang Pauli}: ``When I die, I will ask God two questions: Why relativity? And why 137? I believe he will have an answer for the first one.''
	\item \textbf{Max Born}: ``If $\alpha$ were larger, molecules could not exist, and there would be no life.''
\end{itemize}

\subsection{Electromagnetic Duality as the Key}
\label{ParameterSystem:L-ParameterSystemdipendentEn-0742}

What all these statements overlook: The fine-structure constant possesses two mathematically equivalent representations that reveal its true nature:

\begin{align}
	\alpha_{\text{EM}} &= \frac{e^2}{4\pi\varepsilon_0\hbar c} \quad \text{(Standard form)} \label{ParameterSystem:L-ParameterSystemdipendentEn-0743}\\
	\alpha_{\text{EM}} &= \frac{e^2 \mu_0 c}{4\pi \hbar} \quad \text{(Dual form)} \label{ParameterSystem:L-ParameterSystemdipendentEn-0744}
\end{align}

This equivalence is based on the Maxwell relation $c^2 = \frac{1}{\varepsilon_0\mu_0}$ and reveals a fundamental electromagnetic duality:

\begin{equation}
	\frac{1}{\varepsilon_0 c} = \mu_0 c
	\label{ParameterSystem:L-ParameterSystemdipendentEn-0745}
\end{equation}

\subsection{The Dual Nature of : System-Dependent yet Invariant}
\label{ParameterSystem:L-ParameterSystemdipendentEn-0746}

The fine-structure constant possesses a remarkable dual nature:

\subsubsection{As an Invariant Ratio of Physical Quantities}
\label{ParameterSystem:L-ParameterSystemdipendentEn-0747}

Regardless of the chosen system of units, $\alpha$ remains constant as a \textbf{ratio} of fundamental lengths:

\begin{equation}
	\alpha_{\text{EM}} = \frac{r_e}{\lambda_C} = \frac{\text{Classical electron radius}}{\text{Compton wavelength}}
	\label{ParameterSystem:L-ParameterSystemdipendentEn-0748}
\end{equation}

Similarly, the inverse ratio:

\begin{equation}
	\alpha_{\text{EM}}^{-1} = \frac{a_0}{\lambda_C/2\pi} = \frac{\text{Bohr radius}}{\text{Reduced Compton wavelength}} = 137.036...
	\label{ParameterSystem:L-ParameterSystemdipendentEn-0749}
\end{equation}

These ratios are \textbf{system-of-units invariant} -- they have the same numerical value in any consistent system of units, as the units cancel out in the ratio.

\subsubsection{As a System-Dependent Numerical Value}
\label{ParameterSystem:L-ParameterSystemdipendentEn-0750}

Simultaneously, the numerical value of $\alpha$ depends on the choice of fundamental units:

\begin{itemize}
	\item \textbf{SI system}: $\alpha = \frac{e^2}{4\pi\varepsilon_0\hbar c} \approx 1/137$
	\item \textbf{Natural units}: $\alpha = 1$ (by suitable choice)
	\item \textbf{Gaussian units}: $\alpha = \frac{e^2}{\hbar c} \approx 1/137$
\end{itemize}

\subsection{The System Dependency of}
\label{ParameterSystem:L-ParameterSystemdipendentEn-0751}

The numerical value $\alpha_{\text{EM}} = 1/137$ is \textbf{valid exclusively in the SI system}:

\begin{align}
	\text{SI system:} \quad &\alpha_{\text{EM}}^{\text{SI}} = \frac{e^2}{4\pi\varepsilon_0\hbar c} \approx \frac{1}{137.036} \\
	\text{Natural system of units:} \quad &\alpha_{\text{EM}}^{\text{nat}} = 1 \text{ (by suitable choice of units)}
\end{align}

\section*{Transformation factor:}
\begin{equation}
	\frac{\alpha_{\text{EM}}^{\text{nat}}}{\alpha_{\text{EM}}^{\text{SI}}} = 137.036
\end{equation}

\subsection{The Natural System of Units with}
\label{ParameterSystem:L-ParameterSystemdipendentEn-0752}

In a natural system of units that respects electromagnetic duality, we obtain:

\begin{itemize}
	\item $\hbar_{\text{nat}} = 1$ (quantum mechanical scale)
	\item $c_{\text{nat}} = 1$ (relativistic scale)
	\item $\varepsilon_{0,\text{nat}} = 1$ (electric constant)
	\item $\mu_{0,\text{nat}} = 1$ (magnetic constant)
	\item $e_{\text{nat}}^2 = 4\pi$ (elementary charge)
\end{itemize}

With these values, $\alpha = 1$ is verified in both the standard form and the dual form:

\begin{equation}
	\alpha = \frac{4\pi}{4\pi \cdot 1 \cdot 1 \cdot 1} = 1
\end{equation}

\subsection{The Resolution of the ``Mystery''}
\label{ParameterSystem:L-ParameterSystemdipendentEn-0753}

The apparent mystification of $1/137$ arises from:

\begin{enumerate}
	\item \textbf{Confusion of two aspects}: The invariance of the ratios is conflated with the system-dependency of the numerical representation.
	
	\item \textbf{Treatment of the SI system as absolute}: The historically evolved SI units (meter, second, kilogram, ampere) force electromagnetic constants to take ``unnatural'' values.
	
	\item \textbf{Forgetting the construction of unit systems}: All unit systems are human constructs. Nature knows no preferred units.
	
	\item \textbf{Search for deeper meaning in conversion factors}: The number 137 has no deeper cosmic significance than, say, the factor 1609.344 between miles and meters.
\end{enumerate}

\subsection{The Anthropic Fallacy}
\label{ParameterSystem:L-ParameterSystemdipendentEn-0754}

Typical anthropic arguments claim:
\begin{itemize}
	\item ``If $\alpha_{\text{EM}} = 1/200$ $\rightarrow$ no atoms $\rightarrow$ no life''
	\item ``If $\alpha_{\text{EM}} = 1/80$ $\rightarrow$ no stars $\rightarrow$ no life''
	\item ``Therefore, $\alpha_{\text{EM}} = 1/137$ is `fine-tuned' for life''
\end{itemize}

\textbf{The problem}: These arguments presuppose the SI system as absolute!

\textbf{In natural units}: $\alpha_{\text{EM}} = 1$ is perfectly natural and requires no fine-tuning whatsoever. The electromagnetic interaction has unit strength in the natural system of units, which respects the fundamental structure of quantum mechanics and relativity.

\subsection{Sommerfeld's Harmonic Imprinting}
\label{ParameterSystem:L-ParameterSystemdipendentEn-0755}

An often overlooked historical aspect: In 1916, Arnold Sommerfeld actively searched for \textbf{harmonic ratios} in atomic spectra, guided by the philosophical conviction that nature follows musical principles.

His methodological approach:
\begin{enumerate}
	\item \textbf{Expectation} of musical ratios in quantum transitions
	\item \textbf{Calibration} of measurement systems to produce harmonic values
	\item \textbf{Definition} of $\alpha_{\text{EM}}$ based on harmonic spectroscopic adjustments
	\item \textbf{Attribution} of the resulting ratio to fundamental physics
\end{enumerate}

The apparent ``harmony'' in $\alpha_{\text{EM}}^{-1} = 137 \approx (6/5)^{27}$ is therefore not a cosmic discovery, but the result of Sommerfeld's harmonic expectations embedded into the definition of the unit system.

\subsection{Physical Interpretation}
\label{ParameterSystem:L-T0_Energie-0291}

In natural units, $\alpha = 1$ represents the perfect balance between:

\begin{itemize}
	\item \textbf{Electric field coupling} (via $\varepsilon_0$ with $c^{-1}$)
	\item \textbf{Magnetic field coupling} (via $\mu_0$ with $c^{+1}$)
	\item \textbf{Quantum mechanical scale} (via $\hbar$)
	\item \textbf{Relativistic scale} (via $c$)
\end{itemize}

The electromagnetic duality $\frac{1}{\varepsilon_0 c} = \mu_0 c$ ensures this perfect balance.

\subsection{Summary: The True Lesson}
\label{ParameterSystem:L-ParameterSystemdipendentEn-0756}

The fine-structure constant teaches us a profound lesson about the nature of physical laws:

\section*{The fundamental relationships of the universe are elegant and simple when expressed in their natural language.}

The apparent complexity and mystery of ``1/137'' are merely artifacts of our historical decision to measure electromagnetic phenomena with units originally defined for mechanical quantities.

The ``fine-tuning problem'' completely dissolves once we recognize:

\begin{itemize}
	\item $\alpha = 1/137$ is not a fundamental number, but a unit conversion factor
	\item $\alpha = 1$ represents the natural strength of the electromagnetic coupling
	\item The apparent ``mystery'' arises from treating arbitrary SI units as absolute
	\item The fundamental relationships of nature are simple in their natural language
\end{itemize}

\subsection{Historical Warning: The Eddington Saga}
\label{ParameterSystem:L-ParameterSystemdipendentEn-0757}

Arthur Eddington (1882-1944) attempted to ``prove'' $\alpha_{\text{EM}} = 1/137$ from first principles and developed elaborate numerological theories. The result was entirely speculative and wrong -- a warning against mystifying system-dependent numbers.

However, modern analysis shows that the fine-structure constant is indeed derivable from fundamental electromagnetic vacuum constants and that $\alpha_{\text{EM}} = 1$ in natural units is not only possible but reveals the arbitrary nature of our choice of unit system.
	\section{The Parameter}
	\label{ParameterSystem:L-ParameterSystemdipendentEn-0758}
	
	\subsection{Empirical vs. Theoretical Values}
	\label{ParameterSystem:L-ParameterSystemdipendentEn-0759}
	
	The $\beta_{\text{T}}$ parameter shows the same system dependency:
	
	\begin{align}
		\betaTSI &\approx 0.008 \text{ (from astrophysical observations)} \\
		\betaTnat &= 1 \text{ (in T0-natural units)}
	\end{align}
	
\section*{Transformation factor:}
	$$\frac{\betaTnat}{\betaTSI} = \frac{1}{0.008} = 125$$
	
	\subsection{Theoretical Foundation from Field Theory}
	\label{ParameterSystem:L-ParameterSystemdipendentEn-0760}
	
	The T0 model establishes $\beta_{\text{T}} = 1$ through the fundamental field-theoretic relationship \cite{pascher_derivation_beta_2025}:
	
	\begin{equation}
		\beta_{\text{T}} = \frac{\lambdah^2 v^2}{16\pichar^3 m_h^2 \xipar} = 1
		\label{ParameterSystem:L-ParameterSystemdipendentEn-0761}
	\end{equation}
	
	This relationship, combined with the Higgs-derived value of $\xipar$, uniquely determines $\beta_{\text{T}} = 1$ in natural units, eliminating any free parameters from the theory.
	
	\subsection{Circularity in SI Determination}
	\label{ParameterSystem:L-ParameterSystemdipendentEn-0762}
	
	The SI value $\betaTSI$ is determined through:
	$$z(\lambda) = z_0\left(1 + \beta_{\text{T}} \ln\frac{\lambda}{\lambda_0}\right)$$
	
	But this involves:
	\begin{itemize}
		\item Hubble constant $H_0$ $\rightarrow$ distance measurements
		\item Distance ladder $\rightarrow$ standard candles
		\item Photometry $\rightarrow$ Planck radiation law $\rightarrow$ fundamental constants
	\end{itemize}
	
\section*{The determination is circular through cosmological parameters!}
	
	\section{The Wien Constant}
	\label{ParameterSystem:L-ParameterSystemdipendentEn-0763}
	
	\subsection{Mathematical vs. Conventional Values}
	\label{ParameterSystem:L-ParameterSystemdipendentEn-0764}
	
	Wien's displacement law gives:
	
	\begin{align}
		\text{SI system:} \quad &\alphaWSI = 2.8977719... \\
		\text{T0 system:} \quad &\alphaWnat = 1
	\end{align}
	
\section*{Transformation factor:}
	$$\frac{\alphaWSI}{\alphaWnat} = 2.898$$
	
	\section{Parameter Comparison Table}
	\label{ParameterSystem:L-ParameterSystemdipendentEn-0765}
	
	\begin{table}[htbp]
		\centering
		\begin{tabular}{lcccc}
			\toprule
			\textbf{Parameter} & \textbf{SI Value} & \textbf{T0-nat Value} & \textbf{Ratio} & \textbf{Factor} \\
			\midrule
			$\xipar$ (electron) & $7.5 \times 10^{-6}$ & $3.1 \times 10^{-2}$ & 4100 & $10^{3.6}$ \\
			$\alpha_{\text{EM}}$ & $7.3 \times 10^{-3}$ & $1$ & 137 & $10^{2.1}$ \\
			$\beta_{\text{T}}$ & $0.008$ & $1$ & 125 & $10^{2.1}$ \\
			$\alpha_{\text{W}}$ & $2.898$ & $1$ & 2.9 & $10^{0.5}$ \\
			\bottomrule
		\end{tabular}
		\caption{Systematic parameter differences between unit systems}
		\label{ParameterSystem:L-T0_Energie-0431}
	\end{table}
	
\section*{All parameters show 0.5-4 orders of magnitude difference between systems!}
	
	\section{Yukawa Parameters: Variable and System-Dependent}
	\label{ParameterSystem:L-ParameterSystemdipendentEn-0766}
	
	\subsection{The Hierarchy of Yukawa Couplings}
	\label{ParameterSystem:L-ParameterSystemdipendentEn-0767}
	
	In the Standard Model, Yukawa couplings vary dramatically:
	
	\begin{table}[htbp]
		\centering
		\begin{tabular}{lc}
			\toprule
			\textbf{Particle} & \textbf{$y_i$ (SI system)} \\
			\midrule
			Electron & $2.94 \times 10^{-6}$ \\
			Muon & $6.09 \times 10^{-4}$ \\
			Tau & $1.03 \times 10^{-2}$ \\
			Up quark & $1.27 \times 10^{-5}$ \\
			Top quark & $1.00$ \\
			Bottom quark & $2.25 \times 10^{-2}$ \\
			\bottomrule
		\end{tabular}
		\caption{Yukawa coupling hierarchy (5 orders of magnitude variation)}
		\label{ParameterSystem:L-ParameterSystemdipendentEn-0768}
	\end{table}
	
	\subsection{Transformation Uncertainty}
	\label{ParameterSystem:L-ParameterSystemdipendentEn-0769}
	
	The transformation of Yukawa parameters between systems requires careful consideration of the Higgs mechanism. The general form would be:
	
	$$y_{i,\text{nat}} = y_{i,\text{SI}} \times T_{\text{Yukawa}}$$
	
	where $T_{\text{Yukawa}}$ depends on the transformation of Higgs vacuum expectation value and particle masses.
	
	\subsection{Consistency Requirements}
	\label{ParameterSystem:L-ParameterSystemdipendentEn-0770}
	
	The Higgs mechanism requires:
	$$m_h^2 = \frac{\lambdah v^2}{2}$$
	
	For transformation consistency:
	$$T_m^2 = T_\lambda \times T_v^2$$
	
	This gives:
	$$y_{i,\text{nat}} = y_{i,\text{SI}} \times \sqrt{T_\lambda}$$
	
\section*{However, $T_\lambda$ requires detailed specification of the T0-natural unit system transformation rules.}
	
	\section{Universal Warning: No Direct Parameter Transfer}
	\label{ParameterSystem:L-ParameterSystemdipendentEn-0771}
	
	\subsection{The Systematic Problem}
	\label{ParameterSystem:L-ParameterSystemdipendentEn-0772}
	
\section*{Warning}
\section*{EVERY parameter symbol in T0-model documents may have different values than in SI system calculations!}
% end box warning
	
\section*{Concrete danger zones:}
	
	\begin{align}
		G_{\text{nat}} &= 1 \quad \text{vs.} \quad G_{\text{SI}} = 6.674 \times 10^{-11} \text{ m}^3/(\text{kg} \cdot \text{s}^2) \\
		\alpha_{\text{EM,nat}} &= 1 \quad \text{vs.} \quad \alpha_{\text{EM,SI}} = 1/137 \\
		e_{\text{nat}} &= 2\sqrt{\pichar} \quad \text{vs.} \quad e_{\text{SI}} = 1.602 \times 10^{-19} \text{ C}
	\end{align}
	
	\textbf{Direct transfer leads to errors of factors $10^2$ to $10^{11}$!}
	
	\subsection{Required Transformation Protocol}
	\label{ParameterSystem:L-ParameterSystemdipendentEn-0773}
	
	For every parameter, explicitly specify:
	
	\begin{enumerate}
		\item \textbf{Which unit system} is being used
		\item \textbf{How transformation occurs} between systems
		\item \textbf{Which factors must be considered}
		\item \textbf{Which consistency conditions} must be satisfied
	\end{enumerate}
	
\section*{Example of complete specification:}
	\subsubsection*{Parameter Specification Template}
\textbf{Parameter:} Fine structure constant $\alpha_{\text{EM}}$ \\
		\textbf{SI value:} $\alphaEMSI = 1/137.036$ \\
		\textbf{T0 value:} $\alphaEMnat = 1$ \\
		\textbf{Transformation:} $\alphaEMnat = \alphaEMSI \times 137.036$ \\
		\textbf{Consistency:} Dimensional analysis verified \\
		\textbf{Usage:} Specify system before calculation

	
	\subsection{Experimental Prediction Guidelines}
	\label{ParameterSystem:L-ParameterSystemdipendentEn-0774}
	
\section*{For QED calculations:}
	\begin{align}
		\text{WRONG:} \quad &\alpha_{\text{EM}} = 1 \text{ from T0-model directly in SI formulas} \\
		\text{CORRECT:} \quad &\alphaEMSI = 1/137 \text{ with transformation to } \alphaEMnat = 1
	\end{align}
	
\section*{For gravitational calculations:}
	\begin{align}
		\text{WRONG:} \quad &G = 1 \text{ from T0-model directly in Newton's formulas} \\
		\text{CORRECT:} \quad &G_{\text{SI}} = 6.674 \times 10^{-11} \text{ with transformation to } G_{\text{nat}} = 1
	\end{align}
	
	\section{The Circularity Resolution}
	\label{ParameterSystem:L-ParameterSystemdipendentEn-0775}
	
	\subsection{Apparent vs. Real Circularity}
	\label{ParameterSystem:L-ParameterSystemdipendentEn-0776}
	
	The circularity problem that seemed to plague T0-model parameter determination is resolved by recognizing:
	
	\begin{enumerate}
		\item \textbf{No real circularity exists} within each consistent system
		\item \textbf{Both SI and T0 systems are internally consistent}
		\item \textbf{The apparent contradiction} arose from comparing parameters across different systems
		\item \textbf{Proper transformation} eliminates all apparent inconsistencies
	\end{enumerate}
	
	\subsection{System Consistency Verification}
	\label{ParameterSystem:L-ParameterSystemdipendentEn-0777}
	
\section*{SI system consistency:}
	$$\Rzero = \frac{m_e c \left(\alphaEMSI\right)^2}{2\hbar} \quad \checkmark \text{ (experimentally verified to 0.000001\%)}$$
	
\section*{T0 system consistency:}
	$$\text{All parameters = 1} \quad \checkmark \text{ (by construction)}$$
	
\section*{Both systems work perfectly within their own frameworks!}
	
	\section{Implications for T0-Model Testing}
	\label{ParameterSystem:L-ParameterSystemdipendentEn-0778}
	
	\subsection{System-Specific Predictions}
	\label{ParameterSystem:L-ParameterSystemdipendentEn-0779}
	
	Experimental tests must clearly specify which parameter system is used:
	
	\begin{table}[htbp]
		\centering
		\begin{tabular}{lcc}
			\toprule
			\textbf{Test Type} & \textbf{SI-based Prediction} & \textbf{T0-based Prediction} \\
			\midrule
			QED anomaly & $a_e \propto \alphaEMSI = 1/137$ & $a_e \propto \alphaEMnat = 1$ \\
			Galaxy rotation & $v^2 \propto \xipar_{\text{SI}} \sim 10^{38}$ & $v^2 \propto \xipar_{\text{nat}} \sim 10^{41}$ \\
			CMB temperature & $T \propto \betaTSI = 0.008$ & $T \propto \betaTnat = 1$ \\
			\bottomrule
		\end{tabular}
		\caption{System-specific experimental predictions}
		\label{ParameterSystem:L-ParameterSystemdipendentEn-0780}
	\end{table}
	
	\subsection{Transformation Validation}
	\label{ParameterSystem:L-ParameterSystemdipendentEn-0781}
	
	The transformation factors can be validated by checking:
	
	\begin{enumerate}
		\item \textbf{Dimensional consistency} in both systems
		\item \textbf{Known limits} are reproduced correctly
		\item \textbf{Ratios remain invariant} between systems
		\item \textbf{Internal consistency} of each system
	\end{enumerate}
	




% Bibliography
\begin{thebibliography}{99}
	
	\bibitem{pdg2024}
	Particle Data Group Collaboration (2024). 
	\textit{Review of Particle Physics}. 
	Progress of Theoretical and Experimental Physics, 2024(8), 083C01.
	\url{https://pdg.lbl.gov}
	
	\bibitem{flag2024}
	Aoki, Y., et al. (FLAG Collaboration) (2024). 
	\textit{FLAG Review 2024 of Lattice Results for Low-Energy Constants}. 
	arXiv:2411.04268.
	\url{https://arxiv.org/abs/2411.04268}
	
	\bibitem{fermilab_muon_g2}
	Abi, B., et al. (Muon g-2 Collaboration) (2021). 
	\textit{Measurement of the Positive Muon Anomalous Magnetic Moment to 0.46 ppm}. 
	Physical Review Letters, 126, 141801.
	
	\bibitem{peskin_schroeder}
	Peskin, M. E., \& Schroeder, D. V. (1995). 
	\textit{An Introduction to Quantum Field Theory}. 
	Addison-Wesley.
	
	\bibitem{weinberg_qft}
	Weinberg, S. (1995). 
	\textit{The Quantum Theory of Fields, Vol. I--III}. 
	Cambridge University Press.
	
	\bibitem{griffiths_particle}
	Griffiths, D. (2008). 
	\textit{Introduction to Elementary Particles}. 
	Wiley-VCH.
	
	\bibitem{mandl_shaw}
	Mandl, F., \& Shaw, G. (2010). 
	\textit{Quantum Field Theory (2nd ed.)}. 
	Wiley.
	
	\bibitem{srednicki_qft}
	Srednicki, M. (2007). 
	\textit{Quantum Field Theory}. 
	Cambridge University Press.
	
	\bibitem{t0_fundamentals}
	Pascher, J. (2024). 
	\textit{T0-Theory: Foundations of Time-Mass Duality}. 
	Unpublished manuscript, HTL Leonding.
	
	\bibitem{t0_fine_structure}
	Pascher, J. (2024). 
	\textit{T0-Theory: The Fine Structure Constant}. 
	Unpublished manuscript, HTL Leonding.
	
	\bibitem{t0_neutrinos}
	Pascher, J. (2024). 
	\textit{T0-Theory: Neutrino Masses and PMNS Mixing}. 
	Unpublished manuscript, HTL Leonding.
	
	\bibitem{t0_github}
	Pascher, J. (2024--2025). 
	\textit{T0-Time-Mass-Duality Repository}. 
	GitHub.
	\url{https://github.com/jpascher/T0-Time-Mass-Duality}
	
	\bibitem{lattice_qcd_review}
	Kronfeld, A. S. (2012). 
	\textit{Twenty-first Century Lattice Gauge Theory: Results from the QCD Lagrangian}. 
	Annual Review of Nuclear and Particle Science, 62, 265--284.
	
	\bibitem{neutrino_mixing_pdg}
	Particle Data Group Collaboration (2024). 
	\textit{Neutrino Masses, Mixing, and Oscillations}. 
	PDG Review 2024.
	\url{https://pdg.lbl.gov/2024/reviews/rpp2024-rev-neutrino-mixing.pdf}
	
	\bibitem{higgs_discovery}
	ATLAS and CMS Collaborations (2012). 
	\textit{Observation of a New Particle in the Search for the Standard Model Higgs Boson}. 
	Physics Letters B, 716, 1--29.
	
	\bibitem{Brannen2005}
	C. P. Brannen, ``Estimate of neutrino masses from Koide's relation'', \textit{arXiv:hep-ph/0505028} (2005).
	\url{https://arxiv.org/abs/hep-ph/0505028}
	
	\bibitem{Brannen2006}
	C. P. Brannen, ``Koide Mass Formula for Neutrinos'', \textit{arXiv:0702.0052} (2006).
	\url{http://brannenworks.com/MASSES.pdf}
	
	\bibitem{PhaseVectors2025}
	Anonymous, ``The Koide Relation and Lepton Mass Hierarchy from Phase Vectors'', \textit{rXiv:2507.0040} (2025).
	\url{https://rxiv.org/pdf/2507.0040v1.pdf}
	
	\bibitem{PDG2025}
	Particle Data Group, ``Review of Particle Physics'', \textit{Phys. Rev. D} \textbf{112} (2025) 030001.
	\url{https://pdg.lbl.gov/2025/}
	
	\bibitem{terrell2024}
	Terrell et al. (2024). 
	\textit{Single-Clock Metrology in Nature}. 
	Nature Physics.
	
	\bibitem{hossenfelder2024}
	Hossenfelder, S. (2024). 
	\textit{Single Clock Video Explanation}. 
	YouTube.
	
	\bibitem{hundert1931}
	Hundert (1931). 
	\textit{Reference Work}. 
	Publisher.
	
	\bibitem{terrell2025}
	Terrell et al. (2025). 
	\textit{Advanced Clock Synchronization Methods}. 
	Physical Review Letters.
	
	\bibitem{pascher_t0_2025}
	Pascher, J. (2025). 
	\textit{T0-Theory: Complete Framework and Applications}. 
	Unpublished manuscript, HTL Leonding.
	
	\bibitem{t0qm}
	Pascher, J. (2024). 
	\textit{T0-Theory: Quantum Mechanics Formulation}. 
	Unpublished manuscript, HTL Leonding.
	
	\bibitem{t0anomale}
	Pascher, J. (2024). 
	\textit{T0-Theory: Anomalous Magnetic Moments}. 
	Unpublished manuscript, HTL Leonding.
	
	\bibitem{muong2complete}
	Abi, B., et al. (Muon g-2 Collaboration) (2023). 
	\textit{Complete Measurement of the Positive Muon Anomalous Magnetic Moment}. 
	Physical Review Letters, 131, 161802.
	
	\bibitem{penrose2004}
	Penrose, R. (2004). 
	\textit{The Road to Reality: A Complete Guide to the Laws of the Universe}. 
	Jonathan Cape.
	
	\bibitem{planck1900}
	Planck, M. (1900). 
	\textit{On the Theory of the Energy Distribution Law of the Normal Spectrum}. 
	Verhandlungen der Deutschen Physikalischen Gesellschaft, 2, 237.
	
	\bibitem{T0Theory}
	Pascher, J. (2024). 
	\textit{T0-Theory: Fundamental Principles}. 
	Unpublished manuscript, HTL Leonding.
	
	% Additional bibliography entries for all undefined citations
	\bibitem{6g_roadmap}
	6G Research Consortium (2024).
	\textit{6G Technology Roadmap}.
	Technical Report.
	
	\bibitem{Born2013}
	Born, M. (2013).
	\textit{Einstein's Theory of Relativity}.
	Dover Publications.
	
	\bibitem{Casimir1948}
	Casimir, H. B. G. (1948).
	\textit{On the attraction between two perfectly conducting plates}.
	Proc. Kon. Ned. Akad. Wetensch. B51, 793--795.
	
	\bibitem{Einstein1905}
	Einstein, A. (1905).
	\textit{On the Electrodynamics of Moving Bodies}.
	Annalen der Physik, 17, 891--921.
	
	\bibitem{Feynman2006}
	Feynman, R. P. (2006).
	\textit{QED: The Strange Theory of Light and Matter}.
	Princeton University Press.
	
	\bibitem{Griffiths2017}
	Griffiths, D. J. (2017).
	\textit{Introduction to Electrodynamics (4th ed.)}.
	Cambridge University Press.
	
	\bibitem{Jackson1999}
	Jackson, J. D. (1999).
	\textit{Classical Electrodynamics (3rd ed.)}.
	Wiley.
	
	\bibitem{Mohr2016}
	Mohr, P. J., et al. (2016).
	\textit{CODATA Recommended Values of the Fundamental Physical Constants: 2014}.
	Rev. Mod. Phys. 88, 035009.
	
	\bibitem{Parker2018}
	Parker, R. H., et al. (2018).
	\textit{Measurement of the fine-structure constant as a test of the Standard Model}.
	Science, 360, 191--195.
	
	\bibitem{Planck1900}
	Planck, M. (1900).
	\textit{On the Theory of the Energy Distribution Law of the Normal Spectrum}.
	Verhandlungen der Deutschen Physikalischen Gesellschaft, 2, 237.
	
	\bibitem{Planck2018}
	Planck Collaboration (2018).
	\textit{Planck 2018 results. VI. Cosmological parameters}.
	Astronomy \& Astrophysics, 641, A6.
	
	\bibitem{QFT_T0}
	Pascher, J. (2024).
	\textit{T0-Theory and QFT Connections}.
	Unpublished manuscript, HTL Leonding.
	
	\bibitem{Sommerfeld1916}
	Sommerfeld, A. (1916).
	\textit{On the Quantum Theory of Spectral Lines}.
	Annalen der Physik, 51, 1--94.
	
	\bibitem{T0_Feinstruktur}
	Pascher, J. (2024).
	\textit{T0-Theory: Fine Structure Analysis}.
	Unpublished manuscript, HTL Leonding.
	
	\bibitem{T0_SI}
	Pascher, J. (2024).
	\textit{T0-Theory and SI Units}.
	Unpublished manuscript, HTL Leonding.
	
	\bibitem{T0_fine_structure}
	Pascher, J. (2024).
	\textit{T0-Theory: The Fine Structure Constant}.
	Unpublished manuscript, HTL Leonding.
	
	\bibitem{T0_g2_erweiterung}
	Pascher, J. (2024).
	\textit{T0-Theory: g-2 Extensions}.
	Unpublished manuscript, HTL Leonding.
	
	\bibitem{T0_gravitational_constant}
	Pascher, J. (2024).
	\textit{T0-Theory: Gravitational Constant Derivation}.
	Unpublished manuscript, HTL Leonding.
	
	\bibitem{T0_netze_en}
	Pascher, J. (2024).
	\textit{T0-Theory: Network Structures}.
	Unpublished manuscript, HTL Leonding.
	
	\bibitem{T0_tm_erweiterung}
	Pascher, J. (2024).
	\textit{T0-Theory: Time-Mass Extensions}.
	Unpublished manuscript, HTL Leonding.
	
	\bibitem{Uzan2003}
	Uzan, J.-P. (2003).
	\textit{The fundamental constants and their variation}.
	Rev. Mod. Phys. 75, 403--455.
	
	\bibitem{Weinberg1995}
	Weinberg, S. (1995).
	\textit{The Quantum Theory of Fields, Vol. I}.
	Cambridge University Press.
	
	\bibitem{albrecht1999}
	Albrecht, A. \& Magueijo, J. (1999).
	\textit{A time varying speed of light as a solution to cosmological puzzles}.
	Phys. Rev. D 59, 043516.
	
	\bibitem{alice2023}
	ALICE Collaboration (2023).
	\textit{Recent results from ALICE}.
	CERN-EP-2023-XXX.
	
	\bibitem{analog_optical}
	Smith, J. et al. (2024).
	\textit{Analog optical computing systems}.
	Nature Photonics.
	
	\bibitem{ashtekar2004}
	Ashtekar, A. \& Lewandowski, J. (2004).
	\textit{Background independent quantum gravity}.
	Class. Quantum Grav. 21, R53.
	
	\bibitem{atlas2023}
	ATLAS Collaboration (2023).
	\textit{ATLAS physics results}.
	CERN-PH-EP-2023-XXX.
	
	\bibitem{atlas2023higgs}
	ATLAS Collaboration (2023).
	\textit{Higgs boson measurements}.
	Phys. Rev. Lett.
	
	\bibitem{barbour1999}
	Barbour, J. (1999).
	\textit{The End of Time}.
	Oxford University Press.
	
	\bibitem{barrow1999}
	Barrow, J. D. (1999).
	\textit{Cosmologies with varying light speed}.
	Phys. Rev. D 59, 043515.
	
	\bibitem{becker2007}
	Becker, K. et al. (2007).
	\textit{String Theory and M-Theory}.
	Cambridge University Press.
	
	\bibitem{bell_muon}
	Bennett, G. W., et al. (Muon g-2 Collaboration) (2006).
	\textit{Final report of the E821 muon anomalous magnetic moment measurement}.
	Phys. Rev. D 73, 072003.
	
	\bibitem{bondi1948}
	Bondi, H. \& Gold, T. (1948).
	\textit{The steady-state theory of the expanding universe}.
	Mon. Not. R. Astron. Soc. 108, 252--270.
	
	\bibitem{brewer2019}
	Brewer, S. M. et al. (2019).
	\textit{Al+ Quantum-Logic Clock with Systematic Uncertainty below $10^{-18}$}.
	Phys. Rev. Lett. 123, 033201.
	
	\bibitem{cms2023top}
	CMS Collaboration (2023).
	\textit{Top quark measurements at CMS}.
	JHEP 2023.
	
	\bibitem{cms2024}
	CMS Collaboration (2024).
	\textit{CMS physics results 2024}.
	CERN-PH-EP-2024-XXX.
	
	\bibitem{codata2019}
	Tiesinga, E. et al. (2019).
	\textit{The 2018 CODATA Recommended Values}.
	J. Phys. Chem. Ref. Data.
	
	\bibitem{desi2025}
	DESI Collaboration (2025).
	\textit{DESI 2025 Cosmology Results}.
	arXiv preprint.
	
	\bibitem{differential_optical}
	Wang, X. et al. (2024).
	\textit{Differential optical computing}.
	Optica.
	
	\bibitem{dingle1972}
	Dingle, H. (1972).
	\textit{Science at the Crossroads}.
	Martin Brian \& O'Keeffe.
	
	\bibitem{divalentino2021}
	Di Valentino, E. et al. (2021).
	\textit{In the realm of the Hubble tension}.
	Class. Quantum Grav. 38, 153001.
	
	\bibitem{elnaschie2004}
	El Naschie, M. S. (2004).
	\textit{A review of E infinity theory}.
	Chaos, Solitons \& Fractals, 19, 209--236.
	
	\bibitem{fabrication_heterogeneous}
	Chen, Y. et al. (2024).
	\textit{Heterogeneous photonic integration}.
	Nature Electronics.
	
	\bibitem{fermilab2023}
	Fermilab (2023).
	\textit{Muon g-2 results}.
	Phys. Rev. Lett.
	
	\bibitem{flexible_wafer}
	Kim, S. et al. (2024).
	\textit{Flexible wafer-scale photonics}.
	Science Advances.
	
	\bibitem{francesco1997}
	Di Francesco, P. et al. (1997).
	\textit{Conformal Field Theory}.
	Springer.
	
	\bibitem{hartree1957}
	Hartree, D. R. (1957).
	\textit{The Calculation of Atomic Structures}.
	Wiley.
	
	\bibitem{hhi_6g}
	Fraunhofer HHI (2024).
	\textit{6G Photonic Integration}.
	Technical Report.
	
	\bibitem{hossenfelder2025}
	Hossenfelder, S. (2025).
	\textit{Science without the gobbledygook}.
	YouTube/Blog.
	
	\bibitem{hossenfelder_single_clock_video}
	Hossenfelder, S. (2024).
	\textit{The Single Clock Problem}.
	YouTube.
	
	\bibitem{hoyle1948}
	Hoyle, F. (1948).
	\textit{A new model for the expanding universe}.
	Mon. Not. R. Astron. Soc. 108, 372--382.
	
	\bibitem{integration_microelectronic}
	Liu, A. et al. (2024).
	\textit{Microelectronic photonic integration}.
	IEEE Journal.
	
	\bibitem{jacobson1995}
	Jacobson, T. (1995).
	\textit{Thermodynamics of spacetime}.
	Phys. Rev. Lett. 75, 1260.
	
	\bibitem{kasevich2023}
	Kasevich, M. et al. (2023).
	\textit{Atom interferometry tests}.
	Nature Physics.
	
	\bibitem{lerner2014}
	Lerner, E. J. (2014).
	\textit{An open letter on cosmology}.
	New Scientist.
	
	\bibitem{lisa2017}
	LISA Consortium (2017).
	\textit{Laser Interferometer Space Antenna}.
	ESA Technical Report.
	
	\bibitem{lithium_tantalate}
	Zhang, M. et al. (2024).
	\textit{Thin-film lithium tantalate photonics}.
	Nature Photonics.
	
	\bibitem{lopez2010}
	Lopez-Corredoira, M. (2010).
	\textit{Tests and problems of the standard model in cosmology}.
	Int. J. Mod. Phys. D.
	
	\bibitem{ludlow2015}
	Ludlow, A. D. et al. (2015).
	\textit{Optical atomic clocks}.
	Rev. Mod. Phys. 87, 637.
	
	\bibitem{mach1883}
	Mach, E. (1883).
	\textit{Die Mechanik in ihrer Entwickelung}.
	F.A. Brockhaus.
	
	\bibitem{maldacena1998}
	Maldacena, J. (1998).
	\textit{The large N limit of superconformal field theories}.
	Adv. Theor. Math. Phys. 2, 231--252.
	
	\bibitem{mueller2014}
	Müller, H. et al. (2014).
	\textit{Atom interferometry tests of the gravitational redshift}.
	Phys. Rev. Lett.
	
	\bibitem{mug2_final_2025}
	Muon g-2 Collaboration (2025).
	\textit{Final muon g-2 measurement}.
	Phys. Rev. Lett.
	
	\bibitem{muong2_2023}
	Muon g-2 Collaboration (2023).
	\textit{Updated muon g-2 results}.
	Phys. Rev. Lett.
	
	\bibitem{nathan2024}
	Nathan, A. et al. (2024).
	\textit{Quantum computing advances}.
	Nature.
	
	\bibitem{newell2018}
	Newell, D. B. et al. (2018).
	\textit{The CODATA 2017 values of h, e, k, and $N_A$}.
	Metrologia 55, L13.
	
	\bibitem{nottale1993}
	Nottale, L. (1993).
	\textit{Fractal Space-Time and Microphysics}.
	World Scientific.
	
	\bibitem{on_chip_lithium}
	Wang, C. et al. (2024).
	\textit{On-chip lithium niobate photonics}.
	Nature Communications.
	
	\bibitem{optical_advantages}
	Shastri, B. J. et al. (2024).
	\textit{Advantages of optical computing}.
	Nature Reviews Physics.
	
	\bibitem{pascher2025cmb}
	Pascher, J. (2025).
	\textit{T0-Theory: CMB Analysis}.
	Unpublished manuscript, HTL Leonding.
	
	\bibitem{pascher2025g2}
	Pascher, J. (2025).
	\textit{T0-Theory: g-2 Predictions}.
	Unpublished manuscript, HTL Leonding.
	
	\bibitem{pascher2025qm}
	Pascher, J. (2025).
	\textit{T0-Theory: Quantum Mechanics}.
	Unpublished manuscript, HTL Leonding.
	
	\bibitem{pascher2025si}
	Pascher, J. (2025).
	\textit{T0-Theory: SI Unit System}.
	Unpublished manuscript, HTL Leonding.
	
	\bibitem{pascher2025t0}
	Pascher, J. (2025).
	\textit{T0-Theory: Complete Framework}.
	Unpublished manuscript, HTL Leonding.
	
	\bibitem{pascher:fundamentals}
	Pascher, J. (2024).
	\textit{T0-Theory: Fundamentals}.
	Unpublished manuscript, HTL Leonding.
	
	\bibitem{pascher:g2_rev9}
	Pascher, J. (2024).
	\textit{T0-Theory: g-2 Revision 9}.
	Unpublished manuscript, HTL Leonding.
	
	\bibitem{pascher:geometric_formalism}
	Pascher, J. (2024).
	\textit{T0-Theory: Geometric Formalism}.
	Unpublished manuscript, HTL Leonding.
	
	\bibitem{pascher:ml_addendum}
	Pascher, J. (2024).
	\textit{T0-Theory: Machine Learning Addendum}.
	Unpublished manuscript, HTL Leonding.
	
	\bibitem{pascher:t0_foundations}
	Pascher, J. (2024).
	\textit{T0-Theory: Foundations}.
	Unpublished manuscript, HTL Leonding.
	
	\bibitem{pascher_derivation_beta_2025}
	Pascher, J. (2025).
	\textit{T0-Theory: Derivation of Beta}.
	Unpublished manuscript, HTL Leonding.
	
	\bibitem{pascher_higgs_connection_2025}
	Pascher, J. (2025).
	\textit{T0-Theory: Higgs Connection}.
	Unpublished manuscript, HTL Leonding.
	
	\bibitem{pascher_lagrangian_extended_2025}
	Pascher, J. (2025).
	\textit{T0-Theory: Extended Lagrangian}.
	Unpublished manuscript, HTL Leonding.
	
	\bibitem{pascher_mathematical_structure_2025}
	Pascher, J. (2025).
	\textit{T0-Theory: Mathematical Structure}.
	Unpublished manuscript, HTL Leonding.
	
	\bibitem{pascher_t0_cmb_2025}
	Pascher, J. (2025).
	\textit{T0-Theory: CMB Predictions}.
	Unpublished manuscript, HTL Leonding.
	
	\bibitem{pascher_t0_energie_2025}
	Pascher, J. (2025).
	\textit{T0-Theory: Energy}.
	Unpublished manuscript, HTL Leonding.
	
	\bibitem{pascher_t0_energy_2025}
	Pascher, J. (2025).
	\textit{T0-Theory: Energy Framework}.
	Unpublished manuscript, HTL Leonding.
	
	\bibitem{pascher_t0_theory_2025}
	Pascher, J. (2025).
	\textit{T0-Theory: Complete Theory}.
	Unpublished manuscript, HTL Leonding.
	
	\bibitem{penrose1959}
	Penrose, R. (1959).
	\textit{The apparent shape of a relativistically moving sphere}.
	Proc. Cambridge Phil. Soc. 55, 137--139.
	
	\bibitem{penrose1967}
	Penrose, R. (1967).
	\textit{Twistor algebra}.
	J. Math. Phys. 8, 345--366.
	
	\bibitem{peratt1992}
	Peratt, A. L. (1992).
	\textit{Physics of the Plasma Universe}.
	Springer-Verlag.
	
	\bibitem{peskin1995}
	Peskin, M. E. \& Schroeder, D. V. (1995).
	\textit{An Introduction to Quantum Field Theory}.
	Addison-Wesley.
	
	\bibitem{peskin_schroeder_1995}
	Peskin, M. E. \& Schroeder, D. V. (1995).
	\textit{An Introduction to Quantum Field Theory}.
	Addison-Wesley.
	
	\bibitem{phoquant}
	PhoQuant (2024).
	\textit{Photonic quantum computing}.
	Technical Report.
	
	\bibitem{photonics_ai}
	Wetzstein, G. et al. (2024).
	\textit{Photonics for AI}.
	Nature.
	
	\bibitem{planck1906}
	Planck, M. (1906).
	\textit{The Theory of Heat Radiation}.
	Johann Ambrosius Barth.
	
	\bibitem{planck2018}
	Planck Collaboration (2018).
	\textit{Planck 2018 results}.
	A\&A 641, A6.
	
	\bibitem{polchinski1998}
	Polchinski, J. (1998).
	\textit{String Theory}.
	Cambridge University Press.
	
	\bibitem{qant_nps}
	QANT (2024).
	\textit{Quantum photonics systems}.
	Technical Report.
	
	\bibitem{quantenjahr25}
	Quantenjahr (2025).
	\textit{International Year of Quantum}.
	UNESCO.
	
	\bibitem{recurrent_photonics}
	Tait, A. N. et al. (2024).
	\textit{Recurrent photonic neural networks}.
	Optica.
	
	\bibitem{rf_photonics}
	Capmany, J. \& Novak, D. (2024).
	\textit{Microwave photonics}.
	Nature Photonics.
	
	\bibitem{riess2019}
	Riess, A. G. et al. (2019).
	\textit{Large Magellanic Cloud Cepheid Standards}.
	ApJ 876, 85.
	
	\bibitem{riess2022}
	Riess, A. G. et al. (2022).
	\textit{A Comprehensive Measurement of H0}.
	ApJ 934, L7.
	
	\bibitem{rovelli2004}
	Rovelli, C. (2004).
	\textit{Quantum Gravity}.
	Cambridge University Press.
	
	\bibitem{sciama1953}
	Sciama, D. W. (1953).
	\textit{On the origin of inertia}.
	Mon. Not. R. Astron. Soc. 113, 34--42.
	
	\bibitem{sciencedaily2025}
	ScienceDaily (2025).
	\textit{Physics news}.
	Online.
	
	\bibitem{sm_g2_2025}
	Aoyama, T. et al. (2025).
	\textit{Standard Model prediction for g-2}.
	Phys. Rep.
	
	\bibitem{susskind1995}
	Susskind, L. (1995).
	\textit{The world as a hologram}.
	J. Math. Phys. 36, 6377--6396.
	
	\bibitem{t0_kosmologie}
	Pascher, J. (2024).
	\textit{T0-Theory: Cosmology}.
	Unpublished manuscript, HTL Leonding.
	
	\bibitem{terrell1959}
	Terrell, J. (1959).
	\textit{Invisibility of the Lorentz contraction}.
	Phys. Rev. 116, 1041--1045.
	
	\bibitem{terrell_single_clock_nature_2024}
	Terrell, J. et al. (2024).
	\textit{Single clock precision measurements}.
	Nature Physics.
	
	\bibitem{tfln_foundry}
	TFLN Foundry (2024).
	\textit{Thin-film lithium niobate foundry services}.
	Technical Specifications.
	
	\bibitem{thiemann2007}
	Thiemann, T. (2007).
	\textit{Modern Canonical Quantum General Relativity}.
	Cambridge University Press.
	
	\bibitem{thz_epfl}
	EPFL (2024).
	\textit{Terahertz photonics research}.
	Technical Report.
	
	\bibitem{unnikrishnan2004}
	Unnikrishnan, C. S. (2004).
	\textit{On Einstein's resolution of the twin clock paradox}.
	Current Science, 86, 704--709.
	
	\bibitem{verlinde2011}
	Verlinde, E. (2011).
	\textit{On the origin of gravity and the laws of Newton}.
	JHEP 2011, 29.
	
	\bibitem{video2025}
	Video (2025).
	\textit{Physics video explanation}.
	YouTube.
	
	\bibitem{weinberg1995}
	Weinberg, S. (1995).
	\textit{The Quantum Theory of Fields}.
	Cambridge University Press.
	
	\bibitem{weiskopf2000}
	Weiskopf, D. (2000).
	\textit{Visualization of special relativity}.
	PhD thesis, University of Tübingen.
	
	\bibitem{wheeler1990}
	Wheeler, J. A. (1990).
	\textit{A Journey into Gravity and Spacetime}.
	Scientific American Library.
	
	\bibitem{wiki_bell}
	Wikipedia (2024).
	\textit{Bell's theorem}.
	Online encyclopedia.
	
	\bibitem{zwicky1929}
	Zwicky, F. (1929).
	\textit{On the red shift of spectral lines through interstellar space}.
	Proc. Natl. Acad. Sci. 15, 773--779.

\end{thebibliography}


\end{document}
