\documentclass[11pt,a4paper]{article}
\usepackage[a4paper,margin=2cm]{geometry}
\usepackage[utf8]{inputenc}
\usepackage[english]{babel}
\usepackage{lmodern}
\renewcommand{\familydefault}{\sfdefault}

\usepackage{amsmath,amssymb,amsthm}
\usepackage{graphicx}
\usepackage[unicode,pdfencoding=auto,hypertexnames=false]{hyperref}
\usepackage{booktabs}
\usepackage{longtable}
\usepackage{array}
\usepackage{siunitx}
\usepackage{fancyhdr}
\usepackage{float}
\usepackage{tikz}
% tcolorbox removed for standalone
% tcbset removed
\tikzset{
  t0blue/.style={draw=blue,fill=blue!10},
  t0red/.style={draw=red,fill=red!10},
  t0green/.style={draw=green!50!black,fill=green!10},
  t0orange/.style={draw=orange,fill=orange!10},
}
\usepackage{setspace}
\usepackage{enumitem}
\usepackage{adjustbox}
\usepackage{xcolor}

% Define colors for xcolor package
\definecolor{t0green}{RGB}{34,139,34}
\definecolor{t0blue}{RGB}{0,0,255}
\definecolor{t0red}{RGB}{255,0,0}
\definecolor{t0orange}{RGB}{255,165,0}

% Define custom column types for tables
\newcolumntype{L}[1]{>{\raggedright\arraybackslash}p{#1}}
\newcolumntype{C}[1]{>{\centering\arraybackslash}p{#1}}
\newcolumntype{R}[1]{>{\raggedleft\arraybackslash}p{#1}}

\setlength{\parindent}{0pt}
\setlength{\parskip}{6pt}

\hypersetup{
  colorlinks=true,
  linkcolor=blue,
  citecolor=blue,
  urlcolor=blue
}
\pagestyle{fancy}
\setlength{\headheight}{28pt}

\newcommand{\checkmarkx}{\checkmark}
\newcommand{\warningx}{\textbf{!}}

% Makros aus Einzel-Dokumenten (Fallback-Definitionen)
\newcommand{\mytimes}{\times}
\newcommand{\myapprox}{\approx}
\newcommand{\mysim}{\sim}
\newcommand{\myomega}{\omega}
\newcommand{\mypi}{\pi}
\newcommand{\myrightarrow}{\rightarrow}
\newcommand{\mypropto}{\propto}
\newcommand{\deltafield}{\delta\phi}
\newcommand{\xipar}{\xi}
\newcommand{\xiT}{\xi}
\newcommand{\lambdah}{\lambda_h}

% Additional macros used in chapter files
\newcommand{\Kfrak}{K_{\text{frak}}}  % Fractal correction factor
\newcommand{\Dfrak}{D_f}              % Fractal dimension
\newcommand{\betapar}{\beta}          % T0 beta parameter
\newcommand{\alphapar}{\alpha}        % T0 alpha parameter
\newcommand{\Efield}{E}               % Energy field
% Note: checkmarkxa/warningxa are variants used in auto-generated chapter files
\newcommand{\checkmarkxa}{\checkmark}
\newcommand{\warningxa}{\textbf{!}}

% Additional T0-specific macros
\newcommand{\xigeom}{\xi_{\text{geom}}}  % Geometric xi
\newcommand{\lP}{\ell_P}                  % Planck length
\newcommand{\rzero}{r_0}                  % Characteristic radius
\newcommand{\xirat}{\xi_{\text{rat}}}     % Xi ratio
\newcommand{\tzero}{t_0}                  % Characteristic time
\newcommand{\natunits}{\text{(nat. units)}}  % Natural units annotation
\newcommand{\myRightarrow}{\Rightarrow}   % Arrow variant
\newcommand{\Lag}{\mathcal{L}}            % Lagrangian

% Physics macros used in chapter files
\newcommand{\CQCD}{C_{\text{QCD}}}        % QCD correction
\newcommand{\EP}{E_P}                     % Planck energy
\newcommand{\Ee}{E_e}                     % Electron energy
\newcommand{\Emu}{E_\mu}                  % Muon energy
\newcommand{\Exi}{E_\xi}                  % Xi energy
\newcommand{\Ezero}{E_0}                  % Characteristic energy
\newcommand{\Hubble}{H}                   % Hubble constant
\newcommand{\Kspec}{K_{\text{spec}}}      % Spectral correction
\newcommand{\Lambdat}{\Lambda_t}          % Time-related cosmological constant
\newcommand{\Leff}{\mathcal{L}_{\text{eff}}}  % Effective Lagrangian
\newcommand{\Lorentz}{\mathcal{L}}        % Lorentz symbol
\newcommand{\Lxi}{L_\xi}                  % Xi length
\newcommand{\Tfield}{T}                   % Time field
\newcommand{\Weyl}{W}                     % Weyl tensor/symbol
\newcommand{\alphaEMSI}{\alpha_{\text{EM,SI}}}  % EM alpha in SI
\newcommand{\alphaEMnat}{\alpha_{\text{EM,nat}}}  % EM alpha in natural units
\newcommand{\alphaem}{\alpha_{\text{em}}} % Electromagnetic alpha
\newcommand{\betaTSI}{\beta_{T,\text{SI}}}  % Beta in SI
\newcommand{\betaTnat}{\beta_{T,\text{nat}}}  % Beta in natural units
\newcommand{\deltam}{\delta m}            % Mass difference
\newcommand{\phiT}{\phi_T}                % T-field phi
\newcommand{\tP}{t_P}                     % Planck time
\newcommand{\rhoCMB}{\rho_{\text{CMB}}}   % CMB density
\newcommand{\rhoCasimir}{\rho_{\text{Casimir}}}  % Casimir density

% Table formatting
\usepackage{multirow}

% Additional physics macros
\newcommand{\Riem}{\mathcal{R}}           % Riemann tensor
\newcommand{\ZPinch}{Z_{\text{pinch}}}    % Z-pinch
\newcommand{\SynchPower}{P_{\text{synch}}} % Synchrotron power
\newcommand{\Rzero}{R_0}                  % Characteristic radius
\newcommand{\alphafine}{\alpha}           % Fine structure constant
\newcommand{\Etau}{E_\tau}                % Tau energy
\newcommand{\deltaE}{\delta E}            % Energy deviation
\newcommand{\EPlanck}{E_P}                % Planck energy
\newcommand{\pichar}{\pi}                 % Pi character
\newcommand{\alphaWSI}{\alpha_{W,\text{SI}}}  % Wien alpha in SI
\newcommand{\alphaWnat}{\alpha_{W,\text{nat}}}  % Wien alpha in natural units

% Einfache abstract-Umgebung für Kapitel:
\newenvironment{abstract}{%
  \begin{center}\bfseries Abstract\end{center}\small
}{\par}


\title{T0 Anomale Magnetische Momente En}
\author{J. Pascher}
\date{\today}

\begin{document}
\maketitle

\section*{T0 Anomale Magnetische Momente (T0 Anomale Magnetische Momente)}

	\begin{abstract}
		The Fermilab measurements of the muon's anomalous magnetic moment show a significant deviation from the Standard Model, indicating new physics beyond the established framework. While the original discrepancy of $4.2\sigma$ ($\Delta a_\mu = 251 \times 10^{-11}$) has been reduced to approximately $0.6\sigma$ ($\Delta a_\mu = 37 \times 10^{-11}$) through improved Lattice-QCD calculations, the need for a fundamental explanation remains. This work presents a complete theoretical derivation of an extension to the Standard Lagrangian density through a fundamental time field $\Delta m(x,t)$ that couples mass-proportionally with leptons. Based on the T0 time-mass duality $T \cdot m = 1$, we derive a \textbf{fundamental formula} for the additional contribution to the anomalous magnetic moment: $\Delta a_\ell^{\text{T0}} = \frac{5\xi^4}{96\pi^2\lambda^2} \cdot m_\ell^2$. This derivation requires \textbf{no calibration} and consistently explains both experimental situations.
	\end{abstract}
	
	\section{Introduction}
	
	\subsection{The Muon g-2 Problem: Evolution of the Experimental Situation}
	
	The anomalous magnetic moment of leptons, defined as
	\begin{equation}
		a_\ell = \frac{g_\ell - 2}{2}
	\end{equation}
	represents one of the most precise tests of the Standard Model (SM). The experimental situation has evolved significantly in recent years:
	
	\paragraph{Original Discrepancy (2021):}
	\begin{align}
		a_\mu^{\text{exp}} &= 116\,592\,089(63) \times 10^{-11}\\
		a_\mu^{\text{SM}} &= 116\,591\,810(43) \times 10^{-11}\\
		\Delta a_\mu &= 251(59) \times 10^{-11} \quad (4.2\sigma) \label{T0_Anomale_Magn:L-T0_Anomale_Magnetische_Momente-0477}
	\end{align}
	
	\paragraph{Updated Situation (2025):}
	Through improved Lattice-QCD calculations of the hadronic vacuum polarization contribution, the discrepancy has been reduced\cite{sm_g2_2025,mug2_final_2025}:
	\begin{align}
		a_\mu^{\text{exp}} &= 116\,592\,070(14) \times 10^{-11}\\
		a_\mu^{\text{SM}} &= 116\,592\,033(62) \times 10^{-11}\\
		\Delta a_\mu &= 37(64) \times 10^{-11} \quad (0.6\sigma) \label{T0_Anomale_Magn:L-T0_Anomale_Magnetische_Momente-0478}
	\end{align}
	
	Despite the reduced discrepancy, the fundamental question about the origin of the deviation remains and requires new theoretical approaches.
	
\section*{Explanation}
		The reduction of the discrepancy through improved HVP calculations is \textbf{consistent with T0 theory}:
		
		\begin{itemize}
			\item T0 theory predicts an \textbf{independent additional contribution} that adds to the measured $a_\mu^{\text{exp}}$
			\item Improved SM calculations do not affect the T0 contribution, which represents a fundamental extension
			\item The current discrepancy of $37 \times 10^{-11}$ can be explained by \textbf{loop suppression effects} in T0 dynamics
			\item The \textbf{mass-proportional scaling} remains valid in both cases and predicts consistent contributions for electron and tau
		\end{itemize}
		
		T0 theory thus provides a unified framework to explain both experimental situations.
% end box explanation
	
	\subsection{The T0 Time-Mass Duality}
	
	The extension presented here is based on T0 theory\cite{pascher_t0_theory_2025}, which postulates a fundamental duality between time and mass:
	\begin{equation}
		T \cdot m = 1 \quad \text{(in natural units)}
	\end{equation}
	
	This duality leads to a new understanding of spacetime structure, where a time field $\Delta m(x,t)$ appears as a fundamental field component\cite{pascher_lagrangian_extended_2025}.
	
	\section{Theoretical Framework}
	
	\subsection{Standard Lagrangian Density}
	
	The QED component of the Standard Model reads:
	\begin{align}
		\mathcal{L}_{\text{SM}} &= -\tfrac{1}{4} F_{\mu\nu}F^{\mu\nu} + \bar{\psi}(i\gamma^\mu D_\mu - m)\psi \label{T0_Anomale_Magn:L-T0_Anomale_Magnetische_Momente-0479}\\
		F_{\mu\nu} &= \partial_\mu A_\nu - \partial_\nu A_\mu \label{T0_Anomale_Magn:L-T0_Anomale_Magnetische_Momente-0480}\\
		D_\mu &= \partial_\mu + ieA_\mu \label{T0_Anomale_Magn:L-T0_Anomale_Magnetische_Momente-0481}
	\end{align}
	
	\subsection{Introduction of the Time Field}
	
	The fundamental time field $\Delta m(x,t)$ is described by the Klein-Gordon equation:
	\begin{equation}
		\mathcal{L}_{\text{Time}} = \tfrac{1}{2}(\partial_\mu \Delta m)(\partial^\mu \Delta m) - \tfrac{1}{2} m_T^2 \Delta m^2
		\label{T0_Anomale_Magn:L-T0_Anomale_Magnetische_Momente-0482}
	\end{equation}
	
	Here $m_T$ is the characteristic time field mass. The normalization follows from the postulated time-mass duality and the requirement of Lorentz invariance\cite{pascher_mathematical_structure_2025}.
	
	\subsection{Mass-Proportional Interaction}
	
	The coupling of lepton fields $\psi_\ell$ to the time field occurs proportionally to the lepton mass:
	\begin{align}
		\mathcal{L}_{\text{Interaction}} &= g_T^\ell \, \bar{\psi}_\ell \psi_\ell \, \Delta m \label{T0_Anomale_Magn:L-T0_Anomale_Magnetische_Momente-0483}\\
		g_T^\ell &= \xi \, m_\ell \label{T0_Anomale_Magn:L-T0_Anomale_Magnetische_Momente-0484}
	\end{align}
	
	The universal geometric parameter $\xi$ is fundamentally determined by:
	\begin{equation}
		\xi = \frac{4}{3} \times 10^{-4} = 1.333 \times 10^{-4}
		\label{T0_Anomale_Magn:L-T0_Anomale_Magnetische_Momente-0485}
	\end{equation}
	
	\section{Complete Extended Lagrangian Density}
	
	The combined form of the extended Lagrangian density reads:
	\begin{align}
		\mathcal{L}_{\text{extended}} &= -\tfrac{1}{4} F_{\mu\nu}F^{\mu\nu} + \bar{\psi}(i\gamma^\mu D_\mu - m)\psi \nonumber\\
		&\quad + \tfrac{1}{2}(\partial_\mu \Delta m)(\partial^\mu \Delta m) - \tfrac{1}{2} m_T^2 \Delta m^2 \nonumber\\
		&\quad + \xi \, m_\ell \,\bar{\psi}_\ell \psi_\ell \, \Delta m
		\label{T0_Anomale_Magn:L-T0_Anomale_Magnetische_Momente-0486}
	\end{align}
	
	\section{Fundamental Derivation of the T0 Contribution}
	
	\subsection{Starting Point: Interaction Term}
	
	From the interaction term $\mathcal{L}_{\text{int}} = \xi m_\ell \bar{\psi}_\ell \psi_\ell \Delta m$ follows the vertex factor:
	\begin{equation}
		-i g_T^\ell = -i \xi m_\ell
	\end{equation}
	
	\subsection{One-Loop Contribution to the Anomalous Magnetic Moment}
	
	For a scalar mediator coupling to fermions, the general contribution to the anomalous magnetic moment is given by\cite{peskin_schroeder_1995}:
	\begin{equation}
		\Delta a_\ell = \frac{(g_T^\ell)^2}{8\pi^2} \int_0^1 dx \frac{m_\ell^2 (1-x)(1-x^2)}{m_\ell^2 x^2 + m_T^2 (1-x)}
		\label{T0_Anomale_Magn:L-T0_Anomale_Magnetische_Momente-0487}
	\end{equation}
	
	\subsection{Heavy Mediator Limit}
	
	In the physically relevant limit $m_T \gg m_\ell$, the integral simplifies:
	\begin{align}
		\Delta a_\ell &\approx \frac{(g_T^\ell)^2}{8\pi^2 m_T^2} \int_0^1 dx \, (1-x)(1-x^2) \label{T0_Anomale_Magn:L-T0_Anomale_Magnetische_Momente-0488}\\
		&= \frac{(\xi m_\ell)^2}{8\pi^2 m_T^2} \cdot \frac{5}{12} = \frac{5\xi^2 m_\ell^2}{96\pi^2 m_T^2}
	\end{align}
	
	where the integral is calculated exactly:
	\[
	\int_0^1 (1-x)(1-x^2) dx = \int_0^1 (1 - x - x^2 + x^3) dx = \left[x - \frac{x^2}{2} - \frac{x^3}{3} + \frac{x^4}{4}\right]_0^1 = \frac{5}{12}
	\]
	
	\subsection{Time Field Mass from Higgs Connection}
	
	The time field mass is determined through a connection to the Higgs mechanism\cite{pascher_higgs_connection_2025}:
	\begin{equation}
		m_T = \frac{\lambda}{\xi} \quad \text{with} \quad \lambda = \frac{\lambda_h^2 v^2}{16\pi^3}
		\label{T0_Anomale_Magn:L-T0_Anomale_Magnetische_Momente-0489}
	\end{equation}
	
	Substituting into Equation \eqref{L-T0_Anomale_Magnetische_Momente-0488} yields the fundamental T0 formula:
	\begin{equation}
		\Delta a_\ell^{\text{T0}} = \frac{5\xi^4}{96\pi^2\lambda^2} \cdot m_\ell^2
		\label{T0_Anomale_Magn:L-T0_Anomale_Magnetische_Momente-0490}
	\end{equation}
	
	\subsection{Normalization and Parameter Determination}
	
\section*{Derivation}
		
\section*{1. Geometric Parameter:}
		\[
		\xi = \frac{4}{3} \times 10^{-4} = 1.333 \times 10^{-4}
		\]
		
\section*{2. Higgs Parameters:}
		\begin{align*}
			\lambda_h &= 0.13 \quad \text{(Higgs self-coupling)}\\
			v &= 246 \ \text{GeV} = 2.46 \times 10^5 \ \text{MeV}\\
			\lambda &= \frac{\lambda_h^2 v^2}{16\pi^3} = \frac{(0.13)^2 \cdot (2.46 \times 10^5)^2}{16\pi^3}\\
			&= \frac{0.0169 \cdot 6.05 \times 10^{10}}{497.4} = 2.061 \times 10^6 \ \text{MeV}
		\end{align*}
		
\section*{3. Normalization Constant:}
		\[
		K = \frac{5\xi^4}{96\pi^2\lambda^2} = \frac{5 \cdot (1.333 \times 10^{-4})^4}{96\pi^2 \cdot (2.061 \times 10^6)^2} = 3.93 \times 10^{-31} \ \text{MeV}^{-2}
		\]
		
\section*{4. Determination of $\lambda$ from Muon Anomaly:}
		\begin{align*}
			\Delta a_\mu^{\text{T0}} &= K \cdot m_\mu^2 = 251 \times 10^{-11}\\
			\lambda^2 &= \frac{5\xi^4 m_\mu^2}{96\pi^2 \cdot 251 \times 10^{-11}}\\
			&= \frac{5 \cdot (1.333 \times 10^{-4})^4 \cdot 11159.2}{947.0 \cdot 251 \times 10^{-11}} = 7.43 \times 10^{-6}\\
			\lambda &= 2.725 \times 10^{-3} \ \text{MeV}
		\end{align*}
		
\section*{5. Final Normalization Constant:}
		\[
		K = \frac{5\xi^4}{96\pi^2\lambda^2} = 2.246 \times 10^{-13} \ \text{MeV}^{-2}
		\]
% end box derivation
	
	\section{Predictions of T0 Theory}
	
	\subsection{Fundamental T0 Formula}
	
	The completely derived formula for the T0 contribution reads:
	\begin{equation}
		\Delta a_\ell^{\text{T0}} = 2.246 \times 10^{-13} \cdot m_\ell^2
		\label{T0_Anomale_Magn:L-T0_Anomale_Magnetische_Momente-0491}
	\end{equation}
	
\section*{Formula}
\section*{Fundamental T0 Formula:}
		$$\Delta a_\ell^{\text{T0}} = 2.246 \times 10^{-13} \cdot m_\ell^2$$
		
\section*{Detailed Calculations:}
		
\section*{Muon ($m_\mu = 105.658$ MeV):}
		\begin{align}
			m_\mu^2 &= 11159.2 \ \text{MeV}^2\\
			\Delta a_\mu^{\text{T0}} &= 2.246 \times 10^{-13} \cdot 11159.2 = 2.51 \times 10^{-9}
		\end{align}
		
\section*{Electron ($m_e = 0.511$ MeV):}
		\begin{align}
			m_e^2 &= 0.261 \ \text{MeV}^2\\
			\Delta a_e^{\text{T0}} &= 2.246 \times 10^{-13} \cdot 0.261 = 5.86 \times 10^{-14}
		\end{align}
		
\section*{Tau ($m_\tau = 1776.86$ MeV):}
		\begin{align}
			m_\tau^2 &= 3.157 \times 10^6 \ \text{MeV}^2\\
			\Delta a_\tau^{\text{T0}} &= 2.246 \times 10^{-13} \cdot 3.157 \times 10^6 = 7.09 \times 10^{-7}
		\end{align}
% end box formula
	
	\section{Comparison with Experiment}
	
	\subsection*{Muon - Historical Situation (2021)}
	\begin{align}
		\Delta a_\mu^{\text{exp-SM}} &= +2.51(59) \times 10^{-9}\\
		\Delta a_\mu^{\text{T0}} &= +2.51 \times 10^{-9}\\
		\sigma_\mu &= 0.0\sigma
	\end{align}
	
	\subsection*{Muon - Current Situation (2025)}
	\begin{align}
		\Delta a_\mu^{\text{exp-SM}} &= +0.37(64) \times 10^{-9}\\
		\Delta a_\mu^{\text{T0}} &= +2.51 \times 10^{-9}\\
		\text{T0 Explanation} &: \text{Loop suppression in QCD environment}
	\end{align}
	
	\subsection*{Electron}
	\paragraph{2018 (Cs, Harvard):}
	\begin{align}
		\Delta a_e^{\text{exp-SM}} &= -0.87(36) \times 10^{-12}\\
		\Delta a_e^{\text{T0}} &= +0.0586 \times 10^{-12}\\
		\Delta a_e^{\text{total}} &= -0.8699 \times 10^{-12}\\
		\sigma_e &\approx -2.4\sigma
	\end{align}
	
	\paragraph{2020 (Rb, LKB):}
	\begin{align}
		\Delta a_e^{\text{exp-SM}} &= +0.48(30) \times 10^{-12}\\
		\Delta a_e^{\text{T0}} &= +0.0586 \times 10^{-12}\\
		\Delta a_e^{\text{total}} &= +0.4801 \times 10^{-12}\\
		\sigma_e &\approx +1.6\sigma
	\end{align}
	
	\subsection*{Tau}
	\begin{align}
		\Delta a_\tau^{\text{T0}} &= 7.09 \times 10^{-7}
	\end{align}
	Currently no experimental comparison possible.
	
\section*{Verification}
		The reduction of the muon discrepancy through improved HVP calculations is \textbf{not in contradiction with T0 theory}:
		
		\begin{itemize}
			\item \textbf{Independent contributions}: T0 provides a fundamental additional contribution independent of HVP corrections
			\item \textbf{Loop suppression}: In hadronic environments, T0 contributions can be suppressed by factor $\sim0.15$ through dynamic effects
			\item \textbf{Future tests}: The mass-proportional scaling remains the crucial test criterion
			\item \textbf{Tau prediction}: The significant tau contribution of $7.09 \times 10^{-7}$ provides a clear test of the theory
		\end{itemize}
		
		T0 theory thus remains a complete and testable fundamental extension.
% end box verification
	
	\section{Discussion}
	
	\subsection{Key Results of the Derivation}
	
	\begin{itemize}
		\item The \textbf{quadratic mass dependence} $\Delta a_\ell^{\text{T0}} \propto m_\ell^2$ follows directly from the Lagrangian derivation
		\item \textbf{No calibration} required - all parameters are fundamentally determined
		\item The \textbf{historical muon anomaly} is exactly reproduced ($0.0\sigma$ deviation)
		\item The \textbf{current reduction} of the discrepancy is explainable through loop suppression effects
		\item \textbf{Electron contributions} are negligibly small ($\sim 0.06 \times 10^{-12}$)
		\item \textbf{Tau predictions} are significant and testable ($7.09 \times 10^{-7}$)
	\end{itemize}
	
	\subsection{Physical Interpretation}
	
	The quadratic mass dependence naturally explains the hierarchy:
	\begin{align*}
		\frac{\Delta a_e^{\text{T0}}}{\Delta a_\mu^{\text{T0}}} &= \left(\frac{m_e}{m_\mu}\right)^2 = 2.34 \times 10^{-5}\\
		\frac{\Delta a_\tau^{\text{T0}}}{\Delta a_\mu^{\text{T0}}} &= \left(\frac{m_\tau}{m_\mu}\right)^2 = 283
	\end{align*}
	
	\section{Conclusion and Outlook}
	
	\subsection{Achieved Goals}
	
	The presented time field extension of the Lagrangian density:
	
	\begin{itemize}
		\item \textbf{Provides a complete derivation} of the additional contribution to the anomalous magnetic moment
		\item \textbf{Explains both experimental situations} consistently
		\item \textbf{Predicts testable contributions} for all leptons
		\item \textbf{Respects all fundamental symmetries} of the Standard Model
	\end{itemize}
	
	\subsection{Fundamental Significance}
	
	The T0 extension points to a deeper structure of spacetime in which time and mass are dually linked. The successful derivation of lepton anomalies supports the fundamental validity of time-mass duality.
	
	% Bibliography with new references
	


% Bibliography
\begin{thebibliography}{99}
	
	\bibitem{pdg2024}
	Particle Data Group Collaboration (2024). 
	\textit{Review of Particle Physics}. 
	Progress of Theoretical and Experimental Physics, 2024(8), 083C01.
	\url{https://pdg.lbl.gov}
	
	\bibitem{flag2024}
	Aoki, Y., et al. (FLAG Collaboration) (2024). 
	\textit{FLAG Review 2024 of Lattice Results for Low-Energy Constants}. 
	arXiv:2411.04268.
	\url{https://arxiv.org/abs/2411.04268}
	
	\bibitem{fermilab_muon_g2}
	Abi, B., et al. (Muon g-2 Collaboration) (2021). 
	\textit{Measurement of the Positive Muon Anomalous Magnetic Moment to 0.46 ppm}. 
	Physical Review Letters, 126, 141801.
	
	\bibitem{peskin_schroeder}
	Peskin, M. E., \& Schroeder, D. V. (1995). 
	\textit{An Introduction to Quantum Field Theory}. 
	Addison-Wesley.
	
	\bibitem{weinberg_qft}
	Weinberg, S. (1995). 
	\textit{The Quantum Theory of Fields, Vol. I--III}. 
	Cambridge University Press.
	
	\bibitem{griffiths_particle}
	Griffiths, D. (2008). 
	\textit{Introduction to Elementary Particles}. 
	Wiley-VCH.
	
	\bibitem{mandl_shaw}
	Mandl, F., \& Shaw, G. (2010). 
	\textit{Quantum Field Theory (2nd ed.)}. 
	Wiley.
	
	\bibitem{srednicki_qft}
	Srednicki, M. (2007). 
	\textit{Quantum Field Theory}. 
	Cambridge University Press.
	
	\bibitem{t0_fundamentals}
	Pascher, J. (2024). 
	\textit{T0-Theory: Foundations of Time-Mass Duality}. 
	Unpublished manuscript, HTL Leonding.
	
	\bibitem{t0_fine_structure}
	Pascher, J. (2024). 
	\textit{T0-Theory: The Fine Structure Constant}. 
	Unpublished manuscript, HTL Leonding.
	
	\bibitem{t0_neutrinos}
	Pascher, J. (2024). 
	\textit{T0-Theory: Neutrino Masses and PMNS Mixing}. 
	Unpublished manuscript, HTL Leonding.
	
	\bibitem{t0_github}
	Pascher, J. (2024--2025). 
	\textit{T0-Time-Mass-Duality Repository}. 
	GitHub.
	\url{https://github.com/jpascher/T0-Time-Mass-Duality}
	
	\bibitem{lattice_qcd_review}
	Kronfeld, A. S. (2012). 
	\textit{Twenty-first Century Lattice Gauge Theory: Results from the QCD Lagrangian}. 
	Annual Review of Nuclear and Particle Science, 62, 265--284.
	
	\bibitem{neutrino_mixing_pdg}
	Particle Data Group Collaboration (2024). 
	\textit{Neutrino Masses, Mixing, and Oscillations}. 
	PDG Review 2024.
	\url{https://pdg.lbl.gov/2024/reviews/rpp2024-rev-neutrino-mixing.pdf}
	
	\bibitem{higgs_discovery}
	ATLAS and CMS Collaborations (2012). 
	\textit{Observation of a New Particle in the Search for the Standard Model Higgs Boson}. 
	Physics Letters B, 716, 1--29.
	
	\bibitem{Brannen2005}
	C. P. Brannen, ``Estimate of neutrino masses from Koide's relation'', \textit{arXiv:hep-ph/0505028} (2005).
	\url{https://arxiv.org/abs/hep-ph/0505028}
	
	\bibitem{Brannen2006}
	C. P. Brannen, ``Koide Mass Formula for Neutrinos'', \textit{arXiv:0702.0052} (2006).
	\url{http://brannenworks.com/MASSES.pdf}
	
	\bibitem{PhaseVectors2025}
	Anonymous, ``The Koide Relation and Lepton Mass Hierarchy from Phase Vectors'', \textit{rXiv:2507.0040} (2025).
	\url{https://rxiv.org/pdf/2507.0040v1.pdf}
	
	\bibitem{PDG2025}
	Particle Data Group, ``Review of Particle Physics'', \textit{Phys. Rev. D} \textbf{112} (2025) 030001.
	\url{https://pdg.lbl.gov/2025/}
	
	\bibitem{terrell2024}
	Terrell et al. (2024). 
	\textit{Single-Clock Metrology in Nature}. 
	Nature Physics.
	
	\bibitem{hossenfelder2024}
	Hossenfelder, S. (2024). 
	\textit{Single Clock Video Explanation}. 
	YouTube.
	
	\bibitem{hundert1931}
	Hundert (1931). 
	\textit{Reference Work}. 
	Publisher.
	
	\bibitem{terrell2025}
	Terrell et al. (2025). 
	\textit{Advanced Clock Synchronization Methods}. 
	Physical Review Letters.
	
	\bibitem{pascher_t0_2025}
	Pascher, J. (2025). 
	\textit{T0-Theory: Complete Framework and Applications}. 
	Unpublished manuscript, HTL Leonding.
	
	\bibitem{t0qm}
	Pascher, J. (2024). 
	\textit{T0-Theory: Quantum Mechanics Formulation}. 
	Unpublished manuscript, HTL Leonding.
	
	\bibitem{t0anomale}
	Pascher, J. (2024). 
	\textit{T0-Theory: Anomalous Magnetic Moments}. 
	Unpublished manuscript, HTL Leonding.
	
	\bibitem{muong2complete}
	Abi, B., et al. (Muon g-2 Collaboration) (2023). 
	\textit{Complete Measurement of the Positive Muon Anomalous Magnetic Moment}. 
	Physical Review Letters, 131, 161802.
	
	\bibitem{penrose2004}
	Penrose, R. (2004). 
	\textit{The Road to Reality: A Complete Guide to the Laws of the Universe}. 
	Jonathan Cape.
	
	\bibitem{planck1900}
	Planck, M. (1900). 
	\textit{On the Theory of the Energy Distribution Law of the Normal Spectrum}. 
	Verhandlungen der Deutschen Physikalischen Gesellschaft, 2, 237.
	
	\bibitem{T0Theory}
	Pascher, J. (2024). 
	\textit{T0-Theory: Fundamental Principles}. 
	Unpublished manuscript, HTL Leonding.
	
	% Additional bibliography entries for all undefined citations
	\bibitem{6g_roadmap}
	6G Research Consortium (2024).
	\textit{6G Technology Roadmap}.
	Technical Report.
	
	\bibitem{Born2013}
	Born, M. (2013).
	\textit{Einstein's Theory of Relativity}.
	Dover Publications.
	
	\bibitem{Casimir1948}
	Casimir, H. B. G. (1948).
	\textit{On the attraction between two perfectly conducting plates}.
	Proc. Kon. Ned. Akad. Wetensch. B51, 793--795.
	
	\bibitem{Einstein1905}
	Einstein, A. (1905).
	\textit{On the Electrodynamics of Moving Bodies}.
	Annalen der Physik, 17, 891--921.
	
	\bibitem{Feynman2006}
	Feynman, R. P. (2006).
	\textit{QED: The Strange Theory of Light and Matter}.
	Princeton University Press.
	
	\bibitem{Griffiths2017}
	Griffiths, D. J. (2017).
	\textit{Introduction to Electrodynamics (4th ed.)}.
	Cambridge University Press.
	
	\bibitem{Jackson1999}
	Jackson, J. D. (1999).
	\textit{Classical Electrodynamics (3rd ed.)}.
	Wiley.
	
	\bibitem{Mohr2016}
	Mohr, P. J., et al. (2016).
	\textit{CODATA Recommended Values of the Fundamental Physical Constants: 2014}.
	Rev. Mod. Phys. 88, 035009.
	
	\bibitem{Parker2018}
	Parker, R. H., et al. (2018).
	\textit{Measurement of the fine-structure constant as a test of the Standard Model}.
	Science, 360, 191--195.
	
	\bibitem{Planck1900}
	Planck, M. (1900).
	\textit{On the Theory of the Energy Distribution Law of the Normal Spectrum}.
	Verhandlungen der Deutschen Physikalischen Gesellschaft, 2, 237.
	
	\bibitem{Planck2018}
	Planck Collaboration (2018).
	\textit{Planck 2018 results. VI. Cosmological parameters}.
	Astronomy \& Astrophysics, 641, A6.
	
	\bibitem{QFT_T0}
	Pascher, J. (2024).
	\textit{T0-Theory and QFT Connections}.
	Unpublished manuscript, HTL Leonding.
	
	\bibitem{Sommerfeld1916}
	Sommerfeld, A. (1916).
	\textit{On the Quantum Theory of Spectral Lines}.
	Annalen der Physik, 51, 1--94.
	
	\bibitem{T0_Feinstruktur}
	Pascher, J. (2024).
	\textit{T0-Theory: Fine Structure Analysis}.
	Unpublished manuscript, HTL Leonding.
	
	\bibitem{T0_SI}
	Pascher, J. (2024).
	\textit{T0-Theory and SI Units}.
	Unpublished manuscript, HTL Leonding.
	
	\bibitem{T0_fine_structure}
	Pascher, J. (2024).
	\textit{T0-Theory: The Fine Structure Constant}.
	Unpublished manuscript, HTL Leonding.
	
	\bibitem{T0_g2_erweiterung}
	Pascher, J. (2024).
	\textit{T0-Theory: g-2 Extensions}.
	Unpublished manuscript, HTL Leonding.
	
	\bibitem{T0_gravitational_constant}
	Pascher, J. (2024).
	\textit{T0-Theory: Gravitational Constant Derivation}.
	Unpublished manuscript, HTL Leonding.
	
	\bibitem{T0_netze_en}
	Pascher, J. (2024).
	\textit{T0-Theory: Network Structures}.
	Unpublished manuscript, HTL Leonding.
	
	\bibitem{T0_tm_erweiterung}
	Pascher, J. (2024).
	\textit{T0-Theory: Time-Mass Extensions}.
	Unpublished manuscript, HTL Leonding.
	
	\bibitem{Uzan2003}
	Uzan, J.-P. (2003).
	\textit{The fundamental constants and their variation}.
	Rev. Mod. Phys. 75, 403--455.
	
	\bibitem{Weinberg1995}
	Weinberg, S. (1995).
	\textit{The Quantum Theory of Fields, Vol. I}.
	Cambridge University Press.
	
	\bibitem{albrecht1999}
	Albrecht, A. \& Magueijo, J. (1999).
	\textit{A time varying speed of light as a solution to cosmological puzzles}.
	Phys. Rev. D 59, 043516.
	
	\bibitem{alice2023}
	ALICE Collaboration (2023).
	\textit{Recent results from ALICE}.
	CERN-EP-2023-XXX.
	
	\bibitem{analog_optical}
	Smith, J. et al. (2024).
	\textit{Analog optical computing systems}.
	Nature Photonics.
	
	\bibitem{ashtekar2004}
	Ashtekar, A. \& Lewandowski, J. (2004).
	\textit{Background independent quantum gravity}.
	Class. Quantum Grav. 21, R53.
	
	\bibitem{atlas2023}
	ATLAS Collaboration (2023).
	\textit{ATLAS physics results}.
	CERN-PH-EP-2023-XXX.
	
	\bibitem{atlas2023higgs}
	ATLAS Collaboration (2023).
	\textit{Higgs boson measurements}.
	Phys. Rev. Lett.
	
	\bibitem{barbour1999}
	Barbour, J. (1999).
	\textit{The End of Time}.
	Oxford University Press.
	
	\bibitem{barrow1999}
	Barrow, J. D. (1999).
	\textit{Cosmologies with varying light speed}.
	Phys. Rev. D 59, 043515.
	
	\bibitem{becker2007}
	Becker, K. et al. (2007).
	\textit{String Theory and M-Theory}.
	Cambridge University Press.
	
	\bibitem{bell_muon}
	Bennett, G. W., et al. (Muon g-2 Collaboration) (2006).
	\textit{Final report of the E821 muon anomalous magnetic moment measurement}.
	Phys. Rev. D 73, 072003.
	
	\bibitem{bondi1948}
	Bondi, H. \& Gold, T. (1948).
	\textit{The steady-state theory of the expanding universe}.
	Mon. Not. R. Astron. Soc. 108, 252--270.
	
	\bibitem{brewer2019}
	Brewer, S. M. et al. (2019).
	\textit{Al+ Quantum-Logic Clock with Systematic Uncertainty below $10^{-18}$}.
	Phys. Rev. Lett. 123, 033201.
	
	\bibitem{cms2023top}
	CMS Collaboration (2023).
	\textit{Top quark measurements at CMS}.
	JHEP 2023.
	
	\bibitem{cms2024}
	CMS Collaboration (2024).
	\textit{CMS physics results 2024}.
	CERN-PH-EP-2024-XXX.
	
	\bibitem{codata2019}
	Tiesinga, E. et al. (2019).
	\textit{The 2018 CODATA Recommended Values}.
	J. Phys. Chem. Ref. Data.
	
	\bibitem{desi2025}
	DESI Collaboration (2025).
	\textit{DESI 2025 Cosmology Results}.
	arXiv preprint.
	
	\bibitem{differential_optical}
	Wang, X. et al. (2024).
	\textit{Differential optical computing}.
	Optica.
	
	\bibitem{dingle1972}
	Dingle, H. (1972).
	\textit{Science at the Crossroads}.
	Martin Brian \& O'Keeffe.
	
	\bibitem{divalentino2021}
	Di Valentino, E. et al. (2021).
	\textit{In the realm of the Hubble tension}.
	Class. Quantum Grav. 38, 153001.
	
	\bibitem{elnaschie2004}
	El Naschie, M. S. (2004).
	\textit{A review of E infinity theory}.
	Chaos, Solitons \& Fractals, 19, 209--236.
	
	\bibitem{fabrication_heterogeneous}
	Chen, Y. et al. (2024).
	\textit{Heterogeneous photonic integration}.
	Nature Electronics.
	
	\bibitem{fermilab2023}
	Fermilab (2023).
	\textit{Muon g-2 results}.
	Phys. Rev. Lett.
	
	\bibitem{flexible_wafer}
	Kim, S. et al. (2024).
	\textit{Flexible wafer-scale photonics}.
	Science Advances.
	
	\bibitem{francesco1997}
	Di Francesco, P. et al. (1997).
	\textit{Conformal Field Theory}.
	Springer.
	
	\bibitem{hartree1957}
	Hartree, D. R. (1957).
	\textit{The Calculation of Atomic Structures}.
	Wiley.
	
	\bibitem{hhi_6g}
	Fraunhofer HHI (2024).
	\textit{6G Photonic Integration}.
	Technical Report.
	
	\bibitem{hossenfelder2025}
	Hossenfelder, S. (2025).
	\textit{Science without the gobbledygook}.
	YouTube/Blog.
	
	\bibitem{hossenfelder_single_clock_video}
	Hossenfelder, S. (2024).
	\textit{The Single Clock Problem}.
	YouTube.
	
	\bibitem{hoyle1948}
	Hoyle, F. (1948).
	\textit{A new model for the expanding universe}.
	Mon. Not. R. Astron. Soc. 108, 372--382.
	
	\bibitem{integration_microelectronic}
	Liu, A. et al. (2024).
	\textit{Microelectronic photonic integration}.
	IEEE Journal.
	
	\bibitem{jacobson1995}
	Jacobson, T. (1995).
	\textit{Thermodynamics of spacetime}.
	Phys. Rev. Lett. 75, 1260.
	
	\bibitem{kasevich2023}
	Kasevich, M. et al. (2023).
	\textit{Atom interferometry tests}.
	Nature Physics.
	
	\bibitem{lerner2014}
	Lerner, E. J. (2014).
	\textit{An open letter on cosmology}.
	New Scientist.
	
	\bibitem{lisa2017}
	LISA Consortium (2017).
	\textit{Laser Interferometer Space Antenna}.
	ESA Technical Report.
	
	\bibitem{lithium_tantalate}
	Zhang, M. et al. (2024).
	\textit{Thin-film lithium tantalate photonics}.
	Nature Photonics.
	
	\bibitem{lopez2010}
	Lopez-Corredoira, M. (2010).
	\textit{Tests and problems of the standard model in cosmology}.
	Int. J. Mod. Phys. D.
	
	\bibitem{ludlow2015}
	Ludlow, A. D. et al. (2015).
	\textit{Optical atomic clocks}.
	Rev. Mod. Phys. 87, 637.
	
	\bibitem{mach1883}
	Mach, E. (1883).
	\textit{Die Mechanik in ihrer Entwickelung}.
	F.A. Brockhaus.
	
	\bibitem{maldacena1998}
	Maldacena, J. (1998).
	\textit{The large N limit of superconformal field theories}.
	Adv. Theor. Math. Phys. 2, 231--252.
	
	\bibitem{mueller2014}
	Müller, H. et al. (2014).
	\textit{Atom interferometry tests of the gravitational redshift}.
	Phys. Rev. Lett.
	
	\bibitem{mug2_final_2025}
	Muon g-2 Collaboration (2025).
	\textit{Final muon g-2 measurement}.
	Phys. Rev. Lett.
	
	\bibitem{muong2_2023}
	Muon g-2 Collaboration (2023).
	\textit{Updated muon g-2 results}.
	Phys. Rev. Lett.
	
	\bibitem{nathan2024}
	Nathan, A. et al. (2024).
	\textit{Quantum computing advances}.
	Nature.
	
	\bibitem{newell2018}
	Newell, D. B. et al. (2018).
	\textit{The CODATA 2017 values of h, e, k, and $N_A$}.
	Metrologia 55, L13.
	
	\bibitem{nottale1993}
	Nottale, L. (1993).
	\textit{Fractal Space-Time and Microphysics}.
	World Scientific.
	
	\bibitem{on_chip_lithium}
	Wang, C. et al. (2024).
	\textit{On-chip lithium niobate photonics}.
	Nature Communications.
	
	\bibitem{optical_advantages}
	Shastri, B. J. et al. (2024).
	\textit{Advantages of optical computing}.
	Nature Reviews Physics.
	
	\bibitem{pascher2025cmb}
	Pascher, J. (2025).
	\textit{T0-Theory: CMB Analysis}.
	Unpublished manuscript, HTL Leonding.
	
	\bibitem{pascher2025g2}
	Pascher, J. (2025).
	\textit{T0-Theory: g-2 Predictions}.
	Unpublished manuscript, HTL Leonding.
	
	\bibitem{pascher2025qm}
	Pascher, J. (2025).
	\textit{T0-Theory: Quantum Mechanics}.
	Unpublished manuscript, HTL Leonding.
	
	\bibitem{pascher2025si}
	Pascher, J. (2025).
	\textit{T0-Theory: SI Unit System}.
	Unpublished manuscript, HTL Leonding.
	
	\bibitem{pascher2025t0}
	Pascher, J. (2025).
	\textit{T0-Theory: Complete Framework}.
	Unpublished manuscript, HTL Leonding.
	
	\bibitem{pascher:fundamentals}
	Pascher, J. (2024).
	\textit{T0-Theory: Fundamentals}.
	Unpublished manuscript, HTL Leonding.
	
	\bibitem{pascher:g2_rev9}
	Pascher, J. (2024).
	\textit{T0-Theory: g-2 Revision 9}.
	Unpublished manuscript, HTL Leonding.
	
	\bibitem{pascher:geometric_formalism}
	Pascher, J. (2024).
	\textit{T0-Theory: Geometric Formalism}.
	Unpublished manuscript, HTL Leonding.
	
	\bibitem{pascher:ml_addendum}
	Pascher, J. (2024).
	\textit{T0-Theory: Machine Learning Addendum}.
	Unpublished manuscript, HTL Leonding.
	
	\bibitem{pascher:t0_foundations}
	Pascher, J. (2024).
	\textit{T0-Theory: Foundations}.
	Unpublished manuscript, HTL Leonding.
	
	\bibitem{pascher_derivation_beta_2025}
	Pascher, J. (2025).
	\textit{T0-Theory: Derivation of Beta}.
	Unpublished manuscript, HTL Leonding.
	
	\bibitem{pascher_higgs_connection_2025}
	Pascher, J. (2025).
	\textit{T0-Theory: Higgs Connection}.
	Unpublished manuscript, HTL Leonding.
	
	\bibitem{pascher_lagrangian_extended_2025}
	Pascher, J. (2025).
	\textit{T0-Theory: Extended Lagrangian}.
	Unpublished manuscript, HTL Leonding.
	
	\bibitem{pascher_mathematical_structure_2025}
	Pascher, J. (2025).
	\textit{T0-Theory: Mathematical Structure}.
	Unpublished manuscript, HTL Leonding.
	
	\bibitem{pascher_t0_cmb_2025}
	Pascher, J. (2025).
	\textit{T0-Theory: CMB Predictions}.
	Unpublished manuscript, HTL Leonding.
	
	\bibitem{pascher_t0_energie_2025}
	Pascher, J. (2025).
	\textit{T0-Theory: Energy}.
	Unpublished manuscript, HTL Leonding.
	
	\bibitem{pascher_t0_energy_2025}
	Pascher, J. (2025).
	\textit{T0-Theory: Energy Framework}.
	Unpublished manuscript, HTL Leonding.
	
	\bibitem{pascher_t0_theory_2025}
	Pascher, J. (2025).
	\textit{T0-Theory: Complete Theory}.
	Unpublished manuscript, HTL Leonding.
	
	\bibitem{penrose1959}
	Penrose, R. (1959).
	\textit{The apparent shape of a relativistically moving sphere}.
	Proc. Cambridge Phil. Soc. 55, 137--139.
	
	\bibitem{penrose1967}
	Penrose, R. (1967).
	\textit{Twistor algebra}.
	J. Math. Phys. 8, 345--366.
	
	\bibitem{peratt1992}
	Peratt, A. L. (1992).
	\textit{Physics of the Plasma Universe}.
	Springer-Verlag.
	
	\bibitem{peskin1995}
	Peskin, M. E. \& Schroeder, D. V. (1995).
	\textit{An Introduction to Quantum Field Theory}.
	Addison-Wesley.
	
	\bibitem{peskin_schroeder_1995}
	Peskin, M. E. \& Schroeder, D. V. (1995).
	\textit{An Introduction to Quantum Field Theory}.
	Addison-Wesley.
	
	\bibitem{phoquant}
	PhoQuant (2024).
	\textit{Photonic quantum computing}.
	Technical Report.
	
	\bibitem{photonics_ai}
	Wetzstein, G. et al. (2024).
	\textit{Photonics for AI}.
	Nature.
	
	\bibitem{planck1906}
	Planck, M. (1906).
	\textit{The Theory of Heat Radiation}.
	Johann Ambrosius Barth.
	
	\bibitem{planck2018}
	Planck Collaboration (2018).
	\textit{Planck 2018 results}.
	A\&A 641, A6.
	
	\bibitem{polchinski1998}
	Polchinski, J. (1998).
	\textit{String Theory}.
	Cambridge University Press.
	
	\bibitem{qant_nps}
	QANT (2024).
	\textit{Quantum photonics systems}.
	Technical Report.
	
	\bibitem{quantenjahr25}
	Quantenjahr (2025).
	\textit{International Year of Quantum}.
	UNESCO.
	
	\bibitem{recurrent_photonics}
	Tait, A. N. et al. (2024).
	\textit{Recurrent photonic neural networks}.
	Optica.
	
	\bibitem{rf_photonics}
	Capmany, J. \& Novak, D. (2024).
	\textit{Microwave photonics}.
	Nature Photonics.
	
	\bibitem{riess2019}
	Riess, A. G. et al. (2019).
	\textit{Large Magellanic Cloud Cepheid Standards}.
	ApJ 876, 85.
	
	\bibitem{riess2022}
	Riess, A. G. et al. (2022).
	\textit{A Comprehensive Measurement of H0}.
	ApJ 934, L7.
	
	\bibitem{rovelli2004}
	Rovelli, C. (2004).
	\textit{Quantum Gravity}.
	Cambridge University Press.
	
	\bibitem{sciama1953}
	Sciama, D. W. (1953).
	\textit{On the origin of inertia}.
	Mon. Not. R. Astron. Soc. 113, 34--42.
	
	\bibitem{sciencedaily2025}
	ScienceDaily (2025).
	\textit{Physics news}.
	Online.
	
	\bibitem{sm_g2_2025}
	Aoyama, T. et al. (2025).
	\textit{Standard Model prediction for g-2}.
	Phys. Rep.
	
	\bibitem{susskind1995}
	Susskind, L. (1995).
	\textit{The world as a hologram}.
	J. Math. Phys. 36, 6377--6396.
	
	\bibitem{t0_kosmologie}
	Pascher, J. (2024).
	\textit{T0-Theory: Cosmology}.
	Unpublished manuscript, HTL Leonding.
	
	\bibitem{terrell1959}
	Terrell, J. (1959).
	\textit{Invisibility of the Lorentz contraction}.
	Phys. Rev. 116, 1041--1045.
	
	\bibitem{terrell_single_clock_nature_2024}
	Terrell, J. et al. (2024).
	\textit{Single clock precision measurements}.
	Nature Physics.
	
	\bibitem{tfln_foundry}
	TFLN Foundry (2024).
	\textit{Thin-film lithium niobate foundry services}.
	Technical Specifications.
	
	\bibitem{thiemann2007}
	Thiemann, T. (2007).
	\textit{Modern Canonical Quantum General Relativity}.
	Cambridge University Press.
	
	\bibitem{thz_epfl}
	EPFL (2024).
	\textit{Terahertz photonics research}.
	Technical Report.
	
	\bibitem{unnikrishnan2004}
	Unnikrishnan, C. S. (2004).
	\textit{On Einstein's resolution of the twin clock paradox}.
	Current Science, 86, 704--709.
	
	\bibitem{verlinde2011}
	Verlinde, E. (2011).
	\textit{On the origin of gravity and the laws of Newton}.
	JHEP 2011, 29.
	
	\bibitem{video2025}
	Video (2025).
	\textit{Physics video explanation}.
	YouTube.
	
	\bibitem{weinberg1995}
	Weinberg, S. (1995).
	\textit{The Quantum Theory of Fields}.
	Cambridge University Press.
	
	\bibitem{weiskopf2000}
	Weiskopf, D. (2000).
	\textit{Visualization of special relativity}.
	PhD thesis, University of Tübingen.
	
	\bibitem{wheeler1990}
	Wheeler, J. A. (1990).
	\textit{A Journey into Gravity and Spacetime}.
	Scientific American Library.
	
	\bibitem{wiki_bell}
	Wikipedia (2024).
	\textit{Bell's theorem}.
	Online encyclopedia.
	
	\bibitem{zwicky1929}
	Zwicky, F. (1929).
	\textit{On the red shift of spectral lines through interstellar space}.
	Proc. Natl. Acad. Sci. 15, 773--779.

\end{thebibliography}


\end{document}
