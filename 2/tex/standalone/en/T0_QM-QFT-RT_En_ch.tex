\documentclass[11pt,a4paper]{article}
\usepackage[a4paper,margin=2cm]{geometry}
\usepackage[utf8]{inputenc}
\usepackage[english]{babel}
\usepackage{lmodern}
\renewcommand{\familydefault}{\sfdefault}

\usepackage{amsmath,amssymb,amsthm}
\usepackage{graphicx}
\usepackage[unicode,pdfencoding=auto,hypertexnames=false]{hyperref}
\usepackage{booktabs}
\usepackage{longtable}
\usepackage{array}
\usepackage{siunitx}
\usepackage{fancyhdr}
\usepackage{float}
\usepackage{tikz}
% tcolorbox removed for standalone
% tcbset removed
\tikzset{
  t0blue/.style={draw=blue,fill=blue!10},
  t0red/.style={draw=red,fill=red!10},
  t0green/.style={draw=green!50!black,fill=green!10},
  t0orange/.style={draw=orange,fill=orange!10},
}
\usepackage{setspace}
\usepackage{enumitem}
\usepackage{adjustbox}
\usepackage{xcolor}

% Define colors for xcolor package
\definecolor{t0green}{RGB}{34,139,34}
\definecolor{t0blue}{RGB}{0,0,255}
\definecolor{t0red}{RGB}{255,0,0}
\definecolor{t0orange}{RGB}{255,165,0}

% Define custom column types for tables
\newcolumntype{L}[1]{>{\raggedright\arraybackslash}p{#1}}
\newcolumntype{C}[1]{>{\centering\arraybackslash}p{#1}}
\newcolumntype{R}[1]{>{\raggedleft\arraybackslash}p{#1}}

\setlength{\parindent}{0pt}
\setlength{\parskip}{6pt}

\hypersetup{
  colorlinks=true,
  linkcolor=blue,
  citecolor=blue,
  urlcolor=blue
}
\pagestyle{fancy}
\setlength{\headheight}{28pt}

\newcommand{\checkmarkx}{\checkmark}
\newcommand{\warningx}{\textbf{!}}

% Makros aus Einzel-Dokumenten (Fallback-Definitionen)
\newcommand{\mytimes}{\times}
\newcommand{\myapprox}{\approx}
\newcommand{\mysim}{\sim}
\newcommand{\myomega}{\omega}
\newcommand{\mypi}{\pi}
\newcommand{\myrightarrow}{\rightarrow}
\newcommand{\mypropto}{\propto}
\newcommand{\deltafield}{\delta\phi}
\newcommand{\xipar}{\xi}
\newcommand{\xiT}{\xi}
\newcommand{\lambdah}{\lambda_h}

% Additional macros used in chapter files
\newcommand{\Kfrak}{K_{\text{frak}}}  % Fractal correction factor
\newcommand{\Dfrak}{D_f}              % Fractal dimension
\newcommand{\betapar}{\beta}          % T0 beta parameter
\newcommand{\alphapar}{\alpha}        % T0 alpha parameter
\newcommand{\Efield}{E}               % Energy field
% Note: checkmarkxa/warningxa are variants used in auto-generated chapter files
\newcommand{\checkmarkxa}{\checkmark}
\newcommand{\warningxa}{\textbf{!}}

% Additional T0-specific macros
\newcommand{\xigeom}{\xi_{\text{geom}}}  % Geometric xi
\newcommand{\lP}{\ell_P}                  % Planck length
\newcommand{\rzero}{r_0}                  % Characteristic radius
\newcommand{\xirat}{\xi_{\text{rat}}}     % Xi ratio
\newcommand{\tzero}{t_0}                  % Characteristic time
\newcommand{\natunits}{\text{(nat. units)}}  % Natural units annotation
\newcommand{\myRightarrow}{\Rightarrow}   % Arrow variant
\newcommand{\Lag}{\mathcal{L}}            % Lagrangian

% Physics macros used in chapter files
\newcommand{\CQCD}{C_{\text{QCD}}}        % QCD correction
\newcommand{\EP}{E_P}                     % Planck energy
\newcommand{\Ee}{E_e}                     % Electron energy
\newcommand{\Emu}{E_\mu}                  % Muon energy
\newcommand{\Exi}{E_\xi}                  % Xi energy
\newcommand{\Ezero}{E_0}                  % Characteristic energy
\newcommand{\Hubble}{H}                   % Hubble constant
\newcommand{\Kspec}{K_{\text{spec}}}      % Spectral correction
\newcommand{\Lambdat}{\Lambda_t}          % Time-related cosmological constant
\newcommand{\Leff}{\mathcal{L}_{\text{eff}}}  % Effective Lagrangian
\newcommand{\Lorentz}{\mathcal{L}}        % Lorentz symbol
\newcommand{\Lxi}{L_\xi}                  % Xi length
\newcommand{\Tfield}{T}                   % Time field
\newcommand{\Weyl}{W}                     % Weyl tensor/symbol
\newcommand{\alphaEMSI}{\alpha_{\text{EM,SI}}}  % EM alpha in SI
\newcommand{\alphaEMnat}{\alpha_{\text{EM,nat}}}  % EM alpha in natural units
\newcommand{\alphaem}{\alpha_{\text{em}}} % Electromagnetic alpha
\newcommand{\betaTSI}{\beta_{T,\text{SI}}}  % Beta in SI
\newcommand{\betaTnat}{\beta_{T,\text{nat}}}  % Beta in natural units
\newcommand{\deltam}{\delta m}            % Mass difference
\newcommand{\phiT}{\phi_T}                % T-field phi
\newcommand{\tP}{t_P}                     % Planck time
\newcommand{\rhoCMB}{\rho_{\text{CMB}}}   % CMB density
\newcommand{\rhoCasimir}{\rho_{\text{Casimir}}}  % Casimir density

% Table formatting
\usepackage{multirow}

% Additional physics macros
\newcommand{\Riem}{\mathcal{R}}           % Riemann tensor
\newcommand{\ZPinch}{Z_{\text{pinch}}}    % Z-pinch
\newcommand{\SynchPower}{P_{\text{synch}}} % Synchrotron power
\newcommand{\Rzero}{R_0}                  % Characteristic radius
\newcommand{\alphafine}{\alpha}           % Fine structure constant
\newcommand{\Etau}{E_\tau}                % Tau energy
\newcommand{\deltaE}{\delta E}            % Energy deviation
\newcommand{\EPlanck}{E_P}                % Planck energy
\newcommand{\pichar}{\pi}                 % Pi character
\newcommand{\alphaWSI}{\alpha_{W,\text{SI}}}  % Wien alpha in SI
\newcommand{\alphaWnat}{\alpha_{W,\text{nat}}}  % Wien alpha in natural units

% Einfache abstract-Umgebung für Kapitel:
\newenvironment{abstract}{%
  \begin{center}\bfseries Abstract\end{center}\small
}{\par}


\title{T0 QM-QFT-RT En}
\author{J. Pascher}
\date{\today}

\begin{document}
\maketitle

\section*{T0 Qm Qft Rt (T0 QM-QFT-RT)}

	\begin{abstract}
		This comprehensive presentation of the T0 Quantum Field Theory systematically develops all fundamental aspects of quantum field theory, quantum mechanics, and quantum computer technology within the T0-Framework. Based on the time-mass duality $T_{\text{field}} \cdot \Efield = 1$ and the universal parameter $\xipar = \frac{4}{3} \times 10^{-4}$, the Schrödinger and Dirac equations are fundamentally extended, Bell inequalities are modified, and deterministic quantum computers are developed. The theory solves the measurement problem of quantum mechanics and restores locality and realism, while enabling practical applications in quantum technology.
	\end{abstract}
	
	
	\section{Introduction: T0 Revolution in QFT and QM}
	
	The T0-Theory not only revolutionizes quantum field theory, but also the fundamental equations of quantum mechanics and opens up entirely new possibilities for quantum computer technologies.
	
	\subsubsection*{{T0 Basic Principles for QFT and QM}}
\section*{Fundamental T0 Relations:}
		\begin{align}
			T_{\text{field}}(x,t) \cdot \Efield(x,t) &= 1 \quad \text{(Time-Energy Duality)} \\
			\square \deltaE + \xipar \cdot \mathcal{F}[\deltaE] &= 0 \quad \text{(Universal Field Equation)} \\
			\mathcal{L} &= \frac{\xipar}{\EPlanck^2} (\partial \deltaE)^2 \quad \text{(T0 Lagrangian Density)}
		\end{align}

	
	\section{T0 Field Quantization}
	
	\subsection{Canonical Quantization with Dynamic Time}
	
	The fundamental innovation of T0-QFT lies in the treatment of time as a dynamic field:
	
	\subsubsection*{{T0 Canonical Quantization}}
\section*{Modified Canonical Commutation Relations:}
		\begin{align}
			[\hat{\phi}(x), \hat{\pi}(y)] &= i\hbar \delta^3(x-y) \cdot T_{\text{field}}(x,t) \\
			[\hat{\Efield}(x), \hat{\Pi}_E(y)] &= i\hbar \delta^3(x-y) \cdot \frac{\xipar}{\EPlanck^2}
		\end{align}

	
	The field operators take an extended form:
	
	\begin{equation}
		\hat{\phi}(x,t) = \int \frac{d^3k}{(2\pi)^3} \frac{1}{\sqrt{2\omega_k \cdot T_{\text{field}}(t)}} \left[\hat{a}_k e^{-ik \cdot x} + \hat{b}^\dagger_k e^{ik \cdot x}\right]
	\end{equation}
	
	\subsection{T0-Modified Dispersion Relation}
	
	The energy-momentum relation is modified by the time field:
	
	\begin{equation}
		\boxed{\omega_k = \sqrt{k^2 + m^2} \cdot \left(1 + \xipar \cdot \frac{\langle\deltaE\rangle}{\EPlanck}\right)}
	\end{equation}
	
	\section{T0 Renormalization: Natural Cutoff}
	
	\subsubsection*{{T0 Renormalization}}
\section*{Natural UV-Cutoff:}
		\begin{equation}
			\Lambda_{\text{T0}} = \frac{\EPlanck}{\xipar} \approx 7.5 \times 10^{15} \text{ GeV}
		\end{equation}
		
		All loop integrals automatically converge at this fundamental scale.

	
	The beta functions are modified by T0 corrections:
	
	\begin{equation}
		\beta_g^{\text{T0}} = \beta_g^{\text{SM}} + \xipar \cdot \frac{g^3}{(4\pi)^2} \cdot f_{\text{T0}}(g)
	\end{equation}
	
	\section{T0 Quantum Mechanics: Fundamental Equations Understood Anew}
	
	\subsection{T0-Modified Schrödinger Equation}
	
	The Schrödinger equation receives a revolutionary extension through the dynamic time field:
	
	\subsubsection*{{T0 Schrödinger Equation}}
\section*{Time Field-Dependent Schrödinger Equation:}
		\begin{equation}
			i\hbar \cdot T_{\text{field}}(x,t) \frac{\partial\psi}{\partial t} = \hat{H}_0 \psi + \hat{V}_{\text{T0}}(x,t) \psi
		\end{equation}
		
		where:
		\begin{align}
			\hat{H}_0 &= -\frac{\hbar^2}{2m} \nabla^2 + V_{\text{extern}}(x) \\
			\hat{V}_{\text{T0}}(x,t) &= \xipar \hbar^2 \cdot \frac{\deltaE(x,t)}{E_{\text{Pl}}}
		\end{align}

	
	\subsubsection{Physical Interpretation}
	
	The T0 modification leads to three fundamental changes:
	
	\begin{enumerate}
		\item \textbf{Variable Time Evolution:} The quantum evolution proceeds more slowly in regions of high energy density
		\item \textbf{Energy Field Coupling:} The T0 potential couples quantum particles to local field fluctuations
		\item \textbf{Deterministic Corrections:} Subtle, but measurable deviations from standard QM predictions
	\end{enumerate}
	
	\subsubsection{Hydrogen Atom with T0 Corrections}
	
	For the hydrogen atom, the result is:
	
	\begin{align}
		E_n^{\text{T0}} &= E_n^{\text{Bohr}} \left(1 + \xipar \frac{E_n}{\EPlanck}\right) \\
		&= -13.6 \text{ eV} \cdot \frac{1}{n^2} \left(1 + \xipar \frac{13.6 \text{ eV}}{1.22 \times 10^{19} \text{ GeV}}\right)
	\end{align}
	
	The correction is tiny ($\sim 10^{-32}$ eV), but in principle measurable with ultra-precision spectroscopy.
	
	\subsection{T0-Modified Dirac Equation}
	
	Relativistic quantum mechanics is fundamentally altered by the T0 time field:
	
	\subsubsection*{{T0 Dirac Equation}}
\section*{Time Field-Dependent Dirac Equation:}
		\begin{equation}
			\left[i\gamma^\mu \left(\partial_\mu + \frac{\xipar}{\EPlanck} \Gamma_\mu^{(T)}\right) - m\right]\psi = 0
		\end{equation}
		
		where the T0 spinor connection is:
		\begin{equation}
			\Gamma_\mu^{(T)} = \frac{1}{\Tfield(x)} \partial_\mu \Tfield(x) = -\frac{\partial_\mu \deltaE}{\deltaE^2}
		\end{equation}

	
	\subsubsection{Spin and T0 Fields}
	
	The spin properties are modified by the time field:
	
	\begin{align}
		\vec{S}^{\text{T0}} &= \vec{S}^{\text{Standard}} \left(1 + \xipar \frac{\langle\deltaE\rangle}{\EPlanck}\right) \\
		g_{\text{factor}}^{\text{T0}} &= 2 + \xipar \frac{m^2}{M_{\text{Pl}}^2}
	\end{align}
	
	This explains the anomalous magnetic moments of the electron and muon!
	
	\section{T0 Quantum Computers: Revolution in Information Processing}
	
	\subsection{Deterministic Quantum Logic}
	
	The T0 theory enables a completely new type of quantum computers:
	
	\subsubsection*{{T0 Quantum Computer Principles}}
\section*{Fundamental Differences from Standard QC:}
		\begin{itemize}
			\item \textbf{Deterministic Evolution:} Quantum gates are fully predictable
			\item \textbf{Energy Field-Based Qubits:} $|0\rangle$, $|1\rangle$ as energy field configurations
			\item \textbf{Time Field Control:} Manipulation through local time field modulation
			\item \textbf{Natural Error Correction:} Self-stabilizing energy fields
		\end{itemize}

	
	\subsection{T0 Qubit Representation}
	
	A T0 qubit is realized through energy field configurations:
	
	\begin{align}
		|0\rangle_{\text{T0}} &\leftrightarrow \deltaE_0(x,t) = E_0 \cdot f_0(x,t) \\
		|1\rangle_{\text{T0}} &\leftrightarrow \deltaE_1(x,t) = E_1 \cdot f_1(x,t) \\
		|\psi\rangle_{\text{T0}} &= \alpha|0\rangle + \beta|1\rangle \leftrightarrow \alpha\deltaE_0 + \beta\deltaE_1
	\end{align}
	
	\subsubsection{T0 Quantum Gates}
	
	Quantum gates are realized through targeted time field manipulation:
	
\section*{T0 Hadamard Gate:}
	\begin{equation}
		H_{\text{T0}} = \frac{1}{\sqrt{2}}\begin{pmatrix} 1 & 1 \\ 1 & -1 \end{pmatrix} \cdot \left(1 + \xipar \frac{\langle\deltaE\rangle}{\EPlanck}\right)
	\end{equation}
	
\section*{T0 CNOT Gate:}
	\begin{equation}
		\text{CNOT}_{\text{T0}} = \begin{pmatrix} 1 & 0 & 0 & 0 \\ 0 & 1 & 0 & 0 \\ 0 & 0 & 0 & 1 \\ 0 & 0 & 1 & 0 \end{pmatrix} \cdot \left(\mathbb{I} + \xipar \frac{\delta\Efield}{\EPlanck} \sigma_z \otimes \sigma_x\right)
	\end{equation}
	
	\subsection{Quantum Algorithms with T0 Improvements}
	
	\subsubsection{T0 Shor Algorithm}
	
	The factorization algorithm is improved by deterministic T0 evolution:
	
	\begin{equation}
		P_{\text{Erfolg}}^{\text{T0}} = P_{\text{Erfolg}}^{\text{Standard}} \cdot \left(1 + \xipar \sqrt{n}\right)
	\end{equation}
	
	where $n$ is the number to be factored. For RSA-2048, this means an improved success probability of $\sim 10^{-2}$.
	
	\subsubsection{T0 Grover Algorithm}
	
	The database search is optimized through energy field focusing:
	
	\begin{equation}
		N_{\text{Iterationen}}^{\text{T0}} = \frac{\pi}{4}\sqrt{N} \left(1 - \xipar \ln N\right)
	\end{equation}
	
	This leads to logarithmic improvements for large databases.
	
	\section{Bell Inequalities and T0 Locality}
	
	\subsection{T0-Modified Bell Inequalities}
	
	The famous Bell inequalities receive subtle corrections through the T0 time field:
	
	\subsubsection*{{T0 Bell Corrections}}
\section*{Modified CHSH Inequality:}
		\begin{equation}
			|E(a,b) - E(a,b') + E(a',b) + E(a',b')| \leq 2 + \xipar \Delta_{\text{T0}}
		\end{equation}
		
		where $\Delta_{\text{T0}}$ is the time field correction:
		\begin{equation}
			\Delta_{\text{T0}} = \frac{\langle|\deltaE_A - \deltaE_B|\rangle}{\EPlanck}
		\end{equation}

	
	\subsection{Local Reality with T0 Fields}
	
	The T0 theory provides a local realistic explanation for quantum correlations:
	
	\subsubsection{Hidden Variable: The Time Field}
	
	The T0 time field acts as a local hidden variable:
	
	\begin{equation}
		P(A,B|a,b,\lambda_{\text{T0}}) = P_A(A|a,T_{\text{field},A}) \cdot P_B(B|b,T_{\text{field},B})
	\end{equation}
	
	where $\lambda_{\text{T0}} = \{T_{\text{field},A}(t), T_{\text{field},B}(t)\}$ are the local time field configurations.
	
	\subsubsection{Superdeterminism through T0 Correlations}
	
	The T0 time field establishes superdeterminism without ''spooky action at a distance'':
	
	\begin{align}
		T_{\text{field},A}(t) &= T_{\text{field},\text{common}}(t-r/c) + \delta T_{\text{field},A}(t) \\
		T_{\text{field},B}(t) &= T_{\text{field},\text{common}}(t-r/c) + \delta T_{\text{field},B}(t)
	\end{align}
	
	The common time field history explains the correlations without violating locality.
	
	\section{Experimental Tests of T0 Quantum Mechanics}
	
	\subsection{High-Precision Interferometry}
	
	\subsubsection{Atom Interferometer with T0 Signatures}
	
	Atom interferometers could detect T0 effects through phase shifts:
	
	\begin{equation}
		\Delta\phi_{\text{T0}} = \frac{m \cdot v \cdot L}{\hbar} \cdot \xipar \frac{\langle\deltaE\rangle}{\EPlanck}
	\end{equation}
	
	For cesium atoms in a 1-meter interferometer:
	\begin{equation}
		\Delta\phi_{\text{T0}} \sim 10^{-18} \text{ rad} \times \frac{\langle\deltaE\rangle}{1 \text{ eV}}
	\end{equation}
	
	\subsubsection{Gravitational Wave Interferometry}
	
	LIGO/Virgo could measure T0 corrections in gravitational wave signals:
	
	\begin{equation}
		h_{\text{T0}}(f) = h_{\text{GR}}(f) \left(1 + \xipar \left(\frac{f}{f_{\text{Planck}}}\right)^2\right)
	\end{equation}
	
	\subsection{Quantum Computer Benchmarks}
	
	\subsubsection{T0 Quantum Error Rate}
	
	T0 quantum computers should exhibit systematically lower error rates:
	
	\begin{equation}
		\epsilon_{\text{gate}}^{\text{T0}} = \epsilon_{\text{gate}}^{\text{Standard}} \cdot \left(1 - \xipar \frac{E_{\text{gate}}}{\EPlanck}\right)
	\end{equation}
	
	\section{Philosophical Implications of T0 Quantum Mechanics}
	
	\subsection{Determinism vs. Quantum Randomness}
	
	The T0 theory solves the centuries-old problem of quantum randomness:
	
	\subsubsection*{{T0 Determinism}}
\section*{Quantum Randomness as an Illusion:}
		
		What appears as fundamental randomness in standard QM is deterministic time field dynamics in the T0 theory with practically unpredictable, but in principle determined outcomes.
		
		\begin{equation}
			\text{``Randomness''} = \text{Deterministic Time Field Evolution} + \text{Practical Unpredictability}
		\end{equation}

	
	\subsection{Measurement Problem Solved}
	
	The notorious measurement problem of quantum mechanics is resolved by T0 fields:
	
	\begin{itemize}
		\item \textbf{No Collapse:} Wave functions evolve continuously
		\item \textbf{Measurement Devices:} Macroscopic T0 field configurations
		\item \textbf{Definite Outcomes:} Deterministic time field interactions
		\item \textbf{Born Rule:} Emergent from T0 field dynamics
	\end{itemize}
	
	\subsection{Locality and Realism Restored}
	
	The T0 theory restores both locality and realism:
	
	\begin{align}
		\text{Locality:} &\quad \text{All interactions mediated by local T0 fields} \\
		\text{Realism:} &\quad \text{Particles have definite properties before measurement} \\
		\text{Causality:} &\quad \text{No superluminal information transfer}
	\end{align}
	
	\section{Technological Applications}
	
	\subsection{T0 Quantum Computer Architecture}
	
	\subsubsection{Hardware Implementation}
	
	T0 quantum computers could be realized through controlled time field manipulation:
	
	\begin{itemize}
		\item \textbf{Time Field Modulators:} High-frequency electromagnetic fields
		\item \textbf{Energy Field Sensors:} Ultra-precise field measurement devices
		\item \textbf{Coherence Control:} Stabilization through time field feedback
		\item \textbf{Scalability:} Natural decoupling of neighboring qubits
	\end{itemize}
	
	\subsubsection{Quantum Error Correction with T0}
	
	T0-specific error correction codes:
	
	\begin{equation}
		|\psi_{\text{kodiert}}\rangle = \sum_i c_i |i\rangle \otimes |T_{\text{field},i}\rangle
	\end{equation}
	
	The time field acts as a natural syndrome for error detection.
	
	\subsection{Precision Measurement Technology}
	
	\subsubsection{T0-Enhanced Atomic Clocks}
	
	Atomic clocks with T0 corrections could achieve record precision:
	
	\begin{equation}
		\delta f / f_0 = \delta f_{\text{Standard}} / f_0 - \xipar \frac{\Delta E_{\text{Transition}}}{\EPlanck}
	\end{equation}
	
	\subsubsection{Gravitational Wave Detectors}
	
	Improved sensitivity through T0 field calibration:
	
	\begin{equation}
		h_{\text{min}}^{\text{T0}} = h_{\text{min}}^{\text{Standard}} \cdot \left(1 - \xipar \sqrt{f \cdot t_{\text{int}}}\right)
	\end{equation}
	
	\section{Standard Model Extensions}
	
	\subsection{T0-Extended Standard Model}
	
	The complete Standard Model is integrated into the T0 framework:
	
	\begin{equation}
		\mathcal{L}_{\text{SM}}^{\text{T0}} = \mathcal{L}_{\text{SM}} + \mathcal{L}_{\text{T0-Feld}} + \mathcal{L}_{\text{T0-Interaction}}
	\end{equation}
	
	where:
	\begin{align}
		\mathcal{L}_{\text{T0-Feld}} &= \frac{\xipar}{\EPlanck^2} (\partial \Tfield)^2 \\
		\mathcal{L}_{\text{T0-Interaction}} &= \xipar \sum_i g_i \bar{\psi}_i \gamma^\mu \partial_\mu \Tfield \psi_i
	\end{align}
	
	\subsection{Hierarchy Problem Solution}
	
	The notorious hierarchy problem is solved by the T0 structure:
	
	\begin{equation}
		\frac{M_{\text{Planck}}}{M_{\text{EW}}} = \frac{1}{\sqrt{\xipar}} \approx \frac{1}{\sqrt{1.33 \times 10^{-4}}} \approx 87
	\end{equation}
	
	instead of the problematic $10^{16}$ in the Standard Model.
	
	
	\section{Conclusions}
	
	\subsection{Paradigm Shift in Quantum Theory}
	
	The T0 theory represents a fundamental paradigm shift:
	
	\subsubsection*{{T0 Revolution}}
\section*{From Standard QM/QFT to T0 Theory:}
		
		\begin{itemize}
			\item \textbf{Time}: From parameter to dynamic field
			\item \textbf{Quantum Randomness}: From fundamental to emergent-deterministic
			\item \textbf{Measurement Problem}: From philosophical puzzle to physical solution
			\item \textbf{Bell Inequalities}: From non-locality to local reality
			\item \textbf{Quantum Computers}: From probabilistic to deterministic
			\item \textbf{Renormalization}: From artificial cutoffs to natural scales
		\end{itemize}

	
	\subsection{Experimental Verifiability}
	
	The T0 theory makes concrete, testable predictions:
	
	\begin{enumerate}
		\item \textbf{Quantum Mechanics Tests}: Spectroscopic corrections at the $10^{-32}$ eV level
		\item \textbf{Quantum Computer Improvements}: Systematically lower error rates
		\item \textbf{Bell Test Modifications}: Subtle corrections due to time field effects
		\item \textbf{Interferometry}: Phase shifts of $10^{-18}$ rad
		\item \textbf{Gravitational Waves}: Frequency-dependent T0 corrections
	\end{enumerate}
	
	\subsection{Societal Impacts}
	
	The T0 revolution could bring about profound societal changes:
	
	\subsubsection{Technological Breakthroughs}
	
	\begin{itemize}
		\item \textbf{Quantum Computer Supremacy}: Deterministic T0-QC surpasses classical computers
		\item \textbf{Cryptography}: New secure encryption methods based on time field properties
		\item \textbf{Communication}: T0 field-modulated signal transmission
		\item \textbf{Precision Measurements}: Revolutionary improvements in science and industry
	\end{itemize}
	
	\subsubsection{Scientific Worldview}
	
	\begin{itemize}
		\item \textbf{Determinism Restored}: End of fundamentally probabilistic physics
		\item \textbf{Locality Preserved}: No spooky action at a distance required
		\item \textbf{Realism Vindicated}: Physical properties exist objectively
		\item \textbf{Unification}: One parameter ($\xi$) describes all fundamental phenomena
	\end{itemize}
	
	\section{Future Directions}
	
	\subsection{Theoretical Developments}
	
	\subsubsection*{{Open Research Fields}}
\begin{enumerate}
			\item \textbf{Non-Perturbative T0-QFT}: Exact solutions beyond perturbation theory
			\item \textbf{T0-String Theory}: Integration into higher-dimensional frameworks  
			\item \textbf{Cosmological T0 Applications}: Dark energy and matter
			\item \textbf{T0 Quantum Gravity}: Complete unification of all forces
			\item \textbf{Consciousness Interface}: T0 fields and neural activity
		\end{enumerate}

	
	\subsection{Experimental Priorities}
	
	\begin{table}[htbp]
		\centering
		\begin{tabular}{lcc}
			\toprule
			\textbf{Research Area} & \textbf{Priority} & \textbf{Expected Impact} \\
			\midrule
			T0 Quantum Computer Prototype & Very High & Technological Revolution \\
			High-Precision Bell Tests & High & Fundamental Understanding \\
			Atom Interferometry with T0 & High & Direct Field Measurement \\
			Gravitational Wave Analysis & Medium & Cosmological Confirmation \\
			Spectroscopic T0 Search & Medium & Quantum Mechanics Verification \\
			\bottomrule
		\end{tabular}
		\caption{Research Priorities for T0 Theory}
		\label{T0_QM_QFT_RT:L-T0_QM-QFT-RT-0501}
	\end{table}
	
	\subsection{Long-Term Visions}
	
	\subsubsection{T0-Based Civilization}
	
	A fully T0-based technological civilization could be characterized by:
	
	\begin{itemize}
		\item \textbf{Universal Field Control}: Direct manipulation of T0 time fields
		\item \textbf{Deterministic Predictions}: Perfect predictability through complete field information
		\item \textbf{Energy Field Communication}: Instantaneous information via T0 field modulation
		\item \textbf{Consciousness Expansion}: Interface between T0 fields and the human mind
	\end{itemize}
	
	\subsubsection{Fundamental Understanding}
	
	The complete development of the T0 theory could lead to the following:
	
	\begin{align}
		\text{Ultimate Reality} &= \text{Universal T0 Time Field} + \text{Geometric Structures} \\
		\text{All Physics} &= \text{Various Manifestations of } \xi\text{-modulated Fields} \\
		\text{Consciousness} &= \text{Complex T0 Field Configurations in the Brain}
	\end{align}
	
	\section{Critical Evaluation and Limitations}
	
	
	
	\subsection{Experimental Challenges}
	
	The experimental verification of the T0 theory requires:
	
	\begin{itemize}
		\item \textbf{Ultra-High Precision}: Measurements at the $10^{-18}$-$10^{-32}$ level
		\item \textbf{New Technologies}: T0 field-specific measurement devices
		\item \textbf{Long-Term Stability}: Consistent measurements over years
		\item \textbf{Systematic Control}: Elimination of all other effects
	\end{itemize}
	
	\subsection{Philosophical Implications}
	
	The T0 theory raises profound philosophical questions:
	
	\begin{itemize}
		\item \textbf{Free Will}: Is determinism compatible with human freedom of decision?
		\item \textbf{Epistemology}: How can we fully recognize the T0 reality?
		\item \textbf{Reductionism}: Are all phenomena reducible to T0 fields?
		\item \textbf{Emergence}: What role do emergent properties play?
	\end{itemize}
	
	\section{Conclusion: The T0 Revolution}
	
	The T0 Quantum Field Theory and its extensions to quantum mechanics and quantum computer technology may represent the most significant theoretical development since Einstein. The theory:
	
	\begin{itemize}
		\item \textbf{Unifies} all fundamental areas of physics
		\item \textbf{Solves} long-standing conceptual problems
		\item \textbf{Makes} concrete experimental predictions
		\item \textbf{Enables} revolutionary technologies
		\item \textbf{Changes} our fundamental worldview
	\end{itemize}
	
	The coming decades will show whether this theoretical vision withstands reality. The experimental verification of T0 predictions will not only revolutionize our understanding of physics, but could transform the entire human civilization.
	
	\subsubsection*{{Closing Remarks}}
The T0 theory shows that nature may be much more elegant, deterministic, and comprehensible than current physics suggests. A single parameter $\xi$ could be the key to everything – from quantum mechanics to cosmology, from consciousness to technology.
		
\section*{The future of physics is T0.}

	
	


% Bibliography
\begin{thebibliography}{99}
	
	\bibitem{pdg2024}
	Particle Data Group Collaboration (2024). 
	\textit{Review of Particle Physics}. 
	Progress of Theoretical and Experimental Physics, 2024(8), 083C01.
	\url{https://pdg.lbl.gov}
	
	\bibitem{flag2024}
	Aoki, Y., et al. (FLAG Collaboration) (2024). 
	\textit{FLAG Review 2024 of Lattice Results for Low-Energy Constants}. 
	arXiv:2411.04268.
	\url{https://arxiv.org/abs/2411.04268}
	
	\bibitem{fermilab_muon_g2}
	Abi, B., et al. (Muon g-2 Collaboration) (2021). 
	\textit{Measurement of the Positive Muon Anomalous Magnetic Moment to 0.46 ppm}. 
	Physical Review Letters, 126, 141801.
	
	\bibitem{peskin_schroeder}
	Peskin, M. E., \& Schroeder, D. V. (1995). 
	\textit{An Introduction to Quantum Field Theory}. 
	Addison-Wesley.
	
	\bibitem{weinberg_qft}
	Weinberg, S. (1995). 
	\textit{The Quantum Theory of Fields, Vol. I--III}. 
	Cambridge University Press.
	
	\bibitem{griffiths_particle}
	Griffiths, D. (2008). 
	\textit{Introduction to Elementary Particles}. 
	Wiley-VCH.
	
	\bibitem{mandl_shaw}
	Mandl, F., \& Shaw, G. (2010). 
	\textit{Quantum Field Theory (2nd ed.)}. 
	Wiley.
	
	\bibitem{srednicki_qft}
	Srednicki, M. (2007). 
	\textit{Quantum Field Theory}. 
	Cambridge University Press.
	
	\bibitem{t0_fundamentals}
	Pascher, J. (2024). 
	\textit{T0-Theory: Foundations of Time-Mass Duality}. 
	Unpublished manuscript, HTL Leonding.
	
	\bibitem{t0_fine_structure}
	Pascher, J. (2024). 
	\textit{T0-Theory: The Fine Structure Constant}. 
	Unpublished manuscript, HTL Leonding.
	
	\bibitem{t0_neutrinos}
	Pascher, J. (2024). 
	\textit{T0-Theory: Neutrino Masses and PMNS Mixing}. 
	Unpublished manuscript, HTL Leonding.
	
	\bibitem{t0_github}
	Pascher, J. (2024--2025). 
	\textit{T0-Time-Mass-Duality Repository}. 
	GitHub.
	\url{https://github.com/jpascher/T0-Time-Mass-Duality}
	
	\bibitem{lattice_qcd_review}
	Kronfeld, A. S. (2012). 
	\textit{Twenty-first Century Lattice Gauge Theory: Results from the QCD Lagrangian}. 
	Annual Review of Nuclear and Particle Science, 62, 265--284.
	
	\bibitem{neutrino_mixing_pdg}
	Particle Data Group Collaboration (2024). 
	\textit{Neutrino Masses, Mixing, and Oscillations}. 
	PDG Review 2024.
	\url{https://pdg.lbl.gov/2024/reviews/rpp2024-rev-neutrino-mixing.pdf}
	
	\bibitem{higgs_discovery}
	ATLAS and CMS Collaborations (2012). 
	\textit{Observation of a New Particle in the Search for the Standard Model Higgs Boson}. 
	Physics Letters B, 716, 1--29.
	
	\bibitem{Brannen2005}
	C. P. Brannen, ``Estimate of neutrino masses from Koide's relation'', \textit{arXiv:hep-ph/0505028} (2005).
	\url{https://arxiv.org/abs/hep-ph/0505028}
	
	\bibitem{Brannen2006}
	C. P. Brannen, ``Koide Mass Formula for Neutrinos'', \textit{arXiv:0702.0052} (2006).
	\url{http://brannenworks.com/MASSES.pdf}
	
	\bibitem{PhaseVectors2025}
	Anonymous, ``The Koide Relation and Lepton Mass Hierarchy from Phase Vectors'', \textit{rXiv:2507.0040} (2025).
	\url{https://rxiv.org/pdf/2507.0040v1.pdf}
	
	\bibitem{PDG2025}
	Particle Data Group, ``Review of Particle Physics'', \textit{Phys. Rev. D} \textbf{112} (2025) 030001.
	\url{https://pdg.lbl.gov/2025/}
	
	\bibitem{terrell2024}
	Terrell et al. (2024). 
	\textit{Single-Clock Metrology in Nature}. 
	Nature Physics.
	
	\bibitem{hossenfelder2024}
	Hossenfelder, S. (2024). 
	\textit{Single Clock Video Explanation}. 
	YouTube.
	
	\bibitem{hundert1931}
	Hundert (1931). 
	\textit{Reference Work}. 
	Publisher.
	
	\bibitem{terrell2025}
	Terrell et al. (2025). 
	\textit{Advanced Clock Synchronization Methods}. 
	Physical Review Letters.
	
	\bibitem{pascher_t0_2025}
	Pascher, J. (2025). 
	\textit{T0-Theory: Complete Framework and Applications}. 
	Unpublished manuscript, HTL Leonding.
	
	\bibitem{t0qm}
	Pascher, J. (2024). 
	\textit{T0-Theory: Quantum Mechanics Formulation}. 
	Unpublished manuscript, HTL Leonding.
	
	\bibitem{t0anomale}
	Pascher, J. (2024). 
	\textit{T0-Theory: Anomalous Magnetic Moments}. 
	Unpublished manuscript, HTL Leonding.
	
	\bibitem{muong2complete}
	Abi, B., et al. (Muon g-2 Collaboration) (2023). 
	\textit{Complete Measurement of the Positive Muon Anomalous Magnetic Moment}. 
	Physical Review Letters, 131, 161802.
	
	\bibitem{penrose2004}
	Penrose, R. (2004). 
	\textit{The Road to Reality: A Complete Guide to the Laws of the Universe}. 
	Jonathan Cape.
	
	\bibitem{planck1900}
	Planck, M. (1900). 
	\textit{On the Theory of the Energy Distribution Law of the Normal Spectrum}. 
	Verhandlungen der Deutschen Physikalischen Gesellschaft, 2, 237.
	
	\bibitem{T0Theory}
	Pascher, J. (2024). 
	\textit{T0-Theory: Fundamental Principles}. 
	Unpublished manuscript, HTL Leonding.
	
	% Additional bibliography entries for all undefined citations
	\bibitem{6g_roadmap}
	6G Research Consortium (2024).
	\textit{6G Technology Roadmap}.
	Technical Report.
	
	\bibitem{Born2013}
	Born, M. (2013).
	\textit{Einstein's Theory of Relativity}.
	Dover Publications.
	
	\bibitem{Casimir1948}
	Casimir, H. B. G. (1948).
	\textit{On the attraction between two perfectly conducting plates}.
	Proc. Kon. Ned. Akad. Wetensch. B51, 793--795.
	
	\bibitem{Einstein1905}
	Einstein, A. (1905).
	\textit{On the Electrodynamics of Moving Bodies}.
	Annalen der Physik, 17, 891--921.
	
	\bibitem{Feynman2006}
	Feynman, R. P. (2006).
	\textit{QED: The Strange Theory of Light and Matter}.
	Princeton University Press.
	
	\bibitem{Griffiths2017}
	Griffiths, D. J. (2017).
	\textit{Introduction to Electrodynamics (4th ed.)}.
	Cambridge University Press.
	
	\bibitem{Jackson1999}
	Jackson, J. D. (1999).
	\textit{Classical Electrodynamics (3rd ed.)}.
	Wiley.
	
	\bibitem{Mohr2016}
	Mohr, P. J., et al. (2016).
	\textit{CODATA Recommended Values of the Fundamental Physical Constants: 2014}.
	Rev. Mod. Phys. 88, 035009.
	
	\bibitem{Parker2018}
	Parker, R. H., et al. (2018).
	\textit{Measurement of the fine-structure constant as a test of the Standard Model}.
	Science, 360, 191--195.
	
	\bibitem{Planck1900}
	Planck, M. (1900).
	\textit{On the Theory of the Energy Distribution Law of the Normal Spectrum}.
	Verhandlungen der Deutschen Physikalischen Gesellschaft, 2, 237.
	
	\bibitem{Planck2018}
	Planck Collaboration (2018).
	\textit{Planck 2018 results. VI. Cosmological parameters}.
	Astronomy \& Astrophysics, 641, A6.
	
	\bibitem{QFT_T0}
	Pascher, J. (2024).
	\textit{T0-Theory and QFT Connections}.
	Unpublished manuscript, HTL Leonding.
	
	\bibitem{Sommerfeld1916}
	Sommerfeld, A. (1916).
	\textit{On the Quantum Theory of Spectral Lines}.
	Annalen der Physik, 51, 1--94.
	
	\bibitem{T0_Feinstruktur}
	Pascher, J. (2024).
	\textit{T0-Theory: Fine Structure Analysis}.
	Unpublished manuscript, HTL Leonding.
	
	\bibitem{T0_SI}
	Pascher, J. (2024).
	\textit{T0-Theory and SI Units}.
	Unpublished manuscript, HTL Leonding.
	
	\bibitem{T0_fine_structure}
	Pascher, J. (2024).
	\textit{T0-Theory: The Fine Structure Constant}.
	Unpublished manuscript, HTL Leonding.
	
	\bibitem{T0_g2_erweiterung}
	Pascher, J. (2024).
	\textit{T0-Theory: g-2 Extensions}.
	Unpublished manuscript, HTL Leonding.
	
	\bibitem{T0_gravitational_constant}
	Pascher, J. (2024).
	\textit{T0-Theory: Gravitational Constant Derivation}.
	Unpublished manuscript, HTL Leonding.
	
	\bibitem{T0_netze_en}
	Pascher, J. (2024).
	\textit{T0-Theory: Network Structures}.
	Unpublished manuscript, HTL Leonding.
	
	\bibitem{T0_tm_erweiterung}
	Pascher, J. (2024).
	\textit{T0-Theory: Time-Mass Extensions}.
	Unpublished manuscript, HTL Leonding.
	
	\bibitem{Uzan2003}
	Uzan, J.-P. (2003).
	\textit{The fundamental constants and their variation}.
	Rev. Mod. Phys. 75, 403--455.
	
	\bibitem{Weinberg1995}
	Weinberg, S. (1995).
	\textit{The Quantum Theory of Fields, Vol. I}.
	Cambridge University Press.
	
	\bibitem{albrecht1999}
	Albrecht, A. \& Magueijo, J. (1999).
	\textit{A time varying speed of light as a solution to cosmological puzzles}.
	Phys. Rev. D 59, 043516.
	
	\bibitem{alice2023}
	ALICE Collaboration (2023).
	\textit{Recent results from ALICE}.
	CERN-EP-2023-XXX.
	
	\bibitem{analog_optical}
	Smith, J. et al. (2024).
	\textit{Analog optical computing systems}.
	Nature Photonics.
	
	\bibitem{ashtekar2004}
	Ashtekar, A. \& Lewandowski, J. (2004).
	\textit{Background independent quantum gravity}.
	Class. Quantum Grav. 21, R53.
	
	\bibitem{atlas2023}
	ATLAS Collaboration (2023).
	\textit{ATLAS physics results}.
	CERN-PH-EP-2023-XXX.
	
	\bibitem{atlas2023higgs}
	ATLAS Collaboration (2023).
	\textit{Higgs boson measurements}.
	Phys. Rev. Lett.
	
	\bibitem{barbour1999}
	Barbour, J. (1999).
	\textit{The End of Time}.
	Oxford University Press.
	
	\bibitem{barrow1999}
	Barrow, J. D. (1999).
	\textit{Cosmologies with varying light speed}.
	Phys. Rev. D 59, 043515.
	
	\bibitem{becker2007}
	Becker, K. et al. (2007).
	\textit{String Theory and M-Theory}.
	Cambridge University Press.
	
	\bibitem{bell_muon}
	Bennett, G. W., et al. (Muon g-2 Collaboration) (2006).
	\textit{Final report of the E821 muon anomalous magnetic moment measurement}.
	Phys. Rev. D 73, 072003.
	
	\bibitem{bondi1948}
	Bondi, H. \& Gold, T. (1948).
	\textit{The steady-state theory of the expanding universe}.
	Mon. Not. R. Astron. Soc. 108, 252--270.
	
	\bibitem{brewer2019}
	Brewer, S. M. et al. (2019).
	\textit{Al+ Quantum-Logic Clock with Systematic Uncertainty below $10^{-18}$}.
	Phys. Rev. Lett. 123, 033201.
	
	\bibitem{cms2023top}
	CMS Collaboration (2023).
	\textit{Top quark measurements at CMS}.
	JHEP 2023.
	
	\bibitem{cms2024}
	CMS Collaboration (2024).
	\textit{CMS physics results 2024}.
	CERN-PH-EP-2024-XXX.
	
	\bibitem{codata2019}
	Tiesinga, E. et al. (2019).
	\textit{The 2018 CODATA Recommended Values}.
	J. Phys. Chem. Ref. Data.
	
	\bibitem{desi2025}
	DESI Collaboration (2025).
	\textit{DESI 2025 Cosmology Results}.
	arXiv preprint.
	
	\bibitem{differential_optical}
	Wang, X. et al. (2024).
	\textit{Differential optical computing}.
	Optica.
	
	\bibitem{dingle1972}
	Dingle, H. (1972).
	\textit{Science at the Crossroads}.
	Martin Brian \& O'Keeffe.
	
	\bibitem{divalentino2021}
	Di Valentino, E. et al. (2021).
	\textit{In the realm of the Hubble tension}.
	Class. Quantum Grav. 38, 153001.
	
	\bibitem{elnaschie2004}
	El Naschie, M. S. (2004).
	\textit{A review of E infinity theory}.
	Chaos, Solitons \& Fractals, 19, 209--236.
	
	\bibitem{fabrication_heterogeneous}
	Chen, Y. et al. (2024).
	\textit{Heterogeneous photonic integration}.
	Nature Electronics.
	
	\bibitem{fermilab2023}
	Fermilab (2023).
	\textit{Muon g-2 results}.
	Phys. Rev. Lett.
	
	\bibitem{flexible_wafer}
	Kim, S. et al. (2024).
	\textit{Flexible wafer-scale photonics}.
	Science Advances.
	
	\bibitem{francesco1997}
	Di Francesco, P. et al. (1997).
	\textit{Conformal Field Theory}.
	Springer.
	
	\bibitem{hartree1957}
	Hartree, D. R. (1957).
	\textit{The Calculation of Atomic Structures}.
	Wiley.
	
	\bibitem{hhi_6g}
	Fraunhofer HHI (2024).
	\textit{6G Photonic Integration}.
	Technical Report.
	
	\bibitem{hossenfelder2025}
	Hossenfelder, S. (2025).
	\textit{Science without the gobbledygook}.
	YouTube/Blog.
	
	\bibitem{hossenfelder_single_clock_video}
	Hossenfelder, S. (2024).
	\textit{The Single Clock Problem}.
	YouTube.
	
	\bibitem{hoyle1948}
	Hoyle, F. (1948).
	\textit{A new model for the expanding universe}.
	Mon. Not. R. Astron. Soc. 108, 372--382.
	
	\bibitem{integration_microelectronic}
	Liu, A. et al. (2024).
	\textit{Microelectronic photonic integration}.
	IEEE Journal.
	
	\bibitem{jacobson1995}
	Jacobson, T. (1995).
	\textit{Thermodynamics of spacetime}.
	Phys. Rev. Lett. 75, 1260.
	
	\bibitem{kasevich2023}
	Kasevich, M. et al. (2023).
	\textit{Atom interferometry tests}.
	Nature Physics.
	
	\bibitem{lerner2014}
	Lerner, E. J. (2014).
	\textit{An open letter on cosmology}.
	New Scientist.
	
	\bibitem{lisa2017}
	LISA Consortium (2017).
	\textit{Laser Interferometer Space Antenna}.
	ESA Technical Report.
	
	\bibitem{lithium_tantalate}
	Zhang, M. et al. (2024).
	\textit{Thin-film lithium tantalate photonics}.
	Nature Photonics.
	
	\bibitem{lopez2010}
	Lopez-Corredoira, M. (2010).
	\textit{Tests and problems of the standard model in cosmology}.
	Int. J. Mod. Phys. D.
	
	\bibitem{ludlow2015}
	Ludlow, A. D. et al. (2015).
	\textit{Optical atomic clocks}.
	Rev. Mod. Phys. 87, 637.
	
	\bibitem{mach1883}
	Mach, E. (1883).
	\textit{Die Mechanik in ihrer Entwickelung}.
	F.A. Brockhaus.
	
	\bibitem{maldacena1998}
	Maldacena, J. (1998).
	\textit{The large N limit of superconformal field theories}.
	Adv. Theor. Math. Phys. 2, 231--252.
	
	\bibitem{mueller2014}
	Müller, H. et al. (2014).
	\textit{Atom interferometry tests of the gravitational redshift}.
	Phys. Rev. Lett.
	
	\bibitem{mug2_final_2025}
	Muon g-2 Collaboration (2025).
	\textit{Final muon g-2 measurement}.
	Phys. Rev. Lett.
	
	\bibitem{muong2_2023}
	Muon g-2 Collaboration (2023).
	\textit{Updated muon g-2 results}.
	Phys. Rev. Lett.
	
	\bibitem{nathan2024}
	Nathan, A. et al. (2024).
	\textit{Quantum computing advances}.
	Nature.
	
	\bibitem{newell2018}
	Newell, D. B. et al. (2018).
	\textit{The CODATA 2017 values of h, e, k, and $N_A$}.
	Metrologia 55, L13.
	
	\bibitem{nottale1993}
	Nottale, L. (1993).
	\textit{Fractal Space-Time and Microphysics}.
	World Scientific.
	
	\bibitem{on_chip_lithium}
	Wang, C. et al. (2024).
	\textit{On-chip lithium niobate photonics}.
	Nature Communications.
	
	\bibitem{optical_advantages}
	Shastri, B. J. et al. (2024).
	\textit{Advantages of optical computing}.
	Nature Reviews Physics.
	
	\bibitem{pascher2025cmb}
	Pascher, J. (2025).
	\textit{T0-Theory: CMB Analysis}.
	Unpublished manuscript, HTL Leonding.
	
	\bibitem{pascher2025g2}
	Pascher, J. (2025).
	\textit{T0-Theory: g-2 Predictions}.
	Unpublished manuscript, HTL Leonding.
	
	\bibitem{pascher2025qm}
	Pascher, J. (2025).
	\textit{T0-Theory: Quantum Mechanics}.
	Unpublished manuscript, HTL Leonding.
	
	\bibitem{pascher2025si}
	Pascher, J. (2025).
	\textit{T0-Theory: SI Unit System}.
	Unpublished manuscript, HTL Leonding.
	
	\bibitem{pascher2025t0}
	Pascher, J. (2025).
	\textit{T0-Theory: Complete Framework}.
	Unpublished manuscript, HTL Leonding.
	
	\bibitem{pascher:fundamentals}
	Pascher, J. (2024).
	\textit{T0-Theory: Fundamentals}.
	Unpublished manuscript, HTL Leonding.
	
	\bibitem{pascher:g2_rev9}
	Pascher, J. (2024).
	\textit{T0-Theory: g-2 Revision 9}.
	Unpublished manuscript, HTL Leonding.
	
	\bibitem{pascher:geometric_formalism}
	Pascher, J. (2024).
	\textit{T0-Theory: Geometric Formalism}.
	Unpublished manuscript, HTL Leonding.
	
	\bibitem{pascher:ml_addendum}
	Pascher, J. (2024).
	\textit{T0-Theory: Machine Learning Addendum}.
	Unpublished manuscript, HTL Leonding.
	
	\bibitem{pascher:t0_foundations}
	Pascher, J. (2024).
	\textit{T0-Theory: Foundations}.
	Unpublished manuscript, HTL Leonding.
	
	\bibitem{pascher_derivation_beta_2025}
	Pascher, J. (2025).
	\textit{T0-Theory: Derivation of Beta}.
	Unpublished manuscript, HTL Leonding.
	
	\bibitem{pascher_higgs_connection_2025}
	Pascher, J. (2025).
	\textit{T0-Theory: Higgs Connection}.
	Unpublished manuscript, HTL Leonding.
	
	\bibitem{pascher_lagrangian_extended_2025}
	Pascher, J. (2025).
	\textit{T0-Theory: Extended Lagrangian}.
	Unpublished manuscript, HTL Leonding.
	
	\bibitem{pascher_mathematical_structure_2025}
	Pascher, J. (2025).
	\textit{T0-Theory: Mathematical Structure}.
	Unpublished manuscript, HTL Leonding.
	
	\bibitem{pascher_t0_cmb_2025}
	Pascher, J. (2025).
	\textit{T0-Theory: CMB Predictions}.
	Unpublished manuscript, HTL Leonding.
	
	\bibitem{pascher_t0_energie_2025}
	Pascher, J. (2025).
	\textit{T0-Theory: Energy}.
	Unpublished manuscript, HTL Leonding.
	
	\bibitem{pascher_t0_energy_2025}
	Pascher, J. (2025).
	\textit{T0-Theory: Energy Framework}.
	Unpublished manuscript, HTL Leonding.
	
	\bibitem{pascher_t0_theory_2025}
	Pascher, J. (2025).
	\textit{T0-Theory: Complete Theory}.
	Unpublished manuscript, HTL Leonding.
	
	\bibitem{penrose1959}
	Penrose, R. (1959).
	\textit{The apparent shape of a relativistically moving sphere}.
	Proc. Cambridge Phil. Soc. 55, 137--139.
	
	\bibitem{penrose1967}
	Penrose, R. (1967).
	\textit{Twistor algebra}.
	J. Math. Phys. 8, 345--366.
	
	\bibitem{peratt1992}
	Peratt, A. L. (1992).
	\textit{Physics of the Plasma Universe}.
	Springer-Verlag.
	
	\bibitem{peskin1995}
	Peskin, M. E. \& Schroeder, D. V. (1995).
	\textit{An Introduction to Quantum Field Theory}.
	Addison-Wesley.
	
	\bibitem{peskin_schroeder_1995}
	Peskin, M. E. \& Schroeder, D. V. (1995).
	\textit{An Introduction to Quantum Field Theory}.
	Addison-Wesley.
	
	\bibitem{phoquant}
	PhoQuant (2024).
	\textit{Photonic quantum computing}.
	Technical Report.
	
	\bibitem{photonics_ai}
	Wetzstein, G. et al. (2024).
	\textit{Photonics for AI}.
	Nature.
	
	\bibitem{planck1906}
	Planck, M. (1906).
	\textit{The Theory of Heat Radiation}.
	Johann Ambrosius Barth.
	
	\bibitem{planck2018}
	Planck Collaboration (2018).
	\textit{Planck 2018 results}.
	A\&A 641, A6.
	
	\bibitem{polchinski1998}
	Polchinski, J. (1998).
	\textit{String Theory}.
	Cambridge University Press.
	
	\bibitem{qant_nps}
	QANT (2024).
	\textit{Quantum photonics systems}.
	Technical Report.
	
	\bibitem{quantenjahr25}
	Quantenjahr (2025).
	\textit{International Year of Quantum}.
	UNESCO.
	
	\bibitem{recurrent_photonics}
	Tait, A. N. et al. (2024).
	\textit{Recurrent photonic neural networks}.
	Optica.
	
	\bibitem{rf_photonics}
	Capmany, J. \& Novak, D. (2024).
	\textit{Microwave photonics}.
	Nature Photonics.
	
	\bibitem{riess2019}
	Riess, A. G. et al. (2019).
	\textit{Large Magellanic Cloud Cepheid Standards}.
	ApJ 876, 85.
	
	\bibitem{riess2022}
	Riess, A. G. et al. (2022).
	\textit{A Comprehensive Measurement of H0}.
	ApJ 934, L7.
	
	\bibitem{rovelli2004}
	Rovelli, C. (2004).
	\textit{Quantum Gravity}.
	Cambridge University Press.
	
	\bibitem{sciama1953}
	Sciama, D. W. (1953).
	\textit{On the origin of inertia}.
	Mon. Not. R. Astron. Soc. 113, 34--42.
	
	\bibitem{sciencedaily2025}
	ScienceDaily (2025).
	\textit{Physics news}.
	Online.
	
	\bibitem{sm_g2_2025}
	Aoyama, T. et al. (2025).
	\textit{Standard Model prediction for g-2}.
	Phys. Rep.
	
	\bibitem{susskind1995}
	Susskind, L. (1995).
	\textit{The world as a hologram}.
	J. Math. Phys. 36, 6377--6396.
	
	\bibitem{t0_kosmologie}
	Pascher, J. (2024).
	\textit{T0-Theory: Cosmology}.
	Unpublished manuscript, HTL Leonding.
	
	\bibitem{terrell1959}
	Terrell, J. (1959).
	\textit{Invisibility of the Lorentz contraction}.
	Phys. Rev. 116, 1041--1045.
	
	\bibitem{terrell_single_clock_nature_2024}
	Terrell, J. et al. (2024).
	\textit{Single clock precision measurements}.
	Nature Physics.
	
	\bibitem{tfln_foundry}
	TFLN Foundry (2024).
	\textit{Thin-film lithium niobate foundry services}.
	Technical Specifications.
	
	\bibitem{thiemann2007}
	Thiemann, T. (2007).
	\textit{Modern Canonical Quantum General Relativity}.
	Cambridge University Press.
	
	\bibitem{thz_epfl}
	EPFL (2024).
	\textit{Terahertz photonics research}.
	Technical Report.
	
	\bibitem{unnikrishnan2004}
	Unnikrishnan, C. S. (2004).
	\textit{On Einstein's resolution of the twin clock paradox}.
	Current Science, 86, 704--709.
	
	\bibitem{verlinde2011}
	Verlinde, E. (2011).
	\textit{On the origin of gravity and the laws of Newton}.
	JHEP 2011, 29.
	
	\bibitem{video2025}
	Video (2025).
	\textit{Physics video explanation}.
	YouTube.
	
	\bibitem{weinberg1995}
	Weinberg, S. (1995).
	\textit{The Quantum Theory of Fields}.
	Cambridge University Press.
	
	\bibitem{weiskopf2000}
	Weiskopf, D. (2000).
	\textit{Visualization of special relativity}.
	PhD thesis, University of Tübingen.
	
	\bibitem{wheeler1990}
	Wheeler, J. A. (1990).
	\textit{A Journey into Gravity and Spacetime}.
	Scientific American Library.
	
	\bibitem{wiki_bell}
	Wikipedia (2024).
	\textit{Bell's theorem}.
	Online encyclopedia.
	
	\bibitem{zwicky1929}
	Zwicky, F. (1929).
	\textit{On the red shift of spectral lines through interstellar space}.
	Proc. Natl. Acad. Sci. 15, 773--779.

\end{thebibliography}


\end{document}
