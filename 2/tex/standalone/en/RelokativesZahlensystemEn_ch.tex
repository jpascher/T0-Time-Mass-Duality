\documentclass[11pt,a4paper]{article}
\usepackage[a4paper,margin=2cm]{geometry}
\usepackage[utf8]{inputenc}
\usepackage[english]{babel}
\usepackage{lmodern}
\renewcommand{\familydefault}{\sfdefault}

\usepackage{amsmath,amssymb,amsthm}
\usepackage{graphicx}
\usepackage[unicode,pdfencoding=auto,hypertexnames=false]{hyperref}
\usepackage{booktabs}
\usepackage{longtable}
\usepackage{array}
\usepackage{siunitx}
\usepackage{fancyhdr}
\usepackage{float}
\usepackage{tikz}
% tcolorbox removed for standalone
% tcbset removed
\tikzset{
  t0blue/.style={draw=blue,fill=blue!10},
  t0red/.style={draw=red,fill=red!10},
  t0green/.style={draw=green!50!black,fill=green!10},
  t0orange/.style={draw=orange,fill=orange!10},
}
\usepackage{setspace}
\usepackage{enumitem}
\usepackage{adjustbox}
\usepackage{xcolor}

% Define colors for xcolor package
\definecolor{t0green}{RGB}{34,139,34}
\definecolor{t0blue}{RGB}{0,0,255}
\definecolor{t0red}{RGB}{255,0,0}
\definecolor{t0orange}{RGB}{255,165,0}

% Define custom column types for tables
\newcolumntype{L}[1]{>{\raggedright\arraybackslash}p{#1}}
\newcolumntype{C}[1]{>{\centering\arraybackslash}p{#1}}
\newcolumntype{R}[1]{>{\raggedleft\arraybackslash}p{#1}}

\setlength{\parindent}{0pt}
\setlength{\parskip}{6pt}

\hypersetup{
  colorlinks=true,
  linkcolor=blue,
  citecolor=blue,
  urlcolor=blue
}
\pagestyle{fancy}
\setlength{\headheight}{28pt}

\newcommand{\checkmarkx}{\checkmark}
\newcommand{\warningx}{\textbf{!}}

% Makros aus Einzel-Dokumenten (Fallback-Definitionen)
\newcommand{\mytimes}{\times}
\newcommand{\myapprox}{\approx}
\newcommand{\mysim}{\sim}
\newcommand{\myomega}{\omega}
\newcommand{\mypi}{\pi}
\newcommand{\myrightarrow}{\rightarrow}
\newcommand{\mypropto}{\propto}
\newcommand{\deltafield}{\delta\phi}
\newcommand{\xipar}{\xi}
\newcommand{\xiT}{\xi}
\newcommand{\lambdah}{\lambda_h}

% Additional macros used in chapter files
\newcommand{\Kfrak}{K_{\text{frak}}}  % Fractal correction factor
\newcommand{\Dfrak}{D_f}              % Fractal dimension
\newcommand{\betapar}{\beta}          % T0 beta parameter
\newcommand{\alphapar}{\alpha}        % T0 alpha parameter
\newcommand{\Efield}{E}               % Energy field
% Note: checkmarkxa/warningxa are variants used in auto-generated chapter files
\newcommand{\checkmarkxa}{\checkmark}
\newcommand{\warningxa}{\textbf{!}}

% Additional T0-specific macros
\newcommand{\xigeom}{\xi_{\text{geom}}}  % Geometric xi
\newcommand{\lP}{\ell_P}                  % Planck length
\newcommand{\rzero}{r_0}                  % Characteristic radius
\newcommand{\xirat}{\xi_{\text{rat}}}     % Xi ratio
\newcommand{\tzero}{t_0}                  % Characteristic time
\newcommand{\natunits}{\text{(nat. units)}}  % Natural units annotation
\newcommand{\myRightarrow}{\Rightarrow}   % Arrow variant
\newcommand{\Lag}{\mathcal{L}}            % Lagrangian

% Physics macros used in chapter files
\newcommand{\CQCD}{C_{\text{QCD}}}        % QCD correction
\newcommand{\EP}{E_P}                     % Planck energy
\newcommand{\Ee}{E_e}                     % Electron energy
\newcommand{\Emu}{E_\mu}                  % Muon energy
\newcommand{\Exi}{E_\xi}                  % Xi energy
\newcommand{\Ezero}{E_0}                  % Characteristic energy
\newcommand{\Hubble}{H}                   % Hubble constant
\newcommand{\Kspec}{K_{\text{spec}}}      % Spectral correction
\newcommand{\Lambdat}{\Lambda_t}          % Time-related cosmological constant
\newcommand{\Leff}{\mathcal{L}_{\text{eff}}}  % Effective Lagrangian
\newcommand{\Lorentz}{\mathcal{L}}        % Lorentz symbol
\newcommand{\Lxi}{L_\xi}                  % Xi length
\newcommand{\Tfield}{T}                   % Time field
\newcommand{\Weyl}{W}                     % Weyl tensor/symbol
\newcommand{\alphaEMSI}{\alpha_{\text{EM,SI}}}  % EM alpha in SI
\newcommand{\alphaEMnat}{\alpha_{\text{EM,nat}}}  % EM alpha in natural units
\newcommand{\alphaem}{\alpha_{\text{em}}} % Electromagnetic alpha
\newcommand{\betaTSI}{\beta_{T,\text{SI}}}  % Beta in SI
\newcommand{\betaTnat}{\beta_{T,\text{nat}}}  % Beta in natural units
\newcommand{\deltam}{\delta m}            % Mass difference
\newcommand{\phiT}{\phi_T}                % T-field phi
\newcommand{\tP}{t_P}                     % Planck time
\newcommand{\rhoCMB}{\rho_{\text{CMB}}}   % CMB density
\newcommand{\rhoCasimir}{\rho_{\text{Casimir}}}  % Casimir density

% Table formatting
\usepackage{multirow}

% Additional physics macros
\newcommand{\Riem}{\mathcal{R}}           % Riemann tensor
\newcommand{\ZPinch}{Z_{\text{pinch}}}    % Z-pinch
\newcommand{\SynchPower}{P_{\text{synch}}} % Synchrotron power
\newcommand{\Rzero}{R_0}                  % Characteristic radius
\newcommand{\alphafine}{\alpha}           % Fine structure constant
\newcommand{\Etau}{E_\tau}                % Tau energy
\newcommand{\deltaE}{\delta E}            % Energy deviation
\newcommand{\EPlanck}{E_P}                % Planck energy
\newcommand{\pichar}{\pi}                 % Pi character
\newcommand{\alphaWSI}{\alpha_{W,\text{SI}}}  % Wien alpha in SI
\newcommand{\alphaWnat}{\alpha_{W,\text{nat}}}  % Wien alpha in natural units

% Einfache abstract-Umgebung für Kapitel:
\newenvironment{abstract}{%
  \begin{center}\bfseries Abstract\end{center}\small
}{\par}


\title{RelokativesZahlensystemEn}
\author{J. Pascher}
\date{\today}

\begin{document}
\maketitle

\section*{Relokativeszahlensystemen (RelokativesZahlensystemEn)}

	\begin{abstract}
		Prime numbers correspond to ratios in an alternative number system that is fundamentally more basic than our familiar set-based system. This document develops a relational number system in which prime numbers are defined as elementary, indivisible ratios or proportional transformations. By shifting the reference point from absolute quantities to pure relations, a system emerges that establishes multiplication as the primary operation and reflects the logarithmic structure of many natural laws.
	\end{abstract}
	
	
	\section{List of Symbols and Notation}
	
	{\small
		\begin{table}[htbp]
			\centering
			\begin{adjustbox}{width=0.98\textwidth}
				\begin{tabular}{lll}
					\toprule
					\textbf{Symbol} & \textbf{Meaning} & \textbf{Notes} \\
					\midrule
					\multicolumn{3}{c}{\textbf{Relational Basic Operations}} \\
					$\primrel{1}$ & Identity relation & $1:1$, starting point of all transformations \\
					$\primrel{2}$ & Doubling relation & $2:1$, elementary scaling \\
					$\primrel{3}$ & Fifth relation & $3:2$, musical fifth \\
					$\primrel{5}$ & Third relation & $5:4$, musical major third \\
					$\primrel{p}$ & Prime number relation & Elementary, indivisible proportion \\
					\midrule
					\multicolumn{3}{c}{\textbf{Interval Representation}} \\
					$I$ & Musical interval & As frequency ratio \\
					$\vect{v}$ & Exponent vector & $(a_1, a_2, a_3, \ldots)$ for $2^{a_1} \cdot 3^{a_2} \cdot 5^{a_3} \cdots$ \\
					$p_i$ & i-th prime number & $p_1=2, p_2=3, p_3=5, p_4=7, \ldots$ \\
					$a_i$ & Exponent of i-th prime & Integer, can be negative \\
					$n\text{-limit}$ & Prime number limitation & System with primes up to $n$ \\
					\midrule
					\multicolumn{3}{c}{\textbf{Operations}} \\
					$\circ$ & Composition of relations & Corresponds to multiplication \\
					$\oplus$ & Addition of exponent vectors & Logarithmic addition \\
					$\log$ & Logarithmic transformation & Multiplication $\to$ addition \\
					$\exp$ & Exponential function & Addition $\to$ multiplication \\
					\midrule
					\multicolumn{3}{c}{\textbf{Transformations}} \\
					$\text{FFT}$ & Fast Fourier Transform & Practical application \\
					$\text{QFT}$ & Quantum Fourier Transform & Quantum algorithm \\
					$\text{Shor}$ & Shor's Algorithm & Prime factorization \\
					\bottomrule
				\end{tabular}
			\end{adjustbox}
			\caption{Symbols and notation of the relational number system}
			\label{RelokativesZahl:L-RelokativesZahlensystemEn-1022}
		\end{table}
	
	
	\section{Introduction: Shifting the Reference Point}
	
	The idea of shifting the reference point to construct a number system based on ratios while reinterpreting the role of prime numbers is the key to a more fundamental understanding of mathematics. \textbf{Prime numbers correspond to ratios in an alternative number system that is fundamentally more basic} than our familiar set-based system.
	
	\subsection{What does shifting the reference point mean?}
	
	Previously, we have thought of the reference point (the denominator in a fraction like $P/X$) often as 1, representing a fixed, absolute unit. However, when we shift the reference point, we no longer think of absolute numerical values, but of \textbf{relational steps or transformations}.
	
	Imagine we define numbers not as three apples, but as the \textbf{relationship or operation} that transforms one quantity into another.
	
	\section{Music as a Model: Intervals as Operations}
	
	In music, an interval (e.g., a fifth, $3/2$) is not just a static ratio, but an \textbf{operation} that transforms one tone into another. When you shift a tone up by a fifth, you multiply its frequency by $3/2$.
	
	\subsection{Musical Intervals as a Ratio System}
	
	In just intonation, intervals are represented as ratios of whole numbers:
	
	\begin{table}[htbp]
		\centering
		\begin{adjustbox}{width=0.85\textwidth}
			\begin{tabular}{lccc}
				\toprule
				\textbf{Interval} & \textbf{Ratio} & \textbf{Prime Factor} & \textbf{Vector} \\
				\midrule
				Octave & $2:1$ & $2^1$ & $(1, 0, 0)$ \\
				Fifth & $3:2$ & $2^{-1} \cdot 3^1$ & $(-1, 1, 0)$ \\
				Fourth & $4:3$ & $2^2 \cdot 3^{-1}$ & $(2, -1, 0)$ \\
				Major third & $5:4$ & $2^{-2} \cdot 5^1$ & $(-2, 0, 1)$ \\
				Minor third & $6:5$ & $2^1 \cdot 3^1 \cdot 5^{-1}$ & $(1, 1, -1)$ \\
				\bottomrule
			\end{tabular}
		\end{adjustbox}
		\caption{Musical intervals in relational representation}
		\label{RelokativesZahl:L-RelokativesZahlensystemEn-1023}
	\end{table}
	
	These ratios can be written as \textbf{products of prime numbers with integer exponents}:
	
	\begin{equation}
		\text{Interval} = 2^a \cdot 3^b \cdot 5^c \cdot 7^d \cdot \ldots
	\end{equation}
	
	Depending on how many prime numbers one allows (2, 3, 5 – or also 7, 11, 13 \ldots), one speaks of a \textbf{5-limit}, \textbf{7-limit} or \textbf{13-limit} system.
	
\section*{Example}
		The major third ($5/4$) can be expressed as $2^{-2} \cdot 5^1$:
		\begin{align}
			\frac{5}{4} &= 2^{-2} \cdot 5^1 \\
			\text{Exponent vector:} \quad &(-2, 0, 1) \text{ for } (2, 3, 5)
		\end{align}
		
		Here this means:
		\begin{itemize}
			\item $2^{-2}$: The prime number 2 appears twice in the denominator
			\item $5^{+1}$: The prime number 5 appears once in the numerator
		\end{itemize}
% end box example
	
	\subsection{Vector Representation of Intervals}
	
	A useful representation is:
	
\section*{Definition}
		\begin{equation}
			I = (a_1, a_2, a_3, \ldots) \text{ with } I = \prod_{i} p_i^{a_i}
		\end{equation}
		
		Where:
		\begin{itemize}
			\item $p_i$: the $i$-th prime number $(2, 3, 5, 7, \ldots)$
			\item $a_i$: integer exponent (can be negative)
		\end{itemize}
% end box definition
	
	This allows a clear \textbf{algebraic structure} for intervals, including addition, inversion, etc. over the exponent vectors.
	
	\subsection{Application: Interval Multiplication = Exponent Addition}
	
\section*{Example}
		A C major chord in the 5-limit system:
		\begin{align}
			\text{C-E-G} &= \primrel{1} \circ \text{Major third} \circ \text{Fifth} \\
			&= (0,0,0) \oplus (-2,0,1) \oplus (-1,1,0) \\
			&= (-3,1,1) \\
			&= \frac{2^{-3} \cdot 3^1 \cdot 5^1}{1} = \frac{15}{8}
		\end{align}
		This shows how complex harmonic structures emerge as compositions of elementary prime relations.
% end box example
	
	\section{Historical Precedents}
	
	The relational number system stands in a long tradition of mathematical-philosophical approaches:
	
	\begin{itemize}
		\item \textbf{Pythagorean harmony doctrine}: The Pythagoreans already recognized that \emph{Everything is number} -- understood as ratio, not as quantity
		\item \textbf{Euler's Tonnetz} (1739): Prime number-based representation of musical intervals in a two-dimensional lattice
		\item \textbf{Grassmann's Ausdehnungslehre} (1844): Multiplication as fundamental operation that creates new geometric objects
		\item \textbf{Dedekind cuts} (1872): Numbers as relations between rational sets
	\end{itemize}
	
	\section{Category-Theoretic Foundation}
	
\section*{Category}
		The relational system can be interpreted as a free monoidal category, where:
		\begin{itemize}
			\item \textbf{Objects} = ratio vectors $\vect{v} = (a_1, a_2, a_3, \ldots)$
			\item \textbf{Morphisms} = proportional transformations between relations
			\item \textbf{Tensor product} $\otimes$ = composition $\circ$ of relations
			\item \textbf{Unit object} = identity relation $\primrel{1}$
		\end{itemize}
		
		This structure makes explicit that the relational system has a natural category-theoretic interpretation.
% end box category
	
	\section{Prime Numbers as Elementary Relations}
	
	If we transfer this musical approach to numbers, we can interpret prime numbers not as independent numbers, but as \textbf{fundamental, irreducible proportional steps or transformations}:
	
	\subsection{The Elementary Ratios}
	
\section*{Definition}
		\begin{align}
			\primrel{1}: \quad &\text{Identity relation } (1:1) \\
			&\text{The state of equality, starting point of all transformations} \\[0.5em]
			\primrel{2}: \quad &\text{Doubling relation } (2:1) \\
			&\text{The elementary gesture of doubling} \\[0.5em]
			\primrel{3}: \quad &\text{Fifth relation } (3:2) \\
			&\text{Fundamental proportional transformation} \\[0.5em]
			\primrel{5}: \quad &\text{Third relation } (5:4) \\
			&\text{Further elementary proportional transformation}
		\end{align}
% end box definition
	
	\subsection{Numbers as Compositions of Ratios}
	
	In a relational system, numbers would not be static quantities, but \textbf{compositions of ratios}:
	
	\begin{itemize}
		\item \textbf{Starting point}: Base unit $(1:1)$
		\item \textbf{Numbers as paths}: Each number is a path of operations
		\begin{itemize}
			\item The number 2: Path of the $2:1$ operation
			\item The number 3: Path of the $3:1$ operation  
			\item The number 6: Path $2:1$ followed by $3:1$
			\item The number 12: $2 \times 2 \times 3$ (three operations)
		\end{itemize}
	\end{itemize}
	
	\section{Axiomatic Foundations}
	
\section*{Axiom}
		For all relations $\primrel{a}, \primrel{b}, \primrel{c}$ in a relational number system:
		\begin{enumerate}
			\item \textbf{Associativity}: $(\primrel{a} \circ \primrel{b}) \circ \primrel{c} = \primrel{a} \circ (\primrel{b} \circ \primrel{c})$
			\item \textbf{Neutral element}: $\exists \primrel{1} \forall \primrel{a}: \primrel{a} \circ \primrel{1} = \primrel{a}$
			\item \textbf{Invertibility}: $\forall \primrel{a} \exists \primrel{a}^{-1}: \primrel{a} \circ \primrel{a}^{-1} = \primrel{1}$
			\item \textbf{Commutativity}: $\primrel{a} \circ \primrel{b} = \primrel{b} \circ \primrel{a}$
		\end{enumerate}
% end box axiom
	
	These axioms establish the relational system as an abelian group under the composition operation $\circ$.
	
	\section{The Fundamental Difference: Addition vs. Multiplication}
	
	\subsection{Addition: The Parts Continue to Exist}
	
	When we add, we essentially bring things together that exist side by side or sequentially. The original components remain preserved in some way:
	
	\begin{itemize}
		\item \textbf{Sets}: $2 + 3 = 5$ apples (original parts recognizable as subsets)
		\item \textbf{Wave superposition}: Frequencies $f_1$ and $f_2$ are still detectable in the spectrum
		\item \textbf{Forces}: Vector addition - both original forces are present
	\end{itemize}
	
	\subsection{Multiplication: Something New Emerges}
	
	With multiplication, something fundamentally different happens. This involves scaling, transformation, or the creation of a new quality:
	
	\begin{itemize}
		\item \textbf{Area calculation}: $2m \times 3m = 6m^2$ (new dimension)
		\item \textbf{Proportional change}: Doubling $\circ$ tripling = sixfolding
		\item \textbf{Musical intervals}: Fifth $\times$ octave = new harmonic position
	\end{itemize}
	
	\section{The Power of the Logarithm: Multiplication Becomes Addition}
	
	The fact that taking logarithms turns multiplications into additions is fundamental:
	
	\begin{equation}
		\log(A \times B) = \log(A) + \log(B)
	\end{equation}
	
	\subsection{What does logarithmization teach us?}
	
	\begin{enumerate}
		\item \textbf{Scale transformation}: From proportional to linear scale
		\item \textbf{Nature of perception}: Many sensory perceptions are logarithmic
		\begin{itemize}
			\item \textbf{Hearing}: Frequency ratios as equal steps
			\item \textbf{Light}: Logarithmic brightness perception
			\item \textbf{Sound}: Decibel scale
		\end{itemize}
		\item \textbf{Physical systems}: Exponential growth becomes linear
		\item \textbf{Unification}: Addition and multiplication are connected by transformation
	\end{enumerate}
	
	\subsection{Logarithmic Perception}
	
	The nature of perception follows the Weber-Fechner law, which reflects the logarithmic structure of relational systems:
	
	\begin{figure}[htbp]
		\centering
		\begin{tikzpicture}[scale=0.8]
			\draw[->] (0,0) -- (6,0) node[right] {Stimulus intensity $I$};
			\draw[->] (0,0) -- (0,4) node[above] {Perception $W$};
			\draw[domain=0.1:5.5, smooth, blue, thick] plot (\x, {1.5*ln(\x + 0.5)});
			\node[blue] at (4,2.5) {$W = k \log(I/I_0)$};
			\node at (3,0.8) {\footnotesize Weber-Fechner law};
			\draw[dashed, gray] (1,0) -- (1,1.04);
			\draw[dashed, gray] (2,0) -- (2,1.66);
			\draw[dashed, gray] (4,0) -- (4,2.28);
			\node[below] at (1,0) {\footnotesize $I_1$};
			\node[below] at (2,0) {\footnotesize $2I_1$};
			\node[below] at (4,0) {\footnotesize $4I_1$};
		\end{tikzpicture}
		\caption{Logarithmic perception corresponds to the structure of relational systems}
		\label{RelokativesZahl:L-RelokativesZahlensystemEn-1024}
	\end{figure}
	
	\section{Physical Analogies and Applications}
	
	\subsection{Renormalization Group Flow}
	
	A remarkable parallel exists between relational composition and renormalization group flow in quantum field theory:
	
	\begin{equation}
		\beta(g) = \mu\frac{dg}{d\mu} = \sum_{k=1}^n \primrel{p_k} \circ \log\left(\frac{E}{E_0}\right)
	\end{equation}
	
	Here the energy scaling corresponds to the composition of prime relations.
	
	\subsection{Quantum Entanglement and Relations}
	
	\begin{table}[htbp]
		\centering
		\begin{adjustbox}{width=0.85\textwidth}
			\begin{tabular}{ll}
				\toprule
				\textbf{Relational System} & \textbf{Quantum Mechanics} \\
				\midrule
				Prime relation $\primrel{p}$ & Basis state $|p\rangle$ \\
				Composition $\circ$ & Tensor product $\otimes$ \\
				Vector addition $\oplus$ & Superposition principle \\
				Logarithmic structure & Phase relationships \\
				\bottomrule
			\end{tabular}
		\end{adjustbox}
		\caption{Structural analogies between relational and quantum systems}
		\label{RelokativesZahl:L-RelokativesZahlensystemEn-1025}
	\end{table}
	
	\section{Additive and Multiplicative Modulation in Nature}
	
	\subsection{Electromagnetism and Physics}
	
	\begin{table}[htbp]
		\centering
		\begin{adjustbox}{width=0.9\textwidth}
			\begin{tabular}{lll}
				\toprule
				\textbf{Modulation} & \textbf{Description} & \textbf{Examples} \\
				\midrule
				Multiplicative (AM) & Proportional amplitude change & Amplitude modulation, scaling \\
				Additive (FM) & Superposition of frequencies & Frequency modulation, interference \\
				\bottomrule
			\end{tabular}
		\end{adjustbox}
		\caption{Modulation in physics and technology}
		\label{RelokativesZahl:L-RelokativesZahlensystemEn-1026}
	\end{table}
	
	\subsection{Music and Acoustics}
	
	\begin{itemize}
		\item \textbf{Timbre}: Additive superposition of harmonic overtones with multiplicative frequency ratios
		\item \textbf{Harmony}: Consonance through simple multiplicative ratios ($3:2$, $5:4$)
		\item \textbf{Melody}: Multiplicative frequency steps in additive time sequence
	\end{itemize}
	
	\section{The Elimination of Absolute Quantities}
	
	A central feature of this system is that the concrete assignment to a quantity is not necessary in the fundamental definitions. \textbf{The assignment to a specific quantity can be omitted and only becomes important when these relational numbers are applied to real things.}
	
\section*{Definition}
		\begin{itemize}
			\item \textbf{Fundamental level}: Numbers are abstract relationships
			\item \textbf{Application level}: Measurement in concrete units (meters, kilograms, hertz)
			\item \textbf{Natural units}: $E = m$ (energy-mass identity as pure relation)
		\end{itemize}
% end box definition
	
	\section{FFT, QFT and Shor's Algorithm: Practical Applications}
	
	These algorithms already use the relational principle:
	
	\subsection{Fast Fourier Transform (FFT)}
	
	The FFT reduces complexity from $O(N^2)$ to $O(N \log N)$ through:
	\begin{itemize}
		\item Decomposition of the DFT matrix into sparsely populated factors
		\item Rader's algorithm for prime-sized transforms uses multiplicative groups
		\item Works with frequency ratios instead of absolute values
	\end{itemize}
	
	\subsection{Quantum Fourier Transform (QFT)}
	
	\begin{itemize}
		\item Quantum version of the classical DFT
		\item Core component of Shor's algorithm
		\item Works with exponential functions for period finding
	\end{itemize}
	
	\subsection{Algorithmic Details: Shor's Algorithm}
	
\section*{Algorithm}
		\caption{Shor's Algorithm for Prime Factorization}
		\label{RelokativesZahl:L-RelokativesZahlensystemEn-1027}
\section*{Algorithmic}
			\STATE \textbf{Input:} Odd composite number $N$
			\STATE \textbf{Output:} Non-trivial factor of $N$
			\STATE 
			\STATE Choose random $a$ with $1 < a < N$ and $\gcd(a,N) = 1$
			\STATE Use quantum computer for period finding:
			\STATE \quad Find period $r$ of function $f(x) = a^x \bmod N$
			\STATE \quad Use QFT for efficient computation
			\IF{$r$ is odd OR $a^{r/2} \equiv -1 \pmod{N}$}
			\STATE Go to step 4 (choose new $a$)
			\ENDIF
			\STATE Compute $d_1 = \gcd(a^{r/2} - 1, N)$
			\STATE Compute $d_2 = \gcd(a^{r/2} + 1, N)$
			\IF{$1 < d_1 < N$}
			\RETURN $d_1$
			\ELSIF{$1 < d_2 < N$}
			\RETURN $d_2$
			\ELSE
			\STATE Go to step 4
			\ENDIF
% end box algorithmic
% end box algorithm
	
	The key lies in period finding through QFT, which recognizes relational patterns in modular arithmetic.
	
	\begin{table}[htbp]
		\centering
		\begin{adjustbox}{width=0.85\textwidth}
			\begin{tabular}{llll}
				\toprule
				\textbf{Algorithm} & \textbf{Property} & \textbf{Complexity} & \textbf{Application} \\
				\midrule
				FFT & Ratios & $O(N \log N)$ & Signal processing \\
				QFT & Superposition & Polynomial & Quantum algorithms \\
				Shor & Period patterns & Polynomial & Cryptography \\
				\bottomrule
			\end{tabular}
		\end{adjustbox}
		\caption{Relational algorithms in practice}
		\label{RelokativesZahl:L-RelokativesZahlensystemEn-1028}
	\end{table}
	
	\section{Mathematical Framework}
	
	\subsection{Formal Definition of the Relational System}
	
\section*{Theorem}
		A relational number system $\mathcal{R}$ is defined by:
		\begin{enumerate}
			\item A set of prime number relations $\{\primrel{p_1}, \primrel{p_2}, \ldots\}$
			\item A composition operation $\circ$ (corresponds to multiplication)
			\item A vector representation $\vect{v} = (a_1, a_2, \ldots)$ with $\prod_i p_i^{a_i}$
			\item A logarithmic addition operation $\oplus$ on vectors
		\end{enumerate}
% end box theorem
	
	\subsection{Properties of the System}
	
	\begin{itemize}
		\item \textbf{Closure}: $\primrel{a} \circ \primrel{b} \in \mathcal{R}$
		\item \textbf{Associativity}: $(\primrel{a} \circ \primrel{b}) \circ \primrel{c} = \primrel{a} \circ (\primrel{b} \circ \primrel{c})$
		\item \textbf{Identity}: $\primrel{1}$ is neutral element
		\item \textbf{Inverses}: Each relation $\primrel{a}$ has inverse $\primrel{a}^{-1}$
	\end{itemize}
	
	\section{Advantages and Challenges}
	
	\subsection{Advantages of the Relational System}
	
	\begin{enumerate}
		\item \textbf{Fundamental nature}: Captures the essence of relationships
		\item \textbf{Logarithmic harmony}: Compatible with natural laws
		\item \textbf{Multiplicative primary operation}: Natural connection
		\item \textbf{Practical application}: Already implemented in FFT/QFT/Shor
	\end{enumerate}
	
	\subsection{Challenges}
	
	\begin{enumerate}
		\item \textbf{Addition}: Complex definition in purely relational spaces
		\item \textbf{Intuition}: Unfamiliar for set-based thinking
		\item \textbf{Practical implementation}: Requires new mathematical tools
	\end{enumerate}
	
	\section{Epistemological Implications}
	
	The relational number system has profound philosophical consequences:
	
	\begin{itemize}
		\item \textbf{Operationalism}: Numbers are defined by their transformative effects, not by static properties
		\item \textbf{Process ontology}: Being is understood as a dynamic network of transformations
		\item \textbf{Neo-Pythagoreanism}: Mathematical relations as fundamental substrate of reality
		\item \textbf{Structuralism}: The structure of relationships is primary over \emph{objects}
	\end{itemize}
	
	\section{Open Research Questions}
	
	The relational number system opens various research directions:
	
	\begin{enumerate}
		\item \textbf{Canonical addition}: How can addition be naturally defined in the relational system without transitioning to logarithmic space?
		\item \textbf{Topological structure}: Is there a natural topology on the space of prime relations?
		\item \textbf{Non-commutative generalizations}: Can the system capture quantum groups and non-commutative structures?
		\item \textbf{Algorithmic complexity}: Which computational problems become easier or harder in the relational system?
		\item \textbf{Cognitive modeling}: How is relational thinking reflected in neural structures?
	\end{enumerate}
	
	\section{Conclusion}
	
	The relational number system represents a paradigm shift: from "How much?" to "How does it relate?". 
	
	\textbf{Core insights}:
	\begin{enumerate}
		\item Prime numbers are elementary, indivisible ratios
		\item Multiplication is the natural, primary operation
		\item The system is intrinsically logarithmically structured
		\item Practical applications already exist in computer science
		\item Energy can serve as a universal relational dimension
	\end{enumerate}
	
	This framework offers both theoretical insights and practical tools for a deeper understanding of the mathematical structure of reality.
	
	\section{Appendix A: Practical Application - T0-Framework Factorization Tool}
	
	This appendix shows a real implementation of the relational number system in a factorization tool that practically implements the theoretical concepts.
	
	\subsection{Adaptive Relational Parameter Scaling}
	
	The T0-Framework implements adaptive \xi -parameters that follow the relational principle:
	
\section*{Algorithm}
		\caption{Adaptive $\xi$-Parameters in the Relational System}
		\label{RelokativesZahl:L-RelokativesZahlensystemEn-1029}
\section*{Algorithmic}
			\STATE \textbf{function} adaptive\_xi\_for\_hardware(problem\_bits):
			\IF{problem\_bits $\leq$ 64}
			\STATE base\_xi = $1 \times 10^{-5}$ \COMMENT{Standard relations}
			\ELSIF{problem\_bits $\leq$ 256}
			\STATE base\_xi = $1 \times 10^{-6}$ \COMMENT{Reduced coupling}
			\ELSIF{problem\_bits $\leq$ 1024}
			\STATE base\_xi = $1 \times 10^{-7}$ \COMMENT{Minimal coupling}
			\ELSE
			\STATE base\_xi = $1 \times 10^{-8}$ \COMMENT{Extreme stability}
			\ENDIF
			\RETURN base\_xi $\times$ hardware\_factor
% end box algorithmic
% end box algorithm
	
	This scaling demonstrates the \textbf{relational principle}: The parameter $\xi$ is not set absolutely, but \textbf{relative to the problem size}.
	
	\subsection{Energy Field Relations instead of Absolute Values}
	
	The T0-Framework defines physical constants relationally:
	
	\begin{align}
		c^2 &= 1 + \xi \quad \text{(relational coupling)} \\
		\text{correction} &= 1 + \xi \quad \text{(adaptive correction factor)} \\
		E_{\text{corr}} &= \xi \cdot \frac{E_1 \cdot E_2}{r^2} \quad \text{(energy field ratio)}
	\end{align}
	
	The wave velocity is defined \textbf{not as an absolute constant}, but as a \textbf{relation to $\xi$}.
	
	\subsection{Quantum Gates as Relational Transformations}
	
	The implementation shows how quantum operations function as **compositions of ratios**:
	
\section*{Example}
		\begin{align}
			\text{correction} &= 1 + \xi \\
			E_{\text{out},0} &= \frac{E_0 + E_1}{\sqrt{2}} \cdot \text{correction} \\
			E_{\text{out},1} &= \frac{E_0 - E_1}{\sqrt{2}} \cdot \text{correction}
		\end{align}
		
		The Hadamard gate uses \textbf{relational corrections} instead of fixed transformations.
% end box example
	
\section*{Example}
\section*{Algorithmic}
			\IF{$|$control\_field$|$ > threshold}
			\STATE target\_out = $-$target\_field $\times$ correction
			\ELSE
			\STATE target\_out = target\_field $\times$ correction
			\ENDIF
% end box algorithmic
		
		The CNOT operation is based on \textbf{ratios and thresholds}, not on discrete states.
% end box example
	
	\subsection{Period Finding through Resonance Relations}
	
	The heart of prime factorization uses **relational resonances**:
	
	\begin{align}
		\omega &= \frac{2\pi}{r} \quad \text{(period frequency)} \\
		E_{\text{corr}} &= \xi \cdot \frac{E_1 \cdot E_2}{r^2} \quad \text{(energy field correlation)} \\
		\text{resonance}_{\text{base}} &= \exp\left(-\frac{(\omega - \pi)^2}{4|\xi|}\right) \\
		\text{resonance}_{\text{total}} &= \text{resonance}_{\text{base}} \cdot (1 + E_{\text{corr}})^{2.5}
	\end{align}
	
	This implementation shows how \textbf{Shor's period finding} is replaced by \textbf{relational energy field correlations}.
	
	\subsection{Bell State Verification as Relational Consistency}
	
	The tool implements Bell states with relational corrections:
	
\section*{Algorithm}
		\caption{T0-Bell State Generation}
		\label{RelokativesZahl:L-RelokativesZahlensystemEn-1030}
\section*{Algorithmic}
			\STATE Start: $|00\rangle$
			\STATE correction = $1 + \xi$
			\STATE inv\_sqrt2 = $1/\sqrt{2}$
			\STATE 
			\COMMENT{Hadamard on first qubit}
			\STATE $E_{00} = 1.0 \times$ inv\_sqrt2 $\times$ correction
			\STATE $E_{10} = 1.0 \times$ inv\_sqrt2 $\times$ correction
			\STATE 
			\COMMENT{CNOT: $|10\rangle \to |11\rangle$}
			\STATE $E_{11} = E_{10} \times$ correction
			\STATE $E_{10} = 0$
			\STATE 
			\COMMENT{Final result: $(|00\rangle + |11\rangle)/\sqrt{2}$ with \xi -correction}
			\RETURN $\{P(00), P(01), P(10), P(11)\}$
% end box algorithmic
% end box algorithm
	
	\subsection{Empirical Validation of Relational Theory}
	
	The tool conducts **ablation studies** that confirm the relational principle:
	
	\begin{table}[htbp]
		\centering
		\begin{adjustbox}{width=0.9\textwidth}
			\begin{tabular}{lccc}
				\toprule
				\textbf{$\xi$-Parameter} & \textbf{Success Rate} & \textbf{Average Time} & \textbf{Stability} \\
				\midrule
				$\xi = 1 \times 10^{-5}$ (relational) & 100\% & 1.2s & Stable up to 64-bit \\
				$\xi = 1.33 \times 10^{-4}$ (absolute) & 95\% & 1.8s & Unstable at >32-bit \\
				$\xi = 1 \times 10^{-4}$ (absolute) & 90\% & 2.1s & Overflow problems \\
				$\xi = 5 \times 10^{-5}$ (absolute) & 98\% & 1.4s & Good but not optimal \\
				\bottomrule
			\end{tabular}
		\end{adjustbox}
		\caption{Empirical validation: Relational vs. absolute $\xi$-parameters}
		\label{RelokativesZahl:L-RelokativesZahlensystemEn-1031}
	\end{table}
	
	The results show: \textbf{Relational parameters} (that adapt to problem size) are \textbf{significantly more effective} than absolute constants.
	
	\subsection{Implementation Code Examples}
	
	\subsubsection{Relational Parameter Adaptation}
	\begin{verbatim}
		def adaptive_xi_for_hardware(self, hardware_type: str = "standard") -> float:
		# Adaptive xi-scaling based on problem size
		if self.rsa_bits <= 64:
		base_xi = 1e-5  # Optimal for standard problems
		elif self.rsa_bits <= 256:
		base_xi = 1e-6  # Reduced coupling for medium sizes
		elif self.rsa_bits <= 1024:
		base_xi = 1e-7  # Minimal coupling for large problems
		else:
		base_xi = 1e-8  # Extremely reduced for stability
		
		hardware_factor = {"standard": 1.0, "gpu": 1.2, "quantum": 0.5}
		return base_xi * hardware_factor.get(hardware_type, 1.0)
	\end{verbatim}
	
	\subsubsection{Energy Field Relations}
	\begin{verbatim}
		def solve_energy_field(self, x: np.ndarray, t: np.ndarray) -> np.ndarray:
		# T0-Framework: c² = 1 + xi (relational coupling)
		c_squared = 1.0 + abs(self.xi)  # NOT just xi!
		
		for i in range(2, len(t)):
		for j in range(1, len(x)-1):
		spatial_laplacian = (E[j+1,i-1] - 2*E[j,i-1] + E[j-1,i-1]) / (dx**2)
		# Wave equation with relational velocity
		E[j,i] = 2*E[j,i-1] - E[j,i-2] + c_squared * (dt**2) * spatial_laplacian
	\end{verbatim}
	
	\subsubsection{Relational Quantum Gates}
	\begin{verbatim}
		def hadamard_t0(self, E_field_0: float, E_field_1: float) -> Tuple[float, float]:
		xi = self.adaptive_xi_for_hardware()
		correction = 1 + xi  # Relational correction, not absolute
		inv_sqrt2 = 1 / math.sqrt(2)
		
		# Hadamard with relational xi-correction
		E_out_0 = (E_field_0 + E_field_1) * inv_sqrt2 * correction
		E_out_1 = (E_field_0 - E_field_1) * inv_sqrt2 * correction
		return (E_out_0, E_out_1)
	\end{verbatim}
	
	\subsubsection{Period Finding through Ratio Resonance}
	\begin{verbatim}
		def quantum_period_finding(self, a: int) -> Optional[int]:
		for r in range(1, max_period):
		if self.mod_pow(a, r, self.rsa_N) == 1:
		omega = 2 * math.pi / r
		
		# Relational energy field correlation instead of absolute calculation
		E_corr = self.xi * (E1 * E2) / (r**2)
		base_resonance = math.exp(-((omega - math.pi)**2) / (4 * abs(self.xi)))
		
		# Resonance amplified by ratio correlations
		total_resonance = base_resonance * (1 + E_corr)**2.5
	\end{verbatim}
	
	\subsection{Insights for the Relational Number System}
	
	The T0-Framework implementation demonstrates several core principles of the relational number system:
	
	\begin{enumerate}
		\item \textbf{Adaptive parameters}: No universal constants, but context-sensitive relations
		\item \textbf{Ratio-based operations}: All calculations use correction factors like $(1 + \xi)$
		\item \textbf{Logarithmic scaling}: Parameters change exponentially with problem size
		\item \textbf{Composition of relations}: Complex operations as concatenation of simple ratios
		\item \textbf{Empirical validation}: Relational approaches measurably outperform absolute constants
	\end{enumerate}
	
	This implementation shows that the \textbf{relational number system is not only theoretically elegant}, but also \textbf{practically superior} for complex calculations like prime factorization.
	
	\section{Outlook}
	
	\subsection{Future Research Directions}
	
	\begin{itemize}
		\item Development of a complete addition theory for relational numbers
		\item Application to quantum field theory and string theory
		\item Computer algebra systems for relational arithmetic
		\item Pedagogical approaches for relational mathematics education
	\end{itemize}
	
	\subsection{Potential Applications}
	
	\begin{itemize}
		\item New algorithms for prime factorization
		\item Improved quantum computing protocols
		\item Innovative approaches in music theory and acoustics
		\item Fundamentally new perspectives in theoretical physics
	\end{itemize}
	


% Bibliography
\begin{thebibliography}{99}
	
	\bibitem{pdg2024}
	Particle Data Group Collaboration (2024). 
	\textit{Review of Particle Physics}. 
	Progress of Theoretical and Experimental Physics, 2024(8), 083C01.
	\url{https://pdg.lbl.gov}
	
	\bibitem{flag2024}
	Aoki, Y., et al. (FLAG Collaboration) (2024). 
	\textit{FLAG Review 2024 of Lattice Results for Low-Energy Constants}. 
	arXiv:2411.04268.
	\url{https://arxiv.org/abs/2411.04268}
	
	\bibitem{fermilab_muon_g2}
	Abi, B., et al. (Muon g-2 Collaboration) (2021). 
	\textit{Measurement of the Positive Muon Anomalous Magnetic Moment to 0.46 ppm}. 
	Physical Review Letters, 126, 141801.
	
	\bibitem{peskin_schroeder}
	Peskin, M. E., \& Schroeder, D. V. (1995). 
	\textit{An Introduction to Quantum Field Theory}. 
	Addison-Wesley.
	
	\bibitem{weinberg_qft}
	Weinberg, S. (1995). 
	\textit{The Quantum Theory of Fields, Vol. I--III}. 
	Cambridge University Press.
	
	\bibitem{griffiths_particle}
	Griffiths, D. (2008). 
	\textit{Introduction to Elementary Particles}. 
	Wiley-VCH.
	
	\bibitem{mandl_shaw}
	Mandl, F., \& Shaw, G. (2010). 
	\textit{Quantum Field Theory (2nd ed.)}. 
	Wiley.
	
	\bibitem{srednicki_qft}
	Srednicki, M. (2007). 
	\textit{Quantum Field Theory}. 
	Cambridge University Press.
	
	\bibitem{t0_fundamentals}
	Pascher, J. (2024). 
	\textit{T0-Theory: Foundations of Time-Mass Duality}. 
	Unpublished manuscript, HTL Leonding.
	
	\bibitem{t0_fine_structure}
	Pascher, J. (2024). 
	\textit{T0-Theory: The Fine Structure Constant}. 
	Unpublished manuscript, HTL Leonding.
	
	\bibitem{t0_neutrinos}
	Pascher, J. (2024). 
	\textit{T0-Theory: Neutrino Masses and PMNS Mixing}. 
	Unpublished manuscript, HTL Leonding.
	
	\bibitem{t0_github}
	Pascher, J. (2024--2025). 
	\textit{T0-Time-Mass-Duality Repository}. 
	GitHub.
	\url{https://github.com/jpascher/T0-Time-Mass-Duality}
	
	\bibitem{lattice_qcd_review}
	Kronfeld, A. S. (2012). 
	\textit{Twenty-first Century Lattice Gauge Theory: Results from the QCD Lagrangian}. 
	Annual Review of Nuclear and Particle Science, 62, 265--284.
	
	\bibitem{neutrino_mixing_pdg}
	Particle Data Group Collaboration (2024). 
	\textit{Neutrino Masses, Mixing, and Oscillations}. 
	PDG Review 2024.
	\url{https://pdg.lbl.gov/2024/reviews/rpp2024-rev-neutrino-mixing.pdf}
	
	\bibitem{higgs_discovery}
	ATLAS and CMS Collaborations (2012). 
	\textit{Observation of a New Particle in the Search for the Standard Model Higgs Boson}. 
	Physics Letters B, 716, 1--29.
	
	\bibitem{Brannen2005}
	C. P. Brannen, ``Estimate of neutrino masses from Koide's relation'', \textit{arXiv:hep-ph/0505028} (2005).
	\url{https://arxiv.org/abs/hep-ph/0505028}
	
	\bibitem{Brannen2006}
	C. P. Brannen, ``Koide Mass Formula for Neutrinos'', \textit{arXiv:0702.0052} (2006).
	\url{http://brannenworks.com/MASSES.pdf}
	
	\bibitem{PhaseVectors2025}
	Anonymous, ``The Koide Relation and Lepton Mass Hierarchy from Phase Vectors'', \textit{rXiv:2507.0040} (2025).
	\url{https://rxiv.org/pdf/2507.0040v1.pdf}
	
	\bibitem{PDG2025}
	Particle Data Group, ``Review of Particle Physics'', \textit{Phys. Rev. D} \textbf{112} (2025) 030001.
	\url{https://pdg.lbl.gov/2025/}
	
	\bibitem{terrell2024}
	Terrell et al. (2024). 
	\textit{Single-Clock Metrology in Nature}. 
	Nature Physics.
	
	\bibitem{hossenfelder2024}
	Hossenfelder, S. (2024). 
	\textit{Single Clock Video Explanation}. 
	YouTube.
	
	\bibitem{hundert1931}
	Hundert (1931). 
	\textit{Reference Work}. 
	Publisher.
	
	\bibitem{terrell2025}
	Terrell et al. (2025). 
	\textit{Advanced Clock Synchronization Methods}. 
	Physical Review Letters.
	
	\bibitem{pascher_t0_2025}
	Pascher, J. (2025). 
	\textit{T0-Theory: Complete Framework and Applications}. 
	Unpublished manuscript, HTL Leonding.
	
	\bibitem{t0qm}
	Pascher, J. (2024). 
	\textit{T0-Theory: Quantum Mechanics Formulation}. 
	Unpublished manuscript, HTL Leonding.
	
	\bibitem{t0anomale}
	Pascher, J. (2024). 
	\textit{T0-Theory: Anomalous Magnetic Moments}. 
	Unpublished manuscript, HTL Leonding.
	
	\bibitem{muong2complete}
	Abi, B., et al. (Muon g-2 Collaboration) (2023). 
	\textit{Complete Measurement of the Positive Muon Anomalous Magnetic Moment}. 
	Physical Review Letters, 131, 161802.
	
	\bibitem{penrose2004}
	Penrose, R. (2004). 
	\textit{The Road to Reality: A Complete Guide to the Laws of the Universe}. 
	Jonathan Cape.
	
	\bibitem{planck1900}
	Planck, M. (1900). 
	\textit{On the Theory of the Energy Distribution Law of the Normal Spectrum}. 
	Verhandlungen der Deutschen Physikalischen Gesellschaft, 2, 237.
	
	\bibitem{T0Theory}
	Pascher, J. (2024). 
	\textit{T0-Theory: Fundamental Principles}. 
	Unpublished manuscript, HTL Leonding.
	
	% Additional bibliography entries for all undefined citations
	\bibitem{6g_roadmap}
	6G Research Consortium (2024).
	\textit{6G Technology Roadmap}.
	Technical Report.
	
	\bibitem{Born2013}
	Born, M. (2013).
	\textit{Einstein's Theory of Relativity}.
	Dover Publications.
	
	\bibitem{Casimir1948}
	Casimir, H. B. G. (1948).
	\textit{On the attraction between two perfectly conducting plates}.
	Proc. Kon. Ned. Akad. Wetensch. B51, 793--795.
	
	\bibitem{Einstein1905}
	Einstein, A. (1905).
	\textit{On the Electrodynamics of Moving Bodies}.
	Annalen der Physik, 17, 891--921.
	
	\bibitem{Feynman2006}
	Feynman, R. P. (2006).
	\textit{QED: The Strange Theory of Light and Matter}.
	Princeton University Press.
	
	\bibitem{Griffiths2017}
	Griffiths, D. J. (2017).
	\textit{Introduction to Electrodynamics (4th ed.)}.
	Cambridge University Press.
	
	\bibitem{Jackson1999}
	Jackson, J. D. (1999).
	\textit{Classical Electrodynamics (3rd ed.)}.
	Wiley.
	
	\bibitem{Mohr2016}
	Mohr, P. J., et al. (2016).
	\textit{CODATA Recommended Values of the Fundamental Physical Constants: 2014}.
	Rev. Mod. Phys. 88, 035009.
	
	\bibitem{Parker2018}
	Parker, R. H., et al. (2018).
	\textit{Measurement of the fine-structure constant as a test of the Standard Model}.
	Science, 360, 191--195.
	
	\bibitem{Planck1900}
	Planck, M. (1900).
	\textit{On the Theory of the Energy Distribution Law of the Normal Spectrum}.
	Verhandlungen der Deutschen Physikalischen Gesellschaft, 2, 237.
	
	\bibitem{Planck2018}
	Planck Collaboration (2018).
	\textit{Planck 2018 results. VI. Cosmological parameters}.
	Astronomy \& Astrophysics, 641, A6.
	
	\bibitem{QFT_T0}
	Pascher, J. (2024).
	\textit{T0-Theory and QFT Connections}.
	Unpublished manuscript, HTL Leonding.
	
	\bibitem{Sommerfeld1916}
	Sommerfeld, A. (1916).
	\textit{On the Quantum Theory of Spectral Lines}.
	Annalen der Physik, 51, 1--94.
	
	\bibitem{T0_Feinstruktur}
	Pascher, J. (2024).
	\textit{T0-Theory: Fine Structure Analysis}.
	Unpublished manuscript, HTL Leonding.
	
	\bibitem{T0_SI}
	Pascher, J. (2024).
	\textit{T0-Theory and SI Units}.
	Unpublished manuscript, HTL Leonding.
	
	\bibitem{T0_fine_structure}
	Pascher, J. (2024).
	\textit{T0-Theory: The Fine Structure Constant}.
	Unpublished manuscript, HTL Leonding.
	
	\bibitem{T0_g2_erweiterung}
	Pascher, J. (2024).
	\textit{T0-Theory: g-2 Extensions}.
	Unpublished manuscript, HTL Leonding.
	
	\bibitem{T0_gravitational_constant}
	Pascher, J. (2024).
	\textit{T0-Theory: Gravitational Constant Derivation}.
	Unpublished manuscript, HTL Leonding.
	
	\bibitem{T0_netze_en}
	Pascher, J. (2024).
	\textit{T0-Theory: Network Structures}.
	Unpublished manuscript, HTL Leonding.
	
	\bibitem{T0_tm_erweiterung}
	Pascher, J. (2024).
	\textit{T0-Theory: Time-Mass Extensions}.
	Unpublished manuscript, HTL Leonding.
	
	\bibitem{Uzan2003}
	Uzan, J.-P. (2003).
	\textit{The fundamental constants and their variation}.
	Rev. Mod. Phys. 75, 403--455.
	
	\bibitem{Weinberg1995}
	Weinberg, S. (1995).
	\textit{The Quantum Theory of Fields, Vol. I}.
	Cambridge University Press.
	
	\bibitem{albrecht1999}
	Albrecht, A. \& Magueijo, J. (1999).
	\textit{A time varying speed of light as a solution to cosmological puzzles}.
	Phys. Rev. D 59, 043516.
	
	\bibitem{alice2023}
	ALICE Collaboration (2023).
	\textit{Recent results from ALICE}.
	CERN-EP-2023-XXX.
	
	\bibitem{analog_optical}
	Smith, J. et al. (2024).
	\textit{Analog optical computing systems}.
	Nature Photonics.
	
	\bibitem{ashtekar2004}
	Ashtekar, A. \& Lewandowski, J. (2004).
	\textit{Background independent quantum gravity}.
	Class. Quantum Grav. 21, R53.
	
	\bibitem{atlas2023}
	ATLAS Collaboration (2023).
	\textit{ATLAS physics results}.
	CERN-PH-EP-2023-XXX.
	
	\bibitem{atlas2023higgs}
	ATLAS Collaboration (2023).
	\textit{Higgs boson measurements}.
	Phys. Rev. Lett.
	
	\bibitem{barbour1999}
	Barbour, J. (1999).
	\textit{The End of Time}.
	Oxford University Press.
	
	\bibitem{barrow1999}
	Barrow, J. D. (1999).
	\textit{Cosmologies with varying light speed}.
	Phys. Rev. D 59, 043515.
	
	\bibitem{becker2007}
	Becker, K. et al. (2007).
	\textit{String Theory and M-Theory}.
	Cambridge University Press.
	
	\bibitem{bell_muon}
	Bennett, G. W., et al. (Muon g-2 Collaboration) (2006).
	\textit{Final report of the E821 muon anomalous magnetic moment measurement}.
	Phys. Rev. D 73, 072003.
	
	\bibitem{bondi1948}
	Bondi, H. \& Gold, T. (1948).
	\textit{The steady-state theory of the expanding universe}.
	Mon. Not. R. Astron. Soc. 108, 252--270.
	
	\bibitem{brewer2019}
	Brewer, S. M. et al. (2019).
	\textit{Al+ Quantum-Logic Clock with Systematic Uncertainty below $10^{-18}$}.
	Phys. Rev. Lett. 123, 033201.
	
	\bibitem{cms2023top}
	CMS Collaboration (2023).
	\textit{Top quark measurements at CMS}.
	JHEP 2023.
	
	\bibitem{cms2024}
	CMS Collaboration (2024).
	\textit{CMS physics results 2024}.
	CERN-PH-EP-2024-XXX.
	
	\bibitem{codata2019}
	Tiesinga, E. et al. (2019).
	\textit{The 2018 CODATA Recommended Values}.
	J. Phys. Chem. Ref. Data.
	
	\bibitem{desi2025}
	DESI Collaboration (2025).
	\textit{DESI 2025 Cosmology Results}.
	arXiv preprint.
	
	\bibitem{differential_optical}
	Wang, X. et al. (2024).
	\textit{Differential optical computing}.
	Optica.
	
	\bibitem{dingle1972}
	Dingle, H. (1972).
	\textit{Science at the Crossroads}.
	Martin Brian \& O'Keeffe.
	
	\bibitem{divalentino2021}
	Di Valentino, E. et al. (2021).
	\textit{In the realm of the Hubble tension}.
	Class. Quantum Grav. 38, 153001.
	
	\bibitem{elnaschie2004}
	El Naschie, M. S. (2004).
	\textit{A review of E infinity theory}.
	Chaos, Solitons \& Fractals, 19, 209--236.
	
	\bibitem{fabrication_heterogeneous}
	Chen, Y. et al. (2024).
	\textit{Heterogeneous photonic integration}.
	Nature Electronics.
	
	\bibitem{fermilab2023}
	Fermilab (2023).
	\textit{Muon g-2 results}.
	Phys. Rev. Lett.
	
	\bibitem{flexible_wafer}
	Kim, S. et al. (2024).
	\textit{Flexible wafer-scale photonics}.
	Science Advances.
	
	\bibitem{francesco1997}
	Di Francesco, P. et al. (1997).
	\textit{Conformal Field Theory}.
	Springer.
	
	\bibitem{hartree1957}
	Hartree, D. R. (1957).
	\textit{The Calculation of Atomic Structures}.
	Wiley.
	
	\bibitem{hhi_6g}
	Fraunhofer HHI (2024).
	\textit{6G Photonic Integration}.
	Technical Report.
	
	\bibitem{hossenfelder2025}
	Hossenfelder, S. (2025).
	\textit{Science without the gobbledygook}.
	YouTube/Blog.
	
	\bibitem{hossenfelder_single_clock_video}
	Hossenfelder, S. (2024).
	\textit{The Single Clock Problem}.
	YouTube.
	
	\bibitem{hoyle1948}
	Hoyle, F. (1948).
	\textit{A new model for the expanding universe}.
	Mon. Not. R. Astron. Soc. 108, 372--382.
	
	\bibitem{integration_microelectronic}
	Liu, A. et al. (2024).
	\textit{Microelectronic photonic integration}.
	IEEE Journal.
	
	\bibitem{jacobson1995}
	Jacobson, T. (1995).
	\textit{Thermodynamics of spacetime}.
	Phys. Rev. Lett. 75, 1260.
	
	\bibitem{kasevich2023}
	Kasevich, M. et al. (2023).
	\textit{Atom interferometry tests}.
	Nature Physics.
	
	\bibitem{lerner2014}
	Lerner, E. J. (2014).
	\textit{An open letter on cosmology}.
	New Scientist.
	
	\bibitem{lisa2017}
	LISA Consortium (2017).
	\textit{Laser Interferometer Space Antenna}.
	ESA Technical Report.
	
	\bibitem{lithium_tantalate}
	Zhang, M. et al. (2024).
	\textit{Thin-film lithium tantalate photonics}.
	Nature Photonics.
	
	\bibitem{lopez2010}
	Lopez-Corredoira, M. (2010).
	\textit{Tests and problems of the standard model in cosmology}.
	Int. J. Mod. Phys. D.
	
	\bibitem{ludlow2015}
	Ludlow, A. D. et al. (2015).
	\textit{Optical atomic clocks}.
	Rev. Mod. Phys. 87, 637.
	
	\bibitem{mach1883}
	Mach, E. (1883).
	\textit{Die Mechanik in ihrer Entwickelung}.
	F.A. Brockhaus.
	
	\bibitem{maldacena1998}
	Maldacena, J. (1998).
	\textit{The large N limit of superconformal field theories}.
	Adv. Theor. Math. Phys. 2, 231--252.
	
	\bibitem{mueller2014}
	Müller, H. et al. (2014).
	\textit{Atom interferometry tests of the gravitational redshift}.
	Phys. Rev. Lett.
	
	\bibitem{mug2_final_2025}
	Muon g-2 Collaboration (2025).
	\textit{Final muon g-2 measurement}.
	Phys. Rev. Lett.
	
	\bibitem{muong2_2023}
	Muon g-2 Collaboration (2023).
	\textit{Updated muon g-2 results}.
	Phys. Rev. Lett.
	
	\bibitem{nathan2024}
	Nathan, A. et al. (2024).
	\textit{Quantum computing advances}.
	Nature.
	
	\bibitem{newell2018}
	Newell, D. B. et al. (2018).
	\textit{The CODATA 2017 values of h, e, k, and $N_A$}.
	Metrologia 55, L13.
	
	\bibitem{nottale1993}
	Nottale, L. (1993).
	\textit{Fractal Space-Time and Microphysics}.
	World Scientific.
	
	\bibitem{on_chip_lithium}
	Wang, C. et al. (2024).
	\textit{On-chip lithium niobate photonics}.
	Nature Communications.
	
	\bibitem{optical_advantages}
	Shastri, B. J. et al. (2024).
	\textit{Advantages of optical computing}.
	Nature Reviews Physics.
	
	\bibitem{pascher2025cmb}
	Pascher, J. (2025).
	\textit{T0-Theory: CMB Analysis}.
	Unpublished manuscript, HTL Leonding.
	
	\bibitem{pascher2025g2}
	Pascher, J. (2025).
	\textit{T0-Theory: g-2 Predictions}.
	Unpublished manuscript, HTL Leonding.
	
	\bibitem{pascher2025qm}
	Pascher, J. (2025).
	\textit{T0-Theory: Quantum Mechanics}.
	Unpublished manuscript, HTL Leonding.
	
	\bibitem{pascher2025si}
	Pascher, J. (2025).
	\textit{T0-Theory: SI Unit System}.
	Unpublished manuscript, HTL Leonding.
	
	\bibitem{pascher2025t0}
	Pascher, J. (2025).
	\textit{T0-Theory: Complete Framework}.
	Unpublished manuscript, HTL Leonding.
	
	\bibitem{pascher:fundamentals}
	Pascher, J. (2024).
	\textit{T0-Theory: Fundamentals}.
	Unpublished manuscript, HTL Leonding.
	
	\bibitem{pascher:g2_rev9}
	Pascher, J. (2024).
	\textit{T0-Theory: g-2 Revision 9}.
	Unpublished manuscript, HTL Leonding.
	
	\bibitem{pascher:geometric_formalism}
	Pascher, J. (2024).
	\textit{T0-Theory: Geometric Formalism}.
	Unpublished manuscript, HTL Leonding.
	
	\bibitem{pascher:ml_addendum}
	Pascher, J. (2024).
	\textit{T0-Theory: Machine Learning Addendum}.
	Unpublished manuscript, HTL Leonding.
	
	\bibitem{pascher:t0_foundations}
	Pascher, J. (2024).
	\textit{T0-Theory: Foundations}.
	Unpublished manuscript, HTL Leonding.
	
	\bibitem{pascher_derivation_beta_2025}
	Pascher, J. (2025).
	\textit{T0-Theory: Derivation of Beta}.
	Unpublished manuscript, HTL Leonding.
	
	\bibitem{pascher_higgs_connection_2025}
	Pascher, J. (2025).
	\textit{T0-Theory: Higgs Connection}.
	Unpublished manuscript, HTL Leonding.
	
	\bibitem{pascher_lagrangian_extended_2025}
	Pascher, J. (2025).
	\textit{T0-Theory: Extended Lagrangian}.
	Unpublished manuscript, HTL Leonding.
	
	\bibitem{pascher_mathematical_structure_2025}
	Pascher, J. (2025).
	\textit{T0-Theory: Mathematical Structure}.
	Unpublished manuscript, HTL Leonding.
	
	\bibitem{pascher_t0_cmb_2025}
	Pascher, J. (2025).
	\textit{T0-Theory: CMB Predictions}.
	Unpublished manuscript, HTL Leonding.
	
	\bibitem{pascher_t0_energie_2025}
	Pascher, J. (2025).
	\textit{T0-Theory: Energy}.
	Unpublished manuscript, HTL Leonding.
	
	\bibitem{pascher_t0_energy_2025}
	Pascher, J. (2025).
	\textit{T0-Theory: Energy Framework}.
	Unpublished manuscript, HTL Leonding.
	
	\bibitem{pascher_t0_theory_2025}
	Pascher, J. (2025).
	\textit{T0-Theory: Complete Theory}.
	Unpublished manuscript, HTL Leonding.
	
	\bibitem{penrose1959}
	Penrose, R. (1959).
	\textit{The apparent shape of a relativistically moving sphere}.
	Proc. Cambridge Phil. Soc. 55, 137--139.
	
	\bibitem{penrose1967}
	Penrose, R. (1967).
	\textit{Twistor algebra}.
	J. Math. Phys. 8, 345--366.
	
	\bibitem{peratt1992}
	Peratt, A. L. (1992).
	\textit{Physics of the Plasma Universe}.
	Springer-Verlag.
	
	\bibitem{peskin1995}
	Peskin, M. E. \& Schroeder, D. V. (1995).
	\textit{An Introduction to Quantum Field Theory}.
	Addison-Wesley.
	
	\bibitem{peskin_schroeder_1995}
	Peskin, M. E. \& Schroeder, D. V. (1995).
	\textit{An Introduction to Quantum Field Theory}.
	Addison-Wesley.
	
	\bibitem{phoquant}
	PhoQuant (2024).
	\textit{Photonic quantum computing}.
	Technical Report.
	
	\bibitem{photonics_ai}
	Wetzstein, G. et al. (2024).
	\textit{Photonics for AI}.
	Nature.
	
	\bibitem{planck1906}
	Planck, M. (1906).
	\textit{The Theory of Heat Radiation}.
	Johann Ambrosius Barth.
	
	\bibitem{planck2018}
	Planck Collaboration (2018).
	\textit{Planck 2018 results}.
	A\&A 641, A6.
	
	\bibitem{polchinski1998}
	Polchinski, J. (1998).
	\textit{String Theory}.
	Cambridge University Press.
	
	\bibitem{qant_nps}
	QANT (2024).
	\textit{Quantum photonics systems}.
	Technical Report.
	
	\bibitem{quantenjahr25}
	Quantenjahr (2025).
	\textit{International Year of Quantum}.
	UNESCO.
	
	\bibitem{recurrent_photonics}
	Tait, A. N. et al. (2024).
	\textit{Recurrent photonic neural networks}.
	Optica.
	
	\bibitem{rf_photonics}
	Capmany, J. \& Novak, D. (2024).
	\textit{Microwave photonics}.
	Nature Photonics.
	
	\bibitem{riess2019}
	Riess, A. G. et al. (2019).
	\textit{Large Magellanic Cloud Cepheid Standards}.
	ApJ 876, 85.
	
	\bibitem{riess2022}
	Riess, A. G. et al. (2022).
	\textit{A Comprehensive Measurement of H0}.
	ApJ 934, L7.
	
	\bibitem{rovelli2004}
	Rovelli, C. (2004).
	\textit{Quantum Gravity}.
	Cambridge University Press.
	
	\bibitem{sciama1953}
	Sciama, D. W. (1953).
	\textit{On the origin of inertia}.
	Mon. Not. R. Astron. Soc. 113, 34--42.
	
	\bibitem{sciencedaily2025}
	ScienceDaily (2025).
	\textit{Physics news}.
	Online.
	
	\bibitem{sm_g2_2025}
	Aoyama, T. et al. (2025).
	\textit{Standard Model prediction for g-2}.
	Phys. Rep.
	
	\bibitem{susskind1995}
	Susskind, L. (1995).
	\textit{The world as a hologram}.
	J. Math. Phys. 36, 6377--6396.
	
	\bibitem{t0_kosmologie}
	Pascher, J. (2024).
	\textit{T0-Theory: Cosmology}.
	Unpublished manuscript, HTL Leonding.
	
	\bibitem{terrell1959}
	Terrell, J. (1959).
	\textit{Invisibility of the Lorentz contraction}.
	Phys. Rev. 116, 1041--1045.
	
	\bibitem{terrell_single_clock_nature_2024}
	Terrell, J. et al. (2024).
	\textit{Single clock precision measurements}.
	Nature Physics.
	
	\bibitem{tfln_foundry}
	TFLN Foundry (2024).
	\textit{Thin-film lithium niobate foundry services}.
	Technical Specifications.
	
	\bibitem{thiemann2007}
	Thiemann, T. (2007).
	\textit{Modern Canonical Quantum General Relativity}.
	Cambridge University Press.
	
	\bibitem{thz_epfl}
	EPFL (2024).
	\textit{Terahertz photonics research}.
	Technical Report.
	
	\bibitem{unnikrishnan2004}
	Unnikrishnan, C. S. (2004).
	\textit{On Einstein's resolution of the twin clock paradox}.
	Current Science, 86, 704--709.
	
	\bibitem{verlinde2011}
	Verlinde, E. (2011).
	\textit{On the origin of gravity and the laws of Newton}.
	JHEP 2011, 29.
	
	\bibitem{video2025}
	Video (2025).
	\textit{Physics video explanation}.
	YouTube.
	
	\bibitem{weinberg1995}
	Weinberg, S. (1995).
	\textit{The Quantum Theory of Fields}.
	Cambridge University Press.
	
	\bibitem{weiskopf2000}
	Weiskopf, D. (2000).
	\textit{Visualization of special relativity}.
	PhD thesis, University of Tübingen.
	
	\bibitem{wheeler1990}
	Wheeler, J. A. (1990).
	\textit{A Journey into Gravity and Spacetime}.
	Scientific American Library.
	
	\bibitem{wiki_bell}
	Wikipedia (2024).
	\textit{Bell's theorem}.
	Online encyclopedia.
	
	\bibitem{zwicky1929}
	Zwicky, F. (1929).
	\textit{On the red shift of spectral lines through interstellar space}.
	Proc. Natl. Acad. Sci. 15, 773--779.

\end{thebibliography}


\end{document}
