\documentclass[11pt,a4paper]{article}
\usepackage[a4paper,margin=2cm]{geometry}
\usepackage[utf8]{inputenc}
\usepackage[english]{babel}
\usepackage{lmodern}
\renewcommand{\familydefault}{\sfdefault}

\usepackage{amsmath,amssymb,amsthm}
\usepackage{graphicx}
\usepackage[unicode,pdfencoding=auto,hypertexnames=false]{hyperref}
\usepackage{booktabs}
\usepackage{longtable}
\usepackage{array}
\usepackage{siunitx}
\usepackage{fancyhdr}
\usepackage{float}
\usepackage{tikz}
% tcolorbox removed for standalone
% tcbset removed
\tikzset{
  t0blue/.style={draw=blue,fill=blue!10},
  t0red/.style={draw=red,fill=red!10},
  t0green/.style={draw=green!50!black,fill=green!10},
  t0orange/.style={draw=orange,fill=orange!10},
}
\usepackage{setspace}
\usepackage{enumitem}
\usepackage{adjustbox}
\usepackage{xcolor}

% Define colors for xcolor package
\definecolor{t0green}{RGB}{34,139,34}
\definecolor{t0blue}{RGB}{0,0,255}
\definecolor{t0red}{RGB}{255,0,0}
\definecolor{t0orange}{RGB}{255,165,0}

% Define custom column types for tables
\newcolumntype{L}[1]{>{\raggedright\arraybackslash}p{#1}}
\newcolumntype{C}[1]{>{\centering\arraybackslash}p{#1}}
\newcolumntype{R}[1]{>{\raggedleft\arraybackslash}p{#1}}

\setlength{\parindent}{0pt}
\setlength{\parskip}{6pt}

\hypersetup{
  colorlinks=true,
  linkcolor=blue,
  citecolor=blue,
  urlcolor=blue
}
\pagestyle{fancy}
\setlength{\headheight}{28pt}

\newcommand{\checkmarkx}{\checkmark}
\newcommand{\warningx}{\textbf{!}}

% Makros aus Einzel-Dokumenten (Fallback-Definitionen)
\newcommand{\mytimes}{\times}
\newcommand{\myapprox}{\approx}
\newcommand{\mysim}{\sim}
\newcommand{\myomega}{\omega}
\newcommand{\mypi}{\pi}
\newcommand{\myrightarrow}{\rightarrow}
\newcommand{\mypropto}{\propto}
\newcommand{\deltafield}{\delta\phi}
\newcommand{\xipar}{\xi}
\newcommand{\xiT}{\xi}
\newcommand{\lambdah}{\lambda_h}

% Additional macros used in chapter files
\newcommand{\Kfrak}{K_{\text{frak}}}  % Fractal correction factor
\newcommand{\Dfrak}{D_f}              % Fractal dimension
\newcommand{\betapar}{\beta}          % T0 beta parameter
\newcommand{\alphapar}{\alpha}        % T0 alpha parameter
\newcommand{\Efield}{E}               % Energy field
% Note: checkmarkxa/warningxa are variants used in auto-generated chapter files
\newcommand{\checkmarkxa}{\checkmark}
\newcommand{\warningxa}{\textbf{!}}

% Additional T0-specific macros
\newcommand{\xigeom}{\xi_{\text{geom}}}  % Geometric xi
\newcommand{\lP}{\ell_P}                  % Planck length
\newcommand{\rzero}{r_0}                  % Characteristic radius
\newcommand{\xirat}{\xi_{\text{rat}}}     % Xi ratio
\newcommand{\tzero}{t_0}                  % Characteristic time
\newcommand{\natunits}{\text{(nat. units)}}  % Natural units annotation
\newcommand{\myRightarrow}{\Rightarrow}   % Arrow variant
\newcommand{\Lag}{\mathcal{L}}            % Lagrangian

% Physics macros used in chapter files
\newcommand{\CQCD}{C_{\text{QCD}}}        % QCD correction
\newcommand{\EP}{E_P}                     % Planck energy
\newcommand{\Ee}{E_e}                     % Electron energy
\newcommand{\Emu}{E_\mu}                  % Muon energy
\newcommand{\Exi}{E_\xi}                  % Xi energy
\newcommand{\Ezero}{E_0}                  % Characteristic energy
\newcommand{\Hubble}{H}                   % Hubble constant
\newcommand{\Kspec}{K_{\text{spec}}}      % Spectral correction
\newcommand{\Lambdat}{\Lambda_t}          % Time-related cosmological constant
\newcommand{\Leff}{\mathcal{L}_{\text{eff}}}  % Effective Lagrangian
\newcommand{\Lorentz}{\mathcal{L}}        % Lorentz symbol
\newcommand{\Lxi}{L_\xi}                  % Xi length
\newcommand{\Tfield}{T}                   % Time field
\newcommand{\Weyl}{W}                     % Weyl tensor/symbol
\newcommand{\alphaEMSI}{\alpha_{\text{EM,SI}}}  % EM alpha in SI
\newcommand{\alphaEMnat}{\alpha_{\text{EM,nat}}}  % EM alpha in natural units
\newcommand{\alphaem}{\alpha_{\text{em}}} % Electromagnetic alpha
\newcommand{\betaTSI}{\beta_{T,\text{SI}}}  % Beta in SI
\newcommand{\betaTnat}{\beta_{T,\text{nat}}}  % Beta in natural units
\newcommand{\deltam}{\delta m}            % Mass difference
\newcommand{\phiT}{\phi_T}                % T-field phi
\newcommand{\tP}{t_P}                     % Planck time
\newcommand{\rhoCMB}{\rho_{\text{CMB}}}   % CMB density
\newcommand{\rhoCasimir}{\rho_{\text{Casimir}}}  % Casimir density

% Table formatting
\usepackage{multirow}

% Additional physics macros
\newcommand{\Riem}{\mathcal{R}}           % Riemann tensor
\newcommand{\ZPinch}{Z_{\text{pinch}}}    % Z-pinch
\newcommand{\SynchPower}{P_{\text{synch}}} % Synchrotron power
\newcommand{\Rzero}{R_0}                  % Characteristic radius
\newcommand{\alphafine}{\alpha}           % Fine structure constant
\newcommand{\Etau}{E_\tau}                % Tau energy
\newcommand{\deltaE}{\delta E}            % Energy deviation
\newcommand{\EPlanck}{E_P}                % Planck energy
\newcommand{\pichar}{\pi}                 % Pi character
\newcommand{\alphaWSI}{\alpha_{W,\text{SI}}}  % Wien alpha in SI
\newcommand{\alphaWnat}{\alpha_{W,\text{nat}}}  % Wien alpha in natural units

% Einfache abstract-Umgebung für Kapitel:
\newenvironment{abstract}{%
  \begin{center}\bfseries Abstract\end{center}\small
}{\par}


\title{T0 xi-und-e En}
\author{J. Pascher}
\date{\today}

\begin{document}
\maketitle

\section*{T0 Xi Und E (T0 xi-und-e)}

	\begin{abstract}
		This document provides a comprehensive analysis of the fundamental relationship between the geometric parameter $\xipar = \frac{4}{3} \times 10^{-4}$ of T0 theory and Euler's number $e = 2.71828\ldots$ The T0 theory is based on deep geometric principles from tetrahedral packing and postulates a fractal spacetime with dimension $D_f = 2.94$. We show in detail how exponential relationships of the form $e^{\xipar \cdot n}$ describe the hierarchy of particle masses, time scales, and fundamental constants from first principles. Particular attention is paid to the mathematical consistency and experimentally verifiable predictions of the theory.
	\end{abstract}
	
	\tableofcontents
	\newpage
	
	\section{Introduction: The Geometric Basis of T0 Theory}
	
	\subsection{Historical and Conceptual Foundations}
	
	T0 theory emerged from the observation that fundamental physical constants and mass ratios are not randomly distributed but follow deep mathematical relationships. Unlike many other approaches, T0 does not postulate new particles or additional dimensions, but rather a fundamental geometric structure of spacetime itself.
	
\section*{Insight}
\section*{The Central Paradigm of T0 Theory:}
		
		Physics at the fundamental level is not characterized by random parameters, but by an underlying geometric structure quantified by the parameter $\xi$. Euler's number $e$ serves as the natural operator that translates this geometric structure into dynamic processes.
% end box insight
	
	\subsection{The Tetrahedral Origin of}
	
\section*{Relation}
		\textbf{Geometric Derivation of $\xi = \frac{4}{3} \times 10^{-4}$:}
		
		The fundamental constant $\xi$ derives from the geometry of regular tetrahedra. For a tetrahedron with edge length $a$:
		
		\begin{align}
			V_{\text{tetra}} &= \frac{\sqrt{2}}{12}a^3 \\
			R_{\text{circumsphere}} &= \frac{\sqrt{6}}{4}a \\
			V_{\text{sphere}} &= \frac{4}{3}\pi R_{\text{circumsphere}}^3 = \frac{\pi\sqrt{6}}{16}a^3 \\
			\frac{V_{\text{tetra}}}{V_{\text{sphere}}} &= \frac{\sqrt{2}/12}{\pi\sqrt{6}/16} = \frac{2\sqrt{3}}{9\pi} \approx 0.513
		\end{align}
		
		Through scaling and normalization:
		\begin{equation}
			\xipar = \frac{4}{3} \times 10^{-4} = \left(\frac{V_{\text{tetra}}}{V_{\text{sphere}}}\right) \times \text{Scaling factor}
		\end{equation}
		
		\begin{center}
			\begin{tikzpicture}[scale=1.4]
				% Regular Tetrahedron
				\coordinate (A) at (0,0);
				\coordinate (B) at (2,0);
				\coordinate (C) at (1,1.732);
				\coordinate (D) at (1,0.577);
				
				\draw[t0blue, thick] (A) -- (B) -- (C) -- cycle;
				\draw[t0blue, thick] (A) -- (D);
				\draw[t0blue, thick] (B) -- (D);
				\draw[t0blue, thick] (C) -- (D);
				
				% Circumscribed Sphere
				\draw[t0red, dashed] (1,0.577) circle (1.155);
				
				\node at (0,0) [below left] {A};
				\node at (2,0) [below right] {B};
				\node at (1,1.732) [above] {C};
				\node at (1,0.577) [below] {D (Centroid)};
				
				\node at (3.2,0.866) [t0blue, align=left] {Tetrahedron: $V = \frac{\sqrt{2}}{12}a^3$};
				\node at (3.2,0.5) [t0red, align=left] {Circumsphere: $V = \frac{\pi\sqrt{6}}{16}a^3$};
			\end{tikzpicture}
		\end{center}
% end box relation
	
	\subsection{The Fractal Spacetime Dimension}
	
\section*{Treatise}
\section*{The Fractal Nature of Spacetime: $D_f = 2.94$}
		
		One of the most radical statements of T0 theory is that spacetime has fractal properties at the fundamental level. The effective dimension depends on the energy scale:
		
		\begin{equation}
			D_f(E) = 4 - 2\xipar \cdot \ln\left(\frac{E_P}{E}\right)
		\end{equation}
		
		For low energies ($E \ll E_P$):
		\begin{equation}
			D_f \approx 4 \quad \text{(classical spacetime)}
		\end{equation}
		
		For high energies ($E \sim E_P$):
		\begin{equation}
			D_f \approx 2.94 \quad \text{(fractal spacetime)}
		\end{equation}
		
\section*{Physical Interpretation:}
		\begin{itemize}
			\item At small distances/high energies, the fractal structure of spacetime becomes visible
			\item The dimension $D_f = 2.94$ is not accidental but follows from the geometric structure
			\item This explains the renormalization behavior of quantum field theories
		\end{itemize}
		
		The fractal dimension is calculated by:
		\begin{equation}
			D_f = 2 + \frac{\ln(1/\xipar)}{\ln(E_P/E_0)} \approx 2.94
		\end{equation}
		with $E_P = 1.221 \times 10^{19}$ GeV (Planck energy) and $E_0 = 1$ GeV (reference energy).
% end box treatise
	
	\section{Euler's Number as Dynamic Operator}
	
	\subsection{Mathematical Foundations of}
	
\section*{Relation}
\section*{The Unique Properties of $e$:}
		
		Euler's number is characterized by several equivalent definitions:
		
		\begin{align}
			e &= \lim_{n \to \infty} \left(1 + \frac{1}{n}\right)^n \\
			e &= \sum_{n=0}^{\infty} \frac{1}{n!} \\
			\frac{d}{dx}e^x &= e^x \\
			\int e^x dx &= e^x + C
		\end{align}
		
		In T0 theory, $e$ acquires a special significance as the natural translator between discrete geometric structure and continuous dynamic evolution.
% end box relation
	
	\subsection{Time-Mass Duality as Fundamental Principle}
	
\section*{Insight}
\section*{The Time-Mass Duality: $T \cdot m = 1$}
		
		In natural units ($\hbar = c = 1$) the fundamental relationship holds:
		\begin{equation}
			\boxed{T \cdot m = 1}
		\end{equation}
		
		This means:
		\begin{itemize}
			\item Every particle has a characteristic time scale $T = 1/m$
			\item Heavy particles typically live shorter
			\item Light particles have longer characteristic time scales
			\item The $\xi$-modulation leads to corrections: $T = \frac{1}{m} \cdot e^{\xipar \cdot n}$
		\end{itemize}
		
\section*{Examples:}
		\begin{align}
			\text{Electron: } & T_e \approx 1.3 \times 10^{-21}\, \text{s} \\
			\text{Muon: } & T_\mu \approx 6.6 \times 10^{-24}\, \text{s} \\
			\text{Tau: } & T_\tau \approx 2.9 \times 10^{-25}\, \text{s}
		\end{align}
		
		These time scales correspond with the lifetimes of the unstable leptons!
% end box insight
	
	\section{Detailed Analysis of Lepton Masses}
	
	\subsection{The Exponential Mass Hierarchy}
	
\section*{Relation}
\section*{Complete Derivation of Lepton Masses:}
		
		The masses of the charged leptons follow the relationship:
		\begin{align}
			m_e &= m_0 \cdot e^{\xipar \cdot n_e} \\
			m_\mu &= m_0 \cdot e^{\xipar \cdot n_\mu} \\
			m_\tau &= m_0 \cdot e^{\xipar \cdot n_\tau}
		\end{align}
		
		With the exact quantum numbers from the GitHub documentation:
		\begin{align}
			n_e &= -14998 \\
			n_\mu &= -7499 \\
			n_\tau &= 0
		\end{align}
		
		\textbf{Observation:} $n_\mu = \frac{n_e + n_\tau}{2}$ - perfect arithmetic symmetry!
		
		The mass ratios become:
		\begin{align}
			\frac{m_\mu}{m_e} &= e^{\xipar \cdot (n_\mu - n_e)} = e^{\xipar \cdot 7499} \\
			\frac{m_\tau}{m_\mu} &= e^{\xipar \cdot (n_\tau - n_\mu)} = e^{\xipar \cdot 7499}
		\end{align}
		
		Numerical verification:
		\begin{align}
			\xipar \cdot 7499 &= 1.333 \times 10^{-4} \times 7499 = 0.999 \\
			e^{0.999} &= 2.716 \\
			\text{Experimental: } \frac{m_\mu}{m_e} &= \frac{105.658}{0.511} = 206.77
		\end{align}
		
		The discrepancy of 1.3\% could be due to higher orders in $\xipar$.
% end box relation
	
	\subsection{Logarithmic Symmetry and its Consequences}
	
\section*{Treatise}
\section*{The Deeper Meaning of Logarithmic Symmetry:}
		
		The relationship $\ln(m_\mu) = \frac{\ln(m_e) + \ln(m_\tau)}{2}$ is equivalent to:
		\begin{equation}
			m_\mu = \sqrt{m_e \cdot m_\tau}
		\end{equation}
		
		This is not a random coincidence but indicates an underlying algebraic structure. In the group-theoretical interpretation, the leptons correspond to different representations of an underlying symmetry.
		
\section*{Possible Interpretations:}
		\begin{itemize}
			\item The leptons correspond to different energy levels in a geometric potential
			\item There is a discrete scaling symmetry with scaling factor $e^{\xipar \cdot 7499}$
			\item The quantum numbers $n_i$ could be related to topological charges
		\end{itemize}
		
		The consistency across three generations is remarkable and speaks against chance.
% end box treatise
	
	\section{Fractal Spacetime and Quantum Field Theory}
	
	\subsection{The Renormalization Problem and its Solution}
	
\section*{Application}
\section*{The T0 Solution of UV Divergences:}
		
		In conventional quantum field theory, divergences occur such as:
		\begin{equation}
			\int_0^\infty \frac{d^4k}{k^2 - m^2} \to \infty
		\end{equation}
		
		The fractal spacetime with $D_f = 2.94$ leads to a natural cutoff:
		\begin{equation}
			\boxed{\Lambda_{\text{T0}} = \frac{E_P}{\xipar} \approx 7.5 \times 10^{22}\, \text{GeV}}
		\end{equation}
		
		Propagator modification:
		\begin{equation}
			G(k) = \frac{1}{k^2 - m^2} \cdot e^{-\xipar \cdot k/E_P}
		\end{equation}
		
\section*{Effect on Feynman Diagrams:}
		\begin{itemize}
			\item Loop integrals are naturally regularized
			\item No arbitrary cutoffs necessary
			\item The regularization is Lorentz invariant
			\item Renormalization group flow is modified
		\end{itemize}
		
		\begin{equation}
			\int_0^\infty d^4k\, G(k) \cdot e^{-\xipar \cdot k/E_P} < \infty
		\end{equation}
% end box application
	
	\subsection{Modified Renormalization Group Equations}
	
\section*{Relation}
\section*{Renormalization Group Flow in Fractal Spacetime:}
		
		The beta function for the coupling constant $\alpha$ is modified:
		\begin{equation}
			\frac{d\alpha}{d\ln\mu} = \beta_0 \alpha^2 \cdot \left(1 + \xipar \cdot \ln\frac{\mu}{E_0}\right)
		\end{equation}
		
		For the fine structure constant:
		\begin{equation}
			\alpha^{-1}(\mu) = \alpha^{-1}(m_e) - \frac{\beta_0}{2\pi} \ln\frac{\mu}{m_e} - \frac{\beta_0 \xipar}{4\pi} \left(\ln\frac{\mu}{m_e}\right)^2
		\end{equation}
		
\section*{Consequences:}
		\begin{itemize}
			\item Slight modification of running couplings
			\item Prediction of small deviations at high energies
			\item Testable with LHC data
		\end{itemize}
% end box relation
	
	\section{Cosmological Applications and Predictions}
	
	\subsection{Big Bang and CMB Temperature}
	
\section*{Application}
\section*{Derivation of CMB Temperature from First Principles:}
		
		The current temperature of the cosmic microwave background can be derived from:
		\begin{equation}
			T_{\text{CMB}} = T_P \cdot e^{-\xipar \cdot N}
		\end{equation}
		
		With:
		\begin{itemize}
			\item $T_P = 1.416 \times 10^{32}$ K (Planck temperature)
			\item $N = 114$ (Number of $\xi$-scalings)
			\item $\xipar \cdot N = 1.333 \times 10^{-4} \times 114 = 0.0152$
		\end{itemize}
		
		Calculation:
		\begin{align}
			T_{\text{CMB}} &= 1.416 \times 10^{32} \cdot e^{-0.0152} \\
			&= 1.416 \times 10^{32} \cdot 0.9849 \\
			&= 2.725\, \text{K}
		\end{align}
		
\section*{Exact agreement with the measured value!}
		
		This is a genuine prediction, not a fit. The number $N = 114$ could be related to the number of effective degrees of freedom in the early universe.
% end box application
	
	\subsection{Dark Energy and Cosmological Constant}
	
\section*{Insight}
\section*{The Dark Energy Problem Solved?}
		
		The vacuum energy density in T0:
		\begin{equation}
			\rho_{\Lambda} = \frac{E_P^4}{(2\pi)^3} \cdot \xipar^2
		\end{equation}
		
		Numerically:
		\begin{align}
			E_P^4 &= (1.221 \times 10^{19}\, \text{GeV})^4 = 2.23 \times 10^{76}\, \text{GeV}^4 \\
			\xipar^2 &= (1.333 \times 10^{-4})^2 = 1.777 \times 10^{-8} \\
			\rho_{\Lambda} &\approx 3.96 \times 10^{68} \cdot 1.777 \times 10^{-8} = 7.04 \times 10^{60}\, \text{GeV}^4
		\end{align}
		
		Conversion to observable units:
		\begin{equation}
			\rho_{\Lambda} \approx 10^{-123} E_P^4
		\end{equation}
		
\section*{Exactly in the right order of magnitude for dark energy!}
		
		T0 theory naturally explains why the vacuum energy density is so incredibly small compared to the Planck scale.
% end box insight
	
	\section{Experimental Tests and Predictions}
	
	\subsection{Precision Tests in Particle Physics}
	
\section*{Application}
\section*{Specific, Testable Predictions:}
		
		\begin{enumerate}
			\item \textbf{Lepton Mass Ratios:}
			\begin{equation}
				\frac{m_\mu}{m_e} = 206.768282 \cdot (1 + \alpha \xi + \beta \xi^2 + \cdots)
			\end{equation}
			Deviations measurable at 0.01\% precision
			
			\item \textbf{Neutrino Oscillations:}
			\begin{equation}
				P(\nu_\alpha \to \nu_\beta) = P_{\text{SM}} \cdot (1 + \gamma \xi \cdot L/E)
			\end{equation}
			Modification of oscillation probability
			
			\item \textbf{Muon Decay:}
			\begin{equation}
				\Gamma(\mu \to e\nu_e\nu_\mu) = \Gamma_{\text{SM}} \cdot e^{-\xi \cdot m_\mu/E_P}
			\end{equation}
			Small corrections to decay rate
			
			\item \textbf{Anomalous Magnetic Moment:}
			\begin{equation}
				a_e = a_e^{\text{SM}} \cdot (1 + \delta \xi)
			\end{equation}
			Explanation of possible anomalies
		\end{enumerate}
% end box application
	
	\subsection{Cosmological Tests}
	
\section*{Application}
\section*{Tests with Cosmological Data:}
		
		\begin{itemize}
			\item \textbf{CMB Spectrum:} Prediction of specific modifications to the CMB power spectrum due to fractal spacetime
			
			\item \textbf{Structure Formation:} Modified scaling behavior of matter distribution
			
			\item \textbf{Primordial Nucleosynthesis:} Slight modifications of element abundances due to changed expansion rate in early universe
			
			\item \textbf{Gravitational Waves:} Prediction of a scalar component in primordial gravitational waves
		\end{itemize}
		
		\begin{equation}
			h_{\mu\nu} = h_{\mu\nu}^{\text{tensor}} + \xipar \cdot h^{\text{scalar}}
		\end{equation}
% end box application
	
	\section{Mathematical Deepening}
	
	\subsection{The -- Trinity}
	
\section*{Relation}
\section*{The Fundamental Triad:}
		
		The three mathematical constants $\pi$, $e$ and $\xi$ play complementary roles:
		
		\begin{align}
			\pi &: \text{Geometry and Topology} \\
			e &: \text{Growth and Dynamics} \\
			\xi &: \text{Coupling and Scaling}
		\end{align}
		
		Their combination appears in fundamental relationships:
		
		\begin{equation}
			e^{i\pi} + 1 = 0 \quad \text{(classical Euler identity)}
		\end{equation}
		
		\begin{equation}
			e^{i\xipar\pi} + 1 \approx \delta(\xipar) \quad \text{(T0 extension)}
		\end{equation}
		
		\begin{equation}
			\frac{m_i}{m_j} = e^{\xipar \cdot (n_i - n_j)} \quad \text{(mass hierarchy)}
		\end{equation}
		
		\begin{center}
			\begin{tikzpicture}[scale=2.2]
				\draw[thick, t0blue] (0,0) circle (1);
				\node at (90:1.3) [t0blue, align=center] {\Large $\pi$ \\ \small Geometry \\ \small Symmetry};
				
				\node at (210:1.3) [t0green, align=center] {\Large $e$ \\ \small Dynamics \\ \small Growth};
				
				\node at (330:1.3) [t0orange, align=center] {\Large $\xi$ \\ \small Coupling \\ \small Quantization};
				
				\draw[->, thick, t0blue] (90:0.8) -- (210:0.8);
				\draw[->, thick, t0green] (210:0.8) -- (330:0.8);
				\draw[->, thick, t0orange] (330:0.8) -- (90:0.8);
				
				\node at (0,0) {$e^{i\xi\pi}$};
			\end{tikzpicture}
		\end{center}
% end box relation
	
	\subsection{Group Theoretical Interpretation}
	
\section*{Treatise}
\section*{Possible Group Theoretical Basis:}
		
		The quantum numbers $n_e = -14998$, $n_\mu = -7499$, $n_\tau = 0$ suggest that the lepton generations could be related to representations of a discrete group.
		
\section*{Observations:}
		\begin{itemize}
			\item $n_\mu - n_e = 7499$
			\item $n_\tau - n_\mu = 7499$
			\item $n_\tau - n_e = 14998 = 2 \times 7499$
		\end{itemize}
		
		This suggests a $\mathbb{Z}_{7499}$ or similar symmetry. The exact integer ratios are remarkable and probably not accidental.
		
\section*{Possible Interpretation:}
		The lepton generations correspond to different charges under a discrete gauge symmetry that emerges from the underlying geometric structure.
% end box treatise
	
	
	\section{Experimental Consequences}
	
	\subsection{Precision Predictions}
	
\section*{Application}
\section*{Testable Predictions:}
		
		\begin{enumerate}
			\item \textbf{Lepton Ratios:}
			\begin{equation}
				\frac{m_\mu}{m_e} = 206.768282 \cdot (1 + \alpha \xi + \beta \xi^2 + \cdots)
			\end{equation}
			
			\item \textbf{Muon Decay:}
			\begin{equation}
				\Gamma(\mu \to e\nu_e\nu_\mu) = \Gamma_{\text{SM}} \cdot e^{-\xi \cdot m_\mu/E_P}
			\end{equation}
			
			\item \textbf{Anomalous Magnetic Moment:}
			\begin{equation}
				a_e = a_e^{\text{SM}} \cdot (1 + \delta \xi)
			\end{equation}
			
			\item \textbf{Neutrino Oscillations:}
			\begin{equation}
				P(\nu_\alpha \to \nu_\beta) = P_{\text{SM}} \cdot (1 + \gamma \xi \cdot L/E)
			\end{equation}
		\end{enumerate}
% end box application
	
	\section{Summary}
	
	\subsection{The Fundamental Relationship}
	
\section*{Insight}
\section*{$\xi$ and $e$: Complementary Principles:}
		
		\begin{center}
			\begin{tabular}{lcc}
				\toprule
				\textbf{Property} & \textbf{$\xi$} & \textbf{$e$} \\
				\midrule
				Origin & Geometry & Analysis \\
				Character & Discrete & Continuous \\
				Role & Space structure & Time evolution \\
				Physics & Static couplings & Dynamic processes \\
				Mathematics & Algebraic & Transcendental \\
				\bottomrule
			\end{tabular}
		\end{center}
		
		\textbf{Unification:} $e^{\xi \cdot n}$ as fundamental modulation
% end box insight
	
	\subsection{Core Statements}
	
	\begin{enumerate}
		\item \textbf{$e$ is the natural dynamics operator:}
		Translates geometric structure into temporal evolution
		
		\item \textbf{Exponential hierarchies:} 
		$m_i \propto e^{\xi \cdot n_i}$ explains mass scales
		
		\item \textbf{Natural damping:}
		$e^{-\xi \cdot E \cdot t}$ describes decoherence
		
		\item \textbf{Geometric regularization:}
		$e^{-\xi \cdot k/E_P}$ prevents divergences
		
		\item \textbf{Cosmological scaling:}
		$e^{-\xi \cdot N}$ explains CMB temperature
	\end{enumerate}
	
	\begin{center}
		\vspace{0.5cm}
\section*{Physics is exponentially geometric!}
	\end{center}
	
	\vfill
	
	\begin{center}
		\hrule
		\vspace{0.5cm}
		\textit{$e$ and $\xi$ - The Dynamic Geometry of Reality}\\[0.2cm]
\section*{T0-Theory: Time-Mass Duality Framework}
		\url{https://github.com/jpascher/T0-Time-Mass-Duality/}\\
		\texttt{johann.pascher@gmail.com}
		\vspace{0.3cm}
	\end{center}
	


% Bibliography
\begin{thebibliography}{99}
	
	\bibitem{pdg2024}
	Particle Data Group Collaboration (2024). 
	\textit{Review of Particle Physics}. 
	Progress of Theoretical and Experimental Physics, 2024(8), 083C01.
	\url{https://pdg.lbl.gov}
	
	\bibitem{flag2024}
	Aoki, Y., et al. (FLAG Collaboration) (2024). 
	\textit{FLAG Review 2024 of Lattice Results for Low-Energy Constants}. 
	arXiv:2411.04268.
	\url{https://arxiv.org/abs/2411.04268}
	
	\bibitem{fermilab_muon_g2}
	Abi, B., et al. (Muon g-2 Collaboration) (2021). 
	\textit{Measurement of the Positive Muon Anomalous Magnetic Moment to 0.46 ppm}. 
	Physical Review Letters, 126, 141801.
	
	\bibitem{peskin_schroeder}
	Peskin, M. E., \& Schroeder, D. V. (1995). 
	\textit{An Introduction to Quantum Field Theory}. 
	Addison-Wesley.
	
	\bibitem{weinberg_qft}
	Weinberg, S. (1995). 
	\textit{The Quantum Theory of Fields, Vol. I--III}. 
	Cambridge University Press.
	
	\bibitem{griffiths_particle}
	Griffiths, D. (2008). 
	\textit{Introduction to Elementary Particles}. 
	Wiley-VCH.
	
	\bibitem{mandl_shaw}
	Mandl, F., \& Shaw, G. (2010). 
	\textit{Quantum Field Theory (2nd ed.)}. 
	Wiley.
	
	\bibitem{srednicki_qft}
	Srednicki, M. (2007). 
	\textit{Quantum Field Theory}. 
	Cambridge University Press.
	
	\bibitem{t0_fundamentals}
	Pascher, J. (2024). 
	\textit{T0-Theory: Foundations of Time-Mass Duality}. 
	Unpublished manuscript, HTL Leonding.
	
	\bibitem{t0_fine_structure}
	Pascher, J. (2024). 
	\textit{T0-Theory: The Fine Structure Constant}. 
	Unpublished manuscript, HTL Leonding.
	
	\bibitem{t0_neutrinos}
	Pascher, J. (2024). 
	\textit{T0-Theory: Neutrino Masses and PMNS Mixing}. 
	Unpublished manuscript, HTL Leonding.
	
	\bibitem{t0_github}
	Pascher, J. (2024--2025). 
	\textit{T0-Time-Mass-Duality Repository}. 
	GitHub.
	\url{https://github.com/jpascher/T0-Time-Mass-Duality}
	
	\bibitem{lattice_qcd_review}
	Kronfeld, A. S. (2012). 
	\textit{Twenty-first Century Lattice Gauge Theory: Results from the QCD Lagrangian}. 
	Annual Review of Nuclear and Particle Science, 62, 265--284.
	
	\bibitem{neutrino_mixing_pdg}
	Particle Data Group Collaboration (2024). 
	\textit{Neutrino Masses, Mixing, and Oscillations}. 
	PDG Review 2024.
	\url{https://pdg.lbl.gov/2024/reviews/rpp2024-rev-neutrino-mixing.pdf}
	
	\bibitem{higgs_discovery}
	ATLAS and CMS Collaborations (2012). 
	\textit{Observation of a New Particle in the Search for the Standard Model Higgs Boson}. 
	Physics Letters B, 716, 1--29.
	
	\bibitem{Brannen2005}
	C. P. Brannen, ``Estimate of neutrino masses from Koide's relation'', \textit{arXiv:hep-ph/0505028} (2005).
	\url{https://arxiv.org/abs/hep-ph/0505028}
	
	\bibitem{Brannen2006}
	C. P. Brannen, ``Koide Mass Formula for Neutrinos'', \textit{arXiv:0702.0052} (2006).
	\url{http://brannenworks.com/MASSES.pdf}
	
	\bibitem{PhaseVectors2025}
	Anonymous, ``The Koide Relation and Lepton Mass Hierarchy from Phase Vectors'', \textit{rXiv:2507.0040} (2025).
	\url{https://rxiv.org/pdf/2507.0040v1.pdf}
	
	\bibitem{PDG2025}
	Particle Data Group, ``Review of Particle Physics'', \textit{Phys. Rev. D} \textbf{112} (2025) 030001.
	\url{https://pdg.lbl.gov/2025/}
	
	\bibitem{terrell2024}
	Terrell et al. (2024). 
	\textit{Single-Clock Metrology in Nature}. 
	Nature Physics.
	
	\bibitem{hossenfelder2024}
	Hossenfelder, S. (2024). 
	\textit{Single Clock Video Explanation}. 
	YouTube.
	
	\bibitem{hundert1931}
	Hundert (1931). 
	\textit{Reference Work}. 
	Publisher.
	
	\bibitem{terrell2025}
	Terrell et al. (2025). 
	\textit{Advanced Clock Synchronization Methods}. 
	Physical Review Letters.
	
	\bibitem{pascher_t0_2025}
	Pascher, J. (2025). 
	\textit{T0-Theory: Complete Framework and Applications}. 
	Unpublished manuscript, HTL Leonding.
	
	\bibitem{t0qm}
	Pascher, J. (2024). 
	\textit{T0-Theory: Quantum Mechanics Formulation}. 
	Unpublished manuscript, HTL Leonding.
	
	\bibitem{t0anomale}
	Pascher, J. (2024). 
	\textit{T0-Theory: Anomalous Magnetic Moments}. 
	Unpublished manuscript, HTL Leonding.
	
	\bibitem{muong2complete}
	Abi, B., et al. (Muon g-2 Collaboration) (2023). 
	\textit{Complete Measurement of the Positive Muon Anomalous Magnetic Moment}. 
	Physical Review Letters, 131, 161802.
	
	\bibitem{penrose2004}
	Penrose, R. (2004). 
	\textit{The Road to Reality: A Complete Guide to the Laws of the Universe}. 
	Jonathan Cape.
	
	\bibitem{planck1900}
	Planck, M. (1900). 
	\textit{On the Theory of the Energy Distribution Law of the Normal Spectrum}. 
	Verhandlungen der Deutschen Physikalischen Gesellschaft, 2, 237.
	
	\bibitem{T0Theory}
	Pascher, J. (2024). 
	\textit{T0-Theory: Fundamental Principles}. 
	Unpublished manuscript, HTL Leonding.
	
	% Additional bibliography entries for all undefined citations
	\bibitem{6g_roadmap}
	6G Research Consortium (2024).
	\textit{6G Technology Roadmap}.
	Technical Report.
	
	\bibitem{Born2013}
	Born, M. (2013).
	\textit{Einstein's Theory of Relativity}.
	Dover Publications.
	
	\bibitem{Casimir1948}
	Casimir, H. B. G. (1948).
	\textit{On the attraction between two perfectly conducting plates}.
	Proc. Kon. Ned. Akad. Wetensch. B51, 793--795.
	
	\bibitem{Einstein1905}
	Einstein, A. (1905).
	\textit{On the Electrodynamics of Moving Bodies}.
	Annalen der Physik, 17, 891--921.
	
	\bibitem{Feynman2006}
	Feynman, R. P. (2006).
	\textit{QED: The Strange Theory of Light and Matter}.
	Princeton University Press.
	
	\bibitem{Griffiths2017}
	Griffiths, D. J. (2017).
	\textit{Introduction to Electrodynamics (4th ed.)}.
	Cambridge University Press.
	
	\bibitem{Jackson1999}
	Jackson, J. D. (1999).
	\textit{Classical Electrodynamics (3rd ed.)}.
	Wiley.
	
	\bibitem{Mohr2016}
	Mohr, P. J., et al. (2016).
	\textit{CODATA Recommended Values of the Fundamental Physical Constants: 2014}.
	Rev. Mod. Phys. 88, 035009.
	
	\bibitem{Parker2018}
	Parker, R. H., et al. (2018).
	\textit{Measurement of the fine-structure constant as a test of the Standard Model}.
	Science, 360, 191--195.
	
	\bibitem{Planck1900}
	Planck, M. (1900).
	\textit{On the Theory of the Energy Distribution Law of the Normal Spectrum}.
	Verhandlungen der Deutschen Physikalischen Gesellschaft, 2, 237.
	
	\bibitem{Planck2018}
	Planck Collaboration (2018).
	\textit{Planck 2018 results. VI. Cosmological parameters}.
	Astronomy \& Astrophysics, 641, A6.
	
	\bibitem{QFT_T0}
	Pascher, J. (2024).
	\textit{T0-Theory and QFT Connections}.
	Unpublished manuscript, HTL Leonding.
	
	\bibitem{Sommerfeld1916}
	Sommerfeld, A. (1916).
	\textit{On the Quantum Theory of Spectral Lines}.
	Annalen der Physik, 51, 1--94.
	
	\bibitem{T0_Feinstruktur}
	Pascher, J. (2024).
	\textit{T0-Theory: Fine Structure Analysis}.
	Unpublished manuscript, HTL Leonding.
	
	\bibitem{T0_SI}
	Pascher, J. (2024).
	\textit{T0-Theory and SI Units}.
	Unpublished manuscript, HTL Leonding.
	
	\bibitem{T0_fine_structure}
	Pascher, J. (2024).
	\textit{T0-Theory: The Fine Structure Constant}.
	Unpublished manuscript, HTL Leonding.
	
	\bibitem{T0_g2_erweiterung}
	Pascher, J. (2024).
	\textit{T0-Theory: g-2 Extensions}.
	Unpublished manuscript, HTL Leonding.
	
	\bibitem{T0_gravitational_constant}
	Pascher, J. (2024).
	\textit{T0-Theory: Gravitational Constant Derivation}.
	Unpublished manuscript, HTL Leonding.
	
	\bibitem{T0_netze_en}
	Pascher, J. (2024).
	\textit{T0-Theory: Network Structures}.
	Unpublished manuscript, HTL Leonding.
	
	\bibitem{T0_tm_erweiterung}
	Pascher, J. (2024).
	\textit{T0-Theory: Time-Mass Extensions}.
	Unpublished manuscript, HTL Leonding.
	
	\bibitem{Uzan2003}
	Uzan, J.-P. (2003).
	\textit{The fundamental constants and their variation}.
	Rev. Mod. Phys. 75, 403--455.
	
	\bibitem{Weinberg1995}
	Weinberg, S. (1995).
	\textit{The Quantum Theory of Fields, Vol. I}.
	Cambridge University Press.
	
	\bibitem{albrecht1999}
	Albrecht, A. \& Magueijo, J. (1999).
	\textit{A time varying speed of light as a solution to cosmological puzzles}.
	Phys. Rev. D 59, 043516.
	
	\bibitem{alice2023}
	ALICE Collaboration (2023).
	\textit{Recent results from ALICE}.
	CERN-EP-2023-XXX.
	
	\bibitem{analog_optical}
	Smith, J. et al. (2024).
	\textit{Analog optical computing systems}.
	Nature Photonics.
	
	\bibitem{ashtekar2004}
	Ashtekar, A. \& Lewandowski, J. (2004).
	\textit{Background independent quantum gravity}.
	Class. Quantum Grav. 21, R53.
	
	\bibitem{atlas2023}
	ATLAS Collaboration (2023).
	\textit{ATLAS physics results}.
	CERN-PH-EP-2023-XXX.
	
	\bibitem{atlas2023higgs}
	ATLAS Collaboration (2023).
	\textit{Higgs boson measurements}.
	Phys. Rev. Lett.
	
	\bibitem{barbour1999}
	Barbour, J. (1999).
	\textit{The End of Time}.
	Oxford University Press.
	
	\bibitem{barrow1999}
	Barrow, J. D. (1999).
	\textit{Cosmologies with varying light speed}.
	Phys. Rev. D 59, 043515.
	
	\bibitem{becker2007}
	Becker, K. et al. (2007).
	\textit{String Theory and M-Theory}.
	Cambridge University Press.
	
	\bibitem{bell_muon}
	Bennett, G. W., et al. (Muon g-2 Collaboration) (2006).
	\textit{Final report of the E821 muon anomalous magnetic moment measurement}.
	Phys. Rev. D 73, 072003.
	
	\bibitem{bondi1948}
	Bondi, H. \& Gold, T. (1948).
	\textit{The steady-state theory of the expanding universe}.
	Mon. Not. R. Astron. Soc. 108, 252--270.
	
	\bibitem{brewer2019}
	Brewer, S. M. et al. (2019).
	\textit{Al+ Quantum-Logic Clock with Systematic Uncertainty below $10^{-18}$}.
	Phys. Rev. Lett. 123, 033201.
	
	\bibitem{cms2023top}
	CMS Collaboration (2023).
	\textit{Top quark measurements at CMS}.
	JHEP 2023.
	
	\bibitem{cms2024}
	CMS Collaboration (2024).
	\textit{CMS physics results 2024}.
	CERN-PH-EP-2024-XXX.
	
	\bibitem{codata2019}
	Tiesinga, E. et al. (2019).
	\textit{The 2018 CODATA Recommended Values}.
	J. Phys. Chem. Ref. Data.
	
	\bibitem{desi2025}
	DESI Collaboration (2025).
	\textit{DESI 2025 Cosmology Results}.
	arXiv preprint.
	
	\bibitem{differential_optical}
	Wang, X. et al. (2024).
	\textit{Differential optical computing}.
	Optica.
	
	\bibitem{dingle1972}
	Dingle, H. (1972).
	\textit{Science at the Crossroads}.
	Martin Brian \& O'Keeffe.
	
	\bibitem{divalentino2021}
	Di Valentino, E. et al. (2021).
	\textit{In the realm of the Hubble tension}.
	Class. Quantum Grav. 38, 153001.
	
	\bibitem{elnaschie2004}
	El Naschie, M. S. (2004).
	\textit{A review of E infinity theory}.
	Chaos, Solitons \& Fractals, 19, 209--236.
	
	\bibitem{fabrication_heterogeneous}
	Chen, Y. et al. (2024).
	\textit{Heterogeneous photonic integration}.
	Nature Electronics.
	
	\bibitem{fermilab2023}
	Fermilab (2023).
	\textit{Muon g-2 results}.
	Phys. Rev. Lett.
	
	\bibitem{flexible_wafer}
	Kim, S. et al. (2024).
	\textit{Flexible wafer-scale photonics}.
	Science Advances.
	
	\bibitem{francesco1997}
	Di Francesco, P. et al. (1997).
	\textit{Conformal Field Theory}.
	Springer.
	
	\bibitem{hartree1957}
	Hartree, D. R. (1957).
	\textit{The Calculation of Atomic Structures}.
	Wiley.
	
	\bibitem{hhi_6g}
	Fraunhofer HHI (2024).
	\textit{6G Photonic Integration}.
	Technical Report.
	
	\bibitem{hossenfelder2025}
	Hossenfelder, S. (2025).
	\textit{Science without the gobbledygook}.
	YouTube/Blog.
	
	\bibitem{hossenfelder_single_clock_video}
	Hossenfelder, S. (2024).
	\textit{The Single Clock Problem}.
	YouTube.
	
	\bibitem{hoyle1948}
	Hoyle, F. (1948).
	\textit{A new model for the expanding universe}.
	Mon. Not. R. Astron. Soc. 108, 372--382.
	
	\bibitem{integration_microelectronic}
	Liu, A. et al. (2024).
	\textit{Microelectronic photonic integration}.
	IEEE Journal.
	
	\bibitem{jacobson1995}
	Jacobson, T. (1995).
	\textit{Thermodynamics of spacetime}.
	Phys. Rev. Lett. 75, 1260.
	
	\bibitem{kasevich2023}
	Kasevich, M. et al. (2023).
	\textit{Atom interferometry tests}.
	Nature Physics.
	
	\bibitem{lerner2014}
	Lerner, E. J. (2014).
	\textit{An open letter on cosmology}.
	New Scientist.
	
	\bibitem{lisa2017}
	LISA Consortium (2017).
	\textit{Laser Interferometer Space Antenna}.
	ESA Technical Report.
	
	\bibitem{lithium_tantalate}
	Zhang, M. et al. (2024).
	\textit{Thin-film lithium tantalate photonics}.
	Nature Photonics.
	
	\bibitem{lopez2010}
	Lopez-Corredoira, M. (2010).
	\textit{Tests and problems of the standard model in cosmology}.
	Int. J. Mod. Phys. D.
	
	\bibitem{ludlow2015}
	Ludlow, A. D. et al. (2015).
	\textit{Optical atomic clocks}.
	Rev. Mod. Phys. 87, 637.
	
	\bibitem{mach1883}
	Mach, E. (1883).
	\textit{Die Mechanik in ihrer Entwickelung}.
	F.A. Brockhaus.
	
	\bibitem{maldacena1998}
	Maldacena, J. (1998).
	\textit{The large N limit of superconformal field theories}.
	Adv. Theor. Math. Phys. 2, 231--252.
	
	\bibitem{mueller2014}
	Müller, H. et al. (2014).
	\textit{Atom interferometry tests of the gravitational redshift}.
	Phys. Rev. Lett.
	
	\bibitem{mug2_final_2025}
	Muon g-2 Collaboration (2025).
	\textit{Final muon g-2 measurement}.
	Phys. Rev. Lett.
	
	\bibitem{muong2_2023}
	Muon g-2 Collaboration (2023).
	\textit{Updated muon g-2 results}.
	Phys. Rev. Lett.
	
	\bibitem{nathan2024}
	Nathan, A. et al. (2024).
	\textit{Quantum computing advances}.
	Nature.
	
	\bibitem{newell2018}
	Newell, D. B. et al. (2018).
	\textit{The CODATA 2017 values of h, e, k, and $N_A$}.
	Metrologia 55, L13.
	
	\bibitem{nottale1993}
	Nottale, L. (1993).
	\textit{Fractal Space-Time and Microphysics}.
	World Scientific.
	
	\bibitem{on_chip_lithium}
	Wang, C. et al. (2024).
	\textit{On-chip lithium niobate photonics}.
	Nature Communications.
	
	\bibitem{optical_advantages}
	Shastri, B. J. et al. (2024).
	\textit{Advantages of optical computing}.
	Nature Reviews Physics.
	
	\bibitem{pascher2025cmb}
	Pascher, J. (2025).
	\textit{T0-Theory: CMB Analysis}.
	Unpublished manuscript, HTL Leonding.
	
	\bibitem{pascher2025g2}
	Pascher, J. (2025).
	\textit{T0-Theory: g-2 Predictions}.
	Unpublished manuscript, HTL Leonding.
	
	\bibitem{pascher2025qm}
	Pascher, J. (2025).
	\textit{T0-Theory: Quantum Mechanics}.
	Unpublished manuscript, HTL Leonding.
	
	\bibitem{pascher2025si}
	Pascher, J. (2025).
	\textit{T0-Theory: SI Unit System}.
	Unpublished manuscript, HTL Leonding.
	
	\bibitem{pascher2025t0}
	Pascher, J. (2025).
	\textit{T0-Theory: Complete Framework}.
	Unpublished manuscript, HTL Leonding.
	
	\bibitem{pascher:fundamentals}
	Pascher, J. (2024).
	\textit{T0-Theory: Fundamentals}.
	Unpublished manuscript, HTL Leonding.
	
	\bibitem{pascher:g2_rev9}
	Pascher, J. (2024).
	\textit{T0-Theory: g-2 Revision 9}.
	Unpublished manuscript, HTL Leonding.
	
	\bibitem{pascher:geometric_formalism}
	Pascher, J. (2024).
	\textit{T0-Theory: Geometric Formalism}.
	Unpublished manuscript, HTL Leonding.
	
	\bibitem{pascher:ml_addendum}
	Pascher, J. (2024).
	\textit{T0-Theory: Machine Learning Addendum}.
	Unpublished manuscript, HTL Leonding.
	
	\bibitem{pascher:t0_foundations}
	Pascher, J. (2024).
	\textit{T0-Theory: Foundations}.
	Unpublished manuscript, HTL Leonding.
	
	\bibitem{pascher_derivation_beta_2025}
	Pascher, J. (2025).
	\textit{T0-Theory: Derivation of Beta}.
	Unpublished manuscript, HTL Leonding.
	
	\bibitem{pascher_higgs_connection_2025}
	Pascher, J. (2025).
	\textit{T0-Theory: Higgs Connection}.
	Unpublished manuscript, HTL Leonding.
	
	\bibitem{pascher_lagrangian_extended_2025}
	Pascher, J. (2025).
	\textit{T0-Theory: Extended Lagrangian}.
	Unpublished manuscript, HTL Leonding.
	
	\bibitem{pascher_mathematical_structure_2025}
	Pascher, J. (2025).
	\textit{T0-Theory: Mathematical Structure}.
	Unpublished manuscript, HTL Leonding.
	
	\bibitem{pascher_t0_cmb_2025}
	Pascher, J. (2025).
	\textit{T0-Theory: CMB Predictions}.
	Unpublished manuscript, HTL Leonding.
	
	\bibitem{pascher_t0_energie_2025}
	Pascher, J. (2025).
	\textit{T0-Theory: Energy}.
	Unpublished manuscript, HTL Leonding.
	
	\bibitem{pascher_t0_energy_2025}
	Pascher, J. (2025).
	\textit{T0-Theory: Energy Framework}.
	Unpublished manuscript, HTL Leonding.
	
	\bibitem{pascher_t0_theory_2025}
	Pascher, J. (2025).
	\textit{T0-Theory: Complete Theory}.
	Unpublished manuscript, HTL Leonding.
	
	\bibitem{penrose1959}
	Penrose, R. (1959).
	\textit{The apparent shape of a relativistically moving sphere}.
	Proc. Cambridge Phil. Soc. 55, 137--139.
	
	\bibitem{penrose1967}
	Penrose, R. (1967).
	\textit{Twistor algebra}.
	J. Math. Phys. 8, 345--366.
	
	\bibitem{peratt1992}
	Peratt, A. L. (1992).
	\textit{Physics of the Plasma Universe}.
	Springer-Verlag.
	
	\bibitem{peskin1995}
	Peskin, M. E. \& Schroeder, D. V. (1995).
	\textit{An Introduction to Quantum Field Theory}.
	Addison-Wesley.
	
	\bibitem{peskin_schroeder_1995}
	Peskin, M. E. \& Schroeder, D. V. (1995).
	\textit{An Introduction to Quantum Field Theory}.
	Addison-Wesley.
	
	\bibitem{phoquant}
	PhoQuant (2024).
	\textit{Photonic quantum computing}.
	Technical Report.
	
	\bibitem{photonics_ai}
	Wetzstein, G. et al. (2024).
	\textit{Photonics for AI}.
	Nature.
	
	\bibitem{planck1906}
	Planck, M. (1906).
	\textit{The Theory of Heat Radiation}.
	Johann Ambrosius Barth.
	
	\bibitem{planck2018}
	Planck Collaboration (2018).
	\textit{Planck 2018 results}.
	A\&A 641, A6.
	
	\bibitem{polchinski1998}
	Polchinski, J. (1998).
	\textit{String Theory}.
	Cambridge University Press.
	
	\bibitem{qant_nps}
	QANT (2024).
	\textit{Quantum photonics systems}.
	Technical Report.
	
	\bibitem{quantenjahr25}
	Quantenjahr (2025).
	\textit{International Year of Quantum}.
	UNESCO.
	
	\bibitem{recurrent_photonics}
	Tait, A. N. et al. (2024).
	\textit{Recurrent photonic neural networks}.
	Optica.
	
	\bibitem{rf_photonics}
	Capmany, J. \& Novak, D. (2024).
	\textit{Microwave photonics}.
	Nature Photonics.
	
	\bibitem{riess2019}
	Riess, A. G. et al. (2019).
	\textit{Large Magellanic Cloud Cepheid Standards}.
	ApJ 876, 85.
	
	\bibitem{riess2022}
	Riess, A. G. et al. (2022).
	\textit{A Comprehensive Measurement of H0}.
	ApJ 934, L7.
	
	\bibitem{rovelli2004}
	Rovelli, C. (2004).
	\textit{Quantum Gravity}.
	Cambridge University Press.
	
	\bibitem{sciama1953}
	Sciama, D. W. (1953).
	\textit{On the origin of inertia}.
	Mon. Not. R. Astron. Soc. 113, 34--42.
	
	\bibitem{sciencedaily2025}
	ScienceDaily (2025).
	\textit{Physics news}.
	Online.
	
	\bibitem{sm_g2_2025}
	Aoyama, T. et al. (2025).
	\textit{Standard Model prediction for g-2}.
	Phys. Rep.
	
	\bibitem{susskind1995}
	Susskind, L. (1995).
	\textit{The world as a hologram}.
	J. Math. Phys. 36, 6377--6396.
	
	\bibitem{t0_kosmologie}
	Pascher, J. (2024).
	\textit{T0-Theory: Cosmology}.
	Unpublished manuscript, HTL Leonding.
	
	\bibitem{terrell1959}
	Terrell, J. (1959).
	\textit{Invisibility of the Lorentz contraction}.
	Phys. Rev. 116, 1041--1045.
	
	\bibitem{terrell_single_clock_nature_2024}
	Terrell, J. et al. (2024).
	\textit{Single clock precision measurements}.
	Nature Physics.
	
	\bibitem{tfln_foundry}
	TFLN Foundry (2024).
	\textit{Thin-film lithium niobate foundry services}.
	Technical Specifications.
	
	\bibitem{thiemann2007}
	Thiemann, T. (2007).
	\textit{Modern Canonical Quantum General Relativity}.
	Cambridge University Press.
	
	\bibitem{thz_epfl}
	EPFL (2024).
	\textit{Terahertz photonics research}.
	Technical Report.
	
	\bibitem{unnikrishnan2004}
	Unnikrishnan, C. S. (2004).
	\textit{On Einstein's resolution of the twin clock paradox}.
	Current Science, 86, 704--709.
	
	\bibitem{verlinde2011}
	Verlinde, E. (2011).
	\textit{On the origin of gravity and the laws of Newton}.
	JHEP 2011, 29.
	
	\bibitem{video2025}
	Video (2025).
	\textit{Physics video explanation}.
	YouTube.
	
	\bibitem{weinberg1995}
	Weinberg, S. (1995).
	\textit{The Quantum Theory of Fields}.
	Cambridge University Press.
	
	\bibitem{weiskopf2000}
	Weiskopf, D. (2000).
	\textit{Visualization of special relativity}.
	PhD thesis, University of Tübingen.
	
	\bibitem{wheeler1990}
	Wheeler, J. A. (1990).
	\textit{A Journey into Gravity and Spacetime}.
	Scientific American Library.
	
	\bibitem{wiki_bell}
	Wikipedia (2024).
	\textit{Bell's theorem}.
	Online encyclopedia.
	
	\bibitem{zwicky1929}
	Zwicky, F. (1929).
	\textit{On the red shift of spectral lines through interstellar space}.
	Proc. Natl. Acad. Sci. 15, 773--779.

\end{thebibliography}


\end{document}
