\documentclass[11pt,a4paper]{article}
\usepackage[a4paper,margin=2cm]{geometry}
\usepackage[utf8]{inputenc}
\usepackage[english]{babel}
\usepackage{lmodern}
\renewcommand{\familydefault}{\sfdefault}

\usepackage{amsmath,amssymb,amsthm}
\usepackage{graphicx}
\usepackage[unicode,pdfencoding=auto,hypertexnames=false]{hyperref}
\usepackage{booktabs}
\usepackage{longtable}
\usepackage{array}
\usepackage{siunitx}
\usepackage{fancyhdr}
\usepackage{float}
\usepackage{tikz}
% tcolorbox removed for standalone
% tcbset removed
\tikzset{
  t0blue/.style={draw=blue,fill=blue!10},
  t0red/.style={draw=red,fill=red!10},
  t0green/.style={draw=green!50!black,fill=green!10},
  t0orange/.style={draw=orange,fill=orange!10},
}
\usepackage{setspace}
\usepackage{enumitem}
\usepackage{adjustbox}
\usepackage{xcolor}

% Define colors for xcolor package
\definecolor{t0green}{RGB}{34,139,34}
\definecolor{t0blue}{RGB}{0,0,255}
\definecolor{t0red}{RGB}{255,0,0}
\definecolor{t0orange}{RGB}{255,165,0}

% Define custom column types for tables
\newcolumntype{L}[1]{>{\raggedright\arraybackslash}p{#1}}
\newcolumntype{C}[1]{>{\centering\arraybackslash}p{#1}}
\newcolumntype{R}[1]{>{\raggedleft\arraybackslash}p{#1}}

\setlength{\parindent}{0pt}
\setlength{\parskip}{6pt}

\hypersetup{
  colorlinks=true,
  linkcolor=blue,
  citecolor=blue,
  urlcolor=blue
}
\pagestyle{fancy}
\setlength{\headheight}{28pt}

\newcommand{\checkmarkx}{\checkmark}
\newcommand{\warningx}{\textbf{!}}

% Makros aus Einzel-Dokumenten (Fallback-Definitionen)
\newcommand{\mytimes}{\times}
\newcommand{\myapprox}{\approx}
\newcommand{\mysim}{\sim}
\newcommand{\myomega}{\omega}
\newcommand{\mypi}{\pi}
\newcommand{\myrightarrow}{\rightarrow}
\newcommand{\mypropto}{\propto}
\newcommand{\deltafield}{\delta\phi}
\newcommand{\xipar}{\xi}
\newcommand{\xiT}{\xi}
\newcommand{\lambdah}{\lambda_h}

% Additional macros used in chapter files
\newcommand{\Kfrak}{K_{\text{frak}}}  % Fractal correction factor
\newcommand{\Dfrak}{D_f}              % Fractal dimension
\newcommand{\betapar}{\beta}          % T0 beta parameter
\newcommand{\alphapar}{\alpha}        % T0 alpha parameter
\newcommand{\Efield}{E}               % Energy field
% Note: checkmarkxa/warningxa are variants used in auto-generated chapter files
\newcommand{\checkmarkxa}{\checkmark}
\newcommand{\warningxa}{\textbf{!}}

% Additional T0-specific macros
\newcommand{\xigeom}{\xi_{\text{geom}}}  % Geometric xi
\newcommand{\lP}{\ell_P}                  % Planck length
\newcommand{\rzero}{r_0}                  % Characteristic radius
\newcommand{\xirat}{\xi_{\text{rat}}}     % Xi ratio
\newcommand{\tzero}{t_0}                  % Characteristic time
\newcommand{\natunits}{\text{(nat. units)}}  % Natural units annotation
\newcommand{\myRightarrow}{\Rightarrow}   % Arrow variant
\newcommand{\Lag}{\mathcal{L}}            % Lagrangian

% Physics macros used in chapter files
\newcommand{\CQCD}{C_{\text{QCD}}}        % QCD correction
\newcommand{\EP}{E_P}                     % Planck energy
\newcommand{\Ee}{E_e}                     % Electron energy
\newcommand{\Emu}{E_\mu}                  % Muon energy
\newcommand{\Exi}{E_\xi}                  % Xi energy
\newcommand{\Ezero}{E_0}                  % Characteristic energy
\newcommand{\Hubble}{H}                   % Hubble constant
\newcommand{\Kspec}{K_{\text{spec}}}      % Spectral correction
\newcommand{\Lambdat}{\Lambda_t}          % Time-related cosmological constant
\newcommand{\Leff}{\mathcal{L}_{\text{eff}}}  % Effective Lagrangian
\newcommand{\Lorentz}{\mathcal{L}}        % Lorentz symbol
\newcommand{\Lxi}{L_\xi}                  % Xi length
\newcommand{\Tfield}{T}                   % Time field
\newcommand{\Weyl}{W}                     % Weyl tensor/symbol
\newcommand{\alphaEMSI}{\alpha_{\text{EM,SI}}}  % EM alpha in SI
\newcommand{\alphaEMnat}{\alpha_{\text{EM,nat}}}  % EM alpha in natural units
\newcommand{\alphaem}{\alpha_{\text{em}}} % Electromagnetic alpha
\newcommand{\betaTSI}{\beta_{T,\text{SI}}}  % Beta in SI
\newcommand{\betaTnat}{\beta_{T,\text{nat}}}  % Beta in natural units
\newcommand{\deltam}{\delta m}            % Mass difference
\newcommand{\phiT}{\phi_T}                % T-field phi
\newcommand{\tP}{t_P}                     % Planck time
\newcommand{\rhoCMB}{\rho_{\text{CMB}}}   % CMB density
\newcommand{\rhoCasimir}{\rho_{\text{Casimir}}}  % Casimir density

% Table formatting
\usepackage{multirow}

% Additional physics macros
\newcommand{\Riem}{\mathcal{R}}           % Riemann tensor
\newcommand{\ZPinch}{Z_{\text{pinch}}}    % Z-pinch
\newcommand{\SynchPower}{P_{\text{synch}}} % Synchrotron power
\newcommand{\Rzero}{R_0}                  % Characteristic radius
\newcommand{\alphafine}{\alpha}           % Fine structure constant
\newcommand{\Etau}{E_\tau}                % Tau energy
\newcommand{\deltaE}{\delta E}            % Energy deviation
\newcommand{\EPlanck}{E_P}                % Planck energy
\newcommand{\pichar}{\pi}                 % Pi character
\newcommand{\alphaWSI}{\alpha_{W,\text{SI}}}  % Wien alpha in SI
\newcommand{\alphaWnat}{\alpha_{W,\text{nat}}}  % Wien alpha in natural units

% Einfache abstract-Umgebung für Kapitel:
\newenvironment{abstract}{%
  \begin{center}\bfseries Abstract\end{center}\small
}{\par}


\title{NoGoEn}
\author{J. Pascher}
\date{\today}

\begin{document}
\maketitle

\section*{Nogoen (NoGoEn)}

	\begin{abstract}
		This document presents a comprehensive theoretical analysis of how the T0-energy field formulation confronts and potentially circumvents fundamental no-go theorems in quantum mechanics, particularly Bell's theorem and the Kochen-Specker theorem. We demonstrate that T0 theory employs a sophisticated strategy based on "superdeterminism" and violation of measurement freedom assumptions to reproduce quantum mechanical correlations while maintaining local realism. Through detailed mathematical analysis, we show that T0 can violate Bell inequalities via spatially extended energy field correlations that couple measurement apparatus orientations with quantum system properties. While this approach is mathematically consistent and offers testable predictions, it comes at the philosophical cost of restricting measurement freedom and introducing controversial superdeterministic elements. The analysis reveals both the theoretical elegance and the conceptual challenges of attempting to restore deterministic local realism in quantum mechanics.
	\end{abstract}
	
	
	\section{Introduction: The Fundamental Challenge}
	
	\subsection{The No-Go Theorem Landscape}
	
	Quantum mechanics faces several fundamental no-go theorems that constrain possible interpretations:
	
	\begin{enumerate}
		\item \textbf{Bell's Theorem (1964)}: No local realistic theory can reproduce all quantum mechanical predictions
		\item \textbf{Kochen-Specker Theorem (1967)}: Quantum observables cannot have simultaneous definite values
		\item \textbf{PBR Theorem (2012)}: Quantum states are ontological, not merely epistemological
		\item \textbf{Hardy's Theorem (1993)}: Quantum nonlocality without inequalities
	\end{enumerate}
	
	\subsection{The T0 Challenge}
	
	The T0-energy field formulation makes apparently contradictory claims:
	
	\subsubsection*{T0 Claims vs No-Go Theorems}
\textbf{T0 Claims}:
		\begin{itemize}
			\item Local deterministic dynamics: $\partial^2 \Efield = 0$
			\item Realistic energy fields: $\Efield(x,t)$ exist independently
			\item Perfect QM reproduction: Identical predictions for all experiments
		\end{itemize}
		
		\textbf{No-Go Theorems}: Such a theory is impossible!
		
		\textbf{Question}: How does T0 circumvent these fundamental limitations?

	
	This document provides a comprehensive analysis of T0's strategy for addressing no-go theorems and evaluates its theoretical viability.
	
	\section{Bell's Theorem: Mathematical Foundation}
	
	\subsection{CHSH Inequality}
	
	The Clauser-Horne-Shimony-Holt (CHSH) form of Bell's inequality provides the most general test:
	
	\begin{equation}
		S = E(a,b) - E(a,b') + E(a',b) + E(a',b') \leq 2
		\label{NoGoEn_ch_tex:L-NoGoEn-1033}
	\end{equation}
	
	where $E(a,b)$ represents the correlation between measurements in directions $a$ and $b$.
	
	\subsection{Bell's Theorem Assumptions}
	
	Bell's proof relies on three key assumptions:
	
	\begin{enumerate}
		\item \textbf{Locality}: No superluminal influences
		\item \textbf{Realism}: Properties exist before measurement
		\item \textbf{Measurement freedom}: Free choice of measurement settings
	\end{enumerate}
	
	\textbf{Bell's conclusion}: Any theory satisfying all three assumptions must satisfy $|S| \leq 2$.
	
	\subsection{Quantum Mechanical Violation}
	
	For the Bell state $|\Psi^-\rangle = \frac{1}{\sqrt{2}}(|\uparrow\downarrow\rangle - |\downarrow\uparrow\rangle)$:
	
	\begin{equation}
		E_{QM}(a,b) = -\cos(\theta_{ab})
	\end{equation}
	
	where $\theta_{ab}$ is the angle between measurement directions.
	
	\textbf{Optimal measurement angles}: $a = 0°$, $a' = 45°$, $b = 22.5°$, $b' = 67.5°$
	
	\begin{align}
		E(a,b) &= -\cos(22.5°) = -0.9239 \\
		E(a,b') &= -\cos(67.5°) = -0.3827 \\
		E(a',b) &= -\cos(22.5°) = -0.9239 \\
		E(a',b') &= -\cos(22.5°) = -0.9239
	\end{align}
	
	\begin{equation}
		S_{QM} = -0.9239 - (-0.3827) + (-0.9239) + (-0.9239) = -2.389
	\end{equation}
	
	\textbf{Bell violation}: $|S_{QM}| = 2.389 > 2$
	
	\section{T0 Response to Bell's Theorem}
	
	\subsection{T0 Bell State Representation}
	
	In T0 formulation, the Bell state becomes:
	
	\begin{equation}
		\text{Standard: } |\Psi^-\rangle = \frac{1}{\sqrt{2}}(|\uparrow\downarrow\rangle - |\downarrow\uparrow\rangle)
	\end{equation}
	
	\begin{equation}
		\text{T0: } \{\Efield_{\uparrow\downarrow} = 0.5, \Efield_{\downarrow\uparrow} = -0.5, \Efield_{\uparrow\uparrow} = 0, \Efield_{\downarrow\downarrow} = 0\}
	\end{equation}
	
	\subsection{T0 Correlation Formula}
	
	T0 correlations arise from energy field interactions:
	
	\begin{equation}
		E_{T0}(a,b) = \frac{\langle \Efield_1(a) \cdot \Efield_2(b) \rangle}{\langle |\Efield_1| \rangle \langle |\Efield_2| \rangle}
	\end{equation}
	
	With $\xipar$-parameter corrections:
	
	\begin{equation}
		E_{T0}(a,b) = E_{QM}(a,b) \times (1 + \xipar \cdot f_{corr}(a,b))
	\end{equation}
	
	where $\xipar = 1.33 \times 10^{-4}$ and $f_{corr}$ represents correlation structure.
	
	\subsection{T0 Extended Bell Inequality}
	
	The original T0 documents propose a modified Bell inequality:
	
	\begin{equation}
		|E(a,b) - E(a,c)| + |E(a',b) + E(a',c)| \leq 2 + \varepsilon_{T0}
	\end{equation}
	
	where the T0 correction term is:
	
	\begin{equation}
		\varepsilon_{T0} = \xipar \cdot \left|\frac{E_1 - E_2}{E_1 + E_2}\right| \cdot \frac{2G\langle E \rangle}{r_{12}}
	\end{equation}
	
	\textbf{Numerical evaluation}: For typical atomic systems with $r_{12} \sim 1$ m, $\langle E \rangle \sim 1$ eV:
	
	\begin{equation}
		\varepsilon_{T0} \approx 1.33 \times 10^{-4} \times 1 \times \frac{2 \times 6.7 \times 10^{-11} \times 1.6 \times 10^{-19}}{1} \approx 2.8 \times 10^{-34}
	\end{equation}
	
	\textbf{Problem}: This correction is experimentally unmeasurable!
	
	\textbf{Alternative interpretation}: Direct $\xipar$-corrections without gravitational suppression:
	
	\begin{equation}
		\varepsilon_{T0,direct} = \xipar = 1.33 \times 10^{-4}
	\end{equation}
	
	This would be measurable in precision Bell tests, predicting:
	
	\begin{equation}
		|S_{T0}| = 2.389 + 1.33 \times 10^{-4} = 2.389133
	\end{equation}
	
	\textbf{Testable T0 prediction}: Bell violation exceeds quantum mechanical limit by 133 ppm!
	
	\subsubsection*{Critical Question}
\section*{How can a local deterministic theory violate Bell's inequality?}
		
		This apparent contradiction requires careful analysis of Bell's theorem assumptions.

	
	\section{T0's Circumvention Strategy: Violation of Measurement Freedom}
	
	\subsection{The Key Insight: Spatially Extended Energy Fields}
	
	T0's solution relies on a subtle violation of Bell's measurement freedom assumption:
	
	\begin{equation}
		\Efield(x,t) = \Efield_{intrinsic}(x,t) + \Efield_{apparatus}(x,t)
	\end{equation}
	
	\textbf{Physical picture}:
	\begin{itemize}
		\item Energy fields $\Efield(x,t)$ are spatially extended
		\item Measurement apparatus at location A influences $\Efield(x,t)$ throughout space
		\item This creates correlations between apparatus settings and distant measurements
		\item The correlation is local in field dynamics but appears nonlocal in outcomes
	\end{itemize}
	
	\subsection{Mathematical Formulation}
	
	The T0 correlation includes apparatus-dependent terms:
	
	\begin{equation}
		E_{T0}(a,b) = E_{intrinsic}(a,b) + E_{apparatus}(a,b) + E_{cross}(a,b)
	\end{equation}
	
	where:
	\begin{itemize}
		\item $E_{intrinsic}$: Direct particle-particle correlation
		\item $E_{apparatus}$: Apparatus-particle correlations
		\item $E_{cross}$: Cross-correlations between apparatus and particles
	\end{itemize}
	
	\subsection{Superdeterminism}
	
	T0 implements a form of "superdeterminism":
	
	\subsubsection*{T0 Superdeterminism}
\textbf{Definition}: The choice of measurement settings $a$ and $b$ is not truly free but correlated with the quantum system's initial conditions through energy field dynamics.
		
		\textbf{Mechanism}: Spatially extended energy fields create subtle correlations between:
		\begin{itemize}
			\item Experimenter's "choice" of measurement direction
			\item Quantum system properties
			\item Measurement apparatus configuration
		\end{itemize}
		
		\textbf{Result}: Bell's measurement freedom assumption is violated

	
	\subsection{Experimental Consequences}
	
	T0 superdeterminism makes specific predictions:
	
	\begin{enumerate}
		\item \textbf{Measurement direction correlations}: Statistical bias in "random" measurement choices
		\item \textbf{Spatial energy structure}: Extended field patterns around measurement apparatus
		\item \textbf{$\xipar$-corrections}: $133$ ppm systematic deviations in correlations
		\item \textbf{Apparatus-dependent effects}: Measurement outcomes depend on apparatus history
	\end{enumerate}
	
	\section{Kochen-Specker Theorem}
	
	\subsection{The Contextuality Problem}
	
	The Kochen-Specker theorem states that quantum observables cannot have simultaneous definite values independent of measurement context.
	
	\textbf{Classic example}: Spin measurements in orthogonal directions
	\begin{align}
		\sigma_x^2 + \sigma_y^2 + \sigma_z^2 &= 3 \quad \text{(if all simultaneously definite)} \\
		\langle\sigma_x^2\rangle + \langle\sigma_y^2\rangle + \langle\sigma_z^2\rangle &= 3 \quad \text{(quantum prediction)} \\
		\text{But individual values are context-dependent!}
	\end{align}
	
	\subsection{T0 Response: Energy Field Contextuality}
	
	T0 addresses contextuality through measurement-induced field modifications:
	
	\begin{equation}
		\Efield_{measured,x} = \Efield_{intrinsic,x} + \Delta\Efield_x(\text{apparatus state})
	\end{equation}
	
	\textbf{Key insight}: 
	\begin{itemize}
		\item All energy field components $\Efield_x$, $\Efield_y$, $\Efield_z$ exist simultaneously
		\item Measurement in direction $x$ modifies $\Efield_y$ and $\Efield_z$ through apparatus interaction
		\item Context dependence arises from measurement-apparatus-field coupling
		\item "Hidden variables" are the complete energy field configuration $\{\Efield(x,t)\}$
	\end{itemize}
	
	\subsection{Mathematical Framework}
	
	\begin{equation}
		\frac{\partial \Efield_i}{\partial t} = f_i(\{\Efield_j\}, \{\text{apparatus}_k\})
	\end{equation}
	
	The evolution of each field component depends on:
	\begin{itemize}
		\item All other field components (quantum correlations)
		\item All measurement apparatus configurations (contextuality)
		\item Spatial field structure (nonlocal correlations)
	\end{itemize}
	
	\section{Other No-Go Theorems}
	
	\subsection{PBR Theorem (Pusey-Barrett-Rudolph)}
	
	\textbf{PBR claim}: Quantum states must be ontologically real, not merely epistemological.
	
	\textbf{T0 response}: Perfect compatibility
	\begin{itemize}
		\item Energy fields $\Efield(x,t)$ are ontologically real
		\item Quantum states correspond to energy field configurations
		\item No epistemological interpretation needed
	\end{itemize}
	
	\subsection{Hardy's Theorem}
	
	\textbf{Hardy's claim}: Quantum nonlocality can be demonstrated without inequalities.
	
	\textbf{T0 response}: Energy field correlations can reproduce Hardy's paradoxical situations through spatially extended field dynamics.
	
	\subsection{GHZ Theorem}
	
	\textbf{GHZ claim}: Three-particle correlations provide perfect demonstration of quantum nonlocality.
	
	\textbf{T0 response}: Three-particle energy field configurations with extended correlation structures.
	
	\section{Critical Evaluation}
	
	\subsection{Strengths of T0 Approach}
	
	\begin{enumerate}
		\item \textbf{Distinct predictions}: Makes **different** testable predictions from standard QM
		\item \textbf{Concrete mechanisms}: Provides specific energy field dynamics
		\item \textbf{Multiple testable signatures}: 
		\begin{itemize}
			\item Enhanced Bell violation (133 ppm excess)
			\item Perfect quantum algorithm repeatability  
			\item Spatial energy field structure
			\item Deterministic single-measurement predictions
		\end{itemize}
		\item \textbf{Theoretical elegance}: Unified framework for all quantum phenomena
		\item \textbf{Interpretational clarity}: Eliminates measurement problem and wave function collapse
		\item \textbf{Quantum computing advantages}: Deterministic algorithms with perfect predictability
		\item \textbf{Falsifiability}: Clear experimental criteria for disproof
	\end{enumerate}
	
	\subsection{Weaknesses and Criticisms}
	
	\begin{enumerate}
		\item \textbf{Superdeterminism controversy}: Most physicists consider it implausible
		\item \textbf{Measurement freedom violation}: Challenges fundamental experimental methodology
		\item \textbf{Mathematical development}: Energy field dynamics not fully developed
		\item \textbf{Relativistic compatibility}: Unclear how T0 integrates with special relativity
		\item \textbf{High precision requirements}: 133 ppm measurements technically challenging
		\item \textbf{Falsification risk}: **T0 predictions could be experimentally disproven**
		\item \textbf{Philosophical cost}: Eliminates measurement freedom and true randomness
	\end{enumerate}
	
	\subsection{Experimental Tests}
	
	\begin{table}[htbp]
		\centering
		\begin{tabular}{lcc}
			\toprule
			\textbf{Test} & \textbf{Standard QM} & \textbf{T0 Prediction} \\
			\midrule
			Bell correlations & Violate inequalities & Enhanced violation + $\xipar$ \\
			Extended Bell inequality & $|S| \leq 2$ & $|S| \leq 2 + 1.33 \times 10^{-4}$ \\
			Algorithm repeatability & Statistical variation & Perfect repeatability \\
			Single measurements & Probabilistic outcomes & Deterministic predictions \\
			Spatial structure & Point-like & Extended E(x,t) patterns \\
			Measurement randomness & True randomness & Subtle correlations \\
			Spatial field structure & Point-like & Extended patterns \\
			Apparatus dependence & Minimal & Systematic effects \\
			Superdeterminism & No evidence & Statistical biases \\
			\bottomrule
		\end{tabular}
		\caption{Experimental discrimination between standard QM and T0}
	\end{table}
	
	\section{Philosophical Implications}
	
	\subsection{The Price of Local Realism}
	
	T0's restoration of local realism comes at significant philosophical cost:
	
	\subsubsection*{Philosophical Trade-offs}
\textbf{Gained}:
		\begin{itemize}
			\item Local realism restored
			\item Deterministic physics
			\item Clear ontology (energy fields)
			\item No measurement problem
		\end{itemize}
		
		\textbf{Lost}:
		\begin{itemize}
			\item Traditional measurement interpretation
			\item Apparent fundamental randomness
			\item Simple non-contextual locality
			\item Some current experimental methodologies
		\end{itemize}

	
	\subsection{Superdeterminism and Free Will}
	
	T0's superdeterminism has significant implications:
	
	\begin{itemize}
		\item Experimental choices show subtle correlations with quantum systems
		\item Initial conditions of universe influence all measurement outcomes
		\item "Random" number generators exhibit systematic patterns
		\item Bell test "loopholes" become fundamental features rather than flaws
	\end{itemize}
	
	\section{Conclusion: A Viable Alternative?}
	
	\subsection{Summary of Analysis}
	
	This comprehensive analysis reveals that T0 theory offers a sophisticated strategy for circumventing no-go theorems while making **distinct, testable predictions** that differ from standard quantum mechanics:
	
	\begin{enumerate}
		\item \textbf{Bell's Theorem}: Circumvented through violation of measurement freedom via spatially extended energy field correlations, with **measurable enhanced Bell violation**
		\item \textbf{Kochen-Specker}: Addressed through measurement-apparatus-field coupling creating contextuality
		\item \textbf{Other theorems}: Generally compatible with T0's ontological energy field framework
		\item \textbf{Quantum Computing}: **Perfect algorithmic equivalence** with deterministic advantages (Deutsch, Bell states, Grover, Shor)
	\end{enumerate}
	
	\subsection{Theoretical Viability}
	
	\textbf{T0 is theoretically viable} as a **genuine alternative** (not reinterpretation) to standard quantum mechanics, offering:
	
	\textbf{Advantages}:
	\begin{itemize}
		\item **Distinct testable predictions** differing from QM
		\item **Deterministic quantum computing** with perfect algorithmic equivalence
		\item **Enhanced Bell violation** exceeding quantum limits by 133 ppm
		\item **Perfect repeatability** in quantum measurements
		\item **Spatial energy field structure** extending beyond point particles
		\item **Single-measurement predictability** for quantum algorithms
	\end{itemize}
	
	\textbf{Requirements}:
	\begin{itemize}
		\item Acceptance of superdeterminism
		\item Violation of measurement freedom
		\item Complex energy field dynamics
		\item **Falsifiability risk**: negative precision tests would disprove T0
	\end{itemize}
	
	\subsection{Experimental Resolution}
	
	The ultimate test of T0 vs standard QM lies in **precision experiments** with **clear discrimination criteria**:
	
	\begin{enumerate}
		\item \textbf{Enhanced Bell violation tests}: Search for |S| > 2.389 (QM limit)
		\begin{itemize}
			\item **Target precision**: 133 ppm or better
			\item **T0 prediction**: |S| = 2.389133 ± measurement error
			\item **Decisive test**: Any excess violation supports T0
		\end{itemize}
		
		\item \textbf{Quantum algorithm repeatability}: 1000× identical algorithm execution
		\begin{itemize}
			\item **QM expectation**: Statistical variation within error bars
			\item **T0 prediction**: Perfect repeatability (zero variance)
			\item **Algorithms**: Deutsch, Grover, Bell states, Shor
		\end{itemize}
		
		\item \textbf{Spatial energy field mapping}: Detect extended field structures
		\begin{itemize}
			\item **QM expectation**: Point-like measurement events
			\item **T0 prediction**: Spatially extended energy patterns E(x,t)
			\item **Technology**: High-resolution quantum interferometry
		\end{itemize}
		
		\item \textbf{Superdeterminism signatures}: Search for measurement choice correlations
		\begin{itemize}
			\item **QM expectation**: True randomness in measurement settings
			\item **T0 prediction**: Subtle statistical biases in "random" choices
			\item **Challenge**: Requires careful statistical analysis
		\end{itemize}
	\end{enumerate}
	
	\subsubsection*{Final Assessment}
\section*{T0 theory provides a mathematically consistent, experimentally testable alternative to standard quantum mechanics that circumvents no-go theorems through sophisticated superdeterministic mechanisms.}
		
		\textbf{Key insight}: T0 is not merely a reinterpretation but makes distinct, falsifiable predictions that can definitively distinguish it from standard QM through precision experiments.
		
		\textbf{Critical tests}: Enhanced Bell violation (133 ppm), perfect quantum algorithm repeatability, and spatial energy field mapping provide clear experimental discrimination criteria.
		
		\textbf{Verdict}: The ultimate decision between T0 and standard QM rests on experimental evidence, not theoretical preference.

	
	The T0 approach demonstrates that local realistic alternatives to quantum mechanics are theoretically possible and experimentally distinguishable. While requiring controversial superdeterministic assumptions, T0 offers concrete predictions that can definitively resolve the debate between deterministic and probabilistic quantum mechanics.
	
	


% Bibliography
\begin{thebibliography}{99}
	
	\bibitem{pdg2024}
	Particle Data Group Collaboration (2024). 
	\textit{Review of Particle Physics}. 
	Progress of Theoretical and Experimental Physics, 2024(8), 083C01.
	\url{https://pdg.lbl.gov}
	
	\bibitem{flag2024}
	Aoki, Y., et al. (FLAG Collaboration) (2024). 
	\textit{FLAG Review 2024 of Lattice Results for Low-Energy Constants}. 
	arXiv:2411.04268.
	\url{https://arxiv.org/abs/2411.04268}
	
	\bibitem{fermilab_muon_g2}
	Abi, B., et al. (Muon g-2 Collaboration) (2021). 
	\textit{Measurement of the Positive Muon Anomalous Magnetic Moment to 0.46 ppm}. 
	Physical Review Letters, 126, 141801.
	
	\bibitem{peskin_schroeder}
	Peskin, M. E., \& Schroeder, D. V. (1995). 
	\textit{An Introduction to Quantum Field Theory}. 
	Addison-Wesley.
	
	\bibitem{weinberg_qft}
	Weinberg, S. (1995). 
	\textit{The Quantum Theory of Fields, Vol. I--III}. 
	Cambridge University Press.
	
	\bibitem{griffiths_particle}
	Griffiths, D. (2008). 
	\textit{Introduction to Elementary Particles}. 
	Wiley-VCH.
	
	\bibitem{mandl_shaw}
	Mandl, F., \& Shaw, G. (2010). 
	\textit{Quantum Field Theory (2nd ed.)}. 
	Wiley.
	
	\bibitem{srednicki_qft}
	Srednicki, M. (2007). 
	\textit{Quantum Field Theory}. 
	Cambridge University Press.
	
	\bibitem{t0_fundamentals}
	Pascher, J. (2024). 
	\textit{T0-Theory: Foundations of Time-Mass Duality}. 
	Unpublished manuscript, HTL Leonding.
	
	\bibitem{t0_fine_structure}
	Pascher, J. (2024). 
	\textit{T0-Theory: The Fine Structure Constant}. 
	Unpublished manuscript, HTL Leonding.
	
	\bibitem{t0_neutrinos}
	Pascher, J. (2024). 
	\textit{T0-Theory: Neutrino Masses and PMNS Mixing}. 
	Unpublished manuscript, HTL Leonding.
	
	\bibitem{t0_github}
	Pascher, J. (2024--2025). 
	\textit{T0-Time-Mass-Duality Repository}. 
	GitHub.
	\url{https://github.com/jpascher/T0-Time-Mass-Duality}
	
	\bibitem{lattice_qcd_review}
	Kronfeld, A. S. (2012). 
	\textit{Twenty-first Century Lattice Gauge Theory: Results from the QCD Lagrangian}. 
	Annual Review of Nuclear and Particle Science, 62, 265--284.
	
	\bibitem{neutrino_mixing_pdg}
	Particle Data Group Collaboration (2024). 
	\textit{Neutrino Masses, Mixing, and Oscillations}. 
	PDG Review 2024.
	\url{https://pdg.lbl.gov/2024/reviews/rpp2024-rev-neutrino-mixing.pdf}
	
	\bibitem{higgs_discovery}
	ATLAS and CMS Collaborations (2012). 
	\textit{Observation of a New Particle in the Search for the Standard Model Higgs Boson}. 
	Physics Letters B, 716, 1--29.
	
	\bibitem{Brannen2005}
	C. P. Brannen, ``Estimate of neutrino masses from Koide's relation'', \textit{arXiv:hep-ph/0505028} (2005).
	\url{https://arxiv.org/abs/hep-ph/0505028}
	
	\bibitem{Brannen2006}
	C. P. Brannen, ``Koide Mass Formula for Neutrinos'', \textit{arXiv:0702.0052} (2006).
	\url{http://brannenworks.com/MASSES.pdf}
	
	\bibitem{PhaseVectors2025}
	Anonymous, ``The Koide Relation and Lepton Mass Hierarchy from Phase Vectors'', \textit{rXiv:2507.0040} (2025).
	\url{https://rxiv.org/pdf/2507.0040v1.pdf}
	
	\bibitem{PDG2025}
	Particle Data Group, ``Review of Particle Physics'', \textit{Phys. Rev. D} \textbf{112} (2025) 030001.
	\url{https://pdg.lbl.gov/2025/}
	
	\bibitem{terrell2024}
	Terrell et al. (2024). 
	\textit{Single-Clock Metrology in Nature}. 
	Nature Physics.
	
	\bibitem{hossenfelder2024}
	Hossenfelder, S. (2024). 
	\textit{Single Clock Video Explanation}. 
	YouTube.
	
	\bibitem{hundert1931}
	Hundert (1931). 
	\textit{Reference Work}. 
	Publisher.
	
	\bibitem{terrell2025}
	Terrell et al. (2025). 
	\textit{Advanced Clock Synchronization Methods}. 
	Physical Review Letters.
	
	\bibitem{pascher_t0_2025}
	Pascher, J. (2025). 
	\textit{T0-Theory: Complete Framework and Applications}. 
	Unpublished manuscript, HTL Leonding.
	
	\bibitem{t0qm}
	Pascher, J. (2024). 
	\textit{T0-Theory: Quantum Mechanics Formulation}. 
	Unpublished manuscript, HTL Leonding.
	
	\bibitem{t0anomale}
	Pascher, J. (2024). 
	\textit{T0-Theory: Anomalous Magnetic Moments}. 
	Unpublished manuscript, HTL Leonding.
	
	\bibitem{muong2complete}
	Abi, B., et al. (Muon g-2 Collaboration) (2023). 
	\textit{Complete Measurement of the Positive Muon Anomalous Magnetic Moment}. 
	Physical Review Letters, 131, 161802.
	
	\bibitem{penrose2004}
	Penrose, R. (2004). 
	\textit{The Road to Reality: A Complete Guide to the Laws of the Universe}. 
	Jonathan Cape.
	
	\bibitem{planck1900}
	Planck, M. (1900). 
	\textit{On the Theory of the Energy Distribution Law of the Normal Spectrum}. 
	Verhandlungen der Deutschen Physikalischen Gesellschaft, 2, 237.
	
	\bibitem{T0Theory}
	Pascher, J. (2024). 
	\textit{T0-Theory: Fundamental Principles}. 
	Unpublished manuscript, HTL Leonding.
	
	% Additional bibliography entries for all undefined citations
	\bibitem{6g_roadmap}
	6G Research Consortium (2024).
	\textit{6G Technology Roadmap}.
	Technical Report.
	
	\bibitem{Born2013}
	Born, M. (2013).
	\textit{Einstein's Theory of Relativity}.
	Dover Publications.
	
	\bibitem{Casimir1948}
	Casimir, H. B. G. (1948).
	\textit{On the attraction between two perfectly conducting plates}.
	Proc. Kon. Ned. Akad. Wetensch. B51, 793--795.
	
	\bibitem{Einstein1905}
	Einstein, A. (1905).
	\textit{On the Electrodynamics of Moving Bodies}.
	Annalen der Physik, 17, 891--921.
	
	\bibitem{Feynman2006}
	Feynman, R. P. (2006).
	\textit{QED: The Strange Theory of Light and Matter}.
	Princeton University Press.
	
	\bibitem{Griffiths2017}
	Griffiths, D. J. (2017).
	\textit{Introduction to Electrodynamics (4th ed.)}.
	Cambridge University Press.
	
	\bibitem{Jackson1999}
	Jackson, J. D. (1999).
	\textit{Classical Electrodynamics (3rd ed.)}.
	Wiley.
	
	\bibitem{Mohr2016}
	Mohr, P. J., et al. (2016).
	\textit{CODATA Recommended Values of the Fundamental Physical Constants: 2014}.
	Rev. Mod. Phys. 88, 035009.
	
	\bibitem{Parker2018}
	Parker, R. H., et al. (2018).
	\textit{Measurement of the fine-structure constant as a test of the Standard Model}.
	Science, 360, 191--195.
	
	\bibitem{Planck1900}
	Planck, M. (1900).
	\textit{On the Theory of the Energy Distribution Law of the Normal Spectrum}.
	Verhandlungen der Deutschen Physikalischen Gesellschaft, 2, 237.
	
	\bibitem{Planck2018}
	Planck Collaboration (2018).
	\textit{Planck 2018 results. VI. Cosmological parameters}.
	Astronomy \& Astrophysics, 641, A6.
	
	\bibitem{QFT_T0}
	Pascher, J. (2024).
	\textit{T0-Theory and QFT Connections}.
	Unpublished manuscript, HTL Leonding.
	
	\bibitem{Sommerfeld1916}
	Sommerfeld, A. (1916).
	\textit{On the Quantum Theory of Spectral Lines}.
	Annalen der Physik, 51, 1--94.
	
	\bibitem{T0_Feinstruktur}
	Pascher, J. (2024).
	\textit{T0-Theory: Fine Structure Analysis}.
	Unpublished manuscript, HTL Leonding.
	
	\bibitem{T0_SI}
	Pascher, J. (2024).
	\textit{T0-Theory and SI Units}.
	Unpublished manuscript, HTL Leonding.
	
	\bibitem{T0_fine_structure}
	Pascher, J. (2024).
	\textit{T0-Theory: The Fine Structure Constant}.
	Unpublished manuscript, HTL Leonding.
	
	\bibitem{T0_g2_erweiterung}
	Pascher, J. (2024).
	\textit{T0-Theory: g-2 Extensions}.
	Unpublished manuscript, HTL Leonding.
	
	\bibitem{T0_gravitational_constant}
	Pascher, J. (2024).
	\textit{T0-Theory: Gravitational Constant Derivation}.
	Unpublished manuscript, HTL Leonding.
	
	\bibitem{T0_netze_en}
	Pascher, J. (2024).
	\textit{T0-Theory: Network Structures}.
	Unpublished manuscript, HTL Leonding.
	
	\bibitem{T0_tm_erweiterung}
	Pascher, J. (2024).
	\textit{T0-Theory: Time-Mass Extensions}.
	Unpublished manuscript, HTL Leonding.
	
	\bibitem{Uzan2003}
	Uzan, J.-P. (2003).
	\textit{The fundamental constants and their variation}.
	Rev. Mod. Phys. 75, 403--455.
	
	\bibitem{Weinberg1995}
	Weinberg, S. (1995).
	\textit{The Quantum Theory of Fields, Vol. I}.
	Cambridge University Press.
	
	\bibitem{albrecht1999}
	Albrecht, A. \& Magueijo, J. (1999).
	\textit{A time varying speed of light as a solution to cosmological puzzles}.
	Phys. Rev. D 59, 043516.
	
	\bibitem{alice2023}
	ALICE Collaboration (2023).
	\textit{Recent results from ALICE}.
	CERN-EP-2023-XXX.
	
	\bibitem{analog_optical}
	Smith, J. et al. (2024).
	\textit{Analog optical computing systems}.
	Nature Photonics.
	
	\bibitem{ashtekar2004}
	Ashtekar, A. \& Lewandowski, J. (2004).
	\textit{Background independent quantum gravity}.
	Class. Quantum Grav. 21, R53.
	
	\bibitem{atlas2023}
	ATLAS Collaboration (2023).
	\textit{ATLAS physics results}.
	CERN-PH-EP-2023-XXX.
	
	\bibitem{atlas2023higgs}
	ATLAS Collaboration (2023).
	\textit{Higgs boson measurements}.
	Phys. Rev. Lett.
	
	\bibitem{barbour1999}
	Barbour, J. (1999).
	\textit{The End of Time}.
	Oxford University Press.
	
	\bibitem{barrow1999}
	Barrow, J. D. (1999).
	\textit{Cosmologies with varying light speed}.
	Phys. Rev. D 59, 043515.
	
	\bibitem{becker2007}
	Becker, K. et al. (2007).
	\textit{String Theory and M-Theory}.
	Cambridge University Press.
	
	\bibitem{bell_muon}
	Bennett, G. W., et al. (Muon g-2 Collaboration) (2006).
	\textit{Final report of the E821 muon anomalous magnetic moment measurement}.
	Phys. Rev. D 73, 072003.
	
	\bibitem{bondi1948}
	Bondi, H. \& Gold, T. (1948).
	\textit{The steady-state theory of the expanding universe}.
	Mon. Not. R. Astron. Soc. 108, 252--270.
	
	\bibitem{brewer2019}
	Brewer, S. M. et al. (2019).
	\textit{Al+ Quantum-Logic Clock with Systematic Uncertainty below $10^{-18}$}.
	Phys. Rev. Lett. 123, 033201.
	
	\bibitem{cms2023top}
	CMS Collaboration (2023).
	\textit{Top quark measurements at CMS}.
	JHEP 2023.
	
	\bibitem{cms2024}
	CMS Collaboration (2024).
	\textit{CMS physics results 2024}.
	CERN-PH-EP-2024-XXX.
	
	\bibitem{codata2019}
	Tiesinga, E. et al. (2019).
	\textit{The 2018 CODATA Recommended Values}.
	J. Phys. Chem. Ref. Data.
	
	\bibitem{desi2025}
	DESI Collaboration (2025).
	\textit{DESI 2025 Cosmology Results}.
	arXiv preprint.
	
	\bibitem{differential_optical}
	Wang, X. et al. (2024).
	\textit{Differential optical computing}.
	Optica.
	
	\bibitem{dingle1972}
	Dingle, H. (1972).
	\textit{Science at the Crossroads}.
	Martin Brian \& O'Keeffe.
	
	\bibitem{divalentino2021}
	Di Valentino, E. et al. (2021).
	\textit{In the realm of the Hubble tension}.
	Class. Quantum Grav. 38, 153001.
	
	\bibitem{elnaschie2004}
	El Naschie, M. S. (2004).
	\textit{A review of E infinity theory}.
	Chaos, Solitons \& Fractals, 19, 209--236.
	
	\bibitem{fabrication_heterogeneous}
	Chen, Y. et al. (2024).
	\textit{Heterogeneous photonic integration}.
	Nature Electronics.
	
	\bibitem{fermilab2023}
	Fermilab (2023).
	\textit{Muon g-2 results}.
	Phys. Rev. Lett.
	
	\bibitem{flexible_wafer}
	Kim, S. et al. (2024).
	\textit{Flexible wafer-scale photonics}.
	Science Advances.
	
	\bibitem{francesco1997}
	Di Francesco, P. et al. (1997).
	\textit{Conformal Field Theory}.
	Springer.
	
	\bibitem{hartree1957}
	Hartree, D. R. (1957).
	\textit{The Calculation of Atomic Structures}.
	Wiley.
	
	\bibitem{hhi_6g}
	Fraunhofer HHI (2024).
	\textit{6G Photonic Integration}.
	Technical Report.
	
	\bibitem{hossenfelder2025}
	Hossenfelder, S. (2025).
	\textit{Science without the gobbledygook}.
	YouTube/Blog.
	
	\bibitem{hossenfelder_single_clock_video}
	Hossenfelder, S. (2024).
	\textit{The Single Clock Problem}.
	YouTube.
	
	\bibitem{hoyle1948}
	Hoyle, F. (1948).
	\textit{A new model for the expanding universe}.
	Mon. Not. R. Astron. Soc. 108, 372--382.
	
	\bibitem{integration_microelectronic}
	Liu, A. et al. (2024).
	\textit{Microelectronic photonic integration}.
	IEEE Journal.
	
	\bibitem{jacobson1995}
	Jacobson, T. (1995).
	\textit{Thermodynamics of spacetime}.
	Phys. Rev. Lett. 75, 1260.
	
	\bibitem{kasevich2023}
	Kasevich, M. et al. (2023).
	\textit{Atom interferometry tests}.
	Nature Physics.
	
	\bibitem{lerner2014}
	Lerner, E. J. (2014).
	\textit{An open letter on cosmology}.
	New Scientist.
	
	\bibitem{lisa2017}
	LISA Consortium (2017).
	\textit{Laser Interferometer Space Antenna}.
	ESA Technical Report.
	
	\bibitem{lithium_tantalate}
	Zhang, M. et al. (2024).
	\textit{Thin-film lithium tantalate photonics}.
	Nature Photonics.
	
	\bibitem{lopez2010}
	Lopez-Corredoira, M. (2010).
	\textit{Tests and problems of the standard model in cosmology}.
	Int. J. Mod. Phys. D.
	
	\bibitem{ludlow2015}
	Ludlow, A. D. et al. (2015).
	\textit{Optical atomic clocks}.
	Rev. Mod. Phys. 87, 637.
	
	\bibitem{mach1883}
	Mach, E. (1883).
	\textit{Die Mechanik in ihrer Entwickelung}.
	F.A. Brockhaus.
	
	\bibitem{maldacena1998}
	Maldacena, J. (1998).
	\textit{The large N limit of superconformal field theories}.
	Adv. Theor. Math. Phys. 2, 231--252.
	
	\bibitem{mueller2014}
	Müller, H. et al. (2014).
	\textit{Atom interferometry tests of the gravitational redshift}.
	Phys. Rev. Lett.
	
	\bibitem{mug2_final_2025}
	Muon g-2 Collaboration (2025).
	\textit{Final muon g-2 measurement}.
	Phys. Rev. Lett.
	
	\bibitem{muong2_2023}
	Muon g-2 Collaboration (2023).
	\textit{Updated muon g-2 results}.
	Phys. Rev. Lett.
	
	\bibitem{nathan2024}
	Nathan, A. et al. (2024).
	\textit{Quantum computing advances}.
	Nature.
	
	\bibitem{newell2018}
	Newell, D. B. et al. (2018).
	\textit{The CODATA 2017 values of h, e, k, and $N_A$}.
	Metrologia 55, L13.
	
	\bibitem{nottale1993}
	Nottale, L. (1993).
	\textit{Fractal Space-Time and Microphysics}.
	World Scientific.
	
	\bibitem{on_chip_lithium}
	Wang, C. et al. (2024).
	\textit{On-chip lithium niobate photonics}.
	Nature Communications.
	
	\bibitem{optical_advantages}
	Shastri, B. J. et al. (2024).
	\textit{Advantages of optical computing}.
	Nature Reviews Physics.
	
	\bibitem{pascher2025cmb}
	Pascher, J. (2025).
	\textit{T0-Theory: CMB Analysis}.
	Unpublished manuscript, HTL Leonding.
	
	\bibitem{pascher2025g2}
	Pascher, J. (2025).
	\textit{T0-Theory: g-2 Predictions}.
	Unpublished manuscript, HTL Leonding.
	
	\bibitem{pascher2025qm}
	Pascher, J. (2025).
	\textit{T0-Theory: Quantum Mechanics}.
	Unpublished manuscript, HTL Leonding.
	
	\bibitem{pascher2025si}
	Pascher, J. (2025).
	\textit{T0-Theory: SI Unit System}.
	Unpublished manuscript, HTL Leonding.
	
	\bibitem{pascher2025t0}
	Pascher, J. (2025).
	\textit{T0-Theory: Complete Framework}.
	Unpublished manuscript, HTL Leonding.
	
	\bibitem{pascher:fundamentals}
	Pascher, J. (2024).
	\textit{T0-Theory: Fundamentals}.
	Unpublished manuscript, HTL Leonding.
	
	\bibitem{pascher:g2_rev9}
	Pascher, J. (2024).
	\textit{T0-Theory: g-2 Revision 9}.
	Unpublished manuscript, HTL Leonding.
	
	\bibitem{pascher:geometric_formalism}
	Pascher, J. (2024).
	\textit{T0-Theory: Geometric Formalism}.
	Unpublished manuscript, HTL Leonding.
	
	\bibitem{pascher:ml_addendum}
	Pascher, J. (2024).
	\textit{T0-Theory: Machine Learning Addendum}.
	Unpublished manuscript, HTL Leonding.
	
	\bibitem{pascher:t0_foundations}
	Pascher, J. (2024).
	\textit{T0-Theory: Foundations}.
	Unpublished manuscript, HTL Leonding.
	
	\bibitem{pascher_derivation_beta_2025}
	Pascher, J. (2025).
	\textit{T0-Theory: Derivation of Beta}.
	Unpublished manuscript, HTL Leonding.
	
	\bibitem{pascher_higgs_connection_2025}
	Pascher, J. (2025).
	\textit{T0-Theory: Higgs Connection}.
	Unpublished manuscript, HTL Leonding.
	
	\bibitem{pascher_lagrangian_extended_2025}
	Pascher, J. (2025).
	\textit{T0-Theory: Extended Lagrangian}.
	Unpublished manuscript, HTL Leonding.
	
	\bibitem{pascher_mathematical_structure_2025}
	Pascher, J. (2025).
	\textit{T0-Theory: Mathematical Structure}.
	Unpublished manuscript, HTL Leonding.
	
	\bibitem{pascher_t0_cmb_2025}
	Pascher, J. (2025).
	\textit{T0-Theory: CMB Predictions}.
	Unpublished manuscript, HTL Leonding.
	
	\bibitem{pascher_t0_energie_2025}
	Pascher, J. (2025).
	\textit{T0-Theory: Energy}.
	Unpublished manuscript, HTL Leonding.
	
	\bibitem{pascher_t0_energy_2025}
	Pascher, J. (2025).
	\textit{T0-Theory: Energy Framework}.
	Unpublished manuscript, HTL Leonding.
	
	\bibitem{pascher_t0_theory_2025}
	Pascher, J. (2025).
	\textit{T0-Theory: Complete Theory}.
	Unpublished manuscript, HTL Leonding.
	
	\bibitem{penrose1959}
	Penrose, R. (1959).
	\textit{The apparent shape of a relativistically moving sphere}.
	Proc. Cambridge Phil. Soc. 55, 137--139.
	
	\bibitem{penrose1967}
	Penrose, R. (1967).
	\textit{Twistor algebra}.
	J. Math. Phys. 8, 345--366.
	
	\bibitem{peratt1992}
	Peratt, A. L. (1992).
	\textit{Physics of the Plasma Universe}.
	Springer-Verlag.
	
	\bibitem{peskin1995}
	Peskin, M. E. \& Schroeder, D. V. (1995).
	\textit{An Introduction to Quantum Field Theory}.
	Addison-Wesley.
	
	\bibitem{peskin_schroeder_1995}
	Peskin, M. E. \& Schroeder, D. V. (1995).
	\textit{An Introduction to Quantum Field Theory}.
	Addison-Wesley.
	
	\bibitem{phoquant}
	PhoQuant (2024).
	\textit{Photonic quantum computing}.
	Technical Report.
	
	\bibitem{photonics_ai}
	Wetzstein, G. et al. (2024).
	\textit{Photonics for AI}.
	Nature.
	
	\bibitem{planck1906}
	Planck, M. (1906).
	\textit{The Theory of Heat Radiation}.
	Johann Ambrosius Barth.
	
	\bibitem{planck2018}
	Planck Collaboration (2018).
	\textit{Planck 2018 results}.
	A\&A 641, A6.
	
	\bibitem{polchinski1998}
	Polchinski, J. (1998).
	\textit{String Theory}.
	Cambridge University Press.
	
	\bibitem{qant_nps}
	QANT (2024).
	\textit{Quantum photonics systems}.
	Technical Report.
	
	\bibitem{quantenjahr25}
	Quantenjahr (2025).
	\textit{International Year of Quantum}.
	UNESCO.
	
	\bibitem{recurrent_photonics}
	Tait, A. N. et al. (2024).
	\textit{Recurrent photonic neural networks}.
	Optica.
	
	\bibitem{rf_photonics}
	Capmany, J. \& Novak, D. (2024).
	\textit{Microwave photonics}.
	Nature Photonics.
	
	\bibitem{riess2019}
	Riess, A. G. et al. (2019).
	\textit{Large Magellanic Cloud Cepheid Standards}.
	ApJ 876, 85.
	
	\bibitem{riess2022}
	Riess, A. G. et al. (2022).
	\textit{A Comprehensive Measurement of H0}.
	ApJ 934, L7.
	
	\bibitem{rovelli2004}
	Rovelli, C. (2004).
	\textit{Quantum Gravity}.
	Cambridge University Press.
	
	\bibitem{sciama1953}
	Sciama, D. W. (1953).
	\textit{On the origin of inertia}.
	Mon. Not. R. Astron. Soc. 113, 34--42.
	
	\bibitem{sciencedaily2025}
	ScienceDaily (2025).
	\textit{Physics news}.
	Online.
	
	\bibitem{sm_g2_2025}
	Aoyama, T. et al. (2025).
	\textit{Standard Model prediction for g-2}.
	Phys. Rep.
	
	\bibitem{susskind1995}
	Susskind, L. (1995).
	\textit{The world as a hologram}.
	J. Math. Phys. 36, 6377--6396.
	
	\bibitem{t0_kosmologie}
	Pascher, J. (2024).
	\textit{T0-Theory: Cosmology}.
	Unpublished manuscript, HTL Leonding.
	
	\bibitem{terrell1959}
	Terrell, J. (1959).
	\textit{Invisibility of the Lorentz contraction}.
	Phys. Rev. 116, 1041--1045.
	
	\bibitem{terrell_single_clock_nature_2024}
	Terrell, J. et al. (2024).
	\textit{Single clock precision measurements}.
	Nature Physics.
	
	\bibitem{tfln_foundry}
	TFLN Foundry (2024).
	\textit{Thin-film lithium niobate foundry services}.
	Technical Specifications.
	
	\bibitem{thiemann2007}
	Thiemann, T. (2007).
	\textit{Modern Canonical Quantum General Relativity}.
	Cambridge University Press.
	
	\bibitem{thz_epfl}
	EPFL (2024).
	\textit{Terahertz photonics research}.
	Technical Report.
	
	\bibitem{unnikrishnan2004}
	Unnikrishnan, C. S. (2004).
	\textit{On Einstein's resolution of the twin clock paradox}.
	Current Science, 86, 704--709.
	
	\bibitem{verlinde2011}
	Verlinde, E. (2011).
	\textit{On the origin of gravity and the laws of Newton}.
	JHEP 2011, 29.
	
	\bibitem{video2025}
	Video (2025).
	\textit{Physics video explanation}.
	YouTube.
	
	\bibitem{weinberg1995}
	Weinberg, S. (1995).
	\textit{The Quantum Theory of Fields}.
	Cambridge University Press.
	
	\bibitem{weiskopf2000}
	Weiskopf, D. (2000).
	\textit{Visualization of special relativity}.
	PhD thesis, University of Tübingen.
	
	\bibitem{wheeler1990}
	Wheeler, J. A. (1990).
	\textit{A Journey into Gravity and Spacetime}.
	Scientific American Library.
	
	\bibitem{wiki_bell}
	Wikipedia (2024).
	\textit{Bell's theorem}.
	Online encyclopedia.
	
	\bibitem{zwicky1929}
	Zwicky, F. (1929).
	\textit{On the red shift of spectral lines through interstellar space}.
	Proc. Natl. Acad. Sci. 15, 773--779.

\end{thebibliography}


\end{document}
