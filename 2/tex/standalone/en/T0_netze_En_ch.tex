\documentclass[11pt,a4paper]{article}
\usepackage[a4paper,margin=2cm]{geometry}
\usepackage[utf8]{inputenc}
\usepackage[english]{babel}
\usepackage{lmodern}
\renewcommand{\familydefault}{\sfdefault}

\usepackage{amsmath,amssymb,amsthm}
\usepackage{graphicx}
\usepackage[unicode,pdfencoding=auto,hypertexnames=false]{hyperref}
\usepackage{booktabs}
\usepackage{longtable}
\usepackage{array}
\usepackage{siunitx}
\usepackage{fancyhdr}
\usepackage{float}
\usepackage{tikz}
% tcolorbox removed for standalone
% tcbset removed
\tikzset{
  t0blue/.style={draw=blue,fill=blue!10},
  t0red/.style={draw=red,fill=red!10},
  t0green/.style={draw=green!50!black,fill=green!10},
  t0orange/.style={draw=orange,fill=orange!10},
}
\usepackage{setspace}
\usepackage{enumitem}
\usepackage{adjustbox}
\usepackage{xcolor}

% Define colors for xcolor package
\definecolor{t0green}{RGB}{34,139,34}
\definecolor{t0blue}{RGB}{0,0,255}
\definecolor{t0red}{RGB}{255,0,0}
\definecolor{t0orange}{RGB}{255,165,0}

% Define custom column types for tables
\newcolumntype{L}[1]{>{\raggedright\arraybackslash}p{#1}}
\newcolumntype{C}[1]{>{\centering\arraybackslash}p{#1}}
\newcolumntype{R}[1]{>{\raggedleft\arraybackslash}p{#1}}

\setlength{\parindent}{0pt}
\setlength{\parskip}{6pt}

\hypersetup{
  colorlinks=true,
  linkcolor=blue,
  citecolor=blue,
  urlcolor=blue
}
\pagestyle{fancy}
\setlength{\headheight}{28pt}

\newcommand{\checkmarkx}{\checkmark}
\newcommand{\warningx}{\textbf{!}}

% Makros aus Einzel-Dokumenten (Fallback-Definitionen)
\newcommand{\mytimes}{\times}
\newcommand{\myapprox}{\approx}
\newcommand{\mysim}{\sim}
\newcommand{\myomega}{\omega}
\newcommand{\mypi}{\pi}
\newcommand{\myrightarrow}{\rightarrow}
\newcommand{\mypropto}{\propto}
\newcommand{\deltafield}{\delta\phi}
\newcommand{\xipar}{\xi}
\newcommand{\xiT}{\xi}
\newcommand{\lambdah}{\lambda_h}

% Additional macros used in chapter files
\newcommand{\Kfrak}{K_{\text{frak}}}  % Fractal correction factor
\newcommand{\Dfrak}{D_f}              % Fractal dimension
\newcommand{\betapar}{\beta}          % T0 beta parameter
\newcommand{\alphapar}{\alpha}        % T0 alpha parameter
\newcommand{\Efield}{E}               % Energy field
% Note: checkmarkxa/warningxa are variants used in auto-generated chapter files
\newcommand{\checkmarkxa}{\checkmark}
\newcommand{\warningxa}{\textbf{!}}

% Additional T0-specific macros
\newcommand{\xigeom}{\xi_{\text{geom}}}  % Geometric xi
\newcommand{\lP}{\ell_P}                  % Planck length
\newcommand{\rzero}{r_0}                  % Characteristic radius
\newcommand{\xirat}{\xi_{\text{rat}}}     % Xi ratio
\newcommand{\tzero}{t_0}                  % Characteristic time
\newcommand{\natunits}{\text{(nat. units)}}  % Natural units annotation
\newcommand{\myRightarrow}{\Rightarrow}   % Arrow variant
\newcommand{\Lag}{\mathcal{L}}            % Lagrangian

% Physics macros used in chapter files
\newcommand{\CQCD}{C_{\text{QCD}}}        % QCD correction
\newcommand{\EP}{E_P}                     % Planck energy
\newcommand{\Ee}{E_e}                     % Electron energy
\newcommand{\Emu}{E_\mu}                  % Muon energy
\newcommand{\Exi}{E_\xi}                  % Xi energy
\newcommand{\Ezero}{E_0}                  % Characteristic energy
\newcommand{\Hubble}{H}                   % Hubble constant
\newcommand{\Kspec}{K_{\text{spec}}}      % Spectral correction
\newcommand{\Lambdat}{\Lambda_t}          % Time-related cosmological constant
\newcommand{\Leff}{\mathcal{L}_{\text{eff}}}  % Effective Lagrangian
\newcommand{\Lorentz}{\mathcal{L}}        % Lorentz symbol
\newcommand{\Lxi}{L_\xi}                  % Xi length
\newcommand{\Tfield}{T}                   % Time field
\newcommand{\Weyl}{W}                     % Weyl tensor/symbol
\newcommand{\alphaEMSI}{\alpha_{\text{EM,SI}}}  % EM alpha in SI
\newcommand{\alphaEMnat}{\alpha_{\text{EM,nat}}}  % EM alpha in natural units
\newcommand{\alphaem}{\alpha_{\text{em}}} % Electromagnetic alpha
\newcommand{\betaTSI}{\beta_{T,\text{SI}}}  % Beta in SI
\newcommand{\betaTnat}{\beta_{T,\text{nat}}}  % Beta in natural units
\newcommand{\deltam}{\delta m}            % Mass difference
\newcommand{\phiT}{\phi_T}                % T-field phi
\newcommand{\tP}{t_P}                     % Planck time
\newcommand{\rhoCMB}{\rho_{\text{CMB}}}   % CMB density
\newcommand{\rhoCasimir}{\rho_{\text{Casimir}}}  % Casimir density

% Table formatting
\usepackage{multirow}

% Additional physics macros
\newcommand{\Riem}{\mathcal{R}}           % Riemann tensor
\newcommand{\ZPinch}{Z_{\text{pinch}}}    % Z-pinch
\newcommand{\SynchPower}{P_{\text{synch}}} % Synchrotron power
\newcommand{\Rzero}{R_0}                  % Characteristic radius
\newcommand{\alphafine}{\alpha}           % Fine structure constant
\newcommand{\Etau}{E_\tau}                % Tau energy
\newcommand{\deltaE}{\delta E}            % Energy deviation
\newcommand{\EPlanck}{E_P}                % Planck energy
\newcommand{\pichar}{\pi}                 % Pi character
\newcommand{\alphaWSI}{\alpha_{W,\text{SI}}}  % Wien alpha in SI
\newcommand{\alphaWnat}{\alpha_{W,\text{nat}}}  % Wien alpha in natural units

% Einfache abstract-Umgebung für Kapitel:
\newenvironment{abstract}{%
  \begin{center}\bfseries Abstract\end{center}\small
}{\par}


\title{T0 netze En}
\author{J. Pascher}
\date{\today}

\begin{document}
\maketitle

\section*{T0 Netze (T0 netze)}

	\begin{abstract}
		This analysis examines the network representation of the T0 model with a particular focus on the dimensional aspects and their impacts on factorization processes. The T0 model can be formulated as a multidimensional network, where nodes represent spacetime points with associated time and energy fields. A crucial insight is that different dimensionalities require different $\xi$-parameters, as the geometric scaling factor $G_d = 2^{d-1}/d$ varies with the dimension $d$. In the context of factorization, this dimensional dependence generates a hierarchy of optimal $\xi_{\text{res}}$-values that scale inversely proportional to the problem size. Neural network implementations offer a promising approach to modeling the T0 framework, with dimension-adaptive architectures providing the flexibility required for both the representation of physical space and the mapping of the number space. The fundamental difference between the 3+1-dimensional physical space and the potentially infinitely-dimensional number space requires a careful mathematical transformation, which is realized through spectral methods and dimension-specific network designs. This extension builds on the established principles of the T0 theory, as described in previous works on fractal corrections and time-mass duality, and integrates them seamlessly into a broader, dimension-spanning framework.
	\end{abstract}
	
	
	\section{Introduction: Network Interpretation of the T0 Model}
	\label{T0_netze:L-T0_tm-erweiterung-x6-0008}
	
	The T0 model, grounded in the universal geometric parameter $\xipar = \frac{4}{3} \mytimes 10^{-4}$, can effectively be reformulated as a multidimensional network structure. This approach provides a mathematical framework that naturally accounts for both the representation of physical space and the mapping of the number space underlying factorization applications. The network perspective enables the intrinsic dualities of the theory -- such as the time-mass or time-energy relation -- to be modeled as local properties of nodes and edges, allowing for scalable extensions to higher dimensions. In the following, we will delve in detail into the formal definition, the dimensional implications, and the practical applications to demonstrate how this interpretation enriches the T0 theory and extends its applicability in areas such as quantum field theory and cryptography.
	
	\subsection{Network Formalism in the T0 Framework}
	\label{T0_netze:L-T0_netze-0508}
	
	A T0 network can be mathematically defined as:
	
	\begin{equation}
		\mathcal{N} = (V, E, \{T(v), E(v)\}_{v \in V})
	\end{equation}
	
	Where:
	\begin{itemize}
		\item $V$ represents the set of vertices (nodes) in spacetime, encompassing not only spatial positions but also temporal components to reflect the 3+1-dimensionality of physical space;
		\item $E$ represents the set of edges (connections between nodes), modeling interactions and field propagations, including non-local effects through $\xi$-dependent scalings;
		\item $T(v)$ represents the time field value at node $v$, integrating the absolute time $t_0$ as a fundamental scale;
		\item $E(v)$ represents the energy field value at node $v$, linked to the mass duality.
	\end{itemize}
	
	The fundamental time-energy duality relation $T(v) \cdot E(v) = 1$ is maintained at each node, ensuring consistent preservation of invariance across the entire network. This definition is fully compatible with the Lagrangian extensions in the T0 theory, as described in \cite{T0_tm_erweiterung}, and allows for discrete discretization of continuous fields.
	
	\subsection{Dimensional Aspects of the Network Structure}
	\label{T0_netze:L-T0_netze-0509}
	
	The dimensionality of the network plays a decisive role in determining its properties and opens pathways to modeling phenomena beyond classical 3+1-dimensionality. The following box extends the basic properties with additional considerations on scalability and complexity:
	
	\subsubsection*{Dimensional Network Properties}
In a $d$-dimensional network:
		\begin{itemize}
			\item Each node has up to $2d$ direct connections, causing connectivity to grow exponentially with dimension and leading to increased computational complexity;
			\item The geometric factor scales as $G_d = \frac{2^{d-1}}{d}$, normalizing volume and surface measures in higher dimensions and directly linked to the $\xi$-scaling;
			\item Field propagation follows $d$-dimensional wave equations, which can be generalized to $\partial^2 \deltafield = 0$ in hyperbolic spaces;
			\item Boundary conditions require $d$-dimensional specification, which in practice is approximated by periodic or Dirichlet-like conditions to ensure stability.
		\end{itemize}

	
	These properties form the basis for dimension-adaptive adjustment, which is detailed in later sections.
	
	\section{Dimensionality and -Parameter Variations}
	\label{T0_netze:L-T0_netze-0510}
	
	\subsection{Geometric Factor Dependence on Dimension}
	\label{T0_netze:L-T0_netze-0511}
	
	One of the most significant discoveries in the T0 theory is the dimensional dependence of the geometric factor, which shapes the fundamental structure of the model across all scales:
	
	\begin{equation}
		G_d = \frac{2^{d-1}}{d}
	\end{equation}
	
	For our familiar 3-dimensional space, we obtain $G_3 = \frac{2^2}{3} = \frac{4}{3}$, which appears as a fundamental geometric constant in the T0 model and directly corresponds to the derivation of the fine-structure constant $\alpha$ in \cite{T0_Feinstruktur}. This formula enables a unified description of volume integrals in variable dimensions, which is particularly useful for cosmological extensions.
	
	\begin{table}[htbp]
		\centering
		\begin{tabular}{cccc}
			\toprule
			\textbf{Dimension ($d$)} & \textbf{Geometric Factor ($G_d$)} & \textbf{Ratio to $G_3$} & \textbf{Application Example} \\
			\midrule
			1 & 1/1 = 1 & 0.75 & Linear chain models in 1D dynamics \\
			2 & 2/2 = 1 & 0.75 & Surface-based Casimir effects \\
			3 & 4/3 = 1.333... & 1.00 & Standard physical space (T0 core) \\
			4 & 8/4 = 2 & 1.50 & Kaluza-Klein-like extensions \\
			5 & 16/5 = 3.2 & 2.40 & Fractal scalings in CMB \\
			6 & 32/6 = 5.333... & 4.00 & Hexagonal networks in quantum computing \\
			10 & 512/10 = 51.2 & 38.40 & High-dimensional information spaces \\
			\bottomrule
		\end{tabular}
		\caption{Geometric factors for various dimensionalities, extended with application examples}
		\label{T0_netze:L-T0_netze-0512}
	\end{table}
	
	\subsection{Dimension-Dependent -Parameters}
	\label{T0_netze:L-T0_netze-0513}
	
	A crucial insight is that the $\xipar$-parameter must be adjusted for different dimensionalities to maintain the consistency of duality relations:
	
	\begin{equation}
		\xipar_d = \frac{G_d}{G_3} \cdot \xipar_3 = \frac{d \cdot 2^{d-3}}{3} \cdot \frac{4}{3} \mytimes 10^{-4}
	\end{equation}
	
	This means that different dimensional contexts require different $\xipar$-values for consistent physical behavior, bridging to the fractal corrections in \cite{T0_g2_erweiterung}, where $D_f = 3 - \xipar$ serves as a sub-dimensional variant.
	
\section*{Revolutionary}
		It is a fundamental error to treat $\xipar$ as a single universal constant. Instead:
		
		\begin{itemize}
			\item $\xipar_{\text{geom}}$: The geometric parameter ($\frac{4}{3} \mytimes 10^{-4}$) in 3D space, derived from space geometry;
			\item $\xipar_{\text{res}}$: The resonance parameter ($\approx 0.1$) for factorization, modulating spectral resolutions;
			\item $\xipar_d$: Dimension-specific parameters scaling with $G_d$ and generating a hierarchy across dimensions.
		\end{itemize}
		
		Each parameter serves a specific mathematical purpose and scales differently with dimension, making the theory robust against dimensional variations.
% end box revolutionary
	
	\section{Factorization and Dimensional Effects}
	\label{T0_netze:L-T0_netze-0514}
	
	\subsection{Factorization Requires Different -Values}
	\label{T0_netze:L-T0_netze-0515}
	
	A profound insight from the T0 theory is that factorization processes require different $\xipar$-values because they operate in effectively different dimensions. This dependence arises from the necessity to model prime factor searches as spectral resonances in a dimension-dependent field:
	
	\begin{equation}
		\xipar_{\text{res}}(d) = \frac{\xipar_{\text{res}}(3)}{d-1} = \frac{0,1}{d-1}
	\end{equation}
	
	Where $d$ represents the effective dimensionality of the factorization problem and adjusts resonance frequencies to the number's complexity.
	
	\subsection{Effective Dimensionality of Factorization}
	\label{T0_netze:L-T0_netze-0516}
	
	The effective dimensionality of a factorization problem scales with the size of the number to be factored and reflects the increasing entropy of the prime factor distribution:
	
	\begin{equation}
		d_{\text{eff}}(n) \approx \log_2\left(\frac{n}{\xipar_{\text{res}}}\right)
	\end{equation}
	
	This leads to a profound insight: Larger numbers exist in higher effective dimensions, explaining why factorization becomes exponentially more difficult with growing numbers and why classical algorithms like Pollard's Rho or the General Number Field Sieve exhibit dimensional limits.
	
	\begin{table}[htbp]
		\centering
		\begin{tabular}{cccc}
			\toprule
			\textbf{Number Range} & \textbf{Effective Dimension} & \textbf{Optimal $\xipar_{\text{res}}$} & \textbf{Comparison to RSA Security} \\
			\midrule
			$10^2$ - $10^3$ & 3-4 & 0.05 - 0.1 & Weak (fast factorization) \\
			$10^4$ - $10^6$ & 5-7 & 0.02 - 0.05 & Medium (moderately difficult) \\
			$10^8$ - $10^{12}$ & 8-12 & 0.01 - 0.02 & Strong (RSA-2048 equivalent) \\
			$10^{15}$+ & 15+ & $<0.01$ & Extreme (quantum-resistant scaling) \\
			\bottomrule
		\end{tabular}
		\caption{Effective dimensions and optimal resonance parameters, extended with RSA comparisons}
		\label{T0_netze:L-T0_netze-0517}
	\end{table}
	
	\subsection{Mathematical Formulation of Dimensionality Effects}
	\label{T0_netze:L-T0_netze-0518}
	
	The optimal resonance parameter for factoring a number $n$ can be calculated as:
	
	\begin{equation}
		\xipar_{\text{res,opt}}(n) = \frac{0,1}{d_{\text{eff}}(n)-1} = \frac{0,1}{\log_2\left(\frac{n}{0,1}\right)-1}
	\end{equation}
	
	This relation explains why different $\xipar$-values are required for different factorization problems and provides a mathematical framework for determining the optimal parameter. It integrates seamlessly into the spectral methods of the T0 theory and enables numerical simulations that can be implemented in neural networks.
	
	\section{Number Space vs. Physical Space}
	\label{T0_netze:L-T0_netze-0519}
	
	\subsection{Fundamental Dimensional Differences}
	\label{T0_netze:L-T0_netze-0520}
	
	A central insight in the T0 theory is the recognition that number space and physical space exhibit fundamentally different dimensional structures, highlighting a fundamental duality between discrete mathematics and continuous physics:
	
\section*{Important}
		\begin{itemize}
			\item \textbf{Physical Space}: 3+1 dimensions (3 spatial + 1 temporal), fixed by observation and consistent with the $\xi$-derivation from 3D geometry;
			\item \textbf{Number Space}: Potentially infinite dimensions (each prime factor represents a dimension), modulated by the Riemann hypothesis and $\zeta$-functions;
			\item \textbf{Effective Dimension}: Determined by problem complexity, not fixed, and dynamically adjustable via $\xi_{\text{res}}$.
		\end{itemize}
% end box important
	
	\subsection{Mathematical Transformation Between Spaces}
	\label{T0_netze:L-T0_netze-0521}
	
	The transformation between number space and physical space requires a sophisticated mathematical mapping that establishes isomorphisms between discrete and continuous structures:
	
	\begin{equation}
		\mathcal{T}: \mathbb{Z}_n \to \mathbb{R}^d, \quad \mathcal{T}(n) = \{E_i(x,t)\}
	\end{equation}
	
	This transformation maps numbers from the integer space $\mathbb{Z}_n$ to field configurations in the $d$-dimensional real space $\mathbb{R}^d$ and accounts for $\xi$-dependent rescalings to preserve invariances.
	
	\subsection{Spectral Methods for Dimensional Mapping}
	\label{T0_netze:L-T0_netze-0522}
	
	Spectral methods offer an elegant approach to mapping between spaces by utilizing Fourier-like decompositions to connect frequency domains:
	
	\begin{equation}
		\Psi_n(\omega, \xipar_{\text{res}}) = \sum_i A_i \times \frac{1}{\sqrt{4\pi\xipar_{\text{res}}}} \times \exp\left(-\frac{(\omega-\omega_i)^2}{4\xipar_{\text{res}}}\right)
	\end{equation}
	
	Where:
	\begin{itemize}
		\item $\Psi_n$ represents the spectral representation of the number $n$, encoding prime factors as resonances;
		\item $\omega_i$ represents the frequency associated with the prime factor $p_i$, proportional to $\log(p_i)$;
		\item $A_i$ represents the amplitude coefficient, derived from multiplicity;
		\item $\xipar_{\text{res}}$ controls the spectral resolution and determines the sharpness of the peaks.
	\end{itemize}
	
	This formulation allows efficient numerics and is compatible with quantum algorithms like Shor's.
	
	\section{Neural Network Implementation of the T0 Model}
	\label{T0_netze:L-T0_netze-0523}
	
	\subsection{Optimal Network Architectures}
	\label{T0_netze:L-T0_netze-0524}
	
	Neural networks offer a promising approach to implementing the T0 model, with several architectures particularly suited to handling dimension-dependent scalings:
	
	\begin{table}[htbp]
		\centering
		\begin{tabular}{lp{8cm}}
			\toprule
			\textbf{Architecture} & \textbf{Advantages for T0 Implementation} \\
			\midrule
			Graph Neural Networks & Natural representation of spacetime network structure with nodes and edges, including $\xi$-weighted propagation \\
			Convolutional Networks & Efficient processing of regular grid patterns in various dimensions, ideal for fractal $D_f$ corrections \\
			Fourier Neural Operators & Handles spectral transformations required for number-field mapping, with fast convergence \\
			Recurrent Networks & Models temporal evolution of field patterns, adhering to $T \cdot E = 1$ duality over timesteps \\
			Transformers & Captures long-range correlations in field values, useful for infinite-dimensional projections \\
			\bottomrule
		\end{tabular}
		\caption{Neural network architectures for T0 implementation, extended with specific T0 advantages}
		\label{T0_netze:L-T0_netze-0525}
	\end{table}
	
	\subsection{Dimension-Adaptive Networks}
	\label{T0_netze:L-T0_netze-0526}
	
	A key innovation for T0 implementation is dimension-adaptive networks that dynamically respond to effective dimensionality:
	
\section*{Formula}
		Effective T0 networks should adapt their dimensionality based on:
		\begin{itemize}
			\item \textbf{Problem Domain}: Physical (3+1D) vs. number space (variable $D$), with automatic switching via layer dropout;
			\item \textbf{Problem Complexity}: Higher dimensions for larger factorization tasks, scaled logarithmically with $n$;
			\item \textbf{Resource Constraints}: Dimensional optimization for computational efficiency through tensor reduction;
			\item \textbf{Accuracy Requirements}: Higher dimensions for more precise results, validated by loss functions with $\xi$-penalty.
		\end{itemize}
% end box formula
	
	\subsection{Mathematical Formulation of Neural T0 Networks}
	\label{T0_netze:L-T0_netze-0527}
	
	For Graph Neural Networks, the T0 model can be implemented as:
	
	\begin{equation}
		h_v^{(l+1)} = \sigma\left(W^{(l)} \cdot h_v^{(l)} + \sum_{u \in \mathcal{N}(v)} \alpha_{vu} \cdot M^{(l)} \cdot h_u^{(l)}\right)
	\end{equation}
	
	Where:
	\begin{itemize}
		\item $h_v^{(l)}$ is the state vector at node $v$ in layer $l$, initialized with $T(v)$ and $E(v)$;
		\item $\mathcal{N}(v)$ is the neighborhood of node $v$, extended by $\xi$-weighted distances;
		\item $W^{(l)}$ and $M^{(l)}$ are learnable weight matrices incorporating $G_d$;
		\item $\alpha_{vu}$ are attention coefficients, computed via softmax over edges;
		\item $\sigma$ is a non-linear activation function, e.g., ReLU with duality constraint.
	\end{itemize}
	
	For spectral methods with Fourier Neural Operators:
	
	\begin{equation}
		(\mathcal{K}\phi)(x) = \int_{\Omega} \kappa(x,y) \phi(y) dy \approx \mathcal{F}^{-1}(R \cdot \mathcal{F}(\phi))
	\end{equation}
	
	Where $\mathcal{F}$ is the Fourier transform, $R$ is a learnable filter, and $\phi$ is the field configuration, with $\xi_{\text{res}}$ as bandwidth parameter.
	
	\section{Dimensional Hierarchy and Scale Relations}
	\label{T0_netze:L-T0_netze-0528}
	
	\subsection{Dimensional Scale Separation}
	\label{T0_netze:L-T0_netze-0529}
	
	The T0 model reveals a natural dimensional hierarchy connecting scales from Planck length to cosmological horizons:
	
	\begin{equation}
		\frac{\xipar_{\text{res}}(d)}{\xipar_{\text{geom}}(d)} = \frac{d-1}{d \cdot 2^{d-3}} \cdot \frac{3 \cdot 10^1}{4 \cdot 10^{-4}} \approx \frac{d-1}{d \cdot 2^{d-3}} \cdot 7,5 \cdot 10^4
	\end{equation}
	
	This relation shows how resonance and geometric parameters scale differently with dimension, generating a natural scale separation comparable to the hierarchy in fine-structure constant derivation.
	
	\subsection{Mathematical Relation to Number Space}
	\label{T0_netze:L-T0_netze-0530}
	
	The number space has a fundamentally different dimensional structure than physical space, shaped by infinite prime density:
	
	\begin{equation}
		\dim(\mathbb{Z}_n) = \infty \quad \text{(infinite for prime distribution)}
	\end{equation}
	
	This infinitely-dimensional structure must be projected onto finite-dimensional networks, with the effective dimension:
	
	\begin{equation}
		d_{\text{effective}} = \log_2\left(\frac{n}{\xipar_{\text{res}}}\right)
	\end{equation}
	
	This projection enables treating RSA keys as high-dimensional fields.
	
	\subsection{Information Mapping Between Dimensional Spaces}
	\label{T0_netze:L-T0_netze-0531}
	
	The information mapping between number space and physical space can be quantified by:
	
	\begin{equation}
		\mathcal{I}(n, d) = \int \Psi_n(\omega, \xipar_{\text{res}}) \cdot \Phi_d(\omega, \xipar_{\text{geom}}) \, d\omega
	\end{equation}
	
	Where $\Psi_n$ is the spectral representation of number $n$ and $\Phi_d$ is the $d$-dimensional field configuration, with a mutual information metric for evaluating mapping fidelity.
	
	\section{Hybrid Network Models for T0 Implementation}
	\label{T0_netze:L-T0_netze-0532}
	
	\subsection{Dual-Space Network Architecture}
	\label{T0_netze:L-T0_netze-0533}
	
	An optimal T0 implementation requires a hybrid network addressing both physical and number spaces, enabling bidirectional communication:
	
	\begin{equation}
		\mathcal{N}_{\text{hybrid}} = \mathcal{N}_{\text{phys}} \oplus \mathcal{N}_{\text{info}}
	\end{equation}
	
	Where $\mathcal{N}_{\text{phys}}$ is a 3+1D network for physical space and $\mathcal{N}_{\text{info}}$ is a network with variable dimension for information space, connected by a $\xi$-driven interface.
	
	\subsection{Implementation Strategy}
	\label{T0_netze:L-T0_netze-0534}
	
\section*{Experiment}
		\begin{enumerate}
			\item \textbf{Base Layer}: 3D Graph Neural Network with physical time as fourth dimension, initialized with T0 scales;
			\item \textbf{Field Layer}: Node features encoding $E_{\text{field}}$ and $T_{\text{field}}$ values, adhering to duality;
			\item \textbf{Spectral Layer}: Fourier transformations for mapping between spaces, with $\xi_{\text{res}}$ as filter parameter;
			\item \textbf{Dimension Adapter}: Dynamically adjusts network dimensionality based on problem complexity, via autoencoder-like modules;
			\item \textbf{Resonance Detector}: Implements variable $\xipar_{\text{res}}$ based on number size, with feedback loops for convergence.
		\end{enumerate}
% end box experiment
	
	\subsection{Training Approach for Neural Networks}
	\label{T0_netze:L-T0_netze-0535}
	
	Training a T0 neural network requires a multi-stage approach combining physical constraints with machine learning:
	
	\begin{enumerate}
		\item \textbf{Physical Constraint Learning}: Train the network to respect $T \cdot E = 1$ at each node, using Lagrangian-based loss terms;
		\item \textbf{Wave Equation Dynamics}: Train to solve $\partial^2 \deltafield = 0$ in various dimensions, with numerical solvers as ground truth;
		\item \textbf{Dimension Transfer}: Train the mapping between different dimensional spaces, evaluated by information metrics;
		\item \textbf{Factorization Tasks}: Fine-tuning on specific factorization problems with appropriate $\xipar_{\text{res}}$, including transfer learning from small to large $n$.
	\end{enumerate}
	
	\section{Practical Applications and Experimental Verification}
	\label{T0_netze:L-T0_netze-0536}
	
	\subsection{Factorization Experiments}
	\label{T0_netze:L-T0_netze-0537}
	
	The dimensional theory of T0 networks leads to testable predictions for factorization, which can be validated through simulations:
	
	\begin{table}[htbp]
		\centering
		\begin{tabular}{cccc}
			\toprule
			\textbf{Number Size} & \textbf{Predicted Optimal $\xipar_{\text{res}}$} & \textbf{Predicted Success Rate} & \textbf{Validation Metric} \\
			\midrule
			$10^3$ & 0.05 & 95\% & Hit rate in 100 simulations \\
			$10^6$ & 0.025 & 80\% & Convergence time in ms \\
			$10^9$ & 0.015 & 65\% & Error rate < 5\% \\
			$10^{12}$ & 0.01 & 50\% & Scalability on GPU \\
			\bottomrule
		\end{tabular}
		\caption{Factorization predictions from the dimensional T0 theory, extended with validation metrics}
		\label{T0_netze:L-T0_netze-0538}
	\end{table}
	
	\subsection{Verification Methods}
	\label{T0_netze:L-T0_netze-0539}
	
	The dimensional aspects of the T0 model can be verified through:
	
	\begin{itemize}
		\item \textbf{Dimensional Scaling Tests}: Check how performance scales with network dimension, through benchmarking on synthetic datasets;
		\item \textbf{$\xipar$-Optimization}: Confirm that optimal $\xipar_{\text{res}}$-values match theoretical predictions, via gradient descent logs;
		\item \textbf{Computational Complexity}: Measure how factorization difficulty scales with number size, compared to classical algorithms;
		\item \textbf{Spectral Analysis}: Validate spectral patterns for various number factorizations, using FFT libraries.
	\end{itemize}
	
	\subsection{Hardware Implementation Considerations}
	\label{T0_netze:L-T0_netze-0540}
	
	T0 networks can be implemented on various hardware platforms, each offering specific advantages for dimensional scaling:
	
	\begin{table}[htbp]
		\centering
		\begin{tabular}{lp{8cm}}
			\toprule
			\textbf{Hardware Platform} & \textbf{Dimensional Implementation Approach} \\
			\midrule
			GPU Arrays & Parallel processing of multiple dimensions with tensor cores, optimized for batch factorization \\
			Quantum Processors & Natural implementation of superposition across dimensions, for exponential speedups \\
			Neuromorphic Chips & Dimension-specific neural circuits with adaptive connectivity, energy-efficient for edge computing \\
			FPGA Systems & Reconfigurable architecture for variable dimensional processing, with real-time $\xi$-adjustment \\
			\bottomrule
		\end{tabular}
		\caption{Hardware implementation approaches, extended with platform-specific optimizations}
		\label{T0_netze:L-T0_netze-0541}
	\end{table}
	
	\section{Theoretical Implications and Future Directions}
	\label{T0_netze:L-T0_netze-0542}
	
	\subsection{Unified Mathematical Framework}
	\label{T0_netze:L-T0_netze-0543}
	
	The dimensional analysis of T0 networks reveals a unified mathematical framework uniting physics, mathematics, and informatics:
	
\section*{Revolutionary}
		\begin{equation}
			\boxed{\text{All Reality} = \text{Universal Field } \deltafield(x,t) \text{ dancing in } G_d\text{-characterized }d\text{-dimensional Spacetime}}
		\end{equation}
		
		With $G_d = 2^{d-1}/d$, providing the geometric foundation across all dimensions and ensuring universal invariance.
% end box revolutionary
	
	\subsection{Future Research Directions}
	\label{T0_netze:L-xi_parmater_partikel-0144}
	
	This analysis suggests several promising research directions to further develop the T0 theory:
	
	\begin{enumerate}
		\item \textbf{Dimension-Optimal Networks}: Develop neural architectures that automatically determine optimal dimensionality, through reinforcement learning;
		\item \textbf{Factorization Algorithms}: Create algorithms that adjust $\xipar_{\text{res}}$ based on number size, focusing on post-quantum secure variants;
		\item \textbf{Quantum T0 Networks}: Explore quantum implementations that naturally handle higher dimensions, integrated with NISQ devices;
		\item \textbf{Physical-Number Space Transformations}: Develop improved mappings between physical and number spaces, validated by experimental data from CMB;
		\item \textbf{Adaptive Dimensional Scaling}: Implement networks that dynamically scale dimensions based on problem complexity, with applications in AI-supported physics simulation.
	\end{enumerate}
	
	\subsection{Philosophical Implications}
	\label{T0_netze:L-T0_netze-0544}
	
	The dimensional analysis of T0 networks suggests profound philosophical implications that dissolve the boundaries between reality and abstraction:
	
	\begin{itemize}
		\item \textbf{Reality as Dimensional Projection}: Physical reality could be a 3+1D projection of higher-dimensional information spaces, akin to holographic principles;
		\item \textbf{Dimensionality as Complexity Measure}: The effective dimension of a system reflects its intrinsic complexity and offers a new paradigm for entropy;
		\item \textbf{Unified Geometric Foundation}: The factor $G_d = 2^{d-1}/d$ could represent a universal geometric principle across all dimensions, uniting mathematics and physics;
		\item \textbf{Number Space Connection}: Mathematical structures (like numbers) and physical structures could be fundamentally connected through dimensional mapping, with implications for the nature of causality.
	\end{itemize}
	
	\section{Conclusion: The Dimensional Nature of T0 Networks}
	\label{T0_netze:L-T0_Energie-0331}
	
	\subsection{Summary of Key Findings}
	\label{T0_netze:L-xi_parmater_partikel-0137}
	
	This analysis has revealed several profound insights that elevate the T0 theory to a new level:
	
	\begin{enumerate}
		\item Different $\xipar$-parameters are required for different dimensionalities, with $\xipar_d$ scaling with $G_d = 2^{d-1}/d$ and enabling universal geometry;
		\item Factorization problems require different $\xipar_{\text{res}}$-values as they operate in effectively different dimensions, quantifying complexity logarithmically;
		\item The effective dimensionality of a factorization problem scales logarithmically with number size, offering a new perspective on cryptography;
		\item Neural network implementations must adapt their dimensionality based on problem domain and complexity for scalable applications;
		\item Number space and physical space have fundamentally different dimensional structures requiring sophisticated mapping, but solvable through spectral methods.
	\end{enumerate}
	
	\subsection{The Power of Dimensional Understanding}
	\label{T0_netze:L-T0_netze-0545}
	
	Understanding the dimensional aspects of T0 networks provides powerful insights extending beyond theoretical physics:
	
\section*{Important}
		\begin{itemize}
			\item The challenge of factorization is fundamentally a dimensional problem solvable through $\xi$-adjustment;
			\item Large numbers exist in higher effective dimensions than small numbers, explaining algorithm scalability;
			\item Different $\xipar$-values represent geometric factors in various dimensions, forming a parameter hierarchy;
			\item Neural networks must adapt their dimensionality to the problem context for optimal performance;
			\item Physical 3+1D space is merely a specific case of the general $d$-dimensional T0 framework, open for future extensions.
		\end{itemize}
% end box important
	
	\subsection{Final Synthesis}
	\label{T0_netze:L-T0_netze-0546}
	
	The dimensional analysis of T0 networks reveals a profound unity between mathematics, physics, and computation, crowned by an elegant synthesis:
	
	\begin{equation}
		\boxed{\text{T0 Unification} = \text{Geometry} (G_d) + \text{Field Dynamics} (\partial^2\deltafield = 0) + \text{Dimensional Adaptation} (d_{\text{eff}})}
	\end{equation}
	
	This unified framework offers a powerful approach to understanding both physical reality and mathematical structures like factorization, all within a single elegant geometric framework characterized by the dimension-dependent factor $G_d = 2^{d-1}/d$. Future work will leverage this foundation to advance empirical validations and practical implementations.
	
	


% Bibliography
\begin{thebibliography}{99}
	
	\bibitem{pdg2024}
	Particle Data Group Collaboration (2024). 
	\textit{Review of Particle Physics}. 
	Progress of Theoretical and Experimental Physics, 2024(8), 083C01.
	\url{https://pdg.lbl.gov}
	
	\bibitem{flag2024}
	Aoki, Y., et al. (FLAG Collaboration) (2024). 
	\textit{FLAG Review 2024 of Lattice Results for Low-Energy Constants}. 
	arXiv:2411.04268.
	\url{https://arxiv.org/abs/2411.04268}
	
	\bibitem{fermilab_muon_g2}
	Abi, B., et al. (Muon g-2 Collaboration) (2021). 
	\textit{Measurement of the Positive Muon Anomalous Magnetic Moment to 0.46 ppm}. 
	Physical Review Letters, 126, 141801.
	
	\bibitem{peskin_schroeder}
	Peskin, M. E., \& Schroeder, D. V. (1995). 
	\textit{An Introduction to Quantum Field Theory}. 
	Addison-Wesley.
	
	\bibitem{weinberg_qft}
	Weinberg, S. (1995). 
	\textit{The Quantum Theory of Fields, Vol. I--III}. 
	Cambridge University Press.
	
	\bibitem{griffiths_particle}
	Griffiths, D. (2008). 
	\textit{Introduction to Elementary Particles}. 
	Wiley-VCH.
	
	\bibitem{mandl_shaw}
	Mandl, F., \& Shaw, G. (2010). 
	\textit{Quantum Field Theory (2nd ed.)}. 
	Wiley.
	
	\bibitem{srednicki_qft}
	Srednicki, M. (2007). 
	\textit{Quantum Field Theory}. 
	Cambridge University Press.
	
	\bibitem{t0_fundamentals}
	Pascher, J. (2024). 
	\textit{T0-Theory: Foundations of Time-Mass Duality}. 
	Unpublished manuscript, HTL Leonding.
	
	\bibitem{t0_fine_structure}
	Pascher, J. (2024). 
	\textit{T0-Theory: The Fine Structure Constant}. 
	Unpublished manuscript, HTL Leonding.
	
	\bibitem{t0_neutrinos}
	Pascher, J. (2024). 
	\textit{T0-Theory: Neutrino Masses and PMNS Mixing}. 
	Unpublished manuscript, HTL Leonding.
	
	\bibitem{t0_github}
	Pascher, J. (2024--2025). 
	\textit{T0-Time-Mass-Duality Repository}. 
	GitHub.
	\url{https://github.com/jpascher/T0-Time-Mass-Duality}
	
	\bibitem{lattice_qcd_review}
	Kronfeld, A. S. (2012). 
	\textit{Twenty-first Century Lattice Gauge Theory: Results from the QCD Lagrangian}. 
	Annual Review of Nuclear and Particle Science, 62, 265--284.
	
	\bibitem{neutrino_mixing_pdg}
	Particle Data Group Collaboration (2024). 
	\textit{Neutrino Masses, Mixing, and Oscillations}. 
	PDG Review 2024.
	\url{https://pdg.lbl.gov/2024/reviews/rpp2024-rev-neutrino-mixing.pdf}
	
	\bibitem{higgs_discovery}
	ATLAS and CMS Collaborations (2012). 
	\textit{Observation of a New Particle in the Search for the Standard Model Higgs Boson}. 
	Physics Letters B, 716, 1--29.
	
	\bibitem{Brannen2005}
	C. P. Brannen, ``Estimate of neutrino masses from Koide's relation'', \textit{arXiv:hep-ph/0505028} (2005).
	\url{https://arxiv.org/abs/hep-ph/0505028}
	
	\bibitem{Brannen2006}
	C. P. Brannen, ``Koide Mass Formula for Neutrinos'', \textit{arXiv:0702.0052} (2006).
	\url{http://brannenworks.com/MASSES.pdf}
	
	\bibitem{PhaseVectors2025}
	Anonymous, ``The Koide Relation and Lepton Mass Hierarchy from Phase Vectors'', \textit{rXiv:2507.0040} (2025).
	\url{https://rxiv.org/pdf/2507.0040v1.pdf}
	
	\bibitem{PDG2025}
	Particle Data Group, ``Review of Particle Physics'', \textit{Phys. Rev. D} \textbf{112} (2025) 030001.
	\url{https://pdg.lbl.gov/2025/}
	
	\bibitem{terrell2024}
	Terrell et al. (2024). 
	\textit{Single-Clock Metrology in Nature}. 
	Nature Physics.
	
	\bibitem{hossenfelder2024}
	Hossenfelder, S. (2024). 
	\textit{Single Clock Video Explanation}. 
	YouTube.
	
	\bibitem{hundert1931}
	Hundert (1931). 
	\textit{Reference Work}. 
	Publisher.
	
	\bibitem{terrell2025}
	Terrell et al. (2025). 
	\textit{Advanced Clock Synchronization Methods}. 
	Physical Review Letters.
	
	\bibitem{pascher_t0_2025}
	Pascher, J. (2025). 
	\textit{T0-Theory: Complete Framework and Applications}. 
	Unpublished manuscript, HTL Leonding.
	
	\bibitem{t0qm}
	Pascher, J. (2024). 
	\textit{T0-Theory: Quantum Mechanics Formulation}. 
	Unpublished manuscript, HTL Leonding.
	
	\bibitem{t0anomale}
	Pascher, J. (2024). 
	\textit{T0-Theory: Anomalous Magnetic Moments}. 
	Unpublished manuscript, HTL Leonding.
	
	\bibitem{muong2complete}
	Abi, B., et al. (Muon g-2 Collaboration) (2023). 
	\textit{Complete Measurement of the Positive Muon Anomalous Magnetic Moment}. 
	Physical Review Letters, 131, 161802.
	
	\bibitem{penrose2004}
	Penrose, R. (2004). 
	\textit{The Road to Reality: A Complete Guide to the Laws of the Universe}. 
	Jonathan Cape.
	
	\bibitem{planck1900}
	Planck, M. (1900). 
	\textit{On the Theory of the Energy Distribution Law of the Normal Spectrum}. 
	Verhandlungen der Deutschen Physikalischen Gesellschaft, 2, 237.
	
	\bibitem{T0Theory}
	Pascher, J. (2024). 
	\textit{T0-Theory: Fundamental Principles}. 
	Unpublished manuscript, HTL Leonding.
	
	% Additional bibliography entries for all undefined citations
	\bibitem{6g_roadmap}
	6G Research Consortium (2024).
	\textit{6G Technology Roadmap}.
	Technical Report.
	
	\bibitem{Born2013}
	Born, M. (2013).
	\textit{Einstein's Theory of Relativity}.
	Dover Publications.
	
	\bibitem{Casimir1948}
	Casimir, H. B. G. (1948).
	\textit{On the attraction between two perfectly conducting plates}.
	Proc. Kon. Ned. Akad. Wetensch. B51, 793--795.
	
	\bibitem{Einstein1905}
	Einstein, A. (1905).
	\textit{On the Electrodynamics of Moving Bodies}.
	Annalen der Physik, 17, 891--921.
	
	\bibitem{Feynman2006}
	Feynman, R. P. (2006).
	\textit{QED: The Strange Theory of Light and Matter}.
	Princeton University Press.
	
	\bibitem{Griffiths2017}
	Griffiths, D. J. (2017).
	\textit{Introduction to Electrodynamics (4th ed.)}.
	Cambridge University Press.
	
	\bibitem{Jackson1999}
	Jackson, J. D. (1999).
	\textit{Classical Electrodynamics (3rd ed.)}.
	Wiley.
	
	\bibitem{Mohr2016}
	Mohr, P. J., et al. (2016).
	\textit{CODATA Recommended Values of the Fundamental Physical Constants: 2014}.
	Rev. Mod. Phys. 88, 035009.
	
	\bibitem{Parker2018}
	Parker, R. H., et al. (2018).
	\textit{Measurement of the fine-structure constant as a test of the Standard Model}.
	Science, 360, 191--195.
	
	\bibitem{Planck1900}
	Planck, M. (1900).
	\textit{On the Theory of the Energy Distribution Law of the Normal Spectrum}.
	Verhandlungen der Deutschen Physikalischen Gesellschaft, 2, 237.
	
	\bibitem{Planck2018}
	Planck Collaboration (2018).
	\textit{Planck 2018 results. VI. Cosmological parameters}.
	Astronomy \& Astrophysics, 641, A6.
	
	\bibitem{QFT_T0}
	Pascher, J. (2024).
	\textit{T0-Theory and QFT Connections}.
	Unpublished manuscript, HTL Leonding.
	
	\bibitem{Sommerfeld1916}
	Sommerfeld, A. (1916).
	\textit{On the Quantum Theory of Spectral Lines}.
	Annalen der Physik, 51, 1--94.
	
	\bibitem{T0_Feinstruktur}
	Pascher, J. (2024).
	\textit{T0-Theory: Fine Structure Analysis}.
	Unpublished manuscript, HTL Leonding.
	
	\bibitem{T0_SI}
	Pascher, J. (2024).
	\textit{T0-Theory and SI Units}.
	Unpublished manuscript, HTL Leonding.
	
	\bibitem{T0_fine_structure}
	Pascher, J. (2024).
	\textit{T0-Theory: The Fine Structure Constant}.
	Unpublished manuscript, HTL Leonding.
	
	\bibitem{T0_g2_erweiterung}
	Pascher, J. (2024).
	\textit{T0-Theory: g-2 Extensions}.
	Unpublished manuscript, HTL Leonding.
	
	\bibitem{T0_gravitational_constant}
	Pascher, J. (2024).
	\textit{T0-Theory: Gravitational Constant Derivation}.
	Unpublished manuscript, HTL Leonding.
	
	\bibitem{T0_netze_en}
	Pascher, J. (2024).
	\textit{T0-Theory: Network Structures}.
	Unpublished manuscript, HTL Leonding.
	
	\bibitem{T0_tm_erweiterung}
	Pascher, J. (2024).
	\textit{T0-Theory: Time-Mass Extensions}.
	Unpublished manuscript, HTL Leonding.
	
	\bibitem{Uzan2003}
	Uzan, J.-P. (2003).
	\textit{The fundamental constants and their variation}.
	Rev. Mod. Phys. 75, 403--455.
	
	\bibitem{Weinberg1995}
	Weinberg, S. (1995).
	\textit{The Quantum Theory of Fields, Vol. I}.
	Cambridge University Press.
	
	\bibitem{albrecht1999}
	Albrecht, A. \& Magueijo, J. (1999).
	\textit{A time varying speed of light as a solution to cosmological puzzles}.
	Phys. Rev. D 59, 043516.
	
	\bibitem{alice2023}
	ALICE Collaboration (2023).
	\textit{Recent results from ALICE}.
	CERN-EP-2023-XXX.
	
	\bibitem{analog_optical}
	Smith, J. et al. (2024).
	\textit{Analog optical computing systems}.
	Nature Photonics.
	
	\bibitem{ashtekar2004}
	Ashtekar, A. \& Lewandowski, J. (2004).
	\textit{Background independent quantum gravity}.
	Class. Quantum Grav. 21, R53.
	
	\bibitem{atlas2023}
	ATLAS Collaboration (2023).
	\textit{ATLAS physics results}.
	CERN-PH-EP-2023-XXX.
	
	\bibitem{atlas2023higgs}
	ATLAS Collaboration (2023).
	\textit{Higgs boson measurements}.
	Phys. Rev. Lett.
	
	\bibitem{barbour1999}
	Barbour, J. (1999).
	\textit{The End of Time}.
	Oxford University Press.
	
	\bibitem{barrow1999}
	Barrow, J. D. (1999).
	\textit{Cosmologies with varying light speed}.
	Phys. Rev. D 59, 043515.
	
	\bibitem{becker2007}
	Becker, K. et al. (2007).
	\textit{String Theory and M-Theory}.
	Cambridge University Press.
	
	\bibitem{bell_muon}
	Bennett, G. W., et al. (Muon g-2 Collaboration) (2006).
	\textit{Final report of the E821 muon anomalous magnetic moment measurement}.
	Phys. Rev. D 73, 072003.
	
	\bibitem{bondi1948}
	Bondi, H. \& Gold, T. (1948).
	\textit{The steady-state theory of the expanding universe}.
	Mon. Not. R. Astron. Soc. 108, 252--270.
	
	\bibitem{brewer2019}
	Brewer, S. M. et al. (2019).
	\textit{Al+ Quantum-Logic Clock with Systematic Uncertainty below $10^{-18}$}.
	Phys. Rev. Lett. 123, 033201.
	
	\bibitem{cms2023top}
	CMS Collaboration (2023).
	\textit{Top quark measurements at CMS}.
	JHEP 2023.
	
	\bibitem{cms2024}
	CMS Collaboration (2024).
	\textit{CMS physics results 2024}.
	CERN-PH-EP-2024-XXX.
	
	\bibitem{codata2019}
	Tiesinga, E. et al. (2019).
	\textit{The 2018 CODATA Recommended Values}.
	J. Phys. Chem. Ref. Data.
	
	\bibitem{desi2025}
	DESI Collaboration (2025).
	\textit{DESI 2025 Cosmology Results}.
	arXiv preprint.
	
	\bibitem{differential_optical}
	Wang, X. et al. (2024).
	\textit{Differential optical computing}.
	Optica.
	
	\bibitem{dingle1972}
	Dingle, H. (1972).
	\textit{Science at the Crossroads}.
	Martin Brian \& O'Keeffe.
	
	\bibitem{divalentino2021}
	Di Valentino, E. et al. (2021).
	\textit{In the realm of the Hubble tension}.
	Class. Quantum Grav. 38, 153001.
	
	\bibitem{elnaschie2004}
	El Naschie, M. S. (2004).
	\textit{A review of E infinity theory}.
	Chaos, Solitons \& Fractals, 19, 209--236.
	
	\bibitem{fabrication_heterogeneous}
	Chen, Y. et al. (2024).
	\textit{Heterogeneous photonic integration}.
	Nature Electronics.
	
	\bibitem{fermilab2023}
	Fermilab (2023).
	\textit{Muon g-2 results}.
	Phys. Rev. Lett.
	
	\bibitem{flexible_wafer}
	Kim, S. et al. (2024).
	\textit{Flexible wafer-scale photonics}.
	Science Advances.
	
	\bibitem{francesco1997}
	Di Francesco, P. et al. (1997).
	\textit{Conformal Field Theory}.
	Springer.
	
	\bibitem{hartree1957}
	Hartree, D. R. (1957).
	\textit{The Calculation of Atomic Structures}.
	Wiley.
	
	\bibitem{hhi_6g}
	Fraunhofer HHI (2024).
	\textit{6G Photonic Integration}.
	Technical Report.
	
	\bibitem{hossenfelder2025}
	Hossenfelder, S. (2025).
	\textit{Science without the gobbledygook}.
	YouTube/Blog.
	
	\bibitem{hossenfelder_single_clock_video}
	Hossenfelder, S. (2024).
	\textit{The Single Clock Problem}.
	YouTube.
	
	\bibitem{hoyle1948}
	Hoyle, F. (1948).
	\textit{A new model for the expanding universe}.
	Mon. Not. R. Astron. Soc. 108, 372--382.
	
	\bibitem{integration_microelectronic}
	Liu, A. et al. (2024).
	\textit{Microelectronic photonic integration}.
	IEEE Journal.
	
	\bibitem{jacobson1995}
	Jacobson, T. (1995).
	\textit{Thermodynamics of spacetime}.
	Phys. Rev. Lett. 75, 1260.
	
	\bibitem{kasevich2023}
	Kasevich, M. et al. (2023).
	\textit{Atom interferometry tests}.
	Nature Physics.
	
	\bibitem{lerner2014}
	Lerner, E. J. (2014).
	\textit{An open letter on cosmology}.
	New Scientist.
	
	\bibitem{lisa2017}
	LISA Consortium (2017).
	\textit{Laser Interferometer Space Antenna}.
	ESA Technical Report.
	
	\bibitem{lithium_tantalate}
	Zhang, M. et al. (2024).
	\textit{Thin-film lithium tantalate photonics}.
	Nature Photonics.
	
	\bibitem{lopez2010}
	Lopez-Corredoira, M. (2010).
	\textit{Tests and problems of the standard model in cosmology}.
	Int. J. Mod. Phys. D.
	
	\bibitem{ludlow2015}
	Ludlow, A. D. et al. (2015).
	\textit{Optical atomic clocks}.
	Rev. Mod. Phys. 87, 637.
	
	\bibitem{mach1883}
	Mach, E. (1883).
	\textit{Die Mechanik in ihrer Entwickelung}.
	F.A. Brockhaus.
	
	\bibitem{maldacena1998}
	Maldacena, J. (1998).
	\textit{The large N limit of superconformal field theories}.
	Adv. Theor. Math. Phys. 2, 231--252.
	
	\bibitem{mueller2014}
	Müller, H. et al. (2014).
	\textit{Atom interferometry tests of the gravitational redshift}.
	Phys. Rev. Lett.
	
	\bibitem{mug2_final_2025}
	Muon g-2 Collaboration (2025).
	\textit{Final muon g-2 measurement}.
	Phys. Rev. Lett.
	
	\bibitem{muong2_2023}
	Muon g-2 Collaboration (2023).
	\textit{Updated muon g-2 results}.
	Phys. Rev. Lett.
	
	\bibitem{nathan2024}
	Nathan, A. et al. (2024).
	\textit{Quantum computing advances}.
	Nature.
	
	\bibitem{newell2018}
	Newell, D. B. et al. (2018).
	\textit{The CODATA 2017 values of h, e, k, and $N_A$}.
	Metrologia 55, L13.
	
	\bibitem{nottale1993}
	Nottale, L. (1993).
	\textit{Fractal Space-Time and Microphysics}.
	World Scientific.
	
	\bibitem{on_chip_lithium}
	Wang, C. et al. (2024).
	\textit{On-chip lithium niobate photonics}.
	Nature Communications.
	
	\bibitem{optical_advantages}
	Shastri, B. J. et al. (2024).
	\textit{Advantages of optical computing}.
	Nature Reviews Physics.
	
	\bibitem{pascher2025cmb}
	Pascher, J. (2025).
	\textit{T0-Theory: CMB Analysis}.
	Unpublished manuscript, HTL Leonding.
	
	\bibitem{pascher2025g2}
	Pascher, J. (2025).
	\textit{T0-Theory: g-2 Predictions}.
	Unpublished manuscript, HTL Leonding.
	
	\bibitem{pascher2025qm}
	Pascher, J. (2025).
	\textit{T0-Theory: Quantum Mechanics}.
	Unpublished manuscript, HTL Leonding.
	
	\bibitem{pascher2025si}
	Pascher, J. (2025).
	\textit{T0-Theory: SI Unit System}.
	Unpublished manuscript, HTL Leonding.
	
	\bibitem{pascher2025t0}
	Pascher, J. (2025).
	\textit{T0-Theory: Complete Framework}.
	Unpublished manuscript, HTL Leonding.
	
	\bibitem{pascher:fundamentals}
	Pascher, J. (2024).
	\textit{T0-Theory: Fundamentals}.
	Unpublished manuscript, HTL Leonding.
	
	\bibitem{pascher:g2_rev9}
	Pascher, J. (2024).
	\textit{T0-Theory: g-2 Revision 9}.
	Unpublished manuscript, HTL Leonding.
	
	\bibitem{pascher:geometric_formalism}
	Pascher, J. (2024).
	\textit{T0-Theory: Geometric Formalism}.
	Unpublished manuscript, HTL Leonding.
	
	\bibitem{pascher:ml_addendum}
	Pascher, J. (2024).
	\textit{T0-Theory: Machine Learning Addendum}.
	Unpublished manuscript, HTL Leonding.
	
	\bibitem{pascher:t0_foundations}
	Pascher, J. (2024).
	\textit{T0-Theory: Foundations}.
	Unpublished manuscript, HTL Leonding.
	
	\bibitem{pascher_derivation_beta_2025}
	Pascher, J. (2025).
	\textit{T0-Theory: Derivation of Beta}.
	Unpublished manuscript, HTL Leonding.
	
	\bibitem{pascher_higgs_connection_2025}
	Pascher, J. (2025).
	\textit{T0-Theory: Higgs Connection}.
	Unpublished manuscript, HTL Leonding.
	
	\bibitem{pascher_lagrangian_extended_2025}
	Pascher, J. (2025).
	\textit{T0-Theory: Extended Lagrangian}.
	Unpublished manuscript, HTL Leonding.
	
	\bibitem{pascher_mathematical_structure_2025}
	Pascher, J. (2025).
	\textit{T0-Theory: Mathematical Structure}.
	Unpublished manuscript, HTL Leonding.
	
	\bibitem{pascher_t0_cmb_2025}
	Pascher, J. (2025).
	\textit{T0-Theory: CMB Predictions}.
	Unpublished manuscript, HTL Leonding.
	
	\bibitem{pascher_t0_energie_2025}
	Pascher, J. (2025).
	\textit{T0-Theory: Energy}.
	Unpublished manuscript, HTL Leonding.
	
	\bibitem{pascher_t0_energy_2025}
	Pascher, J. (2025).
	\textit{T0-Theory: Energy Framework}.
	Unpublished manuscript, HTL Leonding.
	
	\bibitem{pascher_t0_theory_2025}
	Pascher, J. (2025).
	\textit{T0-Theory: Complete Theory}.
	Unpublished manuscript, HTL Leonding.
	
	\bibitem{penrose1959}
	Penrose, R. (1959).
	\textit{The apparent shape of a relativistically moving sphere}.
	Proc. Cambridge Phil. Soc. 55, 137--139.
	
	\bibitem{penrose1967}
	Penrose, R. (1967).
	\textit{Twistor algebra}.
	J. Math. Phys. 8, 345--366.
	
	\bibitem{peratt1992}
	Peratt, A. L. (1992).
	\textit{Physics of the Plasma Universe}.
	Springer-Verlag.
	
	\bibitem{peskin1995}
	Peskin, M. E. \& Schroeder, D. V. (1995).
	\textit{An Introduction to Quantum Field Theory}.
	Addison-Wesley.
	
	\bibitem{peskin_schroeder_1995}
	Peskin, M. E. \& Schroeder, D. V. (1995).
	\textit{An Introduction to Quantum Field Theory}.
	Addison-Wesley.
	
	\bibitem{phoquant}
	PhoQuant (2024).
	\textit{Photonic quantum computing}.
	Technical Report.
	
	\bibitem{photonics_ai}
	Wetzstein, G. et al. (2024).
	\textit{Photonics for AI}.
	Nature.
	
	\bibitem{planck1906}
	Planck, M. (1906).
	\textit{The Theory of Heat Radiation}.
	Johann Ambrosius Barth.
	
	\bibitem{planck2018}
	Planck Collaboration (2018).
	\textit{Planck 2018 results}.
	A\&A 641, A6.
	
	\bibitem{polchinski1998}
	Polchinski, J. (1998).
	\textit{String Theory}.
	Cambridge University Press.
	
	\bibitem{qant_nps}
	QANT (2024).
	\textit{Quantum photonics systems}.
	Technical Report.
	
	\bibitem{quantenjahr25}
	Quantenjahr (2025).
	\textit{International Year of Quantum}.
	UNESCO.
	
	\bibitem{recurrent_photonics}
	Tait, A. N. et al. (2024).
	\textit{Recurrent photonic neural networks}.
	Optica.
	
	\bibitem{rf_photonics}
	Capmany, J. \& Novak, D. (2024).
	\textit{Microwave photonics}.
	Nature Photonics.
	
	\bibitem{riess2019}
	Riess, A. G. et al. (2019).
	\textit{Large Magellanic Cloud Cepheid Standards}.
	ApJ 876, 85.
	
	\bibitem{riess2022}
	Riess, A. G. et al. (2022).
	\textit{A Comprehensive Measurement of H0}.
	ApJ 934, L7.
	
	\bibitem{rovelli2004}
	Rovelli, C. (2004).
	\textit{Quantum Gravity}.
	Cambridge University Press.
	
	\bibitem{sciama1953}
	Sciama, D. W. (1953).
	\textit{On the origin of inertia}.
	Mon. Not. R. Astron. Soc. 113, 34--42.
	
	\bibitem{sciencedaily2025}
	ScienceDaily (2025).
	\textit{Physics news}.
	Online.
	
	\bibitem{sm_g2_2025}
	Aoyama, T. et al. (2025).
	\textit{Standard Model prediction for g-2}.
	Phys. Rep.
	
	\bibitem{susskind1995}
	Susskind, L. (1995).
	\textit{The world as a hologram}.
	J. Math. Phys. 36, 6377--6396.
	
	\bibitem{t0_kosmologie}
	Pascher, J. (2024).
	\textit{T0-Theory: Cosmology}.
	Unpublished manuscript, HTL Leonding.
	
	\bibitem{terrell1959}
	Terrell, J. (1959).
	\textit{Invisibility of the Lorentz contraction}.
	Phys. Rev. 116, 1041--1045.
	
	\bibitem{terrell_single_clock_nature_2024}
	Terrell, J. et al. (2024).
	\textit{Single clock precision measurements}.
	Nature Physics.
	
	\bibitem{tfln_foundry}
	TFLN Foundry (2024).
	\textit{Thin-film lithium niobate foundry services}.
	Technical Specifications.
	
	\bibitem{thiemann2007}
	Thiemann, T. (2007).
	\textit{Modern Canonical Quantum General Relativity}.
	Cambridge University Press.
	
	\bibitem{thz_epfl}
	EPFL (2024).
	\textit{Terahertz photonics research}.
	Technical Report.
	
	\bibitem{unnikrishnan2004}
	Unnikrishnan, C. S. (2004).
	\textit{On Einstein's resolution of the twin clock paradox}.
	Current Science, 86, 704--709.
	
	\bibitem{verlinde2011}
	Verlinde, E. (2011).
	\textit{On the origin of gravity and the laws of Newton}.
	JHEP 2011, 29.
	
	\bibitem{video2025}
	Video (2025).
	\textit{Physics video explanation}.
	YouTube.
	
	\bibitem{weinberg1995}
	Weinberg, S. (1995).
	\textit{The Quantum Theory of Fields}.
	Cambridge University Press.
	
	\bibitem{weiskopf2000}
	Weiskopf, D. (2000).
	\textit{Visualization of special relativity}.
	PhD thesis, University of Tübingen.
	
	\bibitem{wheeler1990}
	Wheeler, J. A. (1990).
	\textit{A Journey into Gravity and Spacetime}.
	Scientific American Library.
	
	\bibitem{wiki_bell}
	Wikipedia (2024).
	\textit{Bell's theorem}.
	Online encyclopedia.
	
	\bibitem{zwicky1929}
	Zwicky, F. (1929).
	\textit{On the red shift of spectral lines through interstellar space}.
	Proc. Natl. Acad. Sci. 15, 773--779.

\end{thebibliography}


\end{document}
