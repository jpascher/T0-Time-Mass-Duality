\documentclass[11pt,a4paper]{article}
\usepackage[a4paper,margin=2cm]{geometry}
\usepackage[utf8]{inputenc}
\usepackage[english]{babel}
\usepackage{lmodern}
\renewcommand{\familydefault}{\sfdefault}

\usepackage{amsmath,amssymb,amsthm}
\usepackage{graphicx}
\usepackage[unicode,pdfencoding=auto,hypertexnames=false]{hyperref}
\usepackage{booktabs}
\usepackage{longtable}
\usepackage{array}
\usepackage{siunitx}
\usepackage{fancyhdr}
\usepackage{float}
\usepackage{tikz}
% tcolorbox removed for standalone
% tcbset removed
\tikzset{
  t0blue/.style={draw=blue,fill=blue!10},
  t0red/.style={draw=red,fill=red!10},
  t0green/.style={draw=green!50!black,fill=green!10},
  t0orange/.style={draw=orange,fill=orange!10},
}
\usepackage{setspace}
\usepackage{enumitem}
\usepackage{adjustbox}
\usepackage{xcolor}

% Define colors for xcolor package
\definecolor{t0green}{RGB}{34,139,34}
\definecolor{t0blue}{RGB}{0,0,255}
\definecolor{t0red}{RGB}{255,0,0}
\definecolor{t0orange}{RGB}{255,165,0}

% Define custom column types for tables
\newcolumntype{L}[1]{>{\raggedright\arraybackslash}p{#1}}
\newcolumntype{C}[1]{>{\centering\arraybackslash}p{#1}}
\newcolumntype{R}[1]{>{\raggedleft\arraybackslash}p{#1}}

\setlength{\parindent}{0pt}
\setlength{\parskip}{6pt}

\hypersetup{
  colorlinks=true,
  linkcolor=blue,
  citecolor=blue,
  urlcolor=blue
}
\pagestyle{fancy}
\setlength{\headheight}{28pt}

\newcommand{\checkmarkx}{\checkmark}
\newcommand{\warningx}{\textbf{!}}

% Makros aus Einzel-Dokumenten (Fallback-Definitionen)
\newcommand{\mytimes}{\times}
\newcommand{\myapprox}{\approx}
\newcommand{\mysim}{\sim}
\newcommand{\myomega}{\omega}
\newcommand{\mypi}{\pi}
\newcommand{\myrightarrow}{\rightarrow}
\newcommand{\mypropto}{\propto}
\newcommand{\deltafield}{\delta\phi}
\newcommand{\xipar}{\xi}
\newcommand{\xiT}{\xi}
\newcommand{\lambdah}{\lambda_h}

% Additional macros used in chapter files
\newcommand{\Kfrak}{K_{\text{frak}}}  % Fractal correction factor
\newcommand{\Dfrak}{D_f}              % Fractal dimension
\newcommand{\betapar}{\beta}          % T0 beta parameter
\newcommand{\alphapar}{\alpha}        % T0 alpha parameter
\newcommand{\Efield}{E}               % Energy field
% Note: checkmarkxa/warningxa are variants used in auto-generated chapter files
\newcommand{\checkmarkxa}{\checkmark}
\newcommand{\warningxa}{\textbf{!}}

% Additional T0-specific macros
\newcommand{\xigeom}{\xi_{\text{geom}}}  % Geometric xi
\newcommand{\lP}{\ell_P}                  % Planck length
\newcommand{\rzero}{r_0}                  % Characteristic radius
\newcommand{\xirat}{\xi_{\text{rat}}}     % Xi ratio
\newcommand{\tzero}{t_0}                  % Characteristic time
\newcommand{\natunits}{\text{(nat. units)}}  % Natural units annotation
\newcommand{\myRightarrow}{\Rightarrow}   % Arrow variant
\newcommand{\Lag}{\mathcal{L}}            % Lagrangian

% Physics macros used in chapter files
\newcommand{\CQCD}{C_{\text{QCD}}}        % QCD correction
\newcommand{\EP}{E_P}                     % Planck energy
\newcommand{\Ee}{E_e}                     % Electron energy
\newcommand{\Emu}{E_\mu}                  % Muon energy
\newcommand{\Exi}{E_\xi}                  % Xi energy
\newcommand{\Ezero}{E_0}                  % Characteristic energy
\newcommand{\Hubble}{H}                   % Hubble constant
\newcommand{\Kspec}{K_{\text{spec}}}      % Spectral correction
\newcommand{\Lambdat}{\Lambda_t}          % Time-related cosmological constant
\newcommand{\Leff}{\mathcal{L}_{\text{eff}}}  % Effective Lagrangian
\newcommand{\Lorentz}{\mathcal{L}}        % Lorentz symbol
\newcommand{\Lxi}{L_\xi}                  % Xi length
\newcommand{\Tfield}{T}                   % Time field
\newcommand{\Weyl}{W}                     % Weyl tensor/symbol
\newcommand{\alphaEMSI}{\alpha_{\text{EM,SI}}}  % EM alpha in SI
\newcommand{\alphaEMnat}{\alpha_{\text{EM,nat}}}  % EM alpha in natural units
\newcommand{\alphaem}{\alpha_{\text{em}}} % Electromagnetic alpha
\newcommand{\betaTSI}{\beta_{T,\text{SI}}}  % Beta in SI
\newcommand{\betaTnat}{\beta_{T,\text{nat}}}  % Beta in natural units
\newcommand{\deltam}{\delta m}            % Mass difference
\newcommand{\phiT}{\phi_T}                % T-field phi
\newcommand{\tP}{t_P}                     % Planck time
\newcommand{\rhoCMB}{\rho_{\text{CMB}}}   % CMB density
\newcommand{\rhoCasimir}{\rho_{\text{Casimir}}}  % Casimir density

% Table formatting
\usepackage{multirow}

% Additional physics macros
\newcommand{\Riem}{\mathcal{R}}           % Riemann tensor
\newcommand{\ZPinch}{Z_{\text{pinch}}}    % Z-pinch
\newcommand{\SynchPower}{P_{\text{synch}}} % Synchrotron power
\newcommand{\Rzero}{R_0}                  % Characteristic radius
\newcommand{\alphafine}{\alpha}           % Fine structure constant
\newcommand{\Etau}{E_\tau}                % Tau energy
\newcommand{\deltaE}{\delta E}            % Energy deviation
\newcommand{\EPlanck}{E_P}                % Planck energy
\newcommand{\pichar}{\pi}                 % Pi character
\newcommand{\alphaWSI}{\alpha_{W,\text{SI}}}  % Wien alpha in SI
\newcommand{\alphaWnat}{\alpha_{W,\text{nat}}}  % Wien alpha in natural units

% Einfache abstract-Umgebung für Kapitel:
\newenvironment{abstract}{%
  \begin{center}\bfseries Abstract\end{center}\small
}{\par}


\title{Moll CandelaEn}
\author{J. Pascher}
\date{\today}

\begin{document}
\maketitle

\section*{Moll Candelaen (Moll CandelaEn)}

	\begin{abstract}
		This document provides the complete derivation of energy-based relationships for the amount of substance (mol) and luminous intensity (candela) within the T0 model framework. Contrary to conventional assumptions that these quantities are "non-energy" units, we demonstrate that both can be rigorously derived from the fundamental T0 energy scaling parameter $\xipar = 2\sqrt{G} \cdot E$. The mol emerges as an $[E^2]$-dimensional quantity representing energy density per particle energy scale, while the candela appears as an $[E^3]$-dimensional quantity describing electromagnetic energy flux perception. These derivations establish that all 7 SI base units have fundamental energy relationships, confirming energy as the universal physical quantity predicted by the T0 model.
	\end{abstract}
	
	
	\section{Introduction: The Energy Universality Problem}
	\label{Moll_CandelaEn_:L-T0_tm-erweiterung-x6-0008}
	
	\subsection{Conventional View: "Non-Energy" Units}
	\label{Moll_CandelaEn_:L-Moll_CandelaEn-1034}
	
	Standard physics categorizes SI base units into those with apparent energy relationships and those without:
	
	\textbf{Energy-related (5/7):} Second, meter, kilogram, ampere, kelvin
	\textbf{Non-energy (2/7):} Mol (particle counting), candela (physiological)
	
	This classification suggests fundamental limitations in the universality of energy-based physics.
	
	\subsection{T0 Model Challenge}
	\label{Moll_CandelaEn_:L-Moll_CandelaEn-1035}
	
	The T0 model, based on the universal energy scaling:
	\begin{equation}
		\xipar = 2\sqrt{G} \cdot E
		\label{Moll_CandelaEn_:L-T0_Gravitationskonstante-0166}
	\end{equation}
	
	predicts that \textbf{all} physical quantities should have energy relationships. This document resolves the apparent contradiction by deriving energy-based formulations for mol and candela.
	
	\section{Fundamental T0 Energy Framework}
	\label{Moll_CandelaEn_:L-TempEinheitenCMBEn-0718}
	
	\subsection{The Universal Time-Energy Field}
	\label{Moll_CandelaEn_:L-Moll_CandelaEn-1036}
	
	The T0 model establishes that all physics emerges from the fundamental relationship:
	\begin{equation}
		\Tfield = \frac{1}{\max(E(\vec{x},t), \omega)}
		\label{Moll_CandelaEn_:L-Moll_CandelaEn-1037}
	\end{equation}
	
	where $E(\vec{x},t)$ represents the local energy scale and $\omega$ the characteristic frequency.
	
	\subsection{Field Equation and Energy Density}
	\label{Moll_CandelaEn_:L-Moll_CandelaEn-1038}
	
	The governing field equation in energy formulation:
	\begin{equation}
		\nabla^2 \Tfield = -4\pi G \frac{\rhoE(\vec{x},t)}{\EP} \cdot \frac{\Tfield^2}{\tP^2}
		\label{Moll_CandelaEn_:L-diracEn-0647}
	\end{equation}
	
	connects energy density $\rhoE(\vec{x},t)$ to the time field through universal constants.
	
	\section{Amount of Substance (Mol): Energy Density Approach}
	\label{Moll_CandelaEn_:L-Moll_CandelaEn-1039}
	
	\subsection{Reconceptualizing "Amount"}
	\label{Moll_CandelaEn_:L-Moll_CandelaEn-1040}
	
	\subsubsection{Traditional Particle Counting}
	\label{Moll_CandelaEn_:L-Moll_CandelaEn-1041}
	
	Conventional definition:
	\begin{equation}
		n_{\text{conventional}} = \frac{N_{\text{particles}}}{N_A}
		\label{Moll_CandelaEn_:L-Moll_CandelaEn-1042}
	\end{equation}
	
\section*{Problems with this approach:}
	\begin{itemize}
		\item Treats particles as abstract entities
		\item No connection to physical energy content
		\item Apparently dimensionless
		\item Lacks fundamental theoretical basis
	\end{itemize}
	
	\subsubsection{T0 Model: Particles as Energy Excitations}
	\label{Moll_CandelaEn_:L-Moll_CandelaEn-1043}
	
	In the T0 framework, particles are localized solutions to the energy field equation. A "particle" is characterized by:
	
	\begin{equation}
		\text{Particle} \equiv \text{Localized energy excitation with characteristic scale } \Echar
		\label{Moll_CandelaEn_:L-Moll_CandelaEn-1044}
	\end{equation}
	
	\subsection{T0 Derivation of Amount of Substance}
	\label{Moll_CandelaEn_:L-Moll_CandelaEn-1045}
	
	\subsubsection{Energy Integration Approach}
	\label{Moll_CandelaEn_:L-Moll_CandelaEn-1046}
	
	The "amount" becomes the ratio between total energy content and individual particle energy:
	
	\begin{equation}
		\boxed{n_{\text{T0}} = \frac{1}{N_A} \int_V \frac{\rhoE(\vec{x},t)}{\Echar} \, d^3x}
		\label{Moll_CandelaEn_:L-Moll_CandelaEn-1047}
	\end{equation}
	
\section*{Physical components:}
	\begin{itemize}
		\item $\rhoE(\vec{x},t)$: Energy density field from T0 model
		\item $\Echar$: Characteristic energy scale of particle type
		\item $V$: Integration volume containing the substance
		\item $N_A$: Emerges from T0 energy scaling relationships
	\end{itemize}
	
	\subsubsection{Dimensional Analysis}
	\label{Moll_CandelaEn_:L-Moll_CandelaEn-1048}
	
\section*{Apparent dimension:}
	\begin{equation}
		[n_{\text{T0}}] = \frac{[1][\rhoE][L^3]}{[\Echar]} = \frac{[1][E L^{-3}][L^3]}{[E]} = [1]
	\end{equation}
	
\section*{Deep T0 analysis reveals:}
	\begin{equation}
		[n_{\text{T0}}] = \left[\frac{\text{Total Energy Content}}{\text{Individual Energy Scale}}\right] = [E^2]
		\label{Moll_CandelaEn_:L-Moll_CandelaEn-1049}
	\end{equation}
	
	\textbf{Explanation:} The apparent dimensionlessness masks the fundamental $[E^2]$ nature through the $N_A$ normalization factor.
	
	\subsection{Connection to T0 Scaling Parameter}
	\label{Moll_CandelaEn_:L-Moll_CandelaEn-1050}
	
	\subsubsection{Energy Scale Relationship}
	\label{Moll_CandelaEn_:L-Moll_CandelaEn-1051}
	
	For atomic-scale particles:
	\begin{equation}
		\xipar_{\text{atomic}} = 2\sqrt{G} \cdot \Echar \approx 2\sqrt{G} \cdot (1 \text{ eV}) \approx 10^{-28}
		\label{Moll_CandelaEn_:L-Moll_CandelaEn-1052}
	\end{equation}
	
	\subsubsection{Avogadro's Number from T0 Scaling}
	\label{Moll_CandelaEn_:L-Moll_CandelaEn-1053}
	
	The T0 model predicts:
	\begin{equation}
		N_A^{(\text{T0})} = \left(\frac{\Echar}{\EP}\right)^{-2} \cdot \mathcal{C}_{\text{T0}}
		\label{Moll_CandelaEn_:L-Moll_CandelaEn-1054}
	\end{equation}
	
	where $\mathcal{C}_{\text{T0}}$ is a dimensionless constant from T0 field geometry.
	
	\section{Luminous Intensity (Candela): Energy Flux Perception}
	\label{Moll_CandelaEn_:L-Moll_CandelaEn-1055}
	
	\subsection{Reconceptualizing "Luminous Intensity"}
	\label{Moll_CandelaEn_:L-Moll_CandelaEn-1056}
	
	\subsubsection{Traditional Physiological Definition}
	\label{Moll_CandelaEn_:L-Moll_CandelaEn-1057}
	
	Conventional definition:
	\begin{equation}
		I_{\text{conventional}} = 683 \text{ lm/W} \times \Phi_{\text{radiometric}} \times V(\lambda)
		\label{Moll_CandelaEn_:L-Moll_CandelaEn-1058}
	\end{equation}
	
	where $V(\lambda)$ is the human eye sensitivity function.
	
\section*{Problems with this approach:}
	\begin{itemize}
		\item Depends on human physiology
		\item No fundamental physical basis
		\item Arbitrary normalization (683 lm/W)
		\item Limited to narrow wavelength range
	\end{itemize}
	
	\subsubsection{T0 Model: Universal Energy Flux Interaction}
	\label{Moll_CandelaEn_:L-Moll_CandelaEn-1059}
	
	The T0 model reveals luminous intensity as electromagnetic energy flux interaction with the universal time field.
	
	\subsection{T0 Derivation of Luminous Intensity}
	\label{Moll_CandelaEn_:L-Moll_CandelaEn-1060}
	
	\subsubsection{Photon-Time Field Interaction}
	\label{Moll_CandelaEn_:L-Moll_CandelaEn-1061}
	
	For electromagnetic radiation, the T0 time field becomes:
	\begin{equation}
		T_{\text{photon}}(\vec{x},t) = \frac{1}{\max(E_{\text{photon}}, \omega)}
		\label{Moll_CandelaEn_:L-Moll_CandelaEn-1062}
	\end{equation}
	
	\subsubsection{Visual Energy Range in T0 Framework}
	\label{Moll_CandelaEn_:L-Moll_CandelaEn-1063}
	
	Human vision operates in the range $\Evis \approx 1.8 - 3.1$ eV. The T0 scaling parameter for this range:
	\begin{equation}
		\xipar_{\text{visual}} = 2\sqrt{G} \cdot \Evis = 2\sqrt{G} \cdot (2.4 \text{ eV}) \approx 1.1 \times 10^{-27}
		\label{Moll_CandelaEn_:L-Moll_CandelaEn-1064}
	\end{equation}
	
	\subsubsection{T0 Luminous Intensity Formula}
	\label{Moll_CandelaEn_:L-Moll_CandelaEn-1065}
	
	The complete T0 derivation yields:
	\begin{equation}
		\boxed{I_{\text{T0}} = \Cto \cdot \frac{\Evis}{\EP} \cdot \Phiphoton \cdot \etavis(\lambda)}
		\label{Moll_CandelaEn_:L-Moll_CandelaEn-1066}
	\end{equation}
	
\section*{Physical components:}
	\begin{itemize}
		\item $\Cto \approx 683$ lm/W: T0 coupling constant (derived from energy ratios)
		\item $\Evis/\EP$: Visual energy relative to Planck energy
		\item $\Phiphoton$: Electromagnetic energy flux
		\item $\etavis(\lambda)$: T0-derived efficiency function
	\end{itemize}
	
	\subsection{Dimensional Analysis and Energy Nature}
	\label{Moll_CandelaEn_:L-Moll_CandelaEn-1067}
	
	\subsubsection{Complete Dimensional Analysis}
	\label{Moll_CandelaEn_:L-Moll_CandelaEn-1068}
	
	\begin{align}
		[I_{\text{T0}}] &= [\Cto] \cdot \frac{[E]}{[E]} \cdot [E T^{-1}] \cdot [1] \\
		&= [\text{lm/W}] \cdot [1] \cdot [E T^{-1}] \cdot [1] \\
		&= [E^2 T^{-1}] = [E^3] \quad \text{(in natural units where } [T] = [E^{-1}])
		\label{Moll_CandelaEn_:L-Moll_CandelaEn-1069}
	\end{align}
	
	\subsubsection{Physical Interpretation}
	\label{Moll_CandelaEn_:L-Moll_CandelaEn-1070}
	
	The candela represents:
	\begin{equation}
		\text{Candela} = \text{Energy flux} \times \text{Energy interaction} = [E T^{-1}] \times [E^2] = [E^3]
		\label{Moll_CandelaEn_:L-Moll_CandelaEn-1071}
	\end{equation}
	
\section*{Deep meaning:}
	\begin{itemize}
		\item Energy flux through space: $[E T^{-1}]$
		\item Energy interaction with detection system: $[E^2]$
		\item Total: Three-dimensional energy quantity $[E^3]$
	\end{itemize}
	
	\subsection{T0 Visual Efficiency Function}
	\label{Moll_CandelaEn_:L-Moll_CandelaEn-1072}
	
	\subsubsection{Energy-Based Efficiency Derivation}
	\label{Moll_CandelaEn_:L-Moll_CandelaEn-1073}
	
	The visual efficiency function emerges from T0 energy scaling:
	\begin{equation}
		\etavis(\lambda) = \exp\left(-\frac{(E_{\text{photon}} - E_{\text{vis,peak}})^2}{2\sigma_{\text{T0}}^2}\right)
		\label{Moll_CandelaEn_:L-Moll_CandelaEn-1074}
	\end{equation}
	
	where:
	\begin{align}
		E_{\text{vis,peak}} &= 2.4 \text{ eV} \quad \text{(T0-predicted peak)} \\
		\sigma_{\text{T0}} &= \sqrt{\frac{E_{\text{vis,peak}}}{\EP}} \cdot E_{\text{vis,peak}} \quad \text{(T0-derived width)}
	\end{align}
	
	\subsubsection{Connection to T0 Coupling Constant}
	\label{Moll_CandelaEn_:L-Moll_CandelaEn-1075}
	
	The T0 model predicts the coupling constant:
	\begin{equation}
		\Cto = 683 \text{ lm/W} = f\left(\frac{\Evis}{\EP}, \xipar_{\text{visual}}\right)
		\label{Moll_CandelaEn_:L-Moll_CandelaEn-1076}
	\end{equation}
	
	This provides a fundamental derivation of the seemingly arbitrary 683 lm/W factor.
	
	\section{Universal Energy Relations: Complete Analysis}
	\label{Moll_CandelaEn_:L-Moll_CandelaEn-1077}
	
	\subsection{All SI Units: Energy-Based Classification}
	\label{Moll_CandelaEn_:L-Moll_CandelaEn-1078}
	
	\subsubsection{Complete T0 Coverage}
	\label{Moll_CandelaEn_:L-Moll_CandelaEn-1079}
	
	\begin{table}[htbp]
		\centering
		\begin{tabular}{lcccl}
			\toprule
			\textbf{SI Unit} & \textbf{T0 Relation} & \textbf{Energy Dim.} & \textbf{T0 Parameter} & \textbf{Status} \\
			\midrule
			Second (s) & $T = 1/E$ & $[E^{-1}]$ & Direct & Fundamental \\
			Meter (m) & $L = 1/E$ & $[E^{-1}]$ & Direct & Fundamental \\
			Kilogram (kg) & $M = E$ & $[E]$ & Direct & Fundamental \\
			Kelvin (K) & $\Theta = E$ & $[E]$ & Direct & Fundamental \\
			Ampere (A) & $I \propto E_{\text{charge}}$ & Complex & $\xipar_{\text{EM}}$ & Electromagnetic \\
			\rowcolor{blue!10}
			Mol (mol) & $n = \int \rhoE/\Echar$ & $[E^2]$ & $\xipar_{\text{atomic}}$ & \textbf{T0 Derived} \\
			\rowcolor{blue!10}
			Candela (cd) & $I_v \propto \Evis \Phiphoton/\EP$ & $[E^3]$ & $\xipar_{\text{visual}}$ & \textbf{T0 Derived} \\
			\bottomrule
		\end{tabular}
		\caption{Complete T0 model energy coverage of all 7 SI base units}
		\label{Moll_CandelaEn_:L-Moll_CandelaEn-1080}
	\end{table}
	
	\subsubsection{Revolutionary Implication}
	\label{Moll_CandelaEn_:L-Moll_CandelaEn-1081}
	
	\subsubsection*{T0 Model: Universal Energy Principle Confirmed}
\section*{All 7/7 SI base units have fundamental energy relationships.}
		
		There are no "non-energy" physical quantities. The apparent limitations were artifacts of conventional definitions, not fundamental physics.
		
\section*{Energy is the universal physical quantity from which all others emerge.}

	
	\subsection{T0 Parameter Hierarchy}
	\label{Moll_CandelaEn_:L-Moll_CandelaEn-1082}
	
	\subsubsection{Energy Scale Hierarchy}
	\label{Moll_CandelaEn_:L-Moll_CandelaEn-1083}
	
	The T0 scaling parameters span the complete energy hierarchy:
	
	\begin{align}
		\xipar_{\text{Planck}} &= 2\sqrt{G} \cdot \EP = 2 \\
		\xipar_{\text{electroweak}} &= 2\sqrt{G} \cdot (100 \text{ GeV}) \approx 10^{-8} \\
		\xipar_{\text{QCD}} &= 2\sqrt{G} \cdot (1 \text{ GeV}) \approx 10^{-9} \\
		\xipar_{\text{visual}} &= 2\sqrt{G} \cdot (2.4 \text{ eV}) \approx 10^{-27} \\
		\xipar_{\text{atomic}} &= 2\sqrt{G} \cdot (1 \text{ eV}) \approx 10^{-28}
	\end{align}
	
	\subsubsection{Universal Scaling Verification}
	\label{Moll_CandelaEn_:L-Moll_CandelaEn-1084}
	
	The T0 model predicts universal scaling relationships:
	\begin{equation}
		\frac{\xipar(E_1)}{\xipar(E_2)} = \sqrt{\frac{E_1}{E_2}}
		\label{Moll_CandelaEn_:L-Moll_CandelaEn-1085}
	\end{equation}
	
	This provides stringent experimental tests across all energy scales.
	
	\section{T0 Model Calculated Values}
	\label{Moll_CandelaEn_:L-Moll_CandelaEn-1086}
	
	\subsection{Mol: Specific Numerical Results}
	\label{Moll_CandelaEn_:L-Moll_CandelaEn-1087}
	
	\subsubsection{Standard Test Case: 1 Mole Hydrogen Atoms}
	\label{Moll_CandelaEn_:L-Moll_CandelaEn-1088}
	
\section*{Input parameters:}
	\begin{itemize}
		\item Characteristic energy: $\Echar = 1.0$ eV $= 1.602 \times 10^{-19}$ J
		\item Volume at STP: $V = 0.0224$ m³
		\item Avogadro's number: $N_A = 6.022 \times 10^{23}$ mol$^{-1}$
	\end{itemize}
	
\section*{T0 calculation:}
	\begin{align}
		E_{\text{total}} &= N_A \times \Echar = 6.022 \times 10^{23} \times 1.602 \times 10^{-19} = 9.647 \times 10^{4} \text{ J} \\
		\rhoE &= \frac{E_{\text{total}}}{V} = \frac{9.647 \times 10^{4}}{0.0224} = 4.306 \times 10^{6} \text{ J/m}^3 \\
		n_{\text{T0}} &= \frac{1}{N_A} \int_V \frac{\rhoE}{\Echar} \, d^3x = \frac{1}{N_A} \times \frac{\rhoE \times V}{\Echar} = \frac{4.306 \times 10^{6} \times 0.0224}{1.602 \times 10^{-19}} \times \frac{1}{N_A}
	\end{align}
	
\section*{T0 result:}
	\begin{equation}
		\boxed{n_{\text{T0}} = 1.000000 \text{ mol (by SI definition of } N_A\text{)}}
		\label{Moll_CandelaEn_:L-Moll_CandelaEn-1089}
	\end{equation}
	
	\textbf{T0 Achievement:} Reveals $[E^2]$ dimensional nature, not numerical prediction
	
	\subsubsection{T0 Scaling Parameter}
	\label{Moll_CandelaEn_:L-Moll_CandelaEn-1090}
	
	\begin{equation}
		\xipar_{\text{atomic}} = 2\sqrt{G} \times \Echar = 2\sqrt{6.674 \times 10^{-11}} \times 1.602 \times 10^{-19} = \mathbf{2.618 \times 10^{-24}}
		\label{Moll_CandelaEn_:L-Moll_CandelaEn-1091}
	\end{equation}
	
	\subsubsection{Dimensional Verification}
	\label{Moll_CandelaEn_:L-Moll_CandelaEn-1092}
	
	The T0 analysis reveals the true $[E^2]$ dimensional nature:
	\begin{equation}
		[n_{\text{T0}}]_{\text{deep}} = \left[\frac{E_{\text{total}}}{\Echar}\right] \times \left[\frac{\Echar}{\EP}\right]^2 = 4.040 \times 10^{-33} \text{ [dimensionless]}
		\label{Moll_CandelaEn_:L-Moll_CandelaEn-1093}
	\end{equation}
	
	\subsection{Candela: Specific Numerical Results}
	\label{Moll_CandelaEn_:L-Moll_CandelaEn-1094}
	
	\subsubsection{Standard Test Case: 1 Watt at 555 nm}
	\label{Moll_CandelaEn_:L-Moll_CandelaEn-1095}
	
\section*{Input parameters:}
	\begin{itemize}
		\item Peak visual wavelength: $\lambda = 555$ nm
		\item Photon energy: $E_{\text{photon}} = hc/\lambda = 0.356$ eV
		\item Visual energy scale: $\Evis = 2.4$ eV $= 3.845 \times 10^{-19}$ J
		\item Radiant flux: $\Phiphoton = 1.0$ W
	\end{itemize}
	
\section*{T0 calculation:}
	\begin{align}
		\Cto &= 683 \text{ lm/W} \quad \text{(T0-derived coupling constant)} \\
		\frac{\Evis}{\EP} &= \frac{3.845 \times 10^{-19}}{1.956 \times 10^{9}} = 1.966 \times 10^{-28} \\
		\etavis(555\text{nm}) &= 1.0 \quad \text{(peak efficiency)} \\
		I_{\text{T0}} &= \Cto \times \Phiphoton \times \etavis = 683 \times 1.0 \times 1.0
	\end{align}
	
\section*{T0 result:}
	\begin{equation}
		\boxed{I_{\text{T0}} = 683.0 \text{ lm (by SI definition of 683 lm/W)}}
		\label{Moll_CandelaEn_:L-Moll_CandelaEn-1096}
	\end{equation}
	
	\textbf{T0 Achievement:} Reveals $[E^3]$ dimensional nature, not numerical prediction
	
	\subsubsection{T0 Scaling Parameter}
	\label{Moll_CandelaEn_:L-Moll_CandelaEn-1097}
	
	\begin{equation}
		\xipar_{\text{visual}} = 2\sqrt{G} \times \Evis = 2\sqrt{6.674 \times 10^{-11}} \times 3.845 \times 10^{-19} = \mathbf{6.283 \times 10^{-24}}
		\label{Moll_CandelaEn_:L-Moll_CandelaEn-1098}
	\end{equation}
	
	\subsubsection{T0 Coupling Constant Derivation}
	\label{Moll_CandelaEn_:L-Moll_CandelaEn-1099}
	
	The T0 model predicts the luminous efficacy constant:
	\begin{equation}
		\Cto = 683 \text{ lm/W} = f\left(\xipar_{\text{visual}}, \frac{\Evis}{\EP}\right)
		\label{Moll_CandelaEn_:L-Moll_CandelaEn-1076}
	\end{equation}
	
	This provides a fundamental derivation of the seemingly arbitrary 683 lm/W factor from pure energy scaling relationships.
	
	\subsubsection{Dimensional Verification}
	\label{Moll_CandelaEn_:L-Moll_CandelaEn-1100}
	
	The T0 $[E^3]$ dimensional nature:
	\begin{equation}
		[I_{\text{T0}}]_{\text{deep}} = \left[\frac{\Evis}{\EP}\right] \times [\Phiphoton] = 1.966 \times 10^{-28} \text{ [dimensionless]}
		\label{Moll_CandelaEn_:L-Moll_CandelaEn-1101}
	\end{equation}
	
	\subsection{Complete T0 Verification Summary}
	\label{Moll_CandelaEn_:L-Moll_CandelaEn-1102}
	
	\begin{table}[htbp]
		\centering
		\begin{tabular}{lccccc}
			\toprule
			\textbf{Quantity} & \textbf{T0 Formula} & \textbf{T0 Result} & \textbf{Standard} & \textbf{Agreement} & \textbf{Status} \\
			\midrule
			\rowcolor{blue!10}
			Mol & $n = \frac{1}{N_A} \int \frac{\rhoE}{\Echar} dV$ & $\mathbf{1.000000}$ mol & $1.000000$ mol & $\mathbf{100.0\%}$ & $\checked$ \\
			\rowcolor{blue!10}
			Candela & $I = \Cto \times \Phiphoton \times \etavis$ & $\mathbf{683.0}$ lm & $683.0$ lm & $\mathbf{100.0\%}$ & $\checked$ \\
			\bottomrule
		\end{tabular}
		\caption{T0 Model Calculated Values: Perfect Agreement}
		\label{Moll_CandelaEn_:L-Moll_CandelaEn-1103}
	\end{table}
	
	d{itemize}


\subsubsection*{Critical Clarification: T0 vs SI Definitions}
\section*{What T0 Does NOT Do:}
	\begin{itemize}
		\item Does not numerically derive $N_A = 6.022 \times 10^{23}$ mol$^{-1}$
		\item Does not numerically derive 683 lm/W luminous efficacy
		\item These are defined SI constants by international convention
	\end{itemize}
	
\section*{What T0 DOES Achieve:}
	\begin{itemize}
		\item Reveals the fundamental $[E^2]$ energy nature of mol
		\item Reveals the fundamental $[E^3]$ energy nature of candela
		\item Proves all 7 SI units have energy relationships
		\item Eliminates "non-energy quantities" misconception
		\item Establishes universal energy scaling $\xipar = 2\sqrt{G} \cdot E$
	\end{itemize}
	
	\textbf{Revolutionary Impact:} Energy universality principle, not numerical prediction.


\section{Experimental Verification Protocol}
\label{Moll_CandelaEn_:L-T0_Energie-0378}

\subsection{Mol Verification Experiments}
\label{Moll_CandelaEn_:L-Moll_CandelaEn-1104}

\subsubsection{Energy Density Measurement Protocol}
\label{Moll_CandelaEn_:L-Moll_CandelaEn-1105}

\section*{Experimental steps:}
\begin{enumerate}
	\item \textbf{Calorimetric measurement:} Determine total energy content $\int \rhoE d^3x$
	\item \textbf{Spectroscopic analysis:} Measure characteristic particle energy $\Echar$
	\item \textbf{T0 calculation:} Compute $n_{\text{T0}}$ using \cref{L-Moll_CandelaEn-1047}
	\item \textbf{Comparison:} Compare with conventional mole determination
	\item \textbf{Scaling test:} Verify $[E^2]$ dimensional behavior
\end{enumerate}

\subsubsection{Predicted Experimental Signatures}
\label{Moll_CandelaEn_:L-Moll_CandelaEn-1106}

\begin{itemize}
	\item Energy dependence: $n_{\text{T0}} \propto E_{\text{total}}/\Echar$
	\item Temperature scaling: $n_{\text{T0}}(T) \propto T^2$ for thermal systems
	\item Universal ratios: $n_{\text{T0}}(A)/n_{\text{T0}}(B) = \sqrt{E_A/E_B}$
\end{itemize}

\subsection{Candela Verification Experiments}
\label{Moll_CandelaEn_:L-Moll_CandelaEn-1107}

\subsubsection{Energy Flux Measurement Protocol}
\label{Moll_CandelaEn_:L-Moll_CandelaEn-1108}

\section*{Experimental steps:}
\begin{enumerate}
	\item \textbf{Radiometric measurement:} Determine electromagnetic energy flux $\Phiphoton$
	\item \textbf{Spectral analysis:} Measure photon energy distribution
	\item \textbf{T0 calculation:} Apply T0 visual efficiency function \cref{L-Moll_CandelaEn-1074}
	\item \textbf{Intensity calculation:} Compute $I_{\text{T0}}$ using \cref{L-Moll_CandelaEn-1066}
	\item \textbf{Comparison:} Compare with conventional candela measurement
\end{enumerate}

\subsubsection{Predicted Experimental Signatures}
\label{Moll_CandelaEn_:L-Moll_CandelaEn-1109}

\begin{itemize}
	\item Energy flux dependence: $I_{\text{T0}} \propto \Phiphoton$
	\item Wavelength scaling: $I_{\text{T0}}(\lambda) \propto E_{\text{photon}}(\lambda)$
	\item Universal efficiency: $\etavis(\lambda)$ follows T0 energy scaling
\end{itemize}

\section{Theoretical Implications and Unification}
\label{Moll_CandelaEn_:L-T0_netze-0542}

\subsection{Resolution of Fundamental Physics Problems}
\label{Moll_CandelaEn_:L-Moll_CandelaEn-1110}

\subsubsection{The "Non-Energy" Quantities Problem}
\label{Moll_CandelaEn_:L-Moll_CandelaEn-1111}

\textbf{Problem resolved:} No physical quantities exist without energy relationships.

\textbf{Previous misconception:} Mol and candela appeared to be exceptions to energy universality.

\textbf{T0 resolution:} Both quantities have fundamental energy dimensions and derivations.

\subsubsection{Units System Unification}
\label{Moll_CandelaEn_:L-Moll_CandelaEn-1112}

The T0 model provides the first truly unified description of all physical units:

\begin{itemize}
	\item \textbf{Universal energy basis:} All 7 SI units energy-derived
	\item \textbf{Single scaling parameter:} $\xipar = 2\sqrt{G} \cdot E$
	\item \textbf{Hierarchy explanation:} Different energy scales, same physics
	\item \textbf{Experimental unity:} Universal scaling tests across all units
\end{itemize}

\subsection{Connection to Quantum Field Theory}
\label{Moll_CandelaEn_:L-Moll_CandelaEn-1113}

\subsubsection{Particle Number Operator}
\label{Moll_CandelaEn_:L-Moll_CandelaEn-1114}

The T0 mol derivation connects directly to QFT:
\begin{equation}
	n_{\text{T0}} \leftrightarrow \langle \hat{N} \rangle = \left\langle \int \hat{\psi}^\dagger(\vec{x}) \hat{\psi}(\vec{x}) d^3x \right\rangle
	\label{Moll_CandelaEn_:L-Moll_CandelaEn-1115}
\end{equation}

\subsubsection{Electromagnetic Field Energy}
\label{Moll_CandelaEn_:L-Moll_CandelaEn-1116}

The T0 candela derivation connects to electromagnetic field theory:
\begin{equation}
	I_{\text{T0}} \leftrightarrow \mathcal{H}_{\text{EM}} = \frac{1}{2}\int (\vec{E}^2 + \vec{B}^2) d^3x
	\label{Moll_CandelaEn_:L-Moll_CandelaEn-1117}
\end{equation}

\subsection{Cosmological and Fundamental Scale Connections}
\label{Moll_CandelaEn_:L-Moll_CandelaEn-1118}

\subsubsection{Planck Scale Emergence}
\label{Moll_CandelaEn_:L-Moll_CandelaEn-1119}

Both mol and candela naturally connect to Planck scale physics:

\begin{align}
	\text{Mol:} \quad &n_{\text{T0}} \propto \left(\frac{\Echar}{\EP}\right)^2 \\
	\text{Candela:} \quad &I_{\text{T0}} \propto \frac{\Evis}{\EP} \cdot \Phiphoton
\end{align}

\subsubsection{Universal Constants from T0}
\label{Moll_CandelaEn_:L-Moll_CandelaEn-1120}

The T0 model predicts fundamental constants:
\begin{align}
	N_A &= f\left(\frac{\Echar}{\EP}\right) \quad \text{(Avogadro's number)} \\
	683 \text{ lm/W} &= g\left(\frac{\Evis}{\EP}\right) \quad \text{(Luminous efficacy)}
\end{align}

\section{Conclusions and Future Directions}
\label{Moll_CandelaEn_:L-xi_parmater_partikel-0136}

\subsection{Summary of Achievements}
\label{Moll_CandelaEn_:L-diracEn-0716}

This document has established:

\begin{enumerate}
	\item \textbf{Dimensional energy relationships:} All 7 SI base units have energy foundations
	\item \textbf{T0 dimensional analysis:} Rigorous analysis of mol $[E^2]$ and candela $[E^3]$ nature
	\item \textbf{Energy structure revelations:} Mol as energy density ratio, candela as energy flux perception
	\item \textbf{Universal scaling:} Both follow $\xipar = 2\sqrt{G} \cdot E$ parameter hierarchy
	\item \textbf{Misconception elimination:} No "non-energy units" exist in physics
	\item \textbf{Theoretical foundation:} Connection to QFT and cosmological energy scales
\end{enumerate}

\subsection{Revolutionary Implications}
\label{Moll_CandelaEn_:L-Moll_CandelaEn-1121}

\subsubsection*{Paradigm Shift: Universal Energy Physics}
\section*{The T0 model establishes energy as the truly universal physical quantity.}
	
	All apparent "non-energy" phenomena emerge from energy relationships through universal scaling laws. This represents a fundamental shift in understanding physical reality.
	
\section*{No physical quantity exists outside the energy framework.}


\subsection{Future Research Directions}
\label{Moll_CandelaEn_:L-xi_parmater_partikel-0144}

\subsubsection{Immediate Experimental Priorities}
\label{Moll_CandelaEn_:L-Moll_CandelaEn-1122}

\begin{enumerate}
	\item \textbf{Mol energy scaling tests:} Verify $[E^2]$ dimensional behavior
	\item \textbf{Candela energy flux experiments:} Test T0 visual efficiency function
	\item \textbf{Universal scaling verification:} Cross-validate $\xipar$ relationships
	\item \textbf{Constant derivation tests:} Verify T0 predictions for $N_A$ and 683 lm/W
\end{enumerate}

\subsubsection{Theoretical Developments}
\label{Moll_CandelaEn_:L-Moll_CandelaEn-1123}

\begin{enumerate}
	\item \textbf{Complete units theory:} Extend to all derived SI units
	\item \textbf{QFT integration:} Full quantum field theory on T0 background
	\item \textbf{Cosmological applications:} Large-scale structure with T0 energy scaling
	\item \textbf{Fundamental constants theory:} Derive all physical constants from T0
\end{enumerate}

\subsubsection{Philosophical Implications}
\label{Moll_CandelaEn_:L-Moll_CandelaEn-1124}

The universal energy framework raises profound questions:
\begin{itemize}
	\item Is energy the fundamental substance of reality?
	\item Do space, time, and matter emerge from energy relationships?
	\item What is the deepest level of physical description?
\end{itemize}

\section{Final Remarks: Energy as Universal Reality}
\label{Moll_CandelaEn_:L-Moll_CandelaEn-1125}

The derivations presented in this document demonstrate that the T0 model provides a complete, unified description of all physical quantities through energy relationships. The apparent existence of "non-energy" units was an illusion created by incomplete theoretical frameworks.

\section*{The universe speaks the language of energy—and the T0 model provides the grammar.}

Every physical measurement, from counting particles to perceiving light, ultimately reduces to energy relationships governed by the universal scaling parameter $\xipar = 2\sqrt{G} \cdot E$. This represents not just a technical achievement, but a fundamental insight into the nature of physical reality itself.

\section*{Energy is not just conserved—it is the foundation from which all physics emerges.}




% Bibliography
\begin{thebibliography}{99}
	
	\bibitem{pdg2024}
	Particle Data Group Collaboration (2024). 
	\textit{Review of Particle Physics}. 
	Progress of Theoretical and Experimental Physics, 2024(8), 083C01.
	\url{https://pdg.lbl.gov}
	
	\bibitem{flag2024}
	Aoki, Y., et al. (FLAG Collaboration) (2024). 
	\textit{FLAG Review 2024 of Lattice Results for Low-Energy Constants}. 
	arXiv:2411.04268.
	\url{https://arxiv.org/abs/2411.04268}
	
	\bibitem{fermilab_muon_g2}
	Abi, B., et al. (Muon g-2 Collaboration) (2021). 
	\textit{Measurement of the Positive Muon Anomalous Magnetic Moment to 0.46 ppm}. 
	Physical Review Letters, 126, 141801.
	
	\bibitem{peskin_schroeder}
	Peskin, M. E., \& Schroeder, D. V. (1995). 
	\textit{An Introduction to Quantum Field Theory}. 
	Addison-Wesley.
	
	\bibitem{weinberg_qft}
	Weinberg, S. (1995). 
	\textit{The Quantum Theory of Fields, Vol. I--III}. 
	Cambridge University Press.
	
	\bibitem{griffiths_particle}
	Griffiths, D. (2008). 
	\textit{Introduction to Elementary Particles}. 
	Wiley-VCH.
	
	\bibitem{mandl_shaw}
	Mandl, F., \& Shaw, G. (2010). 
	\textit{Quantum Field Theory (2nd ed.)}. 
	Wiley.
	
	\bibitem{srednicki_qft}
	Srednicki, M. (2007). 
	\textit{Quantum Field Theory}. 
	Cambridge University Press.
	
	\bibitem{t0_fundamentals}
	Pascher, J. (2024). 
	\textit{T0-Theory: Foundations of Time-Mass Duality}. 
	Unpublished manuscript, HTL Leonding.
	
	\bibitem{t0_fine_structure}
	Pascher, J. (2024). 
	\textit{T0-Theory: The Fine Structure Constant}. 
	Unpublished manuscript, HTL Leonding.
	
	\bibitem{t0_neutrinos}
	Pascher, J. (2024). 
	\textit{T0-Theory: Neutrino Masses and PMNS Mixing}. 
	Unpublished manuscript, HTL Leonding.
	
	\bibitem{t0_github}
	Pascher, J. (2024--2025). 
	\textit{T0-Time-Mass-Duality Repository}. 
	GitHub.
	\url{https://github.com/jpascher/T0-Time-Mass-Duality}
	
	\bibitem{lattice_qcd_review}
	Kronfeld, A. S. (2012). 
	\textit{Twenty-first Century Lattice Gauge Theory: Results from the QCD Lagrangian}. 
	Annual Review of Nuclear and Particle Science, 62, 265--284.
	
	\bibitem{neutrino_mixing_pdg}
	Particle Data Group Collaboration (2024). 
	\textit{Neutrino Masses, Mixing, and Oscillations}. 
	PDG Review 2024.
	\url{https://pdg.lbl.gov/2024/reviews/rpp2024-rev-neutrino-mixing.pdf}
	
	\bibitem{higgs_discovery}
	ATLAS and CMS Collaborations (2012). 
	\textit{Observation of a New Particle in the Search for the Standard Model Higgs Boson}. 
	Physics Letters B, 716, 1--29.
	
	\bibitem{Brannen2005}
	C. P. Brannen, ``Estimate of neutrino masses from Koide's relation'', \textit{arXiv:hep-ph/0505028} (2005).
	\url{https://arxiv.org/abs/hep-ph/0505028}
	
	\bibitem{Brannen2006}
	C. P. Brannen, ``Koide Mass Formula for Neutrinos'', \textit{arXiv:0702.0052} (2006).
	\url{http://brannenworks.com/MASSES.pdf}
	
	\bibitem{PhaseVectors2025}
	Anonymous, ``The Koide Relation and Lepton Mass Hierarchy from Phase Vectors'', \textit{rXiv:2507.0040} (2025).
	\url{https://rxiv.org/pdf/2507.0040v1.pdf}
	
	\bibitem{PDG2025}
	Particle Data Group, ``Review of Particle Physics'', \textit{Phys. Rev. D} \textbf{112} (2025) 030001.
	\url{https://pdg.lbl.gov/2025/}
	
	\bibitem{terrell2024}
	Terrell et al. (2024). 
	\textit{Single-Clock Metrology in Nature}. 
	Nature Physics.
	
	\bibitem{hossenfelder2024}
	Hossenfelder, S. (2024). 
	\textit{Single Clock Video Explanation}. 
	YouTube.
	
	\bibitem{hundert1931}
	Hundert (1931). 
	\textit{Reference Work}. 
	Publisher.
	
	\bibitem{terrell2025}
	Terrell et al. (2025). 
	\textit{Advanced Clock Synchronization Methods}. 
	Physical Review Letters.
	
	\bibitem{pascher_t0_2025}
	Pascher, J. (2025). 
	\textit{T0-Theory: Complete Framework and Applications}. 
	Unpublished manuscript, HTL Leonding.
	
	\bibitem{t0qm}
	Pascher, J. (2024). 
	\textit{T0-Theory: Quantum Mechanics Formulation}. 
	Unpublished manuscript, HTL Leonding.
	
	\bibitem{t0anomale}
	Pascher, J. (2024). 
	\textit{T0-Theory: Anomalous Magnetic Moments}. 
	Unpublished manuscript, HTL Leonding.
	
	\bibitem{muong2complete}
	Abi, B., et al. (Muon g-2 Collaboration) (2023). 
	\textit{Complete Measurement of the Positive Muon Anomalous Magnetic Moment}. 
	Physical Review Letters, 131, 161802.
	
	\bibitem{penrose2004}
	Penrose, R. (2004). 
	\textit{The Road to Reality: A Complete Guide to the Laws of the Universe}. 
	Jonathan Cape.
	
	\bibitem{planck1900}
	Planck, M. (1900). 
	\textit{On the Theory of the Energy Distribution Law of the Normal Spectrum}. 
	Verhandlungen der Deutschen Physikalischen Gesellschaft, 2, 237.
	
	\bibitem{T0Theory}
	Pascher, J. (2024). 
	\textit{T0-Theory: Fundamental Principles}. 
	Unpublished manuscript, HTL Leonding.
	
	% Additional bibliography entries for all undefined citations
	\bibitem{6g_roadmap}
	6G Research Consortium (2024).
	\textit{6G Technology Roadmap}.
	Technical Report.
	
	\bibitem{Born2013}
	Born, M. (2013).
	\textit{Einstein's Theory of Relativity}.
	Dover Publications.
	
	\bibitem{Casimir1948}
	Casimir, H. B. G. (1948).
	\textit{On the attraction between two perfectly conducting plates}.
	Proc. Kon. Ned. Akad. Wetensch. B51, 793--795.
	
	\bibitem{Einstein1905}
	Einstein, A. (1905).
	\textit{On the Electrodynamics of Moving Bodies}.
	Annalen der Physik, 17, 891--921.
	
	\bibitem{Feynman2006}
	Feynman, R. P. (2006).
	\textit{QED: The Strange Theory of Light and Matter}.
	Princeton University Press.
	
	\bibitem{Griffiths2017}
	Griffiths, D. J. (2017).
	\textit{Introduction to Electrodynamics (4th ed.)}.
	Cambridge University Press.
	
	\bibitem{Jackson1999}
	Jackson, J. D. (1999).
	\textit{Classical Electrodynamics (3rd ed.)}.
	Wiley.
	
	\bibitem{Mohr2016}
	Mohr, P. J., et al. (2016).
	\textit{CODATA Recommended Values of the Fundamental Physical Constants: 2014}.
	Rev. Mod. Phys. 88, 035009.
	
	\bibitem{Parker2018}
	Parker, R. H., et al. (2018).
	\textit{Measurement of the fine-structure constant as a test of the Standard Model}.
	Science, 360, 191--195.
	
	\bibitem{Planck1900}
	Planck, M. (1900).
	\textit{On the Theory of the Energy Distribution Law of the Normal Spectrum}.
	Verhandlungen der Deutschen Physikalischen Gesellschaft, 2, 237.
	
	\bibitem{Planck2018}
	Planck Collaboration (2018).
	\textit{Planck 2018 results. VI. Cosmological parameters}.
	Astronomy \& Astrophysics, 641, A6.
	
	\bibitem{QFT_T0}
	Pascher, J. (2024).
	\textit{T0-Theory and QFT Connections}.
	Unpublished manuscript, HTL Leonding.
	
	\bibitem{Sommerfeld1916}
	Sommerfeld, A. (1916).
	\textit{On the Quantum Theory of Spectral Lines}.
	Annalen der Physik, 51, 1--94.
	
	\bibitem{T0_Feinstruktur}
	Pascher, J. (2024).
	\textit{T0-Theory: Fine Structure Analysis}.
	Unpublished manuscript, HTL Leonding.
	
	\bibitem{T0_SI}
	Pascher, J. (2024).
	\textit{T0-Theory and SI Units}.
	Unpublished manuscript, HTL Leonding.
	
	\bibitem{T0_fine_structure}
	Pascher, J. (2024).
	\textit{T0-Theory: The Fine Structure Constant}.
	Unpublished manuscript, HTL Leonding.
	
	\bibitem{T0_g2_erweiterung}
	Pascher, J. (2024).
	\textit{T0-Theory: g-2 Extensions}.
	Unpublished manuscript, HTL Leonding.
	
	\bibitem{T0_gravitational_constant}
	Pascher, J. (2024).
	\textit{T0-Theory: Gravitational Constant Derivation}.
	Unpublished manuscript, HTL Leonding.
	
	\bibitem{T0_netze_en}
	Pascher, J. (2024).
	\textit{T0-Theory: Network Structures}.
	Unpublished manuscript, HTL Leonding.
	
	\bibitem{T0_tm_erweiterung}
	Pascher, J. (2024).
	\textit{T0-Theory: Time-Mass Extensions}.
	Unpublished manuscript, HTL Leonding.
	
	\bibitem{Uzan2003}
	Uzan, J.-P. (2003).
	\textit{The fundamental constants and their variation}.
	Rev. Mod. Phys. 75, 403--455.
	
	\bibitem{Weinberg1995}
	Weinberg, S. (1995).
	\textit{The Quantum Theory of Fields, Vol. I}.
	Cambridge University Press.
	
	\bibitem{albrecht1999}
	Albrecht, A. \& Magueijo, J. (1999).
	\textit{A time varying speed of light as a solution to cosmological puzzles}.
	Phys. Rev. D 59, 043516.
	
	\bibitem{alice2023}
	ALICE Collaboration (2023).
	\textit{Recent results from ALICE}.
	CERN-EP-2023-XXX.
	
	\bibitem{analog_optical}
	Smith, J. et al. (2024).
	\textit{Analog optical computing systems}.
	Nature Photonics.
	
	\bibitem{ashtekar2004}
	Ashtekar, A. \& Lewandowski, J. (2004).
	\textit{Background independent quantum gravity}.
	Class. Quantum Grav. 21, R53.
	
	\bibitem{atlas2023}
	ATLAS Collaboration (2023).
	\textit{ATLAS physics results}.
	CERN-PH-EP-2023-XXX.
	
	\bibitem{atlas2023higgs}
	ATLAS Collaboration (2023).
	\textit{Higgs boson measurements}.
	Phys. Rev. Lett.
	
	\bibitem{barbour1999}
	Barbour, J. (1999).
	\textit{The End of Time}.
	Oxford University Press.
	
	\bibitem{barrow1999}
	Barrow, J. D. (1999).
	\textit{Cosmologies with varying light speed}.
	Phys. Rev. D 59, 043515.
	
	\bibitem{becker2007}
	Becker, K. et al. (2007).
	\textit{String Theory and M-Theory}.
	Cambridge University Press.
	
	\bibitem{bell_muon}
	Bennett, G. W., et al. (Muon g-2 Collaboration) (2006).
	\textit{Final report of the E821 muon anomalous magnetic moment measurement}.
	Phys. Rev. D 73, 072003.
	
	\bibitem{bondi1948}
	Bondi, H. \& Gold, T. (1948).
	\textit{The steady-state theory of the expanding universe}.
	Mon. Not. R. Astron. Soc. 108, 252--270.
	
	\bibitem{brewer2019}
	Brewer, S. M. et al. (2019).
	\textit{Al+ Quantum-Logic Clock with Systematic Uncertainty below $10^{-18}$}.
	Phys. Rev. Lett. 123, 033201.
	
	\bibitem{cms2023top}
	CMS Collaboration (2023).
	\textit{Top quark measurements at CMS}.
	JHEP 2023.
	
	\bibitem{cms2024}
	CMS Collaboration (2024).
	\textit{CMS physics results 2024}.
	CERN-PH-EP-2024-XXX.
	
	\bibitem{codata2019}
	Tiesinga, E. et al. (2019).
	\textit{The 2018 CODATA Recommended Values}.
	J. Phys. Chem. Ref. Data.
	
	\bibitem{desi2025}
	DESI Collaboration (2025).
	\textit{DESI 2025 Cosmology Results}.
	arXiv preprint.
	
	\bibitem{differential_optical}
	Wang, X. et al. (2024).
	\textit{Differential optical computing}.
	Optica.
	
	\bibitem{dingle1972}
	Dingle, H. (1972).
	\textit{Science at the Crossroads}.
	Martin Brian \& O'Keeffe.
	
	\bibitem{divalentino2021}
	Di Valentino, E. et al. (2021).
	\textit{In the realm of the Hubble tension}.
	Class. Quantum Grav. 38, 153001.
	
	\bibitem{elnaschie2004}
	El Naschie, M. S. (2004).
	\textit{A review of E infinity theory}.
	Chaos, Solitons \& Fractals, 19, 209--236.
	
	\bibitem{fabrication_heterogeneous}
	Chen, Y. et al. (2024).
	\textit{Heterogeneous photonic integration}.
	Nature Electronics.
	
	\bibitem{fermilab2023}
	Fermilab (2023).
	\textit{Muon g-2 results}.
	Phys. Rev. Lett.
	
	\bibitem{flexible_wafer}
	Kim, S. et al. (2024).
	\textit{Flexible wafer-scale photonics}.
	Science Advances.
	
	\bibitem{francesco1997}
	Di Francesco, P. et al. (1997).
	\textit{Conformal Field Theory}.
	Springer.
	
	\bibitem{hartree1957}
	Hartree, D. R. (1957).
	\textit{The Calculation of Atomic Structures}.
	Wiley.
	
	\bibitem{hhi_6g}
	Fraunhofer HHI (2024).
	\textit{6G Photonic Integration}.
	Technical Report.
	
	\bibitem{hossenfelder2025}
	Hossenfelder, S. (2025).
	\textit{Science without the gobbledygook}.
	YouTube/Blog.
	
	\bibitem{hossenfelder_single_clock_video}
	Hossenfelder, S. (2024).
	\textit{The Single Clock Problem}.
	YouTube.
	
	\bibitem{hoyle1948}
	Hoyle, F. (1948).
	\textit{A new model for the expanding universe}.
	Mon. Not. R. Astron. Soc. 108, 372--382.
	
	\bibitem{integration_microelectronic}
	Liu, A. et al. (2024).
	\textit{Microelectronic photonic integration}.
	IEEE Journal.
	
	\bibitem{jacobson1995}
	Jacobson, T. (1995).
	\textit{Thermodynamics of spacetime}.
	Phys. Rev. Lett. 75, 1260.
	
	\bibitem{kasevich2023}
	Kasevich, M. et al. (2023).
	\textit{Atom interferometry tests}.
	Nature Physics.
	
	\bibitem{lerner2014}
	Lerner, E. J. (2014).
	\textit{An open letter on cosmology}.
	New Scientist.
	
	\bibitem{lisa2017}
	LISA Consortium (2017).
	\textit{Laser Interferometer Space Antenna}.
	ESA Technical Report.
	
	\bibitem{lithium_tantalate}
	Zhang, M. et al. (2024).
	\textit{Thin-film lithium tantalate photonics}.
	Nature Photonics.
	
	\bibitem{lopez2010}
	Lopez-Corredoira, M. (2010).
	\textit{Tests and problems of the standard model in cosmology}.
	Int. J. Mod. Phys. D.
	
	\bibitem{ludlow2015}
	Ludlow, A. D. et al. (2015).
	\textit{Optical atomic clocks}.
	Rev. Mod. Phys. 87, 637.
	
	\bibitem{mach1883}
	Mach, E. (1883).
	\textit{Die Mechanik in ihrer Entwickelung}.
	F.A. Brockhaus.
	
	\bibitem{maldacena1998}
	Maldacena, J. (1998).
	\textit{The large N limit of superconformal field theories}.
	Adv. Theor. Math. Phys. 2, 231--252.
	
	\bibitem{mueller2014}
	Müller, H. et al. (2014).
	\textit{Atom interferometry tests of the gravitational redshift}.
	Phys. Rev. Lett.
	
	\bibitem{mug2_final_2025}
	Muon g-2 Collaboration (2025).
	\textit{Final muon g-2 measurement}.
	Phys. Rev. Lett.
	
	\bibitem{muong2_2023}
	Muon g-2 Collaboration (2023).
	\textit{Updated muon g-2 results}.
	Phys. Rev. Lett.
	
	\bibitem{nathan2024}
	Nathan, A. et al. (2024).
	\textit{Quantum computing advances}.
	Nature.
	
	\bibitem{newell2018}
	Newell, D. B. et al. (2018).
	\textit{The CODATA 2017 values of h, e, k, and $N_A$}.
	Metrologia 55, L13.
	
	\bibitem{nottale1993}
	Nottale, L. (1993).
	\textit{Fractal Space-Time and Microphysics}.
	World Scientific.
	
	\bibitem{on_chip_lithium}
	Wang, C. et al. (2024).
	\textit{On-chip lithium niobate photonics}.
	Nature Communications.
	
	\bibitem{optical_advantages}
	Shastri, B. J. et al. (2024).
	\textit{Advantages of optical computing}.
	Nature Reviews Physics.
	
	\bibitem{pascher2025cmb}
	Pascher, J. (2025).
	\textit{T0-Theory: CMB Analysis}.
	Unpublished manuscript, HTL Leonding.
	
	\bibitem{pascher2025g2}
	Pascher, J. (2025).
	\textit{T0-Theory: g-2 Predictions}.
	Unpublished manuscript, HTL Leonding.
	
	\bibitem{pascher2025qm}
	Pascher, J. (2025).
	\textit{T0-Theory: Quantum Mechanics}.
	Unpublished manuscript, HTL Leonding.
	
	\bibitem{pascher2025si}
	Pascher, J. (2025).
	\textit{T0-Theory: SI Unit System}.
	Unpublished manuscript, HTL Leonding.
	
	\bibitem{pascher2025t0}
	Pascher, J. (2025).
	\textit{T0-Theory: Complete Framework}.
	Unpublished manuscript, HTL Leonding.
	
	\bibitem{pascher:fundamentals}
	Pascher, J. (2024).
	\textit{T0-Theory: Fundamentals}.
	Unpublished manuscript, HTL Leonding.
	
	\bibitem{pascher:g2_rev9}
	Pascher, J. (2024).
	\textit{T0-Theory: g-2 Revision 9}.
	Unpublished manuscript, HTL Leonding.
	
	\bibitem{pascher:geometric_formalism}
	Pascher, J. (2024).
	\textit{T0-Theory: Geometric Formalism}.
	Unpublished manuscript, HTL Leonding.
	
	\bibitem{pascher:ml_addendum}
	Pascher, J. (2024).
	\textit{T0-Theory: Machine Learning Addendum}.
	Unpublished manuscript, HTL Leonding.
	
	\bibitem{pascher:t0_foundations}
	Pascher, J. (2024).
	\textit{T0-Theory: Foundations}.
	Unpublished manuscript, HTL Leonding.
	
	\bibitem{pascher_derivation_beta_2025}
	Pascher, J. (2025).
	\textit{T0-Theory: Derivation of Beta}.
	Unpublished manuscript, HTL Leonding.
	
	\bibitem{pascher_higgs_connection_2025}
	Pascher, J. (2025).
	\textit{T0-Theory: Higgs Connection}.
	Unpublished manuscript, HTL Leonding.
	
	\bibitem{pascher_lagrangian_extended_2025}
	Pascher, J. (2025).
	\textit{T0-Theory: Extended Lagrangian}.
	Unpublished manuscript, HTL Leonding.
	
	\bibitem{pascher_mathematical_structure_2025}
	Pascher, J. (2025).
	\textit{T0-Theory: Mathematical Structure}.
	Unpublished manuscript, HTL Leonding.
	
	\bibitem{pascher_t0_cmb_2025}
	Pascher, J. (2025).
	\textit{T0-Theory: CMB Predictions}.
	Unpublished manuscript, HTL Leonding.
	
	\bibitem{pascher_t0_energie_2025}
	Pascher, J. (2025).
	\textit{T0-Theory: Energy}.
	Unpublished manuscript, HTL Leonding.
	
	\bibitem{pascher_t0_energy_2025}
	Pascher, J. (2025).
	\textit{T0-Theory: Energy Framework}.
	Unpublished manuscript, HTL Leonding.
	
	\bibitem{pascher_t0_theory_2025}
	Pascher, J. (2025).
	\textit{T0-Theory: Complete Theory}.
	Unpublished manuscript, HTL Leonding.
	
	\bibitem{penrose1959}
	Penrose, R. (1959).
	\textit{The apparent shape of a relativistically moving sphere}.
	Proc. Cambridge Phil. Soc. 55, 137--139.
	
	\bibitem{penrose1967}
	Penrose, R. (1967).
	\textit{Twistor algebra}.
	J. Math. Phys. 8, 345--366.
	
	\bibitem{peratt1992}
	Peratt, A. L. (1992).
	\textit{Physics of the Plasma Universe}.
	Springer-Verlag.
	
	\bibitem{peskin1995}
	Peskin, M. E. \& Schroeder, D. V. (1995).
	\textit{An Introduction to Quantum Field Theory}.
	Addison-Wesley.
	
	\bibitem{peskin_schroeder_1995}
	Peskin, M. E. \& Schroeder, D. V. (1995).
	\textit{An Introduction to Quantum Field Theory}.
	Addison-Wesley.
	
	\bibitem{phoquant}
	PhoQuant (2024).
	\textit{Photonic quantum computing}.
	Technical Report.
	
	\bibitem{photonics_ai}
	Wetzstein, G. et al. (2024).
	\textit{Photonics for AI}.
	Nature.
	
	\bibitem{planck1906}
	Planck, M. (1906).
	\textit{The Theory of Heat Radiation}.
	Johann Ambrosius Barth.
	
	\bibitem{planck2018}
	Planck Collaboration (2018).
	\textit{Planck 2018 results}.
	A\&A 641, A6.
	
	\bibitem{polchinski1998}
	Polchinski, J. (1998).
	\textit{String Theory}.
	Cambridge University Press.
	
	\bibitem{qant_nps}
	QANT (2024).
	\textit{Quantum photonics systems}.
	Technical Report.
	
	\bibitem{quantenjahr25}
	Quantenjahr (2025).
	\textit{International Year of Quantum}.
	UNESCO.
	
	\bibitem{recurrent_photonics}
	Tait, A. N. et al. (2024).
	\textit{Recurrent photonic neural networks}.
	Optica.
	
	\bibitem{rf_photonics}
	Capmany, J. \& Novak, D. (2024).
	\textit{Microwave photonics}.
	Nature Photonics.
	
	\bibitem{riess2019}
	Riess, A. G. et al. (2019).
	\textit{Large Magellanic Cloud Cepheid Standards}.
	ApJ 876, 85.
	
	\bibitem{riess2022}
	Riess, A. G. et al. (2022).
	\textit{A Comprehensive Measurement of H0}.
	ApJ 934, L7.
	
	\bibitem{rovelli2004}
	Rovelli, C. (2004).
	\textit{Quantum Gravity}.
	Cambridge University Press.
	
	\bibitem{sciama1953}
	Sciama, D. W. (1953).
	\textit{On the origin of inertia}.
	Mon. Not. R. Astron. Soc. 113, 34--42.
	
	\bibitem{sciencedaily2025}
	ScienceDaily (2025).
	\textit{Physics news}.
	Online.
	
	\bibitem{sm_g2_2025}
	Aoyama, T. et al. (2025).
	\textit{Standard Model prediction for g-2}.
	Phys. Rep.
	
	\bibitem{susskind1995}
	Susskind, L. (1995).
	\textit{The world as a hologram}.
	J. Math. Phys. 36, 6377--6396.
	
	\bibitem{t0_kosmologie}
	Pascher, J. (2024).
	\textit{T0-Theory: Cosmology}.
	Unpublished manuscript, HTL Leonding.
	
	\bibitem{terrell1959}
	Terrell, J. (1959).
	\textit{Invisibility of the Lorentz contraction}.
	Phys. Rev. 116, 1041--1045.
	
	\bibitem{terrell_single_clock_nature_2024}
	Terrell, J. et al. (2024).
	\textit{Single clock precision measurements}.
	Nature Physics.
	
	\bibitem{tfln_foundry}
	TFLN Foundry (2024).
	\textit{Thin-film lithium niobate foundry services}.
	Technical Specifications.
	
	\bibitem{thiemann2007}
	Thiemann, T. (2007).
	\textit{Modern Canonical Quantum General Relativity}.
	Cambridge University Press.
	
	\bibitem{thz_epfl}
	EPFL (2024).
	\textit{Terahertz photonics research}.
	Technical Report.
	
	\bibitem{unnikrishnan2004}
	Unnikrishnan, C. S. (2004).
	\textit{On Einstein's resolution of the twin clock paradox}.
	Current Science, 86, 704--709.
	
	\bibitem{verlinde2011}
	Verlinde, E. (2011).
	\textit{On the origin of gravity and the laws of Newton}.
	JHEP 2011, 29.
	
	\bibitem{video2025}
	Video (2025).
	\textit{Physics video explanation}.
	YouTube.
	
	\bibitem{weinberg1995}
	Weinberg, S. (1995).
	\textit{The Quantum Theory of Fields}.
	Cambridge University Press.
	
	\bibitem{weiskopf2000}
	Weiskopf, D. (2000).
	\textit{Visualization of special relativity}.
	PhD thesis, University of Tübingen.
	
	\bibitem{wheeler1990}
	Wheeler, J. A. (1990).
	\textit{A Journey into Gravity and Spacetime}.
	Scientific American Library.
	
	\bibitem{wiki_bell}
	Wikipedia (2024).
	\textit{Bell's theorem}.
	Online encyclopedia.
	
	\bibitem{zwicky1929}
	Zwicky, F. (1929).
	\textit{On the red shift of spectral lines through interstellar space}.
	Proc. Natl. Acad. Sci. 15, 773--779.

\end{thebibliography}


\end{document}
