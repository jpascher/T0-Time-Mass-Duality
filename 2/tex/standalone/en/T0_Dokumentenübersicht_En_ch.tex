\documentclass[11pt,a4paper]{article}
\usepackage[a4paper,margin=2cm]{geometry}
\usepackage[utf8]{inputenc}
\usepackage[english]{babel}
\usepackage{lmodern}
\renewcommand{\familydefault}{\sfdefault}

\usepackage{amsmath,amssymb,amsthm}
\usepackage{graphicx}
\usepackage[unicode,pdfencoding=auto,hypertexnames=false]{hyperref}
\usepackage{booktabs}
\usepackage{longtable}
\usepackage{array}
\usepackage{siunitx}
\usepackage{fancyhdr}
\usepackage{float}
\usepackage{tikz}
% tcolorbox removed for standalone
% tcbset removed
\tikzset{
  t0blue/.style={draw=blue,fill=blue!10},
  t0red/.style={draw=red,fill=red!10},
  t0green/.style={draw=green!50!black,fill=green!10},
  t0orange/.style={draw=orange,fill=orange!10},
}
\usepackage{setspace}
\usepackage{enumitem}
\usepackage{adjustbox}
\usepackage{xcolor}

% Define colors for xcolor package
\definecolor{t0green}{RGB}{34,139,34}
\definecolor{t0blue}{RGB}{0,0,255}
\definecolor{t0red}{RGB}{255,0,0}
\definecolor{t0orange}{RGB}{255,165,0}

% Define custom column types for tables
\newcolumntype{L}[1]{>{\raggedright\arraybackslash}p{#1}}
\newcolumntype{C}[1]{>{\centering\arraybackslash}p{#1}}
\newcolumntype{R}[1]{>{\raggedleft\arraybackslash}p{#1}}

\setlength{\parindent}{0pt}
\setlength{\parskip}{6pt}

\hypersetup{
  colorlinks=true,
  linkcolor=blue,
  citecolor=blue,
  urlcolor=blue
}
\pagestyle{fancy}
\setlength{\headheight}{28pt}

\newcommand{\checkmarkx}{\checkmark}
\newcommand{\warningx}{\textbf{!}}

% Makros aus Einzel-Dokumenten (Fallback-Definitionen)
\newcommand{\mytimes}{\times}
\newcommand{\myapprox}{\approx}
\newcommand{\mysim}{\sim}
\newcommand{\myomega}{\omega}
\newcommand{\mypi}{\pi}
\newcommand{\myrightarrow}{\rightarrow}
\newcommand{\mypropto}{\propto}
\newcommand{\deltafield}{\delta\phi}
\newcommand{\xipar}{\xi}
\newcommand{\xiT}{\xi}
\newcommand{\lambdah}{\lambda_h}

% Additional macros used in chapter files
\newcommand{\Kfrak}{K_{\text{frak}}}  % Fractal correction factor
\newcommand{\Dfrak}{D_f}              % Fractal dimension
\newcommand{\betapar}{\beta}          % T0 beta parameter
\newcommand{\alphapar}{\alpha}        % T0 alpha parameter
\newcommand{\Efield}{E}               % Energy field
% Note: checkmarkxa/warningxa are variants used in auto-generated chapter files
\newcommand{\checkmarkxa}{\checkmark}
\newcommand{\warningxa}{\textbf{!}}

% Additional T0-specific macros
\newcommand{\xigeom}{\xi_{\text{geom}}}  % Geometric xi
\newcommand{\lP}{\ell_P}                  % Planck length
\newcommand{\rzero}{r_0}                  % Characteristic radius
\newcommand{\xirat}{\xi_{\text{rat}}}     % Xi ratio
\newcommand{\tzero}{t_0}                  % Characteristic time
\newcommand{\natunits}{\text{(nat. units)}}  % Natural units annotation
\newcommand{\myRightarrow}{\Rightarrow}   % Arrow variant
\newcommand{\Lag}{\mathcal{L}}            % Lagrangian

% Physics macros used in chapter files
\newcommand{\CQCD}{C_{\text{QCD}}}        % QCD correction
\newcommand{\EP}{E_P}                     % Planck energy
\newcommand{\Ee}{E_e}                     % Electron energy
\newcommand{\Emu}{E_\mu}                  % Muon energy
\newcommand{\Exi}{E_\xi}                  % Xi energy
\newcommand{\Ezero}{E_0}                  % Characteristic energy
\newcommand{\Hubble}{H}                   % Hubble constant
\newcommand{\Kspec}{K_{\text{spec}}}      % Spectral correction
\newcommand{\Lambdat}{\Lambda_t}          % Time-related cosmological constant
\newcommand{\Leff}{\mathcal{L}_{\text{eff}}}  % Effective Lagrangian
\newcommand{\Lorentz}{\mathcal{L}}        % Lorentz symbol
\newcommand{\Lxi}{L_\xi}                  % Xi length
\newcommand{\Tfield}{T}                   % Time field
\newcommand{\Weyl}{W}                     % Weyl tensor/symbol
\newcommand{\alphaEMSI}{\alpha_{\text{EM,SI}}}  % EM alpha in SI
\newcommand{\alphaEMnat}{\alpha_{\text{EM,nat}}}  % EM alpha in natural units
\newcommand{\alphaem}{\alpha_{\text{em}}} % Electromagnetic alpha
\newcommand{\betaTSI}{\beta_{T,\text{SI}}}  % Beta in SI
\newcommand{\betaTnat}{\beta_{T,\text{nat}}}  % Beta in natural units
\newcommand{\deltam}{\delta m}            % Mass difference
\newcommand{\phiT}{\phi_T}                % T-field phi
\newcommand{\tP}{t_P}                     % Planck time
\newcommand{\rhoCMB}{\rho_{\text{CMB}}}   % CMB density
\newcommand{\rhoCasimir}{\rho_{\text{Casimir}}}  % Casimir density

% Table formatting
\usepackage{multirow}

% Additional physics macros
\newcommand{\Riem}{\mathcal{R}}           % Riemann tensor
\newcommand{\ZPinch}{Z_{\text{pinch}}}    % Z-pinch
\newcommand{\SynchPower}{P_{\text{synch}}} % Synchrotron power
\newcommand{\Rzero}{R_0}                  % Characteristic radius
\newcommand{\alphafine}{\alpha}           % Fine structure constant
\newcommand{\Etau}{E_\tau}                % Tau energy
\newcommand{\deltaE}{\delta E}            % Energy deviation
\newcommand{\EPlanck}{E_P}                % Planck energy
\newcommand{\pichar}{\pi}                 % Pi character
\newcommand{\alphaWSI}{\alpha_{W,\text{SI}}}  % Wien alpha in SI
\newcommand{\alphaWnat}{\alpha_{W,\text{nat}}}  % Wien alpha in natural units

% Einfache abstract-Umgebung für Kapitel:
\newenvironment{abstract}{%
  \begin{center}\bfseries Abstract\end{center}\small
}{\par}


\title{T0 Dokumentenübersicht En}
\author{J. Pascher}
\date{\today}

\begin{document}
\maketitle

\section*{T0 Dokumentenübersicht (T0 Dokumentenübersicht)}

	\begin{abstract}
		This overview presents the complete T0-theory series consisting of 8 fundamental documents that represent a revolutionary geometric reformulation of physics. Based on a single parameter $\xipar = \frac{4}{3} \times 10^{-4}$, all fundamental constants, particle masses, and physical phenomena from quantum mechanics to cosmology are uniformly described. The theory achieves over 99\% accuracy in predicting experimental values without free parameters and offers testable predictions for future experiments.
	\end{abstract}
	
	
	\section{The T0 Revolution: A Paradigm Shift}
	
\section*{Overview}
\section*{What is the T0-Theory?}
		
		The T0-Theory is a fundamental reformulation of physics that derives all known physical phenomena from the geometric structure of three-dimensional space. At its center is a single universal parameter:
		
		\begin{equation}
			\boxed{\xipar = \frac{4}{3} \times 10^{-4} = 1.333333... \times 10^{-4}}
		\end{equation}
		
\section*{Revolutionary Reduction:}
		\begin{itemize}
			\item \textbf{Standard Model + Cosmology:} $>25$ free parameters
			\item \textbf{T0-Theory:} 1 geometric parameter
			\item \textbf{Parameter Reduction:} 96\%!
		\end{itemize}
		
		\textbf{Field of Application:} From particle masses to fundamental constants and cosmological structures
% end box overview
	
	\section{Document Series: Systematic Structure}
	
	\subsection{Hierarchical Structure of the 8 Documents}
	
	The T0-document series follows a logical progression from fundamental principles to specific applications:
	
	\begin{center}
		\begin{tikzpicture}[node distance=2cm, auto]
			\tikzstyle{doc} = [rectangle, rounded corners, minimum width=3cm, minimum height=1cm, text centered, draw=t0blue, fill=t0blue!20]
			\tikzstyle{arrow} = [thick,->]
			
			\node [doc] (doc1) {\textbf{1. Foundations}};
			\node [doc, below of=doc1] (doc2) {\textbf{2. Fine Structure}};
			\node [doc, below of=doc2] (doc3) {\textbf{3. Gravitation}};
			\node [doc, below of=doc3] (doc4) {\textbf{4. Particle Masses}};
			\node [doc, right of=doc4, xshift=2cm] (doc5) {\textbf{5. Neutrinos}};
			\node [doc, above of=doc5] (doc6) {\textbf{6. Cosmology}};
			\node [doc, above of=doc6] (doc7) {\textbf{7. g-2 Anomalies}};
			\node [doc, below of=doc7, yshift=-1cm] (doc8) {\textbf{8. QM-QFT-RT}};
			
			\draw [arrow] (doc1) -- (doc2);
			\draw [arrow] (doc2) -- (doc3);
			\draw [arrow] (doc3) -- (doc4);
			\draw [arrow] (doc4) -- (doc5);
			\draw [arrow] (doc4) -- (doc6);
			\draw [arrow] (doc4) -- (doc7);
			\draw [arrow] (doc7) -- (doc8);
		\end{tikzpicture}
	\end{center}
	
	\section{Document 1: T0.pdf}
	
\section*{Document}
		\textbf{Subtitle:} The Geometric Foundations of Physics
		
\section*{Central Contents:}
		\begin{itemize}
			\item \textbf{Fundamental Parameter:} $\xipar = \frac{4}{3} \times 10^{-4}$ as geometric constant
			\item \textbf{Time-Mass Duality:} $T \cdot m = 1$ in natural units
			\item \textbf{Fractal Spacetime Structure:} $D_f = 2.94$ and $K_{\text{frak}} = 0.986$
			\item \textbf{Levels of Interpretation:} Harmonic, geometric, field-theoretic
			\item \textbf{Universal Formula Structure:} Template for all T0 relations
		\end{itemize}
		
\section*{Fundamental Insights:}
		\begin{itemize}
			\item Tetrahedral packing as space base structure
			\item Quantum field theoretic derivation of $10^{-4}$
			\item Characteristic energy scales: $E_0 = 7.398$ MeV
			\item Philosophical implications of geometric physics
		\end{itemize}
		
		\textbf{Status:} Theoretical foundation - fully established
% end box documentbox
	
	\section{Document 2: T0.pdf}
	
\section*{Document}
		\textbf{Subtitle:} Derivation of $\alpha$ from Geometric Principles
		
\section*{Central Formula:}
		\begin{equation}
			\boxed{\alpha = \xipar \cdot \left(\frac{E_0}{1\,\text{MeV}}\right)^2}
		\end{equation}
		
\section*{Key Results:}
		\begin{itemize}
			\item \textbf{T0 Prediction:} $\alpha^{-1} = 137.04$
			\item \textbf{Experiment:} $\alpha^{-1} = 137.036$
			\item \textbf{Deviation:} 0.003\% (excellent agreement)
		\end{itemize}
		
\section*{Theoretical Innovations:}
		\begin{itemize}
			\item Characteristic energy $E_0 = \sqrt{m_e \cdot m_\mu}$
			\item Logarithmic symmetry of lepton masses
			\item Fundamental dependence $\alpha \propto \xipar^{11/2}$
			\item Why numerical ratios must not be simplified
		\end{itemize}
		
		\textbf{Status:} Experimentally confirmed - excellent accuracy
% end box documentbox
	
	\section{Document 3: T0.pdf}
	
\section*{Document}
		\textbf{Subtitle:} Systematic Derivation of $G$ from Geometric Principles
		
\section*{Complete Formula:}
		\begin{equation}
			\boxed{G_{\text{SI}} = \frac{\xipar^2}{4 m_e} \times C_{\text{conv}} \times K_{\text{frak}}}
		\end{equation}
		
\section*{Conversion Factors:}
		\begin{itemize}
			\item \textbf{Dimensional Correction:} $C_1 = 3.521 \times 10^{-2}$ 
			\item \textbf{SI Conversion:} $C_{\text{conv}} = 7.783 \times 10^{-3}$
			\item \textbf{Fractal Correction:} $K_{\text{frak}} = 0.986$
		\end{itemize}
		
\section*{Experimental Verification:}
		\begin{itemize}
			\item \textbf{T0 Prediction:} $G = 6.67429 \times 10^{-11}$ m³/(kg·s²)
			\item \textbf{CODATA 2018:} $G = 6.67430 \times 10^{-11}$ m³/(kg·s²)
			\item \textbf{Deviation:} < 0.0002\% (extraordinary precision)
		\end{itemize}
		
		\textbf{Physical Meaning:} Gravitation as geometric spacetime-matter coupling
		
		\textbf{Status:} Experimentally confirmed - highest precision
% end box documentbox
	
	\section{Document 4: T0.pdf}
	
\section*{Document}
		\textbf{Subtitle:} Parameter-Free Calculation of All Fermion Masses
		
\section*{Two Equivalent Methods:}
		\begin{enumerate}
			\item \textbf{Direct Geometry:} $m_i = \frac{K_{\text{frak}}}{\xi_i} \times C_{\text{conv}}$
			\item \textbf{Extended Yukawa:} $m_i = y_i \times v$ with $y_i = r_i \times \xipar^{p_i}$
		\end{enumerate}
		
		\textbf{Quantum Number System:} Each particle receives $(n,l,j)$-assignment
		
\section*{Experimental Successes:}
		\begin{center}
			\begin{tabular}{lcc}
				\toprule
				\textbf{Particle Class} & \textbf{Number} & \textbf{Avg. Accuracy} \\
				\midrule
				Charged Leptons & 3 & 98.3\% \\
				Up-type Quarks & 3 & 99.1\% \\
				Down-type Quarks & 3 & 98.8\% \\
				Bosons & 3 & 99.4\% \\
				\midrule
				\textbf{Total (established)} & \textbf{12} & \textbf{99.0\%} \\
				\bottomrule
			\end{tabular}
		\end{center}
		
		\textbf{Revolutionary Reduction:} From 15+ free mass parameters to 0!
		
		\textbf{Status:} Experimentally confirmed - systematic successes
% end box documentbox
	
	\section{Document 5: T0.pdf}
	
\section*{Document}
		\textbf{Subtitle:} The Photon Analogy and Geometric Oscillations
		
\section*{Special Treatment Required:}
		\begin{itemize}
			\item \textbf{Photon Analogy:} Neutrinos as "damped photons"
			\item \textbf{Double $\xi$-Suppression:} $m_\nu = \frac{\xipar^2}{2} \times m_e = 4.54$ meV
			\item \textbf{Geometric Oscillations:} Phases instead of mass differences
		\end{itemize}
		
\section*{T0 Predictions:}
		\begin{itemize}
			\item \textbf{Uniform Masses:} All flavors: $m_\nu = 4.54$ meV
			\item \textbf{Sum:} $\Sigma m_\nu = 13.6$ meV
			\item \textbf{Velocity:} $v_\nu = c(1 - \xipar^2/2)$
		\end{itemize}
		
\section*{Experimental Classification:}
		\begin{itemize}
			\item \textbf{Cosmological Limits:} $\Sigma m_\nu < 70$ meV $\checkmark$
			\item \textbf{KATRIN Experiment:} $m_\nu < 800$ meV $\checkmark$
			\item \textbf{Target Value Estimate:} $\sim 15$ meV (T0 at 30\%)
		\end{itemize}
		
		\textbf{Important Note:} Highly speculative - honest scientific limitation
		
		\textbf{Status:} Speculative - testable predictions, but unconfirmed
% end box documentbox
	
	\section{Document 6: T0.pdf}
	
\section*{Document}
		\textbf{Subtitle:} Static Universe and $\xi$-Field Manifestations
		
\section*{Revolutionary Cosmology:}
		\begin{itemize}
			\item \textbf{Static Universe:} No Big Bang, eternally existing
			\item \textbf{Time-Energy Duality:} Big Bang forbidden by $\Delta E \times \Delta t \geq \frac{\hbar}{2}$
			\item \textbf{CMB from $\xi$-Field:} Not from z=1100 decoupling
		\end{itemize}
		
\section*{Casimir-CMB Connection:}
		\begin{itemize}
			\item \textbf{Characteristic Length:} $L_\xi = 100$ $\mu$m
			\item \textbf{Theoretical Ratio:} $|\rho_{\text{Casimir}}|/\rho_{\text{CMB}} = 308$
			\item \textbf{Experimental:} 312 (98.7\% agreement)
		\end{itemize}
		
\section*{Alternative Redshift:}
		\begin{equation}
			z(\lambda_0, d) = \frac{\xipar \cdot d \cdot \lambda_0}{E_\xi}
		\end{equation}
		
\section*{Cosmological Problems Solved:}
		\begin{itemize}
			\item Horizon problem, flatness problem, monopole problem
			\item Hubble tension, age problem, dark energy
			\item Parameters: From 25+ to 1 ($\xipar$)
		\end{itemize}
		
		\textbf{Status:} Testable hypotheses - revolutionary alternative
% end box documentbox
	
	\section{Document 7: T0.pdf}
	
\section*{Document}
		\textbf{Subtitle:} Solution to the Muon g-2 Anomaly through Time Field Extension
		
\section*{The Muon g-2 Problem:}
		\begin{itemize}
			\item \textbf{Experimental Deviation:} $\Delta a_\mu = 251 \times 10^{-11}$ (4.2$\sigma$)
			\item \textbf{Largest Discrepancy:} Between theory and experiment in modern physics
		\end{itemize}
		
\section*{T0 Solution through Time Field:}
		\begin{equation}
			\boxed{\Delta a_\ell = 251 \times 10^{-11} \times \left(\frac{m_\ell}{m_\mu}\right)^2}
		\end{equation}
		
\section*{Universal Predictions:}
		\begin{center}
			\begin{tabular}{lccc}
				\toprule
				\textbf{Lepton} & \textbf{T0 Correction} & \textbf{Experiment} & \textbf{Status} \\
				\midrule
				Electron & $5.8 \times 10^{-15}$ & Agreement & $\checkmark$ \\
				Muon & $2.51 \times 10^{-9}$ & 4.2$\sigma$ Deviation & $\checkmark$ \\
				Tau & $7.11 \times 10^{-7}$ & Prediction & Test \\
				\bottomrule
			\end{tabular}
		\end{center}
		
		\textbf{Theoretical Basis:} Extended Lagrangian density with fundamental time field
		
		\textbf{Status:} Exact solution to current problem - Tau test pending
% end box documentbox
	
	\section{Document 8: T0-QFT-RT.pdf}
	
\section*{Document}
		\textbf{Subtitle:} Unification of QM, QFT, and RT from a Geometric Foundation
		
\section*{Central Contents:}
		\begin{itemize}
			\item \textbf{Universal T0 Field Equation:} $\square \Efield + \xipar \cdot \mathcal{F}[\Efield] = 0$ as basis for all theories
			\item \textbf{Time-Mass Duality:} $T \cdot m = 1$ connects all three pillars of physics
			\item \textbf{Emergent Quantum Properties:} QM as approximation of the energy field
			\item \textbf{Field Description:} All particles as excitations of a fundamental field $\Efield$
			\item \textbf{Renormalization Solution:} Natural cutoff through $\EP/\xipar$
			\item \textbf{Relativistic Extension:} Extended Einstein equations with $\Lambda_{\xipar}$
		\end{itemize}
		
\section*{Fundamental Insights:}
		\begin{itemize}
			\item Deterministic interpretation of quantum mechanics through local time field
			\item Wave-particle duality from field geometry
			\item Energy scales hierarchy: Planck to QCD through $\xipar$-corrections
			\item Gravitation as field curvature, dark energy as $\xipar^2 c^4 / G$
			\item Philosophical implications: Unity of physics through geometric principles
		\end{itemize}
		
		\textbf{Status:} Theoretical unification - builds on all previous documents, testable predictions
% end box documentbox
	
	\section{Scientific Achievements: Quantitative Summary}
	
\section*{Achievement}
\section*{Experimental Confirmations of the T0-Theory:}
		
		\begin{center}
			\begin{longtable}{lccc}
				\caption{Complete Success Statistics of T0 Predictions} \\
				\toprule
				\textbf{Physical Quantity} & \textbf{T0 Prediction} & \textbf{Experiment} & \textbf{Deviation} \\
				\midrule
				\endfirsthead
				\multicolumn{4}{c}{Continuation of the Table} \\
				\toprule
				\textbf{Physical Quantity} & \textbf{T0 Prediction} & \textbf{Experiment} & \textbf{Deviation} \\
				\midrule
				\endhead
				\bottomrule
				\endlastfoot
				
				\multicolumn{4}{l}{\textbf{Fundamental Constants}} \\
				\midrule
				$\alpha^{-1}$ & 137.04 & 137.036 & 0.003\% \\
				$G$ [$10^{-11}$ m³/(kg·s²)] & 6.67429 & 6.67430 & <0.0002\% \\
				\midrule
				
				\multicolumn{4}{l}{\textbf{Charged Leptons [MeV]}} \\
				\midrule
				$m_e$ & 0.504 & 0.511 & 1.4\% \\
				$m_\mu$ & 105.1 & 105.66 & 0.5\% \\
				$m_\tau$ & 1727.6 & 1776.86 & 2.8\% \\
				\midrule
				
				\multicolumn{4}{l}{\textbf{Quarks [MeV]}} \\
				\midrule
				$m_u$ & 2.27 & 2.2 & 3.2\% \\
				$m_d$ & 4.74 & 4.7 & 0.9\% \\
				$m_s$ & 98.5 & 93.4 & 5.5\% \\
				$m_c$ & 1284.1 & 1270 & 1.1\% \\
				$m_b$ & 4264.8 & 4180 & 2.0\% \\
				$m_t$ [GeV] & 171.97 & 172.76 & 0.5\% \\
				\midrule
				
				\multicolumn{4}{l}{\textbf{Bosons [GeV]}} \\
				\midrule
				$m_H$ & 124.8 & 125.1 & 0.2\% \\
				$m_W$ & 79.8 & 80.38 & 0.7\% \\
				$m_Z$ & 90.3 & 91.19 & 1.0\% \\
				\midrule
				
				\multicolumn{4}{l}{\textbf{Anomalous Magnetic Moments}} \\
				\midrule
				$\Delta a_\mu$ [$10^{-9}$] & 2.51 & 2.51$\pm$0.59 & Exact \\
				\midrule
				
				\multicolumn{4}{l}{\textbf{Cosmology}} \\
				\midrule
				Casimir/CMB Ratio & 308 & 312 & 1.3\% \\
				$L_\xi$ [$\mu$m] & 100 & (theoretical) & -- \\
			\end{longtable}
		\end{center}
		
\section*{Overall Statistics of Established Predictions:}
		\begin{itemize}
			\item \textbf{Number of Tested Quantities:} 16
			\item \textbf{Average Accuracy:} 99.1\%
			\item \textbf{Best Prediction:} Gravitational constant (<0.0002\%)
			\item \textbf{Systematic Successes:} All orders of magnitude correct
		\end{itemize}
% end box achievement
	
	\section{Theoretical Innovations}
	
\section*{Foundation}
\section*{Fundamental Breakthroughs of the T0-Theory:}
		
		\begin{enumerate}
			\item \textbf{Parameter Reduction:} From >25 to 1 parameter (96\% reduction)
			
			\item \textbf{Geometric Unification:} All physics from 3D space structure
			
			\item \textbf{Fractal Quantum Spacetime:} Systematic consideration of $K_{\text{frak}} = 0.986$
			
			\item \textbf{Time-Mass Duality:} $T \cdot m = 1$ as fundamental principle
			
			\item \textbf{Harmonic Physics:} $\frac{4}{3}$ as universal geometric constant
			
			\item \textbf{Quantum Number System:} $(n,l,j)$-assignment for all particles
			
			\item \textbf{Two Equivalent Methods:} Direct geometry $\leftrightarrow$ Extended Yukawa
			
			\item \textbf{Experimental Precision:} >99\% without parameter adjustment
			
			\item \textbf{Cosmological Revolution:} Static universe without Big Bang
			
			\item \textbf{Testable Predictions:} Specific, falsifiable hypotheses
		\end{enumerate}
% end box foundation
	
	\section{Comparison with Established Theories}
	
	\begin{center}
		\begin{longtable}{lccc}
			\caption{T0-Theory vs. Standard Approaches} \\
			\toprule
			\textbf{Aspect} & \textbf{Standard Model} & \textbf{$\Lambda$CDM} & \textbf{T0-Theory} \\
			\midrule
			\endfirsthead
			\multicolumn{4}{c}{Continuation of the Table} \\
			\toprule
			\textbf{Aspect} & \textbf{Standard Model} & \textbf{$\Lambda$CDM} & \textbf{T0-Theory} \\
			\midrule
			\endhead
			\bottomrule
			\endlastfoot
			
			Free Parameters & 19+ & 6 & 1 \\
			Theoretical Basis & Empirical & Empirical & Geometric \\
			Particle Masses & Arbitrary & -- & Calculable \\
			Constants & Experimental & Experimental & Derived \\
			Predictive Power & None & Limited & Comprehensive \\
			Dark Matter & New Particles & 26\% unknown & $\xi$-Field \\
			Dark Energy & -- & 69\% unknown & Not Required \\
			Big Bang & -- & Required & Physically Impossible \\
			Hierarchy Problem & Unsolved & -- & Solved by $\xi$ \\
			Fine-Tuning & $>$20 Parameters & Cosmological & None \\
			Experimental Tests & Confirmed & Confirmed & 99\% Accuracy \\
			New Predictions & None & Few & Many Testable \\
		\end{longtable}
	\end{center}
	
	\section{Summary: The T0 Revolution}
	
\section*{Overview}
\section*{What the T0-Theory Has Achieved:}
		
\section*{1. Scientific Successes:}
		\begin{itemize}
			\item 99.1\% average accuracy for 16 tested quantities
			\item Solution to the muon g-2 anomaly with exact prediction
			\item Parameter reduction from >25 to 1 (96\% reduction)
			\item Unified description from particle physics to cosmology
		\end{itemize}
		
\section*{2. Theoretical Innovations:}
		\begin{itemize}
			\item Geometric derivation of all fundamental constants
			\item Fractal spacetime structure as quantum corrections
			\item Time-mass duality as fundamental principle
			\item Alternative cosmology without Big Bang problems
		\end{itemize}
		
\section*{3. Experimental Predictions:}
		\begin{itemize}
			\item Specific, testable hypotheses for all areas
			\item Neutrino masses, cosmological parameters, g-2 anomalies
			\item New phenomena at characteristic $\xi$-scales
		\end{itemize}
		
\section*{4. Paradigm Shift:}
		\begin{itemize}
			\item From empirical adjustment to geometric derivation
			\item From many parameters to universal constant
			\item From fragmented theories to unified framework
		\end{itemize}
% end box overview
	
	
	\section{Philosophical and Philosophy of Science Significance}
	
\section*{Foundation}
\section*{Paradigm Shift through the T0-Theory:}
		
\section*{1. From Complexity to Simplicity:}
		\begin{itemize}
			\item \textbf{Standard Approach:} Many parameters, complex structures
			\item \textbf{T0 Approach:} One parameter, elegant geometry
			\item \textbf{Philosophy:} "Simplex veri sigillum" (Simplicity as the seal of truth)
		\end{itemize}
		
\section*{2. From Empiricism to Rationalism:}
		\begin{itemize}
			\item \textbf{Standard Approach:} Experimental adjustment of parameters
			\item \textbf{T0 Approach:} Mathematical derivation from principles
			\item \textbf{Philosophy:} Geometric order as foundation of reality
		\end{itemize}
		
\section*{3. From Fragmentation to Unification:}
		\begin{itemize}
			\item \textbf{Standard Approach:} Separate theories for different areas
			\item \textbf{T0 Approach:} Unified framework from quantum to cosmos
			\item \textbf{Philosophy:} Universal harmony of natural laws
		\end{itemize}
		
\section*{4. From Stasis to Dynamics:}
		\begin{itemize}
			\item \textbf{Standard Approach:} Constants taken as given
			\item \textbf{T0 Approach:} Constants understood from geometric principles
			\item \textbf{Philosophy:} Understanding rather than mere description
		\end{itemize}
% end box foundation
	
	\section{Limits and Challenges}
	
	\subsection{Known Limitations}
	
	\begin{itemize}
		\item \textbf{Neutrino Sector:} Highly speculative, experimentally unconfirmed
		\item \textbf{QCD Renormalization:} Not fully integrated into T0 framework
		\item \textbf{Electroweak Symmetry Breaking:} Geometric derivation incomplete
		\item \textbf{Supersymmetry:} T0 predictions for superpartners missing
		\item \textbf{Quantum Gravity:} Complete QFT formulation pending
	\end{itemize}
	
	\subsection{Theoretical Challenges}
	
	\begin{itemize}
		\item \textbf{Renormalization:} Systematic treatment of divergences
		\item \textbf{Symmetries:} Connection to known gauge symmetries
		\item \textbf{Quantization:} Complete quantum field theory of the $\xi$-field
		\item \textbf{Mathematical Rigor:} Proofs instead of plausible arguments
		\item \textbf{Cosmological Details:} Structure formation without Big Bang
	\end{itemize}
	
	\subsection{Experimental Challenges}
	
	\begin{itemize}
		\item \textbf{Precision Measurements:} Many tests at accuracy limits
		\item \textbf{New Phenomena:} Characteristic $\xi$-scales hard to access
		\item \textbf{Cosmological Tests:} Observation times of decades
		\item \textbf{Technological Limits:} Some predictions beyond current capabilities
	\end{itemize}
	
	\section{Future Developments}
	
	\subsection{Theoretical Priorities}
	
	\begin{enumerate}
		\item \textbf{Complete QFT:} Quantum field theory of the $\xi$-field
		\item \textbf{Unification:} Integration of all four fundamental forces
		\item \textbf{Mathematical Foundation:} Rigorous proofs of geometric relations
		\item \textbf{Cosmological Elaboration:} Detailed alternative to the standard model
		\item \textbf{Phenomenology:} Systematic derivation of all observable effects
	\end{enumerate}
	
	
	
	\section{The Significance for the Future of Physics}
	
\section*{Foundation}
\section*{Why the T0-Theory is Revolutionary:}
		
		The T0-Theory is not just a new theory, but a fundamental paradigm shift in our understanding of nature:
		
\section*{1. Ontological Revolution:}
		\begin{itemize}
			\item Nature is not complex, but elegantly simple
			\item Geometry is fundamental, particles are derived
			\item The universe follows harmonic, not chaotic principles
		\end{itemize}
		
\section*{2. Epistemological Revolution:}
		\begin{itemize}
			\item Understanding rather than mere description becomes possible again
			\item Mathematical beauty becomes the criterion of truth
			\item Deduction complements induction as a scientific method
		\end{itemize}
		
\section*{3. Methodological Revolution:}
		\begin{itemize}
			\item From "theory of everything" to "formula for everything"
			\item Geometric intuition becomes a method of discovery
			\item Unity rather than diversity becomes the research principle
		\end{itemize}
		
\section*{4. Technological Revolutions:}
		\begin{itemize}
			\item $\xi$-field manipulation for energy generation
			\item Geometric control over fundamental interactions
			\item New materials based on $\xi$-harmonies
		\end{itemize}
% end box foundation
	
	\section{Conclusion}
	
	The T0-Theory, documented in these 8 systematic works, presents a revolutionary alternative to the current understanding of physics. With a single geometric parameter $\xipar = \frac{4}{3} \times 10^{-4}$, all fundamental constants, particle masses, and physical phenomena from the quantum level to the cosmological scale are uniformly described.
	
	The experimental successes with over 99\% average accuracy, the solution to the muon g-2 anomaly, and the systematic reduction of over 25 free parameters to a single one demonstrate the transformative potential of this theory.
	
	While some aspects (especially neutrinos) are still speculative, the T0-Theory offers a coherent, testable alternative to the current standard models of particle physics and cosmology. The coming years will be decisive in testing the far-reaching predictions of this geometric reformulation of physics through targeted experiments.
	
\section*{The T0-Theory is more than a new physical theory - it is an invitation to understand nature as a harmonic, geometrically structured whole, in which simplicity and beauty give rise to the complexity of observed phenomena.}
	
	\vfill
	
	\begin{center}
		\hrule
		\vspace{0.5cm}
		\textit{This overview summarizes the complete T0-document series}\\
		\textit{All 8 documents are available for detailed study}\\
		\vspace{0.3cm}
\section*{T0-Theory: Time-Mass Duality Framework}
		\textit{Johann Pascher, HTL Leonding, Austria}\\
		\textit{GitHub: https://github.com/jpascher/T0-Time-Mass-Duality}
		\vspace{0.3cm}
	\end{center}
	


% Bibliography
\begin{thebibliography}{99}
	
	\bibitem{pdg2024}
	Particle Data Group Collaboration (2024). 
	\textit{Review of Particle Physics}. 
	Progress of Theoretical and Experimental Physics, 2024(8), 083C01.
	\url{https://pdg.lbl.gov}
	
	\bibitem{flag2024}
	Aoki, Y., et al. (FLAG Collaboration) (2024). 
	\textit{FLAG Review 2024 of Lattice Results for Low-Energy Constants}. 
	arXiv:2411.04268.
	\url{https://arxiv.org/abs/2411.04268}
	
	\bibitem{fermilab_muon_g2}
	Abi, B., et al. (Muon g-2 Collaboration) (2021). 
	\textit{Measurement of the Positive Muon Anomalous Magnetic Moment to 0.46 ppm}. 
	Physical Review Letters, 126, 141801.
	
	\bibitem{peskin_schroeder}
	Peskin, M. E., \& Schroeder, D. V. (1995). 
	\textit{An Introduction to Quantum Field Theory}. 
	Addison-Wesley.
	
	\bibitem{weinberg_qft}
	Weinberg, S. (1995). 
	\textit{The Quantum Theory of Fields, Vol. I--III}. 
	Cambridge University Press.
	
	\bibitem{griffiths_particle}
	Griffiths, D. (2008). 
	\textit{Introduction to Elementary Particles}. 
	Wiley-VCH.
	
	\bibitem{mandl_shaw}
	Mandl, F., \& Shaw, G. (2010). 
	\textit{Quantum Field Theory (2nd ed.)}. 
	Wiley.
	
	\bibitem{srednicki_qft}
	Srednicki, M. (2007). 
	\textit{Quantum Field Theory}. 
	Cambridge University Press.
	
	\bibitem{t0_fundamentals}
	Pascher, J. (2024). 
	\textit{T0-Theory: Foundations of Time-Mass Duality}. 
	Unpublished manuscript, HTL Leonding.
	
	\bibitem{t0_fine_structure}
	Pascher, J. (2024). 
	\textit{T0-Theory: The Fine Structure Constant}. 
	Unpublished manuscript, HTL Leonding.
	
	\bibitem{t0_neutrinos}
	Pascher, J. (2024). 
	\textit{T0-Theory: Neutrino Masses and PMNS Mixing}. 
	Unpublished manuscript, HTL Leonding.
	
	\bibitem{t0_github}
	Pascher, J. (2024--2025). 
	\textit{T0-Time-Mass-Duality Repository}. 
	GitHub.
	\url{https://github.com/jpascher/T0-Time-Mass-Duality}
	
	\bibitem{lattice_qcd_review}
	Kronfeld, A. S. (2012). 
	\textit{Twenty-first Century Lattice Gauge Theory: Results from the QCD Lagrangian}. 
	Annual Review of Nuclear and Particle Science, 62, 265--284.
	
	\bibitem{neutrino_mixing_pdg}
	Particle Data Group Collaboration (2024). 
	\textit{Neutrino Masses, Mixing, and Oscillations}. 
	PDG Review 2024.
	\url{https://pdg.lbl.gov/2024/reviews/rpp2024-rev-neutrino-mixing.pdf}
	
	\bibitem{higgs_discovery}
	ATLAS and CMS Collaborations (2012). 
	\textit{Observation of a New Particle in the Search for the Standard Model Higgs Boson}. 
	Physics Letters B, 716, 1--29.
	
	\bibitem{Brannen2005}
	C. P. Brannen, ``Estimate of neutrino masses from Koide's relation'', \textit{arXiv:hep-ph/0505028} (2005).
	\url{https://arxiv.org/abs/hep-ph/0505028}
	
	\bibitem{Brannen2006}
	C. P. Brannen, ``Koide Mass Formula for Neutrinos'', \textit{arXiv:0702.0052} (2006).
	\url{http://brannenworks.com/MASSES.pdf}
	
	\bibitem{PhaseVectors2025}
	Anonymous, ``The Koide Relation and Lepton Mass Hierarchy from Phase Vectors'', \textit{rXiv:2507.0040} (2025).
	\url{https://rxiv.org/pdf/2507.0040v1.pdf}
	
	\bibitem{PDG2025}
	Particle Data Group, ``Review of Particle Physics'', \textit{Phys. Rev. D} \textbf{112} (2025) 030001.
	\url{https://pdg.lbl.gov/2025/}
	
	\bibitem{terrell2024}
	Terrell et al. (2024). 
	\textit{Single-Clock Metrology in Nature}. 
	Nature Physics.
	
	\bibitem{hossenfelder2024}
	Hossenfelder, S. (2024). 
	\textit{Single Clock Video Explanation}. 
	YouTube.
	
	\bibitem{hundert1931}
	Hundert (1931). 
	\textit{Reference Work}. 
	Publisher.
	
	\bibitem{terrell2025}
	Terrell et al. (2025). 
	\textit{Advanced Clock Synchronization Methods}. 
	Physical Review Letters.
	
	\bibitem{pascher_t0_2025}
	Pascher, J. (2025). 
	\textit{T0-Theory: Complete Framework and Applications}. 
	Unpublished manuscript, HTL Leonding.
	
	\bibitem{t0qm}
	Pascher, J. (2024). 
	\textit{T0-Theory: Quantum Mechanics Formulation}. 
	Unpublished manuscript, HTL Leonding.
	
	\bibitem{t0anomale}
	Pascher, J. (2024). 
	\textit{T0-Theory: Anomalous Magnetic Moments}. 
	Unpublished manuscript, HTL Leonding.
	
	\bibitem{muong2complete}
	Abi, B., et al. (Muon g-2 Collaboration) (2023). 
	\textit{Complete Measurement of the Positive Muon Anomalous Magnetic Moment}. 
	Physical Review Letters, 131, 161802.
	
	\bibitem{penrose2004}
	Penrose, R. (2004). 
	\textit{The Road to Reality: A Complete Guide to the Laws of the Universe}. 
	Jonathan Cape.
	
	\bibitem{planck1900}
	Planck, M. (1900). 
	\textit{On the Theory of the Energy Distribution Law of the Normal Spectrum}. 
	Verhandlungen der Deutschen Physikalischen Gesellschaft, 2, 237.
	
	\bibitem{T0Theory}
	Pascher, J. (2024). 
	\textit{T0-Theory: Fundamental Principles}. 
	Unpublished manuscript, HTL Leonding.
	
	% Additional bibliography entries for all undefined citations
	\bibitem{6g_roadmap}
	6G Research Consortium (2024).
	\textit{6G Technology Roadmap}.
	Technical Report.
	
	\bibitem{Born2013}
	Born, M. (2013).
	\textit{Einstein's Theory of Relativity}.
	Dover Publications.
	
	\bibitem{Casimir1948}
	Casimir, H. B. G. (1948).
	\textit{On the attraction between two perfectly conducting plates}.
	Proc. Kon. Ned. Akad. Wetensch. B51, 793--795.
	
	\bibitem{Einstein1905}
	Einstein, A. (1905).
	\textit{On the Electrodynamics of Moving Bodies}.
	Annalen der Physik, 17, 891--921.
	
	\bibitem{Feynman2006}
	Feynman, R. P. (2006).
	\textit{QED: The Strange Theory of Light and Matter}.
	Princeton University Press.
	
	\bibitem{Griffiths2017}
	Griffiths, D. J. (2017).
	\textit{Introduction to Electrodynamics (4th ed.)}.
	Cambridge University Press.
	
	\bibitem{Jackson1999}
	Jackson, J. D. (1999).
	\textit{Classical Electrodynamics (3rd ed.)}.
	Wiley.
	
	\bibitem{Mohr2016}
	Mohr, P. J., et al. (2016).
	\textit{CODATA Recommended Values of the Fundamental Physical Constants: 2014}.
	Rev. Mod. Phys. 88, 035009.
	
	\bibitem{Parker2018}
	Parker, R. H., et al. (2018).
	\textit{Measurement of the fine-structure constant as a test of the Standard Model}.
	Science, 360, 191--195.
	
	\bibitem{Planck1900}
	Planck, M. (1900).
	\textit{On the Theory of the Energy Distribution Law of the Normal Spectrum}.
	Verhandlungen der Deutschen Physikalischen Gesellschaft, 2, 237.
	
	\bibitem{Planck2018}
	Planck Collaboration (2018).
	\textit{Planck 2018 results. VI. Cosmological parameters}.
	Astronomy \& Astrophysics, 641, A6.
	
	\bibitem{QFT_T0}
	Pascher, J. (2024).
	\textit{T0-Theory and QFT Connections}.
	Unpublished manuscript, HTL Leonding.
	
	\bibitem{Sommerfeld1916}
	Sommerfeld, A. (1916).
	\textit{On the Quantum Theory of Spectral Lines}.
	Annalen der Physik, 51, 1--94.
	
	\bibitem{T0_Feinstruktur}
	Pascher, J. (2024).
	\textit{T0-Theory: Fine Structure Analysis}.
	Unpublished manuscript, HTL Leonding.
	
	\bibitem{T0_SI}
	Pascher, J. (2024).
	\textit{T0-Theory and SI Units}.
	Unpublished manuscript, HTL Leonding.
	
	\bibitem{T0_fine_structure}
	Pascher, J. (2024).
	\textit{T0-Theory: The Fine Structure Constant}.
	Unpublished manuscript, HTL Leonding.
	
	\bibitem{T0_g2_erweiterung}
	Pascher, J. (2024).
	\textit{T0-Theory: g-2 Extensions}.
	Unpublished manuscript, HTL Leonding.
	
	\bibitem{T0_gravitational_constant}
	Pascher, J. (2024).
	\textit{T0-Theory: Gravitational Constant Derivation}.
	Unpublished manuscript, HTL Leonding.
	
	\bibitem{T0_netze_en}
	Pascher, J. (2024).
	\textit{T0-Theory: Network Structures}.
	Unpublished manuscript, HTL Leonding.
	
	\bibitem{T0_tm_erweiterung}
	Pascher, J. (2024).
	\textit{T0-Theory: Time-Mass Extensions}.
	Unpublished manuscript, HTL Leonding.
	
	\bibitem{Uzan2003}
	Uzan, J.-P. (2003).
	\textit{The fundamental constants and their variation}.
	Rev. Mod. Phys. 75, 403--455.
	
	\bibitem{Weinberg1995}
	Weinberg, S. (1995).
	\textit{The Quantum Theory of Fields, Vol. I}.
	Cambridge University Press.
	
	\bibitem{albrecht1999}
	Albrecht, A. \& Magueijo, J. (1999).
	\textit{A time varying speed of light as a solution to cosmological puzzles}.
	Phys. Rev. D 59, 043516.
	
	\bibitem{alice2023}
	ALICE Collaboration (2023).
	\textit{Recent results from ALICE}.
	CERN-EP-2023-XXX.
	
	\bibitem{analog_optical}
	Smith, J. et al. (2024).
	\textit{Analog optical computing systems}.
	Nature Photonics.
	
	\bibitem{ashtekar2004}
	Ashtekar, A. \& Lewandowski, J. (2004).
	\textit{Background independent quantum gravity}.
	Class. Quantum Grav. 21, R53.
	
	\bibitem{atlas2023}
	ATLAS Collaboration (2023).
	\textit{ATLAS physics results}.
	CERN-PH-EP-2023-XXX.
	
	\bibitem{atlas2023higgs}
	ATLAS Collaboration (2023).
	\textit{Higgs boson measurements}.
	Phys. Rev. Lett.
	
	\bibitem{barbour1999}
	Barbour, J. (1999).
	\textit{The End of Time}.
	Oxford University Press.
	
	\bibitem{barrow1999}
	Barrow, J. D. (1999).
	\textit{Cosmologies with varying light speed}.
	Phys. Rev. D 59, 043515.
	
	\bibitem{becker2007}
	Becker, K. et al. (2007).
	\textit{String Theory and M-Theory}.
	Cambridge University Press.
	
	\bibitem{bell_muon}
	Bennett, G. W., et al. (Muon g-2 Collaboration) (2006).
	\textit{Final report of the E821 muon anomalous magnetic moment measurement}.
	Phys. Rev. D 73, 072003.
	
	\bibitem{bondi1948}
	Bondi, H. \& Gold, T. (1948).
	\textit{The steady-state theory of the expanding universe}.
	Mon. Not. R. Astron. Soc. 108, 252--270.
	
	\bibitem{brewer2019}
	Brewer, S. M. et al. (2019).
	\textit{Al+ Quantum-Logic Clock with Systematic Uncertainty below $10^{-18}$}.
	Phys. Rev. Lett. 123, 033201.
	
	\bibitem{cms2023top}
	CMS Collaboration (2023).
	\textit{Top quark measurements at CMS}.
	JHEP 2023.
	
	\bibitem{cms2024}
	CMS Collaboration (2024).
	\textit{CMS physics results 2024}.
	CERN-PH-EP-2024-XXX.
	
	\bibitem{codata2019}
	Tiesinga, E. et al. (2019).
	\textit{The 2018 CODATA Recommended Values}.
	J. Phys. Chem. Ref. Data.
	
	\bibitem{desi2025}
	DESI Collaboration (2025).
	\textit{DESI 2025 Cosmology Results}.
	arXiv preprint.
	
	\bibitem{differential_optical}
	Wang, X. et al. (2024).
	\textit{Differential optical computing}.
	Optica.
	
	\bibitem{dingle1972}
	Dingle, H. (1972).
	\textit{Science at the Crossroads}.
	Martin Brian \& O'Keeffe.
	
	\bibitem{divalentino2021}
	Di Valentino, E. et al. (2021).
	\textit{In the realm of the Hubble tension}.
	Class. Quantum Grav. 38, 153001.
	
	\bibitem{elnaschie2004}
	El Naschie, M. S. (2004).
	\textit{A review of E infinity theory}.
	Chaos, Solitons \& Fractals, 19, 209--236.
	
	\bibitem{fabrication_heterogeneous}
	Chen, Y. et al. (2024).
	\textit{Heterogeneous photonic integration}.
	Nature Electronics.
	
	\bibitem{fermilab2023}
	Fermilab (2023).
	\textit{Muon g-2 results}.
	Phys. Rev. Lett.
	
	\bibitem{flexible_wafer}
	Kim, S. et al. (2024).
	\textit{Flexible wafer-scale photonics}.
	Science Advances.
	
	\bibitem{francesco1997}
	Di Francesco, P. et al. (1997).
	\textit{Conformal Field Theory}.
	Springer.
	
	\bibitem{hartree1957}
	Hartree, D. R. (1957).
	\textit{The Calculation of Atomic Structures}.
	Wiley.
	
	\bibitem{hhi_6g}
	Fraunhofer HHI (2024).
	\textit{6G Photonic Integration}.
	Technical Report.
	
	\bibitem{hossenfelder2025}
	Hossenfelder, S. (2025).
	\textit{Science without the gobbledygook}.
	YouTube/Blog.
	
	\bibitem{hossenfelder_single_clock_video}
	Hossenfelder, S. (2024).
	\textit{The Single Clock Problem}.
	YouTube.
	
	\bibitem{hoyle1948}
	Hoyle, F. (1948).
	\textit{A new model for the expanding universe}.
	Mon. Not. R. Astron. Soc. 108, 372--382.
	
	\bibitem{integration_microelectronic}
	Liu, A. et al. (2024).
	\textit{Microelectronic photonic integration}.
	IEEE Journal.
	
	\bibitem{jacobson1995}
	Jacobson, T. (1995).
	\textit{Thermodynamics of spacetime}.
	Phys. Rev. Lett. 75, 1260.
	
	\bibitem{kasevich2023}
	Kasevich, M. et al. (2023).
	\textit{Atom interferometry tests}.
	Nature Physics.
	
	\bibitem{lerner2014}
	Lerner, E. J. (2014).
	\textit{An open letter on cosmology}.
	New Scientist.
	
	\bibitem{lisa2017}
	LISA Consortium (2017).
	\textit{Laser Interferometer Space Antenna}.
	ESA Technical Report.
	
	\bibitem{lithium_tantalate}
	Zhang, M. et al. (2024).
	\textit{Thin-film lithium tantalate photonics}.
	Nature Photonics.
	
	\bibitem{lopez2010}
	Lopez-Corredoira, M. (2010).
	\textit{Tests and problems of the standard model in cosmology}.
	Int. J. Mod. Phys. D.
	
	\bibitem{ludlow2015}
	Ludlow, A. D. et al. (2015).
	\textit{Optical atomic clocks}.
	Rev. Mod. Phys. 87, 637.
	
	\bibitem{mach1883}
	Mach, E. (1883).
	\textit{Die Mechanik in ihrer Entwickelung}.
	F.A. Brockhaus.
	
	\bibitem{maldacena1998}
	Maldacena, J. (1998).
	\textit{The large N limit of superconformal field theories}.
	Adv. Theor. Math. Phys. 2, 231--252.
	
	\bibitem{mueller2014}
	Müller, H. et al. (2014).
	\textit{Atom interferometry tests of the gravitational redshift}.
	Phys. Rev. Lett.
	
	\bibitem{mug2_final_2025}
	Muon g-2 Collaboration (2025).
	\textit{Final muon g-2 measurement}.
	Phys. Rev. Lett.
	
	\bibitem{muong2_2023}
	Muon g-2 Collaboration (2023).
	\textit{Updated muon g-2 results}.
	Phys. Rev. Lett.
	
	\bibitem{nathan2024}
	Nathan, A. et al. (2024).
	\textit{Quantum computing advances}.
	Nature.
	
	\bibitem{newell2018}
	Newell, D. B. et al. (2018).
	\textit{The CODATA 2017 values of h, e, k, and $N_A$}.
	Metrologia 55, L13.
	
	\bibitem{nottale1993}
	Nottale, L. (1993).
	\textit{Fractal Space-Time and Microphysics}.
	World Scientific.
	
	\bibitem{on_chip_lithium}
	Wang, C. et al. (2024).
	\textit{On-chip lithium niobate photonics}.
	Nature Communications.
	
	\bibitem{optical_advantages}
	Shastri, B. J. et al. (2024).
	\textit{Advantages of optical computing}.
	Nature Reviews Physics.
	
	\bibitem{pascher2025cmb}
	Pascher, J. (2025).
	\textit{T0-Theory: CMB Analysis}.
	Unpublished manuscript, HTL Leonding.
	
	\bibitem{pascher2025g2}
	Pascher, J. (2025).
	\textit{T0-Theory: g-2 Predictions}.
	Unpublished manuscript, HTL Leonding.
	
	\bibitem{pascher2025qm}
	Pascher, J. (2025).
	\textit{T0-Theory: Quantum Mechanics}.
	Unpublished manuscript, HTL Leonding.
	
	\bibitem{pascher2025si}
	Pascher, J. (2025).
	\textit{T0-Theory: SI Unit System}.
	Unpublished manuscript, HTL Leonding.
	
	\bibitem{pascher2025t0}
	Pascher, J. (2025).
	\textit{T0-Theory: Complete Framework}.
	Unpublished manuscript, HTL Leonding.
	
	\bibitem{pascher:fundamentals}
	Pascher, J. (2024).
	\textit{T0-Theory: Fundamentals}.
	Unpublished manuscript, HTL Leonding.
	
	\bibitem{pascher:g2_rev9}
	Pascher, J. (2024).
	\textit{T0-Theory: g-2 Revision 9}.
	Unpublished manuscript, HTL Leonding.
	
	\bibitem{pascher:geometric_formalism}
	Pascher, J. (2024).
	\textit{T0-Theory: Geometric Formalism}.
	Unpublished manuscript, HTL Leonding.
	
	\bibitem{pascher:ml_addendum}
	Pascher, J. (2024).
	\textit{T0-Theory: Machine Learning Addendum}.
	Unpublished manuscript, HTL Leonding.
	
	\bibitem{pascher:t0_foundations}
	Pascher, J. (2024).
	\textit{T0-Theory: Foundations}.
	Unpublished manuscript, HTL Leonding.
	
	\bibitem{pascher_derivation_beta_2025}
	Pascher, J. (2025).
	\textit{T0-Theory: Derivation of Beta}.
	Unpublished manuscript, HTL Leonding.
	
	\bibitem{pascher_higgs_connection_2025}
	Pascher, J. (2025).
	\textit{T0-Theory: Higgs Connection}.
	Unpublished manuscript, HTL Leonding.
	
	\bibitem{pascher_lagrangian_extended_2025}
	Pascher, J. (2025).
	\textit{T0-Theory: Extended Lagrangian}.
	Unpublished manuscript, HTL Leonding.
	
	\bibitem{pascher_mathematical_structure_2025}
	Pascher, J. (2025).
	\textit{T0-Theory: Mathematical Structure}.
	Unpublished manuscript, HTL Leonding.
	
	\bibitem{pascher_t0_cmb_2025}
	Pascher, J. (2025).
	\textit{T0-Theory: CMB Predictions}.
	Unpublished manuscript, HTL Leonding.
	
	\bibitem{pascher_t0_energie_2025}
	Pascher, J. (2025).
	\textit{T0-Theory: Energy}.
	Unpublished manuscript, HTL Leonding.
	
	\bibitem{pascher_t0_energy_2025}
	Pascher, J. (2025).
	\textit{T0-Theory: Energy Framework}.
	Unpublished manuscript, HTL Leonding.
	
	\bibitem{pascher_t0_theory_2025}
	Pascher, J. (2025).
	\textit{T0-Theory: Complete Theory}.
	Unpublished manuscript, HTL Leonding.
	
	\bibitem{penrose1959}
	Penrose, R. (1959).
	\textit{The apparent shape of a relativistically moving sphere}.
	Proc. Cambridge Phil. Soc. 55, 137--139.
	
	\bibitem{penrose1967}
	Penrose, R. (1967).
	\textit{Twistor algebra}.
	J. Math. Phys. 8, 345--366.
	
	\bibitem{peratt1992}
	Peratt, A. L. (1992).
	\textit{Physics of the Plasma Universe}.
	Springer-Verlag.
	
	\bibitem{peskin1995}
	Peskin, M. E. \& Schroeder, D. V. (1995).
	\textit{An Introduction to Quantum Field Theory}.
	Addison-Wesley.
	
	\bibitem{peskin_schroeder_1995}
	Peskin, M. E. \& Schroeder, D. V. (1995).
	\textit{An Introduction to Quantum Field Theory}.
	Addison-Wesley.
	
	\bibitem{phoquant}
	PhoQuant (2024).
	\textit{Photonic quantum computing}.
	Technical Report.
	
	\bibitem{photonics_ai}
	Wetzstein, G. et al. (2024).
	\textit{Photonics for AI}.
	Nature.
	
	\bibitem{planck1906}
	Planck, M. (1906).
	\textit{The Theory of Heat Radiation}.
	Johann Ambrosius Barth.
	
	\bibitem{planck2018}
	Planck Collaboration (2018).
	\textit{Planck 2018 results}.
	A\&A 641, A6.
	
	\bibitem{polchinski1998}
	Polchinski, J. (1998).
	\textit{String Theory}.
	Cambridge University Press.
	
	\bibitem{qant_nps}
	QANT (2024).
	\textit{Quantum photonics systems}.
	Technical Report.
	
	\bibitem{quantenjahr25}
	Quantenjahr (2025).
	\textit{International Year of Quantum}.
	UNESCO.
	
	\bibitem{recurrent_photonics}
	Tait, A. N. et al. (2024).
	\textit{Recurrent photonic neural networks}.
	Optica.
	
	\bibitem{rf_photonics}
	Capmany, J. \& Novak, D. (2024).
	\textit{Microwave photonics}.
	Nature Photonics.
	
	\bibitem{riess2019}
	Riess, A. G. et al. (2019).
	\textit{Large Magellanic Cloud Cepheid Standards}.
	ApJ 876, 85.
	
	\bibitem{riess2022}
	Riess, A. G. et al. (2022).
	\textit{A Comprehensive Measurement of H0}.
	ApJ 934, L7.
	
	\bibitem{rovelli2004}
	Rovelli, C. (2004).
	\textit{Quantum Gravity}.
	Cambridge University Press.
	
	\bibitem{sciama1953}
	Sciama, D. W. (1953).
	\textit{On the origin of inertia}.
	Mon. Not. R. Astron. Soc. 113, 34--42.
	
	\bibitem{sciencedaily2025}
	ScienceDaily (2025).
	\textit{Physics news}.
	Online.
	
	\bibitem{sm_g2_2025}
	Aoyama, T. et al. (2025).
	\textit{Standard Model prediction for g-2}.
	Phys. Rep.
	
	\bibitem{susskind1995}
	Susskind, L. (1995).
	\textit{The world as a hologram}.
	J. Math. Phys. 36, 6377--6396.
	
	\bibitem{t0_kosmologie}
	Pascher, J. (2024).
	\textit{T0-Theory: Cosmology}.
	Unpublished manuscript, HTL Leonding.
	
	\bibitem{terrell1959}
	Terrell, J. (1959).
	\textit{Invisibility of the Lorentz contraction}.
	Phys. Rev. 116, 1041--1045.
	
	\bibitem{terrell_single_clock_nature_2024}
	Terrell, J. et al. (2024).
	\textit{Single clock precision measurements}.
	Nature Physics.
	
	\bibitem{tfln_foundry}
	TFLN Foundry (2024).
	\textit{Thin-film lithium niobate foundry services}.
	Technical Specifications.
	
	\bibitem{thiemann2007}
	Thiemann, T. (2007).
	\textit{Modern Canonical Quantum General Relativity}.
	Cambridge University Press.
	
	\bibitem{thz_epfl}
	EPFL (2024).
	\textit{Terahertz photonics research}.
	Technical Report.
	
	\bibitem{unnikrishnan2004}
	Unnikrishnan, C. S. (2004).
	\textit{On Einstein's resolution of the twin clock paradox}.
	Current Science, 86, 704--709.
	
	\bibitem{verlinde2011}
	Verlinde, E. (2011).
	\textit{On the origin of gravity and the laws of Newton}.
	JHEP 2011, 29.
	
	\bibitem{video2025}
	Video (2025).
	\textit{Physics video explanation}.
	YouTube.
	
	\bibitem{weinberg1995}
	Weinberg, S. (1995).
	\textit{The Quantum Theory of Fields}.
	Cambridge University Press.
	
	\bibitem{weiskopf2000}
	Weiskopf, D. (2000).
	\textit{Visualization of special relativity}.
	PhD thesis, University of Tübingen.
	
	\bibitem{wheeler1990}
	Wheeler, J. A. (1990).
	\textit{A Journey into Gravity and Spacetime}.
	Scientific American Library.
	
	\bibitem{wiki_bell}
	Wikipedia (2024).
	\textit{Bell's theorem}.
	Online encyclopedia.
	
	\bibitem{zwicky1929}
	Zwicky, F. (1929).
	\textit{On the red shift of spectral lines through interstellar space}.
	Proc. Natl. Acad. Sci. 15, 773--779.

\end{thebibliography}


\end{document}
