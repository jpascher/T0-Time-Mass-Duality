\documentclass[11pt,a4paper]{article}
\usepackage[a4paper,margin=2cm]{geometry}
\usepackage[utf8]{inputenc}
\usepackage[english]{babel}
\usepackage{lmodern}
\renewcommand{\familydefault}{\sfdefault}

\usepackage{amsmath,amssymb,amsthm}
\usepackage{graphicx}
\usepackage[unicode,pdfencoding=auto,hypertexnames=false]{hyperref}
\usepackage{booktabs}
\usepackage{longtable}
\usepackage{array}
\usepackage{siunitx}
\usepackage{fancyhdr}
\usepackage{float}
\usepackage{tikz}
% tcolorbox removed for standalone
% tcbset removed
\tikzset{
  t0blue/.style={draw=blue,fill=blue!10},
  t0red/.style={draw=red,fill=red!10},
  t0green/.style={draw=green!50!black,fill=green!10},
  t0orange/.style={draw=orange,fill=orange!10},
}
\usepackage{setspace}
\usepackage{enumitem}
\usepackage{adjustbox}
\usepackage{xcolor}

% Define colors for xcolor package
\definecolor{t0green}{RGB}{34,139,34}
\definecolor{t0blue}{RGB}{0,0,255}
\definecolor{t0red}{RGB}{255,0,0}
\definecolor{t0orange}{RGB}{255,165,0}

% Define custom column types for tables
\newcolumntype{L}[1]{>{\raggedright\arraybackslash}p{#1}}
\newcolumntype{C}[1]{>{\centering\arraybackslash}p{#1}}
\newcolumntype{R}[1]{>{\raggedleft\arraybackslash}p{#1}}

\setlength{\parindent}{0pt}
\setlength{\parskip}{6pt}

\hypersetup{
  colorlinks=true,
  linkcolor=blue,
  citecolor=blue,
  urlcolor=blue
}
\pagestyle{fancy}
\setlength{\headheight}{28pt}

\newcommand{\checkmarkx}{\checkmark}
\newcommand{\warningx}{\textbf{!}}

% Makros aus Einzel-Dokumenten (Fallback-Definitionen)
\newcommand{\mytimes}{\times}
\newcommand{\myapprox}{\approx}
\newcommand{\mysim}{\sim}
\newcommand{\myomega}{\omega}
\newcommand{\mypi}{\pi}
\newcommand{\myrightarrow}{\rightarrow}
\newcommand{\mypropto}{\propto}
\newcommand{\deltafield}{\delta\phi}
\newcommand{\xipar}{\xi}
\newcommand{\xiT}{\xi}
\newcommand{\lambdah}{\lambda_h}

% Additional macros used in chapter files
\newcommand{\Kfrak}{K_{\text{frak}}}  % Fractal correction factor
\newcommand{\Dfrak}{D_f}              % Fractal dimension
\newcommand{\betapar}{\beta}          % T0 beta parameter
\newcommand{\alphapar}{\alpha}        % T0 alpha parameter
\newcommand{\Efield}{E}               % Energy field
% Note: checkmarkxa/warningxa are variants used in auto-generated chapter files
\newcommand{\checkmarkxa}{\checkmark}
\newcommand{\warningxa}{\textbf{!}}

% Additional T0-specific macros
\newcommand{\xigeom}{\xi_{\text{geom}}}  % Geometric xi
\newcommand{\lP}{\ell_P}                  % Planck length
\newcommand{\rzero}{r_0}                  % Characteristic radius
\newcommand{\xirat}{\xi_{\text{rat}}}     % Xi ratio
\newcommand{\tzero}{t_0}                  % Characteristic time
\newcommand{\natunits}{\text{(nat. units)}}  % Natural units annotation
\newcommand{\myRightarrow}{\Rightarrow}   % Arrow variant
\newcommand{\Lag}{\mathcal{L}}            % Lagrangian

% Physics macros used in chapter files
\newcommand{\CQCD}{C_{\text{QCD}}}        % QCD correction
\newcommand{\EP}{E_P}                     % Planck energy
\newcommand{\Ee}{E_e}                     % Electron energy
\newcommand{\Emu}{E_\mu}                  % Muon energy
\newcommand{\Exi}{E_\xi}                  % Xi energy
\newcommand{\Ezero}{E_0}                  % Characteristic energy
\newcommand{\Hubble}{H}                   % Hubble constant
\newcommand{\Kspec}{K_{\text{spec}}}      % Spectral correction
\newcommand{\Lambdat}{\Lambda_t}          % Time-related cosmological constant
\newcommand{\Leff}{\mathcal{L}_{\text{eff}}}  % Effective Lagrangian
\newcommand{\Lorentz}{\mathcal{L}}        % Lorentz symbol
\newcommand{\Lxi}{L_\xi}                  % Xi length
\newcommand{\Tfield}{T}                   % Time field
\newcommand{\Weyl}{W}                     % Weyl tensor/symbol
\newcommand{\alphaEMSI}{\alpha_{\text{EM,SI}}}  % EM alpha in SI
\newcommand{\alphaEMnat}{\alpha_{\text{EM,nat}}}  % EM alpha in natural units
\newcommand{\alphaem}{\alpha_{\text{em}}} % Electromagnetic alpha
\newcommand{\betaTSI}{\beta_{T,\text{SI}}}  % Beta in SI
\newcommand{\betaTnat}{\beta_{T,\text{nat}}}  % Beta in natural units
\newcommand{\deltam}{\delta m}            % Mass difference
\newcommand{\phiT}{\phi_T}                % T-field phi
\newcommand{\tP}{t_P}                     % Planck time
\newcommand{\rhoCMB}{\rho_{\text{CMB}}}   % CMB density
\newcommand{\rhoCasimir}{\rho_{\text{Casimir}}}  % Casimir density

% Table formatting
\usepackage{multirow}

% Additional physics macros
\newcommand{\Riem}{\mathcal{R}}           % Riemann tensor
\newcommand{\ZPinch}{Z_{\text{pinch}}}    % Z-pinch
\newcommand{\SynchPower}{P_{\text{synch}}} % Synchrotron power
\newcommand{\Rzero}{R_0}                  % Characteristic radius
\newcommand{\alphafine}{\alpha}           % Fine structure constant
\newcommand{\Etau}{E_\tau}                % Tau energy
\newcommand{\deltaE}{\delta E}            % Energy deviation
\newcommand{\EPlanck}{E_P}                % Planck energy
\newcommand{\pichar}{\pi}                 % Pi character
\newcommand{\alphaWSI}{\alpha_{W,\text{SI}}}  % Wien alpha in SI
\newcommand{\alphaWnat}{\alpha_{W,\text{nat}}}  % Wien alpha in natural units

% Einfache abstract-Umgebung für Kapitel:
\newenvironment{abstract}{%
  \begin{center}\bfseries Abstract\end{center}\small
}{\par}


\title{RSA En}
\author{J. Pascher}
\date{\today}

\begin{document}
\maketitle

\section*{Rsa (RSA)}

	\begin{abstract}
		This paper presents a mathematical analysis of the T0-Shor algorithm based on energy field formulation. We examine the theoretical foundations of the time-mass duality $T(x,t) \cdot m(x,t) = 1$ and its application to integer factorization. The analysis focuses on the mathematical consistency of the field equations, computational complexity implications, and the role of the coupling parameter $\xi$ derived from Higgs field interactions. We provide rigorous derivations of the algorithm's theoretical performance characteristics and identify the fundamental assumptions underlying the T0 framework.
	\end{abstract}
	
	
	\section{Introduction}
	
	The T0-Shor algorithm represents a theoretical extension of Shor's factorization algorithm based on energy field dynamics rather than quantum mechanical superposition. This work examines the mathematical foundations of this approach without making claims about practical implementability or superiority over existing methods.
	
	\subsection{Theoretical Framework}
	
	The T0 model introduces the following fundamental mathematical structures:
	
	\begin{align}
		\text{Time-Mass Duality}: \quad &T(x,t) \cdot m(x,t) = 1 \label{RSA:L-RSA-1007}\\
		\text{Field Equation}: \quad &\nabla^2 T(x) = -\frac{\rho(x)}{T(x)^2} \label{RSA:L-RSA-1008}\\
		\text{Energy Evolution}: \quad &\frac{\partial^2 E}{\partial t^2} = -\omega^2 E \label{RSA:L-RSA-1009}
	\end{align}
	
	The coupling parameter $\xi$ is theoretically derived from Higgs field interactions:
	\begin{equation}
		\xi = g_H \cdot \frac{\langle\phi\rangle}{v_{EW}} \label{RSA:L-RSA-1010}
	\end{equation}
	where $g_H$ is the Higgs coupling constant, $\langle\phi\rangle$ is the vacuum expectation value, and $v_{EW} = 246$ GeV is the electroweak scale.
	
	\section{Mathematical Foundations}
	
	\subsection{Wave-Like Behavior of T0-Fields}
	
	The T0-field exhibits wave-like propagation characteristics analogous to acoustic waves in media. The fundamental wave equation for T0-fields is:
	
	\begin{equation}
		\nabla^2 T - \frac{1}{c_{T0}^2} \frac{\partial^2 T}{\partial t^2} = -\frac{\rho(x,t)}{T(x,t)^2} \label{RSA:L-RSA-1011}
	\end{equation}
	
	where $c_{T0}$ is the T0-field propagation velocity in the medium, analogous to sound velocity.
	
	\subsection{Medium-Dependent Properties}
	
	Similar to acoustic waves, T0-field propagation depends critically on medium properties:
	
	\textbf{T0-field velocity in different media}:
	\begin{align}
		c_{T0,vacuum} &= c \sqrt{\frac{\xi_0}{\xi_{vacuum}}} \\
		c_{T0,metal} &= c \sqrt{\frac{\xi_0 \epsilon_r}{\xi_{vacuum}}} \\
		c_{T0,dielectric} &= \frac{c}{\sqrt{\epsilon_r \mu_r}} \sqrt{\frac{\xi_0}{\xi_{vacuum}}} \\
		c_{T0,plasma} &= c \sqrt{1 - \frac{\omega_p^2}{\omega^2}} \sqrt{\frac{\xi_0}{\xi_{vacuum}}}
	\end{align}
	
	where $\omega_p$ is the plasma frequency and $\epsilon_r$, $\mu_r$ are relative permittivity and permeability.
	
	\subsection{Boundary Conditions and Reflections}
	
	At interfaces between different media, T0-fields satisfy boundary conditions similar to electromagnetic waves:
	
	\textbf{Continuity conditions}:
	\begin{align}
		T_1|_{interface} &= T_2|_{interface} \quad \text{(field continuity)} \\
		\frac{1}{m_1} \frac{\partial T_1}{\partial n}\bigg|_{interface} &= \frac{1}{m_2} \frac{\partial T_2}{\partial n}\bigg|_{interface} \quad \text{(flux continuity)}
	\end{align}
	
	\textbf{Reflection and transmission coefficients}:
	\begin{align}
		r &= \frac{Z_1 - Z_2}{Z_1 + Z_2} \quad \text{(reflection coefficient)} \\
		t &= \frac{2Z_1}{Z_1 + Z_2} \quad \text{(transmission coefficient)}
	\end{align}
	
	where $Z_i = \sqrt{m_i/T_i}$ is the T0-field impedance in medium $i$.
	
	\subsection{Geometric Constraints and Cavity Resonances}
	
	In bounded geometries, T0-fields form standing wave patterns with discrete eigenfrequencies:
	
	\textbf{Rectangular cavity} ($L_x \times L_y \times L_z$):
	\begin{equation}
		f_{mnp} = \frac{c_{T0}}{2} \sqrt{\left(\frac{m}{L_x}\right)^2 + \left(\frac{n}{L_y}\right)^2 + \left(\frac{p}{L_z}\right)^2}
	\end{equation}
	
	\textbf{Cylindrical cavity} (radius $a$, height $h$):
	\begin{equation}
		f_{mnp} = \frac{c_{T0}}{2\pi} \sqrt{\left(\frac{\chi_{mn}}{a}\right)^2 + \left(\frac{p\pi}{h}\right)^2}
	\end{equation}
	
	where $\chi_{mn}$ are zeros of Bessel functions.
	
	\textbf{Spherical cavity} (radius $R$):
	\begin{equation}
		f_{nlm} = \frac{c_{T0}}{2\pi R} \sqrt{n(n+1)}
	\end{equation}
	
	\subsection{Dispersion Relations}
	
	In dispersive media, the T0-field exhibits frequency-dependent propagation:
	
	\begin{equation}
		\omega^2 = c_{T0}^2(\omega) k^2 + \omega_0^2
	\end{equation}
	
	where $\omega_0$ is a characteristic frequency related to the medium's microscopic structure.
	
	\textbf{Group velocity} (important for information propagation):
	\begin{equation}
		v_g = \frac{d\omega}{dk} = \frac{c_{T0}^2 k}{\omega} + \frac{dc_{T0}^2}{d\omega} \frac{k^2}{2}
	\end{equation}
	
	\subsection{Hyperbolical Geometry in Duality Space}
	
	The time-mass duality (Eq.~\ref{RSA:L-RSA-1007}) defines a hyperbolic metric in the $(T,m)$ parameter space:
	
	\begin{equation}
		ds^2 = \frac{dT \cdot dm}{T \cdot m} = \frac{d(\ln T) \cdot d(\ln m)}{T \cdot m}
	\end{equation}
	
	This geometry is characterized by:
	\begin{itemize}
		\item Constant negative curvature: $K = -1$
		\item Invariant measure: $d\mu = \frac{dT \, dm}{T \cdot m}$
		\item Isometry group: $PSL(2,\mathbb{R})$
	\end{itemize}
	
	\subsection{Field Equation Analysis}
	
	For spherically symmetric configurations, Eq.~\ref{RSA:L-RSA-1008} reduces to:
	\begin{equation}
		\frac{1}{r^2}\frac{d}{dr}\left(r^2 \frac{dT}{dr}\right) = -\frac{\rho(r)}{T(r)^2}
	\end{equation}
	
	For a point mass $m$ at the origin with $\rho(r) = mc^2 \delta(r)$, the solution is:
	\begin{equation}
		T(r) = T_0 \left(1 - \frac{r_0}{r}\right) \quad \text{with} \quad r_0 = \frac{Gm}{c^2}
	\end{equation}
	
	where $T_0 = \hbar/(mc^2)$ and $r_0$ corresponds to the Schwarzschild radius.
	
	\section{T0-Shor Algorithm Formulation}
	
	\subsection{Geometric Cavity Design for Period Finding}
	
	The T0-Shor algorithm utilizes geometric resonance cavities to detect periods, analogous to acoustic resonators:
	
	\textbf{Resonance cavity dimensions} for period $r$:
	\begin{equation}
		L_{cavity} = n \cdot \frac{\lambda_{T0}}{2} = n \cdot \frac{c_{T0} \cdot r}{2f_0}
	\end{equation}
	
	where $f_0$ is the fundamental driving frequency and $n$ is the mode number.
	
	\textbf{Quality factor} of the resonance:
	\begin{equation}
		Q = \frac{f_r}{\Delta f} = \frac{\pi}{\xi} \cdot \frac{L_{cavity}}{\lambda_{T0}}
	\end{equation}
	
	Higher $Q$ values provide sharper period detection but require longer observation times.
	
	\subsection{Medium-Dependent Algorithm Optimization}
	
	The algorithm efficiency depends critically on the propagation medium:
	
	\textbf{Metallic substrates}:
	\begin{align}
		c_{T0,metal} &= c \sqrt{\frac{\xi_0}{\xi_0 + \sigma/(\omega \epsilon_0)}} \\
		\text{Skin depth: } \delta &= \sqrt{\frac{2}{\omega \mu_0 \sigma}} \\
		\text{Effective cavity size: } L_{eff} &= \min(L_{cavity}, \delta)
	\end{align}
	
	\textbf{Dielectric materials}:
	\begin{align}
		c_{T0,dielectric} &= \frac{c}{\sqrt{\epsilon_r}} \sqrt{\frac{\xi_0}{\xi_{vacuum}}} \\
		\text{Penetration depth: } \delta_p &= \frac{c}{\omega \sqrt{\epsilon_r}} \text{Im}(\sqrt{\epsilon_r}) \\
		\text{Loss tangent: } \tan \delta &= \frac{\epsilon''}{\epsilon'}
	\end{align}
	
	\subsection{Boundary Condition Engineering}
	
	Strategic boundary condition design enhances period detection:
	
	\textbf{Perfect conductor boundaries}:
	\begin{equation}
		T|_{boundary} = 0 \quad \text{(hard boundary)}
	\end{equation}
	
	\textbf{Absorbing boundaries}:
	\begin{equation}
		\frac{\partial T}{\partial n} + i\frac{\omega}{c_{T0}} T = 0 \quad \text{(radiation boundary)}
	\end{equation}
	
	\textbf{Periodic boundaries} for resonance enhancement:
	\begin{equation}
		T(x + L, y, z, t) = T(x, y, z, t) \cdot e^{i k_x L}
	\end{equation}
	
	\subsection{Multi-Mode Resonance Analysis}
	
	Instead of quantum Fourier transform, the T0-Shor algorithm uses multi-mode cavity analysis:
	
	\begin{align}
		\text{Mode spectrum}: \quad &T(x,y,z,t) = \sum_{mnp} A_{mnp}(t) \psi_{mnp}(x,y,z) \\
		\text{Period detection}: \quad &r = \frac{c_{T0}}{2f_{resonance}} \cdot \frac{geometry\_factor}{mode\_number}
	\end{align}
	
	\textbf{Geometry factors for different cavity shapes}:
	\begin{align}
		\text{Rectangular: } G_{rect} &= \sqrt{(m/L_x)^2 + (n/L_y)^2 + (p/L_z)^2} \\
		\text{Cylindrical: } G_{cyl} &= \sqrt{(\chi_{mn}/a)^2 + (p\pi/h)^2} \\
		\text{Spherical: } G_{sph} &= \sqrt{n(n+1)}/R
	\end{align}
	
	\subsection{Adaptive Impedance Matching}
	
	For optimal energy transfer and period detection:
	
	\begin{equation}
		Z_{optimal} = \sqrt{\frac{Z_{source} \cdot Z_{cavity}}{1 + (Q \cdot \Delta f / f_0)^2}}
	\end{equation}
	
	The matching network adjusts the effective mass field distribution:
	\begin{equation}
		m_{matched}(r) = m_0(r) \cdot \frac{Z_{optimal}(r)}{Z_0}
	\end{equation}
	
	\section{Physical Implementation Considerations}
	
	\subsection{Substrate Material Selection}
	
	Different substrate materials provide different T0-field characteristics:
	
	\begin{table}[htbp]
		\centering
		\begin{tabular}{lccccc}
			\toprule
			\textbf{Material} & $\boldsymbol{\epsilon_r}$ & $\boldsymbol{\mu_r}$ & $\boldsymbol{c_{T0}/c}$ & $\boldsymbol{\xi_{eff}/\xi_0}$ & \textbf{Applications} \\
			\midrule
			Vacuum & 1.0 & 1.0 & 1.0 & 1.0 & Reference \\
			Silicon & 11.9 & 1.0 & 0.29 & 0.84 & Electronics \\
			Sapphire & 9.4 & 1.0 & 0.33 & 0.87 & High-Q resonators \\
			GaAs & 12.9 & 1.0 & 0.28 & 0.83 & High-speed devices \\
			Superconductor & $\infty$ & 0 & 0 & $\Delta/(k_B T_c)$ & Lossless cavities \\
			Metamaterial & $< 0$ & $< 0$ & $> 1$ & Tunable & Engineered properties \\
			\bottomrule
		\end{tabular}
		\caption{Material properties for T0-field propagation}
		\label{RSA:L-RSA-1012}
	\end{table}
	
	\subsection{Geometric Optimization}
	
	\textbf{Cavity shape optimization} for maximum period resolution:
	
	For period $r$ detection, the optimal cavity dimensions follow:
	\begin{align}
		\text{Length: } L &= (2n+1) \frac{c_{T0} r}{4 f_0} \quad \text{(quarter-wave resonator)} \\
		\text{Width: } W &= \frac{c_{T0}}{2 f_0} \sqrt{1 - (f_0/f_{cutoff})^2} \\
		\text{Height: } H &= \frac{c_{T0}}{2 f_0} \sqrt{1 - (f_0/f_{cutoff})^2}
	\end{align}
	
	\textbf{Coupling aperture design}:
	\begin{equation}
		A_{aperture} = \frac{\lambda_{T0}^2}{4\pi} \cdot \frac{Q_{external}}{Q_{internal}} \cdot \sin^2\left(\frac{\pi a}{\lambda_{T0}}\right)
	\end{equation}
	
	where $a$ is the aperture dimension.
	
	\subsection{Temperature and Pressure Dependencies}
	
	Environmental conditions affect T0-field propagation:
	
	\textbf{Temperature dependence}:
	\begin{equation}
		c_{T0}(T) = c_{T0}(T_0) \sqrt{\frac{T}{T_0}} \left(1 + \alpha_T \Delta T + \beta_T (\Delta T)^2\right)
	\end{equation}
	
	\textbf{Pressure dependence}:
	\begin{equation}
		\xi(p) = \xi_0 \left(1 + \kappa \frac{\Delta p}{p_0}\right)
	\end{equation}
	
	where $\kappa$ is the pressure coefficient.
	
	\textbf{Thermal noise limitations}:
	\begin{equation}
		S_{thermal}(f) = \frac{4 k_B T R}{(1 + (2\pi f \tau)^2)} \quad \text{with } \tau = \frac{Q}{2\pi f_0}
	\end{equation}
	
	\subsection{Interface Effects and Surface Roughness}
	
	Surface conditions critically affect T0-field behavior:
	
	\textbf{Surface roughness scattering}:
	\begin{equation}
		\tau_{surface} = \frac{4\pi^2}{\lambda_{T0}^2} \langle h^2 \rangle \ell_c
	\end{equation}
	
	where $\langle h^2 \rangle$ is mean-square roughness and $\ell_c$ is correlation length.
	
	\textbf{Interface reflection coefficient}:
	\begin{equation}
		R = \left|\frac{Z_1 \cos\theta_1 - Z_2 \cos\theta_2}{Z_1 \cos\theta_1 + Z_2 \cos\theta_2}\right|^2
	\end{equation}
	
	for oblique incidence at angle $\theta_1$.
	
	\subsection{Scaling Laws for Cavity Arrays}
	
	For enhanced period detection using cavity arrays:
	
	\textbf{Coherent detection in N-cavity array}:
	\begin{equation}
		SNR_{array} = \sqrt{N} \cdot SNR_{single} \cdot \eta_{coupling}
	\end{equation}
	
	where $\eta_{coupling}$ accounts for inter-cavity coupling efficiency.
	
	\textbf{Optimal spacing between cavities}:
	\begin{equation}
		d_{optimal} = \frac{\lambda_{T0}}{2} \sqrt{1 + (Q/\pi)^2}
	\end{equation}
	
	\textbf{Phase coherence length}:
	\begin{equation}
		L_{coherence} = c_{T0} \tau_{coherence} = \frac{c_{T0} Q}{2\pi f_0}
	\end{equation}
	
	\subsection{Resource Requirements}
	
	\begin{table}[htbp]
		\centering
		\begin{tabular}{lcc}
			\toprule
			\textbf{Resource} & \textbf{Standard Shor} & \textbf{T0-Shor} \\
			\midrule
			Quantum bits & $2n + O(\log n)$ & 0 \\
			Energy fields & 0 & $2n$ \\
			Field operations & $O(n^3)$ & $O(n^{2.5})$ \\
			Memory (bits) & $O(n)$ & $O(n)$ \\
			Success probability & $\approx 0.5$ & 1.0 (theoretical) \\
			\bottomrule
		\end{tabular}
		\caption{Theoretical resource comparison for $n$-bit integer factorization}
		\label{RSA:L-RSA-1013}
	\end{table}
	
	\subsection{Efficiency Factor Analysis}
	
	The theoretical efficiency gain depends on the optimization of the mass field:
	
	\begin{equation}
		F(m) = \frac{\left(\int_0^N \sqrt{P(r|N)} \, dr\right)^2}{\int_0^N P(r|N) \, dr}
	\end{equation}
	
	For uniform distribution: $F(m) = N$
	
	For optimal Gaussian distribution with standard deviation $\sigma$:
	\begin{equation}
		F(m) = \sqrt{\frac{\pi}{2}} \cdot \frac{\sigma}{\sqrt{\sigma^2 + \sigma_P^2}}
	\end{equation}
	
	where $\sigma_P$ is the natural width of the period distribution.
	
	\section{The Role of the Parameter}
	
	\subsection{Higgs-Derived Coupling}
	
	The theoretical derivation of $\xi$ from Higgs field interactions provides a physical foundation:
	
	\begin{equation}
		\xi(E) = \xi_0 \cdot \left(\frac{E}{E_0}\right)^{\gamma}
	\end{equation}
	
	where the scaling exponent $\gamma$ depends on the energy regime:
	\begin{align}
		\gamma &\approx 0 \quad \text{for } E < \Lambda_{QCD} \\
		\gamma &\approx 1/2 \quad \text{for } \Lambda_{QCD} < E < \Lambda_{EW} \\
		\gamma &\approx -1/4 \quad \text{for } E > \Lambda_{EW}
	\end{align}
	
	\subsection{Material Dependence}
	
	For electronic systems (typical energy scale $\sim 1$ eV):
	\begin{equation}
		\xi_{electronic} = \xi_0 \cdot \left(\frac{1 \text{ eV}}{246 \text{ GeV}}\right)^{1/2} \approx 10^{-6} \cdot \xi_0
	\end{equation}
	
	Different materials exhibit different effective $\xi$ values:
	\begin{align}
		\xi_{metal} &= \xi_0 / \sqrt{N(E_F)} \\
		\xi_{SC} &= \xi_0 \cdot \Delta/(k_B T_c) \\
		\xi_{semi} &= \xi_0 / \sqrt{m_{eff}/m_e}
	\end{align}
	
	\section{Mathematical Consistency Checks}
	
	\subsection{Conservation Laws}
	
	The T0 framework preserves several important conservation laws:
	
	\textbf{Energy conservation in weighted form}:
	\begin{equation}
		\int |E(x,t)|^2 m(x) \, dx = \text{constant}
	\end{equation}
	
	\textbf{Modified momentum conservation}:
	\begin{equation}
		P = \int E^*(x) \frac{\nabla E(x)}{im(x)} \, dx = \text{constant}
	\end{equation}
	
	\subsection{Scaling Properties}
	
	Under spatial scaling $x \rightarrow \lambda x$:
	\begin{align}
		m(x) &\rightarrow \lambda^{-d} m(x/\lambda) \\
		T(x) &\rightarrow \lambda^d T(x/\lambda) \\
		E(x) &\rightarrow \lambda^{d/2} E(x/\lambda)
	\end{align}
	
	where $d$ is the spatial dimension.
	
	\section{Stability Analysis}
	
	\subsection{Linear Stability}
	
	Consider perturbations around equilibrium solution $m_0(r)$:
	\begin{equation}
		m(r,t) = m_0(r) + \epsilon \delta m(r) e^{\lambda t}
	\end{equation}
	
	Stability requires $\text{Re}(\lambda) < 0$ for all eigenmodes.
	
	The stability matrix for small perturbations is:
	\begin{equation}
		\mathcal{L}[\delta m] = -\frac{\partial^2}{\partial r^2} + V_{eff}(r)
	\end{equation}
	
	where $V_{eff}(r)$ is an effective potential derived from the field equations.
	
	\subsection{Numerical Stability Conditions}
	
	For numerical implementation, stability requires:
	
	\textbf{CFL condition}:
	\begin{equation}
		\Delta t < \frac{\Delta r^2}{\max(1/m(r))}
	\end{equation}
	
	\textbf{Mass gradient constraint}:
	\begin{equation}
		\left|\frac{\nabla m}{m}\right| < \frac{1}{\Delta r}
	\end{equation}
	
	\section{Theoretical Limitations}
	
	\subsection{Information-Theoretic Bounds}
	
	The fundamental search time is bounded by Shannon's entropy:
	\begin{equation}
		T_{min} \geq \frac{H[P(r|N)]}{\log_2(N)}
	\end{equation}
	
	where $H[P]$ is the Shannon entropy of the period distribution.
	
	\subsection{Uncertainty Relations in T0 Framework}
	
	The T0 framework introduces its own uncertainty relation:
	\begin{equation}
		\Delta T \cdot \Delta m \geq \frac{\hbar}{2}
	\end{equation}
	
	This limits simultaneous localization in time and mass parameters.
	
	\subsection{Dependence on A Priori Knowledge}
	
	The efficiency of the T0-Shor algorithm fundamentally depends on the quality of the a priori distribution $P(r|N)$. Without proper knowledge of this distribution, the algorithm reduces to:
	
	\textbf{Worst-case scenario}: Uniform distribution
	\begin{equation}
		F(m)_{uniform} = 1 \quad \text{(no advantage)}
	\end{equation}
	
	\textbf{Best-case scenario}: Perfect prior knowledge
	\begin{equation}
		F(m)_{perfect} = N \quad \text{(maximum advantage)}
	\end{equation}
	
	\section{Comparison with Classical Methods}
	
	\subsection{Theoretical Operation Counts}
	
	\begin{table}[htbp]
		\centering
		\begin{tabular}{lccc}
			\toprule
			\textbf{Method} & \textbf{Operations} & \textbf{Memory} & \textbf{Success Rate} \\
			\midrule
			Trial Division & $O(\sqrt{N})$ & $O(1)$ & 1.0 \\
			Pollard's $\rho$ & $O(N^{1/4})$ & $O(1)$ & High \\
			Quadratic Sieve & $O(\exp(\sqrt{\log N \log \log N}))$ & $O(\sqrt{N})$ & High \\
			General Number Field Sieve & $O(\exp((\log N)^{1/3}(\log \log N)^{2/3}))$ & $O(\exp(\sqrt{\log N}))$ & High \\
			Standard Shor & $O((\log N)^3)$ & $O(\log N)$ & $\approx 0.5$ \\
			T0-Shor (theoretical) & $O((\log N)^{2.5} / F(m))$ & $O(\log N)$ & 1.0 \\
			\bottomrule
		\end{tabular}
		\caption{Theoretical complexity comparison for factoring $N$-bit integers}
		\label{RSA:L-RSA-1014}
	\end{table}
	
	\section{Mathematical Rigor Assessment}
	
	\subsection{Well-Posed Problem Analysis}
	
	The T0 field equations constitute a well-posed problem if:
	
	\begin{enumerate}
		\item \textbf{Existence}: Solutions exist for given boundary conditions
		\item \textbf{Uniqueness}: Solutions are unique
		\item \textbf{Continuous dependence}: Small changes in data produce small changes in solution
	\end{enumerate}
	
	For the field equation (\ref{RSA:L-RSA-1008}), existence and uniqueness follow from standard PDE theory for elliptic equations with appropriate boundary conditions.
	
	\subsection{Dimensional Analysis Verification}
	
	Checking dimensional consistency of the field equation:
	
	\textbf{Left side}: $[\nabla^2 T] = [L^{-2} \cdot T]$
	
	\textbf{Right side}: $[\rho/T^2] = [M L^{-3} \cdot T^{-2}]$
	
	For dimensional consistency, we require:
	\begin{equation}
		[L^{-2} \cdot T] = [M L^{-3} \cdot T^{-2}]
	\end{equation}
	
	This implies the need for a dimensional constant with units $[M^{-1} L T^3]$, which can be related to gravitational coupling.
	
	\section{Conclusion}
	
	\subsection{Summary of Mathematical Analysis}
	
	The T0-Shor algorithm presents a mathematically consistent framework based on:
	
	\begin{enumerate}
		\item Hyperbolic geometry in time-mass duality space
		\item Field equations derived from variational principles
		\item Coupling parameter $\xi$ with theoretical foundation in Higgs physics
		\item Computational complexity that scales as $O(n^{2.5}/F(m))$
	\end{enumerate}
	
	\subsection{Critical Dependencies}
	
	The algorithm's theoretical advantages depend on:
	
	\begin{itemize}
		\item Quality of a priori knowledge about period distribution
		\item Validity of the time-mass duality assumption
		\item Stability of numerical implementations
		\item Physical realizability of adaptive mass fields
	\end{itemize}
	
	\subsection{Open Mathematical Questions}
	
	Several mathematical aspects require further investigation:
	
	\begin{enumerate}
		\item Rigorous proof of convergence for the field evolution equations
		\item Analysis of non-spherically symmetric configurations
		\item Study of chaotic dynamics in the mass field evolution
		\item Connection between $\xi$ parameter and experimentally measurable quantities
	\end{enumerate}
	
	The T0-Shor algorithm represents an interesting theoretical construction that connects concepts from differential geometry, field theory, and computational complexity. However, its practical advantages over existing methods remain contingent on several unproven assumptions about the physical realizability of the underlying mathematical framework.
	
	


% Bibliography
\begin{thebibliography}{99}
	
	\bibitem{pdg2024}
	Particle Data Group Collaboration (2024). 
	\textit{Review of Particle Physics}. 
	Progress of Theoretical and Experimental Physics, 2024(8), 083C01.
	\url{https://pdg.lbl.gov}
	
	\bibitem{flag2024}
	Aoki, Y., et al. (FLAG Collaboration) (2024). 
	\textit{FLAG Review 2024 of Lattice Results for Low-Energy Constants}. 
	arXiv:2411.04268.
	\url{https://arxiv.org/abs/2411.04268}
	
	\bibitem{fermilab_muon_g2}
	Abi, B., et al. (Muon g-2 Collaboration) (2021). 
	\textit{Measurement of the Positive Muon Anomalous Magnetic Moment to 0.46 ppm}. 
	Physical Review Letters, 126, 141801.
	
	\bibitem{peskin_schroeder}
	Peskin, M. E., \& Schroeder, D. V. (1995). 
	\textit{An Introduction to Quantum Field Theory}. 
	Addison-Wesley.
	
	\bibitem{weinberg_qft}
	Weinberg, S. (1995). 
	\textit{The Quantum Theory of Fields, Vol. I--III}. 
	Cambridge University Press.
	
	\bibitem{griffiths_particle}
	Griffiths, D. (2008). 
	\textit{Introduction to Elementary Particles}. 
	Wiley-VCH.
	
	\bibitem{mandl_shaw}
	Mandl, F., \& Shaw, G. (2010). 
	\textit{Quantum Field Theory (2nd ed.)}. 
	Wiley.
	
	\bibitem{srednicki_qft}
	Srednicki, M. (2007). 
	\textit{Quantum Field Theory}. 
	Cambridge University Press.
	
	\bibitem{t0_fundamentals}
	Pascher, J. (2024). 
	\textit{T0-Theory: Foundations of Time-Mass Duality}. 
	Unpublished manuscript, HTL Leonding.
	
	\bibitem{t0_fine_structure}
	Pascher, J. (2024). 
	\textit{T0-Theory: The Fine Structure Constant}. 
	Unpublished manuscript, HTL Leonding.
	
	\bibitem{t0_neutrinos}
	Pascher, J. (2024). 
	\textit{T0-Theory: Neutrino Masses and PMNS Mixing}. 
	Unpublished manuscript, HTL Leonding.
	
	\bibitem{t0_github}
	Pascher, J. (2024--2025). 
	\textit{T0-Time-Mass-Duality Repository}. 
	GitHub.
	\url{https://github.com/jpascher/T0-Time-Mass-Duality}
	
	\bibitem{lattice_qcd_review}
	Kronfeld, A. S. (2012). 
	\textit{Twenty-first Century Lattice Gauge Theory: Results from the QCD Lagrangian}. 
	Annual Review of Nuclear and Particle Science, 62, 265--284.
	
	\bibitem{neutrino_mixing_pdg}
	Particle Data Group Collaboration (2024). 
	\textit{Neutrino Masses, Mixing, and Oscillations}. 
	PDG Review 2024.
	\url{https://pdg.lbl.gov/2024/reviews/rpp2024-rev-neutrino-mixing.pdf}
	
	\bibitem{higgs_discovery}
	ATLAS and CMS Collaborations (2012). 
	\textit{Observation of a New Particle in the Search for the Standard Model Higgs Boson}. 
	Physics Letters B, 716, 1--29.
	
	\bibitem{Brannen2005}
	C. P. Brannen, ``Estimate of neutrino masses from Koide's relation'', \textit{arXiv:hep-ph/0505028} (2005).
	\url{https://arxiv.org/abs/hep-ph/0505028}
	
	\bibitem{Brannen2006}
	C. P. Brannen, ``Koide Mass Formula for Neutrinos'', \textit{arXiv:0702.0052} (2006).
	\url{http://brannenworks.com/MASSES.pdf}
	
	\bibitem{PhaseVectors2025}
	Anonymous, ``The Koide Relation and Lepton Mass Hierarchy from Phase Vectors'', \textit{rXiv:2507.0040} (2025).
	\url{https://rxiv.org/pdf/2507.0040v1.pdf}
	
	\bibitem{PDG2025}
	Particle Data Group, ``Review of Particle Physics'', \textit{Phys. Rev. D} \textbf{112} (2025) 030001.
	\url{https://pdg.lbl.gov/2025/}
	
	\bibitem{terrell2024}
	Terrell et al. (2024). 
	\textit{Single-Clock Metrology in Nature}. 
	Nature Physics.
	
	\bibitem{hossenfelder2024}
	Hossenfelder, S. (2024). 
	\textit{Single Clock Video Explanation}. 
	YouTube.
	
	\bibitem{hundert1931}
	Hundert (1931). 
	\textit{Reference Work}. 
	Publisher.
	
	\bibitem{terrell2025}
	Terrell et al. (2025). 
	\textit{Advanced Clock Synchronization Methods}. 
	Physical Review Letters.
	
	\bibitem{pascher_t0_2025}
	Pascher, J. (2025). 
	\textit{T0-Theory: Complete Framework and Applications}. 
	Unpublished manuscript, HTL Leonding.
	
	\bibitem{t0qm}
	Pascher, J. (2024). 
	\textit{T0-Theory: Quantum Mechanics Formulation}. 
	Unpublished manuscript, HTL Leonding.
	
	\bibitem{t0anomale}
	Pascher, J. (2024). 
	\textit{T0-Theory: Anomalous Magnetic Moments}. 
	Unpublished manuscript, HTL Leonding.
	
	\bibitem{muong2complete}
	Abi, B., et al. (Muon g-2 Collaboration) (2023). 
	\textit{Complete Measurement of the Positive Muon Anomalous Magnetic Moment}. 
	Physical Review Letters, 131, 161802.
	
	\bibitem{penrose2004}
	Penrose, R. (2004). 
	\textit{The Road to Reality: A Complete Guide to the Laws of the Universe}. 
	Jonathan Cape.
	
	\bibitem{planck1900}
	Planck, M. (1900). 
	\textit{On the Theory of the Energy Distribution Law of the Normal Spectrum}. 
	Verhandlungen der Deutschen Physikalischen Gesellschaft, 2, 237.
	
	\bibitem{T0Theory}
	Pascher, J. (2024). 
	\textit{T0-Theory: Fundamental Principles}. 
	Unpublished manuscript, HTL Leonding.
	
	% Additional bibliography entries for all undefined citations
	\bibitem{6g_roadmap}
	6G Research Consortium (2024).
	\textit{6G Technology Roadmap}.
	Technical Report.
	
	\bibitem{Born2013}
	Born, M. (2013).
	\textit{Einstein's Theory of Relativity}.
	Dover Publications.
	
	\bibitem{Casimir1948}
	Casimir, H. B. G. (1948).
	\textit{On the attraction between two perfectly conducting plates}.
	Proc. Kon. Ned. Akad. Wetensch. B51, 793--795.
	
	\bibitem{Einstein1905}
	Einstein, A. (1905).
	\textit{On the Electrodynamics of Moving Bodies}.
	Annalen der Physik, 17, 891--921.
	
	\bibitem{Feynman2006}
	Feynman, R. P. (2006).
	\textit{QED: The Strange Theory of Light and Matter}.
	Princeton University Press.
	
	\bibitem{Griffiths2017}
	Griffiths, D. J. (2017).
	\textit{Introduction to Electrodynamics (4th ed.)}.
	Cambridge University Press.
	
	\bibitem{Jackson1999}
	Jackson, J. D. (1999).
	\textit{Classical Electrodynamics (3rd ed.)}.
	Wiley.
	
	\bibitem{Mohr2016}
	Mohr, P. J., et al. (2016).
	\textit{CODATA Recommended Values of the Fundamental Physical Constants: 2014}.
	Rev. Mod. Phys. 88, 035009.
	
	\bibitem{Parker2018}
	Parker, R. H., et al. (2018).
	\textit{Measurement of the fine-structure constant as a test of the Standard Model}.
	Science, 360, 191--195.
	
	\bibitem{Planck1900}
	Planck, M. (1900).
	\textit{On the Theory of the Energy Distribution Law of the Normal Spectrum}.
	Verhandlungen der Deutschen Physikalischen Gesellschaft, 2, 237.
	
	\bibitem{Planck2018}
	Planck Collaboration (2018).
	\textit{Planck 2018 results. VI. Cosmological parameters}.
	Astronomy \& Astrophysics, 641, A6.
	
	\bibitem{QFT_T0}
	Pascher, J. (2024).
	\textit{T0-Theory and QFT Connections}.
	Unpublished manuscript, HTL Leonding.
	
	\bibitem{Sommerfeld1916}
	Sommerfeld, A. (1916).
	\textit{On the Quantum Theory of Spectral Lines}.
	Annalen der Physik, 51, 1--94.
	
	\bibitem{T0_Feinstruktur}
	Pascher, J. (2024).
	\textit{T0-Theory: Fine Structure Analysis}.
	Unpublished manuscript, HTL Leonding.
	
	\bibitem{T0_SI}
	Pascher, J. (2024).
	\textit{T0-Theory and SI Units}.
	Unpublished manuscript, HTL Leonding.
	
	\bibitem{T0_fine_structure}
	Pascher, J. (2024).
	\textit{T0-Theory: The Fine Structure Constant}.
	Unpublished manuscript, HTL Leonding.
	
	\bibitem{T0_g2_erweiterung}
	Pascher, J. (2024).
	\textit{T0-Theory: g-2 Extensions}.
	Unpublished manuscript, HTL Leonding.
	
	\bibitem{T0_gravitational_constant}
	Pascher, J. (2024).
	\textit{T0-Theory: Gravitational Constant Derivation}.
	Unpublished manuscript, HTL Leonding.
	
	\bibitem{T0_netze_en}
	Pascher, J. (2024).
	\textit{T0-Theory: Network Structures}.
	Unpublished manuscript, HTL Leonding.
	
	\bibitem{T0_tm_erweiterung}
	Pascher, J. (2024).
	\textit{T0-Theory: Time-Mass Extensions}.
	Unpublished manuscript, HTL Leonding.
	
	\bibitem{Uzan2003}
	Uzan, J.-P. (2003).
	\textit{The fundamental constants and their variation}.
	Rev. Mod. Phys. 75, 403--455.
	
	\bibitem{Weinberg1995}
	Weinberg, S. (1995).
	\textit{The Quantum Theory of Fields, Vol. I}.
	Cambridge University Press.
	
	\bibitem{albrecht1999}
	Albrecht, A. \& Magueijo, J. (1999).
	\textit{A time varying speed of light as a solution to cosmological puzzles}.
	Phys. Rev. D 59, 043516.
	
	\bibitem{alice2023}
	ALICE Collaboration (2023).
	\textit{Recent results from ALICE}.
	CERN-EP-2023-XXX.
	
	\bibitem{analog_optical}
	Smith, J. et al. (2024).
	\textit{Analog optical computing systems}.
	Nature Photonics.
	
	\bibitem{ashtekar2004}
	Ashtekar, A. \& Lewandowski, J. (2004).
	\textit{Background independent quantum gravity}.
	Class. Quantum Grav. 21, R53.
	
	\bibitem{atlas2023}
	ATLAS Collaboration (2023).
	\textit{ATLAS physics results}.
	CERN-PH-EP-2023-XXX.
	
	\bibitem{atlas2023higgs}
	ATLAS Collaboration (2023).
	\textit{Higgs boson measurements}.
	Phys. Rev. Lett.
	
	\bibitem{barbour1999}
	Barbour, J. (1999).
	\textit{The End of Time}.
	Oxford University Press.
	
	\bibitem{barrow1999}
	Barrow, J. D. (1999).
	\textit{Cosmologies with varying light speed}.
	Phys. Rev. D 59, 043515.
	
	\bibitem{becker2007}
	Becker, K. et al. (2007).
	\textit{String Theory and M-Theory}.
	Cambridge University Press.
	
	\bibitem{bell_muon}
	Bennett, G. W., et al. (Muon g-2 Collaboration) (2006).
	\textit{Final report of the E821 muon anomalous magnetic moment measurement}.
	Phys. Rev. D 73, 072003.
	
	\bibitem{bondi1948}
	Bondi, H. \& Gold, T. (1948).
	\textit{The steady-state theory of the expanding universe}.
	Mon. Not. R. Astron. Soc. 108, 252--270.
	
	\bibitem{brewer2019}
	Brewer, S. M. et al. (2019).
	\textit{Al+ Quantum-Logic Clock with Systematic Uncertainty below $10^{-18}$}.
	Phys. Rev. Lett. 123, 033201.
	
	\bibitem{cms2023top}
	CMS Collaboration (2023).
	\textit{Top quark measurements at CMS}.
	JHEP 2023.
	
	\bibitem{cms2024}
	CMS Collaboration (2024).
	\textit{CMS physics results 2024}.
	CERN-PH-EP-2024-XXX.
	
	\bibitem{codata2019}
	Tiesinga, E. et al. (2019).
	\textit{The 2018 CODATA Recommended Values}.
	J. Phys. Chem. Ref. Data.
	
	\bibitem{desi2025}
	DESI Collaboration (2025).
	\textit{DESI 2025 Cosmology Results}.
	arXiv preprint.
	
	\bibitem{differential_optical}
	Wang, X. et al. (2024).
	\textit{Differential optical computing}.
	Optica.
	
	\bibitem{dingle1972}
	Dingle, H. (1972).
	\textit{Science at the Crossroads}.
	Martin Brian \& O'Keeffe.
	
	\bibitem{divalentino2021}
	Di Valentino, E. et al. (2021).
	\textit{In the realm of the Hubble tension}.
	Class. Quantum Grav. 38, 153001.
	
	\bibitem{elnaschie2004}
	El Naschie, M. S. (2004).
	\textit{A review of E infinity theory}.
	Chaos, Solitons \& Fractals, 19, 209--236.
	
	\bibitem{fabrication_heterogeneous}
	Chen, Y. et al. (2024).
	\textit{Heterogeneous photonic integration}.
	Nature Electronics.
	
	\bibitem{fermilab2023}
	Fermilab (2023).
	\textit{Muon g-2 results}.
	Phys. Rev. Lett.
	
	\bibitem{flexible_wafer}
	Kim, S. et al. (2024).
	\textit{Flexible wafer-scale photonics}.
	Science Advances.
	
	\bibitem{francesco1997}
	Di Francesco, P. et al. (1997).
	\textit{Conformal Field Theory}.
	Springer.
	
	\bibitem{hartree1957}
	Hartree, D. R. (1957).
	\textit{The Calculation of Atomic Structures}.
	Wiley.
	
	\bibitem{hhi_6g}
	Fraunhofer HHI (2024).
	\textit{6G Photonic Integration}.
	Technical Report.
	
	\bibitem{hossenfelder2025}
	Hossenfelder, S. (2025).
	\textit{Science without the gobbledygook}.
	YouTube/Blog.
	
	\bibitem{hossenfelder_single_clock_video}
	Hossenfelder, S. (2024).
	\textit{The Single Clock Problem}.
	YouTube.
	
	\bibitem{hoyle1948}
	Hoyle, F. (1948).
	\textit{A new model for the expanding universe}.
	Mon. Not. R. Astron. Soc. 108, 372--382.
	
	\bibitem{integration_microelectronic}
	Liu, A. et al. (2024).
	\textit{Microelectronic photonic integration}.
	IEEE Journal.
	
	\bibitem{jacobson1995}
	Jacobson, T. (1995).
	\textit{Thermodynamics of spacetime}.
	Phys. Rev. Lett. 75, 1260.
	
	\bibitem{kasevich2023}
	Kasevich, M. et al. (2023).
	\textit{Atom interferometry tests}.
	Nature Physics.
	
	\bibitem{lerner2014}
	Lerner, E. J. (2014).
	\textit{An open letter on cosmology}.
	New Scientist.
	
	\bibitem{lisa2017}
	LISA Consortium (2017).
	\textit{Laser Interferometer Space Antenna}.
	ESA Technical Report.
	
	\bibitem{lithium_tantalate}
	Zhang, M. et al. (2024).
	\textit{Thin-film lithium tantalate photonics}.
	Nature Photonics.
	
	\bibitem{lopez2010}
	Lopez-Corredoira, M. (2010).
	\textit{Tests and problems of the standard model in cosmology}.
	Int. J. Mod. Phys. D.
	
	\bibitem{ludlow2015}
	Ludlow, A. D. et al. (2015).
	\textit{Optical atomic clocks}.
	Rev. Mod. Phys. 87, 637.
	
	\bibitem{mach1883}
	Mach, E. (1883).
	\textit{Die Mechanik in ihrer Entwickelung}.
	F.A. Brockhaus.
	
	\bibitem{maldacena1998}
	Maldacena, J. (1998).
	\textit{The large N limit of superconformal field theories}.
	Adv. Theor. Math. Phys. 2, 231--252.
	
	\bibitem{mueller2014}
	Müller, H. et al. (2014).
	\textit{Atom interferometry tests of the gravitational redshift}.
	Phys. Rev. Lett.
	
	\bibitem{mug2_final_2025}
	Muon g-2 Collaboration (2025).
	\textit{Final muon g-2 measurement}.
	Phys. Rev. Lett.
	
	\bibitem{muong2_2023}
	Muon g-2 Collaboration (2023).
	\textit{Updated muon g-2 results}.
	Phys. Rev. Lett.
	
	\bibitem{nathan2024}
	Nathan, A. et al. (2024).
	\textit{Quantum computing advances}.
	Nature.
	
	\bibitem{newell2018}
	Newell, D. B. et al. (2018).
	\textit{The CODATA 2017 values of h, e, k, and $N_A$}.
	Metrologia 55, L13.
	
	\bibitem{nottale1993}
	Nottale, L. (1993).
	\textit{Fractal Space-Time and Microphysics}.
	World Scientific.
	
	\bibitem{on_chip_lithium}
	Wang, C. et al. (2024).
	\textit{On-chip lithium niobate photonics}.
	Nature Communications.
	
	\bibitem{optical_advantages}
	Shastri, B. J. et al. (2024).
	\textit{Advantages of optical computing}.
	Nature Reviews Physics.
	
	\bibitem{pascher2025cmb}
	Pascher, J. (2025).
	\textit{T0-Theory: CMB Analysis}.
	Unpublished manuscript, HTL Leonding.
	
	\bibitem{pascher2025g2}
	Pascher, J. (2025).
	\textit{T0-Theory: g-2 Predictions}.
	Unpublished manuscript, HTL Leonding.
	
	\bibitem{pascher2025qm}
	Pascher, J. (2025).
	\textit{T0-Theory: Quantum Mechanics}.
	Unpublished manuscript, HTL Leonding.
	
	\bibitem{pascher2025si}
	Pascher, J. (2025).
	\textit{T0-Theory: SI Unit System}.
	Unpublished manuscript, HTL Leonding.
	
	\bibitem{pascher2025t0}
	Pascher, J. (2025).
	\textit{T0-Theory: Complete Framework}.
	Unpublished manuscript, HTL Leonding.
	
	\bibitem{pascher:fundamentals}
	Pascher, J. (2024).
	\textit{T0-Theory: Fundamentals}.
	Unpublished manuscript, HTL Leonding.
	
	\bibitem{pascher:g2_rev9}
	Pascher, J. (2024).
	\textit{T0-Theory: g-2 Revision 9}.
	Unpublished manuscript, HTL Leonding.
	
	\bibitem{pascher:geometric_formalism}
	Pascher, J. (2024).
	\textit{T0-Theory: Geometric Formalism}.
	Unpublished manuscript, HTL Leonding.
	
	\bibitem{pascher:ml_addendum}
	Pascher, J. (2024).
	\textit{T0-Theory: Machine Learning Addendum}.
	Unpublished manuscript, HTL Leonding.
	
	\bibitem{pascher:t0_foundations}
	Pascher, J. (2024).
	\textit{T0-Theory: Foundations}.
	Unpublished manuscript, HTL Leonding.
	
	\bibitem{pascher_derivation_beta_2025}
	Pascher, J. (2025).
	\textit{T0-Theory: Derivation of Beta}.
	Unpublished manuscript, HTL Leonding.
	
	\bibitem{pascher_higgs_connection_2025}
	Pascher, J. (2025).
	\textit{T0-Theory: Higgs Connection}.
	Unpublished manuscript, HTL Leonding.
	
	\bibitem{pascher_lagrangian_extended_2025}
	Pascher, J. (2025).
	\textit{T0-Theory: Extended Lagrangian}.
	Unpublished manuscript, HTL Leonding.
	
	\bibitem{pascher_mathematical_structure_2025}
	Pascher, J. (2025).
	\textit{T0-Theory: Mathematical Structure}.
	Unpublished manuscript, HTL Leonding.
	
	\bibitem{pascher_t0_cmb_2025}
	Pascher, J. (2025).
	\textit{T0-Theory: CMB Predictions}.
	Unpublished manuscript, HTL Leonding.
	
	\bibitem{pascher_t0_energie_2025}
	Pascher, J. (2025).
	\textit{T0-Theory: Energy}.
	Unpublished manuscript, HTL Leonding.
	
	\bibitem{pascher_t0_energy_2025}
	Pascher, J. (2025).
	\textit{T0-Theory: Energy Framework}.
	Unpublished manuscript, HTL Leonding.
	
	\bibitem{pascher_t0_theory_2025}
	Pascher, J. (2025).
	\textit{T0-Theory: Complete Theory}.
	Unpublished manuscript, HTL Leonding.
	
	\bibitem{penrose1959}
	Penrose, R. (1959).
	\textit{The apparent shape of a relativistically moving sphere}.
	Proc. Cambridge Phil. Soc. 55, 137--139.
	
	\bibitem{penrose1967}
	Penrose, R. (1967).
	\textit{Twistor algebra}.
	J. Math. Phys. 8, 345--366.
	
	\bibitem{peratt1992}
	Peratt, A. L. (1992).
	\textit{Physics of the Plasma Universe}.
	Springer-Verlag.
	
	\bibitem{peskin1995}
	Peskin, M. E. \& Schroeder, D. V. (1995).
	\textit{An Introduction to Quantum Field Theory}.
	Addison-Wesley.
	
	\bibitem{peskin_schroeder_1995}
	Peskin, M. E. \& Schroeder, D. V. (1995).
	\textit{An Introduction to Quantum Field Theory}.
	Addison-Wesley.
	
	\bibitem{phoquant}
	PhoQuant (2024).
	\textit{Photonic quantum computing}.
	Technical Report.
	
	\bibitem{photonics_ai}
	Wetzstein, G. et al. (2024).
	\textit{Photonics for AI}.
	Nature.
	
	\bibitem{planck1906}
	Planck, M. (1906).
	\textit{The Theory of Heat Radiation}.
	Johann Ambrosius Barth.
	
	\bibitem{planck2018}
	Planck Collaboration (2018).
	\textit{Planck 2018 results}.
	A\&A 641, A6.
	
	\bibitem{polchinski1998}
	Polchinski, J. (1998).
	\textit{String Theory}.
	Cambridge University Press.
	
	\bibitem{qant_nps}
	QANT (2024).
	\textit{Quantum photonics systems}.
	Technical Report.
	
	\bibitem{quantenjahr25}
	Quantenjahr (2025).
	\textit{International Year of Quantum}.
	UNESCO.
	
	\bibitem{recurrent_photonics}
	Tait, A. N. et al. (2024).
	\textit{Recurrent photonic neural networks}.
	Optica.
	
	\bibitem{rf_photonics}
	Capmany, J. \& Novak, D. (2024).
	\textit{Microwave photonics}.
	Nature Photonics.
	
	\bibitem{riess2019}
	Riess, A. G. et al. (2019).
	\textit{Large Magellanic Cloud Cepheid Standards}.
	ApJ 876, 85.
	
	\bibitem{riess2022}
	Riess, A. G. et al. (2022).
	\textit{A Comprehensive Measurement of H0}.
	ApJ 934, L7.
	
	\bibitem{rovelli2004}
	Rovelli, C. (2004).
	\textit{Quantum Gravity}.
	Cambridge University Press.
	
	\bibitem{sciama1953}
	Sciama, D. W. (1953).
	\textit{On the origin of inertia}.
	Mon. Not. R. Astron. Soc. 113, 34--42.
	
	\bibitem{sciencedaily2025}
	ScienceDaily (2025).
	\textit{Physics news}.
	Online.
	
	\bibitem{sm_g2_2025}
	Aoyama, T. et al. (2025).
	\textit{Standard Model prediction for g-2}.
	Phys. Rep.
	
	\bibitem{susskind1995}
	Susskind, L. (1995).
	\textit{The world as a hologram}.
	J. Math. Phys. 36, 6377--6396.
	
	\bibitem{t0_kosmologie}
	Pascher, J. (2024).
	\textit{T0-Theory: Cosmology}.
	Unpublished manuscript, HTL Leonding.
	
	\bibitem{terrell1959}
	Terrell, J. (1959).
	\textit{Invisibility of the Lorentz contraction}.
	Phys. Rev. 116, 1041--1045.
	
	\bibitem{terrell_single_clock_nature_2024}
	Terrell, J. et al. (2024).
	\textit{Single clock precision measurements}.
	Nature Physics.
	
	\bibitem{tfln_foundry}
	TFLN Foundry (2024).
	\textit{Thin-film lithium niobate foundry services}.
	Technical Specifications.
	
	\bibitem{thiemann2007}
	Thiemann, T. (2007).
	\textit{Modern Canonical Quantum General Relativity}.
	Cambridge University Press.
	
	\bibitem{thz_epfl}
	EPFL (2024).
	\textit{Terahertz photonics research}.
	Technical Report.
	
	\bibitem{unnikrishnan2004}
	Unnikrishnan, C. S. (2004).
	\textit{On Einstein's resolution of the twin clock paradox}.
	Current Science, 86, 704--709.
	
	\bibitem{verlinde2011}
	Verlinde, E. (2011).
	\textit{On the origin of gravity and the laws of Newton}.
	JHEP 2011, 29.
	
	\bibitem{video2025}
	Video (2025).
	\textit{Physics video explanation}.
	YouTube.
	
	\bibitem{weinberg1995}
	Weinberg, S. (1995).
	\textit{The Quantum Theory of Fields}.
	Cambridge University Press.
	
	\bibitem{weiskopf2000}
	Weiskopf, D. (2000).
	\textit{Visualization of special relativity}.
	PhD thesis, University of Tübingen.
	
	\bibitem{wheeler1990}
	Wheeler, J. A. (1990).
	\textit{A Journey into Gravity and Spacetime}.
	Scientific American Library.
	
	\bibitem{wiki_bell}
	Wikipedia (2024).
	\textit{Bell's theorem}.
	Online encyclopedia.
	
	\bibitem{zwicky1929}
	Zwicky, F. (1929).
	\textit{On the red shift of spectral lines through interstellar space}.
	Proc. Natl. Acad. Sci. 15, 773--779.

\end{thebibliography}


\end{document}
