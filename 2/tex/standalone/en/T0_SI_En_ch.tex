\documentclass[11pt,a4paper]{article}
\usepackage[a4paper,margin=2cm]{geometry}
\usepackage[utf8]{inputenc}
\usepackage[english]{babel}
\usepackage{lmodern}
\renewcommand{\familydefault}{\sfdefault}

\usepackage{amsmath,amssymb,amsthm}
\usepackage{graphicx}
\usepackage[unicode,pdfencoding=auto,hypertexnames=false]{hyperref}
\usepackage{booktabs}
\usepackage{longtable}
\usepackage{array}
\usepackage{siunitx}
\usepackage{fancyhdr}
\usepackage{float}
\usepackage{tikz}
% tcolorbox removed for standalone
% tcbset removed
\tikzset{
  t0blue/.style={draw=blue,fill=blue!10},
  t0red/.style={draw=red,fill=red!10},
  t0green/.style={draw=green!50!black,fill=green!10},
  t0orange/.style={draw=orange,fill=orange!10},
}
\usepackage{setspace}
\usepackage{enumitem}
\usepackage{adjustbox}
\usepackage{xcolor}

% Define colors for xcolor package
\definecolor{t0green}{RGB}{34,139,34}
\definecolor{t0blue}{RGB}{0,0,255}
\definecolor{t0red}{RGB}{255,0,0}
\definecolor{t0orange}{RGB}{255,165,0}

% Define custom column types for tables
\newcolumntype{L}[1]{>{\raggedright\arraybackslash}p{#1}}
\newcolumntype{C}[1]{>{\centering\arraybackslash}p{#1}}
\newcolumntype{R}[1]{>{\raggedleft\arraybackslash}p{#1}}

\setlength{\parindent}{0pt}
\setlength{\parskip}{6pt}

\hypersetup{
  colorlinks=true,
  linkcolor=blue,
  citecolor=blue,
  urlcolor=blue
}
\pagestyle{fancy}
\setlength{\headheight}{28pt}

\newcommand{\checkmarkx}{\checkmark}
\newcommand{\warningx}{\textbf{!}}

% Makros aus Einzel-Dokumenten (Fallback-Definitionen)
\newcommand{\mytimes}{\times}
\newcommand{\myapprox}{\approx}
\newcommand{\mysim}{\sim}
\newcommand{\myomega}{\omega}
\newcommand{\mypi}{\pi}
\newcommand{\myrightarrow}{\rightarrow}
\newcommand{\mypropto}{\propto}
\newcommand{\deltafield}{\delta\phi}
\newcommand{\xipar}{\xi}
\newcommand{\xiT}{\xi}
\newcommand{\lambdah}{\lambda_h}

% Additional macros used in chapter files
\newcommand{\Kfrak}{K_{\text{frak}}}  % Fractal correction factor
\newcommand{\Dfrak}{D_f}              % Fractal dimension
\newcommand{\betapar}{\beta}          % T0 beta parameter
\newcommand{\alphapar}{\alpha}        % T0 alpha parameter
\newcommand{\Efield}{E}               % Energy field
% Note: checkmarkxa/warningxa are variants used in auto-generated chapter files
\newcommand{\checkmarkxa}{\checkmark}
\newcommand{\warningxa}{\textbf{!}}

% Additional T0-specific macros
\newcommand{\xigeom}{\xi_{\text{geom}}}  % Geometric xi
\newcommand{\lP}{\ell_P}                  % Planck length
\newcommand{\rzero}{r_0}                  % Characteristic radius
\newcommand{\xirat}{\xi_{\text{rat}}}     % Xi ratio
\newcommand{\tzero}{t_0}                  % Characteristic time
\newcommand{\natunits}{\text{(nat. units)}}  % Natural units annotation
\newcommand{\myRightarrow}{\Rightarrow}   % Arrow variant
\newcommand{\Lag}{\mathcal{L}}            % Lagrangian

% Physics macros used in chapter files
\newcommand{\CQCD}{C_{\text{QCD}}}        % QCD correction
\newcommand{\EP}{E_P}                     % Planck energy
\newcommand{\Ee}{E_e}                     % Electron energy
\newcommand{\Emu}{E_\mu}                  % Muon energy
\newcommand{\Exi}{E_\xi}                  % Xi energy
\newcommand{\Ezero}{E_0}                  % Characteristic energy
\newcommand{\Hubble}{H}                   % Hubble constant
\newcommand{\Kspec}{K_{\text{spec}}}      % Spectral correction
\newcommand{\Lambdat}{\Lambda_t}          % Time-related cosmological constant
\newcommand{\Leff}{\mathcal{L}_{\text{eff}}}  % Effective Lagrangian
\newcommand{\Lorentz}{\mathcal{L}}        % Lorentz symbol
\newcommand{\Lxi}{L_\xi}                  % Xi length
\newcommand{\Tfield}{T}                   % Time field
\newcommand{\Weyl}{W}                     % Weyl tensor/symbol
\newcommand{\alphaEMSI}{\alpha_{\text{EM,SI}}}  % EM alpha in SI
\newcommand{\alphaEMnat}{\alpha_{\text{EM,nat}}}  % EM alpha in natural units
\newcommand{\alphaem}{\alpha_{\text{em}}} % Electromagnetic alpha
\newcommand{\betaTSI}{\beta_{T,\text{SI}}}  % Beta in SI
\newcommand{\betaTnat}{\beta_{T,\text{nat}}}  % Beta in natural units
\newcommand{\deltam}{\delta m}            % Mass difference
\newcommand{\phiT}{\phi_T}                % T-field phi
\newcommand{\tP}{t_P}                     % Planck time
\newcommand{\rhoCMB}{\rho_{\text{CMB}}}   % CMB density
\newcommand{\rhoCasimir}{\rho_{\text{Casimir}}}  % Casimir density

% Table formatting
\usepackage{multirow}

% Additional physics macros
\newcommand{\Riem}{\mathcal{R}}           % Riemann tensor
\newcommand{\ZPinch}{Z_{\text{pinch}}}    % Z-pinch
\newcommand{\SynchPower}{P_{\text{synch}}} % Synchrotron power
\newcommand{\Rzero}{R_0}                  % Characteristic radius
\newcommand{\alphafine}{\alpha}           % Fine structure constant
\newcommand{\Etau}{E_\tau}                % Tau energy
\newcommand{\deltaE}{\delta E}            % Energy deviation
\newcommand{\EPlanck}{E_P}                % Planck energy
\newcommand{\pichar}{\pi}                 % Pi character
\newcommand{\alphaWSI}{\alpha_{W,\text{SI}}}  % Wien alpha in SI
\newcommand{\alphaWnat}{\alpha_{W,\text{nat}}}  % Wien alpha in natural units

% Einfache abstract-Umgebung für Kapitel:
\newenvironment{abstract}{%
  \begin{center}\bfseries Abstract\end{center}\small
}{\par}


\title{T0 SI En}
\author{J. Pascher}
\date{\today}

\begin{document}
\maketitle

\section*{T0 Si (T0 SI)}

	\begin{abstract}
		T0-Theory achieves complete parameter freedom: Only the geometric parameter $\xi = \frac{4}{3} \times 10^{-4}$ is fundamental. All physical constants are either derived from $\xi$ or represent unit definitions. This document provides the complete derivation chain including the gravitational constant $G$, the Planck length $l_P$, and the Boltzmann constant $k_B$. The SI reform 2019 unknowingly implemented the unique calibration that is consistent with this geometric foundation.
	\end{abstract}
	
	
	\section{The Geometric Foundation}
	
	\subsection{Single Fundamental Parameter}
	
	\begin{equation}
		\boxed{\xi = \frac{4}{3} \times 10^{-4}}
	\end{equation}
	
	This geometric ratio encodes the fundamental structure of three-dimensional space. All physical quantities emerge as derivable consequences.
	
	\subsection{Complete Derivation Framework}
	
	Detailed mathematical derivations are available at:
	
	\begin{center}
		\url{https://github.com/jpascher/T0-Time-Mass-Duality/tree/main/2/pdf}
	\end{center}
	
	\section{Derivation of the Gravitational Constant from}
	
	\subsection{The Fundamental T0 Gravitational Relation}
	
\section*{Derivation}
\section*{Starting point of T0 gravity theory:}
		
		T0-Theory postulates a fundamental geometric relationship between the characteristic length parameter $\xi$ and the gravitational constant:
		
		\begin{equation}
			\xi = 2\sqrt{G \cdot m_{\text{char}}}
			\label{T0_SI:L-T0_Gravitationskonstante-0166}
		\end{equation}
		
		where $m_{\text{char}}$ represents a characteristic mass of the theory.
		
\section*{Physical interpretation:}
		\begin{itemize}
			\item $\xi$ encodes the geometric structure of space
			\item $G$ describes the coupling between geometry and matter
			\item $m_{\text{char}}$ sets the characteristic mass scale
		\end{itemize}
% end box derivation
	
	\subsection{Resolution for the Gravitational Constant}
	
	Solving equation \eqref{L-T0_Gravitationskonstante-0166} for $G$:
	
	\begin{equation}
		\boxed{G = \frac{\xi^2}{4 m_{\text{char}}}}
		\label{T0_SI:L-T0_Gravitationskonstante-0167}
	\end{equation}
	
	This is the fundamental T0 relationship for the gravitational constant in natural units.
	
	\subsection{Choice of Characteristic Mass}
	
\section*{Insight}
\section*{The electron mass is also derived from $\xi$:}
		
		T0-Theory uses the electron mass as the characteristic scale:
		\begin{equation}
			m_{\text{char}} = m_e = 0.511 \text{ MeV}
			\label{T0_SI:L-T0_Gravitationskonstante-0168}
		\end{equation}
		
		\textbf{Critical point:} The electron mass itself is not an independent parameter, but is derived from $\xi$ through the T0 mass quantization formula:
		\begin{equation}
			m_e = \frac{f(1,0,1/2)^2}{\xi^2} \cdot S_{T0}
		\end{equation}
		
		where $f(n,l,j)$ is the geometric quantum number factor and $S_{T0} = 1$ MeV/$c^2$ is the predicted scaling factor.
		
		Therefore, the entire derivation chain $\xi \to m_e \to G \to l_P$ depends only on $\xi$ as the single fundamental input.
% end box insight
	
	\subsection{Dimensional Analysis in Natural Units}
	
\section*{Derivation}
\section*{Dimensional check in natural units ($\hbar = c = 1$):}
		
		In natural units:
		\begin{align}
			[M] &= [E] \quad \text{(from } E = mc^2 \text{ with } c = 1\text{)} \\
			[L] &= [E^{-1}] \quad \text{(from } \lambda = \hbar/p \text{ with } \hbar = 1\text{)} \\
			[T] &= [E^{-1}] \quad \text{(from } \omega = E/\hbar \text{ with } \hbar = 1\text{)}
		\end{align}
		
		The gravitational constant has the dimension:
		\begin{equation}
			[G] = [M^{-1}L^3T^{-2}] = [E^{-1}][E^{-3}][E^2] = [E^{-2}]
		\end{equation}
		
		Checking equation \eqref{L-T0_Gravitationskonstante-0167}:
		\begin{equation}
			[G] = \frac{[\xi^2]}{[m_e]} = \frac{[1]}{[E]} = [E^{-1}] \neq [E^{-2}]
		\end{equation}
		
		This shows that additional factors are required for dimensional correctness.
% end box derivation
	
	\subsection{Complete Formula with Conversion Factors}
	
\section*{Key Result}
\section*{Complete gravitational constant formula:}
		
		\begin{equation}
			\boxed{G_{\text{SI}} = \frac{\xi_0^2}{4 m_e} \times C_{\text{conv}} \times K_{\text{frak}}}
			\label{T0_SI:L-T0_Gravitationskonstante-0165}
		\end{equation}
		
		where:
		\begin{itemize}
			\item $\xi_0 = 1.333 \times 10^{-4}$ (geometric parameter)
			\item $m_e = 0.511$ MeV (electron mass, derived from $\xi$)
			\item $C_{\text{conv}} = 7.783 \times 10^{-3}$ (systematically derived from $\hbar$, $c$)
			\item $K_{\text{frak}} = 0.986$ (fractal quantum spacetime correction)
		\end{itemize}
		
\section*{Result:}
		\begin{equation}
			G_{\text{SI}} = 6.674 \times 10^{-11} \text{ m}^3/(\text{kg}\cdot\text{s}^2)
		\end{equation}
		
		with $<0.0002\%$ deviation from CODATA-2018 value.
% end box keyresult
	
	\section{Derivation of the Planck Length from and}
	
	\subsection{The Planck Length as Fundamental Reference}
	
\section*{Derivation}
\section*{Definition of the Planck length:}
		
		In standard physics, the Planck length is defined as:
		\begin{equation}
			l_P = \sqrt{\frac{\hbar G}{c^3}}
			\label{T0_SI:L-T0_SI-0462}
		\end{equation}
		
		In natural units ($\hbar = c = 1$) this simplifies to:
		\begin{equation}
			\boxed{l_P = \sqrt{G} = 1 \quad \text{(natural units)}}
			\label{T0_SI:L-T0_SI-0463}
		\end{equation}
		
		\textbf{Physical meaning:} The Planck length represents the characteristic scale of quantum gravitational effects and serves as the natural length unit in theories combining quantum mechanics and general relativity.
% end box derivation
	
	\subsection{T0 Derivation: Planck Length from Only}
	
\section*{Key Result}
\section*{Complete derivation chain:}
		
		Since $G$ is derived from $\xi$ via equation \eqref{L-T0_Gravitationskonstante-0167}:
		\begin{equation}
			G = \frac{\xi^2}{4 m_e}
		\end{equation}
		
		the Planck length follows directly:
		\begin{equation}
			l_P = \sqrt{G} = \sqrt{\frac{\xi^2}{4 m_e}} = \frac{\xi}{2\sqrt{m_e}}
		\end{equation}
		
		In natural units with $m_e = 0.511$ MeV:
		\begin{equation}
			l_P = \frac{1.333 \times 10^{-4}}{2\sqrt{0.511}} \approx 9.33 \times 10^{-5} \text{ (natural units)}
		\end{equation}
		
\section*{Conversion to SI units:}
		\begin{equation}
			\boxed{l_P = 1.616 \times 10^{-35} \text{ m}}
		\end{equation}
% end box keyresult
	
	\subsection{The Characteristic T0 Length Scale}
	
\section*{Insight}
\section*{Connection between $r_0$ and the fundamental energy scale $E_0$:}
		
		The characteristic T0 length $r_0$ for an energy $E$ is defined as:
		\begin{equation}
			r_0(E) = 2GE
		\end{equation}
		
		For the fundamental energy scale $E_0 = \sqrt{m_e \cdot m_\mu}$:
		\begin{equation}
			r_0(E_0) = 2GE_0 \approx 2.7 \times 10^{-14} \text{ m}
		\end{equation}
		
		The minimal sub-Planck length scale is:
		\begin{equation}
			\boxed{L_0 = \xi \cdot l_P = \frac{4}{3} \times 10^{-4} \times 1.616 \times 10^{-35} \text{ m} = 2.155 \times 10^{-39} \text{ m}}
		\end{equation}
		
		\textbf{Fundamental relationship:} In natural units, for any energy $E$:
		\begin{equation}
			r_0(E) = \frac{1}{E} \quad \text{(in natural units with } c = \hbar = 1\text{)}
		\end{equation}
		
		where the time-energy duality $r_0(E) \leftrightarrow E$ defines the characteristic scale. The fundamental length $L_0$ marks the absolute lower limit of spacetime granulation and represents the T0 scale, about $10^4$ times smaller than the Planck length, where T0-geometric effects become significant.
% end box insight
	
	\subsection{The Crucial Convergence: Why T0 and SI Agree}
	
\section*{Historical}
\section*{Two independent paths to the same Planck length:}
		
		There are two completely independent ways to determine the Planck length:
		
\section*{Path 1: SI-based (experimental):}
		\begin{equation}
			l_P^{\text{SI}} = \sqrt{\frac{\hbar G_{\text{measured}}}{c^3}} = 1.616 \times 10^{-35} \text{ m}
		\end{equation}
		
		This uses the experimentally measured gravitational constant $G_{\text{measured}} = 6.674 \times 10^{-11}$ m$^3$/(kg$\cdot$s$^2$) from CODATA.
		
\section*{Path 2: T0-based (pure geometry):}
		\begin{align}
			m_e &= \frac{f_e^2}{\xi^2} \cdot S_{T0} \quad \text{(from } \xi\text{)} \\
			G &= \frac{\xi^2}{4m_e} \times C_{\text{conv}} \times K_{\text{frak}} \quad \text{(from } \xi \text{ and } m_e\text{)} \\
			l_P^{\text{T0}} &= \sqrt{G} = \frac{\xi}{2\sqrt{m_e}} \quad \text{(from } \xi \text{ alone, in natural units)}
		\end{align}
		
\section*{Conversion to SI units:}
		\begin{equation}
			l_P^{\text{SI}} = l_P^{\text{T0}} \times \frac{\hbar c}{1 \text{ MeV}} = l_P^{\text{T0}} \times 1.973 \times 10^{-13} \text{ m}
		\end{equation}
		
		\textbf{Result:} $l_P^{\text{T0}} = 1.616 \times 10^{-35}$ m
		
\section*{The astonishing convergence:}
		\begin{equation}
			\boxed{l_P^{\text{SI}} = l_P^{\text{T0}} \quad \text{with } <0.0002\% \text{ deviation}}
		\end{equation}
% end box historical
	
\section*{Warning}
\section*{Why this agreement is not coincidental:}
		
		The perfect agreement between the SI-derived and T0-derived Planck length reveals a profound truth:
		
		\begin{enumerate}
			\item The SI reform 2019 unknowingly calibrated itself to geometric reality
			
			\item Sommerfeld's 1916 calibration to $\alpha \approx 1/137$ was not arbitrary -- it reflected the fundamental geometric value $\alpha = \xi \cdot E_0^2$
			
			\item The experimental measurement of $G$ does not determine an arbitrary constant -- it measures the geometric structure encoded in $\xi$
			
			\item \textbf{The conversion factor is not arbitrary:} The factor $\frac{\hbar c}{1 \text{ MeV}} = 1.973 \times 10^{-13}$ m appears arbitrary, but it encodes the geometric prediction $S_{T0} = 1$ MeV/$c^2$ for the mass scaling factor. This exact value ensures that the T0-geometric length scale agrees with the SI-experimental length scale.
			
			\item Both paths describe the same underlying geometric reality: \textbf{the universe is pure $\xi$-geometry}
		\end{enumerate}
		
		The SI constants ($c$, $\hbar$, $e$, $k_B$) define \emph{how we measure}, but the \emph{relationships between measurable quantities} are determined by $\xi$-geometry. Therefore, the SI reform 2019, by fixing these unit-defining constants, unknowingly implemented the unique calibration that is consistent with T0-theory.
% end box warning
	
	\section{The Geometric Necessity of the Conversion Factor}
	
	\subsection{Why Exactly 1 MeV/?}
	
\section*{Key Result}
		\textbf{The non-arbitrary nature of $S_{T0} = 1$ MeV/$c^2$:}
		
		T0-Theory predicts that the mass scaling factor must be:
		\begin{equation}
			\boxed{S_{T0} = 1 \text{ MeV}/c^2}
		\end{equation}
		
		This is \textbf{not} a free parameter or convention -- it is a geometric prediction that follows from the requirement of consistency between:
		\begin{itemize}
			\item $\xi$-geometry in natural units
			\item the experimental Planck length $l_P^{\text{SI}} = 1.616 \times 10^{-35}$ m
			\item the measured gravitational constant $G^{\text{SI}} = 6.674 \times 10^{-11}$ m$^3$/(kg$\cdot$s$^2$)
		\end{itemize}
% end box keyresult
	
	\subsection{The Conversion Chain}
	
\section*{Derivation}
\section*{From natural units to SI units:}
		
		The conversion factor between natural T0 units and SI units is:
		\begin{equation}
			\text{Conversion factor} = \frac{\hbar c}{S_{T0}} = \frac{\hbar c}{1 \text{ MeV}} = 1.973 \times 10^{-13} \text{ m}
		\end{equation}
		
		For the Planck length:
		\begin{align}
			l_P^{\text{nat}} &= \frac{\xi}{2\sqrt{m_e}} \approx 9.33 \times 10^{-5} \quad \text{(natural units)} \\
			l_P^{\text{SI}} &= l_P^{\text{nat}} \times \frac{\hbar c}{1 \text{ MeV}} \\
			&= 9.33 \times 10^{-5} \times 1.973 \times 10^{-13} \text{ m} \\
			&= 1.616 \times 10^{-35} \text{ m} \quad \checkmark
		\end{align}
		
		\textbf{The geometric lock:} If $S_{T0}$ were anything other than exactly 1 MeV/$c^2$, the T0-derived Planck length would not agree with the SI-measured value. The fact that they agree proves that $S_{T0} = 1$ MeV/$c^2$ is geometrically determined by $\xi$.
% end box derivation
	
	\subsection{The Triple Consistency}
	
\section*{Insight}
\section*{Three independent measurements lock together:}
		
		The system is overdetermined by three independent experimental values:
		\begin{enumerate}
			\item Fine structure constant: $\alpha = 1/137.035999084$ (measured via quantum Hall effect)
			\item Gravitational constant: $G = 6.674 \times 10^{-11}$ m$^3$/(kg$\cdot$s$^2$) (Cavendish-type experiments)
			\item Planck length: $l_P = 1.616 \times 10^{-35}$ m (derived from $G$, $\hbar$, $c$)
		\end{enumerate}
		
		T0-Theory predicts all three from $\xi$ alone, with the boundary condition:
		\begin{equation}
			S_{T0} = 1 \text{ MeV}/c^2 \quad \text{(unique value that satisfies all three)}
		\end{equation}
		
		This triple consistency is impossible by chance -- it reveals that $\xi$-geometry is the underlying structure of physical reality, and $S_{T0} = 1$ MeV/$c^2$ is the geometric calibration that connects dimensionless geometry with dimensional measurements.
% end box insight
	
	\section{The Speed of Light: Geometric or Conventional?}
	
	\subsection{The Dual Nature of}
	
\section*{Derivation}
\section*{Understanding the role of the speed of light:}
		
		The speed of light has a subtle dual character that requires careful analysis:
		
\section*{Perspective 1: As dimensional convention}
		
		In natural units, setting $c = 1$ is purely conventional:
		\begin{equation}
			[L] = [T] \quad \text{(space and time have the same dimension)}
		\end{equation}
		
		This is analogous to saying 1 hour equals 60 minutes -- it's a choice of measurement units, not physics.
		
\section*{Perspective 2: As geometric ratio}
		
		However, the \emph{specific numerical value} in SI units is not arbitrary. From T0-Theory:
		\begin{align}
			l_P &= \frac{\xi}{2\sqrt{m_e}} \quad \text{(geometric)} \\
			t_P &= \frac{l_P}{c} = \frac{l_P}{1} \quad \text{(in natural units)}
		\end{align}
		
		The Planck time is geometrically linked to the Planck length through the fundamental spacetime structure encoded in $\xi$.
% end box derivation
	
	\subsection{The SI Value is Geometrically Fixed}
	
\section*{Key Result}
\section*{Why $c = 299,792,458$ m/s exactly:}
		
		The SI reform 2019 fixed $c$ by definition, but this value was not arbitrary -- it was chosen to match centuries of measurements. These measurements were actually probing the geometric structure:
		
		\begin{equation}
			c^{\text{SI}} = \frac{l_P^{\text{SI}}}{t_P^{\text{SI}}} = \frac{1.616 \times 10^{-35} \text{ m}}{5.391 \times 10^{-44} \text{ s}}
		\end{equation}
		
		Both $l_P^{\text{SI}}$ and $t_P^{\text{SI}}$ are derived from $\xi$ through:
		\begin{align}
			l_P &= \sqrt{G} = \sqrt{\frac{\xi^2}{4m_e}} \quad \text{(from } \xi\text{)} \\
			t_P &= l_P/c = l_P \quad \text{(natural units)}
		\end{align}
		
		Therefore:
		\begin{equation}
			\boxed{c^{\text{measured}} = c^{\text{geometric}}(\xi) = 299,792,458 \text{ m/s}}
		\end{equation}
		
		The agreement is not coincidental -- it reveals that historical measurements of $c$ were measuring the $\xi$-geometric structure of spacetime.
% end box keyresult
	
	\subsection{The Meter is Defined by , but is Determined by}
	
\section*{Insight}
\section*{The beautiful calibration loop:}
		
		There is a beautiful circularity in the SI-2019 system:
		
		\begin{enumerate}
			\item The meter is \emph{defined} as the distance light travels in $1/299,792,458$ seconds
			\item But the number $299,792,458$ was chosen to match experimental measurements
			\item These measurements probed $\xi$-geometry: $c = l_P/t_P$ where both scales are derived from $\xi$
			\item Therefore, the meter is ultimately calibrated to $\xi$-geometry
		\end{enumerate}
		
		\textbf{Conclusion:} While we use $c$ to \emph{define} the meter, nature uses $\xi$ to \emph{determine} $c$. The SI system unknowingly calibrated itself to fundamental geometry.
% end box insight
	
	\section{Derivation of the Boltzmann Constant}
	
	\subsection{The Temperature Problem in Natural Units}
	
\section*{Warning}
\section*{The Boltzmann constant is NOT fundamental:}
		
		In natural units, where energy is the fundamental dimension, temperature is just another energy scale. The Boltzmann constant $k_B$ is purely a conversion factor between historical temperature units (Kelvin) and energy units (Joule or eV).
% end box warning
	
	\subsection{Definition in the SI System}
	
\section*{Derivation}
\section*{The SI-Reform-2019 definition:}
		
		Since May 20, 2019, the Boltzmann constant is fixed by definition:
		\begin{equation}
			\boxed{k_B = 1.380649 \times 10^{-23} \text{ J/K}}
			\label{T0_SI:L-T0_SI-0464}
		\end{equation}
		
		This defines the Kelvin scale in terms of energy:
		\begin{equation}
			1 \text{ K} = \frac{k_B}{1 \text{ J}} = 1.380649 \times 10^{-23} \text{ energy units}
		\end{equation}
% end box derivation
	
	\subsection{Relation to Fundamental Constants}
	
\section*{Key Result}
\section*{Boltzmann constant from gas constant:}
		
		The Boltzmann constant is defined through the Avogadro number:
		\begin{equation}
			k_B = \frac{R}{N_A}
		\end{equation}
		
		where:
		\begin{itemize}
			\item $R = 8.314462618$ J/(mol$\cdot$K) (ideal gas constant)
			\item $N_A = 6.02214076 \times 10^{23}$ mol$^{-1}$ (Avogadro constant, fixed since 2019)
		\end{itemize}
		
\section*{Result:}
		\begin{equation}
			k_B = \frac{8.314462618}{6.02214076 \times 10^{23}} = 1.380649 \times 10^{-23} \text{ J/K}
		\end{equation}
% end box keyresult
	
	\subsection{T0 Perspective on Temperature}
	
\section*{Insight}
\section*{Temperature as energy scale in T0-Theory:}
		
		In T0-Theory, temperature is naturally expressed as energy:
		\begin{equation}
			T_{\text{natural}} = k_B T_{\text{Kelvin}}
		\end{equation}
		
		For example the CMB temperature:
		\begin{align}
			T_{\text{CMB}} &= 2.725 \text{ K} \\
			T_{\text{CMB}}^{\text{natural}} &= k_B \times 2.725 \text{ K} = 2.35 \times 10^{-4} \text{ eV}
		\end{align}
		
		\textbf{Core statement:} $k_B$ is not derived from $\xi$ because it represents a historical convention for temperature measurement, not a physical property of spacetime geometry.
% end box insight
	
	\section{The Interwoven Network of Constants}
	
	\subsection{The Fundamental Formula Network}
	
\section*{Derivation}
\section*{The SI constants are mathematically linked:}
		
		Since the SI reform 2019, all fundamental constants are connected by exact mathematical relationships:
		
		\begin{align}
			\alpha &= \frac{e^2}{4\pi\varepsilon_0\hbar c} \quad \text{(exact definition)} \\
			\varepsilon_0 &= \frac{e^2}{2\alpha h c} \quad \text{(derived from above)} \\
			\mu_0 &= \frac{2\alpha h}{e^2 c} \quad \text{(via } \varepsilon_0\mu_0c^2 = 1) \\
			k_B &= \frac{R}{N_A} \quad \text{(definition of Boltzmann constant)}
		\end{align}
% end box derivation
	
	\subsection{The Geometric Boundary Condition}
	
\section*{Insight}
\section*{T0-Theory reveals why these specific values are geometrically necessary:}
		
		\begin{equation}
			\alpha = \xi \cdot E_0^2 = \frac{1}{137.036} \quad \text{(geometric derivation)}
		\end{equation}
		
		This fundamental relationship forces the specific numerical values of the interwoven constants:
		
		\begin{equation}
			\frac{e^2}{4\pi\varepsilon_0\hbar c} = \frac{1}{137.036} \quad \text{(geometric boundary condition)}
		\end{equation}
% end box insight
	
	\section{The Nature of Physical Constants}
	
	\subsection{Translation Conventions vs. Physical Quantities}
	
\section*{Key Result}
\section*{Constants fall into three categories:}
		\begin{enumerate}
			\item \textbf{The single fundamental parameter:} $\xi = \frac{4}{3} \times 10^{-4}$
			
			\item \textbf{Geometric quantities derivable from $\xi$:}
			\begin{itemize}
				\item Particle masses (electron, muon, tau, quarks)
				\item Coupling constants ($\alpha$, $\alpha_s$, $\alpha_w$)
				\item Gravitational constant $G$
				\item Planck length $l_P$
				\item Scaling factor $S_{T0} = 1$ MeV/$c^2$
				\item \textbf{Speed of light $c = 299,792,458$ m/s (geometric prediction)}
			\end{itemize}
			
			\item \textbf{Pure translation conventions (SI unit definitions):}
			\begin{itemize}
				\item $\hbar$ (defines energy-time relationship)
				\item $e$ (defines charge scale)
				\item $k_B$ (defines temperature-energy relationship)
			\end{itemize}
		\end{enumerate}
% end box keyresult
	
\section*{Warning}
\section*{Critical clarification about the speed of light:}
		
		The speed of light occupies a unique position in this classification:
		
		\begin{itemize}
			\item \textbf{In natural units ($c = 1$):} $c$ is merely a convention that specifies how we relate length and time
			
			\item \textbf{In SI units:} The numerical value $c = 299,792,458$ m/s is \textbf{geometrically determined by $\xi$} through:
			\begin{equation}
				c = \frac{l_P^{\text{T0}}}{t_P^{\text{T0}}} = \frac{\xi/(2\sqrt{m_e})}{\xi/(2\sqrt{m_e})} = 1 \quad \text{(natural units)}
			\end{equation}
			
			The SI value follows from the conversion:
			\begin{equation}
				c^{\text{SI}} = \frac{l_P^{\text{SI}}}{t_P^{\text{SI}}} = \frac{1.616 \times 10^{-35} \text{ m}}{5.391 \times 10^{-44} \text{ s}} = 299,792,458 \text{ m/s}
			\end{equation}
		\end{itemize}
		
		\textbf{The profound implication:} While we \emph{define} the meter using $c$ (SI 2019), the \emph{relationship} between time and space intervals is geometrically fixed by $\xi$. The specific numerical value of $c$ in SI units emerges from $\xi$-geometry, not human convention.
% end box warning
	
	\subsection{The SI Reform 2019: Geometric Calibration Realized}
	
	The 2019 redefinition fixed constants by definition:
	\begin{align}
		c &= 299,792,458 \text{ m/s} \\
		\hbar &= 1.054571817... \times 10^{-34} \text{ J}\cdot\text{s} \\
		e &= 1.602176634 \times 10^{-19} \text{ C} \\
		k_B &= 1.380649 \times 10^{-23} \text{ J/K}
	\end{align}
	
\section*{Insight}
		This fixation implements the unique calibration that is consistent with $\xi$-geometry. The apparent arbitrariness conceals geometric necessity.
% end box insight
	
	\section{The Mathematical Necessity}
	
	\subsection{Why Constants Must Have Their Specific Values}
	
\section*{Derivation}
\section*{The interlocking system:}
		
		Given the fixed values and their mathematical relationships:
		
		\begin{align}
			h &= 2\pi\hbar = 6.62607015 \times 10^{-34} \text{ J}\cdot\text{s} \\
			\alpha &= \frac{e^2}{4\pi\varepsilon_0\hbar c} = \frac{1}{137.035999084} \\
			\varepsilon_0 &= \frac{e^2}{2\alpha h c} = 8.8541878128 \times 10^{-12} \text{ F/m} \\
			\mu_0 &= \frac{2\alpha h}{e^2 c} = 1.25663706212 \times 10^{-6} \text{ N/A}^2
		\end{align}
		
		These are not independent choices, but mathematically enforced relationships.
% end box derivation
	
	\subsection{The Geometric Explanation}
	
\section*{Historical}
\section*{Sommerfeld's unknowing geometric calibration}
		
		Arnold Sommerfeld's 1916 calibration to $\alpha \approx 1/137$ established the SI system on geometric foundations. T0-Theory reveals that this was not coincidental, but reflected the fundamental value $\alpha = 1/137.036$ derived from $\xi$.
% end box historical
	
	\section{Conclusion: Geometric Unity}
	
\section*{Key Result}
\section*{Complete parameter freedom achieved:}
		\begin{itemize}
			\item \textbf{Single input:} $\xi = \frac{4}{3} \times 10^{-4}$
			
			\item \textbf{Everything derivable from $\xi$ alone:}
			\begin{itemize}
				\item \textbf{First:} All particle masses including electron: $m_e = f_e^2/\xi^2 \cdot S_{T0}$
				\item \textbf{Then:} Gravitational constant: $G = \xi^2/(4m_e) \times$ (conversion factors)
				\item \textbf{Then:} Planck length: $l_P = \sqrt{G} = \xi/(2\sqrt{m_e})$
				\item \textbf{Also:} Speed of light: $c = l_P/t_P$ (geometrically determined)
				\item \textbf{Also:} Characteristic T0 length: $L_0 = \xi \cdot l_P$ (spacetime granulation)
				\item Coupling constants: $\alpha$, $\alpha_s$, $\alpha_w$
				\item Scaling factor: $S_{T0} = 1$ MeV/$c^2$ (prediction, not convention)
			\end{itemize}
			
			\item \textbf{Translation conventions (not derived, define units):}
			\begin{itemize}
				\item $\hbar$ defines energy-time relationship in SI units
				\item $e$ defines charge scale in SI units
				\item $k_B$ defines temperature-energy conversion (historical)
			\end{itemize}
			
			\item \textbf{Mathematical necessity:} Constants interwoven by exact formulas
			
			\item \textbf{Geometric foundation:} SI 2019 unknowingly implements $\xi$-geometry
		\end{itemize}
% end box keyresult
	
	\begin{center}
		\fbox{\parbox{0.9\textwidth}{
				\textbf{Final insight:} The universe is pure geometry, encoded in $\xi$. The complete derivation chain is:
				
				$\xi \to \{m_e, m_\mu, m_\tau, ...\} \to G \to l_P \to c$
				
				with $L_0 = \xi \cdot l_P$ expressing the fundamental sub-Planck scale of spacetime granulation.
				
				\textbf{The profound mystery solved:} Why does the Planck length derived purely from $\xi$-geometry exactly match the Planck length calculated from experimentally measured $G$? Because \emph{both describe the same geometric reality}. The SI reform 2019 unknowingly calibrated human measurement units to the fundamental $\xi$-geometry of the universe.
				
				This is not coincidence -- it is geometric necessity. Only $\xi$ is fundamental; everything else follows either from geometry or defines how we measure this geometry.
		}}
	\end{center}
	


% Bibliography
\begin{thebibliography}{99}
	
	\bibitem{pdg2024}
	Particle Data Group Collaboration (2024). 
	\textit{Review of Particle Physics}. 
	Progress of Theoretical and Experimental Physics, 2024(8), 083C01.
	\url{https://pdg.lbl.gov}
	
	\bibitem{flag2024}
	Aoki, Y., et al. (FLAG Collaboration) (2024). 
	\textit{FLAG Review 2024 of Lattice Results for Low-Energy Constants}. 
	arXiv:2411.04268.
	\url{https://arxiv.org/abs/2411.04268}
	
	\bibitem{fermilab_muon_g2}
	Abi, B., et al. (Muon g-2 Collaboration) (2021). 
	\textit{Measurement of the Positive Muon Anomalous Magnetic Moment to 0.46 ppm}. 
	Physical Review Letters, 126, 141801.
	
	\bibitem{peskin_schroeder}
	Peskin, M. E., \& Schroeder, D. V. (1995). 
	\textit{An Introduction to Quantum Field Theory}. 
	Addison-Wesley.
	
	\bibitem{weinberg_qft}
	Weinberg, S. (1995). 
	\textit{The Quantum Theory of Fields, Vol. I--III}. 
	Cambridge University Press.
	
	\bibitem{griffiths_particle}
	Griffiths, D. (2008). 
	\textit{Introduction to Elementary Particles}. 
	Wiley-VCH.
	
	\bibitem{mandl_shaw}
	Mandl, F., \& Shaw, G. (2010). 
	\textit{Quantum Field Theory (2nd ed.)}. 
	Wiley.
	
	\bibitem{srednicki_qft}
	Srednicki, M. (2007). 
	\textit{Quantum Field Theory}. 
	Cambridge University Press.
	
	\bibitem{t0_fundamentals}
	Pascher, J. (2024). 
	\textit{T0-Theory: Foundations of Time-Mass Duality}. 
	Unpublished manuscript, HTL Leonding.
	
	\bibitem{t0_fine_structure}
	Pascher, J. (2024). 
	\textit{T0-Theory: The Fine Structure Constant}. 
	Unpublished manuscript, HTL Leonding.
	
	\bibitem{t0_neutrinos}
	Pascher, J. (2024). 
	\textit{T0-Theory: Neutrino Masses and PMNS Mixing}. 
	Unpublished manuscript, HTL Leonding.
	
	\bibitem{t0_github}
	Pascher, J. (2024--2025). 
	\textit{T0-Time-Mass-Duality Repository}. 
	GitHub.
	\url{https://github.com/jpascher/T0-Time-Mass-Duality}
	
	\bibitem{lattice_qcd_review}
	Kronfeld, A. S. (2012). 
	\textit{Twenty-first Century Lattice Gauge Theory: Results from the QCD Lagrangian}. 
	Annual Review of Nuclear and Particle Science, 62, 265--284.
	
	\bibitem{neutrino_mixing_pdg}
	Particle Data Group Collaboration (2024). 
	\textit{Neutrino Masses, Mixing, and Oscillations}. 
	PDG Review 2024.
	\url{https://pdg.lbl.gov/2024/reviews/rpp2024-rev-neutrino-mixing.pdf}
	
	\bibitem{higgs_discovery}
	ATLAS and CMS Collaborations (2012). 
	\textit{Observation of a New Particle in the Search for the Standard Model Higgs Boson}. 
	Physics Letters B, 716, 1--29.
	
	\bibitem{Brannen2005}
	C. P. Brannen, ``Estimate of neutrino masses from Koide's relation'', \textit{arXiv:hep-ph/0505028} (2005).
	\url{https://arxiv.org/abs/hep-ph/0505028}
	
	\bibitem{Brannen2006}
	C. P. Brannen, ``Koide Mass Formula for Neutrinos'', \textit{arXiv:0702.0052} (2006).
	\url{http://brannenworks.com/MASSES.pdf}
	
	\bibitem{PhaseVectors2025}
	Anonymous, ``The Koide Relation and Lepton Mass Hierarchy from Phase Vectors'', \textit{rXiv:2507.0040} (2025).
	\url{https://rxiv.org/pdf/2507.0040v1.pdf}
	
	\bibitem{PDG2025}
	Particle Data Group, ``Review of Particle Physics'', \textit{Phys. Rev. D} \textbf{112} (2025) 030001.
	\url{https://pdg.lbl.gov/2025/}
	
	\bibitem{terrell2024}
	Terrell et al. (2024). 
	\textit{Single-Clock Metrology in Nature}. 
	Nature Physics.
	
	\bibitem{hossenfelder2024}
	Hossenfelder, S. (2024). 
	\textit{Single Clock Video Explanation}. 
	YouTube.
	
	\bibitem{hundert1931}
	Hundert (1931). 
	\textit{Reference Work}. 
	Publisher.
	
	\bibitem{terrell2025}
	Terrell et al. (2025). 
	\textit{Advanced Clock Synchronization Methods}. 
	Physical Review Letters.
	
	\bibitem{pascher_t0_2025}
	Pascher, J. (2025). 
	\textit{T0-Theory: Complete Framework and Applications}. 
	Unpublished manuscript, HTL Leonding.
	
	\bibitem{t0qm}
	Pascher, J. (2024). 
	\textit{T0-Theory: Quantum Mechanics Formulation}. 
	Unpublished manuscript, HTL Leonding.
	
	\bibitem{t0anomale}
	Pascher, J. (2024). 
	\textit{T0-Theory: Anomalous Magnetic Moments}. 
	Unpublished manuscript, HTL Leonding.
	
	\bibitem{muong2complete}
	Abi, B., et al. (Muon g-2 Collaboration) (2023). 
	\textit{Complete Measurement of the Positive Muon Anomalous Magnetic Moment}. 
	Physical Review Letters, 131, 161802.
	
	\bibitem{penrose2004}
	Penrose, R. (2004). 
	\textit{The Road to Reality: A Complete Guide to the Laws of the Universe}. 
	Jonathan Cape.
	
	\bibitem{planck1900}
	Planck, M. (1900). 
	\textit{On the Theory of the Energy Distribution Law of the Normal Spectrum}. 
	Verhandlungen der Deutschen Physikalischen Gesellschaft, 2, 237.
	
	\bibitem{T0Theory}
	Pascher, J. (2024). 
	\textit{T0-Theory: Fundamental Principles}. 
	Unpublished manuscript, HTL Leonding.
	
	% Additional bibliography entries for all undefined citations
	\bibitem{6g_roadmap}
	6G Research Consortium (2024).
	\textit{6G Technology Roadmap}.
	Technical Report.
	
	\bibitem{Born2013}
	Born, M. (2013).
	\textit{Einstein's Theory of Relativity}.
	Dover Publications.
	
	\bibitem{Casimir1948}
	Casimir, H. B. G. (1948).
	\textit{On the attraction between two perfectly conducting plates}.
	Proc. Kon. Ned. Akad. Wetensch. B51, 793--795.
	
	\bibitem{Einstein1905}
	Einstein, A. (1905).
	\textit{On the Electrodynamics of Moving Bodies}.
	Annalen der Physik, 17, 891--921.
	
	\bibitem{Feynman2006}
	Feynman, R. P. (2006).
	\textit{QED: The Strange Theory of Light and Matter}.
	Princeton University Press.
	
	\bibitem{Griffiths2017}
	Griffiths, D. J. (2017).
	\textit{Introduction to Electrodynamics (4th ed.)}.
	Cambridge University Press.
	
	\bibitem{Jackson1999}
	Jackson, J. D. (1999).
	\textit{Classical Electrodynamics (3rd ed.)}.
	Wiley.
	
	\bibitem{Mohr2016}
	Mohr, P. J., et al. (2016).
	\textit{CODATA Recommended Values of the Fundamental Physical Constants: 2014}.
	Rev. Mod. Phys. 88, 035009.
	
	\bibitem{Parker2018}
	Parker, R. H., et al. (2018).
	\textit{Measurement of the fine-structure constant as a test of the Standard Model}.
	Science, 360, 191--195.
	
	\bibitem{Planck1900}
	Planck, M. (1900).
	\textit{On the Theory of the Energy Distribution Law of the Normal Spectrum}.
	Verhandlungen der Deutschen Physikalischen Gesellschaft, 2, 237.
	
	\bibitem{Planck2018}
	Planck Collaboration (2018).
	\textit{Planck 2018 results. VI. Cosmological parameters}.
	Astronomy \& Astrophysics, 641, A6.
	
	\bibitem{QFT_T0}
	Pascher, J. (2024).
	\textit{T0-Theory and QFT Connections}.
	Unpublished manuscript, HTL Leonding.
	
	\bibitem{Sommerfeld1916}
	Sommerfeld, A. (1916).
	\textit{On the Quantum Theory of Spectral Lines}.
	Annalen der Physik, 51, 1--94.
	
	\bibitem{T0_Feinstruktur}
	Pascher, J. (2024).
	\textit{T0-Theory: Fine Structure Analysis}.
	Unpublished manuscript, HTL Leonding.
	
	\bibitem{T0_SI}
	Pascher, J. (2024).
	\textit{T0-Theory and SI Units}.
	Unpublished manuscript, HTL Leonding.
	
	\bibitem{T0_fine_structure}
	Pascher, J. (2024).
	\textit{T0-Theory: The Fine Structure Constant}.
	Unpublished manuscript, HTL Leonding.
	
	\bibitem{T0_g2_erweiterung}
	Pascher, J. (2024).
	\textit{T0-Theory: g-2 Extensions}.
	Unpublished manuscript, HTL Leonding.
	
	\bibitem{T0_gravitational_constant}
	Pascher, J. (2024).
	\textit{T0-Theory: Gravitational Constant Derivation}.
	Unpublished manuscript, HTL Leonding.
	
	\bibitem{T0_netze_en}
	Pascher, J. (2024).
	\textit{T0-Theory: Network Structures}.
	Unpublished manuscript, HTL Leonding.
	
	\bibitem{T0_tm_erweiterung}
	Pascher, J. (2024).
	\textit{T0-Theory: Time-Mass Extensions}.
	Unpublished manuscript, HTL Leonding.
	
	\bibitem{Uzan2003}
	Uzan, J.-P. (2003).
	\textit{The fundamental constants and their variation}.
	Rev. Mod. Phys. 75, 403--455.
	
	\bibitem{Weinberg1995}
	Weinberg, S. (1995).
	\textit{The Quantum Theory of Fields, Vol. I}.
	Cambridge University Press.
	
	\bibitem{albrecht1999}
	Albrecht, A. \& Magueijo, J. (1999).
	\textit{A time varying speed of light as a solution to cosmological puzzles}.
	Phys. Rev. D 59, 043516.
	
	\bibitem{alice2023}
	ALICE Collaboration (2023).
	\textit{Recent results from ALICE}.
	CERN-EP-2023-XXX.
	
	\bibitem{analog_optical}
	Smith, J. et al. (2024).
	\textit{Analog optical computing systems}.
	Nature Photonics.
	
	\bibitem{ashtekar2004}
	Ashtekar, A. \& Lewandowski, J. (2004).
	\textit{Background independent quantum gravity}.
	Class. Quantum Grav. 21, R53.
	
	\bibitem{atlas2023}
	ATLAS Collaboration (2023).
	\textit{ATLAS physics results}.
	CERN-PH-EP-2023-XXX.
	
	\bibitem{atlas2023higgs}
	ATLAS Collaboration (2023).
	\textit{Higgs boson measurements}.
	Phys. Rev. Lett.
	
	\bibitem{barbour1999}
	Barbour, J. (1999).
	\textit{The End of Time}.
	Oxford University Press.
	
	\bibitem{barrow1999}
	Barrow, J. D. (1999).
	\textit{Cosmologies with varying light speed}.
	Phys. Rev. D 59, 043515.
	
	\bibitem{becker2007}
	Becker, K. et al. (2007).
	\textit{String Theory and M-Theory}.
	Cambridge University Press.
	
	\bibitem{bell_muon}
	Bennett, G. W., et al. (Muon g-2 Collaboration) (2006).
	\textit{Final report of the E821 muon anomalous magnetic moment measurement}.
	Phys. Rev. D 73, 072003.
	
	\bibitem{bondi1948}
	Bondi, H. \& Gold, T. (1948).
	\textit{The steady-state theory of the expanding universe}.
	Mon. Not. R. Astron. Soc. 108, 252--270.
	
	\bibitem{brewer2019}
	Brewer, S. M. et al. (2019).
	\textit{Al+ Quantum-Logic Clock with Systematic Uncertainty below $10^{-18}$}.
	Phys. Rev. Lett. 123, 033201.
	
	\bibitem{cms2023top}
	CMS Collaboration (2023).
	\textit{Top quark measurements at CMS}.
	JHEP 2023.
	
	\bibitem{cms2024}
	CMS Collaboration (2024).
	\textit{CMS physics results 2024}.
	CERN-PH-EP-2024-XXX.
	
	\bibitem{codata2019}
	Tiesinga, E. et al. (2019).
	\textit{The 2018 CODATA Recommended Values}.
	J. Phys. Chem. Ref. Data.
	
	\bibitem{desi2025}
	DESI Collaboration (2025).
	\textit{DESI 2025 Cosmology Results}.
	arXiv preprint.
	
	\bibitem{differential_optical}
	Wang, X. et al. (2024).
	\textit{Differential optical computing}.
	Optica.
	
	\bibitem{dingle1972}
	Dingle, H. (1972).
	\textit{Science at the Crossroads}.
	Martin Brian \& O'Keeffe.
	
	\bibitem{divalentino2021}
	Di Valentino, E. et al. (2021).
	\textit{In the realm of the Hubble tension}.
	Class. Quantum Grav. 38, 153001.
	
	\bibitem{elnaschie2004}
	El Naschie, M. S. (2004).
	\textit{A review of E infinity theory}.
	Chaos, Solitons \& Fractals, 19, 209--236.
	
	\bibitem{fabrication_heterogeneous}
	Chen, Y. et al. (2024).
	\textit{Heterogeneous photonic integration}.
	Nature Electronics.
	
	\bibitem{fermilab2023}
	Fermilab (2023).
	\textit{Muon g-2 results}.
	Phys. Rev. Lett.
	
	\bibitem{flexible_wafer}
	Kim, S. et al. (2024).
	\textit{Flexible wafer-scale photonics}.
	Science Advances.
	
	\bibitem{francesco1997}
	Di Francesco, P. et al. (1997).
	\textit{Conformal Field Theory}.
	Springer.
	
	\bibitem{hartree1957}
	Hartree, D. R. (1957).
	\textit{The Calculation of Atomic Structures}.
	Wiley.
	
	\bibitem{hhi_6g}
	Fraunhofer HHI (2024).
	\textit{6G Photonic Integration}.
	Technical Report.
	
	\bibitem{hossenfelder2025}
	Hossenfelder, S. (2025).
	\textit{Science without the gobbledygook}.
	YouTube/Blog.
	
	\bibitem{hossenfelder_single_clock_video}
	Hossenfelder, S. (2024).
	\textit{The Single Clock Problem}.
	YouTube.
	
	\bibitem{hoyle1948}
	Hoyle, F. (1948).
	\textit{A new model for the expanding universe}.
	Mon. Not. R. Astron. Soc. 108, 372--382.
	
	\bibitem{integration_microelectronic}
	Liu, A. et al. (2024).
	\textit{Microelectronic photonic integration}.
	IEEE Journal.
	
	\bibitem{jacobson1995}
	Jacobson, T. (1995).
	\textit{Thermodynamics of spacetime}.
	Phys. Rev. Lett. 75, 1260.
	
	\bibitem{kasevich2023}
	Kasevich, M. et al. (2023).
	\textit{Atom interferometry tests}.
	Nature Physics.
	
	\bibitem{lerner2014}
	Lerner, E. J. (2014).
	\textit{An open letter on cosmology}.
	New Scientist.
	
	\bibitem{lisa2017}
	LISA Consortium (2017).
	\textit{Laser Interferometer Space Antenna}.
	ESA Technical Report.
	
	\bibitem{lithium_tantalate}
	Zhang, M. et al. (2024).
	\textit{Thin-film lithium tantalate photonics}.
	Nature Photonics.
	
	\bibitem{lopez2010}
	Lopez-Corredoira, M. (2010).
	\textit{Tests and problems of the standard model in cosmology}.
	Int. J. Mod. Phys. D.
	
	\bibitem{ludlow2015}
	Ludlow, A. D. et al. (2015).
	\textit{Optical atomic clocks}.
	Rev. Mod. Phys. 87, 637.
	
	\bibitem{mach1883}
	Mach, E. (1883).
	\textit{Die Mechanik in ihrer Entwickelung}.
	F.A. Brockhaus.
	
	\bibitem{maldacena1998}
	Maldacena, J. (1998).
	\textit{The large N limit of superconformal field theories}.
	Adv. Theor. Math. Phys. 2, 231--252.
	
	\bibitem{mueller2014}
	Müller, H. et al. (2014).
	\textit{Atom interferometry tests of the gravitational redshift}.
	Phys. Rev. Lett.
	
	\bibitem{mug2_final_2025}
	Muon g-2 Collaboration (2025).
	\textit{Final muon g-2 measurement}.
	Phys. Rev. Lett.
	
	\bibitem{muong2_2023}
	Muon g-2 Collaboration (2023).
	\textit{Updated muon g-2 results}.
	Phys. Rev. Lett.
	
	\bibitem{nathan2024}
	Nathan, A. et al. (2024).
	\textit{Quantum computing advances}.
	Nature.
	
	\bibitem{newell2018}
	Newell, D. B. et al. (2018).
	\textit{The CODATA 2017 values of h, e, k, and $N_A$}.
	Metrologia 55, L13.
	
	\bibitem{nottale1993}
	Nottale, L. (1993).
	\textit{Fractal Space-Time and Microphysics}.
	World Scientific.
	
	\bibitem{on_chip_lithium}
	Wang, C. et al. (2024).
	\textit{On-chip lithium niobate photonics}.
	Nature Communications.
	
	\bibitem{optical_advantages}
	Shastri, B. J. et al. (2024).
	\textit{Advantages of optical computing}.
	Nature Reviews Physics.
	
	\bibitem{pascher2025cmb}
	Pascher, J. (2025).
	\textit{T0-Theory: CMB Analysis}.
	Unpublished manuscript, HTL Leonding.
	
	\bibitem{pascher2025g2}
	Pascher, J. (2025).
	\textit{T0-Theory: g-2 Predictions}.
	Unpublished manuscript, HTL Leonding.
	
	\bibitem{pascher2025qm}
	Pascher, J. (2025).
	\textit{T0-Theory: Quantum Mechanics}.
	Unpublished manuscript, HTL Leonding.
	
	\bibitem{pascher2025si}
	Pascher, J. (2025).
	\textit{T0-Theory: SI Unit System}.
	Unpublished manuscript, HTL Leonding.
	
	\bibitem{pascher2025t0}
	Pascher, J. (2025).
	\textit{T0-Theory: Complete Framework}.
	Unpublished manuscript, HTL Leonding.
	
	\bibitem{pascher:fundamentals}
	Pascher, J. (2024).
	\textit{T0-Theory: Fundamentals}.
	Unpublished manuscript, HTL Leonding.
	
	\bibitem{pascher:g2_rev9}
	Pascher, J. (2024).
	\textit{T0-Theory: g-2 Revision 9}.
	Unpublished manuscript, HTL Leonding.
	
	\bibitem{pascher:geometric_formalism}
	Pascher, J. (2024).
	\textit{T0-Theory: Geometric Formalism}.
	Unpublished manuscript, HTL Leonding.
	
	\bibitem{pascher:ml_addendum}
	Pascher, J. (2024).
	\textit{T0-Theory: Machine Learning Addendum}.
	Unpublished manuscript, HTL Leonding.
	
	\bibitem{pascher:t0_foundations}
	Pascher, J. (2024).
	\textit{T0-Theory: Foundations}.
	Unpublished manuscript, HTL Leonding.
	
	\bibitem{pascher_derivation_beta_2025}
	Pascher, J. (2025).
	\textit{T0-Theory: Derivation of Beta}.
	Unpublished manuscript, HTL Leonding.
	
	\bibitem{pascher_higgs_connection_2025}
	Pascher, J. (2025).
	\textit{T0-Theory: Higgs Connection}.
	Unpublished manuscript, HTL Leonding.
	
	\bibitem{pascher_lagrangian_extended_2025}
	Pascher, J. (2025).
	\textit{T0-Theory: Extended Lagrangian}.
	Unpublished manuscript, HTL Leonding.
	
	\bibitem{pascher_mathematical_structure_2025}
	Pascher, J. (2025).
	\textit{T0-Theory: Mathematical Structure}.
	Unpublished manuscript, HTL Leonding.
	
	\bibitem{pascher_t0_cmb_2025}
	Pascher, J. (2025).
	\textit{T0-Theory: CMB Predictions}.
	Unpublished manuscript, HTL Leonding.
	
	\bibitem{pascher_t0_energie_2025}
	Pascher, J. (2025).
	\textit{T0-Theory: Energy}.
	Unpublished manuscript, HTL Leonding.
	
	\bibitem{pascher_t0_energy_2025}
	Pascher, J. (2025).
	\textit{T0-Theory: Energy Framework}.
	Unpublished manuscript, HTL Leonding.
	
	\bibitem{pascher_t0_theory_2025}
	Pascher, J. (2025).
	\textit{T0-Theory: Complete Theory}.
	Unpublished manuscript, HTL Leonding.
	
	\bibitem{penrose1959}
	Penrose, R. (1959).
	\textit{The apparent shape of a relativistically moving sphere}.
	Proc. Cambridge Phil. Soc. 55, 137--139.
	
	\bibitem{penrose1967}
	Penrose, R. (1967).
	\textit{Twistor algebra}.
	J. Math. Phys. 8, 345--366.
	
	\bibitem{peratt1992}
	Peratt, A. L. (1992).
	\textit{Physics of the Plasma Universe}.
	Springer-Verlag.
	
	\bibitem{peskin1995}
	Peskin, M. E. \& Schroeder, D. V. (1995).
	\textit{An Introduction to Quantum Field Theory}.
	Addison-Wesley.
	
	\bibitem{peskin_schroeder_1995}
	Peskin, M. E. \& Schroeder, D. V. (1995).
	\textit{An Introduction to Quantum Field Theory}.
	Addison-Wesley.
	
	\bibitem{phoquant}
	PhoQuant (2024).
	\textit{Photonic quantum computing}.
	Technical Report.
	
	\bibitem{photonics_ai}
	Wetzstein, G. et al. (2024).
	\textit{Photonics for AI}.
	Nature.
	
	\bibitem{planck1906}
	Planck, M. (1906).
	\textit{The Theory of Heat Radiation}.
	Johann Ambrosius Barth.
	
	\bibitem{planck2018}
	Planck Collaboration (2018).
	\textit{Planck 2018 results}.
	A\&A 641, A6.
	
	\bibitem{polchinski1998}
	Polchinski, J. (1998).
	\textit{String Theory}.
	Cambridge University Press.
	
	\bibitem{qant_nps}
	QANT (2024).
	\textit{Quantum photonics systems}.
	Technical Report.
	
	\bibitem{quantenjahr25}
	Quantenjahr (2025).
	\textit{International Year of Quantum}.
	UNESCO.
	
	\bibitem{recurrent_photonics}
	Tait, A. N. et al. (2024).
	\textit{Recurrent photonic neural networks}.
	Optica.
	
	\bibitem{rf_photonics}
	Capmany, J. \& Novak, D. (2024).
	\textit{Microwave photonics}.
	Nature Photonics.
	
	\bibitem{riess2019}
	Riess, A. G. et al. (2019).
	\textit{Large Magellanic Cloud Cepheid Standards}.
	ApJ 876, 85.
	
	\bibitem{riess2022}
	Riess, A. G. et al. (2022).
	\textit{A Comprehensive Measurement of H0}.
	ApJ 934, L7.
	
	\bibitem{rovelli2004}
	Rovelli, C. (2004).
	\textit{Quantum Gravity}.
	Cambridge University Press.
	
	\bibitem{sciama1953}
	Sciama, D. W. (1953).
	\textit{On the origin of inertia}.
	Mon. Not. R. Astron. Soc. 113, 34--42.
	
	\bibitem{sciencedaily2025}
	ScienceDaily (2025).
	\textit{Physics news}.
	Online.
	
	\bibitem{sm_g2_2025}
	Aoyama, T. et al. (2025).
	\textit{Standard Model prediction for g-2}.
	Phys. Rep.
	
	\bibitem{susskind1995}
	Susskind, L. (1995).
	\textit{The world as a hologram}.
	J. Math. Phys. 36, 6377--6396.
	
	\bibitem{t0_kosmologie}
	Pascher, J. (2024).
	\textit{T0-Theory: Cosmology}.
	Unpublished manuscript, HTL Leonding.
	
	\bibitem{terrell1959}
	Terrell, J. (1959).
	\textit{Invisibility of the Lorentz contraction}.
	Phys. Rev. 116, 1041--1045.
	
	\bibitem{terrell_single_clock_nature_2024}
	Terrell, J. et al. (2024).
	\textit{Single clock precision measurements}.
	Nature Physics.
	
	\bibitem{tfln_foundry}
	TFLN Foundry (2024).
	\textit{Thin-film lithium niobate foundry services}.
	Technical Specifications.
	
	\bibitem{thiemann2007}
	Thiemann, T. (2007).
	\textit{Modern Canonical Quantum General Relativity}.
	Cambridge University Press.
	
	\bibitem{thz_epfl}
	EPFL (2024).
	\textit{Terahertz photonics research}.
	Technical Report.
	
	\bibitem{unnikrishnan2004}
	Unnikrishnan, C. S. (2004).
	\textit{On Einstein's resolution of the twin clock paradox}.
	Current Science, 86, 704--709.
	
	\bibitem{verlinde2011}
	Verlinde, E. (2011).
	\textit{On the origin of gravity and the laws of Newton}.
	JHEP 2011, 29.
	
	\bibitem{video2025}
	Video (2025).
	\textit{Physics video explanation}.
	YouTube.
	
	\bibitem{weinberg1995}
	Weinberg, S. (1995).
	\textit{The Quantum Theory of Fields}.
	Cambridge University Press.
	
	\bibitem{weiskopf2000}
	Weiskopf, D. (2000).
	\textit{Visualization of special relativity}.
	PhD thesis, University of Tübingen.
	
	\bibitem{wheeler1990}
	Wheeler, J. A. (1990).
	\textit{A Journey into Gravity and Spacetime}.
	Scientific American Library.
	
	\bibitem{wiki_bell}
	Wikipedia (2024).
	\textit{Bell's theorem}.
	Online encyclopedia.
	
	\bibitem{zwicky1929}
	Zwicky, F. (1929).
	\textit{On the red shift of spectral lines through interstellar space}.
	Proc. Natl. Acad. Sci. 15, 773--779.

\end{thebibliography}


\end{document}
