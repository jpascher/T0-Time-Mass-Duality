\documentclass[11pt,a4paper]{article}
\usepackage[a4paper,margin=2cm]{geometry}
\usepackage[utf8]{inputenc}
\usepackage[english]{babel}
\usepackage{lmodern}
\renewcommand{\familydefault}{\sfdefault}

\usepackage{amsmath,amssymb,amsthm}
\usepackage{graphicx}
\usepackage[unicode,pdfencoding=auto,hypertexnames=false]{hyperref}
\usepackage{booktabs}
\usepackage{longtable}
\usepackage{array}
\usepackage{siunitx}
\usepackage{fancyhdr}
\usepackage{float}
\usepackage{tikz}
% tcolorbox removed for standalone
% tcbset removed
\tikzset{
  t0blue/.style={draw=blue,fill=blue!10},
  t0red/.style={draw=red,fill=red!10},
  t0green/.style={draw=green!50!black,fill=green!10},
  t0orange/.style={draw=orange,fill=orange!10},
}
\usepackage{setspace}
\usepackage{enumitem}
\usepackage{adjustbox}
\usepackage{xcolor}

% Define colors for xcolor package
\definecolor{t0green}{RGB}{34,139,34}
\definecolor{t0blue}{RGB}{0,0,255}
\definecolor{t0red}{RGB}{255,0,0}
\definecolor{t0orange}{RGB}{255,165,0}

% Define custom column types for tables
\newcolumntype{L}[1]{>{\raggedright\arraybackslash}p{#1}}
\newcolumntype{C}[1]{>{\centering\arraybackslash}p{#1}}
\newcolumntype{R}[1]{>{\raggedleft\arraybackslash}p{#1}}

\setlength{\parindent}{0pt}
\setlength{\parskip}{6pt}

\hypersetup{
  colorlinks=true,
  linkcolor=blue,
  citecolor=blue,
  urlcolor=blue
}
\pagestyle{fancy}
\setlength{\headheight}{28pt}

\newcommand{\checkmarkx}{\checkmark}
\newcommand{\warningx}{\textbf{!}}

% Makros aus Einzel-Dokumenten (Fallback-Definitionen)
\newcommand{\mytimes}{\times}
\newcommand{\myapprox}{\approx}
\newcommand{\mysim}{\sim}
\newcommand{\myomega}{\omega}
\newcommand{\mypi}{\pi}
\newcommand{\myrightarrow}{\rightarrow}
\newcommand{\mypropto}{\propto}
\newcommand{\deltafield}{\delta\phi}
\newcommand{\xipar}{\xi}
\newcommand{\xiT}{\xi}
\newcommand{\lambdah}{\lambda_h}

% Additional macros used in chapter files
\newcommand{\Kfrak}{K_{\text{frak}}}  % Fractal correction factor
\newcommand{\Dfrak}{D_f}              % Fractal dimension
\newcommand{\betapar}{\beta}          % T0 beta parameter
\newcommand{\alphapar}{\alpha}        % T0 alpha parameter
\newcommand{\Efield}{E}               % Energy field
% Note: checkmarkxa/warningxa are variants used in auto-generated chapter files
\newcommand{\checkmarkxa}{\checkmark}
\newcommand{\warningxa}{\textbf{!}}

% Additional T0-specific macros
\newcommand{\xigeom}{\xi_{\text{geom}}}  % Geometric xi
\newcommand{\lP}{\ell_P}                  % Planck length
\newcommand{\rzero}{r_0}                  % Characteristic radius
\newcommand{\xirat}{\xi_{\text{rat}}}     % Xi ratio
\newcommand{\tzero}{t_0}                  % Characteristic time
\newcommand{\natunits}{\text{(nat. units)}}  % Natural units annotation
\newcommand{\myRightarrow}{\Rightarrow}   % Arrow variant
\newcommand{\Lag}{\mathcal{L}}            % Lagrangian

% Physics macros used in chapter files
\newcommand{\CQCD}{C_{\text{QCD}}}        % QCD correction
\newcommand{\EP}{E_P}                     % Planck energy
\newcommand{\Ee}{E_e}                     % Electron energy
\newcommand{\Emu}{E_\mu}                  % Muon energy
\newcommand{\Exi}{E_\xi}                  % Xi energy
\newcommand{\Ezero}{E_0}                  % Characteristic energy
\newcommand{\Hubble}{H}                   % Hubble constant
\newcommand{\Kspec}{K_{\text{spec}}}      % Spectral correction
\newcommand{\Lambdat}{\Lambda_t}          % Time-related cosmological constant
\newcommand{\Leff}{\mathcal{L}_{\text{eff}}}  % Effective Lagrangian
\newcommand{\Lorentz}{\mathcal{L}}        % Lorentz symbol
\newcommand{\Lxi}{L_\xi}                  % Xi length
\newcommand{\Tfield}{T}                   % Time field
\newcommand{\Weyl}{W}                     % Weyl tensor/symbol
\newcommand{\alphaEMSI}{\alpha_{\text{EM,SI}}}  % EM alpha in SI
\newcommand{\alphaEMnat}{\alpha_{\text{EM,nat}}}  % EM alpha in natural units
\newcommand{\alphaem}{\alpha_{\text{em}}} % Electromagnetic alpha
\newcommand{\betaTSI}{\beta_{T,\text{SI}}}  % Beta in SI
\newcommand{\betaTnat}{\beta_{T,\text{nat}}}  % Beta in natural units
\newcommand{\deltam}{\delta m}            % Mass difference
\newcommand{\phiT}{\phi_T}                % T-field phi
\newcommand{\tP}{t_P}                     % Planck time
\newcommand{\rhoCMB}{\rho_{\text{CMB}}}   % CMB density
\newcommand{\rhoCasimir}{\rho_{\text{Casimir}}}  % Casimir density

% Table formatting
\usepackage{multirow}

% Additional physics macros
\newcommand{\Riem}{\mathcal{R}}           % Riemann tensor
\newcommand{\ZPinch}{Z_{\text{pinch}}}    % Z-pinch
\newcommand{\SynchPower}{P_{\text{synch}}} % Synchrotron power
\newcommand{\Rzero}{R_0}                  % Characteristic radius
\newcommand{\alphafine}{\alpha}           % Fine structure constant
\newcommand{\Etau}{E_\tau}                % Tau energy
\newcommand{\deltaE}{\delta E}            % Energy deviation
\newcommand{\EPlanck}{E_P}                % Planck energy
\newcommand{\pichar}{\pi}                 % Pi character
\newcommand{\alphaWSI}{\alpha_{W,\text{SI}}}  % Wien alpha in SI
\newcommand{\alphaWnat}{\alpha_{W,\text{nat}}}  % Wien alpha in natural units

% Einfache abstract-Umgebung für Kapitel:
\newenvironment{abstract}{%
  \begin{center}\bfseries Abstract\end{center}\small
}{\par}


\title{E-mc2 En}
\author{J. Pascher}
\date{\today}

\begin{document}
\maketitle

\section*{E Mc2 (E-mc2)}

	\begin{abstract}
		This work reveals the central point of Einstein's relativity theory: E=mc² is mathematically identical to E=m. The only difference lies in Einstein's treatment of c as a "constant" instead of a dynamic ratio. By fixing c = 299,792,458 m/s, the natural time-mass duality T·m = 1 is artificially "frozen," leading to apparent complexity. The T0 theory shows: c is not a fundamental law of nature, but only a ratio that must be variable if time is variable. Einstein's error was not E=mc² itself, but the constant-setting of c.
	\end{abstract}
	
	
	\section{The Central Thesis: E=mc² = E=m}
	
	\subsubsection*{The Fundamental Recognition}
\section*{E=mc² and E=m are mathematically identical!}
		
		The only difference: Einstein treats c as a "constant," although c is a dynamic ratio.
		
		\textbf{Einstein's error}: c = 299,792,458 m/s = constant
		
		\textbf{T0 truth}: c = L/T = variable ratio

	
	\subsection{The Mathematical Identity}
	
	\textbf{In natural units}:
	\begin{equation}
		E = mc^2 = m \times c^2 = m \times 1^2 = m
	\end{equation}
	
\section*{This is not an approximation - this is exactly the same equation!}
	
	\subsection{What is c really?}
	
	\begin{equation}
		c = \frac{\text{Length}}{\text{Time}} = \frac{L}{T}
	\end{equation}
	
\section*{c is a ratio, not a natural constant!}
	
	\section{Einstein's Fundamental Error: The Constant-Setting}
	
	\subsection{The Act of Constant-Setting}
	
	Einstein set: $c = 299,792,458$ m/s = \textbf{constant}
	
\section*{What does this mean?}
	\begin{equation}
		c = \frac{L}{T} = \text{constant} \quad \Rightarrow \quad \frac{L}{T} = \text{fixed}
	\end{equation}
	
	\textbf{Implication}: If L and T can vary, their \textbf{ratio} must remain constant.
	
	\subsection{The Problem of Time Variability}
	
	\textbf{Einstein recognized himself}: Time dilates!
	\begin{equation}
		t' = \gamma t \quad \text{(time is variable)}
	\end{equation}
	
	\textbf{But simultaneously he claimed}: 
	\begin{equation}
		c = \frac{L}{T} = \text{constant}
	\end{equation}
	
\section*{This is a logical contradiction!}
	
	\subsection{The T0 Resolution}
	
	\textbf{T0 insight}: $\Tfield \cdot m = 1$
	
	This means:
	\begin{itemize}
		\item Time $\Tfield$ \textbf{must} be variable (coupled to mass)
		\item Therefore $c = L/T$ \textbf{cannot} be constant
		\item $c$ is a \textbf{dynamic ratio}, not a constant
	\end{itemize}
	
	\section{The Constants Illusion: How it Works}
	
	\subsection{The Mechanism of the Illusion}
	
	\textbf{Step 1}: Einstein sets c = constant
	\begin{equation}
		c = 299,792,458 \text{ m/s} = \text{fixed}
	\end{equation}
	
	\textbf{Step 2}: Time becomes "frozen" by this
	\begin{equation}
		T = \frac{L}{c} = \frac{L}{\text{constant}} = \text{apparently determined}
	\end{equation}
	
	\textbf{Step 3}: Time dilation becomes "mysterious effect"
	\begin{equation}
		t' = \gamma t \quad \text{(why? $\rightarrow$ complicated relativity theory)}
	\end{equation}
	
	\subsection{What Really Happens (T0 View)}
	
	\textbf{Reality}: Time is naturally variable through $\Tfield \cdot m = 1$
	
	\textbf{Einstein's constant-setting} "freezes" this natural variability artificially
	
	\textbf{Result}: One needs complicated theory to repair the "frozen" dynamics
	
	\section{c as Ratio vs. c as Constant}
	
	\subsection{c as Natural Ratio (T0)}
	
	\begin{equation}
		c(x,t) = \frac{L(x,t)}{T(x,t)}
	\end{equation}
	
	\textbf{Properties}:
	\begin{itemize}
		\item $c$ varies with location and time
		\item $c$ follows the time-mass duality
		\item No artificial constants
		\item Natural simplicity: $E = m$
	\end{itemize}
	
	\subsection{c as Artificial Constant (Einstein)}
	
	\begin{equation}
		c = 299,792,458 \text{ m/s} = \text{constant everywhere}
	\end{equation}
	
	\textbf{Problems}:
	\begin{itemize}
		\item Contradiction to time dilation
		\item Artificial "freezing" of time dynamics
		\item Complicated repair mathematics needed
		\item Inflated formula: $E = mc^2$
	\end{itemize}
	
	\section{The Time Dilation Paradox}
	
	\subsection{Einstein's Contradiction Exposed}
	
	\textbf{Einstein claims simultaneously}:
	\begin{align}
		c &= \text{constant} \\
		t' &= \gamma t \quad \text{(time varies)}
	\end{align}
	
	\textbf{But}:
	\begin{equation}
		c = \frac{L}{T} \quad \text{and} \quad T \text{ varies} \quad \Rightarrow \quad c \text{ cannot be constant!}
	\end{equation}
	
	\subsection{Einstein's Hidden Solution}
	
	Einstein "solves" the contradiction through:
	\begin{itemize}
		\item Complicated Lorentz transformations
		\item Mathematical formalisms
		\item Space-time constructions
		\item \textbf{But the logical contradiction remains!}
	\end{itemize}
	
	\subsection{T0's Natural Solution}
	
	\textbf{No contradiction in T0}:
	\begin{equation}
		\Tfield \cdot m = 1 \quad \Rightarrow \quad \text{time is naturally variable}
	\end{equation}
	
	\begin{equation}
		c = \frac{L}{T} \quad \Rightarrow \quad \text{c is naturally variable}
	\end{equation}
	
\section*{No constant-setting $\rightarrow$ No contradictions $\rightarrow$ No complicated repair mathematics}
	
	\section{The Mathematical Demonstration}
	
	\subsection{From E=mc² to E=m}
	
	\textbf{Starting equation}: $E = mc^2$
	
	\textbf{c in natural units}: $c = 1$
	
	\textbf{Substitution}:
	\begin{equation}
		E = mc^2 = m \times 1^2 = m
	\end{equation}
	
	\textbf{Result}: $E = m$
	
	\subsection{The Reverse Direction: From E=m to E=mc²}
	
	\textbf{Starting equation}: $E = m$
	
	\textbf{Artificial constant introduction}: $c = 299,792,458$ m/s
	
	\textbf{Inflating the equation}:
	\begin{equation}
		E = m = m \times 1 = m \times \frac{c^2}{c^2} = m \times c^2 \times \frac{1}{c^2}
	\end{equation}
	
	\textbf{If one defines $c^2$ as "conversion factor"}:
	\begin{equation}
		E = mc^2
	\end{equation}
	
	\textbf{This shows}: $E = mc^2$ is only $E = m$ with \textbf{artificial inflation factor} $c^2$!
	
	\section{The Arbitrariness of Constant Choice: c or Time?}
	
	\subsection{Einstein's Arbitrary Decision}
	
	\subsubsection*{The Fundamental Choice Option}
\section*{One can choose what should be "constant"!}
		
		\textbf{Option 1 (Einstein's choice)}: c = constant $\rightarrow$ time becomes variable
		
		\textbf{Option 2 (alternative)}: time = constant $\rightarrow$ c becomes variable
		
\section*{Both describe the same physics!}

	
	\subsection{Option 1: Einstein's c-constant}
	
	\textbf{Einstein chose}:
	\begin{align}
		c &= 299,792,458 \text{ m/s} = \text{constant (defined)} \\
		t' &= \gamma t \quad \text{(time becomes automatically variable)}
	\end{align}
	
	\textbf{Language convention}:
	\begin{itemize}
		\item "Speed of light is universally constant"
		\item "Time dilates in strong gravitational fields"
		\item "Clocks run slower at high velocities"
	\end{itemize}
	
	\subsection{Option 2: Time-constant (Einstein could have chosen)}
	
	\textbf{Alternative choice}:
	\begin{align}
		t &= \text{constant (defined)} \\
		c(x,t) &= \frac{L(x,t)}{t} = \text{variable}
	\end{align}
	
	\textbf{Alternative language convention}:
	\begin{itemize}
		\item "Time flows equally everywhere"
		\item "Speed of light varies with location"
		\item "Light becomes slower in strong gravitational fields"
	\end{itemize}
	
	\subsection{Mathematical Equivalence of Both Options}
	
	\textbf{Both descriptions are mathematically identical}:
	
	\begin{table}[htbp]
		\centering
		\begin{tabular}{|l|c|c|}
			\hline
			\textbf{Phenomenon} & \textbf{Einstein view} & \textbf{Time-constant view} \\
			\hline
			Gravitation & Time slows down & Light slows down \\
			Velocity & Time dilation & c-variation \\
			GPS correction & "Clocks run differently" & "c is different" \\
			Measurements & Same numbers & Same numbers \\
			\hline
		\end{tabular}
		\caption{Two views, identical physics}
	\end{table}
	
	\subsection{Why Einstein Chose Option 1}
	
	\textbf{Historical reasons for Einstein's decision}:
	\begin{itemize}
		\item \textbf{Michelson-Morley}: c seemed locally constant
		\item \textbf{Aesthetics}: "Universal constant" sounded elegant
		\item \textbf{Tradition}: Newtonian constant physics
		\item \textbf{Conceivability}: c-constancy easier to imagine than time constancy
		\item \textbf{Authority effect}: Einstein's prestige fixed this choice
	\end{itemize}
	
\section*{But it was only a convention, not a natural law!}
	
	\subsection{T0's Overcoming of Both Options}
	
	\textbf{T0 shows}: Both choices are arbitrary!
	
	\begin{equation}
		\Tfield \cdot m = 1 \quad \text{(natural duality without constant constraint)}
	\end{equation}
	
	\textbf{T0 insight}:
	\begin{itemize}
		\item \textbf{Neither} c nor time are "really" constant
		\item \textbf{Both} are aspects of the same T·m dynamics
		\item \textbf{Constancy} is only definition convention
		\item \textbf{E = m} is the constant-free truth
	\end{itemize}
	
	\subsection{Liberation from Constant Constraint}
	
	\textbf{Instead of choosing between}:
	\begin{itemize}
		\item c constant, time variable (Einstein)
		\item Time constant, c variable (alternative)
	\end{itemize}
	
	\textbf{T0 chooses}:
	\begin{itemize}
		\item \textbf{Both dynamically coupled} via T·m = 1
		\item \textbf{No arbitrary fixations}
		\item \textbf{Natural ratios} instead of artificial constants
	\end{itemize}
	
	\section{The Reference Point Revolution: Earth Sun Nature}
	
	\subsection{The Reference Point Analogy: Geocentric Heliocentric T0}
	
	\subsubsection*{The Reference Point Revolution: From Earth $\rightarrow$ Sun $\rightarrow$ Nature}
\textbf{Geocentric (Ptolemy)}: Earth at center \\
		- Complicated epicycles needed \\
		- Works, but artificially complicated \\
		
		\textbf{Heliocentric (Copernicus)}: Sun at center \\
		- Simple ellipses \\
		- Much more elegant and simple \\
		
		\textbf{T0-centric}: Natural ratios at center \\
		- $\Tfield \cdot m = 1$ (natural reference point) \\
		- Even more elegant: $E = m$

	
	\textbf{Einstein's c-constant corresponds to the geocentric system}:
	\begin{itemize}
		\item \textbf{Human} reference point at center (like Earth at center)
		\item \textbf{Complicated} mathematics needed (like epicycles)
		\item \textbf{Works} locally, but artificially inflated
	\end{itemize}
	
	\textbf{T0's natural ratios correspond to the heliocentric system}:
	\begin{itemize}
		\item \textbf{Natural} reference point at center (like Sun at center)
		\item \textbf{Simple} mathematics (like ellipses)
		\item \textbf{Universally} valid and elegant
	\end{itemize}
	
	\subsection{Why We Need Reference Points}
	
	\textbf{Reference points are necessary and natural}:
	\begin{itemize}
		\item \textbf{For measurements}: We need standards for comparison
		\item \textbf{For communication}: Common basis for exchange
		\item \textbf{For technology}: Practical applications require units
		\item \textbf{For science}: Reproducible experiments need standards
	\end{itemize}
	
	\textbf{The question is not WHETHER, but WHICH reference point}:
	
	\begin{table}[htbp]
		\centering
		\begin{tabular}{|l|c|c|c|}
			\hline
			\textbf{System} & \textbf{Reference Point} & \textbf{Complexity} & \textbf{Elegance} \\
			\hline
			Geocentric & Earth & Epicycles & Low \\
			Heliocentric & Sun & Ellipses & High \\
			Einstein & c-constant & Relativity theory & Medium \\
			T0 & $\Tfield \cdot m = 1$ & $E = m$ & Maximum \\
			\hline
		\end{tabular}
		\caption{Reference point systems comparison}
	\end{table}
	
	\subsection{The Right vs. Wrong Reference Point}
	
	\textbf{Einstein's error was not to choose a reference point}:
	\begin{itemize}
		\item \textbf{But to choose the wrong reference point!}
	\end{itemize}
	
	\textbf{Wrong reference point (Einstein)}: c = 299,792,458 m/s = constant
	\begin{itemize}
		\item Based on human definition
		\item Leads to complicated mathematics
		\item Creates logical contradictions
	\end{itemize}
	
	\textbf{Right reference point (T0)}: $\Tfield \cdot m = 1$
	\begin{itemize}
		\item Based on natural ratio
		\item Leads to simple mathematics: $E = m$
		\item No contradictions, pure elegance
	\end{itemize}
	
	\section{When Something Becomes "Constant"}
	
	\subsection{The Fundamental Reference Point Problem}
	
	\subsubsection*{The Reference Point Illusion}
\section*{Something only becomes "constant" when we define a reference point!}
		
		\textbf{Without reference point}: All ratios are relative and dynamic
		
		\textbf{With reference point}: One ratio becomes artificially "fixed"
		
		\textbf{Einstein's error}: He defined an absolute reference point for c

	
	\subsection{The Natural Stage: Everything is Relative}
	
	\textbf{Before any reference point definition}:
	\begin{align}
		c_1 &= \frac{L_1}{T_1} \\
		c_2 &= \frac{L_2}{T_2} \\
		c_3 &= \frac{L_3}{T_3} \\
		&\vdots
	\end{align}
	
	\textbf{All c-values are relative to each other}. None is "constant".
	
	\subsection{The Moment of Reference Point Setting}
	
	\textbf{Einstein's fatal step}:
	\begin{equation}
		\text{"I define: } c = 299,792,458 \text{ m/s = reference point"}
	\end{equation}
	
	\textbf{What happens at this moment}:
	\begin{itemize}
		\item An \textbf{arbitrary reference point} is set
		\item All other c-values are measured relative to this
		\item The \textbf{dynamic ratio} becomes a "constant"
		\item The \textbf{natural relativity} is artificially "frozen"
	\end{itemize}
	
	\subsection{The Reference Point Problematic}
	
	\textbf{Every reference point is arbitrary}:
	\begin{itemize}
		\item Why 299,792,458 m/s and not 300,000,000 m/s?
		\item Why in m/s and not in other units?
		\item Why measured on Earth and not in space?
		\item Why at this time and not at another?
	\end{itemize}
	
	\subsection{T0's Reference Point-Free Physics}
	
	\textbf{T0 eliminates all reference points}:
	\begin{equation}
		\Tfield \cdot m = 1 \quad \text{(universal relation without reference point)}
	\end{equation}
	
	\begin{itemize}
		\item No arbitrary fixations
		\item All ratios remain dynamic
		\item Natural relativity is preserved
		\item Fundamental simplicity: $E = m$
	\end{itemize}
	
	\subsection{Example: The Meter Definition}
	
	\textbf{Historical development of meter definition}:
	\begin{enumerate}
		\item \textbf{1793}: 1 meter = 1/10,000,000 of Earth meridian (Earth reference point)
		\item \textbf{1889}: 1 meter = prototype meter in Paris (object reference point)  
		\item \textbf{1960}: 1 meter = 1,650,763.73 wavelengths of krypton-86 (atom reference point)
		\item \textbf{1983}: 1 meter = distance light travels in 1/299,792,458 s (c reference point)
	\end{enumerate}
	
\section*{What does this show?}
	\begin{itemize}
		\item Each definition is \textbf{human arbitrariness}
		\item The \textbf{reference point} changes with human technology
		\item There is \textbf{no "natural" length unit} - only human agreements
		\item \textbf{Humans make c "constant" by definition} - not nature!
	\end{itemize}
	
	\subsection{The Circular Error: Humans Define Their Own "Constants"}
	
	\textbf{In 1983 humans defined}:
	\begin{equation}
		1 \text{ meter} = \frac{1}{299,792,458} \times c \times 1 \text{ second}
	\end{equation}
	
	\textbf{This makes c automatically "constant"} - through human definition, not through natural law:
	\begin{equation}
		c = \frac{299,792,458 \text{ meters}}{1 \text{ second}} = 299,792,458 \text{ m/s}
	\end{equation}
	
	\textbf{Circular reasoning}: Humans define c as constant and then "measure" a constant!
	
\section*{Nature is not asked in this process!}
	
	\subsection{T0's Resolution of the Reference Point Illusion}
	
	\textbf{T0 recognizes}:
	\begin{itemize}
		\item \textbf{Definition $\neq$ natural law}
		\item \textbf{Measurement reference point $\neq$ physical constant}
		\item \textbf{Practical agreement $\neq$ fundamental truth}
	\end{itemize}
	
	\textbf{T0 solution}:
	\begin{align}
		\text{For measurements:} \quad &\text{Use practical reference points} \\
		\text{For natural laws:} \quad &\text{Use reference point-free relations}
	\end{align}
	
	\section{Why c-Constancy is Not Provable}
	
	\subsection{The Fundamental Measurement Problem}
	
	\textbf{To measure c, we need}:
	\begin{equation}
		c = \frac{L}{T}
	\end{equation}
	
	\textbf{But}: We measure L and T with \textbf{the same physical processes} that depend on c!
	
	\textbf{Circular problem}:
	\begin{itemize}
		\item Light measures distances $\rightarrow$ c determines L
		\item Atomic clocks use EM transitions $\rightarrow$ c influences T
		\item Then we measure c = L/T $\rightarrow$ \textbf{We measure c with c!}
	\end{itemize}
	
	\subsection{The Gauge Definition Problem}
	
	\textbf{Since 1983}: 1 meter = distance light travels in 1/299,792,458 s
	
	\begin{equation}
		c = 299,792,458 \text{ m/s} \quad \text{(not measured, but defined!)}
	\end{equation}
	
\section*{One cannot "prove" what one has defined!}
	
	\subsection{The Systematic Compensation Problem}
	
	\textbf{If c varies, ALL measuring devices vary equally}:
	\begin{itemize}
		\item \textbf{Laser interferometers}: use light (c-dependent)
		\item \textbf{Atomic clocks}: use EM transitions (c-dependent)
		\item \textbf{Electronics}: uses EM signals (c-dependent)
	\end{itemize}
	
	\textbf{Result}: All devices \textbf{automatically compensate} the c-variation!
	
	\subsection{The Burden of Proof Problem}
	
	\textbf{Scientifically correct}:
	\begin{itemize}
		\item One \textbf{cannot prove} that something is constant
		\item One can only show that it \textbf{appears constant within measurement precision}
		\item \textbf{Each new precision level} could show variation
	\end{itemize}
	
\section*{Einstein's "c-constancy" was belief, not proof!}
	
	\subsection{T0 Prediction for Precise Measurements}
	
	\textbf{T0 predicts}: At highest precision one will find:
	\begin{equation}
		c(x,t) = c_0 \left(1 + \xipar \times \frac{\Tfield(x,t) - \Tfield_0}{\Tfield_0}\right)
	\end{equation}
	
	with $\xipar = 1.33 \times 10^{-4}$ (T0 parameter)
	
	\textbf{c varies tiny ($\sim$10$^{-15}$), but measurable in principle!}
	
	\section{Ontological Consideration: Calculations as Constructs}
	
	\subsection{The Fundamental Epistemological Limit}
	
	\subsubsection*{Ontological Truth}
\section*{All calculations are human constructs!}
		
		They can \textbf{at best} give a certain idea of reality.
		
		\textbf{That calculations are internally consistent proves little} about actual reality.
		
\section*{Mathematical consistency $\neq$ ontological truth}

	
	\subsection{Einstein's Construct vs. T0's Construct}
	
	\textbf{Both are human thought structures}:
	
	\textbf{Einstein's construct}:
	\begin{itemize}
		\item E = mc² (mathematically consistent)
		\item Relativity theory (internally coherent)
		\item 10 field equations (work computationally)
		\item \textbf{But}: Based on arbitrary c-constant setting
	\end{itemize}
	
	\textbf{T0's construct}:
	\begin{itemize}
		\item E = m (mathematically simpler)
		\item T·m = 1 (internally coherent)
		\item $\partial^2 E = 0$ (works computationally)
		\item \textbf{But}: Also only a human thought model
	\end{itemize}
	
	\subsection{The Ontological Relativity}
	
\section*{What is "really" real?}
	\begin{itemize}
		\item \textbf{Einstein's space-time}? (construct)
		\item \textbf{T0's energy field}? (construct)
		\item \textbf{Newton's absolute time}? (construct)
		\item \textbf{Quantum mechanics' probabilities}? (construct)
	\end{itemize}
	
\section*{All are human interpretive frameworks of the inaccessible reality!}
	
	\subsection{Why T0 is Still "Better"}
	
	\textbf{Not because of "absolute truth," but because of}:
	
	\textbf{1. Simplicity (Occam's Razor)}:
	\begin{itemize}
		\item E = m is simpler than E = mc²
		\item One equation is simpler than 10 equations
		\item Fewer arbitrary assumptions
	\end{itemize}
	
	\textbf{2. Consistency}:
	\begin{itemize}
		\item No logical contradictions (like Einstein's)
		\item No constant arbitrariness
		\item Unified thought structure
	\end{itemize}
	
	\textbf{3. Predictive power}:
	\begin{itemize}
		\item Testable predictions
		\item Fewer free parameters
		\item Clearer experimental distinction
	\end{itemize}
	
	\textbf{4. Aesthetics}:
	\begin{itemize}
		\item Mathematical elegance
		\item Conceptual clarity
		\item Unity
	\end{itemize}
	
	\subsection{The Epistemological Humility}
	
\section*{T0 does NOT claim to be "absolute truth."}
	
	\textbf{T0 only says}:
	\begin{itemize}
		\item "Here is a \textbf{simpler} construct"
		\item "With \textbf{fewer} arbitrary assumptions"
		\item "That is \textbf{more consistent} than Einstein's construct"
		\item "And makes \textbf{more testable} predictions"
	\end{itemize}
	
\section*{But ultimately T0 also remains a human thought structure!}
	
	\subsection{The Pragmatic Consequence}
	
	\textbf{Since all theories are constructs}:
	
	\textbf{Evaluation criteria are}:
	\begin{enumerate}
		\item \textbf{Simplicity} (fewer assumptions)
		\item \textbf{Consistency} (no contradictions)
		\item \textbf{Predictive power} (testable consequences)
		\item \textbf{Elegance} (aesthetic criteria)
		\item \textbf{Unity} (fewer separate domains)
	\end{enumerate}
	
\section*{By all these criteria T0 is "better" than Einstein - but not "absolutely true".}
	
	\subsection{The Ontological Humility}
	
	\textbf{The deepest insight}:
	\begin{itemize}
		\item \textbf{Reality itself} is inaccessible
		\item \textbf{All theories} are human constructs
		\item \textbf{Mathematical consistency} proves no ontological truth
		\item \textbf{The best} we have: \textbf{Simpler, more consistent constructs}
	\end{itemize}
	
\section*{Einstein's error was not only the c-constant setting, but also the claim to absolute truth of his mathematical constructs.}
	
\section*{T0's advantage is not absolute truth, but relative superiority as a thought model.}
	
	\section{The Practical Consequences}
	
	\subsection{Why E=mc² "Works"}
	
	\textbf{E=mc² works because}:
	\begin{itemize}
		\item It is mathematically identical to $E = m$
		\item $c^2$ compensates the "frozen" time dynamics
		\item The T0 truth is unconsciously contained
		\item Local approximations usually suffice
	\end{itemize}
	
	\subsection{When E=mc² Fails}
	
	\textbf{The constants illusion breaks down at}:
	\begin{itemize}
		\item Very precise measurements
		\item Extreme conditions (high energies/masses)
		\item Cosmological scales
		\item Quantum gravity
	\end{itemize}
	
	\subsection{T0's Universal Validity}
	
	\textbf{E = m is valid everywhere and always}:
	\begin{itemize}
		\item No approximations needed
		\item No constant assumptions
		\item Universal applicability
		\item Fundamental simplicity
	\end{itemize}
	
	\section{The Correction of Physics History}
	
	\subsection{Einstein's True Achievement}
	
	\textbf{Einstein's actual discovery was}:
	\begin{equation}
		E = m \quad \text{(in natural form)}
	\end{equation}
	
	\textbf{His error was}:
	\begin{equation}
		E = mc^2 \quad \text{(with artificial constant inflation)}
	\end{equation}
	
	\subsection{The Historical Irony}
	
	\subsubsection*{The Great Irony}
Einstein discovered the fundamental simplicity $E = m$, 
		
		but \textbf{hid it behind the constants illusion} $E = mc^2$!
		
		The physics world celebrated the complicated form and overlooked the simple truth.

	
	\section{The T0 Perspective: c as Living Ratio}
	
	\subsection{c as Expression of Time-Mass Duality}
	
	\textbf{In T0 theory}:
	\begin{equation}
		c(x,t) = f\left(\frac{L(x,t)}{\Tfield(x,t)}\right) = f\left(\frac{L(x,t) \cdot m(x,t)}{1}\right)
	\end{equation}
	
	since $\Tfield \cdot m = 1$.
	
\section*{c becomes an expression of the fundamental time-mass duality!}
	
	\subsection{The Dynamic Speed of Light}
	
	\textbf{T0 prediction}: 
	\begin{equation}
		c(x,t) = c_0 \sqrt{1 + \xipar \frac{m(x,t) - m_0}{m_0}}
	\end{equation}
	
\section*{Light moves faster in more massive regions!}
	
	(Tiny effect, but measurable in principle)
	
	\section{Experimental Tests of c-Variability}
	
	\subsection{Proposed Experiments}
	
	\textbf{Test 1 - Gravitational dependence}:
	\begin{itemize}
		\item Measure c in different gravitational fields
		\item T0 prediction: $c$ varies with $\sim \xipar \times \Delta\Phi_{\text{grav}}$
	\end{itemize}
	
	\textbf{Test 2 - Cosmological variation}:
	\begin{itemize}
		\item Measure c over cosmological time periods
		\item T0 prediction: $c$ changes with universe expansion
	\end{itemize}
	
	\textbf{Test 3 - High-energy physics}:
	\begin{itemize}
		\item Measure c in particle accelerators at highest energies
		\item T0 prediction: Tiny deviations at $E \sim$ TeV
	\end{itemize}
	
	\subsection{Expected Results}
	
	\begin{table}[htbp]
		\centering
		\begin{tabular}{|l|c|c|}
			\hline
			\textbf{Experiment} & \textbf{Einstein (c constant)} & \textbf{T0 (c variable)} \\
			\hline
			Gravitational field & $c = 299792458$ m/s & $c(1 \pm 10^{-15})$ \\
			Cosmological time & $c = $ constant & $c(1 + 10^{-12} \times t)$ \\
			High energy & $c = $ constant & $c(1 + 10^{-16})$ \\
			\hline
		\end{tabular}
		\caption{Predicted c-variations}
	\end{table}
	
	\section{Conclusions}
	
	\subsection{The Central Recognition}
	
	\subsubsection*{The Fundamental Truth}
\section*{E=mc² = E=m}
		
		Einstein's "constant" c is in truth a variable ratio.
		
		The constant-setting was Einstein's fundamental error.
		
		T0 corrects this error by returning to natural variability.

	
	\subsection{Physics After the Constants Illusion}
	
	\textbf{The future of physics}:
	\begin{itemize}
		\item No artificial constants
		\item Dynamic ratios everywhere
		\item Living, variable natural laws
		\item Fundamental simplicity: $E = m$
	\end{itemize}
	
	\subsection{Einstein's Corrected Legacy}
	
	\textbf{Einstein's true discovery}: $E = m$ (energy-mass identity)
	
	\textbf{Einstein's error}: Constant-setting of c
	
	\textbf{T0's correction}: Return to natural form $E = m$
	
\section*{Einstein was brilliant - he just stopped one step too early!}


% Bibliography
\begin{thebibliography}{99}
	
	\bibitem{pdg2024}
	Particle Data Group Collaboration (2024). 
	\textit{Review of Particle Physics}. 
	Progress of Theoretical and Experimental Physics, 2024(8), 083C01.
	\url{https://pdg.lbl.gov}
	
	\bibitem{flag2024}
	Aoki, Y., et al. (FLAG Collaboration) (2024). 
	\textit{FLAG Review 2024 of Lattice Results for Low-Energy Constants}. 
	arXiv:2411.04268.
	\url{https://arxiv.org/abs/2411.04268}
	
	\bibitem{fermilab_muon_g2}
	Abi, B., et al. (Muon g-2 Collaboration) (2021). 
	\textit{Measurement of the Positive Muon Anomalous Magnetic Moment to 0.46 ppm}. 
	Physical Review Letters, 126, 141801.
	
	\bibitem{peskin_schroeder}
	Peskin, M. E., \& Schroeder, D. V. (1995). 
	\textit{An Introduction to Quantum Field Theory}. 
	Addison-Wesley.
	
	\bibitem{weinberg_qft}
	Weinberg, S. (1995). 
	\textit{The Quantum Theory of Fields, Vol. I--III}. 
	Cambridge University Press.
	
	\bibitem{griffiths_particle}
	Griffiths, D. (2008). 
	\textit{Introduction to Elementary Particles}. 
	Wiley-VCH.
	
	\bibitem{mandl_shaw}
	Mandl, F., \& Shaw, G. (2010). 
	\textit{Quantum Field Theory (2nd ed.)}. 
	Wiley.
	
	\bibitem{srednicki_qft}
	Srednicki, M. (2007). 
	\textit{Quantum Field Theory}. 
	Cambridge University Press.
	
	\bibitem{t0_fundamentals}
	Pascher, J. (2024). 
	\textit{T0-Theory: Foundations of Time-Mass Duality}. 
	Unpublished manuscript, HTL Leonding.
	
	\bibitem{t0_fine_structure}
	Pascher, J. (2024). 
	\textit{T0-Theory: The Fine Structure Constant}. 
	Unpublished manuscript, HTL Leonding.
	
	\bibitem{t0_neutrinos}
	Pascher, J. (2024). 
	\textit{T0-Theory: Neutrino Masses and PMNS Mixing}. 
	Unpublished manuscript, HTL Leonding.
	
	\bibitem{t0_github}
	Pascher, J. (2024--2025). 
	\textit{T0-Time-Mass-Duality Repository}. 
	GitHub.
	\url{https://github.com/jpascher/T0-Time-Mass-Duality}
	
	\bibitem{lattice_qcd_review}
	Kronfeld, A. S. (2012). 
	\textit{Twenty-first Century Lattice Gauge Theory: Results from the QCD Lagrangian}. 
	Annual Review of Nuclear and Particle Science, 62, 265--284.
	
	\bibitem{neutrino_mixing_pdg}
	Particle Data Group Collaboration (2024). 
	\textit{Neutrino Masses, Mixing, and Oscillations}. 
	PDG Review 2024.
	\url{https://pdg.lbl.gov/2024/reviews/rpp2024-rev-neutrino-mixing.pdf}
	
	\bibitem{higgs_discovery}
	ATLAS and CMS Collaborations (2012). 
	\textit{Observation of a New Particle in the Search for the Standard Model Higgs Boson}. 
	Physics Letters B, 716, 1--29.
	
	\bibitem{Brannen2005}
	C. P. Brannen, ``Estimate of neutrino masses from Koide's relation'', \textit{arXiv:hep-ph/0505028} (2005).
	\url{https://arxiv.org/abs/hep-ph/0505028}
	
	\bibitem{Brannen2006}
	C. P. Brannen, ``Koide Mass Formula for Neutrinos'', \textit{arXiv:0702.0052} (2006).
	\url{http://brannenworks.com/MASSES.pdf}
	
	\bibitem{PhaseVectors2025}
	Anonymous, ``The Koide Relation and Lepton Mass Hierarchy from Phase Vectors'', \textit{rXiv:2507.0040} (2025).
	\url{https://rxiv.org/pdf/2507.0040v1.pdf}
	
	\bibitem{PDG2025}
	Particle Data Group, ``Review of Particle Physics'', \textit{Phys. Rev. D} \textbf{112} (2025) 030001.
	\url{https://pdg.lbl.gov/2025/}
	
	\bibitem{terrell2024}
	Terrell et al. (2024). 
	\textit{Single-Clock Metrology in Nature}. 
	Nature Physics.
	
	\bibitem{hossenfelder2024}
	Hossenfelder, S. (2024). 
	\textit{Single Clock Video Explanation}. 
	YouTube.
	
	\bibitem{hundert1931}
	Hundert (1931). 
	\textit{Reference Work}. 
	Publisher.
	
	\bibitem{terrell2025}
	Terrell et al. (2025). 
	\textit{Advanced Clock Synchronization Methods}. 
	Physical Review Letters.
	
	\bibitem{pascher_t0_2025}
	Pascher, J. (2025). 
	\textit{T0-Theory: Complete Framework and Applications}. 
	Unpublished manuscript, HTL Leonding.
	
	\bibitem{t0qm}
	Pascher, J. (2024). 
	\textit{T0-Theory: Quantum Mechanics Formulation}. 
	Unpublished manuscript, HTL Leonding.
	
	\bibitem{t0anomale}
	Pascher, J. (2024). 
	\textit{T0-Theory: Anomalous Magnetic Moments}. 
	Unpublished manuscript, HTL Leonding.
	
	\bibitem{muong2complete}
	Abi, B., et al. (Muon g-2 Collaboration) (2023). 
	\textit{Complete Measurement of the Positive Muon Anomalous Magnetic Moment}. 
	Physical Review Letters, 131, 161802.
	
	\bibitem{penrose2004}
	Penrose, R. (2004). 
	\textit{The Road to Reality: A Complete Guide to the Laws of the Universe}. 
	Jonathan Cape.
	
	\bibitem{planck1900}
	Planck, M. (1900). 
	\textit{On the Theory of the Energy Distribution Law of the Normal Spectrum}. 
	Verhandlungen der Deutschen Physikalischen Gesellschaft, 2, 237.
	
	\bibitem{T0Theory}
	Pascher, J. (2024). 
	\textit{T0-Theory: Fundamental Principles}. 
	Unpublished manuscript, HTL Leonding.
	
	% Additional bibliography entries for all undefined citations
	\bibitem{6g_roadmap}
	6G Research Consortium (2024).
	\textit{6G Technology Roadmap}.
	Technical Report.
	
	\bibitem{Born2013}
	Born, M. (2013).
	\textit{Einstein's Theory of Relativity}.
	Dover Publications.
	
	\bibitem{Casimir1948}
	Casimir, H. B. G. (1948).
	\textit{On the attraction between two perfectly conducting plates}.
	Proc. Kon. Ned. Akad. Wetensch. B51, 793--795.
	
	\bibitem{Einstein1905}
	Einstein, A. (1905).
	\textit{On the Electrodynamics of Moving Bodies}.
	Annalen der Physik, 17, 891--921.
	
	\bibitem{Feynman2006}
	Feynman, R. P. (2006).
	\textit{QED: The Strange Theory of Light and Matter}.
	Princeton University Press.
	
	\bibitem{Griffiths2017}
	Griffiths, D. J. (2017).
	\textit{Introduction to Electrodynamics (4th ed.)}.
	Cambridge University Press.
	
	\bibitem{Jackson1999}
	Jackson, J. D. (1999).
	\textit{Classical Electrodynamics (3rd ed.)}.
	Wiley.
	
	\bibitem{Mohr2016}
	Mohr, P. J., et al. (2016).
	\textit{CODATA Recommended Values of the Fundamental Physical Constants: 2014}.
	Rev. Mod. Phys. 88, 035009.
	
	\bibitem{Parker2018}
	Parker, R. H., et al. (2018).
	\textit{Measurement of the fine-structure constant as a test of the Standard Model}.
	Science, 360, 191--195.
	
	\bibitem{Planck1900}
	Planck, M. (1900).
	\textit{On the Theory of the Energy Distribution Law of the Normal Spectrum}.
	Verhandlungen der Deutschen Physikalischen Gesellschaft, 2, 237.
	
	\bibitem{Planck2018}
	Planck Collaboration (2018).
	\textit{Planck 2018 results. VI. Cosmological parameters}.
	Astronomy \& Astrophysics, 641, A6.
	
	\bibitem{QFT_T0}
	Pascher, J. (2024).
	\textit{T0-Theory and QFT Connections}.
	Unpublished manuscript, HTL Leonding.
	
	\bibitem{Sommerfeld1916}
	Sommerfeld, A. (1916).
	\textit{On the Quantum Theory of Spectral Lines}.
	Annalen der Physik, 51, 1--94.
	
	\bibitem{T0_Feinstruktur}
	Pascher, J. (2024).
	\textit{T0-Theory: Fine Structure Analysis}.
	Unpublished manuscript, HTL Leonding.
	
	\bibitem{T0_SI}
	Pascher, J. (2024).
	\textit{T0-Theory and SI Units}.
	Unpublished manuscript, HTL Leonding.
	
	\bibitem{T0_fine_structure}
	Pascher, J. (2024).
	\textit{T0-Theory: The Fine Structure Constant}.
	Unpublished manuscript, HTL Leonding.
	
	\bibitem{T0_g2_erweiterung}
	Pascher, J. (2024).
	\textit{T0-Theory: g-2 Extensions}.
	Unpublished manuscript, HTL Leonding.
	
	\bibitem{T0_gravitational_constant}
	Pascher, J. (2024).
	\textit{T0-Theory: Gravitational Constant Derivation}.
	Unpublished manuscript, HTL Leonding.
	
	\bibitem{T0_netze_en}
	Pascher, J. (2024).
	\textit{T0-Theory: Network Structures}.
	Unpublished manuscript, HTL Leonding.
	
	\bibitem{T0_tm_erweiterung}
	Pascher, J. (2024).
	\textit{T0-Theory: Time-Mass Extensions}.
	Unpublished manuscript, HTL Leonding.
	
	\bibitem{Uzan2003}
	Uzan, J.-P. (2003).
	\textit{The fundamental constants and their variation}.
	Rev. Mod. Phys. 75, 403--455.
	
	\bibitem{Weinberg1995}
	Weinberg, S. (1995).
	\textit{The Quantum Theory of Fields, Vol. I}.
	Cambridge University Press.
	
	\bibitem{albrecht1999}
	Albrecht, A. \& Magueijo, J. (1999).
	\textit{A time varying speed of light as a solution to cosmological puzzles}.
	Phys. Rev. D 59, 043516.
	
	\bibitem{alice2023}
	ALICE Collaboration (2023).
	\textit{Recent results from ALICE}.
	CERN-EP-2023-XXX.
	
	\bibitem{analog_optical}
	Smith, J. et al. (2024).
	\textit{Analog optical computing systems}.
	Nature Photonics.
	
	\bibitem{ashtekar2004}
	Ashtekar, A. \& Lewandowski, J. (2004).
	\textit{Background independent quantum gravity}.
	Class. Quantum Grav. 21, R53.
	
	\bibitem{atlas2023}
	ATLAS Collaboration (2023).
	\textit{ATLAS physics results}.
	CERN-PH-EP-2023-XXX.
	
	\bibitem{atlas2023higgs}
	ATLAS Collaboration (2023).
	\textit{Higgs boson measurements}.
	Phys. Rev. Lett.
	
	\bibitem{barbour1999}
	Barbour, J. (1999).
	\textit{The End of Time}.
	Oxford University Press.
	
	\bibitem{barrow1999}
	Barrow, J. D. (1999).
	\textit{Cosmologies with varying light speed}.
	Phys. Rev. D 59, 043515.
	
	\bibitem{becker2007}
	Becker, K. et al. (2007).
	\textit{String Theory and M-Theory}.
	Cambridge University Press.
	
	\bibitem{bell_muon}
	Bennett, G. W., et al. (Muon g-2 Collaboration) (2006).
	\textit{Final report of the E821 muon anomalous magnetic moment measurement}.
	Phys. Rev. D 73, 072003.
	
	\bibitem{bondi1948}
	Bondi, H. \& Gold, T. (1948).
	\textit{The steady-state theory of the expanding universe}.
	Mon. Not. R. Astron. Soc. 108, 252--270.
	
	\bibitem{brewer2019}
	Brewer, S. M. et al. (2019).
	\textit{Al+ Quantum-Logic Clock with Systematic Uncertainty below $10^{-18}$}.
	Phys. Rev. Lett. 123, 033201.
	
	\bibitem{cms2023top}
	CMS Collaboration (2023).
	\textit{Top quark measurements at CMS}.
	JHEP 2023.
	
	\bibitem{cms2024}
	CMS Collaboration (2024).
	\textit{CMS physics results 2024}.
	CERN-PH-EP-2024-XXX.
	
	\bibitem{codata2019}
	Tiesinga, E. et al. (2019).
	\textit{The 2018 CODATA Recommended Values}.
	J. Phys. Chem. Ref. Data.
	
	\bibitem{desi2025}
	DESI Collaboration (2025).
	\textit{DESI 2025 Cosmology Results}.
	arXiv preprint.
	
	\bibitem{differential_optical}
	Wang, X. et al. (2024).
	\textit{Differential optical computing}.
	Optica.
	
	\bibitem{dingle1972}
	Dingle, H. (1972).
	\textit{Science at the Crossroads}.
	Martin Brian \& O'Keeffe.
	
	\bibitem{divalentino2021}
	Di Valentino, E. et al. (2021).
	\textit{In the realm of the Hubble tension}.
	Class. Quantum Grav. 38, 153001.
	
	\bibitem{elnaschie2004}
	El Naschie, M. S. (2004).
	\textit{A review of E infinity theory}.
	Chaos, Solitons \& Fractals, 19, 209--236.
	
	\bibitem{fabrication_heterogeneous}
	Chen, Y. et al. (2024).
	\textit{Heterogeneous photonic integration}.
	Nature Electronics.
	
	\bibitem{fermilab2023}
	Fermilab (2023).
	\textit{Muon g-2 results}.
	Phys. Rev. Lett.
	
	\bibitem{flexible_wafer}
	Kim, S. et al. (2024).
	\textit{Flexible wafer-scale photonics}.
	Science Advances.
	
	\bibitem{francesco1997}
	Di Francesco, P. et al. (1997).
	\textit{Conformal Field Theory}.
	Springer.
	
	\bibitem{hartree1957}
	Hartree, D. R. (1957).
	\textit{The Calculation of Atomic Structures}.
	Wiley.
	
	\bibitem{hhi_6g}
	Fraunhofer HHI (2024).
	\textit{6G Photonic Integration}.
	Technical Report.
	
	\bibitem{hossenfelder2025}
	Hossenfelder, S. (2025).
	\textit{Science without the gobbledygook}.
	YouTube/Blog.
	
	\bibitem{hossenfelder_single_clock_video}
	Hossenfelder, S. (2024).
	\textit{The Single Clock Problem}.
	YouTube.
	
	\bibitem{hoyle1948}
	Hoyle, F. (1948).
	\textit{A new model for the expanding universe}.
	Mon. Not. R. Astron. Soc. 108, 372--382.
	
	\bibitem{integration_microelectronic}
	Liu, A. et al. (2024).
	\textit{Microelectronic photonic integration}.
	IEEE Journal.
	
	\bibitem{jacobson1995}
	Jacobson, T. (1995).
	\textit{Thermodynamics of spacetime}.
	Phys. Rev. Lett. 75, 1260.
	
	\bibitem{kasevich2023}
	Kasevich, M. et al. (2023).
	\textit{Atom interferometry tests}.
	Nature Physics.
	
	\bibitem{lerner2014}
	Lerner, E. J. (2014).
	\textit{An open letter on cosmology}.
	New Scientist.
	
	\bibitem{lisa2017}
	LISA Consortium (2017).
	\textit{Laser Interferometer Space Antenna}.
	ESA Technical Report.
	
	\bibitem{lithium_tantalate}
	Zhang, M. et al. (2024).
	\textit{Thin-film lithium tantalate photonics}.
	Nature Photonics.
	
	\bibitem{lopez2010}
	Lopez-Corredoira, M. (2010).
	\textit{Tests and problems of the standard model in cosmology}.
	Int. J. Mod. Phys. D.
	
	\bibitem{ludlow2015}
	Ludlow, A. D. et al. (2015).
	\textit{Optical atomic clocks}.
	Rev. Mod. Phys. 87, 637.
	
	\bibitem{mach1883}
	Mach, E. (1883).
	\textit{Die Mechanik in ihrer Entwickelung}.
	F.A. Brockhaus.
	
	\bibitem{maldacena1998}
	Maldacena, J. (1998).
	\textit{The large N limit of superconformal field theories}.
	Adv. Theor. Math. Phys. 2, 231--252.
	
	\bibitem{mueller2014}
	Müller, H. et al. (2014).
	\textit{Atom interferometry tests of the gravitational redshift}.
	Phys. Rev. Lett.
	
	\bibitem{mug2_final_2025}
	Muon g-2 Collaboration (2025).
	\textit{Final muon g-2 measurement}.
	Phys. Rev. Lett.
	
	\bibitem{muong2_2023}
	Muon g-2 Collaboration (2023).
	\textit{Updated muon g-2 results}.
	Phys. Rev. Lett.
	
	\bibitem{nathan2024}
	Nathan, A. et al. (2024).
	\textit{Quantum computing advances}.
	Nature.
	
	\bibitem{newell2018}
	Newell, D. B. et al. (2018).
	\textit{The CODATA 2017 values of h, e, k, and $N_A$}.
	Metrologia 55, L13.
	
	\bibitem{nottale1993}
	Nottale, L. (1993).
	\textit{Fractal Space-Time and Microphysics}.
	World Scientific.
	
	\bibitem{on_chip_lithium}
	Wang, C. et al. (2024).
	\textit{On-chip lithium niobate photonics}.
	Nature Communications.
	
	\bibitem{optical_advantages}
	Shastri, B. J. et al. (2024).
	\textit{Advantages of optical computing}.
	Nature Reviews Physics.
	
	\bibitem{pascher2025cmb}
	Pascher, J. (2025).
	\textit{T0-Theory: CMB Analysis}.
	Unpublished manuscript, HTL Leonding.
	
	\bibitem{pascher2025g2}
	Pascher, J. (2025).
	\textit{T0-Theory: g-2 Predictions}.
	Unpublished manuscript, HTL Leonding.
	
	\bibitem{pascher2025qm}
	Pascher, J. (2025).
	\textit{T0-Theory: Quantum Mechanics}.
	Unpublished manuscript, HTL Leonding.
	
	\bibitem{pascher2025si}
	Pascher, J. (2025).
	\textit{T0-Theory: SI Unit System}.
	Unpublished manuscript, HTL Leonding.
	
	\bibitem{pascher2025t0}
	Pascher, J. (2025).
	\textit{T0-Theory: Complete Framework}.
	Unpublished manuscript, HTL Leonding.
	
	\bibitem{pascher:fundamentals}
	Pascher, J. (2024).
	\textit{T0-Theory: Fundamentals}.
	Unpublished manuscript, HTL Leonding.
	
	\bibitem{pascher:g2_rev9}
	Pascher, J. (2024).
	\textit{T0-Theory: g-2 Revision 9}.
	Unpublished manuscript, HTL Leonding.
	
	\bibitem{pascher:geometric_formalism}
	Pascher, J. (2024).
	\textit{T0-Theory: Geometric Formalism}.
	Unpublished manuscript, HTL Leonding.
	
	\bibitem{pascher:ml_addendum}
	Pascher, J. (2024).
	\textit{T0-Theory: Machine Learning Addendum}.
	Unpublished manuscript, HTL Leonding.
	
	\bibitem{pascher:t0_foundations}
	Pascher, J. (2024).
	\textit{T0-Theory: Foundations}.
	Unpublished manuscript, HTL Leonding.
	
	\bibitem{pascher_derivation_beta_2025}
	Pascher, J. (2025).
	\textit{T0-Theory: Derivation of Beta}.
	Unpublished manuscript, HTL Leonding.
	
	\bibitem{pascher_higgs_connection_2025}
	Pascher, J. (2025).
	\textit{T0-Theory: Higgs Connection}.
	Unpublished manuscript, HTL Leonding.
	
	\bibitem{pascher_lagrangian_extended_2025}
	Pascher, J. (2025).
	\textit{T0-Theory: Extended Lagrangian}.
	Unpublished manuscript, HTL Leonding.
	
	\bibitem{pascher_mathematical_structure_2025}
	Pascher, J. (2025).
	\textit{T0-Theory: Mathematical Structure}.
	Unpublished manuscript, HTL Leonding.
	
	\bibitem{pascher_t0_cmb_2025}
	Pascher, J. (2025).
	\textit{T0-Theory: CMB Predictions}.
	Unpublished manuscript, HTL Leonding.
	
	\bibitem{pascher_t0_energie_2025}
	Pascher, J. (2025).
	\textit{T0-Theory: Energy}.
	Unpublished manuscript, HTL Leonding.
	
	\bibitem{pascher_t0_energy_2025}
	Pascher, J. (2025).
	\textit{T0-Theory: Energy Framework}.
	Unpublished manuscript, HTL Leonding.
	
	\bibitem{pascher_t0_theory_2025}
	Pascher, J. (2025).
	\textit{T0-Theory: Complete Theory}.
	Unpublished manuscript, HTL Leonding.
	
	\bibitem{penrose1959}
	Penrose, R. (1959).
	\textit{The apparent shape of a relativistically moving sphere}.
	Proc. Cambridge Phil. Soc. 55, 137--139.
	
	\bibitem{penrose1967}
	Penrose, R. (1967).
	\textit{Twistor algebra}.
	J. Math. Phys. 8, 345--366.
	
	\bibitem{peratt1992}
	Peratt, A. L. (1992).
	\textit{Physics of the Plasma Universe}.
	Springer-Verlag.
	
	\bibitem{peskin1995}
	Peskin, M. E. \& Schroeder, D. V. (1995).
	\textit{An Introduction to Quantum Field Theory}.
	Addison-Wesley.
	
	\bibitem{peskin_schroeder_1995}
	Peskin, M. E. \& Schroeder, D. V. (1995).
	\textit{An Introduction to Quantum Field Theory}.
	Addison-Wesley.
	
	\bibitem{phoquant}
	PhoQuant (2024).
	\textit{Photonic quantum computing}.
	Technical Report.
	
	\bibitem{photonics_ai}
	Wetzstein, G. et al. (2024).
	\textit{Photonics for AI}.
	Nature.
	
	\bibitem{planck1906}
	Planck, M. (1906).
	\textit{The Theory of Heat Radiation}.
	Johann Ambrosius Barth.
	
	\bibitem{planck2018}
	Planck Collaboration (2018).
	\textit{Planck 2018 results}.
	A\&A 641, A6.
	
	\bibitem{polchinski1998}
	Polchinski, J. (1998).
	\textit{String Theory}.
	Cambridge University Press.
	
	\bibitem{qant_nps}
	QANT (2024).
	\textit{Quantum photonics systems}.
	Technical Report.
	
	\bibitem{quantenjahr25}
	Quantenjahr (2025).
	\textit{International Year of Quantum}.
	UNESCO.
	
	\bibitem{recurrent_photonics}
	Tait, A. N. et al. (2024).
	\textit{Recurrent photonic neural networks}.
	Optica.
	
	\bibitem{rf_photonics}
	Capmany, J. \& Novak, D. (2024).
	\textit{Microwave photonics}.
	Nature Photonics.
	
	\bibitem{riess2019}
	Riess, A. G. et al. (2019).
	\textit{Large Magellanic Cloud Cepheid Standards}.
	ApJ 876, 85.
	
	\bibitem{riess2022}
	Riess, A. G. et al. (2022).
	\textit{A Comprehensive Measurement of H0}.
	ApJ 934, L7.
	
	\bibitem{rovelli2004}
	Rovelli, C. (2004).
	\textit{Quantum Gravity}.
	Cambridge University Press.
	
	\bibitem{sciama1953}
	Sciama, D. W. (1953).
	\textit{On the origin of inertia}.
	Mon. Not. R. Astron. Soc. 113, 34--42.
	
	\bibitem{sciencedaily2025}
	ScienceDaily (2025).
	\textit{Physics news}.
	Online.
	
	\bibitem{sm_g2_2025}
	Aoyama, T. et al. (2025).
	\textit{Standard Model prediction for g-2}.
	Phys. Rep.
	
	\bibitem{susskind1995}
	Susskind, L. (1995).
	\textit{The world as a hologram}.
	J. Math. Phys. 36, 6377--6396.
	
	\bibitem{t0_kosmologie}
	Pascher, J. (2024).
	\textit{T0-Theory: Cosmology}.
	Unpublished manuscript, HTL Leonding.
	
	\bibitem{terrell1959}
	Terrell, J. (1959).
	\textit{Invisibility of the Lorentz contraction}.
	Phys. Rev. 116, 1041--1045.
	
	\bibitem{terrell_single_clock_nature_2024}
	Terrell, J. et al. (2024).
	\textit{Single clock precision measurements}.
	Nature Physics.
	
	\bibitem{tfln_foundry}
	TFLN Foundry (2024).
	\textit{Thin-film lithium niobate foundry services}.
	Technical Specifications.
	
	\bibitem{thiemann2007}
	Thiemann, T. (2007).
	\textit{Modern Canonical Quantum General Relativity}.
	Cambridge University Press.
	
	\bibitem{thz_epfl}
	EPFL (2024).
	\textit{Terahertz photonics research}.
	Technical Report.
	
	\bibitem{unnikrishnan2004}
	Unnikrishnan, C. S. (2004).
	\textit{On Einstein's resolution of the twin clock paradox}.
	Current Science, 86, 704--709.
	
	\bibitem{verlinde2011}
	Verlinde, E. (2011).
	\textit{On the origin of gravity and the laws of Newton}.
	JHEP 2011, 29.
	
	\bibitem{video2025}
	Video (2025).
	\textit{Physics video explanation}.
	YouTube.
	
	\bibitem{weinberg1995}
	Weinberg, S. (1995).
	\textit{The Quantum Theory of Fields}.
	Cambridge University Press.
	
	\bibitem{weiskopf2000}
	Weiskopf, D. (2000).
	\textit{Visualization of special relativity}.
	PhD thesis, University of Tübingen.
	
	\bibitem{wheeler1990}
	Wheeler, J. A. (1990).
	\textit{A Journey into Gravity and Spacetime}.
	Scientific American Library.
	
	\bibitem{wiki_bell}
	Wikipedia (2024).
	\textit{Bell's theorem}.
	Online encyclopedia.
	
	\bibitem{zwicky1929}
	Zwicky, F. (1929).
	\textit{On the red shift of spectral lines through interstellar space}.
	Proc. Natl. Acad. Sci. 15, 773--779.

\end{thebibliography}


\end{document}
