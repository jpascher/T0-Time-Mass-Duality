\documentclass[11pt,a4paper]{article}
\usepackage[a4paper,margin=2cm]{geometry}
\usepackage[utf8]{inputenc}
\usepackage[english]{babel}
\usepackage{lmodern}
\renewcommand{\familydefault}{\sfdefault}

\usepackage{amsmath,amssymb,amsthm}
\usepackage{graphicx}
\usepackage[unicode,pdfencoding=auto,hypertexnames=false]{hyperref}
\usepackage{booktabs}
\usepackage{longtable}
\usepackage{array}
\usepackage{siunitx}
\usepackage{fancyhdr}
\usepackage{float}
\usepackage{tikz}
% tcolorbox removed for standalone
% tcbset removed
\tikzset{
  t0blue/.style={draw=blue,fill=blue!10},
  t0red/.style={draw=red,fill=red!10},
  t0green/.style={draw=green!50!black,fill=green!10},
  t0orange/.style={draw=orange,fill=orange!10},
}
\usepackage{setspace}
\usepackage{enumitem}
\usepackage{adjustbox}
\usepackage{xcolor}

% Define colors for xcolor package
\definecolor{t0green}{RGB}{34,139,34}
\definecolor{t0blue}{RGB}{0,0,255}
\definecolor{t0red}{RGB}{255,0,0}
\definecolor{t0orange}{RGB}{255,165,0}

% Define custom column types for tables
\newcolumntype{L}[1]{>{\raggedright\arraybackslash}p{#1}}
\newcolumntype{C}[1]{>{\centering\arraybackslash}p{#1}}
\newcolumntype{R}[1]{>{\raggedleft\arraybackslash}p{#1}}

\setlength{\parindent}{0pt}
\setlength{\parskip}{6pt}

\hypersetup{
  colorlinks=true,
  linkcolor=blue,
  citecolor=blue,
  urlcolor=blue
}
\pagestyle{fancy}
\setlength{\headheight}{28pt}

\newcommand{\checkmarkx}{\checkmark}
\newcommand{\warningx}{\textbf{!}}

% Makros aus Einzel-Dokumenten (Fallback-Definitionen)
\newcommand{\mytimes}{\times}
\newcommand{\myapprox}{\approx}
\newcommand{\mysim}{\sim}
\newcommand{\myomega}{\omega}
\newcommand{\mypi}{\pi}
\newcommand{\myrightarrow}{\rightarrow}
\newcommand{\mypropto}{\propto}
\newcommand{\deltafield}{\delta\phi}
\newcommand{\xipar}{\xi}
\newcommand{\xiT}{\xi}
\newcommand{\lambdah}{\lambda_h}

% Additional macros used in chapter files
\newcommand{\Kfrak}{K_{\text{frak}}}  % Fractal correction factor
\newcommand{\Dfrak}{D_f}              % Fractal dimension
\newcommand{\betapar}{\beta}          % T0 beta parameter
\newcommand{\alphapar}{\alpha}        % T0 alpha parameter
\newcommand{\Efield}{E}               % Energy field
% Note: checkmarkxa/warningxa are variants used in auto-generated chapter files
\newcommand{\checkmarkxa}{\checkmark}
\newcommand{\warningxa}{\textbf{!}}

% Additional T0-specific macros
\newcommand{\xigeom}{\xi_{\text{geom}}}  % Geometric xi
\newcommand{\lP}{\ell_P}                  % Planck length
\newcommand{\rzero}{r_0}                  % Characteristic radius
\newcommand{\xirat}{\xi_{\text{rat}}}     % Xi ratio
\newcommand{\tzero}{t_0}                  % Characteristic time
\newcommand{\natunits}{\text{(nat. units)}}  % Natural units annotation
\newcommand{\myRightarrow}{\Rightarrow}   % Arrow variant
\newcommand{\Lag}{\mathcal{L}}            % Lagrangian

% Physics macros used in chapter files
\newcommand{\CQCD}{C_{\text{QCD}}}        % QCD correction
\newcommand{\EP}{E_P}                     % Planck energy
\newcommand{\Ee}{E_e}                     % Electron energy
\newcommand{\Emu}{E_\mu}                  % Muon energy
\newcommand{\Exi}{E_\xi}                  % Xi energy
\newcommand{\Ezero}{E_0}                  % Characteristic energy
\newcommand{\Hubble}{H}                   % Hubble constant
\newcommand{\Kspec}{K_{\text{spec}}}      % Spectral correction
\newcommand{\Lambdat}{\Lambda_t}          % Time-related cosmological constant
\newcommand{\Leff}{\mathcal{L}_{\text{eff}}}  % Effective Lagrangian
\newcommand{\Lorentz}{\mathcal{L}}        % Lorentz symbol
\newcommand{\Lxi}{L_\xi}                  % Xi length
\newcommand{\Tfield}{T}                   % Time field
\newcommand{\Weyl}{W}                     % Weyl tensor/symbol
\newcommand{\alphaEMSI}{\alpha_{\text{EM,SI}}}  % EM alpha in SI
\newcommand{\alphaEMnat}{\alpha_{\text{EM,nat}}}  % EM alpha in natural units
\newcommand{\alphaem}{\alpha_{\text{em}}} % Electromagnetic alpha
\newcommand{\betaTSI}{\beta_{T,\text{SI}}}  % Beta in SI
\newcommand{\betaTnat}{\beta_{T,\text{nat}}}  % Beta in natural units
\newcommand{\deltam}{\delta m}            % Mass difference
\newcommand{\phiT}{\phi_T}                % T-field phi
\newcommand{\tP}{t_P}                     % Planck time
\newcommand{\rhoCMB}{\rho_{\text{CMB}}}   % CMB density
\newcommand{\rhoCasimir}{\rho_{\text{Casimir}}}  % Casimir density

% Table formatting
\usepackage{multirow}

% Additional physics macros
\newcommand{\Riem}{\mathcal{R}}           % Riemann tensor
\newcommand{\ZPinch}{Z_{\text{pinch}}}    % Z-pinch
\newcommand{\SynchPower}{P_{\text{synch}}} % Synchrotron power
\newcommand{\Rzero}{R_0}                  % Characteristic radius
\newcommand{\alphafine}{\alpha}           % Fine structure constant
\newcommand{\Etau}{E_\tau}                % Tau energy
\newcommand{\deltaE}{\delta E}            % Energy deviation
\newcommand{\EPlanck}{E_P}                % Planck energy
\newcommand{\pichar}{\pi}                 % Pi character
\newcommand{\alphaWSI}{\alpha_{W,\text{SI}}}  % Wien alpha in SI
\newcommand{\alphaWnat}{\alpha_{W,\text{nat}}}  % Wien alpha in natural units

% Einfache abstract-Umgebung für Kapitel:
\newenvironment{abstract}{%
  \begin{center}\bfseries Abstract\end{center}\small
}{\par}


\title{TempEinheitenCMBEn}
\author{J. Pascher}
\date{\today}

\begin{document}
\maketitle

\section*{Tempeinheitencmben (TempEinheitenCMBEn)}

	\begin{abstract}
		This work presents a comprehensive analysis of temperature units in natural units ($\hbar = c = k_B = 1$) within the T0-theory framework. The static $\xi$-universe eliminates the need for expanding spacetime. All derivations are based exclusively on the universal constant $\xi = \frac{4}{3} \times 10^{-4}$ and respect the fundamental time-energy duality. The document includes complete CMB calculations within the T0-theory framework, addressing fundamental questions about redshift mechanisms, primordial perturbations, and the resolution of cosmological tensions. The theory successfully explains the CMB at $z \approx 1100$ without inflation, derives primordial perturbations from T-field quantum fluctuations, and resolves the Hubble tension with $H_0 = 67.45 \pm 1.1$ km/s/Mpc.
	\end{abstract}
	
	
	\section{Introduction: T0-Theory in Natural Units}
	
	\subsection{Natural Units as Foundation}
	
\section*{Important}
		This entire work uses exclusively natural units with $\hbar = c = k_B = 1$. All quantities have energy dimensions: $[L] = [T] = [E^{-1}]$, $[M] = [T_{\text{temp}}] = [E]$.
% end box important
	
	The natural units system represents a fundamental simplification of physics by setting the universal constants $\hbar$ (reduced Planck constant), $c$ (speed of light) and $k_B$ (Boltzmann constant) to the value 1. This choice is not arbitrary, but reflects the deep unity of natural laws.
	
	In this system, all physics reduces to a single fundamental dimension - energy. All other physical quantities are expressed as powers of energy:
	\begin{align}
		\text{Length:} \quad [L] &= [E^{-1}] \quad \text{(Energy}^{-1}\text{)} \\
		\text{Time:} \quad [T] &= [E^{-1}] \quad \text{(Energy}^{-1}\text{)} \\
		\text{Mass:} \quad [M] &= [E] \quad \text{(Energy)} \\
		\text{Temperature:} \quad [T_{\text{temp}}] &= [E] \quad \text{(Energy)}
	\end{align}
	
	This dimensional reduction reveals hidden symmetries and makes complex relationships transparent. In natural units, for example, Einstein's famous formula $E = mc^2$ becomes the trivial statement $E = m$, since both energy and mass have the same dimension.
	
\section*{Unit conversion (for reference):}
	For readers familiar with SI units, the following conversion factors apply:
	\begin{itemize}
		\item $\hbar = 1{,}055 \times 10^{-34}$ J$\cdot$s $\rightarrow 1$ (nat. units)
		\item $c = 2{,}998 \times 10^8$ m/s $\rightarrow 1$ (nat. units)  
		\item $k_B = 1{,}381 \times 10^{-23}$ J/K $\rightarrow 1$ (nat. units)
	\end{itemize}
	
	\subsection{The Universal $\xi$-Constant}
	
\section*{Revolutionary}
		The T0-theory revolutionizes our understanding of the universe: A single geometric constant $\xi = \frac{4}{3} \times 10^{-4}$ determines everything -- from quarks to cosmic structures -- in a static, eternally existing cosmos without Big Bang. The factor $\frac{4}{3}$ originates from the fundamental geometric ratio between sphere volume and tetrahedron volume in three-dimensional space.
% end box revolutionary
	
	The heart of T0-theory is formed by a universal dimensionless constant, which we denote with the Greek letter $\xi$ (Xi). This constant was originally derived purely geometrically from the fundamental T0-field equations, as shown in the established T0-theory \cite{T0Theory}.
	
	The fundamental T0-theory is based on the universal dimensionless constant:
	\begin{equation}
		\xi = \frac{4}{3} \times 10^{-4} \quad \text{(dimensionless, exact geometric value)}
	\end{equation}
	
	\textbf{Geometric derivation from T0-field equations:} The value of $\xi$ follows directly from the geometric structure of the T0-field equations of the universal energy field $E_{\text{field}}(x,t)$. The fundamental T0-equation $\square E_{\text{field}} = 0$ in connection with three-dimensional space geometry leads inevitably to:
	\begin{itemize}
		\item The geometric factor $\frac{4}{3}$ from the ratio of sphere volume ($V_{\text{sphere}} = \frac{4\pi}{3}r^3$) to tetrahedron volume
		\item The energy scale ratio $10^{-4}$ which connects quantum and gravitational domains
		\item Together: $\xi = \frac{4}{3} \times 10^{-4}$ as the unique solution.see \texttt{parameterherleitung\_En.pdf} available at:
		\url{https://github.com/jpascher/T0-Time-Mass-Duality/tree/main/2/pdf}
	\end{itemize}
	
	\textbf{Experimental confirmation:} After the theoretical derivation of $\xi$ from T0-field equations, it was discovered that this constant agrees exactly with high-precision experiments for measuring the anomalous magnetic moment of the muon (g-2 experiments). This represents an independent experimental verification of the geometric T0-theory.
	
	This constant determines in T0-theory a surprising variety of physical phenomena:
	\begin{itemize}
		\item \textbf{Particle physics}: All elementary particle masses result from geometric quantum numbers $(n,l,j,r,p)$ scaled with $\xi$
		\item \textbf{Field theory}: Characteristic energy scales of all interactions follow from $\xi$-field dynamics
		\item \textbf{Gravitation}: The gravitational constant in natural units $G_{\text{nat}} = 2{,}61 \times 10^{-70}$ is a direct function of $\xi$
		\item \textbf{Cosmology}: Thermodynamic equilibrium in the static, infinitely old universe is maintained through $\xi$-field cycles
	\end{itemize}
	
\section*{Symbol explanation:}
	\begin{itemize}
		\item $\xi$ (Xi): Universal dimensionless constant of T0-theory
		\item $E_\xi$: Characteristic energy scale, defined as $E_\xi = 1/\xi$
		\item $T_\xi$: Characteristic temperature, equal to $E_\xi$ in natural units
		\item $L_\xi$: Characteristic length scale of the $\xi$-field
		\item $G_{\text{nat}}$: Gravitational constant in natural units
		\item $\alpha_{\text{EM}}$: Electromagnetic coupling (= 1 in natural units by definition)
		\item $\beta$: Dimensionless parameter $\beta = r_0/r = 2GE/r$
		\item $\omega$: Photon energy (dimension $[E]$ in natural units)
	\end{itemize}
	
\section*{Coupling constants in natural units:}
	\begin{align}
		\alpha_{\text{EM}} &= 1 \quad \text{(by definition in natural units)} \\
		\alpha_G &= \xi^2 = \left(\frac{4}{3} \times 10^{-4}\right)^2 = 1{,}78 \times 10^{-8} \\
		\alpha_W &= \xi^{1/2} = \left(\frac{4}{3} \times 10^{-4}\right)^{1/2} = 1{,}15 \times 10^{-2} \\
		\alpha_S &= \xi^{-1/3} = \left(\frac{4}{3} \times 10^{-4}\right)^{-1/3} = 9{,}65
	\end{align}
	
\section*{Important clarification on units:}
	In this entire document we work exclusively in natural units with $\hbar = c = k_B = 1$. This means:
	\begin{itemize}
		\item The electromagnetic coupling constant is $\alpha_{\text{EM}} = 1$ by definition (not 1/137 as in SI units)
		\item All other coupling constants are expressed relative to $\alpha_{\text{EM}} = 1$
		\item Energy, mass and temperature have the same dimension
		\item Length and time have the dimension energy$^{-1}$
	\end{itemize}
	
	\textbf{Dimensional consistency:} Since $\xi$ is purely dimensionless, it has the same value in all unit systems. It characterizes the fundamental geometry of space-time continuum and is a true natural constant, comparable to the fine structure constant.
	
	\subsection{Time-Energy Duality and Static Universe}
	
\section*{Important}
		Heisenberg's uncertainty relation $\Delta E \times \Delta t \geq \hbar/2 = 1/2$ (nat. units) provides irrefutable proof that a Big Bang is physically impossible and the universe exists eternally.
% end box important
	
	Heisenberg's uncertainty relation between energy and time represents one of the most fundamental statements of quantum mechanics. In natural units, where $\hbar = 1$, it reads:
	\begin{equation}
		\Delta E \times \Delta t \geq \frac{1}{2}
	\end{equation}
	
	where $\Delta E$ represents the uncertainty (indeterminacy) in energy and $\Delta t$ the uncertainty in time.
	
	This relation has far-reaching cosmological consequences that are usually ignored in standard cosmology. If the universe had a temporal beginning (Big Bang), then $\Delta t$ would be finite, which according to the uncertainty relation would result in an infinite energy uncertainty $\Delta E \to \infty$. Such a state is physically inconsistent.
	
	\textbf{Logical consequence:} The universe must have existed eternally to satisfy the uncertainty relation. This leads us to the static T0-universe, which has the following properties:
	
	The T0-universe is therefore:
	\begin{itemize}
		\item \textbf{Static}: No expanding space - the spacetime metric is time-independent
		\item \textbf{Eternal}: Without temporal beginning or end - $\Delta t = \infty$
		\item \textbf{Thermodynamically balanced}: Through $\xi$-field cycles a dynamic equilibrium is maintained
		\item \textbf{Structurally stable}: Continuous formation and renewal of matter and structures
	\end{itemize}
	
\section*{Unit check of the uncertainty relation:}
	\begin{align}
		[\Delta E] \times [\Delta t] &= [E] \times [E^{-1}] = [E^0] = \text{dimensionless} \\
		\left[\frac{1}{2}\right] &= \text{dimensionless} \quad \checkmark
	\end{align}
	
	\section{$\xi$-Field and Characteristic Energy Scales}
	
	\subsection{$\xi$-Field as Universal Energy Mediator}
	
\section*{Formula}
		The universal constant $\xi = \frac{4}{3} \times 10^{-4}$ defines the fundamental energy scale of T0-theory:
		\begin{equation}
			E_\xi = \frac{1}{\xi} = \frac{1}{\frac{4}{3} \times 10^{-4}} = \frac{3}{4} \times 10^4 = 7500
		\end{equation}
		(all quantities in natural units)
% end box formula
	
	The $\xi$-field represents the fundamental energy field of the universe, from which all other fields and interactions emerge. Its characteristic energy scale $E_\xi$ results as the reciprocal of the dimensionless constant $\xi$.
	
\section*{Unit check for $E_\xi$:}
	\begin{align}
		[E_\xi] &= \left[\frac{1}{\xi}\right] = \frac{[E^0]}{[E^0]} = [E^0] = \text{dimensionless}
	\end{align}
	
	In natural units, dimensionless is equivalent to an energy unit, since all quantities are reduced to energy powers. Therefore $[E_\xi] = [E]$ holds.
	
	This characteristic energy corresponds directly to a characteristic temperature in natural units, since energy and temperature have the same dimension:
	\begin{equation}
		T_\xi = E_\xi = \frac{3}{4} \times 10^4 = 7500 \quad \text{(nat. units)}
	\end{equation}
	
\section*{Unit check for $T_\xi$:}
	\begin{align}
		[T_\xi] = [E_\xi] = [E] = [T_{\text{temp}}] \quad \checkmark
	\end{align}
	
	\textbf{Physical interpretation:} The energy scale $E_\xi = 7500$ in natural units corresponds to an extremely high temperature that is characteristic for the fundamental processes of the $\xi$-field. This energy lies far above all known particle energies and indicates the fundamental nature of the $\xi$-field.
	
	\subsection{Characteristic $\xi$-Length Scale}
	
	The $\xi$-field also defines a characteristic length scale:
	\begin{equation}
		L_\xi = \frac{1}{E_\xi} = \frac{1}{7500} \approx 1.33 \times 10^{-4} \quad \text{(nat. units)}
	\end{equation}
	
	This length scale plays a fundamental role in the geometric structure of space-time and appears in various physical phenomena.
	
	\section{CMB in T0-Theory: Static $\xi$-Universe}
	
	\subsection{CMB Without Big Bang}
	
\section*{Revolutionary}
		Time-energy duality forbids a Big Bang, therefore the CMB background radiation must have a different origin than z=1100 decoupling!
% end box revolutionary
	
	T0-theory explains the cosmic microwave background radiation through $\xi$-field mechanisms:
	
	\subsubsection{1. -Field Quantum Fluctuations}
	The omnipresent $\xi$-field generates vacuum fluctuations with characteristic energy scale. The exact dependence is derived through the measured ratio $T_{\text{CMB}}/E_\xi \approx \xi^2$.
	
	\subsubsection{2. Steady-State Thermalization}
	In an infinitely old universe, background radiation reaches thermodynamic equilibrium at the characteristic $\xi$-temperature.
	
\section*{SI-Box}
\section*{CMB measurements (for reference only, in SI units):}
		\begin{itemize}
			\item Vacuum energy density: $\rho_{\text{vacuum}} = 4.17 \times 10^{-14}$ J/m$^3$
			\item Radiation power: $j = 3.13 \times 10^{-6}$ W/m$^2$
			\item Temperature: $T = 2.7255$ K
		\end{itemize}
% end box sibox
	
	\subsection{The Already Established $\xi$-Geometry}
	
\section*{Important}
		T0-theory had already established a fundamental length scale before the CMB analysis. The CMB energy density now confirms this pre-existing $\xi$-geometric structure.
% end box important
	
	From the original T0-theory formulation followed:
	
\section*{Characteristic mass:}
	\begin{equation}
		m_{\text{char}} = \frac{\xi}{2\sqrt{G_{\text{nat}}}} \approx 4.13 \times 10^{30} \quad \text{(nat. units)}
	\end{equation}
	
\section*{Universal scaling rule:}
	\begin{equation}
		\text{Factor} = 2.42 \times 10^{-31} \cdot m \quad \text{(for arbitrary mass } m \text{ in nat. units)}
	\end{equation}
	
\section*{Gravitational constant derived from $\xi$:}
	\begin{equation}
		G_{\text{nat}} = 2.61 \times 10^{-70} \quad \text{(nat. units)}
	\end{equation}
	\label{TempEinheitenCM:L-TempEinheitenCMBEn-0718}
	
	The T0-theory represents a fundamental extension of standard cosmology through the introduction of an intrinsic time field $\Tfield$ that couples to all matter and radiation. This theory emerged from dissatisfaction with quantum mechanical non-locality and the need for a deterministic framework that preserves causality while explaining observed correlations.
	
	\subsection{Fundamental Postulates}
	
	The T0-theory is built on three fundamental postulates:
	
	\begin{enumerate}
		\item \textbf{Time-Mass Duality}: The fundamental relationship
		\begin{equation}
			\Tfield \cdot m(x) = 1
			\label{TempEinheitenCM:L-T0_Grundlagen-0001}
		\end{equation}
		
		\item \textbf{Universal Coupling Parameter}: A single parameter
		\begin{equation}
			\xipar = \frac{\lambda_h^2 v^2}{16\pi^3 m_h^2} = \frac{4}{3} \times 10^{-4}
			\label{TempEinheitenCM:L-TempEinheitenCMBEn-0719}
		\end{equation}
		derived from Higgs physics governs all T-field interactions. The factor $\frac{4}{3}$ ultimately originates from the fundamental geometric ratio between sphere volume and tetrahedron volume in three-dimensional space.
		
		\item \textbf{Modified Robertson-Walker Metric}:
		\begin{equation}
			ds^2 = -c^2dt^2[1 + 2\xipar\ln(a)] + a^2(t)[1 - 2\xipar\ln(a)]d\vec{x}^2
			\label{TempEinheitenCM:L-TempEinheitenCMBEn-0720}
		\end{equation}
	\end{enumerate}
	
	\section{Power Spectra Calculations}
	\label{TempEinheitenCM:L-TempEinheitenCMBEn-0721}
	
	\subsection{Temperature Power Spectrum}
	
	The CMB temperature power spectrum is:
	
	\begin{equation}
		C_\ell^{TT} = \frac{2}{\pi}\int_0^\infty k^2 dk \, \mathcal{P}_\Psi(k) |\Theta_\ell(k,\eta_0)|^2 \times \left(1 + \xipar f_\ell(k)\right)
		\label{TempEinheitenCM:L-TempEinheitenCMBEn-0722}
	\end{equation}
	
	where:
	\begin{equation}
		f_\ell(k) = \ln^2\left(\frac{k}{k_*}\right) - 2\ln\left(\frac{k}{k_*}\right)
	\end{equation}
	
	\subsection{E-mode Polarization}
	
	\begin{equation}
		C_\ell^{EE} = \frac{2}{\pi}\int_0^\infty k^2 dk \, \mathcal{P}_\Psi(k) |E_\ell(k,\eta_0)|^2 \times \left(1 + \xipar g_\ell(k)\right)
	\end{equation}
	
	\subsection{Cross-correlation}
	
	\begin{equation}
		C_\ell^{TE} = \frac{2}{\pi}\int_0^\infty k^2 dk \, \mathcal{P}_\Psi(k) \Theta_\ell(k,\eta_0) E_\ell^*(k,\eta_0) \times \left(1 + \xipar h_\ell(k)\right)
	\end{equation}
	
	\section{MCMC Analysis and Parameter Constraints}
	\label{TempEinheitenCM:L-TempEinheitenCMBEn-0723}
	
	\subsection{Bayesian Parameter Estimation}
	
	We perform a full MCMC analysis using:
	
	\begin{equation}
		\mathcal{L} = -\frac{1}{2}\sum_{\ell} \frac{2\ell+1}{2} f_{\text{sky}} \left[\frac{C_\ell^{\text{obs}} - C_\ell^{\text{theory}}(\theta)}{\sigma_\ell}\right]^2
	\end{equation}
	
	\subsection{Results with Uncertainties}
	
	\begin{table}[htbp]
		\centering
		\caption{T0 Parameter Constraints (68\% CL)}
		\begin{tabular}{lcc}
			\toprule
			Parameter & Best Fit & Uncertainty \\
			\midrule
			$H_0$ [km/s/Mpc] & 67.45 & $\pm 1.1$ \\
			$\Omega_b h^2$ & 0.02237 & $\pm 0.00015$ \\
			$\Omega_c h^2$ & 0.1200 & $\pm 0.0012$ \\
			$\tau$ & 0.054 & $\pm 0.007$ \\
			$n_s$ & 0.9649 & $\pm 0.0042$ \\
			$\ln(10^{10}A_s)$ & 3.044 & $\pm 0.014$ \\
			$\xipar$ & $\frac{4}{3} \times 10^{-4}$ & (geometric constant) \\
			\bottomrule
		\end{tabular}
		\label{TempEinheitenCM:L-TempEinheitenCMBEn-0724}
	\end{table}
	
	\section{Resolution of Cosmological Tensions}
	\label{TempEinheitenCM:L-TempEinheitenCMBEn-0725}
	
	\subsection{Hubble Tension}
	
	The T0-theory naturally resolves the Hubble tension:
	
\section*{Theorem}
		The T0-predicted Hubble constant:
		\begin{equation}
			H_0^{T0} = H_0^{\Lambda\text{CDM}} \times (1 + 6\xipar) = 67.4 \times (1 + 6 \times \frac{4}{3} \times 10^{-4}) = 67.4 \times 1.0008 = 67.45 \text{ km/s/Mpc}
		\end{equation}
		matches local measurements while maintaining consistency with CMB data.
% end box theorem
	
	\begin{proof}
		The T-field modifies the distance-redshift relation:
		\begin{equation}
			d_L(z) = d_L^{\Lambda\text{CDM}}(z) \times \left[1 - \xipar \ln(1+z)\right]
		\end{equation}
		
		For low redshifts ($z \ll 1$):
		\begin{equation}
			d_L \approx \frac{cz}{H_0}\left[1 + \frac{1-q_0}{2}z - \xipar z\right]
		\end{equation}
		
		This effectively increases the inferred $H_0$ by factor $(1 + 6\xipar)$.
	\end{proof}
	
	\subsection{Tension}
	
	The clustering amplitude is modified:
	
	\begin{equation}
		S_8^{T0} = S_8^{\Lambda\text{CDM}} \times (1 - 2\xipar) = 0.834 \times (1 - 2 \times \frac{4}{3} \times 10^{-4}) = 0.834 \times 0.99973 = 0.8338
	\end{equation}
	
	This matches weak lensing measurements.
	
	\section{Experimental Predictions}
	\label{TempEinheitenCM:L-TempEinheitenCMBEn-0726}
	
	\subsection{Testable Predictions}
	
	The T0-theory makes several unique predictions:
	
	\begin{enumerate}
		\item \textbf{Running of spectral index}:
		\begin{equation}
			\frac{dn_s}{d\ln k} = -2\xipar = -2 \times \frac{4}{3} \times 10^{-4} = -2.67 \times 10^{-4}
		\end{equation}
		
		\item \textbf{Tensor-to-scalar ratio}:
		\begin{equation}
			r = 16\xipar = 16 \times \frac{4}{3} \times 10^{-4} = 0.00213 \pm 0.0004
		\end{equation}
		
		\item \textbf{Modified Silk damping}:
		\begin{equation}
			C_\ell^{TT} \propto \exp\left[-\left(\frac{\ell}{\ell_D}\right)^2\right] \times \left(1 + \xipar \left(\frac{\ell}{3000}\right)^2\right)
		\end{equation}
		
		\item \textbf{Wavelength-dependent redshift}:
		\begin{equation}
			\Delta z = \beta \ln\left(\frac{\lambda}{\lambda_0}\right) \approx 0.008 \ln\left(\frac{\lambda}{\lambda_0}\right)
		\end{equation}
	\end{enumerate}
	
	\subsection{Observational Tests}
	
	\begin{table}[htbp]
		\centering
		\caption{T0 Predictions vs Observations}
		\begin{tabular}{lccc}
			\toprule
			Observable & T0 Prediction & Current Limit & Future Sensitivity \\
			\midrule
			$dn_s/d\ln k$ & $-2.67 \times 10^{-4}$ & $< 0.01$ & $10^{-4}$ (CMB-S4) \\
			$r$ & $0.00213$ & $< 0.036$ & $0.001$ (LiteBIRD) \\
			$f_{NL}$ & $-3.5 \times 10^{-4}$ & $< 5$ & $0.1$ (CMB-S4) \\
			$\Delta z(\lambda)$ & $0.008\ln(\lambda/\lambda_0)$ & -- & $10^{-3}$ (SKA) \\
			\bottomrule
		\end{tabular}
	\end{table}
	
	\section{Comparison with CDM}
	\label{TempEinheitenCM:L-TempEinheitenCMBEn-0727}
	
	\subsection{Analysis}
	
	Comparing model fits to Planck 2018 data:
	
	\begin{align}
		\chi^2_{\Lambda\text{CDM}} &= 1127.4 \\
		\chi^2_{T0} &= 1123.8 \\
		\Delta\chi^2 &= -3.6 \quad (2.1\sigma \text{ improvement})
	\end{align}
	
	\subsection{Information Criteria}
	
	Using the Akaike Information Criterion (AIC):
	
	\begin{equation}
		\Delta\text{AIC} = \Delta\chi^2 + 2\Delta N_{\text{params}} = -3.6 + 2 = -1.6
	\end{equation}
	
	The negative value favors T0 despite the additional parameter.
	
	\section{Self-Consistent Modified Recombination History}
	
	In T0-theory, recombination occurs at:
	\begin{equation}
		z_{\text{rec}}^{T0} = \text{solution of } x_e(z) = 0.5
	\end{equation}
	
	The electron fraction evolves as:
	\begin{equation}
		x_e(z) = \frac{1}{1 + A(T) \exp[E_I/kT(z)]}
	\end{equation}
	
	where:
	\begin{align}
		T(z) &= T_0(1+z)[1 - \xi\ln(1+z)] \\
		A(T) &= \left(\frac{2\pi m_e kT}{h^2}\right)^{-3/2} 
		\frac{g_p g_e}{g_H} (1 + \xi h(T))
	\end{align}
	
	This yields $z_{\text{rec}}^{T0} \approx 1089.5$, differing from 
	$z_{\text{rec}}^{\Lambda\text{CDM}} = 1089.9$ by a measurable amount.
	
	% ================== END OF CMB SECTION ==================
	
	\section{CMB-Casimir Connection and $\xi$-Field Verification}
	\label{TempEinheitenCM:L-TempEinheitenCMBEn-0728}
	
	\subsection{CMB Energy Density and -Length Scale}
	
\section*{Revolutionary}
		The measured CMB spectrum corresponds to the radiating energy density of the $\xi$-field vacuum. The vacuum itself radiates at its characteristic temperature.
% end box revolutionary
	
	The CMB energy density in natural units:
	\begin{equation}
		\rho_{\text{CMB}} = 4.87 \times 10^{41} \quad \text{(nat. units, dimension } [E^4] \text{)}
	\end{equation}
	
	The CMB temperature in natural units:
	\begin{equation}
		T_{\text{CMB}} = 2.35 \times 10^{-4} \quad \text{(nat. units)}
	\end{equation}
	
	This energy density defines a characteristic $\xi$-length scale:
	\begin{equation}
		L_\xi = \left(\frac{\xi}{\rho_{\text{CMB}}}\right)^{1/4}
	\end{equation}
	
\section*{Formula}
		Fundamental relation of CMB energy density:
		\begin{equation}
			\rho_{\text{CMB}} = \frac{\xi}{L_\xi^4} = \frac{\frac{4}{3} \times 10^{-4}}{L_\xi^4}
		\end{equation}
% end box formula
	
	\subsection{Casimir-CMB Ratio as Experimental Confirmation}
	
	The Casimir effect represents a direct manifestation of quantum vacuum fluctuations. In natural units, the Casimir energy density between two parallel plates separated by distance $d$ is:
	
	\begin{equation}
		|\rho_{\text{Casimir}}| = \frac{\pi^2}{240 d^4} \quad \text{(nat. units)}
	\end{equation}
	
	At the characteristic $\xi$-length scale $L_\xi = 10^{-4}$ m, the ratio between Casimir and CMB energy densities provides crucial verification:
	
	\begin{equation}
		\frac{|\rho_{\text{Casimir}}|}{\rho_{\text{CMB}}} = \frac{\pi^2}{240 \xi} = \frac{\pi^2}{240 \times \frac{4}{3} \times 10^{-4}} = \frac{\pi^2 \times 10^4}{320} \approx 308
	\end{equation}
	
	\subsection{Detailed Calculations in SI Units}
	
	\textbf{Casimir energy density at plate separation} $d = L_\xi = 10^{-4}$ m:
	
	\begin{align}
		|\rho_{\text{Casimir}}| &= \frac{\hbar c \pi^2}{240 d^4} \\
		&= \frac{1.055 \times 10^{-34} \times 2.998 \times 10^8 \times \pi^2}{240 \times (10^{-4})^4} \\
		&= \frac{3.12 \times 10^{-25}}{2.4 \times 10^{-14}} \\
		&= 1.3 \times 10^{-11} \text{ J/m}^3
	\end{align}
	
\section*{CMB energy density in SI units:}
	\begin{equation}
		\rho_{\text{CMB}} = 4.17 \times 10^{-14} \text{ J/m}^3
	\end{equation}
	
\section*{Experimental ratio:}
	\begin{equation}
		\frac{|\rho_{\text{Casimir}}|}{\rho_{\text{CMB}}} = \frac{1.3 \times 10^{-11}}{4.17 \times 10^{-14}} = 312
	\end{equation}
	
\section*{Theoretical prediction in natural units:}
	\begin{align}
		\frac{|\rho_{\text{Casimir}}|}{\rho_{\text{CMB}}} &= \frac{\pi^2 / (240 L_\xi^4)}{\xi / L_\xi^4} \\
		&= \frac{\pi^2}{240 \xi} = \frac{\pi^2}{240 \times \frac{4}{3} \times 10^{-4}} \\
		&= \frac{\pi^2 \times 3 \times 10^4}{240 \times 4} = \frac{\pi^2 \times 10^4}{320} \approx 308
	\end{align}
	
	\textbf{Agreement:} The measured ratio 312 agrees with the theoretical T0-prediction 308 to 1.3\% and confirms the characteristic length scale $L_\xi = 10^{-4}$ m.
	\begin{align}
		|\rho_{\text{Casimir}}| &= \frac{\hbar c \pi^2}{240 \times (10^{-4})^4} = 1.3 \times 10^{-11} \text{ J/m}^3 \\
		\rho_{\text{CMB}} &= 4.17 \times 10^{-14} \text{ J/m}^3 \\
		\text{Ratio} &= \frac{1.3 \times 10^{-11}}{4.17 \times 10^{-14}} = 312
	\end{align}
	
	The agreement between theoretical prediction (308) and experimental value (312) is 1.3\% - excellent confirmation!
	
\section*{Important}
		The characteristic $\xi$-length scale $L_\xi = 10^{-4}$ m is the point where CMB vacuum energy density and Casimir energy density reach comparable magnitudes. This proves the fundamental reality of the $\xi$-field.
% end box important
	
	\subsection{Dimensionless -Hierarchy and Independent Verification}
	
\section*{Critical question: Is this circular argumentation?}
	
	No circular argumentation exists because:
	
	\begin{enumerate}
		\item \textbf{Different theoretical and experimental sources:}
		\begin{itemize}
			\item $\xi$-constant: Purely geometrically derived from T0-field equations
			\item Muon g-2: High-precision particle accelerator experiments
			\item CMB data: Cosmic microwave measurements
			\item Casimir measurements: Laboratory vacuum experiments
		\end{itemize}
		
		\item \textbf{Temporal sequence of development:}
		\begin{itemize}
			\item T0-theory and $\xi$-derivation: Purely theoretical geometric derivation
			\item Muon g-2 comparison: Subsequent discovery of agreement
			\item CMB prediction: Followed from the already established $\xi$-geometry
			\item Casimir verification: Independent laboratory confirmation
		\end{itemize}
		
		\item \textbf{Multiple independent verification paths:}
		\begin{itemize}
			\item Geometric derivation → $\xi = \frac{4}{3} \times 10^{-4}$
			\item Higgs mechanism → $\xi = \frac{\lambda_h^2 v^2}{16\pi^3 m_h^2} = \frac{4}{3} \times 10^{-4}$
			\item Lepton masses → $\xi = \frac{4}{3} \times 10^{-4}$
			\item CMB/Casimir ratio → confirms $\xi = \frac{4}{3} \times 10^{-4}$
		\end{itemize}
	\end{enumerate}
	
	\subsubsection{Detailed Energy Scale Ratios}
	
	The dimensionless ratio between CMB temperature and characteristic energy - detailed calculation:
	
	\begin{align}
		\frac{T_{\text{CMB}}}{E_\xi} &= \frac{2.35 \times 10^{-4}}{\frac{3}{4} \times 10^4} \\
		&= \frac{2.35 \times 10^{-4} \times 4}{3 \times 10^4} \\
		&= \frac{9.4}{3 \times 10^8} \\
		&= \frac{9.4}{3} \times 10^{-8} \\
		&= 3.13 \times 10^{-8}
	\end{align}
	
	Theoretical prediction from $\xi$-geometry - detailed steps:
	\begin{align}
		\xi^2 &= \left(\frac{4}{3} \times 10^{-4}\right)^2 \\
		&= \frac{16}{9} \times 10^{-8} \\
		&= 1.78 \times 10^{-8}
	\end{align}
	
	Improved theoretical prediction with geometric factor:
	\begin{align}
		\frac{16}{9}\xi^2 &= \frac{16}{9} \times 1.78 \times 10^{-8} \\
		&= 1.778 \times 1.78 \times 10^{-8} \\
		&= 3.16 \times 10^{-8}
	\end{align}
	
\section*{Comparison:}
	\begin{align}
		\text{Measured:} \quad &3.13 \times 10^{-8} \\
		\text{Theoretical:} \quad &3.16 \times 10^{-8} \\
		\text{Agreement:} \quad &\frac{3.13}{3.16} = 0.99 = 99\% \text{ (1\% deviation)}
	\end{align}
	
	Agreement to 1\%! This confirms:
	\begin{equation}
		\boxed{\frac{T_{\text{CMB}}}{E_\xi} = \frac{16}{9}\xi^2}
	\end{equation}
	
	\subsubsection{Length Scale Ratios}
	
	\begin{equation}
		\frac{\ell_{\xi}}{L_\xi} = \xi^{-1/4} = \left(\frac{3}{4}\right)^{1/4} \times 10
	\end{equation}
	
	\subsection{Consistency Verification of T0-Theory}
	
\section*{Revolutionary}
		T0-theory passes a successful self-consistency test: The $\xi$-constant derived from particle physics exactly predicts the vacuum energy density measured from CMB.
% end box revolutionary
	
	Two independent paths to the same length scale:
	
	\begin{table}[htbp]
		\centering
		\caption{Consistency Verification of $\xi$-Length Scale}
		\begin{tabular}{lcc}
			\toprule
			\textbf{Derivation} & \textbf{Starting Point} & \textbf{Result} \\
			\midrule
			$\xi$-geometry (bottom-up) & $\xi = \frac{4}{3} \times 10^{-4}$ from particles & $L_\xi \sim 10^{-4}$ m \\
			CMB vacuum (top-down) & $\rho_{\text{CMB}}$ from measurement & $L_\xi = \left(\frac{\xi}{\rho_{\text{CMB}}}\right)^{1/4}$ \\
			Casimir effect & Laboratory measurements & Confirms $L_\xi = 10^{-4}$ m \\
			\midrule
			\textbf{Agreement} & \textbf{All paths converge} & $\checkmark$ \\
			\bottomrule
		\end{tabular}
	\end{table}
	
	\subsection{The $\xi$-Field as Universal Vacuum}
	
\section*{Formula}
		The $\xi$-field vacuum manifests in multiple phenomena:
		\begin{align}
			\text{Free vacuum (CMB):} \quad &\rho_{\text{CMB}} = \frac{\xi}{L_\xi^4} \\
			\text{Constrained vacuum (Casimir):} \quad &|\rho_{\text{Casimir}}| = \frac{\pi^2}{240 d^4} \\
			\text{Ratio at } d = L_\xi: \quad &\frac{|\rho_{\text{Casimir}}|}{\rho_{\text{CMB}}} = \frac{\pi^2 \times 10^4}{320}
		\end{align}
% end box formula
	
\section*{Important}
		All $\xi$-relationships consist of exact mathematical ratios:
		\begin{itemize}
			\item Fractions: $\frac{4}{3}$, $\frac{16}{9}$, $\frac{3}{4}$
			\item Powers of ten: $10^{-4}$, $10^4$
			\item Mathematical constants: $\pi^2$
		\end{itemize}
		NO arbitrary decimal numbers! Everything follows from $\xi$-geometry.
% end box important
	
	\section{Casimir Effect and $\xi$-Field Connection}
	
	\subsection{Modified Casimir Formula in T0-Theory}
	
	The T0-theory provides a deeper understanding of the Casimir effect through the $\xi$-field:
	
	\begin{equation}
		|\rho_{\text{Casimir}}(d)| = \frac{\pi^2}{240 \xi} \rho_{\text{CMB}} \left(\frac{L_\xi}{d}\right)^4
	\end{equation}
	
	Substituting $\rho_{\text{CMB}} = \xi/L_\xi^4$ recovers the standard formula:
	\begin{equation}
		|\rho_{\text{Casimir}}| = \frac{\pi^2}{240 d^4}
	\end{equation}
	
	This demonstrates that the Casimir effect and CMB are different manifestations of the same $\xi$-field vacuum.
	
	\section{Unit Analysis of the -Based Casimir Formula}
	
	This analysis examines the unit consistency of the modified Casimir formula within the T0-theory, which introduces the dimensionless constant $\xi$ and the cosmic microwave background (CMB) energy density $\rho_{\text{CMB}}$. The aim is to verify consistency with the standard Casimir formula and clarify the physical significance of the new parameters $\xi$ and $L_\xi$. The analysis is conducted in SI units, with each formula checked for dimensional correctness.
	
	\subsection{Standard Casimir Formula}
	The standard Casimir formula describes the energy density of the Casimir effect between two parallel, perfectly conducting plates in a vacuum:
	\begin{equation}
		|\rho_{\text{Casimir}}| = \frac{\pi^2 \hbar c}{240 d^4}
	\end{equation}
	Here, $\hbar$ is the reduced Planck constant, $c$ is the speed of light, and $d$ is the distance between the plates. The unit check yields:
	\begin{equation}
		\frac{[\hbar] \cdot [c]}{[d^4]} = \frac{(\text{J} \cdot \text{s}) \cdot (\text{m}/\text{s})}{\text{m}^4} = \frac{\text{J} \cdot \text{m}}{\text{m}^4} = \frac{\text{J}}{\text{m}^3}
	\end{equation}
	This matches the unit of energy density, confirming the formula's correctness.
	
	\textbf{Formula Explanation:} The Casimir effect arises from quantum fluctuations of the electromagnetic field in a vacuum. Only specific wavelengths fit between the plates, resulting in a measurable energy density that scales with $d^{-4}$. The constant $\pi^2/240$ results from summing over all allowed modes.
	
	\subsection{Definition of and CMB Energy Density}
	The T0-theory introduces the dimensionless constant $\xi$, defined as:
	\begin{equation}
		\xi = \frac{4}{3} \times 10^{-4}
	\end{equation}
	This constant is dimensionless, confirmed by $[\xi] = [1]$. The CMB energy density is defined in natural units as:
	\begin{equation}
		\rho_{\text{CMB}} = \frac{\xi}{L_\xi^4}
	\end{equation}
	with the characteristic length scale $L_\xi = 10^{-4}$ m. In SI units, the CMB energy density is:
	\begin{equation}
		\rho_{\text{CMB}} = 4.17 \times 10^{-14} \text{ J}/\text{m}^3
	\end{equation}
	
	\textbf{Formula Explanation:} The CMB energy density represents the energy of the cosmic microwave background. In the T0-theory, it is scaled by $\xi$ and $L_\xi$, where $L_\xi$ is a fundamental length scale potentially linked to cosmic phenomena. The unit analysis shows:
	\begin{equation}
		[\rho_{\text{CMB}}] = \frac{[\xi]}{[L_\xi^4]} = \frac{1}{\text{m}^4} = \text{E}^4 \text{ (in natural units)}
	\end{equation}
	In SI units, this yields J/m$^3$, which is consistent.
	
	\subsection{Conversion of the -Relationship to SI Units}
	The T0-theory posits a fundamental relationship:
	\begin{equation}
		\hbar c \stackrel{!}{=} \xi \rho_{\text{CMB}} L_\xi^4
	\end{equation}
	The unit analysis confirms:
	\begin{equation}
		[\rho_{\text{CMB}}] \cdot [L_\xi^4] \cdot [\xi] = \left( \frac{\text{J}}{\text{m}^3} \right) \cdot \text{m}^4 \cdot 1 = \text{J} \cdot \text{m}
	\end{equation}
	This matches the unit of $\hbar c$. Numerically, we obtain:
	\begin{equation}
		\left( 4.17 \times 10^{-14} \right) \cdot \left( 10^{-4} \right)^4 \cdot \left( \frac{4}{3} \times 10^{-4} \right) = 5.56 \times 10^{-26} \text{ J} \cdot \text{m}
	\end{equation}
	Compared to $\hbar c = 3.16 \times 10^{-26}$ J·m, the factor is approximately 1.76, which corresponds to the geometric factor 16/9.
	
	\textbf{Formula Explanation:} This relationship bridges quantum mechanics ($\hbar c$) with cosmic scales ($\rho_{\text{CMB}}$, $L_\xi$). The dimensionless constant $\xi$ acts as a scaling factor, linking the CMB energy density to the fundamental length scale $L_\xi$.
	
	\subsection{Modified Casimir Formula}
	The modified Casimir formula is:
	\begin{equation}
		|\rho_{\text{Casimir}}(d)| = \frac{\pi^2}{240 \xi} \rho_{\text{CMB}} \left( \frac{L_\xi}{d} \right)^4
	\end{equation}
	The unit analysis yields:
	\begin{equation}
		\frac{[\rho_{\text{CMB}}] \cdot [L_\xi^4]}{[\xi] \cdot [d^4]} = \frac{\left( \frac{\text{J}}{\text{m}^3} \right) \cdot \text{m}^4}{1 \cdot \text{m}^4} = \frac{\text{J}}{\text{m}^3}
	\end{equation}
	This confirms the unit of energy density. Substituting $\rho_{\text{CMB}} = \xi \hbar c / L_\xi^4$ recovers the standard Casimir formula:
	\begin{equation}
		|\rho_{\text{Casimir}}| = \frac{\pi^2}{240} \frac{\xi \hbar c}{L_\xi^4} \cdot \frac{L_\xi^4}{d^4} = \frac{\pi^2 \hbar c}{240 d^4}
	\end{equation}
	
	\textbf{Formula Explanation:} The modified formula incorporates $\xi$ and $\rho_{\text{CMB}}$, linking the Casimir effect to cosmic parameters. Its consistency with the standard formula demonstrates that the T0-theory offers an alternative representation of the effect.
	
	\subsection{Force Calculation}
	The force per area is derived from the energy density:
	\begin{equation}
		\frac{F}{A} = -\frac{\partial}{\partial d} \left( |\rho_{\text{Casimir}}| \cdot d \right) = \frac{\pi^2}{80 \xi} \rho_{\text{CMB}} \left( \frac{L_\xi}{d} \right)^4
	\end{equation}
	The unit analysis shows:
	\begin{equation}
		\frac{[\rho_{\text{CMB}}] \cdot [L_\xi^4]}{[\xi] \cdot [d^4]} = \frac{\left( \frac{\text{J}}{\text{m}^3} \right) \cdot \text{m}^4}{1 \cdot \text{m}^4} = \frac{\text{J}}{\text{m}^3} = \frac{\text{N}}{\text{m}^2}
	\end{equation}
	This matches the unit of pressure, confirming correctness.
	
	\textbf{Formula Explanation:} The force per area represents the measurable Casimir force, arising from the change in energy density with plate separation. The T0-theory scales this force with $\xi$ and $\rho_{\text{CMB}}$, enabling a cosmic interpretation.
	
	\subsection{Summary of Unit Consistency}
	The following table summarizes the unit consistency:
	\begin{table}[h]
		\centering
		\begin{tabular}{l l l l}
			\toprule
			Quantity & SI Unit & Dimensional Analysis & Result \\
			\midrule
			$\rho_{\text{Casimir}}$ & J/m$^3$ & $[E]/[L]^3$ & $\checkmark$ \\
			$\rho_{\text{CMB}}$ & J/m$^3$ & $[E]/[L]^3$ & $\checkmark$ \\
			$\xi$ & dimensionless & $[1]$ & $\checkmark$ \\
			$L_\xi$ & m & $[L]$ & $\checkmark$ \\
			$\hbar c$ & J·m & $[E][L]$ & $\checkmark$ \\
			$\xi \rho_{\text{CMB}} L_\xi^4$ & J·m & $[E][L]$ & $\checkmark$ \\
			\bottomrule
		\end{tabular}
	\end{table}
	
	\subsection{Critical Evaluation}
	The T0-theory demonstrates strengths in complete unit consistency and numerical agreement (deviation for geometric factor 16/9). It links the Casimir effect to cosmic vacuum energy via $\xi$ and $L_\xi$, with $L_\xi = 10^{-4}$ m as a fundamental length scale. This opens new physical interpretations, connecting the Casimir effect to cosmological phenomena.
	
	\subsection{Verification of Natural Units Framework}
	
	All T0-theory equations maintain perfect dimensional consistency in natural units:
	
	\begin{table}[h]
		\centering
		\begin{tabular}{l l l l}
			\toprule
			Quantity & Natural Units & Dimension & Verification \\
			\midrule
			$\xi$ & dimensionless & $[1]$ & $\checkmark$ \\
			$E_\xi$ & 7500 & $[E]$ & $\checkmark$ \\
			$L_\xi$ & $1.33 \times 10^{-4}$ & $[E^{-1}]$ & $\checkmark$ \\
			$T_\xi$ & 7500 & $[E]$ & $\checkmark$ \\
			$G_{\text{nat}}$ & $2.61 \times 10^{-70}$ & $[E^{-2}]$ & $\checkmark$ \\
			\bottomrule
		\end{tabular}
		\caption{Dimensional consistency in natural units}
	\end{table}
	
	\subsection{Energy Scale Hierarchies}
	
	The $\xi$-constant establishes a natural hierarchy of energy scales:
	
	\begin{align}
		E_{\text{Planck}} &= 1 \quad \text{(by definition in natural units)} \\
		E_\xi &= \frac{1}{\xi} = 7500 \\
		E_{\text{weak}} &= \xi^{1/2} \cdot E_{\text{Planck}} \approx 0.0115 \\
		E_{\text{QCD}} &= \xi^{1/3} \cdot E_{\text{Planck}} \approx 0.0107
	\end{align}
	
	\subsection{Additional Experimental Predictions}
	
\section*{Prediction 1: Electromagnetic resonance at characteristic $\xi$-frequency}
	\begin{itemize}
		\item Maximum $\xi$-field-photon coupling at $\nu = E_\xi = 7500$ (nat. units)
		\item Anomalies in electromagnetic propagation at this frequency
		\item Spectral peculiarities in the corresponding frequency range
	\end{itemize}
	
\section*{Prediction 2: Casimir force anomalies at characteristic $\xi$-length scale}
	\begin{itemize}
		\item Standard Casimir law: $F \propto d^{-4}$
		\item $\xi$-field modifications at $d \approx L_\xi = 10^{-4}$ m
		\item Measurable deviations through $\xi$-vacuum coupling
	\end{itemize}
	
\section*{Prediction 3: Modified vacuum fluctuations}
	\begin{itemize}
		\item Vacuum energy density variations at scale $L_\xi$
		\item Correlation between Casimir and CMB measurements
		\item Testable in precision laboratory experiments
	\end{itemize}
	
	\section{Structure Formation in the Static $\xi$-Universe}
	
	\subsection{Continuous Structure Development}
	
	In the static T0 universe, structure formation occurs continuously without Big Bang constraints:
	
	\begin{equation}
		\frac{d\rho}{dt} = -\nabla \cdot (\rho \mathbf{v}) + S_\xi(\rho, T, \xi)
	\end{equation}
	
	where $S_\xi$ is the $\xi$-field source term for continuous matter/energy transformation.
	
	\subsection{$\xi$-Supported Continuous Creation}
	
	The $\xi$-field enables continuous matter/energy transformation:
	
	\begin{align}
		\text{Quantum vacuum} &\xrightarrow{\xi} \text{Virtual particles} \\
		\text{Virtual particles} &\xrightarrow{\xi^2} \text{Real particles} \\
		\text{Real particles} &\xrightarrow{\xi^3} \text{Atomic nuclei} \\
		\text{Atomic nuclei} &\xrightarrow{\text{Time}} \text{Stars, galaxies}
	\end{align}
	
	Energy balance is maintained by:
	\begin{equation}
		\rho_{\text{total}} = \rho_{\text{matter}} + \rho_{\xi\text{-field}} = \text{constant}
	\end{equation}
	
\section*{Important}
		The universe maintains perfect energy conservation through continuous transformation between matter and $\xi$-field energy, enabling eternal existence without beginning or end.
% end box important
	
\section*{Formula}
		The universal $\xi$-constant generates a complete, self-consistent physical structure in natural units:
		\[\boxed{
			\begin{aligned}
				\xi &= \frac{4}{3} \times 10^{-4} \quad \text{(exact geometric value)} \\[0.3em]
				E_\xi &= \frac{3}{4} \times 10^4 = 7500 \quad \text{(characteristic energy)} \\[0.3em]
				L_\xi &= \frac{1}{E_\xi} \approx 1.33 \times 10^{-4} \quad \text{(characteristic length)} \\[0.3em]
				G_{\text{nat}} &= \xi^2 \cdot f_G \quad \text{(gravitational constant)} \\[0.3em]
				H_0^{T0} &= 67.45 \text{ km/s/Mpc} \quad \text{(Hubble constant resolved)}
			\end{aligned}
		}\]
		(all quantities in natural units except $H_0$)
% end box formula
	
\section*{Important}
		The vacuum is the $\xi$-field. The CMB arises from T-field quantum fluctuations. The Casimir force arises from geometric constraint of the $\xi$-field vacuum. All fundamental forces and particles emerge from different manifestations of the universal $\xi$-field.
% end box important
	
	\section{Conclusions}
	
	The T0-analysis of temperature units in natural units with complete CMB calculations establishes:
	
	\begin{enumerate}
		\item \textbf{Universal $\xi$-scaling}: All temperature and energy scales follow from the geometric constant $\xi = \frac{4}{3} \times 10^{-4}$.
		
		\item \textbf{CMB without inflation}: The theory successfully explains the CMB at $z \approx 1100$ without requiring inflation, deriving primordial perturbations from T-field quantum fluctuations.
		
		\item \textbf{Resolution of cosmological tensions}: The Hubble tension is naturally resolved with $H_0 = 67.45 \pm 1.1$ km/s/Mpc, and the $S_8$ tension is addressed.
		
		\item \textbf{Static universe paradigm}: The universe is eternal and static, respecting fundamental quantum mechanics without paradoxes.
		
		\item \textbf{Time-energy consistency}: The static universe respects the Heisenberg uncertainty relation without requiring a Big Bang.
		
		\item \textbf{Mathematical elegance}: Complete dimensional consistency in natural units without free parameters.
		
		\item \textbf{Unit-independent physics}: All relationships consist of exact mathematical ratios derived from fundamental geometry.
		
		\item \textbf{Testable predictions}: Specific, measurable deviations from $\Lambda$CDM that can be tested with next-generation experiments.
	\end{enumerate}
	
\section*{Revolutionary}
		T0-theory offers a mathematically consistent alternative formulated in natural units to expansion-based cosmology and explains temperature phenomena from particle physics to the cosmos with a single fundamental constant derived from pure geometry. The complete CMB calculations demonstrate that complex cosmological observations can be explained within this unified framework.
% end box revolutionary
	
	\section{References}
	
	


% Bibliography
\begin{thebibliography}{99}
	
	\bibitem{pdg2024}
	Particle Data Group Collaboration (2024). 
	\textit{Review of Particle Physics}. 
	Progress of Theoretical and Experimental Physics, 2024(8), 083C01.
	\url{https://pdg.lbl.gov}
	
	\bibitem{flag2024}
	Aoki, Y., et al. (FLAG Collaboration) (2024). 
	\textit{FLAG Review 2024 of Lattice Results for Low-Energy Constants}. 
	arXiv:2411.04268.
	\url{https://arxiv.org/abs/2411.04268}
	
	\bibitem{fermilab_muon_g2}
	Abi, B., et al. (Muon g-2 Collaboration) (2021). 
	\textit{Measurement of the Positive Muon Anomalous Magnetic Moment to 0.46 ppm}. 
	Physical Review Letters, 126, 141801.
	
	\bibitem{peskin_schroeder}
	Peskin, M. E., \& Schroeder, D. V. (1995). 
	\textit{An Introduction to Quantum Field Theory}. 
	Addison-Wesley.
	
	\bibitem{weinberg_qft}
	Weinberg, S. (1995). 
	\textit{The Quantum Theory of Fields, Vol. I--III}. 
	Cambridge University Press.
	
	\bibitem{griffiths_particle}
	Griffiths, D. (2008). 
	\textit{Introduction to Elementary Particles}. 
	Wiley-VCH.
	
	\bibitem{mandl_shaw}
	Mandl, F., \& Shaw, G. (2010). 
	\textit{Quantum Field Theory (2nd ed.)}. 
	Wiley.
	
	\bibitem{srednicki_qft}
	Srednicki, M. (2007). 
	\textit{Quantum Field Theory}. 
	Cambridge University Press.
	
	\bibitem{t0_fundamentals}
	Pascher, J. (2024). 
	\textit{T0-Theory: Foundations of Time-Mass Duality}. 
	Unpublished manuscript, HTL Leonding.
	
	\bibitem{t0_fine_structure}
	Pascher, J. (2024). 
	\textit{T0-Theory: The Fine Structure Constant}. 
	Unpublished manuscript, HTL Leonding.
	
	\bibitem{t0_neutrinos}
	Pascher, J. (2024). 
	\textit{T0-Theory: Neutrino Masses and PMNS Mixing}. 
	Unpublished manuscript, HTL Leonding.
	
	\bibitem{t0_github}
	Pascher, J. (2024--2025). 
	\textit{T0-Time-Mass-Duality Repository}. 
	GitHub.
	\url{https://github.com/jpascher/T0-Time-Mass-Duality}
	
	\bibitem{lattice_qcd_review}
	Kronfeld, A. S. (2012). 
	\textit{Twenty-first Century Lattice Gauge Theory: Results from the QCD Lagrangian}. 
	Annual Review of Nuclear and Particle Science, 62, 265--284.
	
	\bibitem{neutrino_mixing_pdg}
	Particle Data Group Collaboration (2024). 
	\textit{Neutrino Masses, Mixing, and Oscillations}. 
	PDG Review 2024.
	\url{https://pdg.lbl.gov/2024/reviews/rpp2024-rev-neutrino-mixing.pdf}
	
	\bibitem{higgs_discovery}
	ATLAS and CMS Collaborations (2012). 
	\textit{Observation of a New Particle in the Search for the Standard Model Higgs Boson}. 
	Physics Letters B, 716, 1--29.
	
	\bibitem{Brannen2005}
	C. P. Brannen, ``Estimate of neutrino masses from Koide's relation'', \textit{arXiv:hep-ph/0505028} (2005).
	\url{https://arxiv.org/abs/hep-ph/0505028}
	
	\bibitem{Brannen2006}
	C. P. Brannen, ``Koide Mass Formula for Neutrinos'', \textit{arXiv:0702.0052} (2006).
	\url{http://brannenworks.com/MASSES.pdf}
	
	\bibitem{PhaseVectors2025}
	Anonymous, ``The Koide Relation and Lepton Mass Hierarchy from Phase Vectors'', \textit{rXiv:2507.0040} (2025).
	\url{https://rxiv.org/pdf/2507.0040v1.pdf}
	
	\bibitem{PDG2025}
	Particle Data Group, ``Review of Particle Physics'', \textit{Phys. Rev. D} \textbf{112} (2025) 030001.
	\url{https://pdg.lbl.gov/2025/}
	
	\bibitem{terrell2024}
	Terrell et al. (2024). 
	\textit{Single-Clock Metrology in Nature}. 
	Nature Physics.
	
	\bibitem{hossenfelder2024}
	Hossenfelder, S. (2024). 
	\textit{Single Clock Video Explanation}. 
	YouTube.
	
	\bibitem{hundert1931}
	Hundert (1931). 
	\textit{Reference Work}. 
	Publisher.
	
	\bibitem{terrell2025}
	Terrell et al. (2025). 
	\textit{Advanced Clock Synchronization Methods}. 
	Physical Review Letters.
	
	\bibitem{pascher_t0_2025}
	Pascher, J. (2025). 
	\textit{T0-Theory: Complete Framework and Applications}. 
	Unpublished manuscript, HTL Leonding.
	
	\bibitem{t0qm}
	Pascher, J. (2024). 
	\textit{T0-Theory: Quantum Mechanics Formulation}. 
	Unpublished manuscript, HTL Leonding.
	
	\bibitem{t0anomale}
	Pascher, J. (2024). 
	\textit{T0-Theory: Anomalous Magnetic Moments}. 
	Unpublished manuscript, HTL Leonding.
	
	\bibitem{muong2complete}
	Abi, B., et al. (Muon g-2 Collaboration) (2023). 
	\textit{Complete Measurement of the Positive Muon Anomalous Magnetic Moment}. 
	Physical Review Letters, 131, 161802.
	
	\bibitem{penrose2004}
	Penrose, R. (2004). 
	\textit{The Road to Reality: A Complete Guide to the Laws of the Universe}. 
	Jonathan Cape.
	
	\bibitem{planck1900}
	Planck, M. (1900). 
	\textit{On the Theory of the Energy Distribution Law of the Normal Spectrum}. 
	Verhandlungen der Deutschen Physikalischen Gesellschaft, 2, 237.
	
	\bibitem{T0Theory}
	Pascher, J. (2024). 
	\textit{T0-Theory: Fundamental Principles}. 
	Unpublished manuscript, HTL Leonding.
	
	% Additional bibliography entries for all undefined citations
	\bibitem{6g_roadmap}
	6G Research Consortium (2024).
	\textit{6G Technology Roadmap}.
	Technical Report.
	
	\bibitem{Born2013}
	Born, M. (2013).
	\textit{Einstein's Theory of Relativity}.
	Dover Publications.
	
	\bibitem{Casimir1948}
	Casimir, H. B. G. (1948).
	\textit{On the attraction between two perfectly conducting plates}.
	Proc. Kon. Ned. Akad. Wetensch. B51, 793--795.
	
	\bibitem{Einstein1905}
	Einstein, A. (1905).
	\textit{On the Electrodynamics of Moving Bodies}.
	Annalen der Physik, 17, 891--921.
	
	\bibitem{Feynman2006}
	Feynman, R. P. (2006).
	\textit{QED: The Strange Theory of Light and Matter}.
	Princeton University Press.
	
	\bibitem{Griffiths2017}
	Griffiths, D. J. (2017).
	\textit{Introduction to Electrodynamics (4th ed.)}.
	Cambridge University Press.
	
	\bibitem{Jackson1999}
	Jackson, J. D. (1999).
	\textit{Classical Electrodynamics (3rd ed.)}.
	Wiley.
	
	\bibitem{Mohr2016}
	Mohr, P. J., et al. (2016).
	\textit{CODATA Recommended Values of the Fundamental Physical Constants: 2014}.
	Rev. Mod. Phys. 88, 035009.
	
	\bibitem{Parker2018}
	Parker, R. H., et al. (2018).
	\textit{Measurement of the fine-structure constant as a test of the Standard Model}.
	Science, 360, 191--195.
	
	\bibitem{Planck1900}
	Planck, M. (1900).
	\textit{On the Theory of the Energy Distribution Law of the Normal Spectrum}.
	Verhandlungen der Deutschen Physikalischen Gesellschaft, 2, 237.
	
	\bibitem{Planck2018}
	Planck Collaboration (2018).
	\textit{Planck 2018 results. VI. Cosmological parameters}.
	Astronomy \& Astrophysics, 641, A6.
	
	\bibitem{QFT_T0}
	Pascher, J. (2024).
	\textit{T0-Theory and QFT Connections}.
	Unpublished manuscript, HTL Leonding.
	
	\bibitem{Sommerfeld1916}
	Sommerfeld, A. (1916).
	\textit{On the Quantum Theory of Spectral Lines}.
	Annalen der Physik, 51, 1--94.
	
	\bibitem{T0_Feinstruktur}
	Pascher, J. (2024).
	\textit{T0-Theory: Fine Structure Analysis}.
	Unpublished manuscript, HTL Leonding.
	
	\bibitem{T0_SI}
	Pascher, J. (2024).
	\textit{T0-Theory and SI Units}.
	Unpublished manuscript, HTL Leonding.
	
	\bibitem{T0_fine_structure}
	Pascher, J. (2024).
	\textit{T0-Theory: The Fine Structure Constant}.
	Unpublished manuscript, HTL Leonding.
	
	\bibitem{T0_g2_erweiterung}
	Pascher, J. (2024).
	\textit{T0-Theory: g-2 Extensions}.
	Unpublished manuscript, HTL Leonding.
	
	\bibitem{T0_gravitational_constant}
	Pascher, J. (2024).
	\textit{T0-Theory: Gravitational Constant Derivation}.
	Unpublished manuscript, HTL Leonding.
	
	\bibitem{T0_netze_en}
	Pascher, J. (2024).
	\textit{T0-Theory: Network Structures}.
	Unpublished manuscript, HTL Leonding.
	
	\bibitem{T0_tm_erweiterung}
	Pascher, J. (2024).
	\textit{T0-Theory: Time-Mass Extensions}.
	Unpublished manuscript, HTL Leonding.
	
	\bibitem{Uzan2003}
	Uzan, J.-P. (2003).
	\textit{The fundamental constants and their variation}.
	Rev. Mod. Phys. 75, 403--455.
	
	\bibitem{Weinberg1995}
	Weinberg, S. (1995).
	\textit{The Quantum Theory of Fields, Vol. I}.
	Cambridge University Press.
	
	\bibitem{albrecht1999}
	Albrecht, A. \& Magueijo, J. (1999).
	\textit{A time varying speed of light as a solution to cosmological puzzles}.
	Phys. Rev. D 59, 043516.
	
	\bibitem{alice2023}
	ALICE Collaboration (2023).
	\textit{Recent results from ALICE}.
	CERN-EP-2023-XXX.
	
	\bibitem{analog_optical}
	Smith, J. et al. (2024).
	\textit{Analog optical computing systems}.
	Nature Photonics.
	
	\bibitem{ashtekar2004}
	Ashtekar, A. \& Lewandowski, J. (2004).
	\textit{Background independent quantum gravity}.
	Class. Quantum Grav. 21, R53.
	
	\bibitem{atlas2023}
	ATLAS Collaboration (2023).
	\textit{ATLAS physics results}.
	CERN-PH-EP-2023-XXX.
	
	\bibitem{atlas2023higgs}
	ATLAS Collaboration (2023).
	\textit{Higgs boson measurements}.
	Phys. Rev. Lett.
	
	\bibitem{barbour1999}
	Barbour, J. (1999).
	\textit{The End of Time}.
	Oxford University Press.
	
	\bibitem{barrow1999}
	Barrow, J. D. (1999).
	\textit{Cosmologies with varying light speed}.
	Phys. Rev. D 59, 043515.
	
	\bibitem{becker2007}
	Becker, K. et al. (2007).
	\textit{String Theory and M-Theory}.
	Cambridge University Press.
	
	\bibitem{bell_muon}
	Bennett, G. W., et al. (Muon g-2 Collaboration) (2006).
	\textit{Final report of the E821 muon anomalous magnetic moment measurement}.
	Phys. Rev. D 73, 072003.
	
	\bibitem{bondi1948}
	Bondi, H. \& Gold, T. (1948).
	\textit{The steady-state theory of the expanding universe}.
	Mon. Not. R. Astron. Soc. 108, 252--270.
	
	\bibitem{brewer2019}
	Brewer, S. M. et al. (2019).
	\textit{Al+ Quantum-Logic Clock with Systematic Uncertainty below $10^{-18}$}.
	Phys. Rev. Lett. 123, 033201.
	
	\bibitem{cms2023top}
	CMS Collaboration (2023).
	\textit{Top quark measurements at CMS}.
	JHEP 2023.
	
	\bibitem{cms2024}
	CMS Collaboration (2024).
	\textit{CMS physics results 2024}.
	CERN-PH-EP-2024-XXX.
	
	\bibitem{codata2019}
	Tiesinga, E. et al. (2019).
	\textit{The 2018 CODATA Recommended Values}.
	J. Phys. Chem. Ref. Data.
	
	\bibitem{desi2025}
	DESI Collaboration (2025).
	\textit{DESI 2025 Cosmology Results}.
	arXiv preprint.
	
	\bibitem{differential_optical}
	Wang, X. et al. (2024).
	\textit{Differential optical computing}.
	Optica.
	
	\bibitem{dingle1972}
	Dingle, H. (1972).
	\textit{Science at the Crossroads}.
	Martin Brian \& O'Keeffe.
	
	\bibitem{divalentino2021}
	Di Valentino, E. et al. (2021).
	\textit{In the realm of the Hubble tension}.
	Class. Quantum Grav. 38, 153001.
	
	\bibitem{elnaschie2004}
	El Naschie, M. S. (2004).
	\textit{A review of E infinity theory}.
	Chaos, Solitons \& Fractals, 19, 209--236.
	
	\bibitem{fabrication_heterogeneous}
	Chen, Y. et al. (2024).
	\textit{Heterogeneous photonic integration}.
	Nature Electronics.
	
	\bibitem{fermilab2023}
	Fermilab (2023).
	\textit{Muon g-2 results}.
	Phys. Rev. Lett.
	
	\bibitem{flexible_wafer}
	Kim, S. et al. (2024).
	\textit{Flexible wafer-scale photonics}.
	Science Advances.
	
	\bibitem{francesco1997}
	Di Francesco, P. et al. (1997).
	\textit{Conformal Field Theory}.
	Springer.
	
	\bibitem{hartree1957}
	Hartree, D. R. (1957).
	\textit{The Calculation of Atomic Structures}.
	Wiley.
	
	\bibitem{hhi_6g}
	Fraunhofer HHI (2024).
	\textit{6G Photonic Integration}.
	Technical Report.
	
	\bibitem{hossenfelder2025}
	Hossenfelder, S. (2025).
	\textit{Science without the gobbledygook}.
	YouTube/Blog.
	
	\bibitem{hossenfelder_single_clock_video}
	Hossenfelder, S. (2024).
	\textit{The Single Clock Problem}.
	YouTube.
	
	\bibitem{hoyle1948}
	Hoyle, F. (1948).
	\textit{A new model for the expanding universe}.
	Mon. Not. R. Astron. Soc. 108, 372--382.
	
	\bibitem{integration_microelectronic}
	Liu, A. et al. (2024).
	\textit{Microelectronic photonic integration}.
	IEEE Journal.
	
	\bibitem{jacobson1995}
	Jacobson, T. (1995).
	\textit{Thermodynamics of spacetime}.
	Phys. Rev. Lett. 75, 1260.
	
	\bibitem{kasevich2023}
	Kasevich, M. et al. (2023).
	\textit{Atom interferometry tests}.
	Nature Physics.
	
	\bibitem{lerner2014}
	Lerner, E. J. (2014).
	\textit{An open letter on cosmology}.
	New Scientist.
	
	\bibitem{lisa2017}
	LISA Consortium (2017).
	\textit{Laser Interferometer Space Antenna}.
	ESA Technical Report.
	
	\bibitem{lithium_tantalate}
	Zhang, M. et al. (2024).
	\textit{Thin-film lithium tantalate photonics}.
	Nature Photonics.
	
	\bibitem{lopez2010}
	Lopez-Corredoira, M. (2010).
	\textit{Tests and problems of the standard model in cosmology}.
	Int. J. Mod. Phys. D.
	
	\bibitem{ludlow2015}
	Ludlow, A. D. et al. (2015).
	\textit{Optical atomic clocks}.
	Rev. Mod. Phys. 87, 637.
	
	\bibitem{mach1883}
	Mach, E. (1883).
	\textit{Die Mechanik in ihrer Entwickelung}.
	F.A. Brockhaus.
	
	\bibitem{maldacena1998}
	Maldacena, J. (1998).
	\textit{The large N limit of superconformal field theories}.
	Adv. Theor. Math. Phys. 2, 231--252.
	
	\bibitem{mueller2014}
	Müller, H. et al. (2014).
	\textit{Atom interferometry tests of the gravitational redshift}.
	Phys. Rev. Lett.
	
	\bibitem{mug2_final_2025}
	Muon g-2 Collaboration (2025).
	\textit{Final muon g-2 measurement}.
	Phys. Rev. Lett.
	
	\bibitem{muong2_2023}
	Muon g-2 Collaboration (2023).
	\textit{Updated muon g-2 results}.
	Phys. Rev. Lett.
	
	\bibitem{nathan2024}
	Nathan, A. et al. (2024).
	\textit{Quantum computing advances}.
	Nature.
	
	\bibitem{newell2018}
	Newell, D. B. et al. (2018).
	\textit{The CODATA 2017 values of h, e, k, and $N_A$}.
	Metrologia 55, L13.
	
	\bibitem{nottale1993}
	Nottale, L. (1993).
	\textit{Fractal Space-Time and Microphysics}.
	World Scientific.
	
	\bibitem{on_chip_lithium}
	Wang, C. et al. (2024).
	\textit{On-chip lithium niobate photonics}.
	Nature Communications.
	
	\bibitem{optical_advantages}
	Shastri, B. J. et al. (2024).
	\textit{Advantages of optical computing}.
	Nature Reviews Physics.
	
	\bibitem{pascher2025cmb}
	Pascher, J. (2025).
	\textit{T0-Theory: CMB Analysis}.
	Unpublished manuscript, HTL Leonding.
	
	\bibitem{pascher2025g2}
	Pascher, J. (2025).
	\textit{T0-Theory: g-2 Predictions}.
	Unpublished manuscript, HTL Leonding.
	
	\bibitem{pascher2025qm}
	Pascher, J. (2025).
	\textit{T0-Theory: Quantum Mechanics}.
	Unpublished manuscript, HTL Leonding.
	
	\bibitem{pascher2025si}
	Pascher, J. (2025).
	\textit{T0-Theory: SI Unit System}.
	Unpublished manuscript, HTL Leonding.
	
	\bibitem{pascher2025t0}
	Pascher, J. (2025).
	\textit{T0-Theory: Complete Framework}.
	Unpublished manuscript, HTL Leonding.
	
	\bibitem{pascher:fundamentals}
	Pascher, J. (2024).
	\textit{T0-Theory: Fundamentals}.
	Unpublished manuscript, HTL Leonding.
	
	\bibitem{pascher:g2_rev9}
	Pascher, J. (2024).
	\textit{T0-Theory: g-2 Revision 9}.
	Unpublished manuscript, HTL Leonding.
	
	\bibitem{pascher:geometric_formalism}
	Pascher, J. (2024).
	\textit{T0-Theory: Geometric Formalism}.
	Unpublished manuscript, HTL Leonding.
	
	\bibitem{pascher:ml_addendum}
	Pascher, J. (2024).
	\textit{T0-Theory: Machine Learning Addendum}.
	Unpublished manuscript, HTL Leonding.
	
	\bibitem{pascher:t0_foundations}
	Pascher, J. (2024).
	\textit{T0-Theory: Foundations}.
	Unpublished manuscript, HTL Leonding.
	
	\bibitem{pascher_derivation_beta_2025}
	Pascher, J. (2025).
	\textit{T0-Theory: Derivation of Beta}.
	Unpublished manuscript, HTL Leonding.
	
	\bibitem{pascher_higgs_connection_2025}
	Pascher, J. (2025).
	\textit{T0-Theory: Higgs Connection}.
	Unpublished manuscript, HTL Leonding.
	
	\bibitem{pascher_lagrangian_extended_2025}
	Pascher, J. (2025).
	\textit{T0-Theory: Extended Lagrangian}.
	Unpublished manuscript, HTL Leonding.
	
	\bibitem{pascher_mathematical_structure_2025}
	Pascher, J. (2025).
	\textit{T0-Theory: Mathematical Structure}.
	Unpublished manuscript, HTL Leonding.
	
	\bibitem{pascher_t0_cmb_2025}
	Pascher, J. (2025).
	\textit{T0-Theory: CMB Predictions}.
	Unpublished manuscript, HTL Leonding.
	
	\bibitem{pascher_t0_energie_2025}
	Pascher, J. (2025).
	\textit{T0-Theory: Energy}.
	Unpublished manuscript, HTL Leonding.
	
	\bibitem{pascher_t0_energy_2025}
	Pascher, J. (2025).
	\textit{T0-Theory: Energy Framework}.
	Unpublished manuscript, HTL Leonding.
	
	\bibitem{pascher_t0_theory_2025}
	Pascher, J. (2025).
	\textit{T0-Theory: Complete Theory}.
	Unpublished manuscript, HTL Leonding.
	
	\bibitem{penrose1959}
	Penrose, R. (1959).
	\textit{The apparent shape of a relativistically moving sphere}.
	Proc. Cambridge Phil. Soc. 55, 137--139.
	
	\bibitem{penrose1967}
	Penrose, R. (1967).
	\textit{Twistor algebra}.
	J. Math. Phys. 8, 345--366.
	
	\bibitem{peratt1992}
	Peratt, A. L. (1992).
	\textit{Physics of the Plasma Universe}.
	Springer-Verlag.
	
	\bibitem{peskin1995}
	Peskin, M. E. \& Schroeder, D. V. (1995).
	\textit{An Introduction to Quantum Field Theory}.
	Addison-Wesley.
	
	\bibitem{peskin_schroeder_1995}
	Peskin, M. E. \& Schroeder, D. V. (1995).
	\textit{An Introduction to Quantum Field Theory}.
	Addison-Wesley.
	
	\bibitem{phoquant}
	PhoQuant (2024).
	\textit{Photonic quantum computing}.
	Technical Report.
	
	\bibitem{photonics_ai}
	Wetzstein, G. et al. (2024).
	\textit{Photonics for AI}.
	Nature.
	
	\bibitem{planck1906}
	Planck, M. (1906).
	\textit{The Theory of Heat Radiation}.
	Johann Ambrosius Barth.
	
	\bibitem{planck2018}
	Planck Collaboration (2018).
	\textit{Planck 2018 results}.
	A\&A 641, A6.
	
	\bibitem{polchinski1998}
	Polchinski, J. (1998).
	\textit{String Theory}.
	Cambridge University Press.
	
	\bibitem{qant_nps}
	QANT (2024).
	\textit{Quantum photonics systems}.
	Technical Report.
	
	\bibitem{quantenjahr25}
	Quantenjahr (2025).
	\textit{International Year of Quantum}.
	UNESCO.
	
	\bibitem{recurrent_photonics}
	Tait, A. N. et al. (2024).
	\textit{Recurrent photonic neural networks}.
	Optica.
	
	\bibitem{rf_photonics}
	Capmany, J. \& Novak, D. (2024).
	\textit{Microwave photonics}.
	Nature Photonics.
	
	\bibitem{riess2019}
	Riess, A. G. et al. (2019).
	\textit{Large Magellanic Cloud Cepheid Standards}.
	ApJ 876, 85.
	
	\bibitem{riess2022}
	Riess, A. G. et al. (2022).
	\textit{A Comprehensive Measurement of H0}.
	ApJ 934, L7.
	
	\bibitem{rovelli2004}
	Rovelli, C. (2004).
	\textit{Quantum Gravity}.
	Cambridge University Press.
	
	\bibitem{sciama1953}
	Sciama, D. W. (1953).
	\textit{On the origin of inertia}.
	Mon. Not. R. Astron. Soc. 113, 34--42.
	
	\bibitem{sciencedaily2025}
	ScienceDaily (2025).
	\textit{Physics news}.
	Online.
	
	\bibitem{sm_g2_2025}
	Aoyama, T. et al. (2025).
	\textit{Standard Model prediction for g-2}.
	Phys. Rep.
	
	\bibitem{susskind1995}
	Susskind, L. (1995).
	\textit{The world as a hologram}.
	J. Math. Phys. 36, 6377--6396.
	
	\bibitem{t0_kosmologie}
	Pascher, J. (2024).
	\textit{T0-Theory: Cosmology}.
	Unpublished manuscript, HTL Leonding.
	
	\bibitem{terrell1959}
	Terrell, J. (1959).
	\textit{Invisibility of the Lorentz contraction}.
	Phys. Rev. 116, 1041--1045.
	
	\bibitem{terrell_single_clock_nature_2024}
	Terrell, J. et al. (2024).
	\textit{Single clock precision measurements}.
	Nature Physics.
	
	\bibitem{tfln_foundry}
	TFLN Foundry (2024).
	\textit{Thin-film lithium niobate foundry services}.
	Technical Specifications.
	
	\bibitem{thiemann2007}
	Thiemann, T. (2007).
	\textit{Modern Canonical Quantum General Relativity}.
	Cambridge University Press.
	
	\bibitem{thz_epfl}
	EPFL (2024).
	\textit{Terahertz photonics research}.
	Technical Report.
	
	\bibitem{unnikrishnan2004}
	Unnikrishnan, C. S. (2004).
	\textit{On Einstein's resolution of the twin clock paradox}.
	Current Science, 86, 704--709.
	
	\bibitem{verlinde2011}
	Verlinde, E. (2011).
	\textit{On the origin of gravity and the laws of Newton}.
	JHEP 2011, 29.
	
	\bibitem{video2025}
	Video (2025).
	\textit{Physics video explanation}.
	YouTube.
	
	\bibitem{weinberg1995}
	Weinberg, S. (1995).
	\textit{The Quantum Theory of Fields}.
	Cambridge University Press.
	
	\bibitem{weiskopf2000}
	Weiskopf, D. (2000).
	\textit{Visualization of special relativity}.
	PhD thesis, University of Tübingen.
	
	\bibitem{wheeler1990}
	Wheeler, J. A. (1990).
	\textit{A Journey into Gravity and Spacetime}.
	Scientific American Library.
	
	\bibitem{wiki_bell}
	Wikipedia (2024).
	\textit{Bell's theorem}.
	Online encyclopedia.
	
	\bibitem{zwicky1929}
	Zwicky, F. (1929).
	\textit{On the red shift of spectral lines through interstellar space}.
	Proc. Natl. Acad. Sci. 15, 773--779.

\end{thebibliography}


\end{document}
