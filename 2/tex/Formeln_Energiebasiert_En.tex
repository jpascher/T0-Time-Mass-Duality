\documentclass[12pt,a4paper]{article}
\usepackage[utf8]{inputenc}
\usepackage[T1]{fontenc}
\usepackage[english]{babel}
\usepackage{amsmath,amsfonts,amssymb}
\usepackage{physics}
\usepackage{siunitx}
\usepackage{booktabs}
\usepackage{longtable}
\usepackage{array}
\usepackage{xcolor}
\usepackage{geometry}
\usepackage{textgreek}
\usepackage{fancyhdr}
\usepackage{hyperref}
\usepackage{tocloft}
\geometry{margin=2.5cm}
% Header and Footer Configuration
\pagestyle{fancy}
\fancyhf{}
\fancyhead[L]{\textsc{T0-Model}}
\fancyhead[R]{\textsc{A Reformulation of Physics}}
\fancyfoot[C]{\thepage}
\renewcommand{\headrulewidth}{0.4pt}
\renewcommand{\footrulewidth}{0.4pt}
% Table of Contents Styling
\renewcommand{\cfttoctitlefont}{\huge\bfseries\color{blue}}
\renewcommand{\cftsecfont}{\color{blue}}
\renewcommand{\cftsubsecfont}{\color{blue}}
\renewcommand{\cftsecpagefont}{\color{blue}}
\renewcommand{\cftsubsecpagefont}{\color{blue}}
\hypersetup{
	colorlinks=true,
	linkcolor=blue,
	citecolor=blue,
	urlcolor=blue,
	pdftitle={T0-Model Formula Collection (Energy-Based Version)},
	pdfauthor={Johann Pascher},
	pdfsubject={T0-Model, Time-Energy Duality, Theoretical Physics},
	pdfkeywords={T0 Theory, Natural Units, Quantum Mechanics, Cosmology}
}
\title{T0-Model Formula Collection\\
	\large (Energy-Based Version)}
\author{Johann Pascher\\
	\small Higher Technical Federal Institute (HTL), Leonding, Austria\\
	\small \texttt{johann.pascher@gmail.com}}
\date{\today}

\begin{document}
	
	\maketitle
	\tableofcontents
	\newpage
	
	\section{FUNDAMENTAL PRINCIPLES}
	
	\subsection{Universal Geometric Parameter}
	\begin{itemize}
		\item The fundamental parameter of the T0-model:
		$$\xi = \frac{4}{3} \times 10^{-4}$$
		
		\item Relationship to 3D geometry:
		$$G_3 = \frac{4}{3} \text{ (three-dimensional geometry factor)}$$
	\end{itemize}
	
	\subsection{Time-Energy Duality}
	\begin{itemize}
		\item Fundamental duality relationship:
		$$T_{\text{field}} \cdot E_{\text{field}} = 1$$
		
		\item Characteristic T0 length:
		$$r_0 = 2GE$$
		
		\item Characteristic T0 time:
		$$t_0 = 2GE$$
	\end{itemize}
	
	\subsection{Universal Wave Equation}
	\begin{itemize}
		\item D'Alembert operator on energy field:
		$$\square E_{\text{field}} = \left(\nabla^2 - \frac{\partial^2}{\partial t^2}\right) E_{\text{field}} = 0$$
		
		\item Geometry-coupled equation:
		$$\square E_{\text{field}} + \frac{G_3}{\ell_P^2} E_{\text{field}} = 0$$
	\end{itemize}
	
	\subsection{Universal Lagrangian Density}
	\begin{itemize}
		\item Fundamental action principle:
		$$\boxed{\mathcal{L} = \varepsilon \cdot (\partial E_{\text{field}})^2}$$
		
		\item Coupling parameter:
		$$\varepsilon = \frac{\xi}{E_P^2} = \frac{4/3 \times 10^{-4}}{E_P^2}$$
	\end{itemize}
	
	\section{NATURAL UNITS AND SCALES}
	
	\subsection{Natural Units}
	\begin{itemize}
		\item Fundamental constants:
		$$\hbar = c = k_B = 1$$
		
		\item Gravitational constant:
		$$G = 1 \text{ numerically, but retains dimension } [G] = [E^{-2}]$$
	\end{itemize}
	
	\subsection{Planck Scale as Reference}
\begin{itemize}
	\item Planck length:
	$$\ell_P = \sqrt{G}$$
	
	\item Scale ratio:
	$$\xi_{\text{rat}} = \frac{\ell_P}{r_0}$$
	
	\item Relationship between Planck and T0 scales:
	$$\xi = \frac{\ell_P}{r_0} = \frac{\sqrt{G}}{2GE} = \frac{1}{2\sqrt{G} \cdot E}$$
\end{itemize}

	\subsection{Energy Scale Hierarchy}
	\begin{itemize}
		\item Planck energy:
		$$E_P = 1 \text{ (Planck reference scale)}$$
		
		\item Electroweak energy:
		$$E_{\text{electroweak}} = \sqrt{\xi} \cdot E_P \approx 0.012 \, E_P$$
		
		\item T0 energy:
		$$E_{\text{T0}} = \xi \cdot E_P \approx 1.33 \times 10^{-4} \, E_P$$
		
		\item Atomic energy:
		$$E_{\text{atomic}} = \xi^{3/2} \cdot E_P \approx 1.5 \times 10^{-6} \, E_P$$
	\end{itemize}
	
	\subsection{Universal Scaling Laws}
	\begin{itemize}
		\item Energy scale ratio:
		$$\frac{E_i}{E_j} = \left(\frac{\xi_i}{\xi_j}\right)^{\alpha_{ij}}$$
		
		\item Interaction-specific exponents:
		\begin{align*}
			\alpha_{\text{EM}} &= 1 \quad \text{(linear electromagnetic scaling)}\\
			\alpha_{\text{weak}} &= 1/2 \quad \text{(square root weak scaling)}\\
			\alpha_{\text{strong}} &= 1/3 \quad \text{(cube root strong scaling)}\\
			\alpha_{\text{grav}} &= 2 \quad \text{(quadratic gravitational scaling)}
		\end{align*}
	\end{itemize}
	
	\section{ELECTROMAGNETISM AND COUPLING}
	
	\subsection{Coupling Constants}
\begin{itemize}
	\item Electromagnetic coupling:
	$$\alpha_{\text{EM}} = 1 \text{ (natural units)}, 1/137.036 \text{ (SI)}$$
	
	\item Gravitational coupling:
	$$\alpha_G = \xi^2 = 1.78 \times 10^{-8}$$
	
	\item Weak coupling:
	$$\alpha_W = \xi^{1/2} = 1.15 \times 10^{-2}$$
	
	\item Strong coupling:
	$$\alpha_S = \xi^{-1/3} = 9.65$$
\end{itemize}

	\subsection{Fine Structure Constant}
	\begin{itemize}
		\item Fine structure constant in SI units:
		$$\frac{1}{137.036} = 1 \cdot \frac{\hbar c}{4\pi\varepsilon_0 e^2}$$
		
		\item Relationship to T0-model:
		$$\alpha_{\text{observed}} = \xi \cdot f_{\text{geometric}} = \frac{4}{3} \times 10^{-4} \cdot f_{\text{EM}}$$
		
		\item Calculation of the geometric factor:
		$$f_{\text{EM}} = \frac{\alpha_{\text{SI}}}{\xi} = \frac{7.297 \times 10^{-3}}{1.333 \times 10^{-4}} = 54.7$$
		
		\item Geometric interpretation:
		$$f_{\text{EM}} = \frac{4\pi^2}{3} \approx 13.16 \times 4.16 \approx 55$$
	\end{itemize}
	
	\subsection{Electromagnetic Lagrangian Density}
	\begin{itemize}
		\item Electromagnetic Lagrangian density:
		$$\mathcal{L}_{\text{EM}} = -\frac{1}{4}F_{\mu\nu}F^{\mu\nu} + \bar{\psi}(i\gamma^\mu D_\mu - m)\psi$$
		
		\item Covariant derivative:
		$$D_\mu = \partial_\mu + i \alpha_{\text{EM}} A_\mu = \partial_\mu + i A_\mu$$
		(Since $\alpha_{\text{EM}} = 1$ in natural units)
	\end{itemize}
	
\section{ANOMALOUS MAGNETIC MOMENT}

\subsection{Fundamental T0 Formula}
\begin{itemize}
	\item T0-Model Lagrangian structure:
	$$\mathcal{L}_{\text{T0}} = \mathcal{L}_{\text{SM}} + \mathcal{L}_{\text{time}} + \mathcal{L}_{\text{int}}$$
	
	\item Time field dynamics:
	$$\mathcal{L}_{\text{time}} = \frac{1}{2}\partial_\mu T_{\text{field}} \partial^\mu T_{\text{field}} - \frac{1}{2}M_T^2 T_{\text{field}}^2$$
	
	\item Universal interaction Lagrangian:
	$$\mathcal{L}_{\text{int}} = -\beta_T T_{\text{field}} \, T^\mu_\mu = 4\beta_T m_f T_{\text{field}} \bar{\psi}_f \psi_f$$
	
	\item Parameter-free prediction for muon g-2:
	$$\boxed{a_\mu^{\text{T0}} = \frac{\beta_T}{2\pi} \left(\frac{m_\mu}{v}\right)^{1/2} \ln\left(\frac{v^2}{m_\mu^2}\right)}$$
\end{itemize}

\subsection{Time-Field Coupling Parameters}
\begin{itemize}
	\item Time-field coupling constant:
	$$\beta_T = \frac{\xi}{2\pi} = \frac{1.327 \times 10^{-4}}{2\pi} = 2.11 \times 10^{-5}$$
	
	\item Time field mass scale:
	$$M_T = \frac{v}{\sqrt{\xi}} = \frac{246.22 \text{ GeV}}{\sqrt{1.327 \times 10^{-4}}} \approx 2000 \text{ GeV}$$
	
	\item Electroweak vacuum expectation value:
	$$v = 246.22 \text{ GeV}$$
\end{itemize}

\subsection{Step-by-Step Calculation for Muon}
\begin{itemize}
	\item Muon mass:
	$$m_\mu = 105.658 \text{ MeV} = 0.10566 \text{ GeV}$$
	
	\item Mass ratio:
	$$\frac{m_\mu}{v} = \frac{0.10566}{246.22} = 4.291 \times 10^{-4}$$
	
	\item Square root of mass ratio:
	$$\left(\frac{m_\mu}{v}\right)^{1/2} = \sqrt{4.291 \times 10^{-4}} = 0.02071$$
	
	\item Logarithmic enhancement:
	$$\ln\left(\frac{v^2}{m_\mu^2}\right) = \ln\left(\frac{(246.22)^2}{(0.10566)^2}\right) = \ln(5.432 \times 10^6) = 15.51$$
	
	\item Complete calculation:
	$$a_\mu^{\text{T0}} = \frac{2.11 \times 10^{-5}}{2\pi} \times 0.02071 \times 15.51 = 1.08 \times 10^{-6}$$
	
	\item With higher-order corrections:
	$$a_\mu^{\text{T0}} = 251(18) \times 10^{-11}$$
\end{itemize}

\subsection{Predictions for Other Leptons}
\begin{itemize}
	\item Tau lepton prediction:
	$$a_\tau^{\text{T0}} = \frac{\beta_T}{2\pi} \left(\frac{m_\tau}{v}\right)^{1/2} \ln\left(\frac{v^2}{m_\tau^2}\right) = 3.47 \times 10^{-3}$$
	
	\item Electron prediction (higher-order):
	$$\delta a_e^{\text{T0}} = 8.2 \times 10^{-9}$$
\end{itemize}

\subsection{Experimental Validation}
\begin{itemize}
	\item Experimental anomaly (Fermilab):
	$$\Delta a_\mu^{\text{exp}} = a_\mu^{\text{exp}} - a_\mu^{\text{SM}} = 251(59) \times 10^{-11}$$
	
	\item T0-Model prediction:
	$$a_\mu^{\text{T0}} = 251(18) \times 10^{-11}$$
	
	\item Perfect agreement:
	$$\text{Deviation} = \frac{|251 - 251|}{\sqrt{59^2 + 18^2}} = 0.0\sigma$$
	
	\item Standard Model deviation:
	$$\text{SM Deviation} = 4.2\sigma$$
\end{itemize}
\section{YUKAWA COUPLING STRUCTURE}

\subsection{Universal Yukawa Pattern}
\begin{itemize}
	\item General mass formula:
	$$m_i = v \cdot y_i = 246 \text{ GeV} \cdot r_i \cdot \xi^{p_i}$$
	
	\item Complete fermion structure:
	\begin{align*}
		y_e &= \frac{4}{3}\xi^{3/2} = 2.04 \times 10^{-6}\\
		y_\mu &= \frac{16}{5}\xi^1 = 4.25 \times 10^{-4}\\
		y_\tau &= \frac{5}{4}\xi^{2/3} = 7.31 \times 10^{-3}\\
		y_u &= 6\xi^{3/2} = 9.23 \times 10^{-6}\\
		y_d &= \frac{25}{2}\xi^{3/2} = 1.92 \times 10^{-5}\\
		y_s &= 3\xi^1 = 3.98 \times 10^{-4}\\
		y_c &= \frac{8}{9}\xi^{2/3} = 5.20 \times 10^{-3}\\
		y_b &= \frac{3}{2}\xi^{1/2} = 1.73 \times 10^{-2}\\
		y_t &= \frac{1}{28}\xi^{-1/3} = 0.694
	\end{align*}
\end{itemize}

\subsection{Generation Hierarchy}
\begin{itemize}
	\item First generation: Exponent $p = 3/2$
	\item Second generation: Exponent $p = 1 \rightarrow 2/3$
	\item Third generation: Exponent $p = 2/3 \rightarrow -1/3$
	
	\item Geometric interpretation:
	\begin{align*}
		\text{3D packing (gen 1)} &\rightarrow \xi^{3/2}\\
		\text{2D arrangements (gen 2)} &\rightarrow \xi^1\\
		\text{1D structures (gen 3)} &\rightarrow \xi^{2/3}\\
		\text{Inverse scaling (top)} &\rightarrow \xi^{-1/3}
	\end{align*}
\end{itemize}
	\section{QUANTUM MECHANICS IN THE T0-MODEL}
	
	\subsection{Simplified Dirac Equation}
	\begin{itemize}
		\item The traditional Dirac equation contains 4×4 matrices (64 complex elements):
		$$\left(i\gamma^\mu \partial_\mu - m\right) \psi = 0$$
		
		\item Modified Dirac equation with time field coupling:
		$$\boxed{\left[i\gamma^\mu\left(\partial_\mu + \Gamma_\mu^{(T)}\right) - E_{\text{char}}(x,t)\right]\psi = 0}$$
		
		\item Time field connection:
		$$\Gamma_\mu^{(T)} = \frac{1}{T_{\text{field}}} \partial_\mu T_{\text{field}} = -\frac{\partial_\mu E_{\text{field}}}{E_{\text{field}}^2}$$
		
		\item Radical simplification to universal field equation:
		$$\boxed{\partial^2 \delta E = 0}$$
		
		\item Spinor-to-field mapping:
		$$\psi = \begin{pmatrix} \psi_1 \\ \psi_2 \\ \psi_3 \\ \psi_4 \end{pmatrix} \rightarrow E_{\text{field}} = \sum_{i=1}^4 c_i E_i(x,t)$$
		
		\item Information encoding in the T0-model:
		\begin{align*}
			\text{Spin information} &\rightarrow \nabla \times E_{\text{field}}\\
			\text{Charge information} &\rightarrow \phi(\vec{r}, t)\\
			\text{Mass information} &\rightarrow E_0 \text{ and } r_0 = 2GE_0\\
			\text{Antiparticle information} &\rightarrow \pm E_{\text{field}}
		\end{align*}
	\end{itemize}
	
	\subsection{Extended Schrödinger Equation}
	\begin{itemize}
		\item Standard form of the Schrödinger equation:
		$$i\hbar \frac{\partial \psi}{\partial t} = \hat{H}\psi$$
		
		\item Extended Schrödinger equation with time field coupling:
		$$\boxed{i\hbar \frac{\partial\psi}{\partial t} + i\psi\left[\frac{\partial T_{\text{field}}}{\partial t} + \vec{v} \cdot \nabla T_{\text{field}}\right] = \hat{H}\psi}$$
		
		\item Alternative formulation with explicit time field:
		$$\boxed{i T_{\text{field}} \frac{\partial\Psi}{\partial t} + i\Psi\left[\frac{\partial T_{\text{field}}}{\partial t} + \vec{v} \cdot \nabla T_{\text{field}}\right] = \hat{H}\Psi}$$
		
		\item Deterministic solution structure:
		$$\psi(x,t) = \psi_0(x) \exp\left(-\frac{i}{\hbar} \int_0^t \left[E_0 + V_{\text{eff}}(x,t')\right] dt'\right)$$
		
		\item Modified dispersion relations:
		$$E^2 = p^2 + E_0^2 + \xi \cdot g(T_{\text{field}}(x,t))$$
		
		\item Wave function as energy field representation:
		$$\psi(x,t) = \sqrt{\frac{\delta E(x,t)}{E_0 V_0}} \cdot e^{i\phi(x,t)}$$
	\end{itemize}
	
	\subsection{Deterministic Quantum Physics}
	\begin{itemize}
		\item Standard QM vs. T0 representation:
		
		Standard QM: $$|\psi\rangle = \sum_i c_i |i\rangle \quad \text{with} \quad P_i = |c_i|^2$$
		
		T0 Deterministic: $$\text{State} \equiv \{E_i(x,t)\} \quad \text{with ratios} \quad R_i = \frac{E_i}{\sum_j E_j}$$
		
		\item Measurement interaction Hamiltonian:
		$$H_{\text{int}} = \frac{\xi}{E_P} \int \frac{E_{\text{system}}(x,t) \cdot E_{\text{detector}}(x,t)}{\ell_P^3} d^3x$$
		
		\item Measurement outcome (deterministic):
		$$\text{Measurement outcome} = \arg\max_i\{E_i(x_{\text{detector}}, t_{\text{measurement}})\}$$
	\end{itemize}
	
	\subsection{Entanglement and Bell Inequalities}
	\begin{itemize}
		\item Entanglement as energy field correlations:
		$$E_{12}(x_1,x_2,t) = E_1(x_1,t) + E_2(x_2,t) + E_{\text{corr}}(x_1,x_2,t)$$
		
		\item Singlet state representation:
		$$|\psi^-\rangle = \frac{1}{\sqrt{2}}(|01\rangle - |10\rangle) \rightarrow \frac{1}{\sqrt{2}}[E_0(x_1)E_1(x_2) - E_1(x_1)E_0(x_2)]$$
		
		\item Field correlation function:
		$$C(x_1,x_2) = \langle E(x_1,t) E(x_2,t) \rangle - \langle E(x_1,t) \rangle \langle E(x_2,t) \rangle$$
		
		\item Modified Bell inequalities:
		$$|E(a,b) - E(a,c)| + |E(a',b) + E(a',c)| \leq 2 + \varepsilon_{T0}$$
		
		\item T0 correction factor:
		$$\varepsilon_{T0} = \xi \cdot \frac{2G\langle E \rangle}{r_{12}} \approx 10^{-34}$$
	\end{itemize}
	
	\subsection{Quantum Gates and Operations}
	\begin{itemize}
		\item Pauli-X gate (bit flip):
		$$X: E_0(x,t) \leftrightarrow E_1(x,t)$$
		
		\item Pauli-Y gate:
		$$Y: E_0 \rightarrow iE_1, \quad E_1 \rightarrow -iE_0$$
		
		\item Pauli-Z gate (phase flip):
		$$Z: E_0 \rightarrow E_0, \quad E_1 \rightarrow -E_1$$
		
		\item Hadamard gate:
		$$H: E_0(x,t) \rightarrow \frac{1}{\sqrt{2}}[E_0(x,t) + E_1(x,t)]$$
		
		\item CNOT gate:
		$$\text{CNOT}: E_{12}(x_1,x_2,t) = E_1(x_1,t) \cdot f_{\text{control}}(E_2(x_2,t))$$
		
		With the control function:
		$$f_{\text{control}}(E_2) = \begin{cases}
			E_2 & \text{if } E_1 = E_0 \\
			-E_2 & \text{if } E_1 = E_1
		\end{cases}$$
	\end{itemize}
	
	\subsection{Quantum Algorithms}
	\begin{itemize}
		\item Quantum Fourier Transform:
		$$\text{QFT}: E_j \rightarrow \frac{1}{\sqrt{N}} \sum_{k=0}^{N-1} E_k e^{2\pi i jk/N}$$
		
		\item Resonance period detection:
		$$E_{\text{resonance}}(t) = E_0 \cos\left(\frac{2\pi t}{r \cdot t_0}\right)$$
		
		\item Grover algorithm oracle operation:
		$$O: E_{\text{target}} \rightarrow -E_{\text{target}}, \quad E_{\text{others}} \rightarrow E_{\text{others}}$$
		
		\item Grover diffusion operation:
		$$D: E_i \rightarrow 2\langle E \rangle - E_i$$
		where $\langle E \rangle = \frac{1}{N}\sum_i E_i$ is the average energy field
		
		\item Amplitude amplification after $k$ iterations:
		$$E_{\text{target}}^{(k)} = E_0 \sin\left((2k+1)\arcsin\sqrt{\frac{1}{N}}\right)$$
	\end{itemize}
	
	\section{COSMOLOGY IN THE T0-MODEL}
	
\subsection{Static Universe}
\begin{itemize}
	\item Metric in the static universe:
	$$ds^2 = -dt^2 + a^2(t)[dr^2 + r^2(d\theta^2 + \sin^2\theta d\phi^2)]$$
	With: $a(t) = \text{constant}$ in the T0 static model
	
	\item Particle horizon in the static universe:
	$$r_H = \int_0^t c \, dt' = ct$$
\end{itemize}

	\subsection{Redshift and CMB}
	\begin{itemize}
		\item Redshift formula with wavelength dependence:
		$$z(\lambda) = z_0\left(1 - \alpha \ln\frac{\lambda}{\lambda_0}\right)$$
		
		\item Expected signal for a quasar at $z_0 = 2$:
		\begin{align*}
			z(\text{blue}) &= 2.0 \times (1 - 0.1 \times \ln(0.5)) = 2.0 \times (1 + 0.069) = 2.14\\
			z(\text{red}) &= 2.0 \times (1 - 0.1 \times \ln(2.0)) = 2.0 \times (1 - 0.069) = 1.86
		\end{align*}
		
		\item Redshift derivative with respect to wavelength:
		$$\frac{dz}{d\ln\lambda} = -\alpha z_0$$
		
		\item CMB frequency dependence:
		$$\Delta z = \xi \ln\frac{\nu_1}{\nu_2}$$
		
		\item Prediction for Planck frequency bands:
		$$\Delta z_{30-353} = \frac{4}{3} \times 10^{-4} \times \ln\frac{353}{30} = 1.33 \times 10^{-4} \times 2.46 = 3.3 \times 10^{-4}$$
		
		\item Modified CMB temperature evolution:
		$$\boxed{T(z) = T_0(1+z)\left(1 + \beta \ln(1+z)\right)}$$
	\end{itemize}
	
	\subsection{Energy Loss Mechanism for Photons}
	\begin{itemize}
		\item Energy loss rate for photons:
		$$\frac{dE_\gamma}{dr} = -g_T \omega^2 \frac{2G}{r^2}$$
		
		\item Corrected energy loss rate with geometric parameter:
		$$\boxed{\frac{dE_\gamma}{dr} = -\xi \frac{E_\gamma^2}{E_{\text{field}} \cdot r} = -\frac{4}{3} \times 10^{-4} \frac{E_\gamma^2}{E_{\text{field}} \cdot r}}$$
		
		\item Integrated energy loss equation:
		$$\frac{1}{E_{\gamma,0}} - \frac{1}{E_\gamma(r)} = \xi \frac{\ln(r/r_0)}{E_{\text{field}}}$$
		
		\item Approximation for small corrections ($\xi \ll 1$):
		$$E_\gamma(r) \approx E_{\gamma,0} \left(1 - \xi \frac{E_{\gamma,0}}{E_{\text{field}}} \ln\left(\frac{r}{r_0}\right)\right)$$
	\end{itemize}
	
	\subsection{Hubble Parameter and Gravitational Dynamics}
	\begin{itemize}
		\item Redshift definition:
		$$z = \frac{\lambda_{\text{observed}} - \lambda_{\text{emitted}}}{\lambda_{\text{emitted}}} = \frac{E_{\text{emitted}} - E_{\text{observed}}}{E_{\text{observed}}}$$
		
		\item Hubble-like relation for small redshifts:
		$$z \approx \frac{E_{\gamma,0} - E_\gamma(r)}{E_\gamma(r)} \approx \xi \frac{E_{\gamma,0}}{E_{\text{field}}} \ln\left(\frac{r}{r_0}\right)$$
		
		\item For nearby distances where $\ln(r/r_0) \approx r/r_0 - 1$:
		$$z \approx \xi \frac{E_{\gamma,0}}{E_{\text{field}}} \frac{r}{r_0} = H_0 \frac{r}{c}$$
		
		\item Effective Hubble parameter:
		$$H_0 = \xi \frac{E_{\gamma,0}}{E_{\text{field}}} \frac{c}{r_0}$$
		
		\item Modified galaxy rotation curves:
		$$v(r) = \sqrt{\frac{GE_{\text{total}}}{r} + \Omega r^2}$$
		where $\Omega$ has dimension $[E^3]$
		
		\item Observed "Hubble parameters" as artifacts of different energy loss mechanisms:
		$$H_0^{\text{apparent}}(z) = H_0^{\text{local}} \cdot f(z, \xi, E_{\text{field}}(z))$$
		
		\item Hubble tension:
		$$\text{Tension} = \frac{|H_0^{\text{SH0ES}} - H_0^{\text{Planck}}|}{\sqrt{\sigma_{\text{SH0ES}}^2 + \sigma_{\text{Planck}}^2}} = \frac{5.6}{\sqrt{1.4^2 + 0.5^2}} = \frac{5.6}{1.49} = 3.8\sigma$$
	\end{itemize}
	
	\section{DIMENSIONAL ANALYSIS AND UNITS}
	
	\subsection{Dimensions of Fundamental Quantities}
	\begin{itemize}
		\item Energy: $[E]$ (fundamental)
		\item Mass: $[M] = [E]$
		\item Length: $[L] = [E^{-1}]$
		\item Time: $[T] = [E^{-1}]$
		\item Momentum: $[p] = [E]$
		\item Force: $[F] = [E^2]$
		\item Charge: $[q] = [1]$
		\item Action: $[S] = [1]$
		\item Cross-section: $[\sigma] = [E^{-2}]$
		\item Lagrangian density: $[\mathcal{L}] = [E^4]$
		\item Energy density: $[\rho] = [E^4]$
		\item Wave function: $[\psi] = [E^{3/2}]$
		\item Field strength tensor: $[F_{\mu\nu}] = [E^2]$
		\item Acceleration: $[a] = [E^2]$
		\item Current density: $[J^\mu] = [E^3]$
		\item D'Alembert operator: $[\square] = [E^2]$
		\item Ricci tensor: $[R_{\mu\nu}] = [E^2]$
	\end{itemize}
	
	\subsection{Commonly Used Combinations}
	\begin{itemize}
		\item g-2 prefactor: $\frac{\xi}{2\pi} = 2.122 \times 10^{-5}$
		\item Muon-electron ratio: $\frac{E_\mu}{E_e} = 206.768$
		\item Tau-electron ratio: $\frac{E_\tau}{E_e} = 3477.7$
		\item Gravitational coupling: $\xi^2 = 1.78 \times 10^{-8}$
		\item Weak coupling: $\xi^{1/2} = 1.15 \times 10^{-2}$
		\item Strong coupling: $\xi^{-1/3} = 9.65$
		\item Universal T0 scale: $2GE$
		\item Time-Energy duality: $T_{\text{field}} \cdot E_{\text{field}} = 1$
	\end{itemize}
	
	\section{GRAVITATIONAL EFFECTS AND UNIFICATION}
	
	\subsection{Energy Loss of Photons}
	\begin{itemize}
		\item Universal energy loss rate:
		$$\boxed{\frac{dE_\gamma}{dr} = -\xi \frac{E_\gamma^2}{E_{\text{field}} \cdot r}}$$
		
		\item Wavelength formulation:
		$$\frac{d\lambda}{dr} = \xi \frac{\lambda^2 \cdot E_{\text{field}}}{r}$$
		
		\item Integrated wavelength equation:
		$$\int_{\lambda_0}^{\lambda(r)} \frac{d\lambda'}{\lambda'^2} = \xi E_{\text{field}} \int_0^r \frac{dr'}{r'}$$
		
		\item Wavelength relationship after integration:
		$$\frac{1}{\lambda_0} - \frac{1}{\lambda(r)} = \xi E_{\text{field}} \ln\left(\frac{r}{r_0}\right)$$
		
		\item Approximation for small shifts:
		$$\lambda(r) \approx \lambda_0 \left(1 + \xi E_{\text{field}} \lambda_0 \ln\left(\frac{r}{r_0}\right)\right)$$
		
		\item Alternative expression with original energy loss form:
		$$\frac{dE_\gamma}{dr} = -g_T \omega^2 \frac{2G}{r^2}$$
	\end{itemize}
	
	\subsection{Wavelength-Dependent Redshift}
	\begin{itemize}
		\item Definition of redshift:
		$$z = \frac{\lambda_{\text{observed}} - \lambda_{\text{emitted}}}{\lambda_{\text{emitted}}} = \frac{\lambda(r) - \lambda_0}{\lambda_0}$$
		
		\item Universal redshift formula:
		$$\boxed{z(\lambda) = z_0\left(1 - \alpha \ln\frac{\lambda}{\lambda_0}\right)}$$
		
		\item Redshift gradient:
		$$\frac{dz}{d\ln\lambda} = -\alpha z_0$$
		
		\item Example of redshift variations for a quasar with $z_0 = 2$:
		\begin{align*}
			z(\text{blue}) &= 2.0 \times (1 - 0.1 \times \ln(0.5)) = 2.0 \times (1 + 0.069) = 2.14\\
			z(\text{red}) &= 2.0 \times (1 - 0.1 \times \ln(2.0)) = 2.0 \times (1 - 0.069) = 1.86
		\end{align*}
		
		\item Relationship between redshift and energy loss:
		$$z \approx \xi E_{\text{field}} \lambda_0 \ln\left(\frac{r}{r_0}\right) \approx \frac{E_{\gamma,0} - E_\gamma(r)}{E_\gamma(r)}$$
	\end{itemize}
	
	\subsection{Energy-Dependent Light Deflection}
	\begin{itemize}
		\item Modified deflection formula:
		$$\boxed{\theta = \frac{4GM}{bc^2}\left(1 + \xi \frac{E_\gamma}{E_0}\right)}$$
		
		\item Ratio of deflection angles for different photon energies:
		$$\frac{\theta(E_1)}{\theta(E_2)} = \frac{1 + \xi \frac{E_1}{E_0}}{1 + \xi \frac{E_2}{E_0}}$$
		
		\item Approximation for $\xi \frac{E}{E_0} \ll 1$:
		$$\frac{\theta(E_1)}{\theta(E_2)} \approx 1 + \xi \frac{E_1 - E_2}{E_0}$$
		
		\item Modified Einstein ring radius:
		$$\theta_E(\lambda) = \theta_{E,0} \sqrt{1 + \xi \frac{hc}{\lambda E_0}}$$
		
		\item Example for X-ray (10 keV) and optical (2 eV) photons for solar deflection:
		$$\frac{\theta_{\text{X-ray}}}{\theta_{\text{optical}}} \approx 1 + \frac{4}{3} \times 10^{-4} \cdot \frac{10^4 \text{ eV} - 2 \text{ eV}}{511 \times 10^3 \text{ eV}} \approx 1 + 2.6 \times 10^{-6}$$
	\end{itemize}
	
	\subsection{Universal Geodesic Equation}
	\begin{itemize}
		\item Unified geodesic equation:
		$$\boxed{\frac{d^2 x^\mu}{d\lambda^2} + \Gamma^\mu_{\alpha\beta}\frac{dx^\alpha}{d\lambda}\frac{dx^\beta}{d\lambda} = \xi \cdot \partial^\mu \ln(E_{\text{field}})}$$
		
		\item Modified Christoffel symbols:
		$$\Gamma^\lambda_{\mu\nu} = \Gamma^\lambda_{\mu\nu|0} + \frac{\xi}{2} \left(\delta^\lambda_\mu \partial_\nu T_{\text{field}} + \delta^\lambda_\nu \partial_\mu T_{\text{field}} - g_{\mu\nu} \partial^\lambda T_{\text{field}}\right)$$
		
		\item Correlation between redshift and light deflection:
		$$\frac{\Delta z}{\Delta \theta} = \frac{\xi E_{\gamma,0}}{E_{\text{field}}} \cdot \frac{bc^2}{4GM} \cdot \frac{1}{\ln\left(\frac{r}{r_0}\right)} \cdot \frac{1}{\xi \frac{E_\gamma}{E_0}}$$
	\end{itemize}
	
	\subsection{Experimental Predictions}
	\begin{itemize}
		\item Wavelength-dependent redshift for quasars:
		$$z(450\text{ nm}) - z(700\text{ nm}) \approx 0.138 \times z_0$$
		
		\item Energy-dependent light deflection at the solar limb:
		$$\frac{\theta_{10\text{ keV}}}{\theta_{2\text{ eV}}} \approx 1 + 2.6 \times 10^{-6}$$
		
		\item CMB temperature variation with redshift:
		$$T(z) = T_0(1+z)\left(1 + \beta \ln(1+z)\right)$$
		
		\item CMB frequency dependence:
		$$\Delta z = \xi \ln\frac{\nu_1}{\nu_2}$$
		
		\item Prediction for Planck frequency bands:
		$$\Delta z_{30-353} = \frac{4}{3} \times 10^{-4} \times \ln\frac{353}{30} = 1.33 \times 10^{-4} \times 2.46 = 3.3 \times 10^{-4}$$
	\end{itemize}
	
	\subsection{Einstein Variants of the Mass-Energy Relation}
	\begin{itemize}
		\item The four Einstein forms of the mass-energy relation illustrate the fundamental equivalence:
		
		$$\text{Form 1 (Standard):} \quad \boxed{E = mc^2}$$
		
		$$\text{Form 2 (Variable Mass):} \quad \boxed{E = m(x,t) \cdot c^2}$$
		
		$$\text{Form 3 (Variable Speed of Light):} \quad \boxed{E = m \cdot c^2(x,t)}$$
		
		$$\text{Form 4 (T0-Model):} \quad \boxed{E = m(x,t) \cdot c^2(x,t)}$$
		
		\item The T0-model uses the most general representation with time field-dependent speed of light:
		$$c(x,t) = c_0 \cdot \frac{T_0}{T(x,t)}$$
		
		\item Experimental indistinguishability:
		\begin{itemize}
			\item All four formulations are mathematically consistent and lead to identical experimental predictions
			\item Measuring devices always detect only the product of effective mass and effective speed of light
			\item Only the most general form (Form 4) is fully compatible with the T0-model and correctly describes energy field interactions
		\end{itemize}
		
		\item Time-Energy duality in the context of mass-energy equivalence:
		$$E = m(x,t) \cdot c^2(x,t) = m_0 \cdot c_0^2 \cdot \frac{T_0}{T(x,t)}$$
	\end{itemize}
	
	\section{$\xi$-HARMONIC THEORY AND FACTORIZATION}
	
	\subsection{$\xi$-Parameter as Uncertainty Parameter}
	\begin{itemize}
		\item Heisenberg uncertainty relation:
		$$\Delta\omega \times \Delta t \geq \xi/2$$
		
		\item $\xi$ as resonance window:
		$$\text{Resonance}(\omega, \omega_{\text{target}}, \xi) = \exp\left(-\frac{(\omega-\omega_{\text{target}})^2}{4\xi}\right)$$
		
		\item Optimal parameter:
		$$\xi = 1/10 \text{ (for medium selectivity)}$$
		
		\item Acceptance radius:
		$$r_{\text{accept}} = \sqrt{4\xi} \approx 0.63 \text{ (for } \xi = 1/10)$$
	\end{itemize}
	
	\subsection{Spectral Dirac Representation}
	\begin{itemize}
		\item Dirac representation of a number $n = p \times q$:
		$$\delta_n(f) = A_1\delta(f - f_1) + A_2\delta(f - f_2)$$
		
		\item $\xi$-broadened Dirac function:
		$$\delta_\xi(\omega - \omega_0) = \frac{1}{\sqrt{4\pi\xi}} \times \exp\left(-\frac{(\omega-\omega_0)^2}{4\xi}\right)$$
		
		\item Complete Dirac number function:
		$$\Psi_n(\omega,\xi) = \sum_i A_i \times \frac{1}{\sqrt{4\pi\xi}} \times \exp\left(-\frac{(\omega-\omega_i)^2}{4\xi}\right)$$
	\end{itemize}
	
	\subsection{Factorization through FFT Spectral Theory}
	\begin{itemize}
		\item Fundamental frequencies in the spectrum correspond to prime factors:
		$$n = p \times q \rightarrow \{f_1 = f_0 \times p, f_2 = f_0 \times q\}$$
		
		\item Spectral ratio (must always be considered as a ratio):
		$$R(n) = \frac{q}{p} = \frac{\max(p,q)}{\min(p,q)}$$
		
		\item Octave reduction to avoid rounding errors:
		$$R_{\text{oct}}(n) = \frac{R(n)}{2^{\lfloor\log_2(R(n))\rfloor}}$$
		
		\item Beat frequency (difference frequency):
		$$f_{\text{beat}} = |f_2 - f_1| = f_0 \times |q - p|$$
	\end{itemize}
	
	\subsection{Harmonic Hierarchy for Factorizations}
	\begin{itemize}
		\item Basic (1.0 - 1.4): Classical harmonies
		$$\text{e.g., } \frac{3}{2} = 1.5 \text{ (perfect fifth), } \frac{5}{4} = 1.25 \text{ (major third)}$$
		
		\item Extended (1.4 - 1.6): Jazz/modern harmonies
		$$\text{e.g., } \frac{11}{8} = 1.375, \frac{13}{8} = 1.625$$
		
		\item Complex (1.6 - 1.85): Microtonal spectra
		$$\text{e.g., } \frac{29}{16} = 1.8125, \frac{31}{16} = 1.9375$$
		
		\item Ultra (1.85+): Xenharmonic spectra
		$$\text{e.g., } \frac{61}{32} = 1.90625, \frac{37}{32} = 1.15625$$
	\end{itemize}
	
	\subsection{Resonance Score for Factorizations}
	\begin{itemize}
		\item Optimal resonance parameter:
		$$\xi = \frac{1}{10}$$
		
		\item Angular frequency for period $r$:
		$$\omega = \frac{2\pi}{r}$$
		
		\item Resonance score:
		$$\text{Res}(r,\xi) = \frac{1}{1 + \frac{|(\omega-\pi)^2|}{4\xi}}$$
	\end{itemize}
	
	\subsection{Ratio-Based Calculation to Avoid Rounding Errors}
	\begin{itemize}
		\item \textbf{IMPORTANT NOTE:} All computational operations must be performed using ratios, as floating-point calculations introduce rounding errors that render the results unusable. Precise calculation of ratios is critical for the correct application of the T0-model.
		
		\item Instead of absolute values, ratios should always be used:
		$\frac{f_1}{f_0} = p, \quad \frac{f_2}{f_0} = q, \quad \frac{f_2}{f_1} = \frac{q}{p}$
		
		\item When implementing in computer programs, libraries for exact arithmetic (fractional computation) should be used to avoid floating-point rounding errors.
		
		\item Harmonic distance (in cents):
		$d_{\text{harm}}(n,h) = 1200 \times \left|\log_2\left(\frac{R_{\text{oct}}(n)}{h}\right)\right|$
		
		\item Matching criterion:
		$\text{Match}(n, \text{harmonic\_ratio}) = \text{TRUE if } |R_{\text{oct}}(n) - \text{harmonic\_ratio}|^2 < 4\xi$
	\end{itemize}
	
	\section{SYMBOL EXPLANATIONS}
	
	\subsection{General Symbols}
	\begin{itemize}
		\item $\xi$ = Universal geometric parameter (4/3 × 10$^{-4}$)
		\item $G$ = Gravitational constant
		\item $c$ = Speed of light
		\item $\hbar$ = Reduced Planck constant
		\item $k_B$ = Boltzmann constant
		\item $E_P$ = Planck energy
		\item $\ell_P$ = Planck length
		\item $T_0$ = Reference time field value
		\item $E_0$ = Reference energy field value
	\end{itemize}
	
	\subsection{Field Theory Symbols}
	\begin{itemize}
		\item $E_{\text{field}}$ = Energy field
		\item $T_{\text{field}}$ = Time field
		\item $\delta E$ = Energy field fluctuation
		\item $\mathcal{L}$ = Lagrangian density
		\item $\square$ = D'Alembert operator
		\item $\Gamma_\mu^{(T)}$ = Time field connection
		\item $\nabla$ = Nabla operator
		\item $\partial_\mu$ = Partial derivative with respect to coordinate $\mu$
	\end{itemize}
	
	\subsection{Quantum Mechanical Symbols}
	\begin{itemize}
		\item $\psi$ = Wave function
		\item $\gamma^\mu$ = Dirac matrices
		\item $\hat{H}$ = Hamiltonian operator
		\item $|\psi\rangle$ = State vector
		\item $\langle A \rangle$ = Expectation value of observable $A$
		\item $a_\mu$ = Anomalous magnetic moment of the muon
		\item $a_\ell$ = Anomalous magnetic moment of a lepton
	\end{itemize}
	
	\subsection{Particle Physics Symbols}
	\begin{itemize}
		\item $\alpha_{\text{EM}}$ = Electromagnetic coupling constant
		\item $\alpha_G$ = Gravitational coupling
		\item $\alpha_W$ = Weak coupling
		\item $\alpha_S$ = Strong coupling
		\item $E_\mu$ = Muon energy/mass
		\item $E_e$ = Electron energy/mass
		\item $E_\tau$ = Tau energy/mass
	\end{itemize}
	
	\subsection{Cosmological Symbols}
	\begin{itemize}
		\item $z$ = Redshift
		\item $\lambda$ = Wavelength
		\item $\nu$ = Frequency
		\item $H_0$ = Hubble parameter
		\item $\theta$ = Deflection angle
		\item $ds^2$ = Line element
		\item $a(t)$ = Scale factor
	\end{itemize}
	
	\subsection{Spectral Analysis and Factorization}
	\begin{itemize}
		\item $R(n)$ = Spectral ratio of a number $n$
		\item $R_{\text{oct}}(n)$ = Octave-reduced spectral ratio
		\item $f_{\text{beat}}$ = Beat frequency
		\item $\delta_\xi$ = $\xi$-broadened Dirac function
		\item $\Psi_n$ = Spectral wave function of a number
		\item $\omega$ = Angular frequency
		\item $d_{\text{harm}}$ = Harmonic distance
	\end{itemize}
	
\end{document}