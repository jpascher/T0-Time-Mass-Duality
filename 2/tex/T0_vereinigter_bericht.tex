\documentclass[11pt,a4paper]{article}
\usepackage[utf8]{inputenc}
\usepackage[ngerman]{babel}
\usepackage{amsmath}
\usepackage{amsfonts}
\usepackage{amssymb}
\usepackage{booktabs}
\usepackage{longtable}
\usepackage{geometry}
\usepackage{siunitx}
\usepackage{hyperref}
\geometry{margin=2cm}

\title{T0-Theorie: Vereinigter Rechner Ergebnisse\\
\large Massen und physikalische Konstanten aus geometrischen Prinzipien}
\author{Johann Pascher\\HTL Leonding, Österreich\\
\texttt{Automatisch generiert vom T0-Vereinigten Rechner v3.0}}
\date{\today}

\begin{document}
\maketitle

\tableofcontents
\newpage

\section{Einführung}

Die T0-Theorie stellt einen revolutionären Ansatz dar, bei dem alle physikalischen Konstanten und Teilchenmassen aus nur drei fundamentalen geometrischen Parametern abgeleitet werden. Diese Arbeit präsentiert die vollständigen Ergebnisse des vereinigten T0-Rechners.

\section{Fundamentale Eingabeparameter}

Die gesamte T0-Theorie basiert auf nur drei Eingabewerten:

\begin{align}
\xi &= \frac{4}{3} \times 10^{-4} \approx 1.33333333e-04 \text{ (geometrische Konstante)} \\
\ell_\text{P} &= 1.616000e-35 \text{ m (Planck-Länge)} \\
E_0 &= 7.398 \text{ MeV (charakteristische Energie)} \\
v &= 246.0 \text{ GeV (Higgs-VEV, aus } \xi \text{ abgeleitet)}
\end{align}

\subsection{Geometrische Herleitung von $\xi$}

Die geometrische Konstante $\xi$ entsteht aus der fundamentalen Feldgleichung:
\begin{equation}
\nabla^2 m(x,t) = 4\pi G \rho(x,t) \cdot m(x,t)
\end{equation}

Für eine sphärisch-symmetrische Punktmasse führt dies zur charakteristischen Länge:
\begin{equation}
r_0 = 2Gm \quad \text{und} \quad \xi = \frac{r_0}{\ell_\text{P}}
\end{equation}

\section{Teilchen-Massenberechnungen}

Die T0-Theorie berechnet alle Teilchenmassen über die Yukawa-Methode:
\begin{equation}
m = r \times \xi^p \times v
\end{equation}

wobei $r$ und $p$ teilchenspezifische Parameter aus der geometrischen Struktur sind.

\begin{longtable}{lccccc}
\caption{T0-Massenvorhersagen mit exakten Bruchparametern} \\
\toprule
Teilchen & $r$ & $p$ & T0-Masse [\si{\mega\electronvolt}] & Exp. Masse [\si{\mega\electronvolt}] & Fehler [\%] \\
\midrule
\endfirsthead
\multicolumn{6}{c}{\tablename\ \thetable\ -- Fortsetzung von vorheriger Seite} \\
\toprule
Teilchen & $r$ & $p$ & T0-Masse [\si{\mega\electronvolt}] & Exp. Masse [\si{\mega\electronvolt}] & Fehler [\%] \\
\midrule
\endhead
\bottomrule
\multicolumn{6}{r}{Fortsetzung auf nächster Seite} \\
\endfoot
\bottomrule
\endlastfoot
Elektron & $\frac{4}{3}$ & $\frac{3}{2}$ & 0.5 & 0.5 & 1.18 \\
Myon & $\frac{16}{5}$ & $1$ & 105.0 & 105.7 & 0.66 \\
Tau & $\frac{8}{3}$ & $\frac{2}{3}$ & 1712.1 & 1776.9 & 3.64 \\
Up & $6$ & $\frac{3}{2}$ & 2.3 & 2.3 & 0.11 \\
Down & $\frac{25}{2}$ & $\frac{3}{2}$ & 4.7 & 4.7 & 0.30 \\
Strange & $\frac{26}{9}$ & $1$ & 94.8 & 93.4 & 1.45 \\
Charm & $2$ & $\frac{2}{3}$ & 1284.1 & 1270.0 & 1.11 \\
Bottom & $\frac{3}{2}$ & $\frac{1}{2}$ & 4260.8 & 4180.0 & 1.93 \\
Top & $\frac{1}{28}$ & $\frac{-1}{3}$ & 171974.5 & 172760.0 & 0.45 \\
\end{longtable}

\subsection{Statistische Analyse der Massenergebnisse}

Die T0-Theorie erreicht eine bemerkenswerte Genauigkeit bei der Vorhersage von Teilchenmassen:

\begin{itemize}
\item Anzahl berechneter Teilchen: 9
\item Durchschnittlicher Fehler: 1.20\%
\item Beste Vorhersage: up (0.11\% Fehler)
\item Alle Massen aus nur 3 Parametern berechnet
\end{itemize}

\section{Physikalische Konstanten}

Die T0-Theorie leitet systematisch alle fundamentalen physikalischen Konstanten in einer 8-stufigen Hierarchie ab:

\subsection{Level 1: Primäre Ableitungen}
\begin{align}
\alpha &= \xi \left(\frac{E_0}{1 \text{ MeV}}\right)^2 = 7.297387e-03 \\
m_{\text{char}} &= \frac{\xi}{2} = 6.666667e-05
\end{align}

\subsection{Level 2: Gravitationskonstante}

Die Gravitationskonstante wird direkt aus $\xi$ abgeleitet:
\begin{align}
G_{\text{nat}} &= \frac{\xi^2}{4 m_{\text{char}}} = \frac{\xi}{2} = 6.666667e-05 \text{ (dimensionslos)} \\
G &= G_{\text{nat}} \times \frac{\ell_\text{P}^2 c^3}{\hbar} = 6.672194e-11 \text{ \si{\cubic\meter\per\kilogram\per\second\squared}}
\end{align}

\subsection{Übersicht aller berechneten Konstanten}

\begin{longtable}{p{1.5cm}p{3cm}S[table-format=1.6e2]S[table-format=1.6e2]S[table-format=2.4]}
\caption{T0-Konstantenberechnungen nach Hierarchie-Level} \\
\toprule
{Level} & {Konstante} & {T0-Wert} & {Referenzwert} & {Fehler [\%]} \\
\midrule
\endfirsthead
\multicolumn{5}{c}{\tablename\ \thetable\ -- Fortsetzung von vorheriger Seite} \\
\toprule
{Level} & {Konstante} & {T0-Wert} & {Referenzwert} & {Fehler [\%]} \\
\midrule
\endhead
\bottomrule
\multicolumn{5}{r}{Fortsetzung auf nächster Seite} \\
\endfoot
\bottomrule
\endlastfoot
1 & $\alpha$ & 7.297387e-03 & 7.297353e-03 & 0.0005 \\
1 & $m_{\text{char}}$ & 6.666667e-05 & {T0-abgeleitet} & {-} \\
2 & $G$ & 6.672194e-11 & 6.674300e-11 & 0.0316 \\
2 & $G_{\text{nat}}$ & 6.666667e-05 & {T0-abgeleitet} & {-} \\
2 & $G_{\text{umrechnungsfaktor}}$ & 6.672194e-11 & {T0-abgeleitet} & {-} \\
3 & $c$ & 2.997925e+08 & 2.997925e+08 & 0.0000 \\
3 & $\hbar$ & 1.054572e-34 & 1.054572e-34 & 0.0000 \\
3 & $m_{\text{P}}$ & 2.176778e-08 & 2.176434e-08 & 0.0158 \\
3 & $t_{\text{P}}$ & 5.390396e-44 & 5.391247e-44 & 0.0158 \\
3 & $T_{\text{P}}$ & 1.417008e+32 & 1.416784e+32 & 0.0158 \\
3 & $E_{\text{P}}$ & 1.956390e+09 & 1.956082e+09 & 0.0158 \\
3 & $F_{\text{P}}$ & 1.210638e+44 & 1.210256e+44 & 0.0315 \\
3 & $P_{\text{P}}$ & 3.629400e+52 & 3.628255e+52 & 0.0316 \\
4 & $\mu_0$ & 1.256637e-06 & 1.256637e-06 & 0.0000 \\
4 & $\epsilon_0$ & 8.854188e-12 & 8.854188e-12 & 0.0000 \\
4 & $e$ & 1.602180e-19 & 1.602177e-19 & 0.0002 \\
4 & $Z_0$ & 3.767303e+02 & 3.767303e+02 & 0.0000 \\
4 & $k_{\text{e}}$ & 8.987552e+09 & 8.987552e+09 & 0.0000 \\
5 & $\sigma_{\text{SB}}$ & 5.670374e-08 & 5.670374e-08 & 0.0000 \\
5 & $b_{\text{Wien}}$ & 2.897839e-03 & 2.897772e-03 & 0.0023 \\
5 & $h$ & 6.626070e-34 & 6.626070e-34 & 0.0000 \\
6 & $a_0$ & 5.291747e-11 & 5.291772e-11 & 0.0005 \\
6 & $R_{\infty}$ & 1.097384e+07 & 1.097373e+07 & 0.0009 \\
6 & $\mu_{\text{B}}$ & 9.274032e-24 & 9.274010e-24 & 0.0002 \\
6 & $\mu_{\text{N}}$ & 5.050796e-27 & 5.050784e-27 & 0.0002 \\
6 & $E_{\text{h}}$ & 4.359786e-18 & 4.359745e-18 & 0.0009 \\
6 & $\lambda_{\text{C}}$ & 2.426310e-12 & 2.426310e-12 & 0.0000 \\
6 & $r_{\text{e}}$ & 2.817954e-15 & 2.817940e-15 & 0.0005 \\
7 & $F$ & 9.648556e+04 & 9.648533e+04 & 0.0002 \\
7 & $R_{\text{K}}$ & 2.581268e+04 & 2.581281e+04 & 0.0005 \\
7 & $K_{\text{J}}$ & 4.835990e+14 & 4.835978e+14 & 0.0002 \\
7 & $\Phi_0$ & 2.067829e-15 & 2.067834e-15 & 0.0002 \\
7 & $R_{\text{gas}}$ & 8.314463e+00 & 8.314463e+00 & 0.0000 \\
8 & $H_0$ & 2.196000e-18 & {T0-abgeleitet} & {-} \\
8 & $\Lambda$ & 1.609698e-52 & {T0-abgeleitet} & {-} \\
8 & $t_{\text{universum}}$ & 4.553734e+17 & {T0-abgeleitet} & {-} \\
8 & $\rho_{\text{krit}}$ & 8.627350e-27 & {T0-abgeleitet} & {-} \\
8 & $l_{\text{Hubble}}$ & 1.365175e+26 & {T0-abgeleitet} & {-} \\
\end{longtable}

\section{Zusammenfassung}

\subsection{Schlüsselergebnisse}

Die T0-Theorie erreicht eine bemerkenswerte Vereinigung der Physik:

\begin{enumerate}
\item \textbf{Vollständige Massenberechnung}: Alle 9 Teilchenmassen aus geometrischen Prinzipien
\item \textbf{Konstanten-Hierarchie}: 39 physikalische Konstanten in 8 Stufen abgeleitet
\item \textbf{Hohe Präzision}: Durchschnittlicher Massenfehler nur 1.2 \%
\item \textbf{Minimaler Input}: Nur 3 fundamentale Parameter erforderlich
\item \textbf{Open Source}: Alle Dokumente und Quellcodes sind verfügbar auf \url{https://github.com/jpascher/T0-Time-Mass-Duality} unter der MIT-Lizenz.
\end{enumerate}


\section{Schlussfolgerung}

Der T0-Vereinigte Rechner zeigt, dass geometrische Prinzipien zu erstaunlich präzisen Vorhersagen in der Teilchenphysik führen können. Die numerische Genauigkeit verdient wissenschaftliche Aufmerksamkeit.

\vfill
\begin{center}
\textit{Generiert am \today\ mit dem T0-Vereinigten Rechner v3.0}\\
\textit{Johann Pascher, HTL Leonding, Österreich}
\end{center}

\end{document}
