\documentclass[12pt,a4paper]{article}
\usepackage[utf8]{inputenc}
\usepackage[T1]{fontenc}
\usepackage{geometry}
\usepackage{lmodern}
\usepackage{amsmath}
\usepackage{amssymb}
\usepackage{amsfonts} % Für bessere Mathe-Symbole
\usepackage{hyperref}
\usepackage{booktabs}
\usepackage{enumitem}
\usepackage[table,xcdraw]{xcolor}
\usepackage{newunicodechar}

% Unicode setups for Greek letters and symbols (falls benötigt, aber wir verwenden LaTeX-Makros)
\newunicodechar{ξ}{\ensuremath{\xi}}
\newunicodechar{μ}{\ensuremath{\mu}}
\newunicodechar{π}{\ensuremath{\pi}}

\geometry{left=2cm,right=2cm,top=2cm,bottom=2cm}

\hypersetup{
	colorlinks=true,
	linkcolor=blue,
	citecolor=blue,
	urlcolor=blue,
	pdftitle={The Electron Unit Charge in T0 Theory: Beyond Point Singularities},
	pdfauthor={Johann Pascher},
	pdfsubject={T0 Theory, Electron Charge, Singularities, Electrodynamics}
}

\title{The Electron Unit Charge in T0 Theory:\\Beyond Point Singularities}
\author{Johann Pascher\\
	Department of Communications Engineering\\
	Higher Technical Institute Leonding, Austria\\
	\texttt{johann.pascher@gmail.com}}
\date{October 21, 2025}

\begin{document}
	
	\maketitle
	
	\begin{abstract}
		The classical representation of the electron unit charge as a point singularity encounters fundamental issues in quantum electrodynamics (QED), such as infinite self-energy and ultraviolet divergences. This treatise, authored as the creator of T0 Theory (Time-Mass Duality Framework), demonstrates how T0 resolves these singularities by treating charge as an emergent, geometric property of a universal field. Based on the single parameter $\xi = \frac{4}{3} \times 10^{-4}$ and the Time-Mass Duality $T_{\text{field}} \cdot E_{\text{field}} = 1$, the charge is derived as a fractal pattern of quantized scales (fractal dimension $D_f \approx 2.94$). This avoids infinities, explains observations like the fine-structure constant $\alpha \approx 1/137$, and seamlessly connects to kinematic models in Electromagnetic Mechanics. The GitHub documentation for T0 Theory (current as of October 21, 2025) serves as a reference for detailed derivations.
	\end{abstract}
	
	\tableofcontents
	
	\section{Introduction: The Problem of Point Singularities}
	\label{sec:intro}
	
	In standard physics, the electron unit charge $-e \approx -1.602 \times 10^{-19}$ C is modeled as a Dirac delta function $\rho(\mathbf{r}) = -e \delta(\mathbf{r})$. This leads to a Coulomb field $E(\mathbf{r}) \propto 1/r^2$ and infinite electrostatic self-energy:
	\begin{equation}
		U = \frac{1}{2} \int \epsilon_0 E^2 \, dV \to \infty \quad \text{(as $r \to 0$)}.
	\end{equation}
	
	QED addresses this through renormalization (vacuum polarization), yet the bare point singularity remains a mathematical artifact. Experimentally, the electron appears point-like (to $< 10^{-22}$ m), but this does not preclude extended models at deeper scales. T0 Theory, which I developed as its creator, radically resolves this dilemma: Charge is not an intrinsic point property but an emergent projection of geometric patterns in the universal field.
	
	\section{Alternative Representations of Charge}
	\label{sec:alternatives}
	
	\subsection{Nonlinear Electrodynamics}
	In models like Born-Infeld, the field saturates at maximum strength $\beta \approx 10^{18}$ V/m, yielding an effective charge radius $r_{\text{eff}} \approx 1/\beta$. This results in finite self-energy $U \approx e^2 \beta / (4\pi \epsilon_0)$.
	
	\subsection{Soliton and Vortex Models}
	The electron as a stable wave packet in nonlinear field theories (e.g., sine-Gordon) distributes the charge density $\rho(r)$ over a finite width, with $E \propto q(r)/r^2$ and $q(r) \to 0$ as $r \to 0$.
	
	\subsection{Topological Defects}
	Charge as a Chern-Simons vortex in gauge theories, quantized by topology ($\pi_3(S^2) = \mathbb{Z}$), without a bare singularity.
	
	\begin{table}[h]
		\centering
		\begin{tabular}{lll}
			\toprule
			\textbf{Model} & \textbf{Singularity?} & \textbf{Self-Energy} \\
			\midrule
			Point Charge (QED) & Yes & $\infty$ (renormalized) \\
			Born-Infeld & Effectively no & Finite \\
			Soliton & No & Finite (from field energy) \\
			T0 Geometry & No & From $\xi$-scaling \\
			\bottomrule
		\end{tabular}
		\caption{Comparison of alternative charge representations}
		\label{tab:comparison}
	\end{table}
	
	\section{The Electron Charge in T0 Theory}
	\label{sec:t0-charge}
	
	\subsection{Time-Mass Duality and Emergence}
	T0 Theory unifies quantum mechanics and relativity in a parameter-free framework via $T_{\text{field}} \cdot E_{\text{field}} = 1$. Particles emerge as excitation patterns in the field, governed by $\xi = \frac{4}{3} \times 10^{-4}$. The fine-structure constant arises as:
	\begin{equation}
		\alpha = \xi \cdot \left( \frac{E_0}{1~\mathrm{MeV}} \right)^2, \quad E_0 = 7.400~\mathrm{MeV},
	\end{equation}
	yielding $\alpha \approx 7.300 \times 10^{-3}$ ($1/\alpha \approx 137.00$)—with fractal corrections for the exact CODATA value $137.035999084$.
	
	The charge $-e$ is a dimensionless geometric relation: $q^{\mathrm{T0}} = -1$ (in natural units), projected via $S_{\mathrm{T0}} = 1.782662 \times 10^{-30}$ kg onto SI values. No singularity, as the charge density is fractally distributed:
	\begin{equation}
		\rho(r) \propto \xi \cdot f_{\text{fractal}}\left( \frac{r}{\lambda_{\text{Compton}}} \right),
	\end{equation}
	with $f_{\text{fractal}}(r) = \prod_{n=1}^{137} \left(1 + \delta_n \cdot \xi \cdot \left(\frac{4}{3}\right)^{n-1}\right)$ and fractal dimension $D_f \approx 2.94$.
	
	\subsection{Finite Self-Energy and Quantization}
	The self-energy is finite:
	\begin{align}
		U &= \frac{1}{2} \int \epsilon_0 E^2 \, dV = \frac{e^2}{8\pi \epsilon_0 r_e} \cdot K_{\text{frac}}, \\
		r_e &\approx 2.817 \times 10^{-15}~\mathrm{m} \quad \text{(classical radius from $\xi$-scaling)}, \\
		K_{\text{frac}} &= 0.986 \quad \text{(fractal correction factor)}.
	\end{align}
	Quantization follows from discrete scales: $q_n = -n \cdot e \cdot \xi^{1/2}$, with $n=1$ for the unit charge. This aligns with topological quantization (Chern number = 1), ensuring stability without collapse.
	
	\section{Implications for Electromagnetic Mechanics}
	\label{sec:emm}
	
	T0 integrates with kinematic mechanics: Charge emerges as a rotating EM vortex, stabilized by fractal renormalization. No Dirac delta—$\rho(r)$ is a helical pattern, enabling singularity-free simulations. Applications: g-2 anomaly predictions and LHC mass spectra.
	
	\section{Conclusion}
	
	T0 Theory transforms the electron charge from a problematic singularity into a harmonious geometric emergence—a core tenet of the framework. All constants derive from $\xi$, reducing physics to dimensionless patterns. Future work: Full kinematic derivations in EMM.
	
	\appendix
	\section{Notation}
	\begin{description}[leftmargin=1cm]
		\item[$\xi$] Geometric parameter; $\xi = \frac{4}{3} \times 10^{-4}$
		\item[$S_{\mathrm{T0}}$] Scaling factor; $S_{\mathrm{T0}} = 1.782662 \times 10^{-30}$ kg
		\item[$f_{\text{fractal}}$] Fractal function; $\prod_{n=1}^{137} (1 + \delta_n \cdot \xi \cdot (4/3)^{n-1})$
		\item[$D_f$] Fractal dimension; $D_f \approx 2.94$
	\end{description}
	
	\begin{center}
		\hrule
		\vspace{0.5cm}
		\textit{This document is part of the T0 series: Exploring geometric emergence in physics}\\
		\textit{Johann Pascher, HTL Leonding, Austria}\\
		\vspace{0.3cm}
		\href{https://github.com/jpascher/T0-Time-Mass-Duality}{T0 Theory: Time-Mass Duality Framework}
		\vspace{0.3cm}
	\end{center}
	
\end{document}