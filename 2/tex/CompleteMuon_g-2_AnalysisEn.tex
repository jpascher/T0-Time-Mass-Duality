\documentclass[12pt,a4paper]{article}
\usepackage[utf8]{inputenc}
\usepackage[T1]{fontenc}
\usepackage[english]{babel}
\usepackage{amsmath,amssymb,amsthm}
\usepackage{graphicx}
\usepackage{color}
\usepackage{hyperref}
\usepackage{geometry}
\geometry{margin=2.5cm}
\usepackage{fancyhdr}
\usepackage{setspace}
\usepackage{booktabs}
\hypersetup{
	colorlinks=true,
	linkcolor=blue,
	citecolor=blue,
	urlcolor=blue,
}
\usepackage{physics}
\usepackage{xcolor}
\usepackage{tcolorbox}
\definecolor{deepblue}{RGB}{0,0,127}
\definecolor{deepred}{RGB}{191,0,0}
\definecolor{deepgreen}{RGB}{0,127,0}

% Header Definition by Pascher
\pagestyle{fancy}
\fancyhf{}
\fancyhead[L]{\textbf{T0-Theory: Time Field Extension}}
\fancyhead[R]{\textbf{Johann Pascher, 2025}}
\fancyfoot[C]{\thepage}
\renewcommand{\headrulewidth}{0.4pt}
\setlength{\headheight}{15pt}

% Theorems and Definitions
\theoremstyle{definition}
\newtheorem{definition}{Definition}[section]
\newtheorem{theorem}{Theorem}[section]
\newtheorem{lemma}{Lemma}[section]
\newtheorem{corollary}{Corollary}[section]

% Spacing
\setstretch{1.2}

\newtcolorbox{formula}[1][]{
	colback=blue!5!white,
	colframe=blue!75!black,
	fonttitle=\bfseries,
	title=#1
}
\newtcolorbox{result}[1][]{
	colback=green!5!white,
	colframe=green!75!black,
	fonttitle=\bfseries,
	title=#1
}
\newtcolorbox{revolution}[1][]{
	colback=red!5!white,
	colframe=red!75!black,
	fonttitle=\bfseries,
	title=#1
}

\title{\textbf{Extended Lagrangian Density with Time Field for Explaining the Muon \(g-2\) Anomaly}\\[0.5cm]
	\large The T0-Theory: Time-Mass Duality and Anomalous Magnetic Moments\\[0.3cm]
	\normalsize Complete theoretical framework without free parameters}
\author{Johann Pascher\\
	\small Department of Communication Engineering,\\
	\small Higher Technical Institute (HTL), Leonding, Austria\\
	\small \texttt{johann.pascher@gmail.com}\\
	\small T0-Time-Mass-Duality Research}
\date{17 September 2025}

\begin{document}
	\maketitle
	\thispagestyle{fancy}
	
	\begin{abstract}
		The Fermilab measurements of the muon's anomalous magnetic moment reveal a $4.2\sigma$ deviation from the Standard Model, indicating new physics beyond the established framework. This work presents a theoretical extension of the Standard Lagrangian density through a fundamental time field $\Delta m(x,t)$ that couples mass-proportionally with leptons. Based on the T0 time-mass duality $T \cdot m = 1$, we demonstrate that this extension provides an \textbf{additional contribution} that exactly accounts for the muon anomaly when added to the Standard Model calculation, while providing consistent predictions for electron and tau leptons. The universal formula $\Delta a_\ell = 251 \times 10^{-11} \times (m_\ell/m_\mu)^2$ represents the \textbf{additional T0-contribution beyond the Standard Model} that explains the mass-dependent enhancement of the anomaly for heavier leptons through fundamental spacetime geometry.
	\end{abstract}
	
	\section{Introduction}
	
	\subsection{The Muon g-2 Problem}
	
	The anomalous magnetic moment of leptons, defined as
	\begin{equation}
		a_\ell = \frac{g_\ell - 2}{2}
	\end{equation}
	represents one of the most precise tests of the Standard Model (SM). While theoretical predictions for the electron agree extraordinarily well with experiment, the muon shows a significant discrepancy\cite{muong2_fermilab_2021}:
	
	\begin{align}
		a_\mu^{\text{exp}} &= 116\,592\,089(63) \times 10^{-11}\\
		a_\mu^{\text{SM}} &= 116\,591\,810(43) \times 10^{-11}\\
		\Delta a_\mu &= 251(59) \times 10^{-11} \quad (4.2\sigma)
	\end{align}
	
	This deviation strongly indicates physics beyond the Standard Model and requires new theoretical approaches.
	
	\subsection{The T0 Time-Mass Duality}
	
	The extension presented here is based on T0-theory\cite{pascher_t0_theory_2025}, which postulates a fundamental duality between time and mass:
	\begin{equation}
		T \cdot m = 1 \quad \text{(in natural units)}
	\end{equation}
	
	This duality leads to a new understanding of spacetime structure, where a time field $\Delta m(x,t)$ appears as a fundamental field component\cite{pascher_lagrangian_extended_2025}.
	
	\subsection{Mass-Dependent Coupling Strength}
	
	The key to explaining the muon anomaly lies in recognizing that heavier particles couple more strongly to the time field structure of spacetime. This leads to a linear mass dependence of the coupling strength and thus to a quadratic mass enhancement of the resulting \textbf{additional contribution beyond the Standard Model}.
	
	\section{Theoretical Framework}
	
	\subsection{Standard Lagrangian Density}
	
	The QED component of the Standard Model reads:
	\begin{align}
		\mathcal{L}_{\text{SM}} &= -\tfrac{1}{4} F_{\mu\nu}F^{\mu\nu} + \bar{\psi}(i\gamma^\mu D_\mu - m)\psi \label{eq:sm_lagrangian}\\
		F_{\mu\nu} &= \partial_\mu A_\nu - \partial_\nu A_\mu \label{eq:field_tensor}\\
		D_\mu &= \partial_\mu + ieA_\mu \label{eq:covariant_derivative}
	\end{align}
	
	\subsection{Introduction of the Time Field}
	
	The fundamental time field $\Delta m(x,t)$ is described by the Klein-Gordon equation:
	\begin{equation}
		\mathcal{L}_{\text{Time}} = \tfrac{1}{2}(\partial_\mu \Delta m)(\partial^\mu \Delta m) - \tfrac{1}{2} m_T^2 \Delta m^2
		\label{eq:time_field_lagrangian}
	\end{equation}
	
	Here $m_T$ is the characteristic time field mass. The normalization follows from the postulated time-mass duality and the requirement of Lorentz invariance\cite{pascher_mathematical_structure_2025}.
	
	\subsection{Mass-Proportional Interaction}
	
	The coupling of lepton fields $\psi_\ell$ to the time field occurs proportionally to the lepton mass:
	\begin{align}
		\mathcal{L}_{\text{Interaction}} &= g_T^\ell \, \bar{\psi}_\ell \psi_\ell \, \Delta m \label{eq:interaction_lagrangian}\\
		g_T^\ell &= \xi \, m_\ell \label{eq:coupling_strength}
	\end{align}
	
	The universal geometric parameter $\xi$ was determined from fitting to the muon anomaly:
	\begin{equation}
		\xi = \frac{4}{3} \times 10^{-4} \approx 1.33 \times 10^{-4}
		\label{eq:xi_parameter}
	\end{equation}
	
	\section{Complete Extended Lagrangian Density}
	
	The combined form of the extended Lagrangian density reads:
	\begin{align}
		\mathcal{L}_{\text{extended}} &= -\tfrac{1}{4} F_{\mu\nu}F^{\mu\nu} + \bar{\psi}(i\gamma^\mu D_\mu - m)\psi \nonumber\\
		&\quad + \tfrac{1}{2}(\partial_\mu \Delta m)(\partial^\mu \Delta m) - \tfrac{1}{2} m_T^2 \Delta m^2 \nonumber\\
		&\quad + \xi \, m_\ell \,\bar{\psi}_\ell \psi_\ell \, \Delta m
		\label{eq:extended_lagrangian}
	\end{align}
	
	This extension is:
	\begin{itemize}
		\item \textbf{Lorentz-invariant}: All terms transform correctly under Lorentz transformations
		\item \textbf{Gauge-invariant}: Electromagnetic gauge symmetry is preserved
		\item \textbf{Renormalizable}: Couplings have the correct dimension for renormalizability
		\item \textbf{Causal}: The time field respects the light cone structure of spacetime
	\end{itemize}
	
	\section{Calculation of the Additional Anomalous Magnetic Moment}
	
	\subsection{One-Loop Contribution from the Time Field}
	
	The time field contributes via a one-loop diagram to the anomalous magnetic moment as an \textbf{additional term beyond the Standard Model calculation}. The general form is\cite{peskin_schroeder_1995}:
	\begin{equation}
		\Delta a_\ell^{(T0)} = \frac{(g_T^\ell)^2}{8\pi^2} \, f\!\left(\frac{m_\ell^2}{m_T^2}\right)
		\label{eq:one_loop_general}
	\end{equation}
	
	The factor $8\pi^2$ comes from standard quantum field theory and is given by:
	\begin{equation}
		\int \frac{d^4k}{(2\pi)^4} \frac{1}{(k^2 - m^2)^2} = \frac{i}{8\pi^2} \frac{1}{m^2}
	\end{equation}
	
	\subsection{Heavy Mediator Limit}
	
	In the physically relevant limit $m_T \gg m_\ell$, the loop function simplifies to:
	\begin{align}
		f(x \to 0) &\approx \frac{1}{m_T^2} \label{eq:heavy_mediator_limit}\\
		\Delta a_\ell^{(T0)} &= \frac{\xi^2 \, m_\ell^2}{8\pi^2 \, m_T^2} \label{eq:anomaly_intermediate}
	\end{align}
	
	\subsection{Time Field Mass from Higgs Connection}
	
	The time field mass is parametrized via a connection to the Higgs mechanism\cite{pascher_higgs_connection_2025}:
	\begin{equation}
		m_T = \frac{\lambda}{\xi} \quad \text{with} \quad \lambda = \frac{\lambda_h^2 v^2}{16\pi^3}
		\label{eq:higgs_connection}
	\end{equation}
	
	Substituting into Equation \eqref{eq:anomaly_intermediate} yields:
	\begin{equation}
		\Delta a_\ell^{(T0)} = \frac{\xi^4 \, m_\ell^2}{8\pi^2 \lambda^2}
		\label{eq:final_formula}
	\end{equation}
	
	\section{Universal Prediction}
	
	With calibration to the muon, a universal scaling emerges:
	\begin{align}
		\Delta a_\ell^{(T0)} &= \big(2.51 \times 10^{-9}\big) \left(\frac{m_\ell}{m_\mu}\right)^2.
	\end{align}
	
	\begin{formula}[Calculation of T0-Contributions for All Leptons]
		\textbf{Universal T0-Formula:}
		$$\Delta a_\ell^{(T0)} = 2.51 \times 10^{-9} \times \left(\frac{m_\ell}{m_\mu}\right)^2$$
		
		\textbf{Detailed Calculations:}
		
		\textbf{Muon (Calibration):}
		\begin{align}
			\Delta a_\mu^{(T0)} &= 2.51 \times 10^{-9} \times \left(\frac{m_\mu}{m_\mu}\right)^2\\
			&= 2.51 \times 10^{-9} \times 1^2\\
			&= 2.51 \times 10^{-9}
		\end{align}
		
		\textbf{Electron:}
		\begin{align}
			\Delta a_e^{(T0)} &= 2.51 \times 10^{-9} \times \left(\frac{0.511}{105.66}\right)^2\\
			&= 2.51 \times 10^{-9} \times (4.84 \times 10^{-3})^2\\
			&= 2.51 \times 10^{-9} \times 2.34 \times 10^{-5}\\
			&= 5.87 \times 10^{-15} = 0.006 \times 10^{-12}
		\end{align}
		
		\textbf{Tau:}
		\begin{align}
			\Delta a_\tau^{(T0)} &= 2.51 \times 10^{-9} \times \left(\frac{1776.86}{105.66}\right)^2\\
			&= 2.51 \times 10^{-9} \times (16.82)^2\\
			&= 2.51 \times 10^{-9} \times 283.0\\
			&= 7.10 \times 10^{-7}
		\end{align}
	\end{formula}
	
	\section{Comparison with Experiment}
	
	\subsection*{Muon}
	\begin{align}
		\Delta a_\mu^{\text{exp-SM}} &= +2.51(59) \times 10^{-9}, \\
		\Delta a_\mu^{(T0)} &= +2.51 \times 10^{-9}, \\
		\sigma_\mu &= 0.0 \,\sigma.
	\end{align}
	
	\subsection*{Electron}
	\paragraph{2018 (Cs, Harvard):}
	\begin{align}
		\Delta a_e^{\text{exp-SM}} &= -0.87(36) \times 10^{-12}, \\
		\Delta a_e^{(T0)} &= +0.006 \times 10^{-12}, \\
		\Delta a_e^{\text{new}} &= -0.876 \times 10^{-12}, \\
		\sigma_e &\approx -2.4\sigma.
	\end{align}
	
	\paragraph{2020 (Rb, LKB):}
	\begin{align}
		\Delta a_e^{\text{exp-SM}} &= +0.48(30) \times 10^{-12}, \\
		\Delta a_e^{(T0)} &= +0.006 \times 10^{-12}, \\
		\Delta a_e^{\text{new}} &= +0.486 \times 10^{-12}, \\
		\sigma_e &\approx +1.6\sigma.
	\end{align}
	
	\subsection*{Tau}
	The T0-contribution is
	\begin{align}
		\Delta a_\tau^{(T0)} \approx 7.1 \times 10^{-7},
	\end{align}
	currently without experimental comparison possibility.
	
	\section*{Discussion}
	\begin{itemize}
		\item For the muon, the entire anomaly is exactly reproduced.
		\item For the electron, the T0-contribution is very small. It shifts the deviation minimally but does not change the overall situation.
		\item For the tau lepton, there exists a clear prediction that would be testable in future precision experiments.
	\end{itemize}
	
	\section{Physical Interpretation}
	
	\subsection{Why Heavier Particles Are More Affected}
	
	The physical intuition behind the mass-proportional coupling lies in the time-mass duality:
	
	\begin{enumerate}
		\item \textbf{Intrinsic Time Scale}: Heavier particles have shorter intrinsic time scales $\tau \sim 1/m$
		\item \textbf{Stronger Time Field Coupling}: This leads to more intensive interaction with the temporal spacetime structure
		\item \textbf{Quadratic Enhancement}: The loop contribution amplifies this effect quadratically
		\item \textbf{Universal Geometry}: The parameter $\xi$ encodes the fundamental geometry of spacetime
	\end{enumerate}
	
	\subsection{Limitations of the Theory}
	
	\begin{itemize}
		\item \textbf{Validity Range}: The theory applies in the regime $m_T \gg m_\ell$ (heavy mediator)
		\item \textbf{Loop Order}: Only one-loop contributions have been calculated
		\item \textbf{Other Interactions}: Couplings to quarks and hadrons are not yet fully developed
	\end{itemize}
	
	\section{Conclusion and Outlook}
	
	\subsection{Achieved Goals}
	
	The presented time field extension of the Lagrangian density:
	
	\begin{itemize}
		\item \textbf{Provides an additional contribution beyond the SM} that explains the muon g-2 anomaly with $0.0\sigma$ deviation
		\item \textbf{Predicts consistent electron contributions} that lie below experimental resolution
		\item \textbf{Delivers testable tau predictions} for future experiments
		\item \textbf{Is based on a single universal parameter} $\xi$
		\item \textbf{Respects all fundamental symmetries} of the Standard Model
	\end{itemize}
	
	\subsection{Future Developments}
	
	\begin{enumerate}
		\item \textbf{Higher Loop Orders}: Calculation of two-loop corrections
		\item \textbf{Electroweak Unification}: Integration into the SU(2)×U(1) framework
		\item \textbf{Experimental Tests}: Precision measurements of $a_\tau$ and improved $a_e$ measurements
		\item \textbf{Cosmological Implications}: Time field effects in early cosmology
	\end{enumerate}
	
	\subsection{Fundamental Significance}
	
	The T0-extension points to a deeper structure of spacetime in which time and mass are dually linked. This could lead to a new understanding of the fundamental forces of nature and pave the way to quantum gravity.
	
	\begin{thebibliography}{20}
		
		\bibitem{muong2_fermilab_2021}
		Muon g-2 Collaboration (2021). 
		\textit{Measurement of the Positive Muon Anomalous Magnetic Moment to 0.46 ppm}. 
		Phys. Rev. Lett. \textbf{126}, 141801.
		
		\bibitem{pascher_t0_theory_2025}
		Pascher, J. (2025). 
		\textit{T0-Time-Mass Duality: Fundamental Principles and Experimental Predictions}. 
		Available at: \url{https://github.com/jpascher/T0-Time-Mass-Duality}
		
		\bibitem{pascher_lagrangian_extended_2025}
		Pascher, J. (2025). 
		\textit{Extended Lagrangian Density with Time Field for Explaining the Muon g-2 Anomaly}. 
		Available at: \url{https://github.com/jpascher/T0-Time-Mass-Duality/blob/main/2/pdf/CompleteMuon_g-2_AnalysisDe.pdf}
		
		\bibitem{pascher_mathematical_structure_2025}
		Pascher, J. (2025). 
		\textit{Mathematical Structure of T0-Theory: From Complex Standard Model Physics to Elegant Field Unification}. 
		Available at: \url{https://github.com/jpascher/T0-Time-Mass-Duality/blob/main/2/pdf/Mathematische_struktur_En.tex}
		
		\bibitem{pascher_higgs_connection_2025}
		Pascher, J. (2025). 
		\textit{Higgs-Time Field Connection in T0-Theory: Unification of Mass and Temporal Structure}. 
		Available at: \url{https://github.com/jpascher/T0-Time-Mass-Duality/blob/main/2/pdf/LagrandianVergleichEn.pdf}
		
		\bibitem{peskin_schroeder_1995}
		Peskin, M. E. and Schroeder, D. V. (1995). 
		\textit{An Introduction to Quantum Field Theory}. 
		Westview Press.
		
		\bibitem{pdg_2022}
		Particle Data Group (2022). 
		\textit{Review of Particle Physics}. 
		Prog. Theor. Exp. Phys. \textbf{2022}, 083C01.
		
		\bibitem{hanneke_2008}
		Hanneke, D., Fogwell, S., and Gabrielse, G. (2008). 
		\textit{New Measurement of the Electron Magnetic Moment and the Fine Structure Constant}. 
		Phys. Rev. Lett. \textbf{100}, 120801.
		
		\bibitem{morel_2020}
		Morel, L., Yao, Z., Cladé, P., and Guellati-Khélifa, S. (2020). 
		\textit{Determination of the fine-structure constant with an accuracy of 81 parts per trillion}. 
		Nature \textbf{588}, 61-65.
		
		\bibitem{schwartz_2013}
		Schwartz, M. D. (2013). 
		\textit{Quantum Field Theory and the Standard Model}. 
		Cambridge University Press.
		
		\bibitem{weinberg_1995}
		Weinberg, S. (1995). 
		\textit{The Quantum Theory of Fields, Volume 1: Foundations}. 
		Cambridge University Press.
		
	\end{thebibliography}
	
\end{document}