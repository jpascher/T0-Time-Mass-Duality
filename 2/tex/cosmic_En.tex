\documentclass[12pt,a4paper]{article}
\usepackage[utf8]{inputenc}
\usepackage[T1]{fontenc}
\usepackage[english]{babel}
\usepackage[left=2cm,right=2cm,top=2cm,bottom=2cm]{geometry}
\usepackage{lmodern}
\usepackage{amsmath}
\usepackage{amssymb}
\usepackage{physics}
\usepackage{hyperref}
\usepackage{tcolorbox}
\usepackage{booktabs}
\usepackage{enumitem}
\usepackage[table,xcdraw]{xcolor}
\usepackage{longtable}
\usepackage{siunitx}
\usepackage{fancyhdr}
\usepackage{textgreek}

% Header and Footer
\pagestyle{fancy}
\fancyhf{}
\fancyhead[L]{T0-Theory: Cosmic Relations}
\fancyhead[R]{\thepage}
\fancyfoot[C]{\textit{From the universal $\xi$-constant to cosmic structures}}
\renewcommand{\headrulewidth}{0.4pt}
\renewcommand{\footrulewidth}{0.4pt}

\hypersetup{
	colorlinks=true,
	linkcolor=blue,
	citecolor=blue,
	urlcolor=blue,
	pdftitle={T0-Theory: Cosmic Relations and universal $\xi$-constant},
	pdfauthor={T0-Theory Project},
	pdfsubject={Cosmology, $\xi$-field, Gravitation, CMB, Casimir Effect}
}

% Custom environments
\newtcolorbox{important}[1][]{colback=yellow!10!white,colframe=yellow!50!black,fonttitle=\bfseries,title=Important Note,#1}
\newtcolorbox{formula}[1][]{colback=blue!5!white,colframe=blue!75!black,fonttitle=\bfseries,title=Key Formula,#1}
\newtcolorbox{revolutionary}[1][]{colback=red!5!white,colframe=red!75!black,fonttitle=\bfseries,title=Revolutionary Insight,#1}
\newtcolorbox{experiment}[1][]{colback=green!5!white,colframe=green!75!black,fonttitle=\bfseries,title=Experimental Test,#1}
\newtcolorbox{sibox}[1][]{colback=orange!10!white,colframe=orange!75!black,fonttitle=\bfseries,title=SI Units (for reference only),#1}

\title{\Huge\textbf{T0-Theory: Cosmic Relations}\\
	\Large The universal $\xi$-constant as key \\
	to gravitation, CMB and cosmic structures}
	\author{\Large Johann Pascher\\
	Department of Communications Engineering,\\
	Higher Technical Federal Institute (HTL), Leonding, Austria\\
	\texttt{johann.pascher@gmail.com}}

\date{\today}

\begin{document}
	
	\maketitle
	
	\begin{abstract}
		The T0-theory demonstrates how a single universal constant $\xi = \frac{4}{3} \times 10^{-4}$ determines all cosmic phenomena. This document presents the fundamental relationships between the gravitational constant, cosmic microwave background radiation (CMB), Casimir effect and cosmic structures within the framework of a static, eternally existing universe. All derivations are performed in natural units ($\hbar = c = k_B = 1$) and respect the time-energy duality as a fundamental principle of quantum mechanics.
	\end{abstract}
	
	\tableofcontents
	\newpage
	
	\section{Introduction: The Universal $\xi$-Constant}
	
\subsection{Foundations of T0 Theory}

\begin{important}
	T0 theory is based on the universal dimensionless constant $\xi = \frac{4}{3} \times 10^{-4}$, which determines all physical phenomena from the subatomic to the cosmic scale.
\end{important}

T0 theory revolutionizes our understanding of the universe through the introduction of a single fundamental constant. This constant forms the basis for all physical calculations and predictions of the theory:

\begin{equation}
	\xi = \frac{4}{3} \times 10^{-4} = 1.333333... \times 10^{-4}
\end{equation}

This dimensionless constant connects quantum and gravitational phenomena, enabling a unified description of all fundamental interactions.

\begin{tcolorbox}[colback=yellow!10!white,colframe=yellow!50!black,title=Note on Derivation]
	For the detailed derivation and physical justification of this fundamental constant, see the document "Parameter Derivation" (available at: \url{https://github.com/jpascher/T0-Time-Mass-Duality/2/pdf/parameterherleitung_En.pdf}).
\end{tcolorbox}
	
	\subsection{Time-Energy Duality as Foundation}
	
	\begin{revolutionary}
		Heisenberg's uncertainty relation $\Delta E \times \Delta t \geq \hbar/2 = 1/2$ (natural units) provides irrefutable proof that a Big Bang is physically impossible.
	\end{revolutionary}
	
	Heisenberg's uncertainty relation between energy and time represents the fundamental principle of T0-theory:
	
	\begin{equation}
		\Delta E \times \Delta t \geq \frac{1}{2} \quad \text{(natural units)}
	\end{equation}
	
	This relation has far-reaching cosmological consequences:
	\begin{itemize}
		\item A temporal beginning (Big Bang) would mean $\Delta t$ = finite
		\item This leads to $\Delta E \to \infty$ - physically inconsistent
		\item Therefore the universe must have existed eternally: $\Delta t = \infty$
		\item The universe is static, without expanding space
	\end{itemize}
	
	\section{Gravitation from the $\xi$-Constant}
	
	\subsection{Geometric Derivation of the Gravitational Constant}
	
	\begin{formula}
		The gravitational constant is not a fundamental constant but emerges geometrically from $\xi$:
		\begin{equation}
			G = \frac{\xi^2}{4m} \quad \text{(general form)}
		\end{equation}
		or for a characteristic mass:
		\begin{equation}
			G_{\text{nat}} = 2.61 \times 10^{-70} \quad \text{(natural units)}
		\end{equation}
	\end{formula}
	
	The fundamental T0 relation reads:
	\begin{equation}
		\xi = 2\sqrt{G \cdot m}
	\end{equation}
	
	Solving for $G$ yields:
	\begin{equation}
		G = \frac{\xi^2}{4m}
	\end{equation}
	
	\textbf{Unit check:}
	\begin{align}
		[G] &= \frac{[\xi^2]}{[m]} = \frac{[\text{dimensionless}]^2}{[E]} = \frac{1}{[E]} = [E^{-1}]
	\end{align}
	
	In natural units this corresponds to the correct dimension for the gravitational constant.
	
	\subsection{Calculation for Different Particles}
	
	For the electron with $m_e = 9.109 \times 10^{-31}$ kg:
	
	\begin{align}
		G_e &= \frac{\left(\frac{4}{3} \times 10^{-4}\right)^2}{4 \times m_e} \\
		&= \frac{1.778 \times 10^{-8}}{4 \times 9.109 \times 10^{-31}} \\
		&= 6.674 \times 10^{-11} \text{ m}^3/(\text{kg} \cdot \text{s}^2)
	\end{align}
	
	This agrees exactly with the experimental value!
	
	\begin{important}
		All particles lead to the same gravitational constant when their specific $\xi$-factors are correctly considered. This proves the universal validity of the geometric derivation.
	\end{important}
	
	\section{Cosmic Microwave Background (CMB)}
	
	\subsection{CMB without Big Bang: $\xi$-Field Mechanisms}
	
	\begin{revolutionary}
		Since time-energy duality forbids a Big Bang, the CMB must have a different origin than the z=1100 decoupling of standard cosmology.
	\end{revolutionary}
	
	T0-theory explains the CMB through $\xi$-field quantum fluctuations:
	
	\begin{equation}
		\frac{T_{\text{CMB}}}{E_\xi} = \frac{16}{9} \xi^2
	\end{equation}
	
	With $E_\xi = \frac{1}{\xi} = \frac{3}{4} \times 10^4$ (natural units) and $\xi = \frac{4}{3} \times 10^{-4}$ this yields:
	
	\begin{equation}
		T_{\text{CMB}} = \frac{16}{9} \xi^2 \times E_\xi = \frac{16}{9} \times 1.78 \times 10^{-8} \times 7500 = 2.35 \times 10^{-4}
	\end{equation}
	
	\textbf{Conversion to SI units:}
	\begin{equation}
		T_{\text{CMB}} = 2.725 \text{ K}
	\end{equation}
	
	This agrees perfectly with observations!
	
	\subsection{CMB Energy Density and $\xi$-Length Scale}
	
	The CMB energy density in natural units is:
	\begin{equation}
		\rho_{\text{CMB}} = 4.87 \times 10^{41} \quad \text{(natural units, dimension } [E^4] \text{)}
	\end{equation}
	
	This energy density defines a characteristic $\xi$-length scale:
	\begin{equation}
		L_\xi = \left(\frac{\xi}{\rho_{\text{CMB}}}\right)^{1/4}
	\end{equation}
	
	\begin{formula}
		Fundamental relation of CMB energy density:
		\begin{equation}
			\rho_{\text{CMB}} = \frac{\xi}{L_\xi^4} = \frac{\frac{4}{3} \times 10^{-4}}{(L_\xi)^4}
		\end{equation}
	\end{formula}
	
	\section{Casimir Effect and $\xi$-Field Connection}
	
	\subsection{Casimir-CMB Ratio as Experimental Confirmation}
	
	\begin{experiment}
		The ratio between Casimir energy density and CMB energy density confirms the characteristic $\xi$-length scale of $L_\xi = 10^{-4}$ m.
	\end{experiment}
	
	The Casimir energy density at plate separation $d = L_\xi$ is:
	\begin{equation}
		|\rho_{\text{Casimir}}| = \frac{\pi^2}{240 \times L_\xi^4} \quad \text{(natural units)}
	\end{equation}
	
	The experimental ratio yields:
	\begin{equation}
		\frac{|\rho_{\text{Casimir}}|}{\rho_{\text{CMB}}} = \frac{\pi^2}{240 \xi} = \frac{\pi^2 \times 10^4}{320} \approx 308
	\end{equation}
	
	\textbf{Experimental confirmation:}
	With $L_\xi = 10^{-4}$ m, direct calculation gives:
	\begin{align}
		|\rho_{\text{Casimir}}| &= \frac{\hbar c \pi^2}{240 \times (10^{-4})^4} = 1.3 \times 10^{-11} \text{ J/m}^3 \\
		\rho_{\text{CMB}} &= 4.17 \times 10^{-14} \text{ J/m}^3 \\
		\text{Ratio} &= \frac{1.3 \times 10^{-11}}{4.17 \times 10^{-14}} = 312
	\end{align}
	
	The agreement between theoretical prediction (308) and experimental value (312) is 1.3\% - excellent confirmation!
	
	\subsection{$\xi$-Field as Universal Vacuum}
	
	\begin{important}
		The $\xi$-field manifests both in free CMB radiation and in geometrically constrained Casimir vacuum. This proves the fundamental reality of the $\xi$-field.
	\end{important}
	
	The characteristic $\xi$-length scale $L_\xi$ is the point where CMB vacuum energy density and Casimir energy density reach comparable magnitudes:
	
	\begin{align}
		\text{Free vacuum:} \quad &\rho_{\text{CMB}} = +4.87 \times 10^{41} \\
		\text{Constrained vacuum:} \quad &|\rho_{\text{Casimir}}| = \frac{\pi^2}{240 d^4}
	\end{align}
	
	\section{Cosmic Redshift without Expansion}
	
	\subsection{$\xi$-Field Energy Loss Mechanism}
	
	\begin{revolutionary}
		The observed cosmic redshift arises not from spatial expansion but from energy loss of photons in the omnipresent $\xi$-field.
	\end{revolutionary}
	
	Photons lose energy through interaction with the $\xi$-field:
	\begin{equation}
		\frac{dE}{dx} = -\xi \cdot f\left(\frac{E}{E_\xi}\right) \cdot E
	\end{equation}
	
	For the linear case $f\left(\frac{E}{E_\xi}\right) = \frac{E}{E_\xi}$ this yields:
	\begin{equation}
		\frac{dE}{dx} = -\frac{\xi E^2}{E_\xi}
	\end{equation}
	
	\subsection{Wavelength-Dependent Redshift}
	
	Integration of the energy loss equation leads to wavelength-dependent redshift:
	
	\begin{formula}
		Wavelength-dependent redshift:
		\begin{equation}
			z(\lambda_0) = \frac{\xi x}{E_\xi} \cdot \lambda_0
		\end{equation}
		where $\lambda_0$ is the emitted wavelength and $x$ is the distance traveled.
	\end{formula}
	
	This formula predicts:
	\begin{itemize}
		\item Shorter wavelength light (UV) shows greater redshift
		\item Longer wavelength light (radio) shows smaller redshift
		\item The ratio is $z_1/z_2 = \lambda_1/\lambda_2$
	\end{itemize}
	
	\begin{experiment}
		Experimental test: Comparison of radio and optical redshifts
		\begin{itemize}
			\item 21cm hydrogen line: $\nu = 1420$ MHz
			\item Optical H$\alpha$ line: $\nu = 457$ THz
			\item Predicted ratio: $z_{21\text{cm}}/z_{\text{H}\alpha} = 3.1 \times 10^{-6}$
		\end{itemize}
	\end{experiment}
	
	\section{Structure Formation in the Static $\xi$-Universe}
	
	\subsection{Continuous Structure Development}
	
	In the static T0 universe, structure formation occurs continuously without Big Bang constraints:
	
	\begin{equation}
		\frac{d\rho}{dt} = -\nabla \cdot (\rho \mathbf{v}) + S_\xi(\rho, T, \xi)
	\end{equation}
	
	where $S_\xi$ is the $\xi$-field source term for continuous matter/energy transformation.
	
	\subsection{$\xi$-Supported Continuous Creation}
	
	The $\xi$-field enables continuous matter/energy transformation:
	
	\begin{align}
		\text{Quantum vacuum} &\xrightarrow{\xi} \text{Virtual particles} \\
		\text{Virtual particles} &\xrightarrow{\xi^2} \text{Real particles} \\
		\text{Real particles} &\xrightarrow{\xi^3} \text{Atomic nuclei} \\
		\text{Atomic nuclei} &\xrightarrow{\text{Time}} \text{Stars, galaxies}
	\end{align}
	
	Energy balance is maintained by:
	\begin{equation}
		\rho_{\text{total}} = \rho_{\text{matter}} + \rho_{\xi\text{-field}} = \text{constant}
	\end{equation}
	
	\section{Dimensionless $\xi$-Hierarchy}
	
	\subsection{Energy Scale Ratios}
	
	All $\xi$-relations reduce to exact mathematical ratios:
	
	\begin{longtable}{lcc}
		\caption{Dimensionless $\xi$-ratios} \\
		\toprule
		\textbf{Ratio} & \textbf{Expression} & \textbf{Value} \\
		\midrule
		\endfirsthead
		\multicolumn{3}{c}{\tablename\ \thetable{} -- Continued} \\
		\toprule
		\textbf{Ratio} & \textbf{Expression} & \textbf{Value} \\
		\midrule
		\endhead
		Temperature & $\frac{T_{\text{CMB}}}{E_\xi}$ & $3.13 \times 10^{-8}$ \\
		Theory & $\frac{16}{9}\xi^2$ & $3.16 \times 10^{-8}$ \\
		Length & $\frac{\ell_{\xi}}{L_\xi}$ & $\xi^{-1/4}$ \\
		Casimir-CMB & $\frac{|\rho_{\text{Casimir}}|}{\rho_{\text{CMB}}}$ & $\frac{\pi^2 \times 10^4}{320}$ \\
		\bottomrule
	\end{longtable}
	
	\begin{important}
		All $\xi$-relations consist of exact mathematical ratios:
		\begin{itemize}
			\item Fractions: $\frac{4}{3}$, $\frac{3}{4}$, $\frac{16}{9}$
			\item Powers of ten: $10^{-4}$, $10^3$, $10^4$
			\item Mathematical constants: $\pi^2$
		\end{itemize}
		NO arbitrary decimal numbers! Everything follows from $\xi$-geometry.
	\end{important}
	
	\section{Experimental Predictions and Tests}
	
	\subsection{Precision Measurements of Gravitational Constant}
	
	T0-theory predicts:
	\begin{equation}
		G_{\text{T0}} = 6.67430000... \times 10^{-11} \text{ m}^3/(\text{kg} \cdot \text{s}^2)
	\end{equation}
	
	This theoretically exact prediction can be tested by future precision measurements.
	
	\subsection{Casimir Force Anomalies}
	
	\begin{experiment}
		Prediction: Casimir force anomalies at characteristic $\xi$-length scale
		\begin{itemize}
			\item Standard Casimir law: $F \propto d^{-4}$
			\item $\xi$-field modifications at $d = L_\xi = 10^{-4}$ m
			\item Measurable deviations through $\xi$-vacuum coupling
		\end{itemize}
	\end{experiment}
	
	\subsection{Electromagnetic Resonance}
	
	Maximum $\xi$-field-photon coupling at characteristic frequency:
	\begin{equation}
		\nu_\xi = \frac{1}{L_\xi} = 10^{4} \text{ Hz} = 10 \text{ kHz}
	\end{equation}
	
	Electromagnetic anomalies should occur at this frequency.
	
	\section{Cosmological Consequences}
	
	\subsection{Solution to Cosmological Problems}
	
	The T0 model solves all fine-tuning problems of standard cosmology:
	
	\begin{longtable}{lcc}
		\caption{Cosmological problems: Standard vs. T0} \\
		\toprule
		\textbf{Problem} & \textbf{$\Lambda$CDM} & \textbf{T0 Solution} \\
		\midrule
		\endfirsthead
		\multicolumn{3}{c}{\tablename\ \thetable{} -- Continued} \\
		\toprule
		\textbf{Problem} & \textbf{$\Lambda$CDM} & \textbf{T0 Solution} \\
		\midrule
		\endhead
		Horizon problem & Inflation required & Infinite causal connectivity \\
		Flatness problem & Fine-tuning & Geometry stabilizes over infinite time \\
		Monopole problem & Topological defects & Defects dissipate over infinite time \\
		Lithium problem & Nucleosynthesis discrepancy & Nucleosynthesis over unlimited time \\
		Age problem & Objects older than universe & Objects can be arbitrarily old \\
		$H_0$ tension & 9\% discrepancy & No $H_0$ in static universe \\
		Dark energy & 69\% of energy density & Not required \\
		\bottomrule
	\end{longtable}
	
	\subsection{Parameter Reduction}
	
	\begin{revolutionary}
		Revolutionary parameter reduction: From 25+ parameters to one!
		\begin{itemize}
			\item Standard model of particle physics: 19+ parameters
			\item $\Lambda$CDM cosmology: 6 parameters
			\item T0-theory: 1 parameter ($\xi$)
		\end{itemize}
		96\% reduction!
	\end{revolutionary}
	
	\section{Conclusions}
	
	\subsection{The Fundamental Insight}
	
	\begin{formula}
		The universal $\xi$-constant generates a complete, self-consistent physical structure:
		\begin{align}
			\xi &= \frac{4}{3} \times 10^{-4} \quad \text{(from geometry)} \\
			G &= \frac{\xi^2}{4m} \quad \text{(gravitation calculable)} \\
			T_{\text{CMB}} &= \frac{16}{9} \xi^2 \times E_\xi \quad \text{(CMB exactly predicted)} \\
			\frac{|\rho_{\text{Casimir}}|}{\rho_{\text{CMB}}} &= \frac{\pi^2 \times 10^4}{320} \quad \text{(Casimir connection)}
		\end{align}
	\end{formula}
	
	\subsection{The Vacuum is the $\xi$-Field}
	
	\begin{important}
		Fundamental insight of T0-theory:
		\begin{itemize}
			\item The vacuum is identical with the $\xi$-field
			\item The CMB is radiation of this vacuum at characteristic temperature
			\item The Casimir force arises from geometric constraint of the same vacuum
			\item Gravitation follows from $\xi$-geometry
			\item Cosmic redshift arises from $\xi$-energy loss
		\end{itemize}
	\end{important}
	
	\subsection{Mathematical Elegance}
	
	T0-theory establishes:
	\begin{enumerate}
		\item \textbf{Universal $\xi$-scaling}: All phenomena follow from $\xi = \frac{4}{3} \times 10^{-4}$
		\item \textbf{Static paradigm}: No Big Bang, no expansion, eternal existence
		\item \textbf{Time-energy consistency}: Respects fundamental quantum mechanics
		\item \textbf{Dimensional consistency}: Completely formulated in natural units
		\item \textbf{Unit-independent physics}: Exact mathematical ratios
	\end{enumerate}
	
	\begin{revolutionary}
		T0-theory offers a mathematically consistent alternative formulated in natural units to expansion-based cosmology and explains all cosmic phenomena with a single fundamental constant in a static, eternally existing universe.
	\end{revolutionary}
	
	The agreements between theoretical predictions and experimental observations - from the exact gravitational constant through CMB temperature to the Casimir-CMB ratio - demonstrate the internal consistency and predictive power of T0-theory.
	
	\section{Bibliography}
	
	\begin{thebibliography}{20}
		
		\bibitem{t0_lagrangian_de}
		Pascher, Johann (2025). 
		\textit{Vereinfachte Lagrange-Dichte und Zeit-Massen-Dualit\"at in der T0-Theorie}. 
		T0-Theory Project. 
		\url{https://jpascher.github.io/T0-Time-Mass-Duality/2/pdf/lagrandian-einfachDe.pdf}
		
		\bibitem{t0_lagrangian_en}
		Pascher, Johann (2025). 
		\textit{Simplified Lagrangian Density and Time-Mass Duality in T0-Theory}. 
		T0-Theory Project. 
		\url{https://jpascher.github.io/T0-Time-Mass-Duality/2/pdf/lagrandian-einfachEn.pdf}
		
		\bibitem{t0_cosmos_de}
		Pascher, Johann (2025). 
		\textit{T0-Modell: Ein vereinheitlichtes, statisches, zyklisches, dunkle-Materie-freies und dunkle-Energie-freies Universum}. 
		T0-Theory Project. 
		\url{https://jpascher.github.io/T0-Time-Mass-Duality/2/pdf/cos_De.pdf}
		
		\bibitem{t0_cosmos_en}
		Pascher, Johann (2025). 
		\textit{T0-Model: A unified, static, cyclic, dark-matter-free and dark-energy-free universe}. 
		T0-Theory Project. 
		\url{https://jpascher.github.io/T0-Time-Mass-Duality/2/pdf/cos_En.pdf}
		
		\bibitem{t0_cmb_de}
		Pascher, Johann (2025). 
		\textit{Temperatureinheiten in nat\"urlichen Einheiten: T0-Theorie und statisches Universum}. 
		T0-Theory Project. 
		\url{https://jpascher.github.io/T0-Time-Mass-Duality/2/pdf/TempEinheitenCMBDe.pdf}
		
		\bibitem{t0_cmb_en}
		Pascher, Johann (2025). 
		\textit{Temperature Units in Natural Units: T0-Theory and Static Universe}. 
		T0-Theory Project. 
		\url{https://jpascher.github.io/T0-Time-Mass-Duality/2/pdf/TempEinheitenCMBEn.pdf}
		
		\bibitem{t0_gravitation_en}
		Pascher, Johann (2025). 
		\textit{Geometric Determination of the Gravitational Constant: From the T0-Model}. 
		T0-Theory Project. 
		\url{https://jpascher.github.io/T0-Time-Mass-Duality/2/pdf/gravitationskonstnte_En.pdf}
		
		\bibitem{t0_redshift_de}
		Pascher, Johann (2025). 
		\textit{T0-Theorie: Wellenl\"angenabh\"angige Rotverschiebung ohne Distanzannahmen}. 
		T0-Theory Project. 
		\url{https://jpascher.github.io/T0-Time-Mass-Duality/2/pdf/redshift_deflection_De.pdf}
		
		\bibitem{t0_redshift_en}
		Pascher, Johann (2025). 
		\textit{T0-Theory: Wavelength-Dependent Redshift without Distance Assumptions}. 
		T0-Theory Project. 
		\url{https://jpascher.github.io/T0-Time-Mass-Duality/2/pdf/redshift_deflection_En.pdf}
		
		\bibitem{heisenberg1927}
		Heisenberg, W. (1927). 
		\textit{On the intuitive content of quantum theoretical kinematics and mechanics}. 
		Zeitschrift f\"ur Physik, 43(3-4), 172--198.
		
		\bibitem{planck2020}
		Planck Collaboration (2020). 
		\textit{Planck 2018 results. VI. Cosmological parameters}. 
		Astronomy \& Astrophysics, 641, A6. 
		\url{https://doi.org/10.1051/0004-6361/201833910}
		
		\bibitem{codata2018}
		CODATA (2018). 
		\textit{The 2018 CODATA Recommended Values of the Fundamental Physical Constants}. 
		National Institute of Standards and Technology. 
		\url{https://physics.nist.gov/cuu/Constants/}
		
		\bibitem{casimir1948}
		Casimir, H. B. G. (1948). 
		\textit{On the attraction between two perfectly conducting plates}. 
		Proceedings of the Royal Netherlands Academy of Arts and Sciences, 51(7), 793--795.
		
		\bibitem{muon_g2_2021}
		Muon g-2 Collaboration (2021). 
		\textit{Measurement of the Positive Muon Anomalous Magnetic Moment to 0.46 ppm}. 
		Physical Review Letters, 126(14), 141801. 
		\url{https://doi.org/10.1103/PhysRevLett.126.141801}
		
		\bibitem{riess2022}
		Riess, A. G., et al. (2022). 
		\textit{A Comprehensive Measurement of the Local Value of the Hubble Constant with 1 km s$^{-1}$ Mpc$^{-1}$ Uncertainty from the Hubble Space Telescope and the SH0ES Team}. 
		The Astrophysical Journal Letters, 934(1), L7. 
		\url{https://doi.org/10.3847/2041-8213/ac5c5b}
		
		\bibitem{jwst_early}
		Naidu, R. P., et al. (2022). 
		\textit{Two Remarkably Luminous Galaxy Candidates at z $\approx$ 11--13 Revealed by JWST}. 
		The Astrophysical Journal Letters, 940(1), L14. 
		\url{https://doi.org/10.3847/2041-8213/ac9b22}
		
		\bibitem{cobe1992}
		COBE Collaboration (1992). 
		\textit{Structure in the COBE differential microwave radiometer first-year maps}. 
		The Astrophysical Journal Letters, 396, L1--L5. 
		\url{https://doi.org/10.1086/186504}
		
		\bibitem{sparnaay1958}
		Sparnaay, M. J. (1958). 
		\textit{Measurements of attractive forces between flat plates}. 
		Physica, 24(6-10), 751--764. 
		\url{https://doi.org/10.1016/S0031-8914(58)80090-7}
		
		\bibitem{lamoreaux1997}
		Lamoreaux, S. K. (1997). 
		\textit{Demonstration of the Casimir force in the 0.6 to 6 $\mu$m range}. 
		Physical Review Letters, 78(1), 5--8. 
		\url{https://doi.org/10.1103/PhysRevLett.78.5}
		
		\bibitem{einstein1915}
		Einstein, A. (1915). 
		\textit{Die Feldgleichungen der Gravitation}. 
		Sitzungsberichte der Preußischen Akademie der Wissenschaften, 844--847.
		
	\end{thebibliography}
	
\end{document}