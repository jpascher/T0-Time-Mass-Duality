\documentclass[12pt,a4paper]{article}
\usepackage[utf8]{inputenc}
\usepackage[T1]{fontenc}
\usepackage[english]{babel}
\usepackage[left=2cm,right=2cm,top=2cm,bottom=2cm]{geometry}
\usepackage{lmodern}
\usepackage{amsmath}
\usepackage{amssymb}
\usepackage{physics}
\usepackage{hyperref}
\usepackage{tcolorbox}
\usepackage{booktabs}
\usepackage{enumitem}
\usepackage[table,xcdraw]{xcolor}
\usepackage{pgfplots}
\pgfplotsset{compat=1.18}
\usepackage{graphicx}
\usepackage{float}
\usepackage{mathtools}
\usepackage{amsthm}
\usepackage{cleveref}
\usepackage{siunitx}
\usepackage{fancyhdr}
\usepackage{tocloft}

% Headers and Footers
\pagestyle{fancy}
\fancyhf{}
\fancyhead[L]{Johann Pascher}
\fancyhead[R]{Parameter System-Dependency T0-Model}
\fancyfoot[C]{\thepage}
\renewcommand{\headrulewidth}{0.4pt}
\renewcommand{\footrulewidth}{0.4pt}

% Table of Contents Styling
\renewcommand{\cftsecfont}{\color{blue}}
\renewcommand{\cftsubsecfont}{\color{blue}}
\renewcommand{\cftsecpagefont}{\color{blue}}
\renewcommand{\cftsubsecpagefont}{\color{blue}}
\setlength{\cftsecindent}{1cm}
\setlength{\cftsubsecindent}{2cm}

\hypersetup{
	colorlinks=true,
	linkcolor=blue,
	citecolor=blue,
	urlcolor=blue,
	pdftitle={Parameter System-Dependency in T0-Model: SI vs. Natural Units},
	pdfauthor={Johann Pascher},
	pdfsubject={T0 Model, Unit Systems, Parameter Transformation},
	pdfkeywords={Natural Units, SI Units, Parameter Dependency, T0 Model}
}

% Custom Commands - CORRECTED to avoid double subscript errors
\newcommand{\xipar}{\xi}
\newcommand{\lP}{\ell_{\text{P}}}
\newcommand{\tP}{t_{\text{P}}}
\newcommand{\EP}{E_{\text{P}}}
\newcommand{\lambdah}{\lambda_h}
\newcommand{\epsilonzero}{\varepsilon_0}
\newcommand{\Rzero}{R_\infty}
\newcommand{\pichar}{\pi}

% Specific system-dependent commands to avoid confusion
\newcommand{\alphaEMSI}{\alpha_{\text{EM,SI}}}
\newcommand{\alphaEMnat}{\alpha_{\text{EM,nat}}}
\newcommand{\betaTSI}{\beta_{\text{T,SI}}}
\newcommand{\betaTnat}{\beta_{\text{T,nat}}}
\newcommand{\alphaWSI}{\alpha_{\text{W,SI}}}
\newcommand{\alphaWnat}{\alpha_{\text{W,nat}}}

\newtheorem{theorem}{Theorem}[section]
\newtheorem{proposition}[theorem]{Proposition}
\newtheorem{definition}[theorem]{Definition}
\newtheorem{warning}[theorem]{Warning}

\begin{document}
	
	\title{Parameter System-Dependency in T0-Model: \\
		SI vs. Natural Units and the Danger \\
		of Direct Transfer of Formula Symbols}
	\author{Johann Pascher\\
		Department of Communications Engineering, \\H{\"o}here Technische Bundeslehranstalt (HTL), Leonding, Austria\\
		\texttt{johann.pascher@gmail.com}}
	\date{\today}
	
	\maketitle
	
	\begin{abstract}
		This paper systematically analyzes the parameter dependency between SI units and T0-model natural units, revealing that fundamental parameters like $\xipar$, $\alpha_{\text{EM}}$, $\beta_{\text{T}}$, and Yukawa couplings have dramatically different numerical values in different unit systems. Through detailed calculations, we demonstrate that direct transfer of parameter values between systems leads to errors spanning multiple orders of magnitude. The analysis extends beyond specific parameters to establish universal transformation rules and provides critical warnings against naive parameter transfer. This work establishes that the apparent inconsistencies in T0-model parameters are actually systematic unit-system dependencies that require careful transformation protocols for experimental verification.
	\end{abstract}
	
	\tableofcontents
	\newpage
	
	\section{Introduction}
	\label{sec:introduction}
	
	\subsection{The Parameter Transfer Problem}
	\label{subsec:parameter_problem}
	
	The T0 model, formulated in natural units where $\hbar = c = G = k_B = \alpha_{\text{EM}} = \alpha_{\text{W}} = \beta_{\text{T}} = 1$, presents a fundamental challenge when compared with experimental data expressed in SI units. This paper demonstrates that the apparent inconsistencies between T0-model predictions and experimental observations are not physical contradictions but systematic unit-system dependencies.
	
	The core insight is that parameters such as $\xipar$, $\alpha_{\text{EM}}$, and $\beta_{\text{T}}$ represent fundamentally different quantities when expressed in different unit systems:
	
	$$\xipar_{\text{SI}} \neq \xipar_{\text{nat}}, \quad \alphaEMSI \neq \alphaEMnat, \quad \betaTSI \neq \betaTnat$$
	
	\subsection{Scope and Methodology}
	\label{subsec:scope}
	
	This analysis covers:
	\begin{itemize}
		\item Systematic calculation of parameter ratios between SI and T0-natural units
		\item Demonstration of transformation invariance for dimensionless ratios
		\item Extension to variable parameters like $\xipar$ and Yukawa couplings
		\item Universal warnings against direct parameter transfer
		\item Guidelines for correct experimental comparison protocols
	\end{itemize}
	
	\section{The $\xipar$ Parameter: Variable Across Mass Scales}
	\label{sec:xi_parameter}
	
	\subsection{Definition and Physical Meaning}
	\label{subsec:xi_definition}
	
	The parameter $\xipar$ is defined as the ratio of the Schwarzschild radius to the Planck length:
	
	\begin{equation}
		\xipar = \frac{r_0}{\lP} = \frac{2Gm}{\lP}
		\label{eq:xi_definition}
	\end{equation}
	
	\textbf{Crucially, $\xipar$ is not a universal constant but varies with the mass $m$ of the object under consideration.}
	
	\subsection{Connection to Higgs Physics}
	\label{subsec:xi_higgs_connection}
	
	The T0 model establishes a fundamental connection between $\xipar$ and Higgs sector physics through the relationship derived in the complete field-theoretic framework \cite{pascher_derivation_beta_2025}:
	
	\begin{equation}
		\xipar = \frac{\lambdah^2 v^2}{16\pichar^3 m_h^2} \approx 1.33 \times 10^{-4}
		\label{eq:xi_higgs_fundamental}
	\end{equation}
	
	where:
	\begin{itemize}
		\item $\lambdah \approx 0.13$ (Higgs self-coupling)
		\item $v \approx 246$ GeV (Higgs VEV)
		\item $m_h \approx 125$ GeV (Higgs mass)
	\end{itemize}
	
	This represents the universal scale parameter that emerges from fundamental Standard Model physics, while the mass-dependent form $\xipar = 2Gm/\lP$ applies to specific objects.
	
	\subsection{$\xipar$ Values in the SI System}
	\label{subsec:xi_si_values}
	
	Using SI constants:
	\begin{align}
		G &= 6.674 \times 10^{-11} \text{ m}^3/(\text{kg} \cdot \text{s}^2) \\
		\lP &= 1.616 \times 10^{-35} \text{ m}
	\end{align}
	
	We calculate $\xipar_{\text{SI}}$ for various objects:
	
	\begin{table}[htbp]
		\centering
		\begin{tabular}{lcc}
			\toprule
			\textbf{Object} & \textbf{Mass} & \textbf{$\xipar_{\text{SI}}$} \\
			\midrule
			Electron & $9.109 \times 10^{-31}$ kg & $7.52 \times 10^{-7}$ \\
			Proton & $1.673 \times 10^{-27}$ kg & $1.38 \times 10^{-3}$ \\
			Human (70 kg) & $7.0 \times 10^{1}$ kg & $6.4 \times 10^{6}$ \\
			Earth & $5.972 \times 10^{24}$ kg & $4.1 \times 10^{28}$ \\
			Sun & $1.989 \times 10^{30}$ kg & $1.8 \times 10^{38}$ \\
			Planck mass & $2.176 \times 10^{-8}$ kg & $2.0$ \\
			\bottomrule
		\end{tabular}
		\caption{$\xipar$ values for different objects in SI units}
		\label{tab:xi_si_values}
	\end{table}
	
	\textbf{The parameter $\xipar$ varies over 46 orders of magnitude!}
	
	\subsection{$\xipar$ Transformation to T0-Natural Units}
	\label{subsec:xi_transformation}
	
	Based on the comprehensive transformation analysis, the conversion factor between systems is approximately:
	
	$$\frac{\xipar_{\text{nat}}}{\xipar_{\text{SI}}} \approx 4100$$
	
	This gives T0-natural unit values:
	
	\begin{table}[htbp]
		\centering
		\begin{tabular}{lcc}
			\toprule
			\textbf{Object} & \textbf{$\xipar_{\text{SI}}$} & \textbf{$\xipar_{\text{nat}}$} \\
			\midrule
			Electron & $7.52 \times 10^{-7}$ & $3.1 \times 10^{-3}$ \\
			Proton & $1.38 \times 10^{-3}$ & $5.7$ \\
			Human (70 kg) & $6.4 \times 10^{6}$ & $2.6 \times 10^{10}$ \\
			Sun & $1.8 \times 10^{38}$ & $7.4 \times 10^{41}$ \\
			\bottomrule
		\end{tabular}
		\caption{$\xipar$ transformation between unit systems}
		\label{tab:xi_transformation}
	\end{table}
	
	\subsection{Invariance of Ratios}
	\label{subsec:xi_ratio_invariance}
	
	\textbf{Critical verification:} The ratios between different objects remain identical in both systems:
	
	\begin{align}
		\frac{\xipar_{\text{Sun},\text{SI}}}{\xipar_{\text{e},\text{SI}}} &= \frac{1.8 \times 10^{38}}{7.52 \times 10^{-7}} = 2.4 \times 10^{44} \\
		\frac{\xipar_{\text{Sun},\text{nat}}}{\xipar_{\text{e},\text{nat}}} &= \frac{7.4 \times 10^{41}}{3.1 \times 10^{-3}} = 2.4 \times 10^{44}
	\end{align}
	
	\boxed{\text{Ratios are invariant under system transformation!}}
	
	\section{The Fine Structure Constant $\alpha_{\text{EM}}$}
	\label{sec:alpha_em}
	
	\subsection{The Mystification of 1/137}
	\label{subsec:alpha_mystification}
	
	The fine structure constant $\alpha_{\text{EM}}$ has been mystified by prominent physicists:
	
	\begin{itemize}
		\item \textbf{Richard Feynman}: ``It is one of the greatest damn mysteries of physics: a magic number that comes to us with no understanding.''
		\item \textbf{Wolfgang Pauli}: ``When I die, I will ask God two questions: Why relativity? And why 137? I believe he will have an answer to the first.''
		\item \textbf{Max Born}: ``If $\alpha$ were larger, no molecules could exist, and there would be no life.''
	\end{itemize}
	
	\subsection{The Overlooked System Dependency}
	\label{subsec:alpha_system_dependency}
	
	What all these statements overlook is that $\alpha_{\text{EM}} = 1/137$ is \textbf{only valid in the SI system}!
	
	\begin{align}
		\text{SI system:} \quad &\alphaEMSI = \frac{e^2}{4\pichar\epsilonzero\hbar c} \approx \frac{1}{137.036} \\
		\text{T0-natural system:} \quad &\alphaEMnat = 1 \text{ (by definition)}
	\end{align}
	
	\textbf{Transformation factor:}
	$$\frac{\alphaEMnat}{\alphaEMSI} = 137.036$$
	
	\subsection{The Anthropic Fallacy}
	\label{subsec:anthropic_fallacy}
	
	Typical anthropic arguments state:
	\begin{itemize}
		\item ``If $\alpha_{\text{EM}} = 1/200$ $\rightarrow$ no atoms $\rightarrow$ no life''
		\item ``If $\alpha_{\text{EM}} = 1/80$ $\rightarrow$ no stars $\rightarrow$ no life''
		\item ``Therefore $\alpha_{\text{EM}} = 1/137$ is `fine-tuned' for life''
	\end{itemize}
	
	\textbf{The problem:} These arguments assume the SI system as absolute!
	
	\textbf{In T0 units:} $\alpha_{\text{EM}} = 1$ is perfectly natural, requiring no fine-tuning whatsoever.
	
	As demonstrated in\cite{pascher_feinstruktur_2025}, the fine structure constant reveals its true nature through the electromagnetic duality inherent in Maxwell's equations. The two equivalent representations:
	\begin{align}
		\alpha_{\text{EM}} &= \frac{e^2}{4\pi\varepsilon_0\hbar c} \quad \text{(standard form)}\\
		\alpha_{\text{EM}} &= \frac{e^2 \mu_0 c}{4\pi \hbar} \quad \text{(dual form)}
	\end{align}
	
	demonstrate the electromagnetic duality $\frac{1}{\varepsilon_0 c} = \mu_0 c$, which is precisely Maxwell's relation $c^2 = \frac{1}{\varepsilon_0\mu_0}$.
	
	When $\alpha_{\text{EM}} = 1$ is chosen as the natural unit, this duality is perfectly balanced, and the elementary charge becomes:
	$$e = \sqrt{4\pi\varepsilon_0\hbar c} = \sqrt{\frac{4\pi\hbar}{\mu_0 c}}$$
	
	This reveals that the mystification of $1/137$ is purely a consequence of our historical unit choices, not a fundamental mystery of nature. The electromagnetic interaction has unit strength ($\alpha = 1$) in the natural unit system that respects the fundamental electromagnetic duality of Maxwell's equations.
	
	The ``fine-tuning problem'' dissolves completely once we recognize that:
	\begin{itemize}
		\item $\alpha = 1/137$ is not a fundamental number but a unit conversion factor
		\item $\alpha = 1$ represents the natural strength of electromagnetic coupling
		\item The apparent ``mystery'' arises from treating arbitrary SI units as absolute
		\item Nature's fundamental relationships are simple in their natural language
	\end{itemize}
	
	As shown in the rigorous mathematical proof \cite{pascher_proof_2025}, there exists a consistent natural unit system where $\alpha = 1$ emerges inevitably from the electromagnetic duality, resolving the century-old puzzle through proper understanding of unit systems rather than speculative fine-tuning mechanisms.
	
	\subsection{Historical Warning: The Eddington Saga}
	\label{subsec:eddington_warning}
	
	Arthur Eddington (1882-1944) attempted to ``prove'' $\alpha_{\text{EM}} = 1/137$ from first principles, developing elaborate numerological theories. The result was completely speculative and wrong, serving as a warning against mystifying system-dependent numbers.
	
	However, recent work by Pascher \cite{pascher_feinstruktur_2025} has shown that the fine structure constant can be derived from fundamental electromagnetic vacuum constants and that setting $\alpha_{\text{EM}} = 1$ in natural units is not only possible but reveals the arbitrary nature of our unit system choices.
	
	\section{The $\beta_{\text{T}}$ Parameter}
	\label{sec:beta_t}
	
	\subsection{Empirical vs. Theoretical Values}
	\label{subsec:beta_empirical_theoretical}
	
	The $\beta_{\text{T}}$ parameter shows the same system dependency:
	
	\begin{align}
		\betaTSI &\approx 0.008 \text{ (from astrophysical observations)} \\
		\betaTnat &= 1 \text{ (in T0-natural units)}
	\end{align}
	
	\textbf{Transformation factor:}
	$$\frac{\betaTnat}{\betaTSI} = \frac{1}{0.008} = 125$$
	
	\subsection{Theoretical Foundation from Field Theory}
	\label{subsec:beta_field_theory}
	
	The T0 model establishes $\beta_{\text{T}} = 1$ through the fundamental field-theoretic relationship \cite{pascher_derivation_beta_2025}:
	
	\begin{equation}
		\beta_{\text{T}} = \frac{\lambdah^2 v^2}{16\pichar^3 m_h^2 \xipar} = 1
		\label{eq:beta_t_field_theory}
	\end{equation}
	
	This relationship, combined with the Higgs-derived value of $\xipar$, uniquely determines $\beta_{\text{T}} = 1$ in natural units, eliminating any free parameters from the theory.
	
	\subsection{Circularity in SI Determination}
	\label{subsec:beta_circularity}
	
	The SI value $\betaTSI$ is determined through:
	$$z(\lambda) = z_0\left(1 + \beta_{\text{T}} \ln\frac{\lambda}{\lambda_0}\right)$$
	
	But this involves:
	\begin{itemize}
		\item Hubble constant $H_0$ $\rightarrow$ distance measurements
		\item Distance ladder $\rightarrow$ standard candles
		\item Photometry $\rightarrow$ Planck radiation law $\rightarrow$ fundamental constants
	\end{itemize}
	
	\textbf{The determination is circular through cosmological parameters!}
	
	\section{The Wien Constant $\alpha_{\text{W}}$}
	\label{sec:alpha_w}
	
	\subsection{Mathematical vs. Conventional Values}
	\label{subsec:wien_values}
	
	Wien's displacement law gives:
	
	\begin{align}
		\text{SI system:} \quad &\alphaWSI = 2.8977719... \\
		\text{T0 system:} \quad &\alphaWnat = 1
	\end{align}
	
	\textbf{Transformation factor:}
	$$\frac{\alphaWSI}{\alphaWnat} = 2.898$$
	
	\section{Parameter Comparison Table}
	\label{sec:parameter_comparison}
	
	\begin{table}[htbp]
		\centering
		\begin{tabular}{lcccc}
			\toprule
			\textbf{Parameter} & \textbf{SI Value} & \textbf{T0-nat Value} & \textbf{Ratio} & \textbf{Factor} \\
			\midrule
			$\xipar$ (electron) & $7.5 \times 10^{-6}$ & $3.1 \times 10^{-2}$ & 4100 & $10^{3.6}$ \\
			$\alpha_{\text{EM}}$ & $7.3 \times 10^{-3}$ & $1$ & 137 & $10^{2.1}$ \\
			$\beta_{\text{T}}$ & $0.008$ & $1$ & 125 & $10^{2.1}$ \\
			$\alpha_{\text{W}}$ & $2.898$ & $1$ & 2.9 & $10^{0.5}$ \\
			\bottomrule
		\end{tabular}
		\caption{Systematic parameter differences between unit systems}
		\label{tab:parameter_comparison}
	\end{table}
	
	\textbf{All parameters show 0.5-4 orders of magnitude difference between systems!}
	
	\section{Yukawa Parameters: Variable and System-Dependent}
	\label{sec:yukawa_parameters}
	
	\subsection{The Hierarchy of Yukawa Couplings}
	\label{subsec:yukawa_hierarchy}
	
	In the Standard Model, Yukawa couplings vary dramatically:
	
	\begin{table}[htbp]
		\centering
		\begin{tabular}{lc}
			\toprule
			\textbf{Particle} & \textbf{$y_i$ (SI system)} \\
			\midrule
			Electron & $2.94 \times 10^{-6}$ \\
			Muon & $6.09 \times 10^{-4}$ \\
			Tau & $1.03 \times 10^{-2}$ \\
			Up quark & $1.27 \times 10^{-5}$ \\
			Top quark & $1.00$ \\
			Bottom quark & $2.25 \times 10^{-2}$ \\
			\bottomrule
		\end{tabular}
		\caption{Yukawa coupling hierarchy (5 orders of magnitude variation)}
		\label{tab:yukawa_hierarchy}
	\end{table}
	
	\subsection{Transformation Uncertainty}
	\label{subsec:yukawa_transformation}
	
	The transformation of Yukawa parameters between systems requires careful consideration of the Higgs mechanism. The general form would be:
	
	$$y_{i,\text{nat}} = y_{i,\text{SI}} \times T_{\text{Yukawa}}$$
	
	where $T_{\text{Yukawa}}$ depends on the transformation of Higgs vacuum expectation value and particle masses.
	
	\subsection{Consistency Requirements}
	\label{subsec:yukawa_consistency}
	
	The Higgs mechanism requires:
	$$m_h^2 = \frac{\lambdah v^2}{2}$$
	
	For transformation consistency:
	$$T_m^2 = T_\lambda \times T_v^2$$
	
	This gives:
	$$y_{i,\text{nat}} = y_{i,\text{SI}} \times \sqrt{T_\lambda}$$
	
	\textbf{However, $T_\lambda$ requires detailed specification of the T0-natural unit system transformation rules.}
	
	\section{Universal Warning: No Direct Parameter Transfer}
	\label{sec:universal_warning}
	
	\subsection{The Systematic Problem}
	\label{subsec:systematic_problem}
	
	\begin{warning}
		\textbf{EVERY parameter symbol in T0-model documents may have different values than in SI system calculations!}
	\end{warning}
	
	\textbf{Concrete danger zones:}
	
	\begin{align}
		G_{\text{nat}} &= 1 \quad \text{vs.} \quad G_{\text{SI}} = 6.674 \times 10^{-11} \text{ m}^3/(\text{kg} \cdot \text{s}^2) \\
		\alpha_{\text{EM,nat}} &= 1 \quad \text{vs.} \quad \alpha_{\text{EM,SI}} = 1/137 \\
		e_{\text{nat}} &= 2\sqrt{\pichar} \quad \text{vs.} \quad e_{\text{SI}} = 1.602 \times 10^{-19} \text{ C}
	\end{align}
	
	\textbf{Direct transfer leads to errors of factors $10^2$ to $10^{11}$!}
	
	\subsection{Required Transformation Protocol}
	\label{subsec:transformation_protocol}
	
	For every parameter, explicitly specify:
	
	\begin{enumerate}
		\item \textbf{Which unit system} is being used
		\item \textbf{How transformation occurs} between systems
		\item \textbf{Which factors must be considered}
		\item \textbf{Which consistency conditions} must be satisfied
	\end{enumerate}
	
	\textbf{Example of complete specification:}
	\begin{tcolorbox}[colback=red!5!white,colframe=red!75!black,title=Parameter Specification Template]
		\textbf{Parameter:} Fine structure constant $\alpha_{\text{EM}}$ \\
		\textbf{SI value:} $\alphaEMSI = 1/137.036$ \\
		\textbf{T0 value:} $\alphaEMnat = 1$ \\
		\textbf{Transformation:} $\alphaEMnat = \alphaEMSI \times 137.036$ \\
		\textbf{Consistency:} Dimensional analysis verified \\
		\textbf{Usage:} Specify system before calculation
	\end{tcolorbox}
	
	\subsection{Experimental Prediction Guidelines}
	\label{subsec:experimental_guidelines}
	
	\textbf{For QED calculations:}
	\begin{align}
		\text{WRONG:} \quad &\alpha_{\text{EM}} = 1 \text{ from T0-model directly in SI formulas} \\
		\text{CORRECT:} \quad &\alphaEMSI = 1/137 \text{ with transformation to } \alphaEMnat = 1
	\end{align}
	
	\textbf{For gravitational calculations:}
	\begin{align}
		\text{WRONG:} \quad &G = 1 \text{ from T0-model directly in Newton's formulas} \\
		\text{CORRECT:} \quad &G_{\text{SI}} = 6.674 \times 10^{-11} \text{ with transformation to } G_{\text{nat}} = 1
	\end{align}
	
	\section{The Circularity Resolution}
	\label{sec:circularity_resolution}
	
	\subsection{Apparent vs. Real Circularity}
	\label{subsec:apparent_real_circularity}
	
	The circularity problem that seemed to plague T0-model parameter determination is resolved by recognizing:
	
	\begin{enumerate}
		\item \textbf{No real circularity exists} within each consistent system
		\item \textbf{Both SI and T0 systems are internally consistent}
		\item \textbf{The apparent contradiction} arose from comparing parameters across different systems
		\item \textbf{Proper transformation} eliminates all apparent inconsistencies
	\end{enumerate}
	
	\subsection{System Consistency Verification}
	\label{subsec:system_consistency}
	
	\textbf{SI system consistency:}
	$$\Rzero = \frac{m_e c \left(\alphaEMSI\right)^2}{2\hbar} \quad \checkmark \text{ (experimentally verified to 0.000001\%)}$$
	
	\textbf{T0 system consistency:}
	$$\text{All parameters = 1} \quad \checkmark \text{ (by construction)}$$
	
	\textbf{Both systems work perfectly within their own frameworks!}
	
	\section{Implications for T0-Model Testing}
	\label{sec:testing_implications}
	
	\subsection{System-Specific Predictions}
	\label{subsec:system_specific_predictions}
	
	Experimental tests must clearly specify which parameter system is used:
	
	\begin{table}[htbp]
		\centering
		\begin{tabular}{lcc}
			\toprule
			\textbf{Test Type} & \textbf{SI-based Prediction} & \textbf{T0-based Prediction} \\
			\midrule
			QED anomaly & $a_e \propto \alphaEMSI = 1/137$ & $a_e \propto \alphaEMnat = 1$ \\
			Galaxy rotation & $v^2 \propto \xipar_{\text{SI}} \sim 10^{38}$ & $v^2 \propto \xipar_{\text{nat}} \sim 10^{41}$ \\
			CMB temperature & $T \propto \betaTSI = 0.008$ & $T \propto \betaTnat = 1$ \\
			\bottomrule
		\end{tabular}
		\caption{System-specific experimental predictions}
		\label{tab:system_predictions}
	\end{table}
	
	\subsection{Transformation Validation}
	\label{subsec:transformation_validation}
	
	The transformation factors can be validated by checking:
	
	\begin{enumerate}
		\item \textbf{Dimensional consistency} in both systems
		\item \textbf{Known limits} are reproduced correctly
		\item \textbf{Ratios remain invariant} between systems
		\item \textbf{Internal consistency} of each system
	\end{enumerate}
	
	\section{Conclusions}
	\label{sec:conclusions}
	
	\subsection{Key Findings}
	\label{subsec:key_findings}
	
	This analysis has demonstrated:
	
	\begin{enumerate}
		\item \textbf{All fundamental parameters are system-dependent} with transformation factors ranging from 2.9 to 4100
		\item \textbf{No parameter can be directly transferred} between SI and T0-natural unit systems
		\item \textbf{Apparent inconsistencies} were artifacts of comparing parameters across different systems
		\item \textbf{Both systems are internally consistent} and experimentally viable
		\item \textbf{Variable parameters like $\xipar$} span dozens of orders of magnitude within each system
		\item \textbf{Transformation factors are invariant} across different mass scales
	\end{enumerate}
	
	\subsection{Universal Principles}
	\label{subsec:universal_principles}
	
	\begin{tcolorbox}[colback=green!5!white,colframe=green!75!black,title=Universal Parameter Transfer Rules]
		\begin{enumerate}
			\item \textbf{Never transfer parameter values directly} between unit systems
			\item \textbf{Always specify the unit system} being used in calculations
			\item \textbf{Apply proper transformation factors} when switching systems
			\item \textbf{Verify dimensional consistency} after transformation
			\item \textbf{Check that known limits} are reproduced correctly
			\item \textbf{Maintain system consistency} throughout calculations
		\end{enumerate}
	\end{tcolorbox}
	
	\subsection{Resolution of the Fine-Tuning Problem}
	\label{subsec:fine_tuning_resolution}
	
	The mystification of parameters like $\alpha_{\text{EM}} = 1/137$ dissolves when we recognize:
	
	\begin{itemize}
		\item \textbf{The value 1/137 is system-specific}, not universal
		\item \textbf{In T0-natural units}, $\alpha_{\text{EM}} = 1$ is perfectly natural
		\item \textbf{Anthropic arguments} assume one particular unit system as absolute
		\item \textbf{What is fundamental} are the mathematical relationships, not the numerical values in arbitrary unit systems
	\end{itemize}
	
	\subsection{Future Directions}
	\label{subsec:future_directions}
	
	This work establishes the foundation for:
	
	\begin{enumerate}
		\item \textbf{Systematic experimental protocols} for T0-model testing
		\item \textbf{Complete transformation tables} for all relevant parameters
		\item \textbf{Educational materials} warning against parameter transfer errors
		\item \textbf{Computational tools} for automatic unit system conversion
		\item \textbf{Philosophical examination} of the role of unit systems in fundamental physics
	\end{enumerate}
	
	\section{Final Remarks}
	\label{sec:final_remarks}
	
	The parameter systemabhängigkeit revealed in this work represents more than a technical correction---it challenges our understanding of what constitutes fundamental physics. The same physical reality can be described using dramatically different numerical values depending on our choice of unit system, yet the underlying mathematical relationships remain invariant.
	
	This teaches us that:
	\begin{itemize}
		\item \textbf{Numbers are not physics}---relationships are
		\item \textbf{Unit systems are human constructs}, not universal truths  
		\item \textbf{Apparent mysteries} may be artifacts of conventional choices
		\item \textbf{True universality} lies in mathematical structure, not numerical values
	\end{itemize}
	
	The T0-model, with its natural unit system where fundamental parameters equal unity, may provide a clearer view of the underlying simplicity that our conventional unit systems obscure. Whether this simplicity reflects deeper truth about nature remains to be determined through careful experimental verification---using the proper transformation protocols established in this work.
	
	\begin{thebibliography}{9}
		\bibitem{pascher_derivation_beta_2025}
		Pascher, J. (2025). \textit{Field-Theoretic Derivation of the $\beta_T$ Parameter in Natural Units ($\hbar = c = 1$)}. Available at: \url{https://github.com/jpascher/T0-Time-Mass-Duality/blob/main/2/pdf/DerivationVonBetaEn.pdf}
		
		\bibitem{pascher_feinstruktur_2025}
		Pascher, J. (2025). \textit{The Fine Structure Constant: Various Representations and Relationships - From Fundamental Physics to Natural Units}. Available at: \url{https://github.com/jpascher/T0-Time-Mass-Duality/blob/main/2/pdf/FeinstrukturkonstanteEn.pdf}
		
		\bibitem{pascher_proof_2025}
		Pascher, J. (2025). \textit{Mathematical Proof: The Fine Structure Constant $\alpha = 1$ in Natural Units}. Available at: \url{https://github.com/jpascher/T0-Time-Mass-Duality/blob/main/2/pdf/ResolvingTheConstantsAlfaEn.pdf}
		
		
		
		\bibitem{feynman_1985}
		Feynman, R. P. (1985). \textit{QED: The Strange Theory of Light and Matter}. Princeton University Press.
		
		\bibitem{pauli_1945}
		Pauli, W. (1945). \textit{Exclusion Principle and Quantum Mechanics}. Nobel Lecture.
		
		\bibitem{eddington_1929}
		Eddington, A. S. (1929). \textit{The Nature of the Physical World}. Cambridge University Press.
		
		\bibitem{codata_2018}
		CODATA Task Group on Fundamental Constants (2019). \textit{CODATA Recommended Values of the Fundamental Physical Constants: 2018}. Rev. Mod. Phys. 91, 025009.
		
		\bibitem{planck_collaboration_2020}
		Planck Collaboration (2020). \textit{Planck 2018 results. VI. Cosmological parameters}. Astron. Astrophys. 641, A6.
	\end{thebibliography}
	
\end{document}