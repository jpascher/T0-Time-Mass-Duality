\documentclass[12pt,a4paper]{article}
\usepackage[utf8]{inputenc}
\usepackage[T1]{fontenc}
\usepackage[english]{babel}
\usepackage[left=2cm,right=2cm,top=2cm,bottom=2cm]{geometry}
\usepackage{lmodern}
\usepackage{amsmath}
\usepackage{amssymb}
\usepackage{physics}
\usepackage{hyperref}
\usepackage{tcolorbox}
\usepackage{booktabs}
\usepackage{enumitem}
\usepackage[table,xcdraw]{xcolor}
\usepackage{graphicx}
\usepackage{float}
\usepackage{mathtools}
\usepackage{amsthm}
\usepackage{siunitx}
\usepackage{fancyhdr}
\usepackage{microtype}

% Headers and Footers
\pagestyle{fancy}
\fancyhf{}
\fancyhead[L]{Johann Pascher}
\fancyhead[R]{E=mc² = E=m: The Constants Illusion Exposed}
\fancyfoot[C]{\thepage}
\renewcommand{\headrulewidth}{0.4pt}
\renewcommand{\footrulewidth}{0.4pt}

% Custom Commands
\newcommand{\Tfield}{T}
\newcommand{\xipar}{\xi}

\hypersetup{
	colorlinks=true,
	linkcolor=blue,
	citecolor=blue,
	urlcolor=blue,
	pdftitle={E=mc² = E=m: The Constants Illusion Exposed},
	pdfauthor={Johann Pascher},
	pdfsubject={T0 Model, Einstein Critique, c-constant}
}

\newtheorem{theorem}{Theorem}[section]
\newtheorem{proposition}[theorem]{Proposition}
\newtheorem{definition}[theorem]{Definition}

\begin{document}
	
	\title{E=mc² = E=m: The Constants Illusion Exposed \\
		Why Einstein's c-constant conceals the fundamental error \\
		\large From Dynamic Ratios to the Constants Illusion}
	\author{Johann Pascher\\
		Department of Communications Engineering, \\Higher Technical Federal Institute (HTL), Leonding, Austria\\
		\texttt{johann.pascher@gmail.com}}
	\date{\today}
	
	\maketitle
	
	\begin{abstract}
		This work reveals the central point of Einstein's relativity theory: E=mc² is mathematically identical to E=m. The only difference lies in Einstein's treatment of c as a "constant" instead of a dynamic ratio. By fixing c = 299,792,458 m/s, the natural time-mass duality T·m = 1 is artificially "frozen," leading to apparent complexity. The T0 theory shows: c is not a fundamental law of nature, but only a ratio that must be variable if time is variable. Einstein's error was not E=mc² itself, but the constant-setting of c.
	\end{abstract}
	
	\tableofcontents
	\newpage
	
	\section{The Central Thesis: E=mc² = E=m}
	
	\begin{tcolorbox}[colback=red!5!white,colframe=red!75!black,title=The Fundamental Recognition]
		\textbf{E=mc² and E=m are mathematically identical!}
		
		The only difference: Einstein treats c as a "constant," although c is a dynamic ratio.
		
		\textbf{Einstein's error}: c = 299,792,458 m/s = constant
		
		\textbf{T0 truth}: c = L/T = variable ratio
	\end{tcolorbox}
	
	\subsection{The Mathematical Identity}
	
	\textbf{In natural units}:
	\begin{equation}
		E = mc^2 = m \times c^2 = m \times 1^2 = m
	\end{equation}
	
	\textbf{This is not an approximation - this is exactly the same equation!}
	
	\subsection{What is c really?}
	
	\begin{equation}
		c = \frac{\text{Length}}{\text{Time}} = \frac{L}{T}
	\end{equation}
	
	\textbf{c is a ratio, not a natural constant!}
	
	\section{Einstein's Fundamental Error: The Constant-Setting}
	
	\subsection{The Act of Constant-Setting}
	
	Einstein set: $c = 299,792,458$ m/s = \textbf{constant}
	
	\textbf{What does this mean?}
	\begin{equation}
		c = \frac{L}{T} = \text{constant} \quad \Rightarrow \quad \frac{L}{T} = \text{fixed}
	\end{equation}
	
	\textbf{Implication}: If L and T can vary, their \textbf{ratio} must remain constant.
	
	\subsection{The Problem of Time Variability}
	
	\textbf{Einstein recognized himself}: Time dilates!
	\begin{equation}
		t' = \gamma t \quad \text{(time is variable)}
	\end{equation}
	
	\textbf{But simultaneously he claimed}: 
	\begin{equation}
		c = \frac{L}{T} = \text{constant}
	\end{equation}
	
	\textbf{This is a logical contradiction!}
	
	\subsection{The T0 Resolution}
	
	\textbf{T0 insight}: $\Tfield \cdot m = 1$
	
	This means:
	\begin{itemize}
		\item Time $\Tfield$ \textbf{must} be variable (coupled to mass)
		\item Therefore $c = L/T$ \textbf{cannot} be constant
		\item $c$ is a \textbf{dynamic ratio}, not a constant
	\end{itemize}
	
	\section{The Constants Illusion: How it Works}
	
	\subsection{The Mechanism of the Illusion}
	
	\textbf{Step 1}: Einstein sets c = constant
	\begin{equation}
		c = 299,792,458 \text{ m/s} = \text{fixed}
	\end{equation}
	
	\textbf{Step 2}: Time becomes "frozen" by this
	\begin{equation}
		T = \frac{L}{c} = \frac{L}{\text{constant}} = \text{apparently determined}
	\end{equation}
	
	\textbf{Step 3}: Time dilation becomes "mysterious effect"
	\begin{equation}
		t' = \gamma t \quad \text{(why? $\rightarrow$ complicated relativity theory)}
	\end{equation}
	
	\subsection{What Really Happens (T0 View)}
	
	\textbf{Reality}: Time is naturally variable through $\Tfield \cdot m = 1$
	
	\textbf{Einstein's constant-setting} "freezes" this natural variability artificially
	
	\textbf{Result}: One needs complicated theory to repair the "frozen" dynamics
	
	\section{c as Ratio vs. c as Constant}
	
	\subsection{c as Natural Ratio (T0)}
	
	\begin{equation}
		c(x,t) = \frac{L(x,t)}{T(x,t)}
	\end{equation}
	
	\textbf{Properties}:
	\begin{itemize}
		\item $c$ varies with location and time
		\item $c$ follows the time-mass duality
		\item No artificial constants
		\item Natural simplicity: $E = m$
	\end{itemize}
	
	\subsection{c as Artificial Constant (Einstein)}
	
	\begin{equation}
		c = 299,792,458 \text{ m/s} = \text{constant everywhere}
	\end{equation}
	
	\textbf{Problems}:
	\begin{itemize}
		\item Contradiction to time dilation
		\item Artificial "freezing" of time dynamics
		\item Complicated repair mathematics needed
		\item Inflated formula: $E = mc^2$
	\end{itemize}
	
	\section{The Time Dilation Paradox}
	
	\subsection{Einstein's Contradiction Exposed}
	
	\textbf{Einstein claims simultaneously}:
	\begin{align}
		c &= \text{constant} \\
		t' &= \gamma t \quad \text{(time varies)}
	\end{align}
	
	\textbf{But}:
	\begin{equation}
		c = \frac{L}{T} \quad \text{and} \quad T \text{ varies} \quad \Rightarrow \quad c \text{ cannot be constant!}
	\end{equation}
	
	\subsection{Einstein's Hidden Solution}
	
	Einstein "solves" the contradiction through:
	\begin{itemize}
		\item Complicated Lorentz transformations
		\item Mathematical formalisms
		\item Space-time constructions
		\item \textbf{But the logical contradiction remains!}
	\end{itemize}
	
	\subsection{T0's Natural Solution}
	
	\textbf{No contradiction in T0}:
	\begin{equation}
		\Tfield \cdot m = 1 \quad \Rightarrow \quad \text{time is naturally variable}
	\end{equation}
	
	\begin{equation}
		c = \frac{L}{T} \quad \Rightarrow \quad \text{c is naturally variable}
	\end{equation}
	
	\textbf{No constant-setting $\rightarrow$ No contradictions $\rightarrow$ No complicated repair mathematics}
	
	\section{The Mathematical Demonstration}
	
	\subsection{From E=mc² to E=m}
	
	\textbf{Starting equation}: $E = mc^2$
	
	\textbf{c in natural units}: $c = 1$
	
	\textbf{Substitution}:
	\begin{equation}
		E = mc^2 = m \times 1^2 = m
	\end{equation}
	
	\textbf{Result}: $E = m$
	
	\subsection{The Reverse Direction: From E=m to E=mc²}
	
	\textbf{Starting equation}: $E = m$
	
	\textbf{Artificial constant introduction}: $c = 299,792,458$ m/s
	
	\textbf{Inflating the equation}:
	\begin{equation}
		E = m = m \times 1 = m \times \frac{c^2}{c^2} = m \times c^2 \times \frac{1}{c^2}
	\end{equation}
	
	\textbf{If one defines $c^2$ as "conversion factor"}:
	\begin{equation}
		E = mc^2
	\end{equation}
	
	\textbf{This shows}: $E = mc^2$ is only $E = m$ with \textbf{artificial inflation factor} $c^2$!
	
	\section{The Arbitrariness of Constant Choice: c or Time?}
	
	\subsection{Einstein's Arbitrary Decision}
	
	\begin{tcolorbox}[colback=orange!5!white,colframe=orange!75!black,title=The Fundamental Choice Option]
		\textbf{One can choose what should be "constant"!}
		
		\textbf{Option 1 (Einstein's choice)}: c = constant $\rightarrow$ time becomes variable
		
		\textbf{Option 2 (alternative)}: time = constant $\rightarrow$ c becomes variable
		
		\textbf{Both describe the same physics!}
	\end{tcolorbox}
	
	\subsection{Option 1: Einstein's c-constant}
	
	\textbf{Einstein chose}:
	\begin{align}
		c &= 299,792,458 \text{ m/s} = \text{constant (defined)} \\
		t' &= \gamma t \quad \text{(time becomes automatically variable)}
	\end{align}
	
	\textbf{Language convention}:
	\begin{itemize}
		\item "Speed of light is universally constant"
		\item "Time dilates in strong gravitational fields"
		\item "Clocks run slower at high velocities"
	\end{itemize}
	
	\subsection{Option 2: Time-constant (Einstein could have chosen)}
	
	\textbf{Alternative choice}:
	\begin{align}
		t &= \text{constant (defined)} \\
		c(x,t) &= \frac{L(x,t)}{t} = \text{variable}
	\end{align}
	
	\textbf{Alternative language convention}:
	\begin{itemize}
		\item "Time flows equally everywhere"
		\item "Speed of light varies with location"
		\item "Light becomes slower in strong gravitational fields"
	\end{itemize}
	
	\subsection{Mathematical Equivalence of Both Options}
	
	\textbf{Both descriptions are mathematically identical}:
	
	\begin{table}[htbp]
		\centering
		\begin{tabular}{|l|c|c|}
			\hline
			\textbf{Phenomenon} & \textbf{Einstein view} & \textbf{Time-constant view} \\
			\hline
			Gravitation & Time slows down & Light slows down \\
			Velocity & Time dilation & c-variation \\
			GPS correction & "Clocks run differently" & "c is different" \\
			Measurements & Same numbers & Same numbers \\
			\hline
		\end{tabular}
		\caption{Two views, identical physics}
	\end{table}
	
	\subsection{Why Einstein Chose Option 1}
	
	\textbf{Historical reasons for Einstein's decision}:
	\begin{itemize}
		\item \textbf{Michelson-Morley}: c seemed locally constant
		\item \textbf{Aesthetics}: "Universal constant" sounded elegant
		\item \textbf{Tradition}: Newtonian constant physics
		\item \textbf{Conceivability}: c-constancy easier to imagine than time constancy
		\item \textbf{Authority effect}: Einstein's prestige fixed this choice
	\end{itemize}
	
	\textbf{But it was only a convention, not a natural law!}
	
	\subsection{T0's Overcoming of Both Options}
	
	\textbf{T0 shows}: Both choices are arbitrary!
	
	\begin{equation}
		\Tfield \cdot m = 1 \quad \text{(natural duality without constant constraint)}
	\end{equation}
	
	\textbf{T0 insight}:
	\begin{itemize}
		\item \textbf{Neither} c nor time are "really" constant
		\item \textbf{Both} are aspects of the same T·m dynamics
		\item \textbf{Constancy} is only definition convention
		\item \textbf{E = m} is the constant-free truth
	\end{itemize}
	
	\subsection{Liberation from Constant Constraint}
	
	\textbf{Instead of choosing between}:
	\begin{itemize}
		\item c constant, time variable (Einstein)
		\item Time constant, c variable (alternative)
	\end{itemize}
	
	\textbf{T0 chooses}:
	\begin{itemize}
		\item \textbf{Both dynamically coupled} via T·m = 1
		\item \textbf{No arbitrary fixations}
		\item \textbf{Natural ratios} instead of artificial constants
	\end{itemize}
	
	\section{The Reference Point Revolution: Earth $\rightarrow$ Sun $\rightarrow$ Nature}
	
	\subsection{The Reference Point Analogy: Geocentric $\rightarrow$ Heliocentric $\rightarrow$ T0}
	
	\begin{tcolorbox}[colback=blue!5!white,colframe=blue!75!black,title=The Reference Point Revolution: From Earth $\rightarrow$ Sun $\rightarrow$ Nature]
		\textbf{Geocentric (Ptolemy)}: Earth at center \\
		- Complicated epicycles needed \\
		- Works, but artificially complicated \\
		
		\textbf{Heliocentric (Copernicus)}: Sun at center \\
		- Simple ellipses \\
		- Much more elegant and simple \\
		
		\textbf{T0-centric}: Natural ratios at center \\
		- $\Tfield \cdot m = 1$ (natural reference point) \\
		- Even more elegant: $E = m$
	\end{tcolorbox}
	
	\textbf{Einstein's c-constant corresponds to the geocentric system}:
	\begin{itemize}
		\item \textbf{Human} reference point at center (like Earth at center)
		\item \textbf{Complicated} mathematics needed (like epicycles)
		\item \textbf{Works} locally, but artificially inflated
	\end{itemize}
	
	\textbf{T0's natural ratios correspond to the heliocentric system}:
	\begin{itemize}
		\item \textbf{Natural} reference point at center (like Sun at center)
		\item \textbf{Simple} mathematics (like ellipses)
		\item \textbf{Universally} valid and elegant
	\end{itemize}
	
	\subsection{Why We Need Reference Points}
	
	\textbf{Reference points are necessary and natural}:
	\begin{itemize}
		\item \textbf{For measurements}: We need standards for comparison
		\item \textbf{For communication}: Common basis for exchange
		\item \textbf{For technology}: Practical applications require units
		\item \textbf{For science}: Reproducible experiments need standards
	\end{itemize}
	
	\textbf{The question is not WHETHER, but WHICH reference point}:
	
	\begin{table}[htbp]
		\centering
		\begin{tabular}{|l|c|c|c|}
			\hline
			\textbf{System} & \textbf{Reference Point} & \textbf{Complexity} & \textbf{Elegance} \\
			\hline
			Geocentric & Earth & Epicycles & Low \\
			Heliocentric & Sun & Ellipses & High \\
			Einstein & c-constant & Relativity theory & Medium \\
			T0 & $\Tfield \cdot m = 1$ & $E = m$ & Maximum \\
			\hline
		\end{tabular}
		\caption{Reference point systems comparison}
	\end{table}
	
	\subsection{The Right vs. Wrong Reference Point}
	
	\textbf{Einstein's error was not to choose a reference point}:
	\begin{itemize}
		\item \textbf{But to choose the wrong reference point!}
	\end{itemize}
	
	\textbf{Wrong reference point (Einstein)}: c = 299,792,458 m/s = constant
	\begin{itemize}
		\item Based on human definition
		\item Leads to complicated mathematics
		\item Creates logical contradictions
	\end{itemize}
	
	\textbf{Right reference point (T0)}: $\Tfield \cdot m = 1$
	\begin{itemize}
		\item Based on natural ratio
		\item Leads to simple mathematics: $E = m$
		\item No contradictions, pure elegance
	\end{itemize}
	
	\section{When Something Becomes "Constant"}
	
	\subsection{The Fundamental Reference Point Problem}
	
	\begin{tcolorbox}[colback=red!5!white,colframe=red!75!black,title=The Reference Point Illusion]
		\textbf{Something only becomes "constant" when we define a reference point!}
		
		\textbf{Without reference point}: All ratios are relative and dynamic
		
		\textbf{With reference point}: One ratio becomes artificially "fixed"
		
		\textbf{Einstein's error}: He defined an absolute reference point for c
	\end{tcolorbox}
	
	\subsection{The Natural Stage: Everything is Relative}
	
	\textbf{Before any reference point definition}:
	\begin{align}
		c_1 &= \frac{L_1}{T_1} \\
		c_2 &= \frac{L_2}{T_2} \\
		c_3 &= \frac{L_3}{T_3} \\
		&\vdots
	\end{align}
	
	\textbf{All c-values are relative to each other}. None is "constant".
	
	\subsection{The Moment of Reference Point Setting}
	
	\textbf{Einstein's fatal step}:
	\begin{equation}
		\text{"I define: } c = 299,792,458 \text{ m/s = reference point"}
	\end{equation}
	
	\textbf{What happens at this moment}:
	\begin{itemize}
		\item An \textbf{arbitrary reference point} is set
		\item All other c-values are measured relative to this
		\item The \textbf{dynamic ratio} becomes a "constant"
		\item The \textbf{natural relativity} is artificially "frozen"
	\end{itemize}
	
	\subsection{The Reference Point Problematic}
	
	\textbf{Every reference point is arbitrary}:
	\begin{itemize}
		\item Why 299,792,458 m/s and not 300,000,000 m/s?
		\item Why in m/s and not in other units?
		\item Why measured on Earth and not in space?
		\item Why at this time and not at another?
	\end{itemize}
	
	\subsection{T0's Reference Point-Free Physics}
	
	\textbf{T0 eliminates all reference points}:
	\begin{equation}
		\Tfield \cdot m = 1 \quad \text{(universal relation without reference point)}
	\end{equation}
	
	\begin{itemize}
		\item No arbitrary fixations
		\item All ratios remain dynamic
		\item Natural relativity is preserved
		\item Fundamental simplicity: $E = m$
	\end{itemize}
	
	\subsection{Example: The Meter Definition}
	
	\textbf{Historical development of meter definition}:
	\begin{enumerate}
		\item \textbf{1793}: 1 meter = 1/10,000,000 of Earth meridian (Earth reference point)
		\item \textbf{1889}: 1 meter = prototype meter in Paris (object reference point)  
		\item \textbf{1960}: 1 meter = 1,650,763.73 wavelengths of krypton-86 (atom reference point)
		\item \textbf{1983}: 1 meter = distance light travels in 1/299,792,458 s (c reference point)
	\end{enumerate}
	
	\textbf{What does this show?}
	\begin{itemize}
		\item Each definition is \textbf{human arbitrariness}
		\item The \textbf{reference point} changes with human technology
		\item There is \textbf{no "natural" length unit} - only human agreements
		\item \textbf{Humans make c "constant" by definition} - not nature!
	\end{itemize}
	
	\subsection{The Circular Error: Humans Define Their Own "Constants"}
	
	\textbf{In 1983 humans defined}:
	\begin{equation}
		1 \text{ meter} = \frac{1}{299,792,458} \times c \times 1 \text{ second}
	\end{equation}
	
	\textbf{This makes c automatically "constant"} - through human definition, not through natural law:
	\begin{equation}
		c = \frac{299,792,458 \text{ meters}}{1 \text{ second}} = 299,792,458 \text{ m/s}
	\end{equation}
	
	\textbf{Circular reasoning}: Humans define c as constant and then "measure" a constant!
	
	\textbf{Nature is not asked in this process!}
	
	\subsection{T0's Resolution of the Reference Point Illusion}
	
	\textbf{T0 recognizes}:
	\begin{itemize}
		\item \textbf{Definition $\neq$ natural law}
		\item \textbf{Measurement reference point $\neq$ physical constant}
		\item \textbf{Practical agreement $\neq$ fundamental truth}
	\end{itemize}
	
	\textbf{T0 solution}:
	\begin{align}
		\text{For measurements:} \quad &\text{Use practical reference points} \\
		\text{For natural laws:} \quad &\text{Use reference point-free relations}
	\end{align}
	
	\section{Why c-Constancy is Not Provable}
	
	\subsection{The Fundamental Measurement Problem}
	
	\textbf{To measure c, we need}:
	\begin{equation}
		c = \frac{L}{T}
	\end{equation}
	
	\textbf{But}: We measure L and T with \textbf{the same physical processes} that depend on c!
	
	\textbf{Circular problem}:
	\begin{itemize}
		\item Light measures distances $\rightarrow$ c determines L
		\item Atomic clocks use EM transitions $\rightarrow$ c influences T
		\item Then we measure c = L/T $\rightarrow$ \textbf{We measure c with c!}
	\end{itemize}
	
	\subsection{The Gauge Definition Problem}
	
	\textbf{Since 1983}: 1 meter = distance light travels in 1/299,792,458 s
	
	\begin{equation}
		c = 299,792,458 \text{ m/s} \quad \text{(not measured, but defined!)}
	\end{equation}
	
	\textbf{One cannot "prove" what one has defined!}
	
	\subsection{The Systematic Compensation Problem}
	
	\textbf{If c varies, ALL measuring devices vary equally}:
	\begin{itemize}
		\item \textbf{Laser interferometers}: use light (c-dependent)
		\item \textbf{Atomic clocks}: use EM transitions (c-dependent)
		\item \textbf{Electronics}: uses EM signals (c-dependent)
	\end{itemize}
	
	\textbf{Result}: All devices \textbf{automatically compensate} the c-variation!
	
	\subsection{The Burden of Proof Problem}
	
	\textbf{Scientifically correct}:
	\begin{itemize}
		\item One \textbf{cannot prove} that something is constant
		\item One can only show that it \textbf{appears constant within measurement precision}
		\item \textbf{Each new precision level} could show variation
	\end{itemize}
	
	\textbf{Einstein's "c-constancy" was belief, not proof!}
	
	\subsection{T0 Prediction for Precise Measurements}
	
	\textbf{T0 predicts}: At highest precision one will find:
	\begin{equation}
		c(x,t) = c_0 \left(1 + \xipar \times \frac{\Tfield(x,t) - \Tfield_0}{\Tfield_0}\right)
	\end{equation}
	
	with $\xipar = 1.33 \times 10^{-4}$ (T0 parameter)
	
	\textbf{c varies tiny ($\sim$10$^{-15}$), but measurable in principle!}
	
	\section{Ontological Consideration: Calculations as Constructs}
	
	\subsection{The Fundamental Epistemological Limit}
	
	\begin{tcolorbox}[colback=purple!5!white,colframe=purple!75!black,title=Ontological Truth]
		\textbf{All calculations are human constructs!}
		
		They can \textbf{at best} give a certain idea of reality.
		
		\textbf{That calculations are internally consistent proves little} about actual reality.
		
		\textbf{Mathematical consistency $\neq$ ontological truth}
	\end{tcolorbox}
	
	\subsection{Einstein's Construct vs. T0's Construct}
	
	\textbf{Both are human thought structures}:
	
	\textbf{Einstein's construct}:
	\begin{itemize}
		\item E = mc² (mathematically consistent)
		\item Relativity theory (internally coherent)
		\item 10 field equations (work computationally)
		\item \textbf{But}: Based on arbitrary c-constant setting
	\end{itemize}
	
	\textbf{T0's construct}:
	\begin{itemize}
		\item E = m (mathematically simpler)
		\item T·m = 1 (internally coherent)
		\item $\partial^2 E = 0$ (works computationally)
		\item \textbf{But}: Also only a human thought model
	\end{itemize}
	
	\subsection{The Ontological Relativity}
	
	\textbf{What is "really" real?}
	\begin{itemize}
		\item \textbf{Einstein's space-time}? (construct)
		\item \textbf{T0's energy field}? (construct)
		\item \textbf{Newton's absolute time}? (construct)
		\item \textbf{Quantum mechanics' probabilities}? (construct)
	\end{itemize}
	
	\textbf{All are human interpretive frameworks of the inaccessible reality!}
	
	\subsection{Why T0 is Still "Better"}
	
	\textbf{Not because of "absolute truth," but because of}:
	
	\textbf{1. Simplicity (Occam's Razor)}:
	\begin{itemize}
		\item E = m is simpler than E = mc²
		\item One equation is simpler than 10 equations
		\item Fewer arbitrary assumptions
	\end{itemize}
	
	\textbf{2. Consistency}:
	\begin{itemize}
		\item No logical contradictions (like Einstein's)
		\item No constant arbitrariness
		\item Unified thought structure
	\end{itemize}
	
	\textbf{3. Predictive power}:
	\begin{itemize}
		\item Testable predictions
		\item Fewer free parameters
		\item Clearer experimental distinction
	\end{itemize}
	
	\textbf{4. Aesthetics}:
	\begin{itemize}
		\item Mathematical elegance
		\item Conceptual clarity
		\item Unity
	\end{itemize}
	
	\subsection{The Epistemological Humility}
	
	\textbf{T0 does NOT claim to be "absolute truth."}
	
	\textbf{T0 only says}:
	\begin{itemize}
		\item "Here is a \textbf{simpler} construct"
		\item "With \textbf{fewer} arbitrary assumptions"
		\item "That is \textbf{more consistent} than Einstein's construct"
		\item "And makes \textbf{more testable} predictions"
	\end{itemize}
	
	\textbf{But ultimately T0 also remains a human thought structure!}
	
	\subsection{The Pragmatic Consequence}
	
	\textbf{Since all theories are constructs}:
	
	\textbf{Evaluation criteria are}:
	\begin{enumerate}
		\item \textbf{Simplicity} (fewer assumptions)
		\item \textbf{Consistency} (no contradictions)
		\item \textbf{Predictive power} (testable consequences)
		\item \textbf{Elegance} (aesthetic criteria)
		\item \textbf{Unity} (fewer separate domains)
	\end{enumerate}
	
	\textbf{By all these criteria T0 is "better" than Einstein - but not "absolutely true".}
	
	\subsection{The Ontological Humility}
	
	\textbf{The deepest insight}:
	\begin{itemize}
		\item \textbf{Reality itself} is inaccessible
		\item \textbf{All theories} are human constructs
		\item \textbf{Mathematical consistency} proves no ontological truth
		\item \textbf{The best} we have: \textbf{Simpler, more consistent constructs}
	\end{itemize}
	
	\textbf{Einstein's error was not only the c-constant setting, but also the claim to absolute truth of his mathematical constructs.}
	
	\textbf{T0's advantage is not absolute truth, but relative superiority as a thought model.}
	
	\section{The Practical Consequences}
	
	\subsection{Why E=mc² "Works"}
	
	\textbf{E=mc² works because}:
	\begin{itemize}
		\item It is mathematically identical to $E = m$
		\item $c^2$ compensates the "frozen" time dynamics
		\item The T0 truth is unconsciously contained
		\item Local approximations usually suffice
	\end{itemize}
	
	\subsection{When E=mc² Fails}
	
	\textbf{The constants illusion breaks down at}:
	\begin{itemize}
		\item Very precise measurements
		\item Extreme conditions (high energies/masses)
		\item Cosmological scales
		\item Quantum gravity
	\end{itemize}
	
	\subsection{T0's Universal Validity}
	
	\textbf{E = m is valid everywhere and always}:
	\begin{itemize}
		\item No approximations needed
		\item No constant assumptions
		\item Universal applicability
		\item Fundamental simplicity
	\end{itemize}
	
	\section{The Correction of Physics History}
	
	\subsection{Einstein's True Achievement}
	
	\textbf{Einstein's actual discovery was}:
	\begin{equation}
		E = m \quad \text{(in natural form)}
	\end{equation}
	
	\textbf{His error was}:
	\begin{equation}
		E = mc^2 \quad \text{(with artificial constant inflation)}
	\end{equation}
	
	\subsection{The Historical Irony}
	
	\begin{tcolorbox}[colback=blue!5!white,colframe=blue!75!black,title=The Great Irony]
		Einstein discovered the fundamental simplicity $E = m$, 
		
		but \textbf{hid it behind the constants illusion} $E = mc^2$!
		
		The physics world celebrated the complicated form and overlooked the simple truth.
	\end{tcolorbox}
	
	\section{The T0 Perspective: c as Living Ratio}
	
	\subsection{c as Expression of Time-Mass Duality}
	
	\textbf{In T0 theory}:
	\begin{equation}
		c(x,t) = f\left(\frac{L(x,t)}{\Tfield(x,t)}\right) = f\left(\frac{L(x,t) \cdot m(x,t)}{1}\right)
	\end{equation}
	
	since $\Tfield \cdot m = 1$.
	
	\textbf{c becomes an expression of the fundamental time-mass duality!}
	
	\subsection{The Dynamic Speed of Light}
	
	\textbf{T0 prediction}: 
	\begin{equation}
		c(x,t) = c_0 \sqrt{1 + \xipar \frac{m(x,t) - m_0}{m_0}}
	\end{equation}
	
	\textbf{Light moves faster in more massive regions!}
	
	(Tiny effect, but measurable in principle)
	
	\section{Experimental Tests of c-Variability}
	
	\subsection{Proposed Experiments}
	
	\textbf{Test 1 - Gravitational dependence}:
	\begin{itemize}
		\item Measure c in different gravitational fields
		\item T0 prediction: $c$ varies with $\sim \xipar \times \Delta\Phi_{\text{grav}}$
	\end{itemize}
	
	\textbf{Test 2 - Cosmological variation}:
	\begin{itemize}
		\item Measure c over cosmological time periods
		\item T0 prediction: $c$ changes with universe expansion
	\end{itemize}
	
	\textbf{Test 3 - High-energy physics}:
	\begin{itemize}
		\item Measure c in particle accelerators at highest energies
		\item T0 prediction: Tiny deviations at $E \sim$ TeV
	\end{itemize}
	
	\subsection{Expected Results}
	
	\begin{table}[htbp]
		\centering
		\begin{tabular}{|l|c|c|}
			\hline
			\textbf{Experiment} & \textbf{Einstein (c constant)} & \textbf{T0 (c variable)} \\
			\hline
			Gravitational field & $c = 299792458$ m/s & $c(1 \pm 10^{-15})$ \\
			Cosmological time & $c = $ constant & $c(1 + 10^{-12} \times t)$ \\
			High energy & $c = $ constant & $c(1 + 10^{-16})$ \\
			\hline
		\end{tabular}
		\caption{Predicted c-variations}
	\end{table}
	
	\section{Conclusions}
	
	\subsection{The Central Recognition}
	
	\begin{tcolorbox}[colback=green!5!white,colframe=green!75!black,title=The Fundamental Truth]
		\textbf{E=mc² = E=m}
		
		Einstein's "constant" c is in truth a variable ratio.
		
		The constant-setting was Einstein's fundamental error.
		
		T0 corrects this error by returning to natural variability.
	\end{tcolorbox}
	
	\subsection{Physics After the Constants Illusion}
	
	\textbf{The future of physics}:
	\begin{itemize}
		\item No artificial constants
		\item Dynamic ratios everywhere
		\item Living, variable natural laws
		\item Fundamental simplicity: $E = m$
	\end{itemize}
	
	\subsection{Einstein's Corrected Legacy}
	
	\textbf{Einstein's true discovery}: $E = m$ (energy-mass identity)
	
	\textbf{Einstein's error}: Constant-setting of c
	
	\textbf{T0's correction}: Return to natural form $E = m$
	
	\textbf{Einstein was brilliant - he just stopped one step too early!}
	\begin{thebibliography}{99}
		\bibitem{einstein1905}
		Einstein, A. (1905). \textit{Does the inertia of a body depend upon its energy content?} Annalen der Physik, 18, 639--641.
		
		\bibitem{michelson1887}
		Michelson, A. A. and Morley, E. W. (1887). \textit{On the relative motion of the Earth and the luminiferous ether}. American Journal of Science, 34, 333--345.
		
		\bibitem{pascher_derivation_beta_2025}
		Pascher, J. (2025). \textit{Field-Theoretic Derivation of the $\beta_T$ Parameter in Natural Units}. T0 Model Documentation.
		
		\bibitem{pascher_simplified_dirac_2025}
		Pascher, J. (2025). \textit{Simplified Dirac Equation in T0 Theory}. T0 Model Documentation.
		
		\bibitem{pascher_ratio_physics_2025}
		Pascher, J. (2025). \textit{Pure Energy T0 Theory: The Ratio-Based Revolution}. T0 Model Documentation.
		
		\bibitem{planck1900}
		Planck, M. (1900). \textit{On the theory of the energy distribution law of the normal spectrum}. Verhandlungen der Deutschen Physikalischen Gesellschaft, 2, 237--245.
		
		\bibitem{lorentz1904}
		Lorentz, H. A. (1904). \textit{Electromagnetic phenomena in a system moving with any velocity smaller than that of light}. Proceedings of the Royal Netherlands Academy of Arts and Sciences, 6, 809--831.
		
		\bibitem{weinberg1972}
		Weinberg, S. (1972). \textit{Gravitation and Cosmology}. John Wiley \& Sons.
	\end{thebibliography}
\end{document}