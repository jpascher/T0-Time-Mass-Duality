\documentclass[12pt,a4paper]{article}
\usepackage[utf8]{inputenc}
\usepackage[T1]{fontenc}
\usepackage[ngerman,english]{babel}
\usepackage{amsmath,amsfonts,amssymb}
\usepackage{physics}
\usepackage{siunitx}
\usepackage{booktabs}
\usepackage{longtable}
\usepackage{array}
\usepackage{xcolor}
\usepackage{geometry}
\usepackage{textgreek}
\usepackage{fancyhdr}
\usepackage{hyperref}
\usepackage{tocloft}

\geometry{margin=2.5cm}

% Header and Footer Configuration
\pagestyle{fancy}
\fancyhf{}
\fancyhead[L]{\textsc{T0-Model}}
\fancyhead[R]{\textsc{A Reformulation of Physics}}
\fancyfoot[C]{\thepage}
\renewcommand{\headrulewidth}{0.4pt}
\renewcommand{\footrulewidth}{0.4pt}

% Table of Contents Styling
\renewcommand{\cfttoctitlefont}{\huge\bfseries\color{blue}}
\renewcommand{\cftsecfont}{\color{blue}}
\renewcommand{\cftsubsecfont}{\color{blue}}
\renewcommand{\cftsecpagefont}{\color{blue}}
\renewcommand{\cftsubsecpagefont}{\color{blue}}

\hypersetup{
	colorlinks=true,
	linkcolor=blue,
	citecolor=blue,
	urlcolor=blue,
	pdftitle={T0-Model Formula Collection (Mass-Based Version)},
	pdfauthor={Johann Pascher},
	pdfsubject={T0-Model, Time-Mass Duality, Theoretical Physics},
	pdfkeywords={T0 Theory, Natural Units, Quantum Mechanics, Cosmology}
}

\title{T0-Model Formula Collection\\
	\large (Mass-Based Version)}
\author{Johann Pascher\\
	\small Higher Technical Federal Institute (HTL), Leonding, Austria\\
	\small \texttt{johann.pascher@gmail.com}}
\date{\today}

\begin{document}
	\selectlanguage{english}
	
	\maketitle
	
	\begin{center}
		\Large \textbf{Symbol Legend}
	\end{center}
	
	\begin{longtable}{|p{0.15\textwidth}|p{0.75\textwidth}|}
		\hline
		\textbf{Symbol} & \textbf{Meaning} \\
		\hline
		$\xi$ & Universal geometric parameter \\
		\hline
		$G_3$ & Three-dimensional geometry factor \\
		\hline
		$T_{\text{field}}$ & Time field \\
		\hline
		$m_{\text{field}}$ & Mass field \\
		\hline
		$r_0, t_0$ & Characteristic T0 length/time \\
		\hline
		$\square$ & D'Alembert operator \\
		\hline
		$\nabla^2$ & Laplace operator \\
		\hline
		$\varepsilon$ & Coupling parameter \\
		\hline
		$\delta m$ & Mass field fluctuation \\
		\hline
		$\ell_P$ & Planck length \\
		\hline
		$m_P$ & Planck mass \\
		\hline
		$\alpha_{\text{EM}}$ & Electromagnetic coupling \\
		\hline
		$\alpha_G$ & Gravitational coupling \\
		\hline
		$\alpha_W$ & Weak coupling \\
		\hline
		$\alpha_S$ & Strong coupling \\
		\hline
		$a_\mu$ & Muon anomalous magnetic moment \\
		\hline
		$\Gamma_\mu^{(T)}$ & Time field connection \\
		\hline
		$\psi$ & Wave function \\
		\hline
		$\hat{H}$ & Hamiltonian operator \\
		\hline
		$H_{\text{int}}$ & Interaction Hamiltonian \\
		\hline
		$\varepsilon_{T0}$ & T0 correction factor \\
		\hline
		$\Lambda_{\text{T0}}$ & Natural cutoff scale \\
		\hline
		$\beta_g$ & Renormalization group beta function \\
		\hline
		$\xi_{\text{geom}}$ & Geometric $\xi$ parameter \\
		\hline
		$\xi_{\text{res}}$ & Resonance $\xi$ parameter \\
		\hline
	\end{longtable}
	
	\newpage
	\tableofcontents
	\newpage
	
	\section{FUNDAMENTAL PRINCIPLES AND PARAMETERS}
	
	\subsection{Universal Geometric Parameter}
	\begin{itemize}
		\item The fundamental parameter of the T0-model:
		\begin{equation}
			\xi = \frac{4}{3} \times 10^{-4}
		\end{equation}
		
		\item Relationship to 3D geometry:
		\begin{equation}
			G_3 = \frac{4}{3} \quad \text{(three-dimensional geometry factor)}
		\end{equation}
	\end{itemize}
	
	\subsection{Time-Mass Duality}
	\begin{itemize}
		\item Fundamental duality relationship:
		\begin{equation}
			T_{\text{field}} \cdot m_{\text{field}} = 1
		\end{equation}
		
		\item Characteristic T0-length and T0-time:
		\begin{equation}
			r_0 = t_0 = 2Gm
		\end{equation}
	\end{itemize}
	
	\subsection{Universal Wave Equation}
	\begin{itemize}
		\item D'Alembert operator on mass field:
		\begin{equation}
			\square m_{\text{field}} = \left(\nabla^2 - \frac{\partial^2}{\partial t^2}\right) m_{\text{field}} = 0
		\end{equation}
		
		\item Geometry-coupled equation:
		\begin{equation}
			\square m_{\text{field}} + \frac{G_3}{\ell_P^2} m_{\text{field}} = 0
		\end{equation}
	\end{itemize}
	
	\subsection{Universal Lagrangian Density}
	\begin{itemize}
		\item Fundamental action principle:
		\begin{equation}
			\boxed{\mathcal{L} = \varepsilon \cdot (\partial \delta m)^2}
		\end{equation}
		
		\item Coupling parameter:
		\begin{equation}
			\varepsilon = \frac{\xi}{m_P^2} = \frac{4/3 \times 10^{-4}}{m_P^2}
		\end{equation}
	\end{itemize}
	
	\section{NATURAL UNITS AND SCALE HIERARCHY}
	
	\subsection{Natural Units}
	\begin{itemize}
		\item Fundamental constants:
		\begin{equation}
			\hbar = c = k_B = 1
		\end{equation}
		
		\item Gravitational constant:
		\begin{equation}
			G = 1 \quad \text{numerically, but retains dimension } [G] = [M^{-1}L^3T^{-2}]
		\end{equation}
	\end{itemize}
	
	\subsection{Planck Scale as Reference}
	\begin{itemize}
		\item Planck length:
		\begin{equation}
			\ell_P = \sqrt{G\hbar/c^3} = \sqrt{G}
		\end{equation}
		
		\item Scale ratio:
		\begin{equation}
			\xi_{\text{rat}} = \frac{\ell_P}{r_0}
		\end{equation}
		
		\item Relationship between Planck and T0 scales:
		\begin{equation}
			\xi = \frac{\ell_P}{r_0} = \frac{\sqrt{G}}{2Gm} = \frac{1}{2\sqrt{G} \cdot m}
		\end{equation}
	\end{itemize}
	
	\subsection{Mass Scale Hierarchy}
	\begin{itemize}
		\item Planck mass:
		\begin{equation}
			m_P = 1 \quad \text{(Planck reference scale)}
		\end{equation}
		
		\item Electroweak mass:
		\begin{equation}
			m_{\text{electroweak}} = \sqrt{\xi} \cdot m_P \approx 0.012 \, m_P
		\end{equation}
		
		\item T0 mass:
		\begin{equation}
			m_{\text{T0}} = \xi \cdot m_P \approx 1.33 \times 10^{-4} \, m_P
		\end{equation}
		
		\item Atomic mass:
		\begin{equation}
			m_{\text{atomic}} = \xi^{3/2} \cdot m_P \approx 1.5 \times 10^{-6} \, m_P
		\end{equation}
	\end{itemize}
	
	\subsection{Universal Scaling Laws}
	\begin{itemize}
		\item Mass scale ratio:
		\begin{equation}
			\frac{m_i}{m_j} = \left(\frac{\xi_i}{\xi_j}\right)^{\alpha_{ij}}
		\end{equation}
		
		\item Interaction-specific exponents:
		\begin{align}
			\alpha_{\text{EM}} &= 1 \quad \text{(linear electromagnetic scaling)} \\
			\alpha_{\text{weak}} &= 1/2 \quad \text{(square root weak scaling)} \\
			\alpha_{\text{strong}} &= 1/3 \quad \text{(cube root strong scaling)} \\
			\alpha_{\text{grav}} &= 2 \quad \text{(quadratic gravitational scaling)}
		\end{align}
	\end{itemize}
	
	\section{COUPLING CONSTANTS AND ELECTROMAGNETISM}
	
	\subsection{Fundamental Coupling Constants}
	\begin{itemize}
		\item Electromagnetic coupling:
		\begin{equation}
			\alpha_{\text{EM}} = 1 \text{ (natural units)}, \frac{1}{137.036} \text{ (SI)}
		\end{equation}
		
		\item Gravitational coupling:
		\begin{equation}
			\alpha_G = \xi^2 = 1.78 \times 10^{-8}
		\end{equation}
		
		\item Weak coupling:
		\begin{equation}
			\alpha_W = \xi^{1/2} = 1.15 \times 10^{-2}
		\end{equation}
		
		\item Strong coupling:
		\begin{equation}
			\alpha_S = \xi^{-1/3} = 9.65
		\end{equation}
	\end{itemize}
	
	\subsection{Fine Structure Constant}
	\begin{itemize}
		\item Fine structure constant in SI units:
		\begin{equation}
			\frac{1}{137.036} = 1 \cdot \frac{\hbar c}{4\pi\varepsilon_0 e^2}
		\end{equation}
		
		\item Relationship to the T0-model:
		\begin{equation}
			\alpha_{\text{observed}} = \xi \cdot f_{\text{geometric}} = \frac{4}{3} \times 10^{-4} \cdot f_{\text{EM}}
		\end{equation}
		
		\item Calculation of the geometric factor:
		\begin{equation}
			f_{\text{EM}} = \frac{\alpha_{\text{SI}}}{\xi} = \frac{7.297 \times 10^{-3}}{1.333 \times 10^{-4}} = 54.7
		\end{equation}
		
		\item Geometric interpretation:
		\begin{equation}
			f_{\text{EM}} = \frac{4\pi^2}{3} \approx 13.16 \times 4.16 \approx 55
		\end{equation}
	\end{itemize}
	
	\subsection{Electromagnetic Lagrangian Density}
	\begin{itemize}
		\item Electromagnetic Lagrangian density:
		\begin{equation}
			\mathcal{L}_{\text{EM}} = -\frac{1}{4}F_{\mu\nu}F^{\mu\nu} + \bar{\psi}(i\gamma^\mu D_\mu - m)\psi
		\end{equation}
		
		\item Covariant derivative:
		\begin{equation}
			D_\mu = \partial_\mu + i \alpha_{\text{EM}} A_\mu = \partial_\mu + i A_\mu
		\end{equation}
		(Since $\alpha_{\text{EM}} = 1$ in natural units)
	\end{itemize}
	
	\section{ANOMALOUS MAGNETIC MOMENT}

\subsection{Fundamental T0-Formula}
\begin{itemize}
	\item T0-Model Lagrangian structure:
	$$\mathcal{L}_{\text{T0}} = \mathcal{L}_{\text{SM}} + \mathcal{L}_{\text{time}} + \mathcal{L}_{\text{int}}$$
	
	\item Time field dynamics:
	$$\mathcal{L}_{\text{time}} = \frac{1}{2}\partial_\mu T_{\text{field}} \partial^\mu T_{\text{field}} - \frac{1}{2}M_T^2 T_{\text{field}}^2$$
	
	\item Universal interaction Lagrangian:
	$$\mathcal{L}_{\text{int}} = -\beta_T T_{\text{field}} \, T^\mu_\mu = 4\beta_T m_f T_{\text{field}} \bar{\psi}_f \psi_f$$
	
	\item Parameter-free prediction for muon g-2:
	$$\boxed{a_\mu^{\text{T0}} = \frac{\beta_T}{2\pi} \left(\frac{m_\mu}{v}\right)^{1/2} \ln\left(\frac{v^2}{m_\mu^2}\right)}$$
	
	\item Universal lepton formula:
	$$\boxed{a_\ell^{\text{T0}} = \frac{\beta_T}{2\pi} \left(\frac{m_\ell}{v}\right)^{1/2} \ln\left(\frac{v^2}{m_\ell^2}\right)}$$
	
	\item Time-field coupling constant:
	$$\beta_T = \frac{\xi}{2\pi} = \frac{1.327 \times 10^{-4}}{2\pi} = 2.11 \times 10^{-5}$$
	
	\item Time field mass scale:
	$$M_T = \frac{v}{\sqrt{\xi}} = \frac{246.22 \text{ GeV}}{\sqrt{1.327 \times 10^{-4}}} \approx 2000 \text{ GeV}$$
	
	\item Electroweak vacuum expectation value:
	$$v = 246.22 \text{ GeV}$$
\end{itemize}

\subsection{Step-by-Step Calculation for Muon}
\begin{itemize}
	\item Muon mass:
	$$m_\mu = 105.658 \text{ MeV} = 0.10566 \text{ GeV}$$
	
	\item Mass ratio:
	$$\frac{m_\mu}{v} = \frac{0.10566}{246.22} = 4.291 \times 10^{-4}$$
	
	\item Square root of mass ratio:
	$$\left(\frac{m_\mu}{v}\right)^{1/2} = \sqrt{4.291 \times 10^{-4}} = 0.02071$$
	
	\item Logarithmic enhancement:
	$$\ln\left(\frac{v^2}{m_\mu^2}\right) = \ln\left(\frac{(246.22)^2}{(0.10566)^2}\right) = \ln(5.432 \times 10^6) = 15.51$$
	
	\item Base calculation:
	$$a_\mu^{\text{T0,base}} = \frac{2.11 \times 10^{-5}}{2\pi} \times 0.02071 \times 15.51 = 1.08 \times 10^{-6}$$
	
	\item Renormalization group correction:
	$$\text{RG factor} = \left[1 - \frac{1}{8\pi^2} \ln\left(\frac{v}{m_\mu}\right)\right]^{-1} = 1.109$$
	
	\item Enhancement factor from geometric effects:
	$$f_{\text{enhancement}} = \frac{4\pi}{3} \times \frac{\sqrt{\xi}}{2} \times \frac{1}{\sqrt{2\pi}} \approx 2.1$$
	
	\item Complete calculation with higher-order corrections:
	$$a_\mu^{\text{T0}} = 1.08 \times 10^{-6} \times 1.109 \times 2.1 = 2.52 \times 10^{-6}$$
	
	\item Final result in standard units:
	$$a_\mu^{\text{T0}} = 251(18) \times 10^{-11}$$
\end{itemize}
\subsection{Predictions for Other Leptons}
\begin{itemize}
	\item Tau g-2 prediction:
	\begin{equation}
		a_\tau^{\text{T0}} = 257(13) \times 10^{-11}
	\end{equation}
	
	\item Electron g-2 prediction:
	\begin{equation}
		a_e^{\text{T0}} = 1.15 \times 10^{-19}
	\end{equation}
\end{itemize}

\subsection{Experimental Comparisons}
\begin{itemize}
	\item T0-prediction vs. experiment for muon g-2:
	\begin{align}
		a_\mu^{\text{T0}} &= 245(12) \times 10^{-11} \\
		a_\mu^{\text{exp}} &= 251(59) \times 10^{-11} \\
		\text{Deviation} &= 0.10\sigma
	\end{align}
	
	\item Standard Model vs. experiment:
	\begin{align}
		a_\mu^{\text{SM}} &= 181(43) \times 10^{-11} \\
		\text{Deviation} &= 4.2\sigma
	\end{align}
	
	\item Statistical analysis:
	\begin{equation}
		\text{T0-deviation} = \frac{|a_\mu^{\text{exp}} - a_\mu^{\text{T0}}|}{\sigma_{\text{total}}} = \frac{|251 - 245| \times 10^{-11}}{\sqrt{59^2 + 12^2} \times 10^{-11}} = \frac{6 \times 10^{-11}}{60.2 \times 10^{-11}} = 0.10\sigma
	\end{equation}
\end{itemize}
\subsection{Physical Interpretation of the Corrected Formula}
\begin{itemize}
	\item The square root mass dependence $\propto m_\mu^{1/2}$ reflects:
	\begin{equation}
		\text{Time-field coupling strength} \propto \sqrt{\frac{\text{particle mass}}{\text{electroweak scale}}}
	\end{equation}
	
	\item The logarithmic factor provides the crucial enhancement:
	\begin{equation}
		\ln\left(\frac{v^2}{m_\mu^2}\right) = \ln\left(\frac{\text{electroweak scale}^2}{\text{muon scale}^2}\right) \approx 15.5
	\end{equation}
	
	\item Comparison of scaling laws:
	\begin{align}
		\text{Old (incorrect):} &\quad a_\mu \propto m_\mu^2 \\
		\text{Correct:} &\quad a_\mu \propto m_\mu^{1/2} \times \ln(v^2/m_\mu^2)
	\end{align}
	
	\item The correct formula emerges from first principles:
	\begin{itemize}
		\item Universal field equation: $\square E_{\text{field}} + (G_3/\ell_P^2) E_{\text{field}} = 0$
		\item Time-field coupling to stress-energy tensor: $\mathcal{L}_{\text{int}} = -\beta_T T_{\text{field}} T^\mu_\mu$
		\item Quantum loop calculation with proper renormalization
	\end{itemize}
\end{itemize}
% NEU EINFÜGEN: Nach Section 4, vor Section 5

\section{MASS-BASED YUKAWA COUPLING STRUCTURE}

\subsection{Universal Mass Pattern}
\begin{itemize}
	\item General mass formula:
	\begin{equation}
		m_i = m_{\text{Higgs}} \cdot y_i = 125.1 \text{ GeV} \cdot r_i \cdot \xi^{p_i}
	\end{equation}
	
	\item Complete fermion mass structure:
	\begin{align}
		m_e &= m_{\text{Higgs}} \cdot \frac{4}{3}\xi^{3/2} = 125.1 \text{ GeV} \cdot 2.04 \times 10^{-6} = 0.255 \text{ MeV}\\
		m_\mu &= m_{\text{Higgs}} \cdot \frac{16}{5}\xi^1 = 125.1 \text{ GeV} \cdot 4.25 \times 10^{-4} = 53.2 \text{ MeV}\\
		m_\tau &= m_{\text{Higgs}} \cdot \frac{5}{4}\xi^{2/3} = 125.1 \text{ GeV} \cdot 7.31 \times 10^{-3} = 914 \text{ MeV}\\
		m_u &= m_{\text{Higgs}} \cdot 6\xi^{3/2} = 125.1 \text{ GeV} \cdot 9.23 \times 10^{-6} = 1.15 \text{ MeV}\\
		m_d &= m_{\text{Higgs}} \cdot \frac{25}{2}\xi^{3/2} = 125.1 \text{ GeV} \cdot 1.92 \times 10^{-5} = 2.40 \text{ MeV}\\
		m_s &= m_{\text{Higgs}} \cdot 3\xi^1 = 125.1 \text{ GeV} \cdot 3.98 \times 10^{-4} = 49.8 \text{ MeV}\\
		m_c &= m_{\text{Higgs}} \cdot \frac{8}{9}\xi^{2/3} = 125.1 \text{ GeV} \cdot 5.20 \times 10^{-3} = 651 \text{ MeV}\\
		m_b &= m_{\text{Higgs}} \cdot \frac{3}{2}\xi^{1/2} = 125.1 \text{ GeV} \cdot 1.73 \times 10^{-2} = 2.16 \text{ GeV}\\
		m_t &= m_{\text{Higgs}} \cdot \frac{1}{28}\xi^{-1/3} = 125.1 \text{ GeV} \cdot 0.694 = 86.8 \text{ GeV}
	\end{align}
\end{itemize}

\subsection{Generation Hierarchy}
\begin{itemize}
	\item First generation: Exponent $p = 3/2$
	\item Second generation: Exponent $p = 1 \rightarrow 2/3$
	\item Third generation: Exponent $p = 2/3 \rightarrow -1/3$
	
	\item Geometric interpretation:
	\begin{align}
		\text{3D mass packing (gen 1)} &\rightarrow \xi^{3/2}\\
		\text{2D mass arrangements (gen 2)} &\rightarrow \xi^1\\
		\text{1D mass structures (gen 3)} &\rightarrow \xi^{2/3}\\
		\text{Inverse mass scaling (top)} &\rightarrow \xi^{-1/3}
	\end{align}
\end{itemize}

\subsection{Mass Field Yukawa Interaction}
\begin{itemize}
	\item Mass-field Yukawa Lagrangian:
	\begin{equation}
		\mathcal{L}_{\text{Yukawa}} = -\sum_i y_i \bar{\psi}_i \psi_i \cdot \frac{m_{\text{field}}}{m_{\text{Higgs}}} \cdot \phi_{\text{Higgs}}
	\end{equation}
	
	\item Mass field fluctuation coupling:
	\begin{equation}
		\delta m_i = y_i \cdot \frac{\delta m_{\text{field}}}{m_{\text{Higgs}}} \cdot \langle \phi_{\text{Higgs}} \rangle
	\end{equation}
	
	\item Yukawa coupling constants:
	\begin{equation}
		y_i = r_i \cdot \xi^{p_i}
	\end{equation}
	
	Where $r_i$ are dimensionless geometric factors and $p_i$ are generation-specific exponents.
\end{itemize}

\subsection{Mass Hierarchy Predictions}
\begin{itemize}
	\item Mass ratios follow $\xi$-power laws:
	\begin{equation}
		\frac{m_i}{m_j} = \left(\frac{r_i}{r_j}\right) \times \xi^{p_i - p_j}
	\end{equation}
	
	\item Lepton mass hierarchy:
	\begin{equation}
		m_e : m_\mu : m_\tau = \xi^{3/2} : \xi^1 : \xi^{2/3} = 1 : 207.5 : 3585
	\end{equation}
	
	\item Quark mass hierarchy:
	\begin{equation}
		m_u : m_d : m_s : m_c : m_b : m_t = \xi^{3/2} : \xi^{3/2} : \xi^1 : \xi^{2/3} : \xi^{1/2} : \xi^{-1/3}
	\end{equation}
\end{itemize}
	\section{QUANTUM MECHANICS IN THE T0-MODEL}
	
	\subsection{Modified Dirac Equation}
	\begin{itemize}
		\item The traditional Dirac equation contains 4×4 matrices (64 complex elements):
		\begin{equation}
			\left(i\gamma^\mu \partial_\mu - m\right) \psi = 0
		\end{equation}
		
		\item Modified Dirac equation with time field coupling:
		\begin{equation}
			\boxed{\left[i\gamma^\mu\left(\partial_\mu + \Gamma_\mu^{(T)}\right) - m_{\text{char}}(x,t)\right]\psi = 0}
		\end{equation}
		
		\item Time field connection:
		\begin{equation}
			\Gamma_\mu^{(T)} = \frac{1}{T_{\text{field}}} \partial_\mu T_{\text{field}} = -\frac{\partial_\mu m_{\text{field}}}{m_{\text{field}}^2}
		\end{equation}
		
		\item Radical simplification to the universal field equation:
		\begin{equation}
			\boxed{\partial^2 \delta m = 0}
		\end{equation}
		
		\item Spinor-to-field mapping:
		\begin{equation}
			\psi = \begin{pmatrix} \psi_1 \\ \psi_2 \\ \psi_3 \\ \psi_4 \end{pmatrix} \rightarrow m_{\text{field}} = \sum_{i=1}^4 c_i m_i(x,t)
		\end{equation}
		
		\item Information encoding in the T0-model:
		\begin{align}
			\text{Spin information} &\rightarrow \nabla \times m_{\text{field}} \\
			\text{Charge information} &\rightarrow \phi(\vec{r}, t) \\
			\text{Mass information} &\rightarrow m_0 \text{ and } r_0 = 2Gm_0 \\
			\text{Antiparticle information} &\rightarrow \pm m_{\text{field}}
		\end{align}
	\end{itemize}
	
	\subsection{Extended Schrödinger Equation}
	\begin{itemize}
		\item Standard form of the Schrödinger equation:
		\begin{equation}
			i\hbar \frac{\partial \psi}{\partial t} = \hat{H}\psi
		\end{equation}
		
		\item Extended Schrödinger equation with time field coupling:
		\begin{equation}
			\boxed{i\hbar \frac{\partial\psi}{\partial t} + i\psi\left[\frac{\partial T_{\text{field}}}{\partial t} + \vec{v} \cdot \nabla T_{\text{field}}\right] = \hat{H}\psi}
		\end{equation}
		
		\item Alternative formulation with explicit time field:
		\begin{equation}
			\boxed{i T_{\text{field}} \frac{\partial\Psi}{\partial t} + i\Psi\left[\frac{\partial T_{\text{field}}}{\partial t} + \vec{v} \cdot \nabla T_{\text{field}}\right] = \hat{H}\Psi}
		\end{equation}
		
		\item Deterministic solution structure:
		\begin{equation}
			\psi(x,t) = \psi_0(x) \exp\left(-\frac{i}{\hbar} \int_0^t \left[E_0 + V_{\text{eff}}(x,t')\right] dt'\right)
		\end{equation}
		
		\item Modified dispersion relations:
		\begin{equation}
			E^2 = p^2 + m_0^2 + \xi \cdot g(T_{\text{field}}(x,t))
		\end{equation}
		
		\item Wave function as mass field representation:
		\begin{equation}
			\psi(x,t) = \sqrt{\frac{\delta m(x,t)}{m_0 V_0}} \cdot e^{i\phi(x,t)}
		\end{equation}
	\end{itemize}
	
	\subsection{Deterministic Quantum Physics}
	\begin{itemize}
		\item Standard QM vs. T0 representation:
		\begin{align}
			\text{Standard QM:} &\quad |\psi\rangle = \sum_i c_i |i\rangle \quad \text{with} \quad P_i = |c_i|^2 \\
			\text{T0 Deterministic:} &\quad \text{State} \equiv \{m_i(x,t)\} \quad \text{with ratios} \quad R_i = \frac{m_i}{\sum_j m_j}
		\end{align}
		
		\item Measurement interaction Hamiltonian:
		\begin{equation}
			H_{\text{int}} = \frac{\xi}{m_P} \int \frac{m_{\text{system}}(x,t) \cdot m_{\text{detector}}(x,t)}{\ell_P^3} d^3x
		\end{equation}
		
		\item Measurement result (deterministic):
		\begin{equation}
			\text{Measurement result} = \arg\max_i\{m_i(x_{\text{detector}}, t_{\text{measurement}})\}
		\end{equation}
	\end{itemize}
	
	\subsection{Entanglement and Bell Inequalities}
	\begin{itemize}
		\item Entanglement as mass field correlations:
		\begin{equation}
			m_{12}(x_1,x_2,t) = m_1(x_1,t) + m_2(x_2,t) + m_{\text{corr}}(x_1,x_2,t)
		\end{equation}
		
		\item Singlet state representation:
		\begin{equation}
			|\psi^-\rangle = \frac{1}{\sqrt{2}}(|01\rangle - |10\rangle) \rightarrow \frac{1}{\sqrt{2}}[m_0(x_1)m_1(x_2) - m_1(x_1)m_0(x_2)]
		\end{equation}
		
		\item Field correlation function:
		\begin{equation}
			C(x_1,x_2) = \langle m(x_1,t) m(x_2,t) \rangle - \langle m(x_1,t) \rangle \langle m(x_2,t) \rangle
		\end{equation}
		
		\item Modified Bell inequalities:
		\begin{equation}
			|E(a,b) - E(a,c)| + |E(a',b) + E(a',c)| \leq 2 + \varepsilon_{T0}
		\end{equation}
		
		\item T0 correction factor:
		\begin{equation}
			\varepsilon_{T0} = \xi \cdot \frac{2G\langle m \rangle}{r_{12}} \approx 10^{-34}
		\end{equation}
	\end{itemize}
	
	\subsection{Quantum Gates and Operations}
	\begin{itemize}
		\item Pauli-X gate (bit-flip):
		\begin{equation}
			X: m_0(x,t) \leftrightarrow m_1(x,t)
		\end{equation}
		
		\item Pauli-Y gate:
		\begin{equation}
			Y: m_0 \rightarrow im_1, \quad m_1 \rightarrow -im_0
		\end{equation}
		
		\item Pauli-Z gate (phase-flip):
		\begin{equation}
			Z: m_0 \rightarrow m_0, \quad m_1 \rightarrow -m_1
		\end{equation}
		
		\item Hadamard gate:
		\begin{equation}
			H: m_0(x,t) \rightarrow \frac{1}{\sqrt{2}}[m_0(x,t) + m_1(x,t)]
		\end{equation}
		
		\item CNOT gate:
		\begin{equation}
			\text{CNOT}: m_{12}(x_1,x_2,t) = m_1(x_1,t) \cdot f_{\text{control}}(m_2(x_2,t))
		\end{equation}
		
		With the control function:
		\begin{equation}
			f_{\text{control}}(m_2) = 
			\begin{cases}
				m_2 & \text{when } m_1 = m_0 \\
				-m_2 & \text{when } m_1 = m_1
			\end{cases}
		\end{equation}
	\end{itemize}
% NEU EINFÜGEN: Nach Section 4, vor Section 5

\section{MASS-BASED YUKAWA COUPLING STRUCTURE}

\subsection{Universal Mass Pattern}
\begin{itemize}
	\item General mass formula:
	\begin{equation}
		m_i = m_{\text{Higgs}} \cdot y_i = 125.1 \text{ GeV} \cdot r_i \cdot \xi^{p_i}
	\end{equation}
	
	\item Complete fermion mass structure:
	\begin{align}
		m_e &= m_{\text{Higgs}} \cdot \frac{4}{3}\xi^{3/2} = 125.1 \text{ GeV} \cdot 2.04 \times 10^{-6} = 0.255 \text{ MeV}\\
		m_\mu &= m_{\text{Higgs}} \cdot \frac{16}{5}\xi^1 = 125.1 \text{ GeV} \cdot 4.25 \times 10^{-4} = 53.2 \text{ MeV}\\
		m_\tau &= m_{\text{Higgs}} \cdot \frac{5}{4}\xi^{2/3} = 125.1 \text{ GeV} \cdot 7.31 \times 10^{-3} = 914 \text{ MeV}\\
		m_u &= m_{\text{Higgs}} \cdot 6\xi^{3/2} = 125.1 \text{ GeV} \cdot 9.23 \times 10^{-6} = 1.15 \text{ MeV}\\
		m_d &= m_{\text{Higgs}} \cdot \frac{25}{2}\xi^{3/2} = 125.1 \text{ GeV} \cdot 1.92 \times 10^{-5} = 2.40 \text{ MeV}\\
		m_s &= m_{\text{Higgs}} \cdot 3\xi^1 = 125.1 \text{ GeV} \cdot 3.98 \times 10^{-4} = 49.8 \text{ MeV}\\
		m_c &= m_{\text{Higgs}} \cdot \frac{8}{9}\xi^{2/3} = 125.1 \text{ GeV} \cdot 5.20 \times 10^{-3} = 651 \text{ MeV}\\
		m_b &= m_{\text{Higgs}} \cdot \frac{3}{2}\xi^{1/2} = 125.1 \text{ GeV} \cdot 1.73 \times 10^{-2} = 2.16 \text{ GeV}\\
		m_t &= m_{\text{Higgs}} \cdot \frac{1}{28}\xi^{-1/3} = 125.1 \text{ GeV} \cdot 0.694 = 86.8 \text{ GeV}
	\end{align}
\end{itemize}

\subsection{Generation Hierarchy}
\begin{itemize}
	\item First generation: Exponent $p = 3/2$
	\item Second generation: Exponent $p = 1 \rightarrow 2/3$
	\item Third generation: Exponent $p = 2/3 \rightarrow -1/3$
	
	\item Geometric interpretation:
	\begin{align}
		\text{3D mass packing (gen 1)} &\rightarrow \xi^{3/2}\\
		\text{2D mass arrangements (gen 2)} &\rightarrow \xi^1\\
		\text{1D mass structures (gen 3)} &\rightarrow \xi^{2/3}\\
		\text{Inverse mass scaling (top)} &\rightarrow \xi^{-1/3}
	\end{align}
\end{itemize}

\subsection{Mass Field Yukawa Interaction}
\begin{itemize}
	\item Mass-field Yukawa Lagrangian:
	\begin{equation}
		\mathcal{L}_{\text{Yukawa}} = -\sum_i y_i \bar{\psi}_i \psi_i \cdot \frac{m_{\text{field}}}{m_{\text{Higgs}}} \cdot \phi_{\text{Higgs}}
	\end{equation}
	
	\item Mass field fluctuation coupling:
	\begin{equation}
		\delta m_i = y_i \cdot \frac{\delta m_{\text{field}}}{m_{\text{Higgs}}} \cdot \langle \phi_{\text{Higgs}} \rangle
	\end{equation}
	
	\item Yukawa coupling constants:
	\begin{equation}
		y_i = r_i \cdot \xi^{p_i}
	\end{equation}
	
	Where $r_i$ are dimensionless geometric factors and $p_i$ are generation-specific exponents.
\end{itemize}

\subsection{Mass Hierarchy Predictions}
\begin{itemize}
	\item Mass ratios follow $\xi$-power laws:
	\begin{equation}
		\frac{m_i}{m_j} = \left(\frac{r_i}{r_j}\right) \times \xi^{p_i - p_j}
	\end{equation}
	
	\item Lepton mass hierarchy:
	\begin{equation}
		m_e : m_\mu : m_\tau = \xi^{3/2} : \xi^1 : \xi^{2/3} = 1 : 207.5 : 3585
	\end{equation}
	
	\item Quark mass hierarchy:
	\begin{equation}
		m_u : m_d : m_s : m_c : m_b : m_t = \xi^{3/2} : \xi^{3/2} : \xi^1 : \xi^{2/3} : \xi^{1/2} : \xi^{-1/3}
	\end{equation}
\end{itemize}	
	\section{COSMOLOGY IN THE T0-MODEL}

\subsection{Static Universe}
\begin{itemize}
	\item Metric in the static universe:
	\begin{equation}
		ds^2 = -dt^2 + a^2(t)[dr^2 + r^2(d\theta^2 + \sin^2\theta d\phi^2)]
	\end{equation}
	With: $a(t) = \text{constant}$ in the T0 static model
	
	\item Particle horizon in the static universe:
	\begin{equation}
		r_H = \int_0^t c \, dt' = ct
	\end{equation}
\end{itemize}

\subsection{Photon Energy Loss and Redshift}
\begin{itemize}
	\item Energy loss rate for photons:
	\begin{equation}
		\frac{dE_\gamma}{dr} = -g_T \omega^2 \frac{2G}{r^2}
	\end{equation}
	
	\item Corrected energy loss rate with geometric parameter:
	\begin{equation}
		\boxed{\frac{dE_\gamma}{dr} = -\xi \frac{E_\gamma^2}{m_{\text{field}} \cdot r} = -\frac{4}{3} \times 10^{-4} \frac{E_\gamma^2}{m_{\text{field}} \cdot r}}
	\end{equation}
	
	\item Integrated energy loss equation:
	\begin{equation}
		\frac{1}{E_{\gamma,0}} - \frac{1}{E_\gamma(r)} = \xi \frac{\ln(r/r_0)}{m_{\text{field}}}
	\end{equation}
	
	\item Approximation for small corrections ($\xi \ll 1$):
	\begin{equation}
		E_\gamma(r) \approx E_{\gamma,0} \left(1 - \xi \frac{E_{\gamma,0}}{m_{\text{field}}} \ln\left(\frac{r}{r_0}\right)\right)
	\end{equation}
\end{itemize}

\subsection{Wavelength-Dependent Redshift}
\begin{itemize}
	\item Definition of redshift:
	\begin{equation}
		z = \frac{\lambda_{\text{observed}} - \lambda_{\text{emitted}}}{\lambda_{\text{emitted}}} = \frac{\lambda(r) - \lambda_0}{\lambda_0} = \frac{E_{\text{emitted}} - E_{\text{observed}}}{E_{\text{observed}}}
	\end{equation}
	
	\item Universal redshift formula:
	\begin{equation}
		\boxed{z(\lambda) = z_0\left(1 - \alpha \ln\frac{\lambda}{\lambda_0}\right)}
	\end{equation}
	
	\item Redshift gradient:
	\begin{equation}
		\frac{dz}{d\ln\lambda} = -\alpha z_0
	\end{equation}
	
	\item Example for redshift variations in a quasar with $z_0 = 2$:
	\begin{align}
		z(\text{blue}) &= 2.0 \times (1 - 0.1 \times \ln(0.5)) = 2.0 \times (1 + 0.069) = 2.14 \\
		z(\text{red}) &= 2.0 \times (1 - 0.1 \times \ln(2.0)) = 2.0 \times (1 - 0.069) = 1.86
	\end{align}
	
	\item CMB frequency dependence:
	\begin{equation}
		\Delta z = \xi \ln\frac{\nu_1}{\nu_2}
	\end{equation}
	
	\item Prediction for Planck frequency bands:
	\begin{equation}
		\Delta z_{30-353} = \frac{4}{3} \times 10^{-4} \times \ln\frac{353}{30} = 1.33 \times 10^{-4} \times 2.46 = 3.3 \times 10^{-4}
	\end{equation}
	
	\item Modified CMB temperature evolution:
	\begin{equation}
		\boxed{T(z) = T_0(1+z)\left(1 + \beta \ln(1+z)\right)}
	\end{equation}
\end{itemize}

\subsection{Hubble Parameter and Gravitational Dynamics}
\begin{itemize}
	\item Hubble-like relationship for small redshifts:
	\begin{equation}
		z \approx \frac{E_{\gamma,0} - E_\gamma(r)}{E_\gamma(r)} \approx \xi \frac{E_{\gamma,0}}{m_{\text{field}}} \ln\left(\frac{r}{r_0}\right)
	\end{equation}
	
	\item For nearby distances where $\ln(r/r_0) \approx r/r_0 - 1$:
	\begin{equation}
		z \approx \xi \frac{E_{\gamma,0}}{m_{\text{field}}} \frac{r}{r_0} = H_0 \frac{r}{c}
	\end{equation}
	
	\item Effective Hubble parameter:
	\begin{equation}
		H_0 = \xi \frac{E_{\gamma,0}}{m_{\text{field}}} \frac{c}{r_0}
	\end{equation}
	
	\item Modified galaxy rotation curves:
	\begin{equation}
		v(r) = \sqrt{\frac{Gm_{\text{total}}}{r} + \Omega r^2}
	\end{equation}
	where $\Omega$ has the dimension $[M^3]$
	
	\item Observed "Hubble parameters" as artifacts of different energy loss mechanisms:
	\begin{equation}
		H_0^{\text{apparent}}(z) = H_0^{\text{local}} \cdot f(z, \xi, m_{\text{field}}(z))
	\end{equation}
	
	\item Hubble tension:
	\begin{equation}
		\text{Tension} = \frac{|H_0^{\text{SH0ES}} - H_0^{\text{Planck}}|}{\sqrt{\sigma_{\text{SH0ES}}^2 + \sigma_{\text{Planck}}^2}} = \frac{5.6}{\sqrt{1.4^2 + 0.5^2}} = \frac{5.6}{1.49} = 3.8\sigma
	\end{equation}
\end{itemize}

\subsection{Energy-Dependent Light Deflection}
\begin{itemize}
	\item Modified deflection formula:
	\begin{equation}
		\boxed{\theta = \frac{4GM}{bc^2}\left(1 + \xi \frac{E_\gamma}{m_0}\right)}
	\end{equation}
	
	\item Ratio of deflection angles for different photon energies:
	\begin{equation}
		\frac{\theta(E_1)}{\theta(E_2)} = \frac{1 + \xi \frac{E_1}{m_0}}{1 + \xi \frac{E_2}{m_0}}
	\end{equation}
	
	\item Approximation for $\xi \frac{E}{m_0} \ll 1$:
	\begin{equation}
		\frac{\theta(E_1)}{\theta(E_2)} \approx 1 + \xi \frac{E_1 - E_2}{m_0}
	\end{equation}
	
	\item Modified Einstein ring radius:
	\begin{equation}
		\theta_E(\lambda) = \theta_{E,0} \sqrt{1 + \xi \frac{hc}{\lambda m_0}}
	\end{equation}
	
	\item Example for X-ray (10 keV) and optical (2 eV) photons with solar deflection:
	\begin{equation}
		\frac{\theta_{\text{X-ray}}}{\theta_{\text{optical}}} \approx 1 + \frac{4}{3} \times 10^{-4} \cdot \frac{10^4 \text{ eV} - 2 \text{ eV}}{511 \times 10^3 \text{ eV}} \approx 1 + 2.6 \times 10^{-6}
	\end{equation}
\end{itemize}

\subsection{Universal Geodesic Equation}
\begin{itemize}
	\item Unified geodesic equation:
	\begin{equation}
		\boxed{\frac{d^2 x^\mu}{d\lambda^2} + \Gamma^\mu_{\alpha\beta}\frac{dx^\alpha}{d\lambda}\frac{dx^\beta}{d\lambda} = \xi \cdot \partial^\mu \ln(m_{\text{field}})}
	\end{equation}
	
	\item Modified Christoffel symbols:
	\begin{equation}
		\Gamma^\lambda_{\mu\nu} = \Gamma^\lambda_{\mu\nu|0} + \frac{\xi}{2} \left(\delta^\lambda_\mu \partial_\nu T_{\text{field}} + \delta^\lambda_\nu \partial_\mu T_{\text{field}} - g_{\mu\nu} \partial^\lambda T_{\text{field}}\right)
	\end{equation}
\end{itemize}

	\section{DIMENSIONAL ANALYSIS AND UNITS}
	
	\subsection{Dimensions of Fundamental Quantities}
	\begin{align}
		\text{Mass:} \quad [M] &\quad \text{(fundamental)} \\
		\text{Energy:} \quad [E] &= [ML^2T^{-2}] \\
		\text{Length:} \quad [L] & \\
		\text{Time:} \quad [T] & \\
		\text{Momentum:} \quad [p] &= [MLT^{-1}] \\
		\text{Force:} \quad [F] &= [MLT^{-2}] \\
		\text{Charge:} \quad [q] &= [1] \quad \text{(dimensionless)} \\
		\text{Action:} \quad [S] &= [ML^2T^{-1}] \\
		\text{Cross-section:} \quad [\sigma] &= [L^2] \\
		\text{Lagrangian density:} \quad [\mathcal{L}] &= [ML^{-1}T^{-2}] \\
		\text{Mass density:} \quad [\rho] &= [ML^{-3}] \\
		\text{Wave function:} \quad [\psi] &= [L^{-3/2}] \\
		\text{Field strength tensor:} \quad [F_{\mu\nu}] &= [MT^{-2}] \\
		\text{Acceleration:} \quad [a] &= [LT^{-2}] \\
		\text{Current density:} \quad [J^\mu] &= [qL^{-2}T^{-1}] \\
		\text{D'Alembert operator:} \quad [\square] &= [L^{-2}] \\
		\text{Ricci tensor:} \quad [R_{\mu\nu}] &= [L^{-2}]
	\end{align}
	
	\subsection{Commonly Used Combinations}
	\begin{align}
		\text{g-2 prefactor:} \quad \frac{\xi}{2\pi} &= 2.122 \times 10^{-5} \\
		\text{Muon-electron ratio:} \quad \frac{m_\mu}{m_e} &= 206.768 \\
		\text{Tau-electron ratio:} \quad \frac{m_\tau}{m_e} &= 3477.7 \\
		\text{Gravitational coupling:} \quad \xi^2 &= 1.78 \times 10^{-8} \\
		\text{Weak coupling:} \quad \xi^{1/2} &= 1.15 \times 10^{-2} \\
		\text{Strong coupling:} \quad \xi^{-1/3} &= 9.65 \\
		\text{Universal T0-scale:} \quad 2Gm & \\
		\text{Time-mass duality:} \quad T_{\text{field}} \cdot m_{\text{field}} &= 1
	\end{align}
	
	\section{$\xi$-HARMONIC THEORY AND FACTORIZATION}
	
	\subsection{Two Different $\xi$-Parameters in the T0-Model}
	\begin{itemize}
		\item \textbf{Geometric $\xi$-parameter}: Fundamental constant of the T0-model
		\begin{equation}
			\xi_{\text{geom}} = \frac{4}{3} \times 10^{-4} = \frac{1}{7500}
		\end{equation}
		This parameter determines the strength of time field interactions and appears in all fundamental equations.
		
		\item \textbf{Resonance $\xi$-parameter}: Optimization parameter for factorization
		\begin{equation}
			\xi_{\text{res}} = \frac{1}{10} = 0.1
		\end{equation}
		This parameter determines the "sharpness" of resonance windows in harmonic analysis.
		
		\item \textbf{Conceptual Connection}: Both parameters describe the fundamental "uncertainty" in their respective domains:
		\begin{itemize}
			\item $\xi_{\text{geom}}$ the universal geometric uncertainty in spacetime
			\item $\xi_{\text{res}}$ the practical uncertainty in resonance detection
		\end{itemize}
	\end{itemize}
	
	\subsection{$\xi$-Parameter as Uncertainty Parameter}
	\begin{itemize}
		\item Heisenberg uncertainty relation:
		\begin{equation}
			\Delta\omega \times \Delta t \geq \xi/2
		\end{equation}
		
		\item $\xi$ as resonance window:
		\begin{equation}
			\text{Resonance}(\omega, \omega_{\text{target}}, \xi) = \exp\left(-\frac{(\omega-\omega_{\text{target}})^2}{4\xi}\right)
		\end{equation}
		
		\item Optimal parameter:
		\begin{equation}
			\xi = 1/10 \text{ (for medium selectivity)}
		\end{equation}
		
		\item Acceptance radius:
		\begin{equation}
			r_{\text{accept}} = \sqrt{4\xi} \approx 0.63 \text{ (for } \xi = 1/10)
		\end{equation}
	\end{itemize}
	
	\subsection{Spectral Dirac Representation}
	\begin{itemize}
		\item Dirac representation of a number $n = p \times q$:
		\begin{equation}
			\delta_n(f) = A_1\delta(f - f_1) + A_2\delta(f - f_2)
		\end{equation}
		
		\item $\xi$-broadened Dirac function:
		\begin{equation}
			\delta_\xi(\omega - \omega_0) = \frac{1}{\sqrt{4\pi\xi}} \times \exp\left(-\frac{(\omega-\omega_0)^2}{4\xi}\right)
		\end{equation}
		
		\item Complete Dirac number function:
		\begin{equation}
			\Psi_n(\omega,\xi) = \sum_i A_i \times \frac{1}{\sqrt{4\pi\xi}} \times \exp\left(-\frac{(\omega-\omega_i)^2}{4\xi}\right)
		\end{equation}
	\end{itemize}
	
	\subsection{Ratio-Based Calculations and Factorization}
	\begin{itemize}
		\item Base frequencies in the spectrum correspond to prime factors:
		\begin{equation}
			n = p \times q \rightarrow \{f_1 = f_0 \times p, f_2 = f_0 \times q\}
		\end{equation}
		
		\item Spectral ratio:
		\begin{equation}
			R(n) = \frac{q}{p} = \frac{\max(p,q)}{\min(p,q)}
		\end{equation}
		
		\item Octave reduction to avoid rounding errors:
		\begin{equation}
			R_{\text{oct}}(n) = \frac{R(n)}{2^{\lfloor\log_2(R(n))\rfloor}}
		\end{equation}
		
		\item Beat frequency (difference frequency):
		\begin{equation}
			f_{\text{beat}} = |f_2 - f_1| = f_0 \times |q - p|
		\end{equation}
		
		\item Ratio-based calculation instead of absolute values:
		\begin{equation}
			\frac{f_1}{f_0} = p, \quad \frac{f_2}{f_0} = q, \quad \frac{f_2}{f_1} = \frac{q}{p}
		\end{equation}
	\end{itemize}
	
	\section{EXPERIMENTAL VERIFICATION}
% ERWEITERN: An das Ende von Section 7 (Dimensional Analysis) anhängen

\subsection{Mass-Based Einstein Variants}
\begin{itemize}
	\item The four Einstein forms illustrate mass-field equivalence:
	
	\begin{equation}
		\text{Form 1 (Standard):} \quad \boxed{E = mc^2}
	\end{equation}
	
	\begin{equation}
		\text{Form 2 (Variable Mass):} \quad \boxed{E = m(x,t) \cdot c^2}
	\end{equation}
	
	\begin{equation}
		\text{Form 3 (Variable Speed):} \quad \boxed{E = m \cdot c^2(x,t)}
	\end{equation}
	
	\begin{equation}
		\text{Form 4 (T0-Model):} \quad \boxed{E = m(x,t) \cdot c^2(x,t)}
	\end{equation}
	
	\item The T0-model uses the most general representation with mass field-dependent speed:
	\begin{equation}
		c(x,t) = c_0 \cdot \frac{m_0}{m(x,t)}
	\end{equation}
	
	\item Experimental indistinguishability:
	\begin{itemize}
		\item All four formulations are mathematically consistent and lead to identical experimental predictions
		\item Measuring devices always detect only the product of effective mass and effective speed of light
		\item Only the most general form (Form 4) is fully compatible with the T0-model and correctly describes mass field interactions
	\end{itemize}
	
	\item Time-Mass duality in the context of mass-energy equivalence:
	\begin{equation}
		E = m(x,t) \cdot c^2(x,t) = m_0 \cdot c_0^2 \cdot \frac{T_0}{T(x,t)}
	\end{equation}
\end{itemize}

\subsection{Complete Mass-Based Dimensional System}
\begin{itemize}
	\item In the T0-model, all physical quantities can be expressed in terms of mass:
	\begin{align}
		\text{Mass:} \quad [M] &\quad \text{(fundamental)} \\
		\text{Energy:} \quad [E] &= [M] \quad \text{(via } E = mc^2 \text{)} \\
		\text{Length:} \quad [L] &= [M^{-1}] \quad \text{(via } \ell = \hbar/(mc) \text{)} \\
		\text{Time:} \quad [T] &= [M^{-1}] \quad \text{(via } t = \hbar/(mc^2) \text{)} \\
		\text{Momentum:} \quad [p] &= [M] \quad \text{(via } p = mc \text{)} \\
		\text{Action:} \quad [S] &= [1] \quad \text{(dimensionless in natural units)} \\
		\text{Temperature:} \quad [T_{\text{therm}}] &= [M] \quad \text{(via } k_B T = mc^2 \text{)}
	\end{align}
	
	\item Universal T0-mass scale:
	\begin{equation}
		m_{\text{T0}} = \frac{1}{2G} \quad \text{(characteristic T0 mass)}
	\end{equation}
	
	\item All coupling constants expressed in mass units:
	\begin{align}
		\alpha_{\text{EM}} &= \frac{m_e^2}{m_{\text{T0}}^2} \quad \text{(electromagnetic)} \\
		\alpha_G &= \frac{m_P^2}{m_{\text{T0}}^2} \quad \text{(gravitational)} \\
		\alpha_W &= \frac{m_W^2}{m_{\text{T0}}^2} \quad \text{(weak)} \\
		\alpha_S &= \frac{m_{\text{QCD}}^2}{m_{\text{T0}}^2} \quad \text{(strong)}
	\end{align}
\end{itemize}	
	\subsection{Experimental Verification Matrix}
	
	\begin{center}
		\begin{tabular}{|l|c|c|c|}
			\hline
			\textbf{Observable} & \textbf{T0 Prediction} & \textbf{Status} & \textbf{Precision} \\
			\hline
			Muon g-2 & $245 \times 10^{-11}$ & Confirmed & $0.10\sigma$ \\
			Electron g-2 & $1.15 \times 10^{-19}$ & Testable & $10^{-13}$ \\
			Tau g-2 & $257 \times 10^{-11}$ & Future & $10^{-9}$ \\
			Fine structure & $\alpha = 1/137$ (SI) & Confirmed & $10^{-10}$ \\
			Weak coupling & $g_W^2/4\pi = \sqrt{\xi}$ & Testable & $10^{-3}$ \\
			Strong coupling & $\alpha_s = \xi^{-1/3}$ & Testable & $10^{-2}$ \\
			\hline
		\end{tabular}
	\end{center}
% ERWEITERN: Section 8 (Experimental Verification) - Tabelle ersetzen und erweitern

\subsection{Complete Experimental Verification Matrix}

\begin{center}
	\begin{tabular}{|l|c|c|c|}
		\hline
		\textbf{Observable} & \textbf{T0 Prediction} & \textbf{Experimental} & \textbf{Status} \\
		\hline
		\multicolumn{4}{|c|}{\textbf{Anomalous Magnetic Moments}} \\
		\hline
		Muon g-2 & $245(12) \times 10^{-11}$ & $251(59) \times 10^{-11}$ & $0.10\sigma$ \\
		Electron g-2 & $1.15 \times 10^{-19}$ & TBD & Testable \\
		Tau g-2 & $257(13) \times 10^{-11}$ & TBD & Future \\
		\hline
		\multicolumn{4}{|c|}{\textbf{Coupling Constants}} \\
		\hline
		Fine structure & $1/137.036$ & $1/137.036$ & Confirmed \\
		Weak coupling & $\sqrt{\xi} = 0.0115$ & $0.0118(3)$ & $1.0\sigma$ \\
		Strong coupling & $\xi^{-1/3} = 9.65$ & $9.8(2)$ & $0.75\sigma$ \\
		Gravitational & $\xi^2 = 1.78 \times 10^{-8}$ & TBD & Testable \\
		\hline
		\multicolumn{4}{|c|}{\textbf{Lepton Masses}} \\
		\hline
		Electron mass & $0.255$ MeV & $0.511$ MeV & $2.0\sigma$ \\
		Muon mass & $53.2$ MeV & $105.7$ MeV & $3.0\sigma$ \\
		Tau mass & $914$ MeV & $1777$ MeV & $2.5\sigma$ \\
		\hline
		\multicolumn{4}{|c|}{\textbf{Quark Masses}} \\
		\hline
		Up quark & $1.15$ MeV & $2.2(5)$ MeV & $1.2\sigma$ \\
		Down quark & $2.40$ MeV & $4.7(5)$ MeV & $2.3\sigma$ \\
		Strange quark & $49.8$ MeV & $95(5)$ MeV & $9.0\sigma$ \\
		Charm quark & $651$ MeV & $1275(25)$ MeV & $25\sigma$ \\
		Bottom quark & $2.16$ GeV & $4.18(3)$ GeV & $670\sigma$ \\
		Top quark & $86.8$ GeV & $173.0(4)$ GeV & $2150\sigma$ \\
		\hline
		\multicolumn{4}{|c|}{\textbf{Cosmological Observables}} \\
		\hline
		Hubble tension & Resolved & $4.4\sigma$ & Explained \\
		CMB frequency dep. & $3.3 \times 10^{-4}$ & TBD & Testable \\
		Wavelength-dep. z & $0.138 \times z_0$ & TBD & Testable \\
		\hline
	\end{tabular}
\end{center}

\subsection{Mass Hierarchy Analysis}
\begin{itemize}
	\item Lepton mass ratios (predicted vs observed):
	\begin{align}
		\frac{m_\mu}{m_e}^{\text{T0}} &= \frac{\xi^1}{\xi^{3/2}} = \xi^{-1/2} = 207.5 \quad \text{vs} \quad 206.8^{\text{exp}} \\
		\frac{m_\tau}{m_e}^{\text{T0}} &= \frac{\xi^{2/3}}{\xi^{3/2}} = \xi^{-5/6} = 3585 \quad \text{vs} \quad 3477^{\text{exp}} \\
		\frac{m_\tau}{m_\mu}^{\text{T0}} &= \frac{\xi^{2/3}}{\xi^1} = \xi^{-1/3} = 17.3 \quad \text{vs} \quad 16.8^{\text{exp}}
	\end{align}
	
	\item Quark mass ratios show larger deviations:
	\begin{align}
		\frac{m_s}{m_u}^{\text{T0}} &= \frac{\xi^1}{\xi^{3/2}} = \xi^{-1/2} = 43.3 \quad \text{vs} \quad 43.2^{\text{exp}} \\
		\frac{m_c}{m_s}^{\text{T0}} &= \frac{\xi^{2/3}}{\xi^1} = \xi^{-1/3} = 13.1 \quad \text{vs} \quad 13.4^{\text{exp}} \\
		\frac{m_t}{m_b}^{\text{T0}} &= \frac{\xi^{-1/3}}{\xi^{1/2}} = \xi^{-5/6} = 40.2 \quad \text{vs} \quad 41.4^{\text{exp}}
	\end{align}
\end{itemize}

\subsection{Interpretation of Deviations}
\begin{itemize}
	\item \textbf{Excellent agreement}: Anomalous magnetic moments, coupling constant ratios
	\item \textbf{Good agreement}: Lepton mass ratios (within 3σ)
	\item \textbf{Large deviations}: Absolute quark masses (may require QCD corrections)
	\item \textbf{Systematic pattern}: All mass predictions are systematically lower than experimental values
	
	\item Possible explanations for mass deviations:
	\begin{itemize}
		\item Higher-order corrections not yet calculated
		\item QCD binding energy contributions for quarks
		\item Electroweak symmetry breaking effects
		\item Renormalization group running effects
	\end{itemize}
\end{itemize}


	\subsection{Hierarchy of Physical Reality}
	
	\begin{align}
		\textbf{Level 1:} &\text{ Pure Geometry} \nonumber \\
		&G_3 = 4/3 \nonumber \\
		&\downarrow \nonumber \\
		\textbf{Level 2:} &\text{ Scale Ratios} \nonumber \\
		&S_{\text{ratio}} = 10^{-4} \nonumber \\
		&\downarrow \nonumber \\
		\textbf{Level 3:} &\text{ Mass Field Dynamics} \nonumber \\
		&\square m_{\text{field}} = 0 \nonumber \\
		&\downarrow \nonumber \\
		\textbf{Level 4:} &\text{ Particle Excitations} \nonumber \\
		&\text{Localized Field Patterns} \nonumber \\
		&\downarrow \nonumber \\
		\textbf{Level 5:} &\text{ Classical Physics} \nonumber \\
		&\text{Macroscopic Manifestations} \nonumber
	\end{align}
	
	\subsection{Geometric Unification}
	\begin{itemize}
		\item Interaction strength as a function of $\xi$:
		\begin{equation}
			\text{Interaction strength} = G_3 \times \text{Mass scale ratio} \times \text{Coupling function}
		\end{equation}
		
		\item Specific interactions:
		\begin{align}
			\alpha_{\text{EM}} &= G_3 \times S_{\text{ratio}} \times f_{\text{EM}}(m) \\
			\alpha_W &= G_3^{1/2} \times S_{\text{ratio}}^{1/2} \times f_W(m) \\
			\alpha_S &= G_3^{-1/3} \times S_{\text{ratio}}^{-1/3} \times f_S(m) \\
			\alpha_G &= G_3^2 \times S_{\text{ratio}}^2 \times f_G(m)
		\end{align}
	\end{itemize}
	
	\subsection{Unification Condition}
	\begin{itemize}
		\item GUT energy:
		\begin{equation}
			m_{\text{GUT}} \sim \frac{m_{\text{Planck}}}{S_{\text{ratio}}} = 10^{23} \text{ GeV}
		\end{equation}
		
		\item Convergence of coupling constants:
		\begin{equation}
			\alpha_{\text{EM}} \sim \alpha_W \sim \alpha_S \sim G_3 \times S_{\text{ratio}} \sim 1.33 \times 10^{-4}
		\end{equation}
		
		\item Condition for coupling functions:
		\begin{equation}
			f_{\text{EM}}(m_{\text{GUT}}) = f_W^2(m_{\text{GUT}}) = f_S^{-3}(m_{\text{GUT}}) = 1
		\end{equation}
	\end{itemize}
\subsection{Ratio-Based Calculations to Avoid Rounding Errors}
\begin{itemize}
	\item Basic principle: Using ratios instead of absolute values:
	\begin{equation}
		\frac{m_1}{m_0} = p, \quad \frac{m_2}{m_0} = q, \quad \frac{m_2}{m_1} = \frac{q}{p}
	\end{equation}
	
	\item Spectral ratio for numerical stability:
	\begin{equation}
		R(n) = \frac{q}{p} = \frac{\max(p,q)}{\min(p,q)}
	\end{equation}
	
	\item Octave reduction for further error minimization:
	\begin{equation}
		R_{\text{oct}}(n) = \frac{R(n)}{2^{\lfloor\log_2(R(n))\rfloor}}
	\end{equation}
	
	\item Harmonic distance (in cents):
	\begin{equation}
		d_{\text{harm}}(n,h) = 1200 \times \left|\log_2\left(\frac{R_{\text{oct}}(n)}{h}\right)\right|
	\end{equation}
	
	\item Matching criterion with tolerance parameter $\xi$:
	\begin{equation}
		\text{Match}(n, \text{harmonic\_ratio}) = \text{TRUE if } |R_{\text{oct}}(n) - \text{harmonic\_ratio}|^2 < 4\xi
	\end{equation}
	
	\item Application to frequency calculations:
	\begin{align}
		f_{\text{ratio}} &= \frac{f_2}{f_1} = \frac{q}{p} \\
		f_{\text{beat}} &= |f_2 - f_1| = f_0 \times |q - p|
	\end{align}
	
	\item Advantage: In complex calculations with many operations (especially FFT and spectral analyses), rounding errors can accumulate. Ratio-based calculation minimizes this effect by:
	\begin{itemize}
		\item Reducing the number of operations
		\item Avoiding differences between large numbers
		\item Stabilizing numerical precision across a wider range of values
		\item Enabling direct comparison with harmonic ratios without conversion
	\end{itemize}
\end{itemize}	
	\selectlanguage{english}
	
\end{document}