\documentclass[12pt,a4paper]{article}
\usepackage[utf8]{inputenc}
\usepackage[T1]{fontenc}
\usepackage[german,english]{babel}
\usepackage[left=2.5cm,right=2.5cm,top=2.5cm,bottom=2.5cm]{geometry}
\usepackage{lmodern}
\usepackage{amsmath}
\usepackage{amssymb}
\usepackage{physics}
\usepackage{hyperref}
\usepackage{tcolorbox}
\usepackage{booktabs}
\usepackage{enumitem}
\usepackage[table,xcdraw]{xcolor}
\usepackage{graphicx}
\usepackage{float}
\usepackage{mathtools}
\usepackage{amsthm}
\usepackage{cleveref}
\usepackage{siunitx}
\usepackage{fancyhdr}
\usepackage{textcomp}
\usepackage{gensymb}

% Headers and Footers
\pagestyle{fancy}
\fancyhf{}
\fancyhead[L]{Johann Pascher}
\fancyhead[R]{Proof: $\alpha = 1$ in Natural Units}
\fancyfoot[C]{\thepage}
\renewcommand{\headrulewidth}{0.4pt}
\renewcommand{\footrulewidth}{0.4pt}

\hypersetup{
	colorlinks=true,
	linkcolor=blue,
	citecolor=blue,
	urlcolor=blue,
	pdftitle={Proof: The Fine Structure Constant α = 1 in Natural Units},
	pdfauthor={Johann Pascher},
	pdfsubject={Fine Structure Constant, Natural Units, Fundamental Physics},
	pdfkeywords={Fine Structure Constant, Natural Units, Electromagnetic Duality}
}

% Custom Commands
\newcommand{\alphaem}{\alpha}
\newcommand{\alphanat}{\alpha_{\text{nat}}}
\newcommand{\alphaSI}{\alpha_{\text{SI}}}

\newtheorem{theorem}{Theorem}[section]
\newtheorem{lemma}[theorem]{Lemma}
\newtheorem{proposition}[theorem]{Proposition}
\newtheorem{definition}[theorem]{Definition}
\newtheorem{proof_step}{Proof Step}[section]

\begin{document}
	
	\title{Mathematical Proof: \\
		The Fine Structure Constant $\alpha = 1$ \\
		in Natural Units}
	\author{Johann Pascher\\
		Department of Communications Engineering,\\
		H{\"o}here Technische Bundeslehranstalt (HTL), Leonding, Austria\\
		\texttt{johann.pascher@gmail.com}}
	\date{\today}
	
	\maketitle
	
	\begin{abstract}
		This paper provides a rigorous mathematical proof that the fine structure constant $\alpha$ equals unity ($\alpha = 1$) in natural unit systems. Through systematic analysis of the two equivalent representations of $\alpha$, we demonstrate that the electromagnetic duality between $\varepsilon_0$ and $\mu_0$, connected by the fundamental Maxwell relation $c^2 = 1/(\varepsilon_0\mu_0)$, naturally leads to $\alpha = 1$ when appropriate unit normalizations are applied. This proof establishes that $\alpha = 1/137$ in SI units is purely a consequence of our historical unit choices, not a fundamental mystery of nature.
	\end{abstract}
	
	\tableofcontents
	\newpage
	
	\section{Introduction and Motivation}
	
	The fine structure constant $\alpha \approx 1/137$ has been called one of the greatest mysteries in physics, inspiring famous quotes from Feynman, Pauli, and others. However, this mystification stems from viewing $\alpha$ only within the SI unit system. This paper proves mathematically that $\alpha = 1$ in appropriately chosen natural units, revealing that the ``mystery'' of $1/137$ is merely a consequence of our conventional unit system.
	
	\section{Fundamental Premise}
	
	\begin{definition}[Two Equivalent Forms of $\alpha$]
		The fine structure constant can be expressed in two mathematically equivalent forms:
		\begin{align}
			\text{Form 1:} \quad \alphaem &= \frac{e^2}{4\pi\varepsilon_0\hbar c} \label{eq:alpha_form1}\\
			\text{Form 2:} \quad \alphaem &= \frac{e^2 \mu_0 c}{4\pi \hbar} \label{eq:alpha_form2}
		\end{align}
	\end{definition}
	
	These forms are equivalent through the Maxwell relation $c^2 = 1/(\varepsilon_0\mu_0)$.
	
	\section{The Duality Analysis}
	
	\subsection{Extraction of Common Elements}
	
	\begin{proof_step}[Identification of Common Terms]
		Both forms \eqref{eq:alpha_form1} and \eqref{eq:alpha_form2} contain identical terms:
		\begin{itemize}
			\item $e^2$ - square of elementary charge
			\item $4\pi$ - geometric factor
			\item $\hbar$ - reduced Planck constant
		\end{itemize}
	\end{proof_step}
	
	\begin{proof_step}[Isolation of Differential Terms]
		After factoring out common elements, the essential difference between the two forms is:
		\begin{align}
			\text{Form 1:} \quad \alphaem &\propto \frac{1}{\varepsilon_0 c} \label{eq:diff1}\\
			\text{Form 2:} \quad \alphaem &\propto \mu_0 c \label{eq:diff2}
		\end{align}
	\end{proof_step}
	
	\subsection{The Electromagnetic Duality}
	
	\begin{theorem}[Electromagnetic Duality Relation]
		For the two forms to be equivalent, we must have:
		\begin{equation}
			\frac{1}{\varepsilon_0 c} = \mu_0 c \label{eq:duality}
		\end{equation}
	\end{theorem}
	
	\begin{proof}
		Rearranging equation \eqref{eq:duality}:
		\begin{align}
			\frac{1}{\varepsilon_0 c} &= \mu_0 c\\
			1 &= \varepsilon_0 c \cdot \mu_0 c\\
			1 &= \varepsilon_0 \mu_0 c^2\\
			c^2 &= \frac{1}{\varepsilon_0 \mu_0}
		\end{align}
		This is precisely Maxwell's fundamental relation connecting electromagnetic constants with the speed of light.
	\end{proof}
	
	\section{The Key Insight: Opposite Powers of c}
	
	\begin{lemma}[Sign Duality of c]
		The speed of light $c$ appears with opposite ``signs'' (powers) in the two forms:
		\begin{align}
			\text{Form 1:} \quad c^{-1} \quad &\text{($c$ in denominator)}\\
			\text{Form 2:} \quad c^{+1} \quad &\text{($c$ in numerator)}
		\end{align}
	\end{lemma}
	
	This duality reflects the complementary nature of electric ($\varepsilon_0$) and magnetic ($\mu_0$) aspects of the electromagnetic field.
	
	\section{Construction of Natural Units}
	
	\subsection{The Natural Unit Choice}
	
	\begin{definition}[Natural Unit System for $\alpha = 1$]
		We define a natural unit system where:
		\begin{enumerate}
			\item $\hbar_{\text{nat}} = 1$ (quantum mechanical scale)
			\item $c_{\text{nat}} = 1$ (relativistic scale)  
			\item The electromagnetic constants are normalized such that $\alphaem = 1$
		\end{enumerate}
	\end{definition}
	
	\subsection{Determination of Natural Electromagnetic Constants}
	
	\begin{theorem}[Natural Unit Electromagnetic Constants]
		In the natural unit system where $\alpha = 1$, $\hbar = 1$, and $c = 1$, the electromagnetic constants become:
		\begin{align}
			e_{\text{nat}}^2 &= 4\pi \label{eq:e_nat}\\
			\varepsilon_{0,\text{nat}} &= 1 \label{eq:eps_nat}\\
			\mu_{0,\text{nat}} &= 1 \label{eq:mu_nat}
		\end{align}
	\end{theorem}
	
	\begin{proof}
		From Form 1 with $\alphaem = 1$, $\hbar = 1$, $c = 1$:
		\begin{align}
			1 &= \frac{e^2}{4\pi\varepsilon_0 \cdot 1 \cdot 1}\\
			4\pi\varepsilon_0 &= e^2
		\end{align}
		
		Setting $\varepsilon_0 = 1$ (natural choice), we get $e^2 = 4\pi$.
		
		From the Maxwell relation $c^2 = 1/(\varepsilon_0\mu_0)$ with $c = 1$:
		\begin{align}
			1 &= \frac{1}{\varepsilon_0\mu_0}\\
			\varepsilon_0\mu_0 &= 1
		\end{align}
		
		With $\varepsilon_0 = 1$, we get $\mu_0 = 1$.
	\end{proof}
	
	\section{Verification of $\alpha = 1$}
	
	\subsection{Verification Using Form 1}
	
	\begin{proof_step}[Form 1 Verification]
		\begin{align}
			\alphaem &= \frac{e^2}{4\pi\varepsilon_0\hbar c}\\
			&= \frac{4\pi}{4\pi \cdot 1 \cdot 1 \cdot 1}\\
			&= \frac{4\pi}{4\pi}\\
			&= 1 \quad \checkmark
		\end{align}
	\end{proof_step}
	
	\subsection{Verification Using Form 2}
	
	\begin{proof_step}[Form 2 Verification]
		\begin{align}
			\alphaem &= \frac{e^2 \mu_0 c}{4\pi \hbar}\\
			&= \frac{4\pi \cdot 1 \cdot 1}{4\pi \cdot 1}\\
			&= \frac{4\pi}{4\pi}\\
			&= 1 \quad \checkmark
		\end{align}
	\end{proof_step}
	
	\section{The Duality Verification}
	
	\begin{theorem}[Electromagnetic Duality in Natural Units]
		In natural units, the electromagnetic duality is perfectly satisfied:
		\begin{equation}
			\frac{1}{\varepsilon_{0,\text{nat}} \cdot c_{\text{nat}}} = \mu_{0,\text{nat}} \cdot c_{\text{nat}}
		\end{equation}
	\end{theorem}
	
	\begin{proof}
		\begin{align}
			\text{LHS:} \quad \frac{1}{\varepsilon_{0,\text{nat}} \cdot c_{\text{nat}}} &= \frac{1}{1 \cdot 1} = 1\\
			\text{RHS:} \quad \mu_{0,\text{nat}} \cdot c_{\text{nat}} &= 1 \cdot 1 = 1\\
			\text{Therefore:} \quad \text{LHS} &= \text{RHS} \quad \checkmark
		\end{align}
	\end{proof}
	
	\section{Physical Interpretation}
	
	\subsection{The Naturalness of $\alpha = 1$}
	
	\begin{tcolorbox}[colback=green!5!white,colframe=green!75!black,title=Key Physical Insight]
		In natural units, $\alpha = 1$ represents the perfect balance between:
		\begin{itemize}
			\item \textbf{Electric field coupling} (through $\varepsilon_0$ with $c^{-1}$)
			\item \textbf{Magnetic field coupling} (through $\mu_0$ with $c^{+1}$)
			\item \textbf{Quantum mechanical scale} (through $\hbar$)
			\item \textbf{Relativistic scale} (through $c$)
		\end{itemize}
		
		The electromagnetic duality $\frac{1}{\varepsilon_0 c} = \mu_0 c$ ensures this perfect balance.
	\end{tcolorbox}
	
	\subsection{Resolution of the ``$1/137$ Mystery''}
	
	The famous value $\alpha \approx 1/137$ in SI units arises solely from our historical choices of:
	\begin{itemize}
		\item The meter (length scale)
		\item The second (time scale)  
		\item The kilogram (mass scale)
		\item The ampere (current scale)
	\end{itemize}
	
	These choices force electromagnetic constants to have ``unnatural'' values, making $\alpha$ appear mysteriously small.
	
	\subsubsection{Transformation from Natural Units to SI Units}
	
	To understand how we arrive at the SI value $\alpha_{\text{SI}} = 1/137$, we must transform from our natural unit system back to SI units. The transformation involves scaling factors for each fundamental constant:
	
	\begin{align}
		\hbar_{\text{SI}} &= \hbar_{\text{nat}} \times S_{\hbar} = 1 \times (1.055 \times 10^{-34} \text{ J·s})\\
		c_{\text{SI}} &= c_{\text{nat}} \times S_c = 1 \times (2.998 \times 10^8 \text{ m/s})\\
		\varepsilon_{0,\text{SI}} &= \varepsilon_{0,\text{nat}} \times S_{\varepsilon} = 1 \times (8.854 \times 10^{-12} \text{ F/m})\\
		e_{\text{SI}} &= e_{\text{nat}} \times S_e = \sqrt{4\pi} \times S_e
	\end{align}
	
	The fine structure constant in SI units becomes:
	\begin{align}
		\alpha_{\text{SI}} &= \frac{e_{\text{SI}}^2}{4\pi\varepsilon_{0,\text{SI}}\hbar_{\text{SI}} c_{\text{SI}}}\\
		&= \frac{(\sqrt{4\pi} \times S_e)^2}{4\pi \times (S_{\varepsilon}) \times (S_{\hbar}) \times (S_c)}\\
		&= \frac{4\pi \times S_e^2}{4\pi \times S_{\varepsilon} \times S_{\hbar} \times S_c}\\
		&= \frac{S_e^2}{S_{\varepsilon} \times S_{\hbar} \times S_c}
	\end{align}
	
	The historical SI unit definitions created scaling factors such that this ratio equals approximately $1/137$. In other words:
	$\frac{S_e^2}{S_{\varepsilon} \times S_{\hbar} \times S_c} \approx \frac{1}{137}$
	
	This demonstrates that the ``mysterious'' value $1/137$ is purely a consequence of the arbitrary scaling factors chosen when defining the SI base units, not a fundamental property of electromagnetic interactions themselves. In the natural unit system where these scaling factors are unity, $\alpha = 1$ emerges as the fundamental value.
	
	\section{Mathematical Proof Summary}
	
	\begin{theorem}[Main Result: $\alpha = 1$ in Natural Units]
		There exists a consistent natural unit system where all fundamental constants are normalized to unity, and in this system, the fine structure constant equals exactly 1.
	\end{theorem}
	
	\begin{proof}[Complete Proof]
		\textbf{Step 1:} We established two equivalent forms of $\alpha$:
		$$\alphaem = \frac{e^2}{4\pi\varepsilon_0\hbar c} = \frac{e^2 \mu_0 c}{4\pi \hbar}$$
		
		\textbf{Step 2:} We identified the electromagnetic duality:
		$$\frac{1}{\varepsilon_0 c} = \mu_0 c \quad \Leftrightarrow \quad c^2 = \frac{1}{\varepsilon_0\mu_0}$$
		
		\textbf{Step 3:} We constructed natural units with:
		$$\hbar = 1, \quad c = 1, \quad e^2 = 4\pi, \quad \varepsilon_0 = 1, \quad \mu_0 = 1$$
		
		\textbf{Step 4:} We verified $\alpha = 1$ in both forms:
		\begin{align}
			\text{Form 1:} \quad \alphaem &= \frac{4\pi}{4\pi \cdot 1 \cdot 1 \cdot 1} = 1\\
			\text{Form 2:} \quad \alphaem &= \frac{4\pi \cdot 1 \cdot 1}{4\pi \cdot 1} = 1
		\end{align}
		
		\textbf{Step 5:} We confirmed the duality: $\frac{1}{1 \cdot 1} = 1 \cdot 1 = 1$ $\checkmark$
		
		Therefore, $\alpha = 1$ in natural units. \qed
	\end{proof}
	
	\section{Implications and Conclusions}
	
	\subsection{Philosophical Implications}
	
	This proof demonstrates that:
	
	\begin{enumerate}
		\item \textbf{$\alpha = 1/137$ is not fundamental} - it's a consequence of unit choices
		\item \textbf{$\alpha = 1$ is natural} - it reflects the inherent electromagnetic duality
		\item \textbf{The ``mystery'' dissolves} - there's nothing special about $1/137$
		\item \textbf{Nature is simpler} - fundamental relationships have natural values
	\end{enumerate}
	
	\subsection{Consistency Check}
	
	\begin{tcolorbox}[colback=blue!5!white,colframe=blue!75!black,title=Internal Consistency Verification]
		Our natural unit system satisfies all fundamental relations:
		\begin{align}
			c^2 &= \frac{1}{\varepsilon_0\mu_0} = \frac{1}{1 \cdot 1} = 1 = 1^2 \quad \checkmark\\
			\alphaem &= \frac{e^2}{4\pi\varepsilon_0\hbar c} = \frac{4\pi}{4\pi \cdot 1 \cdot 1 \cdot 1} = 1 \quad \checkmark\\
			\alphaem &= \frac{e^2\mu_0 c}{4\pi\hbar} = \frac{4\pi \cdot 1 \cdot 1}{4\pi \cdot 1} = 1 \quad \checkmark
		\end{align}
	\end{tcolorbox}
	
\section{Resolving the Constants Paradox}

\subsection{The Fundamental Misconception}

The most profound objection to our proof often takes the form: ``How can a \textbf{constant} have different values?'' This apparent paradox lies at the heart of why the fine structure constant has been mystified for over a century.

\subsubsection{The Problem Statement}

The seeming contradiction is:
\begin{itemize}
	\item $\alpha = 1/137$ (in SI units)
	\item $\alpha = 1$ (in natural units)
	\item $\alpha = \sqrt{2}$ (in Gaussian units)
\end{itemize}

How can the ``same'' constant have three different values?

\subsubsection{The Resolution}

The resolution reveals a fundamental misunderstanding about what ``constant'' means in physics.

\textbf{What is truly constant is not the number, but the physical relationship.}

\subsection{The Perfect Analogy: Water's Boiling Point}

Consider the boiling point of water:
\begin{itemize}
	\item $100°\text{C}$ (Celsius scale)
	\item $212°\text{F}$ (Fahrenheit scale)
	\item $373\text{ K}$ (Kelvin scale)
\end{itemize}

\textbf{Question:} At what temperature does water ``really'' boil?

\textbf{Answer:} At the same physical temperature in all cases! Only the numbers differ due to different temperature scales.

\subsection{The Same Principle Applies to $\alpha$}

Just as with temperature scales:
\begin{itemize}
	\item $\alpha = 1/137$ (SI unit scale)
	\item $\alpha = 1$ (natural unit scale)
	\item $\alpha = \sqrt{2}$ (Gaussian unit scale)
\end{itemize}

\textbf{The electromagnetic coupling strength is identical} -- only the measurement scales differ.

\subsection{The Key Insight}

\begin{tcolorbox}[colback=yellow!5!white,colframe=orange!75!black,title=Fundamental Principle]
	``\textbf{CONSTANT}'' does \textbf{NOT} mean ``same number''!
	
	``\textbf{CONSTANT}'' means ``same physical quantity''!
\end{tcolorbox}

\textbf{Examples of this principle:}
\begin{itemize}
	\item $1\text{ meter} = 100\text{ cm} = 3.28\text{ feet}$ $\rightarrow$ The \textbf{length} is constant
	\item $1\text{ kg} = 1000\text{ g} = 2.2\text{ lbs}$ $\rightarrow$ The \textbf{mass} is constant
	\item $\alpha = 1/137 = 1 = \sqrt{2}$ $\rightarrow$ The \textbf{coupling strength} is constant
\end{itemize}

\subsection{Physical Verification}

We can verify that these represent the same physical constant by confirming that all unit systems yield identical experimental results:

\begin{theorem}[Experimental Invariance]
	All unit systems produce identical measurable predictions:
	\begin{itemize}
		\item \textbf{Hydrogen spectrum:} Same frequencies in all systems $\checkmark$
		\item \textbf{Electron scattering:} Same cross-sections in all systems $\checkmark$
		\item \textbf{Lamb shift:} Same energy shifts in all systems $\checkmark$
	\end{itemize}
\end{theorem}

\subsection{The Deeper Truth}

\begin{tcolorbox}[colback=green!5!white,colframe=green!75!black,title=Nature's True Language]
	\textbf{Nature ``knows'' no numbers!}
	
	\textbf{Nature knows only ratios and relationships!}
\end{tcolorbox}

The fine structure constant $\alpha$ is not the mysterious number ``$1/137$'' -- $\alpha$ is the \textbf{ratio} between electromagnetic and quantum mechanical effects.

This ratio is absolutely constant throughout the universe, but the numerical value depends entirely on our arbitrary choice of unit definitions.

\subsection{The Linguistic Problem}

Much of the confusion stems from imprecise language. We incorrectly say:
\begin{itemize}
	\item[\textcolor{red}{$\times$}] ``\textbf{THE} fine structure constant is $1/137$''
\end{itemize}

The correct statements would be:
\begin{itemize}
	\item[\textcolor{green}{$\checkmark$}] ``The fine structure constant has the value $1/137$ \textbf{in SI units}''
	\item[\textcolor{green}{$\checkmark$}] ``The fine structure constant has the value $1$ \textbf{in natural units}''
\end{itemize}

\subsection{Resolution of the Century-Old Mystery}

This analysis reveals that the ``mystery of $1/137$'' is not a physical puzzle but a \textbf{linguistic and conceptual misunderstanding}. The mystification arose from:

\begin{enumerate}
	\item Conflating the numerical value with the physical quantity
	\item Treating the SI unit system as fundamental rather than conventional
	\item Forgetting that all unit systems are human constructs
	\item Seeking deep meaning in what are essentially conversion factors
\end{enumerate}

Once we recognize that $\alpha = 1$ represents the natural strength of electromagnetic interactions, the ``mystery'' dissolves completely. The electromagnetic force has unit strength in the unit system that respects the fundamental structure of quantum mechanics and relativity -- exactly as one would expect from a truly fundamental interaction.

\subsection{Final Perspective}

The fine structure constant teaches us a profound lesson about the nature of physical laws: \textbf{the universe's fundamental relationships are elegant and simple when expressed in their natural language}. The apparent complexity and mystery of ``$1/137$'' is merely an artifact of our historical choice to measure electromagnetic phenomena using units originally defined for mechanical quantities.

In recognizing $\alpha = 1$ as the natural value, we glimpse the inherent simplicity and beauty that underlies the electromagnetic structure of reality.
	
	\section{Acknowledgments}
	
	This work was inspired by the recognition that fundamental physical constants should not be mysterious numbers but should reflect the underlying mathematical structure of nature. The electromagnetic duality revealed through the analysis of the two forms of $\alpha$ provides the key insight that resolves the long-standing puzzle of the fine structure constant.
	
	\begin{thebibliography}{9}
		\bibitem{Jackson1999} Jackson, J. D. (1999). \textit{Classical Electrodynamics} (3rd ed.). John Wiley \& Sons.
		
		\bibitem{Feynman1985} Feynman, R. P. (1985). \textit{QED: The Strange Theory of Light and Matter}. Princeton University Press.
		
		\bibitem{Weinberg1995} Weinberg, S. (1995). \textit{The Quantum Theory of Fields, Volume 1: Foundations}. Cambridge University Press.
		
		\bibitem{Planck1906} Planck, M. (1906). Vorlesungen über die Theorie der Wärmestrahlung. Leipzig: J.A. Barth.
		
		\bibitem{Maxwell1865} Maxwell, J. C. (1865). A Dynamical Theory of the Electromagnetic Field. \textit{Philosophical Transactions of the Royal Society}, 155, 459-512.
		
		\bibitem{CODATA2018} CODATA Task Group on Fundamental Constants (2019). CODATA Recommended Values of the Fundamental Physical Constants: 2018. \textit{Rev. Mod. Phys.}, 91, 025009.
	\end{thebibliography}
	
\end{document}