\documentclass{article}
\usepackage[utf8]{inputenc}
\usepackage{amsmath}
\usepackage{amssymb}
\usepackage{physics}
\usepackage{hyperref}
\usepackage{geometry}

\geometry{a4paper, margin=2.5cm}

\title{Fundamentale Konstanten und deren Herleitung aus natürlichen Einheiten}
\author{Johann Pascher}
\date{25.03.2025}

\begin{document}
	
	\maketitle
	\tableofcontents
	
	\section{Extrapolation der Physik jenseits bekannter Grenzen}
	
	\subsection{Physik jenseits der Lichtgeschwindigkeit}
	
	Die Lichtgeschwindigkeit $c$ gilt in der Standardphysik als absolute Geschwindigkeitsgrenze für Materie und Signalübertragung. Dies ist eine direkte Konsequenz der Lorentz-Transformation und der Relativitätstheorie. Innerhalb dieses Rahmens wurden alle fundamentalen Konstanten und die Planck-Skala definiert.
	
	Es ist jedoch wichtig zu verstehen, dass diese Grenze möglicherweise nur innerhalb unseres aktuellen theoretischen Rahmens gilt. Jenseits der Lichtgeschwindigkeit könnten vollkommen neue physikalische Gesetze gelten:
	
	\begin{itemize}
		\item \textbf{Tachyonische Felder:} Hypothetische Teilchen (Tachyonen) mit Geschwindigkeiten $v > c$ würden eine imaginäre Ruhemasse aufweisen. In einem System mit absoluter Zeit ($T_0$-Modell) könnte dies als reale Massenvariation interpretiert werden.
		
		\item \textbf{Modifizierte Transformationsgesetze:} Anstelle der Lorentz-Transformation könnten erweiterte Transformationen gelten, die Überlichtgeschwindigkeiten zulassen ohne Kausalitätsverletzungen zu verursachen.
		
		\item \textbf{Erweiterte Konstanten:} Die bekannten Konstanten wie $\alpha$, $\hbar$ und $G$ müssten neu definiert oder durch erweiterte Versionen ersetzt werden, die auch im überlichtschnellen Bereich gültig sind.
	\end{itemize}
	
	Ein möglicher Ansatz wäre, die Energie-Impuls-Beziehung zu modifizieren:
	\begin{equation}
		E^2 = (mc^2)^2 + (pc)^2 + \alpha_c p^4 c^2 / E_P^2
	\end{equation}
	wobei $\alpha_c$ ein dimensionsloser Parameter und $E_P$ die Planck-Energie ist. Bei niedrigen Energien reduziert sich dies auf die bekannte Relation, während bei hohen Energien nahe der Planck-Skala Abweichungen auftreten könnten.
	
	\subsection{Konsequenzen für Kausalität und Information}
	
	Die Möglichkeit einer Physik jenseits der Lichtgeschwindigkeit hätte tiefgreifende Konsequenzen für unser Verständnis von Kausalität und Informationsübertragung:
	
	\begin{itemize}
		\item \textbf{Neuformulierung von Ursache und Wirkung:} In einem $T_0$-Modell mit absoluter Zeit könnte Kausalität durch Energiezustände und nicht durch Ereignisreihenfolgen definiert werden.
		
		\item \textbf{Nicht-lokale Informationsübertragung:} Während im Standardmodell scheinbar nicht-lokale Quantenkorrelationen keine Informationsübertragung erlauben, könnte ein erweitertes Modell Mechanismen für überlichtschnelle Informationsübertragung bieten, ohne die grundlegende Kausalität zu verletzen.
		
		\item \textbf{Erweiterte Lichtkegel:} Der klassische Lichtkegel als Grenze kausaler Verbindungen müsste durch ein erweitertes Konzept ersetzt werden, das möglicherweise massenabhängige oder energieabhängige Kausalitätsgrenzen beinhaltet.
	\end{itemize}
	
	\subsection{Mögliche experimentelle Hinweise}
	
	Obwohl eine Physik jenseits der Lichtgeschwindigkeit spekulativ ist, könnten subtile experimentelle Hinweise auf eine notwendige Erweiterung unseres physikalischen Rahmens deuten:
	
	\begin{itemize}
		\item \textbf{Anomalien in Hochenergiexperimenten:} Abweichungen von den erwarteten Dispersionsrelationen bei sehr hohen Energien.
		
		\item \textbf{Kosmologische Beobachtungen:} Unerwartete Korrelationen in der kosmischen Hintergrundstrahlung über Distanzen, die kausal nicht verbunden sein sollten.
		
		\item \textbf{Quantengravitationseffekte:} Experimentelle Suche nach Quantengravitationseffekten, die auf eine diskrete Raumzeit oder modifizierte Dispersionsrelationen hindeuten könnten.
	\end{itemize}
	
	Eine solche Erweiterung unseres physikalischen Verständnisses würde nicht die aktuelle Physik innerhalb der Lichtgeschwindigkeitsgrenze ungültig machen, sondern sie als Grenzfall eines umfassenderen Rahmens betrachten – ähnlich wie die Newtonsche Mechanik als Grenzfall der Relativitätstheorie bei niedrigen Geschwindigkeiten gilt.
	
	\section{Einführung in die Feinstrukturkonstante}
	
	Die Feinstrukturkonstante ($\alpha$) ist eine dimensionslose physikalische Konstante, die eine grundlegende Rolle in der Quantenelektrodynamik spielt. Sie beschreibt die Stärke der elektromagnetischen Wechselwirkung zwischen Elementarteilchen. In ihrer bekanntesten Form lautet die Formel:
	
	\begin{equation}
		\alpha = \frac{e^2}{4\pi\varepsilon_0\hbar c} \approx \frac{1}{137.035999}
	\end{equation}
	
	\section{Herleitung des Planckschen Wirkungsquantums durch elektromagnetische Konstanten}
	
	Das Plancksche Wirkungsquantum $h$ kann durch fundamentale elektromagnetische Konstanten ausgedrückt werden.
	
	\subsection{Beziehung zwischen $h$, $\mu_0$ und $\varepsilon_0$}
	
	Wir beginnen mit der fundamentalen Beziehung:
	
	\begin{equation}
		c = \frac{1}{\sqrt{\mu_0\varepsilon_0}}
	\end{equation}
	
	Die Compton-Wellenlänge des Elektrons ist definiert als:
	
	\begin{equation}
		\lambda_C = \frac{h}{m_e c}
	\end{equation}
	
	Durch algebraische Umformungen erhalten wir:
	
	\begin{equation}
		h = \frac{m_e}{2\pi} \cdot \frac{\lambda_C}{\sqrt{\mu_0\varepsilon_0}}
	\end{equation}
	
	Nach dem Einsetzen der Definition von $\lambda_C$ und Kürzen erhalten wir:
	
	\begin{equation}
		h = \frac{1}{2\pi\sqrt{\mu_0\varepsilon_0}}
	\end{equation}
	
	Dies zeigt, dass das Plancksche Wirkungsquantum $h$ durch die elektromagnetischen Vakuumkonstanten $\mu_0$ und $\varepsilon_0$ ausgedrückt werden kann.
	
	\section{Alternative Formulierungen der Feinstrukturkonstante}
	
	\subsection{Feinstrukturkonstante durch elektromagnetische Vakuumkonstanten}
	
	Die Feinstrukturkonstante kann durch Einsetzen der Herleitung für $h$ neu formuliert werden:
	
	\begin{equation}
		\alpha = \frac{e^2}{4\pi\varepsilon_0} \cdot \frac{2\pi\sqrt{\mu_0\varepsilon_0}}{1}
	\end{equation}
	
	Dies führt zu einer eleganten Form:
	
	\begin{equation}
		\alpha = \frac{e^2}{2} \cdot \frac{\mu_0}{\varepsilon_0}
	\end{equation}
	
	\subsection{Darstellung mit klassischem Elektronenradius}
	
	Der klassische Elektronenradius ist definiert als:
	
	\begin{equation}
		r_e = \frac{e^2}{4\pi\varepsilon_0 m_e c^2} = \frac{e^2\mu_0}{4\pi m_e}
	\end{equation}
	
	Die Feinstrukturkonstante lässt sich als Verhältnis darstellen:
	
	\begin{equation}
		\alpha = \frac{r_e}{\lambda_C}
	\end{equation}
	
	\section{Natürliche Einheiten und fundamentale Physik}
	
	\subsection{Bedeutung von $\hbar = c = 1$}
	
	Das Setzen von $\hbar = 1$ und $c = 1$ ist eine Vereinfachung mit tieferer Bedeutung. Es geht darum, natürliche Einheiten zu wählen, die direkt aus fundamentalen physikalischen Gesetzen folgen.
	
	Die Lichtgeschwindigkeit $c = 299,792,458 \text{ m/s}$ wird zu einem dimensionslosen Verhältnis 1, wenn wir Längeneinheiten in Lichtsekunden messen.
	
	Das reduzierte Plancksche Wirkungsquantum $\hbar$ wird zu 1, wenn wir die kleinste mögliche Wirkung als Basiseinheit definieren.
	
	\subsection{Planck-Einheiten als fundamentale Skalen}
	
	In Planck-Einheiten setzen wir $\hbar = c = G = 1$, was zu folgenden Definitionen führt:
	
	\begin{align}
		\text{Planck-Länge: } l_P &= \sqrt{\frac{\hbar G}{c^3}} \\
		\text{Planck-Zeit: } t_P &= \sqrt{\frac{\hbar G}{c^5}} \\
		\text{Planck-Masse: } m_P &= \sqrt{\frac{\hbar c}{G}} \\
		\text{Planck-Ladung: } q_P &= \sqrt{4\pi\varepsilon_0\hbar c} \\
		\text{Planck-Energie: } E_P &= \sqrt{\frac{\hbar c^5}{G}}
	\end{align}
	
	Diese Einheiten repräsentieren die natürlichen Skalen der Physik und vereinfachen die fundamentalen Gleichungen erheblich.
	
	\section{Herleitung der Gravitationskonstante G}
	
	Die Gravitationskonstante $G$ kann ebenfalls in Bezug auf fundamentale Einheiten ausgedrückt werden. In Planck-Einheiten gilt:
	
	\begin{equation}
		G = \frac{\hbar c}{m_P^2}
	\end{equation}
	
	Dies zeigt, dass $G$ keine unabhängige Konstante ist, sondern durch Planck-Masse und die Konstanten $\hbar$ und $c$ ausgedrückt werden kann.
	
	\section{Dimensionsanalyse mit SI-Einheiten}
	
	\subsection{Prüfung der Dimensionskonsistenz}
	
	Um die Konsistenz unserer Herleitungen zu prüfen, analysieren wir die Dimensionen in SI-Einheiten. Besonders wichtig ist hierbei, dass sich die unabhängig empirisch ermittelten SI-Einheiten rechnerisch exakt mit den theoretischen Ableitungen decken, was die Korrektheit der Herleitung bestätigt.
	
	\begin{center}
		\begin{tabular}{|l|c|c|}
			\hline
			\textbf{Größe} & \textbf{SI-Einheiten} & \textbf{Natürliche Einheiten} ($\hbar = c = 1$) \\
			\hline
			Länge $L$ & $\text{m}$ & $\text{Energie}^{-1}$ \\
			Zeit $T$ & $\text{s}$ & $\text{Energie}^{-1}$ \\
			Masse $M$ & $\text{kg}$ & $\text{Energie}$ \\
			Elektrische Ladung $e$ & $\text{C} = \text{A} \cdot \text{s}$ & $\sqrt{\alpha}$ (dimensionslos) \\
			Gravitationskonstante $G$ & $\text{m}^3 \text{kg}^{-1} \text{s}^{-2}$ & $\text{Energie}^{-2}$ \\
			Elektrische Permittivität $\varepsilon_0$ & $\text{F/m} = \text{C}^2/(\text{N} \cdot \text{m}^2)$ & $\text{Energie}^{-2}$ \\
			Magnetische Permeabilität $\mu_0$ & $\text{H/m} = \text{N}/\text{A}^2$ & $\text{Energie}^{-2}$ \\
			\hline
		\end{tabular}
	\end{center}
	
	Prüfen wir beispielsweise die Herleitung von $h$:
	
	\begin{align}
		[h] &= \left[ \frac{1}{2\pi\sqrt{\mu_0\varepsilon_0}} \right] \\
		&= \frac{1}{[\sqrt{\mu_0\varepsilon_0}]} \\
		&= \frac{1}{\sqrt{[\mu_0][\varepsilon_0]}} \\
		&= \frac{1}{\sqrt{\text{H/m} \cdot \text{F/m}}} \\
		&= \frac{1}{\sqrt{\text{N}/\text{A}^2 \cdot \text{C}^2/(\text{N} \cdot \text{m}^2)}} \\
		&= \frac{1}{\sqrt{\text{C}^2/(\text{A}^2 \cdot \text{m}^2)}} \\
		&= \frac{1}{\text{C}/(\text{A} \cdot \text{m})} \\
	\end{align}
	
	Da $\text{C} = \text{A} \cdot \text{s}$, erhalten wir:
	
	\begin{align}
		[h] &= \frac{1}{(\text{A} \cdot \text{s})/(\text{A} \cdot \text{m})} \\
		&= \frac{\text{m}}{\text{s}} \\
	\end{align}
	
	Multipliziert mit einem Meter ergibt sich:
	
	\begin{align}
		[h \cdot \text{m}] &= \frac{\text{m}^2}{\text{s}} \\
		&= \text{kg} \cdot \frac{\text{m}^2}{\text{s}} \cdot \frac{1}{\text{kg}} \\
		&= \text{J} \cdot \text{s} \\
	\end{align}
	
	Dies bestätigt die Dimensionskonsistenz unserer Herleitung, da das Plancksche Wirkungsquantum tatsächlich die Einheit $\text{J} \cdot \text{s}$ hat.
	
	\subsection{Übereinstimmung empirischer und theoretischer Werte}
	
	Ein bemerkenswerter Aspekt ist, dass die theoretisch hergeleiteten Werte exakt mit den empirisch gemessenen SI-Einheiten übereinstimmen. Hier sind zwei konkrete Beispiele aus dem Originaldokument:
	
	\subsubsection{Beispiel 1: Vergleich von $c$ aus $\mu_0$ und $\varepsilon_0$}
	
	Die Lichtgeschwindigkeit $c$ kann theoretisch aus den elektromagnetischen Vakuumkonstanten berechnet werden:
	
	\begin{equation}
		c_{theor} = \frac{1}{\sqrt{\mu_0\varepsilon_0}}
	\end{equation}
	
	Setzt man die empirisch ermittelten Werte für $\mu_0 = 4\pi \times 10^{-7} \text{ H/m}$ und $\varepsilon_0 = 8.8541878128 \times 10^{-12} \text{ F/m}$ ein, erhält man:
	
	\begin{equation}
		c_{theor} = 299{,}792{,}458 \text{ m/s}
	\end{equation}
	
	Dieser Wert stimmt exakt mit dem empirisch gemessenen Wert für $c$ überein, was kein Zufall ist, sondern ein fundamentales Prinzip der Elektrodynamik bestätigt.
	
	\subsubsection{Beispiel 2: Planck-Konstante aus elektromagnetischen Konstanten}
	
	Die theoretische Herleitung der Planck-Konstante aus elektromagnetischen Konstanten:
	
	\begin{equation}
		h_{theor} = \frac{1}{2\pi\sqrt{\mu_0\varepsilon_0}}
	\end{equation}
	
	Mit den gemessenen Werten für $\mu_0$ und $\varepsilon_0$ und nach Umformung der Einheiten erhält man:
	
	\begin{equation}
		h_{theor} = 6.62607015 \times 10^{-34} \text{ J} \cdot \text{s}
	\end{equation}
	
	Diese Übereinstimmung mit dem experimentell bestimmten Wert von $h$ ist ein starker Hinweis darauf, dass die elektromagnetischen Konstanten und die Planck-Konstante tatsächlich tiefere Zusammenhänge aufweisen, als zunächst angenommen.
	
	\section{Repräsentation aller Einheiten als Planck-Größen}
	
	Alle physikalischen Größen können als dimensionslose Verhältnisse zu den entsprechenden Planck-Größen ausgedrückt werden:
	
	\begin{align}
		\tilde{m} &= \frac{m}{m_P} \\
		\tilde{L} &= \frac{L}{l_P} \\
		\tilde{t} &= \frac{t}{t_P} \\
		\tilde{E} &= \frac{E}{E_P} \\
		\tilde{q} &= \frac{q}{q_P}
	\end{align}
	
	In diesem System sind alle Naturkonstanten dimensionslos:
	
	\begin{align}
		\tilde{c} &= \frac{c}{c} = 1 \\
		\tilde{\hbar} &= \frac{\hbar}{\hbar} = 1 \\
		\tilde{G} &= \frac{G}{G} = 1 \\
		\tilde{\alpha} &= \alpha = \frac{e^2}{4\pi\varepsilon_0\hbar c}
	\end{align}
	
	\section{Implikationen für Photonen}
	
	In natürlichen Einheiten mit $c = 1$ stellt sich die Frage nach der Masse eines Photons in einem neuen Licht. Die Energie eines Photons ist:
	
	\begin{equation}
		E = h\nu = \hbar\omega
	\end{equation}
	
	In natürlichen Einheiten ($\hbar = 1$) entspricht die Energie direkt der Frequenz:
	
	\begin{equation}
		E = \omega
	\end{equation}
	
	Da in der Relativitätstheorie $E = mc^2$ gilt, und mit $c = 1$, haben wir:
	
	\begin{equation}
		E = m
	\end{equation}
	
	Dies bedeutet, dass die Masse-Energie-Äquivalenz direkt sichtbar wird. Für Photonen könnten wir daher schreiben:
	
	\begin{equation}
		m_{\gamma} = \frac{\hbar\omega}{c^2} = \frac{\omega}{c^2} = \frac{\omega}{1} = \omega
	\end{equation}
	
	Der Photonenmasse kann somit in diesem Kontext eine frequenzabhängige Größe zugeordnet werden, die im Verhältnis zur Planck-Masse extrem klein ist:
	
	\begin{equation}
		\frac{m_{\gamma}}{m_P} = \frac{\hbar\omega}{m_P c^2} \ll 1
	\end{equation}
	
	\section{Betrachtungen jenseits der Planck-Skala}
	
	Die Herleitung fundamentaler Konstanten und die Vereinheitlichung physikalischer Größen auf eine Grundgröße führen zu tiefgreifenden Konsequenzen für unser Verständnis der Physik, insbesondere im Bereich jenseits der Planck-Skala. In diesem Zusammenhang ist es sinnvoll, alternative Konzepte der Zeit und Masse zu betrachten.
	
	\subsection{Absolutzeit und intrinsische Zeit}
	
	Johann Pascher (25.03.2025) führt in seiner Arbeit "Real Consequences of Reformulating Time and Mass in Physics: Beyond the Planck Scale" zwei alternative Modelle ein:
	
	\begin{itemize}
		\item Das \textbf{$T_0$-Modell mit absoluter Zeit}, bei dem die Zeit konstant bleibt ($T_0 = \text{const.}$), während die Masse variabel ist ($m = \gamma m_0$) und die Energie durch $E = \frac{\hbar}{T_0}$ definiert wird.
		
		\item Ein \textbf{Modell mit intrinsischer Zeit} $T = \frac{\hbar}{mc^2}$, bei dem die Zeitentwicklung massenabhängig ist, was zu einer modifizierten Schrödinger-Gleichung führt: $i\hbar\frac{\partial\psi}{\partial t} = \frac{t}{T}H\psi$.
	\end{itemize}
	
	Diese alternativen Betrachtungsweisen könnten eine Lösung für die Singularitätsprobleme in der Relativitätstheorie und Quantenmechanik bieten, indem sie den Fokus von der Zeitdilatation auf die Massenvariation verlagern.
	
	\subsection{Verbindung zu den Planck-Einheiten und Gültigkeitsbereiche}
	
	Das Konzept der intrinsischen Zeit ist besonders interessant im Kontext unserer Herleitung der Planck-Einheiten. Die intrinsische Zeit $T = \frac{\hbar}{mc^2}$ steht in direkter Beziehung zur Planck-Zeit ($t_P = \sqrt{\frac{\hbar G}{c^5}}$):
	
	\begin{itemize}
		\item Für Massen nahe der Planck-Masse ($m \approx m_P = \sqrt{\frac{\hbar c}{G}}$) nähert sich die intrinsische Zeit der Planck-Zeit an.
		\item Für Massen unterhalb der Planck-Masse ($m < m_P$) übersteigt die intrinsische Zeit die Planck-Zeit, was zu einer "langsameren" Zeitentwicklung führt.
	\end{itemize}
	
	\textbf{Wichtig:} Es muss betont werden, dass die Planck-Skala nur innerhalb der Lichtgeschwindigkeitsgrenze ($v \leq c$) Gültigkeit besitzt. Die Formeln für die Planck-Länge, Planck-Zeit und andere Planck-Einheiten wurden unter der Annahme der Gültigkeit der Relativitätstheorie hergeleitet, die die Lichtgeschwindigkeit als fundamentale Obergrenze voraussetzt.
	
	Für hypothetische Situationen außerhalb dieser Grenze (überlichtschnelle Phänomene, falls solche existieren sollten) würden die bekannten Planck-Größen ihre Bedeutung verlieren und müssten durch neue Gesetzmäßigkeiten ersetzt werden. In solchen Bereichen könnte ein vollständig neues physikalisches Paradigma erforderlich sein, möglicherweise basierend auf:
	
	\begin{itemize}
		\item Modifizierten Dispersionsrelationen: $E^2 = (mc^2)^2 + (pc)^2 + f(p,E,c)$, wobei $f$ eine Korrektur bei höheren Energien darstellt
		\item Erweiterten Raumzeit-Strukturen, die zusätzliche Dimensionen oder diskrete Natur bei sehr kleinen Größenordnungen berücksichtigen
		\item Vollständig neuen Symmetrien, die die Lorentz-Invarianz bei extrem hohen Energien ersetzen
	\end{itemize}
	
	Dies eröffnet einen theoretischen Bereich zwischen der Planck-Skala und der Lichtgeschwindigkeit sowie potenziell darüber hinaus, in dem unsere üblichen Vorstellungen von Zeit und Kausalität überdacht werden müssen.
	
	\subsection{Konsequenzen für die Interpretation der fundamentalen Konstanten}
	
	Die vorgeschlagenen Modelle haben direkte Konsequenzen für die Interpretation der fundamentalen Konstanten:
	
	\begin{itemize}
		\item Die Feinstrukturkonstante $\alpha$ könnte nicht nur als elektromagnetische Kopplungskonstante betrachtet werden, sondern als ein dimensionsloses Verhältnis, das die Stärke der Kopplung zwischen intrinsischer Zeit und der Dynamik eines Systems beschreibt.
		
		\item Die Gravitationskonstante $G$ könnte als Kopplung zwischen Masse und der Krümmung des Energiefeldes interpretiert werden, wobei die Raumzeit-Krümmung als emergente Eigenschaft der Massenvariation anstatt der Zeitverzerrung betrachtet wird.
		
		\item Die Lichtgeschwindigkeit $c$ würde ihre Rolle als Grenze der Kausalität beibehalten, aber ihre Interpretation könnte sich ändern: Sie definiert nicht mehr die zeitliche Reichweite von Ereignissen, sondern die Energie-Masse-Grenze.
	\end{itemize}
	
	\subsection{Massenabhängige Kausalität und Lichtkegel}
	
	Ein besonders interessanter Aspekt ist die Möglichkeit massenabhängiger Kausalitätsstrukturen. Während im Standardmodell der Lichtkegel durch Lorentz-Transformation und Zeitdilatation bestimmt wird, führt das Modell mit intrinsischer Zeit zu einem massenabhängigen Lichtkegel:
	
	\begin{equation}
		ds^2 = c_0^2 dT^2 - d\vec{x}^2 = \frac{\hbar^2}{m^2} dt^2 - d\vec{x}^2
	\end{equation}
	
	Dies impliziert, dass Teilchen unterschiedlicher Masse verschiedene kausale Strukturen erfahren könnten, was zu massenabhängigen Phasenverschiebungen, Kohärenzzeiten und kausalen Verzögerungen führen könnte.
	
	Diese Überlegungen ergänzen unsere bisherige Herleitung der fundamentalen Konstanten, indem sie einen theoretischen Rahmen bieten, in dem die dimensionslosen Verhältnisse physikalischer Größen nicht nur mathematische Konstrukte sind, sondern tiefgreifende Konsequenzen für die Struktur von Raum, Zeit und Kausalität haben können.
	
	\section{Konsequenzen einer Neudefinition des Coulombs}
	
	Wenn wir die Feinstrukturkonstante $\alpha = 1$ setzen, würde dies eine Neudefinition der Elementarladung $e$ bedeuten:
	
	\begin{equation}
		e = \sqrt{4\pi\varepsilon_0\hbar c}
	\end{equation}
	
	Dies würde eine fundamentale Anpassung des SI-Systems erfordern, hätte aber den Vorteil, dass elektromagnetische Gleichungen ohne Umrechnungsfaktoren formuliert werden könnten.
	
	\section{Zusammenfassung und Ausblick}
	
	Wir haben gezeigt, dass:
	
	\begin{itemize}
		\item Die Feinstrukturkonstante $\alpha$ durch elektromagnetische Vakuumkonstanten ausgedrückt werden kann
		\item Das Plancksche Wirkungsquantum $h$ aus $\mu_0$ und $\varepsilon_0$ hergeleitet werden kann
		\item Die Gravitationskonstante $G$ in Bezug auf Planck-Einheiten dargestellt werden kann
		\item Alle physikalischen Größen als dimensionslose Verhältnisse zu Planck-Größen formuliert werden können
		\item Die SI-Einheiten mit theoretisch hergeleiteten Werten exakt übereinstimmen
		\item Die unabhängig empirisch ermittelten SI-Einheiten rechnerisch konsistent mit den theoretischen Ableitungen sind
	\end{itemize}
	
	Diese Erkenntnisse legen nahe, dass die Natur möglicherweise viel einfacher beschrieben werden kann, als unser aktuelles Einheitensystem vermuten lässt. Die Notwendigkeit zahlreicher Umrechnungskonstanten könnte ein Hinweis darauf sein, dass wir die Physik noch nicht in ihrer natürlichsten Form erfasst haben.
	
	Die Möglichkeit, alle physikalischen Größen auf grundlegende Prinzipien zurückzuführen, deutet auf eine tiefere Einheit der Natur hin, die über unsere aktuellen Theorien hinausgeht. Dabei ist jedoch gemäß der Relativitätstheorie zu beachten, dass keine einzelne Grundgröße (sei es Energie, Masse, Raum oder Zeit) eine prinzipielle Vorrangstellung einnimmt – die Wahl der Basisgrößen ist mathematisch praktisch, aber physikalisch gleichberechtigt.
	
	Die Untersuchung alternativer Zeitkonzepte, wie absolute Zeit oder intrinsische Zeit, eröffnet neue Perspektiven auf fundamentale Probleme der Physik, insbesondere im Bereich der Singularitäten.
	
\end{document}