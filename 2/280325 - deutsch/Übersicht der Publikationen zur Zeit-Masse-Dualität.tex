\documentclass[a4paper,12pt]{article}
\usepackage[utf8]{inputenc}
\usepackage[T1]{fontenc}
\usepackage[german]{babel}
\usepackage{lmodern}
\usepackage{csquotes}
\usepackage{hyperref}
\usepackage{xcolor}
\usepackage{geometry}
\usepackage{booktabs}
\usepackage{array}
\usepackage{tabularx}
\usepackage{fancyhdr}

\geometry{a4paper, margin=2.5cm}
\hypersetup{
	colorlinks=true,
	linkcolor=blue,
	filecolor=magenta,      
	urlcolor=blue,
	pdftitle={Übersicht der Publikationen zur Zeit-Masse-Dualität},
	pdfauthor={Johann Pascher},
	pdfcreator={LaTeX}
}

% Repository base URL - replace this with your actual repository URL
\newcommand{\repobase}{https://github.com/jpascher/T0-Time-Mass-Duality/tree/main/2/}

\pagestyle{fancy}
\fancyhf{}
\rhead{Johann Pascher}
\lhead{Zeit-Masse-Dualität}
\cfoot{\thepage}

\title{Übersicht der Publikationen zur Zeit-Masse-Dualität \\ \Large{Ein theoretischer Rahmen für die Erweiterung der modernen Physik}}
\author{Johann Pascher}
\date{März 2025}

\begin{document}
	
	\maketitle
	
	\begin{abstract}
		Diese Übersicht präsentiert eine zusammenhängende Sammlung von Arbeiten, die einen neuen theoretischen Rahmen zur Erweiterung der modernen Physik entwickeln. Im Zentrum steht das Konzept der Zeit-Masse-Dualität, das eine grundlegende Neuformulierung der Beziehung zwischen Zeit und Masse vorschlägt. Dieser Ansatz bietet nicht nur einen möglichen Weg zur Vereinigung von Quantenmechanik, Quantenfeldtheorie und Allgemeiner Relativitätstheorie, sondern auch neue Perspektiven auf fundamentale Phänomene wie Nichtlokalität, dunkle Materie und dunkle Energie. Die hier aufgeführten Publikationen bilden ein kohärentes Forschungsprogramm, das von theoretischen Grundlagen über konkrete Anwendungen bis hin zu experimentell überprüfbaren Vorhersagen reicht.
	\end{abstract}
	
	\section{Einleitung}
	
	Die nachfolgend aufgeführten Publikationen stellen einen zusammenhängenden Korpus von Arbeiten dar, die verschiedene Aspekte eines neuen theoretischen Rahmens für die moderne Physik entwickeln. Zentral ist dabei das Konzept der Zeit-Masse-Dualität, das eine fundamentale Neuinterpretation der Beziehung zwischen Zeit und Masse vorschlägt, mit weitreichenden Implikationen für unser Verständnis der physikalischen Realität – von der Quantenmechanik bis zur Kosmologie.
	
	Die Arbeiten sind in fünf thematische Bereiche gegliedert:
	\begin{enumerate}
		\item Fundamentale Theorieentwicklung
		\item Spezifische Anwendungen und Implikationen
		\item Fundamentale Konstanten und Einheiten
		\item Kosmologische und Grenzgebiete
		\item Kurzfassungen und Überblicksdarstellungen
	\end{enumerate}
	
	Alle Dokumente sind im Repository verfügbar und über die angegebenen Links direkt zugänglich.
	
	\section{Fundamentale Theorieentwicklung}
	
	Diese Publikationen legen die grundlegenden konzeptionellen und mathematischen Grundlagen der Zeit-Masse-Dualität und ihrer Erweiterungen der Standardphysik.
	
	\subsection{\href{\repobase/pdf/Deutsch/Die Notwendigkeit einer Erweiterung der Standard-Quantenmechanik und Quantenfeldtheorie.pdf}{Die Notwendigkeit einer Erweiterung der Standard-Quantenmechanik und Quantenfeldtheorie}}
	\textit{(284.314 Bytes, 27.03.2025)}
	
	Dieses Grundlagenwerk identifiziert kritische konzeptionelle Grenzen der bestehenden Quantentheorien, insbesondere im Hinblick auf die asymmetrische Behandlung von Zeit und Raum sowie die statische Betrachtung der Masse. Es führt das Konzept der intrinsischen Zeit $T = \hbar/mc^2$ ein und entwickelt einen erweiterten Lagrange-Formalismus, der die Zeit-Masse-Dualität in die Quantenfeldtheorie integriert. Ein zentraler Aspekt ist die Möglichkeit, durch variable Masse als fundamentale verborgene Variable zu einem deterministischen Verständnis der Quantenwelt zurückzukehren, ohne mit experimentellen Befunden in Konflikt zu geraten.
	
	\subsection{\href{\repobase/pdf/Deutsch/Komplementäre Erweiterungen der Physik.pdf}{Komplementäre Erweiterungen der Physik}}
	\textit{(194.895 Bytes, 25.03.2025)}
	
	Diese Arbeit stellt die zwei komplementären Sichtweisen vor, die den Kern der Zeit-Masse-Dualität bilden: das Standardmodell mit konstanter Masse und variabler Zeit (Zeitdilatation) und das $T_0$-Modell mit absoluter Zeit und variabler Masse. Es wird gezeigt, wie diese dualen Perspektiven mathematisch ineinander überführbar sind und die gleichen physikalischen Phänomene beschreiben können, jedoch mit unterschiedlichen konzeptionellen Grundlagen. Die Arbeit entwickelt die formale Struktur für die Transformation zwischen beiden Modellen und diskutiert die philosophischen Implikationen dieser Dualität.
	
	\subsection{\href{\repobase/pdf/Deutsch/Vereinfachte Beschreibung der vier Grundkräfte mit Zeit-Masse-Dualität.pdf}{Vereinfachte Beschreibung der vier Grundkräfte mit Zeit-Masse-Dualität}}
	\textit{(233.251 Bytes, 27.03.2025)}
	
	Diese Publikation bietet eine umfassende mathematische Formulierung aller vier fundamentalen Kräfte (Gravitation, elektromagnetische, starke und schwache Kraft) im Rahmen der Zeit-Masse-Dualität. Sie entwickelt einen erweiterten Lagrange-Formalismus, der die standardmäßige Behandlung der Grundkräfte mit dem Konzept der intrinsischen Zeit verbindet. Besondere Aufmerksamkeit wird der modifizierten Gravitationstheorie und der Neuformulierung des Standardmodells der Teilchenphysik gewidmet, mit dem Ziel, eine kohärente Vereinheitlichung aller Kräfte zu erreichen.
	
	\subsection{\href{\repobase/pdf/Deutsch/Zeit als emergente Eigenschaft in der Quantenmechanik.pdf}{Zeit als emergente Eigenschaft in der Quantenmechanik}}
	\textit{(Im deutschen Verzeichnis)}
	
	Diese Arbeit untersucht, wie Zeit selbst als emergente Eigenschaft aus fundamentaleren Quantenprozessen verstanden werden kann. Sie stellt Verbindungen zwischen relativistischen Theorien, der Feinstrukturkonstante und der Quantendynamik durch das Konzept der intrinsischen Zeit her. Die Arbeit liefert eine detaillierte mathematische Analyse, wie diese Perspektive mehrere konzeptionelle Probleme in der Standardformulierung der Quantenmechanik löst.
	
	\section{Spezifische Anwendungen und Implikationen}
	
	Diese Publikationen untersuchen spezifische Anwendungen der Zeit-Masse-Dualität auf konkrete physikalische Phänomene und ihre Implikationen.
	
	\subsection{\href{\repobase/pdf/Deutsch/Dynamische Masse von Photonen und ihre Implikationen für Nichtlokalität.pdf}{Dynamische Masse von Photonen und ihre Implikationen für Nichtlokalität}}
	\textit{(182.875 Bytes, 25.03.2025)}
	
	Diese Arbeit erweitert das Konzept der Zeit-Masse-Dualität auf masselose Teilchen, insbesondere Photonen. Indem ein dynamisches Massekonzept für Photonen entwickelt wird, das mit ihrer Frequenz korreliert, bietet sie eine neue Perspektive auf Phänomene wie Quantenverschränkung und Nichtlokalität. Es wird argumentiert, dass die scheinbare instantane Korrelation verschränkter Photonen durch subtile, masseabhängige Zeitstrukturen erklärt werden kann, ohne klassische Kausalitätsprinzipien zu verletzen. Die Arbeit enthält quantitative Vorhersagen für experimentelle Tests dieser Hypothese.
	
	\subsection{\href{\repobase/pdf/Deutsch/Massenvariation in Galaxien.pdf}{Massenvariation in Galaxien}}
	\textit{(278.167 Bytes, 27.03.2025)}
	
	Diese Publikation wendet die Zeit-Masse-Dualität auf galaktische Strukturen an und entwickelt ein modifiziertes Gravitationsmodell, das auf systematischen Massenvariationen in Abhängigkeit vom galaktischen Radius basiert. Es wird gezeigt, wie dieses Modell flache Rotationskurven von Galaxien erklären kann, ohne auf das Konzept der dunklen Materie zurückgreifen zu müssen. Das vorgeschlagene modifizierte Gravitationspotential $\Phi(r) = -GM/r + \kappa r$ wird im Detail analysiert und mit astronomischen Beobachtungsdaten verglichen.
	
	\subsection{\href{\repobase/pdf/Deutsch/Vereinheitlichung des T0-Modells Grundlagen - Dunkle Energie und Galaxiendynamik.pdf}{Vereinheitlichung des T0-Modells: Grundlagen - Dunkle Energie und Galaxiendynamik}}
	\textit{(270.331 Bytes, 27.03.2025)}
	
	Diese umfassende Arbeit synthetisiert die Anwendungen des $T_0$-Modells auf kosmologische Phänomene. Sie entwickelt einen theoretischen Rahmen, der die kosmische Expansion, die Natur der dunklen Energie und die Dynamik von Galaxien in einem kohärenten Modell vereint. Die zentrale These ist, dass systematische Massenänderungen auf kosmischen Skalen sowohl die beschleunigte Expansion des Universums als auch die beobachteten Anomalien in der Galaxiendynamik erklären können. Die Arbeit quantifiziert die Wechselwirkung zwischen baryonischer Materie und dem postulierten dunklen Energiefeld durch den Kopplungsparameter $\beta \approx 10^{-3}$.
	
	\section{Fundamentale Konstanten und Einheiten}
	
	Diese Publikationen untersuchen die Beziehungen zwischen fundamentalen physikalischen Konstanten und entwickeln neue Perspektiven auf natürliche Einheitensysteme.
	
	\subsection{\href{\repobase/pdf/Deutsch/Fundamentale Konstanten und deren Herleitung aus natürlichen Einheiten.pdf}{Fundamentale Konstanten und deren Herleitung aus natürlichen Einheiten}}
	\textit{(252.503 Bytes, 25.03.2025)}
	
	Diese Arbeit analysiert fundamentale physikalische Konstanten wie die Feinstrukturkonstante, die Gravitationskonstante und die Planck-Konstante aus der Perspektive der Zeit-Masse-Dualität. Es wird untersucht, wie diese Konstanten als emergente Größen aus einem fundamentaleren theoretischen Rahmen abgeleitet werden können. Besondere Aufmerksamkeit wird der Beziehung zwischen der intrinsischen Zeit und diesen Konstanten gewidmet, mit dem Ziel, die Anzahl wirklich fundamentaler Konstanten zu reduzieren.
	
	\subsection{\href{\repobase/pdf/Deutsch/Natürliche Einheiten mit Feinstrukturkonstante alpha = 1.pdf}{Natürliche Einheiten mit Feinstrukturkonstante alpha = 1}}
	\textit{(196.636 Bytes, 26.03.2025)}
	
	Diese Publikation entwickelt ein revolutionäres natürliches Einheitensystem, in dem die Feinstrukturkonstante $\alpha = 1$ gesetzt wird, im Gegensatz zu ihrem empirischen Wert von etwa 1/137. Es wird argumentiert, dass dieses Einheitensystem tiefere theoretische Einsichten ermöglicht und mathematische Vereinfachungen bietet. Die Arbeit untersucht die Konsequenzen dieser Wahl für die Formulierung der fundamentalen Gesetze der Physik und die Interpretation empirischer Messungen, insbesondere im Kontext der Zeit-Masse-Dualität.
	
	\section{Kosmologische und Grenzgebiete}
	
	Diese Publikation erkundet die Implikationen der Zeit-Masse-Dualität für unser Verständnis der fundamentalsten Strukturen des Universums.
	
	\subsection{\href{\repobase/pdf/Deutsch/Jenseits der Planck-Skala.pdf}{Jenseits der Planck-Skala}}
	\textit{(233.861 Bytes, 25.03.2025)}
	
	Diese Arbeit untersucht die Konsequenzen der Zeit-Masse-Dualität für Phänomene jenseits der Planck-Skala, wo die konventionellen Theorien an ihre Grenzen stoßen. Es wird argumentiert, dass der neue theoretische Rahmen potentiell Singularitäten vermeiden und zu einem kohärenteren Verständnis extremer physikalischer Bedingungen führen kann. Die Arbeit entwickelt mathematische Modelle für den Übergang zwischen klassischen und quantenmechanischen Regimen und diskutiert Implikationen für das frühe Universum und schwarze Löcher.
	
	\section{Kurzfassungen und Überblicksdarstellungen}
	
	Diese Publikationen bieten kompakte Übersichten über die Hauptkonzepte und Erkenntnisse des breiteren Forschungsprogramms.
	
	\subsection{\href{\repobase/pdf/Deutsch kurzgefasst/Eine neue Perspektive auf Zeit und Raum Johann Paschers revolutionäre Ideen.pdf}{Eine neue Perspektive auf Zeit und Raum: Johann Paschers revolutionäre Ideen}}
	\textit{(60.676 Bytes, 25.03.2025)}
	
	Diese prägnante Überblicksarbeit führt die grundlegenden Konzepte des Zeit-Masse-Dualitätsrahmens einem breiteren Publikum vor. Sie fasst die wichtigsten Innovationen und potentiellen Auswirkungen auf unser Verständnis der Physik in einem zugänglichen Format zusammen.
	
	\subsection{\href{\repobase/pdf/Deutsch kurzgefasst/Kurzgefasst - Komplementärer Dualismus in der Physik - Von Welle-Teilchen zum Zeit-Masse-Konzept.pdf}{Kurzgefasst - Komplementärer Dualismus in der Physik - Von Welle-Teilchen zum Zeit-Masse-Konzept}}
	\textit{(144.630 Bytes, 25.03.2025)}
	
	Dieses Zusammenfassungsdokument konzentriert sich speziell auf die komplementäre Natur der Standardmodell- und T0-Modell-Ansätze. Es bietet eine kompakte Erklärung, wie diese dualen Perspektiven dieselbe physikalische Realität aus verschiedenen konzeptionellen Ausgangspunkten beschreiben können.
	
	\subsection{\href{\repobase/pdf/Deutsch kurzgefasst/Zusammenfassung - Fundamentale Konstanten.pdf}{Zusammenfassung - Fundamentale Konstanten}}
	\textit{(87.847 Bytes, 25.03.2025)}
	
	Diese Arbeit präsentiert einen kompakten Überblick darüber, wie fundamentale physikalische Konstanten im Rahmen der Zeit-Masse-Dualität neu interpretiert werden, und hebt das Potenzial hervor, die Anzahl wirklich fundamentaler Konstanten in der Physik zu reduzieren.
	
	\subsection{\href{\repobase/pdf/Deutsch kurzgefasst/Zeit und Masse Ein neuer Blick auf alte Formeln – und die Befreiung von traditionellen Fesseln.pdf}{Zeit und Masse: Ein neuer Blick auf alte Formeln – und die Befreiung von traditionellen Fesseln}}
	\textit{(91.807 Bytes, 25.03.2025)}
	
	Dieses Überblicksdokument untersucht, wie traditionelle physikalische Formeln durch die Linse der Zeit-Masse-Dualität neu interpretiert werden können, wodurch die theoretische Physik potentiell von langjährigen konzeptionellen Einschränkungen befreit und neue Forschungswege eröffnet werden könnten.
	
	\section{Englische Publikationen}
	
	Zusätzlich zu den deutschen Versionen sind viele der Arbeiten auch auf Englisch verfügbar:
	
	\begin{itemize}
		\item \href{\repobase/pdf/English/The Necessity of Extending Standard Quantum Mechanics and Quantum Field Theory.pdf}{The Necessity of Extending Standard Quantum Mechanics and Quantum Field Theory}
		\item \href{\repobase/pdf/English/complementary-extensions.pdf}{Complementary Extensions of Physics}
		\item \href{\repobase/pdf/English/Simplified Description of the Four Fundamental Forces with Time-Mass Duality.pdf}{Simplified Description of the Four Fundamental Forces with Time-Mass Duality}
		\item \href{\repobase/pdf/English/Time as an Emergent Property in Quantum Mechanics.pdf}{Time as an Emergent Property in Quantum Mechanics}
		\item \href{\repobase/pdf/English/Dynamic Mass of Photons and its Implications for Nonlocality.pdf}{Dynamic Mass of Photons and its Implications for Nonlocality}
		\item \href{\repobase/pdf/English/Unification of the T0 Model Foundations - Dark Energy and Galaxy Dynamics.pdf}{Unification of the T0 Model Foundations - Dark Energy and Galaxy Dynamics}
		\item \href{\repobase/pdf/English/fundamental-constants-derivation.pdf}{Fundamental Constants and Their Derivation from Natural Units}
		\item \href{\repobase/pdf/English/Energy as Fundamental Unit alpha = 1.pdf}{Energy as Fundamental Unit alpha = 1}
		\item \href{\repobase/pdf/English/Beyond the Planck Scale.pdf}{Beyond the Planck Scale}
		\item \href{\repobase/pdf/English/A Mathematical Analysis of Energy Dynamics.pdf}{A Mathematical Analysis of Energy Dynamics}
	\end{itemize}
	
	\subsection{Englische Kurzfassungen}
	
	\begin{itemize}
		\item \href{\repobase/pdf/English/A New Perspective on Time and Space Johann Pascher's Revolutionary Ideas.pdf}{A New Perspective on Time and Space: Johann Pascher's Revolutionary Ideas}
		\item \href{\repobase/pdf/English/Summary - Complementary Dualism in Physics.pdf}{Summary - Complementary Dualism in Physics}
		\item \href{\repobase/pdf/English/summary-fundamental-constants.pdf}{Summary - Fundamental Constants}
		\item \href{\repobase/pdf/English/Time and Mass A New Perspective on Old Formulas – and Liberation from Traditional Constraints.pdf}{Time and Mass: A New Perspective on Old Formulas – and Liberation from Traditional Constraints}
	\end{itemize}
	
	\section{Zusammenfassung und Ausblick}
	
	Die vorgestellten Publikationen bilden zusammen ein kohärentes Forschungsprogramm, das einen grundlegend neuen Ansatz zur Vereinheitlichung und Erweiterung der modernen Physik entwickelt. Im Gegensatz zu anderen vereinheitlichenden Ansätzen wie der Stringtheorie oder der Schleifenquantengravitation, die komplexe zusätzliche Strukturen einführen, konzentriert sich die Zeit-Masse-Dualität auf eine Neuformulierung der grundlegendsten Konzepte – Zeit und Masse.
	
	Die zentrale Idee der intrinsischen Zeit $T = \hbar/mc^2$ und der komplementären Modelle (Standardmodell und $T_0$-Modell) bietet nicht nur neue theoretische Perspektiven, sondern auch konkrete experimentell überprüfbare Vorhersagen. Dies umfasst masseabhängige Zeitentwicklung in Quantensystemen, subtile Effekte in Verschränkungsexperimenten und alternatives Verständnis kosmologischer Phänomene wie dunkle Materie und dunkle Energie.
	
	Zukünftige Forschungsrichtungen könnten folgende Bereiche umfassen:
	\begin{itemize}
		\item Entwicklung detaillierter experimenteller Protokolle zur Überprüfung der masseabhängigen Zeitentwicklung
		\item Verfeinerung der mathematischen Formulierung, insbesondere im Hinblick auf Quantengravitation
		\item Erweiterte numerische Simulationen zur Überprüfung der modifizierten Galaxiendynamik
		\item Einbeziehung weiterer fundamentaler Teilchen und Wechselwirkungen in den theoretischen Rahmen
		\item Anwendung der Theorie auf fundamentale Probleme der Quanteninformation und Quantencomputer
	\end{itemize}
	
	Das übergreifende Ziel bleibt die Entwicklung einer umfassenden, mathematisch eleganten und experimentell bestätigten "All-in-One"-Theorie, die Quantenmechanik, Quantenfeldtheorie und Allgemeine Relativitätstheorie in einem kohärenten Rahmen vereint und zugleich die bestehenden Probleme der modernen Physik wie Nichtlokalität, dunkle Materie und dunkle Energie adressiert.
	
\end{document}