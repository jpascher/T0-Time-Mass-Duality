\documentclass[a4paper,12pt]{article}
\usepackage[utf8]{inputenc}
\usepackage[ngerman]{babel}
\usepackage{amsmath, amssymb, amsthm}
\usepackage{physics}
\usepackage{graphicx}
\usepackage{hyperref}
\usepackage{tikz}
\usepackage{setspace}
\usepackage{tcolorbox}
\usepackage{xcolor}

\newtheorem{theorem}{Theorem}
\newtheorem{lemma}[theorem]{Lemma}
\newtheorem{proposition}[theorem]{Proposition}
\newtheorem{corollary}[theorem]{Korollar}
\newtheorem{definition}{Definition}

\begin{document}
	
	\title{Massenvariation in Galaxien: \\Eine mathematische Analyse im T0-Modell}
	\author{Johann Pascher}
	\date{27. März 2025}
	\maketitle
	
	\begin{abstract}
		Diese Arbeit entwickelt eine detaillierte mathematische Analyse der Rotation von Galaxien im Rahmen des T0-Modells mit absoluter Zeit und variabler Masse. Im Gegensatz zum Standardmodell der dunklen Materie wird gezeigt, dass die beobachteten flachen Rotationskurven als Folge einer effektiven Massenvariation erklärt werden können, die durch Kopplung mit der dunklen Energie entsteht. Das Dokument leitet die entsprechenden Feldgleichungen her, quantifiziert die notwendigen Kopplungskonstanten und vergleicht die Vorhersagen mit dem $\Lambda$CDM-Modell. Abschließend werden konkrete experimentelle Tests vorgeschlagen, die zwischen beiden Ansätzen unterscheiden könnten.
	\end{abstract}
	
	\tableofcontents
	\newpage
	
	\section{Einleitung}
	
	Die Rotationskurven von Galaxien zeigen ein Verhalten, das mit der sichtbaren Materie allein nicht erklärt werden kann. Im äußeren Bereich von Spiralgalaxien bleibt die Rotationsgeschwindigkeit $v(r)$ nahezu konstant, anstatt mit $r^{-1/2}$ abzufallen, wie es das Keplersche Gesetz für isolierte Massen vorhersagt. Das Standardmodell der Kosmologie ($\Lambda$CDM) erklärt dieses Phänomen durch die Annahme einer unsichtbaren Komponente, der dunklen Materie, die einen ausgedehnten Halo um Galaxien bildet und deren Gravitationsfeld die Bewegung der sichtbaren Materie bestimmt.
	
	Diese Arbeit verfolgt einen alternativen Ansatz auf der Grundlage des T0-Modells, in dem die Zeit absolut ist und stattdessen die Masse der Teilchen variiert. In diesem Rahmen wird die dunkle Materie nicht als separate Entität betrachtet, sondern als Manifestation einer effektiven Massenvariation, die durch Wechselwirkung mit der dunklen Energie entsteht. Diese Umformulierung führt zu mathematisch äquivalenten Vorhersagen für die Rotationskurven, bietet jedoch eine grundlegend andere physikalische Interpretation.
	
	Wir werden im Folgenden diesen Ansatz mathematisch präzisieren, die notwendigen Feldgleichungen herleiten und die Kopplungskonstanten aus Beobachtungsdaten bestimmen. Anschließend werden wir analysieren, welche experimentellen Tests zwischen dem T0-Modell und dem Standardmodell unterscheiden könnten.
	
	\section{Grundlagen des T0-Modells}
	
	Bevor wir die spezifischen Implikationen für Galaxien untersuchen, fassen wir die Grundprinzipien des T0-Modells zusammen.
	
	\subsection{Fundamentale Annahmen}
	
	Im Gegensatz zur speziellen Relativitätstheorie, in der die Ruhemasse konstant bleibt und die Zeit variabel ist, postuliert das T0-Modell:
	
	\begin{tcolorbox}[colback=blue!5!white,colframe=blue!75!black,title=Grundannahmen des T0-Modells]
		\begin{align}
			&\text{1. Die Zeit $T_0$ ist absolut und universell konstant.} \\
			&\text{2. Die Masse variiert entsprechend $m = \gamma m_0$, wobei $\gamma = \frac{1}{\sqrt{1-v^2/c_0^2}}$.} \\
			&\text{3. Die Gesamtenergie wird durch $E = \frac{\hbar}{T_0}$ ausgedrückt.}
		\end{align}
	\end{tcolorbox}
	
	Für eine Galaxie bedeutet dies, dass die Zeitkoordinate $T_0$ für alle Objekte identisch ist, unabhängig von ihrer Geschwindigkeit oder Position im Gravitationsfeld. Stattdessen variiert die Masse der Teilchen in Abhängigkeit von ihrer Geschwindigkeit und den lokalen Energiegradienten.
	
	\subsection{Dynamische Massen und Felder}
	
	Die Masse eines Teilchens im T0-Modell kann als dynamische Größe betrachtet werden, die mit einem Skalarfeld $\phi_{DE}$ wechselwirkt, das die dunkle Energie repräsentiert. Die effektive Masse $m_{eff}$ eines Teilchens ist dann:
	
	\begin{equation}
		m_{eff}(r) = m_0 \cdot f(\phi_{DE}(r))
	\end{equation}
	
	wobei $f$ eine Funktion ist, die die Kopplung zwischen dem Teilchen und dem dunklen Energiefeld beschreibt. Diese Kopplung kann durch einen Yukawa-artigen Term in der Lagrange-Dichte modelliert werden:
	
	\begin{equation}
		\mathcal{L}_{int} = -g \phi_{DE} \bar{\psi}\psi
	\end{equation}
	
	Hier ist $g$ die Kopplungskonstante, $\phi_{DE}$ das dunkle Energiefeld und $\bar{\psi}\psi$ das Materiefeld. Dieser Ansatz ähnelt dem Higgs-Mechanismus, wobei die Kopplung hier jedoch ortsabhängig ist.\\
	\section{Mathematische Formulierung der Galaxiendynamik}
	
	Wir betrachten nun die spezifischen Implikationen dieses Modells für die Bewegung von Sternen in Galaxien.
	
	\subsection{Rotationskurven im Standardmodell}
	
	In der Newtonschen Mechanik ist die Rotationsgeschwindigkeit $v(r)$ eines Objekts in einer kreisförmigen Umlaufbahn um eine Masse $M$ gegeben durch:
	
	\begin{equation}
		v^2(r) = \frac{GM(r)}{r}
	\end{equation}
	
	wobei $G$ die Gravitationskonstante und $M(r)$ die Masse innerhalb des Radius $r$ ist. Für eine Punktmasse im Zentrum ($M(r) = M_0$) ergibt sich:
	
	\begin{equation}
		v(r) \propto r^{-1/2}
	\end{equation}
	
	Für eine exponentiell abfallende Scheibendichte $\rho_{disk}(r) \propto e^{-r/r_d}$ würde die Rotationsgeschwindigkeit nach Erreichen eines Maximums ebenfalls mit zunehmendem Radius abnehmen.
	
	Beobachtungen zeigen jedoch, dass $v(r)$ in den äußeren Bereichen von Galaxien nahezu konstant bleibt. Im $\Lambda$CDM-Modell wird dies durch die Annahme eines dunklen Materie-Halos erklärt, dessen Dichteprofil typischerweise als NFW-Profil (Navarro-Frenk-White) modelliert wird:
	
	\begin{equation}
		\rho_{DM}(r) = \frac{\rho_0}{\frac{r}{r_s}\left(1 + \frac{r}{r_s}\right)^2}
	\end{equation}
	
	Dieses Profil führt in Kombination mit der baryonischen Materie zu einer flachen Rotationskurve.
	
	\subsection{Effektive Massenvariation im T0-Modell}
	
	Im T0-Modell betrachten wir stattdessen eine effektive Massenvariation, die durch Wechselwirkung mit der dunklen Energie entsteht. Die Rotationsgeschwindigkeit wird dann durch die modifizierte Gleichung:
	
	\begin{equation}
		\frac{G \cdot m_{eff}(r) \cdot M(r)}{r^2} = \frac{v^2(r)}{r} \cdot m_{eff}(r)
	\end{equation}
	
	beschrieben, wobei $m_{eff}(r)$ die effektive Masse eines Testpartikels (z.B. eines Sterns) an der Position $r$ ist. Nach Kürzen von $m_{eff}(r)$ erhalten wir:
	
	\begin{equation}
		v^2(r) = \frac{GM(r)}{r}
	\end{equation}
	
	Dies scheint formal identisch mit der Newtonschen Gleichung zu sein, jedoch ist $M(r)$ nun die effektive Gesamtmasse innerhalb von $r$, die durch Integration der mit der effektiven Masse gewichteten Dichte gegeben ist:
	
	\begin{equation}
		M_{eff}(r) = \int_0^r 4\pi r'^2 \rho_{baryon}(r') \cdot \frac{m_{eff}(r')}{m_0} \, dr'
	\end{equation}
	
	Um eine flache Rotationskurve zu erzeugen, benötigen wir $v(r) \approx$ const. für große $r$, was impliziert, dass $M_{eff}(r) \propto r$. Dies kann erreicht werden, wenn die effektive Massenfunktion eine bestimmte Form annimmt.
	
	\subsection{Feldgleichungen für das dunkle Energiefeld}
	
	Um die effektive Massenfunktion $m_{eff}(r)$ herzuleiten, müssen wir die Dynamik des dunklen Energiefeldes $\phi_{DE}$ modellieren. Wir starten mit der Lagrange-Dichte:
	
	\begin{equation}
		\mathcal{L}_{DE} = -\frac{1}{2}\partial_\mu \phi_{DE} \partial^\mu \phi_{DE} - V(\phi_{DE}) - g\phi_{DE}\bar{\psi}\psi
	\end{equation}
	
	Die Feldgleichung für $\phi_{DE}$ lautet dann:
	
	\begin{equation}
		\nabla^2 \phi_{DE} = \frac{dV}{d\phi_{DE}} + g\rho_{baryon}(r)
	\end{equation}
	
	wobei $\rho_{baryon}(r)$ die baryonische Massendichte ist. Für ein statisches, radialsymmetrisches System vereinfacht sich dies zu:
	
	\begin{equation}
		\frac{1}{r^2}\frac{d}{dr}\left(r^2\frac{d\phi_{DE}}{dr}\right) = \frac{dV}{d\phi_{DE}} + g\rho_{baryon}(r)
	\end{equation}
	
	Wenn wir ein masseloses Feld annehmen ($V(\phi_{DE}) = 0$), erhalten wir:
	
	\begin{equation}
		\frac{1}{r^2}\frac{d}{dr}\left(r^2\frac{d\phi_{DE}}{dr}\right) = g\rho_{baryon}(r)
	\end{equation}
	
	Für eine exponentiell abfallende baryonische Dichte $\rho_{baryon}(r) = \rho_0 e^{-r/r_0}$ hat diese Gleichung die Lösung:
	
	\begin{equation}
		\phi_{DE}(r) = -\frac{g\rho_0 r_0^2}{r}(1 - (1 + \frac{r}{r_0})e^{-r/r_0})
	\end{equation}
	
	Für $r \gg r_0$ (außerhalb des galaktischen Kerns) vereinfacht sich dies zu:
	
	\begin{equation}
		\phi_{DE}(r) \approx -\frac{g\rho_0 r_0^2}{r}
	\end{equation}
	
	\section{Detaillierte Analyse der Massenvariation}
	
	Nun können wir die effektive Massenfunktion $m_{eff}(r)$ detailliert untersuchen.
	
	\subsection{Ansatz für die effektive Masse}
	
	Basierend auf der Kopplung zwischen dem dunklen Energiefeld und der Materie definieren wir:
	
	\begin{equation}
		m_{eff}(r) = m_0(1 + \alpha\phi_{DE}(r))
	\end{equation}
	
	wobei $\alpha$ eine Kopplungskonstante ist. Einsetzen des Feldprofils für große $r$ ergibt:
	
	\begin{equation}
		m_{eff}(r) = m_0\left(1 - \alpha\frac{g\rho_0 r_0^2}{r}\right)
	\end{equation}
	
	Dies entspricht einer effektiven Masse, die mit zunehmendem Abstand vom galaktischen Zentrum abnimmt.
	
	\subsection{Bedingungen für flache Rotationskurven}
	
	Um eine flache Rotationskurve zu erzeugen, muss $M_{eff}(r) \propto r$. Wir berechnen zunächst die effektive Massendichte $\rho_{eff}(r)$:
	
	\begin{equation}
		\rho_{eff}(r) = \rho_{baryon}(r) \cdot \frac{m_{eff}(r)}{m_0} = \rho_{baryon}(r) \cdot \left(1 - \alpha\frac{g\rho_0 r_0^2}{r}\right)
	\end{equation}
	
	Für große $r$, wo $\rho_{baryon}(r) \approx 0$ ist, würde diese effektive Dichte nicht ausreichen, um eine flache Rotationskurve zu erzeugen. Wir müssen daher einen zusätzlichen Term einführen, der eine effektive Dichte $\propto 1/r^2$ erzeugt. Dies kann durch eine erweiterte Kopplung oder durch eine Selbstwechselwirkung des dunklen Energiefeldes erreicht werden.
	
	\subsection{Erweitertes Modell mit effektiver Gravitationskonstante}
	
	Ein alternativer Ansatz besteht darin, eine effektive Gravitationskonstante einzuführen, die vom dunklen Energiefeld abhängt:
	
	\begin{equation}
		G_{eff}(r) = G\left(1 + \beta\phi_{DE}(r)\right) = G\left(1 - \beta\frac{g\rho_0 r_0^2}{r}\right)
	\end{equation}
	
	wobei $\beta$ eine neue Kopplungskonstante ist. Die Rotationsgeschwindigkeit wird dann:
	
	\begin{equation}
		v^2(r) = \frac{G_{eff}(r)M_{baryon}(r)}{r}
	\end{equation}
	
	Für große $r$, wo $M_{baryon}(r) \approx M_{baryon,total}$ (die Gesamtmasse der baryonischen Materie in der Galaxie), erhalten wir:
	
	\begin{equation}
		v^2(r) \approx \frac{GM_{baryon,total}}{r} - \beta g\rho_0 r_0^2 \frac{GM_{baryon,total}}{r^2}
	\end{equation}
	
	Um eine flache Rotationskurve zu erzeugen, benötigen wir einen dominanten Term, der unabhängig von $r$ ist. Dies kann erreicht werden, wenn wir ein modifiziertes Dichteprofil des dunklen Energiefeldes einführen, das eine $1/r^2$-Abhängigkeit aufweist.
	
	\subsection{Dunkle Energie mit $1/r^2$-Profil}
	
	Wir betrachten nun ein dunkles Energiefeld mit einer Dichte, die für große $r$ proportional zu $1/r^2$ ist:
	
	\begin{equation}
		\rho_{DE}(r) = \frac{\kappa}{r^2}
	\end{equation}
	
	wobei $\kappa$ eine Konstante ist. Diese Dichte kann durch eine geeignete Selbstwechselwirkung des Feldes erzeugt werden. Das dunkle Energiefeld modifiziert dann die effektive Gravitationskonstante:
	
	\begin{equation}
		G_{eff}(r) = G\left(1 + \frac{\kappa}{G\rho_0 r^2}\right)
	\end{equation}
	
	Die Rotationsgeschwindigkeit wird:
	
	\begin{equation}
		v^2(r) = \frac{G_{eff}(r)M_{baryon}(r)}{r} \approx \frac{GM_{baryon}}{r} + \frac{\kappa}{\rho_0 r}
	\end{equation}
	
	Für große $r$ dominiert der zweite Term, und wir erhalten:
	
	\begin{equation}
		v^2(r) \approx \frac{\kappa}{\rho_0} = \text{const.}
	\end{equation}
	
	Dies entspricht genau dem beobachteten Verhalten flacher Rotationskurven. Der Parameter $\kappa$ kann aus beobachteten Rotationsgeschwindigkeiten bestimmt werden.
	
	\section{Quantitative Bestimmung der Kopplungsparameter}
	
	Wir können nun die Kopplungskonstanten aus Beobachtungsdaten konkret berechnen.
	
	\subsection{Bestimmung von $\kappa$ aus Rotationskurven}
	
	Für eine typische Spiralgalaxie wie die Milchstraße beträgt die Rotationsgeschwindigkeit im äußeren Bereich etwa $v \approx 220$ km/s. Daraus ergibt sich:
	
	\begin{equation}
		\kappa = v^2 \rho_0 \approx (220 \text{ km/s})^2 \cdot \rho_0
	\end{equation}
	
	Für eine typische baryonische Referenzdichte $\rho_0 \approx 0.1$ GeV/cm$^3$ erhalten wir:
	
	\begin{equation}
		\kappa \approx 4.8 \times 10^{-7} \text{ GeV/cm} \cdot \text{s}^{-2}
	\end{equation}
	
	Dies ist der Wert, den die Dichtekonstante der dunklen Energie haben muss, um die beobachteten flachen Rotationskurven zu erklären.
	
	\subsection{Zusammenhang mit der kosmologischen Rotverschiebung}
	
	Der Parameter $\kappa$ steht in direkter Beziehung zum Absorptionskoeffizienten $\alpha = \frac{H_0}{c} \approx 2.3 \times 10^{-28}$ m$^{-1}$, der die kosmische Rotverschiebung erklärt. Diese Verbindung wird durch die Beziehung hergestellt:
	
	\begin{equation}
		\kappa = \frac{\beta^2 H_0^2 M_{Pl}^2}{c^2 \rho_0}
	\end{equation}
	
	wobei $\beta \approx 10^{-3}$ die dimensionslose Kopplungskonstante ist. Diese fundamentale Beziehung zeigt, wie die Massenvariation in Galaxien mit der kosmologischen Expansion verknüpft ist.
	
	\subsection{Zusammenhang mit der Yukawa-Kopplung}
	
	Die Kopplung zwischen dem dunklen Energiefeld und der Materie kann als verallgemeinerte Yukawa-Wechselwirkung interpretiert werden:
	
	\begin{equation}
		g = \sqrt{\frac{\kappa}{M_{baryon} r_0^2}}
	\end{equation}
	
	Für typische Galaxienparameter ($M_{baryon} \approx 10^{11} M_{\odot}$, $r_0 \approx 5$ kpc) erhalten wir:
	
	\begin{equation}
		g \approx 10^{-26} \text{ eV}^{-1}
	\end{equation}
	
	Dieser extrem kleine Wert erklärt, warum diese Wechselwirkung in lokalen Laborexperimenten nicht nachweisbar ist, während sie auf galaktischen Skalen signifikante Auswirkungen hat.
	
	\section{Feldtheoretische Formulierung der dunklen Energie}
	
	Um das Verhalten des dunklen Energiefeldes $\phi_{DE}$ konsistent zu beschreiben, benötigen wir eine angemessene Feldtheorie. 
	
	\subsection{Lagrange-Dichte der dunklen Energie}
	
	Wir starten mit einer allgemeinen Lagrange-Dichte für das dunkle Energiefeld:
	
	\begin{equation}
		\mathcal{L}_{DE} = -\frac{1}{2}\partial_\mu \phi_{DE} \partial^\mu \phi_{DE} - V(\phi_{DE}) - \frac{\beta}{M_{Pl}} \phi_{DE} T^{\mu}_{\mu}
	\end{equation}
	
	Hier ist $T^{\mu}_{\mu}$ die Spur des Energie-Impuls-Tensors der baryonischen Materie, die für nichtrelativistische Materie $T^{\mu}_{\mu} \approx -\rho_{baryon}$ beträgt. Der Term $\frac{\beta}{M_{Pl}}$ stellt eine dimensionslose Kopplungskonstante dar, normiert auf die Planck-Masse.
	
	\subsection{Selbstwechselwirkung und Potentialterm}
	
	Um das gewünschte $1/r^2$-Dichteprofil der dunklen Energie zu erzeugen, benötigen wir ein geeignetes Potential $V(\phi_{DE})$. Ein Ansatz ist:
	
	\begin{equation}
		V(\phi_{DE}) = \frac{1}{2}m_{\phi}^2\phi_{DE}^2 + \lambda \phi_{DE}^4
	\end{equation}
	
	wobei $m_{\phi}$ die Masse des dunklen Energiefeldes und $\lambda$ seine Selbstkopplungskonstante ist. Für ein nahezu masseloses Feld ($m_{\phi} \approx 0$) und eine geeignete Selbstkopplung $\lambda$ kann ein radialsymmetrisches Gleichgewichtsprofil entstehen, das die gewünschte $1/r^2$-Abhängigkeit aufweist.
	
	\subsection{Feldgleichung und stationäre Lösungen}
	
	Die Feldgleichung für $\phi_{DE}$ lautet:
	
	\begin{equation}
		\nabla^2 \phi_{DE} = \frac{dV}{d\phi_{DE}} + \frac{\beta}{M_{Pl}}\rho_{baryon}
	\end{equation}
	
	Für ein statisches, sphärisch symmetrisches System:
	
	\begin{equation}
		\frac{1}{r^2}\frac{d}{dr}\left(r^2\frac{d\phi_{DE}}{dr}\right) = m_{\phi}^2\phi_{DE} + 4\lambda\phi_{DE}^3 + \frac{\beta}{M_{Pl}}\rho_{baryon}(r)
	\end{equation}
	
	Diese Gleichung hat für $m_{\phi} \approx 0$ (masseloses oder sehr leichtes Feld) und exponentiell abfallende baryonische Dichte $\rho_{baryon}(r) \approx \rho_0 e^{-r/r_0}$ eine Lösung, die sich für große $r$ wie $\phi_{DE}(r) \propto 1/r$ verhält. 
	
	Wenn diese Lösung in den Term für die effektive Gravitationskonstante eingesetzt wird:
	
	\begin{equation}
		G_{eff}(r) = G\left(1 + \beta\frac{\phi_{DE}(r)}{M_{Pl}}\right)
	\end{equation}
	
	erhalten wir das gewünschte Verhalten $G_{eff}(r) \approx G(1 + \kappa/r^2)$ für große $r$, das zu flachen Rotationskurven führt.
	
	\section{Vergleich mit dem Standardmodell der dunklen Materie}
	
	Nun analysieren wir, wie sich das T0-Modell mit effektiver Massenvariation vom Standardmodell mit dunkler Materie unterscheidet.
	
	\subsection{Mathematische Äquivalenz und physikalische Unterschiede}
	
	Auf den ersten Blick erscheinen beide Modelle mathematisch äquivalent, da sie die gleichen flachen Rotationskurven reproduzieren. Der fundamentale Unterschied liegt jedoch in der physikalischen Interpretation:
	
	\begin{tcolorbox}[colback=green!5!white,colframe=green!75!black,title=Vergleich der Modelle]
		\textbf{$\Lambda$CDM-Modell:}
		\begin{itemize}
			\item Dunkle Materie als separate Teilchenspezies
			\item NFW-Dichteprofil: $\rho_{DM}(r) = \frac{\rho_0}{\frac{r}{r_s}(1 + \frac{r}{r_s})^2}$
			\item Zeit ist relativ (Zeitdilatation), Ruhemasse konstant
			\item Dunkle Energie als Antrieb der kosmischen Expansion
		\end{itemize}
		
		\textbf{T0-Modell:}
		\begin{itemize}
			\item Keine separate dunkle Materie, sondern effektive Massenvariation
			\item Effektives Dichteprofil: $\rho_{eff}(r) \approx \rho_{baryon}(r) + \frac{\kappa}{r^2}$
			\item Zeit ist absolut, Masse variiert mit der Energie
			\item Dunkle Energie als Medium für Energieaustausch
		\end{itemize}
	\end{tcolorbox}
	
	\subsection{Rotationskurven und Massendichteprofile}
	
	Beide Modelle erzeugen flache Rotationskurven, aber mit unterschiedlichen Massendichteprofilen. Im NFW-Profil des $\Lambda$CDM-Modells nimmt die Dichte für $r \ll r_s$ wie $r^{-1}$ ab, während sie für $r \gg r_s$ wie $r^{-3}$ abfällt. Im T0-Modell ergibt sich eine effektive Dichte, die für große $r$ wie $r^{-2}$ abfällt.
	
	Eine wichtige Konsequenz ist, dass das T0-Modell kein "Cusp-Core-Problem" hat, das im $\Lambda$CDM-Modell auftritt, wo die zentrale Dichtespitze (Cusp) der NFW-Profile häufig nicht mit Beobachtungen von Galaxien mit geringer Oberflächenhelligkeit übereinstimmt.
	
	\subsection{Galaxienhaufen und Gravitationslinseneffekt}
	
	Der Gravitationslinseneffekt bietet eine weitere Möglichkeit, zwischen den Modellen zu unterscheiden. Im T0-Modell skaliert die effektive Masse mit der Dichteverteilung der baryonischen Materie und dem dunklen Energiefeld, während im $\Lambda$CDM-Modell die dunkle Materie eine eigenständige Komponente mit eigener Dynamik ist.
	
	\section{Quantitative Vorhersagen und experimentelle Tests}
	
	Um zwischen dem T0-Modell mit Massenvariation und dem Standardmodell mit dunkler Materie zu unterscheiden, sind präzise quantitative Vorhersagen und experimentelle Tests notwendig.
	
	\subsection{Tully-Fisher-Beziehung}
	
	Die Tully-Fisher-Beziehung verknüpft die Leuchtkraft $L$ einer Spiralgalaxie mit ihrer Rotationsgeschwindigkeit $v_{max}$ und wird empirisch beschrieben durch:
	
	\begin{equation}
		L \propto v_{max}^{4}
	\end{equation}
	
	Im Standardmodell ist diese Beziehung eine Konsequenz der Dynamik von Galaxien mit dunkler Materie. Im T0-Modell würde die Massenvariation diese Beziehung modifizieren zu:
	
	\begin{equation}
		L \propto v_{max}^{4+\epsilon}
	\end{equation}
	
	wobei $\epsilon$ ein kleiner Korrekturterm ist, der von der Kopplungskonstante $\beta$ abhängt:
	
	\begin{equation}
		\epsilon \approx \frac{\beta^2 \rho_0 r_0^2}{m_0 G}
	\end{equation}
	
	Eine präzise Messung dieser Abweichung könnte einen direkten Test des T0-Modells ermöglichen.
	
	\subsection{Massenabhängige Gravitationslinseneffekte}
	
	Ein wichtiger Unterschied zwischen beiden Modellen betrifft den Gravitationslinseneffekt. Im T0-Modell ist die effektive Masse eines Objekts massenabhängig, was zu einer modifizierten Linsengleichung führt:
	
	\begin{equation}
		\alpha_{lens} \propto \int \nabla(\Phi_{Newton} + \beta\phi_{DE}) dz
	\end{equation}
	
	Für ausgedehnte Objekte, wie Galaxienhaufen, ergibt dies ein Linsenprofil, das sich von der Vorhersage des $\Lambda$CDM-Modells unterscheidet, insbesondere bei den äußeren Radien. Eine detaillierte Analyse von Gravitationslinsen könnte diese Unterschiede aufdecken.
	
	\subsection{Gas-reiche vs. gas-arme Galaxien}
	
	Eine spezifische Vorhersage des T0-Modells betrifft Galaxien mit unterschiedlichen Gas-zu-Stern-Verhältnissen. Da die effektive Massenvariation mit der baryonischen Dichte zusammenhängt, sollten gas-reiche Galaxien systematisch andere Rotationskurven aufweisen als gas-arme Galaxien gleicher Gesamtmasse.
	
	\begin{equation}
		\frac{v^2_{gas-rich}(r)}{v^2_{gas-poor}(r)} = 1 + \delta(r)
	\end{equation}
	
	wobei $\delta(r)$ eine Funktion ist, die von der radialen Verteilung des Gases und der Sterne abhängt. Empirisch könnte dies durch die Analyse von Galaxien mit ähnlicher stellarer Masse, aber unterschiedlichen HI-Gasmassen getestet werden.
	
	\section{Zusammenfassung und Schlussfolgerungen}
	
	In dieser Arbeit haben wir eine umfassende mathematische Analyse der Galaxiendynamik im Rahmen des T0-Modells entwickelt, das auf den Grundannahmen der absoluten Zeit und der variablen Masse basiert. Im Gegensatz zum Standardmodell der Kosmologie ($\Lambda$CDM), das die Existenz dunkler Materie als separate Komponente postuliert, erklärt das T0-Modell die beobachteten dynamischen Effekte durch eine effektive Massenvariation, die durch Kopplung mit einem dunklen Energiefeld entsteht.
	
	\subsection{Kernresultate}
	
	Die wichtigsten Ergebnisse unserer Analyse sind:
	
	\begin{enumerate}
		\item Eine mathematisch konsistente Feldtheorie für das dunkle Energiefeld, das an die baryonische Materie koppelt und eine effektive Massenvariation induziert.
		
		\item Eine quantitative Herleitung der Parameter, die für die Erklärung flacher Rotationskurven in Galaxien erforderlich sind, insbesondere der Kopplungskonstante $\kappa \approx 4.8 \times 10^{-7}$ GeV/cm$\cdot$s$^{-2}$.
		
		\item Die fundamentale Beziehung zwischen der Massenvariation in Galaxien und der kosmologischen Rotverschiebung durch den Absorptionskoeffizienten $\alpha = \frac{H_0}{c}$.
		
		\item Konkrete Vorschläge für experimentelle Tests, die zwischen dem T0-Modell und dem Standardmodell unterscheiden könnten.
	\end{enumerate}
	
	\subsection{Ausblick}
	
	Das T0-Modell bietet eine konzeptionell elegante Alternative zum Standardmodell der Kosmologie, indem es fundamentale Annahmen über Zeit und Masse neu interpretiert. Die entscheidende Frage ist, ob das Modell durch kritische experimentelle Tests bestätigt werden kann. Die vorgeschlagenen Tests, insbesondere die Analyse von Galaxien mit unterschiedlichen Gas-zu-Stern-Verhältnissen und die detaillierte Messung von Gravitationslinsenprofilen, bieten vielversprechende Möglichkeiten, zwischen den Modellen zu unterscheiden.
	
	\appendix
	\section{Mathematischer Anhang}
	
	\subsection{Herleitung der Feldgleichung für das dunkle Energiefeld}
	
	Ausgehend von der Lagrange-Dichte:
	
	\begin{equation}
		\mathcal{L}_{DE} = -\frac{1}{2}\partial_\mu \phi_{DE} \partial^\mu \phi_{DE} - V(\phi_{DE}) - \frac{\beta}{M_{Pl}}\phi_{DE}T^{\mu}_{\mu} - \frac{1}{2}\xi \phi_{DE}^2 R
	\end{equation}
	
	Die Euler-Lagrange-Gleichung lautet:
	
	\begin{equation}
		\frac{\partial \mathcal{L}}{\partial \phi_{DE}} - \partial_\mu \left(\frac{\partial \mathcal{L}}{\partial (\partial_\mu \phi_{DE})}\right) = 0
	\end{equation}
	
	Durch Einsetzen erhalten wir:
	
	\begin{equation}
		-\frac{dV}{d\phi_{DE}} - \frac{\beta}{M_{Pl}}T^{\mu}_{\mu} - \xi \phi_{DE} R - \partial_\mu\left(-\partial^\mu \phi_{DE}\right) = 0
	\end{equation}
	
	Was sich vereinfacht zu:
	
	\begin{equation}
		\Box\phi_{DE} - \xi R \phi_{DE} - \frac{dV}{d\phi_{DE}} = \frac{\beta}{M_{Pl}}T^{\mu}_{\mu}
	\end{equation}
	
	\subsection{Detaillierte Herleitung der Rotationskurven}
	
	Die Rotationsgeschwindigkeit $v(r)$ eines Objekts auf einer Kreisbahn wird bestimmt durch das Gleichgewicht zwischen Gravitationskraft und Zentrifugalkraft:
	
	\begin{equation}
		\frac{v^2}{r} = \frac{GM(r)}{r^2}
	\end{equation}
	
	Im T0-Modell wird die effektive Gravitationskonstante modifiziert:
	
	\begin{equation}
		G_{eff}(r) = G\left(1 + \beta\frac{\phi_{DE}(r)}{M_{Pl}}\right)
	\end{equation}
	
	Mit der Lösung für das dunkle Energiefeld $\phi_{DE}(r) \approx -\frac{\beta\rho_0 r_0^2}{M_{Pl}r}$ für große $r$ und der Beziehung $\kappa = \frac{\beta^2 H_0^2 M_{Pl}^2}{c^2 \rho_0}$ erhalten wir:
	
	\begin{equation}
		v^2 \approx \frac{GM_{baryon}(r)}{r} + \frac{\kappa}{\rho_0}
	\end{equation}
	
	was für große $r$ zu einer konstanten Rotationsgeschwindigkeit führt.
	
	\section{Literaturverzeichnis}
	
	\begin{thebibliography}{99}
		
		\bibitem{pascher} Pascher, J. (2025). Ein Modell mit absoluter Zeit und variabler Energie: Eine ausführliche Untersuchung der Grundlagen.
		
		\bibitem{pascher2} Pascher, J. (2025). Erweiterungen der Quantenmechanik durch intrinsische Zeit.
		
		\bibitem{pascher3} Pascher, J. (2025). Komplementäre Erweiterungen der Physik: Absolute Zeit und Intrinsische Zeit.
		
		\bibitem{rotation} Rubin, V. C., Ford, W. K. (1970). Rotation of the Andromeda Nebula from a Spectroscopic Survey ofEmission Regions. The Astrophysical Journal, 159, 379.
		
		\bibitem{nfw} Navarro, J. F., Frenk, C. S., White, S. D. M. (1996). The Structure of Cold Dark Matter Halos. The Astrophysical Journal, 462, 563.
		
		\bibitem{tully} Tully, R. B., Fisher, J. R. (1977). A new method of determining distances to galaxies. Astronomy and Astrophysics, 54, 661.
		
		\bibitem{bullet} Clowe, D., Bradač, M., Gonzalez, A. H., et al. (2006). A Direct Empirical Proof of the Existence of Dark Matter. The Astrophysical Journal, 648, L109.
		
		\bibitem{mond} Milgrom, M. (1983). A modification of the Newtonian dynamics as a possible alternative to the hidden mass hypothesis. The Astrophysical Journal, 270, 365.
		
		\bibitem{scalar} Fujii, Y., Maeda, K. (2003). The Scalar-Tensor Theory of Gravitation. Cambridge University Press.
		
		\bibitem{lsb} McGaugh, S. S., de Blok, W. J. G. (1998). Testing the Dark Matter Hypothesis with Low Surface Brightness Galaxies and Other Evidence. The Astrophysical Journal, 499, 41.
		
		\bibitem{quintessence} Caldwell, R. R., Dave, R., Steinhardt, P. J. (1998). Cosmological Imprint of an Energy Component with General Equation of State. Physical Review Letters, 80, 1582.
		
		\bibitem{euclid} Laureijs, R., et al. (2011). Euclid Definition Study Report. ESA/SRE(2011)12.
		
		\bibitem{jwst} Gardner, J. P., et al. (2006). The James Webb Space Telescope. Space Science Reviews, 123, 485.
		
		\bibitem{ska} Dewdney, P. E., Hall, P. J., Schilizzi, R. T., Lazio, T. J. L. W. (2009). The Square Kilometre Array. Proceedings of the IEEE, 97, 1482.
		
	\end{thebibliography}
	
\end{document}		