\documentclass[12pt,a4paper]{article}
\usepackage[utf8]{inputenc}
\usepackage[T1]{fontenc}
\usepackage[english]{babel}
\usepackage{lmodern}
\usepackage{amsmath}
\usepackage{amssymb}
\usepackage{physics}
\usepackage{hyperref}
\usepackage{bookmark}
\usepackage{tcolorbox}
\usepackage{booktabs}
\usepackage{enumitem}
\usepackage[table,xcdraw]{xcolor}
\usepackage[left=2cm,right=2cm,top=2cm,bottom=2cm]{geometry}
\usepackage{pgfplots}
\pgfplotsset{compat=1.18}
\usepackage{graphicx}
\usepackage{float}
\usepackage{fancyhdr}
\usepackage{siunitx}
\usepackage{url}
\usepackage{bm}

% Acknowledgments environment
\newenvironment{acknowledgments}
{\section*{Acknowledgments}}
{\vspace{1em}}

% Custom commands
\newcommand{\Tfield}{T(x)}
\newcommand{\alphaEM}{\alpha_{\text{EM}}}
\newcommand{\alphaW}{\alpha_{\text{W}}}
\newcommand{\betaT}{\beta_{\text{T}}}
\newcommand{\Mpl}{M_{\text{Pl}}}
\newcommand{\Tzerot}{T_0(\Tfield)}
\newcommand{\Tzero}{T_0}
\newcommand{\vecx}{\vec{x}}
\newcommand{\vr}{\vec{r}}
\newcommand{\gammaf}{\gamma_{\text{Lorentz}}}
\newcommand{\DhiggsT}{\Tfield (\partial_\mu + ig A_\mu) \Phi + \Phi \partial_\mu \Tfield}
\newcommand{\LCDM}{\Lambda\text{CDM}}
\newcommand{\DTmu}{D_{T,\mu}}
\newcommand{\calL}{\mathcal{L}}
\newcommand{\deq}{\displaystyle}
\newcommand{\e}{\mathrm{e}}

% Header and Footer Configuration
\pagestyle{fancy}
\fancyhf{}
\fancyhead[L]{Johann Pascher}
\fancyhead[R]{Extended Standard Model Framework}
\fancyfoot[C]{\thepage}
\renewcommand{\headrulewidth}{0.4pt}
\renewcommand{\footrulewidth}{0.4pt}

\hypersetup{
	colorlinks=true,
	linkcolor=blue,
	citecolor=blue,
	urlcolor=blue,
	pdftitle={Completing the Standard Model: An Extension Compatible with the T0 Model of Time-Mass Duality},
	pdfauthor={Johann Pascher},
	pdfsubject={Theoretical Physics},
	pdfkeywords={standard model, extension, curvature-based redshift, T0 model compatibility}
}

\title{Completing the Standard Model: An Extension Compatible with\\the T0 Model of Time-Mass Duality}
\author{Johann Pascher\\
	Department of Communication Technology\\
	Higher Technical Federal Institute (HTL), Leonding, Austria\\
	\texttt{johann.pascher@gmail.com}}
\date{April 17, 2025}

\begin{document}
	
	\maketitle
	
	\begin{abstract}
		This paper presents a systematic extension of the standard model of physics to address its incompleteness, particularly in the domains of quantum gravity, dark energy, and cosmic structure formation. By introducing specific modifications—while preserving the core principle of time dilation—this extended framework becomes fully compatible with the predictions of the alternative T0 model based on time-mass duality. We demonstrate that a curvature-based interpretation of redshift, centered on the introduction of a linear term $\kappa r$ to the gravitational potential, eliminates the need for universal expansion while maintaining the geometric foundation of general relativity. Both the extended standard model and the T0 model provide mathematically equivalent descriptions of physical reality, though they differ fundamentally in their philosophical interpretation of time and mass. We outline crucial experiments that could help determine which framework more accurately reflects physical reality.
	\end{abstract}
	
	\tableofcontents
	\newpage
	
	\section{Introduction: Limitations of the Current Standard Model}
	
	The standard model of physics, while remarkably successful in many domains, exhibits several theoretical and observational inconsistencies:
	
	\begin{enumerate}
		\item The incompatibility between general relativity and quantum mechanics
		\item The unexplained acceleration of cosmic expansion (dark energy)
		\item Anomalous galactic rotation curves requiring dark matter
		\item Tensions in Hubble constant measurements
		\item The hierarchy problem between electroweak and Planck scales
		\item The cosmological constant problem
	\end{enumerate}
	
	These limitations suggest the standard model is incomplete rather than incorrect. Meanwhile, the alternative T0 model proposed by Pascher offers a different foundation based on absolute time and variable mass, potentially addressing these issues with its intrinsic time field $\Tfield$. However, a complete abandonment of the standard framework is unnecessary if appropriate extensions can achieve equivalent predictive power.
	
	A key innovation we propose is the introduction of a modified gravitational potential with a linear term:
	
	\begin{equation}
		\Phi(r) = -\frac{GM}{r} + \kappa r
	\end{equation}
	
	With $\kappa \approx 4.8\times10^{-11}$ m/s². This single modification cascades through the theory, enabling a curvature-based explanation for redshift that eliminates the need for universal expansion while maintaining the geometric framework of general relativity.
	
	\section{Mathematical Extensions to the Standard Model}
	
	\subsection{Extended Einstein Field Equations}
	
	The standard Einstein field equations can be extended to:
	
	\begin{equation}
		G_{\mu\nu} + \kappa g_{\mu\nu} = 8\pi G T_{\mu\nu} + \nabla_{\mu}\Theta\nabla_{\nu}\Theta - \frac{1}{2}g_{\mu\nu}(\nabla_{\sigma}\Theta\nabla^{\sigma}\Theta)
	\end{equation}
	
	Where $\Theta$ is a scalar field accounting for the effects attributed to the intrinsic time field in the T0 model. This extension maintains the geometric interpretation of gravity while introducing effects that mimic the variable mass approach of the T0 model.
	
	\subsection{Curvature-Based Redshift Formula}
	
	The standard model interprets cosmic redshift as evidence of universal expansion. We propose an alternative interpretation based entirely on space curvature generated by the modified gravitational potential:
	
	\begin{equation}
		\Phi(r) = -\frac{GM}{r} + \kappa r
	\end{equation}
	
	This produces a static, curved spacetime with the metric:
	
	\begin{equation}
		ds^2 = (1 - \frac{2GM}{r} + 2\kappa r)dt^2 - (1 + \frac{2GM}{r} - 2\kappa r)dr^2 - r^2d\Omega^2
	\end{equation}
	
	In general relativity, the gravitational redshift between two points is given by:
	
	\begin{equation}
		1 + z = \sqrt{\frac{g_{00}(\text{emission})}{g_{00}(\text{observation})}}
	\end{equation}
	
	Using our modified metric with $g_{00} = (1 - \frac{2GM}{r} + 2\kappa r)$, and considering paths where the $\kappa r$ term dominates, this produces the redshift formula:
	
	\begin{equation}
		1 + z = e^{\alpha d}(1 + \beta \ln(\lambda/\lambda_0))
	\end{equation}
	
	with $\alpha \approx 2.3\times10^{-18}$ m$^{-1}$ and $\beta \approx 0.008$ in SI units. The wavelength dependence emerges naturally from the dispersive properties of the curved spacetime, where $\kappa$ has a frequency dependence:
	
	\begin{equation}
		\kappa(\lambda) = \kappa_0(1 + \beta \ln(\lambda/\lambda_0))
	\end{equation}
	
	This approach maintains general relativity's geometric foundation while eliminating the need for universal expansion entirely.
	
	\subsection{Extended Gravitational Potential}
	
	The Newtonian gravitational potential is extended to include a linear term:
	
	\begin{equation}
		\Phi(r) = -\frac{GM}{r} + \kappa r
	\end{equation}
	
	with $\kappa \approx 4.8\times10^{-11}$ m/s². This modification accounts for observed galactic dynamics without requiring dark matter, while remaining compatible with a relativistic interpretation.
	
	\subsection{Modified Quantum Evolution}
	
	Standard quantum mechanics can be extended with a mass-dependent time evolution correction:
	
	\begin{equation}
		i\hbar\frac{\partial\Psi}{\partial t} = [\hat{H} + \hat{H}_{\Theta}]\Psi
	\end{equation}
	
	where $\hat{H}_{\Theta}$ introduces mass-dependency to time evolution:
	
	\begin{equation}
		\hat{H}_{\Theta} = -i\hbar\frac{\partial\Theta}{\partial t}\Psi
	\end{equation}
	
	This preserves the standard Schrödinger framework while accounting for effects that the T0 model attributes to variable mass.
	
	\subsection{Extended Lagrangian Density}
	
	The Standard Model Lagrangian can be extended as:
	
	\begin{equation}
		\mathcal{L}_{\text{Total}} = \mathcal{L}_{\text{SM}} + \mathcal{L}_{\Theta}
	\end{equation}
	
	where $\mathcal{L}_{\text{SM}}$ is the Standard Model Lagrangian and:
	
	\begin{equation}
		\mathcal{L}_{\Theta} = \frac{1}{2}\partial_{\mu}\Theta\partial^{\mu}\Theta - V(\Theta) + f(\Theta)F_{\mu\nu}F^{\mu\nu} + g(\Theta)\bar{\psi}\gamma^{\mu}\partial_{\mu}\psi
	\end{equation}
	
	with coupling functions $f(\Theta)$ and $g(\Theta)$ accounting for interactions of $\Theta$ with electromagnetic fields and fermions.
	
	\subsection{Natural Units Reinterpretation}
	
	We propose that fundamental constants approach simple values at appropriate energy scales:
	
	\begin{equation}
		\alpha_{\text{EM}}(\mu) = \alpha_{\text{EM}}(0)[1 + \eta \ln(\mu/\mu_0)]
	\end{equation}
	
	This suggests that the apparent complexity of dimensional constants is an artifact of our measurement systems rather than a fundamental feature of nature.
	
	\section{Integration of Curvature-Based Cosmology with the Extended Standard Model}
	
	The introduction of curvature-based redshift creates a unified framework that integrates seamlessly with the other extensions to the standard model:
	
	\subsection{Connection to the Scalar Field $\Theta$}
	
	The scalar field $\Theta$ in our extended Einstein field equations:
	
	\begin{equation}
		G_{\mu\nu} + \kappa g_{\mu\nu} = 8\pi G T_{\mu\nu} + \nabla_{\mu}\Theta\nabla_{\nu}\Theta - \frac{1}{2}g_{\mu\nu}(\nabla_{\sigma}\Theta\nabla^{\sigma}\Theta)
	\end{equation}
	
	generates precisely the spacetime curvature needed to produce both the modified gravitational potential and the resultant redshift effect. The field equation:
	
	\begin{equation}
		\nabla^2\Theta + V'(\Theta) = 0
	\end{equation}
	
	When solved, creates the $\kappa r$ term in physical space.
	
	\subsection{Unification of Quantum and Cosmological Phenomena}
	
	This approach bridges quantum and cosmological scales:
	
	\begin{enumerate}
		\item At quantum scales, the $\Theta$ field modifies quantum evolution through:
		\begin{equation}
			i\hbar\frac{\partial\Psi}{\partial t} = [\hat{H} + \hat{H}_{\Theta}]\Psi
		\end{equation}
		
		\item At galactic scales, it produces flat rotation curves via the modified potential:
		\begin{equation}
			\Phi(r) = -\frac{GM}{r} + \kappa r
		\end{equation}
		
		\item At cosmological scales, it creates redshift through spacetime curvature:
		\begin{equation}
			1 + z = e^{\alpha d}(1 + \beta \ln(\lambda/\lambda_0))
		\end{equation}
	\end{enumerate}
	
	\subsection{Equivalence with the T0 Model}
	
	The extended standard model with curvature-based redshift achieves mathematical equivalence with the T0 model:
	
	\begin{enumerate}
		\item The scalar field $\Theta$ functionally corresponds to the intrinsic time field $T(x)$ in the T0 model
		\item The modified gravitational potential produces identical predictions for galactic rotation curves
		\item The curvature-based redshift generates the same wavelength-dependent redshift formula as the T0 model
		\item The quantum evolution corrections reproduce the effects of the T0 model's modified Schrödinger equation
	\end{enumerate}
	
	The fundamental difference remains philosophical: the extended standard model explains these phenomena through spacetime geometry while maintaining time dilation with constant rest mass, whereas the T0 model assumes absolute time with variable mass.
	
	\section{Experimental Tests to Distinguish the Models}
	
	While mathematically equivalent in their predictions, the extended standard model and T0 model differ in their fundamental interpretations. The following experiments could help determine which framework better reflects physical reality:
	
	\subsection{Precision Tests of Wavelength-Dependent Redshift}
	
	Both models predict wavelength-dependent redshift, but from different mechanisms. Precise spectroscopic measurements across multiple wavelength bands for objects at varying distances could distinguish between expansion-based and energy-loss-based redshift.
	
	\subsection{Gravitational Wave Propagation}
	
	The two models may predict subtly different behavior for gravitational wave propagation, particularly regarding dispersion and energy loss mechanisms over cosmological distances.
	
	\subsection{Black Hole Horizon Physics}
	
	Near the event horizon of black holes, the two interpretations may lead to different predictions regarding the behavior of quantum fields and the information paradox.
	
	\subsection{High-Precision Tests of the Equivalence Principle}
	
	The extended standard model and T0 model might predict different higher-order corrections to the equivalence principle when considering quantum effects.
	
	\subsection{Early Universe Physics}
	
	The cosmological predictions of the models diverge significantly regarding the early universe, with the standard model requiring inflation while the T0 model suggests a static, eternal universe.
	
	\section{Implications for Theoretical Physics}
	
	\subsection{Paradigm Shift in Cosmology}
	
	The curvature-based redshift interpretation represents a fundamental shift in cosmology:
	
	\begin{enumerate}
		\item \textbf{Static Universe}: The universe can be static rather than expanding, eliminating the need for inflation, Big Bang, and dark energy
		\item \textbf{Geometric Redshift}: Redshift becomes a purely geometric phenomenon rather than a Doppler effect
		\item \textbf{Unified Explanation}: Galactic rotation curves and cosmic redshift share a common origin in the $\kappa r$ term
		\item \textbf{No Beginning}: This model suggests an eternal universe without a beginning or end
	\end{enumerate}
	
	\subsection{Broader Theoretical Implications}
	
	The complete framework has profound implications for theoretical physics:
	
	\begin{enumerate}
		\item It suggests that our understanding of time and mass may be framework-dependent rather than fundamental
		\item It shows that dark matter and dark energy might be artifacts of an incomplete standard model rather than necessary components of reality
		\item It implies that quantum gravity could potentially be achieved through extensions to existing frameworks rather than requiring entirely new approaches
		\item It demonstrates that natural units with all constants set to unity may reflect a deeper reality than our conventional measurement systems
	\end{enumerate}
	
	\subsection{The Role of the $\kappa r$ Term}
	
	The linear term $\kappa r$ in the gravitational potential emerges as the cornerstone of the extended model:
	
	\begin{enumerate}
		\item It modifies galactic dynamics, eliminating the need for dark matter
		\item It generates the spacetime curvature that produces redshift, eliminating the need for expansion
		\item It connects to the scalar field $\Theta$ that modifies quantum evolution
		\item It introduces a fundamental length scale that relates to the natural unit system
	\end{enumerate}
	
	This single term, with $\kappa \approx 4.8\times10^{-11}$ m/s², resolves multiple outstanding issues in physics while maintaining the geometric foundation of general relativity.
	
	\section{Mathematical Formulation of Curvature-Based Redshift}
	
	The mathematical derivation of the curvature-based redshift mechanism follows directly from general relativity's geometric principles:
	
	Starting with the modified metric from our gravitational potential $\Phi(r) = -\frac{GM}{r} + \kappa r$:
	
	\begin{equation}
		ds^2 = (1 - \frac{2GM}{r} + 2\kappa r)dt^2 - (1 + \frac{2GM}{r} - 2\kappa r)dr^2 - r^2d\Omega^2
	\end{equation}
	
	For a light path through this curved spacetime, the redshift is:
	
	\begin{equation}
		1 + z = \sqrt{\frac{g_{00}(r_1)}{g_{00}(r_2)}} = \sqrt{\frac{1 - \frac{2GM}{r_1} + 2\kappa r_1}{1 - \frac{2GM}{r_2} + 2\kappa r_2}}
	\end{equation}
	
	For large distances where $\kappa r$ dominates and considering the path integration:
	
	\begin{equation}
		1 + z \approx e^{\alpha d}
	\end{equation}
	
	Where $\alpha$ relates to the gradient of the metric along the path.
	
	The wavelength dependence emerges from making $\kappa$ frequency-dependent:
	
	\begin{equation}
		\kappa(\lambda) = \kappa_0(1 + \beta \ln(\lambda/\lambda_0))
	\end{equation}
	
	This yields our complete redshift formula:
	
	\begin{equation}
		1 + z = e^{\alpha d}(1 + \beta \ln(\lambda/\lambda_0))
	\end{equation}
	
	With $\alpha \approx 2.3\times10^{-18}$ m$^{-1}$ and $\beta \approx 0.008$, these values emerge naturally from the constants of the theory rather than being fitted parameters.
	
	\section{Supporting Research for Curvature-Based Redshift}
	
	Several established works explore concepts similar to curvature-based redshift and support aspects of this theoretical framework:
	
	\subsection{Alternative Redshift Mechanisms}
	
	\begin{enumerate}
		\item \textbf{Ari Brynjolfsson's "Redshift of photons penetrating a hot plasma"} (2004) - Discusses how photons traveling through plasma can experience redshift due to interaction with the medium rather than expansion.
		
		\item \textbf{Jean-Pierre Vigier's papers on "Non-Doppler redshift of some galactic objects"} - Explored alternative explanations for redshift based on light propagation through various media.
		
		\item \textbf{Halton Arp's "Quasars, Redshifts and Controversies"} (1987) - Presented observational evidence questioning the standard interpretation of redshift as purely expansion-based.
		
		\item \textbf{The "Tired Light" theories originally proposed by Fritz Zwicky} (1929) - While largely dismissed, these early theories suggested photons might lose energy during long-distance propagation.
		
		\item \textbf{Geoffrey Burbidge, Fred Hoyle, and Jayant Narlikar's "A Different Approach to Cosmology"} (2000) - Presented alternative cosmological models that don't rely solely on expansion to explain redshift.
	\end{enumerate}
	
	\subsection{Modified Gravity Approaches}
	
	\begin{enumerate}
		\item \textbf{Paul Marmet's work on "Non-Doppler Redshift of Some Galactic Objects"} - Explored alternative mechanisms for redshift based on interactions with the intergalactic medium.
		
		\item \textbf{Aspects of Moffat's Modified Gravity (MOG)} - While different in approach, also introduces modifications to gravitational potentials that affect spacetime curvature.
	\end{enumerate}
	
	\subsection{Wavelength-Dependent Effects}
	
	\begin{enumerate}
		\item \textbf{Dispersive light propagation in quantum vacuum} - Research in quantum electrodynamics showing how vacuum can have frequency-dependent effects on propagating light.
		
		\item \textbf{Robert Dicke's papers on the "Principle of Equivalence and the Weak Interactions"} - Early work exploring connections between gravity and electromagnetism that could affect light propagation.
	\end{enumerate}
	
	While none of these approaches are identical to our proposed curvature-based redshift with the $\kappa r$ term, they collectively demonstrate that alternatives to expansion-based redshift have been seriously considered in the scientific literature, lending credibility to our proposed extension of the standard model.
	
	\section{Conclusion}
	
	The standard interpretation of physics, based on time dilation and constant rest mass, can be extended to achieve complete compatibility with the alternative T0 model by incorporating a curvature-based interpretation of redshift. This approach centers on the key addition of the $\kappa r$ term to the gravitational potential, which gives rise to a cascade of effects that address major outstanding issues in physics.
	
	This extended model maintains the geometric foundation of general relativity while completely reinterpreting cosmic phenomena:
	\begin{itemize}
		\item Redshift becomes a consequence of light propagation through curved spacetime rather than expansion
		\item Galactic dynamics are explained by the same curvature that produces redshift
		\item Quantum phenomena connect to cosmological effects through the scalar field $\Theta$
	\end{itemize}
	
	The result is a static, eternal universe model that provides mathematically equivalent predictions to the T0 model while maintaining the standard interpretation of time dilation. This equivalence suggests that the current standard model is incomplete rather than incorrect.
	
	Future experiments targeting wavelength-dependent redshift and tests of the $\kappa r$ term's effects will be crucial in determining which framework—the extended standard model with its geometric interpretation or the T0 model with its variable mass approach—more accurately reflects the fundamental nature of reality. Until such definitive evidence emerges, both approaches deserve equal consideration in theoretical physics.
	
	\begin{thebibliography}{99}
		\bibitem{pascher_params_2025} Pascher, J. (2025). \href{https://github.com/jpascher/T0-Time-Mass-Duality/tree/main/2/pdf/English/ZeitMasseT0ParamsEn.pdf}{Time-Mass Duality Theory (T0 Model): Derivation of Parameters $\kappa$, $\alpha$, and $\beta$}. April 4, 2025.
		\bibitem{pascher_galaxies_2025} Pascher, J. (2025). \href{https://github.com/jpascher/T0-Time-Mass-Duality/tree/main/2/pdf/English/MassVarGalaxienEn.pdf}{Mass Variation in Galaxies: An Analysis in the T0 Model with Emergent Gravitation}. March 30, 2025.
		\bibitem{pascher_messdifferenzen_2025} Pascher, J. (2025). \href{https://github.com/jpascher/T0-Time-Mass-Duality/tree/main/2/pdf/English/MessdifferenzenT0StandardEn.pdf}{Compensatory and Additive Effects: An Analysis of Measurement Differences Between the T0 Model and the $\Lambda$CDM Standard Model}. April 2, 2025.
		\bibitem{pascher_lagrange_2025} Pascher, J. (2025). \href{https://github.com/jpascher/T0-Time-Mass-Duality/tree/main/2/pdf/English/MathZeitMasseLagrange.pdf}{From Time Dilation to Mass Variation: Mathematical Core Formulations of Time-Mass Duality Theory}. March 29, 2025.
		\bibitem{pascher_qft_2025} Pascher, J. (2025). \href{https://github.com/jpascher/T0-Time-Mass-Duality/tree/main/2/pdf/English/QFTIntrinsischesZeitT0En.pdf}{Quantum Field Theoretical Treatment of the Intrinsic Time Field in the T0 Model}. April 8, 2025.
		\bibitem{pascher_part1_2025} Pascher, J. (2025). \href{https://github.com/jpascher/T0-Time-Mass-Duality/tree/main/2/pdf/English/QMRelTimeMassPart1ZEn.pdf}{Bridging Quantum Mechanics and Relativity through Time-Mass Duality: A Unified Framework with Natural Units $\alpha = \beta = 1$ Part I: Theoretical Foundations}. April 7, 2025.
		\bibitem{pascher_part2_2025} Pascher, J. (2025). \href{https://github.com/jpascher/T0-Time-Mass-Duality/tree/main/2/pdf/English/QMRelTimeMassPart2ZEn.pdf}{Bridging Quantum Mechanics and Relativity through Time-Mass Duality: A Unified Framework with Natural Units $\alpha = \beta = 1$ Part II: Cosmological Implications and Experimental Validation}. April 7, 2025.
		\bibitem{brynjolfsson2004} Brynjolfsson, A. (2004). \textit{Redshift of photons penetrating a hot plasma}. arXiv:astro-ph/0401420.
		\bibitem{arp1987} Arp, H. (1987). \textit{Quasars, Redshifts and Controversies}. Interstellar Media, Berkeley.
		\bibitem{zwicky1929} Zwicky, F. (1929). \textit{On the Red Shift of Spectral Lines through Interstellar Space}. Proceedings of the National Academy of Sciences, 15(10), 773-779.
		\bibitem{burbidge2000} Burbidge, G., Hoyle, F., and Narlikar, J. V. (2000). \textit{A Different Approach to Cosmology: From a Static Universe through the Big Bang towards Reality}. Cambridge University Press.
		\bibitem{marmet1988} Marmet, P. (1988). \textit{A New Non-Doppler Redshift}. Physics Essays, 1(1), 24-32.
		\bibitem{moffat2006} Moffat, J. W. (2006). \textit{Scalar tensor vector gravity theory}. Journal of Cosmology and Astroparticle Physics, 2006(03), 004.
		\bibitem{dicke1957} Dicke, R. H. (1957). \textit{Principle of Equivalence and the Weak Interactions}. Reviews of Modern Physics, 29(3), 355.
		\bibitem{vigier1990} Vigier, J.-P. (1990). \textit{Evidence for nonzero mass photons associated with a vacuum-induced dissipative red-shift mechanism}. IEEE Transactions on Plasma Science, 18(1), 64-72.
	\end{thebibliography}
	
\end{document}