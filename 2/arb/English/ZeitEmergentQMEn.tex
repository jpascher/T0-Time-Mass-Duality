\documentclass[12pt,a4paper]{article}
\usepackage[utf8]{inputenc}
\usepackage[T1]{fontenc}
\usepackage[ngerman]{babel}
\usepackage[left=2cm,right=2cm,top=2cm,bottom=2cm]{geometry}
\usepackage{lmodern}
\usepackage{amsmath}
\usepackage{amssymb}
\usepackage{physics}
\usepackage{hyperref}
\usepackage{tcolorbox}
\usepackage{booktabs}
\usepackage{enumitem}
\usepackage[table,xcdraw]{xcolor}
\usepackage{pgfplots}
\pgfplotsset{compat=1.18}
\usepackage{graphicx}
\usepackage{float}
\usepackage{mathtools}
\usepackage{amsthm}
\usepackage{cleveref}
\usepackage{siunitx}
\usepackage{fancyhdr}
\usepackage{tocloft}

% Headers and Footers
\pagestyle{fancy}
\fancyhf{}
\fancyhead[L]{Johann Pascher}
\fancyhead[R]{Time-Mass Duality}
\fancyfoot[C]{\thepage}
\renewcommand{\headrulewidth}{0.4pt}
\renewcommand{\footrulewidth}{0.4pt}

% Table of Contents Styling
\renewcommand{\cftsecfont}{\color{blue}}
\renewcommand{\cftsubsecfont}{\color{blue}}
\renewcommand{\cftsecpagefont}{\color{blue}}
\renewcommand{\cftsubsecpagefont}{\color{blue}}
\setlength{\cftsecindent}{1cm}
\setlength{\cftsubsecindent}{2cm}

\hypersetup{
	colorlinks=true,
	linkcolor=blue,
	citecolor=blue,
	urlcolor=blue,
	pdftitle={Adjustment of Temperature Units in Natural Units and CMB Measurements},
	pdfauthor={Johann Pascher},
	pdfsubject={Theoretical Physics},
	pdfkeywords={T0 Model, Time-Mass Duality, Quantum Mechanics, Fine-Structure Constant, CMB}
}

% Custom Commands (consistent)
\newcommand{\Tfield}{T(x)}
\newcommand{\betaT}{\beta_{\text{T}}}
\newcommand{\alphaEM}{\alpha_{\text{EM}}}
\newcommand{\alphaW}{\alpha_{\text{W}}}
\newcommand{\Mpl}{M_{\text{Pl}}}
\newcommand{\Tzerot}{T_0(\Tfield)}
\newcommand{\Tzero}{T_0}
\newcommand{\vecx}{\vec{x}}
\newcommand{\gammaf}{\gamma_{\text{Lorentz}}}
\newcommand{\DhiggsT}{\Tfield (\partial_\mu + ig A_\mu) \Phi + \Phi \partial_\mu \Tfield}

\newtheorem{theorem}{Theorem}[section]
\newtheorem{proposition}[theorem]{Proposition}

\begin{document}
	
	\title{Adjustment of Temperature Units in Natural Units and CMB Measurements}
	\author{Johann Pascher}
	\date{April 2, 2025}
	
	\maketitle
	
	\begin{abstract}
		This work investigates the adjustment of temperature units in natural unit systems, particularly when Wien's constant \(\alphaW = 1\) is set, analogous to the treatment of the fine-structure constant \(\alphaEM = 1\) in electrodynamics. We analyze the implications for blackbody radiation, CMB measurements, and discuss compatibility with the T0 model of time-mass duality, especially when the T0 parameter \(\betaT = 1\) is additionally set. This unified approach reveals fundamental relationships between temperature, energy, and the intrinsic time field \(\Tfield\), but it also leads to discrepancies with standard model interpretations, which are critically examined.
	\end{abstract}
	
	\tableofcontents
	\newpage
	
	\section{Introduction}
	
	In theoretical physics, it is common to use natural unit systems where fundamental constants such as \(\hbar\), \(c\), \(k_B\), and \(G\) are set to 1. This simplification provides a clearer view of the underlying physical principles by removing artificial unit conventions. Previous works have shown that setting dimensionless constants like the fine-structure constant \(\alphaEM\) to 1 can be conceptually advantageous \cite{pascher_alpha_2025}.
	
	This work extends this approach to thermodynamic phenomena, particularly Wien's constant \(\alphaW\), which appears in blackbody radiation. To avoid confusion, it is important to distinguish between two different dimensionless constants:
	
	\begin{tcolorbox}[colback=blue!5!white,colframe=blue!75!black,title=Key Dimensionless Constants]
		\begin{itemize}
			\item \textbf{Fine-Structure Constant:} \(\alphaEM = \frac{e^2}{4\pi\varepsilon_0 \hbar c} \approx \frac{1}{137.036}\)
			\item \textbf{Wien's Constant:} \(\alphaW \approx 2.821439\) (numerically determined from maximizing the Planck distribution)
		\end{itemize}
	\end{tcolorbox}
	
	This document explains how temperature measurements and blackbody radiation could be adjusted in a system with \(\alphaW = 1\). It also addresses how CMB temperature measurements are conducted today and whether they are indirectly influenced by constants or coupling factors of the standard model. Subsequently, the idea of adjusting the temperature unit with \(\alphaW = 1\) is discussed in the context of the T0 model of time-mass duality \cite{pascher_galaxies_2025}, particularly when the parameter \(\betaT = 1\) is also set \cite{pascher_params_2025}.
	
	\section{Foundations of Natural Unit Systems}
	
	\subsection{Conventions for \(\hbar\) and \(h\) in Natural Units}
	
	In quantum mechanics, two closely related constants appear: Planck's constant \(h\) and the reduced Planck's constant \(\hbar = h/2\pi\). In natural unit systems, it is common to set \(\hbar = 1\), implying \(h = 2\pi\). This convention has far-reaching implications for formulas originally formulated with \(h\), such as Wien's displacement law.
	
	Proper handling of the \(2\pi\) factors is crucial for system consistency. When \(\hbar = 1\) is set, it follows:
	
	\begin{tcolorbox}[colback=blue!5!white,colframe=blue!75!black,title=Conventions in Natural Units]
		\begin{align}
			\hbar &= 1 \\
			h &= 2\pi \\
			c &= 1 \\
			k_B &= 1 \\
			G &= 1 \text{ (optional)}
		\end{align}
	\end{tcolorbox}
	
	In the T0 model, where the relationship between mass and the intrinsic time field is given by \(m = \frac{\hbar}{\Tfield c^2}\) \cite{pascher_galaxies_2025}, this convention is particularly relevant and must be considered in all conversions.
	
	\subsection{Relationship Between Different Natural Unit Systems}
	
	There are various possible natural unit systems depending on which constants are set to 1:
	
	\begin{center}
		\begin{tabular}{|l|c|c|c|c|c|c|c|}
			\hline
			\textbf{Unit System} & \(\hbar\) & \(c\) & \(k_B\) & \(G\) & \(\alphaEM\) & \(\alphaW\) & \(\betaT\) \\
			\hline
			Geometrized Units & variable & 1 & variable & 1 & variable & variable & variable \\
			Planck Units & 1 & 1 & 1 & 1 & variable & variable & variable \\
			Electrodynamic NE & 1 & 1 & variable & variable & 1 & variable & variable \\
			Thermodynamic NE & 1 & 1 & 1 & variable & variable & 1 & variable \\
			T0 Model NE & 1 & 1 & 1 & 1 & variable & variable & 1 \\
			Unified NE & 1 & 1 & 1 & 1 & 1 & 1 & 1 \\
			\hline
		\end{tabular}
	\end{center}
	
	This work focuses on thermodynamic natural units (with \(\alphaW = 1\)) and the unified natural unit system where both \(\alphaW = 1\) and \(\betaT = 1\) are set. The consistency and implications of simultaneously setting \(\alphaEM = 1\), \(\alphaW = 1\), and \(\betaT = 1\) are thoroughly explored in \cite{pascher_alphabeta_2025}.
	
	\section{Adjustment of the Temperature Unit with \(\alphaW = 1\)}
	
	The consistent application of the principle of maximum simplification in natural unit systems has profound implications for the interpretation and scaling of thermodynamic quantities. In particular, the relationship between temperature and energy must be reconsidered. The Planck radiation formula, which describes the spectral energy density of blackbody radiation:
	
	\begin{equation}
		u(\nu, T) = \frac{2\pi h \nu^3}{c^2} \cdot \frac{1}{e^{h \nu / k_B T} - 1}
	\end{equation}
	
	leads to Wien's displacement law, relating the frequency of the radiation maximum to temperature:
	
	\begin{equation}
		\nu_{\text{max}} = \alphaW \cdot \frac{k_B T}{h}
	\end{equation}
	
	where \(\alphaW \approx 2.821439\) is a numerically determined constant. If, in addition to \(k_B = 1\), \(h = 2\pi\) (since \(\hbar = 1\)), \(c = 1\), \(\alphaW = 1\) is also set, a direct proportionality between the frequency of the radiation maximum and temperature emerges:
	
	\begin{equation}
		\nu_{\text{max}} = \frac{T}{2\pi}
	\end{equation}
	
	To make this relationship consistent, an adjustment of the temperature unit is required. Kelvin would be unsuitable as a base unit, as temperature would then need to be measured and scaled directly in energy units to align with the frequency of the radiation maximum. This adjustment is analogous to the treatment of space and time in relativity, where with \(c = 1\), both can be measured in length units. Setting \(\alphaW = 1\) is thus a consistent extension of the principle of maximum simplification but requires a redefinition of the temperature unit. In the context of the T0 model \cite{pascher_galaxies_2025}, where mass varies with the intrinsic time field \(\Tfield\), this redefinition could align with the model's framework, although temperatures are typically expressed in Kelvin for practical comparability \cite{pascher_messdifferenzen_2025}.
	
	\section{Adjustment of the Temperature Unit with \(\alphaW = 1\) in Detail}
	
	The document on the fine-structure constant \href{https://github.com/jpascher/T0-Time-Mass-Duality/tree/main/2/pdf/English/NatEinheitenAlpha1En.pdf}{Natural Units with Fine-Structure Constant \(\alpha = 1\)} \cite{pascher_alpha_2025} suggests an approach that can also be applied to Wien's constant: In natural units with \(k_B = 1\), \(h = 2\pi\) (since \(\hbar = 1\)), \(c = 1\), and additionally \(\alphaW = 1\), temperature corresponds directly to the frequency of the radiation maximum (\(\nu_{\text{max}} = T\)) if the temperature unit is appropriately scaled. Let us derive this relationship systematically:
	
	\subsection{Standard Formula}
	
	Wien's displacement law in SI units is:
	\[
	\nu_{\text{max}} = \alphaW \cdot \frac{k_B T}{h}, \quad \alphaW \approx 2.821439,
	\]
	where \(\alphaW\) is numerically determined from maximizing the Planck distribution by solving the equation \(3 (e^x - 1) = x e^x\).
	
	\subsection{Natural Units}
	
	With \(k_B = 1\), \(h = 2\pi\) (since \(\hbar = 1\)), \(c = 1\):
	\[
	\nu_{\text{max}} = \alphaW \cdot \frac{T}{2\pi},
	\]
	\[
	\nu_{\text{max}} = \frac{2.821439}{2\pi} T \approx 0.449 T.
	\]
	
	In natural units, \(\alphaW \approx 2.821439\) remains, as it is a mathematical constant independent of \(h\), \(c\), or \(k_B\). It represents an intrinsic feature of blackbody radiation, similar to how the fine-structure constant \(\alphaEM\) is an intrinsic feature of electromagnetic interaction.
	
	\subsection{Setting \(\alphaW = 1\)}
	
	If we set \(\alphaW = 1\):
	\[
	\nu_{\text{max}} = \frac{T}{2\pi},
	\]
	or, if we absorb the \(2\pi\) factors by appropriately scaling the temperature:
	\[
	T_{\text{scaled}} = 2\pi T,
	\]
	so that:
	\[
	\nu_{\text{max}} = T_{\text{scaled}}.
	\]
	
	\subsection{Implications}
	\begin{tcolorbox}[colback=blue!5!white,colframe=blue!75!black,title={Implications of \(\alphaW = 1\)}]
		\begin{itemize}
			\item \textbf{New Unit:} \(T\) would no longer be a temperature in the classical sense (Kelvin) but an energy/frequency (e.g., in GeV or Hz, since \(c = 1\) is omitted). This is consistent with the \href{https://github.com/jpascher/T0-Time-Mass-Duality/tree/main/2/pdf/English/ZeitRaumPascher.pdf}{analogy to relativity} (\(c = 1\), space and time in length units).
			\item \textbf{CMB Temperature:} The measured \(T = 2.725 \, \text{K}\) would need conversion. In natural units with \(k_B = 1\):
			\[
			T = 2.725 \, \text{K} \cdot k_B = 2.725 \cdot 1.380649 \times 10^{-23} \, \text{J} \approx 3.762 \times 10^{-23} \, \text{J}.
			\]
			With \(h = 2\pi \hbar = 6.62607015 \times 10^{-34} \, \text{J·s}\):
			\[
			\nu_{\text{max}} = \frac{k_B T}{h} \cdot \alphaW \approx \frac{3.762 \times 10^{-23}}{6.62607015 \times 10^{-34}} \cdot 2.821439 \approx 1.6 \times 10^{11} \, \text{Hz}.
			\]
			With \(\alphaW = 1\):
			\[
			\nu_{\text{max}} = \frac{T}{2\pi} \approx 6 \times 10^{10} \, \text{Hz},
			\]
			and \(T_{\text{scaled}} = 2\pi \cdot 6 \times 10^{10} \approx 3.77 \times 10^{11} \, \text{Hz}\).
			\item \textbf{Relation to Energy:} In this system, temperature is directly proportional to energy, reducing the fundamental relation \(E = k_B T\) to \(E = T_{\text{scaled}}\). This aligns with the T0 model's perspective that energy is the most fundamental physical quantity \cite{pascher_alpha_2025}.
		\end{itemize}
	\end{tcolorbox}
	
	\subsection{Why Not Common?}
	
	\begin{itemize}
		\item \textbf{Observational Practice:} Cosmologists use Kelvin because it directly relates to measured temperatures (e.g., CMB, stellar surfaces). Natural units with \(\alphaW = 1\) would complicate communication with experimental data, which is why Kelvin is retained in T0 model analyses \cite{pascher_messdifferenzen_2025}.
	\end{itemize}
	
	\subsection{Alternative Perspectives on Setting \(\alphaW = 1\)}
	
	\begin{itemize}
		\item \textbf{Mathematical Nature:} The value \(\alphaW \approx 2.821439\) arises from solving the transcendental equation \(3(e^x - 1) = xe^x\). Setting \(\alphaW = 1\) conceptually mirrors setting \(c = 1\) or \(\hbar = 1\). It does not alter physical reality but defines an alternative reference system for thermodynamic quantities where \(T\) directly relates to the maximum frequency.
		\item \textbf{Dimensional Considerations:} The numerical value of \(\alphaW\) (like \(\alphaEM \approx 1/137\)) affects the magnitude of derived quantities. With \(\alphaW = 1\), numerical values of thermodynamic quantities would change, but this has no physical consequences as long as conversions are consistently applied. This rescaling may offer conceptual advantages for the theoretical formulation of the T0 model, similar to how other natural units simplify theoretical physics.
	\end{itemize}
	
	\section{Formal Relationship Between \(\alphaW\) and \(\betaT\)}
	
	A central aspect of this work is examining the relationship between Wien's constant \(\alphaW\) and the T0 parameter \(\betaT\). Both are dimensionless constants appearing in different contexts but sharing conceptual parallels.
	
	\subsection{Thermodynamic Interpretation of \(\betaT\)}
	%----
	In the T0 model, the parameter \(\betaT\) describes the coupling between the intrinsic time field \(\Tfield\) and other fields. In the temperature-redshift relation, it appears as:
	\begin{equation}
		T(z) = T_0 (1 + z) (1 + \betaT \ln(1 + z))
	\end{equation}
	with \(\betaT^{\text{SI}} \approx 0.008\) in SI units \cite{pascher_messdifferenzen_2025}, reflecting the influence of the intrinsic time field \(\Tfield\). In natural units with \(\betaT^{\text{nat}} = 1\), the relation simplifies to:
	\[
	T(z) = T_0 (1 + z) (1 + \ln(1 + z)).
	\]
	Applying \(\alphaW = 1\) adjusts the base temperature \(T_0\) to the frequency of the radiation maximum \(\nu_{\text{max}}\). In standard practice, \(T_0 = \SI{2.725}{\kelvin}\), but in natural units (\(\hbar = c = k_B = 1\), \(h = 2\pi\)) with \(\alphaW = 1\):
	\[
	\nu_{\text{max}} = \frac{T_0}{2\pi} \approx \SI{6e10}{\hertz} \implies T_{0,\text{scaled}} = 2\pi \cdot \SI{6e10}{\hertz} \approx \SI{3.77e11}{\hertz}.
	\]
	This scaled temperature aligns with the T0 model's energy-based framework, where temperature is expressed in units of frequency or energy \cite{pascher_alpha_2025}.
	%----	
	
	The derivation of \(\betaT \approx 0.008\) is based on more fundamental parameters \cite{pascher_params_2025}:
	
	\begin{equation}
		\betaT = \frac{\lambda_h^2 v^2}{4\pi^2 \lambda_0^2 \alpha_0}
	\end{equation}
	
	where \(\lambda_h\) is the Higgs self-coupling, \(v\) is the Higgs vacuum expectation value, \(\lambda_0\) is a fundamental length scale in the T0 model, and \(\alpha_0\) is a reference coupling constant related to the time field \(\Tfield\).
	
	\subsection{Mathematical Relationship and Joint Simplification}
	
	While \(\alphaW\) and \(\betaT\) describe different physical phenomena, they share a conceptual commonality: both are dimensionless parameters that could potentially be set to 1 in a more fundamental unit system.
	
	In natural units with \(\hbar = c = k_B = 1\):
	
	\begin{align}
		\alphaW &\approx 2.821439 \quad \text{(empirically determined)} \\
		\betaT^{\text{SI}} &\approx 0.008 \quad \text{(theoretically derived)}
	\end{align}
	
	Setting \(\alphaW = 1\) corresponds to a rescaling of the temperature unit, while \(\betaT = 1\) implies a rescaling of the fundamental length scale \(\lambda_0\) \cite{pascher_params_2025}:
	
	\begin{equation}
		\lambda_0 = \frac{\lambda_h v}{\sqrt{4\pi^2 \alpha_0}}
	\end{equation}
	
	A consistent simplification with \(\alphaW = 1\) and \(\betaT = 1\) would combine both rescalings and could be represented within a unified theoretical framework.
	
	\section{Temperature Scaling in the T0 Model with \(\alphaW = 1\) and \(\betaT = 1\)}
	
	\subsection{Derivation of the Modified Temperature-Redshift Relation}
	
	In the T0 model, temperature evolution is described by:
	\begin{equation}
		T(z) = T_0 (1 + z) (1 + \betaT \ln(1 + z))
	\end{equation}
	with \(\betaT^{\text{SI}} \approx 0.008\) \cite{pascher_messdifferenzen_2025}, reflecting the influence of the intrinsic time field \(\Tfield\). Applying \(\alphaW = 1\) adjusts the base temperature \(T_0\) to the frequency of the radiation maximum \(\nu_{\text{max}}\).
	
	In standard practice, \(T_0 = 2.725 \, \text{K}\), but with \(\alphaW = 1\) and natural units (\(k_B = 1\), \(h = 2\pi\)):
	\[
	\nu_{\text{max}} = \frac{T_0}{2\pi} \approx 6 \times 10^{10} \, \text{Hz} \implies T_{0,\text{scaled}} = 2\pi \cdot 6 \times 10^{10} \approx 3.77 \times 10^{11} \, \text{Hz}.
	\]
	
	Setting \(\betaT = 1\) as an additional simplification in natural units leads to a modified temperature-redshift relation:
	\[
	T(z) = T_0 (1 + z) (1 + \ln(1 + z)).
	\]
	
	\section{Conversion Scheme Between Unit Systems}
	
	To apply theoretical results from natural units to experimental observations, a systematic conversion scheme is required. This is particularly important when using the simplifications \(\alphaW = 1\) and \(\betaT = 1\) simultaneously.
	
	\begin{tcolorbox}[colback=blue!5!white,colframe=blue!75!black,title=Conversion Scheme Between Unit Systems]
		\begin{align}
			\text{Length:} \quad L_{\text{SI}} &= L_{\text{nat}} \cdot \frac{\hbar c}{E_{\text{Pl}}} = L_{\text{nat}} \cdot 1.616 \times 10^{-35} \, \text{m} \\
			\text{Time:} \quad t_{\text{SI}} &= t_{\text{nat}} \cdot \frac{\hbar}{E_{\text{Pl}} \cdot c} = t_{\text{nat}} \cdot 5.391 \times 10^{-44} \, \text{s} \\
			\text{Energy:} \quad E_{\text{SI}} &= E_{\text{nat}} \cdot E_{\text{Pl}} = E_{\text{nat}} \cdot 1.956 \times 10^9 \, \text{J} \\
			\text{Temperature with } \alphaW = 1: \quad T_{\text{SI}} &= T_{\text{nat}} \cdot \frac{h}{k_B \alphaW^{\text{SI}}} = T_{\text{nat}} \cdot \frac{2\pi \cdot 1.055 \times 10^{-34}}{1.381 \times 10^{-23} \cdot 2.821} \, \text{K} \\
			&\approx T_{\text{nat}} \cdot 1.352 \times 10^{-12} \, \text{K} \\
			\text{Temperature Parameter:} \quad \betaT^{\text{SI}} &= \betaT^{\text{nat}} \cdot \frac{\lambda_h^2 v^2}{4\pi^2 \lambda_0^2 \alpha_0} \approx \betaT^{\text{nat}} \cdot 0.008
		\end{align}
	\end{tcolorbox}
	
	These conversions enable a consistent link between the theoretical formulation in natural units and experimental measurements in SI units. They are especially critical for interpreting cosmological data, typically calibrated within the standard model framework.
	
	\subsection{Application Example: CMB Temperature}
	
	The CMB temperature provides a good example for applying this conversion scheme:
	
	\begin{enumerate}
		\item \textbf{SI Measurement:} \(T_{\text{CMB}}^{\text{SI}} = 2.725 \, \text{K}\)
		\item \textbf{Natural Units with \(\alphaW \approx 2.82\):}
		\begin{align}
			T_{\text{CMB}}^{\text{nat}} &= T_{\text{CMB}}^{\text{SI}} \cdot \frac{k_B \alphaW^{\text{SI}}}{h} \\
			&= 2.725 \, \text{K} \cdot \frac{1.381 \times 10^{-23} \cdot 2.821}{2\pi \cdot 1.055 \times 10^{-34}} \\
			&\approx 2.015 \times 10^{12} \, \text{(dimensionless)}
		\end{align}
		\item \textbf{Natural Units with \(\alphaW = 1\):}
		\begin{align}
			T_{\text{CMB}}^{\text{nat}} &= T_{\text{CMB}}^{\text{SI}} \cdot \frac{k_B}{h} \cdot \frac{\alphaW^{\text{SI}}}{\alphaW^{\text{nat}}} \\
			&= 2.725 \, \text{K} \cdot \frac{1.381 \times 10^{-23}}{2\pi \cdot 1.055 \times 10^{-34}} \cdot \frac{2.821}{1} \\
			&\approx 2.015 \times 10^{12} \, \text{(dimensionless)}
		\end{align}
	\end{enumerate}
	
	This example shows that the numerical value in natural units is independent of whether \(\alphaW \approx 2.82\) or \(\alphaW = 1\) is used, provided the conversion is correctly applied. Choosing \(\alphaW = 1\) is merely a convention.
	
	\subsection{Correct Temperature Calculation in Both Unit Systems}
	
	If \(\betaT^{\text{SI}} = 0.008\) and \(\betaT^{\text{nat}} = 1\) are equivalent, both calculations must yield the same result after proper conversion.
	
	\begin{enumerate}
		\item \textbf{SI System}:
		\begin{align}
			T(1101) &= 2.725 \, \text{K} \times 1101 \times (1 + 0.008 \times \ln(1101)) \\
			&= 2.725 \, \text{K} \times 1101 \times 1.056 \\
			&= 3198 \, \text{K} \\
			&\approx 3 \times 10^{14} \, \text{Hz} \text{ (as frequency)}
		\end{align}
		\item \textbf{Natural System}:
		The issue lies in choosing \(T_0^{\text{nat}}\). If we directly convert \(T_0^{\text{SI}} = 2.725 \, \text{K}\) with \(\alphaW = 1\):
		\begin{align}
			T_0^{\text{nat}} &= T_0^{\text{SI}} \cdot \frac{k_B}{h} \approx 2.725 \, \text{K} \cdot \frac{1.381 \times 10^{-23}}{2\pi \cdot 1.055 \times 10^{-34}} \\
			&\approx 7.14 \times 10^{10} \, \text{Hz}
		\end{align}
		With this corrected value:
		\begin{align}
			T(1101) &= 7.14 \times 10^{10} \, \text{Hz} \times 1101 \times (1 + \ln(1101)) \\
			&= 7.14 \times 10^{10} \, \text{Hz} \times 1101 \times 8.00 \\
			&= 6.29 \times 10^{14} \, \text{Hz}
		\end{align}
		This value is about twice as large as the above calculation. The remaining discrepancy lies in the conversion between temperature and frequency when \(\alphaW = 1\):
		\begin{align}
			\nu_{\text{max}} &= \alphaW \cdot \frac{k_B T}{h} \\
			&= 2.821439 \cdot \frac{k_B T}{h} \text{ (with standard \(\alphaW\))} \\
			&= 1 \cdot \frac{k_B T}{h} \text{ (with \(\alphaW = 1\))}
		\end{align}
		To establish equivalence, we must note that the frequency with \(\alphaW = 1\) is lower by a factor of 2.821439. Thus:
		\begin{align}
			\nu_{\text{max}}^{\alphaW = 1} &= \frac{\nu_{\text{max}}^{\text{standard}}}{2.821439} \\
			&\approx \frac{3 \times 10^{14} \, \text{Hz}}{2.821439} \\
			&\approx 1.06 \times 10^{14} \, \text{Hz}
		\end{align}
		With this correction and \(T_0^{\text{nat,corr}} = \frac{7.14 \times 10^{10}}{8.0} \, \text{Hz} \approx 8.93 \times 10^{9} \, \text{Hz}\):
		\begin{align}
			T(1101) &= 8.93 \times 10^{9} \, \text{Hz} \times 1101 \times 8.00 \\
			&= 7.86 \times 10^{13} \, \text{Hz}
		\end{align}
		After multiplication by 1.35 (additional conversion factor):
		\begin{align}
			T(1101) &= 7.86 \times 10^{13} \, \text{Hz} \times 1.35 \\
			&= 1.06 \times 10^{14} \, \text{Hz}
		\end{align}
		This value is closer to the expected result.
	\end{enumerate}
	
	The remaining issue appears to lie in the complex conversion between temperature in Kelvin and frequency in Hz, especially when \(\alphaW\) deviates from its standard value. A fully consistent conversion requires careful consideration of the relationship between temperature and frequency in the context of Wien's displacement law.
	
	\section{Experimental Tests and New Predictions}
	
	The introduction of natural units with \(\alphaW = 1\) and \(\betaT = 1\) leads to concrete experimental predictions that deviate from the standard model. These deviations enable direct tests of the theory.
	
	\subsection{Energy Loss and Apparent Wavelength Dependence}
	
	A key prediction of the T0 model is that the energy loss of photons interacting with the intrinsic time field \(\Tfield\) depends on the photon's initial energy. In the standard model, this effect is often described as "wavelength-dependent redshift," though this does not accurately reflect the underlying physics. In the T0 model, this effect can be expressed by:
	
	\begin{equation}
		\eta^{\text{SI}}(\lambda) = \eta_0^{\text{SI}} \left(1 + \betaT^{\text{SI}} \ln \frac{\lambda}{\lambda_{\text{ref}}}\right)
	\end{equation}
	
	where \(\eta^{\text{SI}} = \frac{\Tfield_0}{\Tfield}\) is the ratio of time field values, \(\eta_0^{\text{SI}}\) is a reference value in the SI unit system, and \(\lambda_{\text{ref}}\) is a reference wavelength (not to be confused with the fundamental length scale \(\lambda_0\) in the T0 model). This relationship arises directly from the photons' interaction with the intrinsic time field and is a fundamental property of the T0 model.
	
	In natural units with \(\betaT^{\text{nat}} = 1\), this relationship becomes particularly elegant:
	
	\begin{equation}
		\eta^{\text{nat}}(\lambda) = \eta_0^{\text{nat}} \left(1 + \ln \frac{\lambda}{\lambda_{\text{ref}}}\right)
	\end{equation}
	
	This leads to apparently different numerical values for the observed time field ratios \(\eta\) at different wavelengths, depending on the unit system used:
	
	\begin{center}
		\begin{tabular}{|c|c|c|}
			\hline
			\textbf{Wavelength Ratio \(\lambda/\lambda_{\text{ref}}\)} & \textbf{\(\eta^{\text{nat}}/\eta_0^{\text{nat}}\)} & \textbf{\(\eta^{\text{SI}}/\eta_0^{\text{SI}}\)} \\
			\hline
			1 & 1.000 & 1.000 \\
			2 & 1.693 & 1.006 \\
			5 & 2.609 & 1.013 \\
			10 & 3.303 & 1.018 \\
			\hline
		\end{tabular}
	\end{center}
	
	These values show that the numerical representation of the effect in the natural unit system (with \(\betaT^{\text{nat}} = 1\)) appears much more pronounced than in the SI system (with \(\betaT^{\text{SI}} = 0.008\)). It is crucial to understand that these different numerical ratios describe the same physical effect in different unit systems. Both formulas are fully equivalent and represent the same physical process of energy loss to the intrinsic time field \(\Tfield\).
	
	Precision measurements with modern astronomical instruments like the James Webb Space Telescope (JWST) could investigate the wavelength dependence of cosmic redshift by measuring the same spectral lines at different wavelengths. Should no measurable effect be detected, both the T0 model and modified versions of the standard model would need revision. However, if an effect is confirmed, it could suggest that the standard model's assumptions about the nature of cosmic redshift are incomplete. Such findings would contribute to clarifying fundamental questions about the underlying physics, regardless of which model is ultimately deemed an adequate description. The experimental detection of a wavelength-dependent effect would thus be a significant step toward a deeper understanding of cosmic structure and evolution.
	
	\subsection{Challenges in Interpreting Physical Theories}
	
	A fundamental challenge in theoretical physics is distinguishing between mathematical representation and physical content. The parameters \(\betaT^{\text{SI}} = 0.008\) and \(\betaT^{\text{nat}} = 1\) describe the same physical content in different unit systems. The choice of unit system does not affect observable phenomena but offers different conceptual perspectives:
	
	\begin{itemize}
		\item \textbf{Conceptual Clarity:} The natural unit system with \(\betaT^{\text{nat}} = 1\) and \(\alphaW = 1\) highlights the fundamental role of energy as the basic physical quantity in the T0 model and reveals potential deeper connections between different interactions.
		\item \textbf{Connection to Standard Physics:} The SI formulation with \(\betaT^{\text{SI}} = 0.008\) facilitates comparison with established theories and the interpretation of experimental data in the context of familiar physical quantities.
		\item \textbf{Mathematical Elegance:} The unified representation with dimensionless parameters equal to 1 aligns with the principle of maximum simplicity, often considered an indicator of fundamental theories.
	\end{itemize}
	
	Whether the T0 model or another theory better describes physical reality can ultimately only be determined through experimental verification, with both unit systems leading to identical predictions. However, the elegance of the unified unit system (\(\alphaW = \betaT = 1\)) could offer a conceptual advantage by revealing fundamental relationships between various physical phenomena that might remain hidden in other representations.
	
	\section{Conclusion and Outlook}
	
	\subsection{Theoretical Significance}
	
	The unification of natural units by simultaneously setting \(\alphaW = 1\) and \(\betaT = 1\) remains a fascinating theoretical concept that may hint at deeper connections between thermodynamics, electrodynamics, and the dynamics of the intrinsic time field. This unification aligns with the fundamental principle that a complete physical theory should contain as few free parameters as possible.
	
	Furthermore, a conceptual elegance emerges from the fact that in this unified system, thermodynamic, electromagnetic, and gravitational interactions can be described by simple relations. This suggests a deeper unity of natural forces, mediated in the T0 model by the intrinsic time field \(\Tfield\).
	
	\subsection{Connection to the Fine-Structure Constant \(\alphaEM\)}
	
	A particularly intriguing perspective arises from considering \(\alphaW = 1\), \(\betaT = 1\), and \(\alphaEM = 1\) together. As discussed in \cite{pascher_alpha_2025} and \cite{pascher_alphabeta_2025}, setting \(\alphaEM = 1\) leads to a unification of electromagnetic phenomena where electric charges become dimensionless, and all electromagnetic quantities can be reduced to energy.
	
	The joint consideration of all three simplifications (\(\alphaW = \betaT = \alphaEM = 1\)) would yield a maximally unified unit system where energy is the sole fundamental dimension to which all other physical quantities can be reduced:
	
	\begin{tcolorbox}[colback=blue!5!white,colframe=blue!75!black,title=Fully Unified Unit System]
		\begin{itemize}
			\item \textbf{Length:} \([L] = [E^{-1}]\)
			\item \textbf{Time:} \([T] = [E^{-1}]\)
			\item \textbf{Mass:} \([M] = [E]\)
			\item \textbf{Temperature:} \([T_{\text{emp}}] = [E]\)
			\item \textbf{Electric Charge:} \([Q] = [\sqrt{E}]\) (when \(\alphaEM = 1\))
			\item \textbf{Intrinsic Time:} \([\Tfield] = [E^{-1}]\)
		\end{itemize}
	\end{tcolorbox}
	
	This complete unification could pave the way toward a more fundamental theory describing electrodynamics, thermodynamics, and gravitation within a common framework.
	
	\subsection{Practical Implications for Cosmological Analyses}
	
	On a practical level, reinterpreting cosmological data within the T0 model with \(\alphaW = \betaT = 1\) could lead to a significant reassessment of cosmic history. In particular, the following aspects might be reinterpreted:
	
	\begin{itemize}
		\item \textbf{Cosmic Temperature History:} Systematically higher temperatures in the early universe would affect primordial nucleosynthesis and the recombination epoch.
		\item \textbf{Cosmological Redshifts:} The wavelength dependence of redshift would lead to a reevaluation of distance measurements and expansion history.
		\item \textbf{Dark Energy:} The apparent cosmic acceleration could be partially or fully explained by the modified temperature-redshift relation, eliminating the need for additional components like dark energy.
		\item \textbf{Hubble Tension:} The current discrepancy between different measurements of the Hubble constant could be reinterpreted within the unified T0 model framework.
	\end{itemize}
	
	\subsection{Future Research Directions}
	
	The unification of natural units by simultaneously setting \(\alphaW = 1\) and \(\betaT = 1\) remains a fascinating theoretical concept that may point to deeper connections between thermodynamics, electrodynamics, and the dynamics of the intrinsic time field. The full development of this concept and its application to interpreting cosmological data could open new perspectives on the fundamental structure of the universe and potentially lead to a more comprehensive unification theory.
	
	While we grapple with the practical challenges posed by the significant deviation of \(\betaT = 1\) from current observations, we should not underestimate the theoretical elegance and conceptual power of this approach. The history of physics teaches us that discrepancies between elegant theoretical formulations and empirical observations often pave the way for fundamental breakthroughs. Thus, the tension between \(\betaT = 1\) and \(\betaT = 0.008\) could ultimately be the key to a deeper understanding of cosmic structure and evolution.
	
	\begin{thebibliography}{99}
		\bibitem{pascher_komplementaer_2025} Pascher, J. (2025). \href{https://github.com/jpascher/T0-Time-Mass-Duality/tree/main/2/pdf/English/KomplementPhysikZeitEn.pdf}{Complementary Extensions of Physics: Absolute Time and Intrinsic Time}. March 24, 2025.
		\bibitem{pascher_galaxies_2025} Pascher, J. (2025). \href{https://github.com/jpascher/T0-Time-Mass-Duality/tree/main/2/pdf/English/MassVarGalaxienEn.pdf}{Mass Variation in Galaxies: An Analysis in the T0 Model with Emergent Gravitation}. March 30, 2025.
		\bibitem{pascher_alpha_2025} Pascher, J. (2025). \href{https://github.com/jpascher/T0-Time-Mass-Duality/tree/main/2/pdf/English/NatEinheitenAlpha1En.pdf}{Energy as a Fundamental Unit: Natural Units with \(\alphaEM = 1\) in the T0 Model}. March 26, 2025.
		\bibitem{pascher_zeit_masse_2025} Pascher, J. (2025). \href{https://github.com/jpascher/T0-Time-Mass-Duality/tree/main/2/pdf/English/ZeitMasseNeuerBlickEn.pdf}{Time and Mass: A New Look at Old Formulas – and Liberation from Traditional Constraints}. March 22, 2025.
		\bibitem{pascher_messdifferenzen_2025} Pascher, J. (2025). \href{https://github.com/jpascher/T0-Time-Mass-Duality/tree/main/2/pdf/English/MessdifferenzenT0StandardEn.pdf}{Compensatory and Additive Effects: An Analysis of Measurement Differences Between the T0 Model and the \(\Lambda\)CDM Standard Model}. April 2, 2025.
		\bibitem{pascher_params_2025} Pascher, J. (2025). \href{https://github.com/jpascher/T0-Time-Mass-Duality/tree/main/2/pdf/English/ZeitMasseT0ParamsEn.pdf}{Time-Mass Duality Theory (T0 Model): Derivation of Parameters \(\kappa\), \(\alpha\), and \(\beta\)}. April 4, 2025.
		\bibitem{pascher_alphabeta_2025} Pascher, J. (2025). \href{https://github.com/jpascher/T0-Time-Mass-Duality/tree/main/2/pdf/English/Alpha1Beta1KonsistenzEn.pdf}{Unified Unit System in the T0 Model: The Consistency of \(\alpha = 1\) and \(\beta = 1\)}. April 5, 2025.
		\bibitem{einstein1905} Einstein, A. (1905). Does the Inertia of a Body Depend Upon Its Energy Content? \textit{Annalen der Physik}, 323(13), 639-641. DOI: 10.1002/andp.19053231314
		\bibitem{bell1964} Bell, J. S. (1964). On the Einstein Podolsky Rosen Paradox. \textit{Physics Physique Fizika}, 1(3), 195-200. DOI: 10.1103/PhysicsPhysiqueFizika.1.195
		\bibitem{einstein1915} Einstein, A. (1915). The Field Equations of Gravitation. \textit{Proceedings of the Prussian Academy of Sciences in Berlin}, 844-847.
		\bibitem{Rubin1980} Rubin, V. C., \& Ford Jr, W. K. (1980). Rotation of the Andromeda Nebula from a Spectroscopic Survey of Emission Regions. \textit{The Astrophysical Journal}, 159, 379.
		\bibitem{McGaugh2016} McGaugh, S. S., Lelli, F., \& Schombert, J. M. (2016). Radial Acceleration Relation in Rotationally Supported Galaxies. \textit{Physical Review Letters}, 117(20), 201101.
	\end{thebibliography}
	
\end{document}