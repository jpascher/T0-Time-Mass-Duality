\documentclass[12pt,a4paper]{article}
\usepackage[utf8]{inputenc}
\usepackage[T1]{fontenc}
\usepackage[ngerman]{babel}
\usepackage{lmodern}
\usepackage{amsmath}
\usepackage{amssymb}
\usepackage{physics}
\usepackage{hyperref}
\usepackage{tcolorbox}
\usepackage{booktabs}
\usepackage{enumitem}
\usepackage[table,xcdraw]{xcolor}
\usepackage[left=2cm,right=2cm,top=2cm,bottom=2cm]{geometry}
\usepackage{pgfplots}
\pgfplotsset{compat=1.18}
\usepackage{graphicx}
\usepackage{float}
\usepackage{fancyhdr}
\usepackage{siunitx}

% Custom Commands
\newcommand{\Tfield}{T(x)}
\newcommand{\alphaEM}{\alpha_{\text{EM}}}
\newcommand{\betaT}{\beta_{\text{T}}}
\newcommand{\Mpl}{M_{\text{Pl}}}
\newcommand{\Tzerot}{T_0(\Tfield)}
\newcommand{\e}{\mathrm{e}}

% Header and Footer Configuration
\pagestyle{fancy}
\fancyhf{}
\fancyhead[L]{Johann Pascher}
\fancyhead[R]{Unified Unit System in the T0 Model}
\fancyfoot[C]{\thepage}
\renewcommand{\headrulewidth}{0.4pt}
\renewcommand{\footrulewidth}{0.4pt}

\hypersetup{
	colorlinks=true,
	linkcolor=blue,
	citecolor=blue,
	urlcolor=blue,
	pdftitle={Unified Unit System in the T0 Model},
	pdfauthor={Johann Pascher},
	pdfsubject={Theoretical Physics},
	pdfkeywords={T0 Model, natural units, fine-structure constant, time-mass duality}
}

\title{Unified Unit System in the T0 Model: \\The Consistency of \(\alpha = 1\) and \(\beta = 1\)}
\author{Johann Pascher}
\date{\today}

\begin{document}
	
	\maketitle
	
	\begin{abstract}
		This work examines the theoretical consistency and implications of a unified natural unit system in which both the fine-structure constant \(\alpha = 1\) and the T0 model parameter \(\beta = 1\) are set. Through detailed mathematical analyses, dimensional considerations, and an investigation of fundamental interactions, it is demonstrated that this dual simplification leads to a coherent and elegant theoretical framework. The characteristic length scales of the model are identified as dimensionless ratios to the Planck length, revealing a deeper connection between electrodynamics, time-mass duality, and quantum gravitation. This unified perspective offers new insights for the T0 model and could pave the way toward a more fundamental unification theory.
	\end{abstract}
	
	\tableofcontents
	\newpage
	
	\section{Introduction}
	
	The simplification of physical theories through the choice of appropriate unit systems has a long tradition in theoretical physics. In special relativity, the speed of light is set to \(c = 1\), in quantum mechanics the Planck constant is set to \(\hbar = 1\), and in quantum gravitation the gravitational constant is set to \(G = 1\). These simplifications are not merely mathematical but reveal fundamental structures of physics.
	
	In previous works \cite{pascher_alpha_2025, pascher_beta_2025}, two additional simplifications were independently explored: setting the fine-structure constant \(\alpha = 1\) and setting the T0 model parameter \(\beta = 1\). This work takes a step further and systematically examines the consistency and implications of a unified unit system in which both parameters are simultaneously set to 1.
	
	Such a unification is nontrivial, as \(\alpha\) characterizes the electromagnetic interaction, while \(\beta\) relates to the coupling between the intrinsic time field \(\Tfield\) and other fields in the T0 model. The compatibility of both simplifications could provide deeper insights into the fundamental structure of physics and potentially point toward a unification of electrodynamics and gravitation within the T0 model framework.
	
	\section{Foundations of the Unified Unit System}
	
	\subsection{The Fine-Structure Constant \(\alpha\) and Its Setting to 1}
	
	The fine-structure constant \(\alphaEM\) is defined as:
	\begin{equation}
		\alphaEM = \frac{e^2}{4\pi\varepsilon_0 \hbar c} \approx \frac{1}{137.036}
	\end{equation}
	
	When setting \(\alphaEM = 1\), it follows:
	\begin{equation}
		e = \sqrt{4\pi\varepsilon_0 \hbar c}
	\end{equation}
	
	This implies that the elementary charge becomes a dimensionless quantity, precisely defined by fundamental constants. In a system with \(\hbar = c = 1\), this simplifies to:
	\begin{equation}
		e = \sqrt{4\pi\varepsilon_0}
	\end{equation}
	
	The consequence is that electric charges become dimensionless, and all electromagnetic quantities can be reduced to energy.
	
	\subsection{The Parameter \(\beta\) in the T0 Model and Its Setting to 1}
	
	In the T0 model, the parameter \(\betaT\) is defined by the following formula:
	\begin{equation}
		\betaT = \frac{\lambda_h^2 v^2}{4\pi^2 \lambda_0^2 \alpha_0} \approx 0.008
	\end{equation}
	
	where \(\lambda_h\) is the Higgs self-coupling, \(v\) is the Higgs vacuum expectation value, \(\lambda_0\) is a fundamental length scale in the T0 model, and \(\alpha_0\) is a reference coupling constant related to the time field \(\Tfield\).
	
	In natural units with \(\hbar = c = G = 1\) and the assumption that \(\lambda_0 = \xi \cdot l_P\) (where \(l_P\) is the Planck length), \(\betaT\) can be expressed in terms of these parameters.
	
	With \(\betaT^{\text{nat}} = 1\), we can define:
	\begin{equation}
		\lambda_0 = \frac{\lambda_h v}{\sqrt{4\pi^2 \alpha_0}} \approx 1.33 \times 10^{-4} \cdot l_P
	\end{equation}
	
	This implies that \(\lambda_0\) is approximately \(1/7519\) of the Planck length.
	
	\subsection{Unit System with \(\alpha = \beta = 1\)}
	
	In a unified unit system where both \(\alphaEM = 1\) and \(\betaT^{\text{nat}} = 1\) are set, the following fundamental dimensional assignments emerge:
	
	\begin{tcolorbox}[colback=blue!5!white,colframe=blue!75!black,title=Dimensional Assignments in the Unified Unit System]
		\begin{tabular}{ll}
			\textbf{Quantity} & \textbf{Dimension} \\
			\hline
			Length & \([E^{-1}]\) \\
			Time & \([E^{-1}]\) \\
			Mass & \([E]\) \\
			Charge & \([Q] = [\sqrt{E}]\) \\
			Intrinsic Time \(\Tfield\) & \([E^{-1}]\) \\
		\end{tabular}
	\end{tcolorbox}
	
	In this system, energy becomes the fundamental unit to which all other physical quantities can be reduced. This aligns with modern approaches in theoretical physics that view energy as the most basic property of the universe.
	
	\section{Analysis of the Consistency of \(\alpha = 1\) and \(\beta = 1\)}
	
	\subsection{Dimensional Analysis of the \(\beta\)-Formula with \(\alpha = 1\)}
	
	To verify the consistency of \(\alpha = 1\) and \(\beta = 1\), we perform a detailed dimensional analysis of the \(\beta\)-formula:
	
	\begin{equation}
		\betaT = \frac{\lambda_h^2 v^2}{4\pi^2 \lambda_0^2 \alpha_0}
	\end{equation}
	
	Let's analyze the dimensions of each term:
	
	1. \(\lambda_h\) (Higgs self-coupling): dimensionless
	2. \(v\) (Higgs vacuum expectation value): \([E]\) (energy)
	3. \(\lambda_0\) (fundamental length scale): \([L] = [E^{-1}]\)
	4. \(\alpha_0\) (reference coupling constant): dimensionless
	
	Substituting these dimensions into the \(\beta\)-formula:
	\begin{align}
		\betaT &= [1] \cdot [E]^2 / ([1] \cdot [E^{-1}]^2 \cdot [1]) \\
		&= [E]^2 \cdot [E^{2}] \\
		&= [1]
	\end{align}
	
	This dimensional analysis confirms that \(\betaT\) is indeed dimensionless, as required for a fundamental parameter. Setting \(\alpha = 1\) does not affect this dimensional homogeneity, since \(\alpha\) itself is dimensionless and does not explicitly appear in the \(\beta\)-formula.
	
	\subsection{Relationship Between \(\lambda_0\) and the Planck Length with \(\alpha = 1\)}
	
	With \(\beta = 1\), we have derived the relationship \(\lambda_0 = \frac{\lambda_h v}{\sqrt{4\pi^2 \alpha_0}}\). The question now is whether this relationship remains consistent when \(\alpha = 1\).
	
	Setting \(\alpha = 1\) could alter the values of the electroweak parameters \(\lambda_h\), \(v\), and \(m_h\), but their fundamental relationships should persist. In particular, the Standard Model relation holds:
	\begin{equation}
		m_h^2 = 2\lambda_h v^2
	\end{equation}
	
	This relationship between the Higgs mass, Higgs self-coupling, and vacuum expectation value should remain valid even in the unified unit system with \(\alpha = 1\) and \(\beta = 1\).
	
	\subsection{Relationship Between Electroweak Parameters and \(\alpha\)}
	
	In the Standard Model, there are several relationships between the electroweak parameters and the fine-structure constant. For example, fermion masses are given by the Yukawa couplings \(y_f\) and the Higgs vacuum expectation value \(v\):
	\begin{equation}
		m_f = y_f \cdot v
	\end{equation}
	
	The Yukawa couplings themselves are linked to the electroweak coupling and thus indirectly to \(\alpha\). Setting \(\alpha = 1\) would increase the strength of the electromagnetic interaction, potentially affecting the Yukawa couplings as well.
	
	A parameter \(\alpha_0\) appears in the formula for \(\betaT\), which might be related to the fine-structure constant \(\alpha\). If we assume \(\alpha_0 = \alpha\) and set \(\alpha = 1\), \(\beta\) would increase by a factor of approximately 137. This implies that setting both \(\alpha = 1\) and \(\beta = 1\) simultaneously would require further adjustments in the parametrization.
	
	One possibility is that other parameters, such as \(\lambda_0\), would need to be adjusted when \(\alpha = 1\) to maintain consistency. This suggests a deeper connection between the electromagnetic interaction and the intrinsic time field dynamics, warranting further investigation.
	
	\subsection{The Role of \(\alpha_0\) in the Unified System}
	
	A closer examination of the parameter \(\alpha_0\) in the \(\beta\)-formula is crucial for understanding the consistency of the unified unit system. Three possibilities are considered:
	
	\begin{enumerate}
		\item \textbf{\(\alpha_0\) as an Independent Parameter:} In this case, \(\alpha_0\) would be a standalone parameter of the T0 model, independent of the fine-structure constant \(\alpha\). Setting \(\alpha = 1\) and \(\beta = 1\) simultaneously would then be straightforward.
		
		\item \textbf{\(\alpha_0 = \alpha\):} If \(\alpha_0\) is directly identified with the fine-structure constant, setting \(\alpha = 1\) would affect the values of other T0 model parameters. To maintain \(\beta = 1\), for instance, \(\lambda_0\) would need adjustment:
		\begin{equation}
			\lambda_0 = \frac{\lambda_h v}{\sqrt{4\pi^2 \alpha}}
		\end{equation}
		
		\item \textbf{\(\alpha_0\) as a Function of \(\alpha\):} A third possibility is that \(\alpha_0\) is a function of \(\alpha\), such as \(\alpha_0 = f(\alpha)\), where \(f\) is a function to be determined. In this case, the consistency of the unified system could impose additional constraints on \(f\).
	\end{enumerate}
	
	The theoretically most elegant solution would be to treat \(\alpha_0\) and \(\alpha\) as independent parameters that can both be set to 1 in a more fundamental unified framework. This would imply that both electromagnetism and the intrinsic time-mass duality are characterized by natural coupling strengths.
	
	A detailed analysis of the underlying quantum field theory of the T0 model could shed light on the exact nature of \(\alpha_0\) and its relationship to \(\alpha\). Specifically, it would need to be investigated whether \(\alpha_0\) can be derived from fundamental principles or if it is a phenomenological parameter requiring empirical determination.
	
	\section{Field Equations in the Unified Unit System}
	
	\subsection{Maxwell's Equations with \(\alpha = 1\)}
	
	In natural units with \(\alpha = 1\), Maxwell's equations take a particularly simple form:
	\begin{align}
		\nabla \cdot \vec{E} &= \rho \\
		\nabla \times \vec{B} - \frac{\partial \vec{E}}{\partial t} &= \vec{j} \\
		\nabla \cdot \vec{B} &= 0 \\
		\nabla \times \vec{E} + \frac{\partial \vec{B}}{\partial t} &= 0
	\end{align}
	
	Here, the electromagnetic interaction is characterized by a dimensionless charge, highlighting the intrinsic connection between electromagnetism and the fundamental properties of space and time.
	
	\subsection{T0 Model Equations with \(\beta = 1\)}
	
	In the T0 model, the temperature-redshift relation with \(\betaT^{\text{nat}} = 1\) becomes:
	\begin{equation}
		T(z) = T_0 (1+z)(1+\ln(1+z))
	\end{equation}
	
	and the modified gravitational potential takes the form:
	\begin{equation}
		\Phi(r) = -\frac{r_g}{r} + \kappa r\\\text{where } r_g = \frac{GM}{c^2} \text{ (in SI units), which becomes } \frac{M}{M_P^2} \text{ in natural units}, \quad [r_g] = [E^{-1}]

	\end{equation}
	
	with \(\kappa = \frac{y v}{r_g}\) in natural units.
	
	In a system with \(\alpha = \betaT^{\text{nat}} = 1\), these equations become particularly elegant, as all fundamental couplings assume the natural value of 1.
	
	\subsection{Unified Dynamics of Charge and Intrinsic Time}
	
	In the unified unit system with \(\alpha = \betaT^{\text{nat}} = 1\), the possibility arises to describe electromagnetism and the dynamics of the intrinsic time field \(\Tfield\) within a coherent framework. Consider the Lagrangian density of the combined system:
	
	\begin{equation}
		\mathcal{L} = \mathcal{L}_{\text{EM}} + \mathcal{L}_{\text{T}} + \mathcal{L}_{\text{int}}
	\end{equation}
	
	where \(\mathcal{L}_{\text{EM}}\) is the electromagnetic Lagrangian density, \(\mathcal{L}_{\text{T}}\) is the Lagrangian density of the intrinsic time field, and \(\mathcal{L}_{\text{int}}\) describes the interaction between them.
	
	With \(\alpha = \betaT^{\text{nat}} = 1\), the coupling terms take particularly simple forms, potentially revealing new symmetries.
	
	An intriguing hypothesis is that the electromagnetic interaction and the emergent gravitation in the T0 model could be two aspects of a deeper unified interaction, similar to how electromagnetic and weak interactions are unified in the electroweak model.
	
	\section{Theoretical Implications and New Insights}
	
	\subsection{Hierarchy of Dimensionless Constants}
	
	In fundamental physical theories, a natural hierarchy of dimensionless constants can be identified:
	
	\begin{enumerate}[label=\arabic*.]
		\item \textbf{Fundamental Natural Constants as Units:} \(c = \hbar = G = k_B = 1\)
		\item \textbf{Dimensionless Coupling Constants:} \(\alpha = \betaT^{\text{nat}} = \alpha_W = 1\)
		\item \textbf{Derived Dimensionless Ratios:} \(\xi = \lambda_0/l_P \approx 1.33 \times 10^{-4}\)
	\end{enumerate}
	
	This hierarchy reflects the underlying structure of physics and becomes particularly evident in the unified unit system with \(\alpha = \betaT^{\text{nat}} = 1\).
	
	\subsection{Ratios Between Fundamental Length and Energy Scales}
	
	A notable result of our analysis is the identification of specific ratios between the characteristic length and energy scales in the unified system. These ratios may hold profound physical significance:
	
	\begin{tcolorbox}[colback=blue!5!white,colframe=blue!75!black,title=Fundamental Ratios in the Unified Unit System]
		\begin{align}
			\frac{\lambda_0}{l_P} &= \xi \approx 1.33 \times 10^{-4} \\
			\frac{L_T}{l_P} &\approx 3.9 \times 10^{62} \\
			\frac{\lambda_0}{L_T} &\approx \frac{\lambda_h^2 v^4}{4\pi^2 \lambda_0^2 \alpha_0 \Mpl} \approx 3.41 \times 10^{-67}
		\end{align}
	\end{tcolorbox}
	
	These ratios are purely dimensionless and independent of the choice of unit system. They represent fundamental aspects of the theory and may hint at deeper structures.
	
	Particularly striking is that the ratio between the characteristic T0 interaction length \(\lambda_0\) and the cosmological correlation length \(L_T\) is on the order of \((m_e/M_{Pl})^2\), suggesting a possible connection between the electron mass and the T0 model.
	
	\subsection{Potential Deeper Connection Between Electrodynamics and T0 Dynamics}
	
	The consistency of simultaneously setting \(\alpha = 1\) and \(\betaT^{\text{nat}} = 1\) suggests a deeper connection between electrodynamics and the dynamics of the intrinsic time field in the T0 model. This could indicate a common cause or origin for both interactions.
	
	One possibility is that both interactions arise from a more fundamental theory in which \(\alpha\) and \(\betaT\) are not independent parameters but different manifestations of a single coupling constant.
	
	\subsection{Quantum Field Theoretical Interpretation}
	
	From a quantum field theoretical perspective, both \(\alpha\) and \(\betaT\) can be interpreted as renormalization group fixed points. In an ideal unified system, both parameters would flow to the natural value of 1 in the infrared limit:
	
	\begin{equation}
		\lim_{E \to 0} \alpha(E) = \lim_{E \to 0} \betaT(E) = 1
	\end{equation}
	
	The experimental value \(\alpha \approx 1/137\) would then be a result of renormalization group evolution at finite energies, as would \(\betaT^{\text{SI}} \approx 0.008\).
	
	This interpretation aligns with modern approaches in quantum field theory that view dimensionless coupling constants as energy-dependent quantities that assume their "natural" values only at specific energy scales.
	
	\subsection{Comparison with Other Unification Theories}
	
	The unification proposed here through \(\alpha = \betaT^{\text{nat}} = 1\) has parallels with other unification approaches in theoretical physics:
	
	\begin{tcolorbox}[colback=blue!5!white,colframe=blue!75!black,title=Comparison with Other Unification Theories]
		\begin{tabular}{>{\raggedright\arraybackslash}p{3cm}|>{\raggedright\arraybackslash}p{8cm}}
			\textbf{Theory} & \textbf{Unification Approach} \\
			\hline
			Electroweak Theory & Unification of electromagnetic and weak interactions via \(SU(2) \times U(1)\) symmetry \\
			\hline
			Grand Unified Theories & Unification of all non-gravitational interactions in a single gauge group \\
			\hline
			String Theory & Unification of all interactions including gravitation through vibrating strings \\
			\hline
			Loop Quantum Gravity & Quantization of spacetime through spin networks \\
			\hline
			T0 Model with \(\alpha = \betaT^{\text{nat}} = 1\) & Unification of electrodynamics and emergent gravitation through a common energy unit \\
		\end{tabular}
	\end{tcolorbox}
	
	The unified perspective of the T0 model with \(\alpha = \betaT^{\text{nat}} = 1\) offers a unique approach that is conceptually simpler than many established unification theories.
	
	\section{Experimental Tests and Predictions}
	
	\subsection{Direct Tests of the Unified Theory}
	
	To test the unified theory with \(\alpha = \betaT^{\text{nat}} = 1\), the following experiments could be conducted:
	
	\begin{enumerate}
		\item \textbf{Precision Measurements of Wavelength-Dependent Redshift:} The theory predicts a specific wavelength dependence of redshift, which could be tested with modern astronomical instruments like the James Webb Space Telescope.
		
		\item \textbf{Search for Deviations in Electromagnetic Fine-Structure Measurements:} If \(\alpha\) and \(\betaT\) are connected, subtle deviations in fine-structure measurements over cosmological distances might be observable.
		
		\item \textbf{Tests of Modified Gravitational Dynamics:} The unified model predicts specific deviations from Newtonian gravitational dynamics, which could be detected in precise measurements of galaxy dynamics.
	\end{enumerate}
	
	\subsection{Quantitative Predictions of the Unified Theory}
	
	The unified theory with \(\alpha = \betaT^{\text{nat}} = 1\) makes specific quantitative predictions that can be experimentally verified:
	
	\subsection{Wavelength-Dependent Redshift}
	
	In the T0 model with a unified unit system where \(\betaT^{\text{nat}} = 1\), a characteristic wavelength-dependent redshift emerges, described by:
	\begin{equation}
		z(\lambda) = z_0 \left(1 + \ln \frac{\lambda}{\lambda_{\text{ref}}}\right)
	\end{equation}
	where \(z_0\) is the redshift at the reference wavelength \(\lambda_{\text{ref}}\), and \(\lambda\) is the observed wavelength. This formula stems from the T0 model's fundamental assumption that the intrinsic time \(\Tfield\) and its interaction with electromagnetic fields exhibit a logarithmic wavelength dependence when \(\betaT^{\text{nat}} = 1\) in natural units.
	
	This wavelength-dependent redshift is a direct consequence of the time field's coupling to cosmic expansion and differs from standard cosmology, where redshift is typically considered wavelength-independent. In the unified system with \(\alpha = \betaT^{\text{nat}} = 1\), this relationship takes a particularly elegant form, free of additional scaling factors, reflecting the natural unity of the coupling constants.
	
	For experimental comparisons, the relation can be converted to SI units, scaling the parameter \(\betaT\):
	\begin{equation}
		z(\lambda)_{\text{SI}} = z_0 \left(1 + \betaT^{\text{SI}} \ln \frac{\lambda}{\lambda_{\text{ref}}}\right)
	\end{equation}
	with \(\betaT^{\text{SI}} = \betaT^{\text{nat}} \cdot \frac{\lambda_0 \cdot l_{P,\text{SI}}}{r_{0,\text{SI}}} \approx 0.008\), as derived in \cite{pascher_emergente_gravitation_2025} and \cite{pascher_params_2025}. This scaling allows direct comparison with astronomical observations, while the primary formulation in natural units emphasizes the model's theoretical consistency.
	
	The prediction of a wavelength-dependent redshift provides an opportunity to test the T0 model experimentally, such as through multi-frequency observations of distant quasars or galaxies with instruments like the James Webb Space Telescope. A detailed derivation and discussion of this property can be found in \cite{pascher_emergente_gravitation_2025}.
	
	\subsection{Conversion Between Natural and SI Unit Systems}
	
	For practical calculations and comparison with experimental data, it is important to convert between the unified natural unit system and the SI system. Here is a general conversion scheme:
	
	\begin{tcolorbox}[colback=blue!5!white,colframe=blue!75!black,title=Conversion Scheme for Natural and SI Units]
		\begin{enumerate}
			\item \textbf{Length Scales:}
			\begin{align}
				L_{\text{SI}} &= L_{\text{NE}} \cdot l_{P,\text{SI}}
			\end{align}
			\item \textbf{Energy Scales:}
			\begin{align}
				E_{\text{SI}} &= E_{\text{NE}} \cdot E_{P,\text{SI}} \\
				&= E_{\text{NE}} \cdot \sqrt{\frac{\hbar c^5}{G}}
			\end{align}
			\item \textbf{Electric Charge:}
			\begin{align}
				Q_{\text{SI}} &= Q_{\text{NE}} \cdot \sqrt{4\pi\varepsilon_0 \hbar c}
			\end{align}
			\item \textbf{Dimensionless Parameters:}
			\begin{align}
				\alpha_{\text{SI}} &= \frac{\alpha_{\text{NE}}}{137.036} \\
				\betaT^{\text{SI}} &= \betaT^{\text{nat}} \cdot \frac{\lambda_0 \cdot l_{P,\text{SI}}}{r_{0,\text{SI}}}
			\end{align}
		\end{enumerate}
	\end{tcolorbox}
	
	Using this scheme, any physical quantity can be translated between the fundamental formulation (in natural units with \(\alpha = \betaT^{\text{nat}} = 1\)) and the experimental formulation (in SI units).
	
	\section{Conclusions and Outlook}
	
	This work has investigated the theoretical consistency and implications of a unified natural unit system in which both the fine-structure constant \(\alpha = 1\) and the T0 model parameter \(\betaT^{\text{nat}} = 1\) are set. The main findings are:
	
	\begin{enumerate}
		\item Setting \(\alpha = 1\) and \(\betaT^{\text{nat}} = 1\) simultaneously is mathematically consistent and leads to an elegant theoretical framework where energy is the fundamental unit.
		\item The characteristic length scale \(\lambda_0\) of the T0 model can be interpreted as a specific ratio to the Planck length: \(\lambda_0 \approx 1.33 \times 10^{-4} \cdot l_P\), independent of the value of \(\alpha\).
		\item The field equations for both electrodynamics and T0 dynamics take particularly simple forms in this unified unit system, suggesting a deeper connection between the two interactions.
		\item The unified model provides specific predictions for cosmological observations, particularly regarding wavelength-dependent redshift and cosmic temperature evolution.
	\end{enumerate}
	
	The unification presented here opens new perspectives for the T0 model and could lead to a deeper unification theory that describes electrodynamics and emergent gravitation within a coherent framework. Future research should focus on the precise mathematical formulation of this unification and experimental tests to verify the model's validity.
	
	\begin{thebibliography}{99}
		\bibitem{pascher_zeit_2025} Pascher, J. (2025). \href{https://github.com/jpascher/T0-Time-Mass-Duality/tree/main/2/pdf/English/ZeitEmergentQMEn.pdf}{Time as an Emergent Property in Quantum Mechanics: A Connection Between Relativity, Fine-Structure Constant, and Quantum Dynamics}. March 23, 2025.
		\bibitem{pascher_messdifferenzen_2025} Pascher, J. (2025). \href{https://github.com/jpascher/T0-Time-Mass-Duality/tree/main/2/pdf/English/MessdifferenzenT0StandardEn.pdf}{Compensatory and Additive Effects: An Analysis of Measurement Differences Between the T0 Model and the \(\Lambda\)CDM Standard Model}. April 2, 2025.
		\bibitem{pascher_galaxies_2025} Pascher, J. (2025). \href{https://github.com/jpascher/T0-Time-Mass-Duality/tree/main/2/pdf/English/MassVarGalaxienEn.pdf}{Mass Variation in Galaxies: An Analysis in the T0 Model with Emergent Gravitation}. March 30, 2025.
		\bibitem{pascher_params_2025} Pascher, J. (2025). \href{https://github.com/jpascher/T0-Time-Mass-Duality/tree/main/2/pdf/English/ZeitMasseT0ParamsEn.pdf}{Time-Mass Duality Theory (T0 Model): Derivation of Parameters \(\kappa\), \(\alpha\), and \(\beta\)}. March 30, 2025.
			\bibitem{pascher_temp_2025} Pascher, J. (2025). \href{https://github.com/jpascher/T0-Time-Mass-Duality/tree/main/2/pdf/English/TempEinheitenCMBEn.pdf}{Adjustment of Temperature Units in Natural Units and CMB Measurements}. April 2, 2025.
			\bibitem{pascher_alpha_2025} Pascher, J. (2025). \href{https://github.com/jpascher/T0-Time-Mass-Duality/tree/main/2/pdf/English/NatEinheitenAlpha1En.pdf}{Energy as a Fundamental Unit: Natural Units with \(\alpha = 1\) in the T0 Model}. March 26, 2025.
			\bibitem{pascher_beta_2025} Pascher, J. (2025). \href{https://github.com/jpascher/T0-Time-Mass-Duality/tree/main/2/pdf/English/Alpha1Beta1KonsistenzEn.pdf}{Dimensionless Parameters in the T0 Model: Setting \(\beta = 1\) in Natural Units}. April 4, 2025.
			\bibitem{pascher_higgs_2025} Pascher, J. (2025). \href{https://github.com/jpascher/T0-Time-Mass-Duality/tree/main/2/pdf/English/MathHiggsZeitMasseEn.pdf}{Mathematical Formulation of the Higgs Mechanism in Time-Mass Duality}. March 28, 2025.
			\bibitem{pascher_lagrange_2025} Pascher, J. (2025). \href{https://github.com/jpascher/T0-Time-Mass-Duality/tree/main/2/pdf/English/MathZeitMasseLagrangeEn.pdf}{From Time Dilation to Mass Variation: Mathematical Core Formulations of Time-Mass Duality Theory}. March 29, 2025.
			\bibitem{pascher_emergente_gravitation_2025} Pascher, J. (2025). \href{https://github.com/jpascher/T0-Time-Mass-Duality/tree/main/2/pdf/English/EmergentGravT0En.pdf}{Emergent Gravitation in the T0 Model: A Comprehensive Derivation}. April 1, 2025.
			\bibitem{pascher_feldtheorie_2025} Pascher, J. (2025). \href{https://github.com/jpascher/T0-Time-Mass-Duality/tree/main/2/pdf/English/FeldtheorieQuantenEn.pdf}{Field Theory and Quantum Correlations: A New Perspective on Instantaneity}. March 28, 2025.
			\bibitem{pascher_planck_2025} Pascher, J. (2025). \href{https://github.com/jpascher/T0-Time-Mass-Duality/tree/main/2/pdf/English/JenseitsPlanckEn.pdf}{Real Consequences of Reformulating Time and Mass in Physics: Beyond the Planck Scale}. March 24, 2025.
			\bibitem{pascher_erweiterung_2025} Pascher, J. (2025). \href{https://github.com/jpascher/T0-Time-Mass-Duality/tree/main/2/pdf/English/NotwendigkeitQMErweiterungEn.pdf}{The Necessity of Extending Standard Quantum Mechanics and Quantum Field Theory}. March 27, 2025.
			\bibitem{pascher_energiedynamik_2025} Pascher, J. (2025). \href{https://github.com/jpascher/T0-Time-Mass-Duality/tree/main/2/pdf/English/MathEnergiedynamikEn.pdf}{Dark Energy in the T0 Model: A Mathematical Analysis of Energy Dynamics}. April 3, 2025.
			\bibitem{pascher_vereinheitlichung_2025} Pascher, J. (2025). \href{https://github.com/jpascher/T0-Time-Mass-Duality/tree/main/2/pdf/English/T0VereinheitlichungDEGalEn.pdf}{Unification of the T0 Model: Foundations, Dark Energy, and Galaxy Dynamics}. April 4, 2025.
			\bibitem{pascher_formalismen_2025} Pascher, J. (2025). \href{https://github.com/jpascher/T0-Time-Mass-Duality/tree/main/2/pdf/English/MathZeitMasseLagrangeEn.pdf}{From Time Dilation to Mass Variation: Mathematical Core Formulations of Time-Mass Duality Theory}. April 5, 2025.
			\bibitem{pascher_perspektive_2025} Pascher, J. (2025). \href{https://github.com/jpascher/T0-Time-Mass-Duality/tree/main/2/pdf/English/ZeitRaumPascherEn.pdf}{A New Perspective on Time and Space: Johann Pascher's Revolutionary Ideas}. March 25, 2025.
			\bibitem{pascher_dualismus_2025} Pascher, J. (2025). \href{https://github.com/jpascher/T0-Time-Mass-Duality/tree/main/2/pdf/English/KurzKomplementDualPhysikEn.pdf}{In Brief - Complementary Duality in Physics: From Wave-Particle to Time-Mass Concept}. March 26, 2025.
			\bibitem{pascher_grundkraefte_2025} Pascher, J. (2025). \href{https://github.com/jpascher/T0-Time-Mass-Duality/tree/main/2/pdf/English/VierKraefteZeitMasseEn.pdf}{Simplified Description of the Fundamental Forces with Time-Mass Duality}. March 27, 2025.
			\bibitem{pascher_zeit_masse_2025} Pascher, J. (2025). \href{https://github.com/jpascher/T0-Time-Mass-Duality/tree/main/2/pdf/English/ZeitMasseNeuerBlickEn.pdf}{Time and Mass: A New Look at Old Formulas – and Liberation from Traditional Constraints}. March 22, 2025.
			\bibitem{Planck1899} Planck, M. (1899). On Irreversible Radiation Processes. Proceedings of the Prussian Academy of Sciences, 5, 440-480.
			\bibitem{Feynman1985} Feynman, R. P. (1985). QED: The Strange Theory of Light and Matter. Princeton University Press.
			\bibitem{Duff2002} Duff, M. J., Okun, L. B., \& Veneziano, G. (2002). Trialogue on the Number of Fundamental Constants. Journal of High Energy Physics, 2002(03), 023.
			\bibitem{Verlinde2011} Verlinde, E. (2011). On the Origin of Gravity and the Laws of Newton. Journal of High Energy Physics, 2011(4), 29.
			\bibitem{Wilczek2008} Wilczek, F. (2008). The Lightness of Being: Mass, Ether, and the Unification of Forces. Basic Books.
			\bibitem{DiracLargeNumbers} Dirac, P. A. M. (1937). The Cosmological Constants. Nature, 139, 323.
			\bibitem{Eddington1946} Eddington, A. S. (1946). Fundamental Theory. Cambridge University Press.
			\bibitem{WeinbergAsymSafety} Weinberg, S. (1979). Ultraviolet Divergences in Quantum Theories of Gravitation. In General Relativity: An Einstein Centenary Survey, ed. S. W. Hawking and W. Israel, Cambridge University Press, pp. 790-831.
			\bibitem{tHooft1993} 't Hooft, G. (1993). Dimensional Reduction in Quantum Gravity. In Salamfestschrift: A Collection of Talks, World Scientific Series in 20th Century Physics, vol. 4, ed. A. Ali, J. Ellis, and S. Randjbar-Daemi, World Scientific, pp. 284-296.
			\bibitem{Planck2018Temp} Planck Collaboration, Aghanim, N., et al. (2020). \textit{Planck 2018 Results. V. CMB Power Spectra and Likelihoods}. Astronomy \& Astrophysics, 641, A5. DOI: 10.1051/0004-6361/201833887.
			\bibitem{Ferroen2009} Fixsen, D. J. (2009). \textit{The Temperature of the Cosmic Microwave Background}. The Astrophysical Journal, 707(2), 916-920. DOI: 10.1088/0004-637X/707/2/916.
			\bibitem{ACTTemp} Choi, S. K., et al. (2020). \textit{The Atacama Cosmology Telescope: A Measurement of the CMB Power Spectra at 98 and 150 GHz}. Journal of Cosmology and Astroparticle Physics, 2020(12), 045. DOI: 10.1088/1475-7516/2020/12/045.
			\bibitem{SPTTemp} Reichardt, C. L., et al. (2021). \textit{The South Pole Telescope 3G Survey: CMB Temperature and Polarization Spectra}. The Astrophysical Journal, 908(2), 199. DOI: 10.3847/1538-4357/abd407.
			\bibitem{Mather1994} Mather, J. C., et al. (1994). \textit{Measurement of the CMB Spectrum by the COBE FIRAS Instrument}. The Astrophysical Journal, 420, 439-444. DOI: 10.1086/173574.
			\bibitem{SunyaevZeldovich} Birkinshaw, M. (1999). \textit{The Sunyaev-Zel'dovich Effect}. Physics Reports, 310(2-3), 97-195. DOI: 10.1016/S0370-1573(98)00080-5.
			\bibitem{PlanckTech} Planck Collaboration, Tauber, J. A., et al. (2010). \textit{Planck Pre-Launch Status: The Planck Mission}. Astronomy \& Astrophysics, 520, A1. DOI: 10.1051/0004-6361/200912983.
			\bibitem{CMBTheoryTemp} Hu, W., \& Dodelson, S. (2002). \textit{Cosmic Microwave Background Anisotropies}. Annual Review of Astronomy and Astrophysics, 40, 171-216. DOI: 10.1146/annurev.astro.40.060401.093926.
			\bibitem{Einstein1915} Einstein, A. (1915). The Field Equations of Gravitation. Proceedings of the Prussian Academy of Sciences in Berlin, 844-847.
			\bibitem{Higgs1964} Higgs, P. W. (1964). Broken Symmetries and the Masses of Gauge Bosons. Physical Review Letters, 13(16), 508-509.
			\bibitem{Will2014} Will, C. M. (2014). The Confrontation between General Relativity and Experiment. Living Reviews in Relativity, 17(1), 4.
		\end{thebibliography}
	
\end{document}