\documentclass[a4paper,12pt]{article}
\usepackage[utf8]{inputenc}
\usepackage[T1]{fontenc}
\usepackage{lmodern}
\usepackage[ngerman]{babel}
\usepackage{amsmath}
\usepackage{amssymb}
\usepackage{geometry}
\usepackage{tocloft}
\usepackage{tikz}
\usepackage{xcolor}
\usepackage[colorlinks=true, linkcolor=blue, citecolor=blue, urlcolor=blue]{hyperref}
\usepackage{siunitx}
\DeclareSIUnit{\year}{yr}
\DeclareSIUnit{\parsec}{pc}
\usepackage{fancyhdr}

\geometry{a4paper, margin=2cm}

% Headers and Footers
\pagestyle{fancy}
\fancyhf{}
\fancyhead[L]{Johann Pascher}
\fancyhead[R]{Time-Mass Duality}
\fancyfoot[C]{\thepage}
\renewcommand{\headrulewidth}{0.4pt}
\renewcommand{\footrulewidth}{0.4pt}

\renewcommand{\cftsecfont}{\color{blue}}
\renewcommand{\cftsubsecfont}{\color{blue}}
\renewcommand{\cftsecpagefont}{\color{blue}}
\renewcommand{\cftsubsecpagefont}{\color{blue}}
\setlength{\cftsecindent}{1cm}
\setlength{\cftsubsecindent}{2cm}

% Custom commands
\newcommand{\Tfield}{T(x)}
\newcommand{\DcovT}[1]{\Tfield D_\mu #1 + #1 \partial_\mu \Tfield}
\newcommand{\DhiggsT}{\Tfield (\partial_\mu + ig A_\mu) \Phi + \Phi \partial_\mu \Tfield}
\newcommand{\betaT}{\beta_{\text{T}}}
\newcommand{\alphaEM}{\alpha_{\text{EM}}}
\newcommand{\Mpl}{M_{\text{Pl}}}
\newcommand{\Tzerot}{T_0(\Tfield)}
\newcommand{\Tzero}{T_0}
\newcommand{\vecx}{\vec{x}}
\newcommand{\gammaf}{\gamma_{\text{Lorentz}}}

\title{Time and Mass: A New Look at Old Formulas – and Liberation from Traditional Constraints}
\author{Johann Pascher}
\date{March 25, 2025}

\begin{document}
	
	\maketitle
	
	\begin{abstract}
		This work presents a new perspective on time and mass, the time-mass duality, which challenges traditional views in quantum mechanics and relativity. Through extended natural units, physical constants are reinterpreted as dimensionless energy ratios. Without introducing new equations, the approach highlights the incompleteness of existing theories and proposes a more unified, intuitive description of reality, with implications for quantum gravity, entanglement, and cosmological phenomena.
	\end{abstract}
	
	\tableofcontents
	\newpage
	
	\section{Introduction: Traditional Views and the Hidden Perspective}
	
	Physics has achieved impressive successes with concepts like quantum fields and spacetime curvature, but sometimes I wonder if we’ve strayed too far from a simple, intuitive description of the world. The way we view time and mass—shaped by historical units and traditional theories—might be preventing us from gaining a deeper, more unified understanding of nature. In this work, I aim to return to the fundamentals and liberate physics from its traditional constraints by introducing a new perspective: time-mass duality. This approach, developed in the T0 model, questions conventional assumptions and demonstrates how a reconsideration of familiar formulas can lead us to a clearer view of reality.
	
	My goal is not to overburden physics with entirely new equations but to see the existing formulas of quantum mechanics and relativity in a new light—one that highlights energy as the central quantity and removes the artificial barriers of traditional units. This journey begins with a rethinking of physical constants, guides us through time-mass duality, and concludes with concrete implications for some of the greatest mysteries of modern physics.
	
	\section{Natural Constants and Units: More Than Arbitrary Numbers?}
	
	When we talk about meters, seconds, and kilograms, we rarely consider that these units are historical accidents—human constructs that don’t necessarily reflect the fundamental nature of the world. Natural constants like the speed of light \(c\), the reduced Planck constant \(\hbar\), the gravitational constant \(G\), or the fine-structure constant \(\alpha\) are often set to 1 in natural units to simplify mathematics. However, in the T0 model, we take a step further: we view these constants not as independent quantities but as expressions of a single fundamental quantity—energy.
	
	In this perspective, \(c\), \(\hbar\), and \(G\) are not arbitrary numbers but dimensionless ratios derived from energy. The T0 model, as described in “Time-Mass Duality Theory: Derivation of Parameters” \cite{pascher_params_2025}, interprets the physical world as a framework where energy is the unifying force. This redefinition may seem subtle, but it frees us from the limitations of traditional units and opens the way to a more intuitive description of reality.
	
	\section{Time-Mass Duality: An Alternative Perspective}
	
	Traditional physics, particularly special relativity, describes the world with a relative time that adjusts via time dilation (\(t' = \gamma t\)) while rest mass remains constant (\(m_0 = \text{const.}\)). The T0 model reverses this view: time is treated as an absolute quantity \(T_0\), while mass is variable (\(m = \gamma m_0\)). But the approach goes further by introducing an intrinsic time field that governs each particle’s dynamics:
	
	\begin{equation}
		\Tfield = \frac{\hbar}{\max(m c^2, \omega)}
	\end{equation}
	
	This time field, detailed in “Parameter Derivations” \cite{pascher_params_2025}, links a particle’s mass to its internal timescale and leads to a modified Schrödinger equation, developed in “The Necessity of Extending Standard Quantum Mechanics” \cite{pascher_quantum_2025}:
	
	\begin{equation}
		i\hbar \Tfield \frac{\partial}{\partial t} \Psi + i\hbar \Psi \frac{\partial \Tfield}{\partial t} = \hat{H} \Psi
	\end{equation}
	
	This perspective reveals that time and mass do not exist independently but are in a dual relationship that transcends the traditional boundaries of physics.
	
	\section{All Constants Become Natural: Energy as the Unifying Principle}
	
	In the T0 model, physical constants lose their status as standalone quantities and become dimensionless ratios of a single fundamental quantity—energy. The speed of light \(c\) becomes the unit of motion, \(\hbar\) the measure of action, and \(G\) an expression of gravitational interaction—all derived from energy. This view liberates physics from the arbitrary units of the past and establishes energy as the true centerpiece of nature. It requires no new formulas, only a reconsideration of the ones we already know from quantum mechanics and relativity.
	
	\section{No New Formulas, But a Liberated View}
	
	The beauty of this approach lies in its avoidance of entirely new equations. Instead, it takes the familiar formulas of physics—the Schrödinger equation, Lorentz transformations, gravitational laws—and views them in a new framework where all constants are natural. This seemingly small shift has profound consequences. It shows that traditional quantum mechanics is incomplete: when transferred to this new system, its formulas no longer fully describe all phenomena because they fail to capture the dynamic interplay of mass, time, and energy.
	
	This incompleteness is not a flaw but a clue that we must extend quantum mechanics—not through arbitrary assumptions but by more consistently applying the principles we already know, particularly energy conservation and the connection between time and mass. The wave-particle duality and time-mass duality are not mere interpretations but keys to a more real perspective, showing us that we’ve overlooked aspects of reality by clinging to traditional constraints.
	
	\section{Lagrangian Formulation}
	
	The mathematical foundation of the T0 model is provided by a total Lagrangian density, as elaborated in “Mathematical Core Formulations” \cite{pascher_lagrange_2025}:
	
	\begin{equation}
		\mathcal{L}_{\text{Total}} = \mathcal{L}_{\text{Boson}} + \mathcal{L}_{\text{Fermion}} + \mathcal{L}_{\text{Higgs-T}} + \mathcal{L}_{\text{intrinsic}}, \quad \mathcal{L}_{\text{intrinsic}} = \frac{1}{2} \partial_\mu \Tfield \partial^\mu \Tfield - V(\Tfield)
	\end{equation}
	
	This formulation integrates the dynamics of the intrinsic time field and provides a unified description of fundamental interactions, overcoming the traditional divide between quantum mechanics and gravity.
	
	\section{Concrete Implications: Toward a More Comprehensive Theory}
	
	This liberated view of physics leads to concrete implications that shed light on some of the greatest mysteries of modern science. In quantum gravity, unification becomes more tangible as gravity emerges as a property of the time field’s gradients:
	
	\begin{equation}
		\nabla \Tfield = -\frac{\hbar}{m^2 c^2} \nabla m
	\end{equation}
	
	The resulting gravitational potential is:
	
	\begin{equation}
		\Phi(r) = -\frac{G M}{r} + \kappa r, \quad \kappa \approx \SI{4.8e-11}{\meter\per\second\squared}
	\end{equation}
	
	This approach, detailed in “Mass Variation in Galaxies” \cite{pascher_galaxies_2025}, offers a new perspective on gravity without the need for separate dark matter models. For quantum entanglement, the model shows how intrinsic time can influence correlations between particles, an idea further explored in “Dynamic Mass of Photons” \cite{pascher_photons_2025}.
	
	In cosmology, redshift is interpreted not as expansion but as energy loss, described by:
	
	\begin{equation}
		1 + z = e^{\alpha d}, \quad \alpha \approx \SI{2.3e-18}{\per\meter}
	\end{equation}
	
	with a wavelength-dependent component:
	
	\begin{equation}
		z(\lambda) = z_0 (1 + \betaT \ln(\lambda/\lambda_0)), \quad \betaT^{\text{SI}} \approx 0.008
	\end{equation}
	
	These formulas, derived in “Measurement Differences” \cite{pascher_messdifferenzen_2025} and “Parameter Derivations” \cite{pascher_params_2025}, demonstrate how dark energy and cosmological phenomena can be reunderstood. Finally, we gain a deeper understanding of fundamental constants by reducing them all to energy, making physics a more unified science.
	
	\begin{figure}[h]
		\centering
		\begin{tikzpicture}
			\draw[->] (0,0) -- (6,0) node[right] {Mass \(m\)};
			\draw[->] (0,0) -- (0,4) node[above] {Intrinsic Time \(\Tfield\)};
			\draw[scale=0.5, domain=0.1:10, smooth, variable=\x, blue, thick] plot ({\x}, {1/\x});
			\node[blue] at (4.5,2) {\(\Tfield \propto \frac{1}{m}\)};
		\end{tikzpicture}
		\caption{Relationship between mass and intrinsic time: Lighter particles have a slower internal clock.}
	\end{figure}
	
	\section{Experimental Verification and Conclusion}
	
	This approach is not just a theoretical exercise but is experimentally testable. It makes predictions that differ from traditional quantum mechanics—such as through precision measurements with clocks or entangled particles of different masses. Time-mass duality and the reduction of all constants to energy offer a radical yet promising path to extend physics. The goal is not to discard proven formulas but to free them from their historical constraints and return to a more real, intuitive perspective. This could be the beginning of a more comprehensive theory that illuminates the universe’s mysteries—from quantum gravity to dark energy—in a new light.
	
	\begin{thebibliography}{99}
		\bibitem{pascher_params_2025} Pascher, J. (2025). \href{https://github.com/jpascher/T0-Time-Mass-Duality/tree/main/2/pdf/English/ZeitMasseT0ParamsEn.pdf}{Time-Mass Duality Theory (T0 Model): Derivation of Parameters \(\kappa\), \(\alpha\), and \(\beta\)}. April 4, 2025.
		\bibitem{pascher_galaxies_2025} Pascher, J. (2025). \href{https://github.com/jpascher/T0-Time-Mass-Duality/tree/main/2/pdf/English/MassVarGalaxienEn.pdf}{Mass Variation in Galaxies: An Analysis in the T0 Model with Emergent Gravity}. March 30, 2025.
		\bibitem{pascher_messdifferenzen_2025} Pascher, J. (2025). \href{https://github.com/jpascher/T0-Time-Mass-Duality/tree/main/2/pdf/English/MessdifferenzenT0StandardEn.pdf}{Compensatory and Additive Effects: An Analysis of Measurement Differences Between the T0 Model and the \(\Lambda\)CDM Standard Model}. April 2, 2025.
		\bibitem{pascher_lagrange_2025} Pascher, J. (2025). \href{https://github.com/jpascher/T0-Time-Mass-Duality/tree/main/2/pdf/English/MathZeitMasseLagrange.pdf}{From Time Dilation to Mass Variation: Mathematical Core Formulations of Time-Mass Duality Theory}. March 29, 2025.
		\bibitem{pascher_photons_2025} Pascher, J. (2025). \href{https://github.com/jpascher/T0-Time-Mass-Duality/tree/main/2/pdf/English/DynMassePhotonenNichtlokalEn.pdf}{Dynamic Mass of Photons and Its Implications for Nonlocality in the T0 Model}. March 25, 2025.
		\bibitem{pascher_quantum_2025} Pascher, J. (2025). \href{https://github.com/jpascher/T0-Time-Mass-Duality/tree/main/2/pdf/English/NotwendigkeitQMErweiterungEn.pdf}{The Necessity of Extending Standard Quantum Mechanics and Quantum Field Theory}. March 27, 2025.
		\bibitem{einstein} Einstein, A. (1905). \textit{On the Electrodynamics of Moving Bodies}. Annalen der Physik, 322(10), 891-921.
	\end{thebibliography}
	
\end{document}