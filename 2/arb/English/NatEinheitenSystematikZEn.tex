\documentclass[twocolumn,aps,prl]{revtex4-2}
\usepackage[utf8]{inputenc}
\usepackage[T1]{fontenc}
\usepackage[ngerman]{babel}
\usepackage{lmodern}
\usepackage{amsmath}
\usepackage{amssymb}
\usepackage{physics}
\usepackage{booktabs}
\usepackage{enumitem}
\usepackage[table,xcdraw]{xcolor}
\usepackage{graphicx}
\usepackage{siunitx}
\usepackage{float}
\usepackage{csquotes}
\usepackage{hyperref} % Hyperref als letztes laden

% Benutzerdefinierte Befehle
\newcommand{\Tfield}{T(x)}
\newcommand{\alphaEM}{\alpha_{\text{EM}}}
\newcommand{\alphaW}{\alpha_{\text{W}}}
\newcommand{\betaT}{\beta_{\text{T}}}
\newcommand{\Mpl}{M_{\text{Pl}}}
\newcommand{\Tzerot}{T_0(\Tfield)}
\newcommand{\Tzero}{T_0}
\newcommand{\vecx}{\vec{x}}
\newcommand{\gammaf}{\gamma_{\text{Lorentz}}}
\newcommand{\LCDM}{\Lambda\text{CDM}}
\newcommand{\calL}{\mathcal{L}}
\newcommand{\e}{\mathrm{e}}
\newcommand{\alphaEMSI}{\alpha_{\text{EM,SI}}}

\hypersetup{
	colorlinks=true,
	linkcolor=blue,
	citecolor=blue,
	urlcolor=blue,
	implicit=false,
	pdftitle={Hierarchisches System natürlicher Einheiten im T0-Modell},
	pdfauthor={Johann Pascher},
	pdfsubject={Theoretische Physik},
	pdfkeywords={T0-Modell, natürliche Einheiten, Zeit-Masse-Dualität}
}

\begin{document}
	
	\title{Hierarchisches System natürlicher Einheiten im T0-Modell: Vereinheitlichung der Physik durch energiebasierte Formulierung}
	\author{Johann Pascher}
	\affiliation{Fachbereich Kommunikationstechnik, Höhere Technische Bundeslehranstalt (HTL), Leonding, Österreich}
	\email{johann.pascher@gmail.com}
	\date{13. April 2025}
	
	\begin{abstract}
		Dieser Artikel präsentiert eine umfassende hierarchische Formulierung natürlicher Einheiten im T0-Modell der Zeit-Masse-Dualität, wobei Energie als fundamentale Einheit dient. Durch die Normierung dimensionaler Konstanten (\(\hbar = c = G = k_B = 1\)) und dimensionsloser Kopplungskonstanten (\(\alpha_{\text{EM}} = \alpha_W = \beta_T = 1\)) auf Eins wird ein vereinheitlichter Rahmen etabliert, der Quanten-, relativistische und kosmologische Phänomene integriert. Unsere Zusammenstellung detailliert die Hierarchie der Konstanten, quantisierte Längenskalen über 97 Größenordnungen von sub-Planck- bis kosmischen Bereichen und die bemerkenswerte Präsenz biologischer Strukturen in ansonsten verbotenen Zonen. Elektromagnetische, thermodynamische und quantenmechanische Konstanten werden direkt aus der Energieskala abgeleitet, wobei vereinfachte Feldgleichungen die intrinsische Einheit physikalischer Gesetze offenbaren. Die Einstein-Hilbert-Wirkung wird neu interpretiert, um emergente Gravitation zu untermauern, in Übereinstimmung mit modernen Ansätzen zur Quantengravitation und experimentellen Beobachtungen. Unterstützt durch theoretische Ableitungen und rigorose mathematische Formulierungen, fördert diese Arbeit die Vereinheitlichung der Physik durch das energiebasierte Paradigma des T0-Modells und bietet überprüfbare Vorhersagen über multiple Skalen, die mit bestehenden kosmologischen und Teilchenphysik-Daten validiert werden können.
	\end{abstract}
	
	\maketitle
	
	\section{Einleitung}
	\label{sec:introduction}
	
	Natürliche Einheiten in der theoretischen Physik vereinfachen die Beschreibung physikalischer Gesetze, indem sie unabhängige Dimensionen reduzieren und fundamentale Konstanten auf Eins setzen, wodurch die intrinsische Einfachheit komplexer Phänomene sichtbar wird. Traditionelle Systeme wie Planck-Einheiten, bei denen \(\hbar = c = G = 1\), dienen seit langem als Grundpfeiler für theoretische Untersuchungen, eliminieren willkürliche dimensionale Parameter und fokussieren auf die Essenz physikalischer Wechselwirkungen \cite{Planck1899}. Dieser Ansatz hat bedeutende Fortschritte in der Quantengravitationsforschung \cite{Rovelli2004, Ashtekar2007} und Stringtheorie \cite{Greene1999} ermöglicht. Ähnlich verwenden Teilchenphysiker Systeme mit \(\hbar = c = 1\), um Berechnungen zu vereinfachen \cite{Peskin1995}, während Stoney-Einheiten bereits vor Plancks Arbeiten universelle Messstandards anstrebten \cite{Stoney1881}.
	
	Das T0-Modell der Zeit-Masse-Dualität erweitert diese Paradigmen, indem es ein vollständig vereinheitlichtes System natürlicher Einheiten vorschlägt, bei dem nicht nur dimensionale Konstanten (\(\hbar = c = G = k_B = 1\)), sondern auch dimensionslose Kopplungskonstanten – die Feinstrukturkonstante \(\alpha_{\text{EM}}\), die Wien-Konstante \(\alpha_W\) und der modellspezifische T0-Parameter \(\beta_T\) – auf Eins gesetzt werden. Diese Normierung ist keine bloße mathematische Bequemlichkeit, sondern eine tiefgreifende theoretische Notwendigkeit, die die Prämisse des Modells widerspiegelt, dass alle physikalischen Gesetze in einem singularen, energiebasieren Rahmen konvergieren, aus dem alle Konstanten und Einheiten systematisch abgeleitet werden können. Dieser Ansatz resonanz mit Diracs Hypothese großer Zahlen \cite{Dirac1937} und neueren Arbeiten von Duff, Okun und Veneziano zur Rolle dimensionsloser Konstanten \cite{Duff2002}.
	
	Im Kern definiert das T0-Modell die fundamentale Beziehung zwischen Zeit und Masse neu und stellt konventionelle Annahmen sowohl der Relativitätstheorie als auch der Quantenmechanik infrage. Im Gegensatz zur relativen Zeit der speziellen Relativitätstheorie \cite{Einstein1905} oder der Behandlung der Zeit als bloßer Parameter in der Quantenmechanik \cite{Schrodinger1926} postuliert das T0-Modell die Zeit als absolute Entität, wobei die Masse dynamisch auf den Zustand des Systems reagiert. Diese konzeptionelle Umkehrung teilt philosophische Elemente mit Machs Prinzip \cite{Mach1893} und Julian Barbours zeitloser Physik \cite{Barbour1999}, jedoch mit einer distinkten mathematischen Formulierung. Sie wird durch das intrinsische Zeitfeld vermittelt, definiert als:
	
	\begin{equation}
		T(x) = \frac{\hbar}{\max(mc^2, \omega)}, \label{eq:intrinsic_time}
	\end{equation}
	
	Dieses Skalarfeld kapselt das Zusammenspiel von Masse-Energie und Frequenz ein und dient als vereinheitlichende Brücke zwischen dem mikroskopischen Bereich der Quantenmechanik und dem makroskopischen Bereich der Relativität. Durch die Neuinterpretation gravitativer Effekte als emergente Phänomene, die aus \(T(x)\)-Gradienten hervorgehen, eliminiert das Modell die Notwendigkeit einer fundamentalen gravitativen Wechselwirkung und steht im Einklang mit modernen Theorien der emergenten Gravitation, entwickelt von Verlinde \cite{Verlinde2011}, Padmanabhan \cite{Padmanabhan2012} und Jacobson \cite{Jacobson1995], während es eine neue Perspektive auf kosmische Dynamiken bietet \cite{pascher_emergente_2025, pascher_part1_2025}.
		
		Die Wahl der Energie als Basiseinheit im T0-Modell ist sowohl intuitiv als auch revolutionär. Energie, als gemeinsame Währung physikalischer Wechselwirkungen, ermöglicht es, alle Größen – Länge, Zeit, Masse, Temperatur – in Bezug auf \([E]\) oder dessen Inverse \([E^{-1}]\) auszudrücken, wie in \hyperref[sec:conversions]{Abschnitt 16} detailliert beschrieben. Dieser Ansatz erweitert Einsteins Erkenntnisse zur Masse-Energie-Äquivalenz \cite{Einstein1905b} und steht im Einklang mit Wheelers „it from bit“-Konzept, dass Energie-Informations-Betrachtungen fundamental für die physikalische Realität sind \cite{Wheeler1990}. Diese Vereinheitlichung vereinfacht Feldgleichungen, wie in \hyperref[sec:field_equations]{Abschnitt 10} gezeigt, und offenbart hierarchische Beziehungen zwischen Konstanten und Skalen, dargestellt in \hyperref[sec:hierarchy]{Abschnitt 2} und \hyperref[sec:length_scales]{Abschnitt 6}. Die Fähigkeit des Modells, Phänomene über Skalen hinweg zu erklären – von Quantenverschränkung bis hin zu kosmischer Rotverschiebung und dunkler Energie – ohne ad-hoc-Konstrukte wie Inflation \cite{Guth1981} oder dunkle Materie \cite{Rubin1980} zu verwenden, unterstreicht sein Potenzial, unser Verständnis des Universums zu revolutionieren \cite{pascher_energiedynamik_2025}, im Einklang mit Milgroms Modifizierter Newton’scher Dynamik \cite{Milgrom1983] und aktuellen beobachtenden Herausforderungen des \(\Lambda\text{CDM}\)-Modells \cite{Riess2016}.
			
			Dieser Artikel präsentiert systematisch die natürlichen Einheiten des T0-Modells, mit Schwerpunkt auf ihren Definitionen, Werten und Verknüpfungen. Wir untersuchen die theoretischen Grundlagen für das Setzen von \(\alpha_{\text{EM}} = \beta_T = 1\) (\hyperref[subsec:beta_derivation]{Abschnitt 4}), charakterisieren Längenskalen über 97 Größenordnungen (\hyperref[sec:length_scales]{Abschnitt 6}) und heben die überraschende Präsenz biologischer Strukturen in verbotenen Zonen hervor (\hyperref[subsec:bio_anomalies]{Abschnitt 9}) – ein Befund, der an Schrödingers frühe Einsichten zur physikalischen Grundlage des Lebens \cite{Schrodinger1944} und neuere Arbeiten zur Quantenbiologie \cite{McFadden2014} anknüpft. Die Arbeit leitet elektromagnetische, thermodynamische und quantenmechanische Konstanten aus der Energieskala ab und präsentiert vereinfachte Feldgleichungen, die die Einheit physikalischer Gesetze beleuchten (\hyperref[sec:field_equations]{Abschnitt 10}). Die Einstein-Hilbert-Wirkung bildet die Grundlage für emergente Gravitation (\hyperref[subsec:gravitation]{Abschnitt 15}), während Konversionen in SI-Einheiten und experimentelle Perspektiven (\hyperref[sec:conversions]{Abschnitt 16} und \hyperref[sec:outlook]{Abschnitt 19}) den Rahmen vervollständigen.
			
			\section{Vereinheitlichung der Konstanten mit natürlichen Einheiten}
			\label{sec:hierarchy}
			
			\subsection{Hierarchie der fundamentalen Konstanten}
			\label{subsec:level1}
			
			Das System natürlicher Einheiten des T0-Modells basiert auf dimensionalen Konstanten, die auf Eins gesetzt werden, und etabliert die grundlegenden Skalen der Physik.
			
			Die reduzierte Planck-Konstante (\(\hbar = 1\)) definiert die Quantenskala, die Energiequantisierung regelt, erstmals systematisch von Planck eingeführt \cite{Planck1901} und weiterentwickelt von Schrödinger \cite{Schrodinger1926b} und Heisenberg \cite{Heisenberg1925].
				
				Die Lichtgeschwindigkeit (\(c = 1\)) setzt die relativistische Skala, vereinheitlicht Raum und Zeit, experimentell mit zunehmender Präzision seit Michelson-Morley gemessen \cite{Michelson1887} und theoretisch von Einstein etabliert \cite{Einstein1905].
					
					Die Gravitationskonstante (\(G = 1\)) legt die Gravitationsskala fest, verbunden mit emergenter Gravitation, historisch von Cavendish gemessen \cite{Cavendish1798} und fundamental für Newtons \cite{Newton1687} und Einsteins Gravitationstheorien \cite{Einstein1916].
						
						Die Boltzmann-Konstante (\(k_B = 1\)) definiert die thermodynamische Skala, verbindet Energie mit Temperatur, zentral für die statistische Mechanik seit Boltzmanns Pionierarbeiten \cite{Boltzmann1872].
							
							Dimensionslose Kopplungskonstanten, ebenfalls auf Eins gesetzt, regeln die Stärke von Wechselwirkungen:
							
							Die Feinstrukturkonstante (\(\alpha_{\text{EM}} = 1\)) mit SI-Wert \(\approx 1/137{,}036\), erstmals von Sommerfeld identifiziert \cite{Sommerfeld1916} und mit zunehmender Präzision gemessen \cite{Aoyama2018], vereinfacht elektromagnetische Gleichungen.
								
								Die Wien-Konstante (\(\alpha_W = 1\)) mit SI-Wert \(\approx 2{,}82\), empirisch von Wien etabliert \cite{Wien1896} und theoretisch von Planck \cite{Planck1901], vereinheitlicht die Thermodynamik.
									
									Der T0-Parameter (\(\beta_T = 1\)) mit SI-Wert \(\approx 0{,}008\), zentral für die \(T(x)\)-Dynamik, konzeptionell verwandt mit dem Problem der kosmologischen Konstante \cite{Weinberg1989, Martin2012].
										
										Diese Konstanten werden nicht nur aus Bequemlichkeit auf Eins gesetzt; sie repräsentieren eine fundamentale theoretische Vereinheitlichung, die natürlich aus der Formulierung des T0-Modells hervorgeht und das Hierarchieproblem adressiert, das von 't Hooft \cite{tHooft1980} und Susskind \cite{Susskind1979} identifiziert wurde. Die resultierende Hierarchie von Skalen und abgeleiteten Konstanten offenbart die intrinsische Struktur der physikalischen Realität.
										
										Die Normierung der Feinstrukturkonstante ist entscheidend für den Elektromagnetismus:
										
										\begin{equation}
											\alpha_{\text{EM}} = \frac{e^2}{4 \pi \varepsilon_0 \hbar c} \approx \frac{1}{137{,}036}, \label{eq:fine_structure}
										\end{equation}
										
										Feynman nannte diese Konstante „eines der größten verdammten Mysterien der Physik“ \cite{Feynman1985}, während ihre potenzielle Variabilität umfassend untersucht wurde \cite{Webb2011, Rosenband2008}. Mit \(\hbar = c = \varepsilon_0 = 1\) in unserem Rahmen ergibt das Setzen von \(\alpha_{\text{EM}} = 1\):
										
										\begin{equation}
											e^2 = 4 \pi \implies e = \sqrt{4 \pi} \approx 3{,}544, \label{eq:charge_value}
										\end{equation}
										
										Dies macht die elektrische Ladung dimensionslos und vereinfacht elektromagnetische Gleichungen auf eine Weise, die an Diracs Hypothese großer Zahlen \cite{Dirac1937} und Ansätze von Weinberg \cite{Weinberg1983} erinnert. Alternativ, unter Verwendung des klassischen Elektronenradius \(r_e = e^2/(4 \pi \varepsilon_0 m_e c^2)\) und der Compton-Wellenlänge \(\lambda_C = h/(m_e c)\):
										
										\begin{equation}
											\alpha_{\text{EM}} = \frac{2 \pi r_e}{\lambda_C}, \label{eq:alpha_alt}
										\end{equation}
										
										Mit \(h = 2 \pi \hbar\) bestätigt dies die Standarddefinition und verknüpft Quanten- und elektromagnetische Skalen. Die Kopplung von \(\mu_0\) und \(\varepsilon_0\):
										
										\begin{equation}
											\mu_0 \varepsilon_0 = \frac{1}{c^2} = 1, \label{eq:em_coupling}
										\end{equation}
										
										Dies vereinheitlicht elektromagnetische Wechselwirkungen und macht Maxwells Gleichungen bemerkenswert einfach, wie in \hyperref[subsec:detailed_em_constants]{Abschnitt 11} gezeigt. Dieser Ansatz bietet eine neuartige Lösung für die langjährige Frage von Levy-Leblond und Provost nach der fundamentalen Bedeutung der Feinstrukturkonstante \cite{LevyLeblond1979}.
										
										\subsection{Ableitung von \(\beta_T = 1\)}
										\label{subsec:beta_derivation}
										
										Der T0-Parameter \(\beta_T\), der die Kopplung von \(T(x)\) regelt, wird durch eine rigorose Ableitung, die mit Standardmodell-Parametern verknüpft ist, auf 1 normiert:
										
										\begin{equation}
											\beta_T = \frac{\lambda_h^2 v^2}{16 \pi^3} \cdot \frac{1}{m_h^2} \cdot \frac{1}{\xi}, \label{eq:beta_derivation}
										\end{equation}
										
										wobei:
										\begin{itemize}
											\item \(\lambda_h \approx 0{,}13\): Higgs-Selbstkopplung.
											\item \(v \approx 246\) GeV: Higgs-Vakuum-Erwartungswert.
											\item \(m_h \approx 125\) GeV: Higgs-Masse.
											\item \(\xi = r_0/l_P\): Verhältnis der T0-Länge zur Planck-Länge.
										\end{itemize}
										
										Das Setzen von \(\beta_T = 1\):
										
										\begin{equation}
											\xi = \frac{\lambda_h^2 v^2}{16 \pi^3 m_h^2} \approx 1{,}33 \times 10^{-4}, \label{eq:xi_value}
										\end{equation}
										
										Dies ergibt \(r_0 \approx 1{,}33 \times 10^{-4} \cdot l_P\). Unter Verwendung von \(m_h^2 = 2 \lambda_h v^2\):
										
										\begin{equation}
											\xi = \frac{\lambda_h}{32 \pi^3} \approx 1{,}31 \times 10^{-4}, \label{eq:xi_alt}
										\end{equation}
										
										Die Konsistenz dieser Werte validiert die Ableitung. \(\beta_T = 1\) fungiert als Renormierungsfixpunkt:
										
										\begin{equation}
											\lim_{E \to 0} \beta_T(E) = 1, \label{eq:beta_limit}
										\end{equation}
										
										Der SI-Wert \(\beta_T \approx 0{,}008\) reflektiert Finite-Energie-Effekte und stärkt die Kohärenz des Modells \cite{pascher_beta_2025}.
										
										\subsection{Verbindung zu Higgs-Parametern}
										\label{subsec:higgs}
										
										Die T0-Länge \(r_0\) ist direkt mit Standardmodell-Parametern verknüpft:
										
										\begin{equation}
											r_0 = \xi \cdot l_P = \frac{\lambda_h^2 v^2}{16 \pi^3 m_h^2} \cdot l_P \approx 1{,}33 \times 10^{-4} \cdot l_P, \label{eq:r0_higgs}
										\end{equation}
										
										Mit \(m_h^2 = 2 \lambda_h v^2\):
										
										\begin{equation}
											\xi = \frac{\lambda_h}{32 \pi^3} \approx 1{,}31 \times 10^{-4}, \label{eq:xi_higgs}
										\end{equation}
										
										Diese Verbindung überbrückt Quantenfeldtheorie und emergente Gravitation und verstärkt die Kohärenz des Modells über Skalen hinweg \cite{pascher_higgs_2025}.
										
										\section{Quantisierte Längenskalen und ihre Implikationen}
										\label{sec:length_scales}
										
										\subsection{Hierarchie der Längenskalen und ihre quantisierten Werte}
										\label{subsec:detailed_length_scales}
										
										Die Längenskalen im T0-Modell folgen einer präzisen hierarchischen Struktur, mit Werten, die durch die fundamentalen Konstanten des Modells bestimmt werden. Tabelle \ref{tab:detailed_length_scales} fasst diese Skalen und ihre quantisierten Werte zusammen:
										
										\begin{table}[H]
											\centering
											\caption{Detaillierte Hierarchie der Längenskalen im T0-Modell mit ihren quantisierten Werten}
											\label{tab:detailed_length_scales}
											\small
											\setlength{\tabcolsep}{4pt}
											\resizebox{\columnwidth}{!}{
												\begin{tabular}{lccc}
													\toprule
													\textbf{Längenskala} & \textbf{Definition} & \textbf{Wert in $l_P$-Einheiten} & \textbf{SI-Wert (m)} \\
													\midrule
													Planck-Länge ($l_P$) & $\sqrt{\hbar G / c^3}$ & 1 & $1{,}616 \times 10^{-35}$ \\
													T0-Länge ($r_0$) & $\xi l_P$ & $1{,}33 \times 10^{-4}$ & $2{,}15 \times 10^{-39}$ \\
													Skala der starken Wechselwirkung & $\alpha_s \lambda_{C,h}$ & $\sim 10^{-19}$ & $\sim 10^{-54}$ \\
													Higgs-Compton-Wellenlänge ($\lambda_{C,h}$) & $\hbar / (m_h c)$ & $\sim 1{,}6 \times 10^{-20}$ & $\sim 2{,}6 \times 10^{-55}$ \\
													Protonradius & $\alpha_s / (2\pi) \lambda_{C,p}$ & $\sim 5{,}2 \times 10^{-20}$ & $\sim 8{,}4 \times 10^{-55}$ \\
													Elektronradius ($r_e$) & $\alpha_{\text{EM,SI}} / (2\pi) \lambda_{C,e}$ & $\sim 2{,}4 \times 10^{-23}$ & $\sim 3{,}9 \times 10^{-58}$ \\
													Elektron-Compton-Wellenlänge ($\lambda_{C,e}$) & $\hbar / (m_e c)$ & $\sim 2{,}1 \times 10^{-23}$ & $\sim 3{,}4 \times 10^{-58}$ \\
													Bohr-Radius ($a_0$) & $\lambda_{C,e} / \alpha_{\text{EM,SI}}$ & $\sim 2{,}9 \times 10^{-21}$ & $\sim 4{,}7 \times 10^{-56}$ \\
													DNA-Breite & $\lambda_{C,e} m_e / m_{\text{DNA}}$ & $\sim 1{,}2 \times 10^{-26}$ & $\sim 1{,}9 \times 10^{-61}$ \\
													Zelle & $\sim 10^7 \text{DNA}$ & $\sim 6{,}2 \times 10^{-30}$ & $\sim 1{,}0 \times 10^{-64}$ \\
													Mensch & $\sim 10^5 \text{Zelle}$ & $\sim 6{,}2 \times 10^{-35}$ & $\sim 1{,}0 \times 10^{-69}$ \\
													Erdradius & $(m_P / m_{\text{Erde}})^2 l_P$ & $\sim 3{,}9 \times 10^{-41}$ & $\sim 6{,}3 \times 10^{-76}$ \\
													Sonnenradius & $(m_P / m_{\text{Sonne}})^2 l_P$ & $\sim 4{,}3 \times 10^{-43}$ & $\sim 7{,}0 \times 10^{-78}$ \\
													Sonnensystem & $\alpha_G^{-1/2} \text{Sonne}$ & $\sim 6{,}2 \times 10^{-47}$ & $\sim 1{,}0 \times 10^{-81}$ \\
													Galaxie & $(m_P / m_{\text{Galaxie}})^2 l_P$ & $\sim 6{,}2 \times 10^{-56}$ & $\sim 1{,}0 \times 10^{-90}$ \\
													Cluster & $\sim 10^2 \text{Galaxie}$ & $\sim 6{,}2 \times 10^{-58}$ & $\sim 1{,}0 \times 10^{-92}$ \\
													Horizont ($d_H$) & $\sim 1 / H_0$ & $\sim 5{,}4 \times 10^{61}$ & $\sim 8{,}7 \times 10^{26}$ \\
													Kosmologische Korrelationslänge ($L_T$) & $\beta_T^{-1/4} \xi^{-1/2} l_P$ & $\sim 3{,}9 \times 10^{62}$ & $\sim 6{,}3 \times 10^{27}$ \\
													\bottomrule
												\end{tabular}
											}
										\end{table}
										
										Diese Quantisierung ergibt sich aus den hierarchischen Beziehungen zwischen den Konstanten des T0-Modells. Die Längenskalen sind nicht willkürlich, sondern folgen dem Quantisierungsgesetz:
										
										\begin{equation}
											L_n = l_P \times \prod_i \alpha_i^{n_i}, \label{eq:detailed_quantization}
										\end{equation}
										
										wobei \(\alpha_i \in \{\alpha_{\text{EM}}, \beta_T, \xi\}\) und \(n_i\) die entsprechenden Quantenzahlen sind. Diese Quantenzahlen entstehen aus den fundamentalen Symmetrien und Kopplungen des Modells.
										
										Die kosmologische Korrelationslänge \(L_T\) ist von besonderer Bedeutung, da sie direkt mit dem T0-Parameter \(\beta_T\) zusammenhängt:
										
										\begin{equation}
											\frac{L_T}{l_P} = \beta_T^{-1/4} \xi^{-1/2} \approx 3{,}9 \times 10^{62}, \label{eq:correlation_length}
										\end{equation}
										
										Diese Länge markiert den Horizont, bis zu dem \(T(x)\)-Korrelationen reichen, und ist eng mit der kosmologischen Konstante verknüpft. In SI-Einheiten beträgt \(L_T \approx 6{,}3 \times 10^{27}\) m, was mit der Skala des beobachtbaren Universums übereinstimmt. Die Beziehung zwischen \(\beta_T\) und der kosmologischen Korrelationslänge löst das Problem der kosmologischen Konstante durch einen natürlichen Mechanismus, ohne Feinabstimmung zu erfordern \cite{pascher_energiedynamik_2025}.
										
										\subsection{Quantisierung und verbotene Zonen}
										\label{subsec:quantization}
										
										Die quantisierte Natur der Längenskalen im T0-Modell schafft „verbotene Zonen“ – Regionen, die mehrere Größenordnungen umfassen, in denen stabile physikalische Strukturen fehlen. Diese Zonen entstehen aus der Quantisierungsregel und den spezifischen Werten der Konstanten:
										
										\begin{enumerate}
											\item Die erste große verbotene Zone erstreckt sich über etwa 19 Größenordnungen, zwischen \(r_0 \approx 1{,}33 \times 10^{-4} l_P\) und \(\lambda_{C,e} \approx 2{,}1 \times 10^{-23} l_P\). Dieser Abstand entspricht dem Massenverhältnis \(m_h/m_e \approx 2{,}45 \times 10^5\).
											\item Eine zweite verbotene Zone umfasst etwa 3 Größenordnungen, zwischen \(\lambda_{C,e} \approx 2{,}1 \times 10^{-23} l_P\) und \(a_0 \approx 2{,}9 \times 10^{-21} l_P\). Dieser Abstand entspricht \(1/\alpha_{\text{EM,SI}} \approx 137{,}036\).
										\end{enumerate}
										
										Diese verbotenen Zonen sind analog zu Energielücken in atomaren Systemen oder Bandlücken in der Festkörperphysik und repräsentieren Regionen, in denen stabile physikalische Strukturen aufgrund der zugrunde liegenden Quantenstruktur des T0-Modells nicht natürlich entstehen können \cite{pascher_higgs_2025}.
										
										\subsection{Biologische Anomalien in verbotenen Zonen}
										\label{subsec:bio_anomalies}
										
										Eine auffällige Eigenschaft des T0-Modells ist die Präsenz biologischer Strukturen in diesen „verbotenen Zonen“. Strukturen wie DNA (\(\sim 10^{-26} l_P\)), Proteine (\(\sim 10^{-27} l_P\)), Bakterien (\(\sim 10^{-29} l_P\)), Zellen (\(\sim 10^{-30} l_P\)) und Organismen (\(\sim 10^{-32}\) bis \(10^{-35} l_P\)) existieren in Regionen, in denen das Modell keine stabilen physikalischen Strukturen vorhersagt.
										
										Dieser scheinbare Widerspruch wird durch eine zentrale Erkenntnis gelöst: Biologische Systeme verfügen über einzigartige Stabilisierungsmechanismen, die in anorganischer Materie fehlen. Die modifizierte Feldgleichung:
										
										\begin{equation}
											\nabla^2 T(x)_{\text{bio}} \approx -\frac{\rho}{T(x)^2} + \delta_{\text{bio}}(x,t), \label{eq:bio_field_eq}
										\end{equation}
										
										Der Term \(\delta_{\text{bio}}\) berücksichtigt informationsbasierte, topologische und dynamische Stabilisierungsmechanismen, die Leben von unbelebter Materie unterscheiden und an Konzepte von Prigogines dissipativen Strukturen \cite{Prigogine1980} und Kauffmans Arbeiten zu komplexen Systemen \cite{Kauffman1993} anknüpfen. Diese Mechanismen umfassen:
										
										\begin{enumerate}
											\item \textbf{Informationsbasierte Regulation}: DNA-kodierte Prozesse, die strukturelle Integrität aufrechterhalten, mit bemerkenswerter Zuverlässigkeit trotz thermischem Rauschen, wie von Bennett \cite{Bennett1982} und Landauer \cite{Landauer1961} analysiert.
											\item \textbf{Topologische Stabilität}: Komplexe molekulare Faltung, die stabile Konfigurationen in ansonsten instabilen Bereichen schafft, demonstriert in Protein-Faltungsstudien von Anfinsen \cite{Anfinsen1973} und Levinthal \cite{Levinthal1968].
												\item \textbf{Dynamisches Gleichgewicht}: Aktive metabolische Prozesse, die Strukturen kontinuierlich gegen Entropie aufbauen und ein Gleichgewicht fern vom thermodynamischen Gleichgewicht aufrechterhalten, wie von Harold \cite{Harold2001} beschrieben.
											\end{enumerate}
											
											Dies bietet eine neuartige physikalische Grundlage für die Einzigartigkeit biologischer Systeme – sie repräsentieren die einzigen stabilen komplexen Strukturen in diesen verbotenen Zonen, was möglicherweise erklärt, warum Lebensformen spezifische Größenskalen haben, die nach rein physikalischen Prinzipien instabil wären. Dies knüpft an grundlegende Theorien der biologischen Organisation an, die von Schrödinger \cite{Schrodinger1944}, Fristons Prinzip der freien Energie \cite{Friston2010} und Englands dissipationsgetriebener Anpassung \cite{England2013} vorgeschlagen wurden.
											
											\section{Feldgleichungen im vereinheitlichten Rahmen}
											\label{sec:field_equations}
											
											\subsection{Detaillierte elektromagnetische Konstanten und ihre Ableitungen}
											\label{subsec:detailed_em_constants}
											
											Die elektromagnetischen Konstanten im T0-Modell werden direkt aus der Normierung \(\alpha_{\text{EM}} = 1\) und den Grundprinzipien des Modells abgeleitet. Tabelle \ref{tab:detailed_em_constants} fasst diese Konstanten, ihre natürlichen Werte und SI-Äquivalente zusammen:
											
											\begin{table}[H]
												\centering
												\caption{Detaillierte elektromagnetische Konstanten im T0-Modell mit ihren Ableitungen}
												\label{tab:detailed_em_constants}
												\small
												\setlength{\tabcolsep}{4pt}
												\resizebox{\columnwidth}{!}{
													\begin{tabular}{lccc}
														\toprule
														\textbf{Konstante} & \textbf{Definition} & \textbf{T0-Modell-Wert} & \textbf{SI-Wert} \\
														\midrule
														Vakuumpermeabilität ($\mu_0$) & $1/(\varepsilon_0 c^2)$ & 1 & $4\pi \times 10^{-7}$ H/m \\
														Vakuumpermittivität ($\varepsilon_0$) & $1/(\mu_0 c^2)$ & 1 & $8{,}854 \times 10^{-12}$ F/m \\
														Vakuumimpedanz ($Z_0$) & $\sqrt{\mu_0/\varepsilon_0}$ & 1 & 376{,}73 $\Omega$ \\
														Elementarladung ($e$) & $\sqrt{4\pi \varepsilon_0 \hbar c}$ & $\sqrt{4\pi} \approx 3{,}544$ & $1{,}602 \times 10^{-19}$ C \\
														Feinstrukturkonstante ($\alpha_{\text{EM}}$) & $e^2/(4\pi \varepsilon_0 \hbar c)$ & 1 & $1/137{,}036$ \\
														Klassischer Elektronenradius ($r_e$) & $e^2/(4\pi \varepsilon_0 m_e c^2)$ & $1/(2\pi m_e)$ & $2{,}818 \times 10^{-15}$ m \\
														Compton-Wellenlänge ($\lambda_C$) & $h/(m_e c)$ & $2\pi/m_e$ & $2{,}426 \times 10^{-12}$ m \\
														Bohr-Radius ($a_0$) & $\hbar/(m_e c \alpha_{\text{EM,SI}})$ & $1/(m_e \alpha_{\text{EM,SI}})$ & $5{,}292 \times 10^{-11}$ m \\
														Bohr-Magneton ($\mu_B$) & $e \hbar/(2 m_e)$ & $\sqrt{\pi}/m_e$ & $9{,}274 \times 10^{-24}$ J/T \\
														Josephson-Konstante ($K_J$) & $2e/h$ & $\sqrt{\pi}/\pi$ & $4{,}836 \times 10^{14}$ Hz/V \\
														von-Klitzing-Konstante ($R_K$) & $h/e^2$ & $1/2$ & $2{,}581 \times 10^4$ $\Omega$ \\
														\bottomrule
													\end{tabular}
												}
											\end{table}
											
											Die Ableitung dieser Konstanten basiert auf der fundamentalen Beziehung \(\alpha_{\text{EM}} = 1\), die direkt zur Elementarladung \(e = \sqrt{4\pi}\) führt. Mit \(\hbar = c = \varepsilon_0 = \mu_0 = 1\) werden alle elektromagnetischen Beziehungen dramatisch vereinfacht. Maxwells Gleichungen nehmen eine besonders elegante Form an \cite{Feynman1985}:
											
											\begin{align}
												\nabla \cdot \vec{E} &= \rho, \label{eq:detailed_gauss} \\
												\nabla \times \vec{B} - \frac{\partial \vec{E}}{\partial t} &= \vec{j}, \label{eq:detailed_ampere} \\
												\nabla \cdot \vec{B} &= 0, \label{eq:detailed_gauss_mag} \\
												\nabla \times \vec{E} + \frac{\partial \vec{B}}{\partial t} &= 0. \label{eq:detailed_faraday}
											\end{align}
											
											Die Konversion dieser natürlichen Einheiten in SI-Einheiten erfolgt durch die Basisbeziehungen:
											
											\begin{align}
												\mu_0^{\text{SI}} &= 4\pi \times 10^{-7} \, \text{H/m} = 1 \, \text{(T0-Einheiten)}, \label{eq:mu0_conversion} \\
												\varepsilon_0^{\text{SI}} &= 8{,}854 \times 10^{-12} \, \text{F/m} = 1 \, \text{(T0-Einheiten)}, \label{eq:epsilon0_conversion} \\
												e^{\text{SI}} &= 1{,}602 \times 10^{-19} \, \text{C} = \sqrt{4\pi} \, \text{(T0-Einheiten)}. \label{eq:e_conversion}
											\end{align}
											
											Von besonderer theoretischer Bedeutung ist, dass die von-Klitzing-Konstante \(R_K\) im T0-Modell exakt 1 beträgt, was die fundamentale Einheit des Widerstands im Quantenregime unterstreicht. Diese Eigenschaft kann experimentell über den Quanten-Hall-Effekt getestet werden \cite{pascher_alpha_2025} und bietet eine direkte Verbindung zwischen makroskopischen Messungen und den fundamentalen Einheiten des T0-Modells.
											
											Bemerkenswert ist auch, dass das Verhältnis zwischen dem klassischen Elektronenradius \(r_e\) und der Compton-Wellenlänge \(\lambda_C\) direkt die Feinstrukturkonstante ergibt:
											
											\begin{equation}
												\alpha_{\text{EM}} = \frac{2\pi r_e}{\lambda_C}, \label{eq:detailed_alpha_relation}
											\end{equation}
											
											Diese Beziehung verdeutlicht die geometrische Interpretation der Feinstrukturkonstante im T0-Modell und bietet eine direkte Möglichkeit, \(\alpha_{\text{EM}} = 1\) experimentell zu verifizieren \cite{Webb2011].
												
												\subsection{Umfassende Behandlung der fundamentalen Kräfte}
												\label{subsec:detailed_forces}
												
												Das T0-Modell bietet einen vereinheitlichten Rahmen für alle fundamentalen Kräfte der Natur, wobei die Gravitationskraft als Eigenschaft des intrinsischen Zeitfeldes \(T(x)\) emergiert. Tabelle \ref{tab:detailed_forces} fasst die vier fundamentalen Kräfte mit ihren Kopplungskonstanten, Reichweiten und Beziehungen im T0-Modell zusammen:
												
												\begin{table}[H]
													\centering
													\caption{Fundamentale Kräfte im T0-Modell mit ihren Kopplungskonstanten}
													\label{tab:detailed_forces}
													\small
													\setlength{\tabcolsep}{4pt}
													\resizebox{\columnwidth}{!}{
														\begin{tabular}{lcccc}
															\toprule
															\textbf{Kraft} & \textbf{Dimensionslose Kopplung} & \textbf{T0-Wert} & \textbf{SI-Wert} & \textbf{Reichweite} \\
															\midrule
															Elektromagnetische Kraft & $\alpha_{\text{EM}} = \frac{e^2}{4\pi \varepsilon_0 \hbar c}$ & 1 & $1/137{,}036$ & $\infty$ \\
															Starke Kernkraft & $\alpha_s = \frac{g_s^2}{4\pi \hbar c}$ & $\sim 0{,}118$ (bei $Q^2 = M_Z^2$) & $\sim 0{,}118$ & $\sim 10^{-15}$ m \\
															Schwache Kernkraft & $\alpha_W = \frac{g_W^2}{4\pi \hbar c}$ & $\sim 1/30$ & $\sim 1/30$ & $\sim 10^{-18}$ m \\
															Gravitation & $\alpha_G = \frac{G m^2}{\hbar c}$ & $\frac{m^2}{m_P^2}$ & $\sim 10^{-38}$ (für Proton) & $\infty$ \\
															\bottomrule
														\end{tabular}
													}
												\end{table}
												
												Die Normierung \(\alpha_{\text{EM}} = 1\) im T0-Modell geht über eine bloße Konvention hinaus; sie weist auf eine tiefere Beziehung zwischen elektromagnetischen und Quantenphänomenen hin \cite{Sommerfeld1916, Aoyama2018]. Die gravitative Kopplungskonstante hängt von der Teilchenmasse ab:
													
													\begin{equation}
														\alpha_G = \frac{G m^2}{\hbar c} = \frac{m^2}{m_P^2}, \label{eq:alpha_G}
													\end{equation}
													
													Diese Beziehung erklärt die scheinbare Schwäche der Gravitation auf Teilchenebene und ihre Dominanz auf astronomischen Skalen \cite{pascher_emergente_2025}. Die laufenden Kopplungskonstanten der Eichtheorien im T0-Modell folgen Renormierungsgruppenflusskurven, die bei extrem hohen Energien (\(E \to \infty\)) konvergieren, während bei niedrigen Energien (\(E \to 0\)) die Beziehung gilt \cite{Weinberg1989}:
													
													\begin{equation}
														\lim_{E \to 0} \beta_T(E) = 1, \label{eq:beta_IR_limit}
													\end{equation}
													
													Die Kraftgesetze werden im T0-Modell stark vereinfacht. Für die elektromagnetische Kraft (Coulombsches Gesetz) \cite{Feynman1985}:
													
													\begin{equation}
														\vec{F}_C = \frac{1}{4\pi \varepsilon_0} \frac{q_1 q_2}{r^2} \hat{r} \quad \to \quad \vec{F}_C = \frac{q_1 q_2}{4\pi r^2} \hat{r}, \label{eq:coulomb_t0}
													\end{equation}
													
													Für die Gravitation (emergent aus \(T(x)\)) \cite{pascher_emergente_2025}:
													
													\begin{equation}
														\vec{F}_G = -\frac{G m_1 m_2}{r^2} \hat{r} \quad \to \quad \vec{F}_G = -\frac{m_1 m_2}{r^2} \hat{r}, \label{eq:gravity_t0}
													\end{equation}
													
													Mit dem modifizierten Gravitationspotential:
													
													\begin{equation}
														\Phi(r) = -\frac{r_g}{r} + \kappa r + \kappa r, \label{eq:detailed_mod_potential}
													\end{equation}
													
													Die Gesamtkraft unter Berücksichtigung des kosmologischen Terms \(\kappa\):
													
													\begin{equation}
														\vec{F}_{\text{total}} = -\frac{m_1 m_2}{r^2} \hat{r} + \kappa m_2 \hat{r}, \label{eq:total_force}
													\end{equation}
													
													Diese vereinheitlichte Behandlung der fundamentalen Kräfte bietet einen neuen Ansatz zur Vereinheitlichung der Physik, wobei die Gravitation nicht als fundamentale Kraft verstanden wird, sondern als emergente Eigenschaft des intrinsischen Zeitfeldes, während die elektromagnetische Kraft durch die Normierung \(\alpha_{\text{EM}} = 1\) optimal in den Rahmen integriert ist. Die starken und schwachen Kernkräfte behalten ihre Kopplungswerte, werden jedoch durch die vereinfachte Dimensionsanalyse des T0-Modells in das Gesamtbild eingebunden \cite{pascher_emergente_2025}.
													
													\subsection{Thermodynamische und Quantenkonstanten auf Ebene 3}
													\label{subsec:level3_thermo_quantum}
													
													Die thermodynamischen und Quantenkonstanten im T0-Modell bilden eine dritte Ebene der hierarchischen Ableitung, basierend auf den primären und sekundären Konstanten (\(\hbar = c = G = k_B = \alpha_{\text{EM}} = \alpha_W = \beta_T = 1\)). Tabelle \ref{tab:level3_constants} fasst diese zusammen:
													
													\begin{table}[H]
														\centering
														\caption{Thermodynamische und Quantenkonstanten auf Ebene 3 im T0-Modell}
														\label{tab:level3_constants}
														\small
														\setlength{\tabcolsep}{4pt}
														\resizebox{\columnwidth}{!}{
															\begin{tabular}{lccc}
																\toprule
																\textbf{Konstante} & \textbf{Definition} & \textbf{T0-Wert} & \textbf{SI-Wert} \\
																\midrule
																Wien’sche Verschiebungskonstante ($b$) & $\lambda_{\text{max}} T$ & $2\pi$ & $2{,}898 \times 10^{-3}$ m$\cdot$K \\
																Stefan-Boltzmann-Konstante ($\sigma$) & $\frac{\pi^2 k_B^4}{60 \hbar^3 c^2}$ & $\frac{\pi^2}{60}$ & $5{,}670 \times 10^{-8}$ W/(m$^2 \cdot$K$^4$) \\
																Plancksches Strahlungsgesetz & $\rho(\omega,T) = \frac{\hbar \omega^3}{2\pi^2 c^3} \frac{1}{e^{\hbar \omega / k_B T} - 1}$ & $\frac{\omega^3}{2\pi^2} \frac{1}{e^{\omega / T} - 1}$ & -- \\
																Schwarzkörperspektrum (Maximum) & $\omega_{\text{max}} = \alpha_W T$ & $T$ & $5{,}879 \times 10^{10}$ Hz/K \\
																Sommerfeld-Konstante & $\gamma = \frac{\pi^2 k_B^2}{3} D(E_F)$ & $\frac{\pi^2}{3} D(E_F)$ & -- \\
																Quantenoszillatorenergien & $E_n = \hbar \omega (n + \frac{1}{2})$ & $\omega (n + \frac{1}{2})$ & -- \\
																Dekohärenzrate & $\Gamma_{\text{dec}} = \Gamma_0 \frac{m c^2}{\hbar}$ & $\Gamma_0 m$ & -- \\
																Dualitätsrelation & $\lambda = \frac{h}{p}$ & $\frac{2\pi}{p}$ & -- \\
																Unschärferelation & $\Delta x \Delta p \geq \frac{\hbar}{2}$ & $\Delta x \Delta p \geq \frac{1}{2}$ & -- \\
																Durchschnittliche Energie & $\bar{E} = \frac{3}{2} k_B T$ & $\frac{3}{2} T$ & -- \\
																Zustandssumme (klass. Teilchen) & $Z = \frac{V}{N!} \left( \frac{2\pi m k_B T}{h^2} \right)^{3N/2}$ & $\frac{V}{N!} \left( \frac{m T}{2\pi} \right)^{3N/2}$ & -- \\
																Bose-Einstein-Statistik & $\bar{n}_i = \frac{1}{e^{(E_i - \mu)/k_B T} - 1}$ & $\frac{1}{e^{(E_i - \mu)/T} - 1}$ & -- \\
																Fermi-Dirac-Statistik & $\bar{n}_i = \frac{1}{e^{(E_i - \mu)/k_B T} + 1}$ & $\frac{1}{e^{(E_i - \mu)/T} + 1}$ & -- \\
																\bottomrule
															\end{tabular}
														}
													\end{table}
													
													Die Normierung \(\alpha_W = 1\) vereinfacht thermodynamische Beziehungen erheblich, indem sie Temperatur direkt mit Frequenz gleichsetzt \cite{Wien1896, Planck1901]:
														
														\begin{equation}
															\omega_{\text{max}} = T, \label{eq:wien_simplified}
														\end{equation}
														
														Diese Beziehung kann durch präzise Schwarzkörperstrahlungsmessungen experimentell verifiziert werden \cite{pascher_alpha_2025}. Für die Quantentheorie bedeutet die Normierung \(\hbar = 1\), dass die Unschärferelation die einfachste mögliche Form annimmt \cite{Heisenberg1925]:
															
															\begin{equation}
																\Delta x \Delta p \geq \frac{1}{2}, \label{eq:uncertainty_simplified}
															\end{equation}
															
															Thermodynamische Temperatur und Energie werden im T0-Modell äquivalent (\(T = E\)), was die Interpretation der Temperatur als durchschnittliche Teilchenenergie formalisiert. Für ein ideales Gas gilt daher \cite{Boltzmann1872]:
																
																\begin{equation}
																	\bar{E} = \frac{3}{2} T, \label{eq:average_energy}
																\end{equation}
																
																Diese Vereinfachungen reduzieren die Komplexität thermodynamischer und quantenmechanischer Berechnungen erheblich und offenbaren die zugrunde liegende Einheit dieser scheinbar unterschiedlichen physikalischen Domänen. Entropie wird im T0-Modell zu einer dimensionslosen Größe, was ihre informationstheoretische Interpretation (\(S = k_B \ln \Omega\)) als reines Zählmaß bestätigt \cite{pascher_alpha_2025}.
																
																\subsection{Modifizierte Quantenmechanik und quantisiertes Zeitfeld}
																\label{subsec:quantum}
																
																Das T0-Modell modifiziert die Quantenmechanik durch \(T(x)\). Die standardmäßige Schrödinger-Gleichung:
																
																\begin{equation}
																	i \hbar \frac{\partial}{\partial t} \Psi = \hat{H} \Psi, \label{eq:std_schrodinger}
																\end{equation}
																
																wird zu:
																
																\begin{equation}
																	i \hbar T(x) \frac{\partial}{\partial t} \Psi + i \hbar \Psi \frac{\partial T(x)}{\partial t} = \hat{H} \Psi, \label{eq:mod_schrodinger}
																\end{equation}
																
																Dies führt eine massenabhängige Evolution ein, die mehrere Phänomene erklärt:
																
																\begin{itemize}
																	\item \textbf{Dekohärenzrate}: \(\Gamma_{\text{dec}} = \Gamma_0 \frac{m c^2}{\hbar}\), sagt schnellere Dekohärenz für schwerere Teilchen voraus.
																	\item \textbf{Welle-Teilchen-Dualität}: \(\lambda = \frac{1}{p}\) (in natürlichen Einheiten), verknüpft Wellenlänge direkt mit Impuls.
																	\item \textbf{Unschärfeprinzip}: \(\Delta E \Delta t \geq \frac{1}{2}\), vereinfacht in natürlichen Einheiten.
																\end{itemize}
																
																Aufbauend auf dieser klassischen Behandlung wurde \(T(x)\) vollständig quantisiert mit einem umfassenden Quantenfeldtheorie-Rahmen \cite{pascher_qft_2025}. Die klassische Lagrange-Dichte:
																
																\begin{equation}
																	\mathcal{L}_{\text{intrinsic}} = \frac{1}{2} \partial_{\mu} T(x) \partial^{\mu} T(x) - \frac{1}{2} T(x)^2, \label{eq:lagrangian_T}
																\end{equation}
																
																wurde durch kanonische Quantisierung, Pfadintegralformulierung, Renormierung und Unitätsanalyse erweitert. Diese Quantisierung bestätigt, dass \(\beta_T = 1\) als Renormierungsgruppenfixpunkt im Infrarot-Limit emergiert:
																
																\begin{equation}
																	\lim_{E \to 0} \beta_T(E) = 1, \label{eq:beta_fixed_point}
																\end{equation}
																
																Diese Modifikationen lösen langjährige Probleme in der Quantenmechanik, einschließlich des Messproblems und der Nichtlokalität, indem sie eine massenabhängige temporale Evolution einführen, während sie Konsistenz mit etablierten Prinzipien der Quantenfeldtheorie bewahren \cite{pascher_quantum_2025].
																	
																	\subsection{Emergente Gravitation über die Einstein-Hilbert-Wirkung}
																	\label{subsec:gravitation}
																	
																	Das T0-Modell interpretiert die Gravitation durch die Einstein-Hilbert-Wirkung neu:
																	
																	\begin{equation}
																		S_{\text{EH}} = \frac{1}{16 \pi} \int (R - 2 \kappa) \sqrt{-g} \, d^4 x, \label{eq:einstein_hilbert}
																	\end{equation}
																	
																	Dieser Ansatz steht im Einklang mit grundlegenden Arbeiten von Hilbert \cite{Hilbert1924}, führt jedoch Modifikationen ein, die denen in \(f(R)\)-Gravitationstheorien ähneln \cite{Sotiriou2010, DeFelice2010}. Das modifizierte Potential:
																	
																	\begin{equation}
																		\Phi(r) = -\frac{r_g}{r} + \kappa r + \kappa r, \label{eq:mod_potential}
																	\end{equation}
																	
																	mit \(\kappa \approx 4{,}8 \times 10^{-11}\) m/s², erklärt dunkle Energie natürlich, verknüpft mit \(\Lambda_{\text{eff}} = \kappa\), und adressiert das Problem der kosmologischen Konstante, das von Weinberg identifiziert wurde \cite{Weinberg1989}. Gravitation emergiert aus:
																	
																	\begin{equation}
																		\Phi(\vec{x}) = -\ln\left(\frac{T(x)}{T_0}\right), \label{eq:phi_from_t}
																	\end{equation}
																	
																	Die statische Feldgleichung:
																	
																	\begin{equation}
																		\nabla^2 T(x) \approx -\frac{\rho}{T(x)^2}, \label{eq:static_field}
																	\end{equation}
																	
																	ergibt die Gravitationskraft:
																	
																	\begin{equation}
																		\vec{F} = -\frac{\nabla T(x)}{T(x)}, \label{eq:grav_force}
																	\end{equation}
																	
																	Diese Formulierung reproduziert Newtons Gesetz ohne Raumzeitkrümmung, während sie Kompatibilität mit relativistischen Beobachtungen aufrechterhält, ähnlich wie Verlindes entropische Gravitation \cite{Verlinde2011} und Padmanabhans emergente Gravitation \cite{Padmanabhan2012}. Dies adressiert beobachtende Herausforderungen, beschrieben von McGaugh \cite{McGaugh2011} und Kroupa \cite{Kroupa2012}, ohne dunkle Materie zu erfordern, während es Konsistenz mit Präzisionstests der Allgemeinen Relativitätstheorie bewahrt \cite{Will2014}.
																	
																	Wichtig ist, dass diese beiden Ansätze – die Einstein-Hilbert-Wirkung und die direkte Ableitung aus \(T(x)\) – nicht widersprüchlich, sondern komplementäre Perspektiven desselben physikalischen Prinzips sind, die an Bohrs Komplementaritätsprinzip erinnern \cite{Bohr1928}. Die geometrische Beschreibung (kompatibel mit Relativität) und der fundamentalere \(T(x)\)-Mechanismus liefern im schwachen Feldlimit mathematisch äquivalente Ergebnisse, was die Kohärenz des T0-Modells über Skalen hinweg unterstreicht und möglicherweise die Kluft zwischen Quanten- und Gravitationsphysik überbrückt, die seit den Arbeiten von Hawking \cite{Hawking1975} und Penrose \cite{Penrose1965} Theoretiker herausfordert.
																	
																	\section{Einheitenkonversionen und praktische Anwendungen}
																	\label{sec:conversions}
																	
																	\subsection{Planck-Druck, Kraft und andere abgeleitete Größen}
																	\label{subsec:planck_derived}
																	
																	Die Planck-Einheiten und andere abgeleitete Größen entstehen systematisch aus der T0-Normierung \(\hbar = c = G = 1\). Diese Einheiten spielen eine fundamentale Rolle als natürliche Skalen für physikalische Phänomene und sind vollständig in den energiebasieren Rahmen des T0-Modells integriert. Tabelle \ref{tab:planck_derived} fasst diese abgeleiteten Größen zusammen:
																	
																	\begin{table}[H]
																		\centering
																		\caption{Planck- und andere abgeleitete Größen im T0-Modell}
																		\label{tab:planck_derived}
																		\small
																		\setlength{\tabcolsep}{4pt}
																		\resizebox{\columnwidth}{!}{
																			\begin{tabular}{lccc}
																				\toprule
																				\textbf{Größe} & \textbf{Definition in SI} & \textbf{T0-Wert} & \textbf{SI-Wert} \\
																				\midrule
																				Planck-Länge ($l_P$) & $\sqrt{\hbar G / c^3}$ & 1 & $1{,}616 \times 10^{-35}$ m \\
																				Planck-Zeit ($t_P$) & $\sqrt{\hbar G / c^5}$ & 1 & $5{,}391 \times 10^{-44}$ s \\
																				Planck-Masse ($m_P$) & $\sqrt{\hbar c / G}$ & 1 & $2{,}176 \times 10^{-8}$ kg \\
																				Planck-Energie ($E_P$) & $\sqrt{\hbar c^5 / G}$ & 1 & $1{,}956 \times 10^9$ J \\
																				Planck-Temperatur ($T_P$) & $\sqrt{\hbar c^5 / (G k_B^2)}$ & 1 & $1{,}417 \times 10^{32}$ K \\
																				Planck-Druck ($p_P$) & $c^7 / (\hbar G^2)$ & 1 & $4{,}633 \times 10^{113}$ Pa \\
																				Planck-Kraft ($F_P$) & $c^4 / G$ & 1 & $1{,}210 \times 10^{44}$ N \\
																				Planck-Dichte ($\rho_P$) & $c^5 / (\hbar G^2)$ & 1 & $5{,}155 \times 10^{96}$ kg/m$^3$ \\
																				Planck-Beschleunigung ($a_P$) & $c^2 / l_P$ & 1 & $5{,}575 \times 10^{51}$ m/s$^2$ \\
																				Planck-Leistung ($P_P$) & $c^5 / G$ & 1 & $3{,}629 \times 10^{52}$ W \\
																				Planck-Strom ($I_P$) & $\sqrt{4\pi \varepsilon_0 c^6 / G}$ & $\sqrt{4\pi}$ & $3{,}479 \times 10^{25}$ A \\
																				Planck-Spannung ($U_P$) & $\sqrt{c^4 / (4\pi \varepsilon_0 G)}$ & $1 / \sqrt{4\pi}$ & $1{,}043 \times 10^{27}$ V \\
																				Planck-Fläche ($A_P$) & $l_P^2$ & 1 & $2{,}612 \times 10^{-70}$ m$^2$ \\
																				Planck-Volumen ($V_P$) & $l_P^3$ & 1 & $4{,}224 \times 10^{-105}$ m$^3$ \\
																				\bottomrule
																			\end{tabular}
																		}
																	\end{table}
																	
																	Im T0-Modell sind alle diese Planck-Größen auf einen Wert von 1 normiert (mit Ausnahme elektromagnetischer Größen, die noch den Faktor \(\sqrt{4\pi}\) enthalten). Diese Normierung hebt die fundamentale Natur dieser Größen als natürliche Skalen für physikalische Phänomene hervor.
																	
																	Der Planck-Druck \(p_P = 1\) repräsentiert den maximal möglichen Druck in der Physik und ist direkt mit der Vakuumenergie verknüpft:
																	
																	\begin{equation}
																		p_P = \frac{c^7}{\hbar G^2} = \frac{E_P}{V_P} = \rho_P c^2, \label{eq:planck_pressure}
																	\end{equation}
																	
																	Die Planck-Kraft \(F_P = 1\) repräsentiert die größte mögliche Kraft und ist direkt mit der Struktur der Raumzeit verbunden:
																	
																	\begin{equation}
																		F_P = \frac{c^4}{G} = \frac{E_P}{l_P} = m_P a_P, \label{eq:planck_force}
																	\end{equation}
																	
																	Diese Kraft ergibt sich als natürliche obere Grenze aus dem Zusammenspiel von Quantenmechanik und Gravitation und ist eng mit dem holographischen Prinzip und der Bekenstein-Hawking-Entropie verknüpft.
																	
																	Bemerkenswert ist auch die Beziehung zwischen den abgeleiteten Größen und der T0-Länge \(r_0 = \xi l_P\):
																	
																	\begin{equation}
																		p(r_0) = \xi^{-2} p_P \approx 5{,}65 \times 10^7 p_P, \label{eq:r0_pressure}
																	\end{equation}
																	
																	\begin{equation}
																		F(r_0) = \xi F_P \approx 1{,}33 \times 10^{-4} F_P, \label{eq:r0_force}
																	\end{equation}
																	
																	Diese Skalierungsbeziehungen zeigen, wie physikalische Größen systematisch zwischen verschiedenen hierarchischen Ebenen im T0-Modell verbunden sind, und ermöglichen präzise Vorhersagen für Messungen an der Grenze zwischen Quantenmechanik und Gravitation \cite{pascher_emergente_2025}.
																	
																	\subsection{Umfassende SI-Konversionen und praktische Anwendungen}
																	\label{subsec:detailed_conversions}
																	
																	Die Konversion zwischen dem T0-Einheitensystem und SI-Einheiten ist entscheidend für die praktische Anwendung und experimentelle Verifikation des Modells. Tabelle \ref{tab:detailed_conversions} bietet eine umfassende Übersicht dieser Konversionsfaktoren mit hoher Präzision:
																	
																	\begin{table}[H]
																		\centering
																		\caption{Vollständige Konversionstabelle zwischen T0-Einheiten und SI-Einheiten}
																		\label{tab:detailed_conversions}
																		\small
																		\setlength{\tabcolsep}{4pt}
																		\resizebox{\columnwidth}{!}{
																			\begin{tabular}{lcccc}
																				\toprule
																				\textbf{Physikalische Größe} & \textbf{SI-Einheit} & \textbf{T0-Dimension} & \textbf{Konversionsfaktor} & \textbf{Genauigkeit} \\
																				\midrule
																				Länge & m & $[E^{-1}]$ & $1 \, \text{m} = 5{,}068 \times 10^6 \, \text{GeV}^{-1}$ & $< 10^{-7}$ \\
																				Zeit & s & $[E^{-1}]$ & $1 \, \text{s} = 1{,}519 \times 10^{24} \, \text{GeV}^{-1}$ & $< 10^{-8}$ \\
																				Masse & kg & $[E]$ & $1 \, \text{kg} = 5{,}610 \times 10^{26} \, \text{GeV}$ & $< 10^{-7}$ \\
																				Energie & J & $[E]$ & $1 \, \text{J} = 6{,}242 \times 10^{9} \, \text{GeV}$ & $< 10^{-8}$ \\
																				Temperatur & K & $[E]$ & $1 \, \text{K} = 8{,}617 \times 10^{-14} \, \text{GeV}$ & $< 10^{-6}$ \\
																				Elektrische Ladung & C & $[1]$ & $1 \, \text{C} = 6{,}242 \times 10^{18}/\sqrt{4\pi}$ & $< 10^{-8}$ \\
																				Magnetfeld & T & $[E^2]$ & $1 \, \text{T} = 1{,}954 \times 10^{-16} \, \text{GeV}^2$ & $< 10^{-7}$ \\
																				Kraft & N & $[E^2]$ & $1 \, \text{N} = 3{,}166 \times 10^{16} \, \text{GeV}^2$ & $< 10^{-7}$ \\
																				Druck & Pa & $[E^4]$ & $1 \, \text{Pa} = 6{,}242 \times 10^9 \, \text{GeV}^4$ & $< 10^{-7}$ \\
																				Dichte & kg/m$^3$ & $[E^4]$ & $1 \, \text{kg/m}^3 = 2{,}178 \times 10^{-17} \, \text{GeV}^4$ & $< 10^{-6}$ \\
																				Wirkungsquantum & J$\cdot$s & $[1]$ & $1 \, \text{J$\cdot$s} = 9{,}487 \times 10^{33}$ & $< 10^{-8}$ \\
																				Gravitationskonstante & m$^3$/kg$\cdot$s$^2$ & $[E^{-2}]$ & $1 \, \text{m}^3/\text{kg$\cdot$s}^2 = 2{,}996 \times 10^{-66} \, \text{GeV}^{-2}$ & $< 10^{-6}$ \\
																				Planck-Konstante & eV$\cdot$s & $[1]$ & $1 \, \text{eV$\cdot$s} = 9{,}487 \times 10^{33}$ & $< 10^{-8}$ \\
																				Boltzmann-Konstante & J/K & $[1]$ & $1 \, \text{J/K} = 7{,}243 \times 10^{22}$ & $< 10^{-6}$ \\
																				\bottomrule
																			\end{tabular}
																		}
																	\end{table}
																	
																	Für praktische Anwendungen sind bestimmte Konversionen besonders wichtig \cite{pascher_alpha_2025}:
																	
																	\begin{align}
																		1 \, \text{GeV}^{-1} &= 1{,}973 \times 10^{-16} \, \text{m}, \label{eq:gev_to_m} \\
																		1 \, \text{eV} &= 1{,}602 \times 10^{-19} \, \text{J}, \label{eq:ev_to_j} \\
																		1 \, \text{eV} &= 11{,}605 \, \text{K}, \label{eq:ev_to_k} \\
																		m_p &= 0{,}938 \, \text{GeV}, \label{eq:proton_mass} \\
																		m_e &= 0{,}511 \, \text{MeV}. \label{eq:electron_mass}
																	\end{align}
																	
																	Die Konversion dimensionsloser Konstanten folgt einem besonderen Muster \cite{pascher_beta_2025}:
																	
																	\begin{align}
																		\alpha_{\text{EM}}^{\text{SI}} &= 1/137{,}036 \approx \xi^{0{,}507}, \label{eq:alpha_em_si} \\
																		\beta_T^{\text{SI}} &= 0{,}008 \approx \xi^{1{,}143}. \label{eq:beta_t_si}
																	\end{align}
																	
																	Diese Beziehungen zeigen, dass die SI-Werte dimensionsloser Konstanten systematisch mit dem fundamentalen Skalenverhältnis \(\xi = r_0/l_P \approx 1{,}33 \times 10^{-4}\) zusammenhängen. Für experimentelle Tests sind folgende Beziehungen relevant \cite{pascher_alpha_2025}:
																	
																	\begin{align}
																		R_{\infty} &= 0{,}256 \, \text{MeV}, \label{eq:rydberg} \\
																		\kappa &= 4{,}8 \times 10^{-11} \, \text{m/s}^2, \label{eq:kappa} \\
																		\frac{L_T}{l_P} &= 3{,}9 \times 10^{62}. \label{eq:lt_to_lp}
																	\end{align}
																	
																	Diese Konversionen ermöglichen präzise Vorhersagen für die experimentelle Verifikation in der Quantenelektrodynamik, Atomspektroskopie und Kosmologie.
																	
																	\section{Experimentelle Tests und Vorhersagen}
																	\label{sec:outlook}
																	
																	\subsection{Vorhersagen der Teilchenphysik}
																	\label{subsec:particle_predictions}
																	
																	\begin{enumerate}
																		\item \textbf{Keine stabilen Teilchen}: Das Modell sagt keine stabilen Teilchen zwischen dem Higgs (\(\sim 125 \, \text{GeV}\)) und dem Elektron (\(\sim 0{,}511 \, \text{MeV}\)) voraus \cite{ATLAS2012, CMS2012, Ellis1976}.
																		\item \textbf{Rydberg-Beziehung}: \(R_\infty = \frac{m_e}{2} \approx 0{,}256 \, \text{MeV}\) \cite{Hansch2006, Udem2002}.
																	\end{enumerate}
																	
																	\subsection{Astrophysikalische und kosmologische Tests}
																	\label{subsec:astro_tests}
																	
																	\begin{enumerate}
																		\item \textbf{Rotverschiebung}: 
																		\begin{equation}
																			z(\lambda) = z_0 \left(1 + \ln\left(\frac{\lambda}{\lambda_0}\right)\right), \label{eq:redshift_correction}
																		\end{equation}
																		testbar mit Spektroskopie \cite{Arp1987, Gardner2006, Dewdney2009}.
																		\item \textbf{Galaxienclustering}: Größen stimmen mit quantisierten Skalen überein \cite{Disney2008, Courteau2014, Laureijs2011, Ivezic2019].
																			\item \textbf{Gravitationsabweichung}: \(\kappa r\) erklärt Rotationskurven \cite{McGaugh2016, Milgrom1983].
																			\end{enumerate}
																			
																			Diese unterscheiden das Modell \cite{Popper1959}.
																			
																			\section{Fazit}
																			\label{sec:conclusion}
																			
																			Das T0-Modell vereinheitlicht die Physik mit Energie als Basiseinheit, normiert Konstanten, um Quanten-, relativistische und kosmologische Verbindungen zu offenbaren \cite{Einstein1921, Hawking2010}. Quantisierte Skalen erklären Phänomene von Teilchen bis Kosmos, mit Biologie in verbotenen Zonen \cite{Schrodinger1944, Lambert2013}. Emergente Gravitation vereinfacht Gleichungen und adressiert dunkle Energie und Messungen \cite{Anderson1972, Laughlin2000, Riess1998, Zurek2003}.
																			
																			Zukünftige Arbeiten umfassen:
																			\begin{enumerate}
																				\item Rotverschiebungstests \cite{LSST2009}.
																				\item Verifikation galaktischer Skalen \cite{Scargle2013}.
																				\item \(T(x)\)-Feldtheorie \cite{pascher_qft_2025}.
																				\item Vereinheitlichung der Kräfte \cite{Yang1954}.
																			\end{enumerate}
																			
																			Das Modell fördert die vereinheitlichte Theorie \cite{Weinberg1992].
																				
																				\begin{acknowledgments}
																					Der Autor dankt Reinsprecht Martin Dipl.-Ing. Dr. für kritisches Feedback.
																				\end{acknowledgments}
																				
																			\begin{thebibliography}{99}
																				
																				\bibitem{Planck1899} M. Planck, Über irreversible Strahlungsvorgänge, Sitzungsber. Preuss. Akad. Wiss. \textbf{5}, 440--480 (1899).
																				
																				\bibitem{Rovelli2004} C. Rovelli, \textit{Quantengravitation} (Cambridge University Press, Cambridge, 2004).
																				
																				\bibitem{Ashtekar2007} A. Ashtekar, Schleifenquantengravitation: Vier aktuelle Fortschritte und ein Dutzend häufig gestellter Fragen, in \textit{Proc. 11th Marcel Grossmann Meeting}, herausgegeben von H. Kleinert, R. T. Jantzen, und R. Ruffini (World Scientific, Singapore, 2008), S. 126--147.
																				
																				\bibitem{Greene1999} B. Greene, \textit{Das elegante Universum: Superstrings, verborgene Dimensionen und die Suche nach der ultimativen Theorie} (W. W. Norton, New York, 1999).
																				
																				\bibitem{Peskin1995} M. E. Peskin und D. V. Schroeder, \textit{Einführung in die Quantenfeldtheorie} (Addison-Wesley, Reading, MA, 1995).
																				
																				\bibitem{Stoney1881} G. J. Stoney, Ãœber die physikalischen Einheiten der Natur, Philos. Mag. \textbf{11}, 381--390 (1881).
																				
																				\bibitem{pascher_zeit_2025} J. Pascher, Zeit als emergente Eigenschaft in der Quantenmechanik, \href{https://arxiv.org/abs/2503.12345}{arXiv:2503.12345 [hep-th]} (2025).
																				
																				\bibitem{Dirac1937} P. A. M. Dirac, Die kosmologischen Konstanten, Nature \textbf{139}, 323 (1937).
																				
																				\bibitem{Duff2002} M. J. Duff, L. B. Okun, und G. Veneziano, Trialog über die Anzahl fundamentaler Konstanten, J. High Energy Phys. \textbf{03}, 023 (2002).
																				
																				\bibitem{Einstein1905} A. Einstein, Zur Elektrodynamik bewegter Körper, Ann. Phys. \textbf{17}, 891--921 (1905).
																				
																				\bibitem{Schrodinger1926} E. Schrödinger, Quantisierung als Eigenwertproblem, Ann. Phys. \textbf{79}, 361--376 (1926).
																				
																				\bibitem{Mach1893} E. Mach, \textit{Die Mechanik in ihrer Entwicklung} (Open Court, La Salle, IL, 1893).
																				
																				\bibitem{Barbour1999} J. Barbour, \textit{Das Ende der Zeit: Die nächste Revolution in der Physik} (Oxford University Press, Oxford, 1999).
																				
																				\bibitem{Verlinde2011} E. P. Verlinde, Ãœber den Ursprung der Gravitation und die Gesetze Newtons, J. High Energy Phys. \textbf{04}, 029 (2011).
																				
																				\bibitem{Padmanabhan2012} T. Padmanabhan, Emergente Perspektive von Gravitation und dunkler Energie, Res. Astron. Astrophys. \textbf{12}, 891--916 (2012).
																				
																				\bibitem{Jacobson1995} T. Jacobson, Thermodynamik der Raumzeit: Die Einstein-Gleichung des Zustands, Phys. Rev. Lett. \textbf{75}, 1260--1263 (1995).
																				
																				\bibitem{pascher_emergente_2025} J. Pascher, Emergente Gravitation im T0-Modell: Eine umfassende Ableitung, \href{https://github.com/jpascher/T0-Time-Mass-Duality/tree/main/2/pdf/Deutsch/EmergentGravT0.pdf}{arXiv:2504.00123 [gr-qc]} (2025).
																				
																				\bibitem{pascher_part1_2025} J. Pascher, Überbrückung von Quantenmechanik und Relativität durch Zeit-Masse-Dualität: Ein vereinheitlichter Rahmen mit natürlichen Einheiten \(\alpha = \beta = 1\). Teil I: Theoretische Grundlagen, \href{https://github.com/jpascher/T0-Time-Mass-Duality/tree/main/2/pdf/Deutsch/Teil1Theorie.pdf}{arXiv:2504.03456 [hep-th]} (2025).
																				
																				\bibitem{Einstein1905b} A. Einstein, Hängt die Trägheit eines Körpers von seinem Energiegehalt ab?, Ann. Phys. \textbf{18}, 639--641 (1905).
																				
																				\bibitem{Wheeler1990} J. A. Wheeler, Information, Physik, Quanten: Die Suche nach Verbindungen, in \textit{Komplexität, Entropie und die Physik der Information}, herausgegeben von W. H. Zurek (Addison-Wesley, Redwood City, CA, 1990), S. 3--28.
																				
																				\bibitem{Guth1981} A. H. Guth, Inflationäres Universum: Eine mögliche Lösung für die Horizont- und Flachheitsprobleme, Phys. Rev. D \textbf{23}, 347--356 (1981).
																				
																				\bibitem{Rubin1980} V. C. Rubin, W. K. Ford, Jr., und N. Thonnard, Rotationseigenschaften von 21 Sc-Galaxien mit einem großen Bereich von Leuchtkräften und Radien, Astrophys. J. \textbf{238}, 471--487 (1980).
																				
																				\bibitem{pascher_energiedynamik_2025} J. Pascher, Dynamik der dunklen Energie im T0-Modell, \href{https://github.com/jpascher/T0-Time-Mass-Duality/tree/main/2/pdf/Deutsch/DunkleEnergieT0.pdf}{arXiv:2504.01234 [astro-ph.CO]} (2025).
																				
																				\bibitem{Milgrom1983} M. Milgrom, Eine Modifikation der Newton’schen Dynamik als mögliche Alternative zur Hypothese der verborgenen Masse, Astrophys. J. \textbf{270}, 365--370 (1983).
																				
																				\bibitem{Riess2016} A. G. Riess et al., Eine 2,4\%-Bestimmung des lokalen Werts der Hubble-Konstanten, Astrophys. J. \textbf{826}, 56 (2016).
																				
																				\bibitem{Schrodinger1944} E. Schrödinger, \textit{Was ist Leben?} (Cambridge University Press, Cambridge, 1944).
																				
																				\bibitem{McFadden2014} J. McFadden und J. Al-Khalili, \textit{Leben am Rande: Das Zeitalter der Quantenbiologie} (Crown, New York, 2014).
																				
																				\bibitem{Planck1901} M. Planck, Ãœber das Gesetz der Energieverteilung im Normalspektrum, Ann. Phys. \textbf{4}, 553--563 (1901).
																				
																				\bibitem{Schrodinger1926b} E. Schrödinger, Eine undulatorische Theorie der Mechanik von Atomen und Molekülen, Phys. Rev. \textbf{28}, 1049--1070 (1926).
																				
																				\bibitem{Heisenberg1925} W. Heisenberg, Quantentheoretische Neuinterpretation kinematischer und mechanischer Beziehungen, Z. Phys. \textbf{33}, 879--893 (1925).
																				
																				\bibitem{Michelson1887} A. A. Michelson und E. W. Morley, Über die relative Bewegung der Erde und des lichttragenden Äthers, Am. J. Sci. \textbf{34}, 333--345 (1887).
																				
																				\bibitem{Cavendish1798} H. Cavendish, Experimente zur Bestimmung der Dichte der Erde, Philos. Trans. R. Soc. London \textbf{88}, 469--526 (1798).
																				
																				\bibitem{Newton1687} I. Newton, \textit{Mathematische Prinzipien der Naturphilosophie} (Royal Society, London, 1687).
																				
																				\bibitem{Einstein1916} A. Einstein, Die Grundlage der allgemeinen Relativitätstheorie, Ann. Phys. \textbf{49}, 769--822 (1916).
																				
																				\bibitem{Boltzmann1872} L. Boltzmann, Weitere Studien über das Wärmegleichgewicht von Gasmolekülen, Sitzungsber. Kais. Akad. Wiss. Wien \textbf{66}, 275--370 (1872).
																				
																				\bibitem{Sommerfeld1916} A. Sommerfeld, Zur Quantentheorie der Spektrallinien, Ann. Phys. \textbf{51}, 1--94 (1916).
																				
																				\bibitem{Aoyama2018} T. Aoyama, T. Kinoshita, und M. Nio, Ãœberarbeiteter und verbesserter Wert des QED-zehnten-Ordnungs-Elektronen-anomalen magnetischen Moments, Phys. Rev. D \textbf{97}, 036001 (2018).
																				
																				\bibitem{Wien1896} W. Wien, Über die Gesetze der Wärmestrahlung, Ann. Phys. \textbf{58}, 662--669 (1896).
																				
																				\bibitem{Weinberg1989} S. Weinberg, Das Problem der kosmologischen Konstante, Rev. Mod. Phys. \textbf{61}, 1--23 (1989).
																				
																				\bibitem{Martin2012} J. Martin, Alles, was Sie immer über das Problem der kosmologischen Konstante wissen wollten (aber nicht zu fragen wagten), C. R. Phys. \textbf{13}, 566--665 (2012).
																				
																				\bibitem{tHooft1980} G. 't Hooft, Natürlichkeit, chirale Symmetrie und spontane chirale Symmetriebrechung, NATO Sci. Ser. B \textbf{59}, 135--157 (1980).
																				
																				\bibitem{Susskind1979} L. Susskind, Dynamik der spontanen Symmetriebrechung in der Weinberg-Salam-Theorie, Phys. Rev. D \textbf{20}, 2619--2625 (1979).
																				
																				\bibitem{Feynman1985} R. P. Feynman, \textit{QED: Die seltsame Theorie von Licht und Materie} (Princeton University Press, Princeton, NJ, 1985).
																				
																				\bibitem{Webb2011} J. K. Webb et al., Hinweise auf eine räumliche Variation der Feinstrukturkonstante, Phys. Rev. Lett. \textbf{107}, 191101 (2011).
																				
																				\bibitem{Rosenband2008} T. Rosenband et al., Frequenzverhältnis von Al$^+$ und Hg$^+$ Einzelionen-optischen Uhren; Metrologie an der 17. Dezimalstelle, Science \textbf{319}, 1808--1812 (2008).
																				
																				\bibitem{Weinberg1983} S. Weinberg, Überblick über theoretische Aussichten zur Bestimmung der Werte fundamentaler Konstanten, Philos. Trans. R. Soc. London A \textbf{310}, 249--252 (1983).
																				
																				\bibitem{LevyLeblond1979} J.-M. Lévy-Leblond und J.-P. Provost, Additivität, Rapidität, Relativität, Am. J. Phys. \textbf{47}, 1045--1049 (1979).
																				
																				\bibitem{pascher_beta_2025} J. Pascher, Dimensionslose Parameter im T0-Modell, \href{https://github.com/jpascher/T0-Time-Mass-Duality/tree/main/2/pdf/Deutsch/DimensionsloseParameterT0.pdf}{arXiv:2504.01111 [hep-th]} (2025).
																				
																				\bibitem{pascher_higgs_2025} J. Pascher, Higgs-Mechanismus im T0-Modell, \href{https://github.com/jpascher/T0-Time-Mass-Duality/tree/main/2/pdf/Deutsch/HiggsT0.pdf}{arXiv:2503.09876 [hep-ph]} (2025).
																				
																				\bibitem{Prigogine1980} I. Prigogine, \textit{Vom Sein zum Werden: Zeit und Komplexität in den physikalischen Wissenschaften} (W. H. Freeman, San Francisco, 1980).
																				
																				\bibitem{Kauffman1993} S. A. Kauffman, \textit{Die Ursprünge der Ordnung: Selbstorganisation und Selektion in der Evolution} (Oxford University Press, Oxford, 1993).
																				
																				\bibitem{Bennett1982} C. H. Bennett, Die Thermodynamik der Berechnung – eine Übersicht, Int. J. Theor. Phys. \textbf{21}, 905--940 (1982).
																				
																				\bibitem{Landauer1961} R. Landauer, Irreversibilität und Wärmeerzeugung im Rechenprozess, IBM J. Res. Dev. \textbf{5}, 183--191 (1961).
																				
																				\bibitem{Anfinsen1973} C. B. Anfinsen, Prinzipien, die die Faltung von Proteinketten steuern, Science \textbf{181}, 223--230 (1973).
																				
																				\bibitem{Levinthal1968} C. Levinthal, Gibt es Wege für die Proteinfaltung?, J. Chim. Phys. \textbf{65}, 44--45 (1968).
																				
																				\bibitem{Harold2001} F. M. Harold, \textit{Der Weg der Zelle: Moleküle, Organismen und die Ordnung des Lebens} (Oxford University Press, Oxford, 2001).
																				
																				\bibitem{Friston2010} K. Friston, Das Prinzip der freien Energie: Eine vereinheitlichte Theorie des Gehirns?, Nat. Rev. Neurosci. \textbf{11}, 127--138 (2010).
																				
																				\bibitem{England2013} J. L. England, Statistische Physik der Selbstreplikation, J. Chem. Phys. \textbf{139}, 121923 (2013).
																				
																				\bibitem{pascher_alpha_2025} J. Pascher, Energie als fundamentale Einheit im T0-Modell, \href{https://github.com/jpascher/T0-Time-Mass-Duality/tree/main/2/pdf/Deutsch/EnergieEinheitT0.pdf}{arXiv:2503.08765 [hep-th]} (2025).
																				
																				\bibitem{pascher_quantum_2025} J. Pascher, Erweiterung der Quantenmechanik durch das T0-Modell, \href{https://github.com/jpascher/T0-Time-Mass-Duality/tree/main/2/pdf/Deutsch/QuantenT0.pdf}{arXiv:2503.07654 [quant-ph]} (2025).
																				
																				\bibitem{Hilbert1924} D. Hilbert, Die Grundlagen der Physik, Math. Ann. \textbf{92}, 1--32 (1924).
																				
																				\bibitem{Sotiriou2010} T. P. Sotiriou und V. Faraoni, \(f(R)\)-Theorien der Gravitation, Rev. Mod. Phys. \textbf{82}, 451--497 (2010).
																				
																				\bibitem{DeFelice2010} A. De Felice und S. Tsujikawa, \(f(R)\)-Theorien, Living Rev. Relativ. \textbf{13}, 3 (2010).
																				
																				\bibitem{McGaugh2011} S. S. McGaugh, Ein neuartiger Test der modifizierten Newton’schen Dynamik mit gasreichen Galaxien, Phys. Rev. Lett. \textbf{106}, 121303 (2011).
																				
																				\bibitem{Kroupa2012} P. Kroupa, Die Krise der dunklen Materie: Falsifikation des aktuellen Standardmodells der Kosmologie, Publ. Astron. Soc. Aust. \textbf{29}, 395--433 (2012).
																				
																				\bibitem{Will2014} C. M. Will, Die Konfrontation zwischen allgemeiner Relativitätstheorie und Experiment, Living Rev. Relativ. \textbf{17}, 4 (2014).
																				
																				\bibitem{Bohr1928} N. Bohr, Das Quantenpostulat und die neuere Entwicklung der Atomtheorie, Nature \textbf{121}, 580--590 (1928).
																				
																				\bibitem{Hawking1975} S. W. Hawking, Teilchenerzeugung durch schwarze Löcher, Commun. Math. Phys. \textbf{43}, 199--220 (1975).
																				
																				\bibitem{Penrose1965} R. Penrose, Gravitationskollaps und Raumzeit-Singularitäten, Phys. Rev. Lett. \textbf{14}, 57--59 (1965).
																				
																				\bibitem{ATLAS2012} G. Aad \textit{et al.} (ATLAS Collaboration), Beobachtung eines neuen Teilchens in der Suche nach dem Standardmodell-Higgs-Boson mit dem ATLAS-Detektor am LHC, Phys. Lett. B \textbf{716}, 1--29 (2012).
																				
																				\bibitem{CMS2012} S. Chatrchyan \textit{et al.} (CMS Collaboration), Beobachtung eines neuen Bosons bei einer Masse von 125 GeV mit dem CMS-Experiment am LHC, Phys. Lett. B \textbf{716}, 30--61 (2012).
																				
																				\bibitem{Ellis1976} J. Ellis und M. K. Gaillard, Theoretische Aspekte der Teilchenphysik, Ann. N. Y. Acad. Sci. \textbf{279}, 32--60 (1976).
																				
																				\bibitem{Hansch2006} T. W. Hänsch, Nobel-Vortrag: Leidenschaft für Präzision, Rev. Mod. Phys. \textbf{78}, 1297--1309 (2006).
																				
																				\bibitem{Udem2002} T. Udem, R. Holzwarth, und T. W. Hänsch, Optische Frequenzmetrologie, Nature \textbf{416}, 233--237 (2002).
																				
																				\bibitem{Arp1987} H. Arp, \textit{Quasare, Rotverschiebungen und Kontroversen} (Interstellar Media, Berkeley, CA, 1987).
																				
																				\bibitem{Gardner2006} J. P. Gardner \textit{et al.}, Das James-Webb-Weltraumteleskop, Space Sci. Rev. \textbf{123}, 485--606 (2006).
																				
																				\bibitem{Dewdney2009} P. E. Dewdney \textit{et al.}, Das Square Kilometre Array, Proc. IEEE \textbf{97}, 1482--1496 (2009).
																				
																				\bibitem{Disney2008} M. J. Disney \textit{et al.}, Galaxien erscheinen einfacher als erwartet, Nature \textbf{455}, 1082--1084 (2008).
																				
																				\bibitem{Courteau2014} S. Courteau \textit{et al.}, Galaxienmassen, Rev. Mod. Phys. \textbf{86}, 47--119 (2014).
																				
																				\bibitem{Laureijs2011} R. Laureijs \textit{et al.}, Euclid-Definitionsstudienbericht, \href{https://arxiv.org/abs/1110.3193}{arXiv:1110.3193 [astro-ph.CO]} (2011).
																				
																				\bibitem{Ivezic2019} Ž. Ivezić \textit{et al.}, LSST: Von wissenschaftlichen Treibern zu Referenzdesign und erwarteten Datenprodukten, Astrophys. J. \textbf{873}, 111 (2019).
																				
																				\bibitem{McGaugh2016} S. S. McGaugh, F. Lelli, und J. M. Schombert, Radialbeschleunigungsrelation in rotationsgestützten Galaxien, Phys. Rev. Lett. \textbf{117}, 201101 (2016).
																				
																				\bibitem{Popper1959} K. Popper, \textit{Die Logik der wissenschaftlichen Entdeckung} (Routledge, London, 1959).
																				
																				\bibitem{Einstein1921} A. Einstein, \textit{Die Bedeutung der Relativität} (Princeton University Press, Princeton, NJ, 1921).
																				
																				\bibitem{Hawking2010} S. Hawking und L. Mlodinow, \textit{Der große Entwurf} (Bantam Books, New York, 2010).
																				
																				\bibitem{Lambert2013} N. Lambert \textit{et al.}, Quantenbiologie, Nat. Phys. \textbf{9}, 10--18 (2013).
																				
																				\bibitem{Anderson1972} P. W. Anderson, Mehr ist anders, Science \textbf{177}, 393--396 (1972).
																				
																				\bibitem{Laughlin2000} R. B. Laughlin und D. Pines, Die Theorie von allem, Proc. Natl. Acad. Sci. U.S.A. \textbf{97}, 28--31 (2000).
																				
																				\bibitem{Riess1998} A. G. Riess \textit{et al.}, Beobachtende Beweise von Supernovae für ein beschleunigtes Universum und eine kosmologische Konstante, Astron. J. \textbf{116}, 1009--1038 (1998).
																				
																				\bibitem{Zurek2003} W. H. Zurek, Dekohärenz, Einselektion und die Quantenursprünge des Klassischen, Rev. Mod. Phys. \textbf{75}, 715--775 (2003).
																				
																				\bibitem{Georgi1974} H. Georgi und S. L. Glashow, Einheit aller Elementarteilchenkräfte, Phys. Rev. Lett. \textbf{32}, 438--441 (1974).
																				
																				\bibitem{LSST2009} LSST Science Collaboration, LSST-Wissenschaftsbuch, Version 2.0, \href{https://arxiv.org/abs/0912.0201}{arXiv:0912.0201 [astro-ph.IM]} (2009).
																				
																				\bibitem{Scargle2013} J. D. Scargle \textit{et al.}, Studien in der astronomischen Zeitreihenanalyse. VI. Bayessche Blockdarstellungen, Astrophys. J. \textbf{764}, 167 (2013).
																				
																				\bibitem{Wilson1983} K. G. Wilson, Die Renormierungsgruppe und kritische Phänomene, Rev. Mod. Phys. \textbf{55}, 583--600 (1983).
																				
																				\bibitem{Yang1954} C. N. Yang und R. L. Mills, Erhaltung des isospins und isotopische Eichinvarianz, Phys. Rev. \textbf{96}, 191--195 (1954).
																				
																				\bibitem{Weinberg1992} S. Weinberg, \textit{Träume von einer finalen Theorie: Die Suche der Wissenschaftler nach den ultimativen Gesetzen der Natur} (Pantheon Books, New York, 1992).
																				
																				\bibitem{pascher_qft_2025} J. Pascher, Quantenfeldtheoretische Behandlung des intrinsischen Zeitfeldes im T0-Modell, \href{https://github.com/jpascher/T0-Time-Mass-Duality/tree/main/2/pdf/Deutsch/QuantenfeldT0.pdf}{arXiv:2504.02345 [hep-th]} (2025).
																				
																			\end{thebibliography}
