\documentclass[a4paper,12pt]{article}
\usepackage[utf8]{inputenc}
\usepackage[T1]{fontenc}
\usepackage{lmodern}
\usepackage[ngerman]{babel}
\usepackage{csquotes}
\usepackage{amsmath}
\usepackage{amsfonts}
\usepackage{amssymb}
\usepackage{physics}
\usepackage{geometry}
\usepackage{tocloft}
\usepackage{xcolor}
\usepackage{graphicx,tikz,pgfplots}
\pgfplotsset{compat=1.18}
\usepackage{booktabs}
\usepackage{array}
\usepackage{tabularx}
\usepackage{braket}
\usepackage{siunitx}
\DeclareSIUnit{\year}{yr}
\DeclareSIUnit{\parsec}{pc}
\usepackage{amsthm}
\usepackage[colorlinks=true, linkcolor=blue, citecolor=blue, urlcolor=blue]{hyperref}
\usepackage{cleveref}
\usepackage{fancyhdr}

\geometry{a4paper, margin=2cm}

\hypersetup{
	pdftitle={Field Theory and Quantum Correlations: A New Perspective on Instantaneity},
	pdfauthor={Johann Pascher},
	pdfcreator={LaTeX}
}

% Headers and Footers
\pagestyle{fancy}
\fancyhf{}
\fancyhead[L]{Johann Pascher}
\fancyhead[R]{Time-Mass Duality}
\fancyfoot[C]{\thepage}
\renewcommand{\headrulewidth}{0.4pt}
\renewcommand{\footrulewidth}{0.4pt}

\renewcommand{\cftsecfont}{\color{blue}}
\renewcommand{\cftsubsecfont}{\color{blue}}
\renewcommand{\cftsecpagefont}{\color{blue}}
\renewcommand{\cftsubsecpagefont}{\color{blue}}
\setlength{\cftsecindent}{1cm}
\setlength{\cftsubsecindent}{2cm}

% Custom commands
\newcommand{\Tfield}{T(x)}
\newcommand{\DcovT}[1]{\Tfield D_\mu #1 + #1 \partial_\mu \Tfield}
\newcommand{\DhiggsT}{\Tfield (\partial_\mu + ig A_\mu) \Phi + \Phi \partial_\mu \Tfield}
\newcommand{\betaT}{\beta_{\text{T}}}
\newcommand{\alphaEM}{\alpha_{\text{EM}}}
\newcommand{\Mpl}{M_{\text{Pl}}}
\newcommand{\Tzerot}{T_0(\Tfield)}
\newcommand{\Tzero}{T_0}
\newcommand{\vecx}{\vec{x}}
\newcommand{\gammaf}{\gamma_{\text{Lorentz}}}

\newtheorem{theorem}{Theorem}[section]
\newtheorem{proposition}[theorem]{Proposition}
\theoremstyle{definition}
\newtheorem{definition}{Definition}[theorem]
\theoremstyle{remark}
\newtheorem{remark}{Remark}

\title{Field Theory and Quantum Correlations: \\A New Perspective on Instantaneity}
\author{Johann Pascher}
\date{March 28, 2025}

\begin{document}
	
	\maketitle
	
	\begin{abstract}
		This work develops a new perspective on quantum correlations and their apparent instantaneity within the framework of the T0 model. Through a unified field approach, it demonstrates how the nonlocal properties of quantum mechanics can be understood as a natural consequence of an underlying field structure. Particular emphasis is placed on the role of the quantum background and the interpretation of modern Bell experiments. This viewpoint complements the time-mass duality theory and provides a coherent framework to explain quantum mechanical phenomena without assuming “spooky action at a distance.”
	\end{abstract}
	
	\tableofcontents
	\newpage
	
	\section{Introduction}
	
	Quantum mechanics has long presented us with puzzles, particularly regarding the nature of quantum correlations. The notion that two particles, separated by vast distances, could be instantaneously connected has intrigued scientists since Einstein’s famous critique of “spooky action at a distance.” Modern experiments, such as the loophole-free Bell tests starting in 2015, have shown that these correlations are real and surpass classical notions of locality and causality. But how can we understand this phenomenon without abandoning the foundational pillars of physics?
	
	In this work, I propose viewing the nonlocality of the quantum world not as mysterious instantaneity but as a natural property of a unified quantum field developed within the T0 model framework. This model, based on time-mass duality and treating time as an absolute quantity with variable mass, offers a fresh perspective. Rather than seeing particles as isolated objects, I interpret them as nodes or excitations of a fundamental field, whose coherence explains the observed correlations. This viewpoint builds on the time-mass duality work detailed in “Time-Mass Duality Theory: Derivation of Parameters” \cite{pascher_params_2025}, extending it with a field-theoretic foundation. A central element is the intrinsic time field \(\Tfield = \frac{\hbar}{\max(m c^2, \omega)}\), which determines the timescale of field nodes and bridges quantum mechanics and cosmology.
	
	My approach begins with the idea that the vacuum is not an empty space but an active quantum background with defined properties, expressed through the electric and magnetic field constants \(\varepsilon_0\) and \(\mu_0\). Particles are not standalone entities but stable patterns of this field, while quantum correlations reflect the inherent coherence of the field state. Modern experiments, such as the Vienna tests of 2015 \cite{Giustina2015} or the “Big Bell Test” of 2018 \cite{BigBellTest2018}, support this interpretation by confirming nonlocality as a fact without requiring instantaneous communication between particles. To mathematically substantiate this perspective, I introduce a fundamental field equation describing the dynamics of this unified field, linking it to the modified quantum mechanics of the T0 model, as developed in “The Necessity of Extending Standard Quantum Mechanics” \cite{pascher_quantum_2025}. This work aims to create a coherent narrative that unifies both experimental findings and the theoretical foundations of the T0 model.
	
	\section{The Vacuum as a Quantum Background}
	
	The vacuum, as understood in modern physics, is far more than an empty space. It is a dynamic medium characterized by fundamental physical properties such as the electric field constant \(\varepsilon_0\) and the magnetic field constant \(\mu_0\). These constants are not mere mathematical tools but expressions of a deep structure that defines the speed of light and enables the interactions of all fields in the universe. In the T0 model, this vacuum is regarded as an active quantum background forming the foundation for all physical phenomena, including quantum correlations. A central relationship underscoring this role is:
	
	\begin{equation}
		c = \frac{1}{\sqrt{\varepsilon_0 \mu_0}}
	\end{equation}
	
	This background is not a passive stage but a carrier medium enabling electromagnetic waves and other fundamental fields. Its homogeneity ensures that the speed of light remains constant, as required by special relativity. However, in the T0 model, the vacuum’s significance extends further: it directly influences the intrinsic time field \(\Tfield\), which determines each particle’s timescale, as shown in “Time-Mass Duality Theory” \cite{pascher_params_2025}. The vacuum thus becomes key to understanding the nonlocal properties of the quantum world, acting as a coherent medium that sustains correlations across vast distances.
	
	\section{Quantum Correlations in the Field Model}
	
	When we speak of entangled particles, we often think of photons whose polarization is described by a shared state. A typical entangled state is:
	
	\begin{equation}
		|\psi\rangle = \frac{1}{\sqrt{2}} (|H\rangle_A |H\rangle_B + |V\rangle_A |V\rangle_B)
	\end{equation}
	
	In the traditional view, these particles appear as separate objects whose states correlate instantaneously upon measurement. However, in the T0 model’s field approach, this perspective fundamentally changes. Entangled states are not properties of isolated particles but coherent patterns of a unified quantum field permeating space, as described in “Dynamic Mass of Photons” \cite{pascher_photons_2025}.
	
	The Bell inequalities, formulated by John Bell in 1964, clearly demonstrate that local realistic theories cannot account for the observed correlations. Mathematically expressed, one such inequality is:
	
	\begin{equation}
		|E(a,b) - E(a,c)| \leq 1 + E(b,c)
	\end{equation}
	
	In experiments, this inequality is consistently violated, as demonstrated by Alain Aspect in 1982 and subsequent tests \cite{Aspect1982}. In the field model, this violation is unsurprising: the quantum field is inherently nonlocal because it possesses a global structure that connects local measurements without requiring signal transmission.
	
	The Vienna experiments of 2015, led by Anton Zeilinger, marked a milestone by closing all classical loopholes—such as detection efficiency or spatial separation \cite{Giustina2015}. With a significance exceeding 11 standard deviations, they confirmed nonlocality. Equally impressive was the “Big Bell Test” of 2018, where over 100,000 people worldwide controlled measurement settings to address the freedom-of-choice loophole \cite{BigBellTest2018}. These experiments demonstrate that quantum correlations are real, and in the T0 field model, they find a natural explanation as properties of a coherent quantum field.
	
	\section{Field Theory and Instantaneity}
	
	To make the concept of instantaneity more tangible, I draw an analogy to sound waves. When a sound wave travels through a room, it is present everywhere, yet a microphone measures only the local vibration. Two microphones recording the same wave exhibit a correlation not due to instantaneous communication between them but due to the shared structure of the wave. Similarly, in the quantum field model, entangled particles are nodes of a global field, and their correlations are inherent in the field state before measurement occurs.
	
	This analogy resolves the nonlocality paradox. There is no “action at a distance,” but rather an inherent coherence of the field, controlled by the intrinsic time field \(\Tfield\), as described in “Dynamic Mass of Photons” \cite{pascher_photons_2025}. The Higgs field plays a central role here, defining the mass and thus the timescale of the field nodes, linking the observed correlations to time-mass duality.
	
	\section{Field Equations in Dual Formulation}
	
	Quantum mechanics in the T0 model is described by a modified Schrödinger equation that incorporates variable mass. Unlike the classical form \(i\hbar \frac{\partial}{\partial t} \Psi = \hat{H} \Psi\), it is expressed here as:
	
	\begin{equation}
		i\hbar \Tfield \frac{\partial}{\partial t} \Psi + i\hbar \Psi \frac{\partial \Tfield}{\partial t} = \hat{H} \Psi
	\end{equation}
	
	This approach, developed in “The Necessity of Extending Standard Quantum Mechanics” \cite{pascher_quantum_2025}, reflects time-mass duality and integrates it into field theory. The total Lagrangian density of the model is:
	\begin{equation}
		\mathcal{L}_{\text{Total}} = \mathcal{L}_{\text{Boson}} + \mathcal{L}_{\text{Fermion}} + \mathcal{L}_{\text{Higgs-T}} + \mathcal{L}_{\text{intrinsic}}
	\end{equation}
	where \(\mathcal{L}_{\text{intrinsic}} = \frac{1}{2} \partial_\mu \Tfield \partial^\mu \Tfield - V(\Tfield)\) describes the dynamics of the time field, as elaborated in “Mathematical Core Formulations” \cite{pascher_lagrange_2025}.
	
	\section{Cosmological Implications}
	
	The T0 model has far-reaching implications for cosmology that align with the field perspective. The gravitational potential is modified to \(\Phi(r) = -\frac{G M}{r} + \kappa r\), where \(\kappa \approx \SI{4.8e-11}{\meter\per\second\squared}\) emerges from the \(\Tfield\) dynamics, as shown in “Mass Variation in Galaxies” \cite{pascher_galaxies_2025}. Cosmic redshift is described as energy loss: \(1 + z = e^{\alpha d}\), with \(\alpha \approx \SI{2.3e-18}{\per\meter}\), as derived in “Measurement Differences” \cite{pascher_messdifferenzen_2025}. A wavelength-dependent redshift arises with \(z(\lambda) = z_0 (1 + \betaT \ln(\lambda/\lambda_0))\), where \(\betaT^{\text{SI}} \approx 0.008\) and \(\betaT^{\text{nat}} = 1\), as established in “Parameter Derivations” \cite{pascher_params_2025}. These effects demonstrate how the quantum field model seamlessly integrates into the cosmological aspects of the T0 model.
	
	\section{Conclusion}
	
	The T0 model offers a new perspective on quantum correlations by explaining them as natural properties of a unified quantum field controlled by the intrinsic time field \(\Tfield\). This viewpoint resolves the instantaneity paradox by interpreting nonlocality as an inherent coherence of the field, supported by modern Bell experiments such as those by Zeilinger and the “Big Bell Test.” By integrating with time-mass duality, as developed in the quantum mechanics and cosmology works of the T0 model, it creates a coherent framework that transcends the boundaries between quantum physics and field theory.
	
	\begin{thebibliography}{99}
		\bibitem{pascher_params_2025} Pascher, J. (2025). \href{https://github.com/jpascher/T0-Time-Mass-Duality/tree/main/2/pdf/English/ZeitMasseT0ParamsEn.pdf}{Time-Mass Duality Theory (T0 Model): Derivation of Parameters \(\kappa\), \(\alpha\), and \(\beta\)}. April 4, 2025.
		\bibitem{pascher_galaxies_2025} Pascher, J. (2025). \href{https://github.com/jpascher/T0-Time-Mass-Duality/tree/main/2/pdf/English/MassVarGalaxienEn.pdf}{Mass Variation in Galaxies: An Analysis in the T0 Model with Emergent Gravitation}. March 30, 2025.
		\bibitem{pascher_messdifferenzen_2025} Pascher, J. (2025). \href{https://github.com/jpascher/T0-Time-Mass-Duality/tree/main/2/pdf/English/MessdifferenzenT0StandardEn.pdf}{Compensatory and Additive Effects: An Analysis of Measurement Differences Between the T0 Model and the \(\Lambda\)CDM Standard Model}. April 2, 2025.
		\bibitem{pascher_lagrange_2025} Pascher, J. (2025). \href{https://github.com/jpascher/T0-Time-Mass-Duality/tree/main/2/pdf/English/MathZeitMasseLagrange.pdf}{From Time Dilation to Mass Variation: Mathematical Core Formulations of Time-Mass Duality Theory}. March 29, 2025.
		\bibitem{pascher_photons_2025} Pascher, J. (2025). \href{https://github.com/jpascher/T0-Time-Mass-Duality/tree/main/2/pdf/English/DynMassePhotonenNichtlokalEn.pdf}{Dynamic Mass of Photons and Its Implications for Nonlocality in the T0 Model}. March 25, 2025.
		\bibitem{pascher_quantum_2025} Pascher, J. (2025). \href{https://github.com/jpascher/T0-Time-Mass-Duality/tree/main/2/pdf/English/NotwendigkeitQMErweiterungEn.pdf}{The Necessity of Extending Standard Quantum Mechanics and Quantum Field Theory}. March 27, 2025.
		\bibitem{Hensen2015} Hensen, B., et al. (2015). \textit{Loophole-free Bell inequality violation using electron spins separated by 1.3 kilometres}. Nature, 526, 682-686.
		\bibitem{Giustina2015} Giustina, M., et al. (2015). \textit{Significant-loophole-free test of Bell’s theorem with entangled photons}. Physical Review Letters, 115, 250401.
		\bibitem{BigBellTest2018} The BIG Bell Test Collaboration. (2018). \textit{Challenging local realism with human choices}. Nature, 557, 212-216.
		\bibitem{Bell1964} Bell, J. S. (1964). \textit{On the Einstein-Podolsky-Rosen paradox}. Physics, 1(3), 195-200.
		\bibitem{Aspect1982} Aspect, A., et al. (1982). \textit{Experimental test of Bell's inequalities using time-varying analyzers}. Physical Review Letters, 49, 1804-1807.
		\bibitem{Wilczek2008} Wilczek, F. (2008). \textit{The Lightness of Being: Mass, Ether, and the Unification of Forces}. Basic Books.
		\bibitem{Milonni1994} Milonni, P. W. (1994). \textit{The Quantum Vacuum: An Introduction to Quantum Electrodynamics}. Academic Press.
		\bibitem{Aitchison2004} Aitchison, I. J. R. (2004). \textit{An Informal Introduction to Gauge Field Theories}. Cambridge University Press.
		\bibitem{Weinberg1995} Weinberg, S. (1995). \textit{The Quantum Theory of Fields}. Cambridge University Press.
		\bibitem{Fox2006} Fox, M. (2006). \textit{Quantum Optics: An Introduction}. Oxford University Press.
		\bibitem{Zeilinger2010} Zeilinger, A. (2010). \textit{Dance of the Photons: From Einstein to Quantum Teleportation}. Farrar, Straus and Giroux.
		\bibitem{Bohm1980} Bohm, D. (1980). \textit{Wholeness and the Implicate Order}. Routledge.
	\end{thebibliography}
	
\end{document}