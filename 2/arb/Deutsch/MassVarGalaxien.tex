\documentclass[12pt,a4paper]{article}
\usepackage[utf8]{inputenc}
\usepackage[T1]{fontenc}
\usepackage[ngerman]{babel}
\usepackage{lmodern}
\usepackage{csquotes}
\usepackage{amsmath}
\usepackage{amssymb}
\usepackage{physics}
\usepackage{geometry}
\usepackage{tocloft}
\usepackage{xcolor}
\usepackage{graphicx,tikz,pgfplots}
\pgfplotsset{compat=1.18}
\usepackage{booktabs}
\usepackage{siunitx}
\usepackage{amsthm}
\usepackage[colorlinks=true, linkcolor=blue, citecolor=blue, urlcolor=blue]{hyperref}
\usepackage{cleveref}
\usepackage{fancyhdr}

\geometry{a4paper, margin=2cm}

% Headers and Footers
\pagestyle{fancy}
\fancyhf{}
\fancyhead[L]{Johann Pascher}
\fancyhead[R]{Zeit-Masse-Dualität}
\fancyfoot[C]{\thepage}
\renewcommand{\headrulewidth}{0.4pt}
\renewcommand{\footrulewidth}{0.4pt}

% Table of Contents Styling
\renewcommand{\cftsecfont}{\color{blue}}
\renewcommand{\cftsubsecfont}{\color{blue}}
\renewcommand{\cftsecpagefont}{\color{blue}}
\renewcommand{\cftsubsecpagefont}{\color{blue}}
\setlength{\cftsecindent}{1cm}
\setlength{\cftsubsecindent}{2cm}

\hypersetup{
	colorlinks=true,
	linkcolor=blue,
	citecolor=blue,
	urlcolor=blue,
	pdftitle={Massenvariation in Galaxien: Eine Analyse im T0-Modell mit emergenter Gravitation},
	pdfauthor={Johann Pascher},
	pdfsubject={Theoretische Physik},
	pdfkeywords={T0-Modell, Zeit-Masse-Dualität, Galaxiendynamik, Dunkle Materie}
}

% Custom Commands (angepasst an deutsche Konventionen)
\newcommand{\Tfield}{T(x)}
\newcommand{\betaT}{\beta_{\text{T}}}
\newcommand{\alphaEM}{\alpha_{\text{EM}}}
\newcommand{\alphaW}{\alpha_{\text{W}}}
\newcommand{\Mpl}{M_{\text{Pl}}}
\newcommand{\Tzerot}{T_0(\Tfield)}
\newcommand{\Tzero}{T_0}
\newcommand{\vecx}{\vec{x}}
\newcommand{\DhiggsT}{\Tfield (\partial_\mu + ig A_\mu) \Phi + \Phi \partial_\mu \Tfield}
\newcommand{\DcovT}[1]{\Tfield D_\mu #1 + #1 \partial_\mu \Tfield}
\newcommand{\HiggsLagr}{\mathcal{L}_{\text{Higgs-T}}}

\title{Massenvariation in Galaxien: \\Eine Analyse im T0-Modell mit emergenter Gravitation}
\author{Johann Pascher}
\date{30. März 2025}

\begin{document}
	
	\maketitle
	
	\begin{abstract}
		Diese Arbeit analysiert die Dynamik von Galaxien im Rahmen des T0-Modells der Zeit-Masse-Dualitätstheorie, bei dem die Zeit absolut ist und die Masse als \( m = \frac{\hbar}{T c^2} \) variiert, wobei \( \Tfield \) ein dynamisches intrinsisches Zeitfeld ist. Gravitation wird nicht als fundamentale Wechselwirkung eingeführt, sondern entsteht aus den Gradienten von \( \Tfield \). Wir formulieren eine vollständige totale Lagrangedichte, die Beiträge der vier fundamentalen Felder (Higgs, Fermionen, Eichbosonen) und des intrinsischen Zeitfelds umfasst, und zeigen, dass flache Rotationskurven durch die Variation von \( \Tfield \) ohne die Notwendigkeit von dunkler Materie oder separater dunkler Energie erklärt werden können. Es werden experimentelle Tests zur Validierung des Modells vorgeschlagen, einschließlich kosmologischer Implikationen wie der Interpretation des kosmischen Mikrowellenhintergrunds.
	\end{abstract}
	
	\tableofcontents
	\newpage
	
	\section{Einleitung}
	Die Rotationskurven von Galaxien zeigen ein Verhalten, das allein durch sichtbare Materie nicht erklärt werden kann. In den äußeren Regionen von Spiralgalaxien bleibt die Rotationsgeschwindigkeit \( v(r) \) nahezu konstant, anstatt mit \( r^{-1/2} \) abzunehmen, wie es das Keplersche Gesetz für isolierte Massen vorhersagt. Das Standard-Kosmosmodell (\(\Lambda\)CDM) berücksichtigt dieses Phänomen durch die Annahme einer unsichtbaren Komponente, der dunklen Materie, die einen ausgedehnten Halo um Galaxien bildet und die Bewegung sichtbarer Materie durch ihr Gravitationsfeld steuert, ergänzt durch dunkle Energie zur Erklärung der kosmischen Beschleunigung.
	
	Diese Arbeit verfolgt einen alternativen Ansatz, basierend auf dem T0-Modell der Zeit-Masse-Dualitätstheorie, bei dem die Zeit absolut ist und die Teilchenmasse als \( m = \frac{\hbar}{\Tfield c^2} \) variiert, wobei \( \Tfield \) ein dynamisches intrinsisches Zeitfeld ist. In diesem Rahmen wird dunkle Materie nicht als separate Entität betrachtet; stattdessen entstehen die beobachteten dynamischen Effekte aus emergenter Gravitation, die aus den Gradienten von \( \Tfield \) resultiert. Ähnlich werden Effekte, die traditionell der dunklen Energie zugeschrieben werden, wie die Rotverschiebung, durch die räumliche Variation von \( \Tfield \) erklärt, wodurch die Notwendigkeit einer separaten dunklen Energie, wie im \(\Lambda\)CDM-Modell, entfällt. Diese Reformulierung liefert mathematisch äquivalente Vorhersagen für Rotationskurven und bietet eine fundamental andere physikalische Interpretation, die weder dunkle Materie noch separate dunkle Energie erfordert. Eine detaillierte Analyse der kosmologischen Implikationen des T0-Modells, insbesondere in Bezug auf Entfernungsmessungen, Rotverschiebung und die Interpretation des kosmischen Mikrowellenhintergrunds, findet sich in \cite{pascher_messdifferenzen_2025}.
	
	\subsection{Rotverschiebung im T0-Modell}
	Im T0-Modell wird die Rotverschiebung \( z \) durch die Variation des intrinsischen Zeitfelds \( \Tfield \) bestimmt. Die Beziehung zwischen Rotverschiebung und Masse lautet:
	\begin{equation}
		1 + z = \frac{\Tfield_0}{\Tfield} = \frac{m}{m_0},
	\end{equation}
	wobei \( \Tfield_0 \) und \( m_0 \) die Werte des intrinsischen Zeitfelds und der Masse am Ort des Beobachters sind. Diese Interpretation der Rotverschiebung basiert auf der intrinsischen Zeit und erfordert keine kosmische Expansion, im Gegensatz zum \(\Lambda\)CDM-Modell, wo die Rotverschiebung durch die Expansion des Universums erklärt wird:
	\begin{equation}
		1 + z = \frac{a(t_0)}{a(t_{\text{emit}})}.
	\end{equation}
	Die räumliche Variation von \( \Tfield \) kann mit der Entfernung \( d \) verknüpft werden durch \( \Tfield = \Tfield_0 e^{-\alpha d} \), wobei \( \alpha = H_0/c \), was zu einer äquivalenten Form führt:
	\begin{equation}
		1 + z = e^{\alpha d}.
	\end{equation}
	Diese Formulierung stimmt mit dem Energieverlust von Photonen aufgrund der Dynamik von \( \Tfield \) überein, wie in \cite{pascher_messdifferenzen_2025} detailliert beschrieben. Die Beziehung zwischen Rotverschiebung und Entfernung \( d \) im T0-Modell lautet somit:
	\begin{equation}
		d = \frac{c \ln(1 + z)}{H_0},
	\end{equation}
	wobei \( H_0 \) die Hubble-Konstante ist, die im T0-Modell als Maß für die räumliche Variationsrate von \( \Tfield \) neu interpretiert wird, anstatt eine Expansionsrate darzustellen.
	
	\subsection{Kosmologische Implikationen: Entfernungsmessungen und CMB-Interpretation}
	Das T0-Modell hat weitreichende Implikationen für kosmologische Messungen, wie in \cite{pascher_messdifferenzen_2025} detailliert beschrieben. Insbesondere unterscheiden sich die Entfernungsmessungen im T0-Modell von denen im \(\Lambda\)CDM-Modell:
	
	- \textbf{Physische Entfernung \( d \):}
	\[
	d = \frac{c \ln(1 + z)}{H_0},
	\]
	im Vergleich zu \(\Lambda\)CDM:
	\[
	d = \frac{c}{H_0} \int_0^z \frac{dz'}{\sqrt{\Omega_m (1 + z')^3 + \Omega_\Lambda}}.
	\]
	
	- \textbf{Luminositätsentfernung \( d_L \):}
	\[
	d_L = \frac{c}{H_0} \ln(1 + z) (1 + z),
	\]
	im Vergleich zu \(\Lambda\)CDM:
	\[
	d_L = (1 + z) \cdot \frac{c}{H_0} \int_0^z \frac{dz'}{\sqrt{\Omega_m (1 + z')^3 + \Omega_\Lambda}}.
	\]
	
	- \textbf{Winkeldurchmesser-Entfernung \( d_A \):}
	\[
	d_A = \frac{c \ln(1 + z)}{H_0 (1 + z)},
	\]
	im Vergleich zu \(\Lambda\)CDM:
	\[
	d_A = \frac{d}{1 + z}.
	\]
	
	Zusätzlich ist die Temperatur-Rotverschiebungsrelation des CMB im T0-Modell aufgrund der Dynamik von \( \Tfield \) modifiziert:
	\begin{equation}
		T(z) = T_0 (1 + z) (1 + \betaT^{\text{SI}} \ln(1 + z)),
	\end{equation}
	mit \(\betaT^{\text{SI}} \approx 0.008\) in SI-Einheiten, abgeleitet aus fundamentaleren Parametern als \(\betaT = \frac{\lambda_h^2 v^2}{4\pi^2 \lambda_0^2 \alpha_0}\). In natürlichen Einheiten vereinfacht sich dies zu \( T(z) = T_0 (1 + z) (1 + \ln(1 + z)) \) mit \(\betaT^{\text{nat}} = 1\). Das T0-Modell sagt auch eine wellenlängenabhängige Komponente der Rotverschiebung voraus:
	\begin{equation}
		z(\lambda) = z_0 \left(1 + \betaT^{\text{SI}} \ln\left(\frac{\lambda}{\lambda_{\text{ref}}}\right)\right),
	\end{equation}
	wobei \(\lambda_{\text{ref}}\) eine Referenzwellenlänge ist. Diese Beziehungen unterscheiden sich von der \(\Lambda\)CDM-Vorhersage \( T(z) = T_0 (1 + z) \) und können durch präzise spektroskopische Beobachtungen getestet werden.
	
\begin{thebibliography}{99}
	\bibitem{pascher_zeit_2025} Pascher, J. (2025). \href{https://github.com/jpascher/T0-Time-Mass-Duality/tree/main/2/pdf/Deutsch/ZeitEmergentQM.pdf}{Zeit als emergente Eigenschaft in der Quantenmechanik: Eine Verbindung zwischen Relativität, Feinstrukturkonstante und Quantendynamik}. 23. März 2025.
	\bibitem{pascher_messdifferenzen_2025} Pascher, J. (2025). \href{https://github.com/jpascher/T0-Time-Mass-Duality/tree/main/2/pdf/Deutsch/MessdifferenzenT0Standard.pdf}{Kompensatorische und additive Effekte: Eine Analyse der Messunterschiede zwischen dem T0-Modell und dem \(\Lambda\)CDM-Standardmodell}. 2. April 2025.
	\bibitem{pascher_alpha_2025} Pascher, J. (2025). \href{https://github.com/jpascher/T0-Time-Mass-Duality/tree/main/2/pdf/Deutsch/NatEinheitenAlpha1.pdf}{Energie als fundamentale Einheit: Natürliche Einheiten mit \(\alpha = 1\) im T0-Modell}. 26. März 2025.
	\bibitem{pascher_params_2025} Pascher, J. (2025). \href{https://github.com/jpascher/T0-Time-Mass-Duality/tree/main/2/pdf/Deutsch/ZeitMasseT0Params.pdf}{Zeit-Masse-Dualitätstheorie (T0-Modell): Ableitung der Parameter \(\kappa\), \(\alpha\) und \(\beta\)}. 4. April 2025.
	\bibitem{pascher_higgs_2025} Pascher, J. (2025). \href{https://github.com/jpascher/T0-Time-Mass-Duality/tree/main/2/pdf/Deutsch/MathHiggsZeitMasse.pdf}{Mathematische Formulierung des Higgs-Mechanismus in der Zeit-Masse-Dualität}. 28. März 2025.
	\bibitem{pascher_lagrange_2025} Pascher, J. (2025). \href{https://github.com/jpascher/T0-Time-Mass-Duality/tree/main/2/pdf/Deutsch/MathZeitMasseLagrange.pdf}{Von Zeitdilatation zu Massenvariation: Mathematische Kernformulierungen der Zeit-Masse-Dualitätstheorie}. 29. März 2025.
	\bibitem{pascher_emergente_gravitation_2025} Pascher, J. (2025). \href{https://github.com/jpascher/T0-Time-Mass-Duality/tree/main/2/pdf/Deutsch/EmergentGravT0.pdf}{Emergente Gravitation im T0-Modell: Eine umfassende Ableitung}. 1. April 2025.
	\bibitem{pascher_galaxies_2025} Pascher, J. (2025). \href{https://github.com/jpascher/T0-Time-Mass-Duality/tree/main/2/pdf/Deutsch/MassVarGalaxien.pdf}{Massenvariation in Galaxien: Eine Analyse im T0-Modell mit emergenter Gravitation}. 30. März 2025.
	\bibitem{pascher_temp_2025} Pascher, J. (2025). \href{https://github.com/jpascher/T0-Time-Mass-Duality/tree/main/2/pdf/Deutsch/TempEinheitenCMB.pdf}{Anpassung der Temperatureinheiten in natürlichen Einheiten und CMB-Messungen}. 2. April 2025.
	\bibitem{pascher_alphabeta_2025} Pascher, J. (2025). \href{https://github.com/jpascher/T0-Time-Mass-Duality/tree/main/2/pdf/Deutsch/Alpha1Beta1Konsistenz.pdf}{Vereinheitlichtes Einheitensystem im T0-Modell: Die Konsistenz von \(\alpha = 1\) und \(\beta = 1\)}. 5. April 2025.
	\bibitem{pascher_feldtheorie_2025} Pascher, J. (2025). \href{https://github.com/jpascher/T0-Time-Mass-Duality/tree/main/2/pdf/Deutsch/FeldtheorieQuanten.pdf}{Feldtheorie und Quantenkorrelationen: Eine neue Perspektive auf Instantaneität}. 28. März 2025.
	\bibitem{pascher_planck_2025} Pascher, J. (2025). \href{https://github.com/jpascher/T0-Time-Mass-Duality/tree/main/2/pdf/Deutsch/JenseitsPlanck.pdf}{Reale Konsequenzen der Reformulierung von Zeit und Masse in der Physik: Jenseits der Planck-Skala}. 24. März 2025.
	\bibitem{pascher_erweiterung_2025} Pascher, J. (2025). \href{https://github.com/jpascher/T0-Time-Mass-Duality/tree/main/2/pdf/Deutsch/NotwendigkeitQMErweiterung.pdf}{Die Notwendigkeit der Erweiterung der Standard-Quantenmechanik und Quantenfeldtheorie}. 27. März 2025.
	\bibitem{rubin1980} Rubin, V. C., Ford Jr, W. K., \& Thonnard, N. (1980). Rotationsgeschwindigkeiten von 21 Spiralgalaxien mit einem großen Bereich an Leuchtkräften und Radien, von NGC 4605 (R=4 kpc) bis UGC 2885 (R=122 kpc). \textit{The Astrophysical Journal}, 238, 471-487. DOI: 10.1086/158003.
	\bibitem{McGaugh2016} McGaugh, S. S., Lelli, F., \& Schombert, J. M. (2016). Zusammenhang der radialen Beschleunigung in rotationsgestützten Galaxien. \textit{Physical Review Letters}, 117(20), 201101. DOI: 10.1103/PhysRevLett.117.201101.
	\bibitem{Milgrom1983} Milgrom, M. (1983). Eine Modifikation der Newtonschen Dynamik als mögliche Alternative zur Hypothese der verborgenen Masse. \textit{The Astrophysical Journal}, 270, 365-370. DOI: 10.1086/161130.
	\bibitem{Planck2018} Planck Collaboration, Aghanim, N., et al. (2020). Planck 2018 Ergebnisse. VI. Kosmologische Parameter. \textit{Astronomy \& Astrophysics}, 641, A6. DOI: 10.1051/0004-6361/201833910.
\end{thebibliography}
	
\end{document}