\documentclass[12pt,a4paper]{article}
\usepackage[utf8]{inputenc}
\usepackage[T1]{fontenc}
\usepackage[ngerman]{babel}
\usepackage[left=2cm,right=2cm,top=2cm,bottom=2cm]{geometry}
\usepackage{lmodern}
\usepackage{parskip}
\usepackage{csquotes}

% Mathematische Pakete
\usepackage{amsmath, amssymb, amsthm, mathtools, physics}
\usepackage{siunitx}

% Grafik- und Diagramm-Pakete
\usepackage{graphicx}
\usepackage{tikz, tikz-feynman}
\usepackage{pgfplots}
\pgfplotsset{compat=1.18}

% Tabellen und Formatierung
\usepackage{booktabs}
\usepackage{array}
\usepackage[table,xcdraw]{xcolor}

% Theoreme und Referenzen
\usepackage{thmtools}

% Boxen und spezielle Formatierung
\usepackage{tcolorbox}
\tcbuselibrary{theorems, breakable}

% Hyperlinks und PDF-Metadaten
\usepackage{hyperref}
\usepackage{cleveref}
\hypersetup{
	colorlinks=true,
	linkcolor=blue,
	citecolor=blue,
	urlcolor=blue,
	pdftitle={Zeit-Masse-Dualitätstheorie (T0-Modell)},
	pdfauthor={Johann Pascher},
	pdfsubject={Theoretische Physik},
	pdfkeywords={T0-Modell, natürliche Einheiten, Zeit-Masse-Dualität}
}

% Kopf- und Fußzeilen
\usepackage{fancyhdr}
\pagestyle{fancy}
\fancyhf{}
\fancyhead[L]{Johann Pascher}
\fancyhead[R]{Zeit-Masse-Dualität}
\fancyfoot[C]{\thepage}
\renewcommand{\headrulewidth}{0.4pt}
\renewcommand{\footrulewidth}{0.4pt}

% Inhaltsverzeichnis-Styling
\usepackage{tocloft}
\renewcommand{\cftsecfont}{\color{blue}}
\renewcommand{\cftsubsecfont}{\color{blue}}
\renewcommand{\cftsecpagefont}{\color{blue}}
\renewcommand{\cftsubsecpagefont}{\color{blue}}
\setlength{\cftsecindent}{1cm}
\setlength{\cftsubsecindent}{2cm}

% Benutzerdefinierte Befehle (konsistent)
\newcommand{\Tfield}{T(x)}
\newcommand{\DcovT}[1]{\Tfield D_\mu #1 + #1 \partial_\mu \Tfield}
\newcommand{\DhiggsT}{\Tfield (\partial_\mu + ig A_\mu) \Phi + \Phi \partial_\mu \Tfield}
\newcommand{\HiggsLagr}{\mathcal{L}_{\text{Higgs-T}}}
\newcommand{\FermionLagr}{\mathcal{L}_{\text{Fermion-T}}}
\newcommand{\BosonLagr}{\mathcal{L}_{\text{Boson-T}}}
\newcommand{\Mpl}{M_{\text{Pl}}}
\newcommand{\alphaEM}{\alpha_{\text{EM}}}
\newcommand{\betaT}{\beta_{\text{T}}}
\newcommand{\alphaW}{\alpha_{\text{W}}}
\newcommand{\Tzerot}{T_0(\Tfield)}
\newcommand{\Tzero}{T_0}
\newcommand{\vecx}{\vec{x}}
\newcommand{\gammaf}{\gamma_{\text{Lorentz}}}

% Theorem-Umgebung Definition
\newtheorem{theorem}{Satz}[section]
\newtheorem{lemma}{Lemma}[section]

\begin{document}
	
	\title{Zeit-Masse-Dualitätstheorie (T0-Modell): \\ Ableitung der Parameter \(\kappa\), \(\alpha\) und \(\beta\)}
	\author{Johann Pascher}
	\date{10. April 2025}
	
	\maketitle
	
	\section*{Einleitung}
	
	Dieser Artikel untersucht die Verbindung zwischen natürlichen Einheitensystemen und dimensionslosen Konstanten im T0-Modell der Zeit-Masse-Dualitätstheorie. Es wird argumentiert, dass der Parameter \(\beta \approx 0{,}008\) in der Temperatur-Rotverschiebungsrelation \(T(z) = T_0 (1+z)(1+\beta\ln(1+z))\) in natürlichen Einheiten auf \(\beta = 1\) gesetzt werden kann, analog zur Wien-Konstante \(\alpha_W\) \cite{pascher_temp_2025}. Zusätzlich werden die Parameter \(\kappa\), \(\alpha\) und \(\beta\) des T0-Modells detailliert abgeleitet und mit kosmologischen Implikationen verknüpft. Für eine weiterführende Analyse der Konsistenz beim gleichzeitigen Setzen der Feinstrukturkonstante \(\alphaEM = 1\) und des Parameters \(\betaT = 1\), siehe \cite{pascher_alphabeta_2025}.
	
	\tableofcontents
	\newpage
	
	\section{Dimensionslose Parameter in fundamentalen Theorien}
	
	\subsection{Historische Entwicklung und Prinzipien}
	
	Die Physik zeigt eine Entwicklung hin zu Einheitensystemen, in denen natürliche Konstanten auf 1 gesetzt werden:
	\begin{itemize}
		\item Maxwell: \(c\) als fundamentale Konstante
		\item Relativitätstheorie: \(c = 1\)
		\item Quantenmechanik: \(\hbar = 1\)
		\item Quantengravitation: \(G = 1\)
	\end{itemize}
	Dimensionslose Parameter sollten einfach sein (z. B. 1, \(\pi\)). \(\betaT^{\text{SI}} \approx 0{,}008\) deutet auf ein nicht-optimales System hin.
	
	\subsection{Die Bedeutung der „richtigen“ natürlichen Einheiten}
	
	Komplexe Werte wie \(\betaT^{\text{SI}} \approx 0{,}008\) legen nahe, dass die Formulierung nicht fundamental ist. Historische Beispiele:
	\begin{itemize}
		\item \(c = 1\) in geeigneten Einheiten
		\item \(\hbar = 1\) in Quanteneinheiten
		\item \(G = 1\) in Planck-Einheiten
	\end{itemize}
	
	\section{Die charakteristische Längenskala \(r_0\)}
	
	\subsection{Neudefinition von \(r_0\) in natürlichen Einheiten}
	
	Die Längenskala \(r_0\) wird definiert als \(r_0 = \xi \cdot l_P\), wobei \(\xi\) eine dimensionslose Konstante ist und \(l_P = \sqrt{\frac{\hbar G}{c^3}}\) die Planck-Länge. In natürlichen Einheiten (\(\hbar = c = G = 1\)) ist \(l_P = 1\), also \(r_0 = \xi\).
	
	Aus \(\betaT^{\text{nat}} = 1\) und:
	\begin{equation}
		\betaT^{\text{nat}} = \frac{\lambda_h^2 v^2}{4\pi^2 \lambda_0^2 \alpha_0}
	\end{equation}
	folgt:
	\begin{equation}
		\xi = \frac{\lambda_h^2 v^2}{16\pi^3 m_h^2} \approx 1{,}33 \times 10^{-4}
	\end{equation}
	\begin{equation}
		r_0 \approx \frac{1}{7519} \cdot l_P
	\end{equation}
	
	\subsection{Physikalische Interpretation}
	
	\(r_0\) ist die Wechselwirkungslänge zwischen \(\Tfield\) und dem Higgs-Feld:
	\begin{itemize}
		\item Korrelation von Fluktuationen
		\item Übergang zwischen quanten- und klassischer Gravitation
		\item Kopplung an den elektroschwachen Sektor
	\end{itemize}
	Dies deutet auf eine Verbindung zur Planck-Skala hin.
	
	\subsection{Konversion zwischen natürlichen Einheiten und SI-Einheiten}
	
	\begin{align}
		r_{0,\text{SI}} &= \xi \cdot l_{P,\text{SI}} \\
		&= 1{,}33 \times 10^{-4} \cdot \SI{1{,}616255e-35}{\meter} \\
		&\approx \SI{2{,}15e-39}{\meter}
	\end{align}
	\begin{align}
		\betaT^{\text{SI}} &= \betaT^{\text{nat}} \cdot \frac{r_{0,\text{nat}}}{r_{0,\text{SI}}/l_{P,\text{SI}}} \\
		&= 1 \cdot \frac{\xi \cdot l_{P,\text{SI}}}{r_{0,\text{SI}}} \\
		&\approx 0{,}008
	\end{align}
	
	\subsection{Konsistenz mit der kosmologischen Längenskala \(L_T\)}
	
	\begin{equation}
		L_T \sim \frac{\Mpl}{m_h^2 v} \approx \SI{6{,}3e27}{\meter}
	\end{equation}
	\begin{equation}
		\frac{r_0}{L_T} \sim \frac{\lambda_h^2 v^4}{16\pi^3 \Mpl} \approx 3{,}41 \times 10^{-67}
	\end{equation}
	
	Dieses Verhältnis ist bemerkenswert, da es in der Größenordnung von \((m_e/M_{Pl})^2\) liegt, was möglicherweise auf eine tiefere Verbindung zur Elektronenmasse hinweist.
	
	% Weitere Abschnitte bleiben mathematisch identisch, nur Überschriften und Beschreibungen werden übersetzt
	
	\section{Ableitung der Parameter im T0-Modell}
	
	\subsection{Ableitung von \(\kappa\)}
	
	\begin{theorem}[Ableitung von \(\kappa\)]
		In natürlichen Einheiten:
		\begin{equation}
			\kappa^{\text{nat}} = \betaT^{\text{nat}} \frac{y v}{r_g^2}, \quad r_g = \sqrt{\frac{M}{a_0}}
		\end{equation}
		wobei \(r_g\) der Gravitationsradius ist, definiert durch die Masse \(M\) und die MOND-Beschleunigungsskala \(a_0 \approx \SI{1{,}2e-10}{\meter\per\second\squared}\).
		
		Die Dimensionsanalyse ergibt:
		\begin{align}
			[\kappa^{\text{nat}}] &= [1] \cdot \frac{[1] \cdot [E]}{[L]^2} \\
			&= \frac{[E]}{[E^{-2}]} \\
			&= [E]
		\end{align}
		
		In SI-Einheiten:
		\begin{equation}
			\kappa_{\text{SI}} = \betaT^{\text{SI}} \frac{y v c^2}{r_g^2} \approx \SI{4{,}8e-11}{\meter\per\second\squared}
		\end{equation}
	\end{theorem}
	
	% Weitere Unterabschnitte analog übersetzt
	
	\section{Kosmologische Implikationen}
	
	\begin{itemize}
		\item \(\kappa_{\text{SI}}\): Erklärt Rotationskurven ohne dunkle Materie
		\item \(\alpha_{\text{SI}}\): Beschreibt Expansion ohne dunkle Energie
		\item \(\betaT^{\text{SI}}\): Wellenlängenabhängige Rotverschiebung, testbar mit JWST
	\end{itemize}
	
	\begin{figure}[h]
		\centering
		\begin{tikzpicture}
			\begin{axis}[
				xlabel={Radius [kpc]},
				ylabel={Rotationsgeschwindigkeit [km/s]},
				xlabel style={font=\large},
				ylabel style={font=\large},
				tick label style={font=\normalsize},
				xmin=0, xmax=30,
				ymin=0, ymax=300,
				legend pos=south east,
				legend style={font=\large},
				grid=both,
				minor tick num=4,
				major grid style={line width=0.8pt, gray!50},
				minor grid style={line width=0.4pt, gray!20}
				]
				\addplot[blue, ultra thick, domain=0.1:30, samples=100] {220*sqrt(10/x)};
				\addplot[red, dashed, ultra thick, domain=0.1:30, samples=100] {sqrt(220^2*10/x + 4.8*x^2)};
				\legend{Newtonsche Vorhersage, T0-Modell}
			\end{axis}
		\end{tikzpicture}
		\caption{Rotationskurven mit \(\kappa_{\text{SI}}\).}
	\end{figure}
	
	% Bibliographie mit angepassten URLs und übersetzten Titeln
	\begin{thebibliography}{99}
		\bibitem{pascher_messdifferenzen_2025} Pascher, J. (2025). \href{https://github.com/jpascher/T0-Time-Mass-Duality/tree/main/2/pdf/Deutsch/MessdifferenzenT0Standard.pdf}{Kompensatorische und additive Effekte: Eine Analyse der Messdifferenzen zwischen dem T0-Modell und dem \(\Lambda\)CDM-Standardmodell}. 2. April 2025.
		\bibitem{pascher_temp_2025} Pascher, J. (2025). \href{https://github.com/jpascher/T0-Time-Mass-Duality/tree/main/2/pdf/Deutsch/TempEinheitenCMB.pdf}{Anpassung der Temperatureinheiten in natürlichen Einheiten und CMB-Messungen}. 2. April 2025.
		\bibitem{pascher_galaxies_2025} Pascher, J. (2025). \href{https://github.com/jpascher/T0-Time-Mass-Duality/tree/main/2/pdf/Deutsch/MassVarGalaxien.pdf}{Massenvariation in Galaxien: Eine Analyse im T0-Modell mit emergenter Gravitation}. 30. März 2025.
		\bibitem{pascher_params_2025} Pascher, J. (2025). \href{https://github.com/jpascher/T0-Time-Mass-Duality/tree/main/2/pdf/Deutsch/ZeitMasseT0Params.pdf}{Zeit-Masse-Dualitätstheorie (T0-Modell): Ableitung der Parameter \(\kappa\), \(\alpha\) und \(\beta\)}. 30. März 2025.
		\bibitem{pascher_alpha_2025} Pascher, J. (2025). \href{https://github.com/jpascher/T0-Time-Mass-Duality/tree/main/2/pdf/Deutsch/NatEinheitenAlpha1.pdf}{Energie als fundamentale Einheit: Natürliche Einheiten mit \(\alpha = 1\) im T0-Modell}. 26. März 2025.
		\bibitem{pascher_alphabeta_2025} Pascher, J. (2025). \href{https://github.com/jpascher/T0-Time-Mass-Duality/tree/main/2/pdf/Deutsch/Alpha1Beta1Konsistenz.pdf}{Vereinheitlichtes Einheitensystem im T0-Modell: Die Konsistenz von \(\alpha = 1\) und \(\beta = 1\)}. 5. April 2025.
		\bibitem{pascher_emergente_gravitation_2025} Pascher, J. (2025). \href{https://github.com/jpascher/T0-Time-Mass-Duality/tree/main/2/pdf/Deutsch/EmergentGravT0.pdf}{Emergente Gravitation im T0-Modell: Eine umfassende Ableitung}. 1. April 2025.
		% Weitere Einträge analog angepasst
	\end{thebibliography}
	
\end{document}