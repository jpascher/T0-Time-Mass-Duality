\documentclass[a4paper,12pt]{article}
\usepackage[utf8]{inputenc}
\usepackage[T1]{fontenc}
\usepackage{lmodern}
\usepackage[ngerman]{babel}
\usepackage{csquotes}
\usepackage{amsmath}
\usepackage{amsfonts}
\usepackage{amssymb}
\usepackage{physics}
\usepackage{geometry}
\usepackage{tocloft}
\usepackage{xcolor}
\usepackage{graphicx,tikz,pgfplots}
\pgfplotsset{compat=1.18}
\usepackage{booktabs}
\usepackage{array}
\usepackage{tabularx}
\usepackage{braket}
\usepackage{siunitx}
\DeclareSIUnit{\year}{yr}
\DeclareSIUnit{\parsec}{pc}
\usepackage{amsthm}
\usepackage[colorlinks=true, linkcolor=blue, citecolor=blue, urlcolor=blue]{hyperref}
\usepackage{cleveref}
\usepackage{fancyhdr}

\geometry{a4paper, margin=2cm}

\hypersetup{
	pdftitle={Feldtheorie und Quantenkorrelationen: Eine neue Perspektive auf Instantaneität},
	pdfauthor={Johann Pascher},
	pdfcreator={LaTeX}
}

% Kopf- und Fußzeilen
\pagestyle{fancy}
\fancyhf{}
\fancyhead[L]{Johann Pascher}
\fancyhead[R]{Zeit-Masse-Dualität}
\fancyfoot[C]{\thepage}
\renewcommand{\headrulewidth}{0.4pt}
\renewcommand{\footrulewidth}{0.4pt}

\renewcommand{\cftsecfont}{\color{blue}}
\renewcommand{\cftsubsecfont}{\color{blue}}
\renewcommand{\cftsecpagefont}{\color{blue}}
\renewcommand{\cftsubsecpagefont}{\color{blue}}
\setlength{\cftsecindent}{1cm}
\setlength{\cftsubsecindent}{2cm}

% Custom commands
\newcommand{\Tfield}{T(x)}
\newcommand{\DcovT}[1]{\Tfield D_\mu #1 + #1 \partial_\mu \Tfield}
\newcommand{\DhiggsT}{\Tfield (\partial_\mu + ig A_\mu) \Phi + \Phi \partial_\mu \Tfield}
\newcommand{\betaT}{\beta_{\text{T}}}
\newcommand{\alphaEM}{\alpha_{\text{EM}}}
\newcommand{\Mpl}{M_{\text{Pl}}}
\newcommand{\Tzerot}{T_0(\Tfield)}
\newcommand{\Tzero}{T_0}
\newcommand{\vecx}{\vec{x}}
\newcommand{\gammaf}{\gamma_{\text{Lorentz}}}

\newtheorem{theorem}{Satz}[section]
\newtheorem{proposition}[theorem]{Proposition}
\theoremstyle{definition}
\newtheorem{definition}{Definition}[theorem]
\theoremstyle{remark}
\newtheorem{remark}{Bemerkung}

\title{Feldtheorie und Quantenkorrelationen: \\Eine neue Perspektive auf Instantaneität}
\author{Johann Pascher}
\date{28. März 2025}

\begin{document}
	
	\maketitle
	
	\begin{abstract}
		Diese Arbeit entwickelt eine neue Perspektive auf Quantenkorrelationen und deren scheinbare Instantaneität im Rahmen des T0-Modells. Durch einen einheitlichen Feldansatz wird gezeigt, wie die nichtlokalen Eigenschaften der Quantenmechanik als natürliche Folge einer zugrundeliegenden Feldstruktur verstanden werden können. Besonderes Augenmerk liegt auf der Rolle des Quantenhintergrunds und der Interpretation moderner Bell-Experimente. Diese Sichtweise ergänzt die Zeit-Masse-Dualitätstheorie und bietet einen kohärenten Rahmen, um quantenmechanische Phänomene ohne die Annahme von „spukhafter Fernwirkung“ zu erklären.
	\end{abstract}
	
	\tableofcontents
	\newpage
	
	\section{Einführung}
	
	Die Quantenmechanik stellt uns seit jeher vor Rätsel, insbesondere wenn es um die Natur von Quantenkorrelationen geht. Die Vorstellung, dass zwei weit voneinander entfernte Teilchen augenblicklich miteinander verbunden sein könnten, hat Wissenschaftler seit Einsteins berühmter Kritik an der „spukhaften Fernwirkung“ beschäftigt. Moderne Experimente, wie die lückenlosen Bell-Tests ab 2015, haben gezeigt, dass diese Korrelationen real sind und die klassischen Vorstellungen von Lokalität und Kausalität übertreffen. Doch wie können wir dieses Phänomen verstehen, ohne die Grundpfeiler der Physik aufzugeben?
	
	In dieser Arbeit schlage ich vor, die Nichtlokalität der Quantenwelt nicht als mysteriöse Instantaneität zu betrachten, sondern als natürliche Eigenschaft eines einheitlichen Quantenfelds, das im Rahmen des T0-Modells entwickelt wurde. Dieses Modell, das auf der Zeit-Masse-Dualität basiert und die Zeit als absolute Größe mit variabler Masse behandelt, bietet eine frische Perspektive. Anstatt Teilchen als isolierte Objekte zu sehen, interpretiere ich sie als Knoten oder Anregungen eines fundamentalen Feldes, dessen Kohärenz die beobachteten Korrelationen erklärt. Diese Sichtweise knüpft an die Arbeiten zur Zeit-Masse-Dualität an, die in „Zeit-Masse-Dualitätstheorie: Herleitung der Parameter“ \cite{pascher_params_2025} ausführlich beschrieben sind, und erweitert sie um eine feldtheoretische Grundlage. Ein zentrales Element ist das intrinsische Zeitfeld \(\Tfield = \frac{\hbar}{\max(m c^2, \omega)}\), das die Zeitskala der Feldknoten bestimmt und eine Brücke zwischen Quantenmechanik und Kosmologie schlägt.
	
	Mein Ansatz beginnt mit der Idee, dass das Vakuum kein leerer Raum ist, sondern ein aktiver Quantenhintergrund mit definierten Eigenschaften, wie sie durch die elektrischen und magnetischen Feldkonstanten \(\varepsilon_0\) und \(\mu_0\) zum Ausdruck kommen. Teilchen sind keine eigenständigen Entitäten, sondern stabile Muster dieses Feldes, während Quantenkorrelationen die inhärente Kohärenz des Feldzustands widerspiegeln. Moderne Experimente, wie die Wiener Tests von 2015 \cite{Giustina2015} oder der „Big Bell Test“ von 2018 \cite{BigBellTest2018}, unterstützen diese Interpretation, indem sie die Nichtlokalität als Faktum bestätigen, ohne dass eine instantane Kommunikation zwischen Teilchen erforderlich ist. Um diese Perspektive mathematisch zu untermauern, führe ich eine fundamentale Feldgleichung ein, die die Dynamik dieses einheitlichen Feldes beschreibt, und verbinde sie mit der modifizierten Quantenmechanik des T0-Modells, wie sie in „Die Notwendigkeit der Erweiterung der Standard-Quantenmechanik“ \cite{pascher_quantum_2025} entwickelt wurde. Diese Arbeit zielt darauf ab, eine kohärente Erzählung zu schaffen, die sowohl die experimentellen Befunde als auch die theoretischen Grundlagen des T0-Modells vereint.
	
	\section{Das Vakuum als Quantenhintergrund}
	
	Das Vakuum, wie wir es in der modernen Physik verstehen, ist weit mehr als ein leerer Raum. Es ist ein dynamisches Medium, das durch fundamentale physikalische Eigenschaften wie die elektrische Feldkonstante \(\varepsilon_0\) und die magnetische Feldkonstante \(\mu_0\) charakterisiert wird. Diese Konstanten sind nicht bloß mathematische Hilfsmittel, sondern Ausdruck einer tiefen Struktur, die die Lichtgeschwindigkeit definiert und die Wechselwirkungen aller Felder im Universum ermöglicht. Im T0-Modell wird dieses Vakuum als aktiver Quantenhintergrund betrachtet, der die Grundlage für alle physikalischen Phänomene bildet, einschließlich der Quantenkorrelationen. Eine zentrale Beziehung, die diese Rolle unterstreicht, ist:
	
	\begin{equation}
		c = \frac{1}{\sqrt{\varepsilon_0 \mu_0}}
	\end{equation}
	
	Dieser Hintergrund ist kein passiver Schauplatz, sondern ein Trägermedium, das elektromagnetische Wellen und andere fundamentale Felder ermöglicht. Seine Homogenität sorgt dafür, dass die Lichtgeschwindigkeit konstant bleibt, wie es die spezielle Relativitätstheorie fordert. Doch im T0-Modell geht die Bedeutung des Vakuums weiter: Es beeinflusst direkt das intrinsische Zeitfeld \(\Tfield\), das die Zeitskala jedes Teilchens bestimmt, wie in „Zeit-Masse-Dualitätstheorie“ \cite{pascher_params_2025} gezeigt wird. Das Vakuum wird somit zum Schlüssel, um die nichtlokalen Eigenschaften der Quantenwelt zu verstehen, indem es als kohärentes Medium fungiert, das Korrelationen über weite Distanzen hinweg aufrechterhält.
	
	\section{Quantenkorrelationen im Feldmodell}
	
	Wenn wir von verschränkten Teilchen sprechen, denken wir oft an Photonen, deren Polarisation in einem gemeinsamen Zustand beschrieben wird. Ein typischer verschränkter Zustand lautet:
	
	\begin{equation}
		|\psi\rangle = \frac{1}{\sqrt{2}} (|H\rangle_A |H\rangle_B + |V\rangle_A |V\rangle_B)
	\end{equation}
	
	In der traditionellen Sichtweise erscheinen diese Teilchen als separate Objekte, deren Zustände instantan miteinander korrelieren, sobald eine Messung erfolgt. Doch im Feldmodell des T0-Systems ändert sich diese Sichtweise grundlegend. Verschränkte Zustände sind keine Eigenschaften isolierter Teilchen, sondern kohärente Muster eines einheitlichen Quantenfelds, das den Raum durchdringt, wie in „Dynamische Masse von Photonen“ \cite{pascher_photons_2025} beschrieben.
	
	Die Bell-Ungleichungen, die John Bell 1964 formulierte, zeigen klar, dass lokale realistische Theorien die beobachteten Korrelationen nicht erklären können. Mathematisch ausgedrückt lautet eine solche Ungleichung:
	
	\begin{equation}
		|E(a,b) - E(a,c)| \leq 1 + E(b,c)
	\end{equation}
	
	In Experimenten wird diese Ungleichung regelmäßig verletzt, wie Alain Aspect 1982 und spätere Tests bewiesen haben \cite{Aspect1982}. Im Feldmodell ist diese Verletzung keine Überraschung: Das Quantenfeld ist von Natur aus nichtlokal, weil es eine globale Struktur besitzt, die lokale Messungen verbindet, ohne dass eine Signalübertragung erforderlich ist.
	
	Die Wiener Experimente von 2015 unter der Leitung von Anton Zeilinger waren ein Meilenstein, da sie alle klassischen Lücken – wie Detektionseffizienz oder räumliche Trennung – schlossen \cite{Giustina2015}. Mit einer Signifikanz von über 11 Standardabweichungen bestätigten sie die Nichtlokalität. Ebenso beeindruckend war der „Big Bell Test“ von 2018, bei dem über 100.000 Menschen weltweit die Messungseinstellungen steuerten, um die Freiheitswahl-Lücke zu adressieren \cite{BigBellTest2018}. Diese Experimente zeigen, dass Quantenkorrelationen real sind, und im T0-Feldmodell finden sie eine natürliche Erklärung als Eigenschaften eines kohärenten Quantenfelds.
	
	\section{Feldtheorie und Instantaneität}
	
	Um die Idee der Instantaneität greifbarer zu machen, ziehe ich eine Analogie zu Schallwellen heran. Wenn eine Schallwelle durch einen Raum läuft, ist sie überall präsent, doch ein Mikrofon misst nur die lokale Schwingung. Zwei Mikrofone, die dieselbe Welle aufnehmen, zeigen eine Korrelation, die nicht durch eine instantane Kommunikation zwischen ihnen entsteht, sondern durch die gemeinsame Struktur der Welle. Ähnlich verhält es sich im Quantenfeldmodell: Verschränkte Teilchen sind Knoten eines globalen Felds, und ihre Korrelationen sind bereits im Feldzustand enthalten, bevor eine Messung erfolgt.
	
	Diese Analogie löst das Paradoxon der Nichtlokalität auf. Es gibt keine „Fernwirkung“, sondern eine inhärente Kohärenz des Feldes, die durch das intrinsische Zeitfeld \(\Tfield\) gesteuert wird, wie es in „Dynamische Masse von Photonen“ \cite{pascher_photons_2025} beschrieben wird. Das Higgs-Feld spielt hierbei eine zentrale Rolle, indem es die Masse und damit die Zeitskala der Feldknoten definiert, was die beobachteten Korrelationen mit der Zeit-Masse-Dualität verknüpft.
	
	\section{Feldgleichungen in dualer Formulierung}
	
	Die Quantenmechanik im T0-Modell wird durch eine modifizierte Schrödinger-Gleichung beschrieben, die die variable Masse einbezieht. Im Gegensatz zur klassischen Form \(i\hbar \frac{\partial}{\partial t} \Psi = \hat{H} \Psi\) lautet sie hier:
	
	\begin{equation}
		i\hbar \Tfield \frac{\partial}{\partial t} \Psi + i\hbar \Psi \frac{\partial \Tfield}{\partial t} = \hat{H} \Psi
	\end{equation}
	
	Dieser Ansatz, der in „Die Notwendigkeit der Erweiterung der Standard-Quantenmechanik“ \cite{pascher_quantum_2025} entwickelt wurde, reflektiert die Zeit-Masse-Dualität und integriert sie in die Feldtheorie. Die Gesamt-Lagrangedichte des Modells lautet:
	\begin{equation}
		\mathcal{L}_{\text{Total}} = \mathcal{L}_{\text{Boson}} + \mathcal{L}_{\text{Fermion}} + \mathcal{L}_{\text{Higgs-T}} + \mathcal{L}_{\text{intrinsic}}
	\end{equation}
	wobei \(\mathcal{L}_{\text{intrinsic}} = \frac{1}{2} \partial_\mu \Tfield \partial^\mu \Tfield - V(\Tfield)\) die Dynamik des Zeitfelds beschreibt, wie in „Mathematische Kernformulierungen“ \cite{pascher_lagrange_2025} ausgeführt.
	
	\section{Kosmologische Implikationen}
	
	Das T0-Modell hat weitreichende Implikationen für die Kosmologie, die mit der Feldperspektive übereinstimmen. Das Gravitationspotential wird modifiziert zu \(\Phi(r) = -\frac{G M}{r} + \kappa r\), wobei \(\kappa \approx \SI{4.8e-11}{\meter\per\second\squared}\) aus der \(\Tfield\)-Dynamik emergiert, wie in „Massenvariation in Galaxien“ \cite{pascher_galaxies_2025} gezeigt. Die kosmische Rotverschiebung wird als Energieverlust beschrieben: \(1 + z = e^{\alpha d}\), mit \(\alpha \approx \SI{2.3e-18}{\per\meter}\), wie in „Messdifferenzen“ \cite{pascher_messdifferenzen_2025} hergeleitet. Eine wellenlängenabhängige Rotverschiebung ergibt sich mit \(z(\lambda) = z_0 (1 + \betaT \ln(\lambda/\lambda_0))\), wobei \(\betaT^{\text{SI}} \approx 0.008\) und \(\betaT^{\text{nat}} = 1\) ist, wie in „Parameterableitungen“ \cite{pascher_params_2025} festgelegt. Diese Effekte zeigen, wie das Quantenfeldmodell nahtlos in die kosmologischen Aspekte des T0-Modells integriert werden kann.
	
	\section{Schlussfolgerung}
	
	Das T0-Modell bietet eine neue Sichtweise auf Quantenkorrelationen, indem es sie als natürliche Eigenschaften eines einheitlichen Quantenfelds erklärt, das durch das intrinsische Zeitfeld \(\Tfield\) gesteuert wird. Diese Perspektive löst das Paradoxon der Instantaneität auf, indem sie Nichtlokalität als inhärente Kohärenz des Feldes interpretiert, unterstützt durch moderne Bell-Experimente wie die von Zeilinger und dem „Big Bell Test“. Durch die Integration mit der Zeit-Masse-Dualität, wie sie in den Arbeiten zur Quantenmechanik und Kosmologie des T0-Modells entwickelt wurde, entsteht ein kohärenter Rahmen, der die Grenzen zwischen Quantenphysik und Feldtheorie überwindet.
	
	\begin{thebibliography}{99}
		\bibitem{pascher_params_2025} Pascher, J. (2025). \href{https://github.com/jpascher/T0-Time-Mass-Duality/tree/main/2/pdf/Deutsch/ZeitMasseT0Params.pdf}{Zeit-Masse-Dualitätstheorie (T0-Modell): Ableitung der Parameter \(\kappa\), \(\alpha\) und \(\beta\)}. 4. April 2025.
		\bibitem{pascher_galaxies_2025} Pascher, J. (2025). \href{https://github.com/jpascher/T0-Time-Mass-Duality/tree/main/2/pdf/Deutsch/MassVarGalaxien.pdf}{Massenvariation in Galaxien: Eine Analyse im T0-Modell mit emergenter Gravitation}. 30. März 2025.
		\bibitem{pascher_messdifferenzen_2025} Pascher, J. (2025). \href{https://github.com/jpascher/T0-Time-Mass-Duality/tree/main/2/pdf/Deutsch/MessdifferenzenT0Standard.pdf}{Kompensatorische und additive Effekte: Eine Analyse der Messdifferenzen zwischen dem T0-Modell und dem \(\Lambda\)CDM-Standardmodell}. 2. April 2025.
		\bibitem{pascher_lagrange_2025} Pascher, J. (2025). \href{https://github.com/jpascher/T0-Time-Mass-Duality/tree/main/2/pdf/Deutsch/MathZeitMasseLagrange.pdf}{Von Zeitdilatation zu Massenvariation: Mathematische Kernformulierungen der Zeit-Masse-Dualitätstheorie}. 29. März 2025.
		\bibitem{pascher_photons_2025} Pascher, J. (2025). \href{https://github.com/jpascher/T0-Time-Mass-Duality/tree/main/2/pdf/Deutsch/DynMassePhotonenNichtlokal.pdf}{Dynamische Masse von Photonen und ihre Auswirkungen auf Nichtlokalität im T0-Modell}. 25. März 2025.
		\bibitem{pascher_quantum_2025} Pascher, J. (2025). \href{https://github.com/jpascher/T0-Time-Mass-Duality/tree/main/2/pdf/Deutsch/NotwendigkeitQMErweiterung.pdf}{Die Notwendigkeit der Erweiterung der Standard-Quantenmechanik und Quantenfeldtheorie}. 27. März 2025.
		\bibitem{Hensen2015} Hensen, B., et al. (2015). \textit{Loophole-free Bell inequality violation using electron spins separated by 1.3 kilometres}. Nature, 526, 682-686.
		\bibitem{Giustina2015} Giustina, M., et al. (2015). \textit{Significant-loophole-free test of Bell’s theorem with entangled photons}. Physical Review Letters, 115, 250401.
		\bibitem{BigBellTest2018} The BIG Bell Test Collaboration. (2018). \textit{Challenging local realism with human choices}. Nature, 557, 212-216.
		\bibitem{Bell1964} Bell, J. S. (1964). \textit{On the Einstein-Podolsky-Rosen paradox}. Physics, 1(3), 195-200.
		\bibitem{Aspect1982} Aspect, A., et al. (1982). \textit{Experimental test of Bell's inequalities using time-varying analyzers}. Physical Review Letters, 49, 1804-1807.
		\bibitem{Wilczek2008} Wilczek, F. (2008). \textit{The Lightness of Being: Mass, Ether, and the Unification of Forces}. Basic Books.
		\bibitem{Milonni1994} Milonni, P. W. (1994). \textit{The Quantum Vacuum: An Introduction to Quantum Electrodynamics}. Academic Press.
		\bibitem{Aitchison2004} Aitchison, I. J. R. (2004). \textit{An Informal Introduction to Gauge Field Theories}. Cambridge University Press.
		\bibitem{Weinberg1995} Weinberg, S. (1995). \textit{The Quantum Theory of Fields}. Cambridge University Press.
		\bibitem{Fox2006} Fox, M. (2006). \textit{Quantum Optics: An Introduction}. Oxford University Press.
		\bibitem{Zeilinger2010} Zeilinger, A. (2010). \textit{Dance of the Photons: From Einstein to Quantum Teleportation}. Farrar, Straus and Giroux.
		\bibitem{Bohm1980} Bohm, D. (1980). \textit{Wholeness and the Implicate Order}. Routledge.
	\end{thebibliography}
	
\end{document}