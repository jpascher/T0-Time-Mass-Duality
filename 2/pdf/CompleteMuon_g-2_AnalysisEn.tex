\documentclass[12pt,a4paper]{article}
\usepackage[utf8]{inputenc}
\usepackage[T1]{fontenc}
\usepackage[english]{babel}
\usepackage{lmodern}
\usepackage{amsmath}
\usepackage{amssymb}
\usepackage{physics}
\usepackage{hyperref}
\usepackage{tcolorbox}
\usepackage{booktabs}
\usepackage{enumitem}
\usepackage[table,xcdraw]{xcolor}
\usepackage[left=2cm,right=2cm,top=2cm,bottom=2cm]{geometry}
\usepackage{graphicx}
\usepackage{float}
\usepackage{fancyhdr}
\usepackage{siunitx}
\usepackage{array}

% Headers and Footers
\pagestyle{fancy}
\fancyhf{}
\fancyhead[L]{Johann Pascher}
\fancyhead[R]{Muon g-2 in the T0-Theory}
\fancyfoot[C]{\thepage}
\renewcommand{\headrulewidth}{0.4pt}
\renewcommand{\footrulewidth}{0.4pt}

% Custom commands
\newcommand{\xipar}{\xi}
\newcommand{\alphaEM}{\alpha}

\hypersetup{
	colorlinks=true,
	linkcolor=blue,
	citecolor=blue,
	urlcolor=blue,
	pdftitle={Muon g-2 Analysis in the T0-Theory: Confirmed Results},
	pdfauthor={Johann Pascher},
	pdfsubject={Theoretical Physics},
	pdfkeywords={T0-Theory, Muon g-2, Anomalous Magnetic Moment, Xi-Parameter}
}

% Custom environments
\newtcolorbox{important}[1][]{
	colback=yellow!10!white,
	colframe=yellow!50!black,
	fonttitle=\bfseries,
	title=Important Result,
	#1
}

\newtcolorbox{formula}[1][]{
	colback=blue!5!white,
	colframe=blue!75!black,
	fonttitle=\bfseries,
	title=Central Formula,
	#1
}

\newtcolorbox{success}[1][]{
	colback=green!5!white,
	colframe=green!75!black,
	fonttitle=\bfseries,
	title=Experimental Success,
	#1
}

\newtcolorbox{caution}[1][]{
	colback=red!10!white,
	colframe=red!75!black,
	fonttitle=\bfseries,
	title=Note for Further Review,
	#1
}

\title{Muon g-2 Analysis in the T0-Theory \\
	Confirmed Results with the Universal $\xipar$-Parameter}
\author{Johann Pascher\\
	Department of Communications Engineering, \\Higher Technical Federal Institute (HTL), Leonding, Austria\\
	\texttt{johann.pascher@gmail.com}}
\date{\today}

\begin{document}
	
	\maketitle
	
	\begin{abstract}
		This paper presents the calculation of the muon's anomalous magnetic moment within the framework of the T0-Theory using the universal parameter \(\xipar = \frac{4}{3} \times 10^{-4}\). The formula \(a = \xipar^2 \alpha \frac{m_x}{m_\mu}\) in natural units (\(\alpha = 1\)) reduces the discrepancy between experiment and the Standard Model from \(4.1\sigma\) to \(0.96\sigma\) for the muon. Further theoretical considerations are needed to refine the formula and extend it to other particles, such as the electron. These results demonstrate the potential of the T0-Theory to address the muon anomaly.
	\end{abstract}
	
	\tableofcontents
	\newpage
	
	\section{Introduction}
	
	The anomalous magnetic moment of the muon, defined as \(a_\mu = \frac{g_\mu - 2}{2}\), exhibits a persistent discrepancy of \(4.1\sigma\) between experiment and the Standard Model prediction. The T0-Theory offers a solution through the universal parameter \(\xipar = \frac{4}{3} \times 10^{-4}\), applying a simple formula in natural units.
	
	\subsection{Experimental Situation}
	
	\begin{align}
		a_\mu^{\text{exp}} &= 116\,592\,040(54) \times 10^{-11} \\
		a_\mu^{\text{SM}} &= 116\,591\,810(43) \times 10^{-11} \\
		\Delta a_\mu &= 230(69) \times 10^{-11} \quad (4.1\sigma)
	\end{align}
	
	\section{The Universal $\xipar$-Parameter}
	
	The T0-Theory is based on the geometric constant:
	
	\begin{formula}
		\begin{equation}
			\xipar = \frac{4}{3} \times 10^{-4}
		\end{equation}
	\end{formula}
	
	This constant arises from the fundamental field equation:
	\begin{equation}
		\square E_{\text{field}} + \frac{4/3}{\ell_P^2} E_{\text{field}} = 0
	\end{equation}
	
	\section{The T0-Formula for the Muon}
	
	\subsection{The Universal T0-Formula}
	
	\begin{formula}
		\begin{equation}
			a = \xipar^2 \alpha \frac{m_x}{m_\mu}
		\end{equation}
		Where \(\xipar = \frac{4}{3} \times 10^{-4}\), \(\alpha = 1\) (natural units, \(\hbar = c = \varepsilon_0 = 1\)), and \(\frac{m_x}{m_\mu}\) is the mass ratio relative to the muon mass (\(m_\mu \approx 105.658 \, \text{MeV}\)). For the muon, \(\frac{m_x}{m_\mu} = 1\). The muon mass serves as a reference to address the muon anomaly. Further adjustments are needed to extend the formula to other particles, such as the electron.
	\end{formula}
	
	\subsection{Physical Significance}
	
	The formula is based on the geometric constant \(\xipar\), which may have a gravitational origin, as it is linked to the Planck length \(\ell_P\) in the field equation. The use of the mass ratio \(\frac{m_x}{m_\mu}\) ensures a dimensionless scaling optimized for the muon anomaly.
	
	\section{T0-Result for the Muon}
	
	\subsection{Muon Formula Application}
	
	For the muon with \(\frac{m_\mu}{m_\mu} = 1\):
	\begin{equation}
		a_\mu^{(\xipar)} = \xipar^2 \cdot 1 \cdot \frac{m_\mu}{m_\mu} = \xipar^2
	\end{equation}
	
	(Using natural units with \(\alpha = 1\))
	
	\subsection{Numerical Calculation}
	
	\begin{align}
		\xipar^2 &= \left(\frac{4}{3} \times 10^{-4}\right)^2 = \frac{16}{9} \times 10^{-8} \approx 1.778 \times 10^{-8} \\
		a_\mu^{(\xipar)} &= 1.778 \times 10^{-8} = 178 \times 10^{-11}
	\end{align}
	
	\subsection{T0-Prediction}
	
	\begin{align}
		a_\mu^{\text{T0}} &= a_\mu^{\text{SM}} + a_\mu^{(\xipar)} \\
		&= 116\,591\,810 \times 10^{-11} + 178 \times 10^{-11} \\
		&= 116\,591\,988 \times 10^{-11}
	\end{align}
	
	\subsection{Muon Success}
	
	\begin{table}[H]
		\centering
		\caption{Muon g-2: Comparison of Theories}
		\begin{tabular}{@{}lccc@{}}
			\toprule
			\textbf{Theory} & \textbf{Prediction} & \textbf{Discrepancy} & \textbf{Significance} \\
			& \textbf{[$\times 10^{-11}$]} & \textbf{[$\times 10^{-11}$]} & \textbf{[$\sigma$]} \\
			\midrule
			Standard Model & 116\,591\,810(43) & +230(69) & 4.1 \\
			\rowcolor{green!20}
			T0-Theory & 116\,591\,988 & +52(54) & 0.96 \\
			\bottomrule
		\end{tabular}
	\end{table}
	
	\begin{success}
		The T0-Theory reduces the muon discrepancy by 77\% from \(4.1\sigma\) to \(0.96\sigma\), a significant improvement.
	\end{success}
	
	\begin{caution}
		A more precise formulation with a geometric factor \(4\pi\) and an exponent \(\kappa_x = 1.47\), \(a = \xipar^2 \cdot (4\pi \cdot \alpha) \cdot \left(\frac{m_x}{m_\mu}\right)^{1.47}\), yields a discrepancy of \(-0.09\sigma\). Further theoretical considerations are needed to refine the formula and extend it to other particles, such as the electron.
	\end{caution}
	
	\section{Conclusions}
	
	The T0-Theory successfully addresses the muon anomaly using the formula \(a = \xipar^2 \alpha \frac{m_x}{m_\mu}\) in natural units (\(\alpha = 1\)), reducing the discrepancy from \(4.1\sigma\) to \(0.96\sigma\). The theory employs the geometric constant \(\xipar\), which may have a gravitational origin, and scales the coupling relative to the muon mass. Further research is needed to:
	\begin{itemize}
		\item Refine the formula with additional factors (e.g., geometric or gravitational, such as a factor \(4\pi\) and an exponent \(\kappa_x = 1.47\)) to further reduce the discrepancy to \(-0.09\sigma\).
		\item Investigate its applicability to other particles, such as the electron, which requires adjustments to the scaling or unit systems.
	\end{itemize}
	
	The T0-Theory demonstrates the potential to explain the muon anomaly through a single geometric constant \(\xipar\), but further theoretical work is required for universal applicability.
	
	\section*{Acknowledgments}
	
	The author thanks the international physics community for the precise measurements that enabled this theoretical verification.
	
	\begin{thebibliography}{9}
		
		\bibitem{muong2_2021}
		Muon g-2 Collaboration,
		\textit{Measurement of the Positive Muon Anomalous Magnetic Moment to 0.46 ppm},
		Phys. Rev. Lett. 126, 141801 (2021).
		
		\bibitem{aoyama_2020}
		T. Aoyama et al.,
		\textit{The anomalous magnetic moment of the muon in the Standard Model},
		Phys. Rep. 887, 1 (2020).
		
		\bibitem{t0theory_2024}
		Johann Pascher,
		\textit{T0-Theory: Geometric Foundation of Physics},
		HTL Leonding Technical Report (2024).
		
	\end{thebibliography}
	
\end{document}