\documentclass{article}
\usepackage[utf8]{inputenc}
\usepackage{amsmath}
\usepackage{amssymb}
\usepackage{geometry}
\usepackage{hyperref}
\usepackage{siunitx}

\title{The Fine-Structure Constant: Various Representations and Relationships}
\author{Johann Pascher}
\date{03-21-2025}

\begin{document}
	
	\maketitle
	
	\section{Introduction to the Fine-Structure Constant}
	
	The fine-structure constant ($\alpha$) is a dimensionless physical constant that plays a fundamental role in quantum electrodynamics. It describes the strength of the electromagnetic interaction between elementary particles. In its most well-known form, the formula is:
	
	$$\alpha = \frac{e^2}{4\pi\varepsilon_0\hbar c} \approx \frac{1}{137.035999}$$
	\tableofcontents
	\section{Explanation of Symbols and Units Used}
	
	- $\alpha$: Fine-structure constant (dimensionless)
	- $e$: Elementary charge (unit: Coulomb, C)
	- $\varepsilon_0$: Electric field constant (unit: Farad per meter, F/m)
	- $\mu_0$: Magnetic field constant (unit: Newton per Ampere squared, N/A²)
	- $\hbar$: Reduced Planck's constant (unit: Joule-second, Js)
	- $c$: Speed of light (unit: Meter per second, m/s)
	(unit: \si{\meter\per\second})
	- $h$: Planck's constant (unit: Joule-second, Js)
	- $m_e$: Electron mass (unit: Kilogram, kg)
	- $\lambda_C$: Compton wavelength (unit: Meter, m)
	- $Q$: Electric charge (unit: Coulomb, C)
	- $C$: Capacitance (unit: Farad, F)
	- $V$: Voltage (unit: Volt, V)
	- $E$: Energy (unit: Joule, J)
	
	\section{Alternative Formulations of the Fine-Structure Constant}
	
	\subsection{Representation with Permeability}
	Starting from the standard form, we can replace the electric field constant $\varepsilon_0$ with the magnetic field constant $\mu_0$ by using the relationship $c^2 = \frac{1}{\varepsilon_0\mu_0}$:
	
	$$\varepsilon_0 = \frac{1}{\mu_0c^2}$$
	$$\alpha = \frac{e^2}{4\pi\left(\frac{1}{\mu_0c^2}\right)\hbar c}$$
	$$= \frac{e^2\mu_0c^2}{4\pi\hbar c}$$
	$$= \frac{e^2\mu_0c}{4\pi\hbar}$$
	
	Using the relationship $\hbar = \frac{h}{2\pi}$, we obtain an alternative form:
	
	$$\alpha = \frac{\mu_0e^2}{2h}$$
	
	\subsection{Formulation with Electron Mass and Compton Wavelength}
	Planck's constant $h$ can be expressed through other physical quantities:
	
	$$h = \frac{m_e c \lambda_C}{2\pi}$$
	
	where $\lambda_C$ is the Compton wavelength of the electron:
	
	$$\lambda_C = \frac{h}{m_e c}$$
	
	Substituting this into the fine-structure constant:
	
	$$\alpha = \frac{\mu_0e^2}{2h}$$
	$$= \frac{\mu_0e^2}{2\frac{m_e c \lambda_C}{2\pi}}$$
	$$= \frac{\mu_0e^2 \cdot 2\pi}{2m_e c \lambda_C}$$
	$$= \frac{\mu_0e^2\pi}{m_e c \lambda_C}$$
	
	\subsection{Expression with Classical Electron Radius}
	The classical electron radius is defined as:
	
	$$r_e = \frac{e^2}{4\pi\varepsilon_0 m_e c^2}$$
	
	Using $\varepsilon_0 = \frac{1}{\mu_0c^2}$, it follows:
	
	$$r_e = \frac{e^2\mu_0}{4\pi m_e c^2}$$
	
	The fine-structure constant can be written as the ratio of the classical electron radius to the Compton wavelength:
	
	$$\alpha = \frac{r_e}{\lambda_C}$$
	
	This leads to another form:
	
	$$\alpha = \frac{e^2\mu_0}{4\pi m_e c^2} \cdot \frac{2\pi m_e c}{h}$$
	$$= \frac{e^2\mu_0}{2hc}$$
	
	\subsection{Formulation with $\mu_0$ and $\varepsilon_0$ as Fundamental Constants}
	Using the relationship $c = \frac{1}{\sqrt{\mu_0\varepsilon_0}}$, the fine-structure constant can be expressed as:
	
	$$\alpha = \frac{e^2}{4\pi\varepsilon_0\hbar c} \cdot \sqrt{\mu_0\varepsilon_0}$$
	$$= \frac{e^2}{4\pi\varepsilon_0\hbar} \cdot \sqrt{\mu_0\varepsilon_0}$$
	
	These different representations allow for various physical interpretations and demonstrate the connections between fundamental natural constants.
	
	
	\section{Derivation of Planck's Constant Through Fundamental Electromagnetic Constants}
	
	\subsection{Dimensional Interchangeability: A New Perspective}
	
	Before beginning the derivation, it is important to explain the fundamental assumption of this work: In a fundamental theory, the dimensions of quantum quantities (such as Planck's constant) and electromagnetic quantities (such as permittivity and permeability) might be interchangeable or reducible to a common basis.
	
	This assumption is based on the observation that nature, at its most fundamental level, might require a unified description in which the seemingly different dimensions of quantum mechanics and electrodynamics appear as different manifestations of the same underlying structure. In such a theory, the established dimensional barriers would not be absolute but rather artifacts of our macroscopic perspective.
	
	This interchangeability of dimensions allows us to establish a direct connection between $h$ and the electromagnetic constants, going beyond conventional physics and possibly hinting at a deeper structure of the universe.
	
	\subsection{Relationship Between $h$, $\mu_0$, and $\varepsilon_0$}
	
	First, we consider the fundamental relationship between the speed of light $c$, permeability $\mu_0$, and permittivity $\varepsilon_0$:
	(unit: \si{\meter\per\second})
	$$c = \frac{1}{\sqrt{\mu_0\varepsilon_0}}$$
	
	We also use the fundamental relationship between Planck's constant $h$ and the Compton wavelength $\lambda_C$ of the electron:
	
	$$h = \frac{m_e c \lambda_C}{2\pi}$$
	
	The Compton wavelength is defined as:
	
	$$\lambda_C = \frac{h}{m_e c}$$
	
	By substituting the speed of light $c = \frac{1}{\sqrt{\mu_0\varepsilon_0}}$, we obtain:
	(unit: \si{\meter\per\second})
	$$h = \frac{m_e}{2\pi} \cdot \frac{\lambda_C}{\sqrt{\mu_0\varepsilon_0}}$$
	
	Now we replace $\lambda_C$ with its definition:
	
	$$h = \frac{m_e}{2\pi} \cdot \frac{h}{m_e c \sqrt{\mu_0\varepsilon_0}}$$
	
	This leads to:
	
	$$h^2 = \frac{1}{\mu_0\varepsilon_0} \cdot \frac{m_e^2 \lambda_C^2}{4\pi^2}$$
	
	Using $\lambda_C = \frac{h}{m_e c}$, it follows:
	
	$$h^2 = \frac{1}{\mu_0\varepsilon_0} \cdot \frac{m_e^2}{4\pi^2} \cdot \frac{h^2}{m_e^2c^2}$$
	
	After canceling $m_e^2$ and substituting $c^2 = \frac{1}{\mu_0\varepsilon_0}$, we finally obtain:
	
	$$h = \frac{1}{2\pi\sqrt{\mu_0\varepsilon_0}}$$
	
	This equation shows that, under the assumption of dimensional interchangeability, Planck's constant $h$ can indeed be expressed through the electromagnetic vacuum constants $\mu_0$ and $\varepsilon_0$. This relationship suggests a deeper connection between quantum mechanics and electrodynamics.
	
	For the reduced Planck's constant $\hbar = \frac{h}{2\pi}$, it follows accordingly:
	
	$$\hbar = \frac{h}{2\pi} = \frac{1}{2\pi} \cdot \frac{1}{2\pi\sqrt{\mu_0\varepsilon_0}} = \frac{1}{4\pi^2\sqrt{\mu_0\varepsilon_0}}$$
	
	\section{Redefinition of the Fine-Structure Constant}
	
	\subsection{What Does the Elementary Charge $e$ Mean?}
	
	The elementary charge $e$ represents the electric charge of an electron or proton and is approximately $e \approx 1.602 \times 10^{-19}$ C (Coulomb).
	
	\subsection{The Fine-Structure Constant Through Electromagnetic Vacuum Constants}
	
	The fine-structure constant $\alpha$ is traditionally defined as:
	
	$$\alpha = \frac{e^2}{4\pi\varepsilon_0\hbar c}$$
	
	By substituting the derivation for $\hbar = \frac{1}{4\pi^2\sqrt{\mu_0\varepsilon_0}}$, we get:
	
	$$\alpha = \frac{e^2}{4\pi\varepsilon_0 \cdot \frac{1}{4\pi^2\sqrt{\mu_0\varepsilon_0}} \cdot c}$$
	$$= \frac{e^2}{4\pi\varepsilon_0} \cdot 4\pi^2\sqrt{\mu_0\varepsilon_0} \cdot \frac{1}{c}$$
	$$= \frac{e^2 \cdot 4\pi^2 \cdot \sqrt{\mu_0\varepsilon_0}}{4\pi\varepsilon_0 \cdot c}$$
	$$= \frac{\pi e^2 \cdot \sqrt{\mu_0\varepsilon_0}}{\varepsilon_0 \cdot c}$$
	
	Using $c = \frac{1}{\sqrt{\mu_0\varepsilon_0}}$, it follows:
	
	$$\alpha = \frac{\pi e^2 \cdot \sqrt{\mu_0\varepsilon_0}}{\varepsilon_0} \cdot \sqrt{\mu_0\varepsilon_0}$$
	$$= \frac{\pi e^2 \cdot \mu_0\varepsilon_0}{\varepsilon_0}$$
	$$= \pi e^2 \cdot \mu_0$$
	
	Alternatively, with the other derivation for $h = \frac{1}{2\pi\sqrt{\mu_0\varepsilon_0}}$:
	
	$$\alpha = \frac{e^2}{4\pi\varepsilon_0\hbar c}$$
	$$= \frac{e^2}{4\pi\varepsilon_0 \cdot \frac{h}{2\pi} \cdot c}$$
	$$= \frac{e^2 \cdot 2\pi}{4\pi\varepsilon_0 \cdot h \cdot c}$$
	$$= \frac{e^2}{2\varepsilon_0 \cdot h \cdot c}$$
	
	Substituting $h = \frac{1}{2\pi\sqrt{\mu_0\varepsilon_0}}$ here:
	
	$$\alpha = \frac{e^2}{2\varepsilon_0 \cdot \frac{1}{2\pi\sqrt{\mu_0\varepsilon_0}} \cdot c}$$
	$$= \frac{e^2 \cdot 2\pi\sqrt{\mu_0\varepsilon_0}}{2\varepsilon_0 \cdot c}$$
	$$= \frac{\pi e^2 \cdot \sqrt{\mu_0\varepsilon_0}}{\varepsilon_0 \cdot c}$$
	
	Using $c = \frac{1}{\sqrt{\mu_0\varepsilon_0}}$ again:
	
	$$\alpha = \pi e^2 \cdot \mu_0$$
	
	This representation shows that the fine-structure constant can be directly derived from the electromagnetic structure of the vacuum without explicitly involving $h$ or $\hbar$.
	
	\section{Consequences of a Redefinition of Coulomb}
	
	\subsection{Is Coulomb Incorrectly Defined if $\alpha = 1$ is Assumed?}
	
	The hypothesis is that if the fine-structure constant $\alpha = 1$ were set, the definition of the Coulomb, and thus the elementary charge $e$, would need to be adjusted.
	
	\subsection{New Definition of the Elementary Charge}
	
	If we set $\alpha = 1$ and use the original formula $\alpha = \frac{e^2}{4\pi\varepsilon_0\hbar c}$, then for the elementary charge $e$:
	
	$$e^2 = 4\pi\varepsilon_0\hbar c$$
	
	$$e = \sqrt{4\pi\varepsilon_0\hbar c}$$
	
	Alternatively, with the newly derived form $\alpha = \pi e^2 \cdot \mu_0$:
	
	$$1 = \pi e^2 \cdot \mu_0$$
	
	$$e = \sqrt{\frac{1}{\pi \mu_0}}$$
	
	This would mean that the numerical value of $e$ would change because it would then depend directly on $\hbar$, $c$, and $\varepsilon_0$, or on $\mu_0$.
	
	\subsection{Physical Significance}
	
	The unit Coulomb (C) is an arbitrary convention in the SI system. If $\alpha = 1$ were chosen instead, the definition of $e$ would change. In natural unit systems (common in high-energy physics), $\alpha = 1$ is often set, meaning that charge is measured in a unit other than Coulomb.
	
	The current value of the fine-structure constant $\alpha \approx \frac{1}{137}$ is not "wrong" but a consequence of our historical definitions of units. The electromagnetic unit system could originally have been defined such that $\alpha = 1$.
	
	\section{Effects on Other SI Units}
	
	\subsection{What Effects Would a Coulomb Adjustment Have on Other Units?}
	
	Adjusting the charge unit so that $\alpha = 1$ would have consequences for numerous other physical units:
	
	\subsubsection{New Charge Unit}
	The new elementary charge would be:
	$$e = \sqrt{4\pi\varepsilon_0\hbar c} = \sqrt{\frac{1}{\pi \mu_0}}$$
	
	\subsubsection{Change in Electric Current (Ampere)}
	Since $1 \text{ A} = 1 \text{ C}/\text{s}$, the unit of Ampere would also change accordingly.
	
	\subsubsection{Changes in Electromagnetic Constants}
	Since $\varepsilon_0$ and $\mu_0$ are linked to the speed of light:
	(unit: \si{\meter\per\second})
	$$c^2 = \frac{1}{\mu_0\varepsilon_0}$$
	perhaps either $\mu_0$ or $\varepsilon_0$ would need to be adjusted.
	
	\subsubsection{Effects on Capacitance (Farad)}
	Capacitance is defined as $C = \frac{Q}{V}$. Since $Q$ (charge) changes, the unit of Farad would also change.
	
	\subsubsection{Changes in Voltage Unit (Volt)}
	Electric voltage is defined as $1 \text{ V} = 1 \text{ J}/\text{C}$. Since Coulomb would have a different magnitude, the size of the Volt would also shift.
	
	\subsubsection{Indirect Effects on Mass}
	In quantum field theory, the fine-structure constant is linked to the rest mass energy of electrons, which could have indirect effects on the definition of mass.
	
	\subsubsection{Consistency Check of the Fine-Structure Constant with SI Units}
	
	To check whether an adjustment of $\mu_0$ or $\varepsilon_0$ would be necessary, we consider the relationship:
	\[
	c^2 = \frac{1}{\mu_0 \varepsilon_0}
	\]
	with the known values:
	- Speed of light: \( c = 299792458 \) m/s
	- Magnetic field constant: \( \mu_0 = 4\pi \times 10^{-7} \) H/m
	- Electric field constant: \( \varepsilon_0 = \frac{1}{\mu_0 c^2} \) F/m
	
	The fine-structure constant is defined as:
	\[
	\alpha = \frac{e^2}{4\pi \varepsilon_0 \hbar c}
	\]
	with:
	- Planck constant: \( \hbar = 1.054571817 \times 10^{-34} \) J·s
	- Elementary charge: \( e = 1.602176634 \times 10^{-19} \) C
	
	Substituting the values yields:
	\[
	\alpha_{\text{calculated}} = 0.007297352569776441
	\]
	The official value is:
	\[
	\alpha_{\text{official}} = 0.0072973525692838015
	\]
	The difference is \( 4.93 \times 10^{-13} \), which is extremely small. Thus, \( \alpha \) remains consistent within the expected precision, and an adjustment of \( \mu_0 \) or \( \varepsilon_0 \) is not necessary.
	
	% Verification of the Gravitational Constant in the SI System
	The gravitational constant can be expressed through Planck units as:
	\[
	G = \frac{\hbar c}{m_P^2}
	\]
	with:
	- \( \hbar = 1.054571817 \times 10^{-34} \) J·s
	- \( c = 299792458 \) m/s
	- \( m_P = 2.176434 \times 10^{-8} \) kg (Planck mass)
	
	Substituting the values:
	\[
	G_{\text{calculated}} = \frac{(1.054571817 \times 10^{-34}) \times (299792458)}{(2.176434 \times 10^{-8})^2}
	\]
	Result:
	\[
	G_{\text{calculated}} = 6.6743021 \times 10^{-11} \text{ m}^3\text{kg}^{-1}\text{s}^{-2}
	\]
	The official value:
	\[
	G_{\text{official}} = 6.67430 \times 10^{-11} \text{ m}^3\text{kg}^{-1}\text{s}^{-2}
	\]
	The deviation is only \( 2.1 \times 10^{-17} \), which is within measurement accuracy. This means \( G \) is correctly defined in the SI system and remains consistent with Planck units.
	
	% Derivation of Unit Choice for an Alternative Unit System
	In an alternative unit system, we set certain constants to 1:
	- \( c = 1 \) (speed of light)
	- \( \hbar = 1 \) (reduced Planck constant)
	- \( e = 1 \) (elementary charge)
	- \( \alpha = 1 \) (fine-structure constant)
	
	From this, the electric field constant follows as:
	\[
	\varepsilon_0 = \frac{1}{4\pi}
	\]
	and with the relationship \( c^2 = 1/(\mu_0 \varepsilon_0) \), we obtain the magnetic field constant:
	\[
	\mu_0 = 4\pi
	\]
	
	In this system, units change:
	- Charge unit: \( e = 1 \) fixes the unit of charge.
	- Energy unit: Since \( \hbar = 1 \), energy is measured directly in frequency units.
	- Time unit: With \( c = 1 \), length units are directly linked to time units.
	
	The fine-structure constant remains 1 by definition, but physical quantities are scaled differently in this system.
	
	% Definition of Units and Consequences
	In our new unit system, the following constants were set to 1:
	- Speed of light \( c = 1 \)
	- Reduced Planck constant \( \hbar = 1 \)
	- Elementary charge \( e = 1 \)
	- Fine-structure constant \( \alpha = 1 \)
	
	Not set to 1:
	- Planck energy \( E_P \), which serves as a measure for the energy unit.
	- Gravitational constant \( G \), as it relates to Planck units.
	
	This means all other physical constants can be derived from these values.
	
	% Back-Calculation of Energy in SI Units
	The Planck energy is defined as:
	\[
	E_P = \sqrt{\frac{\hbar c^5}{G}}
	\]
	Substituting the known values:
	\[
	E_P = \sqrt{\frac{(1.054571817 \times 10^{-34}) (299792458)^5}{6.67430 \times 10^{-11}}}
	\]
	This yields:
	\[
	E_P \approx 1.956 \times 10^9 \text{ J}
	\]
	Since \( E_P = 1 \) in our new unit system, it follows:
	\[
	1 \text{ energy unit} = 1.956 \times 10^9 \text{ J}
	\]
	Thus, any energy in our system can be converted to Joules by multiplying by this factor.
	
	\section*{Choice of the Fine-Structure Constant in a Natural Unit System}
	
	In a natural unit system, fundamental natural constants can be set to 1 to simplify equations. A key question is whether the fine-structure constant \( \alpha \) should also be set to 1 or if its measured value \( \alpha \approx \frac{1}{137} \) must be retained.
	
	\subsection*{Setting \( \alpha = 1 \)}
	
	If \( \alpha = 1 \) is set, several simplifications arise:
	
	\begin{itemize}
		\item The electric permittivity becomes \( \varepsilon_0 = \frac{1}{4\pi} \), and the magnetic constant \( \mu_0 = 4\pi \).
		\item The Maxwell equations take a particularly simple form.
		\item The elementary charge \( e \) is directly linked to energy units.
		\item All fundamental constants become dimensionless.
		\item Simplified quantum electrodynamics (QED), as the coupling strength no longer appears as a perturbation parameter.
		\item Potential for a direct relationship between gravitation and electrodynamics if \( G \) is chosen accordingly.
	\end{itemize}
	
	However, there are also challenges:
	
	\begin{itemize}
		\item SI values for charge, energy, and electrical constants would need to be rescaled.
	\end{itemize}
	
	\subsection*{Retaining \( \alpha \approx 1/137 \)}
	
	Alternatively, \( \alpha \) can be left at its measured value. Then, the known conversion factors remain:
	
	\begin{align*}
		\alpha &= \frac{e^2}{4\pi \varepsilon_0 \hbar c} \approx \frac{1}{137}
	\end{align*}
	
	This maintains physical consistency with experimental values, but some equations become slightly more complex.
	
	\subsection*{Possible Alternative Formulations}
	
	This natural form allows various alternative formulations:
	
	\begin{itemize}
		\item Directly derivable relationships between electromagnetic and gravitational forces.
		\item Maxwell equations in a particularly simple form.
		\item Simplified quantum electrodynamics, as the coupling strength no longer acts as a perturbation parameter.
		\item Improved comparability of gravitation and electrodynamics through scaled constants.
		\item Simplification of Planck units and their connection to electromagnetic quantities.
	\end{itemize}
	
	\subsection*{Conclusion}
	
	There is no definitive answer as to which choice is better:
	
	\begin{itemize}
		\item \textbf{If simplifying equations is the main goal}, \( \alpha = 1 \) makes sense.
		\item \textbf{If maintaining physical consistency with measured values is desired}, \( \alpha \approx 1/137 \) is preferable.
	\end{itemize}
	
	Since \( c = 1 \), \( \hbar = 1 \), and \( e = 1 \) have already been set in this unit system, it would be consistent to also choose \( \alpha = 1 \) to further simplify the equations.
	
	\subsection*{Sources}
	\begin{enumerate}
		\item \href{https://en.wikipedia.org/wiki/Fine-structure_constant}{Fine-Structure Constant – Wikipedia}
		\item \href{https://www.cosmos-indirekt.de/Physik-Schule/Feinstrukturkonstante}{Fine-Structure Constant – Physics School}
		\item \href{https://www.spektrum.de/lexikon/physik/feinstrukturkonstante/4829}{Fine-Structure Constant – Physics Lexicon}
		\item \href{https://renenyffenegger.ch/notes/Wissenschaft/Physik/Konstanten/Feinstrukturkonstante}{Fine-Structure Constant – René Nyffenegger}
		\item \href{https://mensch-erde-universum.de/feinstrukturkonstante/}{Fine-Structure Constant Alpha – Human-Earth-Universe}
		\item \href{https://en.wikipedia.org/wiki/Electroweak_interaction}{Electroweak Interaction – Wikipedia}
		\item \href{https://www.cosmos-indirekt.de/Physik-Schule/Elektroschwache_Wechselwirkung}{Electroweak Interaction – Physics School}
		\item \href{https://www.spektrum.de/lexikon/physik/elektroschwache-wechselwirkung/4197}{Electroweak Interaction – Physics Lexicon}
		\item \href{https://en.wikipedia.org/wiki/Natural_units}{Natural Units – Wikipedia}
		\item \href{https://www-static.etp.physik.uni-muenchen.de/fp-versuch/node5.html}{Introduction to Particle Physics – LMU}
		\item \href{https://www.thphys.uni-heidelberg.de/~wolschin/alpha.html}{SNO – Heidelberg University}
		\item \href{https://vixra.org/pdf/1408.0018vM.pdf}{Decoding the Fine-Structure Constant}
		\item \href{https://nsosp.org/en/Quantum-Flux-Theory/Electroweak-Interaction-Particle-Transformations-Weak-Isospin_en.php}{Electroweak Interaction, Particle Transformations, and Inner Spin}
		\item \href{https://www.rhetos.de/html/lex/feinstrukturkonstante.htm}{Fine-Structure Constant (Approximately 1/137) – Rhetos}
	\end{enumerate}
	
\end{document}