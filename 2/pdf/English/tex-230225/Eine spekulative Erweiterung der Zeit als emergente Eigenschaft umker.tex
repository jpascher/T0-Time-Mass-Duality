\documentclass{article}
\usepackage[utf8]{inputenc}
\usepackage[T1]{fontenc}
\usepackage{amsmath}
\usepackage{amssymb}
\usepackage{geometry}
\usepackage{hyperref}
\usepackage{siunitx}

\title{Eine spekulative Erweiterung der Zeit als emergente Eigenschaft: \\Dokumentation der Diskussion mit einer neuen Perspektive}
\author{Unter Mitwirkung von Johann Paschers Konzepten}
\date{März 23, 2025}

\begin{document}
	
	\maketitle
	
	\section{Einführung}
	
	Diese Arbeit basiert auf Johann Paschers Konzept (\textit{Die Feinstrukturkonstante: Verschiedene Darstellungen und Zusammenhänge}, 2025) und untersucht eine neue Perspektive: Zeit als unveränderlich, Energie als variabel, mit vakuum-spezifischem \( c \).
	
	\section{Grundlage: Paschers Konzept der intrinsischen Zeit}
	
	\subsection{Definition und Herleitung}
	
	\[
	E = mc^2 = \frac{\hbar}{T}, \quad T = \frac{\hbar}{mc^2}
	\]
	
	\subsection{Energie-Zeit-Unschärferelation}
	
	\[
	\Delta E \cdot \Delta t \geq \frac{\hbar}{2}
	\]
	
	\subsection{Feinstrukturkonstante}
	
	\[
	\alpha = \frac{e^2}{4\pi \varepsilon_0 \hbar c}
	\]
	
	\section{Ursprünglicher Ansatz: Variable Zeit, feste Energie}
	
	\subsection{Überblick}
	
	\[
	E = mc^2, \quad T = \frac{\hbar}{E}
	\]
	
	\section{Neue Perspektive: Unveränderliche Zeit, variable Energie}
	
	\subsection{Hypothese}
	
	Zeit ist \( T_0 \) (z. B. \( t_P = 5.39 \times 10^{-44} \, \text{s} \)):
	
	\[
	E = \frac{\hbar}{T_0}
	\]
	
	\subsection{Mathematische Modifikation}
	
	\subsubsection{Variable Masse bei konstantem \( c \)}
	- \( c = c_0 \):
	\[
	E = m c_0^2 = \frac{\hbar}{T_0}, \quad m = \frac{\hbar}{T_0 c_0^2}
	\]
	- \( m \) variiert mit \( E \).
	
	\subsection{Physikalische Implikationen}
	
	\subsubsection{Quantenmechanik}
	\[
	i\hbar \frac{\partial \Psi}{\partial T_0} = \hat{H}(E) \Psi
	\]
	
	\subsubsection{Relativitätstheorie}
	\[
	\gamma = \frac{1}{\sqrt{1 - \frac{v^2}{c_0^2}}}
	\]
	
	\subsection{Kompatibilität mit \( c \) und ART}
	
	\subsubsection{Mit konstantem \( c \)}
	- \( c = c_0 \) im Vakuum.
	
	\subsection{Energie: Ausgedehntes Universum vs. Davor}
	
	\subsubsection{Ausgedehntes Universum}
	- \( m_{\text{total}} \approx 10^{53} \, \text{kg} \).
	- \( E_{\text{total}} = 10^{70} \, \text{J} \).
	
	\subsubsection{Davor existierend}
	- Dichte: \( 10^{113} \, \text{kg/m}^3 \), Volumen: \( 10^{-25} \, \text{m}^3 \):
	\[
	m_{\text{total}} = 10^{88} \, \text{kg}, \quad E_{\text{total}} = 10^{105} \, \text{J}
	\]
	
	\subsection{Folgen im erfahrbaren Bereich}
	
	\subsubsection{Absolute Zeit und Messungen}
	- \( T_0 \) fest, \( m \) variabel.
	
	\subsubsection{Photonenbasierte Zeitmessung}
	- Lichtlaufzeit: \( t = \frac{d}{c_0} \).
	- Rotverschiebung: \( f' = f (1 + z)^{-1} \).
	
	\subsubsection{Kosmologische Messungen}
	- **Rotverschiebung**: Standard: Expansion, Modell: \( E \)-Verlust.
	- **CMB**: Standard: Zeitdilatation, Modell: \( m \)-Variation.
	- **Gravitationslinsen**: Standard: Zeitverzögerung, Modell: \( m \)-Effekt.
	- **Fehlinterpretation**: Zeitdilatation könnte \( m \)- oder \( E \)-Änderung sein, aber Konsistenz spricht für Standard.
	
	\subsubsection{Fazit}
	- Kosmologische Messungen könnten fehlerbehaftet sein, aber variable Zeit bleibt konsistenter.
	
	\section{Verbindung zu Schwarzen Löchern und Urknall}
	
	\subsection{Hypothese}
	Schwarze Löcher und Urknall als Zustände variabler \( m \).
	
	\section{Diskussionsverlauf}
	
	\subsection{Startpunkt}
	Paschers \( T = \frac{\hbar}{mc^2} \).
	
	\subsection{Neue Perspektive}
	Feste Zeit, variable \( m \), kosmologische Messungen geprüft.
	
	\section{Schlussfolgerung}
	
	Absolute Zeit und variable \( m \) werfen Fragen zur Fehlinterpretation kosmologischer Messungen auf, aber Zeitdilatation bleibt im erfahrbaren Bereich dominant.
	
	\begin{thebibliography}{1}
		\bibitem{pascher} Pascher, J. (2025). \textit{Die Feinstrukturkonstante: Verschiedene Darstellungen und Zusammenhänge}. Unveröffentlichtes Manuskript.
	\end{thebibliography}
	
\end{document}