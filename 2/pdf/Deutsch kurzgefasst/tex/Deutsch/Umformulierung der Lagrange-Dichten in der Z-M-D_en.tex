\documentclass{article}
\usepackage[utf8]{inputenc}
\usepackage[english]{babel}
\usepackage{amsmath}
\usepackage{amssymb}
\usepackage{physics}
\usepackage{graphicx}
\usepackage{hyperref}
\usepackage{xcolor}

\title{Reformulation of Lagrangian Densities in Time-Mass Duality}
\author{Johann Pascher}
\date{March 29, 2025}

\begin{document}
	
	\maketitle
	
	\section*{Introduction}
	I will attempt to develop a consistent reformulation of the fundamental Lagrangian densities based on the time-mass duality theory. The goal is to create a mathematically coherent and physically meaningful formulation that captures all essential aspects of the theory.
	
	\section{Fundamental Principles}
	Let us begin with the fundamental principles of time-mass duality:
	
	\begin{itemize}
		\item Intrinsic Time: \( T = \frac{\hbar}{m c^2} \)
		\item Modified Time Derivative: \( \partial_{t/T} = \frac{\partial}{\partial(t/T)} = T \frac{\partial}{\partial t} \)
		\item Duality between: Standard Picture (time dilation, constant mass) and Alternative Picture (absolute time, variable mass)
	\end{itemize}
	
	\section{Modified Lagrangian Density for Scalar Fields}
	The standard Lagrangian density for a scalar field (e.g., the Higgs field) is:
	
	\begin{equation}
		\mathcal{L}_{\text{scalar}} = \frac{1}{2} (\partial_\mu \phi) (\partial^\mu \phi) - \frac{1}{2} m^2 \phi^2 - V(\phi)
	\end{equation}
	
	In time-mass duality, this becomes:
	
	\begin{equation}
		\mathcal{L}_{\text{scalar-T}} = \frac{1}{2} (D_{T\mu} \phi) (D_T^\mu \phi) - \frac{1}{2} m^2 \phi^2 - V(\phi)
	\end{equation}
	
	where the modified covariant derivative is defined as:
	
	\begin{equation}
		D_{T\mu} \phi = T(x) \partial_\mu \phi + \phi \partial_\mu T(x)
	\end{equation}
	
	Explicitly written:
	
	\begin{equation}
		\mathcal{L}_{\text{scalar-T}} = \frac{1}{2} T(x)^2 \left( \frac{\partial \phi}{\partial t} \right)^2 + T(x) \phi \frac{\partial \phi}{\partial t} \frac{\partial T(x)}{\partial t} - \frac{1}{2} (\nabla \phi)^2 - \frac{1}{2} m^2 \phi^2 - V(\phi)
	\end{equation}
	
	\section{Complete Higgs Lagrangian Density}
	For the Higgs field as a complex doublet, we obtain:
	
	\begin{equation}
		\mathcal{L}_{\text{Higgs-T}} = (D_{T\mu} \Phi_T)^\dagger (D_T^\mu \Phi_T) - V_T(\Phi_T)
	\end{equation}
	
	with the covariant derivative:
	
	\begin{equation}
		D_{T\mu} \Phi_T = T(x) (\partial_\mu + i g \tau^a W_\mu^a + i g' \frac{Y}{2} B_\mu) \Phi_T + \Phi_T \partial_\mu T(x)
	\end{equation}
	
	The Higgs potential retains its form:
	
	\begin{equation}
		V_T(\Phi_T) = -\mu^2 \Phi_T^\dagger \Phi_T + \lambda (\Phi_T^\dagger \Phi_T)^2
	\end{equation}
	
	\section{Reformulated Yukawa Coupling}
	The Yukawa coupling is modified to:
	
	\begin{equation}
		\mathcal{L}_{\text{Yukawa-T}} = -y_f \bar{\psi}_L \Phi_T \psi_R + \text{h.c.}
	\end{equation}
	
	The transformation function \( \mathcal{T}(\gamma) \) is not explicitly needed here, as mass variation is implicitly accounted for by \( T(x) \).
	
	\section{Lagrangian Density for Fermions}
	The Dirac Lagrangian density for fermions becomes:
	
	\begin{equation}
		\mathcal{L}_{\text{Dirac-T}} = \bar{\psi} (i \gamma^\mu D_{T\mu} - m) \psi
	\end{equation}
	
	with:
	
	\begin{equation}
		D_{T\mu} \psi = T(x) D_\mu \psi + \psi \partial_\mu T(x)
	\end{equation}
	
	where \( D_\mu \) is the usual covariant derivative with gauge fields.
	
	\section{Gauge Boson Lagrangian Density}
	For gauge bosons, the Lagrangian density is modified to:
	
	\begin{equation}
		\mathcal{L}_{\text{Gauge-T}} = -\frac{1}{4} T(x)^2 F_{\mu\nu} F^{\mu\nu}
	\end{equation}
	
	with the unchanged field strength tensor:
	
	\begin{equation}
		F_{\mu\nu} = \partial_\mu A_\nu - \partial_\nu A_\mu + i g [A_\mu, A_\nu]
	\end{equation}
	
	\section{Unified Formulation of the Complete Lagrangian Density}
	The total Lagrangian density is now:
	
	\begin{equation}
		\mathcal{L}_{\text{Total-T}} = \mathcal{L}_{\text{Higgs-T}} + \mathcal{L}_{\text{Dirac-T}} + \mathcal{L}_{\text{Yukawa-T}} + \mathcal{L}_{\text{Gauge-T}}
	\end{equation}
	
	\section{Field Equations from the Modified Lagrangian Density}
	The field equations are derived by applying the Euler-Lagrange equations:
	
	For the Higgs field:
	
	\begin{equation}
		D_{T\mu} D_T^\mu \Phi_T + \frac{\partial V_T}{\partial \Phi_T^\dagger} = 0
	\end{equation}
	
	For fermions:
	
	\begin{equation}
		(i \gamma^\mu D_{T\mu} - m) \psi = 0
	\end{equation}
	
	For gauge bosons:
	
	\begin{equation}
		\partial_\mu (T(x)^2 F^{\mu\nu}) + i g [A_\mu, T(x)^2 F^{\mu\nu}] = j^\nu
	\end{equation}
	
	\section{Incorporation of Gravity via Modified Einstein-Hilbert Action}
	The Einstein-Hilbert action is modified to:
	
	\begin{equation}
		S_{\text{Grav-T}} = \frac{1}{16\pi G} \int d^4x \sqrt{-g} T(x) R
	\end{equation}
	
	where \( R \) is the Ricci scalar, adjusted by \( T(x) \).
	
	\section{Summary and Consistency Check}
	The reformulation is based on the consistent introduction of \( T(x) \) into all derivatives and field terms. The theory should:
	
	\begin{itemize}
		\item Remain Lorentz-invariant under consideration of the duality
		\item Correctly describe phenomena such as time dilation and mass variation
		\item Predict testable deviations from the Standard Model
	\end{itemize}
	
	\section{Comprehensive Lagrangian Density of the Time-Mass Duality Theory}
	The complete Lagrangian density is:
	
	\begin{equation}
		\mathcal{L}_{\text{Total-T}} = \mathcal{L}_{\text{Higgs-T}} + \mathcal{L}_{\text{Fermion-T}} + \mathcal{L}_{\text{Gauge-T}} + \mathcal{L}_{\text{Yukawa-T}}
	\end{equation}
	
	\subsection{Higgs Sector}
	\begin{equation}
		\mathcal{L}_{\text{Higgs-T}} = (D_{T\mu} \Phi_T)^\dagger (D_T^\mu \Phi_T) - V_T(\Phi_T)
	\end{equation}
	
	with:
	\begin{itemize}
		\item \( D_{T\mu} \Phi_T = T(x) (\partial_\mu + i g \tau^a W_\mu^a + i g' \frac{Y}{2} B_\mu) \Phi_T + \Phi_T \partial_\mu T(x) \)
		\item \( V_T(\Phi_T) = -\mu^2 \Phi_T^\dagger \Phi_T + \lambda (\Phi_T^\dagger \Phi_T)^2 \)
		\item \( T(x) = \frac{\hbar}{m c^2} \)
	\end{itemize}
	
	\subsection{Fermion Sector}
	\begin{equation}
		\mathcal{L}_{\text{Fermion-T}} = \sum_f \bar{\psi}_f (i \gamma^\mu D_{T\mu} - m_f) \psi_f
	\end{equation}
	
	\subsection{Gauge Boson Sector}
	\begin{equation}
		\mathcal{L}_{\text{Gauge-T}} = -\frac{1}{4} T(x)^2 (G_{\mu\nu}^a G^{a\mu\nu} + W_{\mu\nu}^a W^{a\mu\nu} + B_{\mu\nu} B^{\mu\nu})
	\end{equation}
	
	\subsection{Yukawa Sector}
	\begin{equation}
		\mathcal{L}_{\text{Yukawa-T}} = -\sum_f y_f \bar{\psi}_{fL} \Phi_T \psi_{fR} + \text{h.c.}
	\end{equation}
	
	\subsection{Energy-Momentum Relation}
	The modified energy-momentum relation is:
	
	\begin{equation}
		E^2 = (p c)^2 + (m c^2)^2 + \alpha \frac{\hbar c}{T}
	\end{equation}
	
\end{document}