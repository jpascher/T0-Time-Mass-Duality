\documentclass[12pt,a4paper]{article}
\usepackage[utf8]{inputenc}
\usepackage[T1]{fontenc}
\usepackage[ngerman]{babel}
\usepackage{lmodern}
\usepackage{csquotes}
\usepackage{amsmath}
\usepackage{amssymb}
\usepackage{physics}
\usepackage{geometry}
\usepackage{tocloft}
\usepackage{xcolor}
\usepackage{graphicx,tikz,pgfplots}
\pgfplotsset{compat=1.18}
\usepackage{booktabs}
\usepackage{siunitx}
\usepackage{amsthm}
\usepackage[colorlinks=true, linkcolor=blue, citecolor=blue, urlcolor=blue]{hyperref}
\usepackage{cleveref}

\geometry{a4paper, margin=2cm}

\renewcommand{\cftsecfont}{\color{blue}}
\renewcommand{\cftsubsecfont}{\color{blue}}
\renewcommand{\cftsecpagefont}{\color{blue}}
\renewcommand{\cftsubsecpagefont}{\color{blue}}
\setlength{\cftsecindent}{1cm}
\setlength{\cftsubsecindent}{2cm}

\newcommand{\Tfield}{T(x)}

\newtheorem{theorem}{Theorem}[section]
\newtheorem{proposition}[theorem]{Proposition}

\title{Die Notwendigkeit einer Erweiterung der Standard-Quantenmechanik und Quantenfeldtheorie}
\author{Johann Pascher}
\date{27. März 2025}

\begin{document}
	
	\maketitle
	
	\begin{abstract}
		Diese Arbeit beleuchtet die konzeptionellen Grenzen der Standard-Quantenmechanik (QM) und Quantenfeldtheorie (QFT) und schlägt die Zeit-Masse-Dualität mit einem intrinsischen Zeitfeld als Erweiterung vor. Durch die Einführung von \(\Tfield = \hbar/mc^2\) wird eine Verbindung zwischen Zeit und Masse hergestellt, die den Dualismus zwischen QM und QFT überwindet und einen deterministischen Rahmen bietet. Die Theorie wird durch experimentelle Vorhersagen und kosmologische Implikationen unterstützt.
	\end{abstract}
	
	\tableofcontents
	\newpage
	
	\section{Einleitung: Konzeptionelle Grenzen der etablierten Theorien}
	Die Quantenmechanik und Quantenfeldtheorie stoßen an Grenzen, insbesondere bei der Integration mit der Allgemeinen Relativitätstheorie (ART) und der Natur von Zeit und Masse. Die Zeit-Masse-Dualität bietet einen neuen Ansatz \cite{pascher_wesentl_2025}.
	
	\subsection{Inhärenter Dualismus zwischen QM und QFT}
	\begin{itemize}
		\item QM: Teilchenperspektive \cite{schrodinger}.
		\item QFT: Feldbasierte Sichtweise.
	\end{itemize}
	
	\subsection{Überinterpretation aufgrund unvollständiger theoretischer Grundlagen}
	\begin{itemize}
		\item Messproblem \cite{einstein2}.
		\item Nichtlokalität \cite{bell}.
	\end{itemize}
	
	\section{Asymmetrische Behandlung von Zeit und Raum}
	\subsection{Zeit als Parameter versus Raum als Operator}
	\begin{equation}
		i\hbar \frac{\partial}{\partial t}\Psi(x,t) = \hat{H}\Psi(x,t)
	\end{equation}
	
	\section{Statische Behandlung der Masse}
	\subsection{Masse als unveränderlicher Parameter}
	\begin{equation}
		\hat{H} = \frac{\hat{p}^2}{2m} + V(\hat{x})
	\end{equation}
	
	\section{Das Konzept der intrinsischen Zeit}
	\begin{theorem}[Intrinsische Zeit]
		\begin{equation}
			\Tfield = \frac{\hbar}{mc^2}
		\end{equation}
	\end{theorem}
	
	\section{Zeit-Masse-Dualität: Ein neuer theoretischer Rahmen}
	\subsection{Komplementäre Modelle}
	\begin{itemize}
		\item Standardmodell: Konstante Masse.
		\item T0-Modell: Absolute Zeit.
	\end{itemize}
	
	\subsection{Reformulierung der Schrödinger-Gleichung}
	\begin{equation}
		i\hbar \frac{\partial}{\partial (t/\Tfield)}\Psi = \hat{H}\Psi
	\end{equation}
	
	\section{Konsequenzen für fundamentale Phänomene}
	\subsection{Quantenkohärenz und Dekohärenz}
	\begin{equation}
		\Gamma_{\text{dek}} = \Gamma_0 \cdot \frac{mc^2}{\hbar}
	\end{equation}
	
	\begin{figure}[h]
		\centering
		\begin{tikzpicture}
			\begin{axis}[
				xlabel={Masse [eV]},
				ylabel={Kohärenzzeit [eV\(^{-1}\)]},
				xlabel style={font=\large},
				ylabel style={font=\large},
				tick label style={font=\normalsize},
				xmin=0, xmax=1000,
				ymin=0, ymax=0.01,
				legend pos=north east,
				legend style={font=\large},
				grid=both,
				minor tick num=1
				]
				\addplot[blue, ultra thick, domain=1:1000, samples=100] {1/x};
				\legend{\(\tau \propto 1/m\)}
			\end{axis}
		\end{tikzpicture}
		\caption{Masseabhängige Kohärenzzeit im T0-Modell.}
	\end{figure}
	
	\section{Variable Masse als verborgene Variable}
	\subsection{Modifizierte Quantendynamik}
	\begin{equation}
		i\hbar \frac{\partial}{\partial t}\Psi(x,t) = \hat{H}(m(t))\Psi(x,t)
	\end{equation}
	
	\section{Kosmologische Implikationen}
	\begin{itemize}
		\item Rotverschiebung: \(1 + z = e^{\alpha r}\) \cite{pascher_wesentl_2025}.
		\item Gravitationspotential: \(\Phi(r) = -\frac{GM}{r} + \kappa r\) \cite{pascher_wesentl_2025}.
	\end{itemize}
	
	\begin{thebibliography}{99}
		\bibitem{pascher_wesentl_2025} Pascher, J. (2025). \textit{Wesentliche mathematische Formalismen der Zeit-Masse-Dualitätstheorie mit Lagrange-Dichten}.
		\bibitem{einstein} Einstein, A. (1905). \textit{Ist die Trägheit eines Körpers von seinem Energieinhalt abhängig?}. Annalen der Physik, 323(13), 639-641.
		\bibitem{planck} Planck, M. (1901). \textit{Über das Gesetz der Energieverteilung im Normalspektrum}. Annalen der Physik, 309(3), 553-563.
		\bibitem{schrodinger} Schrödinger, E. (1926). \textit{An Undulatory Theory of the Mechanics of Atoms and Molecules}. Physical Review, 28(6), 1049-1070.
		\bibitem{bell} Bell, J. S. (1964). \textit{On the Einstein Podolsky Rosen Paradox}. Physics, 1(3), 195-200.
		\bibitem{einstein2} Einstein, A., Podolsky, B., Rosen, N. (1935). \textit{Can Quantum-Mechanical Description of Physical Reality Be Considered Complete?}. Physical Review, 47(10), 777-780.
	\end{thebibliography}
	
\end{document}