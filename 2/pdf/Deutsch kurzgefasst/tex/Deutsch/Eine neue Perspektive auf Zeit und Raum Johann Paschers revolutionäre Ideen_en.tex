\documentclass[a4paper,12pt]{article}
\usepackage[utf8]{inputenc}
\usepackage[T1]{fontenc}
\usepackage[english]{babel}
\usepackage{amsmath}
\usepackage{amssymb}
\usepackage{geometry}
\usepackage{lmodern}
\usepackage{graphicx}
\usepackage{tikz}
\usepackage{pgfplots}
\pgfplotsset{compat=1.18}
\usepackage{xcolor}
\usepackage{tocloft}
\usepackage{amsthm}
\usepackage[colorlinks=true, linkcolor=blue, filecolor=blue, citecolor=blue, urlcolor=blue]{hyperref}
\usepackage{cleveref}

\geometry{a4paper, margin={2.5cm}}

\renewcommand{\cftsecfont}{\color{blue}}
\renewcommand{\cftsubsecfont}{\color{blue}}
\renewcommand{\cftsecpagefont}{\color{blue}}
\renewcommand{\cftsubsecpagefont}{\color{blue}}
\setlength{\cftsecindent}{1cm}
\setlength{\cftsubsecindent}{2cm}

\newtheorem{theorem}{Theorem}[section]
\newtheorem{proposition}[theorem]{Proposition}

\title{A New Perspective on Time and Space: Johann Pascher’s Revolutionary Ideas}
\author{Johann Pascher}
\date{March 25, 2025}

\begin{document}
	
	\maketitle
	
	\begin{abstract}
		This work presents Johann Pascher’s revolutionary model, inverting the traditional view of time and space. Unlike Einstein’s relativity, which treats time as variable and mass as constant, Pascher posits absolute time and variable mass. Each particle possesses an intrinsic time, inversely proportional to its mass. This perspective offers new explanations for quantum entanglement, cosmic redshift, and black hole structure, establishing energy as the fundamental quantity. Testable predictions are proposed to validate the model.
	\end{abstract}
	
	\tableofcontents
	\newpage
	
	Imagine looking at a familiar painting, one you’ve seen a hundred times. Then someone tilts it slightly, and suddenly you notice details and patterns you’d never seen before. That’s exactly what Johann Pascher does with our understanding of the universe.
	
	For over a century, Einstein’s theories have dominated our view of time and space. We’ve accepted that time is malleable—it slows down when you move fast or enter a strong gravitational field. Meanwhile, we believe an object’s rest mass is immutable, a fixed property of matter. This understanding has served us well, explaining everything from GPS satellites to the bending of starlight around the sun.
	
	But Pascher turns this view upside down: In his alternative model, time is absolute and flows constantly, while mass varies. This isn’t mere speculation but a fully developed model with its own mathematical formulations, explaining the same experimental observations we associate with the conventional model.
	
	\section{The Clock in Every Particle}
	In Pascher’s model, every particle in the universe—every electron, every proton—has its own characteristic timescale, termed "intrinsic time." This time is inversely proportional to the particle’s mass. Heavier particles have faster-ticking clocks; lighter particles have slower ones.
	
	Take a muon (similar to an electron but about 200 times heavier). In the standard model, we explain its extended lifetime as it travels through our atmosphere via time dilation. In Pascher’s model, the muon’s mass changes instead, while time runs constantly. Mathematically, these two descriptions are equivalent—they yield identical measurable outcomes but offer radically different perspectives on the underlying reality.
	
	This "intrinsic time" isn’t just a theoretical construct but a mathematically precise quantity enabling new insights into quantum phenomena.
	
	\section{When Distant Particles Are Linked}
	Quantum entanglement, where two particles seem connected across any distance, gains a new interpretation in Pascher’s framework. While conventional quantum mechanics describes the phenomenon without truly explaining it (Einstein called it "spooky action at a distance"), Pascher’s model provides a concrete mechanism.
	
	In his model, the connection isn’t instantaneous but depends on the mass of the involved particles. Two entangled particles of different masses evolve at different intrinsic time rates. What appears as simultaneous correlation actually has a mass-dependent delay, proportional to the mass ratio. This delay is measurable, offering a clear, testable prediction absent from standard quantum mechanics.
	
	\begin{theorem}[Mass-Dependent Delay]
		For entangled particles with masses $m_1$ and $m_2$, the correlation delay is $\Delta t \propto \frac{m_1}{m_2}$, where intrinsic time $T \propto \frac{1}{m}$.
	\end{theorem}
	
	\section{Rethinking Beginning and End}
	Our conception of the universe is also inverted. Conventional cosmology describes an expanding space where galaxies move apart, observed as light’s redshift. In Pascher’s alternative view, space is static, and redshift results from light’s energy loss over time, expressed as mass variation.
	
	The Big Bang isn’t the start of time and space but a state of extremely high energy and mass evolving over constant time. This view resolves cosmology’s horizon problem more elegantly than inflation theory and avoids the mathematical singularities plaguing the standard model.
	
	Black holes in Pascher’s model retain a finite structure, lacking the problematic central singularity of the standard model. The event horizon marks a boundary of extreme mass variation, not a point where time ends. This aligns with thermodynamics and sidesteps the information paradox of conventional theory.
	
	\section{A Fundamental Building Block: Energy}
	In Pascher’s expanded model, all fundamental constants of nature—the speed of light, Planck’s constant, the gravitational constant—are reduced to a single fundamental quantity: energy. This unification isn’t speculative but mathematically precise, showing that seemingly independent constants are facets of the same underlying reality.
	
	While the standard model takes these constants as given, Pascher’s approach demonstrates they can be derived from simpler principles. This profound simplification of natural law descriptions is akin to the shift from Ptolemaic to Copernican astronomy.

	
	\section{Putting It to the Test}
	Pascher’s extended quantum mechanics and field theory make clear, testable predictions differing from the standard model’s:
	
	\begin{itemize}
		\item Bell tests with particles of different masses will show measurable correlation delays proportional to the mass ratio.
		\item In quantum coherence systems, coherence times vary with mass, detectable in quantum information experiments.
		\item The modified Schrödinger equation with intrinsic time leads to different dispersion relations for matter waves.
	\end{itemize}
	
	These predictions are precisely formulated, offering clear tests between models feasible with current or near-future technology.
	
	\section{A New Lens, a Clearer Picture}
	Pascher’s approach inverts our usual perspective without altering experimentally confirmed physics laws. The core mathematical equations remain intact but are interpreted and extended within a new framework.
	
	This inversion mirrors the shift from a geocentric to a heliocentric worldview: Observed celestial motions stay the same, but the underlying explanation becomes more elegant and profound.
	
	While standard physics struggles to unify quantum mechanics and gravity, Pascher’s model offers a direct path to this unification through consistent treatment of time and mass.
	
	Modern physics faces major unsolved mysteries—dark matter, dark energy, the black hole information paradox. Both the standard model and Pascher’s theory have open questions here. Yet, while the standard model often relies on additional assumptions and corrections, Pascher’s approach resolves many issues directly through its more fundamental handling of time, mass, and energy.
	
	The history of science teaches that the deepest advances often come not from more data but from new perspectives. Pascher’s work reminds us that sometimes the most significant discoveries arise not from new observations but from viewing known facts in an entirely new way.
	
	\begin{thebibliography}{2}
		\bibitem{pascher2025} Pascher, J. (2025). \textit{Essential Mathematical Formalisms of the Time-Mass Duality Theory with Lagrangian Densities}. March 29, 2025.
		\bibitem{einstein1905} Einstein, A. (1905). \textit{On the Electrodynamics of Moving Bodies}. Annalen der Physik, 322(10), 891-921.
	\end{thebibliography}
	
\end{document}