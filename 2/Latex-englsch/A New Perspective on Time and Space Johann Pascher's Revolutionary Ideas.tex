\documentclass[a4paper,12pt]{article}
\usepackage[utf8]{inputenc}
\usepackage{amsmath}
\usepackage{amssymb}
\usepackage{geometry}
\geometry{a4paper, margin={2.5cm}}

\title{A New Perspective on Time and Space: Johann Pascher's Revolutionary Ideas}
\author{Johann Pascher}
\date{25.03.2025}

\begin{document}
	\maketitle
	
	Imagine looking at a familiar painting, one you've seen a hundred times before. Then someone tilts it slightly, and suddenly you notice details and patterns you never saw. That's what Johann Pascher is doing with our understanding of the universe.
	
	For over a century, Einstein's theories have dominated our view of time and space. We've come to accept that time is stretchy - it slows down when you move fast or enter a strong gravitational field. Meanwhile, we believe that an object's mass at rest is unchanging, a fixed property of matter. This understanding has served us well, explaining everything from GPS satellites to the bending of starlight around the sun.
	
	But Pascher inverts this perspective: In his alternative model, time is absolute and flows constantly, while it is mass that varies. This is not mere speculation, but a fully developed model with its own mathematical formulations that explain exactly the same experimental observations we associate with the conventional model.
	
	\section{The Clock Inside Every Particle}
	
	In Pascher's model, every particle in the universe - every electron, every proton - has its own characteristic time scale, referred to as ``intrinsic time.'' This time is inversely proportional to the particle's mass. Heavier particles have faster-ticking clocks; lighter particles have slower ones.
	
	Consider a muon (similar to an electron, but about 200 times heavier). In the standard model, we explain its extended lifetime during travel through our atmosphere through time dilation. In Pascher's model, instead, the mass of the muon changes while time continues to flow constantly. Mathematically, these two descriptions are equivalent - they lead to identical measurable results but offer completely different perspectives on the underlying reality.
	
	This ``intrinsic time'' is not just a theoretical construct but a mathematically precisely defined quantity that provides new insights into quantum phenomena.
	
	\section{When Distant Particles Are Connected}
	
	Quantum entanglement, where two particles appear connected across arbitrary distances, receives a new interpretation in Pascher's framework. While conventional quantum mechanics describes the phenomenon without really explaining it (Einstein called it ``spooky action at a distance''), Pascher's model offers a concrete mechanism.
	
	In his model, the connection is not instantaneous but depends on the mass of the particles involved. Two entangled particles of different masses evolve at different intrinsic time rates. What appears as simultaneous correlation actually has a mass-dependent delay, proportional to the ratio of the masses. This delay is measurable and represents a clear, verifiable prediction that the standard model of quantum mechanics does not make.
	
	\section{Rethinking Beginning and End}
	
	Our conception of the universe is also inverted. Conventional cosmology describes an expanding space in which galaxies move away from each other, which we observe as redshift of light. In Pascher's alternative view, space is static, and redshift results from a loss of energy of light over time, expressed as mass variation.
	
	The Big Bang is not the beginning of time and space but a state of extremely high energy and mass evolving over constant time. This view solves the horizon problem of cosmology more elegantly than inflation theory and avoids the mathematical singularities that plague standard theory.
	
	Black holes retain a finite structure in Pascher's model, without the problematic central singularity of the standard model. The event horizon marks a boundary of extreme mass variation, not a point where time ends. This is consistent with thermodynamics and avoids the information paradox that occurs in conventional theory.
	
	\section{One Fundamental Building Block: Energy}
	
	In Pascher's extended model, all fundamental constants of nature - the speed of light, Planck's constant, the gravitational constant - are reduced to a single fundamental quantity: energy. This unification is not speculative but mathematically precisely formulated and shows that the seemingly independent constants are different aspects of the same underlying reality.
	
	While the standard model of physics assumes these constants as given, Pascher's approach shows that they can be derived from simpler principles. This is a profound simplification of our description of natural laws, comparable to the transition from Ptolemaic to Copernican astronomy.
	
	\section{Putting It to the Test}
	
	Pascher's extended quantum mechanics and quantum field theory makes clear, testable predictions that differ from those of the standard model:
	
	\begin{itemize}
		\item Bell tests with particles of different masses will show measurable delays in correlations, proportional to the mass ratio.
		\item In systems with quantum mechanical coherence, coherence times vary with mass, which is detectable in quantum information experiments.
		\item The modified version of the Schrödinger equation with intrinsic time leads to different dispersion relations for matter waves.
	\end{itemize}
	
	These predictions are precisely formulated and offer clear tests between the models that are feasible with current or near-future technology.
	
	\section{A New Lens, a Clearer Picture}
	
	Pascher's approach inverts our usual perspective without changing the experimentally confirmed laws of physics. The mathematical equations remain fundamentally intact but are interpreted and extended in a new framework.
	
	This inversion is similar to the shift from the geocentric to the heliocentric worldview: The observed movements of celestial bodies remain the same, but the underlying explanation becomes more elegant and profound.
	
	While standard physics theory continues to struggle to unify quantum mechanics and gravitation, Pascher's approach offers a direct path to this unification through the consistent treatment of time and mass.
	
	Today's physics faces major unsolved puzzles - dark matter, dark energy, the information paradox of black holes. Both the standard model and Pascher's theory have open questions here. But while the standard model often relies on additional assumptions and corrections, Pascher's approach solves many of these problems directly through its more fundamental treatment of time, mass, and energy.
	
	The history of science teaches us that the most profound advances often come not through more data but through new perspectives. Pascher's work reminds us that sometimes the most important discoveries do not come from new observations but from viewing known facts in an entirely new way.
	
\end{document}