\documentclass[a4paper,12pt]{article}
\usepackage[utf8]{inputenc}
\usepackage[english]{babel}
\usepackage{amsmath}
\usepackage{amssymb}
\usepackage{geometry}
\geometry{a4paper, margin={2.5cm}}

\title{Time and Mass: A New Perspective on Old Formulas – and Liberation from Traditional Constraints}
\author{Johann Pascher}
\date{March 25, 2025}

\begin{document}
	\maketitle
	
	\section{Introduction: Traditional Views and the Obscured Perspective}
	
	Physics has achieved tremendous success with abstract concepts like quantum fields and spacetime curvature. But have we perhaps moved too far from an \emph{intuitive}, \emph{real} description of the world? Traditional perspectives, particularly our choice of units of measurement, may have obscured our view of a deeper, \emph{more unified} description of nature. This approach attempts to take a step back to the fundamentals – and to free physics from unnecessary constraints.
	
	\section{Natural Constants and Units: More Than Just Arbitrary Numbers?}
	
	Our units of measurement (meters, seconds, kilograms) have evolved historically and are practical for everyday use, but are they also \emph{fundamental}? In the laws of nature, \emph{natural constants} appear (such as the speed of light \(c\), the reduced Planck constant \(\hbar\), the gravitational constant \(G\), the fine-structure constant \(\alpha\)). Physicists often set \(c = 1\) and \(\hbar = 1\) ("natural units") to simplify formulas. But the traditional view often considers these constants as \emph{independent}, \emph{given} quantities. Is this really the case? Or do they conceal a deeper connection?
	
	\section{The Time-Mass Dualism: An Alternative Perspective}
	
	The \emph{time-mass dualism} offers a new perspective that challenges this traditional view:
	
	*   \textbf{Standard view (Theory of Relativity):} The \emph{rest mass} of an object is constant, while \emph{time} is relative (time dilation).
	*   \textbf{Alternative perspective:} What if \emph{time} is absolute, but \emph{mass} is variable?
	
	Imagine an "internal clock" (\emph{intrinsic time}) for each particle. This clock ticks faster the \emph{heavier} the particle is. Lighter particles have a slower internal clock.
	
	\section{All Constants Become Natural: Energy as the Unifying Principle}
	
	The crucial step is now: The time-mass dualism, combined with an \emph{extended} choice of natural units, allows us to express *all* physical constants as \emph{dimensionless numbers}. They become \emph{ratios} of a single fundamental quantity – and this quantity is \emph{energy}. The traditional constants lose their status as independent, given quantities; they become \emph{derived} quantities that emerge from energy.
	
	\section{No New Formulas, But a Liberated View of Old Formulas}
	
	This approach does \emph{not} lead to completely new equations. We consider the \emph{same} fundamental formulas of quantum mechanics and relativity – but in a \emph{new reference frame} where all constants are dimensionless, i.e., "natural". This seemingly small change has far-reaching consequences because it reveals the \emph{limitations} and \emph{gaps} of previous theories:
	
	1.  \textbf{Incompleteness of quantum mechanics (from existing formulas):} The \emph{known} formulas of quantum mechanics, transferred into this new system, no longer describe \emph{all} phenomena correctly. They are \emph{incomplete} because they do not fully capture the dynamic relationship between mass, time, and \emph{energy}.
	
	2.  \textbf{Extension within the existing framework:} Quantum mechanics \emph{must} be extended. But this extension is not achieved through arbitrary new assumptions, but through a \emph{more consistent} application of the \emph{already existing} principles, particularly energy conservation and the inseparable connection between mass and time.
	
	3.  \textbf{Dual perspectives as the key to reality:} The wave-particle duality and the time-mass duality are not mere "interpretations". They are \emph{hints} that we overlook or misinterpret aspects of reality when we cling to traditional, restricted perspectives. They point the way to a \emph{more real}, \emph{more intuitive}, and \emph{more unified} description of the physical world.
	
	\section{Concrete Implications: Toward a More Comprehensive Theory}
	
	This "liberated" perspective on physics has concrete implications:
	
	*   \textbf{Quantum gravity:} A unification based on an \emph{extended} and \emph{more consistent} QM becomes more tangible.
	*   \textbf{Quantum entanglement:} The interpretation through intrinsic time questions the \emph{current} QM and opens new perspectives.
	*   \textbf{Dark energy/matter:} New, \emph{concrete} relationships emerge between mass, energy, and the expansion of the universe that go beyond current models.
	*   \textbf{Fundamental constants:} A \emph{deeper} understanding, as all constants are traced back to \emph{one} fundamental quantity (energy).
	
	\section{Experimental Verification and Conclusion: A New Beginning}
	
	This approach is not only theoretical but \emph{experimentally verifiable}. It makes *different* predictions than the *current*, incomplete QM (e.g., with precision clocks and entangled particles of different mass).
	
	The time-mass dualism, the "naturalization" of all constants, and the resulting extension of quantum mechanics are a radical but promising path. They show that we must \emph{fundamentally} reconsider physics – not by discarding proven formulas, but by \emph{liberating} ourselves from traditional constraints and returning to a \emph{more real}, \emph{more intuitive}, and above all \emph{more unified} perspective. It is a new beginning toward a more comprehensive theory that could solve the great mysteries of the universe.
\end{document}