\documentclass[a4paper,12pt]{article}
\usepackage[utf8]{inputenc}
\usepackage[english]{babel} % For English language and hyphenation
\usepackage{amsmath, amssymb}
\usepackage{physics}
\usepackage{hyperref}
\usepackage{geometry}

\geometry{a4paper, margin=2.5cm}

\begin{document}
	
	\title{Dynamic Mass of Photons and its Implications for Nonlocality}
	\author{Johann Pascher}
	\date{March 25, 2025}
	\maketitle
	
	\tableofcontents % Table of contents
	\newpage % Optional: Starts the main content on a new page
	
	\section{Introduction}
	This work investigates the consequences of a dynamic, frequency-dependent mass for photons within different time models in quantum mechanics. Particular emphasis is placed on the implications for nonlocality and causality. Developing a consistent theory for massless particles like photons presents a particular challenge, which is solved here by an energy-dependent mass assignment.
	
	\section{Natural Units as a Foundation}
	\subsection{Definition of Natural Units}
	To simplify the theoretical considerations, we use natural units with $\hbar = c = G = 1$. This choice is not only a mathematical simplification but also reveals fundamental relationships between physical quantities:
	
	\begin{itemize}
		\item The speed of light $c = 299,792,458 \text{ m/s}$ becomes $c = 1$, which allows space and time to be measured in the same units.
		\item The reduced Planck constant $\hbar$ becomes $\hbar = 1$, which highlights the quantum nature of all processes.
		\item The gravitational constant $G$ becomes $G = 1$, which completes the unification of mass, space, and time.
	\end{itemize}
	
	In this system, all physical quantities are dimensionless or can be reduced to a single fundamental dimension (usually energy):
	
	\begin{align}
		\text{Length}: [L] &= \text{Energy}^{-1} \\
		\text{Time}: [T] &= \text{Energy}^{-1} \\
		\text{Mass}: [M] &= \text{Energy}
	\end{align}
	
	This unification clarifies that mass and energy are equivalent ($E = m$), which is central to our treatment of photons.
	
	\subsection{Significance for Mass-Energy Equivalence}
	In natural units, the famous formula $E = mc^2$ becomes $E = m$, making the equivalence of mass and energy directly apparent. For photons with frequency $\omega$, we have:
	
	\begin{equation}
		E = \hbar\omega = \omega \quad \text{(in natural units with $\hbar = 1$)}
	\end{equation}
	
	This leads to a remarkable consequence: photons can be assigned a frequency-dependent dynamic mass:
	
	\begin{equation}
		m_{\gamma} = \frac{E}{c^2} = E = \omega \quad \text{(in natural units)}
	\end{equation}
	
	This assignment is crucial for the integration of photons into different time models.
	
	\section{Time Models in Quantum Mechanics}
	\subsection{Limits of the Standard Model}
	In the standard model of quantum mechanics and relativity, time is treated as a continuous, relativistic variable. The Schrödinger equation $i\hbar\frac{\partial\psi}{\partial t} = H\psi$ describes the time evolution of quantum systems. However, for photons with rest mass $m_0 = 0$, this leads to difficulties in describing their dynamics, especially in the context of entanglement and nonlocality.
	
	\subsection{The $T_0$ Model with Absolute Time}
	The $T_0$ model introduces an absolute time, where $T_0 = \text{const.}$ In this model, time is invariant, while mass becomes variable:
	
	\begin{equation}
		m = \gamma m_0 = \frac{m_0}{\sqrt{1-v^2/c^2}}
	\end{equation}
	
	Energy is defined in this model as:
	
	\begin{equation}
		E = \frac{\hbar}{T_0}
	\end{equation}
	
	However, for photons with $m_0 = 0$, this model fails because no unique mass variation can be defined.  With the assignment of a dynamic mass $m_{\gamma} = \omega$, photons can be integrated into the $T_0$ model.
	
	\subsection{The Model with Intrinsic Time}
	An alternative formulation defines an intrinsic, mass-dependent time:
	
	\begin{equation}
		T = \frac{\hbar}{mc^2}
	\end{equation}
	
	This leads to a modified Schrödinger equation:
	
	\begin{equation}
		i\hbar\frac{\partial\psi}{\partial (t/T)} = H\psi
	\end{equation}
	
	For classical massive particles, this results in a well-defined time scale.  However, for photons with traditionally $m = 0$, it would follow that $T \rightarrow \infty$, which would bring the time evolution to a standstill.
	
	\subsection{Extension for Photons: Energy-Dependent Time}
	To solve this problem, we extend the time definition for massless particles to an energy-dependent form:
	
	\begin{equation}
		T = \frac{1}{E}
	\end{equation}
	
	With the assignment $m_{\gamma} = \omega = E$ for photons, we get:
	
	\begin{equation}
		T = \frac{\hbar}{m_{\gamma}c^2} = \frac{\hbar}{Ec^2} = \frac{1}{E} \quad \text{(in natural units)}
	\end{equation}
	
	This fundamental relationship shows that the energy-dependent time definition for photons is a natural extension of the mass-dependent intrinsic time when we assign a dynamic mass to photons. For a photon with energy $E = 2\pi\nu$ (where $\nu$ is the frequency), we obtain $T = \frac{1}{2\pi\nu}$, which corresponds exactly to its period.
	
	\section{Unification of the Models}
	For a unified treatment of all particles, both massless and massive, we can introduce a hybrid time definition:
	
	\begin{equation}
		T = \frac{1}{\max(m, E)}
	\end{equation}
	
	This definition reduces to the original intrinsic time $T = \frac{1}{m}$ for massive particles with $m > E$ and to the energy-dependent time $T = \frac{1}{E}$ for photons.
	
	\section{Implications for Nonlocality and Entanglement}
	\subsection{Energy-Dependent Correlations}
	The introduction of an energy-dependent time scale for photons has far-reaching consequences for the interpretation of nonlocality and entanglement:
	
	\begin{enumerate}
		\item \textbf{Frequency-Dependent Dynamics}: Photons of different frequencies evolve with different intrinsic time scales.
		
		\item \textbf{Delayed Correlations}: For entangled photons with energies $E_1$ and $E_2$, different time scales $T_1 = \frac{1}{E_1}$ and $T_2 = \frac{1}{E_2}$ result. This leads to a delay in the correlation of $|T_1 - T_2| = \left|\frac{1}{E_1} - \frac{1}{E_2}\right|$, in contrast to the instantaneous correlation in the standard model.
		
		\item \textbf{Hybrid Systems}: In entangled systems of photons and massive particles (e.g., electron with $m_e \approx 5.11 \times 10^5 \text{ eV}$ and photon with $E \approx 1 \text{ eV}$), typically $T_{\text{Photon}} \gg T_e$ holds, which implies a significant delay in the state change of the photon.
		
		\item \textbf{Energy-Dependent Light Cone}: The causal structure remains limited by $c = 1$, but the time evolution within the light cone is modified in an energy-dependent way.
	\end{enumerate}
	
	\subsection{Mass-Dependent and Energy-Dependent Causality}
	A fundamental aspect is the emergence of an energy- or mass-dependent causal structure. The metric structure is modified to:
	
	\begin{equation}
		ds^2 = c^2dT^2 - d\vec{x}^2 = \frac{1}{m^2}dt^2 - d\vec{x}^2 \quad \text{(for massive particles)}
	\end{equation}
	
	\begin{equation}
		ds^2 = c^2dT^2 - d\vec{x}^2 = \frac{1}{E^2}dt^2 - d\vec{x}^2 \quad \text{(for photons)}
	\end{equation}
	
	This implies that particles of different mass or energy experience different causal structures, which fundamentally changes the interpretation of nonlocality. Nonlocality thus appears not as an instantaneous connection over arbitrary distances, but as an emergent phenomenon determined by the intrinsic time scales of the involved particles.
	
	\section{Experimental Verification}
	The presented theory leads to specific experimental predictions:
	
	\begin{itemize}
		\item \textbf{Frequency-Dependent Bell Tests}: Measurements of the correlations in entangled photon pairs of different frequencies should show delays proportional to $\left|\frac{1}{E_1} - \frac{1}{E_2}\right|$.
		
		\item \textbf{Hybrid Entanglement Experiments}: In systems with entangled photons and electrons, the correlation times should be determined by the ratio of the intrinsic times $\frac{T_{\text{Photon}}}{T_e} = \frac{m_e}{E}$.
		
		\item \textbf{Coherence Times in Quantum Optics}: The coherence times of quantum states should exhibit an energy-dependent component that goes beyond the known decoherence effects.
	\end{itemize}
	
	Due to the typically very small intrinsic time scales (in the femtosecond range for visible light), these experiments represent a considerable technical challenge but could be feasible with ultrafast measurement techniques.
	
	\section{Physics Beyond the Speed of Light}
	The introduction of dynamic masses for photons and energy-dependent time scales also opens up new perspectives for considering hypothetical superluminal phenomena:
	
	\begin{itemize}
		\item In the $T_0$ model with absolute time, tachyons (hypothetical particles with $v > c$) could be described by real mass variations without the causality problems of the standard model.
		
		\item A modified energy-momentum relation could be formulated as:
		\begin{equation}
			E^2 = (mc^2)^2 + (pc)^2 + \alpha_c p^4 c^2 / E_P^2
		\end{equation}
		where $\alpha_c$ is a dimensionless parameter and $E_P$ is the Planck energy.
	\end{itemize}
	
	These extensions would retain standard physics as a limiting case while providing a theoretical framework for exploring phenomena beyond the current model boundaries.
	
	\section{Conclusion}
	The introduction of a dynamic, frequency-dependent mass for photons and the resulting energy-dependent intrinsic time provides a consistent framework for the integration of photons into different time models of quantum mechanics. This extension leads to a novel interpretation of nonlocality as an energy-dependent, emergent phenomenon instead of an instantaneous correlation.
	
	The presented theory remains within the established physical principles but leads to experimentally testable predictions that differ from the standard model. The results of these experiments could provide important insights into the nature of time, causality, and nonlocality and possibly pave the way for a more comprehensive theory of quantum gravity.
	
\end{document}