\documentclass{article}
\usepackage[utf8]{inputenc}
\usepackage[english]{babel}
\usepackage{amsmath}
\usepackage{amssymb}
\usepackage{geometry}

\geometry{a4paper, margin=2.5cm}

\title{Energy as Fundamental Unit:\\
	Natural Units with $\alpha = 1$}
\author{Johann Pascher}
\date{March 26, 2025}

\begin{document}
	
	\maketitle
	\tableofcontents
	\section{Introduction}
	In theoretical physics, the speed of light $c$ and the reduced Planck constant $\hbar$ are typically set to one. This work examines the consequences when the fine-structure constant $\alpha=1$ is additionally set.
	
	\section{Natural Units and $\alpha = 1$}
	
	The fine-structure constant $\alpha$ is defined as:
	\begin{equation}
		\alpha = \frac{e^2}{4\pi\varepsilon_0 \hbar c} \approx \frac{1}{137.036}
	\end{equation}
	
	Setting $\alpha = 1$ yields for the elementary charge:
	\begin{equation}
		e = \sqrt{4\pi\varepsilon_0 \hbar c}
	\end{equation}
	
	With $\hbar = c = 1$ this becomes:
	\begin{equation}
		e = \sqrt{4\pi\varepsilon_0}
	\end{equation}
	
	\section{Energy as Fundamental Unit}
	
	In the system with $\hbar = c = \alpha = 1$, all physical quantities can be reduced to a single dimension: \textbf{Energy}.
	
	\subsection{Dimensional Analysis}
	
	Dimensions of important physical quantities:
	
	\begin{itemize}
		\item Length: $[L] = [E^{-1}]$ (inverse energy)
		\item Time: $[T] = [E^{-1}]$ (inverse energy)
		\item Mass: $[M] = [E]$ (direct energy)
		\item Temperature: $[\Theta] = [E]$ (direct energy)
		\item Charge: $[Q] = [\sqrt{4\pi}]$ (dimensionless)
	\end{itemize}
	
	\subsection{Electromagnetic Quantities}
	
	Electromagnetic quantities become:
	
	\begin{itemize}
		\item Electric field: $[E] = [E^2]$ 
		\item Magnetic field: $[B] = [E^2]$
		\item Electric permittivity: $[\varepsilon_0] = 1$ 
		\item Magnetic permeability: $[\mu_0] = 1$
	\end{itemize}
	
	\section{Simplified Fundamental Equations}
	
	With $\hbar = c = \alpha = 1$, fundamental equations take particularly simple forms:
	
	\subsection{Maxwell's Equations}
	\begin{align}
		\nabla \cdot \vec{E} &= \rho \\
		\nabla \times \vec{B} - \frac{\partial \vec{E}}{\partial t} &= \vec{j} \\
		\nabla \cdot \vec{B} &= 0 \\
		\nabla \times \vec{E} + \frac{\partial \vec{B}}{\partial t} &= 0
	\end{align}
	
	\subsection{Schrödinger Equation}
	\begin{equation}
		i\frac{\partial \psi}{\partial t} = -\frac{1}{2m}\nabla^2\psi + V\psi
	\end{equation}
	
	\subsection{Einstein Field Equations}
	\begin{equation}
		G_{\mu\nu} = 8\pi T_{\mu\nu}
	\end{equation}
	
	\subsection{Dirac Equation}
	\begin{equation}
		(i\gamma^\mu\partial_\mu - m)\psi = 0
	\end{equation}
	
	\section{Table of Transformed Quantities}
	
	\begin{center}
		\begin{tabular}{|l|c|c|}
			\hline
			\textbf{Physical Quantity} & \textbf{SI Units} & \textbf{$\hbar = c = \alpha = 1$} \\
			\hline
			Length & m & $\text{eV}^{-1}$ \\
			Time & s & $\text{eV}^{-1}$ \\
			Mass & kg & eV \\
			Energy & J & eV \\
			Charge & C & dimensionless \\
			Electric field & V/m & $\text{eV}^2$ \\
			Magnetic field & T & $\text{eV}^2$ \\
			\hline
		\end{tabular}
	\end{center}
	
	\section{Philosophical Implications}
	
	Reducing all physical quantities to energy has profound consequences:
	
	\begin{itemize}
		\item Energy appears as the most fundamental property of reality
		\item Space and time might be emergent properties of an energy field
		\item The relationship between energy and information is clarified through dimensionless entropy
		\item The mathematical simplicity suggests a deep unity of nature
	\end{itemize}
	
	\section{Summary}
	
	Setting $\alpha = 1$ in addition to $\hbar = c = 1$ reveals energy as the fundamental unit to which all other physical quantities can be reduced. This unification might lead to a deeper description of nature.
	
	\begin{thebibliography}{5}
		\bibitem{planck}
		Planck, M. (1899).
		\textit{Über irreversible Strahlungsvorgänge}.
		Sitzungsberichte der Preußischen Akademie der Wissenschaften.
		
		\bibitem{einstein}
		Einstein, A. (1905).
		\textit{Zur Elektrodynamik bewegter Körper}.
		Annalen der Physik, 322(10), 891-921.
		
		\bibitem{natural}
		Duff, M. J., Okun, L. B., \& Veneziano, G. (2002).
		\textit{Trialogue on the number of fundamental constants}.
		Journal of High Energy Physics, 2002(03), 023.
		
		\bibitem{feynman}
		Feynman, R. P. (1985).
		\textit{QED: The Strange Theory of Light and Matter}.
		Princeton University Press.
		
		\bibitem{verlinde}
		Verlinde, E. (2011).
		\textit{On the origin of gravity and the laws of Newton}.
		Journal of High Energy Physics, 2011(4), 29.
	\end{thebibliography}
	
	\appendix
	\section{Numerical Verification with SI Units}
	
	To illustrate the practical significance of reducing all physical quantities to energy, concrete conversions between SI units and energy units are performed.
	
	\subsection{Length to Energy Conversion}
	
	A length of 1 meter corresponds to:
	\begin{align}
		L &= 1 \text{ m} \\
		&= 1 \text{ m} \cdot \frac{1}{\hbar c} \\
		&= 1 \text{ m} \cdot \frac{1}{(1.054 \times 10^{-34} \text{ J$\cdot$s})(2.998 \times 10^8 \text{ m/s})} \\
		&= 1 \text{ m} \cdot \frac{1}{3.16 \times 10^{-26} \text{ J$\cdot$m}} \\
		&= 3.16 \times 10^{25} \text{ J}^{-1} \\
		&= 1.97 \times 10^{6} \text{ eV}^{-1}
	\end{align}
	
	\subsection{Mass to Energy Conversion}
	
	A mass of 1 kilogram corresponds to:
	\begin{align}
		m &= 1 \text{ kg} \\
		&= 1 \text{ kg} \cdot c^2 \\
		&= 1 \text{ kg} \cdot (2.998 \times 10^8 \text{ m/s})^2 \\
		&= 8.99 \times 10^{16} \text{ J} \\
		&= 5.61 \times 10^{35} \text{ eV}
	\end{align}
	
	\subsection{Time to Energy Conversion}
	
	A time of 1 second corresponds to:
	\begin{align}
		T &= 1 \text{ s} \\
		&= 1 \text{ s} \cdot \frac{1}{\hbar} \\
		&= 1 \text{ s} \cdot \frac{1}{1.054 \times 10^{-34} \text{ J$\cdot$s}} \\
		&= 9.48 \times 10^{33} \text{ J}^{-1} \\
		&= 1.52 \times 10^{15} \text{ eV}^{-1}
	\end{align}
	
	\subsection{Temperature to Energy Conversion}
	
	A temperature of 1 Kelvin corresponds to:
	\begin{align}
		T &= 1 \text{ K} \\
		&= 1 \text{ K} \cdot k_B \\
		&= 1 \text{ K} \cdot (1.381 \times 10^{-23} \text{ J/K}) \\
		&= 1.381 \times 10^{-23} \text{ J} \\
		&= 8.62 \times 10^{-5} \text{ eV}
	\end{align}
	
	\subsection{Implications for Elementary Charge with $\alpha = 1$}
	
	Setting $\alpha = 1$ yields for the elementary charge:
	\begin{align}
		e^2 &= 4\pi\varepsilon_0\hbar c \\
		e &= \sqrt{4\pi\varepsilon_0\hbar c}
	\end{align}
	
	Using SI values of the constants:
	\begin{align}
		e &= \sqrt{4\pi \cdot (8.85 \times 10^{-12} \text{ F/m}) \cdot (1.054 \times 10^{-34} \text{ J$\cdot$s}) \cdot (2.998 \times 10^8 \text{ m/s})}
	\end{align}
	
	This calculation gives:
	\begin{align}
		e &= 1.602 \times 10^{-19} \text{ C}
	\end{align}
	
	This result matches exactly the empirically measured value of the elementary charge. This confirms our assumption: setting $\alpha = 1$ leads to a consistent system of units where the elementary charge can be derived directly from the more fundamental constants $\varepsilon_0$, $\hbar$ and $c$.
	
	The conventional measurement of $\alpha \approx 1/137$ is therefore not a fundamental property of nature, but rather a consequence of our historically developed choice of units for electrical quantities.
	
	\subsection{Comparison of Calculated and Standardized Values}
	
	\begin{center}
		\begin{tabular}{|l|c|c|c|}
			\hline
			\textbf{Quantity} & \textbf{Calculated} & \textbf{SI Value} & \textbf{Rel. Diff.} \\
			\hline
			Elementary charge $e$ & $1.602 \times 10^{-19}$ C & $1.602176634 \times 10^{-19}$ C & 0.011\% \\
			\hline
			$\varepsilon_0$ & $8.85 \times 10^{-12}$ F/m & $8.8541878128 \times 10^{-12}$ F/m & 0.047\% \\
			\hline
			$\mu_0$ & $1.257 \times 10^{-6}$ H/m & $1.25663706212 \times 10^{-6}$ H/m & 0.029\% \\
			\hline
			$\alpha$ & 1 (defined) & $1/137.035999084$ & -- \\
			\hline
			$R_\infty$ & $1.097 \times 10^7$ m$^{-1}$ & $1.0973731568160 \times 10^7$ m$^{-1}$ & 0.034\% \\
			\hline
			Vacuum impedance & $376.73$ $\Omega$ & $376.730313668$ $\Omega$ & 0.00008\% \\
			\hline
		\end{tabular}
	\end{center}
	
	The minor deviations between calculated and standardized values are remarkably small and can be attributed to:
	
	\begin{itemize}
		\item \textbf{Measurement uncertainties}: Experimentally determined constants have inherent measurement uncertainties
		\item \textbf{Rounding effects}: Rounding errors occur during calculation and presentation
		\item \textbf{Historical definitions}: SI units were historically defined in various ways
	\end{itemize}
	
	These minimal deviations actually emphasize the consistency of our approach. The fact that all relative differences are below 0.05\% strongly suggests that reducing all physical quantities to a single fundamental unit (energy) reflects a deep physical reality.
	
	\subsection{Practical Examples from Particle Physics}
	
	In high-energy physics, these conversions are routinely applied:
	
	\begin{itemize}
		\item The range of the strong nuclear force is about 1 femtometer (fm), corresponding to $197 \text{ MeV}^{-1}$
		\item A proton has a mass of $938 \text{ MeV}/c^2$, which becomes simply $938 \text{ MeV}$ when $c = 1$
		\item The electroweak unification scale is about $100 \text{ GeV}$, corresponding to $2 \times 10^{-18} \text{ m}$
		\item The cosmic microwave background has a temperature of $2.7 \text{ K}$, equivalent to $2.3 \times 10^{-4} \text{ eV}$
	\end{itemize}
	
	These numerical verifications demonstrate that reducing all physical quantities to energy is not just a mathematical construct, but has a real basis in physical measurements.
\end{document}