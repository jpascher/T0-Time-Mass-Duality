\documentclass[a4paper,12pt]{article}
\usepackage[utf8]{inputenc}
\usepackage[german]{babel} % Für deutsche Sprache und Silbentrennung
\usepackage{amsmath, amssymb}
\usepackage{physics}
\usepackage{hyperref}
\usepackage{geometry}
\geometry{a4paper, margin=2.5cm}

\title{Concise Summary - Complementary Dualism in Physics: From Wave-Particle to Time-Mass Concept}
\author{Johann Pascher}
\date{March 25, 2025}


\begin{document}
	\maketitle
	
	\section{Introduction: Dualism in Modern Physics}
	
	Modern physics is based on dualistic concepts.  The wave-particle duality is one of the fundamental principles that describes how objects like electrons or photons can exhibit both wave and particle properties. These seemingly contradictory descriptions are both correct and complementary.
	
	However, the two main pillars of modern physics -- quantum mechanics (QM) and quantum field theory (QFT) -- themselves represent a kind of dualism. While QM emphasizes the discrete, particle-like nature of matter, QFT focuses on field concepts and continuous aspects.  However, both theories are incomplete:
	
	\begin{itemize}
		\item \textbf{Quantum mechanics} describes quantum phenomena but cannot fully integrate relativistic effects.
		\item \textbf{Quantum field theory} combines quantum effects with special relativity but reaches its limits with the theory of gravity.
	\end{itemize}
	
	Based on this dualism already established in physics, I introduce a new, analogous dualism in my work "Complementary Extensions of Physics: Absolute Time and Intrinsic Time": the time-mass dualism. This could help to close some of the existing gaps between the established theories.
	
	\section{From Particles and Waves to Time and Mass}
	
	\subsection{The Classical Wave-Particle Duality}
	
	In quantum mechanics, we have two complementary descriptions of the same phenomenon:
	
	\begin{itemize}
		\item The \textbf{particle description} focuses on localized objects with defined position and mass.
		\item The \textbf{wave description} considers the phenomenon as a spatially extended wave function.
	\end{itemize}
	
	Mathematically, these descriptions are connected by the Fourier transform:
	\begin{align}
		\Psi(\vec{x}) &= \frac{1}{(2\pi\hbar)^{3/2}} \int \phi(\vec{p}) e^{i\vec{p}\cdot\vec{x}/\hbar} d^3p \\
		\phi(\vec{p}) &= \frac{1}{(2\pi\hbar)^{3/2}} \int \Psi(\vec{x}) e^{-i\vec{p}\cdot\vec{x}/\hbar} d^3x
	\end{align}
	
	\subsection{The New Time-Mass Dualism}
	
	Analogously, I propose that we can consider two complementary descriptions for relativistic phenomena:
	
	\begin{itemize}
		\item The \textbf{time dilation description} (Standard Model): Time is variable ($t' = \gamma t$), while the rest mass remains constant.
		\item The \textbf{mass variation description} (complementary model): Time is absolute ($T_0 = \text{const.}$), while the mass is variable ($m = \gamma m_0$).
	\end{itemize}
	
	Mathematically, these descriptions are also connected by a transformation, which I refer to as the modified Lorentz transformation.
	
	\section{The Concept of Intrinsic Time}
	
	A remarkable concept emerges from the complementary model: intrinsic time. This is defined as:
	\begin{equation}
		T = \frac{\hbar}{mc^2}
	\end{equation}
	
	Intrinsic time is a fundamental property of every object, depending on its mass. It leads to a modified Schrödinger equation:
	\begin{equation}
		i\hbar \frac{\partial}{\partial (t/T)} \Psi = \hat{H} \Psi
	\end{equation}
	
	This means that heavier objects experience a faster internal time evolution than lighter objects -- a kind of "proper time" in the quantum mechanical sense.
	
	\section{The Parallels between the Dualisms}
	
	The parallels between wave-particle duality and time-mass duality are profound:
	
	\begin{enumerate}
		\item \textbf{Complementarity:} Just as position and momentum are complementary observables, time and energy/mass are complementary quantities.
		
		\item \textbf{Uncertainty relations:} The $\Delta x \Delta p \geq \frac{\hbar}{2}$ of wave-particle duality corresponds to $\Delta t \Delta E \geq \frac{\hbar}{2}$ or $\Delta T \Delta m \geq \frac{\hbar}{2c^2}$ in time-mass dualism.
		
		\item \textbf{Transformations:} Both dualisms are connected by mathematical transformations.
	\end{enumerate}
	
	\section{Necessary Extensions of QM and QFT}
	
	Based on the time-mass dualism, I propose concrete extensions to the existing theories:
	
	\subsection{Extension of Quantum Mechanics}
	
	The classical Schrödinger equation must be extended to take intrinsic time into account:
	
	\begin{equation}
		i\hbar \frac{\partial}{\partial (t/T)} \Psi = \hat{H} \Psi
	\end{equation}
	
	This modification leads to:
	\begin{itemize}
		\item A mass-dependent time evolution of quantum systems.
		\item A natural explanation for different decay rates and coherence times.
		\item A new perspective on the measurement problem through the connection of mass and time evolution.
	\end{itemize}
	
	\subsection{Extension of Quantum Field Theory}
	
	QFT must be extended to integrate absolute time or mass-dependent intrinsic time:
	
	\begin{itemize}
		\item Field operators must be reformulated in terms of intrinsic time $T = \frac{\hbar}{mc^2}$.
		\item Renormalization can be reinterpreted through mass-dependent time scales.
		\item Virtual particles could be understood as manifestations of different intrinsic time scales.
	\end{itemize}
	
	These extensions could be particularly fruitful for:
	\begin{itemize}
		\item The integration of gravity into quantum field theory.
		\item The resolution of infinities in quantum field theories.
		\item A deeper understanding of vacuum energy and the cosmological constant.
	\end{itemize}
	
	\section{The Reality of Time Dilation versus Mass Variation}
	
	A central objection to the concept of absolute time is that we can directly measure time dilation -- for example, in GPS corrections or muon decay. However, in my work, I show that all these measurements are fundamentally based on frequency measurements:
	\begin{equation}
		f = \frac{E}{h} = \frac{mc^2}{h}
	\end{equation}
	
	These measurements can therefore be interpreted equally as time dilation or as mass variation. The experimental data are identical -- only our interpretation changes.
	
	\section{Implications for Instantaneity and Nonlocality}
	
	Nonlocality in quantum physics, especially in entangled particles, is often understood as an instantaneous effect over arbitrary distances. My models offer an alternative interpretation of this apparent instantaneity:
	
	\begin{itemize}
		\item In the $T_0$ model with absolute time, quantum correlations could be explained by mass variation rather than by temporal effects. Since time remains absolute, correlations would not be instantaneous, but would arise through a dynamic mass adjustment ($m = \gamma m_0$).
		
		\item In the model with intrinsic time, entangled particles of different masses would experience different time evolutions. A lighter particle with a larger $T$ would react more slowly to changes in state than a heavier particle with a smaller $T$.
		
		\item For photons, the intrinsic time could be defined as $T = \frac{1}{E} = \frac{1}{p}$, which corresponds to the wavelength. A more energetic (shorter wavelength) photon would thus experience a faster time evolution than a less energetic one.
	\end{itemize}
	
	This approach replaces the counterintuitive instantaneous effect with a systematic, mass-dependent dynamic that could be empirically verifiable. Specifically, Bell tests with particles of different masses or photons of different frequencies could show measurable delays in the correlations, proportional to the mass ratio $\frac{m_1}{m_2}$ or energy ratio $\frac{E_1}{E_2}$.
	
	\section{Consequences and Outlook}
	
	The proposed time-mass dualism offers new perspectives that could go beyond the incompleteness of existing theories:
	
	\begin{itemize}
		\item An alternative conceptual framework for problems of quantum gravity, which could address the incompleteness of QFT with respect to gravity.
		\item A new interpretation of nonlocality through mass-dependent time evolution, which could resolve the apparent contradiction between quantum entanglement and the theory of relativity.
		\item A natural connection between discrete quantum phenomena (QM) and continuous fields (QFT) through the concept of intrinsic time.
		\item Possible experimental tests that could differentiate between the models.
	\end{itemize}
	
	Just as wave-particle duality revolutionized quantum mechanics, time-mass dualism could provide new insights for a more complete theory. While QM and QFT each represent parts of the puzzle, time-mass dualism potentially offers a unifying framework that could close the existing gaps between these theories.
	
	The complete mathematical derivations and detailed implications are presented in detail in my works:
	
	\begin{itemize}
		\item "Complementary Extensions of Physics: Absolute Time and Intrinsic Time" (24.03.2025)
		\item "A Model with Absolute Time and Variable Energy: A Detailed Investigation of the Foundations" (24.03.2025)
		\item "Dynamic Mass of Photons and its Implications for Nonlocality" (25.03.2025)
		\item "Fundamental Constants and Their Derivation from Natural Units" (25.03.2025)
		\item "Real Consequences of the Reformulation of Time and Mass in Physics: Beyond the Planck Scale" (24.03.2025)
	\end{itemize}
	
\end{document}