\documentclass[a4paper,12pt]{article}
\usepackage[utf8]{inputenc}
\usepackage[english]{babel}
\usepackage{amsmath, amssymb, amsthm}
\usepackage{physics}
\usepackage{graphicx}
\usepackage{hyperref}
\usepackage{tikz}
\usepackage{setspace}
\usepackage{tcolorbox}
\usepackage{xcolor}
\usepackage{textalpha}  % For Unicode Greek
\usepackage{textgreek}  % For \textdelta, \textalpha etc.
%\usepackage{unicode-math}  % Must be placed BEFORE fontspec if used
%\usepackage{fontspec}      % For XeLaTeX/LuaLaTeX
\newtheorem{theorem}{Theorem}
\newtheorem{lemma}[theorem]{Lemma}
\newtheorem{proposition}[theorem]{Proposition}
\newtheorem{corollary}[theorem]{Corollary}
\newtheorem{definition}{Definition}
\begin{document}


\title{Dark Energy in the T0 Model: \\A Mathematical Analysis of Energy Dynamics}
\author{Johann Pascher}
\date{March 26, 2025}
\maketitle

\begin{abstract}
	This paper develops a detailed mathematical analysis of dark energy within the framework of the T0 model with absolute time and variable mass. In contrast to the $\Lambda$CDM standard model, dark energy is not considered a driving force of cosmic expansion but rather a dynamic medium for energy exchange in a static universe. The document derives the corresponding field equations, characterizes energy transfer rates, analyzes the radial density profile of dark energy, and explains the observed redshift as a result of energy loss by photons. Finally, specific experimental tests are proposed to distinguish between this interpretation and the standard model.
\end{abstract}

\tableofcontents
\newpage

\section{Introduction}

The discovery of accelerated cosmic expansion through supernova observations in the late 1990s led to the introduction of dark energy as the dominant component of the universe. In the standard cosmological model ($\Lambda$CDM), dark energy is modeled as a cosmological constant ($\Lambda$) with negative pressure, accounting for approximately 68\% of the universe's energy content and driving the accelerated expansion.

This work pursues an alternative approach based on the T0 model, where time is absolute and particle masses vary instead. Within this framework, dark energy is not regarded as a force driving expansion but as a medium for energy exchange interacting with matter and radiation. Cosmic redshift is explained not by spatial expansion but by the energy loss of photons to dark energy.

In the following, we will mathematically refine this approach, derive the necessary field equations, determine the energy density and distribution of dark energy, and analyze the implications for astronomical observations. Subsequently, we will explore experimental tests that could differentiate between the T0 model and the standard model.

\section{Fundamentals of the T0 Model for Dark Energy}

We first summarize the basic concepts of the T0 model with respect to dark energy.

\subsection{Fundamental Assumptions}

In contrast to the standard model, where spacetime is dynamic and expands while rest mass remains constant, the T0 model postulates:

\begin{tcolorbox}[colback=blue!5!white,colframe=blue!75!black,title=Fundamental Assumptions of the T0 Model]
	\begin{align}
		&\text{1. Time $T_0$ is absolute and universally constant.} \\
		&\text{2. Mass varies according to $m = \gamma m_0$, where $\gamma = \frac{1}{\sqrt{1-v^2/c_0^2}}$.} \\
		&\text{3. The total energy of the universe is constant.} \\
		&\text{4. Redshift arises from energy loss: $E_2 = E_1(1+z)^{-1}$.}
	\end{align}
\end{tcolorbox}

For dark energy, this implies that it is not treated as a homogeneous background density expanding space but as a dynamic field capable of exchanging energy with matter and radiation, while the total energy of the universe remains constant.

\subsection{Dark Energy as a Dynamic Field}

In the T0 model, dark energy is modeled as a scalar field $\phi_{DE}$ that interacts with matter and radiation. The energy density of this field is not constant but exhibits a spatial structure:

\begin{equation}
	\rho_{DE}(r) = \frac{\kappa}{r^2}
\end{equation}

where $\kappa$ is a constant and $r$ denotes the radial distance. This $1/r^2$ profile significantly differs from the constant energy density $\rho_\Lambda$ of the cosmological constant in the standard model.

The coupling between dark energy and matter/radiation can be described by an interaction term in the Lagrangian density:

\begin{equation}
	\mathcal{L}_{int} = -\frac{\beta}{M_{Pl}} \phi_{DE} T^{\mu}_{\mu}
\end{equation}

Here, $\beta$ is a dimensionless coupling constant, $M_{Pl}$ is the Planck mass, and $T^{\mu}_{\mu}$ is the trace of the energy-momentum tensor of matter and radiation.

\section{Field-Theoretic Description of Dark Energy}

We now develop a complete field-theoretic description of dark energy in the T0 model.

\subsection{Lagrangian Density of Dark Energy}

The complete Lagrangian density for the dark energy field is:

\begin{equation}
	\mathcal{L}_{DE} = -\frac{1}{2}\partial_\mu \phi_{DE} \partial^\mu \phi_{DE} - V(\phi_{DE}) - \frac{\beta}{M_{Pl}} \phi_{DE} T^{\mu}_{\mu} - \frac{1}{2}\xi \phi_{DE}^2 R
\end{equation}

where:
\begin{itemize}
	\item $\partial_\mu \phi_{DE} \partial^\mu \phi_{DE}$ is the kinetic term,
	\item $V(\phi_{DE})$ is the self-interaction potential of the field,
	\item $\frac{\beta}{M_{Pl}} \phi_{DE} T^{\mu}_{\mu}$ is the coupling to matter and radiation,
	\item $\frac{1}{2}\xi \phi_{DE}^2 R$ is a non-minimal coupling to spacetime curvature $R$.
\end{itemize}

For a static universe in the T0 model, we must choose a suitable potential $V(\phi_{DE})$ that allows for a stable equilibrium:

\begin{equation}
	V(\phi_{DE}) = \frac{1}{2}m_{\phi}^2\phi_{DE}^2 + \lambda \phi_{DE}^4
\end{equation}

where $m_{\phi}$ is the mass of the dark energy field and $\lambda$ is the self-coupling constant.

\subsection{Field Equations of Dark Energy}

The Euler-Lagrange equations for the dark energy field follow from the Lagrangian density:

\begin{equation}
	\Box\phi_{DE} - \frac{dV}{d\phi_{DE}} - \frac{\beta}{M_{Pl}}T^{\mu}_{\mu} - \xi \phi_{DE} R = 0
\end{equation}

which simplifies to:

\begin{equation}
	\Box\phi_{DE} - m_{\phi}^2\phi_{DE} - 4\lambda\phi_{DE}^3 - \frac{\beta}{M_{Pl}}T^{\mu}_{\mu} - \xi \phi_{DE} R = 0
\end{equation}

For a static, spherically symmetric system, this equation reduces to:

\begin{equation}
	\frac{1}{r^2}\frac{d}{dr}\left(r^2\frac{d\phi_{DE}}{dr}\right) = m_{\phi}^2\phi_{DE} + 4\lambda\phi_{DE}^3 + \frac{\beta}{M_{Pl}}T^{\mu}_{\mu} + \xi \phi_{DE} R
\end{equation}

\subsection{Energy Density Profile of Dark Energy}

For a massless field ($m_{\phi} \approx 0$) and negligible curvature ($\xi R \approx 0$), the field equation simplifies to:

\begin{equation}
	\frac{1}{r^2}\frac{d}{dr}\left(r^2\frac{d\phi_{DE}}{dr}\right) = 4\lambda\phi_{DE}^3 + \frac{\beta}{M_{Pl}}T^{\mu}_{\mu}
\end{equation}

We seek a solution that reproduces the observed $1/r^2$ density profile of dark energy. For large distances $r$, where $T^{\mu}_{\mu} \approx 0$ (negligible matter density), the self-interaction term $\lambda$ dominates. Choosing an ansatz $\phi_{DE}(r) \propto r^{-\alpha}$, we find $\alpha = 1/2$ by substitution and coefficient comparison, yielding:

\begin{equation}
	\phi_{DE}(r) \approx \left(\frac{1}{8\lambda}\right)^{1/3} r^{-1/2} \quad \text{for } r \gg r_0
\end{equation}

The energy density of dark energy is then:

\begin{equation}
	\rho_{DE}(r) \approx \frac{1}{2}\left(\frac{d\phi_{DE}}{dr}\right)^2 + \frac{1}{2}m_{\phi}^2\phi_{DE}^2 + \lambda\phi_{DE}^4 \approx \frac{\kappa}{r^2}
\end{equation}

with $\kappa \propto \lambda^{-2/3}$. This $1/r^2$ profile is precisely what is needed to explain flat rotation curves in galaxies.

\section{Energy Exchange and Redshift}

A central aspect of the T0 model is the interpretation of cosmic redshift as a result of photon energy loss to dark energy, rather than spatial expansion.

\subsection{Photon Energy Loss}

We consider a photon moving through the dark energy field. The energy change of the photon is described by:

\begin{equation}
	\frac{dE_{\gamma}}{dx} = -\alpha E_{\gamma}
\end{equation}

where $\alpha$ is the absorption rate. This equation has the solution:

\begin{equation}
	E_{\gamma}(x) = E_{\gamma,0} e^{-\alpha x}
\end{equation}

where $E_{\gamma,0}$ is the initial photon energy and $x$ is the distance traveled.

The redshift $z$ is defined as:

\begin{equation}
	1 + z = \frac{E_0}{E} = \frac{\lambda_{obs}}{\lambda_{emit}} = e^{\alpha d}
\end{equation}

where $d$ is the distance. For small $z$ (local distances), this approximates to:

\begin{equation}
	z \approx \alpha d
\end{equation}

To ensure consistency with the observed Hubble relation $z \approx H_0 d/c$, we must have:

\begin{equation}
	\alpha = \frac{H_0}{c} \approx 2.3 \times 10^{-28} \text{ m}^{-1}
\end{equation}

where $H_0 \approx 70 \text{ km/s/Mpc}$ is the Hubble constant.

\subsection{Energy Transfer to Dark Energy}

The energy lost by the photon is transferred to the dark energy field. Energy conservation requires:

\begin{equation}
	\frac{d}{dt}(E_{\gamma} + E_{DE}) = 0
\end{equation}

The rate at which dark energy gains energy is:

\begin{equation}
	\frac{dE_{DE}}{dt} = -\frac{dE_{\gamma}}{dt} = \alpha c E_{\gamma}
\end{equation}

For the energy density of dark energy, this implies:

\begin{equation}
	\frac{d\rho_{DE}}{dt} = \alpha c \rho_{\gamma}
\end{equation}

where $\rho_{\gamma}$ is the energy density of photons.

\subsection{Energy Balance Equation}

In a static universe with constant total energy, we must consider the energy balance. The total energy density $\rho$ is composed of:

\begin{equation}
	\rho_{total} = \rho_{matter} + \rho_{\gamma} + \rho_{DE} = const.
\end{equation}

The balance equation for the time evolution of energy densities is:

\begin{align}
	\frac{d\rho_{matter}}{dt} &= -\alpha_{m} c \rho_{matter} \\
	\frac{d\rho_{\gamma}}{dt} &= -\alpha_{\gamma} c \rho_{\gamma} \\
	\frac{d\rho_{DE}}{dt} &= \alpha_{m} c \rho_{matter} + \alpha_{\gamma} c \rho_{\gamma}
\end{align}

where $\alpha_{m}$ and $\alpha_{\gamma}$ are the energy transfer rates for matter and photons, respectively.

Assuming $\alpha_{\gamma} = \alpha_{m} = \alpha$ (equal transfer rates for all energy forms), the time evolution of the energy densities becomes:

\begin{align}
	\rho_{matter}(t) &= \rho_{matter,0} e^{-\alpha c t} \\
	\rho_{\gamma}(t) &= \rho_{\gamma,0} e^{-\alpha c t} \\
	\rho_{DE}(t) &= \rho_{DE,0} + (\rho_{matter,0} + \rho_{\gamma,0})(1 - e^{-\alpha c t})
\end{align}

For large times ($t \gg (\alpha c)^{-1}$), the universe approaches a state where all energy is in the form of dark energy:

\begin{equation}
	\lim_{t \rightarrow \infty} \rho_{DE}(t) = \rho_{total} = \rho_{DE,0} + \rho_{matter,0} + \rho_{\gamma,0}
\end{equation}

\section{Quantitative Determination of Parameters}

Based on astronomical observations, we can quantitatively estimate the parameters of the T0 model.

\subsection{Total Energy Density of the Universe}

The critical density of the universe is:

\begin{equation}
	\rho_{crit} = \frac{3H_0^2}{8\pi G} \approx 8.5 \times 10^{-27} \text{ kg/m}^3
\end{equation}

In the standard model, dark energy accounts for approximately 68\% of the critical density:

\begin{equation}
	\rho_{\Lambda} \approx 0.68 \rho_{crit} \approx 5.8 \times 10^{-27} \text{ kg/m}^3
\end{equation}

In the T0 model, this density does not correspond to a homogeneous background but to the average of an inhomogeneous field with a $1/r^2$ dependence.

\subsection{Absorption Coefficient and Hubble Constant}

From the relation $\alpha = H_0/c$ and the observed value $H_0 \approx 70 \text{ km/s/Mpc}$, we obtain:

\begin{equation}
	\alpha \approx 2.3 \times 10^{-28} \text{ m}^{-1}
\end{equation}

This extremely small absorption rate explains why photon energy loss to dark energy is not measurable in laboratory experiments but becomes significant over cosmological distances.

\subsection{Coupling Constant to Matter}

The dimensionless coupling constant $\beta$, which describes the interaction between dark energy and matter, can be estimated from the analysis of galaxy rotation curves:

\begin{equation}
	\beta \approx 10^{-3}
\end{equation}

This value is small enough to pass local gravity tests but large enough to explain cosmological effects.

\subsection{Self-Interaction of the Dark Energy Field}

The self-interaction constant $\lambda$ in $V(\phi_{DE}) = \lambda \phi_{DE}^4$ determines the density profile of dark energy. From the relation $\kappa \propto \lambda^{-2/3}$ and the observed value $\kappa \approx 4.8 \times 10^{-7} \text{ GeV/cm} \cdot \text{s}^{-2}$ (from galaxy rotation curves), we can estimate $\lambda$:

\begin{equation}
	\lambda \approx 10^{-120}
\end{equation}

This extremely small self-interaction poses a challenge for the model, similar to the hierarchy problem in the standard model.

\section{Dark Energy and Cosmological Observations}

We now analyze how the T0 model explains various cosmological observations attributed to dark energy in the standard model.

\subsection{Type Ia Supernovae and Cosmic Acceleration}

The observation that Type Ia supernovae at large distances appear fainter than expected in a pure matter universe led to the discovery of "cosmic acceleration." In the $\Lambda$CDM model, this is explained by the accelerated expansion of the universe driven by dark energy with negative pressure.

In the T0 model, an alternative explanation emerges: Photons lose energy to the dark energy field as they travel through the universe, increasing their wavelength (redshift) and decreasing their intensity. The brightness-redshift relationship is then described by:

\begin{equation}
	m - M = 5 \log_{10}(d_L) + 25
\end{equation}

with the luminosity distance:

\begin{equation}
	d_L = \frac{c}{H_0} \ln(1+z) (1+z)
\end{equation}

in contrast to the standard formula:

\begin{equation}
	d_L^{\Lambda CDM} = \frac{c}{H_0} \int_0^z \frac{dz'}{\sqrt{\Omega_m(1+z')^3 + \Omega_\Lambda}}
\end{equation}

Both formulas can fit the observed data equally well, but with different physical interpretations.

\subsection{Cosmic Microwave Background (CMB)}

The CMB exhibits nearly perfect blackbody radiation with a temperature of T = 2.725 K and tiny temperature fluctuations ($\delta T/T \sim 10^{-5}$). In the $\Lambda$CDM model, this is interpreted as a relic of the early, hot universe cooled by cosmic expansion.

In the T0 model, the CMB is regarded as a static thermal field whose temperature is determined by the balance between energy input (e.g., from stars and galaxies) and energy loss to dark energy. The observed anisotropies arise from local variations in the energy density of the dark energy field.

The CMB power spectrum, particularly the characteristic acoustic peaks, must be reinterpreted in this framework. While in the $\Lambda$CDM model these peaks are explained by baryon acoustic oscillations before recombination, in the T0 model they must be understood as a consequence of density fluctuations in the static dark energy field.

\subsection{Large-Scale Structure and Baryon Acoustic Oscillations (BAO)}

The distribution of galaxies shows a characteristic length scale of about 150 Mpc, interpreted in the $\Lambda$CDM model as a result of baryon acoustic oscillations before recombination. This length scale serves as a "standard ruler" for measuring cosmic expansion.

In the T0 model, this length scale must be explained differently, without relying on expansion. A possible explanation is that mass variation and energy exchange with the dark energy field generate characteristic length scales in structure formation.

The mathematical description of these processes requires a detailed analysis of perturbation equations in the T0 model:

\begin{equation}
	\nabla^2 \delta\phi_{DE} - m_{\phi}^2 \delta\phi_{DE} - 12\lambda\phi_{DE}^2 \delta\phi_{DE} = \frac{\beta}{M_{Pl}}\delta T^{\mu}_{\mu}
\end{equation}

where $\delta\phi_{DE}$ is the fluctuation of the dark energy field and $\delta T^{\mu}_{\mu}$ is the fluctuation in the matter distribution.

\section{Experimental Tests and Predictions}

The T0 model of dark energy makes specific predictions that could distinguish it from the cosmological constant of the standard model.

\subsection{Temporal Variation of the Fine Structure Constant}

Since photons in the T0 model lose energy to the dark energy field, this could lead to a temporal variation of fundamental constants, particularly the fine structure constant $\alpha_{fs}$. The rate of change would be:

\begin{equation}
	\frac{d\alpha_{fs}}{dt} \approx \alpha_{fs} \cdot \alpha \cdot c \approx 10^{-18} \text{ year}^{-1}
\end{equation}

This variation is extremely small but could be measured through high-precision spectroscopy of distant quasars. Such measurements could provide evidence of whether the T0 model is consistent with observed cosmological data.

\subsection{Environmental Dependence of Redshift}

Since dark energy in the T0 model is a dynamic field with spatial variations, the absorption rate $\alpha$ should depend on the local energy density:

\begin{equation}
	\alpha(r) = \alpha_0 \cdot \left(1 + \delta\frac{\rho_{baryon}(r)}{\rho_0}\right)
\end{equation}

where $\delta$ is a parameter describing the strength of the coupling. This leads to a prediction: Redshift should differ slightly in dense cosmic regions (e.g., galaxy clusters) compared to cosmic voids. Specifically:

\begin{equation}
	\frac{z_{cluster}}{z_{void}} \approx 1 + \delta\frac{\rho_{cluster} - \rho_{void}}{\rho_0}
\end{equation}

This deviation could be tested through precise measurements of redshift in different cosmic environments.

\subsection{Anomalous Light Propagation in Strong Gravitational Fields}

Since dark energy in the T0 model couples to matter, its density should be higher near massive objects. This leads to modified light propagation, particularly in strong gravitational fields such as near black holes or in galaxy clusters.

The effective refractive index of space would be:

\begin{equation}
	n_{eff}(r) = 1 + \epsilon \frac{\phi_{DE}(r)}{M_{Pl}}
\end{equation}

where $\epsilon$ is a parameter dependent on the precise coupling between the dark energy field and the electromagnetic field.

This anomalous light propagation could manifest as subtle deviations from the gravitational lensing effects predicted by general relativity.

\subsection{Differential Redshift}

Another prediction of the T0 model concerns the wavelength dependence of redshift. Since photon absorption by the dark energy field might be wavelength-dependent, especially if the coupling exhibits frequency dependence:

\begin{equation}
	\alpha(\lambda) = \alpha_0 \cdot \left(1 + \eta\frac{\lambda}{\lambda_0}\right)
\end{equation}

This would result in a differential redshift, where different wavelengths from the same object exhibit slightly different redshifts:

\begin{equation}
	\frac{z(\lambda_1)}{z(\lambda_2)} \approx 1 + \eta\frac{\lambda_1 - \lambda_2}{\lambda_0}
\end{equation}

This prediction could be tested through high-resolution spectroscopy of distant quasars.

\section{Statistical Analysis and Comparison with the Standard Model}

To compare the predictions of the T0 model with the standard model, we conduct a statistical analysis.

\subsection{Bayesian Model Comparison}

We use Bayesian statistics to quantify the evidence for the T0 model compared to the $\Lambda$CDM model. The Bayes evidence is given by:

\begin{equation}
	E(M) = \int L(\theta|D,M) \pi(\theta|M) d\theta
\end{equation}

where $L(\theta|D,M)$ is the likelihood of the data $D$ given the parameters $\theta$ in model $M$, and $\pi(\theta|M)$ is the prior distribution of the parameters.

The Bayes ratio between the models is:

\begin{equation}
	B_{T_0,\Lambda CDM} = \frac{E(T_0)}{E(\Lambda CDM)}
\end{equation}

This ratio quantifies how strongly the observational data favor one model over the other.

\subsection{Fitting to Supernova Data}

Supernova data can be fitted with both the standard model and the T0 model. In the $\Lambda$CDM model, the distance modulus-redshift relationship is described by:

\begin{equation}
	\mu(z) = 5 \log_{10}\left[\frac{c}{H_0}(1+z)\int_0^z \frac{dz'}{\sqrt{\Omega_m(1+z')^3 + \Omega_{\Lambda}}}\right] + 25
\end{equation}

while in the T0 model:

\begin{equation}
	\mu(z) = 5 \log_{10}\left[\frac{c}{H_0}(1+z)\ln(1+z)\right] + 25
\end{equation}

Both models have free parameters (($\Omega_m$, $\Omega_{\Lambda}$, $H_0$) for $\Lambda$CDM and ($\alpha$, $H_0$) for T0) that can be adjusted to the data.

\subsection{Analysis of the CMB Power Spectrum}

The power spectrum of the cosmic microwave background provides a critical test for both models. In the $\Lambda$CDM model, the spectrum is determined by acoustic oscillations before recombination, while in the T0 model, the structure must be explained by density fluctuations in the static dark energy field. The mathematical description of the CMB power spectrum in the T0 model requires a detailed treatment of density fluctuations in the dark energy field:

\begin{equation}
	P(k) = \langle|\delta\phi_{DE}(k)|^2\rangle
\end{equation}

where $\delta\phi_{DE}(k)$ is the Fourier transform of the dark energy field fluctuations.

This theoretical prediction can then be compared with observed data, particularly measurements from the Planck satellite.

\section{Implications for the Future of the Universe}

The two models differ dramatically in their predictions for the future of the universe.

\subsection{Future Evolution in the $\Lambda$CDM Model}

In the standard model, the constant energy density of dark energy leads to an accelerated expansion that becomes increasingly rapid. The future of the universe is either a "Big Rip" or eternal expansion, depending on the exact equation of state of dark energy.

The scale factor evolution follows:

\begin{equation}
	\frac{\ddot{a}}{a} = -\frac{4\pi G}{3}(\rho_m + 3p_\Lambda) = -\frac{4\pi G}{3}\rho_m + \frac{8\pi G}{3}\rho_\Lambda
\end{equation}

Since $\rho_m \propto a^{-3}$ decreases with time while $\rho_\Lambda = const.$, the expansion becomes increasingly accelerated in the long term.

\subsection{Future Evolution in the T0 Model}

In the T0 model, there is no true expansion of the universe, but rather a continuous conversion of matter and radiation energy into dark energy. The energy densities evolve according to:

\begin{align}
	\rho_{matter}(t) &= \rho_{matter,0} e^{-\alpha c t} \\
	\rho_{\gamma}(t) &= \rho_{\gamma,0} e^{-\alpha c t} \\
	\rho_{DE}(t) &= \rho_{DE,0} + (\rho_{matter,0} + \rho_{\gamma,0})(1 - e^{-\alpha c t})
\end{align}

In the long term, the universe approaches a state where all energy exists in the form of dark energy—a "thermal death," but without spatial expansion.

\subsection{Comparison of Long-Term Forecasts}

\begin{tcolorbox}[colback=yellow!5!white,colframe=yellow!75!black,title=Long-Term Evolution of the Universe]
	\begin{tabular}{|p{0.45\textwidth}|p{0.45\textwidth}|}
		\hline
		\textbf{$\Lambda$CDM Model} & \textbf{T0 Model} \\
		\hline
		Accelerated expansion & No expanding space \\
		\hline
		Galaxies move apart increasingly faster & Galaxies remain in the same space but lose energy \\
		\hline
		Eventual dilution of all matter & Continuous conversion of matter into dark energy \\
		\hline
		Ends in a "Big Rip" or eternal expansion & Ends in a state dominated by the dark energy field \\
		\hline
	\end{tabular}
\end{tcolorbox}

\section{Bibliography}

\begin{thebibliography}{99}
	
	\bibitem{pascher} Pascher, J. (2025). A Model with Absolute Time and Variable Energy: A Comprehensive Study of the Foundations.
	
	\bibitem{pascher2} Pascher, J. (2025). Extensions of Quantum Mechanics through Intrinsic Time.
	
	\bibitem{pascher3} Pascher, J. (2025). Complementary Extensions of Physics: Absolute Time and Intrinsic Time.
	
	\bibitem{supernova} Perlmutter, S., et al. (1999). Measurements of $\Omega$ and $\Lambda$ from 42 High-Redshift Supernovae. The Astrophysical Journal, 517, 565.
	
	\bibitem{riess} Riess, A. G., et al. (1998). Observational Evidence from Supernovae for an Accelerating Universe and a Cosmological Constant. The Astronomical Journal, 116, 1009.
	
	\bibitem{planck} Planck Collaboration. (2020). Planck 2018 results. VI. Cosmological parameters. Astronomy \& Astrophysics, 641, A6.
	
	\bibitem{cmb} Bennett, C. L., et al. (2013). Nine-year Wilkinson Microwave Anisotropy Probe (WMAP) Observations: Final Maps and Results. The Astrophysical Journal Supplement Series, 208, 20.
	
	\bibitem{bao} Eisenstein, D. J., et al. (2005). Detection of the Baryon Acoustic Peak in the Large-Scale Correlation Function of SDSS Luminous Red Galaxies. The Astrophysical Journal, 633, 560.
	
	\bibitem{quintessence} Caldwell, R. R., Dave, R., Steinhardt, P. J. (1998). Cosmological Imprint of an Energy Component with General Equation of State. Physical Review Letters, 80, 1582.
	
	\bibitem{euclid} Laureijs, R., et al. (2011). Euclid Definition Study Report. ESA/SRE(2011)12.
	
	\bibitem{tired} Zwicky, F. (1929). On the Red Shift of Spectral Lines through Interstellar Space. Proceedings of the National Academy of Sciences, 15, 773.
	
	\bibitem{alfa} Webb, J. K., et al. (2011). Indications of a Spatial Variation of the Fine Structure Constant. Physical Review Letters, 107, 191101.
	
	\bibitem{vacuum} Weinberg, S. (1989). The Cosmological Constant Problem. Reviews of Modern Physics, 61, 1.
	
	\bibitem{scalar} Fujii, Y., Maeda, K. (2003). The Scalar-Tensor Theory of Gravitation. Cambridge University Press.
	
	\bibitem{lambda} Carroll, S. M. (2001). The Cosmological Constant. Living Reviews in Relativity, 4, 1.
	
\end{thebibliography}
\end{document}