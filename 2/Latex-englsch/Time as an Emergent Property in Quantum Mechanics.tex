\documentclass{article}
\usepackage[utf8]{inputenc}
\usepackage{amsmath}
\usepackage{amssymb}
\usepackage{geometry}
\usepackage{hyperref}
\usepackage{siunitx}

\title{Time as an Emergent Property in Quantum Mechanics: \\A Connection Between Relativity Theory, Fine-Structure Constant, and Quantum Dynamics}

\author{Johann Pascher}
\date{March 22, 2025}
\begin{document}
	
	\maketitle
	
	\section{Introduction}
	
	In modern physics, time and space are treated differently. While spatial coordinates in quantum mechanics are represented by operators, time primarily appears as a parameter. This asymmetric treatment raises fundamental questions about the nature of time. This work explores the extent to which time can be understood as an emergent property related to fundamental constants and the mass of the system under consideration.
	
	\tableofcontents
	\section{Time in Special Relativity}
	
	Einstein's famous formula $E = mc^2$ connects energy and mass through the speed of light. This relationship contains no explicit time variable. To establish a connection with time, additional physical relationships must be considered.
	
	\subsection{Transformation of the Energy-Mass Equivalence}
	
	Starting from $E = mc^2$ and the quantum mechanical relationship between energy and frequency:
	\begin{align}
		E &= mc^2 \\
		E &= h\nu = \frac{h}{T}
	\end{align}
	
	where $h$ is Planck's constant, $\nu$ is the frequency, and $T$ is the period. By equating, we obtain:
	\begin{align}
		mc^2 &= \frac{h}{T} \\
		T &= \frac{h}{mc^2}
	\end{align}
	
	This time $T$ can be interpreted as a characteristic timescale associated with a mass $m$.
	
	\section{Connection to the Fine-Structure Constant}
	
	The fine-structure constant $\alpha$ is a dimensionless physical constant describing the strength of the electromagnetic interaction:
	\begin{equation}
		\alpha = \frac{e^2}{4\pi\varepsilon_0\hbar c} \approx \frac{1}{137.035999}
	\end{equation}
	
	\subsection{Derivation via Electromagnetic Constants}
	
	As shown in previous work, Planck's constant can be expressed through electromagnetic vacuum constants:
	\begin{equation}
		h = \frac{1}{2\pi\sqrt{\mu_0\varepsilon_0}}
	\end{equation}
	
	With this relationship, the characteristic time $T$ can be rewritten:
	\begin{align}
		T &= \frac{h}{mc^2} \\
		&= \frac{1}{2\pi\sqrt{\mu_0\varepsilon_0}} \cdot \frac{1}{mc^2}
	\end{align}
	
	Since $c = \frac{1}{\sqrt{\mu_0\varepsilon_0}}$, we obtain:
	\begin{align}
		T &= \frac{1}{2\pi\sqrt{\mu_0\varepsilon_0}} \cdot \frac{1}{m \cdot \frac{1}{\mu_0\varepsilon_0}} \\
		&= \frac{1}{2\pi m c^3}
	\end{align}
	
	\section{Time in Quantum Mechanics}
	
	\subsection{Standard Treatment of Time}
	
	In conventional quantum mechanics, time appears as a parameter in the Schrödinger equation:
	\begin{equation}
		i\hbar \frac{\partial}{\partial t}\Psi(x,t) = \hat{H}\Psi(x,t)
	\end{equation}
	
	Unlike spatial or momentum coordinates, there is no time operator. Time is treated as a continuous parameter along which quantum states evolve.
	
	\subsection{A New Perspective: Intrinsic Time}
	
	Consider the characteristic time $T = \frac{\hbar}{mc^2}$ as an "intrinsic time" of a quantum object. This time depends on the object's mass and could be interpreted as the minimal timescale on which the object can undergo quantum mechanical changes. It arises directly from the equivalence of energy and mass as well as the quantum mechanical energy-frequency relationship and could represent a fundamental lower limit for temporal processes, as later discussed in the context of instantaneity.
	
	The Schrödinger equation could be modified to account for this intrinsic time:
	\begin{equation}
		i\hbar \frac{\partial}{\partial (t/T)}\Psi = \hat{H}\Psi
	\end{equation}
	
	This would mean that time evolution is no longer uniform for all objects but depends on their mass.
	
	\section{Linking Time, Mass, and the Fine-Structure Constant}
	
	\subsection{A Unified Relationship}
	
	With the relationship $T = \frac{\hbar}{mc^2}$ and the definition of the fine-structure constant, we can establish a direct connection:
	\begin{align}
		T &= \frac{\hbar}{mc^2} \\
		&= \frac{\hbar}{mc^2} \cdot \frac{4\pi\varepsilon_0\hbar c}{e^2} \cdot \frac{e^2}{4\pi\varepsilon_0\hbar c} \\
		&= \frac{\hbar^2 \cdot 4\pi\varepsilon_0 c}{mc^2 \cdot e^2} \cdot \alpha
	\end{align}
	
	This shows that the intrinsic time $T$ is proportional to the fine-structure constant $\alpha$.
	
	\subsection{Interpretation in Natural Units}
	
	In a natural unit system where $c = \hbar = 1$, this relationship simplifies to:
	\begin{equation}
		T = \frac{\alpha}{m} \cdot \frac{4\pi\varepsilon_0}{e^2}
	\end{equation}
	
	And if we additionally set $\alpha = 1$, as discussed in the main document, we obtain:
	\begin{equation}
		T = \frac{1}{m} \cdot \frac{4\pi\varepsilon_0}{e^2}
	\end{equation}
	
	In a fully natural system where $e = 1$ and $\varepsilon_0 = \frac{1}{4\pi}$, the relationship becomes even simpler:
	\begin{equation}
		T = \frac{1}{m}
	\end{equation}
	
	This elegant relationship shows that in such a theoretical framework, the intrinsic time of an object is simply the reciprocal of its mass.
	
	\section{Consequences for Physics}
	
	\subsection{A New Perspective on Time}
	
	The idea that time could be an emergent property dependent on mass and fundamental interaction constants has profound implications:
	\begin{itemize}
		\item The conventional treatment of time as an independent parameter might be an approximation that works well for macroscopic objects.
		\item At a fundamental level, time could be a derived quantity, not a fundamental one.
		\item The "speed" of time evolution could differ for various quantum objects, depending on their mass.
	\end{itemize}
	
	\subsection{Connection to Time Dilation}
	
	Interestingly, this perspective resembles relativistic time dilation, albeit from a completely different theoretical framework. While relativity predicts that moving clocks run slower, our approach suggests that more massive quantum objects could have a "faster" intrinsic time evolution.
	
	\section{A Unified Picture of Time, Mass, and Interaction}
	
	The transformation presented here connects three fundamental aspects of physics:
	\begin{itemize}
		\item The relativistic energy-mass relationship ($E = mc^2$)
		\item The quantum mechanical energy-frequency relationship ($E = h\nu$)
		\item The strength of electromagnetic interaction (fine-structure constant $\alpha$)
	\end{itemize}
	
	This hints at a deeper connection between these seemingly disparate aspects of reality and could be seen as a step toward a more comprehensive theory.
	
	\section{Possibilities for Experimental Verification}
	
	The idea of a mass-dependent intrinsic time could have experimental consequences:
	\begin{itemize}
		\item Differences in coherence times of quantum systems with different masses
		\item Mass-dependent phase shifts in quantum interference experiments
		\item Specific signatures in the spectroscopy of particles with different masses
	\end{itemize}
	
	\section{Implications for Instantaneous Coherence in Quantum Mechanics}
	
	\subsection{The Problem of Instantaneous Coherence}
	
	Conventional quantum mechanics assumes that quantum superpositions and correlations extend instantaneously across the entire system. This becomes particularly evident in entangled states, where measurements on one particle can have immediate effects on the state of another, spatially separated particle.
	
	In a framework with mass-dependent intrinsic time $T = \frac{\hbar}{mc^2}$, we must reconsider this assumption.
	
	\subsection{Mass-Dependent Coherence Times}
	
	If the intrinsic time $T$ of a mass $m$ is inversely proportional, then heavier particles have shorter intrinsic timescales. This could mean that coherence phenomena for heavier quantum objects proceed faster relative to their own intrinsic timescale.
	
	Mathematically, we could express this through a modified decoherence rate:
	\begin{equation}
		\Gamma_{\text{dec}} = \Gamma_0 \cdot \frac{mc^2}{\hbar}
	\end{equation}
	
	where $\Gamma_0$ is the conventional decoherence rate. This would imply that heavier systems decohere more slowly in their intrinsic timescale but faster in an external laboratory time.
	
	\subsection{Mathematical Formulation for Multi-Particle Systems}
	
	For a system with two particles of different masses ($m_1$ and $m_2$), the joint wavefunction $\Psi(x_1, x_2, t)$ would have two different intrinsic timescales. The modified Schrödinger equation for this system could be formulated as:
	\begin{equation}
		i (m_1 + m_2) c^2 \frac{\partial}{\partial t} \Psi(x_1, x_2, t) = \hat{H} \Psi(x_1, x_2, t)
	\end{equation}
	
	This implies that the time evolution depends on the total mass of the system, representing a natural generalization of intrinsic time for multi-particle systems.
	
	\subsection{Implications for Entangled States}
	
	For entangled states with particles of different masses, for example:
	\begin{equation}
		|\Psi\rangle = \frac{1}{\sqrt{2}}(|0\rangle_{m_1} \otimes |1\rangle_{m_2} + |1\rangle_{m_1} \otimes |0\rangle_{m_2})
	\end{equation}
	
	the time evolution for the two particle components would differ:
	\begin{equation}
		|\Psi(t)\rangle = \frac{1}{\sqrt{2}}(|0(t/T_1)\rangle_{m_1} \otimes |1(t/T_2)\rangle_{m_2} + |1(t/T_1)\rangle_{m_1} \otimes |0(t/T_2)\rangle_{m_2})
	\end{equation}
	
	with $T_1 = \frac{\hbar}{m_1 c^2}$ and $T_2 = \frac{\hbar}{m_2 c^2}$. However, coherence is limited by the minimal timescale $T = \frac{\hbar}{mc^2}$, which follows from the energy-time uncertainty, as discussed later.
	
	\subsection{Modified Dispersion Relation}
	
	In standard quantum mechanics, the dispersion relation for a free particle wave is:
	\begin{equation}
		\hbar \omega = \frac{\hbar^2 k^2}{2m} \quad \Rightarrow \quad \omega = \frac{\hbar k^2}{2m}
	\end{equation}
	This relationship describes the frequency of a matter wave as a function of mass $m$ and wave vector $k$. With the introduction of intrinsic time $T = \frac{\hbar}{mc^2}$ as a characteristic timescale of a quantum object, we must examine how this affects the time evolution of the wavefunction.
	
	Consider a plane wave of the form $\Psi \sim e^{i(kx - \omega t)}$. If the time evolution is scaled relative to the intrinsic time $T$, we write the wavefunction as:
	\begin{equation}
		\Psi \sim e^{i(kx - \omega t / T)}
	\end{equation}
	Here, the time $t$ is normalized by $T$, modifying the effective frequency $\omega_{\text{eff}}$. The phase of the wavefunction now reads:
	\begin{equation}
		kx - \omega \frac{t}{T} = kx - \omega \frac{mc^2}{\hbar} t
	\end{equation}
	From this, it follows that the effective frequency relative to the intrinsic timescale is given by:
	\begin{equation}
		\omega_{\text{eff}} = \omega \cdot T = \omega \cdot \frac{\hbar}{mc^2}
	\end{equation}
	Substituting the standard frequency $\omega = \frac{\hbar k^2}{2m}$:
	\begin{equation}
		\omega_{\text{eff}} = \frac{\hbar k^2}{2m} \cdot \frac{\hbar}{mc^2} = \frac{\hbar^2 k^2}{2 m^2 c^2}
	\end{equation}
	This modified dispersion relation remains mass-dependent, unlike a mass-independent form that appeared as an artifact in earlier considerations. The dependence on $m^2$ in the denominator shows that heavier particles have a slower effective frequency, consistent with the interpretation of $T$ as an intrinsic timescale.
	
	Interestingly, this relationship differs significantly from standard quantum mechanics, where $\omega \propto \frac{1}{m}$. The new form $\omega_{\text{eff}} \propto \frac{1}{m^2}$ could lead to experimental differences in the propagation of matter waves, especially for particles of different masses. This could serve as a test for the mass-dependent time theory.
	
	\subsection{New Interpretation for the EPR Paradox and Bell's Inequalities}
	
	Our mass-dependent time theory could offer new interpretations for the EPR paradox and Bell's inequalities. If time is an emergent, mass-dependent property, the question arises whether "instantaneous" is a well-defined concept at the fundamental quantum level.
	
	An entangled system could be viewed in this framework as a composite object whose intrinsic timescale is determined by a combination of the masses of its components. This might cast the seemingly non-local "spooky action at a distance" in a new light.
	
	\subsection{Consistent Formulation of a Mass-Dependent Time Theory}
	
	To develop a fully consistent theory, we could reinterpret the Hamiltonian operator:
	\begin{equation}
		\hat{H}' = \frac{mc^2}{\hbar} \hat{H}
	\end{equation}
	
	This would transform the modified Schrödinger equation into:
	\begin{equation}
		i\hbar \frac{\partial}{\partial t}\Psi = \hat{H}\Psi
	\end{equation}
	
	This reverts to the original form but with a new interpretation: Time evolves at different "speeds" for different quantum objects, depending on their mass, while the relative energy levels and transitions remain preserved.
	
	\subsection{Instantaneous Processes and Minimal Timescales}
	
	An often-overlooked problem in interpreting quantum mechanics is the assumption of absolute instantaneity. The characteristic timescale of a particle $T = \frac{\hbar}{mc^2}$, however, establishes a fundamental minimal timescale. This means that even processes considered "instantaneous" require at least a time on the order of $T$.
	
	Consider the energy-time uncertainty relation:
	\begin{equation}
		\Delta E \cdot \Delta t \geq \frac{\hbar}{2}
	\end{equation}
	If $\Delta E \sim mc^2$, it follows:
	\begin{equation}
		\Delta t \gtrsim \frac{\hbar}{mc^2} = T
	\end{equation}
	This implies that no information can be transmitted in exactly zero time—there is a fundamental lower limit for any quantum interaction, directly following from existing principles.
	
	\section{Conclusions and Outlook}
	
	The transformation of Einstein's energy-mass equivalence to express time as a function of mass, the speed of light, and Planck's constant opens new conceptual perspectives. The further connection with the fine-structure constant shows how fundamental interactions could be related to the time evolution of quantum mechanical systems.
	
	The mass-dependent time theory presented here challenges fundamental assumptions of quantum mechanics, particularly the notion of a uniform, universal time and instantaneous coherence across spatial distances. The intrinsic time $T = \frac{\hbar}{mc^2}$, derived from $E = mc^2$ and $E = h\nu$, serves as a minimal timescale that restricts instantaneity, as confirmed by the energy-time uncertainty. This naturally connects relativity theory, quantum mechanics, and the fine-structure constant, offering a new approach to unifying these theories.
	
	These ideas are speculative and require further theoretical elaboration and experimental verification. However, they provide an interesting starting point for discussions about the nature of time in quantum physics and could potentially lead to new insights into the connection between quantum mechanics and relativity theory.
	
	Particularly promising is the possibility of developing experimental tests that can distinguish between the conventional treatment of time and our mass-dependent approach. The predicted differences in coherence times, dispersion relations, and the dynamics of entangled systems with different masses could provide experimentally verifiable signatures.
	
	\begin{thebibliography}{9}
		
		\bibitem{einstein} Einstein, A. (1905). Does the Inertia of a Body Depend Upon Its Energy Content? \textit{Annalen der Physik}, 323(13), 639-641.
		
		\bibitem{planck} Planck, M. (1901). On the Law of Energy Distribution in the Normal Spectrum. \textit{Annalen der Physik}, 309(3), 553-563.
		
		\bibitem{schrodinger} Schrödinger, E. (1926). An Undulatory Theory of the Mechanics of Atoms and Molecules. \textit{Physical Review}, 28(6), 1049-1070.
		
		\bibitem{sommerfeld} Sommerfeld, A. (1916). On the Quantum Theory of Spectral Lines. \textit{Annalen der Physik}, 356(17), 1-94.
		
		\bibitem{feynman} Feynman, R. P. (1985). QED: The Strange Theory of Light and Matter. Princeton University Press.
		
		\bibitem{rovelli} Rovelli, C. (2018). The Order of Time. Riverhead Books.
		
		\bibitem{pascher} Pascher, J. (2025). Fundamental Constants and Their Derivation from Natural Units.
		
		\bibitem{bell} Bell, J. S. (1964). On the Einstein Podolsky Rosen Paradox. \textit{Physics}, 1(3), 195-200.
		
		\bibitem{aspect} Aspect, A., Dalibard, J., \& Roger, G. (1982). Experimental Test of Bell's Inequalities Using Time-Varying Analyzers. \textit{Physical Review Letters}, 49(25), 1804-1807.
		
		\bibitem{zeh} Zeh, H. D. (1970). On the interpretation of measurement in quantum theory. \textit{Foundations of Physics}, 1(1), 69-76.
		
	\end{thebibliography}
	
\end{document}