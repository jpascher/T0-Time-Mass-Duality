\documentclass{article}
\usepackage[utf8]{inputenc}
\usepackage{amsmath}
\usepackage{amssymb}
\usepackage{hyperref}
\usepackage{geometry}
\usepackage{tocloft}
\geometry{a4paper, margin=2cm}

\title{Simplified Description of the Four Fundamental Forces with Time-Mass Dualism}
\author{Johann Pascher (revised)}
\date{March 25, 2025}

\begin{document}
	
	\maketitle
	
	\tableofcontents
	\newpage
	
	\section{Unified Lagrangian Density with Dual Time-Mass Concept}
	
	The Lagrangian density for the four fundamental forces (strong nuclear force, electromagnetic force, weak nuclear force, and gravity) can be summarized in a simplified form, which now also takes into account the time-mass dualism:
	
	\begin{equation}
		\mathcal{L}_\text{total} = \mathcal{L}_\text{Gravitation} + \mathcal{L}_\text{SM} + \mathcal{L}_\text{Higgs} + \mathcal{L}_\text{intrinsic},
	\end{equation}
	
	where:
	\begin{itemize}
		\item $\mathcal{L}_\text{Gravitation}$ describes the Lagrangian density of gravity,
		\item $\mathcal{L}_\text{SM}$ represents the Lagrangian density of the Standard Model (strong, electromagnetic, and weak force),
		\item $\mathcal{L}_\text{Higgs}$ is the Lagrangian density of the Higgs field,
		\item $\mathcal{L}_\text{intrinsic}$ represents the new Lagrangian density for taking into account the intrinsic time.
	\end{itemize}
	
	\subsection{Gravitation}
	Gravity is described by the Einstein-Hilbert action, but can now be represented in two complementary forms:
	
	\begin{equation}
		\mathcal{L}_\text{Gravitation} = -\frac{1}{16\pi G} \sqrt{-g} R,
	\end{equation}
	
	in the Standard Model (with time dilation), as well as:
	
	\begin{equation}
		\mathcal{L}_\text{Gravitation-T} = -\frac{1}{16\pi G_T} \sqrt{-g_T} R_T,
	\end{equation}
	
	in the complementary model (with absolute time and mass variation), where $G_T$ is a modified Newton constant that depends on the intrinsic time $T = \frac{\hbar}{mc^2}$.
	
	\subsection{Standard Model}
	The Lagrangian density of the Standard Model encompasses the strong, electromagnetic, and weak forces and can also be formulated dually:
	
	\begin{equation}
		\mathcal{L}_\text{SM} = \mathcal{L}_\text{stark} + \mathcal{L}_\text{em} + \mathcal{L}_\text{schwach},
	\end{equation}
	
	where:
	\begin{itemize}
		\item $\mathcal{L}_\text{stark} = -\frac{1}{4} F_{\mu\nu}^a F^{a\mu\nu} + \bar{\psi}(i \gamma^\mu D_\mu - m_\psi(\phi))\psi$ describes the strong nuclear force,
		\item $\mathcal{L}_\text{em} = -\frac{1}{4} F_{\mu\nu} F^{\mu\nu} + \bar{\psi}(i \gamma^\mu D_\mu - m_\psi(\phi))\psi$ describes the electromagnetic force,
		\item $\mathcal{L}_\text{schwach} = -\frac{1}{4} W_{\mu\nu}^a W^{a\mu\nu} + \bar{\psi}(i \gamma^\mu D_\mu - m_\psi(\phi))\psi$ describes the weak nuclear force.
	\end{itemize}
	
	The complementary formulation with intrinsic time is:
	
	\begin{equation}
		\mathcal{L}_\text{SM-T} = \mathcal{L}_\text{stark-T} + \mathcal{L}_\text{em-T} + \mathcal{L}_\text{schwach-T},
	\end{equation}
	
	where the time derivative is now with respect to the intrinsic time $T$: $\partial_t \rightarrow \partial_{t/T}$.
	
	\subsection{Higgs Field}
	The Lagrangian density of the Higgs field is:
	
	\begin{equation}
		\mathcal{L}_\text{Higgs} = (D_\mu \phi)^\dagger (D^\mu \phi) - V(\phi),
	\end{equation}
	
	where $\phi$ is the Higgs field and $V(\phi) = \mu^2 \phi^\dagger \phi + \lambda (\phi^\dagger \phi)^2$ describes the Higgs potential.
	
	In the complementary formulation with intrinsic time, this becomes:
	
	\begin{equation}
		\mathcal{L}_\text{Higgs-T} = (D_{T\mu} \phi_T)^\dagger (D_T^\mu \phi_T) - V_T(\phi_T),
	\end{equation}
	
	where the covariant derivative $D_{T\mu}$ takes into account the intrinsic time.
	
	\subsection{Lagrangian Density for Intrinsic Time}
	The new component that incorporates the time-mass dualism is:
	
	\begin{equation}
		\mathcal{L}_\text{intrinsic} = \bar{\psi}\left(i\hbar\gamma^0 \frac{\partial}{\partial (t/T)} - i\hbar\gamma^0 \frac{\partial}{\partial t}\right)\psi,
	\end{equation}
	
	where $T = \frac{\hbar}{mc^2}$ is the intrinsic time, which depends on the mass of the particle under consideration.
	
	\section{Simplified Description of Mass Terms with Time-Mass Dualism}
	
	The mass terms of the particles can now be represented in two ways:
	
	\begin{itemize}
		\item Standard Model (time dilation): $m_\psi(\phi) = y_\psi \phi$ with constant mass and variable time
		\item Complementary Model (mass variation): $m_\psi(\phi_T) = y_\psi \phi_T \cdot \gamma$ with absolute time and variable mass
	\end{itemize}
	
	The relationship $\gamma = \frac{1}{\sqrt{1-v^2/c^2}}$ holds as the Lorentz factor.
	
	\section{Asymptotic Safety with Intrinsic Time}
	
	Asymptotic safety in quantum gravity can be described by the modified renormalization group flow equation:
	
	\begin{equation}
		\partial_{t/T} \Gamma_k[g] = \frac{1}{2} \text{Tr}\left[\left(\Gamma_k^{(2)}[g] + R_k\right)^{-1} \partial_{t/T} R_k\right]
	\end{equation}
	
	with the effective action $\Gamma_k$ at scale $k$ and the regulator term $R_k$.
	
	The dimensionless couplings are adjusted accordingly:
	
	\begin{align}
		g_k &= G_k k^2 \rightarrow g_{k,T} = G_k (kT)^2 \\
		\lambda_k &= \Lambda_k/k^2 \rightarrow \lambda_{k,T} = \Lambda_k/(kT)^2
	\end{align}
	
	This leads to modified beta functions:
	
	\begin{align}
		\beta_g &= (2 + \eta_N)g_k \rightarrow \beta_{g,T} = (2 + \eta_N + \eta_T)g_{k,T} \\
		\beta_\lambda &= -2\lambda_k + f(g_k,\lambda_k) \rightarrow \beta_{\lambda,T} = -2\lambda_{k,T} + f_T(g_{k,T},\lambda_{k,T})
	\end{align}
	
	where $\eta_T$ represents the anomalous dimension with respect to intrinsic time.
	
	\section{The Higgs Field as a Universal Medium with Intrinsic Time}
	
	The concept of the Higgs field as a medium that plays a role for all other particles and fields is extended by the concept of intrinsic time.  The Higgs field could be responsible not only for mass generation, but also for the intrinsic time scale of particles:
	
	\begin{equation}
		T = \frac{\hbar}{m(\phi)c^2} = \frac{\hbar}{y_\psi \phi \cdot c^2}
	\end{equation}
	
	This relationship shows that the intrinsic time of a particle is inversely proportional to its mass generated by the Higgs field.
	
	\section{The Higgs Field and the Vacuum: A Complex Relationship with Intrinsic Time}
	
	The relationship between the Higgs field and the vacuum becomes even more complex with the concept of intrinsic time.  The vacuum energy could be reinterpreted as:
	
	\begin{equation}
		E_\text{vak} = \sum_i \frac{\hbar \omega_i}{2} = \sum_i \frac{\hbar}{2T_i}
	\end{equation}
	
	This formulation directly connects the vacuum energy to the intrinsic time of quantum fluctuations.
	
	\section{Quantum Entanglement and Nonlocality in Time-Mass Dualism}
	
	The apparent instantaneity in quantum entanglement can be reinterpreted through time-mass dualism:
	
	\begin{itemize}
		\item In the model with absolute time ($T_0$ model), correlations do not occur instantaneously, but through mass variation.
		\item In the model with intrinsic time, entangled particles of different masses would experience different time evolutions, proportional to their intrinsic time scales.
		\item For photons, the intrinsic time could be defined as $T = \frac{1}{E} = \frac{1}{p}$, which corresponds to their wavelength.
	\end{itemize}
	
	This leads to the prediction that Bell tests with particles of different masses or photons of different frequencies could show measurable delays in the correlations, proportional to the mass ratio $\frac{m_1}{m_2}$ or energy ratio $\frac{E_1}{E_2}$.
	
	\section{Summary of the Unified Theory}
	
	The complete unified theory can be described by the following action:
	
	\begin{equation}
		S_\text{unified} = \int \left( \mathcal{L}_\text{standard} + \mathcal{L}_\text{complementary} + \mathcal{L}_\text{coupling} \right) d^4x
	\end{equation}
	
	where:
	\begin{align}
		\mathcal{L}_\text{standard} &= -\frac{1}{16\pi G} \sqrt{-g} R + \mathcal{L}_\text{SM} + (D_\mu \phi)^\dagger (D^\mu \phi) - V(\phi) \\
		\mathcal{L}_\text{complementary} &= -\frac{1}{16\pi G_T} \sqrt{-g_T} R_T + \mathcal{L}_\text{SM-T} + (D_{T\mu} \phi_T)^\dagger (D_T^\mu \phi_T) - V_T(\phi_T) \\
		\mathcal{L}_\text{coupling} &= \int \mathcal{D}[\Psi] \, \Psi^* \left( i\hbar \frac{\partial}{\partial t} - i\hbar \frac{\partial}{\partial (t/T)} \right) \Psi
	\end{align}
	
	This unified theory offers several significant advantages:
	\begin{itemize}
		\item It bridges gaps between quantum mechanics and quantum field theory.
		\item It offers a new perspective on quantum entanglement and nonlocality.
		\item It opens new avenues for quantum gravity.
		\item It provides deeper insights into the Higgs field and vacuum.
		\item It leads to experimentally testable predictions.
	\end{itemize}
	
	\section{Experimental Verification}
	
	The presented unified theory with time-mass dualism leads to several experimentally testable predictions:
	
	\begin{enumerate}
		\item Mass-dependent time evolution in quantum systems, measurable as different coherence times.
		\item Differences in the speed of entanglement for particles of different masses.
		\item Scale-dependent gravitational constant that correlates with intrinsic time.
		\item Modified energy-momentum relation for very massive particles.
		\item Measurable deviations in high-precision experiments that are normally explained by time dilation.
	\end{enumerate}
	
	\section{References to Further Works}
	
	The unified theory presented here builds upon a series of detailed elaborations that address various aspects of time-mass dualism and its applications:
	
	\begin{itemize}
		\item \textbf{"Complementary Extensions of Physics: Absolute Time and Intrinsic Time" (24.03.2025)} \\
		This fundamental work introduces the $T_0$ model with absolute time and variable mass and presents the theoretical foundations of time-mass dualism. It contains the complete mathematical derivation of the transformation between the Standard Model (time dilation) and the complementary model (mass variation).
		
		\item \textbf{"A Model with Absolute Time and Variable Energy: A Detailed Investigation of the Foundations" (24.03.2025)} \\
		This work presents the detailed physical justification for the $T_0$ model and shows how it is compatible with experimental observations. It explains how phenomena usually attributed to time dilation can also be explained by mass variation.
		
		\item \textbf{"Extensions of Quantum Mechanics through Intrinsic Time" (24.03.2025)} \\
		This work develops the complete reformulation of quantum mechanics taking into account intrinsic time. It contains the derivation of the modified Schrödinger equation $i\hbar \frac{\partial}{\partial (t/T)} \Psi = \hat{H} \Psi$ and shows its applications to various quantum mechanical phenomena.
		
		\item \textbf{"Integration of Time-Mass Dualism into Quantum Field Theory" (25.03.2025)} \\
		This work presents the complete reformulation of quantum field theory with intrinsic time. It shows how field operators can be reformulated in terms of intrinsic time and how renormalization can be reinterpreted through mass-dependent time scales.
		
		\item \textbf{"Dynamic Mass of Photons and its Implications for Nonlocality" (25.03.2025)} \\
		This work explores the application of time-mass dualism to massless particles like photons and develops the concept of frequency-dependent intrinsic time for photons. It contains predictions for experiments to measure delays in quantum correlations.
		
		\item \textbf{"Fundamental Constants and Their Derivation from Natural Units" (25.03.2025)} \\
		This work shows how the fundamental constants of physics can be reinterpreted within the framework of time-mass dualism and how they are related to the concept of intrinsic time.
		
		\item \textbf{"Real Consequences of the Reformulation of Time and Mass in Physics: Beyond the Planck Scale" (24.03.2025)} \\
		This work examines the consequences of time-mass dualism for phenomena beyond the Planck scale and shows how it could solve potential problems of quantum gravity.
		
	\end{itemize}
	
	These further works contain the complete mathematical derivations, detailed calculations, and extensive discussions that would exceed the scope of the present simplified description.
	
\end{document}