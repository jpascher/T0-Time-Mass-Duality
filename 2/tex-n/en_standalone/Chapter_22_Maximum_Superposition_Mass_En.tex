% Chapter 22: Maximum Mass for Quantum Superposition (Adapted to T0 Theory)
% English Version

\section*{Chapter 22: Maximum Mass for Quantum Superposition}
\addcontentsline{toc}{section}{Chapter 22: Maximum Mass for Quantum Superposition}

\subsection*{1. Introduction}

This chapter presents the T0-grounded DVFT prediction for the maximum mass and size of molecules or macroscopic objects that can remain in quantum superposition.

This question is directly relevant to the MAST-QG (Macroscopic Superpositions for Quantum Gravity) project.

\textbf{T0 Adaptation:} DVFT provides a mathematically precise, physically motivated cutoff determined by the nonlinear response of the vacuum-phase field derived from T0's time-mass duality $T(x,t) \cdot m(x,t) = 1$. Unlike heuristic models such as Diòsi–Penrose (DP), the cutoff emerges from T0's fundamental parameter $\xi = 4/3 \times 10^{-4}$ with no free parameters.

\subsection*{2. T0-DVFT Mechanism for Superposition Stability}

\subsubsection*{2.1 Vacuum Field from T0}

T0-grounded DVFT describes the vacuum as a complex field:
\[
\Phi(x) = \rho(x) e^{i\theta(x)}
\]
with:
\begin{itemize}
\item $\rho(x) \propto m(x) = 1/T(x)$ — vacuum amplitude from T0's time-mass duality
\item $\theta(x)$ — vacuum phase from T0 node rotations
\end{itemize}

Derived from T0's single fundamental constant $\xi = 4/3 \times 10^{-4}$, which gives:
\begin{itemize}
\item $\rho_0 = 1/\xi^2 \approx 5.625 \times 10^7$ — equilibrium vacuum density
\item $B$ — vacuum phase stiffness $\sim 1/\xi^4$
\end{itemize}

\subsubsection*{2.2 Coherence Criterion}

Quantum coherence survives only when the two branches of a superposition satisfy:
\[
\theta_1(x) \approx \theta_2(x)
\]

Decoherence is not random: it occurs when the vacuum (derived from T0's time field) can no longer sustain two incompatible curvature configurations.

The collapse criterion from T0-DVFT is:
\[
E_\theta = \int |\nabla\theta_1 - \nabla\theta_2|^2 d^3x \geq B \rho_0
\]
where $B = 1/(\xi^4 \lambda_C^3)$ is the vacuum phase stiffness and $\rho_0 = 1/\xi^2$ is the vacuum inertial density—both derived from T0's $\xi$.

\subsection*{3. Collapse Condition Derived from T0}

\subsubsection*{3.1 Phase Curvature Mismatch from Mass Superposition}

A mass $m$ in two positions separated by distance $d$ produces two distinct curvature fields. In T0 Theory, the time field $T(x)$ near mass $m$ varies as:
\[
T(r) \approx T_0 \left(1 - \frac{GM}{c^2 r}\right)
\]

The vacuum phase $\theta$ responds via:
\[
|\nabla\theta| \approx \frac{Gm}{c^2 r^2}
\]

The curvature mismatch between the two branches scales as:
\[
|\Delta\nabla\theta| \approx \frac{Gmd}{c^2 r^3}
\]

and the total mismatch energy is approximately:
\[
E_\theta \approx \frac{G^2 m^2}{c^4 d}
\]

\subsubsection*{3.2 Maximum Mass for Stable Superposition}

The T0-DVFT collapse condition:
\[
E_\theta < B \rho_0 = \frac{1}{\xi^6 \lambda_C^3}
\]
yields the maximum mass:
\[
m_{\max} \approx \sqrt{\frac{B \rho_0 c^4 d}{G^2}} = \sqrt{\frac{c^4 d}{G^2 \xi^6 \lambda_C^3}}
\]

\textbf{Key Point:} All parameters derived from T0's $\xi = 4/3 \times 10^{-4}$ — no free parameters.

\subsection*{4. Numerical Estimates from T0 Constants}

Using T0-derived constants:
\begin{itemize}
\item $B \rho_0 \approx 1/(\xi^6 \lambda_C^3) \sim 10^{-9}$ J/m³
\item $d \approx 10^{-7}$ m (typical MAST-QG target separation)
\item $\lambda_C = \hbar/(m_0 c) \approx 2.4 \times 10^{-12}$ m (Compton wavelength)
\end{itemize}

we obtain:
\[
m_{\max} \approx 10^7 - 10^8 \text{ amu}
\]

This is the physical upper bound for stable quantum superposition according to T0 Theory.

\subsection*{5. Corresponding Size Limit}

Assuming molecular/organic matter density of $\sim$1000 kg/m³, the size corresponding to $m_{\max}$ is:
\[
R_{\max} \approx \left(\frac{3 m_{\max}}{4\pi\rho}\right)^{1/3} \approx 50-200 \text{ nm}
\]

Thus T0 Theory predicts the largest possible coherent object in our universe is approximately:
\begin{itemize}
\item \textbf{Mass:} $10^7-10^8$ amu
\item \textbf{Radius:} 50–200 nm
\item \textbf{Diameter:} $\sim$100 nm scale
\end{itemize}

Beyond this, vacuum-phase curvature (driven by T0's time-field nonlinearity) becomes unstable, and collapse is immediate.

\subsection*{6. Comparison with Other Collapse Models}

\begin{center}
\begin{tabular}{|l|l|l|}
\hline
\textbf{Model} & \textbf{Collapse Mass} & \textbf{Physical Mechanism} \\
\hline
Diòsi–Penrose & $\sim 10^9$ amu & Gravitational self-energy (heuristic) \\
T0-Grounded DVFT & $10^7-10^8$ amu & T0 time-field nonlinearity ($\xi$-derived) \\
Standard QM & No limit & No collapse mechanism \\
\hline
\end{tabular}
\end{center}

\subsubsection*{6.1 Diòsi–Penrose}

DP predicts collapse around $10^9$ amu using gravitational self-energy arguments.

T0-DVFT predicts earlier collapse ($10^7-10^8$ amu) due to nonlinear curvature terms in T0's time field that DP does not include.

\subsubsection*{6.2 Standard GR + QFT}

There is no predicted upper limit in standard theory.

T0 Theory contradicts this and provides a finite, experimentally falsifiable cutoff derived from $\xi$.

\subsection*{7. Implications for MAST-QG and Other Experiments}

T0-grounded DVFT provides the following predictions:
\begin{itemize}
\item Superpositions up to $\sim 10^7$ amu are stable.
\item At $\sim 10^8$ amu, collapse begins due to T0 time-field nonlinearity.
\item At $> 10^8-10^9$ amu, superposition is fundamentally impossible.
\end{itemize}

Therefore:
\begin{itemize}
\item \textbf{If MAST-QG observes superposition at $10^9-10^{10}$ amu} $\rightarrow$ T0 Theory is falsified.
\item \textbf{If collapse occurs in the $10^7-10^8$ amu window} $\rightarrow$ T0 Theory is strongly supported.
\end{itemize}

\subsection*{8. Physical Interpretation in T0 Context}

The maximum mass limit arises because:
\begin{enumerate}
\item Each mass creates a distortion in T0's time field: $\Delta T(r) \propto Gm/(c^2r)$
\item A superposition of two positions creates two incompatible time fields
\item T0's vacuum phase $\theta(x) \propto \int (1/T) dx$ cannot maintain coherence when $|\theta_1 - \theta_2|$ exceeds threshold
\item Threshold set by $B\rho_0 = 1/(\xi^6 \lambda_C^3)$ from T0's fundamental structure
\item Collapse is not measurement-induced but structural: T0's time field cannot support the superposition
\end{enumerate}

\subsection*{9. Experimental Tests}

\subsubsection*{9.1 MAST-QG}

Target: $10^9-10^{10}$ amu molecules.

\textbf{T0 Prediction:} Should observe collapse before reaching this mass scale.

\subsubsection*{9.2 MAQRO}

Target: Nanoparticles $\sim 10^8$ amu.

\textbf{T0 Prediction:} Should observe onset of collapse at this scale.

\subsubsection*{9.3 Nanodiamond Interferometry}

Current: $\sim 10^6$ amu.

\textbf{T0 Prediction:} Still below threshold—coherence maintained.

\subsubsection*{9.4 Levitated Optomechanics}

Approaching $10^7$ amu.

\textbf{T0 Prediction:} Should begin to see T0-induced decoherence effects.

\subsection*{10. Comparison with T³ Experiment}

Folman's T³ experiment (Chapter 21) and the maximum superposition mass (this chapter) are complementary:
\begin{itemize}
\item \textbf{T³:} Validates T0's phase accumulation mechanism via $\theta(x,t)$ dynamics
\item \textbf{Maximum mass:} Validates T0's phase stiffness $B\rho_0$ and collapse threshold
\item Both derived from single parameter $\xi = 4/3 \times 10^{-4}$
\end{itemize}

\subsection*{11. Conclusion}

T0-grounded DVFT gives a clear, first-principles upper bound on the size and mass of quantum superpositions:
\[
m_{\max} \sim 10^7-10^8 \text{ amu} \quad (R_{\max} \sim 100 \text{ nm})
\]

This limit arises from T0's time-field nonlinearity encoded in $\xi = 4/3 \times 10^{-4}$:
\begin{itemize}
\item Not a heuristic model but structural consequence of $T(x,t) \cdot m(x,t) = 1$
\item Predicts a fundamental cutoff testable in MAST-QG, MAQRO, and related experiments
\item If experiments exceed $10^8$ amu without collapse $\rightarrow$ T0 falsified
\item If collapse occurs at $10^7-10^8$ amu $\rightarrow$ T0 strongly validated
\end{itemize}

The maximum superposition mass is a unique, falsifiable prediction of T0 Theory.
