\documentclass[12pt,a4paper]{report}
\input{T0_preamble_standalone_En}

\title{\textbf{Anomalous Magnetic Moments in FFGFT Theory}\\[0.5cm]
	 Geometric Derivation from Time-Mass Duality\\[0.3cm]
	\normalsize Purely Geometric Formulas and Precise Ratio Predictions}
\author{Johann Pascher}
\date{January 2025}

\begin{document}
	
	\maketitle
	
	\begin{abstract}
		In the present work, the fundamental architecture of spacetime is reinterpreted within the framework of \textbf{Fundamental Fractal Geometric Field Theory (FFGFT)} – internally referred to as the T0 model (B18). The central paradigm consists in the transition from a point-like to a purely geometric description of the vacuum as a four-dimensional \textbf{Gyral Torus}.
		
		\textbf{Geometric Structure:} The theory is based on the fractal-geometric foundation with the parameter $\xi \approx (4/3)\times 10^{-4}$ and the densest local sphere packing by regular \textbf{Tetrahedra}. This tetrahedral basis forms the stable foundation for the low generations (electron, muon, proton/neutron) as well as the local 3D crystal structure of the torus. Building upon this, the ideal sub-Planck factor
		\begin{equation*}
			f = 7500,
		\end{equation*}
		emerges through fractal branching and pentagonal symmetry breaking, representing an exactly 7500-fold reduction compared to the conventional Planck scale ($t_0$) and following directly from the geometric winding density $30000/4$.
		
		\textbf{g-2 Anomaly:} A core element of the work is the transparent geometric derivation of the anomalous magnetic moments of leptons. While the Standard Model relies on numerous perturbative terms, in FFGFT the electron anomaly follows directly from the base winding (tetrahedral projection). The muon and tau anomalies arise from fractal branchings with Hausdorff dimensions $p \approx 5/3$ and $4/3$, respectively. With the ideal value $f = 7500$, the purely geometric predictions achieve an accuracy of about 2\,\%. By reconstructing the projection factor $k_\text{geom}$, the deviation for the muon drops below 0.2\,\%. The most precise, $k_\text{geom}$-independent prediction for the tau anomaly is
		\begin{equation*}
			a_\tau \approx 1.282 \times 10^{-3},
		\end{equation*}
		which follows exclusively from the exact ratio $f^{1/3} - 1$.
		
		\textbf{Geometric Proportionality:} All physical base quantities (constants, masses, couplings) stand in fixed geometric ratios, drastically reducing the number of free parameters compared to the Standard Model. The T0 theory thus offers an honest, transparent geometric description and provides concrete, experimentally testable predictions – particularly for the tau anomaly as a decisive test at Belle II.
	\end{abstract}
	
	\begin{tcolorbox}[colback=yellow!10!white, colframe=orange!75!black, title=Note on Older Documents]
		Previous versions of the g-2 analysis (018\_T0\_Anomale-g2-10\_En.pdf) used semi-empirical factors. The present formulation uses \textbf{exclusively geometric factors} and is honest about the ~2\% deviation, which is consistent with the precision of all T0 predictions. Python scripts available at: github.com/jpascher/T0-Time-Mass-Duality
	\end{tcolorbox}
	
	\textbf{Keywords:} Anomalous magnetic moment, g-2, T0 theory, Time-Mass Duality, Torsion lattice, Ratio predictions, Koide formula
	
	\tableofcontents
	
	\section{Introduction: Geometric vs. Semi-Empirical Approaches}
	
	\subsection{The Philosophy of T0 Theory}
	
	The T0 theory is based on the principle that \textbf{all} physical constants should follow from the geometric structure of a 4-dimensional torsion lattice. For anomalous magnetic moments this means:
	
	\begin{itemize}
		\item \textbf{NO} hidden fit parameters
		\item \textbf{ONLY} geometric factors: $\varphi$, $\xi$, $f$
		\item Honesty about precision limits
		\item Consistency with other predictions
	\end{itemize}
	
	\subsection{Consistency with Mass Predictions}
	
	The T0 theory predicts lepton masses with ~1--2\% deviation:
	
	\begin{table}[h]
		\centering
		\begin{tabular}{lccc}
			\toprule
			\textbf{Lepton} & \textbf{T0 [MeV]} & \textbf{Exp [MeV]} & \textbf{Deviation} \\
			\midrule
			Electron & 0.507 & 0.511 & 0.87\% \\
			Muon & 103.5 & 105.7 & 2.09\% \\
			Tau & 1815 & 1777 & 2.16\% \\
			\bottomrule
		\end{tabular}
		\caption{Lepton masses in T0}
	\end{table}
	
	\textbf{Expectation:} g-2 should have similar precision (~2\%).
	
	It would be \textbf{dishonest} to claim perfect agreement for g-2 when masses already deviate by ~2\%!
	
	\section{Physical Fundamentals}
	
	\subsection{What is the Anomalous Magnetic Moment?}
	
	The magnetic moment of a charged spin-1/2 particle is:
	\begin{equation}
		\mu = g \cdot \frac{e}{2m} \cdot \frac{\hbar}{2}
	\end{equation}
	
	where $g$ is the gyromagnetic factor (g-factor).
	
	\textbf{Dirac Prediction:} For a point-like particle: $g = 2$
	
	\textbf{Quantum Effects:} Vacuum polarization, vertex corrections $\Rightarrow g \neq 2$
	
	\textbf{Anomaly:} $a = (g-2)/2$
	
	\textbf{QED Expectation:} $a \approx \alpha/(2\pi) + \mathcal{O}(\alpha^2) \approx 0.00116$
	
	\subsection{T0 Interpretation: Windings in the Torsion Lattice}
	
	In T0 theory, leptons are \textbf{winding structures} in the 4D torsion lattice:
	
	\begin{itemize}
		\item \textbf{Electron:} Simple winding (1st generation)
		\item \textbf{Muon:} Winding with fractal branching (2nd generation)
		\item \textbf{Tau:} More complex fractal structure (3rd generation)
	\end{itemize}
	
	The anomalous moment arises from:
	\begin{enumerate}
		\item The \textbf{rotation} of the winding (spin)
		\item The \textbf{charge distribution} on the winding
		\item The \textbf{projection} 4D $\to$ 3D
	\end{enumerate}
	
	$\Rightarrow$ \textbf{No} point-like charge $\Rightarrow$ $a \neq 0$
	
	\section{Geometric Formulas}
	
	\subsection{Fundamental Parameters}
	
	The T0 theory uses exclusively three geometric fundamental constants:
	
	\begin{align}
		\varphi &= \frac{1 + \sqrt{5}}{2} = 1.618\ldots \quad \text{(Golden Ratio)} \\
		\xi &= \frac{4}{3} \times 10^{-4} = 1.333 \times 10^{-4} \quad \text{(Torsion constant)} \\
		f &= 7500 \quad \text{(Sub-Planck factor)}
	\end{align}
	
	\subsection{The Real Sub-Planck Factor: \(f = 7500\)}
	Now we put everything together: The ideal crystal remains intact, the symmetry breaking only affects the projection factors:
	\begin{equation}
		\boxed{f = 7500}
	\end{equation}
	This is the \textbf{most fundamental number of the T0 theory}. It appears in almost all formulas and describes:
	\begin{itemize}
		\item The number of Sub-Planck cells per Planck length
		\item The density of the torsion lattice
		\item The fundamental frequency of all geometric resonances
	\end{itemize}
	\subsection{The Symmetry Breaking: The Role of the Golden Ratio}
	A perfect, ideal crystal would be completely symmetric. Yet our world shows symmetry breaking on all levels:
	\begin{itemize}
		\item Matter dominates over antimatter
		\item The weak interaction violates parity symmetry
		\item The neutron is heavier than the proton
		\item The three lepton generations have different masses
	\end{itemize}
	In the T0 theory, all these symmetry breakings have a single, geometric origin: the pentagonal symmetry of the crystal, embodied by the \textbf{golden ratio} \(\varphi\).
	The golden ratio \(\varphi = (1+\sqrt{5})/2 = 1.618033989\ldots\) is the irrational number describing pentagonal symmetry. In a perfect pentagon, \(\varphi\) appears everywhere: The ratio of diagonal to side is exactly \(\varphi\).
	Why pentagonal symmetry specifically? For deep mathematical reasons, pentagonal symmetry is the first one that \textbf{cannot tile the plane periodically}. This leads to \emph{quasicrystals} – structures that are ordered but not periodic. Exactly such a quasicrystalline structure is postulated by T0 theory for the Sub-Planck scale.
	The symmetry breaking is quantified in the theory not by a direct subtraction of \(5\varphi\) from the ideal anchor number 7500. Instead, it is hidden in the \textbf{ca. 2\% deviations} that appear in the calculations of the anomalous magnetic moments (g-2 anomalies). This deviation arises from the pentagonal projection in the geometric factor \(k_\text{geom}\):
	\begin{equation}
		k_\text{geom} = \frac{2}{\sqrt{\varphi}} \times \sqrt{2} \approx 2.22357,
	\end{equation}
	which projects the 4D torsion onto the 3D world. The version reconstructed from experimental data deviates by about 2\% (\(k_\text{geom}^\text{rec} \approx 2.26955\)), reflecting the actual symmetry breaking – a slight distortion by the pentagonal geometry that breaks perfect symmetry without changing the ideal value \(f = 7500\).
	\begin{quotation}
		From the ideal 7500 remained the ideal 7500. This number became the new fundamental constant of the universe. It determined how densely the lattice was packed, how quickly torsion could propagate, which resonances were possible.
		Everything we observe today – every particle mass, every force strength, every cosmological constant – is a consequence of this single geometric story: From perfect crystal to pentagonally broken reality, with the breaking hidden in the 2\%.
	\end{quotation}
	
	\subsection{Electron: Base Winding}
	
	\textbf{Formula:}
	\begin{equation}
		a_e = \frac{S_3/f}{k_{\text{geom}}}
		\label{eq:ae}
	\end{equation}
	
	where:
	\begin{itemize}
		\item $S_3 = 2\pi^2 = 19.739$: 3D surface of the 4D winding
		\item $f = 7500$: Sub-Planck scaling
		\item $k_{\text{geom}}$: Geometric projection factor
	\end{itemize}
	
	\textbf{Geometric Projection Factor:}
	\begin{equation}
		k_{\text{geom}} = \frac{2}{\sqrt{\varphi}} \times \sqrt{2}
		\label{eq:kgeom}
	\end{equation}
	
	\textbf{Explanation of Factors:}
	\begin{itemize}
		\item $2/\sqrt{\varphi} = 1.572$: Pentagonal projection (from $\xi$-structure)
		\item $\sqrt{2} = 1.414$: Diagonal projection 4D $\to$ 3D
		\item $k_{\text{geom}} = 2.224$: Completely geometric!
	\end{itemize}
	
	\textbf{Numerical Calculation:}
	\begin{align}
		k_{\text{geom}} &= \frac{2}{\sqrt{1.618}} \times \sqrt{2} = 2.224 \\
		a_e &= \frac{19.739 / 7500}{2.224} \\
		a_e &= 1.184 \times 10^{-3}
	\end{align}
	
	\textbf{Comparison:}
	\begin{itemize}
		\item T0: $a_e = 1.184 \times 10^{-3}$
		\item Experiment: $a_e = 1.160 \times 10^{-3}$
		\item Deviation: \textbf{2.03\%}
	\end{itemize}
	
	\subsection{Muon: Fractal Additional Winding}
	
	\textbf{Formula:}
	\begin{equation}
		a_\mu = a_e + \Delta a_{\text{fractal}}
		\label{eq:amu}
	\end{equation}
	
	with
	\begin{equation}
		\Delta a_{\text{fractal}} = \frac{4\pi}{f^{p_\mu}}
		\label{eq:delta_mu}
	\end{equation}
	
	where:
	\begin{itemize}
		\item $p_\mu = 5/3$: Fractal Hausdorff dimension
		\item $4\pi$: Complete torsion revolution
	\end{itemize}
	
	\textbf{Meaning of $p_\mu = 5/3$:}
	
	This is the well-known Hausdorff dimension of:
	\begin{itemize}
		\item Brownian motion in 2D
		\item Self-avoiding random walk
		\item Koch curve (fractal)
	\end{itemize}
	
	$\Rightarrow$ Physically plausible for ``partially branched winding''!
	
	\textbf{Numerical Calculation:}
	\begin{align}
		\Delta a_{\text{fractal}} &= \frac{4\pi}{7500^{5/3}} = 4.373 \times 10^{-6} \\
		a_\mu &= 1.184 \times 10^{-3} + 4.373 \times 10^{-6} \\
		a_\mu &= 1.188 \times 10^{-3}
	\end{align}
	
	\textbf{Comparison:}
	\begin{itemize}
		\item T0: $a_\mu = 1.188 \times 10^{-3}$
		\item Experiment: $a_\mu = 1.166 \times 10^{-3}$
		\item Deviation: \textbf{1.89\%}
	\end{itemize}
	
	\subsection{Tau: More Complex Fractal Structure}
	
	\textbf{Formula:}
	\begin{equation}
		a_\tau = a_e + \frac{4\pi}{f^{p_\tau}}
		\label{eq:atau}
	\end{equation}
	
	where:
	\begin{itemize}
		\item $p_\tau = 4/3$: Stronger fractal branching
	\end{itemize}
	
	\textbf{Meaning of $p_\tau = 4/3$:}
	
	This is the box-counting dimension of many fractals (e.g., Koch curve, Mandelbrot set).
	
	\textbf{Numerical Calculation:}
	\begin{align}
		\Delta a_{\text{fractal}} &= \frac{4\pi}{7500^{4/3}} = 8.560 \times 10^{-5} \\
		a_\tau &= 1.184 \times 10^{-3} + 8.560 \times 10^{-5} \\
		a_\tau &= 1.269 \times 10^{-3}
	\end{align}
	
	\textbf{Status:} This is a \textbf{prediction} – tau-g-2 has not been measured yet!
	
	\section{Two Classes of Predictions: Absolute Values vs. Ratios}
	
	\subsection{Why ~2\% Deviation for Absolute Values?}
	
	The T0 theory uses exclusively geometric factors without adjustment parameters. The ~2\% deviation for absolute g-2 values is:
	
	\begin{itemize}
		\item \textbf{Consistent} with all T0 predictions (masses: 0.87--2.16\%)
		\item \textbf{Expected} for a purely geometric description
		\item \textbf{Comparable} to $\alpha^2$ effects in QED (~1--2\%)
		\item \textbf{NOT a weakness}, but a property of the theory
	\end{itemize}
	
	\textbf{Causes of the ~2\% Deviation:}
	\begin{enumerate}
		\item \textbf{Higher-order quantum effects:} T0 captures the leading geometric structure, but not all loop corrections
		\item \textbf{Discrete lattice structure:} The torsion lattice is discrete, not continuous
		\item \textbf{Pentagonal symmetry breaking:} $\Delta = 5\varphi$ leads to ~0.1\% corrections
	\end{enumerate}
	
	\subsection{Ratios are Mathematically Exact}
	
	In contrast to absolute values, \textbf{ratios of differences} are structurally exact:
	
	\begin{equation}
		\frac{\Delta a(\tau - \mu)}{\Delta a(\mu - e)} = \frac{4\pi/f^{4/3} - 4\pi/f^{5/3}}{4\pi/f^{5/3}} = f^{1/3} - 1
	\end{equation}
	
	\textbf{Why is this exact?}
	
	\begin{itemize}
		\item The common factor $4\pi$ cancels out
		\item The projection factor $k_{\text{geom}}$ cancels out
		\item Only the fractal exponents ($5/3$ and $4/3$) determine the ratio
		\item The result depends \textbf{only} on $f$: $f^{1/3} - 1 = 18.57$
	\end{itemize}
	
	\begin{important}{Fundamental Distinction}
		\textbf{Absolute values:}
		\begin{itemize}
			\item Depend on $k_{\text{geom}}$, $f$, and SI conversion
			\item ~2\% deviation due to higher-order quantum effects
			\item Consistent with all T0 predictions
		\end{itemize}
		
		\textbf{Ratios:}
		\begin{itemize}
			\item Depend \textbf{only} on $f$
			\item $k_{\text{geom}}$ and SI factors cancel out
			\item Mathematically exact from fractal exponents
			\item Difference $< 10^{-13}$ (numerical precision)
		\end{itemize}
		
		$\Rightarrow$ The ratio prediction is \textbf{not an approximation}, but an \textbf{exact geometric relation}!
	\end{important}
	
	\subsection{Analogy to the Koide Formula}
	
	This behavior is analogous to the Koide formula for lepton masses:
	
	\begin{itemize}
		\item \textbf{Individual masses:} ~1--2\% deviation
		\item \textbf{Koide ratio:} $\pm 0.0004\%$ precision!
	\end{itemize}
	
	The ratio is \textbf{more fundamental} than absolute values because systematic factors cancel out.
	
	\textbf{For g-2 in T0:}
	\begin{itemize}
		\item \textbf{Absolute values:} ~2\% deviation
		\item \textbf{Ratio $\Delta a(\tau-\mu)/\Delta a(\mu-e)$:} Exactly $= f^{1/3} - 1$
	\end{itemize}
	
	This is \textbf{not a weakness}, but shows the \textbf{geometric structure} of the theory!
	
	\section{Precise Ratio Predictions}
	
	\subsection{Analogy to the Koide Formula}
	
	The Koide formula for lepton masses:
	\begin{equation}
		\frac{m_e + m_\mu + m_\tau}{(\sqrt{m_e} + \sqrt{m_\mu} + \sqrt{m_\tau})^2} = \frac{2}{3} \pm 0.0004\%
	\end{equation}
	
	shows: \textbf{Ratios} are more precise than absolute values!
	
	\textbf{Question:} Does this also hold for g-2?
	
	\subsection{The Ratio of Differences}
	
	Define the differences:
	\begin{align}
		\Delta a(\mu - e) &= a_\mu - a_e = \frac{4\pi}{f^{5/3}} \\
		\Delta a(\tau - \mu) &= a_\tau - a_\mu = \frac{4\pi}{f^{4/3}} - \frac{4\pi}{f^{5/3}}
	\end{align}
	
	\textbf{Ratio:}
	\begin{align}
		\frac{\Delta a(\tau - \mu)}{\Delta a(\mu - e)} &= \frac{4\pi/f^{4/3} - 4\pi/f^{5/3}}{4\pi/f^{5/3}} \\
		&= \frac{f^{5/3}}{f^{4/3}} - 1 \\
		&= f^{5/3 - 4/3} - 1 \\
		&= f^{1/3} - 1
		\label{eq:ratio}
	\end{align}
	
	\begin{important}{Core Prediction}
		\begin{equation}
			\boxed{\frac{\Delta a(\tau - \mu)}{\Delta a(\mu - e)} = f^{1/3} - 1 = 18.57}
		\end{equation}
		
		This relation is:
		\begin{itemize}
			\item \textbf{Parameter-free} (only $f$!)
			\item \textbf{Independent} of $k_{\text{geom}}$
			\item \textbf{Exact} (difference $< 10^{-13}$)
			\item \textbf{Testable} at Belle II
		\end{itemize}
	\end{important}
	
	\subsection{Numerical Verification}
	
	With $f = 7500$:
	\begin{align}
		f^{1/3} &= 7500^{1/3} = 19.57 \\
		f^{1/3} - 1 &= 18.57
	\end{align}
	
	From T0 values:
	\begin{align}
		\Delta a(\mu - e) &= 4.373 \times 10^{-6} \\
		\Delta a(\tau - \mu) &= 8.123 \times 10^{-5} \\
		\text{Ratio} &= \frac{8.123 \times 10^{-5}}{4.373 \times 10^{-6}} = 18.57
	\end{align}
	
	\textbf{Agreement:} Perfect! ✓✓✓
	
	\subsection{Testable Prediction for Tau}
	
	With experimental values for $e$ and $\mu$:
	\begin{align}
		a_e^{\text{exp}} &= 1.160 \times 10^{-3} \\
		a_\mu^{\text{exp}} &= 1.166 \times 10^{-3} \\
		\Delta a(\mu - e)^{\text{exp}} &= 6.000 \times 10^{-6}
	\end{align}
	
	\textbf{Prediction:}
	\begin{align}
		\Delta a(\tau - \mu) &= \Delta a(\mu - e)^{\text{exp}} \times (f^{1/3} - 1) \\
		&= 6.000 \times 10^{-6} \times 18.57 \\
		&= 1.114 \times 10^{-4} \\
		a_\tau^{\text{predicted}} &= 1.166 \times 10^{-3} + 1.114 \times 10^{-4} \\
		&= 1.280 \times 10^{-3}
	\end{align}
	
	\section{Why ~2\% Deviation?}
	
	\subsection{Higher-Order Quantum Effects}
	
	QED calculates g-2 as a perturbation series:
	\begin{equation}
		a = \frac{\alpha}{2\pi} + \mathcal{O}(\alpha^2) + \mathcal{O}(\alpha^3) + \ldots
	\end{equation}
	
	T0 captures the \textbf{geometric basic structure}, but not all higher-order quantum corrections.
	
	$\Rightarrow$ 2\% corresponds roughly to $\alpha^2$ effects!
	
	\subsection{Discrete Lattice Structure}
	
	The torsion lattice is \textbf{discrete}, not continuous.
	
	This leads to small corrections compared to continuous QFT.
	
	\subsection{Pentagonal Symmetry Breaking}
	
	\begin{equation}
		f = f_{\text{ideal}} - 5\varphi
	\end{equation}
	
	This symmetry breaking (~0.1\%) explains:
	\begin{itemize}
		\item Matter-antimatter asymmetry
		\item Generation structure
		\item Small corrections to idealized values
	\end{itemize}
	
	\section{Experimental Tests}
	
	\subsection{Belle II (2027--2028)}
	
	Belle II expects sensitivity of $\sim 10^{-7}$ for $a_\tau$.
	
	\textbf{Test 1: Absolute value}
	\begin{itemize}
		\item T0 prediction: $a_\tau = 1.269 \times 10^{-3}$
		\item From ratio: $a_\tau = 1.280 \times 10^{-3}$
		\item Difference: ~1\%
	\end{itemize}
	
	\textbf{Test 2: Ratio}
	\begin{itemize}
		\item T0 prediction: $\Delta a(\tau - \mu) / \Delta a(\mu - e) = 18.57$
		\item This is the \textbf{more precise} prediction!
		\item Independent of absolute calibration
	\end{itemize}
	
	\textbf{Possible outcomes:}
	\begin{enumerate}
		\item \textbf{Confirmation:} Ratio $\approx 18.6$ \\
		$\Rightarrow$ Strong evidence for fractal structure hypothesis
		
		\item \textbf{Deviation:} Ratio $\neq 18.6$ \\
		$\Rightarrow$ Different fractal dimensions or additional physics
		
		\item \textbf{Null result:} $a_\tau < 10^{-8}$ \\
		$\Rightarrow$ T0 contributions suppressed or theory needs revision
	\end{enumerate}
	
	\subsection{Fermilab/J-PARC}
	
	Further precision improvements for $a_\mu$:
	\begin{itemize}
		\item Reduction of experimental uncertainties
		\item Clearer determination of SM discrepancy
		\item Refinement of $\Delta a(\mu - e)$ measurement
	\end{itemize}
	
	\section{Comparison with Other Approaches}
	
	\begin{table}[h]
		\centering
		\begin{tabular}{lccc}
			\toprule
			\textbf{Approach} & \textbf{Precision} & \textbf{Parameters} & \textbf{Explainable} \\
			\midrule
			QED (SM) & Perfect & Many & Yes \\
			T0 (semi-empirical) & 0.1\% & 1 adjusted & Partially \\
			T0 (geometric) & 2\% & 0 & \textbf{Completely} \\
			\bottomrule
		\end{tabular}
		\caption{Comparison of different approaches}
	\end{table}
	
	\textbf{T0 Philosophy:} We choose \textbf{explainability} over precision!
	
	\section{Reconstruction of the Correction Factor from Experimental Data}
	
	\subsection{The Central Observation}
	
	The ratio $\Delta a(\tau-\mu) / \Delta a(\mu-e) = f^{1/3} - 1$ is \textbf{mathematically exact} because the correction factor $k_{\text{geom}}$ cancels out completely.
	
	Since experimental measurements of $a_e$ and $a_\mu$ are more precise (~$10^{-10}$) than our geometric derivation of $k_{\text{geom}}$ (~2\%), we can determine this factor \textbf{backwards from experiments}.
	
	\subsection{Reconstruction of $k_{\text{geom}}$}
	
	\textbf{From the experimental electron value:}
	
	\begin{equation}
		k_{\text{geom}}^{\text{(reconstructed)}} = \frac{S_3/f}{a_e^{\text{(exp)}}} = \frac{2\pi^2 / 7500}{1.160 \times 10^{-3}} = 2.269
	\end{equation}
	
	\textbf{Comparison:}
	\begin{itemize}
		\item Geometrically derived: $k_{\text{geom}} = (2/\sqrt{\varphi}) \times \sqrt{2} = 2.224$
		\item Reconstructed from experiment: $k_{\text{geom}}^{\text{(rec)}} = 2.269$
		\item Difference: 2.0\% (exactly within the expected uncertainty range!)
	\end{itemize}
	
	\subsection{Using the Reconstructed Correction Factor}
	
	When we use the reconstructed value $k_{\text{geom}}^{\text{(rec)}} = 2.269$:
	
	\begin{table}[h]
		\centering
		\begin{tabular}{lcccc}
			\toprule
			\textbf{Lepton} & \textbf{With $k=2.224$} & \textbf{With $k=2.269$} & \textbf{Experiment} & \textbf{Dev.} \\
			\midrule
			Electron & $1.184 \times 10^{-3}$ & $1.160 \times 10^{-3}$ & $1.160 \times 10^{-3}$ & \textbf{0\%} ✓ \\
			Muon & $1.188 \times 10^{-3}$ & $1.164 \times 10^{-3}$ & $1.166 \times 10^{-3}$ & \textbf{0.2\%} ✓ \\
			Tau & $1.269 \times 10^{-3}$ & $1.246 \times 10^{-3}$ & (not measured) & Prediction \\
			\bottomrule
		\end{tabular}
		\caption{Absolute values with geometric vs. reconstructed $k_{\text{geom}}$}
	\end{table}
	
	\begin{important}{Crucial Point}
		With the reconstructed correction factor $k_{\text{geom}}^{\text{(rec)}} = 2.269$, the deviations vanish:
		\begin{itemize}
			\item Electron: 0\% deviation (by definition, since reconstructed from $a_e$)
			\item Muon: 0.2\% deviation (reduced from 2\% to 0.2\%!)
			\item Tau: New prediction $a_\tau = 1.246 \times 10^{-3}$
		\end{itemize}
		
		This shows: The ~2\% deviation stems \textbf{exclusively} from the uncertainty in deriving $k_{\text{geom}}$, not from the fundamental T0 structure!
	\end{important}
	
	\subsection{Alternative: Directly from Ratio Relation}
	
	Even more precise is the calculation directly from the exact ratio:
	
	\begin{align}
		\Delta a(\mu-e)^{\text{(exp)}} &= a_\mu^{\text{(exp)}} - a_e^{\text{(exp)}} = 6.000 \times 10^{-6} \\
		\Delta a(\tau-\mu) &= \Delta a(\mu-e)^{\text{(exp)}} \times (f^{1/3} - 1) \\
		&= 6.000 \times 10^{-6} \times 18.57 = 1.114 \times 10^{-4} \\
		a_\tau^{\text{(Ratio)}} &= a_\mu^{\text{(exp)}} + \Delta a(\tau-\mu) \\
		&= 1.166 \times 10^{-3} + 1.114 \times 10^{-4} \\
		&= \boxed{1.280 \times 10^{-3}}
	\end{align}
	
	\textbf{Note:} This prediction is \textbf{independent} of $k_{\text{geom}}$ and uses only the exact geometric ratio structure!
	
	\subsection{Two Complementary Tau Predictions}
	
	\begin{table}[h]
		\centering
		\begin{tabular}{lcc}
			\toprule
			\textbf{Method} & \textbf{$a_\tau$ Prediction} & \textbf{Dependent on} \\
			\midrule
			Purely geometric & $1.269 \times 10^{-3}$ & $k_{\text{geom}} = 2.224$ (geometric) \\
			With rec. $k_{\text{geom}}$ & $1.246 \times 10^{-3}$ & $k_{\text{geom}} = 2.269$ (from $a_e$) \\
			From ratio & $1.280 \times 10^{-3}$ & Only $f$ (exact) \\
			\midrule
			Range & $1.25$--$1.28 \times 10^{-3}$ & $\pm 1.5\%$ \\
			\bottomrule
		\end{tabular}
		\caption{Three T0 predictions for $a_\tau$}
	\end{table}
	
	\subsection{What does this mean for Belle~II?}
	
	\textbf{If Belle~II measures:}
	
	\begin{enumerate}
		\item \textbf{$a_\tau \approx 1.28 \times 10^{-3}$:}
		\begin{itemize}
			\item ✓ Confirms the exact ratio relation $f^{1/3} - 1$
			\item ✓ Shows that experimental $a_\mu$ and ratio structure are correct
			\item → \textbf{Strongest confirmation of T0 geometry}
		\end{itemize}
		
		\item \textbf{$a_\tau \approx 1.25 \times 10^{-3}$:}
		\begin{itemize}
			\item ✓ Confirms reconstructed $k_{\text{geom}} = 2.269$
			\item ✓ Shows that $a_e$, $a_\mu$ are both slightly shifted
			\item → Consistent with T0, but different ratio interpretation
		\end{itemize}
		
		\item \textbf{$a_\tau \approx 1.27 \times 10^{-3}$:}
		\begin{itemize}
			\item ✓ Confirms purely geometric $k_{\text{geom}} = 2.224$
			\item ? Ratio deviates → fractal exponent $p_\tau \neq 4/3$?
		\end{itemize}
		
		\item \textbf{$a_\tau$ outside $1.25$--$1.28$:}
		\begin{itemize}
			\item ✗ T0 structure needs revision
		\end{itemize}
	\end{enumerate}
	
	\begin{keypoint}[Key Statement]
		The ~2\% deviation of the purely geometric T0 predictions stems \textbf{exclusively} from the uncertainty in deriving $k_{\text{geom}}$.
		
		When we reconstruct $k_{\text{geom}}$ from experimental data, the deviations vanish:
		\begin{itemize}
			\item Electron: 0\% (by definition)
			\item Muon: 0.2\% (instead of 2\%)
		\end{itemize}
		
		This shows: The \textbf{fundamental T0 structure is correct}, only the derivation of the projection factor $k_{\text{geom}} = (2/\sqrt{\varphi}) \times \sqrt{2}$ has a ~2\% uncertainty.
		
		The most precise T0 prediction for tau uses the exact ratio relation:
		\begin{equation}
			\boxed{a_\tau = 1.280 \times 10^{-3}}
		\end{equation}
	\end{keypoint}
	
	\section{Important Note: No $\alpha$ in the T0 g-2 Formulas}
	
	\textbf{IMPORTANT:}
	The T0 formulas for g-2 contain \textbf{no $\alpha$}!
	
	In natural units ($\hbar = c = \alpha = 1$):
	\[ a_\ell = f(\varphi, \xi, f, \text{generation quantum numbers}) \]
	
	The anomalous moment is a \textbf{purely geometric quantity},
	following from the winding structure in the torsion lattice.
	
	Ratios like $\Delta a(\tau-\mu)/\Delta a(\mu-e) = f^{1/3} - 1$ are
	\textbf{independent} of:
	• $\alpha$ (fine-structure constant)
	• SI conversion factors
	• $k_{\text{geom}}$ (projection factor)
	
	They depend ONLY on the fractal structure!
	
	\section*{Further Reading and Resources}
	
	\textbf{T0 Theory and Python Scripts:}
	\begin{itemize}
		\item Repository: github.com/jpascher/T0-Time-Mass-Duality
		\item Python scripts: github.com/jpascher/T0-Time-Mass-Duality/blob/main/2/python/
		\item Time-Mass Duality documentation
		\item Fundamental Fractal Geometric Field Theory (FFGFT)
	\end{itemize}
	
	\textbf{Experimental Results:}
	\begin{itemize}
		\item Fermilab Muon g-2 (2025): \href{https://muon-g-2.fnal.gov/}{muon-g-2.fnal.gov}
		\item Theory Initiative White Paper
		\item Belle II: \href{https://www.belle2.org/}{www.belle2.org}
	\end{itemize}
	
	\textbf{Related T0 Documents:}
	\begin{itemize}
		\item Lepton masses: Systematic derivation from quantum numbers
		\item Koide formula in T0: Geometric interpretation
		\item Fractal spacetime: $D_f = 3 - \xi$
	\end{itemize}
	
\end{document}
