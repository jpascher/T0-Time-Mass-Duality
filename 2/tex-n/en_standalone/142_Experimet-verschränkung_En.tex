\documentclass[12pt,a4paper]{article}

% Now include the shared preamble (it contains all \usepackage commands)
\input{../../../T0_preamble_shared-ebook_En}
\title{Attosecond Prediction of Quantum Entanglement Formation \\
	as Supporting Evidence for the T$_0$-Time-Mass-Duality Theory}
\author{Johann Pascher}
\date{January 15, 2026}

\begin{document}
	
	\maketitle
	
	\begin{abstract}
		This document summarizes the theoretical prediction of the time-resolved formation of quantum entanglement (Jiang et al., 2024) and presents it as supporting evidence for the fundamental time dynamics postulated in the T$_0$-Time-Mass-Duality Theory. All theoretical interpretations are drawn exclusively from the content of the Master Narrative (FFGFT\_Narrative\_Master\_En.pdf) and related documents in the repository:
		\url{https://github.com/jpascher/T0-Time-Mass-Duality/tree/main/2/}.
	\end{abstract}
	
	\section{The Theoretical Work}
	The study by Jiang et al.\ (2024) demonstrates theoretically that quantum entanglement does \textbf{not} form instantaneously, but emerges over a measurable local time window.
	
	\subsection{Key Details from the Simulation}
	\begin{itemize}[leftmargin=*]
		\item \textbf{System}: Helium atom driven by an intense high-frequency EUV laser pulse (photoionization).
		\item \textbf{Process}: One electron absorbs energy and escapes (ionizes), while the second electron is excited to a higher energy state.
		\item \textbf{Superposition}: The escaping electron is in a superposition of different departure times (no single sharp instant).
		\item \textbf{Correlation}: The final energy of the bound electron is directly correlated with the departure time of the escaping electron:
		\begin{itemize}
			\item Higher energy in the bound electron $\to$ escaping electron departed earlier
			\item Lower energy $\to$ escaping electron departed later
		\end{itemize}
		\item \textbf{Predicted Time Window}: Full time-dependent Schrödinger equation simulations yield an entanglement formation window of $\sim$\textbf{232 attoseconds} ($\approx 2.32 \times 10^{-16}$\,s).
		\item \textbf{Proposed Experimental Verification}: A double-pulse scheme (generation pulse + probe pulse) combined with coincidence detection of both electrons to reconstruct the shared quantum history and time the formation process.
	\end{itemize}
	
	\textbf{Important note}: This is a theoretical/numerical prediction. No laboratory measurement has been performed yet. The authors propose a feasible future experiment using current attosecond laser technology.
	
	\subsection{Popular-Science Video}
	Video summary of the work:  
	\url{https://www.youtube.com/watch?v=t3wjY95zvNM}  
	(„Scientists Measure Quantum Entanglement Speed — And It Breaks Physics“, Channel: NASA Space News, Uploaded: January 14, 2026)
	
	\section{Connection to the T$_0$-Time-Mass-Duality Theory}
	This theoretical result provides strong conceptual support for the core postulate of the theory:
	
	\begin{quote}
		„In the T$_0$-Time-Mass-Duality Theory, time is ontologically equivalent to mass and therefore not merely a coordinate, but an active physical quantity with real dynamics on all scales. Quantum correlations (entanglement) therefore do not arise instantaneously, but develop as a temporal, emergent process within a local interaction window. The predicted attosecond formation time of $\sim 232$\,as confirms exactly this finite, dynamical build-up without non-local 'spooky action at a distance' and without violating causality.“
	\end{quote}
	
	This highlights that all quantum phenomena carry intrinsic time dynamics — a direct consequence of the fundamental duality between time and mass.
	
	\section{References}
	\begin{enumerate}[leftmargin=*]
		\item Jiang, W.-C., Zhong, M.-C., Fang, Y.-K., Donsa, S., Březinová, I., Peng, L.-Y., Burgdörfer, J. (2024). \\
		\emph{Time Delays as Attosecond Probe of Interelectronic Coherence and Entanglement}. \\
		\textbf{Physical Review Letters 133, 163201}. \\
		DOI: \href{https://doi.org/10.1103/PhysRevLett.133.163201}{10.1103/PhysRevLett.133.163201}
		\item Video: „Scientists Measure Quantum Entanglement Speed — And It Breaks Physics“. \\
		YouTube, Channel: NASA Space News. \\
		\url{https://www.youtube.com/watch?v=t3wjY95zvNM} (accessed January 15, 2026)
	\end{enumerate}
	
\end{document}