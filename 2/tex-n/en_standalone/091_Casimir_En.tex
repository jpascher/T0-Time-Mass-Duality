\documentclass[12pt,a4paper]{report}
\input{../../../T0_preamble_shared-ebook_En}
\author{}
\date{}

\begin{document}
\hfuzz=200pt
\allowdisplaybreaks

\title{Unification of the Casimir Effect and Cosmic Microwave Background: A Fundamental Vacuum Theory}
\maketitle

\tableofcontents

	\section{Introduction}
	
	This paper develops a novel theoretical description that interprets the microscopic Casimir effect and the macroscopic cosmic microwave background (CMB) as different manifestations of an underlying vacuum structure. By introducing a characteristic vacuum length scale \( L_\xi \) and a fundamental dimensionless coupling constant \( \xi \), it is shown that both phenomena can be described within a unified theoretical framework.
	
	The theory is based on the hypothesis of a granular spacetime with a minimal length scale \( L_0 = \xi \cdot L_P \), at which all physical forces are fully effective. For distances \( d > L_0 \), only parts of these forces become visible through vacuum fluctuations, which is described by the \( 1/d^4 \) dependence of the Casimir force. Due to the extremely small size of \( L_0 \), a direct experimental measurement is currently not possible, which is why the measurable scale \( L_\xi \) serves as a bridge between the fundamental spacetime structure and experimental observations. Gravity is interpreted as an emergent property of a time field, thereby allowing cosmic effects such as the CMB to be explained without the assumption of dark energy or dark matter.
	
	\section{Theoretical Foundations}
	
	\subsection{Fundamental Length Scales}
	
	The proposed framework defines a hierarchy of characteristic length scales:
	
	\begin{align}
		L_0 &= \xi \cdot L_P \label{eq:L0_definition}\\
		L_P &= \sqrt{\frac{\hbar G}{c^3}} \approx \SI{1.616e-35}{\meter} \label{eq:planck_length}\\
		L_\xi &= \text{characteristic vacuum length scale} \approx \SI{100}{\micro\meter} \label{eq:Lxi_definition}
	\end{align}
	
	Here, \( L_0 \) represents the minimal length scale of a granular spacetime at which all vacuum fluctuations are fully effective, while \( L_\xi \) represents the emergent scale for measurable vacuum interactions.
	
	\subsection{The Coupling Constant \( \xi \)}
	
	The dimensionless coupling constant \( \xi \) is determined to be
	
	\begin{equation}
		\xi = \frac{4}{3} \times 10^{-4} = \num{1.333e-4} \label{eq:coupling_constant}
	\end{equation}
	
	This constant serves as a fundamental space parameter that links the granulation of spacetime at \( L_0 \) with measurable effects such as the Casimir effect and the CMB. It can be derived from a Lagrangian that describes the dynamics of a time field.
	
	\section{The CMB-Vacuum Relationship}
	
	\subsection{Basic Equation}
	
	The central relationship of the theory links the energy density of the cosmic microwave background with the characteristic vacuum length scale:
	
	\begin{equation}
		\rho_{\text{CMB}} = \frac{\xi \hbar c}{L_\xi^4} \label{eq:cmb_vacuum_relation}
	\end{equation}
	
	This formula is dimensionally consistent, since
	
	\begin{equation}
		[\rho_{\text{CMB}}] = \frac{[1] \cdot [\hbar c]}{[L_\xi^4]} = \frac{\si{\joule\meter}}{\si{\meter^4}} = \si{\joule\per\meter^3}
	\end{equation}
	
	\subsection{Numerical Determination of \( L_\xi \)}
	
	With the experimentally determined CMB energy density \( \rho_{\text{CMB}} = \SI{4.17e-14}{\joule\per\meter^3} \), \( L_\xi \) can be calculated:
	
	\begin{align}
		L_\xi^4 &= \frac{\xi \hbar c}{\rho_{\text{CMB}}} \label{eq:Lxi_calculation}\\
		L_\xi^4 &= \frac{\num{1.333e-4} \times \SI{3.162e-26}{\joule\meter}}{\SI{4.17e-14}{\joule\per\meter^3}}\\
		L_\xi^4 &= \SI{1.011e-16}{\meter^4}\\
		L_\xi &= \SI{100}{\micro\meter} \label{eq:Lxi_result}
	\end{align}
	
	\section{Modified Casimir Theory}
	
	\subsection{Extended Casimir Formula}
	
	The Casimir effect is described by the following modified formula:
	
	\begin{equation}
		|\rho_{\text{Casimir}}(d)| = \frac{\pi^2}{240\xi} \rho_{\text{CMB}} \left( \frac{L_\xi}{d} \right)^4 \label{eq:modified_casimir}
	\end{equation}
	
	where \( d \) denotes the distance between the Casimir plates.
	
	\subsection{Consistency with the Standard Casimir Formula}
	
	By substituting the CMB-vacuum relationship \eqref{eq:cmb_vacuum_relation} into the modified Casimir formula \eqref{eq:modified_casimir}, the following is obtained:
	
	\begin{align}
		|\rho_{\text{Casimir}}(d)| &= \frac{\pi^2}{240\xi} \cdot \frac{\xi \hbar c}{L_\xi^4} \cdot \frac{L_\xi^4}{d^4} \label{eq:casimir_substitution}\\
		&= \frac{\pi^2 \hbar c}{240 d^4} \label{eq:standard_casimir_recovered}
	\end{align}
	
	This exactly matches the established standard Casimir formula and proves the mathematical consistency of the proposed theory.
	
	\section{Numerical Verification}
	
	\subsection{Comparison Calculations}
	
	To verify the theoretical consistency, Casimir energy densities are calculated for various plate distances:
	
	\begin{table}[H]
		\centering
		\resizebox{\textwidth}{!}{
\begin{tabular}{c S[table-format=1.3e1] S[table-format=1.2e-2] S[table-format=1.2e-2]}
			\toprule
			Distance \( d \) & {\((L_\xi/d)^4\)} & {\(\rho_{\text{Casimir}}\) (\unit{\joule\per\meter\cubed})} & {\(\rho_{\text{Casimir}}\) (\unit{\joule\per\meter\cubed})} \\
			\midrule
			\SI{1}{\micro\meter} & 1.000e8 & 1.30e-3 & 1.30e-3 \\
			\SI{100}{\nano\meter} & 1.000e12 & 1.30e1 & 1.30e1 \\
			\SI{10}{\nano\meter} & 1.000e16 & 1.30e5 & 1.30e5 \\
			\bottomrule
		\end{tabular}
}
		\caption{Comparison of Casimir energy densities between the standard formula and the new theoretical description}
		\label{tab:casimir_comparison}
	\end{table}
	
	The perfect agreement confirms the mathematical correctness of the developed theory.
	
	\subsection{Hierarchy of Characteristic Length Scales}
	
	The theory establishes a clear hierarchy of length scales:
	
	\begin{align}
		L_0 &= \SI{2.155e-39}{\meter} \quad \text{(Sub-Planck)} \label{eq:L0_value}\\
		L_P &= \SI{1.616e-35}{\meter} \quad \text{(Planck)} \label{eq:LP_value}\\
		L_\xi &= \SI{100}{\micro\meter} \quad \text{(Casimir-characteristic)} \label{eq:Lxi_value}
	\end{align}
	
	The ratios of these length scales are:
	
	\begin{align}
		\frac{L_0}{L_P} &= \xi = \num{1.333e-4} \label{eq:L0_LP_ratio}\\
		\frac{L_P}{L_\xi} &= \num{1.616e-31} \label{eq:LP_Lxi_ratio}\\
		\frac{L_0}{L_\xi} &= \num{2.155e-35} \label{eq:L0_Lxi_ratio}
	\end{align}
	
	\section{Physical Interpretation}
	
	\subsection{Multi-Scale Vacuum Model}
	
	The developed theory implies a fundamental structure of the vacuum on various length scales:
	
	\begin{enumerate}
		\item \textbf{Sub-Planck Level} (\( L_0 \)): Minimal length scale of the granular spacetime, at which all physical forces, including vacuum fluctuations, are fully effective. Due to the extremely small size of \( L_0 \approx \SI{2.155e-39}{\meter} \), a direct measurement is currently not possible.
		\item \textbf{Planck Threshold} (\( L_P \)): Transition region between quantum gravity and classical spacetime geometry.
		\item \textbf{Casimir Manifestation} (\( L_\xi \)): Emergent length scale for measurable vacuum interactions that forms a bridge to the CMB.
		\item \textbf{Cosmic Scale}: Large-scale vacuum signature through the CMB, explained by a time field from which gravity emerges.
	\end{enumerate}
	
	\subsection{Granulation of Spacetime at \( L_0 \)}
	
	The minimal length scale \( L_0 = \xi \cdot L_P \approx \SI{2.155e-39}{\meter} \) represents a discrete spacetime structure, at which all vacuum fluctuations causing the Casimir effect and other forces are fully effective. At this distance, all wave modes are present without restriction, leading to a maximum energy density. For distances \( d > L_0 \), only parts of these forces become visible through the \( 1/d^4 \) dependence of the Casimir energy density, as the plates restrict the wave modes. The extremely small size of \( L_0 \) prevents a direct experimental measurement at present, which is why the theory introduces the measurable scale \( L_\xi \approx \SI{100}{\micro\meter} \) to investigate the vacuum structure indirectly.
	
	\subsection{Coupling Constant \( \xi \) as Space Parameter}
	
	The coupling constant \( \xi = \num{1.333e-4} \) is a fundamental space parameter that links the granulation of spacetime at \( L_0 \) with measurable effects. It can be derived from a Lagrangian that describes the dynamics of a time field:
	
	\begin{equation}
		\mathcal{L} = -\frac{1}{4} F_{\mu\nu} F^{\mu\nu} + \frac{1}{2} (\partial_\mu \phi)^2 - \xi \cdot \frac{\hbar c}{L_0^4} \cdot \phi^2 \label{eq:lagrangian}
	\end{equation}
	
	Here, \( \phi \) is a time field that describes the temporal structure of spacetime, and the term \( \xi \cdot \frac{\hbar c}{L_0^4} \cdot \phi^2 \) introduces an energy density that is linked to \( \rho_{\text{CMB}} \).
	
	\subsection{Emergent Gravity}
	
	Gravity is interpreted as an emergent property of a time field \( \phi \), whose fluctuations on the scale \( L_0 \) generate the spacetime structure. The coupling constant \( \xi \) determines the strength of these interactions, thereby allowing cosmic effects such as the CMB to be explained without the assumption of dark energy or dark matter.
	
	\section{Experimental Predictions}
	
	\subsection{Critical Distances}
	
	The theory makes specific predictions for the behavior of the Casimir effect at characteristic distances:
	
	\begin{table}[H]
		\centering
		\begin{tabular}{c S[table-format=1.2e-2] c}
			\toprule
			Distance \( d \) & {\(\rho_{\text{Casimir}}\) (\unit{\joule\per\meter\cubed})} & {Ratio to CMB} \\
			\midrule
			\SI{100}{\micro\meter} & 4.17e-14 & 1.00 \\
			\SI{10}{\micro\meter} & 4.17e-10 & \num{1.0e4} \\
			\SI{1}{\micro\meter} & 4.17e-2 & \num{1.0e12} \\
			\bottomrule
		\end{tabular}
		\caption{Predictions for Casimir energy densities and their ratio to the CMB energy density}
		\label{tab:predictions}
	\end{table}
	
	\subsection{Experimental Tests}
	
	The most important experimental verifications of the theory include:
	
	\begin{enumerate}
		\item \textbf{Precision measurements at \( d = L_\xi \)}: At a plate distance of approximately \SI{100}{\micro\meter}, the Casimir energy density reaches values in the range of the CMB energy density, confirming the connection between vacuum structure and cosmic effects.
		\item \textbf{Scaling behavior}: The \( (1/d^4) \) dependence should be precisely fulfilled down to the micrometer range, supporting the theory.
		\item \textbf{Indirect tests of granulation}: Since the minimal length scale \( L_0 \approx \SI{2.155e-39}{\meter} \) is currently not directly measurable, deviations from the \( 1/d^4 \) scaling at very small distances (\( d \approx \SI{10}{\nano\meter} \)) could provide indications of spacetime granulation.
	\end{enumerate}
	
	\subsection{Experimental Measurement Data}
	
	The experimental \( L_\xi \)-values are:
	\begin{itemize}
		\item Parallel plates: \( \SI{228}{\nano\meter} \) \cite{dhital2024}.
		\item Sphere-plate: \( \SI{1.75}{\micro\meter} \) \cite{xu2022}.
		\item Further value: \( \SI{18}{\micro\meter} \).
	\end{itemize}
	
	The scatter (228 nm to 18 \(\mu\)m) is plausible and reflects geometric differences (\( F \propto 1/L^4 \) for parallel plates, \( F \propto 1/L^3 \) for sphere-plate) as well as experimental conditions.
	
	\section{Theoretical Extensions}
	
	\subsection{Geometry Dependence}
	
	The characteristic length scale \( L_\xi \) may depend on the specific geometry of the Casimir arrangement:
	
	\begin{equation}
		L_\xi = L_\xi(\text{Geometry}, \text{Materials}, \omega) \label{eq:Lxi_dependencies}
	\end{equation}
	
	This would naturally explain the observed scatter in experimental Casimir measurements and make the theory flexible enough to describe various physical situations.
	
	\subsection{Frequency Dependence}
	
	A possible extension of the theory could consider a frequency dependence of the vacuum parameters, leading to dispersive effects in the Casimir force.
	
	\section{Cosmological Implications}
	
	\subsection{Vacuum Energy Density and Apparent Cosmic Expansion}
	
	The developed theory connects local vacuum effects (Casimir) with cosmic observations (CMB) through the fundamental spacetime structure at \( L_0 \). The CMB energy density \( \rho_{\text{CMB}} = \frac{\xi \hbar c}{L_\xi^4} \) is interpreted as a signature of a time field from which gravity emerges. This emergent gravity explains the apparent cosmic expansion without the need for dark energy or dark matter.
	
	\subsection{Early Universe}
	
	In the early phase of the universe, when characteristic length scales were in the range of \( L_\xi \), Casimir-like effects may have played a significant role in cosmic evolution, influenced by the granular spacetime at \( L_0 \).
	
		\rho_{\rm vac} = \hbar c  A_d  k_{\max}^{d+1},
		\qquad
		A_d \equiv \frac{\pi^{-d/2}}{2^d\Gamma(d/2)(d+1)}.
	\end{equation}
	
	Setting $k_{\max}=\alpha/L_\xi$ leads to the matching
	\begin{equation}
		\rho_{\rm vac} = \hbar c  A_d  \frac{\alpha^{d+1}}{L_\xi^{d+1}}
		\quad\Rightarrow\quad
		\xi = A_d \alpha^{d+1}.
	\end{equation}
	
	\subsection{Numerical Sensitivity}
	The numerical sensitivity curve for $\xi(A_d)$ at $d=3+\delta$.
	
	\section{Regularization: Zeta Function (Sketch)}
	The zeta function regularization leads through analytic continuation of the spectral zeta function to the regularized energy at $s=-1$. For details, see Appendix~\ref{app:zeta_full}.
	
	\section{RG Sketch and Models for $\gamma$}
	A useful parameterization approach is
	\begin{equation}
		L_\xi = L_P\xi^{\gamma},
	\end{equation}
	leading to the closed relation (for $d=3$)
	\begin{equation}
		\xi = \left[ C \left(\frac{k_B T_{\rm CMB} L_P}{\hbar c}\right)^4 \right]^{1/(1-4\gamma)},\qquad C=\frac{\pi^2}{15}.
	\end{equation}
	
	The function $\xi(\gamma)$ and its uncertainty band (Monte-Carlo over $\alpha\in[0.5,2]$) is shown in Figure~\ref{fig:xi_gamma_mc}.
	
	\begin{figure}[htbp]
		\centering
		
		\caption{Median and 16--84\% band for $\xi(\gamma)$ with variation of the cutoff factor $\alpha\in[0.5,2]$.}
		\label{fig:xi_gamma_mc}
	\end{figure}
	
	\section{Implicit Coupling Models}
	For the model $\delta(\xi)=\beta\ln\xi$, the implicit equation is $\xi=A_{3+\beta\ln\xi}$; numerical solutions are shown in Figure~\ref{fig:xi_vs_beta}.
	
	\begin{figure}[htbp]
		\centering
		
		\caption{Implicit solutions $\xi(\beta)$ for $\beta\in[-1,1]$.}
		\label{fig:xi_vs_beta}
	\end{figure}
	
	\section{Implications and Connections}
	\label{sec:discussion}
	
	From the calculations, a clear chain of connections emerges:
	
	\begin{enumerate}
		\item \textbf{Fractal Dimension $\delta$:} Even small deviations from $d=3$ significantly affect the zero-point energy. The geometry directly impacts the vacuum energy density.
		\item \textbf{Regularization:} The zeta function regularization reveals that divergences do not disappear but are transferred into an effective constant $\xi$. This constant is physically measurable.
		\item \textbf{Renormalization Group Aspect:} Through the anomalous dimension $\gamma$, a scale dependence of $\xi$ emerges. Thus, the theory has an RG structure similar to quantum field theory.
		\item \textbf{Observations:} The matching to the CMB temperature fixes $\xi$ almost completely. The cosmological observation thus becomes a measuring instrument for a fundamental coupling.
		\item \textbf{Overall View:} A closed chain emerges:
		\[
		\text{Time-Mass Duality} \Rightarrow \text{fractal mode counting}
		\Rightarrow \text{Regularization}
		\Rightarrow \xi
		\Rightarrow T_{\rm CMB}.
		\]
		Changes at the beginning (microstructure) shift the end (macrostructure).
	\end{enumerate}
	
	\textbf{Lesson:} Microstructure (fractal spatial dimension, field excitations) and macrostructure (CMB, cosmological scales) are inseparably linked through the fundamental coupling $\xi$. Thus, the T0 theory builds a bridge between quantum fluctuations and cosmology.
	
	\appendix
	\section{Complete Zeta Regularization: Details}
	\label{app:zeta_full}
	
	This section contains the complete step-by-step evaluation of the zeta function integrals, the transformation into gamma functions, and the treatment of poles. (The detailed derivation can be output in full length upon request.)
	
	\section{Numerical Data}
	The raw data used for the plots are included as a CSV file in the accompanying archive.
	
	\section{Mode Counting and Zero-Point Energy in Fractal Spatial Dimension}
	\label{sec:modecounting}
	
	In this section, we calculate the vacuum energy density resulting from the mode structure of a scalar field in an effective spatial dimension
	\[
	d = 3 + \delta,\qquad |\delta| \ll 1.
	\]
	The goal is to show that the dimensionless prefactor \(\xi\) naturally emerges from the mode counting and depends only on \(d\) (or \(\delta\)).
	
	\subsection{Mode Counting with Hard Cutoff}
	For massless modes with dispersion \(\omega(k)=c|k|\), the zero-point energy density per volume is
	\[
	\rho_{\rm vac} = \frac{\hbar}{2}\int \frac{d^{d}k}{(2\pi)^d}\omega(k)
	= \frac{\hbar c}{2}\int\frac{d^{d}k}{(2\pi)^d}|k|.
	\]
	With the explicit volume element in momentum space
	\[
	\int d^{d}k = S_{d-1}\int_0^{k_{\max}} k^{d-1}dk,
	\qquad
	S_{d-1}=\frac{2\pi^{d/2}}{\Gamma(d/2)},
	\]
	it follows
	\begin{align}
		\rho_{\rm vac}
		&= \frac{\hbar c}{2}\frac{S_{d-1}}{(2\pi)^d}\int_0^{k_{\max}} k^{d}dk
		= \frac{\hbar c}{2}\frac{S_{d-1}}{(2\pi)^d}\frac{k_{\max}^{d+1}}{d+1}
		\nonumber\\
		&= \hbar c  A_d  k_{\max}^{d+1},
		\label{eq:rho_Ad}
	\end{align}
	where we introduce the dimensionless constant
	\[
	\boxed{A_d = \dfrac{\pi^{-d/2}}{2^d\Gamma(d/2)(d+1)}}
	\]
	$A_d$ depends only on the effective spatial dimension \(d\).
	
	Setting the natural cutoff \(k_{\max}=\alpha/L_\xi\) (with \(\alpha\sim O(1)\)), yields
	\[
	\rho_{\rm vac} = \hbar c  A_d  \frac{\alpha^{d+1}}{L_\xi^{d+1}}.
	\tag{\ref{eq:rho_Ad}$'$}
	\]
	
	\subsection{Matching to the T0 Model}
	In your T0 approach, the vacuum energy density is model-wise written as
	\[
	\rho_{\rm model}=\xi\frac{\hbar c}{L_\xi^{d+1}}.
	\]
	Equating with \eqref{eq:rho_Ad}$'$ gives
	\[
	\boxed{\xi = A_d\alpha^{d+1}}.
	\]
	In the simplest case \(\alpha=1\), it immediately follows
	\[
	\boxed{\xi = A_d = \dfrac{\pi^{-d/2}}{2^d\Gamma(d/2)(d+1)}}.
	\]
	Thus, \(\xi\) is a pure, dimensionless prefactor that results solely from the effective spatial dimension \(d\) — a result that exactly matches the ``consequence case'' you aim for: \(\xi\) emerges from the mode counting.
	
	\subsection{Numerical Sensitivity Near \(d=3\)}
	Setting \(d=3+\delta\), \(\xi(\delta)=A_{3+\delta}\). For some representative values of \(\delta\), one obtains (numerically):
	\begin{center}
		\begin{tabular}{r c c}
			\toprule
			\(\delta\) & \(d=3+\delta\) & \(\xi(\delta)=A_d\) \\
			\midrule
			-0.10 & 2.90 & \(7.375872\times10^{-3}\) \\
			-0.05 & 2.95 & \(6.835838\times10^{-3}\) \\
			-0.01 & 2.99 & \(6.430394\times10^{-3}\) \\
			\(0.00\) & 3.00 & \(6.332574\times10^{-3}\) \\
			\(0.01\) & 3.01 & \(6.236135\times10^{-3}\) \\
			\(0.05\) & 3.05 & \(5.863850\times10^{-3}\) \\
			\(0.10\) & 3.10 & \(5.427545\times10^{-3}\) \\
			\bottomrule
		\end{tabular}
	\end{center}
	
	The associated sensitivity curve \(\xi(\delta)\) (for \(\delta\in[-0.1,0.1]\)) 
	
	%\includegraphics[width=0.75\textwidth]{xi_vs_delta.png}
	%*{Sensitivity of the dimensionless prefactor \(\xi=A_{d}\) to small changes in the Hausdorff dimension \(\delta\) (with \(d=3+\delta\)).}
	%\label{fig:xi_vs_delta}
	
	\noindent\textbf{Remark.} The numerical evaluation shows that \(\xi\) near \(d=3\) has an order of magnitude \(\sim 6.3\times10^{-3}\) (for \(\alpha=1\)). Small changes in \(\delta\) change \(\xi\) by a few \(10^{-4}\) — i.e., the sensitivity is measurable but not ``explosive''.
	
	\section{Regularization: Zeta Function (Appendix)}
	\label{app:zeta}
	
	For the formal regularization of the mode sum, zeta function regularization is recommended. The short path (sketch):
	
	\begin{itemize}
		\item Write the unordered sum of zero-point energies as
		\[
		E_0 = \frac{\hbar}{2}\sum_{\mathbf{k}}\omega_{\mathbf{k}} = \frac{\hbar c}{2}\sum_{\mathbf{k}}|\mathbf{k}|.
		\]
		\item Define the spectral zeta function
		\[
		\zeta(s) := \sum_{\mathbf{k}} |\mathbf{k}|^{-s},
		\]
		where the sum runs over the quantized momentum grid; for a continuous momentum space, replace by an integral with a mode density \(\rho(\omega)\propto \omega^{d-1}\).
		\item The regularized zero-point energy is then
		\[
		E_0^{\rm reg} = \frac{\hbar c}{2}\zeta(-1),
		\]
		where \(\zeta(s)\) is analytically continued.
		\item For a continuum momentum space with mode density \(\rho(\omega) \sim \omega^{d-1}\), the zeta integrals can be explicitly evaluated; the result has the same gamma factors as in \eqref{eq:rho_Ad} and consistently leads to the form \(\rho\propto A_d k_{\max}^{d+1}\) after appropriate treatment of poles.
	\end{itemize}
	
	\section{RG Sketch and Derivation of \(\gamma\)}
	\label{sec:rg_gamma}
	
	The question of whether \(L_\xi\) is independent or back-coupled with \(\xi\) is crucial. Two useful model approaches:
	
	\paragraph{(A) Static fractal dimension.} If \(\delta\) is approximately constant, \(\xi=A_{3+\delta}\) (direct determination).
	
	\paragraph{(B) Scale-dependent dimension / coupling feedback.} If \(\delta\) depends on the coupling \(\xi\), e.g., \(\delta(\xi)=\beta\ln\xi\) (model-wise), an implicit equation is obtained
	\[
	\xi = A_{3+\beta\ln\xi},
	\]
	which must be solved numerically. Such equations can show ambiguities or strong nonlinearities, depending on the sign of \(\beta\).
	
	\paragraph{Parameterization over \(\gamma\).} A more useful approach is often
	\[
	L_\xi = L_P\xi^{\gamma},
	\]
	where \(L_P\) is the Planck length. Combining this approach with the observational relationship between \(\rho\) and \(T_{\rm CMB}\) (see main text) yields — for the case \(d=3\) — the closed solution
	\[
	\xi = \left[ C \left(\frac{k_B T_{\rm CMB} L_P}{\hbar c}\right)^4 \right]^{1/(1-4\gamma)},\qquad C=\frac{\pi^2}{15},
	\]
	provided \(1-4\gamma\neq 0\). Thus, every determination of \(\gamma\) (from RG / anomalous dimensions) can be directly converted into a numerical determination of \(\xi\).
	
	\section{Matching to Observations and Error Estimation}
	For matching to the measured CMB temperature \(T_{\rm CMB}=2.725\ \mathrm{K}\), two paths can be followed:
	\begin{enumerate}
		\item \emph{Direct matching} via the fractal calculation: \(\xi=A_{3+\delta}\) and \(\rho_{\rm vac}=\xi\hbar c/L_\xi^{d+1}$. The main uncertainty here is the determination of \(\delta\) and the cutoff factor \(\alpha\).
		\item \emph{Scaling approach} \(L_\xi=L_P\xi^\gamma\): Then the above closed formula offers a direct relation \(\xi(\gamma)\). The measurement uncertainty of \(T_{\rm CMB}\) is negligible compared to the theoretical uncertainties (regularization, \(\delta\), \(\alpha\)).
	\end{enumerate}
	
	\section{Notation}
	\label{sec:notation}
	
	The following table contains all symbols used in this paper and their meanings.
	
	\subsection{Fundamental Constants}
	\begin{longtable}{p{2.0cm} p{8.0cm} p{2.4cm}}
		\toprule
		\textbf{Symbol} & \textbf{Meaning} & \textbf{Value/Unit} \\
		\midrule
		$\hbar$ & Reduced Planck's constant & $1.055 \times 10^{-34}$ J$\cdot$s \\
		$c$ & Speed of light in vacuum & $2.998 \times 10^8$ m/s \\
		$G$ & Gravitational constant & $6.674 \times 10^{-11}$ m$^3$/kg$\cdot$s$^2$ \\
		$k_B$ & Boltzmann constant & $1.381 \times 10^{-23}$ J/K \\
		$\pi$ & Circle constant & $3.14159\ldots$ \\
		\bottomrule
	\end{longtable}
\normalsize
	
	\subsection{Characteristic Length Scales}
	\begin{longtable}{p{2.0cm} p{8.0cm} p{2.4cm}}
		\toprule
		\textbf{Symbol} & \textbf{Meaning} & \textbf{Value/Unit} \\
		\midrule
		$L_P$ & Planck length & $1.616 \times 10^{-35}$ m \\
		$L_0$ & Minimal length scale of granular spacetime & $2.155 \times 10^{-39}$ m \\
		$L_\xi$ & Characteristic vacuum length scale & $\approx 100$ $\mu$m \\
		$d$ & Distance between Casimir plates & Variable [m] \\
		\bottomrule
	\end{longtable}
\normalsize
	
	\subsection{Coupling Parameters and Dimensionless Quantities}
	\begin{longtable}{p{2.0cm} p{8.0cm} p{2.4cm}}
		\toprule
		\textbf{Symbol} & \textbf{Meaning} & \textbf{Value/Unit} \\
		\midrule
		$\xi$ & Fundamental dimensionless coupling constant & $1.333 \times 10^{-4}$ \\
		$\alpha$ & Cutoff factor for mode counting & $\mathcal{O}(1)$ [dimensionless] \\
		$\gamma$ & Anomalous dimension in RG approach & Variable [dimensionless] \\
		$\beta$ & Coupling parameter for fractal dimension & Variable [dimensionless] \\
		$\delta$ & Deviation from spatial dimension 3 & $|\delta| \ll 1$ [dimensionless] \\
		\bottomrule
	\end{longtable}
\normalsize
	
	\subsection{Energy Densities and Temperatures}
	\begin{longtable}{p{2.0cm} p{8.0cm} p{2.4cm}}
		\toprule
		\textbf{Symbol} & \textbf{Meaning} & \textbf{Value/Unit} \\
		\midrule
		$\rho_{\text{CMB}}$ & Energy density of cosmic microwave background & $4.17 \times 10^{-14}$ J/m$^3$ \\
		$\rho_{\text{Casimir}}(d)$ & Casimir energy density as function of distance & [J/m$^3$] \\
		$\rho_{\text{vac}}$ & Vacuum energy density & [J/m$^3$] \\
		$T_{\text{CMB}}$ & Temperature of cosmic microwave background & $2.725$ K \\
		\bottomrule
	\end{longtable}
\normalsize
	
	\subsection{Mathematical Functions and Operators}
	\begin{longtable}{p{2.0cm} p{8.0cm} p{2.4cm}}
		\toprule
		\textbf{Symbol} & \textbf{Meaning} & \textbf{Remark} \\
		\midrule
		$\Gamma(x)$ & Gamma function & $\Gamma(n) = (n-1)!$ for $n \in \mathbb{N}$ \\
		$\zeta(s)$ & Riemann zeta function & Regularization \\
		$A_d$ & Dimension-dependent prefactor & $A_d = \frac{\pi^{-d/2}}{2^d\Gamma(d/2)(d+1)}$ \\
		$S_{d-1}$ & Surface of $(d-1)$-dimensional unit sphere & $S_{d-1} = \frac{2\pi^{d/2}}{\Gamma(d/2)}$ \\
		$\mathcal{L}$ & Lagrangian density & Lagrangian formulation \\
		\bottomrule
	\end{longtable}
\normalsize
	
	\subsection{Fields and Wave Vectors}
	\begin{longtable}{p{2.0cm} p{8.0cm} p{2.4cm}}
		\toprule
		\textbf{Symbol} & \textbf{Meaning} & \textbf{Unit} \\
		\midrule
		$\phi$ & Time field & [dimension-dependent] \\
		$\mathbf{k}$ & Wave vector & [m$^{-1}$] \\
		$k$ & Magnitude of wave vector, $k = |\mathbf{k}|$ & [m$^{-1}$] \\
		$k_{\max}$ & Maximum cutoff wave vector & [m$^{-1}$] \\
		$\omega(k)$ & Dispersion relation & [s$^{-1}$] \\
		$F_{\mu\nu}$ & Field strength tensor & Gauge field theory \\
		\bottomrule
	\end{longtable}
\normalsize
	
	\subsection{Geometric and Topological Parameters}
	\begin{longtable}{p{2.0cm} p{8.0cm} p{2.4cm}}
		\toprule
		\textbf{Symbol} & \textbf{Meaning} & \textbf{Remark} \\
		\midrule
		$d$ & Effective spatial dimension & $d = 3 + \delta$ \\
		$D$ & Hausdorff dimension of spacetime & Fractal geometry \\
		$\partial_\mu$ & Partial derivative with respect to $x^\mu$ & Covariant notation \\
		$\nabla$ & Nabla operator & Spatial derivatives \\
		\bottomrule
	\end{longtable}
\normalsize
	
	\subsection{Experimental Parameters}
	\begin{longtable}{p{2.0cm} p{8.0cm} p{2.4cm}}
		\toprule
		\textbf{Symbol} & \textbf{Meaning} & \textbf{Typical Range} \\
		\midrule
		$d_{\text{exp}}$ & Experimental plate distance (Casimir) & $10$ nm - $10$ $\mu$m \\
		$L_{\xi,\text{exp}}$ & Experimentally determined characteristic length & $228$ nm - $18$ $\mu$m \\
		$F_{\text{Casimir}}$ & Casimir force per unit area & [N/m$^2$] \\
		\bottomrule
	\end{longtable}
\normalsize
	
	\subsection{Ratio Quantities and Scalings}
	\begin{longtable}{p{2.0cm} p{8.0cm} p{2.4cm}}
		\toprule
		\textbf{Symbol} & \textbf{Meaning} & \textbf{Remark} \\
		\midrule
		$\frac{L_0}{L_P}$ & Ratio sub-Planck to Planck & $= \xi = 1.333 \times 10^{-4}$ \\
		$\frac{L_P}{L_\xi}$ & Ratio Planck to Casimir-characteristic & $\approx 1.616 \times 10^{-31}$ \\
		$\frac{L_\xi}{d}$ & Scaling parameter for Casimir effect & Dimensionless \\
		$\left(\frac{L_\xi}{d}\right)^4$ & Casimir scaling factor & Characteristic $d^{-4}$ dependence \\
		\bottomrule
	\end{longtable}
\normalsize
	
	\subsection{Abbreviations and Indices}
	\begin{longtable}{p{2.0cm} p{8.0cm} p{2.4cm}}
		\toprule
		\textbf{Symbol} & \textbf{Meaning} & \textbf{Context} \\
		\midrule
		CMB & Cosmic Microwave Background & Cosmic microwave background \\
		RG & Renormalization Group & Renormalization group \\
		vac & vacuum & Vacuum \\
		exp & experimental & Experimental \\
		reg & regularized & Regularized \\
		$\mu, \nu$ & Lorentz indices & Relativistic notation ($0,1,2,3$) \\
		$i, j, k$ & Spatial indices & Spatial coordinates ($1,2,3$) \\
		\bottomrule
	\end{longtable}
\normalsize
	
	\subsection{Constants in Numerical Formulas}
	\begin{longtable}{p{2.0cm} p{8.0cm} p{2.4cm}}
		\toprule
		\textbf{Symbol} & \textbf{Meaning} & \textbf{Value} \\
		\midrule
		$\frac{4}{3} \times 10^{-4}$ & Numerical value of $\xi$ & $1.333 \times 10^{-4}$ \\
		$\frac{\pi^2}{240}$ & Casimir prefactor & $\approx 0.0411$ \\
		$\frac{\pi^2}{15}$ & Stefan-Boltzmann-related factor & $\approx 0.658$ \\
		$240$ & Denominator in Casimir formula & Exact \\
		\bottomrule
	\end{longtable}
\normalsize

\end{document}

