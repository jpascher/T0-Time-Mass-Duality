% Standardized preamble - 006_T0_ParticleMasses_En.tex
\documentclass[12pt,a4paper]{article}

% Now include the shared preamble (it contains all \usepackage commands)
\input{../../../T0_preamble_shared-ebook_En}

\title{Fractal Spacetime and its Implications in Quantum Gravity}
\author{Johann Pascher}
\date{January 15, 2026}

\begin{document}
	
	\maketitle
	
	\begin{abstract}
		This document summarizes central results from theoretical physics on the fractal structure of spacetime in various approaches to quantum gravity. Particular emphasis is placed on dimensional flow (from spectral dimension $\sim 2$ in the UV to $\sim 4$ in the IR), the resulting renormalizability, and implications for singularities, gravitational potential, and causality. The presentation is based exclusively on published scientific papers.
	\end{abstract}
	
	\section{Introduction: From Fundamentals to the Problem}
	\subsection{The Fundamental Problem of Quantum Gravity}
	Modern physics is based on two revolutionary pillars: \emph{General Relativity} (GR) and \emph{Quantum Mechanics}. GR describes gravity as a curvature of \textbf{spacetime} – a four-dimensional fabric in which space and time are inextricably linked. This theory is extremely successful on large scales (stars, galaxies). Quantum mechanics, on the other hand, describes the behavior of matter and forces (except gravity) on the smallest scales (atoms, particles) with overwhelming accuracy.
	
	The fundamental problem is that these two theories are mathematically and conceptually incompatible. At points of extreme density, such as at the center of a black hole or at the beginning of the universe (\textbf{Big Bang singularity}), both theories yield nonsensical or infinite results. A theory of \textbf{quantum gravity} that unifies both is therefore sought.
	
	\subsection{The Challenge of ''Infinities'' (Renormalizability)}
	A major problem in quantizing gravity is \textbf{divergences} – mathematical terms that go to infinity. In quantum field theory (the language of particle physics), such infinities constantly appear but can be systematically ''subtracted'' and replaced by finite measurement values through a procedure called \textbf{renormalization}. A theory where this is possible is called \textbf{renormalizable}. For gravity in four dimensions, however, this ''trick'' of perturbation theory (\emph{perturbative}) does not work – infinitely many new types of infinities arise that can no longer be controlled. The theory is \textbf{non-renormalizable}.
	
	\subsection{A Radical Idea: What if Spacetime Itself is Different?}
	Standard GR assumes a smooth, continuous spacetime that is everywhere differentiable (one can draw a tangent at every point). But what if this notion is wrong on fundamental, smallest scales (the \textbf{Planck scale}, $\sim 10^{-35}$ meters)? The idea of a \textbf{fractal spacetime} states that spacetime on these scales does not have a simple, smooth structure but is \textbf{rough}, \textbf{broken}, and \textbf{self-similar} – similar to the infinitely detailed coastline of an island that reveals new structures at every magnification. Such a structure is \textbf{non-differentiable}.
	
	For such complex structures, ordinary dimensional notions (1D=line, 2D=surface, 3D=volume) are insufficient. One introduces the \textbf{fractal dimension} (or \textbf{Hausdorff dimension}) $d_H$, which can be non-integer (e.g., ~1.26 for a coastline). An even better indicator for the behavior of a quantum theory on such a structure is the \textbf{spectral dimension} $d_s$. It measures how a particle (or information) ''experiences'' the structure through diffusion. A decisive result of many new approaches is that $d_s$ drops to about \textbf{2} at high energies/small scales (\textbf{UV}, ''ultraviolet''), while at low energies/large scales (\textbf{IR}, ''infrared'') it assumes the value \textbf{4} (three space plus one time dimension). This transition is called \textbf{dimensional flow}.
	
	\subsection{The Central Hypothesis and its Utility}
	The central thesis of this document is that precisely this \textbf{dimensional flow to $d_s \approx 2$ in the UV} solves the renormalization problem. At an effective dimension of 2, the quantum gravity theory becomes \textbf{power-counting renormalizable} – the infinities become controllable or even vanish completely. This provides an elegant, geometry-inherent \textbf{UV cutoff}. Moreover, a fractal, non-continuous structure ''smears out'' the sharp \textbf{singularities} of GR and could thus mitigate problems like the black hole information paradox.
	
	The following chapters unfold this idea in detail, based on concrete research programs.
	
	\section{Dimensional Flow and Fractal Geometry}
	
	\subsection{Hausdorff and Spectral Dimension}
	The spectral dimension $d_s$ is defined via the return probability of a random walk. In the considered models:
	\begin{equation}
		d_s \sim 
		\begin{cases} 
			2 & \text{(UV, Planck scale)} \\
			4 & \text{(IR, macroscopic scales)}
		\end{cases}
	\end{equation}
	
	This flow occurs in the following approaches:
	\begin{itemize}
		\item Asymptotic Safety (Reuter et al.)
		\item Causal Dynamical Triangulations (CDT)
		\item Multifractal Spacetimes (Calcagni)
		\item Approximations in Loop Quantum Gravity
	\end{itemize}
	
	This phenomenon is supported by several independent research strands. Modesto argues that the analysis of Feynman diagrams on a spin foam yields an effective spectral dimension of nearly 2 near the Planck scale \cite{Modesto2008}. Furthermore, Hořava confirms this flow in his approach ''Spectral Dimension of the Universe in Quantum Gravity at a Lifshitz Point'', where \textit{''the spectral dimension of spacetime flows from $d_s=4$ at large scales, to $d_s=2$ at short distances.''} \cite{Horava2009}.
	
	The universal character of this result is emphasized by Modesto: \textit{''This result is consistent with two other approaches to non perturbative quantum gravity: 'causal dynamical triangulation' and 'asymptotically safe quantum gravity'.''} \cite{Modesto2009}.
	
	\subsection{Multifractal Geometries and Fractional Analysis}
	To describe scale-dependent dimensions, fractional calculus is employed. The Lagrangian density is formulated with fractional derivatives, causing the dimension to vary continuously with scale. Calcagni explains this as the basis of multifractal spacetimes: \textit{''Based on fractional calculus, these continuous spacetimes have their dimension changing with the scale.''} \cite{Calcagni2012}.
	
	\textbf{T0 Time-Mass Duality / Fundamental-Fractal-Geometric Field Theory (FFGFT):} In the present approach, the dimensional flow to fractal structure is not treated as an additional postulate but necessarily follows from the fundamental T0 Time-Mass Duality itself. Gravity emerges in this framework as an effective phenomenon of this underlying, structured field theory. Thus, the \textbf{T0 approach} represents an independent path that avoids the assumption of a separate quantum field for gravity and instead bases spacetime geometry and matter on unified principles.
	
	\section{Implications for Gravity}
	
	\subsection{Renormalizability}
	In four dimensions, perturbative quantum gravity is non-renormalizable. By reducing the spectral dimension to $d_s \approx 2$ in the UV, however, the theory becomes \emph{power-counting renormalizable} or even super-renormalizable.
	
	This avoids infinitely many counterterms and enables a consistent quantum theory of gravity. Calcagni specifies for his models that \textit{''A field theory...which lives in fractal spacetime...is argued to be power-counting renormalizable, ultraviolet finite, and causal at microscopic scales.''} \cite{Calcagni2010a}.
	
	An example of this can be found in ''Fractal Quantum Space-Time'' by Modesto, who notes that \textit{''a system of spin-foam models for Euclidean quantum gravity [is] finite to all orders in the perturbative expansion, and that ultraviolet divergences disappear in the non-perturbative regime.''} \cite{Modesto2009}.
	
	\subsection{Singularities}
	Point singularities (black holes, Big Bang) are resolved in fractal geometries. No true points remain, but rather fractally distributed density distributions. This mitigates the information paradox. According to Modesto, the properties of fractal quantum spacetimes suggest that \textit{''the singularity problem seems to be solved in the covariant formulation of quantum gravity in terms of spin-foam models.''} \cite{Modesto2009}.
	
	\subsection{Gravitational Potential and Causality}
	Newton's $1/r^2$ law holds only as a macroscopic approximation. In fractal geometry, the potential scales in a scale-dependent manner as $1/r^{d-1}$. Light cones become diffuse at small scales, implying an effective violation of strict locality. This fundamental mathematical consequence, which follows from dimensional flow, is not directly formulated in any of the cited works but represents a central implication for any phenomenological model that can be derived from these approaches.
	
	\section{The T0 Approach: Time-Mass Duality as Fundamental Fractal-Geometric Field Theory}
	
	\textbf{Yes, exactly – the T0 approach does not involve classical quantization of gravity.}
	
	Based on the core of the theory (as described in the Master Narrative and related documents), gravity is \textbf{not} quantized as a separate quantum field that would need to be treated with gravitons, loop diagrams, or a new quantum field theory of gravity. Instead:
	
	\begin{itemize}
		\item \textbf{Gravity is emergent} from the fundamental ontological duality between time and mass (T0 as the central bridge). It arises as an \textbf{effective geometric phenomenon} in a fractal spacetime regulated by $\xi$-corrections and scale-dependent dimensionality.
		
		\item There is \textbf{no need for perturbative quantization} of the Einstein-Hilbert action (which is notoriously non-renormalizable in 4D). The UV problems of standard quantum gravity dissolve because the fractal structure ($D_f \approx 2.94$ macroscopically, tending to $\sim 2$ in the UV) provides a natural regulator – similar to some other approaches, but here derived purely from the duality, without additional quantization procedure.
		
		\item The T0 approach consciously avoids the classical pitfalls of quantum gravity:
		\begin{itemize}
			\item No gravitons as fundamental particles
			\item No infinite counterterms or Landau poles
			\item No need for string theory, loops, or Asymptotic Safety as separate mechanisms
		\end{itemize}
		Instead, gravity becomes \textbf{finite and consistent through the duality} – it is quasi-classical on large scales but intrinsically regulated by fractal geometry on small scales.
	\end{itemize}
	
	In short: The T0 approach \textbf{circumvents} the quantization problem of gravity rather than solving it. Gravity requires \textbf{no separate quantization} because it already emerges completely and UV-finite from the fundamental principle (Time-Mass Duality + fractal geometry). This is one of the great advantages of this approach compared to conventional quantum gravity programs.
	
	\section{Comparison of Major Approaches}
	
	The central properties of the various approaches to fractal spacetime can be summarized as follows:
	
	\begin{itemize}
		\item \textbf{Asymptotic Safety (Reuter et al.)}
		\begin{itemize}
			\item Spectral Dimension UV: $\sim 2$
			\item Hausdorff Dimension UV: $\sim 2$
			\item Renormalizability: Yes (non-perturbative)
		\end{itemize}
		
		\item \textbf{Causal Dynamical Triangulations (CDT)}
		\begin{itemize}
			\item Spectral Dimension UV: $\sim 2$
			\item Hausdorff Dimension UV: $\sim 2$
			\item Renormalizability: Yes
		\end{itemize}
		
		\item \textbf{Multifractal Spacetime (Calcagni)}
		\begin{itemize}
			\item Spectral Dimension UV: $\sim 2$
			\item Hausdorff Dimension UV: $\sim 2$
			\item Renormalizability: Yes (perturbative)
		\end{itemize}
		
		\item \textbf{Loop Quantum Gravity (Approximations)}
		\begin{itemize}
			\item Spectral Dimension UV: $\sim 2$
			\item Hausdorff Dimension UV: variable
			\item Renormalizability: Yes (partial)
		\end{itemize}
		
		\item \textbf{T0 Approach (Time-Mass Duality / FGFT)}
		\begin{itemize}
			\item Spectral Dimension UV: $\sim 2$ (tendency)
			\item Hausdorff Dimension UV: $\sim 2.94 \rightarrow \sim 2$
			\item Renormalizability: UV-finite (emergent gravity)
			\item Special feature: No separate quantization needed; gravity emerges from Time-Mass Duality and the scale-dependent fractal structure of space.
		\end{itemize}
	\end{itemize}
	
	\section{Conclusion}
	The dimensional flow from $d_s \approx 2$ in the UV to $d_s \approx 4$ in the IR represents a universal, robust result in modern quantum gravity research. It provides an elegant mechanism for solving the renormalizability problem, mitigates singularities, and fundamentally changes our understanding of gravity at fundamental scales. The T0 approach (Time-Mass Duality / FFGFT) represents a radically alternative path that does not attempt to solve the quantization problem through increasingly complex quantum field theories but circumvents it through a fundamental ontological duality and an emergent, fractal spacetime geometry.
	
	\begin{thebibliography}{9}
		\bibitem[Modesto(2009)]{Modesto2009}
		Modesto, L. (2009). \textit{Fractal Quantum Space-Time}. \texttt{arXiv:0905.1665 [gr-qc]}.
		
		\bibitem[Modesto(2008)]{Modesto2008}
		Modesto, L. (2008). \textit{Fractal Structure of Loop Quantum Gravity}. \texttt{arXiv:0812.2214 [gr-qc]}. Published in Class. Quantum Grav. 26 (2009) 242002.
		
		\bibitem{Magliaro2009}
		Magliaro, E., Perini, C., Modesto, L. (2009). \textit{Fractal Space-Time from Spin-Foams}. \texttt{arXiv:0911.0437 [gr-qc]}.
		
		\bibitem[Calcagni(2010)]{Calcagni2010}
		Calcagni, G. (2010). \textit{Fractal universe and quantum gravity}. \texttt{arXiv:0912.3142 [hep-th]}. Phys. Rev. Lett. 104, 251301 (2010).
		
		\bibitem[Calcagni(2010)]{Calcagni2010a}
		Calcagni, G. (2010). \textit{Quantum field theory, gravity and cosmology in a fractal universe}. \texttt{arXiv:1001.0571 [hep-th]}. JHEP 03 (2010) 120.
		
		\bibitem[Calcagni(2012)]{Calcagni2012}
		Calcagni, G. (2012). \textit{Introduction to multifractional spacetimes}. \texttt{arXiv:1209.1110 [hep-th]}. AIP Conf. Proc. 1483 (2012) 31.
		
		\bibitem[Hořava(2009)]{Horava2009}
		Hořava, P. (2009). \textit{Spectral Dimension of the Universe in Quantum Gravity at a Lifshitz Point}. \texttt{arXiv:0902.3657 [hep-th]}. Phys. Rev. Lett. 102, 161301 (2009).
		
		\bibitem{Thurigen2015}
		Thürigen, J. (2015). \textit{Discrete Quantum Geometries}. \texttt{arXiv:1511.08737 [gr-qc]}.
	\end{thebibliography}
	
\end{document}