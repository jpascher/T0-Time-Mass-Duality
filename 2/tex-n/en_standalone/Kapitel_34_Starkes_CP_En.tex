\documentclass[12pt,a4paper]{article}
\usepackage[utf8]{inputenc}
\usepackage{amsmath,amssymb}
\usepackage{hyperref}
\usepackage{geometry}
\geometry{margin=2.5cm}

\title{{Chapter 34: Chapter 34}}
\author{{Dynamic Vacuum Field Theory with T0 Adaptations}}
\date{{\today}}

\begin{document}
\maketitle

CHAPTER 35: QUANTUM PHENOMENA EXPLAINED
DVFT interprets quantum mechanics as the behavior of vacuum-phase and vacuum-amplitude fields. This
chapter provides a unified explanation for twelve major unsolved quantum phenomena, including collapse,
entanglement, zero point energy, decoherence and delayed-choice experiments. DVFT clarifies these
phenomena by grounding them in the physical fields Φ = ρ e^{iθ}.
1. Wavefunction Collapse
In DVFT, collapse is not a postulate. It occurs when vacuum-phase coherence (θ) is disrupted by
macroscopic interactions. Measurement destroys θ-coherence, forcing ψ to localize.
2. Wave–Particle Duality
Waves correspond to coherent vacuum-phase patterns, while particles correspond to localized vacuumamplitude excitations. Duality becomes a property of Φ, not a paradox.
3. Entanglement
Entanglement arises from shared vacuum-phase coherence between separated systems. Global coherence
of θ allows nonlocal correlations without signaling.
4. Zero-Point Energy
International Journal for Multidisciplinary Research (IJFMR)
E-ISSN: 2582-2160 ● Website: www.ijfmr.com ● Email: editor@ijfmr.com
IJFMR250664112 Volume 7, Issue 6, November-December 2025 78
Dynamic vacuum field gives finite, physical zero-point energy ε_vac = ρ₀² (dθ/dt)², connecting vacuum
energy to cosmological acceleration.
5. Delayed Choice & Quantum Eraser
Interference depends on θ-coherence. Which-path detectors scramble θ; erasure restores it. DVFT removes
retrocausality by explaining phase re-coherence.
6. Decoherence
Decoherence is vacuum-phase scrambling. Macroscopic systems distort θ-fields and eliminate
interference patterns physically, not abstractly.
7. Quantum Randomness
Randomness arises from unavoidable vacuum-phase fluctuations: Δθ · ΔE ≥ ħ/2 produces inherent phase
jitter in Φ.
8. Atomic Quantization
Energy quantization corresponds to θ-field circulation conditions: ∮ ∇θ · dl = 2πn. Atomic spectra reflect
dynamic vacuum field waves.
Conclusion
DVFT unifies gravity and quantum mechanics by grounding quantum behavior in vacuum-phase
properties. Interference, collapse, entanglement, and decoherence all follow naturally from Φ = ρ e^{iθ}.


\section*{T0 Theory Integration}
This chapter integrates DVFT concepts with T0 Time-Mass Duality Theory, where the fundamental relation $T(x,t) \cdot m(x,t) = 1$ governs all vacuum field dynamics. The vacuum amplitude $\rho$ is directly related to local time $T$ through $\rho \propto 1/T$.

\end{document}
