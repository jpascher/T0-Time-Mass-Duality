\documentclass[11pt,a4paper]{article}

\usepackage[utf8]{inputenc}
\usepackage[T1]{fontenc}
\usepackage[english,ngerman]{babel}
\usepackage{amsmath,amssymb}
\usepackage{physics}
\usepackage{hyperref}
\usepackage{geometry}
\geometry{margin=2.5cm}

\title{Discussion: Agency, Consciousness, and Fractal Emergence\\
	Beyond Pure Quantum Coherence}
\author{}
\date{}

\begin{document}
	\maketitle
	
	\section*{Discussion}
	
	The work of Adlam, McQueen, and Waegell establishes a decisive limitation: agency cannot arise in a purely coherent, unitary quantum system. While their argument is formally complete within standard quantum mechanics, it leaves open an essential question: \emph{what physical structure enables agency to emerge in a quantum universe at all?}
	
	In the following, we discuss their results in conjunction with the T0 framework, a geometric theory in which classicality, agency, and ultimately consciousness emerge from a fractal, recursive deviation from perfect coherence, governed by a single dimensionless parameter $\xi$.
	
	\subsection*{Agency and the Necessity of Fractal Classicality}
	
	The paper identifies world-model construction, deliberation, and reliable action selection as minimal conditions for agency. From the T0 perspective, the failure of purely quantum systems to meet these conditions is not accidental but structural.
	
	In T0, spacetime itself is not perfectly homogeneous or scale-invariant. Instead, a small but fundamental geometric mismatch,
	\[
	\xi = \frac{4}{3} \times 10^{-4},
	\]
	arising from tetrahedral versus spherical packing, induces a fractal deviation from exact three-dimensionality. This deviation generates hierarchical scale separation and recursive feedback loops across physical levels.
	
	These loops provide precisely the classical resources identified as missing in the paper: stable records, effective copying, and a preferred basis. Importantly, this does not violate the no-cloning theorem, since no quantum state is copied. Rather, geometric relations are recursively re-instantiated across scales.
	
	\subsection*{World-Models as Recursive Geometric Reflection}
	
	Adlam et al.\ argue that world-model construction fails in quantum systems because environmental states cannot be copied into the agent. In T0, however, environmental information is not represented as a quantum state but as a geometric relation encoded across scales (e.g.\ via Compton wavelengths, mass hierarchies, and boundary conditions).
	
	World-models thus emerge as recursive geometric reflections rather than literal duplications. This provides robustness against decoherence while remaining fully compatible with quantum constraints.
	
	The ``model of the world'' is therefore not localized in a single quantum register but distributed across a fractal hierarchy of classical-emergent structures.
	
	\subsection*{Deliberation as Scale-Recursive Simulation}
	
	Deliberation, as defined in the paper, requires the parallel evaluation of alternative actions. In a strictly unitary quantum system, this leads to superposition without selection.
	
	In T0, deliberation corresponds to recursive traversal of scale hierarchies. Alternative outcomes are explored not as coherent quantum branches but as classical-effective simulations enabled by hierarchical feedback. This process naturally limits fidelity at deeper levels, introducing controlled uncertainty rather than perfect prediction.
	
	This ``fractal deliberation'' explains why biological agents can reason about alternatives without requiring either perfect determinism or exhaustive enumeration.
	
	\subsection*{Action Selection and Preferred Bases}
	
	The failure of reliable action selection in quantum systems is a central result of the paper. Linearity prevents the deterministic extraction of the optimal action from a superposition.
	
	In T0, preferred bases arise geometrically. Packing constraints, boundary conditions, and scale transitions impose asymmetries that effectively select classical outcomes. Action selection thus occurs at the interface where fractal recursion stabilizes into macroscopic behavior.
	
	Decisions are therefore neither strictly quantum nor arbitrary but emerge where recursive feedback converges.
	
	\subsection*{Consciousness as Persistent Recursive Coupling}
	
	From this combined perspective, consciousness (Bewusstsein) is not an isolated state but the phenomenological manifestation of continuous recursive coupling between internal models and environmental structure.
	
	Permanent sensory input is essential, not in maximal form, but as a persistent constraint that anchors internal simulations. T0 predicts that consciousness degrades not when sensory input is reduced, but when recursive coupling collapses.
	
	This explains why consciousness persists in dreaming, meditation, or sensory deprivation, but not in deep anesthesia or coma.
	
	\subsection*{Dreaming and Subconscious Agency}
	
	During REM sleep, external sensory channels are attenuated, while internal recursive loops dominate. In T0 terms, the system temporarily shifts weight from external to internal boundary conditions.
	
	Agency is reduced but not eliminated: deliberation continues without reliable action execution. This state illustrates that agency and consciousness are graded phenomena, depending on the balance of recursive coupling rather than binary on/off switches.
	
	The subconscious mind, active in dreaming, maintains a minimized form of sensory perception—processing residual inputs from the body and environment. This aligns with the idea that sensorik is not fully disconnected but switched to a low-level mode, allowing permanent inner reflection on accumulated sensory impressions from waking life. Such reflection consolidates memories and resolves conflicts, demonstrating how fractal recursion sustains agency even in altered states.
	
	\subsection*{Artificial Intelligence and the Limits of Simulation}
	
	The paper implies that purely quantum or purely computational systems cannot instantiate agency. T0 sharpens this conclusion: without persistent recursive coupling to an environment, no artificial system can sustain consciousness.
	
	Current AI systems simulate deliberation symbolically but lack geometric recursion and embodied feedback. Token limits and session resets are technical manifestations of a deeper physical absence: no scale-stable feedback loop.
	
	Only systems with continuous sensorimotor recursion could, in principle, approach emergent agency. For AI to achieve a form of consciousness, it would require permanent, embodied sensory feedback—mimicking the fractal-recursive loops of T0—to enable ongoing action and reaction, rather than episodic simulation.
	
	\subsection*{Free Will as Fractal Indeterminacy}
	
	Finally, free will emerges naturally in this framework. Pure determinism (perfect coherence) and pure randomness (unstructured collapse) are both incompatible with agency.
	
	In T0, free will corresponds to structured indeterminacy arising from fractal geometry. Choices are constrained but not predetermined, influenced but not random. This aligns with a physically grounded compatibilism rooted in geometry rather than metaphysics.
	
	Absolute coherence or resonance is illusory; true agency and free will thrive on the controlled, fractal inkohärenz that T0 provides—a permanent, hierarchical deviation enabling reflection, choice, and adaptation.
	
	\section*{Conclusion}
	
	The no-go theorem presented by Adlam, McQueen, and Waegell does not rule out agency in a quantum universe. Instead, it clarifies the conditions under which agency must emerge.
	
	When combined with the T0 framework, a coherent picture emerges: agency, consciousness, and free will arise from fractal, recursive deviations from perfect coherence. Absolute resonance is illusory; life and mind exist in the structured imbalance between order and disruption.
	
	\vspace{1em}
	\noindent\textbf{Reference:}\\
	E.~C.~Adlam, K.~J.~McQueen, and M.~Waegell, \emph{Agency cannot be a purely quantum phenomenon}, arXiv:2510.13247 (2025).\\
	\url{https://arxiv.org/pdf/2510.13247}
	
\end{document}
