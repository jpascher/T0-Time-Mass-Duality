\documentclass[12pt,a4paper]{article}

\input{T0_preamble_standalone_En}

\title{\textbf{Fine-Structure Constant: Unit Conventions}\\[0.5cm]
	 Why $\alpha = 1$ can be set\\[0.3cm]
	\normalsize Supplement to Document 011}
\author{}
\date{January 2025}

\begin{document}
	
	\maketitle
	
	\begin{abstract}
		This document addresses aspects of the fine-structure constant not discussed in detail in Document 011. The focus is on providing a comprehensive justification for why and how $\alpha = 1$ can be set (Heaviside-Lorentz convention), the physical consequences of different unit systems, and the historical and practical implications of redefining electromagnetic units.
		
		\textbf{For T0-specific derivations} (characteristic energy $E_0$, geometric parameter $\xi$, T0 formula $\alpha = \xi(E_0/1\text{MeV})^2$) see Document 011.
	\end{abstract}
	
	\tableofcontents
	\newpage
	
	% ============================================================================
	\section{Introduction and Reference to Document 011}
	
	\subsection{Delimitation from Document 011}
	
	\textbf{Document 011} covers in detail:
	\begin{itemize}
		\item T0 derivation: $\alpha = \xi(E_0/1\text{MeV})^2$
		\item Characteristic energy: $E_0 = \sqrt{m_e \cdot m_\mu} = 7.398$ MeV
		\item Geometric parameter: $\xi = \frac{4}{3} \times 10^{-4}$
		\item Alternative formulations: with $\mu_0$, with $r_e/\lambda_C$, etc.
		\item Historical context (Sommerfeld)
		\item Natural units and energy as fundamental field
		\item Detailed dimensional analysis of all formulations
	\end{itemize}
	
	\textbf{This document (044)} focuses on:
	\begin{itemize}
		\item \textbf{Why} $\alpha = 1$ can be set (detailed justification)
		\item \textbf{How} different unit conventions work
		\item Consequences of redefining the Coulomb
		\item Heaviside-Lorentz vs. Gauss vs. SI units
		\item Practical aspects and historical development
		\item Fine's inequality vs. fine-structure constant (name confusion)
	\end{itemize}
	
	\subsection{Why Two Documents?}
	
	\textbf{Document 011:} T0 theory and physical derivations
	
	\textbf{Document 044:} Unit systems and conventions
	
	Both complement each other, with minimal overlap.
	
	% ============================================================================
	\section{Different Unit Conventions for $\alpha$}
	
	\subsection{Overview of Systems}
	
	The fine-structure constant can be expressed in different unit systems:
	
	\begin{table}[h]
		\centering
		\begin{tabular}{|l|c|c|}
			\hline
			\textbf{System} & \textbf{Formula} & \textbf{Value} \\
			\hline
			SI Standard & $\alpha = \frac{e^2}{4\pi\varepsilon_0\hbar c}$ & $\approx \frac{1}{137}$ \\
			Heaviside-Lorentz & $\alpha = \frac{e^2}{4\pi}$ (with $\hbar = c = 4\pi\varepsilon_0 = 1$) & $1$ or $\frac{1}{137}$ \\
			Gauss (cgs) & $\alpha = \frac{e^2}{\hbar c}$ & $\approx \frac{1}{137}$ \\
			\hline
		\end{tabular}
		\caption{Unit systems for $\alpha$}
	\end{table}
	
	\textbf{Important:} The numerical value depends on the convention, the \textit{physical} predictions do not!
	
	% ============================================================================
	\section[Heaviside-Lorentz Units in Detail]{Heaviside-Lorentz Units in Detail}
	
	\subsection{What are Heaviside-Lorentz Units?}
	
	The Heaviside-Lorentz system is a variant of natural units, specifically for electrodynamics:
	
	\begin{equation}
		\boxed{\hbar = c = 4\pi\varepsilon_0 = 1}
	\end{equation}
	
	\textbf{Consequences:}
	\begin{itemize}
		\item Electromagnetic equations become more symmetric
		\item The factor $4\pi$ disappears from many formulas
		\item Elementary charge is redefined
	\end{itemize}
	
	\subsection[Why 4 pi epsilon 0 = 1]{Why $4\pi\varepsilon_0 = 1$?}
	
	In SI units, $4\pi$ appears in many electromagnetic formulas:
	\begin{align}
		\vec{E} &= \frac{1}{4\pi\varepsilon_0} \frac{q}{r^2} \hat{r} \quad \text{(Coulomb's law)} \\
		\nabla \cdot \vec{E} &= \frac{\rho}{\varepsilon_0} \quad \text{(Maxwell's equation)}
	\end{align}
	
	With $4\pi\varepsilon_0 = 1$, these become:
	\begin{align}
		\vec{E} &= \frac{q}{r^2} \hat{r} \\
		\nabla \cdot \vec{E} &= 4\pi\rho
	\end{align}
	
	The factor $4\pi$ moves from Coulomb's law to Poisson's equation!
	
	\subsection{Fine-Structure Constant in Heaviside-Lorentz}
	
	\textbf{Starting point (SI):}
	\begin{equation}
		\alpha = \frac{e^2}{4\pi\varepsilon_0\hbar c}
	\end{equation}
	
	\textbf{With $\hbar = c = 4\pi\varepsilon_0 = 1$:}
	\begin{equation}
		\alpha = \frac{e^2}{1 \cdot 1 \cdot 1} = e^2
	\end{equation}
	
	\textbf{Now the crucial question:} What value does $e$ have in this system?
	
	% ============================================================================
	\section[Two Variants of Heaviside-Lorentz]{Two Variants of Heaviside-Lorentz}
	
	\subsection[Variant A]{Variant A: Normalize $e$ so that $\alpha = 1$}
	
	\textbf{Approach:} We define the unit of charge such that $\alpha = 1$.
	
	Since $\alpha = e^2$ in HL units:
	\begin{equation}
		e^2 = 1 \quad \Rightarrow \quad e = 1
	\end{equation}
	
	\textbf{Physical meaning:}
	\begin{itemize}
		\item Elementary charge becomes a \textit{dimensionless unit}
		\item Electromagnetic coupling is ''normalized''
		\item Charge is measured in units of $\sqrt{\hbar c}$
	\end{itemize}
	
	\textbf{What changes?}
	
	The elementary charge gets a new numerical value:
	\begin{equation}
		e_{\text{HL}} = \sqrt{4\pi\varepsilon_0\hbar c} \quad \text{(expressed in SI units)}
	\end{equation}
	
	Numerically:
	\begin{align}
		e_{\text{HL}} &= \sqrt{4\pi \times 8.854 \times 10^{-12} \times 1.055 \times 10^{-34} \times 3 \times 10^8} \\
		&\approx 5.29 \times 10^{-19} \text{ (new charge unit)}
	\end{align}
	
	This is about $\sqrt{137} \times e_{\text{SI}}$!
	
	\subsection{Variant B: Keep $e$, $\alpha \approx 1/137$}
	
	\textbf{Approach:} The elementary charge retains its ''natural'' value.
	
	In this case:
	\begin{equation}
		\alpha = e^2 \approx \frac{1}{137}
	\end{equation}
	
	because $e$ in these units has the value $\approx 1/\sqrt{137}$.
	
	\textbf{Physical meaning:}
	\begin{itemize}
		\item Charge retains physical meaning
		\item $\alpha$ remains $\approx 1/137$
		\item Only the mathematical form simplifies
	\end{itemize}
	
	\subsection{Which variant is used?}
	
	\textbf{In practice:}
	\begin{itemize}
		\item \textbf{T0 theory:} Sets **all** constants = 1 ($c = \hbar = \alpha = G = 1$)
		\item \textbf{Theoretical high-energy physics:} Often $\hbar = c = 1$, sometimes also $\alpha = 1$
		\item \textbf{Numerical calculations:} Often $\hbar = c = 1$, but $\alpha \approx 1/137$
		\item \textbf{Experimental physics:} Almost always SI units (all constants have numerical values)
	\end{itemize}
	
	\textbf{T0 convention:}
	\begin{itemize}
		\item In T0 calculations: $c = \hbar = \alpha = G = 1$ (maximum simplification)
		\item Only free parameter: $\xi = \frac{4}{3} \times 10^{-4}$
		\item When comparing with experiments: SI values ($c = 3 \times 10^8$ m/s, $\alpha \approx 1/137$, etc.)
		\item Both describe the same physics!
	\end{itemize}
	
	% ============================================================================
	\section{Reconstruction of SI Values from T0}
	\label{sec:si_reconstruction}
	
	\subsection{The Central Principle}
	
	\textbf{Important insight:} Although T0 sets all constants to 1, the SI values can be reconstructed!
	
	\begin{tcolorbox}[colback=green!5!white,colframe=green!75!black,title=T0 Reconstruction]
		\textbf{In T0 calculations:}
		\begin{itemize}
			\item All constants = 1: $c = \hbar = \alpha = G = 1$
			\item Only free parameter: $\xi = \frac{4}{3} \times 10^{-4}$
			\item Formulas maximally simplified
		\end{itemize}
		
		\textbf{Reconstruction of SI values:}
		\begin{itemize}
			\item Fine-structure constant: $\alpha_{\text{SI}} = \xi(E_0/1\text{MeV})^2 \approx 1/137$
			\item Gravitational constant: $G_{\text{SI}} = \frac{\xi^2}{4m_e} \times$ factors
			\item All other constants: derivable from $\xi$
		\end{itemize}
	\end{tcolorbox}
	
	\subsection{Example: Fine-Structure Constant}
	
	\textbf{In T0 units:}
	\begin{equation}
		\alpha = 1
	\end{equation}
	
	\textbf{Reconstruction of SI value:}
	\begin{equation}
		\alpha_{\text{SI}} = \xi \cdot \left(\frac{E_0}{1\,\text{MeV}}\right)^2
	\end{equation}
	
	With $\xi = \frac{4}{3} \times 10^{-4}$ and $E_0 = 7.398$ MeV:
	\begin{align}
		\alpha_{\text{SI}} &= 1.3333 \times 10^{-4} \times (7.398)^2 \\
		&= 1.3333 \times 10^{-4} \times 54.73 \\
		&= 7.297 \times 10^{-3} \\
		&= \frac{1}{137.04}
	\end{align}
	
	Experimental: $\alpha_{\text{exp}} = \frac{1}{137.036}$
	
	Agreement: 0.03\% ✓
	
	\subsection{Example: Gravitational Constant}
	
	\textbf{In T0 units:}
	\begin{equation}
		G = 1
	\end{equation}
	
	\textbf{Reconstruction of SI value:}
	\begin{equation}
		G_{\text{SI}} = \frac{\xi^2}{4m_e} \times C_{\text{dim}} \times C_{\text{conv}}
	\end{equation}
	
	where:
	\begin{itemize}
		\item $C_{\text{dim}}$ = Dimension conversion (natural units → SI)
		\item $C_{\text{conv}}$ = Conversion factors (eV → J, etc.)
	\end{itemize}
	
	Detailed derivation see Document 012 (Gravitation).
	
	\subsection{Why does this work?}
	
	\textbf{Key:} $\xi$ is dimensionless and universal!
	
	\begin{enumerate}
		\item In T0: $\xi$ determines all coupling strengths
		
		\item In SI: $\xi$ together with characteristic energies ($E_0$, masses) reconstructs all constants
		
		\item Physical predictions: identical in both systems!
		
		\item Only the mathematical representation differs
	\end{enumerate}
	
	\subsection{Comparison Table}
	
	\begin{table}[h]
		\centering
		\begin{tabular}{|l|c|c|l|}
			\hline
			\textbf{Constant} & \textbf{T0} & \textbf{SI} & \textbf{Reconstruction} \\
			\hline
			$c$ & $1$ & $3 \times 10^8$ m/s & Convention \\
			$\hbar$ & $1$ & $1.055 \times 10^{-34}$ J$\cdot$s & Convention \\
			$\alpha$ & $1$ & $\approx 1/137$ & $\xi(E_0/1\text{MeV})^2$ \\
			$G$ & $1$ & $6.67 \times 10^{-11}$ m³/(kg$\cdot$s²) & $\xi^2/(4m_e) \times$ factors \\
			$\xi$ & $\frac{4}{3} \times 10^{-4}$ & $\frac{4}{3} \times 10^{-4}$ & \textbf{Same!} \\
			\hline
		\end{tabular}
		\caption{T0 vs. SI - Reconstruction of constants}
	\end{table}
	
	\subsection{Important Conclusion}
	
	\begin{itemize}
		\item \textbf{T0 is not new physics}, but a reparametrization
		
		\item Instead of many constants ($c$, $\hbar$, $\alpha$, $G$, ...) only \textbf{one parameter} $\xi$
		
		\item All SI values reconstructable from $\xi$ and energy scales
		
		\item Advantage: Formulas simpler, physical relationships clearer
		
		\item Disadvantage: Conversion to SI needed for experiments
	\end{itemize}
	
	% ============================================================================
	\section{Why can $\alpha = 1$ be set?}
	
	\subsection{Fundamental Insight}
	
	\begin{tcolorbox}[colback=blue!5!white,colframe=blue!75!black,title=Core Statement]
		The fine-structure constant $\alpha$ is a \textbf{dimensionless number}. Its numerical value is \textbf{convention-dependent}, not fundamental!
		
		One can set $\alpha = 1$ by redefining the \textbf{unit of charge} accordingly.
	\end{tcolorbox}
	
	\subsection{Step-by-Step Justification}
	
	\textbf{Step 1: What is a convention?}
	
	SI units are historically evolved definitions:
	\begin{itemize}
		\item 1 meter = originally 1/10,000,000 of the Earth's meridian
		\item 1 second = originally 1/86,400 of a solar day
		\item 1 Coulomb = defined via Ampere and force between currents
	\end{itemize}
	
	None of these is ''fundamental''!
	
	\textbf{Step 2: $\alpha$ in SI}
	
	In SI units:
	\begin{equation}
		\alpha = \frac{e^2}{4\pi\varepsilon_0\hbar c} \approx \frac{1}{137}
	\end{equation}
	
	The value $1/137$ follows from:
	\begin{itemize}
		\item How we defined the Coulomb (historically)
		\item How we defined $\varepsilon_0$ (via $\mu_0$ and $c$)
	\end{itemize}
	
	\textbf{Step 3: Redefinition}
	
	We can say: ''From now on, the elementary charge is no longer $1.602 \times 10^{-19}$ C, but $e = \sqrt{4\pi\varepsilon_0\hbar c}$.''
	
	Then automatically:
	\begin{equation}
		\alpha = \frac{(\sqrt{4\pi\varepsilon_0\hbar c})^2}{4\pi\varepsilon_0\hbar c} = 1
	\end{equation}
	
	\textbf{Step 4: Physical Consequences}
	
	\begin{itemize}
		\item \textbf{No physical predictions change!}
		\item Only the \textit{numbers} in formulas change
		\item All ratios remain the same
		\item All experiments yield the same results
	\end{itemize}
	
	\subsection{Analogy: Temperature Scales}
	
	\textbf{Celsius:} Water freezes at 0°C
	
	\textbf{Fahrenheit:} Water freezes at 32°F
	
	\textbf{Kelvin:} Water freezes at 273.15 K
	
	Is any of these scales ''correct''? No! They are conventions.
	
	Similarly, $\alpha = 1/137$ (SI) vs. $\alpha = 1$ (HL) is just a choice of convention!
	
	% ============================================================================
	\section{Consequences of Redefining the Coulomb}
	
	\subsection{What does it mean to redefine elementary charge?}
	
	If $e$ is redefined such that $\alpha = 1$:
	
	\textbf{Old definition (SI):}
	\begin{equation}
		e = 1.602 \times 10^{-19} \text{ C}
	\end{equation}
	
	\textbf{New definition (HL with $\alpha = 1$):}
	\begin{equation}
		e = 1 \quad \text{(dimensionless in natural units)}
	\end{equation}
	
	or expressed in SI units:
	\begin{equation}
		e_{\text{new}} = \sqrt{4\pi\varepsilon_0\hbar c} \approx 5.29 \times 10^{-19} \text{ (new charge unit)}
	\end{equation}
	
	\subsection{Effects on Electromagnetic Quantities}
	
	\subsubsection{Electric Current (Ampere)}
	
	Since $1 \text{ A} = 1 \text{ C/s}$:
	\begin{equation}
		1 \text{ A}_{\text{new}} = \frac{e_{\text{new}}}{1 \text{ s}} = \sqrt{137} \times 1 \text{ A}_{\text{old}}
	\end{equation}
	
	\subsubsection{Electric Voltage (Volt)}
	
	$1 \text{ V} = 1 \text{ J/C}$:
	\begin{equation}
		1 \text{ V}_{\text{new}} = \frac{1 \text{ J}}{e_{\text{new}}} = \frac{1}{\sqrt{137}} \times 1 \text{ V}_{\text{old}}
	\end{equation}
	
	\subsubsection{Capacitance (Farad)}
	
	\begin{equation}
		1 \text{ F}_{\text{new}} = \frac{e_{\text{new}}}{1 \text{ V}_{\text{new}}} = 137 \times 1 \text{ F}_{\text{old}}
	\end{equation}
	
	\subsection{Are these changes ''real''?}
	
	\textbf{No!} They are only conversion factors, like Celsius → Fahrenheit.
	
	\textbf{All physical ratios remain identical:}
	\begin{itemize}
		\item Capacitance of a capacitor / distance: same
		\item Force between charges / distance²: same
		\item All experiments: same results
	\end{itemize}
	
	Only the \textit{numerical values} we calculate with change!
	
	% ============================================================================
	\section{Practical Impacts on Everyday Calculations}
	\label{sec:practical_calculations}
	
	\subsection{Motivation}
	
	\textbf{Question:} If we set $\alpha = 1$, what does that mean for ordinary electrical calculations with volts, amperes, resistance, capacitance?
	
	\textbf{Answer:} All formulas change, but the \textit{physical results} remain identical!
	
	\subsection{Example 1: Ohm's Law}
	
	\subsubsection{In SI Units (Standard)}
	
	\begin{equation}
		U = R \cdot I
	\end{equation}
	
	\textbf{Numerical example:}
	\begin{itemize}
		\item Resistance: $R = 100$ $\Omega$
		\item Current: $I = 2$ A
		\item Voltage: $U = 100 \times 2 = 200$ V
	\end{itemize}
	
	\subsubsection{In Heaviside-Lorentz with $\alpha = 1$}
	
	The formula remains $U = R \cdot I$, but the \textit{numerical values} change!
	
	\textbf{Unit conversion:}
	\begin{align}
		1 \text{ A}_{\text{new}} &= \sqrt{137} \times 1 \text{ A}_{\text{old}} \approx 11.7 \text{ A}_{\text{old}} \\
		1 \text{ V}_{\text{new}} &= \frac{1}{\sqrt{137}} \times 1 \text{ V}_{\text{old}} \approx 0.085 \text{ V}_{\text{old}} \\
		1 \text{ }\Omega_{\text{new}} &= \frac{1}{137} \times 1 \text{ }\Omega_{\text{old}}
	\end{align}
	
	\textbf{Same circuit in new units:}
	\begin{align}
		R_{\text{new}} &= 100 \times \frac{1}{137} \approx 0.73 \text{ }\Omega_{\text{new}} \\
		I_{\text{new}} &= 2 \times \sqrt{137} \approx 23.4 \text{ A}_{\text{new}} \\
		U_{\text{new}} &= 0.73 \times 23.4 = 17.1 \text{ V}_{\text{new}}
	\end{align}
	
	\textbf{Conversion back to SI:}
	\begin{equation}
		U_{\text{new}} = 17.1 \times 0.085 \text{ V}_{\text{old}} = 200 \text{ V} \quad \checkmark
	\end{equation}
	
	Identical result!
	
	\subsection{Example 2: Power of a Light Bulb}
	
	\subsubsection{In SI Units}
	
	\begin{equation}
		P = U \cdot I = \frac{U^2}{R}
	\end{equation}
	
	\textbf{Light bulb:} 60 W at 230 V
	
	\begin{align}
		R &= \frac{U^2}{P} = \frac{(230)^2}{60} = 882 \text{ }\Omega \\
		I &= \frac{P}{U} = \frac{60}{230} = 0.26 \text{ A}
	\end{align}
	
	\subsubsection{In Heaviside-Lorentz with $\alpha = 1$}
	
	\textbf{Power:} $1 \text{ W}_{\text{new}} = 1 \text{ W}_{\text{old}}$ (energy/time doesn't change in HL!)
	
	\begin{align}
		U_{\text{new}} &= 230 \times \frac{1}{\sqrt{137}} = 19.6 \text{ V}_{\text{new}} \\
		R_{\text{new}} &= 882 \times \frac{1}{137} = 6.44 \text{ }\Omega_{\text{new}} \\
		I_{\text{new}} &= \frac{P}{U_{\text{new}}} = \frac{60}{19.6} = 3.06 \text{ A}_{\text{new}}
	\end{align}
	
	\textbf{Verification:}
	\begin{equation}
		P = U_{\text{new}} \cdot I_{\text{new}} = 19.6 \times 3.06 = 60 \text{ W} \quad \checkmark
	\end{equation}
	
	\subsection{Example 3: Charging a Capacitor}
	
	\subsubsection{In SI Units}
	
	\begin{equation}
		Q = C \cdot U
	\end{equation}
	
	\textbf{Capacitor:} $C = 100$ $\mu$F at $U = 12$ V
	
	\begin{equation}
		Q = 100 \times 10^{-6} \times 12 = 1.2 \times 10^{-3} \text{ C}
	\end{equation}
	
	\textbf{Stored energy:}
	\begin{equation}
		E = \frac{1}{2} C U^2 = \frac{1}{2} \times 100 \times 10^{-6} \times 144 = 7.2 \times 10^{-3} \text{ J}
	\end{equation}
	
	\subsubsection{In Heaviside-Lorentz with $\alpha = 1$}
	
	\textbf{Conversion:}
	\begin{align}
		1 \text{ F}_{\text{new}} &= 137 \times 1 \text{ F}_{\text{old}} \\
		1 \text{ C}_{\text{new}} &= \sqrt{137} \times 1 \text{ C}_{\text{old}}
	\end{align}
	
	\begin{align}
		C_{\text{new}} &= 100 \times 10^{-6} \times 137 = 0.0137 \text{ F}_{\text{new}} \\
		U_{\text{new}} &= 12 \times \frac{1}{\sqrt{137}} = 1.025 \text{ V}_{\text{new}} \\
		Q_{\text{new}} &= 0.0137 \times 1.025 = 0.014 \text{ C}_{\text{new}}
	\end{align}
	
	\textbf{Conversion back:}
	\begin{equation}
		Q_{\text{new}} = 0.014 \times \frac{1}{\sqrt{137}} = 1.2 \times 10^{-3} \text{ C}_{\text{old}} \quad \checkmark
	\end{equation}
	
	\textbf{Energy:}
	\begin{equation}
		E_{\text{new}} = \frac{1}{2} \times 0.0137 \times (1.025)^2 = 7.2 \times 10^{-3} \text{ J} \quad \checkmark
	\end{equation}
	
	Energy is the same in all systems!
	
	\subsection{Example 4: RC Time Constant}
	
	\subsubsection{In SI Units}
	
	\begin{equation}
		\tau = R \cdot C
	\end{equation}
	
	\textbf{Circuit:} $R = 1$ k$\Omega$, $C = 10$ $\mu$F
	
	\begin{equation}
		\tau = 1000 \times 10 \times 10^{-6} = 0.01 \text{ s} = 10 \text{ ms}
	\end{equation}
	
	\subsubsection{In Heaviside-Lorentz with $\alpha = 1$}
	
	\begin{align}
		R_{\text{new}} &= 1000 \times \frac{1}{137} = 7.3 \text{ }\Omega_{\text{new}} \\
		C_{\text{new}} &= 10 \times 10^{-6} \times 137 = 1.37 \times 10^{-3} \text{ F}_{\text{new}}
	\end{align}
	
	\begin{equation}
		\tau_{\text{new}} = 7.3 \times 1.37 \times 10^{-3} = 0.01 \text{ s} = 10 \text{ ms} \quad \checkmark
	\end{equation}
	
	\textbf{Time remains the same!} This is important: Physical timescales do not change!
	
	\subsection{Summary of Practical Calculations}
	
	\begin{table}[h]
		\centering
		\begin{tabular}{|l|c|c|c|}
			\hline
			\textbf{Quantity} & \textbf{SI} & \textbf{HL Factor} & \textbf{HL ($\alpha=1$)} \\
			\hline
			Charge (Q) & C & $\sqrt{137}$ & $\sqrt{137}$ C \\
			Current (I) & A & $\sqrt{137}$ & $\sqrt{137}$ A \\
			Voltage (U) & V & $1/\sqrt{137}$ & V$/\sqrt{137}$ \\
			Resistance (R) & $\Omega$ & $1/137$ & $\Omega/137$ \\
			Capacitance (C) & F & $137$ & $137$ F \\
			Power (P) & W & $1$ & W (unchanged!) \\
			Energy (E) & J & $1$ & J (unchanged!) \\
			Time ($\tau$) & s & $1$ & s (unchanged!) \\
			\hline
		\end{tabular}
		\caption{Conversion factors SI → HL with $\alpha = 1$}
	\end{table}
	
	\subsection{Important Insights}
	
	\begin{tcolorbox}[colback=green!5!white,colframe=green!75!black,title=Core Statement]
		\textbf{What changes:}
		\begin{itemize}
			\item Numerical values for charge, current, voltage, resistance, capacitance
		\end{itemize}
		
		\textbf{What does NOT change:}
		\begin{itemize}
			\item Energy
			\item Power
			\item Time
			\item All physical ratios
			\item All experimental results
		\end{itemize}
		
		\textbf{Conclusion:} It's just a conversion, like meters ↔ feet!
	\end{tcolorbox}
	
	\subsection{Why does nobody use $\alpha = 1$ in practice?}
	
	\textbf{Reasons:}
	\begin{enumerate}
		\item \textbf{Measuring devices:} All voltmeters, ammeters, etc. are calibrated in SI
		
		\item \textbf{Standards:} Worldwide accepted SI definitions
		
		\item \textbf{Intuition:} Engineers know typical values in SI
		\begin{itemize}
			\item Household: 230 V, not 1.96 V$_{\text{new}}$
			\item USB: 5 V, not 0.43 V$_{\text{new}}$
		\end{itemize}
		
		\item \textbf{Conversion is laborious:} $\sqrt{137}$ factors everywhere
		
		\item \textbf{No advantage for practitioners:} Simplification only visible in theoretical formulas
	\end{enumerate}
	
	\textbf{But:} For theoretical calculations (QED, Feynman diagrams), $\alpha = 1$ is often very helpful!
	
	% ============================================================================
	\section{Practical Aspects of Different Systems}
	
	\subsection{Advantages and Disadvantages: SI Units}
	
	\textbf{Advantages:}
	\begin{itemize}
		\item Worldwide standardized
		\item Directly usable for experiments
		\item All measuring devices calibrated in SI
		\item Clear separation of length/time/mass/charge
	\end{itemize}
	
	\textbf{Disadvantages:}
	\begin{itemize}
		\item Many constants in formulas ($4\pi\varepsilon_0$, $\hbar$, $c$)
		\item Physical relationships obscured
		\item Dimensions unwieldy
	\end{itemize}
	
	\subsection{Advantages and Disadvantages: Heaviside-Lorentz with $\alpha = 1$}
	
	\textbf{Advantages:}
	\begin{itemize}
		\item Maximally simplified formulas
		\item Electromagnetic symmetry visible
		\item Theoretical calculations simpler
		\item QED Feynman diagrams more elegant
	\end{itemize}
	
	\textbf{Disadvantages:}
	\begin{itemize}
		\item No direct connection to experiments
		\item Conversion to SI laborious
		\item Unfamiliar for practitioners
		\item Physical ''size'' of $e$ unclear
	\end{itemize}
	
	\subsection{Advantages and Disadvantages: Natural Units with $\alpha \approx 1/137$}
	
	\textbf{Advantages:}
	\begin{itemize}
		\item Simplified formulas ($\hbar = c = 1$)
		\item $\alpha$ retains physical meaning
		\item Good compromise theory/practice
		\item Numerically: $\alpha \ll 1$ → perturbation theory
	\end{itemize}
	
	\textbf{Disadvantages:}
	\begin{itemize}
		\item Still need conversion to SI
		\item Factor $4\pi$ remains in some formulas
	\end{itemize}
	
	\textbf{This is the preferred convention in modern particle physics!}
	
	% ============================================================================
	\section{Historical Development}
	
	\subsection{Gauss Units (cgs)}
	
	\textbf{19th century:} Gauss system (centimeter-gram-second)
	
	\begin{equation}
		\alpha = \frac{e^2}{\hbar c}
	\end{equation}
	
	No $4\pi\varepsilon_0$, because $\varepsilon_0 = 1$ by definition in cgs!
	
	\subsection{SI Units (MKSA)}
	
	\textbf{20th century:} SI system (meter-kilogram-second-ampere)
	
	\begin{equation}
		\alpha = \frac{e^2}{4\pi\varepsilon_0\hbar c}
	\end{equation}
	
	The $4\pi\varepsilon_0$ appears because the SI ampere is defined via force.
	
	\subsection{Heaviside-Lorentz}
	
	\textbf{Theoretical physics:} Heaviside-Lorentz simplifies Maxwell's equations
	
	\begin{equation}
		\nabla \times \vec{E} = -\frac{\partial \vec{B}}{\partial t}, \quad \nabla \times \vec{B} = \vec{j} + \frac{\partial \vec{E}}{\partial t}
	\end{equation}
	
	Symmetric! (In SI, $\mu_0$ and $\varepsilon_0$ appear asymmetrically)
	
	\subsection{Natural Units}
	
	\textbf{Modern high-energy physics:} $\hbar = c = 1$, but different conventions for $\alpha$
	
	% ============================================================================
	\section{Fine's Inequality vs. Fine-Structure Constant}
	
	\subsection{Frequent Confusion}
	
	\textbf{Warning:} \textit{Fine's inequality} and the \textit{fine-structure constant} are completely different concepts!
	
	\subsection{Fine's Inequality}
	
	\textbf{What it is:}
	\begin{itemize}
		\item A form of Bell's inequality
		\item Test for local hidden variables
		\item Quantum entanglement vs. classical correlations
	\end{itemize}
	
	\textbf{Mathematically:}
	\begin{equation}
		|C(\alpha, \beta) - C(\alpha, \beta')| + |C(\alpha', \beta) + C(\alpha', \beta')| \leq 2
	\end{equation}
	
	where $C$ are correlation functions.
	
	\textbf{Physically:} Shows non-locality of quantum mechanics
	
	\subsection{Fine-Structure Constant}
	
	\textbf{What it is:}
	\begin{itemize}
		\item Fundamental physical constant
		\item Strength of the electromagnetic interaction
		\item Dimensionless, $\alpha \approx 1/137$
	\end{itemize}
	
	\textbf{Mathematically:}
	\begin{equation}
		\alpha = \frac{e^2}{4\pi\varepsilon_0\hbar c}
	\end{equation}
	
	\textbf{Physically:} Determines EM coupling strength
	
	\subsection{No Connection!}
	
	The similarity in name is \textbf{pure coincidence}. The two concepts have nothing to do with each other!
	
	% ============================================================================
	\section{Summary}
	
	\subsection{Core Statements}
	
	\begin{enumerate}
		\item $\alpha$ is dimensionless → numerical value is convention-dependent
		
		\item One \textbf{can} set $\alpha = 1$ by redefining the unit of charge
		
		\item \textbf{T0 theory:} Sets **all** constants = 1: $c = \hbar = \alpha = G = 1$
		
		\item Only free parameter in T0: $\xi = \frac{4}{3} \times 10^{-4}$
		
		\item \textbf{No} physical predictions change!
		
		\item Only numerical values in formulas differ
		
		\item When comparing with experiments: SI values ($\alpha \approx 1/137$, $c = 3 \times 10^8$ m/s, etc.)
	\end{enumerate}
	
	\subsection{For Further Details See Document 011}
	
	\begin{itemize}
		\item T0 derivation of $\alpha$
		\item Characteristic energy $E_0$
		\item Geometric parameter $\xi$
		\item Experimental verification
		\item Detailed dimensional analysis
		\item Historical context (Sommerfeld)
	\end{itemize}
	
	% ============================================================================
	\appendix
	
	\section{Conversion Table: SI ↔ Heaviside-Lorentz}
	
	\begin{table}[h]
		\centering
		\begin{tabular}{|l|c|c|}
			\hline
			\textbf{Quantity} & \textbf{SI} & \textbf{HL ($\alpha = 1$)} \\
			\hline
			Elementary charge & $e = 1.602 \times 10^{-19}$ C & $e = 1$ \\
			Fine-structure constant & $\alpha \approx 1/137$ & $\alpha = 1$ \\
			$4\pi\varepsilon_0$ & $1.11 \times 10^{-10}$ F/m & $1$ \\
			$\hbar$ & $1.055 \times 10^{-34}$ J$\cdot$s & $1$ \\
			$c$ & $3 \times 10^8$ m/s & $1$ \\
			\hline
		\end{tabular}
		\caption{Conversion table SI to HL}
	\end{table}
	
	\section{Sample Calculation: Coulomb's Law}
	
	\subsection{In SI Units}
	
	\begin{equation}
		F = \frac{1}{4\pi\varepsilon_0} \frac{e^2}{r^2}
	\end{equation}
	
	Numerically for $r = 1$ Å = $10^{-10}$ m:
	\begin{align}
		F &= \frac{1}{4\pi \times 8.854 \times 10^{-12}} \frac{(1.602 \times 10^{-19})^2}{(10^{-10})^2} \\
		&\approx 2.3 \times 10^{-8} \text{ N}
	\end{align}
	
	\subsection{In HL Units ($\alpha = 1$)}
	
	\begin{equation}
		F = \frac{e^2}{r^2} = \frac{1}{r^2}
	\end{equation}
	
	With $r$ in natural units: $r = 1$ Å $= 0.197 \times 10^6$ eV$^{-1}$
	
	\begin{equation}
		F = \frac{1}{(0.197 \times 10^6)^2} \approx 2.6 \times 10^{-14} \text{ eV}^2
	\end{equation}
	
	Conversion to SI: $1 \text{ eV}^2 \approx 9 \times 10^5$ N
	
	\begin{equation}
		F \approx 2.3 \times 10^{-8} \text{ N}
	\end{equation}
	
	\textbf{Identical!} Only the intermediate steps look different.
	
\end{document}