\documentclass[12pt,a4paper]{article}

% Standardized preamble - 051_dirac_En.pdf
% ==============================================================================
% T0 Theory: Shared GERMAN Preamble – Optimized for eBook/Book
% Version: 2.0 – Final 2026 (LuaLaTeX only) – DEUTSCH korrigiert
% Author: Johann Pascher
% Date: Januar 2026
% ==============================================================================
%
% WICHTIG: Compile EXCLUSIVELY with LuaLaTeX!
% In TeXstudio: Options → Configure TeXstudio → Build → Default Compiler → LuaLaTeX
%
% Required Fonts (install once):
% - Inter: https://fonts.google.com/specimen/Inter
% - JetBrains Mono: https://www.jetbrains.com/lp/mono/
% - Libertinus Math: https://github.com/libertinus-fonts/libertinus
% ==============================================================================

% === KAPITEL 1: GRUNDLEGENDE PAKETE (müssen ZUERST kommen) ===
\RequirePackage{fontspec}
\RequirePackage{unicode-math}

% === KAPITEL 2: SPRACHE (DEUTSCH mit voller Silbentrennung) ===
\usepackage[ngerman]{babel}
\usepackage{microtype}                    % WICHTIG für bessere Silbentrennung!

% Typographie-Einstellungen für besseren deutschen Umbruch
\frenchspacing                     % Korrekte deutsche Abstände nach Satzzeichen
\emergencystretch=3em              % Erlaubt mehr Dehnung bei schwierigen Zeilen
\tolerance=2500                    % Höhere Toleranz für Zeilenumbrüche
\hbadness=10000                    % Unterdrückt "underfull hbox" Warnungen
\hfuzz=2pt                         % Erlaubt minimalen Overfull
\pretolerance=150                  % Bessere Worttrennung

% Bessere Seitenumbrüche verhindern
\clubpenalty=10000           % Keine "Schusterjungen"
\widowpenalty=10000          % Keine "Hurenkinder"  
\displaywidowpenalty=10000   % Auch bei Formeln
\brokenpenalty=10000         % Keine getrennten Wörter über Seiten

% Explizite Trennungen für lange deutsche Wörter
\hyphenation{Fun-da-men-tal Frak-tal-Ge-o-me-trisch Fel-the-o-rie Me-tho-do-lo-gisch}
\hyphenation{Re-vi-si-o-nis-mus Quan-ti-sie-rung U-ni-fi-ka-ti-on Ef-fek-tiv}
\hyphenation{Re-nor-mier-bar-keit Sin-gu-la-ri-tä-ten Kon-zi-li-an-tis-mus}
\hyphenation{E-mer-genz Phä-no-me-no-lo-gisch Do-ku-men-ta-ti-on Ana-ly-se}
\hyphenation{Gra-vi-ta-ti-on Quan-ten-me-cha-nik Do-gma-tis-mus Kon-se-quent}
\hyphenation{Par-al-le-lis-mus Im-ple-men-tie-rung Per-tur-ba-ti-o-nen}
\hyphenation{Ge-o-me-trisch Ar-te-fakt In-ko-mpa-ti-bi-li-tät Kon-struk-tiv}
\hyphenation{Frak-tal Di-men-si-ons-los Un-ter-such-ung Be-schrei-bung}
\hyphenation{In-ter-pre-ta-ti-on Phe-no-me-no-lo-gisch Ma-the-ma-tisch}
\hyphenation{Phi-lo-so-phisch Le-gi-ti-ma-ti-on An-wen-dung Ab-lei-tung}
\hyphenation{Ver-ein-heit-li-chung An-na-hme Vor-stel-lung Er-war-tung}
\hyphenation{Sym-me-trie-ern-wei-te-rung Ge-samt-bild Her-aus-fo-rde-rung}
\hyphenation{Wech-sel-wir-kung Ma-te-ri-al An-satz Per-spek-ti-ve Vor-ge-hen}

% === KAPITEL 3: SCHRIFTEN (mit deutschen Ligaturen) ===
\setmainfont{Inter}[
Scale=1.02,
UprightFont=*-Regular,
BoldFont=*-Bold,
ItalicFont=*-Italic,
BoldItalicFont=*-BoldItalic,
Ligatures=TeX,           % WICHTIG für deutsche Typografie
Language=German          % Explizite Sprachunterstützung
]
\setsansfont{Inter}[
Scale=MatchLowercase,
Ligatures=TeX,
Language=German
]
\setmonofont{JetBrains Mono}[
Scale=0.95,
Language=German
]

% Math Font (simple & stable) – MUSS NACH der Sprachdefinition kommen
% WICHTIG: Libertinus Math für korrekte \underbrace-Darstellung!
\setmathfont{Libertinus Math}[Scale=1.0]

% === KAPITEL 4: MATHEMATIK-PAKETE (in STRENGER Reihenfolge!) ===
% WICHTIG: mathtools muss VOR unicode-math für manche Befehle!
\usepackage{mathtools}           % ZUERST mathtools!

% Dann der Rest
\usepackage{amsmath, amsfonts, amsthm}

% SIUNITX MUSS VOR physics geladen werden!
\usepackage{siunitx}
\sisetup{
	locale=DE,                    % DEUTSCHE Einstellungen für SI-Einheiten!
	group-separator={.},          % Tausendertrennzeichen Punkt
	output-decimal-marker={,},    % Dezimaltrennzeichen Komma
	per-mode=symbol,
	separate-uncertainty=true
}

% Eigene SI-Einheiten für Narrative/Bücher
\DeclareSIUnit\gigalightyear{Gly}
\DeclareSIUnit\mev{MeV}

% physics – MUSS NACH siunitx und mathtools geladen werden
\usepackage{physics}

% === KAPITEL 5: ERGÄNZUNGEN aus pdflatex-Best Practices ===
\usepackage{colortbl}        % Farbige Tabellen (ESSENTIELL!)
\usepackage{placeins}        % Float-Kontrolle: \FloatBarrier
\usepackage{subcaption}      % Unterabbildungen
\usepackage{xurl}            % Bessere URL-Umbrüche
% Hyphenation for URLs in bibliography
\def\UrlBreaks{\do\/\do-}

% === KAPITEL 6: SEITENGESTALTUNG =
\usepackage[paperwidth=8.25in, paperheight=11in, 
left=2.5cm, 
right=2.5cm, 
top=2.5cm, 
bottom=3.5cm,
bindingoffset=0.5cm]{geometry}
\setlength{\headheight}{15pt}
% Page Geometry – Buch-Optimierung
% =============================================================================
%\usepackage[paperwidth=8.25in, paperheight=11in,
%top=1.0in,
%bottom=1.2in,
%inner=1.0in,
%outer=0.75in,
%bindingoffset=0.75in,
%twoside]{geometry}
%\setlength{\headheight}{15pt}

% === KAPITEL 7: GRAFIKEN UND TABELLEN ===
\usepackage{graphicx}
\usepackage[table,xcdraw]{xcolor}
% T0 Markenfarben
\definecolor{gold}{RGB}{255,215,0}
\definecolor{blue}{rgb}{0,0,1}
\definecolor{boxgray}{RGB}{240,240,240}
\definecolor{deepblue}{RGB}{0,0,127}
\definecolor{deepgreen}{RGB}{0,127,0}
\definecolor{deepred}{RGB}{191,0,0}
\definecolor{t0blue}{RGB}{33,150,243}
\definecolor{t0green}{RGB}{76,175,80}
\definecolor{t0orange}{RGB}{255,152,0}
\definecolor{t0purple}{RGB}{156,39,176}
\definecolor{t0red}{RGB}{244,67,54}
\definecolor{t0yellow}{RGB}{255,204,0}
\usepackage{tikz}
\usetikzlibrary{arrows.meta,positioning,shapes.geometric,decorations.pathmorphing,patterns,shapes.arrows,intersections}
\usepackage{pgfplots}
\pgfplotsset{compat=1.18}
\usepackage{quantikz}
\usepackage[most]{tcolorbox}
\tcbuselibrary{breakable}

% === WICHTIG: Algorithm-Konflikt umgehen ===
% Option: algorithmic mit GROSSBUCHSTABEN
% Gemeinsame Box für Experimente
\newtcolorbox{experimentbox}[1][]{
	colback=green!5!white,
	colframe=t0green!80!black,
	fonttitle=\bfseries,
	title={{#1}},
	breakable
}

% Abstract-Fallback
\ifdefined\abstract\else
\newenvironment{abstract}{\section*{\abstractname}\itshape\small\par\bigskip}{\bigskip}
\fi

% === MAKROS SICHER NEU DEFINIEREN / ÜBERSCHREIBEN ===
% Definiere Makros OHNE doppelte Subskripte
\newcommand{\phipar}{\phi_{\mathrm{par}}}
%\newcommand{\xipar}{\xi_{\mathrm{par}}}
\newcommand{\Qphipar}{Q_{\phi_{\mathrm{par}}}}
\newcommand{\rphipar}{r_{\phi_{\mathrm{par}}}}
\newcommand{\logphipar}{\log_{\phi_{\mathrm{par}}}}
\newcommand{\CHSH}{\text{CHSH}}
\usepackage{booktabs}
\usepackage{array}
\usepackage{longtable}
\usepackage{float}
\usepackage{adjustbox}
\usepackage{rotating}
\usepackage{tabularx}
\usepackage{makecell}
\usepackage{multirow}

% === KAPITEL 8: DOKUMENTFORMATIERUNG ===
\usepackage{fancyhdr}
\renewcommand{\headrulewidth}{0.4pt}
\renewcommand{\footrulewidth}{0.4pt}
\usepackage{tocloft}

\usepackage{enumitem}
\setlist[itemize]{leftmargin=*, topsep=2pt, partopsep=0pt, parsep=2pt, itemsep=2pt}
\setlist[enumerate]{leftmargin=*, topsep=2pt, partopsep=0pt, parsep=2pt, itemsep=2pt}
\usepackage{setspace}
\usepackage{ragged2e}
\usepackage{multicol}

% === KAPITEL 9: CODE UND ALGORITHMEN ===
\usepackage{algorithm}
\usepackage{algorithmic}
\usepackage{listings}
\lstset{
	basicstyle=\ttfamily\footnotesize,
	breaklines=true,
	breakatwhitespace=true,
	columns=flexible,
	keepspaces=true,
	showstringspaces=false,
	frame=single,
	xleftmargin=0pt,
	xrightmargin=0pt,
	literate=              % Für deutsche Umlaute in Code-Listings
	{ä}{{\"a}}1 {ö}{{\"o}}1 {ü}{{\"u}}1 {ß}{{\ss}}1
	{Ä}{{\"A}}1 {Ö}{{\"O}}1 {Ü}{{\"U}}1
}
\usepackage{mdframed}

% === KAPITEL 10: ZUSÄTZLICHE PAKETE ===
\usepackage{pdflscape}
\usepackage{braket}
\usepackage{cancel}
\usepackage{caption}
\captionsetup{format=plain, labelfont=bf, justification=centering}
\usepackage{csquotes}
\usepackage{gensymb}
\usepackage{textcomp}
\usepackage{textgreek}
\usepackage{upgreek}
\usepackage{url}
\usepackage{slashed}
\usepackage{bm}

% === KAPITEL 11: HYPERREF (muss als VORLETZTES Paket kommen!) ===
\usepackage{hyperref}
\hypersetup{
	colorlinks=true,
	linkcolor=black,
	citecolor=black,
	urlcolor=black,
	breaklinks=true,           % WICHTIG für deutsche Umlaute in URLs!
	bookmarksnumbered=true,
	unicode=true,
	pdfencoding=auto,
	pdflang=de,                % PDF-Sprache auf Deutsch setzen
	pdfsubject={T0 Theorie - Fundamental Fractal-Geometric Field Theory}
}

% === KAPITEL 12: BOOKMARK (muss NACH hyperref kommen!) ===
\usepackage{bookmark}
% Fix for unicode-math symbols in PDF bookmarks
\pdfstringdefDisableCommands{%
	\def\xi{xi}%
	\def\alpha{alpha}%
	\def\beta{beta}%
	\def\gamma{gamma}%
	\def\delta{delta}%
	\def\Delta{Delta}%
	\def\epsilon{epsilon}%
	\def\varepsilon{epsilon}%
	\def\theta{theta}%
	\def\kappa{kappa}%
	\def\lambda{lambda}%
	\def\mu{mu}%
	\def\nu{nu}%
	\def\pi{pi}%
	\def\rho{rho}%
	\def\sigma{sigma}%
	\def\tau{tau}%
	\def\phi{phi}%
	\def\chi{chi}%
	\def\psi{psi}%
	\def\omega{omega}%
	\def\Omega{Omega}%
	\def\Lambda{Lambda}%
	\def\times{x}%
	\def\cdot{*}%
	\def\pm{+/-}%
	\def\approx{~}%
	\def\sim{~}%
	\def\equiv{=}%
	\def\ell{l}%
	\def\hbar{h}%
	\def\rightarrow{->}%
	\def\leftarrow{<-}%
	\def\Rightarrow{=>}%
	\def\Leftarrow{<=}%
	\def\propto{~}%
	\def\mitxi{xi}%
	\def\mitalpha{alpha}%
	\def\mitbeta{beta}%
	\def\mitgamma{gamma}%
	\def\mitdelta{delta}%
	\def\mitDelta{Delta}%
	\def\mitepsilon{epsilon}%
	\def\mitvarepsilon{epsilon}%
	\def\mittheta{theta}%
	\def\mitkappa{kappa}%
	\def\mitlambda{lambda}%
	\def\mitLambda{Lambda}%
	\def\mitmu{mu}%
	\def\mitnu{nu}%
	\def\mitpi{pi}%
	\def\mitrho{rho}%
	\def\mitsigma{sigma}%
	\def\mittau{tau}%
	\def\mitphi{phi}%
	\def\mitchi{chi}%
	\def\mitpsi{psi}%
	\def\mitomega{omega}%
	\def\mitOmega{Omega}%
}

% === KAPITEL 13: CLEVEREF (DEUTSCHE LABELS) ===
\usepackage[ngerman]{cleveref}
\crefname{equation}{Gleichung}{Gleichungen}
\crefname{figure}{Abbildung}{Abbildungen}
\crefname{table}{Tabelle}{Tabellen}
\crefname{section}{Abschnitt}{Abschnitte}
\crefname{chapter}{Kapitel}{Kapitel}
\crefname{theorem}{Satz}{Sätze}
\crefname{lemma}{Lemma}{Lemmata}
\crefname{definition}{Definition}{Definitionen}
\crefname{example}{Beispiel}{Beispiele}
\crefname{remark}{Bemerkung}{Bemerkungen}

% ==============================================================================
\newenvironment{alternative}{%
	\begin{mdframed}[linecolor=black!30,linewidth=1pt,roundcorner=4pt,backgroundcolor=black!5]%
	}{%
	\end{mdframed}%
}

% Photon/particle environment
\newenvironment{photon}{%
	\begin{mdframed}[linecolor=blue!30,linewidth=1pt,roundcorner=4pt,backgroundcolor=blue!5]%
	}{%
	\end{mdframed}%
}

% Koide formula box environment
\newenvironment{koidebox}{%
	\begin{mdframed}[linecolor=green!30,linewidth=1pt,roundcorner=4pt,backgroundcolor=green!5]%
	}{%
	\end{mdframed}%
}

% Erkenntnis/insight environment
\newenvironment{erkenntnis}{%
	\begin{mdframed}[linecolor=orange!30,linewidth=1pt,roundcorner=4pt,backgroundcolor=orange!5]%
	}{%
	\end{mdframed}%
}

% Beziehung/relationship environment
\newenvironment{beziehung}{%
	\begin{mdframed}[linecolor=purple!30,linewidth=1pt,roundcorner=4pt,backgroundcolor=purple!5]%
	}{%
	\end{mdframed}%
}

% Derivation environment
\newenvironment{derivation}{%
	\begin{mdframed}[linecolor=teal!30,linewidth=1pt,roundcorner=4pt,backgroundcolor=teal!5]%
	}{%
	\end{mdframed}%
}

% Abhandlung/treatise environment
\newenvironment{abhandlung}{%
	\begin{mdframed}[linecolor=brown!30,linewidth=1pt,roundcorner=4pt,backgroundcolor=brown!5]%
	}{%
	\end{mdframed}%
}

% Anwendung/application environment
\newenvironment{anwendung}{%
	\begin{mdframed}[linecolor=cyan!30,linewidth=1pt,roundcorner=4pt,backgroundcolor=cyan!5]%
	}{%
	\end{mdframed}%
}

% Additional common environments
\newenvironment{konsequenz}{%
	\begin{mdframed}[linecolor=red!30,linewidth=1pt,roundcorner=4pt,backgroundcolor=red!5]%
	}{%
	\end{mdframed}%
}

\newenvironment{schlussfolgerung}{%
	\begin{mdframed}[linecolor=gray!30,linewidth=1pt,roundcorner=4pt,backgroundcolor=gray!5]%
	}{%
	\end{mdframed}%
}

\newenvironment{result}{%
	\begin{mdframed}[linecolor=violet!30,linewidth=1pt,roundcorner=4pt,backgroundcolor=violet!5]%
	}{%
	\end{mdframed}%
}

% Formula environment
\newenvironment{formula}{%
	\begin{mdframed}[linecolor=yellow!30,linewidth=1pt,roundcorner=4pt,backgroundcolor=yellow!5]%
	}{%
	\end{mdframed}%
}

% Revolutionaer/revolutionary environment
\newenvironment{revolutionaer}{%
	\begin{mdframed}[linecolor=red!50,linewidth=2pt,roundcorner=4pt,backgroundcolor=red!10]%
	}{%
	\end{mdframed}%
}

% Formel environment (German version of formula)
\newenvironment{formel}{%
	\begin{mdframed}[linecolor=yellow!30,linewidth=1pt,roundcorner=4pt,backgroundcolor=yellow!5]%
	}{%
	\end{mdframed}%
}

% Prinzip/principle environment
\newenvironment{prinzip}{%
	\begin{mdframed}[linecolor=blue!50,linewidth=2pt,roundcorner=4pt,backgroundcolor=blue!10]%
	}{%
	\end{mdframed}%
}

% Experimentell/experimental environment
\newenvironment{experimentell}{%
	\begin{mdframed}[linecolor=magenta!30,linewidth=1pt,roundcorner=4pt,backgroundcolor=magenta!5]%
	}{%
	\end{mdframed}%
}

% Neutrino environment
\newenvironment{neutrino}{%
	\begin{mdframed}[linecolor=cyan!40,linewidth=1pt,roundcorner=4pt,backgroundcolor=cyan!8]%
	}{%
	\end{mdframed}%
}

% Additional missing environments
\newenvironment{schluessel}{%
	\begin{mdframed}[linecolor=yellow!50,linewidth=1pt,roundcorner=4pt,backgroundcolor=yellow!10]%
	}{%
	\end{mdframed}%
}

\newenvironment{summary}{%
	\begin{mdframed}[linecolor=gray!40,linewidth=1pt,roundcorner=4pt,backgroundcolor=gray!8]%
	}{%
	\end{mdframed}%
}

\newenvironment{category}{%
	\begin{mdframed}[linecolor=pink!40,linewidth=1pt,roundcorner=4pt,backgroundcolor=pink!8]%
	}{%
	\end{mdframed}%
}

\newenvironment{sibox}{%
	\begin{mdframed}[linecolor=lime!40,linewidth=1pt,roundcorner=4pt,backgroundcolor=lime!8]%
	}{%
	\end{mdframed}%
}

% More missing environments
\newenvironment{documentbox}{%
	\begin{mdframed}[linecolor=teal!40,linewidth=1pt,roundcorner=4pt,backgroundcolor=teal!8]%
	}{%
	\end{mdframed}%
}

\newenvironment{t0box}{%
	\begin{mdframed}[linecolor=violet!40,linewidth=1pt,roundcorner=4pt,backgroundcolor=violet!8]%
	}{%
	\end{mdframed}%
}

\newenvironment{wichtig}{%
	\begin{mdframed}[linecolor=red!50,linewidth=2pt,roundcorner=4pt,backgroundcolor=red!10]%
	\textbf{Wichtig:} 
	}{%
	\end{mdframed}%
}

\newenvironment{smbox}{%
	\begin{mdframed}[linecolor=orange!40,linewidth=1pt,roundcorner=4pt,backgroundcolor=orange!8]%
	}{%
	\end{mdframed}%
}

\newenvironment{pvbox}{%
	\begin{mdframed}[linecolor=purple!40,linewidth=1pt,roundcorner=4pt,backgroundcolor=purple!8]%
	}{%
	\end{mdframed}%
}

\newenvironment{numerisch}{%
	\begin{mdframed}[linecolor=blue!40,linewidth=1pt,roundcorner=4pt,backgroundcolor=blue!8]%
	}{%
	\end{mdframed}%
}

% More missing environments
\newenvironment{relation}{%
	\begin{mdframed}[linecolor=green!40,linewidth=1pt,roundcorner=4pt,backgroundcolor=green!8]%
	}{%
	\end{mdframed}%
}

\newenvironment{beweis}{%
	\begin{mdframed}[linecolor=brown!40,linewidth=1pt,roundcorner=4pt,backgroundcolor=brown!8]%
	\textbf{Beweis:} 
	}{%
	\end{mdframed}%
}

\newenvironment{revolution}{%
	\begin{mdframed}[linecolor=red!60,linewidth=2pt,roundcorner=4pt,backgroundcolor=red!12]%
	}{%
	\end{mdframed}%
}

\newenvironment{key}{%
	\begin{mdframed}[linecolor=yellow!50,linewidth=1pt,roundcorner=4pt,backgroundcolor=yellow!10]%
	}{%
	\end{mdframed}%
}

\newenvironment{newperspective}{%
	\begin{mdframed}[linecolor=cyan!50,linewidth=1pt,roundcorner=4pt,backgroundcolor=cyan!10]%
	}{%
	\end{mdframed}%
}

\newenvironment{literatur}{%
	\begin{mdframed}[linecolor=gray!50,linewidth=1pt,roundcorner=4pt,backgroundcolor=gray!10]%
	}{%
	\end{mdframed}%
}

\newenvironment{folgerung}{%
	\begin{mdframed}[linecolor=teal!50,linewidth=1pt,roundcorner=4pt,backgroundcolor=teal!10]%
	}{%
	\end{mdframed}%
}

\newenvironment{principle}{%
	\begin{mdframed}[linecolor=blue!60,linewidth=2pt,roundcorner=4pt,backgroundcolor=blue!12]%
	}{%
	\end{mdframed}%
}

% AB HIER: IHRE DEFINITIONEN (angepasst für Deutsch)
% ==============================================================================

\setcounter{tocdepth}{3}

% === ZITATBEFEHLE ===
\providecommand{\citep}[1]{\cite{#1}}
\providecommand{\citet}[1]{\cite{#1}}

% === FARBEN ===
\definecolor{gold}{RGB}{255,215,0}
\definecolor{blue}{rgb}{0,0,1}
\definecolor{boxgray}{RGB}{240,240,240}
\definecolor{deepblue}{RGB}{0,0,127}
\definecolor{deepgreen}{RGB}{0,127,0}
\definecolor{deepred}{RGB}{191,0,0}
\definecolor{t0blue}{RGB}{33,150,243}
\definecolor{t0green}{RGB}{76,175,80}
\definecolor{t0orange}{RGB}{255,152,0}
\definecolor{t0purple}{RGB}{156,39,176}
\definecolor{t0red}{RGB}{244,67,54}
\definecolor{t0yellow}{RGB}{255,204,0}

% === SPALTENTYPEN ===
\newcolumntype{L}[1]{>{\raggedright\arraybackslash}p{#1}}
\newcolumntype{C}[1]{>{\centering\arraybackslash}p{#1}}
\newcolumntype{R}[1]{>{\raggedleft\arraybackslash}p{#1}}

% === HYPERREF-EINSTELLUNGEN (aktualisiert) ===
\hypersetup{
	colorlinks=true,
	linkcolor=t0blue,
	citecolor=t0blue,
	urlcolor=t0blue,
	breaklinks=true,
	bookmarksnumbered=true,
	pdfstartview=FitH,
	pdfencoding=auto,
	pdfdisplaydoctitle=true
}

% === DEUTSCHE THEOREM-UMGEBUNGEN ===
\theoremstyle{plain}
\newtheorem{theorem}{Satz}[section]
\newtheorem{lemma}[theorem]{Lemma}
\newtheorem{proposition}[theorem]{Proposition}
\newtheorem{corollary}[theorem]{Korollar}

\theoremstyle{definition}
\newtheorem{definition}[theorem]{Definition}
\newtheorem{example}[theorem]{Beispiel}
\newtheorem{insight}[theorem]{Erkenntnis}
\newtheorem{discovery}[theorem]{Entdeckung}

\theoremstyle{remark}
\newtheorem{remark}[theorem]{Bemerkung}
\newtheorem{axiom}{Axiom}
%\newtheorem{principle}{Principle}  % Commented out to avoid conflicts with document-specific definitions
\newtheorem{warnung}[theorem]{Warnung}

% === T0-SPEZIFISCHE BEFEHLE ===
% (Hier folgen alle Ihre \newcommand und \providecommand Definitionen)
% Diese bleiben UNVERÄNDERT wie in Ihrer Original-Preamble
% ==============================================================================
% SECTION 14: T0-Specific Commands
% ==============================================================================

% --- Core T0 Fields ---
\newcommand{\Tfield}{T(x,t)}
\providecommand{\Tfieldt}{T(\vec{x},t)}
\newcommand{\Efield}{E(x,t)}
\newcommand{\mfield}{m(x,t)}
\providecommand{\vecx}{\vec{x}}

% --- Lagrangian ---
\newcommand{\Lag}{\mathcal{L}}
\newcommand{\calL}{\mathcal{L}}

% --- Greek Letters and Constants ---
\newcommand{\alphaem}{\alpha}
\newcommand{\betaT}{\beta_T}
\newcommand{\xiT}{\xi}
\newcommand{\xipar}{\xi}

% --- Energy and Planck Units ---
\newcommand{\Ezero}{E_0}
\newcommand{\EPlanck}{E_{\text{Pl}}}
\newcommand{\Mpl}{M_{\text{Pl}}}
\newcommand{\mP}{m_{\text{P}}}
\newcommand{\lP}{\ell_{\text{P}}}
\newcommand{\tP}{t_{\text{P}}}
\newcommand{\LPlanck}{\ell_{\text{Pl}}}
\newcommand{\TPlanck}{t_{\text{Pl}}}

% --- Coupling Constants ---
\newcommand{\Gnat}{G_{\text{nat}}}
\newcommand{\alphaEM}{\alpha_{\text{EM}}}
\newcommand{\alphaSI}{\alpha_{\text{SI}}}
\newcommand{\Hubble}{H_0}
\newcommand{\LCDM}{\Lambda\text{CDM}}
\newcommand{\natunits}{(nat. units)}

% --- T0 Model Parameters ---
\newcommand{\xigeom}{\xi_{\mathrm{geom}}}
\newcommand{\rzero}{r_{0}}
\newcommand{\xirat}{\xi_{\mathrm{rat}}}
\newcommand{\tzero}{t_{0}}
\newcommand{\Lambdat}{\Lambda_{\mathrm{t}}}
\newcommand{\EP}{E_{\text{P}}}
\newcommand{\Emu}{E_{\mu}}
\newcommand{\Ee}{E_{e}}
\newcommand{\Etau}{E_{\tau}}
\newcommand{\alphafine}{\alpha_{\mathrm{fine}}}
\newcommand{\alphal}{\alpha_{\ell}}
\newcommand{\Lzero}{\ell_{0}}
\newcommand{\Lp}{\ell_{\mathrm{P}}}

% --- Additional T0 Commands ---
\newcommand{\Kfrak}{K_{\text{frak}}}
\newcommand{\Dfrak}{D_{\text{frak}}}
\newcommand{\betapar}{\ensuremath{\beta_T}}
\newcommand{\alphapar}{\alpha}
\newcommand{\deltafield}{\delta \phi}
\newcommand{\deltam}{\delta m}
\newcommand{\deltaE}{\delta E}
\newcommand{\Exi}{E_{\xi}}
\newcommand{\Lxi}{\ell_{\xi}}
\newcommand{\rhoCMB}{\rho_{\text{CMB}}}
\newcommand{\rhoCasimir}{\rho_{\text{Casimir}}}
\newcommand{\Leff}{L_{\text{eff}}}
\newcommand{\CQCD}{C_{\mathrm{QCD}}}
\newcommand{\Kspec}{K_{\mathrm{spec}}}
\newcommand{\Tzero}{\ensuremath{T_0}}
\newcommand{\Eabs}{E_{\text{abs}}}
\newcommand{\taupar}{\tau}

% --- Provided Commands ---
\providecommand{\xiconst}{\xi_{\text{const}}}
\providecommand{\DhiggsT}{D_{\text{Higgs-T}}}
\providecommand{\rhoE}{\rho_{E}}
\providecommand{\Echar}{E_{\text{char}}}
\providecommand{\kfrac}{k_{\text{frac}}}
\providecommand{\alphaEMSI}{\alpha_{\text{EM,SI}}}
\providecommand{\alphaEMnat}{\alpha_{\text{EM,nat}}}
\providecommand{\betaTSI}{\beta_{T,\text{SI}}}
\providecommand{\betaTnat}{\beta_{T,\text{nat}}}
\providecommand{\Gsi}{G_{\text{SI}}}
\providecommand{\xiparSI}{\xi_{\text{SI}}}
\providecommand{\xiparnat}{\xi_{\text{nat}}}
\providecommand{\meff}{m_{\text{eff}}}
\providecommand{\Tzerot}{T_{0}(t)}
\providecommand{\mzerot}{m_{0}(t)}
\providecommand{\Ezeroabs}{E_{0,\text{abs}}}
\providecommand{\Epar}{E_{\text{par}}}
\providecommand{\Lnat}{\ell_{\text{nat}}}
\providecommand{\Tnat}{T_{\text{nat}}}
\providecommand{\xifrak}{\xi_{\text{frac}}}
\providecommand{\Tfrak}{T_{\text{frac}}}
\providecommand{\mfrak}{m_{\text{frac}}}
\providecommand{\Dfrac}{D_{\text{frac}}}
\providecommand{\EphotSI}{E_{\gamma,\text{SI}}}
\providecommand{\EphotNat}{E_{\gamma,\text{nat}}}
\providecommand{\Eabsint}{E_{\text{abs,int}}}
\providecommand{\mphoton}{m_{\gamma}}
\providecommand{\Evis}{E_{\text{vis}}}
\providecommand{\Cto}{C_{T0}}
\providecommand{\mytimes}{\times}
\providecommand{\lambdah}{\lambda_h}
\providecommand{\checkmarkx}{\checkmark}
\providecommand{\Enorm}{E_{\text{norm}}}
\providecommand{\Tobs}{T_{\text{obs}}}
\providecommand{\mobs}{m_{\text{obs}}}
\providecommand{\Eobs}{E_{\text{obs}}}
\providecommand{\Lobs}{\ell_{\text{obs}}}
\providecommand{\xobs}{\xi_{\text{obs}}}
\providecommand{\calE}{\mathcal{E}}
\providecommand{\calT}{\mathcal{T}}
\providecommand{\calM}{\mathcal{M}}
\providecommand{\alphag}{\alpha_g}
\providecommand{\Tmax}{T_{\text{max}}}
\providecommand{\mmin}{m_{\text{min}}}
\providecommand{\Lmax}{\ell_{\text{max}}}
\providecommand{\Emin}{E_{\text{min}}}
\providecommand{\Geff}{G_{\text{eff}}}
\providecommand{\rhoeff}{\rho_{\text{eff}}}
\providecommand{\xieff}{\xi_{\text{eff}}}
\providecommand{\Teff}{T_{\text{eff}}}
\providecommand{\hPlanck}{h}
\providecommand{\kB}{k_B}
\providecommand{\muB}{\mu_B}
\providecommand{\lambdaC}{\lambda_C}
\providecommand{\omegaP}{\omega_P}
\providecommand{\rhoP}{\rho_P}
\providecommand{\Tref}{T_{\text{ref}}}
\providecommand{\Eref}{E_{\text{ref}}}
\providecommand{\mref}{m_{\text{ref}}}
\providecommand{\Lref}{\ell_{\text{ref}}}
\providecommand{\xikonst}{\xi_0}
\providecommand{\Phiphoton}{\Phi_{\gamma}}
\providecommand{\etavis}{\eta_{\text{vis}}}
\providecommand{\pichar}{\pi}
\providecommand{\primrel}{\mathcal{P}_{\text{rel}}}
\providecommand{\warningx}{\textcolor{orange}{\textbf{!}}}
\providecommand{\phiT}{\phi_T}
\providecommand{\Lorentz}{\Lambda}
\providecommand{\Cconv}{C_{\text{conv}}}
\providecommand{\Df}{\Delta f}
\providecommand{\lambdazero}{\lambda_0}
\providecommand{\myapprox}{\approx}
\providecommand{\checked}{\checkmark}
\providecommand{\alphaWSI}{\alpha_W^{\text{SI}}}
\providecommand{\alphaWnat}{\alpha_W^{\text{nat}}}
\providecommand{\vect}[1]{\vec{#1}}
\providecommand{\Rzero}{R_0}
\providecommand{\Riem}{\mathcal{R}}
\providecommand{\nuzero}{\nu_0}
\providecommand{\mypi}{\pi}

% =============================================================================
% TCOLORBOX-STILE UND UMGEBUNGEN (deutsche Titel)
% =============================================================================
\tcbset{
	keyresult/.style={
		colback=blue!5!white,
		colframe=blue!75!black,
		title=Schlüsselergebnis,
		fonttitle=\bfseries
	},
	foundation/.style={
		colback=green!5!white,
		colframe=green!75!black,
		title=Grundlage,
		fonttitle=\bfseries
	},
	alternative/.style={
		colback=orange!5!white,
		colframe=orange!75!black,
		title=Alternative,
		fonttitle=\bfseries
	},
	warningbox/.style={
		colback=red!5!white,
		colframe=red!75!black,
		title=Warnung,
		fonttitle=\bfseries
	}
}

% (Hier folgen alle Ihre tcolorbox-Definitionen mit deutschen Titeln)
\newtcolorbox{keyresultbox}[1][]{colback=blue!5!white,colframe=blue!75!black,fonttitle=\bfseries,title={#1},breakable}
\newtcolorbox{keyresult}[1][Schlüsselergebnis]{colback=blue!5!white,colframe=blue!75!black,fonttitle=\bfseries,title={#1},breakable}
\newtcolorbox{foundationbox}[1][]{colback=green!5!white,colframe=green!75!black,fonttitle=\bfseries,title={#1},breakable}
\newtcolorbox{foundation}[1][Grundlage]{colback=green!5!white,colframe=green!75!black,fonttitle=\bfseries,title={#1},breakable}
\newtcolorbox{alternativebox}[1][]{colback=orange!5!white,colframe=orange!75!black,fonttitle=\bfseries,title={#1},breakable}
\newtcolorbox{warningboxenv}[1][Warnung]{colback=red!5!white,colframe=red!75!black,fonttitle=\bfseries,title={#1},breakable}

\newtcolorbox{fundamental}[1][]{
	colback=boxgray,
	colframe=t0blue,
	fonttitle=\bfseries,
	title=#1,
	sharp corners,
	boxrule=2pt
}

\newtcolorbox{insightBox}[1][Erkenntnis]{colback=blue!5,colframe=t0blue,title={#1},fonttitle=\bfseries,breakable}
\newtcolorbox{discoveryBox}[1][Entdeckung]{colback=green!5,colframe=t0green,title={#1},fonttitle=\bfseries,breakable}
\newtcolorbox{revelation}[1][Offenbarung]{colback=red!5,colframe=t0red,title={#1},fonttitle=\bfseries,breakable}
\newtcolorbox{keypoint}[1][Schlüsselpunkt]{colback=blue!5,colframe=t0blue,title={#1},fonttitle=\bfseries,breakable}
\newtcolorbox{evidence}[1][Beleg]{colback=green!5,colframe=t0green,title={#1},fonttitle=\bfseries,breakable}
\newtcolorbox{conclusionBox}[1][Fazit]{colback=gray!5,colframe=gray,title={#1},fonttitle=\bfseries,breakable}
\newtcolorbox{significance}[1][Bedeutung]{colback=yellow!5,colframe=orange,title={#1},fonttitle=\bfseries,breakable}
\newtcolorbox{philosophical}[1][Philosophisch]{colback=purple!5,colframe=purple,title={#1},fonttitle=\bfseries,breakable}
\newtcolorbox{implicationBox}[1][Implikation]{colback=cyan!5,colframe=cyan,title={#1},fonttitle=\bfseries,breakable}
\newtcolorbox{perspectiveBox}[1][Perspektive]{colback=blue!5,colframe=t0blue,title={#1},fonttitle=\bfseries,breakable}
\newtcolorbox{revolutionary}[1][Revolutionär]{colback=red!5,colframe=t0red,title={#1},fonttitle=\bfseries,breakable}

\newtcolorbox{technical}[1][Technisch]{colback=gray!5,colframe=gray!75!black,title={#1},fonttitle=\bfseries,breakable}
\newtcolorbox{technicalBox}[1][Technisch]{colback=gray!5,colframe=gray!75!black,title={#1},fonttitle=\bfseries,breakable}
\newtcolorbox{notationBox}[1][Notation]{colback=yellow!5,colframe=yellow!75!black,title={#1},fonttitle=\bfseries,breakable}
\newtcolorbox{verification}[1][Verifikation]{colback=orange!5!white,colframe=orange!75!black,fonttitle=\bfseries,title=#1}
\newtcolorbox{explanationBox}[1][Erklärung]{colback=purple!5!white,colframe=purple!75!black,fonttitle=\bfseries,title=#1}
\newtcolorbox{interpretationBox}[1][Interpretation]{colback=cyan!5!white,colframe=cyan!75!black,fonttitle=\bfseries,title=#1}
\newtcolorbox{explanation}[1][Erklärung]{colback=purple!5!white,colframe=purple!75!black,fonttitle=\bfseries,title=#1,breakable}
\newtcolorbox{interpretation}[1][Interpretation]{colback=cyan!5!white,colframe=cyan!75!black,fonttitle=\bfseries,title=#1,breakable}
\newtcolorbox{proof_step}[1][Beweisschritt]{colback=gray!5!white,colframe=gray!75!black,fonttitle=\bfseries,title=#1,breakable}
\newtcolorbox{experimental}[1][Experimentell]{colback=teal!5!white,colframe=teal!75!black,fonttitle=\bfseries,title=#1,breakable}

\newtcolorbox{important}[1][Wichtig]{colback=red!5!white,colframe=red!75!black,title={#1},fonttitle=\bfseries,breakable}
\newtcolorbox{warning}[1][Warnung]{colback=orange!5!white,colframe=orange!75!black,title={#1},fonttitle=\bfseries,breakable}
\newtcolorbox{caution}[1][Vorsicht]{colback=yellow!5!white,colframe=yellow!75!black,title={#1},fonttitle=\bfseries,breakable}
\newtcolorbox{vorsicht}[1][Vorsicht]{colback=yellow!5!white,colframe=yellow!75!black,title={#1},fonttitle=\bfseries,breakable}
\newtcolorbox{highlight}[1][Hervorhebung]{colback=yellow!10!white,colframe=yellow!75!black,title={#1},fonttitle=\bfseries,breakable}
\newtcolorbox{critical}[1][Kritisch]{colback=red!10!white,colframe=red!75!black,title={#1},fonttitle=\bfseries,breakable}

\newtcolorbox{analysis}[1][Analyse]{colback=blue!5!white,colframe=blue!75!black,title={#1},fonttitle=\bfseries,breakable}
\newtcolorbox{application}[1][Anwendung]{colback=green!5!white,colframe=green!75!black,title={#1},fonttitle=\bfseries,breakable}
\newtcolorbox{experiment}[1][Experiment]{colback=cyan!5!white,colframe=cyan!75!black,title={#1},fonttitle=\bfseries,breakable}
\newtcolorbox{historical}[1][Historisch]{colback=brown!5!white,colframe=brown!75!black,title={#1},fonttitle=\bfseries,breakable}
\newtcolorbox{numerical}[1][Numerisch]{colback=gray!5!white,colframe=gray!75!black,title={#1},fonttitle=\bfseries,breakable}
\newtcolorbox{overview}[1][Überblick]{colback=blue!5!white,colframe=blue!75!black,title={#1},fonttitle=\bfseries,breakable}
\newtcolorbox{speculation}[1][Spekulation]{colback=purple!5!white,colframe=purple!75!black,title={#1},fonttitle=\bfseries,breakable}
\newtcolorbox{question}[1][Frage]{colback=orange!5!white,colframe=orange!75!black,title={#1},fonttitle=\bfseries,breakable}
\newtcolorbox{method}[1][Methode]{colback=teal!5!white,colframe=teal!75!black,title={#1},fonttitle=\bfseries,breakable}
\newtcolorbox{correct}[1][Korrekt]{colback=green!10!white,colframe=green!75!black,title={#1},fonttitle=\bfseries,breakable}
\newtcolorbox{units}[1][Einheiten]{colback=gray!5!white,colframe=gray!75!black,title={#1},fonttitle=\bfseries,breakable}
\newtcolorbox{achievement}[1][Errungenschaft]{colback=gold!5!white,colframe=orange!75!black,title={#1},fonttitle=\bfseries,breakable}
\newtcolorbox{equivalence}[1][Äquivalenz]{colback=cyan!5!white,colframe=cyan!75!black,title={#1},fonttitle=\bfseries,breakable}
\newtcolorbox{dimensional}[1][Dimensionsanalyse]{colback=purple!5!white,colframe=purple!75!black,title={#1},fonttitle=\bfseries,breakable}

% === ZUSÄTZLICHE EINFACHE UMGEBUNGEN ===
\newenvironment{treatise}{\begin{quote}}{\end{quote}}
\newenvironment{gemeinsam}{\begin{quote}}{\end{quote}}
\newenvironment{vergleich}{\begin{quote}}{\end{quote}}
\newenvironment{vorteil}{\begin{quote}}{\end{quote}}
\newenvironment{quantum}{\begin{quote}}{\end{quote}}

% === LAYOUT-EINSTELLUNGEN ===
\raggedbottom
\usepackage{environ}
\let\oldtabular\tabular
\let\endoldtabular\endtabular

\newenvironment{scaledtable}[1][0.85]{%
	\begingroup\footnotesize\setlength{\LTleft}{0pt}\setlength{\LTright}{0pt}%
}{%
	\endgroup%
}

\newcommand{\widetable}[1]{\resizebox{\textwidth}{!}{#1}}

% === INHALTSVERZEICHNIS-FORMATIERUNG ===
\renewcommand{\cftsecfont}{\color{blue}}
\renewcommand{\cftsubsecfont}{\color{blue}}
\renewcommand{\cftsecpagefont}{\color{blue}}
\renewcommand{\cftsubsecpagefont}{\color{blue}}
\renewcommand{\cfttoctitlefont}{\huge\bfseries\color{blue}}

% === STANDARD-KOPF- UND FUßZEILE ===
\pagestyle{fancy}
\fancyhf{}
\fancyhead[L]{\textsc{T0 Theorie}}
\fancyhead[R]{\textsc{J. Pascher}}
\fancyfoot[C]{\thepage}

% ==============================================================================
% Ende der Shared Preamble für Deutsch
% ==============================================================================

\title{Dirac Equation in T0 Theory: \\
	Geometric Integration with Time-Mass Duality \\
	\large Fractal Spacetime and Dynamic Mass}
\author{}
\date{January 2026}

\begin{document}
	
	\maketitle
	
	\begin{abstract}
		This work fully integrates the Dirac equation into the T0 theory framework. 
		Unlike the standard formulation with constant mass, T0 theory uses the fundamental 
		time-mass duality $T(x) \cdot m(x) = 1$, leading to a spacetime-dependent mass. 
		The fractal dimension $D_f = 3 - \xi$ modifies the underlying metric and thus 
		the differential operator. We show how the Clifford algebra structure naturally 
		connects with the torus topology of T0 theory and how spin-1/2 can be interpreted 
		as a topological winding number. The predictions are formulated as ratio-based 
		statements that are independent of unit systems and phenomenological parameters. 
		Experimental tests at Belle II can directly verify the fundamental quadratic mass scaling.
	\end{abstract}
	
	\tableofcontents
	
	\section{Introduction: T0 Basic Principles}
	
	\subsection{Time-Mass Duality}
	
	The fundamental principle of T0 theory is time-mass duality:
	
	\begin{equation}
		T(x,t) \cdot m(x,t) = \frac{\hbar}{c^2}
		\label{eq:time_mass_duality}
	\end{equation}
	
	In natural units ($\hbar = c = 1$):
	\begin{equation}
		T(x,t) \cdot m(x,t) = 1
		\label{eq:tmd_natural}
	\end{equation}
	
	This means: **Mass is not constant but a dynamic field**, coupled to the intrinsic time field $T(x,t)$.
	
	\subsection{Fractal Spacetime}
	
	T0 theory postulates a fractal spacetime dimension:
	\begin{equation}
		D_f = 3 - \xi \quad \text{with} \quad \xi = \frac{4}{3 \times 10^4} \approx 1.333 \times 10^{-4}
		\label{eq:fractal_dim}
	\end{equation}
	
	This modifies the metric and thus all differential operators.
	
	\subsection{Torus Topology}
	
	The underlying topology is a torus with characteristic scales:
	\begin{itemize}
		\item Large radius: $R \sim 1/\xi$
		\item Small radius: $r \sim R \cdot \xi$
		\item Winding numbers: $(n_\theta, n_\phi)$ for poloidal and toroidal directions
	\end{itemize}
	
	\section{Standard Dirac Equation: Problems}
	
	\subsection{The Standard Form}
	
	The usual Dirac equation reads:
	\begin{equation}
		(i\gamma^\mu \partial_\mu - m)\psi = 0
		\label{eq:standard_dirac}
	\end{equation}
	
	with constant mass $m$ and flat Minkowski metric.
	
	\subsection{Problems for T0 Integration}
	
	\begin{enumerate}
		\item \textbf{Constant mass:} Contradicts time-mass duality
		\item \textbf{Flat metric:} Ignores the fractal structure
		\item \textbf{No topology:} Spin has no geometric origin
		\item \textbf{Static:} No coupling to time field
	\end{enumerate}
% DIESES KAPITEL EINFÜGEN IN 051_dirac_De_v2.pdf
% NACH SECTION 2.2 (Probleme für die T0-Integration)
% VOR SECTION 3 (T0-Dirac-Gleichung: Geometrische Form)

\section{Clifford Algebra: The Fundamental Structure}
\label{sec:clifford_fundamentals}

Before we develop the T0-specific formulation, we need to understand what the Dirac equation \textbf{really} is – beyond the 4×4 matrices.

\subsection{Representation vs. Physics}
\label{subsec:representation_vs_physics}

\textbf{The central insight:} The 4×4 matrices are not the physics, but a \textbf{specific representation} of the physics.

\begin{important}{Fundamental Difference}
	\textbf{Fundamental (Physics):} \\
	The Clifford algebra structure of spacetime
	
	\textbf{Representation (Calculation):} \\
	Specific 4×4 matrices $\gamma^\mu$ in a chosen basis
	
	\vspace{0.3cm}
	
	\textbf{Analogy:} Vectors are fundamental; their components depend on the chosen basis. The physics (vector) is basis-independent, the calculation (components) is not.
\end{important}

\textbf{Example – different representations:}

The same Dirac equation can be written with:
\begin{itemize}
	\item \textbf{Dirac representation:} Specific 4×4 matrices
	\item \textbf{Weyl representation:} Different 4×4 matrices
	\item \textbf{Majorana representation:} Yet different matrices
\end{itemize}

All describe \textbf{the same physics}! The choice is convention, like choosing a coordinate basis.

\subsection{The Abstract Clifford Form}
\label{subsec:abstract_clifford}

The fundamental form of the Dirac equation without explicit matrices is:

\begin{equation}
	\boxed{(i \mathbf{e}_\mu \partial^\mu - m)\Psi = 0}
	\label{eq:clifford_fundamental}
\end{equation}

where:
\begin{itemize}
	\item $\mathbf{e}_\mu$: \textbf{Abstract basis vectors} of spacetime (not matrices!)
	\item $\Psi$: Element in the \textbf{spin bundle} (geometric object)
	\item The \textbf{Clifford product rule}:
	\begin{equation}
		\mathbf{e}_\mu \mathbf{e}_\nu + \mathbf{e}_\nu \mathbf{e}_\mu = 2 g_{\mu\nu}
		\label{eq:clifford_product_rule}
	\end{equation}
\end{itemize}

\textbf{What does the Clifford product mean?}

The product $\mathbf{e}_\mu \mathbf{e}_\nu$ is \textbf{non-commutative}:
\begin{align}
	\mathbf{e}_0 \mathbf{e}_1 &\neq \mathbf{e}_1 \mathbf{e}_0 \\
	\mathbf{e}_0 \mathbf{e}_1 + \mathbf{e}_1 \mathbf{e}_0 &= 0 \quad \text{(since } g_{01} = 0\text{)}
\end{align}

This encodes the \textbf{geometric structure of spacetime}.

\subsection{What Are the $\gamma$-Matrices Really?}
\label{subsec:what_are_gammas}

The familiar $\gamma^\mu$ matrices are simply:

\begin{equation}
	\gamma^\mu \quad \longleftrightarrow \quad \text{Matrix representation of } \mathbf{e}^\mu
\end{equation}

\textbf{Concretely:} One chooses a basis in spin space and writes:
\begin{equation}
	\mathbf{e}^\mu \quad \rightarrow \quad \gamma^\mu = 
	\begin{pmatrix}
		\gamma^\mu_{11} & \gamma^\mu_{12} & \gamma^\mu_{13} & \gamma^\mu_{14} \\
		\gamma^\mu_{21} & \gamma^\mu_{22} & \gamma^\mu_{23} & \gamma^\mu_{24} \\
		\gamma^\mu_{31} & \gamma^\mu_{32} & \gamma^\mu_{33} & \gamma^\mu_{34} \\
		\gamma^\mu_{41} & \gamma^\mu_{42} & \gamma^\mu_{43} & \gamma^\mu_{44}
	\end{pmatrix}
\end{equation}

The specific numbers in the matrix depend on the chosen representation!

\textbf{The physics} (Clifford product rule~\eqref{eq:clifford_product_rule}) is independent of this choice.

\subsection{Spin as Topological Property}
\label{subsec:spin_topology_detailed}

The spin-1/2 character is not a property of the matrices but follows from the Clifford algebra structure.

\subsubsection{The 720° Rotation}

\textbf{Key observation:} A spinor $\Psi$ behaves under rotations as:

\begin{align}
	R(180°) \Psi &= e^{i\pi/2} \Psi = i \Psi \\
	R(360°) \Psi &= e^{i\pi} \Psi = -\Psi \\
	R(720°) \Psi &= e^{i 2\pi} \Psi = \Psi
\end{align}

This is \textbf{not a matrix property}, but follows from Clifford algebra!

\textbf{Why?} The rotation is given by:
\begin{equation}
	R(\theta) = \exp\left(\frac{i\theta}{2} \mathbf{e}_1 \mathbf{e}_2\right)
\end{equation}

The factor $1/2$ in the exponent is \textbf{geometric} (comes from the Clifford algebra structure), not from the matrices!

\subsubsection{Topological Interpretation}

In T0 theory, we can interpret spin geometrically as a \textbf{winding number on a torus}:

\begin{equation}
	\text{Spin-}s \quad \longleftrightarrow \quad \text{Winding } (n_\theta, n_\phi) 
	\text{ with } \frac{n_\phi}{n_\theta} = 2s
	\label{eq:spin_winding_number}
\end{equation}

\textbf{For spin-1/2:} $(n_\theta, n_\phi) = (1, 1)$ or $(2, 1)$

The 720° rotation then corresponds to:
\begin{itemize}
	\item Once around the poloidal circle → $-\Psi$ (360°)
	\item Twice around the poloidal circle → $+\Psi$ (720°)
\end{itemize}

This is \textbf{pure topology}, not a mysterious quantum property!

\begin{figure}[h]
	\centering
	\begin{tikzpicture}[scale=2.0]
		% Torus - outer contour (top view)
		\draw[very thick, blue!60] (0,0) ellipse (2.2cm and 0.9cm);
		
		% Inner hole
		\draw[very thick, blue!60] (0,0) ellipse (0.7cm and 0.5cm);
		
		% Connection lines (optional for 3D effect)
		\draw[thick, blue!40] (-2.2,0) -- (-0.7,0);
		\draw[thick, blue!40] (2.2,0) -- (0.7,0);
		
		% Poloidal circle (small circle) - right outside, DOUBLE
		\begin{scope}[shift={(1.8,0)}]
			\draw[ultra thick, green!60!black] (0,0) circle (0.4cm);
			\draw[ultra thick, green!60!black, ->] (0,0.4) arc (90:270:0.4cm);
			\draw[ultra thick, green!60!black, ->] (0,0.4) arc (90:-90:0.4cm);
		\end{scope}
		\node[green!60!black, right] at (2.5,0) {$n_\theta$ poloidal};
		
		% Toroidal path (large circle around center)
		\draw[ultra thick, red!70!black, ->] 
		(1.4,0) arc[start angle=0, end angle=180, x radius=1.4cm, y radius=0.7cm];
		\draw[ultra thick, red!70!black, ->] 
		(-1.4,0) arc[start angle=180, end angle=360, x radius=1.4cm, y radius=0.7cm];
		\node[red!70!black, below] at (0,-1.1) {$n_\phi$ toroidal};
		
		% Spin-1/2 winding (1,1) - once small, once large
		\draw[ultra thick, purple!70, ->] 
		plot[smooth, tension=0.7] coordinates {
			(1.8,0.4) (1.5,0.5) (1.0,0.6) (0.3,0.5) (-0.3,0.3) 
			(-0.8,0.1) (-1.3,-0.1) (-1.6,-0.3) (-1.7,-0.5)
			(-1.5,-0.6) (-1.0,-0.65) (-0.3,-0.6) (0.4,-0.5)
			(1.0,-0.35) (1.5,-0.15) (1.8,0.1) (1.8,0.4)
		};
		\node[purple!70, above] at (0,0.9) {\textbf{Spin-1/2: $(1,1)$-Winding}};
		
		% Title
		\node[blue!60, font=\large] at (0,1.3) {\textbf{Torus Topology}};
	\end{tikzpicture}
	\caption{Spin-1/2 as topological winding on a torus (top view). The green double arrow shows the poloidal small circle ($n_\theta$, cross-section of the torus tube). The red arrows show the toroidal direction ($n_\phi$, around the central hole). The violet path shows a $(1,1)$-winding: once around the small circle AND once around the large circle. A 720° rotation corresponds to traversing this winding twice.}
	\label{fig:spin_winding}
\end{figure}

\subsection{Common Misconceptions}
\label{subsec:common_misconceptions}

\subsubsection{Can the Matrices Really Be Eliminated?}

\textbf{Answer: Yes and No.}

\begin{itemize}
	\item \textbf{Yes – fundamentally:} The physics does not need specific 4×4 matrices. The Clifford algebra is fundamental.
	
	\item \textbf{No – practically:} For concrete calculations, a representation is necessary, and matrices are often the most practical choice.
\end{itemize}

\textbf{Analogy:} One can formulate vector physics without coordinates (fundamental), but for calculations one chooses coordinates (practical).

\subsubsection{Is Information Lost?}

\textbf{No!} The Clifford algebra formulation contains \textbf{exactly the same information}:

\begin{table}[h]
	\centering
	\begin{tabular}{lcc}
		\toprule
		\textbf{Property} & \textbf{In Matrices} & \textbf{In Clifford Algebra} \\
		\midrule
		Spin-1/2 & In $\gamma$-structure & In Clifford product rule \\
		Lorentz inv. & Explicit in matrices & In $g_{\mu\nu}$-structure \\
		Antiparticles & Negative energy solutions & Chirality components \\
		Measurables & Matrix elements & Invariant under representation \\
		\bottomrule
	\end{tabular}
	\caption{Information identical in both formulations}
\end{table}

\subsubsection{Is This Just a Reformulation?}

\textbf{No – it is a conceptual shift:}

\begin{itemize}
	\item \textbf{Old view:} ``Electrons are point particles with mysterious intrinsic spin, described by complicated 4×4 matrices''
	
	\item \textbf{New view:} ``Electrons are geometric objects in a Clifford-structured spacetime. Spin is a topological property.''
\end{itemize}

This new view enables \textbf{natural integration} into T0 theory:
\begin{itemize}
	\item Fractal metric $\rightarrow$ modified Clifford structure
	\item Torus topology $\rightarrow$ spin as winding number
	\item Time-mass duality $\rightarrow$ dynamic mass $m(x)$
\end{itemize}

\subsection{Preparation for T0 Integration}
\label{subsec:preparation_t0}

With this understanding, we can now introduce the T0-specific modifications:

\begin{enumerate}
	\item \textbf{Fractal metric:} $g_{\mu\nu} \rightarrow g_{\mu\nu}^{\text{(frak)}}$ with $D_f = 3 - \xi$
	
	\item \textbf{Modified Clifford rule:}
	\begin{equation}
		\mathbf{e}_\mu^{\text{(frak)}} \mathbf{e}_\nu^{\text{(frak)}} + 
		\mathbf{e}_\nu^{\text{(frak)}} \mathbf{e}_\mu^{\text{(frak)}} = 
		2 g_{\mu\nu}^{\text{(frak)}}
	\end{equation}
	
	\item \textbf{Dynamic mass:} $m \rightarrow m(x) = 1/(c^2 T(x))$
	
	\item \textbf{Tetrad formulation:} Necessary for curved/fractal spacetime
\end{enumerate}

In the next section, we develop this T0-specific formulation in detail.

\begin{keypoint}[Core Message of This Chapter]
	The Dirac equation is fundamentally a \textbf{geometric equation} in the Clifford algebra of spacetime. The 4×4 matrices are useful calculation tools, but not the physics itself. This insight is \textbf{essential} for integration into T0 theory with its fractal geometry and torus topology.
\end{keypoint}	
	\section{T0 Dirac Equation: Geometric Form}
	
	\subsection{Clifford Algebra in Fractal Spacetime}
	
	Instead of the standard form, we use the Clifford algebra formulation:
	\begin{equation}
		\boxed{(i \partial\!\!\!/_{\text{frak}} - m(x))\Psi(x) = 0}
		\label{eq:t0_dirac}
	\end{equation}
	
	where:
	\begin{align}
		\partial\!\!\!/_{\text{frak}} &= \mathbf{e}^\mu_a(x) \gamma^a \partial_\mu 
		\quad \text{(tetrad-based)} \\
		m(x) &= \frac{1}{c^2 T(x)} \quad \text{(from time-mass duality)} \\
		\mathbf{e}^\mu_a(x) &= \text{Tetrad in fractal metric}
	\end{align}
	
	\subsection{Fractal Metric}
	
	The fractal correction to the metric is:
	\begin{equation}
		g_{\mu\nu}^{\text{(frak)}}(x) = \eta_{\mu\nu} \cdot \left(1 + \xi \cdot f(x)\right)
		\label{eq:fractal_metric}
	\end{equation}
	
	where $f(x)$ is a dimensionless function of coordinates describing the fractal structure.
	
	\subsection{Tetrad Formulation}
	
	The tetrad $\mathbf{e}^\mu_a(x)$ connects the curved spacetime with the local Clifford algebra:
	\begin{equation}
		g_{\mu\nu}^{\text{(frak)}}(x) = \mathbf{e}^\mu_a(x) \mathbf{e}^\nu_b(x) \eta^{ab}
		\label{eq:tetrad_metric}
	\end{equation}
	
	The $\gamma^a$ are the standard Clifford generators in the local Lorentz frame.
	
	\section{Dynamic Mass}
	
	\subsection{Spacetime Dependence}
	
	From time-mass duality follows:
	\begin{equation}
		m(x,t) = \frac{1}{c^2 T(x,t)} = \frac{1}{c^2} \max(\omega(x,t), m_{\text{bg}}(x))
		\label{eq:dynamic_mass}
	\end{equation}
	
	where:
	\begin{itemize}
		\item $\omega(x,t)$: Local frequency/energy density
		\item $m_{\text{bg}}(x)$: Background mass field
	\end{itemize}
	
	\subsection{Coupling to Time Field}
	
	The time field $T(x,t)$ is itself a dynamic field with Lagrangian density:
	\begin{equation}
		\mathcal{L}_T = \frac{1}{2}(\partial_\mu T)(\partial^\mu T) - V(T)
		\label{eq:time_lagrangian}
	\end{equation}
	
	The coupling to fermions occurs through the mass:
	\begin{equation}
		\mathcal{L}_{\text{int}} = \bar{\Psi} m(T(x)) \Psi
		\label{eq:fermion_time_coupling}
	\end{equation}
	
	\section{Spin as Topology}
	
	\subsection{Winding Numbers on the Torus}
	
	In T0 theory, spin is interpreted as a winding number:
	\begin{equation}
		\text{Spin-}s \quad \longleftrightarrow \quad 
		\text{Winding } (n_\theta, n_\phi) \text{ with } n_\phi/n_\theta = 2s
		\label{eq:spin_topology}
	\end{equation}
	
	\textbf{Examples:}
	\begin{align}
		\text{Spin-}0: &\quad (1, 0) \text{ or } (0, 1) \\
		\text{Spin-}1/2: &\quad (1, 1) \text{ or } (2, 1) \\
		\text{Spin-}1: &\quad (1, 2)
	\end{align}
	
	\subsection{720° Rotation Geometrically}
	
	The well-known property of spin-1/2 particles (720° rotation for identity) follows from torus topology:
	
	\begin{itemize}
		\item One poloidal winding: 360° rotation → $-\Psi$
		\item Two poloidal windings: 720° rotation → $+\Psi$
	\end{itemize}
	
	This is not a mysterious property but **pure topology**.
	
	\section{Mass-Proportional Coupling}
	
	\subsection{Interaction Lagrangian}
	
	The coupling of leptons to the time field is mass-proportional:
	\begin{equation}
		\mathcal{L}_{\text{int}} = \xi m_\ell \bar{\Psi}_\ell \Psi_\ell \Delta m(x)
		\label{eq:mass_proportional}
	\end{equation}
	
	where $\Delta m(x) = m(x) - m_0$ is the mass fluctuation.
	
	\subsection{Consequence: Quadratic Scaling}
	
	From this mass-proportional coupling follows for loop diagrams:
	\begin{equation}
		\Delta a_\ell \propto (\xi m_\ell)^2 \cdot \text{(kinematic factors)} \propto m_\ell^2
		\label{eq:quadratic_scaling}
	\end{equation}
	
	This leads to the fundamental ratio prediction:
	\begin{equation}
		\boxed{\frac{\Delta a_{\ell_1}}{\Delta a_{\ell_2}} = \left(\frac{m_{\ell_1}}{m_{\ell_2}}\right)^2}
		\label{eq:ratio_prediction}
	\end{equation}
	
	\section{Ratios vs. Absolute Values}
	
	\subsection{What the T0 Dirac Equation Predicts}
	
	\textbf{Fundamental predictions (parameter-free):}
	\begin{itemize}
		\item Ratio: $a_\tau/a_\mu = (m_\tau/m_\mu)^2 \approx 283$
		\item Structure: $\Delta a \propto m^2$ (quadratic scaling)
		\item Topology: Spin-1/2 as winding number
	\end{itemize}
	
	\textbf{Not predictable (phenomenological):}
	\begin{itemize}
		\item Absolute values: $a_\mu = 37.5 \times 10^{-11}$ (requires normalization)
	\end{itemize}
	
	\subsection{Why Only Ratios?}
	
	Complete calculation of absolute values requires:
	\begin{enumerate}
		\item Solution of time field dynamics in fractal spacetime (too complex)
		\item Loop integrals in non-integer dimension (open)
		\item Renormalization at $D_f = 3 - \xi$ (not fully developed)
		\item Recursive coupling of all fields (non-perturbative)
	\end{enumerate}
	
	This is analogous to QCD in the Standard Model: Fundamental Lagrangian density is clear, 
	but hadronic contributions are not calculable ab initio.
	
	\section{Natural vs. SI Units}
	
	\subsection{In Natural Units}
	
	In natural units ($\hbar = c = 1$, $\alpha = 1$) $\alpha$ disappears from all formulas:
	
	\begin{equation}
		\tilde{a}_\ell = \tilde{C} \cdot \xi \cdot \tilde{m}_\ell^2
		\label{eq:natural_units}
	\end{equation}
	
	The ratio is:
	\begin{equation}
		\frac{\tilde{a}_\tau}{\tilde{a}_\mu} = \left(\frac{\tilde{m}_\tau}{\tilde{m}_\mu}\right)^2
	\end{equation}
	
	**Identical to SI version** – ratios are invariant!
	
	\subsection{Transformation to SI}
	
	The transformation to SI units introduces $\alpha$:
	\begin{equation}
		a_\ell[\text{SI}] = \text{(conversion factor with } \alpha\text{)} \times \tilde{a}_\ell
	\end{equation}
	
	But the **ratio remains unchanged**:
	\begin{equation}
		\frac{a_\tau[\text{SI}]}{a_\mu[\text{SI}]} = \frac{\tilde{a}_\tau}{\tilde{a}_\mu} = 
		\left(\frac{m_\tau}{m_\mu}\right)^2
	\end{equation}
	
	\section{Experimental Tests}
	
	\subsection{Belle II: Critical Test (2027-2028)}
	
	The fundamental prediction:
	\begin{equation}
		\frac{a_\tau}{a_\mu} = \left(\frac{1776.86}{105.658}\right)^2 = 282.8
	\end{equation}
	
	is directly testable at Belle II.
	
	\textbf{Possible outcomes:}
	\begin{itemize}
		\item \textbf{Confirmation:} Strong evidence for mass-proportional coupling
		\item \textbf{Deviation:} Modification of coupling structure needed
		\item \textbf{Null result:} T0 contributions suppressed or incorrect
	\end{itemize}
	
	\subsection{Further Tests}
	
	\begin{table}[h]
		\centering
		\begin{tabular}{lcc}
			\toprule
			\textbf{Test} & \textbf{T0 Prediction} & \textbf{Status} \\
			\midrule
			$a_\tau/a_\mu$ & $(m_\tau/m_\mu)^2 = 283$ & Belle II 2027-28 \\
			$m_\tau/m_\mu$ & $\approx 16.8$ (from torus) & Confirmed ✓ \\
			Spin-statistics & From topology & Confirmed ✓ \\
			Fractal damping & $\propto e^{-\xi n^2}$ & Rydberg atoms \\
			\bottomrule
		\end{tabular}
		\caption{Experimental tests of the T0 Dirac formulation}
	\end{table}
	
	\section{Comparison with Standard Formulation}
	
	\begin{table}[h]
		\centering
		\begin{tabular}{lcc}
			\toprule
			\textbf{Aspect} & \textbf{Standard Dirac} & \textbf{T0 Dirac} \\
			\midrule
			Mass & Constant $m$ & Dynamic $m(x,t)$ \\
			Metric & Minkowski $\eta_{\mu\nu}$ & Fractal $g_{\mu\nu}^{\text{(frak)}}$ \\
			Spin & Matrix property & Topological winding \\
			Dimension & $D = 4$ & $D_f = 3 - \xi$ in space \\
			Topology & None & Torus $(n_\theta, n_\phi)$ \\
			Coupling & Ad-hoc & Time-mass duality \\
			Predictions & Qualitative & Testable ratios \\
			\bottomrule
		\end{tabular}
		\caption{Standard vs. T0 Dirac formulation}
	\end{table}
	
	\section{Limits and Open Questions}
	
	\subsection{What Works}
	
	\begin{itemize}
		\item ✓ Clifford algebra structure clearly defined
		\item ✓ Spin interpretable as topology
		\item ✓ Ratio predictions parameter-free
		\item ✓ Belle II test possible
	\end{itemize}

	\subsection{Honesty About Limits}
	
	As in the Standard Model (hadronic contributions), there are areas where the fundamental 
	theory is clear but explicit calculations are too complex. This is **not a fault of the 
	theory**, but a realistic assessment of mathematical challenges.
	
	\section*{References and Further Reading}
	
	\begin{thebibliography}{9}
		
		\bibitem{T0Foundation}
		J. Pascher,
		\textit{The T0 Foundation: Time-Mass Duality and Fractal Geometry},
		T0-Time-Mass-Duality Repository,
		2026.
		
		\bibitem{XiNarrative}
		J. Pascher,
		\textit{The Xi Narrative: From a Single Number to the Fine-Structure Constant},
		145\_FFGFT\_donat-teil1\_En.pdf,
		2025.
		
		\bibitem{CliffordGeometricAlgebra}
		D. Hestenes,
		\textit{Space-Time Algebra},
		Gordon and Breach, 1966.
		Provides the mathematical foundation for geometric Clifford algebra formulations.
		
		\bibitem{CliffordSpinors}
		P. Lounesto,
		\textit{Clifford Algebras and Spinors},
		Cambridge University Press, 2001.
		Comprehensive treatment of Clifford algebras with applications to spinors.
		
		\bibitem{DiracOriginal}
		P. A. M. Dirac,
		\textit{The Quantum Theory of the Electron},
		Proc. R. Soc. Lond. A, 117, 610–624, 1928.
		The original paper introducing the Dirac equation.
		
		\bibitem{TorusTopologySpin}
		J. Williamson and M. B. van der Mark,
		\textit{Is the Electron a Photon with Toroidal Topology?},
		Annales de la Fondation Louis de Broglie, 22, 133–167, 1997.
		\href{https://fondationlouisdebroglie.org/IMG/pdf/22_2_133.pdf}{[PDF]}
		
		\bibitem{BelleIITauG2}
		Belle II Collaboration,
		\textit{Prospects for Measuring the Anomalous Magnetic Moment of the Tau Lepton at Belle II},
		Belle II Note 0123, 2024.
		\href{https://www.belle2.org}{[Belle II Website]}
		
		\bibitem{FermilabMuonG2}
		Muon g-2 Collaboration,
		\textit{Measurement of the Positive Muon Anomalous Magnetic Moment to 0.20 ppm},
		Phys. Rev. Lett. 131, 161802, 2023.
		Latest results from Fermilab.
		
		\bibitem{GeometricTopologyPhysics}
		M. Nakahara,
		\textit{Geometry, Topology and Physics},
		IOP Publishing, 2003.
		Excellent resource for tetrad formalism and differential geometry in physics.
		
		\bibitem{FractalGeometry}
		K. Falconer,
		\textit{Fractal Geometry: Mathematical Foundations and Applications},
		Wiley, 2014.
		Standard reference for fractal geometry and Hausdorff dimensions.
		
		\bibitem{TimeMassDualityDerivation}
		J. Pascher,
		\textit{Derivation of Time-Mass Duality from Planck Relations},
		T0\_xi\_ursprung.pdf,
		2025.
		
		\bibitem{T0DiracSimplified}
		J. Pascher,
		\textit{Dirac Equation in T0 Theory: Geometric Clifford-Algebra Formulation},
		Document 050\_dirac\_geometric,
		2026.

	\end{thebibliography}
	
\end{document}