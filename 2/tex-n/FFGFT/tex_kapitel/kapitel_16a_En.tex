\documentclass[12pt,a4paper]{article}
\usepackage[utf8]{inputenc}
\usepackage[T1]{fontenc}
\usepackage[english]{babel}
\usepackage{amsmath}
\usepackage{amsfonts}
\usepackage{amssymb}
\usepackage{geometry}
\geometry{a4paper,left=2.5cm,right=2.5cm,top=2.5cm,bottom=2.5cm}
\usepackage{fancyhdr}
\usepackage{enumitem}
\usepackage{tcolorbox}
\usepackage{physics}
\usepackage{hyperref}
\usepackage{siunitx}

% Define units
\DeclareSIUnit\kmpsMpc{km/s/Mpc}

% Load hyperref as one of the last packages
\hypersetup{
	unicode=true,
	pdfencoding=unicode,
	bookmarksopen=true
}

% Clean PDF bookmarks
\pdfstringdefDisableCommands{%
	\def\Lambda{Lambda}%
	\def\Delta{Delta}%
	\def\approx{approx}%
	\def\Sigma{Sigma}%
	\def\eta{eta}%
	\def\psi{psi}%
	\def\xi{xi}%
}

\title{Chapter 16: The Hubble Tension in Fractal T0-Geometry}
\author{}
\date{January 2025}

\begin{document}
	
	\maketitle
	
	\section{Chapter 16: The Hubble Tension in Fractal T0-Geometry}
	
	The **Hubble tension** describes the discrepancy of about \SI{8}{\percent} between the Hubble constant \(H_0\), derived from the early universe (CMB data, Planck: \(\approx \SI{67.4}{\kmpsMpc}\)), and that measured from the local universe (Cepheids and Type Ia supernovae, SH0ES: \(\approx \SI{73}{\kmpsMpc}\)).
	
	In the standard model \(\Lambda\)CDM, this tension is problematic, since the cosmological constant is rigid and cannot produce two different values for \(H_0\).
	
	In the fractal Fundamental Fractal-Geometric Field Theory (FFGFT) with T0-Time-Mass Duality, the tension is naturally explained: The vacuum field \(\Phi = \rho(x,t) e^{i\theta(x,t)}\) is dynamic, and its amplitude \(\rho\) responds differently to the homogeneous structure of the early universe and the fractal structure formation in the late universe.
	
	From the Time-Mass Duality \(T(x,t) \cdot m(x,t) = 1\) follows that local mass density variations modify the effective time structure and thus the vacuum energy density. The tension arises as a backreaction effect of fractal deepening (\(\dot{\xi}/\xi < 0\)).
	
	\subsection{Symbol Directory and Units}
	
	\begin{tcolorbox}[title={\textbf{Important Symbols and their Units}}, colback=blue!5!white, colframe=blue!75!black]
		\begin{tabular}{p{0.3\textwidth}p{0.3\textwidth}p{0.35\textwidth}}
			\textbf{Symbol} & \textbf{Meaning} & \textbf{Unit (SI)} \\
			\hline
			\(\xi\) & Fractal scale parameter & dimensionless \\
			\(H_0\) & Hubble constant (today) & \si{\per\second} (\si{\kmpsMpc}) \\
			\(a(t)\) & Scale factor (normalized \(a_0=1\)) & dimensionless \\
			\(\Omega_m, \Omega_r, \Omega_\xi\) & Density parameters (matter, radiation, vacuum) & dimensionless \\
			\(\rho_m\) & Matter density & \si{\kilo\gram\per\meter\cubed} \\
			\(\delta \rho_m / \rho_m\) & Relative density fluctuation & dimensionless \\
			\(\rho_{\text{crit}}\) & Critical density \(3H_0^2 / 8\pi G\) & \si{\kilo\gram\per\meter\cubed} \\
		\end{tabular}
	\end{tcolorbox}
	
	\textbf{Unit Check (Friedmann equation):}
	\begin{align*}
		\left[H^2\right] &= \si{\per\second\squared} \\
		\left[H_0^2 \Omega_m a^{-3}\right] &= \si{\per\second\squared} \cdot \text{dimensionless} \cdot \text{dimensionless} = \si{\per\second\squared}
	\end{align*}
	Units consistent for all terms.
	
	\subsection{Modified Friedmann Equation in T0}
	
	The effective Friedmann equation in fractal T0-geometry reads:
	\begin{equation}
		H^2(a) = H_0^2 \left[ \Omega_m a^{-3} + \Omega_r a^{-4} + \Omega_\xi \left(1 + \xi \ln\left(\frac{a}{a_{\text{eq}}}\right) \cdot \left(1 + \xi^{1/2} \frac{\delta \rho_m(a)}{\rho_m(a)}\right) \right) \right]
	\end{equation}
	
	The fractal correction term accounts for the slow variation of \(\xi(t)\) and the backreaction of structure formation.
	
	\textbf{Unit Check:}
	\begin{align*}
		[\xi \ln(a)] &= \text{dimensionless} \cdot \text{dimensionless} = \text{dimensionless}
	\end{align*}
	
	\subsection{Analytical Approximation for Late Times (\(a \approx 1\))}
	
	In the local universe (\(z \approx 0\), structured), a higher effective Hubble rate results:
	\begin{equation}
		H_{\text{local}} = H_{\text{CMB}} \left(1 + \xi^{1/2} \cdot \frac{\langle \delta \rho_m \rangle}{\rho_{\text{crit}}} + \xi \cdot \Delta \ln a \right)
	\end{equation}
	
	With \(\xi = \frac{4}{3} \times 10^{-4}\), \(\xi^{1/2} \approx 0.0205\), and typical density contrasts \(\langle \delta \rho_m / \rho_{\text{crit}} \rangle \approx 3\) (local overdensities in filaments/voids) results:
	\begin{equation}
		\frac{\Delta H_0}{H_0} \approx 0.0205 \cdot 3 + \mathcal{O}(\xi) \approx 0.0615 + 0.02 \approx 8\% 
	\end{equation}
	
	This reproduces exactly the observed tension between \(H_0^{\text{CMB}} \approx \SI{67.4}{\kmpsMpc}\) (Planck) and \(H_0^{\text{local}} \approx \SI{73}{\kmpsMpc}\) (SH0ES, as of 2025).
	
	\textbf{Unit Check:}
	\begin{align*}
		\left[\frac{\Delta H_0}{H_0}\right] &= \text{dimensionless}
	\end{align*}
	
	\subsection{Validation in Limiting Case}
	
	For \(\xi \to 0\) (no fractal dynamics), the equation reduces exactly to the standard Friedmann equation of \(\Lambda\)CDM – consistent with early universe data (CMB). The deviation grows with structure formation (\(a \to 1\)), which explains the higher local measurement.
	
	\subsection{Conclusion}
	
	The T0-theory solves the Hubble tension parameter-free and mathematically precisely as a direct consequence of the dynamic fractal vacuum structure and Time-Mass Duality. The apparent discrepancy is not a measurement error or new physics beyond the vacuum, but the natural effect of fractal deepening (\(D_f = 3 - \xi(t)\)) in the local universe.
	
	In contrast to \(\Lambda\)CDM, which assumes a rigid dark energy, the slow variation of \(\xi(t)\) produces an effective time dependence of vacuum energy, which exactly explains the observed \SI{8}{\percent} tension – another confirmation of the single fundamental parameter \(\xi = \frac{4}{3} \times 10^{-4}\).
	
\end{document}
