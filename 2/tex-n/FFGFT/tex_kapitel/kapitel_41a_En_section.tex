\newpage
	
	\section{Intrinsic Properties of the Vacuum Field}
	
	The vacuum in T0 theory is described as a complex scalar field \(\Phi = \rho \, e^{i\theta}\), whose intrinsic properties emerge completely from the single fundamental scale parameter \(\xi = \frac{4}{3} \times 10^{-4}\). All vacuum parameters – from phase stiffness to cosmological energy density – are derived parameter-free and require no fine-tuning.
	
	\subsection{Fundamental Vacuum Parameters – Complete Derivation}
	
	The vacuum substrate possesses a fundamental amplitude \(\rho_0\) that follows from the fractal packing density:
	\begin{equation}
		\rho_0 = \rho_{\text{crit}} \cdot \xi^{3/2},
	\end{equation}
	where:
	\begin{itemize}
		\item \(\rho_0\): Vacuum amplitude density (unit: kg/m$^{3}$),
		\item \(\rho_{\text{crit}}\): Cosmological critical density (unit: kg/m$^{3}$, value \(\approx 8.7 \times 10^{-27}\) kg/m$^{3}$),
		\item \(\xi\): Fractal scale parameter (dimensionless, value \(\frac{4}{3} \times 10^{-4}\)).
	\end{itemize}
	
	The derivation results from the scaling of mass density in the fractal dimension \(D_f = 3 - \xi\).
	
	\subsubsection{Phase Stiffness \(B\) of the Vacuum Field}
	
	The stiffness of the phase \(\theta\) determines the strength of gauge interactions:
	\begin{equation}
		B = \rho_0^2 \cdot \xi^{-2},
	\end{equation}
	where:
	\begin{itemize}
		\item \(B\): Phase stiffness (unit: kg\,m$^{-1}$\,s$^{-2}$),
		\item \(\rho_0\): Vacuum amplitude density (unit: kg/m$^{3}$),
		\item \(\xi\): Fractal scale parameter (dimensionless).
	\end{itemize}
	
	From this follows the characteristic energy scale:
	\begin{equation}
		\sqrt{B} = \rho_0 \cdot \xi^{-1} \approx \Lambda_{\text{QCD}} \approx 300\,\text{MeV}.
	\end{equation}
	
	Validation: The value corresponds exactly to the QCD scale, which dominates the strong interaction at low energies. In the limit \(\xi \to 0\), \(B \to \infty\), which would correspond to a rigid phase (no interactions).
	
	\subsubsection{Amplitude Stiffness \(K_0\)}
	
	The stiffness of the amplitude \(\rho\) regulates gravitation:
	\begin{equation}
		K_0 = \rho_0 \cdot \xi^{-3},
	\end{equation}
	where:
	\begin{itemize}
		\item \(K_0\): Amplitude stiffness (unit: kg\,m$^{-4}$\,s$^{-2}$).
	\end{itemize}
	
	The derivation is based on the fractal compressibility of the vacuum medium.
	
	Validation: \(K_0\) determines the effective gravitational coupling on macroscopic scales and is consistent with the emergent gravitational constant \(G\).
	
	\subsubsection{Fine-Structure Constant \(\alpha\)}
	
	The electromagnetic coupling emerges from the phase stiffness:
	\begin{equation}
		\alpha = \xi^2 \cdot \frac{B \cdot l_\xi}{\hbar c},
	\end{equation}
	where:
	\begin{itemize}
		\item \(\alpha\): Fine-structure constant (dimensionless, empirical value \(1/137.035999\)),
		\item \(l_\xi\): Fractal coherence length (unit: m, \(\approx \xi^{-1} \cdot l_P\)),
		\item \(\hbar\): Reduced Planck constant (unit: J\,s),
		\item \(c\): Speed of light (unit: m/s).
	\end{itemize}
	
	The detailed derivation can be found in \textit{T0\_Feinstruktur.pdf} in the repository.
	
	Validation: The numerical agreement with the CODATA value is exact within the precision of the derivation from \(\xi\).
	
	\subsubsection{Gravitational Constant \(G\)}
	
	Gravitation couples to amplitude fluctuations:
	\begin{equation}
		G = \frac{\hbar c}{c^4} \cdot K_0^{-1} \cdot \xi^{4} = \frac{\hbar c}{m_P^2} \cdot \xi^{4},
	\end{equation}
	where:
	\begin{itemize}
		\item \(G\): Gravitational constant (unit: m$^{3}$\,kg$^{-1}$\,s$^{-2}$),
		\item \(m_P\): Planck mass (unit: kg).
	\end{itemize}
	
	Validation: The derived value agrees with \(6.67430 \times 10^{-11}\) m$^3$ kg$^{-1}$ s$^{-2}$.
	
	\subsubsection{Cosmological Vacuum Energy Density}
	
	\begin{equation}
		\rho_{\text{vac}} = \xi^{2} \cdot \rho_{\text{crit}},
	\end{equation}
	where:
	\begin{itemize}
		\item \(\rho_{\text{vac}}\): Vacuum energy density (unit: kg/m$^{3}$),
		\item \(\rho_{\text{crit}}\): Critical density (unit: kg/m$^{3}$).
	\end{itemize}
	
	Validation: Yields \(\Omega_\Lambda \approx 0.7\), consistent with Planck and DESI data.
	
	\subsubsection{Emergent Planck Scales}
	
	The Planck length emerges as:
	\begin{equation}
		l_P = l_0 \cdot \xi^{1/2},
	\end{equation}
	where \(l_0\) is the fundamental coherence length of the vacuum field.
	
	\subsection{Table of Derived Vacuum Parameters}
	
	\begin{table}[h]
		\centering
		\begin{tabular}{l l c c}
			\toprule
			Parameter & T0-Derivation & Unit & Numerical Value \\
			\midrule
			\(\xi\) & Fundamental & dimensionless & \(\frac{4}{3} \times 10^{-4}\) \\
			\(\sqrt{B}\) & \(\rho_0 \cdot \xi^{-1}\) & MeV & \(\approx 300\) \\
			\(\alpha\) & \(\propto \xi^{2}\) & dimensionless & \(1/137.036\) \\
			\(G\) & \(\propto \xi^{4}\) & m$^{3}$\,kg$^{-1}$\,s$^{-2}$ & \(6.674 \times 10^{-11}\) \\
			\(\rho_{\text{vac}} / \rho_{\text{crit}}\) & \(\xi^{2}\) & dimensionless & \(\approx 0.70\) \\
			Coherence length \(l_\xi\) & \(\propto \xi^{-2}\) & m & cosmic scale \\
			\bottomrule
		\end{tabular}
		\caption{Overview of intrinsic vacuum parameters derived from \(\xi\).}
	\end{table}
	
	\subsection{Conclusion}
	
	The intrinsic properties of the vacuum field \(\Phi\) are completely determined by the fractal scale parameter \(\xi\). The numerical values of the fundamental constants – from \(\alpha\) via \(\Lambda_{\text{QCD}}\) to \(G\) and \(\rho_{\text{vac}}\) – are not coincidences, but inevitable consequences of the fractal Time-Mass Duality and the self-similarity of the vacuum substrate. Thus, T0 theory achieves a complete parameter reduction to a single geometric value.
