\newpage
	
	\section{Quantum Bits, Schrödinger Equation and Dirac Equation in T0}
	
	T0-Time-Mass Duality interprets quantum phenomena not as separate postulates, but as emergent consequences of fractal vacuum dynamics. Quantum bits (qubits), the Schrödinger equation and the Dirac equation are uniformly derived from the vacuum field \(\Phi = \rho \, e^{i\theta}\) with the single parameter \(\xi = \frac{4}{3} \times 10^{-4}\), consistent with Time-Mass Duality and fractal geometry. This chapter integrates the simplified representation of the Dirac equation as field node dynamics, which reduces the complex matrix structure to simple field excitations, considering the geometric foundations and natural units.
	
	\subsection{Quantum Bits as Vacuum Phase States}
	
	In quantum information science, a qubit is a state in two-dimensional Hilbert space:
	\begin{equation}
		|\psi\rangle = \alpha |0\rangle + \beta |1\rangle, \quad |\alpha|^2 + |\beta|^2 = 1,
	\end{equation}
	where:
	\begin{itemize}
		\item \(|\psi\rangle\): Qubit state (dimensionless, as vector in Hilbert space),
		\item \(\alpha, \beta\): Complex amplitudes (dimensionless, with normalization condition),
		\item \(|0\rangle, |1\rangle\): Basis states (dimensionless).
	\end{itemize}
	
	In T0, a qubit is a stable phase configuration of the vacuum field:
	\begin{equation}
		\theta_{\text{qubit}} = \theta_0 + \xi \cdot (\phi_0 |0\rangle + \phi_1 |1\rangle),
	\end{equation}
	where:
	\begin{itemize}
		\item \(\theta_{\text{qubit}}\): Phase configuration for the qubit (dimensionless),
		\item \(\theta_0\): Global vacuum phase (dimensionless),
		\item \(\phi_0, \phi_1\): Fractally scaled phase angles (dimensionless),
		\item \(\xi\): Fractal scale parameter (dimensionless, value \(\frac{4}{3} \times 10^{-4}\)).
	\end{itemize}
	
	Superposition emerges from the global coherence of the vacuum phase \(\theta\), regulated by fractal self-similarity \(\xi\). The Bloch sphere arises from the cylindrical geometry of the complex field (\(\rho\) as radius, \(\theta\) as angle):
	\begin{equation}
		|\psi\rangle = \cos\left(\frac{\vartheta}{2}\right) |0\rangle + e^{i\varphi} \sin\left(\frac{\vartheta}{2}\right) |1\rangle,
	\end{equation}
	where:
	\begin{itemize}
		\item \(\vartheta\): Polar angle (dimensionless, \(\propto \xi \cdot \Delta \rho\)),
		\item \(\varphi\): Azimuthal angle (dimensionless, \(\propto \Delta \theta\)).
	\end{itemize}
	
	Qubit gates like the Hadamard gate are phase rotations:
	\begin{equation}
		H = \frac{1}{\sqrt{2}} \begin{pmatrix} 1 & 1 \\ 1 & -1 \end{pmatrix}, \quad \Delta \theta = \frac{\pi}{\xi^{1/2}},
	\end{equation}
	where:
	\begin{itemize}
		\item \(H\): Hadamard matrix (dimensionless),
		\item \(\Delta \theta\): Phase shift (dimensionless).
	\end{itemize}
	
	The derivation is based on the variation of the fractal action, where \(\xi\) determines the coherence length. T0 predicts robust qubits at room temperature through stable phase configurations.
	
	Validation: In the limit \(\xi \to 0\), the qubit reduces to classical bits, consistent with macroscopic physics.
	
	\subsection{Derivation of Schrödinger Equation from T0}
	
	The Schrödinger equation
	\begin{equation}
		i \hbar \frac{\partial \psi}{\partial t} = -\frac{\hbar^2}{2m} \nabla^2 \psi + V \psi
	\end{equation}
	emerges in T0 from the phase dynamics of the vacuum field.
	
	The T0 vacuum field \(\Phi = \rho \, e^{i\theta}\) obeys the fractal wave equation:
	\begin{equation}
		\square \Phi + \xi \cdot B (\nabla \theta)^2 \Phi = 0,
	\end{equation}
	where:
	\begin{itemize}
		\item \(\square\): D'Alembertian operator (unit: m$^{-2}$ or s$^{-2}$),
		\item \(\Phi\): Vacuum field (dimensionless),
		\item \(B\): Phase stiffness (unit: kg\,m$^{-1}$\,s$^{-2}$),
		\item \(\nabla \theta\): Phase gradient (dimensionless per m),
		\item \(\xi\): Fractal scale parameter (dimensionless).
	\end{itemize}
	
	In the non-relativistic limit one separates:
	\begin{equation}
		\psi = e^{i \theta / \xi}, \quad \rho \approx \rho_0 + \delta \rho.
	\end{equation}
	where:
	\begin{itemize}
		\item \(\psi\): Wave function (dimensionless),
		\item \(\rho_0\): Vacuum fundamental density (unit: kg/m$^{3}$),
		\item \(\delta \rho\): Density deviation (unit: kg/m$^{3}$).
	\end{itemize}
	
	The variation leads to the Hamilton-Jacobi equation with fractal term:
	\begin{equation}
		\frac{\partial \theta}{\partial t} + \frac{(\nabla \theta)^2}{2m} + V + \xi \cdot \frac{\hbar^2}{2m} \frac{\nabla^2 \sqrt{\rho}}{\sqrt{\rho}} = 0,
	\end{equation}
	where:
	\begin{itemize}
		\item \(\theta\): Phase (dimensionless),
		\item \(m\): Mass (unit: kg),
		\item \(V\): Potential (unit: J),
		\item \(\hbar\): Reduced Planck constant (unit: J\,s).
	\end{itemize}
	
	With Madelung transformation follows the Schrödinger equation, where the fractal term regularizes divergences.
	
	Validation: In the limit \(\xi \to 0\) reduces to the classical Hamilton-Jacobi equation.
	
	\subsection{Derivation of Dirac Equation from T0}
	
	The Dirac equation
	\begin{equation}
		i \hbar \gamma^\mu \partial_\mu \psi - m c \psi = 0
	\end{equation}
	emerges in T0 from multi-component vacuum fields, but is simplified to field node dynamics.
	
	In the detailed T0 integration (natural units \(\hbar = c = 1\)), the modified Dirac equation becomes:
	\begin{equation}
		i\gamma^{\mu}(\partial_{\mu} + \Gamma_{\mu}^{(T)}) \psi - m(\vec{x},t) \psi = 0,
	\end{equation}
	where:
	\begin{itemize}
		\item \(\gamma^\mu\): Dirac matrices (dimensionless),
		\item \(\partial_\mu\): Partial derivative operator (unit: m$^{-1}$ or s$^{-1}$),
		\item \(\Gamma_{\mu}^{(T)}\): Time-field connection (unit: m$^{-1}$ or s$^{-1}$, \(\Gamma_{\mu}^{(T)} = -\frac{\partial_{\mu} m}{m^2}\)),
		\item \(m(\vec{x},t)\): Local mass density (unit: kg/m$^{3}$),
		\item \(\psi\): Dirac spinor (dimensionless).
	\end{itemize}
	
	The derivation is based on Time-Mass Duality \(T \cdot m = 1\), with \(T\): time field (unit: s/m$^{3}$), and fractal geometry \(\beta = 2Gm/r\) (dimensionless), \(\xi = 2\sqrt{G} \cdot m\) (dimensionless).
	
	Validation: In the weak field limit (\(\beta \ll 1\)) reduces to the standard Dirac equation, consistent with QED precision measurements (e.g., g-2 of the electron).
	
	\subsubsection{Simplified Dirac Equation as Field Node Dynamics}
	
	In the simplified T0 view, the Dirac equation reduces to:
	\begin{equation}
		\square \delta m = 0,
	\end{equation}
	where:
	\begin{itemize}
		\item \(\square\): D'Alembertian operator (unit: m$^{-2}$ or s$^{-2}$),
		\item \(\delta m\): Field node amplitude (unit: kg/m$^{3}$, as density deviation from vacuum ground \(\rho_0\)).
	\end{itemize}
	
	The spinor \(\psi\) becomes a node pattern:
	\begin{equation}
		\psi(x,t) \to \delta m_{\text{fermion}}(x,t) = \delta m_0 \cdot f_{\text{spin}}(x,t),
	\end{equation}
	where:
	\begin{itemize}
		\item \(\delta m_0\): Node amplitude (unit: kg/m$^{3}$),
		\item \(f_{\text{spin}}(x,t)\): Spin structure function (dimensionless, \(f_{\text{spin}} = A \cdot e^{i(\vec{k} \cdot \vec{x} - \omega t + \phi_{\text{spin}})}\)).
	\end{itemize}
	
	Spin-1/2 emerges from node rotation with frequency \(\omega_{\text{spin}} \propto m c^2 / \hbar \cdot \xi\).
	
	The Lagrangian density simplifies to:
	\begin{equation}
		\mathcal{L} = \varepsilon \cdot (\partial \delta m)^2,
	\end{equation}
	where:
	\begin{itemize}
		\item \(\mathcal{L}\): Lagrangian density (unit: J/m$^{3}$),
		\item \(\varepsilon\): Node energy coefficient (unit: J\,s$^{2}$/kg$^{2}$).
	\end{itemize}
	
	Validation: Remains consistent with precision observables such as lepton anomalous magnetic moments (g-2), which are treated quantitatively only in the dedicated anomaly document \texttt{018\_T0\_Anomale-g2-10\_En.tex}.
	
	\subsection{Comparison with Standard Interpretations}
	
	\begin{table}[h]
		\centering
		\begin{tabular}{l l l}
			\toprule
			Aspect & Standard QM & T0 Theory \\
			\midrule
			Qubits & Hilbert space postulate & Emergent phase coherence \\
			Schrödinger & Postulate & Derivation from vacuum dynamics \\
			Dirac & Postulate with matrices & Simplified node dynamics \\
			Measurement problem & Collapse postulate & Phase scrambling \\
			\bottomrule
		\end{tabular}
		\caption{Comparison of standard QM and T0.}
	\end{table}
	
	T0 solves paradoxes through deterministic node dynamics, consistent with Time-Mass Duality.
	
	\subsection{Conclusion}
	
	Quantum bits, Schrödinger and Dirac equations emerge in T0 parameter-free from fractal vacuum dynamics with \(\xi\). The simplified Dirac equation as field nodes reduces complexity to simple excitations, unifies fermions and bosons and resolves dualities – an inevitable consequence of the vacuum substrate in FFGFT.
