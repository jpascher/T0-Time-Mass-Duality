\documentclass[12pt,a4paper]{article}
\usepackage[utf8]{inputenc}
\usepackage[T1]{fontenc}
\usepackage[english]{babel}
\usepackage{amsmath}
\usepackage{amsfonts}
\usepackage{amssymb}
\usepackage{geometry}
\geometry{a4paper,left=2.5cm,right=2.5cm,top=2.5cm,bottom=2.5cm}
\usepackage{fancyhdr}
\usepackage{enumitem}
\usepackage{tcolorbox}
\usepackage{physics}
\usepackage{hyperref}
\usepackage{siunitx}

\hypersetup{
	unicode=true,
	pdfencoding=unicode,
	bookmarksopen=true
}

\DeclareSIUnit\electronvolt{eV}
\DeclareSIUnit\hertz{Hz}

\pdfstringdefDisableCommands{%
	\def\Lambda{Lambda}%
	\def\Delta{Delta}%
	\def\approx{approx}%
	\def\Sigma{Sigma}%
	\def\eta{eta}%
	\def\psi{psi}%
	\def\xi{xi}%
}

\title{Chapter 31: Photoelectric Effect and Laser Physics in Fractal T0-Geometry}
\author{}
\date{January 2025}

\begin{document}
	
	\maketitle
	
	\section{Chapter 31: Photoelectric Effect and Laser Physics in Fractal T0-Geometry}
	
	The photoelectric effect and the functioning of lasers are considered classic evidence for the quantum nature of light and the necessity of wave-particle duality. In the Standard Model, photons are treated as discrete particles whose energy \(E = h \nu\) overcomes the work function, while intensity only affects the rate. Lasers are based on stimulated emission and population inversion – phenomenologically described by Einstein coefficients.
	
	In the fractal **Fundamental Fractal-Geometric Field Theory (FFGFT)** with **T0-Time-Mass Duality**, duality paradoxes and ad-hoc coefficients completely disappear. Both phenomena emerge parameter-free from the separation of vacuum amplitude \(\rho(x,t)\) (binding, mass-like) and vacuum phase \(\theta(x,t)\) (oscillating, coherent), regulated by the single fundamental parameter \(\xi = \frac{4}{3} \times 10^{-4}\) (dimensionless). Photons are pure phase excitations, electron binding arises from amplitude deformations.
	
	\subsection{Symbol Directory and Units}
	
	\begin{tcolorbox}[title={\textbf{Important Symbols and their Units}}, colback=blue!5!white, colframe=blue!75!black]
		\begin{tabular}{p{0.3\textwidth}p{0.3\textwidth}p{0.35\textwidth}}
			\textbf{Symbol} & \textbf{Meaning} & \textbf{Unit (SI)} \\
			\hline
			\(\xi\) & Fractal scale parameter & dimensionless \\
			\(\rho(x,t)\) & Vacuum amplitude density & \si{\kilo\gram^{1/2}\per\meter^{3/2}} \\
			\(\theta(x,t)\) & Vacuum phase field & dimensionless (\si{\radian}) \\
			\(\Phi(x,t)\) & Complex vacuum field & \si{\kilo\gram^{1/2}\per\meter^{3/2}} \\
			\(\hbar \omega\) & Photon energy & \si{\joule} \\
			\(\omega\) & Angular frequency & \si{\per\second} (\si{\hertz}) \\
			\(E_{\text{bind}}\) & Binding energy/work function & \si{\joule} (\si{\electronvolt}) \\
			\(E_{\text{kin}}\) & Kinetic energy of photoelectron & \si{\joule} \\
			\(\omega_0\) & Threshold frequency & \si{\per\second} \\
			\(\Delta \theta\) & Phase excitation & dimensionless (\si{\radian}) \\
			\(K_0\) & Amplitude stiffness & \si{\kilo\gram^{1/2}\per\meter^{3/2}} \\
			\(V_{\text{atom}}\) & Atomic volume & \si{\meter^3} \\
			\(\gamma\) & Coupling rate & \si{\per\second} \\
			\(\tau_{\text{cav}}\) & Resonator round-trip time & \si{\second} \\
		\end{tabular}
	\end{tcolorbox}
	
	\textbf{Unit check (photon energy):}
	\begin{align*}
		[\hbar \omega] &= \si{\joule\second} \cdot \si{\per\second} = \si{\joule}
	\end{align*}
	Units are consistent.
	
	\subsection{The Problem of Wave-Particle Duality}
	
	Classical wave theory fails at the photoelectric effect (threshold frequency, independent of intensity). Quantum theory postulates discrete photons and Einstein coefficients for stimulated emission – without deeper geometric justification.
	
	\subsection{Photoelectric Effect as Phase Barrier Overcoming}
	
	Photons are pure phase vortices in the vacuum field:
	\begin{equation}
		\hbar \omega = \xi^{-1} \cdot \Delta \theta \cdot k_B T_0,
	\end{equation}
	where \(T_0\) is a fundamental time scale.
	
	Bound electrons create local amplitude barriers:
	\begin{equation}
		E_{\text{bind}} = K_0 \cdot (\delta \rho / \rho_0)^2 \cdot V_{\text{atom}}.
	\end{equation}
	
	Threshold condition:
	\begin{equation}
		\hbar \omega > E_{\text{bind}} \quad \Rightarrow \quad \Delta \theta > \Delta \theta_0 = \xi \cdot \sqrt{\frac{E_{\text{bind}}}{K_0 V_{\text{atom}}}}.
	\end{equation}
	
	Kinetic energy of emitted electron:
	\begin{equation}
		E_{\text{kin}} = \hbar (\omega - \omega_0) = \xi^{-1} \cdot (\Delta \theta - \Delta \theta_0) \cdot k_B T_0.
	\end{equation}
	
	\textbf{Unit check:}
	\begin{align*}
		[E_{\text{kin}}] &= \text{dimensionless} \cdot \text{dimensionless} \cdot \si{\joule} = \si{\joule}
	\end{align*}
	
	Intensity only increases the rate of multiple phase excitations – exactly Einstein's law.
	
	\subsection{Stimulated Emission and Laser as Phase Entrainment}
	
	Stimulated emission arises through resonant phase coupling:
	\begin{equation}
		\dot{\theta}_{\text{atom}} = \gamma \cdot \xi \cdot \sin(\theta_{\text{in}} - \theta_{\text{atom}}).
	\end{equation}
	
	With population inversion (\(\delta \rho > 0\)), amplification occurs:
	\begin{equation}
		\dot{\theta} = \gamma (\delta \rho / \rho_0) \cdot \theta_{\text{in}}.
	\end{equation}
	
	In the resonator, exponential growth:
	\begin{equation}
		\theta(t) = \theta_0 \exp\left( \xi \cdot (\delta \rho / \rho_0) \cdot t / \tau_{\text{cav}} \right).
	\end{equation}
	
	The outcoupled beam is globally phase-synchronized – monochromatic and coherent.
	
	\subsection{Comparison with Other Approaches}
	
	\begin{center}
		\begin{tabular}{p{0.45\textwidth}p{0.45\textwidth}}
			\textbf{Other Models} & \textbf{T0-Fractal FFGFT} \\
			\hline
			Standard QM: Photon as particle, ad-hoc coefficients & Pure phase excitation, emergent coupling \\
			Semiclassical: Wave-particle duality & Unified vacuum field duality \(\rho\)/\(\theta\) \\
			Einstein coefficients: Phenomenological & Geometric entrainment dynamics \\
			Additional postulates & Parameter-free from \(\xi\) \\
		\end{tabular}
	\end{center}
	
	\subsection{Conclusion}
	
	The photoelectric effect and laser physics emerge in T0-theory completely and parameter-free from the duality of vacuum amplitude \(\rho\) (binding) and phase \(\theta\) (light). The threshold effect is barrier overcoming by phase excitation, stimulated emission is resonant entrainment, laser coherence is global phase synchronization. All observed phenomena – threshold frequency, linear kinetic energy, exponential amplification – follow necessarily from the fractal vacuum structure with the single scale parameter \(\xi = \frac{4}{3} \times 10^{-4}\). Wave-particle duality becomes superfluous; everything is geometric dynamics of the dynamic vacuum.
	
\end{document}
