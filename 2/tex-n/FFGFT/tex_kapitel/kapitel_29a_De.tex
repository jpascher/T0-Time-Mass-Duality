\documentclass[12pt,a4paper]{article}
\usepackage[utf8]{inputenc}
\usepackage[T1]{fontenc}
\usepackage[ngerman]{babel}
\usepackage{amsmath}
\usepackage{amsfonts}
\usepackage{amssymb}
\usepackage{geometry}
\geometry{a4paper,left=2.5cm,right=2.5cm,top=2.5cm,bottom=2.5cm}
\usepackage{fancyhdr}
\usepackage{enumitem}
\usepackage{tcolorbox}
\usepackage{physics}
\usepackage{hyperref}
\usepackage{siunitx}

\hypersetup{
	unicode=true,
	pdfencoding=unicode,
	bookmarksopen=true
}

\DeclareSIUnit\radian{rad}

\pdfstringdefDisableCommands{%
	\def\Lambda{Lambda}%
	\def\Delta{Delta}%
	\def\approx{etwa}%
	\def\Sigma{Sigma}%
	\def\eta{eta}%
	\def\psi{psi}%
	\def\xi{xi}%
}

\title{Kapitel 29: Das Delayed-Choice-Quantum-Eraser-Experiment in der fraktalen T0-Geometrie}
\author{}
\date{Januar 2025}

\begin{document}
	
	\maketitle
	
	\section{Kapitel 29: Das Delayed-Choice-Quantum-Eraser-Experiment in der fraktalen T0-Geometrie}
	
	Das **Delayed-Choice-Quantum-Eraser (DCQE)**-Experiment (Kim et al., 2000; Walborn et al., 2002) demonstriert anschaulich die Quantenkomplementarität und Verschränkung. Es scheint Retrokausalität zu implizieren: Eine verzögerte Entscheidung, Which-Path-Information zu löschen oder zu behalten, beeinflusst scheinbar das Interferenzverhalten eines Photons in der Vergangenheit. In der fraktalen **Fundamental Fractal-Geometric Field Theory (FFGFT)** mit **T0-Time-Mass-Dualität** löst sich dieses Paradoxon vollständig auf. Das Phänomen emergiert aus der globalen, fraktalen Kohärenz des Vakuumphasenfeldes \(\theta(x,t)\), reguliert durch den einzigen fundamentalen Parameter \(\xi = \frac{4}{3} \times 10^{-4}\) (dimensionslos). Es gibt keine Retrokausalität – lediglich eine nichtlokale, aber kausale Korrelation in der fraktalen Vakuumstruktur.
	
	In T0 sind Quantenzustände Anregungen des komplexen Vakuumfeldes \(\Phi(x,t) = \rho(x,t) e^{i\theta(x,t)}\). Photonen sind reine Phasenwirbel (\(\delta\rho \approx 0\)), deren Propagation durch Gradienten der Zeitdichte \(T(x,t)\) geleitet wird (Dualität \(T(x,t) \cdot m(x,t) = 1\)). Verschränkung ist globale Phasenkohärenz: \(\theta_{\text{signal}} + \theta_{\text{idler}} = \theta_{\text{total}} =\) konst.
	
	\subsection{Symbolverzeichnis und Einheiten}
	
	\begin{tcolorbox}[title={\textbf{Wichtige Symbole und ihre Einheiten}}, colback=blue!5!white, colframe=blue!75!black]
		\begin{tabular}{p{0.3\textwidth}p{0.3\textwidth}p{0.35\textwidth}}
			\textbf{Symbol} & \textbf{Bedeutung} & \textbf{Einheit (SI)} \\
			\hline
			\(\xi\) & Fraktaler Skalenparameter & dimensionslos \\
			\(\Phi(x,t)\) & Komplexes Vakuumfeld & \si{\kilo\gram^{1/2}\per\meter^{3/2}} \\
			\(\rho(x,t)\) & Vakuum-Amplitudendichte & \si{\kilo\gram^{1/2}\per\meter^{3/2}} \\
			\(\theta(x,t)\) & Vakuumphasenfeld & \si{\radian} (dimensionslos) \\
			\(T(x,t)\) & Zeitdichte & \si{\second\per\meter^3} \\
			\(\psi(x,t)\) & Effektive Wellenfunktion & dimensionslos \\
			\(\Delta\theta\) & Phasenstörung & \si{\radian} \\
			\(l_0\) & Fraktale Korrelationslänge & \si{\meter} \\
			\(\theta_{\text{total}}\) & Globale verschränkte Phase & \si{\radian} \\
			\(\langle \theta(x) \theta(x') \rangle\) & Phasenkorrelation & \si{\radian^2} \\
			\(V\) & Sichtbarkeit der Interferenz & dimensionslos \\
		\end{tabular}
	\end{tcolorbox}
	
	\textbf{Einheitenprüfung (Phasenkorrelation):}
	\begin{align*}
		[\langle \theta \theta \rangle] &= \text{dimensionslos} + \text{dimensionslos} \cdot \ln(\si{\meter}/\si{\meter}) = \text{dimensionslos}
	\end{align*}
	Einheiten konsistent.
	
	\subsection{Das Problem der scheinbaren Retrokausalität}
	
	Im Standardmodell der Quantenmechanik erscheint DCQE paradox: Die totale Verteilung am Signal-Detektor D0 zeigt nie Interferenz. Nur bei Post-Selektion (Korrelation mit Idler-Detektoren) treten Untermengen mit Interferenz (erased) oder Clumping (which-path) auf – auch wenn die Idler-Messung verzögert erfolgt.
	
	Dies führt zu Missverständnissen über Retrokausalität. T0 löst dies parameterfrei durch fraktale Nichtlokalität.
	
	\subsection{Beschreibung des Experiments}
	
	Verschränkte Photonenpaare aus parametrischer Down-Conversion (PDC):
	- Signal-Photon → Doppelspalt → Detektor D0 (beweglich für Scanning).
	- Idler-Photon → verzögertes Setup mit Beam-Splittern und Detektoren (D1–D4).
	
	Ohne Erasure (Which-Path-Detektoren): Keine Interferenz in korrelierten Subsets.  
	Mit Erasure (z. B. Beam-Splitter vor Detektoren): Interferenz in Subsets – verzögerte Wahl klassifiziert nur die Daten.
	
	\subsection{Phasenkohärenz in der T0-Vakuumstruktur}
	
	Die effektive Wellenfunktion ist eine Phasenmodulation:
	\begin{equation}
		\psi(x,t) = e^{i \theta(x,t)/\xi},
	\end{equation}
	da Photonen reine Phase sind (\(\rho \approx \rho_0\)).
	
	Fraktale Korrelation:
	\begin{equation}
		\langle \theta(x) \theta(x') \rangle = \theta_0 + \xi \cdot \ln(|x - x'| / l_0).
	\end{equation}
	
	\textbf{Einheitenprüfung:}
	\begin{align*}
		[\xi \cdot \ln(|x-x'|/l_0)] &= \text{dimensionslos}
	\end{align*}
	
	Für verschränkte Paare:
	\begin{equation}
		\theta_{\text{signal}}(x) + \theta_{\text{idler}}(x') = \theta_{\text{total}} = \text{konstant}.
	\end{equation}
	
	\subsection{Ableitung des Erasure-Effekts}
	
	Which-Path-Markierung stört die Idler-Phase:
	\begin{equation}
		\Delta \theta_{\text{idler}} \approx \pi \quad \Rightarrow \quad \Delta \theta_{\text{signal}} \approx \pi \quad (\text{durch Dualität}),
	\end{equation}
	randomisiert die Phase am D0 → reduzierte Sichtbarkeit \(V \approx 0\).
	
	Erasure (z. B. 50/50 Beam-Splitter):
	\begin{equation}
		\Delta \theta_{\text{idler}} \approx 0 \quad \Rightarrow \quad \Delta \theta_{\text{signal}} \approx 0,
	\end{equation}
	Kohärenz erhalten → \(V \approx 1\) in korrelierten Subsets.
	
	Die "verzögerte Wahl" beeinflusst nur die Post-Selektion der Ereignisse – die globale Phase \(\theta_{\text{total}}\) ist immer kohärent.
	
	Minimale Phasenunsicherheit aus Fraktalität:
	\begin{equation}
		\Delta \theta_{\min} \approx \xi^{3/2} \sqrt{\ln(\xi^{-1})} \approx 4.6 \times 10^{-6}.
	\end{equation}
	
	\subsection{Nichtlokale Korrelation ohne Retrokausalität}
	
	Die Korrelation ist fraktal bedingt:
	\begin{equation}
		\Delta \theta_{\text{signal}} \cdot \Delta \theta_{\text{idler}} \geq \xi.
	\end{equation}
	
	Dies ist deterministisch und kausal – keine Signalübertragung rückwärts.
	
	\subsection{Vergleich mit anderen Interpretationen}
	
	\begin{center}
		\begin{tabular}{p{0.45\textwidth}p{0.45\textwidth}}
			\textbf{Andere Interpretationen} & \textbf{T0-Fraktale FFGFT} \\
			\hline
			Kopenhagen: Kollaps, Beobachter & Deterministisch, vakuumgeometrisch \\
			Many-Worlds: Branching & Einheitliche fraktale Phase \\
			Retrokausalitäts-Modelle: Zeitreise & Keine Retrokausalität nötig \\
			Zusätzliche Annahmen & Parameterfrei aus \(\xi\) \\
		\end{tabular}
	\end{center}
	
	\subsection{Schlussfolgerung}
	
	Das DCQE-Experiment ist in der Fundamentale Fraktalgeometrische Feldtheorie (FFGFT, früher T0-Theorie) kein Paradoxon mehr: Die scheinbare Retrokausalität entsteht aus der globalen, fraktalen Kohärenz des Vakuumphasenfeldes \(\theta(x,t)\). Erasure stellt Kohärenz in korrelierten Subsets wieder her, ohne das vergangene Ereignis zu verändern – lediglich die Klassifikation der Daten. Alles emergiert parameterfrei aus dem einzigen Skalenparameter \(\xi = \frac{4}{3} \times 10^{-4}\), und vereinheitlicht Quantenverschränkung mit der Time-Mass-Dualität als geometrische Notwendigkeit des dynamischen Vakuums.
	
\end{document}
