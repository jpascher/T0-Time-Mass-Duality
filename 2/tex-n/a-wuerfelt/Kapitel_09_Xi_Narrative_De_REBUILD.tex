% Kapitel 09: Kosmologie, Rotverschiebung und CMB
% Korrigiert: ξ_eff = ξ/2 entfernt (nicht physikalisch begründet)
% H₀ = ξ·c = 40 km/s/Mpc als direkte Vorhersage

\chapter{Kosmologie, Rotverschiebung und CMB in der Zeit-Masse-Dualität}

\section{Einführung}

In den vorangegangenen Kapiteln stand die mikroskopische Seite der 
Zeit-Masse-Dualität im Mittelpunkt: Massen, Kopplungen und Quantenphänomene. 
In diesem Kapitel wird gezeigt, wie sich dieselbe Struktur auf großskalige 
Phänomene der Kosmologie auswirkt: Rotverschiebung, kosmische 
Hintergrundstrahlung und effektive Größen wie die Hubble-Skala.

Entscheidend ist: Derselbe Parameter $\xipar = \frac{4}{3} \times 10^{-4}$, 
der auf Teilchenebene die Massen und Kopplungen bestimmt, bestimmt auch die 
kosmologische Rotverschiebung. Es gibt keinen Übergang zu einem anderen 
effektiven Parameter auf großen Skalen -- in einem homogenen, statischen 
Universum gilt überall derselbe $\xipar$.

\section{Rotverschiebung ohne expandierenden Raum}

\subsection{Standard-Interpretation}

Die Standardkosmologie deutet die kosmologische Rotverschiebung hauptsächlich 
als Folge einer expandierenden Raumzeit. Die Wellenlänge eines Photons wird 
mit dem kosmischen Skalenfaktor $a(t)$ mitgedehnt:

\begin{equation}
\frac{\lambda_{\text{obs}}}{\lambda_{\text{emit}}} = \frac{a(t_{\text{obs}})}{a(t_{\text{emit}})} = 1 + z
\end{equation}

\subsection{Zeit-Masse-Dualität Interpretation}

Im Rahmen der Zeit-Masse-Dualität wird die beobachtete Rotverschiebung 
als Folge der fraktalen Tiefenstruktur der Raumzeit verstanden. Ein Photon, 
das durch den fraktalen Raum mit $D_f = 3 - \xipar$ propagiert, verliert 
kontinuierlich Energie an das dynamische Vakuumfeld.

Die T0-Rotverschiebung:

\begin{equation}
z_{\text{T0}} = \int_0^d \xipar(r) \frac{E_\gamma(r)}{E_{\gamma,0}} dr
\end{equation}

Für ein homogenes $\xipar$-Feld vereinfacht sich dies zu:

\begin{equation}
\boxed{z_{\text{T0}} \approx \xipar \cdot d}
\label{eq:z_T0_linear}
\end{equation}

wobei $d$ die Distanz in Megaparsec ist. Diese lineare Beziehung ist die 
T0-Version des Hubble-Gesetzes.

\section{Der Hubble-Parameter}

\subsection{Direkte T0-Vorhersage}

Aus $z = \xipar \cdot d$ folgt unmittelbar das Hubble-Gesetz $z = (H_0/c) \cdot d$ 
mit:

\begin{equation}
\boxed{H_0^{\text{T0}} = \xipar \cdot c = \frac{4}{3} \times 10^{-4} \times 299\,792\,\text{km/s} \approx 40{,}0\,\text{km/s/Mpc}}
\label{eq:H0_T0}
\end{equation}

Dies ist eine parameterfreie Vorhersage: derselbe $\xipar$, der die 
Feinstrukturkonstante auf $0{,}003\%$ und die Leptonenmassen auf Promille-Niveau 
vorhersagt, liefert auch den Hubble-Parameter.

\subsection{Diskrepanz zum Standardwert}

Der experimentell bestimmte Hubble-Parameter beträgt $H_0^{\text{exp}} \approx 
67$--$73\,$km/s/Mpc, je nach Messmethode. Das Verhältnis:

\begin{equation}
\frac{H_0^{\text{exp}}}{H_0^{\text{T0}}} = \frac{70}{40{,}0} \approx 1{,}75 \approx \frac{7}{4}
\label{eq:H0_ratio}
\end{equation}

beträgt also etwa $7/4$ (Genauigkeit: $0{,}07\%$).

\begin{remark}[Modellabhängigkeit des experimentellen $H_0$]
	Der ``experimentelle'' Wert $H_0 \approx 70\,$km/s/Mpc ist keine 
	modellunabhängige Messung. Er wird aus Rohdaten (Rotverschiebungen und 
	Helligkeiten von Standardkerzen) unter der Annahme eines expandierenden 
	Friedmann-Universums extrahiert. Insbesondere:
	\begin{itemize}
	\item Die Luminositätsdistanz $d_L = d \cdot (1+z)$ setzt einen 
	kosmischen Skalenfaktor voraus, den es in T0 nicht gibt.
	\item In T0 ist die physikalische Distanz $d$ direkt messbar, ohne 
	$(1+z)$-Korrektur.
	\item Die modellunabhängige Beobachtung ist die lineare Beziehung 
	$z \propto d$ für kleine $z$. Deren Proportionalitätsfaktor wird 
	in verschiedenen kosmologischen Modellen verschieden interpretiert.
	\end{itemize}
	Die Diskrepanz um den Faktor $7/4$ könnte daher teilweise oder vollständig 
	aus der unterschiedlichen Dateninterpretation folgen. Bemerkenswert ist, 
	dass $7/4$ ein rationaler Bruch ist, der möglicherweise eine geometrische 
	Bedeutung im Torsionskristall-Formalismus hat (4 Dimensionen des Torus 
	$\mathbb{R}^3 \times S^1$, 7 Symmetrieklassen des Kristallgitters). 
	Dies ist eine offene Forschungsfrage.
\end{remark}

\begin{remark}[Warum nicht $\xi_{\text{eff}} = \xi/2$?]
	In einer früheren Formulierung (Ref.\ 201, DVFT) wurde ein effektiver 
	Parameter $\xi_{\text{eff}} = \xi/2$ für kosmologische Skalen postuliert. 
	Diese Halbierung ist jedoch nicht gerechtfertigt:
	\begin{itemize}
	\item In einem homogenen statischen Universum gibt es keinen 
	Mittelungseffekt, der $\xi$ reduzieren könnte.
	\item Die Energiedichte des Vakuumfelds $|\Phi|^2 = \rho^2$ ist 
	phasenunabhängig -- eine $\cos^2$-Mittelung setzt eine unbegründete 
	U(1)-Symmetriebrechung voraus.
	\item Numerisch verschlechtert $\xi/2$ die Übereinstimmung: 
	$H_0(\xi/2) = 20\,$km/s/Mpc liegt weiter vom Experiment entfernt 
	als $H_0(\xi) = 40\,$km/s/Mpc.
	\item Alle Messungen sind lokal. Da die T0-Theorie überall denselben 
	$\xi$-Wert verwendet (Massen, Kopplungen, Gravitation), muss auch die 
	Rotverschiebung mit demselben $\xi$ berechnet werden.
	\end{itemize}
\end{remark}

\section{CMB-Temperatur}

Die CMB-Temperatur:

\begin{equation}
T_{\text{CMB}} = 2.7255\,\text{K}
\end{equation}

wird in der T0-Theorie als thermodynamischer Gleichgewichtszustand der 
$\xipar$-Geometrie interpretiert, nicht als Relikt eines Urknalls. Das 
dynamische Vakuumfeld $\Phi = \rho e^{i\theta}$ hat eine intrinsische 
Phasenentwicklung $\dot{\theta} = m = 1/T$ (aus der Zeit-Masse-Dualität). 
Die CMB-Strahlung ist das thermische Gleichgewichtsspektrum dieses 
universellen Vakuumfelds, dessen Temperatur durch die geometrischen 
Parameter $\xipar$ und $f = 7500$ festgelegt wird.

\section{Statisches Universum im 4D-Torus}

Die T0-Theorie beschreibt ein statisches Universum ohne globale Expansion. 
Die Raumzeit hat die Topologie $\mathbb{R}^3 \times S^1$, wobei $S^1$ die 
Phasenrichtung des Vakuumfelds ist (nicht ``Zeit'' im klassischen Sinn, 
da $T \cdot m = 1$ die Zeit als Kehrwert der Masse definiert).

Das Universum ist ``unendlich'' nicht im Sinne einer unendlichen Ausdehnung, 
sondern weil der Torus in sich geschlossen ist -- es gibt keinen Rand und 
keine Grenze. In dieser Sicht:

\begin{itemize}
\item Rotverschiebung entsteht durch Energieverlust im fraktalen Vakuumfeld, 
nicht durch Expansion
\item Die Hubble-Relation $z = \xipar \cdot d$ ist eine direkte Konsequenz 
der fraktalen Dimension $D_f = 3 - \xipar$
\item Dunkle Energie ist keine separate Substanz, sondern manifestiert sich 
als effektive Eigenschaft des dynamischen Vakuumfelds $\Phi$
\item Die großskalige Homogenität folgt aus der Topologie des Torus, ohne 
Inflation
\end{itemize}

JWST-Beobachtungen entwickelter Galaxien bei $z > 10$, die im 
Standardmodell unerwartet früh erscheinen, sind im T0-Bild natürlich, da 
die Entwicklungszeit unbegrenzt ist.

\section{Zusammenfassung}

Die kosmologischen Vorhersagen der T0-Theorie folgen direkt aus $\xipar$:

\begin{itemize}
\item Rotverschiebung: $z = \xipar \cdot d$ (Energieverlust im Vakuumfeld)
\item Hubble-Parameter: $H_0^{\text{T0}} = \xipar \cdot c \approx 40\,$km/s/Mpc 
(parameterfreie Vorhersage)
\item Diskrepanz zu $H_0^{\text{exp}} \approx 70\,$km/s/Mpc: Faktor $\approx 7/4$ 
(modellabhängige Dateninterpretation, offene Forschungsfrage)
\item CMB als Gleichgewichtszustand der Vakuumgeometrie
\item Statisches Universum im 4D-Torus $\mathbb{R}^3 \times S^1$
\end{itemize}