% Chapter 06: Units, Scales and Constants from xi
% Completely rewritten with correct formulas
% Basis: 013_T0_SI_De.tex, 015_NatEinheitenSystematik_De.tex

\chapter{Units, Scales and Constants from $\xi$}

\section{Introduction}

A central promise of the FFGFT is that all fundamental constants of physics 
can be derived from the single parameter $\xipar$. In this chapter, we show 
how this works concretely -- from the gravitational constant $G$ through 
the Planck length $l_P$ to the Boltzmann constant $k_B$.

\section{Natural Units}

\subsection{The Concept}

In theoretical physics, \textbf{natural units} are frequently used, in which 
fundamental constants are set to 1:

\begin{equation}
\hbar = c = 1
\label{eq:natural_units}
\end{equation}

In this system, all quantities have dimensions of energy $E$ (or powers thereof):

\begin{align}
[M] &= [E] \quad \text{(from } E = mc^2\text{)} \\
[L] &= [E^{-1}] \quad \text{(from } \lambda = \hbar/p\text{)} \\
[T] &= [E^{-1}] \quad \text{(from } \omega = E/\hbar\text{)}
\end{align}

\subsection{Dimensional Analysis of the Gravitational Constant}

The gravitational constant has the dimension in natural units:

\begin{equation}
[G] = [M^{-1}L^3T^{-2}] = [E^{-1}][E^{-3}][E^2] = [E^{-2}]
\label{eq:G_dimension}
\end{equation}

\section{Derivation of the Gravitational Constant}

\subsection{Fundamental T0 Formula}

The gravitational constant follows from $\xipar$ and the electron mass (a 
detailed derivation with dimensional analysis and comparison of alternative 
representations can be found in Chapter 7):

\begin{equation}
G = \frac{\xipar^2}{4 m_e}
\label{eq:G_fundamental}
\end{equation}

in natural units.

\begin{remark}[Alternative representation: $G = \xi/2$]
	In the torsion crystal formalism (Ref.\ 149), $G$ is also written as 
	$G = \xi/2$. This is consistent, since there the reference mass $m = \xi/2$ 
	is used (natural mass scale of the lattice), so that 
	$G = \xi^2/(4 \cdot \xi/2) = \xi/2$. The form $G = \xi^2/(4 m_e)$ used here 
	makes the dependence on the electron mass explicit, which is more useful for 
	SI conversions and numerical verification. Both representations are equivalent 
	in their respective unit systems.
\end{remark}

\subsection{Complete Formula with SI Conversion}

For conversion to SI units, we need the conversion factor:

\begin{equation}
\boxed{G_{\text{SI}} = \frac{\xipar^2}{4 m_e} \times C_{\text{conv}}}
\label{eq:G_complete}
\end{equation}

where:
\begin{itemize}
\item $\xipar = \frac{4}{3} \times 10^{-4}$ (geometric parameter)
\item $m_e = 0.511$ MeV (electron mass, already fractally corrected via $v = 246\,$GeV)
\item $C_{\text{conv}} = 7.783 \times 10^{-3}$ (conversion factor from $\hbar$, $c$)
\end{itemize}

\begin{remark}[Historical factor $K_{\text{frak}}$]
	In earlier formulations, an additional factor $K_{\text{frak}} = 0.986$ appeared 
	in the $G$ formula. In the modern formulation, this fractal correction is already 
	absorbed in the measured Higgs VEV $v = 246\,$GeV and thus in $m_e$. The mass 
	formulas $m_i = r_i \times \xi^{p_i} \times v$ use the measured $v$ value 
	directly, so no separate $K_{\text{frak}}$ factor is needed.
\end{remark}

\subsection{Numerical Result}

\begin{equation}
G_{\text{SI}} = 6.674 \times 10^{-11}\,\text{m}^3/(\text{kg}\cdot\text{s}^2)
\label{eq:G_result}
\end{equation}

with $< 0.001\%$ deviation from the CODATA 2018 value!

\section{The Planck Length}

\subsection{Standard Definition}

The Planck length is defined as:

\begin{equation}
l_P = \sqrt{\frac{\hbar G}{c^3}}
\label{eq:planck_length_standard}
\end{equation}

In natural units ($\hbar = c = 1$) this simplifies to:

\begin{equation}
l_P = \sqrt{G}
\label{eq:planck_length_natural}
\end{equation}

\subsection{T0 Derivation from $\xipar$}

Since $G$ is derived from $\xipar$, the Planck length follows directly:

\begin{equation}
l_P = \sqrt{G} = \sqrt{\frac{\xipar^2}{4 m_e}} = \frac{\xipar}{2\sqrt{m_e}}
\label{eq:planck_from_xi}
\end{equation}

In natural units with $m_e = 0.511$ MeV:

\begin{equation}
l_P = \frac{1.333 \times 10^{-4}}{2\sqrt{0.511}} \approx 9.33 \times 10^{-5}
\label{eq:planck_nat}
\end{equation}

Conversion to SI units:

\begin{equation}
\boxed{l_P = 1.616 \times 10^{-35}\,\text{m}}
\label{eq:planck_si}
\end{equation}

\section{Characteristic T0 Length Scales}

\subsection{The Sub-Planck Scale}

The minimal sub-Planck length scale is:

\begin{equation}
L_0 = \xipar \cdot l_P = \frac{4}{3} \times 10^{-4} \times 1.616 \times 10^{-35}\,\text{m} = 2.155 \times 10^{-39}\,\text{m}
\label{eq:sub_planck}
\end{equation}

This scale is approximately $10^4$ times smaller than the Planck length and marks 
the absolute lower bound of spacetime granulation.

\subsection{Energy-Dependent Length Scales}

The characteristic T0 length for an energy $E$ is:

\begin{equation}
r_0(E) = 2GE
\label{eq:r0_energy}
\end{equation}

In natural units ($G = 1$):

\begin{equation}
r_0(E) = \frac{1}{E}
\label{eq:r0_natural}
\end{equation}

For the fundamental energy scale $\Ezero = \sqrt{m_e \cdot m_\mu}$:

\begin{equation}
r_0(\Ezero) = 2G\Ezero \approx 2.7 \times 10^{-14}\,\text{m}
\label{eq:r0_E0}
\end{equation}

\section{The Boltzmann Constant}

\subsection{Connection to Temperature}

The Boltzmann constant connects temperature with energy:

\begin{equation}
E = k_B T
\label{eq:boltzmann_relation}
\end{equation}

In T0 theory, this is a manifestation of time-mass duality at thermodynamic scales.

\subsection{Derivation from $\xipar$}

In natural units, $k_B$ is dimensionless. The SI conversion follows from the 
energy unit:

\begin{equation}
k_B^{\text{SI}} = \frac{\text{1 eV}}{\text{11604.5 K}} = 1.381 \times 10^{-23}\,\text{J/K}
\label{eq:boltzmann_si}
\end{equation}

T0 theory reproduces this through the connection between energy and temperature 
scales via $\xipar$-derived masses.

\section{The 2019 SI Reform}

\subsection{Fundamental Redefinition}

The 2019 SI reform defined the kilogram via the Planck constant:

\begin{equation}
\hbar = 6.62607015 \times 10^{-34}\,\text{J}\cdot\text{s} \quad \text{(exact)}
\label{eq:planck_const_exact}
\end{equation}

and the Boltzmann constant:

\begin{equation}
k_B = 1.380649 \times 10^{-23}\,\text{J/K} \quad \text{(exact)}
\label{eq:boltzmann_exact}
\end{equation}

\subsection{T0 Consequence}

This reform unwittingly implemented the unique calibration consistent with 
the T0 geometric foundation. The SI units are now implicitly determined 
by $\xipar$:

\begin{equation}
\text{SI system} \leftrightarrow \xipar = \frac{4}{3} \times 10^{-4}
\label{eq:si_xi_connection}
\end{equation}

\section{Scale Hierarchy}

The various length scales in T0 theory:

\begin{align}
L_0 &= 2.155 \times 10^{-39}\,\text{m} \quad \text{(minimal T0 scale)} \\
l_P &= 1.616 \times 10^{-35}\,\text{m} \quad \text{(Planck length)} \\
r_0(\Ezero) &= 2.7 \times 10^{-14}\,\text{m} \quad \text{(characteristic scale)} \\
r_e &= 2.818 \times 10^{-15}\,\text{m} \quad \text{(classical electron radius)}
\end{align}

This hierarchy emerges entirely from $\xipar$ and the fractal structure 
of spacetime.

\section{Summary}

In this chapter, we have shown how all fundamental units and constants 
follow from $\xipar$:

\begin{enumerate}
\item Natural units: $\hbar = c = 1$ simplify the derivations
\item Gravitational constant: $G = \frac{\xipar^2}{4m_e} \times C_{\text{conv}}$ (fractal correction absorbed in $m_e$)
\item Planck length: $l_P = \frac{\xipar}{2\sqrt{m_e}}$
\item Sub-Planck scale: $L_0 = \xipar \cdot l_P$
\item 2019 SI reform: Consistent with T0 geometry
\end{enumerate}

The complete derivation chain $\xipar \to m_e \to G \to l_P$ demonstrates the 
parameter-freedom of the theory. All physical quantities emerge from the 
geometry of three-dimensional space.

