% Chapter 09: Cosmology, Redshift and CMB
% Corrected: ξ_eff = ξ/2 removed (not physically justified)
% H₀ = ξ·c = 40 km/s/Mpc as direct prediction

\chapter{Cosmology, Redshift and CMB in Time-Mass Duality}

\section{Introduction}

In the preceding chapters, the microscopic side of time-mass duality was the 
focus: masses, couplings, and quantum phenomena. This chapter shows how the 
same structure affects large-scale cosmological phenomena: redshift, cosmic 
microwave background radiation, and effective quantities such as the Hubble scale.

The key point is: the same parameter $\xipar = \frac{4}{3} \times 10^{-4}$ 
that determines masses and couplings at the particle level also determines 
the cosmological redshift. There is no transition to a different effective 
parameter at large scales -- in a homogeneous, static universe, the same 
$\xipar$ applies everywhere.

\section{Redshift Without Expanding Space}

\subsection{Standard Interpretation}

Standard cosmology interprets the cosmological redshift primarily as a 
consequence of expanding spacetime. A photon's wavelength is stretched 
with the cosmic scale factor $a(t)$:

\begin{equation}
\frac{\lambda_{\text{obs}}}{\lambda_{\text{emit}}} = \frac{a(t_{\text{obs}})}{a(t_{\text{emit}})} = 1 + z
\end{equation}

\subsection{Time-Mass Duality Interpretation}

Within the framework of time-mass duality, the observed redshift is 
understood as a consequence of the fractal deep structure of spacetime. A 
photon propagating through the fractal space with $D_f = 3 - \xipar$ 
continuously loses energy to the dynamic vacuum field.

The T0 redshift:

\begin{equation}
z_{\text{T0}} = \int_0^d \xipar(r) \frac{E_\gamma(r)}{E_{\gamma,0}} dr
\end{equation}

For a homogeneous $\xipar$ field, this simplifies to:

\begin{equation}
\boxed{z_{\text{T0}} \approx \xipar \cdot d}
\label{eq:z_T0_linear}
\end{equation}

where $d$ is the distance in megaparsecs. This linear relation is the 
T0 version of the Hubble law.

\section{The Hubble Parameter}

\subsection{Direct T0 Prediction}

From $z = \xipar \cdot d$, the Hubble law $z = (H_0/c) \cdot d$ follows 
immediately with:

\begin{equation}
\boxed{H_0^{\text{T0}} = \xipar \cdot c = \frac{4}{3} \times 10^{-4} \times 299{,}792\,\text{km/s} \approx 40.0\,\text{km/s/Mpc}}
\label{eq:H0_T0}
\end{equation}

This is a parameter-free prediction: the same $\xipar$ that predicts the 
fine-structure constant to $0.003\%$ and lepton masses to the per-mille level 
also yields the Hubble parameter.

\subsection{Discrepancy with the Standard Value}

The experimentally determined Hubble parameter is $H_0^{\text{exp}} \approx 
67$--$73\,$km/s/Mpc, depending on the measurement method. The ratio:

\begin{equation}
\frac{H_0^{\text{exp}}}{H_0^{\text{T0}}} = \frac{70}{40.0} \approx 1.75 \approx \frac{7}{4}
\label{eq:H0_ratio}
\end{equation}

is approximately $7/4$ (accuracy: $0.07\%$).

\begin{remark}[Model dependence of the experimental $H_0$]
	The ``experimental'' value $H_0 \approx 70\,$km/s/Mpc is not a 
	model-independent measurement. It is extracted from raw data (redshifts 
	and luminosities of standard candles) under the assumption of an expanding 
	Friedmann universe. In particular:
	\begin{itemize}
	\item The luminosity distance $d_L = d \cdot (1+z)$ presupposes a 
	cosmic scale factor that does not exist in T0.
	\item In T0, the physical distance $d$ is directly measurable, without 
	$(1+z)$ correction.
	\item The model-independent observation is the linear relation 
	$z \propto d$ for small $z$. Its proportionality factor is interpreted 
	differently in different cosmological models.
	\end{itemize}
	The discrepancy by a factor of $7/4$ could therefore partially or fully 
	result from the different data interpretation. Notably, $7/4$ is a 
	rational fraction that may have geometric significance in the torsion 
	crystal formalism (4 dimensions of the torus $\mathbb{R}^3 \times S^1$, 
	7 symmetry classes of the crystal lattice). This is an open research 
	question.
\end{remark}

\begin{remark}[Why not $\xi_{\text{eff}} = \xi/2$?]
	In an earlier formulation (Ref.\ 201, DVFT), an effective parameter 
	$\xi_{\text{eff}} = \xi/2$ was postulated for cosmological scales. 
	However, this halving is not justified:
	\begin{itemize}
	\item In a homogeneous static universe, there is no averaging effect 
	that could reduce $\xi$.
	\item The energy density of the vacuum field $|\Phi|^2 = \rho^2$ is 
	phase-independent -- a $\cos^2$ averaging would require an unjustified 
	U(1) symmetry breaking.
	\item Numerically, $\xi/2$ worsens the agreement: $H_0(\xi/2) = 20\,$km/s/Mpc 
	lies further from experiment than $H_0(\xi) = 40\,$km/s/Mpc.
	\item All measurements are local. Since T0 theory uses the same $\xi$ 
	value everywhere (masses, couplings, gravitation), the redshift must 
	also be calculated with the same $\xi$.
	\end{itemize}
\end{remark}

\section{CMB Temperature}

The CMB temperature:

\begin{equation}
T_{\text{CMB}} = 2.7255\,\text{K}
\end{equation}

is interpreted in T0 theory as a thermodynamic equilibrium state of the 
$\xipar$ geometry, not as a relic of a Big Bang. The dynamic vacuum field 
$\Phi = \rho e^{i\theta}$ has an intrinsic phase evolution $\dot{\theta} = m 
= 1/T$ (from time-mass duality). The CMB radiation is the thermal equilibrium 
spectrum of this universal vacuum field, whose temperature is determined by the 
geometric parameters $\xipar$ and $f = 7500$.

\section{Static Universe in the 4D Torus}

T0 theory describes a static universe without global expansion. Spacetime 
has the topology $\mathbb{R}^3 \times S^1$, where $S^1$ is the phase 
direction of the vacuum field (not ``time'' in the classical sense, since 
$T \cdot m = 1$ defines time as the inverse of mass).

The universe is ``infinite'' not in the sense of infinite extent, but because 
the torus is closed upon itself -- there is no boundary and no edge. In this 
view:

\begin{itemize}
\item Redshift arises from energy loss in the fractal vacuum field, not from 
expansion
\item The Hubble relation $z = \xipar \cdot d$ is a direct consequence of the 
fractal dimension $D_f = 3 - \xipar$
\item Dark energy is not a separate substance but manifests as an effective 
property of the dynamic vacuum field $\Phi$
\item Large-scale homogeneity follows from the topology of the torus, without 
inflation
\end{itemize}

JWST observations of evolved galaxies at $z > 10$, which appear unexpectedly 
early in the Standard Model, are natural in the T0 picture since the 
development time is unlimited.

\section{Summary}

The cosmological predictions of T0 theory follow directly from $\xipar$:

\begin{itemize}
\item Redshift: $z = \xipar \cdot d$ (energy loss in the vacuum field)
\item Hubble parameter: $H_0^{\text{T0}} = \xipar \cdot c \approx 40\,$km/s/Mpc 
(parameter-free prediction)
\item Discrepancy with $H_0^{\text{exp}} \approx 70\,$km/s/Mpc: factor $\approx 7/4$ 
(model-dependent data interpretation, open research question)
\item CMB as equilibrium state of the vacu