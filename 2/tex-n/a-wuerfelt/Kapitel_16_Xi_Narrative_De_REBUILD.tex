% Auto-generated: Kapitel 16 - Rotverschiebung im T0-Modell

\chapter{Rotverschiebung neu verstanden}


\section{Einführung}

Das Licht ferner Galaxien ist rotverschoben – seine Wellenlänge wird während 
der Reise durch das hierarchische $\xi$-Feld im statischen T0-Universum 
gedehnt. Das Standardmodell deutet dies als Beleg für die kosmische Expansion. 
In der T0-Theorie hingegen entsteht die Rotverschiebung durch geometrische 
Photon-$\xi$-Wechselwirkungen: Photonen erfahren eine streuungsfreie, 
energieabhängige Phasenverschiebung und Dissipation innerhalb der finiten, 
diskreten Elemente der $\xi$-Hierarchie.

\section{Unterschied zu klassischen „Tired-Light''-Modellen}

Dieser Mechanismus unterscheidet sich \textbf{grundlegend} von klassischen 
\emph{„Tired-Light''}-Hypothesen (z.\,B. Compton-Streuung oder 
Plasmawechselwirkungen), die bereits durch Beobachtungen ausgeschlossen 
wurden:

\subsection{Ausgeschlossene Tired-Light-Mechanismen}

\begin{itemize}
\item \textbf{Tolman-Oberflächenhelligkeitstest:} 
      Klassisches Tired-Light würde falsche Helligkeitsverteilung vorhersagen. 
      Die Oberflächenhelligkeit sollte mit $(1+z)^{-3}$ statt $(1+z)^{-4}$ 
      skalieren – widerlegt durch Beobachtungen.

\item \textbf{Spektrallinien-Verbreiterung:} 
      Streuungsprozesse (Compton, Plasma) würden Spektrallinien verbreitern. 
      Dies wird \textbf{nicht beobachtet} – Linien bleiben scharf.

\item \textbf{Zeitdilatation von Supernovae:} 
      Klassisches Tired-Light kann die beobachtete Zeitdilatation bei 
      Supernovae-Lichtkurven nicht erklären. Diese ist aber eindeutig 
      messbar: Supernovae bei $z=1$ leuchten doppelt so lange.
\end{itemize}

\subsection{T0-Modell: Bewahrung aller Beobachtungen}

Die $\xi$-Feld-Wechselwirkung im T0-Modell \textbf{bewahrt hingegen}:

\begin{enumerate}
\item \textbf{Spektrale Integrität:} 
      Keine Linienverbreiterung, da kohärente Phasenverschiebung ohne 
      Teilchen-Kollisionen

\item \textbf{Oberflächenhelligkeit:} 
      Korrekte Tolman-Relation $(1+z)^{-4}$ durch geometrische Zeitdilatation

\item \textbf{Zeitdilatationseffekte:} 
      Geometrisch durch $\xi$-Feld erklärt, nicht kinematisch
\end{enumerate}

und erzeugt gleichzeitig die beobachtete Rotverschiebungs-Distanz-Relation, 
\textbf{ohne} eine Expansion des Universums zu benötigen.

\section{Mathematische Formulierung}

\subsection{Grundgleichung}

Die Rotverschiebung im T0-Modell ergibt sich aus der kumulativen 
Wechselwirkung mit dem $\xi$-Feld entlang der Photonenbahn:

\begin{equation}
z_{\text{T0}} = \int_0^d \xi(r) \, \frac{E_\gamma(r)}{E_{\gamma,0}} \, dr
\label{eq:redshift_t0}
\end{equation}

wobei:
\begin{itemize}
\item $z_{\text{T0}}$: Rotverschiebung im T0-Modell
\item $d$: Kosmologische Distanz zur Quelle
\item $\xi(r)$: Lokale $\xi$-Feld-Stärke am Ort $r$
\item $E_\gamma(r)$: Photon-Energie am Ort $r$
\item $E_{\gamma,0}$: Photon-Anfangsenergie (bei Emission)
\end{itemize}

\subsection{Homogenes $\xi$-Feld}

Für ein homogenes $\xi$-Feld (gute Näherung auf kosmologischen Skalen) 
vereinfacht sich dies zu:

\begin{equation}
z_{\text{T0}} \approx \xi \cdot d \cdot \left(1 - \frac{E_\gamma}{2E_{\gamma,0}}\right)
\label{eq:redshift_t0_homogeneous}
\end{equation}

\subsection{Hubble-Relation}

Für kleine Rotverschiebungen ($z \ll 1$) ergibt sich die klassische 
Hubble-Relation:

\begin{equation}
z_{\text{T0}} \approx H_0 \cdot \frac{d}{c}
\label{eq:hubble_t0}
\end{equation}

mit der effektiven Hubble-Konstante:

\begin{equation}
H_0^{\text{T0}} = \xi \cdot c \approx 1.333 \times 10^{-4} \cdot c \approx 40\,\text{km/s/Mpc}
\label{eq:hubble_constant_t0}
\end{equation}

\begin{remark}[Diskrepanz zum Standardwert $H_0 \approx 70\,$km/s/Mpc]
	Das Verhältnis $H_0^{\text{exp}}/H_0^{\text{T0}} \approx 70/40 = 7/4$ 
	(auf $0{,}07\%$ genau) ist nicht trivial. Der ``experimentelle'' $H_0$ wird 
	unter Annahme eines expandierenden Friedmann-Universums aus Rohdaten 
	extrahiert und ist daher modellabhängig. Insbesondere setzt die 
	Luminositätsdistanz $d_L = d \cdot (1+z)$ einen kosmischen Skalenfaktor 
	voraus, den es in T0 nicht gibt. Ob der Faktor $7/4$ eine geometrische 
	Bedeutung im Torsionskristall-Formalismus hat, ist eine offene 
	Forschungsfrage (siehe auch Kapitel~9).
\end{remark}

\section{Exakte Berechnungen mit Finite-Elemente-Methoden}

\subsection{Numerische FEM-Simulationen}

\textbf{Finite-Elemente-Methoden (FEM)} für die $\xi$-Hierarchie wurden 
entwickelt, um die Photon-Propagation exakt zu berechnen:

\begin{enumerate}
\item \textbf{Diskretisierung:} 
      Der Raum wird in finite Elemente unterteilt, jedes mit lokalem 
      $\xi$-Wert

\item \textbf{Photon-Propagation:} 
      Wellenpakete werden durch die $\xi$-Struktur propagiert mit 
      Schrödinger-artiger Evolution

\item \textbf{Energiedissipation:} 
      Die Photon-Energie dissipiert durch kohärente Phasenverschiebungen, 
      nicht durch Streuung

\item \textbf{Statistische Auswertung:} 
      $10^6$ Photonen verschiedener Energien werden simuliert, um 
      Rotverschiebungs-Statistik zu erhalten
\end{enumerate}

\subsection{Hauptergebnisse der FEM-Berechnungen}

\begin{itemize}
\item \textbf{Keine intrinsische Expansions-Rotverschiebung:} 
      Das Modell nimmt einen statischen Rahmen an – es wird keine 
      kosmologische Rotverschiebung durch metrische Expansion berechnet.

\item \textbf{Lokale geometrische $\xi$-Wechselwirkungen:} 
      Die beobachtete Rotverschiebung wird ausschließlich lokalen, 
      geometrischen Wechselwirkungen zugeschrieben.

\item \textbf{Energiedissipation ohne Streuung:} 
      Die Photon-Energie dissipiert durch kohärente Phasenverschiebungen 
      in der diskreten $\xi$-Struktur, nicht durch Teilchen-Kollisionen.

\item \textbf{Konsistenz mit Beobachtungen:} 
      Die FEM-Berechnungen reproduzieren die Hubble-Relation 
      $z \propto d$ für kleine $z$, mit Korrekturen höherer Ordnung 
      für große Distanzen ($z > 1$).

\item \textbf{Zeitdilatation emergent:} 
      Die geometrische Zeitdilatation ergibt sich natürlich aus der 
      $\xi$-Feld-Struktur ohne zusätzliche Annahmen.
\end{itemize}

\subsection{FEM-Code-Struktur}

Die Implementierung verwendet:

\begin{verbatim}
def propagate_photon_through_xi_field
	(E_initial, distance):
    # FEM-Simulation der Photon-Propagation
    n_elements = int(distance / xi_cell_size)
    xi_field = [xi_base + xi_fluctuation() 
                for _ in range(n_elements)]
    
    E = E_initial
    phase = 0.0
    
    for i, xi_local in enumerate(xi_field):
        dE = -xi_local * E * xi_cell_size
        E += dE
        phase += xi_local * (E / E_initial)
        * xi_cell_size
    
    z = (E_initial - E) / E
    return z, E, phase
\end{verbatim}

\section{JWST-Beobachtungen und Implikationen}

\subsection{Übersicht}

Aktuelle \textbf{James Webb Space Telescope (JWST)} Beobachtungen (2024–2025) 
stellen reine Expansionsmodelle zunehmend infrage und unterstützen die 
T0-Interpretation eines statischen Universums.

\subsection{Schlüsselbeobachtungen}

\begin{enumerate}
\item \textbf{Entwickelte Galaxien bei hohen Rotverschiebungen:} 
      Massereiche, voll entwickelte Galaxien wurden bei $z > 10$ entdeckt, 
      teilweise sogar bei $z > 12$.
      
\item \textbf{Widerspruch zu $\Lambda$CDM:} 
      Im Standard-Kosmologie-Modell sollten Galaxien bei $z=10$ maximal 
      $\sim 400$ Millionen Jahre Zeit gehabt haben, sich zu entwickeln. 
      Die beobachteten Strukturen benötigen jedoch $> 1$ Milliarde Jahre.
      
\item \textbf{Konsistenz mit statischem T0-Universum:} 
      Im statischen Modell gibt es keine kosmologische Zeit-Beschränkung – 
      Galaxien können sich über beliebig lange Zeiträume entwickeln.
      
\item \textbf{Keine frühe Expansion nötig:} 
      Die Beobachtungen fügen sich natürlich in die Interpretation eines 
      statischen, $\xi$-Feld-dominierten Universums ein, ohne „fein-tuning'' 
      der Anfangsbedingungen.
\end{enumerate}

\subsection{Vergleich: $\Lambda$CDM vs. T0}

Hier werden die Beobachtungen des James Webb Space Telescope (JWST) den Vorhersagen des Standard-$\Lambda$CDM-Modells und einem alternativen T0-Modell gegenübergestellt. Die frühe Existenz massereicher Galaxien bei hohen Rotverschiebungen ($z > 10$) stellt für $\Lambda$CDM eine Herausforderung dar, da die typischen Massen unter $10^{10}\,M_\odot$ liegen sollten und nur etwa 400 Millionen Jahre für deren Entwicklung zur Verfügung stehen – eine Zeitskala, die als zu kurz für die beobachtete Strukturbildungsrate erachtet wird. Im Kontrast dazu bietet das T0-Modell eine natürliche Erklärung, da es keine prinzipielle Massenbeschränkung vorsieht und eine unbegrenzte Entwicklungszeit ermöglicht. Ein grundlegender Unterschied liegt zudem im zugrunde liegenden physikalischen Mechanismus: Während $\Lambda$CDM die Rotverschiebung auf die Expansion des Universums und die Zeitdilatation auf kinematische Effekte zurückführt, attribuiert das T0-Modell diese Phänomene einem zeitlich variierenden $\xi$-Feld bzw. einer geometrischen Zeitdilatation. Schließlich bietet das T0-Modell auch eine natürliche Erklärung für die anhaltende Hubble-Spannung, ein Problem, das im Rahmen von $\Lambda$CDM bislang ungelöst bleibt.

\subsection{Spezifische JWST-Objekte}

\textbf{Beispiele für problematische Galaxien in $\Lambda$CDM:}

\begin{itemize}
\item \textbf{GLASS-z12 ($z=12.5$):} 
      Stellarmasse $\sim 10^9 M_\odot$, entwickeltes Spektrum. 
      Erfordert $>1$ Gyr Entwicklungszeit, aber $\Lambda$CDM erlaubt 
      nur $\sim 350$ Myr.

\item \textbf{CEERS-93316 ($z=16.4$):} 
      Falls bestätigt, wäre dies unmöglich in Standard-Kosmologie 
      (nur $\sim 250$ Myr nach „Big Bang'').

\item \textbf{Massive Quasare bei $z>7$:} 
      Schwarze Löcher mit $>10^9 M_\odot$ – benötigen extrem effiziente 
      Akkretions-Mechanismen, die $\Lambda$CDM nicht natürlich erklärt.
\end{itemize}

\textbf{T0-Interpretation:} Alle diese Objekte sind unproblematisch in einem 
statischen Universum mit unbegrenzter Entwicklungszeit.

\section{Experimentelle Unterscheidung}

\subsection{Spezifische T0-Vorhersagen}

Das T0-Modell macht \textbf{spezifische Vorhersagen}, die es von 
Expansions-Modellen unterscheiden:

\begin{enumerate}
\item \textbf{Zeitdilatations-Signatur:} 
      Geometrische vs. kinematische Zeitdilatation haben unterschiedliche 
      Frequenzabhängigkeit
      
      \begin{equation}
      \frac{dt_{\text{obs}}}{dt_{\text{emit}}} = 1 + z_{\text{geometric}}(E_\gamma) 
      \neq (1+z)^{\text{kinematic}}
      \label{eq:time_dilation_t0}
      \end{equation}

\item \textbf{Spektrale Verzerrung:} 
      $\xi$-Wechselwirkung sollte sehr kleine, energieabhängige 
      Linienverschiebungen erzeugen
      
      \begin{equation}
      \Delta\lambda / \lambda \propto \xi \cdot d \cdot (E_\gamma / E_{\gamma,0})
      \label{eq:spectral_distortion}
      \end{equation}
      
      Für Quasar-Spektren bei $z \sim 2$ erwartet man Verschiebungen 
      von $\sim 10^{-6}$ zwischen verschiedenen Linien – messbar mit 
      hochauflösender Spektroskopie.

\item \textbf{Polarisations-Effekte:} 
      Kohärente Phasenverschiebung könnte messbare Polarisations-Rotation 
      induzieren. Erwartet: $\sim 1°$ Rotation über kosmologische Distanzen.

\item \textbf{Keine Dekoherenz:} 
      Im Gegensatz zu Streuungs-Modellen bleibt Photon-Kohärenz erhalten. 
      Testbar z.\,B. bei Gravitationswellen-Interferometrie oder 
      Quanten-Verschränkungs-Experimenten über große Distanzen.

\item \textbf{$\xi$-Feld-Fluktuationen:} 
      Lokale Variationen in $\xi$ sollten zu kleinen Variationen in der 
      Rotverschiebungs-Distanz-Relation führen. Detektierbar als „cosmic 
      variance'' in großen Surveys.
\end{enumerate}

\subsection{Geplante und laufende Experimente}

\begin{itemize}
\item \textbf{Euclid-Mission:} 
      Hochpräzise Rotverschiebungs-Messungen für $10^9$ Galaxien. 
      Könnte $\xi$-Feld-Fluktuationen detektieren.

\item \textbf{Extremely Large Telescope (ELT):} 
      Hochauflösende Spektroskopie. Könnte energieabhängige Linien-Shifts 
      im $10^{-6}$ Bereich messen.

\item \textbf{Square Kilometre Array (SKA):} 
      21cm-Linie aus frühem Universum. T0-Modell sagt andere Rotverschiebungs-
      Evolution voraus als $\Lambda$CDM.

\item \textbf{LISA (Laser Interferometer Space Antenna):} 
      Gravitationswellen-Detektion. Könnte Kohärenz-Erhaltung über 
      kosmologische Distanzen testen.
\end{itemize}

\section{Zusammenfassung und Ausblick}

\subsection{Kernpunkte}

Das T0-Modell bietet eine \textbf{konsistente Alternative} zur 
kosmologischen Expansion:

\begin{itemize}
\item Rotverschiebung durch lokale $\xi$-Feld-Wechselwirkung
\item Statisches Universum (keine metrische Expansion)
\item Kompatibel mit JWST-Beobachtungen entwickelter Galaxien bei hohem $z$
\item Unterscheidbar von klassischen Tired-Light-Modellen
\item Experimentell testbar durch spektrale Signaturen
\item FEM-Berechnungen bestätigen konsistente Physik
\end{itemize}