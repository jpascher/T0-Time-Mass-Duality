% Chapter 04: Quantum Information and Fundamental Functions in Time-Mass Duality
% Completely rewritten with correct formulas
% Base: 020_T0_QM-QFT-RT_De.tex, 147_quantum_computing_En.tex
% English translation

\chapter{Quantum Information and Fundamental Functions in Time-Mass Duality}

\section{Introduction}

This chapter describes the connection between the geometric structure of 
FFGFT and quantum information theory. The focus is not on technical circuit 
diagrams, but on the question of how qubits, superposition, and entanglement 
can be understood within the framework of time-mass duality.

\section{Qubits as Effective Degrees of Freedom}

\subsection{Standard Formulation}

In the usual formulation, a qubit is a state vector in a two-dimensional 
Hilbert space:

\begin{equation}
	|\psi\rangle = \alpha |0\rangle + \beta |1\rangle, \quad |\alpha|^2 + |\beta|^2 = 1
	\label{eq:qubit_state}
\end{equation}

where $|0\rangle$ and $|1\rangle$ are basis states and $\alpha, \beta \in \mathbb{C}$ 
are complex amplitudes.

\subsection{FFGFT Interpretation}

In FFGFT, this Hilbert space is not understood as an abstract mathematical 
space without background, but as an effective description of certain fractal 
modes of time-mass duality.

The two basis states $|0\rangle$ and $|1\rangle$ then represent two stabilized 
configurations of an underlying geometric structure (e.g., two locally different 
phases of the field), while the coefficients $\alpha$ and $\beta$ reflect the 
distribution of activation in this structure.

\subsection{Bloch Sphere Representation}

A pure qubit state can be represented on the Bloch sphere:

\begin{equation}
	|\psi\rangle = \cos\left(\frac{\theta}{2}\right)|0\rangle + e^{i\phi}\sin\left(\frac{\theta}{2}\right)|1\rangle
	\label{eq:bloch_sphere}
\end{equation}

with $\theta \in [0,\pi]$ and $\phi \in [0,2\pi)$. This interpretation does 
not change the formal use of qubit algebra; it only makes explicit that the 
parameters are ultimately determined by $\xi$ and the scales derived from it.

\section{Superposition and Interference}

\subsection{Quantum Superposition}

The core of many quantum algorithms is the controlled use of superposition 
and interference. In the usual language, one speaks of a qubit being 
simultaneously "0" and "1" and of these contributions interfering 
constructively or destructively.

In time-mass duality, this describes not a mysterious non-locality, but the 
fact that the underlying fractal time structure supports multiple paths in 
parallel.

\subsection{Hadamard Transformation}

The Hadamard transformation is fundamental for quantum algorithms:

\begin{equation}
	H = \frac{1}{\sqrt{2}}\begin{pmatrix} 1 & 1 \\ 1 & -1 \end{pmatrix}
	\label{eq:hadamard}
\end{equation}

It creates an equal superposition from a basis state:

\begin{align}
	H|0\rangle &= \frac{1}{\sqrt{2}}(|0\rangle + |1\rangle) \label{eq:h_on_0}\\
	H|1\rangle &= \frac{1}{\sqrt{2}}(|0\rangle - |1\rangle) \label{eq:h_on_1}
\end{align}

\section{Entanglement and Bell States}

\subsection{Two-Qubit Systems}

For two qubits, the Hilbert space is four-dimensional with basis 
$\{|00\rangle, |01\rangle, |10\rangle, |11\rangle\}$. A general state is:

\begin{equation}
	|\Psi\rangle = \alpha_{00}|00\rangle + \alpha_{01}|01\rangle + \alpha_{10}|10\rangle + \alpha_{11}|11\rangle
	\label{eq:two_qubit_state}
\end{equation}

with $\sum_{ij}|\alpha_{ij}|^2 = 1$.

\subsection{Bell States}

The maximally entangled Bell states are:

\begin{align}
	|\Phi^+\rangle &= \frac{1}{\sqrt{2}}(|00\rangle + |11\rangle) \label{eq:bell_phi_plus}\\
	|\Phi^-\rangle &= \frac{1}{\sqrt{2}}(|00\rangle - |11\rangle) \label{eq:bell_phi_minus}\\
	|\Psi^+\rangle &= \frac{1}{\sqrt{2}}(|01\rangle + |10\rangle) \label{eq:bell_psi_plus}\\
	|\Psi^-\rangle &= \frac{1}{\sqrt{2}}(|01\rangle - |10\rangle) \label{eq:bell_psi_minus}
\end{align}

These states cannot be represented as a product $|\psi_1\rangle \otimes |\psi_2\rangle$ 
and represent maximal entanglement.

\subsection{T0 Modification of Bell Correlations}

In the T0 theory, Bell correlations are modified by $\xi$. The correlation 
function for entangled photons with measurement directions $a$ and $b$ is:

\begin{equation}
	E(a,b) = -\cos(a-b) \cdot \left(1 - \xi \cdot f(n,l,j)\right)
	\label{eq:bell_correlation_t0}
\end{equation}

where $f(n,l,j)$ is a function of quantum numbers. This leads to a damping 
of the violation of Bell's inequality:

\begin{equation}
	S_{\text{CHSH}} = 2\sqrt{2} \cdot \left(1 - \xi \cdot g(n)\right) \approx 2.827
	\label{eq:chsh_t0}
\end{equation}

compared to the standard value $S_{\text{CHSH}}^{\text{QM}} = 2\sqrt{2} \approx 2.828$.

\section{Quantum Gates}

\subsection{Single-Qubit Gates}

The fundamental single-qubit gates are:

\textbf{Pauli Matrices:}
\begin{align}
	X = \begin{pmatrix} 0 & 1 \\ 1 & 0 \end{pmatrix}, \quad
	Y = \begin{pmatrix} 0 & -i \\ i & 0 \end{pmatrix}, \quad
	Z = \begin{pmatrix} 1 & 0 \\ 0 & -1 \end{pmatrix}
	\label{eq:pauli_matrices}
\end{align}

\textbf{Phase Gates:}
\begin{equation}
	S = \begin{pmatrix} 1 & 0 \\ 0 & i \end{pmatrix}, \quad
	T = \begin{pmatrix} 1 & 0 \\ 0 & e^{i\pi/4} \end{pmatrix}
	\label{eq:phase_gates}
\end{equation}

\subsection{Two-Qubit Gates: CNOT}

The Controlled-NOT gate is fundamental for entanglement:

\begin{equation}
	\text{CNOT} = \begin{pmatrix} 
		1 & 0 & 0 & 0 \\
		0 & 1 & 0 & 0 \\
		0 & 0 & 0 & 1 \\
		0 & 0 & 1 & 0
	\end{pmatrix}
	\label{eq:cnot}
\end{equation}

It acts on two qubits as:
\begin{equation}
	\text{CNOT}|a\rangle|b\rangle = |a\rangle|a \oplus b\rangle
	\label{eq:cnot_action}
\end{equation}

where $\oplus$ is addition modulo 2.

\section{Quantum Algorithms}

\subsection{Quantum Fourier Transform}

The Quantum Fourier Transform (QFT) is central to many algorithms:

\begin{equation}
	\text{QFT}|j\rangle = \frac{1}{\sqrt{N}}\sum_{k=0}^{N-1} e^{2\pi ijk/N}|k\rangle
	\label{eq:qft}
\end{equation}

for an $n$-qubit system with $N = 2^n$ basis states.

\subsection{Shor's Algorithm}

The core of Shor's algorithm for factorization is the mapping:

\begin{equation}
	|x\rangle|0\rangle \mapsto |x\rangle|f(x)\rangle, \quad f(x) = a^x \mod N
	\label{eq:shor_modular_exp}
\end{equation}

followed by a Quantum Fourier Transform. This utilizes the periodicity 
of $f(x)$ to find factors of $N$.

\subsection{T0 Implications}

In the T0 formulation, quantum algorithms are deterministic at the level 
of time field dynamics. The apparent probability arises from projection 
onto the effective Hilbert space. This has implications for:

\begin{itemize}
	\item \textbf{Decoherence:} Geometrically interpreted as damping through $\xi$-corrections
	\item \textbf{Error correction:} Optimization by exploiting the fractal structure
	\item \textbf{Scaling:} $\xi$-dependent limits for large quantum computers
\end{itemize}

\section{Summary}

In this chapter, we have developed the foundations of quantum information 
within the framework of time-mass duality:

\begin{enumerate}
	\item Qubits as effective degrees of freedom of the fractal time structure
	\item Superposition and interference as parallel paths in the geometry
	\item Entanglement with $\xi$-modified Bell correlations
	\item Quantum gates (Hadamard, Pauli, CNOT) with geometric interpretation
	\item Quantum algorithms (QFT, Shor) as deterministic time field dynamics
\end{enumerate}

This formulation shows how $\xi$ not only determines classical physics, but 
also fundamentally governs quantum information – a complete geometric 
foundation for quantum computing technology.


