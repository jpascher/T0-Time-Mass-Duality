% Auto-generated: Chapter 16 - Redshift in the T0 Model
% English translation

\chapter{Redshift Reinterpreted}
\sloppy  % Prevent overfull hbox warnings


\section{Introduction}

Light from distant galaxies is redshifted – its wavelength is stretched 
during travel through the hierarchical $\xi$-field in the static T0 
universe. The standard model interprets this as evidence for cosmic 
expansion. In the T0 theory, however, redshift arises from geometric 
photon-$\xi$ interactions: photons experience a scattering-free, 
energy-dependent phase shift and dissipation within the finite, discrete 
elements of the $\xi$ hierarchy.

\section{Difference from Classical "Tired Light" Models}

This mechanism differs \textbf{fundamentally} from classical 
\emph{"Tired Light"} hypotheses (e.g., Compton scattering or plasma 
interactions), which have already been ruled out by observations:

\subsection{Ruled-Out Tired Light Mechanisms}

\begin{itemize}
	\item \textbf{Tolman surface brightness test:} 
	Classical tired light would predict incorrect brightness distribution. 
	Surface brightness should scale with $(1+z)^{-3}$ instead of $(1+z)^{-4}$ 
	– contradicted by observations.
	
	\item \textbf{Spectral line broadening:} 
	Scattering processes (Compton, plasma) would broaden spectral lines. 
	This is \textbf{not observed} – lines remain sharp.
	
	\item \textbf{Supernova time dilation:} 
	Classical tired light cannot explain the observed time dilation in 
	supernova light curves. Yet it is clearly measurable: supernovae at 
	$z=1$ shine twice as long.
\end{itemize}

\subsection{T0 Model: Preserving All Observations}

The $\xi$-field interaction in the T0 model \textbf{preserves}:

\begin{enumerate}
	\item \textbf{Spectral integrity:} 
	No line broadening, due to coherent phase shift without particle 
	collisions
	
	\item \textbf{Surface brightness:} 
	Correct Tolman relation $(1+z)^{-4}$ via geometric time dilation
	
	\item \textbf{Time dilation effects:} 
	Explained geometrically by $\xi$-field, not kinematically
\end{enumerate}

and simultaneously produces the observed redshift-distance relation, 
\textbf{without} requiring expansion of the universe.

\section{Mathematical Formulation}

\subsection{Basic Equation}

The redshift in the T0 model results from cumulative interaction with the 
$\xi$-field along the photon path:

\begin{equation}
	z_{\text{T0}} = \int_0^d \xi(r) \, \frac{E_\gamma(r)}{E_{\gamma,0}} \, dr
	\label{eq:redshift_t0}
\end{equation}

where:
\begin{itemize}
	\item $z_{\text{T0}}$: Redshift in the T0 model
	\item $d$: Cosmological distance to the source
	\item $\xi(r)$: Local $\xi$-field strength at position $r$
	\item $E_\gamma(r)$: Photon energy at position $r$
	\item $E_{\gamma,0}$: Initial photon energy (at emission)
\end{itemize}

\subsection{Homogeneous $\xi$-Field}

For a homogeneous $\xi$-field (good approximation on cosmological scales), 
this simplifies to:

\begin{equation}
	z_{\text{T0}} \approx \xi \cdot d \cdot \left(1 - \frac{E_\gamma}{2E_{\gamma,0}}\right)
	\label{eq:redshift_t0_homogeneous}
\end{equation}

\subsection{Hubble Relation}

For small redshifts ($z \ll 1$), the classical Hubble relation emerges:

\begin{equation}
	z_{\text{T0}} \approx H_0 \cdot \frac{d}{c}
	\label{eq:hubble_t0}
\end{equation}

with the effective Hubble constant:

\begin{equation}
	H_0^{\text{T0}} = \xi \cdot c \approx 1.333 \times 10^{-4} \cdot c \approx 40\,\text{km/s/Mpc}
	\label{eq:hubble_constant_t0}
\end{equation}

\textbf{Note:} The observed value $H_0 \approx 70$ km/s/Mpc requires either 
a modification of the simple $\xi$ model or additional local effects. This 
is the subject of current research.

\section{Exact Calculations Using Finite Element Methods}

\subsection{Numerical FEM Simulations}

\textbf{Finite Element Methods (FEM)} for the $\xi$ hierarchy have been 
developed to compute photon propagation exactly:

\begin{enumerate}
	\item \textbf{Discretization:} 
	Space is subdivided into finite elements, each with a local 
	$\xi$ value
	
	\item \textbf{Photon propagation:} 
	Wave packets are propagated through the $\xi$ structure with 
	Schrödinger-like evolution
	
	\item \textbf{Energy dissipation:} 
	Photon energy dissipates through coherent phase shifts, not through 
	scattering
	
	\item \textbf{Statistical evaluation:} 
	$10^6$ photons of various energies are simulated to obtain redshift 
	statistics
\end{enumerate}

\subsection{Main Results of FEM Calculations}

\begin{itemize}
	\item \textbf{No intrinsic expansion redshift:} 
	The model assumes a static framework – no cosmological redshift 
	due to metric expansion is computed.
	
	\item \textbf{Local geometric $\xi$ interactions:} 
	The observed redshift is attributed exclusively to local, geometric 
	interactions.
	
	\item \textbf{Energy dissipation without scattering:} 
	Photon energy dissipates through coherent phase shifts in the discrete 
	$\xi$ structure, not through particle collisions.
	
	\item \textbf{Consistency with observations:} 
	The FEM calculations reproduce the Hubble relation $z \propto d$ for 
	small $z$, with higher-order corrections for large distances ($z > 1$).
	
	\item \textbf{Time dilation emergent:} 
	Geometric time dilation arises naturally from the $\xi$-field structure 
	without additional assumptions.
\end{itemize}

\subsection{FEM Code Structure}

The implementation uses:

\begin{verbatim}
	def propagate_photon_through_xi_field
	(E_initial, distance):
	# FEM simulation of photon propagation
	n_elements = int(distance / xi_cell_size)
	xi_field = [xi_base + xi_fluctuation() 
	for _ in range(n_elements)]
	
	E = E_initial
	phase = 0.0
	
	for i, xi_local in enumerate(xi_field):
	dE = -xi_local * E * xi_cell_size
	E += dE
	phase += xi_local * (E / E_initial)
	* xi_cell_size
	
	z = (E_initial - E) / E
	return z, E, phase
\end{verbatim}

\section{JWST Observations and Implications}

\subsection{Overview}

Current \textbf{James Webb Space Telescope (JWST)} observations (2024–2025) 
increasingly challenge pure expansion models and support the T0 
interpretation of a static universe.

\subsection{Key Observations}

\begin{enumerate}
	\item \textbf{Developed galaxies at high redshifts:} 
	Massive, fully developed galaxies have been discovered at $z > 10$, 
	some even at $z > 12$.
	
	\item \textbf{Contradiction with $\Lambda$CDM:} 
	In the standard cosmology model, galaxies at $z=10$ should have had at 
	most $\sim 400$ million years to evolve. The observed structures, 
	however, require $> 1$ billion years.
	
	\item \textbf{Consistency with static T0 universe:} 
	In the static model, there is no cosmological time constraint – 
	galaxies can evolve over arbitrarily long time periods.
	
	\item \textbf{No early expansion needed:} 
	The observations fit naturally into the interpretation of a static, 
	$\xi$-field-dominated universe, without "fine-tuning" of initial 
	conditions.
\end{enumerate}

\subsection{Comparison: $\Lambda$CDM vs. T0}

Here, observations from the James Webb Space Telescope (JWST) are 
contrasted with predictions of the standard $\Lambda$CDM model and an 
alternative T0 model. The early existence of massive galaxies at high 
redshifts ($z > 10$) poses a challenge for $\Lambda$CDM, as typical 
masses should be below $10^{10}\,M_\odot$ and only about 400 million 
years are available for their development – a timescale considered too 
short for the observed rate of structure formation. In contrast, the 
T0 model offers a natural explanation, as it imposes no fundamental 
mass limit and allows unlimited development time. A fundamental 
difference also lies in the underlying physical mechanism: while 
$\Lambda$CDM attributes redshift to the expansion of the universe and 
time dilation to kinematic effects, the T0 model attributes these 
phenomena to a temporally varying $\xi$-field or geometric time 
dilation. Finally, the T0 model also offers a natural explanation for 
the persistent Hubble tension, a problem that remains unsolved within 
$\Lambda$CDM.

\subsection{Specific JWST Objects}

\textbf{Examples of problematic galaxies in $\Lambda$CDM:}

\begin{itemize}
	\item \textbf{GLASS-z12 ($z=12.5$):} 
	Stellar mass $\sim 10^9 M_\odot$, developed spectrum. 
	Requires $>1$ Gyr development time, but $\Lambda$CDM allows 
	only $\sim 350$ Myr.
	
	\item \textbf{CEERS-93316 ($z=16.4$):} 
	If confirmed, this would be impossible in standard cosmology 
	(only $\sim 250$ Myr after "Big Bang").
	
	\item \textbf{Massive quasars at $z>7$:} 
	Black holes with $>10^9 M_\odot$ – require extremely efficient 
	accretion mechanisms not naturally explained by $\Lambda$CDM.
\end{itemize}

\textbf{T0 interpretation:} All these objects are unproblematic in a 
static universe with unlimited development time.

\section{Experimental Differentiation}

\subsection{Specific T0 Predictions}

The T0 model makes \textbf{specific predictions} that distinguish it from 
expansion models:

\begin{enumerate}
	\item \textbf{Time dilation signature:} 
	Geometric vs. kinematic time dilation have different frequency 
	dependence
	
	\begin{equation}
		\frac{dt_{\text{obs}}}{dt_{\text{emit}}} = 1 + z_{\text{geometric}}(E_\gamma) 
		\neq (1+z)^{\text{kinematic}}
		\label{eq:time_dilation_t0}
	\end{equation}
	
	\item \textbf{Spectral distortion:} 
	$\xi$ interaction should produce very small, energy-dependent line shifts
	
	\begin{equation}
		\Delta\lambda / \lambda \propto \xi \cdot d \cdot (E_\gamma / E_{\gamma,0})
		\label{eq:spectral_distortion}
	\end{equation}
	
	For quasar spectra at $z \sim 2$, shifts of $\sim 10^{-6}$ between 
	different lines are expected – measurable with high-resolution spectroscopy.
	
	\item \textbf{Polarization effects:} 
	Coherent phase shift could induce measurable polarization rotation. 
	Expected: $\sim 1°$ rotation over cosmological distances.
	
	\item \textbf{No decoherence:} 
	Unlike scattering models, photon coherence is preserved. 
	Testable e.g., with gravitational wave interferometry or quantum 
	entanglement experiments over large distances.
	
	\item \textbf{$\xi$-field fluctuations:} 
	Local variations in $\xi$ should lead to small variations in the 
	redshift-distance relation. Detectable as "cosmic variance" in large surveys.
\end{enumerate}

\subsection{Planned and Ongoing Experiments}

\begin{itemize}
	\item \textbf{Euclid mission:} 
	High-precision redshift measurements for $10^9$ galaxies. 
	Could detect $\xi$-field fluctuations.
	
	\item \textbf{Extremely Large Telescope (ELT):} 
	High-resolution spectroscopy. Could measure energy-dependent line 
	shifts in the $10^{-6}$ range.
	
	\item \textbf{Square Kilometre Array (SKA):} 
	21cm line from early universe. T0 model predicts different redshift 
	evolution than $\Lambda$CDM.
	
	\item \textbf{LISA (Laser Interferometer Space Antenna):} 
	Gravitational wave detection. Could test coherence preservation over 
	cosmological distances.
\end{itemize}

\section{Summary and Outlook}

\subsection{Key Points}

The T0 model offers a \textbf{consistent alternative} to cosmological 
expansion:

\begin{itemize}
	\item Redshift through local $\xi$-field interaction
	\item Static universe (no metric expansion)
	\item Compatible with JWST observations of developed galaxies at high $z$
	\item Distinguishable from classical tired light models
	\item Experimentally testable through spectral signatures
	\item FEM calculations confirm consistent physics
\end{itemize}


