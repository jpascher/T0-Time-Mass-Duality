% kapitel_27.tex – Stark erweiterte Version mit detaillierten mathematischen Ableitungen
\section{Teilchen-Massenhierarchie und Schwäche der Gravitation in T0}

Zwei der tiefsten Rätsel der Physik sind:
1. Warum spannen die Elementarteilchenmassen 14 Größenordnungen (von Neutrinos bis Top-Quark)?
2. Warum ist die Gravitation im Vergleich zu den anderen Kräften so schwach (\(G_F / G m_p^2 \approx 10^{36}\))?

T0 löst beide durch die Dualität von Amplitude \(\rho\) und Phase \(\theta\) in der fraktalen Vakuumstruktur \(\Phi = \rho e^{i\theta}\).

\subsection{Amplitude und Phase als duale Freiheitsgrade}

Die Lagrangedichte:
\begin{equation}
	\mathcal{L} = \frac{1}{2} K_0 (\partial \rho)^2 + B (\partial \theta)^2 - U(\rho) + \xi \cdot \mathcal{L}_{\text{fractal}}(\rho, \theta),
\end{equation}
mit Stiffness-Parametern
\begin{equation}
	K_0 = \rho_0 \cdot \xi^{-3}, \quad B = \rho_0^2 \cdot \xi^{-2}.
\end{equation}

\subsection{Masse als Amplitude-Deformation}

Stabile Teilchen sind lokalisierte Deformationen:
\begin{equation}
	m = \int (\delta \rho) c^2 \, dV \approx K_0 \cdot (\Delta \rho / \rho_0)^2 \cdot l_0^3.
\end{equation}

Die Hierarchiestufen \(k\) skalieren mit \(\xi\):
\begin{equation}
	m_k \propto \xi^{-k},
\end{equation}
was die exponentielle Hierarchie erzeugt.

Für Leptonen/Quarks:
\begin{equation}
	m_e : m_\mu : m_\tau \approx 1 : \xi^{-2} : \xi^{-4},
\end{equation}
numerisch \(\xi^{-2} \approx 2.25 \times 10^3\), \(\xi^{-4} \approx 5 \times 10^6\) – passend zu beobachteten Verhältnissen.

\subsection{Schwäche der Gravitation}

Gravitation koppelt an Amplitude-Gradienten:
\begin{equation}
	g \sim \nabla \rho / \rho_0 \cdot \xi,
\end{equation}
während Gauge-Kräfte an Phasen-Gradienten:
\begin{equation}
	F \sim \nabla \theta \cdot \xi^{-1/2}.
\end{equation}

Das Verhältnis der Stärken:
\begin{equation}
	\alpha_G / \alpha_{\text{EM}} \approx (K_0 / B) \cdot \xi^2 \approx \xi^{-1} \approx 10^{36},
\end{equation}
exakt die Hierarchie der Kräfte.

\subsection{Detaillierte Ableitung der Hierarchie}

Die Generationsstruktur aus fraktalen Windungen:
\begin{equation}
	\theta_k = 2\pi k / 3 + \xi \cdot \delta_k,
\end{equation}
koppelt Amplitude an Phase:
\begin{equation}
	\delta \rho_k = \rho_0 \cdot \xi \cdot \sin(\theta_k).
\end{equation}

Dies erzeugt die Massenverhältnisse präzise.

\subsection{Schluss}

T0 erklärt Massenhierarchie und Gravitationsschwäche als duale Konsequenzen der Amplitude-Phase-Trennung mit Stiffness-Verhältnis aus \(\xi\). Kein Higgs-Mechanismus oder Extra-Dimensionen nötig – alles parameterfrei.