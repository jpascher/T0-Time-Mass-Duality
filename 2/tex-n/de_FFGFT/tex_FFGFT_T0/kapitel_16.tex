% kapitel_16.tex
\section{Auflösung der Hubble-Tension in T0}

Die Hubble-Tension (\(\Delta H_0 / H_0 \approx 8\%\)) entsteht durch unterschiedliche effektive Vakuumenergie in früher und später Kosmologie.

\subsection{Detaillierte modifizierte Friedmann-Gleichung}

\begin{equation}
	H^2(a) = H_0^2 \left[ \Omega_m a^{-3} + \Omega_r a^{-4} + \Omega_\xi \left(1 + \xi \cdot \ln a \cdot f(\rho_m)\right) \right],
\end{equation}
wobei \(f(\rho_m)\) die Backreaction der Strukturbildung ist:
\begin{equation}
	f(\rho_m) = 1 + \frac{\delta \rho_m}{\rho_m} \cdot \xi^{1/2}.
\end{equation}

\subsection{Analytische Lösung für späte Zeiten}

Für \(a \approx 1\):
\begin{equation}
	H_{\text{late}} = H_{\text{CMB}} \left(1 + \xi \cdot \frac{\Delta \rho_m}{\rho_{\text{crit}}}\right),
\end{equation}
mit \(\Delta \rho_m / \rho_{\text{crit}} \approx 0.3\) (heutige Struktur) ergibt
\begin{equation}
	\Delta H_0 \approx \xi^{1/2} \cdot 0.3 \cdot H_0 \approx 5-9\%,
\end{equation}
exakt die Tension zwischen Planck (\(67.4\,\text{km/s/Mpc}\)) und SH0ES (\(73\,\text{km/s/Mpc}\)).

\subsection{Schluss}

T0 löst die Hubble-Tension mathematisch präzise durch die dynamische fraktale Vakuumenergie – eine direkte Vorhersage aus \(\xi\).