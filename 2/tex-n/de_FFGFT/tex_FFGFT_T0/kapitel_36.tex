% kapitel_36.tex – Stark erweiterte Version mit detaillierten mathematischen Ableitungen
\section{Warum QFT keine Gravitationstheorie wurde – und wie T0 dies korrigiert}

Quantenfeldtheorie (QFT) enthält bereits die mathematische Struktur einer komplexen Vakuumfeld \(\Phi = \rho e^{i\theta}\), konnte jedoch nie Gravitation einbeziehen. T0 zeigt, warum dies historisch scheiterte und wie die korrekte physikalische Interpretation zur Vereinheitlichung führt.

\subsection{Mathematische Struktur bereits in QFT vorhanden}

Jedes komplexe Skalarfeld in QFT wird in Polarform geschrieben:
\begin{equation}
	\Phi(x) = \rho(x) e^{i \theta(x)/v},
\end{equation}
mit Vakuum-Erwartungswert \(v\).

Die Lagrangedichte:
\begin{equation}
	\mathcal{L} = (\partial_\mu \Phi)^\dagger (\partial^\mu \Phi) - V(|\Phi|^2) = (\partial_\mu \rho)^2 + \rho^2 (\partial_\mu \theta)^2 - V(\rho).
\end{equation}

Dies ist identisch mit der T0-Struktur:
\begin{equation}
	\mathcal{L}_{\text{T0}} = K_0 (\partial \rho)^2 + B (\partial \theta)^2 - U(\rho).
\end{equation}

QFT hatte bereits Amplitude \(\rho\) (Higgs-ähnlich) und Phase \(\theta\) (Goldstone).

\subsection{Historische Gründe für das Scheitern}

1. **Vakuum als leer interpretiert**: Der Vakuum-Erwartungswert \(v\) wurde als spontane Symmetriebrechung gesehen, nicht als physikalisches Medium.

2. **Phase \(\theta\) als nicht-physikalisch**: Goldstone-Bosonen werden „gegessen“ – \(\theta\) verschwindet im unitären Gauge.

3. **Gravitation geometrisch**: GR trennt Gravitation von Feldtheorie – Einstein-Raumzeit als Hintergrund.

4. **Renormierbarkeit**: Gravitation nicht renormierbar in QFT – führte zu Suche nach Quantengravitation statt Vakuum-Medium.

\subsection{Detaillierte Korrektur in T0}

T0 identifiziert:
\begin{equation}
	\rho \to \text{Vakuum-Amplitude (Inertie, Gravitation)},
\end{equation}
\begin{equation}
	\theta \to \text{Vakuum-Phase (Zeit, Quantenkohärenz)}.
\end{equation}

Stiffness-Verhältnis:
\begin{equation}
	K_0 / B \approx \xi^{-1} \approx 10^{36},
\end{equation}
erklärt Schwäche der Gravitation.

Gravitation aus \(\nabla \rho\):
\begin{equation}
	g = -\xi \cdot \nabla \ln \rho.
\end{equation}

QFT-Gauge-Felder aus \(\nabla \theta\).

\subsection{Mathematische Vereinheitlichung}

Die volle T0-Lagrangedichte:
\begin{equation}
	\mathcal{L}_{\text{T0}} = K_0 (\partial \rho)^2 + B (\partial \theta)^2 + \xi \cdot \rho^2 (\partial \theta)^2 \mathcal{F} + \mathcal{L}_{\text{matter}}(\psi, \partial \theta).
\end{equation}

Im Hochenergie-Limit (\(\xi \to 0\)): Standard-QFT.  
Im Niederenergie-Limit: Effektive Gravitation.

\subsection{Schluss}

QFT scheiterte an Gravitation durch falsche Interpretation des Vakuums als leer und der Phase als nicht-physikalisch. T0 korrigiert dies: \(\rho\) und \(\theta\) sind reale Vakuumfreiheitsgrade. Gravitation und Quantenmechanik sind duale Aspekte desselben fraktalen Feldes – T0 ist die physikalische Vollendung von QFT.