% kapitel_29.tex – Stark erweiterte Version mit detaillierten mathematischen Ableitungen
\section{Delayed Choice Quantum Eraser Experiment in T0}

Das Delayed Choice Quantum Eraser (DCQE) Experiment (Kim et al., 2000; Walborn et al., 2002) scheint Retrokausalität zu implizieren: Die Entscheidung, Which-Path-Information zu löschen, beeinflusst scheinbar das Interferenzmuster am Detektor – auch wenn die Löschung „nach“ der Detektion erfolgt. T0 erklärt dies vollständig kausal durch die globale Kohärenz der fraktalen Vakuumphase \(\theta\).

\subsection{Das Experiment – Präzise Beschreibung}

Ein Doppelspalt-Setup mit Signal- und Idler-Photonen (parametrischer Down-Conversion):
- Signal-Photon geht zum Doppelspalt-Detektor D0,
- Idler-Photon zu einem verzögerten Eraser-Setup (z. B. Polarisatoren oder Beam-Splitter).

Ohne Erasure: Kein Interferenzmuster an D0 (Which-Path-Information verfügbar).  
Mit Erasure: Interferenzmuster erscheint – auch bei verzögerter Entscheidung.

\subsection{Kohärenz der Vakuumphase in T0}

In T0 ist die Wellenfunktion eine kohärente Phase-Modulation des Vakuums:
\begin{equation}
	\psi(x,t) = \rho_0 \cdot e^{i \theta(x,t)/\xi}.
\end{equation}

Die globale Phase \(\theta\) ist nichtlokal korreliert:
\begin{equation}
	\langle \theta(x) \theta(x') \rangle = \theta_0 + \xi \cdot \ln(|x - x'| / l_0).
\end{equation}

Für verschränkte Photonen:
\begin{equation}
	\theta_{\text{signal}}(x) + \theta_{\text{idler}}(x') = \theta_{\text{total}} = \text{konstant}.
\end{equation}

\subsection{Detaillierte Ableitung des „Erasure“-Effekts}

Which-Path-Detektion (z. B. Polarisator am Idler):
\begin{equation}
	\Delta \theta_{\text{idler}} \approx \pi \quad \Rightarrow \quad \Delta \theta_{\text{signal}} \approx \pi,
\end{equation}
was die Phase am Signal-Detektor randomisiert – kein Interferenzmuster.

Erasure (z. B. 45°-Polarisator):
\begin{equation}
	\Delta \theta_{\text{idler}} \approx 0 \quad \Rightarrow \quad \Delta \theta_{\text{signal}} \approx 0,
\end{equation}
Kohärenz bleibt erhalten – Interferenzmuster erscheint.

Die „verzögerte“ Entscheidung beeinflusst die Klassifikation der Ereignisse an D0 (welche Untermenge man betrachtet), nicht die kausale Propagation.

Mathematisch: Die bedingte Phase
\begin{equation}
	\phi_{\text{cond}} = \theta_{\text{signal}} |_{\text{Erasure}} = \theta_{\text{total}} - \theta_{\text{idler}}^{\text{erased}} \approx \text{konstant}.
\end{equation}

\subsection{Nichtlokale Korrelation ohne Retrokausalität}

Die Korrelation ist nichtlokal durch \(\xi\):
\begin{equation}
	\Delta \theta_{\text{signal}} \cdot \Delta \theta_{\text{idler}} \geq \xi \cdot \hbar / 2,
\end{equation}
analog zur Bellschen Ungleichung, aber deterministisch.

\subsection{Vergleich mit Standard-Interpretationen}

Standard-QM: Kollaps oder Many-Worlds – mysteriös.  
T0: Reine Vakuumphasen-Kohärenz – kausal, lokal in der Phase.

\subsection{Schluss}

DCQE ist in T0 kein Paradoxon: Erasure stellt globale Vakuumphasen-Kohärenz wieder her. Die „verzögerte Wahl“ klassifiziert nur Daten – keine Retrokausalität. T0 erklärt das Experiment vollständig physikalisch durch die fraktale Nichtlokalität mit \(\xi\).