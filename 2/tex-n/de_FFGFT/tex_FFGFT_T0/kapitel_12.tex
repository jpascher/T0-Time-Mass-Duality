% kapitel_12.tex
\section{Kosmologie, Big Bang und Geburt des Universums in T0 – Vergleich mit LQG und Stringtheorie}

Die Standardkosmologie beginnt mit einer Singularität bei \(t=0\). T0 ersetzt diese durch einen regulierten fraktalen Übergang.

\subsection{Fraktale Friedmann-Gleichungen in T0}

\begin{equation}
	\left(\frac{\dot{a}}{a}\right)^2 = \frac{8\pi G}{3} \rho - \frac{k}{a^2} + \xi \cdot \frac{c^2}{l_0^2 a^4},
\end{equation}
Im frühen Universum:
\begin{equation}
	a(t) \propto t^{1/2}.
\end{equation}

\subsection{Vergleich mit Loop Quantum Cosmology (LQC)}

LQC hat modifizierte Friedmann-Gleichung mit \(\rho_{\text{crit}}\), führt zu Big Bounce.

\textbf{Wichtige Unterschiede zu T0}:
\begin{itemize}
	\item LQC quantengeometrisch, Immirzi-Parameter,
	\item T0 klassisch fraktal, nur \(\xi\).
\end{itemize}

\subsection{Vergleich mit Stringtheorie-Kosmologie}

Stringtheorie-Szenarien (Pre-Big-Bang, Ekpyrotisch) benötigen höhere Dimensionen.

\textbf{Wichtige Unterschiede zu T0}:
\begin{itemize}
	\item Stringtheorie komplex, viele Parameter,
	\item T0 minimal, parameterfrei.
\end{itemize}

\subsection{Schluss}

T0 liefert die einfachste Kosmologie: Big Bang als fraktaler Phasenübergang.