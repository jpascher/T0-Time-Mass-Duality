% kapitel_42.tex – Stark erweiterte Version mit detaillierten mathematischen Ableitungen
\section{Planck-Einheiten und Universalkonstanten als emergente Größen in T0}

Planck-Einheiten werden traditionell als fundamentale Skalen betrachtet, abgeleitet aus \(G\), \(c\), \(\hbar\). T0 zeigt, dass sie emergent sind – abgeleitet aus der fraktalen Vakuumstruktur mit dem einzigen Parameter \(\xi\).

\subsection{Traditionelle Planck-Einheiten}

Planck-Länge:
\begin{equation}
	l_P = \sqrt{\frac{\hbar G}{c^3}} \approx 1.616 \times 10^{-35}\,\text{m},
\end{equation}
Planck-Masse:
\begin{equation}
	m_P = \sqrt{\frac{\hbar c}{G}} \approx 2.176 \times 10^{-8}\,\text{kg},
\end{equation}
Planck-Zeit:
\begin{equation}
	t_P = \sqrt{\frac{\hbar G}{c^5}} \approx 5.391 \times 10^{-44}\,\text{s}.
\end{equation}

Diese werden als „fundamentale“ Grenzen gesehen, bei denen Quanteneffekte und Gravitation gleich stark werden.

\subsection{T0 als fundamentale Skala}

In T0 ist die fundamentale Länge \(l_0\) (T0-Länge):
\begin{equation}
	l_0 = l_P \cdot \xi^{-1},
\end{equation}
mit \(\xi = \frac{4}{3} \times 10^{-4}\):
\begin{equation}
	l_0 \approx 1.616 \times 10^{-35} \cdot 1333 \approx 2.15 \times 10^{-32}\,\text{m}.
\end{equation}

Die Planck-Skala ist emergent:
\begin{equation}
	l_P = l_0 \cdot \xi.
\end{equation}

\subsection{Detaillierte Ableitung der Emergenz}

Die Vakuum-Stiffness:
\begin{equation}
	K_0 = \rho_0 \cdot \xi^{-3}, \quad B = \rho_0^2 \cdot \xi^{-2}.
\end{equation}

Lichtgeschwindigkeit \(c\) als Phasen-Ausbreitung:
\begin{equation}
	c = \sqrt{\frac{B}{K_0 / \rho_0}} = \sqrt{\xi}.
\end{equation}

\(\hbar\) aus Phasen-Quantisierung:
\begin{equation}
	\hbar = B \cdot l_0^2 \cdot \xi.
\end{equation}

G aus Amplitude-Kopplung:
\begin{equation}
	G = \frac{l_0^2 c^2}{m_0} \cdot \xi^4.
\end{equation}

Einsetzen ergibt exakt die Planck-Formeln – sie sind nicht fundamental, sondern abgeleitet aus \(l_0\) und \(\xi\).

\subsection{Universalkonstanten als T0-Derivate}

- \(\alpha = \xi^2 \cdot\) (dimensionslose Kopplung),
- \(\Lambda = \xi^2 / l_0^2 \cdot c^2\),
- \(\Lambda_{\text{QCD}} = \sqrt{B}\).

Alle „Konstanten“ sind Verhältnisse von \(l_0\) und \(\xi\).

\subsection{Schluss}

T0 demystifiziert Planck-Einheiten: Sie sind nicht die „Grundbausteine“, sondern emergente Übergangsskalen zwischen fraktaler T0-Struktur und klassischer Physik. Die wahre fundamentale Skala ist \(l_0\), reguliert durch \(\xi\).