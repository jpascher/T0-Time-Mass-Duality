% kapitel_06.tex
\section{Reinterpretation von \(E = mc^2\) in der T0-Time-Mass-Duality}

Die Äquivalenz von Masse und Energie \(E = mc^2\) wird in der Fundamentale Fraktalgeometrische Feldtheorie (FFGFT, früher T0-Theorie) nicht als separates Postulat eingeführt, sondern ergibt sich zwangsläufig aus der fundamentalen Dualität zwischen Zeit und Masse in der fraktalen Skalenhierarchie.

\subsection{Die Time-Mass-Duality als ontologische Grundlage}

In T0 ist jede Ruhemasse \(m\) nichts anderes als ein stabilisiertes, fraktal skaliertes Zeitintervall \(\Delta t\) in der Hierarchie. Die Dualitätsrelation lautet
\begin{equation}
	m = \frac{\hbar}{c^2} \cdot \frac{\Delta t}{T_0 \cdot \xi^k},
\end{equation}
wobei \(T_0\) die fundamentale T0-Zeitskala (aus \(\xi\) abgeleitet), \(\xi = \frac{4}{3} \times 10^{-4}\) der fraktale Skalenparameter und \(k\) eine ganzzahlige Hierarchiestufe ist. Die Energie dieses Zeitintervalls ist die in der fraktalen Struktur gespeicherte potentielle Energie.

\subsection{Detaillierte Ableitung der Ruheenergie}

Durch Anwendung der fraktalen Selbstähnlichkeit auf die Dimensionsanalyse ergibt sich die Ruheenergie \(E_0\) als
\begin{align}
	E_0 &= m c^2 \nonumber \\
	&= \frac{\hbar}{T_0} \cdot \xi^{-k}.
\end{align}
Hier wird \(c\) als maximale Ausbreitungsgeschwindigkeit fraktaler Signale emergent aus der Skala \(\xi\). Die Umwandlung von Masse in Energie (z. B. bei Paarvernichtung oder Kernfusion) entspricht einer Relaxation der fraktalen Zeitstruktur in eine niedrigere Hierarchiestufe, wobei die Differenz \(\Delta E = \Delta m \, c^2\) freigesetzt wird.

\subsection{Physikalische Interpretation und Konsequenzen}

Masse ist in T0 keine intrinsische Teilcheneigenschaft, sondern gespeicherte fraktale Zeitenergie. Dies erklärt:
\begin{itemize}
	\item Warum \(E = mc^2\) universell gilt – alle stabilen Konfigurationen sind Zeitknoten in der fraktalen Hierarchie,
	\item Die Freisetzung von Bindungsenergie in Kernreaktionen als Übergang zwischen Hierarchiestufen,
	\item Die Äquivalenz von Trägheit und Gravitation als duale Manifestationen derselben fraktalen Zeitstruktur.
\end{itemize}

\subsection{Schluss}

T0 gibt der Gleichung \(E = mc^2\) eine tiefere physikalische Bedeutung: Sie ist die direkte Konsequenz der Time-Mass-Duality in einer fraktalen Raumzeit mit dem einzigen Parameter \(\xi\). Kein zusätzliches Postulat ist erforderlich.