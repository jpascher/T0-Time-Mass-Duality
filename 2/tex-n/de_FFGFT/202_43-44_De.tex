\documentclass[12pt,a4paper]{article}
\usepackage[utf8]{inputenc}
\usepackage[T1]{fontenc}
\usepackage[ngerman]{babel}
\usepackage{amsmath}
\usepackage{amsfonts}
\usepackage{amssymb}
\usepackage{geometry}
\geometry{a4paper,left=2.5cm,right=2.5cm,top=2.5cm,bottom=2.5cm}
\usepackage{fancyhdr}
\usepackage{enumitem}
\usepackage{tcolorbox}
\usepackage{physics}
\usepackage{hyperref}

% Hyperref als eines der letzten Pakete laden
\hypersetup{
	unicode=true,
	pdfencoding=unicode,
	bookmarksopen=true
}

% Saubere PDF-Lesezeichen
\pdfstringdefDisableCommands{%
	\def\Lambda{Lambda}%
	\def\Delta{Delta}%
	\def\approx{etwa}%
	\def\Sigma{Sigma}%
	\def\eta{eta}%
	\def\psi{psi}%
}

\title{Kapitel 44 \\ Quantenbits, Schrödinger-Gleichung und Dirac-Gleichung in der T0-Time-Mass-Duality-Theorie }
\author{}
\date{29. Dezember 2025}

\begin{document}
	
	\maketitle
	
	Die fraktale T0-Time-Mass-Duality-Theorie interpretiert Quantenmechanik nicht als separate Postulate, sondern als emergente Konsequenzen der fraktalen Vakuum-Dynamik mit Amplitude \(\rho\) und Phase \(\theta\). Dieses Kapitel leitet Quantenbits (Qubits), die Schrödinger-Gleichung und die Dirac-Gleichung aus der Vakuumfeldstruktur ab. Qubits sind stabile Knoten im Vakuumphasenfeld, die Schrödinger-Gleichung ist die nicht-relativistische Phasendynamik, die Dirac-Gleichung ist die relativistische Erweiterung. Alle emergieren aus der fundamentalen Dualität \(T(x,t) \cdot m(x,t) = 1\) mit fraktaler Dimension \(D_f \approx 2.94\) und \(\epsilon \approx 0.06\).
	
	\section{Quantenbits in der fraktalen Fundamentale Fraktalgeometrische Feldtheorie (FFGFT, früher T0-Theorie)}
	
	Qubits sind die Bausteine der Quanteninformatik – Systeme mit zwei kohärenten Zuständen $|0\rangle$ und $|1\rangle$. In der fraktalen FFGFT sind Qubits stabile topologische Knoten in der Vakuumphase \(\theta\), die zwei stabile Wicklungszahlen annehmen können: $\theta = 0$ oder $\theta = 2\pi$. Die Dualität \(T \cdot m = 1\) erzwingt, dass Phase \(\theta\) und Amplitude \(\rho\) gekoppelt sind, sodass Qubits durch minimale Vakuumfeld-Störungen realisiert werden.
	
	Fraktale Erweiterung: Der Zustand des Qubits ist \(\psi = \alpha |0\rangle + \beta |1\rangle\), wobei \(\alpha, \beta\) durch fraktale Phasenmodulation \(\theta = \alpha \theta_0 + \beta \theta_1 (1 + \epsilon \ln r)\) gegeben sind. Kohärenzzeit \(\tau_c \propto 1/\epsilon\), robust gegen Dekohärenz durch Selbstähnlichkeit. T0 prognostiziert raumtemperaturfähige Phasen-Qubits.
	
	Mathematische Ableitung: Das Vakuumfeld \(\Phi = \rho e^{i\theta}\) hat Knoten mit Wicklungszahl \(n = \frac{1}{2\pi} \oint \nabla\theta \cdot dl = 0,1\). Für Qubits \(n=0\) oder 1. Superposition durch hybride Knoten \(\theta = \alpha \cdot 0 + \beta \cdot 2\pi\). Der Hamiltonian \(H = - \frac{\hbar^2}{2m} \nabla^{D_f} + V\) führt zu stabilen Zuständen.
	
	Vorteil: Kein Messproblem – Kollaps ist makroskopisches Phasen-Scrambling durch Interaktionen.
	
	\section{Schrödinger-Gleichung in der fraktalen Fundamentale Fraktalgeometrische Feldtheorie (FFGFT, früher T0-Theorie)}
	
	Die Schrödinger-Gleichung entsteht als nicht-relativistisches Limit der Vakuumfeld-Dynamik. Die Wellenfunktion \(\psi\) ist eine lokale Störung im fraktalen Zeit-Masse-Feld, mit Phase aus T0-Knotenrotationen.
	
	Fraktale Ableitung: Die Vakuumgleichung \(i\hbar \partial_t \Phi = -\frac{\hbar^2}{2m} \nabla^{D_f} \Phi + V \Phi + \hbar \mu \Phi\), wobei \(\mu = \xi m_0\) die intrinsische Frequenz ist. Im nicht-relativistischen Limit reduziert sich dies zur Schrödinger-Gleichung \(i\hbar \partial_t \psi = -\frac{\hbar^2}{2m} \nabla^{D_f} \psi + V\psi\).
	
	Die komplexe Natur spiegelt die komplexe Struktur von T0s Feld. Alle Parameter (\(\hbar\), \(\mu\)) aus \(\xi\).
	
	\section{Dirac-Gleichung in der fraktalen Fundamentale Fraktalgeometrische Feldtheorie (FFGFT, früher T0-Theorie)}
	
	Die Dirac-Gleichung erweitert die relativistische Quantenmechanik auf Spin-1/2-Teilchen. In fraktaler FFGFT entsteht sie aus der relativistischen Erweiterung der Vakuumdynamik.
	
	Fraktale Ableitung: Dirac-Gleichung \(i\hbar \gamma^\mu \partial_\mu \psi - m c \psi = 0\) als Linearisierung der relativistischen Energie \(E = \sqrt{p^2 c^2 + m^2 c^4}\) mit fraktalem Momentum \(\partial^{D_f}\).
	
	Spin aus topologischen Wicklungen im Phasenfeld, Dirac-Matrizen als Repräsentation der Vakuum-Clifford-Algebra. Elektronenspin aus T0-Knoten mit halbzahliger Wicklung.
	
	Vollständige Integration: Dirac-Gleichung emergiert aus Vakuumphase \(\theta\) und Amplitude \(\rho\), reguliert durch \(\xi\).
	
	Vorteil: Kein Messproblem – Kollaps ist makroskopisches Phasen-Scrambling.
	
	\section{Schluss}
	
	T0 leitet Quantenbits, Schrödinger- und Dirac-Gleichung deterministisch aus der fraktalen Vakuum-Dualität ab. Die Gleichungen sind keine Postulate, sondern zwangsläufige Konsequenzen der Time-Mass-Dualität mit dem einzigen Parameter \(\xi\). Dies vereinheitlicht Quanteninformatik, nicht-relativistische und relativistische QM in einer klassischen, parameterfreien Struktur – eine direkte Erweiterung der FFGFT/T0-Vakuumfeld-Theorie.
	
\end{document}