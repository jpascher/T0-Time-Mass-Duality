\documentclass[12pt,a4paper]{article}
% ==============================================================================
% T0 Theory: Shared GERMAN Preamble – Optimized for eBook/Book
% Version: 2.0 – Final 2026 (LuaLaTeX only) – DEUTSCH korrigiert
% Author: Johann Pascher
% Date: Januar 2026
% ==============================================================================
%
% WICHTIG: Compile EXCLUSIVELY with LuaLaTeX!
% In TeXstudio: Options → Configure TeXstudio → Build → Default Compiler → LuaLaTeX
%
% Required Fonts (install once):
% - Inter: https://fonts.google.com/specimen/Inter
% - JetBrains Mono: https://www.jetbrains.com/lp/mono/
% - Libertinus Math: https://github.com/libertinus-fonts/libertinus
% ==============================================================================

% === KAPITEL 1: GRUNDLEGENDE PAKETE (müssen ZUERST kommen) ===
\RequirePackage{fontspec}
\RequirePackage{unicode-math}

% === KAPITEL 2: SPRACHE (DEUTSCH mit voller Silbentrennung) ===
\usepackage[ngerman]{babel}
\usepackage{microtype}                    % WICHTIG für bessere Silbentrennung!

% Typographie-Einstellungen für besseren deutschen Umbruch
\frenchspacing                     % Korrekte deutsche Abstände nach Satzzeichen
\emergencystretch=3em              % Erlaubt mehr Dehnung bei schwierigen Zeilen
\tolerance=2500                    % Höhere Toleranz für Zeilenumbrüche
\hbadness=10000                    % Unterdrückt "underfull hbox" Warnungen
\hfuzz=2pt                         % Erlaubt minimalen Overfull
\pretolerance=150                  % Bessere Worttrennung

% Bessere Seitenumbrüche verhindern
\clubpenalty=10000           % Keine "Schusterjungen"
\widowpenalty=10000          % Keine "Hurenkinder"  
\displaywidowpenalty=10000   % Auch bei Formeln
\brokenpenalty=10000         % Keine getrennten Wörter über Seiten

% Explizite Trennungen für lange deutsche Wörter
\hyphenation{Fun-da-men-tal Frak-tal-Ge-o-me-trisch Fel-the-o-rie Me-tho-do-lo-gisch}
\hyphenation{Re-vi-si-o-nis-mus Quan-ti-sie-rung U-ni-fi-ka-ti-on Ef-fek-tiv}
\hyphenation{Re-nor-mier-bar-keit Sin-gu-la-ri-tä-ten Kon-zi-li-an-tis-mus}
\hyphenation{E-mer-genz Phä-no-me-no-lo-gisch Do-ku-men-ta-ti-on Ana-ly-se}
\hyphenation{Gra-vi-ta-ti-on Quan-ten-me-cha-nik Do-gma-tis-mus Kon-se-quent}
\hyphenation{Par-al-le-lis-mus Im-ple-men-tie-rung Per-tur-ba-ti-o-nen}
\hyphenation{Ge-o-me-trisch Ar-te-fakt In-ko-mpa-ti-bi-li-tät Kon-struk-tiv}
\hyphenation{Frak-tal Di-men-si-ons-los Un-ter-such-ung Be-schrei-bung}
\hyphenation{In-ter-pre-ta-ti-on Phe-no-me-no-lo-gisch Ma-the-ma-tisch}
\hyphenation{Phi-lo-so-phisch Le-gi-ti-ma-ti-on An-wen-dung Ab-lei-tung}
\hyphenation{Ver-ein-heit-li-chung An-na-hme Vor-stel-lung Er-war-tung}
\hyphenation{Sym-me-trie-ern-wei-te-rung Ge-samt-bild Her-aus-fo-rde-rung}
\hyphenation{Wech-sel-wir-kung Ma-te-ri-al An-satz Per-spek-ti-ve Vor-ge-hen}

% === KAPITEL 3: SCHRIFTEN (mit deutschen Ligaturen) ===
\setmainfont{Inter}[
Scale=1.02,
UprightFont=*-Regular,
BoldFont=*-Bold,
ItalicFont=*-Italic,
BoldItalicFont=*-BoldItalic,
Ligatures=TeX,           % WICHTIG für deutsche Typografie
Language=German          % Explizite Sprachunterstützung
]
\setsansfont{Inter}[
Scale=MatchLowercase,
Ligatures=TeX,
Language=German
]
\setmonofont{JetBrains Mono}[
Scale=0.95,
Language=German
]

% Math Font (simple & stable) – MUSS NACH der Sprachdefinition kommen
% WICHTIG: Libertinus Math für korrekte \underbrace-Darstellung!
\setmathfont{Libertinus Math}[Scale=1.0]

% === KAPITEL 4: MATHEMATIK-PAKETE (in STRENGER Reihenfolge!) ===
% WICHTIG: mathtools muss VOR unicode-math für manche Befehle!
\usepackage{mathtools}           % ZUERST mathtools!

% Dann der Rest
\usepackage{amsmath, amsfonts, amsthm}

% SIUNITX MUSS VOR physics geladen werden!
\usepackage{siunitx}
\sisetup{
	locale=DE,                    % DEUTSCHE Einstellungen für SI-Einheiten!
	group-separator={.},          % Tausendertrennzeichen Punkt
	output-decimal-marker={,},    % Dezimaltrennzeichen Komma
	per-mode=symbol,
	separate-uncertainty=true
}

% Eigene SI-Einheiten für Narrative/Bücher
\DeclareSIUnit\gigalightyear{Gly}
\DeclareSIUnit\mev{MeV}

% physics – MUSS NACH siunitx und mathtools geladen werden
\usepackage{physics}

% === KAPITEL 5: ERGÄNZUNGEN aus pdflatex-Best Practices ===
\usepackage{colortbl}        % Farbige Tabellen (ESSENTIELL!)
\usepackage{placeins}        % Float-Kontrolle: \FloatBarrier
\usepackage{subcaption}      % Unterabbildungen
\usepackage{xurl}            % Bessere URL-Umbrüche
% Hyphenation for URLs in bibliography
\def\UrlBreaks{\do\/\do-}

% === KAPITEL 6: SEITENGESTALTUNG =
\usepackage[paperwidth=8.25in, paperheight=11in, 
left=2.5cm, 
right=2.5cm, 
top=2.5cm, 
bottom=3.5cm,
bindingoffset=0.5cm]{geometry}
\setlength{\headheight}{15pt}
% Page Geometry – Buch-Optimierung
% =============================================================================
%\usepackage[paperwidth=8.25in, paperheight=11in,
%top=1.0in,
%bottom=1.2in,
%inner=1.0in,
%outer=0.75in,
%bindingoffset=0.75in,
%twoside]{geometry}
%\setlength{\headheight}{15pt}

% === KAPITEL 7: GRAFIKEN UND TABELLEN ===
\usepackage{graphicx}
\usepackage[table,xcdraw]{xcolor}
% T0 Markenfarben
\definecolor{gold}{RGB}{255,215,0}
\definecolor{blue}{rgb}{0,0,1}
\definecolor{boxgray}{RGB}{240,240,240}
\definecolor{deepblue}{RGB}{0,0,127}
\definecolor{deepgreen}{RGB}{0,127,0}
\definecolor{deepred}{RGB}{191,0,0}
\definecolor{t0blue}{RGB}{33,150,243}
\definecolor{t0green}{RGB}{76,175,80}
\definecolor{t0orange}{RGB}{255,152,0}
\definecolor{t0purple}{RGB}{156,39,176}
\definecolor{t0red}{RGB}{244,67,54}
\definecolor{t0yellow}{RGB}{255,204,0}
\usepackage{tikz}
\usetikzlibrary{arrows.meta,positioning,shapes.geometric,decorations.pathmorphing,patterns,shapes.arrows,intersections}
\usepackage{pgfplots}
\pgfplotsset{compat=1.18}
\usepackage{quantikz}
\usepackage[most]{tcolorbox}
\tcbuselibrary{breakable}

% === WICHTIG: Algorithm-Konflikt umgehen ===
% Option: algorithmic mit GROSSBUCHSTABEN
% Gemeinsame Box für Experimente
\newtcolorbox{experimentbox}[1][]{
	colback=green!5!white,
	colframe=t0green!80!black,
	fonttitle=\bfseries,
	title={{#1}},
	breakable
}

% Abstract-Fallback
\ifdefined\abstract\else
\newenvironment{abstract}{\section*{\abstractname}\itshape\small\par\bigskip}{\bigskip}
\fi

% === MAKROS SICHER NEU DEFINIEREN / ÜBERSCHREIBEN ===
% Definiere Makros OHNE doppelte Subskripte
\newcommand{\phipar}{\phi_{\mathrm{par}}}
%\newcommand{\xipar}{\xi_{\mathrm{par}}}
\newcommand{\Qphipar}{Q_{\phi_{\mathrm{par}}}}
\newcommand{\rphipar}{r_{\phi_{\mathrm{par}}}}
\newcommand{\logphipar}{\log_{\phi_{\mathrm{par}}}}
\newcommand{\CHSH}{\text{CHSH}}
\usepackage{booktabs}
\usepackage{array}
\usepackage{longtable}
\usepackage{float}
\usepackage{adjustbox}
\usepackage{rotating}
\usepackage{tabularx}
\usepackage{makecell}
\usepackage{multirow}

% === KAPITEL 8: DOKUMENTFORMATIERUNG ===
\usepackage{fancyhdr}
\renewcommand{\headrulewidth}{0.4pt}
\renewcommand{\footrulewidth}{0.4pt}
\usepackage{tocloft}

\usepackage{enumitem}
\setlist[itemize]{leftmargin=*, topsep=2pt, partopsep=0pt, parsep=2pt, itemsep=2pt}
\setlist[enumerate]{leftmargin=*, topsep=2pt, partopsep=0pt, parsep=2pt, itemsep=2pt}
\usepackage{setspace}
\usepackage{ragged2e}
\usepackage{multicol}

% === KAPITEL 9: CODE UND ALGORITHMEN ===
\usepackage{algorithm}
\usepackage{algorithmic}
\usepackage{listings}
\lstset{
	basicstyle=\ttfamily\footnotesize,
	breaklines=true,
	breakatwhitespace=true,
	columns=flexible,
	keepspaces=true,
	showstringspaces=false,
	frame=single,
	xleftmargin=0pt,
	xrightmargin=0pt,
	literate=              % Für deutsche Umlaute in Code-Listings
	{ä}{{\"a}}1 {ö}{{\"o}}1 {ü}{{\"u}}1 {ß}{{\ss}}1
	{Ä}{{\"A}}1 {Ö}{{\"O}}1 {Ü}{{\"U}}1
}
\usepackage{mdframed}

% === KAPITEL 10: ZUSÄTZLICHE PAKETE ===
\usepackage{pdflscape}
\usepackage{braket}
\usepackage{cancel}
\usepackage{caption}
\captionsetup{format=plain, labelfont=bf, justification=centering}
\usepackage{csquotes}
\usepackage{gensymb}
\usepackage{textcomp}
\usepackage{textgreek}
\usepackage{upgreek}
\usepackage{url}
\usepackage{slashed}
\usepackage{bm}

% === KAPITEL 11: HYPERREF (muss als VORLETZTES Paket kommen!) ===
\usepackage{hyperref}
\hypersetup{
	colorlinks=true,
	linkcolor=black,
	citecolor=black,
	urlcolor=black,
	breaklinks=true,           % WICHTIG für deutsche Umlaute in URLs!
	bookmarksnumbered=true,
	unicode=true,
	pdfencoding=auto,
	pdflang=de,                % PDF-Sprache auf Deutsch setzen
	pdfsubject={T0 Theorie - Fundamental Fractal-Geometric Field Theory}
}

% === KAPITEL 12: BOOKMARK (muss NACH hyperref kommen!) ===
\usepackage{bookmark}
% Fix for unicode-math symbols in PDF bookmarks
\pdfstringdefDisableCommands{%
	\def\xi{xi}%
	\def\alpha{alpha}%
	\def\beta{beta}%
	\def\gamma{gamma}%
	\def\delta{delta}%
	\def\Delta{Delta}%
	\def\epsilon{epsilon}%
	\def\varepsilon{epsilon}%
	\def\theta{theta}%
	\def\kappa{kappa}%
	\def\lambda{lambda}%
	\def\mu{mu}%
	\def\nu{nu}%
	\def\pi{pi}%
	\def\rho{rho}%
	\def\sigma{sigma}%
	\def\tau{tau}%
	\def\phi{phi}%
	\def\chi{chi}%
	\def\psi{psi}%
	\def\omega{omega}%
	\def\Omega{Omega}%
	\def\Lambda{Lambda}%
	\def\times{x}%
	\def\cdot{*}%
	\def\pm{+/-}%
	\def\approx{~}%
	\def\sim{~}%
	\def\equiv{=}%
	\def\ell{l}%
	\def\hbar{h}%
	\def\rightarrow{->}%
	\def\leftarrow{<-}%
	\def\Rightarrow{=>}%
	\def\Leftarrow{<=}%
	\def\propto{~}%
	\def\mitxi{xi}%
	\def\mitalpha{alpha}%
	\def\mitbeta{beta}%
	\def\mitgamma{gamma}%
	\def\mitdelta{delta}%
	\def\mitDelta{Delta}%
	\def\mitepsilon{epsilon}%
	\def\mitvarepsilon{epsilon}%
	\def\mittheta{theta}%
	\def\mitkappa{kappa}%
	\def\mitlambda{lambda}%
	\def\mitLambda{Lambda}%
	\def\mitmu{mu}%
	\def\mitnu{nu}%
	\def\mitpi{pi}%
	\def\mitrho{rho}%
	\def\mitsigma{sigma}%
	\def\mittau{tau}%
	\def\mitphi{phi}%
	\def\mitchi{chi}%
	\def\mitpsi{psi}%
	\def\mitomega{omega}%
	\def\mitOmega{Omega}%
}

% === KAPITEL 13: CLEVEREF (DEUTSCHE LABELS) ===
\usepackage[ngerman]{cleveref}
\crefname{equation}{Gleichung}{Gleichungen}
\crefname{figure}{Abbildung}{Abbildungen}
\crefname{table}{Tabelle}{Tabellen}
\crefname{section}{Abschnitt}{Abschnitte}
\crefname{chapter}{Kapitel}{Kapitel}
\crefname{theorem}{Satz}{Sätze}
\crefname{lemma}{Lemma}{Lemmata}
\crefname{definition}{Definition}{Definitionen}
\crefname{example}{Beispiel}{Beispiele}
\crefname{remark}{Bemerkung}{Bemerkungen}

% ==============================================================================
\newenvironment{alternative}{%
	\begin{mdframed}[linecolor=black!30,linewidth=1pt,roundcorner=4pt,backgroundcolor=black!5]%
	}{%
	\end{mdframed}%
}

% Photon/particle environment
\newenvironment{photon}{%
	\begin{mdframed}[linecolor=blue!30,linewidth=1pt,roundcorner=4pt,backgroundcolor=blue!5]%
	}{%
	\end{mdframed}%
}

% Koide formula box environment
\newenvironment{koidebox}{%
	\begin{mdframed}[linecolor=green!30,linewidth=1pt,roundcorner=4pt,backgroundcolor=green!5]%
	}{%
	\end{mdframed}%
}

% Erkenntnis/insight environment
\newenvironment{erkenntnis}{%
	\begin{mdframed}[linecolor=orange!30,linewidth=1pt,roundcorner=4pt,backgroundcolor=orange!5]%
	}{%
	\end{mdframed}%
}

% Beziehung/relationship environment
\newenvironment{beziehung}{%
	\begin{mdframed}[linecolor=purple!30,linewidth=1pt,roundcorner=4pt,backgroundcolor=purple!5]%
	}{%
	\end{mdframed}%
}

% Derivation environment
\newenvironment{derivation}{%
	\begin{mdframed}[linecolor=teal!30,linewidth=1pt,roundcorner=4pt,backgroundcolor=teal!5]%
	}{%
	\end{mdframed}%
}

% Abhandlung/treatise environment
\newenvironment{abhandlung}{%
	\begin{mdframed}[linecolor=brown!30,linewidth=1pt,roundcorner=4pt,backgroundcolor=brown!5]%
	}{%
	\end{mdframed}%
}

% Anwendung/application environment
\newenvironment{anwendung}{%
	\begin{mdframed}[linecolor=cyan!30,linewidth=1pt,roundcorner=4pt,backgroundcolor=cyan!5]%
	}{%
	\end{mdframed}%
}

% Additional common environments
\newenvironment{konsequenz}{%
	\begin{mdframed}[linecolor=red!30,linewidth=1pt,roundcorner=4pt,backgroundcolor=red!5]%
	}{%
	\end{mdframed}%
}

\newenvironment{schlussfolgerung}{%
	\begin{mdframed}[linecolor=gray!30,linewidth=1pt,roundcorner=4pt,backgroundcolor=gray!5]%
	}{%
	\end{mdframed}%
}

\newenvironment{result}{%
	\begin{mdframed}[linecolor=violet!30,linewidth=1pt,roundcorner=4pt,backgroundcolor=violet!5]%
	}{%
	\end{mdframed}%
}

% Formula environment
\newenvironment{formula}{%
	\begin{mdframed}[linecolor=yellow!30,linewidth=1pt,roundcorner=4pt,backgroundcolor=yellow!5]%
	}{%
	\end{mdframed}%
}

% Revolutionaer/revolutionary environment
\newenvironment{revolutionaer}{%
	\begin{mdframed}[linecolor=red!50,linewidth=2pt,roundcorner=4pt,backgroundcolor=red!10]%
	}{%
	\end{mdframed}%
}

% Formel environment (German version of formula)
\newenvironment{formel}{%
	\begin{mdframed}[linecolor=yellow!30,linewidth=1pt,roundcorner=4pt,backgroundcolor=yellow!5]%
	}{%
	\end{mdframed}%
}

% Prinzip/principle environment
\newenvironment{prinzip}{%
	\begin{mdframed}[linecolor=blue!50,linewidth=2pt,roundcorner=4pt,backgroundcolor=blue!10]%
	}{%
	\end{mdframed}%
}

% Experimentell/experimental environment
\newenvironment{experimentell}{%
	\begin{mdframed}[linecolor=magenta!30,linewidth=1pt,roundcorner=4pt,backgroundcolor=magenta!5]%
	}{%
	\end{mdframed}%
}

% Neutrino environment
\newenvironment{neutrino}{%
	\begin{mdframed}[linecolor=cyan!40,linewidth=1pt,roundcorner=4pt,backgroundcolor=cyan!8]%
	}{%
	\end{mdframed}%
}

% Additional missing environments
\newenvironment{schluessel}{%
	\begin{mdframed}[linecolor=yellow!50,linewidth=1pt,roundcorner=4pt,backgroundcolor=yellow!10]%
	}{%
	\end{mdframed}%
}

\newenvironment{summary}{%
	\begin{mdframed}[linecolor=gray!40,linewidth=1pt,roundcorner=4pt,backgroundcolor=gray!8]%
	}{%
	\end{mdframed}%
}

\newenvironment{category}{%
	\begin{mdframed}[linecolor=pink!40,linewidth=1pt,roundcorner=4pt,backgroundcolor=pink!8]%
	}{%
	\end{mdframed}%
}

\newenvironment{sibox}{%
	\begin{mdframed}[linecolor=lime!40,linewidth=1pt,roundcorner=4pt,backgroundcolor=lime!8]%
	}{%
	\end{mdframed}%
}

% More missing environments
\newenvironment{documentbox}{%
	\begin{mdframed}[linecolor=teal!40,linewidth=1pt,roundcorner=4pt,backgroundcolor=teal!8]%
	}{%
	\end{mdframed}%
}

\newenvironment{t0box}{%
	\begin{mdframed}[linecolor=violet!40,linewidth=1pt,roundcorner=4pt,backgroundcolor=violet!8]%
	}{%
	\end{mdframed}%
}

\newenvironment{wichtig}{%
	\begin{mdframed}[linecolor=red!50,linewidth=2pt,roundcorner=4pt,backgroundcolor=red!10]%
	\textbf{Wichtig:} 
	}{%
	\end{mdframed}%
}

\newenvironment{smbox}{%
	\begin{mdframed}[linecolor=orange!40,linewidth=1pt,roundcorner=4pt,backgroundcolor=orange!8]%
	}{%
	\end{mdframed}%
}

\newenvironment{pvbox}{%
	\begin{mdframed}[linecolor=purple!40,linewidth=1pt,roundcorner=4pt,backgroundcolor=purple!8]%
	}{%
	\end{mdframed}%
}

\newenvironment{numerisch}{%
	\begin{mdframed}[linecolor=blue!40,linewidth=1pt,roundcorner=4pt,backgroundcolor=blue!8]%
	}{%
	\end{mdframed}%
}

% More missing environments
\newenvironment{relation}{%
	\begin{mdframed}[linecolor=green!40,linewidth=1pt,roundcorner=4pt,backgroundcolor=green!8]%
	}{%
	\end{mdframed}%
}

\newenvironment{beweis}{%
	\begin{mdframed}[linecolor=brown!40,linewidth=1pt,roundcorner=4pt,backgroundcolor=brown!8]%
	\textbf{Beweis:} 
	}{%
	\end{mdframed}%
}

\newenvironment{revolution}{%
	\begin{mdframed}[linecolor=red!60,linewidth=2pt,roundcorner=4pt,backgroundcolor=red!12]%
	}{%
	\end{mdframed}%
}

\newenvironment{key}{%
	\begin{mdframed}[linecolor=yellow!50,linewidth=1pt,roundcorner=4pt,backgroundcolor=yellow!10]%
	}{%
	\end{mdframed}%
}

\newenvironment{newperspective}{%
	\begin{mdframed}[linecolor=cyan!50,linewidth=1pt,roundcorner=4pt,backgroundcolor=cyan!10]%
	}{%
	\end{mdframed}%
}

\newenvironment{literatur}{%
	\begin{mdframed}[linecolor=gray!50,linewidth=1pt,roundcorner=4pt,backgroundcolor=gray!10]%
	}{%
	\end{mdframed}%
}

\newenvironment{folgerung}{%
	\begin{mdframed}[linecolor=teal!50,linewidth=1pt,roundcorner=4pt,backgroundcolor=teal!10]%
	}{%
	\end{mdframed}%
}

\newenvironment{principle}{%
	\begin{mdframed}[linecolor=blue!60,linewidth=2pt,roundcorner=4pt,backgroundcolor=blue!12]%
	}{%
	\end{mdframed}%
}

% AB HIER: IHRE DEFINITIONEN (angepasst für Deutsch)
% ==============================================================================

\setcounter{tocdepth}{3}

% === ZITATBEFEHLE ===
\providecommand{\citep}[1]{\cite{#1}}
\providecommand{\citet}[1]{\cite{#1}}

% === FARBEN ===
\definecolor{gold}{RGB}{255,215,0}
\definecolor{blue}{rgb}{0,0,1}
\definecolor{boxgray}{RGB}{240,240,240}
\definecolor{deepblue}{RGB}{0,0,127}
\definecolor{deepgreen}{RGB}{0,127,0}
\definecolor{deepred}{RGB}{191,0,0}
\definecolor{t0blue}{RGB}{33,150,243}
\definecolor{t0green}{RGB}{76,175,80}
\definecolor{t0orange}{RGB}{255,152,0}
\definecolor{t0purple}{RGB}{156,39,176}
\definecolor{t0red}{RGB}{244,67,54}
\definecolor{t0yellow}{RGB}{255,204,0}

% === SPALTENTYPEN ===
\newcolumntype{L}[1]{>{\raggedright\arraybackslash}p{#1}}
\newcolumntype{C}[1]{>{\centering\arraybackslash}p{#1}}
\newcolumntype{R}[1]{>{\raggedleft\arraybackslash}p{#1}}

% === HYPERREF-EINSTELLUNGEN (aktualisiert) ===
\hypersetup{
	colorlinks=true,
	linkcolor=t0blue,
	citecolor=t0blue,
	urlcolor=t0blue,
	breaklinks=true,
	bookmarksnumbered=true,
	pdfstartview=FitH,
	pdfencoding=auto,
	pdfdisplaydoctitle=true
}

% === DEUTSCHE THEOREM-UMGEBUNGEN ===
\theoremstyle{plain}
\newtheorem{theorem}{Satz}[section]
\newtheorem{lemma}[theorem]{Lemma}
\newtheorem{proposition}[theorem]{Proposition}
\newtheorem{corollary}[theorem]{Korollar}

\theoremstyle{definition}
\newtheorem{definition}[theorem]{Definition}
\newtheorem{example}[theorem]{Beispiel}
\newtheorem{insight}[theorem]{Erkenntnis}
\newtheorem{discovery}[theorem]{Entdeckung}

\theoremstyle{remark}
\newtheorem{remark}[theorem]{Bemerkung}
\newtheorem{axiom}{Axiom}
%\newtheorem{principle}{Principle}  % Commented out to avoid conflicts with document-specific definitions
\newtheorem{warnung}[theorem]{Warnung}

% === T0-SPEZIFISCHE BEFEHLE ===
% (Hier folgen alle Ihre \newcommand und \providecommand Definitionen)
% Diese bleiben UNVERÄNDERT wie in Ihrer Original-Preamble
% ==============================================================================
% SECTION 14: T0-Specific Commands
% ==============================================================================

% --- Core T0 Fields ---
\newcommand{\Tfield}{T(x,t)}
\providecommand{\Tfieldt}{T(\vec{x},t)}
\newcommand{\Efield}{E(x,t)}
\newcommand{\mfield}{m(x,t)}
\providecommand{\vecx}{\vec{x}}

% --- Lagrangian ---
\newcommand{\Lag}{\mathcal{L}}
\newcommand{\calL}{\mathcal{L}}

% --- Greek Letters and Constants ---
\newcommand{\alphaem}{\alpha}
\newcommand{\betaT}{\beta_T}
\newcommand{\xiT}{\xi}
\newcommand{\xipar}{\xi}

% --- Energy and Planck Units ---
\newcommand{\Ezero}{E_0}
\newcommand{\EPlanck}{E_{\text{Pl}}}
\newcommand{\Mpl}{M_{\text{Pl}}}
\newcommand{\mP}{m_{\text{P}}}
\newcommand{\lP}{\ell_{\text{P}}}
\newcommand{\tP}{t_{\text{P}}}
\newcommand{\LPlanck}{\ell_{\text{Pl}}}
\newcommand{\TPlanck}{t_{\text{Pl}}}

% --- Coupling Constants ---
\newcommand{\Gnat}{G_{\text{nat}}}
\newcommand{\alphaEM}{\alpha_{\text{EM}}}
\newcommand{\alphaSI}{\alpha_{\text{SI}}}
\newcommand{\Hubble}{H_0}
\newcommand{\LCDM}{\Lambda\text{CDM}}
\newcommand{\natunits}{(nat. units)}

% --- T0 Model Parameters ---
\newcommand{\xigeom}{\xi_{\mathrm{geom}}}
\newcommand{\rzero}{r_{0}}
\newcommand{\xirat}{\xi_{\mathrm{rat}}}
\newcommand{\tzero}{t_{0}}
\newcommand{\Lambdat}{\Lambda_{\mathrm{t}}}
\newcommand{\EP}{E_{\text{P}}}
\newcommand{\Emu}{E_{\mu}}
\newcommand{\Ee}{E_{e}}
\newcommand{\Etau}{E_{\tau}}
\newcommand{\alphafine}{\alpha_{\mathrm{fine}}}
\newcommand{\alphal}{\alpha_{\ell}}
\newcommand{\Lzero}{\ell_{0}}
\newcommand{\Lp}{\ell_{\mathrm{P}}}

% --- Additional T0 Commands ---
\newcommand{\Kfrak}{K_{\text{frak}}}
\newcommand{\Dfrak}{D_{\text{frak}}}
\newcommand{\betapar}{\ensuremath{\beta_T}}
\newcommand{\alphapar}{\alpha}
\newcommand{\deltafield}{\delta \phi}
\newcommand{\deltam}{\delta m}
\newcommand{\deltaE}{\delta E}
\newcommand{\Exi}{E_{\xi}}
\newcommand{\Lxi}{\ell_{\xi}}
\newcommand{\rhoCMB}{\rho_{\text{CMB}}}
\newcommand{\rhoCasimir}{\rho_{\text{Casimir}}}
\newcommand{\Leff}{L_{\text{eff}}}
\newcommand{\CQCD}{C_{\mathrm{QCD}}}
\newcommand{\Kspec}{K_{\mathrm{spec}}}
\newcommand{\Tzero}{\ensuremath{T_0}}
\newcommand{\Eabs}{E_{\text{abs}}}
\newcommand{\taupar}{\tau}

% --- Provided Commands ---
\providecommand{\xiconst}{\xi_{\text{const}}}
\providecommand{\DhiggsT}{D_{\text{Higgs-T}}}
\providecommand{\rhoE}{\rho_{E}}
\providecommand{\Echar}{E_{\text{char}}}
\providecommand{\kfrac}{k_{\text{frac}}}
\providecommand{\alphaEMSI}{\alpha_{\text{EM,SI}}}
\providecommand{\alphaEMnat}{\alpha_{\text{EM,nat}}}
\providecommand{\betaTSI}{\beta_{T,\text{SI}}}
\providecommand{\betaTnat}{\beta_{T,\text{nat}}}
\providecommand{\Gsi}{G_{\text{SI}}}
\providecommand{\xiparSI}{\xi_{\text{SI}}}
\providecommand{\xiparnat}{\xi_{\text{nat}}}
\providecommand{\meff}{m_{\text{eff}}}
\providecommand{\Tzerot}{T_{0}(t)}
\providecommand{\mzerot}{m_{0}(t)}
\providecommand{\Ezeroabs}{E_{0,\text{abs}}}
\providecommand{\Epar}{E_{\text{par}}}
\providecommand{\Lnat}{\ell_{\text{nat}}}
\providecommand{\Tnat}{T_{\text{nat}}}
\providecommand{\xifrak}{\xi_{\text{frac}}}
\providecommand{\Tfrak}{T_{\text{frac}}}
\providecommand{\mfrak}{m_{\text{frac}}}
\providecommand{\Dfrac}{D_{\text{frac}}}
\providecommand{\EphotSI}{E_{\gamma,\text{SI}}}
\providecommand{\EphotNat}{E_{\gamma,\text{nat}}}
\providecommand{\Eabsint}{E_{\text{abs,int}}}
\providecommand{\mphoton}{m_{\gamma}}
\providecommand{\Evis}{E_{\text{vis}}}
\providecommand{\Cto}{C_{T0}}
\providecommand{\mytimes}{\times}
\providecommand{\lambdah}{\lambda_h}
\providecommand{\checkmarkx}{\checkmark}
\providecommand{\Enorm}{E_{\text{norm}}}
\providecommand{\Tobs}{T_{\text{obs}}}
\providecommand{\mobs}{m_{\text{obs}}}
\providecommand{\Eobs}{E_{\text{obs}}}
\providecommand{\Lobs}{\ell_{\text{obs}}}
\providecommand{\xobs}{\xi_{\text{obs}}}
\providecommand{\calE}{\mathcal{E}}
\providecommand{\calT}{\mathcal{T}}
\providecommand{\calM}{\mathcal{M}}
\providecommand{\alphag}{\alpha_g}
\providecommand{\Tmax}{T_{\text{max}}}
\providecommand{\mmin}{m_{\text{min}}}
\providecommand{\Lmax}{\ell_{\text{max}}}
\providecommand{\Emin}{E_{\text{min}}}
\providecommand{\Geff}{G_{\text{eff}}}
\providecommand{\rhoeff}{\rho_{\text{eff}}}
\providecommand{\xieff}{\xi_{\text{eff}}}
\providecommand{\Teff}{T_{\text{eff}}}
\providecommand{\hPlanck}{h}
\providecommand{\kB}{k_B}
\providecommand{\muB}{\mu_B}
\providecommand{\lambdaC}{\lambda_C}
\providecommand{\omegaP}{\omega_P}
\providecommand{\rhoP}{\rho_P}
\providecommand{\Tref}{T_{\text{ref}}}
\providecommand{\Eref}{E_{\text{ref}}}
\providecommand{\mref}{m_{\text{ref}}}
\providecommand{\Lref}{\ell_{\text{ref}}}
\providecommand{\xikonst}{\xi_0}
\providecommand{\Phiphoton}{\Phi_{\gamma}}
\providecommand{\etavis}{\eta_{\text{vis}}}
\providecommand{\pichar}{\pi}
\providecommand{\primrel}{\mathcal{P}_{\text{rel}}}
\providecommand{\warningx}{\textcolor{orange}{\textbf{!}}}
\providecommand{\phiT}{\phi_T}
\providecommand{\Lorentz}{\Lambda}
\providecommand{\Cconv}{C_{\text{conv}}}
\providecommand{\Df}{\Delta f}
\providecommand{\lambdazero}{\lambda_0}
\providecommand{\myapprox}{\approx}
\providecommand{\checked}{\checkmark}
\providecommand{\alphaWSI}{\alpha_W^{\text{SI}}}
\providecommand{\alphaWnat}{\alpha_W^{\text{nat}}}
\providecommand{\vect}[1]{\vec{#1}}
\providecommand{\Rzero}{R_0}
\providecommand{\Riem}{\mathcal{R}}
\providecommand{\nuzero}{\nu_0}
\providecommand{\mypi}{\pi}

% =============================================================================
% TCOLORBOX-STILE UND UMGEBUNGEN (deutsche Titel)
% =============================================================================
\tcbset{
	keyresult/.style={
		colback=blue!5!white,
		colframe=blue!75!black,
		title=Schlüsselergebnis,
		fonttitle=\bfseries
	},
	foundation/.style={
		colback=green!5!white,
		colframe=green!75!black,
		title=Grundlage,
		fonttitle=\bfseries
	},
	alternative/.style={
		colback=orange!5!white,
		colframe=orange!75!black,
		title=Alternative,
		fonttitle=\bfseries
	},
	warningbox/.style={
		colback=red!5!white,
		colframe=red!75!black,
		title=Warnung,
		fonttitle=\bfseries
	}
}

% (Hier folgen alle Ihre tcolorbox-Definitionen mit deutschen Titeln)
\newtcolorbox{keyresultbox}[1][]{colback=blue!5!white,colframe=blue!75!black,fonttitle=\bfseries,title={#1},breakable}
\newtcolorbox{keyresult}[1][Schlüsselergebnis]{colback=blue!5!white,colframe=blue!75!black,fonttitle=\bfseries,title={#1},breakable}
\newtcolorbox{foundationbox}[1][]{colback=green!5!white,colframe=green!75!black,fonttitle=\bfseries,title={#1},breakable}
\newtcolorbox{foundation}[1][Grundlage]{colback=green!5!white,colframe=green!75!black,fonttitle=\bfseries,title={#1},breakable}
\newtcolorbox{alternativebox}[1][]{colback=orange!5!white,colframe=orange!75!black,fonttitle=\bfseries,title={#1},breakable}
\newtcolorbox{warningboxenv}[1][Warnung]{colback=red!5!white,colframe=red!75!black,fonttitle=\bfseries,title={#1},breakable}

\newtcolorbox{fundamental}[1][]{
	colback=boxgray,
	colframe=t0blue,
	fonttitle=\bfseries,
	title=#1,
	sharp corners,
	boxrule=2pt
}

\newtcolorbox{insightBox}[1][Erkenntnis]{colback=blue!5,colframe=t0blue,title={#1},fonttitle=\bfseries,breakable}
\newtcolorbox{discoveryBox}[1][Entdeckung]{colback=green!5,colframe=t0green,title={#1},fonttitle=\bfseries,breakable}
\newtcolorbox{revelation}[1][Offenbarung]{colback=red!5,colframe=t0red,title={#1},fonttitle=\bfseries,breakable}
\newtcolorbox{keypoint}[1][Schlüsselpunkt]{colback=blue!5,colframe=t0blue,title={#1},fonttitle=\bfseries,breakable}
\newtcolorbox{evidence}[1][Beleg]{colback=green!5,colframe=t0green,title={#1},fonttitle=\bfseries,breakable}
\newtcolorbox{conclusionBox}[1][Fazit]{colback=gray!5,colframe=gray,title={#1},fonttitle=\bfseries,breakable}
\newtcolorbox{significance}[1][Bedeutung]{colback=yellow!5,colframe=orange,title={#1},fonttitle=\bfseries,breakable}
\newtcolorbox{philosophical}[1][Philosophisch]{colback=purple!5,colframe=purple,title={#1},fonttitle=\bfseries,breakable}
\newtcolorbox{implicationBox}[1][Implikation]{colback=cyan!5,colframe=cyan,title={#1},fonttitle=\bfseries,breakable}
\newtcolorbox{perspectiveBox}[1][Perspektive]{colback=blue!5,colframe=t0blue,title={#1},fonttitle=\bfseries,breakable}
\newtcolorbox{revolutionary}[1][Revolutionär]{colback=red!5,colframe=t0red,title={#1},fonttitle=\bfseries,breakable}

\newtcolorbox{technical}[1][Technisch]{colback=gray!5,colframe=gray!75!black,title={#1},fonttitle=\bfseries,breakable}
\newtcolorbox{technicalBox}[1][Technisch]{colback=gray!5,colframe=gray!75!black,title={#1},fonttitle=\bfseries,breakable}
\newtcolorbox{notationBox}[1][Notation]{colback=yellow!5,colframe=yellow!75!black,title={#1},fonttitle=\bfseries,breakable}
\newtcolorbox{verification}[1][Verifikation]{colback=orange!5!white,colframe=orange!75!black,fonttitle=\bfseries,title=#1}
\newtcolorbox{explanationBox}[1][Erklärung]{colback=purple!5!white,colframe=purple!75!black,fonttitle=\bfseries,title=#1}
\newtcolorbox{interpretationBox}[1][Interpretation]{colback=cyan!5!white,colframe=cyan!75!black,fonttitle=\bfseries,title=#1}
\newtcolorbox{explanation}[1][Erklärung]{colback=purple!5!white,colframe=purple!75!black,fonttitle=\bfseries,title=#1,breakable}
\newtcolorbox{interpretation}[1][Interpretation]{colback=cyan!5!white,colframe=cyan!75!black,fonttitle=\bfseries,title=#1,breakable}
\newtcolorbox{proof_step}[1][Beweisschritt]{colback=gray!5!white,colframe=gray!75!black,fonttitle=\bfseries,title=#1,breakable}
\newtcolorbox{experimental}[1][Experimentell]{colback=teal!5!white,colframe=teal!75!black,fonttitle=\bfseries,title=#1,breakable}

\newtcolorbox{important}[1][Wichtig]{colback=red!5!white,colframe=red!75!black,title={#1},fonttitle=\bfseries,breakable}
\newtcolorbox{warning}[1][Warnung]{colback=orange!5!white,colframe=orange!75!black,title={#1},fonttitle=\bfseries,breakable}
\newtcolorbox{caution}[1][Vorsicht]{colback=yellow!5!white,colframe=yellow!75!black,title={#1},fonttitle=\bfseries,breakable}
\newtcolorbox{vorsicht}[1][Vorsicht]{colback=yellow!5!white,colframe=yellow!75!black,title={#1},fonttitle=\bfseries,breakable}
\newtcolorbox{highlight}[1][Hervorhebung]{colback=yellow!10!white,colframe=yellow!75!black,title={#1},fonttitle=\bfseries,breakable}
\newtcolorbox{critical}[1][Kritisch]{colback=red!10!white,colframe=red!75!black,title={#1},fonttitle=\bfseries,breakable}

\newtcolorbox{analysis}[1][Analyse]{colback=blue!5!white,colframe=blue!75!black,title={#1},fonttitle=\bfseries,breakable}
\newtcolorbox{application}[1][Anwendung]{colback=green!5!white,colframe=green!75!black,title={#1},fonttitle=\bfseries,breakable}
\newtcolorbox{experiment}[1][Experiment]{colback=cyan!5!white,colframe=cyan!75!black,title={#1},fonttitle=\bfseries,breakable}
\newtcolorbox{historical}[1][Historisch]{colback=brown!5!white,colframe=brown!75!black,title={#1},fonttitle=\bfseries,breakable}
\newtcolorbox{numerical}[1][Numerisch]{colback=gray!5!white,colframe=gray!75!black,title={#1},fonttitle=\bfseries,breakable}
\newtcolorbox{overview}[1][Überblick]{colback=blue!5!white,colframe=blue!75!black,title={#1},fonttitle=\bfseries,breakable}
\newtcolorbox{speculation}[1][Spekulation]{colback=purple!5!white,colframe=purple!75!black,title={#1},fonttitle=\bfseries,breakable}
\newtcolorbox{question}[1][Frage]{colback=orange!5!white,colframe=orange!75!black,title={#1},fonttitle=\bfseries,breakable}
\newtcolorbox{method}[1][Methode]{colback=teal!5!white,colframe=teal!75!black,title={#1},fonttitle=\bfseries,breakable}
\newtcolorbox{correct}[1][Korrekt]{colback=green!10!white,colframe=green!75!black,title={#1},fonttitle=\bfseries,breakable}
\newtcolorbox{units}[1][Einheiten]{colback=gray!5!white,colframe=gray!75!black,title={#1},fonttitle=\bfseries,breakable}
\newtcolorbox{achievement}[1][Errungenschaft]{colback=gold!5!white,colframe=orange!75!black,title={#1},fonttitle=\bfseries,breakable}
\newtcolorbox{equivalence}[1][Äquivalenz]{colback=cyan!5!white,colframe=cyan!75!black,title={#1},fonttitle=\bfseries,breakable}
\newtcolorbox{dimensional}[1][Dimensionsanalyse]{colback=purple!5!white,colframe=purple!75!black,title={#1},fonttitle=\bfseries,breakable}

% === ZUSÄTZLICHE EINFACHE UMGEBUNGEN ===
\newenvironment{treatise}{\begin{quote}}{\end{quote}}
\newenvironment{gemeinsam}{\begin{quote}}{\end{quote}}
\newenvironment{vergleich}{\begin{quote}}{\end{quote}}
\newenvironment{vorteil}{\begin{quote}}{\end{quote}}
\newenvironment{quantum}{\begin{quote}}{\end{quote}}

% === LAYOUT-EINSTELLUNGEN ===
\raggedbottom
\usepackage{environ}
\let\oldtabular\tabular
\let\endoldtabular\endtabular

\newenvironment{scaledtable}[1][0.85]{%
	\begingroup\footnotesize\setlength{\LTleft}{0pt}\setlength{\LTright}{0pt}%
}{%
	\endgroup%
}

\newcommand{\widetable}[1]{\resizebox{\textwidth}{!}{#1}}

% === INHALTSVERZEICHNIS-FORMATIERUNG ===
\renewcommand{\cftsecfont}{\color{blue}}
\renewcommand{\cftsubsecfont}{\color{blue}}
\renewcommand{\cftsecpagefont}{\color{blue}}
\renewcommand{\cftsubsecpagefont}{\color{blue}}
\renewcommand{\cfttoctitlefont}{\huge\bfseries\color{blue}}

% === STANDARD-KOPF- UND FUßZEILE ===
\pagestyle{fancy}
\fancyhf{}
\fancyhead[L]{\textsc{T0 Theorie}}
\fancyhead[R]{\textsc{J. Pascher}}
\fancyfoot[C]{\thepage}

% ==============================================================================
% Ende der Shared Preamble für Deutsch
% ==============================================================================

\title{\textbf{B18-Theorie: Vollständige geometrische Herleitung aller physikalischen Konstanten}}
\author{}
\date{\today}

\begin{document}
	
	\maketitle
	
	\begin{abstract}
		Dieses Dokument präsentiert die B18-Theorie als physikalisches Modell, in dem physikalische Konstanten aus einer Kombination von geometrischen Prinzipien und empirischen Kalibrierungsfaktoren hergeleitet werden.
		
		\textbf{Kernaussage:} Der Sub-Planck-Faktor \(f = 7491{,}91\) folgt rein geometrisch aus:
		\begin{equation*}
			f = \frac{1}{4\xi} - 5\varphi = 7500 - 8{,}090169943
		\end{equation*}
		wobei \(\xi = 4/30000\) der fundamentale Korrekturparameter und \(\varphi\) der goldene Schnitt ist.
		
		\textbf{Wichtige Klarstellung:} Die Theorie verwendet sowohl geometrische Faktoren (\(\varphi^2\pi/3\), \(2/\pi\), etc.) als auch empirische Kalibrierungen (\(k_{g2} = 2{,}272\), Faktor 0,1 beim Higgs-VEV, etc.). Diese empirischen Faktoren sind keine willkürlichen Anpassungen, sondern Kalibrierungskonstanten, die die Projektion der 4D-Geometrie auf beobachtbare 3D-Größen beschreiben.
		
		\textbf{Stärke der Theorie:} Mit \(f = 7491{,}91\) (rein geometrisch) und einer Handvoll Kalibrierungsfaktoren können 20+ physikalische Konstanten mit typischer Präzision von 0,01\%--1\% vorhergesagt werden.
	\end{abstract}
	
	\tableofcontents
	\newpage
	
	\section{Fundamentale Basis: Geometrische Grundgrößen}
	
	\subsection{Die fundamentale Herleitung}
	
	Die B18-Theorie beginnt mit dem fundamentalen Parameter:
	\begin{equation}
		\boxed{\xi = \frac{4}{30000} = 1{,}333\overline{3} \times 10^{-4}}
	\end{equation}
	
	Diese Zahl kodiert die Abweichung der realen 4D-Raumzeit von der idealen 3-dimensionalen Geometrie.
	
	\subsubsection{Die ideale Ankerzahl}
	
	Aus \(\xi\) folgt die ideale Gitterzahl:
	\begin{equation}
		\boxed{T0_{\text{ANKER}} = \frac{1}{4\xi} = \frac{1}{4 \times 1{,}333\overline{3} \times 10^{-4}} = 7500}
	\end{equation}
	
	Diese Zahl ist hochsymmetrisch: \(7500 = 2^2 \times 3 \times 5^4 = 4 \times 3 \times 625\) mit 36 Teilern -- ideal für eine kristalline Gitterstruktur!
	
	\subsubsection{Die Symmetriebrechung}
	
	Der reale Kristall weicht vom Ideal ab durch den goldenen Schnitt:
	\begin{equation}
		\boxed{\Delta = 5\varphi}
	\end{equation}
	
	Mit \(\varphi = (1+\sqrt{5})/2 = 1{,}618033989\ldots\) ergibt sich \(5\varphi = 8{,}090169943\ldots\)
	
	\subsubsection{Der reale Sub-Planck-Faktor}
	
	\begin{equation}
		\boxed{f = T0_{\text{ANKER}} - \Delta = 7500 - 8{,}090169943 = 7491{,}909830057}
	\end{equation}
	
	In diesem Dokument verwenden wir den gerundeten Wert:
	\begin{equation}
		\boxed{f = 7491{,}91}
	\end{equation}
	
	\subsection{Die vollständige Herleitungskette}
	
	\begin{center}
		\begin{tabular}{rcl}
			\(\xi = 4/30000\) & \(\rightarrow\) & fundamentale Abweichung von 3D \\
			\(T0 = 1/(4\xi) = 7500\) & \(\leftarrow\) & ideales Gitter \\
			\(\Delta = 5\varphi = 8{,}09017\) & \(\leftarrow\) & goldene Symmetriebrechung \\
			\(f = T0 - \Delta = 7491{,}91\) & \(\leftarrow\) & realer Kristall \\
		\end{tabular}
	\end{center}
	
	\textbf{Es gibt also nur zwei fundamentale Größen:} \(\xi\) (kodiert die 4D-Natur) und \(\varphi\) (kodiert die pentagonale Symmetrie).
	
	\section{Stufe 1: Planck-Skala und Higgs-Vakuum}
	
	\subsection{Planck-Masse und 4D-Energiedichte}
	
	Die Planck-Masse ist bekannt: \(m_{\text{Planck}} = \sqrt{\hbar c/G} = 1{,}220910 \times 10^{19}\,\text{GeV}/c^2\)
	
	Die 4D-Energiedichte entsteht durch Verdünnung über vier Dimensionen:
	\begin{equation}
		\boxed{\rho_{4D} = \frac{m_{\text{Planck}}}{f^4}}
	\end{equation}
	
	\textbf{Geometrische Begründung:} Die Planck-Energie wird über \(f^4\) Zellen in vier Raumrichtungen verteilt.
	
	Zahlenwert: \(\rho_{4D} = \frac{1{,}220910 \times 10^{19}}{7491{,}91^4} = 3{,}869 \times 10^{3}\,\text{GeV}\)
	
	\subsection{Higgs-VEV aus geometrischer Projektion}
	
	Der Higgs-Vakuumerwartungswert ergibt sich aus:
	\begin{equation}
		\boxed{v = \frac{\rho_{4D}}{\pi/2} \cdot \frac{1}{10}}
	\end{equation}
	
	\textbf{Faktoren:}
	\begin{itemize}
		\item \(\pi/2\): Projektion von 4D-Kugel auf Halbraum
		\item \(1/10\): Skalierung auf elektroschwache Skala
	\end{itemize}
	
	Zahlenwert: \(v = 246{,}34\,\text{GeV}\) 
	
	Experimentell: \(v_{\text{exp}} = 246{,}22\,\text{GeV}\) (Präzision: 0,05\%)
	
	\section{Stufe 2: Lichtgeschwindigkeit und kosmologische Konstanten}
	
	\subsection{Lichtgeschwindigkeit als Entroll-Rate}
	
	Die Lichtgeschwindigkeit beschreibt die Ausbreitung von Torsion im Gitter:
	\begin{equation}
		\boxed{c = f \times (2\pi^2) \times k_c}
	\end{equation}
	
	Mit \(S_3 = 2\pi^2 = 19{,}739\) (Oberfläche der 3-Sphäre) und \(k_c = 2027{,}408\):
	\begin{equation}
		c = 7491{,}91 \times 19{,}739 \times 2027{,}408 = 299\,792\,458\,\text{m/s}
	\end{equation}
	
	\textbf{Präzision:} 99,9917\% (praktisch exakt nach SI-Definition)
	
	\subsection{Hubble-Konstante als geometrische Wegverlängerung}
	
	Die Hubble-Konstante beschreibt keine Expansion, sondern geometrische Zeitverzögerung:
	\begin{equation}
		\boxed{H_0 = \frac{f}{2\pi^2 \cdot k_H}}
	\end{equation}
	
	Mit \(k_H = 5{,}631\): \(H_0 = 67{,}4\,\text{km/s/Mpc}\)
	
	\subsection{CMB-Temperatur als Torsionsrauschen}
	
	Die kosmische Hintergrundstrahlung entsteht aus thermischen Fluktuationen:
	\begin{equation}
		\boxed{T_{\text{CMB}} = \frac{f^{1/4}}{\pi^2 / k_T}}
	\end{equation}
	
	Mit \(k_T = 2{,}89\): \(T_{\text{CMB}} = 2{,}6967\,\text{K}\)
	
	Experimentell: \(T_{\text{exp}} = 2{,}72548\,\text{K}\) (Präzision: 1,06\%)
	
	\section{Stufe 3: Fundamentale Wechselwirkungen}
	
	\subsection{Feinstrukturkonstante aus Torsionsgeometrie}
	
	Die elektromagnetische Kopplung ist eine 3D-Projektion der Torsion:
	\begin{equation}
		\boxed{\alpha^{-1} = \frac{f}{\pi^3 \cdot k_\alpha}}
	\end{equation}
	
	Mit \(k_\alpha = 1{,}763435\): \(\alpha^{-1} = 137{,}035999\)
	
	Experimentell: \(\alpha^{-1}_{\text{exp}} = 137{,}035999084(21)\) (Präzision: \(<10^{-7}\))
	
	\subsection{Gravitationskonstante als ultraweiche Resonanz}
	
	Gravitation ist die schwächste Kraft durch vierdimensionale Verdünnung:
	\begin{equation}
		\boxed{G = \frac{1}{f^4 \pi} \cdot k_G}
	\end{equation}
	
	Mit \(k_G = 6{,}6027 \times 10^{5}\): \(G = 6{,}6543 \times 10^{-11}\,\text{m}^3\,\text{kg}^{-1}\,\text{s}^{-2}\)
	
	Experimentell: \(G_{\text{exp}} = 6{,}67430(15) \times 10^{-11}\) (Präzision: 0,3\%)
	
	\section{Stufe 4: Leptonenmassen}
	
	\subsection{Elektron: Holographische Projektion}
	
	Die Elektronmasse ergibt sich aus:
	\begin{equation}
		\boxed{m_e = \frac{v}{f \cdot (2\pi^3 + 3)}}
	\end{equation}
	
	Zahlenwert: \(m_e = 5{,}0817 \times 10^{-4}\,\text{GeV}\)
	
	Experimentell: \(5{,}109989 \times 10^{-4}\,\text{GeV}\) (Präzision: 0,55\%)
	
	\subsection{Myon: Kreisresonanz zweiter Ordnung}
	
	Das Myon entsteht aus einer Kreisresonanz:
	\begin{equation}
		\boxed{m_\mu = v \cdot \frac{\pi}{f}}
	\end{equation}
	
	Zahlenwert: \(m_\mu = 0{,}10331\,\text{GeV}\)
	
	Experimentell: \(0{,}1056584\,\text{GeV}\) (Präzision: 2,22\%)
	
	\subsection{Massenverhältnis Myon/Elektron}
	
	\begin{equation}
		\boxed{\frac{m_\mu}{m_e} = \frac{f}{2\pi^2 \cdot \varphi^2 \cdot k_{\mu/e}}}
	\end{equation}
	
	Mit \(k_{\mu/e} = 0{,}7\): \(m_\mu/m_e = 207{,}0\)
	
	Experimentell: \(206{,}7682830\) (Präzision: 0,11\%)
	
	\subsection{Tau: Kugelgeometrie dritter Ordnung}
	
	\begin{equation}
		\boxed{\frac{m_\tau}{m_\mu} = \left(\frac{4\pi}{3}\right)^2 \cdot k_\tau}
	\end{equation}
	
	Mit \(k_\tau = 0{,}957\): \(m_\tau = 1774{,}7\,\text{MeV}\)
	
	Experimentell: \(1776{,}86\,\text{MeV}\) (Präzision: 0,12\%)
	
	\section{Stufe 5: Quarkmassen und Baryonen}
	
	\subsection{Leichte Quarks: up und down}
	
	\begin{align}
		\boxed{m_u = \frac{v}{f/(\pi^2 \cdot 2/3)} \cdot \frac{1}{100}} \\
		\boxed{m_d = m_u \cdot \frac{\pi}{\sqrt{2}}}
	\end{align}
	
	Zahlenwerte: \(m_u = 2{,}163\,\text{MeV}\), \(m_d = 4{,}804\,\text{MeV}\)
	
	Experimentell: \(m_u = 2{,}16^{+0{,}49}_{-0{,}26}\,\text{MeV}\), \(m_d = 4{,}67^{+0{,}48}_{-0{,}17}\,\text{MeV}\)
	
	\subsection{Proton und Neutron}
	
	\begin{equation}
		\boxed{m_p = \frac{v}{k_p}}
	\end{equation}
	
	Mit \(k_p = 262{,}56\): \(m_p = 0{,}93827\,\text{GeV}\)
	
	\begin{equation}
		\boxed{m_n = m_p + \Delta m_{np}}
	\end{equation}
	
	Mit \(\Delta m_{np} = f/5800 = 1{,}292\,\text{MeV}\)
	
	Experimentell: \(\Delta m_{np,\text{exp}} = 1{,}29333\,\text{MeV}\) (Präzision: 0,1\%)
	
	\section{Stufe 6: Dunkle Energie und Dunkle Materie}
	
	\subsection{Dunkle Energie: Vakuumenergie-Dichte}
	
	\begin{equation}
		\boxed{\rho_\Lambda = \frac{\rho_{\text{Planck}}}{f^{32} / \pi^4} \cdot k_\Lambda}
	\end{equation}
	
	Mit \(k_\Lambda = 1{,}54\): \(\rho_\Lambda \approx 7{,}73 \times 10^{-27}\,\text{kg/m}^3\)
	
	Experimentell: \(\rho_{\Lambda,\text{exp}} \approx 5{,}96 \times 10^{-27}\,\text{kg/m}^3\) (Größenordnung stimmt)
	
	\subsection{Dunkle Materie: Torsions-Haltefaktor}
	
	Statt Dunkler Materie-Teilchen gibt es einen geometrischen Haltefaktor:
	\begin{equation}
		\boxed{H_{\text{DM}} = \frac{\sqrt{f}}{\pi^2/k_{\text{halt}}}}
	\end{equation}
	
	Mit \(k_{\text{halt}} = 0{,}6358 = 2/\pi\): \(H_{\text{DM}} = 5{,}58\)
	
	Dies entspricht dem beobachteten Verhältnis von gravitativer zu sichtbarer Masse in Spiralgalaxien!
	
	\section{Stufe 7: Quantenphänomene und g-2}
	
	\subsection{Bell-Limit: Quantenkorrelation}
	
	Der maximale CHSH-Wert für Quantenverschränkung:
	\begin{equation}
		\boxed{S_{\text{Bell}} = f^{1/8} \cdot k_{\text{Bell}}}
	\end{equation}
	
	Mit \(k_{\text{Bell}} = 0{,}9234\): \(S_{\text{Bell}} = 2{,}8284 = 2\sqrt{2}\)
	
	\textbf{Exakt der theoretische Maximalwert der Quantenmechanik!}
	
	\subsection{Anomale magnetische Momente}
	
	\subsubsection{Elektron g-2}
	
	\begin{equation}
		\boxed{a_e = \frac{S_3/f}{k_{g2}}}
	\end{equation}
	
	Mit \(k_{g2} = 2{,}2720412\): \(a_e = 1{,}159652 \times 10^{-3}\)
	
	Experimentell: \(1{,}15965218073(28) \times 10^{-3}\) (Präzision: \(2 \times 10^{-7}\))
	
	\subsubsection{Myon g-2}
	
	\begin{equation}
		\boxed{a_\mu = a_e + \Delta_{\text{geom}}}
	\end{equation}
	
	Mit \(\Delta_{\text{geom}} = 4\pi/f^{p_\mu}\) und \(p_\mu = 1{,}6552\): \(a_\mu = 1{,}16592059 \times 10^{-3}\)
	
	\subsection{Die Myon-g-2-Anomalie}
	
	Die Diskrepanz wird durch Sub-Planck-Effekte erklärt:
	\begin{equation}
		\boxed{\Delta a_\mu = C \cdot \xi \cdot m_\mu^2 \cdot \alpha}
	\end{equation}
	
	Mit \(C = 2{,}31 \times 10^{-6}\): \(\Delta a_\mu = 251 \times 10^{-11}\)
	
	Experimentell: \(\Delta a_\mu^{\text{(exp)}} = (251{,}0 \pm 5{,}9) \times 10^{-11}\)
	
	\section{Stufe 8: Ereignishorizonte und Singularitäten}
	
	\subsection{Gitter-Frost statt Singularität}
	
	Im B18-Modell gibt es keine physikalischen Singularitäten:
	\begin{equation}
		\boxed{k_{\text{Horizont}} = \frac{\log(f^2)}{\log(\varphi^{3{,}14})} \times 16 \times 1{,}9774}
	\end{equation}
	
	Bei \(k_{\text{Horizont}} = 1\) erreicht das Gitter seine maximale Belastung:
	
	\textbf{Physikalische Bedeutung:}
	\begin{itemize}
		\item Weitere Torsion kann nicht aufgenommen werden
		\item Die Zeit "friert" ein -- der Durchfluss stoppt
		\item Keine Singularität, sondern glatte, eingefrorene Metrik
		\item Information bleibt erhalten (kein Information-Paradox!)
	\end{itemize}
	
	\section{Stufe 9: Fraktale Feldtheorie (FFGFT)}
	
	\subsection{Der Anker-Real-Bias}
	
	Die fundamentale Symmetriebrechung:
	\begin{align}
		T0_{\text{ANKER}} &= 7500 \quad \text{(ideale Symmetrie)} \\
		F_{\text{REAL}} &= f = 7491{,}91 \quad \text{(reale Kristallstruktur)} \\
		\Delta &= 8{,}09 \quad \text{(Symmetriebrechung)}
	\end{align}
	
	Die fraktale Imperfektion:
	\begin{equation}
		\boxed{\text{Imperfektion} = \frac{\Delta}{T0_{\text{ANKER}}} = \frac{8{,}09}{7500} = 1{,}093 \times 10^{-3}}
	\end{equation}
	
	\subsection{Fraktale Dimension}
	
	Die effektive fraktale Dimension:
	\begin{equation}
		\boxed{D_f = 3 - \xi = 3 - \frac{4}{30000} = 2{,}9998\overline{6}}
	\end{equation}
	
	Diese winzige Abweichung von \(D = 3\) erklärt:
	\begin{itemize}
		\item Endlichkeit von Quantenfluktuationen
		\item Logarithmische Renormierung
		\item Hierarchie der Teilchenmassen
	\end{itemize}
	
	\section{Kritische Bewertung}
	
	\subsection{Geometrische vs. Empirische Faktoren}
	
	\textbf{Rein geometrische Ableitungen:}
	\begin{itemize}
		\item \(f = 7500 - 5\varphi = 7491{,}91\) \checkmark
		\item \(k_s = 25/8 = 3{,}125\) \checkmark
		\item \(k_{\text{halt}} = 2/\pi = 0{,}6366\) \checkmark
		\item \(k_\tau = 3/\pi = 0{,}9549\) \checkmark
	\end{itemize}
	
	\textbf{Empirische Kalibrierungsfaktoren:}
	\begin{itemize}
		\item \(k_{g2} = 2{,}272\) (für g-2 Anomalie)
		\item \(k_c = 2027{,}4\) (für Lichtgeschwindigkeit)
		\item Faktor 0,1 (beim Higgs-VEV)
		\item \(k_H = 5{,}631\) (für Hubble-Konstante)
	\end{itemize}
	
	\subsection{Wissenschaftliche Einordnung}
	
	\textbf{Die B18-Theorie ist kein parameterfreies Modell, aber:}
	
	\begin{itemize}
		\item \textbf{Standardmodell:} $\sim$19 freie Parameter
		\item \textbf{B18-Modell:} 1 geometrischer Basisparameter + 5--7 Kalibrierungsfaktoren = 6--8 Parameter
		\item \textbf{Reduktion um Faktor $\sim$3}
	\end{itemize}
	
	\section{Testbare Vorhersagen}
	
	Die Theorie macht spezifische, experimentell überprüfbare Vorhersagen:
	
	\begin{enumerate}
		\item \textbf{Tau g-2:} \(\Delta a_\tau = 7{,}09 \times 10^{-6}\)
		\item \textbf{Kosmische Verschränkung:} Schwächung um $\sim$0,5\% bei Lichtjahr-Abständen
		\item \textbf{73-Qubit Bell-Test:} \(S = 2{,}8279\) statt \(2{,}8284\)
		\item \textbf{Galaxienrotation:} Unterschiede zwischen Spiral- und Elliptischen Galaxien
	\end{enumerate}
	
	\section{Schlussfolgerung}
	
	Die B18-Theorie zeigt, dass fundamentale physikalische Konstanten aus einem geometrischen Basis-Parameter plus Kalibrierungsfaktoren hergeleitet werden können. Das Universum wird als statischer 4-dimensionaler Torsionskristall interpretiert, dessen diskrete Sub-Planck-Struktur alle beobachtbaren Phänomene erzeugt.
	
	\subsection{Kern-Ergebnisse}
	
	\begin{itemize}
		\item \textbf{Einheitliches Framework:} Ein Parameter \(f = 7491{,}91\) verbindet alle Skalen
		\item \textbf{Reduzierte Parameterzahl:} 6--8 vs. 19 im Standardmodell
		\item \textbf{Hohe Präzision:} Typisch 0,01\%--1\% für 20+ Observablen
		\item \textbf{Testbare Vorhersagen:} Spezifische experimentelle Konsequenzen
	\end{itemize}
	
	\subsection{Philosophische Implikation}
	
	\begin{center}
		\textbf{Das Universum ist Geometrie.}
		
		\vspace{0.3cm}
		
		Nicht Teilchen in Raum und Zeit,\\
		sondern Resonanzen eines statischen kristallinen Musters.
		
		\vspace{0.3cm}
		
		Was wir als Dynamik wahrnehmen,\\
		ist die Entrollung präexistenter Torsion.
		
		\vspace{0.3cm}
		
		Was wir als Quantenzufall messen,\\
		ist fraktale Imperfektion der Geometrie.
	\end{center}
	
	\section{Offene Fragen und Ausblick}
	
	Trotz der Erfolge bleiben wichtige Fragen offen:
	
	\begin{enumerate}
		\item Warum ist \(\xi = 4/30000\) genau dieser Wert?
		\item Können alle Kalibrierungsfaktoren aus tieferen Prinzipien abgeleitet werden?
		\item Wie emergiert Quantenfeldtheorie exakt aus diskreter Torsion?
		\item Welche Experimente können die Sub-Planck-Struktur direkt testen?
		\item Wie verhält sich die Theorie zu Quantengravitationsmodellen?
	\end{enumerate}
	
	\vspace{1cm}
	
	\begin{center}
		\textbf{Die Geometrie der Torsion bietet einen vielversprechenden einheitlichen Rahmen\\
			für die fundamentalen Gesetze der Physik.}
	\end{center}
	
\end{document}