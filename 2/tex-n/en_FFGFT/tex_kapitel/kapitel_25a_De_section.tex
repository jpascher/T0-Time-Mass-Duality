Das Neutrino-Massen-Problem umfasst offene Fragen im Standardmodell: Warum sind Neutrinomassen so klein (\(\sim \SIrange{0.01}{0.1}{\ev}/c^2\))? Warum genau drei Generationen? Majorana- oder Dirac-Natur? Willkürliche PMNS-Mischung? In der fraktalen Fundamental Fractal-Geometric Field Theory (FFGFT) mit T0-Time-Mass-Dualität werden alle Rätsel gelöst: Neutrinos sind reine Phasen-Anregungen des Vakuumfeldes \(\Phi = \rho(x,t) e^{i\theta(x,t)}\), reguliert durch den einzigen fundamentalen Parameter \(\xi = \frac{4}{3} \times 10^{-4}\) (dimensionless).
	
	\subsection{Symbolverzeichnis und Einheiten}
	
	\begin{tcolorbox}[title={\textbf{Important Symbols and their Units}}, colback=blue!5!white, colframe=blue!75!black]
		\begin{tabular}{p{0.3\textwidth}p{0.3\textwidth}p{0.35\textwidth}}
			\textbf{Symbol} & \textbf{Meaning} & \textbf{Unit (SI)} \\
			\hline
			\(\xi\) & Fraktaler Skalenparameter & dimensionless \\
			\(m_{\nu_i}\) & Masse des $i$-ten Neutrinos & \si{\kilo\gram} (\si{\ev\per c\squared}) \\
			\(K_\nu\) & Skalenfaktor für Neutrinomassen & \si{\kilo\gram} (\si{\ev\per c\squared}) \\
			\(\theta_{\nu_i}\) & Charakteristische Phase des $i$-ten Neutrinos & dimensionless (radiant) \\
			\(m_0^\nu\) & Referenzmasse für Neutrinos & \si{\kilo\gram} (\si{\ev\per c\squared}) \\
			\(\Delta \theta_{\min}\) & Minimale Phasenverschiebung & dimensionless (radiant) \\
			\(m_1, m_2, m_3\) & Massen der drei Neutrinogenerationen & \si{\kilo\gram} (\si{\ev\per c\squared}) \\
			\(U_{ij}\) & Element der PMNS-Mischungsmatrix & dimensionless \\
			\(\Delta \theta_{ij}\) & Phasenunterschied zwischen Moden $i$ und $j$ & dimensionless (radiant) \\
			\(\nu\) & Neutrino & -- \\
			\(\nu^c\) & Antineutrino (selbstkonjugiert) & -- \\
			\(\sum m_\nu\) & Summe der Neutrinomassen & \si{\kilo\gram} (\si{\ev\per c\squared}) \\
			\(\hbar\) & Reduziertes Plancksches Wirkungsquantum & \si{\joule\second} \\
			\(c\) & Speed of light & \si{\meter\per\second} \\
			\(l_0\) & Fraktale Korrelationslänge & \si{\meter} \\
			\(\Phi\) & Komplexes Vakuumfeld & \si{\kilo\gram^{1/2}\per\meter^{3/2}} \\
			\(\rho(x,t)\) & Vakuum-Amplitudendichte & \si{\kilo\gram^{1/2}\per\meter^{3/2}} \\
			\(\theta(x,t)\) & Vakuumphasenfeld & dimensionless (radiant) \\
			\(\delta_i\) & Perturbation der Phase & dimensionless (radiant) \\
			\(\theta_0\) & Basisphase & dimensionless (radiant) \\
		\end{tabular}
	\end{tcolorbox}
	
	\textbf{Unit Check (Neutrinomasse):}
	\begin{align*}
		[m_{\nu_i}] &= \si{\kilo\gram} \cdot \text{dimensionless} = \si{\kilo\gram} \quad (\text{oder } \si{\ev\per c\squared})
	\end{align*}
	Einheiten konsistent.
	
	\subsection{Neutrinos als reine Phasen-Anregungen}
	
	In T0 haben Neutrinos keine Amplitude-Deformation (\(\delta \rho = 0\)) und sind reine Phasen-Excitationen:
	\begin{equation}
		m_\nu = m_0^\nu \cdot |e^{i \theta_\nu} - 1|^2 = 2 m_0^\nu \sin^2(\theta_\nu / 2)
	\end{equation}
	
	Da Neutrinos reine Phase sind, ist \(m_0^\nu \ll m_0^{\text{lepton}}\) – die Masse entsteht nur aus Phasenverschiebung.
	
	\textbf{Unit Check:}
	\begin{align*}
		[m_\nu] &= \si{\kilo\gram} \cdot \text{dimensionless} = \si{\kilo\gram}
	\end{align*}
	
	\subsection{Drei Generationen aus fraktaler Symmetrie}
	
	Die fraktale Hierarchie erzwingt eine dreifache Rotationalsymmetrie in der Phase:
	\begin{equation}
		\theta_{\nu_i} = \theta_0 + \frac{2\pi (i-1)}{3} + \delta_i \quad (i = 1,2,3)
	\end{equation}
	
	Dies ist analog zur Lepton-Koide-Symmetrie (Chapter 24), aber für Neutrinos fast masselos.
	
	\subsection{Ableitung der Massenhierarchie}
	
	Die minimale Phasenverschiebung ist durch fraktale Fluktuationen begrenzt:
	\begin{equation}
		\Delta \theta_{\min} \approx \xi^{3/2} \cdot \sqrt{\ln(\xi^{-1})}
	\end{equation}
	
	Die Massen:
	\begin{align}
		m_1 &\approx 2 m_0^\nu \cdot \sin^2(\theta_0 / 2), \\
		m_2 &\approx 2 m_0^\nu \cdot \sin^2((\theta_0 + 120^\circ)/2), \\
		m_3 &\approx 2 m_0^\nu \cdot \sin^2((\theta_0 + 240^\circ)/2)
	\end{align}
	
	Mit \(\theta_0 \approx \pi + \xi \cdot \Delta\):
	\begin{equation}
		m_1 : m_2 : m_3 \approx 1 : 3 : 8
	\end{equation}
	in erster Ordnung, passend zur normalen Hierarchie.
	
	Die absolute Skala:
	\begin{equation}
		m_0^\nu \approx \frac{\hbar}{c l_0} \cdot \xi^3 \approx \SI{0.05}{\ev\per c\squared}
	\end{equation}
	
	Summe der Massen:
	\begin{equation}
		\sum m_\nu \approx \SI{0.12}{\ev\per c\squared}
	\end{equation}
	konsistent mit Kosmologie.
	
	\textbf{Unit Check:}
	\begin{align*}
		[m_0^\nu] &= \si{\joule\second} / (\si{\meter\per\second} \cdot \si{\meter}) \cdot \text{dimensionless} = \si{\kilo\gram}
	\end{align*}
	
	\subsection{PMNS-Mischung aus Phasen-Kopplung}
	
	Die Mischungsmatrix ergibt sich aus Überlapp der Phasenmoden:
	\begin{equation}
		U_{ij} = \langle \theta_{\nu_i} | \theta_{l_j} \rangle \approx \cos(\Delta \theta_{ij}) + i \xi \cdot \sin(\Delta \theta_{ij})
	\end{equation}
	
	Dies reproduziert tribimaximale Mischung plus Perturbationen – exakt PMNS-Winkel.
	
	\subsection{Majorana-Natur}
	
	Da Neutrinos reine Phase sind, sind sie Majorana:
	\begin{equation}
		\nu = \nu^c, \quad \text{da } \theta \to -\theta \text{ äquivalent}
	\end{equation}
	
	\subsection{Vergleich: Standardmodell vs. T0}
	
	\begin{center}
		\begin{tabular}{p{0.45\textwidth}p{0.45\textwidth}}
			\textbf{Standardmodell} & \textbf{T0-Fraktale FFGFT} \\
			\hline
			Massen willkürlich, ad-hoc & Emergent aus Phasenmoden \\
			Seesaw-Mechanismus (postuliert) & Reine Phase, keine Amplitude \\
			Drei Generationen ad-hoc & 120°-Symmetrie der Hierarchie \\
			PMNS-Mischung frei & Aus Phasenüberlappungen \\
			Majorana unklar & Zwangsläufig Majorana \\
		\end{tabular}
	\end{center}
	
	\subsection{Conclusion}
	
	Die Fundamentale Fraktalgeometrische Feldtheorie (FFGFT, früher T0-Theorie) löst das Neutrino-Massen-Problem vollständig und parameterfrei: Kleine Massen aus reiner Phasen-Excitation, drei Generationen aus fraktaler 120°-Symmetrie, Hierarchie und Mischung aus Phasenverschiebungen mit \(\xi = \frac{4}{3} \times 10^{-4}\), Majorana-Natur aus selbstkonjugierten Oszillationen.
	
	Alle Werte (z. B. \(\sum m_\nu \approx \SI{0.12}{\ev\per c\squared}\)) emergieren natürlich aus dem einzigen fundamentalen Parameter \(\xi\), und vervollständigen die Beschreibung des Leptonsektors in der FFGFT.
