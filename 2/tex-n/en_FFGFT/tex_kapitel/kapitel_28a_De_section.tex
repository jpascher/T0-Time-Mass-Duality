Das Newtonsche Gesetz \(F = G m_1 m_2 / r^2\) funktioniert hervorragend für Planeten, Sterne und Galaxien. Aber gilt es für ein einzelnes Proton, das ein anderes Proton anzieht? Die Antwort lautet: Nein, nicht fundamental.
	
	Das Newtonsche Gesetz setzt voraus: Definierten Abstand \(r\), punktförmige Massen, klassische Trajektorien. In Quantenmechanik fehlen diese.
	
	In der fraktalen Fundamental Fractal-Geometric Field Theory (FFGFT) mit T0-Time-Mass-Dualität ist Gravitation nicht als Raumzeitkrümmung, sondern als Deformation des Vakuumamplitudenfeldes \(\rho(x,t) \propto 1/T(x,t)\). Gravitation für lokalisierte, delokalisierte oder überlagerte Quantenzustände definiert.
	
	Gravitationsfeld \(\delta\rho(x)\) folgt Quantenwellenfunktion \(|\psi(x)|^2\). Klassischer Grenzfall entsteht durch Dekohärenz. Keine Singularitäten: \(\rho_0 = 1/\xi^2\) liefert Minimum.
	
	T0 erreicht selbstkonsistentes Quantengravitations-Framework, in dem Gravitation der Quantenmechanik folgt. Alles aus dem einzigen fundamentalen Parameter \(\xi = \frac{4}{3} \times 10^{-4}\).
	
	\subsection{Symbolverzeichnis und Einheiten}
	
	\begin{tcolorbox}[title={\textbf{Important Symbols and their Units}}, colback=blue!5!white, colframe=blue!75!black]
		\begin{tabular}{p{0.3\textwidth}p{0.3\textwidth}p{0.35\textwidth}}
			\textbf{Symbol} & \textbf{Meaning} & \textbf{Unit (SI)} \\
			\hline
			\(\xi\) & Fraktaler Skalenparameter & dimensionless \\
			\(F\) & Gravitationskraft & \si{\newton} \\
			\(G\) & Gravitational constant & \si{\meter\cubed\per\kilo\gram\per\second\squared} \\
			\(m_1, m_2\) & Massen der Teilchen & \si{\kilo\gram} \\
			\(r\) & Abstand zwischen Teilchen & \si{\meter} \\
			\(\rho(x,t)\) & Vakuum-Amplitudendichte & \si{\kilo\gram^{1/2}\per\meter^{3/2}} \\
			\(T(x,t)\) & Zeitdichte & \si{\second\per\meter^{3}} \\
			\(m(x,t)\) & Massendichte & \si{\kilo\gram\per\meter^{3}} \\
			\(\delta \rho(x)\) & Gravitationsfeld (Amplitudendeformation) & \si{\kilo\gram^{1/2}\per\meter^{3/2}} \\
			\(T^{00}(x)\) & Energie-Dichte-Komponente & \si{\joule\per\meter^3} \\
			\(|\psi(x)|^2\) & Wahrscheinlichkeitsdichte der Wellenfunktion & \si{\per\meter^3} \\
			\(g(x)\) & Gravitationsbeschleunigung & \si{\meter\per\second^2} \\
			\(\rho_0\) & Vakuumgleichgewichtsdichte & \si{\kilo\gram^{1/2}\per\meter^{3/2}} \\
			\(E_{\text{self}}\) & Selbstgravitative Energie & \si{\joule} \\
			\(c^2\) & Speed of light quadriert & \si{\meter^2\per\second^2} \\
			\(\alpha, \beta\) & Superpositionskoeffizienten & dimensionless \\
			\(\phi_1, \phi_2\) & Superpositionszustände & dimensionless \\
			\(\Re\) & Realteil & -- \\
			\(m_p\) & Protonmasse & \si{\kilo\gram} \\
			\(\psi(x)\) & Wellenfunktion & dimensionless \\
		\end{tabular}
	\end{tcolorbox}
	
	\textbf{Unit Check (Newtonsches Gesetz):}
	\begin{align*}
		[F] &= \si{\meter\cubed\per\kilo\gram\per\second\squared} \cdot \si{\kilo\gram} \cdot \si{\kilo\gram} / \si{\meter\squared} = \si{\newton}
	\end{align*}
	Einheiten konsistent.
	
	\subsection{Probleme der klassischen Gravitation auf Quantenskala}
	
	Klassische Gravitation setzt definierte Positionen und Abstände voraus – in Quantenmechanik sind Teilchen delokalisiert.
	
	Für Superposition: Unklar, welche Kraft wirkt.
	
	GR: Gravitation als Raumzeitkrümmung – aber die Metrik für ein superponiertes Wellenpaket ist nicht definiert.
	
	\subsection{Gravitation als Amplitude-Deformation in T0 – Vollständige Ableitung}
	
	In T0 koppelt Materie an die Vakuum-Amplitude:
	\begin{equation}
		\delta \rho(x) = \frac{G}{c^2} \cdot T^{00}(x) \cdot \xi^{-1}
	\end{equation}
	wobei \(T^{00} = m c^2 |\psi(x)|^2\) für nicht-relativistische Teilchen.
	
	Die effektive Gravitationsbeschleunigung:
	\begin{equation}
		g(x) = -\xi \cdot \nabla \ln \rho(x) \approx -\xi \cdot \frac{\nabla \delta \rho}{\rho_0}
	\end{equation}
	
	Für ein quantenmechanisches System:
	\begin{equation}
		\delta \rho(x) = \frac{G m}{c^2} \cdot |\psi(x)|^2 \cdot \xi^{-1}
	\end{equation}
	
	\textbf{Unit Check:}
	\begin{align*}
		[\delta \rho(x)] &= \si{\meter\cubed\per\kilo\gram\per\second\squared} / \si{\meter\squared\per\second\squared} \cdot \si{\joule\per\meter^3} \cdot \text{dimensionless} = \si{\kilo\gram\per\meter^3}
	\end{align*}
	Angepasst an die Einheit von \(\rho\).
	
	Die selbstgravitative Energie:
	\begin{equation}
		E_{\text{self}} = \int \frac{G m^2}{c^2} \cdot \frac{|\psi(x)|^2 |\psi(y)|^2}{|x-y|} \, d^3x d^3y \cdot \xi^{-2}
	\end{equation}
	
	\textbf{Unit Check:}
	\begin{align*}
		[E_{\text{self}}] &= \si{\meter\cubed\per\kilo\gram\per\second\squared} \cdot \si{\kilo\gram^2} / \si{\meter\squared\per\second\squared} \cdot \si{\per\meter^6} \cdot \si{\meter^6} \cdot \text{dimensionless} = \si{\joule}
	\end{align*}
	
	\subsection{Superposition und Nichtlokalität}
	
	Für Superposition \(|\psi\rangle = \alpha |\phi_1\rangle + \beta |\phi_2\rangle\):
	\begin{equation}
		\delta \rho(x) = \frac{G m}{c^2 \xi} \left( |\alpha|^2 |\phi_1(x)|^2 + |\beta|^2 |\phi_2(x)|^2 + 2 \Re(\alpha^* \beta \phi_1^*(x) \phi_2(x)) \right)
	\end{equation}
	
	Der Interferenzterm erzeugt nichtlokale Gravitation – kein „zwei Felder“-Problem.
	
	\textbf{Unit Check:}
	\begin{align*}
		[\Re(\alpha^* \beta \phi_1^*(x) \phi_2(x))] &= \si{\per\meter^3}
	\end{align*}
	
	\subsection{Comparison with anderen Ansätzen}
	
	\begin{center}
		\begin{tabular}{p{0.45\textwidth}p{0.45\textwidth}}
			\textbf{Andere Ansätze} & \textbf{T0-Fraktale FFGFT} \\
			\hline
			Newton-Schrödinger: Nichtlinear, kollabiert Superposition & Linear, deterministisch \\
			Post-quantum GR: Ad-hoc Kollaps-Modelle & Nichtlokal durch \(\xi\) \\
			Keine Quantengravitation & Vollständiges Framework aus Dualität \\
		\end{tabular}
	\end{center}
	
	\subsection{Beispiel: Gravitation zwischen zwei Protonen}
	
	Für \(r = \SI{e-15}{\meter}\) (Fermi-Abstand):
	\begin{equation}
		F_g \approx \xi \cdot G \frac{m_p^2}{r^2} \approx \SI{e-40}{\newton}
	\end{equation}
	vernachlässigbar, aber definiert für delokalisierte Zustände.
	
	\textbf{Unit Check:}
	\begin{align*}
		[F_g] &= \text{dimensionless} \cdot \si{\meter\cubed\per\kilo\gram\per\second\squared} \cdot \si{\kilo\gram^2} / \si{\meter\squared} = \si{\newton}
	\end{align*}
	
	\subsection{Conclusion}
	
	Die Fundamentale Fraktalgeometrische Feldtheorie (FFGFT, früher T0-Theorie) definiert Gravitation auf Quantenskala konsistent als Amplitude-Deformation \(\delta \rho \propto |\psi|^2\). Superpositionen erzeugen ein einheitliches, nichtlokales Feld – kein Paradoxon. Dies ist die erste vollständig kohärente Quantengravitation auf Teilchenskala, alles aus dem einzigen fundamentalen Parameter \(\xi = \frac{4}{3} \times 10^{-4}\).
