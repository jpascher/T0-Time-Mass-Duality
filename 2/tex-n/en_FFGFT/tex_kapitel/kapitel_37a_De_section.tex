Das Vakuum in der modernen Physik ist nicht leer, sondern ein dynamisches Medium mit Quantenfluktuationen (Casimir-Effekt, Lamb-Shift) und Vakuumenergie (beitragend zur kosmologischen Konstante). Die fundamentalen Konstanten (z.~B. \(\alpha\), \(G\), \(\Lambda_{\text{QCD}}\), \(\Lambda\)) werden im Standardmodell plus ART als unabhängige Parameter behandelt, was zu Hierarchieproblemen und Feinabstimmungsfragen führt.
	
	Aktueller Stand (Dezember 2025): Die Werte der Konstanten sind hochpräzise gemessen (z.~B. \(\alpha \approx 1/137.035999206\), CODATA 2022/2025-Update), aber ihre numerischen Beziehungen bleiben unerklärt. Kosmologische Beobachtungen bestätigen \(\Omega_\Lambda \approx 0.7\), QCD-Skala \(\Lambda_{\text{QCD}} \approx 300\,\si{MeV}\). Keine vereinheitlichte Theorie leitet alle aus einem Parameter ab.
	
	Die fraktale FFGFT (basierend auf Fundamentale Fraktalgeometrische Feldtheorie (FFGFT, früher T0-Theorie)) bietet eine alternative Sicht: Das Vakuumfeld hat zwei intrinsische Freiheitsgrade – Amplitude \(\rho\) und Phase \(\theta\) – deren Parameter vollständig aus dem einzigen Skalenparameter \(\xi = \frac{4}{3} \times 10^{-4}\) (dimensionless) emergieren.
	
	\textbf{Vorteil der T0-Perspektive:} Alle fundamentalen Konstanten werden parameterfrei abgeleitet, Hierarchieprobleme gelöst und numerische Übereinstimmungen erreicht – ohne Feinabstimmung.
	
	\subsection{Fundamentale Vakuumparameter – Ableitung in T0}
	
	Das Vakuumfeld: \(\Phi = \rho e^{i \theta / \xi}\).
	
	1. **Vakuum-Amplitude-Stiffness \(K_0\)**  
	Aus fraktaler Dimensionsanalyse:
	\begin{equation}
		K_0 = \rho_0 \cdot \xi^{-3},
	\end{equation}
	wobei gilt:
	\begin{itemize}
		\item \(K_0\): Steifigkeit der Amplitude (in passenden Einheiten),
		\item \(\rho_0\): Referenz-Amplitude (in \si{kg/m^3} oder äquivalent),
		\item \(\xi\): Skalenparameter (dimensionless).
	\end{itemize}
	
	Referenzdichte:
	\begin{equation}
		\rho_0 = \frac{\hbar c}{l_P^4} \cdot \xi^3,
	\end{equation}
	mit \(l_P\): Planck length (\(\approx 1.616 \times 10^{-35}\,\si{m}\)).
	
	Validierung: Ergibt korrekte Gravitationsskala.
	
	2. **Vakuum-Phasen-Stiffness \(B\)**  
	\begin{equation}
		B = \rho_0^2 \cdot \xi^{-2},
	\end{equation}
	numerisch:
	\begin{equation}
		\sqrt{B} \approx \Lambda_{\text{QCD}} \approx 300\,\si{MeV}.
	\end{equation}
	
	Validierung: Übereinstimmung mit QCD-Confinement-Skala.
	
	3. **Fundamentale Länge \(l_0\)**  
	\begin{equation}
		l_0 = l_P \cdot \xi^{-1} \approx 1.616 \times 10^{-35} \cdot 7500 \approx 1.21 \times 10^{-31}\,\si{m}.
	\end{equation}
	
	Validierung: Zwischen Planck- und QCD-Skala.
	
	4. **Feinstrukturkonstante \(\alpha\)**  
	Aus Phasen-Stiffness:
	\begin{equation}
		\alpha = \xi^2 \cdot \frac{B}{\rho_0 c^2} \approx \frac{1}{137}.
	\end{equation}
	
	Validierung: Numerisch präzise mit gemessenem Wert.
	
	5. **Gravitational constant \(G\)**  
	\begin{equation}
		G = \frac{\hbar c}{m_P^2} \cdot \xi^4,
	\end{equation}
	mit \(m_P\): Planck-Masse.
	
	Validierung: Ergibt beobachteten Wert \(G \approx 6.67430 \times 10^{-11}\,\si{m^3.kg^{-1}.s^{-2}}\).
	
	6. **Kosmologische Vakuumenergie**  
	\begin{equation}
		\rho_{\text{vac}} = \xi^2 \cdot \rho_{\text{crit}} \approx 0.7 \rho_c,
	\end{equation}
	wobei \(\rho_{\text{crit}} = 3 H_0^2 / (8\pi G)\).
	
	Validierung: Übereinstimmung mit \(\Omega_\Lambda \approx 0.7\).
	
	\subsection{Numerische Konsistenz und Vorhersagen}
	
	Abgeleitete Konstanten (T0-Vorhersagen vs. Beobachtung):
	
	\begin{tabular}{lcc}
		Konstante & T0-Wert & Beobachtung (2025) \\
		\hline
		\(\alpha\) & \(\approx 1/137.036\) & \(1/137.035999206\) \\
		\(G\) & \(\approx 6.674 \times 10^{-11}\) & \(6.67430 \times 10^{-11}\,\si{m^3.kg^{-1}.s^{-2}}\) \\
		\(\Lambda\) & \(\xi^2 \cdot 3 H_0^2 / c^2\) & \(\Omega_\Lambda \approx 0.7\) \\
		\(\Lambda_{\text{QCD}}\) & \(\approx \sqrt{B}\) & \(\approx 300\,\si{MeV}\) \\
	\end{tabular}
	
	Validierung: Hohe numerische Übereinstimmung; Abweichungen testbar mit zukünftiger Präzision.
	
	\subsection{Fraktale Kohärenzlänge}
	
	\begin{equation}
		L_{\text{coh}} = l_0 \cdot \xi^{-2} \approx 10^{28}\,\si{m},
	\end{equation}
	entspricht kosmischer Skala (beobachtbares Universum).
	
	Validierung: Erklärt globale Kohärenz in Kosmologie.
	
	\subsection{Schluss}
	
	Im Mainstream-Modell sind fundamentale Konstanten unabhängig und erfordern Feinabstimmung. Die Fundamentale Fraktalgeometrische Feldtheorie (FFGFT, früher T0-Theorie) bietet eine kohärente Alternative: Alle intrinsischen Vakuumparameter emergieren parameterfrei aus dem einzigen Skalenparameter \(\xi\). Dies vereinheitlicht Elektromagnetismus (\(\alpha\)), Gravitation (\(G\)), QCD-Skala (\(\Lambda_{\text{QCD}}\)) und Dunkle Energie (\(\rho_{\text{vac}}\)) in einer numerischen Struktur – konsistent mit allen Beobachtungen.
	
	Validierung: Präzise numerische Übereinstimmungen; testbar durch verbesserte Messungen von \(\alpha\), \(G\) und \(H_0\).
