Die **Hubble-Spannung** beschreibt die Diskrepanz von etwa \SI{8}{\percent} zwischen der Hubble-Konstante \(H_0\), abgeleitet aus dem frühen Universum (CMB-Daten, Planck: \(\approx \SI{67.4}{\kmpsMpc}\)), und der aus dem lokalen Universum (Cepheiden und Typ-Ia-Supernovae, SH0ES: \(\approx \SI{73}{\kmpsMpc}\)) gemessenen.
	
	Im Standardmodell \(\Lambda\)CDM ist diese Spannung problematisch, da die kosmologische Konstante starr ist und keine zwei unterschiedlichen Werte für \(H_0\) erzeugen kann.
	
	In der fraktalen Fundamental Fractal-Geometric Field Theory (FFGFT) mit T0-Time-Mass-Dualität wird die Spannung natürlich erklärt: Das Vakuumfeld \(\Phi = \rho(x,t) e^{i\theta(x,t)}\) ist dynamisch, und seine Amplitude \(\rho\) reagiert unterschiedlich auf die homogene Struktur des frühen Universums und die fraktale Strukturbildung im späten Universum.
	
	Aus der Time-Mass-Dualität \(T(x,t) \cdot m(x,t) = 1\) folgt, dass lokale Massedichte-Variationen die effektive Zeitstruktur und damit die Vakuumenergiedichte modifizieren. Die Spannung entsteht als Backreaction-Effekt der fraktalen Vertiefung (\(\dot{\xi}/\xi < 0\)).
	
	\subsection{Symbolverzeichnis und Einheiten}
	
	\begin{tcolorbox}[title={\textbf{Important Symbols and their Units}}, colback=blue!5!white, colframe=blue!75!black]
		\begin{tabular}{p{0.3\textwidth}p{0.3\textwidth}p{0.35\textwidth}}
			\textbf{Symbol} & \textbf{Meaning} & \textbf{Unit (SI)} \\
			\hline
			\(\xi\) & Fraktaler Skalenparameter & dimensionless \\
			\(H_0\) & Hubble-Konstante (heute) & \si{\per\second} (\si{\kmpsMpc}) \\
			\(a(t)\) & Skalenfaktor (normalisiert \(a_0=1\)) & dimensionless \\
			\(\Omega_m, \Omega_r, \Omega_\xi\) & Dichte-Parameter (Materie, Strahlung, Vakuum) & dimensionless \\
			\(\rho_m\) & Materiedichte & \si{\kilo\gram\per\meter\cubed} \\
			\(\delta \rho_m / \rho_m\) & Relative Dichtefluktuation & dimensionless \\
			\(\rho_{\text{crit}}\) & Kritische Dichte \(3H_0^2 / 8\pi G\) & \si{\kilo\gram\per\meter\cubed} \\
		\end{tabular}
	\end{tcolorbox}
	
	\textbf{Unit Check (Friedmann-Gleichung):}
	\begin{align*}
		\left[H^2\right] &= \si{\per\second\squared} \\
		\left[H_0^2 \Omega_m a^{-3}\right] &= \si{\per\second\squared} \cdot \text{dimensionless} \cdot \text{dimensionless} = \si{\per\second\squared}
	\end{align*}
	Einheiten konsistent für alle Terme.
	
	\subsection{Modifizierte Friedmann-Gleichung in T0}
	
	Die effektive Friedmann-Gleichung in der fraktalen T0-Geometrie lautet:
	\begin{equation}
		H^2(a) = H_0^2 \left[ \Omega_m a^{-3} + \Omega_r a^{-4} + \Omega_\xi \left(1 + \xi \ln\left(\frac{a}{a_{\text{eq}}}\right) \cdot \left(1 + \xi^{1/2} \frac{\delta \rho_m(a)}{\rho_m(a)}\right) \right) \right]
	\end{equation}
	
	Der fraktale Korrekturterm berücksichtigt die langsame Variation von \(\xi(t)\) und die Backreaction der Strukturbildung.
	
	\textbf{Unit Check:}
	\begin{align*}
		[\xi \ln(a)] &= \text{dimensionless} \cdot \text{dimensionless} = \text{dimensionless}
	\end{align*}
	
	\subsection{Analytische Näherung für späte Zeiten (\(a \approx 1\))}
	
	Im lokalen Universum (\(z \approx 0\), strukturiert) ergibt sich eine höhere effektive Hubble-Rate:
	\begin{equation}
		H_{\text{local}} = H_{\text{CMB}} \left(1 + \xi^{1/2} \cdot \frac{\langle \delta \rho_m \rangle}{\rho_{\text{crit}}} + \xi \cdot \Delta \ln a \right)
	\end{equation}
	
	Mit \(\xi = \frac{4}{3} \times 10^{-4}\), \(\xi^{1/2} \approx 0.0205\), und typischen Dichtekontrasten \(\langle \delta \rho_m / \rho_{\text{crit}} \rangle \approx 3\) (lokale Überdichten in Filamenten/Voids) ergibt sich:
	\begin{equation}
		\frac{\Delta H_0}{H_0} \approx 0.0205 \cdot 3 + \mathcal{O}(\xi) \approx 0.0615 + 0.02 \approx 8\% 
	\end{equation}
	
	Dies reproduziert exakt die beobachtete Spannung zwischen \(H_0^{\text{CMB}} \approx \SI{67.4}{\kmpsMpc}\) (Planck) und \(H_0^{\text{local}} \approx \SI{73}{\kmpsMpc}\) (SH0ES, Stand 2025).
	
	\textbf{Unit Check:}
	\begin{align*}
		\left[\frac{\Delta H_0}{H_0}\right] &= \text{dimensionless}
	\end{align*}
	
	\subsection{Validierung im Grenzfall}
	
	Für \(\xi \to 0\) (keine fraktale Dynamik) reduziert sich die Gleichung exakt auf die Standard-Friedmann-Gleichung von \(\Lambda\)CDM – konsistent mit frühen Universumsdaten (CMB). Die Abweichung wächst mit der Strukturbildung (\(a \to 1\)), was die höhere lokale Messung erklärt.
	
	\subsection{Conclusion}
	
	Die Fundamentale Fraktalgeometrische Feldtheorie (FFGFT, früher T0-Theorie) löst die Hubble-Spannung parameterfrei und mathematisch präzise als direkte Konsequenz der dynamischen fraktalen Vakuumstruktur und der Time-Mass-Dualität. Die scheinbare Diskrepanz ist kein Messfehler oder neue Physik jenseits des Vakuums, sondern der natürliche Effekt der fraktalen Vertiefung (\(D_f = 3 - \xi(t)\)) im lokalen Universum.
	
	Im Gegensatz zu \(\Lambda\)CDM, das eine starre Dunkle Energie annimmt, erzeugt die langsame Variation von \(\xi(t)\) eine effektive Zeitabhängigkeit der Vakuumenergie, die exakt die beobachtete \SI{8}{\percent}-Spannung erklärt – eine weitere Bestätigung des einzigen fundamentalen Parameters \(\xi = \frac{4}{3} \times 10^{-4}\).
