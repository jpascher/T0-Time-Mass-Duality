\documentclass[12pt,a4paper]{article}
\usepackage[utf8]{inputenc}
\usepackage[T1]{fontenc}
\usepackage[english]{babel}
\usepackage{amsmath}
\usepackage{amsfonts}
\usepackage{amssymb}
\usepackage{geometry}
\geometry{a4paper,left=2.5cm,right=2.5cm,top=2.5cm,bottom=2.5cm}
\usepackage{fancyhdr}
\usepackage{enumitem}
\usepackage{tcolorbox}
\usepackage{physics}
\usepackage{hyperref}
\usepackage{siunitx} % For correct units

% Load hyperref as one of the last packages
\hypersetup{
	unicode=true,
	pdfencoding=unicode,
	bookmarksopen=true
}

% Clean PDF bookmarks
\pdfstringdefDisableCommands{%
	\def\Lambda{Lambda}%
	\def\Delta{Delta}%
	\def\approx{approximately}%
	\def\Sigma{Sigma}%
	\def\eta{eta}%
	\def\psi{psi}%
}

\title{Chapter 38: Black Holes and Quantum Singularities – T0 Perspective (As of December 2025)}
\author{}
\date{}

\begin{document}
	
	\maketitle
	
	\section{Chapter 38: Black Holes and Quantum Singularities}
	
	Black holes and singularities are central challenges of theoretical physics. In General Relativity (GR), collapse scenarios lead to singularities with infinite curvature (e.g., Schwarzschild radius \(r=0\)). Quantum field theory (QFT) suffers from point-particle singularities (e.g., self-energy divergences). Both problems signal the need for quantum gravity.
	
	Current Status (December 2025): Observations (Event Horizon Telescope, gravitational waves from LIGO/Virgo/KAGRA) confirm black holes, but no singularities directly accessible. Approaches such as Loop Quantum Gravity (LQG), string theory and asymptotic safety regularize singularities, but remain speculative and experimentally untested. Hawking radiation and information paradox are still debated.
	
	Fractal FFGFT (based on T0-theory) offers an alternative regularization: singularities are avoided through fractal vacuum dynamics and the parameter \(\xi = \frac{4}{3} \times 10^{-4}\) (dimensionless) – without quantization of gravitation.
	
	\textbf{Advantage of the T0 perspective:} Unified, classical regularization of both singularity types through vacuum amplitude \(\rho \geq \rho_0 > 0\); finite and testable.
	
	\subsection{Classical Singularities in Black Holes}
	
	In GR, curvature diverges at \(r \to 0\):
	\begin{equation}
		R \propto \frac{G^2 M^2}{\hbar c r^6},
	\end{equation}
	(correctly dimensioned; scalar curvature).
	
	In T0, the metric is modified by vacuum amplitude \(\rho(r)\). Potential:
	\begin{equation}
		U(\rho) = \Lambda_0 + \frac{\kappa}{2} (\rho - \rho_0)^2 + \frac{\lambda}{4} (\rho - \rho_0)^4,
	\end{equation}
	where:
	\begin{itemize}
		\item \(U(\rho)\): Vacuum potential (in energy density),
		\item \(\rho_0\): Equilibrium amplitude (in \si{kg/m^3}),
		\item \(\kappa, \lambda\): Coefficients (positive for stability).
	\end{itemize}
	
	Equation of motion:
	\begin{equation}
		\Box \rho + \frac{dU}{d\rho} + \xi \cdot \rho \cdot \nabla^2 \mathcal{F}(r) = T^{00},
	\end{equation}
	with \(\mathcal{F}(r)\): Fractal correction.
	
	In collapse, \(\rho\) saturates at:
	\begin{equation}
		\rho_{\max} \approx \rho_0 \cdot \xi^{-3/2}.
	\end{equation}
	
	Maximum curvature finite:
	\begin{equation}
		R_{\max} \approx \frac{c^4}{G \hbar} \cdot \xi^2.
	\end{equation}
	
	Validation: No singularity; consistent with GR outside horizon, modified core radius \(\sim l_P \cdot \xi^{-1}\).
	
	\subsection{Quantum Point Singularities}
	
	In QFT, self-energy of a point particle diverges:
	\begin{equation}
		\Delta E \propto \int^{k_{\max}} k^3 \, dk \propto k_{\max}^4.
	\end{equation}
	
	In T0, each particle has finite extent through fractal deformation:
	\begin{equation}
		\delta \rho(x) = \frac{m c^2}{l_0^3} \cdot \xi \cdot \exp\left(-r^2 / (l_0^2 \xi^2)\right),
	\end{equation}
	where:
	\begin{itemize}
		\item \(\delta \rho\): Amplitude perturbation (in \si{kg/m^3}),
		\item \(m\): Rest mass (in \si{kg}),
		\item \(l_0\): Fundamental length (\(\sim 10^{-31}\,\si{m}\)).
	\end{itemize}
	
	Self-energy finite:
	\begin{equation}
		\Delta E \approx \frac{G m^2}{c^2 l_0 \xi}.
	\end{equation}
	
	Validation: Small and negligible; solves UV divergences without renormalization.
	
	\subsection{Comparison with Other Approaches}
	
	\begin{itemize}
		\item LQG: Discrete spacetime, bounce instead of singularity,
		\item String theory: Minimal string length \(l_s\),
		\item Asymptotic safety: UV fixed point of gravitation,
		\item T0: Fractal cut-off through \(\xi\), purely classical from vacuum dynamics.
	\end{itemize}
	
	T0 is minimal – no new quantum degrees of freedom or dimensions.
	
	Validation: Consistent with observed black holes (shadow, waves); predictions for echo chambers in mergers testable.
	
	\subsection{Conclusion}
	
	While mainstream approaches (LQG, strings) regularize singularities through quantization, T0 offers a coherent alternative: classical and quantum mechanical singularities are uniformly eliminated through saturation of vacuum amplitude \(\rho\) and fractal effects with \(\xi\). Everything remains finite – a natural consequence of the fractal vacuum structure.
	
	Validation: Conceptually consistent with GR and QFT; testable through gravitational wave echoes and future black hole images.
	
\end{document}
