\documentclass[12pt,a4paper]{article}
\usepackage[utf8]{inputenc}
\usepackage[T1]{fontenc}
\usepackage[english]{babel}
\usepackage{amsmath}
\usepackage{amsfonts}
\usepackage{amssymb}
\usepackage{geometry}
\geometry{a4paper,left=2.5cm,right=2.5cm,top=2.5cm,bottom=2.5cm}
\usepackage{fancyhdr}
\usepackage{enumitem}
\usepackage{tcolorbox}
\usepackage{physics}
\usepackage{hyperref}
\usepackage{siunitx} % For correct units

% Load hyperref as one of the last packages
\hypersetup{
	unicode=true,
	pdfencoding=unicode,
	bookmarksopen=true
}

% Clean PDF bookmarks
\pdfstringdefDisableCommands{%
	\def\Lambda{Lambda}%
	\def\Delta{Delta}%
	\def\approx{approximately}%
	\def\Sigma{Sigma}%
	\def\eta{eta}%
	\def\psi{psi}%
}

\title{Chapter 35: Explanation of Quantum Mechanical Phenomena – T0 Perspective (As of December 2025)}
\author{}
\date{}

\begin{document}
	
	\maketitle
	
	\section{Chapter 35: Explanation of Quantum Mechanical Phenomena}
	
	Quantum mechanics (QM) describes the behavior of matter and light on atomic and subatomic scales. It is one of the most successful theories in physics, empirically extremely well confirmed, but its interpretation remains controversial: from the Copenhagen interpretation via Many-Worlds to objective collapse models. Decoherence plays a central role in the transition from quantum to classical and is experimentally well studied (e.g., in nanosystems and quantum computers).
	
	Current Status (December 2025): The measurement problem and the interpretation of the wave function remain open. Decoherence explains the apparent collapse through environmental interaction, without fully solving the measurement problem. Phenomena such as entanglement and delayed-choice experiments are confirmed, but interpreted without retrocausality. Bell tests (e.g., with 73-qubit systems) confirm the violation of local realism assumptions, implying non-locality, and demand philosophical reflections (e.g., on EPR paradox and realism).
	
	Fractal FFGFT (based on T0-theory) offers an alternative, unified explanation: quantum phenomena emerge as dynamics of the fractal vacuum field \(\Phi = \rho e^{i\theta / \xi}\), with the scale parameter \(\xi = \frac{4}{3} \times 10^{-4}\) (dimensionless).
	
	\textbf{Advantage of the T0 explanation:} It interprets QM as real vacuum dynamics, makes postulates like wave function collapse unnecessary and unifies it with gravitation – consistent with all data, parameter-free from \(\xi\).
	
	\subsection{Wave Function Collapse and Decoherence}
	
	In mainstream QM, collapse is a postulate; decoherence explains the apparent collapse through phase loss via environment.
	
	In T0: Decoherence as phase scrambling through macroscopic coupling:
	\begin{equation}
		\Gamma_{\text{decoh}} = \xi^2 \cdot \frac{\Delta E}{\hbar},
	\end{equation}
	where:
	\begin{itemize}
		\item \(\Gamma_{\text{decoh}}\): Decoherence rate (in s$^{-1}$),
		\item \(\Delta E\): Energy difference (in J),
		\item \(\hbar\): Reduced Planck constant (in J\,s),
		\item \(\xi\): Fractal parameter (dimensionless).
	\end{itemize}
	
	Mixed state:
	\begin{equation}
		\rho_{\text{mixed}} = \sum_i p_i |\theta_i\rangle\langle\theta_i|.
	\end{equation}
	
	Collapse physically: Local amplitude perturbation \(\delta \rho\).
	
	Validation: Numerical agreement with observed decoherence times; limit \(\xi \to 0\) classical.
	
	\subsection{Wave-Particle Duality}
	
	Waves: Coherent phase patterns \(\theta(kx - \omega t)\).  
	Particles: Localized \(\delta \rho(x)\).
	
	Duality: Aspects of the same field \(\Phi = \rho e^{i\theta}\).
	
	Validation: Consistent with double-slit experiments.
	
	\subsection{Entanglement and Bell Tests}
	
	Entanglement is a global phase correlation in the vacuum field:
	\begin{equation}
		\theta_{\text{total}} = \theta_1 + \theta_2 = \text{constant},
	\end{equation}
	where:
	\begin{itemize}
		\item \(\theta_{\text{total}}\): Total phase (dimensionless),
		\item \(\theta_1, \theta_2\): Phases of entangled systems (dimensionless).
	\end{itemize}
	
	This correlation arises through fractal non-locality of the vacuum substrate and is \textbf{global}, but \textbf{not instantaneously-causal}: There is no signal-transmitting effect across space. The correlation only becomes visible when classically comparing measurement results (subluminally). No violation of relativity theory, as no information is transmitted (no-signaling theorem).
	
	Bell correlations:
	\begin{equation}
		\langle A B \rangle \approx \cos(\Delta \theta_{12}),
	\end{equation}
	(numerically adjusted by \(\xi\)).
	
	Validation: Agreement with Bell tests; no signal transmission.
	
	\subsubsection{Extension to Bell Tests in T0}
	
	Bell's theorem shows that local realistic theories cannot reproduce quantum predictions (CHSH inequality \(\leq 2\), QM up to \(2\sqrt{2} \approx 2.828\)). In T0, entanglement is modified through subtle time field damping, without instantaneity:
	
	\begin{equation}
		E^{T0}(\Delta \theta) = -\cos(\Delta \theta) \cdot (1 - \xi \cdot f(n,l,j)),
	\end{equation}
	where:
	\begin{itemize}
		\item \(E^{T0}\): Correlation function (dimensionless),
		\item \(\Delta \theta = |a-b|\): Angle difference (in radians),
		\item \(f(n,l,j)\): Function of quantum numbers (dimensionless, \(\approx 1\) for photons).
	\end{itemize}
	
	This marginally reduces CHSH to \(\approx 2.827\), preserving locality at \(\xi\)-scale. Fractal extension (non-instantaneous damping):
	\begin{equation}
		E^{T0}_{\text{frac}}(\Delta \theta) = -\cos(\Delta \theta) \cdot \exp\left(-\xi \cdot \frac{|\Delta \theta|^2}{\pi^2} \cdot D_f^{-1}\right),
	\end{equation}
	with \(D_f = 3 - \xi\): Fractal dimension (dimensionless).
	
	Multi-qubit extension:
	\begin{equation}
		E_{n}^{T0}(\Delta \theta) = -\cos(\Delta \theta) \cdot \left(1 - \frac{\xi \cdot n}{\pi} \cdot \sin^2\left(\frac{2|\Delta \theta|}{n}\right)\right).
	\end{equation}
	
	Nonlinear effects at large angles (\(|\Delta \theta| > \pi/4\)) yield \(\Delta E > 10^{-3}\), testable in 73-qubit systems. The damping underscores: correlations are global-fractal, but temporally distributed through \(\xi\)-effects – \textbf{no instantaneous action}.
	
	Validation: Numerical simulations show divergence at high angles, reduced by T0 damping to <0.1\%; consistent with 2025 experiments (e.g., loophole-free tests).
	
	\subsubsection{Philosophical Tensions and Resolution in T0}
	
	The apparent instantaneity in entanglement (EPR paradox) leads to tensions between non-locality and relativity. In T0, entanglement is a \textbf{global, but non-instantaneous correlation}: The vacuum field is fractally connected, effects propagate with finite scale (\(\xi\)-modified), without causal signal transmission. Realism is restored at vacuum scale, non-locality emerges as geometric effect – EPR solved without "spooky action at a distance".
	
	\subsection{Zero-Point Energy and Vacuum Fluctuations}
	
	Mainstream: Zero-point energy leads to divergent vacuum energy problem (cosmological constant).
	
	In T0: Finite through fractal cut-off:
	\begin{equation}
		E_0 \approx \frac{1}{2} \hbar \omega \cdot \frac{\xi}{1-\xi}.
	\end{equation}
	
	Fluctuations:
	\begin{equation}
		\Delta \theta \cdot \Delta E \geq \xi \hbar / 2.
	\end{equation}
	
	Validation: Numerically finite; mitigates cosmological constant problem.
	
	\subsection{Delayed-Choice and Quantum Eraser Experiments}
	
	Interference dependent on global coherence:
	\begin{equation}
		\Delta \phi = \theta_{\text{path1}} - \theta_{\text{path2}}.
	\end{equation}
	
	Which-path marking: \(\Delta \theta = \pi\).  
	Erasure: Erases marking.
	
	No retrocausality – subensemble selection.
	
	Validation: Consistent with experiments; delayed choice only classifies data.
	
	\subsection{Decoherence Rate}
	
	\begin{equation}
		\Gamma = \xi^2 \cdot N \cdot \frac{k_B T}{\hbar}.
	\end{equation}
	where \(N\): Degrees of freedom, \(T\): Temperature (in K).
	
	Macroscopically rapid.
	
	\subsection{Quantum Randomness}
	
	From fractal fluctuations \(\Delta \theta\); inherent, but deterministic on vacuum scale.
	
	\subsection{Atomic Quantization}
	
	From circulation condition:
	\begin{equation}
		\oint \nabla \theta \cdot dl = 2\pi n \cdot \xi^{-1/2}.
	\end{equation}
	
	Stable modes.
	
	\subsection{Further Phenomena}
	
	Tunneling: Phase propagation under barriers.  
	Interference: Phase overlap.  
	Entanglement swapping: Phase reassignment.
	
	\subsection{Conclusion}
	
	While interpretations of QM (decoherence, Many-Worlds, etc.) do not fully solve the measurement problem and vacuum energy, T0 offers a coherent alternative: All phenomena as dynamics of the fractal vacuum field with \(\xi\). Wave function real as \(\theta\), collapse as scrambling, entanglement global and non-instantaneous – parameter-free and unified with gravitation.
	
	Validation: Numerically and conceptually consistent with experiments; testable in extreme regimes.
	
\end{document}
