\documentclass[12pt,a4paper]{article}
% ==============================================================================
% T0 Theory: Shared GERMAN Preamble – Optimized for eBook/Book
% Version: 2.0 – Final 2026 (LuaLaTeX only) – DEUTSCH korrigiert
% Author: Johann Pascher
% Date: Januar 2026
% ==============================================================================
%
% WICHTIG: Compile EXCLUSIVELY with LuaLaTeX!
% In TeXstudio: Options → Configure TeXstudio → Build → Default Compiler → LuaLaTeX
%
% Required Fonts (install once):
% - Inter: https://fonts.google.com/specimen/Inter
% - JetBrains Mono: https://www.jetbrains.com/lp/mono/
% - Libertinus Math: https://github.com/libertinus-fonts/libertinus
% ==============================================================================

% === KAPITEL 1: GRUNDLEGENDE PAKETE (müssen ZUERST kommen) ===
\RequirePackage{fontspec}
\RequirePackage{unicode-math}

% === KAPITEL 2: SPRACHE (DEUTSCH mit voller Silbentrennung) ===
\usepackage[ngerman]{babel}
\usepackage{microtype}                    % WICHTIG für bessere Silbentrennung!

% Typographie-Einstellungen für besseren deutschen Umbruch
\frenchspacing                     % Korrekte deutsche Abstände nach Satzzeichen
\emergencystretch=3em              % Erlaubt mehr Dehnung bei schwierigen Zeilen
\tolerance=2500                    % Höhere Toleranz für Zeilenumbrüche
\hbadness=10000                    % Unterdrückt "underfull hbox" Warnungen
\hfuzz=2pt                         % Erlaubt minimalen Overfull
\pretolerance=150                  % Bessere Worttrennung

% Bessere Seitenumbrüche verhindern
\clubpenalty=10000           % Keine "Schusterjungen"
\widowpenalty=10000          % Keine "Hurenkinder"  
\displaywidowpenalty=10000   % Auch bei Formeln
\brokenpenalty=10000         % Keine getrennten Wörter über Seiten

% Explizite Trennungen für lange deutsche Wörter
\hyphenation{Fun-da-men-tal Frak-tal-Ge-o-me-trisch Fel-the-o-rie Me-tho-do-lo-gisch}
\hyphenation{Re-vi-si-o-nis-mus Quan-ti-sie-rung U-ni-fi-ka-ti-on Ef-fek-tiv}
\hyphenation{Re-nor-mier-bar-keit Sin-gu-la-ri-tä-ten Kon-zi-li-an-tis-mus}
\hyphenation{E-mer-genz Phä-no-me-no-lo-gisch Do-ku-men-ta-ti-on Ana-ly-se}
\hyphenation{Gra-vi-ta-ti-on Quan-ten-me-cha-nik Do-gma-tis-mus Kon-se-quent}
\hyphenation{Par-al-le-lis-mus Im-ple-men-tie-rung Per-tur-ba-ti-o-nen}
\hyphenation{Ge-o-me-trisch Ar-te-fakt In-ko-mpa-ti-bi-li-tät Kon-struk-tiv}
\hyphenation{Frak-tal Di-men-si-ons-los Un-ter-such-ung Be-schrei-bung}
\hyphenation{In-ter-pre-ta-ti-on Phe-no-me-no-lo-gisch Ma-the-ma-tisch}
\hyphenation{Phi-lo-so-phisch Le-gi-ti-ma-ti-on An-wen-dung Ab-lei-tung}
\hyphenation{Ver-ein-heit-li-chung An-na-hme Vor-stel-lung Er-war-tung}
\hyphenation{Sym-me-trie-ern-wei-te-rung Ge-samt-bild Her-aus-fo-rde-rung}
\hyphenation{Wech-sel-wir-kung Ma-te-ri-al An-satz Per-spek-ti-ve Vor-ge-hen}

% === KAPITEL 3: SCHRIFTEN (mit deutschen Ligaturen) ===
\setmainfont{Inter}[
Scale=1.02,
UprightFont=*-Regular,
BoldFont=*-Bold,
ItalicFont=*-Italic,
BoldItalicFont=*-BoldItalic,
Ligatures=TeX,           % WICHTIG für deutsche Typografie
Language=German          % Explizite Sprachunterstützung
]
\setsansfont{Inter}[
Scale=MatchLowercase,
Ligatures=TeX,
Language=German
]
\setmonofont{JetBrains Mono}[
Scale=0.95,
Language=German
]

% Math Font (simple & stable) – MUSS NACH der Sprachdefinition kommen
% WICHTIG: Libertinus Math für korrekte \underbrace-Darstellung!
\setmathfont{Libertinus Math}[Scale=1.0]

% === KAPITEL 4: MATHEMATIK-PAKETE (in STRENGER Reihenfolge!) ===
% WICHTIG: mathtools muss VOR unicode-math für manche Befehle!
\usepackage{mathtools}           % ZUERST mathtools!

% Dann der Rest
\usepackage{amsmath, amsfonts, amsthm}

% SIUNITX MUSS VOR physics geladen werden!
\usepackage{siunitx}
\sisetup{
	locale=DE,                    % DEUTSCHE Einstellungen für SI-Einheiten!
	group-separator={.},          % Tausendertrennzeichen Punkt
	output-decimal-marker={,},    % Dezimaltrennzeichen Komma
	per-mode=symbol,
	separate-uncertainty=true
}

% Eigene SI-Einheiten für Narrative/Bücher
\DeclareSIUnit\gigalightyear{Gly}
\DeclareSIUnit\mev{MeV}

% physics – MUSS NACH siunitx und mathtools geladen werden
\usepackage{physics}

% === KAPITEL 5: ERGÄNZUNGEN aus pdflatex-Best Practices ===
\usepackage{colortbl}        % Farbige Tabellen (ESSENTIELL!)
\usepackage{placeins}        % Float-Kontrolle: \FloatBarrier
\usepackage{subcaption}      % Unterabbildungen
\usepackage{xurl}            % Bessere URL-Umbrüche
% Hyphenation for URLs in bibliography
\def\UrlBreaks{\do\/\do-}

% === KAPITEL 6: SEITENGESTALTUNG =
\usepackage[paperwidth=8.25in, paperheight=11in, 
left=2.5cm, 
right=2.5cm, 
top=2.5cm, 
bottom=3.5cm,
bindingoffset=0.5cm]{geometry}
\setlength{\headheight}{15pt}
% Page Geometry – Buch-Optimierung
% =============================================================================
%\usepackage[paperwidth=8.25in, paperheight=11in,
%top=1.0in,
%bottom=1.2in,
%inner=1.0in,
%outer=0.75in,
%bindingoffset=0.75in,
%twoside]{geometry}
%\setlength{\headheight}{15pt}

% === KAPITEL 7: GRAFIKEN UND TABELLEN ===
\usepackage{graphicx}
\usepackage[table,xcdraw]{xcolor}
% T0 Markenfarben
\definecolor{gold}{RGB}{255,215,0}
\definecolor{blue}{rgb}{0,0,1}
\definecolor{boxgray}{RGB}{240,240,240}
\definecolor{deepblue}{RGB}{0,0,127}
\definecolor{deepgreen}{RGB}{0,127,0}
\definecolor{deepred}{RGB}{191,0,0}
\definecolor{t0blue}{RGB}{33,150,243}
\definecolor{t0green}{RGB}{76,175,80}
\definecolor{t0orange}{RGB}{255,152,0}
\definecolor{t0purple}{RGB}{156,39,176}
\definecolor{t0red}{RGB}{244,67,54}
\definecolor{t0yellow}{RGB}{255,204,0}
\usepackage{tikz}
\usetikzlibrary{arrows.meta,positioning,shapes.geometric,decorations.pathmorphing,patterns,shapes.arrows,intersections}
\usepackage{pgfplots}
\pgfplotsset{compat=1.18}
\usepackage{quantikz}
\usepackage[most]{tcolorbox}
\tcbuselibrary{breakable}

% === WICHTIG: Algorithm-Konflikt umgehen ===
% Option: algorithmic mit GROSSBUCHSTABEN
% Gemeinsame Box für Experimente
\newtcolorbox{experimentbox}[1][]{
	colback=green!5!white,
	colframe=t0green!80!black,
	fonttitle=\bfseries,
	title={{#1}},
	breakable
}

% Abstract-Fallback
\ifdefined\abstract\else
\newenvironment{abstract}{\section*{\abstractname}\itshape\small\par\bigskip}{\bigskip}
\fi

% === MAKROS SICHER NEU DEFINIEREN / ÜBERSCHREIBEN ===
% Definiere Makros OHNE doppelte Subskripte
\newcommand{\phipar}{\phi_{\mathrm{par}}}
%\newcommand{\xipar}{\xi_{\mathrm{par}}}
\newcommand{\Qphipar}{Q_{\phi_{\mathrm{par}}}}
\newcommand{\rphipar}{r_{\phi_{\mathrm{par}}}}
\newcommand{\logphipar}{\log_{\phi_{\mathrm{par}}}}
\newcommand{\CHSH}{\text{CHSH}}
\usepackage{booktabs}
\usepackage{array}
\usepackage{longtable}
\usepackage{float}
\usepackage{adjustbox}
\usepackage{rotating}
\usepackage{tabularx}
\usepackage{makecell}
\usepackage{multirow}

% === KAPITEL 8: DOKUMENTFORMATIERUNG ===
\usepackage{fancyhdr}
\renewcommand{\headrulewidth}{0.4pt}
\renewcommand{\footrulewidth}{0.4pt}
\usepackage{tocloft}

\usepackage{enumitem}
\setlist[itemize]{leftmargin=*, topsep=2pt, partopsep=0pt, parsep=2pt, itemsep=2pt}
\setlist[enumerate]{leftmargin=*, topsep=2pt, partopsep=0pt, parsep=2pt, itemsep=2pt}
\usepackage{setspace}
\usepackage{ragged2e}
\usepackage{multicol}

% === KAPITEL 9: CODE UND ALGORITHMEN ===
\usepackage{algorithm}
\usepackage{algorithmic}
\usepackage{listings}
\lstset{
	basicstyle=\ttfamily\footnotesize,
	breaklines=true,
	breakatwhitespace=true,
	columns=flexible,
	keepspaces=true,
	showstringspaces=false,
	frame=single,
	xleftmargin=0pt,
	xrightmargin=0pt,
	literate=              % Für deutsche Umlaute in Code-Listings
	{ä}{{\"a}}1 {ö}{{\"o}}1 {ü}{{\"u}}1 {ß}{{\ss}}1
	{Ä}{{\"A}}1 {Ö}{{\"O}}1 {Ü}{{\"U}}1
}
\usepackage{mdframed}

% === KAPITEL 10: ZUSÄTZLICHE PAKETE ===
\usepackage{pdflscape}
\usepackage{braket}
\usepackage{cancel}
\usepackage{caption}
\captionsetup{format=plain, labelfont=bf, justification=centering}
\usepackage{csquotes}
\usepackage{gensymb}
\usepackage{textcomp}
\usepackage{textgreek}
\usepackage{upgreek}
\usepackage{url}
\usepackage{slashed}
\usepackage{bm}

% === KAPITEL 11: HYPERREF (muss als VORLETZTES Paket kommen!) ===
\usepackage{hyperref}
\hypersetup{
	colorlinks=true,
	linkcolor=black,
	citecolor=black,
	urlcolor=black,
	breaklinks=true,           % WICHTIG für deutsche Umlaute in URLs!
	bookmarksnumbered=true,
	unicode=true,
	pdfencoding=auto,
	pdflang=de,                % PDF-Sprache auf Deutsch setzen
	pdfsubject={T0 Theorie - Fundamental Fractal-Geometric Field Theory}
}

% === KAPITEL 12: BOOKMARK (muss NACH hyperref kommen!) ===
\usepackage{bookmark}
% Fix for unicode-math symbols in PDF bookmarks
\pdfstringdefDisableCommands{%
	\def\xi{xi}%
	\def\alpha{alpha}%
	\def\beta{beta}%
	\def\gamma{gamma}%
	\def\delta{delta}%
	\def\Delta{Delta}%
	\def\epsilon{epsilon}%
	\def\varepsilon{epsilon}%
	\def\theta{theta}%
	\def\kappa{kappa}%
	\def\lambda{lambda}%
	\def\mu{mu}%
	\def\nu{nu}%
	\def\pi{pi}%
	\def\rho{rho}%
	\def\sigma{sigma}%
	\def\tau{tau}%
	\def\phi{phi}%
	\def\chi{chi}%
	\def\psi{psi}%
	\def\omega{omega}%
	\def\Omega{Omega}%
	\def\Lambda{Lambda}%
	\def\times{x}%
	\def\cdot{*}%
	\def\pm{+/-}%
	\def\approx{~}%
	\def\sim{~}%
	\def\equiv{=}%
	\def\ell{l}%
	\def\hbar{h}%
	\def\rightarrow{->}%
	\def\leftarrow{<-}%
	\def\Rightarrow{=>}%
	\def\Leftarrow{<=}%
	\def\propto{~}%
	\def\mitxi{xi}%
	\def\mitalpha{alpha}%
	\def\mitbeta{beta}%
	\def\mitgamma{gamma}%
	\def\mitdelta{delta}%
	\def\mitDelta{Delta}%
	\def\mitepsilon{epsilon}%
	\def\mitvarepsilon{epsilon}%
	\def\mittheta{theta}%
	\def\mitkappa{kappa}%
	\def\mitlambda{lambda}%
	\def\mitLambda{Lambda}%
	\def\mitmu{mu}%
	\def\mitnu{nu}%
	\def\mitpi{pi}%
	\def\mitrho{rho}%
	\def\mitsigma{sigma}%
	\def\mittau{tau}%
	\def\mitphi{phi}%
	\def\mitchi{chi}%
	\def\mitpsi{psi}%
	\def\mitomega{omega}%
	\def\mitOmega{Omega}%
}

% === KAPITEL 13: CLEVEREF (DEUTSCHE LABELS) ===
\usepackage[ngerman]{cleveref}
\crefname{equation}{Gleichung}{Gleichungen}
\crefname{figure}{Abbildung}{Abbildungen}
\crefname{table}{Tabelle}{Tabellen}
\crefname{section}{Abschnitt}{Abschnitte}
\crefname{chapter}{Kapitel}{Kapitel}
\crefname{theorem}{Satz}{Sätze}
\crefname{lemma}{Lemma}{Lemmata}
\crefname{definition}{Definition}{Definitionen}
\crefname{example}{Beispiel}{Beispiele}
\crefname{remark}{Bemerkung}{Bemerkungen}

% ==============================================================================
\newenvironment{alternative}{%
	\begin{mdframed}[linecolor=black!30,linewidth=1pt,roundcorner=4pt,backgroundcolor=black!5]%
	}{%
	\end{mdframed}%
}

% Photon/particle environment
\newenvironment{photon}{%
	\begin{mdframed}[linecolor=blue!30,linewidth=1pt,roundcorner=4pt,backgroundcolor=blue!5]%
	}{%
	\end{mdframed}%
}

% Koide formula box environment
\newenvironment{koidebox}{%
	\begin{mdframed}[linecolor=green!30,linewidth=1pt,roundcorner=4pt,backgroundcolor=green!5]%
	}{%
	\end{mdframed}%
}

% Erkenntnis/insight environment
\newenvironment{erkenntnis}{%
	\begin{mdframed}[linecolor=orange!30,linewidth=1pt,roundcorner=4pt,backgroundcolor=orange!5]%
	}{%
	\end{mdframed}%
}

% Beziehung/relationship environment
\newenvironment{beziehung}{%
	\begin{mdframed}[linecolor=purple!30,linewidth=1pt,roundcorner=4pt,backgroundcolor=purple!5]%
	}{%
	\end{mdframed}%
}

% Derivation environment
\newenvironment{derivation}{%
	\begin{mdframed}[linecolor=teal!30,linewidth=1pt,roundcorner=4pt,backgroundcolor=teal!5]%
	}{%
	\end{mdframed}%
}

% Abhandlung/treatise environment
\newenvironment{abhandlung}{%
	\begin{mdframed}[linecolor=brown!30,linewidth=1pt,roundcorner=4pt,backgroundcolor=brown!5]%
	}{%
	\end{mdframed}%
}

% Anwendung/application environment
\newenvironment{anwendung}{%
	\begin{mdframed}[linecolor=cyan!30,linewidth=1pt,roundcorner=4pt,backgroundcolor=cyan!5]%
	}{%
	\end{mdframed}%
}

% Additional common environments
\newenvironment{konsequenz}{%
	\begin{mdframed}[linecolor=red!30,linewidth=1pt,roundcorner=4pt,backgroundcolor=red!5]%
	}{%
	\end{mdframed}%
}

\newenvironment{schlussfolgerung}{%
	\begin{mdframed}[linecolor=gray!30,linewidth=1pt,roundcorner=4pt,backgroundcolor=gray!5]%
	}{%
	\end{mdframed}%
}

\newenvironment{result}{%
	\begin{mdframed}[linecolor=violet!30,linewidth=1pt,roundcorner=4pt,backgroundcolor=violet!5]%
	}{%
	\end{mdframed}%
}

% Formula environment
\newenvironment{formula}{%
	\begin{mdframed}[linecolor=yellow!30,linewidth=1pt,roundcorner=4pt,backgroundcolor=yellow!5]%
	}{%
	\end{mdframed}%
}

% Revolutionaer/revolutionary environment
\newenvironment{revolutionaer}{%
	\begin{mdframed}[linecolor=red!50,linewidth=2pt,roundcorner=4pt,backgroundcolor=red!10]%
	}{%
	\end{mdframed}%
}

% Formel environment (German version of formula)
\newenvironment{formel}{%
	\begin{mdframed}[linecolor=yellow!30,linewidth=1pt,roundcorner=4pt,backgroundcolor=yellow!5]%
	}{%
	\end{mdframed}%
}

% Prinzip/principle environment
\newenvironment{prinzip}{%
	\begin{mdframed}[linecolor=blue!50,linewidth=2pt,roundcorner=4pt,backgroundcolor=blue!10]%
	}{%
	\end{mdframed}%
}

% Experimentell/experimental environment
\newenvironment{experimentell}{%
	\begin{mdframed}[linecolor=magenta!30,linewidth=1pt,roundcorner=4pt,backgroundcolor=magenta!5]%
	}{%
	\end{mdframed}%
}

% Neutrino environment
\newenvironment{neutrino}{%
	\begin{mdframed}[linecolor=cyan!40,linewidth=1pt,roundcorner=4pt,backgroundcolor=cyan!8]%
	}{%
	\end{mdframed}%
}

% Additional missing environments
\newenvironment{schluessel}{%
	\begin{mdframed}[linecolor=yellow!50,linewidth=1pt,roundcorner=4pt,backgroundcolor=yellow!10]%
	}{%
	\end{mdframed}%
}

\newenvironment{summary}{%
	\begin{mdframed}[linecolor=gray!40,linewidth=1pt,roundcorner=4pt,backgroundcolor=gray!8]%
	}{%
	\end{mdframed}%
}

\newenvironment{category}{%
	\begin{mdframed}[linecolor=pink!40,linewidth=1pt,roundcorner=4pt,backgroundcolor=pink!8]%
	}{%
	\end{mdframed}%
}

\newenvironment{sibox}{%
	\begin{mdframed}[linecolor=lime!40,linewidth=1pt,roundcorner=4pt,backgroundcolor=lime!8]%
	}{%
	\end{mdframed}%
}

% More missing environments
\newenvironment{documentbox}{%
	\begin{mdframed}[linecolor=teal!40,linewidth=1pt,roundcorner=4pt,backgroundcolor=teal!8]%
	}{%
	\end{mdframed}%
}

\newenvironment{t0box}{%
	\begin{mdframed}[linecolor=violet!40,linewidth=1pt,roundcorner=4pt,backgroundcolor=violet!8]%
	}{%
	\end{mdframed}%
}

\newenvironment{wichtig}{%
	\begin{mdframed}[linecolor=red!50,linewidth=2pt,roundcorner=4pt,backgroundcolor=red!10]%
	\textbf{Wichtig:} 
	}{%
	\end{mdframed}%
}

\newenvironment{smbox}{%
	\begin{mdframed}[linecolor=orange!40,linewidth=1pt,roundcorner=4pt,backgroundcolor=orange!8]%
	}{%
	\end{mdframed}%
}

\newenvironment{pvbox}{%
	\begin{mdframed}[linecolor=purple!40,linewidth=1pt,roundcorner=4pt,backgroundcolor=purple!8]%
	}{%
	\end{mdframed}%
}

\newenvironment{numerisch}{%
	\begin{mdframed}[linecolor=blue!40,linewidth=1pt,roundcorner=4pt,backgroundcolor=blue!8]%
	}{%
	\end{mdframed}%
}

% More missing environments
\newenvironment{relation}{%
	\begin{mdframed}[linecolor=green!40,linewidth=1pt,roundcorner=4pt,backgroundcolor=green!8]%
	}{%
	\end{mdframed}%
}

\newenvironment{beweis}{%
	\begin{mdframed}[linecolor=brown!40,linewidth=1pt,roundcorner=4pt,backgroundcolor=brown!8]%
	\textbf{Beweis:} 
	}{%
	\end{mdframed}%
}

\newenvironment{revolution}{%
	\begin{mdframed}[linecolor=red!60,linewidth=2pt,roundcorner=4pt,backgroundcolor=red!12]%
	}{%
	\end{mdframed}%
}

\newenvironment{key}{%
	\begin{mdframed}[linecolor=yellow!50,linewidth=1pt,roundcorner=4pt,backgroundcolor=yellow!10]%
	}{%
	\end{mdframed}%
}

\newenvironment{newperspective}{%
	\begin{mdframed}[linecolor=cyan!50,linewidth=1pt,roundcorner=4pt,backgroundcolor=cyan!10]%
	}{%
	\end{mdframed}%
}

\newenvironment{literatur}{%
	\begin{mdframed}[linecolor=gray!50,linewidth=1pt,roundcorner=4pt,backgroundcolor=gray!10]%
	}{%
	\end{mdframed}%
}

\newenvironment{folgerung}{%
	\begin{mdframed}[linecolor=teal!50,linewidth=1pt,roundcorner=4pt,backgroundcolor=teal!10]%
	}{%
	\end{mdframed}%
}

\newenvironment{principle}{%
	\begin{mdframed}[linecolor=blue!60,linewidth=2pt,roundcorner=4pt,backgroundcolor=blue!12]%
	}{%
	\end{mdframed}%
}

% AB HIER: IHRE DEFINITIONEN (angepasst für Deutsch)
% ==============================================================================

\setcounter{tocdepth}{3}

% === ZITATBEFEHLE ===
\providecommand{\citep}[1]{\cite{#1}}
\providecommand{\citet}[1]{\cite{#1}}

% === FARBEN ===
\definecolor{gold}{RGB}{255,215,0}
\definecolor{blue}{rgb}{0,0,1}
\definecolor{boxgray}{RGB}{240,240,240}
\definecolor{deepblue}{RGB}{0,0,127}
\definecolor{deepgreen}{RGB}{0,127,0}
\definecolor{deepred}{RGB}{191,0,0}
\definecolor{t0blue}{RGB}{33,150,243}
\definecolor{t0green}{RGB}{76,175,80}
\definecolor{t0orange}{RGB}{255,152,0}
\definecolor{t0purple}{RGB}{156,39,176}
\definecolor{t0red}{RGB}{244,67,54}
\definecolor{t0yellow}{RGB}{255,204,0}

% === SPALTENTYPEN ===
\newcolumntype{L}[1]{>{\raggedright\arraybackslash}p{#1}}
\newcolumntype{C}[1]{>{\centering\arraybackslash}p{#1}}
\newcolumntype{R}[1]{>{\raggedleft\arraybackslash}p{#1}}

% === HYPERREF-EINSTELLUNGEN (aktualisiert) ===
\hypersetup{
	colorlinks=true,
	linkcolor=t0blue,
	citecolor=t0blue,
	urlcolor=t0blue,
	breaklinks=true,
	bookmarksnumbered=true,
	pdfstartview=FitH,
	pdfencoding=auto,
	pdfdisplaydoctitle=true
}

% === DEUTSCHE THEOREM-UMGEBUNGEN ===
\theoremstyle{plain}
\newtheorem{theorem}{Satz}[section]
\newtheorem{lemma}[theorem]{Lemma}
\newtheorem{proposition}[theorem]{Proposition}
\newtheorem{corollary}[theorem]{Korollar}

\theoremstyle{definition}
\newtheorem{definition}[theorem]{Definition}
\newtheorem{example}[theorem]{Beispiel}
\newtheorem{insight}[theorem]{Erkenntnis}
\newtheorem{discovery}[theorem]{Entdeckung}

\theoremstyle{remark}
\newtheorem{remark}[theorem]{Bemerkung}
\newtheorem{axiom}{Axiom}
%\newtheorem{principle}{Principle}  % Commented out to avoid conflicts with document-specific definitions
\newtheorem{warnung}[theorem]{Warnung}

% === T0-SPEZIFISCHE BEFEHLE ===
% (Hier folgen alle Ihre \newcommand und \providecommand Definitionen)
% Diese bleiben UNVERÄNDERT wie in Ihrer Original-Preamble
% ==============================================================================
% SECTION 14: T0-Specific Commands
% ==============================================================================

% --- Core T0 Fields ---
\newcommand{\Tfield}{T(x,t)}
\providecommand{\Tfieldt}{T(\vec{x},t)}
\newcommand{\Efield}{E(x,t)}
\newcommand{\mfield}{m(x,t)}
\providecommand{\vecx}{\vec{x}}

% --- Lagrangian ---
\newcommand{\Lag}{\mathcal{L}}
\newcommand{\calL}{\mathcal{L}}

% --- Greek Letters and Constants ---
\newcommand{\alphaem}{\alpha}
\newcommand{\betaT}{\beta_T}
\newcommand{\xiT}{\xi}
\newcommand{\xipar}{\xi}

% --- Energy and Planck Units ---
\newcommand{\Ezero}{E_0}
\newcommand{\EPlanck}{E_{\text{Pl}}}
\newcommand{\Mpl}{M_{\text{Pl}}}
\newcommand{\mP}{m_{\text{P}}}
\newcommand{\lP}{\ell_{\text{P}}}
\newcommand{\tP}{t_{\text{P}}}
\newcommand{\LPlanck}{\ell_{\text{Pl}}}
\newcommand{\TPlanck}{t_{\text{Pl}}}

% --- Coupling Constants ---
\newcommand{\Gnat}{G_{\text{nat}}}
\newcommand{\alphaEM}{\alpha_{\text{EM}}}
\newcommand{\alphaSI}{\alpha_{\text{SI}}}
\newcommand{\Hubble}{H_0}
\newcommand{\LCDM}{\Lambda\text{CDM}}
\newcommand{\natunits}{(nat. units)}

% --- T0 Model Parameters ---
\newcommand{\xigeom}{\xi_{\mathrm{geom}}}
\newcommand{\rzero}{r_{0}}
\newcommand{\xirat}{\xi_{\mathrm{rat}}}
\newcommand{\tzero}{t_{0}}
\newcommand{\Lambdat}{\Lambda_{\mathrm{t}}}
\newcommand{\EP}{E_{\text{P}}}
\newcommand{\Emu}{E_{\mu}}
\newcommand{\Ee}{E_{e}}
\newcommand{\Etau}{E_{\tau}}
\newcommand{\alphafine}{\alpha_{\mathrm{fine}}}
\newcommand{\alphal}{\alpha_{\ell}}
\newcommand{\Lzero}{\ell_{0}}
\newcommand{\Lp}{\ell_{\mathrm{P}}}

% --- Additional T0 Commands ---
\newcommand{\Kfrak}{K_{\text{frak}}}
\newcommand{\Dfrak}{D_{\text{frak}}}
\newcommand{\betapar}{\ensuremath{\beta_T}}
\newcommand{\alphapar}{\alpha}
\newcommand{\deltafield}{\delta \phi}
\newcommand{\deltam}{\delta m}
\newcommand{\deltaE}{\delta E}
\newcommand{\Exi}{E_{\xi}}
\newcommand{\Lxi}{\ell_{\xi}}
\newcommand{\rhoCMB}{\rho_{\text{CMB}}}
\newcommand{\rhoCasimir}{\rho_{\text{Casimir}}}
\newcommand{\Leff}{L_{\text{eff}}}
\newcommand{\CQCD}{C_{\mathrm{QCD}}}
\newcommand{\Kspec}{K_{\mathrm{spec}}}
\newcommand{\Tzero}{\ensuremath{T_0}}
\newcommand{\Eabs}{E_{\text{abs}}}
\newcommand{\taupar}{\tau}

% --- Provided Commands ---
\providecommand{\xiconst}{\xi_{\text{const}}}
\providecommand{\DhiggsT}{D_{\text{Higgs-T}}}
\providecommand{\rhoE}{\rho_{E}}
\providecommand{\Echar}{E_{\text{char}}}
\providecommand{\kfrac}{k_{\text{frac}}}
\providecommand{\alphaEMSI}{\alpha_{\text{EM,SI}}}
\providecommand{\alphaEMnat}{\alpha_{\text{EM,nat}}}
\providecommand{\betaTSI}{\beta_{T,\text{SI}}}
\providecommand{\betaTnat}{\beta_{T,\text{nat}}}
\providecommand{\Gsi}{G_{\text{SI}}}
\providecommand{\xiparSI}{\xi_{\text{SI}}}
\providecommand{\xiparnat}{\xi_{\text{nat}}}
\providecommand{\meff}{m_{\text{eff}}}
\providecommand{\Tzerot}{T_{0}(t)}
\providecommand{\mzerot}{m_{0}(t)}
\providecommand{\Ezeroabs}{E_{0,\text{abs}}}
\providecommand{\Epar}{E_{\text{par}}}
\providecommand{\Lnat}{\ell_{\text{nat}}}
\providecommand{\Tnat}{T_{\text{nat}}}
\providecommand{\xifrak}{\xi_{\text{frac}}}
\providecommand{\Tfrak}{T_{\text{frac}}}
\providecommand{\mfrak}{m_{\text{frac}}}
\providecommand{\Dfrac}{D_{\text{frac}}}
\providecommand{\EphotSI}{E_{\gamma,\text{SI}}}
\providecommand{\EphotNat}{E_{\gamma,\text{nat}}}
\providecommand{\Eabsint}{E_{\text{abs,int}}}
\providecommand{\mphoton}{m_{\gamma}}
\providecommand{\Evis}{E_{\text{vis}}}
\providecommand{\Cto}{C_{T0}}
\providecommand{\mytimes}{\times}
\providecommand{\lambdah}{\lambda_h}
\providecommand{\checkmarkx}{\checkmark}
\providecommand{\Enorm}{E_{\text{norm}}}
\providecommand{\Tobs}{T_{\text{obs}}}
\providecommand{\mobs}{m_{\text{obs}}}
\providecommand{\Eobs}{E_{\text{obs}}}
\providecommand{\Lobs}{\ell_{\text{obs}}}
\providecommand{\xobs}{\xi_{\text{obs}}}
\providecommand{\calE}{\mathcal{E}}
\providecommand{\calT}{\mathcal{T}}
\providecommand{\calM}{\mathcal{M}}
\providecommand{\alphag}{\alpha_g}
\providecommand{\Tmax}{T_{\text{max}}}
\providecommand{\mmin}{m_{\text{min}}}
\providecommand{\Lmax}{\ell_{\text{max}}}
\providecommand{\Emin}{E_{\text{min}}}
\providecommand{\Geff}{G_{\text{eff}}}
\providecommand{\rhoeff}{\rho_{\text{eff}}}
\providecommand{\xieff}{\xi_{\text{eff}}}
\providecommand{\Teff}{T_{\text{eff}}}
\providecommand{\hPlanck}{h}
\providecommand{\kB}{k_B}
\providecommand{\muB}{\mu_B}
\providecommand{\lambdaC}{\lambda_C}
\providecommand{\omegaP}{\omega_P}
\providecommand{\rhoP}{\rho_P}
\providecommand{\Tref}{T_{\text{ref}}}
\providecommand{\Eref}{E_{\text{ref}}}
\providecommand{\mref}{m_{\text{ref}}}
\providecommand{\Lref}{\ell_{\text{ref}}}
\providecommand{\xikonst}{\xi_0}
\providecommand{\Phiphoton}{\Phi_{\gamma}}
\providecommand{\etavis}{\eta_{\text{vis}}}
\providecommand{\pichar}{\pi}
\providecommand{\primrel}{\mathcal{P}_{\text{rel}}}
\providecommand{\warningx}{\textcolor{orange}{\textbf{!}}}
\providecommand{\phiT}{\phi_T}
\providecommand{\Lorentz}{\Lambda}
\providecommand{\Cconv}{C_{\text{conv}}}
\providecommand{\Df}{\Delta f}
\providecommand{\lambdazero}{\lambda_0}
\providecommand{\myapprox}{\approx}
\providecommand{\checked}{\checkmark}
\providecommand{\alphaWSI}{\alpha_W^{\text{SI}}}
\providecommand{\alphaWnat}{\alpha_W^{\text{nat}}}
\providecommand{\vect}[1]{\vec{#1}}
\providecommand{\Rzero}{R_0}
\providecommand{\Riem}{\mathcal{R}}
\providecommand{\nuzero}{\nu_0}
\providecommand{\mypi}{\pi}

% =============================================================================
% TCOLORBOX-STILE UND UMGEBUNGEN (deutsche Titel)
% =============================================================================
\tcbset{
	keyresult/.style={
		colback=blue!5!white,
		colframe=blue!75!black,
		title=Schlüsselergebnis,
		fonttitle=\bfseries
	},
	foundation/.style={
		colback=green!5!white,
		colframe=green!75!black,
		title=Grundlage,
		fonttitle=\bfseries
	},
	alternative/.style={
		colback=orange!5!white,
		colframe=orange!75!black,
		title=Alternative,
		fonttitle=\bfseries
	},
	warningbox/.style={
		colback=red!5!white,
		colframe=red!75!black,
		title=Warnung,
		fonttitle=\bfseries
	}
}

% (Hier folgen alle Ihre tcolorbox-Definitionen mit deutschen Titeln)
\newtcolorbox{keyresultbox}[1][]{colback=blue!5!white,colframe=blue!75!black,fonttitle=\bfseries,title={#1},breakable}
\newtcolorbox{keyresult}[1][Schlüsselergebnis]{colback=blue!5!white,colframe=blue!75!black,fonttitle=\bfseries,title={#1},breakable}
\newtcolorbox{foundationbox}[1][]{colback=green!5!white,colframe=green!75!black,fonttitle=\bfseries,title={#1},breakable}
\newtcolorbox{foundation}[1][Grundlage]{colback=green!5!white,colframe=green!75!black,fonttitle=\bfseries,title={#1},breakable}
\newtcolorbox{alternativebox}[1][]{colback=orange!5!white,colframe=orange!75!black,fonttitle=\bfseries,title={#1},breakable}
\newtcolorbox{warningboxenv}[1][Warnung]{colback=red!5!white,colframe=red!75!black,fonttitle=\bfseries,title={#1},breakable}

\newtcolorbox{fundamental}[1][]{
	colback=boxgray,
	colframe=t0blue,
	fonttitle=\bfseries,
	title=#1,
	sharp corners,
	boxrule=2pt
}

\newtcolorbox{insightBox}[1][Erkenntnis]{colback=blue!5,colframe=t0blue,title={#1},fonttitle=\bfseries,breakable}
\newtcolorbox{discoveryBox}[1][Entdeckung]{colback=green!5,colframe=t0green,title={#1},fonttitle=\bfseries,breakable}
\newtcolorbox{revelation}[1][Offenbarung]{colback=red!5,colframe=t0red,title={#1},fonttitle=\bfseries,breakable}
\newtcolorbox{keypoint}[1][Schlüsselpunkt]{colback=blue!5,colframe=t0blue,title={#1},fonttitle=\bfseries,breakable}
\newtcolorbox{evidence}[1][Beleg]{colback=green!5,colframe=t0green,title={#1},fonttitle=\bfseries,breakable}
\newtcolorbox{conclusionBox}[1][Fazit]{colback=gray!5,colframe=gray,title={#1},fonttitle=\bfseries,breakable}
\newtcolorbox{significance}[1][Bedeutung]{colback=yellow!5,colframe=orange,title={#1},fonttitle=\bfseries,breakable}
\newtcolorbox{philosophical}[1][Philosophisch]{colback=purple!5,colframe=purple,title={#1},fonttitle=\bfseries,breakable}
\newtcolorbox{implicationBox}[1][Implikation]{colback=cyan!5,colframe=cyan,title={#1},fonttitle=\bfseries,breakable}
\newtcolorbox{perspectiveBox}[1][Perspektive]{colback=blue!5,colframe=t0blue,title={#1},fonttitle=\bfseries,breakable}
\newtcolorbox{revolutionary}[1][Revolutionär]{colback=red!5,colframe=t0red,title={#1},fonttitle=\bfseries,breakable}

\newtcolorbox{technical}[1][Technisch]{colback=gray!5,colframe=gray!75!black,title={#1},fonttitle=\bfseries,breakable}
\newtcolorbox{technicalBox}[1][Technisch]{colback=gray!5,colframe=gray!75!black,title={#1},fonttitle=\bfseries,breakable}
\newtcolorbox{notationBox}[1][Notation]{colback=yellow!5,colframe=yellow!75!black,title={#1},fonttitle=\bfseries,breakable}
\newtcolorbox{verification}[1][Verifikation]{colback=orange!5!white,colframe=orange!75!black,fonttitle=\bfseries,title=#1}
\newtcolorbox{explanationBox}[1][Erklärung]{colback=purple!5!white,colframe=purple!75!black,fonttitle=\bfseries,title=#1}
\newtcolorbox{interpretationBox}[1][Interpretation]{colback=cyan!5!white,colframe=cyan!75!black,fonttitle=\bfseries,title=#1}
\newtcolorbox{explanation}[1][Erklärung]{colback=purple!5!white,colframe=purple!75!black,fonttitle=\bfseries,title=#1,breakable}
\newtcolorbox{interpretation}[1][Interpretation]{colback=cyan!5!white,colframe=cyan!75!black,fonttitle=\bfseries,title=#1,breakable}
\newtcolorbox{proof_step}[1][Beweisschritt]{colback=gray!5!white,colframe=gray!75!black,fonttitle=\bfseries,title=#1,breakable}
\newtcolorbox{experimental}[1][Experimentell]{colback=teal!5!white,colframe=teal!75!black,fonttitle=\bfseries,title=#1,breakable}

\newtcolorbox{important}[1][Wichtig]{colback=red!5!white,colframe=red!75!black,title={#1},fonttitle=\bfseries,breakable}
\newtcolorbox{warning}[1][Warnung]{colback=orange!5!white,colframe=orange!75!black,title={#1},fonttitle=\bfseries,breakable}
\newtcolorbox{caution}[1][Vorsicht]{colback=yellow!5!white,colframe=yellow!75!black,title={#1},fonttitle=\bfseries,breakable}
\newtcolorbox{vorsicht}[1][Vorsicht]{colback=yellow!5!white,colframe=yellow!75!black,title={#1},fonttitle=\bfseries,breakable}
\newtcolorbox{highlight}[1][Hervorhebung]{colback=yellow!10!white,colframe=yellow!75!black,title={#1},fonttitle=\bfseries,breakable}
\newtcolorbox{critical}[1][Kritisch]{colback=red!10!white,colframe=red!75!black,title={#1},fonttitle=\bfseries,breakable}

\newtcolorbox{analysis}[1][Analyse]{colback=blue!5!white,colframe=blue!75!black,title={#1},fonttitle=\bfseries,breakable}
\newtcolorbox{application}[1][Anwendung]{colback=green!5!white,colframe=green!75!black,title={#1},fonttitle=\bfseries,breakable}
\newtcolorbox{experiment}[1][Experiment]{colback=cyan!5!white,colframe=cyan!75!black,title={#1},fonttitle=\bfseries,breakable}
\newtcolorbox{historical}[1][Historisch]{colback=brown!5!white,colframe=brown!75!black,title={#1},fonttitle=\bfseries,breakable}
\newtcolorbox{numerical}[1][Numerisch]{colback=gray!5!white,colframe=gray!75!black,title={#1},fonttitle=\bfseries,breakable}
\newtcolorbox{overview}[1][Überblick]{colback=blue!5!white,colframe=blue!75!black,title={#1},fonttitle=\bfseries,breakable}
\newtcolorbox{speculation}[1][Spekulation]{colback=purple!5!white,colframe=purple!75!black,title={#1},fonttitle=\bfseries,breakable}
\newtcolorbox{question}[1][Frage]{colback=orange!5!white,colframe=orange!75!black,title={#1},fonttitle=\bfseries,breakable}
\newtcolorbox{method}[1][Methode]{colback=teal!5!white,colframe=teal!75!black,title={#1},fonttitle=\bfseries,breakable}
\newtcolorbox{correct}[1][Korrekt]{colback=green!10!white,colframe=green!75!black,title={#1},fonttitle=\bfseries,breakable}
\newtcolorbox{units}[1][Einheiten]{colback=gray!5!white,colframe=gray!75!black,title={#1},fonttitle=\bfseries,breakable}
\newtcolorbox{achievement}[1][Errungenschaft]{colback=gold!5!white,colframe=orange!75!black,title={#1},fonttitle=\bfseries,breakable}
\newtcolorbox{equivalence}[1][Äquivalenz]{colback=cyan!5!white,colframe=cyan!75!black,title={#1},fonttitle=\bfseries,breakable}
\newtcolorbox{dimensional}[1][Dimensionsanalyse]{colback=purple!5!white,colframe=purple!75!black,title={#1},fonttitle=\bfseries,breakable}

% === ZUSÄTZLICHE EINFACHE UMGEBUNGEN ===
\newenvironment{treatise}{\begin{quote}}{\end{quote}}
\newenvironment{gemeinsam}{\begin{quote}}{\end{quote}}
\newenvironment{vergleich}{\begin{quote}}{\end{quote}}
\newenvironment{vorteil}{\begin{quote}}{\end{quote}}
\newenvironment{quantum}{\begin{quote}}{\end{quote}}

% === LAYOUT-EINSTELLUNGEN ===
\raggedbottom
\usepackage{environ}
\let\oldtabular\tabular
\let\endoldtabular\endtabular

\newenvironment{scaledtable}[1][0.85]{%
	\begingroup\footnotesize\setlength{\LTleft}{0pt}\setlength{\LTright}{0pt}%
}{%
	\endgroup%
}

\newcommand{\widetable}[1]{\resizebox{\textwidth}{!}{#1}}

% === INHALTSVERZEICHNIS-FORMATIERUNG ===
\renewcommand{\cftsecfont}{\color{blue}}
\renewcommand{\cftsubsecfont}{\color{blue}}
\renewcommand{\cftsecpagefont}{\color{blue}}
\renewcommand{\cftsubsecpagefont}{\color{blue}}
\renewcommand{\cfttoctitlefont}{\huge\bfseries\color{blue}}

% === STANDARD-KOPF- UND FUßZEILE ===
\pagestyle{fancy}
\fancyhf{}
\fancyhead[L]{\textsc{T0 Theorie}}
\fancyhead[R]{\textsc{J. Pascher}}
\fancyfoot[C]{\thepage}

% ==============================================================================
% Ende der Shared Preamble für Deutsch
% ==============================================================================

\title{\textbf{Anomale magnetische Momente in der T0-Theorie}\\[0.5cm]
	\large Geometrische Herleitung aus der Zeit-Masse-Dualität\\[0.3cm]
	\normalsize Rein geometrische Formeln und präzise Verhältnis-Vorhersagen}
\author{}
\date{Februar 2026}

\begin{document}
	
	\maketitle
	
	\begin{abstract}
		Die T0-Theorie (Fundamental Fraktale Geometrische Feldtheorie) erklärt anomale magnetische Momente der Leptonen aus rein geometrischen Prinzipien. Leptonen sind Windungsstrukturen im 4D-Torsionsgitter, deren räumliche Ausdehnung das anomale Moment erzeugt. Die Formeln verwenden ausschließlich die geometrischen Grundkonstanten $\varphi$ (goldener Schnitt), $\xi = 4/3 \times 10^{-4}$ (Torsionskonstante) und $f = 7500 - 5\varphi$ (Sub-Planck-Faktor) ohne freie Anpassungsparameter. Absolute Werte weichen ~2\% vom Experiment ab (konsistent mit Massenvorhersagen), aber Verhältnisse wie $\Delta a_\tau/\Delta a_\mu = f^{1/3} - 1 \approx 18{,}57$ sind präzise parameterfrei vorhergesagt. Dies ermöglicht testbare Vorhersagen für Tau-g-2 bei Belle~II analog zur Koide-Formel für Massen.
	\end{abstract}
	
	\begin{tcolorbox}[colback=yellow!10!white, colframe=orange!75!black, title=Hinweis zu älteren Dokumenten]
		Frühere Versionen der g-2 Analyse (\href{https://github.com/jpascher/T0-Time-Mass-Duality/blob/main/2/pdf/018_T0_Anomale-g2-9_En.pdf}{018\_T0\_Anomale-g2-9\_En.pdf}) verwendeten semi-empirische Faktoren. Die vorliegende Formulierung verwendet \textbf{ausschließlich geometrische Faktoren} und ist ehrlich über die ~2\% Abweichung, die mit der Präzision aller T0-Vorhersagen konsistent ist. Python-Skripte verfügbar unter: \href{https://github.com/jpascher/T0-Time-Mass-Duality/blob/main/2/python/}{github.com/jpascher/T0-Time-Mass-Duality}
	\end{tcolorbox}
	
	\textbf{Schlüsselwörter:} Anomales magnetisches Moment, g-2, T0-Theorie, Zeit-Masse-Dualität, Torsionsgitter, Verhältnis-Vorhersagen, Koide-Formel
	
	\tableofcontents
	
	\section{Einleitung: Geometrische vs. semi-empirische Ansätze}
	
	\subsection{Die Philosophie der T0-Theorie}
	
	Die T0-Theorie basiert auf dem Prinzip, dass \textbf{alle} physikalischen Konstanten aus der geometrischen Struktur eines 4-dimensionalen Torsionsgitters folgen sollten. Für die anomalen magnetischen Momente bedeutet dies:
	
	\begin{itemize}
		\item \textbf{KEINE} versteckten Fit-Parameter
		\item \textbf{NUR} geometrische Faktoren: $\varphi$, $\xi$, $f$
		\item Ehrlichkeit über Präzisionsgrenzen
		\item Konsistenz mit anderen Vorhersagen
	\end{itemize}
	
	\subsection{Konsistenz mit Massen-Vorhersagen}
	
	Die T0-Theorie sagt Leptonmassen mit ~1--2\% Abweichung vorher:
	
	\begin{table}[h]
		\centering
		\begin{tabular}{lccc}
			\toprule
			\textbf{Lepton} & \textbf{T0 [MeV]} & \textbf{Exp [MeV]} & \textbf{Abweichung} \\
			\midrule
			Elektron & 0{,}507 & 0{,}511 & 0{,}87\% \\
			Myon & 103{,}5 & 105{,}7 & 2{,}09\% \\
			Tau & 1815 & 1777 & 2{,}16\% \\
			\bottomrule
		\end{tabular}
		\caption{Leptonmassen in T0}
	\end{table}
	
	\textbf{Erwartung:} g-2 sollte ähnliche Präzision haben (~2\%).
	
	Es wäre \textbf{unehrlich}, für g-2 perfekte Übereinstimmung zu behaupten, wenn Massen bereits ~2\% abweichen!
	
	\section{Physikalische Grundlagen}
	
	\subsection{Was ist das anomale magnetische Moment?}
	
	Das magnetische Moment eines geladenen Spin-$1/2$ Teilchens ist:
	\begin{equation}
		\mu = g \cdot \frac{e}{2m} \cdot \frac{\hbar}{2}
	\end{equation}
	
	wobei $g$ der gyromagnetische Faktor (g-Faktor) ist.
	
	\textbf{Dirac-Vorhersage:} Für ein punktförmiges Teilchen: $g = 2$
	
	\textbf{Quanteneffekte:} Vakuumpolarisation, Vertex-Korrekturen $\Rightarrow g \neq 2$
	
	\textbf{Anomalie:} $a = (g-2)/2$
	
	\textbf{QED-Erwartung:} $a \approx \alpha/(2\pi) + \mathcal{O}(\alpha^2) \approx 0{,}00116$
	
	\subsection{T0-Interpretation: Windungen im Torsionsgitter}
	
	In der T0-Theorie sind Leptonen \textbf{Windungsstrukturen} im 4D-Torsionsgitter:
	
	\begin{itemize}
		\item \textbf{Elektron:} Einfache Windung (1. Generation)
		\item \textbf{Myon:} Windung mit fraktaler Verzweigung (2. Generation)
		\item \textbf{Tau:} Komplexere fraktale Struktur (3. Generation)
	\end{itemize}
	
	Das anomale Moment entsteht aus:
	\begin{enumerate}
		\item Der \textbf{Rotation} der Windung (Spin)
		\item Der \textbf{Ladungsverteilung} auf der Windung
		\item Der \textbf{Projektion} 4D $\to$ 3D
	\end{enumerate}
	
	$\Rightarrow$ \textbf{Keine} punktförmige Ladung $\Rightarrow$ $a \neq 0$
	
	\section{Geometrische Formeln}
	
	\subsection{Fundamentale Parameter}
	
	Die T0-Theorie verwendet ausschließlich drei geometrische Grundkonstanten:
	
	\begin{align}
		\varphi &= \frac{1 + \sqrt{5}}{2} = 1{,}618\ldots \quad \text{(Goldener Schnitt)} \\
		\xi &= \frac{4}{3} \times 10^{-4} = 1{,}333 \times 10^{-4} \quad \text{(Torsionskonstante)} \\
		f_{\text{ideal}} &= \frac{30000}{4} = 7500 \quad \text{(Ideales Gitter)} \\
		\Delta &= 5\varphi = 8{,}090 \quad \text{(Pentagonale Symmetriebrechung)} \\
		f &= f_{\text{ideal}} - \Delta = 7491{,}91 \quad \text{(Realer Sub-Planck-Faktor)}
	\end{align}
	
	\subsection{Elektron: Basis-Windung}
	
	\textbf{Formel:}
	\begin{equation}
		a_e = \frac{S_3/f}{k_{\text{geom}}}
		\label{eq:ae}
	\end{equation}
	
	wobei:
	\begin{itemize}
		\item $S_3 = 2\pi^2 = 19{,}739$: 3D-Oberfläche der 4D-Windung
		\item $f = 7491{,}91$: Sub-Planck-Skalierung
		\item $k_{\text{geom}}$: Geometrischer Projektionsfaktor
	\end{itemize}
	
	\textbf{Geometrischer Projektionsfaktor:}
	\begin{equation}
		k_{\text{geom}} = \frac{2}{\sqrt{\varphi}} \times \sqrt{2}
		\label{eq:kgeom}
	\end{equation}
	
	\textbf{Erklärung der Faktoren:}
	\begin{itemize}
		\item $2/\sqrt{\varphi} = 1{,}572$: Pentagonale Projektion (aus $\xi$-Struktur)
		\item $\sqrt{2} = 1{,}414$: Diagonalprojektion 4D $\to$ 3D
		\item $k_{\text{geom}} = 2{,}224$: Vollständig geometrisch!
	\end{itemize}
	
	\textbf{Numerische Berechnung:}
	\begin{align}
		k_{\text{geom}} &= \frac{2}{\sqrt{1{,}618}} \times \sqrt{2} = 2{,}224 \\
		a_e &= \frac{19{,}739 / 7491{,}91}{2{,}224} \\
		a_e &= 1{,}185 \times 10^{-3}
	\end{align}
	
	\textbf{Vergleich:}
	\begin{itemize}
		\item T0: $a_e = 1{,}185 \times 10^{-3}$
		\item Experiment: $a_e = 1{,}160 \times 10^{-3}$
		\item Abweichung: \textbf{2{,}18\%}
	\end{itemize}
	
	\subsection{Myon: Fraktale Zusatzwindung}
	
	\textbf{Formel:}
	\begin{equation}
		a_\mu = a_e + \Delta a_{\text{fraktal}}
		\label{eq:amu}
	\end{equation}
	
	mit
	\begin{equation}
		\Delta a_{\text{fraktal}} = \frac{4\pi}{f^{p_\mu}}
		\label{eq:delta_mu}
	\end{equation}
	
	wobei:
	\begin{itemize}
		\item $p_\mu = 5/3$: Fraktale Hausdorff-Dimension
		\item $4\pi$: Vollständiger Torsionsumlauf
	\end{itemize}
	
	\textbf{Bedeutung von $p_\mu = 5/3$:}
	
	Dies ist die bekannte Hausdorff-Dimension von:
	\begin{itemize}
		\item Brownscher Bewegung in 2D
		\item Selbstvermeidendem Random Walk
		\item Koch-Kurve (Fraktal)
	\end{itemize}
	
	$\Rightarrow$ Physikalisch plausibel für ``teilweise verzweigte Windung''!
	
	\textbf{Numerische Berechnung:}
	\begin{align}
		\Delta a_{\text{fraktal}} &= \frac{4\pi}{7491{,}91^{5/3}} = 4{,}381 \times 10^{-6} \\
		a_\mu &= 1{,}185 \times 10^{-3} + 4{,}381 \times 10^{-6} \\
		a_\mu &= 1{,}189 \times 10^{-3}
	\end{align}
	
	\textbf{Vergleich:}
	\begin{itemize}
		\item T0: $a_\mu = 1{,}189 \times 10^{-3}$
		\item Experiment: $a_\mu = 1{,}166 \times 10^{-3}$
		\item Abweichung: \textbf{2{,}00\%}
	\end{itemize}
	
	\subsection{Tau: Komplexere fraktale Struktur}
	
	\textbf{Formel:}
	\begin{equation}
		a_\tau = a_e + \frac{4\pi}{f^{p_\tau}}
		\label{eq:atau}
	\end{equation}
	
	wobei:
	\begin{itemize}
		\item $p_\tau = 4/3$: Stärkere fraktale Verzweigung
	\end{itemize}
	
	\textbf{Bedeutung von $p_\tau = 4/3$:}
	
	Dies ist die Box-Counting-Dimension vieler Fraktale (z.B. Koch-Kurve, Mandelbrot-Menge).
	
	\textbf{Numerische Berechnung:}
	\begin{align}
		\Delta a_{\text{fraktal}} &= \frac{4\pi}{7491{,}91^{4/3}} = 8{,}572 \times 10^{-5} \\
		a_\tau &= 1{,}185 \times 10^{-3} + 8{,}572 \times 10^{-5} \\
		a_\tau &= 1{,}271 \times 10^{-3}
	\end{align}
	
	\textbf{Status:} Dies ist eine \textbf{Vorhersage} -- Tau-g-2 ist noch nicht gemessen!
	
	\section{Zusammenfassung der Absolutwerte}
	
	\begin{table}[h]
		\centering
		\begin{tabular}{lcccc}
			\toprule
			\textbf{Lepton} & \textbf{T0} & \textbf{Experiment} & \textbf{Abw.} & \textbf{Status} \\
			\midrule
			Elektron & $1{,}185 \times 10^{-3}$ & $1{,}160 \times 10^{-3}$ & 2{,}18\% & ✓ \\
			Myon & $1{,}189 \times 10^{-3}$ & $1{,}166 \times 10^{-3}$ & 2{,}00\% & ✓ \\
			Tau & $1{,}271 \times 10^{-3}$ & (nicht gemessen) & -- & Vorhersage \\
			\bottomrule
		\end{tabular}
		\caption{g-2 Absolutwerte: T0 vs. Experiment}
	\end{table}
	
	\textbf{Bewertung:}
	\begin{itemize}
		\item ✓ Alle Faktoren geometrisch erklärt
		\item ✓ Keine versteckten Fit-Parameter
		\item ✓ ~2\% Abweichung konsistent mit Massen
		\item ✓ Ehrlich über Limitationen
	\end{itemize}
	
	\section{Präzise Verhältnis-Vorhersagen}
	
	\subsection{Analog zur Koide-Formel}
	
	Die Koide-Formel für Leptonmassen:
	\begin{equation}
		\frac{m_e + m_\mu + m_\tau}{(\sqrt{m_e} + \sqrt{m_\mu} + \sqrt{m_\tau})^2} = \frac{2}{3} \pm 0{,}0004\%
	\end{equation}
	
	zeigt: \textbf{Verhältnisse} sind präziser als Absolutwerte!
	
	\textbf{Frage:} Gilt das auch für g-2?
	
	\subsection{Das Verhältnis der Differenzen}
	
	Definiere die Differenzen:
	\begin{align}
		\Delta a(\mu - e) &= a_\mu - a_e = \frac{4\pi}{f^{5/3}} \\
		\Delta a(\tau - \mu) &= a_\tau - a_\mu = \frac{4\pi}{f^{4/3}} - \frac{4\pi}{f^{5/3}}
	\end{align}
	
	\textbf{Verhältnis:}
	\begin{align}
		\frac{\Delta a(\tau - \mu)}{\Delta a(\mu - e)} &= \frac{4\pi/f^{4/3} - 4\pi/f^{5/3}}{4\pi/f^{5/3}} \\
		&= \frac{f^{5/3}}{f^{4/3}} - 1 \\
		&= f^{5/3 - 4/3} - 1 \\
		&= f^{1/3} - 1
		\label{eq:ratio}
	\end{align}
	
	\begin{important}{Kernvorhersage}
		\begin{equation}
			\boxed{\frac{\Delta a(\tau - \mu)}{\Delta a(\mu - e)} = f^{1/3} - 1 = 18{,}567}
		\end{equation}
		
		Diese Relation ist:
		\begin{itemize}
			\item \textbf{Parameterfrei} (nur $f$!)
			\item \textbf{Unabhängig} von $k_{\text{geom}}$
			\item \textbf{Exakt} (Differenz $< 10^{-13}$)
			\item \textbf{Testbar} bei Belle II
		\end{itemize}
	\end{important}
	
	\subsection{Numerische Verifikation}
	
	Mit $f = 7491{,}91$:
	\begin{align}
		f^{1/3} &= 7491{,}91^{1/3} = 19{,}567 \\
		f^{1/3} - 1 &= 18{,}567
	\end{align}
	
	Aus T0-Werten:
	\begin{align}
		\Delta a(\mu - e) &= 4{,}381 \times 10^{-6} \\
		\Delta a(\tau - \mu) &= 8{,}134 \times 10^{-5} \\
		\text{Verhältnis} &= \frac{8{,}134 \times 10^{-5}}{4{,}381 \times 10^{-6}} = 18{,}567
	\end{align}
	
	\textbf{Übereinstimmung:} Perfekt! ✓✓✓
	
	\subsection{Testbare Vorhersage für Tau}
	
	Mit experimentellen Werten für $e$ und $\mu$:
	\begin{align}
		a_e^{\text{exp}} &= 1{,}160 \times 10^{-3} \\
		a_\mu^{\text{exp}} &= 1{,}166 \times 10^{-3} \\
		\Delta a(\mu - e)^{\text{exp}} &= 6{,}269 \times 10^{-6}
	\end{align}
	
	\textbf{Vorhersage:}
	\begin{align}
		\Delta a(\tau - \mu) &= \Delta a(\mu - e)^{\text{exp}} \times (f^{1/3} - 1) \\
		&= 6{,}269 \times 10^{-6} \times 18{,}567 \\
		&= 1{,}164 \times 10^{-4} \\
		a_\tau^{\text{vorhergesagt}} &= 1{,}166 \times 10^{-3} + 1{,}164 \times 10^{-4} \\
		&= 1{,}282 \times 10^{-3}
	\end{align}
	
	\section{Warum ~2\% Abweichung?}
	
	\subsection{Quanteneffekte höherer Ordnung}
	
	Die QED berechnet g-2 als Störungsreihe:
	\begin{equation}
		a = \frac{\alpha}{2\pi} + \mathcal{O}(\alpha^2) + \mathcal{O}(\alpha^3) + \ldots
	\end{equation}
	
	T0 erfasst die \textbf{geometrische Grundstruktur}, aber nicht alle Quantenkorrekturen höherer Ordnung.
	
	$\Rightarrow$ 2\% entspricht ungefähr $\alpha^2$-Effekten!
	
	\subsection{Diskrete Gitterstruktur}
	
	Das Torsionsgitter ist \textbf{diskret}, nicht kontinuierlich.
	
	Dies führt zu kleinen Korrekturen gegenüber der kontinuierlichen QFT.
	
	\subsection{Pentagonale Symmetriebrechung}
	
	\begin{equation}
		f = f_{\text{ideal}} - 5\varphi
	\end{equation}
	
	Diese Symmetriebrechung (~0{,}1\%) erklärt:
	\begin{itemize}
		\item Materie-Antimaterie-Asymmetrie
		\item Generationenstruktur
		\item Kleine Korrekturen zu idealisierten Werten
	\end{itemize}
	
	\section{Experimentelle Tests}
	
	\subsection{Belle II (2027--2028)}
	
	Belle II erwartet Sensitivität von $\sim 10^{-7}$ für $a_\tau$.
	
	\textbf{Test 1: Absolutwert}
	\begin{itemize}
		\item T0-Vorhersage: $a_\tau = 1{,}271 \times 10^{-3}$
		\item Aus Verhältnis: $a_\tau = 1{,}282 \times 10^{-3}$
		\item Unterschied: ~1\%
	\end{itemize}
	
	\textbf{Test 2: Verhältnis}
	\begin{itemize}
		\item T0-Vorhersage: $\Delta a(\tau - \mu) / \Delta a(\mu - e) = 18{,}567$
		\item Dies ist die \textbf{präzisere} Vorhersage!
		\item Unabhängig von absoluter Kalibrierung
	\end{itemize}
	
	\textbf{Mögliche Ergebnisse:}
	\begin{enumerate}
		\item \textbf{Bestätigung}: Verhältnis $\approx 18{,}6$ \\
		$\Rightarrow$ Starke Evidenz für fraktale Struktur-Hypothese
		
		\item \textbf{Abweichung}: Verhältnis $\neq 18{,}6$ \\
		$\Rightarrow$ Andere fraktale Dimensionen oder zusätzliche Physik
		
		\item \textbf{Null-Ergebnis}: $a_\tau < 10^{-8}$ \\
		$\Rightarrow$ T0-Beiträge unterdrückt oder Theorie benötigt Revision
	\end{enumerate}
	
	\subsection{Fermilab/J-PARC}
	
	Weitere Präzisionsverbesserungen für $a_\mu$:
	\begin{itemize}
		\item Reduktion experimenteller Unsicherheiten
		\item Klarere Bestimmung der SM-Diskrepanz
		\item Verfeinerung der $\Delta a(\mu - e)$ Messung
	\end{itemize}
	
	\section{Vergleich mit anderen Ansätzen}
	
	\begin{table}[h]
		\centering
		\begin{tabular}{lccc}
			\toprule
			\textbf{Ansatz} & \textbf{Präzision} & \textbf{Parameter} & \textbf{Erklärbar} \\
			\midrule
			QED (SM) & Perfekt & Viele & Ja \\
			T0 (semi-empirisch) & 0{,}1\% & 1 angepasst & Teilweise \\
			T0 (geometrisch) & 2\% & 0 & \textbf{Vollständig} \\
			\bottomrule
		\end{tabular}
		\caption{Vergleich verschiedener Ansätze}
	\end{table}
	
	\textbf{T0-Philosophie:} Wir wählen \textbf{Erklärbarkeit} über Präzision!
	
	\section{Zusammenfassung}
	
	\subsection{Was wir zeigen}
	
	\begin{enumerate}
		\item g-2 folgt aus \textbf{rein geometrischen Prinzipien}:
		\begin{itemize}
			\item $\varphi$ (goldener Schnitt)
			\item $\xi$ (Torsionskonstante)
			\item $f$ (Sub-Planck-Faktor)
		\end{itemize}
		
		\item Absolute Werte: ~2\% Abweichung
		\begin{itemize}
			\item Konsistent mit Massenvorhersagen
			\item Durch Quanteneffekte höherer Ordnung erklärbar
		\end{itemize}
		
		\item \textbf{Verhältnisse sind präzise}:
		\begin{equation}
			\frac{\Delta a(\tau - \mu)}{\Delta a(\mu - e)} = f^{1/3} - 1 = 18{,}567
		\end{equation}
		
		\item Testbare Tau-Vorhersage: $a_\tau = 1{,}28 \times 10^{-3}$
	\end{enumerate}
	
	\subsection{Kernbotschaft}
	
	\begin{keypoint}[Ehrlichkeit und Konsistenz]
		Die T0-Theorie erklärt g-2 aus denselben geometrischen Prinzipien wie Massen, fundamentale Konstanten ($G$, $\alpha$, $v$) und Generationenstruktur. Die ~2\% Abweichung bei Absolutwerten ist konsistent mit der Präzision aller T0-Vorhersagen und ehrlich dargestellt. Verhältnis-Vorhersagen wie $\Delta a(\tau - \mu) / \Delta a(\mu - e) = 18{,}567$ sind parameterfrei und präzise -- analog zur Koide-Formel für Massen. Dies ermöglicht klare experimentelle Tests bei Belle~II.
	\end{keypoint}
	
	\section*{Weiterführende Literatur und Ressourcen}
	
	\textbf{T0-Theorie und Python-Skripte:}
	\begin{itemize}
		\item Repository: \href{https://github.com/jpascher/T0-Time-Mass-Duality}{github.com/jpascher/T0-Time-Mass-Duality}
		\item Python-Skripte: \href{https://github.com/jpascher/T0-Time-Mass-Duality/blob/main/2/python/}{github.com/jpascher/T0-Time-Mass-Duality/blob/main/2/python/}
		\item Dokumentation Zeit-Masse-Dualität
		\item Fundamental Fraktale Geometrische Feldtheorie (FFGFT)
	\end{itemize}
	
	\textbf{Experimentelle Ergebnisse:}
	\begin{itemize}
		\item Fermilab Muon g-2 (2025): \href{https://muon-g-2.fnal.gov/}{muon-g-2.fnal.gov}
		\item Theory Initiative White Paper
		\item Belle II: \href{https://www.belle2.org/}{www.belle2.org}
	\end{itemize}
	
	\textbf{Verwandte T0-Dokumente:}
	\begin{itemize}
		\item Leptonmassen: Systematische Herleitung aus Quantenzahlen
		\item Koide-Formel in T0: Geometrische Interpretation
		\item Fraktale Raumzeit: $D_f = 3 - \xi$
	\end{itemize}
	
\end{document}
