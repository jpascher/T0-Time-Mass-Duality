\chapter{Analysis of FFGF (Fundamental Fractal-Geometric Field Theory) and t₀ Theory}

	
	
	\section{Introduction}
	This analysis describes the mathematical framework of the Fundamental Fractal-Geometric Field Theory (FFGF) and the t₀ theory. The focus is on presenting the internal mathematical consistency and structure.
	
	\section{Foundational Postulates and Fractal Spacetime}
	\subsection{Fractal Dimension of Spacetime}
	The central starting point of the theory is the description of spacetime by a fractal dimension \(D_f\) that lies slightly below the topological dimension 3:
	\begin{equation}
		D_f = 3 - \xi, \quad \text{with} \quad \xi = \frac{4}{3} \times 10^{-4}.
		\label{eq:fractal_dimension}
	\end{equation}
	The parameter \(\xi\) quantifies the fractal dimension deficit and is fundamental for all subsequent scalings and corrections
	(see \texttt{T0\_xi\_ursprung.pdf}).
	
	\subsection{The Fractal Correction Factor \(K_{\text{frak}}\)}
	Over many scaling orders, \(\xi\) leads to an accumulated geometric correction factor:
	\begin{equation}
		K_{\text{frak}} = 1 - 100\xi \approx 0.9867.
		\label{eq:K_frak}
	\end{equation}
	This factor modifies fundamental geometric and physical quantities
	(see \texttt{133\_Fraktale\_Korrektur\_Herleitung\_En.pdf}).
	
	\subsection{Time-Mass Duality and the Planck Scale}
	Equating the Planck relation \(E = hf\) with the Einstein relation \(E = mc^2\) and substituting \(f = 1/T\) yields a fundamental duality:
	\begin{equation}
		m = \frac{h}{c^2 T}.
		\label{eq:time_mass_duality}
	\end{equation}
	\subsubsection{Clarification: Effective Planck Scale vs. Fundamental t₀ Scale}
	In this analysis, the **effective limit** of continuous physics is described by the **Planck time \( t_P \)** and **Planck length \(\ell_P\)** (see the section ``The Planck Scale as Limit'' below). Below this scale, the classical concept of space and time breaks down.
	
	The **fundamental t₀ scale** of the theory, however, is **sub-Planck** and describes the internal granulation of the fractal field:
	\begin{itemize}
		\item Sub-Planck length: \(\Lambda_0 = \xi \cdot \ell_P \approx 1.333 \times 10^{-4} \cdot \ell_P \approx 2.15 \times 10^{-39} \) m
		\item Characteristic t₀ lengths and times: \( r_0 = 2GE \), \( t_0 = 2GE \) (see \texttt{Zeit\_En.pdf} and \texttt{010\_T0\_Energie\_En.pdf})
	\end{itemize}
	
	The Planck scale (\(\ell_P\), \( t_P \)) is thus the **outer reference limit** of the effective theory, while \( t_0 \) represents the **sub-Planck granulation** on which the fractal structure truly operates.
	
	As a complement, two interactive visualizations are provided in the \texttt{2/html} directory (GitHub Pages, open in browser):
	\begin{itemize}
		\item \href{https://jpascher.github.io/T0-Time-Mass-Duality/2/html/torus_geometry_ffgf.html}{\texttt{torus\_geometry\_ffgf.html}} – animated torus geometry with energy flow and selectable scale (proton, planet, galaxy).
		\item \href{https://jpascher.github.io/T0-Time-Mass-Duality/2/html/t0_subplanck_structure.html}{\texttt{t0\_subplanck\_structure.html}} – comparison of the effective Planck boundary and the fundamental t₀ sub-Planck scale (Λ₀, τ₀).
	\end{itemize}
	
	\subsection{Modification of Electromagnetic Laws in Fractal Space}
	In a space with \(D_f = 3-\xi\), Coulomb's law experiences a tiny but in principle measurable modification:
	\begin{equation}
		F_{\text{Coulomb}} \propto \frac{1}{r^{1 + \xi}}.
		\label{eq:fractal_coulomb}
	\end{equation}
	Analogously, the speed of light \(c\) is no longer a fundamental constant but a quantity derived from the medium: \(c = \ell_P / t_P\), with an effective, fractally modified velocity \(c_{\text{eff}} \approx c \cdot (1 + \xi/2)\).
	
	\subsection{Key Concepts in the Document}
	\begin{itemize}
		\item Spacetime has a fractal structure with dimension \( D_f = 3 - \xi \), where \( \xi = \frac{4}{3} \times 10^{-4} \).
		\item Mass and time are proposed as dual aspects of the same phenomenon.
		\item Dark matter and dark energy are reinterpreted as geometric effects, not as actual substances.
		\item The vacuum has a fractal structure that prevents infinities.
	\end{itemize}
	
	\section{Mathematical Concepts}
	
	\subsection{1. The Fractal Dimension \( D_f = 3 - \xi \)}
	Given: \( \xi = \frac{4}{3} \times 10^{-4} \approx 0.0001333\ldots \)
	
	Therefore: \( D_f \approx 2.9998666\ldots \)
	
	Mathematical meaning:
	In classical fractal geometry, the Hausdorff dimension describes how an object ``fills'' space:
	\begin{itemize}
		\item A point: \( D = 0 \)
		\item A line: \( D = 1 \)
		\item A surface: \( D = 2 \)
		\item A volume: \( D = 3 \)
		\item Koch snowflake: \( D \approx 1.26 \) (more than a line, less than a surface)
	\end{itemize}
	
	The meaning of \( D_f < 3 \):
	If space has a dimension of 2.9998666 instead of exactly 3, this mathematically means:
	\begin{itemize}
		\item Space is not ``completely filled''.
		\item There is a kind of ``porosity'' or lacunarity.
		\item These gaps constitute 0.0001333 of the dimensionality.
	\end{itemize}
	
	Scaling behavior:
	For true fractals: When the resolution is increased by a factor \( r \), the number of visible structures increases by \( r^D \).
	
	For \( D_f = 3 - \xi \) this would mean:
	\[
	N(r) \propto r^{(3-\xi)}
	\]
	
	\subsubsection{2. The Factor \( \frac{4}{3} \) – Geometric Interpretation}
	Sphere packing:
	The factor \( \frac{4}{3} \) appears frequently in geometry:
	\begin{itemize}
		\item Sphere volume: \( V = \frac{4}{3}\pi r^3 \)
		\item Ratio of sphere volume to enclosing cube: \( \frac{4\pi}{3}/8 \approx 0.524 \)
	\end{itemize}
	
	Densest sphere packing:
	Maximum packing density: \( \frac{\pi}{\sqrt{18}} \approx 0.7405 \)
	Thus, ~26\% ``gaps'' remain.
	
	Possible interpretation in FFGF:
	If the vacuum consists of ``Planck spheres'' or toroidal structures that cannot be packed perfectly, geometric interstices arise. The factor \( \frac{4}{3} \) might encode this packing geometry.
	
	\subsubsection{3. Time-Mass Duality – Deeper Mathematics}
	The derivation:
	From \( E = mc^2 \) and \( E = hf \) it follows:
	\[
	mc^2 = hf = \frac{h}{T}
	\]
	Thus:
	\[
	m = \frac{h}{c^2 T}
	\]
	
	Dimensional analysis:
	\begin{itemize}
		\item \( [h] = \text{Js} = \text{kg·m}^2\text{·s}^{-1} \)
		\item \( [c^2] = \text{m}^2\text{·s}^{-2} \)
		\item \( [T] = \text{s} \)
		\item \begin{align}
			[m] &= \frac{[h]}{[c^2][T]} = \frac{\text{kg·m}^2\text{·s}^{-1}}{(\text{m}^2\text{·s}^{-2})(\text{s})} \\
			&= \frac{\text{kg·m}^2\text{·s}^{-1}}{\text{m}^2\text{·s}^{-1}} = \text{kg} \quad \checkmark
		\end{align}
	\end{itemize}
	
	Frequency interpretation:
	If we substitute \( f = \frac{1}{T} \):
	\[
	m = \frac{hf}{c^2}
	\]
	This is the Compton relation in inverse form! The Compton wavelength of a particle is:
	\[
	\lambda_C = \frac{h}{mc}
	\]
	Inserting the above relation \( m = \frac{hf}{c^2} \), we get:
	\[
	\lambda_C = \frac{h}{\left(\frac{hf}{c^2}\right)c} = \frac{c}{f}
	\]
	This shows that the Compton wavelength corresponds to the wavelength of the oscillation that generates the mass.
	
	What is new in the FFGF interpretation?
	Standard QFT says: Particles have a Compton wavelength based on their mass.
	
	FFGF reverses it: The high-frequency oscillation in the fractal field generates the mass.
	
	\subsubsection{4. The Planck Scale as Effective Limit}
	Planck units (from \( \hbar, G, c \)):
	\begin{align}
		\ell_P &= \sqrt{\frac{\hbar G}{c^3}} \approx 1.616 \times 10^{-35} \text{ m} \\
		t_P &= \sqrt{\frac{\hbar G}{c^5}} \approx 5.391 \times 10^{-44} \text{ s} \\
		m_P &= \sqrt{\frac{\hbar c}{G}} \approx 2.176 \times 10^{-8} \text{ kg}
	\end{align}
	
	The speed of light from these:
	\[
	c = \frac{\ell_P}{t_P} \approx 2.998 \times 10^8 \text{ m/s} \quad \checkmark
	\]
	
	FFGF interpretation:
	These values are not coincidental but arise from the geometry of the fractal lattice. The Planck length is the ``lattice spacing'' of the effective theory, the Planck time is the ``tick'' of the continuous description. Below this scale, the fundamental t₀ granulation operates (see above).
	
	\subsubsection{5. Vacuum Energy and the Cutoff by \( \xi \)}
	The catastrophe problem:
	The zero-point energy of a harmonic oscillator:
	\[
	E_0 = \frac{1}{2}\hbar\omega
	\]
	Summed over all modes up to the Planck frequency:
	\[
	\rho_{\text{vac}} \sim \int_0^{\omega_P} \omega^3 d\omega \sim \omega_P^4 \sim \left(\frac{c}{\ell_P}\right)^4
	\]
	This yields: \( \rho_{\text{vac}} \sim 10^{113} \text{ J/m}^3 \)
	
	Observed: \( \rho_{\text{dark energy}} \sim 10^{-9} \text{ J/m}^3 \)
	
	Discrepancy: Factor \( 10^{122} \) (The largest mismatch in physics)
	
	FFGF solution with \( \xi \):
	In a fractal space with \( D_f = 3 - \xi \), not all modes fit:
	\[
	\rho_{\text{eff}} = \rho_{\text{Planck}} \times (\xi)^n
	\]
	Where \( n \) is a scaling exponent. With \( \xi \sim 10^{-4} \), one could indeed achieve a drastic suppression factor after multiple scaling (over ~30 orders of magnitude from Planck to cosmological scale).
	
	Mathematically:
	\[
	(10^{-4})^{30} \sim 10^{-120}
	\]
	This would be almost the right order of magnitude!
	
	\subsubsection{6. Gravitational Relationship (implied in the document)}
	Although not explicitly stated, FFGF suggests that gravity follows from geometry:
	
	Einstein: \( R_{\mu\nu} - \frac{1}{2}g_{\mu\nu}R = \frac{8\pi G}{c^4} T_{\mu\nu} \)
	
	FFGF would propose: Curvature arises from the local variation of \( D_f \):
	\[
	D_f(r) = 3 - \xi(r)
	\]
	Where \( \xi(r) \) depends on energy density. High mass density \( \rightarrow \) larger \( \xi \rightarrow \) stronger deviation from \( D=3 \rightarrow \) stronger ``curvature''.
	\section{A Closer Look at the Mathematics of Torus Geometry (mentioned in the document)}
\subsection{Why the Torus?}
The torus in FFGF is not a random choice but the geometrically most natural form for a self-sustaining energy flow in a fractal field.

Topological properties:
\begin{itemize}
	\item Closed: No boundaries, energy can circulate endlessly
	\item Two independent circles: Poloidal (small) and toroidal (large) circulation
	\item Non-trivial topology: Genus value \( g = 1 \) (one ``hole'')
\end{itemize}

\subsection{Mathematical Description of the Torus}
Parametric equations:
\begin{align}
	x(\theta, \phi) &= (R + r \cos \theta) \cos \phi \\
	y(\theta, \phi) &= (R + r \cos \theta) \sin \phi \\
	z(\theta, \phi) &= r \sin \theta
\end{align}
Where:
\begin{itemize}
	\item \( R \) = Major radius (distance from center to tube center)
	\item \( r \) = Tube radius (thickness of the ``tube'')
	\item \( \theta \in [0, 2\pi] \) = Poloidal angle (around the tube)
	\item \( \phi \in [0, 2\pi] \) = Toroidal angle (around the main axis)
\end{itemize}

Geometric quantities:
\begin{itemize}
	\item Surface area: \( A = 4\pi^2 R r \)
	\item Volume: \( V = 2\pi^2 R r^2 \)
	\item Ratio: \( \frac{V}{A} = \frac{r}{2} \)
\end{itemize}
This is important! The ratio depends only on the tube radius.

\subsection{Curvature of the Torus}
Gaussian curvature:
\[
K(\theta) = \frac{\cos \theta}{r(R + r \cos \theta)}
\]
Critical observation:
\begin{itemize}
	\item On the inner side (\( \theta = 0 \)): \( K > 0 \) (positive curvature, like a sphere)
	\item On the outer side (\( \theta = \pi \)): \( K < 0 \) (negative curvature, like a saddle)
	\item Top/bottom (\( \theta = \pm\pi/2 \)): \( K = 0 \)
\end{itemize}
The torus thus has regions with different curvature - this is crucial for FFGF!

\subsection{Energy Flow in the Torus (FFGF Model)}
The document describes a poloidal and toroidal flow:
\begin{itemize}
	\item Poloidal flow (\( \theta \)-direction):
	\begin{itemize}
		\item Energy flows through the ``tube''
		\item At the center: Contraction (inflow)
		\item At the edge: Expansion (outflow)
	\end{itemize}
	\item Toroidal flow (\( \phi \)-direction):
	\begin{itemize}
		\item Rotation around the main axis
		\item Generates angular momentum
		\item Stabilizes the structure
	\end{itemize}
\end{itemize}

Vector field for energy flow:
\[
\vec{v}(\theta, \phi) = v_\theta \vec{e}_\theta + v_\phi \vec{e}_\phi
\]
Where the velocities depend on local curvature.

\subsection{Connection to \( D_f = 3 - \xi \)}
The fractal dimension influences the torus structure:

In a perfect 3D space (\( D = 3 \)), a torus could shrink to \( r \to 0 \) (singularity).

With \( D_f = 3 - \xi \) there is a minimal tube radius:
\[
r_{\text{min}} \propto \frac{\ell_{\text{Planck}}}{\xi^{1/3}}
\]
With \( \xi = \frac{4}{3} \times 10^{-4} \):
\[
r_{\text{min}} \sim \frac{\ell_{\text{Planck}}}{(10^{-4})^{1/3}} \sim \ell_{\text{Planck}} \times 10^{4/3} \sim 21 \times \ell_{\text{Planck}}
\]
Interpretation: The fractal structure prevents the torus from collapsing to a point. There is a natural lower limit!

\subsection{Mass from Torus Geometry}
The FFGF thesis: A particle (e.g., a proton) is a high-frequency rotating torus on the Planck scale.

Angular momentum in the torus:
For a rotating mass in the torus:
\[
L = 2\pi^2 R r^2 \rho \omega
\]
Where:
\begin{itemize}
	\item \( \rho \) = Energy density
	\item \( \omega \) = Rotation frequency
\end{itemize}

Mass from rotation:
If we equate \( E = mc^2 \) with the rotational energy:
\[
E_{\text{rot}} = \frac{1}{2} I \omega^2
\]
For the torus, the moment of inertia is:
\[
I = \pi^2 R r^2 \left(R^2 + \frac{3r^2}{4}\right) \rho
\]

The relationship to time:
With \( \omega = \frac{2\pi}{T} \) and the previously derived relationship \( m = \frac{h}{c^2 T} \):
\[
T = \frac{h}{mc^2}
\]
Inserting this for a proton (\( m_p \approx 1.67 \times 10^{-27} \) kg):
\[
T_p \approx \frac{6.6 \times 10^{-34}}{1.67 \times 10^{-27} \times 9 \times 10^{16}} \approx 4.4 \times 10^{-24} \text{ s}
\]
This is the Compton time of the proton! The torus rotates with this frequency.

\subsection{Scaling: From Proton to Galaxy}
The fractal self-similarity means:
\begin{table}[H]
	\centering
	\begin{tabular}{|c|c|c|c|}
		\hline
		Scale & \( R \) (Major radius) & \( r \) (Tube) & Mass/System \\
		\hline
		Proton & \( \sim 10^{-15} \) m & \( \sim 10^{-16} \) m & \( 1.67 \times 10^{-27} \) kg \\
		Atom & \( \sim 10^{-10} \) m & \( \sim 10^{-11} \) m & Electrons in orbitals \\
		Planet & \( \sim 10^{6} \) m & \( \sim 10^{5} \) m & Magnetic field torus \\
		Star & \( \sim 10^{9} \) m & \( \sim 10^{8} \) m & Convection currents \\
		Galaxy & \( \sim 10^{20} \) m & \( \sim 10^{19} \) m & Spiral arms \\
		\hline
	\end{tabular}
\end{table}
The ratio \( R/r \) often remains constant (typically \( R/r \approx 3-10 \)), showing self-similarity.

\subsection{Why is the Torus Stable?}
Energy minimum:
The torus minimizes energy for a given volume and topology:
\[
E_{\text{total}} = E_{\text{Surface}} + E_{\text{Curvature}} + E_{\text{Rotation}}
\]
Calculus of variations shows that for certain boundary conditions (constant flux, angular momentum) the torus is the most stable form.

In the fractal field:
The dimension \( D_f = 3 - \xi \) means energy experiences ``resistance'' when flowing. The torus is the path of least resistance for circulating energy.

\subsection{Connection to the Schwarzschild Metric}
Interestingly: Considering the Kerr metric (rotating black hole), one also finds a torus structure:

Ergosphere: The region around a rotating black hole where nothing can stand still has a toroidal form!

FFGF would say: This is no coincidence - the black hole is simply a torus on a larger scale.

\section{Connection Between Torus Topology and Quantum Numbers (Spin, Charge)}

\subsection{Topological Quantum Numbers from Torus Geometry – Detailed Derivation}

FFGF and t₀ theory derive the fundamental quantum numbers of elementary particles (spin, electric charge, and color charge) directly from the topological structure of the torus. The torus is considered the most stable and natural geometric form for closed, self-consistent energy flows. All quantum numbers arise from the properties of closed flux lines that must wind on the torus surface or through the torus and close exactly to form stable configurations.

The central idea is that particles are not understood as point particles but as topologically stable vortex and flow structures in the fractally modified torus field. Quantization arises inevitably from the closure conditions of these flux lines – similar to quantized magnetic fluxes or the Aharonov-Bohm effect, but on a fundamental geometric level.

\subsubsection{1. Spin – The Winding Number $w = n_\phi / n_\theta$}

The spin of a particle corresponds to the **winding number** of the closed flux lines on the torus. This is defined as the ratio of revolutions in the two non-trivial directions of the torus:

\begin{equation}
	w = \frac{n_\phi}{n_\theta}
	\label{eq:winding_number}
\end{equation}

where
\begin{itemize}
	\item $n_\phi$ is the number of revolutions in the **toroidal direction** (around the major radius $R$),
	\item $n_\theta$ is the number of revolutions in the **poloidal direction** (around the tube radius $r$).
\end{itemize}

A flux line is only stable if it closes exactly after an integer number of windings. The simplest non-trivial closed orbits occur for rational values of $w$.

The physical assignment is:
\begin{itemize}
	\item $w = 1$ \quad (full revolution before closure) $\quad \to$ **Boson spin** (integer: 0, 1, 2, …)
	\item $w = 1/2$ \quad (half revolution before closure) $\quad \to$ **Fermion spin** (half-integer: 1/2, 3/2, …)
\end{itemize}

This topological definition naturally explains the spin-statistics theorem: Fermions require two half revolutions (720°) to return to the original state, while bosons are identical after 360°. The minimal winding number is limited by the stability condition $r_{\min} \approx 21 \, \ell_{\text{Planck}}$; smaller values lead to unstable configurations.

\subsubsection{2. Electric Charge – Quantized Electric Flux Through the Torus}

The electric charge directly correlates with the number of closed electric flux lines that **traverse** the torus (i.e., run from the inner to the outer region or vice versa).

The quantization condition is:
\begin{equation}
	\Phi = n \cdot \frac{h}{e}
	\label{eq:flux_quantization}
\end{equation}

where
\begin{itemize}
	\item $\Phi$ is the magnetic flux through a suitable cross-section of the torus,
	\item $h$ is Planck's constant,
	\item $e$ is the elementary charge,
	\item $n \in \mathbb{Z}$ is the integer number of traversing flux lines (positive or negative depending on direction).
\end{itemize}

Physical interpretation:
\begin{itemize}
	\item $n = +1$ $\quad \to$ Charge $+e$ \quad (e.g., proton, positron)
	\item $n = -1$ $\quad \to$ Charge $-e$ \quad (e.g., electron)
	\item $n = 0$   $\quad \to$ Electrically neutral \quad (e.g., neutron, neutrino, photon)
	\item $n = +2, -2, \dots$ $\quad \to$ Higher charges (possible in theory but energetically unfavorable or unstable on low scales)
\end{itemize}

The quantization is topologically protected because the torus has two non-contractible loops (toroidal and poloidal). The flux through these loops is invariant under continuous deformations – therefore the charge cannot vary continuously.

\subsubsection{3. Color Charge – Topological Linking of Three Flux Strands}

The color charge (quantum number of the strong interaction) arises from the **topological linking** of exactly **three flux strands** that wind around each other and around the torus. These three strands represent the three colors of QCD: red, green, blue.

The linking configuration determines the color properties:
\begin{itemize}
	\item Three different colors (red–green–blue) in non-trivial linking $\quad \to$ **Quark** \quad (Color charge 1 in each color)
	\item Three identical colors (e.g., red–red–red) $\quad \to$ **Antiquark** \quad (Color charge $-1$ in each color)
	\item One color + its anticolor (e.g., red + antired) $\quad \to$ **Gluon** \quad (Color neutral but color-anticolor combination)
	\item All three colors simultaneously balanced (red + green + blue) $\quad \to$ **Baryon** \quad (Color overall white/neutral)
\end{itemize}

The theory shows that exactly **eight** non-trivial linking states of the three strands are possible (plus the trivial white state). These eight states correspond precisely to the **eight generators of SU(3) color symmetry** – thus the gauge group SU(3)$_C$ of the strong interaction is derived purely topologically without additional postulates.

\subsubsection{Torus Geometry in Quantum Computing}

The fundamental toroidal structure identified in FFGF theory extends 
naturally to quantum information processing. In quantum computing 
applicationsIn quantum computing applications (Quantum Computing in T0 Framework, 2025), the torus manifests through:

\begin{enumerate}
	\item \textbf{Qubit State Space:} Qubits reside on the torus surface, 
	with state described by position $(z, r, \theta)$ in local 
	cylindrical coordinates.
	
	\item \textbf{Local Approximation:} For single-qubit operations, the 
	large toroidal radius $R$ allows a cylindrical approximation:
	\[
	R \gg r \quad \Rightarrow \quad 
	\text{Torus} \approx \text{Cylinder locally}
	\]
	
	\item \textbf{Global Topology:} Multi-qubit entanglement preserves the 
	toroidal topology (Genus-1), enabling:
	\begin{itemize}
		\item Charge quantization via flux through torus hole
		\item Spin quantization via winding numbers
		\item Topologically protected quantum information
	\end{itemize}
	
	\item \textbf{Bell Correlations:} The $\xi$-damping observed in Bell 
	tests arises from the fractal modification of torus geometry.
\end{enumerate}

\textbf{Quantitative Example:}

For a proton modeled as a torus:
\begin{align}
	R_{\text{proton}} &\sim 10^{-15} \text{ m} \quad \text{(major radius)} \\
	r_{\text{proton}} &\sim 21\ell_P \approx 10^{-34} \text{ m} \quad 
	\text{(tube radius)} \\
	R/r &\sim 10^{19} \quad \text{(aspect ratio)}
\end{align}

A qubit encoded in this structure experiences:
\[
\text{Curvature correction} \sim \frac{r}{R} \sim 10^{-19} 
\ll \xi \sim 10^{-4}
\]

Thus, the cylindrical approximation is valid for quantum gates, while 
the toroidal topology remains crucial for fundamental properties 
(charge, spin, entanglement structure).

\section{Torus Geometry in Cosmology – Scale-Invariant Torsional Structures}

A central and particularly ambitious aspect of the Fundamental Fractal-Geometric Field Theory (FFGF) and the t₀ theory is that torus geometry is not only relevant on the Planck scale and the scale of elementary particles, but continues **self-similarly and scale-invariantly** up to the largest observable cosmic structures.

The theory postulates that on every physical scale – from protons to stars and black holes to galaxies and the large-scale cosmic web – the dominant energy and momentum dynamics can be described by **torsion-like, vortex-shaped flow structures** that topologically correspond to a torus. These structures are characterized by the major radius $R$ (toroidal great circle radius) and the tube radius $r$ and are modified by the fractal dimension deficit $\xi$.

\subsubsection{Cross-Scale Torsional Correspondences}

The following overview summarizes the most important cosmological correspondences as described in the documents:

\begin{itemize}
	\item \textbf{Elementary Particle Scale (Planck to Hadron scale)} \\
	$R \sim 10^{-15}\,\text{m}$ (proton radius), $r \sim 10^{-16}\,\text{m}$ to $21\,\ell_P$ \\
	Stabilized energy vortex (``mass torus'') with Compton frequency. \\
	Poloidal and toroidal flows generate rest mass, spin, and internal quantum numbers. \\
	Primary source: 006\_T0\_Teilchenmassen\_En.pdf
	
	\item \textbf{Star and Black Hole Scale} \\
	$R \approx$ Schwarzschild radius $r_S = 2GM/c^2$ \\
	Rotating spacetime vortex corresponding to the Kerr metric. \\
	The accretion disk and the ergosphere together form a macroscopic torus in which kinetic energy, angular momentum, and gravitational binding energy circulate. \\
	The torus stabilizes the extreme rotational and gravitational fields and explains the existence of stable rotating black holes without additional exotic matter. \\
	Primary source: T0\_Kosmologie.pdf
	
	\item \textbf{Galactic Scale} \\
	$R \sim 10^{20}\,\text{m}$ (typical radius of the bulge / central region) \\
	$r \sim 10^{19}\,\text{m}$ (effective thickness of the galactic disk) \\
	Large-scale filamentary vortices in the cosmic web. \\
	The spiral arms are interpreted as standing density waves within a torsional base structure. \\
	The total galactic angular momentum ensures long-term stabilization of the torus configuration. \\
	The flat rotation curve and observed distribution of star velocities arise geometrically from the fractal modification of torus volume and curvature distribution – without additional dark matter. \\
	Primary sources: T0\_Kosmologie.pdf, 145\_FFGFT\_donat-teil1\_En.pdf
	
	\item \textbf{Cosmological Large Structure Scale (cosmic web, filaments, void structures)} \\
	$R \sim 10^{23}$–$10^{24}\,\text{m}$ (order of magnitude of the largest observed filaments and superclusters) \\
	$r \sim 10^{22}$–$10^{23}\,\text{m}$ (thickness of filaments) \\
	The cosmic web is interpreted as a hierarchical system of nested torsion-like vortices. \\
	The large-scale structures (filaments, walls, voids) correspond to the stable nodes and empty spaces of a huge, fractally modulated torus network. \\
	The observed anisotropy (e.g., CMB dipole, Hubble tension, large-scale flows) is explained as a natural consequence of asymmetric torsional flow dynamics – without cosmic expansion or $\Lambda$CDM parameters. \\
	Primary sources: 039\_Zwei-Dipole-CMB\_En.pdf, T0\_Kosmologie.pdf
\end{itemize}

\subsubsection{Core Principle: Scale Invariance and Fractal Self-Similarity}

The torus geometry is **scale-invariant** in FFGF/t₀ theory:
\[
\frac{R}{r} \approx \text{constant} \quad \text{over many orders of magnitude}
\]
(typical values range between 5 and 50, depending on the scale considered).

The fractal dimension deficit $\xi = 4/3 \times 10^{-4}$ ensures that the effective geometric quantities (surface area $A_{\text{frak}}$, volume $V_{\text{frak}}$, curvature $K_{\text{frak}}$) are consistently modified on every scale – enabling the theory to provide a unified description from micro- to macrocosm.

\subsubsection{Cosmological Implications – Without Dark Matter and Without Expansion}

The theory makes the following strong claims:
\begin{itemize}
	\item Galaxy rotation curves arise purely from fractal-torsional geometry (no additional invisible mass needed).
	\item The Hubble tension (discrepancy between local and CMB-based $H_0$) is a geometric effect of different effective torus scales.
	\item The CMB dipole and large-scale flows are manifestations of a global torsional flow (``Two-Dipole Model'').
	\item The universe is static on the largest scale – expansion is not necessary.
\end{itemize}

These predictions and derivations are documented in detail in:
\begin{itemize}
	\item T0\_Kosmologie.pdf
	\item 145\_FFGFT\_donat-teil1\_En.pdf
	\item 039\_Zwei-Dipole-CMB\_En.pdf
\end{itemize}

Torus cosmology thus represents a radical attempt to derive the entire hierarchy of cosmic structures from a single geometric basic form (the fractally modified torus) – an approach that consciously distinguishes itself from the metric-dynamic description of General Relativity.

\subsection{Two-Dipole Model in Detail}

The Two-Dipole Model is a central element of the Fundamental Fractal-Geometric Field Theory (FFGF) and the t₀ theory, specifically developed to explain anomalies in the Cosmic Microwave Background radiation (CMB). It is presented in the repository documents as a geometric approach that solves the observed CMB dipole without the necessity of cosmic expansion or dark energy. Instead, the dipole is interpreted as a manifestation of two superimposed torsional flows arising from the fractal torus structure of spacetime. The detailed derivations are found primarily in 039\_Zwei-Dipole-CMB\_En.pdf, supplemented by cosmological sections in T0\_Kosmologie.pdf and 145\_FFGFT\_donat-teil1\_En.pdf.

\subsubsection{Introduction and Motivation}

The standard $\Lambda$CDM model interprets the CMB dipole (a temperature anisotropy of $\Delta T / T \approx 10^{-3}$) primarily as a kinematic effect due to the peculiar motion of the Milky Way relative to the CMB rest frame (with $v \approx 370\,\text{km/s}$). However, there are persistent discrepancies: The dipole appears stronger and more asymmetric than expected and does not perfectly correspond to large-scale flows (e.g., Shapley Attractor, Laniakea Supercluster). Additionally, the dipole contributes to the Hubble tension ($H_0$ discrepancy between local and CMB-based measurements of about $5\sigma$).

The Two-Dipole Model solves these problems by modeling the dipole as the superposition of **two geometric components**:
\begin{itemize}
	\item \textbf{Kinematic dipole}: Local motion effects (similar to the standard model).
	\item \textbf{Intrinsic geometric dipole}: Fractal-torsional asymmetry of spacetime itself arising from the $\xi$-modified torus structure.
\end{itemize}

This approach leads to a static universe where apparent expansion effects are geometric – without a Big Bang or dark energy.

\subsubsection{Model Description}

The model is based on the assumption that spacetime on a cosmic scale possesses a **global torsional structure** that is self-similar to torus geometry on smaller scales (elementary particles, black holes, galaxies). The CMB dipole arises from two superimposed poles:

1. **Local dipole**: Generated by the motion of the Local Group (Milky Way) in a torsional flow field. This corresponds to the standard dipole but modified by fractal corrections.

2. **Global dipole**: An intrinsic effect of fractal spacetime resulting from the asymmetry of the cosmic torus network. The global flow is scale-invariant and connects the Planck scale ($\ell_P$) with the Hubble scale ($c/H_0$).

The superposition of the two dipoles explains the observed asymmetries: The local dipole dominates on small scales, while the global one becomes visible on large scales (e.g., in CMB multipoles).

\subsubsection{Mathematical Framework}

The dipole moment is modeled as a vector sum:
\begin{equation}
	\vec{D}_{\text{total}} = \vec{D}_{\text{kin}} + \vec{D}_{\text{geo}}
	\label{eq:two_dipole}
\end{equation}

- **Kinematic dipole $\vec{D}_{\text{kin}}$**:
\[
\Delta T(\hat{n}) = T_0 \frac{\vec{v} \cdot \hat{n}}{c} \quad \Rightarrow \quad D_{\text{kin}} \approx 3.35\,\text{mK}
\]
(with $T_0 \approx 2.725\,\text{K}$, $v \approx 370\,\text{km/s}$, $\hat{n}$ line of sight).

- **Geometric dipole $\vec{D}_{\text{geo}}$**:
It arises from the fractal modification of the spacetime metric:
\[
D_{\text{geo}} \sim \xi \cdot \ln\left(\frac{L_{\text{Hubble}}}{\ell_P}\right) \cdot T_0 \approx 0.1\,\text{mK}
\]
where $\xi = 4/3 \times 10^{-4}$ is the dimension deficit, and the logarithm accounts for the scale hierarchy over $\sim 60$ orders of magnitude.

The direction of the global dipole aligns with the axis of the cosmic torus flow, deviating from the galactic dipole by $\sim 48^\circ$ – explaining the observed misalignment.

The Hubble constant $H_0$ is interpreted as a geometric effect:
\[
H_0 = \frac{c \xi}{R_{\text{torus}}} \approx 70\,\text{km/s/Mpc}
\]
where $R_{\text{torus}}$ is the effective cosmic major radius.

\subsubsection{Cosmological Implications}

- **Solution to the Hubble tension**: Local measurements ($H_0 \approx 73\,\text{km/s/Mpc}$) see the kinematic dipole, CMB measurements ($H_0 \approx 67\,\text{km/s/Mpc}$) see the geometric one – the discrepancy arises from the superposition.

- **Static universe**: No expansion needed; redshift $z$ results from fractal energy loss:
\[
z \approx \xi \cdot \ln(d / \ell_P)
\]
(with $d$ distance).

- **CMB anomalies**: The model explains the dipole, quadrupole weakness, and hemispherical asymmetry as torsional effects.

- **Quantitative predictions**: Dipole amplitude $\Delta T \approx 3.36\,\text{mK}$ (consistent with Planck data), misalignment angle $48^\circ$ (consistent with observations).

\subsubsection{Critical Analysis}

The model is elegant and solves several anomalies geometrically without new parameters. However, a formal derivation from field equations is lacking (compared to standard cosmology). Experimental validation is pending; it contradicts the $\Lambda$CDM paradigm. Further details are in the sources.

\subsection{Parallel to the Toroidal Photon Model (Williamson \& van der Mark, 1997)}

Since 1997, an independent semi-classical approach has existed in the literature describing the electron as a circulating, topologically closed photonic entity with toroidal character. The original paper is titled:

\begin{center}
	\textbf{Is the electron a photon with toroidal topology?} \\
	J. G. Williamson and M. B. van der Mark \\
	Annales de la Fondation Louis de Broglie, Vol. 22, No. 2, 1997, pp. 133--167
\end{center}

The full text is freely available online at: \\
\url{https://fondationlouisdebroglie.org/IMG/pdf/22_2_133.pdf}

A very clear and pedagogically excellent popular-science explanation of this model can be found in the following video:

\begin{center}
	\textbf{Is the Electron a Photon with Toroidal Topology?} \\
	YouTube video by \emph{Physics Explained} (2021) \\
	\url{https://www.youtube.com/watch?v=hYyrgDEJLOA}
\end{center}

Although this model was developed independently of the FFGF/t₀ theory, it exhibits striking structural parallels to the toroidal geometry presented here — especially in the derivation of charge, spin, and magnetic moment from a closed, double-loop field configuration.

\subsubsection{Key Parallels to the FFGF Torus Structure}

\begin{itemize}
	\item \textbf{Torus Topology and Double Loop}\\
	In the referenced model, a circularly polarized electromagnetic field of exactly one Compton wavelength $\lambda_C$ is folded into a closed double loop (double helix / double loop). This corresponds precisely to the toroidal + poloidal circulation postulated in the FFGF: energy flows both toroidally ($\phi$-direction, large circle) and poloidally ($\theta$-direction, around the tube). The double circulation (4$\pi$ instead of 2$\pi$) leads — in both approaches — to half-integer spin ($w = 1/2$ in the FFGF winding-number definition).
	
	\item \textbf{Electric Field and Charge as Topological Property}\\
	In the toroidal model, the electric field vector consistently points inward on the outside (electron) or outward (positron) because field rotation is commensurate with the geometry. This is structurally identical to the FFGF derivation: electric charge arises from the quantized number of closed electric flux lines threading the torus ($\Phi = n \cdot h/e$). The direction (inward/outward) is topologically fixed and reflects the orientation of the poloidal/toroidal flux components.
	
	\item \textbf{Magnetic Moment from Toroidal Magnetic Field Configuration}\\
	Both approaches derive the magnetic dipole moment from closed magnetic field lines running parallel to the torus surface (toroidal $B_\phi$ field in FFGF). The net moment along the torus axis arises inevitably from the asymmetry of the internal rotation — exactly as in the FFGF the intrinsic magnetic moment of the electron ($\mu_e = e\hbar / (2m_e)$) follows from rotational energy in the torus.
	
	\item \textbf{Compton Scale as Intrinsic Size}\\
	In the external model, the Compton wavelength $\lambda_C = h/(m_ec)$ determines the length of the closed path and thus the effective size of the object ($\sim \lambda_C / (4\pi)$ for the core radius). This agrees with the FFGF, where the Compton time $T = h/(m c^2)$ sets the fundamental rotation period of the torus and the minimal stable tube radius $r_{\min} \sim 21\,\ell_P$ is limited by the fractal correction $\xi$. Both approaches thereby avoid the infinite self-energy of a point particle.
	
	\item \textbf{Two Chiral Spin States}\\
	The toroidal model distinguishes two non-superimposable chiral variants (handedness) that only return to themselves after 720° rotation — exactly as in the FFGF spin-1/2 arises from the winding number $w = n_\phi / n_\theta = 1/2$ and fermions require two full rotations to return to the original state.
\end{itemize}

\subsubsection{Differences and Extension by the FFGF}

While the 1997 model remains semi-classical and leaves the self-confinement mechanisms (nonlinear effects, topological stability) largely open, the FFGF/t₀ theory provides a more comprehensive foundation:

\begin{itemize}
	\item The fractal dimension modification $D_f = 3 - \xi$ prevents collapse below $r_{\min} \approx 21\,\ell_P$ and explains stability without additional nonlinear vacuum effects.
	\item Energy flow is explicitly poloidal + toroidal and fractally modulated ($\vec{v}(\theta,\phi)$ depending on local curvature $K(\theta)$).
	\item Quantum numbers (including color charge) arise purely topologically from linking numbers and winding numbers — a generalization that extends far beyond the pure electron model.
	\item Mass emerges not only from confined field energy but from the inertia of the inner T₀-scale flow ($m = h/(c^2 T)$ with $T$ as Compton time).
\end{itemize}

\section{Electromagnetic Fields in Torus Geometry}
\subsection{Maxwell's Equations on the Torus}
In curved coordinates, Maxwell's equations must be adapted:

In torus coordinates (\( \theta, \phi, \psi \)):
\begin{align}
	\nabla \times \vec{E} &= -\frac{\partial \vec{B}}{\partial t} \\
	\nabla \times \vec{B} &= \mu_0 \vec{j} + \mu_0 \varepsilon_0 \frac{\partial \vec{E}}{\partial t} \\
	\nabla \cdot \vec{E} &= \frac{\rho}{\varepsilon_0} \\
	\nabla \cdot \vec{B} &= 0
\end{align}

The nabla operator in torus coordinates is more complex:
\[
\nabla = \frac{1}{h_\theta} \frac{\partial}{\partial \theta} \vec{e}_\theta + \frac{1}{h_\phi} \frac{\partial}{\partial \phi} \vec{e}_\phi + \frac{1}{h_\psi} \frac{\partial}{\partial \psi} \vec{e}_\psi
\]
Where \( h_\theta, h_\phi, h_\psi \) are the metric factors.

\subsection{Magnetic Field Configuration in the Torus}
\begin{itemize}
	\item Poloidal magnetic field \( B_\theta \):
	Runs around the tube. Arises from toroidal currents.
	\item Toroidal magnetic field \( B_\phi \):
	Runs around the main axis. Arises from poloidal currents.
\end{itemize}

The total field configuration:
\[
\vec{B} = B_\theta(r, \theta) \vec{e}_\theta + B_\phi(r, \theta) \vec{e}_\phi
\]

\subsection{Stability Condition (Kruskal-Shafranov)}
For a stable torus plasma (as in fusion reactors!) it must hold:
\[
q = \frac{r B_\phi}{R B_\theta} > 1
\]
This is the safety factor \( q \).

In FFGF: Elementary particles are stable because their torus configuration automatically satisfies \( q > 1 \)!

\subsection{Origin of the Magnetic Moment}
A rotating torus with charge generates a magnetic dipole moment:
\[
\mu = I \times A = \left(\frac{Q}{T}\right) \times \pi r^2
\]
Where:
\begin{itemize}
	\item \( Q \) = Charge
	\item \( T \) = Rotation period
	\item \( r \) = Tube radius
\end{itemize}

For an electron:
\[
\mu_e = \frac{e \hbar}{2 m_e} = \text{Bohr magneton}
\]
This is the intrinsic magnetic moment of the electron!

\subsection{Electromagnetic Self-Energy}
The energy stored in the electromagnetic field of a torus:
\[
E_{\text{em}} = \frac{\varepsilon_0}{2} \int E^2 dV + \frac{1}{2\mu_0} \int B^2 dV
\]
For a torus with radius \( R \) and \( r \):
\[
E_{\text{em}} \propto \frac{e^2}{r} \times f\left(\frac{R}{r}\right)
\]
Where \( f(R/r) \) is a geometric factor.

This energy contributes to mass!
\[
m_{\text{em}} = \frac{E_{\text{em}}}{c^2}
\]
A portion of the electron mass (\( \sim 0.1\% \)) stems from this electromagnetic self-energy.

\subsection{Connection to \( \xi \) and \( D_f \)}
In a fractal space with \( D_f = 3 - \xi \), Coulomb's law changes:

Standard physics (\( D = 3 \)):
\[
F \propto \frac{1}{r^2}
\]
Fractal space (\( D_f = 3 - \xi \)):
\[
F \propto \frac{1}{r^{1 + \xi}}
\]
For \( \xi = \frac{4}{3} \times 10^{-4} \):
\[
F \propto \frac{1}{r^{1.0001333\ldots}}
\]
On large scales, this leads to a tiny modification that explains ``dark energy'' effects!

\section{Fluid Dynamics in the Torus (Navier-Stokes on Curved Spaces)}
\subsection{Navier-Stokes in Curved Coordinates}
The Navier-Stokes equations describe the flow of fluids (or in FFGF: the dynamics of the vacuum ``fluid'').

Standard form:
\[
\rho\left(\frac{\partial \vec{v}}{\partial t} + (\vec{v} \cdot \nabla) \vec{v}\right) = -\nabla p + \eta \nabla^2 \vec{v} + \vec{f}
\]

In torus coordinates: we must use the covariant derivative:
\[
\rho\left(\frac{\partial v^i}{\partial t} + v^j \nabla_j v^i\right) = -\nabla^i p + \eta g^{ij} \nabla_j \nabla_k v^k + f^i
\]
Where:
\begin{itemize}
	\item \( g^{ij} \) = Metric tensor
	\item \( \nabla_j \) = Covariant derivative
	\item \( \eta \) = Viscosity of the vacuum medium
\end{itemize}

\subsection{Metric Tensor for the Torus}
For a torus in standard position:
\[
ds^2 = d\theta^2 + (R + r \cos \theta)^2 d\phi^2
\]
Metric tensor:
\[
g = \begin{bmatrix}
	1 & 0 \\
	0 & (R + r \cos \theta)^2
\end{bmatrix}
\]
Determinant:
\[
\sqrt{g} = R + r \cos \theta
\]

\subsection{Velocity Field in the Rotating Torus}
Assumption: Steady rotation with constant angular velocity \( \omega \).

Poloidal component:
\[
v_\theta(r, \theta) = v_0 \sin(n \theta)
\]
Where \( n \) is the number of vortices.

Toroidal component:
\[
v_\phi(r, \theta) = \omega (R + r \cos \theta)
\]

\subsection{Vorticity}
The vorticity is:
\[
\vec{\omega} = \nabla \times \vec{v}
\]
In torus coordinates:
\[
\omega_r = \frac{1}{h_\theta h_\phi} \left[ \frac{\partial (h_\phi v_\phi)}{\partial \theta} - \frac{\partial (h_\theta v_\theta)}{\partial \phi} \right]
\]
For a stable torus vortex: The vorticity must remain positive everywhere (no backflows).

\subsection{Energy Conservation in Torus Flow}
The kinetic energy of the flow:
\[
E_{\text{kin}} = \frac{\rho}{2} \int v^2 dV
\]
For a torus:
\[
E_{\text{kin}} = \frac{\rho}{2} \times 2\pi^2 R r \times \langle v^2 \rangle
\]

Dissipation due to viscosity:
\[
\frac{dE}{dt} = -\eta \int (\nabla \times \vec{v})^2 dV
\]

Equilibrium: If energy input (through vacuum fluctuations on the Planck scale) balances dissipation, the torus is stable.

\subsection{Turbulence and Stability}
The Reynolds number for a torus:
\[
Re = \frac{\rho v R}{\eta}
\]
Critical value: \( Re_{\text{crit}} \approx 2300 \)

For \( Re < Re_{\text{crit}} \): Laminar flow (stable) \\
For \( Re > Re_{\text{crit}} \): Turbulent flow (unstable)

In FFGF:
The ``viscosity'' \( \eta \) of the vacuum is determined by \( \xi \):
\[
\eta \propto \frac{\hbar}{\ell_{\text{Planck}}^3 \times \xi}
\]
With \( \xi = \frac{4}{3} \times 10^{-4} \) results in a very low viscosity \( \rightarrow \) the vacuum behaves like a superfluid!

\subsection{Helmholtz Decomposition}
Any vector field can be decomposed into:
\[
\vec{v} = \nabla \varphi + \nabla \times \vec{A}
\]
\begin{itemize}
	\item Potential part (\( \nabla \varphi \)): Compressible flow
	\item Vortex part (\( \nabla \times \vec{A} \)): Incompressible rotation
\end{itemize}

In the torus: The vortex part dominates! This is the reason for stability.

\subsection{Casimir Effect in the Torus}
Between the two surfaces of the torus (inside/outside) a Casimir pressure arises:
\[
P_{\text{Casimir}} = -\frac{\pi^2 \hbar c}{240 d^4}
\]
Where \( d \) is the distance (here: tube radius \( 2r \)).

This pressure stabilizes the torus against collapse!

\subsection{Connection to Time-Mass Duality}
The effective flow velocity in the torus on the Planck scale is:
\[
v \sim \frac{\ell_{\text{Planck}}}{t_P} = c
\]
This corresponds to the speed of light and shows that \( c \) emerges as an effective velocity from the Planck scale.

On the fundamental t₀ scale (sub-Planck), however:
\[
v_0 \sim \frac{\Lambda_0}{t_0} = \frac{\xi \cdot \ell_{\text{Planck}}}{t_0}
\]
where \( t_0 \) is the sub-Planck time (2GE). Mass arises from the inertia of this internal flow at the t₀ granulation level.

\subsection{Clarification: Effective Planck Scale vs. Fundamental t₀ Scale}
To avoid confusion: In this analysis, the **effective limit** of continuous physics is described by the **Planck length \( \ell_P \)** and **Planck time \( t_P \)**. The minimal stable torus tube is at \( r_{\min} \approx 21 \ell_P \), i.e., significantly above \( \ell_P \).

The **fundamental t₀ scale**, however, is **sub-Planck** and describes the internal granulation of the fractal field:
\begin{itemize}
	\item Sub-Planck length: \( \Lambda_0 = \xi \cdot \ell_P \approx 1.333 \times 10^{-4} \cdot \ell_P \approx 2.15 \times 10^{-39} \) m
	\item Characteristic t₀ lengths and times: \( r_0 = 2GE \), \( t_0 = 2GE \) (see \texttt{Zeit\_En.pdf} and \texttt{010\_T0\_Energie\_En.pdf})
\end{itemize}

The Planck scale is thus the **outer reference limit** of the effective theory, while \( t_0 \) represents the **sub-Planck granulation** on which the fractal structure truly operates.

\subsection{Fractal Turbulence}
In a space with \( D_f = 3 - \xi \), the turbulence energy spectrum changes:

Kolmogorov spectrum (\( D = 3 \)):
\[
E(k) \propto k^{-5/3}
\]
Fractal spectrum (\( D_f = 3 - \xi \)):
\[
E(k) \propto k^{-(5/3 - \xi/3)}
\]
This could be measurable in cosmic plasma structures!

\section{Overall Synthesis: The Three Aspects Together}
\begin{itemize}
	\item Fluid dynamics generates stable vortices (torus form)
	\item Electromagnetic fields arise from the rotation of charged vortices
	\item Quantum numbers are topological properties of linking
\end{itemize}

Everything is connected through:
\begin{itemize}
	\item The fractal dimension \( D_f = 3 - \xi \)
	\item The Planck time \( t_0 \) as fundamental rhythm
	\item The torus geometry as the most stable form
\end{itemize}
