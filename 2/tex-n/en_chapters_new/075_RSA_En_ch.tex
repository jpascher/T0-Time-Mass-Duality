% Chapter file: 075_RSA_En_ch.tex
% Source: 075_RSA_En.tex

\chapter{Mathematical Analysis of T0-Shor Algorithm:}

\hfuzz=200pt
\allowdisplaybreaks

Theoretical Framework and Computational Complexity \\
	\large A Rigorous Examination of the T0-Energy Field Approach to Integer Factorization

\section*{Abstract}
		This paper presents a mathematical analysis of the T0-Shor algorithm based on energy field formulation. We examine the theoretical foundations of the time-mass duality $T(x,t) \cdot m(x,t) = 1$ and its application to integer factorization. The analysis focuses on the mathematical consistency of the field equations, computational complexity implications, and the role of the coupling parameter $\xi$ derived from Higgs field interactions. We provide rigorous derivations of the algorithm's theoretical performance characteristics and identify the fundamental assumptions underlying the T0 framework.
	
	
	\section{Introduction}
	
	The T0-Shor algorithm represents a theoretical extension of Shor's factorization algorithm based on energy field dynamics rather than quantum mechanical superposition. This work examines the mathematical foundations of this approach without making claims about practical implementability or superiority over existing methods.
	
	\subsection{Theoretical Framework}
	
	The T0 model introduces the following fundamental mathematical structures:
	
	\begin{align}
		\text{Time-Mass Duality}: \quad &T(x,t) \cdot m(x,t) = 1 \label{eq:duality}\\
		\text{Field Equation}: \quad &\nabla^2 T(x) = -\frac{\rho(x)}{T(x)^2} \label{eq:field}\\
		\text{Energy Evolution}: \quad &\frac{\partial^2 E}{\partial t^2} = -\omega^2 E \label{eq:evolution}
	\end{align}
	
	The coupling parameter $\xi$ is theoretically derived from Higgs field interactions:
	\begin{equation}
		\xi = g_H \cdot \frac{\langle\phi\rangle}{v_{EW}} \label{eq:xi_higgs}
	\end{equation}
	where $g_H$ is the Higgs coupling constant, $\langle\phi\rangle$ is the vacuum expectation value, and $v_{EW} = 246$ GeV is the electroweak scale.
	
	\section{Mathematical Foundations}
	
	\subsection{Wave-Like Behavior of T0-Fields}
	
	The T0-field exhibits wave-like propagation characteristics analogous to acoustic waves in media. The fundamental wave equation for T0-fields is:
	
	\begin{equation}
		\nabla^2 T - \frac{1}{c_{T0}^2} \frac{\partial^2 T}{\partial t^2} = -\frac{\rho(x,t)}{T(x,t)^2} \label{eq:wave_equation}
	\end{equation}
	
	where $c_{T0}$ is the T0-field propagation velocity in the medium, analogous to sound velocity.
	
	\subsection{Medium-Dependent Properties}
	
	Similar to acoustic waves, T0-field propagation depends critically on medium properties:
	
	\textbf{T0-field velocity in different media}:
	\begin{align}
		c_{T0,vacuum} &= c \sqrt{\frac{\xi_0}{\xi_{vacuum}}} \\
		c_{T0,metal} &= c \sqrt{\frac{\xi_0 \epsilon_r}{\xi_{vacuum}}} \\
		c_{T0,dielectric} &= \frac{c}{\sqrt{\epsilon_r \mu_r}} \sqrt{\frac{\xi_0}{\xi_{vacuum}}} \\
		c_{T0,plasma} &= c \sqrt{1 - \frac{\omega_p^2}{\omega^2}} \sqrt{\frac{\xi_0}{\xi_{vacuum}}}
	\end{align}
	
	where $\omega_p$ is the plasma frequency and $\epsilon_r$, $\mu_r$ are relative permittivity and permeability.
	
	\subsection{Boundary Conditions and Reflections}
	
	At interfaces between different media, T0-fields satisfy boundary conditions similar to electromagnetic waves:
	
	\textbf{Continuity conditions}:
	\begin{align}
		T_1|_{interface} &= T_2|_{interface} \quad \text{(field continuity)} \\
		\frac{1}{m_1} \frac{\partial T_1}{\partial n}\bigg|_{interface} &= \frac{1}{m_2} \frac{\partial T_2}{\partial n}\bigg|_{interface} \quad \text{(flux continuity)}
	\end{align}
	
	\textbf{Reflection and transmission coefficients}:
	\begin{align}
		r &= \frac{Z_1 - Z_2}{Z_1 + Z_2} \quad \text{(reflection coefficient)} \\
		t &= \frac{2Z_1}{Z_1 + Z_2} \quad \text{(transmission coefficient)}
	\end{align}
	
	where $Z_i = \sqrt{m_i/T_i}$ is the T0-field impedance in medium $i$.
	
	\subsection{Geometric Constraints and Cavity Resonances}
	
	In bounded geometries, T0-fields form standing wave patterns with discrete eigenfrequencies:
	
	\textbf{Rectangular cavity} ($L_x \times L_y \times L_z$):
	\begin{equation}
		f_{mnp} = \frac{c_{T0}}{2} \sqrt{\left(\frac{m}{L_x}\right)^2 + \left(\frac{n}{L_y}\right)^2 + \left(\frac{p}{L_z}\right)^2}
	\end{equation}
	
	\textbf{Cylindrical cavity} (radius $a$, height $h$):
	\begin{equation}
		f_{mnp} = \frac{c_{T0}}{2\pi} \sqrt{\left(\frac{\chi_{mn}}{a}\right)^2 + \left(\frac{p\pi}{h}\right)^2}
	\end{equation}
	
	where $\chi_{mn}$ are zeros of Bessel functions.
	
	\textbf{Spherical cavity} (radius $R$):
	\begin{equation}
		f_{nlm} = \frac{c_{T0}}{2\pi R} \sqrt{n(n+1)}
	\end{equation}
	
	\subsection{Dispersion Relations}
	
	In dispersive media, the T0-field exhibits frequency-dependent propagation:
	
	\begin{equation}
		\omega^2 = c_{T0}^2(\omega) k^2 + \omega_0^2
	\end{equation}
	
	where $\omega_0$ is a characteristic frequency related to the medium's microscopic structure.
	
	\textbf{Group velocity} (important for information propagation):
	\begin{equation}
		v_g = \frac{d\omega}{dk} = \frac{c_{T0}^2 k}{\omega} + \frac{dc_{T0}^2}{d\omega} \frac{k^2}{2}
	\end{equation}
	
	\subsection{Hyperbolical Geometry in Duality Space}
	
	The time-mass duality (Eq.~\ref{eq:duality}) defines a hyperbolic metric in the $(T,m)$ parameter space:
	
	\begin{equation}
		ds^2 = \frac{dT \cdot dm}{T \cdot m} = \frac{d(\ln T) \cdot d(\ln m)}{T \cdot m}
	\end{equation}
	
	This geometry is characterized by:
	\begin{itemize}
		\item Constant negative curvature: $K = -1$
		\item Invariant measure: $d\mu = \frac{dT \, dm}{T \cdot m}$
		\item Isometry group: $PSL(2,\mathbb{R})$
	\end{itemize}
	
	\subsection{Field Equation Analysis}
	
	For spherically symmetric configurations, Eq.~\ref{eq:field} reduces to:
	\begin{equation}
		\frac{1}{r^2}\frac{d}{dr}\left(r^2 \frac{dT}{dr}\right) = -\frac{\rho(r)}{T(r)^2}
	\end{equation}
	
	For a point mass $m$ at the origin with $\rho(r) = mc^2 \delta(r)$, the solution is:
	\begin{equation}
		T(r) = T_0 \left(1 - \frac{r_0}{r}\right) \quad \text{with} \quad r_0 = \frac{Gm}{c^2}
	\end{equation}
	
	where $T_0 = \hbar/(mc^2)$ and $r_0$ corresponds to the Schwarzschild radius.
	
	\section{T0-Shor Algorithm Formulation}
	
	\subsection{Geometric Cavity Design for Period Finding}
	
	The T0-Shor algorithm utilizes geometric resonance cavities to detect periods, analogous to acoustic resonators:
	
	\textbf{Resonance cavity dimensions} for period $r$:
	\begin{equation}
		L_{cavity} = n \cdot \frac{\lambda_{T0}}{2} = n \cdot \frac{c_{T0} \cdot r}{2f_0}
	\end{equation}
	
	where $f_0$ is the fundamental driving frequency and $n$ is the mode number.
	
	\textbf{Quality factor} of the resonance:
	\begin{equation}
		Q = \frac{f_r}{\Delta f} = \frac{\pi}{\xi} \cdot \frac{L_{cavity}}{\lambda_{T0}}
	\end{equation}
	
	Higher $Q$ values provide sharper period detection but require longer observation times.
	
	\subsection{Medium-Dependent Algorithm Optimization}
	
	The algorithm efficiency depends critically on the propagation medium:
	
	\textbf{Metallic substrates}:
	\begin{align}
		c_{T0,metal} &= c \sqrt{\frac{\xi_0}{\xi_0 + \sigma/(\omega \epsilon_0)}} \\
		\text{Skin depth: } \delta &= \sqrt{\frac{2}{\omega \mu_0 \sigma}} \\
		\text{Effective cavity size: } L_{eff} &= \min(L_{cavity}, \delta)
	\end{align}
	
	\textbf{Dielectric materials}:
	\begin{align}
		c_{T0,dielectric} &= \frac{c}{\sqrt{\epsilon_r}} \sqrt{\frac{\xi_0}{\xi_{vacuum}}} \\
		\text{Penetration depth: } \delta_p &= \frac{c}{\omega \sqrt{\epsilon_r}} \text{Im}(\sqrt{\epsilon_r}) \\
		\text{Loss tangent: } \tan \delta &= \frac{\epsilon''}{\epsilon'}
	\end{align}
	
	\subsection{Boundary Condition Engineering}
	
	Strategic boundary condition design enhances period detection:
	
	\textbf{Perfect conductor boundaries}:
	\begin{equation}
		T|_{boundary} = 0 \quad \text{(hard boundary)}
	\end{equation}
	
	\textbf{Absorbing boundaries}:
	\begin{equation}
		\frac{\partial T}{\partial n} + i\frac{\omega}{c_{T0}} T = 0 \quad \text{(radiation boundary)}
	\end{equation}
	
	\textbf{Periodic boundaries} for resonance enhancement:
	\begin{equation}
		T(x + L, y, z, t) = T(x, y, z, t) \cdot e^{i k_x L}
	\end{equation}
	
	\subsection{Multi-Mode Resonance Analysis}
	
	Instead of quantum Fourier transform, the T0-Shor algorithm uses multi-mode cavity analysis:
	
	\begin{align}
		\text{Mode spectrum}: \quad &T(x,y,z,t) = \sum_{mnp} A_{mnp}(t) \psi_{mnp}(x,y,z) \\
		\text{Period detection}: \quad &r = \frac{c_{T0}}{2f_{resonance}} \cdot \frac{geometry\_factor}{mode\_number}
	\end{align}
	
	\textbf{Geometry factors for different cavity shapes}:
	\begin{align}
		\text{Rectangular: } G_{rect} &= \sqrt{(m/L_x)^2 + (n/L_y)^2 + (p/L_z)^2} \\
		\text{Cylindrical: } G_{cyl} &= \sqrt{(\chi_{mn}/a)^2 + (p\pi/h)^2} \\
		\text{Spherical: } G_{sph} &= \sqrt{n(n+1)}/R
	\end{align}
	
	\subsection{Adaptive Impedance Matching}
	
	For optimal energy transfer and period detection:
	
	\begin{equation}
		Z_{optimal} = \sqrt{\frac{Z_{source} \cdot Z_{cavity}}{1 + (Q \cdot \Delta f / f_0)^2}}
	\end{equation}
	
	The matching network adjusts the effective mass field distribution:
	\begin{equation}
		m_{matched}(r) = m_0(r) \cdot \frac{Z_{optimal}(r)}{Z_0}
	\end{equation}
	
	\section{Physical Implementation Considerations}
	
	\subsection{Substrate Material Selection}
	
	Different substrate materials provide different T0-field characteristics:
	
	
% TABLE CONVERTED TO LIST FORMAT FOR KDP COMPLIANCE
% Original table was too complex (many columns/rows)

\begin{itemize}
    \item Vacuum -- 1.0 -- 1.0 -- 1.0 -- 1.0 -- Reference
    \item Silicon -- 11.9 -- 1.0 -- 0.29 -- 0.84 -- Electronics
    \item Sapphire -- 9.4 -- 1.0 -- 0.33 -- 0.87 -- High-Q resonators
    \item GaAs -- 12.9 -- 1.0 -- 0.28 -- 0.83 -- High-speed devices
    \item Superconductor -- $\infty$ -- 0 -- 0 -- $\Delta/(k_B T_c)$ -- Lossless cavities
    \item Metamaterial -- $< 0$ -- $< 0$ -- $> 1$ -- Tunable -- Engineered properties
    \item \text{Length: } L -- = (2n+1) \frac{c_{T0} r}{4 f_0} \quad \text{(quarter-wave resonator)}
    \item \text{Width: } W -- = \frac{c_{T0}}{2 f_0} \sqrt{1 - (f_0/f_{cutoff})^2}
    \item \text{Height: } H -- = \frac{c_{T0}}{2 f_0} \sqrt{1 - (f_0/f_{cutoff})^2}
    \item \textbf{Resource} -- \textbf{Standard Shor} -- \textbf{T0-Shor}
    \item Quantum bits -- $2n + O(\log n)$ -- 0
    \item Energy fields -- 0 -- $2n$
    \item Field operations -- $O(n^3)$ -- $O(n^{2.5})$
    \item Memory (bits) -- $O(n)$ -- $O(n)$
    \item Success probability -- $\approx 0.5$ -- 1.0 (theoretical)
    \item \gamma -- \approx 0 \quad \text{for } E < \Lambda_{QCD}
    \item \gamma -- \approx 1/2 \quad \text{for } \Lambda_{QCD} < E < \Lambda_{EW}
    \item \gamma -- \approx -1/4 \quad \text{for } E > \Lambda_{EW}
    \item \xi_{metal} -- = \xi_0 / \sqrt{N(E_F)}
    \item \xi_{SC} -- = \xi_0 \cdot \Delta/(k_B T_c)
    \item \xi_{semi} -- = \xi_0 / \sqrt{m_{eff}/m_e}
    \item m(x) -- \rightarrow \lambda^{-d} m(x/\lambda)
    \item T(x) -- \rightarrow \lambda^d T(x/\lambda)
    \item E(x) -- \rightarrow \lambda^{d/2} E(x/\lambda)
    \item \textbf{Method} -- \textbf{Operations} -- \textbf{Memory} -- \textbf{Success Rate}
    \item Trial Division -- $O(\sqrt{N})$ -- $O(1)$ -- 1.0
    \item Pollard's $\rho$ -- $O(N^{1/4})$ -- $O(1)$ -- High
    \item Quadratic Sieve -- $O(\exp(\sqrt{\log N \log \log N}))$ -- $O(\sqrt{N})$ -- High
    \item General Number Field Sieve -- $O(\exp((\log N)^{1/3}(\log \log N)^{2/3}))$ -- $O(\exp(\sqrt{\log N}))$ -- High
    \item Standard Shor -- $O((\log N)^3)$ -- $O(\log N)$ -- $\approx 0.5$
    \item T0-Shor (theoretical) -- $O((\log N)^{2.5} / F(m))$ -- $O(\log N)$ -- 1.0
    \item Gelfand, I. M., \& Fomin, S. V. (1963). \textit{Calculus of variations}. Prentice-Hall.
    \item Lenstra, A. K., \& Lenstra Jr, H. W. (Eds.). (1993). \textit{The development of the number field sieve}. Springer-Verlag.
    \item Nielsen, M. A., \& Chuang, I. L. (2010). \textit{Quantum computation and quantum information}. Cambridge University Press.
    \item Cover, T. M., \& Thomas, J. A. (2012). \textit{Elements of information theory}. John Wiley \& Sons.
\end{itemize}
