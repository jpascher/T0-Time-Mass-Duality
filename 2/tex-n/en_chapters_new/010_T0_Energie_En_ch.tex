\chapter{The T0 Energy Field Model:\\[0.3cm]
	 Mathematical Formulation}

	
	
\section*{Abstract}
		The T0 model describes physical phenomena through a universal energy field $E_{\text{field}}(x,t)$ with the parameter $\xi = \frac{4}{3} \times 10^{-4}$. The field equation is $\square E_{\text{field}} = 0$, the Lagrangian density $\mathcal{L} = \xi (\partial E)^2$. The model uses standard natural units with $\hbar = c = 1$.
		
		\textbf{Fundamental quantities:}
		\begin{itemize}
			\item Characteristic energy: $E_0 = \sqrt{m_e \cdot m_\mu} = 7.348$ MeV
			\item Fine structure constant: $\alpha = \xi(E_0/1\,\text{MeV})^2 \approx 1/137$
			\item Gravitational constant: $G = \xi^2/(4m_e) \times$ factors
		\end{itemize}
		
		\textbf{Predictions:} Lepton masses with 2\% accuracy, anomalous magnetic moments $a_\ell = \frac{\xi}{2\pi}(E_\ell/E_e)^2$, fine structure constant with 0.03\% agreement.
		
		\textbf{Detailed derivations:} See Document 011 (fine structure), 012 (gravitation), 018 (g-2 geometric), 019 (Lagrangian).

	
	
	
	% ============================================================================
	\section{Units Convention}
	\label{sec:units}
	
	\subsection{T0 natural units (Heaviside-Lorentz)}
	
	T0 uses Heaviside-Lorentz natural units:
	
	\begin{equation}
		\hbar = c = 4\pi\varepsilon_0 = 1
	\end{equation}
	
	In this system:
	\begin{itemize}
		\item Fine structure constant: $\alpha = e^2 = 1$ (by convention in T0)
		\item Energy = Mass: $E = m$
		\item Length = Time = Energy$^{-1}$: $[L] = [T] = [E^{-1}]$
	\end{itemize}
	
	\textbf{Note:} In SI units, $\alpha \approx 1/137$. The choice $\alpha = 1$ in T0 is a unit convention that simplifies formulas.
	
	\subsection{Dimensions in natural units}
	
	\begin{align}
		[E] &= E \\
		[m] &= E \\
		[t] &= E^{-1} \\
		[L] &= E^{-1} \\
		[G] &= E^{-2} \\
		[\partial_\mu] &= E
	\end{align}
	
	% ============================================================================
	\section{Time-Energy Duality}
	\label{sec:duality}
	
	\subsection{Fundamental relation}
	
	\begin{equation}
		T_{\text{field}}(x,t) \cdot E_{\text{field}}(x,t) = 1
		\label{eq:duality}
	\end{equation}
	
	with $[T_{\text{field}}] = E^{-1}$ and $[E_{\text{field}}] = E$.
	
	\subsection{Intrinsic time field}
	
	\begin{equation}
		T_{\text{field}}(x,t) = \frac{1}{E_{\text{field}}(x,t)}
	\end{equation}
	
	% ============================================================================
	\section{Universal Field Equation}
	\label{sec:field_equation}
	
	\subsection{Wave equation}
	
	\begin{equation}
		\square E_{\text{field}} = 0
	\end{equation}
	
	with d'Alembert operator:
	\begin{equation}
		\square = \nabla^2 - \frac{\partial^2}{\partial t^2}
	\end{equation}
	
	\subsection{With sources}
	
	\begin{equation}
		\nabla^2 E_{\text{field}} = 4\pi G \rho \cdot E_{\text{field}}
	\end{equation}
	
	Dimension check: $[E^3] = [E^{-2}][E^4][E] = [E^3]$ ✓
	
	% ============================================================================
	\section{Lagrangian Density}
	\label{sec:lagrangian}
	
	\subsection{Universal Lagrangian density}
	
	\begin{equation}
		\mathcal{L} = \xi \cdot (\partial_\mu E_{\text{field}})(\partial^\mu E_{\text{field}})
	\end{equation}
	
	with $\xi = \frac{4}{3} \times 10^{-4}$.
	
	\subsection{Euler-Lagrange equation}
	
	\begin{equation}
		\frac{\partial \mathcal{L}}{\partial E} - \partial_\mu \frac{\partial \mathcal{L}}{\partial(\partial_\mu E)} = 0
	\end{equation}
	
	yields:
	\begin{equation}
		\square E_{\text{field}} = 0
	\end{equation}
	
	% ============================================================================
	\section{Characteristic Energy}
	\label{sec:characteristic_energy}
	
	\subsection{Definition}
	
	The characteristic energy $E_0$ is the geometric mean of electron and muon mass (derivation in Document 011):
	
	\begin{equation}
		\boxed{E_0 = \sqrt{m_e \cdot m_\mu}}
		\label{eq:E0_definition}
	\end{equation}
	
	\subsection{Numerical values}
	
	From experimental masses:
	\begin{align}
		E_0 &= \sqrt{0.511 \times 105.66} \\
		&= \sqrt{53.99} \\
		&= 7.348 \text{ MeV}
	\end{align}
	
	Theoretical T0 value:
	\begin{equation}
		E_0^{\text{T0}} = 7.398 \text{ MeV}
	\end{equation}
	
	Deviation: 0.7\% (within geometric corrections)
	
	\subsection{Usage}
	
	$E_0$ serves as energy scale for:
	\begin{itemize}
		\item Fine structure constant: $\alpha = \xi (E_0/1\,\text{MeV})^2$
		\item Normalization of electromagnetic effects
		\item Scaling of anomalous magnetic moments
	\end{itemize}
	
	% ============================================================================
	\section{The Parameter $\xi$}
	\label{sec:xi_parameter}
	
	\subsection{Definition}
	
	\begin{equation}
		\boxed{\xi = \frac{4}{3} \times 10^{-4} = 1.3333 \times 10^{-4}}
	\end{equation}
	
	Dimensionless: $[\xi] = 1$.
	
	\subsection{Geometric components}
	
	\begin{equation}
		\xi = G_3 \times S_{\text{ratio}}
	\end{equation}
	
	where:
	\begin{itemize}
		\item $G_3 = \frac{4}{3}$: Geometric factor (sphere-cube ratio)
		\item $S_{\text{ratio}} = 10^{-4}$: Scale ratio
	\end{itemize}
	
	% ============================================================================
	\section{Fine Structure Constant}
	\label{sec:fine_structure}
	
	\subsection{In T0 units}
	
	In T0 natural units: $\alpha = 1$ (by convention)
	
	\subsection{Reconstruction of SI value}
	
	\textbf{Important:} The SI value can be reconstructed from T0 parameters!
	
	\begin{equation}
		\boxed{\alpha_{\text{SI}} = \xi \cdot \left(\frac{E_0}{1\,\text{MeV}}\right)^2}
		\label{eq:alpha_derivation}
	\end{equation}
	
	\subsection{Numerical calculation}
	
	With $\xi = \frac{4}{3} \times 10^{-4}$ and $E_0 = 7.398$ MeV:
	
	\begin{align}
		\alpha_{\text{SI}} &= 1.3333 \times 10^{-4} \times (7.398)^2 \\
		&= 1.3333 \times 10^{-4} \times 54.73 \\
		&= 7.297 \times 10^{-3} \\
		&= \frac{1}{137.04}
	\end{align}
	
	Experimental: $\alpha_{\text{exp}} = \frac{1}{137.036}$
	
	Agreement: 0.03\%
	
	\subsection{Dimension check}
	
	\begin{equation}
		[\alpha_{\text{SI}}] = [\xi] \times \left[\frac{E}{E}\right]^2 = 1 \times 1 = 1 \quad \checkmark
	\end{equation}
	
	% ============================================================================
	\section{Gravitational Constant}
	\label{sec:gravitational_constant}
	
	\subsection{T0 formula}
	
	The gravitational constant is derived from $\xi$ and $m_e$ (derivation in Document 012):
	
	\begin{equation}
		G = \frac{\xi^2}{4m_e} \times C_{\text{dim}} \times C_{\text{conv}}
		\label{eq:G_formula}
	\end{equation}
	
	where:
	\begin{itemize}
		\item $C_{\text{dim}}$: Dimension correction
		\item $C_{\text{conv}}$: SI conversion factor
	\end{itemize}
	
	\subsection{Fundamental relation}
	
	In natural units:
	\begin{equation}
		\xi = 2\sqrt{G \cdot m_e}
	\end{equation}
	
	Solved for $G$:
	\begin{equation}
		G_{\text{nat}} = \frac{\xi^2}{4m_e}
	\end{equation}
	
	Dimension: $[G] = [E^{-2}]$ in natural units.
	
	% ============================================================================
	\section{Characteristic Lengths}
	\label{sec:characteristic_lengths}
	
	\subsection{T0 characteristic length}
	
	\begin{equation}
		r_0 = 2GE
	\end{equation}
	
	Dimension: $[r_0] = [E^{-2}][E] = [E^{-1}] = [L]$ ✓
	
	\subsection{Derivation}
	
	For spherically symmetric point source $\rho(r) = E_0 \delta^3(\vec{r})$:
	
	Solution of $\nabla^2 E = 4\pi G \rho E$:
	\begin{equation}
		E(r) = E_0 \left(1 - \frac{r_0}{r}\right)
	\end{equation}
	
	with $r_0 = 2GE_0$.
	
	\subsection{Time scale}
	
	\begin{equation}
		t_0 = \frac{r_0}{c} = r_0 = 2GE
	\end{equation}
	
	(since $c = 1$)
	
	% ============================================================================
	\section{Scale Hierarchy}
	\label{sec:scale_hierarchy}
	
	\subsection{Planck length as reference}
	
	\begin{equation}
		\ell_P = \sqrt{G} = 1 \quad \text{(in nat. units)}
	\end{equation}
	
	\subsection{Scale ratio}
	
	\begin{equation}
		\xi_{\text{ratio}} = \frac{\ell_P}{r_0} = \frac{\sqrt{G}}{2GE} = \frac{1}{2\sqrt{G} \cdot E}
	\end{equation}
	
	For $E \sim 1$ GeV:
	\begin{equation}
		\frac{r_0}{\ell_P} \sim 10^7 \quad \text{(sub-Planck)}
	\end{equation}
	
	% ============================================================================
	\section{Particles as Field Excitations}
	\label{sec:particles}
	
	\subsection{Classification by energy}
	
	\begin{table}[h]
		\centering
		\begin{tabular}{lc}
			\hline
			\textbf{Particle} & \textbf{Energy [MeV]} \\
			\hline
			Electron & 0.511 \\
			Muon & 105.658 \\
			Tau & 1776.86 \\
			\hline
		\end{tabular}
	\end{table}
	
	\subsection{Antiparticles}
	
	Negative field excitations: $E_{\text{field}} < 0$
	
	% ============================================================================
	\section{Lepton Masses}
	\label{sec:lepton_masses}
	
	The T0 model predicts lepton masses (derivation in Document 003):
	
	\begin{table}[h]
		\centering
		\begin{tabular}{lccc}
			\hline
			\textbf{Lepton} & \textbf{T0 [MeV]} & \textbf{Exp [MeV]} & \textbf{Δ [\%]} \\
			\hline
			Electron & 0.507 & 0.511 & 0.87 \\
			Muon & 103.5 & 105.7 & 2.09 \\
			Tau & 1815 & 1777 & 2.16 \\
			\hline
		\end{tabular}
	\end{table}
	
	% ============================================================================
	\section{Anomalous Magnetic Moments}
	\label{sec:g2}
	
	\subsection{Definition}
	
	Magnetic moment:
	\begin{equation}
		\mu = g \cdot \frac{e}{2m} \cdot \frac{\hbar}{2}
	\end{equation}
	
	Anomalous magnetic moment:
	\begin{equation}
		a = \frac{g-2}{2}
	\end{equation}
	
	\subsection{T0 prediction formula}
	
	\begin{equation}
		\boxed{a_\ell = \frac{\xi}{2\pi} \left(\frac{E_\ell}{E_e}\right)^2}
		\label{eq:g2_formula}
	\end{equation}
	
	\subsection{Muon}
	
	\begin{align}
		\frac{E_\mu}{E_e} &= \frac{105.658}{0.511} = 206.768 \\
		a_\mu &= \frac{1.3333 \times 10^{-4}}{2\pi} \times (206.768)^2 \\
		&= 2.122 \times 10^{-5} \times 42{,}753 \\
		&= 1.166 \times 10^{-3}
	\end{align}
	
	\subsection{Electron}
	
	\begin{equation}
		a_e = \frac{\xi}{2\pi} = 2.122 \times 10^{-5}
	\end{equation}
	
	\subsection{Tau}
	
	\begin{equation}
		a_\tau = \frac{\xi}{2\pi} \left(\frac{1776.86}{0.511}\right)^2 = 1.28 \times 10^{-3}
	\end{equation}
	
	% ============================================================================
	\section{Three Field Geometries}
	\label{sec:field_geometries}
	
	\subsection{Type 1: Localized spherical}
	
	\begin{equation}
		E(r) = E_0 \left(1 - \frac{\beta}{r}\right), \quad \beta = r_0
	\end{equation}
	
	Application: Individual particles (electron, muon, tau)
	
	\subsection{Type 2: Localized non-spherical}
	
	\begin{equation}
		E(\vec{r}) = E_0 \left(1 - \frac{\beta_{ij} r_i r_j}{r^3}\right)
	\end{equation}
	
	Application: Bound systems
	
	\subsection{Type 3: Extended homogeneous}
	
	Effective parameter:
	\begin{equation}
		\xi_{\text{eff}} = \frac{\xi}{2} = \frac{2}{3} \times 10^{-4}
	\end{equation}
	
	Application: Cosmology (see Document 026)
	
	% ============================================================================
	\section{Mathematical Identities}
	\label{sec:identities}
	
	\subsection{Energy field normalization}
	
	\begin{equation}
		E_{\text{field}}(\vec{r}, t) = E_0 \cdot f(\vec{r}, t) \cdot e^{i\phi(\vec{r}, t)}
	\end{equation}
	
	with:
	\begin{itemize}
		\item $E_0$: Characteristic energy
		\item $f(\vec{r}, t)$: Normalized profile
		\item $\phi(\vec{r}, t)$: Phase
	\end{itemize}
	
	\subsection{Duality consistency}
	
	Time-mass (Document 003): $T \cdot m = 1$
	
	Time-energy (this document): $T \cdot E = 1$
	
	In natural units ($c = 1$):
	\begin{equation}
		E = mc^2 = m \quad \Rightarrow \quad T \cdot m = T \cdot E
	\end{equation}
	
	% ============================================================================
	\section{Dimensional Analysis Verifications}
	\label{sec:dimensional_analysis}
	
	\subsection{Field equation}
	
	\begin{align}
		[\nabla^2 E] &= [L^{-2}][E] = [E^2][E] = [E^3] \\
		[4\pi G \rho E] &= [E^{-2}][E^4][E] = [E^3] \quad \checkmark
	\end{align}
	
	\subsection{Characteristic length}
	
	\begin{equation}
		[r_0] = [2GE] = [E^{-2}][E] = [E^{-1}] = [L] \quad \checkmark
	\end{equation}
	
	\subsection{Lagrangian density}
	
	\begin{equation}
		[\mathcal{L}] = [\xi][(\partial E)^2] = [1][E^2] = [E^2] \quad \text{(correct for Lagrangian density)}
	\end{equation}
	
	\subsection{Anomalous magnetic moment}
	
	\begin{equation}
		[a_\ell] = [\xi]\left[\frac{E^2}{E^2}\right] = [1][1] = [1] \quad \checkmark
	\end{equation}
	
	% ============================================================================
	\section{Formula Reference}
	\label{sec:formula_reference}
	
	\subsection{Fundamental equations}
	
	\begin{align}
		\text{Duality:} \quad & T_{\text{field}} \cdot E_{\text{field}} = 1 \\
		\text{Wave equation:} \quad & \square E_{\text{field}} = 0 \\
		\text{With sources:} \quad & \nabla^2 E = 4\pi G \rho E \\
		\text{Lagrangian density:} \quad & \mathcal{L} = \xi (\partial E)^2
	\end{align}
	
	\subsection{Derived constants}
	
	\begin{align}
		\text{Characteristic energy:} \quad & E_0 = \sqrt{m_e \cdot m_\mu} = 7.348 \text{ MeV} \\
		\text{Fine structure constant:} \quad & \alpha = \xi (E_0/1\,\text{MeV})^2 \approx 1/137 \\
		\text{Gravitational constant:} \quad & G = \frac{\xi^2}{4m_e} \times \text{factors}
	\end{align}
	
	\subsection{Characteristic scales}
	
	\begin{align}
		\text{T0 length:} \quad & r_0 = 2GE \\
		\text{T0 time:} \quad & t_0 = 2GE \\
		\text{Planck length:} \quad & \ell_P = \sqrt{G} = 1 \\
		\text{Scale ratio:} \quad & \xi_{\text{ratio}} = \frac{1}{2\sqrt{G} E}
	\end{align}
	
	\subsection{Prediction formulas}
	
	\begin{align}
		\text{g-2 formula:} \quad & a_\ell = \frac{\xi}{2\pi} \left(\frac{E_\ell}{E_e}\right)^2 \\
		\text{Parameter:} \quad & \xi = \frac{4}{3} \times 10^{-4} \\
		\text{Effective parameter:} \quad & \xi_{\text{eff}} = \frac{\xi}{2}
	\end{align}
	
	% ============================================================================
	\section{Numerical Values}
	\label{sec:numerical_values}
	
	\subsection{Fundamental constants (in natural units)}
	
	\begin{align}
		\hbar &= 1 \\
		c &= 1 \\
		\alpha &= \frac{1}{137.036} \approx 7.297 \times 10^{-3} \\
		G &= 1 \text{ (numerically, dimension } [E^{-2}]\text{)}
	\end{align}
	
	\subsection{T0 parameters}
	
	\begin{align}
		\xi &= \frac{4}{3} \times 10^{-4} = 1.3333 \times 10^{-4} \\
		\xi^2 &= 1.7778 \times 10^{-8} \\
		\frac{\xi}{2\pi} &= 2.1221 \times 10^{-5} \\
		\xi_{\text{eff}} &= 6.6667 \times 10^{-5} \\
		E_0 &= 7.348 \text{ MeV (from exp. masses)} \\
		E_0^{\text{T0}} &= 7.398 \text{ MeV (theoretical)}
	\end{align}
	
	\subsection{Lepton energies}
	
	\begin{align}
		E_e &= 0.511 \text{ MeV} \\
		E_\mu &= 105.658 \text{ MeV} \\
		E_\tau &= 1776.86 \text{ MeV}
	\end{align}
	
	\subsection{Energy ratios}
	
	\begin{align}
		\frac{E_\mu}{E_e} &= 206.768 \\
		\frac{E_\tau}{E_e} &= 3477.2 \\
		\frac{E_\tau}{E_\mu} &= 16.817
	\end{align}
	
	% ============================================================================
	\section{Calculation Examples}
	\label{sec:calculations}
	
	\subsection{Muon g-2}
	
	Given:
	\begin{itemize}
		\item $\xi = 1.3333 \times 10^{-4}$
		\item $E_\mu = 105.658$ MeV
		\item $E_e = 0.511$ MeV
	\end{itemize}
	
	Calculation:
	\begin{align}
		\frac{E_\mu}{E_e} &= \frac{105.658}{0.511} = 206.768 \\
		\left(\frac{E_\mu}{E_e}\right)^2 &= 42{,}753.3 \\
		\frac{\xi}{2\pi} &= \frac{1.3333 \times 10^{-4}}{6.2832} = 2.1221 \times 10^{-5} \\
		a_\mu &= 2.1221 \times 10^{-5} \times 42{,}753.3 \\
		&= 1.1659 \times 10^{-3}
	\end{align}
	
	\subsection{Fine structure constant}
	
	Given:
	\begin{itemize}
		\item $\xi = 1.3333 \times 10^{-4}$
		\item $E_0 = 7.398$ MeV
	\end{itemize}
	
	Calculation:
	\begin{align}
		\left(\frac{E_0}{1\,\text{MeV}}\right)^2 &= (7.398)^2 = 54.73 \\
		\alpha &= 1.3333 \times 10^{-4} \times 54.73 \\
		&= 7.297 \times 10^{-3} \\
		&= \frac{1}{137.04}
	\end{align}
	
	Experimental: $\alpha_{\text{exp}} = \frac{1}{137.036}$
	
	Deviation: 0.03\%
	
	\subsection{Characteristic length (electron)}
	
	Given:
	\begin{itemize}
		\item $E_e = 0.511$ MeV $= 0.511 \times 1.6 \times 10^{-13}$ J $= 8.2 \times 10^{-14}$ J
		\item $G = 6.674 \times 10^{-11}$ m$^3$ kg$^{-1}$ s$^{-2}$
		\item $c = 3 \times 10^8$ m/s
	\end{itemize}
	
	Conversion to natural units:
	\begin{equation}
		r_0 = 2GE \approx 10^{-28} \text{ m}
	\end{equation}
	
	Planck comparison:
	\begin{equation}
		\frac{r_0}{\ell_P} = \frac{10^{-28}}{1.6 \times 10^{-35}} \approx 10^7
	\end{equation}
	
	% ============================================================================
	% APPENDIX
	% ============================================================================
	
	\appendix
	
	\section{Symbol Reference}
	
	\begin{longtable}{|c|l|c|}
		\hline
		\textbf{Symbol} & \textbf{Meaning} & \textbf{Dimension} \\
		\hline
		$\xi$ & Fundamental parameter & $1$ \\
		$E_0$ & Characteristic energy & $E$ \\
		$E_{\text{field}}$ & Universal energy field & $E$ \\
		$T_{\text{field}}$ & Intrinsic time field & $E^{-1}$ \\
		$r_0$ & T0 characteristic length & $L = E^{-1}$ \\
		$t_0$ & T0 characteristic time & $T = E^{-1}$ \\
		$\ell_P$ & Planck length & $L = E^{-1}$ \\
		$G$ & Gravitational constant & $E^{-2}$ \\
		$\alpha$ & Fine structure constant & $1$ \\
		$a_\ell$ & Anomalous magnetic moment & $1$ \\
		$E_e, E_\mu, E_\tau$ & Lepton energies & $E$ \\
		$m_e, m_\mu, m_\tau$ & Lepton masses ($= E$ in nat. units) & $E$ \\
		$\mathcal{L}$ & Lagrangian density & $E^4$ \\
		$\square$ & d'Alembert operator & $E^2$ \\
		$\xi_{\text{eff}}$ & Effective parameter ($\xi/2$) & $1$ \\
		\hline
	\end{longtable}
	
	\section{Unit Conversions}
	
	\subsection{Natural → SI}
	
	\begin{align}
		1 \text{ (Energy)} &= 1 \text{ GeV} = 1.6 \times 10^{-10} \text{ J} \\
		1 \text{ (Length)} &= \frac{\hbar c}{1 \text{ GeV}} = 0.197 \text{ fm} \\
		1 \text{ (Time)} &= \frac{\hbar}{1 \text{ GeV}} = 6.58 \times 10^{-25} \text{ s}
	\end{align}
	
	\subsection{Standard natural units}
	
	In standard convention ($\hbar = c = 1$):
	\begin{itemize}
		\item $\alpha = \frac{e^2}{4\pi\epsilon_0} \approx \frac{1}{137}$ (dimensionless)
		\item All quantities in powers of energy
		\item Physical predictions identical to other conventions
	\end{itemize}
	
	\section{Relations to Other Documents}
	
	\begin{itemize}
		\item \textbf{Document 003}: Time-mass duality, foundations, origin of $\xi$
		\item \textbf{Document 011}: Fine structure constant (geometric derivation)
		\item \textbf{Document 012}: Gravitational constant (systematic derivation)
		\item \textbf{Document 018}: Geometric g-2 formulation (fractal geometry)
		\item \textbf{Document 019}: Lagrangian formulation (quantum field theory)
		\item \textbf{Document 026}: Cosmology ($\xi_{\text{eff}} = \xi/2$)
	\end{itemize}
	
	All formulations are based on $\xi = \frac{4}{3} \times 10^{-4}$.
	