\chapter{\textbf{Unit Conventions and the Speed of Light c}\\[0.5cm]
	 E=mc² vs. E=m: Two Equivalent Perspectives\\[0.3cm]
	\normalsize Natural Units, SI Units, and the T0 Viewpoint}

	
	
\section*{Abstract}
		This document examines when one can set c=1 (natural units) and when the full form E=mc² with c=299,792,458 m/s (SI units) is required. Parallel to the treatment of the fine-structure constant α in Document 101, we show: Both perspectives are mathematically equivalent and differ only in the choice of unit system. The T0 theory reveals that c is not a fundamental law of nature but a dynamic ratio L/T. From the T0 perspective, c=1 can be set (Planck units, particle physics), while for technical applications and precision measurements, SI units with explicit c are required. The equivalence E=mc² ↔ E=m holds exactly in natural units. References: Documents 013 (SI system), 014 (nat./SI), 015 (systematics), 077 (E=mc² analysis), 101 (α conventions).

	
	
	
	\section{Introduction: The Question of c=1}
	
	\subsection{The Central Question}
	
	The question ``When can one set c=1?'' is analogous to the question ``When can one set α=1?'' addressed in Document 101. In both cases, it concerns \textbf{unit conventions}, not fundamental physics.
	
	\begin{tcolorbox}[colback=blue!5!white,colframe=blue!75!black,title=Central Thesis]
		\textbf{E=mc² and E=m are mathematically identical!}
		
		\begin{itemize}
			\item In SI units: $E = mc^2$ with $c = 299,792,458$ m/s
			\item In natural units: $E = m$ with $c = 1$
		\end{itemize}
		
		Both forms describe the same physics -- only the unit choice differs.
	\end{tcolorbox}
	
	\subsection{Historical Context}
	
	Einstein wrote the famous formula in 1905:
	\begin{equation}
		E = mc^2
	\end{equation}
	
	This form was necessary because he worked in \textbf{SI units}, where length (meter), time (second), and mass (kilogram) are independent dimensions.
	
	\textbf{Modern particle physics} uses instead:
	\begin{equation}
		E = m \quad \text{(in natural units with } c=\hbar=1\text{)}
	\end{equation}
	
	\section{Natural Units: When c=1 is Valid}
	
	\subsection{Definition of Natural Units}
	
	In natural units, one sets:
	\begin{equation}
		c = 1, \quad \hbar = 1, \quad \text{(optional: } k_B = 1 \text{)}
	\end{equation}
	
	\textbf{Mathematical meaning:}
	\begin{align}
		c = 1 \quad &\Rightarrow \quad \text{Length} \equiv \text{Time} \\
		\hbar = 1 \quad &\Rightarrow \quad \text{Energy} \equiv \text{inverse Time}
	\end{align}
	
	\subsection{Application Domains}
	
	\textbf{Natural units are appropriate in:}
	
	\begin{itemize}
		\item \textbf{Planck scale}: Quantum gravity, fundamental theory
		\item \textbf{Particle physics}: High-energy physics, QFT, Standard Model
		\item \textbf{Cosmology}: Early universe, inflationary models
		\item \textbf{Theoretical work}: Mathematical derivations, symmetries
	\end{itemize}
	
	\textbf{Advantage:} Formulas become simpler, physical relationships clearer.
	
	\subsection{Mathematical Consistency}
	
	In natural units:
	\begin{equation}
		E^2 = p^2 + m^2
	\end{equation}
	
	In the rest frame ($p=0$):
	\begin{equation}
		E = m
	\end{equation}
	
	This is exact -- \textbf{not an approximation}.
	
	\subsection{T0 Perspective: c as a Ratio}
	
	The T0 theory shows (see Document 077):
	\begin{equation}
		c = \frac{L}{T}
	\end{equation}
	
	\textbf{c is not a fundamental law of nature but a \emph{ratio}}!
	
	With the T0 fundamental relation:
	\begin{equation}
		T \cdot m = 1 \quad \text{(Time-Mass Duality)}
	\end{equation}
	
	it follows that c is a dynamic ratio that varies with mass scale.
	
	\textbf{Implication:} In Planck units, where $t_P = \ell_P/c$, c=1 is the natural choice.
	
	\section{SI Units: When c=299,792,458 m/s is Required}
	
	\subsection{The SI Definition (since 2019)}
	
	The modern SI system defines since 2019:
	\begin{equation}
		\boxed{c = 299,792,458 \text{ m/s} \text{ (exact)}}
	\end{equation}
	
	This choice is a \textbf{convention} that defines the meter via the second.
	
	\subsection{Application Domains}
	
	\textbf{SI units with explicit c are required in:}
	
	\begin{itemize}
		\item \textbf{Engineering}: GPS, telecommunications, laser technology
		\item \textbf{Precision measurements}: Atomic clocks, interferometry, metrology
		\item \textbf{Experimental physics}: Laboratory measurements with SI-calibrated devices
		\item \textbf{Applied physics}: Energy calculations, dosimetry
		\item \textbf{Public \& Education}: Comprehensibility, historical continuity
	\end{itemize}
	
	\textbf{Advantage:} Practical calculability with calibrated measurement devices.
	
	\subsection{Mathematical Form}
	
	In SI units:
	\begin{equation}
		E = \gamma m c^2
	\end{equation}
	
	with the Lorentz factor:
	\begin{equation}
		\gamma = \frac{1}{\sqrt{1 - v^2/c^2}}
	\end{equation}
	
	In the rest frame ($v=0$, $\gamma=1$):
	\begin{equation}
		E = mc^2
	\end{equation}
	
	\subsection{Conversion Between Unit Systems}
	
	\textbf{From natural units to SI:}
	
	\begin{align}
		E_{\text{nat}} &= m_{\text{nat}} \\
		\Downarrow \quad &\text{(Multiply by } c^2\text{)} \\
		E_{\text{SI}} &= m_{\text{SI}} \cdot c^2
	\end{align}
	
	\textbf{Example:} Electron mass
	\begin{align}
		m_e &= 0.511 \text{ MeV} \quad \text{(natural units)} \\
		m_e &= 9.109 \times 10^{-31} \text{ kg} \quad \text{(SI)} \\
		E_e &= m_e c^2 = 0.511 \text{ MeV} = 8.187 \times 10^{-14} \text{ J}
	\end{align}
	
	\section{Comparison with α: Parallel Structure}
	
	\subsection{Two Analogous Conventions}
	
	\begin{table}[h]
		\centering
		\begin{tabular}{|l|c|c|}
			\hline
			\textbf{Convention} & \textbf{Fine-structure constant α} & \textbf{Speed of light c} \\
			\hline
			\textbf{Natural} & $\alpha = 1$ (Heaviside-Lorentz) & $c = 1$ (Planck units) \\
			\textbf{SI / Standard} & $\alpha = 1/137.036$ (Gauss-SI) & $c = 299,792,458$ m/s \\
			\hline
			\textbf{Document} & 101 (Circularity-Constants) & 134 (Unit Conventions c) \\
			\hline
		\end{tabular}
		\caption{Parallel structure: α and c as conventions}
	\end{table}
	
	\subsection{Common Principles}
	
	Both cases show:
	\begin{itemize}
		\item \textbf{Physics is invariant} under unit choice
		\item \textbf{Natural units} simplify theoretical work
		\item \textbf{SI units} enable practical applications
		\item \textbf{T0 theory}: Both are derived conventions, not fundamental
	\end{itemize}
	
	\subsection{T0 Reduction}
	
	From the T0 perspective (see Document 101):
	\begin{equation}
		\xi \to D_f \to E_0 \to \alpha \to \hbar, c, G \to \text{all other constants}
	\end{equation}
	
	\textbf{Only $\xi = \frac{4}{3} \times 10^{-4}$ is fundamental.}
	
	Both $\alpha$ and $c$ are derived quantities or conventions.
	
	\section{When to Use Which System?}
	
	\subsection{Decision Matrix}
	
	\begin{table}[h]
		\centering
		\begin{tabular}{|l|c|c|}
			\hline
			\textbf{Context} & \textbf{Natural units (c=1)} & \textbf{SI units ($c$ explicit)} \\
			\hline
			Theoretical physics & \checkmark & \\
			Quantum field theory & \checkmark & \\
			High-energy physics & \checkmark & \\
			Early cosmology & \checkmark & \\
			\hline
			Experimental physics & & \checkmark \\
			Engineering & & \checkmark \\
			Precision measurements & & \checkmark \\
			Applied physics & & \checkmark \\
			Education & & \checkmark \\
			\hline
		\end{tabular}
		\caption{Application domains of unit systems}
	\end{table}
	
	\subsection{Recommendations}
	
	\textbf{Use natural units (c=1) when:}
	\begin{itemize}
		\item Performing theoretical derivations
		\item Symmetries and invariant structures are important
		\item Formulas should be simplified
		\item Working in particle physics or cosmology
	\end{itemize}
	
	\textbf{Use SI units (c explicit) when:}
	\begin{itemize}
		\item Planning or evaluating experimental measurements
		\item Technical calculations are required
		\item Results should be understandable for non-physicists
		\item Historical continuity is important
	\end{itemize}
	
	\section{Common Misconceptions}
	
	\subsection{``c=1 is only an approximation''}
	
	\textbf{FALSE.} c=1 is \textbf{exact} in natural units, not an approximation.
	
	It is a choice of unit system that defines:
	\begin{equation}
		\text{Length unit} = \text{Time unit}
	\end{equation}
	
	Analogously: In Planck units, $\hbar=1$ is exact, not approximate.
	
	\subsection{``E=m only holds for photons''}
	
	\textbf{FALSE.} In natural units, $E=m$ holds for \textbf{all} particles in their rest frame.
	
	For photons ($m=0$): $E = p$ (in natural units) or $E = pc$ (in SI).
	
	\subsection{``c is a fundamental constant of nature''}
	
	\textbf{T0 viewpoint}: c is a \textbf{ratio} $L/T$, not a fundamental constant.
	
	With the T0 duality $T \cdot m = 1$, c varies dynamically with mass scale:
	\begin{equation}
		c = \frac{L}{T} = L \cdot m
	\end{equation}
	
	Only in SI units is c \emph{fixed by definition}.
	
	\subsection{``Natural units change the physics''}
	
	\textbf{FALSE.} Physics is independent of the unit system.
	
	All \textbf{dimensionless} quantities (e.g., $\xi$, $\alpha$, mass ratios) are invariant.
	
	Only dimensional quantities change their numerical values.
	
	\section{T0 Perspective: c as a Dynamic Ratio}
	
	\subsection{The T0 Fundamental Relation}
	
	From Document 077:
	\begin{equation}
		T \cdot m = 1 \quad \text{(Time-Mass Duality)}
	\end{equation}
	
	This means:
	\begin{align}
		T &\propto \frac{1}{m} \\
		L &\propto \frac{1}{m} \quad \text{(via Compton wavelength)} \\
		\Rightarrow \quad c &= \frac{L}{T} \propto \frac{1/m}{1/m} = \text{scale-dependent}
	\end{align}
	
	\subsection{Implications}
	
	\textbf{1. c is not universally constant in the T0 framework:}
	
	Different effective c-values can occur at different mass scales.
	
	\textbf{2. SI definition c=299,792,458 m/s is a calibration:}
	
	This fixation defines the meter via the second -- a metrological convention.
	
	\textbf{3. Natural units c=1 are T0-consistent:}
	
	In Planck units, where $t_P \propto \ell_P$, c=1 is the natural choice.
	
	\subsection{Comparison with Document 077}
	
	Document 077 argues: ``E=mc² = E=m -- The Constant Illusion Exposed''
	
	\textbf{Clarification here:}
	\begin{itemize}
		\item E=mc² (SI) and E=m (natural) are \emph{equivalent}, not identical
		\item The difference lies in the \emph{unit system}, not in physics
		\item Einstein's c-fixation is a \emph{convention}, not an error
		\item T0 shows: c is a ratio that can vary depending on scale
	\end{itemize}
	
	\section{Mathematical Consistency}
	
	\subsection{Energy-Momentum Relation}
	
	\textbf{In natural units ($c=1$):}
	\begin{equation}
		E^2 = p^2 + m^2
	\end{equation}
	
	\textbf{In SI units:}
	\begin{equation}
		E^2 = (pc)^2 + (mc^2)^2
	\end{equation}
	
	Both forms are mathematically equivalent.
	
	\subsection{Lorentz Transformation}
	
	\textbf{In natural units:}
	\begin{equation}
		E' = \gamma (E - p \cdot v)
	\end{equation}
	
	\textbf{In SI units:}
	\begin{equation}
		E' = \gamma (E - p \cdot v \cdot c^2)
	\end{equation}
	
	The physics remains invariant.
	
	\subsection{Klein-Gordon Equation}
	
	\textbf{In natural units:}
	\begin{equation}
		(\partial_\mu \partial^\mu + m^2) \phi = 0
	\end{equation}
	
	\textbf{In SI units:}
	\begin{equation}
		\left(\frac{1}{c^2} \frac{\partial^2}{\partial t^2} - \nabla^2 + \frac{m^2c^2}{\hbar^2}\right) \phi = 0
	\end{equation}
	
	Identical physics, different notation.
	
	\section{References to T0 Documents}
	
	\subsection{Related Documents}
	
	\begin{itemize}
		\item \textbf{Document 013}: SI System and T0 Theory \\

		\item \textbf{Document 014}: Natural vs. SI Units \\

		\item \textbf{Document 015}: Systematics of Natural Units \\

		\item \textbf{Document 077}: E=mc² = E=m Analysis \\

		\item \textbf{Document 101}: Circularity of Constants (α Conventions) \\

		\item \textbf{Document 133}: Fractal Correction K\_frak Derivation \\
		
	\end{itemize}
	
	\subsection{Derivation Hierarchy}
	
	The T0 hierarchy (from Document 101):
	\begin{equation}
		\xi \to D_f \to E_0 \to \alpha \to \hbar, c, G \to \text{mass ratios}
	\end{equation}
	
	shows that both $\alpha$ and $c$ are derived quantities.
