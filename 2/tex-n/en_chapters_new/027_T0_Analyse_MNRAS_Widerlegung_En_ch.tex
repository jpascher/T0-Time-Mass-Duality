\chapter{\textbf{Analysis of MNRAS Paper 544: A Refutation of Modified Gravity Models and an Indirect Confirmation of the T0-Theory}}

\section*{Abstract}
		This document analyzes the findings of the influential paper "Does the Hubble tension eclipse the Solar System?" (MNRAS, 544, 1, 2024) \cite{nathan2024} and places them in the context of the T0-Theory. The paper refutes a significant class of modified gravity theories by demonstrating that they would lead to measurable anomalies in Solar System orbits, which are not observed. We argue that this falsification should be considered strong, indirect evidence for the T0-Theory's approach, as T0-Theory is, by definition, consistent with high-precision Solar System data.

	
	
	\section{Implications for the T0-Theory}
	
	The falsification of a competing model often serves as strong, indirect confirmation for an alternative theory. This is especially true here, as the T0-Theory solves the problem at a more fundamental level and trivially passes the ``test'' described in the paper.
	
	\subsection{T0-Theory Does Not Modify Gravity}
	The crucial difference is that T0-Theory leaves General Relativity untouched on Solar System scales. It does not postulate any ad-hoc modification of gravity. Instead, it addresses the flawed premise upon which the Hubble tension is based: the assumption of cosmic expansion.
	
	\subsection{Redshift as a Geometric Effect}
	In the T0-Theory, there is no accelerated expansion and, consequently, no ``Hubble tension'' to explain. The observed cosmological redshift is instead explained as an emergent, geometric effect.

	
	\subsection{Consistency with Solar System Data}
	The mechanism of geometric redshift is absolutely negligible over the comparatively tiny distances of the Solar System (a few light-hours). The cumulative effect only becomes measurable over millions and billions of light-years.
	
	It follows that:
	\begin{center}
		\textbf{The T0-Theory predicts exactly zero measurable anomalies in the planetary orbits of the Solar System.}
	\end{center}
	It is therefore, by definition, perfectly consistent with the high-precision data from the Cassini mission that refutes the modified gravity models.
	\begin{thebibliography}{9}
		\bibitem{nathan2024}
		E. Nathan, A. Hees, H. W. R. W. Z. Yan, \textit{Does the Hubble tension eclipse the Solar System?}, Monthly Notices of the Royal Astronomical Society, 544(1), 975-983, 2024.
		
		\bibitem{pascher:geometric_cosmology}
		J. Pascher, \textit{T0-Kosmologie: Rotverschiebung als geometrischer Pfad-Effekt in einem statischen Universum}, T0-Dokumentenserie, Nov. 2025.
	\end{thebibliography}
	