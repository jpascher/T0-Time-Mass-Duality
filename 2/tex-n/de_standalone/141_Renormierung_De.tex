% Standardized preamble - 141\_Renormierung\_De.pdf
\documentclass[12pt,a4paper]{article}

% JETZT die shared preamble einbinden (sie enthält alle \usepackage-Befehle)
% Minimale T0 Standalone Preamble - A4 Format - 25 Zeilen
\RequirePackage{fontspec}
\RequirePackage{unicode-math}
\usepackage[ngerman]{babel}
\usepackage{microtype}
\setmainfont{Inter}
\setmonofont{JetBrains Mono}
\setmathfont{Libertinus Math}
\usepackage{amsmath,amsfonts,amsthm}
\usepackage{mathtools}
\usepackage{graphicx}
\usepackage{xcolor}
\definecolor{t0blue}{RGB}{0,102,204}
\definecolor{t0green}{RGB}{34,139,34}
\definecolor{t0red}{RGB}{204,0,0}
\usepackage{geometry}
\geometry{a4paper,margin=2.5cm}
\usepackage[most]{tcolorbox}
\newtcolorbox{keyresult}[1][]{colback=yellow!5,colframe=t0blue!80,fonttitle=\bfseries,title={#1},breakable}
\newtcolorbox{important}[1][]{colback=red!5,colframe=t0red!80,fonttitle=\bfseries,title={#1},breakable}
\newcommand{\Tfield}{\ensuremath{\mathcal{T}}}
\usepackage{hyperref}
\hypersetup{colorlinks=true,linkcolor=t0blue}


\title{Fraktale Raumzeit und ihre Implikationen in der Quantengravitation}
\author{Johann Pascher}
\date{15. Januar 2026}

\begin{document}
	
	\\maketitle

\\tableofcontents
	
	\begin{abstract}
		Dieses Dokument fasst zentrale Ergebnisse aus der theoretischen Physik zur fraktalen Struktur der Raumzeit in verschiedenen Ansätzen zur Quantengravitation zusammen. Besonderes Augenmerk liegt auf dem Dimensionsfluss (von spektraler Dimension $\sim 2$ im UV zu $\sim 4$ im IR), der daraus folgenden Renormalisierbarkeit und den Implikationen für Singularitäten, Gravitationspotential und Kausalität. Die Darstellung basiert ausschließlich auf veröffentlichten wissenschaftlichen Arbeiten.
	\end{abstract}
	
	\section{Einführung: Von den Fundamenten zum Problem}
	\subsection{Das grundlegende Problem der Quantengravitation}
	Die moderne Physik basiert auf zwei revolutionären Säulen: der \emph{Allgemeinen Relativitätstheorie} (ART) und der \emph{Quantenmechanik}. Die ART beschreibt die Schwerkraft (\emph{Gravitation}) als eine Krümmung der \textbf{Raumzeit} – einem vierdimensionalen Geflecht, in dem Raum und Zeit untrennbar verbunden sind. Diese Theorie ist auf großen Skalen (Sterne, Galaxien) äußerst erfolgreich. Die Quantenmechanik hingegen beschreibt das Verhalten von Materie und Kräften (außer der Gravitation) auf kleinsten Skalen (Atome, Teilchen) mit überwältigender Genauigkeit.
	
	Das fundamentale Problem ist, dass diese beiden Theorien mathematisch und konzeptionell unvereinbar sind. An Punkten extremer Dichte, wie im Zentrum eines Schwarzen Lochs oder am Beginn des Universums (\textbf{Urknall-\\Singularität}), liefern beide Theorien unsinnige oder unendliche Ergebnisse. Eine Theorie der \textbf{Quantengravitation}, die beide vereint, wird daher gesucht.
	
	\subsection{Die Herausforderung der ''Unendlichkeiten'' (Renormierbarkeit)}
	Ein Hauptproblem bei der Quantisierung der Gravitation sind \textbf{Divergenzen} – mathematische Terme, die gegen Unendlich gehen. In der Quantenfeldtheorie (der Sprache der Teilchenphysik) treten solche Unendlichkeiten ständig auf, können aber durch ein als \textbf{Renormierung} bezeichnetes Verfahren systematisch ''abgezogen'' und durch endliche Messwerte ersetzt werden. Eine Theorie, bei der dies möglich ist, heißt \textbf{renormierbar}. Für die Gravitation in vier Dimensionen funktioniert dieser ''Trick'' der Störungstheorie (\emph{perturbativ}) jedoch nicht – es entstehen unendlich viele neue Arten von Unendlichkeiten, die nicht mehr kontrolliert werden können. Die Theorie ist \textbf{nicht-renormierbar}.
	
	\subsection{Eine radikale Idee: Was, wenn die Raumzeit selbst anders ist?}
	Die Standard-ART geht von einer glatten, kontinuierlichen Raumzeit aus, überall differenzierbar (man kann an jedem Punkt eine Tangente anlegen). Was aber, wenn diese Vorstellung auf fundamentalen, kleinsten Skalen (der \textbf{Planck-Skala}, $\sim 10^{-35}$ Meter) falsch ist? Die Idee einer \textbf{fraktalen Raumzeit} besagt, dass die Raumzeit auf diesen Skalen keine einfache, glatte Struktur hat, sondern \textbf{rau}, \textbf{gebrochen} und \textbf{selbstähnlich} ist – ähnlich wie die unendlich detaillierte Küste einer Insel, die bei jeder Vergrößerung neue Strukturen zeigt. Eine solche Struktur ist \textbf{nicht-differenzierbar}.
	
	Für solche komplexen Gebilde reicht die gewöhnliche Dimensionsvorstellung (1D=Linie, 2D=Fläche, 3D=Volumen) nicht aus. Man führt die \textbf{fraktale Dimension} (oder \textbf{Hausdorff-Dimension}) $d_H$ ein, die nicht-ganzzahlig sein kann (z.B. ~1.26 für die Küstenlinie). Ein noch besserer Indikator für das Verhalten einer Quantentheorie auf einer solchen Struktur ist die \textbf{spektrale Dimension} $d_s$. Sie misst, wie ein Teilchen (oder Information) durch Diffusion die Struktur ''erfährt''. Ein entscheidendes Ergebnis vieler neuer Ansätze ist, dass $d_s$ bei hohen Energien/kleinen Skalen (\textbf{UV}, ''ultraviolett'') auf etwa \textbf{2} abfällt, während sie bei niedrigen Energien/großen Skalen (\textbf{IR}, ''infrarot'') den Wert \textbf{4} (drei Raum- plus eine Zeitdimension) annimmt. Diesen Übergang nennt man \textbf{Dimensionsfluss}.
	
	\subsection{Die zentrale Hypothese und ihr Nutzen}
	Die zentrale These dieses Dokuments ist, dass genau dieser \textbf{Dimensionsfluss zu $d_s \approx 2$ im UV} das Renormierungsproblem löst. Bei einer effektiven Dimension von 2 wird die Quantengravitationstheorie \textbf{power-counting renormalisierbar} – die Unendlichkeiten werden kontrollierbar oder verschwinden sogar ganz. Dies bietet einen eleganten, von der Geometrie selbst kommenden \textbf{UV-Cutoff}. Darüber hinaus ''verschmiert'' eine fraktale, nicht-kontinuierliche Struktur die scharfen \textbf{Singularitäten} der ART und könnte so Probleme wie das Informationsparadoxon Schwarzer Löcher mildern.
	
	Die folgenden Kapitel entfalten diese Idee im Detail, basierend auf konkreten Forschungsprogrammen.
	
	\section{Dimensionsfluss und fraktale Geometrie}
	
	\subsection{Hausdorff- und spektrale Dimension}
	Die spektrale Dimension $d_s$ wird über die Rückkehrwahrscheinlichkeit eines Random Walk definiert. In den betrachteten Modellen gilt:
	\begin{equation}
		d_s \sim 
		\begin{cases} 
			2 & \text{(UV, Planck-Skala)} \\
			4 & \text{(IR, makroskopische Skalen)}
		\end{cases}
	\end{equation}
	
	Dieser Fluss tritt in folgenden Ansätzen auf:
	\begin{itemize}
		\item Asymptotic Safety (Reuter et al.)
		\item Causal Dynamical Triangulations (CDT)
		\item Multifraktale Raumzeiten (Calcagni)
		\item Approximationen in Loop Quantum Gravity
	\end{itemize}
	
	Dieses Phänomen wird durch mehrere unabhängige Forschungsstränge gestützt. Modesto argumentiert, dass die Analyse von Feynman-Diagrammen auf einem Spin-Schaum eine effektive spektrale Dimension von nahezu 2 nahe der Planck-Skala ergibt \cite{Modesto2008}. Darüber hinaus bestätigt Hořava diesen Fluss in seinem Ansatz ''Spectral Dimension of the Universe in Quantum Gravity at a Lifshitz Point'', wo \textit{''the spectral dimension of spacetime flows from $d_s=4$ at large scales, to $d_s=2$ at short distances.''} \cite{Horava2009}.
	
	Der universelle Charakter dieses Ergebnisses wird von Modesto betont: \textit{''This result is consistent with two other approaches to non perturbative quantum gravity: 'causal dynamical triangulation' and 'asymptotically safe quantum gravity'.''} \cite{Modesto2009}.
	
	\subsection{Multifraktale Geometrien und fraktionale Analysis}
	Zur Beschreibung skalenabhängiger Dimensionen wird fraktionale Analysis (fractional calculus) eingesetzt. Die Lagrange-Dichte wird mit fraktionalen Ableitungen formuliert, wodurch die Dimension kontinuierlich mit der Skala variiert. Calcagni erklärt dies als Basis multifraktaler Raumzeiten: \textit{''Based on fractional calculus, these continuous spacetimes have their dimension changing with the scale.''} \cite{Calcagni2012}.
	
	\textbf{T0-Time-Mass-Duality / Fundamental-Fraktal-geometrische Feldtheorie (FFGFT):} Im vorliegenden Ansatz wird der Dimensionsfluss zur fraktalen Struktur nicht als Zusatzpostulat behandelt, sondern folgt notwendig aus der fundamentalen T0-Time-Mass-Dualität selbst. Gravitation emergiert in diesem Rahmen als effektives Phänomen dieser zugrundeliegenden, strukturierten Feldtheorie. Damit stellt der \textbf{T0-Ansatz} einen eigenständigen Weg dar, der die Annahme eines separaten Quantenfelds für die Gravitation vermeidet und stattdessen Raumzeitgeometrie und Materie auf einheitliche Prinzipien zurückführt.
	
	\section{Implikationen für die Gravitation}
	
	\subsection{Renormalisierbarkeit}
	In vier Dimensionen ist perturbative Quantengravitation nicht-renormalisierbar. Durch Reduktion der spektralen Dimension auf $d_s \approx 2$ im UV wird die Theorie jedoch \emph{power-counting renormalisierbar} oder sogar super-renormalisierbar.
	
	Dies vermeidet unendlich viele Gegen-Terme und ermöglicht eine konsistente Quantentheorie der Gravitation. Calcagni spezifiziert für seine Modelle, dass \textit{''A field theory...which lives in fractal spacetime...is argued to be power-counting renormalizable, ultraviolet finite, and causal at microscopic scales.''} \cite{Calcagni2010a}.
	
	Ein Beispiel hierfür findet sich in der ''Fractal Quantum Space-Time'' nach Modesto, der darauf hinweist, dass \textit{''a system of spin-foam models for Euclidean quantum gravity [is] finite to all orders in the perturbative expansion, and that ultraviolet divergences disappear in the non-perturbative regime.''} \cite{Modesto2009}.
	
	\subsection{Singularitäten}
	Punkt-Singularitäten (Schwarze Löcher, Urknall) werden in fraktalen Geometrien aufgelöst. Es entstehen keine echten Punkte mehr, sondern fraktal verteilte Dichteverteilungen. Dies mildert das Informationsparadoxon. Laut Modesto deuten die Eigenschaften fraktaler Quantenraumzeiten darauf hin, dass \textit{''the singularity problem seems to be solved in the covariant formulation of quantum gravity in terms of spin-foam models.''} \cite{Modesto2009}.
	
	\subsection{Gravitationspotential und Kausalität}
	Das Newtonsche $1/r^2$-Gesetz gilt nur als makroskopische Näherung. In fraktaler Geometrie skaliert das Potential skalenabhängig als $1/r^{d-1}$. Lichtkegel werden auf kleinen Skalen diffus, was eine effektive Verletzung der strengen Lokalität impliziert. Diese grundlegende mathematische Konsequenz, die aus dem Dimensionsfluss folgt, wird in keiner der zitierten Arbeiten direkt formuliert, stellt jedoch eine zentrale Folgerung für jedes phänomenologische Modell, das aus diesen Ansätzen abgeleitet werden kann, dar.
	
	\section{Der T0-Ansatz: Time-Mass-Duality als fundamentale fraktal-geometrische Feldtheorie}
	
	\textbf{Der T0-Ansatz bringt keine klassische Quantisierung der Gravitation mit sich.}
	
	Basierend auf dem Kern der Theorie (wie in der Master-Narrative und den zugehörigen Dokumenten beschrieben) ist Gravitation \textbf{nicht} als separates Quantenfeld quantisiert, das man mit Gravitonen, Schleifen-Diagrammen oder einer neuen Quantenfeldtheorie der Gravitation behandeln müsste. Stattdessen:
	
	\begin{itemize}
		\item \textbf{Gravitation ist emergent} aus der fundamentalen ontologischen Dualität zwischen Zeit und Masse (T0 als zentrale Brücke). Sie entsteht als \textbf{effektives geometrisches Phänomen} in einer fraktalen Raumzeit, die durch $\xi$-Korrekturen und skalenabhängige Dimensionalität reguliert wird.
		
		\item Es gibt \textbf{keine Notwendigkeit für eine perturbative Quantisierung} der \\ Einstein-Hilbert-Action (die ja in 4D notorisch nicht-renormalisierbar ist). Die UV-Probleme der Standard-Quantengravitation lösen sich auf, weil die fraktale Struktur ($D_f \approx 2.94$ makroskopisch, Tendenz zu $\sim 2$ im UV) einen natürlichen Regulator darstellt -- ähnlich wie in manchen anderen Ansätzen, aber hier rein aus der Dualität abgeleitet, ohne zusätzliche Quantisierungsprozedur.
		
		\item Der T0-Ansatz vermeidet bewusst die klassischen Fallstricke der Quantengravitation:
		\begin{itemize}
			\item Keine Gravitonen als fundamentale Teilchen
			\item Keine unendlichen Gegen-Terme oder Landau-Pole
			\item Keine Notwendigkeit für Stringtheorie, Loops oder Asymptotic Safety als separaten Mechanismus
		\end{itemize}
		Stattdessen wird Gravitation \textbf{durch die Dualität finit und konsistent} -- sie ist quasi-klassisch auf großen Skalen, aber intrinsisch reguliert durch die fraktale Geometrie auf kleinen Skalen.
	\end{itemize}
	
	Kurz gesagt: Der T0-Ansatz \textbf{umgeht} das Quantisierungsproblem der Gravitation, anstatt es zu lösen. Die Gravitation braucht \textbf{keine separate Quantisierung}, weil sie aus dem fundamentalen Prinzip (Time-Mass-Dualität + fraktale Geometrie) bereits vollständig und UV-finit hervorgeht. Das ist einer der großen Vorteile dieses Ansatzes im Vergleich zu den konventionellen Quantengravitationsprogrammen.
	
	\section{Vergleich der wichtigsten Ansätze}
	
	Die zentralen Eigenschaften der verschiedenen Ansätze zur fraktalen Raumzeit lassen sich wie folgt zusammenfassen:
	
	\begin{itemize}
		\item \textbf{Asymptotic Safety (Reuter et al.)}
		\begin{itemize}
			\item Spektrale Dimension UV: $\sim 2$
			\item Hausdorff-Dimension UV: $\sim 2$
			\item Renormalisierbarkeit: Ja (non-perturbativ)
		\end{itemize}
		
		\item \textbf{Causal Dynamical Triangulations (CDT)}
		\begin{itemize}
			\item Spektrale Dimension UV: $\sim 2$
			\item Hausdorff-Dimension UV: $\sim 2$
			\item Renormalisierbarkeit: Ja
		\end{itemize}
		
		\item \textbf{Multifractale Raumzeit (Calcagni)}
		\begin{itemize}
			\item Spektrale Dimension UV: $\sim 2$
			\item Hausdorff-Dimension UV: $\sim 2$
			\item Renormalisierbarkeit: Ja (perturbativ)
		\end{itemize}
		
		\item \textbf{Loop Quantum Gravity (Approximationen)}
		\begin{itemize}
			\item Spektrale Dimension UV: $\sim 2$
			\item Hausdorff-Dimension UV: variabel
			\item Renormalisierbarkeit: Ja (teilweise)
		\end{itemize}
		
		\item \textbf{T0-Ansatz (Time-Mass-Duality / FFGFT)}
		\begin{itemize}
			\item Spektrale Dimension UV: $\sim 2$ (Tendenz)
			\item Hausdorff-Dimension UV: $\sim 2.94 \rightarrow \sim 2$
			\item Renormalisierbarkeit: UV-finit (emergente Gravitation)
			\item Besonderheit: Keine separate Quantisierung nötig; Gravitation emergiert aus der Time-Mass-Dualität und der skalenabhängigen fraktalen Struktur des Raums.
		\end{itemize}
	\end{itemize}
	
	\section{Schlussfolgerung}
	Der Dimensionsfluss von $d_s \approx 2$ im UV zu $d_s \approx 4$ im IR stellt ein universelles, robustes Ergebnis in der modernen Quantengravitationsforschung dar. Er bietet einen eleganten Mechanismus zur Lösung des Renormalisierbarkeitsproblems, mildert Singularitäten und verändert grundlegend unser Verständnis von Gravitation auf fundamentalen Skalen. Der T0-Ansatz (Time-Mass-Duality / FFGFT) stellt dabei einen radikal alternativen Weg dar, der das Quantisierungsproblem nicht durch immer komplexere Quantenfeldtheorien zu lösen versucht, sondern es durch eine fundamentale ontologische Dualität und eine emergente, fraktale Raumzeitgeometrie umgeht.
	
	\begin{thebibliography}{9}
		\bibitem[Modesto(2009)]{Modesto2009}
		Modesto, L. (2009). \textit{Fractal Quantum Space-Time}. \texttt{arXiv:0905.1665 [gr-qc]}.
		
		\bibitem[Modesto(2008)]{Modesto2008}
		Modesto, L. (2008). \textit{Fractal Structure of Loop Quantum Gravity}. \texttt{arXiv:0812.2214 [gr-qc]}. Published in Class. Quantum Grav. 26 (2009) 242002.
		
		\bibitem{Magliaro2009}
		Magliaro, E., Perini, C., Modesto, L. (2009). \textit{Fractal Space-Time from Spin-Foams}. \texttt{arXiv:0911.0437 [gr-qc]}.
		
		\bibitem[Calcagni(2010)]{Calcagni2010}
		Calcagni, G. (2010). \textit{Fractal universe and quantum gravity}. \texttt{arXiv:0912.3142 [hep-th]}. Phys. Rev. Lett. 104, 251301 (2010).
		
		\bibitem[Calcagni(2010)]{Calcagni2010a}
		Calcagni, G. (2010). \textit{Quantum field theory, gravity and cosmology in a fractal universe}. \texttt{arXiv:1001.0571 [hep-th]}. JHEP 03 (2010) 120.
		
		\bibitem[Calcagni(2012)]{Calcagni2012}
		Calcagni, G. (2012). \textit{Introduction to multifractional spacetimes}. \texttt{arXiv:1209.1110 [hep-th]}. AIP Conf. Proc. 1483 (2012) 31.
		
		\bibitem[Hořava(2009)]{Horava2009}
		Hořava, P. (2009). \textit{Spectral Dimension of the Universe in Quantum Gravity at a Lifshitz Point}. \texttt{arXiv:0902.3657 [hep-th]}. Phys. Rev. Lett. 102, 161301 (2009).
		
		\bibitem{Thurigen2015}
		Thürigen, J. (2015). \textit{Discrete Quantum Geometries}. \texttt{arXiv:1511.08737 [gr-qc]}.
	\end{thebibliography}
	
\end{document}


