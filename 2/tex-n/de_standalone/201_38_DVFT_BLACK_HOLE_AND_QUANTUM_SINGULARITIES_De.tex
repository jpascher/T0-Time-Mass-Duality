\documentclass[12pt,a4paper]{article}

% ==============================================================================
% T0 Theory: Shared English Preamble
% Version: 1.0
% Author: Johann Pascher
% Date: 2025
% ==============================================================================
%
% This is the standardized shared preamble for all English T0 Theory documents.
% Place this file in your document's directory or use a path like:
%   % ==============================================================================
% T0 Theory: Shared ENGLISH Preamble – Optimized for eBook/Book
% Version: 2.0 – Final 2026 (LuaLaTeX only) – ENGLISH corrected
% Author: Johann Pascher
% Date: January 2026
% ==============================================================================
%
% IMPORTANT: Compile EXCLUSIVELY with LuaLaTeX!
% In TeXstudio: Options → Configure TeXstudio → Build → Default Compiler → LuaLaTeX
%
% Required Fonts (install once):
% - Inter: https://fonts.google.com/specimen/Inter
% - JetBrains Mono: https://www.jetbrains.com/lp/mono/
% - Libertinus Math: https://github.com/libertinus-fonts/libertinus
% ==============================================================================

% === CHAPTER 1: BASIC PACKAGES (must come FIRST) ===
\RequirePackage{fontspec}
\RequirePackage{unicode-math}
\usepackage{chngcntr}
\setcounter{secnumdepth}{1}  % Nur Sections nummerieren (nicht subsections)
\setcounter{tocdepth}{1}     % Nur Sections im TOC (nicht subsections)
\makeatletter
\@ifundefined{c@chapter}{}{\counterwithout{section}{chapter}}  % Falls Kapitel existieren
\makeatother
\counterwithout{subsection}{section}  % Löse Verknüpfung
% === CHAPTER 2: LANGUAGE (ENGLISH) ===
\usepackage[english]{babel}
\usepackage{microtype}                    % IMPORTANT for better hyphenation!

% Typography settings for better line breaking
\frenchspacing                     % Correct English spacing after punctuation
\emergencystretch=3em              % Allows more stretch for difficult lines
\tolerance=2500                    % Higher tolerance for line breaks
\hbadness=10000                    % Suppresses "underfull hbox" warnings
\hfuzz=2pt                         % Allows minimal overfull
\pretolerance=150                  % Better word breaking

% Prevent bad page breaks
\clubpenalty=10000           % No "orphans"
\widowpenalty=10000          % No "widows"
\displaywidowpenalty=10000   % Also with equations
\brokenpenalty=10000         % No broken words across pages

% Explicit hyphenation for long technical words
\hyphenation{Fun-da-men-tal Frac-tal-Ge-o-met-ric Field The-o-ry Meth-od-o-log-i-cal}
\hyphenation{Re-vi-sion-ism Quan-ti-za-tion U-ni-fi-ca-tion Ef-fec-tive}
\hyphenation{Re-nor-mal-iz-a-bil-i-ty Sin-gu-lar-i-ties Con-cil-i-a-tion}
\hyphenation{E-mer-gence Phe-nom-e-no-log-i-cal Doc-u-men-ta-tion A-nal-y-sis}
\hyphenation{Grav-i-ta-tion Quan-tum Me-chan-ics Dog-ma-tism Con-se-quent}
\hyphenation{Par-al-lel-ism Im-ple-men-ta-tion Per-tur-ba-tions}
\hyphenation{Geo-met-ric Ar-ti-fact In-com-pat-i-bil-i-ty Con-struc-tive}
\hyphenation{Frac-tal Di-men-sion-less In-ves-ti-ga-tion De-scrip-tion}
\hyphenation{In-ter-pre-ta-tion Phe-nom-e-no-log-i-cal Math-e-mat-i-cal}
\hyphenation{Phi-lo-soph-i-cal Le-git-i-ma-tion Ap-pli-ca-tion Der-i-va-tion}
\hyphenation{U-ni-fi-ca-tion As-sump-tion Con-cep-tion Ex-pec-ta-tion}
\hyphenation{Sym-me-try-ex-ten-sion O-ver-all-pic-ture Chal-lenge}
\hyphenation{In-ter-ac-tion Ma-te-ri-al Ap-proach Per-spec-tive Pro-ce-dure}

% === CHAPTER 3: FONTS (with proper ligatures) ===
\setmainfont{Inter}[
Scale=1.02,
UprightFont=*-Regular,
BoldFont=*-Bold,
ItalicFont=*-Italic,
BoldItalicFont=*-BoldItalic,
Ligatures=TeX,           % IMPORTANT for proper typography
Language=English         % Explicit language support
]
\setsansfont{Inter}[
Scale=MatchLowercase,
Ligatures=TeX,
Language=English
]
\setmonofont{JetBrains Mono}[
Scale=0.95,
Language=English
]

% Math Font (simple & stable) – MUST come AFTER language definition
% IMPORTANT: Libertinus Math for correct \underbrace display!
\setmathfont{Libertinus Math}[Scale=1.0]

% === CHAPTER 4: MATHEMATICS PACKAGES (in STRICT order!) ===
% IMPORTANT: mathtools must come BEFORE unicode-math for some commands!
\usepackage{mathtools}           % FIRST mathtools!

% Then the rest
\usepackage{amsmath, amsfonts, amsthm}

% SIUNITX MUST be loaded BEFORE physics!
\usepackage{siunitx}
\sisetup{
	locale=US,                    % ENGLISH settings for SI units!
	group-separator={,},          % Thousands separator comma
	output-decimal-marker={.},    % Decimal separator point
	per-mode=symbol,
	separate-uncertainty=true
}

% Custom SI units used in narrative and books
\DeclareSIUnit\gigalightyear{Gly}
\DeclareSIUnit\mev{MeV}

% physics – MUST be loaded AFTER siunitx and mathtools
\usepackage{physics}

% === CHAPTER 5: ADDITIONS from pdflatex best practices ===
\usepackage{colortbl}        % Colored tables (ESSENTIAL!)
\usepackage{placeins}        % Float control: \FloatBarrier
\usepackage{subcaption}      % Subfigures
\usepackage{xurl}            % Better URL line breaking
% Hyphenation for URLs in bibliography
\def\UrlBreaks{\do\/\do-}

% === CHAPTER 6: PAGE LAYOUT
% =============================================================================
% SECTION 2: Page Geometry – 6" × 9" Buchformat
% =============================================================================
\usepackage[paperwidth=6in, paperheight=9in,
top=0.9in,
bottom=1.1in,
inner=0.9in,            % Größerer Innenrand für Bindung
outer=0.6in,            % Kleinerer Außenrand → mehr Text pro Seite
bindingoffset=0.5in,    % Puffer für Bindung (Steg)
twoside]{geometry}
\setlength{\headheight}{15pt}
%\usepackage[paperwidth=8.25in, paperheight=11in,
%top=1.0in,
%bottom=1.0in,
%left=1.0in,
%right=1.0in,
%twoside=false
% === CHAPTER 7: GRAPHICS AND TABLES ===
\usepackage{graphicx}
\usepackage[table,xcdraw]{xcolor}
% T0 brand colors
\definecolor{gold}{RGB}{255,215,0}
\definecolor{blue}{rgb}{0,0,1}
\definecolor{boxgray}{RGB}{240,240,240}
\definecolor{deepblue}{RGB}{0,0,127}
\definecolor{deepgreen}{RGB}{0,127,0}
\definecolor{deepred}{RGB}{191,0,0}
\definecolor{t0blue}{RGB}{33,150,243}
\definecolor{t0green}{RGB}{76,175,80}
\definecolor{t0orange}{RGB}{255,152,0}
\definecolor{t0purple}{RGB}{156,39,176}
\definecolor{t0red}{RGB}{244,67,54}
\definecolor{t0yellow}{RGB}{255,204,0}
\usepackage{tikz}
\usetikzlibrary{arrows.meta,positioning,shapes.geometric,decorations.pathmorphing,patterns,shapes.arrows,intersections}
\usepackage{pgfplots}
\pgfplotsset{compat=1.18}
\usepackage{quantikz}
\usepackage[most]{tcolorbox}
\tcbuselibrary{breakable}

% === WICHTIG: Algorithm-Konflikt umgehen ===
% Option: algorithmic mit GROSSBUCHSTABEN
% Gemeinsame Box für Experimente
\newtcolorbox{experimentbox}[1][]{
	colback=green!5!white,
	colframe=t0green!80!black,
	fonttitle=\bfseries,
	title={{#1}},
	breakable
}

% Abstract-Fallback
\ifdefined\abstract\else
\newenvironment{abstract}{\section*{\abstractname}\itshape\small\par\bigskip}{\bigskip}
\fi

% === MAKROS SICHER NEU DEFINIEREN / ÜBERSCHREIBEN ===
% Definiere Makros OHNE doppelte Subskripte
\newcommand{\phipar}{\phi_{\mathrm{par}}}
%\newcommand{\xipar}{\xi_{\mathrm{par}}}
\newcommand{\Qphipar}{Q_{\phi_{\mathrm{par}}}}
\newcommand{\rphipar}{r_{\phi_{\mathrm{par}}}}
\newcommand{\logphipar}{\log_{\phi_{\mathrm{par}}}}
\newcommand{\CHSH}{\text{CHSH}}
\usepackage{booktabs}
\usepackage{array}
\usepackage{longtable}
\usepackage{float}
\usepackage{adjustbox}
\usepackage{rotating}
\usepackage{tabularx}
\usepackage{makecell}
\usepackage{multirow}

% === CHAPTER 8: DOCUMENT FORMATTING ===
\usepackage{fancyhdr}
\renewcommand{\headrulewidth}{0.4pt}
\renewcommand{\footrulewidth}{0.4pt}
\usepackage{tocloft}

\usepackage{enumitem}
\setlist[itemize]{leftmargin=*, topsep=2pt, partopsep=0pt, parsep=2pt, itemsep=2pt}
\setlist[enumerate]{leftmargin=*, topsep=2pt, partopsep=0pt, parsep=2pt, itemsep=2pt}
\usepackage{setspace}
\usepackage{ragged2e}
\usepackage{multicol}

% === CHAPTER 9: CODE AND ALGORITHMS ===
\usepackage{algorithm}
\usepackage{algorithmic}
\usepackage{listings}
\lstset{
	basicstyle=\ttfamily\footnotesize,
	breaklines=true,
	breakatwhitespace=true,
	columns=flexible,
	keepspaces=true,
	showstringspaces=false,
	frame=single,
	xleftmargin=0pt,
	xrightmargin=0pt,
	literate=              % For special characters in code listings
	{ä}{{\"a}}1 {ö}{{\"o}}1 {ü}{{\"u}}1 {ß}{{\ss}}1
	{Ä}{{\"A}}1 {Ö}{{\"O}}1 {Ü}{{\"U}}1
}
\usepackage{mdframed}

% === CHAPTER 10: ADDITIONAL PACKAGES ===
\usepackage{pdflscape}
\usepackage{braket}
\usepackage{cancel}
\usepackage{caption}
\captionsetup{format=plain, labelfont=bf, justification=centering}
\usepackage{csquotes}
\usepackage{gensymb}
\usepackage{textcomp}
\usepackage{textgreek}
\usepackage{upgreek}
\usepackage{url}
\usepackage{slashed}
\usepackage{bm}

% === CHAPTER 11: HYPERREF (must come SECOND TO LAST!) ===
\usepackage{hyperref}
\hypersetup{
	colorlinks=true,
	linkcolor=black,
	citecolor=black,
	urlcolor=black,
	breaklinks=true,           % IMPORTANT for special characters in URLs!
	bookmarksnumbered=true,
	unicode=true,
	pdfencoding=auto,
	pdflang=en,                % Set PDF language to English
	pdfsubject={T0 Theory - Fundamental Fractal-Geometric Field Theory}
}

% Fix for unicode-math symbols in PDF bookmarks
\pdfstringdefDisableCommands{%
	\def\xi{xi}%
	\def\alpha{alpha}%
	\def\beta{beta}%
	\def\gamma{gamma}%
	\def\delta{delta}%
	\def\Delta{Delta}%
	\def\epsilon{epsilon}%
	\def\varepsilon{epsilon}%
	\def\theta{theta}%
	\def\kappa{kappa}%
	\def\lambda{lambda}%
	\def\mu{mu}%
	\def\nu{nu}%
	\def\pi{pi}%
	\def\rho{rho}%
	\def\sigma{sigma}%
	\def\tau{tau}%
	\def\phi{phi}%
	\def\chi{chi}%
	\def\psi{psi}%
	\def\omega{omega}%
	\def\Omega{Omega}%
	\def\Lambda{Lambda}%
	\def\times{x}%
	\def\cdot{*}%
	\def\pm{+/-}%
	\def\approx{~}%
	\def\sim{~}%
	\def\equiv{=}%
	\def\ell{l}%
	\def\hbar{h}%
	\def\rightarrow{->}%
	\def\leftarrow{<-}%
	\def\Rightarrow{=>}%
	\def\Leftarrow{<=}%
	\def\propto{~}%
	\def\mitxi{xi}%
	\def\mitalpha{alpha}%
	\def\mitbeta{beta}%
	\def\mitgamma{gamma}%
	\def\mitdelta{delta}%
	\def\mitDelta{Delta}%
	\def\mitepsilon{epsilon}%
	\def\mitvarepsilon{epsilon}%
	\def\mittheta{theta}%
	\def\mitkappa{kappa}%
	\def\mitlambda{lambda}%
	\def\mitLambda{Lambda}%
	\def\mitmu{mu}%
	\def\mitnu{nu}%
	\def\mitpi{pi}%
	\def\mitrho{rho}%
	\def\mitsigma{sigma}%
	\def\mittau{tau}%
	\def\mitphi{phi}%
	\def\mitchi{chi}%
	\def\mitpsi{psi}%
	\def\mitomega{omega}%
	\def\mitOmega{Omega}%
}

% === CHAPTER 12: BOOKMARK (must come AFTER hyperref!) ===
\usepackage{bookmark}

% === CHAPTER 13: CLEVEREF (ENGLISH LABELS) ===
\usepackage[english]{cleveref}
\crefname{equation}{Equation}{Equations}
\crefname{figure}{Figure}{Figures}
\crefname{table}{Table}{Tables}
\crefname{section}{Section}{Sections}
\crefname{chapter}{Chapter}{Chapters}
\crefname{theorem}{Theorem}{Theorems}
\crefname{lemma}{Lemma}{Lemmas}
\crefname{definition}{Definition}{Definitions}
\crefname{example}{Example}{Examples}
\crefname{remark}{Remark}{Remarks}

% === CUSTOM ENVIRONMENTS ===
% Alternative interpretation environment
\newenvironment{alternative}{%
	\begin{mdframed}[linecolor=black!30,linewidth=1pt,roundcorner=4pt,backgroundcolor=black!5]%
	}{%
	\end{mdframed}%
}

% Photon/particle environment
\newenvironment{photon}{%
	\begin{mdframed}[linecolor=blue!30,linewidth=1pt,roundcorner=4pt,backgroundcolor=blue!5]%
	}{%
	\end{mdframed}%
}

% Koide formula box environment
\newenvironment{koidebox}{%
	\begin{mdframed}[linecolor=green!30,linewidth=1pt,roundcorner=4pt,backgroundcolor=green!5]%
	}{%
	\end{mdframed}%
}

% Erkenntnis/insight environment
\newenvironment{erkenntnis}{%
	\begin{mdframed}[linecolor=orange!30,linewidth=1pt,roundcorner=4pt,backgroundcolor=orange!5]%
	}{%
	\end{mdframed}%
}

% Beziehung/relationship environment
\newenvironment{beziehung}{%
	\begin{mdframed}[linecolor=purple!30,linewidth=1pt,roundcorner=4pt,backgroundcolor=purple!5]%
	}{%
	\end{mdframed}%
}

% Derivation environment
\newenvironment{derivation}{%
	\begin{mdframed}[linecolor=teal!30,linewidth=1pt,roundcorner=4pt,backgroundcolor=teal!5]%
	}{%
	\end{mdframed}%
}

% Abhandlung/treatise environment
\newenvironment{abhandlung}{%
	\begin{mdframed}[linecolor=brown!30,linewidth=1pt,roundcorner=4pt,backgroundcolor=brown!5]%
	}{%
	\end{mdframed}%
}

% Anwendung/application environment
\newenvironment{anwendung}{%
	\begin{mdframed}[linecolor=cyan!30,linewidth=1pt,roundcorner=4pt,backgroundcolor=cyan!5]%
	}{%
	\end{mdframed}%
}

% Additional common environments
\newenvironment{konsequenz}{%
	\begin{mdframed}[linecolor=red!30,linewidth=1pt,roundcorner=4pt,backgroundcolor=red!5]%
	}{%
	\end{mdframed}%
}

\newenvironment{schlussfolgerung}{%
	\begin{mdframed}[linecolor=gray!30,linewidth=1pt,roundcorner=4pt,backgroundcolor=gray!5]%
	}{%
	\end{mdframed}%
}

\newenvironment{result}{%
	\begin{mdframed}[linecolor=violet!30,linewidth=1pt,roundcorner=4pt,backgroundcolor=violet!5]%
	}{%
	\end{mdframed}%
}

% Formula environment
\newenvironment{formula}{%
	\begin{mdframed}[linecolor=yellow!30,linewidth=1pt,roundcorner=4pt,backgroundcolor=yellow!5]%
	}{%
	\end{mdframed}%
}

% Revolutionaer/revolutionary environment
\newenvironment{revolutionaer}{%
	\begin{mdframed}[linecolor=red!50,linewidth=2pt,roundcorner=4pt,backgroundcolor=red!10]%
	}{%
	\end{mdframed}%
}

% Formel environment (German version of formula)
\newenvironment{formel}{%
	\begin{mdframed}[linecolor=yellow!30,linewidth=1pt,roundcorner=4pt,backgroundcolor=yellow!5]%
	}{%
	\end{mdframed}%
}

% Prinzip/principle environment
\newenvironment{prinzip}{%
	\begin{mdframed}[linecolor=blue!50,linewidth=2pt,roundcorner=4pt,backgroundcolor=blue!10]%
	}{%
	\end{mdframed}%
}

% Experimentell/experimental environment
\newenvironment{experimentell}{%
	\begin{mdframed}[linecolor=magenta!30,linewidth=1pt,roundcorner=4pt,backgroundcolor=magenta!5]%
	}{%
	\end{mdframed}%
}

% Neutrino environment
\newenvironment{neutrino}{%
	\begin{mdframed}[linecolor=cyan!40,linewidth=1pt,roundcorner=4pt,backgroundcolor=cyan!8]%
	}{%
	\end{mdframed}%
}

% Additional missing environments
\newenvironment{schluessel}{%
	\begin{mdframed}[linecolor=yellow!50,linewidth=1pt,roundcorner=4pt,backgroundcolor=yellow!10]%
	}{%
	\end{mdframed}%
}

\newenvironment{summary}{%
	\begin{mdframed}[linecolor=gray!40,linewidth=1pt,roundcorner=4pt,backgroundcolor=gray!8]%
	}{%
	\end{mdframed}%
}

\newenvironment{category}{%
	\begin{mdframed}[linecolor=pink!40,linewidth=1pt,roundcorner=4pt,backgroundcolor=pink!8]%
	}{%
	\end{mdframed}%
}

\newenvironment{sibox}{%
	\begin{mdframed}[linecolor=lime!40,linewidth=1pt,roundcorner=4pt,backgroundcolor=lime!8]%
	}{%
	\end{mdframed}%
}

% More missing environments
\newenvironment{documentbox}{%
	\begin{mdframed}[linecolor=teal!40,linewidth=1pt,roundcorner=4pt,backgroundcolor=teal!8]%
	}{%
	\end{mdframed}%
}

\newenvironment{t0box}{%
	\begin{mdframed}[linecolor=violet!40,linewidth=1pt,roundcorner=4pt,backgroundcolor=violet!8]%
	}{%
	\end{mdframed}%
}

\newenvironment{wichtig}{%
	\begin{mdframed}[linecolor=red!50,linewidth=2pt,roundcorner=4pt,backgroundcolor=red!10]%
	\textbf{Important:} 
	}{%
	\end{mdframed}%
}

\newenvironment{smbox}{%
	\begin{mdframed}[linecolor=orange!40,linewidth=1pt,roundcorner=4pt,backgroundcolor=orange!8]%
	}{%
	\end{mdframed}%
}

\newenvironment{pvbox}{%
	\begin{mdframed}[linecolor=purple!40,linewidth=1pt,roundcorner=4pt,backgroundcolor=purple!8]%
	}{%
	\end{mdframed}%
}

\newenvironment{numerisch}{%
	\begin{mdframed}[linecolor=blue!40,linewidth=1pt,roundcorner=4pt,backgroundcolor=blue!8]%
	}{%
	\end{mdframed}%
}

% More missing environments
\newenvironment{relation}{%
	\begin{mdframed}[linecolor=green!40,linewidth=1pt,roundcorner=4pt,backgroundcolor=green!8]%
	}{%
	\end{mdframed}%
}

\newenvironment{beweis}{%
	\begin{mdframed}[linecolor=brown!40,linewidth=1pt,roundcorner=4pt,backgroundcolor=brown!8]%
	\textbf{Proof:} 
	}{%
	\end{mdframed}%
}

\newenvironment{revolution}{%
	\begin{mdframed}[linecolor=red!60,linewidth=2pt,roundcorner=4pt,backgroundcolor=red!12]%
	}{%
	\end{mdframed}%
}

\newenvironment{key}{%
	\begin{mdframed}[linecolor=yellow!50,linewidth=1pt,roundcorner=4pt,backgroundcolor=yellow!10]%
	}{%
	\end{mdframed}%
}

\newenvironment{newperspective}{%
	\begin{mdframed}[linecolor=cyan!50,linewidth=1pt,roundcorner=4pt,backgroundcolor=cyan!10]%
	}{%
	\end{mdframed}%
}

\newenvironment{literatur}{%
	\begin{mdframed}[linecolor=gray!50,linewidth=1pt,roundcorner=4pt,backgroundcolor=gray!10]%
	}{%
	\end{mdframed}%
}

\newenvironment{folgerung}{%
	\begin{mdframed}[linecolor=teal!50,linewidth=1pt,roundcorner=4pt,backgroundcolor=teal!10]%
	}{%
	\end{mdframed}%
}

\newenvironment{principle}{%
	\begin{mdframed}[linecolor=blue!60,linewidth=2pt,roundcorner=4pt,backgroundcolor=blue!12]%
	}{%
	\end{mdframed}%
}

% Additional common environments
% ==============================================================================
% FROM HERE: YOUR DEFINITIONS (unchanged)
% ==============================================================================

\setcounter{tocdepth}{3}

% === CITATION COMMANDS ===
\providecommand{\citep}[1]{\cite{#1}}
\providecommand{\citet}[1]{\cite{#1}}

% === COLORS ===
\definecolor{gold}{RGB}{255,215,0}
\definecolor{blue}{rgb}{0,0,1}
\definecolor{boxgray}{RGB}{240,240,240}
\definecolor{deepblue}{RGB}{0,0,127}
\definecolor{deepgreen}{RGB}{0,127,0}
\definecolor{deepred}{RGB}{191,0,0}
\definecolor{t0blue}{RGB}{33,150,243}
\definecolor{t0green}{RGB}{76,175,80}
\definecolor{t0orange}{RGB}{255,152,0}
\definecolor{t0purple}{RGB}{156,39,176}
\definecolor{t0red}{RGB}{244,67,54}
\definecolor{t0yellow}{RGB}{255,204,0}

% === COLUMN TYPES ===
\newcolumntype{L}[1]{>{\raggedright\arraybackslash}p{#1}}
\newcolumntype{C}[1]{>{\centering\arraybackslash}p{#1}}
\newcolumntype{R}[1]{>{\raggedleft\arraybackslash}p{#1}}

% === HYPERREF SETTINGS (updated) ===
\hypersetup{
	colorlinks=true,
	linkcolor=t0blue,
	citecolor=t0blue,
	urlcolor=t0blue,
	breaklinks=true,
	bookmarksnumbered=true,
	pdfstartview=FitH,
	pdfencoding=auto,
	pdfdisplaydoctitle=true
}

% === ENGLISH THEOREM ENVIRONMENTS ===
\theoremstyle{plain}
\newtheorem{theorem}{Theorem}[section]
\newtheorem{lemma}[theorem]{Lemma}
\newtheorem{proposition}[theorem]{Proposition}
\newtheorem{corollary}[theorem]{Corollary}

\theoremstyle{definition}
\newtheorem{definition}[theorem]{Definition}
\newtheorem{example}[theorem]{Example}
\newtheorem{insight}[theorem]{Insight}
\newtheorem{discovery}[theorem]{Discovery}

\theoremstyle{remark}
\newtheorem{remark}[theorem]{Remark}
\newtheorem{axiom}{Axiom}
%\newtheorem{principle}{Principle}  % Commented out to avoid conflicts with document-specific definitions
%\newtheorem{warning}[theorem]{Warning}

% === T0-SPECIFIC COMMANDS ===
% (Here follow all your \newcommand and \providecommand definitions)
% These remain UNCHANGED as in your original preamble
% ==============================================================================
% SECTION 14: T0-Specific Commands
% ==============================================================================

% --- Core T0 Fields ---
\newcommand{\Tfield}{T(x,t)}
\providecommand{\Tfieldt}{T(\vec{x},t)}
\newcommand{\Efield}{E(x,t)}
\newcommand{\mfield}{m(x,t)}
\providecommand{\vecx}{\vec{x}}

% --- Lagrangian ---
\newcommand{\Lag}{\mathcal{L}}
\newcommand{\calL}{\mathcal{L}}

% --- Greek Letters and Constants ---
\newcommand{\alphaem}{\alpha}
\newcommand{\betaT}{\beta_T}
\newcommand{\xiT}{\xi}
\newcommand{\xipar}{\xi}

% --- Energy and Planck Units ---
\newcommand{\Ezero}{E_0}
\newcommand{\E}{E}
\newcommand{\EPlanck}{E_{\text{Pl}}}
\newcommand{\Mpl}{M_{\text{Pl}}}
\newcommand{\mP}{m_{\text{P}}}
\newcommand{\lP}{\ell_{\text{P}}}
\newcommand{\tP}{t_{\text{P}}}
\newcommand{\LPlanck}{\ell_{\text{Pl}}}
\newcommand{\TPlanck}{t_{\text{Pl}}}

% --- Coupling Constants ---
\newcommand{\Gnat}{G_{\text{nat}}}
\newcommand{\alphaEM}{\alpha_{\text{EM}}}
\newcommand{\alphaSI}{\alpha_{\text{SI}}}
\newcommand{\Hubble}{H_0}
\newcommand{\LCDM}{\Lambda\text{CDM}}
\newcommand{\natunits}{(nat. units)}

% --- T0 Model Parameters ---
\newcommand{\xigeom}{\xi_{\mathrm{geom}}}
\newcommand{\rzero}{r_{0}}
\newcommand{\xirat}{\xi_{\mathrm{rat}}}
\newcommand{\tzero}{t_{0}}
\newcommand{\Lambdat}{\Lambda_{\mathrm{t}}}
\newcommand{\EP}{E_{\text{P}}}
\newcommand{\Emu}{E_{\mu}}
\newcommand{\Ee}{E_{e}}
\newcommand{\Etau}{E_{\tau}}
\newcommand{\alphafine}{\alpha_{\mathrm{fine}}}
\newcommand{\alphal}{\alpha_{\ell}}
\newcommand{\Lzero}{\ell_{0}}
\newcommand{\Lp}{\ell_{\mathrm{P}}}

% --- Additional T0 Commands ---
\newcommand{\Kfrak}{K_{\text{frak}}}
\newcommand{\Dfrak}{D_{\text{frak}}}
\newcommand{\betapar}{\ensuremath{\beta_T}}
\newcommand{\alphapar}{\alpha}
\newcommand{\deltafield}{\delta \phi}
\newcommand{\deltam}{\delta m}
\newcommand{\deltaE}{\delta E}
\newcommand{\Exi}{E_{\xi}}
\newcommand{\Lxi}{\ell_{\xi}}
\newcommand{\rhoCMB}{\rho_{\text{CMB}}}
\newcommand{\rhoCasimir}{\rho_{\text{Casimir}}}
\newcommand{\Leff}{L_{\text{eff}}}
\newcommand{\CQCD}{C_{\mathrm{QCD}}}
\newcommand{\Kspec}{K_{\mathrm{spec}}}
\newcommand{\Tzero}{\ensuremath{T_0}}
\newcommand{\Eabs}{E_{\text{abs}}}
\newcommand{\taupar}{\tau}

% --- Provided Commands ---
\providecommand{\xiconst}{\xi_{\text{const}}}
\providecommand{\DhiggsT}{D_{\text{Higgs-T}}}
\providecommand{\rhoE}{\rho_{E}}
\providecommand{\Echar}{E_{\text{char}}}
\providecommand{\kfrac}{k_{\text{frac}}}
\providecommand{\alphaEMSI}{\alpha_{\text{EM,SI}}}
\providecommand{\alphaEMnat}{\alpha_{\text{EM,nat}}}
\providecommand{\betaTSI}{\beta_{T,\text{SI}}}
\providecommand{\betaTnat}{\beta_{T,\text{nat}}}
\providecommand{\Gsi}{G_{\text{SI}}}
\providecommand{\xiparSI}{\xi_{\text{SI}}}
\providecommand{\xiparnat}{\xi_{\text{nat}}}
\providecommand{\meff}{m_{\text{eff}}}
\providecommand{\Tzerot}{T_{0}(t)}
\providecommand{\mzerot}{m_{0}(t)}
\providecommand{\Ezeroabs}{E_{0,\text{abs}}}
\providecommand{\Epar}{E_{\text{par}}}
\providecommand{\Lnat}{\ell_{\text{nat}}}
\providecommand{\Tnat}{T_{\text{nat}}}
\providecommand{\xifrak}{\xi_{\text{frac}}}
\providecommand{\Tfrak}{T_{\text{frac}}}
\providecommand{\mfrak}{m_{\text{frac}}}
\providecommand{\Dfrac}{D_{\text{frac}}}
\providecommand{\EphotSI}{E_{\gamma,\text{SI}}}
\providecommand{\EphotNat}{E_{\gamma,\text{nat}}}
\providecommand{\Eabsint}{E_{\text{abs,int}}}
\providecommand{\mphoton}{m_{\gamma}}
\providecommand{\Evis}{E_{\text{vis}}}
\providecommand{\Cto}{C_{T0}}
\providecommand{\mytimes}{\times}
\providecommand{\lambdah}{\lambda_h}
\providecommand{\checkmarkx}{\checkmark}
\providecommand{\Enorm}{E_{\text{norm}}}
\providecommand{\Tobs}{T_{\text{obs}}}
\providecommand{\mobs}{m_{\text{obs}}}
\providecommand{\Eobs}{E_{\text{obs}}}
\providecommand{\Lobs}{\ell_{\text{obs}}}
\providecommand{\xobs}{\xi_{\text{obs}}}
\providecommand{\calE}{\mathcal{E}}
\providecommand{\calT}{\mathcal{T}}
\providecommand{\calM}{\mathcal{M}}
\providecommand{\alphag}{\alpha_g}
\providecommand{\Tmax}{T_{\text{max}}}
\providecommand{\mmin}{m_{\text{min}}}
\providecommand{\Lmax}{\ell_{\text{max}}}
\providecommand{\Emin}{E_{\text{min}}}
\providecommand{\Geff}{G_{\text{eff}}}
\providecommand{\rhoeff}{\rho_{\text{eff}}}
\providecommand{\xieff}{\xi_{\text{eff}}}
\providecommand{\Teff}{T_{\text{eff}}}
\providecommand{\hPlanck}{h}
\providecommand{\kB}{k_B}
\providecommand{\muB}{\mu_B}
\providecommand{\lambdaC}{\lambda_C}
\providecommand{\omegaP}{\omega_P}
\providecommand{\rhoP}{\rho_P}
\providecommand{\Tref}{T_{\text{ref}}}
\providecommand{\Eref}{E_{\text{ref}}}
\providecommand{\mref}{m_{\text{ref}}}
\providecommand{\Lref}{\ell_{\text{ref}}}
\providecommand{\xikonst}{\xi_0}
\providecommand{\Phiphoton}{\Phi_{\gamma}}
\providecommand{\etavis}{\eta_{\text{vis}}}
\providecommand{\pichar}{\pi}
\providecommand{\primrel}{\mathcal{P}_{\text{rel}}}
\providecommand{\warningx}{\textcolor{orange}{\textbf{!}}}
\providecommand{\phiT}{\phi_T}
\providecommand{\Lorentz}{\Lambda}
\providecommand{\Cconv}{C_{\text{conv}}}
\providecommand{\Df}{\Delta f}
\providecommand{\lambdazero}{\lambda_0}
\providecommand{\myapprox}{\approx}
\providecommand{\checked}{\checkmark}
\providecommand{\alphaWSI}{\alpha_W^{\text{SI}}}
\providecommand{\alphaWnat}{\alpha_W^{\text{nat}}}
\providecommand{\vect}[1]{\vec{#1}}
\providecommand{\Rzero}{R_0}
\providecommand{\Riem}{\mathcal{R}}
\providecommand{\nuzero}{\nu_0}
\providecommand{\mypi}{\pi}

% =============================================================================
% TCOLORBOX STYLES AND ENVIRONMENTS (English titles)
% =============================================================================
\tcbset{
	keyresult/.style={
		colback=blue!5!white,
		colframe=blue!75!black,
		title=Key Result,
		fonttitle=\bfseries
	},
	foundation/.style={
		colback=green!5!white,
		colframe=green!75!black,
		title=Foundation,
		fonttitle=\bfseries
	},
	alternative/.style={
		colback=orange!5!white,
		colframe=orange!75!black,
		title=Alternative,
		fonttitle=\bfseries
	},
	warningbox/.style={
		colback=red!5!white,
		colframe=red!75!black,
		title=Warning,
		fonttitle=\bfseries
	}
}

% (Here follow all your tcolorbox definitions with English titles)
\newtcolorbox{keyresultbox}[1][]{colback=blue!5!white,colframe=blue!75!black,fonttitle=\bfseries,title={#1},breakable}
\newtcolorbox{keyresult}[1][Key Result]{colback=blue!5!white,colframe=blue!75!black,fonttitle=\bfseries,title={#1},breakable}
\newtcolorbox{foundationbox}[1][]{colback=green!5!white,colframe=green!75!black,fonttitle=\bfseries,title={#1},breakable}
\newtcolorbox{foundation}[1][Foundation]{colback=green!5!white,colframe=green!75!black,fonttitle=\bfseries,title={#1},breakable}
\newtcolorbox{alternativebox}[1][]{colback=orange!5!white,colframe=orange!75!black,fonttitle=\bfseries,title={#1},breakable}
\newtcolorbox{warningboxenv}[1][Warning]{colback=red!5!white,colframe=red!75!black,fonttitle=\bfseries,title={#1},breakable}

\newtcolorbox{fundamental}[1][]{
	colback=boxgray,
	colframe=t0blue,
	fonttitle=\bfseries,
	title=#1,
	sharp corners,
	boxrule=2pt
}

\newtcolorbox{insightBox}[1][Insight]{colback=blue!5,colframe=t0blue,title={#1},fonttitle=\bfseries,breakable}
\newtcolorbox{discoveryBox}[1][Discovery]{colback=green!5,colframe=t0green,title={#1},fonttitle=\bfseries,breakable}
\newtcolorbox{revelation}[1][Revelation]{colback=red!5,colframe=t0red,title={#1},fonttitle=\bfseries,breakable}
\newtcolorbox{keypoint}[1][Key Point]{colback=blue!5,colframe=t0blue,title={#1},fonttitle=\bfseries,breakable}
\newtcolorbox{evidence}[1][Evidence]{colback=green!5,colframe=t0green,title={#1},fonttitle=\bfseries,breakable}
\newtcolorbox{conclusionBox}[1][Conclusion]{colback=gray!5,colframe=gray,title={#1},fonttitle=\bfseries,breakable}
\newtcolorbox{significance}[1][Significance]{colback=yellow!5,colframe=orange,title={#1},fonttitle=\bfseries,breakable}
\newtcolorbox{philosophical}[1][Philosophical]{colback=purple!5,colframe=purple,title={#1},fonttitle=\bfseries,breakable}
\newtcolorbox{implicationBox}[1][Implication]{colback=cyan!5,colframe=cyan,title={#1},fonttitle=\bfseries,breakable}
\newtcolorbox{perspectiveBox}[1][Perspective]{colback=blue!5,colframe=t0blue,title={#1},fonttitle=\bfseries,breakable}
\newtcolorbox{revolutionary}[1][Revolutionary]{colback=red!5,colframe=t0red,title={#1},fonttitle=\bfseries,breakable}

\newtcolorbox{technical}[1][Technical]{colback=gray!5,colframe=gray!75!black,title={#1},fonttitle=\bfseries,breakable}
\newtcolorbox{technicalBox}[1][Technical]{colback=gray!5,colframe=gray!75!black,title={#1},fonttitle=\bfseries,breakable}
\newtcolorbox{notationBox}[1][Notation]{colback=yellow!5,colframe=yellow!75!black,title={#1},fonttitle=\bfseries,breakable}
\newtcolorbox{verification}[1][Verification]{colback=orange!5!white,colframe=orange!75!black,fonttitle=\bfseries,title=#1}
\newtcolorbox{explanationBox}[1][Explanation]{colback=purple!5!white,colframe=purple!75!black,fonttitle=\bfseries,title=#1}
\newtcolorbox{interpretationBox}[1][Interpretation]{colback=cyan!5!white,colframe=cyan!75!black,fonttitle=\bfseries,title=#1}
\newtcolorbox{explanation}[1][Explanation]{colback=purple!5!white,colframe=purple!75!black,fonttitle=\bfseries,title=#1,breakable}
\newtcolorbox{interpretation}[1][Interpretation]{colback=cyan!5!white,colframe=cyan!75!black,fonttitle=\bfseries,title=#1,breakable}
\newtcolorbox{proof_step}[1][Proof Step]{colback=gray!5!white,colframe=gray!75!black,fonttitle=\bfseries,title=#1,breakable}
\newtcolorbox{experimental}[1][Experimental]{colback=teal!5!white,colframe=teal!75!black,fonttitle=\bfseries,title=#1,breakable}

\newtcolorbox{important}[1][Important]{colback=red!5!white,colframe=red!75!black,title={#1},fonttitle=\bfseries,breakable}
\newtcolorbox{warning}[1][Warning]{colback=orange!5!white,colframe=orange!75!black,title={#1},fonttitle=\bfseries,breakable}
\newtcolorbox{caution}[1][Caution]{colback=yellow!5!white,colframe=yellow!75!black,title={#1},fonttitle=\bfseries,breakable}
\newtcolorbox{highlight}[1][Highlight]{colback=yellow!10!white,colframe=yellow!75!black,title={#1},fonttitle=\bfseries,breakable}
\newtcolorbox{critical}[1][Critical]{colback=red!10!white,colframe=red!75!black,title={#1},fonttitle=\bfseries,breakable}

\newtcolorbox{analysis}[1][Analysis]{colback=blue!5!white,colframe=blue!75!black,title={#1},fonttitle=\bfseries,breakable}
\newtcolorbox{application}[1][Application]{colback=green!5!white,colframe=green!75!black,title={#1},fonttitle=\bfseries,breakable}
\newtcolorbox{experiment}[1][Experiment]{colback=cyan!5!white,colframe=cyan!75!black,title={#1},fonttitle=\bfseries,breakable}
\newtcolorbox{historical}[1][Historical]{colback=brown!5!white,colframe=brown!75!black,title={#1},fonttitle=\bfseries,breakable}
\newtcolorbox{numerical}[1][Numerical]{colback=gray!5!white,colframe=gray!75!black,title={#1},fonttitle=\bfseries,breakable}
\newtcolorbox{overview}[1][Overview]{colback=blue!5!white,colframe=blue!75!black,title={#1},fonttitle=\bfseries,breakable}
\newtcolorbox{speculation}[1][Speculation]{colback=purple!5!white,colframe=purple!75!black,title={#1},fonttitle=\bfseries,breakable}
\newtcolorbox{question}[1][Question]{colback=orange!5!white,colframe=orange!75!black,title={#1},fonttitle=\bfseries,breakable}
\newtcolorbox{method}[1][Method]{colback=teal!5!white,colframe=teal!75!black,title={#1},fonttitle=\bfseries,breakable}
\newtcolorbox{correct}[1][Correct]{colback=green!10!white,colframe=green!75!black,title={#1},fonttitle=\bfseries,breakable}
\newtcolorbox{units}[1][Units]{colback=gray!5!white,colframe=gray!75!black,title={#1},fonttitle=\bfseries,breakable}
\newtcolorbox{achievement}[1][Achievement]{colback=gold!5!white,colframe=orange!75!black,title={#1},fonttitle=\bfseries,breakable}
\newtcolorbox{equivalence}[1][Equivalence]{colback=cyan!5!white,colframe=cyan!75!black,title={#1},fonttitle=\bfseries,breakable}
\newtcolorbox{dimensional}[1][Dimensional Analysis]{colback=purple!5!white,colframe=purple!75!black,title={#1},fonttitle=\bfseries,breakable}

% === ADDITIONAL SIMPLE ENVIRONMENTS ===
\newenvironment{treatise}{\begin{quote}}{\end{quote}}
\newenvironment{gemeinsam}{\begin{quote}}{\end{quote}}
\newenvironment{vergleich}{\begin{quote}}{\end{quote}}
\newenvironment{vorteil}{\begin{quote}}{\end{quote}}
\newenvironment{common}{\begin{quote}}{\end{quote}}
\newenvironment{comparison}{\begin{quote}}{\end{quote}}
\newenvironment{advantage}{\begin{quote}}{\end{quote}}
\newenvironment{quantum}{\begin{quote}}{\end{quote}}

% === LAYOUT SETTINGS ===
\raggedbottom
\usepackage{environ}
\let\oldtabular\tabular
\let\endoldtabular\endtabular

\newenvironment{scaledtable}[1][0.85]{%
	\begingroup\footnotesize\setlength{\LTleft}{0pt}\setlength{\LTright}{0pt}%
}{%
	\endgroup%
}

\newcommand{\widetable}[1]{\resizebox{\textwidth}{!}{#1}}

% === TABLE OF CONTENTS FORMATTING ===
\renewcommand{\cftsecfont}{\color{blue}}
\renewcommand{\cftsubsecfont}{\color{blue}}
\renewcommand{\cftsecpagefont}{\color{blue}}
\renewcommand{\cftsubsecpagefont}{\color{blue}}
\renewcommand{\cfttoctitlefont}{\huge\bfseries\color{blue}}

% === DEFAULT HEADER AND FOOTER ===
\pagestyle{fancy}
\fancyhf{}
\fancyhead[L]{\textsc{T0 Theory}}
\fancyhead[R]{\textsc{J. Pascher}}
\fancyfoot[C]{\thepage}

% ==============================================================================
% End of Shared Preamble for English
% ==============================================================================
%
% Usage:
%   \documentclass[12pt,a4paper]{article}  % or book, report, etc.
%   % ==============================================================================
% T0 Theory: Shared ENGLISH Preamble – Optimized for eBook/Book
% Version: 2.0 – Final 2026 (LuaLaTeX only) – ENGLISH corrected
% Author: Johann Pascher
% Date: January 2026
% ==============================================================================
%
% IMPORTANT: Compile EXCLUSIVELY with LuaLaTeX!
% In TeXstudio: Options → Configure TeXstudio → Build → Default Compiler → LuaLaTeX
%
% Required Fonts (install once):
% - Inter: https://fonts.google.com/specimen/Inter
% - JetBrains Mono: https://www.jetbrains.com/lp/mono/
% - Libertinus Math: https://github.com/libertinus-fonts/libertinus
% ==============================================================================

% === CHAPTER 1: BASIC PACKAGES (must come FIRST) ===
\RequirePackage{fontspec}
\RequirePackage{unicode-math}
\usepackage{chngcntr}
\setcounter{secnumdepth}{1}  % Nur Sections nummerieren (nicht subsections)
\setcounter{tocdepth}{1}     % Nur Sections im TOC (nicht subsections)
\makeatletter
\@ifundefined{c@chapter}{}{\counterwithout{section}{chapter}}  % Falls Kapitel existieren
\makeatother
\counterwithout{subsection}{section}  % Löse Verknüpfung
% === CHAPTER 2: LANGUAGE (ENGLISH) ===
\usepackage[english]{babel}
\usepackage{microtype}                    % IMPORTANT for better hyphenation!

% Typography settings for better line breaking
\frenchspacing                     % Correct English spacing after punctuation
\emergencystretch=3em              % Allows more stretch for difficult lines
\tolerance=2500                    % Higher tolerance for line breaks
\hbadness=10000                    % Suppresses "underfull hbox" warnings
\hfuzz=2pt                         % Allows minimal overfull
\pretolerance=150                  % Better word breaking

% Prevent bad page breaks
\clubpenalty=10000           % No "orphans"
\widowpenalty=10000          % No "widows"
\displaywidowpenalty=10000   % Also with equations
\brokenpenalty=10000         % No broken words across pages

% Explicit hyphenation for long technical words
\hyphenation{Fun-da-men-tal Frac-tal-Ge-o-met-ric Field The-o-ry Meth-od-o-log-i-cal}
\hyphenation{Re-vi-sion-ism Quan-ti-za-tion U-ni-fi-ca-tion Ef-fec-tive}
\hyphenation{Re-nor-mal-iz-a-bil-i-ty Sin-gu-lar-i-ties Con-cil-i-a-tion}
\hyphenation{E-mer-gence Phe-nom-e-no-log-i-cal Doc-u-men-ta-tion A-nal-y-sis}
\hyphenation{Grav-i-ta-tion Quan-tum Me-chan-ics Dog-ma-tism Con-se-quent}
\hyphenation{Par-al-lel-ism Im-ple-men-ta-tion Per-tur-ba-tions}
\hyphenation{Geo-met-ric Ar-ti-fact In-com-pat-i-bil-i-ty Con-struc-tive}
\hyphenation{Frac-tal Di-men-sion-less In-ves-ti-ga-tion De-scrip-tion}
\hyphenation{In-ter-pre-ta-tion Phe-nom-e-no-log-i-cal Math-e-mat-i-cal}
\hyphenation{Phi-lo-soph-i-cal Le-git-i-ma-tion Ap-pli-ca-tion Der-i-va-tion}
\hyphenation{U-ni-fi-ca-tion As-sump-tion Con-cep-tion Ex-pec-ta-tion}
\hyphenation{Sym-me-try-ex-ten-sion O-ver-all-pic-ture Chal-lenge}
\hyphenation{In-ter-ac-tion Ma-te-ri-al Ap-proach Per-spec-tive Pro-ce-dure}

% === CHAPTER 3: FONTS (with proper ligatures) ===
\setmainfont{Inter}[
Scale=1.02,
UprightFont=*-Regular,
BoldFont=*-Bold,
ItalicFont=*-Italic,
BoldItalicFont=*-BoldItalic,
Ligatures=TeX,           % IMPORTANT for proper typography
Language=English         % Explicit language support
]
\setsansfont{Inter}[
Scale=MatchLowercase,
Ligatures=TeX,
Language=English
]
\setmonofont{JetBrains Mono}[
Scale=0.95,
Language=English
]

% Math Font (simple & stable) – MUST come AFTER language definition
% IMPORTANT: Libertinus Math for correct \underbrace display!
\setmathfont{Libertinus Math}[Scale=1.0]

% === CHAPTER 4: MATHEMATICS PACKAGES (in STRICT order!) ===
% IMPORTANT: mathtools must come BEFORE unicode-math for some commands!
\usepackage{mathtools}           % FIRST mathtools!

% Then the rest
\usepackage{amsmath, amsfonts, amsthm}

% SIUNITX MUST be loaded BEFORE physics!
\usepackage{siunitx}
\sisetup{
	locale=US,                    % ENGLISH settings for SI units!
	group-separator={,},          % Thousands separator comma
	output-decimal-marker={.},    % Decimal separator point
	per-mode=symbol,
	separate-uncertainty=true
}

% Custom SI units used in narrative and books
\DeclareSIUnit\gigalightyear{Gly}
\DeclareSIUnit\mev{MeV}

% physics – MUST be loaded AFTER siunitx and mathtools
\usepackage{physics}

% === CHAPTER 5: ADDITIONS from pdflatex best practices ===
\usepackage{colortbl}        % Colored tables (ESSENTIAL!)
\usepackage{placeins}        % Float control: \FloatBarrier
\usepackage{subcaption}      % Subfigures
\usepackage{xurl}            % Better URL line breaking
% Hyphenation for URLs in bibliography
\def\UrlBreaks{\do\/\do-}

% === CHAPTER 6: PAGE LAYOUT
% =============================================================================
% SECTION 2: Page Geometry – 6" × 9" Buchformat
% =============================================================================
\usepackage[paperwidth=6in, paperheight=9in,
top=0.9in,
bottom=1.1in,
inner=0.9in,            % Größerer Innenrand für Bindung
outer=0.6in,            % Kleinerer Außenrand → mehr Text pro Seite
bindingoffset=0.5in,    % Puffer für Bindung (Steg)
twoside]{geometry}
\setlength{\headheight}{15pt}
%\usepackage[paperwidth=8.25in, paperheight=11in,
%top=1.0in,
%bottom=1.0in,
%left=1.0in,
%right=1.0in,
%twoside=false
% === CHAPTER 7: GRAPHICS AND TABLES ===
\usepackage{graphicx}
\usepackage[table,xcdraw]{xcolor}
% T0 brand colors
\definecolor{gold}{RGB}{255,215,0}
\definecolor{blue}{rgb}{0,0,1}
\definecolor{boxgray}{RGB}{240,240,240}
\definecolor{deepblue}{RGB}{0,0,127}
\definecolor{deepgreen}{RGB}{0,127,0}
\definecolor{deepred}{RGB}{191,0,0}
\definecolor{t0blue}{RGB}{33,150,243}
\definecolor{t0green}{RGB}{76,175,80}
\definecolor{t0orange}{RGB}{255,152,0}
\definecolor{t0purple}{RGB}{156,39,176}
\definecolor{t0red}{RGB}{244,67,54}
\definecolor{t0yellow}{RGB}{255,204,0}
\usepackage{tikz}
\usetikzlibrary{arrows.meta,positioning,shapes.geometric,decorations.pathmorphing,patterns,shapes.arrows,intersections}
\usepackage{pgfplots}
\pgfplotsset{compat=1.18}
\usepackage{quantikz}
\usepackage[most]{tcolorbox}
\tcbuselibrary{breakable}

% === WICHTIG: Algorithm-Konflikt umgehen ===
% Option: algorithmic mit GROSSBUCHSTABEN
% Gemeinsame Box für Experimente
\newtcolorbox{experimentbox}[1][]{
	colback=green!5!white,
	colframe=t0green!80!black,
	fonttitle=\bfseries,
	title={{#1}},
	breakable
}

% Abstract-Fallback
\ifdefined\abstract\else
\newenvironment{abstract}{\section*{\abstractname}\itshape\small\par\bigskip}{\bigskip}
\fi

% === MAKROS SICHER NEU DEFINIEREN / ÜBERSCHREIBEN ===
% Definiere Makros OHNE doppelte Subskripte
\newcommand{\phipar}{\phi_{\mathrm{par}}}
%\newcommand{\xipar}{\xi_{\mathrm{par}}}
\newcommand{\Qphipar}{Q_{\phi_{\mathrm{par}}}}
\newcommand{\rphipar}{r_{\phi_{\mathrm{par}}}}
\newcommand{\logphipar}{\log_{\phi_{\mathrm{par}}}}
\newcommand{\CHSH}{\text{CHSH}}
\usepackage{booktabs}
\usepackage{array}
\usepackage{longtable}
\usepackage{float}
\usepackage{adjustbox}
\usepackage{rotating}
\usepackage{tabularx}
\usepackage{makecell}
\usepackage{multirow}

% === CHAPTER 8: DOCUMENT FORMATTING ===
\usepackage{fancyhdr}
\renewcommand{\headrulewidth}{0.4pt}
\renewcommand{\footrulewidth}{0.4pt}
\usepackage{tocloft}

\usepackage{enumitem}
\setlist[itemize]{leftmargin=*, topsep=2pt, partopsep=0pt, parsep=2pt, itemsep=2pt}
\setlist[enumerate]{leftmargin=*, topsep=2pt, partopsep=0pt, parsep=2pt, itemsep=2pt}
\usepackage{setspace}
\usepackage{ragged2e}
\usepackage{multicol}

% === CHAPTER 9: CODE AND ALGORITHMS ===
\usepackage{algorithm}
\usepackage{algorithmic}
\usepackage{listings}
\lstset{
	basicstyle=\ttfamily\footnotesize,
	breaklines=true,
	breakatwhitespace=true,
	columns=flexible,
	keepspaces=true,
	showstringspaces=false,
	frame=single,
	xleftmargin=0pt,
	xrightmargin=0pt,
	literate=              % For special characters in code listings
	{ä}{{\"a}}1 {ö}{{\"o}}1 {ü}{{\"u}}1 {ß}{{\ss}}1
	{Ä}{{\"A}}1 {Ö}{{\"O}}1 {Ü}{{\"U}}1
}
\usepackage{mdframed}

% === CHAPTER 10: ADDITIONAL PACKAGES ===
\usepackage{pdflscape}
\usepackage{braket}
\usepackage{cancel}
\usepackage{caption}
\captionsetup{format=plain, labelfont=bf, justification=centering}
\usepackage{csquotes}
\usepackage{gensymb}
\usepackage{textcomp}
\usepackage{textgreek}
\usepackage{upgreek}
\usepackage{url}
\usepackage{slashed}
\usepackage{bm}

% === CHAPTER 11: HYPERREF (must come SECOND TO LAST!) ===
\usepackage{hyperref}
\hypersetup{
	colorlinks=true,
	linkcolor=black,
	citecolor=black,
	urlcolor=black,
	breaklinks=true,           % IMPORTANT for special characters in URLs!
	bookmarksnumbered=true,
	unicode=true,
	pdfencoding=auto,
	pdflang=en,                % Set PDF language to English
	pdfsubject={T0 Theory - Fundamental Fractal-Geometric Field Theory}
}

% Fix for unicode-math symbols in PDF bookmarks
\pdfstringdefDisableCommands{%
	\def\xi{xi}%
	\def\alpha{alpha}%
	\def\beta{beta}%
	\def\gamma{gamma}%
	\def\delta{delta}%
	\def\Delta{Delta}%
	\def\epsilon{epsilon}%
	\def\varepsilon{epsilon}%
	\def\theta{theta}%
	\def\kappa{kappa}%
	\def\lambda{lambda}%
	\def\mu{mu}%
	\def\nu{nu}%
	\def\pi{pi}%
	\def\rho{rho}%
	\def\sigma{sigma}%
	\def\tau{tau}%
	\def\phi{phi}%
	\def\chi{chi}%
	\def\psi{psi}%
	\def\omega{omega}%
	\def\Omega{Omega}%
	\def\Lambda{Lambda}%
	\def\times{x}%
	\def\cdot{*}%
	\def\pm{+/-}%
	\def\approx{~}%
	\def\sim{~}%
	\def\equiv{=}%
	\def\ell{l}%
	\def\hbar{h}%
	\def\rightarrow{->}%
	\def\leftarrow{<-}%
	\def\Rightarrow{=>}%
	\def\Leftarrow{<=}%
	\def\propto{~}%
	\def\mitxi{xi}%
	\def\mitalpha{alpha}%
	\def\mitbeta{beta}%
	\def\mitgamma{gamma}%
	\def\mitdelta{delta}%
	\def\mitDelta{Delta}%
	\def\mitepsilon{epsilon}%
	\def\mitvarepsilon{epsilon}%
	\def\mittheta{theta}%
	\def\mitkappa{kappa}%
	\def\mitlambda{lambda}%
	\def\mitLambda{Lambda}%
	\def\mitmu{mu}%
	\def\mitnu{nu}%
	\def\mitpi{pi}%
	\def\mitrho{rho}%
	\def\mitsigma{sigma}%
	\def\mittau{tau}%
	\def\mitphi{phi}%
	\def\mitchi{chi}%
	\def\mitpsi{psi}%
	\def\mitomega{omega}%
	\def\mitOmega{Omega}%
}

% === CHAPTER 12: BOOKMARK (must come AFTER hyperref!) ===
\usepackage{bookmark}

% === CHAPTER 13: CLEVEREF (ENGLISH LABELS) ===
\usepackage[english]{cleveref}
\crefname{equation}{Equation}{Equations}
\crefname{figure}{Figure}{Figures}
\crefname{table}{Table}{Tables}
\crefname{section}{Section}{Sections}
\crefname{chapter}{Chapter}{Chapters}
\crefname{theorem}{Theorem}{Theorems}
\crefname{lemma}{Lemma}{Lemmas}
\crefname{definition}{Definition}{Definitions}
\crefname{example}{Example}{Examples}
\crefname{remark}{Remark}{Remarks}

% === CUSTOM ENVIRONMENTS ===
% Alternative interpretation environment
\newenvironment{alternative}{%
	\begin{mdframed}[linecolor=black!30,linewidth=1pt,roundcorner=4pt,backgroundcolor=black!5]%
	}{%
	\end{mdframed}%
}

% Photon/particle environment
\newenvironment{photon}{%
	\begin{mdframed}[linecolor=blue!30,linewidth=1pt,roundcorner=4pt,backgroundcolor=blue!5]%
	}{%
	\end{mdframed}%
}

% Koide formula box environment
\newenvironment{koidebox}{%
	\begin{mdframed}[linecolor=green!30,linewidth=1pt,roundcorner=4pt,backgroundcolor=green!5]%
	}{%
	\end{mdframed}%
}

% Erkenntnis/insight environment
\newenvironment{erkenntnis}{%
	\begin{mdframed}[linecolor=orange!30,linewidth=1pt,roundcorner=4pt,backgroundcolor=orange!5]%
	}{%
	\end{mdframed}%
}

% Beziehung/relationship environment
\newenvironment{beziehung}{%
	\begin{mdframed}[linecolor=purple!30,linewidth=1pt,roundcorner=4pt,backgroundcolor=purple!5]%
	}{%
	\end{mdframed}%
}

% Derivation environment
\newenvironment{derivation}{%
	\begin{mdframed}[linecolor=teal!30,linewidth=1pt,roundcorner=4pt,backgroundcolor=teal!5]%
	}{%
	\end{mdframed}%
}

% Abhandlung/treatise environment
\newenvironment{abhandlung}{%
	\begin{mdframed}[linecolor=brown!30,linewidth=1pt,roundcorner=4pt,backgroundcolor=brown!5]%
	}{%
	\end{mdframed}%
}

% Anwendung/application environment
\newenvironment{anwendung}{%
	\begin{mdframed}[linecolor=cyan!30,linewidth=1pt,roundcorner=4pt,backgroundcolor=cyan!5]%
	}{%
	\end{mdframed}%
}

% Additional common environments
\newenvironment{konsequenz}{%
	\begin{mdframed}[linecolor=red!30,linewidth=1pt,roundcorner=4pt,backgroundcolor=red!5]%
	}{%
	\end{mdframed}%
}

\newenvironment{schlussfolgerung}{%
	\begin{mdframed}[linecolor=gray!30,linewidth=1pt,roundcorner=4pt,backgroundcolor=gray!5]%
	}{%
	\end{mdframed}%
}

\newenvironment{result}{%
	\begin{mdframed}[linecolor=violet!30,linewidth=1pt,roundcorner=4pt,backgroundcolor=violet!5]%
	}{%
	\end{mdframed}%
}

% Formula environment
\newenvironment{formula}{%
	\begin{mdframed}[linecolor=yellow!30,linewidth=1pt,roundcorner=4pt,backgroundcolor=yellow!5]%
	}{%
	\end{mdframed}%
}

% Revolutionaer/revolutionary environment
\newenvironment{revolutionaer}{%
	\begin{mdframed}[linecolor=red!50,linewidth=2pt,roundcorner=4pt,backgroundcolor=red!10]%
	}{%
	\end{mdframed}%
}

% Formel environment (German version of formula)
\newenvironment{formel}{%
	\begin{mdframed}[linecolor=yellow!30,linewidth=1pt,roundcorner=4pt,backgroundcolor=yellow!5]%
	}{%
	\end{mdframed}%
}

% Prinzip/principle environment
\newenvironment{prinzip}{%
	\begin{mdframed}[linecolor=blue!50,linewidth=2pt,roundcorner=4pt,backgroundcolor=blue!10]%
	}{%
	\end{mdframed}%
}

% Experimentell/experimental environment
\newenvironment{experimentell}{%
	\begin{mdframed}[linecolor=magenta!30,linewidth=1pt,roundcorner=4pt,backgroundcolor=magenta!5]%
	}{%
	\end{mdframed}%
}

% Neutrino environment
\newenvironment{neutrino}{%
	\begin{mdframed}[linecolor=cyan!40,linewidth=1pt,roundcorner=4pt,backgroundcolor=cyan!8]%
	}{%
	\end{mdframed}%
}

% Additional missing environments
\newenvironment{schluessel}{%
	\begin{mdframed}[linecolor=yellow!50,linewidth=1pt,roundcorner=4pt,backgroundcolor=yellow!10]%
	}{%
	\end{mdframed}%
}

\newenvironment{summary}{%
	\begin{mdframed}[linecolor=gray!40,linewidth=1pt,roundcorner=4pt,backgroundcolor=gray!8]%
	}{%
	\end{mdframed}%
}

\newenvironment{category}{%
	\begin{mdframed}[linecolor=pink!40,linewidth=1pt,roundcorner=4pt,backgroundcolor=pink!8]%
	}{%
	\end{mdframed}%
}

\newenvironment{sibox}{%
	\begin{mdframed}[linecolor=lime!40,linewidth=1pt,roundcorner=4pt,backgroundcolor=lime!8]%
	}{%
	\end{mdframed}%
}

% More missing environments
\newenvironment{documentbox}{%
	\begin{mdframed}[linecolor=teal!40,linewidth=1pt,roundcorner=4pt,backgroundcolor=teal!8]%
	}{%
	\end{mdframed}%
}

\newenvironment{t0box}{%
	\begin{mdframed}[linecolor=violet!40,linewidth=1pt,roundcorner=4pt,backgroundcolor=violet!8]%
	}{%
	\end{mdframed}%
}

\newenvironment{wichtig}{%
	\begin{mdframed}[linecolor=red!50,linewidth=2pt,roundcorner=4pt,backgroundcolor=red!10]%
	\textbf{Important:} 
	}{%
	\end{mdframed}%
}

\newenvironment{smbox}{%
	\begin{mdframed}[linecolor=orange!40,linewidth=1pt,roundcorner=4pt,backgroundcolor=orange!8]%
	}{%
	\end{mdframed}%
}

\newenvironment{pvbox}{%
	\begin{mdframed}[linecolor=purple!40,linewidth=1pt,roundcorner=4pt,backgroundcolor=purple!8]%
	}{%
	\end{mdframed}%
}

\newenvironment{numerisch}{%
	\begin{mdframed}[linecolor=blue!40,linewidth=1pt,roundcorner=4pt,backgroundcolor=blue!8]%
	}{%
	\end{mdframed}%
}

% More missing environments
\newenvironment{relation}{%
	\begin{mdframed}[linecolor=green!40,linewidth=1pt,roundcorner=4pt,backgroundcolor=green!8]%
	}{%
	\end{mdframed}%
}

\newenvironment{beweis}{%
	\begin{mdframed}[linecolor=brown!40,linewidth=1pt,roundcorner=4pt,backgroundcolor=brown!8]%
	\textbf{Proof:} 
	}{%
	\end{mdframed}%
}

\newenvironment{revolution}{%
	\begin{mdframed}[linecolor=red!60,linewidth=2pt,roundcorner=4pt,backgroundcolor=red!12]%
	}{%
	\end{mdframed}%
}

\newenvironment{key}{%
	\begin{mdframed}[linecolor=yellow!50,linewidth=1pt,roundcorner=4pt,backgroundcolor=yellow!10]%
	}{%
	\end{mdframed}%
}

\newenvironment{newperspective}{%
	\begin{mdframed}[linecolor=cyan!50,linewidth=1pt,roundcorner=4pt,backgroundcolor=cyan!10]%
	}{%
	\end{mdframed}%
}

\newenvironment{literatur}{%
	\begin{mdframed}[linecolor=gray!50,linewidth=1pt,roundcorner=4pt,backgroundcolor=gray!10]%
	}{%
	\end{mdframed}%
}

\newenvironment{folgerung}{%
	\begin{mdframed}[linecolor=teal!50,linewidth=1pt,roundcorner=4pt,backgroundcolor=teal!10]%
	}{%
	\end{mdframed}%
}

\newenvironment{principle}{%
	\begin{mdframed}[linecolor=blue!60,linewidth=2pt,roundcorner=4pt,backgroundcolor=blue!12]%
	}{%
	\end{mdframed}%
}

% Additional common environments
% ==============================================================================
% FROM HERE: YOUR DEFINITIONS (unchanged)
% ==============================================================================

\setcounter{tocdepth}{3}

% === CITATION COMMANDS ===
\providecommand{\citep}[1]{\cite{#1}}
\providecommand{\citet}[1]{\cite{#1}}

% === COLORS ===
\definecolor{gold}{RGB}{255,215,0}
\definecolor{blue}{rgb}{0,0,1}
\definecolor{boxgray}{RGB}{240,240,240}
\definecolor{deepblue}{RGB}{0,0,127}
\definecolor{deepgreen}{RGB}{0,127,0}
\definecolor{deepred}{RGB}{191,0,0}
\definecolor{t0blue}{RGB}{33,150,243}
\definecolor{t0green}{RGB}{76,175,80}
\definecolor{t0orange}{RGB}{255,152,0}
\definecolor{t0purple}{RGB}{156,39,176}
\definecolor{t0red}{RGB}{244,67,54}
\definecolor{t0yellow}{RGB}{255,204,0}

% === COLUMN TYPES ===
\newcolumntype{L}[1]{>{\raggedright\arraybackslash}p{#1}}
\newcolumntype{C}[1]{>{\centering\arraybackslash}p{#1}}
\newcolumntype{R}[1]{>{\raggedleft\arraybackslash}p{#1}}

% === HYPERREF SETTINGS (updated) ===
\hypersetup{
	colorlinks=true,
	linkcolor=t0blue,
	citecolor=t0blue,
	urlcolor=t0blue,
	breaklinks=true,
	bookmarksnumbered=true,
	pdfstartview=FitH,
	pdfencoding=auto,
	pdfdisplaydoctitle=true
}

% === ENGLISH THEOREM ENVIRONMENTS ===
\theoremstyle{plain}
\newtheorem{theorem}{Theorem}[section]
\newtheorem{lemma}[theorem]{Lemma}
\newtheorem{proposition}[theorem]{Proposition}
\newtheorem{corollary}[theorem]{Corollary}

\theoremstyle{definition}
\newtheorem{definition}[theorem]{Definition}
\newtheorem{example}[theorem]{Example}
\newtheorem{insight}[theorem]{Insight}
\newtheorem{discovery}[theorem]{Discovery}

\theoremstyle{remark}
\newtheorem{remark}[theorem]{Remark}
\newtheorem{axiom}{Axiom}
%\newtheorem{principle}{Principle}  % Commented out to avoid conflicts with document-specific definitions
%\newtheorem{warning}[theorem]{Warning}

% === T0-SPECIFIC COMMANDS ===
% (Here follow all your \newcommand and \providecommand definitions)
% These remain UNCHANGED as in your original preamble
% ==============================================================================
% SECTION 14: T0-Specific Commands
% ==============================================================================

% --- Core T0 Fields ---
\newcommand{\Tfield}{T(x,t)}
\providecommand{\Tfieldt}{T(\vec{x},t)}
\newcommand{\Efield}{E(x,t)}
\newcommand{\mfield}{m(x,t)}
\providecommand{\vecx}{\vec{x}}

% --- Lagrangian ---
\newcommand{\Lag}{\mathcal{L}}
\newcommand{\calL}{\mathcal{L}}

% --- Greek Letters and Constants ---
\newcommand{\alphaem}{\alpha}
\newcommand{\betaT}{\beta_T}
\newcommand{\xiT}{\xi}
\newcommand{\xipar}{\xi}

% --- Energy and Planck Units ---
\newcommand{\Ezero}{E_0}
\newcommand{\E}{E}
\newcommand{\EPlanck}{E_{\text{Pl}}}
\newcommand{\Mpl}{M_{\text{Pl}}}
\newcommand{\mP}{m_{\text{P}}}
\newcommand{\lP}{\ell_{\text{P}}}
\newcommand{\tP}{t_{\text{P}}}
\newcommand{\LPlanck}{\ell_{\text{Pl}}}
\newcommand{\TPlanck}{t_{\text{Pl}}}

% --- Coupling Constants ---
\newcommand{\Gnat}{G_{\text{nat}}}
\newcommand{\alphaEM}{\alpha_{\text{EM}}}
\newcommand{\alphaSI}{\alpha_{\text{SI}}}
\newcommand{\Hubble}{H_0}
\newcommand{\LCDM}{\Lambda\text{CDM}}
\newcommand{\natunits}{(nat. units)}

% --- T0 Model Parameters ---
\newcommand{\xigeom}{\xi_{\mathrm{geom}}}
\newcommand{\rzero}{r_{0}}
\newcommand{\xirat}{\xi_{\mathrm{rat}}}
\newcommand{\tzero}{t_{0}}
\newcommand{\Lambdat}{\Lambda_{\mathrm{t}}}
\newcommand{\EP}{E_{\text{P}}}
\newcommand{\Emu}{E_{\mu}}
\newcommand{\Ee}{E_{e}}
\newcommand{\Etau}{E_{\tau}}
\newcommand{\alphafine}{\alpha_{\mathrm{fine}}}
\newcommand{\alphal}{\alpha_{\ell}}
\newcommand{\Lzero}{\ell_{0}}
\newcommand{\Lp}{\ell_{\mathrm{P}}}

% --- Additional T0 Commands ---
\newcommand{\Kfrak}{K_{\text{frak}}}
\newcommand{\Dfrak}{D_{\text{frak}}}
\newcommand{\betapar}{\ensuremath{\beta_T}}
\newcommand{\alphapar}{\alpha}
\newcommand{\deltafield}{\delta \phi}
\newcommand{\deltam}{\delta m}
\newcommand{\deltaE}{\delta E}
\newcommand{\Exi}{E_{\xi}}
\newcommand{\Lxi}{\ell_{\xi}}
\newcommand{\rhoCMB}{\rho_{\text{CMB}}}
\newcommand{\rhoCasimir}{\rho_{\text{Casimir}}}
\newcommand{\Leff}{L_{\text{eff}}}
\newcommand{\CQCD}{C_{\mathrm{QCD}}}
\newcommand{\Kspec}{K_{\mathrm{spec}}}
\newcommand{\Tzero}{\ensuremath{T_0}}
\newcommand{\Eabs}{E_{\text{abs}}}
\newcommand{\taupar}{\tau}

% --- Provided Commands ---
\providecommand{\xiconst}{\xi_{\text{const}}}
\providecommand{\DhiggsT}{D_{\text{Higgs-T}}}
\providecommand{\rhoE}{\rho_{E}}
\providecommand{\Echar}{E_{\text{char}}}
\providecommand{\kfrac}{k_{\text{frac}}}
\providecommand{\alphaEMSI}{\alpha_{\text{EM,SI}}}
\providecommand{\alphaEMnat}{\alpha_{\text{EM,nat}}}
\providecommand{\betaTSI}{\beta_{T,\text{SI}}}
\providecommand{\betaTnat}{\beta_{T,\text{nat}}}
\providecommand{\Gsi}{G_{\text{SI}}}
\providecommand{\xiparSI}{\xi_{\text{SI}}}
\providecommand{\xiparnat}{\xi_{\text{nat}}}
\providecommand{\meff}{m_{\text{eff}}}
\providecommand{\Tzerot}{T_{0}(t)}
\providecommand{\mzerot}{m_{0}(t)}
\providecommand{\Ezeroabs}{E_{0,\text{abs}}}
\providecommand{\Epar}{E_{\text{par}}}
\providecommand{\Lnat}{\ell_{\text{nat}}}
\providecommand{\Tnat}{T_{\text{nat}}}
\providecommand{\xifrak}{\xi_{\text{frac}}}
\providecommand{\Tfrak}{T_{\text{frac}}}
\providecommand{\mfrak}{m_{\text{frac}}}
\providecommand{\Dfrac}{D_{\text{frac}}}
\providecommand{\EphotSI}{E_{\gamma,\text{SI}}}
\providecommand{\EphotNat}{E_{\gamma,\text{nat}}}
\providecommand{\Eabsint}{E_{\text{abs,int}}}
\providecommand{\mphoton}{m_{\gamma}}
\providecommand{\Evis}{E_{\text{vis}}}
\providecommand{\Cto}{C_{T0}}
\providecommand{\mytimes}{\times}
\providecommand{\lambdah}{\lambda_h}
\providecommand{\checkmarkx}{\checkmark}
\providecommand{\Enorm}{E_{\text{norm}}}
\providecommand{\Tobs}{T_{\text{obs}}}
\providecommand{\mobs}{m_{\text{obs}}}
\providecommand{\Eobs}{E_{\text{obs}}}
\providecommand{\Lobs}{\ell_{\text{obs}}}
\providecommand{\xobs}{\xi_{\text{obs}}}
\providecommand{\calE}{\mathcal{E}}
\providecommand{\calT}{\mathcal{T}}
\providecommand{\calM}{\mathcal{M}}
\providecommand{\alphag}{\alpha_g}
\providecommand{\Tmax}{T_{\text{max}}}
\providecommand{\mmin}{m_{\text{min}}}
\providecommand{\Lmax}{\ell_{\text{max}}}
\providecommand{\Emin}{E_{\text{min}}}
\providecommand{\Geff}{G_{\text{eff}}}
\providecommand{\rhoeff}{\rho_{\text{eff}}}
\providecommand{\xieff}{\xi_{\text{eff}}}
\providecommand{\Teff}{T_{\text{eff}}}
\providecommand{\hPlanck}{h}
\providecommand{\kB}{k_B}
\providecommand{\muB}{\mu_B}
\providecommand{\lambdaC}{\lambda_C}
\providecommand{\omegaP}{\omega_P}
\providecommand{\rhoP}{\rho_P}
\providecommand{\Tref}{T_{\text{ref}}}
\providecommand{\Eref}{E_{\text{ref}}}
\providecommand{\mref}{m_{\text{ref}}}
\providecommand{\Lref}{\ell_{\text{ref}}}
\providecommand{\xikonst}{\xi_0}
\providecommand{\Phiphoton}{\Phi_{\gamma}}
\providecommand{\etavis}{\eta_{\text{vis}}}
\providecommand{\pichar}{\pi}
\providecommand{\primrel}{\mathcal{P}_{\text{rel}}}
\providecommand{\warningx}{\textcolor{orange}{\textbf{!}}}
\providecommand{\phiT}{\phi_T}
\providecommand{\Lorentz}{\Lambda}
\providecommand{\Cconv}{C_{\text{conv}}}
\providecommand{\Df}{\Delta f}
\providecommand{\lambdazero}{\lambda_0}
\providecommand{\myapprox}{\approx}
\providecommand{\checked}{\checkmark}
\providecommand{\alphaWSI}{\alpha_W^{\text{SI}}}
\providecommand{\alphaWnat}{\alpha_W^{\text{nat}}}
\providecommand{\vect}[1]{\vec{#1}}
\providecommand{\Rzero}{R_0}
\providecommand{\Riem}{\mathcal{R}}
\providecommand{\nuzero}{\nu_0}
\providecommand{\mypi}{\pi}

% =============================================================================
% TCOLORBOX STYLES AND ENVIRONMENTS (English titles)
% =============================================================================
\tcbset{
	keyresult/.style={
		colback=blue!5!white,
		colframe=blue!75!black,
		title=Key Result,
		fonttitle=\bfseries
	},
	foundation/.style={
		colback=green!5!white,
		colframe=green!75!black,
		title=Foundation,
		fonttitle=\bfseries
	},
	alternative/.style={
		colback=orange!5!white,
		colframe=orange!75!black,
		title=Alternative,
		fonttitle=\bfseries
	},
	warningbox/.style={
		colback=red!5!white,
		colframe=red!75!black,
		title=Warning,
		fonttitle=\bfseries
	}
}

% (Here follow all your tcolorbox definitions with English titles)
\newtcolorbox{keyresultbox}[1][]{colback=blue!5!white,colframe=blue!75!black,fonttitle=\bfseries,title={#1},breakable}
\newtcolorbox{keyresult}[1][Key Result]{colback=blue!5!white,colframe=blue!75!black,fonttitle=\bfseries,title={#1},breakable}
\newtcolorbox{foundationbox}[1][]{colback=green!5!white,colframe=green!75!black,fonttitle=\bfseries,title={#1},breakable}
\newtcolorbox{foundation}[1][Foundation]{colback=green!5!white,colframe=green!75!black,fonttitle=\bfseries,title={#1},breakable}
\newtcolorbox{alternativebox}[1][]{colback=orange!5!white,colframe=orange!75!black,fonttitle=\bfseries,title={#1},breakable}
\newtcolorbox{warningboxenv}[1][Warning]{colback=red!5!white,colframe=red!75!black,fonttitle=\bfseries,title={#1},breakable}

\newtcolorbox{fundamental}[1][]{
	colback=boxgray,
	colframe=t0blue,
	fonttitle=\bfseries,
	title=#1,
	sharp corners,
	boxrule=2pt
}

\newtcolorbox{insightBox}[1][Insight]{colback=blue!5,colframe=t0blue,title={#1},fonttitle=\bfseries,breakable}
\newtcolorbox{discoveryBox}[1][Discovery]{colback=green!5,colframe=t0green,title={#1},fonttitle=\bfseries,breakable}
\newtcolorbox{revelation}[1][Revelation]{colback=red!5,colframe=t0red,title={#1},fonttitle=\bfseries,breakable}
\newtcolorbox{keypoint}[1][Key Point]{colback=blue!5,colframe=t0blue,title={#1},fonttitle=\bfseries,breakable}
\newtcolorbox{evidence}[1][Evidence]{colback=green!5,colframe=t0green,title={#1},fonttitle=\bfseries,breakable}
\newtcolorbox{conclusionBox}[1][Conclusion]{colback=gray!5,colframe=gray,title={#1},fonttitle=\bfseries,breakable}
\newtcolorbox{significance}[1][Significance]{colback=yellow!5,colframe=orange,title={#1},fonttitle=\bfseries,breakable}
\newtcolorbox{philosophical}[1][Philosophical]{colback=purple!5,colframe=purple,title={#1},fonttitle=\bfseries,breakable}
\newtcolorbox{implicationBox}[1][Implication]{colback=cyan!5,colframe=cyan,title={#1},fonttitle=\bfseries,breakable}
\newtcolorbox{perspectiveBox}[1][Perspective]{colback=blue!5,colframe=t0blue,title={#1},fonttitle=\bfseries,breakable}
\newtcolorbox{revolutionary}[1][Revolutionary]{colback=red!5,colframe=t0red,title={#1},fonttitle=\bfseries,breakable}

\newtcolorbox{technical}[1][Technical]{colback=gray!5,colframe=gray!75!black,title={#1},fonttitle=\bfseries,breakable}
\newtcolorbox{technicalBox}[1][Technical]{colback=gray!5,colframe=gray!75!black,title={#1},fonttitle=\bfseries,breakable}
\newtcolorbox{notationBox}[1][Notation]{colback=yellow!5,colframe=yellow!75!black,title={#1},fonttitle=\bfseries,breakable}
\newtcolorbox{verification}[1][Verification]{colback=orange!5!white,colframe=orange!75!black,fonttitle=\bfseries,title=#1}
\newtcolorbox{explanationBox}[1][Explanation]{colback=purple!5!white,colframe=purple!75!black,fonttitle=\bfseries,title=#1}
\newtcolorbox{interpretationBox}[1][Interpretation]{colback=cyan!5!white,colframe=cyan!75!black,fonttitle=\bfseries,title=#1}
\newtcolorbox{explanation}[1][Explanation]{colback=purple!5!white,colframe=purple!75!black,fonttitle=\bfseries,title=#1,breakable}
\newtcolorbox{interpretation}[1][Interpretation]{colback=cyan!5!white,colframe=cyan!75!black,fonttitle=\bfseries,title=#1,breakable}
\newtcolorbox{proof_step}[1][Proof Step]{colback=gray!5!white,colframe=gray!75!black,fonttitle=\bfseries,title=#1,breakable}
\newtcolorbox{experimental}[1][Experimental]{colback=teal!5!white,colframe=teal!75!black,fonttitle=\bfseries,title=#1,breakable}

\newtcolorbox{important}[1][Important]{colback=red!5!white,colframe=red!75!black,title={#1},fonttitle=\bfseries,breakable}
\newtcolorbox{warning}[1][Warning]{colback=orange!5!white,colframe=orange!75!black,title={#1},fonttitle=\bfseries,breakable}
\newtcolorbox{caution}[1][Caution]{colback=yellow!5!white,colframe=yellow!75!black,title={#1},fonttitle=\bfseries,breakable}
\newtcolorbox{highlight}[1][Highlight]{colback=yellow!10!white,colframe=yellow!75!black,title={#1},fonttitle=\bfseries,breakable}
\newtcolorbox{critical}[1][Critical]{colback=red!10!white,colframe=red!75!black,title={#1},fonttitle=\bfseries,breakable}

\newtcolorbox{analysis}[1][Analysis]{colback=blue!5!white,colframe=blue!75!black,title={#1},fonttitle=\bfseries,breakable}
\newtcolorbox{application}[1][Application]{colback=green!5!white,colframe=green!75!black,title={#1},fonttitle=\bfseries,breakable}
\newtcolorbox{experiment}[1][Experiment]{colback=cyan!5!white,colframe=cyan!75!black,title={#1},fonttitle=\bfseries,breakable}
\newtcolorbox{historical}[1][Historical]{colback=brown!5!white,colframe=brown!75!black,title={#1},fonttitle=\bfseries,breakable}
\newtcolorbox{numerical}[1][Numerical]{colback=gray!5!white,colframe=gray!75!black,title={#1},fonttitle=\bfseries,breakable}
\newtcolorbox{overview}[1][Overview]{colback=blue!5!white,colframe=blue!75!black,title={#1},fonttitle=\bfseries,breakable}
\newtcolorbox{speculation}[1][Speculation]{colback=purple!5!white,colframe=purple!75!black,title={#1},fonttitle=\bfseries,breakable}
\newtcolorbox{question}[1][Question]{colback=orange!5!white,colframe=orange!75!black,title={#1},fonttitle=\bfseries,breakable}
\newtcolorbox{method}[1][Method]{colback=teal!5!white,colframe=teal!75!black,title={#1},fonttitle=\bfseries,breakable}
\newtcolorbox{correct}[1][Correct]{colback=green!10!white,colframe=green!75!black,title={#1},fonttitle=\bfseries,breakable}
\newtcolorbox{units}[1][Units]{colback=gray!5!white,colframe=gray!75!black,title={#1},fonttitle=\bfseries,breakable}
\newtcolorbox{achievement}[1][Achievement]{colback=gold!5!white,colframe=orange!75!black,title={#1},fonttitle=\bfseries,breakable}
\newtcolorbox{equivalence}[1][Equivalence]{colback=cyan!5!white,colframe=cyan!75!black,title={#1},fonttitle=\bfseries,breakable}
\newtcolorbox{dimensional}[1][Dimensional Analysis]{colback=purple!5!white,colframe=purple!75!black,title={#1},fonttitle=\bfseries,breakable}

% === ADDITIONAL SIMPLE ENVIRONMENTS ===
\newenvironment{treatise}{\begin{quote}}{\end{quote}}
\newenvironment{gemeinsam}{\begin{quote}}{\end{quote}}
\newenvironment{vergleich}{\begin{quote}}{\end{quote}}
\newenvironment{vorteil}{\begin{quote}}{\end{quote}}
\newenvironment{common}{\begin{quote}}{\end{quote}}
\newenvironment{comparison}{\begin{quote}}{\end{quote}}
\newenvironment{advantage}{\begin{quote}}{\end{quote}}
\newenvironment{quantum}{\begin{quote}}{\end{quote}}

% === LAYOUT SETTINGS ===
\raggedbottom
\usepackage{environ}
\let\oldtabular\tabular
\let\endoldtabular\endtabular

\newenvironment{scaledtable}[1][0.85]{%
	\begingroup\footnotesize\setlength{\LTleft}{0pt}\setlength{\LTright}{0pt}%
}{%
	\endgroup%
}

\newcommand{\widetable}[1]{\resizebox{\textwidth}{!}{#1}}

% === TABLE OF CONTENTS FORMATTING ===
\renewcommand{\cftsecfont}{\color{blue}}
\renewcommand{\cftsubsecfont}{\color{blue}}
\renewcommand{\cftsecpagefont}{\color{blue}}
\renewcommand{\cftsubsecpagefont}{\color{blue}}
\renewcommand{\cfttoctitlefont}{\huge\bfseries\color{blue}}

% === DEFAULT HEADER AND FOOTER ===
\pagestyle{fancy}
\fancyhf{}
\fancyhead[L]{\textsc{T0 Theory}}
\fancyhead[R]{\textsc{J. Pascher}}
\fancyfoot[C]{\thepage}

% ==============================================================================
% End of Shared Preamble for English
% ==============================================================================
%   \begin{document}
%   ...
%   \end{document}
%
% ==============================================================================

% =============================================================================
% SECTION 1: Encoding and Language
% =============================================================================
\usepackage[utf8]{inputenc}
\usepackage[T1]{fontenc}
\usepackage[ngerman]{babel}
\usepackage{lmodern}

% =============================================================================
% SECTION 2: Page Geometry
% =============================================================================
\usepackage[a4paper, left=2.5cm, right=2.5cm, top=2.5cm, bottom=3.5cm]{geometry}
\setlength{\headheight}{15pt}

% =============================================================================
% SECTION 3: Mathematics and Physics
% =============================================================================
\usepackage{amsmath,amssymb,amsfonts,amsthm}
\usepackage{mathtools}
\usepackage{physics}
\usepackage{siunitx}
\sisetup{
    locale=US,
    group-separator={,},
    output-decimal-marker={.},
    per-mode=symbol
}

% =============================================================================
% SECTION 4: Graphics and Tables
% =============================================================================
\usepackage{graphicx}
\usepackage[table,xcdraw]{xcolor}
\usepackage{tikz}
\usetikzlibrary{arrows.meta,positioning,shapes.geometric,decorations.pathmorphing,patterns,shapes.arrows,intersections}
\usepackage{pgfplots}
\pgfplotsset{compat=1.18}
\usepackage[most]{tcolorbox}
\tcbuselibrary{breakable}
\usepackage{booktabs}
\usepackage{array}
\usepackage{longtable}
\usepackage{float}
\usepackage{adjustbox}
\usepackage{rotating}
\usepackage{tabularx}
\usepackage{makecell}
\usepackage{multirow}

% =============================================================================
% SECTION 5: Document Formatting
% =============================================================================
\usepackage{fancyhdr}
\renewcommand{\headrulewidth}{0.4pt}
\renewcommand{\footrulewidth}{0.4pt}
\usepackage{tocloft}
\usepackage{hyperref}
\hypersetup{
  colorlinks=true,
  linkcolor=black,
  citecolor=black,
  urlcolor=black,
  breaklinks=true,
  bookmarksnumbered=true,
  unicode=true
}
\usepackage{bookmark}
\usepackage{cleveref}

% Table of contents: only show chapters (not sections/subsections)
\setcounter{tocdepth}{3}  % Show sections, subsections, and subsubsections
\usepackage{microtype}
\usepackage{enumitem}
\usepackage{setspace}
\usepackage{ragged2e}
\usepackage{multicol}

% =============================================================================
% SECTION 6: Code and Algorithms
% =============================================================================
\usepackage{algorithm}
\usepackage{algorithmic}
\usepackage{listings}
\lstset{
  basicstyle=\ttfamily\footnotesize,
  breaklines=true,
  breakatwhitespace=true,
  columns=flexible,
  keepspaces=true,
  showstringspaces=false,
  frame=single,
  xleftmargin=0pt,
  xrightmargin=0pt
}
\usepackage{mdframed}

% =============================================================================
% SECTION 7: Additional Packages
% =============================================================================
\usepackage{pdflscape}
\usepackage{braket}
\usepackage{cancel}
\usepackage{caption}
\usepackage{csquotes}
\usepackage{gensymb}
\usepackage{hyphenat}
\usepackage{textcomp}
\usepackage{textgreek}
\usepackage{upgreek}
\usepackage{url}
\usepackage{slashed}
\usepackage{bm}
\usepackage{newunicodechar}

% =============================================================================
% SECTION 8: Citation Commands (Compatibility)
% =============================================================================
\providecommand{\citep}[1]{\cite{#1}}
\providecommand{\citet}[1]{\cite{#1}}

% =============================================================================
% SECTION 9: Colors
% =============================================================================
\definecolor{gold}{RGB}{255,215,0}
\definecolor{blue}{rgb}{0,0,1}
\definecolor{boxgray}{RGB}{240,240,240}
\definecolor{deepblue}{RGB}{0,0,127}
\definecolor{deepgreen}{RGB}{0,127,0}
\definecolor{deepred}{RGB}{191,0,0}
\definecolor{t0blue}{RGB}{33,150,243}
\definecolor{t0green}{RGB}{76,175,80}
\definecolor{t0orange}{RGB}{255,152,0}
\definecolor{t0purple}{RGB}{156,39,176}
\definecolor{t0red}{RGB}{244,67,54}
\definecolor{t0yellow}{RGB}{255,204,0}

% =============================================================================
% SECTION 10: Column Types
% =============================================================================
\newcolumntype{L}[1]{>{\raggedright\arraybackslash}p{#1}}
\newcolumntype{C}[1]{>{\centering\arraybackslash}p{#1}}

% =============================================================================
% SECTION 11: Unicode Character Mappings
% =============================================================================
\newunicodechar{ħ}{$\hbar$}
\newunicodechar{↔}{$\leftrightarrow$}
\newunicodechar{⇐}{$\Leftarrow$}
\newunicodechar{⇒}{$\Rightarrow$}
\newunicodechar{⇔}{$\Leftrightarrow$}
\newunicodechar{∂}{$\partial$}
\newunicodechar{∅}{$\emptyset$}
\newunicodechar{∇}{$\nabla$}
\newunicodechar{∈}{$\in$}
\newunicodechar{∉}{$\notin$}
\newunicodechar{∏}{$\prod$}
\newunicodechar{∑}{$\sum$}
% Note: √ is mapped to an empty sqrt; use \sqrt{x} for proper usage
\newunicodechar{√}{\ensuremath{\sqrt{}}}
\newunicodechar{∝}{$\propto$}
\newunicodechar{∞}{$\infty$}
\newunicodechar{∩}{$\cap$}
\newunicodechar{∪}{$\cup$}
\newunicodechar{∫}{$\int$}
\newunicodechar{≈}{$\approx$}
\newunicodechar{≠}{$\neq$}
\newunicodechar{≤}{$\leq$}
\newunicodechar{≥}{$\geq$}
\newunicodechar{ξ}{\ensuremath{\xi}}
\newunicodechar{μ}{\ensuremath{\mu}}
\newunicodechar{ψ}{\ensuremath{\psi}}
\newunicodechar{φ}{\ensuremath{\phi}}
\newunicodechar{π}{\ensuremath{\pi}}
\newunicodechar{λ}{\ensuremath{\lambda}}
\newunicodechar{Δ}{\ensuremath{\Delta}}

% =============================================================================
% SECTION 12: Hyperref Settings
% =============================================================================
\hypersetup{
    colorlinks=true,
    linkcolor=blue,
    citecolor=blue,
    urlcolor=blue,
    breaklinks=true,
    bookmarksnumbered=true,
    pdfstartview=FitH
}

% =============================================================================
% SECTION 13: Theorem Environments (English)
% =============================================================================
\theoremstyle{plain}
\newtheorem{theorem}{Theorem}[section]
\newtheorem{lemma}[theorem]{Lemma}
\newtheorem{proposition}[theorem]{Proposition}
\newtheorem{corollary}[theorem]{Corollary}

\theoremstyle{definition}
\newtheorem{definition}[theorem]{Definition}
\newtheorem{example}[theorem]{Example}
\newtheorem{insight}[theorem]{Insight}
\newtheorem{discovery}[theorem]{Discovery}
% \newtheorem{erkenntnis}[theorem]{Insight}  % Commented out - conflicts with tcolorbox environment below

\theoremstyle{remark}
\newtheorem{remark}[theorem]{Remark}
\newtheorem{axiom}{Axiom}
\newtheorem{principle}{Principle}
\newtheorem{bemerkung}[theorem]{Remark}
\newtheorem{warnung}[theorem]{Warning}

% =============================================================================
% SECTION 14: T0-Specific Commands
% =============================================================================

% --- Core T0 Fields ---
\newcommand{\Tfield}{T(x,t)}
\providecommand{\Tfieldt}{T(\vec{x},t)}
\newcommand{\Efield}{E(x,t)}
\newcommand{\mfield}{m(x,t)}
\providecommand{\vecx}{\vec{x}}

% --- Lagrangian ---
\newcommand{\Lag}{\mathcal{L}}
\newcommand{\calL}{\mathcal{L}}

% --- Greek Letters and Constants ---
\newcommand{\alphaem}{\alpha}
\newcommand{\betaT}{\beta_T}
\newcommand{\xiT}{\xi}
\newcommand{\xipar}{\xi}

% --- Energy and Planck Units ---
\newcommand{\Ezero}{E_0}
\newcommand{\EPlanck}{E_{\text{Pl}}}
\newcommand{\Mpl}{M_{\text{Pl}}}
\newcommand{\mP}{m_{\text{P}}}
\newcommand{\lP}{\ell_{\text{P}}}
\newcommand{\tP}{t_{\text{P}}}
\newcommand{\LPlanck}{\ell_{\text{Pl}}}
\newcommand{\TPlanck}{t_{\text{Pl}}}

% --- Coupling Constants ---
\newcommand{\Gnat}{G_{\text{nat}}}
\newcommand{\alphaEM}{\alpha_{\text{EM}}}
\newcommand{\alphaSI}{\alpha_{\text{SI}}}
\newcommand{\Hubble}{H_0}
\newcommand{\LCDM}{\Lambda\text{CDM}}
\newcommand{\natunits}{(nat. units)}

% --- T0 Model Parameters ---
\newcommand{\xigeom}{\xi_{\mathrm{geom}}}
\newcommand{\rzero}{r_{0}}
\newcommand{\xirat}{\xi_{\mathrm{rat}}}
\newcommand{\tzero}{t_{0}}
\newcommand{\Lambdat}{\Lambda_{\mathrm{t}}}
\newcommand{\EP}{E_{\mathrm{P}}}
\newcommand{\Emu}{E_{\mu}}
\newcommand{\Ee}{E_{e}}
\newcommand{\Etau}{E_{\tau}}
\newcommand{\alphafine}{\alpha_{\mathrm{fine}}}
\newcommand{\alphal}{\alpha_{\ell}}
\newcommand{\Lzero}{\ell_{0}}
\newcommand{\Lp}{\ell_{\mathrm{P}}}

% --- Additional T0 Commands ---
\newcommand{\Kfrak}{K_{\text{frak}}}
\newcommand{\Dfrak}{D_{\text{frak}}}
\newcommand{\betapar}{\beta_T}
\newcommand{\alphapar}{\alpha}
\newcommand{\deltafield}{\delta \phi}
\newcommand{\deltam}{\delta m}
\newcommand{\deltaE}{\delta E}
\newcommand{\Exi}{E_{\xi}}
\newcommand{\Lxi}{\ell_{\xi}}
\newcommand{\rhoCMB}{\rho_{\text{CMB}}}
\newcommand{\rhoCasimir}{\rho_{\text{Casimir}}}
\newcommand{\Leff}{L_{\text{eff}}}
\newcommand{\CQCD}{C_{\mathrm{QCD}}}
\newcommand{\Kspec}{K_{\mathrm{spec}}}
\newcommand{\Tzero}{\ensuremath{T_0}}
\newcommand{\Eabs}{E_{\text{abs}}}
\newcommand{\taupar}{\tau}

% --- Provided Commands (may be redefined elsewhere) ---
\providecommand{\xiconst}{\xi_{\text{const}}}
\providecommand{\DhiggsT}{D_{\text{Higgs-T}}}
\providecommand{\rhoE}{\rho_{E}}
\providecommand{\Echar}{E_{\text{char}}}
\providecommand{\kfrac}{k_{\text{frac}}}
\providecommand{\alphaEMSI}{\alpha_{\text{EM,SI}}}
\providecommand{\alphaEMnat}{\alpha_{\text{EM,nat}}}
\providecommand{\betaTSI}{\beta_{T,\text{SI}}}
\providecommand{\betaTnat}{\beta_{T,\text{nat}}}
\providecommand{\Gsi}{G_{\text{SI}}}
\providecommand{\xiparSI}{\xi_{\text{SI}}}
\providecommand{\xiparnat}{\xi_{\text{nat}}}
\providecommand{\meff}{m_{\text{eff}}}
\providecommand{\Tzerot}{T_{0}(t)}
\providecommand{\mzerot}{m_{0}(t)}
\providecommand{\Ezeroabs}{E_{0,\text{abs}}}
\providecommand{\Epar}{E_{\text{par}}}
\providecommand{\Lnat}{\ell_{\text{nat}}}
\providecommand{\Tnat}{T_{\text{nat}}}
\providecommand{\xifrak}{\xi_{\text{frac}}}
\providecommand{\Tfrak}{T_{\text{frac}}}
\providecommand{\mfrak}{m_{\text{frac}}}
\providecommand{\Dfrac}{D_{\text{frac}}}
\providecommand{\EphotSI}{E_{\gamma,\text{SI}}}
\providecommand{\EphotNat}{E_{\gamma,\text{nat}}}
\providecommand{\Eabsint}{E_{\text{abs,int}}}
\providecommand{\mphoton}{m_{\gamma}}
\providecommand{\Evis}{E_{\text{vis}}}
\providecommand{\Cto}{C_{T0}}
\providecommand{\mytimes}{\times}
\providecommand{\lambdah}{\lambda_h}
\providecommand{\checkmarkx}{\checkmark}
\providecommand{\Enorm}{E_{\text{norm}}}
\providecommand{\Tobs}{T_{\text{obs}}}
\providecommand{\mobs}{m_{\text{obs}}}
\providecommand{\Eobs}{E_{\text{obs}}}
\providecommand{\Lobs}{\ell_{\text{obs}}}
\providecommand{\xobs}{\xi_{\text{obs}}}
\providecommand{\calE}{\mathcal{E}}
\providecommand{\calT}{\mathcal{T}}
\providecommand{\calM}{\mathcal{M}}
\providecommand{\alphag}{\alpha_g}
\providecommand{\Tmax}{T_{\text{max}}}
\providecommand{\mmin}{m_{\text{min}}}
\providecommand{\Lmax}{\ell_{\text{max}}}
\providecommand{\Emin}{E_{\text{min}}}
\providecommand{\Geff}{G_{\text{eff}}}
\providecommand{\rhoeff}{\rho_{\text{eff}}}
\providecommand{\xieff}{\xi_{\text{eff}}}
\providecommand{\Teff}{T_{\text{eff}}}
\providecommand{\hPlanck}{h}
\providecommand{\kB}{k_B}
\providecommand{\muB}{\mu_B}
\providecommand{\lambdaC}{\lambda_C}
\providecommand{\omegaP}{\omega_P}
\providecommand{\rhoP}{\rho_P}
\providecommand{\Tref}{T_{\text{ref}}}
\providecommand{\Eref}{E_{\text{ref}}}
\providecommand{\mref}{m_{\text{ref}}}
\providecommand{\Lref}{\ell_{\text{ref}}}
\providecommand{\xikonst}{\xi_0}
\providecommand{\Phiphoton}{\Phi_{\gamma}}
\providecommand{\etavis}{\eta_{\text{vis}}}
\providecommand{\pichar}{\pi}
\providecommand{\primrel}{\mathcal{P}_{\text{rel}}}
\providecommand{\warningx}{\textcolor{orange}{\textbf{!}}}
\providecommand{\phiT}{\phi_T}
\providecommand{\Lorentz}{\Lambda}
\providecommand{\Cconv}{C_{\text{conv}}}
\providecommand{\Df}{\Delta f}
\providecommand{\lambdazero}{\lambda_0}
\providecommand{\myapprox}{\approx}
\providecommand{\checked}{\checkmark}
\providecommand{\alphaWSI}{\alpha_W^{\text{SI}}}
\providecommand{\alphaWnat}{\alpha_W^{\text{nat}}}
\providecommand{\vect}[1]{\vec{#1}}
\providecommand{\Rzero}{R_0}
\providecommand{\Riem}{\mathcal{R}}
\providecommand{\nuzero}{\nu_0}
\providecommand{\mypi}{\pi}

% =============================================================================
% SECTION 15: tcolorbox Styles and Environments
% =============================================================================

% --- Predefined Styles ---
\tcbset{
    keyresult/.style={
        colback=blue!5!white,
        colframe=blue!75!black,
        title=Key Result,
        fonttitle=\bfseries
    },
    foundation/.style={
        colback=green!5!white,
        colframe=green!75!black,
        title=Foundation,
        fonttitle=\bfseries
    },
    alternative/.style={
        colback=orange!5!white,
        colframe=orange!75!black,
        title=Alternative,
        fonttitle=\bfseries
    },
    warningbox/.style={
        colback=red!5!white,
        colframe=red!75!black,
        title=Warning,
        fonttitle=\bfseries
    }
}

% --- Core Environments ---
\newtcolorbox{keyresultbox}[1][]{colback=blue!5!white,colframe=blue!75!black,fonttitle=\bfseries,title={#1},breakable}
\newtcolorbox{keyresult}[1][Key Result]{colback=blue!5!white,colframe=blue!75!black,fonttitle=\bfseries,title={#1},breakable}
\newtcolorbox{foundationbox}[1][]{colback=green!5!white,colframe=green!75!black,fonttitle=\bfseries,title={#1},breakable}
\newtcolorbox{foundation}[1][Foundation]{colback=green!5!white,colframe=green!75!black,fonttitle=\bfseries,title={#1},breakable}
\newtcolorbox{alternativebox}[1][]{colback=orange!5!white,colframe=orange!75!black,fonttitle=\bfseries,title={#1},breakable}
\newtcolorbox{warningboxenv}[1][]{colback=red!5!white,colframe=red!75!black,fonttitle=\bfseries,title={#1},breakable}

% --- Formula Environments ---
\newtcolorbox{fundamental}[1][]{
    colback=boxgray,
    colframe=t0blue,
    fonttitle=\bfseries,
    title=#1,
    sharp corners,
    boxrule=2pt
}

\newtcolorbox{newperspective}[1][]{
    colback=red!5!white,
    colframe=t0red,
    fonttitle=\bfseries,
    title=#1,
    sharp corners,
    boxrule=2pt
}

\newtcolorbox{formula}[1][]{
    colback=blue!5!white,
    colframe=blue!75!black,
    fonttitle=\bfseries,
    title=#1
}

\newtcolorbox{result}[1][]{
    colback=green!5!white,
    colframe=green!75!black,
    fonttitle=\bfseries,
    title=#1
}

\newtcolorbox{derivation}[1][]{
    colback=green!5!white,
    colframe=green!75!black,
    title=#1,
    fonttitle=\bfseries,
    breakable
}

\newtcolorbox{summary}[1][]{
    colback=gray!10!white,
    colframe=gray!75!black,
    title=#1,
    fonttitle=\bfseries,
    breakable
}

\newtcolorbox{comparison}[1][]{
    colback=purple!5!white,
    colframe=purple!75!black,
    title=#1,
    fonttitle=\bfseries,
    breakable
}

\newtcolorbox{relation}[1][]{
    colback=cyan!5!white,
    colframe=cyan!75!black,
    title=#1,
    fonttitle=\bfseries,
    breakable
}

\newtcolorbox{principleBox}[1][]{
    colback=yellow!5!white,
    colframe=yellow!75!black,
    title=#1,
    fonttitle=\bfseries,
    breakable
}

% --- Insight and Discovery Environments ---
\newtcolorbox{insightBox}[1][]{colback=blue!5,colframe=t0blue,title={#1},fonttitle=\bfseries,breakable}
\newtcolorbox{discoveryBox}[1][]{colback=green!5,colframe=t0green,title={#1},fonttitle=\bfseries,breakable}
\newtcolorbox{revelation}[1][]{colback=red!5,colframe=t0red,title={#1},fonttitle=\bfseries,breakable}
\newtcolorbox{keypoint}[1][]{colback=blue!5,colframe=t0blue,title={#1},fonttitle=\bfseries,breakable}
\newtcolorbox{evidence}[1][]{colback=green!5,colframe=t0green,title={#1},fonttitle=\bfseries,breakable}
\newtcolorbox{conclusionBox}[1][]{colback=gray!5,colframe=gray,title={#1},fonttitle=\bfseries,breakable}
\newtcolorbox{significance}[1][]{colback=yellow!5,colframe=orange,title={#1},fonttitle=\bfseries,breakable}
\newtcolorbox{philosophical}[1][]{colback=purple!5,colframe=purple,title={#1},fonttitle=\bfseries,breakable}
\newtcolorbox{implicationBox}[1][]{colback=cyan!5,colframe=cyan,title={#1},fonttitle=\bfseries,breakable}
\newtcolorbox{perspectiveBox}[1][]{colback=blue!5,colframe=t0blue,title={#1},fonttitle=\bfseries,breakable}
\newtcolorbox{revolutionary}[1][]{colback=red!5,colframe=t0red,title={#1},fonttitle=\bfseries,breakable}

% --- Technical Environments ---
\newtcolorbox{technical}[1][]{colback=gray!5,colframe=gray!75!black,title={#1},fonttitle=\bfseries,breakable}
\newtcolorbox{technicalBox}[1][]{colback=gray!5,colframe=gray!75!black,title={#1},fonttitle=\bfseries,breakable}
\newtcolorbox{notationBox}[1][]{colback=yellow!5,colframe=yellow!75!black,title={#1},fonttitle=\bfseries,breakable}
\newtcolorbox{verification}[1][]{colback=orange!5!white,colframe=orange!75!black,fonttitle=\bfseries,title=#1}
\newtcolorbox{explanationBox}[1][]{colback=purple!5!white,colframe=purple!75!black,fonttitle=\bfseries,title=#1}
\newtcolorbox{interpretationBox}[1][]{colback=cyan!5!white,colframe=cyan!75!black,fonttitle=\bfseries,title=#1}
\newtcolorbox{explanation}[1][]{colback=purple!5!white,colframe=purple!75!black,fonttitle=\bfseries,title=#1,breakable}
\newtcolorbox{interpretation}[1][]{colback=cyan!5!white,colframe=cyan!75!black,fonttitle=\bfseries,title=#1,breakable}
\newtcolorbox{proof_step}[1][]{colback=gray!5!white,colframe=gray!75!black,fonttitle=\bfseries,title=#1,breakable}
\newtcolorbox{experimental}[1][]{colback=teal!5!white,colframe=teal!75!black,fonttitle=\bfseries,title=#1,breakable}

% --- Warning and Alert Environments ---
\newtcolorbox{important}[1][]{colback=red!5!white,colframe=red!75!black,title={#1},fonttitle=\bfseries,breakable}
\newtcolorbox{warning}[1][]{colback=orange!5!white,colframe=orange!75!black,title={#1},fonttitle=\bfseries,breakable}
\newtcolorbox{caution}[1][]{colback=yellow!5!white,colframe=yellow!75!black,title={#1},fonttitle=\bfseries,breakable}
\newtcolorbox{highlight}[1][]{colback=yellow!10!white,colframe=yellow!75!black,title={#1},fonttitle=\bfseries,breakable}

% --- Additional German-specific Environments for Matsas documents ---
\newtcolorbox{literatur}[1][Literatur]{colback=blue!5!white,colframe=blue!75!black,title={#1},fonttitle=\bfseries,breakable}
\newtcolorbox{zusammenfassung}[1][Zusammenfassung]{colback=green!5!white,colframe=green!75!black,title={#1},fonttitle=\bfseries,breakable}
\newtcolorbox{frage}[1][Frage]{colback=orange!5!white,colframe=orange!75!black,title={#1},fonttitle=\bfseries,breakable}
\newtcolorbox{erkenntnis}[1][Erkenntnis]{colback=purple!5!white,colframe=purple!75!black,title={#1},fonttitle=\bfseries,breakable}
\newtcolorbox{critical}[1][]{colback=red!10!white,colframe=red!75!black,title={#1},fonttitle=\bfseries,breakable}

% --- Analysis and Application Environments ---
\newtcolorbox{analysis}[1][]{colback=blue!5!white,colframe=blue!75!black,title={#1},fonttitle=\bfseries,breakable}
\newtcolorbox{application}[1][]{colback=green!5!white,colframe=green!75!black,title={#1},fonttitle=\bfseries,breakable}
\newtcolorbox{experiment}[1][]{colback=cyan!5!white,colframe=cyan!75!black,title={#1},fonttitle=\bfseries,breakable}
\newtcolorbox{historical}[1][]{colback=brown!5!white,colframe=brown!75!black,title={#1},fonttitle=\bfseries,breakable}
\newtcolorbox{numerical}[1][]{colback=gray!5!white,colframe=gray!75!black,title={#1},fonttitle=\bfseries,breakable}
\newtcolorbox{overview}[1][]{colback=blue!5!white,colframe=blue!75!black,title={#1},fonttitle=\bfseries,breakable}
\newtcolorbox{speculation}[1][]{colback=purple!5!white,colframe=purple!75!black,title={#1},fonttitle=\bfseries,breakable}
\newtcolorbox{question}[1][]{colback=orange!5!white,colframe=orange!75!black,title={#1},fonttitle=\bfseries,breakable}
\newtcolorbox{method}[1][]{colback=teal!5!white,colframe=teal!75!black,title={#1},fonttitle=\bfseries,breakable}
\newtcolorbox{correct}[1][]{colback=green!10!white,colframe=green!75!black,title={#1},fonttitle=\bfseries,breakable}
\newtcolorbox{units}[1][]{colback=gray!5!white,colframe=gray!75!black,title={#1},fonttitle=\bfseries,breakable}
\newtcolorbox{achievement}[1][]{colback=gold!5!white,colframe=orange!75!black,title={#1},fonttitle=\bfseries,breakable}
\newtcolorbox{equivalence}[1][]{colback=cyan!5!white,colframe=cyan!75!black,title={#1},fonttitle=\bfseries,breakable}
\newtcolorbox{dimensional}[1][]{colback=purple!5!white,colframe=purple!75!black,title={#1},fonttitle=\bfseries,breakable}

% --- Physics-specific Environments ---
\newtcolorbox{photon}[1][]{colback=yellow!5!white,colframe=yellow!75!black,title={#1},fonttitle=\bfseries,breakable}
\newtcolorbox{neutrino}[1][]{colback=blue!5!white,colframe=blue!75!black,title={#1},fonttitle=\bfseries,breakable}
\newtcolorbox{revolution}[1][]{colback=red!5!white,colframe=red!75!black,title={#1},fonttitle=\bfseries,breakable}
\newtcolorbox{t0box}[1][]{colback=blue!5!white,colframe=t0blue,title={#1},fonttitle=\bfseries,breakable}
\newtcolorbox{documentbox}[1][]{colback=gray!5!white,colframe=gray!75!black,title={#1},fonttitle=\bfseries,breakable}
\newtcolorbox{sibox}[1][]{colback=green!5!white,colframe=green!75!black,title={#1},fonttitle=\bfseries,breakable}
\newtcolorbox{smbox}[1][]{colback=blue!5!white,colframe=blue!75!black,title={#1},fonttitle=\bfseries,breakable}
\newtcolorbox{pvbox}[1][]{colback=purple!5!white,colframe=purple!75!black,title={#1},fonttitle=\bfseries,breakable}
\newtcolorbox{koidebox}[1][]{colback=orange!5!white,colframe=orange!75!black,title={#1},fonttitle=\bfseries,breakable}

% --- German Compatibility Environments ---
\newtcolorbox{formel}[1][]{colback=blue!5!white,colframe=blue!75!black,title={#1},fonttitle=\bfseries,breakable}
\newtcolorbox{schluessel}[1][]{colback=blue!5!white,colframe=blue!75!black,title={#1},fonttitle=\bfseries,breakable}
\newtcolorbox{wichtig}[1][]{colback=red!5!white,colframe=red!75!black,title={#1},fonttitle=\bfseries,breakable}
\newtcolorbox{vorsicht}[1][]{colback=orange!5!white,colframe=orange!75!black,title={#1},fonttitle=\bfseries,breakable}
\newtcolorbox{revolutionaer}[1][]{colback=red!5!white,colframe=red!75!black,title={#1},fonttitle=\bfseries,breakable}
\newtcolorbox{numerisch}[1][]{colback=gray!5!white,colframe=gray!75!black,title={#1},fonttitle=\bfseries,breakable}
\newtcolorbox{experimentell}[1][]{colback=cyan!5!white,colframe=cyan!75!black,title={#1},fonttitle=\bfseries,breakable}
\newtcolorbox{anwendung}[1][]{colback=green!5!white,colframe=green!75!black,title={#1},fonttitle=\bfseries,breakable}
\newtcolorbox{alternative}[1][]{colback=orange!5!white,colframe=orange!75!black,title={#1},fonttitle=\bfseries,breakable}
\newtcolorbox{beziehung}[1][]{colback=cyan!5!white,colframe=cyan!75!black,title={#1},fonttitle=\bfseries,breakable}
\newtcolorbox{folgerung}[1][]{colback=green!5!white,colframe=green!75!black,title={#1},fonttitle=\bfseries,breakable}
\newtcolorbox{abhandlung}[1][]{colback=gray!5!white,colframe=gray!75!black,title={#1},fonttitle=\bfseries,breakable}
\newtcolorbox{prinzipBox}[1][]{colback=blue!5!white,colframe=blue!75!black,title={#1},fonttitle=\bfseries,breakable}
\newtcolorbox{prinzip}[1][]{colback=blue!5!white,colframe=blue!75!black,title={#1},fonttitle=\bfseries,breakable}
\newtcolorbox{beweis}[1][]{colback=gray!5!white,colframe=gray!75!black,title={#1},fonttitle=\bfseries,breakable}
\newtcolorbox{key}[2][]{colback=blue!5!white,colframe=blue!75!black,title={#2},fonttitle=\bfseries,breakable}
\newtcolorbox{category}[1][]{colback=purple!5!white,colframe=purple!75!black,title={#1},fonttitle=\bfseries,breakable}

% =============================================================================
% SECTION 16: Additional Simple Environments
% =============================================================================
\newenvironment{treatise}{\begin{quote}}{\end{quote}}
\newenvironment{gemeinsam}{\begin{quote}}{\end{quote}}
\newenvironment{vergleich}{\begin{quote}}{\end{quote}}
\newenvironment{vorteil}{\begin{quote}}{\end{quote}}
\newenvironment{quantum}{\begin{quote}}{\end{quote}}

% =============================================================================
% SECTION 17: Layout Settings (Kindle-compatible)
% =============================================================================
\sloppy  % Allow more flexible line breaking
\hfuzz=65pt  % Suppress overfull warnings up to 65pt (Kindle compatibility)
\vfuzz=65pt  
\tolerance=9999  % High tolerance for bad line breaks
\emergencystretch=3em  % Extra stretch to avoid overfull boxes
\hbadness=10000  % Suppress underfull box warnings
\raggedbottom

% Environment for wide tables/longtables that need scaling
\newenvironment{scaledtable}[1][0.85]{%
  \begingroup\footnotesize\setlength{\LTleft}{0pt}\setlength{\LTright}{0pt}%
}{%
  \endgroup%
}

% Command for inline table scaling
\newcommand{\widetable}[1]{\resizebox{\textwidth}{!}{#1}}

% =============================================================================
% SECTION 18: Table of Contents Formatting
% =============================================================================
\renewcommand{\cftsecfont}{\color{blue}}
\renewcommand{\cftsubsecfont}{\color{blue}}
\renewcommand{\cftsecpagefont}{\color{blue}}
\renewcommand{\cftsubsecpagefont}{\color{blue}}
\renewcommand{\cfttoctitlefont}{\huge\bfseries\color{blue}}

% =============================================================================
% SECTION 19: Default Header and Footer
% =============================================================================
\pagestyle{fancy}
\fancyhf{}
\fancyhead[L]{\textsc{T0 Theory}}
\fancyhead[R]{\textsc{J. Pascher}}
\fancyfoot[C]{\thepage}

% ==============================================================================
% End of Shared Preamble
% ==============================================================================


\title{DVFT Kapitel 38: BLACK HOLE AND QUANTUM SINGULARITIES\\
\large Angepasst an T0 Zeit-Masse-Dualitätstheorie}

\author{Originalkonzept: Satish B. Thorwe\\
Angepasst an T0-Theorie: Johann Pascher}

\date{27. December 2025}

\begin{document}

\maketitle

\tableofcontents

\newpage


\begin{enumerate}
  \item Introduction
\end{enumerate}
This document presents a full, rigorous DVFT (Dynamic T0-Vakuum field Curvature Theory) explanation of
why *both* classical gravitational singularities (black holes) and quantum singularities (point particles,
infinite self-energy) cannot exi\footnote{Siehe auch T0-Dokument: \texttt{009_T0_xi_ursprung_De.pdf}}st.
In DVFT, spacetime curvature and inertia emerge from the T0-Vakuum amplitude field:
Φ(x,t) = ρ(x,t) e^{i$\theta$(x,t)},
with:
\begin{itemize}
  \item ρ(x,t) – T0-Vakuum amplitude (determines inertia and gravitational potential),
  \item $\theta$(x,t) – phase field (determines quantum coherence and wave-like behavior).
\end{itemize}
Gravity emerges from amplitude gradients:
g = −$\nabla$ρ.
Singularities require ρ → $\infty$ or $\nabla$ρ → $\infty$. DVFT forbids both because the T0-Vakuum has finite stiffness and
inertial density, encoded in the potential U(ρ).
\begin{enumerate}
  \item Why Singularities Cannot Exist in DVFT: The T0-Vakuum Potential U(ρ)
\end{enumerate}
DVFT (als effektive phänomenologische Schicht von T0) postuliert the T0-Vakuum has a microphysical potential:
U(ρ) = Λ₀ + (κ/2)(ρ − ρ₀)² + (λ/4)(ρ − ρ₀)⁴ + …
where:
\begin{itemize}
  \item ρ₀ is the equilibrium T0-Vakuum amplitude,
  \item κ is the elastic stiffness of the T0-Vakuum,
\end{itemize}
International Journal for Multidisciplinary Research (IJFMR)
E-ISSN: 2582-2160 ● Website: www.ijfmr.com ● Email: editor@ijfmr.com
IJFMR250664112 Volume 7, Issue 6, November-December 2025 84
\begin{itemize}
  \item λ stabilizes large deviations of ρ.
\end{itemize}
This potential is strongly convex at large |ρ − ρ₀|.
Thus, any attempt to compress the T0-Vakuum amplitude beyond moderate values requires infinite energy:
U(ρ) → $\infty$ as |ρ − ρ₀| → $\infty$.
Therefore:
\begin{itemize}
  \item ρ cannot diverge,
  \item $\nabla$ρ cannot diverge,
  \item gravitational curvature cannot diverge.
\end{itemize}
This single microphysical fact eliminates *all* singularities in DVFT.
\begin{enumerate}
  \item Removal of Quantum Singularities (Electron, Proton, Point Particles)
\end{enumerate}
Quantum field theory treats electrons and quarks as point particles, leading to:
\begin{itemize}
  \item infinite self-energy,
  \item divergent Coulomb self-field,
  \item undefined gravitational field at r = 0.
\end{itemize}
DVFT replaces a point mass with a finite T0-Vakuum amplitude deformation:
δρ(x) = G m $\int$ d³x' |ψ(x')|² / |x − x'|.
This deformation is always finite because:
\begin{itemize}
  \item |ψ(x)|² is normalizable,
  \item convolution with 1/r smooths the field,
  \item U(ρ) prevents amplitude blow-up.
\end{itemize}
As a result:
\begin{itemize}
  \item no particle has infinite self-energy,
  \item no wavefunction produces a singular potential,
  \item gravity is well-defined even in superposition.
\end{itemize}
Thus quantum singularities are eliminated by T0-Vakuum microphysics, not by renormalization.
\begin{enumerate}
  \item Gravitational Field of a Delocalized Electron
\end{enumerate}
An electron with wavefunction ψ(x,t) generates a T0-Vakuum amplitude profile:
ρ(x,t) = ρ₀ + G m_e $\int$ d³x' |ψ(x',t)|² / |x − x'|.
When ψ(x,t) spreads due to quantum dispersion, the gravitational field spreads with it:
g(x,t) = −$\nabla$ρ(x,t).
This ensures:
\begin{itemize}
  \item gravity is fully compatible with Heisenberg uncertainty,
  \item gravitational fields have finite width,
  \item no r → 0 divergence occurs.
\end{itemize}
DVFT therefore produces the first consistent microscopic definition of gravity for a single quantum
particle.
\begin{enumerate}
  \item Removal of Black Hole Singularities
\end{enumerate}
In classical GR, gravitational collapse leads to infinite curvature at r = 0.
In DVFT, as matter compresses and raises ρ(x), the T0-Vakuum potential U(ρ) rapidly increases. At
sufficiently high density, a phase transition in the T0-Vakuum occurs:
\begin{itemize}
  \item ρ stops increasing (T0-Vakuum stiffness prevents divergence),
  \item $\theta$ becomes phase-locked (coherence inside horizon),
  \item matter transitions into a high-amplitude T0-Vakuum phase state,
\end{itemize}
International Journal for Multidisciplinary Research (IJFMR)
E-ISSN: 2582-2160 ● Website: www.ijfmr.com ● Email: editor@ijfmr.com
IJFMR250664112 Volume 7, Issue 6, November-December 2025 85
\begin{itemize}
  \item gravitational field saturates.
\end{itemize}
Thus the black hole interior is NOT a singularity. It is a region of:
\begin{itemize}
  \item finite ρ,
  \item finite $\nabla$ρ,
  \item finite energy density,
  \item T0-Vakuum-phase condensate.
\end{itemize}
The event horizon may still exist, but the spacetime interior remains regular.
\begin{enumerate}
  \item DVFT Black Hole Interior Structure
\end{enumerate}
DVFT predicts that inside a black hole:
\begin{itemize}
  \item ρ(r) rises toward a maximum allowed value ρ_max,
  \item U(ρ) prevents further growth beyond ρ_max,
  \item curvature saturates,
  \item matter becomes T0-Vakuum-amplitude dominated,
  \item $\theta$ freezes (phase coherence becomes rigid),
  \item no divergence in metric-equivalent quantities occurs.
\end{itemize}
This resembles:
\begin{itemize}
  \item gravastar-like interiors,
  \item T0-Vakuum condensate cores,
  \item nonsingular loop quantum gravity solutions,
  \item but derived *entirely from DVFT microphysics*.
\end{itemize}
\begin{enumerate}
  \item The Deep Reason DVFT Removes Both Types of Singularities
\end{enumerate}
DVFT eliminates singularities because spacetime curvature is not fundamental. It is an *emergent
property* of the T0-Vakuum amplitude field ρ. If ρ cannot diverge, then curvature cannot diverge. The
T0-Vakuum’s elastic potential and finite inertial density are the mechanisms that prevent runaways.
Thus:
\begin{itemize}
  \item matter cannot collapse to infinite density,
  \item wavefunctions cannot create divergent potentials,
  \item curvature cannot become infinite.
\end{itemize}
This is the first unified mechanism eliminating singularities across classical and quantum domains.
\begin{enumerate}
  \item Comparison with GR, LQG, and QFT\footnote{Siehe auch T0-Dokument: \texttt{097_QFT_De.pdf}}
\end{enumerate}
General Relativity (GR):
\begin{itemize}
  \item predicts unavoidable singularities (Hawking-Penrose theorems),
  \item has no internal regulator for curvature.
\end{itemize}
Loop Quantum Gravity (LQG):
\begin{itemize}
  \item introduces discrete geometry,
  \item removes singularities by quantizing spacetime,
  \item but requires radical nonlocality and lacks experimental grounding.
\end{itemize}
Quantum Field Theory:
\begin{itemize}
  \item produces infinite point-particle self-energies,
  \item resolves them only through renormalization,
  \item does not address gravitational singularity.
\end{itemize}
\subsection{Dvft:}
\begin{itemize}
  \item retains continuum spacetime,
\end{itemize}
International Journal for Multidisciplinary Research (IJFMR)
E-ISSN: 2582-2160 ● Website: www.ijfmr.com ● Email: editor@ijfmr.com
IJFMR250664112 Volume 7, Issue 6, November-December 2025 86
\begin{itemize}
  \item derives gravity from a physical T0-Vakuum field,
  \item imposes finite amplitude & stiffness,
  \item eliminates both self-energy and gravitational singularities,
  \item without renormalization,
  \item without quantizing spacetime,
  \item without modifying quantum mechanics.
\end{itemize}
DVFT is the simplest and most physically grounded solution among all three.
\begin{enumerate}
  \item Final Summary
\end{enumerate}
DVFT eliminates singularities through T0-Vakuum amplitude dynamics:
\begin{enumerate}
  \item The T0-Vakuum field Φ = ρ e^{i$\theta$} has finite stiffness and inertial density.
  \item U(ρ) prevents ρ from diverging under collapse.
  \item Quantum particles generate finite T0-Vakuum amplitude deformations from |ψ|².
  \item Gravity emerges as $\nabla$ρ, which can never diverge.
  \item Black holes contain T0-Vakuum-phase condensates, not singularities.
  \item No infinite self-energy, no point divergences, no r → 0 explosion exists.
\end{enumerate}
DVFT therefore provides the first unified, microphysically consistent elimination of:
\begin{itemize}
  \item black hole singularities,
  \item quantum point singularities,
  \item gravitational field singularities.
\end{itemize}
This positions DVFT as a fundamentally complete framework bridging general relativity and quantum
mechanics.

\section{Referenzen zu T0-Dokumenten}

Dieses Kapitel steht in Zusammenhang mit folgenden T0-Dokumenten im Repository \texttt{2/pdf/}:

\begin{itemize}
  \item \texttt{002\_T0\_Grundlagen\_De.pdf} -- T0 Zeit-Masse-Dualität Grundlagen
  \item \texttt{004\_T0\_Energie\_De.pdf} -- T0 Energiefeld-Theorie
  \item \texttt{009\_T0\_xi\_ursprung\_De.pdf} -- Ursprung des geometrischen Parameters $\xi$
  \item \texttt{201\_DVFT-alles\_De.pdf} -- Vollständiges DVFT-Dokument (Rahmenwerk)
\end{itemize}

\end{document}
