\chapter{\textbf{Anomale magnetische Momente in der T0-Theorie}\\[0.5cm]
	\large Phänomenologische Beschreibung mit theoretischer Begründung\\[0.3cm]
	\normalsize Strukturvorhersage und experimentelle Tests}
\let\cleardoublepage\clearpage  % Entfernt leere Seite vor diesem Kapitel
	\section{abstract}
		Die T0-Theorie sagt eindeutig zusätzliche Beiträge zu den anomalen magnetischen 
		Momenten der Leptonen voraus, da die erweiterte Lagrange-Dichte mit Zeitfeld-Termen 
		notwendigerweise die Vertex-Funktionen modifiziert. Wie im Standardmodell verwenden 
		wir eine phänomenologische Parametrisierung dieser Beiträge, die aus der Theorie 
		motiviert ist: $\Delta a_\ell = s_\ell \times \xi^{q_\ell} \times m_\ell^2 \times \alpha$. 
		Die Normierung erfolgt am Myon ($s_\mu = 3.45 \times 10^{-8}$, $q_\mu = 1$), was die 
		Vorhersage für das Tau ermöglicht: $a_\tau = 1.06 \times 10^{-7}$. Die vollständige 
		Berechnung aus ersten Prinzipien ist aufgrund der Komplexität rekursiv 
		ineinanderwirkender Felder in fraktaler Raumzeit derzeit nicht durchführbar -- analog 
		zur Situation hadronischer Beiträge im SM.
	\end{abstract}
	
	\begin{tcolorbox}[colback=yellow!10!white, colframe=orange!75!black, title=Hinweis zu älteren Dokumenten]
		Frühere Versionen der g-2 Analyse in der T0-Theorie 
		(\href{https://github.com/jpascher/T0-Time-Mass-Duality/blob/main/2/pdf/018_T0_Anomale-g2-9_En.pdf}{018\_T0\_Anomale-g2-9\_En.pdf} 
		und \href{https://github.com/jpascher/T0-Time-Mass-Duality/blob/main/2/pdf/031_T0_g2-erweiterung-4_En.pdf}{031\_T0\_g2-erweiterung-4\_En.pdf}) 
		sind mit der hier dargestellten Formulierung \textbf{obsolet}. Diese Dokumente 
		versuchten eine vollständige ab-initio Berechnung, die sich als nicht durchführbar 
		erwies. Sie werden im Repository aufbewahrt aus historischen Gründen und zur 
		Dokumentation des Entwicklungsprozesses, sollten aber nicht mehr als aktuelle 
		Darstellung der Theorie verwendet werden. Die vorliegende phänomenologische 
		Formulierung entspricht dem aktuellen Stand (Januar 2026).
	\end{tcolorbox}
	
	\textbf{Schlüsselwörter:} Anomales magnetisches Moment, g-2, T0-Theorie, 
	Phänomenologie, Strukturvorhersage, Rekursive Felder
	
	\tableofcontents
	
	\section{Einleitung: Phänomenologie vs. Ab-initio Berechnung}
	
	\subsection{Die Situation im Standardmodell}
	
	Im Standardmodell wird das anomale magnetische Moment des Myons durch verschiedene 
	Beiträge beschrieben:
	
	\begin{equation}
		a_\mu^{\text{SM}} = a_\mu^{\text{QED}} + a_\mu^{\text{EW}} + a_\mu^{\text{Had}}
	\end{equation}
	
	Während $a_\mu^{\text{QED}}$ aus ersten Prinzipien berechnet werden kann, ist 
	$a_\mu^{\text{Had}}$ (hadronischer Beitrag) \textbf{nicht} ab initio berechenbar. 
	Stattdessen verwendet man:
	
	\begin{itemize}
		\item \textbf{Datengetriebene Methoden:} Dispersionsrelationen mit $e^+e^-$-Daten
		\item \textbf{Lattice QCD:} Numerische Simulation auf dem Gitter
		\item \textbf{Phänomenologische Modelle:} Parametrisierungen der QCD-Effekte
	\end{itemize}
	
	\textbf{Warum?} Die QCD ist zwar fundamental definiert, aber die Berechnung von 
	Schleifenintegralen mit stark wechselwirkenden Quarks und Gluonen ist extrem komplex.
	
	\subsection{Analoge Situation in der T0-Theorie}
	
	Die T0-Theorie erweitert die Lagrange-Dichte um Zeitfeld-Terme:
	\begin{equation}
		\mathcal{L}_{\text{T0}} = \mathcal{L}_{\text{SM}} + \xi \cdot T_{\text{field}} \cdot (\partial E_{\text{field}})^2 + \text{Kopplungsterme}
	\end{equation}
	
	Diese Erweiterung führt \textbf{zwingend} zu zusätzlichen Beiträgen zum anomalen 
	magnetischen Moment. Jedoch:
	
	\begin{itemize}
		\item Die Zeitfelder wirken \textbf{rekursiv} ineinander
		\item Die fraktale Raumzeit ($D_f = 3 - \xi$) modifiziert Propagatoren
		\item Renormierung in nicht-ganzzahliger Dimension ist hochkomplex
		\item Wechselwirkung mit SM-Feldern führt zu verschränkten Gleichungen
	\end{itemize}
	
	\textbf{Wie im SM:} Die Theorie ist fundamental definiert, aber die explizite 
	Berechnung ist derzeit nicht durchführbar.
	
	\section{Phänomenologische Beschreibung}
	
	\subsection{Theoretisch motivierte Strukturformel}
	
	Aus der Zeit-Masse-Dualität und dimensionaler Analyse folgt die Struktur:
	\begin{equation}
		\Delta a_\ell^{\text{T0}} = s_\ell \times \xi^{q_\ell} \times m_\ell^2 \times \alpha
		\label{eq:phenomenology}
	\end{equation}
	
	Diese Form ist \textbf{nicht willkürlich}, sondern folgt aus:
	
	\begin{enumerate}
		\item \textbf{Zeit-Masse-Dualität:} $T \times m = \hbar$ 
		→ Kopplung $\propto m^2$
		
		\item \textbf{Fraktale Skalierung:} Korrekturen $\propto \xi^q$ analog zur 
		Massenformel $m_\ell = r_\ell \times \xi^{p_\ell} \times v$
		
		\item \textbf{QED-Kopplung:} Vertex-Korrektur enthält Faktor $\alpha$
		
		\item \textbf{Geometrischer Faktor:} $s_\ell$ aus fraktaler Integration 
		(analog zu $r_\ell$ bei Massen)
	\end{enumerate}
	
	\subsection{Normierung am Myon}
	
	Wie im Standardmodell (hadronische Beiträge) verwenden wir experimentelle Daten 
	zur Normierung:
	
	Mit der Myon-Diskrepanz $\Delta a_\mu = 37.5 \times 10^{-11}$ und der Annahme 
	$q_\mu = 1$ (wie $p_\mu = 1$ in der Massenformel) ergibt sich:
	
	\begin{equation}
		s_\mu = \frac{\Delta a_\mu}{\xi \times m_\mu^2 \times \alpha} = 3.45 \times 10^{-8}
	\end{equation}
	
	\textbf{Status:} Dieser Wert ist phänomenologisch bestimmt, aber die 
	\textbf{funktionale Form} von Gleichung~\eqref{eq:phenomenology} ist theoretisch 
	begründet.
	
	\section{Vorhersage für das Tau-Lepton}
	
	\subsection{Strukturvorhersage ohne weitere Parameter}
	
	Mit der Annahme $s_\tau = s_\mu$ (universeller geometrischer Faktor) folgt:
	\begin{equation}
		a_\tau^{\text{T0}} = s_\mu \times \xi \times m_\tau^2 \times \alpha
	\end{equation}
	
	Einsetzen der Zahlenwerte:
	\begin{align*}
		a_\tau^{\text{T0}} &= 3.45 \times 10^{-8} \times 1.333 \times 10^{-4} \times (1777)^2 \times 0.007297 \\
		&= 1.06 \times 10^{-7}
	\end{align*}
	
	\textbf{Wichtig:} Dies ist eine \textbf{Vorhersage}, keine Anpassung! Das 
	Verhältnis ist parameterfrei:
	\begin{equation}
		\frac{a_\tau}{\Delta a_\mu} = \left(\frac{m_\tau}{m_\mu}\right)^2 = 282.8
	\end{equation}
	
	\subsection{Experimenteller Test bei Belle II}
	
	Belle II erwartet bis 2027-2028 eine Sensitivität von $\sim 10^{-7}$ für $a_\tau$.
	
	\textbf{Mögliche Ergebnisse:}
	\begin{itemize}
		\item \textbf{Bestätigung} ($a_\tau \approx 1.1 \times 10^{-7}$): 
		Starke Evidenz für die quadratische Massenskalierung der T0-Theorie
		
		\item \textbf{Abweichung}: Entweder ist $s_\tau \neq s_\mu$ (generationsabhängig) 
		oder die Strukturformel muss modifiziert werden
		
		\item \textbf{Null-Ergebnis} ($a_\tau < 10^{-8}$): Die T0-Beiträge sind 
		unterdrückt oder die Theorie benötigt Revision
	\end{itemize}
	
	\section{Warum die vollständige Berechnung schwierig ist}
	
	\subsection{Natürliche Einheiten vs. physikalische Einheiten}
	
	Die T0-Theorie kann in zwei verschiedenen Formulierungen dargestellt werden:
	
	\textbf{1. Natürliche Einheiten (idealisiert, durchsichtiger):}
	\begin{itemize}
		\item In diesem System: $\hbar = c = \alpha = 1$, $E_0 = 1/\xi$ (dimensionslos)
		\item Alle Größen dimensionslos, nur Verhältnisse
		\item Keine zusätzlichen Konstanten -- alle Physik durch Verhältnisse beschrieben
		\item Nur \textbf{ein Umrechnungsfaktor} zu SI-Einheiten am Ende
		\item Die Physik ist in diesem System durchsichtiger: pure Verhältnisse
	\end{itemize}
	
	\textbf{Beispiel -- Coulomb-Gesetz:}
	\begin{align}
		\text{Physikalisch (SI):} \quad F &= \frac{1}{4\pi\epsilon_0} \frac{e^2}{r^2} 
		= \frac{\alpha \hbar c}{r^2} 
		\quad \text{(viele Konstanten)} \\
		\text{Natürlich:} \quad F &= \frac{1}{r^2} 
		\quad \text{(mit } \alpha = 1 \text{, pure Geometrie)}
	\end{align}
	
	In natürlichen Einheiten wird die Physik auf pure Geometrie reduziert. 
	Am Ende gibt es \textbf{nur einen Umrechnungsfaktor} zu SI-Einheiten.
	
	\textbf{Für g-2 in natürlichen Einheiten:}
	\begin{equation}
		\tilde{a}_\ell = f\left(\frac{m_\ell}{m_\mu}, \xi, D_f\right) 
		\quad \text{(dimensionslos, pure Verhältnisse, KEIN } \alpha \text{!)}
	\end{equation}
	
	Wichtig: In natürlichen Einheiten erscheint $\alpha$ **nirgendwo** in den Formeln! 
	Alle Physik ist durch reine Verhältnisse und Geometrie beschrieben. Man kann 
	Verhältnisse berechnen:
	\begin{equation}
		\frac{\tilde{a}_\tau}{\tilde{a}_\mu} = \left(\frac{m_\tau}{m_\mu}\right)^2 
		\quad \text{(OHNE } \alpha \text{)}
	\end{equation}
	
	Der Wert $\alpha = 1/137$ erscheint **nur** bei der Umrechnung zu SI-Einheiten am 
	Ende:
	\begin{equation}
		a_\ell[\text{SI}] = \left(\text{Umrechnungsfaktor mit } \alpha\right) 
		\times \tilde{a}_\ell
	\end{equation}
	
	\textbf{2. Physikalische SI-Einheiten (gemessen):}
	\begin{itemize}
		\item $E_0 = \sqrt{m_e m_\mu} = 7.398$ MeV (gemessener Wert)
		\item $\alpha = 1/137.036$ (gemessener Wert)
		\item $D_f = 3 - \xi$ (fraktale Dimension)
		\item Alle Konstanten explizit
		\item Direkt experimentell vergleichbar
	\end{itemize}
	
	In SI-Einheiten erscheint $\alpha$ explizit in den Formeln:
	\begin{equation}
		\Delta a_\ell = s_\ell \times \xi \times m_\ell^2 \times \alpha 
		\quad \text{(MIT } \alpha = 1/137 \text{)}
	\end{equation}
	
	\begin{important}{Unsere Formulierung}
		Die Formel $\Delta a_\ell = s_\ell \times \xi \times m_\ell^2 \times \alpha$ mit 
		$\alpha = 1/137$ ist **bereits die SI-Version**! In natürlichen Einheiten wäre 
		die Formel $\tilde{a}_\ell = \tilde{s}_\ell \times \xi \times \tilde{m}_\ell^2$ 
		**ohne** $\alpha$.
	\end{important}
	
	\subsection{Philosophische Bemerkung: Die überbewertete Rolle von $\alpha$}
	
	Ein fundamentales Problem der modernen Physik könnte sein, dass $\alpha$ als zu 
	wichtig angesehen wird. Dies liegt daran, dass wir gewohnt sind, in SI-Einheiten zu 
	rechnen, nicht in Verhältnissen.
	
	\textbf{Die zwei Perspektiven:}
	
	\begin{tabular}{p{7cm}p{7cm}}
		\toprule
		\textbf{Natürliche Einheiten} & \textbf{SI-Einheiten} \\
		\textbf{(fundamental)} & \textbf{(phänomenologisch)} \\
		\midrule
		Pure Verhältnisse & Absolute Werte \\
		Geometrie & Konstanten \\
		$F = 1/r^2$ & $F = \alpha \hbar c / r^2$ \\
		Kein $\alpha$ & $\alpha = 1/137$ \\
		$\xi$ fundamental & $\xi$ und $\alpha$ beide nötig \\
		\bottomrule
	\end{tabular}
	
	\vspace{0.5cm}
	
	\textbf{Die Einsicht:}
	
	$\alpha$ ist eine \textbf{phänomenologische Konstante aus SI-Sicht}, weil wir gewohnt 
	sind, in festen Einheiten (Meter, Kilogramm, Sekunde) zu rechnen statt in 
	Verhältnissen. Sobald man feste Einheiten einführt, braucht man Umrechnungsfaktoren 
	-- und einer davon ist $\alpha$.
	
	\textbf{In natürlichen Einheiten:}
	\begin{itemize}
		\item Alle Physik: Pure Geometrie und Verhältnisse
		\item Coulomb: $F = 1/r^2$ (kein $\alpha$)
		\item Leptonmassen: Verhältnisse aus $(r,p)$ (kein $\alpha$)
		\item g-2: Verhältnisse aus $m^2$-Skalierung (kein $\alpha$)
		\item Fundamentale Konstante: nur $\xi$
	\end{itemize}
	
	\textbf{In SI-Einheiten (unsere Gewohnheit):}
	\begin{itemize}
		\item Feste Längeneinheit (Meter) → braucht $\hbar c$
		\item Feste Masseneinheit (Kilogramm) → braucht $v = 246$ GeV
		\item Feste Ladungseinheit (Coulomb) → braucht $\alpha = 1/137$
		\item Viele Konstanten nötig für Umrechnung
	\end{itemize}
	
	\textbf{Die philosophische Frage:}
	
	Die Frage "Warum ist $\alpha = 1/137$?" ist aus fundamentaler Sicht vielleicht die 
	\textbf{falsche Frage}. Sie fragt nach einem Umrechnungsfaktor zwischen willkürlich 
	gewählten Einheitensystemen. Die richtige Frage wäre: "Welche geometrischen 
	Verhältnisse bestimmen die Physik?" -- und dort erscheint $\alpha$ überhaupt nicht.
	
	\textbf{In der T0-Theorie:}
	\begin{itemize}
		\item Fundamentale Konstante: $\xi = 4/(3 \times 10^4)$ (geometrisch)
		\item Fraktale Dimension: $D_f = 3 - \xi$
		\item Alle Physik aus Verhältnissen
		\item $\alpha$ nur SI-Umrechnungsfaktor (phänomenologisch)
	\end{itemize}
	
	Diese Perspektive könnte erklären, warum viele Versuche, $\alpha$ "aus ersten 
	Prinzipien zu berechnen", scheitern: Man versucht einen phänomenologischen 
	Umrechnungsfaktor zu berechnen, statt die fundamentale Geometrie zu verstehen.
	
	\subsection{Konsequenz für g-2: Verhältnisse sind fundamental}
	
	Die gleiche Einsicht gilt für das anomale magnetische Moment:
	
	\textbf{Unsere aktuelle Herangehensweise (SI-phänomenologisch):}
	\begin{itemize}
		\item Wir versuchen absolute Werte zu berechnen: $a_\mu = 37.5 \times 10^{-11}$
		\item Dafür brauchen wir $s_\mu = 3.45 \times 10^{-8}$ (phänomenologisch!)
		\item Und $\alpha = 1/137$ (phänomenologisch!)
		\item Viele Konstanten, kompliziert
	\end{itemize}
	
	\textbf{Fundamentale Herangehensweise (Verhältnisse):}
	\begin{itemize}
		\item Zuerst das \textbf{Verhältnis} definieren (natürliche Einheiten)
		\item $\tilde{a}_\tau / \tilde{a}_\mu = (m_\tau / m_\mu)^2$ (pure Geometrie!)
		\item Kein $\alpha$, kein $s_\mu$ mit komischen Werten
		\item Nur geometrische Verhältnisse
		\item Dann erst SI-Umrechnung für Experimente
	\end{itemize}
	
	\textbf{Warum das besser wäre:}
	
	Wenn wir g-2 zuerst als \textbf{Verhältnis} definieren, wird klar:
	\begin{enumerate}
		\item Die \textbf{Strukturvorhersage} $a_\tau/a_\mu = (m_\tau/m_\mu)^2$ ist fundamental
		\item Diese ist \textbf{unabhängig} von SI-Einheiten
		\item Keine mysteriösen Konstanten wie $s_\mu = 3.45 \times 10^{-8}$
		\item Pure Geometrie: quadratische Massenskalierung
		\item Experimentell testbar durch Verhältnismessung
	\end{enumerate}
	
	\textbf{In natürlichen Einheiten:}
	\begin{equation}
		\boxed{\frac{\tilde{a}_\tau}{\tilde{a}_\mu} = \left(\frac{m_\tau}{m_\mu}\right)^2 
			\approx 283}
	\end{equation}
	
	Dies ist die \textbf{fundamentale Vorhersage} der T0-Theorie -- eine reine 
	Verhältnisaussage ohne phänomenologische Konstanten!
	
	\textbf{Erst für den SI-Vergleich:}
	\begin{align}
		a_\mu^{\text{SI}} &= (\text{Umrechnungsfaktor}) \times \tilde{a}_\mu \\
		a_\tau^{\text{SI}} &= (\text{Umrechnungsfaktor}) \times \tilde{a}_\tau
	\end{align}
	
	Der Umrechnungsfaktor enthält $\alpha$, $s_\mu$ etc. -- aber das sind 
	\textbf{phänomenologische Größen}, keine fundamentalen Vorhersagen.
	
	\begin{keypoint}[Paradigmenwechsel]
		Die T0-Theorie sagt nicht vorher: "$a_\mu = 37.5 \times 10^{-11}$" (SI-abhängig).
		
		Sie sagt vorher: "$a_\tau/a_\mu = (m_\tau/m_\mu)^2$" (fundamental, SI-unabhängig).
		
		Dieser Verhältniswert ist bei Belle II \textbf{direkt testbar} ohne Kenntnis der 
		absoluten Werte oder der Umrechnungsfaktoren!
	\end{keypoint}
	
	\subsection{Warum Verhältnisse fundamental sind}
	
	Ein entscheidender Vorteil verhältnisbasierter Aussagen:
	
	\textbf{Verhältnisse brauchen keine fraktale Korrektur!}
	
	\begin{itemize}
		\item Fraktale Korrektur: $D_f = 3 - \xi$ ändert alle absoluten Werte
		\item Aber: Verhältnisse bleiben invariant!
		\item $\frac{m_\tau}{m_\mu}$ ist gleich in $D=3$ und $D_f = 3-\xi$
		\item $\frac{a_\tau}{a_\mu}$ ist gleich in $D=3$ und $D_f = 3-\xi$
	\end{itemize}
	
	\textbf{Mathematisch:}
	\begin{align}
		\text{In idealem 3D:} \quad &\frac{\tilde{m}_\tau}{\tilde{m}_\mu} = 
		\frac{r_\tau \xi^{p_\tau}}{r_\mu \xi^{p_\mu}} \\
		\text{In fraktalem } D_f\text{:} \quad &\frac{m_\tau}{m_\mu} = 
		\frac{r_\tau \xi^{p_\tau} \cdot K_{\text{frak}}}{r_\mu \xi^{p_\mu} \cdot K_{\text{frak}}} 
		= \frac{r_\tau \xi^{p_\tau}}{r_\mu \xi^{p_\mu}}
	\end{align}
	
	Der Korrekturfaktor $K_{\text{frak}}$ kürzt sich heraus!
	
	\textbf{Analog für g-2:}
	\begin{align}
		\text{In idealem 3D:} \quad &\frac{\tilde{a}_\tau}{\tilde{a}_\mu} = 
		\left(\frac{\tilde{m}_\tau}{\tilde{m}_\mu}\right)^2 \\
		\text{In fraktalem } D_f\text{:} \quad &\frac{a_\tau}{a_\mu} = 
		\left(\frac{m_\tau}{m_\mu}\right)^2
	\end{align}
	
	Das Verhältnis ist identisch -- unabhängig von der fraktalen Korrektur!
	
	\begin{important}{Fundamentale Vereinfachung}
		Verhältnisbasierte Vorhersagen umgehen das gesamte Problem der fraktalen 
		Korrektur. Man muss nicht wissen:
		\begin{itemize}
			\item Wie genau $K_{\text{frak}}$ berechnet wird
			\item Wie die Transformation $D=3 \to D_f$ funktioniert
			\item Welche Umrechnungsfaktoren zu SI nötig sind
		\end{itemize}
		
		Das Verhältnis $a_\tau/a_\mu = (m_\tau/m_\mu)^2 \approx 283$ ist eine 
		\textbf{reine geometrische Aussage}, gültig in jedem Einheitensystem und 
		unabhängig von allen Korrekturen!
	\end{important}
	
	\textbf{Problem älterer Formulierungen:}
	Einige ältere Dokumente verwendeten $E_0 = 1/\xi = 7500$ GeV (aus natürlichen 
	Einheiten, aber als GeV geschrieben), dann aber direkt mit physikalischem 
	$\alpha = 1/137$ gerechnet. Das ist inkonsistent -- entweder man bleibt in 
	natürlichen Einheiten (α=1, alles dimensionslos) oder man arbeitet in SI 
	(α=1/137, E₀ in MeV).
	
	\textbf{Warum wir SI-Einheiten verwenden:}
	Für experimentelle Vergleiche ist es praktischer, direkt in gemessenen SI-Werten 
	zu rechnen. Das System mit natürlichen Einheiten wäre durchsichtiger für die 
	fundamentale Struktur, aber erfordert am Ende eine Umrechnung zu SI für jeden 
	experimentellen Vergleich.
	
	\subsection{Rekursive Feldgleichungen}
	
	In der T0-Theorie wirken die Felder rekursiv ineinander:
	
	\begin{align}
		T_{\text{field}}(x) &= f(E_{\text{field}}, \psi, A_\mu, ...) \\
		E_{\text{field}}(x) &= g(T_{\text{field}}, \psi, A_\mu, ...) \\
		\psi(x) &= h(T_{\text{field}}, E_{\text{field}}, A_\mu, ...)
	\end{align}
	
	Diese gekoppelten, nichtlinearen Gleichungen können nicht einfach iterativ gelöst 
	werden, da jede Ordnung die vorherigen modifiziert.
	
	\subsection{Fraktale Raumzeit und Renormierung}
	
	In nicht-ganzzahliger Dimension $D_f = 3 - \xi$ ändern sich:
	
	\begin{itemize}
		\item Propagatoren: $\frac{1}{k^2} \to \frac{1}{k^{2-\epsilon}}$ mit 
		$\epsilon = \xi$
		
		\item Phasenraumintegrale: $\int d^3k \to \int d^{3-\xi}k$
		
		\item Renormierungskonstanten: Dimensionsabhängig
		
		\item Unitarität: Muss in fraktaler Dimension neu formuliert werden
	\end{itemize}
	
	Die dimensional regularisierte Quantenfeldtheorie ($D = 4 - \epsilon$) ist etabliert, 
	aber \textbf{fraktale Dimension ist fundamental anders} -- sie ist physikalisch real, 
	nicht nur ein Regularisierungstrick.
	
	\subsection{Vergleich mit QCD-Problemen im SM}
	
	\begin{table}[h]
		\centering
		\begin{tabular}{lcc}
			\toprule
			& \textbf{SM: Hadronische Beiträge} & \textbf{T0: Zeitfeld-Beiträge} \\
			\midrule
			Theorie definiert? & Ja (QCD) & Ja (T0-Lagrange-Dichte) \\
			Ab-initio berechenbar? & Nein (zu komplex) & Nein (zu komplex) \\
			Methode & Dispersion/Lattice & Phänomenologie \\
			Normierung & $e^+e^-$-Daten & Myon-Diskrepanz \\
			Vorhersagekraft & Begrenzt & Tau-Vorhersage \\
			\bottomrule
		\end{tabular}
	\end{table}
	
	\section{Theoretische Begründung der Struktur}
	
	\subsection{Warum $\Delta a \propto m^2$?}
	
	Die quadratische Massenskalierung ist nicht zufällig:
	
	\begin{enumerate}
		\item \textbf{Zeit-Masse-Dualität:} $T \times m = \hbar$ \\
		Zeitfeld-Fluktuation: $\delta T \sim 1/m$ \\
		Energie-Fluktuation: $\delta E \sim m$ \\
		Vertex-Korrektur: $\propto (\delta E)^2 / M^2 \sim m^2/M^2$
		
		\item \textbf{Dimensionale Analyse:} \\
		$[\Delta a] = $ dimensionslos \\
		$[\xi \times m^2 \times \alpha] = [\text{E}]^{-2} \times [\text{E}]^2 \times 1 = 1$ \\
		→ Konsistent!
		
		\item \textbf{Analogie zur Massenformel:} \\
		Massen: $m \propto \xi^p \times v$ (linear in $v$) \\
		g-2: $\Delta a \propto \xi^q \times m^2$ (quadratisch in $m$) \\
		Beide aus derselben geometrischen Struktur
	\end{enumerate}
	
	\subsection{Warum $q_\mu = 1$ wie $p_\mu = 1$?}
	
	Die Beobachtung $p_\mu = 1$ UND $q_\mu = 1$ ist möglicherweise nicht zufällig:
	
	\begin{itemize}
		\item Myon: Zweite Generation, $p_\mu = 1$ (intermediär zwischen $p_e = 3/2$ 
		und $p_\tau = 2/3$)
		
		\item Wenn $q_\ell = p_\ell$ generell gilt, würde das eine tiefere Verbindung 
		zwischen Massen und g-2 andeuten
		
		\item Test: Wenn Belle II $a_\tau$ misst, können wir prüfen ob tatsächlich 
		$q_\tau = p_\tau = 2/3$ gilt
	\end{itemize}
	
	\section{Zusammenfassung und Ausblick}
	
	\subsection{Was wir wissen}
	
	\begin{enumerate}
		\item Die T0-Theorie sagt \textbf{eindeutig} zusätzliche g-2 Beiträge voraus
		
		\item Die \textbf{Struktur} ist theoretisch begründet: 
		$\Delta a_\ell \propto \xi \times m_\ell^2 \times \alpha$
		
		\item Die \textbf{Amplitude} $s_\mu = 3.45 \times 10^{-8}$ ist phänomenologisch 
		(wie hadronische Beiträge im SM)
		
		\item Die \textbf{Tau-Vorhersage} $a_\tau = 1.06 \times 10^{-7}$ ist 
		parameterfrei testbar
	\end{enumerate}
	
	\subsection{Was wir nicht wissen}
	
	\begin{enumerate}
		\item Die \textbf{explizite Berechnung} von $s_\mu$ aus ersten Prinzipien 
		(zu komplex)
		
		\item Ob $s_\tau = s_\mu$ exakt gilt oder generationsabhängig ist
		
		\item Die vollständige Struktur rekursiver Feldgleichungen in fraktaler Raumzeit
		
		\item Ob es eine Relation $s_\ell = f(r_\ell, p_\ell)$ gibt
	\end{enumerate}
	
	\subsection{Experimentelle Tests}
	
	\textbf{Belle II (2027-2028):}
	\begin{itemize}
		\item Test der Vorhersage $a_\tau = 1.06 \times 10^{-7}$
		\item Test der quadratischen Massenskalierung
		\item Möglicher Test von $q_\tau = p_\tau$
	\end{itemize}
	
	\textbf{Fermilab/J-PARC:}
	\begin{itemize}
		\item Weitere Präzisionsverbesserungen für $a_\mu$
		\item Reduktion theoretischer Unsicherheiten
		\item Klarere Bestimmung von $\Delta a_\mu$
	\end{itemize}
	
	\begin{keypoint}[Kernbotschaft]
		Die T0-Theorie behandelt g-2 analog zum Standardmodell: Die fundamentale Theorie 
		ist definiert und sagt Beiträge voraus, aber die vollständige Berechnung ist zu 
		komplex. Wir verwenden eine theoretisch motivierte phänomenologische 
		Parametrisierung, normiert am Myon, die eine parameterfrei Tau-Vorhersage 
		ermöglicht. Die Situation ist vergleichbar mit hadronischen Beiträgen im SM -- 
		nicht ideal, aber pragmatisch und testbar.
	\end{keypoint}
	
	\section*{Weiterführende Literatur}
	
	\textbf{Experimentelle Ergebnisse:}
	\begin{itemize}
		\item Fermilab Muon g-2 (2025): \href{https://muon-g-2.fnal.gov/}{muon-g-2.fnal.gov}
		\item Theory Initiative White Paper: \href{https://arxiv.org/abs/2505.21476}{arXiv:2505.21476}
		\item Belle II: \href{https://www.belle2.org/}{www.belle2.org}
	\end{itemize}
	
	\textbf{Theoretische Hintergründe:}
	\begin{itemize}
		\item Leptonmassen in T0: Systematische Herleitung aus Quantenzahlen
		\item Zeit-Masse-Dualität: Fundamentale Prinzipien
		\item Fraktale Raumzeit: $D_f = 3 - \xi$
	\end{itemize}
	
