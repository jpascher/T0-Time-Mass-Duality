\section{T0-Theorie: Das Delayed-Choice-Quantum-Eraser-Experiment}

Das **Delayed-Choice-Quantum-Eraser (DCQE)**-Experiment gehört zu den faszinierendsten Demonstrationen der Quantenphysik. Es scheint auf Retrokausalität oder darauf hinzudeuten, dass eine zukünftige Messung das vergange Verhalten eines Photons beeinflusst. Diese Sektion analysiert das Experiment im Rahmen der **T0-Zeit-Masse-Dualitäts-Theorie**. Die T0-Interpretation beseitigt Retrokausalität vollständig, indem sie zeigt, dass das Phänomen aus der **fraktalen Phasenkohärenz** im intrinsischen Zeitfeld $T(x,t)$ resultiert. Beim DCQE geht es um Erhaltung, Störung oder Wiederherstellung der Phasenkohärenz im fraktalen Vakuumfeld – nicht um Rückwärtskausalität.

\subsection{Vakuumfeld-Struktur in der T0-Theorie}

In der T0-Theorie entstehen Quantenzustände aus Anregungen des universellen Zeit-Masse-Feldes, das der Dualität
\begin{equation}
	T(x,t) \cdot E(x,t) = 1
\end{equation}
genügt, mit fraktaler Korrektur durch den geometrischen Parameter $\xi = \frac{4}{3} \times 10^{-4}$ und effektiver Dimension $D_f = 3 - \xi \approx 2{,}99987$.

Das Feld lässt sich in Polarform schreiben als
\begin{equation}
	T(x,t) = \rho(x,t) \, e^{i\theta(x,t)}
\end{equation}
wobei $\rho(x,t)$ die Amplitude und $\theta(x,t)$ die durch fraktale Geometrie modulierte intrinsische Phase ist.

Interferenzmuster entstehen aus stabilen relativen Phasen zwischen Feldpfaden:
\begin{equation}
	I(x) = |\Tfield_1(x) + \Tfield_2(x)|^2
\end{equation}
Wenn die Phasendifferenz $\Delta\theta = \theta_1 - \theta_2$ konstant bleibt (modulo fraktaler Dämpfung $e^{-\xi \cdot n}$), ist Interferenz sichtbar. Randomisierung oder Markierung von $\Delta\theta$ zerstört die Interferenz.

\subsection{Was passiert nach den Spalten}

Nach dem Strahlteiler oder den Spalten teilt sich das Zeitfeld in zwei kohärente Zweige:
\begin{equation}
	\Tfield = \Tfield_1 + \Tfield_2
\end{equation}
Diese Kohärenz spiegelt reale Struktur in der fraktalen Phase $\theta(x,t)$ wider. Interferenz entsteht, wenn
\begin{equation}
	\Delta\theta = \text{konstant}
\end{equation}
(innerhalb der $\xi$-Dämpfungsskala). Interferenz ist somit ein **fraktales Phasenkohärenz-Phänomen** im T0-Vakuum, kein probabilistischer Artefakt.

\subsection{Welcher-Weg-Information als fraktale Phasen-Dekohärenz}

Das Einsetzen von Welcher-Weg-Markierungen verknüpft die Zweige mit einem makroskopischen System und führt fraktale Phasenstörungen ein:
\begin{align}
	\theta_1 &\to \theta_1 + \delta\theta_1(\xi) \\
	\theta_2 &\to \theta_2 + \delta\theta_2(\xi)
\end{align}
mit $\delta\theta_1 \neq \delta\theta_2$ auf Skalen größer als $\xi^{-1}$. Die relative Phase $\Delta\theta$ wird undefiniert durch fraktale Dämpfung. Interferenz verschwindet, weil die Phasengradienten nicht mehr kohärent ausgerichtet sind.

\subsection{Der Quantum-Eraser stellt fraktale Phasenkohärenz wieder her}

Der Eraser verändert nicht die Vergangenheit. Er modifiziert die Randbedingungen des Zeitfeldes, indem er die Welcher-Weg-Verknüpfung löscht. Dadurch wird wiederhergestellt:
\begin{equation}
	\Delta\theta = \text{konstant}
\end{equation}
allerdings nur für die korrelierte Teilmenge der Ereignisse, die bei der Koinzidenzzählung ausgewählt wird. Interferenz erscheint ausschließlich in diesen Teilmengen aufgrund wiederhergestellter fraktaler Kohärenz.

\subsection{Warum Delayed Choice in T0 keine Retrokausalität impliziert}

Im T0-Rahmen:
\begin{itemize}
	\item Das intrinsische Zeitfeld $T(x,t)$ spannt den gesamten Aufbau global.
	\item Phasenkohärenz bzw. Dekohärenz ist eine geometrische Eigenschaft des fraktalen Feldes, nicht lokal.
	\item Koinzidenz-Sortierung wählt Ereignisse aus, die mit der wiederhergestellten Phasenbeziehung konsistent sind.
\end{itemize}
Kein Signal läuft rückwärts. Das Feld kodiert bereits alle Korrelationen über die Dualität $T \cdot E = 1$. Die verzögerte Wahl klassifiziert lediglich Ereignisse nach ihrer fraktalen Phasenkompatibilität.

\subsection{T0-Gleichungen für Interferenz und Dekohärenz}

Volle Interferenz:
\begin{equation}
	I(x) = |\Tfield_1(x) + \Tfield_2(x)|^2
\end{equation}

Mit Welcher-Weg-Markierung (Dekohärenz):
\begin{equation}
	\Tfield \to \Tfield_1 e^{i\delta\theta_1(\xi)} + \Tfield_2 e^{i\delta\theta_2(\xi)}, \quad \Delta\theta \to \text{undefiniert}
\end{equation}

Wiederherstellung durch Eraser:
\begin{equation}
	\delta\theta_1(\xi) = \delta\theta_2(\xi) \implies \Delta\theta = \text{konstant}
\end{equation}
Interferenz erscheint wieder nur in der ausgewählten Koinzidenz-Teilmenge.

\subsection{Photonenverhalten in der T0-Theorie}

In T0:
\begin{itemize}
	\item Ein Photon ist eine lokalisierte Anregung auf dem fraktalen Zeitfeld.
	\item Seine „Trajektorie“ wird durch geometrische Phasengradienten in $T(x,t)$ geleitet.
	\item Welcher-Weg-Detektion stört die fraktale Phasenstruktur.
	\item Löschung rekonstruiert die kohärente Phasengeometrie.
\end{itemize}
Dies löst die Paradoxien ohne Retrokausalität oder Beobachterabhängigkeit.

\subsection{Schlussfolgerung}

Das Delayed-Choice-Quantum-Eraser-Experiment benötigt keine Retrokausalität. Die T0-Theorie liefert eine deterministische, geometrische Erklärung: Die fraktale Phase des intrinsischen Zeitfeldes $T(x,t)$ bestimmt die Sichtbarkeit von Interferenz. Welcher-Weg-Information stört fraktale Kohärenz; Löschung stellt sie in korrelierten Teilmengen wieder her. Die verzögerte Wahl beeinflusst die Klassifikation von Ereignissen, nicht ihr Auftreten. T0 vereinigt somit DCQE mit geometrischer Intuition und reproduziert gleichzeitig alle quantenmechanischen Vorhersagen durch die Zeit-Masse-Dualität und $\xi$-Fraktalität.