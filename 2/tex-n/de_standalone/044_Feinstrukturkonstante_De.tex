\documentclass[12pt,a4paper]{article}

% Minimale T0 Standalone Preamble - A4 Format - 25 Zeilen
\RequirePackage{fontspec}
\RequirePackage{unicode-math}
\usepackage[ngerman]{babel}
\usepackage{microtype}
\setmainfont{Inter}
\setmonofont{JetBrains Mono}
\setmathfont{Libertinus Math}
\usepackage{amsmath,amsfonts,amsthm}
\usepackage{mathtools}
\usepackage{graphicx}
\usepackage{xcolor}
\definecolor{t0blue}{RGB}{0,102,204}
\definecolor{t0green}{RGB}{34,139,34}
\definecolor{t0red}{RGB}{204,0,0}
\usepackage{geometry}
\geometry{a4paper,margin=2.5cm}
\usepackage[most]{tcolorbox}
\newtcolorbox{keyresult}[1][]{colback=yellow!5,colframe=t0blue!80,fonttitle=\bfseries,title={#1},breakable}
\newtcolorbox{important}[1][]{colback=red!5,colframe=t0red!80,fonttitle=\bfseries,title={#1},breakable}
\newcommand{\Tfield}{\ensuremath{\mathcal{T}}}
\usepackage{hyperref}
\hypersetup{colorlinks=true,linkcolor=t0blue}


\title{\textbf{Feinstrukturkonstante: Einheitenkonventionen}\\[0.5cm]
	\large Warum $\alpha = 1$ gesetzt werden kann\\[0.3cm]
	\normalsize Ergänzung zu Dokument 011}
\author{}
\date{Januar 2025}

\begin{document}
	
	\maketitle
	
	\begin{abstract}
		Dieses Dokument behandelt die Aspekte der Feinstrukturkonstante, die in Dokument 011 nicht im Detail diskutiert wurden. Der Fokus liegt auf der ausführlichen Begründung, warum und wie $\alpha = 1$ gesetzt werden kann (Heaviside-Lorentz-Konvention), den physikalischen Konsequenzen verschiedener Einheitensysteme, und den historischen sowie praktischen Implikationen der Neudefinition elektromagnetischer Einheiten.
		
		\textbf{Für T0-spezifische Herleitungen} (charakteristische Energie $E_0$, geometrischer Parameter $\xi$, T0-Formel $\alpha = \xi(E_0/1\text{MeV})^2$) siehe Dokument 011.
	\end{abstract}
	
	\tableofcontents
	\newpage
	
	% ============================================================================
	\section{Einleitung und Verweis auf Dokument 011}
	
	\subsection{Abgrenzung zu Dokument 011}
	
	\textbf{Dokument 011} behandelt ausführlich:
	\begin{itemize}
		\item T0-Herleitung: $\alpha = \xi(E_0/1\text{MeV})^2$
		\item Charakteristische Energie: $E_0 = \sqrt{m_e \cdot m_\mu} = 7{,}398$ MeV
		\item Geometrischer Parameter: $\xi = \frac{4}{3} \times 10^{-4}$
		\item Alternative Formulierungen: mit $\mu_0$, mit $r_e/\lambda_C$, etc.
		\item Historischer Kontext (Sommerfeld)
		\item Natürliche Einheiten und Energie als fundamentales Feld
		\item Detaillierte Dimensionsanalyse aller Formulierungen
	\end{itemize}
	
	\textbf{Dieses Dokument (044)} konzentriert sich auf:
	\begin{itemize}
		\item \textbf{Warum} $\alpha = 1$ gesetzt werden kann (ausführliche Begründung)
		\item \textbf{Wie} verschiedene Einheitenkonventionen funktionieren
		\item Konsequenzen einer Neudefinition des Coulomb
		\item Heaviside-Lorentz vs. Gauss vs. SI-Einheiten
		\item Praktische Aspekte und historische Entwicklung
		\item Fine-Ungleichung vs. Feinstrukturkonstante (Namensverwechslung)
	\end{itemize}
	
	\subsection{Warum zwei Dokumente?}
	
	\textbf{Dokument 011:} T0-Theorie und physikalische Herleitungen
	
	\textbf{Dokument 044:} Einheitensysteme und Konventionen
	
	Beide ergänzen sich, überschneiden sich aber minimal.
	
	% ============================================================================
	\section{Verschiedene Einheitenkonventionen für $\alpha$}
	
	\subsection{Überblick der Systeme}
	
	Die Feinstrukturkonstante kann in verschiedenen Einheitensystemen ausgedrückt werden:
	
	\begin{table}[h]
		\centering
		\begin{tabular}{|l|c|c|}
			\hline
			\textbf{System} & \textbf{Formel} & \textbf{Wert} \\
			\hline
			SI-Standard & $\alpha = \frac{e^2}{4\pi\varepsilon_0\hbar c}$ & $\approx \frac{1}{137}$ \\
			Heaviside-Lorentz & $\alpha = \frac{e^2}{4\pi}$ (mit $\hbar = c = 4\pi\varepsilon_0 = 1$) & $1$ oder $\frac{1}{137}$ \\
			Gauss (cgs) & $\alpha = \frac{e^2}{\hbar c}$ & $\approx \frac{1}{137}$ \\
			\hline
		\end{tabular}
		\caption{Einheitensysteme für $\alpha$}
	\end{table}
	
	\textbf{Wichtig:} Der numerische Wert hängt von der Konvention ab, die \textit{physikalischen} Vorhersagen nicht!
	
	% ============================================================================
	\section[Heaviside-Lorentz-Einheiten im Detail]{Heaviside-Lorentz-Einheiten im Detail}
	
	\subsection{Was sind Heaviside-Lorentz-Einheiten?}
	
	Das Heaviside-Lorentz-System ist eine Variante natürlicher Einheiten, speziell für Elektrodynamik:
	
	\begin{equation}
		\boxed{\hbar = c = 4\pi\varepsilon_0 = 1}
	\end{equation}
	
	\textbf{Konsequenzen:}
	\begin{itemize}
		\item Die elektromagnetischen Gleichungen werden symmetrischer
		\item Der Faktor $4\pi$ verschwindet aus vielen Formeln
		\item Die Elementarladung wird umdefiniert
	\end{itemize}
	
	\subsection[Warum 4 pi epsilon 0 = 1]{Warum $4\pi\varepsilon_0 = 1$?}
	
	In SI-Einheiten erscheint $4\pi$ in vielen elektromagnetischen Formeln:
	\begin{align}
		\vec{E} &= \frac{1}{4\pi\varepsilon_0} \frac{q}{r^2} \hat{r} \quad \text{(Coulomb-Gesetz)} \\
		\nabla \cdot \vec{E} &= \frac{\rho}{\varepsilon_0} \quad \text{(Maxwell-Gleichung)}
	\end{align}
	
	Mit $4\pi\varepsilon_0 = 1$ werden diese zu:
	\begin{align}
		\vec{E} &= \frac{q}{r^2} \hat{r} \\
		\nabla \cdot \vec{E} &= 4\pi\rho
	\end{align}
	
	Der Faktor $4\pi$ wandert von Coulomb-Gesetz zu Poisson-Gleichung!
	
	\subsection{Feinstrukturkonstante in Heaviside-Lorentz}
	
	\textbf{Ausgangspunkt (SI):}
	\begin{equation}
		\alpha = \frac{e^2}{4\pi\varepsilon_0\hbar c}
	\end{equation}
	
	\textbf{Mit $\hbar = c = 4\pi\varepsilon_0 = 1$:}
	\begin{equation}
		\alpha = \frac{e^2}{1 \cdot 1 \cdot 1} = e^2
	\end{equation}
	
	\textbf{Jetzt die entscheidende Frage:} Welchen Wert hat $e$ in diesem System?
	
	% ============================================================================
	\section[Zwei Varianten von Heaviside-Lorentz]{Zwei Varianten von Heaviside-Lorentz}
	
	\subsection[Variante A]{Variante A: $e$ so normieren dass $\alpha = 1$}
	
	\textbf{Ansatz:} Wir definieren die Einheit der Ladung so, dass $\alpha = 1$.
	
	Da $\alpha = e^2$ in HL-Einheiten:
	\begin{equation}
		e^2 = 1 \quad \Rightarrow \quad e = 1
	\end{equation}
	
	\textbf{Physikalische Bedeutung:}
	\begin{itemize}
		\item Die Elementarladung wird zur \textit{dimensionslosen Einheit}
		\item Elektromagnetische Kopplung ist ''normiert''
		\item Ladung wird in Einheiten von $\sqrt{\hbar c}$ gemessen
	\end{itemize}
	
	\textbf{Was ändert sich?}
	
	Die Elementarladung bekommt einen neuen numerischen Wert:
	\begin{equation}
		e_{\text{HL}} = \sqrt{4\pi\varepsilon_0\hbar c} \quad \text{(in SI-Einheiten ausgedrückt)}
	\end{equation}
	
	Numerisch:
	\begin{align}
		e_{\text{HL}} &= \sqrt{4\pi \times 8{,}854 \times 10^{-12} \times 1{,}055 \times 10^{-34} \times 3 \times 10^8} \\
		&\approx 5{,}29 \times 10^{-19} \text{ (neue Ladungseinheit)}
	\end{align}
	
	Das ist etwa $\sqrt{137} \times e_{\text{SI}}$!
	
	\subsection{Variante B: $e$ beibehalten, $\alpha \approx 1/137$}
	
	\textbf{Ansatz:} Die Elementarladung behält ihren ''natürlichen'' Wert.
	
	In diesem Fall:
	\begin{equation}
		\alpha = e^2 \approx \frac{1}{137}
	\end{equation}
	
	weil $e$ in diesen Einheiten den Wert $\approx 1/\sqrt{137}$ hat.
	
	\textbf{Physikalische Bedeutung:}
	\begin{itemize}
		\item Ladung behält physikalische Bedeutung
		\item $\alpha$ bleibt $\approx 1/137$
		\item Nur die mathematische Form vereinfacht sich
	\end{itemize}
	
	\subsection{Welche Variante wird verwendet?}
	
	\textbf{In der Praxis:}
	\begin{itemize}
		\item \textbf{T0-Theorie:} Setzt **alle** Konstanten = 1 ($c = \hbar = \alpha = G = 1$)
		\item \textbf{Theoretische Hochenergiephysik:} Oft $\hbar = c = 1$, manchmal auch $\alpha = 1$
		\item \textbf{Numerische Rechnungen:} Oft $\hbar = c = 1$, aber $\alpha \approx 1/137$
		\item \textbf{Experimentelle Physik:} Fast immer SI-Einheiten (alle Konstanten haben Zahlenwerte)
	\end{itemize}
	
	\textbf{T0-Konvention:}
	\begin{itemize}
		\item In T0-Rechnungen: $c = \hbar = \alpha = G = 1$ (maximale Vereinfachung)
		\item Einziger freier Parameter: $\xi = \frac{4}{3} \times 10^{-4}$
		\item Beim Vergleich mit Experimenten: SI-Werte ($c = 3 \times 10^8$ m/s, $\alpha \approx 1/137$, etc.)
		\item Beide beschreiben dieselbe Physik!
	\end{itemize}
	
	% ============================================================================
	\section{Rekonstruktion von SI-Werten aus T0}
	\label{sec:si_reconstruction}
	
	\subsection{Das zentrale Prinzip}
	
	\textbf{Wichtige Erkenntnis:} Obwohl T0 alle Konstanten auf 1 setzt, können die SI-Werte rekonstruiert werden!
	
	\begin{tcolorbox}[colback=green!5!white,colframe=green!75!black,title=T0-Rekonstruktion]
		\textbf{In T0-Rechnungen:}
		\begin{itemize}
			\item Alle Konstanten = 1: $c = \hbar = \alpha = G = 1$
			\item Einziger freier Parameter: $\xi = \frac{4}{3} \times 10^{-4}$
			\item Formeln maximal vereinfacht
		\end{itemize}
		
		\textbf{Rekonstruktion SI-Werte:}
		\begin{itemize}
			\item Feinstrukturkonstante: $\alpha_{\text{SI}} = \xi(E_0/1\text{MeV})^2 \approx 1/137$
			\item Gravitationskonstante: $G_{\text{SI}} = \frac{\xi^2}{4m_e} \times$ Faktoren
			\item Alle anderen Konstanten: aus $\xi$ ableitbar
		\end{itemize}
	\end{tcolorbox}
	
	\subsection{Beispiel: Feinstrukturkonstante}
	
	\textbf{In T0-Einheiten:}
	\begin{equation}
		\alpha = 1
	\end{equation}
	
	\textbf{Rekonstruktion SI-Wert:}
	\begin{equation}
		\alpha_{\text{SI}} = \xi \cdot \left(\frac{E_0}{1\,\text{MeV}}\right)^2
	\end{equation}
	
	Mit $\xi = \frac{4}{3} \times 10^{-4}$ und $E_0 = 7{,}398$ MeV:
	\begin{align}
		\alpha_{\text{SI}} &= 1{,}3333 \times 10^{-4} \times (7{,}398)^2 \\
		&= 1{,}3333 \times 10^{-4} \times 54{,}73 \\
		&= 7{,}297 \times 10^{-3} \\
		&= \frac{1}{137{,}04}
	\end{align}
	
	Experimentell: $\alpha_{\text{exp}} = \frac{1}{137{,}036}$
	
	Übereinstimmung: 0{,}03\% ✓
	
	\subsection{Beispiel: Gravitationskonstante}
	
	\textbf{In T0-Einheiten:}
	\begin{equation}
		G = 1
	\end{equation}
	
	\textbf{Rekonstruktion SI-Wert:}
	\begin{equation}
		G_{\text{SI}} = \frac{\xi^2}{4m_e} \times C_{\text{dim}} \times C_{\text{conv}}
	\end{equation}
	
	wobei:
	\begin{itemize}
		\item $C_{\text{dim}}$ = Dimensionsumrechnung (nat. Einheiten → SI)
		\item $C_{\text{conv}}$ = Konversionsfaktoren (eV → J, etc.)
	\end{itemize}
	
	Detaillierte Herleitung siehe Dokument 012 (Gravitation).
	
	\subsection{Warum funktioniert das?}
	
	\textbf{Schlüssel:} $\xi$ ist dimensionslos und universell!
	
	\begin{enumerate}
		\item In T0: $\xi$ bestimmt alle Kopplungsstärken
		
		\item In SI: $\xi$ zusammen mit charakteristischen Energien ($E_0$, Massen) rekonstruiert alle Konstanten
		
		\item Physikalische Vorhersagen: identisch in beiden Systemen!
		
		\item Nur die mathematische Darstellung unterscheidet sich
	\end{enumerate}
	
	\subsection{Vergleichstabelle}
	
	\begin{table}[h]
		\centering
		\begin{tabular}{|l|c|c|l|}
			\hline
			\textbf{Konstante} & \textbf{T0} & \textbf{SI} & \textbf{Rekonstruktion} \\
			\hline
			$c$ & $1$ & $3 \times 10^8$ m/s & Konvention \\
			$\hbar$ & $1$ & $1{,}055 \times 10^{-34}$ J$\cdot$s & Konvention \\
			$\alpha$ & $1$ & $\approx 1/137$ & $\xi(E_0/1\text{MeV})^2$ \\
			$G$ & $1$ & $6{,}67 \times 10^{-11}$ m³/(kg$\cdot$s²) & $\xi^2/(4m_e) \times$ Faktoren \\
			$\xi$ & $\frac{4}{3} \times 10^{-4}$ & $\frac{4}{3} \times 10^{-4}$ & \textbf{Gleich!} \\
			\hline
		\end{tabular}
		\caption{T0 vs. SI - Rekonstruktion der Konstanten}
	\end{table}
	
	\subsection{Wichtige Schlussfolgerung}
	
	\begin{itemize}
		\item \textbf{T0 ist keine neue Physik}, sondern eine Umparametrisierung
		
		\item Statt vieler Konstanten ($c$, $\hbar$, $\alpha$, $G$, ...) nur \textbf{einen Parameter} $\xi$
		
		\item Alle SI-Werte rekonstruierbar aus $\xi$ und Energieskalen
		
		\item Vorteil: Formeln einfacher, physikalische Zusammenhänge klarer
		
		\item Nachteil: Umrechnung zu SI für Experimente nötig
	\end{itemize}
	
	% ============================================================================
	\section{Warum kann $\alpha = 1$ gesetzt werden?}
	
	\subsection{Fundamentale Einsicht}
	
	\begin{tcolorbox}[colback=blue!5!white,colframe=blue!75!black,title=Kernaussage]
		Die Feinstrukturkonstante $\alpha$ ist eine \textbf{dimensionslose Zahl}. Ihr numerischer Wert ist \textbf{konventionsabhängig}, nicht fundamental!
		
		Man kann $\alpha = 1$ setzen, indem man die \textbf{Einheit der Ladung} entsprechend umdefiniert.
	\end{tcolorbox}
	
	\subsection{Schritt-für-Schritt-Begründung}
	
	\textbf{Schritt 1: Was ist eine Konvention?}
	
	SI-Einheiten sind historisch gewachsene Definitionen:
	\begin{itemize}
		\item 1 Meter = ursprünglich 1/10.000.000 Erdmeridian
		\item 1 Sekunde = ursprünglich 1/86.400 eines Sonnentages
		\item 1 Coulomb = definiert über Ampere und Kraft zwischen Strömen
	\end{itemize}
	
	Keine davon ist ''fundamental''!
	
	\textbf{Schritt 2: $\alpha$ in SI}
	
	In SI-Einheiten:
	\begin{equation}
		\alpha = \frac{e^2}{4\pi\varepsilon_0\hbar c} \approx \frac{1}{137}
	\end{equation}
	
	Der Wert $1/137$ folgt aus:
	\begin{itemize}
		\item Wie wir das Coulomb definiert haben (historisch)
		\item Wie wir $\varepsilon_0$ definiert haben (über $\mu_0$ und $c$)
	\end{itemize}
	
	\textbf{Schritt 3: Umdefinition}
	
	Wir können sagen: Äb jetzt ist die Elementarladung nicht mehr $1{,}602 \times 10^{-19}$ C, sondern $e = \sqrt{4\pi\varepsilon_0\hbar c}$.''
	
	Dann wird automatisch:
	\begin{equation}
		\alpha = \frac{(\sqrt{4\pi\varepsilon_0\hbar c})^2}{4\pi\varepsilon_0\hbar c} = 1
	\end{equation}
	
	\textbf{Schritt 4: Physikalische Konsequenzen}
	
	\begin{itemize}
		\item \textbf{Keine physikalischen Vorhersagen ändern sich!}
		\item Nur die \textit{Zahlen} in Formeln ändern sich
		\item Alle Verhältnisse bleiben gleich
		\item Alle Experimente geben dieselben Ergebnisse
	\end{itemize}
	
	\subsection{Analogie: Temperaturskalen}
	
	\textbf{Celsius:} Wasser gefriert bei 0°C
	
	\textbf{Fahrenheit:} Wasser gefriert bei 32°F
	
	\textbf{Kelvin:} Wasser gefriert bei 273{,}15 K
	
	Ist eine dieser Skalen ''richtig''? Nein! Sie sind Konventionen.
	
	Genauso ist $\alpha = 1/137$ (SI) vs. $\alpha = 1$ (HL) nur eine Konventionswahl!
	
	% ============================================================================
	\section{Konsequenzen einer Neudefinition des Coulomb}
	
	\subsection{Was bedeutet es, die Elementarladung neu zu definieren?}
	
	Wenn $e$ so umdefiniert wird, dass $\alpha = 1$:
	
	\textbf{Alte Definition (SI):}
	\begin{equation}
		e = 1{,}602 \times 10^{-19} \text{ C}
	\end{equation}
	
	\textbf{Neue Definition (HL mit $\alpha = 1$):}
	\begin{equation}
		e = 1 \quad \text{(dimensionslos in nat. Einheiten)}
	\end{equation}
	
	oder in SI-Einheiten ausgedrückt:
	\begin{equation}
		e_{\text{neu}} = \sqrt{4\pi\varepsilon_0\hbar c} \approx 5{,}29 \times 10^{-19} \text{ (neue Ladungseinheit)}
	\end{equation}
	
	\subsection{Auswirkungen auf elektromagnetische Größen}
	
	\subsubsection{Elektrischer Strom (Ampere)}
	
	Da $1 \text{ A} = 1 \text{ C/s}$:
	\begin{equation}
		1 \text{ A}_{\text{neu}} = \frac{e_{\text{neu}}}{1 \text{ s}} = \sqrt{137} \times 1 \text{ A}_{\text{alt}}
	\end{equation}
	
	\subsubsection{Elektrische Spannung (Volt)}
	
	$1 \text{ V} = 1 \text{ J/C}$:
	\begin{equation}
		1 \text{ V}_{\text{neu}} = \frac{1 \text{ J}}{e_{\text{neu}}} = \frac{1}{\sqrt{137}} \times 1 \text{ V}_{\text{alt}}
	\end{equation}
	
	\subsubsection{Kapazität (Farad)}
	
	\begin{equation}
		1 \text{ F}_{\text{neu}} = \frac{e_{\text{neu}}}{1 \text{ V}_{\text{neu}}} = 137 \times 1 \text{ F}_{\text{alt}}
	\end{equation}
	
	\subsection{Sind diese Änderungen ''real''?}
	
	\textbf{Nein!} Es sind nur Umrechnungsfaktoren, wie bei Celsius → Fahrenheit.
	
	\textbf{Alle physikalischen Verhältnisse bleiben identisch:}
	\begin{itemize}
		\item Kapazität eines Kondensators / Abstand: gleich
		\item Kraft zwischen Ladungen / Abstand²: gleich
		\item Alle Experimente: gleiche Ergebnisse
	\end{itemize}
	
	Nur die \textit{Zahlenwerte} mit denen wir rechnen ändern sich!
	
	% ============================================================================
	\section{Praktische Auswirkungen auf alltägliche Berechnungen}
	\label{sec:practical_calculations}
	
	\subsection{Motivation}
	
	\textbf{Frage:} Wenn wir $\alpha = 1$ setzen, was bedeutet das für gewöhnliche elektrische Berechnungen mit Volt, Ampere, Widerstand, Kapazität?
	
	\textbf{Antwort:} Alle Formeln ändern sich, aber die \textit{physikalischen Ergebnisse} bleiben identisch!
	
	\subsection{Beispiel 1: Ohmsches Gesetz}
	
	\subsubsection{In SI-Einheiten (Standard)}
	
	\begin{equation}
		U = R \cdot I
	\end{equation}
	
	\textbf{Numerisches Beispiel:}
	\begin{itemize}
		\item Widerstand: $R = 100$ $\Omega$
		\item Strom: $I = 2$ A
		\item Spannung: $U = 100 \times 2 = 200$ V
	\end{itemize}
	
	\subsubsection{In Heaviside-Lorentz mit $\alpha = 1$}
	
	Die Formel bleibt $U = R \cdot I$, aber die \textit{Zahlenwerte} ändern sich!
	
	\textbf{Umrechnung der Einheiten:}
	\begin{align}
		1 \text{ A}_{\text{neu}} &= \sqrt{137} \times 1 \text{ A}_{\text{alt}} \approx 11{,}7 \text{ A}_{\text{alt}} \\
		1 \text{ V}_{\text{neu}} &= \frac{1}{\sqrt{137}} \times 1 \text{ V}_{\text{alt}} \approx 0{,}085 \text{ V}_{\text{alt}} \\
		1 \text{ }\Omega_{\text{neu}} &= \frac{1}{137} \times 1 \text{ }\Omega_{\text{alt}}
	\end{align}
	
	\textbf{Dieselbe Schaltung in neuen Einheiten:}
	\begin{align}
		R_{\text{neu}} &= 100 \times \frac{1}{137} \approx 0{,}73 \text{ }\Omega_{\text{neu}} \\
		I_{\text{neu}} &= 2 \times \sqrt{137} \approx 23{,}4 \text{ A}_{\text{neu}} \\
		U_{\text{neu}} &= 0{,}73 \times 23{,}4 = 17{,}1 \text{ V}_{\text{neu}}
	\end{align}
	
	\textbf{Umrechnung zurück zu SI:}
	\begin{equation}
		U_{\text{neu}} = 17{,}1 \times 0{,}085 \text{ V}_{\text{alt}} = 200 \text{ V} \quad \checkmark
	\end{equation}
	
	Identisches Ergebnis!
	
	\subsection{Beispiel 2: Leistung einer Glühbirne}
	
	\subsubsection{In SI-Einheiten}
	
	\begin{equation}
		P = U \cdot I = \frac{U^2}{R}
	\end{equation}
	
	\textbf{Glühbirne:} 60 W bei 230 V
	
	\begin{align}
		R &= \frac{U^2}{P} = \frac{(230)^2}{60} = 882 \text{ }\Omega \\
		I &= \frac{P}{U} = \frac{60}{230} = 0{,}26 \text{ A}
	\end{align}
	
	\subsubsection{In Heaviside-Lorentz mit $\alpha = 1$}
	
	\textbf{Leistung:} $1 \text{ W}_{\text{neu}} = 1 \text{ W}_{\text{alt}}$ (Energie/Zeit ändert sich nicht in HL!)
	
	\begin{align}
		U_{\text{neu}} &= 230 \times \frac{1}{\sqrt{137}} = 19{,}6 \text{ V}_{\text{neu}} \\
		R_{\text{neu}} &= 882 \times \frac{1}{137} = 6{,}44 \text{ }\Omega_{\text{neu}} \\
		I_{\text{neu}} &= \frac{P}{U_{\text{neu}}} = \frac{60}{19{,}6} = 3{,}06 \text{ A}_{\text{neu}}
	\end{align}
	
	\textbf{Verifikation:}
	\begin{equation}
		P = U_{\text{neu}} \cdot I_{\text{neu}} = 19{,}6 \times 3{,}06 = 60 \text{ W} \quad \checkmark
	\end{equation}
	
	\subsection{Beispiel 3: Kondensator laden}
	
	\subsubsection{In SI-Einheiten}
	
	\begin{equation}
		Q = C \cdot U
	\end{equation}
	
	\textbf{Kondensator:} $C = 100$ $\mu$F bei $U = 12$ V
	
	\begin{equation}
		Q = 100 \times 10^{-6} \times 12 = 1{,}2 \times 10^{-3} \text{ C}
	\end{equation}
	
	\textbf{Gespeicherte Energie:}
	\begin{equation}
		E = \frac{1}{2} C U^2 = \frac{1}{2} \times 100 \times 10^{-6} \times 144 = 7{,}2 \times 10^{-3} \text{ J}
	\end{equation}
	
	\subsubsection{In Heaviside-Lorentz mit $\alpha = 1$}
	
	\textbf{Umrechnung:}
	\begin{align}
		1 \text{ F}_{\text{neu}} &= 137 \times 1 \text{ F}_{\text{alt}} \\
		1 \text{ C}_{\text{neu}} &= \sqrt{137} \times 1 \text{ C}_{\text{alt}}
	\end{align}
	
	\begin{align}
		C_{\text{neu}} &= 100 \times 10^{-6} \times 137 = 0{,}0137 \text{ F}_{\text{neu}} \\
		U_{\text{neu}} &= 12 \times \frac{1}{\sqrt{137}} = 1{,}025 \text{ V}_{\text{neu}} \\
		Q_{\text{neu}} &= 0{,}0137 \times 1{,}025 = 0{,}014 \text{ C}_{\text{neu}}
	\end{align}
	
	\textbf{Umrechnung zurück:}
	\begin{equation}
		Q_{\text{neu}} = 0{,}014 \times \frac{1}{\sqrt{137}} = 1{,}2 \times 10^{-3} \text{ C}_{\text{alt}} \quad \checkmark
	\end{equation}
	
	\textbf{Energie:}
	\begin{equation}
		E_{\text{neu}} = \frac{1}{2} \times 0{,}0137 \times (1{,}025)^2 = 7{,}2 \times 10^{-3} \text{ J} \quad \checkmark
	\end{equation}
	
	Energie ist in allen Systemen gleich!
	
	\subsection{Beispiel 4: RC-Zeitkonstante}
	
	\subsubsection{In SI-Einheiten}
	
	\begin{equation}
		\tau = R \cdot C
	\end{equation}
	
	\textbf{Schaltung:} $R = 1$ k$\Omega$, $C = 10$ $\mu$F
	
	\begin{equation}
		\tau = 1000 \times 10 \times 10^{-6} = 0{,}01 \text{ s} = 10 \text{ ms}
	\end{equation}
	
	\subsubsection{In Heaviside-Lorentz mit $\alpha = 1$}
	
	\begin{align}
		R_{\text{neu}} &= 1000 \times \frac{1}{137} = 7{,}3 \text{ }\Omega_{\text{neu}} \\
		C_{\text{neu}} &= 10 \times 10^{-6} \times 137 = 1{,}37 \times 10^{-3} \text{ F}_{\text{neu}}
	\end{align}
	
	\begin{equation}
		\tau_{\text{neu}} = 7{,}3 \times 1{,}37 \times 10^{-3} = 0{,}01 \text{ s} = 10 \text{ ms} \quad \checkmark
	\end{equation}
	
	\textbf{Zeit bleibt gleich!} Das ist wichtig: Physikalische Zeitskalen ändern sich nicht!
	
	\subsection{Zusammenfassung praktische Berechnungen}
	
	\begin{table}[h]
		\centering
		\begin{tabular}{|l|c|c|c|}
			\hline
			\textbf{Größe} & \textbf{SI} & \textbf{HL-Faktor} & \textbf{HL ($\alpha=1$)} \\
			\hline
			Ladung (Q) & C & $\sqrt{137}$ & $\sqrt{137}$ C \\
			Strom (I) & A & $\sqrt{137}$ & $\sqrt{137}$ A \\
			Spannung (U) & V & $1/\sqrt{137}$ & V$/\sqrt{137}$ \\
			Widerstand (R) & $\Omega$ & $1/137$ & $\Omega/137$ \\
			Kapazität (C) & F & $137$ & $137$ F \\
			Leistung (P) & W & $1$ & W (unverändert!) \\
			Energie (E) & J & $1$ & J (unverändert!) \\
			Zeit ($\tau$) & s & $1$ & s (unverändert!) \\
			\hline
		\end{tabular}
		\caption{Umrechnungsfaktoren SI → HL mit $\alpha = 1$}
	\end{table}
	
	\subsection{Wichtige Erkenntnisse}
	
	\begin{tcolorbox}[colback=green!5!white,colframe=green!75!black,title=Kernaussage]
		\textbf{Was ändert sich:}
		\begin{itemize}
			\item Zahlenwerte für Ladung, Strom, Spannung, Widerstand, Kapazität
		\end{itemize}
		
		\textbf{Was NICHT ändert sich:}
		\begin{itemize}
			\item Energie
			\item Leistung
			\item Zeit
			\item Alle physikalischen Verhältnisse
			\item Alle experimentellen Ergebnisse
		\end{itemize}
		
		\textbf{Fazit:} Es ist nur eine Umrechnung, wie Meter ↔ Fuß!
	\end{tcolorbox}
	
	\subsection{Warum verwendet niemand $\alpha = 1$ in der Praxis?}
	
	\textbf{Gründe:}
	\begin{enumerate}
		\item \textbf{Messgeräte:} Alle Voltmeter, Amperemeter, etc. sind in SI kalibriert
		
		\item \textbf{Standards:} Weltweit akzeptierte SI-Definitionen
		
		\item \textbf{Intuition:} Ingenieure kennen typische Werte in SI
		\begin{itemize}
			\item Haushalt: 230 V, nicht 1{,}96 V$_{\text{neu}}$
			\item USB: 5 V, nicht 0{,}43 V$_{\text{neu}}$
		\end{itemize}
		
		\item \textbf{Umrechnung aufwendig:} $\sqrt{137}$ Faktoren überall
		
		\item \textbf{Keine Vorteile für Praktiker:} Vereinfachung nur in theoretischen Formeln sichtbar
	\end{enumerate}
	
	\textbf{Aber:} Für theoretische Rechnungen (QED, Feynman-Diagramme) ist $\alpha = 1$ oft sehr hilfreich!
	
	% ============================================================================
	\section{Praktische Aspekte verschiedener Systeme}
	
	\subsection{Vor- und Nachteile: SI-Einheiten}
	
	\textbf{Vorteile:}
	\begin{itemize}
		\item Weltweit standardisiert
		\item Direkt für Experimente verwendbar
		\item Alle Messgeräte kalibriert in SI
		\item Klare Trennung Länge/Zeit/Masse/Ladung
	\end{itemize}
	
	\textbf{Nachteile:}
	\begin{itemize}
		\item Viele Konstanten in Formeln ($4\pi\varepsilon_0$, $\hbar$, $c$)
		\item Physikalische Beziehungen verschleiert
		\item Dimensionen unübersichtlich
	\end{itemize}
	
	\subsection{Vor- und Nachteile: Heaviside-Lorentz mit $\alpha = 1$}
	
	\textbf{Vorteile:}
	\begin{itemize}
		\item Maximal vereinfachte Formeln
		\item Elektromagnetische Symmetrie sichtbar
		\item Theoretische Rechnungen einfacher
		\item QED-Feynman-Diagramme eleganter
	\end{itemize}
	
	\textbf{Nachteile:}
	\begin{itemize}
		\item Keine direkte Verbindung zu Experimenten
		\item Umrechnung zu SI aufwendig
		\item Ungewohnt für Praktiker
		\item Physikalische ''Größe'' von $e$ unklar
	\end{itemize}
	
	\subsection{Vor- und Nachteile: Natürliche Einheiten mit $\alpha \approx 1/137$}
	
	\textbf{Vorteile:}
	\begin{itemize}
		\item Vereinfachte Formeln ($\hbar = c = 1$)
		\item $\alpha$ behält physikalische Bedeutung
		\item Guter Kompromiss Theorie/Praxis
		\item Numerisch: $\alpha \ll 1$ → Störungstheorie
	\end{itemize}
	
	\textbf{Nachteile:}
	\begin{itemize}
		\item Immer noch Umrechnung zu SI nötig
		\item Faktor $4\pi$ bleibt in manchen Formeln
	\end{itemize}
	
	\textbf{Dies ist die bevorzugte Konvention in moderner Teilchenphysik!}
	
	% ============================================================================
	\section{Historische Entwicklung}
	
	\subsection{Gauss-Einheiten (cgs)}
	
	\textbf{19. Jahrhundert:} Gauss-System (Zentimeter-Gramm-Sekunde)
	
	\begin{equation}
		\alpha = \frac{e^2}{\hbar c}
	\end{equation}
	
	Kein $4\pi\varepsilon_0$, weil $\varepsilon_0 = 1$ per Definition in cgs!
	
	\subsection{SI-Einheiten (MKSA)}
	
	\textbf{20. Jahrhundert:} SI-System (Meter-Kilogramm-Sekunde-Ampere)
	
	\begin{equation}
		\alpha = \frac{e^2}{4\pi\varepsilon_0\hbar c}
	\end{equation}
	
	Das $4\pi\varepsilon_0$ erscheint, weil SI-Ampere über Kraft definiert ist.
	
	\subsection{Heaviside-Lorentz}
	
	\textbf{Theoretische Physik:} Heaviside-Lorentz vereinfacht Maxwell-Gleichungen
	
	\begin{equation}
		\nabla \times \vec{E} = -\frac{\partial \vec{B}}{\partial t}, \quad \nabla \times \vec{B} = \vec{j} + \frac{\partial \vec{E}}{\partial t}
	\end{equation}
	
	Symmetrisch! (In SI erscheinen $\mu_0$ und $\varepsilon_0$ unsymmetrisch)
	
	\subsection{Natürliche Einheiten}
	
	\textbf{Moderne Hochenergiephysik:} $\hbar = c = 1$, aber verschiedene Konventionen für $\alpha$
	
	% ============================================================================
	\section{Fine-Ungleichung vs. Feinstrukturkonstante}
	
	\subsection{Häufige Verwechslung}
	
	\textbf{Warnung:} Die \textit{Fine-Ungleichung} und die \textit{Feinstrukturkonstante} sind völlig verschiedene Konzepte!
	
	\subsection{Fine-Ungleichung}
	
	\textbf{Was es ist:}
	\begin{itemize}
		\item Eine Form der Bell-Ungleichung
		\item Test für lokale verborgene Variablen
		\item Quantenverschränkung vs. klassische Korrelationen
	\end{itemize}
	
	\textbf{Mathematisch:}
	\begin{equation}
		|C(\alpha, \beta) - C(\alpha, \beta')| + |C(\alpha', \beta) + C(\alpha', \beta')| \leq 2
	\end{equation}
	
	wobei $C$ Korrelationsfunktionen sind.
	
	\textbf{Physikalisch:} Zeigt Nichtlokalität der Quantenmechanik
	
	\subsection{Feinstrukturkonstante}
	
	\textbf{Was es ist:}
	\begin{itemize}
		\item Fundamentale Naturkonstante
		\item Stärke der elektromagnetischen Wechselwirkung
		\item Dimensionslos, $\alpha \approx 1/137$
	\end{itemize}
	
	\textbf{Mathematisch:}
	\begin{equation}
		\alpha = \frac{e^2}{4\pi\varepsilon_0\hbar c}
	\end{equation}
	
	\textbf{Physikalisch:} Bestimmt EM-Kopplungsstärke
	
	\subsection{Keine Verbindung!}
	
	Die Ähnlichkeit im Namen ist \textbf{reiner Zufall}. Die beiden Konzepte haben nichts miteinander zu tun!
	
	% ============================================================================
	\section{Zusammenfassung}
	
	\subsection{Kernaussagen}
	
	\begin{enumerate}
		\item $\alpha$ ist dimensionslos → numerischer Wert ist konventionsabhängig
		
		\item Man \textbf{kann} $\alpha = 1$ setzen durch Umdefinition der Ladungseinheit
		
		\item \textbf{T0-Theorie:} Setzt **alle** Konstanten = 1: $c = \hbar = \alpha = G = 1$
		
		\item Einziger freier Parameter in T0: $\xi = \frac{4}{3} \times 10^{-4}$
		
		\item \textbf{Keine} physikalischen Vorhersagen ändern sich!
		
		\item Nur Zahlenwerte in Formeln unterscheiden sich
		
		\item Beim Vergleich mit Experimenten: SI-Werte ($\alpha \approx 1/137$, $c = 3 \times 10^8$ m/s, etc.)
	\end{enumerate}
	
	\subsection{Für weitere Details siehe Dokument 011}
	
	\begin{itemize}
		\item T0-Herleitung von $\alpha$
		\item Charakteristische Energie $E_0$
		\item Geometrischer Parameter $\xi$
		\item Experimentelle Verifikation
		\item Detaillierte Dimensionsanalyse
		\item Historischer Kontext (Sommerfeld)
	\end{itemize}
	
	% ============================================================================
	\appendix
	
	\section{Umrechnungstabelle: SI ↔ Heaviside-Lorentz}
	
	\begin{table}[h]
		\centering
		\begin{tabular}{|l|c|c|}
			\hline
			\textbf{Größe} & \textbf{SI} & \textbf{HL ($\alpha = 1$)} \\
			\hline
			Elementarladung & $e = 1{,}602 \times 10^{-19}$ C & $e = 1$ \\
			Feinstrukturkonstante & $\alpha \approx 1/137$ & $\alpha = 1$ \\
			$4\pi\varepsilon_0$ & $1{,}11 \times 10^{-10}$ F/m & $1$ \\
			$\hbar$ & $1{,}055 \times 10^{-34}$ J$\cdot$s & $1$ \\
			$c$ & $3 \times 10^8$ m/s & $1$ \\
			\hline
		\end{tabular}
		\caption{Umrechnungstabelle SI zu HL}
	\end{table}
	
	\section{Beispielrechnung: Coulomb-Gesetz}
	
	\subsection{In SI-Einheiten}
	
	\begin{equation}
		F = \frac{1}{4\pi\varepsilon_0} \frac{e^2}{r^2}
	\end{equation}
	
	Numerisch für $r = 1$ Å = $10^{-10}$ m:
	\begin{align}
		F &= \frac{1}{4\pi \times 8{,}854 \times 10^{-12}} \frac{(1{,}602 \times 10^{-19})^2}{(10^{-10})^2} \\
		&\approx 2{,}3 \times 10^{-8} \text{ N}
	\end{align}
	
	\subsection{In HL-Einheiten ($\alpha = 1$)}
	
	\begin{equation}
		F = \frac{e^2}{r^2} = \frac{1}{r^2}
	\end{equation}
	
	Mit $r$ in natürlichen Einheiten: $r = 1$ Å $= 0{,}197 \times 10^6$ eV$^{-1}$
	
	\begin{equation}
		F = \frac{1}{(0{,}197 \times 10^6)^2} \approx 2{,}6 \times 10^{-14} \text{ eV}^2
	\end{equation}
	
	Umrechnung zu SI: $1 \text{ eV}^2 \approx 9 \times 10^5$ N
	
	\begin{equation}
		F \approx 2{,}3 \times 10^{-8} \text{ N}
	\end{equation}
	
	\textbf{Identisch!} Nur die Zwischenschritte sehen anders aus.
	
\end{document}