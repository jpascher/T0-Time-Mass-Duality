\documentclass[12pt,a4paper]{article}

% Standardized preamble - 085\_T0\_photonenchip-einführung\_De.pdf
\% Minimale T0 Standalone Preamble - A4 Format - 25 Zeilen
\RequirePackage{fontspec}
\RequirePackage{unicode-math}
\usepackage[ngerman]{babel}
\usepackage{microtype}
\setmainfont{Inter}
\setmonofont{JetBrains Mono}
\setmathfont{Libertinus Math}
\usepackage{amsmath,amsfonts,amsthm}
\usepackage{mathtools}
\usepackage{graphicx}
\usepackage{xcolor}
\definecolor{t0blue}{RGB}{0,102,204}
\definecolor{t0green}{RGB}{34,139,34}
\definecolor{t0red}{RGB}{204,0,0}
\usepackage{geometry}
\geometry{a4paper,margin=2.5cm}
\usepackage[most]{tcolorbox}
\newtcolorbox{keyresult}[1][]{colback=yellow!5,colframe=t0blue!80,fonttitle=\bfseries,title={#1},breakable}
\newtcolorbox{important}[1][]{colback=red!5,colframe=t0red!80,fonttitle=\bfseries,title={#1},breakable}
\newcommand{\Tfield}{\ensuremath{\mathcal{T}}}
\usepackage{hyperref}
\hypersetup{colorlinks=true,linkcolor=t0blue}


\title{\Huge\textbf{Einführung in photonische Quantenchips für Nachrichtentechniker}\\
	\large Analoge Realisierungen und Operationen für 6G-Signalverarbeitung}
\author{}
\date{}

\begin{document}

	
	\maketitle
	
	\begin{abstract}
		Photonische integrierte Schaltkreise (PICs) revolutionieren die Nachrichtentechnik: Von latenzarmen RF-Filtern für 6G-Netze bis zu parallelen AI-Operationen in Data-Centern. **Die 6G-Standardisierung beginnt 2025, wobei photonische Komponenten der Schlüssel zur Erschließung des Terahertz (THz)-Frequenzbereichs für extrem hohe Datenraten sind \cite{6g_roadmap}.** Diese Einführung basiert auf aktueller Literatur (2024–2025) und beleuchtet analoge Realisierungsprinzipien (z.\,B. Interferenz via MZI), bevorzugte Operationen (Matrix-Multiplikation, Signal-Filterung) und Relevanz für Echtzeit-Kommunikation. Praxisnah: Tabelle zu Techniken, Ausblick auf Hybride Systeme. Quellen: Reviews aus Nature, SPIE und ScienceDirect. **Aktuelle Forschungen (EPFL/Harvard) haben einen revolutionären optoelektronischen Chip vorgestellt, der THz- und optische Signale auf einem Prozessor verarbeitet \cite{thz_epfl}.**
	\end{abstract}
	
	\tableofcontents
	
	\section{Grundlagen: Photonische Chips in der Nachrichtentechnik}
	
	Photonische Quantenchips nutzen Lichtwellen für hochparallele, energieeffiziente Verarbeitung – essenziell für 6G (Bandbreiten $>\SI{100}{GHz}$, Latenz $<\SI{1}{ms}$). **Die Europäische Kommission hat den Start der 6G-Standardisierung für 2025 angekündigt, mit einem Fokus auf Souveränität und führender Technologieposition \cite{6g_roadmap}. Das Jahr 2025 wurde zudem von den Vereinten Nationen als das Internationale Jahr der Quantenwissenschaften (IYQ) ausgerufen, was die strategische Bedeutung der Photonik untermauert \cite{quantenjahr25}.** Im Gegensatz zu elektronischen CMOS-Chips (Wärme-Limits bei hohen Frequenzen) ermöglichen PICs analoge Signalverarbeitung durch optische Interferenz und Modulation, angelehnt an klassische analoge Optik (z.\,B. aus der RF-Technik der 1980er).
	\begin{important}
		Wichtiger Hinweis: Die Technik ist stark analog: Kontinuierliche Wellentransformationen (Phasenverschiebung, Diffraktion) dominieren, da Photonen intrinsisch parallel (Wellenlängen-Multiplexing) und latenzarm sind. Hybride Systeme (Photonik + Elektronik) ergänzen für Steuerung.
	\end{important}
	
	Aktuelle Trends (2025): Skalierbare Wafer (z.\,B. 6-Zoll-TFLN) für industrielle Einsätze in Data-Centern, mit $1000\times$-Speedup für AI-Workloads \cite{photonics_ai, qant_nps}.
	\section{Realisierung von Operationen: Analoge Prinzipien}
	
	Operationen werden primär durch optische Bauteile realisiert, die analoge Verarbeitung priorisieren. Kernkomponenten:
	
	\begin{itemize}
		\item \textbf{Mach-Zehnder-Interferometer (MZI)}: Für Phasenmodulation und lineare Transformationen; analoge Addition/Multiplikation via Interferenz.
		\item \textbf{Wellenleiter und Modulatoren}: Elektro-optische (z.\,B. LiNbO$_3$) oder thermische Steuerung für kontinuierliche Signale.
		\item \textbf{Monolithische Integration}: Co-Packaging auf Si- oder TFLN-Plattformen minimiert Verluste ($<\SI{1}{dB}$), ermöglicht dynamische Rekonfiguration.
	\end{itemize}
	Die Technik lehnt sich an analoge RF-Systeme an: Statt diskreter Bits kontinuierliche Wellenfelder für Echtzeit-Filterung (z.\,B. Demodulation in 6G) \cite{analog_optical}.
	\begin{formula}
		Beispiel: Lineare Transformation (Matrix-Vektor-Multiplikation) via MZI-Mesh: $y = M \cdot x$, wobei $M$ durch Phasen $\phi_i$ programmiert wird: $\phi_i = \arg(M_{ij})$.
	\end{formula}
	
	\section{Bevorzugte Operationen für photonische Bauteile}
	
	Photonische Chips eignen sich für lineare, frequenzabhängige und parallele Operationen, da analoge Kontinuität Energie spart ($\SI{}{\pico\joule}/\text{Bit}$) und Bandbreite maximiert. Basierend auf 2025-Reviews:
	
	\begin{table}[htbp]
		\resizebox{\textwidth}{!}{%
		\centering
		\begin{tabular}{l p{6cm} p{4cm}}
			\toprule
			\textbf{Operation} & \textbf{Realisierung (analog)} & \textbf{Relevanz für Nachrichtentechnik} \\
			\midrule
			Matrix-Multiplikation (GEMM) & MZI-Arrays für Interferenz-basierte Addition/Multiplikation & AI-Training in Edge-Netzen (z.\,B. Transformer für 6G-Routing) \cite{photonics_ai} \\
			RF-Signal-Filterung & Optische Diffraktion/FFT via Wellenleiter & Demodulation, BSS in 5G/6G (Bandbreite $>\SI{100}{GHz}$) \cite{rf_photonics} \\
			Recurrent-Processing & Programmierte photonische Circuits (PPCs) für sequentielle Transformationen & Echtzeit-Überwachung in Netzen (z.\,B. RNNs für Anomalie-Erkennung) \cite{recurrent_photonics} \\
			Differential-Operationen & Meta-Optik für Gradienten (z.\,B. Edge-Detection) & Bild-/Signal-Enhancement in Optischen Netzen \cite{differential_optical} \\
			Parallele Optimierung & Korrelation via kohärente PICs & Gradient-Descent für Routing-Optimierung \cite{optical_advantages} \\
			\bottomrule
		\end{tabular}}
		\caption{Bevorzugte Operationen auf photonischen Chips – Fokus auf analoge Techniken}
		\label{tab:operations}
	\end{table}
	
	Nicht bevorzugt: Nicht-lineare Logik (z.\,B. AND/OR), da Photonen linear sind; hier Hybride nötig.
	
	\section{Literaturübersicht: Aktuelle Entwicklungen (2024–2025)}
	
	Basierend auf neuesten Reviews (offen zugänglich) und aktuellen Projekten:
	
	\begin{itemize}
		\item \textbf{Analog optical computing: principles, progress, and prospects (2025)}: Überblick über analoge PICs; Fortschritte in rekonfigurierbaren Designs für Echtzeit-Signale \cite{analog_optical}.
		\item \textbf{Integrierte Terahertz-Kommunikation:} Ein revolutionärer optoelektronischer Prozessor (EPFL/Harvard, 2025) integriert die Verarbeitung von **Terahertz-Wellen** und optischen Signalen auf einem Chip. Dieser Durchbruch ist entscheidend für 6G, da er Hochleistung ohne nennenswerten Energieverlust ermöglicht und mit bestehenden photonischen Technologien kompatibel ist \cite{thz_epfl}.
		\item \textbf{Integrierte Photonik für 6G-Forschung:} Projekte wie **6G-ADLANTIK** und **6G-RIC** (Fraunhofer HHI) entwickeln photonisch-elektronische Integrationskomponenten, um den THz-Frequenzbereich für 6G zu erschließen und die Resilienz von Netzwerken zu verbessern (SUSTAINET) \cite{hhi_6g}.
		\item \textbf{Integrated photonic recurrent processors (2025)}: Recurrent-Operationen via PPCs; Anwendungen in sequentieller Verarbeitung (z.\,B. Netzwerk-Überwachung) \cite{recurrent_photonics}.
		\item \textbf{Photonics for sustainable AI (2025)}: GEMM als Kern für AI; photonische Vorteile für energiearme 6G-Inferenz \cite{photonics_ai}.
		\item \textbf{All-optical analog differential operation... (2025)}: Meta-Optik für Differential-Computing; ideal für Signal-Enhancement \cite{differential_optical}.
		\item \textbf{Harnessing optical advantages in computing: a review (2024)}: Parallele Vorteile; Fokus auf FFT und Korrelation für RF \cite{optical_advantages}.
	\end{itemize}
	
	Diese Quellen betonen den Shift zu analogen Hybriden für 6G: Von Prototypen zu skalierbaren Wafern.
	
\end{document}

