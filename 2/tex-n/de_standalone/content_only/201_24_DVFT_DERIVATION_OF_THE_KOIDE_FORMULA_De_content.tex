\begin{enumerate}
  \item Introduction
\end{enumerate}
This document presents a mathematically consistent derivation of the Koide mass formula from the
T0-Vakuum microphysics of DVFT (Dynamic T0-Vakuum field Curvature Theory).
The Koide relation for the charged leptons is:
Q = (m_e + m_$\mu$ + m_τ) / ( ($\sqrt$m_e + $\sqrt$m_$\mu$ + $\sqrt$m_τ)^2 ),
experimentally:
Q = 2/3 $\pm$ 10⁻⁵.
The Standard Model does not explain this.
GUTs do not explain this.
String theory does not explain this.
DVFT explains Koide naturally because particle masses arise from discrete T0-Vakuum phase--amplitude
eigenmodes of the fundamental field:
Φ = ρ e^{i$\theta$},
with masses determined by phase displacement from equilibrium T0-Vakuum structure.
\begin{enumerate}
  \item DVFT Mass Formula for a Localized Particle
\end{enumerate}
In DVFT, the mass of a stable excitation arises from local curvature of the T0-Vakuum potential U(ρ) and
from the phase shift $\theta$ of the oscillation mode:
m_i ∝ $\sqrt$(U''(ρ_i)) · | e^{i$\theta$_i} − 1 |.
Using:
|e^{i$\theta$} − 1|² = 2(1 − cos$\theta$),
the mass becomes:
m_i = K · (1 − cos$\theta$_i),
where K is a T0-Vakuum stiffness constant.
Thus charged lepton masses correspond to specific phase eigenmodes $\theta$_i.
\begin{enumerate}
  \item Phase Quantization Condition That Produces Koide
\end{enumerate}
Assume the T0-Vakuum supports three stable, equally spaced phase eigenmodes:
$\theta$_e = $\theta$₀,
$\theta$_$\mu$ = $\theta$₀ + 2π/3,
$\theta$_τ = $\theta$₀ + 4π/3.
International Journal for Multidisciplinary Research (IJFMR)
E-ISSN: 2582-2160 $\bullet$ Website: www.ijfmr.com $\bullet$ Email: editor@ijfmr.com
IJFMR250664112 Volume 7, Issue 6, November-December 2025 55
Then:
m_e = K(1 − cos$\theta$₀)
m_$\mu$ = K(1 − cos($\theta$₀ + 2π/3))
m_τ = K(1 − cos($\theta$₀ + 4π/3)).
This three-mode 120° phase structure is the simplest nonlinear T0-Vakuum eigenmode solution.
Using the trigonometric identities for 120° shifts, we find the resulting ratios of square roots automatically
satisfy the Koide condition.
Thus Koide is a geometric consequence of DVFT phase quantization.
\begin{enumerate}
  \item Geometric Interpretation of Koide
\end{enumerate}
Define:
a = $\sqrt$m_e, b = $\sqrt$m_$\mu$, c = $\sqrt$m_τ.
Koide’s formula is equivalent to:
a² + b² + c² = 2(ab + bc + ca).
This occurs if the vectors (a, b, c) lie 120° apart on a circle.
DVFT predicts exactly this geometry because T0-Vakuum oscillation modes separated by 120° in phase
naturally yield mass eigenvalues whose square roots form this structure.
Thus Koide is a direct geometric consequence of T0-Vakuum phase symmetry.
\begin{enumerate}
  \item Why DVFT Predicts Exactly Three Leptons
\end{enumerate}
The T0-Vakuum potential:
U(ρ) = κ(ρ − ρ₀)² + λ(ρ − ρ₀)⁴ + …
supports a limited number of stable localized minima.
Nonlinear dynamic media naturally produce:
\begin{itemize}
  \item three stable modes,
  \item 120° phase spacing,
  \item triplet standing waves.
\end{itemize}
Thus DVFT predicts:
\begin{itemize}
  \item three charged leptons,
  \item with masses tied to phase geometry,
  \item not arbitrary Yukawa couplings.
\end{itemize}
The Koide relation therefore reflects T0-Vakuum structure, not coincidence.
\begin{enumerate}
  \item Full DVFT Derivation Summary
\end{enumerate}
DVFT → m_i ∝ (1 − cos$\theta$_i)
Three equally spaced phase eigenmodes → $\theta$_i = $\theta$₀ + 2πi/3
This produces: ($\sqrt$m_e, $\sqrt$m_$\mu$, $\sqrt$m_τ) lying at 120° in mass space.
This enforces the identity:
Q = (m_e + m_$\mu$ + m_τ) / ( ($\sqrt$m_e + $\sqrt$m_$\mu$ + $\sqrt$m_τ)^2 ) = 2/3.
Thus Koide arises from:
\begin{itemize}
  \item phase structure of T0-Vakuum,
  \item amplitude--phase coupling in Φ = ρ e^{i$\theta$},
  \item geometric symmetry of T0-Vakuum eigenmodes.
\end{itemize}
\begin{enumerate}
  \item Implications for Particle Physics
\end{enumerate}
If DVFT explains Koide, then:
\begin{enumerate}
  \item Mass is not from arbitrary Yukawa parameters but from T0-Vakuum phase structure.
\end{enumerate}
International Journal for Multidisciplinary Research (IJFMR)
E-ISSN: 2582-2160 $\bullet$ Website: www.ijfmr.com $\bullet$ Email: editor@ijfmr.com
IJFMR250664112 Volume 7, Issue 6, November-December 2025 56
\begin{enumerate}
  \item Three generations = three stable phase eigenmodes.
  \item DVFT predicts:
\end{enumerate}
− Mass hierarchies,
− Lepton ratios,
− Neutrino mixi\footnote{Siehe auch T0-Dokument: \texttt{009_T0_xi_ursprung_De.pdf}}ng structure (with phase offsets),
− Quark mass relations (with additional interactions).
− Koide becomes evidence of underlying T0-Vakuum-phase geometry.
DVFT therefore provides a candidate unification of mass generation, explaining one of the most precise
numerical relations in physics.

\section{Referenzen zu T0-Dokumenten}

Dieses Kapitel steht in Zusammenhang mit folgenden T0-Dokumenten im Repository \texttt{2/pdf/}:

\begin{itemize}
  \item \texttt{002\_T0\_Grundlagen\_De.pdf} -- T0 Zeit-Masse-Dualität Grundlagen
  \item \texttt{004\_T0\_Energie\_De.pdf} -- T0 Energiefeld-Theorie
  \item \texttt{009\_T0\_xi\_ursprung\_De.pdf} -- Ursprung des geometrischen Parameters $\xi$
  \item \texttt{201\_DVFT-alles\_De.pdf} -- Vollständiges DVFT-Dokument (Rahmenwerk)
\end{itemize}