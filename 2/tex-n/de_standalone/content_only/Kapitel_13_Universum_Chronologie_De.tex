CHAPTER 13: CHRONOLOGY OF THE UNIVERSE CREATION
1. Introduction
International Journal for Multidisciplinary Research (IJFMR)
E-ISSN: 2582-2160 \textbullet{} Website: www.ijfmr.com \textbullet{} Email: editor@ijfmr.com
IJFMR250664112 Volume 7, Issue 6, November-December 2025 31
The origin of the universe is the deepest question in physics. Standard cosmology begins with the Big
Bang but does not explain why the universe started in a low-entropy, coherent state. Quantum Field Theory
assumes vacuum structure but does not explain why the vacuum exists or why fields take the values they
do. General Relativity describes geometry but cannot describe what spacetime physically is.
Dynamic Vacuum Field Theory (DVFT) provides a coherent physical ontology explaining what the
universe was before the Big Bang, why it began in a perfectly coherent state, and how vacuum amplitude ($\\rho_0 = 1/\\xi^2$ from T0),
mass, forces, and time emerged. This chapter presents this explanation step by step.
2. DVFT Foundations: Amplitude ρ and Phase θ
DVFT states that the vacuum is a real physical medium with two intrinsic degrees of freedom:
\textbullet{} ρ(x,t) — vacuum amplitude ($\\rho_0 = 1/\\xi^2$ from T0) (controls inertia, curvature, mass)
\textbullet{} θ(x,t) — vacuum phase (controls light propagation, coherence, quantum behavior)
The relationship between amplitude and phase defines the universe’s dynamics. Time emerges from phase
evolution, and space–curvature emerges from amplitude gradients.
3. The Only Possible Initial State: Pure Phase Vacuum
In the absolute beginning, the vacuum had no structure. Therefore, it could not possess:
\textbullet{} inertia,
\textbullet{} curvature,
\textbullet{} mass,
\textbullet{} energy density,
\textbullet{} spacetime geometry,
\textbullet{} particles,
\textbullet{} entropy.
All of these require nonzero amplitude ρ.
Thus, the only physically possible initial condition for the universe was:
ρ = 0,
θ = constant.
This pure-phase vacuum is perfectly coherent because no gradients, interactions, or decoherence can exist
without amplitude. It is a symmetry-dominated, structureless state—a true physical ‘void.’
4. Why the Initial Vacuum Must Have Been Perfectly Coherent
A pure-phase vacuum cannot sustain:
\textbullet{} waves,
\textbullet{} forces,
\textbullet{} gradients,
\textbullet{} decoherence,
\textbullet{} entropy.
With ρ = 0, vacuum stiffness (K_0) and vacuum inertial density (ρ_0) are also zero:
K_0 = Bρ² → 0,
ρ_0 = Aρ² → 0.
This means:
\textbullet{} no wave equations exist,
\textbullet{} no propagation is possible,
\textbullet{} time cannot flow,
\textbullet{} no physical process can occur.
International Journal for Multidisciplinary Research (IJFMR)
E-ISSN: 2582-2160 \textbullet{} Website: www.ijfmr.com \textbullet{} Email: editor@ijfmr.com
IJFMR250664112 Volume 7, Issue 6, November-December 2025 32
A pure-phase vacuum is therefore forced into perfect coherence. It is not a choice—it is the only
mathematically and physically consistent state that can exist without amplitude.
5. What Triggered the Emergence of Vacuum Amplitude ρ?
DVFT proposes that amplitude emerged because the pure-phase vacuum became unstable. This instability
could arise from any or all of the following mechanisms:
Mechanism A — Phase-Fluctuation Instability
If the initial vacuum phase experienced even an infinitesimal disturbance (δθ ≠ 0), the vacuum would be
unable to propagate or absorb that disturbance unless amplitude ρ emerged. Thus, quantum fluctuations
of θ force the birth of ρ.
Mechanism B — Vacuum Potential Instability
If the vacuum Lagrangian contains a potential:
U(ρ) = λ(ρ² − ρ★²)²,
then ρ = 0 is unstable and spontaneously rolls to ρ = ρ★. This resembles the Higgs mechanism but now
arises from vacuum necessity, not arbitrary symmetry breaking.
Mechanism C — Requirement for Time Evolution
Time in DVFT is vacuum phase evolution. But without amplitude, c² = K_0/ρ_0 = undefined. Therefore, in
order for time to exist, the vacuum must generate amplitude so that phase can propagate.
Thus, amplitude appears because phase evolution requires a medium with stiffness and inertia.
6. Time Begins: Birth of c = √(K_0/ρ_0)
Once amplitude ρ emerged, the vacuum acquired:
\textbullet{} inertia (ρ_0 = Aρ²),
\textbullet{} stiffness (K_0 = Bρ²),
\textbullet{} a well-defined wave speed c = √(K_0/ρ_0).
This enabled phase oscillations to propagate, marking the birth of time:
dτ ∝ dθ.
The universe went from static pure phase to dynamic phase evolution—a physical event more fundamental
than the Big Bang.
7. Curvature and Gravity Emerge
As amplitude ρ varied spatially:
\textbullet{} regions with larger ρ acquired larger inertial density,
\textbullet{} gradients in ρ generated curvature,
\textbullet{} curvature created gravitational effects.
Thus, gravity is born not from spacetime geometry but from amplitude variations in the vacuum.
8. Particle Formation and Matter Genesis
Once time existed and amplitude stabilized at ρ★, nonlinearities in dynamics allowed localized phase–
amplitude knots to form:
\textbullet{} stable solitons,
\textbullet{} topological defects,
\textbullet{} amplitude–phase traps.
These knots became particles:
\textbullet{} photons = pure phase,
\textbullet{} fermions = amplitude + phase,
International Journal for Multidisciplinary Research (IJFMR)
E-ISSN: 2582-2160 \textbullet{} Website: www.ijfmr.com \textbullet{} Email: editor@ijfmr.com
IJFMR250664112 Volume 7, Issue 6, November-December 2025 33
\textbullet{} massive bosons = amplitude-modulated phase.
Thus, matter emerges naturally from vacuum structure.
9. Why the Universe Started in a Low-Entropy State
In DVFT, entropy corresponds to vacuum phase disorder. A pure-phase vacuum has:
\textbullet{} no gradients,
\textbullet{} no decoherence,
\textbullet{} no thermalization,
\textbullet{} no scattering,
\textbullet{} no entropy.
Therefore, the universe did not "begin" in a low-entropy state—it began in the only possible state: perfect
coherence.
Entropy increases only after amplitude appears and interactions begin.
10. Summary: The DVFT Origin of the Universe
The DVFT offers a complete physical explanation of the universe's beginning:
\textbullet{} The universe began as pure phase with ρ = 0 and θ = constant.
\textbullet{} Perfect coherence was mandatory because no amplitude meant no dynamics.
\textbullet{} Instability triggered amplitude emergence.
\textbullet{} Amplitude enabled time (phase propagation), mass, gravity, and structure.
\textbullet{} Entropy and decoherence arose only after amplitude existed.
\textbullet{} Matter formed from vacuum phase–amplitude knots.
This presents the clearest physical ontology for why the universe started in a perfectly coherent state and
how the structured universe emerged from the most minimal possible beginning.


\section*{T0 Theory Integration}
This chapter integrates DVFT concepts with T0 Time-Mass Duality Theory, where the fundamental relation $T(x,t) \cdot m(x,t) = 1$ governs all vacuum field dynamics. The vacuum amplitude $\rho$ is directly related to local time $T$ through $\rho \propto 1/T$.
