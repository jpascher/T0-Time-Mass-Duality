\begin{enumerate}
  \item Introduction
\end{enumerate}
This document explains how Fundamental Fractal-Geometric Field Theory(FFGFT) derives the Planck time, length, and
mass, as well as other 'universal constants', from the fundamental T0-Vakuum parameters:
\begin{itemize}
  \item B -- T0-Vakuum phase stiffness
  \item ?? -- inertial T0-Vakuum density
  \item K? -- amplitude stiffness of T0-Vakuum
  \item ?? -- matter--T0-Vakuum coupling constant
  \item ? -- emerging from topological phase quantization
  \item $\theta$-winding scale -- phase gradient associated with unit charge
\end{itemize}
International Journal for Multidisciplinary Research (IJFMR)
E-ISSN: 2582-2160 $\bullet$ Website: www.ijfmr.com $\bullet$ Email: editor@ijfmr.com
IJFMR250664112 Volume 7, Issue 6, November-December 2025 94
FFGFT shows that Planck units are *not fundamental constants* but emergent mechanical properties of
the T0-Vakuum field \Phi = ? e^{i$\theta$}.
\begin{enumerate}
  \item FFGFT T0-Vakuum Parameters
\end{enumerate}
The key numerical T0-Vakuum parameters are:
\begin{itemize}
  \item Phase stiffness: B $\approx$ 8.7 $\times$ 10???
  \item Inertial T0-Vakuum density: ?? $\approx$ 6 $\times$ 10??? kg/m?
  \item Amplitude stiffness: K? $\approx$ 5.4 $\times$ 10??? J/m?
  \item Phase gradient for one charge: |$\partial$$\theta$/$\partial$x|? $\approx$ 1.63 $\times$ 10?? m??
  \item Speed of light (derived): c = $\sqrt$(K? / ??)
  \item Newton?s G (derived): G = ?? / (4? K?)
  \item Fine-structure constant (derived): ? = (B / ? c)($\partial$$\theta$/$\partial$x)?
\end{itemize}
These constants collectively define the mechanical, gravitational, and quantum architecture of the T0-Vakuum.
\begin{enumerate}
  \item FFGFT Substitutes into Planck Units
\end{enumerate}
Textbook definitions of Planck units are:
\begin{itemize}
  \item t_P = $\sqrt$(? G / c?)
  \item ?_P = $\sqrt$(? G / c?)
  \item m_P = $\sqrt$(? c / G)
\end{itemize}
But in FFGFT, none of ?, c, or G are fundamental:
\begin{itemize}
  \item c = $\sqrt$(K? / ??)
  \item G = ?? / (4? K?)
  \item ? arises from $\theta$-winding quantization
\end{itemize}
Substituting these relations gives the Planck units as explicit composites of FFGFT T0-Vakuum parameters.
\begin{enumerate}
  \item Planck Time from FFGFT
\end{enumerate}
Starting with:
t_P = $\sqrt$(? G / c?)
Insert:
c = $\sqrt$(K?/??)
G = ?? / (4? K?)
Compute:
t_P = $\sqrt${ (? ?? / (4? K?)) / (K?/??)^{5/2} }
Simplify:
t_P = $\sqrt${ ? ?? ??^{5/2} / (4? K?^{7/2}) }.
This is the FFGFT expression for Planck time.
Interpretation:
Planck time is the minimum time scale at which T0-Vakuum amplitude curvature can sustain a stable
oscillation.
It is not a fundamental limit of nature, but a material property of the T0-Vakuum.
\begin{enumerate}
  \item Planck Length from FFGFT
\end{enumerate}
Textbook definition:
?_P = $\sqrt$(? G / c?)
Substitute:
G = ?? / (4? K?)
c? = (K?/??)^{3/2}
International Journal for Multidisciplinary Research (IJFMR)
E-ISSN: 2582-2160 $\bullet$ Website: www.ijfmr.com $\bullet$ Email: editor@ijfmr.com
IJFMR250664112 Volume 7, Issue 6, November-December 2025 95
Result:
?_P = $\sqrt${ ? ?? ??^{3/2} / (4? K?^{5/2}) }.
Interpretation:
Planck length is the smallest stable spatial scale of T0-Vakuum amplitude curvature --- the 'acoustic
wavelength' of the T0-Vakuum medium.
\begin{enumerate}
  \item Planck Mass from FFGFT
\end{enumerate}
Textbook definition:
m_P = $\sqrt$(? c / G)
Insert:
c = $\sqrt$(K?/??)
G = ?? / (4? K?)
Compute:
m_P = $\sqrt${ (? $\sqrt$(K?/??)) (4? K?)/?? }
Simplify:
m_P = $\sqrt${ 4? ? K?^{3/2} / (?? ??^{1/2}) }.
Interpretation:
Planck mass is the amplitude deformation that matches one quantum of phase curvature.
\begin{enumerate}
  \item Physical Meaning: Planck Units Are Emergent T0-Vakuum Properties
\end{enumerate}
In FFGFT:
\begin{itemize}
  \item Planck time $\rightarrow$ minimum oscillation time of T0-Vakuum amplitude
  \item Planck length $\rightarrow$ minimum spatial curvature scale of T0-Vakuum amplitude
  \item Planck mass $\rightarrow$ amplitude curvature equivalent to one phase quantum
\end{itemize}
This new interpretation replaces the vague 'quantum gravity scale' with clear mechanical meaning.
Planck units describe **acoustic-like resonance properties** of the T0-Vakuum medium.
\begin{enumerate}
  \item Other Constants Derived from FFGFT
\end{enumerate}
FFGFT reduces many universal constants to derivatives of T0-Vakuum parameters:
\begin{enumerate}
  \item Speed of light:
\end{enumerate}
c = $\sqrt$(K?/??)
\begin{enumerate}
  \item Gravitational constant:
\end{enumerate}
G = ?? / (4? K?)
\begin{enumerate}
  \item Fine-structure constant:
\end{enumerate}
? = (B / ? c)($\partial$$\theta$/$\partial$x)?
\begin{enumerate}
  \item Electron charge:
\end{enumerate}
e? = 4? ?? ? c ? $\rightarrow$ e arises from B and phase topology
\begin{enumerate}
  \item Dark-energy density:
\end{enumerate}
?_\Lambda c? $\approx$ K?
\begin{enumerate}
  \item Deep-field acceleration scale (MOND-like):
\end{enumerate}
a? $\approx$ c? / L_* (L_* = cosmic coherence length)
\begin{enumerate}
  \item Neutrino mass scale:
\end{enumerate}
m_? ? B (phase oscillation over long coherence lengths)
\begin{enumerate}
  \item Quantum coherence length of T0-Vakuum:
\end{enumerate}
L_coh $\approx$ $\sqrt$(? / B)
Every one of these constants is derived --- none are fundamental.
International Journal for Multidisciplinary Research (IJFMR)
E-ISSN: 2582-2160 $\bullet$ Website: www.ijfmr.com $\bullet$ Email: editor@ijfmr.com
IJFMR250664112 Volume 7, Issue 6, November-December 2025 96
\begin{enumerate}
  \item Consequences for Physics
\end{enumerate}
Because all universal constants are derived from the T0-Vakuum parameters, FFGFT provides:
\begin{itemize}
  \item A complete unification of gravity, quantum mechanics, and electromagnetism
  \item A physical explanation for Planck units
  \item A mechanism for dark energy
  \item The origin of ?, e, c, G, ?
  \item Predictive power across scales from the proton to cosmology
  \item A new foundation for quantum technologies (phase-based computing)
\end{itemize}
FFGFT reinterprets the universe as a material medium with definable mechanical constants B, K?, ??, from
which all physical scales emerge.
Conclusion
FFGFT transforms the Planck constants from unexplained numerology into physically meaningful
emergent properties of the T0-Vakuum?s amplitude--phase structure. This resolves long-standing conceptual
gaps between quantum mechanics, relativity, and cosmology, and positions FFGFT as a unified framework
where the numerical structure of the universe is derived from the underlying nature of the T0-Vakuum.

\section{Referenzen zu T0-Dokumenten}

Dieses Kapitel steht in Zusammenhang mit folgenden T0-Dokumenten im Repository \texttt{2/pdf/}:

\begin{itemize}
  \item \texttt{002\_T0\_Grundlagen\_De.pdf} -- T0 Zeit-Masse-Dualit?t Grundlagen
  \item \texttt{004\_T0\_Energie\_De.pdf} -- T0 Energiefeld-Theorie
  \item \texttt{009\_T0\_xi\_ursprung\_De.pdf} -- Ursprung des geometrischen Parameters $\xi$
  \item \texttt{201\_FFGFT-alles\_De.pdf} -- Vollst?ndiges FFGFT-Dokument (Rahmenwerk)
\end{itemize}