\begin{enumerate}
  \item Introduction
\end{enumerate}
This chapter derives Einstein’s mass--energy relation E = mc²\footnote{Siehe auch T0-Dokument: \texttt{077_E-mc2_De.pdf}} purely from the Dynamic T0-Vakuum Field
Theory (DVFT), without using Einstein’s field equations. The DVFT provides physical explanation of
conversion of mass into energy. The mass is nothing but the knotted compressed T0-Vakuum field. When
mass converts into energy, the compressed T0-Vakuum energy gets released in the form of light.
DVFT treats spacetime as a physical quantum medium described by the phase field $\theta$(x,t). Particles appear
as localized excitations of this T0-Vakuum medium, and their mass is interpreted as stored T0-Vakuum energy.
From this viewpoint, E = mc² emerges naturally from the dynamics of the T0-Vakuum field.
\begin{enumerate}
  \item The DVFT T0-Vakuum Field
\end{enumerate}
The T0-Vakuum is represented by the complex order parameter:
Φ(x) = ρ(x) e^{i$\theta$(x)},
with ρ the T0-Vakuum density and $\theta$ the T0-Vakuum phase.
In flat spacetime, the DVFT kinetic invariant is:
X = (1/c²)($\partial$_t$\theta$)² − ($\nabla$$\theta$)².
A simplified DVFT Lagrangian\footnote{Siehe auch T0-Dokument: \texttt{027_T0_lagrndian_De.pdf}} for deriving particle-like excitations is:
𝓛_$\theta$ = −Λ_v + (ρ₀/2)X − (η/(3a₀²)) X^{3/2}.
To quantize and analyze particle excitations, we expand the T0-Vakuum phase field around a background
value:
$\theta$(x) = $\theta$₀ + φ(x).
International Journal for Multidisciplinary Research (IJFMR)
E-ISSN: 2582-2160 $\bullet$ Website: www.ijfmr.com $\bullet$ Email: editor@ijfmr.com
IJFMR250664112 Volume 7, Issue 6, November-December 2025 15
\begin{enumerate}
  \item Quadratic Expansion of the DVFT Action
\end{enumerate}
For small φ(x), the leading-order dynamics become:
𝓛_free = (ρ₀/2)[ (1/c²)($\partial$_tφ)² − ($\nabla$φ)² ] − (1/2) m_$\theta$² φ².
By defining a canonically normalized field:
φ_c = $\sqrt$ρ₀ φ,
the free field Lagrangian becomes:
𝓛_free = (1/2)[ (1/c²)($\partial$_tφ_c)² − ($\nabla$φ_c)² ] − (1/2) m_$\theta$² φ_c².
This is the standard Klein--Gordon Lagrangian for a relativistic quantum excitation of the T0-Vakuum.
\begin{enumerate}
  \item Dispersion Relation of DVFT T0-Vakuum Excitations
\end{enumerate}
The equation of motion is the Klein--Gordon equation:
(1/c²) $\partial$_t² φ_c − $\nabla$² φ_c + m_$\theta$² φ_c = 0.
Using plane-wave solutions:
φ_c = A e^{i(k·x − ωt)},
we obtain the dispersion relation:
ω² = c²(k² + m_$\theta$²).
Define the particle energy and momentum:
E = ħω,
p = ħk.
Then the dispersion relation becomes:
E² = p²c² + (ħ m_$\theta$ c)².
Identify the particle mass as:
m = ħ m_$\theta$ / c.
Thus, the DVFT T0-Vakuum excitations obey:
E² = p²c² + m² c⁴.
In the rest frame of the T0-Vakuum excitation (p = 0), the dispersion relation reduces to:
E² = m² c⁴.
Taking the positive-energy branch:
E = mc².
This is derived entirely from the DVFT T0-Vakuum field Lagrangian and its excitations---no Einstein field
equations or GR postulates were used.
Thus, in DVFT:
\begin{itemize}
  \item Mass m is the parameter determining the intrinsic oscillation frequency of the T0-Vakuum phase field
\end{itemize}
at zero momentum.
\begin{itemize}
  \item E = mc² states that rest energy equals the stored T0-Vakuum energy in the localized excitation (the
\end{itemize}
particle).
\begin{enumerate}
  \item T0-Vakuum Energy Interpretation of Mass
\end{enumerate}
From the DVFT Hamiltonian density:
𝓗 = (1/2c²)($\partial$_tφ_c)² + (1/2)($\nabla$φ_c)² + (1/2) m_$\theta$² φ_c²,
the total energy of a localized excitation is:
E = $\int$ d³x 𝓗.
For a rest-frame solution, this energy evaluates to:
E = mc².
Thus, mass is the T0-Vakuum energy stored in a stable $\theta$-excitation.
International Journal for Multidisciplinary Research (IJFMR)
E-ISSN: 2582-2160 $\bullet$ Website: www.ijfmr.com $\bullet$ Email: editor@ijfmr.com
IJFMR250664112 Volume 7, Issue 6, November-December 2025 16
No separate ``mass substance`` exi\footnote{Siehe auch T0-Dokument: \texttt{009_T0_xi_ursprung_De.pdf}}sts: mass is simply bound T0-Vakuum energy.
\begin{enumerate}
  \item Physical Meaning of E = mc² in DVFT
\end{enumerate}
DVFT gives a more satisfying interpretation of E = mc²:
\begin{enumerate}
  \item A particle is a localized distortion of the T0-Vakuum phase field.
  \item Its mass m measures the resistance of the T0-Vakuum to changing this localized pattern.
  \item Its rest energy mc² is the total T0-Vakuum energy stored in that pattern.
  \item Nuclear reactions (fission, fusion) release energy not because ``mass turns into energy,`` but because
\end{enumerate}
T0-Vakuum configurations reorganize.
\begin{enumerate}
  \item The difference in T0-Vakuum energy between initial and final configurations gives ΔE = Δ(mc²).
\end{enumerate}
Conclusion
E = mc² emerges naturally from DVFT as the rest-energy relation for quantized T0-Vakuum-phase excitations.
The result is fully derivable from the DVFT Lagrangian using:
\begin{itemize}
  \item Expansion around the T0-Vakuum,
  \item Canonical normalization,
  \item Klein--Gordon dynamics,
  \item Energy--momentum identification.
\end{itemize}
Mass--energy equivalence arises fundamentally from the microstructure of the T0-Vakuum in DVFT.

\section{Referenzen zu T0-Dokumenten}

Dieses Kapitel steht in Zusammenhang mit folgenden T0-Dokumenten im Repository \texttt{2/pdf/}:

\begin{itemize}
  \item \texttt{002\_T0\_Grundlagen\_De.pdf} -- T0 Zeit-Masse-Dualität Grundlagen
  \item \texttt{004\_T0\_Energie\_De.pdf} -- T0 Energiefeld-Theorie
  \item \texttt{009\_T0\_xi\_ursprung\_De.pdf} -- Ursprung des geometrischen Parameters $\xi$
  \item \texttt{201\_DVFT-alles\_De.pdf} -- Vollständiges DVFT-Dokument (Rahmenwerk)
\end{itemize}