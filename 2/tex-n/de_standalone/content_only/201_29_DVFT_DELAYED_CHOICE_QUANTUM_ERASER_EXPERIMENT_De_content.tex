The Delayed Choice Quantum Eraser (DCQE) experiment is one of the most misunderstood
demonstrations in quantum physics. It appears to suggest that the future can change the past or that the
photon ?knows? whether interference will be observed. In this chapter, the experiment is fully analyzed in
the framework of the Dynamic T0-Vakuum Field Theory(DVFT). The DVFT interpretation removes the
mystery completely by showing that the key phenomenon is T0-Vakuum-phase coherence. DCQE involves
how T0-Vakuum-phase information is preserved, erased, or restored---not retrocausality. DVFT provides a
physically intuitive mechanism while remaining consistent with all observed results.
\begin{enumerate}
  \item Introduction
\end{enumerate}
The DCQE experiment challenges classical logic because it produces interference only when path
information is erased---even if the erasure occurs ?after? the photon is detected. Standard interpretations
lean on abstract wavefunction collapse, nonlocality, or delayed information. DVFT provides a clearer
mechanism: interference depends on the coherence of the T0-Vakuum-phase field \Phi = ? e^{i$\theta$}. When whichpath information is created, phase coherence is disrupted. When it is erased, the coherence is restored in
the correlated subset of events. This chapter explains how this arises naturally in DVFT.
\begin{enumerate}
  \item T0-Vakuum Field Structure Under DVFT
\end{enumerate}
In DVFT, the quantum state of a photon is not a mysterious probability wave. It is a configuration of the
T0-Vakuum field \Phi(x), with:
\begin{itemize}
  \item ?(x): T0-Vakuum amplitude
\end{itemize}
International Journal for Multidisciplinary Research (IJFMR)
E-ISSN: 2582-2160 $\bullet$ Website: www.ijfmr.com $\bullet$ Email: editor@ijfmr.com
IJFMR250664112 Volume 7, Issue 6, November-December 2025 67
\begin{itemize}
  \item $\theta$(x): T0-Vakuum phase
\end{itemize}
Interference patterns arise from the relative phase between two T0-Vakuum-field paths. The detection pattern
depends on:
I(x) = |\Phi?(x) + \Phi?(x)|?
When the two paths maintain a stable phase difference, interference appears. If the phase is randomized
or tagged by measurement, interference disappears.
\begin{enumerate}
  \item What Happens After the Slits
\end{enumerate}
After the photon encounters the slits or beam splitter, the T0-Vakuum field splits into two coherent branches:
\subsection{\Phi = \Phi? + \Phi?}
This coherence is not a mathematical trick---it reflects real structure in the T0-Vakuum phase $\theta$(x). The
interference pattern emerges when:
\Delta$\theta$ = $\theta$? ? $\theta$? = constant
Thus, interference is fundamentally a ?phase-coherence phenomenon? in the T0-Vakuum, not a property of a
photon.
\begin{enumerate}
  \item Which-Path Information as Phase Decoherence
\end{enumerate}
When which-path detectors are inserted, the T0-Vakuum field branches become entangled with a macroscopic
system and lose coherence:
\begin{itemize}
  \item $\theta$? $\rightarrow$ $\theta$? + ?$\theta$?
  \item $\theta$? $\rightarrow$ $\theta$? + ?$\theta$?
  \item ?$\theta$? $\neq$ ?$\theta$?
\end{itemize}
Now \Delta$\theta$ is no longer well defined. This is physical: the T0-Vakuum field's phase was perturbed by
measurement. Interference disappears because the phase gradients no longer match.
\begin{enumerate}
  \item The Quantum Eraser Restores Phase Coherence
\end{enumerate}
The 'eraser' does not change the past. Instead, it changes the T0-Vakuum-phase boundary conditions by
removing which-path information stored in entanglement. This restores:
\Delta$\theta$ = constant
But only for a specific subset of correlated events. Thus, interference appears only in the coincidence
counts.
\begin{enumerate}
  \item Why Delayed Choice Does Not Imply Retrocausality
\end{enumerate}
DCQE appears to imply future choices affect past events, but in DVFT:
\begin{itemize}
  \item The T0-Vakuum field \Phi spans the entire apparatus.
  \item Phase coherence or decoherence is global, not local.
  \item The final coincidence sorting groups events by their T0-Vakuum-phase relationships.
\end{itemize}
No signal travels backward in time. No photon changes its past. The T0-Vakuum-phase field already contains
all correlations. The delayed-choice simply selects a subset consistent with restored coherence.
\begin{enumerate}
  \item DVFT Equation for Interference and Decoherence
\end{enumerate}
Full interference:
I(x) = |\Phi?(x) + \Phi?(x)|?
Decoherence from which-path:
\Phi $\rightarrow$ (\Phi? e^{i?$\theta$?}) + (\Phi? e^{i?$\theta$?})
\Delta$\theta$ = $\theta$? ? $\theta$? + (?$\theta$? ? ?$\theta$?) $\rightarrow$ undefined
Eraser restores coherence:
?$\theta$? = ?$\theta$? $\Rightarrow$ \Delta$\theta$ = constant
International Journal for Multidisciplinary Research (IJFMR)
E-ISSN: 2582-2160 $\bullet$ Website: www.ijfmr.com $\bullet$ Email: editor@ijfmr.com
IJFMR250664112 Volume 7, Issue 6, November-December 2025 68
Therefore, interference reappears only in the selected coincidence channel.
\begin{enumerate}
  \item Photon Behavior Under DVFT
\end{enumerate}
In DVFT:
\begin{itemize}
  \item A photon is a localized excitation riding on the T0-Vakuum field.
  \item Its trajectory is not determined by classical paths but by T0-Vakuum-phase geometry.
  \item Which-path detection modifies the T0-Vakuum phase, not the photon itself.
  \item Erasure restores the phase structure, enabling interference to reappear.
\end{itemize}
This interpretation avoids the paradoxes of retrocausal or consciousness-based explanations.
\begin{enumerate}
  \item Why DVFT Explains DCQE Better Than Standard QM
\end{enumerate}
Standard QM says: 'Wavefunction collapse depends on whether information is available.'
But it does not explain *how* or *why* this information physically affects the photon.
DVFT explains DCQE through:
\begin{itemize}
  \item T0-Vakuum-phase coherence
  \item T0-Vakuum-phase decoherence
  \item Entanglement-induced phase tagging
  \item Erasure-induced re-coherence
\end{itemize}
Everything occurs in the T0-Vakuum field \Phi, which is real, continuous, and causal.
Conclusion
The Delayed Choice Quantum Eraser experiment does not require retrocausality or paradoxi\footnote{Siehe auch T0-Dokument: \texttt{009_T0_xi_ursprung_De.pdf}}cal reasoning.
DVFT provides a physically intuitive explanation: the T0-Vakuum field?s phase determines whether
interference appears, not the photon's knowledge or future choices. Which-path information disrupts
T0-Vakuum-phase coherence. Eraser actions restore it. The delayed-choice affects how events are *classified*,
not how they occur. DVFT thus unifies DCQE with classical intuition while preserving quantum
predictions exactly.

\section{Referenzen zu T0-Dokumenten}

Dieses Kapitel steht in Zusammenhang mit folgenden T0-Dokumenten im Repository \texttt{2/pdf/}:

\begin{itemize}
  \item \texttt{002\_T0\_Grundlagen\_De.pdf} -- T0 Zeit-Masse-Dualit?t Grundlagen
  \item \texttt{004\_T0\_Energie\_De.pdf} -- T0 Energiefeld-Theorie
  \item \texttt{009\_T0\_xi\_ursprung\_De.pdf} -- Ursprung des geometrischen Parameters $\xi$
  \item \texttt{201\_DVFT-alles\_De.pdf} -- Vollst?ndiges DVFT-Dokument (Rahmenwerk)
\end{itemize}