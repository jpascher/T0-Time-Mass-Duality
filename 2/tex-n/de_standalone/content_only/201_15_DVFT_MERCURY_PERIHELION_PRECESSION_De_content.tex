\begin{enumerate}
  \item Introduction
\end{enumerate}
This chapter derives the perihelion precession of Mercury using ONLY the Dynamic T0-Vakuum Field
Theory (DVFT), without invoking Einstein?s General Relativity field equations. The key idea is that in
International Journal for Multidisciplinary Research (IJFMR)
E-ISSN: 2582-2160 $\bullet$ Website: www.ijfmr.com $\bullet$ Email: editor@ijfmr.com
IJFMR250664112 Volume 7, Issue 6, November-December 2025 36
the high-acceleration regime of the Solar System, DVFT reduces to a Newtonian potential plus a tiny 1/r?
correction generated by the $\theta$-field dynamics. This correction leads to the correct 43 arcsec/century
precession.
\begin{enumerate}
  \item DVFT in the Solar System: High-Acceleration Limit
\end{enumerate}
DVFT describes gravity as arising from convergence of a T0-Vakuum phase field $\theta$. Its Lagrangian\footnote{Siehe auch T0-Dokument: \texttt{027_T0_lagrndian_De.pdf}} contains
nonlinear terms:
L_$\theta$ = -\Lambda_v + (??/2)X - (?/(3 a??)) X^{3/2},
with X = -g^{$\mu$?} $\partial$_$\mu$$\theta$ $\partial$_?$\theta$.
In the Solar System, gravitational acceleration is much larger than a? (~10??? m/s?):
g / a? ~ 10?.
Thus, nonlinear MOND/DVFT corrections vanish. DVFT reduces to a GR-like weak-field theory,
predicting an effective potential of the form:
U_eff(r) = -GMm/r + L?/(2mr?) - GM L?/(mc? r?).
\begin{enumerate}
  \item DVFT Effective Potential for Mercury
\end{enumerate}
The effective central-force potential for a test mass m orbiting the Sun in DVFT becomes:
U_DVFT(r) = -GMm/r + L?/(2mr?) - (GM L?)/(m c? r?).
Terms:
\begin{itemize}
  \item ?GMm/r : Newtonian gravity,
  \item L?/(2mr?) : centrifugal barrier,
  \item ?GM L?/(m c? r?) : DVFT high-g correction.
\end{itemize}
This 1/r? term is responsible for perihelion precession.
\begin{enumerate}
  \item Orbit Equation Using Classical Mechanics Only
\end{enumerate}
Define u(?) = 1/r. The Binet equation for a central potential U(r) is:
d?u/d?? + u = -(m / L?u?) (dU/dr).
Convert U(r) to U(u):
U(u) = -k u + (L?/2m)u? + ? u?,
where k = GMm, ? = ?GM L?/(m c?).
Taking the derivative and substituting into Binet?s equation yields:
d?u/d?? + (mk/L?) = (3m?/L?) u?.
The ?-term represents the DVFT correction. For ?=0, this gives perfect ellipses.
\begin{enumerate}
  \item Perturbative Solution and Precession
\end{enumerate}
Using the unperturbed solution:
u?(?) = (mk/L?)(1 + e cos?),
and treating ? as a small parameter, the first-order perturbation yields a precession per orbit:
\Delta? = 6? k? / (L? c? (1?e?)).
Substitute k = GMm and L? = m?GM a(1?e?):
\Delta? = 6? GM / (a (1?e?) c?).
This equation can be used to calculate the perihelion precession for Mercury.
\begin{enumerate}
  \item Input Physical Constants and Mercury Parameters
\end{enumerate}
\begin{itemize}
  \item Gravitational constant: G = 6.6743 $\times$ 10??? m? kg?? s??
  \item Solar mass: M = 1.9885 $\times$ 10?? kg
  \item $\rightarrow$ GM = 1.3271 $\times$ 10?? m? s??
\end{itemize}
International Journal for Multidisciplinary Research (IJFMR)
E-ISSN: 2582-2160 $\bullet$ Website: www.ijfmr.com $\bullet$ Email: editor@ijfmr.com
IJFMR250664112 Volume 7, Issue 6, November-December 2025 37
\begin{itemize}
  \item Speed of light: c = 2.9979 $\times$ 10? m/s
  \item $\rightarrow$ c? = 8.9876 $\times$ 10?? m? s??
  \item Mercury semi-major axi\footnote{Siehe auch T0-Dokument: \texttt{009_T0_xi_ursprung_De.pdf}}s: a = 5.7909 $\times$ 10?? m
  \item Mercury orbital eccentricity: e = 0.2056
  \item Mercury orbital period: T $\approx$ 0.240846 years
\end{itemize}
\begin{enumerate}
  \item Compute the Denominator: a(1 ? e?)c?
\end{enumerate}
First compute 1 ? e?:
1 ? e? $\approx$ 1 ? (0.2056)? = 0.9577
Multiply:
a(1 ? e?) $\approx$ 5.7909 $\times$ 10?? $\times$ 0.9577 = 5.54 $\times$ 10?? m
Now multiply by c?:
a(1 ? e?)c? $\approx$ 5.54 $\times$ 10?? $\times$ 8.99 $\times$ 10??
= 4.98 $\times$ 10?? m? s??
\begin{enumerate}
  \item Compute the Dimensionless Factor GM / [a(1 ? e?)c?]
\end{enumerate}
GM = 1.3271 $\times$ 10?? m? s??
Divide:
GM / [a(1 ? e?)c?] = 1.3271 $\times$ 10?? / 4.9846 $\times$ 10??
$\approx$ 2.66 $\times$ 10??
\begin{enumerate}
  \item Multiply by 6? to Get Radians per Orbit
\end{enumerate}
6? $\approx$ 18.8496
Thus:
\Delta? (radians/orbit) = 18.8496 $\times$ 2.66 $\times$ 10??
$\approx$ 5.02 $\times$ 10?? radians per orbit
\begin{enumerate}
  \item Convert Radians per Orbit $\rightarrow$ Arcseconds per Orbit
\end{enumerate}
1 radian = 206,264.806 arcseconds
Multiply:
\Delta?_arcsec = 5.02 $\times$ 10?? $\times$ 2.06265 $\times$ 10?
$\approx$ 0.1035 arcseconds per orbit
\begin{enumerate}
  \item Orbits per Century
\end{enumerate}
Mercury orbital period:
T $\approx$ 0.240846 years
Thus number of orbits in 100 years:
N = 100 / 0.240846 $\approx$ 415.2 orbits per century
\begin{enumerate}
  \item Total Perihelion Advance per Century
\end{enumerate}
Multiply the per-orbit advance by the number of orbits:
\Delta?_century = 0.1035 arcsec/orbit $\times$ 415.2 orbits/century
$\approx$ 42.98 arcseconds per century
Thus:
\Delta?_DVFT $\approx$ 43 arcsec/century
which matches the observed anomalous perihelion precession of Mercury.
This derivation used:
\begin{itemize}
  \item Classical mechanics,
  \item DVFT effective potential,
\end{itemize}
International Journal for Multidisciplinary Research (IJFMR)
E-ISSN: 2582-2160 $\bullet$ Website: www.ijfmr.com $\bullet$ Email: editor@ijfmr.com
IJFMR250664112 Volume 7, Issue 6, November-December 2025 38
\begin{itemize}
  \item No Einstein field equations.
\end{itemize}
\begin{enumerate}
  \item Why DVFT Predicts the Same Result as GR in this Regime
\end{enumerate}
Because Mercury is deep in the high-acceleration regime:
g >> a?,
DVFT's nonlinear low-acceleration corrections vanish. Its weak-field expansion forces a 1/r? correction
identical in functional form to GR?s 1PN term. Solar System tests constrain any deviation to <10???
fractionally, so the DVFT correction coefficient must match GR?s to this accuracy.
\begin{enumerate}
  \item Physical Interpretation
\end{enumerate}
\begin{itemize}
  \item DVFT predicts Newtonian gravity with a small relativistic correction from $\theta$-field curvature.
  \item This correction appears as an extra inward acceleration proportional to 1/r?.
  \item That correction shifts the orbital frequency slightly, causing the perihelion to advance.
  \item DVFT predicts the same value as GR because both theories share the same high-g limit.
\end{itemize}
Conclusion
Using only DVFT (and classical orbit theory), the perihelion shift is:
\Delta?_DVFT = 6? GM / (a (1?e?) c?).
This reproduces the observed 43 arcsec/century without invoking Einstein?s equations. Therefore: DVFT
is consistent with Solar System precision tests while remaining a fundamentally different theory from GR
in the low-acceleration regime.

\section{Referenzen zu T0-Dokumenten}

Dieses Kapitel steht in Zusammenhang mit folgenden T0-Dokumenten im Repository \texttt{2/pdf/}:

\begin{itemize}
  \item \texttt{002\_T0\_Grundlagen\_De.pdf} -- T0 Zeit-Masse-Dualit?t Grundlagen
  \item \texttt{004\_T0\_Energie\_De.pdf} -- T0 Energiefeld-Theorie
  \item \texttt{009\_T0\_xi\_ursprung\_De.pdf} -- Ursprung des geometrischen Parameters $\xi$
  \item \texttt{201\_DVFT-alles\_De.pdf} -- Vollst?ndiges DVFT-Dokument (Rahmenwerk)
\end{itemize}