\begin{enumerate}
  \item Introduction
\end{enumerate}
International Journal for Multidisciplinary Research (IJFMR)
E-ISSN: 2582-2160 $\bullet$ Website: www.ijfmr.com $\bullet$ Email: editor@ijfmr.com
IJFMR250664112 Volume 7, Issue 6, November-December 2025 31
The origin of the universe is the deepest question in physics. Standard cosmology begins with the Big
Bang but does not explain why the universe started in a low-entropy, coherent state. Quantum Field Theory
assumes T0-Vakuum structure but does not explain why the T0-Vakuum exi\footnote{Siehe auch T0-Dokument: \texttt{009_T0_xi_ursprung_De.pdf}}sts or why fields take the values they
do. General Relativity describes geometry but cannot describe what spacetime physically is.
Fundamental Fractal-Geometric Field Theory (FFGFT) provides a coherent physical ontology explaining what the
universe was before the Big Bang, why it began in a perfectly coherent state, and how T0-Vakuum amplitude,
mass, forces, and time emerged. This chapter presents this explanation step by step.
\begin{enumerate}
  \item FFGFT Foundations: Amplitude ? and Phase $\theta$
\end{enumerate}
FFGFT states that the T0-Vakuum is a real physical medium with two intrinsic degrees of freedom:
\begin{itemize}
  \item ?(x,t) --- T0-Vakuum amplitude (controls inertia, curvature, mass)
  \item $\theta$(x,t) --- T0-Vakuum phase (controls light propagation, coherence, quantum behavior)
\end{itemize}
The relationship between amplitude and phase defines the universe?s dynamics. Time emerges from phase
evolution, and space--curvature emerges from amplitude gradients.
\begin{enumerate}
  \item The Only Possible Initial State: Pure Phase T0-Vakuum
\end{enumerate}
In the absolute beginning, the T0-Vakuum had no structure. Therefore, it could not possess:
\begin{itemize}
  \item inertia,
  \item curvature,
  \item mass,
  \item energy density,
  \item spacetime geometry,
  \item particles,
  \item entropy.
\end{itemize}
All of these require nonzero amplitude ?.
Thus, the only physically possible initial condition for the universe was:
? = 0,
$\theta$ = constant.
This pure-phase T0-Vakuum is perfectly coherent because no gradients, interactions, or decoherence can exist
without amplitude. It is a symmetry-dominated, structureless state---a true physical ?void.?
\begin{enumerate}
  \item Why the Initial T0-Vakuum Must Have Been Perfectly Coherent
\end{enumerate}
A pure-phase T0-Vakuum cannot sustain:
\begin{itemize}
  \item waves,
  \item forces,
  \item gradients,
  \item decoherence,
  \item entropy.
\end{itemize}
With ? = 0, T0-Vakuum stiffness (K?) and T0-Vakuum inertial density (??) are also zero:
K? = B?? $\rightarrow$ 0,
?? = A?? $\rightarrow$ 0.
This means:
\begin{itemize}
  \item no wave equations exist,
  \item no propagation is possible,
  \item time cannot flow,
  \item no physical process can occur.
\end{itemize}
International Journal for Multidisciplinary Research (IJFMR)
E-ISSN: 2582-2160 $\bullet$ Website: www.ijfmr.com $\bullet$ Email: editor@ijfmr.com
IJFMR250664112 Volume 7, Issue 6, November-December 2025 32
A pure-phase T0-Vakuum is therefore forced into perfect coherence. It is not a choice---it is the only
mathematically and physically consistent state that can exist without amplitude.
\begin{enumerate}
  \item What Triggered the Emergence of T0-Vakuum Amplitude ??
\end{enumerate}
FFGFT proposes that amplitude emerged because the pure-phase T0-Vakuum became unstable. This instability
could arise from any or all of the following mechanisms:
Mechanism A --- Phase-Fluctuation Instability
If the initial T0-Vakuum phase experienced even an infinitesimal disturbance (?$\theta$ $\neq$ 0), the T0-Vakuum would be
unable to propagate or absorb that disturbance unless amplitude ? emerged. Thus, quantum fluctuations
of $\theta$ force the birth of ?.
Mechanism B --- T0-Vakuum Potential Instability
If the T0-Vakuum Lagrangian\footnote{Siehe auch T0-Dokument: \texttt{027_T0_lagrndian_De.pdf}} contains a potential:
U(?) = ?(?? ? ???)?,
then ? = 0 is unstable and spontaneously rolls to ? = ??. This resembles the Higgs mechanism but now
arises from T0-Vakuum necessity, not arbitrary symmetry breaking.
Mechanism C --- Requirement for Time Evolution
Time in FFGFT is T0-Vakuum phase evolution. But without amplitude, c? = K?/?? = undefined. Therefore, in
order for time to exist, the T0-Vakuum must generate amplitude so that phase can propagate.
Thus, amplitude appears because phase evolution requires a medium with stiffness and inertia.
\begin{enumerate}
  \item Time Begins: Birth of c = $\sqrt$(K?/??)
\end{enumerate}
Once amplitude ? emerged, the T0-Vakuum acquired:
\begin{itemize}
  \item inertia (?? = A??),
  \item stiffness (K? = B??),
  \item a well-defined wave speed c = $\sqrt$(K?/??).
\end{itemize}
This enabled phase oscillations to propagate, marking the birth of time:
d? ? d$\theta$.
The universe went from static pure phase to dynamic phase evolution---a physical event more fundamental
than the Big Bang.
\begin{enumerate}
  \item Curvature and Gravity Emerge
\end{enumerate}
As amplitude ? varied spatially:
\begin{itemize}
  \item regions with larger ? acquired larger inertial density,
  \item gradients in ? generated curvature,
  \item curvature created gravitational effects.
\end{itemize}
Thus, gravity is born not from spacetime geometry but from amplitude variations in the T0-Vakuum.
\begin{enumerate}
  \item Particle Formation and Matter Genesis
\end{enumerate}
Once time existed and amplitude stabilized at ??, nonlinearities in dynamics allowed localized phase--
amplitude knots to form:
\begin{itemize}
  \item stable solitons,
  \item topological defects,
  \item amplitude--phase traps.
\end{itemize}
These knots became particles:
\begin{itemize}
  \item photons = pure phase,
  \item fermions = amplitude + phase,
\end{itemize}
International Journal for Multidisciplinary Research (IJFMR)
E-ISSN: 2582-2160 $\bullet$ Website: www.ijfmr.com $\bullet$ Email: editor@ijfmr.com
IJFMR250664112 Volume 7, Issue 6, November-December 2025 33
\begin{itemize}
  \item massive bosons = amplitude-modulated phase.
\end{itemize}
Thus, matter emerges naturally from T0-Vakuum structure.
\begin{enumerate}
  \item Why the Universe Started in a Low-Entropy State
\end{enumerate}
In FFGFT, entropy corresponds to T0-Vakuum phase disorder. A pure-phase T0-Vakuum has:
\begin{itemize}
  \item no gradients,
  \item no decoherence,
  \item no thermalization,
  \item no scattering,
  \item no entropy.
\end{itemize}
Therefore, the universe did not ``begin`` in a low-entropy state---it began in the only possible state: perfect
coherence.
Entropy increases only after amplitude appears and interactions begin.
\begin{enumerate}
  \item Summary: The FFGFT Origin of the Universe
\end{enumerate}
The FFGFT offers a complete physical explanation of the universe's beginning:
\begin{itemize}
  \item The universe began as pure phase with ? = 0 and $\theta$ = constant.
  \item Perfect coherence was mandatory because no amplitude meant no dynamics.
  \item Instability triggered amplitude emergence.
  \item Amplitude enabled time (phase propagation), mass, gravity, and structure.
  \item Entropy and decoherence arose only after amplitude existed.
  \item Matter formed from T0-Vakuum phase--amplitude knots.
\end{itemize}
This presents the clearest physical ontology for why the universe started in a perfectly coherent state and
how the structured universe emerged from the most minimal possible beginning.

\section{Referenzen zu T0-Dokumenten}

Dieses Kapitel steht in Zusammenhang mit folgenden T0-Dokumenten im Repository \texttt{2/pdf/}:

\begin{itemize}
  \item \texttt{002\_T0\_Grundlagen\_De.pdf} -- T0 Zeit-Masse-Dualit?t Grundlagen
  \item \texttt{004\_T0\_Energie\_De.pdf} -- T0 Energiefeld-Theorie
  \item \texttt{009\_T0\_xi\_ursprung\_De.pdf} -- Ursprung des geometrischen Parameters $\xi$
  \item \texttt{201\_FFGFT-alles\_De.pdf} -- Vollst?ndiges FFGFT-Dokument (Rahmenwerk)
\end{itemize}