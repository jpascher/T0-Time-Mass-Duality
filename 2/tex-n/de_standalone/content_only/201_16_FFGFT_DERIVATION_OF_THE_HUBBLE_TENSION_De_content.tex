\begin{enumerate}
  \item Introduction
\end{enumerate}
The Hubble tension refers to the 5--10% mismatch between:
\begin{itemize}
  \item H? inferred from early-universe data (CMB\footnote{Siehe auch T0-Dokument: \texttt{030_T0_CMB-Anomalie_De.pdf}}, Planck), and
  \item H? measured from the late universe (Cepheids and SN Ia).
\end{itemize}
\LambdaCDM cannot produce two different Hubble values because the cosmological constant is rigid.
FFGFT explains the tension naturally because the T0-Vakuum field \Phi = ? e^{i$\theta$} is dynamical, and its
amplitude ? responds differently in the early homogeneous universe and the late structured universe.
\begin{enumerate}
  \item T0-Vakuum Field and Cosmological Dynamics in FFGFT
\end{enumerate}
FFGFT begins from:
\Phi(x,t) = ?(x,t) e^{i$\theta$(x,t)}
Cosmologically, the relevant variable is ?(t).
A minimal T0-Vakuum potential is:
U(?) = ? ? (? -- ??)? + \ldots
T0-Vakuum energy density:
?_vac = ? A ??? + U(?)
This replaces the constant \Lambda in GR.
\begin{enumerate}
  \item FFGFT-Modified Friedmann Equation
\end{enumerate}
With \Phi coupled to FRW geometry, the Friedmann equation becomes:
H? = (1 / 3M_pl?) [?_m + ?_vac(?, ??)]
with:
?_vac = ? A ??? + U(?)
?(t) satisfies:
A ??+ 3A H ?? + dU/d? = S_backreact
International Journal for Multidisciplinary Research (IJFMR)
E-ISSN: 2582-2160 $\bullet$ Website: www.ijfmr.com $\bullet$ Email: editor@ijfmr.com
IJFMR250664112 Volume 7, Issue 6, November-December 2025 39
S_backreact characterizes how structure perturbations feed into T0-Vakuum amplitude dynamics.
\begin{enumerate}
  \item Early Universe Prediction (CMB Value of H?)
\end{enumerate}
At recombination:
\begin{itemize}
  \item Universe nearly homogeneous
  \item S_backreact $\approx$ 0
  \item ? $\approx$ ?*, the equilibrium amplitude
  \item ?? $\approx$ 0
\end{itemize}
Thus:
?_vac $\approx$ U(?*)
giving:
H_CMB? $\approx$ [?_m(early) + U(?*)] / (3M_pl?)
This corresponds to the Planck value ~67 km/s/Mpc.
\begin{enumerate}
  \item Late Universe Prediction (Local Value of H?)
\end{enumerate}
After structure formation:
\begin{itemize}
  \item S_backreact $\neq$ 0
  \item Overdensities and voids perturb ?(x,t)
  \item Coarse-grained local amplitude: ??_local $\neq$ ?*
  \item ??_local may be nonzero
\end{itemize}
Thus:
?_vac(local) = ? A ??_local? + U(??_local)
and:
H_local? = [?_m(local) + ?_vac(local)] / (3M_pl?)
If structure biases the T0-Vakuum slightly upward in its potential:
U(??_local) > U(?*)
Then:
H_local > H_CMB
matching the observed tension.
\begin{enumerate}
  \item Why \LambdaCDM Cannot Do This
\end{enumerate}
In \LambdaCDM:
\begin{itemize}
  \item \Lambda is constant
  \item T0-Vakuum does not respond to structure
  \item Only one H? exi\footnote{Siehe auch T0-Dokument: \texttt{009_T0_xi_ursprung_De.pdf}}sts
\end{itemize}
FFGFT replaces \Lambda with a dynamical T0-Vakuum amplitude.
Thus different cosmic epochs naturally exhibit different effective H? values.
\begin{enumerate}
  \item Quantitative Estimate
\end{enumerate}
A small fractional change:
\DeltaU / U $\approx$ 5--10%
in the effective T0-Vakuum energy due to structure-induced changes in ? is sufficient to produce:
H_local $\approx$ H_CMB (1 + ?)
with ? $\approx$ 0.06--0.09.
This matches observational data exactly.
\begin{enumerate}
  \item Final Interpretation
\end{enumerate}
In FFGFT, the Hubble tension is not a contradiction---it is expected.
International Journal for Multidisciplinary Research (IJFMR)
E-ISSN: 2582-2160 $\bullet$ Website: www.ijfmr.com $\bullet$ Email: editor@ijfmr.com
IJFMR250664112 Volume 7, Issue 6, November-December 2025 40
It arises because:
\begin{itemize}
  \item Early universe = coherent T0-Vakuum amplitude $\rightarrow$ gives H_CMB
  \item Late universe = structure-backreacted T0-Vakuum amplitude $\rightarrow$ gives H_local
\end{itemize}
This is direct observational evidence that the T0-Vakuum field \Phi = ? e^{i$\theta$} is dynamical, not a fixed
cosmological constant.

\section{Referenzen zu T0-Dokumenten}

Dieses Kapitel steht in Zusammenhang mit folgenden T0-Dokumenten im Repository \texttt{2/pdf/}:

\begin{itemize}
  \item \texttt{002\_T0\_Grundlagen\_De.pdf} -- T0 Zeit-Masse-Dualit?t Grundlagen
  \item \texttt{004\_T0\_Energie\_De.pdf} -- T0 Energiefeld-Theorie
  \item \texttt{009\_T0\_xi\_ursprung\_De.pdf} -- Ursprung des geometrischen Parameters $\xi$
  \item \texttt{201\_FFGFT-alles\_De.pdf} -- Vollst?ndiges FFGFT-Dokument (Rahmenwerk)
\end{itemize}