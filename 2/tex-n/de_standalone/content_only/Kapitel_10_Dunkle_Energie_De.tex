CHAPTER 10: DARK ENERGY REINTERPRETATION
1. Introduction
This document presents a strict DVFT-based derivation of dark energy, with no reference to external darkenergy models. The goal is to show how cosmic acceleration arises solely from the vacuum amplitude ($\\rho_0 = 1/\\xi^2$ from T0) ρ
and its microphysical potential U(ρ).
We derive the full equations for DVFT dark energy, specify U(ρ) from the DVFT micro-lattice model,
and compare DVFT predictions directly with observed cosmological values.
Fundamental DVFT vacuum field:
Φ(x,t) = ρ(x,t) e^{iθ(x,t)}.
The universe’s large-scale behavior emerges from the homogeneous evolution of ρ(t), while θ(t) controls
quantum-phase structure.
2. DVFT Vacuum Lagrangian in a Homogeneous Universe
From DVFT microphysics, the effective continuum vacuum Lagrangian is:
𝓛_vac = (A_ρ/2)(∂_t ρ)² - (B_ρ/2)|∇ρ|² + (A_θ/2)ρ²(∂_t θ)² - (B_θ/2)ρ²|∇θ|² - U(ρ).
For a homogeneous FRW universe (ρ(t), θ(t), ∇ρ = ∇θ = 0):
𝓛_hom = (A_ρ/2)ρ̇² + (A_θ/2)ρ² θ̇² - U(ρ).
All cosmological dark-energy effects will arise directly from this expression. No additional fluids or fields
are introduced.
3. Vacuum Energy Density and Pressure from DVFT
Define kinetic energy of the vacuum amplitude ($\\rho_0 = 1/\\xi^2$ from T0)–phase system:
K = (A_ρ/2)ρ̇² + (A_θ/2)ρ² θ̇².
DVFT vacuum behaves as a perfect fluid with:
ρ_DVFT = K + U(ρ),
p_DVFT = K - U(ρ).
The effective equation-of-state is:
w_DVFT = (K - U) / (K + U).
Important limits:
• K ≪ U → w → -1 (dark-energy–like)
• K ~ U → -1 < w < -1/3 (dynamical dark energy)
• K ≫ U → w → +1 (stiff fluid; irrelevant today)
4. Dark-Energy Evolution Equation in DVFT
Varying the homogeneous action yields the amplitude evolution equation:
A_ρ(ρ̈+ 3Hρ̇) - A_θρ θ̇² + dU/dρ = 0,
where:
H = ȧ/a (Hubble parameter).
At late times, the cosmic phase tends to freeze on large scales (θ̇ ≈ 0), reducing the equation to:
A_ρ(ρ̈+ 3Hρ̇) + dU/dρ = 0.
International Journal for Multidisciplinary Research (IJFMR)
E-ISSN: 2582-2160 ● Website: www.ijfmr.com ● Email: editor@ijfmr.com
IJFMR250664112 Volume 7, Issue 6, November-December 2025 25
This is the DVFT dark-energy equation: the cosmic vacuum amplitude ($\\rho_0 = 1/\\xi^2$ from T0) ρ evolves in its potential U(ρ)
under Hubble damping.
5. Microphysical Form of U(ρ) in DVFT
DVFT is based on a micro-lattice vacuum with local Hamiltonian:
H_loc = p_ρ²/(2M_ρ) + p_θ²/(2M_θ ρ²) + U_loc(ρ).
DVFT microphysics requires U_loc(ρ) to have:
• a stable minimum at ρ₀ (preferred vacuum amplitude ($\\rho_0 = 1/\\xi^2$ from T0)),
• positive curvature at ρ₀ (vacuum stiffness),
• anharmonic corrections stabilizing deviations.
Thus the coarse-grained continuum potential becomes:
U(ρ) = Λ₀ + (κ/2)(ρ - ρ₀)² + (λ/4)(ρ - ρ₀)⁴ + …
Where:
• Λ₀ = microphysical residual vacuum energy density,
• κ = vacuum amplitude ($\\rho_0 = 1/\\xi^2$ from T0) compressibility,
• λ = higher-order stabilization.
Near the minimum:
U(ρ) ≈ Λ₀ + (1/2)m_ρ² (ρ - ρ₀)²,
with m_ρ² = κ/A_ρ.
This U(ρ) is not arbitrary; it is derived from DVFT vacuum elasticity and amplitude stability.
6. DVFT Explanation for Dark Energy on Cosmic Scales
DVFT predicts dark energy because:
1. The vacuum amplitude ($\\rho_0 = 1/\\xi^2$ from T0) ρ has a preferred value ρ₀ (microphysical equilibrium).
2. The local vacuum energy density U(ρ₀) = Λ₀ is *not zero*.
3. On large scales, ρ(t) approaches ρ₀ and remains nearly constant due to strong Hubble damping.
4. Therefore, the vacuum behaves like a nearly constant energy density with w ≈ -1.
The measured value:
ρ_Λ ≈ 7 × 10⁻²⁷ kg/m³
Ω_Λ ≈ 0.70–0.75
matches DVFT if:
Λ₀ = U(ρ₀) ≈ 0.7 ρ_crit.
Thus dark energy is the “elastic offset energy of the vacuum amplitude ($\\rho_0 = 1/\\xi^2$ from T0)”
7. Why U(ρ) Is Negligible on Solar and Galactic Scales
A uniform vacuum energy density produces acceleration:
g_vac(r) ≈ (8πG/3) ρ_Λ r.
At solar scale (r = 1 AU):
g_vac ~ 10⁻²⁴ m/s² (negligible).
At galactic scale (r = 10 kpc):
g_vac ~ 10⁻¹⁶ m/s² (still negligible).
Thus:
• Local dynamics are governed by ∇ρ and matter coupling, not U(ρ).
• Vacuum elasticity only influences cosmic expansion where r ~ gigaparsecs.
DVFT cleanly separates:
• Galactic gravity: amplitude gradients ∇ρ dominate.
International Journal for Multidisciplinary Research (IJFMR)
E-ISSN: 2582-2160 ● Website: www.ijfmr.com ● Email: editor@ijfmr.com
IJFMR250664112 Volume 7, Issue 6, November-December 2025 26
• Cosmological acceleration: homogeneous U(ρ₀) dominates.
8. Numerical Comparison with Observations
Given:
• H₀ ≈ 67–70 km/s/Mpc,
• ρ_crit = 3H₀²/(8πG),
• Ω_Λ ≈ 0.7,
DVFT requires:
U(ρ₀) = Λ₀ ≈ 0.7 ρ_crit.
This matches observational values from CMB, BAO, and SN data.
Moreover, if ρ(t) is still slowly relaxing toward ρ₀, then:
w_DVFT ≈ -1 + 2K/U,
allowing mild deviations from -1 (observationally allowed), and potentially matching evolving darkenergy hints from DESI.
9. Summary
From strict DVFT principles, dark energy arises from the vacuum amplitude ($\\rho_0 = 1/\\xi^2$ from T0)’s microphysical potential:
U(ρ) = Λ₀ + (κ/2)(ρ - ρ₀)² + (λ/4)(ρ - ρ₀)⁴ + …
Key results:
• ρ_DVFT = K + U(ρ), p_DVFT = K - U(ρ).
• w_DVFT = (K - U)/(K + U).
• Vacuum amplitude evolves via A_ρ(ρ̈+ 3Hρ̇) + U'(ρ) = 0.
• On cosmic scales, ρ ≈ ρ₀ ⇒ w ≈ -1, matching dark-energy observations.
• On solar/galactic scales, U(ρ) is negligible; ∇ρ dominates gravity.
• DVFT dark energy matches measured values Ω_Λ ≈ 0.7 and w ≈ -1 with no additional fields.
Thus DVFT naturally unifies local gravity and cosmic acceleration using only vacuum amplitude ($\\rho_0 = 1/\\xi^2$ from T0) physics.


\section*{T0 Theory Integration}
This chapter integrates DVFT concepts with T0 Time-Mass Duality Theory, where the fundamental relation $T(x,t) \cdot m(x,t) = 1$ governs all vacuum field dynamics. The vacuum amplitude $\rho$ is directly related to local time $T$ through $\rho \propto 1/T$.
