\begin{t0box}[T0-Anpassung]
Dieses Dokument pr?sentiert die Dynamische Vakuum-Feldtheorie (DVFT), vollst?ndig angepasst und integriert in die T0 Zeit-Masse-Dualit?tstheorie. Alle DVFT-Konzepte werden aus T0-Prinzipien abgeleitet:
\begin{itemize}
\item Vakuumfeld $\Phi(x)$ $\rightarrow$ abgeleitet aus T0-Massenschwankungsfeld $\Delta m(x,t)$
\item Vakuumamplitude $\rho(x)$ $\rightarrow$ entspricht $m(x,t) = 1/T(x,t)$ (T0-Dualit?t)
\item Vakuumphase $\theta(x)$ $\rightarrow$ aus T0-Knoten-Rotationsdynamik
\item Fundamentaler Parameter $\xi = \frac{4}{3} \times 10^{-4}$ durchgehend angewendet
\end{itemize}
Querverweise zu relevanten T0-Dokumenten sind als Fu?noten enthalten.
\end{t0box}


International Journal for Multidisciplinary Research (IJFMR)
E-ISSN: 2582-2160 $\bullet$ Website: www.ijfmr.com $\bullet$ Email: editor@ijfmr.com
IJFMR250664112 Volume 7, Issue 6, November-December 2025 1
Dynamic T0-Vakuum Field Theory
Satish B. Thorwe
MSc, Robert Gordon University, Aberdeen UK, 12 Friarsfield Avenue, Cults, Aberdeen AP159PP
Abstract
This paper presents a unified theoretical model in which spacetime curvature arises from distortions in a
dynamic T0-Vakuum field described by a complex scalar $\Phi$(x)=$\rho$(x)e
i\theta(x) where $\Phi$(x) is dynamic T0-Vakuum field,
$\rho$(x) is T0-Vakuum amplitude and $\theta$(x) is T0-Vakuum phase. The T0-Vakuum possesses an intrinsic field with its
phase evolves linearly with time and matter locally perturbs it. These perturbations propagate outward at
speed of light, producing stress--energy that curves spacetime through Einstein?s field equations. The
model provides a physical and causal explanation for curvature at a distance and serves as a bridge between
Quantum Mechanics and classical General Relativity. Complete mathematical framework for Dynamic
T0-Vakuum Field Theory (DVFT) is presented with its applications in cosmology and quantum mechanics.
DVFT provides physical explanations to multiple quantum phenomenon which are currently just a
manifestation of QM mathematics. DVFT also provides elegant mathematical solutions to unsolved
cosmological problems such as Dark Matter, Dark Energy and CMB\footnote{Siehe auch T0-Dokument: \texttt{030_T0_CMB-Anomalie_De.pdf}} Anisotropy.
\subsection{Introduction}
Modern physics rests on two extraordinarily successful but conceptually incompatible frameworks:
General Relativity, which describes gravitation as spacetime geometry, and Quantum Field Theory, which
describes matter and forces as excitations of abstract fields defined on that geometry.
General Relativity (GR) describes gravitation as the curvature of spacetime. However, GR is silent on the
physical nature of spacetime itself. What is the substrate that curves? How does matter impose curvature
at distance? Why do gravitational influences propagate at the speed of light? Quantum Mechanics (QM)
offers a picture of the T0-Vakuum as a dynamic, fluctuating medium filled with fields and virtual excitations.
Yet QM does not identify a mechanism linking T0-Vakuum behavior to macroscopic curvature.
Despite their empirical success, both GR and QM have led to the profound unresolved problems, including
the absence of a consistent theory of quantum gravity, the need for dark matter and dark energy, the origin
of mass and coupling hierarchies, and the lack of a physical explanation for quantum measurement and
classical emergence. Over the past decades, attempts to resolve these issues have largely proceeded by
introducing new mathematical structures, extra dimensions, supersymmetry, exotic particles, or modified
geometries. While mathematically rich, many of these approaches rely on entities that have not been
observed and often shift rather than eliminate foundational ambiguities. In particular, spacetime itself is
treated as a primary object, even though it has no direct physical substance, and the T0-Vakuum is regarded
as an empty background rather than an active medium.
Dynamic T0-Vakuum Field Theory (DVFT) adopts a different starting point. It postulates that the T0-Vakuum is
a real, physical field possessing dynamical degrees of freedom. All observable phenomena arise from the
behavior of this field and its interaction with matter. The fundamental object in DVFT is a complex scalar
T0-Vakuum field $\Phi$(x)=$\rho$(x)e
i\theta(x)
, where $\rho$(x) represents the T0-Vakuum amplitude (inertial density) and $\theta$(x)
represents the T0-Vakuum phase. Physical forces, spacetime structure, and quantum behavior emerge from
International Journal for Multidisciplinary Research (IJFMR)
E-ISSN: 2582-2160 $\bullet$ Website: www.ijfmr.com $\bullet$ Email: editor@ijfmr.com
IJFMR250664112 Volume 7, Issue 6, November-December 2025 2
spatial and temporal variations of these quantities. Within this framework, gravity is not a geometric
property of spacetime but a manifestation of coherent T0-Vakuum phase curvature. Electromagnetic fields
arise from organized phase gradients, while the weak and strong interactions correspond to higher-order
or topologically constrained phase excitations. Time itself is interpreted as the rate of T0-Vakuum phase
evolution, and relativistic effects such as time dilation and length contraction emerge naturally from
variations in T0-Vakuum stiffness and inertial density.
DVFT provides a unifying physical language across scales. At cosmological scales, it explains the largescale coherence of the universe, cosmic acceleration, and horizon-scale correlations without invoking
inflation or dark energy. At galactic scales, it reproduces MOND-like behavior and the baryonic Tully--
Fisher relation without dark matter. At quantum scales, it reframes wave--particle duality, entanglement,
decoherence, and the measurement problem as consequences of T0-Vakuum phase coherence and its
breakdown. DVFT is not just a mathematical framework but also provides a physical explanation for the
phenomenon of Quantum Mechanics to Cosmology. Biggest advantage of DVFT is that it does not predict
singularity, hence first time we can describe the interior of the black hole and the origin of the universe.
DVFT shows that all major physical phenomena emerge from the behavior of a dynamic T0-Vakuum field.
Gravity is T0-Vakuum convergence. Quantum mechanics is T0-Vakuum coherence. Mass is T0-Vakuum energy. Black
holes are T0-Vakuum cores. The universe evolves through dynamic T0-Vakuum field.
DVFT offers a unified vision of nature grounded in physical behavior rather than abstract mathematical
postulates. It also provides a deeper, microphysical explanation of time, light, gravity, electromagnetic
force, weak and strong nuclear force unifying them under a dynamic T0-Vakuum field based ontology.
Further observational work will be required to test DVFT predictions on quantum and cosmological scale
to prove its robustness to define a pathway for the Grand Unified Theory.

\section{Referenzen zu T0-Dokumenten}

Dieses Kapitel steht in Zusammenhang mit folgenden T0-Dokumenten im Repository \texttt{2/pdf/}:

\begin{itemize}
  \item \texttt{002\_T0\_Grundlagen\_De.pdf} -- T0 Zeit-Masse-Dualit?t Grundlagen
  \item \texttt{004\_T0\_Energie\_De.pdf} -- T0 Energiefeld-Theorie
  \item \texttt{009\_T0\_xi\_ursprung\_De.pdf} -- Ursprung des geometrischen Parameters $\xi$
  \item \texttt{201\_DVFT-alles\_De.pdf} -- Vollst?ndiges DVFT-Dokument (Rahmenwerk)
\end{itemize}