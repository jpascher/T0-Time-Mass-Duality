\begin{enumerate}
  \item Introduction
\end{enumerate}
Fundamental Fractal-Geometric Field Theory (FFGFT) predicts distinct regimes of gravitational behavior determined by
the magnitude of the T0-Vakuum phase gradient
X = -g^{$\mu$?} $\partial$$\mu$$\theta$ $\partial$?$\theta$.
These regimes---strong field, weak field, deep field, and an ultra-deep cosmological regime---correspond
to different nonlinear responses of the T0-Vakuum. This chapter provides a unified description of T0-Vakuum
behavior from local strong-gravity environments to the largest cosmological scales where dark energy
dominates.
\begin{enumerate}
  \item Strong Field Regime (X >> a??)
\end{enumerate}
In high-acceleration environments such as near stellar surfaces, neutron stars, or black hole exteriors,
phase gradients are large. The T0-Vakuum response
L_X = $\partial$L_$\theta$/$\partial$X
approaches an almost constant value:
L_X $\approx$ ??/2.
Nonlinear terms in the Lagrangian\footnote{Siehe auch T0-Dokument: \texttt{027_T0_lagrndian_De.pdf}},
International Journal for Multidisciplinary Research (IJFMR)
E-ISSN: 2582-2160 $\bullet$ Website: www.ijfmr.com $\bullet$ Email: editor@ijfmr.com
IJFMR250664112 Volume 7, Issue 6, November-December 2025 22
L_$\theta$ = (??/2) X - (?/(3 a??)) X^{3/2} - \Lambda_v,
become negligible compared to the linear X term. In this limit FFGFT reduces to the predictions of General
Relativity with an effective cosmological constant \Lambda_eff set by the residual T0-Vakuum term. Curvature is
dominated by the quasi-linear response of $\theta$, and conventional GR tests are satisfied.
\begin{enumerate}
  \item Weak Field Regime (X ~ a??)
\end{enumerate}
As accelerations approach a?, nonlinear T0-Vakuum effects begin to contribute. Here X is comparable to a??
and the X^{3/2} correction in the Lagrangian becomes relevant. The response function
L_X = ??/2 - (?/(2 a??)) X^{1/2}
departs from a constant and begins to depend on the local phase gradient. Observable consequences
include:
\begin{itemize}
  \item small deviations from Newtonian potential in extended systems,
  \item mild corrections to post-Newtonian parameters,
  \item subtle modifications to gravitational lensing and Shapiro delay.
\end{itemize}
This regime provides a smooth transition between pure GR behavior in strong fields and the deep field
behavior that governs galactic outskirts.
\begin{enumerate}
  \item Deep Field Regime (X << a??, Galactic Scale)
\end{enumerate}
The deep field regime governs low-acceleration environments such as the outskirts of spiral galaxi\footnote{Siehe auch T0-Dokument: \texttt{009_T0_xi_ursprung_De.pdf}}es. In
this limit phase gradients are small, but the nonlinear X^{3/2} term dominates the response of the T0-Vakuum.
Integrating out the amplitude ? and enforcing scale invariance leads to an effective T0-Vakuum energy density
scaling as:
E_vac ? |$\nabla$?|?,
where ? is the gravitational potential related to $\theta$ through the background dynamic T0-Vakuum field. The
resulting field equation in the non-relativistic limit becomes:
$\nabla$ ? [(|$\nabla$?|/a?) $\nabla$?] = 4?G ?_b,
where ?_b is the baryonic matter density. For spherical systems this gives:
g?(r) = a? g_N(r),
with g the true gravitational acceleration and g_N the Newtonian acceleration from baryons alone. This
produces:
\begin{itemize}
  \item flat rotation curves,
  \item the baryonic Tully--Fisher relation v_c? = G M_b a?,
  \item no requirement for dark matter halos.
\end{itemize}
Thus the deep field regime is responsible for MOND-like behavior emerging naturally from FFGFT
T0-Vakuum microphysics.
\begin{enumerate}
  \item Ultra-Deep Cosmological Regime (g << a?, Dark Energy Scale)
\end{enumerate}
On scales comparable to or larger than the Hubble radius, typical gravitational accelerations become far
smaller than a?. In this ultra-deep regime, phase gradients are extremely small and the kinetic contributions
in L_$\theta$ are suppressed relative to the residual T0-Vakuum term. The T0-Vakuum field approaches:
\Phi $\approx$ ?_$\infty$ e^{i $\mu$ t},
with ?_$\infty$ a nearly homogeneous amplitude and $\mu$ the dynamic T0-Vakuum field frequency. The effective
energy density and pressure of the T0-Vakuum become:
?_vac $\approx$ ?_$\infty$? $\mu$? + V(?_$\infty$),
p_vac $\approx$ ?_$\infty$? $\mu$? - V(?_$\infty$),
International Journal for Multidisciplinary Research (IJFMR)
E-ISSN: 2582-2160 $\bullet$ Website: www.ijfmr.com $\bullet$ Email: editor@ijfmr.com
IJFMR250664112 Volume 7, Issue 6, November-December 2025 23
where V(?) is the T0-Vakuum potential. For parameter choices where V(?_$\infty$) dominates over the kinetic term,
one obtains:
p_vac $\approx$ -?_vac,
which corresponds to an equation of state parameter w $\approx$ -1. This is the dark-energy-like regime of FFGFT:
the universe is driven by residual dynamic T0-Vakuum field energy and the nearly constant T0-Vakuum potential.
In this ultra-deep regime:
\begin{itemize}
  \item X $\rightarrow$ 0,
  \item L_X $\rightarrow$ ??/2,
  \item the stress--energy tensor of $\theta$ reduces to an effective cosmological constant term,
  \item the Friedmann equations predict accelerated expansion.
\end{itemize}
Thus, dark energy is not an independent fluid but the asymptotic T0-Vakuum state of \Phi when typical
gravitational gradients fall far below a? on cosmological scales.
\begin{enumerate}
  \item Transitions Across Scales
\end{enumerate}
The three local regimes (strong, weak, deep) and the ultra-deep cosmological regime are not separate
theories; they are different limits of the same underlying dynamics controlled by X and the parameters (??,
?, a?, \Lambda_v). As a characteristic acceleration in a system changes, the T0-Vakuum smoothly interpolates
between:
\begin{itemize}
  \item GR-like behavior in compact objects and Solar System tests,
  \item modified dynamics in galaxies (deep field),
  \item effective dark energy at horizon-scale averages (ultra-deep field).
\end{itemize}
The governing equation
$\nabla$_$\mu$ (L_X $\nabla$^$\mu$ $\theta$) = 0
determines how the phase field adjusts across these regimes. Small, local systems never probe the ultradeep T0-Vakuum; galaxies probe the deep-field regime; the universe as a whole samples the full T0-Vakuum
potential and residual dynamic T0-Vakuum field energy.
\begin{enumerate}
  \item Implications for Cosmology and Structure Formation
\end{enumerate}
Because the same Lagrangian L_$\theta$ governs all regimes, FFGFT ties together:
\begin{itemize}
  \item galactic rotation curves,
  \item cluster dynamics,
  \item cosmic acceleration,
  \item the absence of singularities,
  \item with a single set of T0-Vakuum parameters. Structure formation proceeds in a background where:
  \item early universe: kinetic and potential terms of \Phi drive inflation-like expansion,
  \item intermediate epochs: matter dominates and deep-field corrections shape halo dynamics,
  \item late universe: ultra-deep regime emerges, and dark-energy-like behavior dominates.
\end{itemize}
In contrast to \LambdaCDM, where dark matter and dark energy are independent components, FFGFT describes
both as manifestations of one T0-Vakuum field, viewed in different acceleration regimes.
\begin{enumerate}
  \item Summary
\end{enumerate}
FFGFT organizes gravitational behavior into four coherent regimes:
\begin{itemize}
  \item Strong field: GR limit, X >> a??, linear response, compact objects.
  \item Weak field: transitional, X ~ a??, small nonlinear corrections.
  \item Deep field: galactic scale, X << a?? but gradients still relevant, g? = a? g_N, no dark matter.
\end{itemize}
International Journal for Multidisciplinary Research (IJFMR)
E-ISSN: 2582-2160 $\bullet$ Website: www.ijfmr.com $\bullet$ Email: editor@ijfmr.com
IJFMR250664112 Volume 7, Issue 6, November-December 2025 24
\begin{itemize}
  \item Ultra-deep cosmological field: g << a? on horizon scales, residual T0-Vakuum energy acts as dark energy (w
\end{itemize}
$\approx$ -1).
This regime structure is not an artificial phenomenology; it is the natural consequence of a single dynamic
T0-Vakuum field Lagrangian. As a result, FFGFT provides a unified physical explanation for local gravity tests,
galaxy dynamics, and late-time cosmic acceleration within one coherent framework.

\section{Referenzen zu T0-Dokumenten}

Dieses Kapitel steht in Zusammenhang mit folgenden T0-Dokumenten im Repository \texttt{2/pdf/}:

\begin{itemize}
  \item \texttt{002\_T0\_Grundlagen\_De.pdf} -- T0 Zeit-Masse-Dualit?t Grundlagen
  \item \texttt{004\_T0\_Energie\_De.pdf} -- T0 Energiefeld-Theorie
  \item \texttt{009\_T0\_xi\_ursprung\_De.pdf} -- Ursprung des geometrischen Parameters $\xi$
  \item \texttt{201\_FFGFT-alles\_De.pdf} -- Vollst?ndiges FFGFT-Dokument (Rahmenwerk)
\end{itemize}