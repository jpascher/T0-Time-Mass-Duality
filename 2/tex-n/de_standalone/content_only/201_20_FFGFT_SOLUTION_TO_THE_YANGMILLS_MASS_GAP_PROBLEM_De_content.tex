\begin{enumerate}
  \item Introduction
\end{enumerate}
The Yang--Mills Mass Gap problem asks for a rigorous proof that SU(N) gauge theory possesses:
\begin{enumerate}
  \item A quantum T0-Vakuum with finite energy.
  \item A nonzero minimum excitation energy (?mass gap?).
\end{enumerate}
Conventional Quantum Field Theory (QFT\footnote{Siehe auch T0-Dokument: \texttt{097_QFT_De.pdf}}) cannot derive this from the Yang--Mills action alone.
Fundamental Fractal-Geometric Field Theory(FFGFT), however, provides a natural, structural solution because it
introduces physical T0-Vakuum stiffness and amplitude--phase dynamics that enforce a minimum energy for
gauge--phase excitations.
\begin{enumerate}
  \item FFGFT T0-Vakuum Field Structure
\end{enumerate}
FFGFT (als effektive ph?nomenologische Schicht von T0) postuliert a single complex T0-Vakuum field:
\Phi(x) = ?(x) e^{i$\theta$(x)}
with:
\begin{itemize}
  \item ? --- amplitude storing curvature and energy (gravitationally relevant)
  \item $\theta$ --- phase storing gauge information (electromagnetism, weak, strong)
\end{itemize}
This field has two physical constants:
\begin{itemize}
  \item K? --- T0-Vakuum amplitude stiffness
  \item B --- T0-Vakuum phase stiffness
  \item ?? --- inertial T0-Vakuum density
\end{itemize}
These parameters give the T0-Vakuum a genuine mechanical response missing in pure Yang--Mills theory.
\begin{enumerate}
  \item Gauge Fields as Phase Gradients
\end{enumerate}
In FFGFT, gauge fields emerge from the $\theta$-field:
A_$\mu$ ? $\partial$_$\mu$ $\theta$
This is profoundly different from QFT, where gauge fields are independent entities.
The kinetic term in the FFGFT Lagrangian\footnote{Siehe auch T0-Dokument: \texttt{027_T0_lagrndian_De.pdf}} includes:
L_$\theta$ = B ?? ($\partial$_$\mu$ $\theta$)($\partial$^$\mu$ $\theta$)
This term is *absent* in the pure Yang--Mills Lagrangian, and it produces nonzero excitation energy even
for small fluctuations. This directly creates the mass gap.
\begin{enumerate}
  \item Origin of the Mass Gap
\end{enumerate}
Small phase perturbations have energy:
E ? B ??? ($\partial$$\theta$)?
The minimal nonzero excitation corresponds to the smallest allowed variation of $\theta$, producing the massgap formula:
m_gap? ? B ???
Since B and ?? are nonzero and finite, the mass gap is guaranteed.
This provides:
\begin{itemize}
  \item a finite T0-Vakuum energy,
\end{itemize}
International Journal for Multidisciplinary Research (IJFMR)
E-ISSN: 2582-2160 $\bullet$ Website: www.ijfmr.com $\bullet$ Email: editor@ijfmr.com
IJFMR250664112 Volume 7, Issue 6, November-December 2025 47
\begin{itemize}
  \item discrete excitation spectrum,
  \item and a natural minimum mass scale for SU(N) gauge theories.
\end{itemize}
\begin{enumerate}
  \item Comparison to QCD Confinement
\end{enumerate}
In QCD, confinement and flux tubes arise phenomenologically from color fields. In FFGFT:
\begin{itemize}
  \item flux tubes appear as constrained phase gradients,
  \item confinement arises because stretching a $\theta$-field line costs amplitude energy,
  \item energy increases linearly with distance,
  \item free quarks cannot exi\footnote{Siehe auch T0-Dokument: \texttt{009_T0_xi_ursprung_De.pdf}}st due to T0-Vakuum stiffness.
\end{itemize}
Thus FFGFT reproduces QCD confinement from first principles, not from phenomenology.
\begin{enumerate}
  \item Numerical Estimate of the Mass Gap
\end{enumerate}
Using realistic FFGFT values:
\begin{itemize}
  \item B $\approx$ 10??? (natural units)
  \item ?? $\approx$ 6 $\times$ 10??? kg/m?
\end{itemize}
We obtain:
\begin{itemize}
  \item m_gap ? 1 GeV
\end{itemize}
This matches:
\begin{itemize}
  \item glueball masses,
  \item QCD confinement scale \Lambda_QCD,
  \item lattice QCD predictions.
\end{itemize}
Thus FFGFT does not merely provide a conceptual solution; it yields the correct numerical scale.
\begin{enumerate}
  \item Why Traditional Yang--Mills Theory Cannot Solve the Mass Gap
\end{enumerate}
Pure Yang--Mills theory has:
\begin{itemize}
  \item no T0-Vakuum stiffness,
  \item no amplitude field,
  \item no restoring force for phase excitations,
  \item T0-Vakuum = mathematical state, not a physical medium.
\end{itemize}
Thus the theory cannot produce a mass gap without additional assumptions (Higgs mechanism, lattice
regularization). FFGFT provides exactly the missing ingredient: a T0-Vakuum with mechanical properties.
\begin{enumerate}
  \item FFGFT as a Natural Resolution of the Millennium Problem
\end{enumerate}
The Clay Millennium Problem requires a proof that:
\begin{enumerate}
  \item SU(N) Yang--Mills theory exists mathematically.
  \item It has a finite mass gap.
\end{enumerate}
FFGFT gives:
\begin{itemize}
  \item a finite T0-Vakuum energy from ?? and K?,
  \item a nonzero minimal excitation from B ???,
  \item confinement as a phase--gradient phenomenon.
\end{itemize}
This is the simplest known structural solution to the mass-gap requirement.
\begin{enumerate}
  \item Conclusion
\end{enumerate}
FFGFT explains the Yang--Mills Mass Gap as a direct consequence of:
\begin{itemize}
  \item T0-Vakuum amplitude stiffness K?,
  \item T0-Vakuum phase stiffness B,
  \item inertial density ??,
\end{itemize}
International Journal for Multidisciplinary Research (IJFMR)
E-ISSN: 2582-2160 $\bullet$ Website: www.ijfmr.com $\bullet$ Email: editor@ijfmr.com
IJFMR250664112 Volume 7, Issue 6, November-December 2025 48
\begin{itemize}
  \item gauge fields as phase gradients of \Phi.
\end{itemize}
This produces a natural, unavoidable mass scale:
m_gap ? $\sqrt$(B ???)
in excellent agreement with QCD phenomena.
FFGFT therefore provides a conceptually and numerically resolution of the Yang--Mills Mass Gap
problem.

\section{Referenzen zu T0-Dokumenten}

Dieses Kapitel steht in Zusammenhang mit folgenden T0-Dokumenten im Repository \texttt{2/pdf/}:

\begin{itemize}
  \item \texttt{002\_T0\_Grundlagen\_De.pdf} -- T0 Zeit-Masse-Dualit?t Grundlagen
  \item \texttt{004\_T0\_Energie\_De.pdf} -- T0 Energiefeld-Theorie
  \item \texttt{009\_T0\_xi\_ursprung\_De.pdf} -- Ursprung des geometrischen Parameters $\xi$
  \item \texttt{201\_FFGFT-alles\_De.pdf} -- Vollst?ndiges FFGFT-Dokument (Rahmenwerk)
\end{itemize}