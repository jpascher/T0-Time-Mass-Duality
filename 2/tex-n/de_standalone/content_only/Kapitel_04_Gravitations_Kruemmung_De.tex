CHAPTER 4: GRAVITATIONAL CURVATURE EQUATIONS
1. Introduction
This chapter presents a complete formulation of gravitational curvature using the Dynamic Vacuum Field
Theory (DVFT). Curvature emerges from the interplay between the metric g_{μν} and the vacuum phase
field θ through the DVFT action. The result is a unified set of equations one for the vacuum field θ and
one for the spacetime curvature. GR appears as the high-acceleration limit of DVFT.
2. DVFT Fundamentals
The vacuum is modeled as a dynamic vacuum field described by the complex order parameter:
Φ(x) = ρ(x) e^{iθ(x)}.
The gravitational degrees of freedom include:
\textbullet{} Metric g_{μν}, determining curvature.
\textbullet{} Phase field θ, governing vacuum convergence.
The kinetic invariant is:
X \equiv -g^{μν} \nabla_μθ \nabla_νθ.
The Dynamic vacuum field Curvature Tensor (DVFT) is defined as:
V_{μν} \equiv \nabla_μ\nabla_νθ - (1/4) g_{μν} □θ,
with □θ = g^{αβ} \nabla_α\nabla_βθ.
3. DVFT Action (Pure Gravity + Vacuum + Matter)
The full DVFT action is:
S = \int d^4x \sqrt-g [ (1/(16πG)) R + 𝓛_θ(X, I_1, I_2) + 𝓛_m(g_{μν},ψ_m) ].
Here:
\textbullet{} R is the Ricci scalar (geometry),
\textbullet{} 𝓛_m is matter Lagrangian,
\textbullet{} 𝓛_θ encodes vacuum microphysics:
𝓛_θ = -Λ_v + (ρ_0/2)X - (η/(3a_0^2)) X^{3/2} + α_1 I_1 + α_2 I_2,
with invariants:
I_1 = V_{μν} V^{μν},
I_2 = V_{μ}^{ α} V_{α}^{ β} V_{β}^{ μ}.
4. θ Field Equation (Dynamics)
Varying S with respect to θ gives the DVFT vacuum equation:
\nabla_μ ( 𝓛_X \nabla^μθ ) + α_1 𝓔^{(1)}[θ,g] + α_2 𝓔^{(2)}[θ,g] = 0,
where:
𝓛_X = \partial𝓛_θ/\partialX = ρ_0/2 - (η/(2a_0^2)) X^{1/2}.
This is a nonlinear wave equation for θ. It determines how the vacuum phase converges into matter and
controls weak-field gravity without needing GR.
International Journal for Multidisciplinary Research (IJFMR)
E-ISSN: 2582-2160 \textbullet{} Website: www.ijfmr.com \textbullet{} Email: editor@ijfmr.com
IJFMR250664112 Volume 7, Issue 6, November-December 2025 12
5. Curvature Equation from Metric Variation
Varying S with respect to the metric g_{μν} yields:
G_{μν} = 8πG ( T^{(m)}_{μν} + T^{(θ)}_{μν} ),
where G_{μν} is the Einstein tensor arising from variation of \sqrt-g R.
The vacuum stress-energy T^{(θ)}_{μν} splits into:
1. k-essence (from X):
T^{(θ,kess)}_{μν} = 2 𝓛_X \nabla_μθ \nabla_νθ - g_{μν} 𝓛_θ(kess).
2. DVFT curvature-like part:
T^{(θ,DVFT)}_{μν} = 2α_1 \partialI_1/\partialg^{μν} + 2α_2 \partialI_2/\partialg^{μν} - g_{μν}(α_1 I_1 + α_2 I_2).
Thus, curvature is determined entirely by θ dynamics and matter, not by assuming Einstein’s equation.
6. Pure DVFT Gravitational Equation
Define the total vacuum tensor:
T^{(θ)}_{μν} = T^{(θ,kess)}_{μν} + T^{(θ,DVFT)}_{μν}.
Then the fundamental DVFT gravitational curvature law is:
E_{μν}[θ,g] \equiv (1/(8πG)) G_{μν} - T^{(θ)}_{μν} = T^{(m)}_{μν}.
This replaces Einstein’s equations. GR is recovered when θ’s nonlinearities vanish.
7. GR as a Limiting Case of DVFT
In high-acceleration environments (Solar System, neutron stars):
\textbullet{} X is large \rightarrow 𝓛_X \approx constant.
\textbullet{} DVFT invariants I_1, I_2 are suppressed.
\textbullet{} T^{(θ)}_{μν} \approx -Λ_eff g_{μν}.
Then DVFT Gravitational Equation reduces to:
G_{μν} + Λ_eff g_{μν} \approx 8πG T^{(m)}_{μν},
which is Einstein’s equation with a cosmological constant.
Thus, GR is not fundamental—it's the high-g limit of DVFT.
8. Low-Acceleration Curvature: Pure DVFT Regime
In galaxies (g ~ a_0 or below):
\textbullet{} Nonlinear term X^{3/2} dominates,
\textbullet{} DVFT invariants contribute significantly,
\textbullet{} θ-field deviates strongly from GR predictions.
The curvature now follows pure DVFT dynamics:
G_{μν} \approx 8πG T^{(θ)}_{μν},
leading to flat rotation curves and MOND-like behavior without dark matter. Example of two galaxies
NGC-3198 and Andromeda rotational speed calculation using DVFT has been shown in next chapter.
9. Summary of DVFT-Only Curvature Framework
Using DVFT, gravitational curvature is fully described by:
1. θ-field equation:
\nabla_μ( 𝓛_X \nabla^μθ ) + DVFT terms = 0.
2. Pure DVFT curvature equation:
G_{μν} = 8πG ( T^{(m)}_{μν} + T^{(θ)}_{μν} ).
No Einstein field equations are introduced by hand—GR emerges only as a limiting case. This is a
complete gravitational theory in its own right, derived purely from dynamic vacuum field microphysics.
International Journal for Multidisciplinary Research (IJFMR)
E-ISSN: 2582-2160 \textbullet{} Website: www.ijfmr.com \textbullet{} Email: editor@ijfmr.com
IJFMR250664112 Volume 7, Issue 6, November-December 2025 13


\section*{T0 Theory Integration}
This chapter integrates DVFT concepts with T0 Time-Mass Duality Theory, where the fundamental relation $T(x,t) \cdot m(x,t) = 1$ governs all vacuum field dynamics. The vacuum amplitude $\rho$ is directly related to local time $T$ through $\rho \propto 1/T$.
