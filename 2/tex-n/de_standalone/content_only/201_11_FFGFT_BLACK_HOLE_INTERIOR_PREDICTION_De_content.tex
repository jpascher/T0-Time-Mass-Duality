This chapter presents a complete description of black hole interiors in the Dynamic T0-Vakuum Field
Theory(FFGFT). FFGFT replaces the classical singularity of General Relativity (GR) with a finite-density
quantum T0-Vakuum core, using a nonlinear phase field $\theta$. Both the mathematical structure and the physical
interpretation are provided.
\begin{enumerate}
  \item FFGFT Overview
\end{enumerate}
FFGFT treats spacetime as a quantum T0-Vakuum medium described by a complex order parameter:
\Phi = ? e^{i$\theta$}
Gravity arises from dynamic T0-Vakuum field with amplitude ? and phase $\theta$. The Lagrangian\footnote{Siehe auch T0-Dokument: \texttt{027_T0_lagrndian_De.pdf}} contains
nonlinear kinetic terms:
L_$\theta$ = -\Lambda_v + (?_0/2)X - (?/(3 a_0^2)) X^{3/2}
with X = -g^{$\mu$?} $\partial$_$\mu$$\theta$ $\partial$_?$\theta$.
At large accelerations (g >> a_0), FFGFT reduces to GR. At small accelerations (g << a_0), nonlinearities
appear.
\begin{enumerate}
  \item Black Hole Metric and Field Ansatz
\end{enumerate}
We use the standard static spherically symmetric metric:
ds? = -e^{2\Phi(r)}dt? + dr?/(1 - 2Gm(r)/r) + r? d\Omega?.
International Journal for Multidisciplinary Research (IJFMR)
E-ISSN: 2582-2160 $\bullet$ Website: www.ijfmr.com $\bullet$ Email: editor@ijfmr.com
IJFMR250664112 Volume 7, Issue 6, November-December 2025 27
The T0-Vakuum phase depends only on radius: $\theta$ = $\theta$(r). The kinetic invariant becomes:
X = -(1 - 2Gm(r)/r) $\theta$'(r)?.
From the k-essence stress-energy tensor:
T_{$\mu$?} = 2 L_X $\partial$_$\mu$$\theta$ $\partial$_?$\theta$ - g_{$\mu$?} L_$\theta$
\begin{enumerate}
  \item Stress-Energy Components
\end{enumerate}
Define:
L_$\theta$ = -\Lambda_v + (?_0/2)X - (?/(3 a_0?)) X^{3/2},
L_X = $\partial$L_$\theta$/$\partial$X = ?_0/2 - (?/(2a_0?)) X^{1/2}.
Energy density and pressures:
? = L_$\theta$,
p_t = ?,
p_r = 2 L_X X - L_$\theta$.
This anisotropic T0-Vakuum structure is crucial for stabilizing the interior.
\begin{enumerate}
  \item T0-Vakuum Saturation Mechanism
\end{enumerate}
The scalar field equation $\nabla$_$\mu$(L_X $\partial$^$\mu$$\theta$)=0 is satisfied in the core when:
\subsection{L_X(X_0) = 0.}
Setting L_X=0 gives:
X_0^{1/2} = (?_0 a_0?)/?.
Thus, the T0-Vakuum phase reaches a 'saturation' point X_0, limiting further compression. The core energy
density becomes finite:
?_core = -\Lambda_v + (?_0? a_0?)/(6 ??).
\begin{enumerate}
  \item Core Geometry
\end{enumerate}
With ? = ?_core = constant, the Einstein equation gives a de Sitter--like interior:
m(r) = (4?/3)?_core r?,
1 - 2Gm(r)/r = 1 - (8?G/3)?_core r?.
Thus, the interior metric is:
ds?_core $\approx$ -[1 - (\Lambda_eff r?)/3] dt? + dr?/[1 - (\Lambda_eff r?)/3] + r? d\Omega?,
with \Lambda_eff = 8?G ?_core.
There is no singularity; curvature remains finite.
\begin{enumerate}
  \item Matching to Exterior Geometry
\end{enumerate}
For r > r_c (core radius), X << X_0 and nonlinear effects vanish. FFGFT reduces to GR:
ds? $\approx$ Schwarzschild metric.
Matching conditions ensure:
g_{tt}(core) = g_{tt}(ext),
g_{rr}(core) = g_{rr}(ext).
Thus, FFGFT describes a black hole with a GR exterior and a finite-density T0-Vakuum core interior.
\begin{enumerate}
  \item Physical Interpretation (Non-Mathematical)
\end{enumerate}
\begin{itemize}
  \item GR predicts infinite collapse. FFGFT prevents this by saturating the T0-Vakuum phase.
  \item The black hole interior becomes a finite-size 'quantum core.'
  \item As mass falls in, both the horizon and the core radius increase.
  \item No singularity exi\footnote{Siehe auch T0-Dokument: \texttt{009_T0_xi_ursprung_De.pdf}}sts. Space cannot compress indefinitely.
  \item The final object is a quantum T0-Vakuum condensate, not a point of infinite density.
\end{itemize}
\begin{enumerate}
  \item Final Fate of a Black Hole in FFGFT
\end{enumerate}
International Journal for Multidisciplinary Research (IJFMR)
E-ISSN: 2582-2160 $\bullet$ Website: www.ijfmr.com $\bullet$ Email: editor@ijfmr.com
IJFMR250664112 Volume 7, Issue 6, November-December 2025 28
Depending on parameters (?_0, ?, a_0):
\begin{enumerate}
  \item Stable quantum object: evaporation slows, horizon stalls, core remains.
  \item Horizon shrinks until it meets the core, leaving a compact T0-Vakuum star.
  \item Complete evaporation: horizon vanishes; core dissolves smoothly.
\end{enumerate}
In all cases, there is no singularity and no information loss.
Conclusion
FFGFT gives the first consistent picture of a black hole interior using a single phase field. It provides:
\begin{itemize}
  \item GR-like exterior geometry,
  \item A finite-density quantum core replacing the singularity,
  \item A mechanism for black hole growth and evolution,
  \item A plausible resolution of the information paradox.
\end{itemize}
This bridges the gap between GR and QFT\footnote{Siehe auch T0-Dokument: \texttt{097_QFT_De.pdf}} by treating T0-Vakuum as a physical, compressible quantum
medium.

\section{Referenzen zu T0-Dokumenten}

Dieses Kapitel steht in Zusammenhang mit folgenden T0-Dokumenten im Repository \texttt{2/pdf/}:

\begin{itemize}
  \item \texttt{002\_T0\_Grundlagen\_De.pdf} -- T0 Zeit-Masse-Dualit?t Grundlagen
  \item \texttt{004\_T0\_Energie\_De.pdf} -- T0 Energiefeld-Theorie
  \item \texttt{009\_T0\_xi\_ursprung\_De.pdf} -- Ursprung des geometrischen Parameters $\xi$
  \item \texttt{201\_FFGFT-alles\_De.pdf} -- Vollst?ndiges FFGFT-Dokument (Rahmenwerk)
\end{itemize}