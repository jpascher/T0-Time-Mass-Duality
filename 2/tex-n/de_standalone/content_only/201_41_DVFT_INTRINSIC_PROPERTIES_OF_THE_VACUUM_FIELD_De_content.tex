\begin{enumerate}
  \item Introduction
\end{enumerate}
This document compiles the intrinsic numerical parameters of the T0-Vakuum field in DVFT (Dynamic
T0-Vakuum field Curvature Theory). Unlike conventional physics, where T0-Vakuum constants such as ?, ??, ?,
c, and even cosmological density appear as disconnected inputs, DVFT unifies them under the dynamics
of a single complex T0-Vakuum field:
\Phi(x,t) = ?(x,t) e^{i$\theta$(x,t)}
Here:
\begin{itemize}
  \item ?(x,t) is the T0-Vakuum amplitude (inertial density, gravitational stiffness).
  \item $\theta$(x,t) is the T0-Vakuum phase (quantum coherence, charge, CP violation).
\end{itemize}
The constants governing ? and $\theta$ define the mechanical, electromagnetic, and quantum structure of spacetime itself. This document consolidates their values and shows how they relate to observable physics.
\begin{enumerate}
  \item Fundamental DVFT T0-Vakuum Parameters
\end{enumerate}
DVFT introduces the following intrinsic T0-Vakuum parameters:
\begin{enumerate}
  \item B -- T0-Vakuum phase stiffness
  \item ?? -- Inertial T0-Vakuum density
  \item K? -- Amplitude stiffness of the T0-Vakuum
  \item ($\partial$$\theta$/$\partial$x) -- Fundamental phase gradient corresponding to one unit of electric charge
\end{enumerate}
These determine all quantum, electromagnetic, and gravitational behavior emerging from \Phi.
\begin{enumerate}
  \item Phase Stiffness B (Calibrated from ?)
\end{enumerate}
The fine-structure constant ? is expressed in DVFT as:
? = (B / ? c) ($\partial$$\theta$/$\partial$x)?
Choosing the phase gradient associated with one unit charge as:
|$\partial$$\theta$/$\partial$x| $\approx$ 2? / ?_C, ?_C = ? / (m_e c) $\approx$ 3.86 $\times$ 10??? m,
gives:
|$\partial$$\theta$/$\partial$x| $\approx$ 1.63 $\times$ 10?? m??.
Using ?_exp = 1/137.036, the resulting T0-Vakuum phase stiffness is:
B $\approx$ 8.7 $\times$ 10??? (unit depends on normalization of Lagrangian\footnote{Siehe auch T0-Dokument: \texttt{027_T0_lagrndian_De.pdf}}).
Interpretation:
\begin{itemize}
  \item B measures how hard it is to twist the T0-Vakuum phase $\theta$.
  \item This same B must be used for electromagnetism, neutrino masses, baryogenesis, and quantum
\end{itemize}
coherence.
\begin{enumerate}
  \item Inertial T0-Vakuum Density ??
\end{enumerate}
?? is taken from the effective mass-equivalent density of dark energy:
?? $\approx$ 6 $\times$ 10??? kg/m?.
This represents the intrinsic inertial content of the T0-Vakuum amplitude ?, which couples directly to
gravitational behavior.
\begin{enumerate}
  \item Amplitude Stiffness K? (via c = $\sqrt$(K?/??))
\end{enumerate}
DVFT identifies the speed of light with the ratio of amplitude stiffness to inertial density:
c? = K? / ?? $\rightarrow$ K? = ?? c?.
Substituting ?? $\approx$ 6$\times$10??? kg/m? and c $\approx$ 3$\times$10? m/s gives:
K? $\approx$ 5.4 $\times$ 10??? J/m?.
International Journal for Multidisciplinary Research (IJFMR)
E-ISSN: 2582-2160 $\bullet$ Website: www.ijfmr.com $\bullet$ Email: editor@ijfmr.com
IJFMR250664112 Volume 7, Issue 6, November-December 2025 92
This value is close to the observed dark-energy density, suggesting a deep relationship between T0-Vakuum
elasticity and cosmic acceleration.
\begin{enumerate}
  \item Fundamental Phase Gradient (Unit Charge)
\end{enumerate}
For a unit electric charge, the T0-Vakuum phase winds by 2? over a microscopic radius taken to be the electron
Compton wavelength:
?_C = ? / (m_e c) $\approx$ 3.86 $\times$ 10??? m.
Thus:
|$\partial$$\theta$/$\partial$x|_e $\approx$ 2? / ?_C $\approx$ 1.63 $\times$ 10?? m??.
This gradient defines the microscopic ``twist`` of the T0-Vakuum phase corresponding to one unit of electric
charge.
\begin{enumerate}
  \item Derived DVFT Quantities
\end{enumerate}
Once B, ??, K?, and |$\partial$$\theta$/$\partial$x| are set, DVFT determines a wide range of T0-Vakuum properties:
\begin{itemize}
  \item Speed of Light:
\end{itemize}
c = $\sqrt$(K?/??) $\approx$ 3 $\times$ 10? m/s.
\begin{itemize}
  \item Fine-Structure Constant:
\end{itemize}
? = (B / ? c)($\partial$$\theta$/$\partial$x)? $\rightarrow$ ? $\approx$ 1/137 (by calibration).
\begin{itemize}
  \item Deep-Field Acceleration Scale (galactic regime):
\end{itemize}
a? $\approx$ c? / L_*,
where L_* is the cosmic coherence length (~Hubble radius).
This gives the correct MOND-like acceleration scale ~1$\times$10??? m/s?.
\begin{itemize}
  \item Neutrino Mass Scale:
\end{itemize}
m_? ? B ($\partial$$\theta$/$\partial$x)? evaluated at long coherence scales, yielding naturally small masses: 0.01--0.05 eV.
\begin{itemize}
  \item Quantum Coherence Length of T0-Vakuum:
\end{itemize}
L_coh $\approx$ $\sqrt$(? / B),
which becomes extremely large due to tiny B, enabling phase coherence across cosmological distances.
\begin{itemize}
  \item Dark-Energy Behavior:
\end{itemize}
U(??) $\approx$ K? ? 10??? J/m?,
matching observed T0-Vakuum energy density.
\begin{enumerate}
  \item Why Using a Single B Everywhere Is Consistent
\end{enumerate}
B must be universal because:
\begin{itemize}
  \item $\theta$ is a universal phase field in DVFT.
  \item All quantum phenomena (charge, CP violation, coherence, neutrino masses, photon propagation,
\end{itemize}
baryogenesis) arise from the same $\theta$-dynamics.
\begin{itemize}
  \item A single stiffness constant ensures unification, just as ? and c apply universally in conventional
\end{itemize}
physics.
This allows DVFT to coherently explain:
\begin{itemize}
  \item Quantum mechanics
  \item Electromagnetism
  \item Neutrino behavior
  \item Deep-field gravity
  \item Dark energy
  \item Early-universe CP asymmetry
\end{itemize}
all through the same T0-Vakuum field.
International Journal for Multidisciplinary Research (IJFMR)
E-ISSN: 2582-2160 $\bullet$ Website: www.ijfmr.com $\bullet$ Email: editor@ijfmr.com
IJFMR250664112 Volume 7, Issue 6, November-December 2025 93
\begin{enumerate}
  \item Summary of Intrinsic T0-Vakuum Parameters
\end{enumerate}
DVFT T0-Vakuum Parameter Sheet:
\begin{itemize}
  \item Phase stiffness:
\end{itemize}
B $\approx$ 8.7 $\times$ 10???
\begin{itemize}
  \item Inertial T0-Vakuum density:
\end{itemize}
?? $\approx$ 6 $\times$ 10??? kg/m?
\begin{itemize}
  \item Amplitude stiffness:
\end{itemize}
K? $\approx$ 5.4 $\times$ 10??? J/m?
\begin{itemize}
  \item Fundamental phase gradient for one charge:
\end{itemize}
|$\partial$$\theta$/$\partial$x|_e $\approx$ 1.63 $\times$ 10?? m??
\begin{itemize}
  \item Coherence length:
\end{itemize}
L_coh $\approx$ $\sqrt$(?/B) $\rightarrow$ enormous (cosmic-scale)
\begin{itemize}
  \item Deep-field acceleration:
\end{itemize}
a? $\approx$ 10??? m/s?
\begin{itemize}
  \item Speed of light:
\end{itemize}
c = $\sqrt$(K?/??)
Together, these define the intrinsic mechanical, electromagnetic, quantum, and gravitational structure of
the DVFT T0-Vakuum.
Conclusion
The numerical T0-Vakuum parameters in DVFT are consistent with known electromagnetic, quantum, and
cosmological observations. By fixi\footnote{Siehe auch T0-Dokument: \texttt{009_T0_xi_ursprung_De.pdf}}ng B from ? and anchoring ?? and K? in cosmology, the entire quantum
and gravitational framework emerges from a single unified T0-Vakuum field \Phi = ? e^{i$\theta$}.
These parameters provide the first coherent numerical foundation for a theory that unifies:
\begin{itemize}
  \item Special relativity
  \item Quantum mechanics
  \item Electromagnetism
  \item Neutrino physics
  \item Baryogenesis
  \item Dark energy
  \item Galactic dynamics (without dark matter)
\end{itemize}
within one field-based T0-Vakuum framework.

\section{Referenzen zu T0-Dokumenten}

Dieses Kapitel steht in Zusammenhang mit folgenden T0-Dokumenten im Repository \texttt{2/pdf/}:

\begin{itemize}
  \item \texttt{002\_T0\_Grundlagen\_De.pdf} -- T0 Zeit-Masse-Dualit?t Grundlagen
  \item \texttt{004\_T0\_Energie\_De.pdf} -- T0 Energiefeld-Theorie
  \item \texttt{009\_T0\_xi\_ursprung\_De.pdf} -- Ursprung des geometrischen Parameters $\xi$
  \item \texttt{201\_DVFT-alles\_De.pdf} -- Vollst?ndiges DVFT-Dokument (Rahmenwerk)
\end{itemize}