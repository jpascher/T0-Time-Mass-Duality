CHAPTER 11: BLACK HOLE INTERIOR PREDICTION
This chapter presents a complete description of black hole interiors in the Dynamic Vacuum Field
Theory(DVFT). DVFT replaces the classical singularity of General Relativity (GR) with a finite-density
quantum vacuum core, using a nonlinear phase field θ. Both the mathematical structure and the physical
interpretation are provided.
1. DVFT Overview
DVFT treats spacetime as a quantum vacuum medium described by a complex order parameter:
Φ = ρ e^{iθ}
Gravity arises from dynamic vacuum field with amplitude ρ and phase θ. The Lagrangian contains
nonlinear kinetic terms:
L_θ = -Λ_v + (ρ_0/2)X - (η/(3 a_0^2)) X^{3/2}
with X = -g^{μν} \partial_μθ \partial_νθ.
At large accelerations (g >> a_0), DVFT reduces to GR. At small accelerations (g << a_0), nonlinearities
appear.
2. Black Hole Metric and Field Ansatz
We use the standard static spherically symmetric metric:
ds^2 = -e^{2Φ(r)}dt^2 + dr^2/(1 - 2Gm(r)/r) + r^2 dΩ^2.
International Journal for Multidisciplinary Research (IJFMR)
E-ISSN: 2582-2160 \textbullet{} Website: www.ijfmr.com \textbullet{} Email: editor@ijfmr.com
IJFMR250664112 Volume 7, Issue 6, November-December 2025 27
The vacuum phase depends only on radius: θ = θ(r). The kinetic invariant becomes:
X = -(1 - 2Gm(r)/r) θ'(r)^2.
From the k-essence stress-energy tensor:
T_{μν} = 2 L_X \partial_μθ \partial_νθ - g_{μν} L_θ
3. Stress-Energy Components
Define:
L_θ = -Λ_v + (ρ_0/2)X - (η/(3 a_0^2)) X^{3/2},
L_X = \partialL_θ/\partialX = ρ_0/2 - (η/(2a_0^2)) X^{1/2}.
Energy density and pressures:
ρ = L_θ,
p_t = ρ,
p_r = 2 L_X X - L_θ.
This anisotropic vacuum structure is crucial for stabilizing the interior.
4. Vacuum Saturation Mechanism
The scalar field equation \nabla_μ(L_X \partial^μθ)=0 is satisfied in the core when:
L_X(X_0) = 0.
Setting L_X=0 gives:
X_0^{1/2} = (ρ_0 a_0^2)/η.
Thus, the vacuum phase reaches a 'saturation' point X_0, limiting further compression. The core energy
density becomes finite:
ρ_core = -Λ_v + (ρ_0^3 a_0^4)/(6 η^2).
5. Core Geometry
With ρ = ρ_core = constant, the Einstein equation gives a de Sitter–like interior:
m(r) = (4π/3)ρ_core r^3,
1 - 2Gm(r)/r = 1 - (8πG/3)ρ_core r^2.
Thus, the interior metric is:
ds^2_core \approx -[1 - (Λ_eff r^2)/3] dt^2 + dr^2/[1 - (Λ_eff r^2)/3] + r^2 dΩ^2,
with Λ_eff = 8πG ρ_core.
There is no singularity; curvature remains finite.
6. Matching to Exterior Geometry
For r > r_c (core radius), X << X_0 and nonlinear effects vanish. DVFT reduces to GR:
ds^2 \approx Schwarzschild metric.
Matching conditions ensure:
g_{tt}(core) = g_{tt}(ext),
g_{rr}(core) = g_{rr}(ext).
Thus, DVFT describes a black hole with a GR exterior and a finite-density vacuum core interior.
7. Physical Interpretation (Non-Mathematical)
\textbullet{} GR predicts infinite collapse. DVFT prevents this by saturating the vacuum phase.
\textbullet{} The black hole interior becomes a finite-size 'quantum core.'
\textbullet{} As mass falls in, both the horizon and the core radius increase.
\textbullet{} No singularity exists. Space cannot compress indefinitely.
\textbullet{} The final object is a quantum vacuum condensate, not a point of infinite density.
8. Final Fate of a Black Hole in DVFT
International Journal for Multidisciplinary Research (IJFMR)
E-ISSN: 2582-2160 \textbullet{} Website: www.ijfmr.com \textbullet{} Email: editor@ijfmr.com
IJFMR250664112 Volume 7, Issue 6, November-December 2025 28
Depending on parameters (ρ_0, η, a_0):
1. Stable quantum object: evaporation slows, horizon stalls, core remains.
2. Horizon shrinks until it meets the core, leaving a compact vacuum star.
3. Complete evaporation: horizon vanishes; core dissolves smoothly.
In all cases, there is no singularity and no information loss.
Conclusion
DVFT gives the first consistent picture of a black hole interior using a single phase field. It provides:
\textbullet{} GR-like exterior geometry,
\textbullet{} A finite-density quantum core replacing the singularity,
\textbullet{} A mechanism for black hole growth and evolution,
\textbullet{} A plausible resolution of the information paradox.
This bridges the gap between GR and QFT by treating vacuum as a physical, compressible quantum
medium.


\section*{T0 Theory Integration}
This chapter integrates DVFT concepts with T0 Time-Mass Duality Theory, where the fundamental relation $T(x,t) \cdot m(x,t) = 1$ governs all vacuum field dynamics. The vacuum amplitude $\rho$ is directly related to local time $T$ through $\rho \propto 1/T$.
