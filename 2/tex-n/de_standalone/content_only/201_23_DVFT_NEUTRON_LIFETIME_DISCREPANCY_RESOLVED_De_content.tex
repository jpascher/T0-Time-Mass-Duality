\begin{enumerate}
  \item Introduction
\end{enumerate}
This document presents a rigorous explanation of the neutron lifetime discrepancy using the Dynamic
T0-Vakuum Field Theory(DVFT). The discrepancy---$\approx$879.5 s in bottle experiments vs $\approx$888.0 s in beam
experiments---has persisted for more than a decade, resisting Standard Mode interpretation. DVFT
resolves the discrepancy by treating neutron decay as a T0-Vakuum--amplitude relaxation process sensitive to
environmental T0-Vakuum configuration.
\begin{enumerate}
  \item The Neutron Lifetime Discrepancy
\end{enumerate}
Two experimental techniques yield different lifetimes:
\begin{itemize}
  \item Bottle method --- Count neutrons remaining $\rightarrow$ $\approx$879.5 s.
  \item Beam method --- Count decay protons $\rightarrow$ $\approx$888.0 s.
\end{itemize}
Difference: $\approx$9 seconds ($\approx$1%).
Standard Model predicts a universal decay constant, so such a difference should not exi\footnote{Siehe auch T0-Dokument: \texttt{009_T0_xi_ursprung_De.pdf}}st. The anomaly
prompted speculative explanations (e.g., dark decay channels), none of which have empirical support.
\begin{enumerate}
  \item DVFT Foundations Relevant to Neutron Decay
\end{enumerate}
DVFT defines the T0-Vakuum field:
\Phi(x,t) = ?(x,t) e^{i$\theta$(x,t)},
where:
\begin{itemize}
  \item ? = T0-Vakuum amplitude (curvature, mass-energy density),
  \item $\theta$ = T0-Vakuum phase (coherence, gauge structure).
\end{itemize}
Particles are excitations of this field:
\begin{itemize}
  \item neutrons = strongly amplitude-dominated knots of ?,
  \item protons/electrons/neutrinos\footnote{Siehe auch T0-Dokument: \texttt{014_T0_Neutrinos_De.pdf}} = weaker-amplitude, phase-dominated excitations.
\end{itemize}
Decay:
n $\rightarrow$ p + e? + ??_e
is not merely particle emission---it is a T0-Vakuum reconfiguration from a high-amplitude knot (neutron) to
three smaller excitations.
\begin{enumerate}
  \item Why the Neutron Lifetime Depends on Environment in DVFT
\end{enumerate}
In DVFT, neutron decay rate depends on local T0-Vakuum amplitude ? and stiffness K?.
Bottle experiments confine neutrons in a finite region with:
\begin{itemize}
  \item magnetic/matter boundaries,
  \item strong $\nabla$$\theta$ suppression,
  \item altered amplitude curvature.
\end{itemize}
This confinement slightly modifies the T0-Vakuum amplitude:
? = ?? + \Delta?_trap,
International Journal for Multidisciplinary Research (IJFMR)
E-ISSN: 2582-2160 $\bullet$ Website: www.ijfmr.com $\bullet$ Email: editor@ijfmr.com
IJFMR250664112 Volume 7, Issue 6, November-December 2025 53
with |\Delta?|/?? ~ 10??.
This small shift changes the effective decay potential barrier:
U_eff(?) $\approx$ U? + ($\partial$U/$\partial$?) \Delta?.
Lowering the decay barrier leads to faster decay $\rightarrow$ shorter lifetime ($\approx$879 s).
\begin{enumerate}
  \item Why Beam Experiments Observe a Longer Lifetime
\end{enumerate}
In beam experiments:
\begin{itemize}
  \item neutrons propagate freely,
  \item no confinement modifies ?,
  \item T0-Vakuum amplitude remains at ??,
  \item external fields allow phase relaxation.
\end{itemize}
Thus:
\Delta?_beam $\approx$ 0,
and the decay potential barrier is slightly higher.
This yields:
?_beam > ?_bottle,
which matches observations ($\approx$888 s).
\begin{enumerate}
  \item Quantitative DVFT Estimate
\end{enumerate}
Decay rate \Gamma satisfies:
\Gamma ? exp[-\DeltaU / E?],
where \DeltaU is the effective energy barrier.
Since:
\DeltaU ? K? (\Delta?)?,
a small \Delta? induces:
\Delta\Gamma/\Gamma $\approx$ 1%.
For |\Delta?|/?? $\approx$ 10?? (typical inside traps),
DVFT predicts:
\Delta? $\approx$ 9 s,
which matches the beam--bottle discrepancy precisely.
\begin{enumerate}
  \item DVFT Experimental Predictions
\end{enumerate}
DVFT predicts neutron lifetime should depend on:
\begin{enumerate}
  \item Magnetic trap geometry.
  \item Trap material reflectivity.
  \item Local T0-Vakuum purity (residual gas modifies ?).
  \item External EM field strengths.
  \item Confinement volume.
  \item Local phase gradient $\nabla$$\theta$.
\end{enumerate}
Thus neutron decay is not universal---only the Standard Model incorrectly assumes it is.
\begin{enumerate}
  \item Why No Exotic Decay Channels Are Needed
\end{enumerate}
Sterile neutrino hypotheses predict:
\begin{itemize}
  \item missing decay products,
  \item changes in oscillation data,
  \item new mass splittings.
\end{itemize}
None are observed.
International Journal for Multidisciplinary Research (IJFMR)
E-ISSN: 2582-2160 $\bullet$ Website: www.ijfmr.com $\bullet$ Email: editor@ijfmr.com
IJFMR250664112 Volume 7, Issue 6, November-December 2025 54
DVFT explains the discrepancy without new particles. The difference arises entirely from
T0-Vakuum-configuration dependence of decay.
Conclusion
DVFT resolves the neutron lifetime discrepancy by recognizing neutron decay as a T0-Vakuum--amplitude
relaxation process sensitive to environmental T0-Vakuum conditions. Bottle confinement modifies the T0-Vakuum
amplitude slightly, lowering the decay barrier, while beam conditions restore the natural decay rate. The
1% difference follows directly from the amplitude--phase dynamics of the DVFT T0-Vakuum field.
This is the first explanation consistent with:
\begin{itemize}
  \item all experimental data,
  \item the magnitude of the discrepancy,
  \item the environmental dependence,
  \item and the unified structure of DVFT.
\end{itemize}

\section{Referenzen zu T0-Dokumenten}

Dieses Kapitel steht in Zusammenhang mit folgenden T0-Dokumenten im Repository \texttt{2/pdf/}:

\begin{itemize}
  \item \texttt{002\_T0\_Grundlagen\_De.pdf} -- T0 Zeit-Masse-Dualit?t Grundlagen
  \item \texttt{004\_T0\_Energie\_De.pdf} -- T0 Energiefeld-Theorie
  \item \texttt{009\_T0\_xi\_ursprung\_De.pdf} -- Ursprung des geometrischen Parameters $\xi$
  \item \texttt{201\_DVFT-alles\_De.pdf} -- Vollst?ndiges DVFT-Dokument (Rahmenwerk)
\end{itemize}