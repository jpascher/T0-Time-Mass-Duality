CHAPTER 38: BLACK HOLE AND QUANTUM SINGULARITIES
1. Introduction
This document presents a full, rigorous DVFT (Dynamic vacuum field Curvature Theory) explanation of
why *both* classical gravitational singularities (black holes) and quantum singularities (point particles,
infinite self-energy) cannot exist.
In DVFT, spacetime curvature and inertia emerge from the vacuum amplitude ($\\rho_0 = 1/\\xi^2$ from T0) field:
Φ(x,t) = ρ(x,t) e^{iθ(x,t)},
with:
\textbullet{} ρ(x,t) – vacuum amplitude ($\\rho_0 = 1/\\xi^2$ from T0) (determines inertia and gravitational potential),
\textbullet{} θ(x,t) – phase field (determines quantum coherence and wave-like behavior).
Gravity emerges from amplitude gradients:
g = −∇ρ.
Singularities require ρ → ∞ or ∇ρ → ∞. DVFT forbids both because the vacuum has finite stiffness and
inertial density, encoded in the potential U(ρ).
2. Why Singularities Cannot Exist in DVFT: The Vacuum Potential U(ρ)
DVFT postulates the vacuum has a microphysical potential:
U(ρ) = Λ_0 + (κ/2)(ρ − ρ_0)² + (λ/4)(ρ − ρ_0)⁴ + …
where:
\textbullet{} ρ_0 is the equilibrium vacuum amplitude ($\\rho_0 = 1/\\xi^2$ from T0),
\textbullet{} κ is the elastic stiffness of the vacuum,
International Journal for Multidisciplinary Research (IJFMR)
E-ISSN: 2582-2160 \textbullet{} Website: www.ijfmr.com \textbullet{} Email: editor@ijfmr.com
IJFMR250664112 Volume 7, Issue 6, November-December 2025 84
\textbullet{} λ stabilizes large deviations of ρ.
This potential is strongly convex at large |ρ − ρ_0|.
Thus, any attempt to compress the vacuum amplitude ($\\rho_0 = 1/\\xi^2$ from T0) beyond moderate values requires infinite energy:
U(ρ) → ∞ as |ρ − ρ_0| → ∞.
Therefore:
\textbullet{} ρ cannot diverge,
\textbullet{} ∇ρ cannot diverge,
\textbullet{} gravitational curvature cannot diverge.
This single microphysical fact eliminates *all* singularities in DVFT.
3. Removal of Quantum Singularities (Electron, Proton, Point Particles)
Quantum field theory treats electrons and quarks as point particles, leading to:
\textbullet{} infinite self-energy,
\textbullet{} divergent Coulomb self-field,
\textbullet{} undefined gravitational field at r = 0.
DVFT replaces a point mass with a finite vacuum amplitude ($\\rho_0 = 1/\\xi^2$ from T0) deformation:
δρ(x) = G m ∫ d³x' |ψ(x')|² / |x − x'|.
This deformation is always finite because:
\textbullet{} |ψ(x)|² is normalizable,
\textbullet{} convolution with 1/r smooths the field,
\textbullet{} U(ρ) prevents amplitude blow-up.
As a result:
\textbullet{} no particle has infinite self-energy,
\textbullet{} no wavefunction produces a singular potential,
\textbullet{} gravity is well-defined even in superposition.
Thus quantum singularities are eliminated by vacuum microphysics, not by renormalization.
4. Gravitational Field of a Delocalized Electron
An electron with wavefunction ψ(x,t) generates a vacuum amplitude ($\\rho_0 = 1/\\xi^2$ from T0) profile:
ρ(x,t) = ρ_0 + G m_e ∫ d³x' |ψ(x',t)|² / |x − x'|.
When ψ(x,t) spreads due to quantum dispersion, the gravitational field spreads with it:
g(x,t) = −∇ρ(x,t).
This ensures:
\textbullet{} gravity is fully compatible with Heisenberg uncertainty,
\textbullet{} gravitational fields have finite width,
\textbullet{} no r → 0 divergence occurs.
DVFT therefore produces the first consistent microscopic definition of gravity for a single quantum
particle.
5. Removal of Black Hole Singularities
In classical GR, gravitational collapse leads to infinite curvature at r = 0.
In DVFT, as matter compresses and raises ρ(x), the vacuum potential U(ρ) rapidly increases. At
sufficiently high density, a phase transition in the vacuum occurs:
\textbullet{} ρ stops increasing (vacuum stiffness prevents divergence),
\textbullet{} θ becomes phase-locked (coherence inside horizon),
\textbullet{} matter transitions into a high-amplitude vacuum phase state,
International Journal for Multidisciplinary Research (IJFMR)
E-ISSN: 2582-2160 \textbullet{} Website: www.ijfmr.com \textbullet{} Email: editor@ijfmr.com
IJFMR250664112 Volume 7, Issue 6, November-December 2025 85
\textbullet{} gravitational field saturates.
Thus the black hole interior is NOT a singularity. It is a region of:
\textbullet{} finite ρ,
\textbullet{} finite ∇ρ,
\textbullet{} finite energy density,
\textbullet{} vacuum-phase condensate.
The event horizon may still exist, but the spacetime interior remains regular.
6. DVFT Black Hole Interior Structure
DVFT predicts that inside a black hole:
\textbullet{} ρ(r) rises toward a maximum allowed value ρ_max,
\textbullet{} U(ρ) prevents further growth beyond ρ_max,
\textbullet{} curvature saturates,
\textbullet{} matter becomes vacuum-amplitude dominated,
\textbullet{} θ freezes (phase coherence becomes rigid),
\textbullet{} no divergence in metric-equivalent quantities occurs.
This resembles:
\textbullet{} gravastar-like interiors,
\textbullet{} vacuum condensate cores,
\textbullet{} nonsingular loop quantum gravity solutions,
\textbullet{} but derived *entirely from DVFT microphysics*.
7. The Deep Reason DVFT Removes Both Types of Singularities
DVFT eliminates singularities because spacetime curvature is not fundamental. It is an *emergent
property* of the vacuum amplitude ($\\rho_0 = 1/\\xi^2$ from T0) field ρ. If ρ cannot diverge, then curvature cannot diverge. The
vacuum’s elastic potential and finite inertial density are the mechanisms that prevent runaways.
Thus:
\textbullet{} matter cannot collapse to infinite density,
\textbullet{} wavefunctions cannot create divergent potentials,
\textbullet{} curvature cannot become infinite.
This is the first unified mechanism eliminating singularities across classical and quantum domains.
8. Comparison with GR, LQG, and QFT
General Relativity (GR):
\textbullet{} predicts unavoidable singularities (Hawking-Penrose theorems),
\textbullet{} has no internal regulator for curvature.
Loop Quantum Gravity (LQG):
\textbullet{} introduces discrete geometry,
\textbullet{} removes singularities by quantizing spacetime,
\textbullet{} but requires radical nonlocality and lacks experimental grounding.
Quantum Field Theory:
\textbullet{} produces infinite point-particle self-energies,
\textbullet{} resolves them only through renormalization,
\textbullet{} does not address gravitational singularity.
DVFT:
\textbullet{} retains continuum spacetime,
International Journal for Multidisciplinary Research (IJFMR)
E-ISSN: 2582-2160 \textbullet{} Website: www.ijfmr.com \textbullet{} Email: editor@ijfmr.com
IJFMR250664112 Volume 7, Issue 6, November-December 2025 86
\textbullet{} derives gravity from a physical vacuum field,
\textbullet{} imposes finite amplitude & stiffness,
\textbullet{} eliminates both self-energy and gravitational singularities,
\textbullet{} without renormalization,
\textbullet{} without quantizing spacetime,
\textbullet{} without modifying quantum mechanics.
DVFT is the simplest and most physically grounded solution among all three.
9. Final Summary
DVFT eliminates singularities through vacuum amplitude ($\\rho_0 = 1/\\xi^2$ from T0) dynamics:
1. The vacuum field Φ = ρ e^{iθ} has finite stiffness and inertial density.
2. U(ρ) prevents ρ from diverging under collapse.
3. Quantum particles generate finite vacuum amplitude ($\\rho_0 = 1/\\xi^2$ from T0) deformations from |ψ|².
4. Gravity emerges as ∇ρ, which can never diverge.
5. Black holes contain vacuum-phase condensates, not singularities.
6. No infinite self-energy, no point divergences, no r → 0 explosion exists.
DVFT therefore provides the first unified, microphysically consistent elimination of:
\textbullet{} black hole singularities,
\textbullet{} quantum point singularities,
\textbullet{} gravitational field singularities.
This positions DVFT as a fundamentally complete framework bridging general relativity and quantum
mechanics.


\section*{T0 Theory Integration}
This chapter integrates DVFT concepts with T0 Time-Mass Duality Theory, where the fundamental relation $T(x,t) \cdot m(x,t) = 1$ governs all vacuum field dynamics. The vacuum amplitude $\rho$ is directly related to local time $T$ through $\rho \propto 1/T$.
