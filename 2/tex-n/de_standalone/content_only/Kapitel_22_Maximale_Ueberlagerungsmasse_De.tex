% Kapitel 22: Maximale Masse für Quantenüberlagerung (Angepasst an T0-Theorie)
% Deutsche Version

\section*{Kapitel 22: Maximale Masse für Quantenüberlagerung}
\addcontentsline{toc}{section}{Kapitel 22: Maximale Masse für Quantenüberlagerung}

\subsection*{1. Einführung}

Dieses Kapitel präsentiert die T0-begründete DVFT-Vorhersage für die maximale Masse und Größe von Molekülen oder makroskopischen Objekten, die in Quantenüberlagerung bleiben können.

Diese Frage ist direkt relevant für das MAST-QG-Projekt (Macroscopic Superpositions for Quantum Gravity).

\textbf{T0-Anpassung:} DVFT liefert einen mathematisch präzisen, physikalisch motivierten Grenzwert, bestimmt durch die nichtlineare Antwort des Vakuumphasenfeldes, das aus T0s Zeit-Masse-Dualität $T(x,t) \cdot m(x,t) = 1$ abgeleitet ist. Im Gegensatz zu heuristischen Modellen wie Diòsi–Penrose (DP) entsteht der Grenzwert aus T0s fundamentalem Parameter $\xi = 4/3 \times 10^{-4}$ ohne freie Parameter.

\subsection*{2. T0-DVFT-Mechanismus für Überlagerungsstabilität}

\subsubsection*{2.1 Vakuumfeld aus T0}

T0-begründete DVFT beschreibt das Vakuum als komplexes Feld:
\[
\Phi(x) = \rho(x) e^{i\theta(x)}
\]
mit:
\begin{itemize}
\item $\rho(x) \propto m(x) = 1/T(x)$ — Vakuumamplitude aus T0s Zeit-Masse-Dualität
\item $\theta(x)$ — Vakuumphase aus T0-Knotenrotationen
\end{itemize}

Abgeleitet aus T0s einziger fundamentaler Konstante $\xi = 4/3 \times 10^{-4}$, was ergibt:
\begin{itemize}
\item $\rho_0 = 1/\xi^2 \approx 5,625 \times 10^7$ — Gleichgewichts-Vakuumdichte
\item $B$ — Vakuumphasensteifigkeit $\sim 1/\xi^4$
\end{itemize}

\subsubsection*{2.2 Kohärenzkriterium}

Quantenkohärenz überlebt nur, wenn die zwei Zweige einer Überlagerung erfüllen:
\[
\theta_1(x) \approx \theta_2(x)
\]

Dekohärenz ist nicht zufällig: Sie tritt auf, wenn das Vakuum (abgeleitet aus T0s Zeitfeld) zwei inkompatible Krümmungskonfigurationen nicht mehr aufrechterhalten kann.

Das Kollaps-Kriterium aus T0-DVFT ist:
\[
E_\theta = \int |\nabla\theta_1 - \nabla\theta_2|^2 d^3x \geq B \rho_0
\]
wobei $B = 1/(\xi^4 \lambda_C^3)$ die Vakuumphasensteifigkeit und $\rho_0 = 1/\xi^2$ die Vakuum-Trägheitsdichte ist—beide aus T0s $\xi$ abgeleitet.

\subsection*{3. Kollapsbedingung abgeleitet aus T0}

\subsubsection*{3.1 Phasenkrümmungs-Fehlanpassung durch Massenüberlagerung}

Eine Masse $m$ an zwei Positionen mit Abstand $d$ erzeugt zwei unterschiedliche Krümmungsfelder. In der T0-Theorie variiert das Zeitfeld $T(x)$ nahe Masse $m$ wie:
\[
T(r) \approx T_0 \left(1 - \frac{GM}{c^2 r}\right)
\]

Die Vakuumphase $\theta$ reagiert über:
\[
|\nabla\theta| \approx \frac{Gm}{c^2 r^2}
\]

Die Krümmungsfehlanpassung zwischen den zwei Zweigen skaliert als:
\[
|\Delta\nabla\theta| \approx \frac{Gmd}{c^2 r^3}
\]

und die totale Fehlanpassungsenergie ist ungefähr:
\[
E_\theta \approx \frac{G^2 m^2}{c^4 d}
\]

\subsubsection*{3.2 Maximale Masse für stabile Überlagerung}

Die T0-DVFT-Kollapsbedingung:
\[
E_\theta < B \rho_0 = \frac{1}{\xi^6 \lambda_C^3}
\]
ergibt die maximale Masse:
\[
m_{\max} \approx \sqrt{\frac{B \rho_0 c^4 d}{G^2}} = \sqrt{\frac{c^4 d}{G^2 \xi^6 \lambda_C^3}}
\]

\textbf{Wichtiger Punkt:} Alle Parameter aus T0s $\xi = 4/3 \times 10^{-4}$ abgeleitet — keine freien Parameter.

\subsection*{4. Numerische Schätzungen aus T0-Konstanten}

Unter Verwendung T0-abgeleiteter Konstanten:
\begin{itemize}
\item $B \rho_0 \approx 1/(\xi^6 \lambda_C^3) \sim 10^{-9}$ J/m³
\item $d \approx 10^{-7}$ m (typische MAST-QG-Ziel-Trennung)
\item $\lambda_C = \hbar/(m_0 c) \approx 2,4 \times 10^{-12}$ m (Compton-Wellenlänge)
\end{itemize}

erhalten wir:
\[
m_{\max} \approx 10^7 - 10^8 \text{ amu}
\]

Dies ist die physikalische Obergrenze für stabile Quantenüberlagerung gemäß T0-Theorie.

\subsection*{5. Entsprechende Größengrenze}

Unter Annahme molekularer/organischer Materiedichte von $\sim$1000 kg/m³ ist die Größe entsprechend $m_{\max}$:
\[
R_{\max} \approx \left(\frac{3 m_{\max}}{4\pi\rho}\right)^{1/3} \approx 50-200 \text{ nm}
\]

Somit sagt die T0-Theorie voraus, dass das größte mögliche kohärente Objekt in unserem Universum ungefähr ist:
\begin{itemize}
\item \textbf{Masse:} $10^7-10^8$ amu
\item \textbf{Radius:} 50–200 nm
\item \textbf{Durchmesser:} $\sim$100 nm Skala
\end{itemize}

Darüber hinaus wird die Vakuumphasen-Krümmung (getrieben durch T0s Zeitfeld-Nichtlinearität) instabil, und Kollaps ist sofortig.

\subsection*{6. Vergleich mit anderen Kollapsmodellen}

\begin{center}
\begin{tabular}{|l|l|l|}
\hline
\textbf{Modell} & \textbf{Kollapsmasse} & \textbf{Physikalischer Mechanismus} \\
\hline
Diòsi–Penrose & $\sim 10^9$ amu & Gravitative Selbstenergie (heuristisch) \\
T0-begründete DVFT & $10^7-10^8$ amu & T0-Zeitfeld-Nichtlinearität ($\xi$-abgeleitet) \\
Standard-QM & Keine Grenze & Kein Kollapsmechanismus \\
\hline
\end{tabular}
\end{center}

\subsubsection*{6.1 Diòsi–Penrose}

DP sagt Kollaps um $10^9$ amu voraus unter Verwendung gravitativer Selbstenergie-Argumente.

T0-DVFT sagt früheren Kollaps ($10^7-10^8$ amu) voraus aufgrund nichtlinearer Krümmungsterme in T0s Zeitfeld, die DP nicht einschließt.

\subsubsection*{6.2 Standard-GR + QFT}

Es gibt keine vorhergesagte Obergrenze in der Standardtheorie.

Die T0-Theorie widerspricht dem und liefert einen endlichen, experimentell falsifizierbaren Grenzwert abgeleitet aus $\xi$.

\subsection*{7. Implikationen für MAST-QG und andere Experimente}

T0-begründete DVFT liefert folgende Vorhersagen:
\begin{itemize}
\item Überlagerungen bis $\sim 10^7$ amu sind stabil.
\item Bei $\sim 10^8$ amu beginnt Kollaps aufgrund T0-Zeitfeld-Nichtlinearität.
\item Bei $> 10^8-10^9$ amu ist Überlagerung fundamental unmöglich.
\end{itemize}

Daher:
\begin{itemize}
\item \textbf{Falls MAST-QG Überlagerung bei $10^9-10^{10}$ amu beobachtet} $\rightarrow$ T0-Theorie ist falsifiziert.
\item \textbf{Falls Kollaps im $10^7-10^8$ amu-Fenster auftritt} $\rightarrow$ T0-Theorie ist stark unterstützt.
\end{itemize}

\subsection*{8. Physikalische Interpretation im T0-Kontext}

Die maximale Massengrenze entsteht, weil:
\begin{enumerate}
\item Jede Masse eine Verzerrung in T0s Zeitfeld erzeugt: $\Delta T(r) \propto Gm/(c^2r)$
\item Eine Überlagerung zweier Positionen zwei inkompatible Zeitfelder erzeugt
\item T0s Vakuumphase $\theta(x) \propto \int (1/T) dx$ kann Kohärenz nicht aufrechterhalten, wenn $|\theta_1 - \theta_2|$ Schwelle überschreitet
\item Schwelle gesetzt durch $B\rho_0 = 1/(\xi^6 \lambda_C^3)$ aus T0s fundamentaler Struktur
\item Kollaps ist nicht messungsinduziert sondern strukturell: T0s Zeitfeld kann die Überlagerung nicht unterstützen
\end{enumerate}

\subsection*{9. Experimentelle Tests}

\subsubsection*{9.1 MAST-QG}

Ziel: $10^9-10^{10}$ amu Moleküle.

\textbf{T0-Vorhersage:} Sollte Kollaps beobachten, bevor diese Massenskala erreicht wird.

\subsubsection*{9.2 MAQRO}

Ziel: Nanopartikel $\sim 10^8$ amu.

\textbf{T0-Vorhersage:} Sollte Beginn des Kollapses bei dieser Skala beobachten.

\subsubsection*{9.3 Nanodiamant-Interferometrie}

Aktuell: $\sim 10^6$ amu.

\textbf{T0-Vorhersage:} Noch unter Schwelle—Kohärenz aufrechterhalten.

\subsubsection*{9.4 Schwebende Optomechanik}

Annäherung an $10^7$ amu.

\textbf{T0-Vorhersage:} Sollte beginnen, T0-induzierte Dekohärenzeffekte zu sehen.

\subsection*{10. Vergleich mit T³-Experiment}

Folmans T³-Experiment (Kapitel 21) und die maximale Überlagerungsmasse (dieses Kapitel) sind komplementär:
\begin{itemize}
\item \textbf{T³:} Validiert T0s Phasenakkumulationsmechanismus über $\theta(x,t)$-Dynamik
\item \textbf{Maximale Masse:} Validiert T0s Phasensteifigkeit $B\rho_0$ und Kollaps-Schwelle
\item Beide abgeleitet aus einzelnem Parameter $\xi = 4/3 \times 10^{-4}$
\end{itemize}

\subsection*{11. Schlussfolgerung}

T0-begründete DVFT gibt eine klare, first-principles Obergrenze für Größe und Masse von Quantenüberlagerungen:
\[
m_{\max} \sim 10^7-10^8 \text{ amu} \quad (R_{\max} \sim 100 \text{ nm})
\]

Diese Grenze entsteht aus T0s Zeitfeld-Nichtlinearität kodiert in $\xi = 4/3 \times 10^{-4}$:
\begin{itemize}
\item Kein heuristisches Modell sondern strukturelle Konsequenz von $T(x,t) \cdot m(x,t) = 1$
\item Sagt fundamentalen Grenzwert voraus, testbar in MAST-QG, MAQRO und verwandten Experimenten
\item Falls Experimente $10^8$ amu ohne Kollaps überschreiten $\rightarrow$ T0 falsifiziert
\item Falls Kollaps bei $10^7-10^8$ amu auftritt $\rightarrow$ T0 stark validiert
\end{itemize}

Die maximale Überlagerungsmasse ist eine einzigartige, falsifizierbare Vorhersage der T0-Theorie.
