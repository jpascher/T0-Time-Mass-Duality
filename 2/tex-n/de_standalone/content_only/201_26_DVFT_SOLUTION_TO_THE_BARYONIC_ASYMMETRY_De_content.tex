\begin{enumerate}
  \item Introduction
\end{enumerate}
The observed universe contains far more matter than antimatter, quantified by the baryon-to-photon ratio:
?_B $\approx$ 6 $\times$ 10???.
The Standard Model cannot explain this value. Its allowed sources of baryon number violation and CP
violation are far too small by orders of magnitude.
DVFT (Dynamic T0-Vakuum field Curvature Theory) provides a clean, unified explanation because baryon
number, CP violation, and non-equilibrium dynamics all arise naturally from the structure of the T0-Vakuum
field:
\Phi(x,t) = ?(x,t) e^{i$\theta$(x,t)},
with amplitude ? controlling inertia and gravitational stiffness and phase $\theta$ controlling quantum behavior,
internal symmetries, and charge structure.
\begin{enumerate}
  \item Sakharov Conditions in the DVFT Framework
\end{enumerate}
Any successful theory of baryogenesis must satisfy Sakharov?s three conditions:
\begin{enumerate}
  \item Baryon number violation
  \item C and CP violation
  \item Departure from thermal equilibrium
\end{enumerate}
DVFT satisfies all three using the single T0-Vakuum field \Phi = ? e^{i$\theta$}, without introducing extra fields, new
particles, or arbitrary CP phases. The conditions emerge naturally from the dynamical behavior of the
T0-Vakuum during the early dynamic T0-Vakuum field epoch.
International Journal for Multidisciplinary Research (IJFMR)
E-ISSN: 2582-2160 $\bullet$ Website: www.ijfmr.com $\bullet$ Email: editor@ijfmr.com
IJFMR250664112 Volume 7, Issue 6, November-December 2025 59
\begin{enumerate}
  \item Baryon Number as Topological Winding in DVFT
\end{enumerate}
In DVFT, baryons correspond to localized topological excitations of the T0-Vakuum phase $\theta$:
\begin{itemize}
  \item baryons $\rightarrow$ positive winding number of $\theta$
  \item antibaryons $\rightarrow$ negative winding number of $\theta$
\end{itemize}
Thus baryon number is:
B ? winding number of $\theta$ in internal phase space.
When the T0-Vakuum phase undergoes topological transitions (such as unwinding, knot-decay, or domain
merging),
B can change by integer amounts. This gives:
\begin{itemize}
  \item natural baryon-number violation
  \item no need for sphalerons or beyond-Standard-Model operators
\end{itemize}
Baryon number violation comes directly from the microphysics of $\theta$(x,t).
\begin{enumerate}
  \item CP Violation from T0-Vakuum Phase Dynamics
\end{enumerate}
In DVFT, CP violation is built into the dynamics of $\theta$.
If the T0-Vakuum's phase evolution is not symmetric under:
$\theta$ $\rightarrow$ ?$\theta$ (charge conjugation)
x $\rightarrow$ ?x (parity),
then the T0-Vakuum itself contains a CP-odd bias. This implies:
\begin{itemize}
  \item different energy costs for +B and ?B topological domains
  \item asymmetric decay of baryon vs antibaryon-like structures
  \item a preferred direction for phase unwinding
\end{itemize}
This CP bias is not an arbitrary input (as in the CKM matrix), but emerges from the structure of the
T0-Vakuum-phase Lagrangian\footnote{Siehe auch T0-Dokument: \texttt{027_T0_lagrndian_De.pdf}}. A general T0-Vakuum Lagrangian may include CP-odd terms such as:
L $\supset$ ? ? $\partial$?$\theta$ + ? ? ($\nabla$$\theta$ ? P_odd),
which directly generate CP-violating evolution.
\begin{enumerate}
  \item Non-Equilibrium from DVFT Early Dynamic T0-Vakuum field
\end{enumerate}
The early universe in DVFT undergoes a transition from a highly coherent T0-Vakuum phase (large ?, uniform
$\theta$) to a broken-phase state with rich amplitude and phase structure. This process is rapid and cannot be
adiabatic.
During this epoch:
\begin{itemize}
  \item ? varies rapidly
  \item $\theta$ develops domains and defects
  \item particle masses change dynamically
  \item T0-Vakuum stiffness evolves in time
\end{itemize}
This means the universe is automatically out of thermal equilibrium, satisfying Sakharov's third condition
without requiring an inflation, reheating, or ad hoc transitions.
\begin{enumerate}
  \item DVFT Mechanism of Baryogenesis
\end{enumerate}
The DVFT baryogenesis mechanism proceeds in five stages:
\begin{enumerate}
  \item Early uniform T0-Vakuum: $\theta$ is nearly constant, ? is high.
  \item Dynamic T0-Vakuum field: $\theta$ fractures into domains with different local winding numbers.
  \item CP bias: the dynamics favor survival of domains with +B over those with ?B.
  \item Topological relaxation: as the T0-Vakuum transitions, domain walls collapse, knots unwind, changing
\end{enumerate}
\subsection{B.}
International Journal for Multidisciplinary Research (IJFMR)
E-ISSN: 2582-2160 $\bullet$ Website: www.ijfmr.com $\bullet$ Email: editor@ijfmr.com
IJFMR250664112 Volume 7, Issue 6, November-December 2025 60
\begin{enumerate}
  \item Freezing: once ? stabilizes near ??, baryon-number-changing processes shut off.
\end{enumerate}
Because the CP-odd terms bias the relaxation, the random walk in baryon number becomes biased.
As the universe cools, this generates a net positive baryon asymmetry:
B_final > 0.
\begin{enumerate}
  \item Predicting the Baryon-to-Photon Ratio
\end{enumerate}
To calculate the observed ratio ?_B $\approx$ 6 $\times$ 10???, DVFT requires:
\begin{itemize}
  \item Explicit CP-odd terms in the $\theta$-Lagrangian
  \item T0-Vakuum stiffness parameters A, B, ?, ?
  \item Dynamics of domain-wall collapse rates
  \item Evolution of the dynamic T0-Vakuum field scale
\end{itemize}
The baryon asymmetry emerges from the imbalance in domain decay:
?_B ? (\DeltaE_CP / T_dynamic T0-Vakuum field).
DVFT uniquely provides a physical meaning to \DeltaE_CP as the energy bias between opposite-winding
phase domains. This makes ?_B calculable once the T0-Vakuum potential is fully specified.
\begin{enumerate}
  \item Distinction Between DVFT and Standard Approaches
\end{enumerate}
Standard Model baryogenesis fails because:
\begin{itemize}
  \item Sphaleron transitions are too weak
  \item CKM CP violation is too small
  \item No natural out-of-equilibrium period exi\footnote{Siehe auch T0-Dokument: \texttt{009_T0_xi_ursprung_De.pdf}}sts
\end{itemize}
Leptogenesis works only by adding massive particles whose masses and couplings remain unmeasured.
DVFT differs sharply:
\begin{itemize}
  \item All Sakharov conditions emerge from \Phi = ? e^{i$\theta$}.
  \item Baryon number is topological, not accidental.
  \item CP violation arises from dynamics, not arbitrary phases.
  \item Non-equilibrium is inherent to early dynamic T0-Vakuum field.
  \item No new fields or heavy particles are needed.
\end{itemize}
This produces a conceptually clean and physically transparent framework for baryogenesis.
\begin{enumerate}
  \item Observational Consequences and Tests
\end{enumerate}
DVFT predicts:
\begin{enumerate}
  \item Residual T0-Vakuum-phase textures may survive as cosmological signatures.
  \item Gravitational waves from domain-wall collapse in the early universe.
  \item A specific scale for CP-odd T0-Vakuum terms, constrained by ?_B.
  \item Possible correlations between baryogenesis parameters and dark energy scale.
  \item A unified explanation of matter genesis and gravitational T0-Vakuum structure.
\end{enumerate}
These predictions allow DVFT to be tested against cosmology, gravitational wave astronomy, and
laboratory searches for CP violation.
Conclusion
DVFT provides a natural, unified, and physically grounded solution to baryonic asymmetry:
\begin{itemize}
  \item baryon number as topological phase winding
  \item CP violation from intrinsic T0-Vakuum phase bias
  \item non-equilibrium from early dynamic T0-Vakuum field dynamics
  \item net baryon asymmetry from biased topological relaxation
\end{itemize}
International Journal for Multidisciplinary Research (IJFMR)
E-ISSN: 2582-2160 $\bullet$ Website: www.ijfmr.com $\bullet$ Email: editor@ijfmr.com
IJFMR250664112 Volume 7, Issue 6, November-December 2025 61
What the Standard Model inserts artificially, DVFT derives inevitably.
DVFT therefore offers one of the cleanest and most compelling paths toward a complete theory of
baryogenesis and the origin of matter in the universe.
International Journal for Multidisciplinary Research (IJFMR)
E-ISSN: 2582-2160 $\bullet$ Website: www.ijfmr.com $\bullet$ Email: editor@ijfmr.com
IJFMR250664112 Volume 7, Issue 6, November-December 2025 62

\section{Referenzen zu T0-Dokumenten}

Dieses Kapitel steht in Zusammenhang mit folgenden T0-Dokumenten im Repository \texttt{2/pdf/}:

\begin{itemize}
  \item \texttt{002\_T0\_Grundlagen\_De.pdf} -- T0 Zeit-Masse-Dualit?t Grundlagen
  \item \texttt{004\_T0\_Energie\_De.pdf} -- T0 Energiefeld-Theorie
  \item \texttt{009\_T0\_xi\_ursprung\_De.pdf} -- Ursprung des geometrischen Parameters $\xi$
  \item \texttt{201\_DVFT-alles\_De.pdf} -- Vollst?ndiges DVFT-Dokument (Rahmenwerk)
\end{itemize}