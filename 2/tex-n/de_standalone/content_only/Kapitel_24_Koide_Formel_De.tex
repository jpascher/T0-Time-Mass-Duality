CHAPTER 25: SOLUTION TO THE NEUTRINO MASS PROBLEM
1. Introduction
This document presents the DVFT (Dynamic vacuum field Curvature Theory) resolution of the neutrino
mass problem — one of the deepest gaps left unsolved by the Standard Model (SM).
In the SM:
\textbullet{} neutrinos were originally predicted to be massless,
\textbullet{} oscillations require nonzero masses,
\textbullet{} no mechanism exists for the tiny scale of neutrino masses,
\textbullet{} no explanation exists for why there are exactly three neutrinos,
\textbullet{} Majorana vs Dirac nature is unspecified,
\textbullet{} PMNS mixing is arbitrary.
DVFT resolves all of these by deriving neutrino masses, mixing, and structure from the physical vacuum
field:
Φ(x,t) = ρ(x,t) e^{iθ(x,t)},
with ρ determining inertia & gravity, and θ determining quantum structure & coherence.
2. Why Neutrinos Must Have Mass in DVFT
In DVFT, all particle masses arise from vacuum phase displacement:
m_i = K (1 − cos θ_i),
where θ_i is a stable vacuum phase eigenmode.
If neutrinos have oscillation frequencies, they must correspond to distinct θ-values:
θ_{ν_e} ≠ θ_{ν_μ} ≠ θ_{ν_τ}.
Thus neutrinos cannot be massless. DVFT therefore predicts neutrino masses as a *necessary
consequence* of vacuum phase physics, not as an added assumption.
3. Why Neutrino Masses Are Extremely Small
Charged leptons deform both ρ and θ, but neutrinos correspond to *pure phase-only modes*.
Thus:
\textbullet{} their deformation of vacuum amplitude ($\\rho_0 = 1/\\xi^2$ from T0) ρ(x) is extremely small,
\textbullet{} their energy cost comes primarily from phase oscillation,
\textbullet{} their effective stiffness K_ν is much smaller than for charged leptons.
This produces natural mass suppression:
m_ν ≪ m_e, m_μ, m_τ.
International Journal for Multidisciplinary Research (IJFMR)
E-ISSN: 2582-2160 \textbullet{} Website: www.ijfmr.com \textbullet{} Email: editor@ijfmr.com
IJFMR250664112 Volume 7, Issue 6, November-December 2025 57
No seesaw mechanism is required — neutrino lightness results directly from the structure of the vacuum
fields.
4. Why Exactly Three Neutrinos Exist
The nonlinear vacuum potential:
U(ρ) = κ(ρ − ρ_0)² + λ(ρ − ρ_0)⁴ + …
supports exactly three stable oscillation modes with 120° vacuum phase separation:
θ_{ν_e} = θ_0
θ_{ν_μ} = θ_0 + 2π/3
θ_{ν_τ} = θ_0 + 4π/3.
Thus:
\textbullet{} three leptons,
\textbullet{} three neutrinos,
\textbullet{} three quark families,
all originate from the same vacuum-phase triplet structure. This is a fully predictive explanation absent in
the SM.
5. DVFT Mass Formula for Neutrinos
Given the phase-mode structure, neutrino masses arise from:
m_{ν_i} = K_ν (1 − cos θ_{ν_i}),
with K_ν ≪ K_e.
If θ_i are separated by 2π/3 but slightly perturbed by small vacuum distortions δ_i:
θ_{ν_i} = θ_0 + 2πi/3 + δ_i,
DVFT produces:
\textbullet{} nearly degenerate masses,
\textbullet{} small differences Δm²,
\textbullet{} stable oscillation modes.
This matches the observed structure of solar and atmospheric neutrino oscillations.
6. DVFT Explanation of Neutrino Mixing (PMNS Matrix)
In DVFT, mixing arises from phase-coupling among vacuum modes. The mixing matrix elements are
overlap integrals between phase eigenstates:
U_{ij} ∝ ⟨ θ_i | θ_j ⟩.
Because neutrinos are phase-only modes, their coupling angles are large, producing:
\textbullet{} large θ_1_2 (solar angle),
\textbullet{} large θ_2_3 (atmospheric angle),
\textbullet{} nonzero θ_1_3 (reactor angle).
The PMNS matrix is therefore a natural consequence of vacuum phase geometry, not an arbitrary 3×3
parameterization as in the SM.
7. Majorana vs Dirac Nature in DVFT
In DVFT:
\textbullet{} charged leptons have amplitude-phase excitations → Dirac-like,
\textbullet{} neutrinos have pure phase oscillations → naturally Majorana-like.
Thus DVFT predicts neutrinos to be effectively Majorana particles, arising from self-conjugate phase
oscillations of θ(x,t).
8. DVFT Prediction of the Absolute Neutrino Mass Scale
International Journal for Multidisciplinary Research (IJFMR)
E-ISSN: 2582-2160 \textbullet{} Website: www.ijfmr.com \textbullet{} Email: editor@ijfmr.com
IJFMR250664112 Volume 7, Issue 6, November-December 2025 58
DVFT connects neutrino masses to vacuum stiffness parameters (Aρ, κ, λ). The mass scale is:
m_ν ≈ √(A_ρ) / 10⁶,
giving:
m_ν ≈ 0.01 – 0.05 eV,
matching cosmological and oscillation bounds. This is a direct prediction — not an input parameter as in
the Standard Model.
9. Koide-like Relations for Neutrinos
DVFT predicts perturbed Koide-like mass relations due to small deviations δ_i in θ:
θ_{ν_i} = θ_0 + 2πi/3 + δ_i.
This produces the characteristic neutrino mass hierarchy and mixing structure. SM cannot predict such
relations; DVFT does through vacuum geometry.
10. Summary of DVFT Solutions to the Neutrino Problem
DVFT provides the most complete and natural explanation of neutrino physics to date:
\textbullet{} Neutrinos must have mass (phase eigenvalue separation).
\textbullet{} Masses are extremely small (pure-phase excitations).
\textbullet{} Exactly three neutrinos exist (triplet vacuum-phase structure).
\textbullet{} PMNS mixing arises from vacuum phase-mode coupling.
\textbullet{} Neutrinos are Majorana-like (phase-only oscillations).
\textbullet{} The mass scale (0.01–0.05 eV) emerges from vacuum stiffness.
\textbullet{} Koide-like relations for neutrinos follow from perturbed phase geometry.
DVFT resolves every major unanswered feature of neutrinos in a unified way, completing what the
Standard Model leaves unexplained.


\section*{T0 Theory Integration}
This chapter integrates DVFT concepts with T0 Time-Mass Duality Theory, where the fundamental relation $T(x,t) \cdot m(x,t) = 1$ governs all vacuum field dynamics. The vacuum amplitude $\rho$ is directly related to local time $T$ through $\rho \propto 1/T$.
