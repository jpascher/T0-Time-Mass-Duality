CHAPTER 37: INTRINSIC PROPERTIES OF THE VACUUM FIELD
1. Introduction
This document compiles the intrinsic numerical parameters of the vacuum field in DVFT (Dynamic
vacuum field Curvature Theory). Unlike conventional physics, where vacuum constants such as α, ε_0, ħ,
c, and even cosmological density appear as disconnected inputs, DVFT unifies them under the dynamics
of a single complex vacuum field:
Φ(x,t) = ρ(x,t) e^{iθ(x,t)}
Here:
\textbullet{} ρ(x,t) is the vacuum amplitude ($\\rho_0 = 1/\\xi^2$ from T0) (inertial density, gravitational stiffness).
\textbullet{} θ(x,t) is the vacuum phase (quantum coherence, charge, CP violation).
The constants governing ρ and θ define the mechanical, electromagnetic, and quantum structure of spacetime itself. This document consolidates their values and shows how they relate to observable physics.
2. Fundamental DVFT Vacuum Parameters
DVFT introduces the following intrinsic vacuum parameters:
1. B – Vacuum phase stiffness
2. ρ_0 – Inertial vacuum density
3. K_0 – Amplitude stiffness of the vacuum
4. (\partialθ/\partialx) – Fundamental phase gradient corresponding to one unit of electric charge
These determine all quantum, electromagnetic, and gravitational behavior emerging from Φ.
3. Phase Stiffness B (Calibrated from α)
The fine-structure constant α is expressed in DVFT as:
α = (B / ħ c) (\partialθ/\partialx)^2
Choosing the phase gradient associated with one unit charge as:
|\partialθ/\partialx| \approx 2π / λ_C, λ_C = ħ / (m_e c) \approx 3.86 \times 10⁻^1^3 m,
gives: |\partialθ/\partialx| \approx 1.63 \times 10^1^3 m⁻^1.
Using α_exp = 1/137.036, the resulting vacuum phase stiffness is:
B \approx 8.7 \times 10⁻^5^5 (unit depends on normalization of Lagrangian).
Interpretation:
\textbullet{} B measures how hard it is to twist the vacuum phase θ.
\textbullet{} This same B must be used for electromagnetism, neutrino masses, baryogenesis, and quantum
coherence.
4. Inertial Vacuum Density ρ_0
ρ_0 is taken from the effective mass-equivalent density of dark energy:
ρ_0 \approx 6 \times 10⁻^2^7 kg/m^3.
This represents the intrinsic inertial content of the vacuum amplitude ($\\rho_0 = 1/\\xi^2$ from T0) ρ, which couples directly to
gravitational behavior.
5. Amplitude Stiffness K_0 (via c = \sqrt(K_0/ρ_0))
DVFT identifies the speed of light with the ratio of amplitude stiffness to inertial density:
c^2 = K_0 / ρ_0 \rightarrow K_0 = ρ_0 c^2.
Substituting ρ_0 \approx 6\times10⁻^2^7 kg/m^3 and c \approx 3\times10^8 m/s gives:
K_0 \approx 5.4 \times 10⁻^1^0 J/m^3.
This value is close to the observed dark-energy density, suggesting a deep relationship between vacuum
elasticity and cosmic acceleration.
International Journal for Multidisciplinary Research (IJFMR)
E-ISSN: 2582-2160 \textbullet{} Website: www.ijfmr.com \textbullet{} Email: editor@ijfmr.com
IJFMR250664112 Volume 7, Issue 6, November-December 2025 82
5. Fundamental Phase Gradient (Unit Charge)
For a unit electric charge, the vacuum phase winds by 2π over a microscopic radius taken to be the electron
Compton wavelength:
λ_C = ħ / (m_e c) \approx 3.86 \times 10⁻^1^3 m.
Thus:
|\partialθ/\partialx|_e \approx 2π / λ_C \approx 1.63 \times 10^1^3 m⁻^1.
This gradient defines the microscopic "twist" of the vacuum phase corresponding to one unit of electric
charge.
7. Derived DVFT Quantities
Once B, ρ_0, K_0, and |\partialθ/\partialx| are set, DVFT determines a wide range of vacuum properties:
1. Speed of Light:
c = \sqrt(K_0/ρ_0) \approx 3 \times 10^8 m/s.
1. Fine-Structure Constant:
α = (B / ħ c)(\partialθ/\partialx)^2 \rightarrow α \approx 1/137 (by calibration).
2. Deep-Field Acceleration Scale (galactic regime):
a_0 \approx c^2 / L_*,
where L_* is the cosmic coherence length (~Hubble radius).
This gives the correct MOND-like acceleration scale ~1\times10⁻^1^0 m/s^2.
3. Neutrino Mass Scale:
m_ν \propto B (\partialθ/\partialx)^2 evaluated at long coherence scales,
yielding naturally small masses: 0.01–0.05 eV.
4. Quantum Coherence Length of Vacuum:
L_coh \approx \sqrt(ħ / B),
which becomes extremely large due to tiny B, enabling phase coherence across cosmological distances.
5. Dark-Energy Behavior:
U(ρ_0) \approx K_0 ∼ 10⁻^1^0 J/m^3,
matching observed vacuum energy density.
8. Why Using a Single B Everywhere Is Consistent
B must be universal because:
\textbullet{} θ is a universal phase field in DVFT.
\textbullet{} All quantum phenomena (charge, CP violation, coherence, neutrino masses, photon propagation,
baryogenesis) arise from the same θ-dynamics.
\textbullet{} A single stiffness constant ensures unification, just as ħ and c apply universally in conventional
physics.
This allows DVFT to coherently explain:
\textbullet{} Quantum mechanics
\textbullet{} Electromagnetism
\textbullet{} Neutrino behavior
\textbullet{} Deep-field gravity
\textbullet{} Dark energy
\textbullet{} Early-universe CP asymmetry
all through the same vacuum field.
9. Summary of Intrinsic Vacuum Parameters
International Journal for Multidisciplinary Research (IJFMR)
E-ISSN: 2582-2160 \textbullet{} Website: www.ijfmr.com \textbullet{} Email: editor@ijfmr.com
IJFMR250664112 Volume 7, Issue 6, November-December 2025 83
DVFT Vacuum Parameter Sheet:
\textbullet{} Phase stiffness: B \approx 8.7 \times 10⁻^5^5
\textbullet{} Inertial vacuum density: ρ_0 \approx 6 \times 10⁻^2^7 kg/m^3
\textbullet{} Amplitude stiffness: K_0 \approx 5.4 \times 10⁻^1^0 J/m^3
\textbullet{} Fundamental phase gradient for one charge: |\partialθ/\partialx|_e \approx 1.63 \times 10^1^3 m⁻^1
\textbullet{} Coherence length: L_coh \approx \sqrt(ħ/B) \rightarrow enormous (cosmic-scale)
\textbullet{} Deep-field acceleration: a_0 \approx 10⁻^1^0 m/s^2
\textbullet{} Speed of light: c = \sqrt(K_0/ρ_0)
Together, these define the intrinsic mechanical, electromagnetic, quantum, and gravitational structure of
the DVFT vacuum.
Conclusion
The numerical vacuum parameters in DVFT are consistent with known electromagnetic, quantum, and
cosmological observations. By fixing B from α and anchoring ρ_0 and K_0 in cosmology, the entire quantum
and gravitational framework emerges from a single unified vacuum field Φ = ρ e^{iθ}.
These parameters provide the first coherent numerical foundation for a theory that unifies:
\textbullet{} Special relativity
\textbullet{} Quantum mechanics
\textbullet{} Electromagnetism
\textbullet{} Neutrino physics
\textbullet{} Baryogenesis
\textbullet{} Dark energy
\textbullet{} Galactic dynamics (without dark matter) within one field-based vacuum framework.


\section*{T0 Theory Integration}
This chapter integrates DVFT concepts with T0 Time-Mass Duality Theory, where the fundamental relation $T(x,t) \cdot m(x,t) = 1$ governs all vacuum field dynamics. The vacuum amplitude $\rho$ is directly related to local time $T$ through $\rho \propto 1/T$.
