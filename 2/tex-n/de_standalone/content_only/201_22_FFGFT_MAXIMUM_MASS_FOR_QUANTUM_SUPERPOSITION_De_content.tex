\begin{enumerate}
  \item Introduction
\end{enumerate}
This document presents the Fundamental Fractal-Geometric Field Theory(FFGFT) prediction for the maxi\footnote{Siehe auch T0-Dokument: \texttt{009_T0_xi_ursprung_De.pdf}}mum mass
and size of molecules or macroscopic objects that can remain in quantum superposition.
This question is directly relevant to the MAST-QG (Macroscopic Superpositions for Quantum Gravity)
project.
FFGFT provides a mathematically precise, physically motivated cutoff determined by the nonlinear
response of the T0-Vakuum-phase field, unlike heuristic or empirical models such as the Di?si--Penrose (DP)
model.
Here we derive this limit and provide experimentally testable values.
\begin{enumerate}
  \item FFGFT Mechanism for Superposition Stability
\end{enumerate}
FFGFT describes the T0-Vakuum as a complex field:
\Phi(x) = ?(x) e^{i$\theta$(x)}
with:
\begin{itemize}
  \item ?(x): T0-Vakuum amplitude (inertial content, related to mass),
  \item $\theta$(x): T0-Vakuum phase (curvature field, source of gravity).
\end{itemize}
Quantum coherence survives only when the two branches of a superposition satisfy:
$\theta$?(x) $\approx$ $\theta$?(x).
Decoherence is not random: it occurs when the T0-Vakuum can no longer sustain two incompatible curvature
configurations.
The collapse criterion is:
E_$\theta$ = $\int$ |$\nabla$$\theta$? - $\nabla$$\theta$?|? d?x $\geq$ B ??,
where B is the T0-Vakuum phase stiffness and ?? is the T0-Vakuum inertial density.
This gives a physically sharp limit on superposition-scale objects.
International Journal for Multidisciplinary Research (IJFMR)
E-ISSN: 2582-2160 $\bullet$ Website: www.ijfmr.com $\bullet$ Email: editor@ijfmr.com
IJFMR250664112 Volume 7, Issue 6, November-December 2025 51
\begin{enumerate}
  \item Collapse Condition Derived from FFGFT
\end{enumerate}
3.1 Phase Curvature Mismatch from Mass Superposition
A mass m in two positions separated by distance d produces two distinct curvature fields based on the
weak-field approximation:
|$\nabla$$\theta$| $\approx$ G m / (c? r?).
The curvature mismatch between the two branches scales as:
|\Delta$\nabla$$\theta$| $\approx$ G m d / (c? r?),
and the total mismatch energy is approximately:
E_$\theta$ $\approx$ (G? m? / c?)(1/d).
3.2 Maximum Mass for Stable Superposition
The FFGFT collapse condition:
E_$\theta$ < B ??
yields the maximum mass:
m_max $\approx$ $\sqrt$( B ?? c? d / G? ).
\begin{enumerate}
  \item Numerical Estimates from FFGFT Constants
\end{enumerate}
Using conservative FFGFT constants:
B ?? $\approx$ 10?? J/m?
d $\approx$ 10?? m (typical MAST-QG target separation)
we obtain:
m_max $\approx$ 10? -- 10? amu.
This is the physical upper bound for stable quantum superposition.
\begin{enumerate}
  \item Corresponding Size Limit
\end{enumerate}
Assuming molecular/organic matter density of ~1000 kg/m?, the size corresponding to m_max is:
R_max $\approx$ (3 m_max / 4??)^{1/3}
$\approx$ 50 -- 200 nm.
Thus FFGFT predicts the largest possible coherent object in our universe is approximately:
\begin{itemize}
  \item mass: 10?--10? amu
  \item radius: 50--200 nm
  \item diameter: ~100 nm scale
\end{itemize}
Beyond this, T0-Vakuum-phase curvature becomes nonlinear, and collapse is immediate.
\begin{enumerate}
  \item Comparison with Other Collapse Models
\end{enumerate}
6.1 Di?si--Penrose
DP predicts collapse around 10? amu.
FFGFT predicts earlier collapse (10?--10? amu) due to nonlinear curvature terms.
6.2 Standard GR + QFT\footnote{Siehe auch T0-Dokument: \texttt{097_QFT_De.pdf}}
There is no predicted upper limit in standard theory.
FFGFT contradicts this and provides a finite, experimentally falsifiable cutoff.
\begin{enumerate}
  \item Implications for MAST-QG and Other Experiments
\end{enumerate}
FFGFT provides the following predictions:
\begin{itemize}
  \item Superpositions up to ~10? amu are stable.
  \item At ~10? amu, collapse begins.
  \item At >10?--10? amu, superposition is fundamentally impossible.
\end{itemize}
Therefore:
International Journal for Multidisciplinary Research (IJFMR)
E-ISSN: 2582-2160 $\bullet$ Website: www.ijfmr.com $\bullet$ Email: editor@ijfmr.com
IJFMR250664112 Volume 7, Issue 6, November-December 2025 52
\begin{itemize}
  \item If MAST-QG observes superposition at 10?--10?? amu $\rightarrow$ FFGFT is falsified.
  \item If collapse occurs in this window $\rightarrow$ FFGFT is strongly supported.
\end{itemize}
Conclusion
FFGFT gives a clear, first-principles upper bound on the size and mass of quantum superpositions.
This predicts a fundamental cutoff around 10?--10? amu (100 nm scale).
This limit is directly testable in upcoming macroscopic quantum experiments such as MAST-QG,
MAQRO, nanodiamond interferometry, and levitated optomechanics.

\section{Referenzen zu T0-Dokumenten}

Dieses Kapitel steht in Zusammenhang mit folgenden T0-Dokumenten im Repository \texttt{2/pdf/}:

\begin{itemize}
  \item \texttt{002\_T0\_Grundlagen\_De.pdf} -- T0 Zeit-Masse-Dualit?t Grundlagen
  \item \texttt{004\_T0\_Energie\_De.pdf} -- T0 Energiefeld-Theorie
  \item \texttt{009\_T0\_xi\_ursprung\_De.pdf} -- Ursprung des geometrischen Parameters $\xi$
  \item \texttt{201\_FFGFT-alles\_De.pdf} -- Vollst?ndiges FFGFT-Dokument (Rahmenwerk)
\end{itemize}