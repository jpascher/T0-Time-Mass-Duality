\begin{enumerate}
  \item Introduction
\end{enumerate}
In Dynamic T0-Vakuum field--Curvature Theory (DVFT), physical space exi\footnote{Siehe auch T0-Dokument: \texttt{009_T0_xi_ursprung_De.pdf}}sts only where the T0-Vakuum
amplitude ρ(x,t) is nonzero. Regions with ρ $\approx$ 0 correspond to the primordial pure-phase (pre-space), which
has no geometry, no time, and no light-speed. When the universe ignited, ρ transitioned from 0 → ρ₀,
creating the domain in which spacetime, matter, and physics could exist.
The radius of this activated domain is the true ‘cosmic boundary,’ and its growth defines the ‘speed of
space creation,’ given by the amplitude-front velocity:
v_b(t) = dR(t)/dt.
This appendix derives v_b(t) from DVFT field equations and shows how it yields observational scales
such as the $\approx$46.5 Gly cosmic horizon.
\begin{enumerate}
  \item Fundamental DVFT Amplitude Equation
\end{enumerate}
The DVFT T0-Vakuum field is:
Φ(x,t) = ρ(x,t) e^{i$\theta$(x,t)}.
The amplitude ρ satisfies the Lagrangian\footnote{Siehe auch T0-Dokument: \texttt{027_T0_lagrndian_De.pdf}}:
𝓛_ρ = ½ A ($\partial$ₜρ)² − ½ B ($\nabla$ρ)² − U(ρ),
leading to the Euler--Lagrange equation:
A $\partial$ₜ²ρ − B $\nabla$²ρ + U'(ρ) = 0.
International Journal for Multidisciplinary Research (IJFMR)
E-ISSN: 2582-2160 $\bullet$ Website: www.ijfmr.com $\bullet$ Email: editor@ijfmr.com
IJFMR250664112 Volume 7, Issue 6, November-December 2025 34
This is a local, second-order, hyperbolic partial differential equation. Therefore, all disturbances or fronts
in ρ propagate with finite characteristic speed. This is the fundamental reason DVFT forbids infinite
‘space-creation speed.’
\begin{enumerate}
  \item Definition of the Space--Nonspace Boundary
\end{enumerate}
In DVFT:
\begin{itemize}
  \item Space exists where ρ(x,t) > 0.
  \item Pre-space (non-space) exists where ρ(x,t) = 0.
\end{itemize}
The boundary R(t) is defined implicitly by:
ρ(R(t), t) = ρ_crit $\approx$ 0.
The speed of ‘space creation’ is:
v_b(t) = dR(t)/dt.
It measures how fast the amplitude front propagates into the primordial pure-phase region.
\begin{enumerate}
  \item Planar Traveling-Front Derivation of Finite Boundary Speed
\end{enumerate}
Consider a planar front:
ρ(x,t) = f(ξ), ξ = x − v_b t.
Insert into the amplitude equation:
A v_b² f''(ξ) − B f''(ξ) + U'(f(ξ)) = 0.
Multiply by f'(ξ) and integrate:
(A v_b² − B) ½ f'^2 + U(f) = C.
Assuming U(0) = U(ρ₀) = 0 (degenerate vacua) and front connecting ρ₀ → 0, boundary conditions require
C = 0, so:
(A v_b² − B) ½ f'^2 + U(f) = 0.
Since U(f) $\geq$ 0, a nontrivial front requires:
A v_b² − B < 0,
or:
v_b < sqrt(B/A) ≡ c_ρ.
Thus **DVFT predicts a finite upper bound on space-creation speed**:
v_b(t) $\leq$ c_ρ,
where c_ρ = $\sqrt$(B/A) is the amplitude signal speed.
\begin{enumerate}
  \item Spherical Boundary in an Expanding Universe
\end{enumerate}
In spherical symmetry with cosmological expansion a(t), the amplitude equation becomes:
A($\partial$ₜ²ρ + 3H$\partial$ₜρ) − B($\partial$ᵣ²ρ + 2$\partial$ᵣρ/r) + U'(ρ) = 0,
where H = ȧ/a.
In a thin-front approximation ρ(r,t) $\approx$ f(r − R(t)), the evolution of R(t) obeys:
σ R¨ + 3H σ R˙ + (2σ / R) = ΔU,
where:
\begin{itemize}
  \item σ is surface tension of the amplitude front,
  \item ΔU = U(0) − U(ρ₀) is the T0-Vakuum-energy difference driving expansion.
\end{itemize}
Dividing by σ gives the effective boundary equation:
R¨ + 3H R˙ + 2/R = ΔU/σ.
This determines the actual physical space-creation speed v_b(t) = R˙(t).
\begin{enumerate}
  \item Why the Space-Creation Speed Is Not Infinite
\end{enumerate}
International Journal for Multidisciplinary Research (IJFMR)
E-ISSN: 2582-2160 $\bullet$ Website: www.ijfmr.com $\bullet$ Email: editor@ijfmr.com
IJFMR250664112 Volume 7, Issue 6, November-December 2025 35
The amplitude-front speed is finite because:
\begin{enumerate}
  \item DVFT uses a local field equation; local PDEs forbid instantaneous global change.
  \item The driving potential gradient |U'(ρ)| is finite.
  \item Energy conservation limits how fast ρ can rise from 0 → ρ₀.
  \item The characteristic T0-Vakuum signal speed is c_ρ = $\sqrt$(B/A), bounding v_b.
\end{enumerate}
Thus DVFT naturally rejects infinite expansion speeds without invoking relativity. Relativity (and light
speed c) only applies *inside* the ρ > 0 activated domain.
\begin{enumerate}
  \item Relation to Observational Horizon Size
\end{enumerate}
The comoving radius of the observable universe is:
R_obs $\approx$ 46.5 Gly.
A naive ratio gives:
R_obs / (c t_age) $\approx$ 46.5 / 13.8 $\approx$ 3.36.
This does **not** mean the boundary moved at 3.36 c.
Rather, DVFT predicts:
\begin{itemize}
  \item The front moves at v_b(t) $\leq$ c_ρ ~ c.
  \item The interior region expands with scale factor a(t).
\end{itemize}
The observed comoving radius is:
R_com(t₀) = a(t₀) $\int$₀^{t₀} [v_b(t) / a(t)] dt.
Metric expansion stretches distances so that the final comoving radius corresponds to an ‘effective average
speed’ greater than c *without violating relativity*, since no signals propagate faster than c within space.
\begin{enumerate}
  \item DVFT Prediction and Observational Fit
\end{enumerate}
DVFT predicts:
\begin{itemize}
  \item A finite space-creation speed v_b(t), controlled by T0-Vakuum micro-constants A, B and potential
\end{itemize}
shape U(ρ).
\begin{itemize}
  \item The cosmic horizon size (~46.5 Gly) arises from the combined effect of v_b(t) $\leq$ c_ρ and
\end{itemize}
cosmological scale-factor stretching.
Thus the theory *can be fitted to observational results* by constraining:
ΔU/σ, B/A, and the shape of U(ρ).
This makes DVFT testable against horizon scale, CMB\footnote{Siehe auch T0-Dokument: \texttt{030_T0_CMB-Anomalie_De.pdf}} structure, and early-universe expansion histories.
Conclusion
\begin{itemize}
  \item Space creation corresponds to the outward propagation of the T0-Vakuum amplitude ρ.
  \item The boundary speed v_b(t) is finite because the amplitude field obeys a hyperbolic PDE.
  \item The maximal speed is the T0-Vakuum amplitude signal speed c_ρ = $\sqrt$(B/A).
  \item Cosmological expansion amplifies R(t) → ~46.5 Gly today.
  \item The observed effective 3.36c ratio is not a physical propagation speed but a cumulative result of
\end{itemize}
front evolution + metric expansion.
DVFT therefore provides a complete, physically grounded mechanism for the finite but super-horizon
expansion of space.

\section{Referenzen zu T0-Dokumenten}

Dieses Kapitel steht in Zusammenhang mit folgenden T0-Dokumenten im Repository \texttt{2/pdf/}:

\begin{itemize}
  \item \texttt{002\_T0\_Grundlagen\_De.pdf} -- T0 Zeit-Masse-Dualität Grundlagen
  \item \texttt{004\_T0\_Energie\_De.pdf} -- T0 Energiefeld-Theorie
  \item \texttt{009\_T0\_xi\_ursprung\_De.pdf} -- Ursprung des geometrischen Parameters $\xi$
  \item \texttt{201\_DVFT-alles\_De.pdf} -- Vollständiges DVFT-Dokument (Rahmenwerk)
\end{itemize}