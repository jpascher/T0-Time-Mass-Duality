CHAPTER 17: ALTERNATIVE TO GR + ΛCDM
1. Introduction
This document explains, in a rigorous and logically complete manner, why the Dynamic Vacuum Field
Theory(DVFT) eliminates the need for the cosmological constant, invalidates inflation, removes the
foundations of ΛCDM, and supersedes all geometric or metric-based cosmological frameworks derived
from General Relativity (GR).
2. The Cosmological Constant as the Central Failure of Modern Cosmology
The mismatch between Λ predicted by Quantum Field Theory and Λ inferred from cosmology is ~10^120
— the largest discrepancy in the history of physics.
This alone indicates:
\textbullet{} ΛCDM cannot be fundamental,
\textbullet{} GR + Λ is an effective approximation, not a physical theory,
\textbullet{} the vacuum cannot be a geometric entity.
The cosmological constant problem is not a puzzle — it is evidence that the underlying ontology is
incorrect.
3. Why All Current Cosmological Models Fail
General Relativity (GR):
\textbullet{} offers no physical explanation for Λ,
\textbullet{} requires dark matter,
\textbullet{} requires inflation,
\textbullet{} predicts singularities,
\textbullet{} cannot quantize gravity.
Inflationary models:
\textbullet{} were invented solely to fix GR's horizon and flatness problems,
\textbullet{} have no physical vacuum origin,
\textbullet{} require finely tuned potentials,
\textbullet{} introduce unobservable fields.
Quantum Field Theory vacuum:
\textbullet{} predicts vacuum energy density 120 orders too large,
\textbullet{} cannot include gravitation consistently.
Modified gravity (MOND, f(R), TeVeS):
\textbullet{} work only at galactic scales,
\textbullet{} break at cosmological scales,
\textbullet{} lack microphysical interpretation.
String/LQG cosmologies:
\textbullet{} generate no definite predictions,
\textbullet{} require vast model freedom,
International Journal for Multidisciplinary Research (IJFMR)
E-ISSN: 2582-2160 \textbullet{} Website: www.ijfmr.com \textbullet{} Email: editor@ijfmr.com
IJFMR250664112 Volume 7, Issue 6, November-December 2025 41
\textbullet{} cannot explain Λ or dark energy.
These failures arise because all frameworks assume either:
\textbullet{} ❌ geometry is fundamental (GR),
\textbullet{} ❌ quantum fields sit on geometry (QFT).
Both assumptions are incorrect if the vacuum is physical.
4. DVFT Replaces the Cosmological Constant with Vacuum Amplitude Dynamics
DVFT defines the vacuum as a physical field:
Φ = ρ e^{iθ},
with a vacuum potential:
U(ρ) = (1/2) σ (ρ - ρ_0)².
Cosmic acceleration arises from relaxation of the vacuum amplitude ($\\rho_0 = 1/\\xi^2$ from T0) ρ, not from any constant Λ.
Thus:
\textbullet{} Λ is not fundamental,
\textbullet{} Λ is not constant,
\textbullet{} Λ is an artifact of misinterpreting U(ρ) geometrically.
This removes the cosmological constant problem completely.
DVFT does not solve Λ — it replaces the concept entirely.
5. DVFT Explains CMB Uniformity Without Inflation
GR cannot explain CMB temperature uniformity; inflation was invented to repair this.
DVFT predicts:
\textbullet{} an initially coherent vacuum phase θ,
\textbullet{} uniform amplitude ρ across all space,
\textbullet{} no distinct “regions” before expansion.
Therefore:
\textbullet{} the entire early universe shared a single vacuum state,
\textbullet{} temperature uniformity was intrinsic,
\textbullet{} no horizon problem exists.
CMB uniformity is direct empirical support for DVFT's vacuum ontology.
6. DVFT Explains Galaxy Rotation Without Dark Matter
DVFT deep-field equation:
g² = a_0 g_N
naturally reproduces flat rotation curves and the baryonic Tully–Fisher relation:
v_c⁴ = G M_b a_0.
No dark matter halos are required.
Dark matter appears only when the vacuum is incorrectly modeled using GR's geometry instead of DVFT's
amplitude-phase structure.
7. DVFT Predicts Cosmic Acceleration Without Λ
Since expansion is driven by U(ρ), not Λ:
\textbullet{} acceleration is dynamical, not constant,
\textbullet{} de Sitter space is not fundamental,
\textbullet{} observed late-time acceleration matches DVFT predictions,
\textbullet{} no fine-tuned cosmological constant is required.
International Journal for Multidisciplinary Research (IJFMR)
E-ISSN: 2582-2160 \textbullet{} Website: www.ijfmr.com \textbullet{} Email: editor@ijfmr.com
IJFMR250664112 Volume 7, Issue 6, November-December 2025 42
8. DVFT Provides Yang–Mills Mass Gap Automatically
The Yang–Mills Mass Gap emerges from vacuum phase stiffness B:
m_gap² ∼ B ρ_0².
No other theory provides a natural physical origin for the mass gap.
This is strong evidence for DVFT’s vacuum-field structure.
9. DVFT Eliminates Big Bang and Black Hole Singularities
Because ρ saturates at a maximum value, singularities cannot occur.
Instead:
\textbullet{} Big Bang = release of stored amplitude energy,
\textbullet{} Black hole core = finite-density vacuum-amplitude saturation,
\textbullet{} no divergences in curvature,
\textbullet{} no undefined geometry.
This is impossible in GR but automatic in DVFT.
10. DVFT Unifies All Interactions
DVFT unifies:
\textbullet{} gravity (∇ρ),
\textbullet{} electromagnetism (∇θ),
\textbullet{} weak and strong interactions (phase topology),
\textbullet{} quantum mechanics (θ-coherence),
\textbullet{} cosmology (U(ρ)),
from the single vacuum field Φ = ρ e^{iθ}.
This replaces both GR and QFT as fundamental theories.
Conclusion
DVFT provides:
\textbullet{} physical vacuum ontology,
\textbullet{} automatic solutions to cosmological inconsistencies,
\textbullet{} removal of Λ,
\textbullet{} natural explanation for CMB uniformity,
\textbullet{} galaxy curves without dark matter,
\textbullet{} acceleration without cosmological constant,
\textbullet{} singularity elimination,
\textbullet{} unification of all forces.
Therefore:
The cosmological constant problem, inflation problem, dark matter hypothesis, and GR-based cosmology
collectively point to a single conclusion:
DVFT is the only cosmological theory that remains consistent with all observations and solves all
foundational problems simultaneously. Thus DVFT stands not as an alternative, but as the fundamental
cosmological theory.


\subsection*{Cross-References}
Related: \\href{run:../pdf/Hubble_Analysis_En.pdf}{T0 Hubble Tension Resolution}


\section*{T0 Theory Integration}
This chapter integrates DVFT concepts with T0 Time-Mass Duality Theory, where the fundamental relation $T(x,t) \cdot m(x,t) = 1$ governs all vacuum field dynamics. The vacuum amplitude $\rho$ is directly related to local time $T$ through $\rho \propto 1/T$.
