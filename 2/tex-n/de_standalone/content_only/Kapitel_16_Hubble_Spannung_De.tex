\section{Kapitel 16: Ableitung der Hubble-Spannung (Angepasst an T0)}

\subsection{1. Einführung}

Die Hubble-Spannung bezieht sich auf die 5–10\% Diskrepanz zwischen:
\begin{itemize}
	\item $H_0$ abgeleitet aus Daten des frühen Universums (CMB, Planck), und
	\item $H_0$ gemessen im späten Universum (Cepheiden und SN Ia).
\end{itemize}

$\Lambda$CDM kann keine zwei unterschiedlichen Hubble-Werte erzeugen, da die kosmologische Konstante starr ist.

Angepasste DVFT erklärt die Spannung natürlich, weil das Vakuumfeld $\Phi = \rho e^{i\theta}$ dynamisch ist (abgeleitet aus T0 Zeit-Masse-Dualität), und seine Amplitude $\rho$ unterschiedlich im frühen homogenen Universum und im späten strukturierten Universum reagiert.

Im T0-Kontext: $\rho(x,t) \propto m(x,t) = 1/T(x,t)$, sodass strukturelle Evolution das lokale Zeitfeld ändert und damit die effektive Vakuumamplitude modifiziert.

\subsection{2. Vakuumfeld und kosmologische Dynamik in angepasster DVFT}

Angepasste DVFT beginnt mit:
\[
\Phi(x,t) = \rho(x,t) e^{i\theta(x,t)}
\]
wobei $\rho(x,t)$ aus T0s $\Delta m(x,t)$-Feld und $\theta(x,t)$ aus T0-Knotenrotationen abgeleitet ist.

Kosmologisch ist die relevante Variable $\rho(t)$.

Ein minimales an T0 angepasstes Vakuumpotential ist:
\[
U(\rho) = \frac{1}{2} \sigma (\rho - \rho_0)^2 + \ldots
\]
wobei $\rho_0 = 1/\xi^2$ aus T0s fundamentalem Parameter $\xi = \frac{4}{3} \times 10^{-4}$ abgeleitet ist.

Vakuumenergiedichte:
\[
\rho_{\text{vac}} = \frac{1}{2} A \dot{\rho}^2 + U(\rho)
\]

Dies ersetzt die konstante $\Lambda$ in der ART, begründet in T0s dynamischem Zeit-Masse-Feld.

\subsection{3. Angepasste DVFT-modifizierte Friedmann-Gleichung}

Mit $\Phi$ gekoppelt an die FRW-Geometrie durch T0-Dynamik wird die Friedmann-Gleichung zu:
\[
H^2 = \frac{1}{3M_{\text{pl}}^2} \left[\rho_m + \rho_{\text{vac}}(\rho, \dot{\rho})\right]
\]
mit:
\[
\rho_{\text{vac}} = \frac{1}{2} A \dot{\rho}^2 + U(\rho)
\]

$\rho(t)$ erfüllt die angepasste Bewegungsgleichung:
\[
A \ddot{\rho} + 3A H \dot{\rho} + \frac{dU}{d\rho} = S_{\text{backreact}}
\]

$S_{\text{backreact}}$ charakterisiert, wie Strukturstörungen durch T0-Knoten-Umordnungen in die Vakuumamplitudendynamik einfließen. Dieser Term verschwindet in homogenen Epochen, wird aber signifikant, wenn sich Struktur bildet.

\subsection{4. Vorhersage für das frühe Universum (CMB-Wert von $H_0$)}

Bei der Rekombination (T0s frühe kohärente Phase):
\begin{itemize}
	\item Universum nahezu homogen
	\item $S_{\text{backreact}} \approx 0$
	\item $\rho \approx \rho_*$, die Gleichgewichtsamplitude aus T0
	\item $\dot{\rho} \approx 0$
\end{itemize}

Somit:
\[
\rho_{\text{vac}} \approx U(\rho_*)
\]
ergibt:
\[
H_{\text{CMB}}^2 \approx \frac{\rho_m(\text{früh}) + U(\rho_*)}{3M_{\text{pl}}^2}
\]

Dies entspricht dem Planck-Wert $\sim$67 km/s/Mpc, konsistent mit T0s früher Universums-Feldkonfiguration.

\subsection{5. Vorhersage für das späte Universum (lokaler Wert von $H_0$)}

Nach der Strukturbildung (T0s strukturierte Phase):
\begin{itemize}
	\item $S_{\text{backreact}} \neq 0$
	\item Überdichten und Voids stören $\rho(x,t)$ durch T0-Knoten-Clustering
	\item Lokal gemittelte Amplitude: $\bar{\rho}_{\text{lokal}} \neq \rho_*$
	\item $\dot{\rho}_{\text{lokal}}$ kann ungleich Null sein
\end{itemize}

Somit:
\[
\rho_{\text{vac}}(\text{lokal}) = \frac{1}{2} A \dot{\rho}_{\text{lokal}}^2 + U(\bar{\rho}_{\text{lokal}})
\]
und:
\[
H_{\text{lokal}}^2 = \frac{\rho_m(\text{lokal}) + \rho_{\text{vac}}(\text{lokal})}{3M_{\text{pl}}^2}
\]

Wenn Struktur das Vakuum leicht nach oben in seinem Potential verschiebt (durch T0 Zeit-Masse-Dualitätseffekte):
\[
U(\bar{\rho}_{\text{lokal}}) > U(\rho_*)
\]

Dann:
\[
H_{\text{lokal}} > H_{\text{CMB}}
\]
was der beobachteten Spannung entspricht. Dies ergibt sich natürlich aus T0s lokalen Zeitfeld-Variationen: $T(x,t)$ unterscheidet sich in überdichten vs. unterdichten Regionen, somit variiert $m(x,t) = 1/T(x,t)$, was $\rho \propto m$ modifiziert.

\subsection{6. Warum $\Lambda$CDM dies nicht kann}

In $\Lambda$CDM:
\begin{itemize}
	\item $\Lambda$ ist konstant
	\item Vakuum reagiert nicht auf Struktur
	\item Es existiert nur ein $H_0$
\end{itemize}

Angepasste DVFT ersetzt $\Lambda$ durch eine dynamische Vakuumamplitude, begründet in T0 Zeit-Masse-Dualität.

Somit zeigen verschiedene kosmische Epochen natürlich unterschiedliche effektive $H_0$-Werte, weil T0s Zeitfeld $T(x,t)$ sich in homogenen vs. strukturierten Umgebungen unterschiedlich entwickelt.

\subsection{7. Quantitative Abschätzung}

Eine kleine fraktionale Änderung:
\[
\frac{\Delta U}{U} \approx 5{-}10\%
\]
in der effektiven Vakuumenergie aufgrund struktur-induzierter Änderungen in $\rho$ (durch T0-Knoten-Clustering) ist ausreichend, um zu erzeugen:
\[
H_{\text{lokal}} \approx H_{\text{CMB}} (1 + \varepsilon)
\]
mit $\varepsilon \approx 0,06{-}0,09$.

Dies stimmt exakt mit den Beobachtungsdaten überein.

Aus T0-Perspektive: Strukturbildung erzeugt $\Delta T/T \sim 5{-}10\%$ Variationen in der lokalen Eigenzeit, was sich direkt in $\Delta m/m$-Variationen durch $T \cdot m = 1$ übersetzt und damit die Vakuumamplitude $\rho \propto m$ modifiziert.

\subsection{8. Abschließende Interpretation}

In angepasster DVFT, begründet auf T0-Theorie, ist die Hubble-Spannung kein Widerspruch—sie ist zu erwarten.

Sie entsteht, weil:
\begin{itemize}
	\item \textbf{Frühes Universum} = kohärente Vakuumamplitude aus homogenem T0-Zeitfeld $\rightarrow$ ergibt $H_{\text{CMB}}$
	\item \textbf{Spätes Universum} = struktur-rückgekoppelte Vakuumamplitude aus inhomogenem T0-Zeitfeld $\rightarrow$ ergibt $H_{\text{lokal}}$
\end{itemize}

Dies ist direkter Beobachtungsnachweis dafür, dass:
\begin{enumerate}
	\item Das Vakuumfeld $\Phi = \rho e^{i\theta}$ dynamisch ist, keine feste kosmologische Konstante
	\item Die Vakuumamplitude $\rho$ auf Struktur durch T0s Zeit-Masse-Dualität reagiert
	\item T0-Theorie die fundamentale Erklärung liefert: lokale Zeitvariationen $\Delta T(x,t)$ erzeugen direkt Masse-/Energievariationen $\Delta m(x,t) = \Delta(1/T)$, was die effektive Hubble-Rate modifiziert
\end{enumerate}

\subsection{Fazit zu Kapitel 16}

Die Hubble-Spannung liefert überzeugende Beweise für:
\begin{itemize}
	\item Ein dynamisches Vakuumfeld, abgeleitet aus T0-Prinzipien
	\item Zeit-Masse-Dualität: $T(x,t) \cdot m(x,t) = 1$
	\item Den fundamentalen Parameter $\xi = \frac{4}{3} \times 10^{-4}$, der das Vakuumgleichgewicht $\rho_0 = 1/\xi^2$ setzt
\end{itemize}

Anstatt eine Krise für die Kosmologie zu sein, bestätigt die Hubble-Spannung, dass Raumzeit und Vakuumenergie fundamental durch T0s Zeit-Masse-Feldstruktur verbunden sind. Die "Spannung" ist tatsächlich die Signatur des Übergangs des Universums von einem homogenen zu einem strukturierten Zustand, vermittelt durch T0-Dynamik.
