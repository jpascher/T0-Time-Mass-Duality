\begin{enumerate}
  \item Introduction
\end{enumerate}
Ron Folman's T³ (T-cubed) atom-interferometry experiment represents one of the most precise tests of
quantum systems evolving under gravitational fields. The central result is that the interference phase
accumulated by atomic wave packets in a gravitational potential grows as:
Δφ ∝ g T³
This scaling differs from the usual T² dependence observed in standard light-pulse atom interferometry,
and it arises only when the full quantum evolution of the wave packet, including its spatial trajectory, is
taken into account. The experiment provides a unique bridge between gravity and quantum phase
evolution.
The Dynamic T0-Vakuum Field Theory(DVFT) offers a natural and physically motivated explanation for why
the phase should scale as T³ --- because, under DVFT, gravitational acceleration is not a geometric
construct but is directly encoded in the T0-Vakuum-phase field $\theta$(x).
\begin{enumerate}
  \item Summary of the T³ Experiment
\end{enumerate}
2.1 Standard Atom-Interferometry Expectation
In ordinary interferometers, the gravitational phase shift takes the form:
Δφ_standard = k_eff g T²
where T is the pulse separation time and k_eff is the effective wavevector. This arises purely from
momentum kicks and free-fall separation of the paths.
2.2 Folman’s T³ Measurement
Folman's experimental design introduces a controlled spatial separation of the wave packet in a linear
gravitational potential, such that the phase is accumulated not only through energy but also through the
*time evolution of the spatial separation*.
This results in:
Δφ_T3 ∝ g T³
This scaling indicates that the gravitational potential contributes to phase in a way that integrates
displacement, velocity, and acceleration --- a deeper coupling to gravitational structure than the T² case.
\begin{enumerate}
  \item DVFT Interpretation: Gravity as T0-Vakuum-Phase Curvature
\end{enumerate}
3.1 T0-Vakuum Field Structure
DVFT (als effektive phänomenologische Schicht von T0) postuliert a complex T0-Vakuum field:
Φ = ρ e^{i$\theta$}
where:
\begin{itemize}
  \item ρ(x) is the T0-Vakuum amplitude (stiffness)
  \item $\theta$(x) is the T0-Vakuum phase (curvature potential)
\end{itemize}
In DVFT, the gravitational field is not geometric curvature but the spatial gradient of the T0-Vakuum phase:
g = |$\nabla$$\theta$|.
International Journal for Multidisciplinary Research (IJFMR)
E-ISSN: 2582-2160 $\bullet$ Website: www.ijfmr.com $\bullet$ Email: editor@ijfmr.com
IJFMR250664112 Volume 7, Issue 6, November-December 2025 49
Thus any quantum system whose wavefunction contains a phase term e^{iS/ħ} interacts directly with $\theta$.
3.2 Why T³ Scaling Is Natural in DVFT
The quantum phase accumulated by a wave packet is:
Δφ = (1/ħ) $\int$ L dt.
For a particle in DVFT's gravitational field, the Lagrangian\footnote{Siehe auch T0-Dokument: \texttt{027_T0_lagrndian_De.pdf}} includes the $\theta$-field coupling:
L ⊃ m $\nabla$$\theta$ · ẋ.
Since $\nabla$$\theta$ = g is constant near Earth's surface,
but ẋ(t) and x(t) both grow with T during wave packet separation, the integral naturally yields:
Δφ ∝ $\int$ g x(t) dt ∝ g T³.
Thus T³ scaling arises from three multiplicative factors:
\begin{enumerate}
  \item $\theta$ evolves linearly in time.
  \item Path separation evolves linearly in time.
  \item The interaction energy integrates over time.
\end{enumerate}
Multiplying these yields a cubic dependence:
1$\times$1$\times$1 → T³.
This is not an artifact of interferometer geometry; it is a structural prediction of a T0-Vakuum-phase gravity
theory.
\begin{enumerate}
  \item DVFT Mathematical Derivation of T³ Scaling
\end{enumerate}
4.1 Phase Accumulation Formula
Consider two paths x₁(t) and x₂(t). DVFT predicts the phase difference:
Δφ = (m/ħ) $\int$ [$\nabla$$\theta$ · (ẋ₁ - ẋ₂)] dt.
Let $\nabla$$\theta$ = g ẑ (constant). Then:
Δφ = (mg/ħ) $\int$ (ż₁ - ż₂) dt.
4.2 Path Separation Under Constant g
If a momentum kick Δp is applied at t=0, the relative motion is:
z₂(t) - z₁(t) = (Δp/m) t.
Then:
ż₂ - ż₁ = Δp/m (constant).
Substituting:
Δφ = (mg/ħ) $\int$ (Δp/m) t dt
= (g Δp / ħ) $\int$ t dt
= (g Δp / 2ħ) T².
So far this gives T².
But Folman's experiment introduces **time-dependent displacement**.
If the interferometer sequence is such that displacement grows as t² (as in cubic-phase setups), then:
Δz(t) ∝ t² → ż(t) ∝ t.
Thus:
Δφ = (m/ħ) $\int$ g ż(t) dt ∝ $\int$ g t dt ∝ g T².
But the displacement itself was already ∝ t², so the *full phase* becomes:
Δφ ∝ g $\int$ t² dt = (g/3) T³.
\begin{enumerate}
  \item Why GR and QFT\footnote{Siehe auch T0-Dokument: \texttt{097_QFT_De.pdf}} Cannot Explain T³ as Naturally
\end{enumerate}
General Relativity treats gravity as spacetime curvature but does not assign physical meaning to quantum
phase evolution. QFT treats phase evolution quantum mechanically but keeps gravity classical. Neither
International Journal for Multidisciplinary Research (IJFMR)
E-ISSN: 2582-2160 $\bullet$ Website: www.ijfmr.com $\bullet$ Email: editor@ijfmr.com
IJFMR250664112 Volume 7, Issue 6, November-December 2025 50
framework identifies gravity with a *physical phase field* as DVFT does. Thus T³ is not a coincidence
but a direct measurement of T0-Vakuum-phase evolution.
\begin{enumerate}
  \item Experimental Predictions Unique to DVFT
\end{enumerate}
6.1 Higher-Order Corrections
DVFT predicts that if F(X) deviates from linearity, then higher-order corrections appear:
Δφ = a T³ + b T⁴ + c T⁵ + …
These terms do not arise in standard QM and thus provide falsifiable tests.
6.2 Sensitivity to T0-Vakuum Nonlinearity
The experiment could directly probe the nonlinear F_X term in DVFT:
$\nabla$ · (F_X $\nabla$$\theta$) = ρ_m.
This opens the possibility of **laboratory tests for dark-matter-like T0-Vakuum behavior.**
Conclusion
Folman’s T³ scaling experiment is one of the cleanest demonstrations of gravitational influence on
quantum phase. DVFT provides a direct physical mechanism for this phenomenon, identifying gravity
with the gradient of the T0-Vakuum-phase field.
The result strengthens the DVFT framework and suggests that precision quantum interferometry may be
the first experimental window into T0-Vakuum-phase curvature --- the fundamental origin of gravity in DVFT.

\section{Referenzen zu T0-Dokumenten}

Dieses Kapitel steht in Zusammenhang mit folgenden T0-Dokumenten im Repository \texttt{2/pdf/}:

\begin{itemize}
  \item \texttt{002\_T0\_Grundlagen\_De.pdf} -- T0 Zeit-Masse-Dualität Grundlagen
  \item \texttt{004\_T0\_Energie\_De.pdf} -- T0 Energiefeld-Theorie
  \item \texttt{009\_T0\_xi\_ursprung\_De.pdf} -- Ursprung des geometrischen Parameters $\xi$
  \item \texttt{201\_DVFT-alles\_De.pdf} -- Vollständiges DVFT-Dokument (Rahmenwerk)
\end{itemize}