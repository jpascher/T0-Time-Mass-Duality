\begin{enumerate}
  \item Introduction
\end{enumerate}
DVFT (Dynamic T0-Vakuum field Curvature Theory) provides a natural and structurally unavoidable solution
to the Strong CP Problem, without requiring axi\footnote{Siehe auch T0-Dokument: \texttt{009_T0_xi_ursprung_De.pdf}}ons, Peccei--Quinn symmetry, or fine-tuning. This
document explains rigorously why DVFT forces the QCD $\theta$-angle to zero as a consequence of the T0-Vakuum
field structure.
\begin{enumerate}
  \item Statement of the Strong CP Problem
\end{enumerate}
Quantum Chromodynamics permits a CP-violating term:
L = $\theta$ (g_s² / 32π²) G_{$\mu$ν} ṠG^{$\mu$ν}
Experimentally, neutron EDM measurements require:
$\theta$ < 10⁻¹⁰
But the natural value in QCD is $\theta$ $\approx$ 1. The Standard Model provides no mechanism to set $\theta$ $\approx$ 0. This
discrepancy is the Strong CP Problem.
\begin{enumerate}
  \item Core DVFT Insight: Only One Physical Phase Field
\end{enumerate}
In DVFT, all forces---including QCD---emerge from the single T0-Vakuum field:
Φ(x,t) = ρ(x,t) e^{i$\theta$(x,t)}
Here $\theta$(x,t) is the unique global T0-Vakuum phase. QCD cannot introduce an independent $\theta$ parameter. No
separate strong-sector phase exists; therefore a CP-violating $\theta$-term has no place in the fundamental
Lagrangian\footnote{Siehe auch T0-Dokument: \texttt{027_T0_lagrndian_De.pdf}}.
Thus:
$\theta$_QCD ≡ 0
by structural necessity, not tuning.
\begin{enumerate}
  \item Why Independent QCD $\theta$ Cannot Exist in DVFT
\end{enumerate}
The QCD $\theta$-term arises from instanton topology. DVFT reinterprets instantons as localized amplitude
knots in ρ(x), not as separate phase sectors.
DVFT enforces:
\begin{itemize}
  \item Continuous global $\theta$(x,t)
  \item No multi-sector T0-Vakuum structure
  \item No misalignment between QCD and T0-Vakuum phases
\end{itemize}
Therefore a CP-violating G ṠG term cannot emerge.
International Journal for Multidisciplinary Research (IJFMR)
E-ISSN: 2582-2160 $\bullet$ Website: www.ijfmr.com $\bullet$ Email: editor@ijfmr.com
IJFMR250664112 Volume 7, Issue 6, November-December 2025 77
\begin{enumerate}
  \item Neutron Electric Dipole Moment Prediction
\end{enumerate}
DVFT predicts the neutron EDM is approximately zero because the T0-Vakuum amplitude around neutrons
is CP-symmetric and the global phase $\theta$(x) cannot induce sector-specific asymmetry. Thus:
d_n $\approx$ 0
in perfect agreement with experiment, without axions or symmetry breaking.
\begin{enumerate}
  \item Comparison With Standard Approaches
\end{enumerate}
Standard Model: Offers no explanation; $\theta$ must be tuned < 10⁻¹⁰.
Axion/PQ symmetry: Adds particles + symmetry; no experimental detection.
String theory: Introduces many vacua; not predictive.
DVFT: Eliminates $\theta$ as an independent variable. Simple, natural, enforced.
\begin{enumerate}
  \item Deeper Reason: Correct Ontology
\end{enumerate}
The Strong CP Problem exists only because QCD---incorrectly---treats the T0-Vakuum as empty. If the
T0-Vakuum is physical (as in DVFT), then its phase structure is unique, global, and non-duplicable. The
freedom to choose $\theta$ is eliminated.
Thus:
$\theta$_QCD = 0
is not fine-tuned; it is the only mathematically allowable value.
Conclusion
DVFT resolves the Strong CP Problem cleanly and uniquely:
\begin{itemize}
  \item No axions.
  \item No fine-tuning.
  \item No new symmetries.
  \item Complete alignment with experiment.
  \item Directly derived from the single T0-Vakuum phase field.
\end{itemize}
This constitutes one of the strongest conceptual triumphs of DVFT.

\section{Referenzen zu T0-Dokumenten}

Dieses Kapitel steht in Zusammenhang mit folgenden T0-Dokumenten im Repository \texttt{2/pdf/}:

\begin{itemize}
  \item \texttt{002\_T0\_Grundlagen\_De.pdf} -- T0 Zeit-Masse-Dualität Grundlagen
  \item \texttt{004\_T0\_Energie\_De.pdf} -- T0 Energiefeld-Theorie
  \item \texttt{009\_T0\_xi\_ursprung\_De.pdf} -- Ursprung des geometrischen Parameters $\xi$
  \item \texttt{201\_DVFT-alles\_De.pdf} -- Vollständiges DVFT-Dokument (Rahmenwerk)
\end{itemize}