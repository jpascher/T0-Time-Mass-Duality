Modern astrophysics and cosmology face numerous unresolved problems that General Relativity (GR)
and the \LambdaCDM model cannot fully explain without invoking dark matter particles, fine-tuned inflation
fields, unexplained singularities, or an arbitrary cosmological constant. FFGFT provides a physically
grounded alternative by treating spacetime as a dynamic T0-Vakuum field.
One of the prime achievement of FFGFT is that galaxy rotation anomalies follow directly from FFGFT deep
field physics, eliminating the need for dark matter halos. Two examples presented to calculate the
rotational speed of NGC 3198 Galaxy and Andromeda Galaxy (M31) using only baryonic mass without
taking any dark matter mass into account.
FFGFT defines the T0-Vakuum field as \Phi = ? e^{i$\theta$}. In the weak-field, low-acceleration outer regions of
galaxi\footnote{Siehe auch T0-Dokument: \texttt{009_T0_xi_ursprung_De.pdf}}es where observed rotation curves deviate from Newtonian predictions, FFGFT predicts a nonlinear
International Journal for Multidisciplinary Research (IJFMR)
E-ISSN: 2582-2160 $\bullet$ Website: www.ijfmr.com $\bullet$ Email: editor@ijfmr.com
IJFMR250664112 Volume 7, Issue 6, November-December 2025 19
T0-Vakuum response based on deep field equations derived from T0-Vakuum Lagrangian\footnote{Siehe auch T0-Dokument: \texttt{027_T0_lagrndian_De.pdf}} gives the baryonic
Tully--Fisher relation:
v_c? = G M_b a?
Where, v_c is circular speed, M_b is Baryonic mass and G is Newton?s Gravitational Constant
These equations are derived from the basic FFGFT equation \Phi = ? e^{i$\theta$} and the T0-Vakuum Lagrangian.
Complete derivation of this equation has been given below.
\begin{enumerate}
  \item FFGFT T0-Vakuum Lagrangian and \Phi = ? e^{i$\theta$}
\end{enumerate}
Start with a minimal FFGFT T0-Vakuum Lagrangian:
? = ? A |$\partial$?\Phi|? ? ? B(?) |$\nabla$\Phi|? ? U(?) ? ?_b ?(?,$\theta$),
where:
\begin{itemize}
  \item A is T0-Vakuum temporal inertia,
  \item B(?) is T0-Vakuum spatial stiffness,
  \item U(?) is the T0-Vakuum amplitude potential,
  \item ?_b is baryonic matter density,
  \item ? is the gravitational potential encoded in $\theta$.
\end{itemize}
Substitute \Phi = ? e^{i$\theta$}:
\begin{itemize}
  \item |$\partial$?\Phi|? = ($\partial$??)? + ??($\partial$?$\theta$)?
  \item |$\nabla$\Phi|? = |$\nabla$?|? + ??|$\nabla$$\theta$|?
\end{itemize}
Thus:
? = ?A[($\partial$??)? + ??($\partial$?$\theta$)?] ? ?B(?)[|$\nabla$?|? + ??|$\nabla$$\theta$|?] ? U(?) ? ?_b ?.
\begin{enumerate}
  \item Static Nonrelativistic Limit
\end{enumerate}
For galaxy rotation curves, time derivatives are negligible:
\begin{itemize}
  \item $\partial$?? $\approx$ 0,
  \item $\partial$?$\theta$ $\approx$ constant (background T0-Vakuum oscillation).
\end{itemize}
FFGFT identifies gravitational potential ? through phase evolution:
$\partial$?$\theta$ = ??(1 + ?/c?) $\Rightarrow$ $\nabla$$\theta$ = (??/c?) $\nabla$?.
Thus, the T0-Vakuum energy density becomes:
?_vac $\approx$ ? K(?) |$\nabla$?|? + U(?),
where K(?) = B(?) ?? (??? / c?).
This shows that gravitational behavior arises from spatial variations of ?, mediated by T0-Vakuum amplitude
?.
\begin{enumerate}
  \item Integrating Out the T0-Vakuum Amplitude ?
\end{enumerate}
At equilibrium (static galaxies), ? adjusts to minimize local T0-Vakuum energy:
$\partial$/$\partial$? [?K(?)|$\nabla$?|? + U(?)] = 0.
This yields an algebraic relation:
? K'(?)|$\nabla$?|? + U'(?) = 0.
In high-acceleration regimes, ? $\approx$ ?? (the T0-Vakuum ground amplitude) and Newtonian gravity emerges.
In low-acceleration regimes, the T0-Vakuum becomes nearly coherent, U'(?) $\rightarrow$ 0, allowing ? to respond
strongly to |$\nabla$?|.
Scale invariance of FFGFT in this regime requires the T0-Vakuum energy to scale as:
? ? |$\nabla$?|?.
This corresponds to a T0-Vakuum functional:
F(y) ? y^{3/2}, y = |$\nabla$?|? / a??.
International Journal for Multidisciplinary Research (IJFMR)
E-ISSN: 2582-2160 $\bullet$ Website: www.ijfmr.com $\bullet$ Email: editor@ijfmr.com
IJFMR250664112 Volume 7, Issue 6, November-December 2025 20
\begin{enumerate}
  \item Deep-Field Lagrangian
\end{enumerate}
In the deep-field regime (g ? a?), the T0-Vakuum Lagrangian becomes:
?_eff = ? (a??/8?G) F(|$\nabla$?|?/a??) ? ?_b ?,
with:
F(y) = (2/3) y^{3/2}.
Varying this with respect to ? yields the field equation:
$\nabla$?[(|$\nabla$?|/a?) $\nabla$?] = 4?G ?_b.
Define gravitational acceleration g = |$\nabla$?|; then:
$\nabla$?[(g/a?) g? g] = 4?G ?_b.
\begin{enumerate}
  \item Spherical Galaxy: Deriving g? = a? g_N
\end{enumerate}
For a spherical mass distribution:
g(r) = |$\nabla$?| = d?/dr.
The FFGFT deep-field equation becomes:
(1/r?) d/dr (r? g? / a?) = 4?G ?_b(r).
Integrate from 0 to r:
r? g? / a? = G M_b(r).
Solve for g:
g?(r) = a? (G M_b(r)/r?) = a? g_N(r).
This is exactly the FFGFT deep-field force law:
g? = a? g_N.
\begin{enumerate}
  \item Rotation Curves and Tully--Fisher Relation
\end{enumerate}
The circular velocity satisfies:
g(r) = v_c?(r)/r.
Insert into g? = a? g_N:
(v_c?/r)? = a? (G M_b / r?).
Simplify:
v_c?(r) = G M_b(r) a?.
In the flat part of the rotation curve, M_b(r) $\rightarrow$ constant = M_b, giving the baryonic Tully--Fisher relation
:
v_c? = G M_b a?,
\begin{enumerate}
  \item Physical Meaning in FFGFT
\end{enumerate}
In FFGFT:
\begin{itemize}
  \item amplitude ? determines inertia and curvature,
  \item phase $\theta$ determines wave propagation and time,
  \item gravity arises from phase-time distortions governed by nonlinear T0-Vakuum response.
\end{itemize}
In low-acceleration galactic outskirts, the T0-Vakuum approaches coherent phase, causing gravitational
behavior to shift from Newtonian (linear) to scale-invariant nonlinear regime.
This reproduces:
\begin{itemize}
  \item flat rotation curves,
  \item g? = a? g_N,
  \item the baryonic Tully--Fisher law,
  \item all without dark matter.
\end{itemize}
\begin{enumerate}
  \item Summary
\end{enumerate}
International Journal for Multidisciplinary Research (IJFMR)
E-ISSN: 2582-2160 $\bullet$ Website: www.ijfmr.com $\bullet$ Email: editor@ijfmr.com
IJFMR250664112 Volume 7, Issue 6, November-December 2025 21
Starting from the fundamental FFGFT field \Phi = ? e^{i$\theta$}, we derived:
\begin{itemize}
  \item an effective T0-Vakuum energy ? |$\nabla$?|?,
  \item the deep-field equation $\nabla$?[(g/a?) g] = 4?G?_b,
  \item the spherical solution g? = a? g_N,
  \item and the baryonic Tully--Fisher relation v_c? = G M_b a?.
\end{itemize}
Thus, galaxy rotation anomalies follow directly from FFGFT T0-Vakuum physics, eliminating the need for
dark matter halos.
Let?s use this equation to calculate the galaxy rotational speed only using visible mass without taking dark
matter into account and compare it with actual observational rotation speed of these two galaxies.
\begin{enumerate}
  \item NGC 3198 Galaxy
\end{enumerate}
Rotation curve: nearly flat at v $\approx$ 150 km/s beyond r ? 20 kpc.
Stellar mass from BTFR / photometric fits: total baryonic mass M_b $\approx$ 2.46 $\times$ 10?? M_?.
Rotation Speed using baryonic Tully--Fisher relation v_c? = G M_b a? with a? = 1.2$\times$10??? m/s?:
v_c $\approx$ 141 km/s.
Interpretation: FFGFT prediction close to the observed 150 km/s without dark matter.
\begin{enumerate}
  \item Andromeda Galaxy
\end{enumerate}
Rotation curve: nearly flat at v $\approx$ 220 -- 226 km/s between 20 -35 kpc
Total baryonic mass: $\approx$ 1.6$\times$10?? M_? (Stars + Gas)
Rotation Speed using baryonic Tully--Fisher relation v_c? = G M_b a? with a? = 1.2$\times$10??? m/s?
v_c $\approx$ 220 km/s.
Interpretation: FFGFT prediction close to the observed 220 - 226 km/s without dark matter.
Conclusion
Both NGC 3198 and Andromeda Galaxies behaves exactly as predicted by FFGFT deep field equation
gives a flat rotation curve set directly by baryonic mass, with no requirement for dark matter.
FFGFT provides gravitational equations which eliminates requirement of dark matter in cosmological
calculations.

\section{Referenzen zu T0-Dokumenten}

Dieses Kapitel steht in Zusammenhang mit folgenden T0-Dokumenten im Repository \texttt{2/pdf/}:

\begin{itemize}
  \item \texttt{002\_T0\_Grundlagen\_De.pdf} -- T0 Zeit-Masse-Dualit?t Grundlagen
  \item \texttt{004\_T0\_Energie\_De.pdf} -- T0 Energiefeld-Theorie
  \item \texttt{009\_T0\_xi\_ursprung\_De.pdf} -- Ursprung des geometrischen Parameters $\xi$
  \item \texttt{201\_FFGFT-alles\_De.pdf} -- Vollst?ndiges FFGFT-Dokument (Rahmenwerk)
\end{itemize}