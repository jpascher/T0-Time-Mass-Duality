CHAPTER 39: ENTROPY
1. Introduction
The Second Law of Thermodynamics is one of the most revered and mysterious principles in physics. It
states that entropy never decreases in an isolated system. But mainstream physics never explains why this
law exists—it simply treats it as a statistical tendency or a mathematical result of counting microstates.
Dynamic Vacuum Field Theory (DVFT) offers a deeper explanation. In this framework, entropy is not a
fundamental law but an emergent property arising from the one-way evolution of the vacuum's internal
phase field θ(x,t). Time itself is defined as vacuum phase accumulation. Because this vacuum phase can
never reverse, entropy can never decrease.
This document presents the DVFT interpretation of entropy, irreversibility, and the Second Law of
Thermodynamics.
2. Time as Vacuum Phase Evolution
In DVFT, the vacuum is a physical medium with two continuous fields:
\textbullet{} ρ(x,t) — vacuum amplitude ($\\rho_0 = 1/\\xi^2$ from T0)
\textbullet{} θ(x,t) — vacuum phase
Time is not a coordinate: it is the physical progression of vacuum phase. Proper time τ is proportional to
the accumulated phase along a worldline:
dτ ∝ dθ.
A crucial property is:
θₜ > 0 always.
International Journal for Multidisciplinary Research (IJFMR)
E-ISSN: 2582-2160 \textbullet{} Website: www.ijfmr.com \textbullet{} Email: editor@ijfmr.com
IJFMR250664112 Volume 7, Issue 6, November-December 2025 87
This means vacuum phase evolves monotonically forward. All physical processes—oscillations, clocks,
interactions—are tied to θ. Therefore, the direction of time is the direction of vacuum phase evolution.
3. Why Entropy Increases in DVFT
Entropy increases because physical systems lose phase coherence as vacuum phase evolves. Every
interaction—thermal, electromagnetic, gravitational, or quantum—spreads vacuum phase information
outward. This causes:
\textbullet{} Loss of microscopic coherence: Phase correlations are dispersed in space and cannot be reversed.
\textbullet{} Mixing of amplitude configurations: Local amplitude excitations (mass/energy) relax into more
uniform distributions.
\textbullet{} Irreversible phase dispersion: Since θ evolves only forward, coherence cannot be reconstructed.
\textbullet{} No mechanism for phase reversal: Reversing θ would require reversing every physical process
in the universe, which is impossible.
In DVFT, entropy increase is not a statistical accident. It is the inevitable result of forward vacuum phase
evolution.
4. Entropy and the Arrow of Time
In classical physics, time is a coordinate. In thermodynamics, the arrow of time is assigned to entropy
increase. In quantum mechanics, time is a parameter outside the formalism.
DVFT unifies these by stating:
Arrow of time = direction of vacuum phase evolution.
Entropy does not cause time's arrow; entropy is a symptom of vacuum phase moving forward.
Because θ cannot reverse, entropy cannot reverse.
5. Why Entropy Cannot Decrease
To decrease entropy, a system must:
\textbullet{} restore lost correlations,
\textbullet{} reverse decoherence,
\textbullet{} undo interactions,
\textbullet{} reconstruct past microstates.
But in DVFT, this requires reversing vacuum phase evolution—a physical impossibility because:
\textbullet{} The vacuum phase field θ is globally single-valued.
\textbullet{} θₜ > 0 everywhere due to positive vacuum inertial density.
\textbullet{} Energy positivity forbids θ reversal.
\textbullet{} Past phase information is not stored; it is erased through dispersion.
Thus, the Second Law of Thermodynamics is a direct consequence of vacuum physics: Entropy cannot
decrease because phase cannot un-evolve.
6. Thermalization as Phase Scrambling
In DVFT, heating corresponds to vacuum phase scrambling. Temperature reflects how rapidly phase
gradients fluctuate. When systems interact, their phase gradients mix, driving them toward equilibrium.
Thus:
\textbullet{} Heat flow = flow of phase disorder
\textbullet{} Equilibrium = maximum phase scrambling
\textbullet{} Entropy = measure of vacuum phase uncertainty
7. Quantum Mechanics and Entropy
International Journal for Multidisciplinary Research (IJFMR)
E-ISSN: 2582-2160 \textbullet{} Website: www.ijfmr.com \textbullet{} Email: editor@ijfmr.com
IJFMR250664112 Volume 7, Issue 6, November-December 2025 88
Quantum decoherence is a phase process: loss of relative phase information between amplitude
components. Once decoherence occurs, phase cannot be reconstructed, so entropy increases.
Thus DVFT explains:
\textbullet{} Why measurement increases entropy
\textbullet{} Why superpositions collapse into classical outcomes
\textbullet{} Why quantum information cannot be fully recovered once dispersed
8. Cosmological Entropy in DVFT
DVFT provides a natural explanation for cosmological entropy:
\textbullet{} As the universe expands, vacuum amplitude ($\\rho_0 = 1/\\xi^2$ from T0) relaxes, causing large-scale phase dispersion.
\textbullet{} This dispersion increases entropy on cosmic scales.
\textbullet{} Black holes represent regions of extreme amplitude, freezing phase and maximizing entropy.
The universe’s thermodynamic arrow is just the global vacuum phase arrow.
Conclusion
DVFT transforms the Second Law of Thermodynamics from a statistical rule into a physical inevitability:
\textbullet{} Time = vacuum phase evolution.
\textbullet{} Phase evolves only forward.
\textbullet{} Entropy increases because phase coherence irreversibly spreads and cannot be undone.
\textbullet{} Irreversibility is not probabilistic—it's built into vacuum structure.
Thus entropy is not fundamental; it is emergent. DVFT provides the first physical explanation for the
Second Law and the arrow of time, resolving conceptual gaps in thermodynamics, quantum mechanics,
and cosmology.


\section*{T0 Theory Integration}
This chapter integrates DVFT concepts with T0 Time-Mass Duality Theory, where the fundamental relation $T(x,t) \cdot m(x,t) = 1$ governs all vacuum field dynamics. The vacuum amplitude $\rho$ is directly related to local time $T$ through $\rho \propto 1/T$.
