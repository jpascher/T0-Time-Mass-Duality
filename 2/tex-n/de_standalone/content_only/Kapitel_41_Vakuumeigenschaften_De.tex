\begin{abstract}
Dieses Kapitel kompiliert die intrinsischen numerischen Parameter des Vakuumfelds in der Dynamischen Vakuumfeldtheorie (DVFT), integriert mit der T0-Theorie. Anders als in der konventionellen Physik, wo Vakuumkonstanten als unverbundene Eingaben erscheinen, vereinigt T0-DVFT sie unter der Dynamik eines einzigen komplexen Vakuumfelds $\Phi = \rho e^{i\theta}$, wobei die Amplitude $\rho \propto 1/T(x,t)$ den Kehrwert von T0s fundamentalem Zeitfeld darstellt. Alles elektromagnetische, quantenmechanische und gravitationale Verhalten entspringt diesem vereinheitlichten Framework.
\end{abstract}

\section{Einleitung}

Dieses Dokument stellt die intrinsischen numerischen Parameter des Vakuumfelds in T0-DVFT zusammen. Anders als in der konventionellen Physik, wo Vakuumkonstanten wie $\alpha$, $\varepsilon_0$, $\hbar$, $c$ und kosmologische Dichte als unverbundene Eingaben erscheinen, vereinigt T0-DVFT sie unter der Dynamik eines einzigen komplexen Vakuumfelds:

\begin{equation}
\Phi(x,t) = \rho(x,t) e^{i\theta(x,t)}
\end{equation}

Hierbei:
\begin{itemize}
\item $\rho(x,t) = \frac{1}{T(x,t) \cdot \xi}$ ist die Vakuumamplitude, abgeleitet von T0s Zeitfeld $T(x,t)$
\item $\rho(x,t)$ bestimmt die Trägheitsdichte und gravitative Steifigkeit
\item $\theta(x,t)$ ist die Vakuumphase (Quantenkohärenz, Ladung, CP-Verletzung)
\item $\xi = 4/3 \times 10^{-4}$ ist T0s einziger fundamentaler Parameter
\end{itemize}

Die Konstanten, die $\rho$ und $\theta$ bestimmen, definieren die mechanische, elektromagnetische und quantenmechanische Struktur der Raumzeit selbst. Dieses Dokument konsolidiert ihre Werte und zeigt, wie sie sich auf beobachtbare Physik beziehen.

\section{Fundamentale T0-DVFT-Vakuumparameter}

T0-DVFT führt folgende intrinsische Vakuumparameter ein:

\begin{enumerate}
\item $B$ — Vakuumphasensteifigkeit
\item $\rho_0 = 1/\xi^2$ — Gleichgewichtsvakuumamplitude (aus T0)
\item $K_0$ — Amplitudensteifigkeit des Vakuums
\item $(\partial\theta/\partial x)$ — Fundamentaler Phasengradient entsprechend einer Elementarladungseinheit
\end{enumerate}

Diese bestimmen das gesamte quantenmechanische, elektromagnetische und gravitationale Verhalten, das aus $\Phi$ hervorgeht.

\section{Phasensteifigkeit $B$ (Kalibriert von $\alpha$)}

Die Feinstrukturkonstante $\alpha$ wird in T0-DVFT ausgedrückt als:

\begin{equation}
\alpha = \frac{B}{\hbar c} \left(\frac{\partial\theta}{\partial x}\right)^2
\end{equation}

Die Wahl des Phasengradienten für eine Elementarladung:
\begin{equation}
\left|\frac{\partial\theta}{\partial x}\right| \approx \frac{2\pi}{\lambda_C}, \quad \lambda_C = \frac{\hbar}{m_e c} \approx \SI{3,86e-13}{m}
\end{equation}

Dies ergibt:
\begin{equation}
\left|\frac{\partial\theta}{\partial x}\right| \approx \SI{1,63e13}{m^{-1}}
\end{equation}

Mit $\alpha_{\text{exp}} = 1/137{,}036$ resultiert die Vakuumphasensteifigkeit:

\begin{equation}
B \approx \SI{8,7e-55}{}
\end{equation}

(Einheit hängt von Lagrange-Normalisierung ab.)

\textbf{Interpretation:}
\begin{itemize}
\item $B$ misst, wie schwer es ist, die Vakuumphase $\theta$ zu verdrehen
\item Dasselbe $B$ muss für Elektromagnetismus, Neutrinomassen, Baryogenese und Quantenkohärenz verwendet werden
\item Universelle Anwendbarkeit gewährleistet Vereinheitlichung durch ein einzelnes Vakuumfeld
\end{itemize}

\section{Trägheitsvakuumdichte $\rho_0$}

Aus der T0-Theorie ist die Gleichgewichtsvakuumamplitude:

\begin{equation}
\rho_0 = \frac{1}{\xi^2} \approx \frac{1}{(4/3 \times 10^{-4})^2} \approx \SI{5,6e6}{}
\end{equation}

Dies kann auch mit der effektiven massenäquivalenten Dichte der dunklen Energie in Beziehung gesetzt werden:
\begin{equation}
\rho_0 \sim \SI{6e-27}{kg/m^3}
\end{equation}

Dies repräsentiert den intrinsischen Trägheitsinhalt der Vakuumamplitude $\rho$, der direkt an gravitatives Verhalten koppelt. Durch T0s Zeit-Masse-Dualität $T \cdot m = 1$ manifestieren sich Variationen in $\rho$ als Variationen im lokalen Zeitfluss.

\section{Amplitudensteifigkeit $K_0$}

Die Amplitudensteifigkeit quantifiziert den Widerstand des Vakuums gegenüber räumlichen Gradienten in $\rho$:

\begin{equation}
K_0 \approx \SI{5,4e-10}{J/m^3}
\end{equation}

Dieser Parameter bestimmt:
\begin{itemize}
\item Lichtgeschwindigkeit: $c = \sqrt{K_0/\rho_0}$
\item Gravitationswellengeschwindigkeit
\item Vakuumenergiedichteskala
\item Maximale Krümmung vor Singularitätsvermeidung
\end{itemize}

\textbf{Verbindung zu T0:}

Das Vakuumpotential, das Singularitäten verhindert, lautet:
\begin{equation}
U(\rho) = \Lambda_0 + \frac{\kappa}{2}(\rho - \rho_0)^2 + \frac{\lambda}{4}(\rho - \rho_0)^4
\end{equation}

wobei $\kappa \sim K_0$ und $\rho_0 = 1/\xi^2$ aus T0. Dieses stark konvexe Potential gewährleistet:
\begin{equation}
U(\rho) \to \infty \text{ für } |\rho - \rho_0| \to \infty
\end{equation}

Daher kann $\rho$ nicht divergieren, was alle Singularitäten eliminiert.

\section{Fundamentaler Phasengradient}

Der Phasengradient für eine Elektronenladung beträgt:

\begin{equation}
\left|\frac{\partial\theta}{\partial x}\right|_e \approx \SI{1,63e13}{m^{-1}}
\end{equation}

Dies quantifiziert die elektromagnetische Struktur des Vakuums und verbindet sich mit:
\begin{itemize}
\item Feinstrukturkonstante $\alpha$
\item Quanten-Hall-Leitfähigkeit
\item Magnetflussquantisierung
\item CP-Verletzungsparameter
\end{itemize}

\section{Tieffeld-Gravitationsskala $a_0$}

In der galaktischen Dynamik, wo T0-DVFT die dunkle Materie ersetzt, ist die charakteristische Beschleunigung:

\begin{equation}
a_0 \approx \SI{10e-10}{m/s^2}
\end{equation}

Dies entsteht aus Vakuumphasengradienten in schwachen Feldbereichen und erklärt:
\begin{itemize}
\item Flache Rotationskurven ohne dunkle Materie
\item Tully-Fisher-Relation
\item MOND-Phänomenologie
\end{itemize}

\textbf{T0-Verbindung:}

In Regionen niedriger Gravitationsbeschleunigung treiben $T(x)$-Variationen Phasengradienten:
\begin{equation}
\nabla\theta \propto \nabla T \quad \Rightarrow \quad a_0 \sim \frac{c}{\xi} \nabla T
\end{equation}

\section{Dunkle-Energie-Verhalten}

Die Vakuumenergiedichte im Gleichgewicht entspricht den Beobachtungen:

\begin{equation}
U(\rho_0) \approx K_0 \sim \SI{10e-10}{J/m^3}
\end{equation}

Dies entspricht der kosmologischen Konstante:
\begin{equation}
\Lambda \approx \frac{8\pi G}{c^4} K_0
\end{equation}

Aus T0 entsteht dies natürlich aus dem Gleichgewichtszustand des Zeitfelds ohne Feinabstimmung.

\section{Warum ein einziges $B$ überall konsistent ist}

$B$ muss universell sein, weil:

\begin{itemize}
\item $\theta$ ein universelles Phasenfeld in T0-DVFT ist
\item Alle Quantenphänomene (Ladung, CP-Verletzung, Kohärenz, Neutrinomassen, Photonausbreitung, Baryogenese) aus derselben $\theta$-Dynamik entstehen
\item Eine einzelne Steifigkeitskonstante die Vereinheitlichung gewährleistet, genau wie $\hbar$ und $c$ universell in der konventionellen Physik gelten
\item T0s Prinzip $T \cdot m = 1$ Konsistenz über alle Skalen garantiert
\end{itemize}

Dies erlaubt T0-DVFT kohärent zu erklären:
\begin{itemize}
\item Quantenmechanik
\item Elektromagnetismus
\item Neutrinoverhalten
\item Tieffeld-Gravitation (ohne dunkle Materie)
\item Dunkle Energie
\item Früh-Universum CP-Asymmetrie
\end{itemize}

alles durch dasselbe Vakuumfeld mit einem einzigen Parameter $\xi$.

\section{Zusammenfassung der intrinsischen Vakuumparameter}

\textbf{T0-DVFT Vakuumparameter-Übersicht:}

\begin{itemize}
\item Phasensteifigkeit: $B \approx \SI{8,7e-55}{}$
\item Trägheitsvakuumdichte: $\rho_0 = 1/\xi^2 \approx \SI{5,6e6}{}$ (oder $\sim \SI{6e-27}{kg/m^3}$)
\item Amplitudensteifigkeit: $K_0 \approx \SI{5,4e-10}{J/m^3}$
\item Fundamentaler Phasengradient für eine Ladung: $|\partial\theta/\partial x|_e \approx \SI{1,63e13}{m^{-1}}$
\item Kohärenzlänge: $L_{\text{coh}} \approx \sqrt{\hbar/B}$ → enorm (kosmische Skala)
\item Tieffeld-Beschleunigung: $a_0 \approx \SI{10e-10}{m/s^2}$
\item Lichtgeschwindigkeit: $c = \sqrt{K_0/\rho_0}$ (abgeleitet, nicht postuliert)
\item T0-Parameter: $\xi = 4/3 \times 10^{-4}$ (EINZIGER freier Parameter)
\end{itemize}

Zusammen definieren diese die intrinsische mechanische, elektromagnetische, quantenmechanische und gravitationale Struktur des T0-DVFT-Vakuums.

\section{Schlussfolgerung}

Die numerischen Vakuumparameter in T0-DVFT stimmen mit bekannten elektromagnetischen, quantenmechanischen und kosmologischen Beobachtungen überein. Durch Fixierung von $B$ aus $\alpha$, Ableitung von $\rho_0$ aus T0s einzigem Parameter $\xi$ und Verankerung von $K_0$ in der Kosmologie entsteht das gesamte quantenmechanische und gravitationale Framework aus einem einzigen vereinheitlichten Vakuumfeld:

\begin{equation}
\Phi(x,t) = \frac{1}{T(x,t) \cdot \xi} e^{i\theta(x,t)}
\end{equation}

Diese Parameter liefern die erste kohärente numerische Grundlage für eine Theorie, die vereinheitlicht:

\begin{itemize}
\item Spezielle Relativitätstheorie
\item Quantenmechanik
\item Elektromagnetismus
\item Neutrinophysik
\item Baryogenese
\item Dunkle Energie
\item Galaktische Dynamik (ohne dunkle Materie)
\item Thermodynamik und Entropie
\item Singularitätselimination
\end{itemize}

innerhalb eines einzigen feldbasierten Vakuum-Frameworks mit \textbf{null freien Parametern} (alle abgeleitet von der einzigen T0-Konstante $\xi$).

T0-DVFT demonstriert, dass das Vakuum kein leerer Raum ist, sondern ein dynamisches Medium, dessen intrinsische Eigenschaften alle fundamentale Physik bestimmen. Die Zeit-Masse-Dualität $T \cdot m = 1$ ist die ontologische Grundlage, aus der alle Konstanten, Kräfte und Phänomene hervorgehen.
