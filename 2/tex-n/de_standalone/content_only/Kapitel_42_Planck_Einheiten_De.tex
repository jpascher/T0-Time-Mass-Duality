CHAPTER 43: FUNDAMENTAL AXIOMS AND CONSTANTS
1. Core Axioms of the Dynamic Vacuum Field Theory (DVFT)
Axiom 1 — The Vacuum Is a Physical Medium
The vacuum is not empty. It is a structured, dynamic vacuum field Φ continuum with an amplitude ρ and
phase θ undergoes intrinsic Dynamic vacuum field, and matter acts as a local perturbation that modifies
this Dynamic vacuum field. The resulting phase and amplitude gradients propagate at light speed,
imprinting curvature onto spacetime.
Axiom 2 — All Forces and Particles Emerge from Vacuum Structure
Gauge interactions and gravity originate from the geometry and dynamics of the vacuum fields. Particles
are stable, localized excitations—either dynamic or topological—in these fields.
Axiom 3 — Light Is a Phase Wave of the Vacuum
Photons correspond to phase oscillations θ(x) of the vacuum. Their propagation speed is set by the ratio
of vacuum stiffness to inertial density.
Axiom 4 — Lorentz Invariance Emerges from Vacuum Uniformity
Uniform values of K₀ and ρ₀ across the vacuum ensure that all observers measure the same wave speed c.
Lorentz symmetry reflects the symmetry of the vacuum itself.
Axiom 5 — Gravity and Mass Arise from Vacuum Amplitude Φ
Local variations in Φ determine inertial response and generate spacetime curvature. Massive particles
require Φ excitation; photons do not.
Axiom 7 — Vacuum Constants K₀ and ρ₀ Are Fundamental
Electromagnetic constants ε₀ and μ₀ are not fundamental. They appear as effective parameters describing
how EM probes vacuum properties. The true constants are K₀ and ρ₀, which determine c.
2. Fundamental Constants of DVFT
• K₀ — Vacuum Stiffness Constant
- Resistance of vacuum phase to spatial distortion.
- Fundamental.
• ρ₀ — Vacuum Inertial Density Constant
International Journal for Multidisciplinary Research (IJFMR)
E-ISSN: 2582-2160 ● Website: www.ijfmr.com ● Email: editor@ijfmr.com
IJFMR250664112 Volume 7, Issue 6, November-December 2025 97
- Resistance to temporal acceleration of the vacuum phase.
- Fundamental.
• c — Speed of Light
- Derived from vacuum constants: c = √(K₀ / ρ₀).
• Φ₀ — Vacuum Amplitude (VEV)
- Determines gravitational coupling and vacuum energy scale.
• ε₀ — Electric Permittivity (Emergent)
- Effective: ε₀ ≈ ρ₀.
• μ₀ — Magnetic Permeability (Emergent)
- Effective: μ₀ ≈ 1 / K₀.
• ħ — Quantum of Action
- Fundamental quantum constant.
• G — Newton’s Gravitational Constant
- Couples vacuum energy to curvature.
3. Structural Relationships Among Constants
1. Speed of Light:
c = √(K₀ / ρ₀)
2. Electromagnetic Constants:
ε₀ = ρ₀ (effective)
μ₀ = 1/K₀ (effective)
3. Gravitational–Vacuum Relation:
G relates Φ₀-driven energy density to curvature.
4. Mass Generation:
m ∝ coupling × Φ₀
These show how classical constants emerge from deeper vacuum properties.
Conclusion: DVFT as a First-Principles Framework
DVFT redefines physics from the ground up by treating the vacuum itself as the foundational physical
entity. Rather than postulating constants like c, ε₀, and μ₀, DVFT derives them from intrinsic vacuum
constants K₀ and ρ₀.
This achieves:
• A physical origin for the speed of light,
• A unified vacuum origin for all forces,
• A mechanism for mass, curvature, and gauge symmetry breaking,
• A coherent interpretation connecting microphysics and cosmology.
DVFT proposes a universe where everything—energy, fields, particles, geometry—arises from a single
structured dynamic vacuum field.
REFERENCES
1. Einstein, A. (1915). Die Feldgleichungen der Gravitation. Sitzungsberichte der Preußischen Akademie
der Wissenschaften (Berlin), 844–847.
2. Einstein, A. (1905). Zur Elektrodynamik bewegter Körper. Annalen der Physik, 17, 891–921.
International Journal for Multidisciplinary Research (IJFMR)
E-ISSN: 2582-2160 ● Website: www.ijfmr.com ● Email: editor@ijfmr.com
IJFMR250664112 Volume 7, Issue 6, November-December 2025 98
3. Einstein, A. (1905). Ist die Trägheit eines Körpers von seinem Energieinhalt abhängig? Annalen der
Physik, 18, 639–641.
4. Schwarzschild, K. (1916). Über das Gravitationsfeld eines Massenpunktes nach der Einsteinschen
Theorie. Sitzungsberichte der Königlich Preußischen Akademie der Wissenschaften, 189–196.
5. Kerr, R. P. (1963). Gravitational field of a spinning mass as an example of algebraically special metrics.
Physical Review Letters, 11(5), 237–238. https://doi.org/10.1103/PhysRevLett.11.237
6. Hawking, S. W. (1975). Particle creation by black holes. Communications in Mathematical Physics,
43, 199–220. https://doi.org/10.1007/BF02345020
7. Wald, R. M. (1984). General Relativity. University of Chicago Press.
8. Misner, C. W., Thorne, K. S., & Wheeler, J. A. (1973). Gravitation. W. H. Freeman.
9. Carroll, S. M. (2004). Spacetime and Geometry: An Introduction to General Relativity.
Addison‑Wesley.
10. Weinberg, S. (1972). Gravitation and Cosmology: Principles and Applications of the General Theory
of Relativity. Wiley.
11. Will, C. M. (2014). The Confrontation between General Relativity and Experiment. Living Reviews
in Relativity, 17, 4. https://doi.org/10.12942/lrr-2014-4
12. Clifton, T., Ferreira, P. G., Padilla, A., & Skordis, C. (2012). Modified Gravity and Cosmology. Physics
Reports, 513(1), 1–189. https://doi.org/10.1016/j.physrep.2012.01.001
13. Planck Collaboration (Aghanim, N., et al.). (2020). Planck 2018 results. VI. Cosmological parameters.
Astronomy & Astrophysics, 641, A6. https://doi.org/10.1051/0004-6361/201833910
14. Riess, A. G., Yuan, W., Macri, L. M., et al. (2022). A Comprehensive Measurement of the Local Value
of the Hubble Constant with 1 km s−1 Mpc−1 Uncertainty from the Hubble Space Telescope and the
SH0ES Team. The Astrophysical Journal Letters, 934, L7. https://doi.org/10.3847/2041-8213/ac5c5b
15. Freedman, W. L., Madore, B. F., Hatt, D., et al. (2019). The Carnegie‑Chicago Hubble Program. VIII.
An Independent Determination of the Hubble Constant Based on the Tip of the Red Giant Branch. The
Astrophysical Journal, 882, 34.
16. Kolb, E. W., & Turner, M. S. (1990). The Early Universe. Addison‑Wesley.
17. Dodelson, S. (2003). Modern Cosmology. Academic Press.
18. Mukhanov, V. (2005). Physical Foundations of Cosmology. Cambridge University Press.
19. Guth, A. H. (1981). Inflationary universe: A possible solution to the horizon and flatness problems.
Physical Review D, 23, 347–356. https://doi.org/10.1103/PhysRevD.23.347
20. Linde, A. D. (1982). A New Inflationary Universe Scenario. Physics Letters B, 108(6), 389–393.
https://doi.org/10.1016/0370-2693(82)91219-9
21. Penzias, A. A., & Wilson, R. W. (1965). A Measurement of Excess Antenna Temperature at 4080 Mc/s.
The Astrophysical Journal, 142, 419–421. https://doi.org/10.1086/148307
22. Smoot, G. F., et al. (1992). Structure in the COBE differential microwave radiometer first-year maps.
The Astrophysical Journal Letters, 396, L1–L5. https://doi.org/10.1086/186504
23. Zwicky, F. (1933). Die Rotverschiebung von extragalaktischen Nebeln. Helvetica Physica Acta, 6,
110–127.
24. Rubin, V. C., & Ford, W. K. Jr. (1970). Rotation of the Andromeda Nebula from a Spectroscopic
Survey of Emission Regions. The Astrophysical Journal, 159, 379–403.
https://doi.org/10.1086/150317
International Journal for Multidisciplinary Research (IJFMR)
E-ISSN: 2582-2160 ● Website: www.ijfmr.com ● Email: editor@ijfmr.com
IJFMR250664112 Volume 7, Issue 6, November-December 2025 99
25. Bosma, A. (1978). The distribution and kinematics of neutral hydrogen in spiral galaxies of various
morphological types. PhD thesis, University of Groningen.
26. Navarro, J. F., Frenk, C. S., & White, S. D. M. (1996). The Structure of Cold Dark Matter Halos. The
Astrophysical Journal, 462, 563–575. https://doi.org/10.1086/177173
27. Tully, R. B., & Fisher, J. R. (1977). A new method of determining distances to galaxies. Astronomy &
Astrophysics, 54, 661–673.
28. McGaugh, S. S., Schombert, J. M., Bothun, G. D., & de Blok, W. J. G. (2000). The Baryonic Tully–
Fisher Relation. The Astrophysical Journal Letters, 533, L99–L102.


\section*{T0 Theory Integration}
This chapter integrates DVFT concepts with T0 Time-Mass Duality Theory, where the fundamental relation $T(x,t) \cdot m(x,t) = 1$ governs all vacuum field dynamics. The vacuum amplitude $\rho$ is directly related to local time $T$ through $\rho \propto 1/T$.
