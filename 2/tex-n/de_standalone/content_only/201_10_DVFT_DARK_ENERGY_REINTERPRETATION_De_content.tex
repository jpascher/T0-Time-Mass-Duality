\begin{enumerate}
  \item Introduction
\end{enumerate}
This document presents a strict DVFT-based derivation of dark energy, with no reference to external darkenergy models. The goal is to show how cosmic acceleration arises solely from the T0-Vakuum amplitude ?
and its microphysical potential U(?).
We derive the full equations for DVFT dark energy, specify U(?) from the DVFT micro-lattice model,
and compare DVFT predictions directly with observed cosmological values.
Fundamental DVFT T0-Vakuum field:
\Phi(x,t) = ?(x,t) e^{i$\theta$(x,t)}.
The universe?s large-scale behavior emerges from the homogeneous evolution of ?(t), while $\theta$(t) controls
quantum-phase structure.
\begin{enumerate}
  \item DVFT T0-Vakuum Lagrangian\footnote{Siehe auch T0-Dokument: \texttt{027_T0_lagrndian_De.pdf}} in a Homogeneous Universe
\end{enumerate}
From DVFT microphysics, the effective continuum T0-Vakuum Lagrangian is:
?_vac = (A_?/2)($\partial$_t ?)? - (B_?/2)|$\nabla$?|? + (A_$\theta$/2)??($\partial$_t $\theta$)? - (B_$\theta$/2)??|$\nabla$$\theta$|? - U(?).
For a homogeneous FRW universe (?(t), $\theta$(t), $\nabla$? = $\nabla$$\theta$ = 0):
?_hom = (A_?/2)??? + (A_$\theta$/2)?? $\theta$?? - U(?).
All cosmological dark-energy effects will arise directly from this expression. No additional fluids or fields
are introduced.
\begin{enumerate}
  \item T0-Vakuum Energy Density and Pressure from DVFT
\end{enumerate}
Define kinetic energy of the T0-Vakuum amplitude--phase system:
K = (A_?/2)??? + (A_$\theta$/2)?? $\theta$??.
DVFT T0-Vakuum behaves as a perfect fluid with:
?_DVFT = K + U(?),
p_DVFT = K - U(?).
The effective equation-of-state is:
w_DVFT = (K - U) / (K + U).
Important limits:
\begin{itemize}
  \item K ? U $\rightarrow$ w $\rightarrow$ -1 (dark-energy--like)
  \item K ~ U $\rightarrow$ -1 < w < -1/3 (dynamical dark energy)
  \item K ? U $\rightarrow$ w $\rightarrow$ +1 (stiff fluid; irrelevant today)
\end{itemize}
\begin{enumerate}
  \item Dark-Energy Evolution Equation in DVFT
\end{enumerate}
Varying the homogeneous action yields the amplitude evolution equation:
A_?(??+ 3H??) - A_$\theta$? $\theta$?? + dU/d? = 0,
where:
H = ?/a (Hubble parameter).
At late times, the cosmic phase tends to freeze on large scales ($\theta$? $\approx$ 0), reducing the equation to:
A_?(??+ 3H??) + dU/d? = 0.
International Journal for Multidisciplinary Research (IJFMR)
E-ISSN: 2582-2160 $\bullet$ Website: www.ijfmr.com $\bullet$ Email: editor@ijfmr.com
IJFMR250664112 Volume 7, Issue 6, November-December 2025 25
This is the DVFT dark-energy equation: the cosmic T0-Vakuum amplitude ? evolves in its potential U(?)
under Hubble damping.
\begin{enumerate}
  \item Microphysical Form of U(?) in DVFT
\end{enumerate}
DVFT is based on a micro-lattice T0-Vakuum with local Hamiltonian:
H_loc = p_??/(2M_?) + p_$\theta$?/(2M_$\theta$ ??) + U_loc(?).
DVFT microphysics requires U_loc(?) to have:
\begin{itemize}
  \item a stable minimum at ?? (preferred T0-Vakuum amplitude),
  \item positive curvature at ?? (T0-Vakuum stiffness),
  \item anharmonic corrections stabilizing deviations.
\end{itemize}
Thus the coarse-grained continuum potential becomes:
U(?) = \Lambda? + (?/2)(? - ??)? + (?/4)(? - ??)? + \ldots
Where:
\begin{itemize}
  \item \Lambda? = microphysical residual T0-Vakuum energy density,
  \item ? = T0-Vakuum amplitude compressibility,
  \item ? = higher-order stabilization.
\end{itemize}
Near the minimum:
U(?) $\approx$ \Lambda? + (1/2)m_?? (? - ??)?,
with m_?? = ?/A_?.
This U(?) is not arbitrary; it is derived from DVFT T0-Vakuum elasticity and amplitude stability.
\begin{enumerate}
  \item DVFT Explanation for Dark Energy on Cosmic Scales
\end{enumerate}
DVFT predicts dark energy because:
\begin{enumerate}
  \item The T0-Vakuum amplitude ? has a preferred value ?? (microphysical equilibrium).
  \item The local T0-Vakuum energy density U(??) = \Lambda? is *not zero*.
  \item On large scales, ?(t) approaches ?? and remains nearly constant due to strong Hubble damping.
  \item Therefore, the T0-Vakuum behaves like a nearly constant energy density with w $\approx$ -1.
\end{enumerate}
The measured value:
?_\Lambda $\approx$ 7 $\times$ 10??? kg/m?
\Omega_\Lambda $\approx$ 0.70--0.75
matches DVFT if:
\Lambda? = U(??) $\approx$ 0.7 ?_crit.
Thus dark energy is the ?elastic offset energy of the T0-Vakuum amplitude?
\begin{enumerate}
  \item Why U(?) Is Negligible on Solar and Galactic Scales
\end{enumerate}
A uniform T0-Vakuum energy density produces acceleration:
g_vac(r) $\approx$ (8?G/3) ?_\Lambda r.
At solar scale (r = 1 AU):
g_vac ~ 10??? m/s? (negligible).
At galactic scale (r = 10 kpc):
g_vac ~ 10??? m/s? (still negligible).
Thus:
\begin{itemize}
  \item Local dynamics are governed by $\nabla$? and matter coupling, not U(?).
  \item T0-Vakuum elasticity only influences cosmic expansion where r ~ gigaparsecs.
\end{itemize}
DVFT cleanly separates:
\begin{itemize}
  \item Galactic gravity: amplitude gradients $\nabla$? dominate.
\end{itemize}
International Journal for Multidisciplinary Research (IJFMR)
E-ISSN: 2582-2160 $\bullet$ Website: www.ijfmr.com $\bullet$ Email: editor@ijfmr.com
IJFMR250664112 Volume 7, Issue 6, November-December 2025 26
\begin{itemize}
  \item Cosmological acceleration: homogeneous U(??) dominates.
\end{itemize}
\begin{enumerate}
  \item Numerical Comparison with Observations
\end{enumerate}
Given:
\begin{itemize}
  \item H? $\approx$ 67--70 km/s/Mpc,
  \item ?_crit = 3H??/(8?G),
  \item \Omega_\Lambda $\approx$ 0.7,
\end{itemize}
DVFT requires:
U(??) = \Lambda? $\approx$ 0.7 ?_crit.
This matches observational values from CMB\footnote{Siehe auch T0-Dokument: \texttt{030_T0_CMB-Anomalie_De.pdf}}, BAO, and SN data.
Moreover, if ?(t) is still slowly relaxi\footnote{Siehe auch T0-Dokument: \texttt{009_T0_xi_ursprung_De.pdf}}ng toward ??, then:
w_DVFT $\approx$ -1 + 2K/U,
allowing mild deviations from -1 (observationally allowed), and potentially matching evolving darkenergy hints from DESI.
\begin{enumerate}
  \item Summary
\end{enumerate}
From strict DVFT principles, dark energy arises from the T0-Vakuum amplitude?s microphysical potential:
U(?) = \Lambda? + (?/2)(? - ??)? + (?/4)(? - ??)? + \ldots
Key results:
\begin{itemize}
  \item ?_DVFT = K + U(?), p_DVFT = K - U(?).
  \item w_DVFT = (K - U)/(K + U).
  \item T0-Vakuum amplitude evolves via A_?(??+ 3H??) + U'(?) = 0.
  \item On cosmic scales, ? $\approx$ ?? $\Rightarrow$ w $\approx$ -1, matching dark-energy observations.
  \item On solar/galactic scales, U(?) is negligible; $\nabla$? dominates gravity.
  \item DVFT dark energy matches measured values \Omega_\Lambda $\approx$ 0.7 and w $\approx$ -1 with no additional fields.
\end{itemize}
Thus DVFT naturally unifies local gravity and cosmic acceleration using only T0-Vakuum amplitude physics.

\section{Referenzen zu T0-Dokumenten}

Dieses Kapitel steht in Zusammenhang mit folgenden T0-Dokumenten im Repository \texttt{2/pdf/}:

\begin{itemize}
  \item \texttt{002\_T0\_Grundlagen\_De.pdf} -- T0 Zeit-Masse-Dualit?t Grundlagen
  \item \texttt{004\_T0\_Energie\_De.pdf} -- T0 Energiefeld-Theorie
  \item \texttt{009\_T0\_xi\_ursprung\_De.pdf} -- Ursprung des geometrischen Parameters $\xi$
  \item \texttt{201\_DVFT-alles\_De.pdf} -- Vollst?ndiges DVFT-Dokument (Rahmenwerk)
\end{itemize}