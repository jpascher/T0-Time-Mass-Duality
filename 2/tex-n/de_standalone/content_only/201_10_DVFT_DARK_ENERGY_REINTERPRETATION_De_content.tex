\begin{enumerate}
  \item Introduction
\end{enumerate}
This document presents a strict DVFT-based derivation of dark energy, with no reference to external darkenergy models. The goal is to show how cosmic acceleration arises solely from the T0-Vakuum amplitude ρ
and its microphysical potential U(ρ).
We derive the full equations for DVFT dark energy, specify U(ρ) from the DVFT micro-lattice model,
and compare DVFT predictions directly with observed cosmological values.
Fundamental DVFT T0-Vakuum field:
Φ(x,t) = ρ(x,t) e^{i$\theta$(x,t)}.
The universe’s large-scale behavior emerges from the homogeneous evolution of ρ(t), while $\theta$(t) controls
quantum-phase structure.
\begin{enumerate}
  \item DVFT T0-Vakuum Lagrangian\footnote{Siehe auch T0-Dokument: \texttt{027_T0_lagrndian_De.pdf}} in a Homogeneous Universe
\end{enumerate}
From DVFT microphysics, the effective continuum T0-Vakuum Lagrangian is:
𝓛_vac = (A_ρ/2)($\partial$_t ρ)² - (B_ρ/2)|$\nabla$ρ|² + (A_$\theta$/2)ρ²($\partial$_t $\theta$)² - (B_$\theta$/2)ρ²|$\nabla$$\theta$|² - U(ρ).
For a homogeneous FRW universe (ρ(t), $\theta$(t), $\nabla$ρ = $\nabla$$\theta$ = 0):
𝓛_hom = (A_ρ/2)ρ̇² + (A_$\theta$/2)ρ² $\theta$̇² - U(ρ).
All cosmological dark-energy effects will arise directly from this expression. No additional fluids or fields
are introduced.
\begin{enumerate}
  \item T0-Vakuum Energy Density and Pressure from DVFT
\end{enumerate}
Define kinetic energy of the T0-Vakuum amplitude--phase system:
K = (A_ρ/2)ρ̇² + (A_$\theta$/2)ρ² $\theta$̇².
DVFT T0-Vakuum behaves as a perfect fluid with:
ρ_DVFT = K + U(ρ),
p_DVFT = K - U(ρ).
The effective equation-of-state is:
w_DVFT = (K - U) / (K + U).
Important limits:
\begin{itemize}
  \item K ≪ U → w → -1 (dark-energy--like)
  \item K ~ U → -1 < w < -1/3 (dynamical dark energy)
  \item K ≫ U → w → +1 (stiff fluid; irrelevant today)
\end{itemize}
\begin{enumerate}
  \item Dark-Energy Evolution Equation in DVFT
\end{enumerate}
Varying the homogeneous action yields the amplitude evolution equation:
A_ρ(ρ̈+ 3Hρ̇) - A_$\theta$ρ $\theta$̇² + dU/dρ = 0,
where:
H = ȧ/a (Hubble parameter).
At late times, the cosmic phase tends to freeze on large scales ($\theta$̇ $\approx$ 0), reducing the equation to:
A_ρ(ρ̈+ 3Hρ̇) + dU/dρ = 0.
International Journal for Multidisciplinary Research (IJFMR)
E-ISSN: 2582-2160 $\bullet$ Website: www.ijfmr.com $\bullet$ Email: editor@ijfmr.com
IJFMR250664112 Volume 7, Issue 6, November-December 2025 25
This is the DVFT dark-energy equation: the cosmic T0-Vakuum amplitude ρ evolves in its potential U(ρ)
under Hubble damping.
\begin{enumerate}
  \item Microphysical Form of U(ρ) in DVFT
\end{enumerate}
DVFT is based on a micro-lattice T0-Vakuum with local Hamiltonian:
H_loc = p_ρ²/(2M_ρ) + p_$\theta$²/(2M_$\theta$ ρ²) + U_loc(ρ).
DVFT microphysics requires U_loc(ρ) to have:
\begin{itemize}
  \item a stable minimum at ρ₀ (preferred T0-Vakuum amplitude),
  \item positive curvature at ρ₀ (T0-Vakuum stiffness),
  \item anharmonic corrections stabilizing deviations.
\end{itemize}
Thus the coarse-grained continuum potential becomes:
U(ρ) = Λ₀ + (κ/2)(ρ - ρ₀)² + (λ/4)(ρ - ρ₀)⁴ + …
Where:
\begin{itemize}
  \item Λ₀ = microphysical residual T0-Vakuum energy density,
  \item κ = T0-Vakuum amplitude compressibility,
  \item λ = higher-order stabilization.
\end{itemize}
Near the minimum:
U(ρ) $\approx$ Λ₀ + (1/2)m_ρ² (ρ - ρ₀)²,
with m_ρ² = κ/A_ρ.
This U(ρ) is not arbitrary; it is derived from DVFT T0-Vakuum elasticity and amplitude stability.
\begin{enumerate}
  \item DVFT Explanation for Dark Energy on Cosmic Scales
\end{enumerate}
DVFT predicts dark energy because:
\begin{enumerate}
  \item The T0-Vakuum amplitude ρ has a preferred value ρ₀ (microphysical equilibrium).
  \item The local T0-Vakuum energy density U(ρ₀) = Λ₀ is *not zero*.
  \item On large scales, ρ(t) approaches ρ₀ and remains nearly constant due to strong Hubble damping.
  \item Therefore, the T0-Vakuum behaves like a nearly constant energy density with w $\approx$ -1.
\end{enumerate}
The measured value:
ρ_Λ $\approx$ 7 $\times$ 10⁻²⁷ kg/m³
Ω_Λ $\approx$ 0.70--0.75
matches DVFT if:
Λ₀ = U(ρ₀) $\approx$ 0.7 ρ_crit.
Thus dark energy is the “elastic offset energy of the T0-Vakuum amplitude”
\begin{enumerate}
  \item Why U(ρ) Is Negligible on Solar and Galactic Scales
\end{enumerate}
A uniform T0-Vakuum energy density produces acceleration:
g_vac(r) $\approx$ (8πG/3) ρ_Λ r.
At solar scale (r = 1 AU):
g_vac ~ 10⁻²⁴ m/s² (negligible).
At galactic scale (r = 10 kpc):
g_vac ~ 10⁻¹⁶ m/s² (still negligible).
Thus:
\begin{itemize}
  \item Local dynamics are governed by $\nabla$ρ and matter coupling, not U(ρ).
  \item T0-Vakuum elasticity only influences cosmic expansion where r ~ gigaparsecs.
\end{itemize}
DVFT cleanly separates:
\begin{itemize}
  \item Galactic gravity: amplitude gradients $\nabla$ρ dominate.
\end{itemize}
International Journal for Multidisciplinary Research (IJFMR)
E-ISSN: 2582-2160 $\bullet$ Website: www.ijfmr.com $\bullet$ Email: editor@ijfmr.com
IJFMR250664112 Volume 7, Issue 6, November-December 2025 26
\begin{itemize}
  \item Cosmological acceleration: homogeneous U(ρ₀) dominates.
\end{itemize}
\begin{enumerate}
  \item Numerical Comparison with Observations
\end{enumerate}
Given:
\begin{itemize}
  \item H₀ $\approx$ 67--70 km/s/Mpc,
  \item ρ_crit = 3H₀²/(8πG),
  \item Ω_Λ $\approx$ 0.7,
\end{itemize}
DVFT requires:
U(ρ₀) = Λ₀ $\approx$ 0.7 ρ_crit.
This matches observational values from CMB\footnote{Siehe auch T0-Dokument: \texttt{030_T0_CMB-Anomalie_De.pdf}}, BAO, and SN data.
Moreover, if ρ(t) is still slowly relaxi\footnote{Siehe auch T0-Dokument: \texttt{009_T0_xi_ursprung_De.pdf}}ng toward ρ₀, then:
w_DVFT $\approx$ -1 + 2K/U,
allowing mild deviations from -1 (observationally allowed), and potentially matching evolving darkenergy hints from DESI.
\begin{enumerate}
  \item Summary
\end{enumerate}
From strict DVFT principles, dark energy arises from the T0-Vakuum amplitude’s microphysical potential:
U(ρ) = Λ₀ + (κ/2)(ρ - ρ₀)² + (λ/4)(ρ - ρ₀)⁴ + …
Key results:
\begin{itemize}
  \item ρ_DVFT = K + U(ρ), p_DVFT = K - U(ρ).
  \item w_DVFT = (K - U)/(K + U).
  \item T0-Vakuum amplitude evolves via A_ρ(ρ̈+ 3Hρ̇) + U'(ρ) = 0.
  \item On cosmic scales, ρ $\approx$ ρ₀ ⇒ w $\approx$ -1, matching dark-energy observations.
  \item On solar/galactic scales, U(ρ) is negligible; $\nabla$ρ dominates gravity.
  \item DVFT dark energy matches measured values Ω_Λ $\approx$ 0.7 and w $\approx$ -1 with no additional fields.
\end{itemize}
Thus DVFT naturally unifies local gravity and cosmic acceleration using only T0-Vakuum amplitude physics.

\section{Referenzen zu T0-Dokumenten}

Dieses Kapitel steht in Zusammenhang mit folgenden T0-Dokumenten im Repository \texttt{2/pdf/}:

\begin{itemize}
  \item \texttt{002\_T0\_Grundlagen\_De.pdf} -- T0 Zeit-Masse-Dualität Grundlagen
  \item \texttt{004\_T0\_Energie\_De.pdf} -- T0 Energiefeld-Theorie
  \item \texttt{009\_T0\_xi\_ursprung\_De.pdf} -- Ursprung des geometrischen Parameters $\xi$
  \item \texttt{201\_DVFT-alles\_De.pdf} -- Vollständiges DVFT-Dokument (Rahmenwerk)
\end{itemize}