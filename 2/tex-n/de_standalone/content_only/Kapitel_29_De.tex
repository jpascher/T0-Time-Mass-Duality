\section{T0-Theory: The Delayed Choice Quantum Eraser Experiment}

The **Delayed Choice Quantum Eraser (DCQE)** experiment is one of the most intriguing demonstrations in quantum physics. It appears to suggest retrocausality or that a future measurement influences the past behavior of a photon. This section analyzes the experiment within the framework of the **T0 Time-Mass Duality Theory**. The T0 interpretation eliminates retrocausality entirely by showing that the phenomenon arises from **fractal phase coherence** in the intrinsic time field $T(x,t)$. The DCQE involves preservation, disruption, or restoration of phase coherence in the fractal vacuum field—not backward causation.

\subsection{Vacuum Field Structure in T0 Theory}

In T0 theory, quantum states emerge from excitations of the universal time-mass field satisfying the duality
\begin{equation}
	T(x,t) \cdot E(x,t) = 1
\end{equation}
with fractal correction from the geometric parameter $\xi = \frac{4}{3} \times 10^{-4}$ and effective dimension $D_f = 3 - \xi \approx 2.99987$.

The field can be expressed in polar form as
\begin{equation}
	T(x,t) = \rho(x,t) \, e^{i\theta(x,t)}
\end{equation}
where $\rho(x,t)$ is the amplitude and $\theta(x,t)$ is the intrinsic phase modulated by fractal geometry.

Interference patterns arise from stable relative phases between field paths:
\begin{equation}
	I(x) = |\Tfield_1(x) + \Tfield_2(x)|^2
\end{equation}
When the phase difference $\Delta\theta = \theta_1 - \theta_2$ remains constant (modulo fractal damping $e^{-\xi \cdot n}$), interference is visible. Randomization or tagging of $\Delta\theta$ destroys interference.

\subsection{What Happens After the Slits}

After the beam splitter or slits, the time field splits into two coherent branches:
\begin{equation}
	\Tfield = \Tfield_1 + \Tfield_2
\end{equation}
This coherence reflects real structure in the fractal phase $\theta(x,t)$. Interference emerges when
\begin{equation}
	\Delta\theta = \constant
\end{equation}
(within the $\xi$-damping scale). Thus, interference is a **fractal phase-coherence phenomenon** in the T0 vacuum, not a probabilistic artifact.

\subsection{Which-Path Information as Fractal Phase Decoherence}

Insertion of which-path markers entangles the branches with a macroscopic system, introducing fractal phase perturbations:
\begin{align}
	\theta_1 &\to \theta_1 + \delta\theta_1(\xi) \\
	\theta_2 &\to \theta_2 + \delta\theta_2(\xi)
\end{align}
with $\delta\theta_1 \neq \delta\theta_2$ on scales larger than $\xi^{-1}$. The relative phase $\Delta\theta$ becomes undefined due to fractal damping. Interference vanishes because the phase gradients no longer align coherently.

\subsection{The Quantum Eraser Restores Fractal Phase Coherence}

The eraser does not alter the past. It modifies the boundary conditions of the time field by erasing which-path entanglement. This restores
\begin{equation}
	\Delta\theta = \constant
\end{equation}
but only for the correlated subset of events selected in coincidence counting. Interference reappears exclusively in those subsets due to restored fractal coherence.

\subsection{Why Delayed Choice Implies No Retrocausality in T0}

In the T0 framework:
\begin{itemize}
	\item The intrinsic time field $T(x,t)$ spans the entire apparatus globally.
	\item Phase coherence or decoherence is a geometric property of the fractal field, not local.
	\item Coincidence sorting selects events consistent with the restored phase relation.
\end{itemize}
No signal propagates backward. The field already encodes all correlations via the duality $T \cdot E = 1$. The delayed choice merely classifies events by their fractal phase compatibility.

\subsection{T0 Equations for Interference and Decoherence}

Full interference:
\begin{equation}
	I(x) = |\Tfield_1(x) + \Tfield_2(x)|^2
\end{equation}

With which-path marking (decoherence):
\begin{equation}
	\Tfield \to \Tfield_1 e^{i\delta\theta_1(\xi)} + \Tfield_2 e^{i\delta\theta_2(\xi)}, \quad \Delta\theta \to \undefined
\end{equation}

Eraser restoration:
\begin{equation}
	\delta\theta_1(\xi) = \delta\theta_2(\xi) \implies \Delta\theta = \constant
\end{equation}
Interference reappears only in the selected coincidence subset.

\subsection{Photon Behavior in T0 Theory}

In T0:
\begin{itemize}
	\item A photon is a localized excitation on the fractal time field.
	\item Its "trajectory" is guided by geometric phase gradients in $T(x,t)$.
	\item Which-path detection perturbs the fractal phase structure.
	\item Erasure reconstructs the coherent phase geometry.
\end{itemize}
This resolves paradoxes without invoking retrocausality or observer dependence.

\subsection{Conclusion}

The Delayed Choice Quantum Eraser experiment requires no retrocausality. T0 theory provides a deterministic, geometric explanation: the fractal phase of the intrinsic time field $T(x,t)$ determines interference visibility. Which-path information disrupts fractal coherence; erasure restores it in correlated subsets. The delayed choice affects event classification, not occurrence. T0 thus unifies DCQE with geometric intuition while reproducing all quantum predictions through the time-mass duality and $\xi$-fractality.