\documentclass[12pt,a4paper]{article}
\usepackage[utf8]{inputenc}
\usepackage{amsmath,amssymb}
\usepackage{hyperref}
\usepackage{geometry}
\geometry{margin=2.5cm}

\title{{Chapter 43: Fundamentale Axiome und Konstanten}}
\author{{Dynamic Vacuum Field Theory with T0 Adaptations}}
\date{{\today}}

\begin{document}
\maketitle

29. McGaugh, S. S. (2005). The Baryonic Tully–Fisher Relation of Galaxies with Extended Rotation
Curves and the Stellar Mass of Rotating Galaxies. The Astrophysical Journal, 632, 859–871.
30. Lelli, F., McGaugh, S. S., & Schombert, J. M. (2016). SPARC: Mass Models for 175 Disk Galaxies
with Spitzer Photometry and Accurate Rotation Curves. The Astronomical Journal, 152, 157.
https://doi.org/10.3847/0004-6256/152/6/157
31. Milgrom, M. (1983). A modification of the Newtonian dynamics as a possible alternative to the hidden
mass hypothesis. The Astrophysical Journal, 270, 365–370. https://doi.org/10.1086/161130
32. Bekenstein, J. D. (2004). Relativistic gravitation theory for the modified Newtonian dynamics
paradigm. Physical Review D, 70, 083509. https://doi.org/10.1103/PhysRevD.70.083509
33. Horndeski, G. W. (1974). Second-order scalar-tensor field equations in a four-dimensional space.
International Journal of Theoretical Physics, 10, 363–384. https://doi.org/10.1007/BF01807638
34. Gubitosi, G., Piazza, F., & Vernizzi, F. (2012). The Effective Field Theory of Dark Energy.
arXiv:1210.0201.
35. Frusciante, N., & Perenon, L. (2020). Effective Field Theory of Dark Energy: a review. Physics
Reports, 857, 1–63. https://doi.org/10.1016/j.physrep.2020.02.004
36. Woodard, R. P. (2015). Ostrogradsky’s theorem on Hamiltonian instability. Scholarpedia, 10(8),
32243. https://doi.org/10.4249/scholarpedia.32243
37. Motohashi, H., & Suyama, T. (2015). Third order equations of motion and the Ostrogradsky instability.
Physical Review D, 91, 085009. https://doi.org/10.1103/PhysRevD.91.085009
38. Langlois, D. (2017). Degenerate Higher‑Order Scalar‑Tensor (DHOST) theories. arXiv:1707.03625.
39. Ben Achour, J., Crisostomi, M., Koyama, K., Langlois, D., & Noui, K. (2016). Degenerate higher
order scalar-tensor theories beyond Horndeski and disformal transformations. Physical Review D, 93,
124005. https://doi.org/10.1103/PhysRevD.93.124005
40. Creminelli, P., & Vernizzi, F. (2017). Dark Energy after GW170817 and GRB170817A. Physical
Review Letters, 119, 251302. https://doi.org/10.1103/PhysRevLett.119.251302
41. Ezquiaga, J. M., & Zumalacárregui, M. (2017). Dark Energy after GW170817: dead ends and the road
ahead. Physical Review Letters, 119, 251304. https://doi.org/10.1103/PhysRevLett.119.251304
42. Langlois, D., Ezquiaga, J. M., & Zumalacárregui, M. (2018). Scalar-tensor theories and modified
gravity in the wake of GW170817. Physical Review D, 97, 061501(R).
https://doi.org/10.1103/PhysRevD.97.061501
43. Abbott, B. P., et al. (LIGO Scientific Collaboration and Virgo Collaboration). (2017). GW170817:
Observation of Gravitational Waves from a Binary Neutron Star Inspiral. Physical Review Letters,
119, 161101. https://doi.org/10.1103/PhysRevLett.119.161101
International Journal for Multidisciplinary Research (IJFMR)
E-ISSN: 2582-2160 ● Website: www.ijfmr.com ● Email: editor@ijfmr.com
IJFMR250664112 Volume 7, Issue 6, November-December 2025 100
44. Abbott, B. P., et al. (LIGO Scientific Collaboration and Virgo Collaboration). (2017). Multi-messenger
Observations of a Binary Neutron Star Merger. The Astrophysical Journal Letters, 848, L12–L16.
https://doi.org/10.3847/2041-8213/aa91c9
45. Abbott, B. P., et al. (LIGO Scientific Collaboration and Virgo Collaboration). (2019). Tests of General
Relativity with the Binary Black Hole Signals from the LIGO‑Virgo Catalog GWTC‑1. Physical
Review D, 100, 104036. https://doi.org/10.1103/PhysRevD.100.104036
46. Eardley, D. M., Lee, D. L., Lightman, A. P., Wagoner, R. V., & Will, C. M. (1973). Gravitational-wave
observations as a tool for testing relativistic gravity. Physical Review Letters, 30, 884–886.
https://doi.org/10.1103/PhysRevLett.30.884
47. Nishizawa, A., Taruya, A., Hayama, K., Kawamura, S., & Sakagami, M. (2009). Probing non-tensorial
polarizations of stochastic gravitational-wave backgrounds with ground-based laser interferometers.
Physical Review D, 79, 082002. https://doi.org/10.1103/PhysRevD.79.082002
48. Vainshtein, A. I. (1972). To the problem of nonvanishing gravitation mass. Physics Letters B, 39(3),
393–394. https://doi.org/10.1016/0370-2693(72)90147-5
49. Babichev, E., & Deffayet, C. (2013). An introduction to the Vainshtein mechanism. Classical and
Quantum Gravity, 30(18), 184001. https://doi.org/10.1088/0264-9381/30/18/184001
50. Khoury, J., & Weltman, A. (2004). Chameleon cosmology. Physical Review D, 69, 044026.
https://doi.org/10.1103/PhysRevD.69.044026
51. Burrage, C., & Sakstein, J. (2018). Tests of Chameleon Gravity. Living Reviews in Relativity, 21, 1.
https://doi.org/10.1007/s41114-018-0011-x
52. Schrödinger, E. (1926). Quantisierung als Eigenwertproblem (Parts I–IV). Annalen der Physik, 79–81
(1926).
53. Heisenberg, W. (1927). Über den anschaulichen Inhalt der quantentheoretischen Kinematik und
Mechanik. Zeitschrift für Physik, 43, 172–198. https://doi.org/10.1007/BF01397280
54. Born, M. (1926). Zur Quantenmechanik der Stoßvorgänge. Zeitschrift für Physik, 37, 863–867.
https://doi.org/10.1007/BF01397477
55. von Neumann, J. (1932). Mathematische Grundlagen der Quantenmechanik. Springer (English transl.:
Mathematical Foundations of Quantum Mechanics, Princeton Univ. Press, 1955).
56. Sakurai, J. J., & Napolitano, J. (2017). Modern Quantum Mechanics (2nd ed.). Cambridge University
Press.
57. Zurek, W. H. (2003). Decoherence, einselection, and the quantum origins of the classical. Reviews of
Modern Physics, 75, 715–775. https://doi.org/10.1103/RevModPhys.75.715
58. Joos, E., Zeh, H. D., Kiefer, C., Giulini, D., Kupsch, J., & Stamatescu, I.-O. (2003). Decoherence and
the Appearance of a Classical World in Quantum Theory (2nd ed.). Springer.
https://doi.org/10.1007/978-3-662-05328-7
59. Yang, C. N., & Mills, R. L. (1954). Conservation of isotopic spin and isotopic gauge invariance.
Physical Review, 96(1), 191–195. https://doi.org/10.1103/PhysRev.96.191
60. Faddeev, L. D., & Popov, V. N. (1967). Feynman diagrams for the Yang–Mills field. Physics Letters
B, 25(1), 29–30. https://doi.org/10.1016/0370-2693(67)90067-6
61. Peskin, M. E., & Schroeder, D. V. (1995). An Introduction to Quantum Field Theory. Addison‑Wesley.
62. Weinberg, S. (1995). The Quantum Theory of Fields, Vol. I: Foundations. Cambridge University Press.
63. Clay Mathematics Institute. (2000–present). Yang–Mills existence and mass gap (Millennium Prize
Problem). https://www.claymath.org/millennium/yang-mills-the-maths-gap/
International Journal for Multidisciplinary Research (IJFMR)
E-ISSN: 2582-2160 ● Website: www.ijfmr.com ● Email: editor@ijfmr.com
IJFMR250664112 Volume 7, Issue 6, November-December 2025 101
64. Jaffe, A. (2000). Quantum Yang–Mills Theory (CMI Millennium Prize Problem description; Jaffe–
Witten). Clay Mathematics Institute. (PDF commonly circulated as “yangmills.pdf”).
65. Sakharov, A. D. (1967). Violation of CP invariance, C asymmetry, and baryon asymmetry of the
universe. JETP Letters, 5, 24–27.
66. Penrose, R. (1996). On Gravity’s role in Quantum State Reduction. General Relativity and Gravitation,
28, 581–600. https://doi.org/10.1007/BF02105068
67. Diósi, L. (1989). Models for universal reduction of macroscopic quantum fluctuations. Physical
Review A, 40, 1165–1174. https://doi.org/10.1103/PhysRevA.40.1165
68. Bassi, A., Lochan, K., Satin, S., Singh, T. P., & Ulbricht, H. (2013). Models of wave-function collapse,
underlying theories, and experimental tests. Reviews of Modern Physics, 85, 471–527.
https://doi.org/10.1103/RevModPhys.85.471
69. Arndt, M., & Hornberger, K. (2014). Testing the limits of quantum mechanical superpositions. Nature
Physics, 10, 271–277. https://doi.org/10.1038/nphys2863
70. Marletto, C., & Vedral, V. (2017). Gravitationally Induced Entanglement between Two Massive
Particles is Sufficient Evidence of Quantum Effects in Gravity. Physical Review Letters, 119, 240402.
https://doi.org/10.1103/PhysRevLett.119.240402
71. Margalit, Y., Dobkowski, O., Zhou, Z., et al. (2021). Realization of a complete Stern–Gerlach
interferometer: Toward a test of quantum gravity. Science Advances, 7(22), eabg2879.
https://doi.org/10.1126/sciadv.abg2879
72. Roura, A. (2020). Gravitational Redshift in Quantum-Clock Interferometry. Physical Review X, 10,
021014. https://doi.org/10.1103/PhysRevX.10.021014
73. Dobkowski, O., Trok, B., Skakunenko, P., et al. (2025). Observation of the quantum equivalence
principle for matter-waves. arXiv:2502.14535.
74. This paper positions Dynamic Vacuum Field Theory (DVFT) as a transformative approach to unifying
general relativity, quantum mechanics, and cosmology by reimagining space as a dynamic vacuum


\section*{T0 Theory Integration}
This chapter integrates DVFT concepts with T0 Time-Mass Duality Theory, where the fundamental relation $T(x,t) \cdot m(x,t) = 1$ governs all vacuum field dynamics. The vacuum amplitude $\rho$ is directly related to local time $T$ through $\rho \propto 1/T$.

\end{document}
