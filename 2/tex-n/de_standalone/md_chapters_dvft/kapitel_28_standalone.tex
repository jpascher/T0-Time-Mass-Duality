\documentclass[12pt,a4paper]{article}

% Packages
\usepackage[utf8]{inputenc}
\usepackage[T1]{fontenc}
\usepackage{amsmath}
\usepackage{amssymb}
\usepackage{physics}
\usepackage{graphicx}
\usepackage{hyperref}
\usepackage[margin=2.5cm]{geometry}

% Physics notation
\renewcommand{\varphi}{\phi}

\title{Kapitel 28
}
\author{DVFT - Dynamic Vacuum Field Theory}
\date{\today}

\begin{document}

\section{Kapitel 28
}


GRAVITY AT QUANTUM SCALE
1. Introduction
This document explains why Newton’s Law does not fundamentally apply to gravity between individual
protons, and how DVFT (Dynamic vacuum field Curvature Theory) provides the first self-consistent
gravitational framework at quantum scales.
DVFT treats gravity not as classical curvature but as a deformation of vacuum amplitude:
Φ = \rho e^{i\theta},
where:
• \rho(x,t) = vacuum amplitude → inertia \& gravity
• \theta(x,t) = vacuum phase → quantum behavior
This allows DVFT to define gravity for localized, delocalized, or superposed quantum states a task that
standard GR and Newtonian gravity cannot accomplish without contradiction.
2. Why Newton’s Law Does Not Fundamentally Apply to Protons
Newton’s Law:
F = G m₁ m₂ / r²
works only when:
• objects are classical point masses,
• positions are definite,
• spacetime is continuous.
A proton violates all of these assumptions. It is:
• a quantum wave packet,
• composite (quarks + gluons),
• position-indeterminate,
• governed by vacuum phase \theta, not classical mass density.
Thus applying Newton’s law to protons is not physically correct — it is merely an approximate numerical
shortcut for highly localized states.
3. DVFT: Gravity Comes From Vacuum Amplitude, Not Classical Mass
DVFT defines gravity through vacuum amplitude deformation:
g(x) = −\nabla\rho(x).
International Journal for Multidisciplinary Research (IJFMR)
E-ISSN: 2582-2160 ● Website: www.ijfmr.com ● Email: editor@ijfmr.com
IJFMR250664112 Volume 7, Issue 6, November-December 2025 65
A proton creates a small amplitude bump δ\rho(x):
\rho(x) = \rho₀ + δ\rho(x).
The gravitational field behaves as:
g(r) = G m_p / r²
ONLY when the proton’s wave function is extremely localized.
If the proton is quantum-delocalized, its gravitational field becomes delocalized. Newton’s formula no
longer applies.
4. The Correct DVFT Gravitational Field of a Proton
A proton with wavefunction \psi(x) produces amplitude distortion:
δ\rho_p(x) = G m_p |\psi(x)|² * (1/r).
Its gravitational field is:
g(x) = −\nabla\rho(x).
Thus gravity reflects the \textit{quantum probability distribution}, not a classical point.
This is something general relativity cannot describe without inconsistency.
5. Protons in Quantum Superposition
Let a proton be in the superposition:
|\psi⟩ = (|L⟩ + |R⟩)/\sqrt2.
Newton’s law breaks immediately because:
• r is undefined,
• there is no single mass location,
• force cannot be computed.
DVFT solves this cleanly:
\rho(x) = \rho₀ + G m_p |\psi(x)|².
Gravity is sourced not by “two protons” but by a single distributed amplitude. This keeps both quantum
linearity and gravitational consistency intact.
Thus DVFT predicts:
• A superposed proton produces a single smooth gravitational field.
• Gravity does not collapse quantum states.
• Gravity remains well-defined without classical positions.
6. Two Protons Both in Superposition
If both protons have wavefunctions \psi₁(x) and \psi₂(x), DVFT gives:
\rho(x) = \rho₀ + G m_p(|\psi₁(x)|² + |\psi₂(x)|²).
Their mutual gravitational interaction depends on:
• wavefunction overlap,
• spatial spread,
• relative phase structure.
This is impossible to formulate in Newtonian or GR frameworks but trivial in DVFT.
7. Why Newtonian Gravity Works Only in the Classical Limit
Newton’s Law becomes a good approximation ONLY when:
• proton is highly localized,
• wavefunction spread ≪ separation distance.
Then:
|\psi(x)|² \approx δ³(x − x₀)
International Journal for Multidisciplinary Research (IJFMR)
E-ISSN: 2582-2160 ● Website: www.ijfmr.com ● Email: editor@ijfmr.com
IJFMR250664112 Volume 7, Issue 6, November-December 2025 66
and the amplitude distortion becomes point-like.
DVFT therefore explains why classical gravity emerges at large scales, yet fails at quantum scales.
8. Numerical Example: Gravity Between Two Protons
At r = 10⁻¹⁰ m (atomic distance), Gravitational force:
F_g \approx 2\times10⁻⁴⁴ N.
Gravitational acceleration:
a_g \approx 1\times10⁻¹⁷ m/s².
Electromagnetic force at same distance:
F_E \approx 2\times10⁻⁸ N.
Ratio:
F_E / F_g \approx 10³⁶.
Thus gravity between single protons is negligible — but in DVFT it has a clean quantum definition, unlike
in GR or Newtonian theory.
Conclusion
DVFT resolves deep inconsistencies in combining quantum mechanics with gravity:
• Newtonian gravity is NOT fundamental and fails for quantum particles.
• GR cannot define gravity of a quantum wavefunction.
• DVFT defines gravity as vacuum amplitude deformation \rho(x), valid for both localized and
superposed states.
• A proton in superposition does NOT produce two fields — it produces one unified field ∝ |\psi|².
• Classical gravity emerges only when wave functions become localized.
DVFT is therefore the first framework that consistently describes gravity at quantum scales without
contradiction.

\end{document}
