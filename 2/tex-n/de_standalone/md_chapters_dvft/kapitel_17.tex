% kapitel_17.tex – Vollständig korrigierte und kompilierbare Version
\section{Alternative zu GR + Lambda-CDM in T0}

T0 stellt eine fundamentale, parameterfreie Alternative zur Allgemeinen Relativitätstheorie (GR) kombiniert mit dem Lambda-CDM-Modell dar. Alle beobachteten kosmologischen und gravitativen Phänomene werden durch den einzigen Skalenparameter \(\xi\) erklärt – ohne Dunkle Komponenten oder Singularitäten.

\subsection{Das Lambda-CDM-Modell und seine Probleme}

Das Standardmodell basiert auf den Friedmann-Gleichungen
\begin{align}
	\left( \frac{\dot{a}}{a} \right)^2 &= \frac{8\pi G}{3} (\rho_m + \rho_r + \rho_\Lambda) - \frac{k}{a^2}, \\
	\frac{\ddot{a}}{a} &= -\frac{4\pi G}{3} (\rho_m + \rho_r + 3p_m + 3p_r) + \frac{\Lambda}{3},
\end{align}
mit sechs freien Parametern (\(\Omega_m, \Omega_r, \Omega_\Lambda, \Omega_k, H_0, w\)) und zusätzlichen Annahmen (Inflaton-Feld, Dunkle Materie-Partikel).

Probleme:
\begin{itemize}
	\item Kosmologische Konstanten-Problem: \(\rho_\Lambda^{\text{QFT}} / \rho_\Lambda^{\text{obs}} \approx 10^{120}\),
	\item Feinabstimmung von \(\Omega_\Lambda \approx \Omega_m\) heute (Koinzidenzproblem),
	\item Keine Erklärung für Galaxierotationskurven ohne Dunkle Materie.
\end{itemize}

\subsection{Fraktale T0-Wirkung – Vollständige Ableitung}

Die fundamentale T0-Wirkung ist
\begin{equation}
	S = \int \sqrt{-g} \left[ \frac{R}{16\pi G} + \xi \cdot \rho_0^2 \left( (\partial_\mu \ln a)^2 + \sum_{k=1}^\infty \xi^k (\nabla^k \ln a)^2 \right) + \mathcal{L}_m \right] d^4x,
\end{equation}
wobei der fraktale Term die Selbstähnlichkeit über Hierarchiestufen \(k\) enkodiert.

Durch Resummation der Reihe:
\begin{equation}
	\sum_{k=1}^\infty \xi^k (\nabla^k \ln a)^2 \approx \xi \cdot \frac{(\nabla \ln a)^2}{1 - \xi (\nabla l_0)^2},
\end{equation}
wobei \(l_0\) die fundamentale T0-Länge ist.

\subsection{Ableitung der modifizierten Friedmann-Gleichungen}

Variation nach dem Skalenfaktor \(a(t)\) (FRW-Metrik \(ds^2 = -dt^2 + a^2(t) d\vec{x}^2\)) liefert
\begin{equation}
	\left( \frac{\dot{a}}{a} \right)^2 = \frac{8\pi G}{3} \rho_m + \xi \cdot \frac{c^2}{l_0^2 a^4} \left( 1 + \xi \ln a + \xi^{1/2} \langle \delta^2 \rangle \right),
\end{equation}
und die Beschleunigungsgleichung
\begin{equation}
	\frac{\ddot{a}}{a} = -\frac{4\pi G}{3} (\rho_m + 3p_m) + \xi \cdot \frac{c^2}{l_0^2 a^4} \left( 1 - 3\xi \ln a - 2\xi^{1/2} \langle \delta^2 \rangle \right).
\end{equation}

Der Term \(\xi \cdot c^2 / (l_0^2 a^4)\) dominiert früh und reguliert die Singularität, während der Backreaction-Term \(\langle \delta^2 \rangle\) die Strukturbildung berücksichtigt.

\subsection{Vollständige Lösung für das späte Universum}

Für \(a \gg 1\):
\begin{equation}
	H^2(a) \approx H_0^2 \left( \Omega_b a^{-3} + \xi^2 \left(1 + \xi^{1/2} \frac{\langle \delta^2 \rangle}{a^3} \right) \right).
\end{equation}

Der effektive \(\Omega_\Lambda^{\text{eff}} = \xi^2 \approx 0.7\), exakt passend zu Beobachtungen, ohne Feinabstimmung.

\subsection{Vergleich mit Lambda-CDM}

\begin{itemize}
	\item Lambda-CDM: 6+ freie Parameter, separate Dunkle Komponenten, Inflation ad-hoc,
	\item T0: Nur \(\xi\), alles abgeleitet, keine Dunklen Komponenten, natürliche Regularisierung.
\end{itemize}

\subsection{Schluss}

T0 ist nicht nur eine Alternative, sondern eine tiefere Theorie: GR + Lambda-CDM emergieren als effektive Grenzfälle der fraktalen Time-Mass-Duality. Alle kosmologischen Daten werden parameterfrei reproduziert.