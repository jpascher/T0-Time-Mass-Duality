% Chapter: Kapitel 15
\section{Kapitel 15
}


MERCURY PERIHELION PRECESSION
1. Introduction
This chapter derives the perihelion precession of Mercury using ONLY the Dynamic Vacuum Field
Theory (DVFT), without invoking Einstein’s General Relativity field equations. The key idea is that in
International Journal for Multidisciplinary Research (IJFMR)
E-ISSN: 2582-2160 ● Website: www.ijfmr.com ● Email: editor@ijfmr.com
IJFMR250664112 Volume 7, Issue 6, November-December 2025 36
the high-acceleration regime of the Solar System, DVFT reduces to a Newtonian potential plus a tiny 1/r³
correction generated by the \theta-field dynamics. This correction leads to the correct 43 arcsec/century
precession.
2. DVFT in the Solar System: High-Acceleration Limit
DVFT describes gravity as arising from convergence of a vacuum phase field \theta. Its Lagrangian contains
nonlinear terms:
L_\theta = -\Lambda_v + (\rho₀/2)X - (\eta/(3 a₀²)) X^{3/2},
with X = -g^{\mu\nu} \partial_\mu\theta \partial_\nu\theta.
In the Solar System, gravitational acceleration is much larger than a₀ (~10⁻¹⁰ m/s²):
g / a₀ ~ 10⁹.
Thus, nonlinear MOND/DVFT corrections vanish. DVFT reduces to a GR-like weak-field theory,
predicting an effective potential of the form:
U_eff(r) = -GMm/r + L²/(2mr²) - GM L²/(mc² r³).
3. DVFT Effective Potential for Mercury
The effective central-force potential for a test mass m orbiting the Sun in DVFT becomes:
U_DVFT(r) = -GMm/r + L²/(2mr²) - (GM L²)/(m c² r³).
Terms:
• −GMm/r : Newtonian gravity,
• L²/(2mr²) : centrifugal barrier,
• −GM L²/(m c² r³) : DVFT high-g correction.
This 1/r³ term is responsible for perihelion precession.
4. Orbit Equation Using Classical Mechanics Only
Define u(\varphi) = 1/r. The Binet equation for a central potential U(r) is:
d²u/d\varphi² + u = -(m / L²u²) (dU/dr).
Convert U(r) to U(u):
U(u) = -k u + (L²/2m)u² + \beta u³,
where k = GMm, \beta = −GM L²/(m c²).
Taking the derivative and substituting into Binet’s equation yields:
d²u/d\varphi² + (mk/L²) = (3m\beta/L²) u².
The \beta-term represents the DVFT correction. For \beta=0, this gives perfect ellipses.
5. Perturbative Solution and Precession
Using the unperturbed solution:
u₀(\varphi) = (mk/L²)(1 + e cos\varphi),
and treating \beta as a small parameter, the first-order perturbation yields a precession per orbit:
\Delta\varphi = 6\pi k² / (L² c² (1−e²)).
Substitute k = GMm and L² = m²GM a(1−e²):
\Delta\varphi = 6\pi GM / (a (1−e²) c²).
This equation can be used to calculate the perihelion precession for Mercury.
6. Input Physical Constants and Mercury Parameters
• Gravitational constant: G = 6.6743 \times 10⁻¹¹ m³ kg⁻¹ s⁻²
• Solar mass: M = 1.9885 \times 10³⁰ kg
• → GM = 1.3271 \times 10²⁰ m³ s⁻²
International Journal for Multidisciplinary Research (IJFMR)
E-ISSN: 2582-2160 ● Website: www.ijfmr.com ● Email: editor@ijfmr.com
IJFMR250664112 Volume 7, Issue 6, November-December 2025 37
• Speed of light: c = 2.9979 \times 10⁸ m/s
• → c² = 8.9876 \times 10¹⁶ m² s⁻²
• Mercury semi-major axis: a = 5.7909 \times 10¹⁰ m
• Mercury orbital eccentricity: e = 0.2056
• Mercury orbital period: T \approx 0.240846 years
7. Compute the Denominator: a(1 − e²)c²
First compute 1 − e²:
1 − e² \approx 1 − (0.2056)² = 0.9577
Multiply:
a(1 − e²) \approx 5.7909 \times 10¹⁰ \times 0.9577 = 5.54 \times 10¹⁰ m
Now multiply by c²:
a(1 − e²)c² \approx 5.54 \times 10¹⁰ \times 8.99 \times 10¹⁶
= 4.98 \times 10²⁷ m³ s⁻²
8. Compute the Dimensionless Factor GM / [a(1 − e²)c²]
GM = 1.3271 \times 10²⁰ m³ s⁻²
Divide:
GM / [a(1 − e²)c²] = 1.3271 \times 10²⁰ / 4.9846 \times 10²⁷
\approx 2.66 \times 10⁻⁸
9. Multiply by 6\pi to Get Radians per Orbit
6\pi \approx 18.8496
Thus:
\Delta\varphi (radians/orbit) = 18.8496 \times 2.66 \times 10⁻⁸
\approx 5.02 \times 10⁻⁷ radians per orbit
10. Convert Radians per Orbit → Arcseconds per Orbit
1 radian = 206,264.806 arcseconds
Multiply:
\Delta\varphi_arcsec = 5.02 \times 10⁻⁷ \times 2.06265 \times 10⁵
\approx 0.1035 arcseconds per orbit
11. Orbits per Century
Mercury orbital period:
T \approx 0.240846 years
Thus number of orbits in 100 years:
N = 100 / 0.240846 \approx 415.2 orbits per century
9. Total Perihelion Advance per Century
Multiply the per-orbit advance by the number of orbits:
\Delta\varphi_century = 0.1035 arcsec/orbit \times 415.2 orbits/century
\approx 42.98 arcseconds per century
Thus:
\Delta\varphi_DVFT \approx 43 arcsec/century
which matches the observed anomalous perihelion precession of Mercury.
This derivation used:
• Classical mechanics,
• DVFT effective potential,
International Journal for Multidisciplinary Research (IJFMR)
E-ISSN: 2582-2160 ● Website: www.ijfmr.com ● Email: editor@ijfmr.com
IJFMR250664112 Volume 7, Issue 6, November-December 2025 38
• No Einstein field equations.
6. Why DVFT Predicts the Same Result as GR in this Regime
Because Mercury is deep in the high-acceleration regime:
g >> a₀,
DVFT's nonlinear low-acceleration corrections vanish. Its weak-field expansion forces a 1/r³ correction
identical in functional form to GR’s 1PN term. Solar System tests constrain any deviation to <10⁻¹¹
fractionally, so the DVFT correction coefficient must match GR’s to this accuracy.
7. Physical Interpretation
• DVFT predicts Newtonian gravity with a small relativistic correction from \theta-field curvature.
• This correction appears as an extra inward acceleration proportional to 1/r³.
• That correction shifts the orbital frequency slightly, causing the perihelion to advance.
• DVFT predicts the same value as GR because both theories share the same high-g limit.
Conclusion
Using only DVFT (and classical orbit theory), the perihelion shift is:
\Delta\varphi_DVFT = 6\pi GM / (a (1−e²) c²).
This reproduces the observed 43 arcsec/century without invoking Einstein’s equations. Therefore: DVFT
is consistent with Solar System precision tests while remaining a fundamentally different theory from GR
in the low-acceleration regime.
