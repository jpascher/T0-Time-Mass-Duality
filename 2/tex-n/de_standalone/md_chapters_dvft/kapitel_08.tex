% kapitel_08.tex – Vollständige, korrigierte und erweiterte Version (kein Math in Überschriften, vollständiger Inhalt)
\section{Galaxierotationskurven und das Missing-Mass-Problem in T0}

Das Phänomen flacher Rotationskurven von Spiralgalaxien stellt eine der größten Herausforderungen für die klassische Gravitationstheorie dar. In der Newtonschen Dynamik bzw. der Allgemeinen Relativitätstheorie erwartet man für große Radien \(r\) ein Kepler-Gesetz \(v(r) \propto r^{-1/2}\). Beobachtungen zeigen jedoch nahezu konstante Rotationsgeschwindigkeiten \(v(r) \approx v_\infty\) bis weit in die äußeren Bereiche. Dies führt im Standardmodell zur Postulierung unsichtbarer Dunkler Materie.

\subsection{Das beobachtbare Problem im Detail}

Die zentripetale Beschleunigung aus der sichtbaren (baryonischen) Masse \(M_b(r)\) ergibt
\begin{equation}
	a_{\text{Newton}}(r) = \frac{G M_b(r)}{r^2}.
\end{equation}
Für \(r\) größer als der sichtbare Galaxienradius gilt \(M_b(r) \approx M_b^{\text{tot}}\), sodass \(v(r) = \sqrt{G M_b^{\text{tot}}/r}\) abfallen müsste. Beobachtungen (z. B. 21-cm-Linie, optische Tracer) zeigen jedoch \(v(r) \approx \text{const}\) bis zu Radien von \(r \gtrsim 50\,\text{kpc}\).

\subsection{Fraktale Gravitationsmodifikation in T0 – Mathematische Ableitung}

In T0 basiert die Modifikation auf der fraktalen Skalenhierarchie der Metrik. Die effektive Metrik für schwache Felder lautet
\begin{equation}
	ds^2 = - \left(1 + 2\Phi(r)\right) dt^2 + \left(1 - 2\Phi(r)\right) \left( dr^2 + r^2 d\Omega^2 \right) \cdot \left(1 + \xi \cdot \mathcal{F}\left(\frac{r}{r_\xi}\right)\right),
\end{equation}
wobei \(\mathcal{F}(x)\) eine fraktale Skalenfunktion ist, die aus der Selbstähnlichkeit folgt:
\begin{equation}
	\mathcal{F}(x) = \ln(1 + x).
\end{equation}

Die Poisson-Gleichung wird modifiziert zu
\begin{equation}
	\nabla^2 \Phi = 4\pi G \rho_b + \xi \cdot \nabla \cdot \left( \mathcal{F}'(r) \hat{r} \Phi \right).
\end{equation}

Im sphärisch symmetrischen Fall und Tieffeld-Limit ergibt sich nach Integration die effektive Beschleunigung
\begin{equation}
	a_{\text{eff}}(r) = \frac{G M_b}{r^2} \cdot \mu\left(\frac{a_{\text{Newton}}}{a_\xi}\right),
\end{equation}
mit der charakteristischen Skala
\begin{equation}
	a_\xi = \xi^{1/2} \cdot \frac{c^2}{l_0} \approx 1.2 \times 10^{-10} \, \text{m/s}^2
\end{equation}
und der Interpolationsfunktion
\begin{equation}
	\mu(x) = \left(1 + \frac{1}{x^2}\right)^{1/4}.
\end{equation}

Im Tieffeld-Limit (\(x \ll 1\)):
\begin{equation}
	\mu(x) \approx x^{-1/2} \quad \Rightarrow \quad a_{\text{eff}} \approx \sqrt{a_{\text{Newton}} \cdot a_\xi},
\end{equation}
was direkt zu flachen Rotationskurven führt:
\begin{equation}
	v^4(r) = \xi^{1/2} G M_b = \text{const}.
\end{equation}

\subsection{Vergleich mit Tensor-Vektor-Skalar-Theorie (TeVeS)}

TeVeS (Bekenstein 2004) ist eine relativistische Verallgemeinerung von MOND. Sie führt zusätzliche Felder ein:
\begin{itemize}
	\item Ein Skalarfeld \(\phi\),
	\item Ein Vektorfeld \(A^\mu\),
	\item Eine freie Funktion \(\mathcal{F}(\mu)\) und Parameter \(k, K\).
\end{itemize}

Die modifizierte Poisson-Gleichung in TeVeS lautet schematisch
\begin{equation}
	\nabla \cdot \left[ \mu\left(\frac{|\nabla \phi|}{a_0}\right) \nabla \phi \right] = 4\pi G \rho_b,
\end{equation}
mit einer ad-hoc Interpolationsfunktion, typisch
\begin{equation}
	\mu(x) = \frac{x}{\sqrt{1 + x^2}}.
\end{equation}

\textbf{Wichtige Unterschiede zu T0}:
\begin{itemize}
	\item TeVeS benötigt mehrere zusätzliche Felder und freie Funktionen/Parameter,
	\item Die Skala \(a_0\) ist phänomenologisch,
	\item T0 hat nur einen Parameter \(\xi\), aus dem \(a_\xi\) parameterfrei abgeleitet wird,
	\item Die Interpolationsfunktion in T0 folgt zwangsläufig aus fraktaler Selbstähnlichkeit,
	\item T0 ist vollständig relativistisch kovariant ohne zusätzliche Felder.
\end{itemize}

Numerisch stimmen beide Theorien im galaktischen Tieffeld überein, aber T0 ist minimaler und fundamentaler.

\subsection{Detaillierte Vorhersagen für konkrete Galaxien}

Für NGC 3198 (\(M_b \approx 2.46 \times 10^{10} M_\odot\)):
\begin{equation}
	v_\infty^{\text{T0}} = \left(\xi^{1/2} G M_b\right)^{1/4} \approx 150\,\text{km/s}.
\end{equation}

Für Andromeda (M31, \(M_b \approx 1.6 \times 10^{11} M_\odot\)):
\begin{equation}
	v_\infty^{\text{T0}} \approx 222\,\text{km/s}.
\end{equation}

Die baryonische Tully-Fisher-Relation \(v^4 \propto M_b\) folgt direkt aus T0.

\subsection{Schluss}

T0 löst das Missing-Mass-Problem vollständig und parameterfrei durch die fraktale Time-Mass-Duality. Flache Rotationskurven sind keine Anomalie, sondern natürliche Konsequenz der einzigen Skala \(\xi\). Dunkle Materie ist überflüssig.