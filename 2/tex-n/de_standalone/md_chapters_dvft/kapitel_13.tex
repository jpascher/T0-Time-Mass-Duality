% kapitel_13.tex
\section{Chronologie der Universumsentstehung in T0 – Vergleich mit LQG und Stringtheorie}

T0 beschreibt die Entstehung als deterministischen Übergang aus einer minimalen fraktalen Pre-Phase.

\subsection{Pre-Big-Bang-Phase in T0}

\begin{equation}
	\rho \approx 0, \quad a \approx a_{\min} \approx l_0 \cdot \xi, \quad \Delta \theta = 0.
\end{equation}

\subsection{Übergang}

Fluktuation
\begin{equation}
	\Delta \rho \approx \xi^2 \cdot \rho_P
\end{equation}
löst Wachstum aus.

\subsection{Vergleich mit LQG und Stringtheorie}

LQG/LQC: Quantengeometrische Pre-Phase. Stringtheorie: Höherdimensionale Pre-Phase.

\textbf{Wichtige Unterschiede zu T0}:
\begin{itemize}
	\item LQG: Quantengeometrie, Immirzi-Parameter,
	\item Stringtheorie: Höhere Dimensionen, Landscape,
	\item T0: Minimal, nur \(\xi\), deterministisch.
\end{itemize}

\subsection{Schluss}

T0 bietet die einfachste Ontologie der Universumsentstehung.