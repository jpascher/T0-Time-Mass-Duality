\documentclass[12pt,a4paper]{article}

% Packages
\usepackage[utf8]{inputenc}
\usepackage[T1]{fontenc}
\usepackage{amsmath}
\usepackage{amssymb}
\usepackage{physics}
\usepackage{graphicx}
\usepackage{hyperref}
\usepackage[margin=2.5cm]{geometry}

% Physics notation
\renewcommand{\varphi}{\phi}

\title{Kapitel 38
}
\author{DVFT - Dynamic Vacuum Field Theory}
\date{\today}

\begin{document}

\section{Kapitel 38
}


BLACK HOLE AND QUANTUM SINGULARITIES
1. Introduction
This document presents a full, rigorous DVFT (Dynamic vacuum field Curvature Theory) explanation of
why \textit{both} classical gravitational singularities (black holes) and quantum singularities (point particles,
infinite self-energy) cannot exist.
In DVFT, spacetime curvature and inertia emerge from the vacuum amplitude field:
Φ(x,t) = \rho(x,t) e^{i\theta(x,t)},
with:
• \rho(x,t) – vacuum amplitude (determines inertia and gravitational potential),
• \theta(x,t) – phase field (determines quantum coherence and wave-like behavior).
Gravity emerges from amplitude gradients:
g = −\nabla\rho.
Singularities require \rho → \infty or \nabla\rho → \infty. DVFT forbids both because the vacuum has finite stiffness and
inertial density, encoded in the potential U(\rho).
2. Why Singularities Cannot Exist in DVFT: The Vacuum Potential U(\rho)
DVFT postulates the vacuum has a microphysical potential:
U(\rho) = \Lambda₀ + (\kappa/2)(\rho − \rho₀)² + (\lambda/4)(\rho − \rho₀)⁴ + …
where:
• \rho₀ is the equilibrium vacuum amplitude,
• \kappa is the elastic stiffness of the vacuum,
International Journal for Multidisciplinary Research (IJFMR)
E-ISSN: 2582-2160 ● Website: www.ijfmr.com ● Email: editor@ijfmr.com
IJFMR250664112 Volume 7, Issue 6, November-December 2025 84
• \lambda stabilizes large deviations of \rho.
This potential is strongly convex at large |\rho − \rho₀|.
Thus, any attempt to compress the vacuum amplitude beyond moderate values requires infinite energy:
U(\rho) → \infty as |\rho − \rho₀| → \infty.
Therefore:
• \rho cannot diverge,
• \nabla\rho cannot diverge,
• gravitational curvature cannot diverge.
This single microphysical fact eliminates \textit{all} singularities in DVFT.
3. Removal of Quantum Singularities (Electron, Proton, Point Particles)
Quantum field theory treats electrons and quarks as point particles, leading to:
• infinite self-energy,
• divergent Coulomb self-field,
• undefined gravitational field at r = 0.
DVFT replaces a point mass with a finite vacuum amplitude deformation:
δ\rho(x) = G m \int d³x' |\psi(x')|² / |x − x'|.
This deformation is always finite because:
• |\psi(x)|² is normalizable,
• convolution with 1/r smooths the field,
• U(\rho) prevents amplitude blow-up.
As a result:
• no particle has infinite self-energy,
• no wavefunction produces a singular potential,
• gravity is well-defined even in superposition.
Thus quantum singularities are eliminated by vacuum microphysics, not by renormalization.
4. Gravitational Field of a Delocalized Electron
An electron with wavefunction \psi(x,t) generates a vacuum amplitude profile:
\rho(x,t) = \rho₀ + G m_e \int d³x' |\psi(x',t)|² / |x − x'|.
When \psi(x,t) spreads due to quantum dispersion, the gravitational field spreads with it:
g(x,t) = −\nabla\rho(x,t).
This ensures:
• gravity is fully compatible with Heisenberg uncertainty,
• gravitational fields have finite width,
• no r → 0 divergence occurs.
DVFT therefore produces the first consistent microscopic definition of gravity for a single quantum
particle.
5. Removal of Black Hole Singularities
In classical GR, gravitational collapse leads to infinite curvature at r = 0.
In DVFT, as matter compresses and raises \rho(x), the vacuum potential U(\rho) rapidly increases. At
sufficiently high density, a phase transition in the vacuum occurs:
• \rho stops increasing (vacuum stiffness prevents divergence),
• \theta becomes phase-locked (coherence inside horizon),
• matter transitions into a high-amplitude vacuum phase state,
International Journal for Multidisciplinary Research (IJFMR)
E-ISSN: 2582-2160 ● Website: www.ijfmr.com ● Email: editor@ijfmr.com
IJFMR250664112 Volume 7, Issue 6, November-December 2025 85
• gravitational field saturates.
Thus the black hole interior is NOT a singularity. It is a region of:
• finite \rho,
• finite \nabla\rho,
• finite energy density,
• vacuum-phase condensate.
The event horizon may still exist, but the spacetime interior remains regular.
6. DVFT Black Hole Interior Structure
DVFT predicts that inside a black hole:
• \rho(r) rises toward a maximum allowed value \rho_max,
• U(\rho) prevents further growth beyond \rho_max,
• curvature saturates,
• matter becomes vacuum-amplitude dominated,
• \theta freezes (phase coherence becomes rigid),
• no divergence in metric-equivalent quantities occurs.
This resembles:
• gravastar-like interiors,
• vacuum condensate cores,
• nonsingular loop quantum gravity solutions,
• but derived \textit{entirely from DVFT microphysics}.
7. The Deep Reason DVFT Removes Both Types of Singularities
DVFT eliminates singularities because spacetime curvature is not fundamental. It is an *emergent
property* of the vacuum amplitude field \rho. If \rho cannot diverge, then curvature cannot diverge. The
vacuum’s elastic potential and finite inertial density are the mechanisms that prevent runaways.
Thus:
• matter cannot collapse to infinite density,
• wavefunctions cannot create divergent potentials,
• curvature cannot become infinite.
This is the first unified mechanism eliminating singularities across classical and quantum domains.
8. Comparison with GR, LQG, and QFT
General Relativity (GR):
• predicts unavoidable singularities (Hawking-Penrose theorems),
• has no internal regulator for curvature.
Loop Quantum Gravity (LQG):
• introduces discrete geometry,
• removes singularities by quantizing spacetime,
• but requires radical nonlocality and lacks experimental grounding.
Quantum Field Theory:
• produces infinite point-particle self-energies,
• resolves them only through renormalization,
• does not address gravitational singularity.
DVFT:
• retains continuum spacetime,
International Journal for Multidisciplinary Research (IJFMR)
E-ISSN: 2582-2160 ● Website: www.ijfmr.com ● Email: editor@ijfmr.com
IJFMR250664112 Volume 7, Issue 6, November-December 2025 86
• derives gravity from a physical vacuum field,
• imposes finite amplitude \& stiffness,
• eliminates both self-energy and gravitational singularities,
• without renormalization,
• without quantizing spacetime,
• without modifying quantum mechanics.
DVFT is the simplest and most physically grounded solution among all three.
9. Final Summary
DVFT eliminates singularities through vacuum amplitude dynamics:
1. The vacuum field Φ = \rho e^{i\theta} has finite stiffness and inertial density.
2. U(\rho) prevents \rho from diverging under collapse.
3. Quantum particles generate finite vacuum amplitude deformations from |\psi|².
4. Gravity emerges as \nabla\rho, which can never diverge.
5. Black holes contain vacuum-phase condensates, not singularities.
6. No infinite self-energy, no point divergences, no r → 0 explosion exists.
DVFT therefore provides the first unified, microphysically consistent elimination of:
• black hole singularities,
• quantum point singularities,
• gravitational field singularities.
This positions DVFT as a fundamentally complete framework bridging general relativity and quantum
mechanics.

\end{document}
