% Chapter: Kapitel 18
\section{Kapitel 18
}


SCHRÖDINGER’S EQUATION DERIVATION
This chapter explains how Schrödinger’s equation naturally emerges within the Dynamic Vacuum Field
Theory (DVFT). In standard quantum mechanics, the wavefunction \psi is treated as an abstract object with
no physical interpretation. DVFT resolves this by showing that \psi is a small excitation riding on the vacuum
International Journal for Multidisciplinary Research (IJFMR)
E-ISSN: 2582-2160 ● Website: www.ijfmr.com ● Email: editor@ijfmr.com
IJFMR250664112 Volume 7, Issue 6, November-December 2025 43
field Φ = \rho e^{i\theta}. The vacuum’s phase \theta provides the physical origin of quantum phase evolution,
interference, and wave-particle duality. We show that Schrödinger dynamics arise as the non-relativistic
limit of particle interactions with the dynamic vacuum field, and that the complex nature of quantum
mechanics emerges from the complex structure of the vacuum itself.
1. Introduction
Schrödinger’s equation governs quantum dynamics, yet its physical meaning is obscure in standard
quantum theory. DVFT provides a physical substrate: the vacuum field Φ = \rho e^{i\theta}. In this framework,
matter wavefunctions \psi interact with the vacuum phase \theta, making quantum phase evolution a
manifestation of dynamic vacuum field.
2. The Vacuum Field Φ and Its Phase \theta
In DVFT, spacetime contains a physical vacuum field:
Φ = \rho e^{i\theta}
where \rho is the vacuum amplitude and \theta is the vacuum phase. The phase evolves in proper time:
\theta(\tau) = \mu \tau
This phase rotation provides a universal background oscillation that seeds quantum phase evolution.
3. Wavefunction Phase Origin: \psi Inherits Phase from Φ
The polar decomposition of the wavefunction is:
\psi = R e^{iS/ħ}
In DVFT, the quantum phase S/ħ is directly linked to the vacuum phase \theta:
S/ħ \approx \alpha \theta
Thus \psi = R e^{i\alpha\theta}. The wavefunction phase is not abstract but it is physically tied to the phase of the
vacuum. This also explains why all quantum interference phenomena depend on relative phase differences.
4. Schrödinger Equation from the Vacuum Field
Begin from the Klein–Gordon equation in a vacuum background:
(□ + m²)\psi = 0
Now write \psi = e^{-imt/ħ} \varphi. Taking the non-relativistic limit yields the Schrödinger equation:
iħ \partial\varphi/\partialt = -ħ²/(2m) \nabla²\varphi + V_eff \varphi
In DVFT, the background vacuum phase modifies the effective time experienced by matter:
t → t + \beta \theta(x)
Thus, Schrödinger’s equation becomes the emergent low-energy evolution of matter riding on the dynamic
vacuum field.
5. Why Quantum Mechanics Uses Complex Numbers
Standard QM requires complex numbers but never explains why. DVFT explains it:
• Φ is complex because it is a U(1) field.
• \psi inherits this complex structure from Φ.
• The vacuum’s internal phase rotation causes the appearance of i in quantum dynamics.
In DVFT, the imaginary unit i is not a mathematical trick but a reflection of physical vacuum structure.
6. Why Schrödinger Dynamics Are Linear
DVFT’s dynamic vacuum field is harmonic. Linear perturbations on such a background naturally yield
linear equations. This is identical to how phonons in superfluids or ripples in condensates obey linear wave
equations. Thus, Schrödinger’s equation arises from linearizing the dynamics of matter excitations on a
stable, dynamic vacuum.
7. Quantum Interference via Vacuum Phase Coherence
International Journal for Multidisciplinary Research (IJFMR)
E-ISSN: 2582-2160 ● Website: www.ijfmr.com ● Email: editor@ijfmr.com
IJFMR250664112 Volume 7, Issue 6, November-December 2025 44
DVFT gives physical meaning to interference:
• When the vacuum phase \theta is coherent → \psi interferes.
• Measurement interactions scramble \theta locally → \psi collapses.
• DCQE experiments show that restoring coherence restores interference.
This ties quantum interference directly to vacuum-phase coherence.
8. Measurement and Collapse in DVFT
In DVFT, wavefunction collapse results from the loss of vacuum-phase coherence due to strong coupling
with macroscopic systems. Collapse is not mystical—it is the destruction of a coherent \theta-field pattern.
Conclusion
Schrödinger’s equation:
iħ \partial\psi/\partialt = -ħ²/(2m) \nabla²\psi + V \psi
is not fundamental. In DVFT it emerges from:
• matter excitations coupled to Φ = \rho e^{i\theta}
• vacuum phase evolution \theta(t)
• the complex structure of Φ
• proper-time Dynamic vacuum field
DVFT provides the physical substrate that Schrödinger’s equation lacks, unifying quantum phase,
interference, collapse, and vacuum structure into a single coherent framework.
