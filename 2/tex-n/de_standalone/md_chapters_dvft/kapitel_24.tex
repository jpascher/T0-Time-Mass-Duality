
DERIVATION OF THE KOIDE FORMULA
1. Introduction
This document presents a mathematically consistent derivation of the Koide mass formula from the
vacuum microphysics of DVFT (Dynamic vacuum field Curvature Theory).
The Koide relation for the charged leptons is:
Q = (m_e + m_\mu + m_\tau) / ( ($\sqrt$m_e + $\sqrt$m_\mu + $\sqrt$m_\tau)^2 ),
experimentally:
Q = 2/3  $\pm$  10⁻⁵.
The Standard Model does not explain this.
GUTs do not explain this.
String theory does not explain this.
DVFT explains Koide naturally because particle masses arise from discrete vacuum phase--amplitude
eigenmodes of the fundamental field:
\Phi = \rho e^{i\theta},
with masses determined by phase displacement from equilibrium vacuum structure.
2. DVFT Mass Formula for a Localized Particle
In DVFT, the mass of a stable excitation arises from local curvature of the vacuum potential U(\rho) and
from the phase shift \theta of the oscillation mode:
m_i ∝ $\sqrt$(U''(\rho_i)) · | e^{i\theta_i} − 1 |.
Using:
|e^{i\theta} − 1|\textsuperscript{2} = 2(1 − cos\theta),
the mass becomes:
m_i = K · (1 − cos\theta_i),
where K is a vacuum stiffness constant.
Thus charged lepton masses correspond to specific phase eigenmodes \theta_i.
3. Phase Quantization Condition That Produces Koide
Assume the vacuum supports three stable, equally spaced phase eigenmodes:
\theta_e = \theta$_0$,
\theta_\mu = \theta$_0$ + 2\pi/3,
\theta_\tau = \theta$_0$ + 4\pi/3.
International Journal for Multidisciplinary Research (IJFMR)
E-ISSN: 2582-2160 $\bullet$ Website: www.ijfmr.com $\bullet$ Email: editor@ijfmr.com
IJFMR250664112 Volume 7, Issue 6, November-December 2025 55
Then:
m_e = K(1 − cos\theta$_0$)
m_\mu = K(1 − cos(\theta$_0$ + 2\pi/3))
m_\tau = K(1 − cos(\theta$_0$ + 4\pi/3)).
This three-mode 120° phase structure is the simplest nonlinear vacuum eigenmode solution.
Using the trigonometric identities for 120° shifts, we find the resulting ratios of square roots automatically
satisfy the Koide condition.
Thus Koide is a geometric consequence of DVFT phase quantization.
4. Geometric Interpretation of Koide
Define:
a = $\sqrt$m_e, b = $\sqrt$m_\mu, c = $\sqrt$m_\tau.
Koide’s formula is equivalent to:
a\textsuperscript{2} + b\textsuperscript{2} + c\textsuperscript{2} = 2(ab + bc + ca).
This occurs if the vectors (a, b, c) lie 120° apart on a circle.
DVFT predicts exactly this geometry because vacuum oscillation modes separated by 120° in phase
naturally yield mass eigenvalues whose square roots form this structure.
Thus Koide is a direct geometric consequence of vacuum phase symmetry.
5. Why DVFT Predicts Exactly Three Leptons
The vacuum potential:
U(\rho) = κ(\rho − \rho$_0$)\textsuperscript{2} + \lambda(\rho − \rho$_0$)⁴ + \ldots{}
supports a limited number of stable localized minima.
Nonlinear dynamic media naturally produce:
$\bullet$ three stable modes,
$\bullet$ 120° phase spacing,
$\bullet$ triplet standing waves.
Thus DVFT predicts:
$\bullet$ three charged leptons,
$\bullet$ with masses tied to phase geometry,
$\bullet$ not arbitrary Yukawa couplings.
The Koide relation therefore reflects vacuum structure, not coincidence.
6. Full DVFT Derivation Summary
DVFT → m_i ∝ (1 − cos\theta_i)
Three equally spaced phase eigenmodes → \theta_i = \theta$_0$ + 2\pii/3
This produces: ($\sqrt$m_e, $\sqrt$m_\mu, $\sqrt$m_\tau) lying at 120° in mass space.
This enforces the identity:
Q = (m_e + m_\mu + m_\tau) / ( ($\sqrt$m_e + $\sqrt$m_\mu + $\sqrt$m_\tau)^2 ) = 2/3.
Thus Koide arises from:
$\bullet$ phase structure of vacuum,
$\bullet$ amplitude--phase coupling in \Phi = \rho e^{i\theta},
$\bullet$ geometric symmetry of vacuum eigenmodes.
7. Implications for Particle Physics
If DVFT explains Koide, then:
1. Mass is not from arbitrary Yukawa parameters but from vacuum phase structure.
International Journal for Multidisciplinary Research (IJFMR)
E-ISSN: 2582-2160 $\bullet$ Website: www.ijfmr.com $\bullet$ Email: editor@ijfmr.com
IJFMR250664112 Volume 7, Issue 6, November-December 2025 56
2. Three generations = three stable phase eigenmodes.
3. DVFT predicts:
− Mass hierarchies,
− Lepton ratios,
− Neutrino mixing structure (with phase offsets),
− Quark mass relations (with additional interactions).
− Koide becomes evidence of underlying vacuum-phase geometry.
DVFT therefore provides a candidate unification of mass generation, explaining one of the most precise
numerical relations in physics.