\documentclass[12pt,a4paper]{article}

% Packages
\usepackage[utf8]{inputenc}
\usepackage[T1]{fontenc}
\usepackage{amsmath}
\usepackage{amssymb}
\usepackage{physics}
\usepackage{graphicx}
\usepackage{hyperref}
\usepackage[margin=2.5cm]{geometry}

% Physics notation
\renewcommand{\varphi}{\phi}

\title{Kapitel 07
}
\author{DVFT - Dynamic Vacuum Field Theory}
\date{\today}

\begin{document}

\section{Kapitel 07
}


DERIVING SPECIAL RELATIVITY EQUATIONS
1. Introduction
Special Relativity traditionally begins with Einstein’s postulates, particularly the constancy of the speed
of light and the equivalence of all inertial frames. However, these postulates do not explain why these
statements are true. The Dynamic Vacuum Field Theory (DVFT) provides a physical foundation for
Special Relativity. Instead of postulating relativistic effects, DVFT derives time dilation, length
contraction, and the relativistic mass–energy relation from first principles:
• The vacuum is a structured medium with stiffness K₀ and inertial density \rho₀.
• The fundamental dynamic vacuum field equation defines the propagation of all phase excitations.
• Physical laws must retain their form in every inertial frame.
From these principles alone, the Lorentz transformation, \gamma factor, and all relativistic transformations
follow. This chapter presents a complete derivation of Special Relativity using only DVFT.
2. The Fundamental Dynamic vacuum field Equation
DVFT begins with the fundamental wave equation for the vacuum phase field \theta(x, t):
\rho₀ \partial²_t \theta − K₀ \partial²_x \theta = 0.
Define the natural propagation speed of vacuum phase waves:
c = \sqrt(K₀ / \rho₀).
This yields the canonical form:
(1/c²) \partial²_t \theta − \partial²_x \theta = 0.
DVFT asserts two axioms:
1. Dynamic vacuum field hold in all inertial frames.
2. The phase \theta(x, t) is a physical scalar observable of the vacuum.
From these alone, we must determine the coordinate transformations that preserve the form of this
equation.
3. Deriving Lorentz Transformations from DVFT
International Journal for Multidisciplinary Research (IJFMR)
E-ISSN: 2582-2160 ● Website: www.ijfmr.com ● Email: editor@ijfmr.com
IJFMR250664112 Volume 7, Issue 6, November-December 2025 17
Consider two inertial frames related linearly:
x' = A x + B t,
t' = C x + D t.
Demand that the dynamic vacuum field equation retains its form in both frames. Applying the chain rule
and enforcing invariance leads to the following constraints:
• AD − BC = 1 (preserves phase structure),
• A = D = \gamma,
• B = −\gamma v,
• C = −\gamma v / c²,
where the Lorentz factor emerges naturally:
\gamma = 1 / \sqrt(1 − v²/c²).
This yields the Lorentz transformation:
x' = \gamma (x − vt),
t' = \gamma (t − vx/c²).
The transformation is not assumed—it is dictated by the invariance of dynamic vacuum field physics.
4. Proper Time from Vacuum Phase Oscillations
In DVFT, time is defined physically, not geometrically. A clock corresponds to a local vacuum phase
oscillation:
\theta(\tau) = \omega₀ \tau,
where \tau parametrizes the intrinsic evolution of the vacuum at a point. Because the dynamic vacuum field
equation’s invariant form is:
c² dt² − dx² = c² d\tau²,
proper time is naturally defined as:
d\tau² = dt² − dx²/c².
Thus, the flow of time is the physical evolution of vacuum phase, and \tau is the invariant measure of phase
progression.
5. Time Dilation
A clock at rest in its own frame satisfies dx' = 0. For two ticks separated by \Deltat' = \Delta\tau in the moving frame,
the DVFT Lorentz transform gives:
t' = \gamma (t − vx/c²),
and substituting x = vt (the worldline of the moving clock) gives:
t' = t / \gamma.
Thus:
\Deltat = \gamma \Delta\tau.
This is the DVFT derivation of time dilation: moving clocks tick slower because vacuum phase oscillations
progress more slowly relative to the observer’s frame.
6. Length Contraction
A rigid rod at rest in the primed frame has proper length L₀ = x₂' − x₁'. Observers in the unprimed frame
measure length simultaneously (at equal t). Using the Lorentz inverse transformation:
x = \gamma (x' + vt'),
and enforcing t₁ = t₂, one finds:
L = L₀ / \gamma.
International Journal for Multidisciplinary Research (IJFMR)
E-ISSN: 2582-2160 ● Website: www.ijfmr.com ● Email: editor@ijfmr.com
IJFMR250664112 Volume 7, Issue 6, November-December 2025 18
In DVFT terms, the length of an object is determined by dynamic vacuum field. Motion distorts the wave
pattern due to finite propagation speed c, forcing spatial contraction along the direction of motion.
7. Relativistic Mass and Energy from DVFT Dispersion
A massive particle is a localized, stable excitation of vacuum amplitude Φ and phase fields. Such an
excitation \chi obeys the wave equation:
\rho_\chi \partial²_t \chi − K_\chi \partial²_x \chi + \mu² \chi = 0,
leading to the dispersion relation:
\omega² = c² k² + \omega₀²,
where \omega₀ = m₀ c² / ħ.
Defining energy E = ħ\omega and momentum p = ħk gives:
E² = p² c² + m₀² c⁴.
This produces:
E = \gamma m₀ c²,
p = \gamma m₀ v.
Thus, relativistic energy and momentum emerge naturally from dynamic vacuum field and invariance.
8. Unified Explanation of Relativistic Effects in DVFT
DVFT derives all relativistic phenomena from a single principle: the invariance of the dynamic vacuum
field equation. From this principle follow:
• Lorentz transformations,
• Time dilation,
• Length contraction,
• Relativistic mass increase,
• The energy–momentum relation.
In DVFT, relativity is not a geometric postulate, but a physical necessity caused by the structure of the
vacuum.
Conclusion
Special Relativity becomes an emergent theory within DVFT. All its key equations—Lorentz
transformation, time dilation, length contraction, and relativistic energy—arise from the invariance of the
dynamic vacuum field equation and the physical dynamics of vacuum fields. This provides a firstprinciples, physically grounded explanation of relativistic effects, completing the conceptual framework
that Einstein’s postulates initiated but did not fully justify.

\end{document}
