
PARTICLE MASS HIERARCHY
1. Introduction
This document explains two of the deepest unresolved problems in modern physics:
1. Why do elementary particles have different masses spanning 14 orders of magnitude?
2. Why is gravity extraordinarily weak compared to the other three forces?
Dynamic Vacuum Field Theory(DVFT) provides natural, structural, non-ad-hoc solutions to both
questions by modeling the universe as a dynamic vacuum field with amplitude (\rho) and phase (\theta) degrees
of freedom:
\Phi = \rho e^{i\theta}.
This framework replaces the arbitrary mass assignments of QFT and the geometric interpretation of GR
with a unified vacuum-based mechanism.
2. DVFT Vacuum Field Structure
DVFT defines the vacuum as a physical field with:
$\bullet$ \rho(x) --- amplitude (stores curvature, mass energy, and gravitational coupling)
$\bullet$ \theta(x) --- phase (stores gauge information and coherence)
$\bullet$ K$_0$ --- vacuum amplitude stiffness
$\bullet$ B --- vacuum phase stiffness
$\bullet$ \rho$_0$ --- inertial vacuum density
Mass, gravity, and gauge interactions arise from how matter perturbs this vacuum.
3. Mass as Vacuum Amplitude Deformation
In DVFT, mass is not intrinsic. It is the energy cost of deforming the vacuum amplitude \rho.
For a particle species i:
m_i ∝ $\sqrt$(K$_0$) · Δ\rho_i
Different particles produce different amplitude perturbations Δ\rho_i depending on:
$\bullet$ how strongly their \theta-structure couples to the vacuum,
$\bullet$ their topological winding number,
$\bullet$ the stability of their amplitude-phase configuration,
$\bullet$ their coherence length and vacuum potential U(\rho).
$\bullet$ This provides a structural explanation for:
$\bullet$ why neutrinos are extremely light,
$\bullet$ why electrons are light,
$\bullet$ why muons and taus are heavier,
$\bullet$ why quarks have large masses,
$\bullet$ why W and Z bosons are massive phase-amplitude configurations.
The mass hierarchy emerges naturally from vacuum microstructure, not from arbitrary Yukawa couplings
as in the Standard Model.
4. Massless Particles in DVFT
Massless particles correspond to pure phase excitations:
Δ\rho = 0, only \theta oscillates.
Photons have no amplitude deformation; they are pure \theta-waves.
This explains:
$\bullet$ why they travel at c,
$\bullet$ why they have zero rest mass,
International Journal for Multidisciplinary Research (IJFMR)
E-ISSN: 2582-2160 $\bullet$ Website: www.ijfmr.com $\bullet$ Email: editor@ijfmr.com
IJFMR250664112 Volume 7, Issue 6, November-December 2025 63
$\bullet$ why they do not curve the vacuum amplitude locally.
5. Why Particle Masses Span Many Orders of Magnitude
DVFT predicts that particles differ because they correspond to different stable vacuum configurations
with distinct:
$\bullet$ amplitude curvature energies,
$\bullet$ \theta-winding topologies,
$\bullet$ vacuum coupling strengths,
$\bullet$ deformation radii,
$\bullet$ coherence breakdown thresholds.
Thus the mass spectrum is not arbitrary, it reflects deeper structure in the amplitude-phase vacuum field.
6. Why Gravity Is So Weak
Gravity is the weakest interaction by a factor of ~10\textsuperscript{3}⁸.
DVFT explains this elegantly:
Gauge forces (EM, weak, strong) arise from phase gradients:
F_gauge ∼ $\partial$\theta.
Phase stiffness (B) is extremely small, so gauge interactions are strong or moderate.
Gravity arises from amplitude gradients:
F_grav ∼ $\partial$\rho.
Vacuum amplitude stiffness (K$_0$) is enormous, so even large masses cause only tiny curvature.
Thus:
Gravity ≪ Electromagnetism ≪ Strong force
because:
K$_0$ ≫ B.
This single relationship solves the hierarchy of forces.
7. Why Gravity Cannot Be Unified with Gauge Forces in QFT
QFT treats all fields as gauge or spinor fields on a fixed vacuum, which prevents a natural unification with
gravity.
DVFT unifies all forces because:
$\bullet$ Gauge forces = phase distortions of \theta,
$\bullet$ Gravity = amplitude distortions of \rho,
$\bullet$ Both arise from one vacuum field \Phi.
Gravity is not a gauge force, so its weakness is not a mystery---it is a mechanical property of the vacuum
itself.
8. Gravity Weakness Formula from DVFT
DVFT predicts:
G = \lambda_m / (4\pi K$_0$)
Thus:
$\bullet$ large K$_0$ → small G,
$\bullet$ weak gravity is a direct result of vacuum stiffness.
This provides the first explanation in physics for the relative weakness of gravity.
9. Implications for the Standard Model
DVFT supersedes the Higgs mechanism:
$\bullet$ The Higgs field becomes a special case of amplitude curvature in \rho,
International Journal for Multidisciplinary Research (IJFMR)
E-ISSN: 2582-2160 $\bullet$ Website: www.ijfmr.com $\bullet$ Email: editor@ijfmr.com
IJFMR250664112 Volume 7, Issue 6, November-December 2025 64
$\bullet$ Coupling constants arise from \theta-winding constraints,
$\bullet$ Masses emerge from vacuum geometry, not arbitrary Yukawa parameters.
DVFT therefore provides a deeper, more natural foundation for particle physics.
Conclusion
DVFT solves two of the greatest open problems in physics:
1. Particle Mass Hierarchy:
Mass = vacuum amplitude deformation.
Different particles correspond to different stable excitations of the vacuum.
2. Weakness of Gravity:
Gravity arises from amplitude gradients ($\nabla$\rho) in a vacuum with enormous stiffness K$_0$.
Gauge forces arise from phase gradients ($\nabla$\theta) with tiny stiffness B.
This not only explains known observations but unifies all interactions under a single vacuum field \Phi = \rho
e^{i\theta}, marking a fundamental advance over both GR and QFT.