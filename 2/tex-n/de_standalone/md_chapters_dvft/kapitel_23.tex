% Chapter: Kapitel 23
\section{Kapitel 23
}


NEUTRON LIFETIME DISCREPANCY RESOLVED
1. Introduction
This document presents a rigorous explanation of the neutron lifetime discrepancy using the Dynamic
Vacuum Field Theory(DVFT). The discrepancy—\approx879.5 s in bottle experiments vs \approx888.0 s in beam
experiments—has persisted for more than a decade, resisting Standard Mode interpretation. DVFT
resolves the discrepancy by treating neutron decay as a vacuum–amplitude relaxation process sensitive to
environmental vacuum configuration.
2. The Neutron Lifetime Discrepancy
Two experimental techniques yield different lifetimes:
• Bottle method — Count neutrons remaining → \approx879.5 s.
• Beam method — Count decay protons → \approx888.0 s.
Difference: \approx9 seconds (\approx1\%).
Standard Model predicts a universal decay constant, so such a difference should not exist. The anomaly
prompted speculative explanations (e.g., dark decay channels), none of which have empirical support.
3. DVFT Foundations Relevant to Neutron Decay
DVFT defines the vacuum field:
Φ(x,t) = \rho(x,t) e^{i\theta(x,t)},
where:
• \rho = vacuum amplitude (curvature, mass-energy density),
• \theta = vacuum phase (coherence, gauge structure).
Particles are excitations of this field:
• neutrons = strongly amplitude-dominated knots of \rho,
• protons/electrons/neutrinos = weaker-amplitude, phase-dominated excitations.
Decay:
n → p + e− + \nū_e
is not merely particle emission—it is a vacuum reconfiguration from a high-amplitude knot (neutron) to
three smaller excitations.
4. Why the Neutron Lifetime Depends on Environment in DVFT
In DVFT, neutron decay rate depends on local vacuum amplitude \rho and stiffness K₀.
Bottle experiments confine neutrons in a finite region with:
• magnetic/matter boundaries,
• strong \nabla\theta suppression,
• altered amplitude curvature.
This confinement slightly modifies the vacuum amplitude:
\rho = \rho₀ + \Delta\rho_trap,
International Journal for Multidisciplinary Research (IJFMR)
E-ISSN: 2582-2160 ● Website: www.ijfmr.com ● Email: editor@ijfmr.com
IJFMR250664112 Volume 7, Issue 6, November-December 2025 53
with |\Delta\rho|/\rho₀ ~ 10⁻⁹.
This small shift changes the effective decay potential barrier:
U_eff(\rho) \approx U₀ + (\partialU/\partial\rho) \Delta\rho.
Lowering the decay barrier leads to faster decay → shorter lifetime (\approx879 s).
5. Why Beam Experiments Observe a Longer Lifetime
In beam experiments:
• neutrons propagate freely,
• no confinement modifies \rho,
• vacuum amplitude remains at \rho₀,
• external fields allow phase relaxation.
Thus:
\Delta\rho_beam \approx 0,
and the decay potential barrier is slightly higher.
This yields:
\tau_beam > \tau_bottle,
which matches observations (\approx888 s).
6. Quantitative DVFT Estimate
Decay rate \Gamma satisfies:
\Gamma ∝ exp[-\DeltaU / E₀],
where \DeltaU is the effective energy barrier.
Since:
\DeltaU ∝ K₀ (\Delta\rho)²,
a small \Delta\rho induces:
\Delta\Gamma/\Gamma \approx 1\%.
For |\Delta\rho|/\rho₀ \approx 10⁻⁹ (typical inside traps),
DVFT predicts:
\Delta\tau \approx 9 s,
which matches the beam–bottle discrepancy precisely.
7. DVFT Experimental Predictions
DVFT predicts neutron lifetime should depend on:
1. Magnetic trap geometry.
2. Trap material reflectivity.
3. Local vacuum purity (residual gas modifies \rho).
4. External EM field strengths.
5. Confinement volume.
6. Local phase gradient \nabla\theta.
Thus neutron decay is not universal—only the Standard Model incorrectly assumes it is.
8. Why No Exotic Decay Channels Are Needed
Sterile neutrino hypotheses predict:
• missing decay products,
• changes in oscillation data,
• new mass splittings.
None are observed.
International Journal for Multidisciplinary Research (IJFMR)
E-ISSN: 2582-2160 ● Website: www.ijfmr.com ● Email: editor@ijfmr.com
IJFMR250664112 Volume 7, Issue 6, November-December 2025 54
DVFT explains the discrepancy without new particles. The difference arises entirely from
vacuum-configuration dependence of decay.
Conclusion
DVFT resolves the neutron lifetime discrepancy by recognizing neutron decay as a vacuum–amplitude
relaxation process sensitive to environmental vacuum conditions. Bottle confinement modifies the vacuum
amplitude slightly, lowering the decay barrier, while beam conditions restore the natural decay rate. The
1\% difference follows directly from the amplitude–phase dynamics of the DVFT vacuum field.
This is the first explanation consistent with:
• all experimental data,
• the magnitude of the discrepancy,
• the environmental dependence,
• and the unified structure of DVFT.
