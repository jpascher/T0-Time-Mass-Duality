% kapitel_31.tex – Stark erweiterte Version mit detaillierten mathematischen Ableitungen
\section{Photoelektrischer Effekt und Laserphysik in T0}

Der photoelektrische Effekt und die Laserphysik werden in T0 einheitlich durch die Dualität von Vakuum-Amplitude \(\rho\) und Phase \(\theta\) erklärt – ohne separate Teilchen- oder Wellenmetapher.

\subsection{Photoelektrischer Effekt – Detaillierte Ableitung}

In T0 ist ein Photon eine reine Phasenexcitation:
\begin{equation}
	E_{\text{photon}} = \hbar \omega = \xi \cdot \Delta \theta \cdot \frac{\hbar}{T_0}.
\end{equation}

Ein gebundenes Elektron ist eine Amplitude-Deformation:
\begin{equation}
	E_{\text{bind}} = K_0 \cdot (\delta \rho / \rho_0)^2 \cdot V_{\text{atom}}.
\end{equation}

Der photoelektrische Schwellenprozess:
\begin{equation}
	\hbar \omega > E_{\text{bind}} \quad \Rightarrow \quad \Delta \theta > \Delta \theta_{\text{threshold}} = \sqrt{\frac{2 E_{\text{bind}}}{B}}.
\end{equation}

Die kinetische Energie des Elektrons:
\begin{equation}
	E_{\text{kin}} = \hbar (\omega - \omega_0) = \xi \cdot (\Delta \theta - \Delta \theta_{\text{threshold}}) \cdot \frac{\hbar}{T_0}.
\end{equation}

Intensität erhöht nur die Rate (mehr Photonen → mehr \(\Delta \theta\)-Excitationen), nicht \(E_{\text{kin}}\) – exakt Einstein's Gesetz.

\subsection{Stimulierte Emission und Laser – Phasen-Synchronisation}

Stimulierte Emission: Ein kohärentes Phasenfeld \(\theta_{\text{in}}\) induziert Emission durch Resonanz:
\begin{equation}
	\frac{d \theta_{\text{atom}}}{dt} = \xi \cdot \sin(\theta_{\text{in}} - \theta_{\text{atom}}).
\end{equation}

Inversion (\(\rho > \rho_0\)) erzeugt negative Dämpfung:
\begin{equation}
	\dot{\theta} = \gamma (\rho - \rho_0) \cdot \theta_{\text{in}}.
\end{equation}

Im Resonator wächst die Phase exponentiell:
\begin{equation}
	\theta(t) = \theta_0 \exp\left( (\rho - \rho_0) \xi t / \tau_{\text{cav}} \right).
\end{equation}

Kohärente Ausgabe: Global synchronisierte Phase – Laserstrahl.

\subsection{Vergleich mit Standard-Interpretation}

Standard: Photon als Teilchen, Einstein-Koeffizienten ad-hoc.  
T0: Reine Phasen-Dynamik, stimulierte Emission als Entrainment.

\subsection{Schluss}

Photoelektrischer Effekt und Laser folgen in T0 aus Amplitude-Barriere und Phasen-Synchronisation – einheitlich, ohne Dualitätsparadoxon. Alles parameterfrei aus \(\xi\).