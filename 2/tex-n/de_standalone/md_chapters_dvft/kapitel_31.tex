% Chapter: Kapitel 31
\section{Kapitel 31
}


PHOTOELECTRIC EFFECT AND LASER PHYSICS
1. Introduction
This document explains the \textbf{photoelectric effect} and \textbf{laser physics} using only the principles of
Dynamic vacuum field–Curvature Theory (DVFT). DVFT is based on the vacuum field:
Φ(x,t) = \rho(x,t) e^{i\theta(x,t)},
where:
• \rho(x,t) = vacuum amplitude (energetic, classical-like, binding structure),
• \theta(x,t) = vacuum phase (coherent, quantized excitations → photons).
This amplitude–phase decomposition gives a physically transparent and unified explanation for photon
absorption, electron emission, stimulated emission, coherence, and laser amplification.
2. DVFT Explanation of the Photoelectric Effect
In DVFT, a photon is not a particle but a localized \textbf{\theta-phase excitation} of the vacuum. An electron is
a \textbf{vacuum defect}—a stable configuration where \rho and \theta deviate from equilibrium.
Why frequency matters but intensity does not The \theta-phase oscillation of a photon carries energy:
E_\theta = ħ\omega.
An electron is bound inside a surface by a vacuum amplitude barrier:
E_bind = \DeltaU(\rho).
A photon ejects an electron only if:
ħ\omega > E_bind.
This is because sufficient \theta-phase energy is required to destabilize the electron’s amplitude well. Intensity
increases the \textit{number} of \theta excitations, not their energy. Thus:
International Journal for Multidisciplinary Research (IJFMR)
E-ISSN: 2582-2160 ● Website: www.ijfmr.com ● Email: editor@ijfmr.com
IJFMR250664112 Volume 7, Issue 6, November-December 2025 71
• Low intensity, high frequency → immediate emission.
• High intensity, low frequency → no emission.
• This directly produces Einstein’s photoelectric law.
3. Why Emission is Instantaneous in DVFT
\theta-phase excitations interact directly with the electron defect. If ħ\omega exceeds the binding energy E_bind, the
electron's amplitude structure (\rho) collapses instantly:
δ\theta → δ\rho_e → defect escape.
There is \textbf{no time accumulation}, no gradual heating, and no multi-photon buildup required.
This explains why photoelectric emission exhibits \textit{zero measurable delay} in experiments.
4. Why Kinetic Energy Depends Only on Frequency
Once the electron defect escapes the surface, any excess \theta-phase energy is converted into kinetic energy:
K = ħ\omega - E_bind.
This explains the linear relationship between electron energy and photon frequency, independent of
intensity.
DVFT thus naturally reproduces Einstein’s equation for the photoelectric effect.
5. Laser Physics in DVFT
A laser is a macroscopic system that produces a coherent beam of \theta-phase excitations through
synchronized dynamics.
Stimulated Emission: In DVFT, an excited electron corresponds to a higher-energy amplitude
configuration of Φ. When an external \theta-wave with the same frequency interacts with this excited state:
\theta_external(t) \approx \theta_transition(t),
the excited vacuum defect becomes phase-locked and releases a new \theta-wave that is:
• identical in frequency,
• identical in direction,
• exactly in phase.
This is \textbf{stimulated emission}, seen as vacuum-phase synchronization.
6. Why Laser Photons Are Identical (Coherence)
Coherence in lasers arises naturally in DVFT because all \theta-excitations in the cavity share the same mode
of the vacuum phase field:
• Cavity geometry restricts allowed \theta-modes.
• Population inversion ensures many excited defects ready to emit.
• Stimulated emission entrains all emissions to the same \theta-pattern.
Thus, a laser beam is simply a \textbf{phase-coherent \theta-wave mode amplified by vacuum synchronization}.
7. Vacuum Interpretation of Population Inversion
Population inversion in DVFT corresponds to forcing many vacuum defects (electrons) into an amplitude
configuration with excess stored energy.
This excited configuration is metastable: the vacuum prefers to relax back to equilibrium by releasing \thetawave energy.
Thus, pumping creates a reservoir of amplitude energy that can be converted into coherent \theta-phase
radiation.
8. Laser Amplification and Resonance
In a laser cavity:
• \theta-waves reflect repeatedly between mirrors,
International Journal for Multidisciplinary Research (IJFMR)
E-ISSN: 2582-2160 ● Website: www.ijfmr.com ● Email: editor@ijfmr.com
IJFMR250664112 Volume 7, Issue 6, November-December 2025 72
• each pass triggers stimulated emission in inverted atoms,
• the \theta-wave amplitude increases exponentially.
This is \textbf{vacuum phase amplification} governed by constructive interference of \theta-modes. Output
coupling releases a stable, phase-aligned \theta-beam: the laser.
Conclusion
The photoelectric effect and laser physics follow naturally from the DVFT structure of vacuum fields:
• Photon = \theta-phase excitation
• Electron binding = amplitude barrier in \rho
• Emission requires \theta-frequency above \rho-barrier threshold
• Stimulated emission = phase entrainment of \theta
• Laser coherence = global \theta-mode synchronization
• Laser amplification = repeated \theta-phase reinforcement
DVFT provides a unified, physical explanation for optical and quantum phenomena without relying on
particle metaphors or classical wave–particle duality.
