% kapitel_38.tex – Stark erweiterte Version mit detaillierten mathematischen Ableitungen
\section{Eliminierung klassischer und quantenmechanischer Singularitäten in T0}

T0 eliminiert sowohl klassische gravitative Singularitäten (Schwarze Löcher, Big Bang) als auch quantenmechanische Punkt-Singularitäten (Selbstenergie-Divergenzen) durch die fraktale Regularisierung der Vakuum-Amplitude \(\rho\).

\subsection{Klassische Singularitäten – Mathematische Regularisierung}

Die Schwarzschild-Metrik divergiert bei \(r \to 0\):
\begin{equation}
	R \propto \frac{M^2}{r^4}.
\end{equation}

In T0 ist die Amplitude \(\rho(r)\) durch das Potential \(U(\rho)\) reguliert:
\begin{equation}
	U(\rho) = \Lambda_0 + \frac{\kappa}{2} (\rho - \rho_0)^2 + \frac{\lambda}{4} (\rho - \rho_0)^4.
\end{equation}

Die Bewegungsgleichung für \(\rho\):
\begin{equation}
	\Box \rho + \frac{dU}{d\rho} + \xi \cdot \rho \cdot \nabla^2 \mathcal{F}(r) = T^{00}.
\end{equation}

Im Kollaps: \(\rho\) sättigt bei
\begin{equation}
	\rho_{\max} \approx \rho_0 \cdot \xi^{-3/2},
\end{equation}
da höhere Terme dominieren.

Krümmung bleibt endlich:
\begin{equation}
	R_{\max} \approx \frac{c^4}{G \hbar} \cdot \xi^2.
\end{equation}

\subsection{Quanten-Punkt-Singularitäten – Selbstenergie}

In QFT divergiert die Selbstenergie eines Punktteilchens:
\begin{equation}
	\Delta E \propto \int^{k_{\max}} k^3 dk \approx k_{\max}^4.
\end{equation}

In T0 ist das Teilchen eine Amplitude-Deformation mit Radius \(l_0 \cdot \xi\):
\begin{equation}
	\delta \rho(x) = m c^2 / l_0^3 \cdot \xi \cdot e^{-r^2 / l_0^2 \xi^2}.
\end{equation}

Selbstenergie:
\begin{equation}
	\Delta E = \frac{G m^2}{c^2 l_0 \xi} \cdot \int e^{-2 r^2 / l_0^2 \xi^2} d^3r \approx \frac{G m^2}{c^2 l_0 \xi},
\end{equation}
endlich und klein.

\subsection{Vergleich mit anderen Ansätzen}

\begin{itemize}
	\item LQG: Diskrete Volumenoperatoren,
	\item Stringtheorie: Stringlänge \(l_s\),
	\item Asymptotic Safety: UV-fixer Punkt,
	\item T0: Fraktaler Cut-off durch \(\xi\), klassisch.
\end{itemize}

T0 ist minimaler – kein Quantisieren nötig.

\subsection{Schluss}

T0 eliminiert Singularitäten einheitlich: Klassisch durch Amplitude-Sättigung, quantenmechanisch durch fraktale Ausdehnung. Keine Divergenzen – alles endlich aus \(\xi\).