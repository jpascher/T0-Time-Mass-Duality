
BLACK HOLE AND QUANTUM SINGULARITIES
1. Introduction
This document presents a full, rigorous DVFT (Dynamic vacuum field Curvature Theory) explanation of
why *both* classical gravitational singularities (black holes) and quantum singularities (point particles,
infinite self-energy) cannot exist.
In DVFT, spacetime curvature and inertia emerge from the vacuum amplitude field:
\Phi(x,t) = \rho(x,t) e^{i\theta(x,t)},
with:
$\bullet$ \rho(x,t) -- vacuum amplitude (determines inertia and gravitational potential),
$\bullet$ \theta(x,t) -- phase field (determines quantum coherence and wave-like behavior).
Gravity emerges from amplitude gradients:
g = -$\nabla$\rho.
Singularities require \rho → $\infty$ or $\nabla$\rho → $\infty$. DVFT forbids both because the vacuum has finite stiffness and
inertial density, encoded in the potential U(\rho).
2. Why Singularities Cannot Exist in DVFT: The Vacuum Potential U(\rho)
DVFT postulates the vacuum has a microphysical potential:
U(\rho) = Λ$_0$ + (κ/2)(\rho - \rho$_0$)\textsuperscript{2} + (\lambda/4)(\rho - \rho$_0$)⁴ + \ldots{}
where:
$\bullet$ \rho$_0$ is the equilibrium vacuum amplitude,
$\bullet$ κ is the elastic stiffness of the vacuum,
International Journal for Multidisciplinary Research (IJFMR)
E-ISSN: 2582-2160 $\bullet$ Website: www.ijfmr.com $\bullet$ Email: editor@ijfmr.com
IJFMR250664112 Volume 7, Issue 6, November-December 2025 84
$\bullet$ \lambda stabilizes large deviations of \rho.
This potential is strongly convex at large |\rho - \rho$_0$|.
Thus, any attempt to compress the vacuum amplitude beyond moderate values requires infinite energy:
U(\rho) → $\infty$ as |\rho - \rho$_0$| → $\infty$.
Therefore:
$\bullet$ \rho cannot diverge,
$\bullet$ $\nabla$\rho cannot diverge,
$\bullet$ gravitational curvature cannot diverge.
This single microphysical fact eliminates *all* singularities in DVFT.
3. Removal of Quantum Singularities (Electron, Proton, Point Particles)
Quantum field theory treats electrons and quarks as point particles, leading to:
$\bullet$ infinite self-energy,
$\bullet$ divergent Coulomb self-field,
$\bullet$ undefined gravitational field at r = 0.
DVFT replaces a point mass with a finite vacuum amplitude deformation:
\delta\rho(x) = G m $\int$ d\textsuperscript{3}x' |ψ(x')|\textsuperscript{2} / |x - x'|.
This deformation is always finite because:
$\bullet$ |ψ(x)|\textsuperscript{2} is normalizable,
$\bullet$ convolution with 1/r smooths the field,
$\bullet$ U(\rho) prevents amplitude blow-up.
As a result:
$\bullet$ no particle has infinite self-energy,
$\bullet$ no wavefunction produces a singular potential,
$\bullet$ gravity is well-defined even in superposition.
Thus quantum singularities are eliminated by vacuum microphysics, not by renormalization.
4. Gravitational Field of a Delocalized Electron
An electron with wavefunction ψ(x,t) generates a vacuum amplitude profile:
\rho(x,t) = \rho$_0$ + G m_e $\int$ d\textsuperscript{3}x' |ψ(x',t)|\textsuperscript{2} / |x - x'|.
When ψ(x,t) spreads due to quantum dispersion, the gravitational field spreads with it:
g(x,t) = -$\nabla$\rho(x,t).
This ensures:
$\bullet$ gravity is fully compatible with Heisenberg uncertainty,
$\bullet$ gravitational fields have finite width,
$\bullet$ no r → 0 divergence occurs.
DVFT therefore produces the first consistent microscopic definition of gravity for a single quantum
particle.
5. Removal of Black Hole Singularities
In classical GR, gravitational collapse leads to infinite curvature at r = 0.
In DVFT, as matter compresses and raises \rho(x), the vacuum potential U(\rho) rapidly increases. At
sufficiently high density, a phase transition in the vacuum occurs:
$\bullet$ \rho stops increasing (vacuum stiffness prevents divergence),
$\bullet$ \theta becomes phase-locked (coherence inside horizon),
$\bullet$ matter transitions into a high-amplitude vacuum phase state,
International Journal for Multidisciplinary Research (IJFMR)
E-ISSN: 2582-2160 $\bullet$ Website: www.ijfmr.com $\bullet$ Email: editor@ijfmr.com
IJFMR250664112 Volume 7, Issue 6, November-December 2025 85
$\bullet$ gravitational field saturates.
Thus the black hole interior is NOT a singularity. It is a region of:
$\bullet$ finite \rho,
$\bullet$ finite $\nabla$\rho,
$\bullet$ finite energy density,
$\bullet$ vacuum-phase condensate.
The event horizon may still exist, but the spacetime interior remains regular.
6. DVFT Black Hole Interior Structure
DVFT predicts that inside a black hole:
$\bullet$ \rho(r) rises toward a maximum allowed value \rho_max,
$\bullet$ U(\rho) prevents further growth beyond \rho_max,
$\bullet$ curvature saturates,
$\bullet$ matter becomes vacuum-amplitude dominated,
$\bullet$ \theta freezes (phase coherence becomes rigid),
$\bullet$ no divergence in metric-equivalent quantities occurs.
This resembles:
$\bullet$ gravastar-like interiors,
$\bullet$ vacuum condensate cores,
$\bullet$ nonsingular loop quantum gravity solutions,
$\bullet$ but derived *entirely from DVFT microphysics*.
7. The Deep Reason DVFT Removes Both Types of Singularities
DVFT eliminates singularities because spacetime curvature is not fundamental. It is an *emergent
property* of the vacuum amplitude field \rho. If \rho cannot diverge, then curvature cannot diverge. The
vacuum’s elastic potential and finite inertial density are the mechanisms that prevent runaways.
Thus:
$\bullet$ matter cannot collapse to infinite density,
$\bullet$ wavefunctions cannot create divergent potentials,
$\bullet$ curvature cannot become infinite.
This is the first unified mechanism eliminating singularities across classical and quantum domains.
8. Comparison with GR, LQG, and QFT
General Relativity (GR):
$\bullet$ predicts unavoidable singularities (Hawking-Penrose theorems),
$\bullet$ has no internal regulator for curvature.
Loop Quantum Gravity (LQG):
$\bullet$ introduces discrete geometry,
$\bullet$ removes singularities by quantizing spacetime,
$\bullet$ but requires radical nonlocality and lacks experimental grounding.
Quantum Field Theory:
$\bullet$ produces infinite point-particle self-energies,
$\bullet$ resolves them only through renormalization,
$\bullet$ does not address gravitational singularity.
DVFT:
$\bullet$ retains continuum spacetime,
International Journal for Multidisciplinary Research (IJFMR)
E-ISSN: 2582-2160 $\bullet$ Website: www.ijfmr.com $\bullet$ Email: editor@ijfmr.com
IJFMR250664112 Volume 7, Issue 6, November-December 2025 86
$\bullet$ derives gravity from a physical vacuum field,
$\bullet$ imposes finite amplitude & stiffness,
$\bullet$ eliminates both self-energy and gravitational singularities,
$\bullet$ without renormalization,
$\bullet$ without quantizing spacetime,
$\bullet$ without modifying quantum mechanics.
DVFT is the simplest and most physically grounded solution among all three.
9. Final Summary
DVFT eliminates singularities through vacuum amplitude dynamics:
1. The vacuum field \Phi = \rho e^{i\theta} has finite stiffness and inertial density.
2. U(\rho) prevents \rho from diverging under collapse.
3. Quantum particles generate finite vacuum amplitude deformations from |ψ|\textsuperscript{2}.
4. Gravity emerges as $\nabla$\rho, which can never diverge.
5. Black holes contain vacuum-phase condensates, not singularities.
6. No infinite self-energy, no point divergences, no r → 0 explosion exists.
DVFT therefore provides the first unified, microphysically consistent elimination of:
$\bullet$ black hole singularities,
$\bullet$ quantum point singularities,
$\bullet$ gravitational field singularities.
This positions DVFT as a fundamentally complete framework bridging general relativity and quantum
mechanics.