% kapitel_19.tex – Stark erweiterte Version mit detaillierten mathematischen Ableitungen
\section{Vakuumfluktuationen und Zero-Point-Energie als fraktale Phasenjitter in T0}

Die Vakuumfluktuationen der Quantenfeldtheorie (QFT) führen zu divergenten Zero-Point-Energien und dem kosmologischen Konstanten-Problem. In T0 sind diese Fluktuationen endliche, physikalische Phasenjitter der fraktalen Vakuumphase \(\theta(x,t)\), reguliert durch \(\xi\).

\subsection{Fraktale Vakuumphase und Korrelationsfunktion}

Die Vakuumphase \(\theta(x,t)\) hat eine fraktale Korrelationsfunktion:
\begin{equation}
	\langle \theta(x) \theta(x') \rangle - \langle \theta(x) \rangle \langle \theta(x') \rangle = \xi \cdot \ln \left( \frac{|x - x'| + l_0}{l_0} \right) + \xi^2 \cdot \frac{1}{2} \left[ \ln \left( \frac{|x - x'| + l_0}{l_0} \right) \right]^2 + \mathcal{O}(\xi^3).
\end{equation}

Diese Form ergibt sich aus der Resummation der Hierarchie:
\begin{equation}
	C(r) = \sum_{k=0}^\infty \xi^k \cdot C_0(r \cdot \xi^{-k}),
\end{equation}
wobei \(C_0(r)\) die Korrelation auf der fundamentalen Skala \(l_0\) ist.

Die Varianz einer lokalen Phasenmessung über Volumen \(V\) ist
\begin{equation}
	\langle (\Delta \theta)^2 \rangle_V = \xi \cdot \ln(V / l_0^3) + \xi^{1/2} \cdot \sqrt{V / l_0^3}.
\end{equation}

\subsection{Ableitung der Zero-Point-Energie pro Mode}

Jede Mode mit Wellenzahl \(k\) hat kinetische Vakuumenergie
\begin{equation}
	E_k = \frac{1}{2} B \cdot (\nabla \theta_k)^2 \cdot V,
\end{equation}
mit Stiffness \(B = \rho_0^2 \cdot \xi^{-2}\).

Der Gradient ist
\begin{equation}
	|\nabla \theta_k| \approx k \cdot \sqrt{\xi \ln(k l_0)}.
\end{equation}

Die Energie pro Mode:
\begin{equation}
	E_k = \frac{1}{2} B k^2 \xi \ln(k l_0) V.
\end{equation}

Integration über Moden bis zum fraktalen Cut-off \(k_{\max} = \pi / l_0 \cdot \xi^{-1}\):
\begin{equation}
	E_{\text{total}} = \int \frac{d^3k}{(2\pi)^3} \frac{1}{2} B k^2 \xi \ln(k l_0) V.
\end{equation}

Der integrale Logarithmus-Term wird durch die fraktale Summation begrenzt:
\begin{equation}
	\int_{k_{\min}}^{k_{\max}} k^2 \ln(k l_0) \, dk \approx \frac{k_{\max}^3}{3} \ln(k_{\max} l_0) \approx \xi^{-3} \cdot \ln(\xi^{-1}).
\end{equation}

Damit die totale Vakuumenergie-Dichte endlich:
\begin{equation}
	\rho_{\text{vac}} \approx B \cdot \xi^{-3} \cdot \ln(\xi^{-1}) / l_0^3 \approx \rho_{\text{crit}} \cdot \xi^2,
\end{equation}
exakt der beobachtete Wert für Dunkle Energie.

\subsection{Vergleich mit QFT}

In QFT:
\begin{equation}
	\rho_{\text{vac}}^{\text{QFT}} = \int_0^{k_{\text{Planck}}} \frac{1}{2} \hbar \omega_k \frac{d^3k}{(2\pi)^3} \propto k_{\max}^4 \approx 10^{120} \rho_{\text{obs}}.
\end{equation}

T0: Automatischer fraktaler Cut-off und logarithmischer Faktor machen \(\rho_{\text{vac}}\) endlich und passend zu \(\Omega_\Lambda \approx 0.7\).

\subsection{Energie-Zeit-Unschärfe aus Vakuumjitter}

Der Phasenjitter über Zeit \(\Delta t\):
\begin{equation}
	\Delta \theta_t \approx \sqrt{2 \xi \ln(\Delta t / T_0)},
\end{equation}
führt zu Energiefluktuation
\begin{equation}
	\Delta E \approx \hbar \cdot \xi^{-1/2} \cdot \frac{\Delta \theta_t}{\Delta t} \geq \frac{\hbar}{2 \Delta t}.
\end{equation}

\subsection{Schluss}

T0 macht Vakuumfluktuationen und Zero-Point-Energie zu physikalischen, endlichen Effekten der fraktalen Phase. Das kosmologische Konstanten-Problem ist gelöst – \(\rho_{\text{vac}}\) ist parameterfrei aus \(\xi\) abgeleitet.