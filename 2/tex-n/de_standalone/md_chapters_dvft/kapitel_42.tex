% Chapter: Kapitel 42
\section{Kapitel 42
}


PLANCK UNITS AND UNIVERSAL CONSTANTS
1. Introduction
This document explains how Dynamic Vacuum Field Theory(DVFT) derives the Planck time, length, and
mass, as well as other 'universal constants', from the fundamental vacuum parameters:
• B – vacuum phase stiffness
• \rho₀ – inertial vacuum density
• K₀ – amplitude stiffness of vacuum
• \lambdaₘ – matter–vacuum coupling constant
• ħ – emerging from topological phase quantization
• \theta-winding scale – phase gradient associated with unit charge
International Journal for Multidisciplinary Research (IJFMR)
E-ISSN: 2582-2160 ● Website: www.ijfmr.com ● Email: editor@ijfmr.com
IJFMR250664112 Volume 7, Issue 6, November-December 2025 94
DVFT shows that Planck units are \textit{not fundamental constants} but emergent mechanical properties of
the vacuum field Φ = \rho e^{i\theta}.
2. DVFT Vacuum Parameters
The key numerical vacuum parameters are:
• Phase stiffness: B \approx 8.7 \times 10⁻⁵⁵
• Inertial vacuum density: \rho₀ \approx 6 \times 10⁻²⁷ kg/m³
• Amplitude stiffness: K₀ \approx 5.4 \times 10⁻¹⁰ J/m³
• Phase gradient for one charge: |\partial\theta/\partialx|ₑ \approx 1.63 \times 10¹³ m⁻¹
• Speed of light (derived): c = \sqrt(K₀ / \rho₀)
• Newton’s G (derived): G = \lambdaₘ / (4\pi K₀)
• Fine-structure constant (derived): \alpha = (B / ħ c)(\partial\theta/\partialx)²
These constants collectively define the mechanical, gravitational, and quantum architecture of the vacuum.
3. DVFT Substitutes into Planck Units
Textbook definitions of Planck units are:
• t_P = \sqrt(ħ G / c⁵)
• ℓ_P = \sqrt(ħ G / c³)
• m_P = \sqrt(ħ c / G)
But in DVFT, none of ħ, c, or G are fundamental:
• c = \sqrt(K₀ / \rho₀)
• G = \lambdaₘ / (4\pi K₀)
• ħ arises from \theta-winding quantization
Substituting these relations gives the Planck units as explicit composites of DVFT vacuum parameters.
4. Planck Time from DVFT
Starting with:
t_P = \sqrt(ħ G / c⁵)
Insert:
c = \sqrt(K₀/\rho₀)
G = \lambdaₘ / (4\pi K₀)
Compute:
t_P = \sqrt{ (ħ \lambdaₘ / (4\pi K₀)) / (K₀/\rho₀)^{5/2} }
Simplify:
t_P = \sqrt{ ħ \lambdaₘ \rho₀^{5/2} / (4\pi K₀^{7/2}) }.
This is the DVFT expression for Planck time.
Interpretation:
Planck time is the minimum time scale at which vacuum amplitude curvature can sustain a stable
oscillation.
It is not a fundamental limit of nature, but a material property of the vacuum.
5. Planck Length from DVFT
Textbook definition:
ℓ_P = \sqrt(ħ G / c³)
Substitute:
G = \lambdaₘ / (4\pi K₀)
c³ = (K₀/\rho₀)^{3/2}
International Journal for Multidisciplinary Research (IJFMR)
E-ISSN: 2582-2160 ● Website: www.ijfmr.com ● Email: editor@ijfmr.com
IJFMR250664112 Volume 7, Issue 6, November-December 2025 95
Result:
ℓ_P = \sqrt{ ħ \lambdaₘ \rho₀^{3/2} / (4\pi K₀^{5/2}) }.
Interpretation:
Planck length is the smallest stable spatial scale of vacuum amplitude curvature — the 'acoustic
wavelength' of the vacuum medium.
6. Planck Mass from DVFT
Textbook definition:
m_P = \sqrt(ħ c / G)
Insert:
c = \sqrt(K₀/\rho₀)
G = \lambdaₘ / (4\pi K₀)
Compute:
m_P = \sqrt{ (ħ \sqrt(K₀/\rho₀)) (4\pi K₀)/\lambdaₘ }
Simplify:
m_P = \sqrt{ 4\pi ħ K₀^{3/2} / (\lambdaₘ \rho₀^{1/2}) }.
Interpretation:
Planck mass is the amplitude deformation that matches one quantum of phase curvature.
7. Physical Meaning: Planck Units Are Emergent Vacuum Properties
In DVFT:
• Planck time → minimum oscillation time of vacuum amplitude
• Planck length → minimum spatial curvature scale of vacuum amplitude
• Planck mass → amplitude curvature equivalent to one phase quantum
This new interpretation replaces the vague 'quantum gravity scale' with clear mechanical meaning.
Planck units describe \textbf{acoustic-like resonance properties} of the vacuum medium.
8. Other Constants Derived from DVFT
DVFT reduces many universal constants to derivatives of vacuum parameters:
1. Speed of light:
c = \sqrt(K₀/\rho₀)
2. Gravitational constant:
G = \lambdaₘ / (4\pi K₀)
3. Fine-structure constant:
\alpha = (B / ħ c)(\partial\theta/\partialx)²
4. Electron charge:
e² = 4\pi \epsilon₀ ħ c \alpha → e arises from B and phase topology
5. Dark-energy density:
\rho_\Lambda c² \approx K₀
6. Deep-field acceleration scale (MOND-like):
a₀ \approx c² / L_\textit{ (L_} = cosmic coherence length)
7. Neutrino mass scale:
m_\nu ∝ B (phase oscillation over long coherence lengths)
8. Quantum coherence length of vacuum:
L_coh \approx \sqrt(ħ / B)
Every one of these constants is derived — none are fundamental.
International Journal for Multidisciplinary Research (IJFMR)
E-ISSN: 2582-2160 ● Website: www.ijfmr.com ● Email: editor@ijfmr.com
IJFMR250664112 Volume 7, Issue 6, November-December 2025 96
9. Consequences for Physics
Because all universal constants are derived from the vacuum parameters, DVFT provides:
• A complete unification of gravity, quantum mechanics, and electromagnetism
• A physical explanation for Planck units
• A mechanism for dark energy
• The origin of \alpha, e, c, G, ħ
• Predictive power across scales from the proton to cosmology
• A new foundation for quantum technologies (phase-based computing)
DVFT reinterprets the universe as a material medium with definable mechanical constants B, K₀, \rho₀, from
which all physical scales emerge.
Conclusion
DVFT transforms the Planck constants from unexplained numerology into physically meaningful
emergent properties of the vacuum’s amplitude–phase structure. This resolves long-standing conceptual
gaps between quantum mechanics, relativity, and cosmology, and positions DVFT as a unified framework
where the numerical structure of the universe is derived from the underlying nature of the vacuum.
