\documentclass[12pt,a4paper]{article}

% Packages
\usepackage[utf8]{inputenc}
\usepackage[T1]{fontenc}
\usepackage{amsmath}
\usepackage{amssymb}
\usepackage{physics}
\usepackage{graphicx}
\usepackage{hyperref}
\usepackage[margin=2.5cm]{geometry}

% Physics notation
\renewcommand{\varphi}{\phi}

\title{Kapitel 06
}
\author{DVFT - Dynamic Vacuum Field Theory}
\date{\today}

\begin{document}

\section{Kapitel 06
}


REINTERPRETATION OF E = MC²
1. Introduction
This chapter derives Einstein’s mass–energy relation E = mc² purely from the Dynamic Vacuum Field
Theory (DVFT), without using Einstein’s field equations. The DVFT provides physical explanation of
conversion of mass into energy. The mass is nothing but the knotted compressed vacuum field. When
mass converts into energy, the compressed vacuum energy gets released in the form of light.
DVFT treats spacetime as a physical quantum medium described by the phase field \theta(x,t). Particles appear
as localized excitations of this vacuum medium, and their mass is interpreted as stored vacuum energy.
From this viewpoint, E = mc² emerges naturally from the dynamics of the vacuum field.
2. The DVFT Vacuum Field
The vacuum is represented by the complex order parameter:
Φ(x) = \rho(x) e^{i\theta(x)},
with \rho the vacuum density and \theta the vacuum phase.
In flat spacetime, the DVFT kinetic invariant is:
X = (1/c²)(\partial_t\theta)² − (\nabla\theta)².
A simplified DVFT Lagrangian for deriving particle-like excitations is:
𝓛_\theta = −\Lambda_v + (\rho₀/2)X − (\eta/(3a₀²)) X^{3/2}.
To quantize and analyze particle excitations, we expand the vacuum phase field around a background
value:
\theta(x) = \theta₀ + \varphi(x).
International Journal for Multidisciplinary Research (IJFMR)
E-ISSN: 2582-2160 ● Website: www.ijfmr.com ● Email: editor@ijfmr.com
IJFMR250664112 Volume 7, Issue 6, November-December 2025 15
3. Quadratic Expansion of the DVFT Action
For small \varphi(x), the leading-order dynamics become:
𝓛_free = (\rho₀/2)[ (1/c²)(\partial_t\varphi)² − (\nabla\varphi)² ] − (1/2) m_\theta² \varphi².
By defining a canonically normalized field:
\varphi_c = \sqrt\rho₀ \varphi,
the free field Lagrangian becomes:
𝓛_free = (1/2)[ (1/c²)(\partial_t\varphi_c)² − (\nabla\varphi_c)² ] − (1/2) m_\theta² \varphi_c².
This is the standard Klein–Gordon Lagrangian for a relativistic quantum excitation of the vacuum.
4. Dispersion Relation of DVFT Vacuum Excitations
The equation of motion is the Klein–Gordon equation:
(1/c²) \partial_t² \varphi_c − \nabla² \varphi_c + m_\theta² \varphi_c = 0.
Using plane-wave solutions:
\varphi_c = A e^{i(k·x − \omegat)},
we obtain the dispersion relation:
\omega² = c²(k² + m_\theta²).
Define the particle energy and momentum:
E = ħ\omega,
p = ħk.
Then the dispersion relation becomes:
E² = p²c² + (ħ m_\theta c)².
Identify the particle mass as:
m = ħ m_\theta / c.
Thus, the DVFT vacuum excitations obey:
E² = p²c² + m² c⁴.
In the rest frame of the vacuum excitation (p = 0), the dispersion relation reduces to:
E² = m² c⁴.
Taking the positive-energy branch:
E = mc².
This is derived entirely from the DVFT vacuum field Lagrangian and its excitations—no Einstein field
equations or GR postulates were used.
Thus, in DVFT:
• Mass m is the parameter determining the intrinsic oscillation frequency of the vacuum phase field
at zero momentum.
• E = mc² states that rest energy equals the stored vacuum energy in the localized excitation (the
particle).
5. Vacuum Energy Interpretation of Mass
From the DVFT Hamiltonian density:
𝓗 = (1/2c²)(\partial_t\varphi_c)² + (1/2)(\nabla\varphi_c)² + (1/2) m_\theta² \varphi_c²,
the total energy of a localized excitation is:
E = \int d³x 𝓗.
For a rest-frame solution, this energy evaluates to:
E = mc².
Thus, mass is the vacuum energy stored in a stable \theta-excitation.
International Journal for Multidisciplinary Research (IJFMR)
E-ISSN: 2582-2160 ● Website: www.ijfmr.com ● Email: editor@ijfmr.com
IJFMR250664112 Volume 7, Issue 6, November-December 2025 16
No separate "mass substance" exists: mass is simply bound vacuum energy.
6. Physical Meaning of E = mc² in DVFT
DVFT gives a more satisfying interpretation of E = mc²:
1. A particle is a localized distortion of the vacuum phase field.
2. Its mass m measures the resistance of the vacuum to changing this localized pattern.
3. Its rest energy mc² is the total vacuum energy stored in that pattern.
4. Nuclear reactions (fission, fusion) release energy not because "mass turns into energy," but because
vacuum configurations reorganize.
5. The difference in vacuum energy between initial and final configurations gives \DeltaE = \Delta(mc²).
Conclusion
E = mc² emerges naturally from DVFT as the rest-energy relation for quantized vacuum-phase excitations.
The result is fully derivable from the DVFT Lagrangian using:
• Expansion around the vacuum,
• Canonical normalization,
• Klein–Gordon dynamics,
• Energy–momentum identification.
Mass–energy equivalence arises fundamentally from the microstructure of the vacuum in DVFT.

\end{document}
