\documentclass[12pt,a4paper]{article}

% Packages
\usepackage[utf8]{inputenc}
\usepackage[T1]{fontenc}
\usepackage{amsmath}
\usepackage{amssymb}
\usepackage{physics}
\usepackage{graphicx}
\usepackage{hyperref}
\usepackage[margin=2.5cm]{geometry}

% Physics notation
\renewcommand{\varphi}{\phi}

\title{Kapitel 04
}
\author{DVFT - Dynamic Vacuum Field Theory}
\date{\today}

\begin{document}

\section{Kapitel 04
}


GRAVITATIONAL CURVATURE EQUATIONS
1. Introduction
This chapter presents a complete formulation of gravitational curvature using the Dynamic Vacuum Field
Theory (DVFT). Curvature emerges from the interplay between the metric g_{\mu\nu} and the vacuum phase
field \theta through the DVFT action. The result is a unified set of equations one for the vacuum field \theta and
one for the spacetime curvature. GR appears as the high-acceleration limit of DVFT.
2. DVFT Fundamentals
The vacuum is modeled as a dynamic vacuum field described by the complex order parameter:
Φ(x) = \rho(x) e^{i\theta(x)}.
The gravitational degrees of freedom include:
• Metric g_{\mu\nu}, determining curvature.
• Phase field \theta, governing vacuum convergence.
The kinetic invariant is:
X ≡ -g^{\mu\nu} \nabla_\mu\theta \nabla_\nu\theta.
The Dynamic vacuum field Curvature Tensor (DVFT) is defined as:
V_{\mu\nu} ≡ \nabla_\mu\nabla_\nu\theta − (1/4) g_{\mu\nu} □\theta,
with □\theta = g^{\alpha\beta} \nabla_\alpha\nabla_\beta\theta.
3. DVFT Action (Pure Gravity + Vacuum + Matter)
The full DVFT action is:
S = \int d⁴x \sqrt−g [ (1/(16\piG)) R + 𝓛_\theta(X, I₁, I₂) + 𝓛_m(g_{\mu\nu},\psi_m) ].
Here:
• R is the Ricci scalar (geometry),
• 𝓛_m is matter Lagrangian,
• 𝓛_\theta encodes vacuum microphysics:
𝓛_\theta = −\Lambda_v + (\rho₀/2)X − (\eta/(3a₀²)) X^{3/2} + \alpha₁ I₁ + \alpha₂ I₂,
with invariants:
I₁ = V_{\mu\nu} V^{\mu\nu},
I₂ = V_{\mu}^{ \alpha} V_{\alpha}^{ \beta} V_{\beta}^{ \mu}.
4. \theta Field Equation (Dynamics)
Varying S with respect to \theta gives the DVFT vacuum equation:
\nabla_\mu ( 𝓛_X \nabla^\mu\theta ) + \alpha₁ 𝓔^{(1)}[\theta,g] + \alpha₂ 𝓔^{(2)}[\theta,g] = 0,
where:
𝓛_X = \partial𝓛_\theta/\partialX = \rho₀/2 − (\eta/(2a₀²)) X^{1/2}.
This is a nonlinear wave equation for \theta. It determines how the vacuum phase converges into matter and
controls weak-field gravity without needing GR.
International Journal for Multidisciplinary Research (IJFMR)
E-ISSN: 2582-2160 ● Website: www.ijfmr.com ● Email: editor@ijfmr.com
IJFMR250664112 Volume 7, Issue 6, November-December 2025 12
5. Curvature Equation from Metric Variation
Varying S with respect to the metric g_{\mu\nu} yields:
G_{\mu\nu} = 8\piG ( T^{(m)}_{\mu\nu} + T^{(\theta)}_{\mu\nu} ),
where G_{\mu\nu} is the Einstein tensor arising from variation of \sqrt−g R.
The vacuum stress-energy T^{(\theta)}_{\mu\nu} splits into:
1. k-essence (from X):
T^{(\theta,kess)}_{\mu\nu} = 2 𝓛_X \nabla_\mu\theta \nabla_\nu\theta − g_{\mu\nu} 𝓛_\theta(kess).
2. DVFT curvature-like part:
T^{(\theta,DVFT)}_{\mu\nu} = 2\alpha₁ \partialI₁/\partialg^{\mu\nu} + 2\alpha₂ \partialI₂/\partialg^{\mu\nu} − g_{\mu\nu}(\alpha₁ I₁ + \alpha₂ I₂).
Thus, curvature is determined entirely by \theta dynamics and matter, not by assuming Einstein’s equation.
6. Pure DVFT Gravitational Equation
Define the total vacuum tensor:
T^{(\theta)}_{\mu\nu} = T^{(\theta,kess)}_{\mu\nu} + T^{(\theta,DVFT)}_{\mu\nu}.
Then the fundamental DVFT gravitational curvature law is:
E_{\mu\nu}[\theta,g] ≡ (1/(8\piG)) G_{\mu\nu} − T^{(\theta)}_{\mu\nu} = T^{(m)}_{\mu\nu}.
This replaces Einstein’s equations. GR is recovered when \theta’s nonlinearities vanish.
7. GR as a Limiting Case of DVFT
In high-acceleration environments (Solar System, neutron stars):
• X is large → 𝓛_X \approx constant.
• DVFT invariants I₁, I₂ are suppressed.
• T^{(\theta)}_{\mu\nu} \approx −\Lambda_eff g_{\mu\nu}.
Then DVFT Gravitational Equation reduces to:
G_{\mu\nu} + \Lambda_eff g_{\mu\nu} \approx 8\piG T^{(m)}_{\mu\nu},
which is Einstein’s equation with a cosmological constant.
Thus, GR is not fundamental—it's the high-g limit of DVFT.
8. Low-Acceleration Curvature: Pure DVFT Regime
In galaxies (g ~ a₀ or below):
• Nonlinear term X^{3/2} dominates,
• DVFT invariants contribute significantly,
• \theta-field deviates strongly from GR predictions.
The curvature now follows pure DVFT dynamics:
G_{\mu\nu} \approx 8\piG T^{(\theta)}_{\mu\nu},
leading to flat rotation curves and MOND-like behavior without dark matter. Example of two galaxies
NGC-3198 and Andromeda rotational speed calculation using DVFT has been shown in next chapter.
9. Summary of DVFT-Only Curvature Framework
Using DVFT, gravitational curvature is fully described by:
1. \theta-field equation:
\nabla_\mu( 𝓛_X \nabla^\mu\theta ) + DVFT terms = 0.
2. Pure DVFT curvature equation:
G_{\mu\nu} = 8\piG ( T^{(m)}_{\mu\nu} + T^{(\theta)}_{\mu\nu} ).
No Einstein field equations are introduced by hand—GR emerges only as a limiting case. This is a
complete gravitational theory in its own right, derived purely from dynamic vacuum field microphysics.
International Journal for Multidisciplinary Research (IJFMR)
E-ISSN: 2582-2160 ● Website: www.ijfmr.com ● Email: editor@ijfmr.com
IJFMR250664112 Volume 7, Issue 6, November-December 2025 13

\end{document}
