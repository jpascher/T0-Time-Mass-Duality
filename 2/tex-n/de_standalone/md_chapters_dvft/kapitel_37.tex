
INTRINSIC PROPERTIES OF THE VACUUM FIELD
1. Introduction
This document compiles the intrinsic numerical parameters of the vacuum field in DVFT (Dynamic
vacuum field Curvature Theory). Unlike conventional physics, where vacuum constants such as \alpha, \varepsilon$_0$, ħ,
c, and even cosmological density appear as disconnected inputs, DVFT unifies them under the dynamics
of a single complex vacuum field:
\Phi(x,t) = \rho(x,t) e^{i\theta(x,t)}
Here:
$\bullet$ \rho(x,t) is the vacuum amplitude (inertial density, gravitational stiffness).
$\bullet$ \theta(x,t) is the vacuum phase (quantum coherence, charge, CP violation).
The constants governing \rho and \theta define the mechanical, electromagnetic, and quantum structure of spacetime itself. This document consolidates their values and shows how they relate to observable physics.
2. Fundamental DVFT Vacuum Parameters
DVFT introduces the following intrinsic vacuum parameters:
1. B -- Vacuum phase stiffness
2. \rho$_0$ -- Inertial vacuum density
3. K$_0$ -- Amplitude stiffness of the vacuum
4. ($\partial$\theta/$\partial$x) -- Fundamental phase gradient corresponding to one unit of electric charge
These determine all quantum, electromagnetic, and gravitational behavior emerging from \Phi.
3. Phase Stiffness B (Calibrated from \alpha)
The fine-structure constant \alpha is expressed in DVFT as:
\alpha = (B / ħ c) ($\partial$\theta/$\partial$x)\textsuperscript{2}
Choosing the phase gradient associated with one unit charge as:
|$\partial$\theta/$\partial$x|  $\approx$  2\pi / \lambda_C, \lambda_C = ħ / (m_e c)  $\approx$  3.86  $\times$  10⁻¹\textsuperscript{3} m,
gives: |$\partial$\theta/$\partial$x|  $\approx$  1.63  $\times$  10¹\textsuperscript{3} m⁻¹.
Using \alpha_exp = 1/137.036, the resulting vacuum phase stiffness is:
B  $\approx$  8.7  $\times$  10⁻⁵⁵ (unit depends on normalization of Lagrangian).
Interpretation:
$\bullet$ B measures how hard it is to twist the vacuum phase \theta.
$\bullet$ This same B must be used for electromagnetism, neutrino masses, baryogenesis, and quantum
coherence.
4. Inertial Vacuum Density \rho$_0$
\rho$_0$ is taken from the effective mass-equivalent density of dark energy:
\rho$_0$  $\approx$  6  $\times$  10⁻\textsuperscript{2}⁷ kg/m\textsuperscript{3}.
This represents the intrinsic inertial content of the vacuum amplitude \rho, which couples directly to
gravitational behavior.
5. Amplitude Stiffness K$_0$ (via c = $\sqrt$(K$_0$/\rho$_0$))
DVFT identifies the speed of light with the ratio of amplitude stiffness to inertial density:
c\textsuperscript{2} = K$_0$ / \rho$_0$ → K$_0$ = \rho$_0$ c\textsuperscript{2}.
Substituting \rho$_0$  $\approx$  6 $\times$ 10⁻\textsuperscript{2}⁷ kg/m\textsuperscript{3} and c  $\approx$  3 $\times$ 10⁸ m/s gives:
K$_0$  $\approx$  5.4  $\times$  10⁻¹\textsuperscript{0} J/m\textsuperscript{3}.
This value is close to the observed dark-energy density, suggesting a deep relationship between vacuum
elasticity and cosmic acceleration.
International Journal for Multidisciplinary Research (IJFMR)
E-ISSN: 2582-2160 $\bullet$ Website: www.ijfmr.com $\bullet$ Email: editor@ijfmr.com
IJFMR250664112 Volume 7, Issue 6, November-December 2025 82
5. Fundamental Phase Gradient (Unit Charge)
For a unit electric charge, the vacuum phase winds by 2\pi over a microscopic radius taken to be the electron
Compton wavelength:
\lambda_C = ħ / (m_e c)  $\approx$  3.86  $\times$  10⁻¹\textsuperscript{3} m.
Thus:
|$\partial$\theta/$\partial$x|_e  $\approx$  2\pi / \lambda_C  $\approx$  1.63  $\times$  10¹\textsuperscript{3} m⁻¹.
This gradient defines the microscopic "twist" of the vacuum phase corresponding to one unit of electric
charge.
7. Derived DVFT Quantities
Once B, \rho$_0$, K$_0$, and |$\partial$\theta/$\partial$x| are set, DVFT determines a wide range of vacuum properties:
1. Speed of Light:
c = $\sqrt$(K$_0$/\rho$_0$)  $\approx$  3  $\times$  10⁸ m/s.
1. Fine-Structure Constant:
\alpha = (B / ħ c)($\partial$\theta/$\partial$x)\textsuperscript{2} → \alpha  $\approx$  1/137 (by calibration).
2. Deep-Field Acceleration Scale (galactic regime):
a$_0$  $\approx$  c\textsuperscript{2} / L_*,
where L_* is the cosmic coherence length (~Hubble radius).
This gives the correct MOND-like acceleration scale ~1 $\times$ 10⁻¹\textsuperscript{0} m/s\textsuperscript{2}.
3. Neutrino Mass Scale:
m_ν ∝ B ($\partial$\theta/$\partial$x)\textsuperscript{2} evaluated at long coherence scales,
yielding naturally small masses: 0.01--0.05 eV.
4. Quantum Coherence Length of Vacuum:
L_coh  $\approx$  $\sqrt$(ħ / B),
which becomes extremely large due to tiny B, enabling phase coherence across cosmological distances.
5. Dark-Energy Behavior:
U(\rho$_0$)  $\approx$  K$_0$ ∼ 10⁻¹\textsuperscript{0} J/m\textsuperscript{3},
matching observed vacuum energy density.
8. Why Using a Single B Everywhere Is Consistent
B must be universal because:
$\bullet$ \theta is a universal phase field in DVFT.
$\bullet$ All quantum phenomena (charge, CP violation, coherence, neutrino masses, photon propagation,
baryogenesis) arise from the same \theta-dynamics.
$\bullet$ A single stiffness constant ensures unification, just as ħ and c apply universally in conventional
physics.
This allows DVFT to coherently explain:
$\bullet$ Quantum mechanics
$\bullet$ Electromagnetism
$\bullet$ Neutrino behavior
$\bullet$ Deep-field gravity
$\bullet$ Dark energy
$\bullet$ Early-universe CP asymmetry
all through the same vacuum field.
9. Summary of Intrinsic Vacuum Parameters
International Journal for Multidisciplinary Research (IJFMR)
E-ISSN: 2582-2160 $\bullet$ Website: www.ijfmr.com $\bullet$ Email: editor@ijfmr.com
IJFMR250664112 Volume 7, Issue 6, November-December 2025 83
DVFT Vacuum Parameter Sheet:
$\bullet$ Phase stiffness: B  $\approx$  8.7  $\times$  10⁻⁵⁵
$\bullet$ Inertial vacuum density: \rho$_0$  $\approx$  6  $\times$  10⁻\textsuperscript{2}⁷ kg/m\textsuperscript{3}
$\bullet$ Amplitude stiffness: K$_0$  $\approx$  5.4  $\times$  10⁻¹\textsuperscript{0} J/m\textsuperscript{3}
$\bullet$ Fundamental phase gradient for one charge: |$\partial$\theta/$\partial$x|_e  $\approx$  1.63  $\times$  10¹\textsuperscript{3} m⁻¹
$\bullet$ Coherence length: L_coh  $\approx$  $\sqrt$(ħ/B) → enormous (cosmic-scale)
$\bullet$ Deep-field acceleration: a$_0$  $\approx$  10⁻¹\textsuperscript{0} m/s\textsuperscript{2}
$\bullet$ Speed of light: c = $\sqrt$(K$_0$/\rho$_0$)
Together, these define the intrinsic mechanical, electromagnetic, quantum, and gravitational structure of
the DVFT vacuum.
Conclusion
The numerical vacuum parameters in DVFT are consistent with known electromagnetic, quantum, and
cosmological observations. By fixing B from \alpha and anchoring \rho$_0$ and K$_0$ in cosmology, the entire quantum
and gravitational framework emerges from a single unified vacuum field \Phi = \rho e^{i\theta}.
These parameters provide the first coherent numerical foundation for a theory that unifies:
$\bullet$ Special relativity
$\bullet$ Quantum mechanics
$\bullet$ Electromagnetism
$\bullet$ Neutrino physics
$\bullet$ Baryogenesis
$\bullet$ Dark energy
$\bullet$ Galactic dynamics (without dark matter) within one field-based vacuum framework.