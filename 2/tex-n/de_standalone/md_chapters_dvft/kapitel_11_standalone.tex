\documentclass[12pt,a4paper]{article}

% Packages
\usepackage[utf8]{inputenc}
\usepackage[T1]{fontenc}
\usepackage{amsmath}
\usepackage{amssymb}
\usepackage{physics}
\usepackage{graphicx}
\usepackage{hyperref}
\usepackage[margin=2.5cm]{geometry}

% Physics notation
\renewcommand{\varphi}{\phi}

\title{Kapitel 11
}
\author{DVFT - Dynamic Vacuum Field Theory}
\date{\today}

\begin{document}

\section{Kapitel 11
}


BLACK HOLE INTERIOR PREDICTION
This chapter presents a complete description of black hole interiors in the Dynamic Vacuum Field
Theory(DVFT). DVFT replaces the classical singularity of General Relativity (GR) with a finite-density
quantum vacuum core, using a nonlinear phase field \theta. Both the mathematical structure and the physical
interpretation are provided.
1. DVFT Overview
DVFT treats spacetime as a quantum vacuum medium described by a complex order parameter:
Φ = \rho e^{i\theta}
Gravity arises from dynamic vacuum field with amplitude \rho and phase \theta. The Lagrangian contains
nonlinear kinetic terms:
L_\theta = -\Lambda_v + (\rho_0/2)X - (\eta/(3 a_0^2)) X^{3/2}
with X = -g^{\mu\nu} \partial_\mu\theta \partial_\nu\theta.
At large accelerations (g >> a_0), DVFT reduces to GR. At small accelerations (g << a_0), nonlinearities
appear.
2. Black Hole Metric and Field Ansatz
We use the standard static spherically symmetric metric:
ds² = -e^{2Φ(r)}dt² + dr²/(1 - 2Gm(r)/r) + r² d\Omega².
International Journal for Multidisciplinary Research (IJFMR)
E-ISSN: 2582-2160 ● Website: www.ijfmr.com ● Email: editor@ijfmr.com
IJFMR250664112 Volume 7, Issue 6, November-December 2025 27
The vacuum phase depends only on radius: \theta = \theta(r). The kinetic invariant becomes:
X = -(1 - 2Gm(r)/r) \theta'(r)².
From the k-essence stress-energy tensor:
T_{\mu\nu} = 2 L_X \partial_\mu\theta \partial_\nu\theta - g_{\mu\nu} L_\theta
3. Stress-Energy Components
Define:
L_\theta = -\Lambda_v + (\rho_0/2)X - (\eta/(3 a_0²)) X^{3/2},
L_X = \partialL_\theta/\partialX = \rho_0/2 - (\eta/(2a_0²)) X^{1/2}.
Energy density and pressures:
\rho = L_\theta,
p_t = \rho,
p_r = 2 L_X X - L_\theta.
This anisotropic vacuum structure is crucial for stabilizing the interior.
4. Vacuum Saturation Mechanism
The scalar field equation \nabla_\mu(L_X \partial^\mu\theta)=0 is satisfied in the core when:
L_X(X_0) = 0.
Setting L_X=0 gives:
X_0^{1/2} = (\rho_0 a_0²)/\eta.
Thus, the vacuum phase reaches a 'saturation' point X_0, limiting further compression. The core energy
density becomes finite:
\rho_core = -\Lambda_v + (\rho_0³ a_0⁴)/(6 \eta²).
5. Core Geometry
With \rho = \rho_core = constant, the Einstein equation gives a de Sitter–like interior:
m(r) = (4\pi/3)\rho_core r³,
1 - 2Gm(r)/r = 1 - (8\piG/3)\rho_core r².
Thus, the interior metric is:
ds²_core \approx -[1 - (\Lambda_eff r²)/3] dt² + dr²/[1 - (\Lambda_eff r²)/3] + r² d\Omega²,
with \Lambda_eff = 8\piG \rho_core.
There is no singularity; curvature remains finite.
6. Matching to Exterior Geometry
For r > r_c (core radius), X << X_0 and nonlinear effects vanish. DVFT reduces to GR:
ds² \approx Schwarzschild metric.
Matching conditions ensure:
g_{tt}(core) = g_{tt}(ext),
g_{rr}(core) = g_{rr}(ext).
Thus, DVFT describes a black hole with a GR exterior and a finite-density vacuum core interior.
7. Physical Interpretation (Non-Mathematical)
• GR predicts infinite collapse. DVFT prevents this by saturating the vacuum phase.
• The black hole interior becomes a finite-size 'quantum core.'
• As mass falls in, both the horizon and the core radius increase.
• No singularity exists. Space cannot compress indefinitely.
• The final object is a quantum vacuum condensate, not a point of infinite density.
8. Final Fate of a Black Hole in DVFT
International Journal for Multidisciplinary Research (IJFMR)
E-ISSN: 2582-2160 ● Website: www.ijfmr.com ● Email: editor@ijfmr.com
IJFMR250664112 Volume 7, Issue 6, November-December 2025 28
Depending on parameters (\rho_0, \eta, a_0):
1. Stable quantum object: evaporation slows, horizon stalls, core remains.
2. Horizon shrinks until it meets the core, leaving a compact vacuum star.
3. Complete evaporation: horizon vanishes; core dissolves smoothly.
In all cases, there is no singularity and no information loss.
Conclusion
DVFT gives the first consistent picture of a black hole interior using a single phase field. It provides:
• GR-like exterior geometry,
• A finite-density quantum core replacing the singularity,
• A mechanism for black hole growth and evolution,
• A plausible resolution of the information paradox.
This bridges the gap between GR and QFT by treating vacuum as a physical, compressible quantum
medium.

\end{document}
