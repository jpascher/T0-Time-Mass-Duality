\documentclass[12pt,a4paper]{article}

% Packages
\usepackage[utf8]{inputenc}
\usepackage[T1]{fontenc}
\usepackage{amsmath}
\usepackage{amssymb}
\usepackage{physics}
\usepackage{graphicx}
\usepackage{hyperref}
\usepackage[margin=2.5cm]{geometry}

% Physics notation
\renewcommand{\varphi}{\phi}

\title{Kapitel 02
}
\author{DVFT - Dynamic Vacuum Field Theory}
\date{\today}

\begin{document}

\section{Kapitel 02
}


WHY VACUUM IS A DYNAMIC FIELD
A core postulate of DVFT is the origin of the vacuum's dynamism: Why does the phase \theta evolve as \theta(t) =
\mut in the unperturbed state, rather than remaining static? This chapter demonstrates that the dynamic nature
emerges naturally from the vacuum's symmetry structure, potential, and adherence to fundamental
physical principles. No external trigger is required; the dynamism is an intrinsic property of the vacuum
field.
1. Introduction
The DVFT framework models spacetime as arising from a complex scalar vacuum field Φ(x) = \rho(x)
e^{i\theta(x)}. The phase \theta evolves with an intrinsic frequency \mu, leading to curvature through its gradients.
International Journal for Multidisciplinary Research (IJFMR)
E-ISSN: 2582-2160 ● Website: www.ijfmr.com ● Email: editor@ijfmr.com
IJFMR250664112 Volume 7, Issue 6, November-December 2025 4
This raises the query: What causes this evolution? The answer lies in established physics of symmetry
breaking, wave equations, vacuum stability and Lorentz invariance without invoking metaphysics.
2. The Vacuum Field Structure
In DVFT, the vacuum is modeled as a complex scalar field:
Φ(x) = \rho(x) e^{i\theta(x)}
with two degrees of freedom:
• \rho(x): Amplitude, related to energy density.
• \theta(x): Phase, related to timing and coherence.
In the ground state, \theta evolves linearly in proper time t:
\theta(t) = \mut
yielding:
Φ(t) = \rho₀ e^{-i\mut}
Here, \mu is the intrinsic frequency, determined by the vacuum's potential and symmetry. This evolution is
the lowest-energy configuration, not an arbitrary choice.
3. Symmetry Breaking as the Prime Mover
The vacuum potential is given by:
V(\rho) = \lambda (\rho² − \rho₀²)²
which exhibits a minimum at \rho = \rho₀ and U(1) symmetry in the complex plane (Φ → Φ e^{i\alpha}). At this
minimum, the potential has no preferred phase, leaving \theta free. The ground state thus selects a spontaneous
breaking of the U(1) symmetry, with \theta evolving as:
\theta(t) = \mu t
where \mu arises from the curvature of V at the minimum (\mu² \approx \lambda \rho₀², analogous to the Higgs mass). This
evolution minimizes the action and stabilizes the vacuum, without external input.
4. Oscillation as an Unavoidable Consequence
Fields governed by wave equations inherently support oscillations. The general equation for \theta in a stiff
medium is:
▫𝜃 +
\partial𝑉eff
\partial𝜃
= 0,
where V_eff includes nonlinear terms. For small displacements, this reduces to harmonic motion:
𝜃(𝑡) = 𝜃0 + 𝐴sin(𝜔𝑡 + \Phi).
Phase fields behave like springs: Displacements induce restoring forces, leading to rebound and
oscillation. A static vacuum (constant \theta) would require infinite fine-tuning, violating stability.
5. The True Pre-Mover is Vacuum Phase Stiffness
The pre-mover of the dynamism is the vacuum's stiffness, quantified by:
𝐿𝑋 =
\rho0
2
−
𝜂
2𝑎0
2 𝑋
1/2
,
where \eta and a_0 are parameters derived from the nonlinear response. This acts as an effective spring
constant. Perturbations (e.g., from matter) compress \theta, triggering nonlinear resistance, overshoot, and
oscillation. No initial cause is needed; stiffness ensures dynamic response to any deviation from
equilibrium.
6. Why the Entire Universe Pulsates
The vacuum's universality implies that its dynamism occurs across all scales. Cosmic-scale oscillations
arise from:
International Journal for Multidisciplinary Research (IJFMR)
E-ISSN: 2582-2160 ● Website: www.ijfmr.com ● Email: editor@ijfmr.com
IJFMR250664112 Volume 7, Issue 6, November-December 2025 5
• Matter-induced convergence of \theta.
• Compression of \theta gradients.
• Nonlinear vacuum resistance.
• Rebound leading to sustained dynamism.
This process requires no fine-tuning, emerging from the field's intrinsic properties.
7. Dynamic vacuum field Preserves Lorentz Invariance
A static vacuum would select a preferred rest frame, violating special relativity. However, with \theta(\tau) = \mu \tau
(proper time), the form:
Φ(𝜏) = \rho0 𝑒
𝑖𝜇𝜏
remains invariant under Lorentz transformations. Each inertial observer measures the same vacuum state
in their local frame, as \mu scales with time dilation. Thus, dynamism is essential for relativistic consistency.
8. Dynamic vacuum field Prevents Singularities
DVFT imposes a fundamental bound on the vacuum phase gradient:
|\partial\theta| \leq \theta_max
This prevents curvature from diverging and eliminates singularities. A static vacuum cannot produce this
stabilizing effect. But a vacuum with intrinsic oscillation has built-in restoring forces, similar to a vibrating
string or superfluid. Dynamic vacuum field creates vacuum 'stiffness' that resists infinite compression.
Thus, Dynamic vacuum field guarantees finite curvature everywhere. This is one of the important
advantage of the DVFT to avoid singularities.
9. Dynamic vacuum field from the Big Bang Vacuum Phase Transition
In DVFT cosmology, the early universe began with:
\rho \approx 0, \theta undefined
This was an unstable vacuum state. During the Big Bang, the vacuum transitioned into its stable state:
Φ = \rho₀ e^{i\mut}
The moment when \rho rose from 0 to \rho₀ and \theta gained coherence is the Big Bang. No external trigger was
required. The vacuum simply settled into its natural dynamic vacuum field ground state, just like the Higgs
field acquires a vacuum expectation value.
10. Dynamic vacuum field as an Intrinsic Vacuum Property
Dynamic vacuum field is not something that starts—it’s something that is intrinsic property of spacetime.
Similar intrinsic properties exist in physics:
• Electrons have intrinsic spin
• The Higgs field has a fixed amplitude
• Superfluids have inherent phase coherence
• Quantum fields have zero-point fluctuations
For DVFT, dynamic vacuum field is an intrinsic property of Φ, not the result of an external force or prime
mover.
11. Unified Answer
The vacuum pulsates because:
1. Vacuum is a physical medium with phase and stiffness.
2. Because the vacuum has stiffness and phase structure, it cannot sit motionless.
3. Symmetry-breaking potentials must lead to vacuum phase freedom.
4. Phase freedom must lead to time evolution (Dynamic vacuum field) in the lowest-energy state.
5. Phase fields obey wave equations.
International Journal for Multidisciplinary Research (IJFMR)
E-ISSN: 2582-2160 ● Website: www.ijfmr.com ● Email: editor@ijfmr.com
IJFMR250664112 Volume 7, Issue 6, November-December 2025 6
6. Wave equations produce oscillations.
7. Vacuum stability requires dynamic behavior.
8. Lorentz invariance requires time-dependent phase.
9. The Big Bang naturally initiated phase coherence.
There is no need for an external trigger. Dynamic vacuum field is the natural, unavoidable behavior of the
vacuum field that underlies spacetime.
Conclusion
DVFT does not require a metaphysical prime mover. The Dynamic vacuum field emerges from the internal
structure and symmetries of the field Φ. This Dynamic vacuum field preserves relativity, prevents
singularities, and drives cosmic evolution. Dynamic vacuum field is not triggered; it is built into the fabric
of reality itself.

\end{document}
