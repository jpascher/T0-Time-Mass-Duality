\documentclass[12pt,a4paper]{article}

% Standardized preamble - 043_ResolvingTheConstantsAlfa_De.pdf
% Minimale T0 Standalone Preamble - A4 Format - 25 Zeilen
\RequirePackage{fontspec}
\RequirePackage{unicode-math}
\usepackage[ngerman]{babel}
\usepackage{microtype}
\setmainfont{Inter}
\setmonofont{JetBrains Mono}
\setmathfont{Libertinus Math}
\usepackage{amsmath,amsfonts,amsthm}
\usepackage{mathtools}
\usepackage{graphicx}
\usepackage{xcolor}
\definecolor{t0blue}{RGB}{0,102,204}
\definecolor{t0green}{RGB}{34,139,34}
\definecolor{t0red}{RGB}{204,0,0}
\usepackage{geometry}
\geometry{a4paper,margin=2.5cm}
\usepackage[most]{tcolorbox}
\newtcolorbox{keyresult}[1][]{colback=yellow!5,colframe=t0blue!80,fonttitle=\bfseries,title={#1},breakable}
\newtcolorbox{important}[1][]{colback=red!5,colframe=t0red!80,fonttitle=\bfseries,title={#1},breakable}
\newcommand{\Tfield}{\ensuremath{\mathcal{T}}}
\usepackage{hyperref}
\hypersetup{colorlinks=true,linkcolor=t0blue}


\title{Mathematischer Beweis: \\
		Die Feinstrukturkonstante $\alpha = 1$ \\
		in natürlichen Einheiten}
\author{}
\date{Januar 2025}

\begin{document}

	\maketitle
	
	\begin{abstract}
		Diese Arbeit liefert einen rigorosen mathematischen Beweis, dass die Feinstrukturkonstante $\alpha$ in natürlichen Einheitensystemen gleich Eins ($\alpha = 1$) ist. Durch systematische Analyse der zwei äquivalenten Darstellungen von $\alpha$ demonstrieren wir, dass die elektromagnetische Dualität zwischen $\varepsilon_0$ und $\mu_0$, verbunden durch die fundamentale Maxwell-Beziehung $c^2 = 1/(\varepsilon_0\mu_0)$, natürlich zu $\alpha = 1$ führt, wenn angemessene Einheitennormierungen angewandt werden. Dieser Beweis etabliert, dass $\alpha = 1/137$ in SI-Einheiten rein eine Folge unserer historischen Einheitenwahlen ist, nicht ein fundamentales Mysterium der Natur.
	\end{abstract}
	
	\tableofcontents
	
	\section{Einleitung und Motivation}
	
	Die Feinstrukturkonstante $\alpha \approx 1/137$ wurde als eines der größten Mysterien der Physik bezeichnet und inspirierte berühmte Zitate von Feynman, Pauli und anderen. Diese Mystifizierung entspringt jedoch der Betrachtung von $\alpha$ nur innerhalb des SI-Einheitensystems. Diese Arbeit beweist mathematisch, dass $\alpha = 1$ in angemessen gewählten natürlichen Einheiten, wodurch offenbart wird, dass das \textit{Mysterium} von $1/137$ lediglich eine Folge unseres konventionellen Einheitensystems ist.
	
	\section{Fundamentale Prämisse}
	
	\begin{definition}[Zwei äquivalente Formen von $\alpha$]
		Die Feinstrukturkonstante kann in zwei mathematisch äquivalenten Formen ausgedrückt werden:
		\begin{align}
			\text{Form 1:} \quad \alphaem &= \frac{e^2}{4\pi\varepsilon_0\hbar c} \label{eq:alpha_form1}\\
			\text{Form 2:} \quad \alphaem &= \frac{e^2 \mu_0 c}{4\pi \hbar} \label{eq:alpha_form2}
		\end{align}
	\end{definition}
	
	Diese Formen sind äquivalent durch die Maxwell-Beziehung $c^2 = 1/(\varepsilon_0\mu_0)$.
	
	\section{Die Dualitäts-Analyse}
	
	\subsection{Extraktion gemeinsamer Elemente}
	
	\begin{proof_step}[Identifikation gemeinsamer Terme]
		Beide Formen \eqref{eq:alpha_form1} und \eqref{eq:alpha_form2} enthalten identische Terme:
		\begin{itemize}
			\item $e^2$ - Quadrat der Elementarladung
			\item $4\pi$ - geometrischer Faktor
			\item $\hbar$ - reduzierte Planck-Konstante
		\end{itemize}
	\end{proof_step}
	
	\begin{proof_step}[Isolierung differenzieller Terme]
		Nach Ausklammern gemeinsamer Elemente ist der wesentliche Unterschied zwischen den beiden Formen:
		\begin{align}
			\text{Form 1:} \quad \alphaem &\propto \frac{1}{\varepsilon_0 c} \label{eq:diff1}\\
			\text{Form 2:} \quad \alphaem &\propto \mu_0 c \label{eq:diff2}
		\end{align}
	\end{proof_step}
	
	\subsection{Die elektromagnetische Dualität}
	
	\begin{theorem}[Elektromagnetische Dualitäts-Beziehung]
		Damit die zwei Formen äquivalent sind, müssen wir haben:
		\begin{equation}
			\frac{1}{\varepsilon_0 c} = \mu_0 c \label{eq:dualitaet}
		\end{equation}
	\end{theorem}
	
	\begin{proof}
		Umformen von Gleichung \eqref{eq:dualitaet}:
		\begin{align}
			\frac{1}{\varepsilon_0 c} &= \mu_0 c\\
			1 &= \varepsilon_0 c \cdot \mu_0 c\\
			1 &= \varepsilon_0 \mu_0 c^2\\
			c^2 &= \frac{1}{\varepsilon_0 \mu_0}
		\end{align}
		Dies ist präzise Maxwells fundamentale Beziehung, die elektromagnetische Konstanten mit der Lichtgeschwindigkeit verbindet.
	\end{proof}
	
	\section{Die Schlüsselerkenntnis: Gegensätzliche Potenzen von c}
	
	\begin{lemma}[Vorzeichendualität von c]
		Die Lichtgeschwindigkeit $c$ erscheint mit gegensätzlichen \textit{Vorzeichen} (Potenzen) in den zwei Formen:
		\begin{align}
			\text{Form 1:} \quad c^{-1} \quad &\text{($c$ im Nenner)}\\
			\text{Form 2:} \quad c^{+1} \quad &\text{($c$ im Zähler)}
		\end{align}
	\end{lemma}
	
	Diese Dualität spiegelt die komplementäre Natur elektrischer ($\varepsilon_0$) und magnetischer ($\mu_0$) Aspekte des elektromagnetischen Feldes wider.
	
	\section{Konstruktion natürlicher Einheiten}
	
	\subsection{Die natürliche Einheitenwahl}
	
	\begin{definition}[Natürliches Einheitensystem für $\alpha = 1$]
		Wir definieren ein natürliches Einheitensystem, wo:
		\begin{enumerate}
			\item $\hbar_{\text{nat}} = 1$ (quantenmechanische Skala)
			\item $c_{\text{nat}} = 1$ (relativistische Skala)  
			\item Die elektromagnetischen Konstanten sind so normiert, dass $\alphaem = 1$
		\end{enumerate}
	\end{definition}
	
	\subsection{Bestimmung natürlicher elektromagnetischer Konstanten}
	
	\begin{theorem}[Natürliche Einheiten elektromagnetische Konstanten]
		Im natürlichen Einheitensystem, wo $\alpha = 1$, $\hbar = 1$ und $c = 1$, werden die elektromagnetischen Konstanten zu:
		\begin{align}
			e_{\text{nat}}^2 &= 4\pi \label{eq:e_nat}\\
			\varepsilon_{0,\text{nat}} &= 1 \label{eq:eps_nat}\\
			\mu_{0,\text{nat}} &= 1 \label{eq:mu_nat}
		\end{align}
	\end{theorem}
	
	\begin{proof}
		Aus Form 1 mit $\alphaem = 1$, $\hbar = 1$, $c = 1$:
		\begin{align}
			1 &= \frac{e^2}{4\pi\varepsilon_0 \cdot 1 \cdot 1}\\
			4\pi\varepsilon_0 &= e^2
		\end{align}
		
		Setzen von $\varepsilon_0 = 1$ (natürliche Wahl), erhalten wir $e^2 = 4\pi$.
		
		Aus der Maxwell-Beziehung $c^2 = 1/(\varepsilon_0\mu_0)$ mit $c = 1$:
		\begin{align}
			1 &= \frac{1}{\varepsilon_0\mu_0}\\
			\varepsilon_0\mu_0 &= 1
		\end{align}
		
		Mit $\varepsilon_0 = 1$ erhalten wir $\mu_0 = 1$.
	\end{proof}
	
	\section{Verifikation von $\alpha = 1$}
	
	\subsection{Verifikation mit Form 1}
	
	\begin{proof_step}[Form 1 Verifikation]
		\begin{align}
			\alphaem &= \frac{e^2}{4\pi\varepsilon_0\hbar c}\\
			&= \frac{4\pi}{4\pi \cdot 1 \cdot 1 \cdot 1}\\
			&= \frac{4\pi}{4\pi}\\
			&= 1 \quad \checkmark
		\end{align}
	\end{proof_step}
	
	\subsection{Verifikation mit Form 2}
	
	\begin{proof_step}[Form 2 Verifikation]
		\begin{align}
			\alphaem &= \frac{e^2 \mu_0 c}{4\pi \hbar}\\
			&= \frac{4\pi \cdot 1 \cdot 1}{4\pi \cdot 1}\\
			&= \frac{4\pi}{4\pi}\\
			&= 1 \quad \checkmark
		\end{align}
	\end{proof_step}
	
	\section{Die Dualitäts-Verifikation}
	
	\begin{theorem}[Elektromagnetische Dualität in natürlichen Einheiten]
		In natürlichen Einheiten ist die elektromagnetische Dualität perfekt erfüllt:
		\begin{equation}
			\frac{1}{\varepsilon_{0,\text{nat}} \cdot c_{\text{nat}}} = \mu_{0,\text{nat}} \cdot c_{\text{nat}}
		\end{equation}
	\end{theorem}
	
	\begin{proof}
		\begin{align}
			\text{LHS:} \quad \frac{1}{\varepsilon_{0,\text{nat}} \cdot c_{\text{nat}}} &= \frac{1}{1 \cdot 1} = 1\\
			\text{RHS:} \quad \mu_{0,\text{nat}} \cdot c_{\text{nat}} &= 1 \cdot 1 = 1\\
			\text{Daher:} \quad \text{LHS} &= \text{RHS} \quad \checkmark
		\end{align}
	\end{proof}
	
	\section{Physikalische Interpretation}
	
	\subsection{Die Natürlichkeit von $\alpha = 1$}
	
	\begin{tcolorbox}[colback=green!5!white,colframe=green!75!black,title=Wichtige physikalische Erkenntnis]
		In natürlichen Einheiten repräsentiert $\alpha = 1$ die perfekte Balance zwischen:
		\begin{itemize}
			\item \textbf{Elektrische Feldkopplung} (durch $\varepsilon_0$ mit $c^{-1}$)
			\item \textbf{Magnetische Feldkopplung} (durch $\mu_0$ mit $c^{+1}$)
			\item \textbf{Quantenmechanische Skala} (durch $\hbar$)
			\item \textbf{Relativistische Skala} (durch $c$)
		\end{itemize}
		
		Die elektromagnetische Dualität $\frac{1}{\varepsilon_0 c} = \mu_0 c$ gewährleistet diese perfekte Balance.
	\end{tcolorbox}
	
	\subsection{Auflösung des \textit{$1/137$-Mysteriums}}
	
	Der berühmte Wert $\alpha \approx 1/137$ in SI-Einheiten entsteht ausschließlich aus unseren historischen Wahlen von:
	\begin{itemize}
		\item Dem Meter (Längenskala)
		\item Der Sekunde (Zeitskala)  
		\item Dem Kilogramm (Massenskala)
		\item Dem Ampere (Stromskala)
	\end{itemize}
	
	Diese Wahlen zwingen elektromagnetische Konstanten zu \textit{unnatürlichen} Werten und lassen $\alpha$ geheimnisvoll klein erscheinen.
	
	\subsubsection{Transformation von natürlichen Einheiten zu SI-Einheiten}
	
	Um zu verstehen, wie wir zum SI-Wert $\alpha_{\text{SI}} = 1/137$ gelangen, müssen wir von unserem natürlichen Einheitensystem zurück zu SI-Einheiten transformieren. Die Transformation beinhaltet Skalierungsfaktoren für jede fundamentale Konstante:
	
	\begin{align}
		\hbar_{\text{SI}} &= \hbar_{\text{nat}} \times S_{\hbar} = 1 \times (1.055 \times 10^{-34} \text{ J·s})\\
		c_{\text{SI}} &= c_{\text{nat}} \times S_c = 1 \times (2.998 \times 10^8 \text{ m/s})\\
		\varepsilon_{0,\text{SI}} &= \varepsilon_{0,\text{nat}} \times S_{\varepsilon} = 1 \times (8.854 \times 10^{-12} \text{ F/m})\\
		e_{\text{SI}} &= e_{\text{nat}} \times S_e = \sqrt{4\pi} \times S_e
	\end{align}
	
	Die Feinstrukturkonstante in SI-Einheiten wird zu:
	\begin{align}
		\alpha_{\text{SI}} &= \frac{e_{\text{SI}}^2}{4\pi\varepsilon_{0,\text{SI}}\hbar_{\text{SI}} c_{\text{SI}}}\\
		&= \frac{(\sqrt{4\pi} \times S_e)^2}{4\pi \times (S_{\varepsilon}) \times (S_{\hbar}) \times (S_c)}\\
		&= \frac{4\pi \times S_e^2}{4\pi \times S_{\varepsilon} \times S_{\hbar} \times S_c}\\
		&= \frac{S_e^2}{S_{\varepsilon} \times S_{\hbar} \times S_c}
	\end{align}
	
	Die historischen SI-Einheitendefinitionen schufen Skalierungsfaktoren, sodass dieses Verhältnis ungefähr $1/137$ entspricht. Mit anderen Worten:
	$\frac{S_e^2}{S_{\varepsilon} \times S_{\hbar} \times S_c} \approx \frac{1}{137}$
	
	Dies demonstriert, dass der \textit{geheimnisvolle} Wert $1/137$ rein eine Folge der willkürlichen Skalierungsfaktoren ist, die bei der Definition der SI-Basiseinheiten gewählt wurden, nicht eine fundamentale Eigenschaft elektromagnetischer Wechselwirkungen selbst. Im natürlichen Einheitensystem, wo diese Skalierungsfaktoren Eins sind, ergibt sich $\alpha = 1$ als der fundamentale Wert.
	
	\section{Implikationen und Schlussfolgerungen}
	
	\subsection{Philosophische Implikationen}
	
	Dieser Beweis demonstriert, dass:
	
	\begin{enumerate}
		\item \textbf{$\alpha = 1/137$ ist nicht fundamental} - es ist eine Folge von Einheitenwahlen
		\item \textbf{$\alpha = 1$ ist natürlich} - es reflektiert die inhärente elektromagnetische Dualität
		\item \textbf{Das \textit{Mysterium} löst sich auf} - es gibt nichts Besonderes an $1/137$
		\item \textbf{Die Natur ist einfacher} - fundamentale Beziehungen haben natürliche Werte
	\end{enumerate}
	
	\subsection{Konsistenzprüfung}
	
	\begin{tcolorbox}[colback=blue!5!white,colframe=blue!75!black,title=Interne Konsistenzverifikation]
		Unser natürliches Einheitensystem erfüllt alle fundamentalen Beziehungen:
		\begin{align}
			c^2 &= \frac{1}{\varepsilon_0\mu_0} = \frac{1}{1 \cdot 1} = 1 = 1^2 \quad \checkmark\\
			\alphaem &= \frac{e^2}{4\pi\varepsilon_0\hbar c} = \frac{4\pi}{4\pi \cdot 1 \cdot 1 \cdot 1} = 1 \quad \checkmark\\
			\alphaem &= \frac{e^2\mu_0 c}{4\pi\hbar} = \frac{4\pi \cdot 1 \cdot 1}{4\pi \cdot 1} = 1 \quad \checkmark
		\end{align}
	\end{tcolorbox}
	
	\section{Auflösung des Konstanten-Paradoxons}
	
	\subsection{Das fundamentale Missverständnis}
	
	Der tiefgreifendste Einwand gegen unseren Beweis nimmt oft die Form an: \textit{Wie kann eine \textbf{Konstante} verschiedene Werte haben?} Dieses scheinbare Paradoxon liegt im Herzen, warum die Feinstrukturkonstante über ein Jahrhundert lang mystifiziert wurde.
	
	\subsubsection{Die Problemstellung}
	
	Der scheinbare Widerspruch ist:
	\begin{itemize}
		\item $\alpha = 1/137$ (in SI-Einheiten)
		\item $\alpha = 1$ (in natürlichen Einheiten)
		\item $\alpha = \sqrt{2}$ (in Gauß-Einheiten)
	\end{itemize}
	
	Wie kann dieselbe \textit{Konstante} drei verschiedene Werte haben?
	
	\subsubsection{Die Auflösung}
	
	Die Auflösung offenbart ein fundamentales Missverständnis darüber, was \textit{Konstante} in der Physik bedeutet.
	
	\textbf{Was wirklich konstant ist, ist nicht die Zahl, sondern die physikalische Beziehung.}
	
	\subsection{Die perfekte Analogie: Siedepunkt des Wassers}
	
	Betrachten Sie den Siedepunkt von Wasser:
	\begin{itemize}
		\item $100°\text{C}$ (Celsius-Skala)
		\item $212°\text{F}$ (Fahrenheit-Skala)
		\item $373\text{ K}$ (Kelvin-Skala)
	\end{itemize}
	
	\textbf{Frage:} Bei welcher Temperatur siedet Wasser \textit{wirklich}?
	
	\textbf{Antwort:} Bei derselben physikalischen Temperatur in allen Fällen! Nur die Zahlen unterscheiden sich aufgrund verschiedener Temperaturskalen.
	
	\subsection{Dasselbe Prinzip gilt für $\alpha$}
	
	Genau wie bei Temperaturskalen:
	\begin{itemize}
		\item $\alpha = 1/137$ (SI-Einheitenskala)
		\item $\alpha = 1$ (natürliche Einheitenskala)
		\item $\alpha = \sqrt{2}$ (Gauß-Einheitenskala)
	\end{itemize}
	
	\textbf{Die elektromagnetische Kopplungsstärke ist identisch} -- nur die Messungsskalen unterscheiden sich.
	
	\subsection{Die Schlüsselerkenntnis}
	
	\begin{tcolorbox}[colback=yellow!5!white,colframe=orange!75!black,title=Fundamentales Prinzip]
		\textit{\textbf{KONSTANT}} bedeutet \textbf{NICHT} \textit{dieselbe Zahl}!
		
		\textit{\textbf{KONSTANT}} bedeutet \textit{dieselbe physikalische Größe}!
	\end{tcolorbox}
	
	\textbf{Beispiele dieses Prinzips:}
	\begin{itemize}
		\item $1\text{ Meter} = 100\text{ cm} = 3.28\text{ Fuß}$ $\rightarrow$ Die \textbf{Länge} ist konstant
		\item $1\text{ kg} = 1000\text{ g} = 2.2\text{ lbs}$ $\rightarrow$ Die \textbf{Masse} ist konstant
		\item $\alpha = 1/137 = 1 = \sqrt{2}$ $\rightarrow$ Die \textbf{Kopplungsstärke} ist konstant
	\end{itemize}
	
	\subsection{Physikalische Verifikation}
	
	Wir können verifizieren, dass diese dieselbe physikalische Konstante repräsentieren, indem wir bestätigen, dass alle Einheitensysteme identische messbare Vorhersagen ergeben:
	
	\begin{theorem}[Experimentelle Invarianz]
		Alle Einheitensysteme produzieren identische messbare Vorhersagen:
		\begin{itemize}
			\item \textbf{Wasserstoffspektrum:} Dieselben Frequenzen in allen Systemen $\checkmark$
			\item \textbf{Elektronstreuung:} Dieselben Wirkungsquerschnitte in allen Systemen $\checkmark$
			\item \textbf{Lamb-Verschiebung:} Dieselben Energieverschiebungen in allen Systemen $\checkmark$
		\end{itemize}
	\end{theorem}
	
	\subsection{Die tiefere Wahrheit}
	
	\begin{tcolorbox}[colback=green!5!white,colframe=green!75!black,title=Naturs wahre Sprache]
		\textbf{Die Natur \textit{kennt} keine Zahlen!}
		
		\textbf{Die Natur kennt nur Verhältnisse und Beziehungen!}
	\end{tcolorbox}
	
	Die Feinstrukturkonstante $\alpha$ ist nicht die geheimnisvolle Zahl \textit{$1/137$} -- $\alpha$ ist das \textbf{Verhältnis} zwischen elektromagnetischen und quantenmechanischen Effekten.
	
	Dieses Verhältnis ist absolut konstant im gesamten Universum, aber der numerische Wert hängt vollständig von unserer willkürlichen Wahl der Einheitendefinitionen ab.
	
	\subsection{Das sprachliche Problem}
	
	Viel Verwirrung entspringt unpräziser Sprache. Wir sagen fälschlicherweise:
	\begin{itemize}
		\item[\textcolor{red}{$\times$}] \textit{\textbf{DIE} Feinstrukturkonstante ist $1/137$}
	\end{itemize}
	
	Die korrekten Aussagen wären:
	\begin{itemize}
		\item[\textcolor{green}{$\checkmark$}] \textit{Die Feinstrukturkonstante hat den Wert $1/137$ \textbf{in SI-Einheiten}}
		\item[\textcolor{green}{$\checkmark$}] \textit{Die Feinstrukturkonstante hat den Wert $1$ \textbf{in natürlichen Einheiten}}
	\end{itemize}
	
	\subsection{Auflösung des jahrhundertealten Mysteriums}
	
	Diese Analyse offenbart, dass das \textit{Mysterium von $1/137$} kein physikalisches Rätsel ist, sondern ein \textbf{sprachliches und konzeptuelles Missverständnis}. Die Mystifizierung entstand aus:
	
	\begin{enumerate}
		\item Verwechslung des numerischen Werts mit der physikalischen Größe
		\item Behandlung des SI-Einheitensystems als fundamental anstatt konventionell
		\item Vergessen, dass alle Einheitensysteme menschliche Konstrukte sind
		\item Suche nach tiefer Bedeutung in dem, was im Wesentlichen Umwandlungsfaktoren sind
	\end{enumerate}
	
	Sobald wir erkennen, dass $\alpha = 1$ die natürliche Stärke elektromagnetischer Wechselwirkungen repräsentiert, löst sich das \textit{Mysterium} vollständig auf. Die elektromagnetische Kraft hat Einheitsstärke im Einheitensystem, das die fundamentale Struktur von Quantenmechanik und Relativität respektiert -- genau wie man es von einer wahrhaft fundamentalen Wechselwirkung erwarten würde.
	
	\subsection{Abschließende Perspektive}
	
	Die Feinstrukturkonstante lehrt uns eine tiefgreifende Lektion über die Natur physikalischer Gesetze: \textbf{die fundamentalen Beziehungen des Universums sind elegant und einfach, wenn sie in ihrer natürlichen Sprache ausgedrückt werden}. Die scheinbare Komplexität und das Mysterium von \textit{$1/137$} ist lediglich ein Artefakt unserer historischen Wahl, elektromagnetische Phänomene mit Einheiten zu messen, die ursprünglich für mechanische Größen definiert wurden.
	
	Indem wir $\alpha = 1$ als den natürlichen Wert erkennen, erblicken wir die inhärente Einfachheit und Schönheit, die der elektromagnetischen Struktur der Realität zugrunde liegt.
	
	\section{Anerkennung}
	
	Diese Arbeit wurde durch die Erkenntnis inspiriert, dass fundamentale physikalische Konstanten keine geheimnisvollen Zahlen sein sollten, sondern die zugrundeliegende mathematische Struktur der Natur widerspiegeln sollten. Die elektromagnetische Dualität, die durch die Analyse der zwei Formen von $\alpha$ offenbart wird, liefert die Schlüsselerkenntnis, die das langanhaltende Rätsel der Feinstrukturkonstante auflöst.
	
	\begin{thebibliography}{9}
		\bibitem{Jackson1999} Jackson, J. D. (1999). \textit{Klassische Elektrodynamik} (3. Aufl.). John Wiley \& Sons.
		
		\bibitem{Feynman1985} Feynman, R. P. (1985). \textit{QED: Die seltsame Theorie des Lichts und der Materie}. Princeton University Press.
		
		\bibitem{Weinberg1995} Weinberg, S. (1995). \textit{The Quantum Theory of Fields, Volume 1: Foundations}. Cambridge University Press.
		
		\bibitem{Planck1906} Planck, M. (1906). Vorlesungen über die Theorie der Wärmestrahlung. Leipzig: J.A. Barth.
		
		\bibitem{Maxwell1865} Maxwell, J. C. (1865). A Dynamical Theory of the Electromagnetic Field. \textit{Philosophical Transactions of the Royal Society}, 155, 459-512.
		
		\bibitem{CODATA2018} CODATA Task Group on Fundamental Constants (2019). CODATA Recommended Values of the Fundamental Physical Constants: 2018. \textit{Rev. Mod. Phys.}, 91, 025009.
	\end{thebibliography}
	
\end{document}

