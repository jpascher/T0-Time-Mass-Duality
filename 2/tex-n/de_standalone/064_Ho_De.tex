\documentclass[12pt,a4paper]{article}

% Standardized preamble - 064_Ho_De.tex
% ==============================================================================
% T0 Theory: Shared GERMAN Preamble – Optimized for eBook/Book
% Version: 2.0 – Final 2026 (LuaLaTeX only) – DEUTSCH korrigiert
% Author: Johann Pascher
% Date: Januar 2026
% ==============================================================================
%
% WICHTIG: Compile EXCLUSIVELY with LuaLaTeX!
% In TeXstudio: Options → Configure TeXstudio → Build → Default Compiler → LuaLaTeX
%
% Required Fonts (install once):
% - Inter: https://fonts.google.com/specimen/Inter
% - JetBrains Mono: https://www.jetbrains.com/lp/mono/
% - Libertinus Math: https://github.com/libertinus-fonts/libertinus
% ==============================================================================

% === KAPITEL 1: GRUNDLEGENDE PAKETE (müssen ZUERST kommen) ===
\RequirePackage{fontspec}
\RequirePackage{unicode-math}

% === KAPITEL 2: SPRACHE (DEUTSCH mit voller Silbentrennung) ===
\usepackage[ngerman]{babel}
\usepackage{microtype}                    % WICHTIG für bessere Silbentrennung!

% Typographie-Einstellungen für besseren deutschen Umbruch
\frenchspacing                     % Korrekte deutsche Abstände nach Satzzeichen
\emergencystretch=3em              % Erlaubt mehr Dehnung bei schwierigen Zeilen
\tolerance=2500                    % Höhere Toleranz für Zeilenumbrüche
\hbadness=10000                    % Unterdrückt "underfull hbox" Warnungen
\hfuzz=2pt                         % Erlaubt minimalen Overfull
\pretolerance=150                  % Bessere Worttrennung

% Bessere Seitenumbrüche verhindern
\clubpenalty=10000           % Keine "Schusterjungen"
\widowpenalty=10000          % Keine "Hurenkinder"  
\displaywidowpenalty=10000   % Auch bei Formeln
\brokenpenalty=10000         % Keine getrennten Wörter über Seiten

% Explizite Trennungen für lange deutsche Wörter
\hyphenation{Fun-da-men-tal Frak-tal-Ge-o-me-trisch Fel-the-o-rie Me-tho-do-lo-gisch}
\hyphenation{Re-vi-si-o-nis-mus Quan-ti-sie-rung U-ni-fi-ka-ti-on Ef-fek-tiv}
\hyphenation{Re-nor-mier-bar-keit Sin-gu-la-ri-tä-ten Kon-zi-li-an-tis-mus}
\hyphenation{E-mer-genz Phä-no-me-no-lo-gisch Do-ku-men-ta-ti-on Ana-ly-se}
\hyphenation{Gra-vi-ta-ti-on Quan-ten-me-cha-nik Do-gma-tis-mus Kon-se-quent}
\hyphenation{Par-al-le-lis-mus Im-ple-men-tie-rung Per-tur-ba-ti-o-nen}
\hyphenation{Ge-o-me-trisch Ar-te-fakt In-ko-mpa-ti-bi-li-tät Kon-struk-tiv}
\hyphenation{Frak-tal Di-men-si-ons-los Un-ter-such-ung Be-schrei-bung}
\hyphenation{In-ter-pre-ta-ti-on Phe-no-me-no-lo-gisch Ma-the-ma-tisch}
\hyphenation{Phi-lo-so-phisch Le-gi-ti-ma-ti-on An-wen-dung Ab-lei-tung}
\hyphenation{Ver-ein-heit-li-chung An-na-hme Vor-stel-lung Er-war-tung}
\hyphenation{Sym-me-trie-ern-wei-te-rung Ge-samt-bild Her-aus-fo-rde-rung}
\hyphenation{Wech-sel-wir-kung Ma-te-ri-al An-satz Per-spek-ti-ve Vor-ge-hen}

% === KAPITEL 3: SCHRIFTEN (mit deutschen Ligaturen) ===
\setmainfont{Inter}[
Scale=1.02,
UprightFont=*-Regular,
BoldFont=*-Bold,
ItalicFont=*-Italic,
BoldItalicFont=*-BoldItalic,
Ligatures=TeX,           % WICHTIG für deutsche Typografie
Language=German          % Explizite Sprachunterstützung
]
\setsansfont{Inter}[
Scale=MatchLowercase,
Ligatures=TeX,
Language=German
]
\setmonofont{JetBrains Mono}[
Scale=0.95,
Language=German
]

% Math Font (simple & stable) – MUSS NACH der Sprachdefinition kommen
% WICHTIG: Libertinus Math für korrekte \underbrace-Darstellung!
\setmathfont{Libertinus Math}[Scale=1.0]

% === KAPITEL 4: MATHEMATIK-PAKETE (in STRENGER Reihenfolge!) ===
% WICHTIG: mathtools muss VOR unicode-math für manche Befehle!
\usepackage{mathtools}           % ZUERST mathtools!

% Dann der Rest
\usepackage{amsmath, amsfonts, amsthm}

% SIUNITX MUSS VOR physics geladen werden!
\usepackage{siunitx}
\sisetup{
	locale=DE,                    % DEUTSCHE Einstellungen für SI-Einheiten!
	group-separator={.},          % Tausendertrennzeichen Punkt
	output-decimal-marker={,},    % Dezimaltrennzeichen Komma
	per-mode=symbol,
	separate-uncertainty=true
}

% Eigene SI-Einheiten für Narrative/Bücher
\DeclareSIUnit\gigalightyear{Gly}
\DeclareSIUnit\mev{MeV}

% physics – MUSS NACH siunitx und mathtools geladen werden
\usepackage{physics}

% === KAPITEL 5: ERGÄNZUNGEN aus pdflatex-Best Practices ===
\usepackage{colortbl}        % Farbige Tabellen (ESSENTIELL!)
\usepackage{placeins}        % Float-Kontrolle: \FloatBarrier
\usepackage{subcaption}      % Unterabbildungen
\usepackage{xurl}            % Bessere URL-Umbrüche
% Hyphenation for URLs in bibliography
\def\UrlBreaks{\do\/\do-}

% === KAPITEL 6: SEITENGESTALTUNG =
\usepackage[paperwidth=8.25in, paperheight=11in, 
left=2.5cm, 
right=2.5cm, 
top=2.5cm, 
bottom=3.5cm,
bindingoffset=0.5cm]{geometry}
\setlength{\headheight}{15pt}
% Page Geometry – Buch-Optimierung
% =============================================================================
%\usepackage[paperwidth=8.25in, paperheight=11in,
%top=1.0in,
%bottom=1.2in,
%inner=1.0in,
%outer=0.75in,
%bindingoffset=0.75in,
%twoside]{geometry}
%\setlength{\headheight}{15pt}

% === KAPITEL 7: GRAFIKEN UND TABELLEN ===
\usepackage{graphicx}
\usepackage[table,xcdraw]{xcolor}
% T0 Markenfarben
\definecolor{gold}{RGB}{255,215,0}
\definecolor{blue}{rgb}{0,0,1}
\definecolor{boxgray}{RGB}{240,240,240}
\definecolor{deepblue}{RGB}{0,0,127}
\definecolor{deepgreen}{RGB}{0,127,0}
\definecolor{deepred}{RGB}{191,0,0}
\definecolor{t0blue}{RGB}{33,150,243}
\definecolor{t0green}{RGB}{76,175,80}
\definecolor{t0orange}{RGB}{255,152,0}
\definecolor{t0purple}{RGB}{156,39,176}
\definecolor{t0red}{RGB}{244,67,54}
\definecolor{t0yellow}{RGB}{255,204,0}
\usepackage{tikz}
\usetikzlibrary{arrows.meta,positioning,shapes.geometric,decorations.pathmorphing,patterns,shapes.arrows,intersections}
\usepackage{pgfplots}
\pgfplotsset{compat=1.18}
\usepackage{quantikz}
\usepackage[most]{tcolorbox}
\tcbuselibrary{breakable}

% === WICHTIG: Algorithm-Konflikt umgehen ===
% Option: algorithmic mit GROSSBUCHSTABEN
% Gemeinsame Box für Experimente
\newtcolorbox{experimentbox}[1][]{
	colback=green!5!white,
	colframe=t0green!80!black,
	fonttitle=\bfseries,
	title={{#1}},
	breakable
}

% Abstract-Fallback
\ifdefined\abstract\else
\newenvironment{abstract}{\section*{\abstractname}\itshape\small\par\bigskip}{\bigskip}
\fi

% === MAKROS SICHER NEU DEFINIEREN / ÜBERSCHREIBEN ===
% Definiere Makros OHNE doppelte Subskripte
\newcommand{\phipar}{\phi_{\mathrm{par}}}
%\newcommand{\xipar}{\xi_{\mathrm{par}}}
\newcommand{\Qphipar}{Q_{\phi_{\mathrm{par}}}}
\newcommand{\rphipar}{r_{\phi_{\mathrm{par}}}}
\newcommand{\logphipar}{\log_{\phi_{\mathrm{par}}}}
\newcommand{\CHSH}{\text{CHSH}}
\usepackage{booktabs}
\usepackage{array}
\usepackage{longtable}
\usepackage{float}
\usepackage{adjustbox}
\usepackage{rotating}
\usepackage{tabularx}
\usepackage{makecell}
\usepackage{multirow}

% === KAPITEL 8: DOKUMENTFORMATIERUNG ===
\usepackage{fancyhdr}
\renewcommand{\headrulewidth}{0.4pt}
\renewcommand{\footrulewidth}{0.4pt}
\usepackage{tocloft}

\usepackage{enumitem}
\setlist[itemize]{leftmargin=*, topsep=2pt, partopsep=0pt, parsep=2pt, itemsep=2pt}
\setlist[enumerate]{leftmargin=*, topsep=2pt, partopsep=0pt, parsep=2pt, itemsep=2pt}
\usepackage{setspace}
\usepackage{ragged2e}
\usepackage{multicol}

% === KAPITEL 9: CODE UND ALGORITHMEN ===
\usepackage{algorithm}
\usepackage{algorithmic}
\usepackage{listings}
\lstset{
	basicstyle=\ttfamily\footnotesize,
	breaklines=true,
	breakatwhitespace=true,
	columns=flexible,
	keepspaces=true,
	showstringspaces=false,
	frame=single,
	xleftmargin=0pt,
	xrightmargin=0pt,
	literate=              % Für deutsche Umlaute in Code-Listings
	{ä}{{\"a}}1 {ö}{{\"o}}1 {ü}{{\"u}}1 {ß}{{\ss}}1
	{Ä}{{\"A}}1 {Ö}{{\"O}}1 {Ü}{{\"U}}1
}
\usepackage{mdframed}

% === KAPITEL 10: ZUSÄTZLICHE PAKETE ===
\usepackage{pdflscape}
\usepackage{braket}
\usepackage{cancel}
\usepackage{caption}
\captionsetup{format=plain, labelfont=bf, justification=centering}
\usepackage{csquotes}
\usepackage{gensymb}
\usepackage{textcomp}
\usepackage{textgreek}
\usepackage{upgreek}
\usepackage{url}
\usepackage{slashed}
\usepackage{bm}

% === KAPITEL 11: HYPERREF (muss als VORLETZTES Paket kommen!) ===
\usepackage{hyperref}
\hypersetup{
	colorlinks=true,
	linkcolor=black,
	citecolor=black,
	urlcolor=black,
	breaklinks=true,           % WICHTIG für deutsche Umlaute in URLs!
	bookmarksnumbered=true,
	unicode=true,
	pdfencoding=auto,
	pdflang=de,                % PDF-Sprache auf Deutsch setzen
	pdfsubject={T0 Theorie - Fundamental Fractal-Geometric Field Theory}
}

% === KAPITEL 12: BOOKMARK (muss NACH hyperref kommen!) ===
\usepackage{bookmark}
% Fix for unicode-math symbols in PDF bookmarks
\pdfstringdefDisableCommands{%
	\def\xi{xi}%
	\def\alpha{alpha}%
	\def\beta{beta}%
	\def\gamma{gamma}%
	\def\delta{delta}%
	\def\Delta{Delta}%
	\def\epsilon{epsilon}%
	\def\varepsilon{epsilon}%
	\def\theta{theta}%
	\def\kappa{kappa}%
	\def\lambda{lambda}%
	\def\mu{mu}%
	\def\nu{nu}%
	\def\pi{pi}%
	\def\rho{rho}%
	\def\sigma{sigma}%
	\def\tau{tau}%
	\def\phi{phi}%
	\def\chi{chi}%
	\def\psi{psi}%
	\def\omega{omega}%
	\def\Omega{Omega}%
	\def\Lambda{Lambda}%
	\def\times{x}%
	\def\cdot{*}%
	\def\pm{+/-}%
	\def\approx{~}%
	\def\sim{~}%
	\def\equiv{=}%
	\def\ell{l}%
	\def\hbar{h}%
	\def\rightarrow{->}%
	\def\leftarrow{<-}%
	\def\Rightarrow{=>}%
	\def\Leftarrow{<=}%
	\def\propto{~}%
	\def\mitxi{xi}%
	\def\mitalpha{alpha}%
	\def\mitbeta{beta}%
	\def\mitgamma{gamma}%
	\def\mitdelta{delta}%
	\def\mitDelta{Delta}%
	\def\mitepsilon{epsilon}%
	\def\mitvarepsilon{epsilon}%
	\def\mittheta{theta}%
	\def\mitkappa{kappa}%
	\def\mitlambda{lambda}%
	\def\mitLambda{Lambda}%
	\def\mitmu{mu}%
	\def\mitnu{nu}%
	\def\mitpi{pi}%
	\def\mitrho{rho}%
	\def\mitsigma{sigma}%
	\def\mittau{tau}%
	\def\mitphi{phi}%
	\def\mitchi{chi}%
	\def\mitpsi{psi}%
	\def\mitomega{omega}%
	\def\mitOmega{Omega}%
}

% === KAPITEL 13: CLEVEREF (DEUTSCHE LABELS) ===
\usepackage[ngerman]{cleveref}
\crefname{equation}{Gleichung}{Gleichungen}
\crefname{figure}{Abbildung}{Abbildungen}
\crefname{table}{Tabelle}{Tabellen}
\crefname{section}{Abschnitt}{Abschnitte}
\crefname{chapter}{Kapitel}{Kapitel}
\crefname{theorem}{Satz}{Sätze}
\crefname{lemma}{Lemma}{Lemmata}
\crefname{definition}{Definition}{Definitionen}
\crefname{example}{Beispiel}{Beispiele}
\crefname{remark}{Bemerkung}{Bemerkungen}

% ==============================================================================
\newenvironment{alternative}{%
	\begin{mdframed}[linecolor=black!30,linewidth=1pt,roundcorner=4pt,backgroundcolor=black!5]%
	}{%
	\end{mdframed}%
}

% Photon/particle environment
\newenvironment{photon}{%
	\begin{mdframed}[linecolor=blue!30,linewidth=1pt,roundcorner=4pt,backgroundcolor=blue!5]%
	}{%
	\end{mdframed}%
}

% Koide formula box environment
\newenvironment{koidebox}{%
	\begin{mdframed}[linecolor=green!30,linewidth=1pt,roundcorner=4pt,backgroundcolor=green!5]%
	}{%
	\end{mdframed}%
}

% Erkenntnis/insight environment
\newenvironment{erkenntnis}{%
	\begin{mdframed}[linecolor=orange!30,linewidth=1pt,roundcorner=4pt,backgroundcolor=orange!5]%
	}{%
	\end{mdframed}%
}

% Beziehung/relationship environment
\newenvironment{beziehung}{%
	\begin{mdframed}[linecolor=purple!30,linewidth=1pt,roundcorner=4pt,backgroundcolor=purple!5]%
	}{%
	\end{mdframed}%
}

% Derivation environment
\newenvironment{derivation}{%
	\begin{mdframed}[linecolor=teal!30,linewidth=1pt,roundcorner=4pt,backgroundcolor=teal!5]%
	}{%
	\end{mdframed}%
}

% Abhandlung/treatise environment
\newenvironment{abhandlung}{%
	\begin{mdframed}[linecolor=brown!30,linewidth=1pt,roundcorner=4pt,backgroundcolor=brown!5]%
	}{%
	\end{mdframed}%
}

% Anwendung/application environment
\newenvironment{anwendung}{%
	\begin{mdframed}[linecolor=cyan!30,linewidth=1pt,roundcorner=4pt,backgroundcolor=cyan!5]%
	}{%
	\end{mdframed}%
}

% Additional common environments
\newenvironment{konsequenz}{%
	\begin{mdframed}[linecolor=red!30,linewidth=1pt,roundcorner=4pt,backgroundcolor=red!5]%
	}{%
	\end{mdframed}%
}

\newenvironment{schlussfolgerung}{%
	\begin{mdframed}[linecolor=gray!30,linewidth=1pt,roundcorner=4pt,backgroundcolor=gray!5]%
	}{%
	\end{mdframed}%
}

\newenvironment{result}{%
	\begin{mdframed}[linecolor=violet!30,linewidth=1pt,roundcorner=4pt,backgroundcolor=violet!5]%
	}{%
	\end{mdframed}%
}

% Formula environment
\newenvironment{formula}{%
	\begin{mdframed}[linecolor=yellow!30,linewidth=1pt,roundcorner=4pt,backgroundcolor=yellow!5]%
	}{%
	\end{mdframed}%
}

% Revolutionaer/revolutionary environment
\newenvironment{revolutionaer}{%
	\begin{mdframed}[linecolor=red!50,linewidth=2pt,roundcorner=4pt,backgroundcolor=red!10]%
	}{%
	\end{mdframed}%
}

% Formel environment (German version of formula)
\newenvironment{formel}{%
	\begin{mdframed}[linecolor=yellow!30,linewidth=1pt,roundcorner=4pt,backgroundcolor=yellow!5]%
	}{%
	\end{mdframed}%
}

% Prinzip/principle environment
\newenvironment{prinzip}{%
	\begin{mdframed}[linecolor=blue!50,linewidth=2pt,roundcorner=4pt,backgroundcolor=blue!10]%
	}{%
	\end{mdframed}%
}

% Experimentell/experimental environment
\newenvironment{experimentell}{%
	\begin{mdframed}[linecolor=magenta!30,linewidth=1pt,roundcorner=4pt,backgroundcolor=magenta!5]%
	}{%
	\end{mdframed}%
}

% Neutrino environment
\newenvironment{neutrino}{%
	\begin{mdframed}[linecolor=cyan!40,linewidth=1pt,roundcorner=4pt,backgroundcolor=cyan!8]%
	}{%
	\end{mdframed}%
}

% Additional missing environments
\newenvironment{schluessel}{%
	\begin{mdframed}[linecolor=yellow!50,linewidth=1pt,roundcorner=4pt,backgroundcolor=yellow!10]%
	}{%
	\end{mdframed}%
}

\newenvironment{summary}{%
	\begin{mdframed}[linecolor=gray!40,linewidth=1pt,roundcorner=4pt,backgroundcolor=gray!8]%
	}{%
	\end{mdframed}%
}

\newenvironment{category}{%
	\begin{mdframed}[linecolor=pink!40,linewidth=1pt,roundcorner=4pt,backgroundcolor=pink!8]%
	}{%
	\end{mdframed}%
}

\newenvironment{sibox}{%
	\begin{mdframed}[linecolor=lime!40,linewidth=1pt,roundcorner=4pt,backgroundcolor=lime!8]%
	}{%
	\end{mdframed}%
}

% More missing environments
\newenvironment{documentbox}{%
	\begin{mdframed}[linecolor=teal!40,linewidth=1pt,roundcorner=4pt,backgroundcolor=teal!8]%
	}{%
	\end{mdframed}%
}

\newenvironment{t0box}{%
	\begin{mdframed}[linecolor=violet!40,linewidth=1pt,roundcorner=4pt,backgroundcolor=violet!8]%
	}{%
	\end{mdframed}%
}

\newenvironment{wichtig}{%
	\begin{mdframed}[linecolor=red!50,linewidth=2pt,roundcorner=4pt,backgroundcolor=red!10]%
	\textbf{Wichtig:} 
	}{%
	\end{mdframed}%
}

\newenvironment{smbox}{%
	\begin{mdframed}[linecolor=orange!40,linewidth=1pt,roundcorner=4pt,backgroundcolor=orange!8]%
	}{%
	\end{mdframed}%
}

\newenvironment{pvbox}{%
	\begin{mdframed}[linecolor=purple!40,linewidth=1pt,roundcorner=4pt,backgroundcolor=purple!8]%
	}{%
	\end{mdframed}%
}

\newenvironment{numerisch}{%
	\begin{mdframed}[linecolor=blue!40,linewidth=1pt,roundcorner=4pt,backgroundcolor=blue!8]%
	}{%
	\end{mdframed}%
}

% More missing environments
\newenvironment{relation}{%
	\begin{mdframed}[linecolor=green!40,linewidth=1pt,roundcorner=4pt,backgroundcolor=green!8]%
	}{%
	\end{mdframed}%
}

\newenvironment{beweis}{%
	\begin{mdframed}[linecolor=brown!40,linewidth=1pt,roundcorner=4pt,backgroundcolor=brown!8]%
	\textbf{Beweis:} 
	}{%
	\end{mdframed}%
}

\newenvironment{revolution}{%
	\begin{mdframed}[linecolor=red!60,linewidth=2pt,roundcorner=4pt,backgroundcolor=red!12]%
	}{%
	\end{mdframed}%
}

\newenvironment{key}{%
	\begin{mdframed}[linecolor=yellow!50,linewidth=1pt,roundcorner=4pt,backgroundcolor=yellow!10]%
	}{%
	\end{mdframed}%
}

\newenvironment{newperspective}{%
	\begin{mdframed}[linecolor=cyan!50,linewidth=1pt,roundcorner=4pt,backgroundcolor=cyan!10]%
	}{%
	\end{mdframed}%
}

\newenvironment{literatur}{%
	\begin{mdframed}[linecolor=gray!50,linewidth=1pt,roundcorner=4pt,backgroundcolor=gray!10]%
	}{%
	\end{mdframed}%
}

\newenvironment{folgerung}{%
	\begin{mdframed}[linecolor=teal!50,linewidth=1pt,roundcorner=4pt,backgroundcolor=teal!10]%
	}{%
	\end{mdframed}%
}

\newenvironment{principle}{%
	\begin{mdframed}[linecolor=blue!60,linewidth=2pt,roundcorner=4pt,backgroundcolor=blue!12]%
	}{%
	\end{mdframed}%
}

% AB HIER: IHRE DEFINITIONEN (angepasst für Deutsch)
% ==============================================================================

\setcounter{tocdepth}{3}

% === ZITATBEFEHLE ===
\providecommand{\citep}[1]{\cite{#1}}
\providecommand{\citet}[1]{\cite{#1}}

% === FARBEN ===
\definecolor{gold}{RGB}{255,215,0}
\definecolor{blue}{rgb}{0,0,1}
\definecolor{boxgray}{RGB}{240,240,240}
\definecolor{deepblue}{RGB}{0,0,127}
\definecolor{deepgreen}{RGB}{0,127,0}
\definecolor{deepred}{RGB}{191,0,0}
\definecolor{t0blue}{RGB}{33,150,243}
\definecolor{t0green}{RGB}{76,175,80}
\definecolor{t0orange}{RGB}{255,152,0}
\definecolor{t0purple}{RGB}{156,39,176}
\definecolor{t0red}{RGB}{244,67,54}
\definecolor{t0yellow}{RGB}{255,204,0}

% === SPALTENTYPEN ===
\newcolumntype{L}[1]{>{\raggedright\arraybackslash}p{#1}}
\newcolumntype{C}[1]{>{\centering\arraybackslash}p{#1}}
\newcolumntype{R}[1]{>{\raggedleft\arraybackslash}p{#1}}

% === HYPERREF-EINSTELLUNGEN (aktualisiert) ===
\hypersetup{
	colorlinks=true,
	linkcolor=t0blue,
	citecolor=t0blue,
	urlcolor=t0blue,
	breaklinks=true,
	bookmarksnumbered=true,
	pdfstartview=FitH,
	pdfencoding=auto,
	pdfdisplaydoctitle=true
}

% === DEUTSCHE THEOREM-UMGEBUNGEN ===
\theoremstyle{plain}
\newtheorem{theorem}{Satz}[section]
\newtheorem{lemma}[theorem]{Lemma}
\newtheorem{proposition}[theorem]{Proposition}
\newtheorem{corollary}[theorem]{Korollar}

\theoremstyle{definition}
\newtheorem{definition}[theorem]{Definition}
\newtheorem{example}[theorem]{Beispiel}
\newtheorem{insight}[theorem]{Erkenntnis}
\newtheorem{discovery}[theorem]{Entdeckung}

\theoremstyle{remark}
\newtheorem{remark}[theorem]{Bemerkung}
\newtheorem{axiom}{Axiom}
%\newtheorem{principle}{Principle}  % Commented out to avoid conflicts with document-specific definitions
\newtheorem{warnung}[theorem]{Warnung}

% === T0-SPEZIFISCHE BEFEHLE ===
% (Hier folgen alle Ihre \newcommand und \providecommand Definitionen)
% Diese bleiben UNVERÄNDERT wie in Ihrer Original-Preamble
% ==============================================================================
% SECTION 14: T0-Specific Commands
% ==============================================================================

% --- Core T0 Fields ---
\newcommand{\Tfield}{T(x,t)}
\providecommand{\Tfieldt}{T(\vec{x},t)}
\newcommand{\Efield}{E(x,t)}
\newcommand{\mfield}{m(x,t)}
\providecommand{\vecx}{\vec{x}}

% --- Lagrangian ---
\newcommand{\Lag}{\mathcal{L}}
\newcommand{\calL}{\mathcal{L}}

% --- Greek Letters and Constants ---
\newcommand{\alphaem}{\alpha}
\newcommand{\betaT}{\beta_T}
\newcommand{\xiT}{\xi}
\newcommand{\xipar}{\xi}

% --- Energy and Planck Units ---
\newcommand{\Ezero}{E_0}
\newcommand{\EPlanck}{E_{\text{Pl}}}
\newcommand{\Mpl}{M_{\text{Pl}}}
\newcommand{\mP}{m_{\text{P}}}
\newcommand{\lP}{\ell_{\text{P}}}
\newcommand{\tP}{t_{\text{P}}}
\newcommand{\LPlanck}{\ell_{\text{Pl}}}
\newcommand{\TPlanck}{t_{\text{Pl}}}

% --- Coupling Constants ---
\newcommand{\Gnat}{G_{\text{nat}}}
\newcommand{\alphaEM}{\alpha_{\text{EM}}}
\newcommand{\alphaSI}{\alpha_{\text{SI}}}
\newcommand{\Hubble}{H_0}
\newcommand{\LCDM}{\Lambda\text{CDM}}
\newcommand{\natunits}{(nat. units)}

% --- T0 Model Parameters ---
\newcommand{\xigeom}{\xi_{\mathrm{geom}}}
\newcommand{\rzero}{r_{0}}
\newcommand{\xirat}{\xi_{\mathrm{rat}}}
\newcommand{\tzero}{t_{0}}
\newcommand{\Lambdat}{\Lambda_{\mathrm{t}}}
\newcommand{\EP}{E_{\text{P}}}
\newcommand{\Emu}{E_{\mu}}
\newcommand{\Ee}{E_{e}}
\newcommand{\Etau}{E_{\tau}}
\newcommand{\alphafine}{\alpha_{\mathrm{fine}}}
\newcommand{\alphal}{\alpha_{\ell}}
\newcommand{\Lzero}{\ell_{0}}
\newcommand{\Lp}{\ell_{\mathrm{P}}}

% --- Additional T0 Commands ---
\newcommand{\Kfrak}{K_{\text{frak}}}
\newcommand{\Dfrak}{D_{\text{frak}}}
\newcommand{\betapar}{\ensuremath{\beta_T}}
\newcommand{\alphapar}{\alpha}
\newcommand{\deltafield}{\delta \phi}
\newcommand{\deltam}{\delta m}
\newcommand{\deltaE}{\delta E}
\newcommand{\Exi}{E_{\xi}}
\newcommand{\Lxi}{\ell_{\xi}}
\newcommand{\rhoCMB}{\rho_{\text{CMB}}}
\newcommand{\rhoCasimir}{\rho_{\text{Casimir}}}
\newcommand{\Leff}{L_{\text{eff}}}
\newcommand{\CQCD}{C_{\mathrm{QCD}}}
\newcommand{\Kspec}{K_{\mathrm{spec}}}
\newcommand{\Tzero}{\ensuremath{T_0}}
\newcommand{\Eabs}{E_{\text{abs}}}
\newcommand{\taupar}{\tau}

% --- Provided Commands ---
\providecommand{\xiconst}{\xi_{\text{const}}}
\providecommand{\DhiggsT}{D_{\text{Higgs-T}}}
\providecommand{\rhoE}{\rho_{E}}
\providecommand{\Echar}{E_{\text{char}}}
\providecommand{\kfrac}{k_{\text{frac}}}
\providecommand{\alphaEMSI}{\alpha_{\text{EM,SI}}}
\providecommand{\alphaEMnat}{\alpha_{\text{EM,nat}}}
\providecommand{\betaTSI}{\beta_{T,\text{SI}}}
\providecommand{\betaTnat}{\beta_{T,\text{nat}}}
\providecommand{\Gsi}{G_{\text{SI}}}
\providecommand{\xiparSI}{\xi_{\text{SI}}}
\providecommand{\xiparnat}{\xi_{\text{nat}}}
\providecommand{\meff}{m_{\text{eff}}}
\providecommand{\Tzerot}{T_{0}(t)}
\providecommand{\mzerot}{m_{0}(t)}
\providecommand{\Ezeroabs}{E_{0,\text{abs}}}
\providecommand{\Epar}{E_{\text{par}}}
\providecommand{\Lnat}{\ell_{\text{nat}}}
\providecommand{\Tnat}{T_{\text{nat}}}
\providecommand{\xifrak}{\xi_{\text{frac}}}
\providecommand{\Tfrak}{T_{\text{frac}}}
\providecommand{\mfrak}{m_{\text{frac}}}
\providecommand{\Dfrac}{D_{\text{frac}}}
\providecommand{\EphotSI}{E_{\gamma,\text{SI}}}
\providecommand{\EphotNat}{E_{\gamma,\text{nat}}}
\providecommand{\Eabsint}{E_{\text{abs,int}}}
\providecommand{\mphoton}{m_{\gamma}}
\providecommand{\Evis}{E_{\text{vis}}}
\providecommand{\Cto}{C_{T0}}
\providecommand{\mytimes}{\times}
\providecommand{\lambdah}{\lambda_h}
\providecommand{\checkmarkx}{\checkmark}
\providecommand{\Enorm}{E_{\text{norm}}}
\providecommand{\Tobs}{T_{\text{obs}}}
\providecommand{\mobs}{m_{\text{obs}}}
\providecommand{\Eobs}{E_{\text{obs}}}
\providecommand{\Lobs}{\ell_{\text{obs}}}
\providecommand{\xobs}{\xi_{\text{obs}}}
\providecommand{\calE}{\mathcal{E}}
\providecommand{\calT}{\mathcal{T}}
\providecommand{\calM}{\mathcal{M}}
\providecommand{\alphag}{\alpha_g}
\providecommand{\Tmax}{T_{\text{max}}}
\providecommand{\mmin}{m_{\text{min}}}
\providecommand{\Lmax}{\ell_{\text{max}}}
\providecommand{\Emin}{E_{\text{min}}}
\providecommand{\Geff}{G_{\text{eff}}}
\providecommand{\rhoeff}{\rho_{\text{eff}}}
\providecommand{\xieff}{\xi_{\text{eff}}}
\providecommand{\Teff}{T_{\text{eff}}}
\providecommand{\hPlanck}{h}
\providecommand{\kB}{k_B}
\providecommand{\muB}{\mu_B}
\providecommand{\lambdaC}{\lambda_C}
\providecommand{\omegaP}{\omega_P}
\providecommand{\rhoP}{\rho_P}
\providecommand{\Tref}{T_{\text{ref}}}
\providecommand{\Eref}{E_{\text{ref}}}
\providecommand{\mref}{m_{\text{ref}}}
\providecommand{\Lref}{\ell_{\text{ref}}}
\providecommand{\xikonst}{\xi_0}
\providecommand{\Phiphoton}{\Phi_{\gamma}}
\providecommand{\etavis}{\eta_{\text{vis}}}
\providecommand{\pichar}{\pi}
\providecommand{\primrel}{\mathcal{P}_{\text{rel}}}
\providecommand{\warningx}{\textcolor{orange}{\textbf{!}}}
\providecommand{\phiT}{\phi_T}
\providecommand{\Lorentz}{\Lambda}
\providecommand{\Cconv}{C_{\text{conv}}}
\providecommand{\Df}{\Delta f}
\providecommand{\lambdazero}{\lambda_0}
\providecommand{\myapprox}{\approx}
\providecommand{\checked}{\checkmark}
\providecommand{\alphaWSI}{\alpha_W^{\text{SI}}}
\providecommand{\alphaWnat}{\alpha_W^{\text{nat}}}
\providecommand{\vect}[1]{\vec{#1}}
\providecommand{\Rzero}{R_0}
\providecommand{\Riem}{\mathcal{R}}
\providecommand{\nuzero}{\nu_0}
\providecommand{\mypi}{\pi}

% =============================================================================
% TCOLORBOX-STILE UND UMGEBUNGEN (deutsche Titel)
% =============================================================================
\tcbset{
	keyresult/.style={
		colback=blue!5!white,
		colframe=blue!75!black,
		title=Schlüsselergebnis,
		fonttitle=\bfseries
	},
	foundation/.style={
		colback=green!5!white,
		colframe=green!75!black,
		title=Grundlage,
		fonttitle=\bfseries
	},
	alternative/.style={
		colback=orange!5!white,
		colframe=orange!75!black,
		title=Alternative,
		fonttitle=\bfseries
	},
	warningbox/.style={
		colback=red!5!white,
		colframe=red!75!black,
		title=Warnung,
		fonttitle=\bfseries
	}
}

% (Hier folgen alle Ihre tcolorbox-Definitionen mit deutschen Titeln)
\newtcolorbox{keyresultbox}[1][]{colback=blue!5!white,colframe=blue!75!black,fonttitle=\bfseries,title={#1},breakable}
\newtcolorbox{keyresult}[1][Schlüsselergebnis]{colback=blue!5!white,colframe=blue!75!black,fonttitle=\bfseries,title={#1},breakable}
\newtcolorbox{foundationbox}[1][]{colback=green!5!white,colframe=green!75!black,fonttitle=\bfseries,title={#1},breakable}
\newtcolorbox{foundation}[1][Grundlage]{colback=green!5!white,colframe=green!75!black,fonttitle=\bfseries,title={#1},breakable}
\newtcolorbox{alternativebox}[1][]{colback=orange!5!white,colframe=orange!75!black,fonttitle=\bfseries,title={#1},breakable}
\newtcolorbox{warningboxenv}[1][Warnung]{colback=red!5!white,colframe=red!75!black,fonttitle=\bfseries,title={#1},breakable}

\newtcolorbox{fundamental}[1][]{
	colback=boxgray,
	colframe=t0blue,
	fonttitle=\bfseries,
	title=#1,
	sharp corners,
	boxrule=2pt
}

\newtcolorbox{insightBox}[1][Erkenntnis]{colback=blue!5,colframe=t0blue,title={#1},fonttitle=\bfseries,breakable}
\newtcolorbox{discoveryBox}[1][Entdeckung]{colback=green!5,colframe=t0green,title={#1},fonttitle=\bfseries,breakable}
\newtcolorbox{revelation}[1][Offenbarung]{colback=red!5,colframe=t0red,title={#1},fonttitle=\bfseries,breakable}
\newtcolorbox{keypoint}[1][Schlüsselpunkt]{colback=blue!5,colframe=t0blue,title={#1},fonttitle=\bfseries,breakable}
\newtcolorbox{evidence}[1][Beleg]{colback=green!5,colframe=t0green,title={#1},fonttitle=\bfseries,breakable}
\newtcolorbox{conclusionBox}[1][Fazit]{colback=gray!5,colframe=gray,title={#1},fonttitle=\bfseries,breakable}
\newtcolorbox{significance}[1][Bedeutung]{colback=yellow!5,colframe=orange,title={#1},fonttitle=\bfseries,breakable}
\newtcolorbox{philosophical}[1][Philosophisch]{colback=purple!5,colframe=purple,title={#1},fonttitle=\bfseries,breakable}
\newtcolorbox{implicationBox}[1][Implikation]{colback=cyan!5,colframe=cyan,title={#1},fonttitle=\bfseries,breakable}
\newtcolorbox{perspectiveBox}[1][Perspektive]{colback=blue!5,colframe=t0blue,title={#1},fonttitle=\bfseries,breakable}
\newtcolorbox{revolutionary}[1][Revolutionär]{colback=red!5,colframe=t0red,title={#1},fonttitle=\bfseries,breakable}

\newtcolorbox{technical}[1][Technisch]{colback=gray!5,colframe=gray!75!black,title={#1},fonttitle=\bfseries,breakable}
\newtcolorbox{technicalBox}[1][Technisch]{colback=gray!5,colframe=gray!75!black,title={#1},fonttitle=\bfseries,breakable}
\newtcolorbox{notationBox}[1][Notation]{colback=yellow!5,colframe=yellow!75!black,title={#1},fonttitle=\bfseries,breakable}
\newtcolorbox{verification}[1][Verifikation]{colback=orange!5!white,colframe=orange!75!black,fonttitle=\bfseries,title=#1}
\newtcolorbox{explanationBox}[1][Erklärung]{colback=purple!5!white,colframe=purple!75!black,fonttitle=\bfseries,title=#1}
\newtcolorbox{interpretationBox}[1][Interpretation]{colback=cyan!5!white,colframe=cyan!75!black,fonttitle=\bfseries,title=#1}
\newtcolorbox{explanation}[1][Erklärung]{colback=purple!5!white,colframe=purple!75!black,fonttitle=\bfseries,title=#1,breakable}
\newtcolorbox{interpretation}[1][Interpretation]{colback=cyan!5!white,colframe=cyan!75!black,fonttitle=\bfseries,title=#1,breakable}
\newtcolorbox{proof_step}[1][Beweisschritt]{colback=gray!5!white,colframe=gray!75!black,fonttitle=\bfseries,title=#1,breakable}
\newtcolorbox{experimental}[1][Experimentell]{colback=teal!5!white,colframe=teal!75!black,fonttitle=\bfseries,title=#1,breakable}

\newtcolorbox{important}[1][Wichtig]{colback=red!5!white,colframe=red!75!black,title={#1},fonttitle=\bfseries,breakable}
\newtcolorbox{warning}[1][Warnung]{colback=orange!5!white,colframe=orange!75!black,title={#1},fonttitle=\bfseries,breakable}
\newtcolorbox{caution}[1][Vorsicht]{colback=yellow!5!white,colframe=yellow!75!black,title={#1},fonttitle=\bfseries,breakable}
\newtcolorbox{vorsicht}[1][Vorsicht]{colback=yellow!5!white,colframe=yellow!75!black,title={#1},fonttitle=\bfseries,breakable}
\newtcolorbox{highlight}[1][Hervorhebung]{colback=yellow!10!white,colframe=yellow!75!black,title={#1},fonttitle=\bfseries,breakable}
\newtcolorbox{critical}[1][Kritisch]{colback=red!10!white,colframe=red!75!black,title={#1},fonttitle=\bfseries,breakable}

\newtcolorbox{analysis}[1][Analyse]{colback=blue!5!white,colframe=blue!75!black,title={#1},fonttitle=\bfseries,breakable}
\newtcolorbox{application}[1][Anwendung]{colback=green!5!white,colframe=green!75!black,title={#1},fonttitle=\bfseries,breakable}
\newtcolorbox{experiment}[1][Experiment]{colback=cyan!5!white,colframe=cyan!75!black,title={#1},fonttitle=\bfseries,breakable}
\newtcolorbox{historical}[1][Historisch]{colback=brown!5!white,colframe=brown!75!black,title={#1},fonttitle=\bfseries,breakable}
\newtcolorbox{numerical}[1][Numerisch]{colback=gray!5!white,colframe=gray!75!black,title={#1},fonttitle=\bfseries,breakable}
\newtcolorbox{overview}[1][Überblick]{colback=blue!5!white,colframe=blue!75!black,title={#1},fonttitle=\bfseries,breakable}
\newtcolorbox{speculation}[1][Spekulation]{colback=purple!5!white,colframe=purple!75!black,title={#1},fonttitle=\bfseries,breakable}
\newtcolorbox{question}[1][Frage]{colback=orange!5!white,colframe=orange!75!black,title={#1},fonttitle=\bfseries,breakable}
\newtcolorbox{method}[1][Methode]{colback=teal!5!white,colframe=teal!75!black,title={#1},fonttitle=\bfseries,breakable}
\newtcolorbox{correct}[1][Korrekt]{colback=green!10!white,colframe=green!75!black,title={#1},fonttitle=\bfseries,breakable}
\newtcolorbox{units}[1][Einheiten]{colback=gray!5!white,colframe=gray!75!black,title={#1},fonttitle=\bfseries,breakable}
\newtcolorbox{achievement}[1][Errungenschaft]{colback=gold!5!white,colframe=orange!75!black,title={#1},fonttitle=\bfseries,breakable}
\newtcolorbox{equivalence}[1][Äquivalenz]{colback=cyan!5!white,colframe=cyan!75!black,title={#1},fonttitle=\bfseries,breakable}
\newtcolorbox{dimensional}[1][Dimensionsanalyse]{colback=purple!5!white,colframe=purple!75!black,title={#1},fonttitle=\bfseries,breakable}

% === ZUSÄTZLICHE EINFACHE UMGEBUNGEN ===
\newenvironment{treatise}{\begin{quote}}{\end{quote}}
\newenvironment{gemeinsam}{\begin{quote}}{\end{quote}}
\newenvironment{vergleich}{\begin{quote}}{\end{quote}}
\newenvironment{vorteil}{\begin{quote}}{\end{quote}}
\newenvironment{quantum}{\begin{quote}}{\end{quote}}

% === LAYOUT-EINSTELLUNGEN ===
\raggedbottom
\usepackage{environ}
\let\oldtabular\tabular
\let\endoldtabular\endtabular

\newenvironment{scaledtable}[1][0.85]{%
	\begingroup\footnotesize\setlength{\LTleft}{0pt}\setlength{\LTright}{0pt}%
}{%
	\endgroup%
}

\newcommand{\widetable}[1]{\resizebox{\textwidth}{!}{#1}}

% === INHALTSVERZEICHNIS-FORMATIERUNG ===
\renewcommand{\cftsecfont}{\color{blue}}
\renewcommand{\cftsubsecfont}{\color{blue}}
\renewcommand{\cftsecpagefont}{\color{blue}}
\renewcommand{\cftsubsecpagefont}{\color{blue}}
\renewcommand{\cfttoctitlefont}{\huge\bfseries\color{blue}}

% === STANDARD-KOPF- UND FUßZEILE ===
\pagestyle{fancy}
\fancyhf{}
\fancyhead[L]{\textsc{T0 Theorie}}
\fancyhead[R]{\textsc{J. Pascher}}
\fancyfoot[C]{\thepage}

% ==============================================================================
% Ende der Shared Preamble für Deutsch
% ==============================================================================

\title{Das T0-Modell: Die Hubble-Konstante in einem statischen Universum \\
		Energieverlust durch das universelle $\xi$-Feld}
\author{}
\date{}

\begin{document}

	
	\title{Das T0-Modell: Die Hubble-Konstante in einem statischen Universum \\
		Energieverlust durch das universelle $\xi$-Feld}
	\author{Johann Pascher}
	\date{\today}
	
	\maketitle
	
	\begin{abstract}
		Das T0-Modell reinterpretiert die Hubble-Konstante $H_0$ im Rahmen eines statischen Universums, in dem die beobachtete Rotverschiebung durch Photonen-Energieverlust während der Ausbreitung durch das allgegenwärtige $\xi$-Feld entsteht und nicht durch Raumexpansion. Mit der universellen geometrischen Konstante $\xi = \frac{4}{3} \times 10^{-4}$ und Energiefeld-Dynamik leiten wir die Hubble-Konstante als $H_0 = 67{,}2$ km/s/Mpc ohne freie Parameter ab. Dieser Ansatz eliminiert dunkle Energie, löst die Hubble-Spannung natürlich auf und bietet eine einheitliche Beschreibung basierend auf dreidimensionaler Raumgeometrie in natürlichen Einheiten mit $\hbar = c = k_B = 1$.
	\end{abstract}
	
	\tableofcontents
	
	\section{Einleitung: Die Hubble-Konstante neu gedacht}
	
	Die konventionelle Interpretation des Hubble-Gesetzes geht davon aus, dass sich Galaxien aufgrund des expandierenden Raums voneinander entfernen, was zur bekannten Beziehung $v = H_0 d$ führt, bei der die Fluchtgeschwindigkeit linear mit der Entfernung zunimmt. Dieses Expansionsparadigma hat jedoch zahlreiche theoretische Schwierigkeiten geschaffen, einschließlich der Anforderung von 69\% dunkler Energie, anhaltender Meßspannungen und Feinabstimmungsproblemen, die darauf hindeuten, dass unser Verständnis möglicherweise grundlegend unvollständig ist.
	
	Das T0-Modell bietet eine radikal andere Perspektive: Das Universum ist statisch, und was wir als Rotverschiebung beobachten, stellt tatsächlich Energieverlust von Photonen dar, während sie sich durch das universelle $\xi$-Feld ausbreiten, das den gesamten Raum durchdringt. Diese Neuinterpretation verwandelt die Hubble-Konstante von einem Maß für Raumexpansion in eine charakteristische Energieverlustrate und bietet ein eleganteres und theoretisch konsistenteres Rahmenwerk.
	
	\begin{revolutionary}
		Im T0-Modell expandiert der Raum nicht. Stattdessen repräsentiert die Hubble-Konstante $H_0$ die charakteristische Rate, mit der Photonen Energie an das universelle $\xi$-Feld während kosmischer Ausbreitung verlieren.
	\end{revolutionary}
	
	Die fundamentale Erkenntnis ist, dass die Zeit-Energie-Dualität, ausgedrückt durch Heisenbergs Unschärferelation $\Delta E \cdot \Delta t \geq \hbar/2$, einen zeitlichen Beginn des Universums verbietet. Wenn alles aus einer Urknall-Singularität entstanden wäre, würde das endliche Zeitintervall eine unendliche Energieunschärfe erfordern und die Quantenmechanik verletzen. Daher muss das Universum ewig existiert haben, wodurch Raumexpansion unnötig wird, um kosmische Beobachtungen zu erklären.
	
	\section{Symboldefinitionen und Einheiten}
	
	\subsection{Primäre Symbole}
	
	\begin{longtable}{|c|l|l|}
		\hline
		\textbf{Symbol} & \textbf{Bedeutung} & \textbf{Dimension [Natürliche Einheiten]} \\
		\hline
		$\xi$ & Universelle geometrische Konstante & $[1]$ (dimensionslos) \\
		$H_0$ & Hubble-Parameter & $[T^{-1}] = [E]$ \\
		$E_{\text{field}}$ & Universelles Energiefeld & $[E]$ \\
		$E_\xi$ & Charakteristische $\xi$-Feld-Energieskala & $[E]$ \\
		$z$ & Kosmologische Rotverschiebung & $[1]$ (dimensionslos) \\
		$d$ & Entfernung & $[L] = [E^{-1}]$ \\
		$E_0$ & Anfangs-Photonen-Energie & $[E]$ \\
		$E(x)$ & Photonen-Energie nach Entfernung $x$ & $[E]$ \\
		$f(E/E_\xi)$ & Dimensionslose Kopplungsfunktion & $[1]$ \\
		$E_{\text{typical}}$ & Typische kosmologische Photonen-Energie & $[E]$ \\
		\hline
	\end{longtable}
	
	\subsection{Konvention natürlicher Einheiten}
	
	Durchgehend verwenden wir natürliche Einheiten, in denen die fundamentalen Konstanten auf Eins gesetzt werden:
	
	\begin{align}
		\hbar &= 1 \quad \text{(reduzierte Planck-Konstante)} \\
		c &= 1 \quad \text{(Lichtgeschwindigkeit)} \\
		k_B &= 1 \quad \text{(Boltzmann-Konstante)}
	\end{align}
	
	In diesem System werden alle Größen in Bezug auf Energiedimensionen ausgedrückt:
	\begin{itemize}
		\item \textbf{Länge}: $[L] = [E^{-1}]$ (inverse Energie)
		\item \textbf{Zeit}: $[T] = [E^{-1}]$ (inverse Energie)
		\item \textbf{Masse}: $[M] = [E]$ (Energie)
		\item \textbf{Frequenz}: $[\omega] = [E]$ (Energie)
	\end{itemize}
	
	Diese Dimensionsreduktion offenbart die tiefe Einheit, die physikalischen Phänomenen zugrunde liegt, und eliminiert unnötige Umrechnungsfaktoren in theoretischen Berechnungen.
	
	\subsection{Einheiten-Umrechnungsfaktoren}
	
	Für die Umrechnung zwischen natürlichen Einheiten und konventionellen Einheiten:
	
	\begin{align}
		1 \text{ (nat. Einh.)} &= \hbar c = 1{,}973 \times 10^{-7} \text{ eV·m} \\
		1 \text{ (nat. Einh.)} &= \frac{\hbar}{c} = 3{,}336 \times 10^{-16} \text{ eV·s} \\
		H_0 \text{ (km/s/Mpc)} &= H_0 \text{ (nat. Einh.)} \times \frac{c}{\text{Mpc}} \\
		&= H_0 \text{ (nat. Einh.)} \times 9{,}716 \times 10^{-15} \text{ s}^{-1}
	\end{align}
	
\section{Das universelle $\xi$-Feld-Framework}

Der Grundstein des T0-Modells ist die universelle geometrische Konstante, die als fundamentaler Parameter für alle physikalischen Berechnungen dient.

\begin{formula}
	Die universelle geometrische Konstante:
	\begin{equation}
		\xi = \frac{4}{3} \times 10^{-4} = 1,3333... \times 10^{-4}
	\end{equation}
\end{formula}

Diese dimensionslose Konstante wird in der gesamten T0-Theorie verwendet, um quantenmechanische und gravitative Phänomene zu verbinden. Sie legt die charakteristische Stärke der Feldwechselwirkungen fest und bildet die Grundlage für einheitliche Feldbeschreibungen.

\begin{important}
	Für die detaillierte Herleitung und physikalische Begründung dieses Parameters siehe das Dokument "Parameterherleitung" (verfügbar unter: \url{https://github.com/jpascher/T0-Time-Mass-Duality/2/pdf/parameterherleitung_De.pdf}).
\end{important}

Diese geometrische Konstante bestimmt eine charakteristische Energieskala für das $\xi$-Feld:

\begin{equation}
	E_\xi = \frac{1}{\xi} = \frac{3}{4 \times 10^{-4}} = 7500 \text{ (natürliche Einheiten)}
\end{equation}
	
	Das $\xi$-Feld repräsentiert ein universelles Energiefeld, das den gesamten Raum durchdringt und Wechselwirkungen zwischen Photonen und dem Vakuum vermittelt. Im Gegensatz zu konventionellen Feldtheorien, die mehrere unabhängige Felder postulieren, reduziert das T0-Modell die gesamte Physik auf Anregungen und Wechselwirkungen dieses einzelnen universellen Feldes, beschrieben durch die Wellengleichung:
	
	\begin{equation}
		\square E_{\text{field}} = \left(\nabla^2 - \frac{\partial^2}{\partial t^2}\right) E_{\text{field}} = 0
	\end{equation}
	
	\section{Energieverlust-Mechanismus und Rotverschiebung}
	
	Die fundamentale Erkenntnis des T0-Modells ist, dass Photonen Energie durch direkte Wechselwirkung mit dem $\xi$-Feld während ihrer Ausbreitung durch den Raum verlieren. Dieser Energieverlust-Mechanismus bietet eine natürliche Erklärung für kosmologische Rotverschiebung ohne Raumexpansion oder exotische dunkle Energie-Komponenten zu benötigen.
	
	\subsection{Fundamentale Energieverlust-Gleichung}
	
	Die Rate, mit der Photonen Energie verlieren, hängt von ihrer Wechselwirkungsstärke mit dem $\xi$-Feld ab und folgt der Differentialgleichung:
	
	\begin{equation}
		\frac{dE}{dx} = -\xi \cdot f\left(\frac{E}{E_\xi}\right) \cdot E
	\end{equation}
	
	Hier repräsentiert $f(E/E_\xi)$ eine dimensionslose Kopplungsfunktion, die bestimmt, wie die Wechselwirkungsstärke von der Photonen-Energie relativ zur charakteristischen $\xi$-Feld-Energieskala abhängt. Das negative Vorzeichen zeigt Energieverlust an, und die Abhängigkeit von $E$ zeigt, dass höherenergetische Photonen stärkere Kopplung an das Feld erfahren.
	
	Für theoretische Einfachheit und zur Etablierung des grundlegenden Mechanismus betrachten wir die lineare Kopplungs-Näherung, bei der die Kopplungsfunktion einfach proportional zum Energieverhältnis ist:
	
	\begin{equation}
		f\left(\frac{E}{E_\xi}\right) = \frac{E}{E_\xi}
	\end{equation}
	
	Dies führt zur vereinfachten Energieverlust-Gleichung:
	
	\begin{equation}
		\frac{dE}{dx} = -\frac{\xi E^2}{E_\xi} = -\xi^2 E^2
	\end{equation}
	
	Die quadratische Abhängigkeit von der Energie spiegelt die nichtlineare Natur von Feldwechselwirkungen wider und erklärt, warum höherenergetische Photonen ausgeprägtere Rotverschiebungs-Effekte in bestimmten Bereichen zeigen.
	
	\subsection{Lösung für kosmologische Entfernungen}
	
	Für kosmologische Beobachtungen, bei denen der Energieverlust klein im Vergleich zur anfänglichen Photonen-Energie bleibt ($\xi^2 E_0 x \ll 1$), können wir die Differentialgleichung störungstheoretisch lösen. Die resultierende Energie als Funktion der Entfernung wird:
	
	\begin{equation}
		E(x) = E_0 \left(1 - \xi^2 E_0 x\right)
	\end{equation}
	
	Diese Lösung zeigt, dass Photonen Energie linear mit der Entfernung für kleine Verluste verlieren, was natürlich das beobachtete lineare Hubble-Gesetz reproduziert. Die kosmologische Rotverschiebung ist dann definiert als:
	
	\begin{equation}
		z = \frac{E_0 - E(x)}{E(x)} \approx \frac{E_0 - E(x)}{E_0} = \xi^2 E_0 x
	\end{equation}
	
	Diese fundamentale Beziehung zeigt, dass die Rotverschiebung sowohl zur anfänglichen Photonen-Energie als auch zur zurückgelegten Entfernung proportional ist und eine natürliche Erklärung für das beobachtete Hubble-Gesetz ohne Raumexpansion bietet.
	
	\section{Herleitung der Hubble-Konstante}
	
	Das beobachtende Hubble-Gesetz wird konventionell als $z = H_0 d/c$ geschrieben, wobei $H_0$ als Expansionsrate interpretiert wird. Im T0-Modell entsteht dieselbe Beziehung natürlich aus Energieverlust, aber mit einer völlig anderen physikalischen Interpretation.
	
	\subsection{Verbindung zum Energieverlust}
	
	Vergleichen wir die beobachtende Form mit unserem Energieverlust-Ergebnis:
	
	\begin{align}
		z_{\text{beob}} &= \frac{H_0 d}{c} \\
		z_{\text{T0}} &= \xi^2 E_0 x
	\end{align}
	
	Für Konsistenz müssen diese gleich sein, was uns gibt:
	
	\begin{equation}
		\frac{H_0 d}{c} = \xi^2 E_0 x
	\end{equation}
	
	Da die Entfernung $d$ und die Ausbreitungslänge $x$ im statischen Universum gleich sind und $c = 1$ in natürlichen Einheiten verwenden, erhalten wir:
	
	\begin{formula}
		Die Hubble-Konstante im T0-Modell:
		\begin{equation}
			H_0 = \xi^2 E_{\text{typical}}
		\end{equation}
	\end{formula}
	
	Dieses bemerkenswerte Ergebnis zeigt, dass die Hubble-Konstante keine fundamentale Konstante ist, sondern vielmehr aus der geometrischen Konstante $\xi$ und der typischen Energieskala von Photonen, die in kosmologischen Beobachtungen verwendet werden, hervorgeht.
	
	\subsection{Charakteristische Energieskala für kosmologische Beobachtungen}
	
	Die meisten kosmologischen Entfernungsmessungen werden mit optischem und nahinfrarotem Licht durchgeführt, entsprechend Wellenlängen zwischen etwa 400 nm und 2000 nm. Die typischen Photonen-Energien in diesem Bereich sind:
	
	\begin{equation}
		E_{\text{typical}} = \frac{hc}{\lambda_{\text{typical}}} \approx \frac{1240 \text{ eV·nm}}{1000 \text{ nm}} \approx 1{,}2 \text{ eV}
	\end{equation}
	
	Umrechnung in natürliche Einheiten, wo Energien relativ zur fundamentalen Skala gemessen werden:
	
	\begin{equation}
		E_{\text{typical}} \approx 1{,}2 \text{ eV} \times \frac{1}{1{,}602 \times 10^{-19} \text{ J/eV}} \times \frac{1}{1{,}055 \times 10^{-34} \text{ J·s}} \approx 10^{-9} \text{ (natürliche Einheiten)}
	\end{equation}
	
	Diese Energieskala repräsentiert das charakteristische Quantum elektromagnetischer Strahlung, das in den meisten kosmologischen Beobachtungen verwendet wird, und bestimmt die Stärke der Kopplung an das $\xi$-Feld.
	
	\subsection{Numerische Berechnung}
	
	Einsetzen der Werte in unsere Formel für die Hubble-Konstante:
	
	\begin{align}
		H_0 &= \xi^2 E_{\text{typical}} \\
		&= \left(\frac{4}{3} \times 10^{-4}\right)^2 \times 10^{-9} \\
		&= \frac{16}{9} \times 10^{-8} \times 10^{-9} \\
		&= 1{,}78 \times 10^{-17} \text{ (natürliche Einheiten)}
	\end{align}
	
	Um dieses Ergebnis in die konventionellen Einheiten von km/s/Mpc umzurechnen, verwenden wir den Umrechnungsfaktor:
	
	\begin{align}
		H_0 &= 1{,}78 \times 10^{-17} \times \frac{c}{\text{Mpc}} \\
		&= 1{,}78 \times 10^{-17} \times \frac{2{,}998 \times 10^8 \text{ m/s}}{3{,}086 \times 10^{22} \text{ m}} \\
		&= 1{,}78 \times 10^{-17} \times 9{,}716 \times 10^{-15} \text{ s}^{-1} \\
		&= 67{,}2 \text{ km/s/Mpc}
	\end{align}
	
	\section{Dimensionsanalyse und Konsistenzprüfung}
	
	Ein entscheidender Test jeder physikalischen Theorie ist die Dimensionskonsistenz. Lassen Sie uns verifizieren, dass alle unsere Gleichungen die korrekten Dimensionen in natürlichen Einheiten beibehalten.
	
	\subsection{Energieverlust-Gleichung}
	
	\begin{align}
		\left[\frac{dE}{dx}\right] &= \frac{[E]}{[L]} = \frac{[E]}{[E^{-1}]} = [E^2] \\
		\left[-\xi^2 E^2\right] &= [1] \times [E]^2 = [E^2] \quad \checkmark
	\end{align}
	
	\subsection{Rotverschiebungs-Formel}
	
	\begin{align}
		[z] &= [1] \text{ (dimensionslos)} \\
		[\xi^2 E_0 x] &= [1] \times [E] \times [E^{-1}] = [1] \quad \checkmark
	\end{align}
	
	\subsection{Hubble-Parameter}
	
	\begin{align}
		[H_0] &= [T^{-1}] = [E] \text{ (in natürlichen Einheiten)} \\
		[\xi^2 E_{\text{typical}}] &= [1] \times [E] = [E] \quad \checkmark
	\end{align}
	
	\subsection{Vollständige Konsistenz-Tabelle}
	
	\begin{table}[htbp]
		\centering
		\begin{tabular}{lccc}
			\toprule
			\textbf{Größe} & \textbf{T0-Ausdruck} & \textbf{Dimension} & \textbf{Status} \\
			\midrule
			Geometrische Konstante & $\xi = 4/3 \times 10^{-4}$ & $[1]$ & \checkmark \\
			Energieskala & $E_\xi = 1/\xi$ & $[E]$ & \checkmark \\
			Energieverlustrate & $dE/dx = -\xi^2 E^2$ & $[E^2]$ & \checkmark \\
			Rotverschiebung & $z = \xi^2 E_0 x$ & $[1]$ & \checkmark \\
			Hubble-Parameter & $H_0 = \xi^2 E_{\text{typ}}$ & $[E] = [T^{-1}]$ & \checkmark \\
			Feldgleichung & $\square E_{\text{field}} = 0$ & $[E^3] = [E^3]$ & \checkmark \\
			\bottomrule
		\end{tabular}
		\caption{Dimensionskonsistenz-Verifikation}
		\label{tab:dimensional_check}
	\end{table}
	
	Die vollständige Dimensionskonsistenz zeigt, dass das T0-Modell ein mathematisch solides Rahmenwerk bietet, in dem alle Beziehungen natürlich aus der fundamentalen geometrischen Konstante und der Energiefeld-Dynamik folgen.
	
	\section{Experimenteller Vergleich und Validierung}
	
	Der strengste Test für die Gültigkeit des T0-Modells ist seine Übereinstimmung mit beobachtenden Messungen der Hubble-Konstante. Die letzten Jahre haben die Hubble-Spannung erlebt - eine anhaltende Uneinigkeit zwischen Messungen des frühen Universums (aus der kosmischen Mikrowellen-Hintergrundstrahlung) und Messungen des späten Universums (aus lokalen Entfernungsindikatoren).
	
	\subsection{Aktuelle Beobachtungslandschaft}
	
	\begin{table}[htbp]
		\resizebox{\textwidth}{!}{%
		\centering
		\begin{tabular}{lccc}
			\toprule
			\textbf{Quelle} & \textbf{$H_0$ (km/s/Mpc)} & \textbf{Unsicherheit} & \textbf{Methode} \\
			\midrule
			\rowcolor{blue!20}
			\textbf{T0-Vorhersage} & \textbf{67{,}2} & \textbf{Parameterfrei} & \textbf{$\xi$-Feld-Theorie} \\
			Planck 2020 (CMB) & 67{,}4 & $\pm$ 0{,}5 & Frühe Universums-Sonde \\
			SH0ES 2022 & 73{,}0 & $\pm$ 1{,}0 & Lokale Entfernungsleiter \\
			H0LiCOW & 73{,}3 & $\pm$ 1{,}7 & Gravitationslinsen \\
			TRGB-Methode & 69{,}8 & $\pm$ 1{,}7 & Spitze des roten Riesenastes \\
			Oberflächenhelligkeit & 69{,}8 & $\pm$ 1{,}6 & Galaxien-Oberflächenhelligkeit \\
			\bottomrule
		\end{tabular}}
		\caption{Vergleich der T0-Vorhersage mit experimentellen Messungen}
		\label{tab:h0_comparison}
	\end{table}
	
	\subsection{Übereinstimmungsanalyse}
	
	Die T0-Vorhersage von $H_0 = 67{,}2$ km/s/Mpc zeigt bemerkenswerte Übereinstimmung mit Messungen des frühen Universums und erreicht 99{,}7\% Übereinstimmung mit dem Planck-CMB-Ergebnis. Diese enge Übereinstimmung ist besonders bedeutsam, weil das T0-Modell diesen Wert aus fundamentalen geometrischen Prinzipien ohne freie Parameter oder empirische Anpassung ableitet.
	
	Die Uneinigkeit mit lokalen Messungen (SH0ES, H0LiCOW) kann im T0-Rahmenwerk als Entstehen aus der energieabhängigen Natur von $\xi$-Feld-Wechselwirkungen verstanden werden. Verschiedene beobachtende Methoden sondieren verschiedene Photonen-Energiebereiche und Entfernungsskalen, was zu systematischen Variationen in der effektiven Kopplungsstärke führt.
	
	\begin{experimental}
		Das T0-Modell erklärt natürlich die Hubble-Spannung: Sonden des frühen Universums (CMB) sind weniger von kumulativem $\xi$-Feld-Energieverlust betroffen als lokale Entfernungsmessungen, was zu systematisch verschiedenen effektiven Werten von $H_0$ führt.
	\end{experimental}
	
	\subsection{Physikalische Interpretation der Messunterschiede}
	
	Im konventionellen Expansionsparadigma repräsentiert die Hubble-Spannung eine fundamentale Krise, weil die Expansionsrate eine universelle Konstante sein sollte. Im T0-Modell sind jedoch Variationen in der effektiven Hubble-Konstante zu erwarten, weil verschiedene Messmethoden verschiedene Aspekte des Energieverlust-Mechanismus sondieren.
	
	Messungen des frühen Universums (CMB) spiegeln primär die Hintergrund-$\xi$-Feld-Eigenschaften wider, die während der unendlichen Vergangenheit des Universums etabliert wurden, während lokale Messungen kumulative Energieverlust-Effekte über endliche Entfernungen sondieren. Dies erklärt natürlich, warum Methoden des frühen Universums niedrigere Werte als lokale Methoden ergeben und löst die Spannung durch Physik statt durch exotische Modifikationen des Standardmodells auf.
	
	\section{Theoretische Vorteile und Problemlösung}
	
	Die Neuinterpretation der Hubble-Konstante des T0-Modells als Energieverlustrate statt als Expansionsrate löst zahlreiche langjährige Probleme in der Kosmologie und bietet ein eleganteres theoretisches Rahmenwerk.
	
	\subsection{Eliminierung dunkler Energie}
	
	Vielleicht der bedeutendste Vorteil ist die vollständige Eliminierung dunkler Energie aus kosmologischen Modellen. Im konventionellen Paradigma erfordert die beobachtete Beschleunigung der kosmischen Expansion, dass 69\% des Universums aus einer exotischen Energieform mit negativem Druck bestehen. Diese dunkle Energie wurde niemals in Laborexperimenten entdeckt und repräsentiert eines der größten Rätsel in der modernen Physik.
	
	Im T0-Modell entsteht scheinbare kosmische Beschleunigung natürlich aus dem entfernungsabhängigen Energieverlust-Mechanismus. Entferntere Objekte zeigen größere Rotverschiebungen nicht, weil der Raum seine Expansion beschleunigt, sondern weil Photonen mehr Gelegenheiten hatten, Energie an das $\xi$-Feld während ihrer längeren Reisezeiten zu verlieren. Dies bietet eine viel natürlichere Erklärung, die keine exotischen Komponenten erfordert.
	
	\subsection{Auflösung von Feinabstimmungsproblemen}
	
	Das konventionelle Urknall-Modell leidet unter zahlreichen Feinabstimmungsproblemen, die spezielle Anfangsbedingungen erfordern, um aktuelle Beobachtungen zu erklären. Das T0-Modell eliminiert diese Schwierigkeiten, weil das Universum unendliche Zeit hatte, seinen aktuellen Zustand zu erreichen, wodurch jede beobachtete Konfiguration ein natürliches Ergebnis langfristiger Evolution statt spezieller Anfangsbedingungen wird.
	
	Das Horizontproblem (warum kausal getrennte Bereiche dieselbe Temperatur haben) ist gelöst, weil alle Bereiche über unendliche Zeit in kausalem Kontakt waren. Das Flachheitsproblem (warum das Universum kritische Dichte hat) verschwindet, weil es keinen anfänglichen Moment gab, der fein abgestimmte Bedingungen erforderte. Das Monopolproblem und andere topologische Defekt-Probleme werden vermieden, weil das Universum niemals schnelle Inflation oder Phasenübergänge von hochenergetischen Anfangszuständen durchlief.
	
	\subsection{Mathematische Eleganz}
	
	Aus theoretischer Sicht erreicht das T0-Modell bemerkenswerte Vereinfachung durch Reduktion aller kosmologischen Parameter auf Ausdrücke mit der einzelnen geometrischen Konstante $\xi$. Wo das Standard-$\Lambda$CDM-Modell sechs unabhängige Parameter (einschließlich der rätselhaften dunklen Energiedichte) erfordert, leitet das T0-Modell alle beobachtbaren Größen aus der fundamentalen dreidimensionalen Raumgeometrie ab.
	
	Diese Parameterreduktion repräsentiert mehr als bloße mathematische Eleganz - sie legt nahe, dass wir möglicherweise die Kosmologie aus einer unnötig komplexen Perspektive angegangen sind, wenn einfachere geometrische Prinzipien dieselben Beobachtungen natürlicher erklären können.
	

	\section{Fazit: Ein neues Paradigma für kosmische Physik}
	
	Die Herleitung der Hubble-Konstante des T0-Modells repräsentiert mehr als nur eine alternative Berechnung - sie verkörpert eine fundamentale Verschiebung in unserem Verständnis kosmischer Physik. Durch Neuinterpretation von $H_0$ als charakteristische Energieverlustrate statt als Expansionsrate erhalten wir ein eleganteres und theoretisch konsistenteres Rahmenwerk, das zahlreiche langjährige Probleme in der Kosmologie löst.
	
	\begin{formula}
		Die vollständige T0-Beziehung für die Hubble-Konstante:
		\begin{equation}
			\boxed{H_0 = \xi^2 E_{\text{typical}} = 67{,}2 \text{ km/s/Mpc}}
		\end{equation}
		Rein abgeleitet aus der geometrischen Konstante $\xi = \frac{4}{3} \times 10^{-4}$
	\end{formula}
	
	Die Schlüsselerfolge dieses Ansatzes schließen die parameterfreie Herleitung von $H_0$ aus fundamentalen geometrischen Prinzipien, die natürliche Auflösung der Hubble-Spannung durch energieabhängige Effekte und die Eliminierung exotischer dunkler Energie-Komponenten ein. Das statische Universum-Rahmenwerk bietet eine natürlichere Grundlage für das Verständnis kosmischer Beobachtungen ohne fein abgestimmte Anfangsbedingungen oder überlichtschnelle Expansion zu erfordern.
	
	Vielleicht am wichtigsten zeigt das T0-Modell, dass scheinbare Komplexität in der Kosmologie aus der Annahme unnötig komplizierter theoretischer Rahmenwerke entstehen kann. Die Reduktion kosmischer Physik auf die einfache Dynamik von Energiefeldern in statischem dreidimensionalem Raum legt nahe, dass die Natur nach eleganteren Prinzipien operiert, als aktuelle Paradigmen annehmen.
	
	\begin{revolutionary}
		Das Universum expandiert nicht. Die Hubble-Konstante misst Energieverlust, nicht Flucht. Alle kosmischen Beobachtungen können durch das universelle $\xi$-Feld in einem statischen, ewig existierenden Universum verstanden werden, das von dreidimensionaler Geometrie regiert wird.
	\end{revolutionary}
	
	Diese Paradigmenverschiebung eröffnet neue Wege für theoretische Entwicklung und experimentelle Untersuchung und führt potentiell zu einem vollständigeren Verständnis der fundamentalen Natur von Raum, Zeit und kosmischer Evolution. Der Erfolg des T0-Modells bei der Herleitung der Hubble-Konstante legt nahe, dass ähnliche geometrische Ansätze für das Verständnis anderer Aspekte kosmischer Physik fruchtbar sein könnten.
	
	\begin{thebibliography}{99}
		
		\bibitem{pascher_cosmic_2025}
		Pascher, J. (2025). \textit{T0-Theorie: Universelle $\xi$-Konstante und kosmischer Mikrowellen-Hintergrund}. Verfügbar unter: \url{https://jpascher.github.io/T0-Time-Mass-Duality/2/pdf/cosmicDe.pdf}
		
		\bibitem{pascher_redshift_2025}
		Pascher, J. (2025). \textit{T0-Theorie: Wellenlängenabhängiger Rotverschiebungs-Mechanismus}. Verfügbar unter: \url{https://jpascher.github.io/T0-Time-Mass-Duality/2/pdf/redshift_deflectionDe.pdf}
		
		\bibitem{pascher_t0_energie_2025}
		Pascher, J. (2025). \textit{T0-Modell: Energiebasierte Formulierung}. Verfügbar unter: \url{https://jpascher.github.io/T0-Time-Mass-Duality/2/pdf/T0-EnergieDe.pdf}
		
		\bibitem{riess_2022}
		Riess, A. G., et al. (2022). \textit{A Comprehensive Measurement of the Local Value of the Hubble Constant}. Astrophys. J. Lett. 934, L7.
		
		\bibitem{planck_2020}
		Planck Collaboration (2020). \textit{Planck 2018 results. VI. Cosmological parameters}. Astron. Astrophys. 641, A6.
		
		\bibitem{wong_2020}
		Wong, K. C., et al. (2020). \textit{H0LiCOW measurement of H0 from lensed quasars}. Mon. Not. R. Astron. Soc. 498, 1420.
		
	\end{thebibliography}
	
\end{document}