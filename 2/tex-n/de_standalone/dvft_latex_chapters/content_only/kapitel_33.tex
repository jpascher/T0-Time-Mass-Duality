\chapter{Kapitel 33}
\label{chap:33}







DERIVING PAULI’S EXCLUSION PRINCIPLE
1. Introduction
This document derives Pauli’s Exclusion Principle from the foundational structure of Dynamic vacuum
field–Curvature Theory (DVFT).
In DVFT, the vacuum field is expressed as:
\Phi(x) = \rho(x) e^{i\theta(x)},
where \rho is the vacuum amplitude and \theta is the vacuum phase. Gravity, geometry, and particle behavior
arise from structured excitations in these fields. To explain Pauli exclusion, we extend \Phi into a multicomponent vacuum field whose excitations—topological defects—represent particles. The exclusion
principle then emerges naturally from the topology and energetics of the vacuum configuration space, not
as an added rule.
2. Multi-Component DVFT Field and Particle Species
To model fermions and bosons, DVFT is extended to an N-component vacuum field:
\Phi_A(x) = \rho_A(x) e^{i\theta_A(x)}, A = 1,2,...,N.
Particles correspond to localized topological excitations (defects) of \Phi_A(x).
Different particle types correspond to different topological classes of vacuum excitations. This step is
analogous to how solitons, vortices, and monopoles emerge in non-linear field theories—except here the
excitations live inside the amplitude–phase structure of the vacuum.
3. Configuration Space and Particle Exchange
Consider two identical DVFT excitations located at positions x_1 and x_2.
Their combined configuration is a point in the configuration space:
C_2 = (R^3 \times R^3 − {x_1 = x_2}) / exchange.
Exchanging the two particles corresponds to a continuous loop in configuration space.
International Journal for Multidisciplinary Research (IJFMR)
E-ISSN: 2582-2160 ● Website: www.ijfmr.com ● Email: editor@ijfmr.com
IJFMR250664112 Volume 7, Issue 6, November-December 2025 75
In DVFT, exchanging defects also induces a continuous deformation of the vacuum fields:
\Phi_A(x) \to \Phi'_A(x),
which may return to the same local configuration but with a global phase holonomy. This holonomy
determines whether the species behaves as a boson or fermion.
4. Exchange Holonomy in the Vacuum Phase Field
Under exchange of identical excitations, the many-body vacuum configuration Ψ may acquire a phase
factor:
Ψ \to e^{i\alpha} Ψ.
Repeating the exchange twice corresponds to a 2\pi rotation of the configuration, which must return to the
same state:
(e^{i\alpha})^2 = 1 \to e^{i\alpha} = \pm1.
Thus DVFT allows two topological classes:
• e^{i\alpha} = +1 \to symmetric state \to bosons
• e^{i\alpha} = –1 \to antisymmetric state \to fermions
This is not assumed; it follows from the topology of vacuum phase evolution under exchange loops.
5. Antisymmetry and Pauli Exclusion
For fermions (e^{i\alpha} = –1), the many-body wavefunctional must satisfy:
Ψ(..., x_i, ..., x_j, ...) = –Ψ(..., x_j, ..., x_i, ...).
Evaluate this at coincidence arguments x_i = x_j:
Ψ(..., x, ..., x, ...) = –Ψ(..., x, ..., x, ...)
Therefore:
Ψ(..., x, ..., x, ...) = 0.
This is Pauli’s Exclusion Principle: the probability amplitude for two identical fermions occupying the
same quantum state vanishes exactly. DVFT thus derives exclusion from a topological phase holonomy
of the vacuum—not from Grassmann variables or postulated anticommutation relations.
6. Topological Interpretation of Spin
In DVFT, spin arises from the internal structure of the vacuum excitation itself:
• Bosonic excitations correspond to integer-winding vacuum defects.
• Fermionic excitations correspond to half-winding or twist defects.
A 2\pi rotation of a half-winding defect results in a sign change of the underlying phase configuration: Ψ
\to –Ψ.
Thus spin-½ behavior is a geometric property of the vacuum excitation, not an axiomatic quantum rule.
Spin and statistics are unified as consequences of vacuum topology.
7. Energetic Origin of Pauli Exclusion in DVFT
Beyond wavefunction antisymmetry, DVFT also provides an energetic justification.
When two identical fermionic defects attempt to overlap spatially, the associated amplitude and phase
fields must deform in a way violating the allowed topological class:
• The vacuum amplitude \rho develops extreme gradients (large |\nabla\rho|^2 term).
• The vacuum phase \theta becomes singular or multi-valued (large \rho^2|\nabla\theta|^2 term).
The DVFT energy functional:
E = \int [ (A/2)|\nabla\rho|^2 + (A/2)\rho^2|\nabla\theta|^2 + U(\rho) ] d^3x
diverges for overlapping fermionic defects.
Thus Pauli exclusion is not only a topological rule but an energy-prohibition:
International Journal for Multidisciplinary Research (IJFMR)
E-ISSN: 2582-2160 ● Website: www.ijfmr.com ● Email: editor@ijfmr.com
IJFMR250664112 Volume 7, Issue 6, November-December 2025 76
certain vacuum configurations simply cannot exist.
8. Summary of Derivation
DVFT explains Pauli exclusion through:
1. Vacuum phase topology:
• Exchange of identical DVFT excitations produces a phase factor e^{i\alpha}.
• Only \alpha = 0 or \pi are allowed \to bosons or fermions.
2. Fermionic antisymmetry:
\alpha = \pi \to Ψ is antisymmetric \to Ψ(x,x) = 0 \to exclusion.
3. Energetics of vacuum defects:
Overlapping fermionic defects produce forbidden gradient and phase singularities \to infinite energy cost.
Thus Pauli’s Exclusion Principle is not arbitrary:
It is a direct consequence of the topological and energetic structure of the DVFT vacuum field.