\chapter{Kapitel 34}
\label{chap:34}







SOLUTION TO THE STRONG CP PROBLEM
1. Introduction
DVFT (Dynamic vacuum field Curvature Theory) provides a natural and structurally unavoidable solution
to the Strong CP Problem, without requiring axions, Peccei–Quinn symmetry, or fine-tuning. This
document explains rigorously why DVFT forces the QCD \theta-angle to zero as a consequence of the vacuum
field structure.
2. Statement of the Strong CP Problem
Quantum Chromodynamics permits a CP-violating term:
L = \theta (g_s^2 / 32\pi^2) G_{\mu\nu} ṠG^{\mu\nu}
Experimentally, neutron EDM measurements require:
\theta < 10⁻^1^0
But the natural value in QCD is \theta \approx 1. The Standard Model provides no mechanism to set \theta \approx 0. This
discrepancy is the Strong CP Problem.
3. Core DVFT Insight: Only One Physical Phase Field
In DVFT, all forces—including QCD—emerge from the single vacuum field:
\Phi(x,t) = \rho(x,t) e^{i\theta(x,t)}
Here \theta(x,t) is the unique global vacuum phase. QCD cannot introduce an independent \theta parameter. No
separate strong-sector phase exists; therefore a CP-violating \theta-term has no place in the fundamental
Lagrangian.
Thus:
\theta_QCD ≡ 0
by structural necessity, not tuning.
4. Why Independent QCD \theta Cannot Exist in DVFT
The QCD \theta-term arises from instanton topology. DVFT reinterprets instantons as localized amplitude
knots in \rho(x), not as separate phase sectors.
DVFT enforces:
• Continuous global \theta(x,t)
• No multi-sector vacuum structure
• No misalignment between QCD and vacuum phases
Therefore a CP-violating G ṠG term cannot emerge.
International Journal for Multidisciplinary Research (IJFMR)
E-ISSN: 2582-2160 ● Website: www.ijfmr.com ● Email: editor@ijfmr.com
IJFMR250664112 Volume 7, Issue 6, November-December 2025 77
5. Neutron Electric Dipole Moment Prediction
DVFT predicts the neutron EDM is approximately zero because the vacuum amplitude around neutrons
is CP-symmetric and the global phase \theta(x) cannot induce sector-specific asymmetry. Thus:
d_n \approx 0
in perfect agreement with experiment, without axions or symmetry breaking.
6. Comparison With Standard Approaches
Standard Model: Offers no explanation; \theta must be tuned < 10⁻^1^0.
Axion/PQ symmetry: Adds particles + symmetry; no experimental detection.
String theory: Introduces many vacua; not predictive.
DVFT: Eliminates \theta as an independent variable. Simple, natural, enforced.
7. Deeper Reason: Correct Ontology
The Strong CP Problem exists only because QCD—incorrectly—treats the vacuum as empty. If the
vacuum is physical (as in DVFT), then its phase structure is unique, global, and non-duplicable. The
freedom to choose \theta is eliminated.
Thus:
\theta_QCD = 0
is not fine-tuned; it is the only mathematically allowable value.
Conclusion
DVFT resolves the Strong CP Problem cleanly and uniquely:
• No axions.
• No fine-tuning.
• No new symmetries.
• Complete alignment with experiment.
• Directly derived from the single vacuum phase field.
This constitutes one of the strongest conceptual triumphs of DVFT.