\chapter{Kapitel 22}
\label{chap:22}







MAXIMUM MASS FOR QUANTUM SUPERPOSITION
1. Introduction
This document presents the Dynamic Vacuum Field Theory(DVFT) prediction for the maximum mass
and size of molecules or macroscopic objects that can remain in quantum superposition.
This question is directly relevant to the MAST-QG (Macroscopic Superpositions for Quantum Gravity)
project.
DVFT provides a mathematically precise, physically motivated cutoff determined by the nonlinear
response of the vacuum-phase field, unlike heuristic or empirical models such as the Diòsi–Penrose (DP)
model.
Here we derive this limit and provide experimentally testable values.
2. DVFT Mechanism for Superposition Stability
DVFT describes the vacuum as a complex field:
\Phi(x) = \rho(x) e^{i\theta(x)}
with:
• \rho(x): vacuum amplitude (inertial content, related to mass),
• \theta(x): vacuum phase (curvature field, source of gravity).
Quantum coherence survives only when the two branches of a superposition satisfy:
\theta$_1$(x) \approx \theta$_2$(x).
Decoherence is not random: it occurs when the vacuum can no longer sustain two incompatible curvature
configurations.
The collapse criterion is:
E_\theta = \int |\nabla\theta$_1$ - \nabla\theta$_2$|$^2$ d$^3$x \geq B \rho$_0$,
where B is the vacuum phase stiffness and \rho$_0$ is the vacuum inertial density.
This gives a physically sharp limit on superposition-scale objects.
International Journal for Multidisciplinary Research (IJFMR)
E-ISSN: 2582-2160 ● Website: www.ijfmr.com ● Email: editor@ijfmr.com
IJFMR250664112 Volume 7, Issue 6, November-December 2025 51
3. Collapse Condition Derived from DVFT
3.1 Phase Curvature Mismatch from Mass Superposition
A mass m in two positions separated by distance d produces two distinct curvature fields based on the
weak-field approximation:
|\nabla\theta| \approx G m / (c$^2$ r$^2$).
The curvature mismatch between the two branches scales as:
|Δ\nabla\theta| \approx G m d / (c$^2$ r$^3$),
and the total mismatch energy is approximately:
E_\theta \approx (G$^2$ m$^2$ / c⁴)(1/d).
3.2 Maximum Mass for Stable Superposition
The DVFT collapse condition:
E_\theta < B \rho$_0$
yields the maximum mass:
m_max \approx \sqrt( B \rho$_0$ c⁴ d / G$^2$ ).
4. Numerical Estimates from DVFT Constants
Using conservative DVFT constants:
B \rho$_0$ \approx 10⁻⁹ J/m$^3$
d \approx 10⁻⁷ m (typical MAST-QG target separation)
we obtain:
m_max \approx 10⁷ – 10⁸ amu.
This is the physical upper bound for stable quantum superposition.
5. Corresponding Size Limit
Assuming molecular/organic matter density of ~1000 kg/m$^3$, the size corresponding to m_max is:
R_max \approx (3 m_max / 4\pi\rho)^{1/3}
\approx 50 – 200 nm.
Thus DVFT predicts the largest possible coherent object in our universe is approximately:
• mass: 10⁷–10⁸ amu
• radius: 50–200 nm
• diameter: ~100 nm scale
Beyond this, vacuum-phase curvature becomes nonlinear, and collapse is immediate.
6. Comparison with Other Collapse Models
6.1 Diòsi–Penrose
DP predicts collapse around 10⁹ amu.
DVFT predicts earlier collapse (10⁷–10⁸ amu) due to nonlinear curvature terms.
6.2 Standard GR + QFT
There is no predicted upper limit in standard theory.
DVFT contradicts this and provides a finite, experimentally falsifiable cutoff.
7. Implications for MAST-QG and Other Experiments
DVFT provides the following predictions:
• Superpositions up to ~10⁷ amu are stable.
• At ~10⁸ amu, collapse begins.
• At >10⁸–10⁹ amu, superposition is fundamentally impossible.
Therefore:
International Journal for Multidisciplinary Research (IJFMR)
E-ISSN: 2582-2160 ● Website: www.ijfmr.com ● Email: editor@ijfmr.com
IJFMR250664112 Volume 7, Issue 6, November-December 2025 52
• If MAST-QG observes superposition at 10⁹–10¹⁰ amu \to DVFT is falsified.
• If collapse occurs in this window \to DVFT is strongly supported.
Conclusion
DVFT gives a clear, first-principles upper bound on the size and mass of quantum superpositions.
This predicts a fundamental cutoff around 10⁷–10⁸ amu (100 nm scale).
This limit is directly testable in upcoming macroscopic quantum experiments such as MAST-QG,
MAQRO, nanodiamond interferometry, and levitated optomechanics.