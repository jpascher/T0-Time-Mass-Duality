\chapter{Kapitel 21}
\label{chap:21}







RON FOLMAN'S T^3 QUANTUM GRAVITY EXPERIMENT
1. Introduction
Ron Folman's T^3 (T-cubed) atom-interferometry experiment represents one of the most precise tests of
quantum systems evolving under gravitational fields. The central result is that the interference phase
accumulated by atomic wave packets in a gravitational potential grows as:
Δφ ∝ g T^3
This scaling differs from the usual T^2 dependence observed in standard light-pulse atom interferometry,
and it arises only when the full quantum evolution of the wave packet, including its spatial trajectory, is
taken into account. The experiment provides a unique bridge between gravity and quantum phase
evolution.
The Dynamic Vacuum Field Theory(DVFT) offers a natural and physically motivated explanation for why
the phase should scale as T^3 — because, under DVFT, gravitational acceleration is not a geometric
construct but is directly encoded in the vacuum-phase field \theta(x).
2. Summary of the T^3 Experiment
2.1 Standard Atom-Interferometry Expectation
In ordinary interferometers, the gravitational phase shift takes the form:
Δφ_standard = k_eff g T^2
where T is the pulse separation time and k_eff is the effective wavevector. This arises purely from
momentum kicks and free-fall separation of the paths.
2.2 Folman’s T^3 Measurement
Folman's experimental design introduces a controlled spatial separation of the wave packet in a linear
gravitational potential, such that the phase is accumulated not only through energy but also through the
*time evolution of the spatial separation*.
This results in:
Δφ_T3 ∝ g T^3
This scaling indicates that the gravitational potential contributes to phase in a way that integrates
displacement, velocity, and acceleration — a deeper coupling to gravitational structure than the T^2 case.
3. DVFT Interpretation: Gravity as Vacuum-Phase Curvature
3.1 Vacuum Field Structure
DVFT postulates a complex vacuum field:
\Phi = \rho e^{i\theta}
where:
• \rho(x) is the vacuum amplitude (stiffness)
• \theta(x) is the vacuum phase (curvature potential)
In DVFT, the gravitational field is not geometric curvature but the spatial gradient of the vacuum phase:
g = |\nabla\theta|.
International Journal for Multidisciplinary Research (IJFMR)
E-ISSN: 2582-2160 ● Website: www.ijfmr.com ● Email: editor@ijfmr.com
IJFMR250664112 Volume 7, Issue 6, November-December 2025 49
Thus any quantum system whose wavefunction contains a phase term e^{iS/\hbar} interacts directly with \theta.
3.2 Why T^3 Scaling Is Natural in DVFT
The quantum phase accumulated by a wave packet is:
Δφ = (1/\hbar) \int L dt.
For a particle in DVFT's gravitational field, the Lagrangian includes the \theta-field coupling:
L \supset m \nabla\theta \cdot ẋ.
Since \nabla\theta = g is constant near Earth's surface,
but ẋ(t) and x(t) both grow with T during wave packet separation, the integral naturally yields:
Δφ ∝ \int g x(t) dt ∝ g T^3.
Thus T^3 scaling arises from three multiplicative factors:
1. \theta evolves linearly in time.
2. Path separation evolves linearly in time.
3. The interaction energy integrates over time.
Multiplying these yields a cubic dependence:
1\times1\times1 \to T^3.
This is not an artifact of interferometer geometry; it is a structural prediction of a vacuum-phase gravity
theory.
4. DVFT Mathematical Derivation of T^3 Scaling
4.1 Phase Accumulation Formula
Consider two paths x_1(t) and x_2(t). DVFT predicts the phase difference:
Δφ = (m/\hbar) \int [\nabla\theta \cdot (ẋ_1 - ẋ_2)] dt.
Let \nabla\theta = g ẑ (constant). Then:
Δφ = (mg/\hbar) \int (ż_1 - ż_2) dt.
4.2 Path Separation Under Constant g
If a momentum kick Δp is applied at t=0, the relative motion is:
z_2(t) - z_1(t) = (Δp/m) t.
Then:
ż_2 - ż_1 = Δp/m (constant).
Substituting:
Δφ = (mg/\hbar) \int (Δp/m) t dt
= (g Δp / \hbar) \int t dt
= (g Δp / 2\hbar) T^2.
So far this gives T^2.
But Folman's experiment introduces **time-dependent displacement**.
If the interferometer sequence is such that displacement grows as t^2 (as in cubic-phase setups), then:
Δz(t) ∝ t^2 \to ż(t) ∝ t.
Thus:
Δφ = (m/\hbar) \int g ż(t) dt ∝ \int g t dt ∝ g T^2.
But the displacement itself was already ∝ t^2, so the *full phase* becomes:
Δφ ∝ g \int t^2 dt = (g/3) T^3.
5. Why GR and QFT Cannot Explain T^3 as Naturally
General Relativity treats gravity as spacetime curvature but does not assign physical meaning to quantum
phase evolution. QFT treats phase evolution quantum mechanically but keeps gravity classical. Neither
International Journal for Multidisciplinary Research (IJFMR)
E-ISSN: 2582-2160 ● Website: www.ijfmr.com ● Email: editor@ijfmr.com
IJFMR250664112 Volume 7, Issue 6, November-December 2025 50
framework identifies gravity with a *physical phase field* as DVFT does. Thus T^3 is not a coincidence
but a direct measurement of vacuum-phase evolution.
6. Experimental Predictions Unique to DVFT
6.1 Higher-Order Corrections
DVFT predicts that if F(X) deviates from linearity, then higher-order corrections appear:
Δφ = a T^3 + b T^4 + c T^5 + …
These terms do not arise in standard QM and thus provide falsifiable tests.
6.2 Sensitivity to Vacuum Nonlinearity
The experiment could directly probe the nonlinear F_X term in DVFT:
\nabla \cdot (F_X \nabla\theta) = \rho_m.
This opens the possibility of **laboratory tests for dark-matter-like vacuum behavior.**
Conclusion
Folman’s T^3 scaling experiment is one of the cleanest demonstrations of gravitational influence on
quantum phase. DVFT provides a direct physical mechanism for this phenomenon, identifying gravity
with the gradient of the vacuum-phase field.
The result strengthens the DVFT framework and suggests that precision quantum interferometry may be
the first experimental window into vacuum-phase curvature — the fundamental origin of gravity in DVFT.