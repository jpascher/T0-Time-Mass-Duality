\chapter{Kapitel 16}
\label{chap:16}







DERIVATION OF THE HUBBLE TENSION
1. Introduction
The Hubble tension refers to the 5–10% mismatch between:
• H$_0$ inferred from early-universe data (CMB, Planck), and
• H$_0$ measured from the late universe (Cepheids and SN Ia).
ΛCDM cannot produce two different Hubble values because the cosmological constant is rigid.
DVFT explains the tension naturally because the vacuum field \Phi = \rho e^{i\theta} is dynamical, and its
amplitude \rho responds differently in the early homogeneous universe and the late structured universe.
2. Vacuum Field and Cosmological Dynamics in DVFT
DVFT begins from:
\Phi(x,t) = \rho(x,t) e^{i\theta(x,t)}
Cosmologically, the relevant variable is \rho(t).
A minimal vacuum potential is:
U(\rho) = ½ \sigma (\rho – \rho$_0$)$^2$ + …
Vacuum energy density:
\rho_vac = ½ A \rhȯ$^2$ + U(\rho)
This replaces the constant Λ in GR.
3. DVFT-Modified Friedmann Equation
With \Phi coupled to FRW geometry, the Friedmann equation becomes:
H$^2$ = (1 / 3M_pl$^2$) [\rho_m + \rho_vac(\rho, \rhȯ)]
with:
\rho_vac = ½ A \rhȯ$^2$ + U(\rho)
\rho(t) satisfies:
A \rhö+ 3A H \rhȯ + dU/d\rho = S_backreact
International Journal for Multidisciplinary Research (IJFMR)
E-ISSN: 2582-2160 ● Website: www.ijfmr.com ● Email: editor@ijfmr.com
IJFMR250664112 Volume 7, Issue 6, November-December 2025 39
S_backreact characterizes how structure perturbations feed into vacuum amplitude dynamics.
4. Early Universe Prediction (CMB Value of H$_0$)
At recombination:
• Universe nearly homogeneous
• S_backreact \approx 0
• \rho \approx \rho*, the equilibrium amplitude
• \rhȯ \approx 0
Thus:
\rho_vac \approx U(\rho*)
giving:
H_CMB$^2$ \approx [\rho_m(early) + U(\rho*)] / (3M_pl$^2$)
This corresponds to the Planck value ~67 km/s/Mpc.
5. Late Universe Prediction (Local Value of H$_0$)
After structure formation:
• S_backreact \neq 0
• Overdensities and voids perturb \rho(x,t)
• Coarse-grained local amplitude: \rhō_local \neq \rho*
• \rhȯ_local may be nonzero
Thus:
\rho_vac(local) = ½ A \rhȯ_local$^2$ + U(\rhō_local)
and:
H_local$^2$ = [\rho_m(local) + \rho_vac(local)] / (3M_pl$^2$)
If structure biases the vacuum slightly upward in its potential:
U(\rhō_local) > U(\rho*)
Then:
H_local > H_CMB
matching the observed tension.
6. Why ΛCDM Cannot Do This
In ΛCDM:
• Λ is constant
• Vacuum does not respond to structure
• Only one H$_0$ exists
DVFT replaces Λ with a dynamical vacuum amplitude.
Thus different cosmic epochs naturally exhibit different effective H$_0$ values.
7. Quantitative Estimate
A small fractional change:
ΔU / U \approx 5–10%
in the effective vacuum energy due to structure-induced changes in \rho is sufficient to produce:
H_local \approx H_CMB (1 + \varepsilon)
with \varepsilon \approx 0.06–0.09.
This matches observational data exactly.
8. Final Interpretation
In DVFT, the Hubble tension is not a contradiction—it is expected.
International Journal for Multidisciplinary Research (IJFMR)
E-ISSN: 2582-2160 ● Website: www.ijfmr.com ● Email: editor@ijfmr.com
IJFMR250664112 Volume 7, Issue 6, November-December 2025 40
It arises because:
• Early $universe = coherent$ vacuum amplitude \to gives H_CMB
• Late $universe = structure$-backreacted vacuum amplitude \to gives H_local
This is direct observational evidence that the vacuum field \Phi = \rho e^{i\theta} is dynamical, not a fixed
cosmological constant.