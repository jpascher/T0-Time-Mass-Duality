\chapter{Kapitel 12}
\label{chap:12}







COSMOLOGY, BIG BANG, AND BIRTH OF THE UNIVERSE
This chapter presents a full cosmological formulation of the Dynamic Vacuum Field Theory(DVFT).
Under DVFT, the universe did not begin as a singularity but as a vacuum-phase transition from a nearzero amplitude pre-vacuum state to the stable dynamic vacuum field state described by the field \Phi =
\rho(x)e^{i\theta(x)}. We show how DVFT naturally explains the Big Bang, inflation, cosmic expansion, dark
energy, cosmic horizon problems, and other fundamental mysteries of cosmology.
1. Introduction
Traditional cosmological models built on General Relativity confront a fundamental problem: they begin
with a singularity at t = 0 where curvature, density, and temperature diverge. This singularity eliminates
the possibility of explaining the physical origin of the universe, inflation, or the emergence of space itself.
DVFT replaces the singularity with a physically meaningful vacuum-phase defect, enabling a consistent
explanation of how the Big Bang occurred, what existed before it, and why the universe expanded so
rapidly.
2. The Vacuum Field in Cosmology
In cosmological symmetry, the vacuum field is homogeneous:
\Phi(t) = \rho(t) e^{i\theta(t)}
Here, \rho(t) is the vacuum amplitude determining vacuum energy density, and \theta(t) encodes dynamic vacuum
field.
The vacuum Lagrangian contributes energy density:
\varepsilon_vac = (d\rho/dt)^2 + \rho^2 (d\theta/dt)^2 + V(\rho)
and pressure:
p_vac = (d\rho/dt)^2 + \rho^2 (d\theta/dt)^2 - V(\rho)
This becomes the source term in the Friedmann equations.
3. DVFT Friedmann Equations
The spacetime metric in a homogeneous universe is the FLRW form:
ds^2 = -dt^2 + a(t)^2 [ dr^2/(1-kr^2) + r^2 d\Omega^2 ]
In DVFT, the Friedmann equations become:
(da/dt)^2 / a^2 = (8\piG/3) \varepsilon_vac
d^2a/dt^2 / a = -(4\piG/3)(\varepsilon_vac + 3p_vac)
International Journal for Multidisciplinary Research (IJFMR)
E-ISSN: 2582-2160 ● Website: www.ijfmr.com ● Email: editor@ijfmr.com
IJFMR250664112 Volume 7, Issue 6, November-December 2025 29
The evolution of \rho(t) and \theta(t) determines \varepsilon_vac and p_vac.
Because the vacuum cannot diverge, \varepsilon_vac remains finite even at the earliest times.
4. Pre-Big-Bang Vacuum Phase
Before the Big Bang, the vacuum field was in a near-zero amplitude state:
• \rho(t) \approx 0
• \theta(t) undefined or fluctuating
This state is energetically unstable. The vacuum potential:
V(\rho) = \lambda (\rho^2 - \rho_0^2)^2
encourages a phase transition toward the minimum at \rho = \rho_0.
5. The Vacuum Phase Transition (Big Bang Event)
The Big Bang corresponds to the moment when the vacuum transitioned from the unstable state \rho \approx 0 to
the stable dynamic vacuum field state \rho = \rho_0. This transition releases energy, sets \theta(t) into coherent
oscillation, and generates an explosive increase in \varepsilon_vac.
This triggers rapid expansion of the scale factor a(t).
6. Inflation from Dynamics
Inflation requires rapid acceleration of the universe. DVFT provides this because the vacuum-potential
plateau makes V(\rho) nearly constant during the early evolution.
During the transition:
\varepsilon_vac \approx constant
Thus:
(da/dt)/a \approx constant \Rightarrow exponential expansion
DVFT inflation ends naturally when \rho(t) settles near \rho_0 and \theta(t) becomes coherent.
7. Reheating and Matter Creation
Once the vacuum field settles into coherent dynamic vacuum field, oscillations of \Phi transfer energy into
matter fields via interaction terms of the form:
L_int = -y |\Phi| ψ̄ψ
This generates particle–antiparticle pairs, radiation, and thermal energy. The universe becomes radiation
dominated.
8. Origin of Space Expansion
In GR, space expands, but no mechanism explains *why*. In DVFT, space expands because the vacuum
amplitude \rho(t) increases and the dynamic vacuum field becomes coherent. Vacuum energy determines
curvature, and a rapid change in vacuum energy produces rapid change in the scale factor.
9. Removal of the Cosmological Singularity
The divergence of curvature in GR arises because nothing limits density or curvature.
In DVFT, dynamics impose:
• |d\theta/dt| \leq \theta_max
• \rho(t) finite
• V(\rho) finite
• \varepsilon_vac finite
The energy density never diverges. The curvature invariants remain finite. The Big Bang is replaced by a
finite, smooth vacuum phase transition. There is no singular point.
10. Horizon Problem Resolved
International Journal for Multidisciplinary Research (IJFMR)
E-ISSN: 2582-2160 ● Website: www.ijfmr.com ● Email: editor@ijfmr.com
IJFMR250664112 Volume 7, Issue 6, November-December 2025 30
The classical horizon problem asks why causally disconnected regions of the sky have the same
temperature.
In DVFT:
• Before the Big Bang, the vacuum was nearly homogeneous
• The vacuum phase transition occurred everywhere simultaneously
• Vacuum-phase waves propagate at c, enforcing coherence
No superluminal mechanisms needed.
11. Flatness Problem Resolved
The vacuum phase transition drives rapid inflation, which smooths curvature.
This pushes the universe toward k = 0.
Thus flatness arises automatically.
12. What Caused the Universe to Begin?
In DVFT, the universe begins because the vacuum was unstable in its low-amplitude configuration. When
\rho reached the critical threshold, the vacuum rolled down its potential to \rho_0, initiating dynamic vacuum
field and expansion. This is analogous to phase transitions in condensed-matter systems.
13. What Expanded During the Big Bang?
• Not matter.
• Not energy.
• Not space as pure geometry.
What expanded was:
• the vacuum amplitude \rho(t).
As \rho(t) increased, vacuum energy increased, forcing the metric to inflate. This is the physical meaning
behind the expansion of space.
14. Dark Energy from Residual Dynamic vacuum field
Today, the vacuum still pulsates with frequency \mu. If \mu evolves slowly with time, or if the vacuum
amplitude slightly shifts, this yields a small, nearly constant vacuum energy density. This naturally
produces accelerated expansion of the universe without requiring a cosmological constant.
15. Full Evolution Summary
• Pre-Big-Bang: \rho \approx 0, incoherent vacuum
• Phase transition: \rho grows, \theta becomes coherent
• Inflation: V(\rho) nearly constant
• Reheating: \Phi couples to matter
• Radiation era
• Matter era
• Dark energy era: residual dynamic vacuum field
Conclusion
DVFT replaces the cosmological singularity with a physical vacuum-phase transition. It explains the origin
of the universe, inflation, expansion, dark energy, and smoothness of the cosmos using a single vacuum
field. This eliminates the inconsistencies of classical GR and provides a unified, microphysical picture of
cosmology.