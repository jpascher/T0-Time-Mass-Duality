\chapter{Kapitel 14}
\label{chap:14}







SPACE-CREATION SPEED AND THE COSMIC BOUNDARY
1. Introduction
In Dynamic vacuum field–Curvature Theory (DVFT), physical space exists only where the vacuum
amplitude \rho(x,t) is nonzero. Regions with \rho \approx 0 correspond to the primordial pure-phase (pre-space), which
has no geometry, no time, and no light-speed. When the universe ignited, \rho transitioned from 0 \to \rho$_0$,
creating the domain in which spacetime, matter, and physics could exist.
The radius of this activated domain is the true ‘cosmic boundary,’ and its growth defines the ‘speed of
space creation,’ given by the amplitude-front velocity:
v_b(t) = dR(t)/dt.
This appendix derives v_b(t) from DVFT field equations and shows how it yields observational scales
such as the \approx46.5 Gly cosmic horizon.
2. Fundamental DVFT Amplitude Equation
The DVFT vacuum field is:
\Phi(x,t) = \rho(x,t) e^{i\theta(x,t)}.
The amplitude \rho satisfies the Lagrangian:
𝓛_\rho = ½ A (\partialₜ\rho)$^2$ − ½ B (\nabla\rho)$^2$ − U(\rho),
leading to the Euler–Lagrange equation:
A \partialₜ$^2$\rho − B \nabla$^2$\rho + U'(\rho) = 0.
International Journal for Multidisciplinary Research (IJFMR)
E-ISSN: 2582-2160 ● Website: www.ijfmr.com ● Email: editor@ijfmr.com
IJFMR250664112 Volume 7, Issue 6, November-December 2025 34
This is a local, second-order, hyperbolic partial differential equation. Therefore, all disturbances or fronts
in \rho propagate with finite characteristic speed. This is the fundamental reason DVFT forbids infinite
‘space-creation speed.’
3. Definition of the Space–Nonspace Boundary
In DVFT:
• Space exists where \rho(x,t) > 0.
• Pre-space (non-space) exists where \rho(x,t) = 0.
The boundary R(t) is defined implicitly by:
\rho(R(t), t) = \rho_crit \approx 0.
The speed of ‘space creation’ is:
v_b(t) = dR(t)/dt.
It measures how fast the amplitude front propagates into the primordial pure-phase region.
4. Planar Traveling-Front Derivation of Finite Boundary Speed
Consider a planar front:
\rho(x,t) = f(\xi), \$xi = x$ − v_b t.
Insert into the amplitude equation:
A v_b$^2$ f''(\xi) − B f''(\xi) + U'(f(\xi)) = 0.
Multiply by f'(\xi) and integrate:
(A v_b$^2$ − B) ½ f'^2 + U(f) = C.
Assuming U(0) = U(\rho$_0$) = 0 (degenerate vacua) and front connecting \rho$_0$ \to 0, boundary conditions require
$C = 0$, so:
(A v_b$^2$ − B) ½ f'^2 + U(f) = 0.
Since U(f) \geq 0, a nontrivial front requires:
A v_b$^2$ − B < 0,
or:
v_b < sqrt(B/A) ≡ c_\rho.
Thus **DVFT predicts a finite upper bound on space-creation speed**:
v_b(t) \leq c_\rho,
where c_\rho = \sqrt(B/A) is the amplitude signal speed.
5. Spherical Boundary in an Expanding Universe
In spherical symmetry with cosmological expansion a(t), the amplitude equation becomes:
A(\partialₜ$^2$\rho + 3H\partialₜ\rho) − B(\partialᵣ$^2$\rho + 2\partialᵣ\rho/r) + U'(\rho) = 0,
where H = ȧ/a.
In a thin-front approximation \rho(r,t) \approx f(r − R(t)), the evolution of R(t) obeys:
\sigma R¨ + 3H \sigma R˙ + (2\sigma / R) = ΔU,
where:
• \sigma is surface tension of the amplitude front,
• ΔU = U(0) − U(\rho$_0$) is the vacuum-energy difference driving expansion.
Dividing by \sigma gives the effective boundary equation:
R¨ + 3H R˙ + 2/R = ΔU/\sigma.
This determines the actual physical space-creation speed v_b(t) = R˙(t).
6. Why the Space-Creation Speed Is Not Infinite
International Journal for Multidisciplinary Research (IJFMR)
E-ISSN: 2582-2160 ● Website: www.ijfmr.com ● Email: editor@ijfmr.com
IJFMR250664112 Volume 7, Issue 6, November-December 2025 35
The amplitude-front speed is finite because:
1. DVFT uses a local field equation; local PDEs forbid instantaneous global change.
2. The driving potential gradient |U'(\rho)| is finite.
3. Energy conservation limits how fast \rho can rise from 0 \to \rho$_0$.
4. The characteristic vacuum signal speed is c_\rho = \sqrt(B/A), bounding v_b.
Thus DVFT naturally rejects infinite expansion speeds without invoking relativity. Relativity (and light
speed c) only applies *inside* the \rho > 0 activated domain.
7. Relation to Observational Horizon Size
The comoving radius of the observable universe is:
R_obs \approx 46.5 Gly.
A naive ratio gives:
R_obs / (c t_age) \approx 46.5 / 13.8 \approx 3.36.
This does **not** mean the boundary moved at 3.36 c.
Rather, DVFT predicts:
• The front moves at v_b(t) \leq c_\rho ~ c.
• The interior region expands with scale factor a(t).
The observed comoving radius is:
R_com(t$_0$) = a(t$_0$) \int$_0$^{t$_0$} [v_b(t) / a(t)] dt.
Metric expansion stretches distances so that the final comoving radius corresponds to an ‘effective average
speed’ greater than c *without violating relativity*, since no signals propagate faster than c within space.
8. DVFT Prediction and Observational Fit
DVFT predicts:
• A finite space-creation speed v_b(t), controlled by vacuum micro-constants A, B and potential
shape U(\rho).
• The cosmic horizon size (~46.5 Gly) arises from the combined effect of v_b(t) \leq c_\rho and
cosmological scale-factor stretching.
Thus the theory *can be fitted to observational results* by constraining:
ΔU/\sigma, B/A, and the shape of U(\rho).
This makes DVFT testable against horizon scale, CMB structure, and early-universe expansion histories.
Conclusion
• Space creation corresponds to the outward propagation of the vacuum amplitude \rho.
• The boundary speed v_b(t) is finite because the amplitude field obeys a hyperbolic PDE.
• The maximal speed is the vacuum amplitude signal speed c_\rho = \sqrt(B/A).
• Cosmological expansion amplifies R(t) \to ~46.5 Gly today.
• The observed effective 3.36c ratio is not a physical propagation speed but a cumulative result of
front evolution + metric expansion.
DVFT therefore provides a complete, physically grounded mechanism for the finite but super-horizon
expansion of space.