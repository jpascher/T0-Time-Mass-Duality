\chapter{Kapitel 36}
\label{chap:36}







WHY QFT NEVER BECAME A THEORY OF GRAVITY
1. Introduction
Quantum Field Theory (QFT) contains nearly all the mathematical ingredients needed to develop Dynamic
vacuum field–Curvature Theory (DVFT): amplitude, phase, vacuum expectation values, field propagation,
and even vacuum instability. Yet QFT never evolved into a theory of gravity, and the physics community
resorted instead to geometric General Relativity (GR), which remains incompatible with quantum theory.
This chapter explains in detail why QFT never became a vacuum-curvature theory, how historical biases
prevented scientists from interpreting the vacuum correctly, and how DVFT completes the conceptual
unification that QFT mathematically hinted at for decades.
2. QFT Already Contains DVFT’s Mathematical Structure
QFT expresses every complex field in the form:
\Phi = \rho e^{i\theta},
where:
• \rho = amplitude of the field,
• \theta = phase of the field.
This decomposition is identical to the foundation of DVFT. In DVFT:
• \rho becomes vacuum amplitude (origin of inertia, curvature, gravity, mass),
• \theta becomes vacuum phase (origin of propagation, coherence, time).
Thus, the seeds of DVFT were fully present in QFT formalism. What was missing was the interpretation:
the recognition that \rho and \theta describe the physical vacuum, not just mathematical field components.
3. Why Physicists Rejected Physical Vacuum Models
After the failure of the 19th-century luminiferous aether, physicists became allergic to the idea of a
physical vacuum. Einstein’s formulation of relativity removed the need for a medium, and the scientific
community treated this as a philosophical victory.
This created an ideological barrier: "There must be no vacuum medium."
International Journal for Multidisciplinary Research (IJFMR)
E-ISSN: 2582-2160 ● Website: www.ijfmr.com ● Email: editor@ijfmr.com
IJFMR250664112 Volume 7, Issue 6, November-December 2025 79
As a result:
• QFT’s vacuum amplitude \rho was treated as mathematical,
• QFT’s vacuum phase \theta was treated as gauge redundancy,
• and the vacuum was mistakenly considered "empty."
4. GR Disconnected Gravity from Vacuum Structure
General Relativity treats gravity as pure geometry:
"mass-energy tells spacetime how to curve."
But GR doesn’t define what spacetime is. It provides equations but no physical substrate.
This made physicists believe gravity has no medium, no field, and no underlying physical structure. Thus,
when QFT emerged:
• QFT = fields in empty space,
• GR = curvature of empty geometry.
With two incompatible pictures, no one thought to ask:
"What if gravity is the vacuum’s amplitude response?"
DVFT answers exactly that.
5. The Higgs Mechanism Almost Revealed DVFT
The Higgs field demonstrated that:
• the vacuum has a nonzero amplitude (\rho★),
• particle masses arise from vacuum interaction,
• vacuum amplitude determines inertial properties.
This should have triggered the insight:
"Vacuum amplitude controls inertia \to inertia is gravity \to gravity is vacuum curvature."
But instead, physicists treated the Higgs field as just one field among many—not the universal physical
substrate.
6. The Fundamental Conceptual Error: Quantizing Geometry
To unify gravity with QFT, scientists attempted to quantize GR’s geometric curvature:
• string theory,
• loop quantum gravity,
• spin foams.
Every attempt failed because:
you cannot quantize geometry if geometry is not fundamental.
DVFT avoids this mistake. It says:
• geometry is emergent,
• vacuum amplitude \rho is fundamental,
• curvature is \nabla\rho,
• gravity is amplitude dynamics, not metric structure.
7. Why QFT Never Interpreted \theta as Time
QFT treats the phase of a field (\theta) as gauge freedom — something to remove, not interpret. But DVFT
identifies:
• \thetaₜ \to time evolution,
• \theta propagation \to speed of light,
• pure \theta-waves \to photons.
International Journal for Multidisciplinary Research (IJFMR)
E-ISSN: 2582-2160 ● Website: www.ijfmr.com ● Email: editor@ijfmr.com
IJFMR250664112 Volume 7, Issue 6, November-December 2025 80
This single insight unifies:
• time,
• relativity,
• light propagation,
• electromagnetism.
Mainstream physics never noticed this because \theta was never considered a physical vacuum property. DVFT
positions \theta at the center of physical reality.
8. Why QFT Never Connected Amplitude to Curvature
DVFT identifies:
gravity = curvature of vacuum amplitude = \nabla\rho.
QFT already had amplitude \rho in every field. But because GR insisted gravity was geometry, no one thought
to reinterpret \rho as the origin of curvature.
The failure was conceptual, not mathematical. DVFT simply restores the physical meaning that QFT’s
formalism always contained.
9. DVFT as the Completion of QFT and GR
DVFT completes modern physics by interpreting the vacuum as a physical medium with:
• amplitude (\rho) determining inertia, curvature, mass,
• phase (\theta) determining time, coherence, and light propagation.
Because of this, DVFT:
• unifies gravity with field theory,
• explains relativity from dynamics,
• derives c from vacuum parameters,
• explains mass without ad hoc Higgs interpretation,
• explains quantum collapse as amplitude-phase selection,
• explains cosmic expansion as amplitude activation.
QFT could not do this because it lacked the missing physical interpretation: the vacuum is real.
Conclusion
QFT had all the mathematical structure needed to lead to DVFT, but it failed because:
• the vacuum was treated as empty,
• GR disconnected gravity from field physics,
• physicists rejected vacuum-medium ideas,
• the phase \theta was never interpreted physically,
• attempts to quantize geometry distracted from the real foundation.
DVFT restores the missing ontology, showing that:
• \rho is vacuum curvature,
• \theta is vacuum time-phase,
• c = \sqrt(K_0 / \rho_0) arises naturally,
• gravity is amplitude dynamics,
• photons are pure phase waves,
• matter is amplitude-phase knots.
Thus, DVFT is not an alternative to QFT—it is its physical completion. It reveals the true nature of the
vacuum that QFT always described mathematically but never recognized physically.
International Journal for Multidisciplinary Research (IJFMR)
E-ISSN: 2582-2160 ● Website: www.ijfmr.com ● Email: editor@ijfmr.com
IJFMR250664112 Volume 7, Issue 6, November-December 2025 81