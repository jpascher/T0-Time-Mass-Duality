\chapter{Kapitel 08}
\label{chap:08}







GALAXY ROTATION CURVES AND MISSING MASS PROBLEM
Modern astrophysics and cosmology face numerous unresolved problems that General Relativity (GR)
and the ΛCDM model cannot fully explain without invoking dark matter particles, fine-tuned inflation
fields, unexplained singularities, or an arbitrary cosmological constant. DVFT provides a physically
grounded alternative by treating spacetime as a dynamic vacuum field.
One of the prime achievement of DVFT is that galaxy rotation anomalies follow directly from DVFT deep
field physics, eliminating the need for dark matter halos. Two examples presented to calculate the
rotational speed of NGC 3198 Galaxy and Andromeda Galaxy (M31) using only baryonic mass without
taking any dark matter mass into account.
DVFT defines the vacuum field as \Phi = \rho e^{i\theta}. In the weak-field, low-acceleration outer regions of
galaxies where observed rotation curves deviate from Newtonian predictions, DVFT predicts a nonlinear
International Journal for Multidisciplinary Research (IJFMR)
E-ISSN: 2582-2160 ● Website: www.ijfmr.com ● Email: editor@ijfmr.com
IJFMR250664112 Volume 7, Issue 6, November-December 2025 19
vacuum response based on deep field equations derived from vacuum Lagrangian gives the baryonic
Tully–Fisher relation:
v_c^4 = G M_b a_0
Where, v_c is circular speed, M_b is Baryonic mass and G is Newton’s Gravitational Constant
These equations are derived from the basic DVFT equation \Phi = \rho e^{i\theta} and the vacuum Lagrangian.
Complete derivation of this equation has been given below.
1. DVFT Vacuum Lagrangian and \Phi = \rho e^{i\theta}
Start with a minimal DVFT vacuum Lagrangian:
𝓛 = ½ A |\partialₜ\Phi|^2 − ½ B(\rho) |\nabla\Phi|^2 − U(\rho) − \rho_b φ(\rho,\theta),
where:
• A is vacuum temporal inertia,
• B(\rho) is vacuum spatial stiffness,
• U(\rho) is the vacuum amplitude potential,
• \rho_b is baryonic matter density,
• φ is the gravitational potential encoded in \theta.
Substitute \Phi = \rho e^{i\theta}:
• |\partialₜ\Phi|^2 = (\partialₜ\rho)^2 + \rho^2(\partialₜ\theta)^2
• |\nabla\Phi|^2 = |\nabla\rho|^2 + \rho^2|\nabla\theta|^2
Thus:
𝓛 = ½A[(\partialₜ\rho)^2 + \rho^2(\partialₜ\theta)^2] − ½B(\rho)[|\nabla\rho|^2 + \rho^2|\nabla\theta|^2] − U(\rho) − \rho_b φ.
2. Static Nonrelativistic Limit
For galaxy rotation curves, time derivatives are negligible:
• \partialₜ\rho \approx 0,
• \partialₜ\theta \approx constant (background vacuum oscillation).
DVFT identifies gravitational potential φ through phase evolution:
\partialₜ\theta = \omega_0(1 + φ/c^2) \Rightarrow \nabla\theta = (\omega_0/c^2) \nablaφ.
Thus, the vacuum energy density becomes:
ℰ_vac \approx ½ K(\rho) |\nablaφ|^2 + U(\rho),
where K(\rho) = B(\rho) \rho^2 (\omega_0^2 / c^4).
This shows that gravitational behavior arises from spatial variations of φ, mediated by vacuum amplitude
\rho.
3. Integrating Out the Vacuum Amplitude \rho
At equilibrium (static galaxies), \rho adjusts to minimize local vacuum energy:
\partial/\partial\rho [½K(\rho)|\nablaφ|^2 + U(\rho)] = 0.
This yields an algebraic relation:
½ K'(\rho)|\nablaφ|^2 + U'(\rho) = 0.
In high-acceleration regimes, \rho \approx \rho_0 (the vacuum ground amplitude) and Newtonian gravity emerges.
In low-acceleration regimes, the vacuum becomes nearly coherent, U'(\rho) \to 0, allowing \rho to respond
strongly to |\nablaφ|.
Scale invariance of DVFT in this regime requires the vacuum energy to scale as:
ℰ ∝ |\nablaφ|^3.
This corresponds to a vacuum functional:
F(y) ∝ y^{3/2}, y = |\nablaφ|^2 / a_0^2.
International Journal for Multidisciplinary Research (IJFMR)
E-ISSN: 2582-2160 ● Website: www.ijfmr.com ● Email: editor@ijfmr.com
IJFMR250664112 Volume 7, Issue 6, November-December 2025 20
4. Deep-Field Lagrangian
In the deep-field regime (g ≪ a_0), the vacuum Lagrangian becomes:
𝓛_eff = − (a_0^2/8\piG) F(|\nablaφ|^2/a_0^2) − \rho_b φ,
with:
F(y) = (2/3) y^{3/2}.
Varying this with respect to φ yields the field equation:
\nabla\cdot[(|\nablaφ|/a_0) \nablaφ] = 4\piG \rho_b.
Define gravitational acceleration g = |\nablaφ|; then:
\nabla\cdot[(g/a_0) ĝ g] = 4\piG \rho_b.
5. Spherical Galaxy: Deriving g^2 = a_0 g_N
For a spherical mass distribution:
g(r) = |\nablaφ| = dφ/dr.
The DVFT deep-field equation becomes:
(1/r^2) d/dr (r^2 g^2 / a_0) = 4\piG \rho_b(r).
Integrate from 0 to r:
r^2 g^2 / a_0 = G M_b(r).
Solve for g:
g^2(r) = a_0 (G M_b(r)/r^2) = a_0 g_N(r).
This is exactly the DVFT deep-field force law:
g^2 = a_0 g_N.
6. Rotation Curves and Tully–Fisher Relation
The circular velocity satisfies:
g(r) = v_c^2(r)/r.
Insert into g^2 = a_0 g_N:
(v_c^2/r)^2 = a_0 (G M_b / r^2).
Simplify:
v_c^4(r) = G M_b(r) a_0.
In the flat part of the rotation curve, M_b(r) \to constant = M_b, giving the baryonic Tully–Fisher relation
:
v_c^4 = G M_b a_0,
7. Physical Meaning in DVFT
In DVFT:
• amplitude \rho determines inertia and curvature,
• phase \theta determines wave propagation and time,
• gravity arises from phase-time distortions governed by nonlinear vacuum response.
In low-acceleration galactic outskirts, the vacuum approaches coherent phase, causing gravitational
behavior to shift from Newtonian (linear) to scale-invariant nonlinear regime.
This reproduces:
• flat rotation curves,
• g^2 = a_0 g_N,
• the baryonic Tully–Fisher law,
• all without dark matter.
8. Summary
International Journal for Multidisciplinary Research (IJFMR)
E-ISSN: 2582-2160 ● Website: www.ijfmr.com ● Email: editor@ijfmr.com
IJFMR250664112 Volume 7, Issue 6, November-December 2025 21
Starting from the fundamental DVFT field \Phi = \rho e^{i\theta}, we derived:
• an effective vacuum energy ∝ |\nablaφ|^3,
• the deep-field equation \nabla\cdot[(g/a_0) g] = 4\piG\rho_b,
• the spherical solution g^2 = a_0 g_N,
• and the baryonic Tully–Fisher relation v_c^4 = G M_b a_0.
Thus, galaxy rotation anomalies follow directly from DVFT vacuum physics, eliminating the need for
dark matter halos.
Let’s use this equation to calculate the galaxy rotational speed only using visible mass without taking dark
matter into account and compare it with actual observational rotation speed of these two galaxies.
9. NGC 3198 Galaxy
Rotation curve: nearly flat at v \approx 150 km/s beyond r ≳ 20 kpc.
Stellar mass from BTFR / photometric fits: total baryonic mass M_b \approx 2.46 \times 10^1^0 M_⊙.
Rotation Speed using baryonic Tully–Fisher relation v_c^4 = G M_b a_0 with a_0 = 1.2\times10⁻^1^0 m/s^2:
v_c \approx 141 km/s.
Interpretation: DVFT prediction close to the observed 150 km/s without dark matter.
10. Andromeda Galaxy
Rotation curve: nearly flat at v \approx 220 – 226 km/s between 20 -35 kpc
Total baryonic mass: \approx 1.6\times10^1^1 M_⊙ (Stars + Gas)
Rotation Speed using baryonic Tully–Fisher relation v_c^4 = G M_b a_0 with a_0 = 1.2\times10⁻^1^0 m/s^2
v_c \approx 220 km/s.
Interpretation: DVFT prediction close to the observed 220 - 226 km/s without dark matter.
Conclusion
Both NGC 3198 and Andromeda Galaxies behaves exactly as predicted by DVFT deep field equation
gives a flat rotation curve set directly by baryonic mass, with no requirement for dark matter.
DVFT provides gravitational equations which eliminates requirement of dark matter in cosmological
calculations.