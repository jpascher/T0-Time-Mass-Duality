\documentclass[12pt,a4paper]{article}
\usepackage[utf8]{inputenc}
\usepackage[T1]{fontenc}
\usepackage[ngerman]{babel}
\usepackage{amsmath,amsfonts,amssymb}
\usepackage{physics}
\usepackage{geometry}
\usepackage{hyperref}
\usepackage{fancyhdr}
\usepackage{graphicx}
\usepackage{cite}

\geometry{margin=1in}
\pagestyle{fancy}
\fancyhf{}
\fancyhead[C]{Dynamische Vakuumfeldtheorie}
\fancyfoot[C]{\thepage}

\title{Dynamische Vakuumfeldtheorie}
\author{Satish B. Thorwe, MSc\\Robert Gordon University, Aberdeen UK}
\date{}

\begin{document}

\maketitle

\begin{abstract}
Diese Arbeit präsentiert ein einheitliches theoretisches Modell, in dem die Raumzeit-Krümmung aus Verzerrungen in einem dynamischen Vakuumfeld entsteht, das durch ein komplexes Skalarfeld $\phi(x)=\rho(x)e^{i\theta(x)}$ beschrieben wird, wobei $\phi(x)$ das dynamische Vakuumfeld, $\rho(x)$ die Vakuumamplitude und $\theta(x)$ die Vakuumphase ist. Das Vakuum besitzt ein intrinsisches Feld, dessen Phase sich linear mit der Zeit entwickelt, und Materie stört es lokal. Diese Störungen breiten sich mit Lichtgeschwindigkeit aus und erzeugen Stress-Energie, die die Raumzeit durch Einsteins Feldgleichungen krümmt. Das Modell liefert eine physikalische und kausale Erklärung für Krümmung über Distanz und dient als Brücke zwischen Quantenmechanik und klassischer Allgemeiner Relativitätstheorie. Ein vollständiges mathematisches Rahmenwerk für die Dynamische Vakuumfeldtheorie (DVFT) wird mit ihren Anwendungen in Kosmologie und Quantenmechanik präsentiert.
\end{abstract}

\tableofcontents
\newpage

\section{Hinweis}
Die vollständige deutsche Übersetzung wird maschinell erstellt und muss manuell überprüft werden.

\section{Einleitung}
DVFT-Dokument - Automatische Konvertierung in Arbeit.


\section{References}
\label{sec:references}

References will be added based on citations in the full document.

\end{document}
