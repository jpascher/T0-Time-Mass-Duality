\documentclass[a4paper,11pt]{book}
\usepackage[utf8]{inputenc}
\usepackage[T1]{fontenc}
\usepackage[ngerman]{babel}
\usepackage{graphicx}
\usepackage{amsmath,amssymb,amsthm}
\usepackage{hyperref}
\usepackage{geometry}
\usepackage{fancyhdr}
\usepackage{booktabs}
\usepackage{longtable}
\usepackage{array}
\usepackage{tcolorbox}
\tcbuselibrary{breakable,skins}
\usepackage{xcolor}
\usepackage{tikz}

\setlength{\headheight}{14pt}

% Abstract environment
\newenvironment{abstract}{\section*{Zusammenfassung}}{}

% Colors
\definecolor{t0blue}{RGB}{33,150,243}
\definecolor{gold}{RGB}{255,215,0}

% T0 specific commands
\newcommand{\Tzero}{T_0}
\newcommand{\betaT}{\beta_T}
\newcommand{\xipar}{\xi}
\providecommand{\Kfrak}{K_{\text{frak}}}
\providecommand{\hbar}{\hslash}
\providecommand{\kB}{k_B}
\providecommand{\Tfield}{T}
\providecommand{\alphaEM}{\alpha}
\providecommand{\Lp}{l_P}
\providecommand{\Tp}{t_P}
\providecommand{\Mp}{M_P}

% tcolorbox environments
\newtcolorbox{keyresult}[1][Kernaussage]{colback=blue!5,colframe=blue!75!black,title=#1,breakable}
\newtcolorbox{foundation}[1][Grundlage]{colback=green!5,colframe=green!75!black,title=#1,breakable}
\newtcolorbox{alternative}[1][Alternative]{colback=orange!5,colframe=orange!75!black,title=#1,breakable}
\newtcolorbox{summary}[1][Zusammenfassung]{colback=gray!5,colframe=gray!75!black,title=#1,breakable}
\newtcolorbox{important}[1][Wichtig]{colback=red!5,colframe=red!75!black,title=#1,breakable}

\geometry{margin=2.5cm}
\pagestyle{fancy}

\title{\Huge\textbf{T0-Theorie: Zeit-Masse-Dualität}\\[1cm]\Large Alle Naturkonstanten aus einer Zahl: $\alpha \approx 1/137$}
\author{Johann Pascher}
\date{2024}

\begin{document}

% Cover page with image
\begin{titlepage}
\centering
\includegraphics[width=\textwidth,height=\textheight,keepaspectratio]{T0_deckblatt_De.png}
\end{titlepage}

\frontmatter
\tableofcontents

\mainmatter

\chapter{Einführung in die T0-Theorie}

Die T0-Theorie ist ein neuer Ansatz zur Vereinheitlichung der fundamentalen Physik. Die zentrale These lautet:

\begin{keyresult}[Zentrales Theorem]
Alle Naturkonstanten und physikalischen Parameter können aus einer einzigen dimensionslosen Zahl abgeleitet werden: der Feinstrukturkonstante $\alpha \approx 1/137$.
\end{keyresult}

\section{Zeit-Masse-Dualität}

Das Kernprinzip der T0-Theorie ist die Zeit-Masse-Dualität:
\begin{equation}
T(x) = \frac{\hbar}{E(x)} = \frac{\hbar}{m(x)c^2}
\end{equation}

Diese Beziehung zeigt, dass Zeit und Masse intrinsisch verknüpft sind.

\begin{foundation}[Grundprinzip]
In Regionen mit höherer Energiedichte verläuft die intrinsische Zeit langsamer - genau wie es die Allgemeine Relativitätstheorie für die Gravitation vorhersagt.
\end{foundation}

\section{Der Skalierungsparameter $\xi$}

Der dimensionslose Skalierungsparameter $\xi$ verbindet alle Naturkonstanten:
\begin{equation}
\xi = \frac{4}{3} \times 10^{-4} \approx \sqrt{\alpha}
\end{equation}

\chapter{Teilchenmassen und fundamentale Konstanten}

Die Massen aller Elementarteilchen können aus dem Skalierungsparameter $\xi$ und der Planck-Masse abgeleitet werden.

\section{Leptonenmassen}

Die Koide-Formel findet ihre natürliche Erklärung im T0-Framework:
\begin{equation}
\frac{m_e + m_\mu + m_\tau}{(\sqrt{m_e} + \sqrt{m_\mu} + \sqrt{m_\tau})^2} = \frac{2}{3}
\end{equation}

\chapter{Kosmologische Implikationen}

\section{Die Hubble-Konstante}

Die T0-Theorie liefert eine geometrische Herleitung der Hubble-Konstante:
\begin{equation}
H_0 \approx \frac{c}{\xi \cdot L_P} \cdot \alpha^2
\end{equation}

\section{Dunkle Energie}

Die kosmologische Konstante wird als Konsequenz der intrinsischen Zeit erklärt.

\chapter{Experimentelle Vorhersagen}

\begin{summary}[Testbare Vorhersagen]
\begin{itemize}
\item Anomales magnetisches Moment des Elektrons: $(g-2)_e$
\item Koide-Formel Erweiterungen für Quarks
\item Frequenzunabhängige Effekte
\end{itemize}
\end{summary}

\backmatter

\chapter*{Anhang: Formelsammlung}
\addcontentsline{toc}{chapter}{Anhang: Formelsammlung}

Hier sind die wichtigsten Formeln der T0-Theorie zusammengefasst.

\end{document}

