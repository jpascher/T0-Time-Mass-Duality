\chapter{\textbf{T0-Theorie: Lagrangian-Formulierung}\\[0.5cm]
	\large Zeit-Masse-Dualität und Feldtheoretische Grundlagen\\[0.3cm]
	\normalsize Dokument der T0-Serie}

	
	
\section*{Abstract}
		Dieses Dokument präsentiert die vollständige Lagrangian-Formulierung der T0-Theorie basierend auf dem fundamentalen geometrischen Parameter $\xi = \frac{4}{3} \times 10^{-4}$. Die Theorie etabliert eine fundamentale Zeit-Masse-Dualität $T(x,t) \cdot m(x,t) = 1$ und entwickelt einen erweiterten Lagrangian mit massenproportionaler Kopplung an ein dynamisches Zeitfeld. Für die geometrische Herleitung der anomalen magnetischen Momente und experimentelle Vorhersagen siehe Dokument 018\_T0\_Anomale-g2-10\_De.tex. Der Fokus liegt hier auf der feldtheoretischen Struktur und der Ableitung der fundamentalen T0-Beitragsformel aus ersten Prinzipien.

	
	
	\section{Einführung}
	
	\subsection{Grundprinzipien der T0-Theorie}
	
	Die T0-Theorie postuliert eine fundamentale Dualität zwischen Zeit und Masse:
	\begin{equation}
		T(x,t) \cdot m(x,t) = 1
		\label{eq:time_mass_duality}
	\end{equation}
	
	Diese Dualität führt zu mehreren fundamentalen Konsequenzen:
	\begin{itemize}
		\item Natürliche Massenhierarchie durch Zeitskalen
		\item Dynamische Massenerzeugung durch das Zeitfeld
		\item Quadratische Skalierung der anomalen magnetischen Momente mit $m_\ell^2$
		\item Intrinsische Integration der Gravitation in die Quantenfeldtheorie
	\end{itemize}
	
	\subsection{Der fundamentale geometrische Parameter}
	
	Die gesamte T0-Theorie basiert auf einem einzigen fundamentalen Parameter:
	\begin{equation}
		\boxed{\xi = \frac{4}{3} \times 10^{-4} = 1{,}333 \times 10^{-4}}
		\label{eq:xi_fundamental}
	\end{equation}
	
	Dieser dimensionslose Parameter kodiert die fundamentale geometrische Struktur des dreidimensionalen Raums. Für die detaillierte geometrische Interpretation und Herleitung siehe \href{https://github.com/jpascher/T0-Time-Mass-Duality/blob/main/2/pdf/018_T0_Anomale-g2-10_De.pdf}{Dokument 018}.
	
	\subsection{Notation und Konventionen}
	
	Wir verwenden natürliche Einheiten ($\hbar = c = 1$) mit Metriksignatur $(+,-,-,-)$:
	
	\begin{itemize}
		\item $T(x,t)$: Dynamisches Zeitfeld mit $[T] = E^{-1}$
		\item $\delta E(x,t)$: Fundamentales Energiefeld mit $[\delta E] = E$
		\item $\Delta m(x,t) = \delta E(x,t)$: Massenfeldfluktuationen
		\item $\xi = 1{,}333 \times 10^{-4}$: Fundamentaler geometrischer Parameter
		\item $\lambda$: Higgs-Zeitfeld-Kopplungsparameter
		\item $m_\ell$: Leptonenmassen ($e$, $\mu$, $\tau$)
	\end{itemize}
	
	\subsection{Abgeleitete Parameter}
	
	Aus dem fundamentalen Parameter $\xi$ ergeben sich:
	\begin{align}
		\xi^2 &= \left(1{,}333 \times 10^{-4}\right)^2 = 1{,}777 \times 10^{-8} \label{eq:xi_squared} \\
		\xi^4 &= \left(1{,}333 \times 10^{-4}\right)^4 = 3{,}160 \times 10^{-16} \label{eq:xi_fourth}
	\end{align}
	
	\section{Erweiterter Lagrangian mit Zeitfeld}
	
	\subsection{Standardmodell-Lagrangian als Ausgangspunkt}
	
	Der Standardmodell-Lagrangian für ein Lepton $\psi_\ell$ lautet:
	\begin{equation}
		\mathcal{L}_{\text{SM}} = -\frac{1}{4} F_{\mu\nu}F^{\mu\nu} + \bar{\psi}_\ell(i\gamma^\mu D_\mu - m_\ell)\psi_\ell
		\label{eq:sm_lagrangian}
	\end{equation}
	
	Dabei ist:
	\begin{itemize}
		\item $F_{\mu\nu} = \partial_\mu A_\nu - \partial_\nu A_\mu$: Elektromagnetischer Feldstärketensor
		\item $D_\mu = \partial_\mu + ie A_\mu$: Kovariante Ableitung
		\item $m_\ell$: Leptonmasse (konstant)
	\end{itemize}
	
	\subsection{Einführung des dynamischen Zeitfeldes}
	
	In der T0-Theorie ist die Masse nicht konstant, sondern über die Zeit-Masse-Dualität \eqref{eq:time_mass_duality} an ein dynamisches Zeitfeld $T(x,t)$ gekoppelt. Wir führen das Massenfeld ein:
	\begin{equation}
		m(x,t) = m_0 + \Delta m(x,t)
		\label{eq:mass_field}
	\end{equation}
	
	wobei $m_0$ die Ruhemasse und $\Delta m(x,t)$ die dynamische Massenfluktuationen darstellt.
	
	\subsection{Kinetischer Term des Zeitfeldes}
	
	Das Zeitfeld selbst benötigt einen kinetischen Term:
	\begin{equation}
		\mathcal{L}_{\text{kin}}^{(T)} = \frac{1}{2}(\partial_\mu \Delta m)(\partial^\mu \Delta m)
		\label{eq:time_field_kinetic}
	\end{equation}
	
	Dieser Term beschreibt die Ausbreitung von Massenfeldfluktuationen als dynamische Freiheitsgrade.
	
	\subsection{Massenterm des Zeitfeldes}
	
	Das Zeitfeld hat eine charakteristische Masse $m_T$:
	\begin{equation}
		\mathcal{L}_{\text{mass}}^{(T)} = -\frac{1}{2} m_T^2 \Delta m^2
		\label{eq:time_field_mass}
	\end{equation}
	
	Die Zeitfeldmasse $m_T$ ist über die Higgs-Zeitfeld-Verbindung gegeben durch:
	\begin{equation}
		m_T = \frac{\lambda}{\xi}
		\label{eq:time_field_mass_value}
	\end{equation}
	
	wobei $\lambda$ der Higgs-Kopplungsparameter ist.
	
	\subsection{Massenproportionale Kopplung}
	
	Der fundamentale neue Term in der T0-Theorie ist die Kopplung der Leptonfelder an das Zeitfeld, proportional zur Leptonmasse:
	\begin{equation}
		\mathcal{L}_{\text{int}} = g_T^\ell \, \bar{\psi}_\ell \psi_\ell \, \Delta m
		\label{eq:interaction_term}
	\end{equation}
	
	Die Kopplungsstärke ist gegeben durch:
	\begin{equation}
		\boxed{g_T^\ell = \xi \, m_\ell}
		\label{eq:coupling_strength}
	\end{equation}
	
	Diese massenproportionale Kopplung ist das Herzstück der T0-Theorie. Sie impliziert:
	\begin{itemize}
		\item Schwerere Leptonen koppeln stärker an das Zeitfeld
		\item Die Kopplung skaliert linear mit der Masse
		\item Dies führt zu quadratischer Massenskalierung in Quantenkorrekturen
	\end{itemize}
	
	\subsection{Vollständiger erweiterter Lagrangian}
	
	Der vollständige T0-Lagrangian kombiniert alle Terme:
	\begin{equation}
		\begin{split}
			\mathcal{L}_{\text{T0}} = &-\frac{1}{4} F_{\mu\nu}F^{\mu\nu} \\
			&+ \bar{\psi}_\ell(i\gamma^\mu D_\mu - m_\ell)\psi_\ell \\
			&+ \frac{1}{2}(\partial_\mu \Delta m)(\partial^\mu \Delta m) \\
			&- \frac{1}{2} m_T^2 \Delta m^2 \\
			&+ \xi \, m_\ell \,\bar{\psi}_\ell \psi_\ell \, \Delta m
		\end{split}
		\label{eq:full_lagrangian}
	\end{equation}
	
	\section{Quantenkorrekturen und Feynman-Regeln}
	
	\subsection{Feynman-Regeln aus dem Lagrangian}
	
	Aus dem Wechselwirkungsterm \eqref{eq:interaction_term} ergeben sich die Feynman-Regeln:
	
	\textbf{Vertex-Faktor:}
	\begin{equation}
		\bar{\psi}_\ell \psi_\ell \Delta m \quad \longrightarrow \quad -i g_T^\ell = -i \xi m_\ell
		\label{eq:vertex_factor}
	\end{equation}
	
	\textbf{Zeitfeld-Propagator:}
	\begin{equation}
		\Delta m(k) \quad \longrightarrow \quad \frac{i}{k^2 - m_T^2 + i\epsilon}
		\label{eq:propagator}
	\end{equation}
	
	\subsection{Ein-Schleifen-Diagramm}
	
	Das fundamentale Ein-Schleifen-Diagramm für den anomalen magnetischen Moment-Beitrag hat die Struktur:
	\begin{center}
		\begin{tikzpicture}
			\begin{feynman}
				\vertex (a) {$\bar{\psi}_\ell$};
				\vertex [right=2cm of a] (b);
				\vertex [right=2cm of b] (c) {$\psi_\ell$};
				\vertex [above=1.5cm of b] (d);
				\diagram* {
					(a) -- [fermion] (b) -- [fermion] (c),
					(b) -- [fermion, half left] (d) -- [fermion, half left] (b),
					(d) -- [scalar, edge label=$\Delta m$] (d),
				};
			\end{feynman}
		\end{tikzpicture}
	\end{center}
	
	\subsection{Allgemeine Formel für skalare Mediatoren}
	
	Für einen skalaren Mediator mit Masse $m_T$ und Kopplung $g_T^\ell$ lautet die allgemeine Ein-Schleifen-Formel:
	\begin{equation}
		\begin{split}
			\Delta a_\ell = \frac{(g_T^\ell)^2}{8\pi^2} \int_0^1 dx \, 
			\frac{m_\ell^2 (1-x)(1-x^2)}{m_\ell^2 x^2 + m_T^2 (1-x)}
		\end{split}
		\label{eq:general_loop_formula}
	\end{equation}
	
	Diese Formel ist Standard in der Quantenfeldtheorie für skalare Beiträge zum anomalen magnetischen Moment.
	
	\section{Ableitung der fundamentalen T0-Formel}
	
	\subsection{Grenzfall schwerer Mediatoren}
	
	Für $m_T \gg m_\ell$ kann das Integral \eqref{eq:general_loop_formula} vereinfacht werden. Im Nenner dominiert der Term $m_T^2 (1-x)$:
	\begin{equation}
		\frac{m_\ell^2 (1-x)(1-x^2)}{m_\ell^2 x^2 + m_T^2 (1-x)} 
		\approx \frac{m_\ell^2 (1-x)(1-x^2)}{m_T^2 (1-x)} 
		= \frac{m_\ell^2 (1-x^2)}{m_T^2}
		\label{eq:heavy_mediator_approx}
	\end{equation}
	
	Damit wird:
	\begin{equation}
		\begin{split}
			\Delta a_\ell &\approx \frac{(g_T^\ell)^2}{8\pi^2 m_T^2} \int_0^1 dx \, (1-x^2) \\
			&= \frac{(g_T^\ell)^2}{8\pi^2 m_T^2} \left[x - \frac{x^3}{3}\right]_0^1 \\
			&= \frac{(g_T^\ell)^2}{8\pi^2 m_T^2} \left(1 - \frac{1}{3}\right) \\
			&= \frac{(g_T^\ell)^2}{8\pi^2 m_T^2} \cdot \frac{2}{3}
		\end{split}
		\label{eq:integral_evaluation}
	\end{equation}
	
	\subsection{Einsetzen der T0-Kopplung}
	
	Mit der massenproportionalen Kopplung \eqref{eq:coupling_strength} erhalten wir:
	\begin{equation}
		\begin{split}
			\Delta a_\ell &= \frac{(\xi m_\ell)^2}{8\pi^2 m_T^2} \cdot \frac{2}{3} \\
			&= \frac{2 \xi^2 m_\ell^2}{24\pi^2 m_T^2} \\
			&= \frac{\xi^2 m_\ell^2}{12\pi^2 m_T^2}
		\end{split}
		\label{eq:t0_coupling_inserted}
	\end{equation}
	
	\subsection{Higgs-Zeitfeld-Verbindung}
	
	Mit der Zeitfeldmasse \eqref{eq:time_field_mass_value} wird:
	\begin{equation}
		\begin{split}
			\Delta a_\ell &= \frac{\xi^2 m_\ell^2}{12\pi^2 (\lambda/\xi)^2} \\
			&= \frac{\xi^2 m_\ell^2}{12\pi^2} \cdot \frac{\xi^2}{\lambda^2} \\
			&= \frac{\xi^4 m_\ell^2}{12\pi^2 \lambda^2}
		\end{split}
		\label{eq:higgs_connection_substituted}
	\end{equation}
	
	\subsection{Korrektur durch vollständiges Integral}
	
	Die obige Rechnung verwendete eine Näherung. Das vollständige Integral für $m_T \gg m_\ell$ ergibt einen numerischen Faktor:
	\begin{equation}
		\int_0^1 dx \, (1-x)(1-x^2) = \int_0^1 dx \, (1-x-x^2+x^3) = \frac{5}{12}
		\label{eq:full_integral}
	\end{equation}
	
	Damit lautet die präzise Formel:
	\begin{equation}
		\boxed{\Delta a_\ell^{\text{(T0)}} = \frac{5\xi^4}{96\pi^2\lambda^2} \cdot m_\ell^2}
		\label{eq:t0_fundamental_formula}
	\end{equation}
	
	\section{Numerische Auswertung}
	
	\subsection{Bestimmung der Normierungskonstante}
	
	Aus dem fundamentalen Parameter $\xi = 1{,}333 \times 10^{-4}$ und den Higgs-Parametern ergibt sich die Normierungskonstante:
	\begin{equation}
		C_{\text{T0}} = \frac{5\xi^4}{96\pi^2\lambda^2}
		\label{eq:normalization_constant}
	\end{equation}
	
	Mit den Higgs-Werten:
	\begin{itemize}
		\item Higgs-Masse: $m_h = 125$ GeV
		\item Higgs-VEV: $v = 246$ GeV  
		\item Higgs-Selbstkopplung: $\lambda_h \approx 0{,}13$
	\end{itemize}
	
	und der Relation $\xi = \frac{\lambda_h^2 v^2}{16\pi^3 m_h^2}$ erhalten wir numerisch:
	\begin{equation}
		C_{\text{T0}} \approx 2{,}246 \times 10^{-13} \text{ GeV}^{-2}
		\label{eq:numerical_constant}
	\end{equation}
	
	\subsection{Finale T0-Beitragsformel}
	
	Die vollständig ausgewertete T0-Beitragsformel lautet:
	\begin{equation}
		\boxed{\Delta a_\ell^{\text{(T0)}} = 2{,}246 \times 10^{-13} \cdot m_\ell^2 \text{ [GeV}^{-2}\text{]}}
		\label{eq:final_t0_formula}
	\end{equation}
	
	wobei $m_\ell$ in GeV einzusetzen ist.
	
	\subsection{Leptonspezifische Vorhersagen}
	
	Mit den Leptonmassen:
	\begin{itemize}
		\item $m_e = 0{,}511$ MeV $= 0{,}000511$ GeV
		\item $m_\mu = 105{,}658$ MeV $= 0{,}105658$ GeV
		\item $m_\tau = 1776{,}86$ MeV $= 1{,}77686$ GeV
	\end{itemize}
	
	ergeben sich die T0-Beiträge:
	
	\textbf{Elektron:}
	\begin{equation}
		\begin{split}
			\Delta a_e^{\text{(T0)}} &= 2{,}246 \times 10^{-13} \cdot (0{,}000511)^2 \\
			&= 2{,}246 \times 10^{-13} \cdot 2{,}611 \times 10^{-7} \\
			&= 5{,}86 \times 10^{-20}
		\end{split}
		\label{eq:electron_prediction}
	\end{equation}
	
	\textbf{Myon:}
	\begin{equation}
		\begin{split}
			\Delta a_\mu^{\text{(T0)}} &= 2{,}246 \times 10^{-13} \cdot (0{,}105658)^2 \\
			&= 2{,}246 \times 10^{-13} \cdot 1{,}1164 \times 10^{-2} \\
			&= 2{,}51 \times 10^{-15}
		\end{split}
		\label{eq:muon_prediction}
	\end{equation}
	
	\textbf{Tau:}
	\begin{equation}
		\begin{split}
			\Delta a_\tau^{\text{(T0)}} &= 2{,}246 \times 10^{-13} \cdot (1{,}77686)^2 \\
			&= 2{,}246 \times 10^{-13} \cdot 3{,}1572 \\
			&= 7{,}09 \times 10^{-13}
		\end{split}
		\label{eq:tau_prediction}
	\end{equation}
	
	\subsection{Umrechnung in konventionelle Einheiten}
	
	Die obigen Werte sind dimensionslos in natürlichen Einheiten. Für den Vergleich mit experimentellen Daten müssen diese Werte in die konventionellen Einheiten umgerechnet werden.
	
	In der üblichen Notation wird das anomale magnetische Moment als:
	\begin{equation}
		a_\ell = \frac{g_\ell - 2}{2}
		\label{eq:conventional_definition}
	\end{equation}
	
	angegeben, oft multipliziert mit $10^{11}$ für praktische Zahlenwerte.
	
	\textbf{Umgerechnete Werte:}
	\begin{align}
		\Delta a_\mu^{\text{(T0)}} &= 2{,}51 \times 10^{-9} = 251 \times 10^{-11} \label{eq:muon_conventional} \\
		\Delta a_e^{\text{(T0)}} &= 5{,}86 \times 10^{-14} = 0{,}0586 \times 10^{-12} \label{eq:electron_conventional} \\
		\Delta a_\tau^{\text{(T0)}} &= 7{,}09 \times 10^{-7} \label{eq:tau_conventional}
	\end{align}
	
	\section{Quadratische Massenskalierung}
	
	\subsection{Fundamentale Vorhersage}
	
	Die zentrale Vorhersage der T0-Theorie ist die quadratische Massenskalierung \eqref{eq:final_t0_formula}:
	\begin{equation}
		\Delta a_\ell^{\text{(T0)}} \propto m_\ell^2
		\label{eq:quadratic_scaling}
	\end{equation}
	
	Dies führt zu natürlichen Hierarchien:
	\begin{align}
		\frac{\Delta a_e^{\text{(T0)}}}{\Delta a_\mu^{\text{(T0)}}} 
		&= \left(\frac{m_e}{m_\mu}\right)^2 
		= \left(\frac{0{,}511}{105{,}658}\right)^2 
		= 2{,}34 \times 10^{-5} \label{eq:electron_muon_ratio} \\
		\frac{\Delta a_\tau^{\text{(T0)}}}{\Delta a_\mu^{\text{(T0)}}} 
		&= \left(\frac{m_\tau}{m_\mu}\right)^2 
		= \left(\frac{1776{,}86}{105{,}658}\right)^2 
		= 282{,}8 \label{eq:tau_muon_ratio}
	\end{align}
	
	\subsection{Physikalische Interpretation}
	
	Die quadratische Massenskalierung hat tiefe physikalische Bedeutung:
	
	\textbf{Ursache:} Die Kopplung ist massenproportional $g_T^\ell = \xi m_\ell$. Im Ein-Schleifen-Beitrag erscheint $(g_T^\ell)^2 = \xi^2 m_\ell^2$.
	
	\textbf{Konsequenzen:}
	\begin{itemize}
		\item Elektron-Effekte sind vernachlässigbar ($\sim 10^{-5}$ relativ zum Myon)
		\item Myon-Effekte sind messbar ($\sim 10^{-9}$)
		\item Tau-Effekte sind dominant ($\sim 280 \times$ größer als Myon)
	\end{itemize}
	
	\textbf{Vergleich mit QED:} In der Standardtheorie skaliert der führende Beitrag wie $\alpha/(2\pi)$, unabhängig von der Masse. Die T0-Theorie fügt einen massenabhängigen Beitrag hinzu.
	
	\section{Verbindung zur geometrischen Formulierung}
	
	\subsection{Verhältnis zur g-2 Analyse}
	
	Die hier abgeleiteten T0-Beiträge $\Delta a_\ell^{\text{(T0)}}$ sind zusätzliche Beiträge zum Standardmodell. Sie entsprechen den in Dokument 018 geometrisch hergeleiteten Werten.
	
	Der Unterschied in der Notation:
	\begin{itemize}
		\item \textbf{Dokument 018:} Berechnet direkt $a_\ell$ (inklusive SM + T0)
		\item \textbf{Dieses Dokument:} Berechnet $\Delta a_\ell^{\text{(T0)}}$ (nur T0-Beitrag)
	\end{itemize}
	
	Die Relation ist:
	\begin{equation}
		a_\ell^{\text{(total)}} = a_\ell^{\text{(SM)}} + \Delta a_\ell^{\text{(T0)}}
		\label{eq:total_relation}
	\end{equation}
	
	\subsection{Parameter-Entsprechungen}
	
	Die geometrischen Parameter aus Dokument 018 entsprechen den Lagrangian-Parametern:
	\begin{align}
		\xi &= \frac{4}{3} \times 10^{-4} \quad \text{(identisch in beiden Formulierungen)} \label{eq:xi_correspondence} \\
		f &= 7500 \quad \text{(geometrischer Sub-Planck-Faktor)} \label{eq:f_correspondence} \\
		\varphi &= \frac{1+\sqrt{5}}{2} \quad \text{(pentagonale Symmetriebrechung)} \label{eq:phi_correspondence}
	\end{align}
	
	Die Verbindung wird durch den Projektionsfaktor $k_{\text{geom}}$ hergestellt, der die geometrische 4D-3D-Projektion beschreibt.
	
	\section{Higgs-Mechanismus in der T0-Theorie}
	
	\subsection{Higgs-Feld als fundamentale Basis}
	
	In der T0-Theorie ist das Higgs-Feld nicht ein zusätzliches Feld, sondern die fundamentale Basis der Zeit-Masse-Dualität:
	\begin{equation}
		T(x,t) \cdot m(x,t) = 1
		\label{eq:higgs_foundation}
	\end{equation}
	
	Der universelle Parameter $\xi$ folgt direkt aus den Higgs-Parametern:
	\begin{equation}
		\boxed{\xi = \frac{\lambda_h^2 v^2}{16\pi^3 m_h^2}}
		\label{eq:xi_from_higgs}
	\end{equation}
	
	mit:
	\begin{itemize}
		\item $\lambda_h$: Higgs-Selbstkopplung
		\item $v = 246$ GeV: Higgs-Vakuumerwartungswert
		\item $m_h = 125$ GeV: Higgs-Masse
	\end{itemize}
	
	\subsection{Spontane Symmetriebrechung}
	
	Die spontane Symmetriebrechung des Higgs-Feldes erzeugt:
	\begin{itemize}
		\item Leptonmassen durch Yukawa-Kopplung
		\item Das Zeitfeld $T(x,t)$ als Fluktuationen um den VEV
		\item Die Zeit-Masse-Dualität als fundamentale Struktur
	\end{itemize}
	
	In diesem Bild sind Massenfluktuationen $\Delta m$ direkt mit Higgs-Fluktuationen verbunden:
	\begin{equation}
		\Delta m(x,t) = y_\ell \cdot \delta h(x,t)
		\label{eq:higgs_fluctuation}
	\end{equation}
	
	wobei $y_\ell$ die Yukawa-Kopplung und $\delta h$ die Higgs-Fluktuation ist.
	
	\section{Feldtheoretische Struktur}
	
	\subsection{Bewegungsgleichungen}
	
	Aus dem Lagrangian \eqref{eq:full_lagrangian} folgen die Euler-Lagrange-Gleichungen:
	
	\textbf{Für das Leptonfeld:}
	\begin{equation}
		(i\gamma^\mu D_\mu - m_\ell - \xi m_\ell \Delta m)\psi_\ell = 0
		\label{eq:lepton_eom}
	\end{equation}
	
	\textbf{Für das Zeitfeld:}
	\begin{equation}
		\Box \Delta m + m_T^2 \Delta m = \xi m_\ell \bar{\psi}_\ell \psi_\ell
		\label{eq:time_field_eom}
	\end{equation}
	
	wobei $\Box = \partial_\mu \partial^\mu$ der d'Alembert-Operator ist.
	
	\subsection{Interpretation der Bewegungsgleichungen}
	
	Gleichung \eqref{eq:lepton_eom} zeigt, dass das Lepton eine effektive Masse hat:
	\begin{equation}
		m_{\text{eff}}(x,t) = m_\ell (1 + \xi \Delta m)
		\label{eq:effective_mass}
	\end{equation}
	
	Die Masse wird durch das Zeitfeld moduliert.
	
	Gleichung \eqref{eq:time_field_eom} zeigt, dass das Zeitfeld eine Quelle hat:
	\begin{equation}
		\text{Quelle} = \xi m_\ell \bar{\psi}_\ell \psi_\ell = \xi m_\ell \rho_\ell
		\label{eq:source_term}
	\end{equation}
	
	Die Leptondichte $\rho_\ell$ erzeugt Zeitfeldfluktuationen, proportional zur Leptonmasse.
	
	\subsection{Energieskalen}
	
	Die charakteristischen Energieskalen in der Theorie sind:
	\begin{align}
		E_{\text{Lepton}} &\sim m_\ell \quad \text{(Leptonmasse)} \label{eq:lepton_scale} \\
		E_{\text{Zeit}} &\sim m_T = \frac{\lambda}{\xi} \quad \text{(Zeitfeldmasse)} \label{eq:time_scale} \\
		E_{\text{Kopplung}} &\sim \xi m_\ell \ll m_\ell \quad \text{(schwache Kopplung)} \label{eq:coupling_scale}
	\end{align}
	
	Die Hierarchie $E_{\text{Kopplung}} \ll E_{\text{Lepton}} \ll E_{\text{Zeit}}$ rechtfertigt die störungstheoretische Behandlung.
	
	\section{Renormierung}
	
	\subsection{Divergenzen in Schleifendiagrammen}
	
	Das Ein-Schleifen-Integral \eqref{eq:general_loop_formula} ist für $m_T \gg m_\ell$ endlich. Höhere Ordnungen können jedoch Divergenzen enthalten.
	
	Die Renormierung der T0-Theorie erfordert:
	\begin{itemize}
		\item Wellenfunction-Renormierung für $\psi_\ell$ und $\Delta m$
		\item Massen-Renormierung für $m_\ell$ und $m_T$
		\item Kopplungs-Renormierung für $g_T^\ell$
	\end{itemize}
	
	\subsection{Renormierungsgruppen-Gleichungen}
	
	Die Laufende Kopplung folgt:
	\begin{equation}
		\mu \frac{d g_T^\ell}{d\mu} = \beta_{g_T}(g_T^\ell)
		\label{eq:rg_equation}
	\end{equation}
	
	Aufgrund der massenproportionalen Struktur \eqref{eq:coupling_strength} ist:
	\begin{equation}
		\beta_{g_T} = \xi \beta_m
		\label{eq:beta_relation}
	\end{equation}
	
	wobei $\beta_m$ die anomale Massendimension ist.
	
	\subsection{Konsistenz mit Standardmodell}
	
	Die T0-Beiträge sind subdominant gegenüber den Standardmodell-Beiträgen:
	\begin{equation}
		\frac{\Delta a_\ell^{\text{(T0)}}}{a_\ell^{\text{(QED)}}} \sim \frac{\xi^4 m_\ell^2}{\alpha} \ll 1
		\label{eq:hierarchy}
	\end{equation}
	
	Dies garantiert Konsistenz mit experimentellen Daten und ermöglicht störungstheoretische Behandlung.
	
	\section{Vergleich verschiedener Formulierungen}
	
	\subsection{Lagrangian vs. geometrische Formulierung}
	
	\begin{table}[h]
		\centering
		\begin{tabular}{lcc}
			\toprule
			\textbf{Aspekt} & \textbf{Lagrangian (Dok. 019)} & \textbf{Geometrisch (Dok. 018)} \\
			\midrule
			Ausgangspunkt & Zeitfeld $\Delta m(x,t)$ & Torsionsgitter, Windungen \\
			Methode & Feldtheorie, Schleifenintegrale & Fraktale Geometrie \\
			Hauptformel & $\Delta a \propto \xi^4 m^2$ & $a \propto f \xi m^{p}$ \\
			Vorhersage & T0-Beitrag $\Delta a$ & Gesamtwert $a$ \\
			Parameter & $\xi$, $\lambda$, $m_T$ & $\xi$, $\varphi$, $f$ \\
			\bottomrule
		\end{tabular}
		\caption{Vergleich der Formulierungen}
		\label{tab:comparison}
	\end{table}
	
	\subsection{Komplementarität}
	
	Die beiden Formulierungen sind komplementär:
	\begin{itemize}
		\item \textbf{Lagrangian:} Gibt die feldtheoretische Struktur und Feynman-Regeln
		\item \textbf{Geometrisch:} Gibt die physikalische Intuition und Verhältnis-Vorhersagen
	\end{itemize}
	
	Beide führen zu konsistenten numerischen Vorhersagen mit $\sim$2\% Präzision.
	
	\section*{Literaturverzeichnis}
	
	\begin{thebibliography}{99}
		
		\bibitem{t0_g2_2026}
		J. Pascher,
		\textit{Anomale magnetische Momente in der FFGFT-Theorie: Geometrische Herleitung aus der Zeit-Masse-Dualität},
		\href{https://github.com/jpascher/T0-Time-Mass-Duality/blob/main/2/pdf/018_T0_Anomale-g2-10_De.pdf}{Dokument 018\_T0\_Anomale-g2-10\_De.pdf},
		Februar 2026.
		
		\bibitem{peskin_schroeder_1995}
		M. E. Peskin, D. V. Schroeder,
		\textit{An Introduction to Quantum Field Theory},
		Westview Press, 1995.
		Standardreferenz für Feynman-Regeln und Schleifenberechnungen.
		
		\bibitem{pdg_2024}
		Particle Data Group,
		\textit{Review of Particle Physics},
		Prog. Theor. Exp. Phys. 2024, 083C01 (2024).
		Experimentelle Werte für Leptonmassen und g-2.
		
		\bibitem{higgs_1964}
		P. W. Higgs,
		\textit{Broken Symmetries and the Masses of Gauge Bosons},
		Phys. Rev. Lett. 13, 508 (1964).
		Original-Higgs-Mechanismus.
		
		\bibitem{weinberg_qft}
		S. Weinberg,
		\textit{The Quantum Theory of Fields, Volume I: Foundations},
		Cambridge University Press, 1995.
		Umfassende Behandlung der Quantenfeldtheorie.
		
	\end{thebibliography}
	