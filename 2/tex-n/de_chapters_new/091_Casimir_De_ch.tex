
% TABLE CONVERTED TO LIST FORMAT FOR KDP COMPLIANCE
% Original table was too complex (many columns/rows)

\begin{itemize}
    \item \(\delta\) -- \(d=3+\delta\) -- \(\xi(\delta)=A_d\)
    \item -0.10 -- 2.90 -- \(7.375872\times10^{-3}\)
    \item -0.05 -- 2.95 -- \(6.835838\times10^{-3}\)
    \item -0.01 -- 2.99 -- \(6.430394\times10^{-3}\)
    \item \(0.00\) -- 3.00 -- \(6.332574\times10^{-3}\)
    \item \(0.01\) -- 3.01 -- \(6.236135\times10^{-3}\)
    \item \(0.05\) -- 3.05 -- \(5.863850\times10^{-3}\)
    \item \(0.10\) -- 3.10 -- \(5.427545\times10^{-3}\)
    \item $\hbar$ -- Reduziertes Planck'sches Wirkungsquantum -- $1.055 \times 10^{-34}$ J$\cdot$s
    \item $c$ -- Lichtgeschwindigkeit im Vakuum -- $2.998 \times 10^8$ m/s
    \item $G$ -- Gravitationskonstante -- $6.674 \times 10^{-11}$ m$^3$/kg$\cdot$s$^2$
    \item $k_B$ -- Boltzmann-Konstante -- $1.381 \times 10^{-23}$ J/K
    \item $\pi$ -- Kreiszahl -- $3.14159\ldots$
    \item \textbf{Symbol} -- \textbf{Bedeutung} -- \textbf{Wert/Einheit}
    \item $L_P$ -- Planck-Länge -- $1.616 \times 10^{-35}$ m
    \item $L_0$ -- Minimale Längenskala der granulierten Raumzeit -- $2.155 \times 10^{-39}$ m
    \item $L_\xi$ -- Charakteristische Vakuum-Längenskala -- $\approx 100$ $\mu$m
    \item $d$ -- Abstand zwischen Casimir-Platten -- Variable [m]
    \item \textbf{Symbol} -- \textbf{Bedeutung} -- \textbf{Wert/Einheit}
    \item $\xi$ -- Fundamentale dimensionslose Kopplungskonstante -- $1.333 \times 10^{-4}$
    \item $\alpha$ -- Cutoff-Faktor für Modenzählung -- $\mathcal{O}(1)$ [dimensionslos]
    \item $\gamma$ -- Anomale Dimension im RG-Ansatz -- Variable [dimensionslos]
    \item $\beta$ -- Kopplungsparameter für fraktale Dimension -- Variable [dimensionslos]
    \item $\delta$ -- Abweichung von der räumlichen Dimension 3 -- $|\delta| \ll 1$ [dimensionslos]
    \item \textbf{Symbol} -- \textbf{Bedeutung} -- \textbf{Wert/Einheit}
    \item $\rho_{\text{CMB}}$ -- Energiedichte der kosmischen Hintergrundstrahlung -- $4.17 \times 10^{-14}$ J/m$^3$
    \item $\rho_{\text{Casimir}}(d)$ -- Casimir-Energiedichte als Funktion des Abstands -- [J/m$^3$]
    \item $\rho_{\text{vac}}$ -- Vakuum-Energiedichte -- [J/m$^3$]
    \item $T_{\text{CMB}}$ -- Temperatur der kosmischen Hintergrundstrahlung -- $2.725$ K
    \item \textbf{Symbol} -- \textbf{Bedeutung} -- \textbf{Anmerkung}
    \item $\Gamma(x)$ -- Gamma-Funktion -- $\Gamma(n) = (n-1)!$ für $n \in \mathbb{N}$
    \item $\zeta(s)$ -- Riemannsche Zeta-Funktion -- Regularisierung
    \item $A_d$ -- Dimensionsabhängiger Vorfaktor -- $A_d = \frac{\pi^{-d/2}}{2^d\Gamma(d/2)(d+1)}$
    \item $S_{d-1}$ -- Oberfläche der $(d-1)$-dimensionalen Einheitssphäre -- $S_{d-1} = \frac{2\pi^{d/2}}{\Gamma(d/2)}$
    \item $\mathcal{L}$ -- Lagrange-Dichte -- Lagrangian-Formulierung
    \item \textbf{Symbol} -- \textbf{Bedeutung} -- \textbf{Einheit}
    \item $\phi$ -- Zeitfeld -- [dimensionsabhängig]
    \item $\mathbf{k}$ -- Wellenvektor -- [m$^{-1}$]
    \item $k$ -- Betrag des Wellenvektors, $k = |\mathbf{k}|$ -- [m$^{-1}$]
    \item $k_{\max}$ -- Maximaler Cutoff-Wellenvektor -- [m$^{-1}$]
    \item $\omega(k)$ -- Dispersionsrelation -- [s$^{-1}$]
    \item $F_{\mu\nu}$ -- Feldstärketensor -- Eichfeldtheorie
    \item \textbf{Symbol} -- \textbf{Bedeutung} -- \textbf{Anmerkung}
    \item $d$ -- Effektive räumliche Dimension -- $d = 3 + \delta$
    \item $D$ -- Hausdorff-Dimension der Raumzeit -- Fraktale Geometrie
    \item $\partial_\mu$ -- Partielle Ableitung nach $x^\mu$ -- Kovariante Notation
    \item $\nabla$ -- Nabla-Operator -- Räumliche Ableitungen
    \item \textbf{Symbol} -- \textbf{Bedeutung} -- \textbf{Typischer Bereich}
    \item $d_{\text{exp}}$ -- Experimenteller Plattenabstand (Casimir) -- $10$ nm - $10$ $\mu$m
    \item $L_{\xi,\text{exp}}$ -- Experimentell bestimmte charakteristische Länge -- $228$ nm - $18$ $\mu$m
    \item $F_{\text{Casimir}}$ -- Casimir-Kraft pro Flächeneinheit -- [N/m$^2$]
    \item \textbf{Symbol} -- \textbf{Bedeutung} -- \textbf{Anmerkung}
    \item $\frac{L_0}{L_P}$ -- Verhältnis Sub-Planck zu Planck -- $= \xi = 1.333 \times 10^{-4}$
    \item $\frac{L_P}{L_\xi}$ -- Verhältnis Planck zu Casimir-charakteristisch -- $\approx 1.616 \times 10^{-31}$
    \item $\frac{L_\xi}{d}$ -- Skalierungsparameter für Casimir-Effekt -- Dimensionslos
    \item $\left(\frac{L_\xi}{d}\right)^4$ -- Casimir-Skalierungsfaktor -- Charakteristische $d^{-4}$-Abhängigkeit
    \item \textbf{Symbol} -- \textbf{Bedeutung} -- \textbf{Kontext}
    \item CMB -- Cosmic Microwave Background -- Kosmische Hintergrundstrahlung
    \item RG -- Renormalization Group -- Renormierungsgruppe
    \item vac -- vacuum -- Vakuum
    \item exp -- experimental -- Experimentell
    \item reg -- regularized -- Regularisiert
    \item $\mu, \nu$ -- Lorentz-Indizes -- Relativistische Notation ($0,1,2,3$)
    \item $i, j, k$ -- Räumliche Indizes -- Räumliche Koordinaten ($1,2,3$)
    \item \textbf{Symbol} -- \textbf{Bedeutung} -- \textbf{Wert}
    \item $\frac{4}{3} \times 10^{-4}$ -- Numerischer Wert von $\xi$ -- $1.333 \times 10^{-4}$
    \item $\frac{\pi^2}{240}$ -- Casimir-Vorfaktor -- $\approx 0.0411$
    \item $\frac{\pi^2}{15}$ -- Stefan-Boltzmann-verwandter Faktor -- $\approx 0.658$
    \item $240$ -- Denominator in Casimir-Formel -- Exakt
\end{itemize}

% TABLE CONVERTED TO LIST FORMAT FOR KDP COMPLIANCE
% Original table was too complex (many columns/rows)

\begin{itemize}
    \item Abstand \( d \) -- {\(\rho_{\text{Casimir}}\) (\unit{\joule\per\meter\cubed})} -- {Verhältnis zu CMB}
    \item \SI{100}{\micro\meter} -- 4.17e-14 -- 1.00
    \item \SI{10}{\micro\meter} -- 4.17e-10 -- \num{1.0e4}
    \item \SI{1}{\micro\meter} -- 4.17e-2 -- \num{1.0e12}
    \item = \frac{\hbar c}{2}\frac{S_{d-1}}{(2\pi)^d}\int_0^{k_{\max}} k^{d}dk
    \item = \hbar c  A_d  k_{\max}^{d+1},
    \item \(\delta\) -- \(d=3+\delta\) -- \(\xi(\delta)=A_d\)
    \item -0.10 -- 2.90 -- \(7.375872\times10^{-3}\)
    \item -0.05 -- 2.95 -- \(6.835838\times10^{-3}\)
    \item -0.01 -- 2.99 -- \(6.430394\times10^{-3}\)
    \item \(0.00\) -- 3.00 -- \(6.332574\times10^{-3}\)
    \item \(0.01\) -- 3.01 -- \(6.236135\times10^{-3}\)
    \item \(0.05\) -- 3.05 -- \(5.863850\times10^{-3}\)
    \item \(0.10\) -- 3.10 -- \(5.427545\times10^{-3}\)
    \item $\hbar$ -- Reduziertes Planck'sches Wirkungsquantum -- $1.055 \times 10^{-34}$ J$\cdot$s
    \item $c$ -- Lichtgeschwindigkeit im Vakuum -- $2.998 \times 10^8$ m/s
    \item $G$ -- Gravitationskonstante -- $6.674 \times 10^{-11}$ m$^3$/kg$\cdot$s$^2$
    \item $k_B$ -- Boltzmann-Konstante -- $1.381 \times 10^{-23}$ J/K
    \item $\pi$ -- Kreiszahl -- $3.14159\ldots$
    \item \textbf{Symbol} -- \textbf{Bedeutung} -- \textbf{Wert/Einheit}
    \item $L_P$ -- Planck-Länge -- $1.616 \times 10^{-35}$ m
    \item $L_0$ -- Minimale Längenskala der granulierten Raumzeit -- $2.155 \times 10^{-39}$ m
    \item $L_\xi$ -- Charakteristische Vakuum-Längenskala -- $\approx 100$ $\mu$m
    \item $d$ -- Abstand zwischen Casimir-Platten -- Variable [m]
    \item \textbf{Symbol} -- \textbf{Bedeutung} -- \textbf{Wert/Einheit}
    \item $\xi$ -- Fundamentale dimensionslose Kopplungskonstante -- $1.333 \times 10^{-4}$
    \item $\alpha$ -- Cutoff-Faktor für Modenzählung -- $\mathcal{O}(1)$ [dimensionslos]
    \item $\gamma$ -- Anomale Dimension im RG-Ansatz -- Variable [dimensionslos]
    \item $\beta$ -- Kopplungsparameter für fraktale Dimension -- Variable [dimensionslos]
    \item $\delta$ -- Abweichung von der räumlichen Dimension 3 -- $|\delta| \ll 1$ [dimensionslos]
    \item \textbf{Symbol} -- \textbf{Bedeutung} -- \textbf{Wert/Einheit}
    \item $\rho_{\text{CMB}}$ -- Energiedichte der kosmischen Hintergrundstrahlung -- $4.17 \times 10^{-14}$ J/m$^3$
    \item $\rho_{\text{Casimir}}(d)$ -- Casimir-Energiedichte als Funktion des Abstands -- [J/m$^3$]
    \item $\rho_{\text{vac}}$ -- Vakuum-Energiedichte -- [J/m$^3$]
    \item $T_{\text{CMB}}$ -- Temperatur der kosmischen Hintergrundstrahlung -- $2.725$ K
    \item \textbf{Symbol} -- \textbf{Bedeutung} -- \textbf{Anmerkung}
    \item $\Gamma(x)$ -- Gamma-Funktion -- $\Gamma(n) = (n-1)!$ für $n \in \mathbb{N}$
    \item $\zeta(s)$ -- Riemannsche Zeta-Funktion -- Regularisierung
    \item $A_d$ -- Dimensionsabhängiger Vorfaktor -- $A_d = \frac{\pi^{-d/2}}{2^d\Gamma(d/2)(d+1)}$
    \item $S_{d-1}$ -- Oberfläche der $(d-1)$-dimensionalen Einheitssphäre -- $S_{d-1} = \frac{2\pi^{d/2}}{\Gamma(d/2)}$
    \item $\mathcal{L}$ -- Lagrange-Dichte -- Lagrangian-Formulierung
    \item \textbf{Symbol} -- \textbf{Bedeutung} -- \textbf{Einheit}
    \item $\phi$ -- Zeitfeld -- [dimensionsabhängig]
    \item $\mathbf{k}$ -- Wellenvektor -- [m$^{-1}$]
    \item $k$ -- Betrag des Wellenvektors, $k = |\mathbf{k}|$ -- [m$^{-1}$]
    \item $k_{\max}$ -- Maximaler Cutoff-Wellenvektor -- [m$^{-1}$]
    \item $\omega(k)$ -- Dispersionsrelation -- [s$^{-1}$]
    \item $F_{\mu\nu}$ -- Feldstärketensor -- Eichfeldtheorie
    \item \textbf{Symbol} -- \textbf{Bedeutung} -- \textbf{Anmerkung}
    \item $d$ -- Effektive räumliche Dimension -- $d = 3 + \delta$
    \item $D$ -- Hausdorff-Dimension der Raumzeit -- Fraktale Geometrie
    \item $\partial_\mu$ -- Partielle Ableitung nach $x^\mu$ -- Kovariante Notation
    \item $\nabla$ -- Nabla-Operator -- Räumliche Ableitungen
    \item \textbf{Symbol} -- \textbf{Bedeutung} -- \textbf{Typischer Bereich}
    \item $d_{\text{exp}}$ -- Experimenteller Plattenabstand (Casimir) -- $10$ nm - $10$ $\mu$m
    \item $L_{\xi,\text{exp}}$ -- Experimentell bestimmte charakteristische Länge -- $228$ nm - $18$ $\mu$m
    \item $F_{\text{Casimir}}$ -- Casimir-Kraft pro Flächeneinheit -- [N/m$^2$]
    \item \textbf{Symbol} -- \textbf{Bedeutung} -- \textbf{Anmerkung}
    \item $\frac{L_0}{L_P}$ -- Verhältnis Sub-Planck zu Planck -- $= \xi = 1.333 \times 10^{-4}$
    \item $\frac{L_P}{L_\xi}$ -- Verhältnis Planck zu Casimir-charakteristisch -- $\approx 1.616 \times 10^{-31}$
    \item $\frac{L_\xi}{d}$ -- Skalierungsparameter für Casimir-Effekt -- Dimensionslos
    \item $\left(\frac{L_\xi}{d}\right)^4$ -- Casimir-Skalierungsfaktor -- Charakteristische $d^{-4}$-Abhängigkeit
    \item \textbf{Symbol} -- \textbf{Bedeutung} -- \textbf{Kontext}
    \item CMB -- Cosmic Microwave Background -- Kosmische Hintergrundstrahlung
    \item RG -- Renormalization Group -- Renormierungsgruppe
    \item vac -- vacuum -- Vakuum
    \item exp -- experimental -- Experimentell
    \item reg -- regularized -- Regularisiert
    \item $\mu, \nu$ -- Lorentz-Indizes -- Relativistische Notation ($0,1,2,3$)
    \item $i, j, k$ -- Räumliche Indizes -- Räumliche Koordinaten ($1,2,3$)
    \item \textbf{Symbol} -- \textbf{Bedeutung} -- \textbf{Wert}
    \item $\frac{4}{3} \times 10^{-4}$ -- Numerischer Wert von $\xi$ -- $1.333 \times 10^{-4}$
    \item $\frac{\pi^2}{240}$ -- Casimir-Vorfaktor -- $\approx 0.0411$
    \item $\frac{\pi^2}{15}$ -- Stefan-Boltzmann-verwandter Faktor -- $\approx 0.658$
    \item $240$ -- Denominator in Casimir-Formel -- Exakt
\end{itemize}
