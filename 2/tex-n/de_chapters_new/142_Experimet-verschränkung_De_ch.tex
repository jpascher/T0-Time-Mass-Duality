\chapter{Attosekunden-Vorhersage zur Entstehung von Quantenverschränkung \\
	als Beleg für die T$_0$-Time-Mass-Duality-Theorie}

	

	
\section*{Abstract}
		Dieses Dokument fasst die theoretische Vorhersage zur zeitaufgelösten Entstehung von Quantenverschränkung (Jiang et al., 2024) zusammen und nutzt sie als Beleg für die fundamentale Zeitdynamik, die in der T$_0$-Time-Mass-Duality-Theorie postuliert wird. Alle theoretischen Interpretationen basieren ausschließlich auf dem Inhalt der Master-Narrative (FFGFT\_Narrative\_Master\_De.pdf) und den zugehörigen Dokumenten im Repository:
		\url{https://github.com/jpascher/T0-Time-Mass-Duality/tree/main/2/}.

	
	\section{Die theoretische Arbeit}
	Die Studie von Jiang et al.\ (2024) zeigt theoretisch, dass Quantenverschränkung \textbf{nicht instantan} entsteht, sondern sich über ein messbares lokales Zeitfenster aufbaut.
	
	\subsection{Wichtige Details aus der Simulation}
	\begin{itemize}[leftmargin=*]
		\item \textbf{System}: Helium-Atom unter intensivem hochfrequentem EUV-Laserpuls (Photoionisation).
		\item \textbf{Prozess}: Ein Elektron absorbiert Energie und entweicht (ionisiert), das zweite Elektron wird in einen höheren Energiezustand angeregt.
		\item \textbf{Superposition}: Das entweichende Elektron befindet sich in einer Superposition verschiedener Austrittszeiten (kein scharfer Moment).
		\item \textbf{Korrelation}: Die Endenergie des gebundenen Elektrons korreliert direkt mit der Austrittszeit des entweichenden Elektrons:
		\begin{itemize}
			\item Höhere Energie im gebundenen Elektron $\to$ entweichendes Elektron verließ früher
			\item Niedrigere Energie $\to$ entweichendes Elektron verließ später
		\end{itemize}
		\item \textbf{Vorhergesagtes Zeitfenster}: Vollständige Simulation der zeitabhängigen Sch-rödinger-Gleichung ergibt ein Entstehungsfenster von $\sim$\textbf{232 Attosekunden} ($\approx 2{,}32 \times 10^{-16}$\,s).
		\item \textbf{Vorgeschlagene experimentelle Überprüfung}: Doppelpuls-Verfahren (Erzeugungspuls + Sondierungspuls) kombiniert mit Koinzidenzdetektion beider Elektronen, um die gemeinsame Quantengeschichte zu rekonstruieren und die Entstehung zu timen.
	\end{itemize}
	
	\textbf{Wichtiger Hinweis}: Es handelt sich um eine theoretische/numerische Vorhersage. Bislang wurde kein Laborexperiment durchgeführt. Die Autoren schlagen ein mit aktueller Attosekunden-Lasertechnik machbares Experiment vor.
	
	\subsection{Populärwissenschaftliches Video}
	Zusammenfassung der Arbeit im Video:  
	\url{https://www.youtube.com/watch?v=t3wjY95zvNM}  
	(``Scientists Measure Quantum Entanglement Speed — And It Breaks Physics'', Kanal: NASA Space News, Hochgeladen: 14. Januar 2026)
	
	\section{Verbindung zur T$_0$-Time-Mass-Duality-Theorie}
	Dieses theoretische Ergebnis liefert starken konzeptionellen Beleg für das Kernpostulat der Theorie:
	
	\begin{quote}
		``In der T$_0$-Time-Mass-Duality-Theorie ist Zeit ontologisch äquivalent zu Masse und damit keine bloße Koordinate, sondern eine aktive physikalische Größe mit realer Dynamik auf allen Skalen. Quantenkorrelationen (Verschränkung) entstehen daher nicht augenblicklich, sondern entwickeln sich als zeitlicher, emergenter Prozess innerhalb eines lokalen Interaktionsfensters. Die vorhergesagte Attosekunden-Entstehungszeit von $\sim 232$\,as bestätigt genau diesen endlichen, dynamischen Aufbau ohne nicht-lokale ‚spooky action at a distance‘ und ohne Verletzung der Kausalität.''
	\end{quote}
	
	Dies unterstreicht, dass alle Quantenphänomene intrinsische Zeitdynamik tragen – eine direkte Konsequenz der fundamentalen Dualität zwischen Zeit und Masse.
	
	\section{Literaturverzeichnis}
	\begin{enumerate}[leftmargin=*]
		\item Jiang, W.-C., Zhong, M.-C., Fang, Y.-K., Donsa, S., Březinová, I., Peng, L.-Y., Burgdörfer, J. (2024). \\
		\emph{Time Delays as Attosecond Probe of Interelectronic Coherence and Entanglement}. \\
		\textbf{Physical Review Letters 133, 163201}. \\
		DOI: \href{https://doi.org/10.1103/PhysRevLett.133.163201}{10.1103/PhysRevLett.133.163201}
		\item Video: ``Scientists Measure Quantum Entanglement Speed — And It Breaks Physics''. \\
		YouTube, Kanal: NASA Space News. \\
		\url{https://www.youtube.com/watch?v=t3wjY95zvNM} (abgerufen am 15. Januar 2026)
	\end{enumerate}
	