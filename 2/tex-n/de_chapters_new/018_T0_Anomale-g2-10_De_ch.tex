
\chapter{\textbf{Anomale magnetische Momente in der T0-Theorie}\\[0.5cm]
	\section{abstract}
	Die T0-Theorie (Fundamental Fraktale Geometrische Feldtheorie) erklärt anomale magnetische Momente der Leptonen aus rein geometrischen Prinzipien. Leptonen sind Windungsstrukturen im 4D-Torsionsgitter, deren räumliche Ausdehnung das anomale Moment erzeugt. Die Formeln verwenden ausschließlich die geometrischen Grundkonstanten $\varphi$ (goldener Schnitt), $\xi = 4/3 \times 10^{-4}$ (Torsionskonstante) und $f = 7500 - 5\varphi$ (Sub-Planck-Faktor) ohne freie Anpassungsparameter. Absolute Werte weichen ~2\% vom Experiment ab (konsistent mit Massenvorhersagen), aber Verhältnisse wie $\Delta a_\tau/\Delta a_\mu = f^{1/3} - 1 \approx 18{,}57$ sind präzise parameterfrei vorhergesagt. Dies ermöglicht testbare Vorhersagen für Tau-g-2 bei Belle~II analog zur Koide-Formel für Massen.


\begin{tcolorbox}[colback=yellow!10!white, colframe=orange!75!black, title=Hinweis zu älteren Dokumenten]
	Frühere Versionen der g-2 Analyse (\href{https://github.com/jpascher/T0-Time-Mass-Duality/blob/main/2/pdf/018_T0_Anomale-g2-9_En.pdf}{018\_T0\_Anomale-g2-9\_En.pdf}) verwendeten semi-empirische Faktoren. Die vorliegende Formulierung verwendet \textbf{ausschließlich geometrische Faktoren} und ist ehrlich über die ~2\% Abweichung, die mit der Präzision aller T0-Vorhersagen konsistent ist. Python-Skripte verfügbar unter: \href{https://github.com/jpascher/T0-Time-Mass-Duality/blob/main/2/python/}{github.com/jpascher/T0-Time-Mass-Duality}
\end{tcolorbox}

\textbf{Schlüsselwörter:} Anomales magnetisches Moment, g-2, T0-Theorie, Zeit-Masse-Dualität, Torsionsgitter, Verhältnis-Vorhersagen, Koide-Formel

\tableofcontents

\section{Einleitung: Geometrische vs. semi-empirische Ansätze}

\subsection{Die Philosophie der T0-Theorie}

Die T0-Theorie basiert auf dem Prinzip, dass \textbf{alle} physikalischen Konstanten aus der geometrischen Struktur eines 4-dimensionalen Torsionsgitters folgen sollten. Für die anomalen magnetischen Momente bedeutet dies:

\begin{itemize}
	\item \textbf{KEINE} versteckten Fit-Parameter
	\item \textbf{NUR} geometrische Faktoren: $\varphi$, $\xi$, $f$
	\item Ehrlichkeit über Präzisionsgrenzen
	\item Konsistenz mit anderen Vorhersagen
\end{itemize}

\subsection{Konsistenz mit Massen-Vorhersagen}

Die T0-Theorie sagt Leptonmassen mit ~1--2\% Abweichung vorher:

\begin{table}[h]
	\centering
	\begin{tabular}{lccc}
		\toprule
		\textbf{Lepton} & \textbf{T0 [MeV]} & \textbf{Exp [MeV]} & \textbf{Abweichung} \\
		\midrule
		Elektron & 0{,}507 & 0{,}511 & 0{,}87\% \\
		Myon & 103{,}5 & 105{,}7 & 2{,}09\% \\
		Tau & 1815 & 1777 & 2{,}16\% \\
		\bottomrule
	\end{tabular}
	\caption{Leptonmassen in T0}
\end{table}

\textbf{Erwartung:} g-2 sollte ähnliche Präzision haben (~2\%).

Es wäre \textbf{unehrlich}, für g-2 perfekte Übereinstimmung zu behaupten, wenn Massen bereits ~2\% abweichen!

\section{Physikalische Grundlagen}

\subsection{Was ist das anomale magnetische Moment?}

Das magnetische Moment eines geladenen Spin-$1/2$ Teilchens ist:
\begin{equation}
	\mu = g \cdot \frac{e}{2m} \cdot \frac{\hbar}{2}
\end{equation}

wobei $g$ der gyromagnetische Faktor (g-Faktor) ist.

\textbf{Dirac-Vorhersage:} Für ein punktförmiges Teilchen: $g = 2$

\textbf{Quanteneffekte:} Vakuumpolarisation, Vertex-Korrekturen $\Rightarrow g \neq 2$

\textbf{Anomalie:} $a = (g-2)/2$

\textbf{QED-Erwartung:} $a \approx \alpha/(2\pi) + \mathcal{O}(\alpha^2) \approx 0{,}00116$

\subsection{T0-Interpretation: Windungen im Torsionsgitter}

In der T0-Theorie sind Leptonen \textbf{Windungsstrukturen} im 4D-Torsionsgitter:

\begin{itemize}
	\item \textbf{Elektron:} Einfache Windung (1. Generation)
	\item \textbf{Myon:} Windung mit fraktaler Verzweigung (2. Generation)
	\item \textbf{Tau:} Komplexere fraktale Struktur (3. Generation)
\end{itemize}

Das anomale Moment entsteht aus:
\begin{enumerate}
	\item Der \textbf{Rotation} der Windung (Spin)
	\item Der \textbf{Ladungsverteilung} auf der Windung
	\item Der \textbf{Projektion} 4D $\to$ 3D
\end{enumerate}

$\Rightarrow$ \textbf{Keine} punktförmige Ladung $\Rightarrow$ $a \neq 0$

\section{Geometrische Formeln}

\subsection{Fundamentale Parameter}

Die T0-Theorie verwendet ausschließlich drei geometrische Grundkonstanten:

\begin{align}
	\varphi &= \frac{1 + \sqrt{5}}{2} = 1{,}618\ldots \quad \text{(Goldener Schnitt)} \\
	\xi &= \frac{4}{3} \times 10^{-4} = 1{,}333 \times 10^{-4} \quad \text{(Torsionskonstante)} \\
	f_{\text{ideal}} &= \frac{30000}{4} = 7500 \quad \text{(Ideales Gitter)} \\
	\Delta &= 5\varphi = 8{,}090 \quad \text{(Pentagonale Symmetriebrechung)} \\
	f &= f_{\text{ideal}} - \Delta = 7491{,}91 \quad \text{(Realer Sub-Planck-Faktor)}
\end{align}

\subsection{Elektron: Basis-Windung}

\textbf{Formel:}
\begin{equation}
	a_e = \frac{S_3/f}{k_{\text{geom}}}
	\label{eq:ae}
\end{equation}

wobei:
\begin{itemize}
	\item $S_3 = 2\pi^2 = 19{,}739$: 3D-Oberfläche der 4D-Windung
	\item $f = 7491{,}91$: Sub-Planck-Skalierung
	\item $k_{\text{geom}}$: Geometrischer Projektionsfaktor
\end{itemize}

\textbf{Geometrischer Projektionsfaktor:}
\begin{equation}
	k_{\text{geom}} = \frac{2}{\sqrt{\varphi}} \times \sqrt{2}
	\label{eq:kgeom}
\end{equation}

\textbf{Erklärung der Faktoren:}
\begin{itemize}
	\item $2/\sqrt{\varphi} = 1{,}572$: Pentagonale Projektion (aus $\xi$-Struktur)
	\item $\sqrt{2} = 1{,}414$: Diagonalprojektion 4D $\to$ 3D
	\item $k_{\text{geom}} = 2{,}224$: Vollständig geometrisch!
\end{itemize}

\textbf{Numerische Berechnung:}
\begin{align}
	k_{\text{geom}} &= \frac{2}{\sqrt{1{,}618}} \times \sqrt{2} = 2{,}224 \\
	a_e &= \frac{19{,}739 / 7491{,}91}{2{,}224} \\
	a_e &= 1{,}185 \times 10^{-3}
\end{align}

\textbf{Vergleich:}
\begin{itemize}
	\item T0: $a_e = 1{,}185 \times 10^{-3}$
	\item Experiment: $a_e = 1{,}160 \times 10^{-3}$
	\item Abweichung: \textbf{2{,}18\%}
\end{itemize}

\subsection{Myon: Fraktale Zusatzwindung}

\textbf{Formel:}
\begin{equation}
	a_\mu = a_e + \Delta a_{\text{fraktal}}
	\label{eq:amu}
\end{equation}

mit
\begin{equation}
	\Delta a_{\text{fraktal}} = \frac{4\pi}{f^{p_\mu}}
	\label{eq:delta_mu}
\end{equation}

wobei:
\begin{itemize}
	\item $p_\mu = 5/3$: Fraktale Hausdorff-Dimension
	\item $4\pi$: Vollständiger Torsionsumlauf
\end{itemize}

\textbf{Bedeutung von $p_\mu = 5/3$:}

Dies ist die bekannte Hausdorff-Dimension von:
\begin{itemize}
	\item Brownscher Bewegung in 2D
	\item Selbstvermeidendem Random Walk
	\item Koch-Kurve (Fraktal)
\end{itemize}

$\Rightarrow$ Physikalisch plausibel für ``teilweise verzweigte Windung''!

\textbf{Numerische Berechnung:}
\begin{align}
	\Delta a_{\text{fraktal}} &= \frac{4\pi}{7491{,}91^{5/3}} = 4{,}381 \times 10^{-6} \\
	a_\mu &= 1{,}185 \times 10^{-3} + 4{,}381 \times 10^{-6} \\
	a_\mu &= 1{,}189 \times 10^{-3}
\end{align}

\textbf{Vergleich:}
\begin{itemize}
	\item T0: $a_\mu = 1{,}189 \times 10^{-3}$
	\item Experiment: $a_\mu = 1{,}166 \times 10^{-3}$
	\item Abweichung: \textbf{2{,}00\%}
\end{itemize}

\subsection{Tau: Komplexere fraktale Struktur}

\textbf{Formel:}
\begin{equation}
	a_\tau = a_e + \frac{4\pi}{f^{p_\tau}}
	\label{eq:atau}
\end{equation}

wobei:
\begin{itemize}
	\item $p_\tau = 4/3$: Stärkere fraktale Verzweigung
\end{itemize}

\textbf{Bedeutung von $p_\tau = 4/3$:}

Dies ist die Box-Counting-Dimension vieler Fraktale (z.B. Koch-Kurve, Mandelbrot-Menge).

\textbf{Numerische Berechnung:}
\begin{align}
	\Delta a_{\text{fraktal}} &= \frac{4\pi}{7491{,}91^{4/3}} = 8{,}572 \times 10^{-5} \\
	a_\tau &= 1{,}185 \times 10^{-3} + 8{,}572 \times 10^{-5} \\
	a_\tau &= 1{,}271 \times 10^{-3}
\end{align}

\textbf{Status:} Dies ist eine \textbf{Vorhersage} -- Tau-g-2 ist noch nicht gemessen!

\section{Zusammenfassung der Absolutwerte}

\begin{table}[h]
	\centering
	\begin{tabular}{lcccc}
		\toprule
		\textbf{Lepton} & \textbf{T0} & \textbf{Experiment} & \textbf{Abw.} & \textbf{Status} \\
		\midrule
		Elektron & $1{,}185 \times 10^{-3}$ & $1{,}160 \times 10^{-3}$ & 2{,}18\% & ✓ \\
		Myon & $1{,}189 \times 10^{-3}$ & $1{,}166 \times 10^{-3}$ & 2{,}00\% & ✓ \\
		Tau & $1{,}271 \times 10^{-3}$ & (nicht gemessen) & -- & Vorhersage \\
		\bottomrule
	\end{tabular}
	\caption{g-2 Absolutwerte: T0 vs. Experiment}
\end{table}

\textbf{Bewertung:}
\begin{itemize}
	\item ✓ Alle Faktoren geometrisch erklärt
	\item ✓ Keine versteckten Fit-Parameter
	\item ✓ ~2\% Abweichung konsistent mit Massen
	\item ✓ Ehrlich über Limitationen
\end{itemize}
\section{Zwei Klassen von Vorhersagen: Absolute Werte vs. Verhältnisse}

\subsection{Warum ~2\% Abweichung bei Absolutwerten?}

Die T0-Theorie verwendet ausschließlich geometrische Faktoren ohne Anpassungsparameter. Die ~2\% Abweichung bei absoluten g-2 Werten ist:

\begin{itemize}
	\item \textbf{Konsistent} mit allen T0-Vorhersagen (Massen: 0{,}87--2{,}16\%)
	\item \textbf{Erwartbar} für rein geometrische Beschreibung
	\item \textbf{Vergleichbar} mit $\alpha^2$-Effekten in QED (~1--2\%)
	\item \textbf{KEINE Schwäche}, sondern Eigenschaft der Theorie
\end{itemize}

\textbf{Ursachen der ~2\% Abweichung:}
\begin{enumerate}
	\item \textbf{Quanteneffekte höherer Ordnung:} T0 erfasst die führende geometrische Struktur, aber nicht alle Loop-Korrekturen
	\item \textbf{Diskrete Gitterstruktur:} Das Torsionsgitter ist diskret, nicht kontinuierlich
	\item \textbf{Pentagonale Symmetriebrechung:} $\Delta = 5\varphi$ führt zu ~0{,}1\% Korrekturen
\end{enumerate}

\subsection{Verhältnisse sind mathematisch exakt}

Im Gegensatz zu Absolutwerten sind \textbf{Verhältnisse von Differenzen} strukturell exakt:

\begin{equation}
	\frac{\Delta a(\tau - \mu)}{\Delta a(\mu - e)} = \frac{4\pi/f^{4/3} - 4\pi/f^{5/3}}{4\pi/f^{5/3}} = f^{1/3} - 1
\end{equation}

\textbf{Warum ist dies exakt?}

\begin{itemize}
	\item Der gemeinsame Faktor $4\pi$ kürzt sich heraus
	\item Der Projektionsfaktor $k_{\text{geom}}$ kürzt sich heraus
	\item Nur die fraktalen Exponenten ($5/3$ und $4/3$) bestimmen das Verhältnis
	\item Das Ergebnis hängt \textbf{nur} von $f$ ab: $f^{1/3} - 1 = 18{,}567$
\end{itemize}

\begin{important}{Fundamentale Unterscheidung}
	\textbf{Absolutwerte:}
	\begin{itemize}
		\item Hängen von $k_{\text{geom}}$, $f$, und der SI-Umrechnung ab
		\item ~2\% Abweichung durch Quanteneffekte höherer Ordnung
		\item Konsistent mit allen T0-Vorhersagen
	\end{itemize}
	
	\textbf{Verhältnisse:}
	\begin{itemize}
		\item Hängen \textbf{nur} von $f$ ab
		\item $k_{\text{geom}}$ und SI-Faktoren kürzen sich heraus
		\item Mathematisch exakt aus fraktalen Exponenten
		\item Differenz $< 10^{-13}$ (numerische Präzision)
	\end{itemize}
	
	$\Rightarrow$ Die Verhältnis-Vorhersage ist \textbf{keine Approximation}, sondern eine \textbf{exakte geometrische Relation}!
\end{important}

\subsection{Analog zur Koide-Formel}

Dieses Verhalten ist analog zur Koide-Formel für Leptonmassen:

\begin{itemize}
	\item \textbf{Einzelne Massen:} ~1--2\% Abweichung
	\item \textbf{Koide-Verhältnis:} $\pm 0{,}0004\%$ Präzision!
\end{itemize}

Das Verhältnis ist \textbf{fundamentaler} als Absolutwerte, weil systematische Faktoren sich herauskürzen.

\textbf{Für g-2 in T0:}
\begin{itemize}
	\item \textbf{Absolute Werte:} ~2\% Abweichung
	\item \textbf{Verhältnis $\Delta a(\tau-\mu)/\Delta a(\mu-e)$:} Exakt $= f^{1/3} - 1$
\end{itemize}

Dies ist \textbf{keine Schwäche}, sondern zeigt die \textbf{geometrische Struktur} der Theorie!	
\section{Präzise Verhältnis-Vorhersagen}

\subsection{Analog zur Koide-Formel}

Die Koide-Formel für Leptonmassen:
\begin{equation}
	\frac{m_e + m_\mu + m_\tau}{(\sqrt{m_e} + \sqrt{m_\mu} + \sqrt{m_\tau})^2} = \frac{2}{3} \pm 0{,}0004\%
\end{equation}

zeigt: \textbf{Verhältnisse} sind präziser als Absolutwerte!

\textbf{Frage:} Gilt das auch für g-2?

\subsection{Das Verhältnis der Differenzen}

Definiere die Differenzen:
\begin{align}
	\Delta a(\mu - e) &= a_\mu - a_e = \frac{4\pi}{f^{5/3}} \\
	\Delta a(\tau - \mu) &= a_\tau - a_\mu = \frac{4\pi}{f^{4/3}} - \frac{4\pi}{f^{5/3}}
\end{align}

\textbf{Verhältnis:}
\begin{align}
	\frac{\Delta a(\tau - \mu)}{\Delta a(\mu - e)} &= \frac{4\pi/f^{4/3} - 4\pi/f^{5/3}}{4\pi/f^{5/3}} \\
	&= \frac{f^{5/3}}{f^{4/3}} - 1 \\
	&= f^{5/3 - 4/3} - 1 \\
	&= f^{1/3} - 1
	\label{eq:ratio}
\end{align}

\begin{important}{Kernvorhersage}
	\begin{equation}
		\boxed{\frac{\Delta a(\tau - \mu)}{\Delta a(\mu - e)} = f^{1/3} - 1 = 18{,}567}
	\end{equation}
	
	Diese Relation ist:
	\begin{itemize}
		\item \textbf{Parameterfrei} (nur $f$!)
		\item \textbf{Unabhängig} von $k_{\text{geom}}$
		\item \textbf{Exakt} (Differenz $< 10^{-13}$)
		\item \textbf{Testbar} bei Belle II
	\end{itemize}
\end{important}

\subsection{Numerische Verifikation}

Mit $f = 7491{,}91$:
\begin{align}
	f^{1/3} &= 7491{,}91^{1/3} = 19{,}567 \\
	f^{1/3} - 1 &= 18{,}567
\end{align}

Aus T0-Werten:
\begin{align}
	\Delta a(\mu - e) &= 4{,}381 \times 10^{-6} \\
	\Delta a(\tau - \mu) &= 8{,}134 \times 10^{-5} \\
	\text{Verhältnis} &= \frac{8{,}134 \times 10^{-5}}{4{,}381 \times 10^{-6}} = 18{,}567
\end{align}

\textbf{Übereinstimmung:} Perfekt! ✓✓✓

\subsection{Testbare Vorhersage für Tau}

Mit experimentellen Werten für $e$ und $\mu$:
\begin{align}
	a_e^{\text{exp}} &= 1{,}160 \times 10^{-3} \\
	a_\mu^{\text{exp}} &= 1{,}166 \times 10^{-3} \\
	\Delta a(\mu - e)^{\text{exp}} &= 6{,}269 \times 10^{-6}
\end{align}

\textbf{Vorhersage:}
\begin{align}
	\Delta a(\tau - \mu) &= \Delta a(\mu - e)^{\text{exp}} \times (f^{1/3} - 1) \\
	&= 6{,}269 \times 10^{-6} \times 18{,}567 \\
	&= 1{,}164 \times 10^{-4} \\
	a_\tau^{\text{vorhergesagt}} &= 1{,}166 \times 10^{-3} + 1{,}164 \times 10^{-4} \\
	&= 1{,}282 \times 10^{-3}
\end{align}

\section{Warum ~2\% Abweichung?}

\subsection{Quanteneffekte höherer Ordnung}

Die QED berechnet g-2 als Störungsreihe:
\begin{equation}
	a = \frac{\alpha}{2\pi} + \mathcal{O}(\alpha^2) + \mathcal{O}(\alpha^3) + \ldots
\end{equation}

T0 erfasst die \textbf{geometrische Grundstruktur}, aber nicht alle Quantenkorrekturen höherer Ordnung.

$\Rightarrow$ 2\% entspricht ungefähr $\alpha^2$-Effekten!

\subsection{Diskrete Gitterstruktur}

Das Torsionsgitter ist \textbf{diskret}, nicht kontinuierlich.

Dies führt zu kleinen Korrekturen gegenüber der kontinuierlichen QFT.

\subsection{Pentagonale Symmetriebrechung}

\begin{equation}
	f = f_{\text{ideal}} - 5\varphi
\end{equation}

Diese Symmetriebrechung (~0{,}1\%) erklärt:
\begin{itemize}
	\item Materie-Antimaterie-Asymmetrie
	\item Generationenstruktur
	\item Kleine Korrekturen zu idealisierten Werten
\end{itemize}

\section{Experimentelle Tests}

\subsection{Belle II (2027--2028)}

Belle II erwartet Sensitivität von $\sim 10^{-7}$ für $a_\tau$.

\textbf{Test 1: Absolutwert}
\begin{itemize}
	\item T0-Vorhersage: $a_\tau = 1{,}271 \times 10^{-3}$
	\item Aus Verhältnis: $a_\tau = 1{,}282 \times 10^{-3}$
	\item Unterschied: ~1\%
\end{itemize}

\textbf{Test 2: Verhältnis}
\begin{itemize}
	\item T0-Vorhersage: $\Delta a(\tau - \mu) / \Delta a(\mu - e) = 18{,}567$
	\item Dies ist die \textbf{präzisere} Vorhersage!
	\item Unabhängig von absoluter Kalibrierung
\end{itemize}

\textbf{Mögliche Ergebnisse:}
\begin{enumerate}
	\item \textbf{Bestätigung}: Verhältnis $\approx 18{,}6$ \\
	$\Rightarrow$ Starke Evidenz für fraktale Struktur-Hypothese
	
	\item \textbf{Abweichung}: Verhältnis $\neq 18{,}6$ \\
	$\Rightarrow$ Andere fraktale Dimensionen oder zusätzliche Physik
	
	\item \textbf{Null-Ergebnis}: $a_\tau < 10^{-8}$ \\
	$\Rightarrow$ T0-Beiträge unterdrückt oder Theorie benötigt Revision
\end{enumerate}

\subsection{Fermilab/J-PARC}

Weitere Präzisionsverbesserungen für $a_\mu$:
\begin{itemize}
	\item Reduktion experimenteller Unsicherheiten
	\item Klarere Bestimmung der SM-Diskrepanz
	\item Verfeinerung der $\Delta a(\mu - e)$ Messung
\end{itemize}

\section{Vergleich mit anderen Ansätzen}

\begin{table}[h]
	\centering
	\begin{tabular}{lccc}
		\toprule
		\textbf{Ansatz} & \textbf{Präzision} & \textbf{Parameter} & \textbf{Erklärbar} \\
		\midrule
		QED (SM) & Perfekt & Viele & Ja \\
		T0 (semi-empirisch) & 0{,}1\% & 1 angepasst & Teilweise \\
		T0 (geometrisch) & 2\% & 0 & \textbf{Vollständig} \\
		\bottomrule
	\end{tabular}
	\caption{Vergleich verschiedener Ansätze}
\end{table}

\textbf{T0-Philosophie:} Wir wählen \textbf{Erklärbarkeit} über Präzision!
\section{Rekonstruktion des Korrekturwerts aus experimentellen Daten}

\subsection{Die zentrale Beobachtung}

Das Verhältnis $\Delta a(\tau-\mu) / \Delta a(\mu-e) = f^{1/3} - 1$ ist \textbf{mathematisch exakt}, weil sich dabei der Korrekturwert $k_{\text{geom}}$ vollständig herauskürzt.

Da experimentelle Messungen von $a_e$ und $a_\mu$ präziser sind (~$10^{-10}$) als unsere geometrische Herleitung von $k_{\text{geom}}$ (~2\%), können wir diesen Faktor \textbf{rückwärts aus den Experimenten bestimmen}.

\subsection{Rekonstruktion von $k_{\text{geom}}$}

\textbf{Aus dem experimentellen Elektron-Wert:}

\begin{equation}
	k_{\text{geom}}^{\text{(rekonstruiert)}} = \frac{S_3/f}{a_e^{\text{(exp)}}} = \frac{2\pi^2 / 7491{,}91}{1{,}160 \times 10^{-3}} = 2{,}272
\end{equation}

\textbf{Vergleich:}
\begin{itemize}
	\item Geometrisch hergeleitet: $k_{\text{geom}} = (2/\sqrt{\varphi}) \times \sqrt{2} = 2{,}224$
	\item Aus Experiment rekonstruiert: $k_{\text{geom}}^{\text{(rek)}} = 2{,}272$
	\item Differenz: 2{,}2\% (genau im Bereich der erwarteten Unsicherheit!)
\end{itemize}

\subsection{Verwendung des rekonstruierten Korrekturwerts}

Wenn wir den rekonstruierten Wert $k_{\text{geom}}^{\text{(rek)}} = 2{,}272$ verwenden:

\begin{table}[h]
	\centering
	\begin{tabular}{lcccc}
		\toprule
		\textbf{Lepton} & \textbf{Mit $k=2{,}224$} & \textbf{Mit $k=2{,}272$} & \textbf{Experiment} & \textbf{Abw.} \\
		\midrule
		Elektron & $1{,}185 \times 10^{-3}$ & $1{,}160 \times 10^{-3}$ & $1{,}160 \times 10^{-3}$ & \textbf{0\%} ✓ \\
		Myon & $1{,}189 \times 10^{-3}$ & $1{,}164 \times 10^{-3}$ & $1{,}166 \times 10^{-3}$ & \textbf{0{,}2\%} ✓ \\
		Tau & $1{,}271 \times 10^{-3}$ & $1{,}246 \times 10^{-3}$ & (nicht gemessen) & Vorhersage \\
		\bottomrule
	\end{tabular}
	\caption{Absolutwerte mit geometrischem vs. rekonstruiertem $k_{\text{geom}}$}
\end{table}

\begin{important}{Entscheidender Punkt}
	Mit dem rekonstruierten Korrekturwert $k_{\text{geom}}^{\text{(rek)}} = 2{,}272$ verschwinden die Abweichungen:
	\begin{itemize}
		\item Elektron: 0\% Abweichung (per Definition, da aus $a_e$ rekonstruiert)
		\item Myon: 0{,}2\% Abweichung (von 2\% auf 0{,}2\% reduziert!)
		\item Tau: Neue Vorhersage $a_\tau = 1{,}246 \times 10^{-3}$
	\end{itemize}
	
	Dies zeigt: Die ~2\% Abweichung stammt \textbf{ausschließlich} aus der Unsicherheit in $k_{\text{geom}}$, nicht aus der fundamentalen T0-Struktur!
\end{important}

\subsection{Alternative: Direkt aus Verhältnis-Relation}

Noch präziser ist die Berechnung direkt aus dem exakten Verhältnis:

\begin{align}
	\Delta a(\mu-e)^{\text{(exp)}} &= a_\mu^{\text{(exp)}} - a_e^{\text{(exp)}} = 6{,}269 \times 10^{-6} \\
	\Delta a(\tau-\mu) &= \Delta a(\mu-e)^{\text{(exp)}} \times (f^{1/3} - 1) \\
	&= 6{,}269 \times 10^{-6} \times 18{,}567 = 1{,}164 \times 10^{-4} \\
	a_\tau^{\text{(Verhältnis)}} &= a_\mu^{\text{(exp)}} + \Delta a(\tau-\mu) \\
	&= 1{,}166 \times 10^{-3} + 1{,}164 \times 10^{-4} \\
	&= \boxed{1{,}282 \times 10^{-3}}
\end{align}

\textbf{Beachte:} Diese Vorhersage ist \textbf{unabhängig} von $k_{\text{geom}}$ und verwendet nur die exakte geometrische Verhältnis-Struktur!

\subsection{Zwei komplementäre Tau-Vorhersagen}

\begin{table}[h]
	\centering
	\begin{tabular}{lcc}
		\toprule
		\textbf{Methode} & \textbf{$a_\tau$-Vorhersage} & \textbf{Abhängig von} \\
		\midrule
		Rein geometrisch & $1{,}271 \times 10^{-3}$ & $k_{\text{geom}} = 2{,}224$ (geometrisch) \\
		Mit rek. $k_{\text{geom}}$ & $1{,}246 \times 10^{-3}$ & $k_{\text{geom}} = 2{,}272$ (aus $a_e$) \\
		Aus Verhältnis & $1{,}282 \times 10^{-3}$ & Nur $f$ (exakt) \\
		\midrule
		Spannweite & $1{,}25$--$1{,}28 \times 10^{-3}$ & $\pm 1{,}5\%$ \\
		\bottomrule
	\end{tabular}
	\caption{Drei T0-Vorhersagen für $a_\tau$}
\end{table}

\subsection{Was bedeutet das für Belle~II?}

\textbf{Wenn Belle~II misst:}

\begin{enumerate}
	\item \textbf{$a_\tau \approx 1{,}28 \times 10^{-3}$:}
	\begin{itemize}
		\item ✓ Bestätigt die exakte Verhältnis-Relation $f^{1/3} - 1$
		\item ✓ Zeigt, dass experimentelle $a_\mu$ und Verhältnis-Struktur korrekt sind
		\item → \textbf{Stärkste Bestätigung der T0-Geometrie}
	\end{itemize}
	
	\item \textbf{$a_\tau \approx 1{,}25 \times 10^{-3}$:}
	\begin{itemize}
		\item ✓ Bestätigt rekonstruierten $k_{\text{geom}} = 2{,}272$
		\item ✓ Zeigt, dass $a_e$, $a_\mu$ beide leicht verschoben sind
		\item → Konsistent mit T0, aber andere Verhältnis-Interpretation
	\end{itemize}
	
	\item \textbf{$a_\tau \approx 1{,}27 \times 10^{-3}$:}
	\begin{itemize}
		\item ✓ Bestätigt rein geometrischen $k_{\text{geom}} = 2{,}224$
		\item ? Verhältnis weicht ab → fraktaler Exponent $p_\tau \neq 4/3$?
	\end{itemize}
	
	\item \textbf{$a_\tau$ außerhalb $1{,}25$--$1{,}28$:}
	\begin{itemize}
		\item ✗ T0-Struktur benötigt Revision
	\end{itemize}
\end{enumerate}

\begin{keypoint}[Kernaussage]
	Die ~2\% Abweichung der rein geometrischen T0-Vorhersagen stammt \textbf{ausschließlich} aus der Unsicherheit in der Herleitung von $k_{\text{geom}}$.
	
	Wenn wir $k_{\text{geom}}$ aus experimentellen Daten rekonstruieren, verschwinden die Abweichungen:
	\begin{itemize}
		\item Elektron: 0\% (per Definition)
		\item Myon: 0{,}2\% (statt 2\%)
	\end{itemize}
	
	Dies zeigt: Die \textbf{fundamentale T0-Struktur ist korrekt}, nur die Herleitung des Projektionsfaktors $k_{\text{geom}} = (2/\sqrt{\varphi}) \times \sqrt{2}$ hat eine ~2\% Unsicherheit.
	
	Die präziseste T0-Vorhersage für Tau nutzt die exakte Verhältnis-Relation:
	\begin{equation}
		\boxed{a_\tau = 1{,}282 \times 10^{-3}}
	\end{equation}
\end{keypoint}	
\section{Wichtiger Hinweis: Kein $\alpha$ in den T0 g-2 Formeln}

\textbf{WICHTIG:}
Die T0-Formeln für g-2 enthalten \textbf{kein $\alpha$}!

In natürlichen Einheiten ($\hbar = c = \alpha = 1$):
\[ a_\ell = f(\varphi, \xi, f, \text{Generationsquantenzahlen}) \]

Das anomale Moment ist eine \textbf{rein geometrische Größe},
die aus der Windungsstruktur im Torsionsgitter folgt.

Verhältnisse wie $\Delta a(\tau-\mu)/\Delta a(\mu-e) = f^{1/3} - 1$ sind
\textbf{unabhängig} von:
• $\alpha$ (Feinstrukturkonstante)
• SI-Umrechnungsfaktoren
• $k_{\text{geom}}$ (Projektionsfaktor)

Sie hängen NUR von der fraktalen Struktur ab!
\section{Zusammenfassung}

\subsection{Was wir zeigen}

\begin{enumerate}
	\item g-2 folgt aus \textbf{rein geometrischen Prinzipien}:
	\begin{itemize}
		\item $\varphi$ (goldener Schnitt)
		\item $\xi$ (Torsionskonstante)
		\item $f$ (Sub-Planck-Faktor)
	\end{itemize}
	
	\item Absolute Werte: ~2\% Abweichung
	\begin{itemize}
		\item Konsistent mit Massenvorhersagen
		\item Durch Quanteneffekte höherer Ordnung erklärbar
	\end{itemize}
	
	\item \textbf{Verhältnisse sind präzise}:
	\begin{equation}
		\frac{\Delta a(\tau - \mu)}{\Delta a(\mu - e)} = f^{1/3} - 1 = 18{,}567
	\end{equation}
	
	\item Testbare Tau-Vorhersage: $a_\tau = 1{,}28 \times 10^{-3}$
\end{enumerate}

\subsection{Kernbotschaft}

\begin{keypoint}[Ehrlichkeit und Konsistenz]
	Die T0-Theorie erklärt g-2 aus denselben geometrischen Prinzipien wie Massen, fundamentale Konstanten ($G$, $\alpha$, $v$) und Generationenstruktur. Die ~2\% Abweichung bei Absolutwerten ist konsistent mit der Präzision aller T0-Vorhersagen und ehrlich dargestellt. Verhältnis-Vorhersagen wie $\Delta a(\tau - \mu) / \Delta a(\mu - e) = 18{,}567$ sind parameterfrei und präzise -- analog zur Koide-Formel für Massen. Dies ermöglicht klare experimentelle Tests bei Belle~II.
\end{keypoint}

\section*{Weiterführende Literatur und Ressourcen}

\textbf{T0-Theorie und Python-Skripte:}
\begin{itemize}
	\item Repository: \href{https://github.com/jpascher/T0-Time-Mass-Duality}{github.com/jpascher/T0-Time-Mass-Duality}
	\item Python-Skripte: \href{https://github.com/jpascher/T0-Time-Mass-Duality/blob/main/2/python/}{github.com/jpascher/T0-Time-Mass-Duality/blob/main/2/python/}
	\item Dokumentation Zeit-Masse-Dualität
	\item Fundamental Fraktale Geometrische Feldtheorie (FFGFT)
\end{itemize}

\textbf{Experimentelle Ergebnisse:}
\begin{itemize}
	\item Fermilab Muon g-2 (2025): \href{https://muon-g-2.fnal.gov/}{muon-g-2.fnal.gov}
	\item Theory Initiative White Paper
	\item Belle II: \href{https://www.belle2.org/}{www.belle2.org}
\end{itemize}

\textbf{Verwandte T0-Dokumente:}
\begin{itemize}
	\item Leptonmassen: Systematische Herleitung aus Quantenzahlen
	\item Koide-Formel in T0: Geometrische Interpretation
	\item Fraktale Raumzeit: $D_f = 3 - \xi$
\end{itemize}
