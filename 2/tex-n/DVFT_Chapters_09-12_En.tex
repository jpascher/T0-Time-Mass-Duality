\documentclass[12pt,a4paper]{article}
\usepackage[utf8]{inputenc}
\usepackage[T1]{fontenc}
\usepackage[english]{babel}
\usepackage{amsmath,amsfonts,amssymb}
\usepackage{physics}
\usepackage{geometry}
\usepackage{hyperref}
\usepackage{fancyhdr}
\usepackage{graphicx}
\usepackage{cite}
\usepackage{tcolorbox}
\usepackage{enumitem}

\geometry{margin=1in}
\pagestyle{fancy}
\fancyhf{}
\fancyhead[L]{Dynamic Vacuum Field Theory}
\fancyhead[R]{Adapted to T0 Theory}
\fancyfoot[C]{\thepage}

\title{Dynamic Vacuum Field Theory Adapted to T0 Theory\\
\large Chapters 9--12}
\author{Based on work by Satish B. Thorwe\\
Adapted to T0 Theory Framework}
\date{December 25, 2025}

\begin{document}

\maketitle

\begin{tcolorbox}[colback=blue!5!white,colframe=blue!75!black,title=T0 Theory Framework]
This document presents Dynamic Vacuum Field Theory (DVFT) adapted to align with T0 Theory as its fundamental basis. T0 Theory provides the conclusive core framework with:
\begin{itemize}
\item Time-mass duality: $T(x,t) \cdot m(x,t) = 1$
\item Fundamental parameter: $\xi = \frac{4}{3} \times 10^{-4}$
\item Simplified Lagrangian: $\mathcal{L} = \varepsilon (\partial \Delta m)^2$
\item Extended Lagrangian including time-field interactions
\item Node dynamics for particles and spin
\end{itemize}

DVFT is reformulated as a phenomenological layer on T0, deriving its vacuum field $\Phi = \rho e^{i\theta}$ directly from T0 principles.
\end{tcolorbox}

\tableofcontents
\newpage

\section{STRONG, WEAK, AND DEEP FIELD PHYSICS}
\label{sec:ch09}

\subsection{Introduction}
Dynamic Vacuum Field Theory (DVFT) predicts distinct regimes of gravitational behavior determined by
the magnitude of the vacuum phase gradient
\[X = -g^{$\mu$ν} $\partial$\mu$\theta$ $\partial$ν$\theta$.\]

These regimes—strong field, weak field, deep field, and an ultra-deep cosmological regime—correspond
to different nonlinear responses of the vacuum. This chapter provides a unified description of vacuum
behavior from local strong-gravity environments to the largest cosmological scales where dark energy
dominates.
\subsection{Strong Field Regime (X >> a₀²)}
In high-acceleration environments such as near stellar surfaces, neutron stars, or black hole exteriors,
phase gradients are large. The vacuum response
\[L_X = $\partial$L_$\theta$/$\partial$X\]

approaches an almost constant value:
L_X $\approx\rho$₀/2.
Nonlinear terms in the Lagrangian,
International Journal for Multidisciplinary Research (IJFMR)
E-ISSN: 2582-2160 $\bullet$ Website: www.ijfmr.com $\bullet$ Email: editor@ijfmr.com
IJFMR250664112 Volume 7, Issue 6, November-December 2025 22
\[L_$\theta$ = ($\rho$₀/2) X - (η/(3 a_0^2)) X^{3/2} - $\Lambda$_v,\]

become negligible compared to the linear X term. In this limit DVFT reduces to the predictions of General
Relativity with an effective cosmological constant $\Lambda$_eff set by the residual vacuum term. Curvature is
dominated by the quasi-linear response of $\theta$, and conventional GR tests are satisfied.
\subsection{Weak Field Regime (X ~ a₀²)}
As accelerations approach a_0, nonlinear vacuum effects begin to contribute. Here X is comparable to a_0^2
and the X^{3/2} correction in the Lagrangian becomes relevant. The response function
\[L_X = $\rho$₀/2 - (η/(2 a_0^2)) X^{1/2}\]

departs from a constant and begins to depend on the local phase gradient. Observable consequences
include:
\begin{itemize}
  \item small deviations from Newtonian potential in extended systems,
  \item mild corrections to post-Newtonian parameters,
  \item subtle modifications to gravitational lensing and Shapiro delay.
\end{itemize}
This regime provides a smooth transition between pure GR behavior in strong fields and the deep field
behavior that governs galactic outskirts.
\subsection{Deep Field Regime (X << a₀², Galactic Scale)}
The deep field regime governs low-acceleration environments such as the outskirts of spiral galaxies. In
this limit phase gradients are small, but the nonlinear X^{3/2} term dominates the response of the vacuum.
Integrating out the amplitude $\rho$ and enforcing scale invariance leads to an effective vacuum energy density
scaling as:
E_vac ∝ |$\nabla$\phi$|^3,
where $\phi$ is the gravitational potential related to $\theta$ through the background dynamic vacuum field. The
resulting field equation in the non-relativistic limit becomes:
\[$\nabla$ · [(|$\nabla$\phi$|/a_0) $\nabla$\phi$] = 4$\pi$G $\rho$_b,\]

where $\rho$_b is the baryonic matter density. For spherical systems this gives:
\[g^2(r) = a_0 g_N(r),\]
with g the true gravitational acceleration and g_N the Newtonian acceleration from baryons alone. This
produces:
\begin{itemize}
  \item flat rotation curves,
  \item \[the baryonic Tully–Fisher relation v_c⁴ = G M_b a_0,\]

  \item no requirement for dark matter halos.
\end{itemize}
Thus the deep field regime is responsible for MOND-like behavior emerging naturally from DVFT
vacuum microphysics.
\subsection{Ultra-Deep Cosmological Regime (g << a₀, Dark Energy Scale)}
On scales comparable to or larger than the Hubble radius, typical gravitational accelerations become far
smaller than a_0. In this ultra-deep regime, phase gradients are extremely small and the kinetic contributions
in L_$\theta$ are suppressed relative to the residual vacuum term. The vacuum field approaches:
$\Phi\approx\rho$_$\infty$ e^{i $\mu$ t},
with $\rho$_$\infty$ a nearly homogeneous amplitude and $\mu$ the dynamic vacuum field frequency. The effective
energy density and pressure of the vacuum become:
$\epsilon$_vac $\approx\rho$_$\infty$^2 $\mu$^2 + V($\rho$_$\infty$),
p_vac $\approx\rho$_$\infty$^2 $\mu$^2 - V($\rho$_$\infty$),
International Journal for Multidisciplinary Research (IJFMR)
E-ISSN: 2582-2160 $\bullet$ Website: www.ijfmr.com $\bullet$ Email: editor@ijfmr.com
IJFMR250664112 Volume 7, Issue 6, November-December 2025 23
where V($\rho$) is the vacuum potential. For parameter choices where V($\rho$_$\infty$) dominates over the kinetic term,
one obtains:
p_vac $\approx$ -$\epsilon$_vac,
which corresponds to an equation of state parameter w $\approx$ -1. This is the dark-energy-like regime of DVFT:
the universe is driven by residual dynamic vacuum field energy and the nearly constant vacuum potential.
In this ultra-deep regime:
\begin{itemize}
  \item X $\rightarrow$ 0,
  \item L_X $\rightarrow\rho$₀/2,
  \item the stress–energy tensor of $\theta$ reduces to an effective cosmological constant term,
  \item the Friedmann equations predict accelerated expansion.
\end{itemize}
Thus, dark energy is not an independent fluid but the asymptotic vacuum state of $\Phi$ when typical
gravitational gradients fall far below a_0 on cosmological scales.
\subsection{Transitions Across Scales}
The three local regimes (strong, weak, deep) and the ultra-deep cosmological regime are not separate
theories; they are different limits of the same underlying dynamics controlled by X and the parameters ($\rho$₀,
η, a_0, $\Lambda$_v). As a characteristic acceleration in a system changes, the vacuum smoothly interpolates
between:
\begin{itemize}
  \item GR-like behavior in compact objects and Solar System tests,
  \item modified dynamics in galaxies (deep field),
  \item effective dark energy at horizon-scale averages (ultra-deep field).
\end{itemize}
The governing equation
\[$\nabla$_$\mu$ (L_X $\nabla$^$\mu$ $\theta$) = 0\]

determines how the phase field adjusts across these regimes. Small, local systems never probe the ultradeep vacuum; galaxies probe the deep-field regime; the universe as a whole samples the full vacuum
potential and residual dynamic vacuum field energy.
\subsection{Implications for Cosmology and Structure Formation}
Because the same Lagrangian L_$\theta$ governs all regimes, DVFT ties together:
\begin{itemize}
  \item galactic rotation curves,
  \item cluster dynamics,
  \item cosmic acceleration,
  \item the absence of singularities,
  \item with a single set of vacuum parameters. Structure formation proceeds in a background where:
  \item early universe: kinetic and potential terms of $\Phi$ drive inflation-like expansion,
  \item intermediate epochs: matter dominates and deep-field corrections shape halo dynamics,
  \item late universe: ultra-deep regime emerges, and dark-energy-like behavior dominates.
\end{itemize}
In contrast to $\Lambda$CDM, where dark matter and dark energy are independent components, DVFT describes
both as manifestations of one vacuum field, viewed in different acceleration regimes.
\subsection{Summary}
DVFT organizes gravitational behavior into four coherent regimes:
\begin{itemize}
  \item Strong field: GR limit, X >> a_0^2, linear response, compact objects.
  \item Weak field: transitional, X ~ a_0^2, small nonlinear corrections.
  \item Deep field: galactic scale, X << a_0^2 but gradients still relevant, g^2 = a_0 g_N, no dark matter.
\end{itemize}
International Journal for Multidisciplinary Research (IJFMR)
E-ISSN: 2582-2160 $\bullet$ Website: www.ijfmr.com $\bullet$ Email: editor@ijfmr.com
IJFMR250664112 Volume 7, Issue 6, November-December 2025 24
\begin{itemize}
  \item Ultra-deep cosmological field: g << a_0 on horizon scales, residual vacuum energy acts as dark energy (w
\end{itemize}
$\approx$ -1).
This regime structure is not an artificial phenomenology; it is the natural consequence of a single dynamic
vacuum field Lagrangian. As a result, DVFT provides a unified physical explanation for local gravity tests,
galaxy dynamics, and late-time cosmic acceleration within one coherent framework.

\newpage

\section{DARK ENERGY REINTERPRETATION}
\label{sec:ch10}

\subsection{Introduction}
This document presents a strict DVFT-based derivation of dark energy, with no reference to external darkenergy models. The goal is to show how cosmic acceleration arises solely from the vacuum amplitude $\rho$
and its microphysical potential U($\rho$).
We derive the full equations for DVFT dark energy, specify U($\rho$) from the DVFT micro-lattice model,
and compare DVFT predictions directly with observed cosmological values.
Fundamental DVFT vacuum field:
\[$\Phi$(x,t) = $\rho$(x,t) e^{i$\theta$(x,t)}.\]

The universe’s large-scale behavior emerges from the homogeneous evolution of $\rho$(t), while $\theta$(t) controls
quantum-phase structure.
\subsection{DVFT Vacuum Lagrangian in a Homogeneous Universe}
From DVFT microphysics, the effective continuum vacuum Lagrangian is:
\[𝓛_vac = (A_$\rho$/2)($\partial$_t $\rho$)^2 - (B_$\rho$/2)|$\nabla$\rho$|^2 + (A_$\theta$/2)$\rho$^2($\partial$_t $\theta$)^2 - (B_$\theta$/2)$\rho$^2|$\nabla$\theta$|^2 - U($\rho$).\]
\[For a homogeneous FRW universe ($\rho$(t), $\theta$(t), $\nabla$\rho$ = $\nabla$\theta$ = 0):\]

\[𝓛_hom = (A_$\rho$/2)$\rho$̇^2 + (A_$\theta$/2)$\rho$^2 $\theta$̇^2 - U($\rho$).\]

All cosmological dark-energy effects will arise directly from this expression. No additional fluids or fields
are introduced.
\subsection{Vacuum Energy Density and Pressure from DVFT}
Define kinetic energy of the vacuum amplitude–phase system:
\[K = (A_$\rho$/2)$\rho$̇^2 + (A_$\theta$/2)$\rho$^2 $\theta$̇^2.\]

DVFT vacuum behaves as a perfect fluid with:
\[$\rho$_DVFT = K + U($\rho$),p_DVFT = K - U($\rho$).\]

The effective equation-of-state is:
\[w_DVFT = (K - U) / (K + U).\]
Important limits:
\begin{itemize}
  \item K ≪ U $\rightarrow$ w $\rightarrow$ -1 (dark-energy–like)
  \item K ~ U $\rightarrow$ -1 < w < -1/3 (dynamical dark energy)
  \item K ≫ U $\rightarrow$ w $\rightarrow$ +1 (stiff fluid; irrelevant today)
\end{itemize}
\subsection{Dark-Energy Evolution Equation in DVFT}
Varying the homogeneous action yields the amplitude evolution equation:
\[A_$\rho$($\rho$̈+ 3H$\rho$̇) - A_$\theta$\rho$ $\theta$̇^2 + dU/d$\rho$ = 0,\]

where:
H = ȧ/a (Hubble parameter).
At late times, the cosmic phase tends to freeze on large scales ($\theta$̇ $\approx$ 0), reducing the equation to:
\[A_$\rho$($\rho$̈+ 3H$\rho$̇) + dU/d$\rho$ = 0.\]

International Journal for Multidisciplinary Research (IJFMR)
E-ISSN: 2582-2160 $\bullet$ Website: www.ijfmr.com $\bullet$ Email: editor@ijfmr.com
IJFMR250664112 Volume 7, Issue 6, November-December 2025 25
This is the DVFT dark-energy equation: the cosmic vacuum amplitude $\rho$ evolves in its potential U($\rho$)
under Hubble damping.
\subsection{Microphysical Form of U(ρ) in DVFT}
DVFT is based on a micro-lattice vacuum with local Hamiltonian:
\[H_loc = p_$\rho$^2/(2M_$\rho$) + p_$\theta$^2/(2M_$\theta$ $\rho$^2) + U_loc($\rho$).\]

DVFT microphysics requires U_loc($\rho$) to have:
\begin{itemize}
  \item a stable minimum at $\rho$₀ (preferred vacuum amplitude),
  \item positive curvature at $\rho$₀ (vacuum stiffness),
  \item anharmonic corrections stabilizing deviations.
\end{itemize}
Thus the coarse-grained continuum potential becomes:
\[U($\rho$) = $\Lambda$₀ + (κ/2)($\rho$ - $\rho$₀)^2 + ($\lambda$/4)($\rho$ - $\rho$₀)⁴ + …\]

Where:
\begin{itemize}
  \item \[$\Lambda$₀ = microphysical residual vacuum energy density,\]

  \item κ = vacuum amplitude compressibility,
  \item \[$\lambda$ = higher-order stabilization.\]

\end{itemize}
Near the minimum:
U($\rho$) $\approx\Lambda$₀ + (1/2)m_$\rho$^2 ($\rho$ - $\rho$₀)^2,
\[with m_$\rho$^2 = κ/A_$\rho$.\]

This U($\rho$) is not arbitrary; it is derived from DVFT vacuum elasticity and amplitude stability.
\subsection{DVFT Explanation for Dark Energy on Cosmic Scales}
DVFT predicts dark energy because:
\subsection{The vacuum amplitude ρ has a preferred value ρ₀ (microphysical equilibrium).}
\subsection{The local vacuum energy density U(ρ₀) = Λ₀ is *not zero*.}
\subsection{On large scales, ρ(t) approaches ρ₀ and remains nearly constant due to strong Hubble damping.}
\subsection{Therefore, the vacuum behaves like a nearly constant energy density with w ≈ -1.}
The measured value:
$\rho$_$\Lambda\approx$ 7 $\times$ 10⁻^2⁷ kg/m^3
$\Omega$_$\Lambda\approx$ 0.70–0.75
matches DVFT if:
\[$\Lambda$₀ = U($\rho$₀) $\approx$ 0.7 $\rho$_crit.\]

Thus dark energy is the “elastic offset energy of the vacuum amplitude”
\subsection{Why U(ρ) Is Negligible on Solar and Galactic Scales}
A uniform vacuum energy density produces acceleration:
g_vac(r) $\approx$ (8$\pi$G/3) $\rho$_$\Lambda$ r.
At solar scale (r = 1 AU):
g_vac ~ 10⁻^2⁴ m/s^2 (negligible).
At galactic scale (r = 10 kpc):
g_vac ~ 10⁻¹⁶ m/s^2 (still negligible).
Thus:
\begin{itemize}
  \item Local dynamics are governed by $\nabla$\rho$ and matter coupling, not U($\rho$).
  \item Vacuum elasticity only influences cosmic expansion where r ~ gigaparsecs.
\end{itemize}
DVFT cleanly separates:
\begin{itemize}
  \item Galactic gravity: amplitude gradients $\nabla$\rho$ dominate.
\end{itemize}
International Journal for Multidisciplinary Research (IJFMR)
E-ISSN: 2582-2160 $\bullet$ Website: www.ijfmr.com $\bullet$ Email: editor@ijfmr.com
IJFMR250664112 Volume 7, Issue 6, November-December 2025 26
\begin{itemize}
  \item Cosmological acceleration: homogeneous U($\rho$₀) dominates.
\end{itemize}
\subsection{Numerical Comparison with Observations}
Given:
\begin{itemize}
  \item H_0 $\approx$ 67–70 km/s/Mpc,
  \item \[$\rho$_crit = 3H_0^2/(8$\pi$G),\]

  \item $\Omega$_$\Lambda\approx$ 0.7,
\end{itemize}
DVFT requires:
\[U($\rho$₀) = $\Lambda$₀ $\approx$ 0.7 $\rho$_crit.\]

This matches observational values from CMB, BAO, and SN data.
Moreover, if $\rho$(t) is still slowly relaxing toward $\rho$₀, then:
w_DVFT $\approx$ -1 + 2K/U,
allowing mild deviations from -1 (observationally allowed), and potentially matching evolving darkenergy hints from DESI.
\subsection{Summary}
From strict DVFT principles, dark energy arises from the vacuum amplitude’s microphysical potential:
\[U($\rho$) = $\Lambda$₀ + (κ/2)($\rho$ - $\rho$₀)^2 + ($\lambda$/4)($\rho$ - $\rho$₀)⁴ + …\]

Key results:
\begin{itemize}
  \item \[$\rho$_DVFT = K + U($\rho$), p_DVFT = K - U($\rho$).\]

  \item \[w_DVFT = (K - U)/(K + U).\]

  \item \[Vacuum amplitude evolves via A_$\rho$($\rho$̈+ 3H$\rho$̇) + U'($\rho$) = 0.\]

  \item On cosmic scales, $\rho\approx\rho$₀ ⇒ w $\approx$ -1, matching dark-energy observations.
  \item On solar/galactic scales, U($\rho$) is negligible; $\nabla$\rho$ dominates gravity.
  \item DVFT dark energy matches measured values $\Omega$_$\Lambda\approx$ 0.7 and w $\approx$ -1 with no additional fields.
\end{itemize}
Thus DVFT naturally unifies local gravity and cosmic acceleration using only vacuum amplitude physics.

\newpage

\section{BLACK HOLE INTERIOR PREDICTION}
\label{sec:ch11}

This chapter presents a complete description of black hole interiors in the Dynamic Vacuum Field
Theory(DVFT). DVFT replaces the classical singularity of General Relativity (GR) with a finite-density
quantum vacuum core, using a nonlinear phase field $\theta$. Both the mathematical structure and the physical
interpretation are provided.
\subsection{DVFT Overview}
DVFT treats spacetime as a quantum vacuum medium described by a complex order parameter:
\[$\Phi$ = $\rho$ e^{i$\theta$}\]

Gravity arises from dynamic vacuum field with amplitude $\rho$ and phase $\theta$. The Lagrangian contains
nonlinear kinetic terms:
\[L_$\theta$ = -$\Lambda$_v + ($\rho$_0/2)X - (η/(3 a_0^2)) X^{3/2}with X = -g^{$\mu$ν} $\partial$_$\mu$\theta$ $\partial$_ν$\theta$.\]

At large accelerations (g >> a_0), DVFT reduces to GR. At small accelerations (g << a_0), nonlinearities
appear.
\subsection{Black Hole Metric and Field Ansatz}
We use the standard static spherically symmetric metric:
\[ds^2 = -e^{2$\Phi$(r)}dt^2 + dr^2/(1 - 2Gm(r)/r) + r^2 d$\Omega$^2.\]

International Journal for Multidisciplinary Research (IJFMR)
E-ISSN: 2582-2160 $\bullet$ Website: www.ijfmr.com $\bullet$ Email: editor@ijfmr.com
IJFMR250664112 Volume 7, Issue 6, November-December 2025 27
\[The vacuum phase depends only on radius: $\theta$ = $\theta$(r). The kinetic invariant becomes:X = -(1 - 2Gm(r)/r) $\theta$'(r)^2.\]
From the k-essence stress-energy tensor:
\[T_{$\mu$ν} = 2 L_X $\partial$_$\mu$\theta$ $\partial$_ν$\theta$ - g_{$\mu$ν} L_$\theta$\]

\subsection{Stress-Energy Components}
Define:
\[L_$\theta$ = -$\Lambda$_v + ($\rho$_0/2)X - (η/(3 a_0^2)) X^{3/2},\]

\[L_X = $\partial$L_$\theta$/$\partial$X = $\rho$_0/2 - (η/(2a_0^2)) X^{1/2}.Energy density and pressures:$\rho$ = L_$\theta$,p_t = $\rho$,p_r = 2 L_X X - L_$\theta$.\]
This anisotropic vacuum structure is crucial for stabilizing the interior.
\subsection{Vacuum Saturation Mechanism}
\[The scalar field equation $\nabla$_$\mu$(L_X $\partial$^$\mu$\theta$)=0 is satisfied in the core when:L_X(X_0) = 0.Setting L_X=0 gives:X_0^{1/2} = ($\rho$_0 a_0^2)/η.\]
Thus, the vacuum phase reaches a 'saturation' point X_0, limiting further compression. The core energy
density becomes finite:
\[$\rho$_core = -$\Lambda$_v + ($\rho$_0^3 a_0⁴)/(6 η^2).\]

\subsection{Core Geometry}
\[With $\rho$ = $\rho$_core = constant, the Einstein equation gives a de Sitter–like interior:m(r) = (4$\pi$/3)$\rho$_core r^3,\]
\[1 - 2Gm(r)/r = 1 - (8$\pi$G/3)$\rho$_core r^2.Thus, the interior metric is:\]

ds^2_core $\approx$ -[1 - ($\Lambda$_eff r^2)/3] dt^2 + dr^2/[1 - ($\Lambda$_eff r^2)/3] + r^2 d$\Omega$^2,
\[with $\Lambda$_eff = 8$\pi$G $\rho$_core.\]

There is no singularity; curvature remains finite.
\subsection{Matching to Exterior Geometry}
For r > r_c (core radius), X << X_0 and nonlinear effects vanish. DVFT reduces to GR:
ds^2 $\approx$ Schwarzschild metric.
Matching conditions ensure:
\[g_{tt}(core) = g_{tt}(ext),\]
\[g_{rr}(core) = g_{rr}(ext).\]
Thus, DVFT describes a black hole with a GR exterior and a finite-density vacuum core interior.
\subsection{Physical Interpretation (Non-Mathematical)}
\begin{itemize}
  \item GR predicts infinite collapse. DVFT prevents this by saturating the vacuum phase.
  \item The black hole interior becomes a finite-size 'quantum core.'
  \item As mass falls in, both the horizon and the core radius increase.
  \item No singularity exists. Space cannot compress indefinitely.
  \item The final object is a quantum vacuum condensate, not a point of infinite density.
\end{itemize}
\subsection{Final Fate of a Black Hole in DVFT}
International Journal for Multidisciplinary Research (IJFMR)
E-ISSN: 2582-2160 $\bullet$ Website: www.ijfmr.com $\bullet$ Email: editor@ijfmr.com
IJFMR250664112 Volume 7, Issue 6, November-December 2025 28
Depending on parameters ($\rho$_0, η, a_0):
\subsection{Stable quantum object: evaporation slows, horizon stalls, core remains.}
\subsection{Horizon shrinks until it meets the core, leaving a compact vacuum star.}
\subsection{Complete evaporation: horizon vanishes; core dissolves smoothly.}
In all cases, there is no singularity and no information loss.
Conclusion
DVFT gives the first consistent picture of a black hole interior using a single phase field. It provides:
\begin{itemize}
  \item GR-like exterior geometry,
  \item A finite-density quantum core replacing the singularity,
  \item A mechanism for black hole growth and evolution,
  \item A plausible resolution of the information paradox.
\end{itemize}
This bridges the gap between GR and QFT by treating vacuum as a physical, compressible quantum
medium.

\newpage

\section{COSMOLOGY, BIG BANG, AND BIRTH OF THE UNIVERSE}
\label{sec:ch12}

This chapter presents a full cosmological formulation of the Dynamic Vacuum Field Theory(DVFT).
\[Under DVFT, the universe did not begin as a singularity but as a vacuum-phase transition from a nearzero amplitude pre-vacuum state to the stable dynamic vacuum field state described by the field $\Phi$ =\]
$\rho$(x)e^{i$\theta$(x)}. We show how DVFT naturally explains the Big Bang, inflation, cosmic expansion, dark
energy, cosmic horizon problems, and other fundamental mysteries of cosmology.
\subsection{Introduction}
Traditional cosmological models built on General Relativity confront a fundamental problem: they begin
with a singularity at t = 0 where curvature, density, and temperature diverge. This singularity eliminates
the possibility of explaining the physical origin of the universe, inflation, or the emergence of space itself.
DVFT replaces the singularity with a physically meaningful vacuum-phase defect, enabling a consistent
explanation of how the Big Bang occurred, what existed before it, and why the universe expanded so
rapidly.
\subsection{The Vacuum Field in Cosmology}
In cosmological symmetry, the vacuum field is homogeneous:
\[$\Phi$(t) = $\rho$(t) e^{i$\theta$(t)}\]

Here, $\rho$(t) is the vacuum amplitude determining vacuum energy density, and $\theta$(t) encodes dynamic vacuum
field.
The vacuum Lagrangian contributes energy density:
\[$\epsilon$_vac = (d$\rho$/dt)^2 + $\rho$^2 (d$\theta$/dt)^2 + V($\rho$)\]

and pressure:
\[p_vac = (d$\rho$/dt)^2 + $\rho$^2 (d$\theta$/dt)^2 - V($\rho$)\]

This becomes the source term in the Friedmann equations.
\subsection{DVFT Friedmann Equations}
The spacetime metric in a homogeneous universe is the FLRW form:
\[ds^2 = -dt^2 + a(t)^2 [ dr^2/(1-kr^2) + r^2 d$\Omega$^2 ]\]
In DVFT, the Friedmann equations become:
\[(da/dt)^2 / a^2 = (8$\pi$G/3) $\epsilon$_vac\]
\[d^2a/dt^2 / a = -(4$\pi$G/3)($\epsilon$_vac + 3p_vac)\]
International Journal for Multidisciplinary Research (IJFMR)
E-ISSN: 2582-2160 $\bullet$ Website: www.ijfmr.com $\bullet$ Email: editor@ijfmr.com
IJFMR250664112 Volume 7, Issue 6, November-December 2025 29
The evolution of $\rho$(t) and $\theta$(t) determines $\epsilon$_vac and p_vac.
Because the vacuum cannot diverge, $\epsilon$_vac remains finite even at the earliest times.
\subsection{Pre-Big-Bang Vacuum Phase}
Before the Big Bang, the vacuum field was in a near-zero amplitude state:
\begin{itemize}
  \item $\rho$(t) $\approx$ 0
  \item $\theta$(t) undefined or fluctuating
\end{itemize}
This state is energetically unstable. The vacuum potential:
\[V($\rho$) = $\lambda$ ($\rho$^2 - $\rho$₀^2)^2\]

\[encourages a phase transition toward the minimum at $\rho$ = $\rho$₀.\]

\subsection{The Vacuum Phase Transition (Big Bang Event)}
The Big Bang corresponds to the moment when the vacuum transitioned from the unstable state $\rho\approx$ 0 to
\[the stable dynamic vacuum field state $\rho$ = $\rho$₀. This transition releases energy, sets $\theta$(t) into coherent\]
oscillation, and generates an explosive increase in $\epsilon$_vac.
This triggers rapid expansion of the scale factor a(t).
\subsection{Inflation from Dynamics}
Inflation requires rapid acceleration of the universe. DVFT provides this because the vacuum-potential
plateau makes V($\rho$) nearly constant during the early evolution.
During the transition:
$\epsilon$_vac $\approx$ constant
Thus:
(da/dt)/a $\approx$ constant ⇒ exponential expansion
DVFT inflation ends naturally when $\rho$(t) settles near $\rho$₀ and $\theta$(t) becomes coherent.
\subsection{Reheating and Matter Creation}
Once the vacuum field settles into coherent dynamic vacuum field, oscillations of $\Phi$ transfer energy into
matter fields via interaction terms of the form:
\[L_int = -y |$\Phi$| ψ̄ψ\]

This generates particle–antiparticle pairs, radiation, and thermal energy. The universe becomes radiation
dominated.
\subsection{Origin of Space Expansion}
In GR, space expands, but no mechanism explains *why*. In DVFT, space expands because the vacuum
amplitude $\rho$(t) increases and the dynamic vacuum field becomes coherent. Vacuum energy determines
curvature, and a rapid change in vacuum energy produces rapid change in the scale factor.
\subsection{Removal of the Cosmological Singularity}
The divergence of curvature in GR arises because nothing limits density or curvature.
In DVFT, dynamics impose:
\begin{itemize}
  \item |d$\theta$/dt| $\leq\theta$_max
  \item $\rho$(t) finite
  \item V($\rho$) finite
  \item $\epsilon$_vac finite
\end{itemize}
The energy density never diverges. The curvature invariants remain finite. The Big Bang is replaced by a
finite, smooth vacuum phase transition. There is no singular point.
\subsection{Horizon Problem Resolved}
International Journal for Multidisciplinary Research (IJFMR)
E-ISSN: 2582-2160 $\bullet$ Website: www.ijfmr.com $\bullet$ Email: editor@ijfmr.com
IJFMR250664112 Volume 7, Issue 6, November-December 2025 30
The classical horizon problem asks why causally disconnected regions of the sky have the same
temperature.
In DVFT:
\begin{itemize}
  \item Before the Big Bang, the vacuum was nearly homogeneous
  \item The vacuum phase transition occurred everywhere simultaneously
  \item Vacuum-phase waves propagate at c, enforcing coherence
\end{itemize}
No superluminal mechanisms needed.
\subsection{Flatness Problem Resolved}
The vacuum phase transition drives rapid inflation, which smooths curvature.
This pushes the universe toward k = 0.
Thus flatness arises automatically.
\subsection{What Caused the Universe to Begin?}
In DVFT, the universe begins because the vacuum was unstable in its low-amplitude configuration. When
$\rho$ reached the critical threshold, the vacuum rolled down its potential to $\rho$₀, initiating dynamic vacuum
field and expansion. This is analogous to phase transitions in condensed-matter systems.
\subsection{What Expanded During the Big Bang?}
\begin{itemize}
  \item Not matter.
  \item Not energy.
  \item Not space as pure geometry.
\end{itemize}
What expanded was:
\begin{itemize}
  \item the vacuum amplitude $\rho$(t).
\end{itemize}
As $\rho$(t) increased, vacuum energy increased, forcing the metric to inflate. This is the physical meaning
behind the expansion of space.
\subsection{Dark Energy from Residual Dynamic vacuum field}
Today, the vacuum still pulsates with frequency $\mu$. If $\mu$ evolves slowly with time, or if the vacuum
amplitude slightly shifts, this yields a small, nearly constant vacuum energy density. This naturally
produces accelerated expansion of the universe without requiring a cosmological constant.
\subsection{Full Evolution Summary}
\begin{itemize}
  \item Pre-Big-Bang: $\rho\approx$ 0, incoherent vacuum
  \item Phase transition: $\rho$ grows, $\theta$ becomes coherent
  \item Inflation: V($\rho$) nearly constant
  \item Reheating: $\Phi$ couples to matter
  \item Radiation era
  \item Matter era
  \item Dark energy era: residual dynamic vacuum field
\end{itemize}
Conclusion
DVFT replaces the cosmological singularity with a physical vacuum-phase transition. It explains the origin
of the universe, inflation, expansion, dark energy, and smoothness of the cosmos using a single vacuum
field. This eliminates the inconsistencies of classical GR and provides a unified, microphysical picture of
cosmology.

\newpage


\section*{References and Notes}

This document is part of the DVFT-T0 integration project. For complete details on T0 Theory, refer to the main T0 documentation. DVFT content is based on the work by Satish B. Thorwe, adapted to align with T0 Theory framework.

\subsection*{Key Adaptations}
\begin{enumerate}
\item DVFT's vacuum field $\Phi(x) = \rho(x) e^{i\theta(x)}$ is derived from T0's $\Delta m(x,t)$
\item All DVFT parameters are expressed in terms of T0's $\xi$
\item Vacuum dynamics emerge from T0's time-mass duality
\item Field equations are grounded in T0's extended Lagrangian
\end{enumerate}

\end{document}
