\documentclass[12pt,a4paper]{article}
\usepackage[utf8]{inputenc}
\usepackage[T1]{fontenc}
\usepackage[ngerman]{babel}
\usepackage{amsmath}
\usepackage{amsfonts}
\usepackage{amssymb}
\usepackage{geometry}
\geometry{a4paper,left=2.5cm,right=2.5cm,top=2.5cm,bottom=2.5cm}
\usepackage{fancyhdr}
\usepackage{enumitem}
\usepackage{tcolorbox}
\usepackage{physics}
\usepackage{hyperref}

% Hyperref als eines der letzten Pakete laden
\hypersetup{
	unicode=true,
	pdfencoding=unicode,
	bookmarksopen=true
}

% Saubere PDF-Lesezeichen
\pdfstringdefDisableCommands{%
	\def\Lambda{Lambda}%
	\def\Delta{Delta}%
	\def\approx{etwa}%
	\def\Sigma{Sigma}%
	\def\eta{eta}%
	\def\psi{psi}%
}





\title{Kapitel 26: Lösung der Baryonischen Asymmetrie}
\author{}
\date{}

\begin{document}

\maketitle

\section{Kapitel 26: Lösung der Baryonischen Asymmetrie }
	
	Das beobachtete Universum enthält weit mehr Materie als Antimaterie, quantifiziert durch das Baryon-zu-Photon-Verhältnis \(\eta_B \approx 6 \times 10^{-10}\).
	
	Das Standardmodell kann diesen Wert nicht erklären. Seine erlaubten Quellen für Baryonzahl-Verletzung und CP-Verletzung sind um Größenordnungen zu klein.
	
	Fraktale T0-Anpassung: DVFTs Vakuumfeld \(\Phi(x,t) = \rho(x,t) e^{i\theta(x,t)}\) wird von T0s Zeit-Masse-Feldstruktur \(T(x,t) \cdot m(x,t) = 1\) abgeleitet. Baryonzahl, CP-Verletzung und Nicht-Gleichgewichtsdynamik entstehen aus T0s intrinsischer Asymmetrie in der Zeitfeld-Rotation.
	
	Baryonzahl-Verletzung aus topologischen Wicklungen im Zeitfeld. CP-Verletzung aus asymmetrischer Phasenrotation \(\delta_{CP} \sim \xi^2 \approx 10^{-8}\). Nicht-Gleichgewicht aus frühen Instabilitäten.
	
	Alle drei Sacharow-Bedingungen entstehen aus \(T(x,t) \cdot m(x,t) = 1\). \(\eta_B \sim \xi^4 \approx 10^{-14}\) – richtige Größenordnung nur aus \(\xi\).
	
	Testbare Vorhersagen für Neutrino-Experimente. Löst 50-Jahre-Mystery mit null neuen Parametern.
	
	Das Universum hat mehr Materie, weil T0s Zeitfeld asymmetrische topologische Übergänge durchlief.
	
	Schlussfolgerung: T0-Theorie liefert die erste vollständige, parameterfreie Erklärung der Baryonasymmetrie. Baryogenese ist Validierung, dass T0s Zeit-Masse-Feld das Universum regiert.

\end{document}
