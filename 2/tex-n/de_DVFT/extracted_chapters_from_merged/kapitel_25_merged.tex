\documentclass[12pt,a4paper]{article}
\usepackage[utf8]{inputenc}
\usepackage[T1]{fontenc}
\usepackage[ngerman]{babel}
\usepackage{amsmath}
\usepackage{amsfonts}
\usepackage{amssymb}
\usepackage{geometry}
\geometry{a4paper,left=2.5cm,right=2.5cm,top=2.5cm,bottom=2.5cm}
\usepackage{fancyhdr}
\usepackage{enumitem}
\usepackage{tcolorbox}
\usepackage{physics}
\usepackage{hyperref}

% Hyperref als eines der letzten Pakete laden
\hypersetup{
	unicode=true,
	pdfencoding=unicode,
	bookmarksopen=true
}

% Saubere PDF-Lesezeichen
\pdfstringdefDisableCommands{%
	\def\Lambda{Lambda}%
	\def\Delta{Delta}%
	\def\approx{etwa}%
	\def\Sigma{Sigma}%
	\def\eta{eta}%
	\def\psi{psi}%
}





\title{Kapitel 25: Neutrinomassen-Problem gelöst}
\author{}
\date{}

\begin{document}

\maketitle

\section{Kapitel 25: Neutrinomassen-Problem gelöst }
	
	Dieses Dokument präsentiert die T0-begründete fraktale DVFT-Auflösung des Neutrinomassen-Problems.
	
	Vollständige T0-Lösung aller Neutrino-Rätsel: Neutrinos = reine Phasen-Anregungen von T0s \(\Phi = \rho e^{i\theta}\) Feld. Massen aus Phaseneigenmoden \(m_{\nu_i} = K_\nu(1 - \cos\theta_{\nu_i})\) mit \(K_\nu \ll K_e\). Drei Neutrinos aus SU(3)-Phasensymmetrie bei 120°-Intervallen. Winzige Massenskala \(m_\nu \sim 1/(\xi^3 m_0) \sim 0{,}01-0{,}05\) eV aus T0-Parametern. PMNS-Mischung aus Phasenmoden-Überlappungen. Majorana-Natur aus selbstkonjugierten Phasenoszillationen. Alles aus \(\xi = 4/3 \times 10^{-4}\) – null zusätzliche Parameter.
	
	T0 erklärt: Warum Neutrinos Masse haben (Phaseneigenwerte), warum Massen winzig sind (reine Phasenmoden), warum es drei gibt (SU(3)-Symmetrie), wie sie mischen (Phasenüberlappungen), was sie sind (selbstkonjugierte Phasenoszillationen), was ihre Massen sind (0{,}01-0{,}05 eV).
	
	Dies vervollständigt die Beschreibung des Leptonsektors, demonstrierend T0-Theorys Macht, langjährige Mysterien zu lösen.

\end{document}
