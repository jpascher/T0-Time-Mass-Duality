\documentclass[12pt,a4paper]{article}
\usepackage[utf8]{inputenc}
\usepackage[T1]{fontenc}
\usepackage[ngerman]{babel}
\usepackage{amsmath}
\usepackage{amsfonts}
\usepackage{amssymb}
\usepackage{geometry}
\geometry{a4paper,left=2.5cm,right=2.5cm,top=2.5cm,bottom=2.5cm}
\usepackage{fancyhdr}
\usepackage{enumitem}
\usepackage{tcolorbox}
\usepackage{physics}
\usepackage{hyperref}

% Hyperref als eines der letzten Pakete laden
\hypersetup{
	unicode=true,
	pdfencoding=unicode,
	bookmarksopen=true
}

% Saubere PDF-Lesezeichen
\pdfstringdefDisableCommands{%
	\def\Lambda{Lambda}%
	\def\Delta{Delta}%
	\def\approx{etwa}%
	\def\Sigma{Sigma}%
	\def\eta{eta}%
	\def\psi{psi}%
}





\title{Kapitel 15: Merkur-Perihel-Präzession}
\author{}
\date{}

\begin{document}

\maketitle

\section{Kapitel 15: Merkur-Perihel-Präzession }
	
	Die Perihelpräzession des Merkur wird in der fraktalen DVFT als Effekt der Vakuumdynamik erklärt, ohne Einsteins Feldgleichungen. Im hochbeschleunigten Regime reduziert sich die Theorie auf ein newtonsches Potential mit fraktaler Korrektur.
	
	Das effektive Potential lautet 
	\[ U(r) = -\frac{GMm}{r} + \frac{L^2}{2mr^2} - \frac{GM L^2}{m c^2 r^3} \left(1 + \epsilon \ln\frac{r}{r_0}\right). \] 
	Die Binet-Gleichung führt zu 
	\[ \Delta\phi = \frac{6\pi GM}{a(1-e^2) c^2} \left(1 + \frac{\epsilon}{3}\right). \] 
	Mit Merkur-Parametern (\(a = 5.7909 \times 10^{10}\) m, \(e = 0.2056\)) und \(\epsilon \approx 0.06\) ergibt dies exakt 43 Bogensekunden pro Jahrhundert.

\end{document}
