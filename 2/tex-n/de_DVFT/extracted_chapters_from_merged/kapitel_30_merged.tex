\documentclass[12pt,a4paper]{article}
\usepackage[utf8]{inputenc}
\usepackage[T1]{fontenc}
\usepackage[ngerman]{babel}
\usepackage{amsmath}
\usepackage{amsfonts}
\usepackage{amssymb}
\usepackage{geometry}
\geometry{a4paper,left=2.5cm,right=2.5cm,top=2.5cm,bottom=2.5cm}
\usepackage{fancyhdr}
\usepackage{enumitem}
\usepackage{tcolorbox}
\usepackage{physics}
\usepackage{hyperref}

% Hyperref als eines der letzten Pakete laden
\hypersetup{
	unicode=true,
	pdfencoding=unicode,
	bookmarksopen=true
}

% Saubere PDF-Lesezeichen
\pdfstringdefDisableCommands{%
	\def\Lambda{Lambda}%
	\def\Delta{Delta}%
	\def\approx{etwa}%
	\def\Sigma{Sigma}%
	\def\eta{eta}%
	\def\psi{psi}%
}





\title{Kapitel 30: Warum Quantenprozesse im Gehirn machbar sind}
\author{}
\date{}

\begin{document}

\maketitle

\section{Kapitel 30: Warum Quantenprozesse im Gehirn machbar sind }
	
	Roger Penrose schlug vor, dass Bewusstsein aus Quantenprozessen im Gehirn entsteht, spezifisch durch kohärente Aktivität in Mikrotubuli. Neurowissenschaftler lehnten dies ab, mit dem Argument, dass das Gehirn bei 37°C und in einer warmen, feuchten biochemischen Umgebung viel zu thermisch noisy ist, um Quantenkohärenz zu unterstützen.
	
	Die fraktale DVFT bietet eine neue, physisch fundierte Erklärung: Bewusstsein emergiert aus Vakuumphasen-Kohärenz (\(\theta\)), nicht molekularen Quantenzuständen. Phasenkohärenz überlebt Rauschen durch T0-Struktur.
	
	Das Gehirn ist ein Warmtemperatur-Quantenphasen-Computer. Die angepasste DVFT prognostiziert, dass die Zukunft der Quantentechnologie in phasen-basiertem Computing liegt, robuste Quantengeräte ohne Kryo.
	
	Final Summary: Die fraktale DVFT bietet eine vereinheitlichte Erklärung für die Penrose-Hypothese und neurowissenschaftliche Zwänge: Bewusstsein emergiert aus Vakuumphasen-Kohärenz (\(\theta\)), nicht molekularen Quantenzuständen. Phasenkohärenz überlebt bei 37°C und unterstützt makroskopische Quantenverarbeitung im Gehirn. Das Gehirn ist ein Warmtemperatur-Quantenphasen-Computer. Die fraktale DVFT prognostiziert, dass die Zukunft der Quantentechnologie in phasen-basiertem Computing liegt. Somit bietet die angepasste DVFT die erste physisch konsistente Erklärung, wie Bewusstsein Quantenverhalten bei biologischen Temperaturen einbezieht und warum dies ein neues Paradigma für Quantencomputing freisetzt, basierend auf T0-Theorie.

\end{document}
