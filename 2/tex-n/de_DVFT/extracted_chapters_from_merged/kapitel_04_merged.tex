\documentclass[12pt,a4paper]{article}
\usepackage{amsmath, amssymb, amsthm}
\usepackage{geometry}
\usepackage{titlesec}
\usepackage{tcolorbox}
\usepackage{enumitem}
\usepackage{booktabs}

% Theoreme
\newtheorem{theorem}{Theorem}[section]
\newtheorem{lemma}[theorem]{Lemma}
\newtheorem{corollary}[theorem]{Korollar}
\newtheorem{definition}[theorem]{Definition}

\title{Kapitel 4: Elementarteilchenmassen aus dem Vakuumfeld}
\author{}
\date{29. Dezember 2025}

\begin{document}
	
	\maketitle
	

\section*{Kapitel 4: Elementarteilchenmassen aus dem Vakuumfeld}

Dieses Kapitel ist Teil der Dynamic Vacuum Field Theory (DVFT) mit vollständiger Integration der fraktalen T0-Geometrie.

\textbf{Hinweis:} Der vollständige Inhalt dieses Kapitels muss aus dem bereitgestellten Template-Dokument extrahiert werden, das die Kapitel 1-11 in zusammengeführter Form enthält.

\subsection*{Platzhalter}

Der Inhalt wird aus dem Template extrahiert und hier eingefügt.

	
\end{document}
