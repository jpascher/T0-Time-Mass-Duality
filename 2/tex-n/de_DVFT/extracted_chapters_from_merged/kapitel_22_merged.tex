\documentclass[12pt,a4paper]{article}
\usepackage[utf8]{inputenc}
\usepackage[T1]{fontenc}
\usepackage[ngerman]{babel}
\usepackage{amsmath}
\usepackage{amsfonts}
\usepackage{amssymb}
\usepackage{geometry}
\geometry{a4paper,left=2.5cm,right=2.5cm,top=2.5cm,bottom=2.5cm}
\usepackage{fancyhdr}
\usepackage{enumitem}
\usepackage{tcolorbox}
\usepackage{physics}
\usepackage{hyperref}

% Hyperref als eines der letzten Pakete laden
\hypersetup{
	unicode=true,
	pdfencoding=unicode,
	bookmarksopen=true
}

% Saubere PDF-Lesezeichen
\pdfstringdefDisableCommands{%
	\def\Lambda{Lambda}%
	\def\Delta{Delta}%
	\def\approx{etwa}%
	\def\Sigma{Sigma}%
	\def\eta{eta}%
	\def\psi{psi}%
}





\title{Kapitel 22: Maximale Masse für Quantenüberlagerung}
\author{}
\date{}

\begin{document}

\maketitle

\section{Kapitel 22: Maximale Masse für Quantenüberlagerung }
	
	Dieses Kapitel präsentiert die T0-begründete fraktale DVFT-Vorhersage für die maximale Masse und Größe von Molekülen oder makroskopischen Objekten, die in Quantenüberlagerung bleiben können. Diese Frage ist direkt relevant für das MAST-QG-Projekt (Macroscopic Superpositions for Quantum Gravity).
	
	Fraktale T0-Anpassung: DVFT liefert einen mathematisch präzisen Grenzwert, bestimmt durch die nichtlineare Antwort des Vakuumphasenfeldes, das aus T0s Dualität abgeleitet ist. Im Gegensatz zu heuristischen Modellen wie Penrose's Objective Reduction oder CSL-Modellen ist der Grenzwert strukturell aus T0s Vakuumsteifigkeit abgeleitet.
	
	Die Kohärenzzeit \(\tau_c = \hbar / (\Delta E) (1 - \epsilon/2)\), mit \(\Delta E \sim G m^2 / R r^{\epsilon}\). Obergrenze \(m_{\max} \sim 10^7 - 10^8\) amu (\(R_{\max} \sim 100\) nm).
	
	Narrative: Überlagerung kollabiert, wenn fraktale Amplitude die Selbstähnlichkeit nicht mehr aufrechterhalten kann, spontane Dekohärenz durch T0-Nichtlinearität.
	
	Testbar in MAST-QG, MAQRO; Kollaps bei etwa \(10^8\) amu falsifiziert oder validiert T0.
	
	Hauptergebnisse: Kein heuristisches Modell, sondern strukturelle Konsequenz von \(T(x,t) \cdot m(x,t) = 1\). Sagt fundamentalen Grenzwert voraus. Falls Experimente \(10^8\) amu ohne Kollaps überschreiten, T0 falsifiziert; bei Kollaps T0 validiert.
	
	Die maximale Überlagerungsmasse ist einzigartige, falsifizierbare Vorhersage der T0-Theorie.

\end{document}
