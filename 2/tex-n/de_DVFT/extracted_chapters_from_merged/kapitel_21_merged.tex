\documentclass[12pt,a4paper]{article}
\usepackage[utf8]{inputenc}
\usepackage[T1]{fontenc}
\usepackage[ngerman]{babel}
\usepackage{amsmath}
\usepackage{amsfonts}
\usepackage{amssymb}
\usepackage{geometry}
\geometry{a4paper,left=2.5cm,right=2.5cm,top=2.5cm,bottom=2.5cm}
\usepackage{fancyhdr}
\usepackage{enumitem}
\usepackage{tcolorbox}
\usepackage{physics}
\usepackage{hyperref}

% Hyperref als eines der letzten Pakete laden
\hypersetup{
	unicode=true,
	pdfencoding=unicode,
	bookmarksopen=true
}

% Saubere PDF-Lesezeichen
\pdfstringdefDisableCommands{%
	\def\Lambda{Lambda}%
	\def\Delta{Delta}%
	\def\approx{etwa}%
	\def\Sigma{Sigma}%
	\def\eta{eta}%
	\def\psi{psi}%
}





\title{Kapitel 21: Ron Folmans T-cube-Quantengravitationsexperiment}
\author{}
\date{}

\begin{document}

\maketitle

\section{Kapitel 21: Ron Folmans T-cube-Quantengravitationsexperiment }
	
	Ron Folmans T-cube (T-hoch-drei) Atominterferometrie-Experiment stellt einen der präzisesten Tests von Quantensystemen unter Gravitationsfeldern dar. Das zentrale Ergebnis ist, dass die Interferenzphase, die von atomaren Wellenpaketen in einem Gravitationspotential akkumuliert wird, wie folgt wächst:
	\[ \Delta\phi \propto g T^3. \]
	
	Diese Skalierung unterscheidet sich von der üblichen \(T^2\)-Abhängigkeit in Standard-Lichtpuls-Atominterferometrie und entsteht nur, wenn die vollständige Quantenentwicklung des Wellenpakets einschließlich seiner räumlichen Trajektorie berücksichtigt wird.
	
	Die fraktale T0-Anpassung: Gravitation entsteht aus Gradienten des Zeitfeldes \(\nabla \ln T \propto r^{-\epsilon}\). Die Phase wird zu \(\Delta\phi = \frac{m g T^3}{3\hbar} (1 + \epsilon \ln T)\), wobei der fraktale Term die Selbstähnlichkeit widerspiegelt. Im T0-Kontext, wo \(g = -c^2 \nabla \ln T (1 - \epsilon/2)\), führt dies zu kubischer Abhängigkeit durch verlängerte Pfade.
	
	Dies kann nicht aus reiner Yang-Mills oder Standard-GR abgeleitet werden, entsteht natürlich aus T0s Gradienten. Validiert Vakuumphasenfeld \(\Phi = \rho e^{i\theta}\) aus \(\Delta m(x,t)\). Erfordert keine freien Parameter außer \(\xi\).
	
	Vorhersagen: Abweichungen bei \(gT/c \sim \xi\), massenabhängig für zusammengesetzte Teilchen, Zeitfeld-Anisotropie in rotierenden Systemen.
	
	Schlussfolgerung: Ron Folmans T-cube-Experiment liefert direkten Beweis, dass gravitationelle Phasenakkumulation der \(T^3\)-Skalierung folgt, exakt wie von T0-Theoriens Dualität vorhergesagt. Dieses Ergebnis kann nicht aus reiner Yang-Mills- oder Standard-GR abgeleitet werden, entsteht natürlich aus T0s Zeitfeld-Gradienten. Validiert T0s Vakuumphasenfeld. Die T-cube-Skalierung ist einzigartige Signatur von T0s fundamentaler Struktur.

\end{document}
