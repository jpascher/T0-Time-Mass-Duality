\documentclass[12pt,a4paper]{article}
\usepackage[utf8]{inputenc}
\usepackage[T1]{fontenc}
\usepackage[ngerman]{babel}
\usepackage{amsmath}
\usepackage{amsfonts}
\usepackage{amssymb}
\usepackage{geometry}
\geometry{a4paper,left=2.5cm,right=2.5cm,top=2.5cm,bottom=2.5cm}
\usepackage{fancyhdr}
\usepackage{enumitem}
\usepackage{tcolorbox}
\usepackage{physics}
\usepackage{hyperref}

% Hyperref als eines der letzten Pakete laden
\hypersetup{
	unicode=true,
	pdfencoding=unicode,
	bookmarksopen=true
}

% Saubere PDF-Lesezeichen
\pdfstringdefDisableCommands{%
	\def\Lambda{Lambda}%
	\def\Delta{Delta}%
	\def\approx{etwa}%
	\def\Sigma{Sigma}%
	\def\eta{eta}%
	\def\psi{psi}%
}





\title{Kapitel 28: Warum Newtons Gesetz nicht für Quantenteilchen gilt}
\author{}
\date{}

\begin{document}

\maketitle

\section{Kapitel 28: Warum Newtons Gesetz nicht für Quantenteilchen gilt }
	
	Das Newtonsche Gesetz \(F = G m_1 m_2 / r^2\) funktioniert hervorragend für Planeten, Sterne und Galaxien. Aber gilt es für ein einzelnes Proton, das ein anderes Proton anzieht? Die Antwort lautet: Nein, nicht fundamental.
	
	Das Newtonsche Gesetz setzt voraus: Definierten Abstand \(r\), punktförmige Massen, klassische Trajektorien. In Quantenmechanik fehlen diese.
	
	Fraktale T0-Theorie: Gravitation nicht als Raumzeitkrümmung, sondern als Deformation des Vakuumamplitudenfelds \(\rho(x,t) \propto 1/T(x,t)\). Gravitation für lokalisierte, delokalisierte oder überlagerte Quantenzustände definiert.
	
	Gravitationsfeld \(\delta\rho(x)\) folgt Quantenwellenfunktion \(|\psi(x)|^2\). Klassischer Grenzfall entsteht durch Dekohärenz. Keine Singularitäten: \(\rho_0 = 1/\xi^2\) liefert Minimum.
	
	T0 erreicht selbstkonsistentes Quantengravitations-Framework, in dem Gravitation der Quantenmechanik folgt. Alles aus \(\xi\).

\end{document}
