\documentclass[12pt,a4paper]{article}
\usepackage[utf8]{inputenc}
\usepackage[T1]{fontenc}
\usepackage[ngerman]{babel}
\usepackage{amsmath}
\usepackage{amsfonts}
\usepackage{amssymb}
\usepackage{geometry}
\geometry{a4paper,left=2.5cm,right=2.5cm,top=2.5cm,bottom=2.5cm}
\usepackage{fancyhdr}
\usepackage{enumitem}
\usepackage{tcolorbox}
\usepackage{physics}
\usepackage{hyperref}

% Hyperref als eines der letzten Pakete laden
\hypersetup{
	unicode=true,
	pdfencoding=unicode,
	bookmarksopen=true
}

% Saubere PDF-Lesezeichen
\pdfstringdefDisableCommands{%
	\def\Lambda{Lambda}%
	\def\Delta{Delta}%
	\def\approx{etwa}%
	\def\Sigma{Sigma}%
	\def\eta{eta}%
	\def\psi{psi}%
}





\title{Kapitel 27: Teilchen-Massenhierarchie und Gravitationsschwäche}
\author{}
\date{}

\begin{document}

\maketitle

\section{Kapitel 27: Teilchen-Massenhierarchie und Gravitationsschwäche }
	
	Dieses Kapitel erklärt zwei fundamentale ungelöste Probleme: (1) Warum erstrecken sich Elementarteilchenmassen über 14 Größenordnungen? (2) Warum ist Gravitation außerordentlich schwach? Fraktale T0-Theorie liefert natürliche, strukturelle Lösungen durch Modellierung von Teilchen als Vakuumfeld-Störungen im Zeit-Masse-Feld \(T(x,t) \cdot m(x,t) = 1\). Massenhierarchie entsteht aus verschiedenen Vakuum-Deformationsmoden, Gravitationsschwäche aus T0s verdünnter Struktur \(\rho_0 = 1/\xi^2\).
	
	Die moderne Physik kann nicht erklären: Elektronmasse \(m_e \approx 0{,}5\) MeV, Top-Quark-Masse \(m_t \approx 173\) GeV, Verhältnis \(m_t/m_e \sim 3{,}5 \times 10^5\) (14 Größenordnungen inkl. Neutrinos), Gravitation \(10^{32}\) mal schwächer als schwache Kraft.
	
	Fraktale DVFT: Teilchen als Vakuumdeformationsenergie. Hierarchie = verschiedene Moden. Gravitationsschwäche = verdünnte Vakuumstruktur \(\rho_0 = 1/\xi^2\).
	
	Drei Familien = SU(3)-Phasensymmetrie in T0.
	
	Hauptergebnisse: Alle Massen aus \(\xi = 4/3 \times 10^{-4}\), Massenhierarchie = verschiedene Moden, Gravitationsschwäche = verdünnte Struktur, drei Familien = SU(3).
	
	Von Neutrinomassen (\(10^{-3}\) eV) bis Top-Quark (173 GeV) – alles aus T0s Vakuumstruktur. Keine willkürlichen Parameter. Vollständige strukturelle Erklärung. Experimentell validiert.

\end{document}
