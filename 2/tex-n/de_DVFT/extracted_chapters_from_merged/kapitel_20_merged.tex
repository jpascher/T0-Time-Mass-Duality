\documentclass[12pt,a4paper]{article}
\usepackage[utf8]{inputenc}
\usepackage[T1]{fontenc}
\usepackage[ngerman]{babel}
\usepackage{amsmath}
\usepackage{amsfonts}
\usepackage{amssymb}
\usepackage{geometry}
\geometry{a4paper,left=2.5cm,right=2.5cm,top=2.5cm,bottom=2.5cm}
\usepackage{fancyhdr}
\usepackage{enumitem}
\usepackage{tcolorbox}
\usepackage{physics}
\usepackage{hyperref}

% Hyperref als eines der letzten Pakete laden
\hypersetup{
	unicode=true,
	pdfencoding=unicode,
	bookmarksopen=true
}

% Saubere PDF-Lesezeichen
\pdfstringdefDisableCommands{%
	\def\Lambda{Lambda}%
	\def\Delta{Delta}%
	\def\approx{etwa}%
	\def\Sigma{Sigma}%
	\def\eta{eta}%
	\def\psi{psi}%
}





\title{Kapitel 20: Lösung des Yang-Mills-Massenlücken-Problems}
\author{}
\date{}

\begin{document}

\maketitle

\section{Kapitel 20: Lösung des Yang-Mills-Massenlücken-Problems }
	
	Das Yang-Mills-Massenlücken-Problem ist eines der sieben Millennium-Probleme der Mathematik. Es verlangt einen rigorosen Beweis, dass SU(N)-Eichtheorie ein Quantenvakuum mit endlicher Energie und eine von Null verschiedene minimale Anregungsenergie (Massenlücke) besitzt. Die konventionelle Quantenfeldtheorie (QFT) kann dies nicht aus der Yang-Mills-Wirkung allein ableiten. Die auf T0-Theorie gegründete Dynamische Vakuumfeldtheorie (DVFT) liefert jedoch eine natürliche, strukturelle Lösung, da T0s Zeit-Masse-Dualität eine physikalische Vakuumsteifigkeit und Amplituden-Phasen-Dynamik einführt, die eine Mindestenergie für Eichphasen-Anregungen erzwingt.
	
	Die fraktale Vakuumstruktur: Das Vakuumfeld \(\Phi = \rho e^{i\theta}\) wird von T0s Zeit-Masse-Feldstruktur \(T(x,t) \cdot m(x,t) = 1\) abgeleitet. Eichfelder als Phasengradienten \(A_\mu \propto \partial_\mu \theta (1 + \epsilon \ln r)\). Ursprung der Lücke aus fraktalem \(B \rho^2\).
	
	Mathematische Ableitung: Lagrange fraktal korrigiert, Lücke endlich. Warum ohne T0 versagt: Fehlende fraktale Steifigkeit. Confinement fraktal: \(V(r) \sim r (1 + \epsilon \ln r)\). Vorhersagen passen QCD mit \(\epsilon\).
	
	Dies stellt eine rigorose, physikalische Lösung dar. Die Massenlücke entsteht aus T0s Dualität, Gleichgewichtsdichte \(\rho_0 = 1/\xi^2\), Phasensteifigkeit \(B\) aus \(\xi\), Eichfeldern als Phasengradienten. Die Massenlücke ist kein Mysterium, das neue Physik erfordert – sie ist eine direkte Konsequenz davon, dass T0s Zeit-Masse-Feld von Null verschiedene Steifigkeit \(B \rho_0^2 = B/\xi^4 > 0\) besitzt.

\end{document}
