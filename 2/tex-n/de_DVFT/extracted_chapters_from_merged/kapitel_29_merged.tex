\documentclass[12pt,a4paper]{article}
\usepackage[utf8]{inputenc}
\usepackage[T1]{fontenc}
\usepackage[ngerman]{babel}
\usepackage{amsmath}
\usepackage{amsfonts}
\usepackage{amssymb}
\usepackage{geometry}
\geometry{a4paper,left=2.5cm,right=2.5cm,top=2.5cm,bottom=2.5cm}
\usepackage{fancyhdr}
\usepackage{enumitem}
\usepackage{tcolorbox}
\usepackage{physics}
\usepackage{hyperref}

% Hyperref als eines der letzten Pakete laden
\hypersetup{
	unicode=true,
	pdfencoding=unicode,
	bookmarksopen=true
}

% Saubere PDF-Lesezeichen
\pdfstringdefDisableCommands{%
	\def\Lambda{Lambda}%
	\def\Delta{Delta}%
	\def\approx{etwa}%
	\def\Sigma{Sigma}%
	\def\eta{eta}%
	\def\psi{psi}%
}





\title{Kapitel 29: Delayed-Choice-Quantum-Eraser-Experiment}
\author{}
\date{}

\begin{document}

\maketitle

\section{Kapitel 29: Delayed-Choice-Quantum-Eraser-Experiment }
	
	Das Delayed-Choice-Quantum-Eraser (DCQE)-Experiment gehört zu den faszinierendsten Demonstrationen der Quantenphysik. Es scheint auf Retrokausalität oder darauf hinzudeuten, dass eine zukünftige Messung das vergangene Verhalten eines Photons beeinflusst. Diese Sektion analysiert das Experiment im Rahmen der fraktalen T0-Theorie. Die T0-Interpretation beseitigt Retrokausalität vollständig, indem sie zeigt, dass das Phänomen aus der fraktalen Phasenkohärenz im intrinsischen Zeitfeld \(T(x,t)\) resultiert. Beim DCQE geht es um Erhaltung, Störung oder Wiederherstellung der Phasenkohärenz im fraktalen Vakuumfeld – nicht um Rückwärtskausalität.
	
	Vakuumfeld-Struktur in der T0-Theorie: Quantenzustände aus Anregungen des universellen Zeit-Masse-Feldes, das der Dualität \(T(x,t) \cdot E(x,t) = 1\) genügt. ''Photon'' = Phasenwirbel im Vakuumfeld \(\Phi = \rho e^{i\theta}\). Seine ''Trajektorie'' wird durch geometrische Phasengradienten in \(T(x,t)\) geleitet. Welcher-Weg-Detektion stört die fraktale Phasenstruktur. Löschung rekonstruiert die kohärente Phasengeometrie.
	
	Dies löst die Paradoxien ohne Retrokausalität oder Beobachterabhängigkeit.
	
	Schlussfolgerung: Das Delayed-Choice-Quantum-Eraser-Experiment benötigt keine Retrokausalität. Die T0-Theorie liefert eine deterministische, geometrische Erklärung: Die fraktale Phase des intrinsischen Zeitfeldes \(T(x,t)\) bestimmt die Sichtbarkeit von Interferenz. Welcher-Weg-Information stört fraktale Kohärenz; Löschung stellt sie in korrelierten Teilmengen wieder her. Die verzögerte Wahl beeinflusst die Klassifikation von Ereignissen, nicht ihr Auftreten. T0 vereinigt somit DCQE mit geometrischer Intuition und reproduziert gleichzeitig alle quantenmechanischen Vorhersagen durch die Zeit-Masse-Dualität und \(\xi\)-Fraktalität.

\end{document}
