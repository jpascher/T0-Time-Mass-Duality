\documentclass[12pt,a4paper]{article}
\usepackage[utf8]{inputenc}
\usepackage[T1]{fontenc}
\usepackage[ngerman]{babel}
\usepackage{amsmath}
\usepackage{amsfonts}
\usepackage{amssymb}
\usepackage{geometry}
\geometry{a4paper,left=2.5cm,right=2.5cm,top=2.5cm,bottom=2.5cm}
\usepackage{fancyhdr}
\usepackage{enumitem}
\usepackage{tcolorbox}
\usepackage{physics}
\usepackage{hyperref}

% Hyperref als eines der letzten Pakete laden
\hypersetup{
	unicode=true,
	pdfencoding=unicode,
	bookmarksopen=true
}

% Saubere PDF-Lesezeichen
\pdfstringdefDisableCommands{%
	\def\Lambda{Lambda}%
	\def\Delta{Delta}%
	\def\approx{etwa}%
	\def\Sigma{Sigma}%
	\def\eta{eta}%
	\def\psi{psi}%
}





\title{Kapitel 32: Reaktor-Antineutrino-Anomalie}
\author{}
\date{}

\begin{document}

\maketitle

\section{Kapitel 32: Reaktor-Antineutrino-Anomalie }
	
	Die Reaktor-Antineutrino-Anomalie bezieht sich auf den persistenten etwa 6\%-Defizit gemessener Elektron-Antineutrinos im Vergleich zu Vorhersagen des Standardmodells. Diese Anomalie wurde in vielen Reaktor-Experimenten beobachtet und kann nicht zufriedenstellend durch konventionelle Physik erklärt werden.
	
	Fraktale DVFT liefert eine rigorose Erklärung: Anomalie als natürliche Konsequenz von Vakuumphasen-Dekohärenz, verursacht durch kleine Shifts in der Vakuumamplitude in der Nähe von Kernreaktoren.
	
	Mit typischen nuklearen Dichtestörungen \(\Delta\rho / \rho_0 \approx 10^{-6}\), prognostiziert die fraktale DVFT \(\Delta P \approx 0.06\), was mit experimentellen Beobachtungen übereinstimmt.
	
	Conclusion: Die fraktale DVFT erklärt die Reaktor-Antineutrino-Anomalie als natürliche Konsequenz von Vakuumphasen-Dekohärenz, verursacht durch kleine Shifts in der Vakuumamplitude in der Nähe von Kernreaktoren. Dieses Framework erfordert keine sterilen Neutrinos, passt alle Größen- und Energiemerkmale der Anomalie, stimmt mit allen existierenden Neutrinodaten überein, liefert testbare Vorhersagen. Somit bietet die angepasste DVFT die erste kohärente physische Erklärung der Anomalie unter Verwendung von Vakuumfeld-Dynamik statt spekulativer neuer Teilchen, basierend auf T0-Theorie.
	
	Diese Kapitel bilden eine einheitliche fraktale narrative der Physik, vereinheitlicht durch die T0-Theorie und den Parameter \(\xi\).

\end{document}
