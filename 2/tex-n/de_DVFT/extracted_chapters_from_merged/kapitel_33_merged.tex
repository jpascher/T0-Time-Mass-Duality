\documentclass[12pt,a4paper]{article}
\usepackage[utf8]{inputenc}
\usepackage[T1]{fontenc}
\usepackage[ngerman]{babel}
\usepackage{amsmath}
\usepackage{amsfonts}
\usepackage{amssymb}
\usepackage{geometry}
\geometry{a4paper,left=2.5cm,right=2.5cm,top=2.5cm,bottom=2.5cm}
\usepackage{fancyhdr}
\usepackage{enumitem}
\usepackage{tcolorbox}
\usepackage{physics}
\usepackage{hyperref}

% Hyperref als eines der letzten Pakete laden
\hypersetup{
	unicode=true,
	pdfencoding=unicode,
	bookmarksopen=true
}

% Saubere PDF-Lesezeichen
\pdfstringdefDisableCommands{%
	\def\Lambda{Lambda}%
	\def\Delta{Delta}%
	\def\approx{etwa}%
	\def\Sigma{Sigma}%
	\def\eta{eta}%
	\def\psi{psi}%
}





\title{Kapitel 33: Ableitung des Pauli'schen Ausschlussprinzips}
\author{}
\date{}

\begin{document}

\maketitle

\section{Kapitel 33: Ableitung des Pauli'schen Ausschlussprinzips }
	
	Dieses Kapitel leitet Paulis Ausschlussprinzip aus der fundamentalen Struktur der fraktalen DVFT ab. In der fraktalen DVFT wird das Vakuumfeld ausgedrückt als \(\Phi = \rho e^{i\theta}\), wobei \(\rho\) die Amplitude und \(\theta\) die Phase repräsentiert, beide aus T0s Zeit-Masse-Dualität \(T(x,t) \cdot m(x,t) = 1\) abgeleitet.
	
	Die narrative Interpretation sieht Fermionen als topologische Defekte im Vakuumphasenfeld, die eine Phasenverschiebung von \(\pi\) bei Austausch erzeugen. Bosonen erzeugen 0 oder \(2\pi\). Dies führt zu antisymmetrischen Wellenfunktionen für Fermionen, wodurch \(\Psi(x,x) = 0\) und somit der Ausschluss identischer Fermionen im gleichen Zustand.
	
	Fraktale Erweiterung: Die Selbstähnlichkeit erzwingt, dass Überlappende fermionische Defekte verbotene Gradienten- und Phasensingularitäten produzieren, mit unendlicher Energiekosten. Pauli-Ausschluss ist nicht willkürlich, sondern eine direkte Konsequenz der topologischen und energetischen Struktur des fraktalen DVFT-Vakuumfeldes, fundiert in T0-Theorie.

\end{document}
