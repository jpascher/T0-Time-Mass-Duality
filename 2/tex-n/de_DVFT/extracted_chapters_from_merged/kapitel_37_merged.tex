\documentclass[12pt,a4paper]{article}
\usepackage[utf8]{inputenc}
\usepackage[T1]{fontenc}
\usepackage[ngerman]{babel}
\usepackage{amsmath}
\usepackage{amsfonts}
\usepackage{amssymb}
\usepackage{geometry}
\geometry{a4paper,left=2.5cm,right=2.5cm,top=2.5cm,bottom=2.5cm}
\usepackage{fancyhdr}
\usepackage{enumitem}
\usepackage{tcolorbox}
\usepackage{physics}
\usepackage{hyperref}

% Hyperref als eines der letzten Pakete laden
\hypersetup{
	unicode=true,
	pdfencoding=unicode,
	bookmarksopen=true
}

% Saubere PDF-Lesezeichen
\pdfstringdefDisableCommands{%
	\def\Lambda{Lambda}%
	\def\Delta{Delta}%
	\def\approx{etwa}%
	\def\Sigma{Sigma}%
	\def\eta{eta}%
	\def\psi{psi}%
}





\title{Kapitel 37: Intrinsische Eigenschaften des Vakuumfeldes}
\author{}
\date{}

\begin{document}

\maketitle

\section{Kapitel 37: Intrinsische Eigenschaften des Vakuumfeldes }
	
	Dieses Kapitel kompiliert intrinsische numerische Parameter des Vakuumfeldes in fraktaler DVFT.
	
	Parameter wie \(\rho_0 = 1/\xi^2\), \(B\) aus Feinstrukturkonstante \(\alpha\), \(K_0\) aus Kosmologie. Diese emergieren aus T0 und vereinheitlichen Spezielle Relativität, Quantenmechanik, Elektromagnetismus, Neutrinophysik, Baryogenese, Dunkle Energie, galaktische Dynamik.
	
	Erste kohärente numerische Grundlage für vereinheitlichte Theorie.

\end{document}
