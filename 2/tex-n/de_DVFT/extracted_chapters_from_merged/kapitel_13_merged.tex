\documentclass[12pt,a4paper]{article}
\usepackage[utf8]{inputenc}
\usepackage[T1]{fontenc}
\usepackage[ngerman]{babel}
\usepackage{amsmath}
\usepackage{amsfonts}
\usepackage{amssymb}
\usepackage{geometry}
\geometry{a4paper,left=2.5cm,right=2.5cm,top=2.5cm,bottom=2.5cm}
\usepackage{fancyhdr}
\usepackage{enumitem}
\usepackage{tcolorbox}
\usepackage{physics}
\usepackage{hyperref}

% Hyperref als eines der letzten Pakete laden
\hypersetup{
	unicode=true,
	pdfencoding=unicode,
	bookmarksopen=true
}

% Saubere PDF-Lesezeichen
\pdfstringdefDisableCommands{%
	\def\Lambda{Lambda}%
	\def\Delta{Delta}%
	\def\approx{etwa}%
	\def\Sigma{Sigma}%
	\def\eta{eta}%
	\def\psi{psi}%
}





\title{Kapitel 13: Chronologie der Universumsschöpfung}
\author{}
\date{}

\begin{document}

\maketitle

\section{Kapitel 13: Chronologie der Universumsschöpfung }
	
	Die Chronologie der Universumsschöpfung in der fraktalen DVFT ist eine detaillierte narrative der Emergenz: Am absoluten Anfang existiert ein reines Phasen-Vakuum mit \(\rho = 0\) und konstanter \(\theta\), das durch seine fraktale Natur keine Struktur aufweisen kann. Dieses Vakuum ist perfekt kohärent, da Gradienten oder Fluktuationen eine Amplitude erfordern würden, die fehlt. Die Instabilität entsteht aus der T0-Dualität: Infinitesimale Störungen in \(\delta \theta\) fordern eine nicht-null Amplitude \(\rho > 0\), um zu propagieren, was den Phasenübergang auslöst.
	
	Mathematisch wird dies durch das Potenzial \(V(\rho) = \lambda (\rho^2 - \rho_0^2)^2 (1 + \epsilon \ln(\rho / \rho_0))\) beschrieben, das bei \(\rho = 0\) unstabil ist. Sobald \(\rho\) emergiert, entsteht Zeit als Phasenentwicklung \(d\tau \propto d\theta\), mit Lichtgeschwindigkeit \(c = \sqrt{K_0 / \rho_0} (1 - \epsilon / 2)\), fraktal begrenzt. Gravitation und Krümmung folgen aus fraktalen Gradienten \(\nabla \rho \propto r^{-D_f}\). Teilchen bilden sich als stabile fraktale Knoten in \(\Phi\), mit Massen \(m \propto \rho_0 \xi\). Die niedrige Entropie am Anfang ist unvermeidlich: Das fraktale Vakuum hat null Entropie durch Selbstähnlichkeit, und Entropie wächst nur nach der Emergenz von \(\rho\). Diese Sequenz bietet eine physikalische Ontologie ohne Singularität oder Expansion.

\end{document}
