\documentclass[12pt,a4paper]{article}
\usepackage[utf8]{inputenc}
\usepackage[T1]{fontenc}
\usepackage[ngerman]{babel}
\usepackage{amsmath}
\usepackage{amsfonts}
\usepackage{amssymb}
\usepackage{geometry}
\geometry{a4paper,left=2.5cm,right=2.5cm,top=2.5cm,bottom=2.5cm}
\usepackage{fancyhdr}
\usepackage{enumitem}
\usepackage{tcolorbox}
\usepackage{physics}
\usepackage{hyperref}

% Hyperref als eines der letzten Pakete laden
\hypersetup{
	unicode=true,
	pdfencoding=unicode,
	bookmarksopen=true
}

% Saubere PDF-Lesezeichen
\pdfstringdefDisableCommands{%
	\def\Lambda{Lambda}%
	\def\Delta{Delta}%
	\def\approx{etwa}%
	\def\Sigma{Sigma}%
	\def\eta{eta}%
	\def\psi{psi}%
}





\title{Kapitel 35: Erklärung quantenmechanischer Phänomene}
\author{}
\date{}

\begin{document}

\maketitle

\section{Kapitel 35: Erklärung quantenmechanischer Phänomene }
	
	Die fraktale DVFT interpretiert Quantenmechanik als Verhalten von Vakuumphasen- und Amplitudenfeldern, fundiert in T0s Dualität. Dieses Kapitel erklärt zwölf große Quantenphänomene einheitlich.
	
	Interferenz aus Phasenaddition in \(\theta\). Kollaps als lokale Amplitudenstörung \(\delta\rho\). Verschränkung aus globaler Phasenkopplung. Dekohärenz aus Phasenverstreuung durch Interaktionen.
	
	Superposition aus multiplen Phasenkonfigurationen. Tunneln durch Phasenbarriere. Nullpunktsenergie aus intrinsischer \(\mu = \xi m_0\)-Oszillation. Vakuumfluktuationen \(\Delta\theta \cdot \Delta E \geq \hbar/2\) aus T0-Fluktuationen \(\Delta m\).
	
	Atomare Quantisierung aus \(\theta\)-Zirkulationsbedingungen \(\oint \nabla\theta \cdot dl = 2\pi n\).
	
	Die fraktale DVFT vereinheitlicht Gravitation und Quantenmechanik, quantenmechanisches Verhalten in Vakuumphaseneigenschaften verankert.

\end{document}
