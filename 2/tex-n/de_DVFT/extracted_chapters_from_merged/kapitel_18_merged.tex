\documentclass[12pt,a4paper]{article}
\usepackage[utf8]{inputenc}
\usepackage[T1]{fontenc}
\usepackage[ngerman]{babel}
\usepackage{amsmath}
\usepackage{amsfonts}
\usepackage{amssymb}
\usepackage{geometry}
\geometry{a4paper,left=2.5cm,right=2.5cm,top=2.5cm,bottom=2.5cm}
\usepackage{fancyhdr}
\usepackage{enumitem}
\usepackage{tcolorbox}
\usepackage{physics}
\usepackage{hyperref}

% Hyperref als eines der letzten Pakete laden
\hypersetup{
	unicode=true,
	pdfencoding=unicode,
	bookmarksopen=true
}

% Saubere PDF-Lesezeichen
\pdfstringdefDisableCommands{%
	\def\Lambda{Lambda}%
	\def\Delta{Delta}%
	\def\approx{etwa}%
	\def\Sigma{Sigma}%
	\def\eta{eta}%
	\def\psi{psi}%
}





\title{Kapitel 18: Ableitung der Schrödinger-Gleichung}
\author{}
\date{}

\begin{document}

\maketitle

\section{Kapitel 18: Ableitung der Schrödinger-Gleichung }
	
	In der T0-Theorie ist das Vakuumfeld \(\Phi = \rho e^{i\theta}\) nicht unabhängig, sondern aus dem Massenfeld \(\Delta m(x,t)\) über die Zeit-Masse-Dualität \(T(x,t) \cdot m(x,t) = 1\) abgeleitet. Die Vakuumphase \(\theta\) entsteht aus T0-Knotenrotationen, und \(\rho \propto m = 1/T\). Die Quantenmechanik entsteht als nicht-relativistischer Grenzfall von Teilchen, die mit T0s Zeitfeldstruktur wechselwirken. Die komplexe Natur quantenmechanischer Wellenfunktionen spiegelt die komplexe Struktur von T0s zugrundeliegendem Zeit-Masse-Feld wider. Alle Quantenparameter leiten sich aus T0s fundamentaler Konstante \(\xi = 4/3 \times 10^{-4}\) ab.
	
	Dieses Kapitel erklärt, wie die Schrödinger-Gleichung natürlich innerhalb der Dynamischen Vakuumfeldtheorie (DVFT) entsteht, wenn man den nicht-relativistischen Grenzfall der Vakuumfeldgleichung betrachtet. Die Wellenfunktion \(\psi = R e^{iS/\hbar}\) erbt ihre Phase von der Vakuumphase \(\theta = \mu t\), mit intrinsischer Frequenz \(\mu = \xi m_0\).
	
	Der Quantenhamiltonian ist \(\hat{H} = -\frac{\hbar^2}{2m} \nabla^{D_f} \psi + V \psi + \hbar \mu\), was zu \(i\hbar \partial_t \psi = \hat{H} \psi\) führt. Dies löst das grundlegende Geheimnis der Quantenmechanik: Die Wellenfunktion ist nicht abstrakt, sondern repräsentiert physikalische Störungen in T0s Zeit-Masse-Feld. Die Schrödinger-Gleichung ist nicht postuliert, sondern als nicht-relativistischer Grenzfall von Teilchen-Vakuum-Wechselwirkungen innerhalb des T0-Rahmens abgeleitet.

\end{document}
