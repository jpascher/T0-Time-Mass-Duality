\documentclass[12pt,a4paper]{article}
\usepackage[utf8]{inputenc}
\usepackage[T1]{fontenc}
\usepackage[ngerman]{babel}
\usepackage{amsmath}
\usepackage{amsfonts}
\usepackage{amssymb}
\usepackage{geometry}
\geometry{a4paper,left=2.5cm,right=2.5cm,top=2.5cm,bottom=2.5cm}
\usepackage{fancyhdr}
\usepackage{enumitem}
\usepackage{tcolorbox}
\usepackage{physics}
\usepackage{hyperref}

% Hyperref als eines der letzten Pakete laden
\hypersetup{
	unicode=true,
	pdfencoding=unicode,
	bookmarksopen=true
}

% Saubere PDF-Lesezeichen
\pdfstringdefDisableCommands{%
	\def\Lambda{Lambda}%
	\def\Delta{Delta}%
	\def\approx{etwa}%
	\def\Sigma{Sigma}%
	\def\eta{eta}%
	\def\psi{psi}%
}





\title{Kapitel 34: Lösung des Strong-CP-Problems}
\author{}
\date{}

\begin{document}

\maketitle

\section{Kapitel 34: Lösung des Strong-CP-Problems }
	
	Das Strong-CP-Problem fragt, warum der CP-verletzende Parameter \(\theta_{\text{QCD}}\) in QCD experimentell unter \(10^{-10}\) liegt, obwohl das Standardmodell Werte bis 1 erlaubt. Die fraktale DVFT bietet eine natürliche Lösung ohne Axionen oder Feinabstimmung.
	
	In fraktaler DVFT ist das Vakuumphasenfeld \(\theta\) global und einzig, da es aus T0s universellem Zeitfeld emergiert. Die Phase ist nicht lokal wählbar; die Freiheit, \(\theta\) zu drehen, existiert nicht, weil das Feld physisch und fraktal verbunden ist.
	
	Daher \(\theta_{\text{QCD}} = 0\) ist der einzige mathematisch erlaubte Wert. Die fraktale Selbstähnlichkeit eliminiert Duplizierbarkeit der Phase.
	
	Dies löst das Problem sauber: Keine Axionen, keine Feinabstimmung, volle Übereinstimmung mit Experiment. Starke konzeptionelle Triumph der fraktalen DVFT.

\end{document}
