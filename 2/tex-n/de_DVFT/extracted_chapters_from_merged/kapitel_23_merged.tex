\documentclass[12pt,a4paper]{article}
\usepackage[utf8]{inputenc}
\usepackage[T1]{fontenc}
\usepackage[ngerman]{babel}
\usepackage{amsmath}
\usepackage{amsfonts}
\usepackage{amssymb}
\usepackage{geometry}
\geometry{a4paper,left=2.5cm,right=2.5cm,top=2.5cm,bottom=2.5cm}
\usepackage{fancyhdr}
\usepackage{enumitem}
\usepackage{tcolorbox}
\usepackage{physics}
\usepackage{hyperref}

% Hyperref als eines der letzten Pakete laden
\hypersetup{
	unicode=true,
	pdfencoding=unicode,
	bookmarksopen=true
}

% Saubere PDF-Lesezeichen
\pdfstringdefDisableCommands{%
	\def\Lambda{Lambda}%
	\def\Delta{Delta}%
	\def\approx{etwa}%
	\def\Sigma{Sigma}%
	\def\eta{eta}%
	\def\psi{psi}%
}





\title{Kapitel 23: Neutronenlebensdauer-Diskrepanz gelöst}
\author{}
\date{}

\begin{document}

\maketitle

\section{Kapitel 23: Neutronenlebensdauer-Diskrepanz gelöst }
	
	Dieses Kapitel präsentiert eine rigorose Erklärung der Neutronenlebensdauer-Diskrepanz unter Verwendung der fraktalen T0-DVFT. Die Diskrepanz – etwa 879,5 s in Flaschenexperimenten vs. etwa 888,0 s in Strahlexperimenten – besteht seit mehr als einem Jahrzehnt und widersetzt sich Standardmodell-Interpretation.
	
	Fraktale T0-Anpassung: DVFT löst die Diskrepanz, indem sie Neutronenzerfall als Vakuumamplituden-Relaxationsprozess behandelt, empfindlich auf Umgebungsvakuumkonfiguration. Vakuumfeld \(\Phi = \rho e^{i\theta}\) aus T0s Dualität \(T(x,t) \cdot m(x,t) = 1\), wobei \(\rho \propto 1/T\).
	
	Flaschen-Einschränkung modifiziert \(T(x)\)-Feld leicht: \(\Delta T/T \sim 10^{-9}\). Dies senkt Zerfallsbarriere über \(\rho \propto 1/T\), ergibt \(\tau_{\text{Flasche}} \approx 879\) s. Strahlbedingungen erhalten natürliches \(T_0\), ergibt \(\tau_{\text{Strahl}} \approx 888\) s. Die 1\%-Differenz folgt aus T0s \(\xi = 4/3 \times 10^{-4}\) ohne freie Parameter.
	
	Dies ist die erste Erklärung konsistent mit allen experimentellen Daten, Größe der Diskrepanz (9 s), Umgebungsabhängigkeit, vereinheitlichten T0-DVFT-Struktur. Keine neuen Teilchen oder exotischen Kanäle erforderlich.
	
	Die Neutronenlebensdauer-Diskrepanz ist direkter experimenteller Beweis für T0s fundamentale Zeitfeldstruktur.

\end{document}
