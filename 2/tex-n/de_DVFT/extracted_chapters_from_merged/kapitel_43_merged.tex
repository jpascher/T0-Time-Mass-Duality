\documentclass[12pt,a4paper]{article}
\usepackage[utf8]{inputenc}
\usepackage[T1]{fontenc}
\usepackage[ngerman]{babel}
\usepackage{amsmath}
\usepackage{amsfonts}
\usepackage{amssymb}
\usepackage{geometry}
\geometry{a4paper,left=2.5cm,right=2.5cm,top=2.5cm,bottom=2.5cm}
\usepackage{fancyhdr}
\usepackage{enumitem}
\usepackage{tcolorbox}
\usepackage{physics}
\usepackage{hyperref}

% Hyperref als eines der letzten Pakete laden
\hypersetup{
	unicode=true,
	pdfencoding=unicode,
	bookmarksopen=true
}

% Saubere PDF-Lesezeichen
\pdfstringdefDisableCommands{%
	\def\Lambda{Lambda}%
	\def\Delta{Delta}%
	\def\approx{etwa}%
	\def\Sigma{Sigma}%
	\def\eta{eta}%
	\def\psi{psi}%
}





\title{Kapitel 43: Fundamentale Axiome und Konstanten}
\author{}
\date{}

\begin{document}

\maketitle

\section{Kapitel 43: Fundamentale Axiome und Konstanten }
	
	Core Axiome: Vakuum physisches Medium, Feld \(\Phi\), Dualität, etc.
	
	Universum als materielles Medium mit mechanischen Konstanten aus T0.
	
	Diese Kapitel bilden eine einheitliche fraktale narrative der Physik, vereinheitlicht durch die T0-Theorie und den Parameter \(\xi\).

\end{document}
