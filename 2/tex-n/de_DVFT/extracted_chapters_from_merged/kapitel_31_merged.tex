\documentclass[12pt,a4paper]{article}
\usepackage[utf8]{inputenc}
\usepackage[T1]{fontenc}
\usepackage[ngerman]{babel}
\usepackage{amsmath}
\usepackage{amsfonts}
\usepackage{amssymb}
\usepackage{geometry}
\geometry{a4paper,left=2.5cm,right=2.5cm,top=2.5cm,bottom=2.5cm}
\usepackage{fancyhdr}
\usepackage{enumitem}
\usepackage{tcolorbox}
\usepackage{physics}
\usepackage{hyperref}

% Hyperref als eines der letzten Pakete laden
\hypersetup{
	unicode=true,
	pdfencoding=unicode,
	bookmarksopen=true
}

% Saubere PDF-Lesezeichen
\pdfstringdefDisableCommands{%
	\def\Lambda{Lambda}%
	\def\Delta{Delta}%
	\def\approx{etwa}%
	\def\Sigma{Sigma}%
	\def\eta{eta}%
	\def\psi{psi}%
}





\title{Kapitel 31: Photoelektrischer Effekt und Laserphysik}
\author{}
\date{}

\begin{document}

\maketitle

\section{Kapitel 31: Photoelektrischer Effekt und Laserphysik }
	
	Dieses Dokument erklärt den photoelektrischen Effekt und die Laserphysik nur unter Verwendung der Prinzipien der an T0 angepassten fraktalen Dynamic Vacuum Field Theory (DVFT). Die angepasste DVFT basiert auf dem Vakuumfeld \(\Phi(x,t) = \rho(x,t) e^{i\theta(x,t)}\), wo \(\rho(x,t)\) Vakuumamplitude (energetisch, klassisch-ähnlich, bindende Struktur), proportional zu \(m(x,t)\) aus T0, und \(\theta(x,t)\) Vakuumphase (quantenmechanisch, oszillierend, kohärent).
	
	Photon = \(\theta\)-Phasen-Exzitation. Elektronenbindung = Amplituden-Barriere in \(\rho\). Emission erfordert \(\theta\)-Frequenz über \(\rho\)-Barriereschwelle.
	
	Stimulierte Emission = Phasen-Entrainment von \(\theta\). Laser-Kohärenz = globale \(\theta\)-Modus-Synchronisation. Laser-Verstärkung = wiederholte \(\theta\)-Phasen-Verstärkung, gesteuert durch konstruktive Interferenz von \(\theta\)-Modi. Auskopplung gibt stabilen, phasen-ausgerichteten \(\theta\)-Strahl ab: den Laser, emergierend aus T0-Dynamik.
	
	Conclusion: Der photoelektrische Effekt und die Laserphysik folgen natürlich aus der angepassten DVFT-Struktur der Vakuumfelder: Photon = \(\theta\)-Phasen-Exzitation, Elektronenbindung = Amplituden-Barriere in \(\rho\), Emission erfordert \(\theta\)-Frequenz über \(\rho\)-Barriereschwelle, Stimulierte Emission = Phasen-Entrainment von \(\theta\), Laser-Kohärenz = globale \(\theta\)-Modus-Synchronisation, Laser-Verstärkung = wiederholte \(\theta\)-Phasen-Verstärkung.
	
	Die angepasste DVFT bietet eine vereinheitlichte, physische Erklärung für optische und quantenmechanische Phänomene ohne Teilchen-Metaphern oder klassische Wellen-Teilchen-Dualität, fundiert in T0-Theorie.

\end{document}
