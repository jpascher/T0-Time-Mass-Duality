\documentclass[12pt,a4paper]{article}
\usepackage[utf8]{inputenc}
\usepackage[T1]{fontenc}
\usepackage[ngerman]{babel}
\usepackage{amsmath}
\usepackage{amsfonts}
\usepackage{amssymb}
\usepackage{geometry}
\geometry{a4paper,left=2.5cm,right=2.5cm,top=2.5cm,bottom=2.5cm}
\usepackage{fancyhdr}
\usepackage{enumitem}
\usepackage{tcolorbox}
\usepackage{physics}
\usepackage{hyperref}

% Hyperref als eines der letzten Pakete laden
\hypersetup{
	unicode=true,
	pdfencoding=unicode,
	bookmarksopen=true
}

% Saubere PDF-Lesezeichen
\pdfstringdefDisableCommands{%
	\def\Lambda{Lambda}%
	\def\Delta{Delta}%
	\def\approx{etwa}%
	\def\Sigma{Sigma}%
	\def\eta{eta}%
	\def\psi{psi}%
}





\title{Kapitel 38: Schwarze Löcher und Quantensingularitäten}
\author{}
\date{}

\begin{document}

\maketitle

\section{Kapitel 38: Schwarze Löcher und Quantensingularitäten }
	
	Fraktale DVFT eliminiert beide Singularitäten: Klassische Schwarze Löcher und Quantenpunkt-Singularitäten.
	
	Vakuumamplitude \(\rho \geq \rho_0 > 0\), nie null. Gravitation \(\nabla\rho\), nie divergent. Schwarze Löcher enthalten Vakuumphasen-Kondensate.
	
	Quanten-Singularitäten vermieden durch fraktale Deformationen aus \(|\psi|^2\).
	
	Vereinheitlichte Eliminierung von Singularitäten, überbrückt GR und QM.

\end{document}
