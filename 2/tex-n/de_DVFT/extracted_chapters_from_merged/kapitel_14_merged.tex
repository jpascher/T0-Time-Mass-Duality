\documentclass[12pt,a4paper]{article}
\usepackage[utf8]{inputenc}
\usepackage[T1]{fontenc}
\usepackage[ngerman]{babel}
\usepackage{amsmath}
\usepackage{amsfonts}
\usepackage{amssymb}
\usepackage{geometry}
\geometry{a4paper,left=2.5cm,right=2.5cm,top=2.5cm,bottom=2.5cm}
\usepackage{fancyhdr}
\usepackage{enumitem}
\usepackage{tcolorbox}
\usepackage{physics}
\usepackage{hyperref}

% Hyperref als eines der letzten Pakete laden
\hypersetup{
	unicode=true,
	pdfencoding=unicode,
	bookmarksopen=true
}

% Saubere PDF-Lesezeichen
\pdfstringdefDisableCommands{%
	\def\Lambda{Lambda}%
	\def\Delta{Delta}%
	\def\approx{etwa}%
	\def\Sigma{Sigma}%
	\def\eta{eta}%
	\def\psi{psi}%
}





\title{Kapitel 14: Raum-Schöpfungsgeschwindigkeit und kosmische Grenze}
\author{}
\date{}

\begin{document}

\maketitle

\section{Kapitel 14: Raum-Schöpfungsgeschwindigkeit und kosmische Grenze }
	
	Die Raum-Schöpfung in der fraktalen DVFT ist ein dynamischer Prozess, bei dem das Vakuumfeld \(\Phi\) eine Amplitude-Front ausbreitet, die den ''Raum'' definiert, wo \(\rho > 0\). Im Gegensatz zu expandierenden Modellen ist diese Ausbreitung endlich und fraktal begrenzt, mit Geschwindigkeit \(v_b(t) = dR(t)/dt < c_\rho = \sqrt{B/A} (1 - \epsilon / 2)\). Die Amplitudengleichung wird fraktal zu \(A \partial_t^2 \rho - B \nabla^{D_f} \rho + U'(\rho) (1 + \epsilon \ln \rho) = 0\).
	
	Für eine planare Front ergibt die Integration eine maximale Geschwindigkeit unter Lichtgeschwindigkeit. In sphärischer Symmetrie wird die Grenze durch \(\ddot{R} + 3H \dot{R} + 2/R = \Delta U / \sigma (1 + \epsilon \ln R)\) beschrieben. Die beobachtete Horizontgröße von etwa 46.5 Gly entsteht durch fraktale Wegintegration \(R_{\text{com}} = \int_0^{t_0} v_b(t) r^{\epsilon - 1} dt\).

\end{document}
