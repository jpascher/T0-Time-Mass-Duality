\documentclass[12pt,a4paper]{article}

% ==============================================================================
% T0 Theory: Shared English Preamble
% Version: 1.0
% Author: Johann Pascher
% Date: 2025
% ==============================================================================
%
% This is the standardized shared preamble for all English T0 Theory documents.
% Place this file in your document's directory or use a path like:
%   % ==============================================================================
% T0 Theory: Shared ENGLISH Preamble – Optimized for eBook/Book
% Version: 2.0 – Final 2026 (LuaLaTeX only) – ENGLISH corrected
% Author: Johann Pascher
% Date: January 2026
% ==============================================================================
%
% IMPORTANT: Compile EXCLUSIVELY with LuaLaTeX!
% In TeXstudio: Options → Configure TeXstudio → Build → Default Compiler → LuaLaTeX
%
% Required Fonts (install once):
% - Inter: https://fonts.google.com/specimen/Inter
% - JetBrains Mono: https://www.jetbrains.com/lp/mono/
% - Libertinus Math: https://github.com/libertinus-fonts/libertinus
% ==============================================================================

% === CHAPTER 1: BASIC PACKAGES (must come FIRST) ===
\RequirePackage{fontspec}
\RequirePackage{unicode-math}
\usepackage{chngcntr}
\setcounter{secnumdepth}{1}  % Nur Sections nummerieren (nicht subsections)
\setcounter{tocdepth}{1}     % Nur Sections im TOC (nicht subsections)
\makeatletter
\@ifundefined{c@chapter}{}{\counterwithout{section}{chapter}}  % Falls Kapitel existieren
\makeatother
\counterwithout{subsection}{section}  % Löse Verknüpfung
% === CHAPTER 2: LANGUAGE (ENGLISH) ===
\usepackage[english]{babel}
\usepackage{microtype}                    % IMPORTANT for better hyphenation!

% Typography settings for better line breaking
\frenchspacing                     % Correct English spacing after punctuation
\emergencystretch=3em              % Allows more stretch for difficult lines
\tolerance=2500                    % Higher tolerance for line breaks
\hbadness=10000                    % Suppresses "underfull hbox" warnings
\hfuzz=2pt                         % Allows minimal overfull
\pretolerance=150                  % Better word breaking

% Prevent bad page breaks
\clubpenalty=10000           % No "orphans"
\widowpenalty=10000          % No "widows"
\displaywidowpenalty=10000   % Also with equations
\brokenpenalty=10000         % No broken words across pages

% Explicit hyphenation for long technical words
\hyphenation{Fun-da-men-tal Frac-tal-Ge-o-met-ric Field The-o-ry Meth-od-o-log-i-cal}
\hyphenation{Re-vi-sion-ism Quan-ti-za-tion U-ni-fi-ca-tion Ef-fec-tive}
\hyphenation{Re-nor-mal-iz-a-bil-i-ty Sin-gu-lar-i-ties Con-cil-i-a-tion}
\hyphenation{E-mer-gence Phe-nom-e-no-log-i-cal Doc-u-men-ta-tion A-nal-y-sis}
\hyphenation{Grav-i-ta-tion Quan-tum Me-chan-ics Dog-ma-tism Con-se-quent}
\hyphenation{Par-al-lel-ism Im-ple-men-ta-tion Per-tur-ba-tions}
\hyphenation{Geo-met-ric Ar-ti-fact In-com-pat-i-bil-i-ty Con-struc-tive}
\hyphenation{Frac-tal Di-men-sion-less In-ves-ti-ga-tion De-scrip-tion}
\hyphenation{In-ter-pre-ta-tion Phe-nom-e-no-log-i-cal Math-e-mat-i-cal}
\hyphenation{Phi-lo-soph-i-cal Le-git-i-ma-tion Ap-pli-ca-tion Der-i-va-tion}
\hyphenation{U-ni-fi-ca-tion As-sump-tion Con-cep-tion Ex-pec-ta-tion}
\hyphenation{Sym-me-try-ex-ten-sion O-ver-all-pic-ture Chal-lenge}
\hyphenation{In-ter-ac-tion Ma-te-ri-al Ap-proach Per-spec-tive Pro-ce-dure}

% === CHAPTER 3: FONTS (with proper ligatures) ===
\setmainfont{Inter}[
Scale=1.02,
UprightFont=*-Regular,
BoldFont=*-Bold,
ItalicFont=*-Italic,
BoldItalicFont=*-BoldItalic,
Ligatures=TeX,           % IMPORTANT for proper typography
Language=English         % Explicit language support
]
\setsansfont{Inter}[
Scale=MatchLowercase,
Ligatures=TeX,
Language=English
]
\setmonofont{JetBrains Mono}[
Scale=0.95,
Language=English
]

% Math Font (simple & stable) – MUST come AFTER language definition
% IMPORTANT: Libertinus Math for correct \underbrace display!
\setmathfont{Libertinus Math}[Scale=1.0]

% === CHAPTER 4: MATHEMATICS PACKAGES (in STRICT order!) ===
% IMPORTANT: mathtools must come BEFORE unicode-math for some commands!
\usepackage{mathtools}           % FIRST mathtools!

% Then the rest
\usepackage{amsmath, amsfonts, amsthm}

% SIUNITX MUST be loaded BEFORE physics!
\usepackage{siunitx}
\sisetup{
	locale=US,                    % ENGLISH settings for SI units!
	group-separator={,},          % Thousands separator comma
	output-decimal-marker={.},    % Decimal separator point
	per-mode=symbol,
	separate-uncertainty=true
}

% Custom SI units used in narrative and books
\DeclareSIUnit\gigalightyear{Gly}
\DeclareSIUnit\mev{MeV}

% physics – MUST be loaded AFTER siunitx and mathtools
\usepackage{physics}

% === CHAPTER 5: ADDITIONS from pdflatex best practices ===
\usepackage{colortbl}        % Colored tables (ESSENTIAL!)
\usepackage{placeins}        % Float control: \FloatBarrier
\usepackage{subcaption}      % Subfigures
\usepackage{xurl}            % Better URL line breaking
% Hyphenation for URLs in bibliography
\def\UrlBreaks{\do\/\do-}

% === CHAPTER 6: PAGE LAYOUT
% =============================================================================
% SECTION 2: Page Geometry – 6" × 9" Buchformat
% =============================================================================
\usepackage[paperwidth=6in, paperheight=9in,
top=0.9in,
bottom=1.1in,
inner=0.9in,            % Größerer Innenrand für Bindung
outer=0.6in,            % Kleinerer Außenrand → mehr Text pro Seite
bindingoffset=0.5in,    % Puffer für Bindung (Steg)
twoside]{geometry}
\setlength{\headheight}{15pt}
%\usepackage[paperwidth=8.25in, paperheight=11in,
%top=1.0in,
%bottom=1.0in,
%left=1.0in,
%right=1.0in,
%twoside=false
% === CHAPTER 7: GRAPHICS AND TABLES ===
\usepackage{graphicx}
\usepackage[table,xcdraw]{xcolor}
% T0 brand colors
\definecolor{gold}{RGB}{255,215,0}
\definecolor{blue}{rgb}{0,0,1}
\definecolor{boxgray}{RGB}{240,240,240}
\definecolor{deepblue}{RGB}{0,0,127}
\definecolor{deepgreen}{RGB}{0,127,0}
\definecolor{deepred}{RGB}{191,0,0}
\definecolor{t0blue}{RGB}{33,150,243}
\definecolor{t0green}{RGB}{76,175,80}
\definecolor{t0orange}{RGB}{255,152,0}
\definecolor{t0purple}{RGB}{156,39,176}
\definecolor{t0red}{RGB}{244,67,54}
\definecolor{t0yellow}{RGB}{255,204,0}
\usepackage{tikz}
\usetikzlibrary{arrows.meta,positioning,shapes.geometric,decorations.pathmorphing,patterns,shapes.arrows,intersections}
\usepackage{pgfplots}
\pgfplotsset{compat=1.18}
\usepackage{quantikz}
\usepackage[most]{tcolorbox}
\tcbuselibrary{breakable}

% === WICHTIG: Algorithm-Konflikt umgehen ===
% Option: algorithmic mit GROSSBUCHSTABEN
% Gemeinsame Box für Experimente
\newtcolorbox{experimentbox}[1][]{
	colback=green!5!white,
	colframe=t0green!80!black,
	fonttitle=\bfseries,
	title={{#1}},
	breakable
}

% Abstract-Fallback
\ifdefined\abstract\else
\newenvironment{abstract}{\section*{\abstractname}\itshape\small\par\bigskip}{\bigskip}
\fi

% === MAKROS SICHER NEU DEFINIEREN / ÜBERSCHREIBEN ===
% Definiere Makros OHNE doppelte Subskripte
\newcommand{\phipar}{\phi_{\mathrm{par}}}
%\newcommand{\xipar}{\xi_{\mathrm{par}}}
\newcommand{\Qphipar}{Q_{\phi_{\mathrm{par}}}}
\newcommand{\rphipar}{r_{\phi_{\mathrm{par}}}}
\newcommand{\logphipar}{\log_{\phi_{\mathrm{par}}}}
\newcommand{\CHSH}{\text{CHSH}}
\usepackage{booktabs}
\usepackage{array}
\usepackage{longtable}
\usepackage{float}
\usepackage{adjustbox}
\usepackage{rotating}
\usepackage{tabularx}
\usepackage{makecell}
\usepackage{multirow}

% === CHAPTER 8: DOCUMENT FORMATTING ===
\usepackage{fancyhdr}
\renewcommand{\headrulewidth}{0.4pt}
\renewcommand{\footrulewidth}{0.4pt}
\usepackage{tocloft}

\usepackage{enumitem}
\setlist[itemize]{leftmargin=*, topsep=2pt, partopsep=0pt, parsep=2pt, itemsep=2pt}
\setlist[enumerate]{leftmargin=*, topsep=2pt, partopsep=0pt, parsep=2pt, itemsep=2pt}
\usepackage{setspace}
\usepackage{ragged2e}
\usepackage{multicol}

% === CHAPTER 9: CODE AND ALGORITHMS ===
\usepackage{algorithm}
\usepackage{algorithmic}
\usepackage{listings}
\lstset{
	basicstyle=\ttfamily\footnotesize,
	breaklines=true,
	breakatwhitespace=true,
	columns=flexible,
	keepspaces=true,
	showstringspaces=false,
	frame=single,
	xleftmargin=0pt,
	xrightmargin=0pt,
	literate=              % For special characters in code listings
	{ä}{{\"a}}1 {ö}{{\"o}}1 {ü}{{\"u}}1 {ß}{{\ss}}1
	{Ä}{{\"A}}1 {Ö}{{\"O}}1 {Ü}{{\"U}}1
}
\usepackage{mdframed}

% === CHAPTER 10: ADDITIONAL PACKAGES ===
\usepackage{pdflscape}
\usepackage{braket}
\usepackage{cancel}
\usepackage{caption}
\captionsetup{format=plain, labelfont=bf, justification=centering}
\usepackage{csquotes}
\usepackage{gensymb}
\usepackage{textcomp}
\usepackage{textgreek}
\usepackage{upgreek}
\usepackage{url}
\usepackage{slashed}
\usepackage{bm}

% === CHAPTER 11: HYPERREF (must come SECOND TO LAST!) ===
\usepackage{hyperref}
\hypersetup{
	colorlinks=true,
	linkcolor=black,
	citecolor=black,
	urlcolor=black,
	breaklinks=true,           % IMPORTANT for special characters in URLs!
	bookmarksnumbered=true,
	unicode=true,
	pdfencoding=auto,
	pdflang=en,                % Set PDF language to English
	pdfsubject={T0 Theory - Fundamental Fractal-Geometric Field Theory}
}

% Fix for unicode-math symbols in PDF bookmarks
\pdfstringdefDisableCommands{%
	\def\xi{xi}%
	\def\alpha{alpha}%
	\def\beta{beta}%
	\def\gamma{gamma}%
	\def\delta{delta}%
	\def\Delta{Delta}%
	\def\epsilon{epsilon}%
	\def\varepsilon{epsilon}%
	\def\theta{theta}%
	\def\kappa{kappa}%
	\def\lambda{lambda}%
	\def\mu{mu}%
	\def\nu{nu}%
	\def\pi{pi}%
	\def\rho{rho}%
	\def\sigma{sigma}%
	\def\tau{tau}%
	\def\phi{phi}%
	\def\chi{chi}%
	\def\psi{psi}%
	\def\omega{omega}%
	\def\Omega{Omega}%
	\def\Lambda{Lambda}%
	\def\times{x}%
	\def\cdot{*}%
	\def\pm{+/-}%
	\def\approx{~}%
	\def\sim{~}%
	\def\equiv{=}%
	\def\ell{l}%
	\def\hbar{h}%
	\def\rightarrow{->}%
	\def\leftarrow{<-}%
	\def\Rightarrow{=>}%
	\def\Leftarrow{<=}%
	\def\propto{~}%
	\def\mitxi{xi}%
	\def\mitalpha{alpha}%
	\def\mitbeta{beta}%
	\def\mitgamma{gamma}%
	\def\mitdelta{delta}%
	\def\mitDelta{Delta}%
	\def\mitepsilon{epsilon}%
	\def\mitvarepsilon{epsilon}%
	\def\mittheta{theta}%
	\def\mitkappa{kappa}%
	\def\mitlambda{lambda}%
	\def\mitLambda{Lambda}%
	\def\mitmu{mu}%
	\def\mitnu{nu}%
	\def\mitpi{pi}%
	\def\mitrho{rho}%
	\def\mitsigma{sigma}%
	\def\mittau{tau}%
	\def\mitphi{phi}%
	\def\mitchi{chi}%
	\def\mitpsi{psi}%
	\def\mitomega{omega}%
	\def\mitOmega{Omega}%
}

% === CHAPTER 12: BOOKMARK (must come AFTER hyperref!) ===
\usepackage{bookmark}

% === CHAPTER 13: CLEVEREF (ENGLISH LABELS) ===
\usepackage[english]{cleveref}
\crefname{equation}{Equation}{Equations}
\crefname{figure}{Figure}{Figures}
\crefname{table}{Table}{Tables}
\crefname{section}{Section}{Sections}
\crefname{chapter}{Chapter}{Chapters}
\crefname{theorem}{Theorem}{Theorems}
\crefname{lemma}{Lemma}{Lemmas}
\crefname{definition}{Definition}{Definitions}
\crefname{example}{Example}{Examples}
\crefname{remark}{Remark}{Remarks}

% === CUSTOM ENVIRONMENTS ===
% Alternative interpretation environment
\newenvironment{alternative}{%
	\begin{mdframed}[linecolor=black!30,linewidth=1pt,roundcorner=4pt,backgroundcolor=black!5]%
	}{%
	\end{mdframed}%
}

% Photon/particle environment
\newenvironment{photon}{%
	\begin{mdframed}[linecolor=blue!30,linewidth=1pt,roundcorner=4pt,backgroundcolor=blue!5]%
	}{%
	\end{mdframed}%
}

% Koide formula box environment
\newenvironment{koidebox}{%
	\begin{mdframed}[linecolor=green!30,linewidth=1pt,roundcorner=4pt,backgroundcolor=green!5]%
	}{%
	\end{mdframed}%
}

% Erkenntnis/insight environment
\newenvironment{erkenntnis}{%
	\begin{mdframed}[linecolor=orange!30,linewidth=1pt,roundcorner=4pt,backgroundcolor=orange!5]%
	}{%
	\end{mdframed}%
}

% Beziehung/relationship environment
\newenvironment{beziehung}{%
	\begin{mdframed}[linecolor=purple!30,linewidth=1pt,roundcorner=4pt,backgroundcolor=purple!5]%
	}{%
	\end{mdframed}%
}

% Derivation environment
\newenvironment{derivation}{%
	\begin{mdframed}[linecolor=teal!30,linewidth=1pt,roundcorner=4pt,backgroundcolor=teal!5]%
	}{%
	\end{mdframed}%
}

% Abhandlung/treatise environment
\newenvironment{abhandlung}{%
	\begin{mdframed}[linecolor=brown!30,linewidth=1pt,roundcorner=4pt,backgroundcolor=brown!5]%
	}{%
	\end{mdframed}%
}

% Anwendung/application environment
\newenvironment{anwendung}{%
	\begin{mdframed}[linecolor=cyan!30,linewidth=1pt,roundcorner=4pt,backgroundcolor=cyan!5]%
	}{%
	\end{mdframed}%
}

% Additional common environments
\newenvironment{konsequenz}{%
	\begin{mdframed}[linecolor=red!30,linewidth=1pt,roundcorner=4pt,backgroundcolor=red!5]%
	}{%
	\end{mdframed}%
}

\newenvironment{schlussfolgerung}{%
	\begin{mdframed}[linecolor=gray!30,linewidth=1pt,roundcorner=4pt,backgroundcolor=gray!5]%
	}{%
	\end{mdframed}%
}

\newenvironment{result}{%
	\begin{mdframed}[linecolor=violet!30,linewidth=1pt,roundcorner=4pt,backgroundcolor=violet!5]%
	}{%
	\end{mdframed}%
}

% Formula environment
\newenvironment{formula}{%
	\begin{mdframed}[linecolor=yellow!30,linewidth=1pt,roundcorner=4pt,backgroundcolor=yellow!5]%
	}{%
	\end{mdframed}%
}

% Revolutionaer/revolutionary environment
\newenvironment{revolutionaer}{%
	\begin{mdframed}[linecolor=red!50,linewidth=2pt,roundcorner=4pt,backgroundcolor=red!10]%
	}{%
	\end{mdframed}%
}

% Formel environment (German version of formula)
\newenvironment{formel}{%
	\begin{mdframed}[linecolor=yellow!30,linewidth=1pt,roundcorner=4pt,backgroundcolor=yellow!5]%
	}{%
	\end{mdframed}%
}

% Prinzip/principle environment
\newenvironment{prinzip}{%
	\begin{mdframed}[linecolor=blue!50,linewidth=2pt,roundcorner=4pt,backgroundcolor=blue!10]%
	}{%
	\end{mdframed}%
}

% Experimentell/experimental environment
\newenvironment{experimentell}{%
	\begin{mdframed}[linecolor=magenta!30,linewidth=1pt,roundcorner=4pt,backgroundcolor=magenta!5]%
	}{%
	\end{mdframed}%
}

% Neutrino environment
\newenvironment{neutrino}{%
	\begin{mdframed}[linecolor=cyan!40,linewidth=1pt,roundcorner=4pt,backgroundcolor=cyan!8]%
	}{%
	\end{mdframed}%
}

% Additional missing environments
\newenvironment{schluessel}{%
	\begin{mdframed}[linecolor=yellow!50,linewidth=1pt,roundcorner=4pt,backgroundcolor=yellow!10]%
	}{%
	\end{mdframed}%
}

\newenvironment{summary}{%
	\begin{mdframed}[linecolor=gray!40,linewidth=1pt,roundcorner=4pt,backgroundcolor=gray!8]%
	}{%
	\end{mdframed}%
}

\newenvironment{category}{%
	\begin{mdframed}[linecolor=pink!40,linewidth=1pt,roundcorner=4pt,backgroundcolor=pink!8]%
	}{%
	\end{mdframed}%
}

\newenvironment{sibox}{%
	\begin{mdframed}[linecolor=lime!40,linewidth=1pt,roundcorner=4pt,backgroundcolor=lime!8]%
	}{%
	\end{mdframed}%
}

% More missing environments
\newenvironment{documentbox}{%
	\begin{mdframed}[linecolor=teal!40,linewidth=1pt,roundcorner=4pt,backgroundcolor=teal!8]%
	}{%
	\end{mdframed}%
}

\newenvironment{t0box}{%
	\begin{mdframed}[linecolor=violet!40,linewidth=1pt,roundcorner=4pt,backgroundcolor=violet!8]%
	}{%
	\end{mdframed}%
}

\newenvironment{wichtig}{%
	\begin{mdframed}[linecolor=red!50,linewidth=2pt,roundcorner=4pt,backgroundcolor=red!10]%
	\textbf{Important:} 
	}{%
	\end{mdframed}%
}

\newenvironment{smbox}{%
	\begin{mdframed}[linecolor=orange!40,linewidth=1pt,roundcorner=4pt,backgroundcolor=orange!8]%
	}{%
	\end{mdframed}%
}

\newenvironment{pvbox}{%
	\begin{mdframed}[linecolor=purple!40,linewidth=1pt,roundcorner=4pt,backgroundcolor=purple!8]%
	}{%
	\end{mdframed}%
}

\newenvironment{numerisch}{%
	\begin{mdframed}[linecolor=blue!40,linewidth=1pt,roundcorner=4pt,backgroundcolor=blue!8]%
	}{%
	\end{mdframed}%
}

% More missing environments
\newenvironment{relation}{%
	\begin{mdframed}[linecolor=green!40,linewidth=1pt,roundcorner=4pt,backgroundcolor=green!8]%
	}{%
	\end{mdframed}%
}

\newenvironment{beweis}{%
	\begin{mdframed}[linecolor=brown!40,linewidth=1pt,roundcorner=4pt,backgroundcolor=brown!8]%
	\textbf{Proof:} 
	}{%
	\end{mdframed}%
}

\newenvironment{revolution}{%
	\begin{mdframed}[linecolor=red!60,linewidth=2pt,roundcorner=4pt,backgroundcolor=red!12]%
	}{%
	\end{mdframed}%
}

\newenvironment{key}{%
	\begin{mdframed}[linecolor=yellow!50,linewidth=1pt,roundcorner=4pt,backgroundcolor=yellow!10]%
	}{%
	\end{mdframed}%
}

\newenvironment{newperspective}{%
	\begin{mdframed}[linecolor=cyan!50,linewidth=1pt,roundcorner=4pt,backgroundcolor=cyan!10]%
	}{%
	\end{mdframed}%
}

\newenvironment{literatur}{%
	\begin{mdframed}[linecolor=gray!50,linewidth=1pt,roundcorner=4pt,backgroundcolor=gray!10]%
	}{%
	\end{mdframed}%
}

\newenvironment{folgerung}{%
	\begin{mdframed}[linecolor=teal!50,linewidth=1pt,roundcorner=4pt,backgroundcolor=teal!10]%
	}{%
	\end{mdframed}%
}

\newenvironment{principle}{%
	\begin{mdframed}[linecolor=blue!60,linewidth=2pt,roundcorner=4pt,backgroundcolor=blue!12]%
	}{%
	\end{mdframed}%
}

% Additional common environments
% ==============================================================================
% FROM HERE: YOUR DEFINITIONS (unchanged)
% ==============================================================================

\setcounter{tocdepth}{3}

% === CITATION COMMANDS ===
\providecommand{\citep}[1]{\cite{#1}}
\providecommand{\citet}[1]{\cite{#1}}

% === COLORS ===
\definecolor{gold}{RGB}{255,215,0}
\definecolor{blue}{rgb}{0,0,1}
\definecolor{boxgray}{RGB}{240,240,240}
\definecolor{deepblue}{RGB}{0,0,127}
\definecolor{deepgreen}{RGB}{0,127,0}
\definecolor{deepred}{RGB}{191,0,0}
\definecolor{t0blue}{RGB}{33,150,243}
\definecolor{t0green}{RGB}{76,175,80}
\definecolor{t0orange}{RGB}{255,152,0}
\definecolor{t0purple}{RGB}{156,39,176}
\definecolor{t0red}{RGB}{244,67,54}
\definecolor{t0yellow}{RGB}{255,204,0}

% === COLUMN TYPES ===
\newcolumntype{L}[1]{>{\raggedright\arraybackslash}p{#1}}
\newcolumntype{C}[1]{>{\centering\arraybackslash}p{#1}}
\newcolumntype{R}[1]{>{\raggedleft\arraybackslash}p{#1}}

% === HYPERREF SETTINGS (updated) ===
\hypersetup{
	colorlinks=true,
	linkcolor=t0blue,
	citecolor=t0blue,
	urlcolor=t0blue,
	breaklinks=true,
	bookmarksnumbered=true,
	pdfstartview=FitH,
	pdfencoding=auto,
	pdfdisplaydoctitle=true
}

% === ENGLISH THEOREM ENVIRONMENTS ===
\theoremstyle{plain}
\newtheorem{theorem}{Theorem}[section]
\newtheorem{lemma}[theorem]{Lemma}
\newtheorem{proposition}[theorem]{Proposition}
\newtheorem{corollary}[theorem]{Corollary}

\theoremstyle{definition}
\newtheorem{definition}[theorem]{Definition}
\newtheorem{example}[theorem]{Example}
\newtheorem{insight}[theorem]{Insight}
\newtheorem{discovery}[theorem]{Discovery}

\theoremstyle{remark}
\newtheorem{remark}[theorem]{Remark}
\newtheorem{axiom}{Axiom}
%\newtheorem{principle}{Principle}  % Commented out to avoid conflicts with document-specific definitions
%\newtheorem{warning}[theorem]{Warning}

% === T0-SPECIFIC COMMANDS ===
% (Here follow all your \newcommand and \providecommand definitions)
% These remain UNCHANGED as in your original preamble
% ==============================================================================
% SECTION 14: T0-Specific Commands
% ==============================================================================

% --- Core T0 Fields ---
\newcommand{\Tfield}{T(x,t)}
\providecommand{\Tfieldt}{T(\vec{x},t)}
\newcommand{\Efield}{E(x,t)}
\newcommand{\mfield}{m(x,t)}
\providecommand{\vecx}{\vec{x}}

% --- Lagrangian ---
\newcommand{\Lag}{\mathcal{L}}
\newcommand{\calL}{\mathcal{L}}

% --- Greek Letters and Constants ---
\newcommand{\alphaem}{\alpha}
\newcommand{\betaT}{\beta_T}
\newcommand{\xiT}{\xi}
\newcommand{\xipar}{\xi}

% --- Energy and Planck Units ---
\newcommand{\Ezero}{E_0}
\newcommand{\E}{E}
\newcommand{\EPlanck}{E_{\text{Pl}}}
\newcommand{\Mpl}{M_{\text{Pl}}}
\newcommand{\mP}{m_{\text{P}}}
\newcommand{\lP}{\ell_{\text{P}}}
\newcommand{\tP}{t_{\text{P}}}
\newcommand{\LPlanck}{\ell_{\text{Pl}}}
\newcommand{\TPlanck}{t_{\text{Pl}}}

% --- Coupling Constants ---
\newcommand{\Gnat}{G_{\text{nat}}}
\newcommand{\alphaEM}{\alpha_{\text{EM}}}
\newcommand{\alphaSI}{\alpha_{\text{SI}}}
\newcommand{\Hubble}{H_0}
\newcommand{\LCDM}{\Lambda\text{CDM}}
\newcommand{\natunits}{(nat. units)}

% --- T0 Model Parameters ---
\newcommand{\xigeom}{\xi_{\mathrm{geom}}}
\newcommand{\rzero}{r_{0}}
\newcommand{\xirat}{\xi_{\mathrm{rat}}}
\newcommand{\tzero}{t_{0}}
\newcommand{\Lambdat}{\Lambda_{\mathrm{t}}}
\newcommand{\EP}{E_{\text{P}}}
\newcommand{\Emu}{E_{\mu}}
\newcommand{\Ee}{E_{e}}
\newcommand{\Etau}{E_{\tau}}
\newcommand{\alphafine}{\alpha_{\mathrm{fine}}}
\newcommand{\alphal}{\alpha_{\ell}}
\newcommand{\Lzero}{\ell_{0}}
\newcommand{\Lp}{\ell_{\mathrm{P}}}

% --- Additional T0 Commands ---
\newcommand{\Kfrak}{K_{\text{frak}}}
\newcommand{\Dfrak}{D_{\text{frak}}}
\newcommand{\betapar}{\ensuremath{\beta_T}}
\newcommand{\alphapar}{\alpha}
\newcommand{\deltafield}{\delta \phi}
\newcommand{\deltam}{\delta m}
\newcommand{\deltaE}{\delta E}
\newcommand{\Exi}{E_{\xi}}
\newcommand{\Lxi}{\ell_{\xi}}
\newcommand{\rhoCMB}{\rho_{\text{CMB}}}
\newcommand{\rhoCasimir}{\rho_{\text{Casimir}}}
\newcommand{\Leff}{L_{\text{eff}}}
\newcommand{\CQCD}{C_{\mathrm{QCD}}}
\newcommand{\Kspec}{K_{\mathrm{spec}}}
\newcommand{\Tzero}{\ensuremath{T_0}}
\newcommand{\Eabs}{E_{\text{abs}}}
\newcommand{\taupar}{\tau}

% --- Provided Commands ---
\providecommand{\xiconst}{\xi_{\text{const}}}
\providecommand{\DhiggsT}{D_{\text{Higgs-T}}}
\providecommand{\rhoE}{\rho_{E}}
\providecommand{\Echar}{E_{\text{char}}}
\providecommand{\kfrac}{k_{\text{frac}}}
\providecommand{\alphaEMSI}{\alpha_{\text{EM,SI}}}
\providecommand{\alphaEMnat}{\alpha_{\text{EM,nat}}}
\providecommand{\betaTSI}{\beta_{T,\text{SI}}}
\providecommand{\betaTnat}{\beta_{T,\text{nat}}}
\providecommand{\Gsi}{G_{\text{SI}}}
\providecommand{\xiparSI}{\xi_{\text{SI}}}
\providecommand{\xiparnat}{\xi_{\text{nat}}}
\providecommand{\meff}{m_{\text{eff}}}
\providecommand{\Tzerot}{T_{0}(t)}
\providecommand{\mzerot}{m_{0}(t)}
\providecommand{\Ezeroabs}{E_{0,\text{abs}}}
\providecommand{\Epar}{E_{\text{par}}}
\providecommand{\Lnat}{\ell_{\text{nat}}}
\providecommand{\Tnat}{T_{\text{nat}}}
\providecommand{\xifrak}{\xi_{\text{frac}}}
\providecommand{\Tfrak}{T_{\text{frac}}}
\providecommand{\mfrak}{m_{\text{frac}}}
\providecommand{\Dfrac}{D_{\text{frac}}}
\providecommand{\EphotSI}{E_{\gamma,\text{SI}}}
\providecommand{\EphotNat}{E_{\gamma,\text{nat}}}
\providecommand{\Eabsint}{E_{\text{abs,int}}}
\providecommand{\mphoton}{m_{\gamma}}
\providecommand{\Evis}{E_{\text{vis}}}
\providecommand{\Cto}{C_{T0}}
\providecommand{\mytimes}{\times}
\providecommand{\lambdah}{\lambda_h}
\providecommand{\checkmarkx}{\checkmark}
\providecommand{\Enorm}{E_{\text{norm}}}
\providecommand{\Tobs}{T_{\text{obs}}}
\providecommand{\mobs}{m_{\text{obs}}}
\providecommand{\Eobs}{E_{\text{obs}}}
\providecommand{\Lobs}{\ell_{\text{obs}}}
\providecommand{\xobs}{\xi_{\text{obs}}}
\providecommand{\calE}{\mathcal{E}}
\providecommand{\calT}{\mathcal{T}}
\providecommand{\calM}{\mathcal{M}}
\providecommand{\alphag}{\alpha_g}
\providecommand{\Tmax}{T_{\text{max}}}
\providecommand{\mmin}{m_{\text{min}}}
\providecommand{\Lmax}{\ell_{\text{max}}}
\providecommand{\Emin}{E_{\text{min}}}
\providecommand{\Geff}{G_{\text{eff}}}
\providecommand{\rhoeff}{\rho_{\text{eff}}}
\providecommand{\xieff}{\xi_{\text{eff}}}
\providecommand{\Teff}{T_{\text{eff}}}
\providecommand{\hPlanck}{h}
\providecommand{\kB}{k_B}
\providecommand{\muB}{\mu_B}
\providecommand{\lambdaC}{\lambda_C}
\providecommand{\omegaP}{\omega_P}
\providecommand{\rhoP}{\rho_P}
\providecommand{\Tref}{T_{\text{ref}}}
\providecommand{\Eref}{E_{\text{ref}}}
\providecommand{\mref}{m_{\text{ref}}}
\providecommand{\Lref}{\ell_{\text{ref}}}
\providecommand{\xikonst}{\xi_0}
\providecommand{\Phiphoton}{\Phi_{\gamma}}
\providecommand{\etavis}{\eta_{\text{vis}}}
\providecommand{\pichar}{\pi}
\providecommand{\primrel}{\mathcal{P}_{\text{rel}}}
\providecommand{\warningx}{\textcolor{orange}{\textbf{!}}}
\providecommand{\phiT}{\phi_T}
\providecommand{\Lorentz}{\Lambda}
\providecommand{\Cconv}{C_{\text{conv}}}
\providecommand{\Df}{\Delta f}
\providecommand{\lambdazero}{\lambda_0}
\providecommand{\myapprox}{\approx}
\providecommand{\checked}{\checkmark}
\providecommand{\alphaWSI}{\alpha_W^{\text{SI}}}
\providecommand{\alphaWnat}{\alpha_W^{\text{nat}}}
\providecommand{\vect}[1]{\vec{#1}}
\providecommand{\Rzero}{R_0}
\providecommand{\Riem}{\mathcal{R}}
\providecommand{\nuzero}{\nu_0}
\providecommand{\mypi}{\pi}

% =============================================================================
% TCOLORBOX STYLES AND ENVIRONMENTS (English titles)
% =============================================================================
\tcbset{
	keyresult/.style={
		colback=blue!5!white,
		colframe=blue!75!black,
		title=Key Result,
		fonttitle=\bfseries
	},
	foundation/.style={
		colback=green!5!white,
		colframe=green!75!black,
		title=Foundation,
		fonttitle=\bfseries
	},
	alternative/.style={
		colback=orange!5!white,
		colframe=orange!75!black,
		title=Alternative,
		fonttitle=\bfseries
	},
	warningbox/.style={
		colback=red!5!white,
		colframe=red!75!black,
		title=Warning,
		fonttitle=\bfseries
	}
}

% (Here follow all your tcolorbox definitions with English titles)
\newtcolorbox{keyresultbox}[1][]{colback=blue!5!white,colframe=blue!75!black,fonttitle=\bfseries,title={#1},breakable}
\newtcolorbox{keyresult}[1][Key Result]{colback=blue!5!white,colframe=blue!75!black,fonttitle=\bfseries,title={#1},breakable}
\newtcolorbox{foundationbox}[1][]{colback=green!5!white,colframe=green!75!black,fonttitle=\bfseries,title={#1},breakable}
\newtcolorbox{foundation}[1][Foundation]{colback=green!5!white,colframe=green!75!black,fonttitle=\bfseries,title={#1},breakable}
\newtcolorbox{alternativebox}[1][]{colback=orange!5!white,colframe=orange!75!black,fonttitle=\bfseries,title={#1},breakable}
\newtcolorbox{warningboxenv}[1][Warning]{colback=red!5!white,colframe=red!75!black,fonttitle=\bfseries,title={#1},breakable}

\newtcolorbox{fundamental}[1][]{
	colback=boxgray,
	colframe=t0blue,
	fonttitle=\bfseries,
	title=#1,
	sharp corners,
	boxrule=2pt
}

\newtcolorbox{insightBox}[1][Insight]{colback=blue!5,colframe=t0blue,title={#1},fonttitle=\bfseries,breakable}
\newtcolorbox{discoveryBox}[1][Discovery]{colback=green!5,colframe=t0green,title={#1},fonttitle=\bfseries,breakable}
\newtcolorbox{revelation}[1][Revelation]{colback=red!5,colframe=t0red,title={#1},fonttitle=\bfseries,breakable}
\newtcolorbox{keypoint}[1][Key Point]{colback=blue!5,colframe=t0blue,title={#1},fonttitle=\bfseries,breakable}
\newtcolorbox{evidence}[1][Evidence]{colback=green!5,colframe=t0green,title={#1},fonttitle=\bfseries,breakable}
\newtcolorbox{conclusionBox}[1][Conclusion]{colback=gray!5,colframe=gray,title={#1},fonttitle=\bfseries,breakable}
\newtcolorbox{significance}[1][Significance]{colback=yellow!5,colframe=orange,title={#1},fonttitle=\bfseries,breakable}
\newtcolorbox{philosophical}[1][Philosophical]{colback=purple!5,colframe=purple,title={#1},fonttitle=\bfseries,breakable}
\newtcolorbox{implicationBox}[1][Implication]{colback=cyan!5,colframe=cyan,title={#1},fonttitle=\bfseries,breakable}
\newtcolorbox{perspectiveBox}[1][Perspective]{colback=blue!5,colframe=t0blue,title={#1},fonttitle=\bfseries,breakable}
\newtcolorbox{revolutionary}[1][Revolutionary]{colback=red!5,colframe=t0red,title={#1},fonttitle=\bfseries,breakable}

\newtcolorbox{technical}[1][Technical]{colback=gray!5,colframe=gray!75!black,title={#1},fonttitle=\bfseries,breakable}
\newtcolorbox{technicalBox}[1][Technical]{colback=gray!5,colframe=gray!75!black,title={#1},fonttitle=\bfseries,breakable}
\newtcolorbox{notationBox}[1][Notation]{colback=yellow!5,colframe=yellow!75!black,title={#1},fonttitle=\bfseries,breakable}
\newtcolorbox{verification}[1][Verification]{colback=orange!5!white,colframe=orange!75!black,fonttitle=\bfseries,title=#1}
\newtcolorbox{explanationBox}[1][Explanation]{colback=purple!5!white,colframe=purple!75!black,fonttitle=\bfseries,title=#1}
\newtcolorbox{interpretationBox}[1][Interpretation]{colback=cyan!5!white,colframe=cyan!75!black,fonttitle=\bfseries,title=#1}
\newtcolorbox{explanation}[1][Explanation]{colback=purple!5!white,colframe=purple!75!black,fonttitle=\bfseries,title=#1,breakable}
\newtcolorbox{interpretation}[1][Interpretation]{colback=cyan!5!white,colframe=cyan!75!black,fonttitle=\bfseries,title=#1,breakable}
\newtcolorbox{proof_step}[1][Proof Step]{colback=gray!5!white,colframe=gray!75!black,fonttitle=\bfseries,title=#1,breakable}
\newtcolorbox{experimental}[1][Experimental]{colback=teal!5!white,colframe=teal!75!black,fonttitle=\bfseries,title=#1,breakable}

\newtcolorbox{important}[1][Important]{colback=red!5!white,colframe=red!75!black,title={#1},fonttitle=\bfseries,breakable}
\newtcolorbox{warning}[1][Warning]{colback=orange!5!white,colframe=orange!75!black,title={#1},fonttitle=\bfseries,breakable}
\newtcolorbox{caution}[1][Caution]{colback=yellow!5!white,colframe=yellow!75!black,title={#1},fonttitle=\bfseries,breakable}
\newtcolorbox{highlight}[1][Highlight]{colback=yellow!10!white,colframe=yellow!75!black,title={#1},fonttitle=\bfseries,breakable}
\newtcolorbox{critical}[1][Critical]{colback=red!10!white,colframe=red!75!black,title={#1},fonttitle=\bfseries,breakable}

\newtcolorbox{analysis}[1][Analysis]{colback=blue!5!white,colframe=blue!75!black,title={#1},fonttitle=\bfseries,breakable}
\newtcolorbox{application}[1][Application]{colback=green!5!white,colframe=green!75!black,title={#1},fonttitle=\bfseries,breakable}
\newtcolorbox{experiment}[1][Experiment]{colback=cyan!5!white,colframe=cyan!75!black,title={#1},fonttitle=\bfseries,breakable}
\newtcolorbox{historical}[1][Historical]{colback=brown!5!white,colframe=brown!75!black,title={#1},fonttitle=\bfseries,breakable}
\newtcolorbox{numerical}[1][Numerical]{colback=gray!5!white,colframe=gray!75!black,title={#1},fonttitle=\bfseries,breakable}
\newtcolorbox{overview}[1][Overview]{colback=blue!5!white,colframe=blue!75!black,title={#1},fonttitle=\bfseries,breakable}
\newtcolorbox{speculation}[1][Speculation]{colback=purple!5!white,colframe=purple!75!black,title={#1},fonttitle=\bfseries,breakable}
\newtcolorbox{question}[1][Question]{colback=orange!5!white,colframe=orange!75!black,title={#1},fonttitle=\bfseries,breakable}
\newtcolorbox{method}[1][Method]{colback=teal!5!white,colframe=teal!75!black,title={#1},fonttitle=\bfseries,breakable}
\newtcolorbox{correct}[1][Correct]{colback=green!10!white,colframe=green!75!black,title={#1},fonttitle=\bfseries,breakable}
\newtcolorbox{units}[1][Units]{colback=gray!5!white,colframe=gray!75!black,title={#1},fonttitle=\bfseries,breakable}
\newtcolorbox{achievement}[1][Achievement]{colback=gold!5!white,colframe=orange!75!black,title={#1},fonttitle=\bfseries,breakable}
\newtcolorbox{equivalence}[1][Equivalence]{colback=cyan!5!white,colframe=cyan!75!black,title={#1},fonttitle=\bfseries,breakable}
\newtcolorbox{dimensional}[1][Dimensional Analysis]{colback=purple!5!white,colframe=purple!75!black,title={#1},fonttitle=\bfseries,breakable}

% === ADDITIONAL SIMPLE ENVIRONMENTS ===
\newenvironment{treatise}{\begin{quote}}{\end{quote}}
\newenvironment{gemeinsam}{\begin{quote}}{\end{quote}}
\newenvironment{vergleich}{\begin{quote}}{\end{quote}}
\newenvironment{vorteil}{\begin{quote}}{\end{quote}}
\newenvironment{common}{\begin{quote}}{\end{quote}}
\newenvironment{comparison}{\begin{quote}}{\end{quote}}
\newenvironment{advantage}{\begin{quote}}{\end{quote}}
\newenvironment{quantum}{\begin{quote}}{\end{quote}}

% === LAYOUT SETTINGS ===
\raggedbottom
\usepackage{environ}
\let\oldtabular\tabular
\let\endoldtabular\endtabular

\newenvironment{scaledtable}[1][0.85]{%
	\begingroup\footnotesize\setlength{\LTleft}{0pt}\setlength{\LTright}{0pt}%
}{%
	\endgroup%
}

\newcommand{\widetable}[1]{\resizebox{\textwidth}{!}{#1}}

% === TABLE OF CONTENTS FORMATTING ===
\renewcommand{\cftsecfont}{\color{blue}}
\renewcommand{\cftsubsecfont}{\color{blue}}
\renewcommand{\cftsecpagefont}{\color{blue}}
\renewcommand{\cftsubsecpagefont}{\color{blue}}
\renewcommand{\cfttoctitlefont}{\huge\bfseries\color{blue}}

% === DEFAULT HEADER AND FOOTER ===
\pagestyle{fancy}
\fancyhf{}
\fancyhead[L]{\textsc{T0 Theory}}
\fancyhead[R]{\textsc{J. Pascher}}
\fancyfoot[C]{\thepage}

% ==============================================================================
% End of Shared Preamble for English
% ==============================================================================
%
% Usage:
%   \documentclass[12pt,a4paper]{article}  % or book, report, etc.
%   % ==============================================================================
% T0 Theory: Shared ENGLISH Preamble – Optimized for eBook/Book
% Version: 2.0 – Final 2026 (LuaLaTeX only) – ENGLISH corrected
% Author: Johann Pascher
% Date: January 2026
% ==============================================================================
%
% IMPORTANT: Compile EXCLUSIVELY with LuaLaTeX!
% In TeXstudio: Options → Configure TeXstudio → Build → Default Compiler → LuaLaTeX
%
% Required Fonts (install once):
% - Inter: https://fonts.google.com/specimen/Inter
% - JetBrains Mono: https://www.jetbrains.com/lp/mono/
% - Libertinus Math: https://github.com/libertinus-fonts/libertinus
% ==============================================================================

% === CHAPTER 1: BASIC PACKAGES (must come FIRST) ===
\RequirePackage{fontspec}
\RequirePackage{unicode-math}
\usepackage{chngcntr}
\setcounter{secnumdepth}{1}  % Nur Sections nummerieren (nicht subsections)
\setcounter{tocdepth}{1}     % Nur Sections im TOC (nicht subsections)
\makeatletter
\@ifundefined{c@chapter}{}{\counterwithout{section}{chapter}}  % Falls Kapitel existieren
\makeatother
\counterwithout{subsection}{section}  % Löse Verknüpfung
% === CHAPTER 2: LANGUAGE (ENGLISH) ===
\usepackage[english]{babel}
\usepackage{microtype}                    % IMPORTANT for better hyphenation!

% Typography settings for better line breaking
\frenchspacing                     % Correct English spacing after punctuation
\emergencystretch=3em              % Allows more stretch for difficult lines
\tolerance=2500                    % Higher tolerance for line breaks
\hbadness=10000                    % Suppresses "underfull hbox" warnings
\hfuzz=2pt                         % Allows minimal overfull
\pretolerance=150                  % Better word breaking

% Prevent bad page breaks
\clubpenalty=10000           % No "orphans"
\widowpenalty=10000          % No "widows"
\displaywidowpenalty=10000   % Also with equations
\brokenpenalty=10000         % No broken words across pages

% Explicit hyphenation for long technical words
\hyphenation{Fun-da-men-tal Frac-tal-Ge-o-met-ric Field The-o-ry Meth-od-o-log-i-cal}
\hyphenation{Re-vi-sion-ism Quan-ti-za-tion U-ni-fi-ca-tion Ef-fec-tive}
\hyphenation{Re-nor-mal-iz-a-bil-i-ty Sin-gu-lar-i-ties Con-cil-i-a-tion}
\hyphenation{E-mer-gence Phe-nom-e-no-log-i-cal Doc-u-men-ta-tion A-nal-y-sis}
\hyphenation{Grav-i-ta-tion Quan-tum Me-chan-ics Dog-ma-tism Con-se-quent}
\hyphenation{Par-al-lel-ism Im-ple-men-ta-tion Per-tur-ba-tions}
\hyphenation{Geo-met-ric Ar-ti-fact In-com-pat-i-bil-i-ty Con-struc-tive}
\hyphenation{Frac-tal Di-men-sion-less In-ves-ti-ga-tion De-scrip-tion}
\hyphenation{In-ter-pre-ta-tion Phe-nom-e-no-log-i-cal Math-e-mat-i-cal}
\hyphenation{Phi-lo-soph-i-cal Le-git-i-ma-tion Ap-pli-ca-tion Der-i-va-tion}
\hyphenation{U-ni-fi-ca-tion As-sump-tion Con-cep-tion Ex-pec-ta-tion}
\hyphenation{Sym-me-try-ex-ten-sion O-ver-all-pic-ture Chal-lenge}
\hyphenation{In-ter-ac-tion Ma-te-ri-al Ap-proach Per-spec-tive Pro-ce-dure}

% === CHAPTER 3: FONTS (with proper ligatures) ===
\setmainfont{Inter}[
Scale=1.02,
UprightFont=*-Regular,
BoldFont=*-Bold,
ItalicFont=*-Italic,
BoldItalicFont=*-BoldItalic,
Ligatures=TeX,           % IMPORTANT for proper typography
Language=English         % Explicit language support
]
\setsansfont{Inter}[
Scale=MatchLowercase,
Ligatures=TeX,
Language=English
]
\setmonofont{JetBrains Mono}[
Scale=0.95,
Language=English
]

% Math Font (simple & stable) – MUST come AFTER language definition
% IMPORTANT: Libertinus Math for correct \underbrace display!
\setmathfont{Libertinus Math}[Scale=1.0]

% === CHAPTER 4: MATHEMATICS PACKAGES (in STRICT order!) ===
% IMPORTANT: mathtools must come BEFORE unicode-math for some commands!
\usepackage{mathtools}           % FIRST mathtools!

% Then the rest
\usepackage{amsmath, amsfonts, amsthm}

% SIUNITX MUST be loaded BEFORE physics!
\usepackage{siunitx}
\sisetup{
	locale=US,                    % ENGLISH settings for SI units!
	group-separator={,},          % Thousands separator comma
	output-decimal-marker={.},    % Decimal separator point
	per-mode=symbol,
	separate-uncertainty=true
}

% Custom SI units used in narrative and books
\DeclareSIUnit\gigalightyear{Gly}
\DeclareSIUnit\mev{MeV}

% physics – MUST be loaded AFTER siunitx and mathtools
\usepackage{physics}

% === CHAPTER 5: ADDITIONS from pdflatex best practices ===
\usepackage{colortbl}        % Colored tables (ESSENTIAL!)
\usepackage{placeins}        % Float control: \FloatBarrier
\usepackage{subcaption}      % Subfigures
\usepackage{xurl}            % Better URL line breaking
% Hyphenation for URLs in bibliography
\def\UrlBreaks{\do\/\do-}

% === CHAPTER 6: PAGE LAYOUT
% =============================================================================
% SECTION 2: Page Geometry – 6" × 9" Buchformat
% =============================================================================
\usepackage[paperwidth=6in, paperheight=9in,
top=0.9in,
bottom=1.1in,
inner=0.9in,            % Größerer Innenrand für Bindung
outer=0.6in,            % Kleinerer Außenrand → mehr Text pro Seite
bindingoffset=0.5in,    % Puffer für Bindung (Steg)
twoside]{geometry}
\setlength{\headheight}{15pt}
%\usepackage[paperwidth=8.25in, paperheight=11in,
%top=1.0in,
%bottom=1.0in,
%left=1.0in,
%right=1.0in,
%twoside=false
% === CHAPTER 7: GRAPHICS AND TABLES ===
\usepackage{graphicx}
\usepackage[table,xcdraw]{xcolor}
% T0 brand colors
\definecolor{gold}{RGB}{255,215,0}
\definecolor{blue}{rgb}{0,0,1}
\definecolor{boxgray}{RGB}{240,240,240}
\definecolor{deepblue}{RGB}{0,0,127}
\definecolor{deepgreen}{RGB}{0,127,0}
\definecolor{deepred}{RGB}{191,0,0}
\definecolor{t0blue}{RGB}{33,150,243}
\definecolor{t0green}{RGB}{76,175,80}
\definecolor{t0orange}{RGB}{255,152,0}
\definecolor{t0purple}{RGB}{156,39,176}
\definecolor{t0red}{RGB}{244,67,54}
\definecolor{t0yellow}{RGB}{255,204,0}
\usepackage{tikz}
\usetikzlibrary{arrows.meta,positioning,shapes.geometric,decorations.pathmorphing,patterns,shapes.arrows,intersections}
\usepackage{pgfplots}
\pgfplotsset{compat=1.18}
\usepackage{quantikz}
\usepackage[most]{tcolorbox}
\tcbuselibrary{breakable}

% === WICHTIG: Algorithm-Konflikt umgehen ===
% Option: algorithmic mit GROSSBUCHSTABEN
% Gemeinsame Box für Experimente
\newtcolorbox{experimentbox}[1][]{
	colback=green!5!white,
	colframe=t0green!80!black,
	fonttitle=\bfseries,
	title={{#1}},
	breakable
}

% Abstract-Fallback
\ifdefined\abstract\else
\newenvironment{abstract}{\section*{\abstractname}\itshape\small\par\bigskip}{\bigskip}
\fi

% === MAKROS SICHER NEU DEFINIEREN / ÜBERSCHREIBEN ===
% Definiere Makros OHNE doppelte Subskripte
\newcommand{\phipar}{\phi_{\mathrm{par}}}
%\newcommand{\xipar}{\xi_{\mathrm{par}}}
\newcommand{\Qphipar}{Q_{\phi_{\mathrm{par}}}}
\newcommand{\rphipar}{r_{\phi_{\mathrm{par}}}}
\newcommand{\logphipar}{\log_{\phi_{\mathrm{par}}}}
\newcommand{\CHSH}{\text{CHSH}}
\usepackage{booktabs}
\usepackage{array}
\usepackage{longtable}
\usepackage{float}
\usepackage{adjustbox}
\usepackage{rotating}
\usepackage{tabularx}
\usepackage{makecell}
\usepackage{multirow}

% === CHAPTER 8: DOCUMENT FORMATTING ===
\usepackage{fancyhdr}
\renewcommand{\headrulewidth}{0.4pt}
\renewcommand{\footrulewidth}{0.4pt}
\usepackage{tocloft}

\usepackage{enumitem}
\setlist[itemize]{leftmargin=*, topsep=2pt, partopsep=0pt, parsep=2pt, itemsep=2pt}
\setlist[enumerate]{leftmargin=*, topsep=2pt, partopsep=0pt, parsep=2pt, itemsep=2pt}
\usepackage{setspace}
\usepackage{ragged2e}
\usepackage{multicol}

% === CHAPTER 9: CODE AND ALGORITHMS ===
\usepackage{algorithm}
\usepackage{algorithmic}
\usepackage{listings}
\lstset{
	basicstyle=\ttfamily\footnotesize,
	breaklines=true,
	breakatwhitespace=true,
	columns=flexible,
	keepspaces=true,
	showstringspaces=false,
	frame=single,
	xleftmargin=0pt,
	xrightmargin=0pt,
	literate=              % For special characters in code listings
	{ä}{{\"a}}1 {ö}{{\"o}}1 {ü}{{\"u}}1 {ß}{{\ss}}1
	{Ä}{{\"A}}1 {Ö}{{\"O}}1 {Ü}{{\"U}}1
}
\usepackage{mdframed}

% === CHAPTER 10: ADDITIONAL PACKAGES ===
\usepackage{pdflscape}
\usepackage{braket}
\usepackage{cancel}
\usepackage{caption}
\captionsetup{format=plain, labelfont=bf, justification=centering}
\usepackage{csquotes}
\usepackage{gensymb}
\usepackage{textcomp}
\usepackage{textgreek}
\usepackage{upgreek}
\usepackage{url}
\usepackage{slashed}
\usepackage{bm}

% === CHAPTER 11: HYPERREF (must come SECOND TO LAST!) ===
\usepackage{hyperref}
\hypersetup{
	colorlinks=true,
	linkcolor=black,
	citecolor=black,
	urlcolor=black,
	breaklinks=true,           % IMPORTANT for special characters in URLs!
	bookmarksnumbered=true,
	unicode=true,
	pdfencoding=auto,
	pdflang=en,                % Set PDF language to English
	pdfsubject={T0 Theory - Fundamental Fractal-Geometric Field Theory}
}

% Fix for unicode-math symbols in PDF bookmarks
\pdfstringdefDisableCommands{%
	\def\xi{xi}%
	\def\alpha{alpha}%
	\def\beta{beta}%
	\def\gamma{gamma}%
	\def\delta{delta}%
	\def\Delta{Delta}%
	\def\epsilon{epsilon}%
	\def\varepsilon{epsilon}%
	\def\theta{theta}%
	\def\kappa{kappa}%
	\def\lambda{lambda}%
	\def\mu{mu}%
	\def\nu{nu}%
	\def\pi{pi}%
	\def\rho{rho}%
	\def\sigma{sigma}%
	\def\tau{tau}%
	\def\phi{phi}%
	\def\chi{chi}%
	\def\psi{psi}%
	\def\omega{omega}%
	\def\Omega{Omega}%
	\def\Lambda{Lambda}%
	\def\times{x}%
	\def\cdot{*}%
	\def\pm{+/-}%
	\def\approx{~}%
	\def\sim{~}%
	\def\equiv{=}%
	\def\ell{l}%
	\def\hbar{h}%
	\def\rightarrow{->}%
	\def\leftarrow{<-}%
	\def\Rightarrow{=>}%
	\def\Leftarrow{<=}%
	\def\propto{~}%
	\def\mitxi{xi}%
	\def\mitalpha{alpha}%
	\def\mitbeta{beta}%
	\def\mitgamma{gamma}%
	\def\mitdelta{delta}%
	\def\mitDelta{Delta}%
	\def\mitepsilon{epsilon}%
	\def\mitvarepsilon{epsilon}%
	\def\mittheta{theta}%
	\def\mitkappa{kappa}%
	\def\mitlambda{lambda}%
	\def\mitLambda{Lambda}%
	\def\mitmu{mu}%
	\def\mitnu{nu}%
	\def\mitpi{pi}%
	\def\mitrho{rho}%
	\def\mitsigma{sigma}%
	\def\mittau{tau}%
	\def\mitphi{phi}%
	\def\mitchi{chi}%
	\def\mitpsi{psi}%
	\def\mitomega{omega}%
	\def\mitOmega{Omega}%
}

% === CHAPTER 12: BOOKMARK (must come AFTER hyperref!) ===
\usepackage{bookmark}

% === CHAPTER 13: CLEVEREF (ENGLISH LABELS) ===
\usepackage[english]{cleveref}
\crefname{equation}{Equation}{Equations}
\crefname{figure}{Figure}{Figures}
\crefname{table}{Table}{Tables}
\crefname{section}{Section}{Sections}
\crefname{chapter}{Chapter}{Chapters}
\crefname{theorem}{Theorem}{Theorems}
\crefname{lemma}{Lemma}{Lemmas}
\crefname{definition}{Definition}{Definitions}
\crefname{example}{Example}{Examples}
\crefname{remark}{Remark}{Remarks}

% === CUSTOM ENVIRONMENTS ===
% Alternative interpretation environment
\newenvironment{alternative}{%
	\begin{mdframed}[linecolor=black!30,linewidth=1pt,roundcorner=4pt,backgroundcolor=black!5]%
	}{%
	\end{mdframed}%
}

% Photon/particle environment
\newenvironment{photon}{%
	\begin{mdframed}[linecolor=blue!30,linewidth=1pt,roundcorner=4pt,backgroundcolor=blue!5]%
	}{%
	\end{mdframed}%
}

% Koide formula box environment
\newenvironment{koidebox}{%
	\begin{mdframed}[linecolor=green!30,linewidth=1pt,roundcorner=4pt,backgroundcolor=green!5]%
	}{%
	\end{mdframed}%
}

% Erkenntnis/insight environment
\newenvironment{erkenntnis}{%
	\begin{mdframed}[linecolor=orange!30,linewidth=1pt,roundcorner=4pt,backgroundcolor=orange!5]%
	}{%
	\end{mdframed}%
}

% Beziehung/relationship environment
\newenvironment{beziehung}{%
	\begin{mdframed}[linecolor=purple!30,linewidth=1pt,roundcorner=4pt,backgroundcolor=purple!5]%
	}{%
	\end{mdframed}%
}

% Derivation environment
\newenvironment{derivation}{%
	\begin{mdframed}[linecolor=teal!30,linewidth=1pt,roundcorner=4pt,backgroundcolor=teal!5]%
	}{%
	\end{mdframed}%
}

% Abhandlung/treatise environment
\newenvironment{abhandlung}{%
	\begin{mdframed}[linecolor=brown!30,linewidth=1pt,roundcorner=4pt,backgroundcolor=brown!5]%
	}{%
	\end{mdframed}%
}

% Anwendung/application environment
\newenvironment{anwendung}{%
	\begin{mdframed}[linecolor=cyan!30,linewidth=1pt,roundcorner=4pt,backgroundcolor=cyan!5]%
	}{%
	\end{mdframed}%
}

% Additional common environments
\newenvironment{konsequenz}{%
	\begin{mdframed}[linecolor=red!30,linewidth=1pt,roundcorner=4pt,backgroundcolor=red!5]%
	}{%
	\end{mdframed}%
}

\newenvironment{schlussfolgerung}{%
	\begin{mdframed}[linecolor=gray!30,linewidth=1pt,roundcorner=4pt,backgroundcolor=gray!5]%
	}{%
	\end{mdframed}%
}

\newenvironment{result}{%
	\begin{mdframed}[linecolor=violet!30,linewidth=1pt,roundcorner=4pt,backgroundcolor=violet!5]%
	}{%
	\end{mdframed}%
}

% Formula environment
\newenvironment{formula}{%
	\begin{mdframed}[linecolor=yellow!30,linewidth=1pt,roundcorner=4pt,backgroundcolor=yellow!5]%
	}{%
	\end{mdframed}%
}

% Revolutionaer/revolutionary environment
\newenvironment{revolutionaer}{%
	\begin{mdframed}[linecolor=red!50,linewidth=2pt,roundcorner=4pt,backgroundcolor=red!10]%
	}{%
	\end{mdframed}%
}

% Formel environment (German version of formula)
\newenvironment{formel}{%
	\begin{mdframed}[linecolor=yellow!30,linewidth=1pt,roundcorner=4pt,backgroundcolor=yellow!5]%
	}{%
	\end{mdframed}%
}

% Prinzip/principle environment
\newenvironment{prinzip}{%
	\begin{mdframed}[linecolor=blue!50,linewidth=2pt,roundcorner=4pt,backgroundcolor=blue!10]%
	}{%
	\end{mdframed}%
}

% Experimentell/experimental environment
\newenvironment{experimentell}{%
	\begin{mdframed}[linecolor=magenta!30,linewidth=1pt,roundcorner=4pt,backgroundcolor=magenta!5]%
	}{%
	\end{mdframed}%
}

% Neutrino environment
\newenvironment{neutrino}{%
	\begin{mdframed}[linecolor=cyan!40,linewidth=1pt,roundcorner=4pt,backgroundcolor=cyan!8]%
	}{%
	\end{mdframed}%
}

% Additional missing environments
\newenvironment{schluessel}{%
	\begin{mdframed}[linecolor=yellow!50,linewidth=1pt,roundcorner=4pt,backgroundcolor=yellow!10]%
	}{%
	\end{mdframed}%
}

\newenvironment{summary}{%
	\begin{mdframed}[linecolor=gray!40,linewidth=1pt,roundcorner=4pt,backgroundcolor=gray!8]%
	}{%
	\end{mdframed}%
}

\newenvironment{category}{%
	\begin{mdframed}[linecolor=pink!40,linewidth=1pt,roundcorner=4pt,backgroundcolor=pink!8]%
	}{%
	\end{mdframed}%
}

\newenvironment{sibox}{%
	\begin{mdframed}[linecolor=lime!40,linewidth=1pt,roundcorner=4pt,backgroundcolor=lime!8]%
	}{%
	\end{mdframed}%
}

% More missing environments
\newenvironment{documentbox}{%
	\begin{mdframed}[linecolor=teal!40,linewidth=1pt,roundcorner=4pt,backgroundcolor=teal!8]%
	}{%
	\end{mdframed}%
}

\newenvironment{t0box}{%
	\begin{mdframed}[linecolor=violet!40,linewidth=1pt,roundcorner=4pt,backgroundcolor=violet!8]%
	}{%
	\end{mdframed}%
}

\newenvironment{wichtig}{%
	\begin{mdframed}[linecolor=red!50,linewidth=2pt,roundcorner=4pt,backgroundcolor=red!10]%
	\textbf{Important:} 
	}{%
	\end{mdframed}%
}

\newenvironment{smbox}{%
	\begin{mdframed}[linecolor=orange!40,linewidth=1pt,roundcorner=4pt,backgroundcolor=orange!8]%
	}{%
	\end{mdframed}%
}

\newenvironment{pvbox}{%
	\begin{mdframed}[linecolor=purple!40,linewidth=1pt,roundcorner=4pt,backgroundcolor=purple!8]%
	}{%
	\end{mdframed}%
}

\newenvironment{numerisch}{%
	\begin{mdframed}[linecolor=blue!40,linewidth=1pt,roundcorner=4pt,backgroundcolor=blue!8]%
	}{%
	\end{mdframed}%
}

% More missing environments
\newenvironment{relation}{%
	\begin{mdframed}[linecolor=green!40,linewidth=1pt,roundcorner=4pt,backgroundcolor=green!8]%
	}{%
	\end{mdframed}%
}

\newenvironment{beweis}{%
	\begin{mdframed}[linecolor=brown!40,linewidth=1pt,roundcorner=4pt,backgroundcolor=brown!8]%
	\textbf{Proof:} 
	}{%
	\end{mdframed}%
}

\newenvironment{revolution}{%
	\begin{mdframed}[linecolor=red!60,linewidth=2pt,roundcorner=4pt,backgroundcolor=red!12]%
	}{%
	\end{mdframed}%
}

\newenvironment{key}{%
	\begin{mdframed}[linecolor=yellow!50,linewidth=1pt,roundcorner=4pt,backgroundcolor=yellow!10]%
	}{%
	\end{mdframed}%
}

\newenvironment{newperspective}{%
	\begin{mdframed}[linecolor=cyan!50,linewidth=1pt,roundcorner=4pt,backgroundcolor=cyan!10]%
	}{%
	\end{mdframed}%
}

\newenvironment{literatur}{%
	\begin{mdframed}[linecolor=gray!50,linewidth=1pt,roundcorner=4pt,backgroundcolor=gray!10]%
	}{%
	\end{mdframed}%
}

\newenvironment{folgerung}{%
	\begin{mdframed}[linecolor=teal!50,linewidth=1pt,roundcorner=4pt,backgroundcolor=teal!10]%
	}{%
	\end{mdframed}%
}

\newenvironment{principle}{%
	\begin{mdframed}[linecolor=blue!60,linewidth=2pt,roundcorner=4pt,backgroundcolor=blue!12]%
	}{%
	\end{mdframed}%
}

% Additional common environments
% ==============================================================================
% FROM HERE: YOUR DEFINITIONS (unchanged)
% ==============================================================================

\setcounter{tocdepth}{3}

% === CITATION COMMANDS ===
\providecommand{\citep}[1]{\cite{#1}}
\providecommand{\citet}[1]{\cite{#1}}

% === COLORS ===
\definecolor{gold}{RGB}{255,215,0}
\definecolor{blue}{rgb}{0,0,1}
\definecolor{boxgray}{RGB}{240,240,240}
\definecolor{deepblue}{RGB}{0,0,127}
\definecolor{deepgreen}{RGB}{0,127,0}
\definecolor{deepred}{RGB}{191,0,0}
\definecolor{t0blue}{RGB}{33,150,243}
\definecolor{t0green}{RGB}{76,175,80}
\definecolor{t0orange}{RGB}{255,152,0}
\definecolor{t0purple}{RGB}{156,39,176}
\definecolor{t0red}{RGB}{244,67,54}
\definecolor{t0yellow}{RGB}{255,204,0}

% === COLUMN TYPES ===
\newcolumntype{L}[1]{>{\raggedright\arraybackslash}p{#1}}
\newcolumntype{C}[1]{>{\centering\arraybackslash}p{#1}}
\newcolumntype{R}[1]{>{\raggedleft\arraybackslash}p{#1}}

% === HYPERREF SETTINGS (updated) ===
\hypersetup{
	colorlinks=true,
	linkcolor=t0blue,
	citecolor=t0blue,
	urlcolor=t0blue,
	breaklinks=true,
	bookmarksnumbered=true,
	pdfstartview=FitH,
	pdfencoding=auto,
	pdfdisplaydoctitle=true
}

% === ENGLISH THEOREM ENVIRONMENTS ===
\theoremstyle{plain}
\newtheorem{theorem}{Theorem}[section]
\newtheorem{lemma}[theorem]{Lemma}
\newtheorem{proposition}[theorem]{Proposition}
\newtheorem{corollary}[theorem]{Corollary}

\theoremstyle{definition}
\newtheorem{definition}[theorem]{Definition}
\newtheorem{example}[theorem]{Example}
\newtheorem{insight}[theorem]{Insight}
\newtheorem{discovery}[theorem]{Discovery}

\theoremstyle{remark}
\newtheorem{remark}[theorem]{Remark}
\newtheorem{axiom}{Axiom}
%\newtheorem{principle}{Principle}  % Commented out to avoid conflicts with document-specific definitions
%\newtheorem{warning}[theorem]{Warning}

% === T0-SPECIFIC COMMANDS ===
% (Here follow all your \newcommand and \providecommand definitions)
% These remain UNCHANGED as in your original preamble
% ==============================================================================
% SECTION 14: T0-Specific Commands
% ==============================================================================

% --- Core T0 Fields ---
\newcommand{\Tfield}{T(x,t)}
\providecommand{\Tfieldt}{T(\vec{x},t)}
\newcommand{\Efield}{E(x,t)}
\newcommand{\mfield}{m(x,t)}
\providecommand{\vecx}{\vec{x}}

% --- Lagrangian ---
\newcommand{\Lag}{\mathcal{L}}
\newcommand{\calL}{\mathcal{L}}

% --- Greek Letters and Constants ---
\newcommand{\alphaem}{\alpha}
\newcommand{\betaT}{\beta_T}
\newcommand{\xiT}{\xi}
\newcommand{\xipar}{\xi}

% --- Energy and Planck Units ---
\newcommand{\Ezero}{E_0}
\newcommand{\E}{E}
\newcommand{\EPlanck}{E_{\text{Pl}}}
\newcommand{\Mpl}{M_{\text{Pl}}}
\newcommand{\mP}{m_{\text{P}}}
\newcommand{\lP}{\ell_{\text{P}}}
\newcommand{\tP}{t_{\text{P}}}
\newcommand{\LPlanck}{\ell_{\text{Pl}}}
\newcommand{\TPlanck}{t_{\text{Pl}}}

% --- Coupling Constants ---
\newcommand{\Gnat}{G_{\text{nat}}}
\newcommand{\alphaEM}{\alpha_{\text{EM}}}
\newcommand{\alphaSI}{\alpha_{\text{SI}}}
\newcommand{\Hubble}{H_0}
\newcommand{\LCDM}{\Lambda\text{CDM}}
\newcommand{\natunits}{(nat. units)}

% --- T0 Model Parameters ---
\newcommand{\xigeom}{\xi_{\mathrm{geom}}}
\newcommand{\rzero}{r_{0}}
\newcommand{\xirat}{\xi_{\mathrm{rat}}}
\newcommand{\tzero}{t_{0}}
\newcommand{\Lambdat}{\Lambda_{\mathrm{t}}}
\newcommand{\EP}{E_{\text{P}}}
\newcommand{\Emu}{E_{\mu}}
\newcommand{\Ee}{E_{e}}
\newcommand{\Etau}{E_{\tau}}
\newcommand{\alphafine}{\alpha_{\mathrm{fine}}}
\newcommand{\alphal}{\alpha_{\ell}}
\newcommand{\Lzero}{\ell_{0}}
\newcommand{\Lp}{\ell_{\mathrm{P}}}

% --- Additional T0 Commands ---
\newcommand{\Kfrak}{K_{\text{frak}}}
\newcommand{\Dfrak}{D_{\text{frak}}}
\newcommand{\betapar}{\ensuremath{\beta_T}}
\newcommand{\alphapar}{\alpha}
\newcommand{\deltafield}{\delta \phi}
\newcommand{\deltam}{\delta m}
\newcommand{\deltaE}{\delta E}
\newcommand{\Exi}{E_{\xi}}
\newcommand{\Lxi}{\ell_{\xi}}
\newcommand{\rhoCMB}{\rho_{\text{CMB}}}
\newcommand{\rhoCasimir}{\rho_{\text{Casimir}}}
\newcommand{\Leff}{L_{\text{eff}}}
\newcommand{\CQCD}{C_{\mathrm{QCD}}}
\newcommand{\Kspec}{K_{\mathrm{spec}}}
\newcommand{\Tzero}{\ensuremath{T_0}}
\newcommand{\Eabs}{E_{\text{abs}}}
\newcommand{\taupar}{\tau}

% --- Provided Commands ---
\providecommand{\xiconst}{\xi_{\text{const}}}
\providecommand{\DhiggsT}{D_{\text{Higgs-T}}}
\providecommand{\rhoE}{\rho_{E}}
\providecommand{\Echar}{E_{\text{char}}}
\providecommand{\kfrac}{k_{\text{frac}}}
\providecommand{\alphaEMSI}{\alpha_{\text{EM,SI}}}
\providecommand{\alphaEMnat}{\alpha_{\text{EM,nat}}}
\providecommand{\betaTSI}{\beta_{T,\text{SI}}}
\providecommand{\betaTnat}{\beta_{T,\text{nat}}}
\providecommand{\Gsi}{G_{\text{SI}}}
\providecommand{\xiparSI}{\xi_{\text{SI}}}
\providecommand{\xiparnat}{\xi_{\text{nat}}}
\providecommand{\meff}{m_{\text{eff}}}
\providecommand{\Tzerot}{T_{0}(t)}
\providecommand{\mzerot}{m_{0}(t)}
\providecommand{\Ezeroabs}{E_{0,\text{abs}}}
\providecommand{\Epar}{E_{\text{par}}}
\providecommand{\Lnat}{\ell_{\text{nat}}}
\providecommand{\Tnat}{T_{\text{nat}}}
\providecommand{\xifrak}{\xi_{\text{frac}}}
\providecommand{\Tfrak}{T_{\text{frac}}}
\providecommand{\mfrak}{m_{\text{frac}}}
\providecommand{\Dfrac}{D_{\text{frac}}}
\providecommand{\EphotSI}{E_{\gamma,\text{SI}}}
\providecommand{\EphotNat}{E_{\gamma,\text{nat}}}
\providecommand{\Eabsint}{E_{\text{abs,int}}}
\providecommand{\mphoton}{m_{\gamma}}
\providecommand{\Evis}{E_{\text{vis}}}
\providecommand{\Cto}{C_{T0}}
\providecommand{\mytimes}{\times}
\providecommand{\lambdah}{\lambda_h}
\providecommand{\checkmarkx}{\checkmark}
\providecommand{\Enorm}{E_{\text{norm}}}
\providecommand{\Tobs}{T_{\text{obs}}}
\providecommand{\mobs}{m_{\text{obs}}}
\providecommand{\Eobs}{E_{\text{obs}}}
\providecommand{\Lobs}{\ell_{\text{obs}}}
\providecommand{\xobs}{\xi_{\text{obs}}}
\providecommand{\calE}{\mathcal{E}}
\providecommand{\calT}{\mathcal{T}}
\providecommand{\calM}{\mathcal{M}}
\providecommand{\alphag}{\alpha_g}
\providecommand{\Tmax}{T_{\text{max}}}
\providecommand{\mmin}{m_{\text{min}}}
\providecommand{\Lmax}{\ell_{\text{max}}}
\providecommand{\Emin}{E_{\text{min}}}
\providecommand{\Geff}{G_{\text{eff}}}
\providecommand{\rhoeff}{\rho_{\text{eff}}}
\providecommand{\xieff}{\xi_{\text{eff}}}
\providecommand{\Teff}{T_{\text{eff}}}
\providecommand{\hPlanck}{h}
\providecommand{\kB}{k_B}
\providecommand{\muB}{\mu_B}
\providecommand{\lambdaC}{\lambda_C}
\providecommand{\omegaP}{\omega_P}
\providecommand{\rhoP}{\rho_P}
\providecommand{\Tref}{T_{\text{ref}}}
\providecommand{\Eref}{E_{\text{ref}}}
\providecommand{\mref}{m_{\text{ref}}}
\providecommand{\Lref}{\ell_{\text{ref}}}
\providecommand{\xikonst}{\xi_0}
\providecommand{\Phiphoton}{\Phi_{\gamma}}
\providecommand{\etavis}{\eta_{\text{vis}}}
\providecommand{\pichar}{\pi}
\providecommand{\primrel}{\mathcal{P}_{\text{rel}}}
\providecommand{\warningx}{\textcolor{orange}{\textbf{!}}}
\providecommand{\phiT}{\phi_T}
\providecommand{\Lorentz}{\Lambda}
\providecommand{\Cconv}{C_{\text{conv}}}
\providecommand{\Df}{\Delta f}
\providecommand{\lambdazero}{\lambda_0}
\providecommand{\myapprox}{\approx}
\providecommand{\checked}{\checkmark}
\providecommand{\alphaWSI}{\alpha_W^{\text{SI}}}
\providecommand{\alphaWnat}{\alpha_W^{\text{nat}}}
\providecommand{\vect}[1]{\vec{#1}}
\providecommand{\Rzero}{R_0}
\providecommand{\Riem}{\mathcal{R}}
\providecommand{\nuzero}{\nu_0}
\providecommand{\mypi}{\pi}

% =============================================================================
% TCOLORBOX STYLES AND ENVIRONMENTS (English titles)
% =============================================================================
\tcbset{
	keyresult/.style={
		colback=blue!5!white,
		colframe=blue!75!black,
		title=Key Result,
		fonttitle=\bfseries
	},
	foundation/.style={
		colback=green!5!white,
		colframe=green!75!black,
		title=Foundation,
		fonttitle=\bfseries
	},
	alternative/.style={
		colback=orange!5!white,
		colframe=orange!75!black,
		title=Alternative,
		fonttitle=\bfseries
	},
	warningbox/.style={
		colback=red!5!white,
		colframe=red!75!black,
		title=Warning,
		fonttitle=\bfseries
	}
}

% (Here follow all your tcolorbox definitions with English titles)
\newtcolorbox{keyresultbox}[1][]{colback=blue!5!white,colframe=blue!75!black,fonttitle=\bfseries,title={#1},breakable}
\newtcolorbox{keyresult}[1][Key Result]{colback=blue!5!white,colframe=blue!75!black,fonttitle=\bfseries,title={#1},breakable}
\newtcolorbox{foundationbox}[1][]{colback=green!5!white,colframe=green!75!black,fonttitle=\bfseries,title={#1},breakable}
\newtcolorbox{foundation}[1][Foundation]{colback=green!5!white,colframe=green!75!black,fonttitle=\bfseries,title={#1},breakable}
\newtcolorbox{alternativebox}[1][]{colback=orange!5!white,colframe=orange!75!black,fonttitle=\bfseries,title={#1},breakable}
\newtcolorbox{warningboxenv}[1][Warning]{colback=red!5!white,colframe=red!75!black,fonttitle=\bfseries,title={#1},breakable}

\newtcolorbox{fundamental}[1][]{
	colback=boxgray,
	colframe=t0blue,
	fonttitle=\bfseries,
	title=#1,
	sharp corners,
	boxrule=2pt
}

\newtcolorbox{insightBox}[1][Insight]{colback=blue!5,colframe=t0blue,title={#1},fonttitle=\bfseries,breakable}
\newtcolorbox{discoveryBox}[1][Discovery]{colback=green!5,colframe=t0green,title={#1},fonttitle=\bfseries,breakable}
\newtcolorbox{revelation}[1][Revelation]{colback=red!5,colframe=t0red,title={#1},fonttitle=\bfseries,breakable}
\newtcolorbox{keypoint}[1][Key Point]{colback=blue!5,colframe=t0blue,title={#1},fonttitle=\bfseries,breakable}
\newtcolorbox{evidence}[1][Evidence]{colback=green!5,colframe=t0green,title={#1},fonttitle=\bfseries,breakable}
\newtcolorbox{conclusionBox}[1][Conclusion]{colback=gray!5,colframe=gray,title={#1},fonttitle=\bfseries,breakable}
\newtcolorbox{significance}[1][Significance]{colback=yellow!5,colframe=orange,title={#1},fonttitle=\bfseries,breakable}
\newtcolorbox{philosophical}[1][Philosophical]{colback=purple!5,colframe=purple,title={#1},fonttitle=\bfseries,breakable}
\newtcolorbox{implicationBox}[1][Implication]{colback=cyan!5,colframe=cyan,title={#1},fonttitle=\bfseries,breakable}
\newtcolorbox{perspectiveBox}[1][Perspective]{colback=blue!5,colframe=t0blue,title={#1},fonttitle=\bfseries,breakable}
\newtcolorbox{revolutionary}[1][Revolutionary]{colback=red!5,colframe=t0red,title={#1},fonttitle=\bfseries,breakable}

\newtcolorbox{technical}[1][Technical]{colback=gray!5,colframe=gray!75!black,title={#1},fonttitle=\bfseries,breakable}
\newtcolorbox{technicalBox}[1][Technical]{colback=gray!5,colframe=gray!75!black,title={#1},fonttitle=\bfseries,breakable}
\newtcolorbox{notationBox}[1][Notation]{colback=yellow!5,colframe=yellow!75!black,title={#1},fonttitle=\bfseries,breakable}
\newtcolorbox{verification}[1][Verification]{colback=orange!5!white,colframe=orange!75!black,fonttitle=\bfseries,title=#1}
\newtcolorbox{explanationBox}[1][Explanation]{colback=purple!5!white,colframe=purple!75!black,fonttitle=\bfseries,title=#1}
\newtcolorbox{interpretationBox}[1][Interpretation]{colback=cyan!5!white,colframe=cyan!75!black,fonttitle=\bfseries,title=#1}
\newtcolorbox{explanation}[1][Explanation]{colback=purple!5!white,colframe=purple!75!black,fonttitle=\bfseries,title=#1,breakable}
\newtcolorbox{interpretation}[1][Interpretation]{colback=cyan!5!white,colframe=cyan!75!black,fonttitle=\bfseries,title=#1,breakable}
\newtcolorbox{proof_step}[1][Proof Step]{colback=gray!5!white,colframe=gray!75!black,fonttitle=\bfseries,title=#1,breakable}
\newtcolorbox{experimental}[1][Experimental]{colback=teal!5!white,colframe=teal!75!black,fonttitle=\bfseries,title=#1,breakable}

\newtcolorbox{important}[1][Important]{colback=red!5!white,colframe=red!75!black,title={#1},fonttitle=\bfseries,breakable}
\newtcolorbox{warning}[1][Warning]{colback=orange!5!white,colframe=orange!75!black,title={#1},fonttitle=\bfseries,breakable}
\newtcolorbox{caution}[1][Caution]{colback=yellow!5!white,colframe=yellow!75!black,title={#1},fonttitle=\bfseries,breakable}
\newtcolorbox{highlight}[1][Highlight]{colback=yellow!10!white,colframe=yellow!75!black,title={#1},fonttitle=\bfseries,breakable}
\newtcolorbox{critical}[1][Critical]{colback=red!10!white,colframe=red!75!black,title={#1},fonttitle=\bfseries,breakable}

\newtcolorbox{analysis}[1][Analysis]{colback=blue!5!white,colframe=blue!75!black,title={#1},fonttitle=\bfseries,breakable}
\newtcolorbox{application}[1][Application]{colback=green!5!white,colframe=green!75!black,title={#1},fonttitle=\bfseries,breakable}
\newtcolorbox{experiment}[1][Experiment]{colback=cyan!5!white,colframe=cyan!75!black,title={#1},fonttitle=\bfseries,breakable}
\newtcolorbox{historical}[1][Historical]{colback=brown!5!white,colframe=brown!75!black,title={#1},fonttitle=\bfseries,breakable}
\newtcolorbox{numerical}[1][Numerical]{colback=gray!5!white,colframe=gray!75!black,title={#1},fonttitle=\bfseries,breakable}
\newtcolorbox{overview}[1][Overview]{colback=blue!5!white,colframe=blue!75!black,title={#1},fonttitle=\bfseries,breakable}
\newtcolorbox{speculation}[1][Speculation]{colback=purple!5!white,colframe=purple!75!black,title={#1},fonttitle=\bfseries,breakable}
\newtcolorbox{question}[1][Question]{colback=orange!5!white,colframe=orange!75!black,title={#1},fonttitle=\bfseries,breakable}
\newtcolorbox{method}[1][Method]{colback=teal!5!white,colframe=teal!75!black,title={#1},fonttitle=\bfseries,breakable}
\newtcolorbox{correct}[1][Correct]{colback=green!10!white,colframe=green!75!black,title={#1},fonttitle=\bfseries,breakable}
\newtcolorbox{units}[1][Units]{colback=gray!5!white,colframe=gray!75!black,title={#1},fonttitle=\bfseries,breakable}
\newtcolorbox{achievement}[1][Achievement]{colback=gold!5!white,colframe=orange!75!black,title={#1},fonttitle=\bfseries,breakable}
\newtcolorbox{equivalence}[1][Equivalence]{colback=cyan!5!white,colframe=cyan!75!black,title={#1},fonttitle=\bfseries,breakable}
\newtcolorbox{dimensional}[1][Dimensional Analysis]{colback=purple!5!white,colframe=purple!75!black,title={#1},fonttitle=\bfseries,breakable}

% === ADDITIONAL SIMPLE ENVIRONMENTS ===
\newenvironment{treatise}{\begin{quote}}{\end{quote}}
\newenvironment{gemeinsam}{\begin{quote}}{\end{quote}}
\newenvironment{vergleich}{\begin{quote}}{\end{quote}}
\newenvironment{vorteil}{\begin{quote}}{\end{quote}}
\newenvironment{common}{\begin{quote}}{\end{quote}}
\newenvironment{comparison}{\begin{quote}}{\end{quote}}
\newenvironment{advantage}{\begin{quote}}{\end{quote}}
\newenvironment{quantum}{\begin{quote}}{\end{quote}}

% === LAYOUT SETTINGS ===
\raggedbottom
\usepackage{environ}
\let\oldtabular\tabular
\let\endoldtabular\endtabular

\newenvironment{scaledtable}[1][0.85]{%
	\begingroup\footnotesize\setlength{\LTleft}{0pt}\setlength{\LTright}{0pt}%
}{%
	\endgroup%
}

\newcommand{\widetable}[1]{\resizebox{\textwidth}{!}{#1}}

% === TABLE OF CONTENTS FORMATTING ===
\renewcommand{\cftsecfont}{\color{blue}}
\renewcommand{\cftsubsecfont}{\color{blue}}
\renewcommand{\cftsecpagefont}{\color{blue}}
\renewcommand{\cftsubsecpagefont}{\color{blue}}
\renewcommand{\cfttoctitlefont}{\huge\bfseries\color{blue}}

% === DEFAULT HEADER AND FOOTER ===
\pagestyle{fancy}
\fancyhf{}
\fancyhead[L]{\textsc{T0 Theory}}
\fancyhead[R]{\textsc{J. Pascher}}
\fancyfoot[C]{\thepage}

% ==============================================================================
% End of Shared Preamble for English
% ==============================================================================
%   \begin{document}
%   ...
%   \end{document}
%
% ==============================================================================

% =============================================================================
% SECTION 1: Encoding and Language
% =============================================================================
\usepackage[utf8]{inputenc}
\usepackage[T1]{fontenc}
\usepackage[ngerman]{babel}
\usepackage{lmodern}

% =============================================================================
% SECTION 2: Page Geometry
% =============================================================================
\usepackage[a4paper, left=2.5cm, right=2.5cm, top=2.5cm, bottom=3.5cm]{geometry}
\setlength{\headheight}{15pt}

% =============================================================================
% SECTION 3: Mathematics and Physics
% =============================================================================
\usepackage{amsmath,amssymb,amsfonts,amsthm}
\usepackage{mathtools}
\usepackage{physics}
\usepackage{siunitx}
\sisetup{
    locale=US,
    group-separator={,},
    output-decimal-marker={.},
    per-mode=symbol
}

% =============================================================================
% SECTION 4: Graphics and Tables
% =============================================================================
\usepackage{graphicx}
\usepackage[table,xcdraw]{xcolor}
\usepackage{tikz}
\usetikzlibrary{arrows.meta,positioning,shapes.geometric,decorations.pathmorphing,patterns,shapes.arrows,intersections}
\usepackage{pgfplots}
\pgfplotsset{compat=1.18}
\usepackage[most]{tcolorbox}
\tcbuselibrary{breakable}
\usepackage{booktabs}
\usepackage{array}
\usepackage{longtable}
\usepackage{float}
\usepackage{adjustbox}
\usepackage{rotating}
\usepackage{tabularx}
\usepackage{makecell}
\usepackage{multirow}

% =============================================================================
% SECTION 5: Document Formatting
% =============================================================================
\usepackage{fancyhdr}
\renewcommand{\headrulewidth}{0.4pt}
\renewcommand{\footrulewidth}{0.4pt}
\usepackage{tocloft}
\usepackage{hyperref}
\hypersetup{
  colorlinks=true,
  linkcolor=black,
  citecolor=black,
  urlcolor=black,
  breaklinks=true,
  bookmarksnumbered=true,
  unicode=true
}
\usepackage{bookmark}
\usepackage{cleveref}

% Table of contents: only show chapters (not sections/subsections)
\setcounter{tocdepth}{3}  % Show sections, subsections, and subsubsections
\usepackage{microtype}
\usepackage{enumitem}
\usepackage{setspace}
\usepackage{ragged2e}
\usepackage{multicol}

% =============================================================================
% SECTION 6: Code and Algorithms
% =============================================================================
\usepackage{algorithm}
\usepackage{algorithmic}
\usepackage{listings}
\lstset{
  basicstyle=\ttfamily\footnotesize,
  breaklines=true,
  breakatwhitespace=true,
  columns=flexible,
  keepspaces=true,
  showstringspaces=false,
  frame=single,
  xleftmargin=0pt,
  xrightmargin=0pt
}
\usepackage{mdframed}

% =============================================================================
% SECTION 7: Additional Packages
% =============================================================================
\usepackage{pdflscape}
\usepackage{braket}
\usepackage{cancel}
\usepackage{caption}
\usepackage{csquotes}
\usepackage{gensymb}
\usepackage{hyphenat}
\usepackage{textcomp}
\usepackage{textgreek}
\usepackage{upgreek}
\usepackage{url}
\usepackage{slashed}
\usepackage{bm}
\usepackage{newunicodechar}

% =============================================================================
% SECTION 8: Citation Commands (Compatibility)
% =============================================================================
\providecommand{\citep}[1]{\cite{#1}}
\providecommand{\citet}[1]{\cite{#1}}

% =============================================================================
% SECTION 9: Colors
% =============================================================================
\definecolor{gold}{RGB}{255,215,0}
\definecolor{blue}{rgb}{0,0,1}
\definecolor{boxgray}{RGB}{240,240,240}
\definecolor{deepblue}{RGB}{0,0,127}
\definecolor{deepgreen}{RGB}{0,127,0}
\definecolor{deepred}{RGB}{191,0,0}
\definecolor{t0blue}{RGB}{33,150,243}
\definecolor{t0green}{RGB}{76,175,80}
\definecolor{t0orange}{RGB}{255,152,0}
\definecolor{t0purple}{RGB}{156,39,176}
\definecolor{t0red}{RGB}{244,67,54}
\definecolor{t0yellow}{RGB}{255,204,0}

% =============================================================================
% SECTION 10: Column Types
% =============================================================================
\newcolumntype{L}[1]{>{\raggedright\arraybackslash}p{#1}}
\newcolumntype{C}[1]{>{\centering\arraybackslash}p{#1}}

% =============================================================================
% SECTION 11: Unicode Character Mappings
% =============================================================================
\newunicodechar{ħ}{$\hbar$}
\newunicodechar{↔}{$\leftrightarrow$}
\newunicodechar{⇐}{$\Leftarrow$}
\newunicodechar{⇒}{$\Rightarrow$}
\newunicodechar{⇔}{$\Leftrightarrow$}
\newunicodechar{∂}{$\partial$}
\newunicodechar{∅}{$\emptyset$}
\newunicodechar{∇}{$\nabla$}
\newunicodechar{∈}{$\in$}
\newunicodechar{∉}{$\notin$}
\newunicodechar{∏}{$\prod$}
\newunicodechar{∑}{$\sum$}
% Note: √ is mapped to an empty sqrt; use \sqrt{x} for proper usage
\newunicodechar{√}{\ensuremath{\sqrt{}}}
\newunicodechar{∝}{$\propto$}
\newunicodechar{∞}{$\infty$}
\newunicodechar{∩}{$\cap$}
\newunicodechar{∪}{$\cup$}
\newunicodechar{∫}{$\int$}
\newunicodechar{≈}{$\approx$}
\newunicodechar{≠}{$\neq$}
\newunicodechar{≤}{$\leq$}
\newunicodechar{≥}{$\geq$}
\newunicodechar{ξ}{\ensuremath{\xi}}
\newunicodechar{μ}{\ensuremath{\mu}}
\newunicodechar{ψ}{\ensuremath{\psi}}
\newunicodechar{φ}{\ensuremath{\phi}}
\newunicodechar{π}{\ensuremath{\pi}}
\newunicodechar{λ}{\ensuremath{\lambda}}
\newunicodechar{Δ}{\ensuremath{\Delta}}

% =============================================================================
% SECTION 12: Hyperref Settings
% =============================================================================
\hypersetup{
    colorlinks=true,
    linkcolor=blue,
    citecolor=blue,
    urlcolor=blue,
    breaklinks=true,
    bookmarksnumbered=true,
    pdfstartview=FitH
}

% =============================================================================
% SECTION 13: Theorem Environments (English)
% =============================================================================
\theoremstyle{plain}
\newtheorem{theorem}{Theorem}[section]
\newtheorem{lemma}[theorem]{Lemma}
\newtheorem{proposition}[theorem]{Proposition}
\newtheorem{corollary}[theorem]{Corollary}

\theoremstyle{definition}
\newtheorem{definition}[theorem]{Definition}
\newtheorem{example}[theorem]{Example}
\newtheorem{insight}[theorem]{Insight}
\newtheorem{discovery}[theorem]{Discovery}
% \newtheorem{erkenntnis}[theorem]{Insight}  % Commented out - conflicts with tcolorbox environment below

\theoremstyle{remark}
\newtheorem{remark}[theorem]{Remark}
\newtheorem{axiom}{Axiom}
\newtheorem{principle}{Principle}
\newtheorem{bemerkung}[theorem]{Remark}
\newtheorem{warnung}[theorem]{Warning}

% =============================================================================
% SECTION 14: T0-Specific Commands
% =============================================================================

% --- Core T0 Fields ---
\newcommand{\Tfield}{T(x,t)}
\providecommand{\Tfieldt}{T(\vec{x},t)}
\newcommand{\Efield}{E(x,t)}
\newcommand{\mfield}{m(x,t)}
\providecommand{\vecx}{\vec{x}}

% --- Lagrangian ---
\newcommand{\Lag}{\mathcal{L}}
\newcommand{\calL}{\mathcal{L}}

% --- Greek Letters and Constants ---
\newcommand{\alphaem}{\alpha}
\newcommand{\betaT}{\beta_T}
\newcommand{\xiT}{\xi}
\newcommand{\xipar}{\xi}

% --- Energy and Planck Units ---
\newcommand{\Ezero}{E_0}
\newcommand{\EPlanck}{E_{\text{Pl}}}
\newcommand{\Mpl}{M_{\text{Pl}}}
\newcommand{\mP}{m_{\text{P}}}
\newcommand{\lP}{\ell_{\text{P}}}
\newcommand{\tP}{t_{\text{P}}}
\newcommand{\LPlanck}{\ell_{\text{Pl}}}
\newcommand{\TPlanck}{t_{\text{Pl}}}

% --- Coupling Constants ---
\newcommand{\Gnat}{G_{\text{nat}}}
\newcommand{\alphaEM}{\alpha_{\text{EM}}}
\newcommand{\alphaSI}{\alpha_{\text{SI}}}
\newcommand{\Hubble}{H_0}
\newcommand{\LCDM}{\Lambda\text{CDM}}
\newcommand{\natunits}{(nat. units)}

% --- T0 Model Parameters ---
\newcommand{\xigeom}{\xi_{\mathrm{geom}}}
\newcommand{\rzero}{r_{0}}
\newcommand{\xirat}{\xi_{\mathrm{rat}}}
\newcommand{\tzero}{t_{0}}
\newcommand{\Lambdat}{\Lambda_{\mathrm{t}}}
\newcommand{\EP}{E_{\mathrm{P}}}
\newcommand{\Emu}{E_{\mu}}
\newcommand{\Ee}{E_{e}}
\newcommand{\Etau}{E_{\tau}}
\newcommand{\alphafine}{\alpha_{\mathrm{fine}}}
\newcommand{\alphal}{\alpha_{\ell}}
\newcommand{\Lzero}{\ell_{0}}
\newcommand{\Lp}{\ell_{\mathrm{P}}}

% --- Additional T0 Commands ---
\newcommand{\Kfrak}{K_{\text{frak}}}
\newcommand{\Dfrak}{D_{\text{frak}}}
\newcommand{\betapar}{\beta_T}
\newcommand{\alphapar}{\alpha}
\newcommand{\deltafield}{\delta \phi}
\newcommand{\deltam}{\delta m}
\newcommand{\deltaE}{\delta E}
\newcommand{\Exi}{E_{\xi}}
\newcommand{\Lxi}{\ell_{\xi}}
\newcommand{\rhoCMB}{\rho_{\text{CMB}}}
\newcommand{\rhoCasimir}{\rho_{\text{Casimir}}}
\newcommand{\Leff}{L_{\text{eff}}}
\newcommand{\CQCD}{C_{\mathrm{QCD}}}
\newcommand{\Kspec}{K_{\mathrm{spec}}}
\newcommand{\Tzero}{\ensuremath{T_0}}
\newcommand{\Eabs}{E_{\text{abs}}}
\newcommand{\taupar}{\tau}

% --- Provided Commands (may be redefined elsewhere) ---
\providecommand{\xiconst}{\xi_{\text{const}}}
\providecommand{\DhiggsT}{D_{\text{Higgs-T}}}
\providecommand{\rhoE}{\rho_{E}}
\providecommand{\Echar}{E_{\text{char}}}
\providecommand{\kfrac}{k_{\text{frac}}}
\providecommand{\alphaEMSI}{\alpha_{\text{EM,SI}}}
\providecommand{\alphaEMnat}{\alpha_{\text{EM,nat}}}
\providecommand{\betaTSI}{\beta_{T,\text{SI}}}
\providecommand{\betaTnat}{\beta_{T,\text{nat}}}
\providecommand{\Gsi}{G_{\text{SI}}}
\providecommand{\xiparSI}{\xi_{\text{SI}}}
\providecommand{\xiparnat}{\xi_{\text{nat}}}
\providecommand{\meff}{m_{\text{eff}}}
\providecommand{\Tzerot}{T_{0}(t)}
\providecommand{\mzerot}{m_{0}(t)}
\providecommand{\Ezeroabs}{E_{0,\text{abs}}}
\providecommand{\Epar}{E_{\text{par}}}
\providecommand{\Lnat}{\ell_{\text{nat}}}
\providecommand{\Tnat}{T_{\text{nat}}}
\providecommand{\xifrak}{\xi_{\text{frac}}}
\providecommand{\Tfrak}{T_{\text{frac}}}
\providecommand{\mfrak}{m_{\text{frac}}}
\providecommand{\Dfrac}{D_{\text{frac}}}
\providecommand{\EphotSI}{E_{\gamma,\text{SI}}}
\providecommand{\EphotNat}{E_{\gamma,\text{nat}}}
\providecommand{\Eabsint}{E_{\text{abs,int}}}
\providecommand{\mphoton}{m_{\gamma}}
\providecommand{\Evis}{E_{\text{vis}}}
\providecommand{\Cto}{C_{T0}}
\providecommand{\mytimes}{\times}
\providecommand{\lambdah}{\lambda_h}
\providecommand{\checkmarkx}{\checkmark}
\providecommand{\Enorm}{E_{\text{norm}}}
\providecommand{\Tobs}{T_{\text{obs}}}
\providecommand{\mobs}{m_{\text{obs}}}
\providecommand{\Eobs}{E_{\text{obs}}}
\providecommand{\Lobs}{\ell_{\text{obs}}}
\providecommand{\xobs}{\xi_{\text{obs}}}
\providecommand{\calE}{\mathcal{E}}
\providecommand{\calT}{\mathcal{T}}
\providecommand{\calM}{\mathcal{M}}
\providecommand{\alphag}{\alpha_g}
\providecommand{\Tmax}{T_{\text{max}}}
\providecommand{\mmin}{m_{\text{min}}}
\providecommand{\Lmax}{\ell_{\text{max}}}
\providecommand{\Emin}{E_{\text{min}}}
\providecommand{\Geff}{G_{\text{eff}}}
\providecommand{\rhoeff}{\rho_{\text{eff}}}
\providecommand{\xieff}{\xi_{\text{eff}}}
\providecommand{\Teff}{T_{\text{eff}}}
\providecommand{\hPlanck}{h}
\providecommand{\kB}{k_B}
\providecommand{\muB}{\mu_B}
\providecommand{\lambdaC}{\lambda_C}
\providecommand{\omegaP}{\omega_P}
\providecommand{\rhoP}{\rho_P}
\providecommand{\Tref}{T_{\text{ref}}}
\providecommand{\Eref}{E_{\text{ref}}}
\providecommand{\mref}{m_{\text{ref}}}
\providecommand{\Lref}{\ell_{\text{ref}}}
\providecommand{\xikonst}{\xi_0}
\providecommand{\Phiphoton}{\Phi_{\gamma}}
\providecommand{\etavis}{\eta_{\text{vis}}}
\providecommand{\pichar}{\pi}
\providecommand{\primrel}{\mathcal{P}_{\text{rel}}}
\providecommand{\warningx}{\textcolor{orange}{\textbf{!}}}
\providecommand{\phiT}{\phi_T}
\providecommand{\Lorentz}{\Lambda}
\providecommand{\Cconv}{C_{\text{conv}}}
\providecommand{\Df}{\Delta f}
\providecommand{\lambdazero}{\lambda_0}
\providecommand{\myapprox}{\approx}
\providecommand{\checked}{\checkmark}
\providecommand{\alphaWSI}{\alpha_W^{\text{SI}}}
\providecommand{\alphaWnat}{\alpha_W^{\text{nat}}}
\providecommand{\vect}[1]{\vec{#1}}
\providecommand{\Rzero}{R_0}
\providecommand{\Riem}{\mathcal{R}}
\providecommand{\nuzero}{\nu_0}
\providecommand{\mypi}{\pi}

% =============================================================================
% SECTION 15: tcolorbox Styles and Environments
% =============================================================================

% --- Predefined Styles ---
\tcbset{
    keyresult/.style={
        colback=blue!5!white,
        colframe=blue!75!black,
        title=Key Result,
        fonttitle=\bfseries
    },
    foundation/.style={
        colback=green!5!white,
        colframe=green!75!black,
        title=Foundation,
        fonttitle=\bfseries
    },
    alternative/.style={
        colback=orange!5!white,
        colframe=orange!75!black,
        title=Alternative,
        fonttitle=\bfseries
    },
    warningbox/.style={
        colback=red!5!white,
        colframe=red!75!black,
        title=Warning,
        fonttitle=\bfseries
    }
}

% --- Core Environments ---
\newtcolorbox{keyresultbox}[1][]{colback=blue!5!white,colframe=blue!75!black,fonttitle=\bfseries,title={#1},breakable}
\newtcolorbox{keyresult}[1][Key Result]{colback=blue!5!white,colframe=blue!75!black,fonttitle=\bfseries,title={#1},breakable}
\newtcolorbox{foundationbox}[1][]{colback=green!5!white,colframe=green!75!black,fonttitle=\bfseries,title={#1},breakable}
\newtcolorbox{foundation}[1][Foundation]{colback=green!5!white,colframe=green!75!black,fonttitle=\bfseries,title={#1},breakable}
\newtcolorbox{alternativebox}[1][]{colback=orange!5!white,colframe=orange!75!black,fonttitle=\bfseries,title={#1},breakable}
\newtcolorbox{warningboxenv}[1][]{colback=red!5!white,colframe=red!75!black,fonttitle=\bfseries,title={#1},breakable}

% --- Formula Environments ---
\newtcolorbox{fundamental}[1][]{
    colback=boxgray,
    colframe=t0blue,
    fonttitle=\bfseries,
    title=#1,
    sharp corners,
    boxrule=2pt
}

\newtcolorbox{newperspective}[1][]{
    colback=red!5!white,
    colframe=t0red,
    fonttitle=\bfseries,
    title=#1,
    sharp corners,
    boxrule=2pt
}

\newtcolorbox{formula}[1][]{
    colback=blue!5!white,
    colframe=blue!75!black,
    fonttitle=\bfseries,
    title=#1
}

\newtcolorbox{result}[1][]{
    colback=green!5!white,
    colframe=green!75!black,
    fonttitle=\bfseries,
    title=#1
}

\newtcolorbox{derivation}[1][]{
    colback=green!5!white,
    colframe=green!75!black,
    title=#1,
    fonttitle=\bfseries,
    breakable
}

\newtcolorbox{summary}[1][]{
    colback=gray!10!white,
    colframe=gray!75!black,
    title=#1,
    fonttitle=\bfseries,
    breakable
}

\newtcolorbox{comparison}[1][]{
    colback=purple!5!white,
    colframe=purple!75!black,
    title=#1,
    fonttitle=\bfseries,
    breakable
}

\newtcolorbox{relation}[1][]{
    colback=cyan!5!white,
    colframe=cyan!75!black,
    title=#1,
    fonttitle=\bfseries,
    breakable
}

\newtcolorbox{principleBox}[1][]{
    colback=yellow!5!white,
    colframe=yellow!75!black,
    title=#1,
    fonttitle=\bfseries,
    breakable
}

% --- Insight and Discovery Environments ---
\newtcolorbox{insightBox}[1][]{colback=blue!5,colframe=t0blue,title={#1},fonttitle=\bfseries,breakable}
\newtcolorbox{discoveryBox}[1][]{colback=green!5,colframe=t0green,title={#1},fonttitle=\bfseries,breakable}
\newtcolorbox{revelation}[1][]{colback=red!5,colframe=t0red,title={#1},fonttitle=\bfseries,breakable}
\newtcolorbox{keypoint}[1][]{colback=blue!5,colframe=t0blue,title={#1},fonttitle=\bfseries,breakable}
\newtcolorbox{evidence}[1][]{colback=green!5,colframe=t0green,title={#1},fonttitle=\bfseries,breakable}
\newtcolorbox{conclusionBox}[1][]{colback=gray!5,colframe=gray,title={#1},fonttitle=\bfseries,breakable}
\newtcolorbox{significance}[1][]{colback=yellow!5,colframe=orange,title={#1},fonttitle=\bfseries,breakable}
\newtcolorbox{philosophical}[1][]{colback=purple!5,colframe=purple,title={#1},fonttitle=\bfseries,breakable}
\newtcolorbox{implicationBox}[1][]{colback=cyan!5,colframe=cyan,title={#1},fonttitle=\bfseries,breakable}
\newtcolorbox{perspectiveBox}[1][]{colback=blue!5,colframe=t0blue,title={#1},fonttitle=\bfseries,breakable}
\newtcolorbox{revolutionary}[1][]{colback=red!5,colframe=t0red,title={#1},fonttitle=\bfseries,breakable}

% --- Technical Environments ---
\newtcolorbox{technical}[1][]{colback=gray!5,colframe=gray!75!black,title={#1},fonttitle=\bfseries,breakable}
\newtcolorbox{technicalBox}[1][]{colback=gray!5,colframe=gray!75!black,title={#1},fonttitle=\bfseries,breakable}
\newtcolorbox{notationBox}[1][]{colback=yellow!5,colframe=yellow!75!black,title={#1},fonttitle=\bfseries,breakable}
\newtcolorbox{verification}[1][]{colback=orange!5!white,colframe=orange!75!black,fonttitle=\bfseries,title=#1}
\newtcolorbox{explanationBox}[1][]{colback=purple!5!white,colframe=purple!75!black,fonttitle=\bfseries,title=#1}
\newtcolorbox{interpretationBox}[1][]{colback=cyan!5!white,colframe=cyan!75!black,fonttitle=\bfseries,title=#1}
\newtcolorbox{explanation}[1][]{colback=purple!5!white,colframe=purple!75!black,fonttitle=\bfseries,title=#1,breakable}
\newtcolorbox{interpretation}[1][]{colback=cyan!5!white,colframe=cyan!75!black,fonttitle=\bfseries,title=#1,breakable}
\newtcolorbox{proof_step}[1][]{colback=gray!5!white,colframe=gray!75!black,fonttitle=\bfseries,title=#1,breakable}
\newtcolorbox{experimental}[1][]{colback=teal!5!white,colframe=teal!75!black,fonttitle=\bfseries,title=#1,breakable}

% --- Warning and Alert Environments ---
\newtcolorbox{important}[1][]{colback=red!5!white,colframe=red!75!black,title={#1},fonttitle=\bfseries,breakable}
\newtcolorbox{warning}[1][]{colback=orange!5!white,colframe=orange!75!black,title={#1},fonttitle=\bfseries,breakable}
\newtcolorbox{caution}[1][]{colback=yellow!5!white,colframe=yellow!75!black,title={#1},fonttitle=\bfseries,breakable}
\newtcolorbox{highlight}[1][]{colback=yellow!10!white,colframe=yellow!75!black,title={#1},fonttitle=\bfseries,breakable}

% --- Additional German-specific Environments for Matsas documents ---
\newtcolorbox{literatur}[1][Literatur]{colback=blue!5!white,colframe=blue!75!black,title={#1},fonttitle=\bfseries,breakable}
\newtcolorbox{zusammenfassung}[1][Zusammenfassung]{colback=green!5!white,colframe=green!75!black,title={#1},fonttitle=\bfseries,breakable}
\newtcolorbox{frage}[1][Frage]{colback=orange!5!white,colframe=orange!75!black,title={#1},fonttitle=\bfseries,breakable}
\newtcolorbox{erkenntnis}[1][Erkenntnis]{colback=purple!5!white,colframe=purple!75!black,title={#1},fonttitle=\bfseries,breakable}
\newtcolorbox{critical}[1][]{colback=red!10!white,colframe=red!75!black,title={#1},fonttitle=\bfseries,breakable}

% --- Analysis and Application Environments ---
\newtcolorbox{analysis}[1][]{colback=blue!5!white,colframe=blue!75!black,title={#1},fonttitle=\bfseries,breakable}
\newtcolorbox{application}[1][]{colback=green!5!white,colframe=green!75!black,title={#1},fonttitle=\bfseries,breakable}
\newtcolorbox{experiment}[1][]{colback=cyan!5!white,colframe=cyan!75!black,title={#1},fonttitle=\bfseries,breakable}
\newtcolorbox{historical}[1][]{colback=brown!5!white,colframe=brown!75!black,title={#1},fonttitle=\bfseries,breakable}
\newtcolorbox{numerical}[1][]{colback=gray!5!white,colframe=gray!75!black,title={#1},fonttitle=\bfseries,breakable}
\newtcolorbox{overview}[1][]{colback=blue!5!white,colframe=blue!75!black,title={#1},fonttitle=\bfseries,breakable}
\newtcolorbox{speculation}[1][]{colback=purple!5!white,colframe=purple!75!black,title={#1},fonttitle=\bfseries,breakable}
\newtcolorbox{question}[1][]{colback=orange!5!white,colframe=orange!75!black,title={#1},fonttitle=\bfseries,breakable}
\newtcolorbox{method}[1][]{colback=teal!5!white,colframe=teal!75!black,title={#1},fonttitle=\bfseries,breakable}
\newtcolorbox{correct}[1][]{colback=green!10!white,colframe=green!75!black,title={#1},fonttitle=\bfseries,breakable}
\newtcolorbox{units}[1][]{colback=gray!5!white,colframe=gray!75!black,title={#1},fonttitle=\bfseries,breakable}
\newtcolorbox{achievement}[1][]{colback=gold!5!white,colframe=orange!75!black,title={#1},fonttitle=\bfseries,breakable}
\newtcolorbox{equivalence}[1][]{colback=cyan!5!white,colframe=cyan!75!black,title={#1},fonttitle=\bfseries,breakable}
\newtcolorbox{dimensional}[1][]{colback=purple!5!white,colframe=purple!75!black,title={#1},fonttitle=\bfseries,breakable}

% --- Physics-specific Environments ---
\newtcolorbox{photon}[1][]{colback=yellow!5!white,colframe=yellow!75!black,title={#1},fonttitle=\bfseries,breakable}
\newtcolorbox{neutrino}[1][]{colback=blue!5!white,colframe=blue!75!black,title={#1},fonttitle=\bfseries,breakable}
\newtcolorbox{revolution}[1][]{colback=red!5!white,colframe=red!75!black,title={#1},fonttitle=\bfseries,breakable}
\newtcolorbox{t0box}[1][]{colback=blue!5!white,colframe=t0blue,title={#1},fonttitle=\bfseries,breakable}
\newtcolorbox{documentbox}[1][]{colback=gray!5!white,colframe=gray!75!black,title={#1},fonttitle=\bfseries,breakable}
\newtcolorbox{sibox}[1][]{colback=green!5!white,colframe=green!75!black,title={#1},fonttitle=\bfseries,breakable}
\newtcolorbox{smbox}[1][]{colback=blue!5!white,colframe=blue!75!black,title={#1},fonttitle=\bfseries,breakable}
\newtcolorbox{pvbox}[1][]{colback=purple!5!white,colframe=purple!75!black,title={#1},fonttitle=\bfseries,breakable}
\newtcolorbox{koidebox}[1][]{colback=orange!5!white,colframe=orange!75!black,title={#1},fonttitle=\bfseries,breakable}

% --- German Compatibility Environments ---
\newtcolorbox{formel}[1][]{colback=blue!5!white,colframe=blue!75!black,title={#1},fonttitle=\bfseries,breakable}
\newtcolorbox{schluessel}[1][]{colback=blue!5!white,colframe=blue!75!black,title={#1},fonttitle=\bfseries,breakable}
\newtcolorbox{wichtig}[1][]{colback=red!5!white,colframe=red!75!black,title={#1},fonttitle=\bfseries,breakable}
\newtcolorbox{vorsicht}[1][]{colback=orange!5!white,colframe=orange!75!black,title={#1},fonttitle=\bfseries,breakable}
\newtcolorbox{revolutionaer}[1][]{colback=red!5!white,colframe=red!75!black,title={#1},fonttitle=\bfseries,breakable}
\newtcolorbox{numerisch}[1][]{colback=gray!5!white,colframe=gray!75!black,title={#1},fonttitle=\bfseries,breakable}
\newtcolorbox{experimentell}[1][]{colback=cyan!5!white,colframe=cyan!75!black,title={#1},fonttitle=\bfseries,breakable}
\newtcolorbox{anwendung}[1][]{colback=green!5!white,colframe=green!75!black,title={#1},fonttitle=\bfseries,breakable}
\newtcolorbox{alternative}[1][]{colback=orange!5!white,colframe=orange!75!black,title={#1},fonttitle=\bfseries,breakable}
\newtcolorbox{beziehung}[1][]{colback=cyan!5!white,colframe=cyan!75!black,title={#1},fonttitle=\bfseries,breakable}
\newtcolorbox{folgerung}[1][]{colback=green!5!white,colframe=green!75!black,title={#1},fonttitle=\bfseries,breakable}
\newtcolorbox{abhandlung}[1][]{colback=gray!5!white,colframe=gray!75!black,title={#1},fonttitle=\bfseries,breakable}
\newtcolorbox{prinzipBox}[1][]{colback=blue!5!white,colframe=blue!75!black,title={#1},fonttitle=\bfseries,breakable}
\newtcolorbox{prinzip}[1][]{colback=blue!5!white,colframe=blue!75!black,title={#1},fonttitle=\bfseries,breakable}
\newtcolorbox{beweis}[1][]{colback=gray!5!white,colframe=gray!75!black,title={#1},fonttitle=\bfseries,breakable}
\newtcolorbox{key}[2][]{colback=blue!5!white,colframe=blue!75!black,title={#2},fonttitle=\bfseries,breakable}
\newtcolorbox{category}[1][]{colback=purple!5!white,colframe=purple!75!black,title={#1},fonttitle=\bfseries,breakable}

% =============================================================================
% SECTION 16: Additional Simple Environments
% =============================================================================
\newenvironment{treatise}{\begin{quote}}{\end{quote}}
\newenvironment{gemeinsam}{\begin{quote}}{\end{quote}}
\newenvironment{vergleich}{\begin{quote}}{\end{quote}}
\newenvironment{vorteil}{\begin{quote}}{\end{quote}}
\newenvironment{quantum}{\begin{quote}}{\end{quote}}

% =============================================================================
% SECTION 17: Layout Settings (Kindle-compatible)
% =============================================================================
\sloppy  % Allow more flexible line breaking
\hfuzz=65pt  % Suppress overfull warnings up to 65pt (Kindle compatibility)
\vfuzz=65pt  
\tolerance=9999  % High tolerance for bad line breaks
\emergencystretch=3em  % Extra stretch to avoid overfull boxes
\hbadness=10000  % Suppress underfull box warnings
\raggedbottom

% Environment for wide tables/longtables that need scaling
\newenvironment{scaledtable}[1][0.85]{%
  \begingroup\footnotesize\setlength{\LTleft}{0pt}\setlength{\LTright}{0pt}%
}{%
  \endgroup%
}

% Command for inline table scaling
\newcommand{\widetable}[1]{\resizebox{\textwidth}{!}{#1}}

% =============================================================================
% SECTION 18: Table of Contents Formatting
% =============================================================================
\renewcommand{\cftsecfont}{\color{blue}}
\renewcommand{\cftsubsecfont}{\color{blue}}
\renewcommand{\cftsecpagefont}{\color{blue}}
\renewcommand{\cftsubsecpagefont}{\color{blue}}
\renewcommand{\cfttoctitlefont}{\huge\bfseries\color{blue}}

% =============================================================================
% SECTION 19: Default Header and Footer
% =============================================================================
\pagestyle{fancy}
\fancyhf{}
\fancyhead[L]{\textsc{T0 Theory}}
\fancyhead[R]{\textsc{J. Pascher}}
\fancyfoot[C]{\thepage}

% ==============================================================================
% End of Shared Preamble
% ==============================================================================


\title{Angepasste Dynamische Vakuum-Feldtheorie (DVFT)\\ 
	Vollständig Begründet in der T0 Zeit-Masse-Dualitätstheorie}

\author{Originalkonzept: Satish B. Thorwe\\
	Vollständig Angepasst und Integriert in die T0-Theorie: J. Pascher}

\date{24. Dezember 2025}

\begin{document}
	
	\maketitle
	
	\tableofcontents
	
	\newpage
	
	\begin{t0box}[Abstract]
		Die Dynamische Vakuum-Feldtheorie (DVFT), ursprünglich von Satish B. Thorwe vorgeschlagen, wird in dieser Arbeit vollständig an die T0-Time-Mass-Duality-Theorie angepasst und darin fundamental begründet.
		
		Das Vakuum wird als fraktales komplexes Skalarfeld
		$$  
		\Phi(x,t) = \rho(x,t) \, e^{i \theta(x,t)/\xi}
		$$
		beschrieben, wobei:
		\begin{itemize}
			\item $$ \rho(x,t) $$: Vakuum-Amplitude – verantwortlich für Gravitation, Trägheit und Masse,
			\item $$ \theta(x,t) $$: Vakuum-Phase – verantwortlich für Zeitfortschritt, Quantenkohärenz und Nichtlokalität,
			\item $$ \xi = \frac{4}{3} \times 10^{-4} $$: Einziger fraktaler Skalenparameter.
		\end{itemize}
		
		Alle Konzepte der DVFT (komplexes Vakuumfeld, Polarform, Stiffness $$ K_0 $$ und $$ B $$, emergente Gravitation aus Deformationen von $$ \rho $$) werden nun parameterfrei aus der fraktalen Selbstähnlichkeit und der Time-Mass-Duality von T0 abgeleitet. DVFT bleibt als phänomenologische Beschreibung erhalten und wird vollständig in T0 integriert.
		
		Diese Version berücksichtigt alle relevanten Ableitungen aus den T0-Kapiteln 01–43.
	\end{t0box}
	
	\section{Einführung}
	
	Die moderne Physik beruht auf zwei außerordentlich erfolgreichen, aber konzeptionell inkompatiblen Rahmenwerken:
	Allgemeine Relativitätstheorie, die Gravitation als Raumzeitgeometrie beschreibt, und Quantenfeldtheorie, die Materie und Kräfte als Anregungen abstrakter Felder beschreibt, die auf dieser Geometrie definiert sind.
	
	Die Allgemeine Relativitätstheorie (ART) beschreibt Gravitation als Krümmung der Raumzeit.
	Allerdings schweigt ART über die physische Natur der Raumzeit selbst.
	Was ist das Substrat, das sich krümmt?
	Wie legt Materie Krümmung auf Distanz auf?
	Warum propagieren gravitationelle Einflüsse mit Lichtgeschwindigkeit?
	Die Quantenmechanik (QM)
	bietet ein Bild des Vakuums als dynamisches, fluktuierendes Medium, gefüllt mit Feldern und virtuellen Anregungen.
	Doch QM identifiziert keinen Mechanismus, der Vakuumverhalten mit makroskopischer Krümmung verknüpft.
	
	Trotz ihres empirischen Erfolgs haben sowohl ART als auch QM zu tiefgreifenden ungelösten Problemen geführt, einschließlich
	des Fehlens einer konsistenten Theorie der Quantengravitation, des Bedarfs an dunkler Materie und dunkler Energie, des Ursprungs
	von Masse und Kopplungshierarchien sowie des Fehlens einer physischen Erklärung für Quantenmessung und
	klassische Emergenz.
	
	In den vergangenen Jahrzehnten haben Versuche, diese Probleme zu lösen, weitgehend durch Einführung neuer mathematischer Strukturen, extra Dimensionen, Supersymmetrie, exotischer Partikel oder modifizierter Geometrien verfolgt.
	Während mathematisch reichhaltig, beruhen viele dieser Ansätze auf Entitäten, die nicht beobachtet wurden, und verschieben oft eher als eliminieren grundlegende Ambiguïten.
	Insbesondere wird Raumzeit selbst als primäres Objekt behandelt, obwohl sie keine direkte physische Substanz hat, und das Vakuum wird als leeres Hintergrund betrachtet statt als aktives Medium.
	
	Angepasste Dynamische Vakuum-Feldtheorie (DVFT begründet in T0) wählt einen anderen Ausgangspunkt.
	Sie leitet ab, dass das Vakuum ein reales, physisches Feld ist, das dynamische Freiheitsgrade besitzt, direkt aus T0-Zeit-Masse-Dualität $T(x,t) \cdot m(x,t) = 1$ und dem fundamentalen Parameter $\xi = \frac{4}{3} \times 10^{-4}$.
	
	Alle beobachtbaren Phänomene entstehen aus dem Verhalten dieses Feldes und seiner Interaktion mit Materie.
	
	Das fundamentale Objekt in angepasster DVFT ist ein komplexes Skalarvakuumfeld
	\[
	\Phi(x)=\rho(x)e^{i\theta(x)},
	\]
	abgeleitet aus T0s $\Delta m(x,t)$, wo $\rho(x)$ die Vakuumamplitude darstellt (inertiale Dichte $\propto m(x,t)$) und $\theta(x)$
	die Vakuumphase aus T0-Knoten-Rotationen darstellt.
	
	Physische Kräfte, Raumzeitstruktur und Quantenverhalten entstehen aus räumlichen und temporalen Variationen dieser Größen.
	
	In diesem Rahmen ist Gravitation keine geometrische Eigenschaft der Raumzeit, sondern eine Manifestation kohärenter Vakuumphasenkrümmung, abgeleitet aus T0-Massenschwankungen.
	
	Elektromagnetische Felder entstehen aus organisierten Phasengradienten, während die schwache und starke Interaktion höherordentlichen oder topologisch eingeschränkten Phasenanregungen aus T0-Knoten-Mustern entsprechen.
	
	Zeit selbst wird als Rate der Vakuumphasenentwicklung aus T0-Dualität interpretiert, und relativistische Effekte wie scheinbare Zeitdilatation und Längenkontraktion entstehen natürlich aus Variationen in Vakuumsteifigkeit und inertialer Dichte, begrenzt durch T0-Mediator-Masse $m_T$. Zeitdilatation wird optimal als lokale Massevariation verstanden: höhere Massendichte (höheres $\rho$) führt zu langsameren lokalen Zeitraten, konsistent mit der Dualität $T \cdot m = 1$.
	
	Angepasste DVFT liefert eine vereinheitlichende physische Sprache über Skalen hinweg.
	
	Auf kosmologischen Skalen erklärt sie die großskalige Kohärenz des Universums, kosmische Beschleunigung und Horizontskalen-Korrelationen ohne Inflation oder dunkle Energie über T0 infinite homogene Geometrie ($\xi_{\text{eff}} = \xi/2$) zu rufen. Das Universum ist statisch und unendlich homogen, ohne Expansion.
	
	Auf galaktischen Skalen reproduziert sie MOND-ähnliches Verhalten und die baryonische Tully–Fisher-Relation ohne dunkle Materie aus T0-Niedrigenergie-Lagrangian-Grenzen.
	
	Auf Quantenskala reframiert es Welle-Teilchen-Dualität, Verschränkung, Dekohärenz und das Messproblem als Konsequenzen von Vakuumphasen-Kohärenz und ihrem Zusammenbruch aus T0-Knoten-Dynamik.
	
	Angepasste DVFT ist nicht nur ein mathematischer Rahmen, sondern liefert auch eine physische Erklärung für das Phänomen der Quantenmechanik zur Kosmologie, begründet in T0.
	
	Der größte Vorteil der angepassten DVFT ist, dass sie keine Singularität vorhersagt aufgrund der T0-Mediator-Masse und stabiler Knoten, daher können wir zum ersten Mal das Innere des Schwarzen Lochs und den Ursprung des Universums als stabile T0-Vakuumkerne beschreiben.
	
	Angepasste DVFT zeigt, dass alle majoren physischen Phänomene aus dem Verhalten eines dynamischen Vakuumfeldes abgeleitet aus T0 entstehen.
	
	Gravitation ist Vakuumkonvergenz.
	Quantenmechanik ist Vakuumkohärenz.
	Masse ist Vakuumenergie.
	Schwarze Löcher sind Vakuumkerne (stabile T0-Knoten).
	Das Universum evolviert durch dynamisches Vakuumfeld aus T0-Dualität, ohne globale Expansion.
	
	Angepasste DVFT bietet eine vereinheitlichte Vision der Natur, begründet in T0 physischem Verhalten statt abstrakter mathematischer Postulate.
	
	Es liefert auch eine tiefere, mikrophysische Erklärung von Zeit, Licht, Gravitation, elektromagnetischer Kraft, schwacher und starker Kernkraft, die sie unter einer dynamischen Vakuumfeld-basierten Ontologie abgeleitet aus T0 vereinigt.
	
	Weitere beobachtende Arbeit wird benötigt, um angepasste DVFT-Vorhersagen auf Quanten- und kosmologischer Skala zu testen, um ihre Robustheit zu beweisen, um einen Weg für die Große Vereinheitlichte Theorie als die phänomenologische Schicht der abschließenden T0-Theorie zu definieren.
\input{Kapitel_00a_De_clean}
\input{Kapitel_1a_De_clean}
\input{Kapitel_2a_De_clean}
\input{Kapitel_3a_De_clean}
\input{Kapitel_4a_De_clean}
\input{Kapitel_5a_De_clean}
\input{Kapitel_6a_De_clean}
\input{Kapitel_7a_De_clean}
\input{Kapitel_8a_De_clean}
\input{Kapitel_9a_De_clean}
\input{Kapitel_10a_De_clean}
\input{Kapitel_11a_De_clean}
\input{Kapitel_12a_De_clean}
\input{Kapitel_13a_De_clean}
\input{Kapitel_14a_De_clean}
\input{Kapitel_15a_De_clean}

\input{Kapitel_16a_De_clean.tex}
\input{Kapitel_17a_De_clean.tex}
\input{Kapitel_18a_De_clean.tex}
\input{Kapitel_19a_De_clean.tex}
\input{Kapitel_20a_De_clean.tex}
\input{Kapitel_21a_De_clean.tex}
\input{Kapitel_22a_De_clean.tex}
\input{Kapitel_23a_De_clean.tex}
\input{Kapitel_24a_De_clean.tex}
\input{Kapitel_25a_De_clean.tex}
\input{Kapitel_26a_De_clean.tex}
\input{Kapitel_27a_De_clean.tex}
\input{Kapitel_28a_De_clean.tex}
\input{Kapitel_29a_De_clean}
\input{Kapitel_30a_De_clean}
\input{Kapitel_31a_De_clean}
\input{Kapitel_32a_De_clean}
\input{Kapitel_33a_De_clean}
\input{Kapitel_34a_De_clean}
\input{Kapitel_35a_De_clean}
\input{Kapitel_36a_De_clean}
\input{Kapitel_37a_De_clean}
\input{Kapitel_38a_De_clean}
\input{Kapitel_39a_De_clean}
\input{Kapitel_40a_De_clean}
\input{Kapitel_41a_De_clean}
\input{Kapitel_42a_De_clean}
\input{Kapitel_43a_De_clean}

\begin{thebibliography}{99}
	
	\bibitem{Einstein1915}
	Einstein, A. (1915). Die Feldgleichungen der Gravitation. Sitzungsberichte der Preussischen Akademie der Wissenschaften, 844–847.
	
	\bibitem{Hilbert1915}
	Hilbert, D. (1915). Die Grundlagen der Physik. Nachrichten von der Gesellschaft der Wissenschaften zu Göttingen, Mathematisch-Physikalische Klasse, 395–407.
	
	\bibitem{Schwarzschild1916}
	Schwarzschild, K. (1916). Über das Gravitationsfeld eines Massenpunktes nach der Einsteinschen Theorie. Sitzungsberichte der Preussischen Akademie der Wissenschaften, 189–196.
	
	\bibitem{Kerr1963}
	Kerr, R. P. (1963). Gravitational Field of a Spinning Mass as an Example of Algebraically Special Metrics. Physical Review Letters, 11, 237–238. \url{https://doi.org/10.1103/PhysRevLett.11.237}
	
	\bibitem{Newman1965}
	Newman, E. T., Couch, E., Chinnapared, K., Exton, A., Prakash, A., \& Torrence, R. (1965). Metric of a Rotating, Charged Mass. Journal of Mathematical Physics, 6, 918–919. \url{https://doi.org/10.1006/1.1704351}
	
	\bibitem{Penrose1965}
	Penrose, R. (1965). Gravitational Collapse and Space-Time Singularities. Physical Review Letters, 14, 57–59. \url{https://doi.org/10.1103/PhysRevLett.14.57}
	
	\bibitem{Hawking1974}
	Hawking, S. W. (1974). Black Hole Explosions? Nature, 248, 30–31. \url{https://doi.org/10.1038/248030a0}
	
	\bibitem{Hawking1975}
	Hawking, S. W. (1975). Particle Creation by Black Holes. Communications in Mathematical Physics, 43, 199–220. \url{https://doi.org/10.1007/BF02345020}
	
	\bibitem{Bekenstein1973}
	Bekenstein, J. D. (1973). Black Holes and Entropy. Physical Review D, 7, 2333–2346. \url{https://doi.org/10.1103/PhysRevD.7.2333}
	
	\bibitem{Misner1973}
	Misner, C. W., Thorne, K. S., \& Wheeler, J. A. (1973). Gravitation. W. H. Freeman.
	
	\bibitem{Bosma1978}
	Bosma, A. (1978). The distribution and kinematics of neutral hydrogen in spiral galaxies of various morphological types. PhD thesis, University of Groningen.
	
	\bibitem{Navarro1996}
	Navarro, J. F., Frenk, C. S., \& White, S. D. M. (1996). The Structure of Cold Dark Matter Halos. The Astrophysical Journal, 462, 563–575. \url{https://doi.org/10.1086/177173}
	
	\bibitem{Tully1977}
	Tully, R. B., \& Fisher, J. R. (1977). A new method of determining distances to galaxies. Astronomy \& Astrophysics, 54, 661–673.
	
	\bibitem{McGaugh2000}
	McGaugh, S. S., Schombert, J. M., Bothun, G. D., \& de Blok, W. J. G. (2000). The Baryonic Tully–Fisher Relation. The Astrophysical Journal Letters, 533, L99–L102.
	
	\bibitem{McGaugh2005}
	McGaugh, S. S. (2005). The Baryonic Tully–Fisher Relation of Galaxies with Extended Rotation Curves and the Stellar Mass of Rotating Galaxies. The Astrophysical Journal, 632, 859–871.
	
	\bibitem{Lelli2016}
	Lelli, F., McGaugh, S. S., \& Schombert, J. M. (2016). SPARC: Mass Models for 175 Disk Galaxies with Spitzer Photometry and Accurate Rotation Curves. The Astronomical Journal, 152, 157. \url{https://doi.org/10.3847/0004-6256/152/6/157}
	
	\bibitem{Milgrom1983}
	Milgrom, M. (1983). A modification of the Newtonian dynamics as a possible alternative to the hidden mass hypothesis. The Astrophysical Journal, 270, 365–370. \url{https://doi.org/10.1086/161130}
	
	\bibitem{Bekenstein2004}
	Bekenstein, J. D. (2004). Relativistic gravitation theory for the modified Newtonian dynamics paradigm. Physical Review D, 70, 083509. \url{https://doi.org/10.1103/PhysRevD.70.083509}
	
	\bibitem{Horndeski1974}
	Horndeski, G. W. (1974). Second-order scalar-tensor field equations in a four-dimensional space. International Journal of Theoretical Physics, 10, 363–384. \url{https://doi.org/10.1007/BF01807638}
	
	\bibitem{Gubitosi2012}
	Gubitosi, G., Piazza, F., \& Vernizzi, F. (2012). The Effective Field Theory of Dark Energy. arXiv:1210.0201.
	
	\bibitem{Frusciante2020}
	Frusciante, N., \& Perenon, L. (2020). Effective Field Theory of Dark Energy: a review. Physics Reports, 857, 1–63. \url{https://doi.org/10.1016/j.physrep.2020.02.004}
	
	\bibitem{Woodard2015}
	Woodard, R. P. (2015). Ostrogradsky’s theorem on Hamiltonian instability. Scholarpedia, 10(8), 32243. \url{https://doi.org/10.4249/scholarpedia.32243}
	
	\bibitem{Motohashi2015}
	Motohashi, H., \& Suyama, T. (2015). Third order equations of motion and the Ostrogradsky instability. Physical Review D, 91, 085009. \url{https://doi.org/10.1103/PhysRevD.91.085009}
	
	\bibitem{Langlois2017}
	Langlois, D. (2017). Degenerate Higher-Order Scalar-Tensor (DHOST) theories. arXiv:1707.03625.
	
	\bibitem{BenAchour2016}
	Ben Achour, J., Crisostomi, M., Koyama, K., Langlois, D., \& Noui, K. (2016). Degenerate higher order scalar-tensor theories beyond Horndeski and disformal transformations. Physical Review D, 93, 124005. \url{https://doi.org/10.1103/PhysRevD.93.124005}
	
	\bibitem{Creminelli2017}
	Creminelli, P., \& Vernizzi, F. (2017). Dark Energy after GW170817 and GRB170817A. Physical Review Letters, 119, 251302. \url{https://doi.org/10.1103/PhysRevLett.119.251302}
	
	\bibitem{Ezquiaga2017}
	Ezquiaga, J. M., \& Zumalacárregui, M. (2017). Dark Energy after GW170817: dead ends and the road ahead. Physical Review Letters, 119, 251304. \url{https://doi.org/10.1103/PhysRevLett.119.251304}
	
	\bibitem{Langlois2018}
	Langlois, D., Ezquiaga, J. M., \& Zumalacárregui, M. (2018). Scalar-tensor theories and modified gravity in the wake of GW170817. Physical Review D, 97, 061501(R). \url{https://doi.org/10.1103/PhysRevD.97.061501}
	
	\bibitem{Abbott2017GW}
	Abbott, B. P., et al. (LIGO Scientific Collaboration and Virgo Collaboration). (2017). GW170817: Observation of Gravitational Waves from a Binary Neutron Star Inspiral. Physical Review Letters, 119, 161101. \url{https://doi.org/10.1103/PhysRevLett.119.161101}
	
	\bibitem{Abbott2017MM}
	Abbott, B. P., et al. (LIGO Scientific Collaboration and Virgo Collaboration). (2017). Multi-messenger Observations of a Binary Neutron Star Merger. The Astrophysical Journal Letters, 848, L12–L16. \url{https://doi.org/10.3847/2041-8213/aa91c9}
	
	\bibitem{Abbott2019}
	Abbott, B. P., et al. (LIGO Scientific Collaboration and Virgo Collaboration). (2019). Tests of General Relativity with the Binary Black Hole Signals from the LIGO–Virgo Catalog GWTC-1. Physical Review D, 100, 104036. \url{https://doi.org/10.1103/PhysRevD.100.104036}
	
	\bibitem{Eardley1973}
	Eardley, D. M., Lee, D. L., Lightman, A. P., Wagoner, R. V., \& Will, C. M. (1973). Gravitational-wave observations as a tool for testing relativistic gravity. Physical Review Letters, 30, 884–886. \url{https://doi.org/10.1103/PhysRevLett.30.884}
	
	\bibitem{Nishizawa2009}
	Nishizawa, A., Taruya, A., Hayama, K., Kawamura, S., \& Sakagami, M. (2009). Probing non-tensorial polarizations of stochastic gravitational-wave backgrounds with ground-based laser interferometers. Physical Review D, 79, 082002. \url{https://doi.org/10.1103/PhysRevD.79.082002}
	
	\bibitem{Vainshtein1972}
	Vainshtein, A. I. (1972). To the problem of nonvanishing gravitation mass. Physics Letters B, 39(3), 393–394. \url{https://doi.org/10.1016/0370-2693(72)90147-5}
	
	\bibitem{Babichev2013}
	Babichev, E., \& Deffayet, C. (2013). An introduction to the Vainshtein mechanism. Classical and Quantum Gravity, 30(18), 184001. \url{https://doi.org/10.1088/0264-9381/30/18/184001}
	
	\bibitem{Khoury2004}
	Khoury, J., \& Weltman, A. (2004). Chameleon cosmology. Physical Review D, 69, 044026. \url{https://doi.org/10.1103/PhysRevD.69.044026}
	
	\bibitem{Burrage2018}
	Burrage, C., \& Sakstein, J. (2018). Tests of Chameleon Gravity. Living Reviews in Relativity, 21, 1. \url{https://doi.org/10.1007/s41114-018-0011-x}
	
	\bibitem{Schrodinger1926}
	Schrödinger, E. (1926). Quantisierung als Eigenwertproblem (Parts I–IV). Annalen der Physik, 79–81.
	
	\bibitem{Heisenberg1927}
	Heisenberg, W. (1927). Über den anschaulichen Inhalt der quantentheoretischen Kinematik und Mechanik. Zeitschrift für Physik, 43, 172–198. \url{https://doi.org/10.1007/BF01397280}
	
	\bibitem{Born1926}
	Born, M. (1926). Zur Quantenmechanik der Stoßvorgänge. Zeitschrift für Physik, 37, 863–867. \url{https://doi.org/10.1007/BF01397477}
	
	\bibitem{vonNeumann1932}
	von Neumann, J. (1932). Mathematische Grundlagen der Quantenmechanik. Springer (English transl.: Mathematical Foundations of Quantum Mechanics, Princeton Univ. Press, 1955).
	
	\bibitem{Sakurai2017}
	Sakurai, J. J., \& Napolitano, J. (2017). Modern Quantum Mechanics (2nd ed.). Cambridge University Press.
	
	\bibitem{Zurek2003}
	Zurek, W. H. (2003). Decoherence, einselection, and the quantum origins of the classical. Reviews of Modern Physics, 75, 715–775. \url{https://doi.org/10.1103/RevModPhys.75.715}
	
	\bibitem{Joos2003}
	Joos, E., Zeh, H. D., Kiefer, C., Giulini, D., Kupsch, J., \& Stamatescu, I.-O. (2003). Decoherence and the Appearance of a Classical World in Quantum Theory (2nd ed.). Springer. \url{https://doi.org/10.1007/978-3-662-05328-7}
	
	\bibitem{Yang1954}
	Yang, C. N., \& Mills, R. L. (1954). Conservation of isotopic spin and isotopic gauge invariance. Physical Review, 96(1), 191–195. \url{https://doi.org/10.1103/PhysRev.96.191}
	
	\bibitem{Faddeev1967}
	Faddeev, L. D., \& Popov, V. N. (1967). Feynman diagrams for the Yang–Mills field. Physics Letters B, 25(1), 29–30. \url{https://doi.org/10.1016/0370-2693(67)90067-6}
	
	\bibitem{Peskin1995}
	Peskin, M. E., \& Schroeder, D. V. (1995). An Introduction to Quantum Field Theory. Addison-Wesley.
	
	\bibitem{Weinberg1995}
	Weinberg, S. (1995). The Quantum Theory of Fields, Vol. I: Foundations. Cambridge University Press.
	
	\bibitem{Clay2000}
	Clay Mathematics Institute. (2000–present). Yang–Mills existence and mass gap (Millennium Prize Problem). \url{https://www.claymath.org/millennium/yang-mills-the-maths-gap/}
	
	\bibitem{Jaffe2000}
	Jaffe, A. (2000). Quantum Yang–Mills Theory (CMI Millennium Prize Problem description; Jaffe–Witten). Clay Mathematics Institute.
	
	\bibitem{Sakharov1967}
	Sakharov, A. D. (1967). Violation of CP invariance, C asymmetry, and baryon asymmetry of the universe. JETP Letters, 5, 24–27.
	
	\bibitem{Penrose1996}
	Penrose, R. (1996). On Gravity’s role in Quantum State Reduction. General Relativity and Gravitation, 28, 581–600. \url{https://doi.org/10.1007/BF02105068}
	
	\bibitem{Diosi1989}
	Diósi, L. (1989). Models for universal reduction of macroscopic quantum fluctuations. Physical Review A, 40, 1165–1174. \url{https://doi.org/10.1103/PhysRevA.40.1165}
	
	\bibitem{Bassi2013}
	Bassi, A., Lochan, K., Satin, S., Singh, T. P., \& Ulbricht, H. (2013). Models of wave-function collapse, underlying theories, and experimental tests. Reviews of Modern Physics, 85, 471–527. \url{https://doi.org/10.1103/RevModPhys.85.471}
	
	\bibitem{Arndt2014}
	Arndt, M., \& Hornberger, K. (2014). Testing the limits of quantum mechanical superpositions. Nature Physics, 10, 271–277. \url{https://doi.org/10.1038/nphys2863}
	
	\bibitem{Marletto2017}
	Marletto, C., \& Vedral, V. (2017). Gravitationally Induced Entanglement between Two Massive Particles is Sufficient Evidence of Quantum Effects in Gravity. Physical Review Letters, 119, 240402. \url{https://doi.org/10.1103/PhysRevLett.119.240402}
	
	\bibitem{Margalit2021}
	Margalit, Y., Dobkowski, O., Zhou, Z., et al. (2021). Realization of a complete Stern–Gerlach interferometer: Toward a test of quantum gravity. Science Advances, 7(22), eabg2879. \url{https://doi.org/10.1126/sciadv.abg2879}
	
	\bibitem{Roura2020}
	Roura, A. (2020). Gravitational Redshift in Quantum-Clock Interferometry. Physical Review X, 10, 021014. \url{https://doi.org/10.1103/PhysRevX.10.021014}
	
	\bibitem{Dobkowski2025}
	Dobkowski, O., Trok, B., Skakunenko, P., et al. (2025). Observation of the quantum equivalence principle for matter-waves. arXiv:2502.14535.
	
	\bibitem{finalposition}
	This paper positions Adapted Dynamic Vacuum Field Theory (DVFT fully grounded in T0 time-mass duality) as a transformative phenomenological approach to unifying general relativity, quantum mechanics, and cosmology by reimagining space as a dynamic vacuum field that has amplitude and phase fully derived from T0 duality and node dynamics. This intrinsic dynamic vacuum field behavior opens new theoretical and observational possibilities for understanding the universe’s structure and forces within the conclusive T0 framework.
	\bibitem{PascherT0Intro}
	Pascher, J. (2025). T0 Theory Introduction. Available at: \url{https://github.com/jpascher/T0-Time-Mass-Duality/blob/main/2/pdf/1_T0_Introduction_De.pdf}
	
	\bibitem{PascherT0Grundlagen}
	Pascher, J. (2025). T0 Theory Foundations. Available at: \url{https://github.com/jpascher/T0-Time-Mass-Duality/blob/main/2/pdf/003_T0_Grundlagen_De.pdf}
	
	\bibitem{PascherT0Lagrangian}
	Pascher, J. (2025). T0 Universal Lagrangian. Available at: \url{https://github.com/jpascher/T0-Time-Mass-Duality/blob/main/2/pdf/019_T0_lagrndian_De.pdf}
	
	\bibitem{PascherT0Dirac}
	Pascher, J. (2025). Simplified Dirac Equation in T0 Theory. Available at: \url{https://github.com/jpascher/T0-Time-Mass-Duality/blob/main/2/pdf/050_diracVereinfacht_De.pdf}
	
	\bibitem{PascherT0QM}
	Pascher, J. (2025). Deterministic Quantum Mechanics in T0. Available at: \url{https://github.com/jpascher/T0-Time-Mass-Duality/blob/main/2/pdf/QM-DetrmisticEn.pdf}
	
	\bibitem{PascherT0Cosmology}
	Pascher, J. (2025). T0 Cosmology and Dipole Analysis. Available at: \url{https://github.com/jpascher/T0-Time-Mass-Duality/blob/main/2/pdf/039_Zwei-Dipole-CMB_De.pdf}
	
	\bibitem{PascherT0Casimir}
	Pascher, J. (2025). Unification of Casimir Effect and CMB in T0. Available at: \url{https://github.com/jpascher/T0-Time-Mass-Duality/blob/main/2/pdf/091_Casimir_De.pdf}
	
	\bibitem{PascherT0ParticleMasses}
	Pascher, J. (2025). T0 Particle Masses and Hierarchies. Available at: \url{https://github.com/jpascher/T0-Time-Mass-Duality/blob/main/2/pdf/006_T0_Teilchenmassen_De.pdf}
	
	\bibitem{PascherT0Neutrinos}
	Pascher, J. (2025). T0 Neutrino Masses. Available at: \url{https://github.com/jpascher/T0-Time-Mass-Duality/blob/main/2/pdf/007_T0_Neutrinos_De.pdf}
	
	\bibitem{PascherT0g2}
	Pascher, J. (2025). Anomalous Magnetic Moments in T0. Available at: \url{https://github.com/jpascher/T0-Time-Mass-Duality/blob/main/2/pdf/017_T0_Anomale_Magnetische_Momente_De.pdf}
	
	\bibitem{finalposition}
	This paper positions Adapted Dynamic Vacuum Field Theory (DVFT fully grounded in T0 time-mass duality) as a transformative phenomenological approach to unifying general relativity, quantum mechanics, and cosmology by reimagining space as a dynamic vacuum field that has amplitude and phase fully derived from T0 duality and node dynamics. This intrinsic dynamic vacuum field behavior opens new theoretical and observational possibilities for understanding the universe’s structure and forces within the conclusive T0 framework.
	
	\bibitem{Thorwe}
	Thorwe, Satish B. – Originalkonzept der Dynamischen Vakuum-Feldtheorie (DVFT).
	
	\bibitem{T0repo}
	Pascher, J. (2025). T0-Time-Mass-Duality-Theorie: Vollständige Kapitel und Ableitungen. GitHub: \url{https://github.com/jpascher/T0-Time-Mass-Duality}.
	
	\bibitem{finalposition}
	Diese Arbeit positioniert die Angepasste Dynamische Vakuum-Feldtheorie als phänomenologische Beschreibung, die vollständig in der fundamentalen T0-Time-Mass-Duality-Theorie begründet ist, unter Anerkennung des Originalkonzepts von Satish B. Thorwe.
	
\end{thebibliography}

\end{document}