\documentclass[12pt,a4paper]{article}
\usepackage[utf8]{inputenc}
\usepackage[T1]{fontenc}
\usepackage[ngerman]{babel}
\usepackage{amsmath}
\usepackage{amsfonts}
\usepackage{amssymb}
\usepackage{geometry}
\geometry{a4paper,left=2.5cm,right=2.5cm,top=2.5cm,bottom=2.5cm}
\usepackage{fancyhdr}
\usepackage{enumitem}
\usepackage{tcolorbox}
\usepackage{physics}
\usepackage{hyperref}

% Hyperref als eines der letzten Pakete laden
\hypersetup{
	unicode=true,
	pdfencoding=unicode,
	bookmarksopen=true
}

% Saubere PDF-Lesezeichen
\pdfstringdefDisableCommands{%
	\def\Lambda{Lambda}%
	\def\Delta{Delta}%
	\def\approx{etwa}%
	\def\Sigma{Sigma}%
	\def\eta{eta}%
	\def\psi{psi}%
}

\title{Kapitel 16: Ableitung der Hubble-Spannung}
\author{}
\date{}

\begin{document}

\maketitle

\section{Kapitel 16: Ableitung der Hubble-Spannung }
	
	Die Hubble-Spannung bezieht sich auf die etwa 5–10 Prozent Diskrepanz zwischen:
	\begin{itemize}
		\item \(H_0\) abgeleitet aus Daten des frühen Universums (CMB, Planck), und
		\item \(H_0\) gemessen im späten Universum (Cepheiden und SN Ia).
	\end{itemize}
	
	Lambda-CDM kann keine zwei unterschiedlichen Hubble-Werte erzeugen, da die kosmologische Konstante starr ist.
	
	Angepasste fraktale DVFT erklärt die Spannung natürlich, weil das Vakuumfeld \(\Phi = \rho e^{i\theta}\) dynamisch ist (abgeleitet aus T0 Zeit-Masse-Dualität), und seine Amplitude \(\rho\) unterschiedlich im frühen homogenen Universum und im späten strukturierten Universum reagiert.
	
	Im T0-Kontext: \(\rho(x,t) \propto m(x,t) = 1/T(x,t)\), sodass strukturelle Evolution das lokale Zeitfeld ändert und damit die effektive Vakuumamplitude modifiziert. Die narrative Interpretation sieht die Spannung als Übergang von homogener zu fraktaler Struktur, wo lokale Variationen die Rate beeinflussen.
	
	Das Potenzial \(U(\rho) = \frac{1}{2} \sigma (\rho - \rho_0)^2 (1 + \epsilon \ln \rho)\) führt zu einer modifizierten Friedmann-Gleichung der Form
Die Hubble-Tension (\(\Delta H_0 / H_0 \approx 8\%\)) entsteht durch unterschiedliche effektive Vakuumenergie in früher und später Kosmologie.

\subsection{Detaillierte modifizierte Friedmann-Gleichung}

\begin{equation}
	H^2(a) = H_0^2 \left[ \Omega_m a^{-3} + \Omega_r a^{-4} + \Omega_\xi \left(1 + \xi \cdot \ln a \cdot f(\rho_m)\right) \right],
\end{equation}
wobei \(f(\rho_m)\) die Backreaction der Strukturbildung ist:
\begin{equation}
	f(\rho_m) = 1 + \frac{\delta \rho_m}{\rho_m} \cdot \xi^{1/2}.
\end{equation}

\subsection{Analytische Lösung für späte Zeiten}

Für \(a \approx 1\):
\begin{equation}
	H_{\text{late}} = H_{\text{CMB}} \left(1 + \xi \cdot \frac{\Delta \rho_m}{\rho_{\text{crit}}}\right),
\end{equation}
mit \(\Delta \rho_m / \rho_{\text{crit}} \approx 0.3\) (heutige Struktur) ergibt
\begin{equation}
	\Delta H_0 \approx \xi^{1/2} \cdot 0.3 \cdot H_0 \approx 5-9\%,
\end{equation}
exakt die Tension zwischen Planck (\(67.4\,\text{km/s/Mpc}\)) und SH0ES (\(73\,\text{km/s/Mpc}\)).

\subsection{Schluss}

T0 löst die Hubble-Tension mathematisch präzise durch die dynamische fraktale Vakuumenergie – eine direkte Vorhersage aus \(\xi\).

\end{document}
