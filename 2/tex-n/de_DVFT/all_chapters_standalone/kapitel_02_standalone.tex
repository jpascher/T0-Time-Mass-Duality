\documentclass[12pt,a4paper]{article}
\usepackage[utf8]{inputenc}
\usepackage[ngerman]{babel}
\usepackage{amsmath}
\usepackage{amsfonts}
\usepackage{amssymb}
\usepackage{geometry}
\geometry{a4paper,left=2.5cm,right=2.5cm,top=2.5cm,bottom=2.5cm}
\usepackage{fancyhdr}
\usepackage{enumitem}
\usepackage{hyperref}
\usepackage{booktabs}
\usepackage{float}

\begin{document}

\title{Kapitel 2: Warum die Raumzeit in T0 fraktal und dual ist}
\author{}
\date{Dezember 2025}

\maketitle

\section{Warum die Raumzeit in T0 fraktal und dual ist}

Die klassische Vorstellung einer glatten, kontinuierlichen Raumzeit führt zu Singularitäten, unendlichen Renormierungen und der Notwendigkeit multipler freier Parameter. T0 löst diese Probleme, indem sie die Raumzeit als fraktal mit Skalenparameter $\xi = (4/3)\times 10^{-4}$ und intrinsischer Dualität zwischen Zeit und Masse beschreibt.

\subsection{Notwendigkeit der fraktalen Struktur}

Eine kontinuierliche Raumzeit erzeugt:
\begin{itemize}
	\item Schwarze-Löcher-Singularitäten
	\item UV-Divergenzen in QFT
	\item Willkürliche Parameter (Yukawa-Kopplungen, $\Lambda \approx 10^{-120}$ Fehleinschätzung)
\end{itemize}
Die fraktale Skalierung mit $\xi$ bricht die Kontinuität auf Planck-skalierten Unterskalen und reguliert automatisch alle Divergenzen.

\subsection{Die intrinsische Time-Mass-Duality}

T0 zeigt, dass die Dualität nicht postuliert, sondern aus der fraktalen Selbstähnlichkeit folgt:
\begin{itemize}
	\item Jede Skalentransformation $\xi$ verbindet Zeitintervalle mit Massenskalen
	\item Die Ruhemasse eines Teilchens ist äquivalent zu einem fraktal skalierten Zeitintervall
	\item Die Stabilität der Vakuumstruktur erzwingt $\xi = 4/3 \times 10^{-4}$ als einzige konsistente Lösung
\end{itemize}

\subsection{Warum keine externen Auslöser nötig sind}

Die fraktale Hierarchie und die Dualität sind Eigenschaften der minimalen konsistenten Beschreibung der Raumzeit selbst. Es gibt keine metaphysischen „Ursachen“ – T0 ist die natürliche Konsequenz aus:
\begin{itemize}
	\item Endlichkeit der Skalen
	\item Selbstähnlichkeit
	\item Stabilität des Vakuums
\end{itemize}

\subsection{Schluss}

Die fraktale Natur mit $\xi$ und die Time-Mass-Duality sind keine Annahmen, sondern unvermeidbare Konsequenzen einer singularitätenfreien, parameterarmen Beschreibung der Physik. T0 benötigt keinen externen Auslöser – die Dualität ist in die Struktur der Raumzeit selbst eingebaut.


\end{document}
