\documentclass[12pt,a4paper]{article}
\usepackage[utf8]{inputenc}
\usepackage[T1]{fontenc}
\usepackage[ngerman]{babel}
\usepackage{amsmath}
\usepackage{amsfonts}
\usepackage{amssymb}
\usepackage{geometry}
\geometry{a4paper,left=2.5cm,right=2.5cm,top=2.5cm,bottom=2.5cm}
\usepackage{fancyhdr}
\usepackage{enumitem}
\usepackage{tcolorbox}
\usepackage{physics}
\usepackage{hyperref}

% Hyperref als eines der letzten Pakete laden
\hypersetup{
	unicode=true,
	pdfencoding=unicode,
	bookmarksopen=true
}

% Saubere PDF-Lesezeichen
\pdfstringdefDisableCommands{%
	\def\Lambda{Lambda}%
	\def\Delta{Delta}%
	\def\approx{etwa}%
	\def\Sigma{Sigma}%
	\def\eta{eta}%
	\def\psi{psi}%
}

\title{Kapitel 39: Entropie}
\author{}
\date{}

\begin{document}

\maketitle

\section{Kapitel 39: Entropie }
	
	Der Zweite Hauptsatz – Entropie nimmt zu – ist mysteriös in Standardphysik.
	
	Fraktale DVFT: Zeit = Vakuumphasen-Evolution \(\theta\). Phase evolviert nur vorwärts. Entropie zu durch irreversible Phasenverstreuung.
	
	Irreversibilität eingebaut in Vakuumstruktur. Entropie emergent, nicht fundamental.
	
	Erste physische Erklärung für Zweiten Hauptsatz und Zeitpfeil.
Der Zweite Hauptsatz der Thermodynamik – Entropie nimmt in isolierten Systemen nie ab – wird in der Standardphysik als statistische Tendenz oder mikroskopische Zählung interpretiert. T0 macht ihn zu einer fundamentalen, irreversiblen Konsequenz der Vakuumphasen-Evolution \(\theta(t)\).

\subsection{Zeit als Vakuumphasen-Fortschritt}

In T0 ist Properzeit \(\tau\) proportional zur akkumulierten Phase:
\begin{equation}
	d\tau = \xi \cdot d\theta.
\end{equation}

Die Phase evolviert intrinsisch vorwärts:
\begin{equation}
	\dot{\theta} = \omega_0 + \xi \cdot \nabla \theta > 0,
\end{equation}
da die fraktale Hierarchie eine Richtung erzwingt (Selbstähnlichkeit nur in eine Zeitrichtung).

\subsection{Entropie als Phasen-Disorder}

Entropie \(S\) ist Maß für Phasen-Unkohärenz:
\begin{equation}
	S = k_B \cdot \ln \Omega = k_B \cdot \langle (\Delta \theta)^2 \rangle / \xi.
\end{equation}

In kohärentem Zustand (\(\Delta \theta = 0\)): \(S = 0\).  
Dekohärenz erhöht \(\Delta \theta\):
\begin{equation}
	\frac{dS}{dt} = k_B \cdot \frac{2 \Delta \theta \dot{\theta}}{\xi} \geq 0.
\end{equation}

\subsection{Irreversibilität aus Phasen-Evolution}

Phasen-Rückwärtslauf (\(\dot{\theta} < 0\)) würde fraktale Hierarchie umkehren – energetisch verboten:
\begin{equation}
	\Delta E_{\text{reverse}} \approx B \cdot (\Delta \theta)^2 \cdot \xi^{-1} \to \infty.
\end{equation}

Daher:
\begin{equation}
	\frac{dS}{dt} \geq 0
\end{equation}
zwangsläufig.

\subsection{Messung und Kollaps}

Messung koppelt makroskopisch an \(\theta\):
\begin{equation}
	\Delta \theta_{\text{meas}} \approx \xi \cdot \sqrt{N_{\text{atoms}}} \gg \xi.
\end{equation}

Entropie-Zuwachs:
\begin{equation}
	\Delta S \approx k_B \ln N_{\text{states}} \approx k_B N_{\text{atoms}}.
\end{equation}

Kollaps ist irreversibles Phasen-Scrambling.

\subsection{Kosmologische Entropie}

Universums-Expansion dispergiert Phase:
\begin{equation}
	\Delta \theta_{\text{cosmo}} \propto \xi \cdot \ln a(t).
\end{equation}

Entropie-Wachstum erklärt Arrow of Time.

\subsection{Schluss}

T0 macht den Zweiten Hauptsatz fundamental: Zeit ist Phasen-Fortschritt, Entropie Phasen-Disorder, Irreversibilität aus gerichteter Evolution. Keine statistische Annahme – physikalische Notwendigkeit aus \(\xi\).

\end{document}
