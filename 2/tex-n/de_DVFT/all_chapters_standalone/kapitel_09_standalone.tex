\documentclass[12pt,a4paper]{article}
\usepackage[utf8]{inputenc}
\usepackage[ngerman]{babel}
\usepackage{amsmath}
\usepackage{amsfonts}
\usepackage{amssymb}
\usepackage{geometry}
\geometry{a4paper,left=2.5cm,right=2.5cm,top=2.5cm,bottom=2.5cm}
\usepackage{fancyhdr}
\usepackage{enumitem}
\usepackage{hyperref}
\usepackage{booktabs}
\usepackage{float}

\begin{document}

\title{Kapitel 9: Stark-, Schwach- und Tief-Feld-Regime in T0}
\author{}
\date{Dezember 2025}

\maketitle

\section{Stark-, Schwach- und Tief-Feld-Regime in T0}

T0 prognostiziert eine natürliche Hierarchie gravitativer Regime, bestimmt allein durch den Vergleich der lokalen Beschleunigung \(a\) mit der fundamentalen T0-Skala \(a_\xi \approx 1.2 \times 10^{-10}\,\text{m/s}^2\).

\subsection{Mathematische Definition der Regime}

Die effektive Gravitationsstärke wird durch die Funktion
\begin{equation}
	\mu\left(\frac{a}{a_\xi}\right) = \left(1 + \left(\frac{a_\xi}{a}\right)^2\right)^{1/4}
\end{equation}
beschrieben (abgeleitet aus fraktaler Metrik-Integration).

\begin{itemize}
	\item Starkfeld: \(a \gg a_\xi\) \(\Rightarrow\) \(\mu \approx 1\) \(\Rightarrow\) exakt GR,
	\item Übergang: \(a \sim a_\xi\),
	\item Tieffeld: \(a \ll a_\xi\) \(\Rightarrow\) \(\mu \approx (a/a_\xi)^{-1/2}\).
\end{itemize}

\subsection{Vergleich mit TeVeS}

TeVeS definiert Regime durch die Skalarfeld-Dynamik und die freie Funktion \(\mathcal{F}\). Die Übergangsfunktion ist komplizierter und enthält zusätzliche Parameter. T0 erreicht dasselbe Tieffeld-Verhalten mit nur einem Parameter und ohne zusätzliche Felder.

\subsection{Schluss}

Die Regime in T0 sind fundamental aus \(\xi\) abgeleitet, nicht phänomenologisch wie in TeVeS oder MOND.


\end{document}
