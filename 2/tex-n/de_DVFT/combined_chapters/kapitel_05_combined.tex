\documentclass[12pt,a4paper]{article}
\usepackage{amsmath, amssymb, amsthm}
\usepackage{geometry}
\usepackage{titlesec}
\usepackage{tcolorbox}
\usepackage{enumitem}
\usepackage{booktabs}

% Theoreme
\newtheorem{theorem}{Theorem}[section]
\newtheorem{lemma}[theorem]{Lemma}
\newtheorem{corollary}[theorem]{Korollar}
\newtheorem{definition}[theorem]{Definition}

\title{Kapitel 5: Probleme der Allgemeinen Relativitätstheorie und ihre Lösung durch T0}
\author{}
\date{}

\begin{document}

\maketitle

Die Allgemeine Relativitätstheorie (ART) ist empirisch erfolgreich, leidet jedoch unter konzeptionellen und physikalischen Defiziten.

\subsection{Singularitäten und Informationsverlust}

ART prognostiziert unvermeidbare Singularitäten. T0 verhindert diese durch fraktale Diskretisierung – die Krümmung bleibt bei der T0-Skala endlich.

\subsection{Dunkle Materie und Dunkle Energie}

ART benötigt unobserved Komponenten für Galaxiedynamik und kosmische Beschleunigung. T0 erklärt beide durch fraktale Gravitationsmodifikationen.

\subsection{Quanteninkompatibilität und Parameterproblematik}

ART ist nicht renormierbar und erfordert zusammen mit dem Standardmodell zahlreiche freie Parameter. T0 ist UV-finitt und hat genau einen Parameter \(\xi\).

\subsection{Schluss}

T0 löst alle fundamentalen Probleme der ART durch die fraktale Time-Mass-Duality und liefert eine konsistente Quantengravitation ohne zusätzliche Annahmen.

\end{document}
