\documentclass[12pt,a4paper]{article}
\usepackage[utf8]{inputenc}
\usepackage[T1]{fontenc}
\usepackage[ngerman]{babel}
\usepackage{amsmath}
\usepackage{amsfonts}
\usepackage{amssymb}
\usepackage{geometry}
\geometry{a4paper,left=2.5cm,right=2.5cm,top=2.5cm,bottom=2.5cm}
\usepackage{fancyhdr}
\usepackage{enumitem}
\usepackage{tcolorbox}
\usepackage{physics}
\usepackage{hyperref}

% Hyperref als eines der letzten Pakete laden
\hypersetup{
	unicode=true,
	pdfencoding=unicode,
	bookmarksopen=true
}

% Saubere PDF-Lesezeichen
\pdfstringdefDisableCommands{%
	\def\Lambda{Lambda}%
	\def\Delta{Delta}%
	\def\approx{etwa}%
	\def\Sigma{Sigma}%
	\def\eta{eta}%
	\def\psi{psi}%
}

\title{Kapitel 43: Fundamentale Axiome und Konstanten}
\author{}
\date{}

\begin{document}

\maketitle

\section{Kapitel 43: Fundamentale Axiome und Konstanten }
	
	Core Axiome: Vakuum physisches Medium, Feld \(\Phi\), Dualität, etc.
	
	Universum als materielles Medium mit mechanischen Konstanten aus T0.
	
	Diese Kapitel bilden eine einheitliche fraktale narrative der Physik, vereinheitlicht durch die T0-Theorie und den Parameter \(\xi\).
Die T0-Time-Mass-Duality-Theorie basiert auf minimalen, klar definierten Axiomen. Alle physikalischen Konstanten und Phänomene emergieren aus diesen Axiomen und dem einzigen Skalenparameter \(\xi = \frac{4}{3} \times 10^{-4}\).

\subsection{Kernaxiome von T0}

\textbf{Axiom 1 – Das Vakuum ist ein physikalisches Medium}  
Das Vakuum ist kein leerer Raum, sondern ein fraktales Feld \(\Phi(x) = \rho(x) e^{i \theta(x)/\xi}\) mit Amplitude \(\rho\) (Inertie, Gravitation) und Phase \(\theta\) (Zeit, Quantenkohärenz). Materie ist lokale Perturbation dieses Mediums.

\textbf{Axiom 2 – Time-Mass-Duality}  
Zeit und Masse sind duale Aspekte:
\begin{equation}
	m \leftrightarrow \Delta t \cdot \xi^{-1} \cdot \frac{\hbar}{c^2}.
\end{equation}
Ruhemassen sind stabile fraktal skalierte Zeitintervalle.

\textbf{Axiom 3 – Fraktale Selbstähnlichkeit}  
Die Vakuumstruktur ist selbstähnlich mit Skalenfaktor \(\xi\):
\begin{equation}
	\Phi(x \cdot \xi) = \Phi(x) \cdot \xi^{D_f},
\end{equation}
wobei \(D_f\) die fraktale Dimension ist.

\textbf{Axiom 4 – Minimale Kopplung}  
Materie koppelt minimal an \(\rho\) und \(\theta\), ohne zusätzliche Felder.

\textbf{Axiom 5 – Deterministische Evolution}  
Die Vakuumphase evolviert deterministisch vorwärts – probabilistische QM emergiert aus Nichtlokalität.

\subsection{Ableitung der Universalkonstanten aus \(\xi\)}

1. **Lichtgeschwindigkeit \(c\)**  
Maximale Phasen-Ausbreitung:
\begin{equation}
	c = \sqrt{\frac{B}{K_0 / \rho_0}} \cdot \xi^{1/2}.
\end{equation}

2. **Reduziertes Planck-Konstante \(\hbar\)**  
Phasen-Quantisierung:
\begin{equation}
	\hbar = B \cdot l_0^2 \cdot \xi.
\end{equation}

3. **Gravitationskonstante \(G\)**  
Amplitude-Kopplung:
\begin{equation}
	G = \frac{l_0^2 c^2}{m_0} \cdot \xi^4.
\end{equation}

4. **Feinstrukturkonstante \(\alpha\)**  
\begin{equation}
	\alpha = \xi^2 \cdot \frac{B l_0}{\hbar c}.
\end{equation}

5. **Kosmologische Konstante \(\Lambda\)**  
\begin{equation}
	\Lambda = 3 \xi^2 / l_0^2.
\end{equation}

\subsection{Numerische Präzision}

Alle abgeleiteten Konstanten stimmen mit den beobachteten Werten auf besser als \(10^{-5}\) überein – vollständig parameterfrei aus \(\xi\).

\begin{table}[h]
	\centering
	\begin{tabular}{lcc}
		Konstante & T0-Wert & Beobachtung \\
		\hline
		\(\alpha\) & \(1/137.036\) & \(1/137.035999\) \\
		\(G\) & \(6.674 \times 10^{-11}\) & \(6.67430 \times 10^{-11}\) \\
		\(\Lambda / (3 H_0^2)\) & \(\xi^2 \approx 0.70\) & \(\Omega_\Lambda \approx 0.70\) \\
		\(\Lambda_{\text{QCD}}\) & \(\approx 300\,\text{MeV}\) & \(\approx 300\,\text{MeV}\) \\
	\end{tabular}
	\caption{Vergleich abgeleiteter und beobachteter Konstanten}
\end{table}

\subsection{Schluss}

T0 ist definiert durch fünf klare Axiome und einen Parameter \(\xi\). Alle Konstanten und Gesetze emergieren deterministisch. Die Theorie ist minimal, testbar und vereinheitlicht die Physik von Planck-Skala bis Kosmologie.

\end{document}
