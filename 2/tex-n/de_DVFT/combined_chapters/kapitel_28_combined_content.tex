\section{Kapitel 28: Warum Newtons Gesetz nicht für Quantenteilchen gilt }
	
	Das Newtonsche Gesetz \(F = G m_1 m_2 / r^2\) funktioniert hervorragend für Planeten, Sterne und Galaxien. Aber gilt es für ein einzelnes Proton, das ein anderes Proton anzieht? Die Antwort lautet: Nein, nicht fundamental.
	
	Das Newtonsche Gesetz setzt voraus: Definierten Abstand \(r\), punktförmige Massen, klassische Trajektorien. In Quantenmechanik fehlen diese.
	
	Fraktale T0-Theorie: Gravitation nicht als Raumzeitkrümmung, sondern als Deformation des Vakuumamplitudenfelds \(\rho(x,t) \propto 1/T(x,t)\). Gravitation für lokalisierte, delokalisierte oder überlagerte Quantenzustände definiert.
	
	Gravitationsfeld \(\delta\rho(x)\) folgt Quantenwellenfunktion \(|\psi(x)|^2\). Klassischer Grenzfall entsteht durch Dekohärenz. Keine Singularitäten: \(\rho_0 = 1/\xi^2\) liefert Minimum.
	
	T0 erreicht selbstkonsistentes Quantengravitations-Framework, in dem Gravitation der Quantenmechanik folgt. Alles aus \(\xi\).
Die klassische Gravitation (Newton/GR) ist für quantenmechanische Systeme nicht definiert – z. B. kann man keine Gravitationskraft zwischen zwei superponierten Zuständen eines Protons berechnen. T0 löst dies durch die Kopplung an die fraktale Vakuum-Amplitude \(\rho\).

\subsection{Probleme der klassischen Gravitation auf Quantenskala}

Newtonsche Gravitation:
\begin{equation}
	F = G \frac{m_1 m_2}{r^2}
\end{equation}
setzt definite Positionen und Massen voraus. Für ein Proton in Superposition \(|\psi\rangle = \alpha |x_1\rangle + \beta |x_2\rangle\) ist unklar, welche Kraft wirkt.

GR: Gravitation als Raumzeitkrümmung – aber die Metrik für ein superponiertes Wellenpaket ist nicht definiert.

\subsection{Gravitation als Amplitude-Deformation in T0}

In T0 koppelt Materie an die Vakuum-Amplitude:
\begin{equation}
	\delta \rho(x) = \frac{G}{c^2} \cdot T^{00}(x) \cdot \xi^{-1},
\end{equation}
wobei \(T^{00} = m c^2 |\psi(x)|^2\) für nicht-relativistische Teilchen.

Die effektive Gravitationsbeschleunigung:
\begin{equation}
	g(x) = -\xi \cdot \nabla \ln \rho(x) \approx -\xi \cdot \frac{\nabla \delta \rho}{\rho_0}.
\end{equation}

Für ein quantenmechanisches System:
\begin{equation}
	\delta \rho(x) = \frac{G m}{c^2} \cdot |\psi(x)|^2 \cdot \xi^{-1}.
\end{equation}

Die selbstgravitative Energie:
\begin{equation}
	E_{\text{self}} = \int \frac{G m^2}{c^2} \cdot \frac{|\psi(x)|^2 |\psi(y)|^2}{|x-y|} \, d^3x d^3y \cdot \xi^{-2}.
\end{equation}

\subsection{Superposition und Nichtlokalität}

Für Superposition \(|\psi\rangle = \alpha |\phi_1\rangle + \beta |\phi_2\rangle\):
\begin{equation}
	\delta \rho(x) = \frac{G m}{c^2 \xi} \left( |\alpha|^2 |\phi_1(x)|^2 + |\beta|^2 |\phi_2(x)|^2 + 2 \Re(\alpha^* \beta \phi_1^*(x) \phi_2(x)) \right).
\end{equation}

Der Interferenzterm erzeugt nichtlokale Gravitation – kein „zwei Felder“-Problem.

\subsection{Vergleich mit anderen Ansätzen}

\begin{itemize}
	\item Newton-Schrödinger: Nichtlinear, kollabiert Superposition,
	\item Post-quantum GR: Ad-hoc Kollaps-Modelle,
	\item T0: Linear, deterministisch, nichtlokal durch \(\xi\).
\end{itemize}

\subsection{Beispiel: Gravitation zwischen zwei Protonen}

Für \(r = 10^{-15}\,\text{m}\) (Fermi-Abstand):
\begin{equation}
	F_g \approx \xi \cdot G \frac{m_p^2}{r^2} \approx 10^{-40} \, \text{N},
\end{equation}
vernachlässigbar, aber definiert für delokalisierte Zustände.

\subsection{Schluss}

T0 definiert Gravitation auf Quantenskala konsistent als Amplitude-Deformation \(\delta \rho \propto |\psi|^2\). Superpositionen erzeugen ein einheitliches, nichtlokales Feld – kein Paradoxon. Dies ist die erste vollständig kohärente Quantengravitation auf Teilchenskala.