\documentclass[12pt,a4paper]{article}
\usepackage[utf8]{inputenc}
\usepackage[T1]{fontenc}
\usepackage[ngerman]{babel}
\usepackage{amsmath}
\usepackage{amsfonts}
\usepackage{amssymb}
\usepackage{geometry}
\geometry{a4paper,left=2.5cm,right=2.5cm,top=2.5cm,bottom=2.5cm}
\usepackage{fancyhdr}
\usepackage{enumitem}
\usepackage{tcolorbox}
\usepackage{physics}
\usepackage{hyperref}

% Hyperref als eines der letzten Pakete laden
\hypersetup{
	unicode=true,
	pdfencoding=unicode,
	bookmarksopen=true
}

% Saubere PDF-Lesezeichen
\pdfstringdefDisableCommands{%
	\def\Lambda{Lambda}%
	\def\Delta{Delta}%
	\def\approx{etwa}%
	\def\Sigma{Sigma}%
	\def\eta{eta}%
	\def\psi{psi}%
}

\title{Kapitel 12: Kosmologie, Big Bang und Geburt des Universums}
\author{}
\date{}

\begin{document}

\maketitle

\section{Kapitel 12: Kosmologie, Big Bang und Geburt des Universums }
	
	In der fraktalen DVFT wird die Kosmologie als emergentes Phänomen aus dem Vakuumsubstrat verstanden, das keine klassische Expansion erfährt. Stattdessen ist der Big Bang ein Phasenübergang, bei dem das fraktale Vakuumfeld \(\Phi = \rho(x,t) e^{i\theta(x,t)}\) von einem instabilen Zustand mit \(\rho \approx 0\) zu einem stabilen Gleichgewicht bei \(\rho_0 \propto 1/\xi^2\) übergeht. Dieser Übergang stabilisiert die fraktale Dimension \(D_f = 3 - \epsilon \approx 2.94\), was die scheinbare Homogenität und Isotropie des Universums erklärt, ohne eine explosive Ausdehnung zu benötigen.
	
	Die narrative Interpretation sieht dies als eine selbstähnliche Entfaltung: Das Vakuumsubstrat, anfangs strukturlos, entwickelt durch die Instabilität der reinen Phase \(\theta\) eine Amplitude \(\rho\), die fraktale Muster erzeugt. Beobachtbare Effekte wie die kosmische Mikrowellenhintergrundstrahlung (CMB) entstehen aus fraktalen Fluktuationen in \(\delta \rho / \rho \propto r^{-\epsilon}\), die Anisotropien mit Skaleninvarianz \(n_s \approx 0.96\) produzieren, passend zu Planck-Daten.
	
	Die Friedmann-Gleichungen werden fraktal modifiziert: \(H^2 \propto \varepsilon_{\text{vac}} (1 + \epsilon \ln(\rho / \rho_0))\), wobei \(\varepsilon_{\text{vac}} = \frac{1}{2} A (\partial_t \rho)^2 + \rho^2 (\partial_t \theta)^2 + V(\rho)\) und \(V(\rho) = \lambda (\rho^2 - \rho_0^2)^2 (1 + \epsilon \ln \rho)\) fraktal korrigiert ist. Dies eliminiert die Notwendigkeit für Inflation, da die fraktale Selbstähnlichkeit natürliche Homogenität auf allen Skalen gewährleistet. Die Dunkle Energie emergiert als residuale fraktale Korrektur \(\propto \epsilon \rho_0\), die die beschleunigte ''Expansion'' simuliert, ohne tatsächliche Dynamik. In dieser Erzählung ist das Universum ewig und statisch in seiner fraktalen Struktur, mit Beobachtungen, die durch die Geometrie der Photonpropagation erklärt werden.
Die Standardkosmologie beginnt mit einer Singularität bei \(t=0\). T0 ersetzt diese durch einen regulierten fraktalen Übergang.

\subsection{Fraktale Friedmann-Gleichungen in T0}

\begin{equation}
	\left(\frac{\dot{a}}{a}\right)^2 = \frac{8\pi G}{3} \rho - \frac{k}{a^2} + \xi \cdot \frac{c^2}{l_0^2 a^4},
\end{equation}
Im frühen Universum:
\begin{equation}
	a(t) \propto t^{1/2}.
\end{equation}

\subsection{Vergleich mit Loop Quantum Cosmology (LQC)}

LQC hat modifizierte Friedmann-Gleichung mit \(\rho_{\text{crit}}\), führt zu Big Bounce.

\textbf{Wichtige Unterschiede zu T0}:
\begin{itemize}
	\item LQC quantengeometrisch, Immirzi-Parameter,
	\item T0 klassisch fraktal, nur \(\xi\).
\end{itemize}

\subsection{Vergleich mit Stringtheorie-Kosmologie}

Stringtheorie-Szenarien (Pre-Big-Bang, Ekpyrotisch) benötigen höhere Dimensionen.

\textbf{Wichtige Unterschiede zu T0}:
\begin{itemize}
	\item Stringtheorie komplex, viele Parameter,
	\item T0 minimal, parameterfrei.
\end{itemize}

\subsection{Schluss}

T0 liefert die einfachste Kosmologie: Big Bang als fraktaler Phasenübergang.

\end{document}
