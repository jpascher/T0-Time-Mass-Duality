\documentclass[12pt,a4paper]{article}
\usepackage{amsmath, amssymb, amsthm}
\usepackage{geometry}
\usepackage{titlesec}
\usepackage{tcolorbox}
\usepackage{enumitem}
\usepackage{booktabs}

% Theoreme
\newtheorem{theorem}{Theorem}[section]
\newtheorem{lemma}[theorem]{Lemma}
\newtheorem{corollary}[theorem]{Korollar}
\newtheorem{definition}[theorem]{Definition}

\title{Kapitel 11: Innere Struktur Schwarzer Löcher in T0 – Vergleich mit Loop Quantum Gravity und Stringtheorie}
\author{}
\date{}

\begin{document}

\maketitle

In der Allgemeinen Relativitätstheorie führen kollabierende Sterne unvermeidlich zu Singularitäten. T0 eliminiert diese durch die fraktale Diskretisierung der Raumzeit auf der Skala \(\xi\), während Loop Quantum Gravity (LQG) dies durch Quantisierung der Geometrie und Stringtheorie durch fundamentale Strings erreicht.

\subsection{Mathematische Beschreibung des Kollapses in T0}

Die Schwarzschild-Metrik in Standard-GR lautet
\begin{equation}
	ds^2 = -\left(1 - \frac{2GM}{r}\right) dt^2 + \left(1 - \frac{2GM}{r}\right)^{-1} dr^2 + r^2 d\Omega^2.
\end{equation}
Bei \(r \to 0\) divergiert die Krümmung \(R \propto 1/r^4\).

In T0 wird die Metrik fraktal modifiziert:
\begin{equation}
	ds^2 = -\left(1 - \frac{2GM}{r}\right) dt^2 + \left(1 - \frac{2GM}{r}\right)^{-1} dr^2 \cdot \left(1 + \xi \cdot \Theta(r - r_\xi)\right) + r^2 d\Omega^2,
\end{equation}
wobei \(\Theta\) eine regulierende Schrittfunktion ist und \(r_\xi \approx l_P \cdot \xi^{-1}\) die T0-Kernskala darstellt. Der effektive Ricci-Skalar bleibt endlich:
\begin{equation}
	R_{\text{eff}} \leq R_{\max} \approx \frac{c^4}{G \hbar} \cdot \xi^2.
\end{equation}

Der Kollaps stoppt bei einer endlichen Dichte
\begin{equation}
	\rho_{\text{kern}} \approx \frac{m_P}{l_P^3} \cdot \xi^{-3}.
\end{equation}

\subsection{Vergleich mit Loop Quantum Gravity (LQG)}

LQG quantisiert die Raumzeit durch Spin-Netzwerke. Flächen- und Volumenoperatoren haben diskrete Spektren:
\begin{equation}
	\hat{A} \psi = 8\pi \gamma l_P^2 \sqrt{j(j+1)} \psi,
\end{equation}
wobei \(\gamma\) der Immirzi-Parameter ist. Schwarze Löcher werden zu „Quantum Black Holes“ mit Bounce.

\textbf{Wichtige Unterschiede zu T0}:
\begin{itemize}
	\item LQG ist vollständige Quantengravitation, benötigt Immirzi-Parameter,
	\item Regularisierung durch Quanteneffekte,
	\item T0 regularisiert klassisch durch fraktale Struktur mit nur \(\xi\),
	\item T0 liefert parameterfreie Teilchenmassen.
\end{itemize}

\subsection{Vergleich mit Stringtheorie}

Stringtheorie reguliert durch Stringlänge \(l_s\). Schwarze Löcher als Fuzzballs oder Mikrozustände.

\textbf{Wichtige Unterschiede zu T0}:
\begin{itemize}
	\item Stringtheorie benötigt höhere Dimensionen, Supersymmetrie, Landscape,
	\item T0 ist 4-dimensional, minimal, parameterfrei.
\end{itemize}

\subsection{Schluss}

T0 bietet die einfachste, parameterfreie Regularisierung Schwarzer Löcher durch fraktale Time-Mass-Duality.

\end{document}
