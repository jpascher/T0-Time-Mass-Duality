\documentclass[12pt,a4paper]{article}
\usepackage{amsmath, amssymb, amsthm}
\usepackage{geometry}
\usepackage{titlesec}
\usepackage{tcolorbox}
\usepackage{enumitem}
\usepackage{booktabs}

% Theoreme
\newtheorem{theorem}{Theorem}[section]
\newtheorem{lemma}[theorem]{Lemma}
\newtheorem{corollary}[theorem]{Korollar}
\newtheorem{definition}[theorem]{Definition}

\title{Kapitel 10: Reinterpretation der Dunklen Energie in T0}
\author{}
\date{}

\begin{document}

\maketitle

Die beschleunigte Expansion des Universums wird in T0 nicht durch eine ad-hoc kosmologische Konstante erklärt, sondern als residuale Dynamik der fraktalen Vakuumstruktur.

\subsection{Ableitung der Vakuumenergie-Dichte}

Die fraktale Hierarchie erzeugt eine natürliche Energie-Skala
\begin{equation}
	\rho_{\text{vac}}^{\text{T0}} = \xi^2 \cdot \rho_{\text{crit}} \approx 0.7 \cdot \rho_c,
\end{equation}
wobei \(\rho_c = 3 H_0^2 / (8\pi G)\) die kritische Dichte ist. Dies entspricht exakt dem beobachteten \(\Omega_\Lambda \approx 0.7\).

\subsection{Genauere Herleitung}

Aus der fraktalen Selbstähnlichkeit der Friedmann-Gleichungen ergibt sich ein effektiver Term
\begin{equation}
	\ddot{a}/a = -\frac{4\pi G}{3} (\rho_m + \rho_r) + \xi \cdot \frac{c^2}{l_0^2},
\end{equation}
der für späte Zeiten dominiert und \(w \approx -1\) liefert.

\subsection{Dynamische Aspekte und Hubble-Tension}

Im Gegensatz zur starren \(\Lambda\) ist die T0-Vakuumenergie leicht zeitabhängig, was kleine Variationen von \(H_0\) zwischen früher und später Kosmologie erklärt (Hubble-Tension).

\subsection{Schluss}

Dunkle Energie ist in T0 die makroskopische Manifestation der fraktalen Skala \(\xi\) – parameterfrei abgeleitet und vereinheitlicht mit lokaler Gravitation.

\end{document}
