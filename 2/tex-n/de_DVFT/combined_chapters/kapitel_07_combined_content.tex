Die Spezielle Relativitätstheorie (SRT) ist in T0 keine unabhängige Theorie mit eigenen Postulaten, sondern emergiert zwangsläufig aus der Forderung nach Invarianz der fraktalen Skalenhierarchie unter Transformationen zwischen Inertialsystemen.

\subsection{Fraktale Lorentz-Invarianz als Grundprinzip}

Die fraktale Selbstähnlichkeit mit Parameter \(\xi\) muss in allen Inertialsystemen identisch erhalten bleiben. Dies erzwingt automatisch die Konstanz der Lichtgeschwindigkeit \(c\) als maximale kausale Signalgeschwindigkeit in der fraktalen Struktur.

\subsection{Detaillierte Ableitung der Lorentz-Transformationen}

Betrachten wir zwei Inertialsysteme S und S'. Die Skalenfunktion \(\mathcal{F}(x,t)\) muss invariant sein. Dies führt zu der Transformationsforderung
\begin{align}
	x' &= \gamma (x - v t), \nonumber \\
	t' &= \gamma \left( t - \frac{v x}{c^2} \right), \nonumber \\
	\gamma &= \left(1 - \frac{v^2}{c^2}\right)^{-1/2}.
\end{align}
Die Ableitung erfolgt rein aus der Erhaltung der fraktalen Dimensionsbeziehungen unter Boosts – ohne Postulat der Lichtkonstanz.

\subsection{Zeitdilatation, Längenkontraktion und relativistische Dynamik}

Zeitdilatation entsteht, weil bewegte Uhren eine skalierte Zeitkomponente in der fraktalen Hierarchie erfahren. Längenkontraktion ist die duale Masseneffekt. Die relativistische Energie-Impuls-Relation
\begin{equation}
	E^2 = p^2 c^2 + m^2 c^4
\end{equation}
folgt aus der Erhaltung fraktaler Invarianten unter Lorentz-Boosts.

\subsection{Schluss}

In T0 ist die SRT eine emergente Notwendigkeit der fraktalen Raumzeitinvarianz mit \(\xi\). Alle relativistischen Effekte – inklusive \(E = mc^2\) für bewegte Systeme – sind direkte Konsequenzen der Time-Mass-Duality und benötigen keine separaten Postulate.