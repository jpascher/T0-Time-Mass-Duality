Die T0-Theorie leitet ihre Feldgleichungen direkt aus der fraktalen Skalenhierarchie und der Time-Mass-Duality ab. Im Gegensatz zu modifizierten Gravitationstheorien werden keine zusätzlichen Felder oder Parameter eingeführt.

\subsection{Die fraktale Metrik}

Die effektive Metrik in T0 lautet
\begin{equation}
	g_{\mu\nu}^{\text{eff}} = g_{\mu\nu} + \xi \, h_{\mu\nu}(\mathcal{F}),
\end{equation}
wobei $h_{\mu\nu}(\mathcal{F})$ die fraktale Korrekturterme beschreibt, die von der Skalenfunktion $\mathcal{F}(r)$ abhängen.

\subsection{Ableitung der Gravitationsgleichungen}

Durch Variation der fraktalen Wirkung ergibt sich
\begin{equation}
	R_{\mu\nu} - \frac{1}{2} R g_{\mu\nu} + \xi \Delta T_{\mu\nu}^{\text{fractal}} = 8\pi G T_{\mu\nu},
\end{equation}
wobei $\Delta T_{\mu\nu}^{\text{fractal}}$ der effektive Energie-Impuls-Tensor der fraktalen Struktur ist. Auf makroskopischen Skalen ($\xi \to 0$) reduzieren sich die Gleichungen exakt auf die Einstein-Gleichungen.

\subsection{Schluss}

Die Feldgleichungen von T0 sind parameterfrei und entstehen allein aus der fraktalen Selbstähnlichkeit in Kombination mit der Time-Mass-Duality. Sie vereinigen Gravitation und Quanteneffekte in einer einzigen Struktur.