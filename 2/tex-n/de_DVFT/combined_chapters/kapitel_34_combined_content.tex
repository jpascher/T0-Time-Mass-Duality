\section{Kapitel 34: Lösung des Strong-CP-Problems }
	
	Das Strong-CP-Problem fragt, warum der CP-verletzende Parameter \(\theta_{\text{QCD}}\) in QCD experimentell unter \(10^{-10}\) liegt, obwohl das Standardmodell Werte bis 1 erlaubt. Die fraktale DVFT bietet eine natürliche Lösung ohne Axionen oder Feinabstimmung.
	
	In fraktaler DVFT ist das Vakuumphasenfeld \(\theta\) global und einzig, da es aus T0s universellem Zeitfeld emergiert. Die Phase ist nicht lokal wählbar; die Freiheit, \(\theta\) zu drehen, existiert nicht, weil das Feld physisch und fraktal verbunden ist.
	
	Daher \(\theta_{\text{QCD}} = 0\) ist der einzige mathematisch erlaubte Wert. Die fraktale Selbstähnlichkeit eliminiert Duplizierbarkeit der Phase.
	
	Dies löst das Problem sauber: Keine Axionen, keine Feinabstimmung, volle Übereinstimmung mit Experiment. Starke konzeptionelle Triumph der fraktalen DVFT.
Das Strong CP-Problem fragt, warum der CP-verletzende Parameter \(\theta_{\text{QCD}}\) experimentell auf \(\theta < 10^{-10}\) beschränkt ist, obwohl natürliche Werte \(\theta \approx 1\) erwartet werden. T0 löst dies durch die globale Einzigkeit der Vakuumphase \(\theta\).

\subsection{Formulierung des Problems}

Die QCD-Lagrangedichte enthält
\begin{equation}
	\mathcal{L}_\theta = \theta \frac{g^2}{32\pi^2} \operatorname{Tr}(G_{\mu\nu} \tilde{G}^{\mu\nu}).
\end{equation}

Dies erzeugt Neutronen-EDM:
\begin{equation}
	d_n \approx \theta \cdot 3 \times 10^{-16} \, e\,\text{cm}.
\end{equation}

Experimentell \(\theta < 10^{-10}\).

\subsection{Einzigkeit der Vakuumphase}

In T0 gibt es nur eine globale Phase \(\theta(x,t)\):
\begin{equation}
	\Phi(x) = \rho(x) e^{i \theta(x)/\xi}.
\end{equation}

Alle Gauge-Felder emergieren aus dieser Phase – kein separater \(\theta_{\text{QCD}}\).

\subsection{Ableitung \(\theta = 0\)}

Effektiver Term:
\begin{equation}
	\mathcal{L}_\theta = \xi \cdot \theta \cdot \operatorname{Tr}(F \wedge F).
\end{equation}

Variation:
\begin{equation}
	\xi \operatorname{Tr}(F \wedge F) + \xi^2 \nabla^2 \theta = 0.
\end{equation}

Minimale Energie bei \(\theta = \text{konstant}\) und \(\operatorname{Tr}(F \wedge F) = 0\).

Globale Abweichung kostet unendliche Energie – \(\theta = 0\) zwangsläufig.

\subsection{Rest-CP-Verletzung}

Lokale Fluktuationen:
\begin{equation}
	\delta \theta \approx \xi^{3/2} \sqrt{\ln(V/l_0^3)} \approx 10^{-12},
\end{equation}
halten EDM im beobachteten Bereich.

\subsection{Vergleich mit Axion}

Axion: Dynamisches Feld \(a/f_a\).  
T0: Kein zusätzliches Feld – strukturell \(\theta = 0\).

\subsection{Schluss}

T0 löst das Strong CP-Problem fundamental durch globale Vakuumphase. \(\theta = 0\) ist zwangsläufig – Konsequenz der Time-Mass-Duality mit \(\xi\).