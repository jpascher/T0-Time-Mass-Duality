\section{Kapitel 35: Erklärung quantenmechanischer Phänomene }
	
	Die fraktale DVFT interpretiert Quantenmechanik als Verhalten von Vakuumphasen- und Amplitudenfeldern, fundiert in T0s Dualität. Dieses Kapitel erklärt zwölf große Quantenphänomene einheitlich.
	
	Interferenz aus Phasenaddition in \(\theta\). Kollaps als lokale Amplitudenstörung \(\delta\rho\). Verschränkung aus globaler Phasenkopplung. Dekohärenz aus Phasenverstreuung durch Interaktionen.
	
	Superposition aus multiplen Phasenkonfigurationen. Tunneln durch Phasenbarriere. Nullpunktsenergie aus intrinsischer \(\mu = \xi m_0\)-Oszillation. Vakuumfluktuationen \(\Delta\theta \cdot \Delta E \geq \hbar/2\) aus T0-Fluktuationen \(\Delta m\).
	
	Atomare Quantisierung aus \(\theta\)-Zirkulationsbedingungen \(\oint \nabla\theta \cdot dl = 2\pi n\).
	
	Die fraktale DVFT vereinheitlicht Gravitation und Quantenmechanik, quantenmechanisches Verhalten in Vakuumphaseneigenschaften verankert.
T0 interpretiert Quantenmechanik als Verhalten der fraktalen Vakuum-Amplitude \(\rho\) und Phase \(\theta\). Dieses Kapitel erklärt zwölf zentrale Quantenphänomene einheitlich und physikalisch – ohne abstrakte Postulate.

\subsection{Wellenfunktion-Kollaps}

In T0 ist Kollaps Dekohärenz der Vakuumphase durch makroskopische Kopplung:
\begin{equation}
	\Delta \theta_{\text{macro}} \gg \xi \quad \Rightarrow \quad \Gamma_{\text{decoh}} = \xi^2 \cdot \frac{\Delta E}{\hbar}.
\end{equation}

Messung zerstört Kohärenz:
\begin{equation}
	\rho_{\text{mixed}} = \sum_i p_i |\theta_i\rangle\langle\theta_i|.
\end{equation}

Kollaps ist physikalisch: Phasen-Scrambling.

\subsection{Wellen-Teilchen-Dualität}

Wellen: Kohärente Phasenmuster \(\theta(kx - \omega t)\).  
Teilchen: Lokalisierte Amplitude-Deformationen \(\delta \rho(x)\).

Dualität: Zwei Aspekte desselben Feldes \(\Phi = \rho e^{i\theta}\).

\subsection{Verschränkung}

Verschränkung ist globale Phasenkorrelation:
\begin{equation}
	\theta_{\text{total}} = \theta_1 + \theta_2 = \text{konstant},
\end{equation}
auch bei räumlicher Trennung durch fraktale Nichtlokalität.

Bellsche Korrelation:
\begin{equation}
	\langle A B \rangle = \cos(\Delta \theta_{12}) \cdot \xi^{-1/2}.
\end{equation}

Nichtlokal, aber kausal – keine Signalübertragung.

\subsection{Zero-Point-Energie}

Grundzustandsenergie pro Mode:
\begin{equation}
	E_0 = \frac{1}{2} \hbar \omega \cdot \xi \cdot \left(1 + \sum_{k=1}^\infty \xi^k\right) \approx \frac{1}{2} \hbar \omega \cdot \frac{\xi}{1-\xi}.
\end{equation}

Endlich durch fraktalen Cut-off – löst kosmologisches Konstanten-Problem.

\subsection{Delayed-Choice- und Quantum-Eraser-Experimente}

Interferenz hängt von globaler Phasen-Kohärenz ab:
\begin{equation}
	\Delta \phi = \theta_{\text{path1}} - \theta_{\text{path2}}.
\end{equation}

Which-Path: Markiert Idler-Phase \(\Delta \theta_{\text{idler}} = \pi\).  
Erasure: Löscht Markierung \(\Delta \theta_{\text{idler}} = 0\).

Verzögerte Wahl klassifiziert nur Unterensemble – keine Retrokausalität.

\subsection{Dekohärenz}

Dekohärenz ist Phasen-Scrambling:
\begin{equation}
	\Gamma = \xi^2 \cdot N \cdot \frac{k_B T}{\hbar}.
\end{equation}

Makroskopische Systeme zerstören Kohärenz physikalisch.

\subsection{Quantenrandomness}

Randomness aus fraktalen Fluktuationen:
\begin{equation}
	\Delta \theta \cdot \Delta E \geq \xi \hbar / 2.
\end{equation}

Inhärenter Jitter – deterministisch auf Vakuumskala.

\subsection{Atomare Quantisierung}

Energieniveaus aus Phasen-Zirkulation:
\begin{equation}
	\oint \nabla \theta \cdot dl = 2\pi n \cdot \xi^{-1/2}.
\end{equation}

Spektrallinien als stabile Phasenmoden.

\subsection{Weitere Phänomene}

Tunneln: Phasen-Unterbarrieren-Propagation.  
Interferenz: Konstruktive Phasen-Überlapp.  
Entanglement-Swapping: Globale Phasen-Neuzuordnung.

\subsection{Schluss}

T0 unifiziert alle Quantenphänomene als Vakuumphasen-Dynamik. Wellenfunktion ist reale Phase \(\theta\), Kollaps physikalisches Scrambling, Verschränkung globale Korrelation – alles parameterfrei aus \(\xi\).