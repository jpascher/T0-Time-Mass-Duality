\documentclass[12pt,a4paper]{article}
\usepackage{amsmath, amssymb, amsthm}
\usepackage{geometry}
\usepackage{titlesec}
\usepackage{tcolorbox}
\usepackage{enumitem}
\usepackage{booktabs}

% Theoreme
\newtheorem{theorem}{Theorem}[section]
\newtheorem{lemma}[theorem]{Lemma}
\newtheorem{corollary}[theorem]{Korollar}
\newtheorem{definition}[theorem]{Definition}

\title{Kapitel 1: Feldgleichungen der T0-Time-Mass-Duality}
\author{}
\date{}

\begin{document}

\maketitle

Die Feldgleichungen von T0 entstehen durch Variation einer fraktalen Wirkung, die ausschließlich aus der Metrik und dem Skalenparameter \(\xi\) aufgebaut ist.

\subsection{Die fraktale Wirkung}

Die T0-Wirkung lautet
\[S = \int \left( \frac{R}{16\pi G} + \xi \cdot \mathcal{L}_{\text{fractal}} \right) \sqrt{-g} \, d^4x,\]
wobei \(\mathcal{L}_{\text{fractal}}\) die fraktale Korrektur-Lagrangedichte beschreibt.

\subsection{Ableitung der modifizierten Einstein-Gleichungen}

Durch Variation nach der Metrik ergeben sich
\[R_{\mu\nu} - \frac{1}{2} R g_{\mu\nu} + \xi \cdot T_{\mu\nu}^{\text{fractal}} = 8\pi G \left( T_{\mu\nu}^{\text{matter}} + T_{\mu\nu}^{\text{vac}} \right),\]
wobei \(T_{\mu\nu}^{\text{fractal}}\) der effektive Energie-Impuls-Tensor der fraktalen Struktur ist, der auf makroskopischen Skalen null wird.

\subsection{Kopplung an Materie}

Materiefelder koppeln minimal an die effektive Metrik, wodurch nichtlokale Quanteneffekte auf kleinen Skalen entstehen.

\subsection{Schluss}

Die T0-Feldgleichungen sind parameterfrei, reduzieren im Grenzfall \(\xi \to 0\) exakt auf die Einstein-Gleichungen und erzeugen auf kleinen Skalen deterministische quantengravitative Effekte.

\end{document}
