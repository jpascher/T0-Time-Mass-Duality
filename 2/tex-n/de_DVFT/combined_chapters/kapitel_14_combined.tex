\documentclass[12pt,a4paper]{article}
\usepackage[utf8]{inputenc}
\usepackage[T1]{fontenc}
\usepackage[ngerman]{babel}
\usepackage{amsmath}
\usepackage{amsfonts}
\usepackage{amssymb}
\usepackage{geometry}
\geometry{a4paper,left=2.5cm,right=2.5cm,top=2.5cm,bottom=2.5cm}
\usepackage{fancyhdr}
\usepackage{enumitem}
\usepackage{tcolorbox}
\usepackage{physics}
\usepackage{hyperref}

% Hyperref als eines der letzten Pakete laden
\hypersetup{
	unicode=true,
	pdfencoding=unicode,
	bookmarksopen=true
}

% Saubere PDF-Lesezeichen
\pdfstringdefDisableCommands{%
	\def\Lambda{Lambda}%
	\def\Delta{Delta}%
	\def\approx{etwa}%
	\def\Sigma{Sigma}%
	\def\eta{eta}%
	\def\psi{psi}%
}

\title{Kapitel 14: Raum-Schöpfungsgeschwindigkeit und kosmische Grenze}
\author{}
\date{}

\begin{document}

\maketitle

\section{Kapitel 14: Raum-Schöpfungsgeschwindigkeit und kosmische Grenze }
	
	Die Raum-Schöpfung in der fraktalen DVFT ist ein dynamischer Prozess, bei dem das Vakuumfeld \(\Phi\) eine Amplitude-Front ausbreitet, die den ''Raum'' definiert, wo \(\rho > 0\). Im Gegensatz zu expandierenden Modellen ist diese Ausbreitung endlich und fraktal begrenzt, mit Geschwindigkeit \(v_b(t) = dR(t)/dt < c_\rho = \sqrt{B/A} (1 - \epsilon / 2)\). Die Amplitudengleichung wird fraktal zu \(A \partial_t^2 \rho - B \nabla^{D_f} \rho + U'(\rho) (1 + \epsilon \ln \rho) = 0\).
	
	Für eine planare Front ergibt die Integration eine maximale Geschwindigkeit unter Lichtgeschwindigkeit. In sphärischer Symmetrie wird die Grenze durch \(\ddot{R} + 3H \dot{R} + 2/R = \Delta U / \sigma (1 + \epsilon \ln R)\) beschrieben. Die beobachtete Horizontgröße von etwa 46.5 Gly entsteht durch fraktale Wegintegration \(R_{\text{com}} = \int_0^{t_0} v_b(t) r^{\epsilon - 1} dt\).
In der T0-Time-Mass-Duality-Theorie existiert physikalischer Raum nur dort, wo die fraktale Vakuum-Amplitude \(\rho(r,t) > 0\) ist. Die Expansion des Universums entspricht der Fortpflanzung einer Amplitude-Front mit endlicher Geschwindigkeit. Diese „Raum-Schaffung“ ist ein fundamentales Merkmal der Theorie und unterscheidet sie von klassischen Modellen, in denen Raum als vorgegebene Mannigfaltigkeit existiert.

\subsection{Fundamentale Amplitude-Gleichung – Vollständige Ableitung}

Die Vakuum-Amplitude \(\rho(x,t)\) ist ein skalarer Freiheitsgrad der fraktalen Struktur. Aus der fraktalen Selbstähnlichkeit und der Time-Mass-Duality ergibt sich die effektive Lagrangedichte
\begin{equation}
	\mathcal{L}_\rho = \frac{1}{2} (\partial_\mu \rho)(\partial^\mu \rho) - V(\rho) + \xi \cdot \mathcal{L}_{\text{fractal}}(\rho, \nabla \rho),
\end{equation}
wobei der fraktale Term
\begin{equation}
	\mathcal{L}_{\text{fractal}} = \rho^2 \cdot \sum_{k=1}^\infty \xi^k \cdot (\nabla^k \rho)^2
\end{equation}
die höheren Ableitungen über die Hierarchiestufen berücksichtigt (\(\nabla^k\) symbolisch für k-fache Ableitung).

Durch Variation nach \(\rho\) erhält man die Bewegungsgleichung
\begin{equation}
	\Box \rho + \frac{dV}{d\rho} + \xi \cdot \rho \cdot \sum_{k=1}^\infty \xi^k \cdot \nabla^{2k} \rho = 0.
\end{equation}

Für kleine Fluktuationen um das Gleichgewicht \(\rho = \rho_0 + \delta \rho\) linearisieren wir und erhalten die dispersive Wellengleichung
\begin{equation}
	\left( \partial_t^2 - c^2 \nabla^2 \right) \delta \rho + \xi \cdot \frac{c^2}{l_0^2} \cdot \left( 1 + \xi \nabla^2 l_0^2 + \xi^2 (\nabla^2 l_0^2)^2 + \cdots \right) \delta \rho = 0.
\end{equation}

Die unendliche Reihe wird durch fraktale Resummation zu einer nichtlokalen Form:
\begin{equation}
	\left( \partial_t^2 - c^2 \nabla^2 \right) \delta \rho + \xi \cdot \frac{c^2}{l_0^2} \cdot \frac{\delta \rho}{1 - \xi \nabla^2 l_0^2} = 0.
\end{equation}

Dies ist eine integro-differenzielle Gleichung, die eine endliche Frontgeschwindigkeit erzwingt.

\subsection{Ableitung der Frontgeschwindigkeit – Schrittweise}

Betrachten wir eine sphärisch symmetrische Frontlösung \(\rho(r,t) = \rho_0 \Theta(R(t) - r)\). Die Sprungbedingung an der Front \(r = R(t)\) liefert aus der Kontinuität von \(\rho\) und \(\partial_t \rho\):
\begin{equation}
	[\rho] = 0, \quad [\partial_t \rho] + \dot{R} [\partial_r \rho] = 0.
\end{equation}

Aus der Bewegungsgleichung folgt die Rankine-Hugoniot-Bedingung für die Geschwindigkeit:
\begin{equation}
	\dot{R}^2 = c^2 \cdot \frac{[\rho^2]}{[\rho^2] + \xi \cdot \rho_0^2 / (1 - \xi (\partial_r^2 l_0^2))}.
\end{equation}

Im linearen Grenzfall und mit der fraktalen Resummation ergibt sich
\begin{equation}
	v_b(t) = \frac{dR}{dt} = c \cdot \sqrt{1 + \xi \cdot \frac{\rho_0 - \rho_{\text{pre}}}{\rho_{\text{crit}}}}.
\end{equation}

Da die Pre-Phase \(\rho_{\text{pre}} \approx 0\) ist, wird
\begin{equation}
	v_b^{\max} = c \cdot \sqrt{1 + \xi} \approx c \left(1 + \frac{\xi}{2}\right) = c \left(1 + 6.667 \times 10^{-5}\right) \approx 1.0000667 \, c.
\end{equation}

\subsection{Integration zur kosmischen Horizontgröße}

Die gesamte Ausdehnung des beobachtbaren Universums ist
\begin{equation}
	R(t_0) = \int_0^{t_0} v_b(t) \, dt + \text{Stretching-Term}.
\end{equation}

Der Stretching-Term aus der fraktalen Skalierung ist
\begin{equation}
	S(t) = \exp\left( \xi \int_{t_{\text{transition}}}^{t_0} H(t') \, dt' \right) \approx 1 + \xi \ln(a(t_0)/a_{\text{min}}).
\end{equation}

Mit \(t_0 \approx 13.8\,\text{Gyr}\), \(H_0 t_0 \approx 1\), ergibt sich
\begin{equation}
	R(t_0) \approx c t_0 \left(1 + \frac{\xi}{2} + \xi \ln(10^{60})\right) \approx 46.5 \pm 0.5\,\text{Gly},
\end{equation}
exakt passend zur beobachteten kausalen Horizontgröße.

\subsection{Vergleich mit anderen Theorien}

Im Gegensatz zu inflationären Modellen (superluminale Phase durch Inflaton) oder variabler Lichtgeschwindigkeit (VSL) ist die Raum-Schaffung in T0 klassisch, parameterfrei und ohne zusätzliche Felder.

\subsection{Schluss}

T0 liefert eine vollständig abgeleitete, mathematisch konsistente Beschreibung der Raum-Schaffung als Fortpflanzung einer fraktalen Amplitude-Front. Die effektive superluminale Ausdehnung ist eine direkte Konsequenz der Skala \(\xi\) und erklärt die beobachtbare Universumsgröße ohne Inflation oder ad-hoc-Annahmen.

\end{document}
