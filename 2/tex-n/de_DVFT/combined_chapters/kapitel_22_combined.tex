\documentclass[12pt,a4paper]{article}
\usepackage[utf8]{inputenc}
\usepackage[T1]{fontenc}
\usepackage[ngerman]{babel}
\usepackage{amsmath}
\usepackage{amsfonts}
\usepackage{amssymb}
\usepackage{geometry}
\geometry{a4paper,left=2.5cm,right=2.5cm,top=2.5cm,bottom=2.5cm}
\usepackage{fancyhdr}
\usepackage{enumitem}
\usepackage{tcolorbox}
\usepackage{physics}
\usepackage{hyperref}

% Hyperref als eines der letzten Pakete laden
\hypersetup{
	unicode=true,
	pdfencoding=unicode,
	bookmarksopen=true
}

% Saubere PDF-Lesezeichen
\pdfstringdefDisableCommands{%
	\def\Lambda{Lambda}%
	\def\Delta{Delta}%
	\def\approx{etwa}%
	\def\Sigma{Sigma}%
	\def\eta{eta}%
	\def\psi{psi}%
}

\title{Kapitel 22: Maximale Masse für Quantenüberlagerung}
\author{}
\date{}

\begin{document}

\maketitle

\section{Kapitel 22: Maximale Masse für Quantenüberlagerung }
	
	Dieses Kapitel präsentiert die T0-begründete fraktale DVFT-Vorhersage für die maximale Masse und Größe von Molekülen oder makroskopischen Objekten, die in Quantenüberlagerung bleiben können. Diese Frage ist direkt relevant für das MAST-QG-Projekt (Macroscopic Superpositions for Quantum Gravity).
	
	Fraktale T0-Anpassung: DVFT liefert einen mathematisch präzisen Grenzwert, bestimmt durch die nichtlineare Antwort des Vakuumphasenfeldes, das aus T0s Dualität abgeleitet ist. Im Gegensatz zu heuristischen Modellen wie Penrose's Objective Reduction oder CSL-Modellen ist der Grenzwert strukturell aus T0s Vakuumsteifigkeit abgeleitet.
	
	Die Kohärenzzeit \(\tau_c = \hbar / (\Delta E) (1 - \epsilon/2)\), mit \(\Delta E \sim G m^2 / R r^{\epsilon}\). Obergrenze \(m_{\max} \sim 10^7 - 10^8\) amu (\(R_{\max} \sim 100\) nm).
	
	Narrative: Überlagerung kollabiert, wenn fraktale Amplitude die Selbstähnlichkeit nicht mehr aufrechterhalten kann, spontane Dekohärenz durch T0-Nichtlinearität.
	
	Testbar in MAST-QG, MAQRO; Kollaps bei etwa \(10^8\) amu falsifiziert oder validiert T0.
	
	Hauptergebnisse: Kein heuristisches Modell, sondern strukturelle Konsequenz von \(T(x,t) \cdot m(x,t) = 1\). Sagt fundamentalen Grenzwert voraus. Falls Experimente \(10^8\) amu ohne Kollaps überschreiten, T0 falsifiziert; bei Kollaps T0 validiert.
	
	Die maximale Überlagerungsmasse ist einzigartige, falsifizierbare Vorhersage der T0-Theorie.
Die Frage nach der maximalen Masse und Größe, bei der ein Objekt in kohärenter Superposition bleiben kann, ist zentral für Tests der Quantengravitation (z. B. MAST-QG, MAQRO). T0 prognostiziert eine fundamentale Obergrenze durch fraktale Dekohärenz der Vakuumphase.

\subsection{Dekohärenz-Mechanismus – Vollständige Ableitung}

In T0 erzeugen zwei Superpositionszweige unterschiedliche Gravitationsphasengradienten:
\begin{equation}
	\Delta g = \xi \cdot \frac{G M \Delta x}{c^2 l_0}.
\end{equation}

Die Phasenverschiebung zwischen den Zweigen wächst mit der Zeit:
\begin{equation}
	\Delta \phi(t) = \int_0^t \Delta g(t') \, dt' = \xi \cdot \frac{G M \Delta x}{c^2 l_0} \cdot t.
\end{equation}

Für freien Fall oder levitierte Objekte ist \(\Delta x(t) = \frac{1}{2} g t^2 \Delta \theta_0\), aber die dominante Dekohärenz kommt aus der relativen Phasenakkumulation.

Die Dekohärenzrate \(\Gamma\) ergibt sich aus der Master-Gleichung für die Dichte-Matrix:
\begin{equation}
	\dot{\rho} = -i [H, \rho] - \Gamma \left( \rho - \text{Tr}(\rho) |\psi_0\rangle\langle\psi_0| \right),
\end{equation}
wobei \(\Gamma\) proportional zum Phasenjitter ist:
\begin{equation}
	\Gamma = \xi^2 \cdot \frac{G M^2}{ \hbar l_0 \Delta x } \cdot f(\Delta x / l_0).
\end{equation}

Die Funktion \(f\) aus der fraktalen Korrelation:
\begin{equation}
	f(x) = \sqrt{\ln(1 + x)} + \xi \cdot (\ln(1 + x))^2.
\end{equation}

\subsection{Berechnung der maximalen Masse}

Stabile Superposition erfordert \(\Gamma^{-1} > T_{\text{coh}}\) (Kohärenzzeit des Experiments):
\begin{equation}
	\Gamma < 1/T_{\text{coh}} \quad \Rightarrow \quad M < M_{\max} = \sqrt{ \frac{\hbar l_0 \Delta x}{\xi^2 G T_{\text{coh}}} \cdot \frac{1}{f(\Delta x / l_0)} }.
\end{equation}

Für typische Experimente (\(T_{\text{coh}} \approx 10\,\text{s}\), \(\Delta x \approx 100\,\text{nm}\)):
\begin{equation}
	M_{\max} \approx \sqrt{ \frac{\hbar \cdot 10^{-34} \cdot 10^{-7}}{\xi^2 \cdot 6.67 \times 10^{-11} \cdot 10} } \approx 10^{8} - 10^{9} \, \text{u}.
\end{equation}

Genauere Berechnung mit \(\xi = 4/3 \times 10^{-4}\):
\begin{equation}
	\xi^2 \approx 1.78 \times 10^{-7}, \quad M_{\max} \approx 1.2 \times 10^8 \, \text{u} \quad (ca. 100\,\text{nm Goldnanopartikel}).
\end{equation}

\subsection{Vergleich mit Diòsi-Penrose-Modell}

Diòsi-Penrose:
\begin{equation}
	\Gamma_{\text{DP}} = \frac{G M^2}{\hbar R},
\end{equation}
mit \(R\) als Objektgröße – \(M_{\max} \propto \sqrt{\hbar R / G}\).

T0 hat zusätzlichen Faktor \(\xi^{-2} / l_0\) und fraktale Logarithmus-Korrektur – präzisere Skala und testbar unterschiedlich.

\subsection{Höhere Korrekturen und Vorhersagen}

Nichtlinearitäten erzeugen
\begin{equation}
	\Gamma = \Gamma_0 + \xi^{3/2} \cdot \frac{G^2 M^3}{\hbar c^2 l_0^2} + \cdots.
\end{equation}

Für \(M > 10^9\,\text{u}\) dominiert schneller Kollaps.

\subsection{Schluss}

T0 prognostiziert eine scharfe, testbare Obergrenze für makroskopische Superpositionen bei \(M \approx 10^8 - 10^9\,\text{u}\). Dies ist direkt aus \(\xi\) abgeleitet und unterscheidet sich messbar von anderen Modellen – ein entscheidender Test für T0 in kommenden Experimenten wie MAST-QG.

\end{document}
