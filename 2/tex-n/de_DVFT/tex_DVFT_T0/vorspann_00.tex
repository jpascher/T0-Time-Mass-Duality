% kapitel_00_vorspann.tex
\section*{Vorspann: T0 – Time-Mass-Duality}
\addcontentsline{toc}{section}{Vorspann: T0 – Time-Mass-Duality}

\textbf{Autor:} Johann Pascher \\
\textbf{Datum:} Dezember 2025

\textbf{Abstract}

Die T0-Time-Mass-Duality-Theorie stellt eine fundamentale, parameterfreie Vereinheitlichung der Physik dar. Sie basiert ausschließlich auf der intrinsischen Dualität zwischen Zeit und Masse sowie einer fraktalen Struktur der Raumzeit mit dem einzigen Skalenparameter
\[\xi = \frac{4}{3} \times 10^{-4}.\]

Dieser Parameter \(\xi\) markiert den Übergang von der klassischen kontinuierlichen Beschreibung zur fraktalen Quantenskala und ist die einzige freie Größe der Theorie. Aus \(\xi\) werden durch fraktale Selbstähnlichkeit und Dimensionsanalyse sämtliche Naturkonstanten parameterfrei abgeleitet, darunter die Gravitationskonstante \(G\), die Lichtgeschwindigkeit \(c\) (als emergente Größe), das reduzierte Plancksche Wirkungsquantum \(\hbar\), die kosmologische Konstante \(\Lambda\) sowie die Feinstrukturkonstante \(\alpha \approx 1/137\).

T0 liefert präzise Vorhersagen für die Ruhemassen aller bekannten Elementarteilchen mit einer Genauigkeit von über 98\,\% (Elektron, Muon, Tau-Lepton, Up-, Down-, Strange-, Charm-, Bottom-, Top-Quarks, Neutrinos, W- und Z-Bosonen sowie das Higgs-Boson). Diese Massen ergeben sich deterministisch aus fraktal skalierten Zeitintervallen in der T0-Hierarchie.

Die Theorie eliminiert physikalische Singularitäten (Schwarze Löcher, Big Bang), erklärt Galaxierotationskurven und kosmologische Beschleunigung ohne Dunkle Materie oder Dunkle Energie, beschreibt die Quantenmechanik als deterministisch und nichtlokal durch die fraktale Struktur und vereinigt Gravitation, Quantenmechanik sowie das Standardmodell in einer einzigen ontologischen Grundlage.

T0 ist testbar, falsifizierbar und liefert neue Vorhersagen in der Teilchenphysik, Kosmologie und Quantengravitation. Die folgenden Kapitel entwickeln die mathematische Struktur und die physikalischen Konsequenzen der Theorie systematisch.