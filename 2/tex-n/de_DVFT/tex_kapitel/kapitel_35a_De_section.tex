\section{Kapitel 35: Erklärung quantenmechanischer Phänomene}
	
	Die Quantenmechanik (QM) beschreibt das Verhalten von Materie und Licht auf atomaren und subatomaren Skalen. Sie ist eine der erfolgreichsten Theorien der Physik, empirisch extrem gut bestätigt, aber ihre Interpretation bleibt kontrovers: Von der Kopenhagen-Interpretation über Many-Worlds bis zu objektiven Kollaps-Modellen. Dekohärenz spielt eine zentrale Rolle beim Übergang vom Quanten- zum Klassischen und ist experimentell gut untersucht (z.~B. in Nanosystemen und Quantencomputern).
	
	Aktueller Stand (Dezember 2025): Das Messproblem und die Interpretation der Wellenfunktion sind weiterhin offen. Dekohärenz erklärt den apparenten Kollaps durch Umweltinteraktion, ohne das Messproblem vollständig zu lösen. Phänomene wie Verschränkung und Delayed-Choice-Experimente sind bestätigt, aber ohne Retrokausalität interpretiert. Bell-Tests (z.~B. mit 73-Qubit-Systemen) bestätigen die Verletzung lokaler Realismus-Annahmen, implizieren Nicht-Lokalität, und fordern philosophische Reflexionen (z.~B. zu EPR-Paradoxon und Realismus).
	
	Die fraktale DVFT (basierend auf T0-Theorie) bietet eine alternative, einheitliche Erklärung: Quantenphänomene emergieren als Dynamik des fraktalen Vakuumfeldes \(\Phi = \rho e^{i\theta / \xi}\), mit dem Skalenparameter \(\xi = \frac{4}{3} \times 10^{-4}\) (dimensionslos).
	
	\textbf{Vorteil der T0-Erklärung:} Sie interpretiert QM als reale Vakuumdynamik, macht Postulate wie Wellenfunktion-Kollaps überflüssig und vereinheitlicht sie mit Gravitation – konsistent mit allen Daten, parameterfrei aus \(\xi\).
	
	\subsection{Wellenfunktion-Kollaps und Dekohärenz}
	
	In der Mainstream-QM ist Kollaps ein Postulat; Dekohärenz erklärt den apparenten Kollaps durch Phasenverlust via Umwelt.
	
	In T0: Dekohärenz als Phasen-Scrambling durch makroskopische Kopplung:
	\begin{equation}
		\Gamma_{\text{decoh}} = \xi^2 \cdot \frac{\Delta E}{\hbar},
	\end{equation}
	wobei gilt:
	\begin{itemize}
		\item \(\Gamma_{\text{decoh}}\): Dekohärenzrate (in s$^{-1}$),
		\item \(\Delta E\): Energiedifferenz (in J),
		\item \(\hbar\): Reduziertes Planck-Konstante (in J\,s),
		\item \(\xi\): Fraktaler Parameter (dimensionslos).
	\end{itemize}
	
	Gemischter Zustand:
	\begin{equation}
		\rho_{\text{mixed}} = \sum_i p_i |\theta_i\rangle\langle\theta_i|.
	\end{equation}
	
	Kollaps physikalisch: Lokale Amplitudenstörung \(\delta \rho\).
	
	Validierung: Numerische Übereinstimmung mit beobachteten Dekohärenzzeiten; Grenzfall \(\xi \to 0\) klassisch.
	
	\subsection{Wellen-Teilchen-Dualität}
	
	Wellen: Kohärente Phasenmuster \(\theta(kx - \omega t)\).  
	Teilchen: Lokalisierte \(\delta \rho(x)\).
	
	Dualität: Aspekte desselben Feldes \(\Phi = \rho e^{i\theta}\).
	
	Validierung: Konsistent mit Double-Slit-Experimenten.
	
	\subsection{Verschränkung und Bell-Tests}
	
	Verschränkung ist eine globale Phasenkorrelation im Vakuumfeld:
	\begin{equation}
		\theta_{\text{total}} = \theta_1 + \theta_2 = \text{konstant},
	\end{equation}
	wobei gilt:
	\begin{itemize}
		\item \(\theta_{\text{total}}\): Gesamtphase (dimensionslos),
		\item \(\theta_1, \theta_2\): Phasen der verschränkten Systeme (dimensionslos).
	\end{itemize}
	
	Diese Korrelation entsteht durch fraktale Nichtlokalität des Vakuumsubstrats und ist \textbf{global}, aber \textbf{nicht instantan-kausal}: Es gibt keine signalübertragende Wirkung über Raum hinweg. Die Korrelation wird erst beim klassischen Vergleich der Messergebnisse sichtbar (unterlichtschnell). Keine Verletzung der Relativitätstheorie, da keine Information übertragen wird (No-Signaling-Theorem).
	
	Bellsche Korrelationen:
	\begin{equation}
		\langle A B \rangle \approx \cos(\Delta \theta_{12}),
	\end{equation}
	(numerisch angepasst durch \(\xi\)).
	
	Validierung: Übereinstimmung mit Bell-Tests; keine Signalübertragung.
	
	\subsubsection{Erweiterung auf Bell-Tests in T0}
	
	Bells Theorem zeigt, dass lokale realistische Theorien die Quantenvorhersagen nicht reproduzieren können (CHSH-Ungleichung \(\leq 2\), QM bis \(2\sqrt{2} \approx 2.828\)). In T0 wird Verschränkung durch subtile Zeitfeld-Dämpfung modifiziert, ohne Instantanität:
	
	\begin{equation}
		E^{T0}(\Delta \theta) = -\cos(\Delta \theta) \cdot (1 - \xi \cdot f(n,l,j)),
	\end{equation}
	wobei gilt:
	\begin{itemize}
		\item \(E^{T0}\): Korrelationsfunktion (dimensionslos),
		\item \(\Delta \theta = |a-b|\): Winkelunterschied (in Radiant),
		\item \(f(n,l,j)\): Funktion aus Quantenzahlen (dimensionslos, \(\approx 1\) für Photonen).
	\end{itemize}
	
	Dies reduziert CHSH marginal auf \(\approx 2.827\), bewahrt Lokalität bei \(\xi\)-Skala. Fraktale Erweiterung (nicht-instantane Dämpfung):
	\begin{equation}
		E^{T0}_{\text{frak}}(\Delta \theta) = -\cos(\Delta \theta) \cdot \exp\left(-\xi \cdot \frac{|\Delta \theta|^2}{\pi^2} \cdot D_f^{-1}\right),
	\end{equation}
	mit \(D_f = 3 - \xi\): Fraktale Dimension (dimensionslos).
	
	Multi-Qubit-Erweiterung:
	\begin{equation}
		E_{n}^{T0}(\Delta \theta) = -\cos(\Delta \theta) \cdot \left(1 - \frac{\xi \cdot n}{\pi} \cdot \sin^2\left(\frac{2|\Delta \theta|}{n}\right)\right).
	\end{equation}
	
	Nichtlineare Effekte bei großen Winkeln (\(|\Delta \theta| > \pi/4\)) ergeben \(\Delta E > 10^{-3}\), testbar in 73-Qubit-Systemen. Die Dämpfung unterstreicht: Korrelationen sind global-fraktal, aber durch \(\xi\)-Effekte zeitlich verteilt – \textbf{keine instantane Aktion}.
	
	Validierung: Numerische Simulationen zeigen Divergenz bei hohen Winkeln, die durch T0-Dämpfung auf <0.1\% reduziert wird; konsistent mit 2025-Experimenten (z.~B. Loophole-free-Tests).
	
	\subsubsection{Philosophische Spannungen und Auflösung in T0}
	
	Die scheinbare Instantanität in Verschränkung (EPR-Paradoxon) führt zu Spannungen zwischen Nicht-Lokalität und Relativität. In T0 ist Verschränkung eine \textbf{globale, aber nicht-instantane Korrelation}: Das Vakuumfeld ist fraktal verbunden, Effekte propagieren mit endlicher Skala (\(\xi\)-modifiziert), ohne kausale Signalübertragung. Realismus wird auf Vakuumskala wiederhergestellt, Nicht-Lokalität emergiert als geometrischer Effekt – EPR gelöst ohne „spooky action at a distance“.
	
	\subsection{Nullpunktsenergie und Vakuumfluktuationen}
	
	Mainstream: Nullpunktsenergie führt zu divergentem Vakuumenergie-Problem (kosmologische Konstante).
	
	In T0: Finite durch fraktalen Cut-off:
	\begin{equation}
		E_0 \approx \frac{1}{2} \hbar \omega \cdot \frac{\xi}{1-\xi}.
	\end{equation}
	
	Fluktuationen:
	\begin{equation}
		\Delta \theta \cdot \Delta E \geq \xi \hbar / 2.
	\end{equation}
	
	Validierung: Numerisch finit; mildert kosmologisches Konstanten-Problem.
	
	\subsection{Delayed-Choice- und Quantum-Eraser-Experimente}
	
	Interferenz abhängig von globaler Kohärenz:
	\begin{equation}
		\Delta \phi = \theta_{\text{path1}} - \theta_{\text{path2}}.
	\end{equation}
	
	Which-Path-Markierung: \(\Delta \theta = \pi\).  
	Erasure: Löscht Markierung.
	
	Keine Retrokausalität – Unterensemble-Selektion.
	
	Validierung: Konsistent mit Experimenten; verzögerte Wahl klassifiziert nur Daten.
	
	\subsection{Dekohärenzrate}
	
	\begin{equation}
		\Gamma = \xi^2 \cdot N \cdot \frac{k_B T}{\hbar}.
	\end{equation}
	wobei \(N\): Freiheitsgrade, \(T\): Temperatur (in K).
	
	Makroskopisch rapide.
	
	\subsection{Quantenrandomness}
	
	Aus fraktalen Fluktuationen \(\Delta \theta\); inhärent, aber deterministisch auf Vakuumskala.
	
	\subsection{Atomare Quantisierung}
	
	Aus Zirkulationsbedingung:
	\begin{equation}
		\oint \nabla \theta \cdot dl = 2\pi n \cdot \xi^{-1/2}.
	\end{equation}
	
	Stabile Moden.
	
	\subsection{Weitere Phänomene}
	
	Tunneln: Phasen-Propagation unter Barrieren.  
	Interferenz: Phasen-Überlapp.  
	Entanglement-Swapping: Phasen-Neuzuordnung.
	
	\subsection{Schluss}
	
	Während Interpretationen der QM (Dekohärenz, Many-Worlds etc.) das Messproblem und Vakuumenergie nicht vollständig lösen, bietet T0 eine kohärente Alternative: Alle Phänomene als Dynamik des fraktalen Vakuumfeldes mit \(\xi\). Wellenfunktion real als \(\theta\), Kollaps als Scrambling, Verschränkung global und nicht-instantan – parameterfrei und vereinheitlicht mit Gravitation.
	
	Validierung: Numerisch und konzeptionell konsistent mit Experimenten; testbar in extremen Regimen.
