\documentclass[12pt,a4paper]{article}
\usepackage[utf8]{inputenc}
\usepackage[T1]{fontenc}
\usepackage[english]{babel}
\usepackage{amsmath}
\usepackage{amsfonts}
\usepackage{amssymb}
\usepackage{geometry}
\geometry{a4paper,left=2.5cm,right=2.5cm,top=2.5cm,bottom=2.5cm}
\usepackage{fancyhdr}
\usepackage{enumitem}
\usepackage{tcolorbox}
\usepackage{physics}
\usepackage{hyperref}
\usepackage{siunitx} % For correct units

% Load hyperref as one of the last packages
\hypersetup{
	unicode=true,
	pdfencoding=unicode,
	bookmarksopen=true
}

% Clean PDF bookmarks
\pdfstringdefDisableCommands{%
	\def\Lambda{Lambda}%
	\def\Delta{Delta}%
	\def\approx{approximately}%
	\def\Sigma{Sigma}%
	\def\eta{eta}%
	\def\psi{psi}%
}

\title{Chapter 36: Why Quantum Field Theory (QFT) Did Not Become a Gravity Theory – T0 Perspective (As of December 2025)}
\author{}
\date{}

\begin{document}
	
	\maketitle
	
	\section{Chapter 36: Why Quantum Field Theory (QFT) Did Not Become a Gravity Theory}
	
	Quantum field theory (QFT) is the most successful description of the three non-gravitational forces (electromagnetic, weak, strong) in the Standard Model of particle physics. It is renormalizable and empirically extremely precise. However, the inclusion of gravitation fails: perturbative quantum gravity is non-renormalizable (divergences from second loop), leading to approaches such as string theory, loop quantum gravity, or asymptotic safety.
	
	Current Status (December 2025): No experimentally confirmed quantum gravity theory exists. The Standard Model plus General Relativity (GR) remains effective, but incompatible at Planck scale. The hierarchy problem and vacuum energy (cosmological constant) remain unsolved. Recent work (e.g., on fractal approaches in QFT) explores alternative interpretations, but remains speculative.
	
	Fractal DVFT (based on T0-theory) offers an alternative view: QFT already contains the mathematical structure for gravitation, but failed due to the interpretation of vacuum as "empty" and phase as non-physical. T0 makes \(\rho\) and \(\theta\) real vacuum degrees of freedom with parameter \(\xi = \frac{4}{3} \times 10^{-4}\) (dimensionless).
	
	\textbf{Advantage of the T0 perspective:} It unifies QFT and gravitation without new particles or dimensions – purely through physical interpretation of the complex vacuum field.
	
	\subsection{Mathematical Structure Already Present in QFT}
	
	Complex scalar field in QFT (polar form):
	\begin{equation}
		\Phi(x) = \rho(x) e^{i \theta(x)/v},
	\end{equation}
	where:
	\begin{itemize}
		\item \(\Phi(x)\): Scalar field (complex),
		\item \(\rho(x)\): Amplitude (real, positive),
		\item \(\theta(x)\): Phase (in radians, dimensionless),
		\item \(v\): Vacuum expectation value (VEV, in energy units, e.g., GeV).
	\end{itemize}
	
	Lagrangian density:
	\begin{equation}
		\mathcal{L} = (\partial_\mu \Phi)^\dagger (\partial^\mu \Phi) - V(|\Phi|^2) = (\partial_\mu \rho)^2 + \rho^2 (\partial_\mu \theta)^2 - V(\rho).
	\end{equation}
	
	This corresponds structurally to the T0 form:
	\begin{equation}
		\mathcal{L}_{\text{T0}} = K_0 (\partial \rho)^2 + B (\partial \theta)^2 - U(\rho).
	\end{equation}
	where:
	\begin{itemize}
		\item \(K_0, B\): Stiffness coefficients (in suitable units for energy density),
		\item \(U(\rho)\): Potential (in energy density).
	\end{itemize}
	
	Validation: Mathematically identical; QFT already had amplitude (Higgs-like) and phase (Goldstone).
	
	\subsection{Historical and Conceptual Reasons for Failure}
	
	1. Vacuum interpreted as "empty" – VEV \(v\) as spontaneous symmetry breaking, not as physical medium.
	
	2. Phase \(\theta\) as non-physical: Goldstone bosons are "eaten" in Higgs mechanism (unitary gauge).
	
	3. Gravitation as pure geometry (GR): Spacetime as dynamic background, not as field in vacuum.
	
	4. Renormalizability problem: Perturbative quantization of metric leads to non-renormalizable divergences.
	
	Validation: These interpretations are empirically successful in the Standard Model, but prevent unification with gravitation.
	
	\subsection{Correction Through T0 Interpretation}
	
	T0 identifies:
	\begin{equation}
		\rho \leftrightarrow \text{Vacuum amplitude (inertia, curvature)},
	\end{equation}
	\begin{equation}
		\theta \leftrightarrow \text{Vacuum phase (time flow, quantum coherence)}.
	\end{equation}
	
	Stiffness ratio:
	\begin{equation}
		K_0 / B \approx \xi^{-1} \approx 7.5 \times 10^{3},
	\end{equation}
	where \(\xi^{-1} \approx 7500\) (dimensionless); explains hierarchy between gravitation and other forces.
	
	Gravitational acceleration:
	\begin{equation}
		g = -\xi \cdot \nabla \ln \rho.
	\end{equation}
	where:
	\begin{itemize}
		\item \(g\): Gravitational acceleration (in \si{m/s^2}),
		\item \(\nabla \ln \rho\): Gradient of logarithmic amplitude (in m$^{-1}$).
	\end{itemize}
	
	Gauge fields emerge from \(\nabla \theta\).
	
	Validation: In the limit \(\xi \to 0\) reduces to standard QFT without gravitational effects.
	
	\subsection{Mathematical Unification in T0}
	
	Extended Lagrangian density:
	\begin{equation}
		\mathcal{L}_{\text{T0}} = K_0 (\partial \rho)^2 + B (\partial \theta)^2 + \xi \cdot \rho^2 (\partial \theta)^2 \mathcal{F} + \mathcal{L}_{\text{matter}}(\psi, \partial \theta).
	\end{equation}
	where:
	\begin{itemize}
		\item \(\mathcal{F}\): Fractal correction terms (dimensionless or adjusted),
		\item \(\mathcal{L}_{\text{matter}}\): Matter terms, coupled to \(\partial \theta\).
	\end{itemize}
	
	High-energy limit (\(\xi \to 0\)): Standard QFT.  
	Low-energy limit: Effective gravitation (GR-like).
	
	Validation: Renormalizability through fractal cut-off; finite vacuum energy.
	
	\subsection{Conclusion}
	
	Mainstream QFT fails at unification with gravitation due to historical interpretations (empty vacuum, non-physical phase, geometric gravitation) and technical problems (non-renormalizability). T0 theory offers a coherent alternative: Through physical interpretation of \(\rho\) and \(\theta\) as real vacuum degrees of freedom, gravitation emerges naturally from fractal vacuum dynamics with \(\xi\). T0 is thus a possible completion of QFT structure – parameter-free and unified.
	
	Validation: Conceptually consistent with QFT successes and GR; testable in hierarchy and vacuum energy predictions.
	
\end{document}
