\documentclass[12pt,a4paper]{article}
\usepackage[utf8]{inputenc}
\usepackage[T1]{fontenc}
\usepackage[english]{babel}
\usepackage{amsmath,amssymb,amsthm}
\usepackage{geometry}
\usepackage{titlesec}
\usepackage{tcolorbox}
\usepackage{enumitem}
\usepackage{booktabs}
\usepackage{hyperref}

\geometry{margin=2.5cm}

% Theorems
\newtheorem{theorem}{Theorem}[section]
\newtheorem{lemma}[theorem]{Lemma}
\newtheorem{corollary}[theorem]{Corollary}
\newtheorem{definition}[theorem]{Definition}

% Title
\title{
	\textbf{Dynamic Vacuum Field Theory (DVFT)} \\
	\Large Complete Integration of Fractal T0-Geometry \\
	\normalsize With Detailed Scientific Explanations and Formula Analyses
}
\author{}
\date{December 2025}

\begin{document}
	
	\maketitle
	
	\section*{Abstract}

		This document presents the completely revised \textbf{Dynamic Vacuum Field Theory (DVFT)} with consistent integration of \textbf{fractal T0-geometry}. It demonstrates how all fundamental physical phenomena emerge from a unified fractal vacuum substrate with scale parameter \(\xi = \frac{4}{3} \times 10^{-4}\) and Time-Mass Duality. The presentation is self-explanatory and replaces all previous versions. Formulas are extensively explained, including definitions of symbols, units, and possible validations through limiting cases or comparisons with known empirical values.
	
	
	\vspace{1cm}
	
	{\centering\textbf{Fundamental Basis of T0-Theory}\par}
	
	In T0-theory there is exactly \textbf{one single fundamental parameter}: the geometric scale parameter \(\xi = \frac{4}{3} \times 10^{-4}\). All other quantities – including the fractal dimension \(D_f\), the fine-structure constant \(\alpha\), Planck's constant \(\hbar\) (as well as \(h = 2\pi \hbar\)), the speed of light \(c\), the gravitational constant \(G\), and all characteristic scales (Planck length, time, mass, etc.) – are \textbf{necessarily and parameter-free derived from \(\xi\)}. In particular:
	\begin{itemize}
		\item The fractal dimension \(D_f = 3 - \xi\) is not an assumption but a direct geometric consequence of the packing deficit in the vacuum substrate.
		\item The fine-structure constant \(\alpha\) emerges from fractal self-similarity and mass hierarchies.
		\item The quantum of action \(\hbar\) results from discretization of action on the effective Planck scale.
	\end{itemize}
	
	A detailed derivation of all constants from \(\xi\) can be found in supplementary documents in the repository, e.g.:
	\begin{itemize}
		\item \textit{T0\_Feinstruktur.pdf} (Derivation of \(\alpha\)),
		\item \textit{T0\_unified\_report.pdf} / \textit{T0\_vereinigter\_bericht.pdf} (Unified derivation of all constants),
		\item \textit{133\_Fraktale\_Korrektur\_Herleitung.pdf} (Proof of \(D_f = 3 - \xi\) and \(K_{\text{frak}}\)).
	\end{itemize}
	
	Available at: \url{https://github.com/jpascher/T0-Time-Mass-Duality/tree/main/2/pdf}
	
	\tableofcontents
	\newpage
	
	\section{Introduction to T0-Time-Mass Duality and its Field Equations}
	
	T0-theory extends wave-particle duality to a complementary Time-Mass Duality, whereby absolute time and variable mass are viewed as aspects of a unified geometric field. This enables unification of quantum mechanics and general relativity through a fractal vacuum substrate with scale parameter \(\xi = \frac{4}{3} \times 10^{-4}\) (dimensionless, as a measure of fractal packing deficit) and fractal dimension \(D_f = 3 - \xi \approx 2.999867\) (dimensionless, Hausdorff dimension of effective spacetime).
	
	\subsection{The Fractal Action and its Derivation}
	
	The fundamental action in T0 is an extension of the Einstein-Hilbert action with fractal corrections:
	\begin{equation}
		S = \int \left( \frac{R}{16\pi G} + \xi \cdot \mathcal{L}_{\text{fractal}} \right) \sqrt{-g} \, d^4x,
	\end{equation}
	where:
	\begin{itemize}
		\item \(S\): The action (unit: J\,s, as variational principle for field equations),
		\item \(R\): Ricci scalar (unit: m$^{-2}$, measure of spacetime curvature),
		\item \(G\): Gravitational constant (unit: m$^{3}$\,kg$^{-1}$\,s$^{-2}$),
		\item \(\xi\): Fractal scale parameter (dimensionless, value \(\frac{4}{3} \times 10^{-4}\)),
		\item \(\mathcal{L}_{\text{fractal}}\): Fractal Lagrangian density (unit: J/m$^{3}$, correction term for self-similarity),
		\item \(g\): Determinant of the metric (dimensionless),
		\item \(d^4x\): Volume element (unit: m$^{4}$).
	\end{itemize}
	
	The derivation proceeds from variation of a fractal metric that accounts for self-similarity of spacetime. The parameter \(\xi\) represents the geometric packing deficit in three-dimensional space, derived from tetrahedral symmetry and the golden ratio \(\phi = (1 + \sqrt{5})/2 \approx 1.618\) (dimensionless). The term \(\xi \cdot \mathcal{L}_{\text{fractal}}\) regulates ultraviolet divergences through discretization on Planck scales (\(l_P \approx 1.62 \times 10^{-35}\)~m) and describes the vacuum as a compressible medium in which Time-Mass Duality \(T(x,t) \cdot m(x,t) = 1\) holds (T: time density in s/m$^{3}$, m: mass density in kg/m$^{3}$, product dimensionless = 1).
	
	Validation: In the limit \(\xi \to 0\), the action reduces exactly to the classical Einstein-Hilbert action, consistent with all known tests of general relativity (e.g., Mercury's perihelion precession).
	
	\subsection{Derivation of Modified Einstein Equations}
	
	Variation of the action with respect to the metric \(g_{\mu\nu}\) yields the field equations
	\begin{equation}
		R_{\mu\nu} - \frac{1}{2} R g_{\mu\nu} + \xi \cdot T_{\mu\nu}^{\text{fractal}} = 8\pi G \left( T_{\mu\nu}^{\text{matter}} + T_{\mu\nu}^{\text{vac}} \right),
	\end{equation}
	where:
	\begin{itemize}
		\item \(R_{\mu\nu}\): Ricci tensor (unit: m$^{-2}$),
		\item \(g_{\mu\nu}\): Metric tensor (dimensionless),
		\item \(T_{\mu\nu}^{\text{fractal}}\): Fractal energy-momentum tensor (unit: J/m$^{3}$),
		\item \(T_{\mu\nu}^{\text{matter}}\): Matter energy-momentum tensor (unit: J/m$^{3}$),
		\item \(T_{\mu\nu}^{\text{vac}}\): Vacuum energy-momentum tensor (unit: J/m$^{3}$).
	\end{itemize}
	
	The variation leads to standard contributions from \(R\) as well as additional terms from \(\xi \cdot \mathcal{L}_{\text{fractal}}\), which vanish on macroscopic scales (\(r \gg 10^{-15}\)~m). The effective metric reads \(g_{\mu\nu}^{\text{eff}} = g_{\mu\nu} + \xi h_{\mu\nu}(\mathcal{F})\) with scale function \(\mathcal{F}(r) = \ln(1 + r/r_\xi)\) (dimensionless, r: distance in m, \(r_\xi\): fractal core scale \(\approx 10^{-15}\)~m). The fractal term explains dark matter as a geometric effect and ensures UV-finiteness without renormalization.
	
	Validation: On cosmological scales, the equation reduces to the Friedmann equations, consistent with CMB data (Planck mission).
	
	\subsection{Conclusion}
	
	The T0 field equations are parameter-free (only \(\xi\)) and emerge from fractal self-similarity combined with Time-Mass Duality.
	
	\section{Why Spacetime in T0 is Fractal and Dual}
	
	A continuous spacetime leads to singularities and divergences. T0 describes spacetime as fractal with \(\xi = \frac{4}{3} \times 10^{-4}\) and intrinsic Time-Mass Duality.
	
	\subsection{Necessity of Fractal Structure}
	
	The fractal dimension \(D_f = 3 - \xi\) regulates singularities and UV divergences. It results from the packing density of tetrahedral structures:
	\begin{equation}
		D_f = \lim_{\epsilon \to 0} \frac{\ln N(\epsilon)}{\ln(1/\epsilon)},
	\end{equation}
	where:
	\begin{itemize}
		\item \(D_f\): Fractal dimension (dimensionless),
		\item \(N(\epsilon)\): Number of self-similar units at resolution \(\epsilon\) (dimensionless),
		\item \(\epsilon\): Scale factor (dimensionless).
	\end{itemize}
	
	The volume scaling \(V \sim r^{D_f}\) (V: volume in m$^{3}$, r: radius in m) breaks continuity on Planck scales and makes the theory finite.
	
	Validation: The value \(D_f \approx 2.999867\) lies close to 3, consistent with macroscopic 3D spacetime, but introduces quantum effects on small scales.
	
	\subsection{The Intrinsic Time-Mass Duality}
	
	The fundamental relation
	\begin{equation}
		T(x,t) \cdot m(x,t) = 1
	\end{equation}
	follows from fractal self-similarity: scale transformations \(\xi^k\) link time intervals with mass scales such that the product remains invariant (T: time density in s/m$^{3}$, m: mass density in kg/m$^{3}$, product dimensionless = 1). Vacuum stability enforces this constancy.
	
	Validation: In limiting cases of high mass density (e.g., neutron stars), the effective time density decreases, consistent with relativistic time dilation.
	
	\subsection{Conclusion}
	
	Fractality and duality are unavoidable consequences of a singularity-free, low-parameter spacetime description.
	
	\section{Problems of General Relativity and their Solution through T0}
	
	General relativity (GR) suffers from singularities, dark matter/energy, and quantum incompatibility. T0 solves these through fractal Time-Mass Duality.
	
	\subsection{Singularities and Information Loss}
	
	In GR, curvature diverges as \(R \propto 1/r^{4}\) (R: Ricci scalar in m$^{-2}$, r: radius in m). In T0, the effective Ricci scalar remains finite:
	\begin{equation}
		R_{\text{eff}} \leq \frac{c^4}{G \hbar} \cdot \xi^2,
	\end{equation}
	where:
	\begin{itemize}
		\item \(c\): Speed of light (\(3 \times 10^{8}\)~m/s),
		\item \(\hbar\): Reduced Planck constant (\(1.05 \times 10^{-34}\)~J\,s).
	\end{itemize}
	
	Validation: The maximum value is finite, avoids information loss, and is consistent with quantum information principles.
	
	\subsection{Dark Matter and Dark Energy}
	
	Both are explained by fractal modifications with \(\xi\), without unobserved components.
	
	\subsection{Quantum Incompatibility}
	
	T0 is UV-finite with only one parameter \(\xi\).
	
	\subsection{Conclusion}
	
	T0 provides a consistent quantum gravity without additional assumptions.
	
	\section{Reinterpretation of \(E = mc^2\) in T0-Time-Mass Duality}
	
	The equivalence emerges from the duality.
	
	\subsection{Derivation of Rest Energy}
	
	Rest mass is a stabilized time interval:
	\begin{equation}
		m = \frac{\hbar}{c^2} \cdot \frac{\Delta t}{T_0 \cdot \xi^k}, \quad E_0 = m c^2 = \frac{\hbar}{T_0} \cdot \xi^{-k}.
	\end{equation}
	where:
	\begin{itemize}
		\item \(m\): Mass (kg),
		\item \(\Delta t\): Time interval (s),
		\item \(T_0\): Fundamental time scale (s),
		\item \(k\): Hierarchy level (integer, dimensionless).
	\end{itemize}
	
	The derivation is based on fractal hierarchy and self-similarity; \(c\) emerges as maximum signal speed (\(3 \times 10^{8}\)~m/s).
	
	Validation: In the limit \(k=0\), reduces to classical rest energy, consistent with \(E=mc^2\) from special relativity.
	
	\subsection{Physical Interpretation}
	
	Mass is stored fractal time energy, explaining the universality of \(E = mc^2\).
	
	\subsection{Conclusion}
	
	No separate postulate needed – direct consequence of duality.
	
	\section{Derivation of Special Relativity from T0}
	
	Special relativity (SR) emerges from invariance of the fractal hierarchy.
	
	\subsection{Lorentz Transformations}
	
	Conservation of the scale function \(\mathcal{F}(x,t)\) leads to
	\begin{equation}
		x' = \gamma (x - v t), \quad t' = \gamma \left( t - \frac{v x}{c^2} \right), \quad \gamma = \left(1 - \frac{v^2}{c^2}\right)^{-1/2}.
	\end{equation}
	where:
	\begin{itemize}
		\item \(x, t\): Coordinates (m, s),
		\item \(v\): Relative velocity (m/s),
		\item \(\gamma\): Lorentz factor (dimensionless).
	\end{itemize}
	
	Validation: For \(v \ll c\), reduces to Galilean transformation, consistent with classical mechanics.
	
	\subsection{Conclusion}
	
	All relativistic effects are consequences of fractal invariance with \(\xi\).
	
	\section{Galaxy Rotation Curves and the Missing Mass Problem in T0}
	
	Flat rotation curves arise without dark matter.
	
	\subsection{Fractal Modification}
	
	The effective acceleration in the deep-field limit reads
	\begin{equation}
		a_{\text{eff}} = \sqrt{a_{\text{Newton}} \cdot a_\xi}, \quad a_\xi = \xi^{1/2} \frac{c^2}{l_0} \approx 1.2 \times 10^{-10} \, \text{m/s}^{2},
	\end{equation}
	where:
	\begin{itemize}
		\item \(a_{\text{eff}}\): Effective acceleration (m/s$^{2}$),
		\item \(a_{\text{Newton}}\): Newtonian acceleration (m/s$^{2}$),
		\item \(a_\xi\): Characteristic acceleration (m/s$^{2}$),
		\item \(l_0\): Characteristic length scale (m, derived from cosmological parameters).
	\end{itemize}
	
	Derived from the modified Poisson equation with fractal scale function.
	
	Validation: The value \(a_\xi \approx 1.2 \times 10^{-10}\)~m/s$^{2}$ matches the empirical \(a_0\) in Modified Newtonian Dynamics (MOND), known from observations of galaxy rotation curves.
	
	\subsection{Comparison with TeVeS}
	
	T0 is minimal and parameter-free unlike TeVeS.
	
	\subsection{Conclusion}
	
	Dark matter is superfluous – geometric effect from \(\xi\).
	
	\section{Strong, Weak, and Deep Field Regimes in T0}
	
	The regimes are defined by the interpolation function
	\begin{equation}
		\mu\left(\frac{a}{a_\xi}\right) = \left(1 + \left(\frac{a_\xi}{a}\right)^2\right)^{1/4}
	\end{equation}
	where:
	\begin{itemize}
		\item \(\mu\): Interpolation function (dimensionless),
		\item \(a\): Local acceleration (m/s$^{2}$).
	\end{itemize}
	
	Derived from fractal metric integration.
	
	Strong field: \(\mu \approx 1\) (GR), deep field: \(\mu \approx (a/a_\xi)^{-1/2}\).
	
	Validation: In the strong-field limit (\(a \gg a_\xi\)), reduces to Newtonian law, consistent with solar system observations.
	
	\subsection{Conclusion}
	
	The regimes follow fundamentally from \(\xi\).
	
	\section{Reinterpretation of Dark Energy in T0}
	
	Dark energy as residual fractal dynamics:
	\begin{equation}
		\rho_{\text{vac}} = \xi^2 \rho_{\text{crit}} \approx 0.7 \rho_c,
	\end{equation}
	where:
	\begin{itemize}
		\item \(\rho_{\text{vac}}\): Vacuum energy density (kg/m$^{3}$),
		\item \(\rho_{\text{crit}}\): Critical density (kg/m$^{3}$, \(3 H_0^2 / (8 \pi G)\)).
	\end{itemize}
	
	Slight time dependence explains Hubble tension.
	
	Validation: The factor 0.7 agrees with cosmological observations for \(\Omega_\Lambda\).
	
	\subsection{Conclusion}
	
	Unified with local gravitation through \(\xi\).
	
	\section{Internal Structure of Black Holes in T0}
	
	Modified Schwarzschild metric:
	\begin{equation}
		ds^2 = -\left(1 - \frac{2GM}{r}\right) dt^2 + \left(1 - \frac{2GM}{r}\right)^{-1} dr^2 \left(1 + \xi \Theta(r - r_\xi)\right) + r^2 d\Omega^2.
	\end{equation}
	where:
	\begin{itemize}
		\item \(ds^2\): Line element (m$^{2}$),
		\item \(M\): Mass (kg),
		\item \(\Theta\): Heaviside step function (dimensionless).
	\end{itemize}
	
	Finite core density, no singularity.
	
	Validation: Outside \(r_\xi\), reduces to Schwarzschild metric, consistent with gravitational wave observations (LIGO/Virgo).
	
	\subsection{Comparison with Loop Quantum Gravity and String Theory}
	
	T0 is 4-dimensional and parameter-free.
	
	\subsection{Conclusion}
	
	Simplest regularization through duality.
	
	\section{Testable Predictions and Observations}
	
	Modified black hole shadow:
	\begin{equation}
		\theta_{\text{shadow}} = \frac{3\sqrt{3}GM}{c^2 D} \left[1 + \frac{\kappa}{r_c^{D_f-2}}\right].
	\end{equation}
	where:
	\begin{itemize}
		\item \(\theta_{\text{shadow}}\): Angular radius (rad),
		\item \(D\): Distance (m),
		\item \(\kappa\): Correction constant (dimensionless),
		\item \(r_c\): Core radius (m).
	\end{itemize}
	
	Further predictions: echo chambers, modified quasi-normal modes, Hawking radiation modifications.
	
	Validation: The correction term is small (0.1–1\,\%), testable with future Event Horizon Telescope data.
	
	\subsection{Conclusion}
	
	Precise, testable deviations from general relativity.
	
	\section{Summary – Bridge between GR and QFT}
	
	DVFT with T0-Time-Mass Duality and fractal geometry unifies all fundamental phenomena from a single parameter \(\xi\). Black holes become windows into fractal spacetime structure, singularities and paradoxes are resolved, and the theory delivers parameter-free, testable predictions.
	
	Physics reaches a new level of harmony: everything emerges from the dynamic, fractal nature of the vacuum itself.
	
\begin{thebibliography}{9}
	\bibitem{mandelbrot} B. B. Mandelbrot, \textit{The Fractal Geometry of Nature}, W.H. Freeman, 1982
	\bibitem{calcagni} G. Calcagni, Fractal spacetime and quantum gravity, Phys. Rev. Lett. 104, 2010
	\bibitem{weinberg} S. Weinberg, \textit{Gravitation and Cosmology}, Wiley, 1972
	\bibitem{feinstruktur} Derivation of fine structure constant from parameter xi (see file T0 Feinstruktur.pdf in repository jpascher/T0-Time-Mass-Duality)
	\bibitem{unified} Unified derivation of all constants from parameter xi (see file T0 unified report.pdf in repository jpascher/T0-Time-Mass-Duality)
	\bibitem{korrektur} Mathematical proof of fractal correction Kfrak (see file 133 Fraktale Korrektur Herleitung.pdf in repository jpascher/T0-Time-Mass-Duality)
\end{thebibliography}

\end{document}
