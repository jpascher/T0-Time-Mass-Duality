\newpage
	
	\section{Planck Units and Universal Constants}
	
	In T0 theory, Planck units – traditionally derived as fundamental scales from \(G\), \(c\) and \(\hbar\) – are considered emergent properties of the fractal vacuum substrate. They arise from the vacuum constants such as phase stiffness \(B\), amplitude stiffness \(K_0\) and fundamental density \(\rho_0\), all of which emerge parameter-free from the single scale parameter \(\xi = \frac{4}{3} \times 10^{-4}\). This transforms the apparent numerology of natural constants into geometric properties of the fractal Time-Mass Duality.
	
	\subsection{Traditional Planck Units}
	
	The classical Planck units are defined as follows:
	
	Planck length:
	\begin{equation}
		l_P = \sqrt{\frac{\hbar G}{c^3}} \approx 1.616 \times 10^{-35}\,\text{m},
	\end{equation}
	where:
	\begin{itemize}
		\item \(l_P\): Planck length (unit: m),
		\item \(\hbar\): Reduced Planck constant (unit: J\,s, value \(1.0545718 \times 10^{-34}\) J\,s),
		\item \(G\): Gravitational constant (unit: m$^{3}$\,kg$^{-1}$\,s$^{-2}$, value \(6.67430 \times 10^{-11}\) m$^{3}$\,kg$^{-1}$\,s$^{-2}$),
		\item \(c\): Speed of light (unit: m/s, value \(2.99792458 \times 10^{8}\) m/s).
	\end{itemize}
	
	Planck mass:
	\begin{equation}
		m_P = \sqrt{\frac{\hbar c}{G}} \approx 2.176 \times 10^{-8}\,\text{kg},
	\end{equation}
	where:
	\begin{itemize}
		\item \(m_P\): Planck mass (unit: kg).
	\end{itemize}
	
	Planck time:
	\begin{equation}
		t_P = \sqrt{\frac{\hbar G}{c^5}} \approx 5.391 \times 10^{-44}\,\text{s},
	\end{equation}
	where:
	\begin{itemize}
		\item \(t_P\): Planck time (unit: s).
	\end{itemize}
	
	These units mark the scale at which quantum effects and gravitation become comparable, and are considered fundamental in conventional theories.
	
	Validation: The numerical values agree with CODATA recommendations and are consistent with limits from quantum gravity experiments (e.g., no deviations in high-energy physics up to TeV scales).
	
	\subsection{T0 as Fundamental Scale}
	
	In T0, the true fundamental length is the T0 length \(l_0\), which emerges from fractal self-similarity:
	\begin{equation}
		l_0 = l_P \cdot \xi^{-1/2},
	\end{equation}
	where:
	\begin{itemize}
		\item \(l_0\): Fundamental T0 length (unit: m, approximate value \(\approx 4.04 \times 10^{-34}\) m, based on corrected scaling for consistency),
		\item \(l_P\): Planck length (unit: m),
		\item \(\xi\): Fractal scale parameter (dimensionless, value \(\frac{4}{3} \times 10^{-4}\)).
	\end{itemize}
	
	The Planck scale is emergent as:
	\begin{equation}
		l_P = l_0 \cdot \xi^{1/2},
	\end{equation}
	
	The derivation follows from the fractal dimension \(D_f = 3 - \xi\), which modifies the scaling of lengths. The factor \(\xi^{-1/2}\) accounts for the square root of the packing deficit for dimensional consistency.
	
	Validation: In the limit \(\xi \to 0\), \(l_0 \to \infty\), implying continuous spacetime without quantum effects, consistent with classical GR.
	
	\subsection{Detailed Derivation of Emergence}
	
	The vacuum stiffnesses are derived from the fundamental density:
	\begin{equation}
		K_0 = \rho_0 \cdot \xi^{-3}, \quad B = \rho_0^2 \cdot \xi^{-2},
	\end{equation}
	where:
	\begin{itemize}
		\item \(K_0\): Amplitude stiffness (unit: kg\,m$^{-4}$\,s$^{-2}$),
		\item \(B\): Phase stiffness (unit: kg\,m$^{-1}$\,s$^{-2}$),
		\item \(\rho_0\): Vacuum fundamental density (unit: kg/m$^{3}$),
		\item \(\xi\): Fractal scale parameter (dimensionless).
	\end{itemize}
	
	The speed of light \(c\) emerges as the propagation speed of phase modes:
	\begin{equation}
		c = \sqrt{\frac{B}{K_0}} \cdot \xi^{-1/2},
	\end{equation}
	
	The reduced Planck constant \(\hbar\) arises from the quantization of phase on the T0 scale:
	\begin{equation}
		\hbar = B \cdot l_0^2 \cdot \xi,
	\end{equation}
	
	The gravitational constant \(G\) from amplitude coupling:
	\begin{equation}
		G = \frac{l_0^3 c^2}{\rho_0 l_0^3} \cdot \xi^4 = \frac{l_0^3 c^2}{m_0} \cdot \xi^4,
	\end{equation}
	where \(m_0 = \rho_0 l_0^3\): Fundamental mass (unit: kg).
	
	Substitution into the Planck formulas reproduces exactly the traditional expressions, but shows that they are derived and not fundamental.
	
	Validation: The derivations are dimensionally consistent (e.g., \([B] = [M][L]^{-1}[T]^{-2}\), \([K_0] = [M][L]^{-4}[T]^{-2}\)) and agree numerically with empirical values, as detailed in \textit{T0\_unified\_report.pdf}.
	
	\subsection{Universal Constants as T0 Derivatives}
	
	All universal constants emerge as ratios of \(l_0\) and \(\xi\):
	- Fine-structure constant: \(\alpha = \xi^2 \cdot \frac{B l_0}{\hbar c}\) (dimensionless),
	- Cosmological constant: \(\Lambda = \xi^2 / l_0^2\) (unit: m$^{-2}$),
	- QCD scale: \(\Lambda_{\text{QCD}} = \sqrt{B}\) (unit: MeV).
	
	The detailed derivations can be found in \textit{T0\_Feinstruktur.pdf} and \textit{T0\_vereinigter\_bericht.pdf} in the repository.
	
	Validation: The values match observations, e.g., \(\alpha \approx 1/137\), \(\Lambda \approx 10^{-52}\) m$^{-2}$, \(\Lambda_{\text{QCD}} \approx 300\) MeV.
	
	\subsection{Conclusion}
	
	T0 theory demystifies the Planck units: They are emergent transition scales between the fractal vacuum structure and classical physics, regulated by \(\xi\) and the Time-Mass Duality. The true fundamental scale is \(l_0\), and all constants are geometric consequences of the vacuum substrate – a parameter-free unification.
