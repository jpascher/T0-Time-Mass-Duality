
\chapter{Das T0-Energiefeld-Modell:\\[0.3cm]
	\large Mathematische Formulierung}

	\section{abstract}
		Das T0-Modell beschreibt physikalische Phänomene durch ein universelles Energiefeld $E_{\text{field}}(x,t)$ mit dem Parameter $\xi = \frac{4}{3} \times 10^{-4}$. Die Feldgleichung ist $\square E_{\text{field}} = 0$, die Lagrange-Dichte $\mathcal{L} = \xi (\partial E)^2$. Das Modell verwendet Standard natürliche Einheiten mit $\hbar = c = 1$.
		
		\textbf{Fundamentale Größen:}
		\begin{itemize}
			\item Charakteristische Energie: $E_0 = \sqrt{m_e \cdot m_\mu} = 7{,}348$ MeV
			\item Feinstrukturkonstante: $\alpha = \xi(E_0/1\,\text{MeV})^2 \approx 1/137$
			\item Gravitationskonstante: $G = \xi^2/(4m_e) \times$ Faktoren
		\end{itemize}
		
		\textbf{Vorhersagen:} Leptonmassen mit 2\% Genauigkeit, anomale magnetische Momente $a_\ell = \frac{\xi}{2\pi}(E_\ell/E_e)^2$, Feinstrukturkonstante mit 0,03\% Übereinstimmung.
		
		\textbf{Detaillierte Herleitungen:} Siehe Dokument 011 (Feinstruktur), 012 (Gravitation), 018 (g-2 geometrisch), 019 (Lagrangian).

	
	% ============================================================================
	\section{Einheitenkonvention}
	\label{sec:units}
	
	\subsection{Standard natürliche Einheiten}
	
	Dieses Dokument verwendet die Standard-Konvention der Teilchenphysik:
	
	\begin{equation}
		\hbar = c = 1
	\end{equation}
	
	In diesem System:
	\begin{itemize}
		\item Feinstrukturkonstante: $\alpha \approx \frac{1}{137,036} = 7,297 \times 10^{-3}$
		\item Energie = Masse: $E = m$
		\item Länge = Zeit = Energie$^{-1}$: $[L] = [T] = [E^{-1}]$
	\end{itemize}
	
	\textbf{Hinweis:} Alternative Heaviside-Lorentz-Einheiten ($4\pi\epsilon_0 = 1$, dann $\alpha = e^2 = 1$) führen zu denselben physikalischen Ergebnissen, nur mit anderer mathematischer Form.
	
	\subsection{Dimensionen in natürlichen Einheiten}
	
	\begin{align}
		[E] &= E \\
		[m] &= E \\
		[t] &= E^{-1} \\
		[L] &= E^{-1} \\
		[G] &= E^{-2} \\
		[\partial_\mu] &= E
	\end{align}
	
	% ============================================================================
	\section{Zeit-Energie-Dualität}
	\label{sec:duality}
	
	\subsection{Fundamentale Relation}
	
	\begin{equation}
		T_{\text{field}}(x,t) \cdot E_{\text{field}}(x,t) = 1
		\label{eq:duality}
	\end{equation}
	
	mit $[T_{\text{field}}] = E^{-1}$ und $[E_{\text{field}}] = E$.
	
	\subsection{Intrinsisches Zeitfeld}
	
	\begin{equation}
		T_{\text{field}}(x,t) = \frac{1}{E_{\text{field}}(x,t)}
	\end{equation}
	
	% ============================================================================
	\section{Universelle Feldgleichung}
	\label{sec:field_equation}
	
	\subsection{Wellengleichung}
	
	\begin{equation}
		\square E_{\text{field}} = 0
	\end{equation}
	
	mit d'Alembert-Operator:
	\begin{equation}
		\square = \nabla^2 - \frac{\partial^2}{\partial t^2}
	\end{equation}
	
	\subsection{Mit Quellen}
	
	\begin{equation}
		\nabla^2 E_{\text{field}} = 4\pi G \rho \cdot E_{\text{field}}
	\end{equation}
	
	Dimensionscheck: $[E^3] = [E^{-2}][E^4][E] = [E^3]$ ✓
	
	% ============================================================================
	\section{Lagrange-Dichte}
	\label{sec:lagrangian}
	
	\subsection{Universelle Lagrange-Dichte}
	
	\begin{equation}
		\mathcal{L} = \xi \cdot (\partial_\mu E_{\text{field}})(\partial^\mu E_{\text{field}})
	\end{equation}
	
	mit $\xi = \frac{4}{3} \times 10^{-4}$.
	
	\subsection{Euler-Lagrange-Gleichung}
	
	\begin{equation}
		\frac{\partial \mathcal{L}}{\partial E} - \partial_\mu \frac{\partial \mathcal{L}}{\partial(\partial_\mu E)} = 0
	\end{equation}
	
	ergibt:
	\begin{equation}
		\square E_{\text{field}} = 0
	\end{equation}
	
	% ============================================================================
	\section{Charakteristische Längen}
	\label{sec:characteristic_lengths}
	
	\subsection{T0-charakteristische Länge}
	
	\begin{equation}
		r_0 = 2GE
	\end{equation}
	
	Dimension: $[r_0] = [E^{-2}][E] = [E^{-1}] = [L]$ ✓
	
	\subsection{Herleitung}
	
	Für sphärisch symmetrische Punktquelle $\rho(r) = E_0 \delta^3(\vec{r})$:
	
	Lösung von $\nabla^2 E = 4\pi G \rho E$:
	\begin{equation}
		E(r) = E_0 \left(1 - \frac{r_0}{r}\right)
	\end{equation}
	
	mit $r_0 = 2GE_0$.
	
	\subsection{Zeitskala}
	
	\begin{equation}
		t_0 = \frac{r_0}{c} = r_0 = 2GE
	\end{equation}
	
	(da $c = 1$)
	
	% ============================================================================
	\section{Charakteristische Energie}
	\label{sec:characteristic_energy}
	
	\subsection{Definition}
	
	Die charakteristische Energie $E_0$ ist das geometrische Mittel der Elektron- und Myonmasse (Herleitung in Dokument 011):
	
	\begin{equation}
		\boxed{E_0 = \sqrt{m_e \cdot m_\mu}}
		\label{eq:E0_definition}
	\end{equation}
	
	\subsection{Numerische Werte}
	
	Aus experimentellen Massen:
	\begin{align}
		E_0 &= \sqrt{0{,}511 \times 105{,}66} \\
		&= \sqrt{53{,}99} \\
		&= 7{,}348 \text{ MeV}
	\end{align}
	
	Theoretischer T0-Wert:
	\begin{equation}
		E_0^{\text{T0}} = 7{,}398 \text{ MeV}
	\end{equation}
	
	Abweichung: 0,7\% (im Rahmen geometrischer Korrekturen)
	
	\subsection{Verwendung}
	
	$E_0$ dient als Energieskala für:
	\begin{itemize}
		\item Feinstrukturkonstante: $\alpha = \xi (E_0/1\,\text{MeV})^2$
		\item Normierung elektromagnetischer Effekte
		\item Skalierung anomaler magnetischer Momente
	\end{itemize}
	
	% ============================================================================
	\section{Der Parameter $\xi$}
	\label{sec:xi_parameter}
	
	\subsection{Definition}
	
	\begin{equation}
		\boxed{\xi = \frac{4}{3} \times 10^{-4} = 1{,}3333 \times 10^{-4}}
	\end{equation}
	
	Dimensionslos: $[\xi] = 1$.
	
	\subsection{Geometrische Komponenten}
	
	\begin{equation}
		\xi = G_3 \times S_{\text{ratio}}
	\end{equation}
	
	wobei:
	\begin{itemize}
		\item $G_3 = \frac{4}{3}$: Geometrischer Faktor (Kugel-Würfel-Verhältnis)
		\item $S_{\text{ratio}} = 10^{-4}$: Skalenverhältnis
	\end{itemize}
	
	% ============================================================================
	\section{Skalenhierarchie}
	\label{sec:scale_hierarchy}
	
	\subsection{Planck-Länge als Referenz}
	
	\begin{equation}
		\ell_P = \sqrt{G} = 1 \quad \text{(in nat. Einheiten)}
	\end{equation}
	
	\subsection{Skalenverhältnis}
	
	\begin{equation}
		\xi_{\text{ratio}} = \frac{\ell_P}{r_0} = \frac{\sqrt{G}}{2GE} = \frac{1}{2\sqrt{G} \cdot E}
	\end{equation}
	
	Für $E \sim 1$ GeV:
	\begin{equation}
		\frac{r_0}{\ell_P} \sim 10^7 \quad \text{(sub-Planck)}
	\end{equation}
	
	% ============================================================================
	\section{Teilchen als Feldanregungen}
	\label{sec:particles}
	
	\subsection{Klassifikation nach Energie}
	
	\begin{table}[h]
		\centering
		\begin{tabular}{lc}
			\hline
			\textbf{Teilchen} & \textbf{Energie [MeV]} \\
			\hline
			Elektron & 0,511 \\
			Myon & 105,658 \\
			Tau & 1776,86 \\
			\hline
		\end{tabular}
	\end{table}
	
	\subsection{Antiteilchen}
	
	Negative Feldanregungen: $E_{\text{field}} < 0$
	
	% ============================================================================
	\section{Feinstrukturkonstante}
	\label{sec:fine_structure}
	
	\subsection{T0-Herleitung}
	
	Die Feinstrukturkonstante folgt aus $\xi$ und $E_0$ (Herleitung in Dokument 011):
	
	\begin{equation}
		\boxed{\alpha = \xi \cdot \left(\frac{E_0}{1\,\text{MeV}}\right)^2}
		\label{eq:alpha_derivation}
	\end{equation}
	
	\subsection{Numerische Berechnung}
	
	Mit $\xi = \frac{4}{3} \times 10^{-4}$ und $E_0 = 7{,}398$ MeV:
	
	\begin{align}
		\alpha &= 1{,}3333 \times 10^{-4} \times (7{,}398)^2 \\
		&= 1{,}3333 \times 10^{-4} \times 54{,}73 \\
		&= 7{,}297 \times 10^{-3} \\
		&= \frac{1}{137{,}04}
	\end{align}
	
	Experimentell: $\alpha_{\text{exp}} = \frac{1}{137{,}036}$
	
	Übereinstimmung: 0,03\%
	
	\subsection{Dimensionscheck}
	
	\begin{equation}
		[\alpha] = [\xi] \times \left[\frac{E}{E}\right]^2 = 1 \times 1 = 1 \quad \checkmark
	\end{equation}
	
	% ============================================================================
	\section{Gravitationskonstante}
	\label{sec:gravitational_constant}
	
	\subsection{T0-Formel}
	
	Die Gravitationskonstante wird aus $\xi$ und $m_e$ hergeleitet (Herleitung in Dokument 012):
	
	\begin{equation}
		G = \frac{\xi^2}{4m_e} \times C_{\text{dim}} \times C_{\text{conv}}
		\label{eq:G_formula}
	\end{equation}
	
	wobei:
	\begin{itemize}
		\item $C_{\text{dim}}$: Dimensionskorrektur
		\item $C_{\text{conv}}$: SI-Umrechnungsfaktor
	\end{itemize}
	
	\subsection{Fundamentale Beziehung}
	
	In natürlichen Einheiten:
	\begin{equation}
		\xi = 2\sqrt{G \cdot m_e}
	\end{equation}
	
	Aufgelöst nach $G$:
	\begin{equation}
		G_{\text{nat}} = \frac{\xi^2}{4m_e}
	\end{equation}
	
	Dimension: $[G] = [E^{-2}]$ in natürlichen Einheiten.
	
	% ============================================================================
	\section{Leptonmassen}
	\label{sec:lepton_masses}
	
	Das T0-Modell sagt Leptonmassen voraus (Herleitung in Dokument 003):
	
	\begin{table}[h]
		\centering
		\begin{tabular}{lccc}
			\hline
			\textbf{Lepton} & \textbf{T0 [MeV]} & \textbf{Exp [MeV]} & \textbf{Δ [\%]} \\
			\hline
			Elektron & 0,507 & 0,511 & 0,87 \\
			Myon & 103,5 & 105,7 & 2,09 \\
			Tau & 1815 & 1777 & 2,16 \\
			\hline
		\end{tabular}
	\end{table}
	
	% ============================================================================
	\section{Anomale magnetische Momente}
	\label{sec:g2}
	
	\subsection{Definition}
	
	Magnetisches Moment:
	\begin{equation}
		\mu = g \cdot \frac{e}{2m} \cdot \frac{\hbar}{2}
	\end{equation}
	
	Anomales magnetisches Moment:
	\begin{equation}
		a = \frac{g-2}{2}
	\end{equation}
	
	\subsection{T0-Vorhersageformel}
	
	\begin{equation}
		\boxed{a_\ell = \frac{\xi}{2\pi} \left(\frac{E_\ell}{E_e}\right)^2}
		\label{eq:g2_formula}
	\end{equation}
	
	\subsection{Myon}
	
	\begin{align}
		\frac{E_\mu}{E_e} &= \frac{105{,}658}{0{,}511} = 206{,}768 \\
		a_\mu &= \frac{1{,}3333 \times 10^{-4}}{2\pi} \times (206{,}768)^2 \\
		&= 2{,}122 \times 10^{-5} \times 42\,753 \\
		&= 1{,}166 \times 10^{-3}
	\end{align}
	
	\subsection{Elektron}
	
	\begin{equation}
		a_e = \frac{\xi}{2\pi} = 2{,}122 \times 10^{-5}
	\end{equation}
	
	\subsection{Tau}
	
	\begin{equation}
		a_\tau = \frac{\xi}{2\pi} \left(\frac{1776{,}86}{0{,}511}\right)^2 = 1{,}28 \times 10^{-3}
	\end{equation}
	
	% ============================================================================
	\section{Drei Feldgeometrien}
	\label{sec:field_geometries}
	
	\subsection{Typ 1: Lokalisiert sphärisch}
	
	\begin{equation}
		E(r) = E_0 \left(1 - \frac{\beta}{r}\right), \quad \beta = r_0
	\end{equation}
	
	Anwendung: Einzelteilchen (Elektron, Myon, Tau)
	
	\subsection{Typ 2: Lokalisiert nicht-sphärisch}
	
	\begin{equation}
		E(\vec{r}) = E_0 \left(1 - \frac{\beta_{ij} r_i r_j}{r^3}\right)
	\end{equation}
	
	Anwendung: Verbundene Systeme
	
	\subsection{Typ 3: Ausgedehnt homogen}
	
	Effektiver Parameter:
	\begin{equation}
		\xi_{\text{eff}} = \frac{\xi}{2} = \frac{2}{3} \times 10^{-4}
	\end{equation}
	
	Anwendung: Kosmologie (siehe Dokument 026)
	
	% ============================================================================
	\section{Mathematische Identitäten}
	\label{sec:identities}
	
	\subsection{Energiefeld-Normierung}
	
	\begin{equation}
		E_{\text{field}}(\vec{r}, t) = E_0 \cdot f(\vec{r}, t) \cdot e^{i\phi(\vec{r}, t)}
	\end{equation}
	
	mit:
	\begin{itemize}
		\item $E_0$: Charakteristische Energie
		\item $f(\vec{r}, t)$: Normiertes Profil
		\item $\phi(\vec{r}, t)$: Phase
	\end{itemize}
	
	\subsection{Dualitäts-Konsistenz}
	
	Zeit-Masse (Dokument 003): $T \cdot m = 1$
	
	Zeit-Energie (dieses Dokument): $T \cdot E = 1$
	
	In natürlichen Einheiten ($c = 1$):
	\begin{equation}
		E = mc^2 = m \quad \Rightarrow \quad T \cdot m = T \cdot E
	\end{equation}
	
	% ============================================================================
	\section{Dimensionsanalyse-Verifikationen}
	\label{sec:dimensional_analysis}
	
	\subsection{Feldgleichung}
	
	\begin{align}
		[\nabla^2 E] &= [L^{-2}][E] = [E^2][E] = [E^3] \\
		[4\pi G \rho E] &= [E^{-2}][E^4][E] = [E^3] \quad \checkmark
	\end{align}
	
	\subsection{Charakteristische Länge}
	
	\begin{equation}
		[r_0] = [2GE] = [E^{-2}][E] = [E^{-1}] = [L] \quad \checkmark
	\end{equation}
	
	\subsection{Lagrange-Dichte}
	
	\begin{equation}
		[\mathcal{L}] = [\xi][(\partial E)^2] = [1][E^2] = [E^2] \quad \text{(korrekt für Lagrange-Dichte)}
	\end{equation}
	
	\subsection{Anomales magnetisches Moment}
	
	\begin{equation}
		[a_\ell] = [\xi]\left[\frac{E^2}{E^2}\right] = [1][1] = [1] \quad \checkmark
	\end{equation}
	
	% ============================================================================
	\section{Formeln-Referenz}
	\label{sec:formula_reference}
	
	\subsection{Fundamentale Gleichungen}
	
	\begin{align}
		\text{Dualität:} \quad & T_{\text{field}} \cdot E_{\text{field}} = 1 \\
		\text{Wellengleichung:} \quad & \square E_{\text{field}} = 0 \\
		\text{Mit Quellen:} \quad & \nabla^2 E = 4\pi G \rho E \\
		\text{Lagrange-Dichte:} \quad & \mathcal{L} = \xi (\partial E)^2
	\end{align}
	
	\subsection{Abgeleitete Konstanten}
	
	\begin{align}
		\text{Charakteristische Energie:} \quad & E_0 = \sqrt{m_e \cdot m_\mu} = 7{,}348 \text{ MeV} \\
		\text{Feinstrukturkonstante:} \quad & \alpha = \xi (E_0/1\,\text{MeV})^2 \approx 1/137 \\
		\text{Gravitationskonstante:} \quad & G = \frac{\xi^2}{4m_e} \times \text{Faktoren}
	\end{align}
	
	\subsection{Charakteristische Skalen}
	
	\begin{align}
		\text{T0-Länge:} \quad & r_0 = 2GE \\
		\text{T0-Zeit:} \quad & t_0 = 2GE \\
		\text{Planck-Länge:} \quad & \ell_P = \sqrt{G} = 1 \\
		\text{Skalenverhältnis:} \quad & \xi_{\text{ratio}} = \frac{1}{2\sqrt{G} E}
	\end{align}
	
	\subsection{Vorhersageformeln}
	
	\begin{align}
		\text{g-2 Formel:} \quad & a_\ell = \frac{\xi}{2\pi} \left(\frac{E_\ell}{E_e}\right)^2 \\
		\text{Parameter:} \quad & \xi = \frac{4}{3} \times 10^{-4} \\
		\text{Effektiver Parameter:} \quad & \xi_{\text{eff}} = \frac{\xi}{2}
	\end{align}
	
	% ============================================================================
	\section{Numerische Werte}
	\label{sec:numerical_values}
	
	\subsection{Fundamentale Konstanten (in natürlichen Einheiten)}
	
	\begin{align}
		\hbar &= 1 \\
		c &= 1 \\
		\alpha &= \frac{1}{137{,}036} \approx 7{,}297 \times 10^{-3} \\
		G &= 1 \text{ (numerisch, Dimension } [E^{-2}]\text{)}
	\end{align}
	
	\subsection{T0-Parameter}
	
	\begin{align}
		\xi &= \frac{4}{3} \times 10^{-4} = 1{,}3333 \times 10^{-4} \\
		\xi^2 &= 1{,}7778 \times 10^{-8} \\
		\frac{\xi}{2\pi} &= 2{,}1221 \times 10^{-5} \\
		\xi_{\text{eff}} &= 6{,}6667 \times 10^{-5} \\
		E_0 &= 7{,}348 \text{ MeV (aus exp. Massen)} \\
		E_0^{\text{T0}} &= 7{,}398 \text{ MeV (theoretisch)}
	\end{align}
	
	\subsection{Leptonenergieen}
	
	\begin{align}
		E_e &= 0{,}511 \text{ MeV} \\
		E_\mu &= 105{,}658 \text{ MeV} \\
		E_\tau &= 1776{,}86 \text{ MeV}
	\end{align}
	
	\subsection{Energieverhältnisse}
	
	\begin{align}
		\frac{E_\mu}{E_e} &= 206{,}768 \\
		\frac{E_\tau}{E_e} &= 3477{,}2 \\
		\frac{E_\tau}{E_\mu} &= 16{,}817
	\end{align}
	
	% ============================================================================
	\section{Berechnungsbeispiele}
	\label{sec:calculations}
	
	\subsection{Myon g-2}
	
	Gegeben:
	\begin{itemize}
		\item $\xi = 1{,}3333 \times 10^{-4}$
		\item $E_\mu = 105{,}658$ MeV
		\item $E_e = 0{,}511$ MeV
	\end{itemize}
	
	Berechnung:
	\begin{align}
		\frac{E_\mu}{E_e} &= \frac{105{,}658}{0{,}511} = 206{,}768 \\
		\left(\frac{E_\mu}{E_e}\right)^2 &= 42\,753{,}3 \\
		\frac{\xi}{2\pi} &= \frac{1{,}3333 \times 10^{-4}}{6{,}2832} = 2{,}1221 \times 10^{-5} \\
		a_\mu &= 2{,}1221 \times 10^{-5} \times 42\,753{,}3 \\
		&= 1{,}1659 \times 10^{-3}
	\end{align}
	
	\subsection{Feinstrukturkonstante}
	
	Gegeben:
	\begin{itemize}
		\item $\xi = 1{,}3333 \times 10^{-4}$
		\item $E_0 = 7{,}398$ MeV
	\end{itemize}
	
	Berechnung:
	\begin{align}
		\left(\frac{E_0}{1\,\text{MeV}}\right)^2 &= (7{,}398)^2 = 54{,}73 \\
		\alpha &= 1{,}3333 \times 10^{-4} \times 54{,}73 \\
		&= 7{,}297 \times 10^{-3} \\
		&= \frac{1}{137{,}04}
	\end{align}
	
	Experimentell: $\alpha_{\text{exp}} = \frac{1}{137{,}036}$
	
	Abweichung: 0,03\%
	
	\subsection{Charakteristische Länge (Elektron)}
	
	Gegeben:
	\begin{itemize}
		\item $E_e = 0{,}511$ MeV $= 0{,}511 \times 1{,}6 \times 10^{-13}$ J $= 8{,}2 \times 10^{-14}$ J
		\item $G = 6{,}674 \times 10^{-11}$ m$^3$ kg$^{-1}$ s$^{-2}$
		\item $c = 3 \times 10^8$ m/s
	\end{itemize}
	
	Umrechnung in natürliche Einheiten:
	\begin{equation}
		r_0 = 2GE \approx 10^{-28} \text{ m}
	\end{equation}
	
	Planck-Vergleich:
	\begin{equation}
		\frac{r_0}{\ell_P} = \frac{10^{-28}}{1{,}6 \times 10^{-35}} \approx 10^7
	\end{equation}
	
	% ============================================================================
	% APPENDIX
	% ============================================================================
	
	\appendix
	
	\section{Symbolverzeichnis}
	
	\begin{longtable}{|c|l|c|}
		\hline
		\textbf{Symbol} & \textbf{Bedeutung} & \textbf{Dimension} \\
		\hline
		$\xi$ & Fundamentaler Parameter & $1$ \\
		$E_0$ & Charakteristische Energie & $E$ \\
		$E_{\text{field}}$ & Universelles Energiefeld & $E$ \\
		$T_{\text{field}}$ & Intrinsisches Zeitfeld & $E^{-1}$ \\
		$r_0$ & T0-charakteristische Länge & $L = E^{-1}$ \\
		$t_0$ & T0-charakteristische Zeit & $T = E^{-1}$ \\
		$\ell_P$ & Planck-Länge & $L = E^{-1}$ \\
		$G$ & Gravitationskonstante & $E^{-2}$ \\
		$\alpha$ & Feinstrukturkonstante & $1$ \\
		$a_\ell$ & Anomales magnetisches Moment & $1$ \\
		$E_e, E_\mu, E_\tau$ & Leptonenergieen & $E$ \\
		$m_e, m_\mu, m_\tau$ & Leptonmassen ($= E$ in nat. Einh.) & $E$ \\
		$\mathcal{L}$ & Lagrange-Dichte & $E^4$ \\
		$\square$ & d'Alembert-Operator & $E^2$ \\
		$\xi_{\text{eff}}$ & Effektiver Parameter ($\xi/2$) & $1$ \\
		\hline
	\end{longtable}
	
	\section{Einheiten-Umrechnungen}
	
	\subsection{Natürliche → SI}
	
	\begin{align}
		1 \text{ (Energie)} &= 1 \text{ GeV} = 1{,}6 \times 10^{-10} \text{ J} \\
		1 \text{ (Länge)} &= \frac{\hbar c}{1 \text{ GeV}} = 0{,}197 \text{ fm} \\
		1 \text{ (Zeit)} &= \frac{\hbar}{1 \text{ GeV}} = 6{,}58 \times 10^{-25} \text{ s}
	\end{align}
	
	\subsection{Standard natürliche Einheiten}
	
	In Standard-Konvention ($\hbar = c = 1$):
	\begin{itemize}
		\item $\alpha = \frac{e^2}{4\pi\epsilon_0} \approx \frac{1}{137}$ (dimensionslos)
		\item Alle Größen in Potenzen von Energie
		\item Physikalische Vorhersagen identisch zu anderen Konventionen
	\end{itemize}
	
	\section{Beziehung zu anderen Dokumenten}
	
	\begin{itemize}
		\item \textbf{Dokument 003}: Zeit-Masse-Dualität, Grundlagen, Ursprung von $\xi$
		\item \textbf{Dokument 018}: Geometrische g-2-Formulierung (fraktale Geometrie)
		\item \textbf{Dokument 019}: Lagrangian-Formulierung (Quantenfeldtheorie)
		\item \textbf{Dokument 026}: Kosmologie ($\xi_{\text{eff}} = \xi/2$)
	\end{itemize}
	
	Alle Formulierungen basieren auf $\xi = \frac{4}{3} \times 10^{-4}$.
