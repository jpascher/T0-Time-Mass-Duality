% Chapter file: 075_RSA_De_ch.tex
% Source: 075_RSA_De.tex

\chapter{Mathematical Analysis of the T0-Shor Algorithm: Theoretical Framework and Computational Complexity A Rigorous Investigation of the T0 Energy Field Approach to Integer Factorization}
\let\cleardoublepage\clearpage  % Removes blank page before this chapter

\section*{Abstract}
This work presents a mathematical analysis of the T0-Shor Algorithm based on an energy field formulation. We examine the theoretical foundations of the time-mass duality $T(x,t) \cdot m(x,t) = 1$ and its application to integer factorization. The analysis encompasses field equations, wave-like behavior similar to acoustic propagation, and material-dependent parameters derived from vacuum physics. We derive scaling relations for various spatial dimensions and investigate the role of computational accuracy for algorithm performance. The mathematical framework is checked for consistency, and practical limitations are identified.


\section{Introduction}

The T0-Shor Algorithm represents a theoretical extension of Shor's factorization algorithm, based on energy field dynamics instead of quantum mechanical superposition. This work examines the mathematical foundations of this approach without making claims about practical implementability or superiority over existing methods.

\subsection{Theoretical Framework}

The T0 model introduces the following fundamental mathematical structures:

\begin{align}
	\text{Time-Mass Duality}: \quad &T(x,t) \cdot m(x,t) = 1 \label{eq:duality}\\
	\text{Field Equation}: \quad &\nabla^2 T(x) = -\frac{\rho(x)}{T(x)^2} \label{eq:field}\\
	\text{Energy Evolution}: \quad &\frac{\partial^2 E}{\partial t^2} = -\omega^2 E \label{eq:evolution}
\end{align}

The coupling parameter $\xipar$ is theoretically derived from Higgs field interactions:
\begin{equation}
	\xipar = g_H \cdot \frac{\langle\phi\rangle}{v_{EW}} \label{eq:xi_higgs}
\end{equation}
where $g_H$ is the Higgs coupling constant, $\langle\phi\rangle$ is the vacuum expectation value, and $v_{EW} = 246$ GeV is the electroweak scale.

\section{Mathematical Foundations}

\subsection{Wave-like Behavior of T0 Fields}

The T0 field exhibits wave-like propagation characteristics analogous to acoustic waves in media. The fundamental wave equation for T0 fields is:

\begin{equation}
	\nabla^2 T - \frac{1}{c_{T0}^2} \frac{\partial^2 T}{\partial t^2} = -\frac{\rho(x,t)}{T(x,t)^2} \label{eq:wave_equation}
\end{equation}

where $c_{T0}$ is the T0 field propagation velocity in the medium, analogous to the speed of sound.

\subsection{Medium-Dependent Properties}

Similar to acoustic waves, T0 field propagation critically depends on medium properties:

\textbf{T0 Field Velocity in Different Media}:
\begin{align}
	c_{T0,vacuum} &= c \sqrt{\frac{\xipar_0}{\xipar_{vacuum}}} \\
	c_{T0,metal} &= c \sqrt{\frac{\xipar_0 \epsilon_r}{\xipar_{vacuum}}} \\
	c_{T0,dielectric} &= \frac{c}{\sqrt{\epsilon_r \mu_r}} \sqrt{\frac{\xipar_0}{\xipar_{vacuum}}} \\
	c_{T0,plasma} &= c \sqrt{1 - \frac{\omega_p^2}{\omega^2}} \sqrt{\frac{\xipar_0}{\xipar_{vacuum}}}
\end{align}

where $\omega_p$ is the plasma frequency, and $\epsilon_r$, $\mu_r$ are the relative permittivity and permeability.

\subsection{Boundary Conditions and Reflections}

At interfaces between different media, T0 fields satisfy boundary conditions similar to electromagnetic waves:

\textbf{Continuity Conditions}:
\begin{align}
	T_1|_{interface} &= T_2|_{interface} \quad \text{(Field continuity)} \\
	\frac{1}{m_1} \frac{\partial T_1}{\partial n}\bigg|_{interface} &= \frac{1}{m_2} \frac{\partial T_2}{\partial n}\bigg|_{interface} \quad \text{(Flux continuity)}
\end{align}

\textbf{Reflection and Transmission Coefficients}:
\begin{align}
	r &= \frac{Z_1 - Z_2}{Z_1 + Z_2} \quad \text{(Reflection coefficient)} \\
	t &= \frac{2Z_1}{Z_1 + Z_2} \quad \text{(Transmission coefficient)}
\end{align}

where $Z_i = \sqrt{m_i/T_i}$ is the T0 field impedance in medium $i$.

\subsection{Hyperbolic Geometry in Duality Space}

The time-mass duality (Eq.~\ref{eq:duality}) defines a hyperbolic metric in the $(T,m)$ parameter space:

\begin{equation}
	ds^2 = \frac{dT \cdot dm}{T \cdot m} = \frac{d(\ln T) \cdot d(\ln m)}{T \cdot m}
\end{equation}

This geometry is characterized by:
\begin{itemize}
	\item Constant negative curvature: $K = -1$
	\item Invariant measure: $d\mu = \frac{dT \, dm}{T \cdot m}$
	\item Isometry group: $PSL(2,\mathbb{R})$
\end{itemize}

\subsection{Atomic-scale T0 Field Parameters}

Since vacuum conditions are known, atomic T0 field behavior can be derived from fundamental constants:

\textbf{Vacuum T0 Field Baseline}:
\begin{align}
	c_{T0,vacuum} &= c = 2.998 \times 10^8 \text{ m/s} \\
	\xipar_{vacuum} &= \xipar_0 = \frac{g_H \langle\phi\rangle}{v_{EW}} \\
	Z_{vacuum} &= Z_0 = \sqrt{\frac{\mu_0}{\epsilon_0}} = 376.73 \text{ $\Omega$}
\end{align}

\textbf{Atomic-scale Derivations}:

For the hydrogen atom (fundamental case):
\begin{align}
	a_0 &= \frac{4\pi\epsilon_0\hbar^2}{m_e e^2} = 5.292 \times 10^{-11} \text{ m} \quad \text{(Bohr radius)} \\
	\alpha &= \frac{e^2}{4\pi\epsilon_0\hbar c} = 7.297 \times 10^{-3} \quad \text{(Fine-structure constant)} \\
	r_{e} &= \frac{e^2}{4\pi\epsilon_0 m_e c^2} = 2.818 \times 10^{-15} \text{ m} \quad \text{(Classical electron radius)}
\end{align}

\textbf{T0 Field Atomic Parameters}:
\begin{align}
	c_{T0,atom} &= c \cdot \alpha = 2.19 \times 10^6 \text{ m/s} \\
	\xipar_{atom} &= \xipar_0 \cdot \frac{E_{Rydberg}}{m_e c^2} = \xipar_0 \cdot \frac{\alpha^2}{2} \\
	\lambda_{T0,atom} &= \frac{2\pi a_0}{\alpha} = 2.426 \times 10^{-9} \text{ m}
\end{align}

\textbf{Scaling for Different Atoms}:

For an atom with nuclear charge $Z$ and mass number $A$:
\begin{align}
	c_{T0,Z} &= c_{T0,atom} \cdot Z^{2/3} \quad \text{(Velocity scaling)} \\
	\xipar_{Z} &= \xipar_{atom} \cdot \frac{Z^4}{A} \quad \text{(Coupling scaling)} \\
	a_{Z} &= \frac{a_0}{Z} \quad \text{(Size scaling)} \\
	E_{binding,Z} &= 13.6 \text{ eV} \cdot Z^2 \quad \text{(Energy scaling)}
\end{align}

\section{T0-Shor Algorithm Formulation}

\subsection{Geometric Cavity Design for Period Finding}

The T0-Shor Algorithm uses geometric resonance cavities for period detection, analogous to acoustic resonators:

\textbf{Resonance Cavity Dimensions} for period $r$:
\begin{equation}
	L_{cavity} = n \cdot \frac{\lambda_{T0}}{2} = n \cdot \frac{c_{T0} \cdot r}{2f_0}
\end{equation}

where $f_0$ is the fundamental drive frequency and $n$ is the mode number.

\textbf{Quality Factor} of the resonance:
\begin{equation}
	Q = \frac{f_r}{\Delta f} = \frac{\pi}{\xipar} \cdot \frac{L_{cavity}}{\lambda_{T0}}
\end{equation}

Higher $Q$ values provide sharper period detection but require longer observation times.

\subsection{Multi-Mode Resonance Analysis}

Instead of the Quantum Fourier Transform, the T0-Shor Algorithm uses multi-mode cavity analysis:

\begin{align}
	\text{Mode Spectrum}: \quad &T(x,y,z,t) = \sum_{mnp} A_{mnp}(t) \psi_{mnp}(x,y,z) \\
	\text{Period Detection}: \quad &r = \frac{c_{T0}}{2f_{resonance}} \cdot \frac{geometry\_factor}{mode\_number}
\end{align}

\section{Self-Amplifying $\xipar$ Optimization: The Error Reduction Feedback Loop}

\subsection{The Fundamental Discovery: Computational Errors Degrade $\xipar$}

A critical insight emerges: Computational accuracy directly influences $\xipar$ parameter values and creates a self-amplifying optimization cycle:

\textbf{Error-Dependent $\xipar$ Degradation}:
\begin{equation}
	\xipar_{effective} = \xipar_{ideal} \cdot \exp\left(-\alpha \sum_{i} p_{error,i} \cdot w_i\right)
\end{equation}

where $p_{error,i}$ are error probabilities and $w_i$ are criticality weights.
\textbf{The Self-Amplifying Relationship}:
\begin{equation}
	\begin{split}
		&\text{Fewer Errors} \rightarrow \text{Higher } \xipar \\
		&\quad \rightarrow \text{Better Field Coherence} \rightarrow \text{Even Fewer Errors}
	\end{split}
\end{equation}

\subsection{Mathematical Model of the Feedback Loop}

\textbf{Differential Equation for $\xipar$ Evolution}:
\begin{equation}
	\frac{d\xipar}{dt} = \beta \xipar \left(1 - \frac{R_{error}}{R_{threshold}}\right) - \gamma \xipar \frac{R_{error}}{R_{reference}}
\end{equation}

Critical insight: If $R_{error} < R_{threshold}$, $\xipar$ grows exponentially.

\textbf{Typical Threshold Values}:
\begin{align}
	R_{critical} &\approx 10^{-12} \text{ errors per operation} \\
	R_{64bit} &\approx 10^{-16} \text{ (already below threshold)} \\
	R_{32bit} &\approx 10^{-7} \text{ (above threshold)}
\end{align}

Standard 64-bit arithmetic is already in the $\xipar$ amplification range.

\section{Vacuum-derived Atomic Parameters: No Free Parameters}

\subsection{Fundamental Parameter Derivation}

Since vacuum conditions are known, all atomic T0 parameters can be derived from fundamental constants:

\textbf{Vacuum Baseline}:
\begin{align}
	c_{T0,vacuum} &= c = 2.998 \times 10^8 \text{ m/s} \\
	\xipar_{vacuum} &= \xipar_0 = \frac{g_H \langle\phi\rangle}{v_{EW}} \quad \text{(Higgs-derived)} \\
	Z_{vacuum} &= Z_0 = 376.73 \text{ $\Omega$}
\end{align}

\textbf{Material-Specific Predictions}:

No free parameters - all $\xipar$ values are calculable:
\begin{align}
	\xipar_{Si} &= \xipar_0 \cdot 0.98 \cdot \frac{E_g}{k_B T} = 43.7 \xipar_0 \quad \text{(at 300K)} \\
	\xipar_{metal} &= \xipar_0 \sqrt{\frac{n e^2}{\epsilon_0 m_e \omega^2}} \approx (10^{-4} \text{ to } 10^{-3}) \xipar_0 \\
	\xipar_{SC} &= \xipar_0 \cdot \frac{\Delta}{k_B T_c} \cdot \tanh\left(\frac{\Delta}{2k_B T}\right)
\end{align}

\textbf{Experimentally Testable Predictions}:
\begin{align}
	\text{Temperature Scaling}: \quad &\xipar(T_2)/\xipar(T_1) = T_1/T_2 \\
	\text{Isotope Effect}: \quad &\xipar(^{13}C)/\xipar(^{12}C) = \sqrt{12/13} = 0.962 \\
	\text{Pressure Dependence}: \quad &\xipar(p) = \xipar_0 \left(1 + \kappa \frac{\Delta p}{p_0}\right)
\end{align}

\section{$\xipar$ as a Multifunctional Parameter: Beyond Simple Coupling}

\subsection{Multiple Hidden Functions of $\xipar$}

$\xipar$ fulfills several fundamental roles beyond simple field-matter coupling:

\begin{align}
	\text{1. Coupling Strength}: \quad &\xipar_{coupling} = \text{Field-Matter Interaction} \\
	\text{2. Asymmetry Generator}: \quad &\xipar_{asymmetry} = \text{Directional Preference} \\
	\text{3. Scale Setter}: \quad &\xipar_{scale} = \text{Characteristic Length/Time} \\
	\text{4. Information Encoder}: \quad &\xipar_{info} = \text{Computational} \\
	&\quad \text{Complexity Modifier} \\
	\text{5. Symmetry Breaker}: \quad &\xipar_{symmetry} = \text{Spontaneous Order}
\end{align}

\subsection{$\xipar$-Induced Computational Asymmetries}

\textbf{Computational Chirality}:

Even in mathematically symmetric operations, $\xipar$ creates computational preferences:
\begin{align}
	\text{Forward Calculation}: \quad &\xipar_{forward} = \xipar_0 \\
	\text{Inverse Calculation}: \quad &\xipar_{inverse} = \xipar_0 / \alpha \quad (\alpha > 1) \\
	\text{Verification}: \quad &\xipar_{verify} = \xipar_0 \cdot \beta \quad (\beta > 1)
\end{align}

This creates computational chirality where verification is easier than calculation.

\subsection{$\xipar$ Memory and History Dependence}

\textbf{$\xipar$ retains computational history}:
\begin{equation}
	\xipar(t) = \xipar_0 + \int_0^t K(t-\tau) \cdot f(\text{computation}(\tau)) \, d\tau
\end{equation}

where $K(t-\tau)$ is a memory kernel.

\section{Dimensional Scaling: Fundamental Differences Between 2D and 3D}

\subsection{Wave Propagation Scaling Laws}

The fundamental difference between 2D and 3D space profoundly influences T0 field behavior:

\textbf{Dimensional Field Equations}:
\begin{align}
	\text{2D}: \quad &\frac{1}{r} \frac{\partial}{\partial r}\left(r \frac{\partial T}{\partial r}\right) = -\frac{\rho(r)}{T(r)^2} \\
	\text{3D}: \quad &\frac{1}{r^2} \frac{\partial}{\partial r}\left(r^2 \frac{\partial T}{\partial r}\right) = -\frac{\rho(r)}{T(r)^2}
\end{align}

\textbf{Green's Function Differences}:
\begin{align}
	G_{2D}(r) &= -\frac{1}{2\pi} \ln(r) \quad \text{(Logarithmic decay)} \\
	G_{3D}(r) &= \frac{1}{4\pi r} \quad \text{(Power law decay)}
\end{align}

\subsection{Critical Dimension Thresholds}

\textbf{Lower Critical Dimension}: $d_c^{lower} = 2$

Below 2D, T0 fields cannot propagate conventionally:
\begin{equation}
	\text{1D}: \quad T(x) = T_0 + A|x| \quad \text{(Linear growth, unphysical)}
\end{equation}

\textbf{Upper Critical Dimension}: $d_c^{upper} = 4$

Above 4D, the mean-field theory becomes exact:
\begin{equation}
	\text{4D+}: \quad \xipar_{eff} = \xipar_0 \quad \text{(Dimension-independent)}
\end{equation}

\subsection{Algorithmic Performance Scaling}

\textbf{Dimensional Scaling Affects T0-Shor Performance}:
\begin{align}
	\text{2D Implementation}: \quad F_{2D} &= \sqrt{\ln(N)} \quad \text{(Logarithmic)} \\
	\text{3D Implementation}: \quad F_{3D} &= N^{1/3} \quad \text{(Power law)}
\end{align}

\textbf{Optimal Geometries by Dimension}:
\begin{align}
	\text{2D}: \quad &\text{Long, thin structures preferred} \\
	&Q \propto L/\lambda_{T0} \\
	\text{3D}: \quad &\text{Compact, spherical geometries optimal} \\
	&Q \propto (V/\lambda_{T0}^3)^{1/3}
\end{align}

\section{The Fundamental Nature of Numbers and Prime Structure}

\subsection{Prime Numbers as the Scaffolding of Mathematics}

The reason why all period-finding algorithms work (FFT, Quantum Shor, T0-Shor) lies in the fundamental structure of our number system:

\textbf{Prime Numbers as Mathematical Atoms}:
\begin{equation}
	\text{Every integer } n > 1: \quad n = p_1^{a_1} \cdot p_2^{a_2} \cdot ... \cdot p_k^{a_k} \quad \text{(Unique)}
\end{equation}

Prime numbers form the fundamental scaffolding - every number is completely determined by primes.

\textbf{Why Periodicity Arises from Prime Structure}:
\begin{align}
	\text{Euler's Theorem}: &\quad a^{\phi(N)} \equiv 1 \pmod{N} \\
	\text{Periodicity}: &\quad f(x) = a^x \bmod N \text{ is inherently periodic} \\
	\text{Universal Principle}: &\quad \text{Prime Structure} \rightarrow \text{Periodicity} \nonumber \\
	&\quad \rightarrow \text{Fourier Detection}
\end{align}


\textbf{Why the Period Contains Factorization Information}:
\begin{equation}
	a^r \equiv 1 \pmod{N} \Rightarrow a^r - 1 = (a^{r/2} - 1)(a^{r/2} + 1) \equiv 0 \pmod{N}
\end{equation}

The period $r$ encodes the prime factors through this algebraic relationship.

\section{Critical Assessment: Why T0-Shor Only Works for Small Numbers}

\subsection{The Precision Barrier}

Despite the theoretical elegance, T0-Shor faces a fundamental precision limitation that restricts its practical applicability:

\textbf{Required Resonance Precision for period r}:
\begin{equation}
	\Delta f_{required} = \frac{f_0}{r} - \frac{f_0}{r+1} = \frac{f_0}{r(r+1)} \approx \frac{f_0}{r^2}
\end{equation}

For cryptographically relevant numbers where $r \approx N$:
\begin{equation}
	\Delta f_{required} \approx \frac{f_0}{N^2}
\end{equation}

\textbf{Computational Precision Limits}:
\begin{align}
	\text{64-Bit Precision}: \quad &\epsilon \approx 10^{-16} \rightarrow N_{max} \approx 10^8 \text{ (27 bits)} \\
	\text{128-Bit Precision}: \quad &\epsilon \approx 10^{-34} \rightarrow N_{max} \approx 10^{17} \text{ (56 bits)} \\
	\text{1024-Bit RSA requires}: \quad &\epsilon \approx 10^{-617} \text{ (impossible)}
\end{align}

\subsection{The Precision Barrier and Scaling Limitations}

Important clarification: T0-Shor theoretically works for large numbers. The limitations are practical, not theoretical:

\textbf{Fundamental Scaling Challenges}:
\begin{align}
	\text{Memory Requirements}: \quad &M(N) = O(N) \text{ field points} \\
	\text{Computational Precision}: \quad &\epsilon_{required} = O(1/N^2) \\
	\text{Field Resolution}: \quad &\Delta r = O(1/N) \text{ for period detection} \\
	\text{Number of Operations}: \quad &\text{Still } O(\log N) \text{ per successful prediction}
\end{align}

\subsection{Comparison with Existing Methods}

\begin{table}[htbp]
	\centering
	%
	\begin{tabular}{p{2.4cm}p{2.4cm}p{2.4cm}p{2.4cm}p{2.4cm}}
		\toprule
		\textbf{Method} & \textbf{Operations (small N)} & \textbf{Operations (large N)} & \textbf{Success Rate} & \textbf{Hardware} \\
		\midrule
		Trivial Factorization & $O(\sqrt{N})$ & $O(\sqrt{N})$ & 100\% & Standard \\
		Classical FFT & $O(N \log N)$ & $O(N \log N)$ & 100\% & Standard \\
		Quantum Shor & $O((\log N)^3)$ & $O((\log N)^3)$ & $\approx$50\% & Quantum \\
		T0-Shor (Prediction Hit) & $O(\log N)$ & $O(\log N)$ & Variable & Standard \\
		T0-Shor (No Prediction) & $O(N \log N)$ & Limited by Precision & Variable & Standard \\
		\bottomrule
	\end{tabular}
	\caption{Realistic Comparison of Factorization Methods}
	\label{tab:method_comparison_realistic}
\end{table}

\textbf{Quantum Computers and the I/O Bottleneck}:

Quantum computers with electron-based memory have a theoretical memory advantage but face the same fundamental I/O limitations:

\begin{table}[htbp]
	\centering
	%
	\begin{tabular}{p{2.4cm}p{2.4cm}p{2.4cm}p{2.4cm}p{2.4cm}}
		\toprule
		\textbf{System} & \textbf{Memory} & \textbf{Input Mapping} & \textbf{Output Reading} & \textbf{Bottleneck} \\
		\midrule
		T0-Shor & RAM Limitation & Direct & Direct & Memory Scaling \\
		QC & Electron States & Exponential Encoding & Measurement Collapse & I/O Complexity \\
		T0 + QC & Electron States & Same QC Problem & Same QC Problem & I/O Complexity \\
		\bottomrule
	\end{tabular}
	\caption{Memory Systems and Their Fundamental Bottlenecks}
	\label{tab:memory_bottlenecks}
\end{table}

\section{Conclusions}

\subsection{Central Insights}

The time-mass duality leads to a mathematically consistent extension of the Shor algorithm with the following properties:

\begin{enumerate}
	\item Theoretical Framework: Hyperbolic geometry in duality space
	\item Wave Characteristics: T0 fields behave similarly to acoustic waves
	\item Vacuum Derivation: All parameters calculable from fundamental constants
	\item Self-Amplification: Error reduction improves the $\xipar$ parameter
	\item Multifunctionality: $\xipar$ has roles beyond simple coupling
	\item Dimensional Effects: 2D and 3D behave fundamentally differently
	\item Practical Limits: Precision and memory requirements limit applicability
\end{enumerate}

\subsection{Open Mathematical Questions}

Several mathematical aspects require further investigation:

\begin{enumerate}
	\item Rigorous convergence proof for field evolution equations
	\item Analysis of non-spherically symmetric configurations
	\item Investigation of chaotic dynamics in mass field evolution
	\item Connection between $\xipar$ parameter and experimentally measurable quantities
\end{enumerate}

The T0-Shor Algorithm represents an interesting theoretical construction that connects concepts from differential geometry, field theory, and computational complexity. However, its practical advantages over existing methods remain dependent on several unproven assumptions about the physical realizability of the underlying mathematical framework.

\begin{thebibliography}{99}
	\bibitem{shor1994}
	Shor, P. W. (1994). Algorithms for quantum computation: discrete logarithms and factoring. \textit{Proceedings 35th Annual Symposium on Foundations of Computer Science}, 124--134.
	
	\bibitem{higgs1964}
	Higgs, P. W. (1964). Broken symmetries and the masses of gauge bosons. \textit{Physical Review Letters}, 13(16), 508--509.
	
	\bibitem{weinberg1967}
	Weinberg, S. (1967). A model of leptons. \textit{Physical Review Letters}, 19(21), 1264--1266.
	
	\bibitem{gelfand1963}
	Gelfand, I. M., \& Fomin, S. V. (1963). \textit{Calculus of variations}. Prentice-Hall.
	
	\bibitem{arnold1989}
	Arnold, V. I. (1989). \textit{Mathematical methods of classical mechanics}. Springer-Verlag.
	
	\bibitem{evans2010}
	Evans, L. C. (2010). \textit{Partial differential equations}. American Mathematical Society.
	
	\bibitem{shannon1948}
	Shannon, C. E. (1948). A mathematical theory of communication. \textit{Bell System Technical Journal}, 27(3), 379--423.
	
	\bibitem{pollard1975}
	Pollard, J. M. (1975). A Monte Carlo method for factorization. \textit{BIT Numerical Mathematics}, 15(3), 331--334.
	
	\bibitem{lenstra1993}
	Lenstra, A. K., \& Lenstra Jr, H. W. (Eds.). (1993). \textit{The development of the number field sieve}. Springer-Verlag.
	
	\bibitem{nielsen_chuang2010}
	Nielsen, M. A., \& Chuang, I. L. (2010). \textit{Quantum computation and quantum information}. Cambridge University Press.
	
	\bibitem{riemannian_geometry}
	Lee, J. M. (2018). \textit{Introduction to Riemannian manifolds}. Springer.
	
	\bibitem{variational_calculus}
	Kot, M. (2014). \textit{A first course in the calculus of variations}. American Mathematical Society.
	
	\bibitem{pde_stability}
	Strikwerda, J. C. (2004). \textit{Finite difference schemes and partial differential equations}. SIAM.
	
	\bibitem{computational_complexity}
	Sipser, M. (2012). \textit{Introduction to the theory of computation}. Cengage Learning.
	
	\bibitem{information_theory}
	Cover, T. M., \& Thomas, J. A. (2012). \textit{Elements of information theory}. John Wiley \& Sons.
\end{thebibliography}