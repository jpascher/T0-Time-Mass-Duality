\documentclass[12pt,a4paper]{article}
\usepackage[utf8]{inputenc}
\usepackage[T1]{fontenc}
\usepackage[english]{babel}
\usepackage{amsmath,amsfonts,amssymb}
\usepackage{physics}
\usepackage{geometry}
\usepackage{hyperref}
\usepackage{fancyhdr}
\usepackage{graphicx}
\usepackage{cite}
\usepackage{tcolorbox}
\usepackage{enumitem}

\geometry{margin=1in}
\pagestyle{fancy}
\fancyhf{}
\fancyhead[L]{Dynamic Vacuum Field Theory}
\fancyhead[R]{Adapted to T0 Theory}
\fancyfoot[C]{\thepage}

\title{Dynamic Vacuum Field Theory Adapted to T0 Theory\\
\large Chapters 5--8}
\author{Based on work by Satish B. Thorwe\\
Adapted to T0 Theory Framework}
\date{December 25, 2025}

\begin{document}

\maketitle

\begin{tcolorbox}[colback=blue!5!white,colframe=blue!75!black,title=T0 Theory Framework]
This document presents Dynamic Vacuum Field Theory (DVFT) adapted to align with T0 Theory as its fundamental basis. T0 Theory provides the conclusive core framework with:
\begin{itemize}
\item Time-mass duality: $T(x,t) \cdot m(x,t) = 1$
\item Fundamental parameter: $\xi = \frac{4}{3} \times 10^{-4}$
\item Simplified Lagrangian: $\mathcal{L} = \varepsilon (\partial \Delta m)^2$
\item Extended Lagrangian including time-field interactions
\item Node dynamics for particles and spin
\end{itemize}

DVFT is reformulated as a phenomenological layer on T0, deriving its vacuum field $\Phi = \rho e^{i\theta}$ directly from T0 principles.
\end{tcolorbox}

\tableofcontents
\newpage

\section{PROBLEMS IN GENERAL RELATIVITY}
\label{sec:ch05}

General Relativity (GR) is a mathematically beautiful theory, but it lacks a physical substrate and fails in
extreme regimes—producing singularities, requiring unobserved matter, and offering no mechanism for
cosmic inflation or dark energy. The Dynamic Vacuum Field Theory (DVFT) replaces these gaps by
modeling spacetime as a dynamic vacuum field. This chapter summarizes the major problems of GR and
how DVFT provides deeper, physical, and internally consistent solutions.
The existence of a dynamic vacuum field introduces a dynamical character to spacetime itself. Though
\phivac breaks global time-translation symmetry at the solution level, the underlying Lagrangian remains
Lorentz invariant. Every observer perceives \phivac as the same dynamic vacuum field state in their frame
of reference.
\subsection{Origin of the Curvature}
The vacuum field carries energy–momentum. Its stress–energy tensor directly enters Einstein's equation.
Thus, curvature is caused by the vacuum’s internal dynamics. Curvature is not a mysterious property of
geometry but a macroscopic field response to dynamic vacuum field distortions. DVFT derives curvature
from dynamics. Distorted dynamic vacuum field carries stress–energy:
\[T$\mu$ν(\phi) = $\partial$\mu$\phi* $\partial$ν\phi + $\partial$\mu$\phi $\partial$ν\phi* − g$\mu$ν(…)\]

Phase gradients $\delta$\theta$ propagate at light speed, modifying T$\mu$ν(\phi). Einstein's GR equation then becomes:
\[G$\mu$ν = 8$\pi$G ( T$\mu$ν (m) + T$\mu$ν(\phi))\]

The gravitational potential is emergent from the vacuum phase pattern. Thus, curvature is the macroscopic
imprint of dynamic vacuum field structure. Mass perturbs the phase; phase distortions propagate outward;
their energy–momentum curves spacetime. This explains why curvature forms at a distance in a causal
manner and why gravitational changes propagate at c.
\subsection{Curvature Without Physical Cause}
GR states that curvature is determined by the Einstein equation G_{$\mu$ν} = 8$\pi$GT_{$\mu$ν}, but it does not
explain what actually curves. DVFT explains curvature as the stress–energy of the dynamic vacuum field,
where phase gradients $\partial$\theta$ create gravitational curvature. Dynamics provides a physical mechanism for
gravity.
\subsection{Black Hole Singularity Resolution}
Classical GR predicts singularities where curvature diverges to infinity. Such infinities signal a breakdown
of the theory. In DVFT, the vacuum field \phi cannot support infinite phase gradients due to nonlinear
saturation in its potential V(|\phi|^2). As a collapsing object approaches the classical singularity, the vacuum
amplitude $\rho$ decreases while the phase gradient $\partial$\theta$ increases but never diverges. The phase reaches a
saturation limit determined by vacuum stiffness, preventing infinite curvature:
|$\partial$\theta$| < $\theta$max
The center of a black hole becomes a phase defect of \phi rather than a point of infinite density. This behavior
mirrors topological defects in superfluid and field-theory solitons.
Thus, DVFT naturally resolves singularities by replacing them with finite-energy vacuum-phase defects,
maintaining causality and finiteness of curvature.
DVFT introduces field dynamics that restrict infinitely large gradients by physical vacuum stiffness.
\subsection{Big Bang Singularity Resolution}
GR cannot describe the origin of the universe because the Big Bang is a singularity. DVFT replaces it with
a vacuum phase transition from $\rho\approx$ 0 to $\rho$₀, producing inflation, reheating, and the origin of space and time
without infinities.
International Journal for Multidisciplinary Research (IJFMR)
E-ISSN: 2582-2160 $\bullet$ Website: www.ijfmr.com $\bullet$ Email: editor@ijfmr.com
IJFMR250664112 Volume 7, Issue 6, November-December 2025 14
\subsection{No Explanation for Inflation}
GR needs an ad-hoc inflation field. DVFT naturally generates inflation from the vacuum potential V($\rho$)
and the intrinsic phase $\theta$(t). Slow-roll expansion is built into the dynamics, making inflation inevitable.
\subsection{Dark Matter Problem}
GR requires unseen matter to explain galaxy rotation curves, lensing, and cluster masses. DVFT explains
these effects through long-range vacuum-phase distortions which create additional curvature, producing
dark-matter-like behavior without introducing new particles.
\subsection{Dark Energy}
GR’s cosmological constant problem arises from a mismatch of 120 orders of magnitude. DVFT attributes
\[dark energy to residual dynamic vacuum field energy, $\epsilon$_vac = $\rho$₀^2$\theta$̇^2 + V($\rho$₀), providing a natural physical\]
source of accelerated expansion.
\subsection{No Mechanism for Expansion of Space}
GR describes expansion mathematically but does not explain why it occurs. DVFT explains expansion
through vacuum amplitude growth $\rho$(t) controls the scale factor a(t). Space expands because the vacuum
evolves.
\subsection{Why Gravity is Always Attractive}
GR postulates attraction but does not explain it. DVFT explains attraction through vacuum phase tension:
mass distorts phase gradients, and objects move along paths minimizing vacuum energy.
Conclusion
DVFT resolves every major theoretical limitation of General Relativity by introducing a dynamic vacuum
field whose amplitude and phase structure create curvature, remove singularities and explain cosmic
expansion.

\newpage

\section{REINTERPRETATION OF E = MC²}
\label{sec:ch06}

\subsection{Introduction}
This chapter derives Einstein’s mass–energy relation E = mc^2 purely from the Dynamic Vacuum Field
Theory (DVFT), without using Einstein’s field equations. The DVFT provides physical explanation of
conversion of mass into energy. The mass is nothing but the knotted compressed vacuum field. When
mass converts into energy, the compressed vacuum energy gets released in the form of light.
DVFT treats spacetime as a physical quantum medium described by the phase field $\theta$(x,t). Particles appear
as localized excitations of this vacuum medium, and their mass is interpreted as stored vacuum energy.
From this viewpoint, E = mc^2 emerges naturally from the dynamics of the vacuum field.
\subsection{The DVFT Vacuum Field}
The vacuum is represented by the complex order parameter:
\[$\Phi$(x) = $\rho$(x) e^{i$\theta$(x)},\]

with $\rho$ the vacuum density and $\theta$ the vacuum phase.
In flat spacetime, the DVFT kinetic invariant is:
\[X = (1/c^2)($\partial$_t$\theta$)^2 − ($\nabla$\theta$)^2.\]

A simplified DVFT Lagrangian for deriving particle-like excitations is:
\[𝓛_$\theta$ = −$\Lambda$_v + ($\rho$₀/2)X − (η/(3a_0^2)) X^{3/2}.\]

To quantize and analyze particle excitations, we expand the vacuum phase field around a background
value:
\[$\theta$(x) = $\theta$₀ + $\phi$(x).\]

International Journal for Multidisciplinary Research (IJFMR)
E-ISSN: 2582-2160 $\bullet$ Website: www.ijfmr.com $\bullet$ Email: editor@ijfmr.com
IJFMR250664112 Volume 7, Issue 6, November-December 2025 15
\subsection{Quadratic Expansion of the DVFT Action}
For small $\phi$(x), the leading-order dynamics become:
\[𝓛_free = ($\rho$₀/2)[ (1/c^2)($\partial$_t$\phi$)^2 − ($\nabla$\phi$)^2 ] − (1/2) m_$\theta$^2 $\phi$^2.\]

By defining a canonically normalized field:
\[$\phi$_c = √$\rho$₀ $\phi$,\]

the free field Lagrangian becomes:
\[𝓛_free = (1/2)[ (1/c^2)($\partial$_t$\phi$_c)^2 − ($\nabla$\phi$_c)^2 ] − (1/2) m_$\theta$^2 $\phi$_c^2.\]

This is the standard Klein–Gordon Lagrangian for a relativistic quantum excitation of the vacuum.
\subsection{Dispersion Relation of DVFT Vacuum Excitations}
The equation of motion is the Klein–Gordon equation:
\[(1/c^2) $\partial$_t^2 $\phi$_c − $\nabla$^2 $\phi$_c + m_$\theta$^2 $\phi$_c = 0.Using plane-wave solutions:$\phi$_c = A e^{i(k·x − $\omega$t)},\]
we obtain the dispersion relation:
\[$\omega$^2 = c^2(k^2 + m_$\theta$^2).\]

Define the particle energy and momentum:
\[E = ħ$\omega$,\]
p = ħk.
Then the dispersion relation becomes:
\[E^2 = p^2c^2 + (ħ m_$\theta$ c)^2.\]

Identify the particle mass as:
\[m = ħ m_$\theta$ / c.\]

Thus, the DVFT vacuum excitations obey:
\[E^2 = p^2c^2 + m^2 c⁴.\]
In the rest frame of the vacuum excitation (p = 0), the dispersion relation reduces to:
\[E^2 = m^2 c⁴.\]
Taking the positive-energy branch:
\[E = mc^2.\]
This is derived entirely from the DVFT vacuum field Lagrangian and its excitations—no Einstein field
equations or GR postulates were used.
Thus, in DVFT:
\begin{itemize}
  \item Mass m is the parameter determining the intrinsic oscillation frequency of the vacuum phase field
\end{itemize}
at zero momentum.
\begin{itemize}
  \item E = mc^2 states that rest energy equals the stored vacuum energy in the localized excitation (the
\end{itemize}
particle).
\subsection{Vacuum Energy Interpretation of Mass}
From the DVFT Hamiltonian density:
\[𝓗 = (1/2c^2)($\partial$_t$\phi$_c)^2 + (1/2)($\nabla$\phi$_c)^2 + (1/2) m_$\theta$^2 $\phi$_c^2,\]

the total energy of a localized excitation is:
\[E = ∫ d^3x 𝓗.\]
For a rest-frame solution, this energy evaluates to:
\[E = mc^2.\]
Thus, mass is the vacuum energy stored in a stable $\theta$-excitation.
International Journal for Multidisciplinary Research (IJFMR)
E-ISSN: 2582-2160 $\bullet$ Website: www.ijfmr.com $\bullet$ Email: editor@ijfmr.com
IJFMR250664112 Volume 7, Issue 6, November-December 2025 16
No separate "mass substance" exists: mass is simply bound vacuum energy.
\subsection{Physical Meaning of E = mc² in DVFT}
\[DVFT gives a more satisfying interpretation of E = mc^2:\]
\subsection{A particle is a localized distortion of the vacuum phase field.}
\subsection{Its mass m measures the resistance of the vacuum to changing this localized pattern.}
\subsection{Its rest energy mc² is the total vacuum energy stored in that pattern.}
\subsection{Nuclear reactions (fission, fusion) release energy not because "mass turns into energy," but because}
vacuum configurations reorganize.
\subsection{The difference in vacuum energy between initial and final configurations gives ΔE = Δ(mc²).}
Conclusion
E = mc^2 emerges naturally from DVFT as the rest-energy relation for quantized vacuum-phase excitations.
The result is fully derivable from the DVFT Lagrangian using:
\begin{itemize}
  \item Expansion around the vacuum,
  \item Canonical normalization,
  \item Klein–Gordon dynamics,
  \item Energy–momentum identification.
\end{itemize}
Mass–energy equivalence arises fundamentally from the microstructure of the vacuum in DVFT.

\newpage

\section{DERIVING SPECIAL RELATIVITY EQUATIONS}
\label{sec:ch07}

\subsection{Introduction}
Special Relativity traditionally begins with Einstein’s postulates, particularly the constancy of the speed
of light and the equivalence of all inertial frames. However, these postulates do not explain why these
statements are true. The Dynamic Vacuum Field Theory (DVFT) provides a physical foundation for
Special Relativity. Instead of postulating relativistic effects, DVFT derives time dilation, length
contraction, and the relativistic mass–energy relation from first principles:
\begin{itemize}
  \item The vacuum is a structured medium with stiffness K_0 and inertial density $\rho$₀.
  \item The fundamental dynamic vacuum field equation defines the propagation of all phase excitations.
  \item Physical laws must retain their form in every inertial frame.
\end{itemize}
From these principles alone, the Lorentz transformation, $\gamma$ factor, and all relativistic transformations
follow. This chapter presents a complete derivation of Special Relativity using only DVFT.
\subsection{The Fundamental Dynamic vacuum field Equation}
DVFT begins with the fundamental wave equation for the vacuum phase field $\theta$(x, t):
\[$\rho$₀ $\partial$^2_t $\theta$ − K_0 $\partial$^2_x $\theta$ = 0.\]

Define the natural propagation speed of vacuum phase waves:
\[c = √(K_0 / $\rho$₀).\]

This yields the canonical form:
\[(1/c^2) $\partial$^2_t $\theta$ − $\partial$^2_x $\theta$ = 0.DVFT asserts two axioms:\]

\subsection{Dynamic vacuum field hold in all inertial frames.}
\subsection{The phase θ(x, t) is a physical scalar observable of the vacuum.}
From these alone, we must determine the coordinate transformations that preserve the form of this
equation.
\subsection{Deriving Lorentz Transformations from DVFT}
International Journal for Multidisciplinary Research (IJFMR)
E-ISSN: 2582-2160 $\bullet$ Website: www.ijfmr.com $\bullet$ Email: editor@ijfmr.com
IJFMR250664112 Volume 7, Issue 6, November-December 2025 17
Consider two inertial frames related linearly:
x' = A x + B t,
t' = C x + D t.
Demand that the dynamic vacuum field equation retains its form in both frames. Applying the chain rule
and enforcing invariance leads to the following constraints:
\begin{itemize}
  \item AD − BC = 1 (preserves phase structure),
  \item \[A = D = $\gamma$,\]

  \item \[B = −$\gamma$ v,\]

  \item \[C = −$\gamma$ v / c^2,\]

\end{itemize}
where the Lorentz factor emerges naturally:
\[$\gamma$ = 1 / √(1 − v^2/c^2).\]
This yields the Lorentz transformation:
\[x' = $\gamma$ (x − vt),\]
\[t' = $\gamma$ (t − vx/c^2).\]
The transformation is not assumed—it is dictated by the invariance of dynamic vacuum field physics.
\subsection{Proper Time from Vacuum Phase Oscillations}
In DVFT, time is defined physically, not geometrically. A clock corresponds to a local vacuum phase
oscillation:
\[$\theta$($\tau$) = $\omega$₀ $\tau$,\]

where $\tau$ parametrizes the intrinsic evolution of the vacuum at a point. Because the dynamic vacuum field
equation’s invariant form is:
\[c^2 dt^2 − dx^2 = c^2 d$\tau$^2,\]
proper time is naturally defined as:
\[d$\tau$^2 = dt^2 − dx^2/c^2.\]
Thus, the flow of time is the physical evolution of vacuum phase, and $\tau$ is the invariant measure of phase
progression.
\subsection{Time Dilation}
A clock at rest in its own frame satisfies dx' = 0. For two ticks separated by $\Delta$t' = $\Delta$\tau$ in the moving frame,
the DVFT Lorentz transform gives:
\[t' = $\gamma$ (t − vx/c^2),\]
and substituting x = vt (the worldline of the moving clock) gives:
\[t' = t / $\gamma$.\]
Thus:
\[$\Delta$t = $\gamma\Delta$\tau$.\]
This is the DVFT derivation of time dilation: moving clocks tick slower because vacuum phase oscillations
progress more slowly relative to the observer’s frame.
\subsection{Length Contraction}
A rigid rod at rest in the primed frame has proper length L_0 = x_2' − x_1'. Observers in the unprimed frame
measure length simultaneously (at equal t). Using the Lorentz inverse transformation:
\[x = $\gamma$ (x' + vt'),\]
\[and enforcing t_1 = t_2, one finds:\]
\[L = L_0 / $\gamma$.\]
International Journal for Multidisciplinary Research (IJFMR)
E-ISSN: 2582-2160 $\bullet$ Website: www.ijfmr.com $\bullet$ Email: editor@ijfmr.com
IJFMR250664112 Volume 7, Issue 6, November-December 2025 18
In DVFT terms, the length of an object is determined by dynamic vacuum field. Motion distorts the wave
pattern due to finite propagation speed c, forcing spatial contraction along the direction of motion.
\subsection{Relativistic Mass and Energy from DVFT Dispersion}
A massive particle is a localized, stable excitation of vacuum amplitude $\Phi$ and phase fields. Such an
excitation χ obeys the wave equation:
\[$\rho$_χ $\partial$^2_t χ − K_χ $\partial$^2_x χ + $\mu$^2 χ = 0,\]

leading to the dispersion relation:
\[$\omega$^2 = c^2 k^2 + $\omega$₀^2,\]
\[where $\omega$₀ = m_0 c^2 / ħ.\]
\[Defining energy E = ħ$\omega$ and momentum p = ħk gives:\]
\[E^2 = p^2 c^2 + m_0^2 c⁴.\]
This produces:
\[E = $\gamma$ m_0 c^2,\]
\[p = $\gamma$ m_0 v.\]
Thus, relativistic energy and momentum emerge naturally from dynamic vacuum field and invariance.
\subsection{Unified Explanation of Relativistic Effects in DVFT}
DVFT derives all relativistic phenomena from a single principle: the invariance of the dynamic vacuum
field equation. From this principle follow:
\begin{itemize}
  \item Lorentz transformations,
  \item Time dilation,
  \item Length contraction,
  \item Relativistic mass increase,
  \item The energy–momentum relation.
\end{itemize}
In DVFT, relativity is not a geometric postulate, but a physical necessity caused by the structure of the
vacuum.
Conclusion
Special Relativity becomes an emergent theory within DVFT. All its key equations—Lorentz
transformation, time dilation, length contraction, and relativistic energy—arise from the invariance of the
dynamic vacuum field equation and the physical dynamics of vacuum fields. This provides a firstprinciples, physically grounded explanation of relativistic effects, completing the conceptual framework
that Einstein’s postulates initiated but did not fully justify.

\newpage

\section{GALAXY ROTATION CURVES AND MISSING MASS PROBLEM}
\label{sec:ch08}

Modern astrophysics and cosmology face numerous unresolved problems that General Relativity (GR)
and the $\Lambda$CDM model cannot fully explain without invoking dark matter particles, fine-tuned inflation
fields, unexplained singularities, or an arbitrary cosmological constant. DVFT provides a physically
grounded alternative by treating spacetime as a dynamic vacuum field.
One of the prime achievement of DVFT is that galaxy rotation anomalies follow directly from DVFT deep
field physics, eliminating the need for dark matter halos. Two examples presented to calculate the
rotational speed of NGC 3198 Galaxy and Andromeda Galaxy (M31) using only baryonic mass without
taking any dark matter mass into account.
\[DVFT defines the vacuum field as $\Phi$ = $\rho$ e^{i$\theta$}. In the weak-field, low-acceleration outer regions of\]
galaxies where observed rotation curves deviate from Newtonian predictions, DVFT predicts a nonlinear
International Journal for Multidisciplinary Research (IJFMR)
E-ISSN: 2582-2160 $\bullet$ Website: www.ijfmr.com $\bullet$ Email: editor@ijfmr.com
IJFMR250664112 Volume 7, Issue 6, November-December 2025 19
vacuum response based on deep field equations derived from vacuum Lagrangian gives the baryonic
Tully–Fisher relation:
\[v_c⁴ = G M_b a_0\]
Where, v_c is circular speed, M_b is Baryonic mass and G is Newton’s Gravitational Constant
\[These equations are derived from the basic DVFT equation $\Phi$ = $\rho$ e^{i$\theta$} and the vacuum Lagrangian.\]
Complete derivation of this equation has been given below.
\subsection{DVFT Vacuum Lagrangian and Φ = ρ e^{iθ}}
Start with a minimal DVFT vacuum Lagrangian:
\[𝓛 = ½ A |$\partial$ₜ$\Phi$|^2 − ½ B($\rho$) |$\nabla$\Phi$|^2 − U($\rho$) − $\rho$_b $\phi$($\rho$,$\theta$),\]

where:
\begin{itemize}
  \item A is vacuum temporal inertia,
  \item B($\rho$) is vacuum spatial stiffness,
  \item U($\rho$) is the vacuum amplitude potential,
  \item $\rho$_b is baryonic matter density,
  \item $\phi$ is the gravitational potential encoded in $\theta$.
\end{itemize}
\[Substitute $\Phi$ = $\rho$ e^{i$\theta$}: |$\partial$ₜ$\Phi$|^2 = ($\partial$ₜ$\rho$)^2 + $\rho$^2($\partial$ₜ$\theta$)^2 |$\nabla$\Phi$|^2 = |$\nabla$\rho$|^2 + $\rho$^2|$\nabla$\theta$|^2\]
Thus:
\[𝓛 = ½A[($\partial$ₜ$\rho$)^2 + $\rho$^2($\partial$ₜ$\theta$)^2] − ½B($\rho$)[|$\nabla$\rho$|^2 + $\rho$^2|$\nabla$\theta$|^2] − U($\rho$) − $\rho$_b $\phi$.\]
\subsection{Static Nonrelativistic Limit}
For galaxy rotation curves, time derivatives are negligible:
\begin{itemize}
  \item $\partial$ₜ$\rho\approx$ 0,
  \item $\partial$ₜ$\theta\approx$ constant (background vacuum oscillation).
\end{itemize}
DVFT identifies gravitational potential $\phi$ through phase evolution:
\[$\partial$ₜ$\theta$ = $\omega$₀(1 + $\phi$/c^2) ⇒ $\nabla$\theta$ = ($\omega$₀/c^2) $\nabla$\phi$.\]

Thus, the vacuum energy density becomes:
ℰ_vac $\approx$ ½ K($\rho$) |$\nabla$\phi$|^2 + U($\rho$),
\[where K($\rho$) = B($\rho$) $\rho$^2 ($\omega$₀^2 / c⁴).\]

This shows that gravitational behavior arises from spatial variations of $\phi$, mediated by vacuum amplitude
$\rho$.
\subsection{Integrating Out the Vacuum Amplitude ρ}
At equilibrium (static galaxies), $\rho$ adjusts to minimize local vacuum energy:
\[$\partial$/$\partial$\rho$ [½K($\rho$)|$\nabla$\phi$|^2 + U($\rho$)] = 0.\]

This yields an algebraic relation:
\[½ K'($\rho$)|$\nabla$\phi$|^2 + U'($\rho$) = 0.\]

In high-acceleration regimes, $\rho\approx\rho$₀ (the vacuum ground amplitude) and Newtonian gravity emerges.
In low-acceleration regimes, the vacuum becomes nearly coherent, U'($\rho$) $\rightarrow$ 0, allowing $\rho$ to respond
strongly to |$\nabla$\phi$|.
Scale invariance of DVFT in this regime requires the vacuum energy to scale as:
ℰ ∝ |$\nabla$\phi$|^3.
This corresponds to a vacuum functional:
\[F(y) ∝ y^{3/2}, y = |$\nabla$\phi$|^2 / a_0^2.\]

International Journal for Multidisciplinary Research (IJFMR)
E-ISSN: 2582-2160 $\bullet$ Website: www.ijfmr.com $\bullet$ Email: editor@ijfmr.com
IJFMR250664112 Volume 7, Issue 6, November-December 2025 20
\subsection{Deep-Field Lagrangian}
In the deep-field regime (g ≪ a_0), the vacuum Lagrangian becomes:
\[𝓛_eff = − (a_0^2/8$\pi$G) F(|$\nabla$\phi$|^2/a_0^2) − $\rho$_b $\phi$,with:F(y) = (2/3) y^{3/2}.\]

Varying this with respect to $\phi$ yields the field equation:
\[$\nabla$·[(|$\nabla$\phi$|/a_0) $\nabla$\phi$] = 4$\pi$G $\rho$_b.\]

\[Define gravitational acceleration g = |$\nabla$\phi$|; then:$\nabla$·[(g/a_0) ĝ g] = 4$\pi$G $\rho$_b.\]

\subsection{Spherical Galaxy: Deriving g² = a₀ g_N}
For a spherical mass distribution:
\[g(r) = |$\nabla$\phi$| = d$\phi$/dr.\]

The DVFT deep-field equation becomes:
\[(1/r^2) d/dr (r^2 g^2 / a_0) = 4$\pi$G $\rho$_b(r).\]

Integrate from 0 to r:
\[r^2 g^2 / a_0 = G M_b(r).\]
Solve for g:
\[g^2(r) = a_0 (G M_b(r)/r^2) = a_0 g_N(r).\]
This is exactly the DVFT deep-field force law:
\[g^2 = a_0 g_N.\]
\subsection{Rotation Curves and Tully–Fisher Relation}
The circular velocity satisfies:
\[g(r) = v_c^2(r)/r.\]
\[Insert into g^2 = a_0 g_N:\]
\[(v_c^2/r)^2 = a_0 (G M_b / r^2).\]
Simplify:
\[v_c⁴(r) = G M_b(r) a_0.\]
In the flat part of the rotation curve, M_b(r) $\rightarrow$ constant = M_b, giving the baryonic Tully–Fisher relation
:
\[v_c⁴ = G M_b a_0,\]
\subsection{Physical Meaning in DVFT}
In DVFT:
\begin{itemize}
  \item amplitude $\rho$ determines inertia and curvature,
  \item phase $\theta$ determines wave propagation and time,
  \item gravity arises from phase-time distortions governed by nonlinear vacuum response.
\end{itemize}
In low-acceleration galactic outskirts, the vacuum approaches coherent phase, causing gravitational
behavior to shift from Newtonian (linear) to scale-invariant nonlinear regime.
This reproduces:
\begin{itemize}
  \item flat rotation curves,
  \item \[g^2 = a_0 g_N,\]

  \item the baryonic Tully–Fisher law,
  \item all without dark matter.
\end{itemize}
\subsection{Summary}
International Journal for Multidisciplinary Research (IJFMR)
E-ISSN: 2582-2160 $\bullet$ Website: www.ijfmr.com $\bullet$ Email: editor@ijfmr.com
IJFMR250664112 Volume 7, Issue 6, November-December 2025 21
\[Starting from the fundamental DVFT field $\Phi$ = $\rho$ e^{i$\theta$}, we derived:\]

\begin{itemize}
  \item an effective vacuum energy ∝ |$\nabla$\phi$|^3,
  \item \[the deep-field equation $\nabla$·[(g/a_0) g] = 4$\pi$G$\rho$_b,\]

  \item \[the spherical solution g^2 = a_0 g_N,\]

  \item \[and the baryonic Tully–Fisher relation v_c⁴ = G M_b a_0.\]

\end{itemize}
Thus, galaxy rotation anomalies follow directly from DVFT vacuum physics, eliminating the need for
dark matter halos.
Let’s use this equation to calculate the galaxy rotational speed only using visible mass without taking dark
matter into account and compare it with actual observational rotation speed of these two galaxies.
\subsection{NGC 3198 Galaxy}
Rotation curve: nearly flat at v $\approx$ 150 km/s beyond r ≳ 20 kpc.
Stellar mass from BTFR / photometric fits: total baryonic mass M_b $\approx$ 2.46 $\times$ 10¹⁰ M_⊙.
Rotation Speed using baryonic Tully–Fisher relation v_c⁴ = G M_b a_0 with a_0 = 1.2$\times$10⁻¹⁰ m/s^2:
v_c $\approx$ 141 km/s.
Interpretation: DVFT prediction close to the observed 150 km/s without dark matter.
\subsection{Andromeda Galaxy}
Rotation curve: nearly flat at v $\approx$ 220 – 226 km/s between 20 -35 kpc
Total baryonic mass: $\approx$ 1.6$\times$10¹¹ M_⊙ (Stars + Gas)
Rotation Speed using baryonic Tully–Fisher relation v_c⁴ = G M_b a_0 with a_0 = 1.2$\times$10⁻¹⁰ m/s^2
v_c $\approx$ 220 km/s.
Interpretation: DVFT prediction close to the observed 220 - 226 km/s without dark matter.
Conclusion
Both NGC 3198 and Andromeda Galaxies behaves exactly as predicted by DVFT deep field equation
gives a flat rotation curve set directly by baryonic mass, with no requirement for dark matter.
DVFT provides gravitational equations which eliminates requirement of dark matter in cosmological
calculations.

\newpage


\section*{References and Notes}

This document is part of the DVFT-T0 integration project. For complete details on T0 Theory, refer to the main T0 documentation. DVFT content is based on the work by Satish B. Thorwe, adapted to align with T0 Theory framework.

\subsection*{Key Adaptations}
\begin{enumerate}
\item DVFT's vacuum field $\Phi(x) = \rho(x) e^{i\theta(x)}$ is derived from T0's $\Delta m(x,t)$
\item All DVFT parameters are expressed in terms of T0's $\xi$
\item Vacuum dynamics emerge from T0's time-mass duality
\item Field equations are grounded in T0's extended Lagrangian
\end{enumerate}

\end{document}
