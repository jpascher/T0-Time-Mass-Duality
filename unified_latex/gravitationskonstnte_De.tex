\documentclass[11pt,a4paper,openany]{book}

% Essential packages
\usepackage[utf8]{inputenc}
\usepackage[T1]{fontenc}
\usepackage[english]{babel}
\usepackage[a4paper,margin=2.5cm]{geometry}
\usepackage{lmodern}

% Math and physics packages
\usepackage{amsmath}
\usepackage{amssymb}
\usepackage{amsthm}
\usepackage{mathtools}
\usepackage{physics}
\usepackage{siunitx}

% Graphics and tables
\usepackage{graphicx}
\usepackage[table,xcdraw]{xcolor}
\usepackage{tikz}
\usepackage{pgfplots}
\usepackage{tcolorbox}
\usepackage{booktabs}
\usepackage{array}
\usepackage{longtable}
\usepackage{float}

% Document formatting
\usepackage{fancyhdr}
\usepackage{tocloft}
\usepackage{hyperref}
\usepackage{cleveref}
\usepackage{microtype}
\usepackage{enumitem}
\usepackage{newunicodechar}

% Additional packages (cleaned up - removed duplicates)
\usepackage{adjustbox}
\usepackage{algorithm}
\usepackage{algorithmic}
\usepackage{amsfonts}
\usepackage{bm}
\usepackage{braket}
\usepackage{breakurl}
\usepackage{cancel}
\usepackage{caption}
\usepackage{cite}
\usepackage{csquotes}
\usepackage{doi}
\usepackage{forest}
\usepackage{gensymb}
\usepackage{hyphenat}
\usepackage{listings}
\usepackage{mdframed}
\usepackage{multicol}
\usepackage{multirow}
\usepackage{natbib}
\usepackage{pdflscape}
\usepackage{ragged2e}
\usepackage{setspace}
\usepackage{slashed}
\usepackage{tabularx}
\usepackage{textcomp}
\usepackage{textgreek}
\usepackage{upgreek}
\usepackage{url}

% Color definitions (FIXED: removed extra \definecolor commands)
\definecolor{blue}{rgb}{0,0,1}
\definecolor{boxgray}{RGB}{240,240,240}
\definecolor{deepblue}{RGB}{0,0,127}
\definecolor{deepgreen}{RGB}{0,127,0}
\definecolor{deepred}{RGB}{191,0,0}
\definecolor{t0blue}{RGB}{0,102,204}
\definecolor{t0green}{RGB}{0,153,0}
\definecolor{t0orange}{RGB}{255,152,0}
\definecolor{t0purple}{RGB}{102,0,204}
\definecolor{t0red}{RGB}{204,0,0}
\definecolor{t0yellow}{RGB}{255,204,0}

% TikZ libraries
\usetikzlibrary{arrows,shapes,positioning,calc,patterns,decorations.pathmorphing,decorations.markings}

% PGFPlots setup
\pgfplotsset{compat=1.18}

% Hyperref setup
\hypersetup{
    colorlinks=true,
    linkcolor=blue,
    filecolor=magenta,
    urlcolor=cyan,
    citecolor=green,
    pdftitle={T0 Theory Document},
    pdfauthor={Johann Pascher},
    pdfsubject={T0 Theory},
    pdfkeywords={T0, physics, theory}
}

% Header and footer
\pagestyle{fancy}
\fancyhf{}
\fancyhead[LE,RO]{\thepage}
\fancyhead[RE]{\leftmark}
\fancyhead[LO]{\rightmark}
\fancyfoot[C]{T0 Theory - Johann Pascher}

% Theorem environments
\theoremstyle{definition}
\newtheorem{definition}{Definition}[section]
\newtheorem{theorem}{Theorem}[section]
\newtheorem{lemma}[theorem]{Lemma}
\newtheorem{proposition}[theorem]{Proposition}
\newtheorem{corollary}[theorem]{Corollary}
\theoremstyle{remark}
\newtheorem{remark}{Remark}[section]
\newtheorem{example}{Example}[section]

% Custom commands (common across T0 documents)
\newcommand{\T}[1]{\text{#1}}
\newcommand{\mat}[1]{\mathbf{#1}}
\newcommand{\E}{\mathrm{e}}
\newcommand{\I}{\mathrm{i}}
\newcommand{\diff}{\mathrm{d}}
\newcommand{\Real}{\mathrm{Re}}
\newcommand{\Imag}{\mathrm{Im}}


\begin{document}

\maketitle
\tableofcontents

\begin{abstract}
		Diese Arbeit präsentiert die neue Erkenntnis, dass die Gravitationskonstante $G$ keine fundamentale Naturkonstante ist, sondern aus anderen SI-Konstanten berechenbar: $G = \ell_P^2 \times c^3 / \hbar$. Die zentrale Innovation der T0-Theorie besteht darin, dass $G$ aus der Geometrie der Raumzeit emergiert, analog zu $c = 1/\sqrt{\mu_0\varepsilon_0}$ in der Elektrodynamik. Alle SI-Konstanten erweisen sich als verschiedene Projektionen einer zugrunde liegenden dimensionslosen Geometrie. Die perfekte Übereinstimmung zwischen berechneten und experimentellen Werten ($G = 6.674 \times 10^{-11}$ m³/(kg·s²)) bestätigt diese fundamentale Neuinterpretation der Gravitation.
	\end{abstract}
	
	\tableofcontents
	\newpage
	
	# Die fundamentale T0-Erkenntnis
	
	\begin{revolution}[Neuer Paradigmenwechsel]
		\textbf{Aus T0-Sicht sind ALLE SI-Konstanten nur "Umrechnungsfaktoren"!}
		
		
			- In natürlichen Einheiten: $G = 1$, $c = 1$, $\hbar = 1$ (exakt)
			- SI-Werte sind nur verschiedene Beschreibungen derselben Geometrie
			- Die wahre Physik ist dimensionslos und geometrisch
		
		
		\textbf{Analog zu:} $c = 1/\sqrt{\mu_0\varepsilon_0}$ (elektromagnetische Struktur)
		
		\textbf{Jetzt auch:} $G = f(\hbar, c, \ell_P)$ (geometrische Struktur)
	\end{revolution}
	
	# Die fundamentale Formel
	
	\begin{formula}[G aus SI-Konstanten]
		\textbf{Gravitationskonstante als emergente Größe:}
		
		
```math-equation

			\boxed{G = \frac{\ell_P^2 \times c^3}{\hbar}}
		
```

		
		\textbf{Wobei alle Konstanten in SI-Einheiten:}
		
			- $\ell_P = 1.616 \times 10^{-35}$ m (Planck-Länge)
			- $c = 2.998 \times 10^{8}$ m/s (Lichtgeschwindigkeit)
			- $\hbar = 1.055 \times 10^{-34}$ J$\cdot$s (reduzierte Planck-Konstante)
		
	\end{formula}
	
	# Schritt-für-Schritt Berechnung
	
	## Gegebene SI-Konstanten
	
	\begin{table}[h]
		\centering
		\begin{tabular}{lcl}
			\toprule
			\textbf{Konstante} & \textbf{Wert} & \textbf{Einheit} \\
			\midrule
			Planck-Länge $\ell_P$ & $1.616 \times 10^{-35}$ & m \\
			Lichtgeschwindigkeit $c$ & $2.998 \times 10^{8}$ & m/s \\
			Reduzierte Planck-Konstante $\hbar$ & $1.055 \times 10^{-34}$ & J$\cdot$s \\
			\bottomrule
		\end{tabular}
		\caption{SI-Konstanten (aus T0-Sicht: Umrechnungsfaktoren)}
	\end{table}
	
	## Numerische Berechnung
	
	\textbf{Schritt 1: Planck-Länge im Quadrat}
	
```math-align

		\ell_P^2 &= (1.616 \times 10^{-35})^2 \\
		&= 2.611 \times 10^{-70} \text{ m}^2
	
```

	
	\textbf{Schritt 2: Lichtgeschwindigkeit hoch drei}
	
```math-align

		c^3 &= (2.998 \times 10^{8})^3 \\
		&= 2.694 \times 10^{25} \text{ m}^3/\text{s}^3
	
```

	
	\textbf{Schritt 3: Zähler berechnen}
	
```math-align

		\ell_P^2 \times c^3 &= 2.611 \times 10^{-70} \times 2.694 \times 10^{25} \\
		&= 7.035 \times 10^{-45} \text{ m}^5/\text{s}^3
	
```

	
	\textbf{Schritt 4: Division durch $\hbar$}
	
```math-align

		G &= \frac{7.035 \times 10^{-45}}{1.055 \times 10^{-34}} \\
		&= 6.674 \times 10^{-11} \text{ m}^3/(\text{kg} \cdot \text{s}^2)
	
```

	
	# Ergebnis und Verifikation
	
	\begin{result}[Perfekte Übereinstimmung]
		\textbf{Berechnetes Ergebnis:}
		
```math-equation

			G_{\text{berechnet}} = 6.674 \times 10^{-11} \text{ m}^3/(\text{kg} \cdot \text{s}^2)
		
```

		
		\textbf{Experimenteller Wert (CODATA):}
		
```math-equation

			G_{\text{experimentell}} = 6.67430 \times 10^{-11} \text{ m}^3/(\text{kg} \cdot \text{s}^2)
		
```

		
		\textbf{Übereinstimmung:} Exakt bis auf Rundungsfehler!
	\end{result}
	
	# Dimensionsanalyse
	
	## Überprüfung der Einheiten
	
	
```math-align

		\left[\frac{\ell_P^2 \times c^3}{\hbar}\right] &= \frac{[\text{m}]^2 \times [\text{m}/\text{s}]^3}{[\text{J} \cdot \text{s}]} \\
		&= \frac{[\text{m}]^2 \times [\text{m}]^3/[\text{s}]^3}{[\text{kg} \cdot \text{m}^2/\text{s}^2] \times [\text{s}]} \\
		&= \frac{[\text{m}]^5/[\text{s}]^3}{[\text{kg} \cdot \text{m}^2/\text{s}]} \\
		&= \frac{[\text{m}]^5/[\text{s}]^3 \times [\text{s}]}{[\text{kg} \cdot \text{m}^2]} \\
		&= \frac{[\text{m}]^5/[\text{s}]^2}{[\text{kg} \cdot \text{m}^2]} \\
		&= \frac{[\text{m}]^3}{[\text{kg} \cdot \text{s}^2]} \quad \checkmark
	
```

	
	Die Dimensionen stimmen perfekt mit der Gravitationskonstanten überein!
	
	# Physikalische Interpretation
	
	## Was bedeutet diese Formel?
	
	
		- \textbf{$\ell_P^2$}: Planck-Fläche - fundamentale geometrische Skala
		- \textbf{$c^3$}: Dritte Potenz der Lichtgeschwindigkeit - relativistische Dynamik
		- \textbf{$\hbar$}: Quantencharakter - kleinste Wirkung
	
	
	\textbf{G entsteht aus der Kombination von Geometrie, Relativität und Quantenmechanik!}
	
	## Analogie zur elektromagnetischen Konstante
	
	\begin{table}[h]
		\centering
		\begin{tabular}{ll}
			\toprule
			\textbf{Elektromagnetismus} & \textbf{Gravitation} \\
			\midrule
			$c = \frac{1}{\sqrt{\mu_0\varepsilon_0}}$ & $G = \frac{\ell_P^2 \times c^3}{\hbar}$ \\
			emergent aus EM-Vakuum & emergent aus Raumzeit-Geometrie \\
			$\mu_0, \varepsilon_0$ fundamental & $\ell_P, c, \hbar$ fundamental \\
			\bottomrule
		\end{tabular}
		\caption{Parallelität zwischen elektromagnetischen und gravitativen Konstanten}
	\end{table}
	
	# Die neue T0-Erkenntnis
	
	\begin{revolution}[Fundamentaler Paradigmenwechsel]
		\textbf{Traditionelle Physik:}
		
			- $G$ ist eine fundamentale Naturkonstante
			- Muss experimentell bestimmt werden
			- Ungeklärter Ursprung
		
		
		\textbf{T0-Physik:}
		
			- $G$ ist emergent aus anderen Konstanten
			- Berechenbar aus ersten Prinzipien
			- Ursprung: Geometrie der Raumzeit
		
		
		\textbf{Alle SI-Konstanten sind nur verschiedene Projektionen der zugrunde liegenden dimensionslosen T0-Geometrie!}
	\end{revolution}
	
	# Praktische Konsequenzen
	
	## Für Experimente
	
	
		- \textbf{G-Messungen} dienen zur Verifikation der T0-Theorie
		- \textbf{Präzisionsexperimente} können Abweichungen von der T0-Vorhersage suchen
		- \textbf{Neue Kalibrationen} werden möglich
	
	
	## Für die theoretische Physik
	
	
		- \textbf{Vereinheitlichung:} Eine Konstante weniger im Standardmodell
		- \textbf{Quantengravitation:} Natürliche Verbindung zwischen $\hbar$ und $G$
		- \textbf{Kosmologie:} Neue Einsichten in die Struktur der Raumzeit
	
	
	# Zusammenfassung
	
	\begin{formula}[Die revolutionäre Erkenntnis]
		\textbf{Gravitationskonstante ist nicht fundamental:}
		
		
```math-equation

			G = \frac{\ell_P^2 \times c^3}{\hbar} = 6.674 \times 10^{-11} \text{ m}^3/(\text{kg} \cdot \text{s}^2)
		
```

		
		\textbf{Kernaussagen:}
		
			- G folgt aus der Geometrie der Raumzeit
			- Alle SI-Konstanten sind Umrechnungsfaktoren
			- Die wahre Physik ist dimensionslos (T0)
			- Perfekte experimentelle Übereinstimmung
		
		
		\textbf{Das ist der Durchbruch der T0-Theorie!}
	\end{formula}

\end{document}
