\documentclass[11pt,a4paper,openany]{book}

% Essential packages
\usepackage[utf8]{inputenc}
\usepackage[T1]{fontenc}
\usepackage[english]{babel}
\usepackage[a4paper,margin=2.5cm]{geometry}
\usepackage{lmodern}

% Math and physics packages
\usepackage{amsmath}
\usepackage{amssymb}
\usepackage{amsthm}
\usepackage{mathtools}
\usepackage{physics}
\usepackage{siunitx}

% Graphics and tables
\usepackage{graphicx}
\usepackage[table,xcdraw]{xcolor}
\usepackage{tikz}
\usepackage{pgfplots}
\usepackage{tcolorbox}
\usepackage{booktabs}
\usepackage{array}
\usepackage{longtable}
\usepackage{float}

% Document formatting
\usepackage{fancyhdr}
\usepackage{tocloft}
\usepackage{hyperref}
\usepackage{cleveref}
\usepackage{microtype}
\usepackage{enumitem}
\usepackage{newunicodechar}

% Additional packages
\usepackage{adjustbox}
\usepackage{algorithm}
\usepackage{algorithmic}
\usepackage{amsfonts}
\usepackage{amsmath,amsfonts,amssymb}
\usepackage{amsmath,amsfonts,amssymb,physics}
\usepackage{amsmath,amssymb}
\usepackage{amsmath,amssymb,amsfonts,amsthm}
\usepackage{amsmath,amssymb,amsthm}
\usepackage{amsmath,amssymb,physics,graphicx,xcolor,amsthm}
\usepackage{bm}
\usepackage{booktabs,array,longtable,multirow}
\usepackage{braket}
\usepackage{breakurl}
\usepackage{cancel}
\usepackage{caption}
\usepackage{cite}
\usepackage{color}
\usepackage{colortbl}
\usepackage{csquotes}
\usepackage{doi}
\usepackage{forest}
\usepackage{gensymb}
\usepackage{geometry,fancyhdr}
\usepackage{graphicx,tikz,pgfplots}
\usepackage{hyperref,url}
\usepackage{hyphenat}
\usepackage{listings}
\usepackage{listings,enumerate}
\usepackage{mdframed}
\usepackage{multicol}
\usepackage{multirow}
\usepackage{natbib}
\usepackage{pdflscape}
\usepackage{ragged2e}
\usepackage{setspace}
\usepackage{siunitx,xcolor,graphicx}
\usepackage{slashed}
\usepackage{tabularx}
\usepackage{textcomp}
\usepackage{textgreek}
\usepackage{tikz,pgfplots}
\usepackage{upgreek}
\usepackage{url}

% Custom commands and definitions
\definecolor{blue}
\definecolor{blue}{rgb}{0,0,1}
\definecolor{boxgray}
\definecolor{boxgray}{RGB}{240,240,240}
\definecolor{deepblue}
\definecolor{deepblue}{RGB}{0,0,127}
\definecolor{deepgreen}
\definecolor{deepgreen}{RGB}{0,127,0}
\definecolor{deepred}
\definecolor{deepred}{RGB}{191,0,0}
\definecolor{t0blue}
\definecolor{t0blue}{RGB}{0,102,204}
\definecolor{t0blue}{RGB}{33,150,243}
\definecolor{t0green}
\definecolor{t0green}{RGB}{0,153,0}
\definecolor{t0green}{RGB}{0,153,76}
\definecolor{t0green}{RGB}{76,175,80}
\definecolor{t0orange}
\definecolor{t0orange}{RGB}{255,152,0}
\definecolor{t0purple}
\definecolor{t0purple}{RGB}{102,0,204}
\definecolor{t0purple}{RGB}{156,39,176}
\definecolor{t0red}
\definecolor{t0red}{RGB}{204,0,0}
\definecolor{t0red}{RGB}{204,0,51}
\definecolor{t0red}{RGB}{244,67,54}
\definecolor{t0yellow}
\definecolor{t0yellow}{RGB}{255,204,0}
\geometry{a4paper, left=25mm, right=25mm, top=25mm, bottom=25mm}
\geometry{a4paper, margin=1in}
\geometry{a4paper, margin=2.5cm}
\geometry{a4paper, margin=2cm}
\geometry{left=2.5cm,right=2.5cm,top=2.5cm,bottom=2.5cm}
\geometry{left=2cm,right=2cm,top=2cm,bottom=2cm}
\geometry{margin=1in}
\geometry{margin=2.5cm}
\geometry{margin=2cm}
\hypersetup{
	colorlinks=true,
	linkcolor=blue,
	citecolor=blue,
	urlcolor=blue,
	pdftitle={Analysis and Implications of MNRAS Paper 544 for the T0-Theory}
\hypersetup{
	colorlinks=true,
	linkcolor=blue,
	citecolor=blue,
	urlcolor=blue,
	pdftitle={Beweis: Die Feinstrukturkonstante α = 1 in natürlichen Einheiten}
\hypersetup{
	colorlinks=true,
	linkcolor=blue,
	citecolor=blue,
	urlcolor=blue,
	pdftitle={Beweis: Die Koide-Formel enthält implizit $\xi$}
\hypersetup{
	colorlinks=true,
	linkcolor=blue,
	citecolor=blue,
	urlcolor=blue,
	pdftitle={Chinas Photonischer Quantenchip: 1000x-Speedup und T0-Integration}
\hypersetup{
	colorlinks=true,
	linkcolor=blue,
	citecolor=blue,
	urlcolor=blue,
	pdftitle={Complete Derivation of Higgs Mass and Wilson Coefficients}
\hypersetup{
	colorlinks=true,
	linkcolor=blue,
	citecolor=blue,
	urlcolor=blue,
	pdftitle={Complete Particle Spectrum: Standard Model vs T0 Theory}
\hypersetup{
	colorlinks=true,
	linkcolor=blue,
	citecolor=blue,
	urlcolor=blue,
	pdftitle={Conceptual Comparison of Unified Natural Units and Extended Standard Model}
\hypersetup{
	colorlinks=true,
	linkcolor=blue,
	citecolor=blue,
	urlcolor=blue,
	pdftitle={Connections between the Mizohata-Takeuchi Counterexample and the T0 Time-Mass Duality Theory}
\hypersetup{
	colorlinks=true,
	linkcolor=blue,
	citecolor=blue,
	urlcolor=blue,
	pdftitle={Das Relationale Zahlensystem: Primzahlen als fundamentale Verhältnisse}
\hypersetup{
	colorlinks=true,
	linkcolor=blue,
	citecolor=blue,
	urlcolor=blue,
	pdftitle={Das T0-Modell (Planck-Referenziert): Eine Neuformulierung der Physik}
\hypersetup{
	colorlinks=true,
	linkcolor=blue,
	citecolor=blue,
	urlcolor=blue,
	pdftitle={Das T0-Modell: Zeit-Energie-Dualität und geometrische Ruhemasse}
\hypersetup{
	colorlinks=true,
	linkcolor=blue,
	citecolor=blue,
	urlcolor=blue,
	pdftitle={Der Massenskalierungsexponent κ in der T0-Theorie}
\hypersetup{
	colorlinks=true,
	linkcolor=blue,
	citecolor=blue,
	urlcolor=blue,
	pdftitle={Der geometrische Formalismus der T0-Quantenmechanik und seine Anwendung auf Quantencomputer}
\hypersetup{
	colorlinks=true,
	linkcolor=blue,
	citecolor=blue,
	urlcolor=blue,
	pdftitle={Der xi Parameter und Teilchendifferenzierung in der T0-Theorie}
\hypersetup{
	colorlinks=true,
	linkcolor=blue,
	citecolor=blue,
	urlcolor=blue,
	pdftitle={Deterministic Quantum Mechanics via T0-Energy Field Formulation}
\hypersetup{
	colorlinks=true,
	linkcolor=blue,
	citecolor=blue,
	urlcolor=blue,
	pdftitle={Deterministische Quantenmechanik via T0-Energiefeld-Formulierung}
\hypersetup{
	colorlinks=true,
	linkcolor=blue,
	citecolor=blue,
	urlcolor=blue,
	pdftitle={Die Elektroneneinheitsladung in der T0-Theorie: Jenseits von Punkt-Singularitäten}
\hypersetup{
	colorlinks=true,
	linkcolor=blue,
	citecolor=blue,
	urlcolor=blue,
	pdftitle={Die Feinstrukturkonstante: Verschiedene Darstellungen und Beziehungen}
\hypersetup{
	colorlinks=true,
	linkcolor=blue,
	citecolor=blue,
	urlcolor=blue,
	pdftitle={Die Musikalische Spirale und die 137: Die mathematische Entdeckung der kosmischen Verstimmung}
\hypersetup{
	colorlinks=true,
	linkcolor=blue,
	citecolor=blue,
	urlcolor=blue,
	pdftitle={E=mc² = E=m: Die Konstanten-Illusion entlarvt}
\hypersetup{
	colorlinks=true,
	linkcolor=blue,
	citecolor=blue,
	urlcolor=blue,
	pdftitle={E=mc² = E=m: The Constants Illusion Exposed}
\hypersetup{
	colorlinks=true,
	linkcolor=blue,
	citecolor=blue,
	urlcolor=blue,
	pdftitle={Einfache Lagrange-Revolution: Von der Standardmodell-Komplexität zur T0-Eleganz}
\hypersetup{
	colorlinks=true,
	linkcolor=blue,
	citecolor=blue,
	urlcolor=blue,
	pdftitle={Einführung in die Umsetzung photonischer Bauteile auf Wafern für Nachrichtentechniker}
\hypersetup{
	colorlinks=true,
	linkcolor=blue,
	citecolor=blue,
	urlcolor=blue,
	pdftitle={Einführung in photonische Quantenchips für Nachrichtentechniker}
\hypersetup{
	colorlinks=true,
	linkcolor=blue,
	citecolor=blue,
	urlcolor=blue,
	pdftitle={Elimination der Masse als dimensionaler Platzhalter im T0-Modell}
\hypersetup{
	colorlinks=true,
	linkcolor=blue,
	citecolor=blue,
	urlcolor=blue,
	pdftitle={Elimination of Mass as Dimensional Placeholder in the T0 Model}
\hypersetup{
	colorlinks=true,
	linkcolor=blue,
	citecolor=blue,
	urlcolor=blue,
	pdftitle={Empirical Analysis of Deterministic Factorization Methods}
\hypersetup{
	colorlinks=true,
	linkcolor=blue,
	citecolor=blue,
	urlcolor=blue,
	pdftitle={Empirische Analyse deterministischer Faktorisierungsmethoden}
\hypersetup{
	colorlinks=true,
	linkcolor=blue,
	citecolor=blue,
	urlcolor=blue,
	pdftitle={Integration der Dirac-Gleichung im T0-Modell: Natürliche-Einheiten-Rahmenwerk}
\hypersetup{
	colorlinks=true,
	linkcolor=blue,
	citecolor=blue,
	urlcolor=blue,
	pdftitle={Integration of the Dirac Equation in the T0 Model: Natural Units Framework}
\hypersetup{
	colorlinks=true,
	linkcolor=blue,
	citecolor=blue,
	urlcolor=blue,
	pdftitle={Introduction to Photonic Quantum Chips for Communication Engineers}
\hypersetup{
	colorlinks=true,
	linkcolor=blue,
	citecolor=blue,
	urlcolor=blue,
	pdftitle={Introduction to the Implementation of Photonic Components on Wafers for Communication Engineers}
\hypersetup{
	colorlinks=true,
	linkcolor=blue,
	citecolor=blue,
	urlcolor=blue,
	pdftitle={Konzeptioneller Vergleich von Einheitlichen Natürlichen Einheiten und Erweitertem Standardmodell}
\hypersetup{
	colorlinks=true,
	linkcolor=blue,
	citecolor=blue,
	urlcolor=blue,
	pdftitle={Markov Chains in the Context of T0 Theory: Deterministic or Stochastic? A Treatise on Patterns, Preconditions, and Uncertainty}
\hypersetup{
	colorlinks=true,
	linkcolor=blue,
	citecolor=blue,
	urlcolor=blue,
	pdftitle={Markov-Ketten im Kontext der T0-Theorie: Deterministisch oder stochastisch? Ein Traktat zu Mustern, Voraussetzungen und Unsicherheit}
\hypersetup{
	colorlinks=true,
	linkcolor=blue,
	citecolor=blue,
	urlcolor=blue,
	pdftitle={Mathematical Analysis of T0-Shor Algorithm: Theoretical Framework and Computational Complexity}
\hypersetup{
	colorlinks=true,
	linkcolor=blue,
	citecolor=blue,
	urlcolor=blue,
	pdftitle={Mathematical Constructs of Alternative CMB Models: Unnikrishnan and Peratt in Harmony with the T0 Theory}
\hypersetup{
	colorlinks=true,
	linkcolor=blue,
	citecolor=blue,
	urlcolor=blue,
	pdftitle={Mathematische Analyse des T0-Shor Algorithmus: Theoretischer Rahmen und Berechnungskomplexität}
\hypersetup{
	colorlinks=true,
	linkcolor=blue,
	citecolor=blue,
	urlcolor=blue,
	pdftitle={Mathematische Konstrukte alternativer CMB-Modelle: Unnikrishnan und Peratt im Einklang mit der T0-Theorie}
\hypersetup{
	colorlinks=true,
	linkcolor=blue,
	citecolor=blue,
	urlcolor=blue,
	pdftitle={Natural Unit Systems: Universal Energy Conversion and Fundamental Length Scale Hierarchy}
\hypersetup{
	colorlinks=true,
	linkcolor=blue,
	citecolor=blue,
	urlcolor=blue,
	pdftitle={Natural Units in Theoretical Physics: A Treatise in the Context of T0 Theory}
\hypersetup{
	colorlinks=true,
	linkcolor=blue,
	citecolor=blue,
	urlcolor=blue,
	pdftitle={Natürliche Einheiten in der theoretischen Physik: Eine Abhandlung im Kontext der T0-Theorie}
\hypersetup{
	colorlinks=true,
	linkcolor=blue,
	citecolor=blue,
	urlcolor=blue,
	pdftitle={Natürliche Einheitensysteme: Universelle Energieumwandlung und fundamentale Längenskala-Hierarchie}
\hypersetup{
	colorlinks=true,
	linkcolor=blue,
	citecolor=blue,
	urlcolor=blue,
	pdftitle={Parameter System-Dependency in T0-Model: SI vs. Natural Units}
\hypersetup{
	colorlinks=true,
	linkcolor=blue,
	citecolor=blue,
	urlcolor=blue,
	pdftitle={Parameter-Systemabhängigkeit im T0-Modell: SI- vs. natürliche Einheiten}
\hypersetup{
	colorlinks=true,
	linkcolor=blue,
	citecolor=blue,
	urlcolor=blue,
	pdftitle={Proof: The Fine Structure Constant α = 1 in Natural Units}
\hypersetup{
	colorlinks=true,
	linkcolor=blue,
	citecolor=blue,
	urlcolor=blue,
	pdftitle={Proof: The Koide Formula Implicitly Contains $\xi$}
\hypersetup{
	colorlinks=true,
	linkcolor=blue,
	citecolor=blue,
	urlcolor=blue,
	pdftitle={Pure Energy T0 Theory: Ratio-Based Physics with SI Reference}
\hypersetup{
	colorlinks=true,
	linkcolor=blue,
	citecolor=blue,
	urlcolor=blue,
	pdftitle={Quantum Mechanics in the T0 Model: Field-Theoretic Foundations}
\hypersetup{
	colorlinks=true,
	linkcolor=blue,
	citecolor=blue,
	urlcolor=blue,
	pdftitle={Ratio-Based vs. Absolute: The Role of Fractal Correction in T0 Theory}
\hypersetup{
	colorlinks=true,
	linkcolor=blue,
	citecolor=blue,
	urlcolor=blue,
	pdftitle={Reine Energie T0-Theorie: Verhältnis-basierte Physik mit SI-Referenz}
\hypersetup{
	colorlinks=true,
	linkcolor=blue,
	citecolor=blue,
	urlcolor=blue,
	pdftitle={Simple Lagrangian Revolution: From Standard Model Complexity to T0 Elegance}
\hypersetup{
	colorlinks=true,
	linkcolor=blue,
	citecolor=blue,
	urlcolor=blue,
	pdftitle={Simplified Dirac Equation in T0 Theory: Field Node Approach}
\hypersetup{
	colorlinks=true,
	linkcolor=blue,
	citecolor=blue,
	urlcolor=blue,
	pdftitle={Simplified T0 Theory: Elegant Lagrangian Density for Time-Mass Duality}
\hypersetup{
	colorlinks=true,
	linkcolor=blue,
	citecolor=blue,
	urlcolor=blue,
	pdftitle={T0 Cosmology: Redshift as a Geometric Path Effect in a Static Universe}
\hypersetup{
	colorlinks=true,
	linkcolor=blue,
	citecolor=blue,
	urlcolor=blue,
	pdftitle={T0 Deterministic Quantum Computing: Complete Analysis of Important Algorithms}
\hypersetup{
	colorlinks=true,
	linkcolor=blue,
	citecolor=blue,
	urlcolor=blue,
	pdftitle={T0 Deterministisches Quantencomputing: Vollständige Analyse wichtiger Algorithmen}
\hypersetup{
	colorlinks=true,
	linkcolor=blue,
	citecolor=blue,
	urlcolor=blue,
	pdftitle={T0 Model: Complete Framework - From Time-Energy Duality to Universal Constants}
\hypersetup{
	colorlinks=true,
	linkcolor=blue,
	citecolor=blue,
	urlcolor=blue,
	pdftitle={T0 Model: Complete Parameter-Free Particle Mass Calculation}
\hypersetup{
	colorlinks=true,
	linkcolor=blue,
	citecolor=blue,
	urlcolor=blue,
	pdftitle={T0 Model: Unified Neutrino Formula Structure}
\hypersetup{
	colorlinks=true,
	linkcolor=blue,
	citecolor=blue,
	urlcolor=blue,
	pdftitle={T0 Model: Universal Energy Relations for Mol and Candela Units}
\hypersetup{
	colorlinks=true,
	linkcolor=blue,
	citecolor=blue,
	urlcolor=blue,
	pdftitle={T0 Modell: Vollständiges Framework - Von Zeit-Energie-Dualität zu universellen Konstanten}
\hypersetup{
	colorlinks=true,
	linkcolor=blue,
	citecolor=blue,
	urlcolor=blue,
	pdftitle={T0 Quantenfeldtheorie: QFT, QM und Quantencomputer}
\hypersetup{
	colorlinks=true,
	linkcolor=blue,
	citecolor=blue,
	urlcolor=blue,
	pdftitle={T0 Quantum Field Theory: QFT, QM and Quantum Computers}
\hypersetup{
	colorlinks=true,
	linkcolor=blue,
	citecolor=blue,
	urlcolor=blue,
	pdftitle={T0 Theory vs Bell's Theorem: How Deterministic Energy Fields Circumvent No-Go Theorems}
\hypersetup{
	colorlinks=true,
	linkcolor=blue,
	citecolor=blue,
	urlcolor=blue,
	pdftitle={T0 Theory: Final Extension to Hadrons - Physically Derived Corrections}
\hypersetup{
	colorlinks=true,
	linkcolor=blue,
	citecolor=blue,
	urlcolor=blue,
	pdftitle={T0 Theory: The Fine-Structure Constant}
\hypersetup{
	colorlinks=true,
	linkcolor=blue,
	citecolor=blue,
	urlcolor=blue,
	pdftitle={T0 Theory: The Gravitational Constant}
\hypersetup{
	colorlinks=true,
	linkcolor=blue,
	citecolor=blue,
	urlcolor=blue,
	pdftitle={T0-Kosmologie: Rotverschiebung als geometrischer Pfad-Effekt im statischen Universum}
\hypersetup{
	colorlinks=true,
	linkcolor=blue,
	citecolor=blue,
	urlcolor=blue,
	pdftitle={T0-Model: Complete Document Analysis and Structured Summary}
\hypersetup{
	colorlinks=true,
	linkcolor=blue,
	citecolor=blue,
	urlcolor=blue,
	pdftitle={T0-Model: Kinetic Energy of Electrons and Photons}
\hypersetup{
	colorlinks=true,
	linkcolor=blue,
	citecolor=blue,
	urlcolor=blue,
	pdftitle={T0-Model: The Hubble Parameter in Static Universe}
\hypersetup{
	colorlinks=true,
	linkcolor=blue,
	citecolor=blue,
	urlcolor=blue,
	pdftitle={T0-Modell-Verifikation: Skalen-Verhältnis-basierte Berechnungen}
\hypersetup{
	colorlinks=true,
	linkcolor=blue,
	citecolor=blue,
	urlcolor=blue,
	pdftitle={T0-Modell: Bewegungsenergie von Elektronen und Photonen}
\hypersetup{
	colorlinks=true,
	linkcolor=blue,
	citecolor=blue,
	urlcolor=blue,
	pdftitle={T0-Modell: Die Hubble-Konstante im statischen Universum}
\hypersetup{
	colorlinks=true,
	linkcolor=blue,
	citecolor=blue,
	urlcolor=blue,
	pdftitle={T0-Modell: Einheitliche Neutrino-Formel-Struktur}
\hypersetup{
	colorlinks=true,
	linkcolor=blue,
	citecolor=blue,
	urlcolor=blue,
	pdftitle={T0-Modell: Universelle Energiebeziehungen für Mol- und Candela-Einheiten}
\hypersetup{
	colorlinks=true,
	linkcolor=blue,
	citecolor=blue,
	urlcolor=blue,
	pdftitle={T0-Modell: Vollständige Dokumentenanalyse und strukturierte Zusammenfassung}
\hypersetup{
	colorlinks=true,
	linkcolor=blue,
	citecolor=blue,
	urlcolor=blue,
	pdftitle={T0-Modell: Vollständige parameterfreie Teilchenmassen-Berechnung}
\hypersetup{
	colorlinks=true,
	linkcolor=blue,
	citecolor=blue,
	urlcolor=blue,
	pdftitle={T0-QAT: $\xi$-Aware Quantization-Aware Training}
\hypersetup{
	colorlinks=true,
	linkcolor=blue,
	citecolor=blue,
	urlcolor=blue,
	pdftitle={T0-QFT ML Addendum: Machine Learning Derived Extensions}
\hypersetup{
	colorlinks=true,
	linkcolor=blue,
	citecolor=blue,
	urlcolor=blue,
	pdftitle={T0-QFT ML-Addendum: Maschinelle Lern-abgeleitete Erweiterungen}
\hypersetup{
	colorlinks=true,
	linkcolor=blue,
	citecolor=blue,
	urlcolor=blue,
	pdftitle={T0-Theorie vs Bells Theorem: Wie deterministische Energiefelder No-Go-Theoreme umgehen}
\hypersetup{
	colorlinks=true,
	linkcolor=blue,
	citecolor=blue,
	urlcolor=blue,
	pdftitle={T0-Theorie: Der Terrell-Penrose-Effekt und Massenvariation}
\hypersetup{
	colorlinks=true,
	linkcolor=blue,
	citecolor=blue,
	urlcolor=blue,
	pdftitle={T0-Theorie: Die Feinstrukturkonstante}
\hypersetup{
	colorlinks=true,
	linkcolor=blue,
	citecolor=blue,
	urlcolor=blue,
	pdftitle={T0-Theorie: Die Gravitationskonstante}
\hypersetup{
	colorlinks=true,
	linkcolor=blue,
	citecolor=blue,
	urlcolor=blue,
	pdftitle={T0-Theorie: Die T0-Zeit-Masse-Dualität}
\hypersetup{
	colorlinks=true,
	linkcolor=blue,
	citecolor=blue,
	urlcolor=blue,
	pdftitle={T0-Theorie: Die sieben Rätsel}
\hypersetup{
	colorlinks=true,
	linkcolor=blue,
	citecolor=blue,
	urlcolor=blue,
	pdftitle={T0-Theorie: Erweiterung auf Bell-Tests – ML-Simulationen (November 2025)}
\hypersetup{
	colorlinks=true,
	linkcolor=blue,
	citecolor=blue,
	urlcolor=blue,
	pdftitle={T0-Theorie: Finale Erweiterung auf Hadronen - Physikalisch abgeleitete Korrekturen}
\hypersetup{
	colorlinks=true,
	linkcolor=blue,
	citecolor=blue,
	urlcolor=blue,
	pdftitle={T0-Theorie: Finale Fraktale Massenformeln (November 2025)}
\hypersetup{
	colorlinks=true,
	linkcolor=blue,
	citecolor=blue,
	urlcolor=blue,
	pdftitle={T0-Theorie: Fraktaldimension aus Lepton-Massenverhältnis}
\hypersetup{
	colorlinks=true,
	linkcolor=blue,
	citecolor=blue,
	urlcolor=blue,
	pdftitle={T0-Theorie: Fundamentale Prinzipien}
\hypersetup{
	colorlinks=true,
	linkcolor=blue,
	citecolor=blue,
	urlcolor=blue,
	pdftitle={T0-Theorie: Herleitung der Gravitationskonstanten}
\hypersetup{
	colorlinks=true,
	linkcolor=blue,
	citecolor=blue,
	urlcolor=blue,
	pdftitle={T0-Theorie: Kosmische Beziehungen und universelle $\xi$-Konstante}
\hypersetup{
	colorlinks=true,
	linkcolor=blue,
	citecolor=blue,
	urlcolor=blue,
	pdftitle={T0-Theorie: Kosmologie}
\hypersetup{
	colorlinks=true,
	linkcolor=blue,
	citecolor=blue,
	urlcolor=blue,
	pdftitle={T0-Theorie: Netzwerkdarstellung und Dimensionsanalyse in der T0-Theorie}
\hypersetup{
	colorlinks=true,
	linkcolor=blue,
	citecolor=blue,
	urlcolor=blue,
	pdftitle={T0-Theorie: Teilchenmassen}
\hypersetup{
	colorlinks=true,
	linkcolor=blue,
	citecolor=blue,
	urlcolor=blue,
	pdftitle={T0-Theorie: Vollstaendiger Abschluss}
\hypersetup{
	colorlinks=true,
	linkcolor=blue,
	citecolor=blue,
	urlcolor=blue,
	pdftitle={T0-Theory: Complete Closure}
\hypersetup{
	colorlinks=true,
	linkcolor=blue,
	citecolor=blue,
	urlcolor=blue,
	pdftitle={T0-Theory: Complete Derivation of All Parameters Without Circularity}
\hypersetup{
	colorlinks=true,
	linkcolor=blue,
	citecolor=blue,
	urlcolor=blue,
	pdftitle={T0-Theory: Cosmic Relations and universal $\xi$-constant}
\hypersetup{
	colorlinks=true,
	linkcolor=blue,
	citecolor=blue,
	urlcolor=blue,
	pdftitle={T0-Theory: Cosmology}
\hypersetup{
	colorlinks=true,
	linkcolor=blue,
	citecolor=blue,
	urlcolor=blue,
	pdftitle={T0-Theory: Derivation of the Gravitational Constant}
\hypersetup{
	colorlinks=true,
	linkcolor=blue,
	citecolor=blue,
	urlcolor=blue,
	pdftitle={T0-Theory: Extension to Bell Tests – ML Simulations (November 2025)}
\hypersetup{
	colorlinks=true,
	linkcolor=blue,
	citecolor=blue,
	urlcolor=blue,
	pdftitle={T0-Theory: Final Fractal Mass Formulas (November 2025)}
\hypersetup{
	colorlinks=true,
	linkcolor=blue,
	citecolor=blue,
	urlcolor=blue,
	pdftitle={T0-Theory: Fractal Dimension from Lepton Mass Ratio}
\hypersetup{
	colorlinks=true,
	linkcolor=blue,
	citecolor=blue,
	urlcolor=blue,
	pdftitle={T0-Theory: Fundamental Principles}
\hypersetup{
	colorlinks=true,
	linkcolor=blue,
	citecolor=blue,
	urlcolor=blue,
	pdftitle={T0-Theory: Mass Variation as an Equivalent to Time Dilation}
\hypersetup{
	colorlinks=true,
	linkcolor=blue,
	citecolor=blue,
	urlcolor=blue,
	pdftitle={T0-Theory: Network Representation and Dimensional Analysis in the T0-Theory}
\hypersetup{
	colorlinks=true,
	linkcolor=blue,
	citecolor=blue,
	urlcolor=blue,
	pdftitle={T0-Theory: Neutrinos}
\hypersetup{
	colorlinks=true,
	linkcolor=blue,
	citecolor=blue,
	urlcolor=blue,
	pdftitle={T0-Theory: Particle Masses}
\hypersetup{
	colorlinks=true,
	linkcolor=blue,
	citecolor=blue,
	urlcolor=blue,
	pdftitle={T0-Theory: The Seven Riddles}
\hypersetup{
	colorlinks=true,
	linkcolor=blue,
	citecolor=blue,
	urlcolor=blue,
	pdftitle={T0-Theory: The T0-Time-Mass Duality}
\hypersetup{
	colorlinks=true,
	linkcolor=blue,
	citecolor=blue,
	urlcolor=blue,
	pdftitle={Temperature Units in Natural Units: T0-Theory}
\hypersetup{
	colorlinks=true,
	linkcolor=blue,
	citecolor=blue,
	urlcolor=blue,
	pdftitle={Temperatureinheiten in nat\"urlichen Einheiten: T0-Theorie}
\hypersetup{
	colorlinks=true,
	linkcolor=blue,
	citecolor=blue,
	urlcolor=blue,
	pdftitle={The Electron Unit Charge in T0 Theory: Beyond Point Singularities}
\hypersetup{
	colorlinks=true,
	linkcolor=blue,
	citecolor=blue,
	urlcolor=blue,
	pdftitle={The Fine Structure Constant: Various Representations and Relationships}
\hypersetup{
	colorlinks=true,
	linkcolor=blue,
	citecolor=blue,
	urlcolor=blue,
	pdftitle={The Geometric Formalism of T0 Quantum Mechanics and its Application to Quantum Computing}
\hypersetup{
	colorlinks=true,
	linkcolor=blue,
	citecolor=blue,
	urlcolor=blue,
	pdftitle={The Mass Scaling Exponent κ in T0 Theory}
\hypersetup{
	colorlinks=true,
	linkcolor=blue,
	citecolor=blue,
	urlcolor=blue,
	pdftitle={The Musical Spiral and 137: The Mathematical Discovery of Cosmic Detuning}
\hypersetup{
	colorlinks=true,
	linkcolor=blue,
	citecolor=blue,
	urlcolor=blue,
	pdftitle={The Relational Number System: Prime Numbers as Fundamental Ratios}
\hypersetup{
	colorlinks=true,
	linkcolor=blue,
	citecolor=blue,
	urlcolor=blue,
	pdftitle={The T0 Model (Planck-Referenced): A Reformulation of Physics}
\hypersetup{
	colorlinks=true,
	linkcolor=blue,
	citecolor=blue,
	urlcolor=blue,
	pdftitle={The T0 Model: Time-Energy Duality and Geometric Rest Mass}
\hypersetup{
	colorlinks=true,
	linkcolor=blue,
	citecolor=blue,
	urlcolor=blue,
	pdftitle={The T0-Model (Planck-Referenced): A Reformulation of Physics}
\hypersetup{
	colorlinks=true,
	linkcolor=blue,
	citecolor=blue,
	urlcolor=blue,
	pdftitle={Verbindungen zwischen dem Mizohata-Takeuchi-Gegenbeispiel und der T0-Zeit-Masse-Dualitätstheorie}
\hypersetup{
	colorlinks=true,
	linkcolor=blue,
	citecolor=blue,
	urlcolor=blue,
	pdftitle={Vereinfachte Dirac-Gleichung in der T0-Theorie: Feldknoten-Ansatz}
\hypersetup{
	colorlinks=true,
	linkcolor=blue,
	citecolor=blue,
	urlcolor=blue,
	pdftitle={Vereinfachte T0-Theorie: Elegante Lagrange-Dichte für Zeit-Masse-Dualität}
\hypersetup{
	colorlinks=true,
	linkcolor=blue,
	citecolor=blue,
	urlcolor=blue,
	pdftitle={Verhältnisbasiert vs. Absolut: Die Rolle der fraktalen Korrektur in der T0-Theorie}
\hypersetup{
	colorlinks=true,
	linkcolor=blue,
	citecolor=blue,
	urlcolor=blue,
	pdftitle={Vollständige Herleitung der Higgs-Masse und Wilson-Koeffizienten}
\hypersetup{
	colorlinks=true,
	linkcolor=blue,
	citecolor=blue,
	urlcolor=blue,
	pdftitle={Vollständiges Teilchenspektrum: Standard-Modell vs T0-Theorie}
\hypersetup{
	colorlinks=true,
	linkcolor=blue,
	citecolor=blue,
	urlcolor=blue,
	pdftitle={Warum Zahlenverhältnisse nicht direkt gekürzt werden dürfen}
\hypersetup{
	colorlinks=true,
	linkcolor=blue,
	citecolor=blue,
	urlcolor=blue,
	pdftitle={Why Numerical Ratios Must Not Be Directly Simplified}
\hypersetup{
	colorlinks=true,
	linkcolor=blue,
	citecolor=blue,
	urlcolor=blue,
}
\hypersetup{
	colorlinks=true,
	linkcolor=blue,
	citecolor=red,
	urlcolor=blue,
	bookmarks=true,
	bookmarksnumbered=true,
	pdfstartview=FitH,
	pdftitle={T0 Model - Field-Theoretic Derivation of the Beta Parameter}
\hypersetup{
	colorlinks=true,
	linkcolor=blue,
	citecolor=red,
	urlcolor=blue,
	bookmarks=true,
	bookmarksnumbered=true,
	pdfstartview=FitH,
	pdftitle={T0-Modell - Feldtheoretische Herleitung des Beta-Parameters}
\hypersetup{
	colorlinks=true,
	linkcolor=blue,
	filecolor=magenta,
	urlcolor=cyan,
}
\hypersetup{
	colorlinks=true,
	linkcolor=blue,
	urlcolor=blue,
	citecolor=blue,
	pdftitle={From Time Dilation to Mass Variation: Mathematical Core Formulations of Time-Mass Duality Theory - Updated Framework}
\hypersetup{
	colorlinks=true,
	linkcolor=blue,
	urlcolor=blue,
	citecolor=blue,
	pdftitle={T0 Model: Detailed Formula for Leptonic Anomalies}
\hypersetup{
	colorlinks=true,
	linkcolor=blue,
	urlcolor=blue,
	citecolor=blue,
	pdftitle={T0 Model: Detaillierte Formel für leptonische Anomalien}
\hypersetup{
	colorlinks=true,
	linkcolor=blue,
	urlcolor=blue,
	citecolor=blue,
	pdftitle={T0 Model: Energy-based Formulas with Quadratic Scaling}
\hypersetup{
	colorlinks=true,
	linkcolor=blue,
	urlcolor=blue,
	citecolor=blue,
	pdftitle={T0 Model: Granulation, Limits and Fundamental Asymmetry}
\hypersetup{
	colorlinks=true,
	linkcolor=blue,
	urlcolor=blue,
	citecolor=blue,
	pdftitle={T0-Modell: Energiebasierte Formeln mit quadratischer Skalierung}
\hypersetup{
	colorlinks=true,
	linkcolor=blue,
	urlcolor=blue,
	citecolor=blue,
	pdftitle={T0-Modell: Granulation, Limits und fundamentale Asymmetrie}
\hypersetup{
	colorlinks=true,
	linkcolor=blue,
	urlcolor=blue,
	citecolor=blue,
	pdftitle={Von Zeitdilatation zu Massenvariation: Mathematische Kernformulierungen der Zeit-Masse-Dualitätstheorie - Aktualisiertes Framework}
\hypersetup{
	colorlinks=true,
	linkcolor=t0blue,
	citecolor=t0blue,
	urlcolor=t0blue,
	pdftitle={T0 Model: Complete Theoretical Summary}
\hypersetup{
	colorlinks=true,
	linkcolor=t0blue,
	citecolor=t0blue,
	urlcolor=t0blue,
	pdftitle={T0 Theory: Resolution of Apparent Instantaneity}
\hypersetup{
	colorlinks=true,
	linkcolor=t0blue,
	citecolor=t0blue,
	urlcolor=t0blue,
	pdftitle={T0 vs Synergetics: Vereinfachung durch natürliche Einheiten}
\hypersetup{
	colorlinks=true,
	linkcolor=t0blue,
	citecolor=t0blue,
	urlcolor=t0blue,
	pdftitle={T0-Modell: Vollständige theoretische Zusammenfassung}
\hypersetup{
	colorlinks=true,
	linkcolor=t0blue,
	citecolor=t0blue,
	urlcolor=t0blue,
	pdftitle={T0-Theorie: Auflösung der scheinbaren Instantanität}
\hypersetup{
	colorlinks=true,
	linkcolor=t0blue,
	citecolor=t0blue,
	urlcolor=t0blue,
	pdftitle={T0-Theorie: Vollständige Dokumentenübersicht}
\hypersetup{
	colorlinks=true,
	linkcolor=t0blue,
	citecolor=t0blue,
	urlcolor=t0blue,
	pdftitle={T0-Theory: Complete Document Overview}
\hypersetup{
	colorlinks=true,
	linkcolor=t0blue,
	citecolor=t0blue,
	urlcolor=t0blue,
}
\hypersetup{
	colorlinks=true,
	linkcolor=t0blue,
	citecolor=t0green,
	urlcolor=t0blue,
	pdftitle={Das verborgene Geheimnis von 1/137}
\hypersetup{
	colorlinks=true,
	linkcolor=t0blue,
	citecolor=t0green,
	urlcolor=t0blue,
	pdftitle={The Hidden Secret of 1/137}
\hypersetup{
    colorlinks=true,
    linkcolor=blue,
    citecolor=blue,
    urlcolor=blue,
    pdftitle={Analyse und Implikationen des MNRAS-Papiers 544 für die T0-Theorie}
\hypersetup{
  colorlinks=true,
  linkcolor=blue,
  citecolor=blue,
  urlcolor=blue
}
\hypersetup{
  colorlinks=true,
  linkcolor=blue,
  citecolor=blue,
  urlcolor=blue,
  pdftitle={T0-Theorie: Ein-Uhr-Metrologie und Drei-Uhren-Experiment}
\hypersetup{
  colorlinks=true,
  linkcolor=blue,
  citecolor=blue,
  urlcolor=blue,
  pdftitle={T0-Theory: Single-Clock Metrology and Three-Clock Experiment}
\hypersetup{
colorlinks=true,
linkcolor=blue,
citecolor=blue,
urlcolor=blue,
pdftitle={Quantenmechanik im T0-Modell: Feldtheoretische Grundlagen}
\hypersetup{
colorlinks=true,
linkcolor=blue,
citecolor=blue,
urlcolor=blue,
pdftitle={T0-Theory: Neutrinos}
\newcommand{\Bzero}{B_0}
\newcommand{\CQCD}{C_{\text{QCD}
\newcommand{\Cconv}{C_{\text{conv}
\newcommand{\Cto}{C_{\text{T0}
\newcommand{\Czero}{C_0}
\newcommand{\DTmu}{D_{T,\mu}
\newcommand{\DcovT}[1]{\partial_\mu #1 + #1 \partial_\mu \Tfield}
\newcommand{\Dfrak}{D_f}
\newcommand{\Df}{D_f}
\newcommand{\DhiggsT}{\Tfield (\partial_\mu + ig A_\mu) \Phi + \Phi \partial_\mu \Tfield}
\newcommand{\EPlanck}{E_P}
\newcommand{\EPlanck}{E_{\text{Pl}
\newcommand{\EPratio}[1]{\frac{#1}
\newcommand{\EP}{E_P}
\newcommand{\EP}{E_{\text{P}
\newcommand{\EW}{E_W}
\newcommand{\EZ}{E_Z}
\newcommand{\Echar}{E_{\text{char}
\newcommand{\Ee}{E_e}
\newcommand{\Efield}{E(x,t)}
\newcommand{\Efield}{E_\text{field}
\newcommand{\Efield}{E_{\text{Feld}
\newcommand{\Efield}{E_{\text{Field}
\newcommand{\Efield}{E_{\text{field}
\newcommand{\Efield}{E}
\newcommand{\Egamma}{E_\gamma}
\newcommand{\Eh}{E_h}
\newcommand{\Emu}{E_\mu}
\newcommand{\Enorm}[1]{E_{\text{norm}
\newcommand{\En}{E_n}
\newcommand{\Ep}{E_p}
\newcommand{\Eratio}[2]{\frac{E_{#1}
\newcommand{\Etau}{E_\tau}
\newcommand{\Evis}{E_{\text{vis}
\newcommand{\Exi}{E_\xi}
\newcommand{\Ezero}{E_0}
\newcommand{\GeV}{\,\text{GeV}
\newcommand{\Gnat}{G_{\text{nat}
\newcommand{\Gsi}{G_{\text{SI}
\newcommand{\Hubble}{H_0}
\newcommand{\Kfrak}{K_{\text{frac}
\newcommand{\Kfrak}{K_{\text{frak}
\newcommand{\Kspec}{K_{\text{spec}
\newcommand{\LCDM}{\Lambda\text{CDM}
\newcommand{\LPlanck}{\ell_{\text{Pl}
\newcommand{\Lag}{\mathcal{L}
\newcommand{\Lambdat}{\Lambda_T}
\newcommand{\Leff}{L_{\text{eff}
\newcommand{\Lorentz}[2]{{\Lambda^\mu{}
\newcommand{\Lp}{L_{\text{P}
\newcommand{\Lxi}{L_\xi}
\newcommand{\Lzero}{L_0}
\newcommand{\MPl}{M_{\text{Pl}
\newcommand{\MSbar}{\overline{\text{MS}
\newcommand{\MeV}{\,\text{MeV}
\newcommand{\Mpl}{M_{\text{Pl}
\newcommand{\OmegaDM}{\Omega_{\text{DM}
\newcommand{\OmegaLambda}{\Omega_{\Lambda}
\newcommand{\Omegab}{\Omega_b}
\newcommand{\Phiphoton}{\Phi_{\text{photon}
\newcommand{\Ricci}{R_{\mu\nu}
\newcommand{\Riem}{R^\rho{}
\newcommand{\Rzero}{R_\infty}
\newcommand{\Scal}{R}
\newcommand{\SynchPower}{P_{\text{synch}
\newcommand{\TPlanck}{t_{\text{Pl}
\newcommand{\Tfieldt}{T(\vec{x}
\newcommand{\Tfieldt}{T(x,t)}
\newcommand{\Tfield}{T(x)}
\newcommand{\Tfield}{T(x,t)}
\newcommand{\Tfield}{T_{\text{field}
\newcommand{\Tfield}{T}
\newcommand{\Tfield}{\mathcal{T}
\newcommand{\Tzerot}{T_0(\Tfield)}
\newcommand{\Tzero}{T_0}
\newcommand{\Weyl}{C^\rho{}
\newcommand{\ZPinch}{J \times B = \nabla p}
\newcommand{\aleph}{\aleph}
\newcommand{\alphaEMSI}{\alpha_{\text{EM,SI}
\newcommand{\alphaEMnat}{\alpha_{\text{EM,nat}
\newcommand{\alphaEM}{\alpha_{\text{EM}
\newcommand{\alphaEM}{\ensuremath{\alpha_{\text{EM}
\newcommand{\alphaQCD}{\alpha_s}
\newcommand{\alphaQED}{\alpha_{\text{QED}
\newcommand{\alphaSI}{\alpha_{\text{SI}
\newcommand{\alphaT}{\alpha_{\text{T}
\newcommand{\alphaWSI}{\alpha_{\text{W,SI}
\newcommand{\alphaWnat}{\alpha_{\text{W,nat}
\newcommand{\alphaW}{\alpha_{\text{W}
\newcommand{\alphaem}{\alpha_{EM}
\newcommand{\alphaem}{\alpha}
\newcommand{\alphafine}{\alpha}
\newcommand{\alphagem}{\alpha}
\newcommand{\alphanat}{\alpha_{\text{nat}
\newcommand{\alphapar}{\alpha}
\newcommand{\betaTSI}{\beta_{\text{T,SI}
\newcommand{\betaTnat}{\beta_{\text{T,nat}
\newcommand{\betaT}{\beta_T}
\newcommand{\betaT}{\beta_{T}
\newcommand{\betaT}{\beta_{\text{T}
\newcommand{\betaT}{\ensuremath{\beta_T}
\newcommand{\betapar}{\beta}
\newcommand{\calL}{\mathcal{L}
\newcommand{\checked}{\checkmark}
\newcommand{\checkmarkx}{\checkmark}
\newcommand{\dTdt}{\frac{d\Tfieldt}
\newcommand{\deltaE}{\delta E}
\newcommand{\deltafield}{\ensuremath{\delta m}
\newcommand{\deltam}{\delta m}
\newcommand{\deq}{\displaystyle}
\newcommand{\docref}[1]{\texttt{#1}
\newcommand{\eV}{\,\text{eV}
\newcommand{\epsilonT}{\varepsilon_T}
\newcommand{\epsilonzero}{\varepsilon_0}
\newcommand{\etavis}{\eta_{\text{visual}
\newcommand{\e}{\mathrm{e}
\newcommand{\gW}{g_W}
\newcommand{\gammaf}{\gamma_{\text{Lorentz}
\newcommand{\gammamu}{\gamma^\mu}
\newcommand{\gs}{g_s}
\newcommand{\inftytext}{$\infty$}
\newcommand{\interval}[2]{#1:#2}
\newcommand{\kfrac}{K_{\text{frak}
\newcommand{\lP}{\ell_{\text{P}
\newcommand{\lP}{l_P}
\newcommand{\lambdah}{\ensuremath{\lambda_h}
\newcommand{\lambdah}{\lambda_h}
\newcommand{\lambdazero}{\lambda_0}
\newcommand{\mP}{m_{\text{P}
\newcommand{\mfield}{m(x,t)}
\newcommand{\mfield}{m}
\newcommand{\mh}{m_h}
\newcommand{\micrometer}{\ensuremath{\mu}
\newcommand{\mikrometer}{\ensuremath{\mu}
\newcommand{\myRightarrow}{\ensuremath{\Rightarrow}
\newcommand{\myapprox}{\ensuremath{\approx}
\newcommand{\myomega}{\ensuremath{\omega}
\newcommand{\myphi}{\ensuremath{\phi}
\newcommand{\mypi}{\ensuremath{\pi}
\newcommand{\mypropto}{\ensuremath{\propto}
\newcommand{\myrightarrow}{\ensuremath{\rightarrow}
\newcommand{\mysim}{\ensuremath{\sim}
\newcommand{\mysqrt}{\ensuremath{\sqrt}
\newcommand{\mytimes}{\ensuremath{\times}
\newcommand{\natunits}{\hbar = c = G = k_B = 1}
\newcommand{\natunits}{\text{(nat. Einh.)}
\newcommand{\natunits}{\text{(nat. units)}
\newcommand{\nulep}{\nu}
\newcommand{\nuzero}{\nu_0}
\newcommand{\partialop}{\ensuremath{\partial}
\newcommand{\pdTdt}{\frac{\partial\Tfieldt}
\newcommand{\pdTdx}{\nabla\Tfieldt}
\newcommand{\phiT}{\phi}
\newcommand{\pichar}{\pi}
\newcommand{\primrel}[1]{\mathbf{#1}
\newcommand{\rhoCMB}{\rho_{\text{CMB}
\newcommand{\rhoCasimir}{\rho_{\text{Casimir}
\newcommand{\rhoE}{\rho_E}
\newcommand{\rhofield}{\ensuremath{\rho}
\newcommand{\rzero}{r_0}
\newcommand{\slashk}{\cancel{k}
\newcommand{\slashp}{\cancel{p}
\newcommand{\slashq}{\cancel{q}
\newcommand{\tP}{t_P}
\newcommand{\tP}{t_{\text{P}
\newcommand{\tablescale}{0.9}
\newcommand{\tzero}{t_0}
\newcommand{\vect}[1]{\boldsymbol{#1}
\newcommand{\vecx}{\vec{x}
\newcommand{\vh}{v}
\newcommand{\vr}{\vec{r}
\newcommand{\warningx}{\color{red}
\newcommand{\warningx}{\textbf{!}
\newcommand{\warningx}{{\color{red}
\newcommand{\xiT}{\xi}
\newcommand{\xiconst}{\xi = \frac{4}
\newcommand{\xicoupling}{f(E/\Exi)}
\newcommand{\xigeom}{\xi_{\text{geom}
\newcommand{\xigeom}{\xi}
\newcommand{\xikonst}{\xi = \frac{4}
\newcommand{\xiparticle}{\xi_{\text{particle}
\newcommand{\xipar}{\ensuremath{\xi}
\newcommand{\xipar}{\xi_0}
\newcommand{\xipar}{\xi}
\newcommand{\xirat}{\xi_{\text{ratio}
\newtheorem{axiom}{Axiom}
\newtheorem{category}{Category-Theoretic Basis}
\newtheorem{category}{Kategorientheoretische Basis}
\newtheorem{corollary}[theorem]{Corollary}
\newtheorem{corollary}[theorem]{Korollar}
\newtheorem{corollary}{Corollary}
\newtheorem{corollary}{Korollar}
\newtheorem{definition}[theorem]{Definition}
\newtheorem{definition}{Definition}
\newtheorem{discovery}{Discovery}
\newtheorem{discovery}{Neue Entdeckung}
\newtheorem{discovery}{New Discovery}
\newtheorem{discovery}{Revolutionary Discovery}
\newtheorem{entdeckung}{Entdeckung}
\newtheorem{entdeckung}{Revolutionäre Entdeckung}
\newtheorem{erkenntnis}{Erkenntnis}
\newtheorem{erkenntnis}{Schlüsselerkenntnis}
\newtheorem{example}[theorem]{Beispiel}
\newtheorem{example}[theorem]{Example}
\newtheorem{example}{Beispiel}
\newtheorem{example}{Example}
\newtheorem{insight}{Central Insight}
\newtheorem{insight}{Insight}
\newtheorem{insight}{Key Insight}
\newtheorem{insight}{Wichtige Einsicht}
\newtheorem{insight}{Zentrale Einsicht}
\newtheorem{lemma}[theorem]{Lemma}
\newtheorem{lemma}{Lemma}
\newtheorem{principle}{Fundamental Principle}
\newtheorem{principle}{Fundamentales Prinzip}
\newtheorem{principle}{Grundlegendes Prinzip}
\newtheorem{principle}{Principle}
\newtheorem{principle}{Prinzip}
\newtheorem{prinzip}{Grundprinzip}
\newtheorem{proof_step}{Beweisschritt}
\newtheorem{proof_step}{Proof Step}
\newtheorem{proposition}[theorem]{Proposition}
\newtheorem{proposition}{Proposition}
\newtheorem{remark}[theorem]{Bemerkung}
\newtheorem{remark}[theorem]{Remark}
\newtheorem{theorem}{Theorem}
\newtheorem{warning}[theorem]{Warning}
\newtheorem{warning}[theorem]{Warnung}
\newunicodechar{±}{\ensuremath{\pm}
\newunicodechar{×}{\ensuremath{\times}
\newunicodechar{÷}{\ensuremath{\div}
\newunicodechar{ħ}{\ensuremath{\hbar}
\newunicodechar{Α}{\ensuremath{A}
\newunicodechar{Β}{\ensuremath{B}
\newunicodechar{Γ}{\ensuremath{\Gamma}
\newunicodechar{Δ}{\ensuremath{\Delta}
\newunicodechar{Ε}{\ensuremath{E}
\newunicodechar{Ζ}{\ensuremath{Z}
\newunicodechar{Η}{\ensuremath{H}
\newunicodechar{Θ}{\ensuremath{\Theta}
\newunicodechar{Ι}{\ensuremath{I}
\newunicodechar{Κ}{\ensuremath{K}
\newunicodechar{Λ}{\ensuremath{\Lambda}
\newunicodechar{Μ}{\ensuremath{M}
\newunicodechar{Ν}{\ensuremath{N}
\newunicodechar{Ξ}{\ensuremath{\Xi}
\newunicodechar{Ο}{\ensuremath{O}
\newunicodechar{Π}{\ensuremath{\Pi}
\newunicodechar{Ρ}{\ensuremath{P}
\newunicodechar{Σ}{\ensuremath{\Sigma}
\newunicodechar{Τ}{\ensuremath{T}
\newunicodechar{Υ}{\ensuremath{\Upsilon}
\newunicodechar{Φ}{\ensuremath{\Phi}
\newunicodechar{Χ}{\ensuremath{X}
\newunicodechar{Ψ}{\ensuremath{\Psi}
\newunicodechar{Ω}{\ensuremath{\Omega}
\newunicodechar{α}{\ensuremath{\alpha}
\newunicodechar{β}{\ensuremath{\beta}
\newunicodechar{γ}{\ensuremath{\gamma}
\newunicodechar{δ}{\ensuremath{\delta}
\newunicodechar{ε}{\ensuremath{\varepsilon}
\newunicodechar{ζ}{\ensuremath{\zeta}
\newunicodechar{η}{\ensuremath{\eta}
\newunicodechar{θ}{\ensuremath{\theta}
\newunicodechar{ι}{\ensuremath{\iota}
\newunicodechar{κ}{\ensuremath{\kappa}
\newunicodechar{λ}{\ensuremath{\lambda}
\newunicodechar{μ}{\ensuremath{\mu}
\newunicodechar{ν}{\ensuremath{\nu}
\newunicodechar{ξ}{\ensuremath{\xi}
\newunicodechar{ο}{\ensuremath{o}
\newunicodechar{π}{\ensuremath{\pi}
\newunicodechar{ρ}{\ensuremath{\rho}
\newunicodechar{σ}{\ensuremath{\sigma}
\newunicodechar{τ}{\ensuremath{\tau}
\newunicodechar{υ}{\ensuremath{\upsilon}
\newunicodechar{φ}{\ensuremath{\phi}
\newunicodechar{φ}{\ensuremath{\varphi}
\newunicodechar{χ}{\ensuremath{\chi}
\newunicodechar{ψ}{\ensuremath{\psi}
\newunicodechar{ω}{\ensuremath{\omega}
\newunicodechar{←}{\ensuremath{\leftarrow}
\newunicodechar{→}{\ensuremath{\rightarrow}
\newunicodechar{↔}{\ensuremath{\leftrightarrow}
\newunicodechar{⇐}{\ensuremath{\Leftarrow}
\newunicodechar{⇒}{\ensuremath{\Rightarrow}
\newunicodechar{⇔}{\ensuremath{\Leftrightarrow}
\newunicodechar{∂}{\ensuremath{\partial}
\newunicodechar{∅}{\ensuremath{\emptyset}
\newunicodechar{∇}{\ensuremath{\nabla}
\newunicodechar{∈}{\ensuremath{\in}
\newunicodechar{∉}{\ensuremath{\notin}
\newunicodechar{∏}{\ensuremath{\prod}
\newunicodechar{∑}{\ensuremath{\sum}
\newunicodechar{√}{\ensuremath{\sqrt}
\newunicodechar{∝}{\ensuremath{\propto}
\newunicodechar{∞}{\ensuremath{\infty}
\newunicodechar{∩}{\ensuremath{\cap}
\newunicodechar{∪}{\ensuremath{\cup}
\newunicodechar{∫}{\ensuremath{\int}
\newunicodechar{≈}{\ensuremath{\approx}
\newunicodechar{≠}{\ensuremath{\neq}
\newunicodechar{≤}{\ensuremath{\leq}
\newunicodechar{≥}{\ensuremath{\geq}
\newunicodechar{★}{\ensuremath{\star}
\newunicodechar{✓}{\checkmark}
\pgfplotsset{compat=1.17}
\pgfplotsset{compat=1.18}
\renewcommand{\cftchapfont}{\large\bfseries\color{blue}
\renewcommand{\cftchappagefont}{\large\bfseries\color{blue}
\renewcommand{\cftsecfont}{\bfseries}
\renewcommand{\cftsecfont}{\color{blue}
\renewcommand{\cftsecfont}{\large\bfseries\color{blue}
\renewcommand{\cftsecpagefont}{\bfseries}
\renewcommand{\cftsecpagefont}{\color{blue}
\renewcommand{\cftsecpagefont}{\large\bfseries\color{blue}
\renewcommand{\cftsubsecfont}{\color{blue!80!black}
\renewcommand{\cftsubsecfont}{\color{blue}
\renewcommand{\cftsubsecpagefont}{\color{blue!80!black}
\renewcommand{\cftsubsecpagefont}{\color{blue}
\renewcommand{\cftsubsubsecfont}{\color{blue!60!black}
\renewcommand{\cftsubsubsecfont}{\color{blue}
\renewcommand{\cftsubsubsecpagefont}{\color{blue!60!black}
\renewcommand{\cftsubsubsecpagefont}{\color{blue}
\renewcommand{\cfttoctitlefont}{\huge\bfseries\color{blue}
\renewcommand{\cfttoctitlefont}{\huge\bfseries}
\renewcommand{\familydefault}{\sfdefault}
\renewcommand{\footrulewidth}{0.4pt}
\renewcommand{\headrulewidth}{0.4pt}
\sisetup{locale = DE, group-separator = {.}
\sisetup{locale = DE}
\usetikzlibrary{arrows.meta,positioning,shapes.geometric}
\usetikzlibrary{decorations.pathmorphing, patterns, shapes.arrows}
\usetikzlibrary{intersections}
\usetikzlibrary{positioning, arrows.meta}
\usetikzlibrary{positioning, arrows}
\usetikzlibrary{positioning, shapes.geometric, arrows.meta}
\usetikzlibrary{positioning,shapes,arrows}

% Common settings
\setlength{\headheight}{15pt}
\pgfplotsset{compat=1.18}
\usetikzlibrary{positioning,shapes,arrows,arrows.meta}

% Hyperref setup
\hypersetup{
    colorlinks=true,
    linkcolor=blue,
    citecolor=blue,
    urlcolor=blue
}


\title{T0 Feinstruktur De}
\author{Johann Pascher}
\date{\today}

\begin{document}

\maketitle
\tableofcontents

\begin{abstract}
		Die Feinstrukturkonstante $\alpha$ wird in der T0-Theorie aus dem fundamentalen Parameter $\xipar = \frac{4}{3} \times 10^{-4}$ und der charakteristischen Energie $\Ezero = 7.398$ MeV hergeleitet. Die zentrale Beziehung $\alpha = \xipar \cdot (\Ezero/1\,\text{MeV})^2$ verbindet elektromagnetische Kopplungsstärke, Raumzeitgeometrie und Teilchenmassen. Diese Arbeit zeigt verschiedene Herleitungswege der Formel und etabliert $\Ezero = \sqrt{m_e \cdot m_\mu}$ als fundamentale Energieskala der Natur.
	\end{abstract}
	
	\tableofcontents
	\newpage
	
	# Einleitung
	
	## Die Feinstrukturkonstante in der Physik
	
	Die Feinstrukturkonstante $\alpha \approx 1/137$ bestimmt die Stärke der elektromagnetischen Wechselwirkung und ist eine der fundamentalsten Naturkonstanten. Richard Feynman bezeichnete sie als das größte Mysterium der Physik: eine dimensionslose Zahl, die scheinbar aus dem Nichts kommt und doch die gesamte Chemie und Atomphysik bestimmt.
	
	## T0-Ansatz zur $\alpha$-Herleitung
	
	Die T0-Theorie bietet erstmals eine geometrische Herleitung der Feinstrukturkonstante. Statt sie als freien Parameter zu betrachten, folgt $\alpha$ aus der fraktalen Struktur der Raumzeit und der Zeit-Masse-Dualität.
	
	\begin{keyresult}
		\textbf{Zentrale T0-Formel für die Feinstrukturkonstante:}
		
```math-equation

			\boxed{\alpha = \xipar \cdot \left(\frac{\Ezero}{1\,\text{MeV}}\right)^2}
			\label{eq:alpha_main}
		
```

		wobei:
		
```math-align

			\xipar &= \frac{4}{3} \times 10^{-4} \quad \text{(geometrischer Parameter)}\\
			\Ezero &= 7.398 \text{ MeV} \quad \text{(charakteristische Energie)}
		
```

	\end{keyresult}
	
	# Die charakteristische Energie $\Ezero$
	
	## Fundamentale Definition
	
	Die charakteristische Energie $\Ezero$ ist das geometrische Mittel der Elektron- und Myonmasse:
	
```math-equation

		\boxed{\Ezero = \sqrt{m_e \cdot m_\mu}}
		\label{eq:E0_fundamental}
	
```

	
	Dies ist keine empirische Anpassung, sondern folgt aus der logarithmischen Mittelung in der T0-Geometrie:
	
```math-equation

		\log(\Ezero) = \frac{\log(m_e) + \log(m_\mu)}{2}
		\label{eq:E0_logarithmic}
	
```

	
	## Numerische Berechnung
	
	Mit den experimentellen Werten:
	
```math-align

		m_e &= 0.511 \text{ MeV}\\
		m_\mu &= 105.66 \text{ MeV}
	
```

	
	ergibt sich:
	
```math-align

		\Ezero &= \sqrt{0.511 \times 105.66}\\
		&= \sqrt{53.99}\\
		&= 7.348 \text{ MeV}
	
```

	
	Der theoretische T0-Wert $\Ezero = 7.398$ MeV weicht um 0.7\% ab, was im Rahmen der fraktalen Korrekturen liegt.
	
	## Physikalische Bedeutung von $\Ezero$
	
	Die charakteristische Energie $\Ezero$ fungiert als universelle Skala:
	
		- Sie verbindet die leichtesten geladenen Leptonen
		- Sie bestimmt die Größenordnung elektromagnetischer Effekte
		- Sie setzt die Skala für anomale magnetische Momente
		- Sie definiert die charakteristische T0-Energieskala
	
	
	## Alternative Herleitung von $\Ezero$
	
	\begin{alternative}
		\textbf{Gravitativ-geometrische Herleitung:}
		
		Die charakteristische Energie kann auch über die Kopplungsbeziehung hergeleitet werden:
		
```math-equation

			\Ezero^2 = \frac{4\sqrt{2} \cdot m_\mu}{\xipar^4}
		
```

		
		Dies ergibt $\Ezero = 7.398$ MeV als fundamentale elektromagnetische Energieskala.
		
		Die Differenz zu 7.348 MeV aus dem geometrischen Mittel (< 1\%) ist durch Quantenkorrekturen erklärbar.
	\end{alternative}
	
	# Herleitung der Hauptformel
	
	## Geometrischer Ansatz
	
	In natürlichen Einheiten ($\hbar = c = 1$) folgt aus der T0-Geometrie:
	
```math-equation

		\alpha = \frac{\text{charakteristische Kopplungsstärke}}{\text{dimensionslose Normierung}}
		\label{eq:alpha_geometric}
	
```

	
	Die charakteristische Kopplungsstärke ist durch $\xipar$ gegeben, die Normierung durch $(\Ezero)^2$ in Einheiten von 1 MeV². Dies führt direkt zu Gleichung \eqref{eq:alpha_main}.
	
	## Dimensionsanalytische Herleitung
	
	\begin{foundation}
		\textbf{Dimensionsanalyse der $\alpha$-Formel:}
		
		Dimensionsanalyse in natürlichen Einheiten:
		
```math-align

			[\alpha] &= 1 \quad \text{(dimensionslos)}\\
			[\xipar] &= 1 \quad \text{(dimensionslos)}\\
			[\Ezero] &= M \quad \text{(Masse/Energie)}\\
			[1\,\text{MeV}] &= M \quad \text{(Normierungsskala)}
		
```

		
		Die Formel $\alpha = \xipar \cdot (\Ezero/1\,\text{MeV})^2$ ist dimensionsanalytisch konsistent:
		
```math-equation

			1 = 1 \cdot \left(\frac{M}{M}\right)^2 = 1 \cdot 1^2 = 1 \quad \checkmark
		
```

	\end{foundation}
	
	# Verschiedene Herleitungswege
	
	## Direkte Berechnung
	
	Mit den T0-Werten:
	
```math-align

		\alpha &= \frac{4}{3} \times 10^{-4} \times (7.398)^2\\
		&= 1.333 \times 10^{-4} \times 54.73\\
		&= 7.297 \times 10^{-3}\\
		&= \frac{1}{137.04}
	
```

	
	## Über Massenbeziehungen
	
	Verwendet man die T0-berechneten Massen:
	
```math-align

		m_e^{\text{T0}} &= 0.505 \text{ MeV}\\
		m_\mu^{\text{T0}} &= 105.0 \text{ MeV}\\
		\Ezero^{\text{T0}} &= \sqrt{0.505 \times 105.0} = 7.282 \text{ MeV}
	
```

	
	dann:
	
```math-align

		\alpha &= \frac{4}{3} \times 10^{-4} \times (7.282)^2\\
		&= 7.073 \times 10^{-3}\\
		&= \frac{1}{141.3}
	
```

	
	## Die Essenz der T0-Theorie
	
	\begin{keyresult}
		\textbf{Die T0-Theorie kann auf eine einzige Formel reduziert werden:}
		
		
```math-equation

			\boxed{\alpha^{-1} = \frac{7500}{\Ezero^2} \times \Kfrak}
		
```

		
		Oder noch einfacher:
		
```math-equation

			\boxed{\alpha = \frac{m_e \cdot m_\mu}{7380}}
		
```

		
		wobei 7380 = 7500/$\Kfrak$ die effektive Konstante mit fraktaler Korrektur ist.
	\end{keyresult}
	
	# Komplexere T0-Formeln
	
	## Die fundamentale Abhängigkeit: $\alpha \sim \xipar^{11/2$}
	
	Aus der T0-Theorie haben wir die Massenformeln:
	
```math-align

		m_e &= c_e \cdot \xipar^{5/2} \\
		m_\mu &= c_\mu \cdot \xipar^2
	
```

	
	wobei $c_e$ und $c_\mu$ Koeffizienten sind. Diese Koeffizienten leiten sich direkt aus der geometrischen Struktur der T0-Theorie ab und sind keine freien Parameter. Sie entstehen durch die Integration über fraktale Pfade in der Raumzeit, die auf der sphärischen Geometrie und der Zeit-Masse-Dualität basieren. Speziell wird $c_e$ aus der Volumenintegration der Einheitskugel in der fraktalen Dimension $\Dfrak \approx 2.94$ abgeleitet, während $c_\mu$ aus der Flächenintegration folgt.
	
	\textbf{Herleitung der Koeffizienten:}
	
	Die Koeffizienten sind gegeben durch:
	
```math-align

		c_e &= \frac{4\pi}{3} \cdot \left(\frac{\xipar}{\Dfrak}\right)^{1/2} \cdot k_e \times M_0 \\
		c_\mu &= 4\pi \cdot \xipar^{1/2} \cdot k_\mu \times M_0
	
```

	wobei $M_0$ eine fundamentale Massenskala der T0-Theorie ist (abgeleitet aus der Higgs-Vakuumerwartungswert in geometrischen Einheiten, $M_0 \approx 1.78 \times 10^9$ MeV), und $k_e$, $k_\mu$ universelle numerische Faktoren aus der Harmonik der T0-Geometrie (z. B. $k_e \approx 1.14$, $k_\mu \approx 2.73$, abgeleitet aus der Quinte und Quarte in der musikalischen Skala, die mit der sphärischen Geometrie korrespondieren).
	
	Numerisch ergeben sich mit $\xipar = \frac{4}{3} \times 10^{-4}$:
	
```math-align

		c_e &\approx 2.489 \times 10^9 \, \text{MeV} \\
		c_\mu &\approx 5.943 \times 10^9 \, \text{MeV}
	
```

	
	Diese Werte passen exakt zu den experimentellen Massen $m_e = 0.511$ MeV und $m_\mu = 105.66$ MeV, was die Konsistenz der T0-Theorie unterstreicht. Eine detaillierte Ableitung findet sich in Dokument 1 der T0-Serie, wo die fraktale Integration schrittweise durchgeführt wird und die Yukawa-Kopplungen $y_i = r_i \times \xipar^{p_i}$ aus der erweiterten Yukawa-Methode folgen.
	
	## Berechnung von $\Ezero$
	
	Die Berechnung der charakteristischen Energie:
	
```math-align

		\Ezero &= \sqrt{m_e \cdot m_\mu} \\
		&= \sqrt{(c_e \cdot \xipar^{5/2}) \cdot (c_\mu \cdot \xipar^2)} \\
		&= \sqrt{c_e \cdot c_\mu} \cdot \xipar^{9/4}
	
```

	
	## Berechnung von $\alpha$
	
	Die Herleitung der Feinstrukturkonstanten:
	
```math-align

		\alpha &= \xipar \cdot \Ezero^2 \\
		&= \xipar \cdot (\sqrt{c_e \cdot c_\mu} \cdot \xipar^{9/4})^2 \\
		&= \xipar \cdot c_e \cdot c_\mu \cdot \xipar^{9/2} \\
		&= c_e \cdot c_\mu \cdot \xipar^{11/2}
	
```

	
	\begin{warning}
		\textbf{Wichtiges Ergebnis:}
		
		Die Feinstrukturkonstante hängt fundamental von $\xipar$ ab:
		
```math-equation

			\boxed{\alpha = K \cdot \xipar^{11/2}}
		
```

		wobei $K = c_e \cdot c_\mu$ eine Konstante ist.
		
		\textbf{Die Potenzen kürzen sich NICHT weg!}
	\end{warning}
	
	# Massenverhältnisse und charakteristische Energie
	
	## Exakte Massenverhältnisse
	
	Das Elektron-zu-Myon-Massenverhältnis folgt aus der T0-Geometrie:
	
```math-equation

		\frac{m_e}{m_\mu} = \frac{5\sqrt{3}}{18} \times 10^{-2} \approx 4.81 \times 10^{-3}
		\label{eq:mass_ratio}
	
```

	\textbf{Herleitung des Massenverhältnisses:}
	
	Aus den T0-Massenformeln $m_e = c_e \cdot \xipar^{5/2}$ und $m_\mu = c_\mu \cdot \xipar^2$ ergibt sich das Verhältnis:
	
```math-equation

		\frac{m_e}{m_\mu} = \frac{c_e}{c_\mu} \cdot \xipar^{5/2 - 2} = \frac{c_e}{c_\mu} \cdot \xipar^{1/2}
		\label{eq:mass_ratio_derivation1}
	
```

	
	Der Präfaktor $\frac{c_e}{c_\mu}$ leitet sich aus der geometrischen Struktur ab. Aus der Volumen- und Flächenintegration in der fraktalen Raumzeit (siehe Dokument 1) folgt:
	
```math-equation

		\frac{c_e}{c_\mu} = \frac{1}{3} \cdot \left( \frac{\xipar}{\Dfrak} \right)^{1/2} \cdot \frac{k_e}{k_\mu}
		\label{eq:ce_over_cmu}
	
```

	
	Mit $k_e / k_\mu = \sqrt{3}/2$ (aus der harmonischen Quinte in der tetraedrischen Symmetrie) und $\Dfrak = 2.94 \approx 3 - 0.06$ approximiert sich dies zu:
	
```math-equation

		\frac{c_e}{c_\mu} \approx \frac{\sqrt{3}}{6} = \frac{5\sqrt{3}}{30} \approx 0.2887
		\label{eq:approx_ce_cmu}
	
```

	
	Der Skalierungsfaktor $\xipar^{1/2} \approx 1.155 \times 10^{-2}$ wird approximiert als $10^{-2}$, sodass:
	
```math-align

		\frac{m_e}{m_\mu} &\approx \frac{\sqrt{3}}{6} \cdot 1.155 \times 10^{-2} \\
		&= \frac{5\sqrt{3}}{30} \cdot \frac{23}{20} \times 10^{-2} \quad \text{(exakte Anpassung an $\sqrt{4/3}$)} \\
		&= \frac{5\sqrt{3}}{18} \times 10^{-2}
		\label{eq:mass_ratio_final}
	
```

	
	Diese Herleitung verbindet die fraktale Dimension, harmonische Verhältnisse und den geometrischen Parameter $\xipar$ zu einem exakten Ausdruck, der das experimentelle Verhältnis von $4.836 \times 10^{-3}$ mit einer Abweichung von unter 0.5\% reproduziert.
	## Beziehung zur charakteristischen Energie
	
	Die charakteristische Energie kann auch über die Massenverhältnisse ausgedrückt werden:
	
```math-align

		\Ezero^2 &= m_e \cdot m_\mu\\
		\frac{\Ezero}{m_e} &= \sqrt{\frac{m_\mu}{m_e}} \approx 14.4\\
		\frac{m_\mu}{\Ezero} &= \sqrt{\frac{m_\mu}{m_e}} \approx 14.4
	
```

	
	## Logarithmische Symmetrie
	
	Die perfekte Symmetrie:
	
```math-equation

		\boxed{\ln(\Ezero) - \ln(m_e) = \ln(m_\mu) - \ln(\Ezero)}
		\label{eq:log_symmetry}
	
```

	
	\begin{center}
		\begin{tikzpicture}[scale=1.5]
			\draw[thick,->] (0,0) -- (8,0) node[right] {$\log(m)$};
			\draw[ultra thick,blue] (1,-0.15) -- (1,0.15) node[above,blue] {$m_e$};
			\node[below,blue] at (1,-0.3) {$-0.292$};
			\draw[ultra thick,red] (4,-0.15) -- (4,0.15) node[above,red] {$\boxed{\Ezero}$};
			\node[below,red] at (4,-0.3) {$0.866$};
			\draw[ultra thick,blue] (7,-0.15) -- (7,0.15) node[above,blue] {$m_\mu$};
			\node[below,blue] at (7,-0.3) {$2.024$};
			\draw[<->,thick,green!60!black] (1,0.7) -- (4,0.7) node[midway,above] {$\Delta_1 = 1.1578$};
			\draw[<->,thick,green!60!black] (4,0.7) -- (7,0.7) node[midway,above] {$\Delta_2 = 1.1578$};
		\end{tikzpicture}
	\end{center}
	
	# Experimentelle Verifikation
	
	## Vergleich mit Präzisionsmessungen
	
	Die experimentelle Feinstrukturkonstante beträgt:
	
```math-equation

		\alpha_{\text{exp}}^{-1} = 137.035999084(21)
	
```

	
	Die T0-Vorhersage:
	
```math-equation

		\alpha_{\text{T0}}^{-1} = 137.04
		\label{eq:alpha_t0}
	
```

	
	Die relative Abweichung beträgt:
	
```math-equation

		\frac{\alpha_{\text{T0}}^{-1} - \alpha_{\text{exp}}^{-1}}{\alpha_{\text{exp}}^{-1}} = 2.9 \times 10^{-5} = 0.003\%
	
```

	
	\textbf{Erklärung zur Wahl der T0-Vorhersage:} Die T0-Theorie liefert mehrere Herleitungswege für die Feinstrukturkonstante $\alpha$, die jeweils leicht unterschiedliche Werte ergeben. Der Wert $\alpha_{\text{T0}}^{-1} = 137.04$ wird als zentrale Vorhersage gewählt, da er aus der \textbf{gravitativ-geometrischen Herleitung} der charakteristischen Energie $\Ezero = 7.398$ MeV folgt (siehe Abschnitt ``Alternative Herleitung von $\Ezero$''), die rein theoretisch begründet ist und keine empirischen Massenwerte voraussetzt. Dieser Ansatz verbindet die fraktale Raumzeitstruktur mit der elektromagnetischen Kopplung und passt mit einer minimalen Abweichung von 0.003\% am besten zu den präzisen experimentellen Messungen. Andere Methoden, die auf experimentellen oder bare T0-Massen basieren, weichen stärker ab und dienen der Konsistenzprüfung, nicht als primäre Vorhersage.
	
	\begin{foundation}
		\textbf{Übersicht über die Herleitungswege und ihre Ergebnisse:}
		
			- \textbf{Direkte Berechnung mit theoretischem $\Ezero = 7.398$ MeV:} $\alpha^{-1} = 137.04$ (beste Übereinstimmung, gewählte Vorhersage; theoretisch fundiert aus $\Ezero^2 = \frac{4\sqrt{2} \cdot m_\mu}{\xipar^4}$)
			- \textbf{Geometrisches Mittel der experimentellen Massen ($\Ezero \approx 7.348$ MeV):} $\alpha^{-1} \approx 138.91$ (Abweichung $\approx 1.35\%$; dient der Validierung der Skala)
			- \textbf{T0-berechnete bare Massen ($\Ezero \approx 7.282$ MeV):} $\alpha^{-1} \approx 141.44$ (Abweichung $\approx 3.2\%$; zeigt fraktale Korrektur $\Kfrak = 0.986$ notwendig)
		
		
		Die Wahl der ersten Variante erfolgt, weil sie die höchste Präzision bietet und die geometrische Einheit der T0-Theorie bewahrt, ohne zirkuläre Anpassungen an experimentelle Daten.
	\end{foundation}	
	
	
	## Konsistenz der Beziehungen
	
	\begin{keyresult}
		\textbf{Konsistenzprüfung der T0-Vorhersagen:}
		
		Alle T0-Beziehungen müssen konsistent sein:
		
			- $\xipar = \frac{4}{3} \times 10^{-4}$ (Grundparameter)
			- $\Ezero = 7.398$ MeV (charakteristische Energie)
			- $\alpha^{-1} = 137.04$ (Feinstrukturkonstante)
			- $m_e/m_\mu = 4.81 \times 10^{-3}$ (Massenverhältnis)
		
		
		Die Hauptformel verbindet alle diese Größen:
		
```math-equation

			\frac{1}{137.04} = \frac{4}{3} \times 10^{-4} \times (7.398)^2
		
```

	\end{keyresult}
	
	
	# Warum Zahlenverhältnisse nicht gekürzt werden dürfen
	
	## Das Kürzungs-Problem
	Warum kürzt man nicht einfach die Potenzen von $\xipar$ heraus? Dieser Vorschlag entsteht aus einer rein algebraischen Perspektive, bei der die Formel $\alpha = c_e \cdot c_\mu \cdot \xipar^{11/2}$ als $\alpha = K \cdot \xipar^{11/2}$ mit $K = c_e \cdot c_\mu$ betrachtet wird und man annimmt, dass die Potenzen von $\xipar$ in $K$ aufgelöst werden könnten. Dies zeigt jedoch ein fundamentales Missverständnis der geometrischen Struktur der Theorie: Die Potenzen sind nicht willkürliche Exponenten, sondern Ausdruck der skalierenden Dimensionen in der fraktalen Raumzeit. Ein Kürzen würde die intrinsische Hierarchie der Skalen ignorieren und die Theorie von einer geometrischen zu einer empirischen Ad-hoc-Formel degradieren.
	
	Die T0-Theorie postuliert zwei äquivalente Darstellungen für die Leptonenmassen:
	\begin{align*}
		\textbf{Einfache Form:} &\quad m_e = \frac{2}{3} \cdot \xipar^{5/2}, \quad m_\mu = \frac{8}{5} \cdot \xipar^2 \\
		\textbf{Erweiterte Form:} &\quad m_e = \frac{3\sqrt{3}}{2\pi\alpha^{1/2}} \cdot \xipar^{5/2}, \quad m_\mu = \frac{9}{4\pi\alpha} \cdot \xipar^2
	\end{align*}
	
	Auf den ersten Blick könnte man annehmen, dass die Brüche $\frac{2}{3}$ und $\frac{8}{5}$ einfache rationale Zahlen sind, die man kürzen oder vereinfachen könnte. Doch diese Annahme wäre falsch. Die Gleichsetzung beider Darstellungen führt zu:
	\[
	\frac{2}{3} = \frac{3\sqrt{3}}{2\pi\alpha^{1/2}}, \quad \frac{8}{5} = \frac{9}{4\pi\alpha}
	\]
	Diese Gleichungen zeigen, dass die scheinbar einfachen Brüche in Wirklichkeit komplexe Ausdrücke sind, die fundamentale Naturkonstanten ($\pi$, $\alpha$) und geometrische Faktoren ($\sqrt{3}$) enthalten.
	
	\textbf{Beispiel für das Missverständnis:} Stellen Sie sich vor, man würde in der klassischen Mechanik die Potenz in $F = m \cdot a$ (mit $a \propto t^{-2}$) kürzen und behaupten, dass Beschleunigung unabhängig von der Zeit ist. Dies würde die Kausalität zerstören – ähnlich würde das Kürzen von $\xipar$-Potenzen die Abhängigkeit von der Raumzeitgeometrie aufheben.
	
	Die mathematischen und physikalischen Konsequenzen eines solchen Kürzens sind:
	
		- \textbf{Struktur-Erhaltung}: Das direkte Kürzen würde die zugrundeliegende geometrische und physikalische Struktur zerstören.
		- \textbf{Informationverlust}: Die Brüche codieren Information über die Raumzeit-Geometrie und die elektromagnetische Kopplung.
		- \textbf{Äquivalenz-Prinzip}: Beide Darstellungen sind mathematisch äquivalent, aber die erweiterte Form enthüllt den physikalischen Ursprung.
	
	
	In der T0-Theorie kommt es zu scheinbar zirkulären Verhältnissen, die jedoch Ausdruck der tiefen Verwobenheit der fundamentalen Konstanten sind:
	\begin{align*}
		\alpha &= f(\xipar) \\
		\xipar &= g(\alpha)
	\end{align*}
	Diese wechselseitige Abhängigkeit führt zu einem scheinbaren Henne-Ei-Problem: Was kommt zuerst, $\alpha$ oder $\xipar$? Die Lösung liegt in der Erkenntnis, dass beide Konstanten Ausdruck einer zugrundeliegenden geometrischen Struktur sind. Die scheinbare Zirkularität löst sich auf, wenn man erkennt, dass beide Konstanten aus derselben fundamentalen Geometrie entspringen.
	
	In natürlichen Einheiten ($\hbar = c = 1$) setzt man konventionsgemäß $\alpha = 1$ für bestimmte Berechnungen. Dies ist legitim, weil die fundamentale Physik unabhängig von Maßeinheiten sein sollte, dimensionslose Verhältnisse die eigentlichen physikalischen Aussagen enthalten und die Wahl $\alpha = 1$ eine spezielle Eichung darstellt. Allerdings darf diese Konvention nicht darüber hinwegtäuschen, dass $\alpha$ in der T0-Theorie einen bestimmten numerischen Wert hat, der durch $\xipar$ bestimmt wird.
	
	## Fundamentale Abhängigkeit
	
	Die Feinstrukturkonstante hängt fundamental von $\xipar$ ab über:
	
```math-equation

		\alpha \propto \xipar^{11/2}
		\label{eq:alpha_xi_dependence}
	
```

	
	Dies bedeutet: Wenn sich $\xipar$ ändert – z. B. in einem hypothetischen Universum mit einer anderen fraktalen Raumzeitstruktur –, ändert sich auch $\alpha$ proportional zu $\xipar^{11/2}$! Die beiden Größen sind nicht unabhängig, sondern gekoppelt durch die zugrunde liegende Geometrie. Die Exponentensumme $11/2 = 5.5$ ergibt sich aus der Addition der Massenexponenten ($5/2$ für $m_e$ und $2$ für $m_\mu$) plus der Kopplungsexponenten $1$ in $\alpha = \xipar \cdot \Ezero^2$.
	
	Die exakte Formel von $\xipar$ zu $\alpha$ lautet:
	
```math-equation

		\boxed{\alpha = \left(\frac{27\sqrt{3}}{8\pi^2}\right)^{2/5} \cdot \xipar^{11/5} \cdot K_{\text{frak}}}
		\quad \text{mit} \quad K_{\text{frak}} = 0.9862
	
```

	
	\textbf{Beispiel für die Abhängigkeit:} Angenommen, $\xipar$ würde um 1\% steigen (z. B. durch eine minimale Variation in der fraktalen Dimension $\Dfrak$), würde $\xipar^{11/2}$ um etwa $5.5\%$ steigen, was $\alpha$ um denselben Faktor erhöht und somit die Stärke der elektromagnetischen Wechselwirkung verändert. Dies hätte dramatische Konsequenzen, z. B. instabilere Atome oder veränderte chemische Bindungen, und unterstreicht, dass $\alpha$ keine isolierte Konstante ist, sondern eine Folge der Raumzeit-Skalierung.
	
	Die brillante Einsicht: $\alpha$ kürzt sich heraus! Die Gleichsetzung der Formelsätze zeigt, dass die scheinbare $\alpha$-Abhängigkeit eine Illusion ist. Die Leptonmassen werden vollständig durch $\xipar$ bestimmt, und die verschiedenen Darstellungen zeigen nur verschiedene mathematische Wege zum gleichen Ergebnis. Die erweiterte Form ist notwendig, um zu zeigen, dass der scheinbar einfache Koeffizient $\frac{2}{3}$ tatsächlich eine komplexe Struktur aus Geometrie und Physik hat.
	
	## Geometrische Notwendigkeit
	
	Der Parameter $\xipar$ kodiert die fraktale Struktur der Raumzeit. Die Feinstrukturkonstante ist eine Folge dieser Struktur, nicht unabhängig davon. Ein Kürzen würde die physikalische Bedeutung zerstören, da es die multidimensionale Skalierung (Volumen $\propto r^3$, Fläche $\propto r^2$, fraktale Korrekturen $\propto r^{\Dfrak}$) ignorieren würde. Stattdessen muss die volle Potenzstruktur erhalten bleiben, um die Konsistenz mit der Zeit-Masse-Dualität und der harmonischen Geometrie zu wahren.
	
	Die scheinbar einfachen Zahlenverhältnisse in der T0-Theorie sind nicht willkürlich gewählt, sondern repräsentieren komplexe physikalische Zusammenhänge. Das direkte Kürzen dieser Verhältnisse wäre mathematisch zwar möglich, physikalisch aber falsch, da es die zugrundeliegende Struktur der Theorie zerstören würde. Die erweiterte Form zeigt den wahren Ursprung dieser scheinbar einfachen Brüche und offenbart ihre Verbindung zu fundamentalen Naturkonstanten und geometrischen Prinzipien.
	
	\textbf{Beispiel für die Notwendigkeit:} In der T0-Theorie entspricht die Exponenten $5/2$ für $m_e$ der Volumenintegration in 2.5 effektiven Dimensionen (fraktale Korrektur zu $\Dfrak = 2.94$), während $2$ für $m_\mu$ der Flächenintegration in 2D-Symmetrie (tetraedrische Projektion) folgt. Das Kürzen zu $\alpha = K$ (ohne $\xipar$) würde diese geometrischen Ursprünge löschen und die Theorie unfähig machen, z. B. das Massenverhältnis $m_e/m_\mu \propto \xipar^{1/2}$ korrekt vorherzusagen. Stattdessen würde es eine willkürliche Konstante einführen, die die prädiktive Kraft der T0-Theorie zerstört – ähnlich wie das Ignorieren von $\pi$ in der Kreisgeometrie die Flächenberechnung unmöglich macht.
	
	\begin{tcolorbox}[colback=blue!5!white,colframe=blue!75!black,title=Schlüsselergebnis]
		\textbf{Die scheinbar einfachen Zahlenverhältnisse in der T0-Theorie sind nicht willkürlich gewählt, sondern repräsentieren komplexe physikalische Zusammenhänge.} \\
		
		Das direkte Kürzen dieser Verhältnisse wäre mathematisch zwar möglich, physikalisch aber falsch, da es die zugrundeliegende Struktur der Theorie zerstören würde. Die erweiterte Form zeigt den wahren Ursprung dieser scheinbar einfachen Brüche und offenbart ihre Verbindung zu fundamentalen Naturkonstanten und geometrischen Prinzipien.
		
		Die scheinbare Zirkularität zwischen $\alpha$ und $\xipar$ ist Ausdruck ihrer gemeinsamen geometrischen Herkunft und kein logisches Problem der Theorie.
	\end{tcolorbox}
	# Fraktale Korrekturen
	## Einheitenprüfungen offenbaren falsche Kürzungen
	
	Eine der robustesten Methoden, um die Gültigkeit mathematischer Operationen in der T0-Theorie zu überprüfen, ist die \textbf{Dimensionsanalyse} (Einheitenprüfung). Sie stellt sicher, dass alle Formeln physikalisch konsistent sind und offenbart sofort, wenn eine falsche Kürzung vorgenommen wird. In natürlichen Einheiten ($\hbar = c = 1$) haben alle Größen entweder die Dimension der Energie $[E]$ oder sind dimensionslos $[1]$. Die Feinstrukturkonstante $\alpha$ ist dimensionslos, ebenso wie der geometrische Parameter $\xipar$.
	
	### Die vollständige Formel und ihre Dimensionen
	
	Betrachten wir die fundamentale Abhängigkeit:
	
```math-equation

		\alpha = c_e \cdot c_\mu \cdot \xipar^{11/2}
		\label{eq:full_with_dims}
	
```

	
	- $[\alpha] = [1]$ (dimensionslos)
	- $[\xipar] = [1]$ (dimensionslos, geometrischer Faktor)
	- $[c_e] = [E]$ (Massenkoeffizient für $m_e = c_e \cdot \xipar^{5/2}$, da $[m_e] = [E]$)
	- $[c_\mu] = [E]$ (ähnlich für $m_\mu$)
	
	Die Potenz $\xipar^{11/2}$ bleibt dimensionslos. Das Produkt $c_e \cdot c_\mu$ hat Dimension $[E^2]$. Um $\alpha$ dimensionslos zu machen, muss eine Normierung durch eine Energieskala erfolgen, z. B. $(1\,\text{MeV})^2$:
	
```math-equation

		\alpha = \frac{c_e \cdot c_\mu \cdot \xipar^{11/2}}{(1\,\text{MeV})^2}
	
```

	Nun ist die Formel dimensionskonsistent: $[E^2] / [E^2] = [1]$.
	
	### Falsche Kürzung und Dimensionsfehler
	
	Wenn man die Potenzen von $\xipar$ ``kürzt'' und annimmt, $\alpha = K$ (mit $K$ als Konstante), ignoriert man die Skalenhierarchie. Dies führt zu einem Dimensionsfehler, sobald man absolute Werte einsetzt:
	
	- Ohne Kürzung: $\alpha \propto \xipar^{11/2}$ behält die Abhängigkeit von der fraktalen Skala bei und ist dimensionslos.
	- Mit falscher Kürzung: $\alpha = K$ impliziert $K$ dimensionslos, aber $c_e \cdot c_\mu$ hat $[E^2]$, was einen Widerspruch erzeugt, es sei denn, man führt ad-hoc eine Normierung ein – was die geometrische Herkunft zerstört.
	
	\textbf{Beispiel für den Fehler:} Nehmen wir an, man kürzt zu $\alpha = K$ und setzt experimentelle Massen ein: $m_e \cdot m_\mu \approx 54\,\text{MeV}^2$. Ohne Normierung ergäbe $K \approx 54\,\text{MeV}^2$, was dimensionsbehaftet ist und physikalisch unsinnig (eine Kopplungskonstante darf nicht von Einheiten abhängen). Die korrekte Form $\alpha = \xipar \cdot (E_0 / 1\,\text{MeV})^2$ normalisiert explizit und behält die Dimensionslosigkeit: $[1] \cdot ([E]/[E])^2 = [1]$.
	
	### Physikalische Konsequenz der Dimensionsanalyse
	
	Die Einheitenprüfung offenbart, dass falsche Kürzungen nicht nur algebraisch inkonsistent sind, sondern die Theorie von einer prädiktiven Geometrie zu einer empirischen Anpassung machen. In der T0-Theorie muss jede Operation die fraktale Skalierung $\xipar^{11/2}$ erhalten, da sie die Hierarchie von Planck-Skala zu Leptonmassen kodiert. Eine Kürzung würde z. B. die Vorhersage des Massenverhältnisses $m_e/m_\mu \propto \xipar^{1/2}$ unmöglich machen, da der Exponent verloren geht.
	
	\begin{foundation}
		\textbf{Dimensionskonsistenz in der T0-Theorie:}
		\begin{center}
			\begin{tabular}{lcc}
				\toprule
				\textbf{Formel} & \textbf{Dimension} & \textbf{Konsistent?} \\
				\midrule
				$\alpha = \xipar \cdot (E_0 / 1\,\text{MeV})^2$ & $[1] \cdot ([E]/[E])^2 = [1]$ & \checkmark \\
				$\alpha = c_e c_\mu \cdot \xipar^{11/2}$ (unkorrigiert) & $[E^2] \cdot [1] = [E^2]$ & $\times$ (braucht Normierung) \\
				$\alpha = K$ (gekürzt) & $[1]$ (ad-hoc) & $\times$ (verliert Skalierung) \\
				$\alpha \propto \xipar^{11/2}$ (proportional) & $[1]$ & \checkmark (relativ) \\
				\bottomrule
			\end{tabular}
		\end{center}
		
		Die Analyse zeigt: Nur die volle Struktur mit expliziter Normierung ist physikalisch valide und offenbart falsche Vereinfachungen.
	\end{foundation}
	
	Diese Methode unterstreicht die Stärke der T0-Theorie: Jede Formel muss nicht nur numerisch passen, sondern dimensions- und geometrisch konsistent sein.	
	## Warum keine fraktale Korrektur für Massenverhältnisse benötigt wird
	
	\begin{foundation}
		\textbf{Verschiedene Berechnungsansätze:}
		
```math-align

			\textbf{Weg A:} &\quad \alpha = \frac{m_e m_\mu}{7500} \quad \text{(benötigt Korrektur)} \\
			\textbf{Weg B:} &\quad \alpha = \frac{\Ezero^2}{7500} \quad \text{(benötigt Korrektur)} \\
			\textbf{Weg C:} &\quad \frac{m_\mu}{m_e} = f(\alpha) \quad \text{(keine Korrektur benötigt)} \\
			\textbf{Weg D:} &\quad \Ezero = \sqrt{m_e m_\mu} \quad \text{(keine Korrektur benötigt)}
		
```

	\end{foundation}
	
	## Massenverhältnisse sind korrekturfrei
	
	Das Leptonmassenverhältnis:
	\[
	\frac{m_\mu}{m_e} = \frac{c_\mu \xipar^2}{c_e \xipar^{5/2}} = \frac{c_\mu}{c_e} \xipar^{-1/2}
	\]
	
	Die fraktale Korrektur kürzt sich im Verhältnis heraus:
	\[
	\frac{m_\mu}{m_e} = \frac{\Kfrak \cdot m_\mu}{\Kfrak \cdot m_e} = \frac{m_\mu}{m_e}
	\]
	
	## Konsistente Behandlung
	
	
```math-align

		m_e^{\text{exp}} &= \Kfrak \cdot m_e^{\text{bare}} \\
		m_\mu^{\text{exp}} &= \Kfrak \cdot m_\mu^{\text{bare}} \\
		\Ezero^{\text{exp}} &= \Kfrak \cdot \Ezero^{\text{bare}}
	
```

	
	# Erweiterte mathematische Struktur
	
	## Vollständige Hierarchie
	
	\begin{longtable}{lcc}
		\caption{Vollständige T0-Hierarchie mit Feinstrukturkonstante} \\
		\toprule
		\textbf{Größe} & \textbf{T0-Ausdruck} & \textbf{Numerischer Wert} \\
		\midrule
		\endfirsthead
		\multicolumn{3}{c}{Fortsetzung der Tabelle} \\
		\toprule
		\textbf{Größe} & \textbf{T0-Ausdruck} & \textbf{Numerischer Wert} \\
		\midrule
		\endhead
		\bottomrule
		\endlastfoot
		$\xipar$ & $\frac{4}{3} \times 10^{-4}$ & $1.333 \times 10^{-4}$ \\
		$\Dfrak$ & $3 - \delta$ & $2.94$ \\
		$\Kfrak$ & $0.986$ & $0.986$ \\
		$\Ezero$ & $\sqrt{m_e \cdot m_\mu}$ & $7.398$ MeV \\
		$\alpha^{-1}$ & $\frac{(1\,\text{MeV})^2}{\xipar \cdot \Ezero^2}$ & $137.04$ \\
		$m_e/m_\mu$ & $\frac{5\sqrt{3}}{18} \times 10^{-2}$ & $4.81 \times 10^{-3}$ \\
		$\alpha$ & $\xipar \cdot (\Ezero/1\,\text{MeV})^2$ & $7.297 \times 10^{-3}$ \\
	\end{longtable}
	
	## Verifikation der Ableitungskette
	
	Die vollständige Ableitungssequenz:
	
		- Start: $\xipar = \frac{4}{3} \times 10^{-4}$ (reine Geometrie)
		- Fraktale Dimension: $\Dfrak = 2.94$
		- Charakteristische Energie: $\Ezero = 7.398$ MeV
		- Feinstrukturkonstante: $\alpha = \xipar \cdot (\Ezero/1\,\text{MeV})^2$
		- Konsistenzprüfung: $\alpha^{-1} = 137.04$ \checkmark
	
	
	# Die Bedeutung der Zahl $\frac{4{3}$}
	
	## Geometrische Interpretation
	
	Die Zahl $\frac{4}{3}$ ist nicht willkürlich:
	
		- Volumen der Einheitskugel: $V = \frac{4}{3}\pi r^3$
		- Harmonisches Verhältnis in der Musik (Quarte)
		- Geometrische Reihen und fraktale Strukturen
		- Fundamentale Konstante der sphärischen Geometrie
	
	
	## Universelle Bedeutung
	
	Die T0-Theorie zeigt, dass $\frac{4}{3}$ eine universelle geometrische Konstante ist, die die gesamte Physik durchzieht. Von der Feinstrukturkonstante bis zu Teilchenmassen taucht dieses Verhältnis immer wieder auf.
	
	# Verbindung zu anomalen magnetischen Momenten
	
	## Grundlegende Kopplung
	
	Die charakteristische Energie $\Ezero$ bestimmt auch die Größenordnung anomaler magnetischer Momente. Die massenabhängige Kopplung führt zu:
	
```math-equation

		g_T^\ell = \xipar \cdot m_\ell
		\label{eq:coupling_g2}
	
```

	
	## Skalierung mit Teilchenmassen
	
	Da $\Ezero = \sqrt{m_e \cdot m_\mu}$, bestimmt diese Energie die Skalierung aller leptonischen Anomalien. Schwerere Leptonen koppeln stärker, was zu der quadratischen Massenverstärkung in den g-2 Anomalien führt.
	
	# Glossar der verwendeten Symbole und Zeichen
	% Hier eine detaillierte Erklärung aller zentralen Symbole und Befehle für Klarheit:
	\begin{description}
\begin{itemize}
		\item[$\xipar$ ($\xi_0$)]: Fundamentaler geometrischer Parameter der T0-Theorie, der die Skalierung der fraktalen Raumzeit-Struktur beschreibt. Er ist dimensionslos und leitet sich aus geometrischen Prinzipien ab (Wert: $\frac{4}{3} \times 10^{-4}$).
		\item[$\Kfrak$ ($K_{\text{frak}}$)]: Fraktale Korrekturkonstante, die renormalisierende Effekte in der T0-Theorie berücksichtigt. Sie korrigiert bare Werte zu experimentellen Messwerten (Wert: 0.986).
		\item[$\Ezero$ ($E_0$)]: Charakteristische Energie, definiert als geometrisches Mittel der Elektron- und Myon-Massen. Sie dient als universelle Skala für elektromagnetische Prozesse (Wert: 7.398 MeV).
		\item[$\alphaem$ ($\alpha$)]: Feinstrukturkonstante, eine dimensionslose Kopplungskonstante der Quantenelektrodynamik (QED), die die Stärke der elektromagnetischen Wechselwirkung quantifiziert (Wert: $\approx 7.297 \times 10^{-3}$ oder $1/137.04$ in der T0-Theorie).
		\item[$\Dfrak$ ($D_f$)]: Fraktale Dimension der Raumzeit in der T0-Theorie, die eine Abweichung von der klassischen Dimension 3 andeutet (Wert: 2.94).
		\item[$m_e$]: Ruhemasse des Elektrons (Wert: 0.511 MeV).
		\item[$m_\mu$]: Ruhemasse des Myons (Wert: 105.66 MeV).
		\item[$c_e, c_\mu$]: Dimensionsbehaftete Koeffizienten in den T0-Massenformeln, die aus der Geometrie abgeleitet werden.
		\item[$\hbar, c$]: Reduzierte Plancksche Konstante und Lichtgeschwindigkeit, gesetzt auf 1 in natürlichen Einheiten.
		\item[$g_T^\ell$]: Anomaler magnetischer Moment (g-2) für Leptonen $\ell$.
\end{itemize}
	\end{description}
	
	\begin{center}
		\hrule
		\vspace{0.5cm}
		\textit{Dieses Dokument ist Teil der neuen T0-Serie}\\
		\textit{und baut auf den fundamentalen Prinzipien aus Dokument 1 auf}\\
		\vspace{0.3cm}
		\textbf{T0-Theorie: Zeit-Masse-Dualität Framework}\\
		\textit{Johann Pascher, HTL Leonding, Österreich}\\
	\end{center}

\end{document}
