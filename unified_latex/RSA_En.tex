\documentclass[11pt,a4paper,openany]{book}

% Essential packages
\usepackage[utf8]{inputenc}
\usepackage[T1]{fontenc}
\usepackage[english]{babel}
\usepackage[a4paper,margin=2.5cm]{geometry}
\usepackage{lmodern}

% Math and physics packages
\usepackage{amsmath}
\usepackage{amssymb}
\usepackage{amsthm}
\usepackage{mathtools}
\usepackage{physics}
\usepackage{siunitx}

% Graphics and tables
\usepackage{graphicx}
\usepackage[table,xcdraw]{xcolor}
\usepackage{tikz}
\usepackage{pgfplots}
\usepackage{tcolorbox}
\usepackage{booktabs}
\usepackage{array}
\usepackage{longtable}
\usepackage{float}

% Document formatting
\usepackage{fancyhdr}
\usepackage{tocloft}
\usepackage{hyperref}
\usepackage{cleveref}
\usepackage{microtype}
\usepackage{enumitem}
\usepackage{newunicodechar}

% Additional packages
\usepackage{adjustbox}
\usepackage{algorithm}
\usepackage{algorithmic}
\usepackage{amsfonts}
\usepackage{amsmath,amsfonts,amssymb}
\usepackage{amsmath,amsfonts,amssymb,physics}
\usepackage{amsmath,amssymb}
\usepackage{amsmath,amssymb,amsfonts,amsthm}
\usepackage{amsmath,amssymb,amsthm}
\usepackage{amsmath,amssymb,physics,graphicx,xcolor,amsthm}
\usepackage{bm}
\usepackage{booktabs,array,longtable,multirow}
\usepackage{braket}
\usepackage{breakurl}
\usepackage{cancel}
\usepackage{caption}
\usepackage{cite}
\usepackage{color}
\usepackage{colortbl}
\usepackage{csquotes}
\usepackage{doi}
\usepackage{forest}
\usepackage{gensymb}
\usepackage{geometry,fancyhdr}
\usepackage{graphicx,tikz,pgfplots}
\usepackage{hyperref,url}
\usepackage{hyphenat}
\usepackage{listings}
\usepackage{listings,enumerate}
\usepackage{mdframed}
\usepackage{multicol}
\usepackage{multirow}
\usepackage{natbib}
\usepackage{pdflscape}
\usepackage{ragged2e}
\usepackage{setspace}
\usepackage{siunitx,xcolor,graphicx}
\usepackage{slashed}
\usepackage{tabularx}
\usepackage{textcomp}
\usepackage{textgreek}
\usepackage{tikz,pgfplots}
\usepackage{upgreek}
\usepackage{url}

% Custom commands and definitions
\definecolor{blue}
\definecolor{blue}{rgb}{0,0,1}
\definecolor{boxgray}
\definecolor{boxgray}{RGB}{240,240,240}
\definecolor{deepblue}
\definecolor{deepblue}{RGB}{0,0,127}
\definecolor{deepgreen}
\definecolor{deepgreen}{RGB}{0,127,0}
\definecolor{deepred}
\definecolor{deepred}{RGB}{191,0,0}
\definecolor{t0blue}
\definecolor{t0blue}{RGB}{0,102,204}
\definecolor{t0blue}{RGB}{33,150,243}
\definecolor{t0green}
\definecolor{t0green}{RGB}{0,153,0}
\definecolor{t0green}{RGB}{0,153,76}
\definecolor{t0green}{RGB}{76,175,80}
\definecolor{t0orange}
\definecolor{t0orange}{RGB}{255,152,0}
\definecolor{t0purple}
\definecolor{t0purple}{RGB}{102,0,204}
\definecolor{t0purple}{RGB}{156,39,176}
\definecolor{t0red}
\definecolor{t0red}{RGB}{204,0,0}
\definecolor{t0red}{RGB}{204,0,51}
\definecolor{t0red}{RGB}{244,67,54}
\definecolor{t0yellow}
\definecolor{t0yellow}{RGB}{255,204,0}
\geometry{a4paper, left=25mm, right=25mm, top=25mm, bottom=25mm}
\geometry{a4paper, margin=1in}
\geometry{a4paper, margin=2.5cm}
\geometry{a4paper, margin=2cm}
\geometry{left=2.5cm,right=2.5cm,top=2.5cm,bottom=2.5cm}
\geometry{left=2cm,right=2cm,top=2cm,bottom=2cm}
\geometry{margin=1in}
\geometry{margin=2.5cm}
\geometry{margin=2cm}
\hypersetup{
	colorlinks=true,
	linkcolor=blue,
	citecolor=blue,
	urlcolor=blue,
	pdftitle={Analysis and Implications of MNRAS Paper 544 for the T0-Theory}
\hypersetup{
	colorlinks=true,
	linkcolor=blue,
	citecolor=blue,
	urlcolor=blue,
	pdftitle={Beweis: Die Feinstrukturkonstante α = 1 in natürlichen Einheiten}
\hypersetup{
	colorlinks=true,
	linkcolor=blue,
	citecolor=blue,
	urlcolor=blue,
	pdftitle={Beweis: Die Koide-Formel enthält implizit $\xi$}
\hypersetup{
	colorlinks=true,
	linkcolor=blue,
	citecolor=blue,
	urlcolor=blue,
	pdftitle={Chinas Photonischer Quantenchip: 1000x-Speedup und T0-Integration}
\hypersetup{
	colorlinks=true,
	linkcolor=blue,
	citecolor=blue,
	urlcolor=blue,
	pdftitle={Complete Derivation of Higgs Mass and Wilson Coefficients}
\hypersetup{
	colorlinks=true,
	linkcolor=blue,
	citecolor=blue,
	urlcolor=blue,
	pdftitle={Complete Particle Spectrum: Standard Model vs T0 Theory}
\hypersetup{
	colorlinks=true,
	linkcolor=blue,
	citecolor=blue,
	urlcolor=blue,
	pdftitle={Conceptual Comparison of Unified Natural Units and Extended Standard Model}
\hypersetup{
	colorlinks=true,
	linkcolor=blue,
	citecolor=blue,
	urlcolor=blue,
	pdftitle={Connections between the Mizohata-Takeuchi Counterexample and the T0 Time-Mass Duality Theory}
\hypersetup{
	colorlinks=true,
	linkcolor=blue,
	citecolor=blue,
	urlcolor=blue,
	pdftitle={Das Relationale Zahlensystem: Primzahlen als fundamentale Verhältnisse}
\hypersetup{
	colorlinks=true,
	linkcolor=blue,
	citecolor=blue,
	urlcolor=blue,
	pdftitle={Das T0-Modell (Planck-Referenziert): Eine Neuformulierung der Physik}
\hypersetup{
	colorlinks=true,
	linkcolor=blue,
	citecolor=blue,
	urlcolor=blue,
	pdftitle={Das T0-Modell: Zeit-Energie-Dualität und geometrische Ruhemasse}
\hypersetup{
	colorlinks=true,
	linkcolor=blue,
	citecolor=blue,
	urlcolor=blue,
	pdftitle={Der Massenskalierungsexponent κ in der T0-Theorie}
\hypersetup{
	colorlinks=true,
	linkcolor=blue,
	citecolor=blue,
	urlcolor=blue,
	pdftitle={Der geometrische Formalismus der T0-Quantenmechanik und seine Anwendung auf Quantencomputer}
\hypersetup{
	colorlinks=true,
	linkcolor=blue,
	citecolor=blue,
	urlcolor=blue,
	pdftitle={Der xi Parameter und Teilchendifferenzierung in der T0-Theorie}
\hypersetup{
	colorlinks=true,
	linkcolor=blue,
	citecolor=blue,
	urlcolor=blue,
	pdftitle={Deterministic Quantum Mechanics via T0-Energy Field Formulation}
\hypersetup{
	colorlinks=true,
	linkcolor=blue,
	citecolor=blue,
	urlcolor=blue,
	pdftitle={Deterministische Quantenmechanik via T0-Energiefeld-Formulierung}
\hypersetup{
	colorlinks=true,
	linkcolor=blue,
	citecolor=blue,
	urlcolor=blue,
	pdftitle={Die Elektroneneinheitsladung in der T0-Theorie: Jenseits von Punkt-Singularitäten}
\hypersetup{
	colorlinks=true,
	linkcolor=blue,
	citecolor=blue,
	urlcolor=blue,
	pdftitle={Die Feinstrukturkonstante: Verschiedene Darstellungen und Beziehungen}
\hypersetup{
	colorlinks=true,
	linkcolor=blue,
	citecolor=blue,
	urlcolor=blue,
	pdftitle={Die Musikalische Spirale und die 137: Die mathematische Entdeckung der kosmischen Verstimmung}
\hypersetup{
	colorlinks=true,
	linkcolor=blue,
	citecolor=blue,
	urlcolor=blue,
	pdftitle={E=mc² = E=m: Die Konstanten-Illusion entlarvt}
\hypersetup{
	colorlinks=true,
	linkcolor=blue,
	citecolor=blue,
	urlcolor=blue,
	pdftitle={E=mc² = E=m: The Constants Illusion Exposed}
\hypersetup{
	colorlinks=true,
	linkcolor=blue,
	citecolor=blue,
	urlcolor=blue,
	pdftitle={Einfache Lagrange-Revolution: Von der Standardmodell-Komplexität zur T0-Eleganz}
\hypersetup{
	colorlinks=true,
	linkcolor=blue,
	citecolor=blue,
	urlcolor=blue,
	pdftitle={Einführung in die Umsetzung photonischer Bauteile auf Wafern für Nachrichtentechniker}
\hypersetup{
	colorlinks=true,
	linkcolor=blue,
	citecolor=blue,
	urlcolor=blue,
	pdftitle={Einführung in photonische Quantenchips für Nachrichtentechniker}
\hypersetup{
	colorlinks=true,
	linkcolor=blue,
	citecolor=blue,
	urlcolor=blue,
	pdftitle={Elimination der Masse als dimensionaler Platzhalter im T0-Modell}
\hypersetup{
	colorlinks=true,
	linkcolor=blue,
	citecolor=blue,
	urlcolor=blue,
	pdftitle={Elimination of Mass as Dimensional Placeholder in the T0 Model}
\hypersetup{
	colorlinks=true,
	linkcolor=blue,
	citecolor=blue,
	urlcolor=blue,
	pdftitle={Empirical Analysis of Deterministic Factorization Methods}
\hypersetup{
	colorlinks=true,
	linkcolor=blue,
	citecolor=blue,
	urlcolor=blue,
	pdftitle={Empirische Analyse deterministischer Faktorisierungsmethoden}
\hypersetup{
	colorlinks=true,
	linkcolor=blue,
	citecolor=blue,
	urlcolor=blue,
	pdftitle={Integration der Dirac-Gleichung im T0-Modell: Natürliche-Einheiten-Rahmenwerk}
\hypersetup{
	colorlinks=true,
	linkcolor=blue,
	citecolor=blue,
	urlcolor=blue,
	pdftitle={Integration of the Dirac Equation in the T0 Model: Natural Units Framework}
\hypersetup{
	colorlinks=true,
	linkcolor=blue,
	citecolor=blue,
	urlcolor=blue,
	pdftitle={Introduction to Photonic Quantum Chips for Communication Engineers}
\hypersetup{
	colorlinks=true,
	linkcolor=blue,
	citecolor=blue,
	urlcolor=blue,
	pdftitle={Introduction to the Implementation of Photonic Components on Wafers for Communication Engineers}
\hypersetup{
	colorlinks=true,
	linkcolor=blue,
	citecolor=blue,
	urlcolor=blue,
	pdftitle={Konzeptioneller Vergleich von Einheitlichen Natürlichen Einheiten und Erweitertem Standardmodell}
\hypersetup{
	colorlinks=true,
	linkcolor=blue,
	citecolor=blue,
	urlcolor=blue,
	pdftitle={Markov Chains in the Context of T0 Theory: Deterministic or Stochastic? A Treatise on Patterns, Preconditions, and Uncertainty}
\hypersetup{
	colorlinks=true,
	linkcolor=blue,
	citecolor=blue,
	urlcolor=blue,
	pdftitle={Markov-Ketten im Kontext der T0-Theorie: Deterministisch oder stochastisch? Ein Traktat zu Mustern, Voraussetzungen und Unsicherheit}
\hypersetup{
	colorlinks=true,
	linkcolor=blue,
	citecolor=blue,
	urlcolor=blue,
	pdftitle={Mathematical Analysis of T0-Shor Algorithm: Theoretical Framework and Computational Complexity}
\hypersetup{
	colorlinks=true,
	linkcolor=blue,
	citecolor=blue,
	urlcolor=blue,
	pdftitle={Mathematical Constructs of Alternative CMB Models: Unnikrishnan and Peratt in Harmony with the T0 Theory}
\hypersetup{
	colorlinks=true,
	linkcolor=blue,
	citecolor=blue,
	urlcolor=blue,
	pdftitle={Mathematische Analyse des T0-Shor Algorithmus: Theoretischer Rahmen und Berechnungskomplexität}
\hypersetup{
	colorlinks=true,
	linkcolor=blue,
	citecolor=blue,
	urlcolor=blue,
	pdftitle={Mathematische Konstrukte alternativer CMB-Modelle: Unnikrishnan und Peratt im Einklang mit der T0-Theorie}
\hypersetup{
	colorlinks=true,
	linkcolor=blue,
	citecolor=blue,
	urlcolor=blue,
	pdftitle={Natural Unit Systems: Universal Energy Conversion and Fundamental Length Scale Hierarchy}
\hypersetup{
	colorlinks=true,
	linkcolor=blue,
	citecolor=blue,
	urlcolor=blue,
	pdftitle={Natural Units in Theoretical Physics: A Treatise in the Context of T0 Theory}
\hypersetup{
	colorlinks=true,
	linkcolor=blue,
	citecolor=blue,
	urlcolor=blue,
	pdftitle={Natürliche Einheiten in der theoretischen Physik: Eine Abhandlung im Kontext der T0-Theorie}
\hypersetup{
	colorlinks=true,
	linkcolor=blue,
	citecolor=blue,
	urlcolor=blue,
	pdftitle={Natürliche Einheitensysteme: Universelle Energieumwandlung und fundamentale Längenskala-Hierarchie}
\hypersetup{
	colorlinks=true,
	linkcolor=blue,
	citecolor=blue,
	urlcolor=blue,
	pdftitle={Parameter System-Dependency in T0-Model: SI vs. Natural Units}
\hypersetup{
	colorlinks=true,
	linkcolor=blue,
	citecolor=blue,
	urlcolor=blue,
	pdftitle={Parameter-Systemabhängigkeit im T0-Modell: SI- vs. natürliche Einheiten}
\hypersetup{
	colorlinks=true,
	linkcolor=blue,
	citecolor=blue,
	urlcolor=blue,
	pdftitle={Proof: The Fine Structure Constant α = 1 in Natural Units}
\hypersetup{
	colorlinks=true,
	linkcolor=blue,
	citecolor=blue,
	urlcolor=blue,
	pdftitle={Proof: The Koide Formula Implicitly Contains $\xi$}
\hypersetup{
	colorlinks=true,
	linkcolor=blue,
	citecolor=blue,
	urlcolor=blue,
	pdftitle={Pure Energy T0 Theory: Ratio-Based Physics with SI Reference}
\hypersetup{
	colorlinks=true,
	linkcolor=blue,
	citecolor=blue,
	urlcolor=blue,
	pdftitle={Quantum Mechanics in the T0 Model: Field-Theoretic Foundations}
\hypersetup{
	colorlinks=true,
	linkcolor=blue,
	citecolor=blue,
	urlcolor=blue,
	pdftitle={Ratio-Based vs. Absolute: The Role of Fractal Correction in T0 Theory}
\hypersetup{
	colorlinks=true,
	linkcolor=blue,
	citecolor=blue,
	urlcolor=blue,
	pdftitle={Reine Energie T0-Theorie: Verhältnis-basierte Physik mit SI-Referenz}
\hypersetup{
	colorlinks=true,
	linkcolor=blue,
	citecolor=blue,
	urlcolor=blue,
	pdftitle={Simple Lagrangian Revolution: From Standard Model Complexity to T0 Elegance}
\hypersetup{
	colorlinks=true,
	linkcolor=blue,
	citecolor=blue,
	urlcolor=blue,
	pdftitle={Simplified Dirac Equation in T0 Theory: Field Node Approach}
\hypersetup{
	colorlinks=true,
	linkcolor=blue,
	citecolor=blue,
	urlcolor=blue,
	pdftitle={Simplified T0 Theory: Elegant Lagrangian Density for Time-Mass Duality}
\hypersetup{
	colorlinks=true,
	linkcolor=blue,
	citecolor=blue,
	urlcolor=blue,
	pdftitle={T0 Cosmology: Redshift as a Geometric Path Effect in a Static Universe}
\hypersetup{
	colorlinks=true,
	linkcolor=blue,
	citecolor=blue,
	urlcolor=blue,
	pdftitle={T0 Deterministic Quantum Computing: Complete Analysis of Important Algorithms}
\hypersetup{
	colorlinks=true,
	linkcolor=blue,
	citecolor=blue,
	urlcolor=blue,
	pdftitle={T0 Deterministisches Quantencomputing: Vollständige Analyse wichtiger Algorithmen}
\hypersetup{
	colorlinks=true,
	linkcolor=blue,
	citecolor=blue,
	urlcolor=blue,
	pdftitle={T0 Model: Complete Framework - From Time-Energy Duality to Universal Constants}
\hypersetup{
	colorlinks=true,
	linkcolor=blue,
	citecolor=blue,
	urlcolor=blue,
	pdftitle={T0 Model: Complete Parameter-Free Particle Mass Calculation}
\hypersetup{
	colorlinks=true,
	linkcolor=blue,
	citecolor=blue,
	urlcolor=blue,
	pdftitle={T0 Model: Unified Neutrino Formula Structure}
\hypersetup{
	colorlinks=true,
	linkcolor=blue,
	citecolor=blue,
	urlcolor=blue,
	pdftitle={T0 Model: Universal Energy Relations for Mol and Candela Units}
\hypersetup{
	colorlinks=true,
	linkcolor=blue,
	citecolor=blue,
	urlcolor=blue,
	pdftitle={T0 Modell: Vollständiges Framework - Von Zeit-Energie-Dualität zu universellen Konstanten}
\hypersetup{
	colorlinks=true,
	linkcolor=blue,
	citecolor=blue,
	urlcolor=blue,
	pdftitle={T0 Quantenfeldtheorie: QFT, QM und Quantencomputer}
\hypersetup{
	colorlinks=true,
	linkcolor=blue,
	citecolor=blue,
	urlcolor=blue,
	pdftitle={T0 Quantum Field Theory: QFT, QM and Quantum Computers}
\hypersetup{
	colorlinks=true,
	linkcolor=blue,
	citecolor=blue,
	urlcolor=blue,
	pdftitle={T0 Theory vs Bell's Theorem: How Deterministic Energy Fields Circumvent No-Go Theorems}
\hypersetup{
	colorlinks=true,
	linkcolor=blue,
	citecolor=blue,
	urlcolor=blue,
	pdftitle={T0 Theory: Final Extension to Hadrons - Physically Derived Corrections}
\hypersetup{
	colorlinks=true,
	linkcolor=blue,
	citecolor=blue,
	urlcolor=blue,
	pdftitle={T0 Theory: The Fine-Structure Constant}
\hypersetup{
	colorlinks=true,
	linkcolor=blue,
	citecolor=blue,
	urlcolor=blue,
	pdftitle={T0 Theory: The Gravitational Constant}
\hypersetup{
	colorlinks=true,
	linkcolor=blue,
	citecolor=blue,
	urlcolor=blue,
	pdftitle={T0-Kosmologie: Rotverschiebung als geometrischer Pfad-Effekt im statischen Universum}
\hypersetup{
	colorlinks=true,
	linkcolor=blue,
	citecolor=blue,
	urlcolor=blue,
	pdftitle={T0-Model: Complete Document Analysis and Structured Summary}
\hypersetup{
	colorlinks=true,
	linkcolor=blue,
	citecolor=blue,
	urlcolor=blue,
	pdftitle={T0-Model: Kinetic Energy of Electrons and Photons}
\hypersetup{
	colorlinks=true,
	linkcolor=blue,
	citecolor=blue,
	urlcolor=blue,
	pdftitle={T0-Model: The Hubble Parameter in Static Universe}
\hypersetup{
	colorlinks=true,
	linkcolor=blue,
	citecolor=blue,
	urlcolor=blue,
	pdftitle={T0-Modell-Verifikation: Skalen-Verhältnis-basierte Berechnungen}
\hypersetup{
	colorlinks=true,
	linkcolor=blue,
	citecolor=blue,
	urlcolor=blue,
	pdftitle={T0-Modell: Bewegungsenergie von Elektronen und Photonen}
\hypersetup{
	colorlinks=true,
	linkcolor=blue,
	citecolor=blue,
	urlcolor=blue,
	pdftitle={T0-Modell: Die Hubble-Konstante im statischen Universum}
\hypersetup{
	colorlinks=true,
	linkcolor=blue,
	citecolor=blue,
	urlcolor=blue,
	pdftitle={T0-Modell: Einheitliche Neutrino-Formel-Struktur}
\hypersetup{
	colorlinks=true,
	linkcolor=blue,
	citecolor=blue,
	urlcolor=blue,
	pdftitle={T0-Modell: Universelle Energiebeziehungen für Mol- und Candela-Einheiten}
\hypersetup{
	colorlinks=true,
	linkcolor=blue,
	citecolor=blue,
	urlcolor=blue,
	pdftitle={T0-Modell: Vollständige Dokumentenanalyse und strukturierte Zusammenfassung}
\hypersetup{
	colorlinks=true,
	linkcolor=blue,
	citecolor=blue,
	urlcolor=blue,
	pdftitle={T0-Modell: Vollständige parameterfreie Teilchenmassen-Berechnung}
\hypersetup{
	colorlinks=true,
	linkcolor=blue,
	citecolor=blue,
	urlcolor=blue,
	pdftitle={T0-QAT: $\xi$-Aware Quantization-Aware Training}
\hypersetup{
	colorlinks=true,
	linkcolor=blue,
	citecolor=blue,
	urlcolor=blue,
	pdftitle={T0-QFT ML Addendum: Machine Learning Derived Extensions}
\hypersetup{
	colorlinks=true,
	linkcolor=blue,
	citecolor=blue,
	urlcolor=blue,
	pdftitle={T0-QFT ML-Addendum: Maschinelle Lern-abgeleitete Erweiterungen}
\hypersetup{
	colorlinks=true,
	linkcolor=blue,
	citecolor=blue,
	urlcolor=blue,
	pdftitle={T0-Theorie vs Bells Theorem: Wie deterministische Energiefelder No-Go-Theoreme umgehen}
\hypersetup{
	colorlinks=true,
	linkcolor=blue,
	citecolor=blue,
	urlcolor=blue,
	pdftitle={T0-Theorie: Der Terrell-Penrose-Effekt und Massenvariation}
\hypersetup{
	colorlinks=true,
	linkcolor=blue,
	citecolor=blue,
	urlcolor=blue,
	pdftitle={T0-Theorie: Die Feinstrukturkonstante}
\hypersetup{
	colorlinks=true,
	linkcolor=blue,
	citecolor=blue,
	urlcolor=blue,
	pdftitle={T0-Theorie: Die Gravitationskonstante}
\hypersetup{
	colorlinks=true,
	linkcolor=blue,
	citecolor=blue,
	urlcolor=blue,
	pdftitle={T0-Theorie: Die T0-Zeit-Masse-Dualität}
\hypersetup{
	colorlinks=true,
	linkcolor=blue,
	citecolor=blue,
	urlcolor=blue,
	pdftitle={T0-Theorie: Die sieben Rätsel}
\hypersetup{
	colorlinks=true,
	linkcolor=blue,
	citecolor=blue,
	urlcolor=blue,
	pdftitle={T0-Theorie: Erweiterung auf Bell-Tests – ML-Simulationen (November 2025)}
\hypersetup{
	colorlinks=true,
	linkcolor=blue,
	citecolor=blue,
	urlcolor=blue,
	pdftitle={T0-Theorie: Finale Erweiterung auf Hadronen - Physikalisch abgeleitete Korrekturen}
\hypersetup{
	colorlinks=true,
	linkcolor=blue,
	citecolor=blue,
	urlcolor=blue,
	pdftitle={T0-Theorie: Finale Fraktale Massenformeln (November 2025)}
\hypersetup{
	colorlinks=true,
	linkcolor=blue,
	citecolor=blue,
	urlcolor=blue,
	pdftitle={T0-Theorie: Fraktaldimension aus Lepton-Massenverhältnis}
\hypersetup{
	colorlinks=true,
	linkcolor=blue,
	citecolor=blue,
	urlcolor=blue,
	pdftitle={T0-Theorie: Fundamentale Prinzipien}
\hypersetup{
	colorlinks=true,
	linkcolor=blue,
	citecolor=blue,
	urlcolor=blue,
	pdftitle={T0-Theorie: Herleitung der Gravitationskonstanten}
\hypersetup{
	colorlinks=true,
	linkcolor=blue,
	citecolor=blue,
	urlcolor=blue,
	pdftitle={T0-Theorie: Kosmische Beziehungen und universelle $\xi$-Konstante}
\hypersetup{
	colorlinks=true,
	linkcolor=blue,
	citecolor=blue,
	urlcolor=blue,
	pdftitle={T0-Theorie: Kosmologie}
\hypersetup{
	colorlinks=true,
	linkcolor=blue,
	citecolor=blue,
	urlcolor=blue,
	pdftitle={T0-Theorie: Netzwerkdarstellung und Dimensionsanalyse in der T0-Theorie}
\hypersetup{
	colorlinks=true,
	linkcolor=blue,
	citecolor=blue,
	urlcolor=blue,
	pdftitle={T0-Theorie: Teilchenmassen}
\hypersetup{
	colorlinks=true,
	linkcolor=blue,
	citecolor=blue,
	urlcolor=blue,
	pdftitle={T0-Theorie: Vollstaendiger Abschluss}
\hypersetup{
	colorlinks=true,
	linkcolor=blue,
	citecolor=blue,
	urlcolor=blue,
	pdftitle={T0-Theory: Complete Closure}
\hypersetup{
	colorlinks=true,
	linkcolor=blue,
	citecolor=blue,
	urlcolor=blue,
	pdftitle={T0-Theory: Complete Derivation of All Parameters Without Circularity}
\hypersetup{
	colorlinks=true,
	linkcolor=blue,
	citecolor=blue,
	urlcolor=blue,
	pdftitle={T0-Theory: Cosmic Relations and universal $\xi$-constant}
\hypersetup{
	colorlinks=true,
	linkcolor=blue,
	citecolor=blue,
	urlcolor=blue,
	pdftitle={T0-Theory: Cosmology}
\hypersetup{
	colorlinks=true,
	linkcolor=blue,
	citecolor=blue,
	urlcolor=blue,
	pdftitle={T0-Theory: Derivation of the Gravitational Constant}
\hypersetup{
	colorlinks=true,
	linkcolor=blue,
	citecolor=blue,
	urlcolor=blue,
	pdftitle={T0-Theory: Extension to Bell Tests – ML Simulations (November 2025)}
\hypersetup{
	colorlinks=true,
	linkcolor=blue,
	citecolor=blue,
	urlcolor=blue,
	pdftitle={T0-Theory: Final Fractal Mass Formulas (November 2025)}
\hypersetup{
	colorlinks=true,
	linkcolor=blue,
	citecolor=blue,
	urlcolor=blue,
	pdftitle={T0-Theory: Fractal Dimension from Lepton Mass Ratio}
\hypersetup{
	colorlinks=true,
	linkcolor=blue,
	citecolor=blue,
	urlcolor=blue,
	pdftitle={T0-Theory: Fundamental Principles}
\hypersetup{
	colorlinks=true,
	linkcolor=blue,
	citecolor=blue,
	urlcolor=blue,
	pdftitle={T0-Theory: Mass Variation as an Equivalent to Time Dilation}
\hypersetup{
	colorlinks=true,
	linkcolor=blue,
	citecolor=blue,
	urlcolor=blue,
	pdftitle={T0-Theory: Network Representation and Dimensional Analysis in the T0-Theory}
\hypersetup{
	colorlinks=true,
	linkcolor=blue,
	citecolor=blue,
	urlcolor=blue,
	pdftitle={T0-Theory: Neutrinos}
\hypersetup{
	colorlinks=true,
	linkcolor=blue,
	citecolor=blue,
	urlcolor=blue,
	pdftitle={T0-Theory: Particle Masses}
\hypersetup{
	colorlinks=true,
	linkcolor=blue,
	citecolor=blue,
	urlcolor=blue,
	pdftitle={T0-Theory: The Seven Riddles}
\hypersetup{
	colorlinks=true,
	linkcolor=blue,
	citecolor=blue,
	urlcolor=blue,
	pdftitle={T0-Theory: The T0-Time-Mass Duality}
\hypersetup{
	colorlinks=true,
	linkcolor=blue,
	citecolor=blue,
	urlcolor=blue,
	pdftitle={Temperature Units in Natural Units: T0-Theory}
\hypersetup{
	colorlinks=true,
	linkcolor=blue,
	citecolor=blue,
	urlcolor=blue,
	pdftitle={Temperatureinheiten in nat\"urlichen Einheiten: T0-Theorie}
\hypersetup{
	colorlinks=true,
	linkcolor=blue,
	citecolor=blue,
	urlcolor=blue,
	pdftitle={The Electron Unit Charge in T0 Theory: Beyond Point Singularities}
\hypersetup{
	colorlinks=true,
	linkcolor=blue,
	citecolor=blue,
	urlcolor=blue,
	pdftitle={The Fine Structure Constant: Various Representations and Relationships}
\hypersetup{
	colorlinks=true,
	linkcolor=blue,
	citecolor=blue,
	urlcolor=blue,
	pdftitle={The Geometric Formalism of T0 Quantum Mechanics and its Application to Quantum Computing}
\hypersetup{
	colorlinks=true,
	linkcolor=blue,
	citecolor=blue,
	urlcolor=blue,
	pdftitle={The Mass Scaling Exponent κ in T0 Theory}
\hypersetup{
	colorlinks=true,
	linkcolor=blue,
	citecolor=blue,
	urlcolor=blue,
	pdftitle={The Musical Spiral and 137: The Mathematical Discovery of Cosmic Detuning}
\hypersetup{
	colorlinks=true,
	linkcolor=blue,
	citecolor=blue,
	urlcolor=blue,
	pdftitle={The Relational Number System: Prime Numbers as Fundamental Ratios}
\hypersetup{
	colorlinks=true,
	linkcolor=blue,
	citecolor=blue,
	urlcolor=blue,
	pdftitle={The T0 Model (Planck-Referenced): A Reformulation of Physics}
\hypersetup{
	colorlinks=true,
	linkcolor=blue,
	citecolor=blue,
	urlcolor=blue,
	pdftitle={The T0 Model: Time-Energy Duality and Geometric Rest Mass}
\hypersetup{
	colorlinks=true,
	linkcolor=blue,
	citecolor=blue,
	urlcolor=blue,
	pdftitle={The T0-Model (Planck-Referenced): A Reformulation of Physics}
\hypersetup{
	colorlinks=true,
	linkcolor=blue,
	citecolor=blue,
	urlcolor=blue,
	pdftitle={Verbindungen zwischen dem Mizohata-Takeuchi-Gegenbeispiel und der T0-Zeit-Masse-Dualitätstheorie}
\hypersetup{
	colorlinks=true,
	linkcolor=blue,
	citecolor=blue,
	urlcolor=blue,
	pdftitle={Vereinfachte Dirac-Gleichung in der T0-Theorie: Feldknoten-Ansatz}
\hypersetup{
	colorlinks=true,
	linkcolor=blue,
	citecolor=blue,
	urlcolor=blue,
	pdftitle={Vereinfachte T0-Theorie: Elegante Lagrange-Dichte für Zeit-Masse-Dualität}
\hypersetup{
	colorlinks=true,
	linkcolor=blue,
	citecolor=blue,
	urlcolor=blue,
	pdftitle={Verhältnisbasiert vs. Absolut: Die Rolle der fraktalen Korrektur in der T0-Theorie}
\hypersetup{
	colorlinks=true,
	linkcolor=blue,
	citecolor=blue,
	urlcolor=blue,
	pdftitle={Vollständige Herleitung der Higgs-Masse und Wilson-Koeffizienten}
\hypersetup{
	colorlinks=true,
	linkcolor=blue,
	citecolor=blue,
	urlcolor=blue,
	pdftitle={Vollständiges Teilchenspektrum: Standard-Modell vs T0-Theorie}
\hypersetup{
	colorlinks=true,
	linkcolor=blue,
	citecolor=blue,
	urlcolor=blue,
	pdftitle={Warum Zahlenverhältnisse nicht direkt gekürzt werden dürfen}
\hypersetup{
	colorlinks=true,
	linkcolor=blue,
	citecolor=blue,
	urlcolor=blue,
	pdftitle={Why Numerical Ratios Must Not Be Directly Simplified}
\hypersetup{
	colorlinks=true,
	linkcolor=blue,
	citecolor=blue,
	urlcolor=blue,
}
\hypersetup{
	colorlinks=true,
	linkcolor=blue,
	citecolor=red,
	urlcolor=blue,
	bookmarks=true,
	bookmarksnumbered=true,
	pdfstartview=FitH,
	pdftitle={T0 Model - Field-Theoretic Derivation of the Beta Parameter}
\hypersetup{
	colorlinks=true,
	linkcolor=blue,
	citecolor=red,
	urlcolor=blue,
	bookmarks=true,
	bookmarksnumbered=true,
	pdfstartview=FitH,
	pdftitle={T0-Modell - Feldtheoretische Herleitung des Beta-Parameters}
\hypersetup{
	colorlinks=true,
	linkcolor=blue,
	filecolor=magenta,
	urlcolor=cyan,
}
\hypersetup{
	colorlinks=true,
	linkcolor=blue,
	urlcolor=blue,
	citecolor=blue,
	pdftitle={From Time Dilation to Mass Variation: Mathematical Core Formulations of Time-Mass Duality Theory - Updated Framework}
\hypersetup{
	colorlinks=true,
	linkcolor=blue,
	urlcolor=blue,
	citecolor=blue,
	pdftitle={T0 Model: Detailed Formula for Leptonic Anomalies}
\hypersetup{
	colorlinks=true,
	linkcolor=blue,
	urlcolor=blue,
	citecolor=blue,
	pdftitle={T0 Model: Detaillierte Formel für leptonische Anomalien}
\hypersetup{
	colorlinks=true,
	linkcolor=blue,
	urlcolor=blue,
	citecolor=blue,
	pdftitle={T0 Model: Energy-based Formulas with Quadratic Scaling}
\hypersetup{
	colorlinks=true,
	linkcolor=blue,
	urlcolor=blue,
	citecolor=blue,
	pdftitle={T0 Model: Granulation, Limits and Fundamental Asymmetry}
\hypersetup{
	colorlinks=true,
	linkcolor=blue,
	urlcolor=blue,
	citecolor=blue,
	pdftitle={T0-Modell: Energiebasierte Formeln mit quadratischer Skalierung}
\hypersetup{
	colorlinks=true,
	linkcolor=blue,
	urlcolor=blue,
	citecolor=blue,
	pdftitle={T0-Modell: Granulation, Limits und fundamentale Asymmetrie}
\hypersetup{
	colorlinks=true,
	linkcolor=blue,
	urlcolor=blue,
	citecolor=blue,
	pdftitle={Von Zeitdilatation zu Massenvariation: Mathematische Kernformulierungen der Zeit-Masse-Dualitätstheorie - Aktualisiertes Framework}
\hypersetup{
	colorlinks=true,
	linkcolor=t0blue,
	citecolor=t0blue,
	urlcolor=t0blue,
	pdftitle={T0 Model: Complete Theoretical Summary}
\hypersetup{
	colorlinks=true,
	linkcolor=t0blue,
	citecolor=t0blue,
	urlcolor=t0blue,
	pdftitle={T0 Theory: Resolution of Apparent Instantaneity}
\hypersetup{
	colorlinks=true,
	linkcolor=t0blue,
	citecolor=t0blue,
	urlcolor=t0blue,
	pdftitle={T0 vs Synergetics: Vereinfachung durch natürliche Einheiten}
\hypersetup{
	colorlinks=true,
	linkcolor=t0blue,
	citecolor=t0blue,
	urlcolor=t0blue,
	pdftitle={T0-Modell: Vollständige theoretische Zusammenfassung}
\hypersetup{
	colorlinks=true,
	linkcolor=t0blue,
	citecolor=t0blue,
	urlcolor=t0blue,
	pdftitle={T0-Theorie: Auflösung der scheinbaren Instantanität}
\hypersetup{
	colorlinks=true,
	linkcolor=t0blue,
	citecolor=t0blue,
	urlcolor=t0blue,
	pdftitle={T0-Theorie: Vollständige Dokumentenübersicht}
\hypersetup{
	colorlinks=true,
	linkcolor=t0blue,
	citecolor=t0blue,
	urlcolor=t0blue,
	pdftitle={T0-Theory: Complete Document Overview}
\hypersetup{
	colorlinks=true,
	linkcolor=t0blue,
	citecolor=t0blue,
	urlcolor=t0blue,
}
\hypersetup{
	colorlinks=true,
	linkcolor=t0blue,
	citecolor=t0green,
	urlcolor=t0blue,
	pdftitle={Das verborgene Geheimnis von 1/137}
\hypersetup{
	colorlinks=true,
	linkcolor=t0blue,
	citecolor=t0green,
	urlcolor=t0blue,
	pdftitle={The Hidden Secret of 1/137}
\hypersetup{
    colorlinks=true,
    linkcolor=blue,
    citecolor=blue,
    urlcolor=blue,
    pdftitle={Analyse und Implikationen des MNRAS-Papiers 544 für die T0-Theorie}
\hypersetup{
  colorlinks=true,
  linkcolor=blue,
  citecolor=blue,
  urlcolor=blue
}
\hypersetup{
  colorlinks=true,
  linkcolor=blue,
  citecolor=blue,
  urlcolor=blue,
  pdftitle={T0-Theorie: Ein-Uhr-Metrologie und Drei-Uhren-Experiment}
\hypersetup{
  colorlinks=true,
  linkcolor=blue,
  citecolor=blue,
  urlcolor=blue,
  pdftitle={T0-Theory: Single-Clock Metrology and Three-Clock Experiment}
\hypersetup{
colorlinks=true,
linkcolor=blue,
citecolor=blue,
urlcolor=blue,
pdftitle={Quantenmechanik im T0-Modell: Feldtheoretische Grundlagen}
\hypersetup{
colorlinks=true,
linkcolor=blue,
citecolor=blue,
urlcolor=blue,
pdftitle={T0-Theory: Neutrinos}
\newcommand{\Bzero}{B_0}
\newcommand{\CQCD}{C_{\text{QCD}
\newcommand{\Cconv}{C_{\text{conv}
\newcommand{\Cto}{C_{\text{T0}
\newcommand{\Czero}{C_0}
\newcommand{\DTmu}{D_{T,\mu}
\newcommand{\DcovT}[1]{\partial_\mu #1 + #1 \partial_\mu \Tfield}
\newcommand{\Dfrak}{D_f}
\newcommand{\Df}{D_f}
\newcommand{\DhiggsT}{\Tfield (\partial_\mu + ig A_\mu) \Phi + \Phi \partial_\mu \Tfield}
\newcommand{\EPlanck}{E_P}
\newcommand{\EPlanck}{E_{\text{Pl}
\newcommand{\EPratio}[1]{\frac{#1}
\newcommand{\EP}{E_P}
\newcommand{\EP}{E_{\text{P}
\newcommand{\EW}{E_W}
\newcommand{\EZ}{E_Z}
\newcommand{\Echar}{E_{\text{char}
\newcommand{\Ee}{E_e}
\newcommand{\Efield}{E(x,t)}
\newcommand{\Efield}{E_\text{field}
\newcommand{\Efield}{E_{\text{Feld}
\newcommand{\Efield}{E_{\text{Field}
\newcommand{\Efield}{E_{\text{field}
\newcommand{\Efield}{E}
\newcommand{\Egamma}{E_\gamma}
\newcommand{\Eh}{E_h}
\newcommand{\Emu}{E_\mu}
\newcommand{\Enorm}[1]{E_{\text{norm}
\newcommand{\En}{E_n}
\newcommand{\Ep}{E_p}
\newcommand{\Eratio}[2]{\frac{E_{#1}
\newcommand{\Etau}{E_\tau}
\newcommand{\Evis}{E_{\text{vis}
\newcommand{\Exi}{E_\xi}
\newcommand{\Ezero}{E_0}
\newcommand{\GeV}{\,\text{GeV}
\newcommand{\Gnat}{G_{\text{nat}
\newcommand{\Gsi}{G_{\text{SI}
\newcommand{\Hubble}{H_0}
\newcommand{\Kfrak}{K_{\text{frac}
\newcommand{\Kfrak}{K_{\text{frak}
\newcommand{\Kspec}{K_{\text{spec}
\newcommand{\LCDM}{\Lambda\text{CDM}
\newcommand{\LPlanck}{\ell_{\text{Pl}
\newcommand{\Lag}{\mathcal{L}
\newcommand{\Lambdat}{\Lambda_T}
\newcommand{\Leff}{L_{\text{eff}
\newcommand{\Lorentz}[2]{{\Lambda^\mu{}
\newcommand{\Lp}{L_{\text{P}
\newcommand{\Lxi}{L_\xi}
\newcommand{\Lzero}{L_0}
\newcommand{\MPl}{M_{\text{Pl}
\newcommand{\MSbar}{\overline{\text{MS}
\newcommand{\MeV}{\,\text{MeV}
\newcommand{\Mpl}{M_{\text{Pl}
\newcommand{\OmegaDM}{\Omega_{\text{DM}
\newcommand{\OmegaLambda}{\Omega_{\Lambda}
\newcommand{\Omegab}{\Omega_b}
\newcommand{\Phiphoton}{\Phi_{\text{photon}
\newcommand{\Ricci}{R_{\mu\nu}
\newcommand{\Riem}{R^\rho{}
\newcommand{\Rzero}{R_\infty}
\newcommand{\Scal}{R}
\newcommand{\SynchPower}{P_{\text{synch}
\newcommand{\TPlanck}{t_{\text{Pl}
\newcommand{\Tfieldt}{T(\vec{x}
\newcommand{\Tfieldt}{T(x,t)}
\newcommand{\Tfield}{T(x)}
\newcommand{\Tfield}{T(x,t)}
\newcommand{\Tfield}{T_{\text{field}
\newcommand{\Tfield}{T}
\newcommand{\Tfield}{\mathcal{T}
\newcommand{\Tzerot}{T_0(\Tfield)}
\newcommand{\Tzero}{T_0}
\newcommand{\Weyl}{C^\rho{}
\newcommand{\ZPinch}{J \times B = \nabla p}
\newcommand{\aleph}{\aleph}
\newcommand{\alphaEMSI}{\alpha_{\text{EM,SI}
\newcommand{\alphaEMnat}{\alpha_{\text{EM,nat}
\newcommand{\alphaEM}{\alpha_{\text{EM}
\newcommand{\alphaEM}{\ensuremath{\alpha_{\text{EM}
\newcommand{\alphaQCD}{\alpha_s}
\newcommand{\alphaQED}{\alpha_{\text{QED}
\newcommand{\alphaSI}{\alpha_{\text{SI}
\newcommand{\alphaT}{\alpha_{\text{T}
\newcommand{\alphaWSI}{\alpha_{\text{W,SI}
\newcommand{\alphaWnat}{\alpha_{\text{W,nat}
\newcommand{\alphaW}{\alpha_{\text{W}
\newcommand{\alphaem}{\alpha_{EM}
\newcommand{\alphaem}{\alpha}
\newcommand{\alphafine}{\alpha}
\newcommand{\alphagem}{\alpha}
\newcommand{\alphanat}{\alpha_{\text{nat}
\newcommand{\alphapar}{\alpha}
\newcommand{\betaTSI}{\beta_{\text{T,SI}
\newcommand{\betaTnat}{\beta_{\text{T,nat}
\newcommand{\betaT}{\beta_T}
\newcommand{\betaT}{\beta_{T}
\newcommand{\betaT}{\beta_{\text{T}
\newcommand{\betaT}{\ensuremath{\beta_T}
\newcommand{\betapar}{\beta}
\newcommand{\calL}{\mathcal{L}
\newcommand{\checked}{\checkmark}
\newcommand{\checkmarkx}{\checkmark}
\newcommand{\dTdt}{\frac{d\Tfieldt}
\newcommand{\deltaE}{\delta E}
\newcommand{\deltafield}{\ensuremath{\delta m}
\newcommand{\deltam}{\delta m}
\newcommand{\deq}{\displaystyle}
\newcommand{\docref}[1]{\texttt{#1}
\newcommand{\eV}{\,\text{eV}
\newcommand{\epsilonT}{\varepsilon_T}
\newcommand{\epsilonzero}{\varepsilon_0}
\newcommand{\etavis}{\eta_{\text{visual}
\newcommand{\e}{\mathrm{e}
\newcommand{\gW}{g_W}
\newcommand{\gammaf}{\gamma_{\text{Lorentz}
\newcommand{\gammamu}{\gamma^\mu}
\newcommand{\gs}{g_s}
\newcommand{\inftytext}{$\infty$}
\newcommand{\interval}[2]{#1:#2}
\newcommand{\kfrac}{K_{\text{frak}
\newcommand{\lP}{\ell_{\text{P}
\newcommand{\lP}{l_P}
\newcommand{\lambdah}{\ensuremath{\lambda_h}
\newcommand{\lambdah}{\lambda_h}
\newcommand{\lambdazero}{\lambda_0}
\newcommand{\mP}{m_{\text{P}
\newcommand{\mfield}{m(x,t)}
\newcommand{\mfield}{m}
\newcommand{\mh}{m_h}
\newcommand{\micrometer}{\ensuremath{\mu}
\newcommand{\mikrometer}{\ensuremath{\mu}
\newcommand{\myRightarrow}{\ensuremath{\Rightarrow}
\newcommand{\myapprox}{\ensuremath{\approx}
\newcommand{\myomega}{\ensuremath{\omega}
\newcommand{\myphi}{\ensuremath{\phi}
\newcommand{\mypi}{\ensuremath{\pi}
\newcommand{\mypropto}{\ensuremath{\propto}
\newcommand{\myrightarrow}{\ensuremath{\rightarrow}
\newcommand{\mysim}{\ensuremath{\sim}
\newcommand{\mysqrt}{\ensuremath{\sqrt}
\newcommand{\mytimes}{\ensuremath{\times}
\newcommand{\natunits}{\hbar = c = G = k_B = 1}
\newcommand{\natunits}{\text{(nat. Einh.)}
\newcommand{\natunits}{\text{(nat. units)}
\newcommand{\nulep}{\nu}
\newcommand{\nuzero}{\nu_0}
\newcommand{\partialop}{\ensuremath{\partial}
\newcommand{\pdTdt}{\frac{\partial\Tfieldt}
\newcommand{\pdTdx}{\nabla\Tfieldt}
\newcommand{\phiT}{\phi}
\newcommand{\pichar}{\pi}
\newcommand{\primrel}[1]{\mathbf{#1}
\newcommand{\rhoCMB}{\rho_{\text{CMB}
\newcommand{\rhoCasimir}{\rho_{\text{Casimir}
\newcommand{\rhoE}{\rho_E}
\newcommand{\rhofield}{\ensuremath{\rho}
\newcommand{\rzero}{r_0}
\newcommand{\slashk}{\cancel{k}
\newcommand{\slashp}{\cancel{p}
\newcommand{\slashq}{\cancel{q}
\newcommand{\tP}{t_P}
\newcommand{\tP}{t_{\text{P}
\newcommand{\tablescale}{0.9}
\newcommand{\tzero}{t_0}
\newcommand{\vect}[1]{\boldsymbol{#1}
\newcommand{\vecx}{\vec{x}
\newcommand{\vh}{v}
\newcommand{\vr}{\vec{r}
\newcommand{\warningx}{\color{red}
\newcommand{\warningx}{\textbf{!}
\newcommand{\warningx}{{\color{red}
\newcommand{\xiT}{\xi}
\newcommand{\xiconst}{\xi = \frac{4}
\newcommand{\xicoupling}{f(E/\Exi)}
\newcommand{\xigeom}{\xi_{\text{geom}
\newcommand{\xigeom}{\xi}
\newcommand{\xikonst}{\xi = \frac{4}
\newcommand{\xiparticle}{\xi_{\text{particle}
\newcommand{\xipar}{\ensuremath{\xi}
\newcommand{\xipar}{\xi_0}
\newcommand{\xipar}{\xi}
\newcommand{\xirat}{\xi_{\text{ratio}
\newtheorem{axiom}{Axiom}
\newtheorem{category}{Category-Theoretic Basis}
\newtheorem{category}{Kategorientheoretische Basis}
\newtheorem{corollary}[theorem]{Corollary}
\newtheorem{corollary}[theorem]{Korollar}
\newtheorem{corollary}{Corollary}
\newtheorem{corollary}{Korollar}
\newtheorem{definition}[theorem]{Definition}
\newtheorem{definition}{Definition}
\newtheorem{discovery}{Discovery}
\newtheorem{discovery}{Neue Entdeckung}
\newtheorem{discovery}{New Discovery}
\newtheorem{discovery}{Revolutionary Discovery}
\newtheorem{entdeckung}{Entdeckung}
\newtheorem{entdeckung}{Revolutionäre Entdeckung}
\newtheorem{erkenntnis}{Erkenntnis}
\newtheorem{erkenntnis}{Schlüsselerkenntnis}
\newtheorem{example}[theorem]{Beispiel}
\newtheorem{example}[theorem]{Example}
\newtheorem{example}{Beispiel}
\newtheorem{example}{Example}
\newtheorem{insight}{Central Insight}
\newtheorem{insight}{Insight}
\newtheorem{insight}{Key Insight}
\newtheorem{insight}{Wichtige Einsicht}
\newtheorem{insight}{Zentrale Einsicht}
\newtheorem{lemma}[theorem]{Lemma}
\newtheorem{lemma}{Lemma}
\newtheorem{principle}{Fundamental Principle}
\newtheorem{principle}{Fundamentales Prinzip}
\newtheorem{principle}{Grundlegendes Prinzip}
\newtheorem{principle}{Principle}
\newtheorem{principle}{Prinzip}
\newtheorem{prinzip}{Grundprinzip}
\newtheorem{proof_step}{Beweisschritt}
\newtheorem{proof_step}{Proof Step}
\newtheorem{proposition}[theorem]{Proposition}
\newtheorem{proposition}{Proposition}
\newtheorem{remark}[theorem]{Bemerkung}
\newtheorem{remark}[theorem]{Remark}
\newtheorem{theorem}{Theorem}
\newtheorem{warning}[theorem]{Warning}
\newtheorem{warning}[theorem]{Warnung}
\newunicodechar{±}{\ensuremath{\pm}
\newunicodechar{×}{\ensuremath{\times}
\newunicodechar{÷}{\ensuremath{\div}
\newunicodechar{ħ}{\ensuremath{\hbar}
\newunicodechar{Α}{\ensuremath{A}
\newunicodechar{Β}{\ensuremath{B}
\newunicodechar{Γ}{\ensuremath{\Gamma}
\newunicodechar{Δ}{\ensuremath{\Delta}
\newunicodechar{Ε}{\ensuremath{E}
\newunicodechar{Ζ}{\ensuremath{Z}
\newunicodechar{Η}{\ensuremath{H}
\newunicodechar{Θ}{\ensuremath{\Theta}
\newunicodechar{Ι}{\ensuremath{I}
\newunicodechar{Κ}{\ensuremath{K}
\newunicodechar{Λ}{\ensuremath{\Lambda}
\newunicodechar{Μ}{\ensuremath{M}
\newunicodechar{Ν}{\ensuremath{N}
\newunicodechar{Ξ}{\ensuremath{\Xi}
\newunicodechar{Ο}{\ensuremath{O}
\newunicodechar{Π}{\ensuremath{\Pi}
\newunicodechar{Ρ}{\ensuremath{P}
\newunicodechar{Σ}{\ensuremath{\Sigma}
\newunicodechar{Τ}{\ensuremath{T}
\newunicodechar{Υ}{\ensuremath{\Upsilon}
\newunicodechar{Φ}{\ensuremath{\Phi}
\newunicodechar{Χ}{\ensuremath{X}
\newunicodechar{Ψ}{\ensuremath{\Psi}
\newunicodechar{Ω}{\ensuremath{\Omega}
\newunicodechar{α}{\ensuremath{\alpha}
\newunicodechar{β}{\ensuremath{\beta}
\newunicodechar{γ}{\ensuremath{\gamma}
\newunicodechar{δ}{\ensuremath{\delta}
\newunicodechar{ε}{\ensuremath{\varepsilon}
\newunicodechar{ζ}{\ensuremath{\zeta}
\newunicodechar{η}{\ensuremath{\eta}
\newunicodechar{θ}{\ensuremath{\theta}
\newunicodechar{ι}{\ensuremath{\iota}
\newunicodechar{κ}{\ensuremath{\kappa}
\newunicodechar{λ}{\ensuremath{\lambda}
\newunicodechar{μ}{\ensuremath{\mu}
\newunicodechar{ν}{\ensuremath{\nu}
\newunicodechar{ξ}{\ensuremath{\xi}
\newunicodechar{ο}{\ensuremath{o}
\newunicodechar{π}{\ensuremath{\pi}
\newunicodechar{ρ}{\ensuremath{\rho}
\newunicodechar{σ}{\ensuremath{\sigma}
\newunicodechar{τ}{\ensuremath{\tau}
\newunicodechar{υ}{\ensuremath{\upsilon}
\newunicodechar{φ}{\ensuremath{\phi}
\newunicodechar{φ}{\ensuremath{\varphi}
\newunicodechar{χ}{\ensuremath{\chi}
\newunicodechar{ψ}{\ensuremath{\psi}
\newunicodechar{ω}{\ensuremath{\omega}
\newunicodechar{←}{\ensuremath{\leftarrow}
\newunicodechar{→}{\ensuremath{\rightarrow}
\newunicodechar{↔}{\ensuremath{\leftrightarrow}
\newunicodechar{⇐}{\ensuremath{\Leftarrow}
\newunicodechar{⇒}{\ensuremath{\Rightarrow}
\newunicodechar{⇔}{\ensuremath{\Leftrightarrow}
\newunicodechar{∂}{\ensuremath{\partial}
\newunicodechar{∅}{\ensuremath{\emptyset}
\newunicodechar{∇}{\ensuremath{\nabla}
\newunicodechar{∈}{\ensuremath{\in}
\newunicodechar{∉}{\ensuremath{\notin}
\newunicodechar{∏}{\ensuremath{\prod}
\newunicodechar{∑}{\ensuremath{\sum}
\newunicodechar{√}{\ensuremath{\sqrt}
\newunicodechar{∝}{\ensuremath{\propto}
\newunicodechar{∞}{\ensuremath{\infty}
\newunicodechar{∩}{\ensuremath{\cap}
\newunicodechar{∪}{\ensuremath{\cup}
\newunicodechar{∫}{\ensuremath{\int}
\newunicodechar{≈}{\ensuremath{\approx}
\newunicodechar{≠}{\ensuremath{\neq}
\newunicodechar{≤}{\ensuremath{\leq}
\newunicodechar{≥}{\ensuremath{\geq}
\newunicodechar{★}{\ensuremath{\star}
\newunicodechar{✓}{\checkmark}
\pgfplotsset{compat=1.17}
\pgfplotsset{compat=1.18}
\renewcommand{\cftchapfont}{\large\bfseries\color{blue}
\renewcommand{\cftchappagefont}{\large\bfseries\color{blue}
\renewcommand{\cftsecfont}{\bfseries}
\renewcommand{\cftsecfont}{\color{blue}
\renewcommand{\cftsecfont}{\large\bfseries\color{blue}
\renewcommand{\cftsecpagefont}{\bfseries}
\renewcommand{\cftsecpagefont}{\color{blue}
\renewcommand{\cftsecpagefont}{\large\bfseries\color{blue}
\renewcommand{\cftsubsecfont}{\color{blue!80!black}
\renewcommand{\cftsubsecfont}{\color{blue}
\renewcommand{\cftsubsecpagefont}{\color{blue!80!black}
\renewcommand{\cftsubsecpagefont}{\color{blue}
\renewcommand{\cftsubsubsecfont}{\color{blue!60!black}
\renewcommand{\cftsubsubsecfont}{\color{blue}
\renewcommand{\cftsubsubsecpagefont}{\color{blue!60!black}
\renewcommand{\cftsubsubsecpagefont}{\color{blue}
\renewcommand{\cfttoctitlefont}{\huge\bfseries\color{blue}
\renewcommand{\cfttoctitlefont}{\huge\bfseries}
\renewcommand{\familydefault}{\sfdefault}
\renewcommand{\footrulewidth}{0.4pt}
\renewcommand{\headrulewidth}{0.4pt}
\sisetup{locale = DE, group-separator = {.}
\sisetup{locale = DE}
\usetikzlibrary{arrows.meta,positioning,shapes.geometric}
\usetikzlibrary{decorations.pathmorphing, patterns, shapes.arrows}
\usetikzlibrary{intersections}
\usetikzlibrary{positioning, arrows.meta}
\usetikzlibrary{positioning, arrows}
\usetikzlibrary{positioning, shapes.geometric, arrows.meta}
\usetikzlibrary{positioning,shapes,arrows}

% Common settings
\setlength{\headheight}{15pt}
\pgfplotsset{compat=1.18}
\usetikzlibrary{positioning,shapes,arrows,arrows.meta}

% Hyperref setup
\hypersetup{
    colorlinks=true,
    linkcolor=blue,
    citecolor=blue,
    urlcolor=blue
}


\title{RSA En}
\author{Johann Pascher}
\date{\today}

\begin{document}

\maketitle
\tableofcontents

\begin{abstract}
		This paper presents a mathematical analysis of the T0-Shor algorithm based on energy field formulation. We examine the theoretical foundations of the time-mass duality $T(x,t) \cdot m(x,t) = 1$ and its application to integer factorization. The analysis focuses on the mathematical consistency of the field equations, computational complexity implications, and the role of the coupling parameter $\xi$ derived from Higgs field interactions. We provide rigorous derivations of the algorithm's theoretical performance characteristics and identify the fundamental assumptions underlying the T0 framework.
	\end{abstract}
	
	\tableofcontents
	\newpage
	
	# Introduction
	
	The T0-Shor algorithm represents a theoretical extension of Shor's factorization algorithm based on energy field dynamics rather than quantum mechanical superposition. This work examines the mathematical foundations of this approach without making claims about practical implementability or superiority over existing methods.
	
	## Theoretical Framework
	
	The T0 model introduces the following fundamental mathematical structures:
	
	
```math-align

		\text{Time-Mass Duality}: \quad &T(x,t) \cdot m(x,t) = 1 \label{eq:duality}\\
		\text{Field Equation}: \quad &\nabla^2 T(x) = -\frac{\rho(x)}{T(x)^2} \label{eq:field}\\
		\text{Energy Evolution}: \quad &\frac{\partial^2 E}{\partial t^2} = -\omega^2 E \label{eq:evolution}
	
```

	
	The coupling parameter $\xi$ is theoretically derived from Higgs field interactions:
	
```math-equation

		\xi = g_H \cdot \frac{\langle\phi\rangle}{v_{EW}} \label{eq:xi_higgs}
	
```

	where $g_H$ is the Higgs coupling constant, $\langle\phi\rangle$ is the vacuum expectation value, and $v_{EW} = 246$ GeV is the electroweak scale.
	
	# Mathematical Foundations
	
	## Wave-Like Behavior of T0-Fields
	
	The T0-field exhibits wave-like propagation characteristics analogous to acoustic waves in media. The fundamental wave equation for T0-fields is:
	
	
```math-equation

		\nabla^2 T - \frac{1}{c_{T0}^2} \frac{\partial^2 T}{\partial t^2} = -\frac{\rho(x,t)}{T(x,t)^2} \label{eq:wave_equation}
	
```

	
	where $c_{T0}$ is the T0-field propagation velocity in the medium, analogous to sound velocity.
	
	## Medium-Dependent Properties
	
	Similar to acoustic waves, T0-field propagation depends critically on medium properties:
	
	\textbf{T0-field velocity in different media}:
	
```math-align

		c_{T0,vacuum} &= c \sqrt{\frac{\xi_0}{\xi_{vacuum}}} \\
		c_{T0,metal} &= c \sqrt{\frac{\xi_0 \epsilon_r}{\xi_{vacuum}}} \\
		c_{T0,dielectric} &= \frac{c}{\sqrt{\epsilon_r \mu_r}} \sqrt{\frac{\xi_0}{\xi_{vacuum}}} \\
		c_{T0,plasma} &= c \sqrt{1 - \frac{\omega_p^2}{\omega^2}} \sqrt{\frac{\xi_0}{\xi_{vacuum}}}
	
```

	
	where $\omega_p$ is the plasma frequency and $\epsilon_r$, $\mu_r$ are relative permittivity and permeability.
	
	## Boundary Conditions and Reflections
	
	At interfaces between different media, T0-fields satisfy boundary conditions similar to electromagnetic waves:
	
	\textbf{Continuity conditions}:
	
```math-align

		T_1|_{interface} &= T_2|_{interface} \quad \text{(field continuity)} \\
		\frac{1}{m_1} \frac{\partial T_1}{\partial n}\bigg|_{interface} &= \frac{1}{m_2} \frac{\partial T_2}{\partial n}\bigg|_{interface} \quad \text{(flux continuity)}
	
```

	
	\textbf{Reflection and transmission coefficients}:
	
```math-align

		r &= \frac{Z_1 - Z_2}{Z_1 + Z_2} \quad \text{(reflection coefficient)} \\
		t &= \frac{2Z_1}{Z_1 + Z_2} \quad \text{(transmission coefficient)}
	
```

	
	where $Z_i = \sqrt{m_i/T_i}$ is the T0-field impedance in medium $i$.
	
	## Geometric Constraints and Cavity Resonances
	
	In bounded geometries, T0-fields form standing wave patterns with discrete eigenfrequencies:
	
	\textbf{Rectangular cavity} ($L_x \times L_y \times L_z$):
	
```math-equation

		f_{mnp} = \frac{c_{T0}}{2} \sqrt{\left(\frac{m}{L_x}\right)^2 + \left(\frac{n}{L_y}\right)^2 + \left(\frac{p}{L_z}\right)^2}
	
```

	
	\textbf{Cylindrical cavity} (radius $a$, height $h$):
	
```math-equation

		f_{mnp} = \frac{c_{T0}}{2\pi} \sqrt{\left(\frac{\chi_{mn}}{a}\right)^2 + \left(\frac{p\pi}{h}\right)^2}
	
```

	
	where $\chi_{mn}$ are zeros of Bessel functions.
	
	\textbf{Spherical cavity} (radius $R$):
	
```math-equation

		f_{nlm} = \frac{c_{T0}}{2\pi R} \sqrt{n(n+1)}
	
```

	
	## Dispersion Relations
	
	In dispersive media, the T0-field exhibits frequency-dependent propagation:
	
	
```math-equation

		\omega^2 = c_{T0}^2(\omega) k^2 + \omega_0^2
	
```

	
	where $\omega_0$ is a characteristic frequency related to the medium's microscopic structure.
	
	\textbf{Group velocity} (important for information propagation):
	
```math-equation

		v_g = \frac{d\omega}{dk} = \frac{c_{T0}^2 k}{\omega} + \frac{dc_{T0}^2}{d\omega} \frac{k^2}{2}
	
```

	
	## Hyperbolical Geometry in Duality Space
	
	The time-mass duality (Eq.~\ref{eq:duality}) defines a hyperbolic metric in the $(T,m)$ parameter space:
	
	
```math-equation

		ds^2 = \frac{dT \cdot dm}{T \cdot m} = \frac{d(\ln T) \cdot d(\ln m)}{T \cdot m}
	
```

	
	This geometry is characterized by:
	
		- Constant negative curvature: $K = -1$
		- Invariant measure: $d\mu = \frac{dT \, dm}{T \cdot m}$
		- Isometry group: $PSL(2,\mathbb{R})$
	
	
	## Field Equation Analysis
	
	For spherically symmetric configurations, Eq.~\ref{eq:field} reduces to:
	
```math-equation

		\frac{1}{r^2}\frac{d}{dr}\left(r^2 \frac{dT}{dr}\right) = -\frac{\rho(r)}{T(r)^2}
	
```

	
	For a point mass $m$ at the origin with $\rho(r) = mc^2 \delta(r)$, the solution is:
	
```math-equation

		T(r) = T_0 \left(1 - \frac{r_0}{r}\right) \quad \text{with} \quad r_0 = \frac{Gm}{c^2}
	
```

	
	where $T_0 = \hbar/(mc^2)$ and $r_0$ corresponds to the Schwarzschild radius.
	
	# T0-Shor Algorithm Formulation
	
	## Geometric Cavity Design for Period Finding
	
	The T0-Shor algorithm utilizes geometric resonance cavities to detect periods, analogous to acoustic resonators:
	
	\textbf{Resonance cavity dimensions} for period $r$:
	
```math-equation

		L_{cavity} = n \cdot \frac{\lambda_{T0}}{2} = n \cdot \frac{c_{T0} \cdot r}{2f_0}
	
```

	
	where $f_0$ is the fundamental driving frequency and $n$ is the mode number.
	
	\textbf{Quality factor} of the resonance:
	
```math-equation

		Q = \frac{f_r}{\Delta f} = \frac{\pi}{\xi} \cdot \frac{L_{cavity}}{\lambda_{T0}}
	
```

	
	Higher $Q$ values provide sharper period detection but require longer observation times.
	
	## Medium-Dependent Algorithm Optimization
	
	The algorithm efficiency depends critically on the propagation medium:
	
	\textbf{Metallic substrates}:
	
```math-align

		c_{T0,metal} &= c \sqrt{\frac{\xi_0}{\xi_0 + \sigma/(\omega \epsilon_0)}} \\
		\text{Skin depth: } \delta &= \sqrt{\frac{2}{\omega \mu_0 \sigma}} \\
		\text{Effective cavity size: } L_{eff} &= \min(L_{cavity}, \delta)
	
```

	
	\textbf{Dielectric materials}:
	
```math-align

		c_{T0,dielectric} &= \frac{c}{\sqrt{\epsilon_r}} \sqrt{\frac{\xi_0}{\xi_{vacuum}}} \\
		\text{Penetration depth: } \delta_p &= \frac{c}{\omega \sqrt{\epsilon_r}} \text{Im}(\sqrt{\epsilon_r}) \\
		\text{Loss tangent: } \tan \delta &= \frac{\epsilon''}{\epsilon'}
	
```

	
	## Boundary Condition Engineering
	
	Strategic boundary condition design enhances period detection:
	
	\textbf{Perfect conductor boundaries}:
	
```math-equation

		T|_{boundary} = 0 \quad \text{(hard boundary)}
	
```

	
	\textbf{Absorbing boundaries}:
	
```math-equation

		\frac{\partial T}{\partial n} + i\frac{\omega}{c_{T0}} T = 0 \quad \text{(radiation boundary)}
	
```

	
	\textbf{Periodic boundaries} for resonance enhancement:
	
```math-equation

		T(x + L, y, z, t) = T(x, y, z, t) \cdot e^{i k_x L}
	
```

	
	## Multi-Mode Resonance Analysis
	
	Instead of quantum Fourier transform, the T0-Shor algorithm uses multi-mode cavity analysis:
	
	
```math-align

		\text{Mode spectrum}: \quad &T(x,y,z,t) = \sum_{mnp} A_{mnp}(t) \psi_{mnp}(x,y,z) \\
		\text{Period detection}: \quad &r = \frac{c_{T0}}{2f_{resonance}} \cdot \frac{geometry\_factor}{mode\_number}
	
```

	
	\textbf{Geometry factors for different cavity shapes}:
	
```math-align

		\text{Rectangular: } G_{rect} &= \sqrt{(m/L_x)^2 + (n/L_y)^2 + (p/L_z)^2} \\
		\text{Cylindrical: } G_{cyl} &= \sqrt{(\chi_{mn}/a)^2 + (p\pi/h)^2} \\
		\text{Spherical: } G_{sph} &= \sqrt{n(n+1)}/R
	
```

	
	## Adaptive Impedance Matching
	
	For optimal energy transfer and period detection:
	
	
```math-equation

		Z_{optimal} = \sqrt{\frac{Z_{source} \cdot Z_{cavity}}{1 + (Q \cdot \Delta f / f_0)^2}}
	
```

	
	The matching network adjusts the effective mass field distribution:
	
```math-equation

		m_{matched}(r) = m_0(r) \cdot \frac{Z_{optimal}(r)}{Z_0}
	
```

	
	# Physical Implementation Considerations
	
	## Substrate Material Selection
	
	Different substrate materials provide different T0-field characteristics:
	
	\begin{table}[htbp]
		\centering
		\begin{tabular}{lccccc}
			\toprule
			\textbf{Material} & $\boldsymbol{\epsilon_r}$ & $\boldsymbol{\mu_r}$ & $\boldsymbol{c_{T0}/c}$ & $\boldsymbol{\xi_{eff}/\xi_0}$ & \textbf{Applications} \\
			\midrule
			Vacuum & 1.0 & 1.0 & 1.0 & 1.0 & Reference \\
			Silicon & 11.9 & 1.0 & 0.29 & 0.84 & Electronics \\
			Sapphire & 9.4 & 1.0 & 0.33 & 0.87 & High-Q resonators \\
			GaAs & 12.9 & 1.0 & 0.28 & 0.83 & High-speed devices \\
			Superconductor & $\infty$ & 0 & 0 & $\Delta/(k_B T_c)$ & Lossless cavities \\
			Metamaterial & $< 0$ & $< 0$ & $> 1$ & Tunable & Engineered properties \\
			\bottomrule
		\end{tabular}
		\caption{Material properties for T0-field propagation}
		\label{tab:materials}
	\end{table}
	
	## Geometric Optimization
	
	\textbf{Cavity shape optimization} for maximum period resolution:
	
	For period $r$ detection, the optimal cavity dimensions follow:
	
```math-align

		\text{Length: } L &= (2n+1) \frac{c_{T0} r}{4 f_0} \quad \text{(quarter-wave resonator)} \\
		\text{Width: } W &= \frac{c_{T0}}{2 f_0} \sqrt{1 - (f_0/f_{cutoff})^2} \\
		\text{Height: } H &= \frac{c_{T0}}{2 f_0} \sqrt{1 - (f_0/f_{cutoff})^2}
	
```

	
	\textbf{Coupling aperture design}:
	
```math-equation

		A_{aperture} = \frac{\lambda_{T0}^2}{4\pi} \cdot \frac{Q_{external}}{Q_{internal}} \cdot \sin^2\left(\frac{\pi a}{\lambda_{T0}}\right)
	
```

	
	where $a$ is the aperture dimension.
	
	## Temperature and Pressure Dependencies
	
	Environmental conditions affect T0-field propagation:
	
	\textbf{Temperature dependence}:
	
```math-equation

		c_{T0}(T) = c_{T0}(T_0) \sqrt{\frac{T}{T_0}} \left(1 + \alpha_T \Delta T + \beta_T (\Delta T)^2\right)
	
```

	
	\textbf{Pressure dependence}:
	
```math-equation

		\xi(p) = \xi_0 \left(1 + \kappa \frac{\Delta p}{p_0}\right)
	
```

	
	where $\kappa$ is the pressure coefficient.
	
	\textbf{Thermal noise limitations}:
	
```math-equation

		S_{thermal}(f) = \frac{4 k_B T R}{(1 + (2\pi f \tau)^2)} \quad \text{with } \tau = \frac{Q}{2\pi f_0}
	
```

	
	## Interface Effects and Surface Roughness
	
	Surface conditions critically affect T0-field behavior:
	
	\textbf{Surface roughness scattering}:
	
```math-equation

		\tau_{surface} = \frac{4\pi^2}{\lambda_{T0}^2} \langle h^2 \rangle \ell_c
	
```

	
	where $\langle h^2 \rangle$ is mean-square roughness and $\ell_c$ is correlation length.
	
	\textbf{Interface reflection coefficient}:
	
```math-equation

		R = \left|\frac{Z_1 \cos\theta_1 - Z_2 \cos\theta_2}{Z_1 \cos\theta_1 + Z_2 \cos\theta_2}\right|^2
	
```

	
	for oblique incidence at angle $\theta_1$.
	
	## Scaling Laws for Cavity Arrays
	
	For enhanced period detection using cavity arrays:
	
	\textbf{Coherent detection in N-cavity array}:
	
```math-equation

		SNR_{array} = \sqrt{N} \cdot SNR_{single} \cdot \eta_{coupling}
	
```

	
	where $\eta_{coupling}$ accounts for inter-cavity coupling efficiency.
	
	\textbf{Optimal spacing between cavities}:
	
```math-equation

		d_{optimal} = \frac{\lambda_{T0}}{2} \sqrt{1 + (Q/\pi)^2}
	
```

	
	\textbf{Phase coherence length}:
	
```math-equation

		L_{coherence} = c_{T0} \tau_{coherence} = \frac{c_{T0} Q}{2\pi f_0}
	
```

	
	## Resource Requirements
	
	\begin{table}[htbp]
		\centering
		\begin{tabular}{lcc}
			\toprule
			\textbf{Resource} & \textbf{Standard Shor} & \textbf{T0-Shor} \\
			\midrule
			Quantum bits & $2n + O(\log n)$ & 0 \\
			Energy fields & 0 & $2n$ \\
			Field operations & $O(n^3)$ & $O(n^{2.5})$ \\
			Memory (bits) & $O(n)$ & $O(n)$ \\
			Success probability & $\approx 0.5$ & 1.0 (theoretical) \\
			\bottomrule
		\end{tabular}
		\caption{Theoretical resource comparison for $n$-bit integer factorization}
		\label{tab:complexity}
	\end{table}
	
	## Efficiency Factor Analysis
	
	The theoretical efficiency gain depends on the optimization of the mass field:
	
	
```math-equation

		F(m) = \frac{\left(\int_0^N \sqrt{P(r|N)} \, dr\right)^2}{\int_0^N P(r|N) \, dr}
	
```

	
	For uniform distribution: $F(m) = N$
	
	For optimal Gaussian distribution with standard deviation $\sigma$:
	
```math-equation

		F(m) = \sqrt{\frac{\pi}{2}} \cdot \frac{\sigma}{\sqrt{\sigma^2 + \sigma_P^2}}
	
```

	
	where $\sigma_P$ is the natural width of the period distribution.
	
	# The Role of the $\xi$ Parameter
	
	## Higgs-Derived Coupling
	
	The theoretical derivation of $\xi$ from Higgs field interactions provides a physical foundation:
	
	
```math-equation

		\xi(E) = \xi_0 \cdot \left(\frac{E}{E_0}\right)^{\gamma}
	
```

	
	where the scaling exponent $\gamma$ depends on the energy regime:
	
```math-align

		\gamma &\approx 0 \quad \text{for } E < \Lambda_{QCD} \\
		\gamma &\approx 1/2 \quad \text{for } \Lambda_{QCD} < E < \Lambda_{EW} \\
		\gamma &\approx -1/4 \quad \text{for } E > \Lambda_{EW}
	
```

	
	## Material Dependence
	
	For electronic systems (typical energy scale $\sim 1$ eV):
	
```math-equation

		\xi_{electronic} = \xi_0 \cdot \left(\frac{1 \text{ eV}}{246 \text{ GeV}}\right)^{1/2} \approx 10^{-6} \cdot \xi_0
	
```

	
	Different materials exhibit different effective $\xi$ values:
	
```math-align

		\xi_{metal} &= \xi_0 / \sqrt{N(E_F)} \\
		\xi_{SC} &= \xi_0 \cdot \Delta/(k_B T_c) \\
		\xi_{semi} &= \xi_0 / \sqrt{m_{eff}/m_e}
	
```

	
	# Mathematical Consistency Checks
	
	## Conservation Laws
	
	The T0 framework preserves several important conservation laws:
	
	\textbf{Energy conservation in weighted form}:
	
```math-equation

		\int |E(x,t)|^2 m(x) \, dx = \text{constant}
	
```

	
	\textbf{Modified momentum conservation}:
	
```math-equation

		P = \int E^*(x) \frac{\nabla E(x)}{im(x)} \, dx = \text{constant}
	
```

	
	## Scaling Properties
	
	Under spatial scaling $x \rightarrow \lambda x$:
	
```math-align

		m(x) &\rightarrow \lambda^{-d} m(x/\lambda) \\
		T(x) &\rightarrow \lambda^d T(x/\lambda) \\
		E(x) &\rightarrow \lambda^{d/2} E(x/\lambda)
	
```

	
	where $d$ is the spatial dimension.
	
	# Stability Analysis
	
	## Linear Stability
	
	Consider perturbations around equilibrium solution $m_0(r)$:
	
```math-equation

		m(r,t) = m_0(r) + \epsilon \delta m(r) e^{\lambda t}
	
```

	
	Stability requires $\text{Re}(\lambda) < 0$ for all eigenmodes.
	
	The stability matrix for small perturbations is:
	
```math-equation

		\mathcal{L}[\delta m] = -\frac{\partial^2}{\partial r^2} + V_{eff}(r)
	
```

	
	where $V_{eff}(r)$ is an effective potential derived from the field equations.
	
	## Numerical Stability Conditions
	
	For numerical implementation, stability requires:
	
	\textbf{CFL condition}:
	
```math-equation

		\Delta t < \frac{\Delta r^2}{\max(1/m(r))}
	
```

	
	\textbf{Mass gradient constraint}:
	
```math-equation

		\left|\frac{\nabla m}{m}\right| < \frac{1}{\Delta r}
	
```

	
	# Theoretical Limitations
	
	## Information-Theoretic Bounds
	
	The fundamental search time is bounded by Shannon's entropy:
	
```math-equation

		T_{min} \geq \frac{H[P(r|N)]}{\log_2(N)}
	
```

	
	where $H[P]$ is the Shannon entropy of the period distribution.
	
	## Uncertainty Relations in T0 Framework
	
	The T0 framework introduces its own uncertainty relation:
	
```math-equation

		\Delta T \cdot \Delta m \geq \frac{\hbar}{2}
	
```

	
	This limits simultaneous localization in time and mass parameters.
	
	## Dependence on A Priori Knowledge
	
	The efficiency of the T0-Shor algorithm fundamentally depends on the quality of the a priori distribution $P(r|N)$. Without proper knowledge of this distribution, the algorithm reduces to:
	
	\textbf{Worst-case scenario}: Uniform distribution
	
```math-equation

		F(m)_{uniform} = 1 \quad \text{(no advantage)}
	
```

	
	\textbf{Best-case scenario}: Perfect prior knowledge
	
```math-equation

		F(m)_{perfect} = N \quad \text{(maximum advantage)}
	
```

	
	# Comparison with Classical Methods
	
	## Theoretical Operation Counts
	
	\begin{table}[htbp]
		\centering
		\begin{tabular}{lccc}
			\toprule
			\textbf{Method} & \textbf{Operations} & \textbf{Memory} & \textbf{Success Rate} \\
			\midrule
			Trial Division & $O(\sqrt{N})$ & $O(1)$ & 1.0 \\
			Pollard's $\rho$ & $O(N^{1/4})$ & $O(1)$ & High \\
			Quadratic Sieve & $O(\exp(\sqrt{\log N \log \log N}))$ & $O(\sqrt{N})$ & High \\
			General Number Field Sieve & $O(\exp((\log N)^{1/3}(\log \log N)^{2/3}))$ & $O(\exp(\sqrt{\log N}))$ & High \\
			Standard Shor & $O((\log N)^3)$ & $O(\log N)$ & $\approx 0.5$ \\
			T0-Shor (theoretical) & $O((\log N)^{2.5} / F(m))$ & $O(\log N)$ & 1.0 \\
			\bottomrule
		\end{tabular}
		\caption{Theoretical complexity comparison for factoring $N$-bit integers}
		\label{tab:method_comparison}
	\end{table}
	
	# Mathematical Rigor Assessment
	
	## Well-Posed Problem Analysis
	
	The T0 field equations constitute a well-posed problem if:
	
	
		- \textbf{Existence}: Solutions exist for given boundary conditions
		- \textbf{Uniqueness}: Solutions are unique
		- \textbf{Continuous dependence}: Small changes in data produce small changes in solution
	
	
	For the field equation (\ref{eq:field}), existence and uniqueness follow from standard PDE theory for elliptic equations with appropriate boundary conditions.
	
	## Dimensional Analysis Verification
	
	Checking dimensional consistency of the field equation:
	
	\textbf{Left side}: $[\nabla^2 T] = [L^{-2} \cdot T]$
	
	\textbf{Right side}: $[\rho/T^2] = [M L^{-3} \cdot T^{-2}]$
	
	For dimensional consistency, we require:
	
```math-equation

		[L^{-2} \cdot T] = [M L^{-3} \cdot T^{-2}]
	
```

	
	This implies the need for a dimensional constant with units $[M^{-1} L T^3]$, which can be related to gravitational coupling.
	
	# Conclusion
	
	## Summary of Mathematical Analysis
	
	The T0-Shor algorithm presents a mathematically consistent framework based on:
	
	
		- Hyperbolic geometry in time-mass duality space
		- Field equations derived from variational principles
		- Coupling parameter $\xi$ with theoretical foundation in Higgs physics
		- Computational complexity that scales as $O(n^{2.5}/F(m))$
	
	
	## Critical Dependencies
	
	The algorithm's theoretical advantages depend on:
	
	
		- Quality of a priori knowledge about period distribution
		- Validity of the time-mass duality assumption
		- Stability of numerical implementations
		- Physical realizability of adaptive mass fields
	
	
	## Open Mathematical Questions
	
	Several mathematical aspects require further investigation:
	
	
		- Rigorous proof of convergence for the field evolution equations
		- Analysis of non-spherically symmetric configurations
		- Study of chaotic dynamics in the mass field evolution
		- Connection between $\xi$ parameter and experimentally measurable quantities
	
	
	The T0-Shor algorithm represents an interesting theoretical construction that connects concepts from differential geometry, field theory, and computational complexity. However, its practical advantages over existing methods remain contingent on several unproven assumptions about the physical realizability of the underlying mathematical framework.

\end{document}
