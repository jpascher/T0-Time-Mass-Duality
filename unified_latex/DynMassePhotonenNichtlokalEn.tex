\documentclass[11pt,a4paper,openany]{book}

% Essential packages
\usepackage[utf8]{inputenc}
\usepackage[T1]{fontenc}
\usepackage[english]{babel}
\usepackage[a4paper,margin=2.5cm]{geometry}
\usepackage{lmodern}

% Math and physics packages
\usepackage{amsmath}
\usepackage{amssymb}
\usepackage{amsthm}
\usepackage{mathtools}
\usepackage{physics}
\usepackage{siunitx}

% Graphics and tables
\usepackage{graphicx}
\usepackage[table,xcdraw]{xcolor}
\usepackage{tikz}
\usepackage{pgfplots}
\usepackage{tcolorbox}
\usepackage{booktabs}
\usepackage{array}
\usepackage{longtable}
\usepackage{float}

% Document formatting
\usepackage{fancyhdr}
\usepackage{tocloft}
\usepackage{hyperref}
\usepackage{cleveref}
\usepackage{microtype}
\usepackage{enumitem}
\usepackage{newunicodechar}

% Additional packages (cleaned up - removed duplicates)
\usepackage{adjustbox}
\usepackage{algorithm}
\usepackage{algorithmic}
\usepackage{amsfonts}
\usepackage{bm}
\usepackage{braket}
\usepackage{breakurl}
\usepackage{cancel}
\usepackage{caption}
\usepackage{cite}
\usepackage{csquotes}
\usepackage{doi}
\usepackage{forest}
\usepackage{gensymb}
\usepackage{hyphenat}
\usepackage{listings}
\usepackage{mdframed}
\usepackage{multicol}
\usepackage{multirow}
\usepackage{natbib}
\usepackage{pdflscape}
\usepackage{ragged2e}
\usepackage{setspace}
\usepackage{slashed}
\usepackage{tabularx}
\usepackage{textcomp}
\usepackage{textgreek}
\usepackage{upgreek}
\usepackage{url}

% Color definitions (FIXED: removed extra \definecolor commands)
\definecolor{blue}{rgb}{0,0,1}
\definecolor{boxgray}{RGB}{240,240,240}
\definecolor{deepblue}{RGB}{0,0,127}
\definecolor{deepgreen}{RGB}{0,127,0}
\definecolor{deepred}{RGB}{191,0,0}
\definecolor{t0blue}{RGB}{0,102,204}
\definecolor{t0green}{RGB}{0,153,0}
\definecolor{t0orange}{RGB}{255,152,0}
\definecolor{t0purple}{RGB}{102,0,204}
\definecolor{t0red}{RGB}{204,0,0}
\definecolor{t0yellow}{RGB}{255,204,0}

% TikZ libraries
\usetikzlibrary{arrows,shapes,positioning,calc,patterns,decorations.pathmorphing,decorations.markings}

% PGFPlots setup
\pgfplotsset{compat=1.18}

% Hyperref setup
\hypersetup{
    colorlinks=true,
    linkcolor=blue,
    filecolor=magenta,
    urlcolor=cyan,
    citecolor=green,
    pdftitle={T0 Theory Document},
    pdfauthor={Johann Pascher},
    pdfsubject={T0 Theory},
    pdfkeywords={T0, physics, theory}
}

% Header and footer
\pagestyle{fancy}
\fancyhf{}
\fancyhead[LE,RO]{\thepage}
\fancyhead[RE]{\leftmark}
\fancyhead[LO]{\rightmark}
\fancyfoot[C]{T0 Theory - Johann Pascher}

% Theorem environments
\theoremstyle{definition}
\newtheorem{definition}{Definition}[section]
\newtheorem{theorem}{Theorem}[section]
\newtheorem{lemma}[theorem]{Lemma}
\newtheorem{proposition}[theorem]{Proposition}
\newtheorem{corollary}[theorem]{Corollary}
\theoremstyle{remark}
\newtheorem{remark}{Remark}[section]
\newtheorem{example}{Example}[section]

% Custom commands (common across T0 documents)
\newcommand{\T}[1]{\text{#1}}
\newcommand{\mat}[1]{\mathbf{#1}}
\newcommand{\E}{\mathrm{e}}
\newcommand{\I}{\mathrm{i}}
\newcommand{\diff}{\mathrm{d}}
\newcommand{\Real}{\mathrm{Re}}
\newcommand{\Imag}{\mathrm{Im}}


\begin{document}

\maketitle
\tableofcontents

\begin{abstract}
		This updated work examines the implications of assigning a dynamic, frequency-dependent effective mass to photons within the comprehensive framework of the T0 model, building upon the complete field-theoretic derivation and natural units system where $\hbar = c = \alpha_{\text{EM}} = \beta_{\text{T}} = 1$. The theory establishes the fundamental relationship $\Tfield = \frac{1}{\max(m, \omega)}$ with dimension $[E^{-1}]$, providing a unified treatment of massive particles and photons through the three fundamental field geometries. The dynamic photon mass $m_\gamma = \omega$ introduces energy-dependent nonlocality effects, with testable predictions. All formulations maintain strict dimensional consistency with the fixed T0 parameters $\beta = 2Gm/r$, $\xi = 2\sqrt{G} \cdot m$, and the cosmic screening factor $\xi_{\text{eff}} = \xi/2$ for infinite fields.
	\end{abstract}
	
	\tableofcontents
	\newpage
	
	# Introduction: T0 Model Foundation for Photon Dynamics
	
	This updated analysis builds upon the comprehensive T0 model framework established in the field-theoretic derivation, incorporating the complete geometric foundations and natural units system. The dynamic effective mass concept for photons emerges naturally from the T0 model's fundamental time-mass duality principle.
	
	## Fundamental T0 Model Framework
	
	The T0 model is based on the intrinsic time field definition:
	
	
```math-equation

		\boxed{\Tfield = \frac{1}{\max(m(\vec{x},t), \omega)}}
		\label{eq:intrinsic_time_field}
	
```

	
	\textbf{Dimensional verification}: $[\Tfield] = [1/E] = [E^{-1}]$ in natural units \checkmark
	
	This field satisfies the fundamental field equation:
	
```math-equation

		\nabla^2 m(\vec{x},t) = 4\pi G \rho(\vec{x},t) \cdot m(\vec{x},t)
		\label{eq:field_equation}
	
```

	
	From this foundation emerge the key parameters:
	
	\begin{tcolorbox}[colback=blue!5!white,colframe=blue!75!black,title=T0 Model Parameters for Photon Analysis]
		
```math-align

			\beta &= \frac{2Gm}{r} \quad [1] \text{ (dimensionless)} \\
			\xi &= 2\sqrt{G} \cdot m \quad [1] \text{ (dimensionless)} \\
			\beta_T &= 1 \quad [1] \text{ (natural units)} \\
			\alpha_{\text{EM}} &= 1 \quad [1] \text{ (natural units)}
		
```

	\end{tcolorbox}
	
	## Photon Integration in Time-Mass Duality
	
	For photons, the T0 model assigns an effective mass:
	
```math-equation

		m_\gamma = \omega
		\label{eq:photon_effective_mass}
	
```

	
	\textbf{Dimensional verification}: $[m_\gamma] = [\omega] = [E]$ in natural units \checkmark
	
	This gives the photon's intrinsic time field:
	
```math-equation

		\Tfield_\gamma = \frac{1}{\omega}
		\label{eq:photon_time_field}
	
```

\begin{tcolorbox}[colback=yellow!5!white,colframe=orange!75!black,title=Praktische Vereinfachung]
	\textbf{Vereinfachung:} Da alle Messungen in unserem endlichen, beobachtbaren Universum lokal erfolgen, wird nur die \textbf{lokalisierte Feldgeometrie} verwendet:
	
	$\xi = 2\sqrt{G} \cdot m$ und $\beta = \frac{2Gm}{r}$ für alle Anwendungen.
	
	Der kosmische Abschirmfaktor $\xi_{\text{eff}} = \xi/2$ entfällt.
\end{tcolorbox}	
	\textbf{Physical interpretation}: Higher-energy photons have shorter intrinsic time scales, creating energy-dependent temporal dynamics.
	
	# Energy-Dependent Nonlocality and Quantum Correlations
	
	## Entangled Photon Systems
	
	For entangled photons with energies $\omega_1$ and $\omega_2$, the time field difference is:
	
```math-equation

		\Delta T_\gamma = \left|\frac{1}{\omega_1} - \frac{1}{\omega_2}\right|
		\label{eq:time_field_difference}
	
```

	
	\textbf{Physical consequence}: Quantum correlations experience energy-dependent delays.
	
	## Modified Bell Inequality
	
	The energy-dependent time fields lead to a modified Bell inequality:
	
```math-equation

		|E(a,b) - E(a,c)| + |E(a',b) + E(a',c)| \leq 2 + \epsilon(\omega_1, \omega_2)
		\label{eq:modified_bell_inequality}
	
```

	
	where:
	
```math-equation

		\epsilon(\omega_1, \omega_2) = \alpha_{\text{corr}} \left|\frac{1}{\omega_1} - \frac{1}{\omega_2}\right| \frac{2G\langle m \rangle}{r}
		\label{eq:bell_correction}
	
```

	
	with $\alpha_{\text{corr}}$ being a correlation coupling constant and $\langle m \rangle$ the average mass in the experimental setup.
	

	# Experimental Predictions and Tests
	
	## High-Precision Quantum Optics Tests
	
	### Energy-Dependent Bell Tests
	
	Predicted time delay between entangled photons:
	
```math-equation

		\Delta t_{\text{corr}} = \frac{G\langle m \rangle}{r} \left|\frac{1}{\omega_1} - \frac{1}{\omega_2}\right|
		\label{eq:correlation_time_delay}
	
```

	
	For laboratory conditions with $\langle m \rangle \sim 10^{-3}$ kg, $r \sim 10$ m, and $\omega_1,\omega_2 \sim 1$ eV:
	
```math-equation

		\Delta t_{\text{corr}} \sim 10^{-21} \text{ s}
		\label{eq:laboratory_delay}
	
```

	

	# Dimensional Consistency Verification
	
	\begin{table}[htbp]
		\centering
		\begin{tabular}{lccl}
			\toprule
			\textbf{Equation} & \textbf{Left Side} & \textbf{Right Side} & \textbf{Status} \\
			\midrule
			Photon effective mass & $[m_\gamma] = [E]$ & $[\omega] = [E]$ & \checkmark \\
			Photon time field & $[T_\gamma] = [E^{-1}]$ & $[1/\omega] = [E^{-1}]$ & \checkmark \\
			Energy loss rate & $[d\omega/dr] = [E^2]$ & $[g_T \omega^2 2G/r^2] = [E^2]$ & \checkmark \\
			Time field difference & $[\Delta T_\gamma] = [E^{-1}]$ & $[|1/\omega_1 - 1/\omega_2|] = [E^{-1}]$ & \checkmark \\
			Bell correction & $[\epsilon] = [1]$ & $[\alpha_{\text{corr}} \Delta T_\gamma \beta] = [1]$ & \checkmark \\
			\bottomrule
		\end{tabular}
		\caption{Dimensional consistency verification for photon dynamics in T0 model}
	\end{table}
	
	# Conclusions
	
	## Summary of Key Results
	
	This updated analysis demonstrates that the dynamic photon mass concept integrates seamlessly into the comprehensive T0 model framework:
	
	
		- \textbf{Unified treatment}: Photons and massive particles follow the same fundamental relationship $T = 1/\max(m,\omega)$
		- \textbf{Energy-dependent effects}: Photon dynamics depend on frequency through the intrinsic time field
		- \textbf{Modified nonlocality}: Quantum correlations experience energy-dependent delays
		- \textbf{Testable predictions}: Specific experimental signatures distinguish T0 from standard theory
		- \textbf{Dimensional consistency}: All equations verified in natural units framework
		- \textbf{Parameter-free theory}: All effects determined by fundamental T0 parameters

\end{document}
