\documentclass[11pt,a4paper,openany]{book}

% Essential packages
\usepackage[utf8]{inputenc}
\usepackage[T1]{fontenc}
\usepackage[english]{babel}
\usepackage[a4paper,margin=2.5cm]{geometry}
\usepackage{lmodern}

% Math and physics packages
\usepackage{amsmath}
\usepackage{amssymb}
\usepackage{amsthm}
\usepackage{mathtools}
\usepackage{physics}
\usepackage{siunitx}

% Graphics and tables
\usepackage{graphicx}
\usepackage[table,xcdraw]{xcolor}
\usepackage{tikz}
\usepackage{pgfplots}
\usepackage{tcolorbox}
\usepackage{booktabs}
\usepackage{array}
\usepackage{longtable}
\usepackage{float}

% Document formatting
\usepackage{fancyhdr}
\usepackage{tocloft}
\usepackage{hyperref}
\usepackage{cleveref}
\usepackage{microtype}
\usepackage{enumitem}
\usepackage{newunicodechar}

% Additional packages (cleaned up - removed duplicates)
\usepackage{adjustbox}
\usepackage{algorithm}
\usepackage{algorithmic}
\usepackage{amsfonts}
\usepackage{bm}
\usepackage{braket}
\usepackage{breakurl}
\usepackage{cancel}
\usepackage{caption}
\usepackage{cite}
\usepackage{csquotes}
\usepackage{doi}
\usepackage{forest}
\usepackage{gensymb}
\usepackage{hyphenat}
\usepackage{listings}
\usepackage{mdframed}
\usepackage{multicol}
\usepackage{multirow}
\usepackage{natbib}
\usepackage{pdflscape}
\usepackage{ragged2e}
\usepackage{setspace}
\usepackage{slashed}
\usepackage{tabularx}
\usepackage{textcomp}
\usepackage{textgreek}
\usepackage{upgreek}
\usepackage{url}

% Color definitions (FIXED: removed extra \definecolor commands)
\definecolor{blue}{rgb}{0,0,1}
\definecolor{boxgray}{RGB}{240,240,240}
\definecolor{deepblue}{RGB}{0,0,127}
\definecolor{deepgreen}{RGB}{0,127,0}
\definecolor{deepred}{RGB}{191,0,0}
\definecolor{t0blue}{RGB}{0,102,204}
\definecolor{t0green}{RGB}{0,153,0}
\definecolor{t0orange}{RGB}{255,152,0}
\definecolor{t0purple}{RGB}{102,0,204}
\definecolor{t0red}{RGB}{204,0,0}
\definecolor{t0yellow}{RGB}{255,204,0}

% TikZ libraries
\usetikzlibrary{arrows,shapes,positioning,calc,patterns,decorations.pathmorphing,decorations.markings}

% PGFPlots setup
\pgfplotsset{compat=1.18}

% Hyperref setup
\hypersetup{
    colorlinks=true,
    linkcolor=blue,
    filecolor=magenta,
    urlcolor=cyan,
    citecolor=green,
    pdftitle={T0 Theory Document},
    pdfauthor={Johann Pascher},
    pdfsubject={T0 Theory},
    pdfkeywords={T0, physics, theory}
}

% Header and footer
\pagestyle{fancy}
\fancyhf{}
\fancyhead[LE,RO]{\thepage}
\fancyhead[RE]{\leftmark}
\fancyhead[LO]{\rightmark}
\fancyfoot[C]{T0 Theory - Johann Pascher}

% Theorem environments
\theoremstyle{definition}
\newtheorem{definition}{Definition}[section]
\newtheorem{theorem}{Theorem}[section]
\newtheorem{lemma}[theorem]{Lemma}
\newtheorem{proposition}[theorem]{Proposition}
\newtheorem{corollary}[theorem]{Corollary}
\theoremstyle{remark}
\newtheorem{remark}{Remark}[section]
\newtheorem{example}{Example}[section]

% Custom commands (common across T0 documents)
\newcommand{\T}[1]{\text{#1}}
\newcommand{\mat}[1]{\mathbf{#1}}
\newcommand{\E}{\mathrm{e}}
\newcommand{\I}{\mathrm{i}}
\newcommand{\diff}{\mathrm{d}}
\newcommand{\Real}{\mathrm{Re}}
\newcommand{\Imag}{\mathrm{Im}}


\begin{document}

\maketitle
\tableofcontents

\begin{abstract}
		This work presents the new insight that the gravitational constant $G$ is not a fundamental constant of nature but is calculable from other SI constants: $G = \ell_P^2 \times c^3 / \hbar$. The central innovation of the T0-Theory is that $G$ emerges from the geometry of spacetime, analogous to $c = 1/\sqrt{\mu_0\varepsilon_0}$ in electrodynamics. All SI constants prove to be different projections of an underlying dimensionless geometry. The perfect agreement between calculated and experimental values ($G = 6.674 \times 10^{-11}$ m³/(kg·s²)) confirms this fundamental reinterpretation of gravity.
	\end{abstract}
	
	\tableofcontents
	\newpage
	
	# The Fundamental T0-Insight
	
	\begin{revolution}[New Paradigm Shift]
		\textbf{From the T0 perspective, ALL SI constants are merely "conversion factors"!}
		
		
			- In natural units: $G = 1$, $c = 1$, $\hbar = 1$ (exactly)
			- SI values are only different descriptions of the same geometry
			- The true physics is dimensionless and geometric
		
		
		\textbf{Analogue to:} $c = 1/\sqrt{\mu_0\varepsilon_0}$ (electromagnetic structure)
		
		\textbf{Now also:} $G = f(\hbar, c, \ell_P)$ (geometric structure)
	\end{revolution}
	
	# The Fundamental Formula
	
	\begin{formula}[G from SI Constants]
		\textbf{Gravitational constant as an emergent quantity:}
		
		
```math-equation

			\boxed{G = \frac{\ell_P^2 \times c^3}{\hbar}}
		
```

		
		\textbf{Where all constants are in SI units:}
		
			- $\ell_P = 1.616 \times 10^{-35}$ m (Planck length)
			- $c = 2.998 \times 10^{8}$ m/s (Speed of light)
			- $\hbar = 1.055 \times 10^{-34}$ J$\cdot$s (Reduced Planck constant)
		
	\end{formula}
	
	# Step-by-Step Calculation
	
	## Given SI Constants
	
	\begin{table}[h]
		\centering
		\begin{tabular}{lcl}
			\toprule
			\textbf{Constant} & \textbf{Value} & \textbf{Unit} \\
			\midrule
			Planck length $\ell_P$ & $1.616 \times 10^{-35}$ & m \\
			Speed of light $c$ & $2.998 \times 10^{8}$ & m/s \\
			Reduced Planck constant $\hbar$ & $1.055 \times 10^{-34}$ & J$\cdot$s \\
			\bottomrule
		\end{tabular}
		\caption{SI Constants (from T0 perspective: conversion factors)}
	\end{table}
	
	## Numerical Calculation
	
	\textbf{Step 1: Planck length squared}
	
```math-align

		\ell_P^2 &= (1.616 \times 10^{-35})^2 \\
		&= 2.611 \times 10^{-70} \text{ m}^2
	
```

	
	\textbf{Step 2: Speed of light cubed}
	
```math-align

		c^3 &= (2.998 \times 10^{8})^3 \\
		&= 2.694 \times 10^{25} \text{ m}^3/\text{s}^3
	
```

	
	\textbf{Step 3: Calculate numerator}
	
```math-align

		\ell_P^2 \times c^3 &= 2.611 \times 10^{-70} \times 2.694 \times 10^{25} \\
		&= 7.035 \times 10^{-45} \text{ m}^5/\text{s}^3
	
```

	
	\textbf{Step 4: Division by $\hbar$}
	
```math-align

		G &= \frac{7.035 \times 10^{-45}}{1.055 \times 10^{-34}} \\
		&= 6.674 \times 10^{-11} \text{ m}^3/(\text{kg} \cdot \text{s}^2)
	
```

	
	# Result and Verification
	
	\begin{result}[Perfect Agreement]
		\textbf{Calculated result:}
		
```math-equation

			G_{\text{calculated}} = 6.674 \times 10^{-11} \text{ m}^3/(\text{kg} \cdot \text{s}^2)
		
```

		
		\textbf{Experimental value (CODATA):}
		
```math-equation

			G_{\text{experimental}} = 6.67430 \times 10^{-11} \text{ m}^3/(\text{kg} \cdot \text{s}^2)
		
```

		
		\textbf{Agreement:} Exact up to rounding errors!
	\end{result}
	
	# Dimensional Analysis
	
	## Unit Verification
	
	
```math-align

		\left[\frac{\ell_P^2 \times c^3}{\hbar}\right] &= \frac{[\text{m}]^2 \times [\text{m}/\text{s}]^3}{[\text{J} \cdot \text{s}]} \\
		&= \frac{[\text{m}]^2 \times [\text{m}]^3/[\text{s}]^3}{[\text{kg} \cdot \text{m}^2/\text{s}^2] \times [\text{s}]} \\
		&= \frac{[\text{m}]^5/[\text{s}]^3}{[\text{kg} \cdot \text{m}^2/\text{s}]} \\
		&= \frac{[\text{m}]^5/[\text{s}]^3 \times [\text{s}]}{[\text{kg} \cdot \text{m}^2]} \\
		&= \frac{[\text{m}]^5/[\text{s}]^2}{[\text{kg} \cdot \text{m}^2]} \\
		&= \frac{[\text{m}]^3}{[\text{kg} \cdot \text{s}^2]} \quad \checkmark
	
```

	
	The dimensions perfectly match those of the gravitational constant!
	
	# Physical Interpretation
	
	## What does this formula mean?
	
	
		- \textbf{$\ell_P^2$}: Planck area - fundamental geometric scale
		- \textbf{$c^3$}: Third power of the speed of light - relativistic dynamics
		- \textbf{$\hbar$}: Quantum character - smallest action
	
	
	\textbf{G arises from the combination of geometry, relativity, and quantum mechanics!}
	
	## Analogy to the electromagnetic constant
	
	\begin{table}[h]
		\centering
		\begin{tabular}{ll}
			\toprule
			\textbf{Electromagnetism} & \textbf{Gravitation} \\
			\midrule
			$c = \frac{1}{\sqrt{\mu_0\varepsilon_0}}$ & $G = \frac{\ell_P^2 \times c^3}{\hbar}$ \\
			emergent from EM vacuum & emergent from spacetime geometry \\
			$\mu_0, \varepsilon_0$ fundamental & $\ell_P, c, \hbar$ fundamental \\
			\bottomrule
		\end{tabular}
		\caption{Parallel between electromagnetic and gravitational constants}
	\end{table}
	
	# The New T0-Insight
	
	\begin{revolution}[Fundamental Paradigm Shift]
		\textbf{Traditional physics:}
		
			- $G$ is a fundamental constant of nature
			- Must be determined experimentally
			- Unexplained origin
		
		
		\textbf{T0-Physics:}
		
			- $G$ is emergent from other constants
			- Calculable from first principles
			- Origin: Geometry of spacetime
		
		
		\textbf{All SI constants are merely different projections of the underlying dimensionless T0-geometry!}
	\end{revolution}
	
	# Practical Consequences
	
	## For Experiments
	
	
		- \textbf{G-measurements} serve to verify the T0-Theory
		- \textbf{Precision experiments} can search for deviations from the T0 prediction
		- \textbf{New calibrations} become possible
	
	
	## For Theoretical Physics
	
	
		- \textbf{Unification:} One constant less in the standard model
		- \textbf{Quantum gravity:} Natural connection between $\hbar$ and $G$
		- \textbf{Cosmology:} New insights into the structure of spacetime
	
	
	# Summary
	
	\begin{formula}[The Revolutionary Insight]
		\textbf{Gravitational constant is not fundamental:}
		
		
```math-equation

			G = \frac{\ell_P^2 \times c^3}{\hbar} = 6.674 \times 10^{-11} \text{ m}^3/(\text{kg} \cdot \text{s}^2)
		
```

		
		\textbf{Key statements:}
		
			- G follows from the geometry of spacetime
			- All SI constants are conversion factors
			- The true physics is dimensionless (T0)
			- Perfect experimental agreement
		
		
		\textbf{This is the breakthrough of the T0-Theory!}
	\end{formula}

\end{document}
