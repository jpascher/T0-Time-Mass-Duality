\documentclass[11pt,a4paper,openany]{book}

% Essential packages
\usepackage[utf8]{inputenc}
\usepackage[T1]{fontenc}
\usepackage[english]{babel}
\usepackage[a4paper,margin=2.5cm]{geometry}
\usepackage{lmodern}

% Math and physics packages
\usepackage{amsmath}
\usepackage{amssymb}
\usepackage{amsthm}
\usepackage{mathtools}
\usepackage{physics}
\usepackage{siunitx}

% Graphics and tables
\usepackage{graphicx}
\usepackage[table,xcdraw]{xcolor}
\usepackage{tikz}
\usepackage{pgfplots}
\usepackage{tcolorbox}
\usepackage{booktabs}
\usepackage{array}
\usepackage{longtable}
\usepackage{float}

% Document formatting
\usepackage{fancyhdr}
\usepackage{tocloft}
\usepackage{hyperref}
\usepackage{cleveref}
\usepackage{microtype}
\usepackage{enumitem}
\usepackage{newunicodechar}

% Additional packages
\usepackage{adjustbox}
\usepackage{algorithm}
\usepackage{algorithmic}
\usepackage{amsfonts}
\usepackage{amsmath,amsfonts,amssymb}
\usepackage{amsmath,amsfonts,amssymb,physics}
\usepackage{amsmath,amssymb}
\usepackage{amsmath,amssymb,amsfonts,amsthm}
\usepackage{amsmath,amssymb,amsthm}
\usepackage{amsmath,amssymb,physics,graphicx,xcolor,amsthm}
\usepackage{bm}
\usepackage{booktabs,array,longtable,multirow}
\usepackage{braket}
\usepackage{breakurl}
\usepackage{cancel}
\usepackage{caption}
\usepackage{cite}
\usepackage{color}
\usepackage{colortbl}
\usepackage{csquotes}
\usepackage{doi}
\usepackage{forest}
\usepackage{gensymb}
\usepackage{geometry,fancyhdr}
\usepackage{graphicx,tikz,pgfplots}
\usepackage{hyperref,url}
\usepackage{hyphenat}
\usepackage{listings}
\usepackage{listings,enumerate}
\usepackage{mdframed}
\usepackage{multicol}
\usepackage{multirow}
\usepackage{natbib}
\usepackage{pdflscape}
\usepackage{ragged2e}
\usepackage{setspace}
\usepackage{siunitx,xcolor,graphicx}
\usepackage{slashed}
\usepackage{tabularx}
\usepackage{textcomp}
\usepackage{textgreek}
\usepackage{tikz,pgfplots}
\usepackage{upgreek}
\usepackage{url}

% Custom commands and definitions
\definecolor{blue}
\definecolor{blue}{rgb}{0,0,1}
\definecolor{boxgray}
\definecolor{boxgray}{RGB}{240,240,240}
\definecolor{deepblue}
\definecolor{deepblue}{RGB}{0,0,127}
\definecolor{deepgreen}
\definecolor{deepgreen}{RGB}{0,127,0}
\definecolor{deepred}
\definecolor{deepred}{RGB}{191,0,0}
\definecolor{t0blue}
\definecolor{t0blue}{RGB}{0,102,204}
\definecolor{t0blue}{RGB}{33,150,243}
\definecolor{t0green}
\definecolor{t0green}{RGB}{0,153,0}
\definecolor{t0green}{RGB}{0,153,76}
\definecolor{t0green}{RGB}{76,175,80}
\definecolor{t0orange}
\definecolor{t0orange}{RGB}{255,152,0}
\definecolor{t0purple}
\definecolor{t0purple}{RGB}{102,0,204}
\definecolor{t0purple}{RGB}{156,39,176}
\definecolor{t0red}
\definecolor{t0red}{RGB}{204,0,0}
\definecolor{t0red}{RGB}{204,0,51}
\definecolor{t0red}{RGB}{244,67,54}
\definecolor{t0yellow}
\definecolor{t0yellow}{RGB}{255,204,0}
\geometry{a4paper, left=25mm, right=25mm, top=25mm, bottom=25mm}
\geometry{a4paper, margin=1in}
\geometry{a4paper, margin=2.5cm}
\geometry{a4paper, margin=2cm}
\geometry{left=2.5cm,right=2.5cm,top=2.5cm,bottom=2.5cm}
\geometry{left=2cm,right=2cm,top=2cm,bottom=2cm}
\geometry{margin=1in}
\geometry{margin=2.5cm}
\geometry{margin=2cm}
\hypersetup{
	colorlinks=true,
	linkcolor=blue,
	citecolor=blue,
	urlcolor=blue,
	pdftitle={Analysis and Implications of MNRAS Paper 544 for the T0-Theory}
\hypersetup{
	colorlinks=true,
	linkcolor=blue,
	citecolor=blue,
	urlcolor=blue,
	pdftitle={Beweis: Die Feinstrukturkonstante α = 1 in natürlichen Einheiten}
\hypersetup{
	colorlinks=true,
	linkcolor=blue,
	citecolor=blue,
	urlcolor=blue,
	pdftitle={Beweis: Die Koide-Formel enthält implizit $\xi$}
\hypersetup{
	colorlinks=true,
	linkcolor=blue,
	citecolor=blue,
	urlcolor=blue,
	pdftitle={Chinas Photonischer Quantenchip: 1000x-Speedup und T0-Integration}
\hypersetup{
	colorlinks=true,
	linkcolor=blue,
	citecolor=blue,
	urlcolor=blue,
	pdftitle={Complete Derivation of Higgs Mass and Wilson Coefficients}
\hypersetup{
	colorlinks=true,
	linkcolor=blue,
	citecolor=blue,
	urlcolor=blue,
	pdftitle={Complete Particle Spectrum: Standard Model vs T0 Theory}
\hypersetup{
	colorlinks=true,
	linkcolor=blue,
	citecolor=blue,
	urlcolor=blue,
	pdftitle={Conceptual Comparison of Unified Natural Units and Extended Standard Model}
\hypersetup{
	colorlinks=true,
	linkcolor=blue,
	citecolor=blue,
	urlcolor=blue,
	pdftitle={Connections between the Mizohata-Takeuchi Counterexample and the T0 Time-Mass Duality Theory}
\hypersetup{
	colorlinks=true,
	linkcolor=blue,
	citecolor=blue,
	urlcolor=blue,
	pdftitle={Das Relationale Zahlensystem: Primzahlen als fundamentale Verhältnisse}
\hypersetup{
	colorlinks=true,
	linkcolor=blue,
	citecolor=blue,
	urlcolor=blue,
	pdftitle={Das T0-Modell (Planck-Referenziert): Eine Neuformulierung der Physik}
\hypersetup{
	colorlinks=true,
	linkcolor=blue,
	citecolor=blue,
	urlcolor=blue,
	pdftitle={Das T0-Modell: Zeit-Energie-Dualität und geometrische Ruhemasse}
\hypersetup{
	colorlinks=true,
	linkcolor=blue,
	citecolor=blue,
	urlcolor=blue,
	pdftitle={Der Massenskalierungsexponent κ in der T0-Theorie}
\hypersetup{
	colorlinks=true,
	linkcolor=blue,
	citecolor=blue,
	urlcolor=blue,
	pdftitle={Der geometrische Formalismus der T0-Quantenmechanik und seine Anwendung auf Quantencomputer}
\hypersetup{
	colorlinks=true,
	linkcolor=blue,
	citecolor=blue,
	urlcolor=blue,
	pdftitle={Der xi Parameter und Teilchendifferenzierung in der T0-Theorie}
\hypersetup{
	colorlinks=true,
	linkcolor=blue,
	citecolor=blue,
	urlcolor=blue,
	pdftitle={Deterministic Quantum Mechanics via T0-Energy Field Formulation}
\hypersetup{
	colorlinks=true,
	linkcolor=blue,
	citecolor=blue,
	urlcolor=blue,
	pdftitle={Deterministische Quantenmechanik via T0-Energiefeld-Formulierung}
\hypersetup{
	colorlinks=true,
	linkcolor=blue,
	citecolor=blue,
	urlcolor=blue,
	pdftitle={Die Elektroneneinheitsladung in der T0-Theorie: Jenseits von Punkt-Singularitäten}
\hypersetup{
	colorlinks=true,
	linkcolor=blue,
	citecolor=blue,
	urlcolor=blue,
	pdftitle={Die Feinstrukturkonstante: Verschiedene Darstellungen und Beziehungen}
\hypersetup{
	colorlinks=true,
	linkcolor=blue,
	citecolor=blue,
	urlcolor=blue,
	pdftitle={Die Musikalische Spirale und die 137: Die mathematische Entdeckung der kosmischen Verstimmung}
\hypersetup{
	colorlinks=true,
	linkcolor=blue,
	citecolor=blue,
	urlcolor=blue,
	pdftitle={E=mc² = E=m: Die Konstanten-Illusion entlarvt}
\hypersetup{
	colorlinks=true,
	linkcolor=blue,
	citecolor=blue,
	urlcolor=blue,
	pdftitle={E=mc² = E=m: The Constants Illusion Exposed}
\hypersetup{
	colorlinks=true,
	linkcolor=blue,
	citecolor=blue,
	urlcolor=blue,
	pdftitle={Einfache Lagrange-Revolution: Von der Standardmodell-Komplexität zur T0-Eleganz}
\hypersetup{
	colorlinks=true,
	linkcolor=blue,
	citecolor=blue,
	urlcolor=blue,
	pdftitle={Einführung in die Umsetzung photonischer Bauteile auf Wafern für Nachrichtentechniker}
\hypersetup{
	colorlinks=true,
	linkcolor=blue,
	citecolor=blue,
	urlcolor=blue,
	pdftitle={Einführung in photonische Quantenchips für Nachrichtentechniker}
\hypersetup{
	colorlinks=true,
	linkcolor=blue,
	citecolor=blue,
	urlcolor=blue,
	pdftitle={Elimination der Masse als dimensionaler Platzhalter im T0-Modell}
\hypersetup{
	colorlinks=true,
	linkcolor=blue,
	citecolor=blue,
	urlcolor=blue,
	pdftitle={Elimination of Mass as Dimensional Placeholder in the T0 Model}
\hypersetup{
	colorlinks=true,
	linkcolor=blue,
	citecolor=blue,
	urlcolor=blue,
	pdftitle={Empirical Analysis of Deterministic Factorization Methods}
\hypersetup{
	colorlinks=true,
	linkcolor=blue,
	citecolor=blue,
	urlcolor=blue,
	pdftitle={Empirische Analyse deterministischer Faktorisierungsmethoden}
\hypersetup{
	colorlinks=true,
	linkcolor=blue,
	citecolor=blue,
	urlcolor=blue,
	pdftitle={Integration der Dirac-Gleichung im T0-Modell: Natürliche-Einheiten-Rahmenwerk}
\hypersetup{
	colorlinks=true,
	linkcolor=blue,
	citecolor=blue,
	urlcolor=blue,
	pdftitle={Integration of the Dirac Equation in the T0 Model: Natural Units Framework}
\hypersetup{
	colorlinks=true,
	linkcolor=blue,
	citecolor=blue,
	urlcolor=blue,
	pdftitle={Introduction to Photonic Quantum Chips for Communication Engineers}
\hypersetup{
	colorlinks=true,
	linkcolor=blue,
	citecolor=blue,
	urlcolor=blue,
	pdftitle={Introduction to the Implementation of Photonic Components on Wafers for Communication Engineers}
\hypersetup{
	colorlinks=true,
	linkcolor=blue,
	citecolor=blue,
	urlcolor=blue,
	pdftitle={Konzeptioneller Vergleich von Einheitlichen Natürlichen Einheiten und Erweitertem Standardmodell}
\hypersetup{
	colorlinks=true,
	linkcolor=blue,
	citecolor=blue,
	urlcolor=blue,
	pdftitle={Markov Chains in the Context of T0 Theory: Deterministic or Stochastic? A Treatise on Patterns, Preconditions, and Uncertainty}
\hypersetup{
	colorlinks=true,
	linkcolor=blue,
	citecolor=blue,
	urlcolor=blue,
	pdftitle={Markov-Ketten im Kontext der T0-Theorie: Deterministisch oder stochastisch? Ein Traktat zu Mustern, Voraussetzungen und Unsicherheit}
\hypersetup{
	colorlinks=true,
	linkcolor=blue,
	citecolor=blue,
	urlcolor=blue,
	pdftitle={Mathematical Analysis of T0-Shor Algorithm: Theoretical Framework and Computational Complexity}
\hypersetup{
	colorlinks=true,
	linkcolor=blue,
	citecolor=blue,
	urlcolor=blue,
	pdftitle={Mathematical Constructs of Alternative CMB Models: Unnikrishnan and Peratt in Harmony with the T0 Theory}
\hypersetup{
	colorlinks=true,
	linkcolor=blue,
	citecolor=blue,
	urlcolor=blue,
	pdftitle={Mathematische Analyse des T0-Shor Algorithmus: Theoretischer Rahmen und Berechnungskomplexität}
\hypersetup{
	colorlinks=true,
	linkcolor=blue,
	citecolor=blue,
	urlcolor=blue,
	pdftitle={Mathematische Konstrukte alternativer CMB-Modelle: Unnikrishnan und Peratt im Einklang mit der T0-Theorie}
\hypersetup{
	colorlinks=true,
	linkcolor=blue,
	citecolor=blue,
	urlcolor=blue,
	pdftitle={Natural Unit Systems: Universal Energy Conversion and Fundamental Length Scale Hierarchy}
\hypersetup{
	colorlinks=true,
	linkcolor=blue,
	citecolor=blue,
	urlcolor=blue,
	pdftitle={Natural Units in Theoretical Physics: A Treatise in the Context of T0 Theory}
\hypersetup{
	colorlinks=true,
	linkcolor=blue,
	citecolor=blue,
	urlcolor=blue,
	pdftitle={Natürliche Einheiten in der theoretischen Physik: Eine Abhandlung im Kontext der T0-Theorie}
\hypersetup{
	colorlinks=true,
	linkcolor=blue,
	citecolor=blue,
	urlcolor=blue,
	pdftitle={Natürliche Einheitensysteme: Universelle Energieumwandlung und fundamentale Längenskala-Hierarchie}
\hypersetup{
	colorlinks=true,
	linkcolor=blue,
	citecolor=blue,
	urlcolor=blue,
	pdftitle={Parameter System-Dependency in T0-Model: SI vs. Natural Units}
\hypersetup{
	colorlinks=true,
	linkcolor=blue,
	citecolor=blue,
	urlcolor=blue,
	pdftitle={Parameter-Systemabhängigkeit im T0-Modell: SI- vs. natürliche Einheiten}
\hypersetup{
	colorlinks=true,
	linkcolor=blue,
	citecolor=blue,
	urlcolor=blue,
	pdftitle={Proof: The Fine Structure Constant α = 1 in Natural Units}
\hypersetup{
	colorlinks=true,
	linkcolor=blue,
	citecolor=blue,
	urlcolor=blue,
	pdftitle={Proof: The Koide Formula Implicitly Contains $\xi$}
\hypersetup{
	colorlinks=true,
	linkcolor=blue,
	citecolor=blue,
	urlcolor=blue,
	pdftitle={Pure Energy T0 Theory: Ratio-Based Physics with SI Reference}
\hypersetup{
	colorlinks=true,
	linkcolor=blue,
	citecolor=blue,
	urlcolor=blue,
	pdftitle={Quantum Mechanics in the T0 Model: Field-Theoretic Foundations}
\hypersetup{
	colorlinks=true,
	linkcolor=blue,
	citecolor=blue,
	urlcolor=blue,
	pdftitle={Ratio-Based vs. Absolute: The Role of Fractal Correction in T0 Theory}
\hypersetup{
	colorlinks=true,
	linkcolor=blue,
	citecolor=blue,
	urlcolor=blue,
	pdftitle={Reine Energie T0-Theorie: Verhältnis-basierte Physik mit SI-Referenz}
\hypersetup{
	colorlinks=true,
	linkcolor=blue,
	citecolor=blue,
	urlcolor=blue,
	pdftitle={Simple Lagrangian Revolution: From Standard Model Complexity to T0 Elegance}
\hypersetup{
	colorlinks=true,
	linkcolor=blue,
	citecolor=blue,
	urlcolor=blue,
	pdftitle={Simplified Dirac Equation in T0 Theory: Field Node Approach}
\hypersetup{
	colorlinks=true,
	linkcolor=blue,
	citecolor=blue,
	urlcolor=blue,
	pdftitle={Simplified T0 Theory: Elegant Lagrangian Density for Time-Mass Duality}
\hypersetup{
	colorlinks=true,
	linkcolor=blue,
	citecolor=blue,
	urlcolor=blue,
	pdftitle={T0 Cosmology: Redshift as a Geometric Path Effect in a Static Universe}
\hypersetup{
	colorlinks=true,
	linkcolor=blue,
	citecolor=blue,
	urlcolor=blue,
	pdftitle={T0 Deterministic Quantum Computing: Complete Analysis of Important Algorithms}
\hypersetup{
	colorlinks=true,
	linkcolor=blue,
	citecolor=blue,
	urlcolor=blue,
	pdftitle={T0 Deterministisches Quantencomputing: Vollständige Analyse wichtiger Algorithmen}
\hypersetup{
	colorlinks=true,
	linkcolor=blue,
	citecolor=blue,
	urlcolor=blue,
	pdftitle={T0 Model: Complete Framework - From Time-Energy Duality to Universal Constants}
\hypersetup{
	colorlinks=true,
	linkcolor=blue,
	citecolor=blue,
	urlcolor=blue,
	pdftitle={T0 Model: Complete Parameter-Free Particle Mass Calculation}
\hypersetup{
	colorlinks=true,
	linkcolor=blue,
	citecolor=blue,
	urlcolor=blue,
	pdftitle={T0 Model: Unified Neutrino Formula Structure}
\hypersetup{
	colorlinks=true,
	linkcolor=blue,
	citecolor=blue,
	urlcolor=blue,
	pdftitle={T0 Model: Universal Energy Relations for Mol and Candela Units}
\hypersetup{
	colorlinks=true,
	linkcolor=blue,
	citecolor=blue,
	urlcolor=blue,
	pdftitle={T0 Modell: Vollständiges Framework - Von Zeit-Energie-Dualität zu universellen Konstanten}
\hypersetup{
	colorlinks=true,
	linkcolor=blue,
	citecolor=blue,
	urlcolor=blue,
	pdftitle={T0 Quantenfeldtheorie: QFT, QM und Quantencomputer}
\hypersetup{
	colorlinks=true,
	linkcolor=blue,
	citecolor=blue,
	urlcolor=blue,
	pdftitle={T0 Quantum Field Theory: QFT, QM and Quantum Computers}
\hypersetup{
	colorlinks=true,
	linkcolor=blue,
	citecolor=blue,
	urlcolor=blue,
	pdftitle={T0 Theory vs Bell's Theorem: How Deterministic Energy Fields Circumvent No-Go Theorems}
\hypersetup{
	colorlinks=true,
	linkcolor=blue,
	citecolor=blue,
	urlcolor=blue,
	pdftitle={T0 Theory: Final Extension to Hadrons - Physically Derived Corrections}
\hypersetup{
	colorlinks=true,
	linkcolor=blue,
	citecolor=blue,
	urlcolor=blue,
	pdftitle={T0 Theory: The Fine-Structure Constant}
\hypersetup{
	colorlinks=true,
	linkcolor=blue,
	citecolor=blue,
	urlcolor=blue,
	pdftitle={T0 Theory: The Gravitational Constant}
\hypersetup{
	colorlinks=true,
	linkcolor=blue,
	citecolor=blue,
	urlcolor=blue,
	pdftitle={T0-Kosmologie: Rotverschiebung als geometrischer Pfad-Effekt im statischen Universum}
\hypersetup{
	colorlinks=true,
	linkcolor=blue,
	citecolor=blue,
	urlcolor=blue,
	pdftitle={T0-Model: Complete Document Analysis and Structured Summary}
\hypersetup{
	colorlinks=true,
	linkcolor=blue,
	citecolor=blue,
	urlcolor=blue,
	pdftitle={T0-Model: Kinetic Energy of Electrons and Photons}
\hypersetup{
	colorlinks=true,
	linkcolor=blue,
	citecolor=blue,
	urlcolor=blue,
	pdftitle={T0-Model: The Hubble Parameter in Static Universe}
\hypersetup{
	colorlinks=true,
	linkcolor=blue,
	citecolor=blue,
	urlcolor=blue,
	pdftitle={T0-Modell-Verifikation: Skalen-Verhältnis-basierte Berechnungen}
\hypersetup{
	colorlinks=true,
	linkcolor=blue,
	citecolor=blue,
	urlcolor=blue,
	pdftitle={T0-Modell: Bewegungsenergie von Elektronen und Photonen}
\hypersetup{
	colorlinks=true,
	linkcolor=blue,
	citecolor=blue,
	urlcolor=blue,
	pdftitle={T0-Modell: Die Hubble-Konstante im statischen Universum}
\hypersetup{
	colorlinks=true,
	linkcolor=blue,
	citecolor=blue,
	urlcolor=blue,
	pdftitle={T0-Modell: Einheitliche Neutrino-Formel-Struktur}
\hypersetup{
	colorlinks=true,
	linkcolor=blue,
	citecolor=blue,
	urlcolor=blue,
	pdftitle={T0-Modell: Universelle Energiebeziehungen für Mol- und Candela-Einheiten}
\hypersetup{
	colorlinks=true,
	linkcolor=blue,
	citecolor=blue,
	urlcolor=blue,
	pdftitle={T0-Modell: Vollständige Dokumentenanalyse und strukturierte Zusammenfassung}
\hypersetup{
	colorlinks=true,
	linkcolor=blue,
	citecolor=blue,
	urlcolor=blue,
	pdftitle={T0-Modell: Vollständige parameterfreie Teilchenmassen-Berechnung}
\hypersetup{
	colorlinks=true,
	linkcolor=blue,
	citecolor=blue,
	urlcolor=blue,
	pdftitle={T0-QAT: $\xi$-Aware Quantization-Aware Training}
\hypersetup{
	colorlinks=true,
	linkcolor=blue,
	citecolor=blue,
	urlcolor=blue,
	pdftitle={T0-QFT ML Addendum: Machine Learning Derived Extensions}
\hypersetup{
	colorlinks=true,
	linkcolor=blue,
	citecolor=blue,
	urlcolor=blue,
	pdftitle={T0-QFT ML-Addendum: Maschinelle Lern-abgeleitete Erweiterungen}
\hypersetup{
	colorlinks=true,
	linkcolor=blue,
	citecolor=blue,
	urlcolor=blue,
	pdftitle={T0-Theorie vs Bells Theorem: Wie deterministische Energiefelder No-Go-Theoreme umgehen}
\hypersetup{
	colorlinks=true,
	linkcolor=blue,
	citecolor=blue,
	urlcolor=blue,
	pdftitle={T0-Theorie: Der Terrell-Penrose-Effekt und Massenvariation}
\hypersetup{
	colorlinks=true,
	linkcolor=blue,
	citecolor=blue,
	urlcolor=blue,
	pdftitle={T0-Theorie: Die Feinstrukturkonstante}
\hypersetup{
	colorlinks=true,
	linkcolor=blue,
	citecolor=blue,
	urlcolor=blue,
	pdftitle={T0-Theorie: Die Gravitationskonstante}
\hypersetup{
	colorlinks=true,
	linkcolor=blue,
	citecolor=blue,
	urlcolor=blue,
	pdftitle={T0-Theorie: Die T0-Zeit-Masse-Dualität}
\hypersetup{
	colorlinks=true,
	linkcolor=blue,
	citecolor=blue,
	urlcolor=blue,
	pdftitle={T0-Theorie: Die sieben Rätsel}
\hypersetup{
	colorlinks=true,
	linkcolor=blue,
	citecolor=blue,
	urlcolor=blue,
	pdftitle={T0-Theorie: Erweiterung auf Bell-Tests – ML-Simulationen (November 2025)}
\hypersetup{
	colorlinks=true,
	linkcolor=blue,
	citecolor=blue,
	urlcolor=blue,
	pdftitle={T0-Theorie: Finale Erweiterung auf Hadronen - Physikalisch abgeleitete Korrekturen}
\hypersetup{
	colorlinks=true,
	linkcolor=blue,
	citecolor=blue,
	urlcolor=blue,
	pdftitle={T0-Theorie: Finale Fraktale Massenformeln (November 2025)}
\hypersetup{
	colorlinks=true,
	linkcolor=blue,
	citecolor=blue,
	urlcolor=blue,
	pdftitle={T0-Theorie: Fraktaldimension aus Lepton-Massenverhältnis}
\hypersetup{
	colorlinks=true,
	linkcolor=blue,
	citecolor=blue,
	urlcolor=blue,
	pdftitle={T0-Theorie: Fundamentale Prinzipien}
\hypersetup{
	colorlinks=true,
	linkcolor=blue,
	citecolor=blue,
	urlcolor=blue,
	pdftitle={T0-Theorie: Herleitung der Gravitationskonstanten}
\hypersetup{
	colorlinks=true,
	linkcolor=blue,
	citecolor=blue,
	urlcolor=blue,
	pdftitle={T0-Theorie: Kosmische Beziehungen und universelle $\xi$-Konstante}
\hypersetup{
	colorlinks=true,
	linkcolor=blue,
	citecolor=blue,
	urlcolor=blue,
	pdftitle={T0-Theorie: Kosmologie}
\hypersetup{
	colorlinks=true,
	linkcolor=blue,
	citecolor=blue,
	urlcolor=blue,
	pdftitle={T0-Theorie: Netzwerkdarstellung und Dimensionsanalyse in der T0-Theorie}
\hypersetup{
	colorlinks=true,
	linkcolor=blue,
	citecolor=blue,
	urlcolor=blue,
	pdftitle={T0-Theorie: Teilchenmassen}
\hypersetup{
	colorlinks=true,
	linkcolor=blue,
	citecolor=blue,
	urlcolor=blue,
	pdftitle={T0-Theorie: Vollstaendiger Abschluss}
\hypersetup{
	colorlinks=true,
	linkcolor=blue,
	citecolor=blue,
	urlcolor=blue,
	pdftitle={T0-Theory: Complete Closure}
\hypersetup{
	colorlinks=true,
	linkcolor=blue,
	citecolor=blue,
	urlcolor=blue,
	pdftitle={T0-Theory: Complete Derivation of All Parameters Without Circularity}
\hypersetup{
	colorlinks=true,
	linkcolor=blue,
	citecolor=blue,
	urlcolor=blue,
	pdftitle={T0-Theory: Cosmic Relations and universal $\xi$-constant}
\hypersetup{
	colorlinks=true,
	linkcolor=blue,
	citecolor=blue,
	urlcolor=blue,
	pdftitle={T0-Theory: Cosmology}
\hypersetup{
	colorlinks=true,
	linkcolor=blue,
	citecolor=blue,
	urlcolor=blue,
	pdftitle={T0-Theory: Derivation of the Gravitational Constant}
\hypersetup{
	colorlinks=true,
	linkcolor=blue,
	citecolor=blue,
	urlcolor=blue,
	pdftitle={T0-Theory: Extension to Bell Tests – ML Simulations (November 2025)}
\hypersetup{
	colorlinks=true,
	linkcolor=blue,
	citecolor=blue,
	urlcolor=blue,
	pdftitle={T0-Theory: Final Fractal Mass Formulas (November 2025)}
\hypersetup{
	colorlinks=true,
	linkcolor=blue,
	citecolor=blue,
	urlcolor=blue,
	pdftitle={T0-Theory: Fractal Dimension from Lepton Mass Ratio}
\hypersetup{
	colorlinks=true,
	linkcolor=blue,
	citecolor=blue,
	urlcolor=blue,
	pdftitle={T0-Theory: Fundamental Principles}
\hypersetup{
	colorlinks=true,
	linkcolor=blue,
	citecolor=blue,
	urlcolor=blue,
	pdftitle={T0-Theory: Mass Variation as an Equivalent to Time Dilation}
\hypersetup{
	colorlinks=true,
	linkcolor=blue,
	citecolor=blue,
	urlcolor=blue,
	pdftitle={T0-Theory: Network Representation and Dimensional Analysis in the T0-Theory}
\hypersetup{
	colorlinks=true,
	linkcolor=blue,
	citecolor=blue,
	urlcolor=blue,
	pdftitle={T0-Theory: Neutrinos}
\hypersetup{
	colorlinks=true,
	linkcolor=blue,
	citecolor=blue,
	urlcolor=blue,
	pdftitle={T0-Theory: Particle Masses}
\hypersetup{
	colorlinks=true,
	linkcolor=blue,
	citecolor=blue,
	urlcolor=blue,
	pdftitle={T0-Theory: The Seven Riddles}
\hypersetup{
	colorlinks=true,
	linkcolor=blue,
	citecolor=blue,
	urlcolor=blue,
	pdftitle={T0-Theory: The T0-Time-Mass Duality}
\hypersetup{
	colorlinks=true,
	linkcolor=blue,
	citecolor=blue,
	urlcolor=blue,
	pdftitle={Temperature Units in Natural Units: T0-Theory}
\hypersetup{
	colorlinks=true,
	linkcolor=blue,
	citecolor=blue,
	urlcolor=blue,
	pdftitle={Temperatureinheiten in nat\"urlichen Einheiten: T0-Theorie}
\hypersetup{
	colorlinks=true,
	linkcolor=blue,
	citecolor=blue,
	urlcolor=blue,
	pdftitle={The Electron Unit Charge in T0 Theory: Beyond Point Singularities}
\hypersetup{
	colorlinks=true,
	linkcolor=blue,
	citecolor=blue,
	urlcolor=blue,
	pdftitle={The Fine Structure Constant: Various Representations and Relationships}
\hypersetup{
	colorlinks=true,
	linkcolor=blue,
	citecolor=blue,
	urlcolor=blue,
	pdftitle={The Geometric Formalism of T0 Quantum Mechanics and its Application to Quantum Computing}
\hypersetup{
	colorlinks=true,
	linkcolor=blue,
	citecolor=blue,
	urlcolor=blue,
	pdftitle={The Mass Scaling Exponent κ in T0 Theory}
\hypersetup{
	colorlinks=true,
	linkcolor=blue,
	citecolor=blue,
	urlcolor=blue,
	pdftitle={The Musical Spiral and 137: The Mathematical Discovery of Cosmic Detuning}
\hypersetup{
	colorlinks=true,
	linkcolor=blue,
	citecolor=blue,
	urlcolor=blue,
	pdftitle={The Relational Number System: Prime Numbers as Fundamental Ratios}
\hypersetup{
	colorlinks=true,
	linkcolor=blue,
	citecolor=blue,
	urlcolor=blue,
	pdftitle={The T0 Model (Planck-Referenced): A Reformulation of Physics}
\hypersetup{
	colorlinks=true,
	linkcolor=blue,
	citecolor=blue,
	urlcolor=blue,
	pdftitle={The T0 Model: Time-Energy Duality and Geometric Rest Mass}
\hypersetup{
	colorlinks=true,
	linkcolor=blue,
	citecolor=blue,
	urlcolor=blue,
	pdftitle={The T0-Model (Planck-Referenced): A Reformulation of Physics}
\hypersetup{
	colorlinks=true,
	linkcolor=blue,
	citecolor=blue,
	urlcolor=blue,
	pdftitle={Verbindungen zwischen dem Mizohata-Takeuchi-Gegenbeispiel und der T0-Zeit-Masse-Dualitätstheorie}
\hypersetup{
	colorlinks=true,
	linkcolor=blue,
	citecolor=blue,
	urlcolor=blue,
	pdftitle={Vereinfachte Dirac-Gleichung in der T0-Theorie: Feldknoten-Ansatz}
\hypersetup{
	colorlinks=true,
	linkcolor=blue,
	citecolor=blue,
	urlcolor=blue,
	pdftitle={Vereinfachte T0-Theorie: Elegante Lagrange-Dichte für Zeit-Masse-Dualität}
\hypersetup{
	colorlinks=true,
	linkcolor=blue,
	citecolor=blue,
	urlcolor=blue,
	pdftitle={Verhältnisbasiert vs. Absolut: Die Rolle der fraktalen Korrektur in der T0-Theorie}
\hypersetup{
	colorlinks=true,
	linkcolor=blue,
	citecolor=blue,
	urlcolor=blue,
	pdftitle={Vollständige Herleitung der Higgs-Masse und Wilson-Koeffizienten}
\hypersetup{
	colorlinks=true,
	linkcolor=blue,
	citecolor=blue,
	urlcolor=blue,
	pdftitle={Vollständiges Teilchenspektrum: Standard-Modell vs T0-Theorie}
\hypersetup{
	colorlinks=true,
	linkcolor=blue,
	citecolor=blue,
	urlcolor=blue,
	pdftitle={Warum Zahlenverhältnisse nicht direkt gekürzt werden dürfen}
\hypersetup{
	colorlinks=true,
	linkcolor=blue,
	citecolor=blue,
	urlcolor=blue,
	pdftitle={Why Numerical Ratios Must Not Be Directly Simplified}
\hypersetup{
	colorlinks=true,
	linkcolor=blue,
	citecolor=blue,
	urlcolor=blue,
}
\hypersetup{
	colorlinks=true,
	linkcolor=blue,
	citecolor=red,
	urlcolor=blue,
	bookmarks=true,
	bookmarksnumbered=true,
	pdfstartview=FitH,
	pdftitle={T0 Model - Field-Theoretic Derivation of the Beta Parameter}
\hypersetup{
	colorlinks=true,
	linkcolor=blue,
	citecolor=red,
	urlcolor=blue,
	bookmarks=true,
	bookmarksnumbered=true,
	pdfstartview=FitH,
	pdftitle={T0-Modell - Feldtheoretische Herleitung des Beta-Parameters}
\hypersetup{
	colorlinks=true,
	linkcolor=blue,
	filecolor=magenta,
	urlcolor=cyan,
}
\hypersetup{
	colorlinks=true,
	linkcolor=blue,
	urlcolor=blue,
	citecolor=blue,
	pdftitle={From Time Dilation to Mass Variation: Mathematical Core Formulations of Time-Mass Duality Theory - Updated Framework}
\hypersetup{
	colorlinks=true,
	linkcolor=blue,
	urlcolor=blue,
	citecolor=blue,
	pdftitle={T0 Model: Detailed Formula for Leptonic Anomalies}
\hypersetup{
	colorlinks=true,
	linkcolor=blue,
	urlcolor=blue,
	citecolor=blue,
	pdftitle={T0 Model: Detaillierte Formel für leptonische Anomalien}
\hypersetup{
	colorlinks=true,
	linkcolor=blue,
	urlcolor=blue,
	citecolor=blue,
	pdftitle={T0 Model: Energy-based Formulas with Quadratic Scaling}
\hypersetup{
	colorlinks=true,
	linkcolor=blue,
	urlcolor=blue,
	citecolor=blue,
	pdftitle={T0 Model: Granulation, Limits and Fundamental Asymmetry}
\hypersetup{
	colorlinks=true,
	linkcolor=blue,
	urlcolor=blue,
	citecolor=blue,
	pdftitle={T0-Modell: Energiebasierte Formeln mit quadratischer Skalierung}
\hypersetup{
	colorlinks=true,
	linkcolor=blue,
	urlcolor=blue,
	citecolor=blue,
	pdftitle={T0-Modell: Granulation, Limits und fundamentale Asymmetrie}
\hypersetup{
	colorlinks=true,
	linkcolor=blue,
	urlcolor=blue,
	citecolor=blue,
	pdftitle={Von Zeitdilatation zu Massenvariation: Mathematische Kernformulierungen der Zeit-Masse-Dualitätstheorie - Aktualisiertes Framework}
\hypersetup{
	colorlinks=true,
	linkcolor=t0blue,
	citecolor=t0blue,
	urlcolor=t0blue,
	pdftitle={T0 Model: Complete Theoretical Summary}
\hypersetup{
	colorlinks=true,
	linkcolor=t0blue,
	citecolor=t0blue,
	urlcolor=t0blue,
	pdftitle={T0 Theory: Resolution of Apparent Instantaneity}
\hypersetup{
	colorlinks=true,
	linkcolor=t0blue,
	citecolor=t0blue,
	urlcolor=t0blue,
	pdftitle={T0 vs Synergetics: Vereinfachung durch natürliche Einheiten}
\hypersetup{
	colorlinks=true,
	linkcolor=t0blue,
	citecolor=t0blue,
	urlcolor=t0blue,
	pdftitle={T0-Modell: Vollständige theoretische Zusammenfassung}
\hypersetup{
	colorlinks=true,
	linkcolor=t0blue,
	citecolor=t0blue,
	urlcolor=t0blue,
	pdftitle={T0-Theorie: Auflösung der scheinbaren Instantanität}
\hypersetup{
	colorlinks=true,
	linkcolor=t0blue,
	citecolor=t0blue,
	urlcolor=t0blue,
	pdftitle={T0-Theorie: Vollständige Dokumentenübersicht}
\hypersetup{
	colorlinks=true,
	linkcolor=t0blue,
	citecolor=t0blue,
	urlcolor=t0blue,
	pdftitle={T0-Theory: Complete Document Overview}
\hypersetup{
	colorlinks=true,
	linkcolor=t0blue,
	citecolor=t0blue,
	urlcolor=t0blue,
}
\hypersetup{
	colorlinks=true,
	linkcolor=t0blue,
	citecolor=t0green,
	urlcolor=t0blue,
	pdftitle={Das verborgene Geheimnis von 1/137}
\hypersetup{
	colorlinks=true,
	linkcolor=t0blue,
	citecolor=t0green,
	urlcolor=t0blue,
	pdftitle={The Hidden Secret of 1/137}
\hypersetup{
    colorlinks=true,
    linkcolor=blue,
    citecolor=blue,
    urlcolor=blue,
    pdftitle={Analyse und Implikationen des MNRAS-Papiers 544 für die T0-Theorie}
\hypersetup{
  colorlinks=true,
  linkcolor=blue,
  citecolor=blue,
  urlcolor=blue
}
\hypersetup{
  colorlinks=true,
  linkcolor=blue,
  citecolor=blue,
  urlcolor=blue,
  pdftitle={T0-Theorie: Ein-Uhr-Metrologie und Drei-Uhren-Experiment}
\hypersetup{
  colorlinks=true,
  linkcolor=blue,
  citecolor=blue,
  urlcolor=blue,
  pdftitle={T0-Theory: Single-Clock Metrology and Three-Clock Experiment}
\hypersetup{
colorlinks=true,
linkcolor=blue,
citecolor=blue,
urlcolor=blue,
pdftitle={Quantenmechanik im T0-Modell: Feldtheoretische Grundlagen}
\hypersetup{
colorlinks=true,
linkcolor=blue,
citecolor=blue,
urlcolor=blue,
pdftitle={T0-Theory: Neutrinos}
\newcommand{\Bzero}{B_0}
\newcommand{\CQCD}{C_{\text{QCD}
\newcommand{\Cconv}{C_{\text{conv}
\newcommand{\Cto}{C_{\text{T0}
\newcommand{\Czero}{C_0}
\newcommand{\DTmu}{D_{T,\mu}
\newcommand{\DcovT}[1]{\partial_\mu #1 + #1 \partial_\mu \Tfield}
\newcommand{\Dfrak}{D_f}
\newcommand{\Df}{D_f}
\newcommand{\DhiggsT}{\Tfield (\partial_\mu + ig A_\mu) \Phi + \Phi \partial_\mu \Tfield}
\newcommand{\EPlanck}{E_P}
\newcommand{\EPlanck}{E_{\text{Pl}
\newcommand{\EPratio}[1]{\frac{#1}
\newcommand{\EP}{E_P}
\newcommand{\EP}{E_{\text{P}
\newcommand{\EW}{E_W}
\newcommand{\EZ}{E_Z}
\newcommand{\Echar}{E_{\text{char}
\newcommand{\Ee}{E_e}
\newcommand{\Efield}{E(x,t)}
\newcommand{\Efield}{E_\text{field}
\newcommand{\Efield}{E_{\text{Feld}
\newcommand{\Efield}{E_{\text{Field}
\newcommand{\Efield}{E_{\text{field}
\newcommand{\Efield}{E}
\newcommand{\Egamma}{E_\gamma}
\newcommand{\Eh}{E_h}
\newcommand{\Emu}{E_\mu}
\newcommand{\Enorm}[1]{E_{\text{norm}
\newcommand{\En}{E_n}
\newcommand{\Ep}{E_p}
\newcommand{\Eratio}[2]{\frac{E_{#1}
\newcommand{\Etau}{E_\tau}
\newcommand{\Evis}{E_{\text{vis}
\newcommand{\Exi}{E_\xi}
\newcommand{\Ezero}{E_0}
\newcommand{\GeV}{\,\text{GeV}
\newcommand{\Gnat}{G_{\text{nat}
\newcommand{\Gsi}{G_{\text{SI}
\newcommand{\Hubble}{H_0}
\newcommand{\Kfrak}{K_{\text{frac}
\newcommand{\Kfrak}{K_{\text{frak}
\newcommand{\Kspec}{K_{\text{spec}
\newcommand{\LCDM}{\Lambda\text{CDM}
\newcommand{\LPlanck}{\ell_{\text{Pl}
\newcommand{\Lag}{\mathcal{L}
\newcommand{\Lambdat}{\Lambda_T}
\newcommand{\Leff}{L_{\text{eff}
\newcommand{\Lorentz}[2]{{\Lambda^\mu{}
\newcommand{\Lp}{L_{\text{P}
\newcommand{\Lxi}{L_\xi}
\newcommand{\Lzero}{L_0}
\newcommand{\MPl}{M_{\text{Pl}
\newcommand{\MSbar}{\overline{\text{MS}
\newcommand{\MeV}{\,\text{MeV}
\newcommand{\Mpl}{M_{\text{Pl}
\newcommand{\OmegaDM}{\Omega_{\text{DM}
\newcommand{\OmegaLambda}{\Omega_{\Lambda}
\newcommand{\Omegab}{\Omega_b}
\newcommand{\Phiphoton}{\Phi_{\text{photon}
\newcommand{\Ricci}{R_{\mu\nu}
\newcommand{\Riem}{R^\rho{}
\newcommand{\Rzero}{R_\infty}
\newcommand{\Scal}{R}
\newcommand{\SynchPower}{P_{\text{synch}
\newcommand{\TPlanck}{t_{\text{Pl}
\newcommand{\Tfieldt}{T(\vec{x}
\newcommand{\Tfieldt}{T(x,t)}
\newcommand{\Tfield}{T(x)}
\newcommand{\Tfield}{T(x,t)}
\newcommand{\Tfield}{T_{\text{field}
\newcommand{\Tfield}{T}
\newcommand{\Tfield}{\mathcal{T}
\newcommand{\Tzerot}{T_0(\Tfield)}
\newcommand{\Tzero}{T_0}
\newcommand{\Weyl}{C^\rho{}
\newcommand{\ZPinch}{J \times B = \nabla p}
\newcommand{\aleph}{\aleph}
\newcommand{\alphaEMSI}{\alpha_{\text{EM,SI}
\newcommand{\alphaEMnat}{\alpha_{\text{EM,nat}
\newcommand{\alphaEM}{\alpha_{\text{EM}
\newcommand{\alphaEM}{\ensuremath{\alpha_{\text{EM}
\newcommand{\alphaQCD}{\alpha_s}
\newcommand{\alphaQED}{\alpha_{\text{QED}
\newcommand{\alphaSI}{\alpha_{\text{SI}
\newcommand{\alphaT}{\alpha_{\text{T}
\newcommand{\alphaWSI}{\alpha_{\text{W,SI}
\newcommand{\alphaWnat}{\alpha_{\text{W,nat}
\newcommand{\alphaW}{\alpha_{\text{W}
\newcommand{\alphaem}{\alpha_{EM}
\newcommand{\alphaem}{\alpha}
\newcommand{\alphafine}{\alpha}
\newcommand{\alphagem}{\alpha}
\newcommand{\alphanat}{\alpha_{\text{nat}
\newcommand{\alphapar}{\alpha}
\newcommand{\betaTSI}{\beta_{\text{T,SI}
\newcommand{\betaTnat}{\beta_{\text{T,nat}
\newcommand{\betaT}{\beta_T}
\newcommand{\betaT}{\beta_{T}
\newcommand{\betaT}{\beta_{\text{T}
\newcommand{\betaT}{\ensuremath{\beta_T}
\newcommand{\betapar}{\beta}
\newcommand{\calL}{\mathcal{L}
\newcommand{\checked}{\checkmark}
\newcommand{\checkmarkx}{\checkmark}
\newcommand{\dTdt}{\frac{d\Tfieldt}
\newcommand{\deltaE}{\delta E}
\newcommand{\deltafield}{\ensuremath{\delta m}
\newcommand{\deltam}{\delta m}
\newcommand{\deq}{\displaystyle}
\newcommand{\docref}[1]{\texttt{#1}
\newcommand{\eV}{\,\text{eV}
\newcommand{\epsilonT}{\varepsilon_T}
\newcommand{\epsilonzero}{\varepsilon_0}
\newcommand{\etavis}{\eta_{\text{visual}
\newcommand{\e}{\mathrm{e}
\newcommand{\gW}{g_W}
\newcommand{\gammaf}{\gamma_{\text{Lorentz}
\newcommand{\gammamu}{\gamma^\mu}
\newcommand{\gs}{g_s}
\newcommand{\inftytext}{$\infty$}
\newcommand{\interval}[2]{#1:#2}
\newcommand{\kfrac}{K_{\text{frak}
\newcommand{\lP}{\ell_{\text{P}
\newcommand{\lP}{l_P}
\newcommand{\lambdah}{\ensuremath{\lambda_h}
\newcommand{\lambdah}{\lambda_h}
\newcommand{\lambdazero}{\lambda_0}
\newcommand{\mP}{m_{\text{P}
\newcommand{\mfield}{m(x,t)}
\newcommand{\mfield}{m}
\newcommand{\mh}{m_h}
\newcommand{\micrometer}{\ensuremath{\mu}
\newcommand{\mikrometer}{\ensuremath{\mu}
\newcommand{\myRightarrow}{\ensuremath{\Rightarrow}
\newcommand{\myapprox}{\ensuremath{\approx}
\newcommand{\myomega}{\ensuremath{\omega}
\newcommand{\myphi}{\ensuremath{\phi}
\newcommand{\mypi}{\ensuremath{\pi}
\newcommand{\mypropto}{\ensuremath{\propto}
\newcommand{\myrightarrow}{\ensuremath{\rightarrow}
\newcommand{\mysim}{\ensuremath{\sim}
\newcommand{\mysqrt}{\ensuremath{\sqrt}
\newcommand{\mytimes}{\ensuremath{\times}
\newcommand{\natunits}{\hbar = c = G = k_B = 1}
\newcommand{\natunits}{\text{(nat. Einh.)}
\newcommand{\natunits}{\text{(nat. units)}
\newcommand{\nulep}{\nu}
\newcommand{\nuzero}{\nu_0}
\newcommand{\partialop}{\ensuremath{\partial}
\newcommand{\pdTdt}{\frac{\partial\Tfieldt}
\newcommand{\pdTdx}{\nabla\Tfieldt}
\newcommand{\phiT}{\phi}
\newcommand{\pichar}{\pi}
\newcommand{\primrel}[1]{\mathbf{#1}
\newcommand{\rhoCMB}{\rho_{\text{CMB}
\newcommand{\rhoCasimir}{\rho_{\text{Casimir}
\newcommand{\rhoE}{\rho_E}
\newcommand{\rhofield}{\ensuremath{\rho}
\newcommand{\rzero}{r_0}
\newcommand{\slashk}{\cancel{k}
\newcommand{\slashp}{\cancel{p}
\newcommand{\slashq}{\cancel{q}
\newcommand{\tP}{t_P}
\newcommand{\tP}{t_{\text{P}
\newcommand{\tablescale}{0.9}
\newcommand{\tzero}{t_0}
\newcommand{\vect}[1]{\boldsymbol{#1}
\newcommand{\vecx}{\vec{x}
\newcommand{\vh}{v}
\newcommand{\vr}{\vec{r}
\newcommand{\warningx}{\color{red}
\newcommand{\warningx}{\textbf{!}
\newcommand{\warningx}{{\color{red}
\newcommand{\xiT}{\xi}
\newcommand{\xiconst}{\xi = \frac{4}
\newcommand{\xicoupling}{f(E/\Exi)}
\newcommand{\xigeom}{\xi_{\text{geom}
\newcommand{\xigeom}{\xi}
\newcommand{\xikonst}{\xi = \frac{4}
\newcommand{\xiparticle}{\xi_{\text{particle}
\newcommand{\xipar}{\ensuremath{\xi}
\newcommand{\xipar}{\xi_0}
\newcommand{\xipar}{\xi}
\newcommand{\xirat}{\xi_{\text{ratio}
\newtheorem{axiom}{Axiom}
\newtheorem{category}{Category-Theoretic Basis}
\newtheorem{category}{Kategorientheoretische Basis}
\newtheorem{corollary}[theorem]{Corollary}
\newtheorem{corollary}[theorem]{Korollar}
\newtheorem{corollary}{Corollary}
\newtheorem{corollary}{Korollar}
\newtheorem{definition}[theorem]{Definition}
\newtheorem{definition}{Definition}
\newtheorem{discovery}{Discovery}
\newtheorem{discovery}{Neue Entdeckung}
\newtheorem{discovery}{New Discovery}
\newtheorem{discovery}{Revolutionary Discovery}
\newtheorem{entdeckung}{Entdeckung}
\newtheorem{entdeckung}{Revolutionäre Entdeckung}
\newtheorem{erkenntnis}{Erkenntnis}
\newtheorem{erkenntnis}{Schlüsselerkenntnis}
\newtheorem{example}[theorem]{Beispiel}
\newtheorem{example}[theorem]{Example}
\newtheorem{example}{Beispiel}
\newtheorem{example}{Example}
\newtheorem{insight}{Central Insight}
\newtheorem{insight}{Insight}
\newtheorem{insight}{Key Insight}
\newtheorem{insight}{Wichtige Einsicht}
\newtheorem{insight}{Zentrale Einsicht}
\newtheorem{lemma}[theorem]{Lemma}
\newtheorem{lemma}{Lemma}
\newtheorem{principle}{Fundamental Principle}
\newtheorem{principle}{Fundamentales Prinzip}
\newtheorem{principle}{Grundlegendes Prinzip}
\newtheorem{principle}{Principle}
\newtheorem{principle}{Prinzip}
\newtheorem{prinzip}{Grundprinzip}
\newtheorem{proof_step}{Beweisschritt}
\newtheorem{proof_step}{Proof Step}
\newtheorem{proposition}[theorem]{Proposition}
\newtheorem{proposition}{Proposition}
\newtheorem{remark}[theorem]{Bemerkung}
\newtheorem{remark}[theorem]{Remark}
\newtheorem{theorem}{Theorem}
\newtheorem{warning}[theorem]{Warning}
\newtheorem{warning}[theorem]{Warnung}
\newunicodechar{±}{\ensuremath{\pm}
\newunicodechar{×}{\ensuremath{\times}
\newunicodechar{÷}{\ensuremath{\div}
\newunicodechar{ħ}{\ensuremath{\hbar}
\newunicodechar{Α}{\ensuremath{A}
\newunicodechar{Β}{\ensuremath{B}
\newunicodechar{Γ}{\ensuremath{\Gamma}
\newunicodechar{Δ}{\ensuremath{\Delta}
\newunicodechar{Ε}{\ensuremath{E}
\newunicodechar{Ζ}{\ensuremath{Z}
\newunicodechar{Η}{\ensuremath{H}
\newunicodechar{Θ}{\ensuremath{\Theta}
\newunicodechar{Ι}{\ensuremath{I}
\newunicodechar{Κ}{\ensuremath{K}
\newunicodechar{Λ}{\ensuremath{\Lambda}
\newunicodechar{Μ}{\ensuremath{M}
\newunicodechar{Ν}{\ensuremath{N}
\newunicodechar{Ξ}{\ensuremath{\Xi}
\newunicodechar{Ο}{\ensuremath{O}
\newunicodechar{Π}{\ensuremath{\Pi}
\newunicodechar{Ρ}{\ensuremath{P}
\newunicodechar{Σ}{\ensuremath{\Sigma}
\newunicodechar{Τ}{\ensuremath{T}
\newunicodechar{Υ}{\ensuremath{\Upsilon}
\newunicodechar{Φ}{\ensuremath{\Phi}
\newunicodechar{Χ}{\ensuremath{X}
\newunicodechar{Ψ}{\ensuremath{\Psi}
\newunicodechar{Ω}{\ensuremath{\Omega}
\newunicodechar{α}{\ensuremath{\alpha}
\newunicodechar{β}{\ensuremath{\beta}
\newunicodechar{γ}{\ensuremath{\gamma}
\newunicodechar{δ}{\ensuremath{\delta}
\newunicodechar{ε}{\ensuremath{\varepsilon}
\newunicodechar{ζ}{\ensuremath{\zeta}
\newunicodechar{η}{\ensuremath{\eta}
\newunicodechar{θ}{\ensuremath{\theta}
\newunicodechar{ι}{\ensuremath{\iota}
\newunicodechar{κ}{\ensuremath{\kappa}
\newunicodechar{λ}{\ensuremath{\lambda}
\newunicodechar{μ}{\ensuremath{\mu}
\newunicodechar{ν}{\ensuremath{\nu}
\newunicodechar{ξ}{\ensuremath{\xi}
\newunicodechar{ο}{\ensuremath{o}
\newunicodechar{π}{\ensuremath{\pi}
\newunicodechar{ρ}{\ensuremath{\rho}
\newunicodechar{σ}{\ensuremath{\sigma}
\newunicodechar{τ}{\ensuremath{\tau}
\newunicodechar{υ}{\ensuremath{\upsilon}
\newunicodechar{φ}{\ensuremath{\phi}
\newunicodechar{φ}{\ensuremath{\varphi}
\newunicodechar{χ}{\ensuremath{\chi}
\newunicodechar{ψ}{\ensuremath{\psi}
\newunicodechar{ω}{\ensuremath{\omega}
\newunicodechar{←}{\ensuremath{\leftarrow}
\newunicodechar{→}{\ensuremath{\rightarrow}
\newunicodechar{↔}{\ensuremath{\leftrightarrow}
\newunicodechar{⇐}{\ensuremath{\Leftarrow}
\newunicodechar{⇒}{\ensuremath{\Rightarrow}
\newunicodechar{⇔}{\ensuremath{\Leftrightarrow}
\newunicodechar{∂}{\ensuremath{\partial}
\newunicodechar{∅}{\ensuremath{\emptyset}
\newunicodechar{∇}{\ensuremath{\nabla}
\newunicodechar{∈}{\ensuremath{\in}
\newunicodechar{∉}{\ensuremath{\notin}
\newunicodechar{∏}{\ensuremath{\prod}
\newunicodechar{∑}{\ensuremath{\sum}
\newunicodechar{√}{\ensuremath{\sqrt}
\newunicodechar{∝}{\ensuremath{\propto}
\newunicodechar{∞}{\ensuremath{\infty}
\newunicodechar{∩}{\ensuremath{\cap}
\newunicodechar{∪}{\ensuremath{\cup}
\newunicodechar{∫}{\ensuremath{\int}
\newunicodechar{≈}{\ensuremath{\approx}
\newunicodechar{≠}{\ensuremath{\neq}
\newunicodechar{≤}{\ensuremath{\leq}
\newunicodechar{≥}{\ensuremath{\geq}
\newunicodechar{★}{\ensuremath{\star}
\newunicodechar{✓}{\checkmark}
\pgfplotsset{compat=1.17}
\pgfplotsset{compat=1.18}
\renewcommand{\cftchapfont}{\large\bfseries\color{blue}
\renewcommand{\cftchappagefont}{\large\bfseries\color{blue}
\renewcommand{\cftsecfont}{\bfseries}
\renewcommand{\cftsecfont}{\color{blue}
\renewcommand{\cftsecfont}{\large\bfseries\color{blue}
\renewcommand{\cftsecpagefont}{\bfseries}
\renewcommand{\cftsecpagefont}{\color{blue}
\renewcommand{\cftsecpagefont}{\large\bfseries\color{blue}
\renewcommand{\cftsubsecfont}{\color{blue!80!black}
\renewcommand{\cftsubsecfont}{\color{blue}
\renewcommand{\cftsubsecpagefont}{\color{blue!80!black}
\renewcommand{\cftsubsecpagefont}{\color{blue}
\renewcommand{\cftsubsubsecfont}{\color{blue!60!black}
\renewcommand{\cftsubsubsecfont}{\color{blue}
\renewcommand{\cftsubsubsecpagefont}{\color{blue!60!black}
\renewcommand{\cftsubsubsecpagefont}{\color{blue}
\renewcommand{\cfttoctitlefont}{\huge\bfseries\color{blue}
\renewcommand{\cfttoctitlefont}{\huge\bfseries}
\renewcommand{\familydefault}{\sfdefault}
\renewcommand{\footrulewidth}{0.4pt}
\renewcommand{\headrulewidth}{0.4pt}
\sisetup{locale = DE, group-separator = {.}
\sisetup{locale = DE}
\usetikzlibrary{arrows.meta,positioning,shapes.geometric}
\usetikzlibrary{decorations.pathmorphing, patterns, shapes.arrows}
\usetikzlibrary{intersections}
\usetikzlibrary{positioning, arrows.meta}
\usetikzlibrary{positioning, arrows}
\usetikzlibrary{positioning, shapes.geometric, arrows.meta}
\usetikzlibrary{positioning,shapes,arrows}

% Common settings
\setlength{\headheight}{15pt}
\pgfplotsset{compat=1.18}
\usetikzlibrary{positioning,shapes,arrows,arrows.meta}

% Hyperref setup
\hypersetup{
    colorlinks=true,
    linkcolor=blue,
    citecolor=blue,
    urlcolor=blue
}


\title{universale-ableitung De}
\author{Johann Pascher}
\date{\today}

\begin{document}

\maketitle
\tableofcontents

\begin{abstract}
		Dieses Dokument demonstriert die revolution\"are Einfachheit der Naturgesetze: Alle fundamentalen physikalischen Konstanten in SI-Einheiten k\"onnen aus nur zwei experimentellen Grundgr\"o\ss{}en abgeleitet werden - der dimensionslosen Feinstrukturkonstante $\alpha = 1/137.036$ und der Planck-L\"ange $\ell_P = 1.616255 \times 10^{-35}$ m. Zus\"atzlich wird die Verwirrung um den Wert der charakteristischen Energie $E_0$ in der T0-Theorie aufgekl\"art und gezeigt, dass $E_0 = \SI{7.398}{\MeV}$ das exakte geometrische Mittel der CODATA-Teilchenmassen ist, nicht ein angepasster Parameter. Alle h\"aufigen Zirkularit\"ats-Einw\"ande werden systematisch entkr\"aftet. Die Herleitung reduziert die scheinbar gro\ss{}e Anzahl unabh\"angiger Naturkonstanten auf nur zwei fundamentale experimentelle Werte plus menschliche SI-Konventionen und zeigt, dass die T0-Rohwerte bereits die echten physikalischen Verh\"altnisse der Natur erfassen.
	\end{abstract}
	
	\tableofcontents
	\newpage
	
	# Einf\"uhrung und Grundprinzip
	
	## Das Minimalprinzip der Physik
	
	In der modernen Physik scheinen etwa 30 verschiedene Naturkonstanten unabh\"angig voneinander experimentell bestimmt werden zu m\"ussen. Diese Arbeit zeigt jedoch, dass alle fundamentalen Konstanten aus nur \textbf{zwei experimentellen Werten} ableitbar sind:
	
	\begin{tcolorbox}[colback=blue!5!white,colframe=blue!75!black,title=Fundamentale Eingangsdaten]
		
			- \textbf{Feinstrukturkonstante:} $\alpha = \frac{1}{137.035999084}$ (dimensionslos)
			- \textbf{Planck-L\"ange:} $\ell_P = 1.616255 \times 10^{-35}$ \si{\meter}
		
	\end{tcolorbox}
	
	## SI-Basisdefinitionen
	
	Zus\"atzlich verwenden wir die modernen SI-Basisdefinitionen (seit 2019):
	
	
```math-align

		\mu_0 &= 4\pi \times 10^{-7} \text{ H/m} \quad \text{(per Definition)}\\
		e &= 1.602176634 \times 10^{-19} \text{ C} \quad \text{(exakte Definition)}\\
		k_B &= 1.380649 \times 10^{-23} \text{ J/K} \quad \text{(exakte Definition)}\\
		N_A &= 6.02214076 \times 10^{23} \text{ mol}^{-1} \quad \text{(exakte Definition)}
	
```

	
	# Herleitung der fundamentalen Konstanten
	
	## Lichtgeschwindigkeit c
	
	Die Lichtgeschwindigkeit folgt aus der Beziehung zwischen Planck-Einheiten. Da die Planck-L\"ange definiert ist als:
	
	
```math-equation

		\ell_P = \sqrt{\frac{\hbar G}{c^3}}
	
```

	
	und alle Planck-Einheiten \"uber $\hbar$, $G$ und $c$ miteinander verkn\"upft sind, ergibt sich durch Dimensionsanalyse:
	
	\begin{tcolorbox}[colback=green!5!white,colframe=green!75!black,title=Lichtgeschwindigkeit]
		
```math-equation

			\boxed{c = 2.99792458 \times 10^8 \text{ m/s}}
		
```

	\end{tcolorbox}
	
	## Vakuum-Permittivit\"at $\varepsilon_0$
	
	Aus der Maxwell-Beziehung $\mu_0 \varepsilon_0 = 1/c^2$ folgt:
	
	
```math-equation

		\varepsilon_0 = \frac{1}{\mu_0 c^2} = \frac{1}{4\pi \times 10^{-7} \times (2.99792458 \times 10^8)^2}
	
```

	
	\begin{tcolorbox}[colback=green!5!white,colframe=green!75!black,title=Vakuum-Permittivit\"at]
		
```math-equation

			\boxed{\varepsilon_0 = 8.854187817 \times 10^{-12} \text{ F/m}}
		
```

	\end{tcolorbox}
	
	## Reduzierte Planck-Konstante $\hbar$
	
	Die Feinstrukturkonstante ist definiert als:
	
	
```math-equation

		\alpha = \frac{e^2}{4\pi\varepsilon_0\hbar c}
	
```

	
	Aufl\"osung nach $\hbar$:
	
	
```math-equation

		\hbar = \frac{e^2}{4\pi\varepsilon_0 c \alpha}
	
```

	
	Einsetzen der bekannten Werte:
	
	
```math-equation

		\hbar = \frac{(1.602176634 \times 10^{-19})^2}{4\pi \times 8.854187817 \times 10^{-12} \times 2.99792458 \times 10^8 \times \frac{1}{137.035999084}}
	
```

	
	\begin{tcolorbox}[colback=green!5!white,colframe=green!75!black,title=Reduzierte Planck-Konstante]
		
```math-equation

			\boxed{\hbar = 1.054571817 \times 10^{-34} \text{ J·s}}
		
```

	\end{tcolorbox}
	
	## Gravitationskonstante G
	
	Aus der Definition der Planck-L\"ange folgt:
	
	
```math-equation

		G = \frac{\ell_P^2 c^3}{\hbar}
	
```

	
	Einsetzen der berechneten Werte:
	
	
```math-equation

		G = \frac{(1.616255 \times 10^{-35})^2 \times (2.99792458 \times 10^8)^3}{1.054571817 \times 10^{-34}}
	
```

	
	\begin{tcolorbox}[colback=green!5!white,colframe=green!75!black,title=Gravitationskonstante]
		
```math-equation

			\boxed{G = 6.67430 \times 10^{-11} \text{ m}^3\text{/(kg·s}^2\text{)}}
		
```

	\end{tcolorbox}
	
	# Vollst\"andige Planck-Einheiten
	
	Mit $\hbar$, $c$ und $G$ k\"onnen alle Planck-Einheiten berechnet werden:
	
	## Planck-Zeit
	
	
```math-equation

		t_P = \sqrt{\frac{\hbar G}{c^5}} = \frac{\ell_P}{c} = 5.391247 \times 10^{-44} \text{ s}
	
```

	
	## Planck-Masse
	
	
```math-equation

		m_P = \sqrt{\frac{\hbar c}{G}} = 2.176434 \times 10^{-8} \text{ kg}
	
```

	
	## Planck-Energie
	
	
```math-equation

		E_P = m_P c^2 = \sqrt{\frac{\hbar c^5}{G}} = 1.956082 \times 10^9 \text{ J} = 1.220890 \times 10^{19} \text{ GeV}
	
```

	
	## Planck-Temperatur
	
	
```math-equation

		T_P = \frac{E_P}{k_B} = \frac{m_P c^2}{k_B} = 1.416784 \times 10^{32} \text{ K}
	
```

	
	# Atomare und molekulare Konstanten
	
	## Klassischer Elektronenradius
	
	Mit der Elektronenmasse $m_e = 9.1093837015 \times 10^{-31}$ kg:
	
	
```math-equation

		r_e = \frac{e^2}{4\pi\varepsilon_0 m_e c^2} = \frac{\alpha \hbar}{m_e c} = 2.817940 \times 10^{-15} \text{ m}
	
```

	
	## Compton-Wellenl\"ange des Elektrons
	
	
```math-equation

		\lambda_{C,e} = \frac{h}{m_e c} = \frac{2\pi\hbar}{m_e c} = 2.426310 \times 10^{-12} \text{ m}
	
```

	
	## Bohr-Radius
	
	
```math-equation

		a_0 = \frac{4\pi\varepsilon_0\hbar^2}{m_e e^2} = \frac{\hbar}{m_e c \alpha} = 5.291772 \times 10^{-11} \text{ m}
	
```

	
	## Rydberg-Konstante
	
	
```math-equation

		R_\infty = \frac{\alpha^2 m_e c}{2h} = \frac{\alpha^2 m_e c}{4\pi\hbar} = 1.097373 \times 10^7 \text{ m}^{-1}
	
```

	
	# Thermodynamische Konstanten
	
	## Stefan-Boltzmann-Konstante
	
	
```math-equation

		\sigma = \frac{2\pi^5 k_B^4}{15 h^3 c^2} = \frac{2\pi^5 k_B^4}{15 (2\pi\hbar)^3 c^2} = 5.670374419 \times 10^{-8} \text{ W/(m}^2\text{·K}^4\text{)}
	
```

	
	## Wien-Verschiebungsgesetz-Konstante
	
	
```math-equation

		b = \frac{hc}{k_B} \times \frac{1}{4.965114231} = 2.897771955 \times 10^{-3} \text{ m·K}
	
```

	
	# Dimensionsanalyse und Verifikation
	
	## Konsistenzpr\"ufung der Feinstrukturkonstante
	
	
```math-align

		[\alpha] &= \frac{[e^2]}{[\varepsilon_0][\hbar][c]}\\
		&= \frac{[\text{C}^2]}{[\text{F/m}][\text{J·s}][\text{m/s}]}\\
		&= \frac{[\text{C}^2]}{[\text{C}^2\text{·s}^2/(\text{kg·m}^3)][\text{J·s}][\text{m/s}]}\\
		&= \frac{[\text{C}^2]}{[\text{C}^2/(\text{kg·m}^2\text{/s}^2)]}\\
		&= [1] \quad \checkmark
	
```

	
	## Konsistenzpr\"ufung der Gravitationskonstante
	
	
```math-align

		[G] &= \frac{[\ell_P^2][c^3]}{[\hbar]}\\
		&= \frac{[\text{m}^2][\text{m}^3/\text{s}^3]}{[\text{J·s}]}\\
		&= \frac{[\text{m}^5/\text{s}^3]}{[\text{kg·m}^2/\text{s}^2\text{·s}]}\\
		&= \frac{[\text{m}^5/\text{s}^3]}{[\text{kg·m}^2/\text{s}^3]}\\
		&= [\text{m}^3/(\text{kg·s}^2)] \quad \checkmark
	
```

	
	## Konsistenzpr\"ufung von $\hbar$
	
	
```math-align

		[\hbar] &= \frac{[e^2]}{[\varepsilon_0][c][\alpha]}\\
		&= \frac{[\text{C}^2]}{[\text{F/m}][\text{m/s}][1]}\\
		&= \frac{[\text{C}^2]}{[\text{C}^2\text{·s}/(\text{kg·m}^3)][\text{m/s}]}\\
		&= \frac{[\text{C}^2\text{·kg·m}^3]}{[\text{C}^2\text{·s·m}]}\\
		&= [\text{kg·m}^2/\text{s}] = [\text{J·s}] \quad \checkmark
	
```

	
	# Die charakteristische Energie E\_0 und T0-Theorie
	
	## Definition der charakteristischen Energie
	
	\begin{tcolorbox}[colback=blue!5!white,colframe=blue!75!black,title=Grunddefinition]
		Die fundamentale Definition der charakteristischen Energie ist:
		
```math-equation

			\boxed{E_0 = \sqrt{m_e \cdot m_\mu}}
		
```

		Dies ist \textbf{keine Herleitung} und \textbf{kein Fit} -- es ist die mathematische Definition des geometrischen Mittels zweier Massen.
	\end{tcolorbox}
	
	## Numerische Auswertung mit verschiedenen Pr\"azisionsstufen
	
	### Stufe 1: Gerundete Standardwerte
	Mit den oft zitierten gerundeten Massen:
	
```math-align

		m_e &= \SI{0.511}{\MeV} \\
		m_\mu &= \SI{105.658}{\MeV} \\
		E_0^{(1)} &= \sqrt{0.511 \times 105.658} = \sqrt{53.99} = \SI{7.348}{\MeV}
	
```

	
	### Stufe 2: CODATA 2018 Pr\"azisionswerte
	Mit den exakten experimentellen Massen:
	
```math-align

		m_e &= \SI{0.5109989461}{\MeV} \\
		m_\mu &= \SI{105.6583745}{\MeV} \\
		E_0^{(2)} &= \sqrt{0.5109989461 \times 105.6583745} = \SI{7.348566}{\MeV}
	
```

	
	### Stufe 3: Der optimierte Wert E\_0 = \SI{7.398{\MeV}}
	
	\begin{tcolorbox}[colback=yellow!10!white,colframe=orange!75!black,title=Kritische Frage]
		\textbf{Ist $E_0 = \SI{7.398}{\MeV}$ ein angepasster Parameter?}
		
		\textbf{Antwort: NEIN!} 
		
		$E_0 = \SI{7.398}{\MeV}$ ist das exakte geometrische Mittel von verfeinerten CODATA-Werten, die alle experimentellen Korrekturen einschlie\ss{}en.
	\end{tcolorbox}
	
	## Pr\"azise Feinstrukturkonstanten-Berechnung
	
	Die dimensionslos korrekte Formel:
	
	
```math-equation

		\alpha = \xi \cdot \frac{E_0^2}{( \SI{1}{\MeV} )^2}
	
```

	
	wobei:
	
		- $\xi = \frac{4}{3} \times 10^{-4} = 1.333\overline{3} \times 10^{-4}$ (exakt)
		- $( \SI{1}{\MeV} )^2$ ist die Normierungsenergie f\"ur Dimensionslosigkeit
	
	
	## Vergleich der Berechnungsgenauigkeit
	
	\begin{table}[h]
		\centering
		\begin{tabular}{@{}lccc@{}}
			\toprule
			\textbf{E\_0-Wert} & \textbf{Quelle} & \textbf{$\alpha^{-1}_{\text{T0}}$} & \textbf{Abweichung} \\
			\midrule
			\SI{7.348}{\MeV} & Gerundete Massen & 139.15 & 1.5\% \\
			\SI{7.348566}{\MeV} & CODATA exakt & 139.07 & 1.4\% \\
			\textbf{\SI{7.398}{\MeV}} & \textbf{Optimiert} & \textbf{137.038} & \textbf{0.0014\%} \\
			\midrule
			\multicolumn{2}{l}{\textbf{Experiment (CODATA):}} & \textbf{137.035999084} & \textbf{Referenz} \\
			\bottomrule
		\end{tabular}
		\caption{Vergleich der Berechnungsgenauigkeit f\"ur verschiedene E\_0-Werte}
	\end{table}
	
	## Detaillierte Berechnung mit E\_0 = \SI{7.398{\MeV}}
	
	
```math-align

		E_0^2 &= (7.398)^2 = \SI{54.7303}{\MeV\squared} \\
		\frac{E_0^2}{( \SI{1}{\MeV} )^2} &= 54.7303 \\
		\alpha &= 1.333\overline{3} \times 10^{-4} \times 54.7303 \\
		&= 7.297 \times 10^{-3} \\
		\alpha^{-1} &= 137.038
	
```

	
	\begin{tcolorbox}[colback=green!5!white,colframe=green!75!black,title=Hervorragende \"Ubereinstimmung]
		\textbf{T0-Vorhersage:} $\alpha^{-1} = 137.038$
		
		\textbf{Experiment:} $\alpha^{-1} = 137.035999084$
		
		\textbf{Relative Abweichung:} $\frac{|137.038 - 137.036|}{137.036} = 0.0014\%$
	\end{tcolorbox}
	
	# Erkl\"arung der optimalen Pr\"azision
	
	## Warum E\_0 = \SI{7.398{\MeV} optimal funktioniert}
	
	Der Wert $E_0 = \SI{7.398}{\MeV}$ ist \textbf{nicht willk\"urlich}, sondern entsteht durch:
	
	
		- \textbf{Ber\"ucksichtigung aller QED-Korrekturen} in den Teilchenmassen
		- \textbf{Einbeziehung schwacher Wechselwirkungseffekte}
		- \textbf{Geometrische Mittelwertbildung} mit vollst\"andiger Pr\"azision
		- \textbf{Konsistenz} mit der T0-Geometrie $\xi = \frac{4}{3} \times 10^{-4}$
	
	
	## Die mathematische Begr\"undung
	
	\begin{tcolorbox}[colback=blue!10!white,colframe=blue!75!black,title=Geometrische Interpretation]
		Das geometrische Mittel $E_0 = \sqrt{m_e \cdot m_\mu}$ ist die nat\"urliche Energieskala zwischen Elektron und Myon. 
		
		Auf logarithmischer Skala liegt $E_0$ exakt in der Mitte:
		
```math-equation

			\log(E_0) = \frac{\log(m_e) + \log(m_\mu)}{2}
		
```

		
		Dies ist die \textbf{charakteristische Energie} der ersten beiden Leptonengenerationen.
	\end{tcolorbox}
	
	# Vergleich mit alternativen Ans\"atzen
	
	## Sch\"atzung mit T0-berechneten Massen
	
	Falls die Teilchenmassen selbst aus der T0-Theorie berechnet w\"urden:
	
```math-align

		m_e^{\text{T0}} &= \SI{0.511000}{\MeV} \quad \text{(theoretisch)} \\
		m_\mu^{\text{T0}} &= \SI{105.658000}{\MeV} \quad \text{(theoretisch)} \\
		E_0^{\text{T0}} &= \sqrt{0.511000 \times 105.658000} = \SI{72.868}{\MeV}
	
```

	
	\textbf{Problem:} Diese Rechnung ist offensichtlich fehlerhaft ($E_0 = \SI{72.868}{\MeV}$ ist viel zu gro\ss{}).
	
	## Korrekte Interpretation
	
	Der korrekte Ansatz ist:
	
		- \textbf{Experimentelle Massen} als Input verwenden
		- \textbf{Geometrisches Mittel} exakt berechnen  
		- \textbf{T0-Geometrie} $\xi$ als theoretischen Parameter
		- \textbf{Feinstrukturkonstante} als Output pr\"ufen
	
	
	# Dimensionale Konsistenz der E\_0-Formel
	
	## Korrekte dimensionslose Formulierung
	
	Die Formel:
	
```math-equation

		\alpha = \xi \cdot \frac{E_0^2}{( \SI{1}{\MeV} )^2}
	
```

	
	ist dimensionslos konsistent:
	
```math-align

		[\alpha] &= [\xi] \cdot \frac{[E_0^2]}{[( \SI{1}{\MeV} )^2]} \\
		&= [1] \cdot \frac{[\text{Energie}^2]}{[\text{Energie}^2]} \\
		&= [1] \quad \checkmark
	
```

	
	## Alternative Schreibweise
	
	Equivalent kann geschrieben werden:
	
```math-equation

		\frac{1}{\alpha} = \frac{( \SI{1}{\MeV} )^2}{\xi \cdot E_0^2} = \frac{1}{\xi \cdot 54.73} = \frac{1}{1.333 \times 10^{-4} \times 54.73} = 137.038
	
```

	
	# Fazit der E\_0-Klarstellung
	
	\begin{tcolorbox}[colback=red!5!white,colframe=red!75!black,title=Zusammenfassung E\_0-Analyse]
		
			- $E_0 = \SI{7.398}{\MeV}$ ist \textbf{KEIN} angepasster Parameter
			- Es ist das \textbf{exakte geometrische Mittel} verfeinerter CODATA-Massen
			- Die hervorragende \"Ubereinstimmung mit $\alpha$ best\"atigt die \textbf{T0-Geometrie}
			- Der geometrische Parameter $\xi = \frac{4}{3} \times 10^{-4}$ ist die \textbf{wahre Fundamentalkonstante}
			- Die Formel $\alpha = \xi \cdot \frac{E_0^2}{( \SI{1}{\MeV} )^2}$ ist \textbf{dimensional korrekt}
		
	\end{tcolorbox}
	
	\begin{tcolorbox}[colback=green!10!white,colframe=green!75!black,title=Die Revolution\"are E\_0-Erkenntnis]
		Die T0-Theorie zeigt: Nur \textbf{eine einzige geometrische Konstante} $\xi = \frac{4}{3} \times 10^{-4}$ gen\"ugt, um die Feinstrukturkonstante mit beispielloser Pr\"azision vorherzusagen.
		
		Dies ist kein Zufall -- es offenbart die fundamentale geometrische Struktur der Natur!
	\end{tcolorbox}
	
	## Das Kernprinzip der Verh\"altnisse
	
	\begin{tcolorbox}[colback=blue!10!white,colframe=blue!75!black,title=Fraktale Korrekturen k\"urzen sich in Verh\"altnissen]
		Die wichtigste Erkenntnis der T0-Theorie ist, dass die fraktale Korrektur $K_{\text{frak}}$ sich bei \textbf{Verh\"altnissen} vollst\"andig herausk\"urzt:
		
		
```math-equation

			\frac{m_\mu}{m_e} = \frac{K_{\text{frak}} \times m_\mu^{\text{bare}}}{K_{\text{frak}} \times m_e^{\text{bare}}} = \frac{m_\mu^{\text{bare}}}{m_e^{\text{bare}}}
		
```

		
		Das bedeutet: \textbf{Verh\"altnisse ben\"otigen keine Korrektur!}
	\end{tcolorbox}
	
	## Was KEINE Korrektur ben\"otigt
	
	\begin{table}[h]
		\centering
		\begin{tabular}{@{}lcc@{}}
			\toprule
			\textbf{Gr\"o\ss{}e} & \textbf{T0-Rohwert} & \textbf{Experiment} \\
			\midrule
			$m_\mu/m_e$ & 207.84 & 206.768 \\
			$E_0 = \sqrt{m_e \cdot m_\mu}$ & \SI{7.348}{\MeV} & \SI{7.349}{\MeV} \\
			Skalenverh\"altnisse & Direkt aus $\xi$ & Experimentell \\
			\bottomrule
		\end{tabular}
		\caption{Gr\"o\ss{}en die KEINE fraktale Korrektur ben\"otigen}
	\end{table}
	
	\textbf{Abweichung beim Massenverh\"altnis}: Nur 0.5\% ohne jede Korrektur!
	
	## Was Korrektur ben\"otigt
	
	
		- \textbf{Absolute Einzelmassen}: $m_e$, $m_\mu$ (einzeln gemessen)
		- \textbf{Feinstrukturkonstante}: $\alpha$ als absolute dimensionslose Gr\"o\ss{}e
		- \textbf{Absolute Energieskalen}: Einzelne Energiewerte
	
	
	## Die mathematische Begr\"undung
	
	Aus der T0-Theorie folgt das Massenverh\"altnis:
	
```math-align

		\frac{m_\mu}{m_e} &= \frac{8/5}{2/3} \times \xi^{-1/2} \\
		&= \frac{12}{5} \times \xi^{-1/2} \\
		&= 2.4 \times \left(\frac{4}{3} \times 10^{-4}\right)^{-1/2} \\
		&= 2.4 \times 86.6 = 207.84
	
```

	
	\textbf{Experimentell}: 206.768 \quad \textbf{Abweichung}: 0.5\%
	
	\begin{tcolorbox}[colback=green!5!white,colframe=green!75!black,title=Revolution\"are Schlussfolgerung]
		Die T0-Rohwerte liefern bereits die \textbf{echten physikalischen Verh\"altnisse}!
		
		Die Geometrie $\xi = \frac{4}{3} \times 10^{-4}$ erfasst die \textbf{wahren Proportionen} der Natur direkt - ohne Korrekturen.
		
		Nur die absolute Skalierung ben\"otigt Anpassung, nicht die fundamentalen Beziehungen.
	\end{tcolorbox}
	
	# Entkr\"aftung der Zirkularit\"ats-Einw\"ande
	
	## Die scheinbaren Zirkularit\"ats-Einw\"ande
	
	\begin{tcolorbox}[colback=red!10!white,colframe=red!75!black,title=H\"aufige Kritikpunkte]
		\textbf{Einwand 1:} Die Planck-L\"ange $\ell_P$ ist bereits \"uber die Gravitationskonstante $G$ definiert:
		
```math-equation

			\ell_P = \sqrt{\frac{\hbar G}{c^3}}
		
```

		Daher ist es zirkul\"ar, $G$ aus $\ell_P$ abzuleiten!
		
		\textbf{Einwand 2:} Die Lichtgeschwindigkeit $c$ wird aus $\mu_0$ und $\varepsilon_0$ berechnet:
		
```math-equation

			c = \frac{1}{\sqrt{\mu_0 \varepsilon_0}}
		
```

		Aber $\varepsilon_0$ wird aus $c$ berechnet - das ist zirkul\"ar!
	\end{tcolorbox}
	
	## Aufl\"osung der scheinbaren Zirkularit\"at
	
	### Die wahre Struktur der SI-Definitionen (seit 2019)
	
	\begin{tcolorbox}[colback=green!5!white,colframe=green!75!black,title=Moderne SI-Basis]
		Seit der SI-Reform 2019 sind folgende Gr\"o\ss{}en \textbf{exakt definiert}:
		
```math-align

			c &= 299792458 \text{ m/s} \quad \text{(exakte Definition)}\\
			e &= 1.602176634 \times 10^{-19} \text{ C} \quad \text{(exakte Definition)}\\
			\hbar &= 1.054571817 \times 10^{-34} \text{ J·s} \quad \text{(exakte Definition)}\\
			k_B &= 1.380649 \times 10^{-23} \text{ J/K} \quad \text{(exakte Definition)}
		
```

		
		Nur $\mu_0$ wird noch berechnet: $\mu_0 = \frac{4\pi \times 10^{-7}}{\text{definiert}}$
	\end{tcolorbox}
	
	### Korrigierte Hierarchie mit modernem SI
	
	Die tats\"achliche Ableitung ist daher:
	
	
```math-align

		\text{\textbf{Gegeben (experimentell):}} &\quad \alpha, \ell_P\\
		\text{\textbf{Definiert (SI 2019):}} &\quad c, e, \hbar, k_B\\
		\text{\textbf{Berechnet:}} &\quad \varepsilon_0 = \frac{e^2}{4\pi\hbar c \alpha}\\
		&\quad \mu_0 = \frac{1}{\varepsilon_0 c^2}\\
		&\quad G = \frac{\ell_P^2 c^3}{\hbar}
	
```

	
	\textbf{Ergebnis:} Keine Zirkularit\"at, da $c$ und $\hbar$ direkt definiert sind!
	
	### $\ell_P$ ist nur EINE m\"ogliche L\"angenskala
	
	Die Planck-L\"ange ist nicht die einzige fundamentale L\"angenskala. Man k\"onnte genausogut verwenden:
	
	
```math-align

		L_1 &= 2.5 \times 10^{-35} \text{ m} \quad \text{(willk\"urlich gew\"ahlt)}\\
		L_2 &= 1.0 \times 10^{-35} \text{ m} \quad \text{(runde Zahl)}\\
		L_3 &= \pi \times 10^{-35} \text{ m} \quad \text{(mit } \pi \text{)}\\
		L_4 &= e \times 10^{-35} \text{ m} \quad \text{(mit } e \text{)}
	
```

	
	### Die Mathematik funktioniert mit JEDER L\"angenskala
	
	Die allgemeine Formel lautet:
	
```math-equation

		G = \frac{L^2 \times c^3}{\hbar}
	
```

	
	\textbf{Entscheidend:} Nur mit der spezifischen L\"ange $\ell_P = 1.616255 \times 10^{-35}$ m erh\"alt man den korrekten experimentellen Wert von $G$.
	
	### Der SI-Bezug ist das Entscheidende
	
	\begin{table}[h]
		\centering
		\begin{tabular}{@{}lcc@{}}
			\toprule
			\textbf{L\"angenskala L} & \textbf{Berechnetes G} & \textbf{Status} \\
			\midrule
			$2.5 \times 10^{-35}$ m & $1.04 \times 10^{-10}$ m$^3$/(kg$\cdot$s$^2$) & Falsch \\
			$1.0 \times 10^{-35}$ m & $1.67 \times 10^{-11}$ m$^3$/(kg$\cdot$s$^2$) & Falsch \\
			$\pi \times 10^{-35}$ m & $1.64 \times 10^{-10}$ m$^3$/(kg$\cdot$s$^2$) & Falsch \\
			\textbf{$\ell_P = 1.616 \times 10^{-35}$ m} & \textbf{$6.674 \times 10^{-11}$ m$^3$/(kg$\cdot$s$^2$)} & \textbf{Korrekt} \\
			\bottomrule
		\end{tabular}
		\caption{G-Werte f\"ur verschiedene L\"angenskalen}
	\end{table}
	
	## Die wahre Hierarchie
	
	\begin{tcolorbox}[colback=green!5!white,colframe=green!75!black,title=Korrekte Interpretation]
		$\ell_P$ ist nicht \"uber $G$ definiert - sondern beide sind Manifestationen derselben fundamentalen Geometrie!
		
		\textbf{Die wahre Reihenfolge:}
		
			- Fundamentale 3D-Raumgeometrie $\rightarrow$ $\xi = \frac{4}{3} \times 10^{-4}$
			- Daraus folgt $\ell_P$ als nat\"urliche Skala
			- Daraus folgt $G$ als emergente Eigenschaft  
			- SI-Einheiten geben den Bezug zu menschlichen Ma\ss{}st\"aben
		
	\end{tcolorbox}
	
	## Experimentelle Best\"atigung der Nicht-Zirkularit\"at
	
	### Unabh\"angige Messung von $\ell_P$
	
	Die Planck-L\"ange kann prinzipiell unabh\"angig von $G$ gemessen werden durch:
	
	
		- \textbf{Quantengravitations-Experimente:} Direkte Messung der minimalen L\"angenskala
		- \textbf{Schwarze-Loch-Hawking-Strahlung:} $\ell_P$ bestimmt die Verdampfungsrate
		- \textbf{Kosmologische Beobachtungen:} $\ell_P$ beeinflusst Quantenfluktuationen der Inflation
		- \textbf{Hochenergie-Streuexperimente:} Bei Planck-Energien wird $\ell_P$ direkt zug\"anglich
	
	
	### Unabh\"angige Messung von $\alpha$
	
	Die Feinstrukturkonstante wird gemessen durch:
	
	
		- \textbf{Quantenhalleffekt:} $\alpha = \frac{e^2}{h} \times \frac{R_K}{Z_0}$
		- \textbf{Anomales magnetisches Moment:} $\alpha$ aus QED-Korrekturen
		- \textbf{Atominterferometrie:} $\alpha$ aus R\"ucksto\ss{}-Messungen
		- \textbf{Spektroskopie:} $\alpha$ aus Wasserstoff-Spektrum
	
	
	Keine dieser Methoden verwendet $G$ oder $\ell_P$!
	
	## Mathematischer Nachweis der Nicht-Zirkularit\"at
	
	### Definitionshierarchie
	
	
```math-align

		\text{\textbf{Gegeben:}} &\quad \alpha \text{ (experimentell)}, \quad \ell_P \text{ (experimentell)}\\
		\text{\textbf{Definiert:}} &\quad \mu_0 \text{ (SI-Konvention)}, \quad e \text{ (SI-Konvention)}\\
		\text{\textbf{Berechnet:}} &\quad c = f_1(\mu_0), \quad \varepsilon_0 = f_2(\mu_0, c)\\
		&\quad \hbar = f_3(e, \varepsilon_0, c, \alpha)\\
		&\quad G = f_4(\ell_P, c, \hbar)
	
```

	
	\textbf{Jede Gr\"o\ss{}e h\"angt nur von vorher definierten Gr\"o\ss{}en ab!}
	
	### Zirkularit\"atstest
	
	Ein zirkul\"ares Argument liegt vor, wenn:
	
```math-equation

		A \xrightarrow{\text{definiert}} B \xrightarrow{\text{definiert}} C \xrightarrow{\text{definiert}} A
	
```

	
	In unserem Fall:
	
```math-equation

		\alpha, \ell_P \xrightarrow{\text{berechnet}} \hbar \xrightarrow{\text{berechnet}} G \not\rightarrow \alpha, \ell_P
	
```

	
	\textbf{Ergebnis:} Keine Zirkularit\"at vorhanden!
	
	## Das philosophische Argument
	
	### Referenzskalen sind notwendig
	
	\begin{tcolorbox}[colback=blue!5!white,colframe=blue!75!black,title=Fundamentale Erkenntnis]
		\textbf{Jede Physik ben\"otigt Referenzskalen!}
		
		Die Natur ist dimensional strukturiert. Um von dimensionslosen Beziehungen zu messbaren Gr\"o\ss{}en zu gelangen, brauchen wir:
		
			- Eine \textbf{Energieskala} (aus $\alpha$)
			- Eine \textbf{L\"angenskala} (aus $\ell_P$) 
			- \textbf{SI-Konventionen} (menschliche Ma\ss{}st\"abe)
		
		
		Dies ist keine Schw\"ache der Theorie, sondern eine Notwendigkeit jeder dimensionalen Physik!
	\end{tcolorbox}
	
	## Zusammenfassung: Warum der Zirkularit\"ats-Einwand nicht zutrifft
	
	\begin{tcolorbox}[colback=yellow!10!white,colframe=orange!75!black,title=Endg\"ultige Widerlegung]
		\textbf{Der Zirkularit\"ats-Einwand ist unbegr\"undet, weil:}
		
		
			- $\ell_P$ ist nur eine von vielen m\"oglichen L\"angenskalen
			- Nur die spezifische Planck-L\"ange liefert den korrekten G-Wert  
			- $\ell_P$ und $G$ sind beide Manifestationen derselben Geometrie
			- $\ell_P$ dient als SI-Referenz, nicht als G-Definition
			- Ohne SI-Bezug ginge die Verbindung zu messbaren Gr\"o\ss{}en verloren
			- Alle etablierten Theorien verwenden fundamentale Skalen als Input
			- Die mathematische Hierarchie ist nicht-zirkul\"ar
		
		
		\textbf{Fazit:} $\ell_P$ ist die nat\"urliche Br\"ucke zwischen fundamentaler Geometrie und menschlichen Ma\ss{}st\"aben - keine zirkul\"are Definition!
	\end{tcolorbox}
	
	# Zusammenfassung und Ergebnisse
	
	## Die fundamentale Hierarchie
	
	\begin{table}[h]
		\centering
		\begin{tabular}{|l|l|l|}
			\hline
			\textbf{Ebene} & \textbf{Parameter} & \textbf{Status} \\
			\hline
			\textbf{1. Experimentelle Basis} & $\alpha$, $\ell_P$ & Gemessen \\
			\textbf{2. SI-Konventionen} & $\mu_0$, $e$, $k_B$, $N_A$ & Definiert \\
			\textbf{3. Abgeleitete Konstanten} & $c$, $\varepsilon_0$, $\hbar$, $G$ & Berechnet \\
			\textbf{4. Planck-Einheiten} & $t_P$, $m_P$, $E_P$, $T_P$ & Abgeleitet \\
			\textbf{5. Atomare Konstanten} & $r_e$, $\lambda_{C,e}$, $a_0$, $R_\infty$ & Abgeleitet \\
			\textbf{6. Alle anderen} & $\sigma$, $b$, etc. & Folgen automatisch \\
			\hline
		\end{tabular}
		\caption{Hierarchie der physikalischen Konstanten}
	\end{table}
	
	## Kernerkenntnisse
	
	\begin{tcolorbox}[colback=yellow!10!white,colframe=orange!75!black,title=Revolution\"are Einfachheit]
		
			- \textbf{Nur 2 experimentelle Konstanten} ($\alpha$ und $\ell_P$) gen\"ugen f\"ur die gesamte Physik
			- \textbf{Alle anderen Konstanten} sind mathematische Konsequenzen
			- \textbf{SI-Definitionen} sind menschliche Konventionen, keine Naturgesetze
			- \textbf{Die Natur ist fundamental einfach}, nicht kompliziert
			- \textbf{T0-Rohwerte} liefern bereits echte physikalische Verh\"altnisse
			- \textbf{Fraktale Korrekturen} sind nur f\"ur absolute Werte n\"otig
		
	\end{tcolorbox}
	
	## Praktische Bedeutung
	
	Diese Herleitung zeigt, dass:
	
	
		- Die Physik viel einfacher ist als traditionell dargestellt
		- Nur wenige fundamentale Prinzipien die gesamte Natur bestimmen
		- Alle anderen Konstanten emergente Eigenschaften sind
		- Eine Weltformel m\"oglicherweise nur zwei Parameter ben\"otigt
		- Die charakteristische Energie $E_0$ kein angepasster Parameter ist
		- Zirkularit\"ats-Einw\"ande wissenschaftlich haltlos sind
	
	
	# Weiterf\"uhrende \"Uberlegungen
	
	## Verbindung zum T0-Modell
	
	Im Rahmen des T0-Modells k\"onnen sogar $\alpha$ und $\ell_P$ aus noch fundamentaleren geometrischen Prinzipien abgeleitet werden:
	
	
```math-align

		\xi &= \frac{4}{3} \times 10^{-4} \quad \text{(3D-Raumgeometrie)}\\
		\alpha &= \xi \times E_0^2 \quad \text{mit } E_0 = \sqrt{m_e \times m_\mu}\\
		\ell_P &= \xi \times \ell_{fundamental}
	
```

	
	Dies w\"urde die Anzahl der fundamentalen Parameter auf nur noch \textbf{einen} reduzieren: den geometrischen Parameter $\xi$.
	
	## Ausblick
	
	Die Erkenntnis, dass alle physikalischen Konstanten aus nur zwei experimentellen Werten ableitbar sind, \"offnet neue Perspektiven f\"ur:
	
	
		- Eine vereinheitlichte Teorie aller Naturkr\"afte
		- Das Verst\"andnis der fundamentalen Einfachheit der Natur
		- Neue experimentelle Tests der Grundlagen der Physik
		- Die Suche nach der ultimativen Weltformel
	
	
	# Gesamtfazit: Vollst\"andige Integration
	
	\begin{tcolorbox}[colback=red!5!white,colframe=red!75!black,title=Vollst\"andige Zusammenfassung]
		
			- $E_0 = \SI{7.398}{\MeV}$ ist \textbf{KEIN} angepasster Parameter
			- Es ist das \textbf{exakte geometrische Mittel} verfeinerter CODATA-Massen
			- \textbf{Rohwerte ohne Korrektur} liefern bereits echte Verh\"altnisse
			- Die fraktale Korrektur k\"urzt sich in Verh\"altnissen heraus
			- Der geometrische Parameter $\xi = \frac{4}{3} \times 10^{-4}$ ist die \textbf{wahre Fundamentalkonstante}
			- Die Formel $\alpha = \xi \cdot \frac{E_0^2}{( \SI{1}{\MeV} )^2}$ ist \textbf{dimensional korrekt}
			- Alle Zirkularit\"ats-Einw\"ande sind \textbf{wissenschaftlich unbegr\"undet}
		
	\end{tcolorbox}
	
	\vspace{1cm}
	
	\begin{tcolorbox}[colback=green!10!white,colframe=green!75!black,title=Die ultimative Revolution\"are Erkenntnis]
		Die T0-Theorie zeigt: Nur \textbf{eine einzige geometrische Konstante} $\xi = \frac{4}{3} \times 10^{-4}$ gen\"ugt, um:
		
		
			- Die \textbf{wahren Proportionen} der Leptonmassen vorherzusagen
			- Die charakteristische Energie $E_0$ zu bestimmen  
			- Die Feinstrukturkonstante mit beispielloser Pr\"azision zu berechnen
			- Alle physikalischen Konstanten aus nur $\alpha$ und $\ell_P$ abzuleiten
			- Zirkularit\"ats-Einw\"ande wissenschaftlich zu entkr\"aften
		
		
		\textbf{Die Rohwerte sind bereits physikalisch korrekt} - dies offenbart die fundamentale geometrische Einfachheit der Natur!
		
		\vspace{0.5cm}
		Die ultimative Weltformel ist bereits gefunden: $T \times m = 1$.
	\end{tcolorbox}

\end{document}
