\documentclass[11pt,a4paper,openany]{book}

% Essential packages
\usepackage[utf8]{inputenc}
\usepackage[T1]{fontenc}
\usepackage[english]{babel}
\usepackage[a4paper,margin=2.5cm]{geometry}
\usepackage{lmodern}

% Math and physics packages
\usepackage{amsmath}
\usepackage{amssymb}
\usepackage{amsthm}
\usepackage{mathtools}
\usepackage{physics}
\usepackage{siunitx}

% Graphics and tables
\usepackage{graphicx}
\usepackage[table,xcdraw]{xcolor}
\usepackage{tikz}
\usepackage{pgfplots}
\usepackage{tcolorbox}
\usepackage{booktabs}
\usepackage{array}
\usepackage{longtable}
\usepackage{float}

% Document formatting
\usepackage{fancyhdr}
\usepackage{tocloft}
\usepackage{hyperref}
\usepackage{cleveref}
\usepackage{microtype}
\usepackage{enumitem}
\usepackage{newunicodechar}

% Additional packages
\usepackage{adjustbox}
\usepackage{algorithm}
\usepackage{algorithmic}
\usepackage{amsfonts}
\usepackage{amsmath,amsfonts,amssymb}
\usepackage{amsmath,amsfonts,amssymb,physics}
\usepackage{amsmath,amssymb}
\usepackage{amsmath,amssymb,amsfonts,amsthm}
\usepackage{amsmath,amssymb,amsthm}
\usepackage{amsmath,amssymb,physics,graphicx,xcolor,amsthm}
\usepackage{bm}
\usepackage{booktabs,array,longtable,multirow}
\usepackage{braket}
\usepackage{breakurl}
\usepackage{cancel}
\usepackage{caption}
\usepackage{cite}
\usepackage{color}
\usepackage{colortbl}
\usepackage{csquotes}
\usepackage{doi}
\usepackage{forest}
\usepackage{gensymb}
\usepackage{geometry,fancyhdr}
\usepackage{graphicx,tikz,pgfplots}
\usepackage{hyperref,url}
\usepackage{hyphenat}
\usepackage{listings}
\usepackage{listings,enumerate}
\usepackage{mdframed}
\usepackage{multicol}
\usepackage{multirow}
\usepackage{natbib}
\usepackage{pdflscape}
\usepackage{ragged2e}
\usepackage{setspace}
\usepackage{siunitx,xcolor,graphicx}
\usepackage{slashed}
\usepackage{tabularx}
\usepackage{textcomp}
\usepackage{textgreek}
\usepackage{tikz,pgfplots}
\usepackage{upgreek}
\usepackage{url}

% Custom commands and definitions
\definecolor{blue}
\definecolor{blue}{rgb}{0,0,1}
\definecolor{boxgray}
\definecolor{boxgray}{RGB}{240,240,240}
\definecolor{deepblue}
\definecolor{deepblue}{RGB}{0,0,127}
\definecolor{deepgreen}
\definecolor{deepgreen}{RGB}{0,127,0}
\definecolor{deepred}
\definecolor{deepred}{RGB}{191,0,0}
\definecolor{t0blue}
\definecolor{t0blue}{RGB}{0,102,204}
\definecolor{t0blue}{RGB}{33,150,243}
\definecolor{t0green}
\definecolor{t0green}{RGB}{0,153,0}
\definecolor{t0green}{RGB}{0,153,76}
\definecolor{t0green}{RGB}{76,175,80}
\definecolor{t0orange}
\definecolor{t0orange}{RGB}{255,152,0}
\definecolor{t0purple}
\definecolor{t0purple}{RGB}{102,0,204}
\definecolor{t0purple}{RGB}{156,39,176}
\definecolor{t0red}
\definecolor{t0red}{RGB}{204,0,0}
\definecolor{t0red}{RGB}{204,0,51}
\definecolor{t0red}{RGB}{244,67,54}
\definecolor{t0yellow}
\definecolor{t0yellow}{RGB}{255,204,0}
\geometry{a4paper, left=25mm, right=25mm, top=25mm, bottom=25mm}
\geometry{a4paper, margin=1in}
\geometry{a4paper, margin=2.5cm}
\geometry{a4paper, margin=2cm}
\geometry{left=2.5cm,right=2.5cm,top=2.5cm,bottom=2.5cm}
\geometry{left=2cm,right=2cm,top=2cm,bottom=2cm}
\geometry{margin=1in}
\geometry{margin=2.5cm}
\geometry{margin=2cm}
\hypersetup{
	colorlinks=true,
	linkcolor=blue,
	citecolor=blue,
	urlcolor=blue,
	pdftitle={Analysis and Implications of MNRAS Paper 544 for the T0-Theory}
\hypersetup{
	colorlinks=true,
	linkcolor=blue,
	citecolor=blue,
	urlcolor=blue,
	pdftitle={Beweis: Die Feinstrukturkonstante α = 1 in natürlichen Einheiten}
\hypersetup{
	colorlinks=true,
	linkcolor=blue,
	citecolor=blue,
	urlcolor=blue,
	pdftitle={Beweis: Die Koide-Formel enthält implizit $\xi$}
\hypersetup{
	colorlinks=true,
	linkcolor=blue,
	citecolor=blue,
	urlcolor=blue,
	pdftitle={Chinas Photonischer Quantenchip: 1000x-Speedup und T0-Integration}
\hypersetup{
	colorlinks=true,
	linkcolor=blue,
	citecolor=blue,
	urlcolor=blue,
	pdftitle={Complete Derivation of Higgs Mass and Wilson Coefficients}
\hypersetup{
	colorlinks=true,
	linkcolor=blue,
	citecolor=blue,
	urlcolor=blue,
	pdftitle={Complete Particle Spectrum: Standard Model vs T0 Theory}
\hypersetup{
	colorlinks=true,
	linkcolor=blue,
	citecolor=blue,
	urlcolor=blue,
	pdftitle={Conceptual Comparison of Unified Natural Units and Extended Standard Model}
\hypersetup{
	colorlinks=true,
	linkcolor=blue,
	citecolor=blue,
	urlcolor=blue,
	pdftitle={Connections between the Mizohata-Takeuchi Counterexample and the T0 Time-Mass Duality Theory}
\hypersetup{
	colorlinks=true,
	linkcolor=blue,
	citecolor=blue,
	urlcolor=blue,
	pdftitle={Das Relationale Zahlensystem: Primzahlen als fundamentale Verhältnisse}
\hypersetup{
	colorlinks=true,
	linkcolor=blue,
	citecolor=blue,
	urlcolor=blue,
	pdftitle={Das T0-Modell (Planck-Referenziert): Eine Neuformulierung der Physik}
\hypersetup{
	colorlinks=true,
	linkcolor=blue,
	citecolor=blue,
	urlcolor=blue,
	pdftitle={Das T0-Modell: Zeit-Energie-Dualität und geometrische Ruhemasse}
\hypersetup{
	colorlinks=true,
	linkcolor=blue,
	citecolor=blue,
	urlcolor=blue,
	pdftitle={Der Massenskalierungsexponent κ in der T0-Theorie}
\hypersetup{
	colorlinks=true,
	linkcolor=blue,
	citecolor=blue,
	urlcolor=blue,
	pdftitle={Der geometrische Formalismus der T0-Quantenmechanik und seine Anwendung auf Quantencomputer}
\hypersetup{
	colorlinks=true,
	linkcolor=blue,
	citecolor=blue,
	urlcolor=blue,
	pdftitle={Der xi Parameter und Teilchendifferenzierung in der T0-Theorie}
\hypersetup{
	colorlinks=true,
	linkcolor=blue,
	citecolor=blue,
	urlcolor=blue,
	pdftitle={Deterministic Quantum Mechanics via T0-Energy Field Formulation}
\hypersetup{
	colorlinks=true,
	linkcolor=blue,
	citecolor=blue,
	urlcolor=blue,
	pdftitle={Deterministische Quantenmechanik via T0-Energiefeld-Formulierung}
\hypersetup{
	colorlinks=true,
	linkcolor=blue,
	citecolor=blue,
	urlcolor=blue,
	pdftitle={Die Elektroneneinheitsladung in der T0-Theorie: Jenseits von Punkt-Singularitäten}
\hypersetup{
	colorlinks=true,
	linkcolor=blue,
	citecolor=blue,
	urlcolor=blue,
	pdftitle={Die Feinstrukturkonstante: Verschiedene Darstellungen und Beziehungen}
\hypersetup{
	colorlinks=true,
	linkcolor=blue,
	citecolor=blue,
	urlcolor=blue,
	pdftitle={Die Musikalische Spirale und die 137: Die mathematische Entdeckung der kosmischen Verstimmung}
\hypersetup{
	colorlinks=true,
	linkcolor=blue,
	citecolor=blue,
	urlcolor=blue,
	pdftitle={E=mc² = E=m: Die Konstanten-Illusion entlarvt}
\hypersetup{
	colorlinks=true,
	linkcolor=blue,
	citecolor=blue,
	urlcolor=blue,
	pdftitle={E=mc² = E=m: The Constants Illusion Exposed}
\hypersetup{
	colorlinks=true,
	linkcolor=blue,
	citecolor=blue,
	urlcolor=blue,
	pdftitle={Einfache Lagrange-Revolution: Von der Standardmodell-Komplexität zur T0-Eleganz}
\hypersetup{
	colorlinks=true,
	linkcolor=blue,
	citecolor=blue,
	urlcolor=blue,
	pdftitle={Einführung in die Umsetzung photonischer Bauteile auf Wafern für Nachrichtentechniker}
\hypersetup{
	colorlinks=true,
	linkcolor=blue,
	citecolor=blue,
	urlcolor=blue,
	pdftitle={Einführung in photonische Quantenchips für Nachrichtentechniker}
\hypersetup{
	colorlinks=true,
	linkcolor=blue,
	citecolor=blue,
	urlcolor=blue,
	pdftitle={Elimination der Masse als dimensionaler Platzhalter im T0-Modell}
\hypersetup{
	colorlinks=true,
	linkcolor=blue,
	citecolor=blue,
	urlcolor=blue,
	pdftitle={Elimination of Mass as Dimensional Placeholder in the T0 Model}
\hypersetup{
	colorlinks=true,
	linkcolor=blue,
	citecolor=blue,
	urlcolor=blue,
	pdftitle={Empirical Analysis of Deterministic Factorization Methods}
\hypersetup{
	colorlinks=true,
	linkcolor=blue,
	citecolor=blue,
	urlcolor=blue,
	pdftitle={Empirische Analyse deterministischer Faktorisierungsmethoden}
\hypersetup{
	colorlinks=true,
	linkcolor=blue,
	citecolor=blue,
	urlcolor=blue,
	pdftitle={Integration der Dirac-Gleichung im T0-Modell: Natürliche-Einheiten-Rahmenwerk}
\hypersetup{
	colorlinks=true,
	linkcolor=blue,
	citecolor=blue,
	urlcolor=blue,
	pdftitle={Integration of the Dirac Equation in the T0 Model: Natural Units Framework}
\hypersetup{
	colorlinks=true,
	linkcolor=blue,
	citecolor=blue,
	urlcolor=blue,
	pdftitle={Introduction to Photonic Quantum Chips for Communication Engineers}
\hypersetup{
	colorlinks=true,
	linkcolor=blue,
	citecolor=blue,
	urlcolor=blue,
	pdftitle={Introduction to the Implementation of Photonic Components on Wafers for Communication Engineers}
\hypersetup{
	colorlinks=true,
	linkcolor=blue,
	citecolor=blue,
	urlcolor=blue,
	pdftitle={Konzeptioneller Vergleich von Einheitlichen Natürlichen Einheiten und Erweitertem Standardmodell}
\hypersetup{
	colorlinks=true,
	linkcolor=blue,
	citecolor=blue,
	urlcolor=blue,
	pdftitle={Markov Chains in the Context of T0 Theory: Deterministic or Stochastic? A Treatise on Patterns, Preconditions, and Uncertainty}
\hypersetup{
	colorlinks=true,
	linkcolor=blue,
	citecolor=blue,
	urlcolor=blue,
	pdftitle={Markov-Ketten im Kontext der T0-Theorie: Deterministisch oder stochastisch? Ein Traktat zu Mustern, Voraussetzungen und Unsicherheit}
\hypersetup{
	colorlinks=true,
	linkcolor=blue,
	citecolor=blue,
	urlcolor=blue,
	pdftitle={Mathematical Analysis of T0-Shor Algorithm: Theoretical Framework and Computational Complexity}
\hypersetup{
	colorlinks=true,
	linkcolor=blue,
	citecolor=blue,
	urlcolor=blue,
	pdftitle={Mathematical Constructs of Alternative CMB Models: Unnikrishnan and Peratt in Harmony with the T0 Theory}
\hypersetup{
	colorlinks=true,
	linkcolor=blue,
	citecolor=blue,
	urlcolor=blue,
	pdftitle={Mathematische Analyse des T0-Shor Algorithmus: Theoretischer Rahmen und Berechnungskomplexität}
\hypersetup{
	colorlinks=true,
	linkcolor=blue,
	citecolor=blue,
	urlcolor=blue,
	pdftitle={Mathematische Konstrukte alternativer CMB-Modelle: Unnikrishnan und Peratt im Einklang mit der T0-Theorie}
\hypersetup{
	colorlinks=true,
	linkcolor=blue,
	citecolor=blue,
	urlcolor=blue,
	pdftitle={Natural Unit Systems: Universal Energy Conversion and Fundamental Length Scale Hierarchy}
\hypersetup{
	colorlinks=true,
	linkcolor=blue,
	citecolor=blue,
	urlcolor=blue,
	pdftitle={Natural Units in Theoretical Physics: A Treatise in the Context of T0 Theory}
\hypersetup{
	colorlinks=true,
	linkcolor=blue,
	citecolor=blue,
	urlcolor=blue,
	pdftitle={Natürliche Einheiten in der theoretischen Physik: Eine Abhandlung im Kontext der T0-Theorie}
\hypersetup{
	colorlinks=true,
	linkcolor=blue,
	citecolor=blue,
	urlcolor=blue,
	pdftitle={Natürliche Einheitensysteme: Universelle Energieumwandlung und fundamentale Längenskala-Hierarchie}
\hypersetup{
	colorlinks=true,
	linkcolor=blue,
	citecolor=blue,
	urlcolor=blue,
	pdftitle={Parameter System-Dependency in T0-Model: SI vs. Natural Units}
\hypersetup{
	colorlinks=true,
	linkcolor=blue,
	citecolor=blue,
	urlcolor=blue,
	pdftitle={Parameter-Systemabhängigkeit im T0-Modell: SI- vs. natürliche Einheiten}
\hypersetup{
	colorlinks=true,
	linkcolor=blue,
	citecolor=blue,
	urlcolor=blue,
	pdftitle={Proof: The Fine Structure Constant α = 1 in Natural Units}
\hypersetup{
	colorlinks=true,
	linkcolor=blue,
	citecolor=blue,
	urlcolor=blue,
	pdftitle={Proof: The Koide Formula Implicitly Contains $\xi$}
\hypersetup{
	colorlinks=true,
	linkcolor=blue,
	citecolor=blue,
	urlcolor=blue,
	pdftitle={Pure Energy T0 Theory: Ratio-Based Physics with SI Reference}
\hypersetup{
	colorlinks=true,
	linkcolor=blue,
	citecolor=blue,
	urlcolor=blue,
	pdftitle={Quantum Mechanics in the T0 Model: Field-Theoretic Foundations}
\hypersetup{
	colorlinks=true,
	linkcolor=blue,
	citecolor=blue,
	urlcolor=blue,
	pdftitle={Ratio-Based vs. Absolute: The Role of Fractal Correction in T0 Theory}
\hypersetup{
	colorlinks=true,
	linkcolor=blue,
	citecolor=blue,
	urlcolor=blue,
	pdftitle={Reine Energie T0-Theorie: Verhältnis-basierte Physik mit SI-Referenz}
\hypersetup{
	colorlinks=true,
	linkcolor=blue,
	citecolor=blue,
	urlcolor=blue,
	pdftitle={Simple Lagrangian Revolution: From Standard Model Complexity to T0 Elegance}
\hypersetup{
	colorlinks=true,
	linkcolor=blue,
	citecolor=blue,
	urlcolor=blue,
	pdftitle={Simplified Dirac Equation in T0 Theory: Field Node Approach}
\hypersetup{
	colorlinks=true,
	linkcolor=blue,
	citecolor=blue,
	urlcolor=blue,
	pdftitle={Simplified T0 Theory: Elegant Lagrangian Density for Time-Mass Duality}
\hypersetup{
	colorlinks=true,
	linkcolor=blue,
	citecolor=blue,
	urlcolor=blue,
	pdftitle={T0 Cosmology: Redshift as a Geometric Path Effect in a Static Universe}
\hypersetup{
	colorlinks=true,
	linkcolor=blue,
	citecolor=blue,
	urlcolor=blue,
	pdftitle={T0 Deterministic Quantum Computing: Complete Analysis of Important Algorithms}
\hypersetup{
	colorlinks=true,
	linkcolor=blue,
	citecolor=blue,
	urlcolor=blue,
	pdftitle={T0 Deterministisches Quantencomputing: Vollständige Analyse wichtiger Algorithmen}
\hypersetup{
	colorlinks=true,
	linkcolor=blue,
	citecolor=blue,
	urlcolor=blue,
	pdftitle={T0 Model: Complete Framework - From Time-Energy Duality to Universal Constants}
\hypersetup{
	colorlinks=true,
	linkcolor=blue,
	citecolor=blue,
	urlcolor=blue,
	pdftitle={T0 Model: Complete Parameter-Free Particle Mass Calculation}
\hypersetup{
	colorlinks=true,
	linkcolor=blue,
	citecolor=blue,
	urlcolor=blue,
	pdftitle={T0 Model: Unified Neutrino Formula Structure}
\hypersetup{
	colorlinks=true,
	linkcolor=blue,
	citecolor=blue,
	urlcolor=blue,
	pdftitle={T0 Model: Universal Energy Relations for Mol and Candela Units}
\hypersetup{
	colorlinks=true,
	linkcolor=blue,
	citecolor=blue,
	urlcolor=blue,
	pdftitle={T0 Modell: Vollständiges Framework - Von Zeit-Energie-Dualität zu universellen Konstanten}
\hypersetup{
	colorlinks=true,
	linkcolor=blue,
	citecolor=blue,
	urlcolor=blue,
	pdftitle={T0 Quantenfeldtheorie: QFT, QM und Quantencomputer}
\hypersetup{
	colorlinks=true,
	linkcolor=blue,
	citecolor=blue,
	urlcolor=blue,
	pdftitle={T0 Quantum Field Theory: QFT, QM and Quantum Computers}
\hypersetup{
	colorlinks=true,
	linkcolor=blue,
	citecolor=blue,
	urlcolor=blue,
	pdftitle={T0 Theory vs Bell's Theorem: How Deterministic Energy Fields Circumvent No-Go Theorems}
\hypersetup{
	colorlinks=true,
	linkcolor=blue,
	citecolor=blue,
	urlcolor=blue,
	pdftitle={T0 Theory: Final Extension to Hadrons - Physically Derived Corrections}
\hypersetup{
	colorlinks=true,
	linkcolor=blue,
	citecolor=blue,
	urlcolor=blue,
	pdftitle={T0 Theory: The Fine-Structure Constant}
\hypersetup{
	colorlinks=true,
	linkcolor=blue,
	citecolor=blue,
	urlcolor=blue,
	pdftitle={T0 Theory: The Gravitational Constant}
\hypersetup{
	colorlinks=true,
	linkcolor=blue,
	citecolor=blue,
	urlcolor=blue,
	pdftitle={T0-Kosmologie: Rotverschiebung als geometrischer Pfad-Effekt im statischen Universum}
\hypersetup{
	colorlinks=true,
	linkcolor=blue,
	citecolor=blue,
	urlcolor=blue,
	pdftitle={T0-Model: Complete Document Analysis and Structured Summary}
\hypersetup{
	colorlinks=true,
	linkcolor=blue,
	citecolor=blue,
	urlcolor=blue,
	pdftitle={T0-Model: Kinetic Energy of Electrons and Photons}
\hypersetup{
	colorlinks=true,
	linkcolor=blue,
	citecolor=blue,
	urlcolor=blue,
	pdftitle={T0-Model: The Hubble Parameter in Static Universe}
\hypersetup{
	colorlinks=true,
	linkcolor=blue,
	citecolor=blue,
	urlcolor=blue,
	pdftitle={T0-Modell-Verifikation: Skalen-Verhältnis-basierte Berechnungen}
\hypersetup{
	colorlinks=true,
	linkcolor=blue,
	citecolor=blue,
	urlcolor=blue,
	pdftitle={T0-Modell: Bewegungsenergie von Elektronen und Photonen}
\hypersetup{
	colorlinks=true,
	linkcolor=blue,
	citecolor=blue,
	urlcolor=blue,
	pdftitle={T0-Modell: Die Hubble-Konstante im statischen Universum}
\hypersetup{
	colorlinks=true,
	linkcolor=blue,
	citecolor=blue,
	urlcolor=blue,
	pdftitle={T0-Modell: Einheitliche Neutrino-Formel-Struktur}
\hypersetup{
	colorlinks=true,
	linkcolor=blue,
	citecolor=blue,
	urlcolor=blue,
	pdftitle={T0-Modell: Universelle Energiebeziehungen für Mol- und Candela-Einheiten}
\hypersetup{
	colorlinks=true,
	linkcolor=blue,
	citecolor=blue,
	urlcolor=blue,
	pdftitle={T0-Modell: Vollständige Dokumentenanalyse und strukturierte Zusammenfassung}
\hypersetup{
	colorlinks=true,
	linkcolor=blue,
	citecolor=blue,
	urlcolor=blue,
	pdftitle={T0-Modell: Vollständige parameterfreie Teilchenmassen-Berechnung}
\hypersetup{
	colorlinks=true,
	linkcolor=blue,
	citecolor=blue,
	urlcolor=blue,
	pdftitle={T0-QAT: $\xi$-Aware Quantization-Aware Training}
\hypersetup{
	colorlinks=true,
	linkcolor=blue,
	citecolor=blue,
	urlcolor=blue,
	pdftitle={T0-QFT ML Addendum: Machine Learning Derived Extensions}
\hypersetup{
	colorlinks=true,
	linkcolor=blue,
	citecolor=blue,
	urlcolor=blue,
	pdftitle={T0-QFT ML-Addendum: Maschinelle Lern-abgeleitete Erweiterungen}
\hypersetup{
	colorlinks=true,
	linkcolor=blue,
	citecolor=blue,
	urlcolor=blue,
	pdftitle={T0-Theorie vs Bells Theorem: Wie deterministische Energiefelder No-Go-Theoreme umgehen}
\hypersetup{
	colorlinks=true,
	linkcolor=blue,
	citecolor=blue,
	urlcolor=blue,
	pdftitle={T0-Theorie: Der Terrell-Penrose-Effekt und Massenvariation}
\hypersetup{
	colorlinks=true,
	linkcolor=blue,
	citecolor=blue,
	urlcolor=blue,
	pdftitle={T0-Theorie: Die Feinstrukturkonstante}
\hypersetup{
	colorlinks=true,
	linkcolor=blue,
	citecolor=blue,
	urlcolor=blue,
	pdftitle={T0-Theorie: Die Gravitationskonstante}
\hypersetup{
	colorlinks=true,
	linkcolor=blue,
	citecolor=blue,
	urlcolor=blue,
	pdftitle={T0-Theorie: Die T0-Zeit-Masse-Dualität}
\hypersetup{
	colorlinks=true,
	linkcolor=blue,
	citecolor=blue,
	urlcolor=blue,
	pdftitle={T0-Theorie: Die sieben Rätsel}
\hypersetup{
	colorlinks=true,
	linkcolor=blue,
	citecolor=blue,
	urlcolor=blue,
	pdftitle={T0-Theorie: Erweiterung auf Bell-Tests – ML-Simulationen (November 2025)}
\hypersetup{
	colorlinks=true,
	linkcolor=blue,
	citecolor=blue,
	urlcolor=blue,
	pdftitle={T0-Theorie: Finale Erweiterung auf Hadronen - Physikalisch abgeleitete Korrekturen}
\hypersetup{
	colorlinks=true,
	linkcolor=blue,
	citecolor=blue,
	urlcolor=blue,
	pdftitle={T0-Theorie: Finale Fraktale Massenformeln (November 2025)}
\hypersetup{
	colorlinks=true,
	linkcolor=blue,
	citecolor=blue,
	urlcolor=blue,
	pdftitle={T0-Theorie: Fraktaldimension aus Lepton-Massenverhältnis}
\hypersetup{
	colorlinks=true,
	linkcolor=blue,
	citecolor=blue,
	urlcolor=blue,
	pdftitle={T0-Theorie: Fundamentale Prinzipien}
\hypersetup{
	colorlinks=true,
	linkcolor=blue,
	citecolor=blue,
	urlcolor=blue,
	pdftitle={T0-Theorie: Herleitung der Gravitationskonstanten}
\hypersetup{
	colorlinks=true,
	linkcolor=blue,
	citecolor=blue,
	urlcolor=blue,
	pdftitle={T0-Theorie: Kosmische Beziehungen und universelle $\xi$-Konstante}
\hypersetup{
	colorlinks=true,
	linkcolor=blue,
	citecolor=blue,
	urlcolor=blue,
	pdftitle={T0-Theorie: Kosmologie}
\hypersetup{
	colorlinks=true,
	linkcolor=blue,
	citecolor=blue,
	urlcolor=blue,
	pdftitle={T0-Theorie: Netzwerkdarstellung und Dimensionsanalyse in der T0-Theorie}
\hypersetup{
	colorlinks=true,
	linkcolor=blue,
	citecolor=blue,
	urlcolor=blue,
	pdftitle={T0-Theorie: Teilchenmassen}
\hypersetup{
	colorlinks=true,
	linkcolor=blue,
	citecolor=blue,
	urlcolor=blue,
	pdftitle={T0-Theorie: Vollstaendiger Abschluss}
\hypersetup{
	colorlinks=true,
	linkcolor=blue,
	citecolor=blue,
	urlcolor=blue,
	pdftitle={T0-Theory: Complete Closure}
\hypersetup{
	colorlinks=true,
	linkcolor=blue,
	citecolor=blue,
	urlcolor=blue,
	pdftitle={T0-Theory: Complete Derivation of All Parameters Without Circularity}
\hypersetup{
	colorlinks=true,
	linkcolor=blue,
	citecolor=blue,
	urlcolor=blue,
	pdftitle={T0-Theory: Cosmic Relations and universal $\xi$-constant}
\hypersetup{
	colorlinks=true,
	linkcolor=blue,
	citecolor=blue,
	urlcolor=blue,
	pdftitle={T0-Theory: Cosmology}
\hypersetup{
	colorlinks=true,
	linkcolor=blue,
	citecolor=blue,
	urlcolor=blue,
	pdftitle={T0-Theory: Derivation of the Gravitational Constant}
\hypersetup{
	colorlinks=true,
	linkcolor=blue,
	citecolor=blue,
	urlcolor=blue,
	pdftitle={T0-Theory: Extension to Bell Tests – ML Simulations (November 2025)}
\hypersetup{
	colorlinks=true,
	linkcolor=blue,
	citecolor=blue,
	urlcolor=blue,
	pdftitle={T0-Theory: Final Fractal Mass Formulas (November 2025)}
\hypersetup{
	colorlinks=true,
	linkcolor=blue,
	citecolor=blue,
	urlcolor=blue,
	pdftitle={T0-Theory: Fractal Dimension from Lepton Mass Ratio}
\hypersetup{
	colorlinks=true,
	linkcolor=blue,
	citecolor=blue,
	urlcolor=blue,
	pdftitle={T0-Theory: Fundamental Principles}
\hypersetup{
	colorlinks=true,
	linkcolor=blue,
	citecolor=blue,
	urlcolor=blue,
	pdftitle={T0-Theory: Mass Variation as an Equivalent to Time Dilation}
\hypersetup{
	colorlinks=true,
	linkcolor=blue,
	citecolor=blue,
	urlcolor=blue,
	pdftitle={T0-Theory: Network Representation and Dimensional Analysis in the T0-Theory}
\hypersetup{
	colorlinks=true,
	linkcolor=blue,
	citecolor=blue,
	urlcolor=blue,
	pdftitle={T0-Theory: Neutrinos}
\hypersetup{
	colorlinks=true,
	linkcolor=blue,
	citecolor=blue,
	urlcolor=blue,
	pdftitle={T0-Theory: Particle Masses}
\hypersetup{
	colorlinks=true,
	linkcolor=blue,
	citecolor=blue,
	urlcolor=blue,
	pdftitle={T0-Theory: The Seven Riddles}
\hypersetup{
	colorlinks=true,
	linkcolor=blue,
	citecolor=blue,
	urlcolor=blue,
	pdftitle={T0-Theory: The T0-Time-Mass Duality}
\hypersetup{
	colorlinks=true,
	linkcolor=blue,
	citecolor=blue,
	urlcolor=blue,
	pdftitle={Temperature Units in Natural Units: T0-Theory}
\hypersetup{
	colorlinks=true,
	linkcolor=blue,
	citecolor=blue,
	urlcolor=blue,
	pdftitle={Temperatureinheiten in nat\"urlichen Einheiten: T0-Theorie}
\hypersetup{
	colorlinks=true,
	linkcolor=blue,
	citecolor=blue,
	urlcolor=blue,
	pdftitle={The Electron Unit Charge in T0 Theory: Beyond Point Singularities}
\hypersetup{
	colorlinks=true,
	linkcolor=blue,
	citecolor=blue,
	urlcolor=blue,
	pdftitle={The Fine Structure Constant: Various Representations and Relationships}
\hypersetup{
	colorlinks=true,
	linkcolor=blue,
	citecolor=blue,
	urlcolor=blue,
	pdftitle={The Geometric Formalism of T0 Quantum Mechanics and its Application to Quantum Computing}
\hypersetup{
	colorlinks=true,
	linkcolor=blue,
	citecolor=blue,
	urlcolor=blue,
	pdftitle={The Mass Scaling Exponent κ in T0 Theory}
\hypersetup{
	colorlinks=true,
	linkcolor=blue,
	citecolor=blue,
	urlcolor=blue,
	pdftitle={The Musical Spiral and 137: The Mathematical Discovery of Cosmic Detuning}
\hypersetup{
	colorlinks=true,
	linkcolor=blue,
	citecolor=blue,
	urlcolor=blue,
	pdftitle={The Relational Number System: Prime Numbers as Fundamental Ratios}
\hypersetup{
	colorlinks=true,
	linkcolor=blue,
	citecolor=blue,
	urlcolor=blue,
	pdftitle={The T0 Model (Planck-Referenced): A Reformulation of Physics}
\hypersetup{
	colorlinks=true,
	linkcolor=blue,
	citecolor=blue,
	urlcolor=blue,
	pdftitle={The T0 Model: Time-Energy Duality and Geometric Rest Mass}
\hypersetup{
	colorlinks=true,
	linkcolor=blue,
	citecolor=blue,
	urlcolor=blue,
	pdftitle={The T0-Model (Planck-Referenced): A Reformulation of Physics}
\hypersetup{
	colorlinks=true,
	linkcolor=blue,
	citecolor=blue,
	urlcolor=blue,
	pdftitle={Verbindungen zwischen dem Mizohata-Takeuchi-Gegenbeispiel und der T0-Zeit-Masse-Dualitätstheorie}
\hypersetup{
	colorlinks=true,
	linkcolor=blue,
	citecolor=blue,
	urlcolor=blue,
	pdftitle={Vereinfachte Dirac-Gleichung in der T0-Theorie: Feldknoten-Ansatz}
\hypersetup{
	colorlinks=true,
	linkcolor=blue,
	citecolor=blue,
	urlcolor=blue,
	pdftitle={Vereinfachte T0-Theorie: Elegante Lagrange-Dichte für Zeit-Masse-Dualität}
\hypersetup{
	colorlinks=true,
	linkcolor=blue,
	citecolor=blue,
	urlcolor=blue,
	pdftitle={Verhältnisbasiert vs. Absolut: Die Rolle der fraktalen Korrektur in der T0-Theorie}
\hypersetup{
	colorlinks=true,
	linkcolor=blue,
	citecolor=blue,
	urlcolor=blue,
	pdftitle={Vollständige Herleitung der Higgs-Masse und Wilson-Koeffizienten}
\hypersetup{
	colorlinks=true,
	linkcolor=blue,
	citecolor=blue,
	urlcolor=blue,
	pdftitle={Vollständiges Teilchenspektrum: Standard-Modell vs T0-Theorie}
\hypersetup{
	colorlinks=true,
	linkcolor=blue,
	citecolor=blue,
	urlcolor=blue,
	pdftitle={Warum Zahlenverhältnisse nicht direkt gekürzt werden dürfen}
\hypersetup{
	colorlinks=true,
	linkcolor=blue,
	citecolor=blue,
	urlcolor=blue,
	pdftitle={Why Numerical Ratios Must Not Be Directly Simplified}
\hypersetup{
	colorlinks=true,
	linkcolor=blue,
	citecolor=blue,
	urlcolor=blue,
}
\hypersetup{
	colorlinks=true,
	linkcolor=blue,
	citecolor=red,
	urlcolor=blue,
	bookmarks=true,
	bookmarksnumbered=true,
	pdfstartview=FitH,
	pdftitle={T0 Model - Field-Theoretic Derivation of the Beta Parameter}
\hypersetup{
	colorlinks=true,
	linkcolor=blue,
	citecolor=red,
	urlcolor=blue,
	bookmarks=true,
	bookmarksnumbered=true,
	pdfstartview=FitH,
	pdftitle={T0-Modell - Feldtheoretische Herleitung des Beta-Parameters}
\hypersetup{
	colorlinks=true,
	linkcolor=blue,
	filecolor=magenta,
	urlcolor=cyan,
}
\hypersetup{
	colorlinks=true,
	linkcolor=blue,
	urlcolor=blue,
	citecolor=blue,
	pdftitle={From Time Dilation to Mass Variation: Mathematical Core Formulations of Time-Mass Duality Theory - Updated Framework}
\hypersetup{
	colorlinks=true,
	linkcolor=blue,
	urlcolor=blue,
	citecolor=blue,
	pdftitle={T0 Model: Detailed Formula for Leptonic Anomalies}
\hypersetup{
	colorlinks=true,
	linkcolor=blue,
	urlcolor=blue,
	citecolor=blue,
	pdftitle={T0 Model: Detaillierte Formel für leptonische Anomalien}
\hypersetup{
	colorlinks=true,
	linkcolor=blue,
	urlcolor=blue,
	citecolor=blue,
	pdftitle={T0 Model: Energy-based Formulas with Quadratic Scaling}
\hypersetup{
	colorlinks=true,
	linkcolor=blue,
	urlcolor=blue,
	citecolor=blue,
	pdftitle={T0 Model: Granulation, Limits and Fundamental Asymmetry}
\hypersetup{
	colorlinks=true,
	linkcolor=blue,
	urlcolor=blue,
	citecolor=blue,
	pdftitle={T0-Modell: Energiebasierte Formeln mit quadratischer Skalierung}
\hypersetup{
	colorlinks=true,
	linkcolor=blue,
	urlcolor=blue,
	citecolor=blue,
	pdftitle={T0-Modell: Granulation, Limits und fundamentale Asymmetrie}
\hypersetup{
	colorlinks=true,
	linkcolor=blue,
	urlcolor=blue,
	citecolor=blue,
	pdftitle={Von Zeitdilatation zu Massenvariation: Mathematische Kernformulierungen der Zeit-Masse-Dualitätstheorie - Aktualisiertes Framework}
\hypersetup{
	colorlinks=true,
	linkcolor=t0blue,
	citecolor=t0blue,
	urlcolor=t0blue,
	pdftitle={T0 Model: Complete Theoretical Summary}
\hypersetup{
	colorlinks=true,
	linkcolor=t0blue,
	citecolor=t0blue,
	urlcolor=t0blue,
	pdftitle={T0 Theory: Resolution of Apparent Instantaneity}
\hypersetup{
	colorlinks=true,
	linkcolor=t0blue,
	citecolor=t0blue,
	urlcolor=t0blue,
	pdftitle={T0 vs Synergetics: Vereinfachung durch natürliche Einheiten}
\hypersetup{
	colorlinks=true,
	linkcolor=t0blue,
	citecolor=t0blue,
	urlcolor=t0blue,
	pdftitle={T0-Modell: Vollständige theoretische Zusammenfassung}
\hypersetup{
	colorlinks=true,
	linkcolor=t0blue,
	citecolor=t0blue,
	urlcolor=t0blue,
	pdftitle={T0-Theorie: Auflösung der scheinbaren Instantanität}
\hypersetup{
	colorlinks=true,
	linkcolor=t0blue,
	citecolor=t0blue,
	urlcolor=t0blue,
	pdftitle={T0-Theorie: Vollständige Dokumentenübersicht}
\hypersetup{
	colorlinks=true,
	linkcolor=t0blue,
	citecolor=t0blue,
	urlcolor=t0blue,
	pdftitle={T0-Theory: Complete Document Overview}
\hypersetup{
	colorlinks=true,
	linkcolor=t0blue,
	citecolor=t0blue,
	urlcolor=t0blue,
}
\hypersetup{
	colorlinks=true,
	linkcolor=t0blue,
	citecolor=t0green,
	urlcolor=t0blue,
	pdftitle={Das verborgene Geheimnis von 1/137}
\hypersetup{
	colorlinks=true,
	linkcolor=t0blue,
	citecolor=t0green,
	urlcolor=t0blue,
	pdftitle={The Hidden Secret of 1/137}
\hypersetup{
    colorlinks=true,
    linkcolor=blue,
    citecolor=blue,
    urlcolor=blue,
    pdftitle={Analyse und Implikationen des MNRAS-Papiers 544 für die T0-Theorie}
\hypersetup{
  colorlinks=true,
  linkcolor=blue,
  citecolor=blue,
  urlcolor=blue
}
\hypersetup{
  colorlinks=true,
  linkcolor=blue,
  citecolor=blue,
  urlcolor=blue,
  pdftitle={T0-Theorie: Ein-Uhr-Metrologie und Drei-Uhren-Experiment}
\hypersetup{
  colorlinks=true,
  linkcolor=blue,
  citecolor=blue,
  urlcolor=blue,
  pdftitle={T0-Theory: Single-Clock Metrology and Three-Clock Experiment}
\hypersetup{
colorlinks=true,
linkcolor=blue,
citecolor=blue,
urlcolor=blue,
pdftitle={Quantenmechanik im T0-Modell: Feldtheoretische Grundlagen}
\hypersetup{
colorlinks=true,
linkcolor=blue,
citecolor=blue,
urlcolor=blue,
pdftitle={T0-Theory: Neutrinos}
\newcommand{\Bzero}{B_0}
\newcommand{\CQCD}{C_{\text{QCD}
\newcommand{\Cconv}{C_{\text{conv}
\newcommand{\Cto}{C_{\text{T0}
\newcommand{\Czero}{C_0}
\newcommand{\DTmu}{D_{T,\mu}
\newcommand{\DcovT}[1]{\partial_\mu #1 + #1 \partial_\mu \Tfield}
\newcommand{\Dfrak}{D_f}
\newcommand{\Df}{D_f}
\newcommand{\DhiggsT}{\Tfield (\partial_\mu + ig A_\mu) \Phi + \Phi \partial_\mu \Tfield}
\newcommand{\EPlanck}{E_P}
\newcommand{\EPlanck}{E_{\text{Pl}
\newcommand{\EPratio}[1]{\frac{#1}
\newcommand{\EP}{E_P}
\newcommand{\EP}{E_{\text{P}
\newcommand{\EW}{E_W}
\newcommand{\EZ}{E_Z}
\newcommand{\Echar}{E_{\text{char}
\newcommand{\Ee}{E_e}
\newcommand{\Efield}{E(x,t)}
\newcommand{\Efield}{E_\text{field}
\newcommand{\Efield}{E_{\text{Feld}
\newcommand{\Efield}{E_{\text{Field}
\newcommand{\Efield}{E_{\text{field}
\newcommand{\Efield}{E}
\newcommand{\Egamma}{E_\gamma}
\newcommand{\Eh}{E_h}
\newcommand{\Emu}{E_\mu}
\newcommand{\Enorm}[1]{E_{\text{norm}
\newcommand{\En}{E_n}
\newcommand{\Ep}{E_p}
\newcommand{\Eratio}[2]{\frac{E_{#1}
\newcommand{\Etau}{E_\tau}
\newcommand{\Evis}{E_{\text{vis}
\newcommand{\Exi}{E_\xi}
\newcommand{\Ezero}{E_0}
\newcommand{\GeV}{\,\text{GeV}
\newcommand{\Gnat}{G_{\text{nat}
\newcommand{\Gsi}{G_{\text{SI}
\newcommand{\Hubble}{H_0}
\newcommand{\Kfrak}{K_{\text{frac}
\newcommand{\Kfrak}{K_{\text{frak}
\newcommand{\Kspec}{K_{\text{spec}
\newcommand{\LCDM}{\Lambda\text{CDM}
\newcommand{\LPlanck}{\ell_{\text{Pl}
\newcommand{\Lag}{\mathcal{L}
\newcommand{\Lambdat}{\Lambda_T}
\newcommand{\Leff}{L_{\text{eff}
\newcommand{\Lorentz}[2]{{\Lambda^\mu{}
\newcommand{\Lp}{L_{\text{P}
\newcommand{\Lxi}{L_\xi}
\newcommand{\Lzero}{L_0}
\newcommand{\MPl}{M_{\text{Pl}
\newcommand{\MSbar}{\overline{\text{MS}
\newcommand{\MeV}{\,\text{MeV}
\newcommand{\Mpl}{M_{\text{Pl}
\newcommand{\OmegaDM}{\Omega_{\text{DM}
\newcommand{\OmegaLambda}{\Omega_{\Lambda}
\newcommand{\Omegab}{\Omega_b}
\newcommand{\Phiphoton}{\Phi_{\text{photon}
\newcommand{\Ricci}{R_{\mu\nu}
\newcommand{\Riem}{R^\rho{}
\newcommand{\Rzero}{R_\infty}
\newcommand{\Scal}{R}
\newcommand{\SynchPower}{P_{\text{synch}
\newcommand{\TPlanck}{t_{\text{Pl}
\newcommand{\Tfieldt}{T(\vec{x}
\newcommand{\Tfieldt}{T(x,t)}
\newcommand{\Tfield}{T(x)}
\newcommand{\Tfield}{T(x,t)}
\newcommand{\Tfield}{T_{\text{field}
\newcommand{\Tfield}{T}
\newcommand{\Tfield}{\mathcal{T}
\newcommand{\Tzerot}{T_0(\Tfield)}
\newcommand{\Tzero}{T_0}
\newcommand{\Weyl}{C^\rho{}
\newcommand{\ZPinch}{J \times B = \nabla p}
\newcommand{\aleph}{\aleph}
\newcommand{\alphaEMSI}{\alpha_{\text{EM,SI}
\newcommand{\alphaEMnat}{\alpha_{\text{EM,nat}
\newcommand{\alphaEM}{\alpha_{\text{EM}
\newcommand{\alphaEM}{\ensuremath{\alpha_{\text{EM}
\newcommand{\alphaQCD}{\alpha_s}
\newcommand{\alphaQED}{\alpha_{\text{QED}
\newcommand{\alphaSI}{\alpha_{\text{SI}
\newcommand{\alphaT}{\alpha_{\text{T}
\newcommand{\alphaWSI}{\alpha_{\text{W,SI}
\newcommand{\alphaWnat}{\alpha_{\text{W,nat}
\newcommand{\alphaW}{\alpha_{\text{W}
\newcommand{\alphaem}{\alpha_{EM}
\newcommand{\alphaem}{\alpha}
\newcommand{\alphafine}{\alpha}
\newcommand{\alphagem}{\alpha}
\newcommand{\alphanat}{\alpha_{\text{nat}
\newcommand{\alphapar}{\alpha}
\newcommand{\betaTSI}{\beta_{\text{T,SI}
\newcommand{\betaTnat}{\beta_{\text{T,nat}
\newcommand{\betaT}{\beta_T}
\newcommand{\betaT}{\beta_{T}
\newcommand{\betaT}{\beta_{\text{T}
\newcommand{\betaT}{\ensuremath{\beta_T}
\newcommand{\betapar}{\beta}
\newcommand{\calL}{\mathcal{L}
\newcommand{\checked}{\checkmark}
\newcommand{\checkmarkx}{\checkmark}
\newcommand{\dTdt}{\frac{d\Tfieldt}
\newcommand{\deltaE}{\delta E}
\newcommand{\deltafield}{\ensuremath{\delta m}
\newcommand{\deltam}{\delta m}
\newcommand{\deq}{\displaystyle}
\newcommand{\docref}[1]{\texttt{#1}
\newcommand{\eV}{\,\text{eV}
\newcommand{\epsilonT}{\varepsilon_T}
\newcommand{\epsilonzero}{\varepsilon_0}
\newcommand{\etavis}{\eta_{\text{visual}
\newcommand{\e}{\mathrm{e}
\newcommand{\gW}{g_W}
\newcommand{\gammaf}{\gamma_{\text{Lorentz}
\newcommand{\gammamu}{\gamma^\mu}
\newcommand{\gs}{g_s}
\newcommand{\inftytext}{$\infty$}
\newcommand{\interval}[2]{#1:#2}
\newcommand{\kfrac}{K_{\text{frak}
\newcommand{\lP}{\ell_{\text{P}
\newcommand{\lP}{l_P}
\newcommand{\lambdah}{\ensuremath{\lambda_h}
\newcommand{\lambdah}{\lambda_h}
\newcommand{\lambdazero}{\lambda_0}
\newcommand{\mP}{m_{\text{P}
\newcommand{\mfield}{m(x,t)}
\newcommand{\mfield}{m}
\newcommand{\mh}{m_h}
\newcommand{\micrometer}{\ensuremath{\mu}
\newcommand{\mikrometer}{\ensuremath{\mu}
\newcommand{\myRightarrow}{\ensuremath{\Rightarrow}
\newcommand{\myapprox}{\ensuremath{\approx}
\newcommand{\myomega}{\ensuremath{\omega}
\newcommand{\myphi}{\ensuremath{\phi}
\newcommand{\mypi}{\ensuremath{\pi}
\newcommand{\mypropto}{\ensuremath{\propto}
\newcommand{\myrightarrow}{\ensuremath{\rightarrow}
\newcommand{\mysim}{\ensuremath{\sim}
\newcommand{\mysqrt}{\ensuremath{\sqrt}
\newcommand{\mytimes}{\ensuremath{\times}
\newcommand{\natunits}{\hbar = c = G = k_B = 1}
\newcommand{\natunits}{\text{(nat. Einh.)}
\newcommand{\natunits}{\text{(nat. units)}
\newcommand{\nulep}{\nu}
\newcommand{\nuzero}{\nu_0}
\newcommand{\partialop}{\ensuremath{\partial}
\newcommand{\pdTdt}{\frac{\partial\Tfieldt}
\newcommand{\pdTdx}{\nabla\Tfieldt}
\newcommand{\phiT}{\phi}
\newcommand{\pichar}{\pi}
\newcommand{\primrel}[1]{\mathbf{#1}
\newcommand{\rhoCMB}{\rho_{\text{CMB}
\newcommand{\rhoCasimir}{\rho_{\text{Casimir}
\newcommand{\rhoE}{\rho_E}
\newcommand{\rhofield}{\ensuremath{\rho}
\newcommand{\rzero}{r_0}
\newcommand{\slashk}{\cancel{k}
\newcommand{\slashp}{\cancel{p}
\newcommand{\slashq}{\cancel{q}
\newcommand{\tP}{t_P}
\newcommand{\tP}{t_{\text{P}
\newcommand{\tablescale}{0.9}
\newcommand{\tzero}{t_0}
\newcommand{\vect}[1]{\boldsymbol{#1}
\newcommand{\vecx}{\vec{x}
\newcommand{\vh}{v}
\newcommand{\vr}{\vec{r}
\newcommand{\warningx}{\color{red}
\newcommand{\warningx}{\textbf{!}
\newcommand{\warningx}{{\color{red}
\newcommand{\xiT}{\xi}
\newcommand{\xiconst}{\xi = \frac{4}
\newcommand{\xicoupling}{f(E/\Exi)}
\newcommand{\xigeom}{\xi_{\text{geom}
\newcommand{\xigeom}{\xi}
\newcommand{\xikonst}{\xi = \frac{4}
\newcommand{\xiparticle}{\xi_{\text{particle}
\newcommand{\xipar}{\ensuremath{\xi}
\newcommand{\xipar}{\xi_0}
\newcommand{\xipar}{\xi}
\newcommand{\xirat}{\xi_{\text{ratio}
\newtheorem{axiom}{Axiom}
\newtheorem{category}{Category-Theoretic Basis}
\newtheorem{category}{Kategorientheoretische Basis}
\newtheorem{corollary}[theorem]{Corollary}
\newtheorem{corollary}[theorem]{Korollar}
\newtheorem{corollary}{Corollary}
\newtheorem{corollary}{Korollar}
\newtheorem{definition}[theorem]{Definition}
\newtheorem{definition}{Definition}
\newtheorem{discovery}{Discovery}
\newtheorem{discovery}{Neue Entdeckung}
\newtheorem{discovery}{New Discovery}
\newtheorem{discovery}{Revolutionary Discovery}
\newtheorem{entdeckung}{Entdeckung}
\newtheorem{entdeckung}{Revolutionäre Entdeckung}
\newtheorem{erkenntnis}{Erkenntnis}
\newtheorem{erkenntnis}{Schlüsselerkenntnis}
\newtheorem{example}[theorem]{Beispiel}
\newtheorem{example}[theorem]{Example}
\newtheorem{example}{Beispiel}
\newtheorem{example}{Example}
\newtheorem{insight}{Central Insight}
\newtheorem{insight}{Insight}
\newtheorem{insight}{Key Insight}
\newtheorem{insight}{Wichtige Einsicht}
\newtheorem{insight}{Zentrale Einsicht}
\newtheorem{lemma}[theorem]{Lemma}
\newtheorem{lemma}{Lemma}
\newtheorem{principle}{Fundamental Principle}
\newtheorem{principle}{Fundamentales Prinzip}
\newtheorem{principle}{Grundlegendes Prinzip}
\newtheorem{principle}{Principle}
\newtheorem{principle}{Prinzip}
\newtheorem{prinzip}{Grundprinzip}
\newtheorem{proof_step}{Beweisschritt}
\newtheorem{proof_step}{Proof Step}
\newtheorem{proposition}[theorem]{Proposition}
\newtheorem{proposition}{Proposition}
\newtheorem{remark}[theorem]{Bemerkung}
\newtheorem{remark}[theorem]{Remark}
\newtheorem{theorem}{Theorem}
\newtheorem{warning}[theorem]{Warning}
\newtheorem{warning}[theorem]{Warnung}
\newunicodechar{±}{\ensuremath{\pm}
\newunicodechar{×}{\ensuremath{\times}
\newunicodechar{÷}{\ensuremath{\div}
\newunicodechar{ħ}{\ensuremath{\hbar}
\newunicodechar{Α}{\ensuremath{A}
\newunicodechar{Β}{\ensuremath{B}
\newunicodechar{Γ}{\ensuremath{\Gamma}
\newunicodechar{Δ}{\ensuremath{\Delta}
\newunicodechar{Ε}{\ensuremath{E}
\newunicodechar{Ζ}{\ensuremath{Z}
\newunicodechar{Η}{\ensuremath{H}
\newunicodechar{Θ}{\ensuremath{\Theta}
\newunicodechar{Ι}{\ensuremath{I}
\newunicodechar{Κ}{\ensuremath{K}
\newunicodechar{Λ}{\ensuremath{\Lambda}
\newunicodechar{Μ}{\ensuremath{M}
\newunicodechar{Ν}{\ensuremath{N}
\newunicodechar{Ξ}{\ensuremath{\Xi}
\newunicodechar{Ο}{\ensuremath{O}
\newunicodechar{Π}{\ensuremath{\Pi}
\newunicodechar{Ρ}{\ensuremath{P}
\newunicodechar{Σ}{\ensuremath{\Sigma}
\newunicodechar{Τ}{\ensuremath{T}
\newunicodechar{Υ}{\ensuremath{\Upsilon}
\newunicodechar{Φ}{\ensuremath{\Phi}
\newunicodechar{Χ}{\ensuremath{X}
\newunicodechar{Ψ}{\ensuremath{\Psi}
\newunicodechar{Ω}{\ensuremath{\Omega}
\newunicodechar{α}{\ensuremath{\alpha}
\newunicodechar{β}{\ensuremath{\beta}
\newunicodechar{γ}{\ensuremath{\gamma}
\newunicodechar{δ}{\ensuremath{\delta}
\newunicodechar{ε}{\ensuremath{\varepsilon}
\newunicodechar{ζ}{\ensuremath{\zeta}
\newunicodechar{η}{\ensuremath{\eta}
\newunicodechar{θ}{\ensuremath{\theta}
\newunicodechar{ι}{\ensuremath{\iota}
\newunicodechar{κ}{\ensuremath{\kappa}
\newunicodechar{λ}{\ensuremath{\lambda}
\newunicodechar{μ}{\ensuremath{\mu}
\newunicodechar{ν}{\ensuremath{\nu}
\newunicodechar{ξ}{\ensuremath{\xi}
\newunicodechar{ο}{\ensuremath{o}
\newunicodechar{π}{\ensuremath{\pi}
\newunicodechar{ρ}{\ensuremath{\rho}
\newunicodechar{σ}{\ensuremath{\sigma}
\newunicodechar{τ}{\ensuremath{\tau}
\newunicodechar{υ}{\ensuremath{\upsilon}
\newunicodechar{φ}{\ensuremath{\phi}
\newunicodechar{φ}{\ensuremath{\varphi}
\newunicodechar{χ}{\ensuremath{\chi}
\newunicodechar{ψ}{\ensuremath{\psi}
\newunicodechar{ω}{\ensuremath{\omega}
\newunicodechar{←}{\ensuremath{\leftarrow}
\newunicodechar{→}{\ensuremath{\rightarrow}
\newunicodechar{↔}{\ensuremath{\leftrightarrow}
\newunicodechar{⇐}{\ensuremath{\Leftarrow}
\newunicodechar{⇒}{\ensuremath{\Rightarrow}
\newunicodechar{⇔}{\ensuremath{\Leftrightarrow}
\newunicodechar{∂}{\ensuremath{\partial}
\newunicodechar{∅}{\ensuremath{\emptyset}
\newunicodechar{∇}{\ensuremath{\nabla}
\newunicodechar{∈}{\ensuremath{\in}
\newunicodechar{∉}{\ensuremath{\notin}
\newunicodechar{∏}{\ensuremath{\prod}
\newunicodechar{∑}{\ensuremath{\sum}
\newunicodechar{√}{\ensuremath{\sqrt}
\newunicodechar{∝}{\ensuremath{\propto}
\newunicodechar{∞}{\ensuremath{\infty}
\newunicodechar{∩}{\ensuremath{\cap}
\newunicodechar{∪}{\ensuremath{\cup}
\newunicodechar{∫}{\ensuremath{\int}
\newunicodechar{≈}{\ensuremath{\approx}
\newunicodechar{≠}{\ensuremath{\neq}
\newunicodechar{≤}{\ensuremath{\leq}
\newunicodechar{≥}{\ensuremath{\geq}
\newunicodechar{★}{\ensuremath{\star}
\newunicodechar{✓}{\checkmark}
\pgfplotsset{compat=1.17}
\pgfplotsset{compat=1.18}
\renewcommand{\cftchapfont}{\large\bfseries\color{blue}
\renewcommand{\cftchappagefont}{\large\bfseries\color{blue}
\renewcommand{\cftsecfont}{\bfseries}
\renewcommand{\cftsecfont}{\color{blue}
\renewcommand{\cftsecfont}{\large\bfseries\color{blue}
\renewcommand{\cftsecpagefont}{\bfseries}
\renewcommand{\cftsecpagefont}{\color{blue}
\renewcommand{\cftsecpagefont}{\large\bfseries\color{blue}
\renewcommand{\cftsubsecfont}{\color{blue!80!black}
\renewcommand{\cftsubsecfont}{\color{blue}
\renewcommand{\cftsubsecpagefont}{\color{blue!80!black}
\renewcommand{\cftsubsecpagefont}{\color{blue}
\renewcommand{\cftsubsubsecfont}{\color{blue!60!black}
\renewcommand{\cftsubsubsecfont}{\color{blue}
\renewcommand{\cftsubsubsecpagefont}{\color{blue!60!black}
\renewcommand{\cftsubsubsecpagefont}{\color{blue}
\renewcommand{\cfttoctitlefont}{\huge\bfseries\color{blue}
\renewcommand{\cfttoctitlefont}{\huge\bfseries}
\renewcommand{\familydefault}{\sfdefault}
\renewcommand{\footrulewidth}{0.4pt}
\renewcommand{\headrulewidth}{0.4pt}
\sisetup{locale = DE, group-separator = {.}
\sisetup{locale = DE}
\usetikzlibrary{arrows.meta,positioning,shapes.geometric}
\usetikzlibrary{decorations.pathmorphing, patterns, shapes.arrows}
\usetikzlibrary{intersections}
\usetikzlibrary{positioning, arrows.meta}
\usetikzlibrary{positioning, arrows}
\usetikzlibrary{positioning, shapes.geometric, arrows.meta}
\usetikzlibrary{positioning,shapes,arrows}

% Common settings
\setlength{\headheight}{15pt}
\pgfplotsset{compat=1.18}
\usetikzlibrary{positioning,shapes,arrows,arrows.meta}

% Hyperref setup
\hypersetup{
    colorlinks=true,
    linkcolor=blue,
    citecolor=blue,
    urlcolor=blue
}


\title{T0 Dokumentenübersicht En}
\author{Johann Pascher}
\date{\today}

\begin{document}

\maketitle
\tableofcontents

\begin{abstract}
		This overview presents the complete T0-theory series consisting of 8 fundamental documents that represent a revolutionary geometric reformulation of physics. Based on a single parameter $\xipar = \frac{4}{3} \times 10^{-4}$, all fundamental constants, particle masses, and physical phenomena from quantum mechanics to cosmology are uniformly described. The theory achieves over 99\% accuracy in predicting experimental values without free parameters and offers testable predictions for future experiments.
	\end{abstract}
	
	\tableofcontents
	\newpage
	
	# The T0 Revolution: A Paradigm Shift
	
	\begin{overview}
		\textbf{What is the T0-Theory?}
		
		The T0-Theory is a fundamental reformulation of physics that derives all known physical phenomena from the geometric structure of three-dimensional space. At its center is a single universal parameter:
		
		
```math-equation

			\boxed{\xipar = \frac{4}{3} \times 10^{-4} = 1.333333... \times 10^{-4}}
		
```

		
		\textbf{Revolutionary Reduction:}
		
			- \textbf{Standard Model + Cosmology:} $>25$ free parameters
			- \textbf{T0-Theory:} 1 geometric parameter
			- \textbf{Parameter Reduction:} 96\%!
		
		
		\textbf{Field of Application:} From particle masses to fundamental constants and cosmological structures
	\end{overview}
	
	# Document Series: Systematic Structure
	
	## Hierarchical Structure of the 8 Documents
	
	The T0-document series follows a logical progression from fundamental principles to specific applications:
	
	\begin{center}
		\begin{tikzpicture}[node distance=2cm, auto]
			\tikzstyle{doc} = [rectangle, rounded corners, minimum width=3cm, minimum height=1cm, text centered, draw=t0blue, fill=t0blue!20]
			\tikzstyle{arrow} = [thick,->]
			
			\node [doc] (doc1) {\textbf{1. Foundations}};
			\node [doc, below of=doc1] (doc2) {\textbf{2. Fine Structure}};
			\node [doc, below of=doc2] (doc3) {\textbf{3. Gravitation}};
			\node [doc, below of=doc3] (doc4) {\textbf{4. Particle Masses}};
			\node [doc, right of=doc4, xshift=2cm] (doc5) {\textbf{5. Neutrinos}};
			\node [doc, above of=doc5] (doc6) {\textbf{6. Cosmology}};
			\node [doc, above of=doc6] (doc7) {\textbf{7. g-2 Anomalies}};
			\node [doc, below of=doc7, yshift=-1cm] (doc8) {\textbf{8. QM-QFT-RT}};
			
			\draw [arrow] (doc1) -- (doc2);
			\draw [arrow] (doc2) -- (doc3);
			\draw [arrow] (doc3) -- (doc4);
			\draw [arrow] (doc4) -- (doc5);
			\draw [arrow] (doc4) -- (doc6);
			\draw [arrow] (doc4) -- (doc7);
			\draw [arrow] (doc7) -- (doc8);
		\end{tikzpicture}
	\end{center}
	
	# Document 1: T0\_Foundations\_En.pdf
	
	\begin{documentbox}
		\textbf{Subtitle:} The Geometric Foundations of Physics
		
		\textbf{Central Contents:}
		
			- \textbf{Fundamental Parameter:} $\xipar = \frac{4}{3} \times 10^{-4}$ as geometric constant
			- \textbf{Time-Mass Duality:} $T \cdot m = 1$ in natural units
			- \textbf{Fractal Spacetime Structure:} $D_f = 2.94$ and $K_{\text{frak}} = 0.986$
			- \textbf{Levels of Interpretation:} Harmonic, geometric, field-theoretic
			- \textbf{Universal Formula Structure:} Template for all T0 relations
		
		
		\textbf{Fundamental Insights:}
		
			- Tetrahedral packing as space base structure
			- Quantum field theoretic derivation of $10^{-4}$
			- Characteristic energy scales: $E_0 = 7.398$ MeV
			- Philosophical implications of geometric physics
		
		
		\textbf{Status:} Theoretical foundation - fully established
	\end{documentbox}
	
	# Document 2: T0\_FineStructure\_En.pdf
	
	\begin{documentbox}
		\textbf{Subtitle:} Derivation of $\alpha$ from Geometric Principles
		
		\textbf{Central Formula:}
		
```math-equation

			\boxed{\alpha = \xipar \cdot \left(\frac{E_0}{1\,\text{MeV}}\right)^2}
		
```

		
		\textbf{Key Results:}
		
			- \textbf{T0 Prediction:} $\alpha^{-1} = 137.04$
			- \textbf{Experiment:} $\alpha^{-1} = 137.036$
			- \textbf{Deviation:} 0.003\% (excellent agreement)
		
		
		\textbf{Theoretical Innovations:}
		
			- Characteristic energy $E_0 = \sqrt{m_e \cdot m_\mu}$
			- Logarithmic symmetry of lepton masses
			- Fundamental dependence $\alpha \propto \xipar^{11/2}$
			- Why numerical ratios must not be simplified
		
		
		\textbf{Status:} Experimentally confirmed - excellent accuracy
	\end{documentbox}
	
	# Document 3: T0\_GravitationalConstant\_En.pdf
	
	\begin{documentbox}
		\textbf{Subtitle:} Systematic Derivation of $G$ from Geometric Principles
		
		\textbf{Complete Formula:}
		
```math-equation

			\boxed{G_{\text{SI}} = \frac{\xipar^2}{4 m_e} \times C_{\text{conv}} \times K_{\text{frak}}}
		
```

		
		\textbf{Conversion Factors:}
		
			- \textbf{Dimensional Correction:} $C_1 = 3.521 \times 10^{-2}$ 
			- \textbf{SI Conversion:} $C_{\text{conv}} = 7.783 \times 10^{-3}$
			- \textbf{Fractal Correction:} $K_{\text{frak}} = 0.986$
		
		
		\textbf{Experimental Verification:}
		
			- \textbf{T0 Prediction:} $G = 6.67429 \times 10^{-11}$ m³/(kg·s²)
			- \textbf{CODATA 2018:} $G = 6.67430 \times 10^{-11}$ m³/(kg·s²)
			- \textbf{Deviation:} < 0.0002\% (extraordinary precision)
		
		
		\textbf{Physical Meaning:} Gravitation as geometric spacetime-matter coupling
		
		\textbf{Status:} Experimentally confirmed - highest precision
	\end{documentbox}
	
	# Document 4: T0\_ParticleMasses\_En.pdf
	
	\begin{documentbox}
		\textbf{Subtitle:} Parameter-Free Calculation of All Fermion Masses
		
		\textbf{Two Equivalent Methods:}
		
			- \textbf{Direct Geometry:} $m_i = \frac{K_{\text{frak}}}{\xi_i} \times C_{\text{conv}}$
			- \textbf{Extended Yukawa:} $m_i = y_i \times v$ with $y_i = r_i \times \xipar^{p_i}$
		
		
		\textbf{Quantum Number System:} Each particle receives $(n,l,j)$-assignment
		
		\textbf{Experimental Successes:}
		\begin{center}
			\begin{tabular}{lcc}
				\toprule
				\textbf{Particle Class} & \textbf{Number} & \textbf{Avg. Accuracy} \\
				\midrule
				Charged Leptons & 3 & 98.3\% \\
				Up-type Quarks & 3 & 99.1\% \\
				Down-type Quarks & 3 & 98.8\% \\
				Bosons & 3 & 99.4\% \\
				\midrule
				\textbf{Total (established)} & \textbf{12} & \textbf{99.0\%} \\
				\bottomrule
			\end{tabular}
		\end{center}
		
		\textbf{Revolutionary Reduction:} From 15+ free mass parameters to 0!
		
		\textbf{Status:} Experimentally confirmed - systematic successes
	\end{documentbox}
	
	# Document 5: T0\_Neutrinos\_En.pdf
	
	\begin{documentbox}
		\textbf{Subtitle:} The Photon Analogy and Geometric Oscillations
		
		\textbf{Special Treatment Required:}
		
			- \textbf{Photon Analogy:} Neutrinos as "damped photons"
			- \textbf{Double $\xi$-Suppression:} $m_\nu = \frac{\xipar^2}{2} \times m_e = 4.54$ meV
			- \textbf{Geometric Oscillations:} Phases instead of mass differences
		
		
		\textbf{T0 Predictions:}
		
			- \textbf{Uniform Masses:} All flavors: $m_\nu = 4.54$ meV
			- \textbf{Sum:} $\Sigma m_\nu = 13.6$ meV
			- \textbf{Velocity:} $v_\nu = c(1 - \xipar^2/2)$
		
		
		\textbf{Experimental Classification:}
		
			- \textbf{Cosmological Limits:} $\Sigma m_\nu < 70$ meV $\checkmark$
			- \textbf{KATRIN Experiment:} $m_\nu < 800$ meV $\checkmark$
			- \textbf{Target Value Estimate:} $\sim 15$ meV (T0 at 30\%)
		
		
		\textbf{Important Note:} Highly speculative - honest scientific limitation
		
		\textbf{Status:} Speculative - testable predictions, but unconfirmed
	\end{documentbox}
	
	# Document 6: T0\_Cosmology\_En.pdf
	
	\begin{documentbox}
		\textbf{Subtitle:} Static Universe and $\xi$-Field Manifestations
		
		\textbf{Revolutionary Cosmology:}
		
			- \textbf{Static Universe:} No Big Bang, eternally existing
			- \textbf{Time-Energy Duality:} Big Bang forbidden by $\Delta E \times \Delta t \geq \frac{\hbar}{2}$
			- \textbf{CMB from $\xi$-Field:} Not from z=1100 decoupling
		
		
		\textbf{Casimir-CMB Connection:}
		
			- \textbf{Characteristic Length:} $L_\xi = 100$ $\mu$m
			- \textbf{Theoretical Ratio:} $|\rho_{\text{Casimir}}|/\rho_{\text{CMB}} = 308$
			- \textbf{Experimental:} 312 (98.7\% agreement)
		
		
		\textbf{Alternative Redshift:}
		
```math-equation

			z(\lambda_0, d) = \frac{\xipar \cdot d \cdot \lambda_0}{E_\xi}
		
```

		
		\textbf{Cosmological Problems Solved:}
		
			- Horizon problem, flatness problem, monopole problem
			- Hubble tension, age problem, dark energy
			- Parameters: From 25+ to 1 ($\xipar$)
		
		
		\textbf{Status:} Testable hypotheses - revolutionary alternative
	\end{documentbox}
	
	# Document 7: T0\_Anomalous\_Magnetic\_Moments\_En.pdf
	
	\begin{documentbox}
		\textbf{Subtitle:} Solution to the Muon g-2 Anomaly through Time Field Extension
		
		\textbf{The Muon g-2 Problem:}
		
			- \textbf{Experimental Deviation:} $\Delta a_\mu = 251 \times 10^{-11}$ (4.2$\sigma$)
			- \textbf{Largest Discrepancy:} Between theory and experiment in modern physics
		
		
		\textbf{T0 Solution through Time Field:}
		
```math-equation

			\boxed{\Delta a_\ell = 251 \times 10^{-11} \times \left(\frac{m_\ell}{m_\mu}\right)^2}
		
```

		
		\textbf{Universal Predictions:}
		\begin{center}
			\begin{tabular}{lccc}
				\toprule
				\textbf{Lepton} & \textbf{T0 Correction} & \textbf{Experiment} & \textbf{Status} \\
				\midrule
				Electron & $5.8 \times 10^{-15}$ & Agreement & $\checkmark$ \\
				Muon & $2.51 \times 10^{-9}$ & 4.2$\sigma$ Deviation & $\checkmark$ \\
				Tau & $7.11 \times 10^{-7}$ & Prediction & Test \\
				\bottomrule
			\end{tabular}
		\end{center}
		
		\textbf{Theoretical Basis:} Extended Lagrangian density with fundamental time field
		
		\textbf{Status:} Exact solution to current problem - Tau test pending
	\end{documentbox}
	
	# Document 8: T0\_QM-QFT-RT\_En.pdf
	
	\begin{documentbox}
		\textbf{Subtitle:} Unification of QM, QFT, and RT from a Geometric Foundation
		
		\textbf{Central Contents:}
		
			- \textbf{Universal T0 Field Equation:} $\square \Efield + \xipar \cdot \mathcal{F}[\Efield] = 0$ as basis for all theories
			- \textbf{Time-Mass Duality:} $T \cdot m = 1$ connects all three pillars of physics
			- \textbf{Emergent Quantum Properties:} QM as approximation of the energy field
			- \textbf{Field Description:} All particles as excitations of a fundamental field $\Efield$
			- \textbf{Renormalization Solution:} Natural cutoff through $\EP/\xipar$
			- \textbf{Relativistic Extension:} Extended Einstein equations with $\Lambda_{\xipar}$
		
		
		\textbf{Fundamental Insights:}
		
			- Deterministic interpretation of quantum mechanics through local time field
			- Wave-particle duality from field geometry
			- Energy scales hierarchy: Planck to QCD through $\xipar$-corrections
			- Gravitation as field curvature, dark energy as $\xipar^2 c^4 / G$
			- Philosophical implications: Unity of physics through geometric principles
		
		
		\textbf{Status:} Theoretical unification - builds on all previous documents, testable predictions
	\end{documentbox}
	
	# Scientific Achievements: Quantitative Summary
	
	\begin{achievement}
		\textbf{Experimental Confirmations of the T0-Theory:}
		
		\begin{center}
			\begin{longtable}{lccc}
				\caption{Complete Success Statistics of T0 Predictions} \\
				\toprule
				\textbf{Physical Quantity} & \textbf{T0 Prediction} & \textbf{Experiment} & \textbf{Deviation} \\
				\midrule
				\endfirsthead
				\multicolumn{4}{c}{Continuation of the Table} \\
				\toprule
				\textbf{Physical Quantity} & \textbf{T0 Prediction} & \textbf{Experiment} & \textbf{Deviation} \\
				\midrule
				\endhead
				\bottomrule
				\endlastfoot
				
				\multicolumn{4}{l}{\textbf{Fundamental Constants}} \\
				\midrule
				$\alpha^{-1}$ & 137.04 & 137.036 & 0.003\% \\
				$G$ [$10^{-11}$ m³/(kg·s²)] & 6.67429 & 6.67430 & <0.0002\% \\
				\midrule
				
				\multicolumn{4}{l}{\textbf{Charged Leptons [MeV]}} \\
				\midrule
				$m_e$ & 0.504 & 0.511 & 1.4\% \\
				$m_\mu$ & 105.1 & 105.66 & 0.5\% \\
				$m_\tau$ & 1727.6 & 1776.86 & 2.8\% \\
				\midrule
				
				\multicolumn{4}{l}{\textbf{Quarks [MeV]}} \\
				\midrule
				$m_u$ & 2.27 & 2.2 & 3.2\% \\
				$m_d$ & 4.74 & 4.7 & 0.9\% \\
				$m_s$ & 98.5 & 93.4 & 5.5\% \\
				$m_c$ & 1284.1 & 1270 & 1.1\% \\
				$m_b$ & 4264.8 & 4180 & 2.0\% \\
				$m_t$ [GeV] & 171.97 & 172.76 & 0.5\% \\
				\midrule
				
				\multicolumn{4}{l}{\textbf{Bosons [GeV]}} \\
				\midrule
				$m_H$ & 124.8 & 125.1 & 0.2\% \\
				$m_W$ & 79.8 & 80.38 & 0.7\% \\
				$m_Z$ & 90.3 & 91.19 & 1.0\% \\
				\midrule
				
				\multicolumn{4}{l}{\textbf{Anomalous Magnetic Moments}} \\
				\midrule
				$\Delta a_\mu$ [$10^{-9}$] & 2.51 & 2.51$\pm$0.59 & Exact \\
				\midrule
				
				\multicolumn{4}{l}{\textbf{Cosmology}} \\
				\midrule
				Casimir/CMB Ratio & 308 & 312 & 1.3\% \\
				$L_\xi$ [$\mu$m] & 100 & (theoretical) & -- \\
			\end{longtable}
		\end{center}
		
		\textbf{Overall Statistics of Established Predictions:}
		
			- \textbf{Number of Tested Quantities:} 16
			- \textbf{Average Accuracy:} 99.1\%
			- \textbf{Best Prediction:} Gravitational constant (<0.0002\%)
			- \textbf{Systematic Successes:} All orders of magnitude correct
		
	\end{achievement}
	
	# Theoretical Innovations
	
	\begin{foundation}
		\textbf{Fundamental Breakthroughs of the T0-Theory:}
		
		
			- \textbf{Parameter Reduction:} From >25 to 1 parameter (96\% reduction)
			
			- \textbf{Geometric Unification:} All physics from 3D space structure
			
			- \textbf{Fractal Quantum Spacetime:} Systematic consideration of $K_{\text{frak}} = 0.986$
			
			- \textbf{Time-Mass Duality:} $T \cdot m = 1$ as fundamental principle
			
			- \textbf{Harmonic Physics:} $\frac{4}{3}$ as universal geometric constant
			
			- \textbf{Quantum Number System:} $(n,l,j)$-assignment for all particles
			
			- \textbf{Two Equivalent Methods:} Direct geometry $\leftrightarrow$ Extended Yukawa
			
			- \textbf{Experimental Precision:} >99\% without parameter adjustment
			
			- \textbf{Cosmological Revolution:} Static universe without Big Bang
			
			- \textbf{Testable Predictions:} Specific, falsifiable hypotheses
		
	\end{foundation}
	
	# Comparison with Established Theories
	
	\begin{center}
		\begin{longtable}{lccc}
			\caption{T0-Theory vs. Standard Approaches} \\
			\toprule
			\textbf{Aspect} & \textbf{Standard Model} & \textbf{$\Lambda$CDM} & \textbf{T0-Theory} \\
			\midrule
			\endfirsthead
			\multicolumn{4}{c}{Continuation of the Table} \\
			\toprule
			\textbf{Aspect} & \textbf{Standard Model} & \textbf{$\Lambda$CDM} & \textbf{T0-Theory} \\
			\midrule
			\endhead
			\bottomrule
			\endlastfoot
			
			Free Parameters & 19+ & 6 & 1 \\
			Theoretical Basis & Empirical & Empirical & Geometric \\
			Particle Masses & Arbitrary & -- & Calculable \\
			Constants & Experimental & Experimental & Derived \\
			Predictive Power & None & Limited & Comprehensive \\
			Dark Matter & New Particles & 26\% unknown & $\xi$-Field \\
			Dark Energy & -- & 69\% unknown & Not Required \\
			Big Bang & -- & Required & Physically Impossible \\
			Hierarchy Problem & Unsolved & -- & Solved by $\xi$ \\
			Fine-Tuning & $>$20 Parameters & Cosmological & None \\
			Experimental Tests & Confirmed & Confirmed & 99\% Accuracy \\
			New Predictions & None & Few & Many Testable \\
		\end{longtable}
	\end{center}
	
	# Summary: The T0 Revolution
	
	\begin{overview}
		\textbf{What the T0-Theory Has Achieved:}
		
		\textbf{1. Scientific Successes:}
		
			- 99.1\% average accuracy for 16 tested quantities
			- Solution to the muon g-2 anomaly with exact prediction
			- Parameter reduction from >25 to 1 (96\% reduction)
			- Unified description from particle physics to cosmology
		
		
		\textbf{2. Theoretical Innovations:}
		
			- Geometric derivation of all fundamental constants
			- Fractal spacetime structure as quantum corrections
			- Time-mass duality as fundamental principle
			- Alternative cosmology without Big Bang problems
		
		
		\textbf{3. Experimental Predictions:}
		
			- Specific, testable hypotheses for all areas
			- Neutrino masses, cosmological parameters, g-2 anomalies
			- New phenomena at characteristic $\xi$-scales
		
		
		\textbf{4. Paradigm Shift:}
		
			- From empirical adjustment to geometric derivation
			- From many parameters to universal constant
			- From fragmented theories to unified framework
		
	\end{overview}
	
	
	# Philosophical and Philosophy of Science Significance
	
	\begin{foundation}
		\textbf{Paradigm Shift through the T0-Theory:}
		
		\textbf{1. From Complexity to Simplicity:}
		
			- \textbf{Standard Approach:} Many parameters, complex structures
			- \textbf{T0 Approach:} One parameter, elegant geometry
			- \textbf{Philosophy:} "Simplex veri sigillum" (Simplicity as the seal of truth)
		
		
		\textbf{2. From Empiricism to Rationalism:}
		
			- \textbf{Standard Approach:} Experimental adjustment of parameters
			- \textbf{T0 Approach:} Mathematical derivation from principles
			- \textbf{Philosophy:} Geometric order as foundation of reality
		
		
		\textbf{3. From Fragmentation to Unification:}
		
			- \textbf{Standard Approach:} Separate theories for different areas
			- \textbf{T0 Approach:} Unified framework from quantum to cosmos
			- \textbf{Philosophy:} Universal harmony of natural laws
		
		
		\textbf{4. From Stasis to Dynamics:}
		
			- \textbf{Standard Approach:} Constants taken as given
			- \textbf{T0 Approach:} Constants understood from geometric principles
			- \textbf{Philosophy:} Understanding rather than mere description
		
	\end{foundation}
	
	# Limits and Challenges
	
	## Known Limitations
	
	
		- \textbf{Neutrino Sector:} Highly speculative, experimentally unconfirmed
		- \textbf{QCD Renormalization:} Not fully integrated into T0 framework
		- \textbf{Electroweak Symmetry Breaking:} Geometric derivation incomplete
		- \textbf{Supersymmetry:} T0 predictions for superpartners missing
		- \textbf{Quantum Gravity:} Complete QFT formulation pending
	
	
	## Theoretical Challenges
	
	
		- \textbf{Renormalization:} Systematic treatment of divergences
		- \textbf{Symmetries:} Connection to known gauge symmetries
		- \textbf{Quantization:} Complete quantum field theory of the $\xi$-field
		- \textbf{Mathematical Rigor:} Proofs instead of plausible arguments
		- \textbf{Cosmological Details:} Structure formation without Big Bang
	
	
	## Experimental Challenges
	
	
		- \textbf{Precision Measurements:} Many tests at accuracy limits
		- \textbf{New Phenomena:} Characteristic $\xi$-scales hard to access
		- \textbf{Cosmological Tests:} Observation times of decades
		- \textbf{Technological Limits:} Some predictions beyond current capabilities
	
	
	# Future Developments
	
	## Theoretical Priorities
	
	
		- \textbf{Complete QFT:} Quantum field theory of the $\xi$-field
		- \textbf{Unification:} Integration of all four fundamental forces
		- \textbf{Mathematical Foundation:} Rigorous proofs of geometric relations
		- \textbf{Cosmological Elaboration:} Detailed alternative to the standard model
		- \textbf{Phenomenology:} Systematic derivation of all observable effects
	
	
	
	
	# The Significance for the Future of Physics
	
	\begin{foundation}
		\textbf{Why the T0-Theory is Revolutionary:}
		
		The T0-Theory is not just a new theory, but a fundamental paradigm shift in our understanding of nature:
		
		\textbf{1. Ontological Revolution:}
		
			- Nature is not complex, but elegantly simple
			- Geometry is fundamental, particles are derived
			- The universe follows harmonic, not chaotic principles
		
		
		\textbf{2. Epistemological Revolution:}
		
			- Understanding rather than mere description becomes possible again
			- Mathematical beauty becomes the criterion of truth
			- Deduction complements induction as a scientific method
		
		
		\textbf{3. Methodological Revolution:}
		
			- From "theory of everything" to "formula for everything"
			- Geometric intuition becomes a method of discovery
			- Unity rather than diversity becomes the research principle
		
		
		\textbf{4. Technological Revolutions:}
		
			- $\xi$-field manipulation for energy generation
			- Geometric control over fundamental interactions
			- New materials based on $\xi$-harmonies
		
	\end{foundation}
	
	# Conclusion
	
	The T0-Theory, documented in these 8 systematic works, presents a revolutionary alternative to the current understanding of physics. With a single geometric parameter $\xipar = \frac{4}{3} \times 10^{-4}$, all fundamental constants, particle masses, and physical phenomena from the quantum level to the cosmological scale are uniformly described.
	
	The experimental successes with over 99\% average accuracy, the solution to the muon g-2 anomaly, and the systematic reduction of over 25 free parameters to a single one demonstrate the transformative potential of this theory.
	
	While some aspects (especially neutrinos) are still speculative, the T0-Theory offers a coherent, testable alternative to the current standard models of particle physics and cosmology. The coming years will be decisive in testing the far-reaching predictions of this geometric reformulation of physics through targeted experiments.
	
	\textbf{The T0-Theory is more than a new physical theory - it is an invitation to understand nature as a harmonic, geometrically structured whole, in which simplicity and beauty give rise to the complexity of observed phenomena.}
	
	\vfill
	
	\begin{center}
		\hrule
		\vspace{0.5cm}
		\textit{This overview summarizes the complete T0-document series}\\
		\textit{All 8 documents are available for detailed study}\\
		\vspace{0.3cm}
		\textbf{T0-Theory: Time-Mass Duality Framework}\\
		\textit{Johann Pascher, HTL Leonding, Austria}\\
		\textit{GitHub: https://github.com/jpascher/T0-Time-Mass-Duality}
		\vspace{0.3cm}
	\end{center}

\end{document}
