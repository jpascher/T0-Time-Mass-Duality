\documentclass[11pt,a4paper,openany]{book}

% Essential packages
\usepackage[utf8]{inputenc}
\usepackage[T1]{fontenc}
\usepackage[english]{babel}
\usepackage[a4paper,margin=2.5cm]{geometry}
\usepackage{lmodern}

% Math and physics packages
\usepackage{amsmath}
\usepackage{amssymb}
\usepackage{amsthm}
\usepackage{mathtools}
\usepackage{physics}
\usepackage{siunitx}

% Graphics and tables
\usepackage{graphicx}
\usepackage[table,xcdraw]{xcolor}
\usepackage{tikz}
\usepackage{pgfplots}
\usepackage{tcolorbox}
\usepackage{booktabs}
\usepackage{array}
\usepackage{longtable}
\usepackage{float}

% Document formatting
\usepackage{fancyhdr}
\usepackage{tocloft}
\usepackage{hyperref}
\usepackage{cleveref}
\usepackage{microtype}
\usepackage{enumitem}
\usepackage{newunicodechar}

% Additional packages
\usepackage{adjustbox}
\usepackage{algorithm}
\usepackage{algorithmic}
\usepackage{amsfonts}
\usepackage{amsmath,amsfonts,amssymb}
\usepackage{amsmath,amsfonts,amssymb,physics}
\usepackage{amsmath,amssymb}
\usepackage{amsmath,amssymb,amsfonts,amsthm}
\usepackage{amsmath,amssymb,amsthm}
\usepackage{amsmath,amssymb,physics,graphicx,xcolor,amsthm}
\usepackage{bm}
\usepackage{booktabs,array,longtable,multirow}
\usepackage{braket}
\usepackage{breakurl}
\usepackage{cancel}
\usepackage{caption}
\usepackage{cite}
\usepackage{color}
\usepackage{colortbl}
\usepackage{csquotes}
\usepackage{doi}
\usepackage{forest}
\usepackage{gensymb}
\usepackage{geometry,fancyhdr}
\usepackage{graphicx,tikz,pgfplots}
\usepackage{hyperref,url}
\usepackage{hyphenat}
\usepackage{listings}
\usepackage{listings,enumerate}
\usepackage{mdframed}
\usepackage{multicol}
\usepackage{multirow}
\usepackage{natbib}
\usepackage{pdflscape}
\usepackage{ragged2e}
\usepackage{setspace}
\usepackage{siunitx,xcolor,graphicx}
\usepackage{slashed}
\usepackage{tabularx}
\usepackage{textcomp}
\usepackage{textgreek}
\usepackage{tikz,pgfplots}
\usepackage{upgreek}
\usepackage{url}

% Custom commands and definitions
\definecolor{blue}
\definecolor{blue}{rgb}{0,0,1}
\definecolor{boxgray}
\definecolor{boxgray}{RGB}{240,240,240}
\definecolor{deepblue}
\definecolor{deepblue}{RGB}{0,0,127}
\definecolor{deepgreen}
\definecolor{deepgreen}{RGB}{0,127,0}
\definecolor{deepred}
\definecolor{deepred}{RGB}{191,0,0}
\definecolor{t0blue}
\definecolor{t0blue}{RGB}{0,102,204}
\definecolor{t0blue}{RGB}{33,150,243}
\definecolor{t0green}
\definecolor{t0green}{RGB}{0,153,0}
\definecolor{t0green}{RGB}{0,153,76}
\definecolor{t0green}{RGB}{76,175,80}
\definecolor{t0orange}
\definecolor{t0orange}{RGB}{255,152,0}
\definecolor{t0purple}
\definecolor{t0purple}{RGB}{102,0,204}
\definecolor{t0purple}{RGB}{156,39,176}
\definecolor{t0red}
\definecolor{t0red}{RGB}{204,0,0}
\definecolor{t0red}{RGB}{204,0,51}
\definecolor{t0red}{RGB}{244,67,54}
\definecolor{t0yellow}
\definecolor{t0yellow}{RGB}{255,204,0}
\geometry{a4paper, left=25mm, right=25mm, top=25mm, bottom=25mm}
\geometry{a4paper, margin=1in}
\geometry{a4paper, margin=2.5cm}
\geometry{a4paper, margin=2cm}
\geometry{left=2.5cm,right=2.5cm,top=2.5cm,bottom=2.5cm}
\geometry{left=2cm,right=2cm,top=2cm,bottom=2cm}
\geometry{margin=1in}
\geometry{margin=2.5cm}
\geometry{margin=2cm}
\hypersetup{
	colorlinks=true,
	linkcolor=blue,
	citecolor=blue,
	urlcolor=blue,
	pdftitle={Analysis and Implications of MNRAS Paper 544 for the T0-Theory}
\hypersetup{
	colorlinks=true,
	linkcolor=blue,
	citecolor=blue,
	urlcolor=blue,
	pdftitle={Beweis: Die Feinstrukturkonstante α = 1 in natürlichen Einheiten}
\hypersetup{
	colorlinks=true,
	linkcolor=blue,
	citecolor=blue,
	urlcolor=blue,
	pdftitle={Beweis: Die Koide-Formel enthält implizit $\xi$}
\hypersetup{
	colorlinks=true,
	linkcolor=blue,
	citecolor=blue,
	urlcolor=blue,
	pdftitle={Chinas Photonischer Quantenchip: 1000x-Speedup und T0-Integration}
\hypersetup{
	colorlinks=true,
	linkcolor=blue,
	citecolor=blue,
	urlcolor=blue,
	pdftitle={Complete Derivation of Higgs Mass and Wilson Coefficients}
\hypersetup{
	colorlinks=true,
	linkcolor=blue,
	citecolor=blue,
	urlcolor=blue,
	pdftitle={Complete Particle Spectrum: Standard Model vs T0 Theory}
\hypersetup{
	colorlinks=true,
	linkcolor=blue,
	citecolor=blue,
	urlcolor=blue,
	pdftitle={Conceptual Comparison of Unified Natural Units and Extended Standard Model}
\hypersetup{
	colorlinks=true,
	linkcolor=blue,
	citecolor=blue,
	urlcolor=blue,
	pdftitle={Connections between the Mizohata-Takeuchi Counterexample and the T0 Time-Mass Duality Theory}
\hypersetup{
	colorlinks=true,
	linkcolor=blue,
	citecolor=blue,
	urlcolor=blue,
	pdftitle={Das Relationale Zahlensystem: Primzahlen als fundamentale Verhältnisse}
\hypersetup{
	colorlinks=true,
	linkcolor=blue,
	citecolor=blue,
	urlcolor=blue,
	pdftitle={Das T0-Modell (Planck-Referenziert): Eine Neuformulierung der Physik}
\hypersetup{
	colorlinks=true,
	linkcolor=blue,
	citecolor=blue,
	urlcolor=blue,
	pdftitle={Das T0-Modell: Zeit-Energie-Dualität und geometrische Ruhemasse}
\hypersetup{
	colorlinks=true,
	linkcolor=blue,
	citecolor=blue,
	urlcolor=blue,
	pdftitle={Der Massenskalierungsexponent κ in der T0-Theorie}
\hypersetup{
	colorlinks=true,
	linkcolor=blue,
	citecolor=blue,
	urlcolor=blue,
	pdftitle={Der geometrische Formalismus der T0-Quantenmechanik und seine Anwendung auf Quantencomputer}
\hypersetup{
	colorlinks=true,
	linkcolor=blue,
	citecolor=blue,
	urlcolor=blue,
	pdftitle={Der xi Parameter und Teilchendifferenzierung in der T0-Theorie}
\hypersetup{
	colorlinks=true,
	linkcolor=blue,
	citecolor=blue,
	urlcolor=blue,
	pdftitle={Deterministic Quantum Mechanics via T0-Energy Field Formulation}
\hypersetup{
	colorlinks=true,
	linkcolor=blue,
	citecolor=blue,
	urlcolor=blue,
	pdftitle={Deterministische Quantenmechanik via T0-Energiefeld-Formulierung}
\hypersetup{
	colorlinks=true,
	linkcolor=blue,
	citecolor=blue,
	urlcolor=blue,
	pdftitle={Die Elektroneneinheitsladung in der T0-Theorie: Jenseits von Punkt-Singularitäten}
\hypersetup{
	colorlinks=true,
	linkcolor=blue,
	citecolor=blue,
	urlcolor=blue,
	pdftitle={Die Feinstrukturkonstante: Verschiedene Darstellungen und Beziehungen}
\hypersetup{
	colorlinks=true,
	linkcolor=blue,
	citecolor=blue,
	urlcolor=blue,
	pdftitle={Die Musikalische Spirale und die 137: Die mathematische Entdeckung der kosmischen Verstimmung}
\hypersetup{
	colorlinks=true,
	linkcolor=blue,
	citecolor=blue,
	urlcolor=blue,
	pdftitle={E=mc² = E=m: Die Konstanten-Illusion entlarvt}
\hypersetup{
	colorlinks=true,
	linkcolor=blue,
	citecolor=blue,
	urlcolor=blue,
	pdftitle={E=mc² = E=m: The Constants Illusion Exposed}
\hypersetup{
	colorlinks=true,
	linkcolor=blue,
	citecolor=blue,
	urlcolor=blue,
	pdftitle={Einfache Lagrange-Revolution: Von der Standardmodell-Komplexität zur T0-Eleganz}
\hypersetup{
	colorlinks=true,
	linkcolor=blue,
	citecolor=blue,
	urlcolor=blue,
	pdftitle={Einführung in die Umsetzung photonischer Bauteile auf Wafern für Nachrichtentechniker}
\hypersetup{
	colorlinks=true,
	linkcolor=blue,
	citecolor=blue,
	urlcolor=blue,
	pdftitle={Einführung in photonische Quantenchips für Nachrichtentechniker}
\hypersetup{
	colorlinks=true,
	linkcolor=blue,
	citecolor=blue,
	urlcolor=blue,
	pdftitle={Elimination der Masse als dimensionaler Platzhalter im T0-Modell}
\hypersetup{
	colorlinks=true,
	linkcolor=blue,
	citecolor=blue,
	urlcolor=blue,
	pdftitle={Elimination of Mass as Dimensional Placeholder in the T0 Model}
\hypersetup{
	colorlinks=true,
	linkcolor=blue,
	citecolor=blue,
	urlcolor=blue,
	pdftitle={Empirical Analysis of Deterministic Factorization Methods}
\hypersetup{
	colorlinks=true,
	linkcolor=blue,
	citecolor=blue,
	urlcolor=blue,
	pdftitle={Empirische Analyse deterministischer Faktorisierungsmethoden}
\hypersetup{
	colorlinks=true,
	linkcolor=blue,
	citecolor=blue,
	urlcolor=blue,
	pdftitle={Integration der Dirac-Gleichung im T0-Modell: Natürliche-Einheiten-Rahmenwerk}
\hypersetup{
	colorlinks=true,
	linkcolor=blue,
	citecolor=blue,
	urlcolor=blue,
	pdftitle={Integration of the Dirac Equation in the T0 Model: Natural Units Framework}
\hypersetup{
	colorlinks=true,
	linkcolor=blue,
	citecolor=blue,
	urlcolor=blue,
	pdftitle={Introduction to Photonic Quantum Chips for Communication Engineers}
\hypersetup{
	colorlinks=true,
	linkcolor=blue,
	citecolor=blue,
	urlcolor=blue,
	pdftitle={Introduction to the Implementation of Photonic Components on Wafers for Communication Engineers}
\hypersetup{
	colorlinks=true,
	linkcolor=blue,
	citecolor=blue,
	urlcolor=blue,
	pdftitle={Konzeptioneller Vergleich von Einheitlichen Natürlichen Einheiten und Erweitertem Standardmodell}
\hypersetup{
	colorlinks=true,
	linkcolor=blue,
	citecolor=blue,
	urlcolor=blue,
	pdftitle={Markov Chains in the Context of T0 Theory: Deterministic or Stochastic? A Treatise on Patterns, Preconditions, and Uncertainty}
\hypersetup{
	colorlinks=true,
	linkcolor=blue,
	citecolor=blue,
	urlcolor=blue,
	pdftitle={Markov-Ketten im Kontext der T0-Theorie: Deterministisch oder stochastisch? Ein Traktat zu Mustern, Voraussetzungen und Unsicherheit}
\hypersetup{
	colorlinks=true,
	linkcolor=blue,
	citecolor=blue,
	urlcolor=blue,
	pdftitle={Mathematical Analysis of T0-Shor Algorithm: Theoretical Framework and Computational Complexity}
\hypersetup{
	colorlinks=true,
	linkcolor=blue,
	citecolor=blue,
	urlcolor=blue,
	pdftitle={Mathematical Constructs of Alternative CMB Models: Unnikrishnan and Peratt in Harmony with the T0 Theory}
\hypersetup{
	colorlinks=true,
	linkcolor=blue,
	citecolor=blue,
	urlcolor=blue,
	pdftitle={Mathematische Analyse des T0-Shor Algorithmus: Theoretischer Rahmen und Berechnungskomplexität}
\hypersetup{
	colorlinks=true,
	linkcolor=blue,
	citecolor=blue,
	urlcolor=blue,
	pdftitle={Mathematische Konstrukte alternativer CMB-Modelle: Unnikrishnan und Peratt im Einklang mit der T0-Theorie}
\hypersetup{
	colorlinks=true,
	linkcolor=blue,
	citecolor=blue,
	urlcolor=blue,
	pdftitle={Natural Unit Systems: Universal Energy Conversion and Fundamental Length Scale Hierarchy}
\hypersetup{
	colorlinks=true,
	linkcolor=blue,
	citecolor=blue,
	urlcolor=blue,
	pdftitle={Natural Units in Theoretical Physics: A Treatise in the Context of T0 Theory}
\hypersetup{
	colorlinks=true,
	linkcolor=blue,
	citecolor=blue,
	urlcolor=blue,
	pdftitle={Natürliche Einheiten in der theoretischen Physik: Eine Abhandlung im Kontext der T0-Theorie}
\hypersetup{
	colorlinks=true,
	linkcolor=blue,
	citecolor=blue,
	urlcolor=blue,
	pdftitle={Natürliche Einheitensysteme: Universelle Energieumwandlung und fundamentale Längenskala-Hierarchie}
\hypersetup{
	colorlinks=true,
	linkcolor=blue,
	citecolor=blue,
	urlcolor=blue,
	pdftitle={Parameter System-Dependency in T0-Model: SI vs. Natural Units}
\hypersetup{
	colorlinks=true,
	linkcolor=blue,
	citecolor=blue,
	urlcolor=blue,
	pdftitle={Parameter-Systemabhängigkeit im T0-Modell: SI- vs. natürliche Einheiten}
\hypersetup{
	colorlinks=true,
	linkcolor=blue,
	citecolor=blue,
	urlcolor=blue,
	pdftitle={Proof: The Fine Structure Constant α = 1 in Natural Units}
\hypersetup{
	colorlinks=true,
	linkcolor=blue,
	citecolor=blue,
	urlcolor=blue,
	pdftitle={Proof: The Koide Formula Implicitly Contains $\xi$}
\hypersetup{
	colorlinks=true,
	linkcolor=blue,
	citecolor=blue,
	urlcolor=blue,
	pdftitle={Pure Energy T0 Theory: Ratio-Based Physics with SI Reference}
\hypersetup{
	colorlinks=true,
	linkcolor=blue,
	citecolor=blue,
	urlcolor=blue,
	pdftitle={Quantum Mechanics in the T0 Model: Field-Theoretic Foundations}
\hypersetup{
	colorlinks=true,
	linkcolor=blue,
	citecolor=blue,
	urlcolor=blue,
	pdftitle={Ratio-Based vs. Absolute: The Role of Fractal Correction in T0 Theory}
\hypersetup{
	colorlinks=true,
	linkcolor=blue,
	citecolor=blue,
	urlcolor=blue,
	pdftitle={Reine Energie T0-Theorie: Verhältnis-basierte Physik mit SI-Referenz}
\hypersetup{
	colorlinks=true,
	linkcolor=blue,
	citecolor=blue,
	urlcolor=blue,
	pdftitle={Simple Lagrangian Revolution: From Standard Model Complexity to T0 Elegance}
\hypersetup{
	colorlinks=true,
	linkcolor=blue,
	citecolor=blue,
	urlcolor=blue,
	pdftitle={Simplified Dirac Equation in T0 Theory: Field Node Approach}
\hypersetup{
	colorlinks=true,
	linkcolor=blue,
	citecolor=blue,
	urlcolor=blue,
	pdftitle={Simplified T0 Theory: Elegant Lagrangian Density for Time-Mass Duality}
\hypersetup{
	colorlinks=true,
	linkcolor=blue,
	citecolor=blue,
	urlcolor=blue,
	pdftitle={T0 Cosmology: Redshift as a Geometric Path Effect in a Static Universe}
\hypersetup{
	colorlinks=true,
	linkcolor=blue,
	citecolor=blue,
	urlcolor=blue,
	pdftitle={T0 Deterministic Quantum Computing: Complete Analysis of Important Algorithms}
\hypersetup{
	colorlinks=true,
	linkcolor=blue,
	citecolor=blue,
	urlcolor=blue,
	pdftitle={T0 Deterministisches Quantencomputing: Vollständige Analyse wichtiger Algorithmen}
\hypersetup{
	colorlinks=true,
	linkcolor=blue,
	citecolor=blue,
	urlcolor=blue,
	pdftitle={T0 Model: Complete Framework - From Time-Energy Duality to Universal Constants}
\hypersetup{
	colorlinks=true,
	linkcolor=blue,
	citecolor=blue,
	urlcolor=blue,
	pdftitle={T0 Model: Complete Parameter-Free Particle Mass Calculation}
\hypersetup{
	colorlinks=true,
	linkcolor=blue,
	citecolor=blue,
	urlcolor=blue,
	pdftitle={T0 Model: Unified Neutrino Formula Structure}
\hypersetup{
	colorlinks=true,
	linkcolor=blue,
	citecolor=blue,
	urlcolor=blue,
	pdftitle={T0 Model: Universal Energy Relations for Mol and Candela Units}
\hypersetup{
	colorlinks=true,
	linkcolor=blue,
	citecolor=blue,
	urlcolor=blue,
	pdftitle={T0 Modell: Vollständiges Framework - Von Zeit-Energie-Dualität zu universellen Konstanten}
\hypersetup{
	colorlinks=true,
	linkcolor=blue,
	citecolor=blue,
	urlcolor=blue,
	pdftitle={T0 Quantenfeldtheorie: QFT, QM und Quantencomputer}
\hypersetup{
	colorlinks=true,
	linkcolor=blue,
	citecolor=blue,
	urlcolor=blue,
	pdftitle={T0 Quantum Field Theory: QFT, QM and Quantum Computers}
\hypersetup{
	colorlinks=true,
	linkcolor=blue,
	citecolor=blue,
	urlcolor=blue,
	pdftitle={T0 Theory vs Bell's Theorem: How Deterministic Energy Fields Circumvent No-Go Theorems}
\hypersetup{
	colorlinks=true,
	linkcolor=blue,
	citecolor=blue,
	urlcolor=blue,
	pdftitle={T0 Theory: Final Extension to Hadrons - Physically Derived Corrections}
\hypersetup{
	colorlinks=true,
	linkcolor=blue,
	citecolor=blue,
	urlcolor=blue,
	pdftitle={T0 Theory: The Fine-Structure Constant}
\hypersetup{
	colorlinks=true,
	linkcolor=blue,
	citecolor=blue,
	urlcolor=blue,
	pdftitle={T0 Theory: The Gravitational Constant}
\hypersetup{
	colorlinks=true,
	linkcolor=blue,
	citecolor=blue,
	urlcolor=blue,
	pdftitle={T0-Kosmologie: Rotverschiebung als geometrischer Pfad-Effekt im statischen Universum}
\hypersetup{
	colorlinks=true,
	linkcolor=blue,
	citecolor=blue,
	urlcolor=blue,
	pdftitle={T0-Model: Complete Document Analysis and Structured Summary}
\hypersetup{
	colorlinks=true,
	linkcolor=blue,
	citecolor=blue,
	urlcolor=blue,
	pdftitle={T0-Model: Kinetic Energy of Electrons and Photons}
\hypersetup{
	colorlinks=true,
	linkcolor=blue,
	citecolor=blue,
	urlcolor=blue,
	pdftitle={T0-Model: The Hubble Parameter in Static Universe}
\hypersetup{
	colorlinks=true,
	linkcolor=blue,
	citecolor=blue,
	urlcolor=blue,
	pdftitle={T0-Modell-Verifikation: Skalen-Verhältnis-basierte Berechnungen}
\hypersetup{
	colorlinks=true,
	linkcolor=blue,
	citecolor=blue,
	urlcolor=blue,
	pdftitle={T0-Modell: Bewegungsenergie von Elektronen und Photonen}
\hypersetup{
	colorlinks=true,
	linkcolor=blue,
	citecolor=blue,
	urlcolor=blue,
	pdftitle={T0-Modell: Die Hubble-Konstante im statischen Universum}
\hypersetup{
	colorlinks=true,
	linkcolor=blue,
	citecolor=blue,
	urlcolor=blue,
	pdftitle={T0-Modell: Einheitliche Neutrino-Formel-Struktur}
\hypersetup{
	colorlinks=true,
	linkcolor=blue,
	citecolor=blue,
	urlcolor=blue,
	pdftitle={T0-Modell: Universelle Energiebeziehungen für Mol- und Candela-Einheiten}
\hypersetup{
	colorlinks=true,
	linkcolor=blue,
	citecolor=blue,
	urlcolor=blue,
	pdftitle={T0-Modell: Vollständige Dokumentenanalyse und strukturierte Zusammenfassung}
\hypersetup{
	colorlinks=true,
	linkcolor=blue,
	citecolor=blue,
	urlcolor=blue,
	pdftitle={T0-Modell: Vollständige parameterfreie Teilchenmassen-Berechnung}
\hypersetup{
	colorlinks=true,
	linkcolor=blue,
	citecolor=blue,
	urlcolor=blue,
	pdftitle={T0-QAT: $\xi$-Aware Quantization-Aware Training}
\hypersetup{
	colorlinks=true,
	linkcolor=blue,
	citecolor=blue,
	urlcolor=blue,
	pdftitle={T0-QFT ML Addendum: Machine Learning Derived Extensions}
\hypersetup{
	colorlinks=true,
	linkcolor=blue,
	citecolor=blue,
	urlcolor=blue,
	pdftitle={T0-QFT ML-Addendum: Maschinelle Lern-abgeleitete Erweiterungen}
\hypersetup{
	colorlinks=true,
	linkcolor=blue,
	citecolor=blue,
	urlcolor=blue,
	pdftitle={T0-Theorie vs Bells Theorem: Wie deterministische Energiefelder No-Go-Theoreme umgehen}
\hypersetup{
	colorlinks=true,
	linkcolor=blue,
	citecolor=blue,
	urlcolor=blue,
	pdftitle={T0-Theorie: Der Terrell-Penrose-Effekt und Massenvariation}
\hypersetup{
	colorlinks=true,
	linkcolor=blue,
	citecolor=blue,
	urlcolor=blue,
	pdftitle={T0-Theorie: Die Feinstrukturkonstante}
\hypersetup{
	colorlinks=true,
	linkcolor=blue,
	citecolor=blue,
	urlcolor=blue,
	pdftitle={T0-Theorie: Die Gravitationskonstante}
\hypersetup{
	colorlinks=true,
	linkcolor=blue,
	citecolor=blue,
	urlcolor=blue,
	pdftitle={T0-Theorie: Die T0-Zeit-Masse-Dualität}
\hypersetup{
	colorlinks=true,
	linkcolor=blue,
	citecolor=blue,
	urlcolor=blue,
	pdftitle={T0-Theorie: Die sieben Rätsel}
\hypersetup{
	colorlinks=true,
	linkcolor=blue,
	citecolor=blue,
	urlcolor=blue,
	pdftitle={T0-Theorie: Erweiterung auf Bell-Tests – ML-Simulationen (November 2025)}
\hypersetup{
	colorlinks=true,
	linkcolor=blue,
	citecolor=blue,
	urlcolor=blue,
	pdftitle={T0-Theorie: Finale Erweiterung auf Hadronen - Physikalisch abgeleitete Korrekturen}
\hypersetup{
	colorlinks=true,
	linkcolor=blue,
	citecolor=blue,
	urlcolor=blue,
	pdftitle={T0-Theorie: Finale Fraktale Massenformeln (November 2025)}
\hypersetup{
	colorlinks=true,
	linkcolor=blue,
	citecolor=blue,
	urlcolor=blue,
	pdftitle={T0-Theorie: Fraktaldimension aus Lepton-Massenverhältnis}
\hypersetup{
	colorlinks=true,
	linkcolor=blue,
	citecolor=blue,
	urlcolor=blue,
	pdftitle={T0-Theorie: Fundamentale Prinzipien}
\hypersetup{
	colorlinks=true,
	linkcolor=blue,
	citecolor=blue,
	urlcolor=blue,
	pdftitle={T0-Theorie: Herleitung der Gravitationskonstanten}
\hypersetup{
	colorlinks=true,
	linkcolor=blue,
	citecolor=blue,
	urlcolor=blue,
	pdftitle={T0-Theorie: Kosmische Beziehungen und universelle $\xi$-Konstante}
\hypersetup{
	colorlinks=true,
	linkcolor=blue,
	citecolor=blue,
	urlcolor=blue,
	pdftitle={T0-Theorie: Kosmologie}
\hypersetup{
	colorlinks=true,
	linkcolor=blue,
	citecolor=blue,
	urlcolor=blue,
	pdftitle={T0-Theorie: Netzwerkdarstellung und Dimensionsanalyse in der T0-Theorie}
\hypersetup{
	colorlinks=true,
	linkcolor=blue,
	citecolor=blue,
	urlcolor=blue,
	pdftitle={T0-Theorie: Teilchenmassen}
\hypersetup{
	colorlinks=true,
	linkcolor=blue,
	citecolor=blue,
	urlcolor=blue,
	pdftitle={T0-Theorie: Vollstaendiger Abschluss}
\hypersetup{
	colorlinks=true,
	linkcolor=blue,
	citecolor=blue,
	urlcolor=blue,
	pdftitle={T0-Theory: Complete Closure}
\hypersetup{
	colorlinks=true,
	linkcolor=blue,
	citecolor=blue,
	urlcolor=blue,
	pdftitle={T0-Theory: Complete Derivation of All Parameters Without Circularity}
\hypersetup{
	colorlinks=true,
	linkcolor=blue,
	citecolor=blue,
	urlcolor=blue,
	pdftitle={T0-Theory: Cosmic Relations and universal $\xi$-constant}
\hypersetup{
	colorlinks=true,
	linkcolor=blue,
	citecolor=blue,
	urlcolor=blue,
	pdftitle={T0-Theory: Cosmology}
\hypersetup{
	colorlinks=true,
	linkcolor=blue,
	citecolor=blue,
	urlcolor=blue,
	pdftitle={T0-Theory: Derivation of the Gravitational Constant}
\hypersetup{
	colorlinks=true,
	linkcolor=blue,
	citecolor=blue,
	urlcolor=blue,
	pdftitle={T0-Theory: Extension to Bell Tests – ML Simulations (November 2025)}
\hypersetup{
	colorlinks=true,
	linkcolor=blue,
	citecolor=blue,
	urlcolor=blue,
	pdftitle={T0-Theory: Final Fractal Mass Formulas (November 2025)}
\hypersetup{
	colorlinks=true,
	linkcolor=blue,
	citecolor=blue,
	urlcolor=blue,
	pdftitle={T0-Theory: Fractal Dimension from Lepton Mass Ratio}
\hypersetup{
	colorlinks=true,
	linkcolor=blue,
	citecolor=blue,
	urlcolor=blue,
	pdftitle={T0-Theory: Fundamental Principles}
\hypersetup{
	colorlinks=true,
	linkcolor=blue,
	citecolor=blue,
	urlcolor=blue,
	pdftitle={T0-Theory: Mass Variation as an Equivalent to Time Dilation}
\hypersetup{
	colorlinks=true,
	linkcolor=blue,
	citecolor=blue,
	urlcolor=blue,
	pdftitle={T0-Theory: Network Representation and Dimensional Analysis in the T0-Theory}
\hypersetup{
	colorlinks=true,
	linkcolor=blue,
	citecolor=blue,
	urlcolor=blue,
	pdftitle={T0-Theory: Neutrinos}
\hypersetup{
	colorlinks=true,
	linkcolor=blue,
	citecolor=blue,
	urlcolor=blue,
	pdftitle={T0-Theory: Particle Masses}
\hypersetup{
	colorlinks=true,
	linkcolor=blue,
	citecolor=blue,
	urlcolor=blue,
	pdftitle={T0-Theory: The Seven Riddles}
\hypersetup{
	colorlinks=true,
	linkcolor=blue,
	citecolor=blue,
	urlcolor=blue,
	pdftitle={T0-Theory: The T0-Time-Mass Duality}
\hypersetup{
	colorlinks=true,
	linkcolor=blue,
	citecolor=blue,
	urlcolor=blue,
	pdftitle={Temperature Units in Natural Units: T0-Theory}
\hypersetup{
	colorlinks=true,
	linkcolor=blue,
	citecolor=blue,
	urlcolor=blue,
	pdftitle={Temperatureinheiten in nat\"urlichen Einheiten: T0-Theorie}
\hypersetup{
	colorlinks=true,
	linkcolor=blue,
	citecolor=blue,
	urlcolor=blue,
	pdftitle={The Electron Unit Charge in T0 Theory: Beyond Point Singularities}
\hypersetup{
	colorlinks=true,
	linkcolor=blue,
	citecolor=blue,
	urlcolor=blue,
	pdftitle={The Fine Structure Constant: Various Representations and Relationships}
\hypersetup{
	colorlinks=true,
	linkcolor=blue,
	citecolor=blue,
	urlcolor=blue,
	pdftitle={The Geometric Formalism of T0 Quantum Mechanics and its Application to Quantum Computing}
\hypersetup{
	colorlinks=true,
	linkcolor=blue,
	citecolor=blue,
	urlcolor=blue,
	pdftitle={The Mass Scaling Exponent κ in T0 Theory}
\hypersetup{
	colorlinks=true,
	linkcolor=blue,
	citecolor=blue,
	urlcolor=blue,
	pdftitle={The Musical Spiral and 137: The Mathematical Discovery of Cosmic Detuning}
\hypersetup{
	colorlinks=true,
	linkcolor=blue,
	citecolor=blue,
	urlcolor=blue,
	pdftitle={The Relational Number System: Prime Numbers as Fundamental Ratios}
\hypersetup{
	colorlinks=true,
	linkcolor=blue,
	citecolor=blue,
	urlcolor=blue,
	pdftitle={The T0 Model (Planck-Referenced): A Reformulation of Physics}
\hypersetup{
	colorlinks=true,
	linkcolor=blue,
	citecolor=blue,
	urlcolor=blue,
	pdftitle={The T0 Model: Time-Energy Duality and Geometric Rest Mass}
\hypersetup{
	colorlinks=true,
	linkcolor=blue,
	citecolor=blue,
	urlcolor=blue,
	pdftitle={The T0-Model (Planck-Referenced): A Reformulation of Physics}
\hypersetup{
	colorlinks=true,
	linkcolor=blue,
	citecolor=blue,
	urlcolor=blue,
	pdftitle={Verbindungen zwischen dem Mizohata-Takeuchi-Gegenbeispiel und der T0-Zeit-Masse-Dualitätstheorie}
\hypersetup{
	colorlinks=true,
	linkcolor=blue,
	citecolor=blue,
	urlcolor=blue,
	pdftitle={Vereinfachte Dirac-Gleichung in der T0-Theorie: Feldknoten-Ansatz}
\hypersetup{
	colorlinks=true,
	linkcolor=blue,
	citecolor=blue,
	urlcolor=blue,
	pdftitle={Vereinfachte T0-Theorie: Elegante Lagrange-Dichte für Zeit-Masse-Dualität}
\hypersetup{
	colorlinks=true,
	linkcolor=blue,
	citecolor=blue,
	urlcolor=blue,
	pdftitle={Verhältnisbasiert vs. Absolut: Die Rolle der fraktalen Korrektur in der T0-Theorie}
\hypersetup{
	colorlinks=true,
	linkcolor=blue,
	citecolor=blue,
	urlcolor=blue,
	pdftitle={Vollständige Herleitung der Higgs-Masse und Wilson-Koeffizienten}
\hypersetup{
	colorlinks=true,
	linkcolor=blue,
	citecolor=blue,
	urlcolor=blue,
	pdftitle={Vollständiges Teilchenspektrum: Standard-Modell vs T0-Theorie}
\hypersetup{
	colorlinks=true,
	linkcolor=blue,
	citecolor=blue,
	urlcolor=blue,
	pdftitle={Warum Zahlenverhältnisse nicht direkt gekürzt werden dürfen}
\hypersetup{
	colorlinks=true,
	linkcolor=blue,
	citecolor=blue,
	urlcolor=blue,
	pdftitle={Why Numerical Ratios Must Not Be Directly Simplified}
\hypersetup{
	colorlinks=true,
	linkcolor=blue,
	citecolor=blue,
	urlcolor=blue,
}
\hypersetup{
	colorlinks=true,
	linkcolor=blue,
	citecolor=red,
	urlcolor=blue,
	bookmarks=true,
	bookmarksnumbered=true,
	pdfstartview=FitH,
	pdftitle={T0 Model - Field-Theoretic Derivation of the Beta Parameter}
\hypersetup{
	colorlinks=true,
	linkcolor=blue,
	citecolor=red,
	urlcolor=blue,
	bookmarks=true,
	bookmarksnumbered=true,
	pdfstartview=FitH,
	pdftitle={T0-Modell - Feldtheoretische Herleitung des Beta-Parameters}
\hypersetup{
	colorlinks=true,
	linkcolor=blue,
	filecolor=magenta,
	urlcolor=cyan,
}
\hypersetup{
	colorlinks=true,
	linkcolor=blue,
	urlcolor=blue,
	citecolor=blue,
	pdftitle={From Time Dilation to Mass Variation: Mathematical Core Formulations of Time-Mass Duality Theory - Updated Framework}
\hypersetup{
	colorlinks=true,
	linkcolor=blue,
	urlcolor=blue,
	citecolor=blue,
	pdftitle={T0 Model: Detailed Formula for Leptonic Anomalies}
\hypersetup{
	colorlinks=true,
	linkcolor=blue,
	urlcolor=blue,
	citecolor=blue,
	pdftitle={T0 Model: Detaillierte Formel für leptonische Anomalien}
\hypersetup{
	colorlinks=true,
	linkcolor=blue,
	urlcolor=blue,
	citecolor=blue,
	pdftitle={T0 Model: Energy-based Formulas with Quadratic Scaling}
\hypersetup{
	colorlinks=true,
	linkcolor=blue,
	urlcolor=blue,
	citecolor=blue,
	pdftitle={T0 Model: Granulation, Limits and Fundamental Asymmetry}
\hypersetup{
	colorlinks=true,
	linkcolor=blue,
	urlcolor=blue,
	citecolor=blue,
	pdftitle={T0-Modell: Energiebasierte Formeln mit quadratischer Skalierung}
\hypersetup{
	colorlinks=true,
	linkcolor=blue,
	urlcolor=blue,
	citecolor=blue,
	pdftitle={T0-Modell: Granulation, Limits und fundamentale Asymmetrie}
\hypersetup{
	colorlinks=true,
	linkcolor=blue,
	urlcolor=blue,
	citecolor=blue,
	pdftitle={Von Zeitdilatation zu Massenvariation: Mathematische Kernformulierungen der Zeit-Masse-Dualitätstheorie - Aktualisiertes Framework}
\hypersetup{
	colorlinks=true,
	linkcolor=t0blue,
	citecolor=t0blue,
	urlcolor=t0blue,
	pdftitle={T0 Model: Complete Theoretical Summary}
\hypersetup{
	colorlinks=true,
	linkcolor=t0blue,
	citecolor=t0blue,
	urlcolor=t0blue,
	pdftitle={T0 Theory: Resolution of Apparent Instantaneity}
\hypersetup{
	colorlinks=true,
	linkcolor=t0blue,
	citecolor=t0blue,
	urlcolor=t0blue,
	pdftitle={T0 vs Synergetics: Vereinfachung durch natürliche Einheiten}
\hypersetup{
	colorlinks=true,
	linkcolor=t0blue,
	citecolor=t0blue,
	urlcolor=t0blue,
	pdftitle={T0-Modell: Vollständige theoretische Zusammenfassung}
\hypersetup{
	colorlinks=true,
	linkcolor=t0blue,
	citecolor=t0blue,
	urlcolor=t0blue,
	pdftitle={T0-Theorie: Auflösung der scheinbaren Instantanität}
\hypersetup{
	colorlinks=true,
	linkcolor=t0blue,
	citecolor=t0blue,
	urlcolor=t0blue,
	pdftitle={T0-Theorie: Vollständige Dokumentenübersicht}
\hypersetup{
	colorlinks=true,
	linkcolor=t0blue,
	citecolor=t0blue,
	urlcolor=t0blue,
	pdftitle={T0-Theory: Complete Document Overview}
\hypersetup{
	colorlinks=true,
	linkcolor=t0blue,
	citecolor=t0blue,
	urlcolor=t0blue,
}
\hypersetup{
	colorlinks=true,
	linkcolor=t0blue,
	citecolor=t0green,
	urlcolor=t0blue,
	pdftitle={Das verborgene Geheimnis von 1/137}
\hypersetup{
	colorlinks=true,
	linkcolor=t0blue,
	citecolor=t0green,
	urlcolor=t0blue,
	pdftitle={The Hidden Secret of 1/137}
\hypersetup{
    colorlinks=true,
    linkcolor=blue,
    citecolor=blue,
    urlcolor=blue,
    pdftitle={Analyse und Implikationen des MNRAS-Papiers 544 für die T0-Theorie}
\hypersetup{
  colorlinks=true,
  linkcolor=blue,
  citecolor=blue,
  urlcolor=blue
}
\hypersetup{
  colorlinks=true,
  linkcolor=blue,
  citecolor=blue,
  urlcolor=blue,
  pdftitle={T0-Theorie: Ein-Uhr-Metrologie und Drei-Uhren-Experiment}
\hypersetup{
  colorlinks=true,
  linkcolor=blue,
  citecolor=blue,
  urlcolor=blue,
  pdftitle={T0-Theory: Single-Clock Metrology and Three-Clock Experiment}
\hypersetup{
colorlinks=true,
linkcolor=blue,
citecolor=blue,
urlcolor=blue,
pdftitle={Quantenmechanik im T0-Modell: Feldtheoretische Grundlagen}
\hypersetup{
colorlinks=true,
linkcolor=blue,
citecolor=blue,
urlcolor=blue,
pdftitle={T0-Theory: Neutrinos}
\newcommand{\Bzero}{B_0}
\newcommand{\CQCD}{C_{\text{QCD}
\newcommand{\Cconv}{C_{\text{conv}
\newcommand{\Cto}{C_{\text{T0}
\newcommand{\Czero}{C_0}
\newcommand{\DTmu}{D_{T,\mu}
\newcommand{\DcovT}[1]{\partial_\mu #1 + #1 \partial_\mu \Tfield}
\newcommand{\Dfrak}{D_f}
\newcommand{\Df}{D_f}
\newcommand{\DhiggsT}{\Tfield (\partial_\mu + ig A_\mu) \Phi + \Phi \partial_\mu \Tfield}
\newcommand{\EPlanck}{E_P}
\newcommand{\EPlanck}{E_{\text{Pl}
\newcommand{\EPratio}[1]{\frac{#1}
\newcommand{\EP}{E_P}
\newcommand{\EP}{E_{\text{P}
\newcommand{\EW}{E_W}
\newcommand{\EZ}{E_Z}
\newcommand{\Echar}{E_{\text{char}
\newcommand{\Ee}{E_e}
\newcommand{\Efield}{E(x,t)}
\newcommand{\Efield}{E_\text{field}
\newcommand{\Efield}{E_{\text{Feld}
\newcommand{\Efield}{E_{\text{Field}
\newcommand{\Efield}{E_{\text{field}
\newcommand{\Efield}{E}
\newcommand{\Egamma}{E_\gamma}
\newcommand{\Eh}{E_h}
\newcommand{\Emu}{E_\mu}
\newcommand{\Enorm}[1]{E_{\text{norm}
\newcommand{\En}{E_n}
\newcommand{\Ep}{E_p}
\newcommand{\Eratio}[2]{\frac{E_{#1}
\newcommand{\Etau}{E_\tau}
\newcommand{\Evis}{E_{\text{vis}
\newcommand{\Exi}{E_\xi}
\newcommand{\Ezero}{E_0}
\newcommand{\GeV}{\,\text{GeV}
\newcommand{\Gnat}{G_{\text{nat}
\newcommand{\Gsi}{G_{\text{SI}
\newcommand{\Hubble}{H_0}
\newcommand{\Kfrak}{K_{\text{frac}
\newcommand{\Kfrak}{K_{\text{frak}
\newcommand{\Kspec}{K_{\text{spec}
\newcommand{\LCDM}{\Lambda\text{CDM}
\newcommand{\LPlanck}{\ell_{\text{Pl}
\newcommand{\Lag}{\mathcal{L}
\newcommand{\Lambdat}{\Lambda_T}
\newcommand{\Leff}{L_{\text{eff}
\newcommand{\Lorentz}[2]{{\Lambda^\mu{}
\newcommand{\Lp}{L_{\text{P}
\newcommand{\Lxi}{L_\xi}
\newcommand{\Lzero}{L_0}
\newcommand{\MPl}{M_{\text{Pl}
\newcommand{\MSbar}{\overline{\text{MS}
\newcommand{\MeV}{\,\text{MeV}
\newcommand{\Mpl}{M_{\text{Pl}
\newcommand{\OmegaDM}{\Omega_{\text{DM}
\newcommand{\OmegaLambda}{\Omega_{\Lambda}
\newcommand{\Omegab}{\Omega_b}
\newcommand{\Phiphoton}{\Phi_{\text{photon}
\newcommand{\Ricci}{R_{\mu\nu}
\newcommand{\Riem}{R^\rho{}
\newcommand{\Rzero}{R_\infty}
\newcommand{\Scal}{R}
\newcommand{\SynchPower}{P_{\text{synch}
\newcommand{\TPlanck}{t_{\text{Pl}
\newcommand{\Tfieldt}{T(\vec{x}
\newcommand{\Tfieldt}{T(x,t)}
\newcommand{\Tfield}{T(x)}
\newcommand{\Tfield}{T(x,t)}
\newcommand{\Tfield}{T_{\text{field}
\newcommand{\Tfield}{T}
\newcommand{\Tfield}{\mathcal{T}
\newcommand{\Tzerot}{T_0(\Tfield)}
\newcommand{\Tzero}{T_0}
\newcommand{\Weyl}{C^\rho{}
\newcommand{\ZPinch}{J \times B = \nabla p}
\newcommand{\aleph}{\aleph}
\newcommand{\alphaEMSI}{\alpha_{\text{EM,SI}
\newcommand{\alphaEMnat}{\alpha_{\text{EM,nat}
\newcommand{\alphaEM}{\alpha_{\text{EM}
\newcommand{\alphaEM}{\ensuremath{\alpha_{\text{EM}
\newcommand{\alphaQCD}{\alpha_s}
\newcommand{\alphaQED}{\alpha_{\text{QED}
\newcommand{\alphaSI}{\alpha_{\text{SI}
\newcommand{\alphaT}{\alpha_{\text{T}
\newcommand{\alphaWSI}{\alpha_{\text{W,SI}
\newcommand{\alphaWnat}{\alpha_{\text{W,nat}
\newcommand{\alphaW}{\alpha_{\text{W}
\newcommand{\alphaem}{\alpha_{EM}
\newcommand{\alphaem}{\alpha}
\newcommand{\alphafine}{\alpha}
\newcommand{\alphagem}{\alpha}
\newcommand{\alphanat}{\alpha_{\text{nat}
\newcommand{\alphapar}{\alpha}
\newcommand{\betaTSI}{\beta_{\text{T,SI}
\newcommand{\betaTnat}{\beta_{\text{T,nat}
\newcommand{\betaT}{\beta_T}
\newcommand{\betaT}{\beta_{T}
\newcommand{\betaT}{\beta_{\text{T}
\newcommand{\betaT}{\ensuremath{\beta_T}
\newcommand{\betapar}{\beta}
\newcommand{\calL}{\mathcal{L}
\newcommand{\checked}{\checkmark}
\newcommand{\checkmarkx}{\checkmark}
\newcommand{\dTdt}{\frac{d\Tfieldt}
\newcommand{\deltaE}{\delta E}
\newcommand{\deltafield}{\ensuremath{\delta m}
\newcommand{\deltam}{\delta m}
\newcommand{\deq}{\displaystyle}
\newcommand{\docref}[1]{\texttt{#1}
\newcommand{\eV}{\,\text{eV}
\newcommand{\epsilonT}{\varepsilon_T}
\newcommand{\epsilonzero}{\varepsilon_0}
\newcommand{\etavis}{\eta_{\text{visual}
\newcommand{\e}{\mathrm{e}
\newcommand{\gW}{g_W}
\newcommand{\gammaf}{\gamma_{\text{Lorentz}
\newcommand{\gammamu}{\gamma^\mu}
\newcommand{\gs}{g_s}
\newcommand{\inftytext}{$\infty$}
\newcommand{\interval}[2]{#1:#2}
\newcommand{\kfrac}{K_{\text{frak}
\newcommand{\lP}{\ell_{\text{P}
\newcommand{\lP}{l_P}
\newcommand{\lambdah}{\ensuremath{\lambda_h}
\newcommand{\lambdah}{\lambda_h}
\newcommand{\lambdazero}{\lambda_0}
\newcommand{\mP}{m_{\text{P}
\newcommand{\mfield}{m(x,t)}
\newcommand{\mfield}{m}
\newcommand{\mh}{m_h}
\newcommand{\micrometer}{\ensuremath{\mu}
\newcommand{\mikrometer}{\ensuremath{\mu}
\newcommand{\myRightarrow}{\ensuremath{\Rightarrow}
\newcommand{\myapprox}{\ensuremath{\approx}
\newcommand{\myomega}{\ensuremath{\omega}
\newcommand{\myphi}{\ensuremath{\phi}
\newcommand{\mypi}{\ensuremath{\pi}
\newcommand{\mypropto}{\ensuremath{\propto}
\newcommand{\myrightarrow}{\ensuremath{\rightarrow}
\newcommand{\mysim}{\ensuremath{\sim}
\newcommand{\mysqrt}{\ensuremath{\sqrt}
\newcommand{\mytimes}{\ensuremath{\times}
\newcommand{\natunits}{\hbar = c = G = k_B = 1}
\newcommand{\natunits}{\text{(nat. Einh.)}
\newcommand{\natunits}{\text{(nat. units)}
\newcommand{\nulep}{\nu}
\newcommand{\nuzero}{\nu_0}
\newcommand{\partialop}{\ensuremath{\partial}
\newcommand{\pdTdt}{\frac{\partial\Tfieldt}
\newcommand{\pdTdx}{\nabla\Tfieldt}
\newcommand{\phiT}{\phi}
\newcommand{\pichar}{\pi}
\newcommand{\primrel}[1]{\mathbf{#1}
\newcommand{\rhoCMB}{\rho_{\text{CMB}
\newcommand{\rhoCasimir}{\rho_{\text{Casimir}
\newcommand{\rhoE}{\rho_E}
\newcommand{\rhofield}{\ensuremath{\rho}
\newcommand{\rzero}{r_0}
\newcommand{\slashk}{\cancel{k}
\newcommand{\slashp}{\cancel{p}
\newcommand{\slashq}{\cancel{q}
\newcommand{\tP}{t_P}
\newcommand{\tP}{t_{\text{P}
\newcommand{\tablescale}{0.9}
\newcommand{\tzero}{t_0}
\newcommand{\vect}[1]{\boldsymbol{#1}
\newcommand{\vecx}{\vec{x}
\newcommand{\vh}{v}
\newcommand{\vr}{\vec{r}
\newcommand{\warningx}{\color{red}
\newcommand{\warningx}{\textbf{!}
\newcommand{\warningx}{{\color{red}
\newcommand{\xiT}{\xi}
\newcommand{\xiconst}{\xi = \frac{4}
\newcommand{\xicoupling}{f(E/\Exi)}
\newcommand{\xigeom}{\xi_{\text{geom}
\newcommand{\xigeom}{\xi}
\newcommand{\xikonst}{\xi = \frac{4}
\newcommand{\xiparticle}{\xi_{\text{particle}
\newcommand{\xipar}{\ensuremath{\xi}
\newcommand{\xipar}{\xi_0}
\newcommand{\xipar}{\xi}
\newcommand{\xirat}{\xi_{\text{ratio}
\newtheorem{axiom}{Axiom}
\newtheorem{category}{Category-Theoretic Basis}
\newtheorem{category}{Kategorientheoretische Basis}
\newtheorem{corollary}[theorem]{Corollary}
\newtheorem{corollary}[theorem]{Korollar}
\newtheorem{corollary}{Corollary}
\newtheorem{corollary}{Korollar}
\newtheorem{definition}[theorem]{Definition}
\newtheorem{definition}{Definition}
\newtheorem{discovery}{Discovery}
\newtheorem{discovery}{Neue Entdeckung}
\newtheorem{discovery}{New Discovery}
\newtheorem{discovery}{Revolutionary Discovery}
\newtheorem{entdeckung}{Entdeckung}
\newtheorem{entdeckung}{Revolutionäre Entdeckung}
\newtheorem{erkenntnis}{Erkenntnis}
\newtheorem{erkenntnis}{Schlüsselerkenntnis}
\newtheorem{example}[theorem]{Beispiel}
\newtheorem{example}[theorem]{Example}
\newtheorem{example}{Beispiel}
\newtheorem{example}{Example}
\newtheorem{insight}{Central Insight}
\newtheorem{insight}{Insight}
\newtheorem{insight}{Key Insight}
\newtheorem{insight}{Wichtige Einsicht}
\newtheorem{insight}{Zentrale Einsicht}
\newtheorem{lemma}[theorem]{Lemma}
\newtheorem{lemma}{Lemma}
\newtheorem{principle}{Fundamental Principle}
\newtheorem{principle}{Fundamentales Prinzip}
\newtheorem{principle}{Grundlegendes Prinzip}
\newtheorem{principle}{Principle}
\newtheorem{principle}{Prinzip}
\newtheorem{prinzip}{Grundprinzip}
\newtheorem{proof_step}{Beweisschritt}
\newtheorem{proof_step}{Proof Step}
\newtheorem{proposition}[theorem]{Proposition}
\newtheorem{proposition}{Proposition}
\newtheorem{remark}[theorem]{Bemerkung}
\newtheorem{remark}[theorem]{Remark}
\newtheorem{theorem}{Theorem}
\newtheorem{warning}[theorem]{Warning}
\newtheorem{warning}[theorem]{Warnung}
\newunicodechar{±}{\ensuremath{\pm}
\newunicodechar{×}{\ensuremath{\times}
\newunicodechar{÷}{\ensuremath{\div}
\newunicodechar{ħ}{\ensuremath{\hbar}
\newunicodechar{Α}{\ensuremath{A}
\newunicodechar{Β}{\ensuremath{B}
\newunicodechar{Γ}{\ensuremath{\Gamma}
\newunicodechar{Δ}{\ensuremath{\Delta}
\newunicodechar{Ε}{\ensuremath{E}
\newunicodechar{Ζ}{\ensuremath{Z}
\newunicodechar{Η}{\ensuremath{H}
\newunicodechar{Θ}{\ensuremath{\Theta}
\newunicodechar{Ι}{\ensuremath{I}
\newunicodechar{Κ}{\ensuremath{K}
\newunicodechar{Λ}{\ensuremath{\Lambda}
\newunicodechar{Μ}{\ensuremath{M}
\newunicodechar{Ν}{\ensuremath{N}
\newunicodechar{Ξ}{\ensuremath{\Xi}
\newunicodechar{Ο}{\ensuremath{O}
\newunicodechar{Π}{\ensuremath{\Pi}
\newunicodechar{Ρ}{\ensuremath{P}
\newunicodechar{Σ}{\ensuremath{\Sigma}
\newunicodechar{Τ}{\ensuremath{T}
\newunicodechar{Υ}{\ensuremath{\Upsilon}
\newunicodechar{Φ}{\ensuremath{\Phi}
\newunicodechar{Χ}{\ensuremath{X}
\newunicodechar{Ψ}{\ensuremath{\Psi}
\newunicodechar{Ω}{\ensuremath{\Omega}
\newunicodechar{α}{\ensuremath{\alpha}
\newunicodechar{β}{\ensuremath{\beta}
\newunicodechar{γ}{\ensuremath{\gamma}
\newunicodechar{δ}{\ensuremath{\delta}
\newunicodechar{ε}{\ensuremath{\varepsilon}
\newunicodechar{ζ}{\ensuremath{\zeta}
\newunicodechar{η}{\ensuremath{\eta}
\newunicodechar{θ}{\ensuremath{\theta}
\newunicodechar{ι}{\ensuremath{\iota}
\newunicodechar{κ}{\ensuremath{\kappa}
\newunicodechar{λ}{\ensuremath{\lambda}
\newunicodechar{μ}{\ensuremath{\mu}
\newunicodechar{ν}{\ensuremath{\nu}
\newunicodechar{ξ}{\ensuremath{\xi}
\newunicodechar{ο}{\ensuremath{o}
\newunicodechar{π}{\ensuremath{\pi}
\newunicodechar{ρ}{\ensuremath{\rho}
\newunicodechar{σ}{\ensuremath{\sigma}
\newunicodechar{τ}{\ensuremath{\tau}
\newunicodechar{υ}{\ensuremath{\upsilon}
\newunicodechar{φ}{\ensuremath{\phi}
\newunicodechar{φ}{\ensuremath{\varphi}
\newunicodechar{χ}{\ensuremath{\chi}
\newunicodechar{ψ}{\ensuremath{\psi}
\newunicodechar{ω}{\ensuremath{\omega}
\newunicodechar{←}{\ensuremath{\leftarrow}
\newunicodechar{→}{\ensuremath{\rightarrow}
\newunicodechar{↔}{\ensuremath{\leftrightarrow}
\newunicodechar{⇐}{\ensuremath{\Leftarrow}
\newunicodechar{⇒}{\ensuremath{\Rightarrow}
\newunicodechar{⇔}{\ensuremath{\Leftrightarrow}
\newunicodechar{∂}{\ensuremath{\partial}
\newunicodechar{∅}{\ensuremath{\emptyset}
\newunicodechar{∇}{\ensuremath{\nabla}
\newunicodechar{∈}{\ensuremath{\in}
\newunicodechar{∉}{\ensuremath{\notin}
\newunicodechar{∏}{\ensuremath{\prod}
\newunicodechar{∑}{\ensuremath{\sum}
\newunicodechar{√}{\ensuremath{\sqrt}
\newunicodechar{∝}{\ensuremath{\propto}
\newunicodechar{∞}{\ensuremath{\infty}
\newunicodechar{∩}{\ensuremath{\cap}
\newunicodechar{∪}{\ensuremath{\cup}
\newunicodechar{∫}{\ensuremath{\int}
\newunicodechar{≈}{\ensuremath{\approx}
\newunicodechar{≠}{\ensuremath{\neq}
\newunicodechar{≤}{\ensuremath{\leq}
\newunicodechar{≥}{\ensuremath{\geq}
\newunicodechar{★}{\ensuremath{\star}
\newunicodechar{✓}{\checkmark}
\pgfplotsset{compat=1.17}
\pgfplotsset{compat=1.18}
\renewcommand{\cftchapfont}{\large\bfseries\color{blue}
\renewcommand{\cftchappagefont}{\large\bfseries\color{blue}
\renewcommand{\cftsecfont}{\bfseries}
\renewcommand{\cftsecfont}{\color{blue}
\renewcommand{\cftsecfont}{\large\bfseries\color{blue}
\renewcommand{\cftsecpagefont}{\bfseries}
\renewcommand{\cftsecpagefont}{\color{blue}
\renewcommand{\cftsecpagefont}{\large\bfseries\color{blue}
\renewcommand{\cftsubsecfont}{\color{blue!80!black}
\renewcommand{\cftsubsecfont}{\color{blue}
\renewcommand{\cftsubsecpagefont}{\color{blue!80!black}
\renewcommand{\cftsubsecpagefont}{\color{blue}
\renewcommand{\cftsubsubsecfont}{\color{blue!60!black}
\renewcommand{\cftsubsubsecfont}{\color{blue}
\renewcommand{\cftsubsubsecpagefont}{\color{blue!60!black}
\renewcommand{\cftsubsubsecpagefont}{\color{blue}
\renewcommand{\cfttoctitlefont}{\huge\bfseries\color{blue}
\renewcommand{\cfttoctitlefont}{\huge\bfseries}
\renewcommand{\familydefault}{\sfdefault}
\renewcommand{\footrulewidth}{0.4pt}
\renewcommand{\headrulewidth}{0.4pt}
\sisetup{locale = DE, group-separator = {.}
\sisetup{locale = DE}
\usetikzlibrary{arrows.meta,positioning,shapes.geometric}
\usetikzlibrary{decorations.pathmorphing, patterns, shapes.arrows}
\usetikzlibrary{intersections}
\usetikzlibrary{positioning, arrows.meta}
\usetikzlibrary{positioning, arrows}
\usetikzlibrary{positioning, shapes.geometric, arrows.meta}
\usetikzlibrary{positioning,shapes,arrows}

% Common settings
\setlength{\headheight}{15pt}
\pgfplotsset{compat=1.18}
\usetikzlibrary{positioning,shapes,arrows,arrows.meta}

% Hyperref setup
\hypersetup{
    colorlinks=true,
    linkcolor=blue,
    citecolor=blue,
    urlcolor=blue
}


\title{NatEinheitenSystematikDe}
\author{Johann Pascher}
\date{\today}

\begin{document}

\maketitle
\tableofcontents

\begin{abstract}
		Dieses grundlegende Dokument etabliert das natürliche Einheitensystem, das im gesamten T0-Modell-Framework verwendet wird. Durch Setzen fundamentaler Konstanten auf Eins und Annahme von Energie als Basisdimension können alle physikalischen Größen als Potenzen der Energie ausgedrückt werden. Dieses Dokument dient als Referenz für Einheitenumwandlungen und Dimensionsanalyse über alle T0-Modell-Anwendungen hinweg.
	\end{abstract}
	
	\tableofcontents
	\newpage
	
	# Liste der Symbole und Notation
	
	{\small
		\begin{table}[htbp]
			\centering
			\begin{adjustbox}{width=0.98\textwidth}
				\begin{tabular}{lll}
					\toprule
					\textbf{Symbol} & \textbf{Bedeutung} & \textbf{Einheiten/Notizen} \\
					\midrule
					\multicolumn{3}{c}{\textbf{Fundamentale Konstanten}} \\
					$\hbar$ & Reduzierte Planck-Konstante & Auf 1 gesetzt \\
					$c$ & Lichtgeschwindigkeit & Auf 1 gesetzt \\
					$G$ & Gravitationskonstante & Auf 1 gesetzt \\
					$k_B$ & Boltzmann-Konstante & Auf 1 gesetzt \\
					$e$ & Elementarladung & $[E^0]$ (dimensionslos) \\
					$\varepsilon_0, \mu_0$ & Vakuum-Permittivität, -Permeabilität & In QED-Einheiten auf 1 gesetzt \\
					\midrule
					\multicolumn{3}{c}{\textbf{Einheiten}} \\
					$l_P, t_P, m_P, E_P, T_P$ & Planck-Länge, -Zeit, -Masse, -Energie, -Temp. & Natürliche Basiseinheiten \\
					$m_e, a_0, E_h$ & Elektronmasse, Bohr-Radius, Hartree-Energie & Atomare Einheiten \\
					\midrule
					\multicolumn{3}{c}{\textbf{Kopplungskonstanten}} \\
					$\alpha_{\text{EM}}$ & Feinstrukturkonstante & $e^2/(4\pi) = 1$ (nat.), $\approx 1/137$ (SI) \\
					$\alpha_s, \alpha_W, \alpha_G$ & Starke, schwache, Gravitations-Kopplung & Dimensionslos \\
					\midrule
					\multicolumn{3}{c}{\textbf{Physikalische Größen}} \\
					$E, m, \Theta$ & Energie, Masse, Temperatur & $[E]$ \\
					$L, r, \lambda, t$ & Länge, Radius, Wellenlänge, Zeit & $[E^{-1}]$ \\
					$p, \omega, \nu$ & Impuls, Kreisfrequenz, Frequenz & $[E]$ \\
					$F$ & Kraft & $[E^2]$ \\
					$v$ & Geschwindigkeit & Dimensionslos \\
					$q$ & Elektrische Ladung & $[E^0]$ (dimensionslos) \\
					\midrule
					\multicolumn{3}{c}{\textbf{Spezielle Skalen \& Notation}} \\
					$r_0, \xi$ & T0-Länge, Skalierungsparameter & $\xi l_P, \xi \approx 1.33 \times 10^{-4}$ \\
					$\lambda_{C,e}, r_e$ & Compton-Wellenlänge, klassischer e-Radius & $\hbar/(m_e c), e^2/(4\pi\varepsilon_0 m_e c^2)$ \\
					$[X], [E^n]$ & Dimension von X, Energiedimension & Dimensionsanalyse \\
					$\sim, \leftrightarrow$ & Ungefähr, Umwandlung & Größenordnung, Einheiten \\
					\bottomrule
				\end{tabular}
			\end{adjustbox}
			\caption{Symbole und Notation}
			\label{tab:symbole}
		\end{table}
	}
	
	\newpage
	
	# Einleitung
	
	Natürliche Einheiten sind Einheitensysteme, in denen fundamentale physikalische Konstanten auf Eins gesetzt werden, um Berechnungen zu vereinfachen und die zugrundeliegende mathematische Struktur physikalischer Gesetze zu offenbaren. Die bekanntesten Systeme sind \textbf{Planck-Einheiten} (für Gravitation und Quantenphysik) und \textbf{atomare Einheiten} (für Quantenchemie).
	
	Dieses Dokument etabliert das vollständige Framework für das natürliche Einheitensystem, das im T0-Modell verwendet wird, welches auf Planck-Einheiten mit Energie als fundamentaler Dimension basiert. Die Schlüsselerkenntnis ist, dass Energie $[E]$ als universelle Dimension dient, aus der alle anderen physikalischen Größen abgeleitet werden.
	
	## Vergleich mit anderen natürlichen Einheitensystemen
	
	\begin{table}[htbp]
		\centering
		\begin{adjustbox}{width=0.95\textwidth}
			\begin{tabular}{lllll}
				\toprule
				\textbf{System} & \textbf{Konstanten = 1} & \textbf{Basiseinheiten} & \textbf{Anwendungen} & \textbf{Notizen} \\
				\midrule
				Planck-Einheiten & $\hbar, c, G, k_B = 1$ & $l_P, t_P, m_P, E_P$ & Quantengravitation, Kosmologie & Universelle Bedeutung \\
				Atomare Einheiten & $m_e, e, \hbar, \frac{1}{4\pi\varepsilon_0} = 1$ & $a_0, E_h$ & Quantenchemie, Atome & Chemieanwendungen \\
				Teilchenphysik & $\hbar, c = 1$ & GeV & Hochenergiephysik & Praktisch für Collider \\
				T0-Modell & $\hbar, c, G, k_B = 1$ & Energie $[E]$ & Vereinheitlichte Physik & Energie als Basisdimension \\
				\bottomrule
			\end{tabular}
		\end{adjustbox}
		\caption{Vergleich natürlicher Einheitensysteme}
		\label{tab:einheitensysteme}
	\end{table}
	
	# Grundlagen natürlicher Einheitensysteme
	
	## Planck-Einheiten
	
	Die Planck-Einheiten wurden 1899 von Max Planck vorgeschlagen \cite{planck1900,planck1906} und basieren auf den fundamentalen Naturkonstanten:
	
```math-align

		G &= 1 \quad \text{(Gravitationskonstante)} \\
		c &= 1 \quad \text{(Lichtgeschwindigkeit)} \\
		\hbar &= 1 \quad \text{(reduzierte Planck-Konstante)}
	
```

	
	Planck erkannte, dass diese Einheiten \textit{ihre Bedeutung für alle Zeiten und für alle, einschließlich außerirdischer und nicht-menschlicher Kulturen notwendigerweise behalten} \cite{planck1900}.
	
	## Atomare Einheiten
	
	Die atomaren Einheiten, 1927 von Hartree eingeführt \cite{hartree1957}, setzen:
	
```math-align

		m_e &= 1 \quad \text{(Elektronmasse)} \\
		e &= 1 \quad \text{(Elementarladung)} \\
		\hbar &= 1 \\
		\frac{1}{4\pi\varepsilon_0} &= 1 \quad \text{(Coulomb-Konstante)}
	
```

	
	## Quantenoptische Einheiten
	
	Für Quantenfeldtheorie-Anwendungen werden häufig quantenoptische Einheiten verwendet:
	
```math-align

		c &= 1 \quad \text{(Lichtgeschwindigkeit)} \\
		\hbar &= 1 \quad \text{(reduzierte Planck-Konstante)} \\
		\varepsilon_0 &= 1 \quad \text{(Permittivität)} \\
		\mu_0 &= 1 \quad \text{(Permeabilität, da } c = 1/\sqrt{\varepsilon_0 \mu_0}\text{)}
	
```

	
	## Vorteile natürlicher Einheiten
	
	Natürliche Einheiten bieten mehrere Schlüsselvorteile:
	
		- \textbf{Vereinfachte Gleichungen} (z.B. $E = m$ statt $E = mc^2$)
		- \textbf{Keine überflüssigen Konstanten} in Berechnungen
		- \textbf{Universelle Skalierung} für fundamentale Physik
		- \textbf{Offenbaren fundamentaler Beziehungen} zwischen physikalischen Größen
		- \textbf{Bieten Dimensionskonsistenz-Prüfungen}
		- \textbf{Eliminieren willkürliche Umwandlungsfaktoren}
		- \textbf{Heben die universelle Rolle der Energie hervor}
	
	
	# Mathematischer Beweis der Energieäquivalenz
	
	## Fundamentale dimensionale Beziehungen
	
	In natürlichen Einheiten haben alle physikalischen Größen Dimensionen, die als Potenzen der Energie $[E]$ ausgedrückt werden können \cite{weinberg1995,peskin1995}:
	
	
```math-align

		[L] &= [E]^{-1} \quad \text{(aus } \hbar c = 1\text{)} \\
		[T] &= [E]^{-1} \quad \text{(aus } \hbar = 1\text{)} \\
		[M] &= [E] \quad \text{(aus } c = 1\text{)}
	
```

	
	## Umwandlung fundamentaler Größen
	
	\textbf{Länge:} Aus der Beziehung $\hbar c = 1$ folgt:
	
```math-equation

		[L] = \frac{[\hbar][c]}{[E]} = [E]^{-1}
	
```

	
	\textbf{Zeit:} Aus $\hbar = 1$ und $E = \hbar \omega$ folgt:
	
```math-equation

		[T] = \frac{[\hbar]}{[E]} = [E]^{-1}
	
```

	
	\textbf{Masse:} Aus $E = mc^2$ und $c = 1$ folgt:
	
```math-equation

		[M] = [E]
	
```

	
	\textbf{Geschwindigkeit:} 
	
```math-equation

		[v] = \frac{[L]}{[T]} = \frac{[E]^{-1}}{[E]^{-1}} = [E]^0 = \text{dimensionslos}
	
```

	
	\textbf{Impuls:}
	
```math-equation

		[p] = [M][v] = [E] \cdot [E]^0 = [E]
	
```

	
	\textbf{Kraft:}
	
```math-equation

		[F] = [M][a] = [E] \cdot [E]^{-1} = [E]^2
	
```

	
	\textbf{Ladung:} In Planck-Einheiten aus $F = \frac{1}{4\pi\varepsilon_0} \frac{q^2}{r^2}$:
	
```math-equation

		[q] = [E]^{1/2}
	
```

	
	## Verallgemeinerung
	
	Jede physikalische Größe $G$ kann als Produkt von Potenzen der fundamentalen Konstanten dargestellt werden:
	
```math-equation

		G = c^a \cdot \hbar^b \cdot G^c \cdot k_B^d \cdot \ldots
	
```

	
	In natürlichen Einheiten wird dies zu:
	
```math-equation

		[G] = [E]^n \quad \text{für ein spezifisches } n \in \mathbb{Q}
	
```

	
	\begin{table}[htbp]
		\centering
		\begin{adjustbox}{width=0.9\textwidth}
			\begin{tabular}{lccc}
				\toprule
				\textbf{Physikalische Größe} & \textbf{SI-Dimension} & \textbf{Natürliche Dimension} & \textbf{Herleitung} \\
				\midrule
				Energie & $[ML^2T^{-2}]$ & $[E]$ & Basisdimension \\
				Masse & $[M]$ & $[E]$ & $E = mc^2, c = 1$ \\
				Temperatur & $[\Theta]$ & $[E]$ & $E = k_BT, k_B = 1$ \\
				Länge & $[L]$ & $[E^{-1}]$ & $l_P = \sqrt{\hbar G/c^3} = 1$ \\
				Zeit & $[T]$ & $[E^{-1}]$ & $t_P = \sqrt{\hbar G/c^5} = 1$ \\
				Impuls & $[MLT^{-1}]$ & $[E]$ & $p = mv, v = [E^0]$ \\
				Kraft & $[MLT^{-2}]$ & $[E^2]$ & $F = ma = [E][E] = [E^2]$ \\
				Leistung & $[ML^2T^{-3}]$ & $[E^2]$ & $P = E/t = [E]/[E^{-1}] = [E^2]$ \\
				Ladung & $[AT]$ & $[E^0]$ & Dimensionslos in Planck-Einheiten \\
				Elektrisches Feld & $[MLT^{-3}A^{-1}]$ & $[E^2]$ & $\vec{E} = \vec{F}/q$ \\
				Magnetisches Feld & $[MT^{-2}A^{-1}]$ & $[E^2]$ & $\vec{B} = \vec{F}/(qv)$ \\
				\bottomrule
			\end{tabular}
		\end{adjustbox}
		\caption{Universelle Energiedimensionen physikalischer Größen}
		\label{tab:energiedimensionen}
	\end{table}
	
	## Fundamentale Beziehungen
	
	Die Schlüsselbeziehungen in natürlichen Einheiten werden zu:
	
```math-align

		E &= m \quad \text{(Masse-Energie-Äquivalenz)} \\
		E &= T \quad \text{(Temperatur-Energie-Äquivalenz)} \\
		[L] &= [T] = [E^{-1}] \quad \text{(Raum-Zeit-Einheit)} \\
		\omega &= E \quad \text{(Frequenz-Energie-Äquivalenz)} \\
		p &= E \quad \text{(Impuls-Energie-Äquivalenz für masselose Teilchen)}
	
```

	
	# Längenskala-Hierarchie
	
	## Standard-Längenskalen
	
	Physikalische Systeme organisieren sich um charakteristische Längenskalen:
	
	\begin{table}[htbp]
		\centering
		\begin{adjustbox}{width=0.95\textwidth}
			\begin{tabular}{lccc}
				\toprule
				\textbf{Skala} & \textbf{Symbol} & \textbf{SI-Wert (m)} & \textbf{Natürliche Einheiten ($l_P = 1$)} \\
				\midrule
				Planck-Länge & $l_P$ & $1.616 \times 10^{-35}$ & $1$ \\
				Compton (Elektron) & $\lambda_{C,e}$ & $2.426 \times 10^{-12}$ & $1.5 \times 10^{23}$ \\
				Klassischer Elektronradius & $r_e$ & $2.818 \times 10^{-15}$ & $1.7 \times 10^{20}$ \\
				Bohr-Radius & $a_0$ & $5.292 \times 10^{-11}$ & $3.3 \times 10^{24}$ \\
				Kernskala & $\sim 10^{-15}$ & $10^{-15}$ & $6.2 \times 10^{19}$ \\
				Atomare Skala & $\sim 10^{-10}$ & $10^{-10}$ & $6.2 \times 10^{24}$ \\
				Menschliche Skala & $\sim 1$ & $1$ & $6.2 \times 10^{34}$ \\
				Erdradius & $R_\oplus$ & $6.371 \times 10^6$ & $3.9 \times 10^{41}$ \\
				Sonnensystem & $\sim 10^{12}$ & $10^{12}$ & $6.2 \times 10^{46}$ \\
				Galaktische Skala & $\sim 10^{21}$ & $10^{21}$ & $6.2 \times 10^{55}$ \\
				\bottomrule
			\end{tabular}
		\end{adjustbox}
		\caption{Standard-Längenskalen in natürlichen Einheiten}
		\label{tab:laengenskalen}
	\end{table}
	
	## Die T0-Längenskala
	
	Das T0-Modell führt eine sub-Plancksche Längenskala ein:
	
	\begin{definition}[T0-Länge]
		
```math-equation

			r_0 = \xi \cdot l_P
		
```

		wobei $\xi \approx 1.33 \times 10^{-4}$ ein dimensionsloser Parameter ist.
	\end{definition}
	
	Dies ergibt:
	
```math-align

		r_0 &= \xi \cdot l_P = 1.33 \times 10^{-4} \times 1.616 \times 10^{-35}\,\text{m} \\
		&= 2.15 \times 10^{-39}\,\text{m}
	
```

	
	In natürlichen Einheiten mit $l_P = 1$:
	
```math-equation

		r_0 = \xi \approx 1.33 \times 10^{-4}
	
```

	
	# Einheitenumwandlungen
	
	## Energie als Referenz
	
	Verwendung des Elektronvolts (eV) als praktische Energieeinheit:
	
	\begin{table}[htbp]
		\centering
		\begin{adjustbox}{width=0.9\textwidth}
			\begin{tabular}{lll}
				\toprule
				\textbf{Physikalische Größe} & \textbf{Umwandlung zu SI} & \textbf{Beispiel (1 GeV)} \\
				\midrule
				Energie & $\SI{1}{\electronvolt} = \SI{1.602e-19}{\joule}$ & $\SI{1.602e-10}{\joule}$ \\
				Masse & $E(\text{eV}) \times \SI{1.783e-36}{\kilogram\per\electronvolt}$ & $\SI{1.783e-27}{\kilogram}$ \\
				Länge & $E(\text{eV})^{-1} \times \SI{1.973e-7}{\meter\electronvolt}$ & $\SI{1.973e-16}{\meter}$ \\
				Zeit & $E(\text{eV})^{-1} \times \SI{6.582e-16}{\second\electronvolt}$ & $\SI{6.582e-25}{\second}$ \\
				Temperatur & $E(\text{eV}) \times \SI{1.161e4}{\kelvin\per\electronvolt}$ & $\SI{1.161e13}{\kelvin}$ \\
				\bottomrule
			\end{tabular}
		\end{adjustbox}
		\caption{Umwandlungsfaktoren von natürlichen zu SI-Einheiten}
		\label{tab:umwandlungen}
	\end{table}
	
	## Planck-Skala-Umwandlungen
	
	Umwandlung zwischen Planck-Einheiten und SI:
	
	\begin{table}[htbp]
		\centering
		\begin{adjustbox}{width=0.8\textwidth}
			\begin{tabular}{lll}
				\toprule
				\textbf{Planck-Einheit} & \textbf{Natürlicher Wert} & \textbf{SI-Wert} \\
				\midrule
				Länge ($l_P$) & $1$ & $\SI{1.616e-35}{\meter}$ \\
				Zeit ($t_P$) & $1$ & $\SI{5.391e-44}{\second}$ \\
				Masse ($m_P$) & $1$ & $\SI{2.176e-8}{\kilogram}$ \\
				Energie ($E_P$) & $1$ & $\SI{1.220e19}{\giga\electronvolt}$ \\
				Temperatur ($T_P$) & $1$ & $\SI{1.417e32}{\kelvin}$ \\
				\bottomrule
			\end{tabular}
		\end{adjustbox}
		\caption{Planck-Einheiten-Umwandlungen}
		\label{tab:planck_umwandlungen}
	\end{table}
	
	# Mathematisches Framework
	
	## Vereinfachte Gleichungen
	
	In natürlichen Einheiten werden fundamentale Gleichungen elegant einfach:
	
	### Quantenmechanik
	
```math-align

		\text{Schrödinger-Gleichung:} \quad & i\frac{\partial\psi}{\partial t} = H\psi \\
		\text{Unschärferelation:} \quad & \Delta E \Delta t \geq \frac{1}{2} \\
		\text{de-Broglie-Beziehung:} \quad & \lambda = \frac{1}{p}
	
```

	
	### Spezielle Relativitätstheorie
	
```math-align

		\text{Masse-Energie:} \quad & E = m \\
		\text{Energie-Impuls:} \quad & E^2 = p^2 + m^2 \\
		\text{Lorentz-Faktor:} \quad & \gamma = \frac{1}{\sqrt{1-v^2}}
	
```

	
	### Allgemeine Relativitätstheorie
	
```math-align

		\text{Einstein-Gleichungen:} \quad & G_{\mu\nu} = 8\pi T_{\mu\nu} \\
		\text{Schwarzschild-Radius:} \quad & r_s = 2M
	
```

	
	### Elektromagnetismus
	
```math-align

		\text{Coulomb-Gesetz:} \quad & F = \frac{q_1 q_2}{4\pi r^2} \\
		\text{Feinstrukturkonstante:} \quad & \alpha = \frac{e^2}{4\pi}
		\text{(mit } 4\pi\varepsilon_0 = 1\text{)}
	
```

	
	### Thermodynamik
	
```math-align

		\text{Stefan-Boltzmann:} \quad & j = \sigma T^4 \\
		\text{Wien-Gesetz:} \quad & \lambda_{max} T = b \\
		\text{Boltzmann-Verteilung:} \quad & P \propto e^{-E/T}
	
```

	
	# Vorteile und Anwendungen
	
	## Vorteile natürlicher Einheiten
	
		- \textbf{Vereinfachte Gleichungen} (z.B. $E = m$ statt $E = mc^2$)
		- \textbf{Keine überflüssigen Konstanten} in Berechnungen
		- \textbf{Universelle Skalierung} für fundamentale Physik
		- \textbf{Offenbaren fundamentaler Beziehungen} zwischen physikalischen Größen
		- \textbf{Bieten Dimensionskonsistenz-Prüfungen}
		- \textbf{Eliminieren willkürliche Umwandlungsfaktoren}
		- \textbf{Heben die universelle Rolle der Energie hervor}
	
	
	## Nachteile
	
		- \textbf{Unintuitive für makroskopische Anwendungen}
		- \textbf{Umwandlung zu SI erfordert Kenntnis} fundamentaler Konstanten
		- \textbf{Anfängliche Unvertrautheit} für an SI-Einheiten Gewöhnte
		- \textbf{Ingenieurspräferenz} für praktische SI-Einheiten
	
	
	## Praktische Anwendungen
	
		- Teilchenphysik-Berechnungen
		- Quantenfeldtheorie
		- Allgemeine Relativität und Kosmologie
		- Hochenergie-Astrophysik
		- Stringtheorie und Quantengravitation
		- Fundamentale Konstanten-Beziehungen
	
	
	# Arbeiten mit natürlichen Einheiten
	
	## Arbeiten mit natürlichen Einheiten
	
	Um eine Berechnung von SI zu natürlichen Einheiten umzuwandeln:
	
		- Alle Größen in Energieeinheiten (eV oder GeV) ausdrücken
		- $\hbar = c = G = k_B = 1$ setzen
		- Die Berechnung durchführen
		- Ergebnisse bei Bedarf zurück zu SI umwandeln
	
	
	## Dimensionsprüfung
	
	Immer Dimensionskonsistenz verifizieren:
	
		- Alle Terme in einer Gleichung müssen dieselbe Energiedimension haben
		- Prüfen, dass Exponenten konsistent sind
		- Dimensionsanalyse zur Verifikation der Ergebnisse verwenden
	
	
	## Fundamentale Kräfte in natürlichen Einheiten
	
	Die vier fundamentalen Kräfte können durch ihre dimensionslosen Kopplungskonstanten charakterisiert werden:
	
	\begin{table}[htbp]
		\centering
		\begin{adjustbox}{width=0.9\textwidth}
			\begin{tabular}{llll}
				\toprule
				\textbf{Kraft} & \textbf{Dimensionslose Kopplung} & \textbf{Typischer Wert} & \textbf{Reichweite} \\
				\midrule
				Elektromagnetisch & $\alpha_{\text{EM}}$ & $\sim 1/137$ & $\infty$ \\
				Stark & $\alpha_s$ & $\sim 0.118$ bei $Q^2 = M_Z^2$ & $\sim \SI{1e-15}{\meter}$ \\
				Schwach & $\alpha_W = g^2/(4\pi)$ & $\sim 1/30$ & $\sim \SI{1e-18}{\meter}$ \\
				Gravitation & $\alpha_G = G m^2/(\hbar c)$ & $m^2/m_P^2$ & $\infty$ \\
				\bottomrule
			\end{tabular}
		\end{adjustbox}
		\caption{Fundamentale Kräfte charakterisiert durch Kopplungskonstanten}
		\label{tab:kraefte}
	\end{table}
	
	## Umfassende Einheitenumwandlungen
	
	\begin{table}[htbp]
		\centering
		\begin{adjustbox}{width=0.95\textwidth}
			\begin{tabular}{lcccc}
				\toprule
				\textbf{SI-Einheit} & \textbf{SI-Dimension} & \textbf{Natürliche Dimension} & \textbf{Umwandlung} & \textbf{Genauigkeit} \\
				\midrule
				Meter & $[L]$ & $[E^{-1}]$ & $\SI{1}{\meter} \leftrightarrow (\SI{197}{\mega\electronvolt})^{-1}$ & $< 0.001\%$ \\
				Sekunde & $[T]$ & $[E^{-1}]$ & $\SI{1}{\second} \leftrightarrow (\SI{6.58e-22}{\mega\electronvolt})^{-1}$ & $< 0.00001\%$ \\
				Kilogramm & $[M]$ & $[E]$ & $\SI{1}{\kilogram} \leftrightarrow \SI{5.61e26}{\mega\electronvolt}$ & $< 0.001\%$ \\
				Ampere & $[I]$ & $[E]^{1/2}$ & $\SI{1}{\ampere} \leftrightarrow (\SI{6.24e18}{\electronvolt})^{1/2}/\si{\second}$ & $< 0.005\%$ \\
				Kelvin & $[\Theta]$ & $[E]$ & $\SI{1}{\kelvin} \leftrightarrow \SI{8.62e-5}{\electronvolt}$ & $< 0.01\%$ \\
				Volt & $[ML^2 T^{-3} I^{-1}]$ & $[E]$ & $\SI{1}{\volt} \leftrightarrow \SI{1}{\electronvolt}/e$ & $< 0.0001\%$ \\
				Coulomb & $[T I]$ & $[E^0]$ & $\SI{1}{\coulomb} \leftrightarrow 6.24 \times 10^{18} \, e$ & $< 0.0001\%$ \\
				\bottomrule
			\end{tabular}
		\end{adjustbox}
		\caption{Umfassende Einheitenumwandlungen von SI zu natürlichen Einheiten}
		\label{tab:umwandlung}
	\end{table}
	
	# Schlussfolgerung
	
	Dieses natürliche Einheitensystem bildet die Grundlage für alle T0-Modell-Berechnungen. Durch Etablierung der Energie als universelle Dimension und Setzen fundamentaler Konstanten auf Eins offenbaren wir die zugrundeliegende Einheit physikalischer Gesetze über alle Skalen von der sub-Planckschen T0-Länge bis zu kosmologischen Entfernungen.
	
	Schlüsselprinzipien:
	
		- Energie ist die fundamentale Dimension
		- Alle physikalischen Größen sind Potenzen der Energie
		- Die T0-Länge erweitert die Physik unter die Planck-Skala
		- Natürliche Einheiten vereinfachen fundamentale Gleichungen
		- Dimensionskonsistenz ist von höchster Bedeutung
	
	
	Dieses Framework dient als Basis für alle weiteren Entwicklungen im T0-Modell und bietet sowohl Rechenwerkzeuge als auch konzeptuelle Einsichten in die Natur der physikalischen Realität.

\end{document}
