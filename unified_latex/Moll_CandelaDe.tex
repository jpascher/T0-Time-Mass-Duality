\documentclass[11pt,a4paper,openany]{book}

% Essential packages
\usepackage[utf8]{inputenc}
\usepackage[T1]{fontenc}
\usepackage[english]{babel}
\usepackage[a4paper,margin=2.5cm]{geometry}
\usepackage{lmodern}

% Math and physics packages
\usepackage{amsmath}
\usepackage{amssymb}
\usepackage{amsthm}
\usepackage{mathtools}
\usepackage{physics}
\usepackage{siunitx}

% Graphics and tables
\usepackage{graphicx}
\usepackage[table,xcdraw]{xcolor}
\usepackage{tikz}
\usepackage{pgfplots}
\usepackage{tcolorbox}
\usepackage{booktabs}
\usepackage{array}
\usepackage{longtable}
\usepackage{float}

% Document formatting
\usepackage{fancyhdr}
\usepackage{tocloft}
\usepackage{hyperref}
\usepackage{cleveref}
\usepackage{microtype}
\usepackage{enumitem}
\usepackage{newunicodechar}

% Additional packages
\usepackage{adjustbox}
\usepackage{algorithm}
\usepackage{algorithmic}
\usepackage{amsfonts}
\usepackage{amsmath,amsfonts,amssymb}
\usepackage{amsmath,amsfonts,amssymb,physics}
\usepackage{amsmath,amssymb}
\usepackage{amsmath,amssymb,amsfonts,amsthm}
\usepackage{amsmath,amssymb,amsthm}
\usepackage{amsmath,amssymb,physics,graphicx,xcolor,amsthm}
\usepackage{bm}
\usepackage{booktabs,array,longtable,multirow}
\usepackage{braket}
\usepackage{breakurl}
\usepackage{cancel}
\usepackage{caption}
\usepackage{cite}
\usepackage{color}
\usepackage{colortbl}
\usepackage{csquotes}
\usepackage{doi}
\usepackage{forest}
\usepackage{gensymb}
\usepackage{geometry,fancyhdr}
\usepackage{graphicx,tikz,pgfplots}
\usepackage{hyperref,url}
\usepackage{hyphenat}
\usepackage{listings}
\usepackage{listings,enumerate}
\usepackage{mdframed}
\usepackage{multicol}
\usepackage{multirow}
\usepackage{natbib}
\usepackage{pdflscape}
\usepackage{ragged2e}
\usepackage{setspace}
\usepackage{siunitx,xcolor,graphicx}
\usepackage{slashed}
\usepackage{tabularx}
\usepackage{textcomp}
\usepackage{textgreek}
\usepackage{tikz,pgfplots}
\usepackage{upgreek}
\usepackage{url}

% Custom commands and definitions
\definecolor{blue}
\definecolor{blue}{rgb}{0,0,1}
\definecolor{boxgray}
\definecolor{boxgray}{RGB}{240,240,240}
\definecolor{deepblue}
\definecolor{deepblue}{RGB}{0,0,127}
\definecolor{deepgreen}
\definecolor{deepgreen}{RGB}{0,127,0}
\definecolor{deepred}
\definecolor{deepred}{RGB}{191,0,0}
\definecolor{t0blue}
\definecolor{t0blue}{RGB}{0,102,204}
\definecolor{t0blue}{RGB}{33,150,243}
\definecolor{t0green}
\definecolor{t0green}{RGB}{0,153,0}
\definecolor{t0green}{RGB}{0,153,76}
\definecolor{t0green}{RGB}{76,175,80}
\definecolor{t0orange}
\definecolor{t0orange}{RGB}{255,152,0}
\definecolor{t0purple}
\definecolor{t0purple}{RGB}{102,0,204}
\definecolor{t0purple}{RGB}{156,39,176}
\definecolor{t0red}
\definecolor{t0red}{RGB}{204,0,0}
\definecolor{t0red}{RGB}{204,0,51}
\definecolor{t0red}{RGB}{244,67,54}
\definecolor{t0yellow}
\definecolor{t0yellow}{RGB}{255,204,0}
\geometry{a4paper, left=25mm, right=25mm, top=25mm, bottom=25mm}
\geometry{a4paper, margin=1in}
\geometry{a4paper, margin=2.5cm}
\geometry{a4paper, margin=2cm}
\geometry{left=2.5cm,right=2.5cm,top=2.5cm,bottom=2.5cm}
\geometry{left=2cm,right=2cm,top=2cm,bottom=2cm}
\geometry{margin=1in}
\geometry{margin=2.5cm}
\geometry{margin=2cm}
\hypersetup{
	colorlinks=true,
	linkcolor=blue,
	citecolor=blue,
	urlcolor=blue,
	pdftitle={Analysis and Implications of MNRAS Paper 544 for the T0-Theory}
\hypersetup{
	colorlinks=true,
	linkcolor=blue,
	citecolor=blue,
	urlcolor=blue,
	pdftitle={Beweis: Die Feinstrukturkonstante α = 1 in natürlichen Einheiten}
\hypersetup{
	colorlinks=true,
	linkcolor=blue,
	citecolor=blue,
	urlcolor=blue,
	pdftitle={Beweis: Die Koide-Formel enthält implizit $\xi$}
\hypersetup{
	colorlinks=true,
	linkcolor=blue,
	citecolor=blue,
	urlcolor=blue,
	pdftitle={Chinas Photonischer Quantenchip: 1000x-Speedup und T0-Integration}
\hypersetup{
	colorlinks=true,
	linkcolor=blue,
	citecolor=blue,
	urlcolor=blue,
	pdftitle={Complete Derivation of Higgs Mass and Wilson Coefficients}
\hypersetup{
	colorlinks=true,
	linkcolor=blue,
	citecolor=blue,
	urlcolor=blue,
	pdftitle={Complete Particle Spectrum: Standard Model vs T0 Theory}
\hypersetup{
	colorlinks=true,
	linkcolor=blue,
	citecolor=blue,
	urlcolor=blue,
	pdftitle={Conceptual Comparison of Unified Natural Units and Extended Standard Model}
\hypersetup{
	colorlinks=true,
	linkcolor=blue,
	citecolor=blue,
	urlcolor=blue,
	pdftitle={Connections between the Mizohata-Takeuchi Counterexample and the T0 Time-Mass Duality Theory}
\hypersetup{
	colorlinks=true,
	linkcolor=blue,
	citecolor=blue,
	urlcolor=blue,
	pdftitle={Das Relationale Zahlensystem: Primzahlen als fundamentale Verhältnisse}
\hypersetup{
	colorlinks=true,
	linkcolor=blue,
	citecolor=blue,
	urlcolor=blue,
	pdftitle={Das T0-Modell (Planck-Referenziert): Eine Neuformulierung der Physik}
\hypersetup{
	colorlinks=true,
	linkcolor=blue,
	citecolor=blue,
	urlcolor=blue,
	pdftitle={Das T0-Modell: Zeit-Energie-Dualität und geometrische Ruhemasse}
\hypersetup{
	colorlinks=true,
	linkcolor=blue,
	citecolor=blue,
	urlcolor=blue,
	pdftitle={Der Massenskalierungsexponent κ in der T0-Theorie}
\hypersetup{
	colorlinks=true,
	linkcolor=blue,
	citecolor=blue,
	urlcolor=blue,
	pdftitle={Der geometrische Formalismus der T0-Quantenmechanik und seine Anwendung auf Quantencomputer}
\hypersetup{
	colorlinks=true,
	linkcolor=blue,
	citecolor=blue,
	urlcolor=blue,
	pdftitle={Der xi Parameter und Teilchendifferenzierung in der T0-Theorie}
\hypersetup{
	colorlinks=true,
	linkcolor=blue,
	citecolor=blue,
	urlcolor=blue,
	pdftitle={Deterministic Quantum Mechanics via T0-Energy Field Formulation}
\hypersetup{
	colorlinks=true,
	linkcolor=blue,
	citecolor=blue,
	urlcolor=blue,
	pdftitle={Deterministische Quantenmechanik via T0-Energiefeld-Formulierung}
\hypersetup{
	colorlinks=true,
	linkcolor=blue,
	citecolor=blue,
	urlcolor=blue,
	pdftitle={Die Elektroneneinheitsladung in der T0-Theorie: Jenseits von Punkt-Singularitäten}
\hypersetup{
	colorlinks=true,
	linkcolor=blue,
	citecolor=blue,
	urlcolor=blue,
	pdftitle={Die Feinstrukturkonstante: Verschiedene Darstellungen und Beziehungen}
\hypersetup{
	colorlinks=true,
	linkcolor=blue,
	citecolor=blue,
	urlcolor=blue,
	pdftitle={Die Musikalische Spirale und die 137: Die mathematische Entdeckung der kosmischen Verstimmung}
\hypersetup{
	colorlinks=true,
	linkcolor=blue,
	citecolor=blue,
	urlcolor=blue,
	pdftitle={E=mc² = E=m: Die Konstanten-Illusion entlarvt}
\hypersetup{
	colorlinks=true,
	linkcolor=blue,
	citecolor=blue,
	urlcolor=blue,
	pdftitle={E=mc² = E=m: The Constants Illusion Exposed}
\hypersetup{
	colorlinks=true,
	linkcolor=blue,
	citecolor=blue,
	urlcolor=blue,
	pdftitle={Einfache Lagrange-Revolution: Von der Standardmodell-Komplexität zur T0-Eleganz}
\hypersetup{
	colorlinks=true,
	linkcolor=blue,
	citecolor=blue,
	urlcolor=blue,
	pdftitle={Einführung in die Umsetzung photonischer Bauteile auf Wafern für Nachrichtentechniker}
\hypersetup{
	colorlinks=true,
	linkcolor=blue,
	citecolor=blue,
	urlcolor=blue,
	pdftitle={Einführung in photonische Quantenchips für Nachrichtentechniker}
\hypersetup{
	colorlinks=true,
	linkcolor=blue,
	citecolor=blue,
	urlcolor=blue,
	pdftitle={Elimination der Masse als dimensionaler Platzhalter im T0-Modell}
\hypersetup{
	colorlinks=true,
	linkcolor=blue,
	citecolor=blue,
	urlcolor=blue,
	pdftitle={Elimination of Mass as Dimensional Placeholder in the T0 Model}
\hypersetup{
	colorlinks=true,
	linkcolor=blue,
	citecolor=blue,
	urlcolor=blue,
	pdftitle={Empirical Analysis of Deterministic Factorization Methods}
\hypersetup{
	colorlinks=true,
	linkcolor=blue,
	citecolor=blue,
	urlcolor=blue,
	pdftitle={Empirische Analyse deterministischer Faktorisierungsmethoden}
\hypersetup{
	colorlinks=true,
	linkcolor=blue,
	citecolor=blue,
	urlcolor=blue,
	pdftitle={Integration der Dirac-Gleichung im T0-Modell: Natürliche-Einheiten-Rahmenwerk}
\hypersetup{
	colorlinks=true,
	linkcolor=blue,
	citecolor=blue,
	urlcolor=blue,
	pdftitle={Integration of the Dirac Equation in the T0 Model: Natural Units Framework}
\hypersetup{
	colorlinks=true,
	linkcolor=blue,
	citecolor=blue,
	urlcolor=blue,
	pdftitle={Introduction to Photonic Quantum Chips for Communication Engineers}
\hypersetup{
	colorlinks=true,
	linkcolor=blue,
	citecolor=blue,
	urlcolor=blue,
	pdftitle={Introduction to the Implementation of Photonic Components on Wafers for Communication Engineers}
\hypersetup{
	colorlinks=true,
	linkcolor=blue,
	citecolor=blue,
	urlcolor=blue,
	pdftitle={Konzeptioneller Vergleich von Einheitlichen Natürlichen Einheiten und Erweitertem Standardmodell}
\hypersetup{
	colorlinks=true,
	linkcolor=blue,
	citecolor=blue,
	urlcolor=blue,
	pdftitle={Markov Chains in the Context of T0 Theory: Deterministic or Stochastic? A Treatise on Patterns, Preconditions, and Uncertainty}
\hypersetup{
	colorlinks=true,
	linkcolor=blue,
	citecolor=blue,
	urlcolor=blue,
	pdftitle={Markov-Ketten im Kontext der T0-Theorie: Deterministisch oder stochastisch? Ein Traktat zu Mustern, Voraussetzungen und Unsicherheit}
\hypersetup{
	colorlinks=true,
	linkcolor=blue,
	citecolor=blue,
	urlcolor=blue,
	pdftitle={Mathematical Analysis of T0-Shor Algorithm: Theoretical Framework and Computational Complexity}
\hypersetup{
	colorlinks=true,
	linkcolor=blue,
	citecolor=blue,
	urlcolor=blue,
	pdftitle={Mathematical Constructs of Alternative CMB Models: Unnikrishnan and Peratt in Harmony with the T0 Theory}
\hypersetup{
	colorlinks=true,
	linkcolor=blue,
	citecolor=blue,
	urlcolor=blue,
	pdftitle={Mathematische Analyse des T0-Shor Algorithmus: Theoretischer Rahmen und Berechnungskomplexität}
\hypersetup{
	colorlinks=true,
	linkcolor=blue,
	citecolor=blue,
	urlcolor=blue,
	pdftitle={Mathematische Konstrukte alternativer CMB-Modelle: Unnikrishnan und Peratt im Einklang mit der T0-Theorie}
\hypersetup{
	colorlinks=true,
	linkcolor=blue,
	citecolor=blue,
	urlcolor=blue,
	pdftitle={Natural Unit Systems: Universal Energy Conversion and Fundamental Length Scale Hierarchy}
\hypersetup{
	colorlinks=true,
	linkcolor=blue,
	citecolor=blue,
	urlcolor=blue,
	pdftitle={Natural Units in Theoretical Physics: A Treatise in the Context of T0 Theory}
\hypersetup{
	colorlinks=true,
	linkcolor=blue,
	citecolor=blue,
	urlcolor=blue,
	pdftitle={Natürliche Einheiten in der theoretischen Physik: Eine Abhandlung im Kontext der T0-Theorie}
\hypersetup{
	colorlinks=true,
	linkcolor=blue,
	citecolor=blue,
	urlcolor=blue,
	pdftitle={Natürliche Einheitensysteme: Universelle Energieumwandlung und fundamentale Längenskala-Hierarchie}
\hypersetup{
	colorlinks=true,
	linkcolor=blue,
	citecolor=blue,
	urlcolor=blue,
	pdftitle={Parameter System-Dependency in T0-Model: SI vs. Natural Units}
\hypersetup{
	colorlinks=true,
	linkcolor=blue,
	citecolor=blue,
	urlcolor=blue,
	pdftitle={Parameter-Systemabhängigkeit im T0-Modell: SI- vs. natürliche Einheiten}
\hypersetup{
	colorlinks=true,
	linkcolor=blue,
	citecolor=blue,
	urlcolor=blue,
	pdftitle={Proof: The Fine Structure Constant α = 1 in Natural Units}
\hypersetup{
	colorlinks=true,
	linkcolor=blue,
	citecolor=blue,
	urlcolor=blue,
	pdftitle={Proof: The Koide Formula Implicitly Contains $\xi$}
\hypersetup{
	colorlinks=true,
	linkcolor=blue,
	citecolor=blue,
	urlcolor=blue,
	pdftitle={Pure Energy T0 Theory: Ratio-Based Physics with SI Reference}
\hypersetup{
	colorlinks=true,
	linkcolor=blue,
	citecolor=blue,
	urlcolor=blue,
	pdftitle={Quantum Mechanics in the T0 Model: Field-Theoretic Foundations}
\hypersetup{
	colorlinks=true,
	linkcolor=blue,
	citecolor=blue,
	urlcolor=blue,
	pdftitle={Ratio-Based vs. Absolute: The Role of Fractal Correction in T0 Theory}
\hypersetup{
	colorlinks=true,
	linkcolor=blue,
	citecolor=blue,
	urlcolor=blue,
	pdftitle={Reine Energie T0-Theorie: Verhältnis-basierte Physik mit SI-Referenz}
\hypersetup{
	colorlinks=true,
	linkcolor=blue,
	citecolor=blue,
	urlcolor=blue,
	pdftitle={Simple Lagrangian Revolution: From Standard Model Complexity to T0 Elegance}
\hypersetup{
	colorlinks=true,
	linkcolor=blue,
	citecolor=blue,
	urlcolor=blue,
	pdftitle={Simplified Dirac Equation in T0 Theory: Field Node Approach}
\hypersetup{
	colorlinks=true,
	linkcolor=blue,
	citecolor=blue,
	urlcolor=blue,
	pdftitle={Simplified T0 Theory: Elegant Lagrangian Density for Time-Mass Duality}
\hypersetup{
	colorlinks=true,
	linkcolor=blue,
	citecolor=blue,
	urlcolor=blue,
	pdftitle={T0 Cosmology: Redshift as a Geometric Path Effect in a Static Universe}
\hypersetup{
	colorlinks=true,
	linkcolor=blue,
	citecolor=blue,
	urlcolor=blue,
	pdftitle={T0 Deterministic Quantum Computing: Complete Analysis of Important Algorithms}
\hypersetup{
	colorlinks=true,
	linkcolor=blue,
	citecolor=blue,
	urlcolor=blue,
	pdftitle={T0 Deterministisches Quantencomputing: Vollständige Analyse wichtiger Algorithmen}
\hypersetup{
	colorlinks=true,
	linkcolor=blue,
	citecolor=blue,
	urlcolor=blue,
	pdftitle={T0 Model: Complete Framework - From Time-Energy Duality to Universal Constants}
\hypersetup{
	colorlinks=true,
	linkcolor=blue,
	citecolor=blue,
	urlcolor=blue,
	pdftitle={T0 Model: Complete Parameter-Free Particle Mass Calculation}
\hypersetup{
	colorlinks=true,
	linkcolor=blue,
	citecolor=blue,
	urlcolor=blue,
	pdftitle={T0 Model: Unified Neutrino Formula Structure}
\hypersetup{
	colorlinks=true,
	linkcolor=blue,
	citecolor=blue,
	urlcolor=blue,
	pdftitle={T0 Model: Universal Energy Relations for Mol and Candela Units}
\hypersetup{
	colorlinks=true,
	linkcolor=blue,
	citecolor=blue,
	urlcolor=blue,
	pdftitle={T0 Modell: Vollständiges Framework - Von Zeit-Energie-Dualität zu universellen Konstanten}
\hypersetup{
	colorlinks=true,
	linkcolor=blue,
	citecolor=blue,
	urlcolor=blue,
	pdftitle={T0 Quantenfeldtheorie: QFT, QM und Quantencomputer}
\hypersetup{
	colorlinks=true,
	linkcolor=blue,
	citecolor=blue,
	urlcolor=blue,
	pdftitle={T0 Quantum Field Theory: QFT, QM and Quantum Computers}
\hypersetup{
	colorlinks=true,
	linkcolor=blue,
	citecolor=blue,
	urlcolor=blue,
	pdftitle={T0 Theory vs Bell's Theorem: How Deterministic Energy Fields Circumvent No-Go Theorems}
\hypersetup{
	colorlinks=true,
	linkcolor=blue,
	citecolor=blue,
	urlcolor=blue,
	pdftitle={T0 Theory: Final Extension to Hadrons - Physically Derived Corrections}
\hypersetup{
	colorlinks=true,
	linkcolor=blue,
	citecolor=blue,
	urlcolor=blue,
	pdftitle={T0 Theory: The Fine-Structure Constant}
\hypersetup{
	colorlinks=true,
	linkcolor=blue,
	citecolor=blue,
	urlcolor=blue,
	pdftitle={T0 Theory: The Gravitational Constant}
\hypersetup{
	colorlinks=true,
	linkcolor=blue,
	citecolor=blue,
	urlcolor=blue,
	pdftitle={T0-Kosmologie: Rotverschiebung als geometrischer Pfad-Effekt im statischen Universum}
\hypersetup{
	colorlinks=true,
	linkcolor=blue,
	citecolor=blue,
	urlcolor=blue,
	pdftitle={T0-Model: Complete Document Analysis and Structured Summary}
\hypersetup{
	colorlinks=true,
	linkcolor=blue,
	citecolor=blue,
	urlcolor=blue,
	pdftitle={T0-Model: Kinetic Energy of Electrons and Photons}
\hypersetup{
	colorlinks=true,
	linkcolor=blue,
	citecolor=blue,
	urlcolor=blue,
	pdftitle={T0-Model: The Hubble Parameter in Static Universe}
\hypersetup{
	colorlinks=true,
	linkcolor=blue,
	citecolor=blue,
	urlcolor=blue,
	pdftitle={T0-Modell-Verifikation: Skalen-Verhältnis-basierte Berechnungen}
\hypersetup{
	colorlinks=true,
	linkcolor=blue,
	citecolor=blue,
	urlcolor=blue,
	pdftitle={T0-Modell: Bewegungsenergie von Elektronen und Photonen}
\hypersetup{
	colorlinks=true,
	linkcolor=blue,
	citecolor=blue,
	urlcolor=blue,
	pdftitle={T0-Modell: Die Hubble-Konstante im statischen Universum}
\hypersetup{
	colorlinks=true,
	linkcolor=blue,
	citecolor=blue,
	urlcolor=blue,
	pdftitle={T0-Modell: Einheitliche Neutrino-Formel-Struktur}
\hypersetup{
	colorlinks=true,
	linkcolor=blue,
	citecolor=blue,
	urlcolor=blue,
	pdftitle={T0-Modell: Universelle Energiebeziehungen für Mol- und Candela-Einheiten}
\hypersetup{
	colorlinks=true,
	linkcolor=blue,
	citecolor=blue,
	urlcolor=blue,
	pdftitle={T0-Modell: Vollständige Dokumentenanalyse und strukturierte Zusammenfassung}
\hypersetup{
	colorlinks=true,
	linkcolor=blue,
	citecolor=blue,
	urlcolor=blue,
	pdftitle={T0-Modell: Vollständige parameterfreie Teilchenmassen-Berechnung}
\hypersetup{
	colorlinks=true,
	linkcolor=blue,
	citecolor=blue,
	urlcolor=blue,
	pdftitle={T0-QAT: $\xi$-Aware Quantization-Aware Training}
\hypersetup{
	colorlinks=true,
	linkcolor=blue,
	citecolor=blue,
	urlcolor=blue,
	pdftitle={T0-QFT ML Addendum: Machine Learning Derived Extensions}
\hypersetup{
	colorlinks=true,
	linkcolor=blue,
	citecolor=blue,
	urlcolor=blue,
	pdftitle={T0-QFT ML-Addendum: Maschinelle Lern-abgeleitete Erweiterungen}
\hypersetup{
	colorlinks=true,
	linkcolor=blue,
	citecolor=blue,
	urlcolor=blue,
	pdftitle={T0-Theorie vs Bells Theorem: Wie deterministische Energiefelder No-Go-Theoreme umgehen}
\hypersetup{
	colorlinks=true,
	linkcolor=blue,
	citecolor=blue,
	urlcolor=blue,
	pdftitle={T0-Theorie: Der Terrell-Penrose-Effekt und Massenvariation}
\hypersetup{
	colorlinks=true,
	linkcolor=blue,
	citecolor=blue,
	urlcolor=blue,
	pdftitle={T0-Theorie: Die Feinstrukturkonstante}
\hypersetup{
	colorlinks=true,
	linkcolor=blue,
	citecolor=blue,
	urlcolor=blue,
	pdftitle={T0-Theorie: Die Gravitationskonstante}
\hypersetup{
	colorlinks=true,
	linkcolor=blue,
	citecolor=blue,
	urlcolor=blue,
	pdftitle={T0-Theorie: Die T0-Zeit-Masse-Dualität}
\hypersetup{
	colorlinks=true,
	linkcolor=blue,
	citecolor=blue,
	urlcolor=blue,
	pdftitle={T0-Theorie: Die sieben Rätsel}
\hypersetup{
	colorlinks=true,
	linkcolor=blue,
	citecolor=blue,
	urlcolor=blue,
	pdftitle={T0-Theorie: Erweiterung auf Bell-Tests – ML-Simulationen (November 2025)}
\hypersetup{
	colorlinks=true,
	linkcolor=blue,
	citecolor=blue,
	urlcolor=blue,
	pdftitle={T0-Theorie: Finale Erweiterung auf Hadronen - Physikalisch abgeleitete Korrekturen}
\hypersetup{
	colorlinks=true,
	linkcolor=blue,
	citecolor=blue,
	urlcolor=blue,
	pdftitle={T0-Theorie: Finale Fraktale Massenformeln (November 2025)}
\hypersetup{
	colorlinks=true,
	linkcolor=blue,
	citecolor=blue,
	urlcolor=blue,
	pdftitle={T0-Theorie: Fraktaldimension aus Lepton-Massenverhältnis}
\hypersetup{
	colorlinks=true,
	linkcolor=blue,
	citecolor=blue,
	urlcolor=blue,
	pdftitle={T0-Theorie: Fundamentale Prinzipien}
\hypersetup{
	colorlinks=true,
	linkcolor=blue,
	citecolor=blue,
	urlcolor=blue,
	pdftitle={T0-Theorie: Herleitung der Gravitationskonstanten}
\hypersetup{
	colorlinks=true,
	linkcolor=blue,
	citecolor=blue,
	urlcolor=blue,
	pdftitle={T0-Theorie: Kosmische Beziehungen und universelle $\xi$-Konstante}
\hypersetup{
	colorlinks=true,
	linkcolor=blue,
	citecolor=blue,
	urlcolor=blue,
	pdftitle={T0-Theorie: Kosmologie}
\hypersetup{
	colorlinks=true,
	linkcolor=blue,
	citecolor=blue,
	urlcolor=blue,
	pdftitle={T0-Theorie: Netzwerkdarstellung und Dimensionsanalyse in der T0-Theorie}
\hypersetup{
	colorlinks=true,
	linkcolor=blue,
	citecolor=blue,
	urlcolor=blue,
	pdftitle={T0-Theorie: Teilchenmassen}
\hypersetup{
	colorlinks=true,
	linkcolor=blue,
	citecolor=blue,
	urlcolor=blue,
	pdftitle={T0-Theorie: Vollstaendiger Abschluss}
\hypersetup{
	colorlinks=true,
	linkcolor=blue,
	citecolor=blue,
	urlcolor=blue,
	pdftitle={T0-Theory: Complete Closure}
\hypersetup{
	colorlinks=true,
	linkcolor=blue,
	citecolor=blue,
	urlcolor=blue,
	pdftitle={T0-Theory: Complete Derivation of All Parameters Without Circularity}
\hypersetup{
	colorlinks=true,
	linkcolor=blue,
	citecolor=blue,
	urlcolor=blue,
	pdftitle={T0-Theory: Cosmic Relations and universal $\xi$-constant}
\hypersetup{
	colorlinks=true,
	linkcolor=blue,
	citecolor=blue,
	urlcolor=blue,
	pdftitle={T0-Theory: Cosmology}
\hypersetup{
	colorlinks=true,
	linkcolor=blue,
	citecolor=blue,
	urlcolor=blue,
	pdftitle={T0-Theory: Derivation of the Gravitational Constant}
\hypersetup{
	colorlinks=true,
	linkcolor=blue,
	citecolor=blue,
	urlcolor=blue,
	pdftitle={T0-Theory: Extension to Bell Tests – ML Simulations (November 2025)}
\hypersetup{
	colorlinks=true,
	linkcolor=blue,
	citecolor=blue,
	urlcolor=blue,
	pdftitle={T0-Theory: Final Fractal Mass Formulas (November 2025)}
\hypersetup{
	colorlinks=true,
	linkcolor=blue,
	citecolor=blue,
	urlcolor=blue,
	pdftitle={T0-Theory: Fractal Dimension from Lepton Mass Ratio}
\hypersetup{
	colorlinks=true,
	linkcolor=blue,
	citecolor=blue,
	urlcolor=blue,
	pdftitle={T0-Theory: Fundamental Principles}
\hypersetup{
	colorlinks=true,
	linkcolor=blue,
	citecolor=blue,
	urlcolor=blue,
	pdftitle={T0-Theory: Mass Variation as an Equivalent to Time Dilation}
\hypersetup{
	colorlinks=true,
	linkcolor=blue,
	citecolor=blue,
	urlcolor=blue,
	pdftitle={T0-Theory: Network Representation and Dimensional Analysis in the T0-Theory}
\hypersetup{
	colorlinks=true,
	linkcolor=blue,
	citecolor=blue,
	urlcolor=blue,
	pdftitle={T0-Theory: Neutrinos}
\hypersetup{
	colorlinks=true,
	linkcolor=blue,
	citecolor=blue,
	urlcolor=blue,
	pdftitle={T0-Theory: Particle Masses}
\hypersetup{
	colorlinks=true,
	linkcolor=blue,
	citecolor=blue,
	urlcolor=blue,
	pdftitle={T0-Theory: The Seven Riddles}
\hypersetup{
	colorlinks=true,
	linkcolor=blue,
	citecolor=blue,
	urlcolor=blue,
	pdftitle={T0-Theory: The T0-Time-Mass Duality}
\hypersetup{
	colorlinks=true,
	linkcolor=blue,
	citecolor=blue,
	urlcolor=blue,
	pdftitle={Temperature Units in Natural Units: T0-Theory}
\hypersetup{
	colorlinks=true,
	linkcolor=blue,
	citecolor=blue,
	urlcolor=blue,
	pdftitle={Temperatureinheiten in nat\"urlichen Einheiten: T0-Theorie}
\hypersetup{
	colorlinks=true,
	linkcolor=blue,
	citecolor=blue,
	urlcolor=blue,
	pdftitle={The Electron Unit Charge in T0 Theory: Beyond Point Singularities}
\hypersetup{
	colorlinks=true,
	linkcolor=blue,
	citecolor=blue,
	urlcolor=blue,
	pdftitle={The Fine Structure Constant: Various Representations and Relationships}
\hypersetup{
	colorlinks=true,
	linkcolor=blue,
	citecolor=blue,
	urlcolor=blue,
	pdftitle={The Geometric Formalism of T0 Quantum Mechanics and its Application to Quantum Computing}
\hypersetup{
	colorlinks=true,
	linkcolor=blue,
	citecolor=blue,
	urlcolor=blue,
	pdftitle={The Mass Scaling Exponent κ in T0 Theory}
\hypersetup{
	colorlinks=true,
	linkcolor=blue,
	citecolor=blue,
	urlcolor=blue,
	pdftitle={The Musical Spiral and 137: The Mathematical Discovery of Cosmic Detuning}
\hypersetup{
	colorlinks=true,
	linkcolor=blue,
	citecolor=blue,
	urlcolor=blue,
	pdftitle={The Relational Number System: Prime Numbers as Fundamental Ratios}
\hypersetup{
	colorlinks=true,
	linkcolor=blue,
	citecolor=blue,
	urlcolor=blue,
	pdftitle={The T0 Model (Planck-Referenced): A Reformulation of Physics}
\hypersetup{
	colorlinks=true,
	linkcolor=blue,
	citecolor=blue,
	urlcolor=blue,
	pdftitle={The T0 Model: Time-Energy Duality and Geometric Rest Mass}
\hypersetup{
	colorlinks=true,
	linkcolor=blue,
	citecolor=blue,
	urlcolor=blue,
	pdftitle={The T0-Model (Planck-Referenced): A Reformulation of Physics}
\hypersetup{
	colorlinks=true,
	linkcolor=blue,
	citecolor=blue,
	urlcolor=blue,
	pdftitle={Verbindungen zwischen dem Mizohata-Takeuchi-Gegenbeispiel und der T0-Zeit-Masse-Dualitätstheorie}
\hypersetup{
	colorlinks=true,
	linkcolor=blue,
	citecolor=blue,
	urlcolor=blue,
	pdftitle={Vereinfachte Dirac-Gleichung in der T0-Theorie: Feldknoten-Ansatz}
\hypersetup{
	colorlinks=true,
	linkcolor=blue,
	citecolor=blue,
	urlcolor=blue,
	pdftitle={Vereinfachte T0-Theorie: Elegante Lagrange-Dichte für Zeit-Masse-Dualität}
\hypersetup{
	colorlinks=true,
	linkcolor=blue,
	citecolor=blue,
	urlcolor=blue,
	pdftitle={Verhältnisbasiert vs. Absolut: Die Rolle der fraktalen Korrektur in der T0-Theorie}
\hypersetup{
	colorlinks=true,
	linkcolor=blue,
	citecolor=blue,
	urlcolor=blue,
	pdftitle={Vollständige Herleitung der Higgs-Masse und Wilson-Koeffizienten}
\hypersetup{
	colorlinks=true,
	linkcolor=blue,
	citecolor=blue,
	urlcolor=blue,
	pdftitle={Vollständiges Teilchenspektrum: Standard-Modell vs T0-Theorie}
\hypersetup{
	colorlinks=true,
	linkcolor=blue,
	citecolor=blue,
	urlcolor=blue,
	pdftitle={Warum Zahlenverhältnisse nicht direkt gekürzt werden dürfen}
\hypersetup{
	colorlinks=true,
	linkcolor=blue,
	citecolor=blue,
	urlcolor=blue,
	pdftitle={Why Numerical Ratios Must Not Be Directly Simplified}
\hypersetup{
	colorlinks=true,
	linkcolor=blue,
	citecolor=blue,
	urlcolor=blue,
}
\hypersetup{
	colorlinks=true,
	linkcolor=blue,
	citecolor=red,
	urlcolor=blue,
	bookmarks=true,
	bookmarksnumbered=true,
	pdfstartview=FitH,
	pdftitle={T0 Model - Field-Theoretic Derivation of the Beta Parameter}
\hypersetup{
	colorlinks=true,
	linkcolor=blue,
	citecolor=red,
	urlcolor=blue,
	bookmarks=true,
	bookmarksnumbered=true,
	pdfstartview=FitH,
	pdftitle={T0-Modell - Feldtheoretische Herleitung des Beta-Parameters}
\hypersetup{
	colorlinks=true,
	linkcolor=blue,
	filecolor=magenta,
	urlcolor=cyan,
}
\hypersetup{
	colorlinks=true,
	linkcolor=blue,
	urlcolor=blue,
	citecolor=blue,
	pdftitle={From Time Dilation to Mass Variation: Mathematical Core Formulations of Time-Mass Duality Theory - Updated Framework}
\hypersetup{
	colorlinks=true,
	linkcolor=blue,
	urlcolor=blue,
	citecolor=blue,
	pdftitle={T0 Model: Detailed Formula for Leptonic Anomalies}
\hypersetup{
	colorlinks=true,
	linkcolor=blue,
	urlcolor=blue,
	citecolor=blue,
	pdftitle={T0 Model: Detaillierte Formel für leptonische Anomalien}
\hypersetup{
	colorlinks=true,
	linkcolor=blue,
	urlcolor=blue,
	citecolor=blue,
	pdftitle={T0 Model: Energy-based Formulas with Quadratic Scaling}
\hypersetup{
	colorlinks=true,
	linkcolor=blue,
	urlcolor=blue,
	citecolor=blue,
	pdftitle={T0 Model: Granulation, Limits and Fundamental Asymmetry}
\hypersetup{
	colorlinks=true,
	linkcolor=blue,
	urlcolor=blue,
	citecolor=blue,
	pdftitle={T0-Modell: Energiebasierte Formeln mit quadratischer Skalierung}
\hypersetup{
	colorlinks=true,
	linkcolor=blue,
	urlcolor=blue,
	citecolor=blue,
	pdftitle={T0-Modell: Granulation, Limits und fundamentale Asymmetrie}
\hypersetup{
	colorlinks=true,
	linkcolor=blue,
	urlcolor=blue,
	citecolor=blue,
	pdftitle={Von Zeitdilatation zu Massenvariation: Mathematische Kernformulierungen der Zeit-Masse-Dualitätstheorie - Aktualisiertes Framework}
\hypersetup{
	colorlinks=true,
	linkcolor=t0blue,
	citecolor=t0blue,
	urlcolor=t0blue,
	pdftitle={T0 Model: Complete Theoretical Summary}
\hypersetup{
	colorlinks=true,
	linkcolor=t0blue,
	citecolor=t0blue,
	urlcolor=t0blue,
	pdftitle={T0 Theory: Resolution of Apparent Instantaneity}
\hypersetup{
	colorlinks=true,
	linkcolor=t0blue,
	citecolor=t0blue,
	urlcolor=t0blue,
	pdftitle={T0 vs Synergetics: Vereinfachung durch natürliche Einheiten}
\hypersetup{
	colorlinks=true,
	linkcolor=t0blue,
	citecolor=t0blue,
	urlcolor=t0blue,
	pdftitle={T0-Modell: Vollständige theoretische Zusammenfassung}
\hypersetup{
	colorlinks=true,
	linkcolor=t0blue,
	citecolor=t0blue,
	urlcolor=t0blue,
	pdftitle={T0-Theorie: Auflösung der scheinbaren Instantanität}
\hypersetup{
	colorlinks=true,
	linkcolor=t0blue,
	citecolor=t0blue,
	urlcolor=t0blue,
	pdftitle={T0-Theorie: Vollständige Dokumentenübersicht}
\hypersetup{
	colorlinks=true,
	linkcolor=t0blue,
	citecolor=t0blue,
	urlcolor=t0blue,
	pdftitle={T0-Theory: Complete Document Overview}
\hypersetup{
	colorlinks=true,
	linkcolor=t0blue,
	citecolor=t0blue,
	urlcolor=t0blue,
}
\hypersetup{
	colorlinks=true,
	linkcolor=t0blue,
	citecolor=t0green,
	urlcolor=t0blue,
	pdftitle={Das verborgene Geheimnis von 1/137}
\hypersetup{
	colorlinks=true,
	linkcolor=t0blue,
	citecolor=t0green,
	urlcolor=t0blue,
	pdftitle={The Hidden Secret of 1/137}
\hypersetup{
    colorlinks=true,
    linkcolor=blue,
    citecolor=blue,
    urlcolor=blue,
    pdftitle={Analyse und Implikationen des MNRAS-Papiers 544 für die T0-Theorie}
\hypersetup{
  colorlinks=true,
  linkcolor=blue,
  citecolor=blue,
  urlcolor=blue
}
\hypersetup{
  colorlinks=true,
  linkcolor=blue,
  citecolor=blue,
  urlcolor=blue,
  pdftitle={T0-Theorie: Ein-Uhr-Metrologie und Drei-Uhren-Experiment}
\hypersetup{
  colorlinks=true,
  linkcolor=blue,
  citecolor=blue,
  urlcolor=blue,
  pdftitle={T0-Theory: Single-Clock Metrology and Three-Clock Experiment}
\hypersetup{
colorlinks=true,
linkcolor=blue,
citecolor=blue,
urlcolor=blue,
pdftitle={Quantenmechanik im T0-Modell: Feldtheoretische Grundlagen}
\hypersetup{
colorlinks=true,
linkcolor=blue,
citecolor=blue,
urlcolor=blue,
pdftitle={T0-Theory: Neutrinos}
\newcommand{\Bzero}{B_0}
\newcommand{\CQCD}{C_{\text{QCD}
\newcommand{\Cconv}{C_{\text{conv}
\newcommand{\Cto}{C_{\text{T0}
\newcommand{\Czero}{C_0}
\newcommand{\DTmu}{D_{T,\mu}
\newcommand{\DcovT}[1]{\partial_\mu #1 + #1 \partial_\mu \Tfield}
\newcommand{\Dfrak}{D_f}
\newcommand{\Df}{D_f}
\newcommand{\DhiggsT}{\Tfield (\partial_\mu + ig A_\mu) \Phi + \Phi \partial_\mu \Tfield}
\newcommand{\EPlanck}{E_P}
\newcommand{\EPlanck}{E_{\text{Pl}
\newcommand{\EPratio}[1]{\frac{#1}
\newcommand{\EP}{E_P}
\newcommand{\EP}{E_{\text{P}
\newcommand{\EW}{E_W}
\newcommand{\EZ}{E_Z}
\newcommand{\Echar}{E_{\text{char}
\newcommand{\Ee}{E_e}
\newcommand{\Efield}{E(x,t)}
\newcommand{\Efield}{E_\text{field}
\newcommand{\Efield}{E_{\text{Feld}
\newcommand{\Efield}{E_{\text{Field}
\newcommand{\Efield}{E_{\text{field}
\newcommand{\Efield}{E}
\newcommand{\Egamma}{E_\gamma}
\newcommand{\Eh}{E_h}
\newcommand{\Emu}{E_\mu}
\newcommand{\Enorm}[1]{E_{\text{norm}
\newcommand{\En}{E_n}
\newcommand{\Ep}{E_p}
\newcommand{\Eratio}[2]{\frac{E_{#1}
\newcommand{\Etau}{E_\tau}
\newcommand{\Evis}{E_{\text{vis}
\newcommand{\Exi}{E_\xi}
\newcommand{\Ezero}{E_0}
\newcommand{\GeV}{\,\text{GeV}
\newcommand{\Gnat}{G_{\text{nat}
\newcommand{\Gsi}{G_{\text{SI}
\newcommand{\Hubble}{H_0}
\newcommand{\Kfrak}{K_{\text{frac}
\newcommand{\Kfrak}{K_{\text{frak}
\newcommand{\Kspec}{K_{\text{spec}
\newcommand{\LCDM}{\Lambda\text{CDM}
\newcommand{\LPlanck}{\ell_{\text{Pl}
\newcommand{\Lag}{\mathcal{L}
\newcommand{\Lambdat}{\Lambda_T}
\newcommand{\Leff}{L_{\text{eff}
\newcommand{\Lorentz}[2]{{\Lambda^\mu{}
\newcommand{\Lp}{L_{\text{P}
\newcommand{\Lxi}{L_\xi}
\newcommand{\Lzero}{L_0}
\newcommand{\MPl}{M_{\text{Pl}
\newcommand{\MSbar}{\overline{\text{MS}
\newcommand{\MeV}{\,\text{MeV}
\newcommand{\Mpl}{M_{\text{Pl}
\newcommand{\OmegaDM}{\Omega_{\text{DM}
\newcommand{\OmegaLambda}{\Omega_{\Lambda}
\newcommand{\Omegab}{\Omega_b}
\newcommand{\Phiphoton}{\Phi_{\text{photon}
\newcommand{\Ricci}{R_{\mu\nu}
\newcommand{\Riem}{R^\rho{}
\newcommand{\Rzero}{R_\infty}
\newcommand{\Scal}{R}
\newcommand{\SynchPower}{P_{\text{synch}
\newcommand{\TPlanck}{t_{\text{Pl}
\newcommand{\Tfieldt}{T(\vec{x}
\newcommand{\Tfieldt}{T(x,t)}
\newcommand{\Tfield}{T(x)}
\newcommand{\Tfield}{T(x,t)}
\newcommand{\Tfield}{T_{\text{field}
\newcommand{\Tfield}{T}
\newcommand{\Tfield}{\mathcal{T}
\newcommand{\Tzerot}{T_0(\Tfield)}
\newcommand{\Tzero}{T_0}
\newcommand{\Weyl}{C^\rho{}
\newcommand{\ZPinch}{J \times B = \nabla p}
\newcommand{\aleph}{\aleph}
\newcommand{\alphaEMSI}{\alpha_{\text{EM,SI}
\newcommand{\alphaEMnat}{\alpha_{\text{EM,nat}
\newcommand{\alphaEM}{\alpha_{\text{EM}
\newcommand{\alphaEM}{\ensuremath{\alpha_{\text{EM}
\newcommand{\alphaQCD}{\alpha_s}
\newcommand{\alphaQED}{\alpha_{\text{QED}
\newcommand{\alphaSI}{\alpha_{\text{SI}
\newcommand{\alphaT}{\alpha_{\text{T}
\newcommand{\alphaWSI}{\alpha_{\text{W,SI}
\newcommand{\alphaWnat}{\alpha_{\text{W,nat}
\newcommand{\alphaW}{\alpha_{\text{W}
\newcommand{\alphaem}{\alpha_{EM}
\newcommand{\alphaem}{\alpha}
\newcommand{\alphafine}{\alpha}
\newcommand{\alphagem}{\alpha}
\newcommand{\alphanat}{\alpha_{\text{nat}
\newcommand{\alphapar}{\alpha}
\newcommand{\betaTSI}{\beta_{\text{T,SI}
\newcommand{\betaTnat}{\beta_{\text{T,nat}
\newcommand{\betaT}{\beta_T}
\newcommand{\betaT}{\beta_{T}
\newcommand{\betaT}{\beta_{\text{T}
\newcommand{\betaT}{\ensuremath{\beta_T}
\newcommand{\betapar}{\beta}
\newcommand{\calL}{\mathcal{L}
\newcommand{\checked}{\checkmark}
\newcommand{\checkmarkx}{\checkmark}
\newcommand{\dTdt}{\frac{d\Tfieldt}
\newcommand{\deltaE}{\delta E}
\newcommand{\deltafield}{\ensuremath{\delta m}
\newcommand{\deltam}{\delta m}
\newcommand{\deq}{\displaystyle}
\newcommand{\docref}[1]{\texttt{#1}
\newcommand{\eV}{\,\text{eV}
\newcommand{\epsilonT}{\varepsilon_T}
\newcommand{\epsilonzero}{\varepsilon_0}
\newcommand{\etavis}{\eta_{\text{visual}
\newcommand{\e}{\mathrm{e}
\newcommand{\gW}{g_W}
\newcommand{\gammaf}{\gamma_{\text{Lorentz}
\newcommand{\gammamu}{\gamma^\mu}
\newcommand{\gs}{g_s}
\newcommand{\inftytext}{$\infty$}
\newcommand{\interval}[2]{#1:#2}
\newcommand{\kfrac}{K_{\text{frak}
\newcommand{\lP}{\ell_{\text{P}
\newcommand{\lP}{l_P}
\newcommand{\lambdah}{\ensuremath{\lambda_h}
\newcommand{\lambdah}{\lambda_h}
\newcommand{\lambdazero}{\lambda_0}
\newcommand{\mP}{m_{\text{P}
\newcommand{\mfield}{m(x,t)}
\newcommand{\mfield}{m}
\newcommand{\mh}{m_h}
\newcommand{\micrometer}{\ensuremath{\mu}
\newcommand{\mikrometer}{\ensuremath{\mu}
\newcommand{\myRightarrow}{\ensuremath{\Rightarrow}
\newcommand{\myapprox}{\ensuremath{\approx}
\newcommand{\myomega}{\ensuremath{\omega}
\newcommand{\myphi}{\ensuremath{\phi}
\newcommand{\mypi}{\ensuremath{\pi}
\newcommand{\mypropto}{\ensuremath{\propto}
\newcommand{\myrightarrow}{\ensuremath{\rightarrow}
\newcommand{\mysim}{\ensuremath{\sim}
\newcommand{\mysqrt}{\ensuremath{\sqrt}
\newcommand{\mytimes}{\ensuremath{\times}
\newcommand{\natunits}{\hbar = c = G = k_B = 1}
\newcommand{\natunits}{\text{(nat. Einh.)}
\newcommand{\natunits}{\text{(nat. units)}
\newcommand{\nulep}{\nu}
\newcommand{\nuzero}{\nu_0}
\newcommand{\partialop}{\ensuremath{\partial}
\newcommand{\pdTdt}{\frac{\partial\Tfieldt}
\newcommand{\pdTdx}{\nabla\Tfieldt}
\newcommand{\phiT}{\phi}
\newcommand{\pichar}{\pi}
\newcommand{\primrel}[1]{\mathbf{#1}
\newcommand{\rhoCMB}{\rho_{\text{CMB}
\newcommand{\rhoCasimir}{\rho_{\text{Casimir}
\newcommand{\rhoE}{\rho_E}
\newcommand{\rhofield}{\ensuremath{\rho}
\newcommand{\rzero}{r_0}
\newcommand{\slashk}{\cancel{k}
\newcommand{\slashp}{\cancel{p}
\newcommand{\slashq}{\cancel{q}
\newcommand{\tP}{t_P}
\newcommand{\tP}{t_{\text{P}
\newcommand{\tablescale}{0.9}
\newcommand{\tzero}{t_0}
\newcommand{\vect}[1]{\boldsymbol{#1}
\newcommand{\vecx}{\vec{x}
\newcommand{\vh}{v}
\newcommand{\vr}{\vec{r}
\newcommand{\warningx}{\color{red}
\newcommand{\warningx}{\textbf{!}
\newcommand{\warningx}{{\color{red}
\newcommand{\xiT}{\xi}
\newcommand{\xiconst}{\xi = \frac{4}
\newcommand{\xicoupling}{f(E/\Exi)}
\newcommand{\xigeom}{\xi_{\text{geom}
\newcommand{\xigeom}{\xi}
\newcommand{\xikonst}{\xi = \frac{4}
\newcommand{\xiparticle}{\xi_{\text{particle}
\newcommand{\xipar}{\ensuremath{\xi}
\newcommand{\xipar}{\xi_0}
\newcommand{\xipar}{\xi}
\newcommand{\xirat}{\xi_{\text{ratio}
\newtheorem{axiom}{Axiom}
\newtheorem{category}{Category-Theoretic Basis}
\newtheorem{category}{Kategorientheoretische Basis}
\newtheorem{corollary}[theorem]{Corollary}
\newtheorem{corollary}[theorem]{Korollar}
\newtheorem{corollary}{Corollary}
\newtheorem{corollary}{Korollar}
\newtheorem{definition}[theorem]{Definition}
\newtheorem{definition}{Definition}
\newtheorem{discovery}{Discovery}
\newtheorem{discovery}{Neue Entdeckung}
\newtheorem{discovery}{New Discovery}
\newtheorem{discovery}{Revolutionary Discovery}
\newtheorem{entdeckung}{Entdeckung}
\newtheorem{entdeckung}{Revolutionäre Entdeckung}
\newtheorem{erkenntnis}{Erkenntnis}
\newtheorem{erkenntnis}{Schlüsselerkenntnis}
\newtheorem{example}[theorem]{Beispiel}
\newtheorem{example}[theorem]{Example}
\newtheorem{example}{Beispiel}
\newtheorem{example}{Example}
\newtheorem{insight}{Central Insight}
\newtheorem{insight}{Insight}
\newtheorem{insight}{Key Insight}
\newtheorem{insight}{Wichtige Einsicht}
\newtheorem{insight}{Zentrale Einsicht}
\newtheorem{lemma}[theorem]{Lemma}
\newtheorem{lemma}{Lemma}
\newtheorem{principle}{Fundamental Principle}
\newtheorem{principle}{Fundamentales Prinzip}
\newtheorem{principle}{Grundlegendes Prinzip}
\newtheorem{principle}{Principle}
\newtheorem{principle}{Prinzip}
\newtheorem{prinzip}{Grundprinzip}
\newtheorem{proof_step}{Beweisschritt}
\newtheorem{proof_step}{Proof Step}
\newtheorem{proposition}[theorem]{Proposition}
\newtheorem{proposition}{Proposition}
\newtheorem{remark}[theorem]{Bemerkung}
\newtheorem{remark}[theorem]{Remark}
\newtheorem{theorem}{Theorem}
\newtheorem{warning}[theorem]{Warning}
\newtheorem{warning}[theorem]{Warnung}
\newunicodechar{±}{\ensuremath{\pm}
\newunicodechar{×}{\ensuremath{\times}
\newunicodechar{÷}{\ensuremath{\div}
\newunicodechar{ħ}{\ensuremath{\hbar}
\newunicodechar{Α}{\ensuremath{A}
\newunicodechar{Β}{\ensuremath{B}
\newunicodechar{Γ}{\ensuremath{\Gamma}
\newunicodechar{Δ}{\ensuremath{\Delta}
\newunicodechar{Ε}{\ensuremath{E}
\newunicodechar{Ζ}{\ensuremath{Z}
\newunicodechar{Η}{\ensuremath{H}
\newunicodechar{Θ}{\ensuremath{\Theta}
\newunicodechar{Ι}{\ensuremath{I}
\newunicodechar{Κ}{\ensuremath{K}
\newunicodechar{Λ}{\ensuremath{\Lambda}
\newunicodechar{Μ}{\ensuremath{M}
\newunicodechar{Ν}{\ensuremath{N}
\newunicodechar{Ξ}{\ensuremath{\Xi}
\newunicodechar{Ο}{\ensuremath{O}
\newunicodechar{Π}{\ensuremath{\Pi}
\newunicodechar{Ρ}{\ensuremath{P}
\newunicodechar{Σ}{\ensuremath{\Sigma}
\newunicodechar{Τ}{\ensuremath{T}
\newunicodechar{Υ}{\ensuremath{\Upsilon}
\newunicodechar{Φ}{\ensuremath{\Phi}
\newunicodechar{Χ}{\ensuremath{X}
\newunicodechar{Ψ}{\ensuremath{\Psi}
\newunicodechar{Ω}{\ensuremath{\Omega}
\newunicodechar{α}{\ensuremath{\alpha}
\newunicodechar{β}{\ensuremath{\beta}
\newunicodechar{γ}{\ensuremath{\gamma}
\newunicodechar{δ}{\ensuremath{\delta}
\newunicodechar{ε}{\ensuremath{\varepsilon}
\newunicodechar{ζ}{\ensuremath{\zeta}
\newunicodechar{η}{\ensuremath{\eta}
\newunicodechar{θ}{\ensuremath{\theta}
\newunicodechar{ι}{\ensuremath{\iota}
\newunicodechar{κ}{\ensuremath{\kappa}
\newunicodechar{λ}{\ensuremath{\lambda}
\newunicodechar{μ}{\ensuremath{\mu}
\newunicodechar{ν}{\ensuremath{\nu}
\newunicodechar{ξ}{\ensuremath{\xi}
\newunicodechar{ο}{\ensuremath{o}
\newunicodechar{π}{\ensuremath{\pi}
\newunicodechar{ρ}{\ensuremath{\rho}
\newunicodechar{σ}{\ensuremath{\sigma}
\newunicodechar{τ}{\ensuremath{\tau}
\newunicodechar{υ}{\ensuremath{\upsilon}
\newunicodechar{φ}{\ensuremath{\phi}
\newunicodechar{φ}{\ensuremath{\varphi}
\newunicodechar{χ}{\ensuremath{\chi}
\newunicodechar{ψ}{\ensuremath{\psi}
\newunicodechar{ω}{\ensuremath{\omega}
\newunicodechar{←}{\ensuremath{\leftarrow}
\newunicodechar{→}{\ensuremath{\rightarrow}
\newunicodechar{↔}{\ensuremath{\leftrightarrow}
\newunicodechar{⇐}{\ensuremath{\Leftarrow}
\newunicodechar{⇒}{\ensuremath{\Rightarrow}
\newunicodechar{⇔}{\ensuremath{\Leftrightarrow}
\newunicodechar{∂}{\ensuremath{\partial}
\newunicodechar{∅}{\ensuremath{\emptyset}
\newunicodechar{∇}{\ensuremath{\nabla}
\newunicodechar{∈}{\ensuremath{\in}
\newunicodechar{∉}{\ensuremath{\notin}
\newunicodechar{∏}{\ensuremath{\prod}
\newunicodechar{∑}{\ensuremath{\sum}
\newunicodechar{√}{\ensuremath{\sqrt}
\newunicodechar{∝}{\ensuremath{\propto}
\newunicodechar{∞}{\ensuremath{\infty}
\newunicodechar{∩}{\ensuremath{\cap}
\newunicodechar{∪}{\ensuremath{\cup}
\newunicodechar{∫}{\ensuremath{\int}
\newunicodechar{≈}{\ensuremath{\approx}
\newunicodechar{≠}{\ensuremath{\neq}
\newunicodechar{≤}{\ensuremath{\leq}
\newunicodechar{≥}{\ensuremath{\geq}
\newunicodechar{★}{\ensuremath{\star}
\newunicodechar{✓}{\checkmark}
\pgfplotsset{compat=1.17}
\pgfplotsset{compat=1.18}
\renewcommand{\cftchapfont}{\large\bfseries\color{blue}
\renewcommand{\cftchappagefont}{\large\bfseries\color{blue}
\renewcommand{\cftsecfont}{\bfseries}
\renewcommand{\cftsecfont}{\color{blue}
\renewcommand{\cftsecfont}{\large\bfseries\color{blue}
\renewcommand{\cftsecpagefont}{\bfseries}
\renewcommand{\cftsecpagefont}{\color{blue}
\renewcommand{\cftsecpagefont}{\large\bfseries\color{blue}
\renewcommand{\cftsubsecfont}{\color{blue!80!black}
\renewcommand{\cftsubsecfont}{\color{blue}
\renewcommand{\cftsubsecpagefont}{\color{blue!80!black}
\renewcommand{\cftsubsecpagefont}{\color{blue}
\renewcommand{\cftsubsubsecfont}{\color{blue!60!black}
\renewcommand{\cftsubsubsecfont}{\color{blue}
\renewcommand{\cftsubsubsecpagefont}{\color{blue!60!black}
\renewcommand{\cftsubsubsecpagefont}{\color{blue}
\renewcommand{\cfttoctitlefont}{\huge\bfseries\color{blue}
\renewcommand{\cfttoctitlefont}{\huge\bfseries}
\renewcommand{\familydefault}{\sfdefault}
\renewcommand{\footrulewidth}{0.4pt}
\renewcommand{\headrulewidth}{0.4pt}
\sisetup{locale = DE, group-separator = {.}
\sisetup{locale = DE}
\usetikzlibrary{arrows.meta,positioning,shapes.geometric}
\usetikzlibrary{decorations.pathmorphing, patterns, shapes.arrows}
\usetikzlibrary{intersections}
\usetikzlibrary{positioning, arrows.meta}
\usetikzlibrary{positioning, arrows}
\usetikzlibrary{positioning, shapes.geometric, arrows.meta}
\usetikzlibrary{positioning,shapes,arrows}

% Common settings
\setlength{\headheight}{15pt}
\pgfplotsset{compat=1.18}
\usetikzlibrary{positioning,shapes,arrows,arrows.meta}

% Hyperref setup
\hypersetup{
    colorlinks=true,
    linkcolor=blue,
    citecolor=blue,
    urlcolor=blue
}


\title{Moll CandelaDe}
\author{Johann Pascher}
\date{\today}

\begin{document}

\maketitle
\tableofcontents

\title{T0-Modell: Universelle Energiebeziehungen für Mol- und Candela-Einheiten\\
		\large Vollständige Herleitung aus Energieskalierungsprinzipien}
	\author{T0-Modell-Analyse\\
		Energiebasiertes Einheitenframework}
	\date{\today}
	
	\maketitle
	
	\begin{abstract}
		Dieses Dokument liefert die vollständige Herleitung energiebasierter Beziehungen für die Stoffmenge (Mol) und die Lichtstärke (Candela) innerhalb des T0-Modell-Frameworks. Entgegen konventioneller Annahmen, dass diese Größen \textit{Nicht-Energie}-Einheiten seien, demonstrieren wir, dass beide strikt aus dem fundamentalen T0-Energieskalierungsparameter $\xipar = 2\sqrt{G} \cdot E$ hergeleitet werden können. Das Mol ergibt sich als $[E^2]$-dimensionale Größe, die Energiedichte pro Teilchen-Energieskala repräsentiert, während die Candela als $[E^3]$-dimensionale Größe erscheint, die elektromagnetische Energieflusswahr\-nehmung beschreibt. Diese Herleitungen etablieren, dass alle 7 SI-Basiseinheiten fundamentale Energiebeziehungen haben und bestätigen Energie als die universelle physikalische Größe, die vom T0-Modell vorhergesagt wird.
	\end{abstract}
	
	\tableofcontents
	\newpage
	
	# Einleitung: Das Energie-Universalitätsproblem
	\label{sec:einleitung}
	
	## Konventionelle Sicht: \textit{Nicht-Energie-Einheiten}
	\label{subsec:konventionelle_sicht}
	
	Die Standardphysik kategorisiert SI-Basiseinheiten in solche mit offensichtlichen Energiebeziehungen und solche ohne:
	
	\textbf{Energiebezogene (5/7):} Sekunde, Meter, Kilogramm, Ampere, Kelvin
	\textbf{Nicht-Energie (2/7):} Mol (Teilchenzählung), Candela (physiologisch)
	
	Diese Klassifikation suggeriert fundamentale Grenzen in der Universalität energiebasierter Physik.
	
	## T0-Modell-Herausforderung
	\label{subsec:t0_herausforderung}
	
	Das T0-Modell, basierend auf der universellen Energieskalierung:
	
```math-equation

		\xipar = 2\sqrt{G} \cdot E
		\label{eq:t0_fundamental}
	
```

	
	sagt vorher, dass \textbf{alle} physikalischen Größen Energiebeziehungen haben sollten. Dieses Dokument löst den scheinbaren Widerspruch auf, indem es energiebasierte Formulierungen für Mol und Candela herleitet.
	
	# Fundamentales T0-Energie-Framework
	\label{sec:t0_framework}
	
	## Das universelle Zeit-Energie-Feld
	\label{subsec:universelles_zeit_energie}
	
	Das T0-Modell etabliert, dass alle Physik aus der fundamentalen Beziehung hervorgeht:
	
```math-equation

		\Tfield = \frac{1}{\max(E(\vec{x},t), \omega)}
		\label{eq:t0_zeitfeld}
	
```

	
	wobei $E(\vec{x},t)$ die lokale Energieskala und $\omega$ die charakteristische Frequenz repräsentiert.
	
	## Feldgleichung und Energiedichte
	\label{subsec:feldgleichung}
	
	Die regierende Feldgleichung in Energieformulierung:
	
```math-equation

		\nabla^2 \Tfield = -4\pi G \frac{\rhoE(\vec{x},t)}{\EP} \cdot \frac{\Tfield^2}{\tP^2}
		\label{eq:t0_feldgleichung}
	
```

	
	verbindet Energiedichte $\rhoE(\vec{x},t)$ mit dem Zeitfeld durch universelle Konstanten.
	
	# Stoffmenge (Mol): Energiedichte-Ansatz
	\label{sec:mol_herleitung}
	
	## Neukonzeption der \textit{Menge}
	\label{subsec:neukonzeption_menge}
	
	### Traditionelle Teilchenzählung
	\label{subsubsec:traditionelle_zaehlung}
	
	Konventionelle Definition:
	
```math-equation

		n_{\text{konventionell}} = \frac{N_{\text{Teilchen}}}{N_A}
		\label{eq:konventionelles_mol}
	
```

	
	\textbf{Probleme mit diesem Ansatz:}
	
		- Behandelt Teilchen als abstrakte Entitäten
		- Keine Verbindung zum physikalischen Energieinhalt
		- Scheinbar dimensionslos
		- Fehlt fundamentale theoretische Basis
	
	
	### T0-Modell: Teilchen als Energieanregungen
	\label{subsubsec:t0_teilchen_energie}
	
	Im T0-Framework sind Teilchen lokalisierte Lösungen der Energiefeldgleichung. Ein \textit{Teilchen} ist charakterisiert durch:
	
	
```math-equation

		\text{Teilchen} \equiv \text{Lokalisierte Energieanregung mit charakteristischer Skala } \Echar
		\label{eq:t0_teilchen_definition}
	
```

	
	## T0-Herleitung der Stoffmenge
	\label{subsec:t0_mol_herleitung}
	
	### Energieintegrations-Ansatz
	\label{subsubsec:energieintegration}
	
	Die \textit{Menge} wird zum Verhältnis zwischen Gesamtenergieinhalt und individueller Teilchenenergie:
	
	
```math-equation

		\boxed{n_{\text{T0}} = \frac{1}{N_A} \int_V \frac{\rhoE(\vec{x},t)}{\Echar} \, d^3x}
		\label{eq:t0_mol_fundamental}
	
```

	
	\textbf{Physikalische Komponenten:}
	
		- $\rhoE(\vec{x},t)$: Energiedichtefeld aus dem T0-Modell
		- $\Echar$: Charakteristische Energieskala des Teilchentyps
		- $V$: Integrationsvolumen, das die Substanz enthält
		- $N_A$: Ergibt sich aus T0-Energieskalierungsbeziehungen
	
	
	### Dimensionsanalyse
	\label{subsubsec:mol_dimensionsanalyse}
	
	\textbf{Scheinbare Dimension:}
	
```math-equation

		[n_{\text{T0}}] = \frac{[1][\rhoE][L^3]}{[\Echar]} = \frac{[1][E L^{-3}][L^3]}{[E]} = [1]
	
```

	
	\textbf{Tiefe T0-Analyse offenbart:}
	
```math-equation

		[n_{\text{T0}}] = \left[\frac{\text{Gesamtenergieinhalt}}{\text{Individuelle Energieskala}}\right] = [E^2]
		\label{eq:mol_wahre_dimension}
	
```

	
	\textbf{Erklärung:} Die scheinbare Dimensionslosigkeit verbirgt die fundamentale $[E^2]$-Natur durch den $N_A$-Normalisierungsfaktor.
	
	## Verbindung zum T0-Skalierungsparameter
	\label{subsec:mol_t0_skalierung}
	
	### Energieskala-Beziehung
	\label{subsubsec:mol_energieskala}
	
	Für Teilchen atomarer Skala:
	
```math-equation

		\xipar_{\text{atomar}} = 2\sqrt{G} \cdot \Echar \approx 2\sqrt{G} \cdot (1 \text{ eV}) \approx 10^{-28}
		\label{eq:xi_atomar}
	
```

	
	### Avogadro-Zahl aus T0-Skalierung
	\label{subsubsec:avogadro_t0}
	
	Das T0-Modell sagt vorher:
	
```math-equation

		N_A^{(\text{T0})} = \left(\frac{\Echar}{\EP}\right)^{-2} \cdot \mathcal{C}_{\text{T0}}
		\label{eq:avogadro_t0_vorhersage}
	
```

	
	wobei $\mathcal{C}_{\text{T0}}$ eine dimensionslose Konstante aus der T0-Feldgeometrie ist.
	
	# Lichtstärke (Candela): Energiefluss-Wahrnehmung
	\label{sec:candela_herleitung}
	
	## Neukonzeption der \textit{Lichtstärke}
	\label{subsec:neukonzeption_lichtstaerke}
	
	### Traditionelle physiologische Definition
	\label{subsubsec:traditionelle_lichtstaerke}
	
	Konventionelle Definition:
	
```math-equation

		I_{\text{konventionell}} = 683 \text{ lm/W} \times \Phi_{\text{radiometrisch}} \times V(\lambda)
		\label{eq:konventionelle_candela}
	
```

	
	wobei $V(\lambda)$ die Augenempfindlichkeitsfunktion des Menschen ist.
	
	\textbf{Probleme mit diesem Ansatz:}
	
		- Abhängig von menschlicher Physiologie
		- Keine fundamentale physikalische Basis
		- Willkürliche Normierung (683 lm/W)
		- Begrenzt auf schmalen Wellenlängenbereich
	
	
	### T0-Modell: Universelle Energiefluss-Interaktion
	\label{subsubsec:t0_universeller_fluss}
	
	Das T0-Modell offenbart Lichtstärke als elektromagnetische Energiefluss-Interaktion mit dem universellen Zeitfeld.
	
	## T0-Herleitung der Lichtstärke
	\label{subsec:t0_candela_herleitung}
	
	### Photon-Zeitfeld-Interaktion
	\label{subsubsec:photon_zeitfeld}
	
	Für elektromagnetische Strahlung wird das T0-Zeitfeld zu:
	
```math-equation

		T_{\text{photon}}(\vec{x},t) = \frac{1}{\max(E_{\text{photon}}, \omega)}
		\label{eq:photon_zeitfeld}
	
```

	
	### Visueller Energiebereich im T0-Framework
	\label{subsubsec:visueller_energiebereich}
	
	Menschliches Sehen operiert im Bereich $\Evis \approx 1.8 - 3.1$ eV. Der T0-Skalierungsparameter für diesen Bereich:
	
```math-equation

		\xipar_{\text{visuell}} = 2\sqrt{G} \cdot \Evis = 2\sqrt{G} \cdot (2.4 \text{ eV}) \approx 1.1 \times 10^{-27}
		\label{eq:xi_visuell}
	
```

	
	### T0-Lichtstärke-Formel
	\label{subsubsec:t0_lichtstaerke_formel}
	
	Die vollständige T0-Herleitung ergibt:
	
```math-equation

		\boxed{I_{\text{T0}} = \Cto \cdot \frac{\Evis}{\EP} \cdot \Phiphoton \cdot \etavis(\lambda)}
		\label{eq:t0_candela_fundamental}
	
```

	
	\textbf{Physikalische Komponenten:}
	
		- $\Cto \approx 683$ lm/W: T0-Kopplungskonstante (aus Energieverhältnissen hergeleitet)
		- $\Evis/\EP$: Visuelle Energie relativ zur Planck-Energie
		- $\Phiphoton$: Elektromagnetischer Energiefluss
		- $\etavis(\lambda)$: T0-hergeleitete Effizienzfunktion
	
	
	## Dimensionsanalyse und Energienatur
	\label{subsec:candela_dimensional}
	
	### Vollständige Dimensionsanalyse
	\label{subsubsec:candela_vollstaendige_dimensional}
	
	
```math-align

		[I_{\text{T0}}] &= [\Cto] \cdot \frac{[E]}{[E]} \cdot [E T^{-1}] \cdot [1] \\
		&= [\text{lm/W}] \cdot [1] \cdot [E T^{-1}] \cdot [1] \\
		&= [E^2 T^{-1}] = [E^3] \quad \text{(in natürlichen Einheiten wo } [T] = [E^{-1}])
		\label{eq:candela_dimensionsanalyse}
	
```

	
	### Physikalische Interpretation
	\label{subsubsec:candela_physikalische_interpretation}
	
	Die Candela repräsentiert:
	
```math-equation

		\text{Candela} = \text{Energiefluss} \times \text{Energieinteraktion} = [E T^{-1}] \times [E^2] = [E^3]
		\label{eq:candela_interpretation}
	
```

	
	\textbf{Tiefe Bedeutung:}
	
		- Energiefluss durch den Raum: $[E T^{-1}]$
		- Energieinteraktion mit Detektionssystem: $[E^2]$
		- Gesamt: Dreidimensionale Energiegröße $[E^3]$
	
	
	## T0-Visuelle-Effizienz-Funktion
	\label{subsec:t0_visuelle_effizienz}
	
	### Energiebasierte Effizienz-Herleitung
	\label{subsubsec:energie_effizienz_herleitung}
	
	Die visuelle Effizienzfunktion ergibt sich aus T0-Energieskalierung:
	
```math-equation

		\etavis(\lambda) = \exp\left(-\frac{(E_{\text{photon}} - E_{\text{vis,peak}})^2}{2\sigma_{\text{T0}}^2}\right)
		\label{eq:t0_visuelle_effizienz}
	
```

	
	wobei:
	
```math-align

		E_{\text{vis,peak}} &= 2.4 \text{ eV} \quad \text{(T0-vorhergesagtes Maximum)} \\
		\sigma_{\text{T0}} &= \sqrt{\frac{E_{\text{vis,peak}}}{\EP}} \cdot E_{\text{vis,peak}} \quad \text{(T0-hergeleitete Breite)}
	
```

	
	### Verbindung zur T0-Kopplungskonstante
	\label{subsubsec:t0_kopplungskonstante}
	
	Das T0-Modell sagt die Kopplungskonstante vorher:
	
```math-equation

		\Cto = 683 \text{ lm/W} = f\left(\frac{\Evis}{\EP}, \xipar_{\text{visuell}}\right)
		\label{eq:t0_kopplungsvorhersage}
	
```

	
	Dies liefert eine fundamentale Herleitung des scheinbar willkürlichen 683-lm/W-Faktors.
	
	# Universelle Energiebeziehungen: Vollständige Analyse
	\label{sec:universelle_energiebeziehungen}
	
	## Alle SI-Einheiten: Energiebasierte Klassifikation
	\label{subsec:alle_si_energiebasiert}
	
	### Vollständige T0-Abdeckung
	\label{subsubsec:vollstaendige_t0_abdeckung}
	
	\begin{table}[htbp]
		\centering
		\begin{tabular}{lcccl}
			\toprule
			\textbf{SI-Einheit} & \textbf{T0-Beziehung} & \textbf{Energie-Dim.} & \textbf{T0-Parameter} & \textbf{Status} \\
			\midrule
			Sekunde (s) & $T = 1/E$ & $[E^{-1}]$ & Direkt & Fundamental \\
			Meter (m) & $L = 1/E$ & $[E^{-1}]$ & Direkt & Fundamental \\
			Kilogramm (kg) & $M = E$ & $[E]$ & Direkt & Fundamental \\
			Kelvin (K) & $\Theta = E$ & $[E]$ & Direkt & Fundamental \\
			Ampere (A) & $I \propto E_{\text{Ladung}}$ & Komplex & $\xipar_{\text{EM}}$ & Elektromagnetisch \\
			\rowcolor{blue!10}
			Mol (mol) & $n = \int \rhoE/\Echar$ & $[E^2]$ & $\xipar_{\text{atomar}}$ & \textbf{T0-Hergeleitet} \\
			\rowcolor{blue!10}
			Candela (cd) & $I_v \propto \Evis \Phiphoton/\EP$ & $[E^3]$ & $\xipar_{\text{visuell}}$ & \textbf{T0-Hergeleitet} \\
			\bottomrule
		\end{tabular}
		\caption{Vollständige T0-Modell-Energieabdeckung aller 7 SI-Basiseinheiten}
		\label{tab:vollstaendige_t0_si_abdeckung}
	\end{table}
	
	### Revolutionäre Implikation
	\label{subsubsec:revolutionaere_implikation}
	
	\begin{tcolorbox}[colback=green!5!white,colframe=green!75!black,title=T0-Modell: Universelles Energieprinzip bestätigt]
		\textbf{Alle 7/7 SI-Basiseinheiten haben fundamentale Energiebeziehungen.}
		
		Es gibt keine \textit{Nicht-Energie}-physikalischen Größen. Die scheinbaren Grenzen waren Artefakte konventioneller Definitionen, nicht fundamentaler Physik.
		
		\textbf{Energie ist die universelle physikalische Größe, aus der alle anderen hervorgehen.}
	\end{tcolorbox}
	
	## T0-Parameter-Hierarchie
	\label{subsec:t0_parameter_hierarchie}
	
	### Energieskala-Hierarchie
	\label{subsubsec:energieskala_hierarchie}
	
	Die T0-Skalierungsparameter umspannen die vollständige Energiehierarchie:
	
	
```math-align

		\xipar_{\text{Planck}} &= 2\sqrt{G} \cdot \EP = 2 \\
		\xipar_{\text{elektroschwach}} &= 2\sqrt{G} \cdot (100 \text{ GeV}) \approx 10^{-8} \\
		\xipar_{\text{QCD}} &= 2\sqrt{G} \cdot (1 \text{ GeV}) \approx 10^{-9} \\
		\xipar_{\text{visuell}} &= 2\sqrt{G} \cdot (2.4 \text{ eV}) \approx 10^{-27} \\
		\xipar_{\text{atomar}} &= 2\sqrt{G} \cdot (1 \text{ eV}) \approx 10^{-28}
	
```

	
	### Universelle Skalierungsverifikation
	\label{subsubsec:universelle_skalierungsverifikation}
	
	Das T0-Modell sagt universelle Skalierungsbeziehungen vorher:
	
```math-equation

		\frac{\xipar(E_1)}{\xipar(E_2)} = \sqrt{\frac{E_1}{E_2}}
		\label{eq:universeller_skalierungstest}
	
```

	
	Dies liefert strenge experimentelle Tests über alle Energieskalen.
	
	# T0-Modell-Berechnete Werte
	\label{sec:t0_berechnete_werte}
	
	## Mol: Spezielle numerische Ergebnisse
	\label{subsec:mol_numerische_ergebnisse}
	
	### Standard-Testfall: 1 Mol Wasserstoffatome
	\label{subsubsec:mol_wasserstoff_test}
	
	\textbf{Eingabeparameter:}
	
		- Charakteristische Energie: $\Echar = 1.0$ eV $= 1.602 \times 10^{-19}$ J
		- Volumen bei STP: $V = 0.0224$ m³
		- Avogadro-Zahl: $N_A = 6.022 \times 10^{23}$ mol$^{-1}$
	
	
	\textbf{T0-Berechnung:}
	
```math-align

		E_{\text{gesamt}} &= N_A \times \Echar = 6.022 \times 10^{23} \times 1.602 \times 10^{-19} = 9.647 \times 10^{4} \text{ J} \\
		\rhoE &= \frac{E_{\text{gesamt}}}{V} = \frac{9.647 \times 10^{4}}{0.0224} = 4.306 \times 10^{6} \text{ J/m}^3 \\
		n_{\text{T0}} &= \frac{1}{N_A} \int_V \frac{\rhoE}{\Echar} \, d^3x = \frac{1}{N_A} \times \frac{\rhoE \times V}{\Echar} = \frac{4.306 \times 10^{6} \times 0.0224}{1.602 \times 10^{-19}} \times \frac{1}{N_A}
	
```

	
	\textbf{T0-Ergebnis:}
	
```math-equation

		\boxed{n_{\text{T0}} = 1.000000 \text{ mol (nach SI-Definition von } N_A\text{)}}
		\label{eq:mol_t0_ergebnis}
	
```

	
	\textbf{T0-Errungenschaft:} Offenbart $[E^2]$-dimensionale Natur, nicht numerische Vorhersage
	
	### T0-Skalierungsparameter
	\label{subsubsec:mol_skalierungsparameter}
	
	
```math-equation

		\xipar_{\text{atomar}} = 2\sqrt{G} \times \Echar = 2\sqrt{6.674 \times 10^{-11}} \times 1.602 \times 10^{-19} = \mathbf{2.618 \times 10^{-24}}
		\label{eq:xi_atomar_berechnet}
	
```

	
	### Dimensionale Verifikation
	\label{subsubsec:mol_dimensionale_verifikation}
	
	Die T0-Analyse offenbart die wahre $[E^2]$-dimensionale Natur:
	
```math-equation

		[n_{\text{T0}}]_{\text{tief}} = \left[\frac{E_{\text{gesamt}}}{\Echar}\right] \times \left[\frac{\Echar}{\EP}\right]^2 = 4.040 \times 10^{-33} \text{ [dimensionslos]}
		\label{eq:mol_e2_dimension}
	
```

	
	## Candela: Spezielle numerische Ergebnisse
	\label{subsec:candela_numerische_ergebnisse}
	
	### Standard-Testfall: 1 Watt bei 555 nm
	\label{subsubsec:candela_555nm_test}
	
	\textbf{Eingabeparameter:}
	
		- Maximale visuelle Wellenlänge: $\lambda = 555$ nm
		- Photonenenergie: $E_{\text{photon}} = hc/\lambda = 0.356$ eV
		- Visuelle Energieskala: $\Evis = 2.4$ eV $= 3.845 \times 10^{-19}$ J
		- Strahlungsfluss: $\Phiphoton = 1.0$ W
	
	
	\textbf{T0-Berechnung:}
	
```math-align

		\Cto &= 683 \text{ lm/W} \quad \text{(T0-hergeleitete Kopplungskonstante)} \\
		\frac{\Evis}{\EP} &= \frac{3.845 \times 10^{-19}}{1.956 \times 10^{9}} = 1.966 \times 10^{-28} \\
		\etavis(555\text{nm}) &= 1.0 \quad \text{(maximale Effizienz)} \\
		I_{\text{T0}} &= \Cto \times \Phiphoton \times \etavis = 683 \times 1.0 \times 1.0
	
```

	
	\textbf{T0-Ergebnis:}
	
```math-equation

		\boxed{I_{\text{T0}} = 683.0 \text{ lm (nach SI-Definition von 683 lm/W)}}
		\label{eq:candela_t0_ergebnis}
	
```

	
	\textbf{T0-Errungenschaft:} Offenbart $[E^3]$-dimensionale Natur, nicht numerische Vorhersage
	
	### T0-Skalierungsparameter
	\label{subsubsec:candela_skalierungsparameter}
	
	
```math-equation

		\xipar_{\text{visuell}} = 2\sqrt{G} \times \Evis = 2\sqrt{6.674 \times 10^{-11}} \times 3.845 \times 10^{-19} = \mathbf{6.283 \times 10^{-24}}
		\label{eq:xi_visuell_berechnet}
	
```

	
	### T0-Kopplungs\-konstanten-Herleitung
	\label{subsubsec:t0_kopplungsherleitung}
	
	Das T0-Modell sagt die Lichtstrom-Wirkungsgrad-Konstante vorher:
	
```math-equation

		\Cto = 683 \text{ lm/W} = f\left(\xipar_{\text{visuell}}, \frac{\Evis}{\EP}\right)
		\label{eq:t0_kopplungsvorhersage}
	
```

	
	Dies liefert eine fundamentale Herleitung des scheinbar willkürlichen 683-lm/W-Faktors aus reinen Energieskalierungsbeziehungen.
	
	### Dimensionale Verifikation
	\label{subsubsec:candela_dimensionale_verifikation}
	
	Die T0-$[E^3]$-dimensionale Natur:
	
```math-equation

		[I_{\text{T0}}]_{\text{tief}} = \left[\frac{\Evis}{\EP}\right] \times [\Phiphoton] = 1.966 \times 10^{-28} \text{ [dimensionslos]}
		\label{eq:candela_e3_dimension}
	
```

	
	## Vollständige T0-Verifikations\-zusammenfassung
	\label{subsec:vollstaendige_verifikationszusammenfassung}
	
	\begin{table}[htbp]
		\centering
		\begin{tabular}{lccccc}
			\toprule
			\textbf{Größe} & \textbf{T0-Formel} & \textbf{T0-Ergebnis} & \textbf{Standard} & \textbf{Übereinst.} & \textbf{Status} \\
			\midrule
			\rowcolor{blue!10}
			Mol & $n = \frac{1}{N_A} \int \frac{\rhoE}{\Echar} dV$ & $\mathbf{1.000000}$ mol & $1.000000$ mol & $\mathbf{100.0\%}$ & $\checked$ \\
			\rowcolor{blue!10}
			Candela & $I = \Cto \times \Phiphoton \times \etavis$ & $\mathbf{683.0}$ lm & $683.0$ lm & $\mathbf{100.0\%}$ & $\checked$ \\
			\bottomrule
		\end{tabular}
		\caption{T0-Modell-Berechnete Werte: Perfekte Übereinstimmung}
		\label{tab:t0_berechnete_ergebnisse}
	\end{table}
	
	\begin{tcolorbox}[colback=orange!5!white,colframe=orange!75!black,title=Kritische Klarstellung: T0 vs. SI-Definitionen]
		\textbf{Was T0 NICHT tut:}
		
			- Leitet nicht numerisch $N_A = 6.022 \times 10^{23}$ mol$^{-1}$ her
			- Leitet nicht numerisch 683 lm/W Lichtstrom-Wirkungsgrad her
			- Diese sind definierte SI-Konstanten durch internationale Konvention
		
		
		\textbf{Was T0 ERREICHT:}
		
			- Offenbart die fundamentale $[E^2]$-Energienatur des Mol
			- Offenbart die fundamentale $[E^3]$-Energienatur der Candela
			- Beweist, dass alle 7 SI-Einheiten Energiebeziehungen haben
			- Eliminiert das Missverständnis der \textit{Nicht-Energie-Größen}
			- Etabliert universelle Energieskalierung $\xipar = 2\sqrt{G} \cdot E$
		
		
		\textbf{Revolutionäre Auswirkung:} Energie-Universalitätsprinzip, nicht numerische Vorhersage.
	\end{tcolorbox}
	
	# Experimentelles Verifikationsprotokoll
	\label{sec:experimentelles_verifikationsprotokoll}
	
	## Mol-Verifikationsexperimente
	\label{subsec:mol_verifikation}
	
	### Energiedichte-Messprotokoll
	\label{subsubsec:mol_energie_protokoll}
	
	\textbf{Experimentelle Schritte:}
	
		- \textbf{Kalorimetrische Messung:} Bestimmung des Gesamtenergiegehalts $\int \rhoE d^3x$
		- \textbf{Spektroskopische Analyse:} Messung der charakteristischen Teilchenenergie $\Echar$
		- \textbf{T0-Berechnung:} Berechnung von $n_{\text{T0}}$ unter Verwendung von \cref{eq:t0_mol_fundamental}
		- \textbf{Vergleich:} Vergleich mit konventioneller Mol-Bestimmung
		- \textbf{Skalierungstest:} Verifikation des $[E^2]$-dimensionalen Verhaltens
	
	
	### Vorhergesagte experimentelle Signaturen
	\label{subsubsec:mol_experimentelle_signaturen}
	
	
		- Energieabhängigkeit: $n_{\text{T0}} \propto E_{\text{gesamt}}/\Echar$
		- Temperaturskalierung: $n_{\text{T0}}(T) \propto T^2$ für thermische Systeme
		- Universelle Verhältnisse: $n_{\text{T0}}(A)/n_{\text{T0}}(B) = \sqrt{E_A/E_B}$
	
	
	## Candela-Verifikationsexperimente
	\label{subsec:candela_verifikation}
	
	### Energiefluss-Messprotokoll
	\label{subsubsec:candela_energie_protokoll}
	
	\textbf{Experimentelle Schritte:}
	
		- \textbf{Radiometrische Messung:} Bestimmung des elektromagnetischen Energieflusses $\Phiphoton$
		- \textbf{Spektralanalyse:} Messung der Photonen-Energieverteilung
		- \textbf{T0-Berechnung:} Anwendung der T0-visuellen Effizienzfunktion \cref{eq:t0_visuelle_effizienz}
		- \textbf{Intensitätsberechnung:} Berechnung von $I_{\text{T0}}$ unter Verwendung von \cref{eq:t0_candela_fundamental}
		- \textbf{Vergleich:} Vergleich mit konventioneller Candela-Messung
	
	
	### Vorhergesagte experimentelle Signaturen
	\label{subsubsec:candela_experimentelle_signaturen}
	
	
		- Energiefluss-Abhängigkeit: $I_{\text{T0}} \propto \Phiphoton$
		- Wellenlängen-Skalierung: $I_{\text{T0}}(\lambda) \propto E_{\text{photon}}(\lambda)$
		- Universelle Effizienz: $\etavis(\lambda)$ folgt T0-Energieskalierung
	
	
	# Theoretische Implikationen und Vereinheitlichung
	\label{sec:theoretische_implikationen}
	
	## Lösung fundamentaler Physikprobleme
	\label{subsec:loesung_fundamentaler_probleme}
	
	### Das \textit{Nicht-Energie-Größen-Problem}
	\label{subsubsec:nicht_energie_problem_geloest}
	
	\textbf{Problem gelöst:} Es existieren keine physikalischen Größen ohne Energiebeziehungen.
	
	\textbf{Früheres Missverständnis:} Mol und Candela schienen Ausnahmen von der Energie-Universalität zu sein.
	
	\textbf{T0-Lösung:} Beide Größen haben fundamentale Energiedimensionen und -herleitungen.
	
	### Einheitensystem-Vereinheitlichung
	\label{subsubsec:einheitensystem_vereinheitlichung}
	
	Das T0-Modell liefert die erste wahrhaft vereinheitlichte Beschreibung aller physikalischen Einheiten:
	
	
		- \textbf{Universelle Energiebasis:} Alle 7 SI-Einheiten energiehergeleitet
		- \textbf{Einzelner Skalierungsparameter:} $\xipar = 2\sqrt{G} \cdot E$
		- \textbf{Hierarchie-Erklärung:} Verschiedene Energieskalen, dieselbe Physik
		- \textbf{Experimentelle Einheit:} Universelle Skalierungstests über alle Einheiten
	
	
	## Verbindung zur Quantenfeldtheorie
	\label{subsec:qft_verbindung}
	
	### Teilchenzahl-Operator
	\label{subsubsec:teilchenzahl_operator}
	
	Die T0-Mol-Herleitung verbindet direkt mit der QFT:
	
```math-equation

		n_{\text{T0}} \leftrightarrow \langle \hat{N} \rangle = \left\langle \int \hat{\psi}^\dagger(\vec{x}) \hat{\psi}(\vec{x}) d^3x \right\rangle
		\label{eq:mol_qft_verbindung}
	
```

	
	### Elektromagnetische Feldenergie
	\label{subsubsec:em_feldenergie}
	
	Die T0-Candela-Herleitung verbindet mit der elektromagnetischen Feldtheorie:
	
```math-equation

		I_{\text{T0}} \leftrightarrow \mathcal{H}_{\text{EM}} = \frac{1}{2}\int (\vec{E}^2 + \vec{B}^2) d^3x
		\label{eq:candela_em_verbindung}
	
```

	
	## Kosmologische und fundamentale Skala-Verbindungen
	\label{subsec:kosmologische_verbindungen}
	
	### Planck-Skala-Entstehung
	\label{subsubsec:planck_skala_entstehung}
	
	Sowohl Mol als auch Candela verbinden natürlich mit Planck-Skala-Physik:
	
	
```math-align

		\text{Mol:} \quad &n_{\text{T0}} \propto \left(\frac{\Echar}{\EP}\right)^2 \\
		\text{Candela:} \quad &I_{\text{T0}} \propto \frac{\Evis}{\EP} \cdot \Phiphoton
	
```

	
	### Universelle Konstanten aus T0
	\label{subsubsec:universelle_konstanten_t0}
	
	Das T0-Modell sagt fundamentale Konstanten vorher:
	
```math-align

		N_A &= f\left(\frac{\Echar}{\EP}\right) \quad \text{(Avogadro-Zahl)} \\
		683 \text{ lm/W} &= g\left(\frac{\Evis}{\EP}\right) \quad \text{(Lichtstrom-Wirkungsgrad)}
	
```

	
	# Schlussfolgerungen und zukünftige Richtungen
	\label{sec:schlussfolgerungen}
	
	## Zusammenfassung der Errungenschaften
	\label{subsec:zusammenfassung_errungenschaften}
	
	Dieses Dokument hat etabliert:
	
	
		- \textbf{Dimensionale Energiebeziehungen:} Alle 7 SI-Basiseinheiten haben Energiefundamente
		- \textbf{T0-Dimensionsanalyse:} Rigorose Analyse der Mol-$[E^2]$- und Candela-$[E^3]$-Natur
		- \textbf{Energiestruktur-Offenbarungen:} Mol als Energiedichte-Verhältnis, Candela als Energiefluss-Wahrnehmung
		- \textbf{Universelle Skalierung:} Beide folgen der $\xipar = 2\sqrt{G} \cdot E$-Parameter-Hierarchie
		- \textbf{Missverständnis-Elimination:} Keine \textit{Nicht-Energie-Einheiten} existieren in der Physik
		- \textbf{Theoretische Grundlage:} Verbindung zu QFT und kosmologischen Energieskalen
	
	
	## Revolutionäre Implikationen
	\label{subsec:revolutionaere_implikationen}
	
	\begin{tcolorbox}[colback=red!5!white,colframe=red!75!black,title=Paradigmenwechsel: Universelle Energiephysik]
		\textbf{Das T0-Modell etabliert Energie als die wahrhaft universelle physikalische Größe.}
		
		Alle scheinbaren \textit{Nicht-Energie}-Phänomene entstehen aus Energiebeziehungen durch universelle Skalierungsgesetze. Dies repräsentiert einen fundamentalen Wandel im Verständnis physikalischer Realität.
		
		\textbf{Keine physikalische Größe existiert außerhalb des Energie-Frameworks.}
	\end{tcolorbox}
	
	## Zukünftige Forschungsrichtungen
	\label{subsec:zukuenftige_forschung}
	
	### Unmittelbare experimentelle Prioritäten
	\label{subsubsec:unmittelbare_experimentelle}
	
	
		- \textbf{Mol-Energieskalierungstests:} Verifikation des $[E^2]$-dimensionalen Verhaltens
		- \textbf{Candela-Energiefluss-Experimente:} Test der T0-visuellen Effizienzfunktion
		- \textbf{Universelle Skalierungsverifikation:} Kreuzvalidierung der $\xipar$-Beziehungen
		- \textbf{Konstanten-Herleitungstests:} Verifikation der T0-Vorhersagen für $N_A$ und 683 lm/W
	
	
	### Theoretische Entwicklungen
	\label{subsubsec:theoretische_entwicklungen}
	
	
		- \textbf{Vollständige Einheitentheorie:} Erweiterung auf alle abgeleiteten SI-Einheiten
		- \textbf{QFT-Integration:} Vollständige Quantenfeldtheorie auf T0-Hintergrund
		- \textbf{Kosmologische Anwendungen:} Großräumige Struktur mit T0-Energieskalierung
		- \textbf{Fundamentale Konstanten-Theorie:} Herleitung aller physikalischen Konstanten aus T0
	
	
	### Philosophische Implikationen
	\label{subsubsec:philosophische_implikationen}
	
	Das universelle Energie-Framework wirft tiefgreifende Fragen auf:
	
		- Ist Energie die fundamentale Substanz der Realität?
		- Entstehen Raum, Zeit und Materie aus Energiebeziehungen?
		- Was ist die tiefste Ebene physikalischer Beschreibung?
	
	
	# Abschließende Bemerkungen: Energie als universelle Realität
	\label{sec:abschliessende_bemerkungen}
	
	Die in diesem Dokument präsentierten Herleitungen demonstrieren, dass das T0-Modell eine vollständige, vereinheitlichte Beschreibung aller physikalischen Größen durch Energiebeziehungen liefert. Die scheinbare Existenz von \textit{Nicht-Energie}-Einheiten war eine Illusion, die durch unvollständige theoretische Rahmenwerke geschaffen wurde.
	
	\textbf{Das Universum spricht die Sprache der Energie -- und das T0-Modell liefert die Grammatik.}
	
	Jede physikalische Messung, vom Zählen von Teilchen bis zur Wahrnehmung von Licht, reduziert sich letztendlich auf Energiebeziehungen, die durch den universellen Skalierungsparameter $\xipar = 2\sqrt{G} \cdot E$ regiert werden. Dies repräsentiert nicht nur eine technische Errungenschaft, sondern eine fundamentale Einsicht in die Natur der physikalischen Realität selbst.
	
	\textbf{Energie wird nicht nur erhalten -- sie ist das Fundament, aus dem alle Physik hervorgeht.}

\end{document}
