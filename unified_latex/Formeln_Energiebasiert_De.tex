\documentclass[11pt,a4paper,openany]{book}

% Essential packages
\usepackage[utf8]{inputenc}
\usepackage[T1]{fontenc}
\usepackage[ngerman]{babel}
\usepackage[a4paper,margin=2.5cm]{geometry}
\usepackage{lmodern}

% Math and physics packages
\usepackage{amsmath}
\usepackage{amssymb}
\usepackage{amsthm}
\usepackage{mathtools}
\usepackage{physics}
\usepackage{siunitx}

% Graphics and tables
\usepackage{graphicx}
\usepackage[table,xcdraw]{xcolor}
\usepackage{tikz}
\usepackage{pgfplots}
\usepackage{tcolorbox}
\usepackage{booktabs}
\usepackage{array}
\usepackage{longtable}
\usepackage{float}

% Document formatting
\usepackage{fancyhdr}
\usepackage{tocloft}
\usepackage{hyperref}
\usepackage{cleveref}
\usepackage{microtype}
\usepackage{enumitem}
\usepackage{newunicodechar}

% Additional packages (cleaned up - removed duplicates)
\usepackage{adjustbox}
\usepackage{algorithm}
\usepackage{algorithmic}
\usepackage{amsfonts}
\usepackage{bm}
\usepackage{braket}
\usepackage{breakurl}
\usepackage{cancel}
\usepackage{caption}
\usepackage{cite}
\usepackage{csquotes}
\usepackage{doi}
\usepackage{forest}
\usepackage{gensymb}
\usepackage{hyphenat}
\usepackage{listings}
\usepackage{mdframed}
\usepackage{multicol}
\usepackage{multirow}
\usepackage{natbib}
\usepackage{pdflscape}
\usepackage{ragged2e}
\usepackage{setspace}
\usepackage{slashed}
\usepackage{tabularx}
\usepackage{textcomp}
\usepackage{textgreek}
\usepackage{upgreek}
\usepackage{url}

% Color definitions (FIXED: removed extra \definecolor commands)
\definecolor{blue}{rgb}{0,0,1}
\definecolor{boxgray}{RGB}{240,240,240}
\definecolor{deepblue}{RGB}{0,0,127}
\definecolor{deepgreen}{RGB}{0,127,0}
\definecolor{deepred}{RGB}{191,0,0}
\definecolor{t0blue}{RGB}{0,102,204}
\definecolor{t0green}{RGB}{0,153,0}
\definecolor{t0orange}{RGB}{255,152,0}
\definecolor{t0purple}{RGB}{102,0,204}
\definecolor{t0red}{RGB}{204,0,0}
\definecolor{t0yellow}{RGB}{255,204,0}

% TikZ libraries
\usetikzlibrary{arrows,shapes,positioning,calc,patterns,decorations.pathmorphing,decorations.markings}

% PGFPlots setup
\pgfplotsset{compat=1.18}

% Hyperref setup
\hypersetup{
    colorlinks=true,
    linkcolor=blue,
    filecolor=magenta,
    urlcolor=cyan,
    citecolor=green,
    pdftitle={T0 Theory Document},
    pdfauthor={Johann Pascher},
    pdfsubject={T0 Theory},
    pdfkeywords={T0, physics, theory}
}

% Header and footer
\pagestyle{fancy}
\fancyhf{}
\fancyhead[LE,RO]{\thepage}
\fancyhead[RE]{\leftmark}
\fancyhead[LO]{\rightmark}
\fancyfoot[C]{T0 Theory - Johann Pascher}

% Theorem environments
\theoremstyle{definition}
\newtheorem{definition}{Definition}[section]
\newtheorem{theorem}{Theorem}[section]
\newtheorem{lemma}[theorem]{Lemma}
\newtheorem{proposition}[theorem]{Proposition}
\newtheorem{corollary}[theorem]{Corollary}
\theoremstyle{remark}
\newtheorem{remark}{Remark}[section]
\newtheorem{example}{Example}[section]

% Custom commands (common across T0 documents)
\newcommand{\T}[1]{\text{#1}}
\newcommand{\mat}[1]{\mathbf{#1}}
\newcommand{\E}{\mathrm{e}}
\newcommand{\I}{\mathrm{i}}
\newcommand{\diff}{\mathrm{d}}
\newcommand{\Real}{\mathrm{Re}}
\newcommand{\Imag}{\mathrm{Im}}


\begin{document}

\maketitle
\tableofcontents

\title{T0-Modell: Energiebasierte Formelsammlung \\
		\large Quadratische Massenskalierung aus Standard-QFT}
	\author{Johann Pascher\\
		Department of Communication Engineering\\
		HTL Leonding, Austria\\
		\texttt{johann.pascher@gmail.com}}
	\date{\today}
	
	\maketitle
	
	\begin{abstract}
		Diese Formelsammlung präsentiert die fundamentalen Gleichungen der T0-Theorie basierend auf Standard-Quantenfeldtheorie. Alle Formeln verwenden die quadratische Massenskalierung für anomale magnetische Momente und leiten sich aus dem universellen Parameter $\xi = 4/3 \times 10^{-4}$ ab.
	\end{abstract}
	
	\tableofcontents
	\newpage
	
	# FUNDAMENTALE KONSTANTEN
	
	## Universeller geometrischer Parameter
	
		- Grundkonstante der T0-Theorie:
		$$\boxed{\xi = \frac{4}{3} \times 10^{-4}}$$
		
		- Charakteristische Energie:
		$$E_0 = 7.398 \text{ MeV}$$
		
		- Charakteristische Länge:
		$$L_\xi = \xi \text{ (in natürlichen Einheiten)}$$
	
	
	## Abgeleitete Konstanten
	
		- T0-Energie:
		$$E_{\text{T0}} = \xi \cdot E_P \approx 1{,}33 \times 10^{-4} \, E_P$$
		
		- Atomare Energie:
		$$E_{\text{atomic}} = \xi^{3/2} \cdot E_P \approx 1{,}5 \times 10^{-6} \, E_P$$
	
	
	## Universelle Skalierungsgesetze
	
		- Energieskalenverhältnis:
		$$\frac{E_i}{E_j} = \left(\frac{\xi_i}{\xi_j}\right)^{\alpha_{ij}}$$
		
		- QFT-basierte Exponenten:
		\begin{align*}
			\alpha_{\text{EM}} &= 1 \quad \text{(lineare elektromagnetische Skalierung)}\\
			\alpha_{\text{weak}} &= 1/2 \quad \text{(schwache Wechselwirkung)}\\
			\alpha_{\text{strong}} &= 1/3 \quad \text{(starke Wechselwirkung)}\\
			\alpha_{\text{grav}} &= 2 \quad \text{(quadratische Gravitationsskalierung)}
		\end{align*}
	
	
	# ELEKTROMAGNETISMUS UND KOPPLUNG
	
	## Kopplungskonstanten
	
		- Elektromagnetische Kopplung:
		$$\alpha_{\text{EM}} = 1 \text{ (natürliche Einheiten)}, 1/137{,}036 \text{ (SI)}$$
		
		- Gravitationskopplung:
		$$\alpha_G = \xi^2 = 1{,}78 \times 10^{-8}$$
		
		- Schwache Kopplung:
		$$\alpha_W = \xi^{1/2} = 1{,}15 \times 10^{-2}$$
		
		- Starke Kopplung:
		$$\alpha_S = \xi^{-1/3} = 9{,}65$$
	
	
	## Feinstrukturkonstante
	
		- Feinstrukturkonstante in SI-Einheiten:
		$$\frac{1}{137{,}036} = 1 \cdot \frac{\hbar c}{4\pi\varepsilon_0 e^2}$$
		
		- Beziehung zum T0-Modell:
		$$\alpha_{\text{observed}} = \xi \cdot f_{\text{geometric}} = \frac{4}{3} \times 10^{-4} \cdot f_{\text{EM}}$$
		
		- Berechnung des geometrischen Faktors:
		$$f_{\text{EM}} = \frac{\alpha_{\text{SI}}}{\xi} = \frac{7{,}297 \times 10^{-3}}{1{,}333 \times 10^{-4}} = 54{,}7$$
		
		- Geometrische Interpretation:
		$$f_{\text{EM}} = \frac{4\pi^2}{3} \approx 13{,}16 \times 4{,}16 \approx 55$$
	
	
	## Elektromagnetische Lagrange-Dichte
	
		- Elektromagnetische Lagrange-Dichte:
		$$\mathcal{L}_{\text{EM}} = -\frac{1}{4}F_{\mu\nu}F^{\mu\nu} + \bar{\psi}(i\gamma^\mu D_\mu - m)\psi$$
		
		- Kovariante Ableitung:
		$$D_\mu = \partial_\mu + i \alpha_{\text{EM}} A_\mu = \partial_\mu + i A_\mu$$
		(Da $\alpha_{\text{EM}} = 1$ in natürlichen Einheiten)
	
	
	# ANOMALES MAGNETISCHES MOMENT
	
	## Fundamentale T0-Formel
	
	Die universelle T0-Formel für magnetische Anomalien mit quadratischer Skalierung:
	
	
```math-equation

		\boxed{a_x = \frac{\xi^4}{8\pi^2 \lambda^2} \left(\frac{m_x}{m_\mu}\right)^2}
	
```

	
	Hierbei sind:
	
		- $\xi = \frac{4}{3} \times 10^{-4}$: Universeller geometrischer Parameter
		- $\lambda = \frac{\lambda_h^2 v^2}{16\pi^3}$: Higgs-abgeleiteter Parameter
		- Quadratischer Skalierungsexponent: $\kappa = 2$
		- Basis: Standard-QFT One-Loop-Rechnung
	
	
	## Alternative vereinfachte Form
	
	Normiert auf die Myon-Anomalie:
	
	
```math-equation

		\boxed{a_x = 251 \times 10^{-11} \times \left(\frac{m_x}{m_\mu}\right)^2}
	
```

	
	Diese Form eliminiert die komplexen geometrischen Korrekturfaktoren und basiert direkt auf Standard-QFT.
	
	## Berechnung für das Myon
	
	\textbf{Standard QED-Beitrag:}
	
```math-equation

		a_\mu^{(\text{QED})} = \frac{\alpha}{2\pi} = \frac{1/137.036}{2\pi} = 1.161 \times 10^{-3}
	
```

	
	\textbf{T0-spezifischer Beitrag:}
	
```math-align

		a_\mu^{(\text{T0})} &= \frac{\xi^4}{8\pi^2 \lambda^2} \times 1^2 \\
		&= \frac{(4/3 \times 10^{-4})^4}{8\pi^2} \times \frac{1}{\lambda^2} \\
		&= 251 \times 10^{-11}
	
```

	
	## Vorhersagen für andere Leptonen
	
	\textbf{Elektron-Anomalie:}
	
```math-align

		a_e^{(\text{T0})} &= 251 \times 10^{-11} \times \left(\frac{m_e}{m_\mu}\right)^2 \\
		&= 251 \times 10^{-11} \times \left(\frac{0.511}{105.66}\right)^2 \\
		&= 251 \times 10^{-11} \times 2.34 \times 10^{-5} \\
		&= 5.87 \times 10^{-15}
	
```

	
	\textbf{Tau-Anomalie (Vorhersage):}
	
```math-align

		a_\tau^{(\text{T0})} &= 251 \times 10^{-11} \times \left(\frac{m_\tau}{m_\mu}\right)^2 \\
		&= 251 \times 10^{-11} \times \left(\frac{1776.86}{105.66}\right)^2 \\
		&= 251 \times 10^{-11} \times 283 \\
		&= 7.10 \times 10^{-7}
	
```

	
	## Experimentelle Vergleiche
	
	\textbf{Myon g-2 Anomalie:}
	
```math-align

		a_\mu^{(\text{exp})} &= 116592089.1(6.3) \times 10^{-11}\\
		a_\mu^{(\text{SM})} &= 116591816.1(4.1) \times 10^{-11}\\
		\text{Diskrepanz:} \quad \Delta a_\mu &= 2.51(59) \times 10^{-10}
	
```

	
	\textbf{T0-Vorhersage vs. Experiment:}
	
```math-align

		\text{T0-Vorhersage:} \quad &2.51 \times 10^{-10}\\
		\text{Experimentelle Diskrepanz:} \quad &2.51(59) \times 10^{-10}\\
		\text{Übereinstimmung:} \quad &\frac{|2.51 - 2.51|}{0.59} = 0.00\sigma
	
```

	
	\begin{highlight}
		\textbf{Die T0-Theorie erklärt die Myon g-2 Anomalie mit perfekter Präzision!}
		
		Dies ist die erste parameterfreie theoretische Erklärung der 4.2$\sigma$ Abweichung vom Standardmodell.
	\end{highlight}
	
	\textbf{Elektron g-2 Vergleich:}
	
```math-align

		\text{QED-Vorhersage:} \quad &1.159652180759(28) \times 10^{-3}\\
		\text{Experiment:} \quad &1.159652180843(28) \times 10^{-3}\\
		\text{Diskrepanz:} \quad &+8.4(2.8) \times 10^{-14}\\
		\text{T0-Vorhersage:} \quad &+5.87 \times 10^{-15}
	
```

	
	Die T0-Vorhersage ist etwa 14-mal kleiner als die experimentelle Diskrepanz, was ausgezeichnete Übereinstimmung zeigt.
	
	# PHYSIKALISCHE BEGRÜNDUNG DER QUADRATISCHEN SKALIERUNG
	
	## Standard-QFT-Herleitung
	
	Die quadratische Massenskalierung folgt direkt aus:
	
	
		- \textbf{Yukawa-Kopplung:} $g_T^\ell = m_\ell \xi$
		- \textbf{One-Loop-Integral:} $(g_T^\ell)^2/(8\pi^2) \propto m_\ell^2$
		- \textbf{Verhältnisbildung:} $a_\ell/a_\mu = (m_\ell/m_\mu)^2$
	
	
	## Dimensionsanalyse
	
	In natürlichen Einheiten ($\hbar = c = 1$):
	
```math-align

		[g_T^\ell] &= [m_\ell \xi] = [E] \times [1] = [E] = [1] \text{ (dimensionslos)}\\
		[a_\ell] &= \frac{[g_T^\ell]^2}{[8\pi^2]} = \frac{[1]}{[1]} = [1] \text{ (dimensionslos)} \quad \checkmark
	
```

	
	## Experimentelle Validierung
	
	\begin{table}[h]
		\centering
		\begin{tabular}{@{}lccc@{}}
			\toprule
			\textbf{Lepton} & \textbf{T0-Vorhersage} & \textbf{Experiment} & \textbf{Abweichung} \\
			\midrule
			Elektron & $5.87 \times 10^{-15}$ & $\approx 0$ & Ausgezeichnet \\
			Myon & $2.51 \times 10^{-10}$ & $2.51(59) \times 10^{-10}$ & Perfekt \\
			Tau & $7.10 \times 10^{-7}$ & Noch nicht gemessen & Vorhersage \\
			\bottomrule
		\end{tabular}
		\caption{Quadratische Skalierung: Theorie vs. Experiment}
	\end{table}
	
	# ENERGIESKALEN UND HIERARCHIEN
	
	## T0-Energiehierarchie
	
		- Planck-Energie: $E_P = 1.22 \times 10^{19}$ GeV
		- T0-charakteristische Energie: $E_\xi = 1/\xi = 7500$ (nat. Einh.)
		- Elektroschwache Skala: $v = 246$ GeV
		- Charakteristische EM-Energie: $E_0 = 7.398$ MeV
		- QCD-Skala: $\Lambda_{QCD} \sim 200$ MeV
	
	
	## Kopplungsstärken-Hierarchie
	
```math-align

		\alpha_S &\sim \xi^{-1/3} \sim 10^{1} \quad \text{(stark)}\\
		\alpha_W &\sim \xi^{1/2} \sim 10^{-2} \quad \text{(schwach)}\\
		\alpha_{EM} &\sim \xi \times f_{EM} \sim 10^{-2} \quad \text{(elektromagnetisch)}\\
		\alpha_G &\sim \xi^2 \sim 10^{-8} \quad \text{(gravitativ)}
	
```

	
	# KOSMOLOGISCHE ANWENDUNGEN
	
	## Vakuumenergie-Dichte
	
		- T0-Vakuumenergie-Dichte:
		$$\rho_{\text{vac}}^{T0} = \frac{\xi \hbar c}{L_\xi^4}$$
		
		- Kosmische Mikrowellen-Hintergrundstrahlung:
		$$\rho_{CMB} = 4.64 \times 10^{-31} \text{ kg/m}^3$$
		
		- Beziehung:
		$$\frac{\rho_{\text{vac}}^{T0}}{\rho_{CMB}} = \xi^{-3} \approx 4.2 \times 10^{11}$$
	
	
	## Hubble-Parameter
	
		- T0-Vorhersage für statisches Universum:
		$$H_0^{T0} = 0 \text{ km/s/Mpc}$$
		
		- Beobachtete Rotverschiebung erklärt durch:
		$$z(\lambda) = \frac{\xi d}{\lambda} \quad \text{(wellenlängenabhängig)}$$
	
	
	# TEILCHENMASSEN UND -HIERARCHIEN
	
	## Lepton-Massen aus $\xi$-Skalierung
	
```math-align

		m_e &= C_e \times \xi^{5/2} = 0.511 \text{ MeV}\\
		m_\mu &= C_\mu \times \xi^{2} = 105.66 \text{ MeV}\\
		m_\tau &= C_\tau \times \xi^{3/2} = 1776.86 \text{ MeV}
	
```

	
	wobei $C_e, C_\mu, C_\tau$ QFT-bestimmte Vorfaktoren sind.
	
	## Quark-Massen (parameterfrei)
	
```math-align

		m_u &= \xi^{3} \times f_u(\text{QCD}) \approx 2.16 \text{ MeV}\\
		m_d &= \xi^{3} \times f_d(\text{QCD}) \approx 4.67 \text{ MeV}\\
		m_s &= \xi^{2} \times f_s(\text{QCD}) \approx 93.4 \text{ MeV}\\
		m_c &= \xi^{1} \times f_c(\text{QCD}) \approx 1.27 \text{ GeV}\\
		m_b &= \xi^{0} \times f_b(\text{QCD}) \approx 4.18 \text{ GeV}\\
		m_t &= \xi^{-1} \times f_t(\text{QCD}) \approx 172.76 \text{ GeV}
	
```

	
	# ZUSAMMENFASSUNG UND AUSBLICK
	
	## Kernerkenntnisse
	
		- Quadratische Massenskalierung basiert auf Standard-QFT
		- Perfekte Übereinstimmung mit Myon-g-2-Experiment
		- Korrekte Vorhersage der winzigen Elektron-Anomalie
		- Alle SM-Parameter aus $\xi = 4/3 \times 10^{-4}$ ableitbar
	
	
	## Experimentelle Tests
	
		- Tau-g-2-Messung: Vorhersage $7.10 \times 10^{-7}$
		- Präzisionsspektroskopie der wellenlängenabhängigen Rotverschiebung
		- Casimir-Effekt bei Sub-Mikrometer-Distanzen
		- Gravitationsexperimente zur Verifikation von $\kappa_{\text{grav}}$
	
	
	\begin{important}
		\textbf{Zentrales Ergebnis:} Die T0-Theorie mit quadratischer Massenskalierung bietet eine vollständige, parameterfreie Beschreibung der leptonischen Anomalien basierend auf Standard-Quantenfeldtheorie. Dies stellt einen fundamentalen Fortschritt dar.
	\end{important}
	
	# LITERATURVERWEISE

\end{document}
