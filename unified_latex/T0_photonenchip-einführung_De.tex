\documentclass[11pt,a4paper,openany]{book}

% Essential packages
\usepackage[utf8]{inputenc}
\usepackage[T1]{fontenc}
\usepackage[ngerman]{babel}
\usepackage[a4paper,margin=2.5cm]{geometry}
\usepackage{lmodern}

% Math and physics packages
\usepackage{amsmath}
\usepackage{amssymb}
\usepackage{amsthm}
\usepackage{mathtools}
\usepackage{physics}
\usepackage{siunitx}

% Graphics and tables
\usepackage{graphicx}
\usepackage[table,xcdraw]{xcolor}
\usepackage{tikz}
\usepackage{pgfplots}
\usepackage{tcolorbox}
\usepackage{booktabs}
\usepackage{array}
\usepackage{longtable}
\usepackage{float}

% Document formatting
\usepackage{fancyhdr}
\usepackage{tocloft}
\usepackage{hyperref}
\usepackage{cleveref}
\usepackage{microtype}
\usepackage{enumitem}
\usepackage{newunicodechar}

% Additional packages (cleaned up - removed duplicates)
\usepackage{adjustbox}
\usepackage{algorithm}
\usepackage{algorithmic}
\usepackage{amsfonts}
\usepackage{bm}
\usepackage{braket}
\usepackage{breakurl}
\usepackage{cancel}
\usepackage{caption}
\usepackage{cite}
\usepackage{csquotes}
\usepackage{doi}
\usepackage{forest}
\usepackage{gensymb}
\usepackage{hyphenat}
\usepackage{listings}
\usepackage{mdframed}
\usepackage{multicol}
\usepackage{multirow}
\usepackage{natbib}
\usepackage{pdflscape}
\usepackage{ragged2e}
\usepackage{setspace}
\usepackage{slashed}
\usepackage{tabularx}
\usepackage{textcomp}
\usepackage{textgreek}
\usepackage{upgreek}
\usepackage{url}

% Color definitions (FIXED: removed extra \definecolor commands)
\definecolor{blue}{rgb}{0,0,1}
\definecolor{boxgray}{RGB}{240,240,240}
\definecolor{deepblue}{RGB}{0,0,127}
\definecolor{deepgreen}{RGB}{0,127,0}
\definecolor{deepred}{RGB}{191,0,0}
\definecolor{t0blue}{RGB}{0,102,204}
\definecolor{t0green}{RGB}{0,153,0}
\definecolor{t0orange}{RGB}{255,152,0}
\definecolor{t0purple}{RGB}{102,0,204}
\definecolor{t0red}{RGB}{204,0,0}
\definecolor{t0yellow}{RGB}{255,204,0}

% TikZ libraries
\usetikzlibrary{arrows,shapes,positioning,calc,patterns,decorations.pathmorphing,decorations.markings}

% PGFPlots setup
\pgfplotsset{compat=1.18}

% Hyperref setup
\hypersetup{
    colorlinks=true,
    linkcolor=blue,
    filecolor=magenta,
    urlcolor=cyan,
    citecolor=green,
    pdftitle={T0 Theory Document},
    pdfauthor={Johann Pascher},
    pdfsubject={T0 Theory},
    pdfkeywords={T0, physics, theory}
}

% Header and footer
\pagestyle{fancy}
\fancyhf{}
\fancyhead[LE,RO]{\thepage}
\fancyhead[RE]{\leftmark}
\fancyhead[LO]{\rightmark}
\fancyfoot[C]{T0 Theory - Johann Pascher}

% Theorem environments
\theoremstyle{definition}
\newtheorem{definition}{Definition}[section]
\newtheorem{theorem}{Theorem}[section]
\newtheorem{lemma}[theorem]{Lemma}
\newtheorem{proposition}[theorem]{Proposition}
\newtheorem{corollary}[theorem]{Corollary}
\theoremstyle{remark}
\newtheorem{remark}{Remark}[section]
\newtheorem{example}{Example}[section]

% Custom commands (common across T0 documents)
\newcommand{\T}[1]{\text{#1}}
\newcommand{\mat}[1]{\mathbf{#1}}
\newcommand{\E}{\mathrm{e}}
\newcommand{\I}{\mathrm{i}}
\newcommand{\diff}{\mathrm{d}}
\newcommand{\Real}{\mathrm{Re}}
\newcommand{\Imag}{\mathrm{Im}}


\begin{document}

\maketitle
\tableofcontents

\begin{abstract}
		Photonische integrierte Schaltkreise (PICs) revolutionieren die Nachrichtentechnik: Von latenzarmen RF-Filtern für 6G-Netze bis zu parallelen AI-Operationen in Data-Centern. \textbf{Die 6G-Standardisierung beginnt 2025, wobei photonische Komponenten der Schlüssel zur Erschließung des Terahertz (THz)-Frequenzbereichs für extrem hohe Datenraten sind \cite{6g_roadmap}.} Diese Einführung basiert auf aktueller Literatur (2024–2025) und beleuchtet analoge Realisierungsprinzipien (z.\,B. Interferenz via MZI), bevorzugte Operationen (Matrix-Multiplikation, Signal-Filterung) und Relevanz für Echtzeit-Kommunikation. Praxisnah: Tabelle zu Techniken, Ausblick auf Hybride Systeme. Quellen: Reviews aus Nature, SPIE und ScienceDirect. \textbf{Aktuelle Forschungen (EPFL/Harvard) haben einen revolutionären optoelektronischen Chip vorgestellt, der THz- und optische Signale auf einem Prozessor verarbeitet \cite{thz_epfl}.}
	\end{abstract}
	
	\tableofcontents
	\newpage
	
	# Grundlagen: Photonische Chips in der Nachrichtentechnik
	
	Photonische Quantenchips nutzen Lichtwellen für hochparallele, energieeffiziente Verarbeitung – essenziell für 6G (Bandbreiten $>\SI{100}{GHz}$, Latenz $<\SI{1}{ms}$). \textbf{Die Europäische Kommission hat den Start der 6G-Standardisierung für 2025 angekündigt, mit einem Fokus auf Souveränität und führender Technologieposition \cite{6g_roadmap}. Das Jahr 2025 wurde zudem von den Vereinten Nationen als das Internationale Jahr der Quantenwissenschaften (IYQ) ausgerufen, was die strategische Bedeutung der Photonik untermauert \cite{quantenjahr25}.} Im Gegensatz zu elektronischen CMOS-Chips (Wärme-Limits bei hohen Frequenzen) ermöglichen PICs analoge Signalverarbeitung durch optische Interferenz und Modulation, angelehnt an klassische analoge Optik (z.\,B. aus der RF-Technik der 1980er).
	\begin{important}
		Wichtiger Hinweis: Die Technik ist stark analog: Kontinuierliche Wellentransformationen (Phasenverschiebung, Diffraktion) dominieren, da Photonen intrinsisch parallel (Wellenlängen-Multiplexing) und latenzarm sind. Hybride Systeme (Photonik + Elektronik) ergänzen für Steuerung.
	\end{important}
	
	Aktuelle Trends (2025): Skalierbare Wafer (z.\,B. 6-Zoll-TFLN) für industrielle Einsätze in Data-Centern, mit $1000\times$-Speedup für AI-Workloads \cite{photonics_ai, qant_nps}.
	# Realisierung von Operationen: Analoge Prinzipien
	
	Operationen werden primär durch optische Bauteile realisiert, die analoge Verarbeitung priorisieren. Kernkomponenten:
	
	
		- \textbf{Mach-Zehnder-Interferometer (MZI)}: Für Phasenmodulation und lineare Transformationen; analoge Addition/Multiplikation via Interferenz.
		- \textbf{Wellenleiter und Modulatoren}: Elektro-optische (z.\,B. LiNbO$_3$) oder thermische Steuerung für kontinuierliche Signale.
		- \textbf{Monolithische Integration}: Co-Packaging auf Si- oder TFLN-Plattformen minimiert Verluste ($<\SI{1}{dB}$), ermöglicht dynamische Rekonfiguration.
	
	Die Technik lehnt sich an analoge RF-Systeme an: Statt diskreter Bits kontinuierliche Wellenfelder für Echtzeit-Filterung (z.\,B. Demodulation in 6G) \cite{analog_optical}.
	\begin{formula}
		Beispiel: Lineare Transformation (Matrix-Vektor-Multiplikation) via MZI-Mesh: $y = M \cdot x$, wobei $M$ durch Phasen $\phi_i$ programmiert wird: $\phi_i = \arg(M_{ij})$.
	\end{formula}
	
	# Bevorzugte Operationen für photonische Bauteile
	
	Photonische Chips eignen sich für lineare, frequenzabhängige und parallele Operationen, da analoge Kontinuität Energie spart ($\SI{}{\pico\joule}/\text{Bit}$) und Bandbreite maximiert. Basierend auf 2025-Reviews:
	
	\begin{table}[htbp]
		\centering
		\begin{tabular}{l p{6cm} p{4cm}}
			\toprule
			\textbf{Operation} & \textbf{Realisierung (analog)} & \textbf{Relevanz für Nachrichtentechnik} \\
			\midrule
			Matrix-Multiplikation (GEMM) & MZI-Arrays für Interferenz-basierte Addition/Multiplikation & AI-Training in Edge-Netzen (z.\,B. Transformer für 6G-Routing) \cite{photonics_ai} \\
			RF-Signal-Filterung & Optische Diffraktion/FFT via Wellenleiter & Demodulation, BSS in 5G/6G (Bandbreite $>\SI{100}{GHz}$) \cite{rf_photonics} \\
			Recurrent-Processing & Programmierte photonische Circuits (PPCs) für sequentielle Transformationen & Echtzeit-Überwachung in Netzen (z.\,B. RNNs für Anomalie-Erkennung) \cite{recurrent_photonics} \\
			Differential-Operationen & Meta-Optik für Gradienten (z.\,B. Edge-Detection) & Bild-/Signal-Enhancement in Optischen Netzen \cite{differential_optical} \\
			Parallele Optimierung & Korrelation via kohärente PICs & Gradient-Descent für Routing-Optimierung \cite{optical_advantages} \\
			\bottomrule
		\end{tabular}
		\caption{Bevorzugte Operationen auf photonischen Chips – Fokus auf analoge Techniken}
		\label{tab:operations}
	\end{table}
	
	Nicht bevorzugt: Nicht-lineare Logik (z.\,B. AND/OR), da Photonen linear sind; hier Hybride nötig.
	
	# Literaturübersicht: Aktuelle Entwicklungen (2024–2025)
	
	Basierend auf neuesten Reviews (offen zugänglich) und aktuellen Projekten:
	
	
		- \textbf{Analog optical computing: principles, progress, and prospects (2025)}: Überblick über analoge PICs; Fortschritte in rekonfigurierbaren Designs für Echtzeit-Signale \cite{analog_optical}.
		- \textbf{Integrierte Terahertz-Kommunikation:} Ein revolutionärer optoelektronischer Prozessor (EPFL/Harvard, 2025) integriert die Verarbeitung von \textbf{Terahertz-Wellen} und optischen Signalen auf einem Chip. Dieser Durchbruch ist entscheidend für 6G, da er Hochleistung ohne nennenswerten Energieverlust ermöglicht und mit bestehenden photonischen Technologien kompatibel ist \cite{thz_epfl}.
		- \textbf{Integrierte Photonik für 6G-Forschung:} Projekte wie \textbf{6G-ADLANTIK} und \textbf{6G-RIC} (Fraunhofer HHI) entwickeln photonisch-elektronische Integrationskomponenten, um den THz-Frequenzbereich für 6G zu erschließen und die Resilienz von Netzwerken zu verbessern (SUSTAINET) \cite{hhi_6g}.
		- \textbf{Integrated photonic recurrent processors (2025)}: Recurrent-Operationen via PPCs; Anwendungen in sequentieller Verarbeitung (z.\,B. Netzwerk-Überwachung) \cite{recurrent_photonics}.
		- \textbf{Photonics for sustainable AI (2025)}: GEMM als Kern für AI; photonische Vorteile für energiearme 6G-Inferenz \cite{photonics_ai}.
		- \textbf{All-optical analog differential operation... (2025)}: Meta-Optik für Differential-Computing; ideal für Signal-Enhancement \cite{differential_optical}.
		- \textbf{Harnessing optical advantages in computing: a review (2024)}: Parallele Vorteile; Fokus auf FFT und Korrelation für RF \cite{optical_advantages}.
	
	
	Diese Quellen betonen den Shift zu analogen Hybriden für 6G: Von Prototypen zu skalierbaren Wafern.
	
	# Ausblick: Photonik in 6G-Netzen
	
	Photonische Chips ermöglichen latenzarme, skalierbare Kommunikation: Z.\,B. optische BSS für Multi-User-MIMO in 6G. Herausforderungen: Verluste minimieren (via InAs-QDs). Zukünftig: Voll-integrierte PICs für Edge-Computing in Basissationen. \textbf{Das Fraunhofer HHI bietet bereits anwendungsspezifische PICs auf der Siliziumnitrid (SiN) Plattform an, die auch in Biowissenschaften und Sensorik eingesetzt werden \cite{hhi_6g}.}

\end{document}
