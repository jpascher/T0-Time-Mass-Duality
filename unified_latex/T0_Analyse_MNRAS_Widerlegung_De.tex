\documentclass[11pt,a4paper,openany]{book}

% Essential packages
\usepackage[utf8]{inputenc}
\usepackage[T1]{fontenc}
\usepackage[english]{babel}
\usepackage[a4paper,margin=2.5cm]{geometry}
\usepackage{lmodern}

% Math and physics packages
\usepackage{amsmath}
\usepackage{amssymb}
\usepackage{amsthm}
\usepackage{mathtools}
\usepackage{physics}
\usepackage{siunitx}

% Graphics and tables
\usepackage{graphicx}
\usepackage[table,xcdraw]{xcolor}
\usepackage{tikz}
\usepackage{pgfplots}
\usepackage{tcolorbox}
\usepackage{booktabs}
\usepackage{array}
\usepackage{longtable}
\usepackage{float}

% Document formatting
\usepackage{fancyhdr}
\usepackage{tocloft}
\usepackage{hyperref}
\usepackage{cleveref}
\usepackage{microtype}
\usepackage{enumitem}
\usepackage{newunicodechar}

% Additional packages (cleaned up - removed duplicates)
\usepackage{adjustbox}
\usepackage{algorithm}
\usepackage{algorithmic}
\usepackage{amsfonts}
\usepackage{bm}
\usepackage{braket}
\usepackage{breakurl}
\usepackage{cancel}
\usepackage{caption}
\usepackage{cite}
\usepackage{csquotes}
\usepackage{doi}
\usepackage{forest}
\usepackage{gensymb}
\usepackage{hyphenat}
\usepackage{listings}
\usepackage{mdframed}
\usepackage{multicol}
\usepackage{multirow}
\usepackage{natbib}
\usepackage{pdflscape}
\usepackage{ragged2e}
\usepackage{setspace}
\usepackage{slashed}
\usepackage{tabularx}
\usepackage{textcomp}
\usepackage{textgreek}
\usepackage{upgreek}
\usepackage{url}

% Color definitions (FIXED: removed extra \definecolor commands)
\definecolor{blue}{rgb}{0,0,1}
\definecolor{boxgray}{RGB}{240,240,240}
\definecolor{deepblue}{RGB}{0,0,127}
\definecolor{deepgreen}{RGB}{0,127,0}
\definecolor{deepred}{RGB}{191,0,0}
\definecolor{t0blue}{RGB}{0,102,204}
\definecolor{t0green}{RGB}{0,153,0}
\definecolor{t0orange}{RGB}{255,152,0}
\definecolor{t0purple}{RGB}{102,0,204}
\definecolor{t0red}{RGB}{204,0,0}
\definecolor{t0yellow}{RGB}{255,204,0}

% TikZ libraries
\usetikzlibrary{arrows,shapes,positioning,calc,patterns,decorations.pathmorphing,decorations.markings}

% PGFPlots setup
\pgfplotsset{compat=1.18}

% Hyperref setup
\hypersetup{
    colorlinks=true,
    linkcolor=blue,
    filecolor=magenta,
    urlcolor=cyan,
    citecolor=green,
    pdftitle={T0 Theory Document},
    pdfauthor={Johann Pascher},
    pdfsubject={T0 Theory},
    pdfkeywords={T0, physics, theory}
}

% Header and footer
\pagestyle{fancy}
\fancyhf{}
\fancyhead[LE,RO]{\thepage}
\fancyhead[RE]{\leftmark}
\fancyhead[LO]{\rightmark}
\fancyfoot[C]{T0 Theory - Johann Pascher}

% Theorem environments
\theoremstyle{definition}
\newtheorem{definition}{Definition}[section]
\newtheorem{theorem}{Theorem}[section]
\newtheorem{lemma}[theorem]{Lemma}
\newtheorem{proposition}[theorem]{Proposition}
\newtheorem{corollary}[theorem]{Corollary}
\theoremstyle{remark}
\newtheorem{remark}{Remark}[section]
\newtheorem{example}{Example}[section]

% Custom commands (common across T0 documents)
\newcommand{\T}[1]{\text{#1}}
\newcommand{\mat}[1]{\mathbf{#1}}
\newcommand{\E}{\mathrm{e}}
\newcommand{\I}{\mathrm{i}}
\newcommand{\diff}{\mathrm{d}}
\newcommand{\Real}{\mathrm{Re}}
\newcommand{\Imag}{\mathrm{Im}}


\begin{document}

\maketitle
\tableofcontents

\thispagestyle{fancy}

\begin{abstract}
    Dieses Dokument analysiert die Ergebnisse des einflussreichen Papers "Does the Hubble tension eclipse the Solar System?" (MNRAS, 544, 1, 2024) \cite{nathan2024} und setzt sie in den Kontext der T0-Theorie. Das Paper widerlegt eine bedeutende Klasse von modifizierten Gravitationstheorien, indem es zeigt, dass diese zu messbaren Anomalien in den Umlaufbahnen des Sonnensystems führen würden, die jedoch nicht beobachtet werden. Wir argumentieren, dass diese Falsifizierung als starke, indirekte Evidenz für den Ansatz der T0-Theorie zu werten ist, da die T0-Theorie per Definition mit den hochpräzisen Daten des Sonnensystems konsistent ist.
\end{abstract}

\tableofcontents
\newpage

\chapter{Zusammenfassung des MNRAS-Papiers}

Die sogenannte "Hubble-Spannung" – die Diskrepanz zwischen den Messungen der Expansionsrate des Universums im nahen und fernen Kosmos – ist eines der größten Rätsel der modernen Kosmologie. Ein populärer Lösungsansatz besteht darin, die Allgemeine Relativitätstheorie auf kosmologischen Skalen zu modifizieren.

Das in \textit{Monthly Notices of the Royal Astronomical Society} (MNRAS) publizierte Paper von Nathan et al. \cite{nathan2024} verfolgt einen rigorosen Testansatz für diese Hypothese:

    - \textbf{Annahme:} Die Autoren nehmen eine Klasse von modifizierten Gravitationstheorien an, die konstruiert sind, um die Hubble-Spannung aufzulösen.
    - \textbf{Test im Sonnensystem:} Sie wenden dieselbe Theorie auf unser lokales Umfeld an und berechnen die theoretisch zu erwartenden Auswirkungen auf die hochpräzise bekannte Umlaufbahn des Planeten Saturn.
    - \textbf{Ergebnis:} Die Modifikationen, die notwendig wären, um die Hubble-Spannung zu erklären, würden zu signifikanten, leicht messbaren Abweichungen in Saturns Orbit führen.
    - \textbf{Falsifizierung:} Hochpräzise Messdaten, insbesondere von der Cassini-Raumsonde, zeigen keinerlei Anzeichen dieser vorhergesagten Anomalien. Die beobachtete Umlaufbahn stimmt exakt mit den Vorhersagen der unveränderten Allgemeinen Relativitätstheorie überein.

Die Schlussfolgerung des Papers ist unmissverständlich: Diese spezifische Klasse von modifizierten Gravitationstheorien ist mit den Beobachtungen unvereinbar und somit als Erklärung für die Hubble-Spannung widerlegt.

\chapter{Die Implikationen für die T0-Theorie}

Die Falsifizierung eines konkurrierenden Modells ist oft eine starke indirekte Bestätigung für eine alternative Theorie. Dies ist hier in besonderem Maße der Fall, da die T0-Theorie das Problem auf einer fundamentaleren Ebene löst und den im Paper beschriebenen "Test" trivial besteht.

\section{Die T0-Theorie modifiziert nicht die Gravitation}
Der entscheidende Unterschied ist, dass die T0-Theorie die Allgemeine Relativitätstheorie auf Skalen des Sonnensystems unangetastet lässt. Sie postuliert keine Ad-hoc-Modifikation der Gravitation. Stattdessen adressiert sie die fehlerhafte Prämisse, auf der die Hubble-Spannung überhaupt erst basiert: die Annahme einer kosmischen Expansion.

\section{Rotverschiebung als geometrischer Effekt}
In der T0-Theorie existiert keine beschleunigte Expansion und folglich auch keine "Hubble-Spannung", die erklärt werden müsste. Die beobachtete kosmologische Rotverschiebung wird stattdessen als ein emergenter, geometrischer Effekt erklärt:

    - Licht verliert auf seiner Reise durch das T0-Vakuum Energie durch eine kumulative Interaktion mit der fraktalen Geometrie des Feldes.
    - Dieser Effekt manifestiert sich als eine systematische Rotverschiebung, die proportional zur zurückgelegten Distanz ist.

\section{Konsistenz mit den Daten des Sonnensystems}
Der Mechanismus der geometrischen Rotverschiebung ist über die vergleichsweise winzigen Distanzen des Sonnensystems (wenige Lichtstunden) absolut vernachlässigbar. Der kumulative Effekt ist erst über Millionen und Milliarden von Lichtjahren messbar.

Daraus folgt:
\begin{center}
    \textbf{Die T0-Theorie sagt exakt null messbare Anomalien in den Planetenbahnen des Sonnensystems voraus.}
\end{center}
Sie ist somit per Definition perfekt konsistent mit den hochpräzisen Daten der Cassini-Mission, die die modifizierten Gravitationsmodelle widerlegen.

\chapter{Schlussfolgerung}

Das Paper von Nathan et al. \cite{nathan2024} leistet einen wichtigen Beitrag, indem es einen spekulativen und inkonsistenten Lösungsweg für die Hubble-Spannung schließt. Gleichzeitig unterstreicht es die Stärke eines fundamentaleren Ansatzes, wie ihn die T0-Theorie verfolgt.

Indem die T0-Theorie nicht an den Symptomen (der Expansion) ansetzt, sondern die Ursache (die Interpretation der Rotverschiebung) korrigiert, löst sie nicht nur die Hubble-Spannung auf, sondern bleibt dabei in voller Übereinstimmung mit den präzisesten Beobachtungen in unserem eigenen Sonnensystem. Das Scheitern der modifizierten Gravitation ist somit ein Erfolg für die physikalische Konsistenz der T0-Kosmologie.

\end{document}
