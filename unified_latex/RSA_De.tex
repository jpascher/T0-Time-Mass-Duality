\documentclass[11pt,a4paper,openany]{book}

% Essential packages
\usepackage[utf8]{inputenc}
\usepackage[T1]{fontenc}
\usepackage[english]{babel}
\usepackage[a4paper,margin=2.5cm]{geometry}
\usepackage{lmodern}

% Math and physics packages
\usepackage{amsmath}
\usepackage{amssymb}
\usepackage{amsthm}
\usepackage{mathtools}
\usepackage{physics}
\usepackage{siunitx}

% Graphics and tables
\usepackage{graphicx}
\usepackage[table,xcdraw]{xcolor}
\usepackage{tikz}
\usepackage{pgfplots}
\usepackage{tcolorbox}
\usepackage{booktabs}
\usepackage{array}
\usepackage{longtable}
\usepackage{float}

% Document formatting
\usepackage{fancyhdr}
\usepackage{tocloft}
\usepackage{hyperref}
\usepackage{cleveref}
\usepackage{microtype}
\usepackage{enumitem}
\usepackage{newunicodechar}

% Additional packages
\usepackage{adjustbox}
\usepackage{algorithm}
\usepackage{algorithmic}
\usepackage{amsfonts}
\usepackage{amsmath,amsfonts,amssymb}
\usepackage{amsmath,amsfonts,amssymb,physics}
\usepackage{amsmath,amssymb}
\usepackage{amsmath,amssymb,amsfonts,amsthm}
\usepackage{amsmath,amssymb,amsthm}
\usepackage{amsmath,amssymb,physics,graphicx,xcolor,amsthm}
\usepackage{bm}
\usepackage{booktabs,array,longtable,multirow}
\usepackage{braket}
\usepackage{breakurl}
\usepackage{cancel}
\usepackage{caption}
\usepackage{cite}
\usepackage{color}
\usepackage{colortbl}
\usepackage{csquotes}
\usepackage{doi}
\usepackage{forest}
\usepackage{gensymb}
\usepackage{geometry,fancyhdr}
\usepackage{graphicx,tikz,pgfplots}
\usepackage{hyperref,url}
\usepackage{hyphenat}
\usepackage{listings}
\usepackage{listings,enumerate}
\usepackage{mdframed}
\usepackage{multicol}
\usepackage{multirow}
\usepackage{natbib}
\usepackage{pdflscape}
\usepackage{ragged2e}
\usepackage{setspace}
\usepackage{siunitx,xcolor,graphicx}
\usepackage{slashed}
\usepackage{tabularx}
\usepackage{textcomp}
\usepackage{textgreek}
\usepackage{tikz,pgfplots}
\usepackage{upgreek}
\usepackage{url}

% Custom commands and definitions
\definecolor{blue}
\definecolor{blue}{rgb}{0,0,1}
\definecolor{boxgray}
\definecolor{boxgray}{RGB}{240,240,240}
\definecolor{deepblue}
\definecolor{deepblue}{RGB}{0,0,127}
\definecolor{deepgreen}
\definecolor{deepgreen}{RGB}{0,127,0}
\definecolor{deepred}
\definecolor{deepred}{RGB}{191,0,0}
\definecolor{t0blue}
\definecolor{t0blue}{RGB}{0,102,204}
\definecolor{t0blue}{RGB}{33,150,243}
\definecolor{t0green}
\definecolor{t0green}{RGB}{0,153,0}
\definecolor{t0green}{RGB}{0,153,76}
\definecolor{t0green}{RGB}{76,175,80}
\definecolor{t0orange}
\definecolor{t0orange}{RGB}{255,152,0}
\definecolor{t0purple}
\definecolor{t0purple}{RGB}{102,0,204}
\definecolor{t0purple}{RGB}{156,39,176}
\definecolor{t0red}
\definecolor{t0red}{RGB}{204,0,0}
\definecolor{t0red}{RGB}{204,0,51}
\definecolor{t0red}{RGB}{244,67,54}
\definecolor{t0yellow}
\definecolor{t0yellow}{RGB}{255,204,0}
\geometry{a4paper, left=25mm, right=25mm, top=25mm, bottom=25mm}
\geometry{a4paper, margin=1in}
\geometry{a4paper, margin=2.5cm}
\geometry{a4paper, margin=2cm}
\geometry{left=2.5cm,right=2.5cm,top=2.5cm,bottom=2.5cm}
\geometry{left=2cm,right=2cm,top=2cm,bottom=2cm}
\geometry{margin=1in}
\geometry{margin=2.5cm}
\geometry{margin=2cm}
\hypersetup{
	colorlinks=true,
	linkcolor=blue,
	citecolor=blue,
	urlcolor=blue,
	pdftitle={Analysis and Implications of MNRAS Paper 544 for the T0-Theory}
\hypersetup{
	colorlinks=true,
	linkcolor=blue,
	citecolor=blue,
	urlcolor=blue,
	pdftitle={Beweis: Die Feinstrukturkonstante α = 1 in natürlichen Einheiten}
\hypersetup{
	colorlinks=true,
	linkcolor=blue,
	citecolor=blue,
	urlcolor=blue,
	pdftitle={Beweis: Die Koide-Formel enthält implizit $\xi$}
\hypersetup{
	colorlinks=true,
	linkcolor=blue,
	citecolor=blue,
	urlcolor=blue,
	pdftitle={Chinas Photonischer Quantenchip: 1000x-Speedup und T0-Integration}
\hypersetup{
	colorlinks=true,
	linkcolor=blue,
	citecolor=blue,
	urlcolor=blue,
	pdftitle={Complete Derivation of Higgs Mass and Wilson Coefficients}
\hypersetup{
	colorlinks=true,
	linkcolor=blue,
	citecolor=blue,
	urlcolor=blue,
	pdftitle={Complete Particle Spectrum: Standard Model vs T0 Theory}
\hypersetup{
	colorlinks=true,
	linkcolor=blue,
	citecolor=blue,
	urlcolor=blue,
	pdftitle={Conceptual Comparison of Unified Natural Units and Extended Standard Model}
\hypersetup{
	colorlinks=true,
	linkcolor=blue,
	citecolor=blue,
	urlcolor=blue,
	pdftitle={Connections between the Mizohata-Takeuchi Counterexample and the T0 Time-Mass Duality Theory}
\hypersetup{
	colorlinks=true,
	linkcolor=blue,
	citecolor=blue,
	urlcolor=blue,
	pdftitle={Das Relationale Zahlensystem: Primzahlen als fundamentale Verhältnisse}
\hypersetup{
	colorlinks=true,
	linkcolor=blue,
	citecolor=blue,
	urlcolor=blue,
	pdftitle={Das T0-Modell (Planck-Referenziert): Eine Neuformulierung der Physik}
\hypersetup{
	colorlinks=true,
	linkcolor=blue,
	citecolor=blue,
	urlcolor=blue,
	pdftitle={Das T0-Modell: Zeit-Energie-Dualität und geometrische Ruhemasse}
\hypersetup{
	colorlinks=true,
	linkcolor=blue,
	citecolor=blue,
	urlcolor=blue,
	pdftitle={Der Massenskalierungsexponent κ in der T0-Theorie}
\hypersetup{
	colorlinks=true,
	linkcolor=blue,
	citecolor=blue,
	urlcolor=blue,
	pdftitle={Der geometrische Formalismus der T0-Quantenmechanik und seine Anwendung auf Quantencomputer}
\hypersetup{
	colorlinks=true,
	linkcolor=blue,
	citecolor=blue,
	urlcolor=blue,
	pdftitle={Der xi Parameter und Teilchendifferenzierung in der T0-Theorie}
\hypersetup{
	colorlinks=true,
	linkcolor=blue,
	citecolor=blue,
	urlcolor=blue,
	pdftitle={Deterministic Quantum Mechanics via T0-Energy Field Formulation}
\hypersetup{
	colorlinks=true,
	linkcolor=blue,
	citecolor=blue,
	urlcolor=blue,
	pdftitle={Deterministische Quantenmechanik via T0-Energiefeld-Formulierung}
\hypersetup{
	colorlinks=true,
	linkcolor=blue,
	citecolor=blue,
	urlcolor=blue,
	pdftitle={Die Elektroneneinheitsladung in der T0-Theorie: Jenseits von Punkt-Singularitäten}
\hypersetup{
	colorlinks=true,
	linkcolor=blue,
	citecolor=blue,
	urlcolor=blue,
	pdftitle={Die Feinstrukturkonstante: Verschiedene Darstellungen und Beziehungen}
\hypersetup{
	colorlinks=true,
	linkcolor=blue,
	citecolor=blue,
	urlcolor=blue,
	pdftitle={Die Musikalische Spirale und die 137: Die mathematische Entdeckung der kosmischen Verstimmung}
\hypersetup{
	colorlinks=true,
	linkcolor=blue,
	citecolor=blue,
	urlcolor=blue,
	pdftitle={E=mc² = E=m: Die Konstanten-Illusion entlarvt}
\hypersetup{
	colorlinks=true,
	linkcolor=blue,
	citecolor=blue,
	urlcolor=blue,
	pdftitle={E=mc² = E=m: The Constants Illusion Exposed}
\hypersetup{
	colorlinks=true,
	linkcolor=blue,
	citecolor=blue,
	urlcolor=blue,
	pdftitle={Einfache Lagrange-Revolution: Von der Standardmodell-Komplexität zur T0-Eleganz}
\hypersetup{
	colorlinks=true,
	linkcolor=blue,
	citecolor=blue,
	urlcolor=blue,
	pdftitle={Einführung in die Umsetzung photonischer Bauteile auf Wafern für Nachrichtentechniker}
\hypersetup{
	colorlinks=true,
	linkcolor=blue,
	citecolor=blue,
	urlcolor=blue,
	pdftitle={Einführung in photonische Quantenchips für Nachrichtentechniker}
\hypersetup{
	colorlinks=true,
	linkcolor=blue,
	citecolor=blue,
	urlcolor=blue,
	pdftitle={Elimination der Masse als dimensionaler Platzhalter im T0-Modell}
\hypersetup{
	colorlinks=true,
	linkcolor=blue,
	citecolor=blue,
	urlcolor=blue,
	pdftitle={Elimination of Mass as Dimensional Placeholder in the T0 Model}
\hypersetup{
	colorlinks=true,
	linkcolor=blue,
	citecolor=blue,
	urlcolor=blue,
	pdftitle={Empirical Analysis of Deterministic Factorization Methods}
\hypersetup{
	colorlinks=true,
	linkcolor=blue,
	citecolor=blue,
	urlcolor=blue,
	pdftitle={Empirische Analyse deterministischer Faktorisierungsmethoden}
\hypersetup{
	colorlinks=true,
	linkcolor=blue,
	citecolor=blue,
	urlcolor=blue,
	pdftitle={Integration der Dirac-Gleichung im T0-Modell: Natürliche-Einheiten-Rahmenwerk}
\hypersetup{
	colorlinks=true,
	linkcolor=blue,
	citecolor=blue,
	urlcolor=blue,
	pdftitle={Integration of the Dirac Equation in the T0 Model: Natural Units Framework}
\hypersetup{
	colorlinks=true,
	linkcolor=blue,
	citecolor=blue,
	urlcolor=blue,
	pdftitle={Introduction to Photonic Quantum Chips for Communication Engineers}
\hypersetup{
	colorlinks=true,
	linkcolor=blue,
	citecolor=blue,
	urlcolor=blue,
	pdftitle={Introduction to the Implementation of Photonic Components on Wafers for Communication Engineers}
\hypersetup{
	colorlinks=true,
	linkcolor=blue,
	citecolor=blue,
	urlcolor=blue,
	pdftitle={Konzeptioneller Vergleich von Einheitlichen Natürlichen Einheiten und Erweitertem Standardmodell}
\hypersetup{
	colorlinks=true,
	linkcolor=blue,
	citecolor=blue,
	urlcolor=blue,
	pdftitle={Markov Chains in the Context of T0 Theory: Deterministic or Stochastic? A Treatise on Patterns, Preconditions, and Uncertainty}
\hypersetup{
	colorlinks=true,
	linkcolor=blue,
	citecolor=blue,
	urlcolor=blue,
	pdftitle={Markov-Ketten im Kontext der T0-Theorie: Deterministisch oder stochastisch? Ein Traktat zu Mustern, Voraussetzungen und Unsicherheit}
\hypersetup{
	colorlinks=true,
	linkcolor=blue,
	citecolor=blue,
	urlcolor=blue,
	pdftitle={Mathematical Analysis of T0-Shor Algorithm: Theoretical Framework and Computational Complexity}
\hypersetup{
	colorlinks=true,
	linkcolor=blue,
	citecolor=blue,
	urlcolor=blue,
	pdftitle={Mathematical Constructs of Alternative CMB Models: Unnikrishnan and Peratt in Harmony with the T0 Theory}
\hypersetup{
	colorlinks=true,
	linkcolor=blue,
	citecolor=blue,
	urlcolor=blue,
	pdftitle={Mathematische Analyse des T0-Shor Algorithmus: Theoretischer Rahmen und Berechnungskomplexität}
\hypersetup{
	colorlinks=true,
	linkcolor=blue,
	citecolor=blue,
	urlcolor=blue,
	pdftitle={Mathematische Konstrukte alternativer CMB-Modelle: Unnikrishnan und Peratt im Einklang mit der T0-Theorie}
\hypersetup{
	colorlinks=true,
	linkcolor=blue,
	citecolor=blue,
	urlcolor=blue,
	pdftitle={Natural Unit Systems: Universal Energy Conversion and Fundamental Length Scale Hierarchy}
\hypersetup{
	colorlinks=true,
	linkcolor=blue,
	citecolor=blue,
	urlcolor=blue,
	pdftitle={Natural Units in Theoretical Physics: A Treatise in the Context of T0 Theory}
\hypersetup{
	colorlinks=true,
	linkcolor=blue,
	citecolor=blue,
	urlcolor=blue,
	pdftitle={Natürliche Einheiten in der theoretischen Physik: Eine Abhandlung im Kontext der T0-Theorie}
\hypersetup{
	colorlinks=true,
	linkcolor=blue,
	citecolor=blue,
	urlcolor=blue,
	pdftitle={Natürliche Einheitensysteme: Universelle Energieumwandlung und fundamentale Längenskala-Hierarchie}
\hypersetup{
	colorlinks=true,
	linkcolor=blue,
	citecolor=blue,
	urlcolor=blue,
	pdftitle={Parameter System-Dependency in T0-Model: SI vs. Natural Units}
\hypersetup{
	colorlinks=true,
	linkcolor=blue,
	citecolor=blue,
	urlcolor=blue,
	pdftitle={Parameter-Systemabhängigkeit im T0-Modell: SI- vs. natürliche Einheiten}
\hypersetup{
	colorlinks=true,
	linkcolor=blue,
	citecolor=blue,
	urlcolor=blue,
	pdftitle={Proof: The Fine Structure Constant α = 1 in Natural Units}
\hypersetup{
	colorlinks=true,
	linkcolor=blue,
	citecolor=blue,
	urlcolor=blue,
	pdftitle={Proof: The Koide Formula Implicitly Contains $\xi$}
\hypersetup{
	colorlinks=true,
	linkcolor=blue,
	citecolor=blue,
	urlcolor=blue,
	pdftitle={Pure Energy T0 Theory: Ratio-Based Physics with SI Reference}
\hypersetup{
	colorlinks=true,
	linkcolor=blue,
	citecolor=blue,
	urlcolor=blue,
	pdftitle={Quantum Mechanics in the T0 Model: Field-Theoretic Foundations}
\hypersetup{
	colorlinks=true,
	linkcolor=blue,
	citecolor=blue,
	urlcolor=blue,
	pdftitle={Ratio-Based vs. Absolute: The Role of Fractal Correction in T0 Theory}
\hypersetup{
	colorlinks=true,
	linkcolor=blue,
	citecolor=blue,
	urlcolor=blue,
	pdftitle={Reine Energie T0-Theorie: Verhältnis-basierte Physik mit SI-Referenz}
\hypersetup{
	colorlinks=true,
	linkcolor=blue,
	citecolor=blue,
	urlcolor=blue,
	pdftitle={Simple Lagrangian Revolution: From Standard Model Complexity to T0 Elegance}
\hypersetup{
	colorlinks=true,
	linkcolor=blue,
	citecolor=blue,
	urlcolor=blue,
	pdftitle={Simplified Dirac Equation in T0 Theory: Field Node Approach}
\hypersetup{
	colorlinks=true,
	linkcolor=blue,
	citecolor=blue,
	urlcolor=blue,
	pdftitle={Simplified T0 Theory: Elegant Lagrangian Density for Time-Mass Duality}
\hypersetup{
	colorlinks=true,
	linkcolor=blue,
	citecolor=blue,
	urlcolor=blue,
	pdftitle={T0 Cosmology: Redshift as a Geometric Path Effect in a Static Universe}
\hypersetup{
	colorlinks=true,
	linkcolor=blue,
	citecolor=blue,
	urlcolor=blue,
	pdftitle={T0 Deterministic Quantum Computing: Complete Analysis of Important Algorithms}
\hypersetup{
	colorlinks=true,
	linkcolor=blue,
	citecolor=blue,
	urlcolor=blue,
	pdftitle={T0 Deterministisches Quantencomputing: Vollständige Analyse wichtiger Algorithmen}
\hypersetup{
	colorlinks=true,
	linkcolor=blue,
	citecolor=blue,
	urlcolor=blue,
	pdftitle={T0 Model: Complete Framework - From Time-Energy Duality to Universal Constants}
\hypersetup{
	colorlinks=true,
	linkcolor=blue,
	citecolor=blue,
	urlcolor=blue,
	pdftitle={T0 Model: Complete Parameter-Free Particle Mass Calculation}
\hypersetup{
	colorlinks=true,
	linkcolor=blue,
	citecolor=blue,
	urlcolor=blue,
	pdftitle={T0 Model: Unified Neutrino Formula Structure}
\hypersetup{
	colorlinks=true,
	linkcolor=blue,
	citecolor=blue,
	urlcolor=blue,
	pdftitle={T0 Model: Universal Energy Relations for Mol and Candela Units}
\hypersetup{
	colorlinks=true,
	linkcolor=blue,
	citecolor=blue,
	urlcolor=blue,
	pdftitle={T0 Modell: Vollständiges Framework - Von Zeit-Energie-Dualität zu universellen Konstanten}
\hypersetup{
	colorlinks=true,
	linkcolor=blue,
	citecolor=blue,
	urlcolor=blue,
	pdftitle={T0 Quantenfeldtheorie: QFT, QM und Quantencomputer}
\hypersetup{
	colorlinks=true,
	linkcolor=blue,
	citecolor=blue,
	urlcolor=blue,
	pdftitle={T0 Quantum Field Theory: QFT, QM and Quantum Computers}
\hypersetup{
	colorlinks=true,
	linkcolor=blue,
	citecolor=blue,
	urlcolor=blue,
	pdftitle={T0 Theory vs Bell's Theorem: How Deterministic Energy Fields Circumvent No-Go Theorems}
\hypersetup{
	colorlinks=true,
	linkcolor=blue,
	citecolor=blue,
	urlcolor=blue,
	pdftitle={T0 Theory: Final Extension to Hadrons - Physically Derived Corrections}
\hypersetup{
	colorlinks=true,
	linkcolor=blue,
	citecolor=blue,
	urlcolor=blue,
	pdftitle={T0 Theory: The Fine-Structure Constant}
\hypersetup{
	colorlinks=true,
	linkcolor=blue,
	citecolor=blue,
	urlcolor=blue,
	pdftitle={T0 Theory: The Gravitational Constant}
\hypersetup{
	colorlinks=true,
	linkcolor=blue,
	citecolor=blue,
	urlcolor=blue,
	pdftitle={T0-Kosmologie: Rotverschiebung als geometrischer Pfad-Effekt im statischen Universum}
\hypersetup{
	colorlinks=true,
	linkcolor=blue,
	citecolor=blue,
	urlcolor=blue,
	pdftitle={T0-Model: Complete Document Analysis and Structured Summary}
\hypersetup{
	colorlinks=true,
	linkcolor=blue,
	citecolor=blue,
	urlcolor=blue,
	pdftitle={T0-Model: Kinetic Energy of Electrons and Photons}
\hypersetup{
	colorlinks=true,
	linkcolor=blue,
	citecolor=blue,
	urlcolor=blue,
	pdftitle={T0-Model: The Hubble Parameter in Static Universe}
\hypersetup{
	colorlinks=true,
	linkcolor=blue,
	citecolor=blue,
	urlcolor=blue,
	pdftitle={T0-Modell-Verifikation: Skalen-Verhältnis-basierte Berechnungen}
\hypersetup{
	colorlinks=true,
	linkcolor=blue,
	citecolor=blue,
	urlcolor=blue,
	pdftitle={T0-Modell: Bewegungsenergie von Elektronen und Photonen}
\hypersetup{
	colorlinks=true,
	linkcolor=blue,
	citecolor=blue,
	urlcolor=blue,
	pdftitle={T0-Modell: Die Hubble-Konstante im statischen Universum}
\hypersetup{
	colorlinks=true,
	linkcolor=blue,
	citecolor=blue,
	urlcolor=blue,
	pdftitle={T0-Modell: Einheitliche Neutrino-Formel-Struktur}
\hypersetup{
	colorlinks=true,
	linkcolor=blue,
	citecolor=blue,
	urlcolor=blue,
	pdftitle={T0-Modell: Universelle Energiebeziehungen für Mol- und Candela-Einheiten}
\hypersetup{
	colorlinks=true,
	linkcolor=blue,
	citecolor=blue,
	urlcolor=blue,
	pdftitle={T0-Modell: Vollständige Dokumentenanalyse und strukturierte Zusammenfassung}
\hypersetup{
	colorlinks=true,
	linkcolor=blue,
	citecolor=blue,
	urlcolor=blue,
	pdftitle={T0-Modell: Vollständige parameterfreie Teilchenmassen-Berechnung}
\hypersetup{
	colorlinks=true,
	linkcolor=blue,
	citecolor=blue,
	urlcolor=blue,
	pdftitle={T0-QAT: $\xi$-Aware Quantization-Aware Training}
\hypersetup{
	colorlinks=true,
	linkcolor=blue,
	citecolor=blue,
	urlcolor=blue,
	pdftitle={T0-QFT ML Addendum: Machine Learning Derived Extensions}
\hypersetup{
	colorlinks=true,
	linkcolor=blue,
	citecolor=blue,
	urlcolor=blue,
	pdftitle={T0-QFT ML-Addendum: Maschinelle Lern-abgeleitete Erweiterungen}
\hypersetup{
	colorlinks=true,
	linkcolor=blue,
	citecolor=blue,
	urlcolor=blue,
	pdftitle={T0-Theorie vs Bells Theorem: Wie deterministische Energiefelder No-Go-Theoreme umgehen}
\hypersetup{
	colorlinks=true,
	linkcolor=blue,
	citecolor=blue,
	urlcolor=blue,
	pdftitle={T0-Theorie: Der Terrell-Penrose-Effekt und Massenvariation}
\hypersetup{
	colorlinks=true,
	linkcolor=blue,
	citecolor=blue,
	urlcolor=blue,
	pdftitle={T0-Theorie: Die Feinstrukturkonstante}
\hypersetup{
	colorlinks=true,
	linkcolor=blue,
	citecolor=blue,
	urlcolor=blue,
	pdftitle={T0-Theorie: Die Gravitationskonstante}
\hypersetup{
	colorlinks=true,
	linkcolor=blue,
	citecolor=blue,
	urlcolor=blue,
	pdftitle={T0-Theorie: Die T0-Zeit-Masse-Dualität}
\hypersetup{
	colorlinks=true,
	linkcolor=blue,
	citecolor=blue,
	urlcolor=blue,
	pdftitle={T0-Theorie: Die sieben Rätsel}
\hypersetup{
	colorlinks=true,
	linkcolor=blue,
	citecolor=blue,
	urlcolor=blue,
	pdftitle={T0-Theorie: Erweiterung auf Bell-Tests – ML-Simulationen (November 2025)}
\hypersetup{
	colorlinks=true,
	linkcolor=blue,
	citecolor=blue,
	urlcolor=blue,
	pdftitle={T0-Theorie: Finale Erweiterung auf Hadronen - Physikalisch abgeleitete Korrekturen}
\hypersetup{
	colorlinks=true,
	linkcolor=blue,
	citecolor=blue,
	urlcolor=blue,
	pdftitle={T0-Theorie: Finale Fraktale Massenformeln (November 2025)}
\hypersetup{
	colorlinks=true,
	linkcolor=blue,
	citecolor=blue,
	urlcolor=blue,
	pdftitle={T0-Theorie: Fraktaldimension aus Lepton-Massenverhältnis}
\hypersetup{
	colorlinks=true,
	linkcolor=blue,
	citecolor=blue,
	urlcolor=blue,
	pdftitle={T0-Theorie: Fundamentale Prinzipien}
\hypersetup{
	colorlinks=true,
	linkcolor=blue,
	citecolor=blue,
	urlcolor=blue,
	pdftitle={T0-Theorie: Herleitung der Gravitationskonstanten}
\hypersetup{
	colorlinks=true,
	linkcolor=blue,
	citecolor=blue,
	urlcolor=blue,
	pdftitle={T0-Theorie: Kosmische Beziehungen und universelle $\xi$-Konstante}
\hypersetup{
	colorlinks=true,
	linkcolor=blue,
	citecolor=blue,
	urlcolor=blue,
	pdftitle={T0-Theorie: Kosmologie}
\hypersetup{
	colorlinks=true,
	linkcolor=blue,
	citecolor=blue,
	urlcolor=blue,
	pdftitle={T0-Theorie: Netzwerkdarstellung und Dimensionsanalyse in der T0-Theorie}
\hypersetup{
	colorlinks=true,
	linkcolor=blue,
	citecolor=blue,
	urlcolor=blue,
	pdftitle={T0-Theorie: Teilchenmassen}
\hypersetup{
	colorlinks=true,
	linkcolor=blue,
	citecolor=blue,
	urlcolor=blue,
	pdftitle={T0-Theorie: Vollstaendiger Abschluss}
\hypersetup{
	colorlinks=true,
	linkcolor=blue,
	citecolor=blue,
	urlcolor=blue,
	pdftitle={T0-Theory: Complete Closure}
\hypersetup{
	colorlinks=true,
	linkcolor=blue,
	citecolor=blue,
	urlcolor=blue,
	pdftitle={T0-Theory: Complete Derivation of All Parameters Without Circularity}
\hypersetup{
	colorlinks=true,
	linkcolor=blue,
	citecolor=blue,
	urlcolor=blue,
	pdftitle={T0-Theory: Cosmic Relations and universal $\xi$-constant}
\hypersetup{
	colorlinks=true,
	linkcolor=blue,
	citecolor=blue,
	urlcolor=blue,
	pdftitle={T0-Theory: Cosmology}
\hypersetup{
	colorlinks=true,
	linkcolor=blue,
	citecolor=blue,
	urlcolor=blue,
	pdftitle={T0-Theory: Derivation of the Gravitational Constant}
\hypersetup{
	colorlinks=true,
	linkcolor=blue,
	citecolor=blue,
	urlcolor=blue,
	pdftitle={T0-Theory: Extension to Bell Tests – ML Simulations (November 2025)}
\hypersetup{
	colorlinks=true,
	linkcolor=blue,
	citecolor=blue,
	urlcolor=blue,
	pdftitle={T0-Theory: Final Fractal Mass Formulas (November 2025)}
\hypersetup{
	colorlinks=true,
	linkcolor=blue,
	citecolor=blue,
	urlcolor=blue,
	pdftitle={T0-Theory: Fractal Dimension from Lepton Mass Ratio}
\hypersetup{
	colorlinks=true,
	linkcolor=blue,
	citecolor=blue,
	urlcolor=blue,
	pdftitle={T0-Theory: Fundamental Principles}
\hypersetup{
	colorlinks=true,
	linkcolor=blue,
	citecolor=blue,
	urlcolor=blue,
	pdftitle={T0-Theory: Mass Variation as an Equivalent to Time Dilation}
\hypersetup{
	colorlinks=true,
	linkcolor=blue,
	citecolor=blue,
	urlcolor=blue,
	pdftitle={T0-Theory: Network Representation and Dimensional Analysis in the T0-Theory}
\hypersetup{
	colorlinks=true,
	linkcolor=blue,
	citecolor=blue,
	urlcolor=blue,
	pdftitle={T0-Theory: Neutrinos}
\hypersetup{
	colorlinks=true,
	linkcolor=blue,
	citecolor=blue,
	urlcolor=blue,
	pdftitle={T0-Theory: Particle Masses}
\hypersetup{
	colorlinks=true,
	linkcolor=blue,
	citecolor=blue,
	urlcolor=blue,
	pdftitle={T0-Theory: The Seven Riddles}
\hypersetup{
	colorlinks=true,
	linkcolor=blue,
	citecolor=blue,
	urlcolor=blue,
	pdftitle={T0-Theory: The T0-Time-Mass Duality}
\hypersetup{
	colorlinks=true,
	linkcolor=blue,
	citecolor=blue,
	urlcolor=blue,
	pdftitle={Temperature Units in Natural Units: T0-Theory}
\hypersetup{
	colorlinks=true,
	linkcolor=blue,
	citecolor=blue,
	urlcolor=blue,
	pdftitle={Temperatureinheiten in nat\"urlichen Einheiten: T0-Theorie}
\hypersetup{
	colorlinks=true,
	linkcolor=blue,
	citecolor=blue,
	urlcolor=blue,
	pdftitle={The Electron Unit Charge in T0 Theory: Beyond Point Singularities}
\hypersetup{
	colorlinks=true,
	linkcolor=blue,
	citecolor=blue,
	urlcolor=blue,
	pdftitle={The Fine Structure Constant: Various Representations and Relationships}
\hypersetup{
	colorlinks=true,
	linkcolor=blue,
	citecolor=blue,
	urlcolor=blue,
	pdftitle={The Geometric Formalism of T0 Quantum Mechanics and its Application to Quantum Computing}
\hypersetup{
	colorlinks=true,
	linkcolor=blue,
	citecolor=blue,
	urlcolor=blue,
	pdftitle={The Mass Scaling Exponent κ in T0 Theory}
\hypersetup{
	colorlinks=true,
	linkcolor=blue,
	citecolor=blue,
	urlcolor=blue,
	pdftitle={The Musical Spiral and 137: The Mathematical Discovery of Cosmic Detuning}
\hypersetup{
	colorlinks=true,
	linkcolor=blue,
	citecolor=blue,
	urlcolor=blue,
	pdftitle={The Relational Number System: Prime Numbers as Fundamental Ratios}
\hypersetup{
	colorlinks=true,
	linkcolor=blue,
	citecolor=blue,
	urlcolor=blue,
	pdftitle={The T0 Model (Planck-Referenced): A Reformulation of Physics}
\hypersetup{
	colorlinks=true,
	linkcolor=blue,
	citecolor=blue,
	urlcolor=blue,
	pdftitle={The T0 Model: Time-Energy Duality and Geometric Rest Mass}
\hypersetup{
	colorlinks=true,
	linkcolor=blue,
	citecolor=blue,
	urlcolor=blue,
	pdftitle={The T0-Model (Planck-Referenced): A Reformulation of Physics}
\hypersetup{
	colorlinks=true,
	linkcolor=blue,
	citecolor=blue,
	urlcolor=blue,
	pdftitle={Verbindungen zwischen dem Mizohata-Takeuchi-Gegenbeispiel und der T0-Zeit-Masse-Dualitätstheorie}
\hypersetup{
	colorlinks=true,
	linkcolor=blue,
	citecolor=blue,
	urlcolor=blue,
	pdftitle={Vereinfachte Dirac-Gleichung in der T0-Theorie: Feldknoten-Ansatz}
\hypersetup{
	colorlinks=true,
	linkcolor=blue,
	citecolor=blue,
	urlcolor=blue,
	pdftitle={Vereinfachte T0-Theorie: Elegante Lagrange-Dichte für Zeit-Masse-Dualität}
\hypersetup{
	colorlinks=true,
	linkcolor=blue,
	citecolor=blue,
	urlcolor=blue,
	pdftitle={Verhältnisbasiert vs. Absolut: Die Rolle der fraktalen Korrektur in der T0-Theorie}
\hypersetup{
	colorlinks=true,
	linkcolor=blue,
	citecolor=blue,
	urlcolor=blue,
	pdftitle={Vollständige Herleitung der Higgs-Masse und Wilson-Koeffizienten}
\hypersetup{
	colorlinks=true,
	linkcolor=blue,
	citecolor=blue,
	urlcolor=blue,
	pdftitle={Vollständiges Teilchenspektrum: Standard-Modell vs T0-Theorie}
\hypersetup{
	colorlinks=true,
	linkcolor=blue,
	citecolor=blue,
	urlcolor=blue,
	pdftitle={Warum Zahlenverhältnisse nicht direkt gekürzt werden dürfen}
\hypersetup{
	colorlinks=true,
	linkcolor=blue,
	citecolor=blue,
	urlcolor=blue,
	pdftitle={Why Numerical Ratios Must Not Be Directly Simplified}
\hypersetup{
	colorlinks=true,
	linkcolor=blue,
	citecolor=blue,
	urlcolor=blue,
}
\hypersetup{
	colorlinks=true,
	linkcolor=blue,
	citecolor=red,
	urlcolor=blue,
	bookmarks=true,
	bookmarksnumbered=true,
	pdfstartview=FitH,
	pdftitle={T0 Model - Field-Theoretic Derivation of the Beta Parameter}
\hypersetup{
	colorlinks=true,
	linkcolor=blue,
	citecolor=red,
	urlcolor=blue,
	bookmarks=true,
	bookmarksnumbered=true,
	pdfstartview=FitH,
	pdftitle={T0-Modell - Feldtheoretische Herleitung des Beta-Parameters}
\hypersetup{
	colorlinks=true,
	linkcolor=blue,
	filecolor=magenta,
	urlcolor=cyan,
}
\hypersetup{
	colorlinks=true,
	linkcolor=blue,
	urlcolor=blue,
	citecolor=blue,
	pdftitle={From Time Dilation to Mass Variation: Mathematical Core Formulations of Time-Mass Duality Theory - Updated Framework}
\hypersetup{
	colorlinks=true,
	linkcolor=blue,
	urlcolor=blue,
	citecolor=blue,
	pdftitle={T0 Model: Detailed Formula for Leptonic Anomalies}
\hypersetup{
	colorlinks=true,
	linkcolor=blue,
	urlcolor=blue,
	citecolor=blue,
	pdftitle={T0 Model: Detaillierte Formel für leptonische Anomalien}
\hypersetup{
	colorlinks=true,
	linkcolor=blue,
	urlcolor=blue,
	citecolor=blue,
	pdftitle={T0 Model: Energy-based Formulas with Quadratic Scaling}
\hypersetup{
	colorlinks=true,
	linkcolor=blue,
	urlcolor=blue,
	citecolor=blue,
	pdftitle={T0 Model: Granulation, Limits and Fundamental Asymmetry}
\hypersetup{
	colorlinks=true,
	linkcolor=blue,
	urlcolor=blue,
	citecolor=blue,
	pdftitle={T0-Modell: Energiebasierte Formeln mit quadratischer Skalierung}
\hypersetup{
	colorlinks=true,
	linkcolor=blue,
	urlcolor=blue,
	citecolor=blue,
	pdftitle={T0-Modell: Granulation, Limits und fundamentale Asymmetrie}
\hypersetup{
	colorlinks=true,
	linkcolor=blue,
	urlcolor=blue,
	citecolor=blue,
	pdftitle={Von Zeitdilatation zu Massenvariation: Mathematische Kernformulierungen der Zeit-Masse-Dualitätstheorie - Aktualisiertes Framework}
\hypersetup{
	colorlinks=true,
	linkcolor=t0blue,
	citecolor=t0blue,
	urlcolor=t0blue,
	pdftitle={T0 Model: Complete Theoretical Summary}
\hypersetup{
	colorlinks=true,
	linkcolor=t0blue,
	citecolor=t0blue,
	urlcolor=t0blue,
	pdftitle={T0 Theory: Resolution of Apparent Instantaneity}
\hypersetup{
	colorlinks=true,
	linkcolor=t0blue,
	citecolor=t0blue,
	urlcolor=t0blue,
	pdftitle={T0 vs Synergetics: Vereinfachung durch natürliche Einheiten}
\hypersetup{
	colorlinks=true,
	linkcolor=t0blue,
	citecolor=t0blue,
	urlcolor=t0blue,
	pdftitle={T0-Modell: Vollständige theoretische Zusammenfassung}
\hypersetup{
	colorlinks=true,
	linkcolor=t0blue,
	citecolor=t0blue,
	urlcolor=t0blue,
	pdftitle={T0-Theorie: Auflösung der scheinbaren Instantanität}
\hypersetup{
	colorlinks=true,
	linkcolor=t0blue,
	citecolor=t0blue,
	urlcolor=t0blue,
	pdftitle={T0-Theorie: Vollständige Dokumentenübersicht}
\hypersetup{
	colorlinks=true,
	linkcolor=t0blue,
	citecolor=t0blue,
	urlcolor=t0blue,
	pdftitle={T0-Theory: Complete Document Overview}
\hypersetup{
	colorlinks=true,
	linkcolor=t0blue,
	citecolor=t0blue,
	urlcolor=t0blue,
}
\hypersetup{
	colorlinks=true,
	linkcolor=t0blue,
	citecolor=t0green,
	urlcolor=t0blue,
	pdftitle={Das verborgene Geheimnis von 1/137}
\hypersetup{
	colorlinks=true,
	linkcolor=t0blue,
	citecolor=t0green,
	urlcolor=t0blue,
	pdftitle={The Hidden Secret of 1/137}
\hypersetup{
    colorlinks=true,
    linkcolor=blue,
    citecolor=blue,
    urlcolor=blue,
    pdftitle={Analyse und Implikationen des MNRAS-Papiers 544 für die T0-Theorie}
\hypersetup{
  colorlinks=true,
  linkcolor=blue,
  citecolor=blue,
  urlcolor=blue
}
\hypersetup{
  colorlinks=true,
  linkcolor=blue,
  citecolor=blue,
  urlcolor=blue,
  pdftitle={T0-Theorie: Ein-Uhr-Metrologie und Drei-Uhren-Experiment}
\hypersetup{
  colorlinks=true,
  linkcolor=blue,
  citecolor=blue,
  urlcolor=blue,
  pdftitle={T0-Theory: Single-Clock Metrology and Three-Clock Experiment}
\hypersetup{
colorlinks=true,
linkcolor=blue,
citecolor=blue,
urlcolor=blue,
pdftitle={Quantenmechanik im T0-Modell: Feldtheoretische Grundlagen}
\hypersetup{
colorlinks=true,
linkcolor=blue,
citecolor=blue,
urlcolor=blue,
pdftitle={T0-Theory: Neutrinos}
\newcommand{\Bzero}{B_0}
\newcommand{\CQCD}{C_{\text{QCD}
\newcommand{\Cconv}{C_{\text{conv}
\newcommand{\Cto}{C_{\text{T0}
\newcommand{\Czero}{C_0}
\newcommand{\DTmu}{D_{T,\mu}
\newcommand{\DcovT}[1]{\partial_\mu #1 + #1 \partial_\mu \Tfield}
\newcommand{\Dfrak}{D_f}
\newcommand{\Df}{D_f}
\newcommand{\DhiggsT}{\Tfield (\partial_\mu + ig A_\mu) \Phi + \Phi \partial_\mu \Tfield}
\newcommand{\EPlanck}{E_P}
\newcommand{\EPlanck}{E_{\text{Pl}
\newcommand{\EPratio}[1]{\frac{#1}
\newcommand{\EP}{E_P}
\newcommand{\EP}{E_{\text{P}
\newcommand{\EW}{E_W}
\newcommand{\EZ}{E_Z}
\newcommand{\Echar}{E_{\text{char}
\newcommand{\Ee}{E_e}
\newcommand{\Efield}{E(x,t)}
\newcommand{\Efield}{E_\text{field}
\newcommand{\Efield}{E_{\text{Feld}
\newcommand{\Efield}{E_{\text{Field}
\newcommand{\Efield}{E_{\text{field}
\newcommand{\Efield}{E}
\newcommand{\Egamma}{E_\gamma}
\newcommand{\Eh}{E_h}
\newcommand{\Emu}{E_\mu}
\newcommand{\Enorm}[1]{E_{\text{norm}
\newcommand{\En}{E_n}
\newcommand{\Ep}{E_p}
\newcommand{\Eratio}[2]{\frac{E_{#1}
\newcommand{\Etau}{E_\tau}
\newcommand{\Evis}{E_{\text{vis}
\newcommand{\Exi}{E_\xi}
\newcommand{\Ezero}{E_0}
\newcommand{\GeV}{\,\text{GeV}
\newcommand{\Gnat}{G_{\text{nat}
\newcommand{\Gsi}{G_{\text{SI}
\newcommand{\Hubble}{H_0}
\newcommand{\Kfrak}{K_{\text{frac}
\newcommand{\Kfrak}{K_{\text{frak}
\newcommand{\Kspec}{K_{\text{spec}
\newcommand{\LCDM}{\Lambda\text{CDM}
\newcommand{\LPlanck}{\ell_{\text{Pl}
\newcommand{\Lag}{\mathcal{L}
\newcommand{\Lambdat}{\Lambda_T}
\newcommand{\Leff}{L_{\text{eff}
\newcommand{\Lorentz}[2]{{\Lambda^\mu{}
\newcommand{\Lp}{L_{\text{P}
\newcommand{\Lxi}{L_\xi}
\newcommand{\Lzero}{L_0}
\newcommand{\MPl}{M_{\text{Pl}
\newcommand{\MSbar}{\overline{\text{MS}
\newcommand{\MeV}{\,\text{MeV}
\newcommand{\Mpl}{M_{\text{Pl}
\newcommand{\OmegaDM}{\Omega_{\text{DM}
\newcommand{\OmegaLambda}{\Omega_{\Lambda}
\newcommand{\Omegab}{\Omega_b}
\newcommand{\Phiphoton}{\Phi_{\text{photon}
\newcommand{\Ricci}{R_{\mu\nu}
\newcommand{\Riem}{R^\rho{}
\newcommand{\Rzero}{R_\infty}
\newcommand{\Scal}{R}
\newcommand{\SynchPower}{P_{\text{synch}
\newcommand{\TPlanck}{t_{\text{Pl}
\newcommand{\Tfieldt}{T(\vec{x}
\newcommand{\Tfieldt}{T(x,t)}
\newcommand{\Tfield}{T(x)}
\newcommand{\Tfield}{T(x,t)}
\newcommand{\Tfield}{T_{\text{field}
\newcommand{\Tfield}{T}
\newcommand{\Tfield}{\mathcal{T}
\newcommand{\Tzerot}{T_0(\Tfield)}
\newcommand{\Tzero}{T_0}
\newcommand{\Weyl}{C^\rho{}
\newcommand{\ZPinch}{J \times B = \nabla p}
\newcommand{\aleph}{\aleph}
\newcommand{\alphaEMSI}{\alpha_{\text{EM,SI}
\newcommand{\alphaEMnat}{\alpha_{\text{EM,nat}
\newcommand{\alphaEM}{\alpha_{\text{EM}
\newcommand{\alphaEM}{\ensuremath{\alpha_{\text{EM}
\newcommand{\alphaQCD}{\alpha_s}
\newcommand{\alphaQED}{\alpha_{\text{QED}
\newcommand{\alphaSI}{\alpha_{\text{SI}
\newcommand{\alphaT}{\alpha_{\text{T}
\newcommand{\alphaWSI}{\alpha_{\text{W,SI}
\newcommand{\alphaWnat}{\alpha_{\text{W,nat}
\newcommand{\alphaW}{\alpha_{\text{W}
\newcommand{\alphaem}{\alpha_{EM}
\newcommand{\alphaem}{\alpha}
\newcommand{\alphafine}{\alpha}
\newcommand{\alphagem}{\alpha}
\newcommand{\alphanat}{\alpha_{\text{nat}
\newcommand{\alphapar}{\alpha}
\newcommand{\betaTSI}{\beta_{\text{T,SI}
\newcommand{\betaTnat}{\beta_{\text{T,nat}
\newcommand{\betaT}{\beta_T}
\newcommand{\betaT}{\beta_{T}
\newcommand{\betaT}{\beta_{\text{T}
\newcommand{\betaT}{\ensuremath{\beta_T}
\newcommand{\betapar}{\beta}
\newcommand{\calL}{\mathcal{L}
\newcommand{\checked}{\checkmark}
\newcommand{\checkmarkx}{\checkmark}
\newcommand{\dTdt}{\frac{d\Tfieldt}
\newcommand{\deltaE}{\delta E}
\newcommand{\deltafield}{\ensuremath{\delta m}
\newcommand{\deltam}{\delta m}
\newcommand{\deq}{\displaystyle}
\newcommand{\docref}[1]{\texttt{#1}
\newcommand{\eV}{\,\text{eV}
\newcommand{\epsilonT}{\varepsilon_T}
\newcommand{\epsilonzero}{\varepsilon_0}
\newcommand{\etavis}{\eta_{\text{visual}
\newcommand{\e}{\mathrm{e}
\newcommand{\gW}{g_W}
\newcommand{\gammaf}{\gamma_{\text{Lorentz}
\newcommand{\gammamu}{\gamma^\mu}
\newcommand{\gs}{g_s}
\newcommand{\inftytext}{$\infty$}
\newcommand{\interval}[2]{#1:#2}
\newcommand{\kfrac}{K_{\text{frak}
\newcommand{\lP}{\ell_{\text{P}
\newcommand{\lP}{l_P}
\newcommand{\lambdah}{\ensuremath{\lambda_h}
\newcommand{\lambdah}{\lambda_h}
\newcommand{\lambdazero}{\lambda_0}
\newcommand{\mP}{m_{\text{P}
\newcommand{\mfield}{m(x,t)}
\newcommand{\mfield}{m}
\newcommand{\mh}{m_h}
\newcommand{\micrometer}{\ensuremath{\mu}
\newcommand{\mikrometer}{\ensuremath{\mu}
\newcommand{\myRightarrow}{\ensuremath{\Rightarrow}
\newcommand{\myapprox}{\ensuremath{\approx}
\newcommand{\myomega}{\ensuremath{\omega}
\newcommand{\myphi}{\ensuremath{\phi}
\newcommand{\mypi}{\ensuremath{\pi}
\newcommand{\mypropto}{\ensuremath{\propto}
\newcommand{\myrightarrow}{\ensuremath{\rightarrow}
\newcommand{\mysim}{\ensuremath{\sim}
\newcommand{\mysqrt}{\ensuremath{\sqrt}
\newcommand{\mytimes}{\ensuremath{\times}
\newcommand{\natunits}{\hbar = c = G = k_B = 1}
\newcommand{\natunits}{\text{(nat. Einh.)}
\newcommand{\natunits}{\text{(nat. units)}
\newcommand{\nulep}{\nu}
\newcommand{\nuzero}{\nu_0}
\newcommand{\partialop}{\ensuremath{\partial}
\newcommand{\pdTdt}{\frac{\partial\Tfieldt}
\newcommand{\pdTdx}{\nabla\Tfieldt}
\newcommand{\phiT}{\phi}
\newcommand{\pichar}{\pi}
\newcommand{\primrel}[1]{\mathbf{#1}
\newcommand{\rhoCMB}{\rho_{\text{CMB}
\newcommand{\rhoCasimir}{\rho_{\text{Casimir}
\newcommand{\rhoE}{\rho_E}
\newcommand{\rhofield}{\ensuremath{\rho}
\newcommand{\rzero}{r_0}
\newcommand{\slashk}{\cancel{k}
\newcommand{\slashp}{\cancel{p}
\newcommand{\slashq}{\cancel{q}
\newcommand{\tP}{t_P}
\newcommand{\tP}{t_{\text{P}
\newcommand{\tablescale}{0.9}
\newcommand{\tzero}{t_0}
\newcommand{\vect}[1]{\boldsymbol{#1}
\newcommand{\vecx}{\vec{x}
\newcommand{\vh}{v}
\newcommand{\vr}{\vec{r}
\newcommand{\warningx}{\color{red}
\newcommand{\warningx}{\textbf{!}
\newcommand{\warningx}{{\color{red}
\newcommand{\xiT}{\xi}
\newcommand{\xiconst}{\xi = \frac{4}
\newcommand{\xicoupling}{f(E/\Exi)}
\newcommand{\xigeom}{\xi_{\text{geom}
\newcommand{\xigeom}{\xi}
\newcommand{\xikonst}{\xi = \frac{4}
\newcommand{\xiparticle}{\xi_{\text{particle}
\newcommand{\xipar}{\ensuremath{\xi}
\newcommand{\xipar}{\xi_0}
\newcommand{\xipar}{\xi}
\newcommand{\xirat}{\xi_{\text{ratio}
\newtheorem{axiom}{Axiom}
\newtheorem{category}{Category-Theoretic Basis}
\newtheorem{category}{Kategorientheoretische Basis}
\newtheorem{corollary}[theorem]{Corollary}
\newtheorem{corollary}[theorem]{Korollar}
\newtheorem{corollary}{Corollary}
\newtheorem{corollary}{Korollar}
\newtheorem{definition}[theorem]{Definition}
\newtheorem{definition}{Definition}
\newtheorem{discovery}{Discovery}
\newtheorem{discovery}{Neue Entdeckung}
\newtheorem{discovery}{New Discovery}
\newtheorem{discovery}{Revolutionary Discovery}
\newtheorem{entdeckung}{Entdeckung}
\newtheorem{entdeckung}{Revolutionäre Entdeckung}
\newtheorem{erkenntnis}{Erkenntnis}
\newtheorem{erkenntnis}{Schlüsselerkenntnis}
\newtheorem{example}[theorem]{Beispiel}
\newtheorem{example}[theorem]{Example}
\newtheorem{example}{Beispiel}
\newtheorem{example}{Example}
\newtheorem{insight}{Central Insight}
\newtheorem{insight}{Insight}
\newtheorem{insight}{Key Insight}
\newtheorem{insight}{Wichtige Einsicht}
\newtheorem{insight}{Zentrale Einsicht}
\newtheorem{lemma}[theorem]{Lemma}
\newtheorem{lemma}{Lemma}
\newtheorem{principle}{Fundamental Principle}
\newtheorem{principle}{Fundamentales Prinzip}
\newtheorem{principle}{Grundlegendes Prinzip}
\newtheorem{principle}{Principle}
\newtheorem{principle}{Prinzip}
\newtheorem{prinzip}{Grundprinzip}
\newtheorem{proof_step}{Beweisschritt}
\newtheorem{proof_step}{Proof Step}
\newtheorem{proposition}[theorem]{Proposition}
\newtheorem{proposition}{Proposition}
\newtheorem{remark}[theorem]{Bemerkung}
\newtheorem{remark}[theorem]{Remark}
\newtheorem{theorem}{Theorem}
\newtheorem{warning}[theorem]{Warning}
\newtheorem{warning}[theorem]{Warnung}
\newunicodechar{±}{\ensuremath{\pm}
\newunicodechar{×}{\ensuremath{\times}
\newunicodechar{÷}{\ensuremath{\div}
\newunicodechar{ħ}{\ensuremath{\hbar}
\newunicodechar{Α}{\ensuremath{A}
\newunicodechar{Β}{\ensuremath{B}
\newunicodechar{Γ}{\ensuremath{\Gamma}
\newunicodechar{Δ}{\ensuremath{\Delta}
\newunicodechar{Ε}{\ensuremath{E}
\newunicodechar{Ζ}{\ensuremath{Z}
\newunicodechar{Η}{\ensuremath{H}
\newunicodechar{Θ}{\ensuremath{\Theta}
\newunicodechar{Ι}{\ensuremath{I}
\newunicodechar{Κ}{\ensuremath{K}
\newunicodechar{Λ}{\ensuremath{\Lambda}
\newunicodechar{Μ}{\ensuremath{M}
\newunicodechar{Ν}{\ensuremath{N}
\newunicodechar{Ξ}{\ensuremath{\Xi}
\newunicodechar{Ο}{\ensuremath{O}
\newunicodechar{Π}{\ensuremath{\Pi}
\newunicodechar{Ρ}{\ensuremath{P}
\newunicodechar{Σ}{\ensuremath{\Sigma}
\newunicodechar{Τ}{\ensuremath{T}
\newunicodechar{Υ}{\ensuremath{\Upsilon}
\newunicodechar{Φ}{\ensuremath{\Phi}
\newunicodechar{Χ}{\ensuremath{X}
\newunicodechar{Ψ}{\ensuremath{\Psi}
\newunicodechar{Ω}{\ensuremath{\Omega}
\newunicodechar{α}{\ensuremath{\alpha}
\newunicodechar{β}{\ensuremath{\beta}
\newunicodechar{γ}{\ensuremath{\gamma}
\newunicodechar{δ}{\ensuremath{\delta}
\newunicodechar{ε}{\ensuremath{\varepsilon}
\newunicodechar{ζ}{\ensuremath{\zeta}
\newunicodechar{η}{\ensuremath{\eta}
\newunicodechar{θ}{\ensuremath{\theta}
\newunicodechar{ι}{\ensuremath{\iota}
\newunicodechar{κ}{\ensuremath{\kappa}
\newunicodechar{λ}{\ensuremath{\lambda}
\newunicodechar{μ}{\ensuremath{\mu}
\newunicodechar{ν}{\ensuremath{\nu}
\newunicodechar{ξ}{\ensuremath{\xi}
\newunicodechar{ο}{\ensuremath{o}
\newunicodechar{π}{\ensuremath{\pi}
\newunicodechar{ρ}{\ensuremath{\rho}
\newunicodechar{σ}{\ensuremath{\sigma}
\newunicodechar{τ}{\ensuremath{\tau}
\newunicodechar{υ}{\ensuremath{\upsilon}
\newunicodechar{φ}{\ensuremath{\phi}
\newunicodechar{φ}{\ensuremath{\varphi}
\newunicodechar{χ}{\ensuremath{\chi}
\newunicodechar{ψ}{\ensuremath{\psi}
\newunicodechar{ω}{\ensuremath{\omega}
\newunicodechar{←}{\ensuremath{\leftarrow}
\newunicodechar{→}{\ensuremath{\rightarrow}
\newunicodechar{↔}{\ensuremath{\leftrightarrow}
\newunicodechar{⇐}{\ensuremath{\Leftarrow}
\newunicodechar{⇒}{\ensuremath{\Rightarrow}
\newunicodechar{⇔}{\ensuremath{\Leftrightarrow}
\newunicodechar{∂}{\ensuremath{\partial}
\newunicodechar{∅}{\ensuremath{\emptyset}
\newunicodechar{∇}{\ensuremath{\nabla}
\newunicodechar{∈}{\ensuremath{\in}
\newunicodechar{∉}{\ensuremath{\notin}
\newunicodechar{∏}{\ensuremath{\prod}
\newunicodechar{∑}{\ensuremath{\sum}
\newunicodechar{√}{\ensuremath{\sqrt}
\newunicodechar{∝}{\ensuremath{\propto}
\newunicodechar{∞}{\ensuremath{\infty}
\newunicodechar{∩}{\ensuremath{\cap}
\newunicodechar{∪}{\ensuremath{\cup}
\newunicodechar{∫}{\ensuremath{\int}
\newunicodechar{≈}{\ensuremath{\approx}
\newunicodechar{≠}{\ensuremath{\neq}
\newunicodechar{≤}{\ensuremath{\leq}
\newunicodechar{≥}{\ensuremath{\geq}
\newunicodechar{★}{\ensuremath{\star}
\newunicodechar{✓}{\checkmark}
\pgfplotsset{compat=1.17}
\pgfplotsset{compat=1.18}
\renewcommand{\cftchapfont}{\large\bfseries\color{blue}
\renewcommand{\cftchappagefont}{\large\bfseries\color{blue}
\renewcommand{\cftsecfont}{\bfseries}
\renewcommand{\cftsecfont}{\color{blue}
\renewcommand{\cftsecfont}{\large\bfseries\color{blue}
\renewcommand{\cftsecpagefont}{\bfseries}
\renewcommand{\cftsecpagefont}{\color{blue}
\renewcommand{\cftsecpagefont}{\large\bfseries\color{blue}
\renewcommand{\cftsubsecfont}{\color{blue!80!black}
\renewcommand{\cftsubsecfont}{\color{blue}
\renewcommand{\cftsubsecpagefont}{\color{blue!80!black}
\renewcommand{\cftsubsecpagefont}{\color{blue}
\renewcommand{\cftsubsubsecfont}{\color{blue!60!black}
\renewcommand{\cftsubsubsecfont}{\color{blue}
\renewcommand{\cftsubsubsecpagefont}{\color{blue!60!black}
\renewcommand{\cftsubsubsecpagefont}{\color{blue}
\renewcommand{\cfttoctitlefont}{\huge\bfseries\color{blue}
\renewcommand{\cfttoctitlefont}{\huge\bfseries}
\renewcommand{\familydefault}{\sfdefault}
\renewcommand{\footrulewidth}{0.4pt}
\renewcommand{\headrulewidth}{0.4pt}
\sisetup{locale = DE, group-separator = {.}
\sisetup{locale = DE}
\usetikzlibrary{arrows.meta,positioning,shapes.geometric}
\usetikzlibrary{decorations.pathmorphing, patterns, shapes.arrows}
\usetikzlibrary{intersections}
\usetikzlibrary{positioning, arrows.meta}
\usetikzlibrary{positioning, arrows}
\usetikzlibrary{positioning, shapes.geometric, arrows.meta}
\usetikzlibrary{positioning,shapes,arrows}

% Common settings
\setlength{\headheight}{15pt}
\pgfplotsset{compat=1.18}
\usetikzlibrary{positioning,shapes,arrows,arrows.meta}

% Hyperref setup
\hypersetup{
    colorlinks=true,
    linkcolor=blue,
    citecolor=blue,
    urlcolor=blue
}


\title{RSA De}
\author{Johann Pascher}
\date{\today}

\begin{document}

\maketitle
\tableofcontents

\begin{abstract}
		Diese Arbeit präsentiert eine mathematische Analyse des T0-Shor Algorithmus basierend auf einer Energiefeld-Formulierung. Wir untersuchen die theoretischen Grundlagen der Zeit-Masse-Dualität $T(x,t) \cdot m(x,t) = 1$ und deren Anwendung auf die Ganzzahlfaktorisierung. Die Analyse umfasst Feldgleichungen, wellenartiges Verhalten ähnlich der akustischen Ausbreitung und materialabhängige Parameter abgeleitet aus der Vakuumphysik. Wir leiten Skalierungsbeziehungen für verschiedene Raumdimensionen ab und untersuchen die Rolle der Rechengenauigkeit für die Algorithmusleistung. Das mathematische Framework wird auf Konsistenz überprüft und praktische Limitationen werden identifiziert.
	\end{abstract}
	
	\tableofcontents
	\newpage
	
	# Einleitung
	
	Der T0-Shor Algorithmus stellt eine theoretische Erweiterung von Shors Faktorisierungsalgorithmus dar, basierend auf Energiefelddynamik anstelle quantenmechanischer Superposition. Diese Arbeit untersucht die mathematischen Grundlagen dieses Ansatzes ohne Behauptungen über praktische Implementierbarkeit oder Überlegenheit gegenüber bestehenden Methoden.
	
	## Theoretisches Framework
	
	Das T0-Modell führt folgende fundamentale mathematische Strukturen ein:
	
	
```math-align

		\text{Zeit-Masse-Dualität}: \quad &T(x,t) \cdot m(x,t) = 1 \label{eq:duality}\\
		\text{Feldgleichung}: \quad &\nabla^2 T(x) = -\frac{\rho(x)}{T(x)^2} \label{eq:field}\\
		\text{Energieentwicklung}: \quad &\frac{\partial^2 E}{\partial t^2} = -\omega^2 E \label{eq:evolution}
	
```

	
	Der Kopplungsparameter $\xipar$ wird theoretisch aus Higgs-Feld-Wechselwirkungen abgeleitet:
	
```math-equation

		\xipar = g_H \cdot \frac{\langle\phi\rangle}{v_{EW}} \label{eq:xi_higgs}
	
```

	wobei $g_H$ die Higgs-Kopplungskonstante, $\langle\phi\rangle$ der Vakuumerwartungswert und $v_{EW} = 246$ GeV die elektroschwache Skala ist.
	
	# Mathematische Grundlagen
	
	## Wellenartiges Verhalten von T0-Feldern
	
	Das T0-Feld zeigt wellenartige Ausbreitungscharakteristika analog zu akustischen Wellen in Medien. Die fundamentale Wellengleichung für T0-Felder lautet:
	
	
```math-equation

		\nabla^2 T - \frac{1}{c_{T0}^2} \frac{\partial^2 T}{\partial t^2} = -\frac{\rho(x,t)}{T(x,t)^2} \label{eq:wave_equation}
	
```

	
	wobei $c_{T0}$ die T0-Feld-Ausbreitungsgeschwindigkeit im Medium ist, analog zur Schallgeschwindigkeit.
	
	## Mediumabhängige Eigenschaften
	
	Ähnlich wie akustische Wellen hängt die T0-Feld-Ausbreitung kritisch von den Mediumeigenschaften ab:
	
	\textbf{T0-Feld-Geschwindigkeit in verschiedenen Medien}:
	
```math-align

		c_{T0,vacuum} &= c \sqrt{\frac{\xipar_0}{\xipar_{vacuum}}} \\
		c_{T0,metal} &= c \sqrt{\frac{\xipar_0 \epsilon_r}{\xipar_{vacuum}}} \\
		c_{T0,dielectric} &= \frac{c}{\sqrt{\epsilon_r \mu_r}} \sqrt{\frac{\xipar_0}{\xipar_{vacuum}}} \\
		c_{T0,plasma} &= c \sqrt{1 - \frac{\omega_p^2}{\omega^2}} \sqrt{\frac{\xipar_0}{\xipar_{vacuum}}}
	
```

	
	wobei $\omega_p$ die Plasmafrequenz und $\epsilon_r$, $\mu_r$ die relative Permittivität und Permeabilität sind.
	
	## Randbedingungen und Reflexionen
	
	An Grenzflächen zwischen verschiedenen Medien erfüllen T0-Felder Randbedingungen ähnlich elektromagnetischen Wellen:
	
	\textbf{Kontinuitätsbedingungen}:
	
```math-align

		T_1|_{interface} &= T_2|_{interface} \quad \text{(Feldkontinuität)} \\
		\frac{1}{m_1} \frac{\partial T_1}{\partial n}\bigg|_{interface} &= \frac{1}{m_2} \frac{\partial T_2}{\partial n}\bigg|_{interface} \quad \text{(Flusskontinuität)}
	
```

	
	\textbf{Reflexions- und Transmissionskoeffizienten}:
	
```math-align

		r &= \frac{Z_1 - Z_2}{Z_1 + Z_2} \quad \text{(Reflexionskoeffizient)} \\
		t &= \frac{2Z_1}{Z_1 + Z_2} \quad \text{(Transmissionskoeffizient)}
	
```

	
	wobei $Z_i = \sqrt{m_i/T_i}$ die T0-Feld-Impedanz in Medium $i$ ist.
	
	## Hyperbolische Geometrie im Dualitätsraum
	
	Die Zeit-Masse-Dualität (Gl.~\ref{eq:duality}) definiert eine hyperbolische Metrik im $(T,m)$ Parameterraum:
	
	
```math-equation

		ds^2 = \frac{dT \cdot dm}{T \cdot m} = \frac{d(\ln T) \cdot d(\ln m)}{T \cdot m}
	
```

	
	Diese Geometrie ist charakterisiert durch:
	
		- Konstante negative Krümmung: $K = -1$
		- Invariantes Maß: $d\mu = \frac{dT \, dm}{T \cdot m}$
		- Isometriegruppe: $PSL(2,\mathbb{R})$
	
	
	## Atomskalige T0-Feld-Parameter
	
	Da die Vakuumbedingungen bekannt sind, kann das atomare T0-Feld-Verhalten aus Fundamentalkonstanten abgeleitet werden:
	
	\textbf{Vakuum T0-Feld-Basislinie}:
	
```math-align

		c_{T0,vacuum} &= c = 2,998 \times 10^8 \text{ m/s} \\
		\xipar_{vacuum} &= \xipar_0 = \frac{g_H \langle\phi\rangle}{v_{EW}} \\
		Z_{vacuum} &= Z_0 = \sqrt{\frac{\mu_0}{\epsilon_0}} = 376,73 \text{ $\Omega$}
	
```

	
	\textbf{Atomskalige Ableitungen}:
	
	Für das Wasserstoffatom (Fundamentalfall):
	
```math-align

		a_0 &= \frac{4\pi\epsilon_0\hbar^2}{m_e e^2} = 5,292 \times 10^{-11} \text{ m} \quad \text{(Bohr-Radius)} \\
		\alpha &= \frac{e^2}{4\pi\epsilon_0\hbar c} = 7,297 \times 10^{-3} \quad \text{(Feinstrukturkonstante)} \\
		r_{e} &= \frac{e^2}{4\pi\epsilon_0 m_e c^2} = 2,818 \times 10^{-15} \text{ m} \quad \text{(klassischer Elektronenradius)}
	
```

	
	\textbf{T0-Feld-Atomparameter}:
	
```math-align

		c_{T0,atom} &= c \cdot \alpha = 2,19 \times 10^6 \text{ m/s} \\
		\xipar_{atom} &= \xipar_0 \cdot \frac{E_{Rydberg}}{m_e c^2} = \xipar_0 \cdot \frac{\alpha^2}{2} \\
		\lambda_{T0,atom} &= \frac{2\pi a_0}{\alpha} = 2,426 \times 10^{-9} \text{ m}
	
```

	
	\textbf{Skalierung für verschiedene Atome}:
	
	Für Atom mit Kernladung $Z$ und Massenzahl $A$:
	
```math-align

		c_{T0,Z} &= c_{T0,atom} \cdot Z^{2/3} \quad \text{(Geschwindigkeitsskalierung)} \\
		\xipar_{Z} &= \xipar_{atom} \cdot \frac{Z^4}{A} \quad \text{(Kopplungsskalierung)} \\
		a_{Z} &= \frac{a_0}{Z} \quad \text{(Größenskalierung)} \\
		E_{binding,Z} &= 13,6 \text{ eV} \cdot Z^2 \quad \text{(Energieskalierung)}
	
```

	
	# T0-Shor Algorithmus-Formulierung
	
	## Geometrisches Hohlraum-Design für Periodenfindung
	
	Der T0-Shor Algorithmus nutzt geometrische Resonanzhohlräume zur Periodendetektion, analog zu akustischen Resonatoren:
	
	\textbf{Resonanzhohlraum-Dimensionen} für Periode $r$:
	
```math-equation

		L_{cavity} = n \cdot \frac{\lambda_{T0}}{2} = n \cdot \frac{c_{T0} \cdot r}{2f_0}
	
```

	
	wobei $f_0$ die fundamentale Antriebsfrequenz und $n$ die Modenzahl ist.
	
	\textbf{Gütefaktor} der Resonanz:
	
```math-equation

		Q = \frac{f_r}{\Delta f} = \frac{\pi}{\xipar} \cdot \frac{L_{cavity}}{\lambda_{T0}}
	
```

	
	Höhere $Q$-Werte bieten schärfere Periodendetektion, erfordern aber längere Beobachtungszeiten.
	
	## Multi-Moden-Resonanzanalyse
	
	Anstelle der Quanten-Fourier-Transformation verwendet der T0-Shor Algorithmus Multi-Moden-Hohlraumanalyse:
	
	
```math-align

		\text{Modenspektrum}: \quad &T(x,y,z,t) = \sum_{mnp} A_{mnp}(t) \psi_{mnp}(x,y,z) \\
		\text{Periodendetektion}: \quad &r = \frac{c_{T0}}{2f_{resonance}} \cdot \frac{geometry\_factor}{mode\_number}
	
```

	
	# Selbstverstärkende $\xipar$-Optimierung: Die Fehlerreduktions-Rückkopplungsschleife
	
	## Die fundamentale Entdeckung: Rechenfehler verschlechtern $\xipar$
	
	Eine kritische Erkenntnis ergibt sich: Rechengenauigkeit beeinflusst direkt $\xipar$-Parameter-Werte und erschafft einen selbstverstärkenden Optimierungszyklus:
	
	\textbf{Fehlerabhängige $\xipar$-Verschlechterung}:
	
```math-equation

		\xipar_{effective} = \xipar_{ideal} \cdot \exp\left(-\alpha \sum_{i} p_{error,i} \cdot w_i\right)
	
```

	
	wobei $p_{error,i}$ Fehlerwahrscheinlichkeiten und $w_i$ Kritikalitätsgewichte sind.
	
	\textbf{Die selbstverstärkende Beziehung}:
	
```math-equation

		\text{Weniger Fehler} \rightarrow \text{Höheres } \xipar \rightarrow \text{Bessere Feldkohärenz} \rightarrow \text{Noch weniger Fehler}
	
```

	
	## Mathematisches Modell der Rückkopplungsschleife
	
	\textbf{Differentialgleichung für $\xipar$-Entwicklung}:
	
```math-equation

		\frac{d\xipar}{dt} = \beta \xipar \left(1 - \frac{R_{error}}{R_{threshold}}\right) - \gamma \xipar \frac{R_{error}}{R_{reference}}
	
```

	
	Kritische Erkenntnis: Wenn $R_{error} < R_{threshold}$, wächst $\xipar$ exponentiell.
	
	\textbf{Typische Schwellenwerte}:
	
```math-align

		R_{critical} &\approx 10^{-12} \text{ Fehler pro Operation} \\
		R_{64bit} &\approx 10^{-16} \text{ (bereits unter Schwellenwert)} \\
		R_{32bit} &\approx 10^{-7} \text{ (über Schwellenwert)}
	
```

	
	Standard 64-Bit Arithmetik ist bereits im $\xipar$-Verstärkungsbereich.
	
	# Vakuum-abgeleitete Atomparameter: Keine freien Parameter
	
	## Fundamentale Parameter-Ableitung
	
	Da Vakuumbedingungen bekannt sind, können alle atomaren T0-Parameter aus Fundamentalkonstanten abgeleitet werden:
	
	\textbf{Vakuum-Basislinie}:
	
```math-align

		c_{T0,vacuum} &= c = 2,998 \times 10^8 \text{ m/s} \\
		\xipar_{vacuum} &= \xipar_0 = \frac{g_H \langle\phi\rangle}{v_{EW}} \quad \text{(Higgs-abgeleitet)} \\
		Z_{vacuum} &= Z_0 = 376,73 \text{ $\Omega$}
	
```

	
	\textbf{Materialspezifische Vorhersagen}:
	
	Keine freien Parameter - alle $\xipar$-Werte sind berechenbar:
	
```math-align

		\xipar_{Si} &= \xipar_0 \cdot 0,98 \cdot \frac{E_g}{k_B T} = 43,7 \xipar_0 \quad \text{(bei 300K)} \\
		\xipar_{metal} &= \xipar_0 \sqrt{\frac{n e^2}{\epsilon_0 m_e \omega^2}} \approx (10^{-4} \text{ bis } 10^{-3}) \xipar_0 \\
		\xipar_{SC} &= \xipar_0 \cdot \frac{\Delta}{k_B T_c} \cdot \tanh\left(\frac{\Delta}{2k_B T}\right)
	
```

	
	\textbf{Experimentell testbare Vorhersagen}:
	
```math-align

		\text{Temperaturskalierung}: \quad &\xipar(T_2)/\xipar(T_1) = T_1/T_2 \\
		\text{Isotopeffekt}: \quad &\xipar(^{13}C)/\xipar(^{12}C) = \sqrt{12/13} = 0,962 \\
		\text{Druckabhängigkeit}: \quad &\xipar(p) = \xipar_0 \left(1 + \kappa \frac{\Delta p}{p_0}\right)
	
```

	
	# $\xipar$ als multifunktionaler Parameter: Jenseits einfacher Kopplung
	
	## Multiple versteckte Funktionen von $\xipar$
	
	$\xipar$ erfüllt mehrere fundamentale Rollen jenseits einfacher Feld-Materie-Kopplung:
	
	
```math-align

		\text{1. Kopplungsstärke}: \quad &\xipar_{coupling} = \text{Feld-Materie-Wechselwirkung} \\
		\text{2. Asymmetrie-Generator}: \quad &\xipar_{asymmetry} = \text{Richtungspräferenz} \\
		\text{3. Skalen-Setzer}: \quad &\xipar_{scale} = \text{charakteristische Länge/Zeit} \\
		\text{4. Informations-Kodierer}: \quad &\xipar_{info} = \text{Berechnungskomplexitäts-Modifikator} \\
		\text{5. Symmetriebrecher}: \quad &\xipar_{symmetry} = \text{spontane Ordnung}
	
```

	
	## $\xipar$-induzierte Berechnungsasymmetrien
	
	\textbf{Berechnungschiralität}:
	
	Auch in mathematisch symmetrischen Operationen erschafft $\xipar$ Berechnungspräferenzen:
	
```math-align

		\text{Vorwärtsberechnung}: \quad &\xipar_{forward} = \xipar_0 \\
		\text{Umkehrberechnung}: \quad &\xipar_{inverse} = \xipar_0 / \alpha \quad (\alpha > 1) \\
		\text{Verifikation}: \quad &\xipar_{verify} = \xipar_0 \cdot \beta \quad (\beta > 1)
	
```

	
	Dies erschafft Berechnungschiralität wo Verifikation einfacher ist als Berechnung.
	
	## $\xipar$-Gedächtnis und Geschichtsabhängigkeit
	
	\textbf{$\xipar$ behält Berechnungsgeschichte}:
	
```math-equation

		\xipar(t) = \xipar_0 + \int_0^t K(t-\tau) \cdot f(\text{computation}(\tau)) \, d\tau
	
```

	
	wobei $K(t-\tau)$ ein Gedächtniskern ist.
	
	# Dimensionale Skalierung: Fundamentale Unterschiede zwischen 2D und 3D
	
	## Wellenausbreitungs-Skalierungsgesetze
	
	Der fundamentale Unterschied zwischen 2D und 3D Raum beeinflusst T0-Feld-Verhalten tiefgreifend:
	
	\textbf{Dimensionale Feldgleichungen}:
	
```math-align

		\text{2D}: \quad &\frac{1}{r} \frac{\partial}{\partial r}\left(r \frac{\partial T}{\partial r}\right) = -\frac{\rho(r)}{T(r)^2} \\
		\text{3D}: \quad &\frac{1}{r^2} \frac{\partial}{\partial r}\left(r^2 \frac{\partial T}{\partial r}\right) = -\frac{\rho(r)}{T(r)^2}
	
```

	
	\textbf{Green-Funktions-Unterschiede}:
	
```math-align

		G_{2D}(r) &= -\frac{1}{2\pi} \ln(r) \quad \text{(logarithmischer Abfall)} \\
		G_{3D}(r) &= \frac{1}{4\pi r} \quad \text{(Potenzgesetz-Abfall)}
	
```

	
	## Kritische Dimensionsschwellenwerte
	
	\textbf{Untere kritische Dimension}: $d_c^{lower} = 2$
	
	Unter 2D können T0-Felder nicht konventionell propagieren:
	
```math-equation

		\text{1D}: \quad T(x) = T_0 + A|x| \quad \text{(lineares Wachstum, unphysikalisch)}
	
```

	
	\textbf{Obere kritische Dimension}: $d_c^{upper} = 4$
	
	Über 4D wird die Molekularfeld-Theorie exakt:
	
```math-equation

		\text{4D+}: \quad \xipar_{eff} = \xipar_0 \quad \text{(dimensionsunabhängig)}
	
```

	
	## Algorithmische Leistungsskalierung
	
	\textbf{Dimensionale Skalierung beeinflusst T0-Shor Leistung}:
	
```math-align

		\text{2D Implementierung}: \quad F_{2D} &= \sqrt{\ln(N)} \quad \text{(logarithmisch)} \\
		\text{3D Implementierung}: \quad F_{3D} &= N^{1/3} \quad \text{(Potenzgesetz)}
	
```

	
	\textbf{Optimale Geometrien nach Dimension}:
	
```math-align

		\text{2D}: \quad &\text{Lange, dünne Strukturen bevorzugt} \\
		&Q \propto L/\lambda_{T0} \\
		\text{3D}: \quad &\text{Kompakte, sphärische Geometrien optimal} \\
		&Q \propto (V/\lambda_{T0}^3)^{1/3}
	
```

	
	# Die fundamentale Natur von Zahlen und Primstruktur
	
	## Primzahlen als das Gerüst der Mathematik
	
	Der Grund warum alle Periodenfindungsalgorithmen funktionieren (FFT, Quanten-Shor, T0-Shor) liegt in der fundamentalen Struktur unseres Zahlensystems:
	
	\textbf{Primzahlen als mathematische Atome}:
	
```math-equation

		\text{Jede Ganzzahl } n > 1: \quad n = p_1^{a_1} \cdot p_2^{a_2} \cdot ... \cdot p_k^{a_k} \quad \text{(eindeutig)}
	
```

	
	Primzahlen bilden das fundamentale Gerüst - jede Zahl ist vollständig durch Primzahlen bestimmt.
	
	\textbf{Warum Periodizität aus Primstruktur entsteht}:
	
```math-align

		\text{Euler-Theorem}: \quad &a^{\phi(N)} \equiv 1 \pmod{N} \\
		\text{Periodizität}: \quad &f(x) = a^x \bmod N \text{ ist inhärent periodisch} \\
		\text{Universelles Prinzip}: \quad &\text{Primstruktur} \rightarrow \text{Periodizität} \rightarrow \text{Fourier-Detektion}
	
```

	
	\textbf{Warum Periode Faktorisierungsinformation enthält}:
	
```math-equation

		a^r \equiv 1 \pmod{N} \Rightarrow a^r - 1 = (a^{r/2} - 1)(a^{r/2} + 1) \equiv 0 \pmod{N}
	
```

	
	Die Periode $r$ kodiert die Primfaktoren durch diese algebraische Beziehung.
	
	# Kritische Bewertung: Warum T0-Shor nur für kleine Zahlen funktioniert
	
	## Die Präzisionsbarriere
	
	Trotz der theoretischen Eleganz steht T0-Shor vor einer fundamentalen Präzisionslimitierung die seine praktische Anwendbarkeit einschränkt:
	
	\textbf{Erforderliche Resonanzpräzision für Periode r}:
	
```math-equation

		\Delta f_{required} = \frac{f_0}{r} - \frac{f_0}{r+1} = \frac{f_0}{r(r+1)} \approx \frac{f_0}{r^2}
	
```

	
	Für kryptographisch relevante Zahlen wo $r \approx N$:
	
```math-equation

		\Delta f_{required} \approx \frac{f_0}{N^2}
	
```

	
	\textbf{Rechenpräzisionsgrenzen}:
	
```math-align

		\text{64-Bit Präzision}: \quad &\epsilon \approx 10^{-16} \rightarrow N_{max} \approx 10^8 \text{ (27 Bits)} \\
		\text{128-Bit Präzision}: \quad &\epsilon \approx 10^{-34} \rightarrow N_{max} \approx 10^{17} \text{ (56 Bits)} \\
		\text{1024-Bit RSA erfordert}: \quad &\epsilon \approx 10^{-617} \text{ (unmöglich)}
	
```

	
	## Die Präzisionsbarriere und Skalierungslimitationen
	
	Wichtige Klarstellung: T0-Shor funktioniert theoretisch für große Zahlen. Die Limitationen sind praktisch, nicht theoretisch:
	
	\textbf{Fundamentale Skalierungsherausforderungen}:
	
```math-align

		\text{Speicheranforderungen}: \quad &M(N) = O(N) \text{ Feldpunkte} \\
		\text{Rechenpräzision}: \quad &\epsilon_{required} = O(1/N^2) \\
		\text{Feldauflösung}: \quad &\Delta r = O(1/N) \text{ für Periodendetektion} \\
		\text{Operationszahl}: \quad &\text{Immer noch } O(\log N) \text{ pro erfolgreicher Vorhersage}
	
```

	
	## Vergleich mit bestehenden Methoden
	
	\begin{table}[htbp]
		\centering
		\begin{tabular}{lcccc}
			\toprule
			\textbf{Methode} & \textbf{Operationen (kleine N)} & \textbf{Operationen (große N)} & \textbf{Erfolgsrate} & \textbf{Hardware} \\
			\midrule
			Triviale Faktorisierung & $O(\sqrt{N})$ & $O(\sqrt{N})$ & 100\% & Standard \\
			Klassische FFT & $O(N \log N)$ & $O(N \log N)$ & 100\% & Standard \\
			Quanten-Shor & $O((\log N)^3)$ & $O((\log N)^3)$ & $\approx$50\% & Quantum \\
			T0-Shor (Vorhersage-Treffer) & $O(\log N)$ & $O(\log N)$ & Variabel & Standard \\
			T0-Shor (keine Vorhersage) & $O(N \log N)$ & Durch Präzision begrenzt & Variabel & Standard \\
			\bottomrule
		\end{tabular}
		\caption{Realistische Vergleich von Faktorisierungsmethoden}
		\label{tab:method_comparison_realistic}
	\end{table}
	
	\textbf{Quantencomputer und das I/O-Engpass}:
	
	Quantencomputer mit elektronenbasiertem Speicher haben einen theoretischen Speichervorteil, stehen aber vor denselben fundamentalen I/O-Limitationen:
	
	\begin{table}[htbp]
		\centering
		\begin{tabular}{lcccc}
			\toprule
			\textbf{System} & \textbf{Speicher} & \textbf{Eingabe-Abbildung} & \textbf{Ausgabe-Auslesen} & \textbf{Engpass} \\
			\midrule
			T0-Shor & RAM-Limitierung & Direkt & Direkt & Speicherskalierung \\
			QC & Elektronenzustände & Exponentielle Kodierung & Messkollaps & I/O-Komplexität \\
			T0 + QC & Elektronenzustände & Selbes QC-Problem & Selbes QC-Problem & I/O-Komplexität \\
			\bottomrule
		\end{tabular}
		\caption{Speichersysteme und ihre fundamentalen Engpässe}
		\label{tab:memory_bottlenecks}
	\end{table}
	
	# Schlussfolgerungen
	
	## Zentrale Erkenntnisse
	
	Die Zeit-Masse-Dualität führt zu einer mathematisch konsistenten Erweiterung des Shor-Algorithmus mit folgenden Eigenschaften:
	
	
		- Theoretischer Rahmen: Hyperbolische Geometrie im Dualitätsraum
		- Wellencharakteristik: T0-Felder verhalten sich ähnlich akustischen Wellen
		- Vakuum-Ableitung: Alle Parameter aus Fundamentalkonstanten berechenbar
		- Selbstverstärkung: Fehlerreduktion verbessert $\xipar$-Parameter
		- Multifunktionalität: $\xipar$ hat Rollen jenseits einfacher Kopplung
		- Dimensionale Effekte: 2D und 3D verhalten sich fundamental unterschiedlich
		- Praktische Grenzen: Präzisions- und Speicheranforderungen begrenzen Anwendbarkeit
	
	
	## Offene mathematische Fragen
	
	Mehrere mathematische Aspekte erfordern weitere Untersuchung:
	
	
		- Rigoroser Konvergenzbeweis für Feldentwicklungsgleichungen
		- Analyse nicht-sphärisch symmetrischer Konfigurationen
		- Untersuchung chaotischer Dynamik in Massenfeld-Evolution
		- Verbindung zwischen $\xipar$-Parameter und experimentell messbaren Größen
	
	
	Der T0-Shor Algorithmus stellt eine interessante theoretische Konstruktion dar, die Konzepte aus Differentialgeometrie, Feldtheorie und Berechnungskomplexität verbindet. Seine praktischen Vorteile gegenüber bestehenden Methoden bleiben jedoch abhängig von mehreren unbewiesenen Annahmen über die physikalische Realisierbarkeit des zugrundeliegenden mathematischen Frameworks.

\end{document}
