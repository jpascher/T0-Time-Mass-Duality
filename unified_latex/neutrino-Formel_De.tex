\documentclass[11pt,a4paper,openany]{book}

% Essential packages
\usepackage[utf8]{inputenc}
\usepackage[T1]{fontenc}
\usepackage[english]{babel}
\usepackage[a4paper,margin=2.5cm]{geometry}
\usepackage{lmodern}

% Math and physics packages
\usepackage{amsmath}
\usepackage{amssymb}
\usepackage{amsthm}
\usepackage{mathtools}
\usepackage{physics}
\usepackage{siunitx}

% Graphics and tables
\usepackage{graphicx}
\usepackage[table,xcdraw]{xcolor}
\usepackage{tikz}
\usepackage{pgfplots}
\usepackage{tcolorbox}
\usepackage{booktabs}
\usepackage{array}
\usepackage{longtable}
\usepackage{float}

% Document formatting
\usepackage{fancyhdr}
\usepackage{tocloft}
\usepackage{hyperref}
\usepackage{cleveref}
\usepackage{microtype}
\usepackage{enumitem}
\usepackage{newunicodechar}

% Additional packages (cleaned up - removed duplicates)
\usepackage{adjustbox}
\usepackage{algorithm}
\usepackage{algorithmic}
\usepackage{amsfonts}
\usepackage{bm}
\usepackage{braket}
\usepackage{breakurl}
\usepackage{cancel}
\usepackage{caption}
\usepackage{cite}
\usepackage{csquotes}
\usepackage{doi}
\usepackage{forest}
\usepackage{gensymb}
\usepackage{hyphenat}
\usepackage{listings}
\usepackage{mdframed}
\usepackage{multicol}
\usepackage{multirow}
\usepackage{natbib}
\usepackage{pdflscape}
\usepackage{ragged2e}
\usepackage{setspace}
\usepackage{slashed}
\usepackage{tabularx}
\usepackage{textcomp}
\usepackage{textgreek}
\usepackage{upgreek}
\usepackage{url}

% Color definitions (FIXED: removed extra \definecolor commands)
\definecolor{blue}{rgb}{0,0,1}
\definecolor{boxgray}{RGB}{240,240,240}
\definecolor{deepblue}{RGB}{0,0,127}
\definecolor{deepgreen}{RGB}{0,127,0}
\definecolor{deepred}{RGB}{191,0,0}
\definecolor{t0blue}{RGB}{0,102,204}
\definecolor{t0green}{RGB}{0,153,0}
\definecolor{t0orange}{RGB}{255,152,0}
\definecolor{t0purple}{RGB}{102,0,204}
\definecolor{t0red}{RGB}{204,0,0}
\definecolor{t0yellow}{RGB}{255,204,0}

% TikZ libraries
\usetikzlibrary{arrows,shapes,positioning,calc,patterns,decorations.pathmorphing,decorations.markings}

% PGFPlots setup
\pgfplotsset{compat=1.18}

% Hyperref setup
\hypersetup{
    colorlinks=true,
    linkcolor=blue,
    filecolor=magenta,
    urlcolor=cyan,
    citecolor=green,
    pdftitle={T0 Theory Document},
    pdfauthor={Johann Pascher},
    pdfsubject={T0 Theory},
    pdfkeywords={T0, physics, theory}
}

% Header and footer
\pagestyle{fancy}
\fancyhf{}
\fancyhead[LE,RO]{\thepage}
\fancyhead[RE]{\leftmark}
\fancyhead[LO]{\rightmark}
\fancyfoot[C]{T0 Theory - Johann Pascher}

% Theorem environments
\theoremstyle{definition}
\newtheorem{definition}{Definition}[section]
\newtheorem{theorem}{Theorem}[section]
\newtheorem{lemma}[theorem]{Lemma}
\newtheorem{proposition}[theorem]{Proposition}
\newtheorem{corollary}[theorem]{Corollary}
\theoremstyle{remark}
\newtheorem{remark}{Remark}[section]
\newtheorem{example}{Example}[section]

% Custom commands (common across T0 documents)
\newcommand{\T}[1]{\text{#1}}
\newcommand{\mat}[1]{\mathbf{#1}}
\newcommand{\E}{\mathrm{e}}
\newcommand{\I}{\mathrm{i}}
\newcommand{\diff}{\mathrm{d}}
\newcommand{\Real}{\mathrm{Re}}
\newcommand{\Imag}{\mathrm{Im}}


\begin{document}

\maketitle
\tableofcontents

\begin{abstract}
		Dieses Dokument präsentiert eine mathematisch konsistente Formel-Struktur für Neutrino-Berechnungen im Rahmen des T0-Modells, basierend auf der Hypothese gleicher Massen für alle Flavour-Zustände (\(\nu_e, \nu_\mu, \nu_\tau\)). Die Neutrino-Masse wird durch die Photon-Analogie (\(\frac{\xipar^2}{2}\)-Suppression) abgeleitet, und Oszillationen werden durch geometrische Phasen basierend auf \( T_x \cdot m_x = 1 \) erklärt, wobei die Quantenzahlen (\(n, \ell, j\)) die Phasenunterschiede bestimmen. Ein plausibler Zielwert für die Neutrino-Masse (\(m_\nu = 15 \text{ meV}\)) wird aus empirischen Daten (kosmologische Grenzen) abgeleitet. Die T0-Theorie basiert auf spekulativen geometrischen Harmonien ohne empirische Basis und ist mit hoher Wahrscheinlichkeit unvollständig oder falsch. Die wissenschaftliche Integrität erfordert die klare Trennung zwischen mathematischer Korrektheit und physikalischer Gültigkeit.
	\end{abstract}
	
	\tableofcontents
	\newpage
	
	# Präambel: Wissenschaftliche Ehrlichkeit
	
	\begin{warning}
		\textbf{KRITISCHE EINSCHRÄNKUNG:} Die folgenden Formeln für Neutrino-Massen sind \textbf{spekulative Extrapolationen} basierend auf der ungetesteten Hypothese, dass Neutrinos geometrischen Harmonien folgen und alle Flavour-Zustände gleiche Massen besitzen. Diese Hypothese hat \textbf{keine empirische Basis} und ist mit hoher Wahrscheinlichkeit unvollständig oder falsch. Die mathematischen Formeln sind dennoch intern konsistent und fehlerfrei formuliert.
		
		\vspace{0.5cm}
		\textbf{Wissenschaftliche Integrität bedeutet:}
		
			- Ehrlichkeit über spekulative Natur der Vorhersagen
			- Mathematische Korrektheit trotz physikalischer Unsicherheit
			- Klare Trennung zwischen Hypothesen und verifizierten Fakten
		
	\end{warning}
	
	# Neutrinos als ''fast-masselose Photonen'': Die T0-Photon-Analogie
	
	\begin{speculation}
		\textbf{Fundamentale T0-Einsicht:} Neutrinos können als ''gedämpfte Photonen'' verstanden werden.
		
		Die bemerkenswerte Ähnlichkeit zwischen Photonen und Neutrinos legt eine tiefere geometrische Verwandtschaft nahe:
		
			- \textbf{Geschwindigkeit:} Beide propagieren nahezu mit Lichtgeschwindigkeit
			- \textbf{Durchdringung:} Beide haben extreme Durchdringungsfähigkeit
			- \textbf{Masse:} Photon exakt masselos, Neutrino quasi-masselos
			- \textbf{Wechselwirkung:} Photon elektromagnetisch, Neutrino schwach
		
	\end{speculation}
	
	## Photon-Neutrino-Korrespondenz
	
	\begin{important}
		\textbf{Physikalische Parallelen:}
		
```math-align

			\text{Photon:} \quad &E^2 = (pc)^2 + 0 \quad \text{(perfekt masselos)} \\
			\text{Neutrino:} \quad &E^2 = (pc)^2 + \left(\sqrt{\frac{\xipar^2}{2}} m c^2\right)^2 \quad \text{(quasi-masselos)}
		
```

		
		\textbf{Geschwindigkeitsvergleich:}
		
```math-align

			v_\gamma &= c \quad \text{(exakt)} \\
			v_\nu &= c \times \left(1 - \frac{\xipar^2}{2}\right) \approx 0.9999999911 \times c
		
```

		
		Die Geschwindigkeitsdifferenz beträgt nur \(8.89 \times 10^{-9}\) -- praktisch unmessbar!
	\end{important}
	
	## Doppelte \(\xipar\)-Suppression aus Photon-Analogie
	
	\begin{formula}
		\textbf{T0-Hypothese:} Neutrino = Photon mit geometrischer Doppeldämpfung
		
		Wenn Neutrinos ''fast-Photonen'' sind, dann ergeben sich zwei Suppressionsfaktoren:
		
			- \textbf{Erster \(\xipar\)-Faktor:} ''Fast masselos'' (wie Photon, aber nicht perfekt)
			- \textbf{Zweiter \(\xipar\)-Faktor:} ''Schwache Wechselwirkung'' (geometrische Kopplung)
			- \textbf{Resultat:} \(m_\nu \propto \frac{\xipar^2}{2}\), konsistent mit der Geschwindigkeitsdifferenz \(v_\nu = c \times \left(1 - \frac{\xipar^2}{2}\right)\)
		
		
		\textbf{Wechselwirkungsstärken-Vergleich:}
		
```math-align

			\sigma_\gamma &\sim \alpha_{\text{EM}} \approx \frac{1}{137} \\
			\sigma_\nu &\sim \frac{\xipar^2}{2} \times G_F \approx 8.888888 \times 10^{-9}
		
```

		
		Das Verhältnis \(\sigma_\nu/\sigma_\gamma \sim \frac{\xipar^2}{2}\) bestätigt die geometrische Suppression!
	\end{formula}
	
	# Neutrino-Oszillationen
	
	\begin{important}
		\textbf{Neutrino-Oszillationen:} Neutrinos können ihre Identität (Flavour) während des Fluges ändern – ein Phänomen, das als Neutrino-Oszillation bekannt ist. Ein Neutrino, das als Elektron-Neutrino (\(\nu_e\)) erzeugt wurde, kann sich später als Myon-Neutrino (\(\nu_\mu\)) oder Tau-Neutrino (\(\nu_\tau\)) messen lassen und umgekehrt.
		
		Dieses Verhalten wird in der Standardphysik durch die Mischung der Masseneigenzustände (\(\nu_1, \nu_2, \nu_3\)) beschrieben, die durch die PMNS-Matrix (Pontecorvo-Maki-Nakagawa-Sakata) mit den Flavour-Zuständen (\(\nu_e, \nu_\mu, \nu_\tau\)) verbunden sind:
		
```math-align

			\begin{pmatrix}
				\nu_e \\ \nu_\mu \\ \nu_\tau
			\end{pmatrix}
			=
			U_{\text{PMNS}}
			\begin{pmatrix}
				\nu_1 \\ \nu_2 \\ \nu_3
			\end{pmatrix},
		
```

		wobei \(U_{\text{PMNS}}\) die Mischungsmatrix ist.
		
		Die Oszillationen hängen von den Massendifferenzen \(\Delta m^2_{ij} = m_i^2 - m_j^2\) und den Mischungswinkeln ab. Aktuelle experimentelle Daten (2025) liefern:
		
```math-align

			\Delta m^2_{21} &\approx 7.53 \times 10^{-5} \text{ eV}^2 \quad \text{[Solar]} \\
			\Delta m^2_{32} &\approx 2.44 \times 10^{-3} \text{ eV}^2 \quad \text{[Atmosphärisch]} \\
			m_\nu &> 0.06 \text{ eV} \quad \text{[Mindestens ein Neutrino, 3}\sigma\text{]}
		
```

		
		\textbf{Implikationen für T0:}
		
			- Die T0-Theorie postuliert gleiche Massen für die Flavour-Zustände (\(\nu_e, \nu_\mu, \nu_\tau\)), was \(\Delta m^2_{ij} = 0\) impliziert und mit Standard-Oszillationen inkompatibel ist.
			- Um Oszillationen zu erklären, verwendet die T0-Theorie geometrische Phasen basierend auf \( T_x \cdot m_x = 1 \), wobei die Quantenzahlen (\(n, \ell, j\)) die Phasenunterschiede bestimmen.
		
	\end{important}
	
	## Geometrische Phasen als Oszillationsmechanismus
	
	\begin{speculation}
		\textbf{T0-Hypothese: Geometrische Phasen für Oszillationen}
		
		Um die Hypothese gleicher Massen (\(m_{\nu_e} = m_{\nu_\mu} = m_{\nu_\tau} = m_\nu\)) mit Neutrino-Oszillationen zu vereinbaren, wird spekuliert, dass Oszillationen in der T0-Theorie durch geometrische Phasen statt durch Massendifferenzen verursacht werden. Dies basiert auf der T0-Beziehung:
		\[
		T_x \cdot m_x = 1,
		\]
		wobei \(m_x = m_\nu = 4.54 \text{ meV}\) die Neutrino-Masse ist und \(T_x\) eine charakteristische Zeit oder Frequenz:
		\[
		T_x = \frac{1}{m_\nu} = \frac{1}{4.54 \times 10^{-3} \text{ eV}} \approx 2.2026 \times 10^2 \text{ eV}^{-1} \approx 1.449 \times 10^{-13} \text{ s}.
		\]
		
		Die geometrische Phase wird durch die T0-Quantenzahlen (\(n, \ell, j\)) bestimmt:
		\[
		\phi_{\text{geo}, i} \propto f(n, \ell, j) \cdot \frac{L}{E} \cdot \frac{1}{T_x},
		\]
		wobei \(f(n, \ell, j) = \frac{n^6}{\ell^3}\) (oder 1 für \(\ell = 0\)) die geometrischen Faktoren sind:
		
```math-align

			f_{\nu_e} &= 1, \\
			f_{\nu_\mu} &= 64, \\
			f_{\nu_\tau} &= 91.125.
		
```

		
		\textbf{Berechnete Phasenunterschiede:}
		
```math-align

			\phi_{\nu_e} &\propto 1 \cdot \frac{L}{E} \cdot \frac{1}{T_x}, \\
			\phi_{\nu_\mu} &\propto 64 \cdot \frac{L}{E} \cdot \frac{1}{T_x}, \\
			\phi_{\nu_\tau} &\propto 91.125 \cdot \frac{L}{E} \cdot \frac{1}{T_x}.
		
```

		
		Diese Phasenunterschiede könnten Oszillationen zwischen Flavour-Zuständen verursachen, ohne dass unterschiedliche Massen erforderlich sind. Die genaue Form der Oszillationswahrscheinlichkeit müsste weiter entwickelt werden, bleibt aber hochspekulativ.
		
		\textbf{WARNUNG:} Dieser Ansatz ist rein hypothetisch und ohne empirische Bestätigung. Er widerspricht der etablierten Theorie, dass Oszillationen durch \(\Delta m^2_{ij} \neq 0\) verursacht werden.
	\end{speculation}
	
	# Fundamentale Konstanten und Einheiten
	
	## Basis-Parameter
	
	\begin{formula}
		\textbf{T0-Grundkonstanten:}
		
```math-align

			\xipar &= \frac{4}{3} \times 10^{-4} \approx 1.333333 \times 10^{-4} \quad \text{[dimensionslos]} \\
			\frac{\xipar^2}{2} &= \frac{\left(\frac{4}{3} \times 10^{-4}\right)^2}{2} \approx 8.888888 \times 10^{-9} \quad \text{[dimensionslos]} \\
			v &= 246.22 \text{ GeV} \quad \text{[Higgs VEV]} \\
			\hbar c &= 0.19733 \text{ GeV·fm} \quad \text{[Umrechnungskonstante]} \\
			T_x &= \frac{1}{4.54 \times 10^{-3} \text{ eV}} \approx 2.2026 \times 10^2 \text{ eV}^{-1} \approx 1.449 \times 10^{-13} \text{ s} \quad \text{[T0-Masse]}
		
```

	\end{formula}
	
	## Einheiten-Konventionen
	
	\begin{important}
		\textbf{Konsistente Einheiten-Hierarchie:}
		
```math-align

			\text{Standard:} &\quad \text{GeV} \\
			\text{Submultiples:} &\quad 1 \text{ eV} = 10^{-9} \text{ GeV} \\
			&\quad 1 \text{ meV} = 10^{-12} \text{ GeV} = 10^{-3} \text{ eV} \\
			\text{Massen:} &\quad m[\text{GeV}/c^2] = E[\text{GeV}]/c^2 \approx E[\text{GeV}] \text{ (natürliche Einheiten)} \\
			\text{Zeit:} &\quad 1 \text{ eV}^{-1} \approx 6.582 \times 10^{-16} \text{ s}
		
```

	\end{important}
	
	# Geladene Lepton-Referenzmassen
	
	## Präzise experimentelle Werte (PDG 2024)
	
	\begin{experimental}
		\textbf{Verifizierte Teilchenmassen:}
		
```math-align

			m_e &= 0.51099895000 \times 10^{-3} \text{ GeV} = 510.99895 \text{ keV} \\
			m_\mu &= 105.6583745 \times 10^{-3} \text{ GeV} = 105.6583745 \text{ MeV} \\
			m_\tau &= 1776.86 \times 10^{-3} \text{ GeV} = 1.77686 \text{ GeV}
		
```

		
		\textbf{Einheiten-Umrechnung zu eV:}
		
```math-align

			m_e &= 510998.95 \text{ eV} = 510998950 \text{ meV} \\
			m_\mu &= 105658374.5 \text{ eV} \\
			m_\tau &= 1776860000 \text{ eV}
		
```

	\end{experimental}
	
	# Neutrino-Quantenzahlen (T0-Hypothese)
	
	## Postulierte Quantenzahl-Zuordnung
	
	\begin{speculation}
		\textbf{Hypothetische Neutrino-Quantenzahlen:}
		
```math-align

			\nu_e: &\quad n=1, \ell=0, j=1/2 \quad \text{[Grundzustand-Neutrino]} \\
			\nu_\mu: &\quad n=2, \ell=1, j=1/2 \quad \text{[Erste Anregung]} \\
			\nu_\tau: &\quad n=3, \ell=2, j=1/2 \quad \text{[Zweite Anregung]}
		
```

		
		\textbf{Rolle der Quantenzahlen:}
		Die Quantenzahlen beeinflussen nicht die Neutrino-Massen (da \(m_{\nu_e} = m_{\nu_\mu} = m_{\nu_\tau}\)), sondern bestimmen die geometrischen Faktoren \(f(n, \ell, j)\), die die Oszillationsphasen steuern.
		
		\textbf{WARNUNG:} Diese Zuordnungen sind reine Spekulationen ohne experimentelle Basis.
	\end{speculation}
	
	## Geometrische Faktoren
	
	\begin{formula}
		\textbf{T0-Geometrische Faktoren:}
		
```math-align

			f(n,\ell,j) &= \frac{n^6}{\ell^3} \quad \text{für } \ell > 0 \\
			f(1,0,j) &= 1 \quad \text{für } \ell = 0 \text{ (Spezialfall)}
		
```

		
		\textbf{Berechnete Werte:}
		
```math-align

			f_{\nu_e} &= f(1,0,1/2) = 1 \\
			f_{\nu_\mu} &= f(2,1,1/2) = \frac{2^6}{1^3} = 64 \\
			f_{\nu_\tau} &= f(3,2,1/2) = \frac{3^6}{2^3} = \frac{729}{8} = 91.125
		
```

	\end{formula}
	
	# Neutrino-Masse-Formel
	
	## T0-Hypothese: Gleiche Massen mit Geometrischen Phasen
	
	\begin{speculation}
		\textbf{T0-Hypothese: Gleiche Neutrino-Massen mit Geometrischen Phasen}
		
		Die T0-Theorie postuliert, dass alle Flavour-Zustände (\(\nu_e, \nu_\mu, \nu_\tau\)) die gleiche Masse haben:
		\[
		m_{\nu_e} = m_{\nu_\mu} = m_{\nu_\tau} = m_\nu = 4.54 \text{ meV}.
		\]
		Die Masse wird aus der Photon-Analogie abgeleitet:
		\[
		m_\nu = \frac{\xipar^2}{2} \times m_e = \left(8.888888 \times 10^{-9}\right) \times (0.51099895 \times 10^{-3} \text{ GeV}) = 4.54 \text{ meV}.
		\]
		
		Um Oszillationen zu erklären, wird ein geometrischer Mechanismus postuliert, basierend auf der T0-Beziehung:
		\[
		T_x \cdot m_x = 1, \quad m_x = 4.54 \text{ meV}, \quad T_x \approx 2.2026 \times 10^2 \text{ eV}^{-1} \approx 1.449 \times 10^{-13} \text{ s}.
		\]
		
		Die Oszillationsphasen werden durch geometrische Faktoren \(f(n, \ell, j)\) bestimmt:
		\[
		\phi_{\text{geo}, i} \propto f_{\nu_i} \cdot \frac{L}{E} \cdot \frac{1}{T_x},
		\]
		wobei \(f_{\nu_e} = 1\), \(f_{\nu_\mu} = 64\), \(f_{\nu_\tau} = 91.125\).
		
		\textbf{Begründung:}
		
			- Die Masse \(4.54 \text{ meV}\) ist konsistent mit der kosmologischen Grenze (\(\Sigma m_\nu = 0.01362 \text{ eV} < 0.07 \text{ eV}\)).
			- Geometrische Phasen ermöglichen Oszillationen ohne Massendifferenzen, was die Hypothese gleicher Massen unterstützt.
			- Diese Hypothese ist hochspekulativ und ohne empirische Bestätigung.
		
	\end{speculation}
	
	\begin{formula}
		\textbf{Formel:} \(m_{\nu_i} = 4.54 \text{ meV}\)
		
		\textbf{Gesamtmasse:}
		\[
		\Sigma m_\nu = 3 \times 4.54 \text{ meV} = 13.62 \text{ meV} = 0.01362 \text{ eV}
		\]
		
		\textbf{Vergleich mit plausiblen Zielwert:}
		
			- \(\nu_e, \nu_\mu, \nu_\tau\): \(4.54 \text{ meV}\) vs. \(15 \text{ meV}\) (Übereinstimmung: \(30.3\%\))
			- \(\Sigma m_\nu\): \(13.62 \text{ meV}\) vs. \(45 \text{ meV}\) (Abweichung: Faktor \(\approx 3.30\))
		
	\end{formula}
	
	\begin{warning}
		\textbf{KRITISCHER BEFUND:} Die Hypothese gleicher Massen mit geometrischen Phasen ist inkompatibel mit den experimentellen Oszillationsdaten (\(\Delta m^2_{21} \approx 7.53 \times 10^{-5} \text{ eV}^2\), \(\Delta m^2_{32} \approx 2.44 \times 10^{-3} \text{ eV}^2\)), da sie \(\Delta m^2_{ij} = 0\) impliziert. Der geometrische Ansatz ist rein spekulativ und erfordert weitere theoretische und experimentelle Validierung.
	\end{warning}
	
	# Plausibler Zielwert basierend auf empirischen Daten
	
	## Ableitung aus Messdaten
	
	\begin{experimental}
		\textbf{Plausibler Zielwert:}
		Die T0-Theorie postuliert gleiche Massen für alle Flavour-Zustände (\(\nu_e, \nu_\mu, \nu_\tau\)). Daher wird ein einziger Zielwert für die Neutrino-Masse \(m_\nu\) abgeleitet, basierend auf empirischen Daten (Stand 2025):
		
			- Kosmologische Grenze: \(\Sigma m_\nu = 3 m_\nu < 0.07 \text{ eV} \implies m_\nu < 23.33 \text{ meV}\).
			- Oszillationsdaten: \(\Delta m^2_{21} \approx 7.53 \times 10^{-5} \text{ eV}^2\), \(\Delta m^2_{32} \approx 2.44 \times 10^{-3} \text{ eV}^2\), was normalerweise unterschiedliche Massen erfordert. Die T0-Theorie umgeht dies durch geometrische Phasen.
			- Plausibler Zielwert: \(m_\nu \approx 15 \text{ meV}\), was zwischen der solaren (\(8.68 \text{ meV}\)) und atmosphärischen Skala (\(50.15 \text{ meV}\)) liegt und die kosmologische Grenze erfüllt:
			\[
			\Sigma m_\nu = 3 \times 15 \text{ meV} = 45 \text{ meV} = 0.045 \text{ eV} < 0.07 \text{ eV}.
			\]
		
		
		\textbf{Begründung:}
		
			- Der Zielwert ist konsistent mit der kosmologischen Grenze und liegt in der Größenordnung der Oszillationsdaten.
			- Die Hypothese gleicher Massen wird durch geometrische Phasen unterstützt, was die T0-Theorie von der Standardphysik abgrenzt.
			- Der Wert ist plausibel, aber nicht direkt gemessen, da Flavour-Massen Mischungen der Eigenzustände sind.
			- Die T0-Masse (\(4.54 \text{ meV}\)) liegt unter dem Zielwert (\(30.3\%\)), ist aber ebenfalls kosmologisch konsistent.
		
	\end{experimental}
	
	# Experimentelle Vergleichsgrößen
	
	## Aktuelle experimentelle Obergrenzen (2025)
	
	\begin{experimental}
		\textbf{Experimentelle Grenzen:}
		
```math-align

			m_{\nu_e} &< 0.45 \text{ eV} \quad \text{[KATRIN, 90\% CL]} \\
			m_{\nu_\mu} &< 0.17 \text{ MeV} \quad \text{[Myon-Zerfall, indirekt]} \\
			m_{\nu_\tau} &< 18.2 \text{ MeV} \quad \text{[Tau-Zerfall, indirekt]} \\
			\Sigma m_\nu &< 0.07 \text{ eV} \quad \text{[DESI+Planck, 95\% CL]} \\
			\Delta m^2_{21} &\approx 7.53 \times 10^{-5} \text{ eV}^2 \quad \text{[Solar]} \\
			\Delta m^2_{32} &\approx 2.44 \times 10^{-3} \text{ eV}^2 \quad \text{[Atmosphärisch]} \\
			m_\nu &> 0.06 \text{ eV} \quad \text{[Mindestens ein Neutrino, 3}\sigma\text{]}
		
```

	\end{experimental}
	
	## Sicherheitsmargen für T0-Hypothese
	
	\begin{longtable}[c]{@{}lcc@{}}
		\caption{Sicherheitsmargen der T0-Hypothese zu experimentellen Grenzen} \\
		\toprule
		\textbf{Parameter} & \textbf{T0-Masse (\(4.54 \text{ meV}\))} & \textbf{Zielwert (\(15 \text{ meV}\))} \\
		\midrule
		\endfirsthead
		\toprule
		\textbf{Parameter} & \textbf{T0-Masse (\(4.54 \text{ meV}\))} & \textbf{Zielwert (\(15 \text{ meV}\))} \\
		\midrule
		\endhead
		$m_{\nu_e}$ vs 0.45 eV & 99200× & 30× \\
		$m_{\nu_\mu}$ vs 0.17 MeV & 3.74E7× & 11333× \\
		$m_{\nu_\tau}$ vs 18.2 MeV & 4.01E9× & 1.21E6× \\
		\midrule
		$\Sigma m_\nu$ vs 0.07 eV & 5.14× & 1.56× \\
		$\Sigma m_\nu$ vs 0.06 eV & 4.41× & 1.33× \\
		\bottomrule
	\end{longtable}
	
	\begin{important}
		\textbf{T0-Hypothese:}
		
			- Die T0-Masse (\(4.54 \text{ meV}\)) ist kompatibel mit kosmologischen Grenzen (\(\Sigma m_\nu = 0.01362 \text{ eV} < 0.07 \text{ eV}\)) und liegt unter dem Zielwert (\(15 \text{ meV}\), \(30.3\%\)).
			- Geometrische Phasen (\(T_x \cdot m_x = 1\)) bieten einen spekulativen Mechanismus für Oszillationen, sind aber inkompatibel mit Standard-Oszillationen.
			- Physikalische Begründung: Die Masse basiert auf der \(\frac{\xipar^2}{2}\)-Suppression, konsistent mit der Geschwindigkeitsdifferenz \(v_\nu = c \times \left(1 - \frac{\xipar^2}{2}\right)\).
		
	\end{important}
	
	# Konsistenz-Checks und Validierung
	
	## Dimensionale Analyse
	
	\begin{formula}
		\textbf{Dimensionale Konsistenz:}
		
```math-align

			[\xipar] &= 1 \quad \checkmark \text{ dimensionslos} \\
			[m_e] &= \text{GeV} \quad \checkmark \text{ Energie/Masse} \\
			\left[\frac{\xipar^2}{2} \times m_e\right] &= \text{GeV} \quad \checkmark \text{ Energie/Masse} \\
			[f_{\nu_i}] &= 1 \quad \checkmark \text{ dimensionslos} \\
			[m_\nu] &= \text{eV} \quad \checkmark \text{ (festgelegte Masse)} \\
			[T_x] &= \text{eV}^{-1} \quad \checkmark \text{ (Zeit)}
		
```

		Alle Formeln sind dimensional konsistent.
	\end{formula}
	
	## Mathematische Konsistenz
	
	\begin{important}
		\textbf{Konsistenz der Hypothese:}
		
			- Die Formel \(m_\nu = \frac{\xipar^2}{2} \times m_e = 4.54 \text{ meV}\) ist physikalisch begründet durch die Photon-Analogie und konsistent mit der Geschwindigkeitsdifferenz.
			- Geometrische Phasen basierend auf \(f(n, \ell, j)\) und \(T_x \cdot m_x = 1\) bieten einen spekulativen Mechanismus für Oszillationen.
			- Keine freien Parameter außer \(\xipar\), was die Theorie vereinfacht.
		
	\end{important}
	
	## Experimentelle Validierung
	
	\begin{experimental}
		\textbf{Validierungsstatus (Stand 2025):}
		
			- Die T0-Masse (\(4.54 \text{ meV}\)) erfüllt kosmologische Grenzen (\(\Sigma m_\nu = 0.01362 \text{ eV} < 0.07 \text{ eV}\)) und liegt unter dem Zielwert (\(15 \text{ meV}\), \(30.3\%\)).
			- Inkompatibel mit Standard-Oszillationen (\(\Delta m^2_{ij} = 0\)), aber geometrische Phasen bieten einen spekulativen Ausweg.
			- Der Zielwert (\(15 \text{ meV}\)) ist konsistent mit kosmologischen Grenzen, aber nicht direkt gemessen.
		
	\end{experimental}
	
	# Fazit
	
	\begin{important}
		\textbf{Zusammenfassung und Ausblick:}
		
			- Die T0-Theorie postuliert gleiche Neutrino-Massen (\(m_\nu = 4.54 \text{ meV}\)) basierend auf der Photon-Analogie (\(\frac{\xipar^2}{2} \times m_e\)), konsistent mit der Geschwindigkeitsdifferenz (\(v_\nu = c \times \left(1 - \frac{\xipar^2}{2}\right)\)).
			- Geometrische Phasen basierend auf \(T_x \cdot m_x = 1\) und den Quantenzahlen (\(f_{\nu_e} = 1\), \(f_{\nu_\mu} = 64\), \(f_{\nu_\tau} = 91.125\)) erklären Oszillationen spekulative, ohne Massendifferenzen.
			- Der plausible Zielwert (\(m_\nu = 15 \text{ meV}\)) basiert auf empirischen Daten (kosmologische Grenze) und liegt in der Größenordnung der Oszillationsdaten, ist aber nicht direkt gemessen.
			- Die T0-Masse (\(4.54 \text{ meV}\)) ist relativ nahe am Zielwert (\(30.3\%\)), erfüllt kosmologische Grenzen, ist aber inkompatibel mit Standard-Oszillationen.
			- Die T0-Theorie bleibt spekulativ, da sie auf geometrischen Harmonien ohne empirische Basis basiert.
			- Zukünftige Experimente (2025–2030, z. B. KATRIN-Upgrade, DESI, Euclid) könnten die T0-Hypothese, insbesondere den geometrischen Oszillationsmechanismus, weiter prüfen oder widerlegen.
			- Die wissenschaftliche Integrität erfordert, die spekulative Natur der T0-Theorie klar zu kommunizieren und weitere Tests abzuwarten.
		
	\end{important}

\end{document}
