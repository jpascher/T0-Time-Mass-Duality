\documentclass[11pt,a4paper,openany]{book}

% Essential packages
\usepackage[utf8]{inputenc}
\usepackage[T1]{fontenc}
\usepackage[english]{babel}
\usepackage[a4paper,margin=2.5cm]{geometry}
\usepackage{lmodern}

% Math and physics packages
\usepackage{amsmath}
\usepackage{amssymb}
\usepackage{amsthm}
\usepackage{mathtools}
\usepackage{physics}
\usepackage{siunitx}

% Graphics and tables
\usepackage{graphicx}
\usepackage[table,xcdraw]{xcolor}
\usepackage{tikz}
\usepackage{pgfplots}
\usepackage{tcolorbox}
\usepackage{booktabs}
\usepackage{array}
\usepackage{longtable}
\usepackage{float}

% Document formatting
\usepackage{fancyhdr}
\usepackage{tocloft}
\usepackage{hyperref}
\usepackage{cleveref}
\usepackage{microtype}
\usepackage{enumitem}
\usepackage{newunicodechar}

% Additional packages
\usepackage{adjustbox}
\usepackage{algorithm}
\usepackage{algorithmic}
\usepackage{amsfonts}
\usepackage{amsmath,amsfonts,amssymb}
\usepackage{amsmath,amsfonts,amssymb,physics}
\usepackage{amsmath,amssymb}
\usepackage{amsmath,amssymb,amsfonts,amsthm}
\usepackage{amsmath,amssymb,amsthm}
\usepackage{amsmath,amssymb,physics,graphicx,xcolor,amsthm}
\usepackage{bm}
\usepackage{booktabs,array,longtable,multirow}
\usepackage{braket}
\usepackage{breakurl}
\usepackage{cancel}
\usepackage{caption}
\usepackage{cite}
\usepackage{color}
\usepackage{colortbl}
\usepackage{csquotes}
\usepackage{doi}
\usepackage{forest}
\usepackage{gensymb}
\usepackage{geometry,fancyhdr}
\usepackage{graphicx,tikz,pgfplots}
\usepackage{hyperref,url}
\usepackage{hyphenat}
\usepackage{listings}
\usepackage{listings,enumerate}
\usepackage{mdframed}
\usepackage{multicol}
\usepackage{multirow}
\usepackage{natbib}
\usepackage{pdflscape}
\usepackage{ragged2e}
\usepackage{setspace}
\usepackage{siunitx,xcolor,graphicx}
\usepackage{slashed}
\usepackage{tabularx}
\usepackage{textcomp}
\usepackage{textgreek}
\usepackage{tikz,pgfplots}
\usepackage{upgreek}
\usepackage{url}

% Custom commands and definitions
\definecolor{blue}
\definecolor{blue}{rgb}{0,0,1}
\definecolor{boxgray}
\definecolor{boxgray}{RGB}{240,240,240}
\definecolor{deepblue}
\definecolor{deepblue}{RGB}{0,0,127}
\definecolor{deepgreen}
\definecolor{deepgreen}{RGB}{0,127,0}
\definecolor{deepred}
\definecolor{deepred}{RGB}{191,0,0}
\definecolor{t0blue}
\definecolor{t0blue}{RGB}{0,102,204}
\definecolor{t0blue}{RGB}{33,150,243}
\definecolor{t0green}
\definecolor{t0green}{RGB}{0,153,0}
\definecolor{t0green}{RGB}{0,153,76}
\definecolor{t0green}{RGB}{76,175,80}
\definecolor{t0orange}
\definecolor{t0orange}{RGB}{255,152,0}
\definecolor{t0purple}
\definecolor{t0purple}{RGB}{102,0,204}
\definecolor{t0purple}{RGB}{156,39,176}
\definecolor{t0red}
\definecolor{t0red}{RGB}{204,0,0}
\definecolor{t0red}{RGB}{204,0,51}
\definecolor{t0red}{RGB}{244,67,54}
\definecolor{t0yellow}
\definecolor{t0yellow}{RGB}{255,204,0}
\geometry{a4paper, left=25mm, right=25mm, top=25mm, bottom=25mm}
\geometry{a4paper, margin=1in}
\geometry{a4paper, margin=2.5cm}
\geometry{a4paper, margin=2cm}
\geometry{left=2.5cm,right=2.5cm,top=2.5cm,bottom=2.5cm}
\geometry{left=2cm,right=2cm,top=2cm,bottom=2cm}
\geometry{margin=1in}
\geometry{margin=2.5cm}
\geometry{margin=2cm}
\hypersetup{
	colorlinks=true,
	linkcolor=blue,
	citecolor=blue,
	urlcolor=blue,
	pdftitle={Analysis and Implications of MNRAS Paper 544 for the T0-Theory}
\hypersetup{
	colorlinks=true,
	linkcolor=blue,
	citecolor=blue,
	urlcolor=blue,
	pdftitle={Beweis: Die Feinstrukturkonstante α = 1 in natürlichen Einheiten}
\hypersetup{
	colorlinks=true,
	linkcolor=blue,
	citecolor=blue,
	urlcolor=blue,
	pdftitle={Beweis: Die Koide-Formel enthält implizit $\xi$}
\hypersetup{
	colorlinks=true,
	linkcolor=blue,
	citecolor=blue,
	urlcolor=blue,
	pdftitle={Chinas Photonischer Quantenchip: 1000x-Speedup und T0-Integration}
\hypersetup{
	colorlinks=true,
	linkcolor=blue,
	citecolor=blue,
	urlcolor=blue,
	pdftitle={Complete Derivation of Higgs Mass and Wilson Coefficients}
\hypersetup{
	colorlinks=true,
	linkcolor=blue,
	citecolor=blue,
	urlcolor=blue,
	pdftitle={Complete Particle Spectrum: Standard Model vs T0 Theory}
\hypersetup{
	colorlinks=true,
	linkcolor=blue,
	citecolor=blue,
	urlcolor=blue,
	pdftitle={Conceptual Comparison of Unified Natural Units and Extended Standard Model}
\hypersetup{
	colorlinks=true,
	linkcolor=blue,
	citecolor=blue,
	urlcolor=blue,
	pdftitle={Connections between the Mizohata-Takeuchi Counterexample and the T0 Time-Mass Duality Theory}
\hypersetup{
	colorlinks=true,
	linkcolor=blue,
	citecolor=blue,
	urlcolor=blue,
	pdftitle={Das Relationale Zahlensystem: Primzahlen als fundamentale Verhältnisse}
\hypersetup{
	colorlinks=true,
	linkcolor=blue,
	citecolor=blue,
	urlcolor=blue,
	pdftitle={Das T0-Modell (Planck-Referenziert): Eine Neuformulierung der Physik}
\hypersetup{
	colorlinks=true,
	linkcolor=blue,
	citecolor=blue,
	urlcolor=blue,
	pdftitle={Das T0-Modell: Zeit-Energie-Dualität und geometrische Ruhemasse}
\hypersetup{
	colorlinks=true,
	linkcolor=blue,
	citecolor=blue,
	urlcolor=blue,
	pdftitle={Der Massenskalierungsexponent κ in der T0-Theorie}
\hypersetup{
	colorlinks=true,
	linkcolor=blue,
	citecolor=blue,
	urlcolor=blue,
	pdftitle={Der geometrische Formalismus der T0-Quantenmechanik und seine Anwendung auf Quantencomputer}
\hypersetup{
	colorlinks=true,
	linkcolor=blue,
	citecolor=blue,
	urlcolor=blue,
	pdftitle={Der xi Parameter und Teilchendifferenzierung in der T0-Theorie}
\hypersetup{
	colorlinks=true,
	linkcolor=blue,
	citecolor=blue,
	urlcolor=blue,
	pdftitle={Deterministic Quantum Mechanics via T0-Energy Field Formulation}
\hypersetup{
	colorlinks=true,
	linkcolor=blue,
	citecolor=blue,
	urlcolor=blue,
	pdftitle={Deterministische Quantenmechanik via T0-Energiefeld-Formulierung}
\hypersetup{
	colorlinks=true,
	linkcolor=blue,
	citecolor=blue,
	urlcolor=blue,
	pdftitle={Die Elektroneneinheitsladung in der T0-Theorie: Jenseits von Punkt-Singularitäten}
\hypersetup{
	colorlinks=true,
	linkcolor=blue,
	citecolor=blue,
	urlcolor=blue,
	pdftitle={Die Feinstrukturkonstante: Verschiedene Darstellungen und Beziehungen}
\hypersetup{
	colorlinks=true,
	linkcolor=blue,
	citecolor=blue,
	urlcolor=blue,
	pdftitle={Die Musikalische Spirale und die 137: Die mathematische Entdeckung der kosmischen Verstimmung}
\hypersetup{
	colorlinks=true,
	linkcolor=blue,
	citecolor=blue,
	urlcolor=blue,
	pdftitle={E=mc² = E=m: Die Konstanten-Illusion entlarvt}
\hypersetup{
	colorlinks=true,
	linkcolor=blue,
	citecolor=blue,
	urlcolor=blue,
	pdftitle={E=mc² = E=m: The Constants Illusion Exposed}
\hypersetup{
	colorlinks=true,
	linkcolor=blue,
	citecolor=blue,
	urlcolor=blue,
	pdftitle={Einfache Lagrange-Revolution: Von der Standardmodell-Komplexität zur T0-Eleganz}
\hypersetup{
	colorlinks=true,
	linkcolor=blue,
	citecolor=blue,
	urlcolor=blue,
	pdftitle={Einführung in die Umsetzung photonischer Bauteile auf Wafern für Nachrichtentechniker}
\hypersetup{
	colorlinks=true,
	linkcolor=blue,
	citecolor=blue,
	urlcolor=blue,
	pdftitle={Einführung in photonische Quantenchips für Nachrichtentechniker}
\hypersetup{
	colorlinks=true,
	linkcolor=blue,
	citecolor=blue,
	urlcolor=blue,
	pdftitle={Elimination der Masse als dimensionaler Platzhalter im T0-Modell}
\hypersetup{
	colorlinks=true,
	linkcolor=blue,
	citecolor=blue,
	urlcolor=blue,
	pdftitle={Elimination of Mass as Dimensional Placeholder in the T0 Model}
\hypersetup{
	colorlinks=true,
	linkcolor=blue,
	citecolor=blue,
	urlcolor=blue,
	pdftitle={Empirical Analysis of Deterministic Factorization Methods}
\hypersetup{
	colorlinks=true,
	linkcolor=blue,
	citecolor=blue,
	urlcolor=blue,
	pdftitle={Empirische Analyse deterministischer Faktorisierungsmethoden}
\hypersetup{
	colorlinks=true,
	linkcolor=blue,
	citecolor=blue,
	urlcolor=blue,
	pdftitle={Integration der Dirac-Gleichung im T0-Modell: Natürliche-Einheiten-Rahmenwerk}
\hypersetup{
	colorlinks=true,
	linkcolor=blue,
	citecolor=blue,
	urlcolor=blue,
	pdftitle={Integration of the Dirac Equation in the T0 Model: Natural Units Framework}
\hypersetup{
	colorlinks=true,
	linkcolor=blue,
	citecolor=blue,
	urlcolor=blue,
	pdftitle={Introduction to Photonic Quantum Chips for Communication Engineers}
\hypersetup{
	colorlinks=true,
	linkcolor=blue,
	citecolor=blue,
	urlcolor=blue,
	pdftitle={Introduction to the Implementation of Photonic Components on Wafers for Communication Engineers}
\hypersetup{
	colorlinks=true,
	linkcolor=blue,
	citecolor=blue,
	urlcolor=blue,
	pdftitle={Konzeptioneller Vergleich von Einheitlichen Natürlichen Einheiten und Erweitertem Standardmodell}
\hypersetup{
	colorlinks=true,
	linkcolor=blue,
	citecolor=blue,
	urlcolor=blue,
	pdftitle={Markov Chains in the Context of T0 Theory: Deterministic or Stochastic? A Treatise on Patterns, Preconditions, and Uncertainty}
\hypersetup{
	colorlinks=true,
	linkcolor=blue,
	citecolor=blue,
	urlcolor=blue,
	pdftitle={Markov-Ketten im Kontext der T0-Theorie: Deterministisch oder stochastisch? Ein Traktat zu Mustern, Voraussetzungen und Unsicherheit}
\hypersetup{
	colorlinks=true,
	linkcolor=blue,
	citecolor=blue,
	urlcolor=blue,
	pdftitle={Mathematical Analysis of T0-Shor Algorithm: Theoretical Framework and Computational Complexity}
\hypersetup{
	colorlinks=true,
	linkcolor=blue,
	citecolor=blue,
	urlcolor=blue,
	pdftitle={Mathematical Constructs of Alternative CMB Models: Unnikrishnan and Peratt in Harmony with the T0 Theory}
\hypersetup{
	colorlinks=true,
	linkcolor=blue,
	citecolor=blue,
	urlcolor=blue,
	pdftitle={Mathematische Analyse des T0-Shor Algorithmus: Theoretischer Rahmen und Berechnungskomplexität}
\hypersetup{
	colorlinks=true,
	linkcolor=blue,
	citecolor=blue,
	urlcolor=blue,
	pdftitle={Mathematische Konstrukte alternativer CMB-Modelle: Unnikrishnan und Peratt im Einklang mit der T0-Theorie}
\hypersetup{
	colorlinks=true,
	linkcolor=blue,
	citecolor=blue,
	urlcolor=blue,
	pdftitle={Natural Unit Systems: Universal Energy Conversion and Fundamental Length Scale Hierarchy}
\hypersetup{
	colorlinks=true,
	linkcolor=blue,
	citecolor=blue,
	urlcolor=blue,
	pdftitle={Natural Units in Theoretical Physics: A Treatise in the Context of T0 Theory}
\hypersetup{
	colorlinks=true,
	linkcolor=blue,
	citecolor=blue,
	urlcolor=blue,
	pdftitle={Natürliche Einheiten in der theoretischen Physik: Eine Abhandlung im Kontext der T0-Theorie}
\hypersetup{
	colorlinks=true,
	linkcolor=blue,
	citecolor=blue,
	urlcolor=blue,
	pdftitle={Natürliche Einheitensysteme: Universelle Energieumwandlung und fundamentale Längenskala-Hierarchie}
\hypersetup{
	colorlinks=true,
	linkcolor=blue,
	citecolor=blue,
	urlcolor=blue,
	pdftitle={Parameter System-Dependency in T0-Model: SI vs. Natural Units}
\hypersetup{
	colorlinks=true,
	linkcolor=blue,
	citecolor=blue,
	urlcolor=blue,
	pdftitle={Parameter-Systemabhängigkeit im T0-Modell: SI- vs. natürliche Einheiten}
\hypersetup{
	colorlinks=true,
	linkcolor=blue,
	citecolor=blue,
	urlcolor=blue,
	pdftitle={Proof: The Fine Structure Constant α = 1 in Natural Units}
\hypersetup{
	colorlinks=true,
	linkcolor=blue,
	citecolor=blue,
	urlcolor=blue,
	pdftitle={Proof: The Koide Formula Implicitly Contains $\xi$}
\hypersetup{
	colorlinks=true,
	linkcolor=blue,
	citecolor=blue,
	urlcolor=blue,
	pdftitle={Pure Energy T0 Theory: Ratio-Based Physics with SI Reference}
\hypersetup{
	colorlinks=true,
	linkcolor=blue,
	citecolor=blue,
	urlcolor=blue,
	pdftitle={Quantum Mechanics in the T0 Model: Field-Theoretic Foundations}
\hypersetup{
	colorlinks=true,
	linkcolor=blue,
	citecolor=blue,
	urlcolor=blue,
	pdftitle={Ratio-Based vs. Absolute: The Role of Fractal Correction in T0 Theory}
\hypersetup{
	colorlinks=true,
	linkcolor=blue,
	citecolor=blue,
	urlcolor=blue,
	pdftitle={Reine Energie T0-Theorie: Verhältnis-basierte Physik mit SI-Referenz}
\hypersetup{
	colorlinks=true,
	linkcolor=blue,
	citecolor=blue,
	urlcolor=blue,
	pdftitle={Simple Lagrangian Revolution: From Standard Model Complexity to T0 Elegance}
\hypersetup{
	colorlinks=true,
	linkcolor=blue,
	citecolor=blue,
	urlcolor=blue,
	pdftitle={Simplified Dirac Equation in T0 Theory: Field Node Approach}
\hypersetup{
	colorlinks=true,
	linkcolor=blue,
	citecolor=blue,
	urlcolor=blue,
	pdftitle={Simplified T0 Theory: Elegant Lagrangian Density for Time-Mass Duality}
\hypersetup{
	colorlinks=true,
	linkcolor=blue,
	citecolor=blue,
	urlcolor=blue,
	pdftitle={T0 Cosmology: Redshift as a Geometric Path Effect in a Static Universe}
\hypersetup{
	colorlinks=true,
	linkcolor=blue,
	citecolor=blue,
	urlcolor=blue,
	pdftitle={T0 Deterministic Quantum Computing: Complete Analysis of Important Algorithms}
\hypersetup{
	colorlinks=true,
	linkcolor=blue,
	citecolor=blue,
	urlcolor=blue,
	pdftitle={T0 Deterministisches Quantencomputing: Vollständige Analyse wichtiger Algorithmen}
\hypersetup{
	colorlinks=true,
	linkcolor=blue,
	citecolor=blue,
	urlcolor=blue,
	pdftitle={T0 Model: Complete Framework - From Time-Energy Duality to Universal Constants}
\hypersetup{
	colorlinks=true,
	linkcolor=blue,
	citecolor=blue,
	urlcolor=blue,
	pdftitle={T0 Model: Complete Parameter-Free Particle Mass Calculation}
\hypersetup{
	colorlinks=true,
	linkcolor=blue,
	citecolor=blue,
	urlcolor=blue,
	pdftitle={T0 Model: Unified Neutrino Formula Structure}
\hypersetup{
	colorlinks=true,
	linkcolor=blue,
	citecolor=blue,
	urlcolor=blue,
	pdftitle={T0 Model: Universal Energy Relations for Mol and Candela Units}
\hypersetup{
	colorlinks=true,
	linkcolor=blue,
	citecolor=blue,
	urlcolor=blue,
	pdftitle={T0 Modell: Vollständiges Framework - Von Zeit-Energie-Dualität zu universellen Konstanten}
\hypersetup{
	colorlinks=true,
	linkcolor=blue,
	citecolor=blue,
	urlcolor=blue,
	pdftitle={T0 Quantenfeldtheorie: QFT, QM und Quantencomputer}
\hypersetup{
	colorlinks=true,
	linkcolor=blue,
	citecolor=blue,
	urlcolor=blue,
	pdftitle={T0 Quantum Field Theory: QFT, QM and Quantum Computers}
\hypersetup{
	colorlinks=true,
	linkcolor=blue,
	citecolor=blue,
	urlcolor=blue,
	pdftitle={T0 Theory vs Bell's Theorem: How Deterministic Energy Fields Circumvent No-Go Theorems}
\hypersetup{
	colorlinks=true,
	linkcolor=blue,
	citecolor=blue,
	urlcolor=blue,
	pdftitle={T0 Theory: Final Extension to Hadrons - Physically Derived Corrections}
\hypersetup{
	colorlinks=true,
	linkcolor=blue,
	citecolor=blue,
	urlcolor=blue,
	pdftitle={T0 Theory: The Fine-Structure Constant}
\hypersetup{
	colorlinks=true,
	linkcolor=blue,
	citecolor=blue,
	urlcolor=blue,
	pdftitle={T0 Theory: The Gravitational Constant}
\hypersetup{
	colorlinks=true,
	linkcolor=blue,
	citecolor=blue,
	urlcolor=blue,
	pdftitle={T0-Kosmologie: Rotverschiebung als geometrischer Pfad-Effekt im statischen Universum}
\hypersetup{
	colorlinks=true,
	linkcolor=blue,
	citecolor=blue,
	urlcolor=blue,
	pdftitle={T0-Model: Complete Document Analysis and Structured Summary}
\hypersetup{
	colorlinks=true,
	linkcolor=blue,
	citecolor=blue,
	urlcolor=blue,
	pdftitle={T0-Model: Kinetic Energy of Electrons and Photons}
\hypersetup{
	colorlinks=true,
	linkcolor=blue,
	citecolor=blue,
	urlcolor=blue,
	pdftitle={T0-Model: The Hubble Parameter in Static Universe}
\hypersetup{
	colorlinks=true,
	linkcolor=blue,
	citecolor=blue,
	urlcolor=blue,
	pdftitle={T0-Modell-Verifikation: Skalen-Verhältnis-basierte Berechnungen}
\hypersetup{
	colorlinks=true,
	linkcolor=blue,
	citecolor=blue,
	urlcolor=blue,
	pdftitle={T0-Modell: Bewegungsenergie von Elektronen und Photonen}
\hypersetup{
	colorlinks=true,
	linkcolor=blue,
	citecolor=blue,
	urlcolor=blue,
	pdftitle={T0-Modell: Die Hubble-Konstante im statischen Universum}
\hypersetup{
	colorlinks=true,
	linkcolor=blue,
	citecolor=blue,
	urlcolor=blue,
	pdftitle={T0-Modell: Einheitliche Neutrino-Formel-Struktur}
\hypersetup{
	colorlinks=true,
	linkcolor=blue,
	citecolor=blue,
	urlcolor=blue,
	pdftitle={T0-Modell: Universelle Energiebeziehungen für Mol- und Candela-Einheiten}
\hypersetup{
	colorlinks=true,
	linkcolor=blue,
	citecolor=blue,
	urlcolor=blue,
	pdftitle={T0-Modell: Vollständige Dokumentenanalyse und strukturierte Zusammenfassung}
\hypersetup{
	colorlinks=true,
	linkcolor=blue,
	citecolor=blue,
	urlcolor=blue,
	pdftitle={T0-Modell: Vollständige parameterfreie Teilchenmassen-Berechnung}
\hypersetup{
	colorlinks=true,
	linkcolor=blue,
	citecolor=blue,
	urlcolor=blue,
	pdftitle={T0-QAT: $\xi$-Aware Quantization-Aware Training}
\hypersetup{
	colorlinks=true,
	linkcolor=blue,
	citecolor=blue,
	urlcolor=blue,
	pdftitle={T0-QFT ML Addendum: Machine Learning Derived Extensions}
\hypersetup{
	colorlinks=true,
	linkcolor=blue,
	citecolor=blue,
	urlcolor=blue,
	pdftitle={T0-QFT ML-Addendum: Maschinelle Lern-abgeleitete Erweiterungen}
\hypersetup{
	colorlinks=true,
	linkcolor=blue,
	citecolor=blue,
	urlcolor=blue,
	pdftitle={T0-Theorie vs Bells Theorem: Wie deterministische Energiefelder No-Go-Theoreme umgehen}
\hypersetup{
	colorlinks=true,
	linkcolor=blue,
	citecolor=blue,
	urlcolor=blue,
	pdftitle={T0-Theorie: Der Terrell-Penrose-Effekt und Massenvariation}
\hypersetup{
	colorlinks=true,
	linkcolor=blue,
	citecolor=blue,
	urlcolor=blue,
	pdftitle={T0-Theorie: Die Feinstrukturkonstante}
\hypersetup{
	colorlinks=true,
	linkcolor=blue,
	citecolor=blue,
	urlcolor=blue,
	pdftitle={T0-Theorie: Die Gravitationskonstante}
\hypersetup{
	colorlinks=true,
	linkcolor=blue,
	citecolor=blue,
	urlcolor=blue,
	pdftitle={T0-Theorie: Die T0-Zeit-Masse-Dualität}
\hypersetup{
	colorlinks=true,
	linkcolor=blue,
	citecolor=blue,
	urlcolor=blue,
	pdftitle={T0-Theorie: Die sieben Rätsel}
\hypersetup{
	colorlinks=true,
	linkcolor=blue,
	citecolor=blue,
	urlcolor=blue,
	pdftitle={T0-Theorie: Erweiterung auf Bell-Tests – ML-Simulationen (November 2025)}
\hypersetup{
	colorlinks=true,
	linkcolor=blue,
	citecolor=blue,
	urlcolor=blue,
	pdftitle={T0-Theorie: Finale Erweiterung auf Hadronen - Physikalisch abgeleitete Korrekturen}
\hypersetup{
	colorlinks=true,
	linkcolor=blue,
	citecolor=blue,
	urlcolor=blue,
	pdftitle={T0-Theorie: Finale Fraktale Massenformeln (November 2025)}
\hypersetup{
	colorlinks=true,
	linkcolor=blue,
	citecolor=blue,
	urlcolor=blue,
	pdftitle={T0-Theorie: Fraktaldimension aus Lepton-Massenverhältnis}
\hypersetup{
	colorlinks=true,
	linkcolor=blue,
	citecolor=blue,
	urlcolor=blue,
	pdftitle={T0-Theorie: Fundamentale Prinzipien}
\hypersetup{
	colorlinks=true,
	linkcolor=blue,
	citecolor=blue,
	urlcolor=blue,
	pdftitle={T0-Theorie: Herleitung der Gravitationskonstanten}
\hypersetup{
	colorlinks=true,
	linkcolor=blue,
	citecolor=blue,
	urlcolor=blue,
	pdftitle={T0-Theorie: Kosmische Beziehungen und universelle $\xi$-Konstante}
\hypersetup{
	colorlinks=true,
	linkcolor=blue,
	citecolor=blue,
	urlcolor=blue,
	pdftitle={T0-Theorie: Kosmologie}
\hypersetup{
	colorlinks=true,
	linkcolor=blue,
	citecolor=blue,
	urlcolor=blue,
	pdftitle={T0-Theorie: Netzwerkdarstellung und Dimensionsanalyse in der T0-Theorie}
\hypersetup{
	colorlinks=true,
	linkcolor=blue,
	citecolor=blue,
	urlcolor=blue,
	pdftitle={T0-Theorie: Teilchenmassen}
\hypersetup{
	colorlinks=true,
	linkcolor=blue,
	citecolor=blue,
	urlcolor=blue,
	pdftitle={T0-Theorie: Vollstaendiger Abschluss}
\hypersetup{
	colorlinks=true,
	linkcolor=blue,
	citecolor=blue,
	urlcolor=blue,
	pdftitle={T0-Theory: Complete Closure}
\hypersetup{
	colorlinks=true,
	linkcolor=blue,
	citecolor=blue,
	urlcolor=blue,
	pdftitle={T0-Theory: Complete Derivation of All Parameters Without Circularity}
\hypersetup{
	colorlinks=true,
	linkcolor=blue,
	citecolor=blue,
	urlcolor=blue,
	pdftitle={T0-Theory: Cosmic Relations and universal $\xi$-constant}
\hypersetup{
	colorlinks=true,
	linkcolor=blue,
	citecolor=blue,
	urlcolor=blue,
	pdftitle={T0-Theory: Cosmology}
\hypersetup{
	colorlinks=true,
	linkcolor=blue,
	citecolor=blue,
	urlcolor=blue,
	pdftitle={T0-Theory: Derivation of the Gravitational Constant}
\hypersetup{
	colorlinks=true,
	linkcolor=blue,
	citecolor=blue,
	urlcolor=blue,
	pdftitle={T0-Theory: Extension to Bell Tests – ML Simulations (November 2025)}
\hypersetup{
	colorlinks=true,
	linkcolor=blue,
	citecolor=blue,
	urlcolor=blue,
	pdftitle={T0-Theory: Final Fractal Mass Formulas (November 2025)}
\hypersetup{
	colorlinks=true,
	linkcolor=blue,
	citecolor=blue,
	urlcolor=blue,
	pdftitle={T0-Theory: Fractal Dimension from Lepton Mass Ratio}
\hypersetup{
	colorlinks=true,
	linkcolor=blue,
	citecolor=blue,
	urlcolor=blue,
	pdftitle={T0-Theory: Fundamental Principles}
\hypersetup{
	colorlinks=true,
	linkcolor=blue,
	citecolor=blue,
	urlcolor=blue,
	pdftitle={T0-Theory: Mass Variation as an Equivalent to Time Dilation}
\hypersetup{
	colorlinks=true,
	linkcolor=blue,
	citecolor=blue,
	urlcolor=blue,
	pdftitle={T0-Theory: Network Representation and Dimensional Analysis in the T0-Theory}
\hypersetup{
	colorlinks=true,
	linkcolor=blue,
	citecolor=blue,
	urlcolor=blue,
	pdftitle={T0-Theory: Neutrinos}
\hypersetup{
	colorlinks=true,
	linkcolor=blue,
	citecolor=blue,
	urlcolor=blue,
	pdftitle={T0-Theory: Particle Masses}
\hypersetup{
	colorlinks=true,
	linkcolor=blue,
	citecolor=blue,
	urlcolor=blue,
	pdftitle={T0-Theory: The Seven Riddles}
\hypersetup{
	colorlinks=true,
	linkcolor=blue,
	citecolor=blue,
	urlcolor=blue,
	pdftitle={T0-Theory: The T0-Time-Mass Duality}
\hypersetup{
	colorlinks=true,
	linkcolor=blue,
	citecolor=blue,
	urlcolor=blue,
	pdftitle={Temperature Units in Natural Units: T0-Theory}
\hypersetup{
	colorlinks=true,
	linkcolor=blue,
	citecolor=blue,
	urlcolor=blue,
	pdftitle={Temperatureinheiten in nat\"urlichen Einheiten: T0-Theorie}
\hypersetup{
	colorlinks=true,
	linkcolor=blue,
	citecolor=blue,
	urlcolor=blue,
	pdftitle={The Electron Unit Charge in T0 Theory: Beyond Point Singularities}
\hypersetup{
	colorlinks=true,
	linkcolor=blue,
	citecolor=blue,
	urlcolor=blue,
	pdftitle={The Fine Structure Constant: Various Representations and Relationships}
\hypersetup{
	colorlinks=true,
	linkcolor=blue,
	citecolor=blue,
	urlcolor=blue,
	pdftitle={The Geometric Formalism of T0 Quantum Mechanics and its Application to Quantum Computing}
\hypersetup{
	colorlinks=true,
	linkcolor=blue,
	citecolor=blue,
	urlcolor=blue,
	pdftitle={The Mass Scaling Exponent κ in T0 Theory}
\hypersetup{
	colorlinks=true,
	linkcolor=blue,
	citecolor=blue,
	urlcolor=blue,
	pdftitle={The Musical Spiral and 137: The Mathematical Discovery of Cosmic Detuning}
\hypersetup{
	colorlinks=true,
	linkcolor=blue,
	citecolor=blue,
	urlcolor=blue,
	pdftitle={The Relational Number System: Prime Numbers as Fundamental Ratios}
\hypersetup{
	colorlinks=true,
	linkcolor=blue,
	citecolor=blue,
	urlcolor=blue,
	pdftitle={The T0 Model (Planck-Referenced): A Reformulation of Physics}
\hypersetup{
	colorlinks=true,
	linkcolor=blue,
	citecolor=blue,
	urlcolor=blue,
	pdftitle={The T0 Model: Time-Energy Duality and Geometric Rest Mass}
\hypersetup{
	colorlinks=true,
	linkcolor=blue,
	citecolor=blue,
	urlcolor=blue,
	pdftitle={The T0-Model (Planck-Referenced): A Reformulation of Physics}
\hypersetup{
	colorlinks=true,
	linkcolor=blue,
	citecolor=blue,
	urlcolor=blue,
	pdftitle={Verbindungen zwischen dem Mizohata-Takeuchi-Gegenbeispiel und der T0-Zeit-Masse-Dualitätstheorie}
\hypersetup{
	colorlinks=true,
	linkcolor=blue,
	citecolor=blue,
	urlcolor=blue,
	pdftitle={Vereinfachte Dirac-Gleichung in der T0-Theorie: Feldknoten-Ansatz}
\hypersetup{
	colorlinks=true,
	linkcolor=blue,
	citecolor=blue,
	urlcolor=blue,
	pdftitle={Vereinfachte T0-Theorie: Elegante Lagrange-Dichte für Zeit-Masse-Dualität}
\hypersetup{
	colorlinks=true,
	linkcolor=blue,
	citecolor=blue,
	urlcolor=blue,
	pdftitle={Verhältnisbasiert vs. Absolut: Die Rolle der fraktalen Korrektur in der T0-Theorie}
\hypersetup{
	colorlinks=true,
	linkcolor=blue,
	citecolor=blue,
	urlcolor=blue,
	pdftitle={Vollständige Herleitung der Higgs-Masse und Wilson-Koeffizienten}
\hypersetup{
	colorlinks=true,
	linkcolor=blue,
	citecolor=blue,
	urlcolor=blue,
	pdftitle={Vollständiges Teilchenspektrum: Standard-Modell vs T0-Theorie}
\hypersetup{
	colorlinks=true,
	linkcolor=blue,
	citecolor=blue,
	urlcolor=blue,
	pdftitle={Warum Zahlenverhältnisse nicht direkt gekürzt werden dürfen}
\hypersetup{
	colorlinks=true,
	linkcolor=blue,
	citecolor=blue,
	urlcolor=blue,
	pdftitle={Why Numerical Ratios Must Not Be Directly Simplified}
\hypersetup{
	colorlinks=true,
	linkcolor=blue,
	citecolor=blue,
	urlcolor=blue,
}
\hypersetup{
	colorlinks=true,
	linkcolor=blue,
	citecolor=red,
	urlcolor=blue,
	bookmarks=true,
	bookmarksnumbered=true,
	pdfstartview=FitH,
	pdftitle={T0 Model - Field-Theoretic Derivation of the Beta Parameter}
\hypersetup{
	colorlinks=true,
	linkcolor=blue,
	citecolor=red,
	urlcolor=blue,
	bookmarks=true,
	bookmarksnumbered=true,
	pdfstartview=FitH,
	pdftitle={T0-Modell - Feldtheoretische Herleitung des Beta-Parameters}
\hypersetup{
	colorlinks=true,
	linkcolor=blue,
	filecolor=magenta,
	urlcolor=cyan,
}
\hypersetup{
	colorlinks=true,
	linkcolor=blue,
	urlcolor=blue,
	citecolor=blue,
	pdftitle={From Time Dilation to Mass Variation: Mathematical Core Formulations of Time-Mass Duality Theory - Updated Framework}
\hypersetup{
	colorlinks=true,
	linkcolor=blue,
	urlcolor=blue,
	citecolor=blue,
	pdftitle={T0 Model: Detailed Formula for Leptonic Anomalies}
\hypersetup{
	colorlinks=true,
	linkcolor=blue,
	urlcolor=blue,
	citecolor=blue,
	pdftitle={T0 Model: Detaillierte Formel für leptonische Anomalien}
\hypersetup{
	colorlinks=true,
	linkcolor=blue,
	urlcolor=blue,
	citecolor=blue,
	pdftitle={T0 Model: Energy-based Formulas with Quadratic Scaling}
\hypersetup{
	colorlinks=true,
	linkcolor=blue,
	urlcolor=blue,
	citecolor=blue,
	pdftitle={T0 Model: Granulation, Limits and Fundamental Asymmetry}
\hypersetup{
	colorlinks=true,
	linkcolor=blue,
	urlcolor=blue,
	citecolor=blue,
	pdftitle={T0-Modell: Energiebasierte Formeln mit quadratischer Skalierung}
\hypersetup{
	colorlinks=true,
	linkcolor=blue,
	urlcolor=blue,
	citecolor=blue,
	pdftitle={T0-Modell: Granulation, Limits und fundamentale Asymmetrie}
\hypersetup{
	colorlinks=true,
	linkcolor=blue,
	urlcolor=blue,
	citecolor=blue,
	pdftitle={Von Zeitdilatation zu Massenvariation: Mathematische Kernformulierungen der Zeit-Masse-Dualitätstheorie - Aktualisiertes Framework}
\hypersetup{
	colorlinks=true,
	linkcolor=t0blue,
	citecolor=t0blue,
	urlcolor=t0blue,
	pdftitle={T0 Model: Complete Theoretical Summary}
\hypersetup{
	colorlinks=true,
	linkcolor=t0blue,
	citecolor=t0blue,
	urlcolor=t0blue,
	pdftitle={T0 Theory: Resolution of Apparent Instantaneity}
\hypersetup{
	colorlinks=true,
	linkcolor=t0blue,
	citecolor=t0blue,
	urlcolor=t0blue,
	pdftitle={T0 vs Synergetics: Vereinfachung durch natürliche Einheiten}
\hypersetup{
	colorlinks=true,
	linkcolor=t0blue,
	citecolor=t0blue,
	urlcolor=t0blue,
	pdftitle={T0-Modell: Vollständige theoretische Zusammenfassung}
\hypersetup{
	colorlinks=true,
	linkcolor=t0blue,
	citecolor=t0blue,
	urlcolor=t0blue,
	pdftitle={T0-Theorie: Auflösung der scheinbaren Instantanität}
\hypersetup{
	colorlinks=true,
	linkcolor=t0blue,
	citecolor=t0blue,
	urlcolor=t0blue,
	pdftitle={T0-Theorie: Vollständige Dokumentenübersicht}
\hypersetup{
	colorlinks=true,
	linkcolor=t0blue,
	citecolor=t0blue,
	urlcolor=t0blue,
	pdftitle={T0-Theory: Complete Document Overview}
\hypersetup{
	colorlinks=true,
	linkcolor=t0blue,
	citecolor=t0blue,
	urlcolor=t0blue,
}
\hypersetup{
	colorlinks=true,
	linkcolor=t0blue,
	citecolor=t0green,
	urlcolor=t0blue,
	pdftitle={Das verborgene Geheimnis von 1/137}
\hypersetup{
	colorlinks=true,
	linkcolor=t0blue,
	citecolor=t0green,
	urlcolor=t0blue,
	pdftitle={The Hidden Secret of 1/137}
\hypersetup{
    colorlinks=true,
    linkcolor=blue,
    citecolor=blue,
    urlcolor=blue,
    pdftitle={Analyse und Implikationen des MNRAS-Papiers 544 für die T0-Theorie}
\hypersetup{
  colorlinks=true,
  linkcolor=blue,
  citecolor=blue,
  urlcolor=blue
}
\hypersetup{
  colorlinks=true,
  linkcolor=blue,
  citecolor=blue,
  urlcolor=blue,
  pdftitle={T0-Theorie: Ein-Uhr-Metrologie und Drei-Uhren-Experiment}
\hypersetup{
  colorlinks=true,
  linkcolor=blue,
  citecolor=blue,
  urlcolor=blue,
  pdftitle={T0-Theory: Single-Clock Metrology and Three-Clock Experiment}
\hypersetup{
colorlinks=true,
linkcolor=blue,
citecolor=blue,
urlcolor=blue,
pdftitle={Quantenmechanik im T0-Modell: Feldtheoretische Grundlagen}
\hypersetup{
colorlinks=true,
linkcolor=blue,
citecolor=blue,
urlcolor=blue,
pdftitle={T0-Theory: Neutrinos}
\newcommand{\Bzero}{B_0}
\newcommand{\CQCD}{C_{\text{QCD}
\newcommand{\Cconv}{C_{\text{conv}
\newcommand{\Cto}{C_{\text{T0}
\newcommand{\Czero}{C_0}
\newcommand{\DTmu}{D_{T,\mu}
\newcommand{\DcovT}[1]{\partial_\mu #1 + #1 \partial_\mu \Tfield}
\newcommand{\Dfrak}{D_f}
\newcommand{\Df}{D_f}
\newcommand{\DhiggsT}{\Tfield (\partial_\mu + ig A_\mu) \Phi + \Phi \partial_\mu \Tfield}
\newcommand{\EPlanck}{E_P}
\newcommand{\EPlanck}{E_{\text{Pl}
\newcommand{\EPratio}[1]{\frac{#1}
\newcommand{\EP}{E_P}
\newcommand{\EP}{E_{\text{P}
\newcommand{\EW}{E_W}
\newcommand{\EZ}{E_Z}
\newcommand{\Echar}{E_{\text{char}
\newcommand{\Ee}{E_e}
\newcommand{\Efield}{E(x,t)}
\newcommand{\Efield}{E_\text{field}
\newcommand{\Efield}{E_{\text{Feld}
\newcommand{\Efield}{E_{\text{Field}
\newcommand{\Efield}{E_{\text{field}
\newcommand{\Efield}{E}
\newcommand{\Egamma}{E_\gamma}
\newcommand{\Eh}{E_h}
\newcommand{\Emu}{E_\mu}
\newcommand{\Enorm}[1]{E_{\text{norm}
\newcommand{\En}{E_n}
\newcommand{\Ep}{E_p}
\newcommand{\Eratio}[2]{\frac{E_{#1}
\newcommand{\Etau}{E_\tau}
\newcommand{\Evis}{E_{\text{vis}
\newcommand{\Exi}{E_\xi}
\newcommand{\Ezero}{E_0}
\newcommand{\GeV}{\,\text{GeV}
\newcommand{\Gnat}{G_{\text{nat}
\newcommand{\Gsi}{G_{\text{SI}
\newcommand{\Hubble}{H_0}
\newcommand{\Kfrak}{K_{\text{frac}
\newcommand{\Kfrak}{K_{\text{frak}
\newcommand{\Kspec}{K_{\text{spec}
\newcommand{\LCDM}{\Lambda\text{CDM}
\newcommand{\LPlanck}{\ell_{\text{Pl}
\newcommand{\Lag}{\mathcal{L}
\newcommand{\Lambdat}{\Lambda_T}
\newcommand{\Leff}{L_{\text{eff}
\newcommand{\Lorentz}[2]{{\Lambda^\mu{}
\newcommand{\Lp}{L_{\text{P}
\newcommand{\Lxi}{L_\xi}
\newcommand{\Lzero}{L_0}
\newcommand{\MPl}{M_{\text{Pl}
\newcommand{\MSbar}{\overline{\text{MS}
\newcommand{\MeV}{\,\text{MeV}
\newcommand{\Mpl}{M_{\text{Pl}
\newcommand{\OmegaDM}{\Omega_{\text{DM}
\newcommand{\OmegaLambda}{\Omega_{\Lambda}
\newcommand{\Omegab}{\Omega_b}
\newcommand{\Phiphoton}{\Phi_{\text{photon}
\newcommand{\Ricci}{R_{\mu\nu}
\newcommand{\Riem}{R^\rho{}
\newcommand{\Rzero}{R_\infty}
\newcommand{\Scal}{R}
\newcommand{\SynchPower}{P_{\text{synch}
\newcommand{\TPlanck}{t_{\text{Pl}
\newcommand{\Tfieldt}{T(\vec{x}
\newcommand{\Tfieldt}{T(x,t)}
\newcommand{\Tfield}{T(x)}
\newcommand{\Tfield}{T(x,t)}
\newcommand{\Tfield}{T_{\text{field}
\newcommand{\Tfield}{T}
\newcommand{\Tfield}{\mathcal{T}
\newcommand{\Tzerot}{T_0(\Tfield)}
\newcommand{\Tzero}{T_0}
\newcommand{\Weyl}{C^\rho{}
\newcommand{\ZPinch}{J \times B = \nabla p}
\newcommand{\aleph}{\aleph}
\newcommand{\alphaEMSI}{\alpha_{\text{EM,SI}
\newcommand{\alphaEMnat}{\alpha_{\text{EM,nat}
\newcommand{\alphaEM}{\alpha_{\text{EM}
\newcommand{\alphaEM}{\ensuremath{\alpha_{\text{EM}
\newcommand{\alphaQCD}{\alpha_s}
\newcommand{\alphaQED}{\alpha_{\text{QED}
\newcommand{\alphaSI}{\alpha_{\text{SI}
\newcommand{\alphaT}{\alpha_{\text{T}
\newcommand{\alphaWSI}{\alpha_{\text{W,SI}
\newcommand{\alphaWnat}{\alpha_{\text{W,nat}
\newcommand{\alphaW}{\alpha_{\text{W}
\newcommand{\alphaem}{\alpha_{EM}
\newcommand{\alphaem}{\alpha}
\newcommand{\alphafine}{\alpha}
\newcommand{\alphagem}{\alpha}
\newcommand{\alphanat}{\alpha_{\text{nat}
\newcommand{\alphapar}{\alpha}
\newcommand{\betaTSI}{\beta_{\text{T,SI}
\newcommand{\betaTnat}{\beta_{\text{T,nat}
\newcommand{\betaT}{\beta_T}
\newcommand{\betaT}{\beta_{T}
\newcommand{\betaT}{\beta_{\text{T}
\newcommand{\betaT}{\ensuremath{\beta_T}
\newcommand{\betapar}{\beta}
\newcommand{\calL}{\mathcal{L}
\newcommand{\checked}{\checkmark}
\newcommand{\checkmarkx}{\checkmark}
\newcommand{\dTdt}{\frac{d\Tfieldt}
\newcommand{\deltaE}{\delta E}
\newcommand{\deltafield}{\ensuremath{\delta m}
\newcommand{\deltam}{\delta m}
\newcommand{\deq}{\displaystyle}
\newcommand{\docref}[1]{\texttt{#1}
\newcommand{\eV}{\,\text{eV}
\newcommand{\epsilonT}{\varepsilon_T}
\newcommand{\epsilonzero}{\varepsilon_0}
\newcommand{\etavis}{\eta_{\text{visual}
\newcommand{\e}{\mathrm{e}
\newcommand{\gW}{g_W}
\newcommand{\gammaf}{\gamma_{\text{Lorentz}
\newcommand{\gammamu}{\gamma^\mu}
\newcommand{\gs}{g_s}
\newcommand{\inftytext}{$\infty$}
\newcommand{\interval}[2]{#1:#2}
\newcommand{\kfrac}{K_{\text{frak}
\newcommand{\lP}{\ell_{\text{P}
\newcommand{\lP}{l_P}
\newcommand{\lambdah}{\ensuremath{\lambda_h}
\newcommand{\lambdah}{\lambda_h}
\newcommand{\lambdazero}{\lambda_0}
\newcommand{\mP}{m_{\text{P}
\newcommand{\mfield}{m(x,t)}
\newcommand{\mfield}{m}
\newcommand{\mh}{m_h}
\newcommand{\micrometer}{\ensuremath{\mu}
\newcommand{\mikrometer}{\ensuremath{\mu}
\newcommand{\myRightarrow}{\ensuremath{\Rightarrow}
\newcommand{\myapprox}{\ensuremath{\approx}
\newcommand{\myomega}{\ensuremath{\omega}
\newcommand{\myphi}{\ensuremath{\phi}
\newcommand{\mypi}{\ensuremath{\pi}
\newcommand{\mypropto}{\ensuremath{\propto}
\newcommand{\myrightarrow}{\ensuremath{\rightarrow}
\newcommand{\mysim}{\ensuremath{\sim}
\newcommand{\mysqrt}{\ensuremath{\sqrt}
\newcommand{\mytimes}{\ensuremath{\times}
\newcommand{\natunits}{\hbar = c = G = k_B = 1}
\newcommand{\natunits}{\text{(nat. Einh.)}
\newcommand{\natunits}{\text{(nat. units)}
\newcommand{\nulep}{\nu}
\newcommand{\nuzero}{\nu_0}
\newcommand{\partialop}{\ensuremath{\partial}
\newcommand{\pdTdt}{\frac{\partial\Tfieldt}
\newcommand{\pdTdx}{\nabla\Tfieldt}
\newcommand{\phiT}{\phi}
\newcommand{\pichar}{\pi}
\newcommand{\primrel}[1]{\mathbf{#1}
\newcommand{\rhoCMB}{\rho_{\text{CMB}
\newcommand{\rhoCasimir}{\rho_{\text{Casimir}
\newcommand{\rhoE}{\rho_E}
\newcommand{\rhofield}{\ensuremath{\rho}
\newcommand{\rzero}{r_0}
\newcommand{\slashk}{\cancel{k}
\newcommand{\slashp}{\cancel{p}
\newcommand{\slashq}{\cancel{q}
\newcommand{\tP}{t_P}
\newcommand{\tP}{t_{\text{P}
\newcommand{\tablescale}{0.9}
\newcommand{\tzero}{t_0}
\newcommand{\vect}[1]{\boldsymbol{#1}
\newcommand{\vecx}{\vec{x}
\newcommand{\vh}{v}
\newcommand{\vr}{\vec{r}
\newcommand{\warningx}{\color{red}
\newcommand{\warningx}{\textbf{!}
\newcommand{\warningx}{{\color{red}
\newcommand{\xiT}{\xi}
\newcommand{\xiconst}{\xi = \frac{4}
\newcommand{\xicoupling}{f(E/\Exi)}
\newcommand{\xigeom}{\xi_{\text{geom}
\newcommand{\xigeom}{\xi}
\newcommand{\xikonst}{\xi = \frac{4}
\newcommand{\xiparticle}{\xi_{\text{particle}
\newcommand{\xipar}{\ensuremath{\xi}
\newcommand{\xipar}{\xi_0}
\newcommand{\xipar}{\xi}
\newcommand{\xirat}{\xi_{\text{ratio}
\newtheorem{axiom}{Axiom}
\newtheorem{category}{Category-Theoretic Basis}
\newtheorem{category}{Kategorientheoretische Basis}
\newtheorem{corollary}[theorem]{Corollary}
\newtheorem{corollary}[theorem]{Korollar}
\newtheorem{corollary}{Corollary}
\newtheorem{corollary}{Korollar}
\newtheorem{definition}[theorem]{Definition}
\newtheorem{definition}{Definition}
\newtheorem{discovery}{Discovery}
\newtheorem{discovery}{Neue Entdeckung}
\newtheorem{discovery}{New Discovery}
\newtheorem{discovery}{Revolutionary Discovery}
\newtheorem{entdeckung}{Entdeckung}
\newtheorem{entdeckung}{Revolutionäre Entdeckung}
\newtheorem{erkenntnis}{Erkenntnis}
\newtheorem{erkenntnis}{Schlüsselerkenntnis}
\newtheorem{example}[theorem]{Beispiel}
\newtheorem{example}[theorem]{Example}
\newtheorem{example}{Beispiel}
\newtheorem{example}{Example}
\newtheorem{insight}{Central Insight}
\newtheorem{insight}{Insight}
\newtheorem{insight}{Key Insight}
\newtheorem{insight}{Wichtige Einsicht}
\newtheorem{insight}{Zentrale Einsicht}
\newtheorem{lemma}[theorem]{Lemma}
\newtheorem{lemma}{Lemma}
\newtheorem{principle}{Fundamental Principle}
\newtheorem{principle}{Fundamentales Prinzip}
\newtheorem{principle}{Grundlegendes Prinzip}
\newtheorem{principle}{Principle}
\newtheorem{principle}{Prinzip}
\newtheorem{prinzip}{Grundprinzip}
\newtheorem{proof_step}{Beweisschritt}
\newtheorem{proof_step}{Proof Step}
\newtheorem{proposition}[theorem]{Proposition}
\newtheorem{proposition}{Proposition}
\newtheorem{remark}[theorem]{Bemerkung}
\newtheorem{remark}[theorem]{Remark}
\newtheorem{theorem}{Theorem}
\newtheorem{warning}[theorem]{Warning}
\newtheorem{warning}[theorem]{Warnung}
\newunicodechar{±}{\ensuremath{\pm}
\newunicodechar{×}{\ensuremath{\times}
\newunicodechar{÷}{\ensuremath{\div}
\newunicodechar{ħ}{\ensuremath{\hbar}
\newunicodechar{Α}{\ensuremath{A}
\newunicodechar{Β}{\ensuremath{B}
\newunicodechar{Γ}{\ensuremath{\Gamma}
\newunicodechar{Δ}{\ensuremath{\Delta}
\newunicodechar{Ε}{\ensuremath{E}
\newunicodechar{Ζ}{\ensuremath{Z}
\newunicodechar{Η}{\ensuremath{H}
\newunicodechar{Θ}{\ensuremath{\Theta}
\newunicodechar{Ι}{\ensuremath{I}
\newunicodechar{Κ}{\ensuremath{K}
\newunicodechar{Λ}{\ensuremath{\Lambda}
\newunicodechar{Μ}{\ensuremath{M}
\newunicodechar{Ν}{\ensuremath{N}
\newunicodechar{Ξ}{\ensuremath{\Xi}
\newunicodechar{Ο}{\ensuremath{O}
\newunicodechar{Π}{\ensuremath{\Pi}
\newunicodechar{Ρ}{\ensuremath{P}
\newunicodechar{Σ}{\ensuremath{\Sigma}
\newunicodechar{Τ}{\ensuremath{T}
\newunicodechar{Υ}{\ensuremath{\Upsilon}
\newunicodechar{Φ}{\ensuremath{\Phi}
\newunicodechar{Χ}{\ensuremath{X}
\newunicodechar{Ψ}{\ensuremath{\Psi}
\newunicodechar{Ω}{\ensuremath{\Omega}
\newunicodechar{α}{\ensuremath{\alpha}
\newunicodechar{β}{\ensuremath{\beta}
\newunicodechar{γ}{\ensuremath{\gamma}
\newunicodechar{δ}{\ensuremath{\delta}
\newunicodechar{ε}{\ensuremath{\varepsilon}
\newunicodechar{ζ}{\ensuremath{\zeta}
\newunicodechar{η}{\ensuremath{\eta}
\newunicodechar{θ}{\ensuremath{\theta}
\newunicodechar{ι}{\ensuremath{\iota}
\newunicodechar{κ}{\ensuremath{\kappa}
\newunicodechar{λ}{\ensuremath{\lambda}
\newunicodechar{μ}{\ensuremath{\mu}
\newunicodechar{ν}{\ensuremath{\nu}
\newunicodechar{ξ}{\ensuremath{\xi}
\newunicodechar{ο}{\ensuremath{o}
\newunicodechar{π}{\ensuremath{\pi}
\newunicodechar{ρ}{\ensuremath{\rho}
\newunicodechar{σ}{\ensuremath{\sigma}
\newunicodechar{τ}{\ensuremath{\tau}
\newunicodechar{υ}{\ensuremath{\upsilon}
\newunicodechar{φ}{\ensuremath{\phi}
\newunicodechar{φ}{\ensuremath{\varphi}
\newunicodechar{χ}{\ensuremath{\chi}
\newunicodechar{ψ}{\ensuremath{\psi}
\newunicodechar{ω}{\ensuremath{\omega}
\newunicodechar{←}{\ensuremath{\leftarrow}
\newunicodechar{→}{\ensuremath{\rightarrow}
\newunicodechar{↔}{\ensuremath{\leftrightarrow}
\newunicodechar{⇐}{\ensuremath{\Leftarrow}
\newunicodechar{⇒}{\ensuremath{\Rightarrow}
\newunicodechar{⇔}{\ensuremath{\Leftrightarrow}
\newunicodechar{∂}{\ensuremath{\partial}
\newunicodechar{∅}{\ensuremath{\emptyset}
\newunicodechar{∇}{\ensuremath{\nabla}
\newunicodechar{∈}{\ensuremath{\in}
\newunicodechar{∉}{\ensuremath{\notin}
\newunicodechar{∏}{\ensuremath{\prod}
\newunicodechar{∑}{\ensuremath{\sum}
\newunicodechar{√}{\ensuremath{\sqrt}
\newunicodechar{∝}{\ensuremath{\propto}
\newunicodechar{∞}{\ensuremath{\infty}
\newunicodechar{∩}{\ensuremath{\cap}
\newunicodechar{∪}{\ensuremath{\cup}
\newunicodechar{∫}{\ensuremath{\int}
\newunicodechar{≈}{\ensuremath{\approx}
\newunicodechar{≠}{\ensuremath{\neq}
\newunicodechar{≤}{\ensuremath{\leq}
\newunicodechar{≥}{\ensuremath{\geq}
\newunicodechar{★}{\ensuremath{\star}
\newunicodechar{✓}{\checkmark}
\pgfplotsset{compat=1.17}
\pgfplotsset{compat=1.18}
\renewcommand{\cftchapfont}{\large\bfseries\color{blue}
\renewcommand{\cftchappagefont}{\large\bfseries\color{blue}
\renewcommand{\cftsecfont}{\bfseries}
\renewcommand{\cftsecfont}{\color{blue}
\renewcommand{\cftsecfont}{\large\bfseries\color{blue}
\renewcommand{\cftsecpagefont}{\bfseries}
\renewcommand{\cftsecpagefont}{\color{blue}
\renewcommand{\cftsecpagefont}{\large\bfseries\color{blue}
\renewcommand{\cftsubsecfont}{\color{blue!80!black}
\renewcommand{\cftsubsecfont}{\color{blue}
\renewcommand{\cftsubsecpagefont}{\color{blue!80!black}
\renewcommand{\cftsubsecpagefont}{\color{blue}
\renewcommand{\cftsubsubsecfont}{\color{blue!60!black}
\renewcommand{\cftsubsubsecfont}{\color{blue}
\renewcommand{\cftsubsubsecpagefont}{\color{blue!60!black}
\renewcommand{\cftsubsubsecpagefont}{\color{blue}
\renewcommand{\cfttoctitlefont}{\huge\bfseries\color{blue}
\renewcommand{\cfttoctitlefont}{\huge\bfseries}
\renewcommand{\familydefault}{\sfdefault}
\renewcommand{\footrulewidth}{0.4pt}
\renewcommand{\headrulewidth}{0.4pt}
\sisetup{locale = DE, group-separator = {.}
\sisetup{locale = DE}
\usetikzlibrary{arrows.meta,positioning,shapes.geometric}
\usetikzlibrary{decorations.pathmorphing, patterns, shapes.arrows}
\usetikzlibrary{intersections}
\usetikzlibrary{positioning, arrows.meta}
\usetikzlibrary{positioning, arrows}
\usetikzlibrary{positioning, shapes.geometric, arrows.meta}
\usetikzlibrary{positioning,shapes,arrows}

% Common settings
\setlength{\headheight}{15pt}
\pgfplotsset{compat=1.18}
\usetikzlibrary{positioning,shapes,arrows,arrows.meta}

% Hyperref setup
\hypersetup{
    colorlinks=true,
    linkcolor=blue,
    citecolor=blue,
    urlcolor=blue
}


\title{neutrino-Formel De}
\author{Johann Pascher}
\date{\today}

\begin{document}

\maketitle
\tableofcontents

\begin{abstract}
		Dieses Dokument präsentiert eine mathematisch konsistente Formel-Struktur für Neutrino-Berechnungen im Rahmen des T0-Modells, basierend auf der Hypothese gleicher Massen für alle Flavour-Zustände (\(\nu_e, \nu_\mu, \nu_\tau\)). Die Neutrino-Masse wird durch die Photon-Analogie (\(\frac{\xipar^2}{2}\)-Suppression) abgeleitet, und Oszillationen werden durch geometrische Phasen basierend auf \( T_x \cdot m_x = 1 \) erklärt, wobei die Quantenzahlen (\(n, \ell, j\)) die Phasenunterschiede bestimmen. Ein plausibler Zielwert für die Neutrino-Masse (\(m_\nu = 15 \text{ meV}\)) wird aus empirischen Daten (kosmologische Grenzen) abgeleitet. Die T0-Theorie basiert auf spekulativen geometrischen Harmonien ohne empirische Basis und ist mit hoher Wahrscheinlichkeit unvollständig oder falsch. Die wissenschaftliche Integrität erfordert die klare Trennung zwischen mathematischer Korrektheit und physikalischer Gültigkeit.
	\end{abstract}
	
	\tableofcontents
	\newpage
	
	# Präambel: Wissenschaftliche Ehrlichkeit
	
	\begin{warning}
		\textbf{KRITISCHE EINSCHRÄNKUNG:} Die folgenden Formeln für Neutrino-Massen sind \textbf{spekulative Extrapolationen} basierend auf der ungetesteten Hypothese, dass Neutrinos geometrischen Harmonien folgen und alle Flavour-Zustände gleiche Massen besitzen. Diese Hypothese hat \textbf{keine empirische Basis} und ist mit hoher Wahrscheinlichkeit unvollständig oder falsch. Die mathematischen Formeln sind dennoch intern konsistent und fehlerfrei formuliert.
		
		\vspace{0.5cm}
		\textbf{Wissenschaftliche Integrität bedeutet:}
		
			- Ehrlichkeit über spekulative Natur der Vorhersagen
			- Mathematische Korrektheit trotz physikalischer Unsicherheit
			- Klare Trennung zwischen Hypothesen und verifizierten Fakten
		
	\end{warning}
	
	# Neutrinos als ''fast-masselose Photonen'': Die T0-Photon-Analogie
	
	\begin{speculation}
		\textbf{Fundamentale T0-Einsicht:} Neutrinos können als ''gedämpfte Photonen'' verstanden werden.
		
		Die bemerkenswerte Ähnlichkeit zwischen Photonen und Neutrinos legt eine tiefere geometrische Verwandtschaft nahe:
		
			- \textbf{Geschwindigkeit:} Beide propagieren nahezu mit Lichtgeschwindigkeit
			- \textbf{Durchdringung:} Beide haben extreme Durchdringungsfähigkeit
			- \textbf{Masse:} Photon exakt masselos, Neutrino quasi-masselos
			- \textbf{Wechselwirkung:} Photon elektromagnetisch, Neutrino schwach
		
	\end{speculation}
	
	## Photon-Neutrino-Korrespondenz
	
	\begin{important}
		\textbf{Physikalische Parallelen:}
		
```math-align

			\text{Photon:} \quad &E^2 = (pc)^2 + 0 \quad \text{(perfekt masselos)} \\
			\text{Neutrino:} \quad &E^2 = (pc)^2 + \left(\sqrt{\frac{\xipar^2}{2}} m c^2\right)^2 \quad \text{(quasi-masselos)}
		
```

		
		\textbf{Geschwindigkeitsvergleich:}
		
```math-align

			v_\gamma &= c \quad \text{(exakt)} \\
			v_\nu &= c \times \left(1 - \frac{\xipar^2}{2}\right) \approx 0.9999999911 \times c
		
```

		
		Die Geschwindigkeitsdifferenz beträgt nur \(8.89 \times 10^{-9}\) -- praktisch unmessbar!
	\end{important}
	
	## Doppelte \(\xipar\)-Suppression aus Photon-Analogie
	
	\begin{formula}
		\textbf{T0-Hypothese:} Neutrino = Photon mit geometrischer Doppeldämpfung
		
		Wenn Neutrinos ''fast-Photonen'' sind, dann ergeben sich zwei Suppressionsfaktoren:
		
			- \textbf{Erster \(\xipar\)-Faktor:} ''Fast masselos'' (wie Photon, aber nicht perfekt)
			- \textbf{Zweiter \(\xipar\)-Faktor:} ''Schwache Wechselwirkung'' (geometrische Kopplung)
			- \textbf{Resultat:} \(m_\nu \propto \frac{\xipar^2}{2}\), konsistent mit der Geschwindigkeitsdifferenz \(v_\nu = c \times \left(1 - \frac{\xipar^2}{2}\right)\)
		
		
		\textbf{Wechselwirkungsstärken-Vergleich:}
		
```math-align

			\sigma_\gamma &\sim \alpha_{\text{EM}} \approx \frac{1}{137} \\
			\sigma_\nu &\sim \frac{\xipar^2}{2} \times G_F \approx 8.888888 \times 10^{-9}
		
```

		
		Das Verhältnis \(\sigma_\nu/\sigma_\gamma \sim \frac{\xipar^2}{2}\) bestätigt die geometrische Suppression!
	\end{formula}
	
	# Neutrino-Oszillationen
	
	\begin{important}
		\textbf{Neutrino-Oszillationen:} Neutrinos können ihre Identität (Flavour) während des Fluges ändern – ein Phänomen, das als Neutrino-Oszillation bekannt ist. Ein Neutrino, das als Elektron-Neutrino (\(\nu_e\)) erzeugt wurde, kann sich später als Myon-Neutrino (\(\nu_\mu\)) oder Tau-Neutrino (\(\nu_\tau\)) messen lassen und umgekehrt.
		
		Dieses Verhalten wird in der Standardphysik durch die Mischung der Masseneigenzustände (\(\nu_1, \nu_2, \nu_3\)) beschrieben, die durch die PMNS-Matrix (Pontecorvo-Maki-Nakagawa-Sakata) mit den Flavour-Zuständen (\(\nu_e, \nu_\mu, \nu_\tau\)) verbunden sind:
		
```math-align

			\begin{pmatrix}
				\nu_e \\ \nu_\mu \\ \nu_\tau
			\end{pmatrix}
			=
			U_{\text{PMNS}}
			\begin{pmatrix}
				\nu_1 \\ \nu_2 \\ \nu_3
			\end{pmatrix},
		
```

		wobei \(U_{\text{PMNS}}\) die Mischungsmatrix ist.
		
		Die Oszillationen hängen von den Massendifferenzen \(\Delta m^2_{ij} = m_i^2 - m_j^2\) und den Mischungswinkeln ab. Aktuelle experimentelle Daten (2025) liefern:
		
```math-align

			\Delta m^2_{21} &\approx 7.53 \times 10^{-5} \text{ eV}^2 \quad \text{[Solar]} \\
			\Delta m^2_{32} &\approx 2.44 \times 10^{-3} \text{ eV}^2 \quad \text{[Atmosphärisch]} \\
			m_\nu &> 0.06 \text{ eV} \quad \text{[Mindestens ein Neutrino, 3}\sigma\text{]}
		
```

		
		\textbf{Implikationen für T0:}
		
			- Die T0-Theorie postuliert gleiche Massen für die Flavour-Zustände (\(\nu_e, \nu_\mu, \nu_\tau\)), was \(\Delta m^2_{ij} = 0\) impliziert und mit Standard-Oszillationen inkompatibel ist.
			- Um Oszillationen zu erklären, verwendet die T0-Theorie geometrische Phasen basierend auf \( T_x \cdot m_x = 1 \), wobei die Quantenzahlen (\(n, \ell, j\)) die Phasenunterschiede bestimmen.
		
	\end{important}
	
	## Geometrische Phasen als Oszillationsmechanismus
	
	\begin{speculation}
		\textbf{T0-Hypothese: Geometrische Phasen für Oszillationen}
		
		Um die Hypothese gleicher Massen (\(m_{\nu_e} = m_{\nu_\mu} = m_{\nu_\tau} = m_\nu\)) mit Neutrino-Oszillationen zu vereinbaren, wird spekuliert, dass Oszillationen in der T0-Theorie durch geometrische Phasen statt durch Massendifferenzen verursacht werden. Dies basiert auf der T0-Beziehung:
		\[
		T_x \cdot m_x = 1,
		\]
		wobei \(m_x = m_\nu = 4.54 \text{ meV}\) die Neutrino-Masse ist und \(T_x\) eine charakteristische Zeit oder Frequenz:
		\[
		T_x = \frac{1}{m_\nu} = \frac{1}{4.54 \times 10^{-3} \text{ eV}} \approx 2.2026 \times 10^2 \text{ eV}^{-1} \approx 1.449 \times 10^{-13} \text{ s}.
		\]
		
		Die geometrische Phase wird durch die T0-Quantenzahlen (\(n, \ell, j\)) bestimmt:
		\[
		\phi_{\text{geo}, i} \propto f(n, \ell, j) \cdot \frac{L}{E} \cdot \frac{1}{T_x},
		\]
		wobei \(f(n, \ell, j) = \frac{n^6}{\ell^3}\) (oder 1 für \(\ell = 0\)) die geometrischen Faktoren sind:
		
```math-align

			f_{\nu_e} &= 1, \\
			f_{\nu_\mu} &= 64, \\
			f_{\nu_\tau} &= 91.125.
		
```

		
		\textbf{Berechnete Phasenunterschiede:}
		
```math-align

			\phi_{\nu_e} &\propto 1 \cdot \frac{L}{E} \cdot \frac{1}{T_x}, \\
			\phi_{\nu_\mu} &\propto 64 \cdot \frac{L}{E} \cdot \frac{1}{T_x}, \\
			\phi_{\nu_\tau} &\propto 91.125 \cdot \frac{L}{E} \cdot \frac{1}{T_x}.
		
```

		
		Diese Phasenunterschiede könnten Oszillationen zwischen Flavour-Zuständen verursachen, ohne dass unterschiedliche Massen erforderlich sind. Die genaue Form der Oszillationswahrscheinlichkeit müsste weiter entwickelt werden, bleibt aber hochspekulativ.
		
		\textbf{WARNUNG:} Dieser Ansatz ist rein hypothetisch und ohne empirische Bestätigung. Er widerspricht der etablierten Theorie, dass Oszillationen durch \(\Delta m^2_{ij} \neq 0\) verursacht werden.
	\end{speculation}
	
	# Fundamentale Konstanten und Einheiten
	
	## Basis-Parameter
	
	\begin{formula}
		\textbf{T0-Grundkonstanten:}
		
```math-align

			\xipar &= \frac{4}{3} \times 10^{-4} \approx 1.333333 \times 10^{-4} \quad \text{[dimensionslos]} \\
			\frac{\xipar^2}{2} &= \frac{\left(\frac{4}{3} \times 10^{-4}\right)^2}{2} \approx 8.888888 \times 10^{-9} \quad \text{[dimensionslos]} \\
			v &= 246.22 \text{ GeV} \quad \text{[Higgs VEV]} \\
			\hbar c &= 0.19733 \text{ GeV·fm} \quad \text{[Umrechnungskonstante]} \\
			T_x &= \frac{1}{4.54 \times 10^{-3} \text{ eV}} \approx 2.2026 \times 10^2 \text{ eV}^{-1} \approx 1.449 \times 10^{-13} \text{ s} \quad \text{[T0-Masse]}
		
```

	\end{formula}
	
	## Einheiten-Konventionen
	
	\begin{important}
		\textbf{Konsistente Einheiten-Hierarchie:}
		
```math-align

			\text{Standard:} &\quad \text{GeV} \\
			\text{Submultiples:} &\quad 1 \text{ eV} = 10^{-9} \text{ GeV} \\
			&\quad 1 \text{ meV} = 10^{-12} \text{ GeV} = 10^{-3} \text{ eV} \\
			\text{Massen:} &\quad m[\text{GeV}/c^2] = E[\text{GeV}]/c^2 \approx E[\text{GeV}] \text{ (natürliche Einheiten)} \\
			\text{Zeit:} &\quad 1 \text{ eV}^{-1} \approx 6.582 \times 10^{-16} \text{ s}
		
```

	\end{important}
	
	# Geladene Lepton-Referenzmassen
	
	## Präzise experimentelle Werte (PDG 2024)
	
	\begin{experimental}
		\textbf{Verifizierte Teilchenmassen:}
		
```math-align

			m_e &= 0.51099895000 \times 10^{-3} \text{ GeV} = 510.99895 \text{ keV} \\
			m_\mu &= 105.6583745 \times 10^{-3} \text{ GeV} = 105.6583745 \text{ MeV} \\
			m_\tau &= 1776.86 \times 10^{-3} \text{ GeV} = 1.77686 \text{ GeV}
		
```

		
		\textbf{Einheiten-Umrechnung zu eV:}
		
```math-align

			m_e &= 510998.95 \text{ eV} = 510998950 \text{ meV} \\
			m_\mu &= 105658374.5 \text{ eV} \\
			m_\tau &= 1776860000 \text{ eV}
		
```

	\end{experimental}
	
	# Neutrino-Quantenzahlen (T0-Hypothese)
	
	## Postulierte Quantenzahl-Zuordnung
	
	\begin{speculation}
		\textbf{Hypothetische Neutrino-Quantenzahlen:}
		
```math-align

			\nu_e: &\quad n=1, \ell=0, j=1/2 \quad \text{[Grundzustand-Neutrino]} \\
			\nu_\mu: &\quad n=2, \ell=1, j=1/2 \quad \text{[Erste Anregung]} \\
			\nu_\tau: &\quad n=3, \ell=2, j=1/2 \quad \text{[Zweite Anregung]}
		
```

		
		\textbf{Rolle der Quantenzahlen:}
		Die Quantenzahlen beeinflussen nicht die Neutrino-Massen (da \(m_{\nu_e} = m_{\nu_\mu} = m_{\nu_\tau}\)), sondern bestimmen die geometrischen Faktoren \(f(n, \ell, j)\), die die Oszillationsphasen steuern.
		
		\textbf{WARNUNG:} Diese Zuordnungen sind reine Spekulationen ohne experimentelle Basis.
	\end{speculation}
	
	## Geometrische Faktoren
	
	\begin{formula}
		\textbf{T0-Geometrische Faktoren:}
		
```math-align

			f(n,\ell,j) &= \frac{n^6}{\ell^3} \quad \text{für } \ell > 0 \\
			f(1,0,j) &= 1 \quad \text{für } \ell = 0 \text{ (Spezialfall)}
		
```

		
		\textbf{Berechnete Werte:}
		
```math-align

			f_{\nu_e} &= f(1,0,1/2) = 1 \\
			f_{\nu_\mu} &= f(2,1,1/2) = \frac{2^6}{1^3} = 64 \\
			f_{\nu_\tau} &= f(3,2,1/2) = \frac{3^6}{2^3} = \frac{729}{8} = 91.125
		
```

	\end{formula}
	
	# Neutrino-Masse-Formel
	
	## T0-Hypothese: Gleiche Massen mit Geometrischen Phasen
	
	\begin{speculation}
		\textbf{T0-Hypothese: Gleiche Neutrino-Massen mit Geometrischen Phasen}
		
		Die T0-Theorie postuliert, dass alle Flavour-Zustände (\(\nu_e, \nu_\mu, \nu_\tau\)) die gleiche Masse haben:
		\[
		m_{\nu_e} = m_{\nu_\mu} = m_{\nu_\tau} = m_\nu = 4.54 \text{ meV}.
		\]
		Die Masse wird aus der Photon-Analogie abgeleitet:
		\[
		m_\nu = \frac{\xipar^2}{2} \times m_e = \left(8.888888 \times 10^{-9}\right) \times (0.51099895 \times 10^{-3} \text{ GeV}) = 4.54 \text{ meV}.
		\]
		
		Um Oszillationen zu erklären, wird ein geometrischer Mechanismus postuliert, basierend auf der T0-Beziehung:
		\[
		T_x \cdot m_x = 1, \quad m_x = 4.54 \text{ meV}, \quad T_x \approx 2.2026 \times 10^2 \text{ eV}^{-1} \approx 1.449 \times 10^{-13} \text{ s}.
		\]
		
		Die Oszillationsphasen werden durch geometrische Faktoren \(f(n, \ell, j)\) bestimmt:
		\[
		\phi_{\text{geo}, i} \propto f_{\nu_i} \cdot \frac{L}{E} \cdot \frac{1}{T_x},
		\]
		wobei \(f_{\nu_e} = 1\), \(f_{\nu_\mu} = 64\), \(f_{\nu_\tau} = 91.125\).
		
		\textbf{Begründung:}
		
			- Die Masse \(4.54 \text{ meV}\) ist konsistent mit der kosmologischen Grenze (\(\Sigma m_\nu = 0.01362 \text{ eV} < 0.07 \text{ eV}\)).
			- Geometrische Phasen ermöglichen Oszillationen ohne Massendifferenzen, was die Hypothese gleicher Massen unterstützt.
			- Diese Hypothese ist hochspekulativ und ohne empirische Bestätigung.
		
	\end{speculation}
	
	\begin{formula}
		\textbf{Formel:} \(m_{\nu_i} = 4.54 \text{ meV}\)
		
		\textbf{Gesamtmasse:}
		\[
		\Sigma m_\nu = 3 \times 4.54 \text{ meV} = 13.62 \text{ meV} = 0.01362 \text{ eV}
		\]
		
		\textbf{Vergleich mit plausiblen Zielwert:}
		
			- \(\nu_e, \nu_\mu, \nu_\tau\): \(4.54 \text{ meV}\) vs. \(15 \text{ meV}\) (Übereinstimmung: \(30.3\%\))
			- \(\Sigma m_\nu\): \(13.62 \text{ meV}\) vs. \(45 \text{ meV}\) (Abweichung: Faktor \(\approx 3.30\))
		
	\end{formula}
	
	\begin{warning}
		\textbf{KRITISCHER BEFUND:} Die Hypothese gleicher Massen mit geometrischen Phasen ist inkompatibel mit den experimentellen Oszillationsdaten (\(\Delta m^2_{21} \approx 7.53 \times 10^{-5} \text{ eV}^2\), \(\Delta m^2_{32} \approx 2.44 \times 10^{-3} \text{ eV}^2\)), da sie \(\Delta m^2_{ij} = 0\) impliziert. Der geometrische Ansatz ist rein spekulativ und erfordert weitere theoretische und experimentelle Validierung.
	\end{warning}
	
	# Plausibler Zielwert basierend auf empirischen Daten
	
	## Ableitung aus Messdaten
	
	\begin{experimental}
		\textbf{Plausibler Zielwert:}
		Die T0-Theorie postuliert gleiche Massen für alle Flavour-Zustände (\(\nu_e, \nu_\mu, \nu_\tau\)). Daher wird ein einziger Zielwert für die Neutrino-Masse \(m_\nu\) abgeleitet, basierend auf empirischen Daten (Stand 2025):
		
			- Kosmologische Grenze: \(\Sigma m_\nu = 3 m_\nu < 0.07 \text{ eV} \implies m_\nu < 23.33 \text{ meV}\).
			- Oszillationsdaten: \(\Delta m^2_{21} \approx 7.53 \times 10^{-5} \text{ eV}^2\), \(\Delta m^2_{32} \approx 2.44 \times 10^{-3} \text{ eV}^2\), was normalerweise unterschiedliche Massen erfordert. Die T0-Theorie umgeht dies durch geometrische Phasen.
			- Plausibler Zielwert: \(m_\nu \approx 15 \text{ meV}\), was zwischen der solaren (\(8.68 \text{ meV}\)) und atmosphärischen Skala (\(50.15 \text{ meV}\)) liegt und die kosmologische Grenze erfüllt:
			\[
			\Sigma m_\nu = 3 \times 15 \text{ meV} = 45 \text{ meV} = 0.045 \text{ eV} < 0.07 \text{ eV}.
			\]
		
		
		\textbf{Begründung:}
		
			- Der Zielwert ist konsistent mit der kosmologischen Grenze und liegt in der Größenordnung der Oszillationsdaten.
			- Die Hypothese gleicher Massen wird durch geometrische Phasen unterstützt, was die T0-Theorie von der Standardphysik abgrenzt.
			- Der Wert ist plausibel, aber nicht direkt gemessen, da Flavour-Massen Mischungen der Eigenzustände sind.
			- Die T0-Masse (\(4.54 \text{ meV}\)) liegt unter dem Zielwert (\(30.3\%\)), ist aber ebenfalls kosmologisch konsistent.
		
	\end{experimental}
	
	# Experimentelle Vergleichsgrößen
	
	## Aktuelle experimentelle Obergrenzen (2025)
	
	\begin{experimental}
		\textbf{Experimentelle Grenzen:}
		
```math-align

			m_{\nu_e} &< 0.45 \text{ eV} \quad \text{[KATRIN, 90\% CL]} \\
			m_{\nu_\mu} &< 0.17 \text{ MeV} \quad \text{[Myon-Zerfall, indirekt]} \\
			m_{\nu_\tau} &< 18.2 \text{ MeV} \quad \text{[Tau-Zerfall, indirekt]} \\
			\Sigma m_\nu &< 0.07 \text{ eV} \quad \text{[DESI+Planck, 95\% CL]} \\
			\Delta m^2_{21} &\approx 7.53 \times 10^{-5} \text{ eV}^2 \quad \text{[Solar]} \\
			\Delta m^2_{32} &\approx 2.44 \times 10^{-3} \text{ eV}^2 \quad \text{[Atmosphärisch]} \\
			m_\nu &> 0.06 \text{ eV} \quad \text{[Mindestens ein Neutrino, 3}\sigma\text{]}
		
```

	\end{experimental}
	
	## Sicherheitsmargen für T0-Hypothese
	
	\begin{longtable}[c]{@{}lcc@{}}
		\caption{Sicherheitsmargen der T0-Hypothese zu experimentellen Grenzen} \\
		\toprule
		\textbf{Parameter} & \textbf{T0-Masse (\(4.54 \text{ meV}\))} & \textbf{Zielwert (\(15 \text{ meV}\))} \\
		\midrule
		\endfirsthead
		\toprule
		\textbf{Parameter} & \textbf{T0-Masse (\(4.54 \text{ meV}\))} & \textbf{Zielwert (\(15 \text{ meV}\))} \\
		\midrule
		\endhead
		$m_{\nu_e}$ vs 0.45 eV & 99200× & 30× \\
		$m_{\nu_\mu}$ vs 0.17 MeV & 3.74E7× & 11333× \\
		$m_{\nu_\tau}$ vs 18.2 MeV & 4.01E9× & 1.21E6× \\
		\midrule
		$\Sigma m_\nu$ vs 0.07 eV & 5.14× & 1.56× \\
		$\Sigma m_\nu$ vs 0.06 eV & 4.41× & 1.33× \\
		\bottomrule
	\end{longtable}
	
	\begin{important}
		\textbf{T0-Hypothese:}
		
			- Die T0-Masse (\(4.54 \text{ meV}\)) ist kompatibel mit kosmologischen Grenzen (\(\Sigma m_\nu = 0.01362 \text{ eV} < 0.07 \text{ eV}\)) und liegt unter dem Zielwert (\(15 \text{ meV}\), \(30.3\%\)).
			- Geometrische Phasen (\(T_x \cdot m_x = 1\)) bieten einen spekulativen Mechanismus für Oszillationen, sind aber inkompatibel mit Standard-Oszillationen.
			- Physikalische Begründung: Die Masse basiert auf der \(\frac{\xipar^2}{2}\)-Suppression, konsistent mit der Geschwindigkeitsdifferenz \(v_\nu = c \times \left(1 - \frac{\xipar^2}{2}\right)\).
		
	\end{important}
	
	# Konsistenz-Checks und Validierung
	
	## Dimensionale Analyse
	
	\begin{formula}
		\textbf{Dimensionale Konsistenz:}
		
```math-align

			[\xipar] &= 1 \quad \checkmark \text{ dimensionslos} \\
			[m_e] &= \text{GeV} \quad \checkmark \text{ Energie/Masse} \\
			\left[\frac{\xipar^2}{2} \times m_e\right] &= \text{GeV} \quad \checkmark \text{ Energie/Masse} \\
			[f_{\nu_i}] &= 1 \quad \checkmark \text{ dimensionslos} \\
			[m_\nu] &= \text{eV} \quad \checkmark \text{ (festgelegte Masse)} \\
			[T_x] &= \text{eV}^{-1} \quad \checkmark \text{ (Zeit)}
		
```

		Alle Formeln sind dimensional konsistent.
	\end{formula}
	
	## Mathematische Konsistenz
	
	\begin{important}
		\textbf{Konsistenz der Hypothese:}
		
			- Die Formel \(m_\nu = \frac{\xipar^2}{2} \times m_e = 4.54 \text{ meV}\) ist physikalisch begründet durch die Photon-Analogie und konsistent mit der Geschwindigkeitsdifferenz.
			- Geometrische Phasen basierend auf \(f(n, \ell, j)\) und \(T_x \cdot m_x = 1\) bieten einen spekulativen Mechanismus für Oszillationen.
			- Keine freien Parameter außer \(\xipar\), was die Theorie vereinfacht.
		
	\end{important}
	
	## Experimentelle Validierung
	
	\begin{experimental}
		\textbf{Validierungsstatus (Stand 2025):}
		
			- Die T0-Masse (\(4.54 \text{ meV}\)) erfüllt kosmologische Grenzen (\(\Sigma m_\nu = 0.01362 \text{ eV} < 0.07 \text{ eV}\)) und liegt unter dem Zielwert (\(15 \text{ meV}\), \(30.3\%\)).
			- Inkompatibel mit Standard-Oszillationen (\(\Delta m^2_{ij} = 0\)), aber geometrische Phasen bieten einen spekulativen Ausweg.
			- Der Zielwert (\(15 \text{ meV}\)) ist konsistent mit kosmologischen Grenzen, aber nicht direkt gemessen.
		
	\end{experimental}
	
	# Fazit
	
	\begin{important}
		\textbf{Zusammenfassung und Ausblick:}
		
			- Die T0-Theorie postuliert gleiche Neutrino-Massen (\(m_\nu = 4.54 \text{ meV}\)) basierend auf der Photon-Analogie (\(\frac{\xipar^2}{2} \times m_e\)), konsistent mit der Geschwindigkeitsdifferenz (\(v_\nu = c \times \left(1 - \frac{\xipar^2}{2}\right)\)).
			- Geometrische Phasen basierend auf \(T_x \cdot m_x = 1\) und den Quantenzahlen (\(f_{\nu_e} = 1\), \(f_{\nu_\mu} = 64\), \(f_{\nu_\tau} = 91.125\)) erklären Oszillationen spekulative, ohne Massendifferenzen.
			- Der plausible Zielwert (\(m_\nu = 15 \text{ meV}\)) basiert auf empirischen Daten (kosmologische Grenze) und liegt in der Größenordnung der Oszillationsdaten, ist aber nicht direkt gemessen.
			- Die T0-Masse (\(4.54 \text{ meV}\)) ist relativ nahe am Zielwert (\(30.3\%\)), erfüllt kosmologische Grenzen, ist aber inkompatibel mit Standard-Oszillationen.
			- Die T0-Theorie bleibt spekulativ, da sie auf geometrischen Harmonien ohne empirische Basis basiert.
			- Zukünftige Experimente (2025–2030, z. B. KATRIN-Upgrade, DESI, Euclid) könnten die T0-Hypothese, insbesondere den geometrischen Oszillationsmechanismus, weiter prüfen oder widerlegen.
			- Die wissenschaftliche Integrität erfordert, die spekulative Natur der T0-Theorie klar zu kommunizieren und weitere Tests abzuwarten.
		
	\end{important}

\end{document}
