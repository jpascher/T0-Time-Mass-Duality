\documentclass[11pt,a4paper,openany]{book}

% Essential packages
\usepackage[utf8]{inputenc}
\usepackage[T1]{fontenc}
\usepackage[english]{babel}
\usepackage[a4paper,margin=2.5cm]{geometry}
\usepackage{lmodern}

% Math and physics packages
\usepackage{amsmath}
\usepackage{amssymb}
\usepackage{amsthm}
\usepackage{mathtools}
\usepackage{physics}
\usepackage{siunitx}

% Graphics and tables
\usepackage{graphicx}
\usepackage[table,xcdraw]{xcolor}
\usepackage{tikz}
\usepackage{pgfplots}
\usepackage{tcolorbox}
\usepackage{booktabs}
\usepackage{array}
\usepackage{longtable}
\usepackage{float}

% Document formatting
\usepackage{fancyhdr}
\usepackage{tocloft}
\usepackage{hyperref}
\usepackage{cleveref}
\usepackage{microtype}
\usepackage{enumitem}
\usepackage{newunicodechar}

% Additional packages
\usepackage{adjustbox}
\usepackage{algorithm}
\usepackage{algorithmic}
\usepackage{amsfonts}
\usepackage{amsmath,amsfonts,amssymb}
\usepackage{amsmath,amsfonts,amssymb,physics}
\usepackage{amsmath,amssymb}
\usepackage{amsmath,amssymb,amsfonts,amsthm}
\usepackage{amsmath,amssymb,amsthm}
\usepackage{amsmath,amssymb,physics,graphicx,xcolor,amsthm}
\usepackage{bm}
\usepackage{booktabs,array,longtable,multirow}
\usepackage{braket}
\usepackage{breakurl}
\usepackage{cancel}
\usepackage{caption}
\usepackage{cite}
\usepackage{color}
\usepackage{colortbl}
\usepackage{csquotes}
\usepackage{doi}
\usepackage{forest}
\usepackage{gensymb}
\usepackage{geometry,fancyhdr}
\usepackage{graphicx,tikz,pgfplots}
\usepackage{hyperref,url}
\usepackage{hyphenat}
\usepackage{listings}
\usepackage{listings,enumerate}
\usepackage{mdframed}
\usepackage{multicol}
\usepackage{multirow}
\usepackage{natbib}
\usepackage{pdflscape}
\usepackage{ragged2e}
\usepackage{setspace}
\usepackage{siunitx,xcolor,graphicx}
\usepackage{slashed}
\usepackage{tabularx}
\usepackage{textcomp}
\usepackage{textgreek}
\usepackage{tikz,pgfplots}
\usepackage{upgreek}
\usepackage{url}

% Custom commands and definitions
\definecolor{blue}
\definecolor{blue}{rgb}{0,0,1}
\definecolor{boxgray}
\definecolor{boxgray}{RGB}{240,240,240}
\definecolor{deepblue}
\definecolor{deepblue}{RGB}{0,0,127}
\definecolor{deepgreen}
\definecolor{deepgreen}{RGB}{0,127,0}
\definecolor{deepred}
\definecolor{deepred}{RGB}{191,0,0}
\definecolor{t0blue}
\definecolor{t0blue}{RGB}{0,102,204}
\definecolor{t0blue}{RGB}{33,150,243}
\definecolor{t0green}
\definecolor{t0green}{RGB}{0,153,0}
\definecolor{t0green}{RGB}{0,153,76}
\definecolor{t0green}{RGB}{76,175,80}
\definecolor{t0orange}
\definecolor{t0orange}{RGB}{255,152,0}
\definecolor{t0purple}
\definecolor{t0purple}{RGB}{102,0,204}
\definecolor{t0purple}{RGB}{156,39,176}
\definecolor{t0red}
\definecolor{t0red}{RGB}{204,0,0}
\definecolor{t0red}{RGB}{204,0,51}
\definecolor{t0red}{RGB}{244,67,54}
\definecolor{t0yellow}
\definecolor{t0yellow}{RGB}{255,204,0}
\geometry{a4paper, left=25mm, right=25mm, top=25mm, bottom=25mm}
\geometry{a4paper, margin=1in}
\geometry{a4paper, margin=2.5cm}
\geometry{a4paper, margin=2cm}
\geometry{left=2.5cm,right=2.5cm,top=2.5cm,bottom=2.5cm}
\geometry{left=2cm,right=2cm,top=2cm,bottom=2cm}
\geometry{margin=1in}
\geometry{margin=2.5cm}
\geometry{margin=2cm}
\hypersetup{
	colorlinks=true,
	linkcolor=blue,
	citecolor=blue,
	urlcolor=blue,
	pdftitle={Analysis and Implications of MNRAS Paper 544 for the T0-Theory}
\hypersetup{
	colorlinks=true,
	linkcolor=blue,
	citecolor=blue,
	urlcolor=blue,
	pdftitle={Beweis: Die Feinstrukturkonstante α = 1 in natürlichen Einheiten}
\hypersetup{
	colorlinks=true,
	linkcolor=blue,
	citecolor=blue,
	urlcolor=blue,
	pdftitle={Beweis: Die Koide-Formel enthält implizit $\xi$}
\hypersetup{
	colorlinks=true,
	linkcolor=blue,
	citecolor=blue,
	urlcolor=blue,
	pdftitle={Chinas Photonischer Quantenchip: 1000x-Speedup und T0-Integration}
\hypersetup{
	colorlinks=true,
	linkcolor=blue,
	citecolor=blue,
	urlcolor=blue,
	pdftitle={Complete Derivation of Higgs Mass and Wilson Coefficients}
\hypersetup{
	colorlinks=true,
	linkcolor=blue,
	citecolor=blue,
	urlcolor=blue,
	pdftitle={Complete Particle Spectrum: Standard Model vs T0 Theory}
\hypersetup{
	colorlinks=true,
	linkcolor=blue,
	citecolor=blue,
	urlcolor=blue,
	pdftitle={Conceptual Comparison of Unified Natural Units and Extended Standard Model}
\hypersetup{
	colorlinks=true,
	linkcolor=blue,
	citecolor=blue,
	urlcolor=blue,
	pdftitle={Connections between the Mizohata-Takeuchi Counterexample and the T0 Time-Mass Duality Theory}
\hypersetup{
	colorlinks=true,
	linkcolor=blue,
	citecolor=blue,
	urlcolor=blue,
	pdftitle={Das Relationale Zahlensystem: Primzahlen als fundamentale Verhältnisse}
\hypersetup{
	colorlinks=true,
	linkcolor=blue,
	citecolor=blue,
	urlcolor=blue,
	pdftitle={Das T0-Modell (Planck-Referenziert): Eine Neuformulierung der Physik}
\hypersetup{
	colorlinks=true,
	linkcolor=blue,
	citecolor=blue,
	urlcolor=blue,
	pdftitle={Das T0-Modell: Zeit-Energie-Dualität und geometrische Ruhemasse}
\hypersetup{
	colorlinks=true,
	linkcolor=blue,
	citecolor=blue,
	urlcolor=blue,
	pdftitle={Der Massenskalierungsexponent κ in der T0-Theorie}
\hypersetup{
	colorlinks=true,
	linkcolor=blue,
	citecolor=blue,
	urlcolor=blue,
	pdftitle={Der geometrische Formalismus der T0-Quantenmechanik und seine Anwendung auf Quantencomputer}
\hypersetup{
	colorlinks=true,
	linkcolor=blue,
	citecolor=blue,
	urlcolor=blue,
	pdftitle={Der xi Parameter und Teilchendifferenzierung in der T0-Theorie}
\hypersetup{
	colorlinks=true,
	linkcolor=blue,
	citecolor=blue,
	urlcolor=blue,
	pdftitle={Deterministic Quantum Mechanics via T0-Energy Field Formulation}
\hypersetup{
	colorlinks=true,
	linkcolor=blue,
	citecolor=blue,
	urlcolor=blue,
	pdftitle={Deterministische Quantenmechanik via T0-Energiefeld-Formulierung}
\hypersetup{
	colorlinks=true,
	linkcolor=blue,
	citecolor=blue,
	urlcolor=blue,
	pdftitle={Die Elektroneneinheitsladung in der T0-Theorie: Jenseits von Punkt-Singularitäten}
\hypersetup{
	colorlinks=true,
	linkcolor=blue,
	citecolor=blue,
	urlcolor=blue,
	pdftitle={Die Feinstrukturkonstante: Verschiedene Darstellungen und Beziehungen}
\hypersetup{
	colorlinks=true,
	linkcolor=blue,
	citecolor=blue,
	urlcolor=blue,
	pdftitle={Die Musikalische Spirale und die 137: Die mathematische Entdeckung der kosmischen Verstimmung}
\hypersetup{
	colorlinks=true,
	linkcolor=blue,
	citecolor=blue,
	urlcolor=blue,
	pdftitle={E=mc² = E=m: Die Konstanten-Illusion entlarvt}
\hypersetup{
	colorlinks=true,
	linkcolor=blue,
	citecolor=blue,
	urlcolor=blue,
	pdftitle={E=mc² = E=m: The Constants Illusion Exposed}
\hypersetup{
	colorlinks=true,
	linkcolor=blue,
	citecolor=blue,
	urlcolor=blue,
	pdftitle={Einfache Lagrange-Revolution: Von der Standardmodell-Komplexität zur T0-Eleganz}
\hypersetup{
	colorlinks=true,
	linkcolor=blue,
	citecolor=blue,
	urlcolor=blue,
	pdftitle={Einführung in die Umsetzung photonischer Bauteile auf Wafern für Nachrichtentechniker}
\hypersetup{
	colorlinks=true,
	linkcolor=blue,
	citecolor=blue,
	urlcolor=blue,
	pdftitle={Einführung in photonische Quantenchips für Nachrichtentechniker}
\hypersetup{
	colorlinks=true,
	linkcolor=blue,
	citecolor=blue,
	urlcolor=blue,
	pdftitle={Elimination der Masse als dimensionaler Platzhalter im T0-Modell}
\hypersetup{
	colorlinks=true,
	linkcolor=blue,
	citecolor=blue,
	urlcolor=blue,
	pdftitle={Elimination of Mass as Dimensional Placeholder in the T0 Model}
\hypersetup{
	colorlinks=true,
	linkcolor=blue,
	citecolor=blue,
	urlcolor=blue,
	pdftitle={Empirical Analysis of Deterministic Factorization Methods}
\hypersetup{
	colorlinks=true,
	linkcolor=blue,
	citecolor=blue,
	urlcolor=blue,
	pdftitle={Empirische Analyse deterministischer Faktorisierungsmethoden}
\hypersetup{
	colorlinks=true,
	linkcolor=blue,
	citecolor=blue,
	urlcolor=blue,
	pdftitle={Integration der Dirac-Gleichung im T0-Modell: Natürliche-Einheiten-Rahmenwerk}
\hypersetup{
	colorlinks=true,
	linkcolor=blue,
	citecolor=blue,
	urlcolor=blue,
	pdftitle={Integration of the Dirac Equation in the T0 Model: Natural Units Framework}
\hypersetup{
	colorlinks=true,
	linkcolor=blue,
	citecolor=blue,
	urlcolor=blue,
	pdftitle={Introduction to Photonic Quantum Chips for Communication Engineers}
\hypersetup{
	colorlinks=true,
	linkcolor=blue,
	citecolor=blue,
	urlcolor=blue,
	pdftitle={Introduction to the Implementation of Photonic Components on Wafers for Communication Engineers}
\hypersetup{
	colorlinks=true,
	linkcolor=blue,
	citecolor=blue,
	urlcolor=blue,
	pdftitle={Konzeptioneller Vergleich von Einheitlichen Natürlichen Einheiten und Erweitertem Standardmodell}
\hypersetup{
	colorlinks=true,
	linkcolor=blue,
	citecolor=blue,
	urlcolor=blue,
	pdftitle={Markov Chains in the Context of T0 Theory: Deterministic or Stochastic? A Treatise on Patterns, Preconditions, and Uncertainty}
\hypersetup{
	colorlinks=true,
	linkcolor=blue,
	citecolor=blue,
	urlcolor=blue,
	pdftitle={Markov-Ketten im Kontext der T0-Theorie: Deterministisch oder stochastisch? Ein Traktat zu Mustern, Voraussetzungen und Unsicherheit}
\hypersetup{
	colorlinks=true,
	linkcolor=blue,
	citecolor=blue,
	urlcolor=blue,
	pdftitle={Mathematical Analysis of T0-Shor Algorithm: Theoretical Framework and Computational Complexity}
\hypersetup{
	colorlinks=true,
	linkcolor=blue,
	citecolor=blue,
	urlcolor=blue,
	pdftitle={Mathematical Constructs of Alternative CMB Models: Unnikrishnan and Peratt in Harmony with the T0 Theory}
\hypersetup{
	colorlinks=true,
	linkcolor=blue,
	citecolor=blue,
	urlcolor=blue,
	pdftitle={Mathematische Analyse des T0-Shor Algorithmus: Theoretischer Rahmen und Berechnungskomplexität}
\hypersetup{
	colorlinks=true,
	linkcolor=blue,
	citecolor=blue,
	urlcolor=blue,
	pdftitle={Mathematische Konstrukte alternativer CMB-Modelle: Unnikrishnan und Peratt im Einklang mit der T0-Theorie}
\hypersetup{
	colorlinks=true,
	linkcolor=blue,
	citecolor=blue,
	urlcolor=blue,
	pdftitle={Natural Unit Systems: Universal Energy Conversion and Fundamental Length Scale Hierarchy}
\hypersetup{
	colorlinks=true,
	linkcolor=blue,
	citecolor=blue,
	urlcolor=blue,
	pdftitle={Natural Units in Theoretical Physics: A Treatise in the Context of T0 Theory}
\hypersetup{
	colorlinks=true,
	linkcolor=blue,
	citecolor=blue,
	urlcolor=blue,
	pdftitle={Natürliche Einheiten in der theoretischen Physik: Eine Abhandlung im Kontext der T0-Theorie}
\hypersetup{
	colorlinks=true,
	linkcolor=blue,
	citecolor=blue,
	urlcolor=blue,
	pdftitle={Natürliche Einheitensysteme: Universelle Energieumwandlung und fundamentale Längenskala-Hierarchie}
\hypersetup{
	colorlinks=true,
	linkcolor=blue,
	citecolor=blue,
	urlcolor=blue,
	pdftitle={Parameter System-Dependency in T0-Model: SI vs. Natural Units}
\hypersetup{
	colorlinks=true,
	linkcolor=blue,
	citecolor=blue,
	urlcolor=blue,
	pdftitle={Parameter-Systemabhängigkeit im T0-Modell: SI- vs. natürliche Einheiten}
\hypersetup{
	colorlinks=true,
	linkcolor=blue,
	citecolor=blue,
	urlcolor=blue,
	pdftitle={Proof: The Fine Structure Constant α = 1 in Natural Units}
\hypersetup{
	colorlinks=true,
	linkcolor=blue,
	citecolor=blue,
	urlcolor=blue,
	pdftitle={Proof: The Koide Formula Implicitly Contains $\xi$}
\hypersetup{
	colorlinks=true,
	linkcolor=blue,
	citecolor=blue,
	urlcolor=blue,
	pdftitle={Pure Energy T0 Theory: Ratio-Based Physics with SI Reference}
\hypersetup{
	colorlinks=true,
	linkcolor=blue,
	citecolor=blue,
	urlcolor=blue,
	pdftitle={Quantum Mechanics in the T0 Model: Field-Theoretic Foundations}
\hypersetup{
	colorlinks=true,
	linkcolor=blue,
	citecolor=blue,
	urlcolor=blue,
	pdftitle={Ratio-Based vs. Absolute: The Role of Fractal Correction in T0 Theory}
\hypersetup{
	colorlinks=true,
	linkcolor=blue,
	citecolor=blue,
	urlcolor=blue,
	pdftitle={Reine Energie T0-Theorie: Verhältnis-basierte Physik mit SI-Referenz}
\hypersetup{
	colorlinks=true,
	linkcolor=blue,
	citecolor=blue,
	urlcolor=blue,
	pdftitle={Simple Lagrangian Revolution: From Standard Model Complexity to T0 Elegance}
\hypersetup{
	colorlinks=true,
	linkcolor=blue,
	citecolor=blue,
	urlcolor=blue,
	pdftitle={Simplified Dirac Equation in T0 Theory: Field Node Approach}
\hypersetup{
	colorlinks=true,
	linkcolor=blue,
	citecolor=blue,
	urlcolor=blue,
	pdftitle={Simplified T0 Theory: Elegant Lagrangian Density for Time-Mass Duality}
\hypersetup{
	colorlinks=true,
	linkcolor=blue,
	citecolor=blue,
	urlcolor=blue,
	pdftitle={T0 Cosmology: Redshift as a Geometric Path Effect in a Static Universe}
\hypersetup{
	colorlinks=true,
	linkcolor=blue,
	citecolor=blue,
	urlcolor=blue,
	pdftitle={T0 Deterministic Quantum Computing: Complete Analysis of Important Algorithms}
\hypersetup{
	colorlinks=true,
	linkcolor=blue,
	citecolor=blue,
	urlcolor=blue,
	pdftitle={T0 Deterministisches Quantencomputing: Vollständige Analyse wichtiger Algorithmen}
\hypersetup{
	colorlinks=true,
	linkcolor=blue,
	citecolor=blue,
	urlcolor=blue,
	pdftitle={T0 Model: Complete Framework - From Time-Energy Duality to Universal Constants}
\hypersetup{
	colorlinks=true,
	linkcolor=blue,
	citecolor=blue,
	urlcolor=blue,
	pdftitle={T0 Model: Complete Parameter-Free Particle Mass Calculation}
\hypersetup{
	colorlinks=true,
	linkcolor=blue,
	citecolor=blue,
	urlcolor=blue,
	pdftitle={T0 Model: Unified Neutrino Formula Structure}
\hypersetup{
	colorlinks=true,
	linkcolor=blue,
	citecolor=blue,
	urlcolor=blue,
	pdftitle={T0 Model: Universal Energy Relations for Mol and Candela Units}
\hypersetup{
	colorlinks=true,
	linkcolor=blue,
	citecolor=blue,
	urlcolor=blue,
	pdftitle={T0 Modell: Vollständiges Framework - Von Zeit-Energie-Dualität zu universellen Konstanten}
\hypersetup{
	colorlinks=true,
	linkcolor=blue,
	citecolor=blue,
	urlcolor=blue,
	pdftitle={T0 Quantenfeldtheorie: QFT, QM und Quantencomputer}
\hypersetup{
	colorlinks=true,
	linkcolor=blue,
	citecolor=blue,
	urlcolor=blue,
	pdftitle={T0 Quantum Field Theory: QFT, QM and Quantum Computers}
\hypersetup{
	colorlinks=true,
	linkcolor=blue,
	citecolor=blue,
	urlcolor=blue,
	pdftitle={T0 Theory vs Bell's Theorem: How Deterministic Energy Fields Circumvent No-Go Theorems}
\hypersetup{
	colorlinks=true,
	linkcolor=blue,
	citecolor=blue,
	urlcolor=blue,
	pdftitle={T0 Theory: Final Extension to Hadrons - Physically Derived Corrections}
\hypersetup{
	colorlinks=true,
	linkcolor=blue,
	citecolor=blue,
	urlcolor=blue,
	pdftitle={T0 Theory: The Fine-Structure Constant}
\hypersetup{
	colorlinks=true,
	linkcolor=blue,
	citecolor=blue,
	urlcolor=blue,
	pdftitle={T0 Theory: The Gravitational Constant}
\hypersetup{
	colorlinks=true,
	linkcolor=blue,
	citecolor=blue,
	urlcolor=blue,
	pdftitle={T0-Kosmologie: Rotverschiebung als geometrischer Pfad-Effekt im statischen Universum}
\hypersetup{
	colorlinks=true,
	linkcolor=blue,
	citecolor=blue,
	urlcolor=blue,
	pdftitle={T0-Model: Complete Document Analysis and Structured Summary}
\hypersetup{
	colorlinks=true,
	linkcolor=blue,
	citecolor=blue,
	urlcolor=blue,
	pdftitle={T0-Model: Kinetic Energy of Electrons and Photons}
\hypersetup{
	colorlinks=true,
	linkcolor=blue,
	citecolor=blue,
	urlcolor=blue,
	pdftitle={T0-Model: The Hubble Parameter in Static Universe}
\hypersetup{
	colorlinks=true,
	linkcolor=blue,
	citecolor=blue,
	urlcolor=blue,
	pdftitle={T0-Modell-Verifikation: Skalen-Verhältnis-basierte Berechnungen}
\hypersetup{
	colorlinks=true,
	linkcolor=blue,
	citecolor=blue,
	urlcolor=blue,
	pdftitle={T0-Modell: Bewegungsenergie von Elektronen und Photonen}
\hypersetup{
	colorlinks=true,
	linkcolor=blue,
	citecolor=blue,
	urlcolor=blue,
	pdftitle={T0-Modell: Die Hubble-Konstante im statischen Universum}
\hypersetup{
	colorlinks=true,
	linkcolor=blue,
	citecolor=blue,
	urlcolor=blue,
	pdftitle={T0-Modell: Einheitliche Neutrino-Formel-Struktur}
\hypersetup{
	colorlinks=true,
	linkcolor=blue,
	citecolor=blue,
	urlcolor=blue,
	pdftitle={T0-Modell: Universelle Energiebeziehungen für Mol- und Candela-Einheiten}
\hypersetup{
	colorlinks=true,
	linkcolor=blue,
	citecolor=blue,
	urlcolor=blue,
	pdftitle={T0-Modell: Vollständige Dokumentenanalyse und strukturierte Zusammenfassung}
\hypersetup{
	colorlinks=true,
	linkcolor=blue,
	citecolor=blue,
	urlcolor=blue,
	pdftitle={T0-Modell: Vollständige parameterfreie Teilchenmassen-Berechnung}
\hypersetup{
	colorlinks=true,
	linkcolor=blue,
	citecolor=blue,
	urlcolor=blue,
	pdftitle={T0-QAT: $\xi$-Aware Quantization-Aware Training}
\hypersetup{
	colorlinks=true,
	linkcolor=blue,
	citecolor=blue,
	urlcolor=blue,
	pdftitle={T0-QFT ML Addendum: Machine Learning Derived Extensions}
\hypersetup{
	colorlinks=true,
	linkcolor=blue,
	citecolor=blue,
	urlcolor=blue,
	pdftitle={T0-QFT ML-Addendum: Maschinelle Lern-abgeleitete Erweiterungen}
\hypersetup{
	colorlinks=true,
	linkcolor=blue,
	citecolor=blue,
	urlcolor=blue,
	pdftitle={T0-Theorie vs Bells Theorem: Wie deterministische Energiefelder No-Go-Theoreme umgehen}
\hypersetup{
	colorlinks=true,
	linkcolor=blue,
	citecolor=blue,
	urlcolor=blue,
	pdftitle={T0-Theorie: Der Terrell-Penrose-Effekt und Massenvariation}
\hypersetup{
	colorlinks=true,
	linkcolor=blue,
	citecolor=blue,
	urlcolor=blue,
	pdftitle={T0-Theorie: Die Feinstrukturkonstante}
\hypersetup{
	colorlinks=true,
	linkcolor=blue,
	citecolor=blue,
	urlcolor=blue,
	pdftitle={T0-Theorie: Die Gravitationskonstante}
\hypersetup{
	colorlinks=true,
	linkcolor=blue,
	citecolor=blue,
	urlcolor=blue,
	pdftitle={T0-Theorie: Die T0-Zeit-Masse-Dualität}
\hypersetup{
	colorlinks=true,
	linkcolor=blue,
	citecolor=blue,
	urlcolor=blue,
	pdftitle={T0-Theorie: Die sieben Rätsel}
\hypersetup{
	colorlinks=true,
	linkcolor=blue,
	citecolor=blue,
	urlcolor=blue,
	pdftitle={T0-Theorie: Erweiterung auf Bell-Tests – ML-Simulationen (November 2025)}
\hypersetup{
	colorlinks=true,
	linkcolor=blue,
	citecolor=blue,
	urlcolor=blue,
	pdftitle={T0-Theorie: Finale Erweiterung auf Hadronen - Physikalisch abgeleitete Korrekturen}
\hypersetup{
	colorlinks=true,
	linkcolor=blue,
	citecolor=blue,
	urlcolor=blue,
	pdftitle={T0-Theorie: Finale Fraktale Massenformeln (November 2025)}
\hypersetup{
	colorlinks=true,
	linkcolor=blue,
	citecolor=blue,
	urlcolor=blue,
	pdftitle={T0-Theorie: Fraktaldimension aus Lepton-Massenverhältnis}
\hypersetup{
	colorlinks=true,
	linkcolor=blue,
	citecolor=blue,
	urlcolor=blue,
	pdftitle={T0-Theorie: Fundamentale Prinzipien}
\hypersetup{
	colorlinks=true,
	linkcolor=blue,
	citecolor=blue,
	urlcolor=blue,
	pdftitle={T0-Theorie: Herleitung der Gravitationskonstanten}
\hypersetup{
	colorlinks=true,
	linkcolor=blue,
	citecolor=blue,
	urlcolor=blue,
	pdftitle={T0-Theorie: Kosmische Beziehungen und universelle $\xi$-Konstante}
\hypersetup{
	colorlinks=true,
	linkcolor=blue,
	citecolor=blue,
	urlcolor=blue,
	pdftitle={T0-Theorie: Kosmologie}
\hypersetup{
	colorlinks=true,
	linkcolor=blue,
	citecolor=blue,
	urlcolor=blue,
	pdftitle={T0-Theorie: Netzwerkdarstellung und Dimensionsanalyse in der T0-Theorie}
\hypersetup{
	colorlinks=true,
	linkcolor=blue,
	citecolor=blue,
	urlcolor=blue,
	pdftitle={T0-Theorie: Teilchenmassen}
\hypersetup{
	colorlinks=true,
	linkcolor=blue,
	citecolor=blue,
	urlcolor=blue,
	pdftitle={T0-Theorie: Vollstaendiger Abschluss}
\hypersetup{
	colorlinks=true,
	linkcolor=blue,
	citecolor=blue,
	urlcolor=blue,
	pdftitle={T0-Theory: Complete Closure}
\hypersetup{
	colorlinks=true,
	linkcolor=blue,
	citecolor=blue,
	urlcolor=blue,
	pdftitle={T0-Theory: Complete Derivation of All Parameters Without Circularity}
\hypersetup{
	colorlinks=true,
	linkcolor=blue,
	citecolor=blue,
	urlcolor=blue,
	pdftitle={T0-Theory: Cosmic Relations and universal $\xi$-constant}
\hypersetup{
	colorlinks=true,
	linkcolor=blue,
	citecolor=blue,
	urlcolor=blue,
	pdftitle={T0-Theory: Cosmology}
\hypersetup{
	colorlinks=true,
	linkcolor=blue,
	citecolor=blue,
	urlcolor=blue,
	pdftitle={T0-Theory: Derivation of the Gravitational Constant}
\hypersetup{
	colorlinks=true,
	linkcolor=blue,
	citecolor=blue,
	urlcolor=blue,
	pdftitle={T0-Theory: Extension to Bell Tests – ML Simulations (November 2025)}
\hypersetup{
	colorlinks=true,
	linkcolor=blue,
	citecolor=blue,
	urlcolor=blue,
	pdftitle={T0-Theory: Final Fractal Mass Formulas (November 2025)}
\hypersetup{
	colorlinks=true,
	linkcolor=blue,
	citecolor=blue,
	urlcolor=blue,
	pdftitle={T0-Theory: Fractal Dimension from Lepton Mass Ratio}
\hypersetup{
	colorlinks=true,
	linkcolor=blue,
	citecolor=blue,
	urlcolor=blue,
	pdftitle={T0-Theory: Fundamental Principles}
\hypersetup{
	colorlinks=true,
	linkcolor=blue,
	citecolor=blue,
	urlcolor=blue,
	pdftitle={T0-Theory: Mass Variation as an Equivalent to Time Dilation}
\hypersetup{
	colorlinks=true,
	linkcolor=blue,
	citecolor=blue,
	urlcolor=blue,
	pdftitle={T0-Theory: Network Representation and Dimensional Analysis in the T0-Theory}
\hypersetup{
	colorlinks=true,
	linkcolor=blue,
	citecolor=blue,
	urlcolor=blue,
	pdftitle={T0-Theory: Neutrinos}
\hypersetup{
	colorlinks=true,
	linkcolor=blue,
	citecolor=blue,
	urlcolor=blue,
	pdftitle={T0-Theory: Particle Masses}
\hypersetup{
	colorlinks=true,
	linkcolor=blue,
	citecolor=blue,
	urlcolor=blue,
	pdftitle={T0-Theory: The Seven Riddles}
\hypersetup{
	colorlinks=true,
	linkcolor=blue,
	citecolor=blue,
	urlcolor=blue,
	pdftitle={T0-Theory: The T0-Time-Mass Duality}
\hypersetup{
	colorlinks=true,
	linkcolor=blue,
	citecolor=blue,
	urlcolor=blue,
	pdftitle={Temperature Units in Natural Units: T0-Theory}
\hypersetup{
	colorlinks=true,
	linkcolor=blue,
	citecolor=blue,
	urlcolor=blue,
	pdftitle={Temperatureinheiten in nat\"urlichen Einheiten: T0-Theorie}
\hypersetup{
	colorlinks=true,
	linkcolor=blue,
	citecolor=blue,
	urlcolor=blue,
	pdftitle={The Electron Unit Charge in T0 Theory: Beyond Point Singularities}
\hypersetup{
	colorlinks=true,
	linkcolor=blue,
	citecolor=blue,
	urlcolor=blue,
	pdftitle={The Fine Structure Constant: Various Representations and Relationships}
\hypersetup{
	colorlinks=true,
	linkcolor=blue,
	citecolor=blue,
	urlcolor=blue,
	pdftitle={The Geometric Formalism of T0 Quantum Mechanics and its Application to Quantum Computing}
\hypersetup{
	colorlinks=true,
	linkcolor=blue,
	citecolor=blue,
	urlcolor=blue,
	pdftitle={The Mass Scaling Exponent κ in T0 Theory}
\hypersetup{
	colorlinks=true,
	linkcolor=blue,
	citecolor=blue,
	urlcolor=blue,
	pdftitle={The Musical Spiral and 137: The Mathematical Discovery of Cosmic Detuning}
\hypersetup{
	colorlinks=true,
	linkcolor=blue,
	citecolor=blue,
	urlcolor=blue,
	pdftitle={The Relational Number System: Prime Numbers as Fundamental Ratios}
\hypersetup{
	colorlinks=true,
	linkcolor=blue,
	citecolor=blue,
	urlcolor=blue,
	pdftitle={The T0 Model (Planck-Referenced): A Reformulation of Physics}
\hypersetup{
	colorlinks=true,
	linkcolor=blue,
	citecolor=blue,
	urlcolor=blue,
	pdftitle={The T0 Model: Time-Energy Duality and Geometric Rest Mass}
\hypersetup{
	colorlinks=true,
	linkcolor=blue,
	citecolor=blue,
	urlcolor=blue,
	pdftitle={The T0-Model (Planck-Referenced): A Reformulation of Physics}
\hypersetup{
	colorlinks=true,
	linkcolor=blue,
	citecolor=blue,
	urlcolor=blue,
	pdftitle={Verbindungen zwischen dem Mizohata-Takeuchi-Gegenbeispiel und der T0-Zeit-Masse-Dualitätstheorie}
\hypersetup{
	colorlinks=true,
	linkcolor=blue,
	citecolor=blue,
	urlcolor=blue,
	pdftitle={Vereinfachte Dirac-Gleichung in der T0-Theorie: Feldknoten-Ansatz}
\hypersetup{
	colorlinks=true,
	linkcolor=blue,
	citecolor=blue,
	urlcolor=blue,
	pdftitle={Vereinfachte T0-Theorie: Elegante Lagrange-Dichte für Zeit-Masse-Dualität}
\hypersetup{
	colorlinks=true,
	linkcolor=blue,
	citecolor=blue,
	urlcolor=blue,
	pdftitle={Verhältnisbasiert vs. Absolut: Die Rolle der fraktalen Korrektur in der T0-Theorie}
\hypersetup{
	colorlinks=true,
	linkcolor=blue,
	citecolor=blue,
	urlcolor=blue,
	pdftitle={Vollständige Herleitung der Higgs-Masse und Wilson-Koeffizienten}
\hypersetup{
	colorlinks=true,
	linkcolor=blue,
	citecolor=blue,
	urlcolor=blue,
	pdftitle={Vollständiges Teilchenspektrum: Standard-Modell vs T0-Theorie}
\hypersetup{
	colorlinks=true,
	linkcolor=blue,
	citecolor=blue,
	urlcolor=blue,
	pdftitle={Warum Zahlenverhältnisse nicht direkt gekürzt werden dürfen}
\hypersetup{
	colorlinks=true,
	linkcolor=blue,
	citecolor=blue,
	urlcolor=blue,
	pdftitle={Why Numerical Ratios Must Not Be Directly Simplified}
\hypersetup{
	colorlinks=true,
	linkcolor=blue,
	citecolor=blue,
	urlcolor=blue,
}
\hypersetup{
	colorlinks=true,
	linkcolor=blue,
	citecolor=red,
	urlcolor=blue,
	bookmarks=true,
	bookmarksnumbered=true,
	pdfstartview=FitH,
	pdftitle={T0 Model - Field-Theoretic Derivation of the Beta Parameter}
\hypersetup{
	colorlinks=true,
	linkcolor=blue,
	citecolor=red,
	urlcolor=blue,
	bookmarks=true,
	bookmarksnumbered=true,
	pdfstartview=FitH,
	pdftitle={T0-Modell - Feldtheoretische Herleitung des Beta-Parameters}
\hypersetup{
	colorlinks=true,
	linkcolor=blue,
	filecolor=magenta,
	urlcolor=cyan,
}
\hypersetup{
	colorlinks=true,
	linkcolor=blue,
	urlcolor=blue,
	citecolor=blue,
	pdftitle={From Time Dilation to Mass Variation: Mathematical Core Formulations of Time-Mass Duality Theory - Updated Framework}
\hypersetup{
	colorlinks=true,
	linkcolor=blue,
	urlcolor=blue,
	citecolor=blue,
	pdftitle={T0 Model: Detailed Formula for Leptonic Anomalies}
\hypersetup{
	colorlinks=true,
	linkcolor=blue,
	urlcolor=blue,
	citecolor=blue,
	pdftitle={T0 Model: Detaillierte Formel für leptonische Anomalien}
\hypersetup{
	colorlinks=true,
	linkcolor=blue,
	urlcolor=blue,
	citecolor=blue,
	pdftitle={T0 Model: Energy-based Formulas with Quadratic Scaling}
\hypersetup{
	colorlinks=true,
	linkcolor=blue,
	urlcolor=blue,
	citecolor=blue,
	pdftitle={T0 Model: Granulation, Limits and Fundamental Asymmetry}
\hypersetup{
	colorlinks=true,
	linkcolor=blue,
	urlcolor=blue,
	citecolor=blue,
	pdftitle={T0-Modell: Energiebasierte Formeln mit quadratischer Skalierung}
\hypersetup{
	colorlinks=true,
	linkcolor=blue,
	urlcolor=blue,
	citecolor=blue,
	pdftitle={T0-Modell: Granulation, Limits und fundamentale Asymmetrie}
\hypersetup{
	colorlinks=true,
	linkcolor=blue,
	urlcolor=blue,
	citecolor=blue,
	pdftitle={Von Zeitdilatation zu Massenvariation: Mathematische Kernformulierungen der Zeit-Masse-Dualitätstheorie - Aktualisiertes Framework}
\hypersetup{
	colorlinks=true,
	linkcolor=t0blue,
	citecolor=t0blue,
	urlcolor=t0blue,
	pdftitle={T0 Model: Complete Theoretical Summary}
\hypersetup{
	colorlinks=true,
	linkcolor=t0blue,
	citecolor=t0blue,
	urlcolor=t0blue,
	pdftitle={T0 Theory: Resolution of Apparent Instantaneity}
\hypersetup{
	colorlinks=true,
	linkcolor=t0blue,
	citecolor=t0blue,
	urlcolor=t0blue,
	pdftitle={T0 vs Synergetics: Vereinfachung durch natürliche Einheiten}
\hypersetup{
	colorlinks=true,
	linkcolor=t0blue,
	citecolor=t0blue,
	urlcolor=t0blue,
	pdftitle={T0-Modell: Vollständige theoretische Zusammenfassung}
\hypersetup{
	colorlinks=true,
	linkcolor=t0blue,
	citecolor=t0blue,
	urlcolor=t0blue,
	pdftitle={T0-Theorie: Auflösung der scheinbaren Instantanität}
\hypersetup{
	colorlinks=true,
	linkcolor=t0blue,
	citecolor=t0blue,
	urlcolor=t0blue,
	pdftitle={T0-Theorie: Vollständige Dokumentenübersicht}
\hypersetup{
	colorlinks=true,
	linkcolor=t0blue,
	citecolor=t0blue,
	urlcolor=t0blue,
	pdftitle={T0-Theory: Complete Document Overview}
\hypersetup{
	colorlinks=true,
	linkcolor=t0blue,
	citecolor=t0blue,
	urlcolor=t0blue,
}
\hypersetup{
	colorlinks=true,
	linkcolor=t0blue,
	citecolor=t0green,
	urlcolor=t0blue,
	pdftitle={Das verborgene Geheimnis von 1/137}
\hypersetup{
	colorlinks=true,
	linkcolor=t0blue,
	citecolor=t0green,
	urlcolor=t0blue,
	pdftitle={The Hidden Secret of 1/137}
\hypersetup{
    colorlinks=true,
    linkcolor=blue,
    citecolor=blue,
    urlcolor=blue,
    pdftitle={Analyse und Implikationen des MNRAS-Papiers 544 für die T0-Theorie}
\hypersetup{
  colorlinks=true,
  linkcolor=blue,
  citecolor=blue,
  urlcolor=blue
}
\hypersetup{
  colorlinks=true,
  linkcolor=blue,
  citecolor=blue,
  urlcolor=blue,
  pdftitle={T0-Theorie: Ein-Uhr-Metrologie und Drei-Uhren-Experiment}
\hypersetup{
  colorlinks=true,
  linkcolor=blue,
  citecolor=blue,
  urlcolor=blue,
  pdftitle={T0-Theory: Single-Clock Metrology and Three-Clock Experiment}
\hypersetup{
colorlinks=true,
linkcolor=blue,
citecolor=blue,
urlcolor=blue,
pdftitle={Quantenmechanik im T0-Modell: Feldtheoretische Grundlagen}
\hypersetup{
colorlinks=true,
linkcolor=blue,
citecolor=blue,
urlcolor=blue,
pdftitle={T0-Theory: Neutrinos}
\newcommand{\Bzero}{B_0}
\newcommand{\CQCD}{C_{\text{QCD}
\newcommand{\Cconv}{C_{\text{conv}
\newcommand{\Cto}{C_{\text{T0}
\newcommand{\Czero}{C_0}
\newcommand{\DTmu}{D_{T,\mu}
\newcommand{\DcovT}[1]{\partial_\mu #1 + #1 \partial_\mu \Tfield}
\newcommand{\Dfrak}{D_f}
\newcommand{\Df}{D_f}
\newcommand{\DhiggsT}{\Tfield (\partial_\mu + ig A_\mu) \Phi + \Phi \partial_\mu \Tfield}
\newcommand{\EPlanck}{E_P}
\newcommand{\EPlanck}{E_{\text{Pl}
\newcommand{\EPratio}[1]{\frac{#1}
\newcommand{\EP}{E_P}
\newcommand{\EP}{E_{\text{P}
\newcommand{\EW}{E_W}
\newcommand{\EZ}{E_Z}
\newcommand{\Echar}{E_{\text{char}
\newcommand{\Ee}{E_e}
\newcommand{\Efield}{E(x,t)}
\newcommand{\Efield}{E_\text{field}
\newcommand{\Efield}{E_{\text{Feld}
\newcommand{\Efield}{E_{\text{Field}
\newcommand{\Efield}{E_{\text{field}
\newcommand{\Efield}{E}
\newcommand{\Egamma}{E_\gamma}
\newcommand{\Eh}{E_h}
\newcommand{\Emu}{E_\mu}
\newcommand{\Enorm}[1]{E_{\text{norm}
\newcommand{\En}{E_n}
\newcommand{\Ep}{E_p}
\newcommand{\Eratio}[2]{\frac{E_{#1}
\newcommand{\Etau}{E_\tau}
\newcommand{\Evis}{E_{\text{vis}
\newcommand{\Exi}{E_\xi}
\newcommand{\Ezero}{E_0}
\newcommand{\GeV}{\,\text{GeV}
\newcommand{\Gnat}{G_{\text{nat}
\newcommand{\Gsi}{G_{\text{SI}
\newcommand{\Hubble}{H_0}
\newcommand{\Kfrak}{K_{\text{frac}
\newcommand{\Kfrak}{K_{\text{frak}
\newcommand{\Kspec}{K_{\text{spec}
\newcommand{\LCDM}{\Lambda\text{CDM}
\newcommand{\LPlanck}{\ell_{\text{Pl}
\newcommand{\Lag}{\mathcal{L}
\newcommand{\Lambdat}{\Lambda_T}
\newcommand{\Leff}{L_{\text{eff}
\newcommand{\Lorentz}[2]{{\Lambda^\mu{}
\newcommand{\Lp}{L_{\text{P}
\newcommand{\Lxi}{L_\xi}
\newcommand{\Lzero}{L_0}
\newcommand{\MPl}{M_{\text{Pl}
\newcommand{\MSbar}{\overline{\text{MS}
\newcommand{\MeV}{\,\text{MeV}
\newcommand{\Mpl}{M_{\text{Pl}
\newcommand{\OmegaDM}{\Omega_{\text{DM}
\newcommand{\OmegaLambda}{\Omega_{\Lambda}
\newcommand{\Omegab}{\Omega_b}
\newcommand{\Phiphoton}{\Phi_{\text{photon}
\newcommand{\Ricci}{R_{\mu\nu}
\newcommand{\Riem}{R^\rho{}
\newcommand{\Rzero}{R_\infty}
\newcommand{\Scal}{R}
\newcommand{\SynchPower}{P_{\text{synch}
\newcommand{\TPlanck}{t_{\text{Pl}
\newcommand{\Tfieldt}{T(\vec{x}
\newcommand{\Tfieldt}{T(x,t)}
\newcommand{\Tfield}{T(x)}
\newcommand{\Tfield}{T(x,t)}
\newcommand{\Tfield}{T_{\text{field}
\newcommand{\Tfield}{T}
\newcommand{\Tfield}{\mathcal{T}
\newcommand{\Tzerot}{T_0(\Tfield)}
\newcommand{\Tzero}{T_0}
\newcommand{\Weyl}{C^\rho{}
\newcommand{\ZPinch}{J \times B = \nabla p}
\newcommand{\aleph}{\aleph}
\newcommand{\alphaEMSI}{\alpha_{\text{EM,SI}
\newcommand{\alphaEMnat}{\alpha_{\text{EM,nat}
\newcommand{\alphaEM}{\alpha_{\text{EM}
\newcommand{\alphaEM}{\ensuremath{\alpha_{\text{EM}
\newcommand{\alphaQCD}{\alpha_s}
\newcommand{\alphaQED}{\alpha_{\text{QED}
\newcommand{\alphaSI}{\alpha_{\text{SI}
\newcommand{\alphaT}{\alpha_{\text{T}
\newcommand{\alphaWSI}{\alpha_{\text{W,SI}
\newcommand{\alphaWnat}{\alpha_{\text{W,nat}
\newcommand{\alphaW}{\alpha_{\text{W}
\newcommand{\alphaem}{\alpha_{EM}
\newcommand{\alphaem}{\alpha}
\newcommand{\alphafine}{\alpha}
\newcommand{\alphagem}{\alpha}
\newcommand{\alphanat}{\alpha_{\text{nat}
\newcommand{\alphapar}{\alpha}
\newcommand{\betaTSI}{\beta_{\text{T,SI}
\newcommand{\betaTnat}{\beta_{\text{T,nat}
\newcommand{\betaT}{\beta_T}
\newcommand{\betaT}{\beta_{T}
\newcommand{\betaT}{\beta_{\text{T}
\newcommand{\betaT}{\ensuremath{\beta_T}
\newcommand{\betapar}{\beta}
\newcommand{\calL}{\mathcal{L}
\newcommand{\checked}{\checkmark}
\newcommand{\checkmarkx}{\checkmark}
\newcommand{\dTdt}{\frac{d\Tfieldt}
\newcommand{\deltaE}{\delta E}
\newcommand{\deltafield}{\ensuremath{\delta m}
\newcommand{\deltam}{\delta m}
\newcommand{\deq}{\displaystyle}
\newcommand{\docref}[1]{\texttt{#1}
\newcommand{\eV}{\,\text{eV}
\newcommand{\epsilonT}{\varepsilon_T}
\newcommand{\epsilonzero}{\varepsilon_0}
\newcommand{\etavis}{\eta_{\text{visual}
\newcommand{\e}{\mathrm{e}
\newcommand{\gW}{g_W}
\newcommand{\gammaf}{\gamma_{\text{Lorentz}
\newcommand{\gammamu}{\gamma^\mu}
\newcommand{\gs}{g_s}
\newcommand{\inftytext}{$\infty$}
\newcommand{\interval}[2]{#1:#2}
\newcommand{\kfrac}{K_{\text{frak}
\newcommand{\lP}{\ell_{\text{P}
\newcommand{\lP}{l_P}
\newcommand{\lambdah}{\ensuremath{\lambda_h}
\newcommand{\lambdah}{\lambda_h}
\newcommand{\lambdazero}{\lambda_0}
\newcommand{\mP}{m_{\text{P}
\newcommand{\mfield}{m(x,t)}
\newcommand{\mfield}{m}
\newcommand{\mh}{m_h}
\newcommand{\micrometer}{\ensuremath{\mu}
\newcommand{\mikrometer}{\ensuremath{\mu}
\newcommand{\myRightarrow}{\ensuremath{\Rightarrow}
\newcommand{\myapprox}{\ensuremath{\approx}
\newcommand{\myomega}{\ensuremath{\omega}
\newcommand{\myphi}{\ensuremath{\phi}
\newcommand{\mypi}{\ensuremath{\pi}
\newcommand{\mypropto}{\ensuremath{\propto}
\newcommand{\myrightarrow}{\ensuremath{\rightarrow}
\newcommand{\mysim}{\ensuremath{\sim}
\newcommand{\mysqrt}{\ensuremath{\sqrt}
\newcommand{\mytimes}{\ensuremath{\times}
\newcommand{\natunits}{\hbar = c = G = k_B = 1}
\newcommand{\natunits}{\text{(nat. Einh.)}
\newcommand{\natunits}{\text{(nat. units)}
\newcommand{\nulep}{\nu}
\newcommand{\nuzero}{\nu_0}
\newcommand{\partialop}{\ensuremath{\partial}
\newcommand{\pdTdt}{\frac{\partial\Tfieldt}
\newcommand{\pdTdx}{\nabla\Tfieldt}
\newcommand{\phiT}{\phi}
\newcommand{\pichar}{\pi}
\newcommand{\primrel}[1]{\mathbf{#1}
\newcommand{\rhoCMB}{\rho_{\text{CMB}
\newcommand{\rhoCasimir}{\rho_{\text{Casimir}
\newcommand{\rhoE}{\rho_E}
\newcommand{\rhofield}{\ensuremath{\rho}
\newcommand{\rzero}{r_0}
\newcommand{\slashk}{\cancel{k}
\newcommand{\slashp}{\cancel{p}
\newcommand{\slashq}{\cancel{q}
\newcommand{\tP}{t_P}
\newcommand{\tP}{t_{\text{P}
\newcommand{\tablescale}{0.9}
\newcommand{\tzero}{t_0}
\newcommand{\vect}[1]{\boldsymbol{#1}
\newcommand{\vecx}{\vec{x}
\newcommand{\vh}{v}
\newcommand{\vr}{\vec{r}
\newcommand{\warningx}{\color{red}
\newcommand{\warningx}{\textbf{!}
\newcommand{\warningx}{{\color{red}
\newcommand{\xiT}{\xi}
\newcommand{\xiconst}{\xi = \frac{4}
\newcommand{\xicoupling}{f(E/\Exi)}
\newcommand{\xigeom}{\xi_{\text{geom}
\newcommand{\xigeom}{\xi}
\newcommand{\xikonst}{\xi = \frac{4}
\newcommand{\xiparticle}{\xi_{\text{particle}
\newcommand{\xipar}{\ensuremath{\xi}
\newcommand{\xipar}{\xi_0}
\newcommand{\xipar}{\xi}
\newcommand{\xirat}{\xi_{\text{ratio}
\newtheorem{axiom}{Axiom}
\newtheorem{category}{Category-Theoretic Basis}
\newtheorem{category}{Kategorientheoretische Basis}
\newtheorem{corollary}[theorem]{Corollary}
\newtheorem{corollary}[theorem]{Korollar}
\newtheorem{corollary}{Corollary}
\newtheorem{corollary}{Korollar}
\newtheorem{definition}[theorem]{Definition}
\newtheorem{definition}{Definition}
\newtheorem{discovery}{Discovery}
\newtheorem{discovery}{Neue Entdeckung}
\newtheorem{discovery}{New Discovery}
\newtheorem{discovery}{Revolutionary Discovery}
\newtheorem{entdeckung}{Entdeckung}
\newtheorem{entdeckung}{Revolutionäre Entdeckung}
\newtheorem{erkenntnis}{Erkenntnis}
\newtheorem{erkenntnis}{Schlüsselerkenntnis}
\newtheorem{example}[theorem]{Beispiel}
\newtheorem{example}[theorem]{Example}
\newtheorem{example}{Beispiel}
\newtheorem{example}{Example}
\newtheorem{insight}{Central Insight}
\newtheorem{insight}{Insight}
\newtheorem{insight}{Key Insight}
\newtheorem{insight}{Wichtige Einsicht}
\newtheorem{insight}{Zentrale Einsicht}
\newtheorem{lemma}[theorem]{Lemma}
\newtheorem{lemma}{Lemma}
\newtheorem{principle}{Fundamental Principle}
\newtheorem{principle}{Fundamentales Prinzip}
\newtheorem{principle}{Grundlegendes Prinzip}
\newtheorem{principle}{Principle}
\newtheorem{principle}{Prinzip}
\newtheorem{prinzip}{Grundprinzip}
\newtheorem{proof_step}{Beweisschritt}
\newtheorem{proof_step}{Proof Step}
\newtheorem{proposition}[theorem]{Proposition}
\newtheorem{proposition}{Proposition}
\newtheorem{remark}[theorem]{Bemerkung}
\newtheorem{remark}[theorem]{Remark}
\newtheorem{theorem}{Theorem}
\newtheorem{warning}[theorem]{Warning}
\newtheorem{warning}[theorem]{Warnung}
\newunicodechar{±}{\ensuremath{\pm}
\newunicodechar{×}{\ensuremath{\times}
\newunicodechar{÷}{\ensuremath{\div}
\newunicodechar{ħ}{\ensuremath{\hbar}
\newunicodechar{Α}{\ensuremath{A}
\newunicodechar{Β}{\ensuremath{B}
\newunicodechar{Γ}{\ensuremath{\Gamma}
\newunicodechar{Δ}{\ensuremath{\Delta}
\newunicodechar{Ε}{\ensuremath{E}
\newunicodechar{Ζ}{\ensuremath{Z}
\newunicodechar{Η}{\ensuremath{H}
\newunicodechar{Θ}{\ensuremath{\Theta}
\newunicodechar{Ι}{\ensuremath{I}
\newunicodechar{Κ}{\ensuremath{K}
\newunicodechar{Λ}{\ensuremath{\Lambda}
\newunicodechar{Μ}{\ensuremath{M}
\newunicodechar{Ν}{\ensuremath{N}
\newunicodechar{Ξ}{\ensuremath{\Xi}
\newunicodechar{Ο}{\ensuremath{O}
\newunicodechar{Π}{\ensuremath{\Pi}
\newunicodechar{Ρ}{\ensuremath{P}
\newunicodechar{Σ}{\ensuremath{\Sigma}
\newunicodechar{Τ}{\ensuremath{T}
\newunicodechar{Υ}{\ensuremath{\Upsilon}
\newunicodechar{Φ}{\ensuremath{\Phi}
\newunicodechar{Χ}{\ensuremath{X}
\newunicodechar{Ψ}{\ensuremath{\Psi}
\newunicodechar{Ω}{\ensuremath{\Omega}
\newunicodechar{α}{\ensuremath{\alpha}
\newunicodechar{β}{\ensuremath{\beta}
\newunicodechar{γ}{\ensuremath{\gamma}
\newunicodechar{δ}{\ensuremath{\delta}
\newunicodechar{ε}{\ensuremath{\varepsilon}
\newunicodechar{ζ}{\ensuremath{\zeta}
\newunicodechar{η}{\ensuremath{\eta}
\newunicodechar{θ}{\ensuremath{\theta}
\newunicodechar{ι}{\ensuremath{\iota}
\newunicodechar{κ}{\ensuremath{\kappa}
\newunicodechar{λ}{\ensuremath{\lambda}
\newunicodechar{μ}{\ensuremath{\mu}
\newunicodechar{ν}{\ensuremath{\nu}
\newunicodechar{ξ}{\ensuremath{\xi}
\newunicodechar{ο}{\ensuremath{o}
\newunicodechar{π}{\ensuremath{\pi}
\newunicodechar{ρ}{\ensuremath{\rho}
\newunicodechar{σ}{\ensuremath{\sigma}
\newunicodechar{τ}{\ensuremath{\tau}
\newunicodechar{υ}{\ensuremath{\upsilon}
\newunicodechar{φ}{\ensuremath{\phi}
\newunicodechar{φ}{\ensuremath{\varphi}
\newunicodechar{χ}{\ensuremath{\chi}
\newunicodechar{ψ}{\ensuremath{\psi}
\newunicodechar{ω}{\ensuremath{\omega}
\newunicodechar{←}{\ensuremath{\leftarrow}
\newunicodechar{→}{\ensuremath{\rightarrow}
\newunicodechar{↔}{\ensuremath{\leftrightarrow}
\newunicodechar{⇐}{\ensuremath{\Leftarrow}
\newunicodechar{⇒}{\ensuremath{\Rightarrow}
\newunicodechar{⇔}{\ensuremath{\Leftrightarrow}
\newunicodechar{∂}{\ensuremath{\partial}
\newunicodechar{∅}{\ensuremath{\emptyset}
\newunicodechar{∇}{\ensuremath{\nabla}
\newunicodechar{∈}{\ensuremath{\in}
\newunicodechar{∉}{\ensuremath{\notin}
\newunicodechar{∏}{\ensuremath{\prod}
\newunicodechar{∑}{\ensuremath{\sum}
\newunicodechar{√}{\ensuremath{\sqrt}
\newunicodechar{∝}{\ensuremath{\propto}
\newunicodechar{∞}{\ensuremath{\infty}
\newunicodechar{∩}{\ensuremath{\cap}
\newunicodechar{∪}{\ensuremath{\cup}
\newunicodechar{∫}{\ensuremath{\int}
\newunicodechar{≈}{\ensuremath{\approx}
\newunicodechar{≠}{\ensuremath{\neq}
\newunicodechar{≤}{\ensuremath{\leq}
\newunicodechar{≥}{\ensuremath{\geq}
\newunicodechar{★}{\ensuremath{\star}
\newunicodechar{✓}{\checkmark}
\pgfplotsset{compat=1.17}
\pgfplotsset{compat=1.18}
\renewcommand{\cftchapfont}{\large\bfseries\color{blue}
\renewcommand{\cftchappagefont}{\large\bfseries\color{blue}
\renewcommand{\cftsecfont}{\bfseries}
\renewcommand{\cftsecfont}{\color{blue}
\renewcommand{\cftsecfont}{\large\bfseries\color{blue}
\renewcommand{\cftsecpagefont}{\bfseries}
\renewcommand{\cftsecpagefont}{\color{blue}
\renewcommand{\cftsecpagefont}{\large\bfseries\color{blue}
\renewcommand{\cftsubsecfont}{\color{blue!80!black}
\renewcommand{\cftsubsecfont}{\color{blue}
\renewcommand{\cftsubsecpagefont}{\color{blue!80!black}
\renewcommand{\cftsubsecpagefont}{\color{blue}
\renewcommand{\cftsubsubsecfont}{\color{blue!60!black}
\renewcommand{\cftsubsubsecfont}{\color{blue}
\renewcommand{\cftsubsubsecpagefont}{\color{blue!60!black}
\renewcommand{\cftsubsubsecpagefont}{\color{blue}
\renewcommand{\cfttoctitlefont}{\huge\bfseries\color{blue}
\renewcommand{\cfttoctitlefont}{\huge\bfseries}
\renewcommand{\familydefault}{\sfdefault}
\renewcommand{\footrulewidth}{0.4pt}
\renewcommand{\headrulewidth}{0.4pt}
\sisetup{locale = DE, group-separator = {.}
\sisetup{locale = DE}
\usetikzlibrary{arrows.meta,positioning,shapes.geometric}
\usetikzlibrary{decorations.pathmorphing, patterns, shapes.arrows}
\usetikzlibrary{intersections}
\usetikzlibrary{positioning, arrows.meta}
\usetikzlibrary{positioning, arrows}
\usetikzlibrary{positioning, shapes.geometric, arrows.meta}
\usetikzlibrary{positioning,shapes,arrows}

% Common settings
\setlength{\headheight}{15pt}
\pgfplotsset{compat=1.18}
\usetikzlibrary{positioning,shapes,arrows,arrows.meta}

% Hyperref setup
\hypersetup{
    colorlinks=true,
    linkcolor=blue,
    citecolor=blue,
    urlcolor=blue
}


\title{ParameterSystemdipendentEn}
\author{Johann Pascher}
\date{\today}

\begin{document}

\maketitle
\tableofcontents

\title{Parameter System-Dependency in T0-Model: \\
		SI vs. Natural Units and the Danger \\
		of Direct Transfer of Formula Symbols}
	\author{Johann Pascher\\
		Department of Communications Engineering, \\H{\"o}here Technische Bundeslehranstalt (HTL), Leonding, Austria\\
		\texttt{johann.pascher@gmail.com}}
	\date{\today}
	
	\maketitle
	
	\begin{abstract}
		This paper systematically analyzes the parameter dependency between SI units and T0-model natural units, revealing that fundamental parameters like $\xipar$, $\alpha_{\text{EM}}$, $\beta_{\text{T}}$, and Yukawa couplings have dramatically different numerical values in different unit systems. Through detailed calculations, we demonstrate that direct transfer of parameter values between systems leads to errors spanning multiple orders of magnitude. The analysis extends beyond specific parameters to establish universal transformation rules and provides critical warnings against naive parameter transfer. This work establishes that the apparent inconsistencies in T0-model parameters are actually systematic unit-system dependencies that require careful transformation protocols for experimental verification.
	\end{abstract}
	
	\tableofcontents
	\newpage
	
	# Introduction
	\label{sec:introduction}
	
	## The Parameter Transfer Problem
	\label{subsec:parameter_problem}
	
	The T0 model, formulated in natural units where $\hbar = c = G = k_B = \alpha_{\text{EM}} = \alpha_{\text{W}} = \beta_{\text{T}} = 1$, presents a fundamental challenge when compared with experimental data expressed in SI units. This paper demonstrates that the apparent inconsistencies between T0-model predictions and experimental observations are not physical contradictions but systematic unit-system dependencies.
	
	The core insight is that parameters such as $\xipar$, $\alpha_{\text{EM}}$, and $\beta_{\text{T}}$ represent fundamentally different quantities when expressed in different unit systems:
	
	$$\xipar_{\text{SI}} \neq \xipar_{\text{nat}}, \quad \alphaEMSI \neq \alphaEMnat, \quad \betaTSI \neq \betaTnat$$
	
	## Scope and Methodology
	\label{subsec:scope}
	
	This analysis covers:
	
		- Systematic calculation of parameter ratios between SI and T0-natural units
		- Demonstration of transformation invariance for dimensionless ratios
		- Extension to variable parameters like $\xipar$ and Yukawa couplings
		- Universal warnings against direct parameter transfer
		- Guidelines for correct experimental comparison protocols
	
	
	# The $\xipar$ Parameter: Variable Across Mass Scales
	\label{sec:xi_parameter}
	# The Universal $\xi$-Field Framework
	
	The cornerstone of the T0-model is the universal geometric constant that serves as the fundamental parameter for all physical calculations.
	

		The universal geometric constant:
		
```math-equation

			\xi = \frac{4}{3} \times 10^{-4} = 1.3333... \times 10^{-4}
		
```

	
	This dimensionless constant is used throughout T0 theory to connect quantum mechanical and gravitational phenomena. It establishes the characteristic strength of field interactions and provides the foundation for unified field descriptions.
	

		For the detailed derivation and physical justification of this parameter, see the document "Parameter Derivation" (available at: \url{https://github.com/jpascher/T0-Time-Mass-Duality/2/pdf/parameterherleitung_En.pdf}).

	
	This geometric constant determines a characteristic energy scale for the $\xi$-field:
	
	
```math-equation

		E_\xi = \frac{1}{\xi} = \frac{3}{4 \times 10^{-4}} = 7500 \text{ (natural units)}
	
```

	## Definition and Physical Meaning
	\label{subsec:xi_definition}
	
	The parameter $\xipar$ is also the ratio of the Schwarzschild radius to the Planck length:
	
	
```math-equation

		\xipar = \frac{r_0}{\lP} = \frac{2Gm}{\lP}
		\label{eq:xi_definition}
	
```

	
\textbf{Crucial:} The parameter $\xipar$ scales with the mass of the object under consideration according to $\xipar(m) = 2Gm/\lP$. The Higgs mass defines the fundamental reference scale $\xipar_0 = 1.33 \times 10^{-4}$, to which all other masses are normalized in the T0 model.
	
	## Connection to Higgs Physics
	\label{subsec:xi_higgs_connection}
	
	The T0 model establishes a fundamental connection between $\xipar$ and Higgs sector physics through the relationship derived in the complete field-theoretic framework 
	
	
```math-equation

		\xipar = \frac{\lambdah^2 v^2}{16\pichar^3 m_h^2} \approx 1.33 \times 10^{-4}
		\label{eq:xi_higgs_fundamental}
	
```

	
	where:
	
		- $\lambdah \approx 0.13$ (Higgs self-coupling)
		- $v \approx 246$ GeV (Higgs VEV)
		- $m_h \approx 125$ GeV (Higgs mass)
	
	
	This represents the universal scale parameter that emerges from fundamental Standard Model physics, while the mass-dependent form $\xipar = 2Gm/\lP$ applies to specific objects.
	
	## $\xipar$ Values in the SI System
	\label{subsec:xi_si_values}
	
	Using SI constants:
	
```math-align

		G &= 6.674 \times 10^{-11} \text{ m}^3/(\text{kg} \cdot \text{s}^2) \\
		\lP &= 1.616 \times 10^{-35} \text{ m}
	
```

	
	We calculate $\xipar_{\text{SI}}$ for various objects:
	
	\begin{table}[htbp]
		\centering
		\begin{tabular}{lcc}
			\toprule
			\textbf{Object} & \textbf{Mass} & \textbf{$\xipar_{\text{SI}}$} \\
			\midrule
			Electron & $9.109 \times 10^{-31}$ kg & $7.52 \times 10^{-7}$ \\
			Proton & $1.673 \times 10^{-27}$ kg & $1.38 \times 10^{-3}$ \\
			Human (70 kg) & $7.0 \times 10^{1}$ kg & $6.4 \times 10^{6}$ \\
			Earth & $5.972 \times 10^{24}$ kg & $4.1 \times 10^{28}$ \\
			Sun & $1.989 \times 10^{30}$ kg & $1.8 \times 10^{38}$ \\
			Planck mass & $2.176 \times 10^{-8}$ kg & $2.0$ \\
			\bottomrule
		\end{tabular}
		\caption{$\xipar$ values for different objects in SI units}
		\label{tab:xi_si_values}
	\end{table}
	
	\textbf{The parameter $\xipar$ varies over 46 orders of magnitude!}
	
	## $\xipar$ Transformation to T0-Natural Units
	\label{subsec:xi_transformation}
	
	Based on the comprehensive transformation analysis, the conversion factor between systems is approximately:
	
	$$\frac{\xipar_{\text{nat}}}{\xipar_{\text{SI}}} \approx 4100$$
	
	This gives T0-natural unit values:
	
	\begin{table}[htbp]
		\centering
		\begin{tabular}{lcc}
			\toprule
			\textbf{Object} & \textbf{$\xipar_{\text{SI}}$} & \textbf{$\xipar_{\text{nat}}$} \\
			\midrule
			Electron & $7.52 \times 10^{-7}$ & $3.1 \times 10^{-3}$ \\
			Proton & $1.38 \times 10^{-3}$ & $5.7$ \\
			Human (70 kg) & $6.4 \times 10^{6}$ & $2.6 \times 10^{10}$ \\
			Sun & $1.8 \times 10^{38}$ & $7.4 \times 10^{41}$ \\
			\bottomrule
		\end{tabular}
		\caption{$\xipar$ transformation between unit systems}
		\label{tab:xi_transformation}
	\end{table}
	
	## Invariance of Ratios
	\label{subsec:xi_ratio_invariance}
	
	\textbf{Critical verification:} The ratios between different objects remain identical in both systems:
	
	
```math-align

		\frac{\xipar_{\text{Sun},\text{SI}}}{\xipar_{\text{e},\text{SI}}} &= \frac{1.8 \times 10^{38}}{7.52 \times 10^{-7}} = 2.4 \times 10^{44} \\
		\frac{\xipar_{\text{Sun},\text{nat}}}{\xipar_{\text{e},\text{nat}}} &= \frac{7.4 \times 10^{41}}{3.1 \times 10^{-3}} = 2.4 \times 10^{44}
	
```

	
	\boxed{\text{Ratios are invariant under system transformation!}}
	
\chapter{The Fine-Structure Constant $\alpha_{\text{EM}$}}
\label{sec:alpha_em}

\section{The Mystification of 1/137}
\label{subsec:alpha_mystification}

The fine-structure constant $\alpha_{\text{EM}} \approx 1/137$ has been declared one of the greatest mysteries of physics by prominent physicists:

	- \textbf{Richard Feynman}: ``It is one of the greatest damn mysteries of physics: a magic number that comes to us with no understanding whatsoever.''
	- \textbf{Wolfgang Pauli}: ``When I die, I will ask God two questions: Why relativity? And why 137? I believe he will have an answer for the first one.''
	- \textbf{Max Born}: ``If $\alpha$ were larger, molecules could not exist, and there would be no life.''

\section{Electromagnetic Duality as the Key}
\label{subsec:electromagnetic_duality}

What all these statements overlook: The fine-structure constant possesses two mathematically equivalent representations that reveal its true nature:

```math-align

	\alpha_{\text{EM}} &= \frac{e^2}{4\pi\varepsilon_0\hbar c} \quad \text{(Standard form)} \label{eq:alpha_standard}\\
	\alpha_{\text{EM}} &= \frac{e^2 \mu_0 c}{4\pi \hbar} \quad \text{(Dual form)} \label{eq:alpha_dual}

```

This equivalence is based on the Maxwell relation $c^2 = \frac{1}{\varepsilon_0\mu_0}$ and reveals a fundamental electromagnetic duality:

```math-equation

	\frac{1}{\varepsilon_0 c} = \mu_0 c
	\label{eq:em_duality}

```

\section{The Dual Nature of $\alpha$: System-Dependent yet Invariant}
\label{subsec:double_nature}

The fine-structure constant possesses a remarkable dual nature:

\subsection{As an Invariant Ratio of Physical Quantities}
\label{subsubsec:invariant_ratio}

Regardless of the chosen system of units, $\alpha$ remains constant as a \textbf{ratio} of fundamental lengths:

```math-equation

	\alpha_{\text{EM}} = \frac{r_e}{\lambda_C} = \frac{\text{Classical electron radius}}{\text{Compton wavelength}}
	\label{eq:alpha_ratio_re}

```

Similarly, the inverse ratio:

```math-equation

	\alpha_{\text{EM}}^{-1} = \frac{a_0}{\lambda_C/2\pi} = \frac{\text{Bohr radius}}{\text{Reduced Compton wavelength}} = 137.036...
	\label{eq:alpha_ratio_bohr}

```

These ratios are \textbf{system-of-units invariant} -- they have the same numerical value in any consistent system of units, as the units cancel out in the ratio.

\subsection{As a System-Dependent Numerical Value}
\label{subsubsec:system_dependent}

Simultaneously, the numerical value of $\alpha$ depends on the choice of fundamental units:

	- \textbf{SI system}: $\alpha = \frac{e^2}{4\pi\varepsilon_0\hbar c} \approx 1/137$
	- \textbf{Natural units}: $\alpha = 1$ (by suitable choice)
	- \textbf{Gaussian units}: $\alpha = \frac{e^2}{\hbar c} \approx 1/137$

\section{The System Dependency of $\alpha$}
\label{subsec:alpha_system_dependency}

The numerical value $\alpha_{\text{EM}} = 1/137$ is \textbf{valid exclusively in the SI system}:

```math-align

	\text{SI system:} \quad &\alpha_{\text{EM}}^{\text{SI}} = \frac{e^2}{4\pi\varepsilon_0\hbar c} \approx \frac{1}{137.036} \\
	\text{Natural system of units:} \quad &\alpha_{\text{EM}}^{\text{nat}} = 1 \text{ (by suitable choice of units)}

```

\textbf{Transformation factor:}

```math-equation

	\frac{\alpha_{\text{EM}}^{\text{nat}}}{\alpha_{\text{EM}}^{\text{SI}}} = 137.036

```

\section{The Natural System of Units with $\alpha = 1$}
\label{subsec:natural_units}

In a natural system of units that respects electromagnetic duality, we obtain:

	- $\hbar_{\text{nat}} = 1$ (quantum mechanical scale)
	- $c_{\text{nat}} = 1$ (relativistic scale)
	- $\varepsilon_{0,\text{nat}} = 1$ (electric constant)
	- $\mu_{0,\text{nat}} = 1$ (magnetic constant)
	- $e_{\text{nat}}^2 = 4\pi$ (elementary charge)

With these values, $\alpha = 1$ is verified in both the standard form and the dual form:

```math-equation

	\alpha = \frac{4\pi}{4\pi \cdot 1 \cdot 1 \cdot 1} = 1

```

\section{The Resolution of the ``Mystery''}
\label{subsec:mystery_resolution}

The apparent mystification of $1/137$ arises from:

	- \textbf{Confusion of two aspects}: The invariance of the ratios is conflated with the system-dependency of the numerical representation.
	
	- \textbf{Treatment of the SI system as absolute}: The historically evolved SI units (meter, second, kilogram, ampere) force electromagnetic constants to take ``unnatural'' values.
	
	- \textbf{Forgetting the construction of unit systems}: All unit systems are human constructs. Nature knows no preferred units.
	
	- \textbf{Search for deeper meaning in conversion factors}: The number 137 has no deeper cosmic significance than, say, the factor 1609.344 between miles and meters.

\section{The Anthropic Fallacy}
\label{subsec:anthropic_fallacy}

Typical anthropic arguments claim:

	- ``If $\alpha_{\text{EM}} = 1/200$ $\rightarrow$ no atoms $\rightarrow$ no life''
	- ``If $\alpha_{\text{EM}} = 1/80$ $\rightarrow$ no stars $\rightarrow$ no life''
	- ``Therefore, $\alpha_{\text{EM}} = 1/137$ is `fine-tuned' for life''

\textbf{The problem}: These arguments presuppose the SI system as absolute!

\textbf{In natural units}: $\alpha_{\text{EM}} = 1$ is perfectly natural and requires no fine-tuning whatsoever. The electromagnetic interaction has unit strength in the natural system of units, which respects the fundamental structure of quantum mechanics and relativity.

\section{Sommerfeld's Harmonic Imprinting}
\label{subsec:sommerfeld_harmonic}

An often overlooked historical aspect: In 1916, Arnold Sommerfeld actively searched for \textbf{harmonic ratios} in atomic spectra, guided by the philosophical conviction that nature follows musical principles.

His methodological approach:

	- \textbf{Expectation} of musical ratios in quantum transitions
	- \textbf{Calibration} of measurement systems to produce harmonic values
	- \textbf{Definition} of $\alpha_{\text{EM}}$ based on harmonic spectroscopic adjustments
	- \textbf{Attribution} of the resulting ratio to fundamental physics

The apparent ``harmony'' in $\alpha_{\text{EM}}^{-1} = 137 \approx (6/5)^{27}$ is therefore not a cosmic discovery, but the result of Sommerfeld's harmonic expectations embedded into the definition of the unit system.

\section{Physical Interpretation}
\label{subsec:physical_interpretation}

In natural units, $\alpha = 1$ represents the perfect balance between:

	- \textbf{Electric field coupling} (via $\varepsilon_0$ with $c^{-1}$)
	- \textbf{Magnetic field coupling} (via $\mu_0$ with $c^{+1}$)
	- \textbf{Quantum mechanical scale} (via $\hbar$)
	- \textbf{Relativistic scale} (via $c$)

The electromagnetic duality $\frac{1}{\varepsilon_0 c} = \mu_0 c$ ensures this perfect balance.

\section{Summary: The True Lesson}
\label{subsec:true_lesson}

The fine-structure constant teaches us a profound lesson about the nature of physical laws:

\textbf{The fundamental relationships of the universe are elegant and simple when expressed in their natural language.}

The apparent complexity and mystery of ``1/137'' are merely artifacts of our historical decision to measure electromagnetic phenomena with units originally defined for mechanical quantities.

The ``fine-tuning problem'' completely dissolves once we recognize:

	- $\alpha = 1/137$ is not a fundamental number, but a unit conversion factor
	- $\alpha = 1$ represents the natural strength of the electromagnetic coupling
	- The apparent ``mystery'' arises from treating arbitrary SI units as absolute
	- The fundamental relationships of nature are simple in their natural language

\section{Historical Warning: The Eddington Saga}
\label{subsec:eddington_warning}

Arthur Eddington (1882-1944) attempted to ``prove'' $\alpha_{\text{EM}} = 1/137$ from first principles and developed elaborate numerological theories. The result was entirely speculative and wrong -- a warning against mystifying system-dependent numbers.

However, modern analysis shows that the fine-structure constant is indeed derivable from fundamental electromagnetic vacuum constants and that $\alpha_{\text{EM}} = 1$ in natural units is not only possible but reveals the arbitrary nature of our choice of unit system.
	# The $\beta_{\text{T}$ Parameter}
	\label{sec:beta_t}
	
	## Empirical vs. Theoretical Values
	\label{subsec:beta_empirical_theoretical}
	
	The $\beta_{\text{T}}$ parameter shows the same system dependency:
	
	
```math-align

		\betaTSI &\approx 0.008 \text{ (from astrophysical observations)} \\
		\betaTnat &= 1 \text{ (in T0-natural units)}
	
```

	
	\textbf{Transformation factor:}
	$$\frac{\betaTnat}{\betaTSI} = \frac{1}{0.008} = 125$$
	
	## Theoretical Foundation from Field Theory
	\label{subsec:beta_field_theory}
	
	The T0 model establishes $\beta_{\text{T}} = 1$ through the fundamental field-theoretic relationship \cite{pascher_derivation_beta_2025}:
	
	
```math-equation

		\beta_{\text{T}} = \frac{\lambdah^2 v^2}{16\pichar^3 m_h^2 \xipar} = 1
		\label{eq:beta_t_field_theory}
	
```

	
	This relationship, combined with the Higgs-derived value of $\xipar$, uniquely determines $\beta_{\text{T}} = 1$ in natural units, eliminating any free parameters from the theory.
	
	## Circularity in SI Determination
	\label{subsec:beta_circularity}
	
	The SI value $\betaTSI$ is determined through:
	$$z(\lambda) = z_0\left(1 + \beta_{\text{T}} \ln\frac{\lambda}{\lambda_0}\right)$$
	
	But this involves:
	
		- Hubble constant $H_0$ $\rightarrow$ distance measurements
		- Distance ladder $\rightarrow$ standard candles
		- Photometry $\rightarrow$ Planck radiation law $\rightarrow$ fundamental constants
	
	
	\textbf{The determination is circular through cosmological parameters!}
	
	# The Wien Constant $\alpha_{\text{W}$}
	\label{sec:alpha_w}
	
	## Mathematical vs. Conventional Values
	\label{subsec:wien_values}
	
	Wien's displacement law gives:
	
	
```math-align

		\text{SI system:} \quad &\alphaWSI = 2.8977719... \\
		\text{T0 system:} \quad &\alphaWnat = 1
	
```

	
	\textbf{Transformation factor:}
	$$\frac{\alphaWSI}{\alphaWnat} = 2.898$$
	
	# Parameter Comparison Table
	\label{sec:parameter_comparison}
	
	\begin{table}[htbp]
		\centering
		\begin{tabular}{lcccc}
			\toprule
			\textbf{Parameter} & \textbf{SI Value} & \textbf{T0-nat Value} & \textbf{Ratio} & \textbf{Factor} \\
			\midrule
			$\xipar$ (electron) & $7.5 \times 10^{-6}$ & $3.1 \times 10^{-2}$ & 4100 & $10^{3.6}$ \\
			$\alpha_{\text{EM}}$ & $7.3 \times 10^{-3}$ & $1$ & 137 & $10^{2.1}$ \\
			$\beta_{\text{T}}$ & $0.008$ & $1$ & 125 & $10^{2.1}$ \\
			$\alpha_{\text{W}}$ & $2.898$ & $1$ & 2.9 & $10^{0.5}$ \\
			\bottomrule
		\end{tabular}
		\caption{Systematic parameter differences between unit systems}
		\label{tab:parameter_comparison}
	\end{table}
	
	\textbf{All parameters show 0.5-4 orders of magnitude difference between systems!}
	
	# Yukawa Parameters: Variable and System-Dependent
	\label{sec:yukawa_parameters}
	
	## The Hierarchy of Yukawa Couplings
	\label{subsec:yukawa_hierarchy}
	
	In the Standard Model, Yukawa couplings vary dramatically:
	
	\begin{table}[htbp]
		\centering
		\begin{tabular}{lc}
			\toprule
			\textbf{Particle} & \textbf{$y_i$ (SI system)} \\
			\midrule
			Electron & $2.94 \times 10^{-6}$ \\
			Muon & $6.09 \times 10^{-4}$ \\
			Tau & $1.03 \times 10^{-2}$ \\
			Up quark & $1.27 \times 10^{-5}$ \\
			Top quark & $1.00$ \\
			Bottom quark & $2.25 \times 10^{-2}$ \\
			\bottomrule
		\end{tabular}
		\caption{Yukawa coupling hierarchy (5 orders of magnitude variation)}
		\label{tab:yukawa_hierarchy}
	\end{table}
	
	## Transformation Uncertainty
	\label{subsec:yukawa_transformation}
	
	The transformation of Yukawa parameters between systems requires careful consideration of the Higgs mechanism. The general form would be:
	
	$$y_{i,\text{nat}} = y_{i,\text{SI}} \times T_{\text{Yukawa}}$$
	
	where $T_{\text{Yukawa}}$ depends on the transformation of Higgs vacuum expectation value and particle masses.
	
	## Consistency Requirements
	\label{subsec:yukawa_consistency}
	
	The Higgs mechanism requires:
	$$m_h^2 = \frac{\lambdah v^2}{2}$$
	
	For transformation consistency:
	$$T_m^2 = T_\lambda \times T_v^2$$
	
	This gives:
	$$y_{i,\text{nat}} = y_{i,\text{SI}} \times \sqrt{T_\lambda}$$
	
	\textbf{However, $T_\lambda$ requires detailed specification of the T0-natural unit system transformation rules.}
	
	# Universal Warning: No Direct Parameter Transfer
	\label{sec:universal_warning}
	
	## The Systematic Problem
	\label{subsec:systematic_problem}
	
	\begin{warning}
		\textbf{EVERY parameter symbol in T0-model documents may have different values than in SI system calculations!}
	\end{warning}
	
	\textbf{Concrete danger zones:}
	
	
```math-align

		G_{\text{nat}} &= 1 \quad \text{vs.} \quad G_{\text{SI}} = 6.674 \times 10^{-11} \text{ m}^3/(\text{kg} \cdot \text{s}^2) \\
		\alpha_{\text{EM,nat}} &= 1 \quad \text{vs.} \quad \alpha_{\text{EM,SI}} = 1/137 \\
		e_{\text{nat}} &= 2\sqrt{\pichar} \quad \text{vs.} \quad e_{\text{SI}} = 1.602 \times 10^{-19} \text{ C}
	
```

	
	\textbf{Direct transfer leads to errors of factors $10^2$ to $10^{11}$!}
	
	## Required Transformation Protocol
	\label{subsec:transformation_protocol}
	
	For every parameter, explicitly specify:
	
	
		- \textbf{Which unit system} is being used
		- \textbf{How transformation occurs} between systems
		- \textbf{Which factors must be considered}
		- \textbf{Which consistency conditions} must be satisfied
	
	
	\textbf{Example of complete specification:}
	\begin{tcolorbox}[colback=red!5!white,colframe=red!75!black,title=Parameter Specification Template]
		\textbf{Parameter:} Fine structure constant $\alpha_{\text{EM}}$ \\
		\textbf{SI value:} $\alphaEMSI = 1/137.036$ \\
		\textbf{T0 value:} $\alphaEMnat = 1$ \\
		\textbf{Transformation:} $\alphaEMnat = \alphaEMSI \times 137.036$ \\
		\textbf{Consistency:} Dimensional analysis verified \\
		\textbf{Usage:} Specify system before calculation
	\end{tcolorbox}
	
	## Experimental Prediction Guidelines
	\label{subsec:experimental_guidelines}
	
	\textbf{For QED calculations:}
	
```math-align

		\text{WRONG:} \quad &\alpha_{\text{EM}} = 1 \text{ from T0-model directly in SI formulas} \\
		\text{CORRECT:} \quad &\alphaEMSI = 1/137 \text{ with transformation to } \alphaEMnat = 1
	
```

	
	\textbf{For gravitational calculations:}
	
```math-align

		\text{WRONG:} \quad &G = 1 \text{ from T0-model directly in Newton's formulas} \\
		\text{CORRECT:} \quad &G_{\text{SI}} = 6.674 \times 10^{-11} \text{ with transformation to } G_{\text{nat}} = 1
	
```

	
	# The Circularity Resolution
	\label{sec:circularity_resolution}
	
	## Apparent vs. Real Circularity
	\label{subsec:apparent_real_circularity}
	
	The circularity problem that seemed to plague T0-model parameter determination is resolved by recognizing:
	
	
		- \textbf{No real circularity exists} within each consistent system
		- \textbf{Both SI and T0 systems are internally consistent}
		- \textbf{The apparent contradiction} arose from comparing parameters across different systems
		- \textbf{Proper transformation} eliminates all apparent inconsistencies
	
	
	## System Consistency Verification
	\label{subsec:system_consistency}
	
	\textbf{SI system consistency:}
	$$\Rzero = \frac{m_e c \left(\alphaEMSI\right)^2}{2\hbar} \quad \checkmark \text{ (experimentally verified to 0.000001\%)}$$
	
	\textbf{T0 system consistency:}
	$$\text{All parameters = 1} \quad \checkmark \text{ (by construction)}$$
	
	\textbf{Both systems work perfectly within their own frameworks!}
	
	# Implications for T0-Model Testing
	\label{sec:testing_implications}
	
	## System-Specific Predictions
	\label{subsec:system_specific_predictions}
	
	Experimental tests must clearly specify which parameter system is used:
	
	\begin{table}[htbp]
		\centering
		\begin{tabular}{lcc}
			\toprule
			\textbf{Test Type} & \textbf{SI-based Prediction} & \textbf{T0-based Prediction} \\
			\midrule
			QED anomaly & $a_e \propto \alphaEMSI = 1/137$ & $a_e \propto \alphaEMnat = 1$ \\
			Galaxy rotation & $v^2 \propto \xipar_{\text{SI}} \sim 10^{38}$ & $v^2 \propto \xipar_{\text{nat}} \sim 10^{41}$ \\
			CMB temperature & $T \propto \betaTSI = 0.008$ & $T \propto \betaTnat = 1$ \\
			\bottomrule
		\end{tabular}
		\caption{System-specific experimental predictions}
		\label{tab:system_predictions}
	\end{table}
	
	## Transformation Validation
	\label{subsec:transformation_validation}
	
	The transformation factors can be validated by checking:
	
	
		- \textbf{Dimensional consistency} in both systems
		- \textbf{Known limits} are reproduced correctly
		- \textbf{Ratios remain invariant} between systems
		- \textbf{Internal consistency} of each system

\end{document}
