\documentclass[11pt,a4paper,openany]{book}

% Essential packages
\usepackage[utf8]{inputenc}
\usepackage[T1]{fontenc}
\usepackage[english]{babel}
\usepackage[a4paper,margin=2.5cm]{geometry}
\usepackage{lmodern}

% Math and physics packages
\usepackage{amsmath}
\usepackage{amssymb}
\usepackage{amsthm}
\usepackage{mathtools}
\usepackage{physics}
\usepackage{siunitx}

% Graphics and tables
\usepackage{graphicx}
\usepackage[table,xcdraw]{xcolor}
\usepackage{tikz}
\usepackage{pgfplots}
\usepackage{tcolorbox}
\usepackage{booktabs}
\usepackage{array}
\usepackage{longtable}
\usepackage{float}

% Document formatting
\usepackage{fancyhdr}
\usepackage{tocloft}
\usepackage{hyperref}
\usepackage{cleveref}
\usepackage{microtype}
\usepackage{enumitem}
\usepackage{newunicodechar}

% Additional packages
\usepackage{adjustbox}
\usepackage{algorithm}
\usepackage{algorithmic}
\usepackage{amsfonts}
\usepackage{amsmath,amsfonts,amssymb}
\usepackage{amsmath,amsfonts,amssymb,physics}
\usepackage{amsmath,amssymb}
\usepackage{amsmath,amssymb,amsfonts,amsthm}
\usepackage{amsmath,amssymb,amsthm}
\usepackage{amsmath,amssymb,physics,graphicx,xcolor,amsthm}
\usepackage{bm}
\usepackage{booktabs,array,longtable,multirow}
\usepackage{braket}
\usepackage{breakurl}
\usepackage{cancel}
\usepackage{caption}
\usepackage{cite}
\usepackage{color}
\usepackage{colortbl}
\usepackage{csquotes}
\usepackage{doi}
\usepackage{forest}
\usepackage{gensymb}
\usepackage{geometry,fancyhdr}
\usepackage{graphicx,tikz,pgfplots}
\usepackage{hyperref,url}
\usepackage{hyphenat}
\usepackage{listings}
\usepackage{listings,enumerate}
\usepackage{mdframed}
\usepackage{multicol}
\usepackage{multirow}
\usepackage{natbib}
\usepackage{pdflscape}
\usepackage{ragged2e}
\usepackage{setspace}
\usepackage{siunitx,xcolor,graphicx}
\usepackage{slashed}
\usepackage{tabularx}
\usepackage{textcomp}
\usepackage{textgreek}
\usepackage{tikz,pgfplots}
\usepackage{upgreek}
\usepackage{url}

% Custom commands and definitions
\definecolor{blue}
\definecolor{blue}{rgb}{0,0,1}
\definecolor{boxgray}
\definecolor{boxgray}{RGB}{240,240,240}
\definecolor{deepblue}
\definecolor{deepblue}{RGB}{0,0,127}
\definecolor{deepgreen}
\definecolor{deepgreen}{RGB}{0,127,0}
\definecolor{deepred}
\definecolor{deepred}{RGB}{191,0,0}
\definecolor{t0blue}
\definecolor{t0blue}{RGB}{0,102,204}
\definecolor{t0blue}{RGB}{33,150,243}
\definecolor{t0green}
\definecolor{t0green}{RGB}{0,153,0}
\definecolor{t0green}{RGB}{0,153,76}
\definecolor{t0green}{RGB}{76,175,80}
\definecolor{t0orange}
\definecolor{t0orange}{RGB}{255,152,0}
\definecolor{t0purple}
\definecolor{t0purple}{RGB}{102,0,204}
\definecolor{t0purple}{RGB}{156,39,176}
\definecolor{t0red}
\definecolor{t0red}{RGB}{204,0,0}
\definecolor{t0red}{RGB}{204,0,51}
\definecolor{t0red}{RGB}{244,67,54}
\definecolor{t0yellow}
\definecolor{t0yellow}{RGB}{255,204,0}
\geometry{a4paper, left=25mm, right=25mm, top=25mm, bottom=25mm}
\geometry{a4paper, margin=1in}
\geometry{a4paper, margin=2.5cm}
\geometry{a4paper, margin=2cm}
\geometry{left=2.5cm,right=2.5cm,top=2.5cm,bottom=2.5cm}
\geometry{left=2cm,right=2cm,top=2cm,bottom=2cm}
\geometry{margin=1in}
\geometry{margin=2.5cm}
\geometry{margin=2cm}
\hypersetup{
	colorlinks=true,
	linkcolor=blue,
	citecolor=blue,
	urlcolor=blue,
	pdftitle={Analysis and Implications of MNRAS Paper 544 for the T0-Theory}
\hypersetup{
	colorlinks=true,
	linkcolor=blue,
	citecolor=blue,
	urlcolor=blue,
	pdftitle={Beweis: Die Feinstrukturkonstante α = 1 in natürlichen Einheiten}
\hypersetup{
	colorlinks=true,
	linkcolor=blue,
	citecolor=blue,
	urlcolor=blue,
	pdftitle={Beweis: Die Koide-Formel enthält implizit $\xi$}
\hypersetup{
	colorlinks=true,
	linkcolor=blue,
	citecolor=blue,
	urlcolor=blue,
	pdftitle={Chinas Photonischer Quantenchip: 1000x-Speedup und T0-Integration}
\hypersetup{
	colorlinks=true,
	linkcolor=blue,
	citecolor=blue,
	urlcolor=blue,
	pdftitle={Complete Derivation of Higgs Mass and Wilson Coefficients}
\hypersetup{
	colorlinks=true,
	linkcolor=blue,
	citecolor=blue,
	urlcolor=blue,
	pdftitle={Complete Particle Spectrum: Standard Model vs T0 Theory}
\hypersetup{
	colorlinks=true,
	linkcolor=blue,
	citecolor=blue,
	urlcolor=blue,
	pdftitle={Conceptual Comparison of Unified Natural Units and Extended Standard Model}
\hypersetup{
	colorlinks=true,
	linkcolor=blue,
	citecolor=blue,
	urlcolor=blue,
	pdftitle={Connections between the Mizohata-Takeuchi Counterexample and the T0 Time-Mass Duality Theory}
\hypersetup{
	colorlinks=true,
	linkcolor=blue,
	citecolor=blue,
	urlcolor=blue,
	pdftitle={Das Relationale Zahlensystem: Primzahlen als fundamentale Verhältnisse}
\hypersetup{
	colorlinks=true,
	linkcolor=blue,
	citecolor=blue,
	urlcolor=blue,
	pdftitle={Das T0-Modell (Planck-Referenziert): Eine Neuformulierung der Physik}
\hypersetup{
	colorlinks=true,
	linkcolor=blue,
	citecolor=blue,
	urlcolor=blue,
	pdftitle={Das T0-Modell: Zeit-Energie-Dualität und geometrische Ruhemasse}
\hypersetup{
	colorlinks=true,
	linkcolor=blue,
	citecolor=blue,
	urlcolor=blue,
	pdftitle={Der Massenskalierungsexponent κ in der T0-Theorie}
\hypersetup{
	colorlinks=true,
	linkcolor=blue,
	citecolor=blue,
	urlcolor=blue,
	pdftitle={Der geometrische Formalismus der T0-Quantenmechanik und seine Anwendung auf Quantencomputer}
\hypersetup{
	colorlinks=true,
	linkcolor=blue,
	citecolor=blue,
	urlcolor=blue,
	pdftitle={Der xi Parameter und Teilchendifferenzierung in der T0-Theorie}
\hypersetup{
	colorlinks=true,
	linkcolor=blue,
	citecolor=blue,
	urlcolor=blue,
	pdftitle={Deterministic Quantum Mechanics via T0-Energy Field Formulation}
\hypersetup{
	colorlinks=true,
	linkcolor=blue,
	citecolor=blue,
	urlcolor=blue,
	pdftitle={Deterministische Quantenmechanik via T0-Energiefeld-Formulierung}
\hypersetup{
	colorlinks=true,
	linkcolor=blue,
	citecolor=blue,
	urlcolor=blue,
	pdftitle={Die Elektroneneinheitsladung in der T0-Theorie: Jenseits von Punkt-Singularitäten}
\hypersetup{
	colorlinks=true,
	linkcolor=blue,
	citecolor=blue,
	urlcolor=blue,
	pdftitle={Die Feinstrukturkonstante: Verschiedene Darstellungen und Beziehungen}
\hypersetup{
	colorlinks=true,
	linkcolor=blue,
	citecolor=blue,
	urlcolor=blue,
	pdftitle={Die Musikalische Spirale und die 137: Die mathematische Entdeckung der kosmischen Verstimmung}
\hypersetup{
	colorlinks=true,
	linkcolor=blue,
	citecolor=blue,
	urlcolor=blue,
	pdftitle={E=mc² = E=m: Die Konstanten-Illusion entlarvt}
\hypersetup{
	colorlinks=true,
	linkcolor=blue,
	citecolor=blue,
	urlcolor=blue,
	pdftitle={E=mc² = E=m: The Constants Illusion Exposed}
\hypersetup{
	colorlinks=true,
	linkcolor=blue,
	citecolor=blue,
	urlcolor=blue,
	pdftitle={Einfache Lagrange-Revolution: Von der Standardmodell-Komplexität zur T0-Eleganz}
\hypersetup{
	colorlinks=true,
	linkcolor=blue,
	citecolor=blue,
	urlcolor=blue,
	pdftitle={Einführung in die Umsetzung photonischer Bauteile auf Wafern für Nachrichtentechniker}
\hypersetup{
	colorlinks=true,
	linkcolor=blue,
	citecolor=blue,
	urlcolor=blue,
	pdftitle={Einführung in photonische Quantenchips für Nachrichtentechniker}
\hypersetup{
	colorlinks=true,
	linkcolor=blue,
	citecolor=blue,
	urlcolor=blue,
	pdftitle={Elimination der Masse als dimensionaler Platzhalter im T0-Modell}
\hypersetup{
	colorlinks=true,
	linkcolor=blue,
	citecolor=blue,
	urlcolor=blue,
	pdftitle={Elimination of Mass as Dimensional Placeholder in the T0 Model}
\hypersetup{
	colorlinks=true,
	linkcolor=blue,
	citecolor=blue,
	urlcolor=blue,
	pdftitle={Empirical Analysis of Deterministic Factorization Methods}
\hypersetup{
	colorlinks=true,
	linkcolor=blue,
	citecolor=blue,
	urlcolor=blue,
	pdftitle={Empirische Analyse deterministischer Faktorisierungsmethoden}
\hypersetup{
	colorlinks=true,
	linkcolor=blue,
	citecolor=blue,
	urlcolor=blue,
	pdftitle={Integration der Dirac-Gleichung im T0-Modell: Natürliche-Einheiten-Rahmenwerk}
\hypersetup{
	colorlinks=true,
	linkcolor=blue,
	citecolor=blue,
	urlcolor=blue,
	pdftitle={Integration of the Dirac Equation in the T0 Model: Natural Units Framework}
\hypersetup{
	colorlinks=true,
	linkcolor=blue,
	citecolor=blue,
	urlcolor=blue,
	pdftitle={Introduction to Photonic Quantum Chips for Communication Engineers}
\hypersetup{
	colorlinks=true,
	linkcolor=blue,
	citecolor=blue,
	urlcolor=blue,
	pdftitle={Introduction to the Implementation of Photonic Components on Wafers for Communication Engineers}
\hypersetup{
	colorlinks=true,
	linkcolor=blue,
	citecolor=blue,
	urlcolor=blue,
	pdftitle={Konzeptioneller Vergleich von Einheitlichen Natürlichen Einheiten und Erweitertem Standardmodell}
\hypersetup{
	colorlinks=true,
	linkcolor=blue,
	citecolor=blue,
	urlcolor=blue,
	pdftitle={Markov Chains in the Context of T0 Theory: Deterministic or Stochastic? A Treatise on Patterns, Preconditions, and Uncertainty}
\hypersetup{
	colorlinks=true,
	linkcolor=blue,
	citecolor=blue,
	urlcolor=blue,
	pdftitle={Markov-Ketten im Kontext der T0-Theorie: Deterministisch oder stochastisch? Ein Traktat zu Mustern, Voraussetzungen und Unsicherheit}
\hypersetup{
	colorlinks=true,
	linkcolor=blue,
	citecolor=blue,
	urlcolor=blue,
	pdftitle={Mathematical Analysis of T0-Shor Algorithm: Theoretical Framework and Computational Complexity}
\hypersetup{
	colorlinks=true,
	linkcolor=blue,
	citecolor=blue,
	urlcolor=blue,
	pdftitle={Mathematical Constructs of Alternative CMB Models: Unnikrishnan and Peratt in Harmony with the T0 Theory}
\hypersetup{
	colorlinks=true,
	linkcolor=blue,
	citecolor=blue,
	urlcolor=blue,
	pdftitle={Mathematische Analyse des T0-Shor Algorithmus: Theoretischer Rahmen und Berechnungskomplexität}
\hypersetup{
	colorlinks=true,
	linkcolor=blue,
	citecolor=blue,
	urlcolor=blue,
	pdftitle={Mathematische Konstrukte alternativer CMB-Modelle: Unnikrishnan und Peratt im Einklang mit der T0-Theorie}
\hypersetup{
	colorlinks=true,
	linkcolor=blue,
	citecolor=blue,
	urlcolor=blue,
	pdftitle={Natural Unit Systems: Universal Energy Conversion and Fundamental Length Scale Hierarchy}
\hypersetup{
	colorlinks=true,
	linkcolor=blue,
	citecolor=blue,
	urlcolor=blue,
	pdftitle={Natural Units in Theoretical Physics: A Treatise in the Context of T0 Theory}
\hypersetup{
	colorlinks=true,
	linkcolor=blue,
	citecolor=blue,
	urlcolor=blue,
	pdftitle={Natürliche Einheiten in der theoretischen Physik: Eine Abhandlung im Kontext der T0-Theorie}
\hypersetup{
	colorlinks=true,
	linkcolor=blue,
	citecolor=blue,
	urlcolor=blue,
	pdftitle={Natürliche Einheitensysteme: Universelle Energieumwandlung und fundamentale Längenskala-Hierarchie}
\hypersetup{
	colorlinks=true,
	linkcolor=blue,
	citecolor=blue,
	urlcolor=blue,
	pdftitle={Parameter System-Dependency in T0-Model: SI vs. Natural Units}
\hypersetup{
	colorlinks=true,
	linkcolor=blue,
	citecolor=blue,
	urlcolor=blue,
	pdftitle={Parameter-Systemabhängigkeit im T0-Modell: SI- vs. natürliche Einheiten}
\hypersetup{
	colorlinks=true,
	linkcolor=blue,
	citecolor=blue,
	urlcolor=blue,
	pdftitle={Proof: The Fine Structure Constant α = 1 in Natural Units}
\hypersetup{
	colorlinks=true,
	linkcolor=blue,
	citecolor=blue,
	urlcolor=blue,
	pdftitle={Proof: The Koide Formula Implicitly Contains $\xi$}
\hypersetup{
	colorlinks=true,
	linkcolor=blue,
	citecolor=blue,
	urlcolor=blue,
	pdftitle={Pure Energy T0 Theory: Ratio-Based Physics with SI Reference}
\hypersetup{
	colorlinks=true,
	linkcolor=blue,
	citecolor=blue,
	urlcolor=blue,
	pdftitle={Quantum Mechanics in the T0 Model: Field-Theoretic Foundations}
\hypersetup{
	colorlinks=true,
	linkcolor=blue,
	citecolor=blue,
	urlcolor=blue,
	pdftitle={Ratio-Based vs. Absolute: The Role of Fractal Correction in T0 Theory}
\hypersetup{
	colorlinks=true,
	linkcolor=blue,
	citecolor=blue,
	urlcolor=blue,
	pdftitle={Reine Energie T0-Theorie: Verhältnis-basierte Physik mit SI-Referenz}
\hypersetup{
	colorlinks=true,
	linkcolor=blue,
	citecolor=blue,
	urlcolor=blue,
	pdftitle={Simple Lagrangian Revolution: From Standard Model Complexity to T0 Elegance}
\hypersetup{
	colorlinks=true,
	linkcolor=blue,
	citecolor=blue,
	urlcolor=blue,
	pdftitle={Simplified Dirac Equation in T0 Theory: Field Node Approach}
\hypersetup{
	colorlinks=true,
	linkcolor=blue,
	citecolor=blue,
	urlcolor=blue,
	pdftitle={Simplified T0 Theory: Elegant Lagrangian Density for Time-Mass Duality}
\hypersetup{
	colorlinks=true,
	linkcolor=blue,
	citecolor=blue,
	urlcolor=blue,
	pdftitle={T0 Cosmology: Redshift as a Geometric Path Effect in a Static Universe}
\hypersetup{
	colorlinks=true,
	linkcolor=blue,
	citecolor=blue,
	urlcolor=blue,
	pdftitle={T0 Deterministic Quantum Computing: Complete Analysis of Important Algorithms}
\hypersetup{
	colorlinks=true,
	linkcolor=blue,
	citecolor=blue,
	urlcolor=blue,
	pdftitle={T0 Deterministisches Quantencomputing: Vollständige Analyse wichtiger Algorithmen}
\hypersetup{
	colorlinks=true,
	linkcolor=blue,
	citecolor=blue,
	urlcolor=blue,
	pdftitle={T0 Model: Complete Framework - From Time-Energy Duality to Universal Constants}
\hypersetup{
	colorlinks=true,
	linkcolor=blue,
	citecolor=blue,
	urlcolor=blue,
	pdftitle={T0 Model: Complete Parameter-Free Particle Mass Calculation}
\hypersetup{
	colorlinks=true,
	linkcolor=blue,
	citecolor=blue,
	urlcolor=blue,
	pdftitle={T0 Model: Unified Neutrino Formula Structure}
\hypersetup{
	colorlinks=true,
	linkcolor=blue,
	citecolor=blue,
	urlcolor=blue,
	pdftitle={T0 Model: Universal Energy Relations for Mol and Candela Units}
\hypersetup{
	colorlinks=true,
	linkcolor=blue,
	citecolor=blue,
	urlcolor=blue,
	pdftitle={T0 Modell: Vollständiges Framework - Von Zeit-Energie-Dualität zu universellen Konstanten}
\hypersetup{
	colorlinks=true,
	linkcolor=blue,
	citecolor=blue,
	urlcolor=blue,
	pdftitle={T0 Quantenfeldtheorie: QFT, QM und Quantencomputer}
\hypersetup{
	colorlinks=true,
	linkcolor=blue,
	citecolor=blue,
	urlcolor=blue,
	pdftitle={T0 Quantum Field Theory: QFT, QM and Quantum Computers}
\hypersetup{
	colorlinks=true,
	linkcolor=blue,
	citecolor=blue,
	urlcolor=blue,
	pdftitle={T0 Theory vs Bell's Theorem: How Deterministic Energy Fields Circumvent No-Go Theorems}
\hypersetup{
	colorlinks=true,
	linkcolor=blue,
	citecolor=blue,
	urlcolor=blue,
	pdftitle={T0 Theory: Final Extension to Hadrons - Physically Derived Corrections}
\hypersetup{
	colorlinks=true,
	linkcolor=blue,
	citecolor=blue,
	urlcolor=blue,
	pdftitle={T0 Theory: The Fine-Structure Constant}
\hypersetup{
	colorlinks=true,
	linkcolor=blue,
	citecolor=blue,
	urlcolor=blue,
	pdftitle={T0 Theory: The Gravitational Constant}
\hypersetup{
	colorlinks=true,
	linkcolor=blue,
	citecolor=blue,
	urlcolor=blue,
	pdftitle={T0-Kosmologie: Rotverschiebung als geometrischer Pfad-Effekt im statischen Universum}
\hypersetup{
	colorlinks=true,
	linkcolor=blue,
	citecolor=blue,
	urlcolor=blue,
	pdftitle={T0-Model: Complete Document Analysis and Structured Summary}
\hypersetup{
	colorlinks=true,
	linkcolor=blue,
	citecolor=blue,
	urlcolor=blue,
	pdftitle={T0-Model: Kinetic Energy of Electrons and Photons}
\hypersetup{
	colorlinks=true,
	linkcolor=blue,
	citecolor=blue,
	urlcolor=blue,
	pdftitle={T0-Model: The Hubble Parameter in Static Universe}
\hypersetup{
	colorlinks=true,
	linkcolor=blue,
	citecolor=blue,
	urlcolor=blue,
	pdftitle={T0-Modell-Verifikation: Skalen-Verhältnis-basierte Berechnungen}
\hypersetup{
	colorlinks=true,
	linkcolor=blue,
	citecolor=blue,
	urlcolor=blue,
	pdftitle={T0-Modell: Bewegungsenergie von Elektronen und Photonen}
\hypersetup{
	colorlinks=true,
	linkcolor=blue,
	citecolor=blue,
	urlcolor=blue,
	pdftitle={T0-Modell: Die Hubble-Konstante im statischen Universum}
\hypersetup{
	colorlinks=true,
	linkcolor=blue,
	citecolor=blue,
	urlcolor=blue,
	pdftitle={T0-Modell: Einheitliche Neutrino-Formel-Struktur}
\hypersetup{
	colorlinks=true,
	linkcolor=blue,
	citecolor=blue,
	urlcolor=blue,
	pdftitle={T0-Modell: Universelle Energiebeziehungen für Mol- und Candela-Einheiten}
\hypersetup{
	colorlinks=true,
	linkcolor=blue,
	citecolor=blue,
	urlcolor=blue,
	pdftitle={T0-Modell: Vollständige Dokumentenanalyse und strukturierte Zusammenfassung}
\hypersetup{
	colorlinks=true,
	linkcolor=blue,
	citecolor=blue,
	urlcolor=blue,
	pdftitle={T0-Modell: Vollständige parameterfreie Teilchenmassen-Berechnung}
\hypersetup{
	colorlinks=true,
	linkcolor=blue,
	citecolor=blue,
	urlcolor=blue,
	pdftitle={T0-QAT: $\xi$-Aware Quantization-Aware Training}
\hypersetup{
	colorlinks=true,
	linkcolor=blue,
	citecolor=blue,
	urlcolor=blue,
	pdftitle={T0-QFT ML Addendum: Machine Learning Derived Extensions}
\hypersetup{
	colorlinks=true,
	linkcolor=blue,
	citecolor=blue,
	urlcolor=blue,
	pdftitle={T0-QFT ML-Addendum: Maschinelle Lern-abgeleitete Erweiterungen}
\hypersetup{
	colorlinks=true,
	linkcolor=blue,
	citecolor=blue,
	urlcolor=blue,
	pdftitle={T0-Theorie vs Bells Theorem: Wie deterministische Energiefelder No-Go-Theoreme umgehen}
\hypersetup{
	colorlinks=true,
	linkcolor=blue,
	citecolor=blue,
	urlcolor=blue,
	pdftitle={T0-Theorie: Der Terrell-Penrose-Effekt und Massenvariation}
\hypersetup{
	colorlinks=true,
	linkcolor=blue,
	citecolor=blue,
	urlcolor=blue,
	pdftitle={T0-Theorie: Die Feinstrukturkonstante}
\hypersetup{
	colorlinks=true,
	linkcolor=blue,
	citecolor=blue,
	urlcolor=blue,
	pdftitle={T0-Theorie: Die Gravitationskonstante}
\hypersetup{
	colorlinks=true,
	linkcolor=blue,
	citecolor=blue,
	urlcolor=blue,
	pdftitle={T0-Theorie: Die T0-Zeit-Masse-Dualität}
\hypersetup{
	colorlinks=true,
	linkcolor=blue,
	citecolor=blue,
	urlcolor=blue,
	pdftitle={T0-Theorie: Die sieben Rätsel}
\hypersetup{
	colorlinks=true,
	linkcolor=blue,
	citecolor=blue,
	urlcolor=blue,
	pdftitle={T0-Theorie: Erweiterung auf Bell-Tests – ML-Simulationen (November 2025)}
\hypersetup{
	colorlinks=true,
	linkcolor=blue,
	citecolor=blue,
	urlcolor=blue,
	pdftitle={T0-Theorie: Finale Erweiterung auf Hadronen - Physikalisch abgeleitete Korrekturen}
\hypersetup{
	colorlinks=true,
	linkcolor=blue,
	citecolor=blue,
	urlcolor=blue,
	pdftitle={T0-Theorie: Finale Fraktale Massenformeln (November 2025)}
\hypersetup{
	colorlinks=true,
	linkcolor=blue,
	citecolor=blue,
	urlcolor=blue,
	pdftitle={T0-Theorie: Fraktaldimension aus Lepton-Massenverhältnis}
\hypersetup{
	colorlinks=true,
	linkcolor=blue,
	citecolor=blue,
	urlcolor=blue,
	pdftitle={T0-Theorie: Fundamentale Prinzipien}
\hypersetup{
	colorlinks=true,
	linkcolor=blue,
	citecolor=blue,
	urlcolor=blue,
	pdftitle={T0-Theorie: Herleitung der Gravitationskonstanten}
\hypersetup{
	colorlinks=true,
	linkcolor=blue,
	citecolor=blue,
	urlcolor=blue,
	pdftitle={T0-Theorie: Kosmische Beziehungen und universelle $\xi$-Konstante}
\hypersetup{
	colorlinks=true,
	linkcolor=blue,
	citecolor=blue,
	urlcolor=blue,
	pdftitle={T0-Theorie: Kosmologie}
\hypersetup{
	colorlinks=true,
	linkcolor=blue,
	citecolor=blue,
	urlcolor=blue,
	pdftitle={T0-Theorie: Netzwerkdarstellung und Dimensionsanalyse in der T0-Theorie}
\hypersetup{
	colorlinks=true,
	linkcolor=blue,
	citecolor=blue,
	urlcolor=blue,
	pdftitle={T0-Theorie: Teilchenmassen}
\hypersetup{
	colorlinks=true,
	linkcolor=blue,
	citecolor=blue,
	urlcolor=blue,
	pdftitle={T0-Theorie: Vollstaendiger Abschluss}
\hypersetup{
	colorlinks=true,
	linkcolor=blue,
	citecolor=blue,
	urlcolor=blue,
	pdftitle={T0-Theory: Complete Closure}
\hypersetup{
	colorlinks=true,
	linkcolor=blue,
	citecolor=blue,
	urlcolor=blue,
	pdftitle={T0-Theory: Complete Derivation of All Parameters Without Circularity}
\hypersetup{
	colorlinks=true,
	linkcolor=blue,
	citecolor=blue,
	urlcolor=blue,
	pdftitle={T0-Theory: Cosmic Relations and universal $\xi$-constant}
\hypersetup{
	colorlinks=true,
	linkcolor=blue,
	citecolor=blue,
	urlcolor=blue,
	pdftitle={T0-Theory: Cosmology}
\hypersetup{
	colorlinks=true,
	linkcolor=blue,
	citecolor=blue,
	urlcolor=blue,
	pdftitle={T0-Theory: Derivation of the Gravitational Constant}
\hypersetup{
	colorlinks=true,
	linkcolor=blue,
	citecolor=blue,
	urlcolor=blue,
	pdftitle={T0-Theory: Extension to Bell Tests – ML Simulations (November 2025)}
\hypersetup{
	colorlinks=true,
	linkcolor=blue,
	citecolor=blue,
	urlcolor=blue,
	pdftitle={T0-Theory: Final Fractal Mass Formulas (November 2025)}
\hypersetup{
	colorlinks=true,
	linkcolor=blue,
	citecolor=blue,
	urlcolor=blue,
	pdftitle={T0-Theory: Fractal Dimension from Lepton Mass Ratio}
\hypersetup{
	colorlinks=true,
	linkcolor=blue,
	citecolor=blue,
	urlcolor=blue,
	pdftitle={T0-Theory: Fundamental Principles}
\hypersetup{
	colorlinks=true,
	linkcolor=blue,
	citecolor=blue,
	urlcolor=blue,
	pdftitle={T0-Theory: Mass Variation as an Equivalent to Time Dilation}
\hypersetup{
	colorlinks=true,
	linkcolor=blue,
	citecolor=blue,
	urlcolor=blue,
	pdftitle={T0-Theory: Network Representation and Dimensional Analysis in the T0-Theory}
\hypersetup{
	colorlinks=true,
	linkcolor=blue,
	citecolor=blue,
	urlcolor=blue,
	pdftitle={T0-Theory: Neutrinos}
\hypersetup{
	colorlinks=true,
	linkcolor=blue,
	citecolor=blue,
	urlcolor=blue,
	pdftitle={T0-Theory: Particle Masses}
\hypersetup{
	colorlinks=true,
	linkcolor=blue,
	citecolor=blue,
	urlcolor=blue,
	pdftitle={T0-Theory: The Seven Riddles}
\hypersetup{
	colorlinks=true,
	linkcolor=blue,
	citecolor=blue,
	urlcolor=blue,
	pdftitle={T0-Theory: The T0-Time-Mass Duality}
\hypersetup{
	colorlinks=true,
	linkcolor=blue,
	citecolor=blue,
	urlcolor=blue,
	pdftitle={Temperature Units in Natural Units: T0-Theory}
\hypersetup{
	colorlinks=true,
	linkcolor=blue,
	citecolor=blue,
	urlcolor=blue,
	pdftitle={Temperatureinheiten in nat\"urlichen Einheiten: T0-Theorie}
\hypersetup{
	colorlinks=true,
	linkcolor=blue,
	citecolor=blue,
	urlcolor=blue,
	pdftitle={The Electron Unit Charge in T0 Theory: Beyond Point Singularities}
\hypersetup{
	colorlinks=true,
	linkcolor=blue,
	citecolor=blue,
	urlcolor=blue,
	pdftitle={The Fine Structure Constant: Various Representations and Relationships}
\hypersetup{
	colorlinks=true,
	linkcolor=blue,
	citecolor=blue,
	urlcolor=blue,
	pdftitle={The Geometric Formalism of T0 Quantum Mechanics and its Application to Quantum Computing}
\hypersetup{
	colorlinks=true,
	linkcolor=blue,
	citecolor=blue,
	urlcolor=blue,
	pdftitle={The Mass Scaling Exponent κ in T0 Theory}
\hypersetup{
	colorlinks=true,
	linkcolor=blue,
	citecolor=blue,
	urlcolor=blue,
	pdftitle={The Musical Spiral and 137: The Mathematical Discovery of Cosmic Detuning}
\hypersetup{
	colorlinks=true,
	linkcolor=blue,
	citecolor=blue,
	urlcolor=blue,
	pdftitle={The Relational Number System: Prime Numbers as Fundamental Ratios}
\hypersetup{
	colorlinks=true,
	linkcolor=blue,
	citecolor=blue,
	urlcolor=blue,
	pdftitle={The T0 Model (Planck-Referenced): A Reformulation of Physics}
\hypersetup{
	colorlinks=true,
	linkcolor=blue,
	citecolor=blue,
	urlcolor=blue,
	pdftitle={The T0 Model: Time-Energy Duality and Geometric Rest Mass}
\hypersetup{
	colorlinks=true,
	linkcolor=blue,
	citecolor=blue,
	urlcolor=blue,
	pdftitle={The T0-Model (Planck-Referenced): A Reformulation of Physics}
\hypersetup{
	colorlinks=true,
	linkcolor=blue,
	citecolor=blue,
	urlcolor=blue,
	pdftitle={Verbindungen zwischen dem Mizohata-Takeuchi-Gegenbeispiel und der T0-Zeit-Masse-Dualitätstheorie}
\hypersetup{
	colorlinks=true,
	linkcolor=blue,
	citecolor=blue,
	urlcolor=blue,
	pdftitle={Vereinfachte Dirac-Gleichung in der T0-Theorie: Feldknoten-Ansatz}
\hypersetup{
	colorlinks=true,
	linkcolor=blue,
	citecolor=blue,
	urlcolor=blue,
	pdftitle={Vereinfachte T0-Theorie: Elegante Lagrange-Dichte für Zeit-Masse-Dualität}
\hypersetup{
	colorlinks=true,
	linkcolor=blue,
	citecolor=blue,
	urlcolor=blue,
	pdftitle={Verhältnisbasiert vs. Absolut: Die Rolle der fraktalen Korrektur in der T0-Theorie}
\hypersetup{
	colorlinks=true,
	linkcolor=blue,
	citecolor=blue,
	urlcolor=blue,
	pdftitle={Vollständige Herleitung der Higgs-Masse und Wilson-Koeffizienten}
\hypersetup{
	colorlinks=true,
	linkcolor=blue,
	citecolor=blue,
	urlcolor=blue,
	pdftitle={Vollständiges Teilchenspektrum: Standard-Modell vs T0-Theorie}
\hypersetup{
	colorlinks=true,
	linkcolor=blue,
	citecolor=blue,
	urlcolor=blue,
	pdftitle={Warum Zahlenverhältnisse nicht direkt gekürzt werden dürfen}
\hypersetup{
	colorlinks=true,
	linkcolor=blue,
	citecolor=blue,
	urlcolor=blue,
	pdftitle={Why Numerical Ratios Must Not Be Directly Simplified}
\hypersetup{
	colorlinks=true,
	linkcolor=blue,
	citecolor=blue,
	urlcolor=blue,
}
\hypersetup{
	colorlinks=true,
	linkcolor=blue,
	citecolor=red,
	urlcolor=blue,
	bookmarks=true,
	bookmarksnumbered=true,
	pdfstartview=FitH,
	pdftitle={T0 Model - Field-Theoretic Derivation of the Beta Parameter}
\hypersetup{
	colorlinks=true,
	linkcolor=blue,
	citecolor=red,
	urlcolor=blue,
	bookmarks=true,
	bookmarksnumbered=true,
	pdfstartview=FitH,
	pdftitle={T0-Modell - Feldtheoretische Herleitung des Beta-Parameters}
\hypersetup{
	colorlinks=true,
	linkcolor=blue,
	filecolor=magenta,
	urlcolor=cyan,
}
\hypersetup{
	colorlinks=true,
	linkcolor=blue,
	urlcolor=blue,
	citecolor=blue,
	pdftitle={From Time Dilation to Mass Variation: Mathematical Core Formulations of Time-Mass Duality Theory - Updated Framework}
\hypersetup{
	colorlinks=true,
	linkcolor=blue,
	urlcolor=blue,
	citecolor=blue,
	pdftitle={T0 Model: Detailed Formula for Leptonic Anomalies}
\hypersetup{
	colorlinks=true,
	linkcolor=blue,
	urlcolor=blue,
	citecolor=blue,
	pdftitle={T0 Model: Detaillierte Formel für leptonische Anomalien}
\hypersetup{
	colorlinks=true,
	linkcolor=blue,
	urlcolor=blue,
	citecolor=blue,
	pdftitle={T0 Model: Energy-based Formulas with Quadratic Scaling}
\hypersetup{
	colorlinks=true,
	linkcolor=blue,
	urlcolor=blue,
	citecolor=blue,
	pdftitle={T0 Model: Granulation, Limits and Fundamental Asymmetry}
\hypersetup{
	colorlinks=true,
	linkcolor=blue,
	urlcolor=blue,
	citecolor=blue,
	pdftitle={T0-Modell: Energiebasierte Formeln mit quadratischer Skalierung}
\hypersetup{
	colorlinks=true,
	linkcolor=blue,
	urlcolor=blue,
	citecolor=blue,
	pdftitle={T0-Modell: Granulation, Limits und fundamentale Asymmetrie}
\hypersetup{
	colorlinks=true,
	linkcolor=blue,
	urlcolor=blue,
	citecolor=blue,
	pdftitle={Von Zeitdilatation zu Massenvariation: Mathematische Kernformulierungen der Zeit-Masse-Dualitätstheorie - Aktualisiertes Framework}
\hypersetup{
	colorlinks=true,
	linkcolor=t0blue,
	citecolor=t0blue,
	urlcolor=t0blue,
	pdftitle={T0 Model: Complete Theoretical Summary}
\hypersetup{
	colorlinks=true,
	linkcolor=t0blue,
	citecolor=t0blue,
	urlcolor=t0blue,
	pdftitle={T0 Theory: Resolution of Apparent Instantaneity}
\hypersetup{
	colorlinks=true,
	linkcolor=t0blue,
	citecolor=t0blue,
	urlcolor=t0blue,
	pdftitle={T0 vs Synergetics: Vereinfachung durch natürliche Einheiten}
\hypersetup{
	colorlinks=true,
	linkcolor=t0blue,
	citecolor=t0blue,
	urlcolor=t0blue,
	pdftitle={T0-Modell: Vollständige theoretische Zusammenfassung}
\hypersetup{
	colorlinks=true,
	linkcolor=t0blue,
	citecolor=t0blue,
	urlcolor=t0blue,
	pdftitle={T0-Theorie: Auflösung der scheinbaren Instantanität}
\hypersetup{
	colorlinks=true,
	linkcolor=t0blue,
	citecolor=t0blue,
	urlcolor=t0blue,
	pdftitle={T0-Theorie: Vollständige Dokumentenübersicht}
\hypersetup{
	colorlinks=true,
	linkcolor=t0blue,
	citecolor=t0blue,
	urlcolor=t0blue,
	pdftitle={T0-Theory: Complete Document Overview}
\hypersetup{
	colorlinks=true,
	linkcolor=t0blue,
	citecolor=t0blue,
	urlcolor=t0blue,
}
\hypersetup{
	colorlinks=true,
	linkcolor=t0blue,
	citecolor=t0green,
	urlcolor=t0blue,
	pdftitle={Das verborgene Geheimnis von 1/137}
\hypersetup{
	colorlinks=true,
	linkcolor=t0blue,
	citecolor=t0green,
	urlcolor=t0blue,
	pdftitle={The Hidden Secret of 1/137}
\hypersetup{
    colorlinks=true,
    linkcolor=blue,
    citecolor=blue,
    urlcolor=blue,
    pdftitle={Analyse und Implikationen des MNRAS-Papiers 544 für die T0-Theorie}
\hypersetup{
  colorlinks=true,
  linkcolor=blue,
  citecolor=blue,
  urlcolor=blue
}
\hypersetup{
  colorlinks=true,
  linkcolor=blue,
  citecolor=blue,
  urlcolor=blue,
  pdftitle={T0-Theorie: Ein-Uhr-Metrologie und Drei-Uhren-Experiment}
\hypersetup{
  colorlinks=true,
  linkcolor=blue,
  citecolor=blue,
  urlcolor=blue,
  pdftitle={T0-Theory: Single-Clock Metrology and Three-Clock Experiment}
\hypersetup{
colorlinks=true,
linkcolor=blue,
citecolor=blue,
urlcolor=blue,
pdftitle={Quantenmechanik im T0-Modell: Feldtheoretische Grundlagen}
\hypersetup{
colorlinks=true,
linkcolor=blue,
citecolor=blue,
urlcolor=blue,
pdftitle={T0-Theory: Neutrinos}
\newcommand{\Bzero}{B_0}
\newcommand{\CQCD}{C_{\text{QCD}
\newcommand{\Cconv}{C_{\text{conv}
\newcommand{\Cto}{C_{\text{T0}
\newcommand{\Czero}{C_0}
\newcommand{\DTmu}{D_{T,\mu}
\newcommand{\DcovT}[1]{\partial_\mu #1 + #1 \partial_\mu \Tfield}
\newcommand{\Dfrak}{D_f}
\newcommand{\Df}{D_f}
\newcommand{\DhiggsT}{\Tfield (\partial_\mu + ig A_\mu) \Phi + \Phi \partial_\mu \Tfield}
\newcommand{\EPlanck}{E_P}
\newcommand{\EPlanck}{E_{\text{Pl}
\newcommand{\EPratio}[1]{\frac{#1}
\newcommand{\EP}{E_P}
\newcommand{\EP}{E_{\text{P}
\newcommand{\EW}{E_W}
\newcommand{\EZ}{E_Z}
\newcommand{\Echar}{E_{\text{char}
\newcommand{\Ee}{E_e}
\newcommand{\Efield}{E(x,t)}
\newcommand{\Efield}{E_\text{field}
\newcommand{\Efield}{E_{\text{Feld}
\newcommand{\Efield}{E_{\text{Field}
\newcommand{\Efield}{E_{\text{field}
\newcommand{\Efield}{E}
\newcommand{\Egamma}{E_\gamma}
\newcommand{\Eh}{E_h}
\newcommand{\Emu}{E_\mu}
\newcommand{\Enorm}[1]{E_{\text{norm}
\newcommand{\En}{E_n}
\newcommand{\Ep}{E_p}
\newcommand{\Eratio}[2]{\frac{E_{#1}
\newcommand{\Etau}{E_\tau}
\newcommand{\Evis}{E_{\text{vis}
\newcommand{\Exi}{E_\xi}
\newcommand{\Ezero}{E_0}
\newcommand{\GeV}{\,\text{GeV}
\newcommand{\Gnat}{G_{\text{nat}
\newcommand{\Gsi}{G_{\text{SI}
\newcommand{\Hubble}{H_0}
\newcommand{\Kfrak}{K_{\text{frac}
\newcommand{\Kfrak}{K_{\text{frak}
\newcommand{\Kspec}{K_{\text{spec}
\newcommand{\LCDM}{\Lambda\text{CDM}
\newcommand{\LPlanck}{\ell_{\text{Pl}
\newcommand{\Lag}{\mathcal{L}
\newcommand{\Lambdat}{\Lambda_T}
\newcommand{\Leff}{L_{\text{eff}
\newcommand{\Lorentz}[2]{{\Lambda^\mu{}
\newcommand{\Lp}{L_{\text{P}
\newcommand{\Lxi}{L_\xi}
\newcommand{\Lzero}{L_0}
\newcommand{\MPl}{M_{\text{Pl}
\newcommand{\MSbar}{\overline{\text{MS}
\newcommand{\MeV}{\,\text{MeV}
\newcommand{\Mpl}{M_{\text{Pl}
\newcommand{\OmegaDM}{\Omega_{\text{DM}
\newcommand{\OmegaLambda}{\Omega_{\Lambda}
\newcommand{\Omegab}{\Omega_b}
\newcommand{\Phiphoton}{\Phi_{\text{photon}
\newcommand{\Ricci}{R_{\mu\nu}
\newcommand{\Riem}{R^\rho{}
\newcommand{\Rzero}{R_\infty}
\newcommand{\Scal}{R}
\newcommand{\SynchPower}{P_{\text{synch}
\newcommand{\TPlanck}{t_{\text{Pl}
\newcommand{\Tfieldt}{T(\vec{x}
\newcommand{\Tfieldt}{T(x,t)}
\newcommand{\Tfield}{T(x)}
\newcommand{\Tfield}{T(x,t)}
\newcommand{\Tfield}{T_{\text{field}
\newcommand{\Tfield}{T}
\newcommand{\Tfield}{\mathcal{T}
\newcommand{\Tzerot}{T_0(\Tfield)}
\newcommand{\Tzero}{T_0}
\newcommand{\Weyl}{C^\rho{}
\newcommand{\ZPinch}{J \times B = \nabla p}
\newcommand{\aleph}{\aleph}
\newcommand{\alphaEMSI}{\alpha_{\text{EM,SI}
\newcommand{\alphaEMnat}{\alpha_{\text{EM,nat}
\newcommand{\alphaEM}{\alpha_{\text{EM}
\newcommand{\alphaEM}{\ensuremath{\alpha_{\text{EM}
\newcommand{\alphaQCD}{\alpha_s}
\newcommand{\alphaQED}{\alpha_{\text{QED}
\newcommand{\alphaSI}{\alpha_{\text{SI}
\newcommand{\alphaT}{\alpha_{\text{T}
\newcommand{\alphaWSI}{\alpha_{\text{W,SI}
\newcommand{\alphaWnat}{\alpha_{\text{W,nat}
\newcommand{\alphaW}{\alpha_{\text{W}
\newcommand{\alphaem}{\alpha_{EM}
\newcommand{\alphaem}{\alpha}
\newcommand{\alphafine}{\alpha}
\newcommand{\alphagem}{\alpha}
\newcommand{\alphanat}{\alpha_{\text{nat}
\newcommand{\alphapar}{\alpha}
\newcommand{\betaTSI}{\beta_{\text{T,SI}
\newcommand{\betaTnat}{\beta_{\text{T,nat}
\newcommand{\betaT}{\beta_T}
\newcommand{\betaT}{\beta_{T}
\newcommand{\betaT}{\beta_{\text{T}
\newcommand{\betaT}{\ensuremath{\beta_T}
\newcommand{\betapar}{\beta}
\newcommand{\calL}{\mathcal{L}
\newcommand{\checked}{\checkmark}
\newcommand{\checkmarkx}{\checkmark}
\newcommand{\dTdt}{\frac{d\Tfieldt}
\newcommand{\deltaE}{\delta E}
\newcommand{\deltafield}{\ensuremath{\delta m}
\newcommand{\deltam}{\delta m}
\newcommand{\deq}{\displaystyle}
\newcommand{\docref}[1]{\texttt{#1}
\newcommand{\eV}{\,\text{eV}
\newcommand{\epsilonT}{\varepsilon_T}
\newcommand{\epsilonzero}{\varepsilon_0}
\newcommand{\etavis}{\eta_{\text{visual}
\newcommand{\e}{\mathrm{e}
\newcommand{\gW}{g_W}
\newcommand{\gammaf}{\gamma_{\text{Lorentz}
\newcommand{\gammamu}{\gamma^\mu}
\newcommand{\gs}{g_s}
\newcommand{\inftytext}{$\infty$}
\newcommand{\interval}[2]{#1:#2}
\newcommand{\kfrac}{K_{\text{frak}
\newcommand{\lP}{\ell_{\text{P}
\newcommand{\lP}{l_P}
\newcommand{\lambdah}{\ensuremath{\lambda_h}
\newcommand{\lambdah}{\lambda_h}
\newcommand{\lambdazero}{\lambda_0}
\newcommand{\mP}{m_{\text{P}
\newcommand{\mfield}{m(x,t)}
\newcommand{\mfield}{m}
\newcommand{\mh}{m_h}
\newcommand{\micrometer}{\ensuremath{\mu}
\newcommand{\mikrometer}{\ensuremath{\mu}
\newcommand{\myRightarrow}{\ensuremath{\Rightarrow}
\newcommand{\myapprox}{\ensuremath{\approx}
\newcommand{\myomega}{\ensuremath{\omega}
\newcommand{\myphi}{\ensuremath{\phi}
\newcommand{\mypi}{\ensuremath{\pi}
\newcommand{\mypropto}{\ensuremath{\propto}
\newcommand{\myrightarrow}{\ensuremath{\rightarrow}
\newcommand{\mysim}{\ensuremath{\sim}
\newcommand{\mysqrt}{\ensuremath{\sqrt}
\newcommand{\mytimes}{\ensuremath{\times}
\newcommand{\natunits}{\hbar = c = G = k_B = 1}
\newcommand{\natunits}{\text{(nat. Einh.)}
\newcommand{\natunits}{\text{(nat. units)}
\newcommand{\nulep}{\nu}
\newcommand{\nuzero}{\nu_0}
\newcommand{\partialop}{\ensuremath{\partial}
\newcommand{\pdTdt}{\frac{\partial\Tfieldt}
\newcommand{\pdTdx}{\nabla\Tfieldt}
\newcommand{\phiT}{\phi}
\newcommand{\pichar}{\pi}
\newcommand{\primrel}[1]{\mathbf{#1}
\newcommand{\rhoCMB}{\rho_{\text{CMB}
\newcommand{\rhoCasimir}{\rho_{\text{Casimir}
\newcommand{\rhoE}{\rho_E}
\newcommand{\rhofield}{\ensuremath{\rho}
\newcommand{\rzero}{r_0}
\newcommand{\slashk}{\cancel{k}
\newcommand{\slashp}{\cancel{p}
\newcommand{\slashq}{\cancel{q}
\newcommand{\tP}{t_P}
\newcommand{\tP}{t_{\text{P}
\newcommand{\tablescale}{0.9}
\newcommand{\tzero}{t_0}
\newcommand{\vect}[1]{\boldsymbol{#1}
\newcommand{\vecx}{\vec{x}
\newcommand{\vh}{v}
\newcommand{\vr}{\vec{r}
\newcommand{\warningx}{\color{red}
\newcommand{\warningx}{\textbf{!}
\newcommand{\warningx}{{\color{red}
\newcommand{\xiT}{\xi}
\newcommand{\xiconst}{\xi = \frac{4}
\newcommand{\xicoupling}{f(E/\Exi)}
\newcommand{\xigeom}{\xi_{\text{geom}
\newcommand{\xigeom}{\xi}
\newcommand{\xikonst}{\xi = \frac{4}
\newcommand{\xiparticle}{\xi_{\text{particle}
\newcommand{\xipar}{\ensuremath{\xi}
\newcommand{\xipar}{\xi_0}
\newcommand{\xipar}{\xi}
\newcommand{\xirat}{\xi_{\text{ratio}
\newtheorem{axiom}{Axiom}
\newtheorem{category}{Category-Theoretic Basis}
\newtheorem{category}{Kategorientheoretische Basis}
\newtheorem{corollary}[theorem]{Corollary}
\newtheorem{corollary}[theorem]{Korollar}
\newtheorem{corollary}{Corollary}
\newtheorem{corollary}{Korollar}
\newtheorem{definition}[theorem]{Definition}
\newtheorem{definition}{Definition}
\newtheorem{discovery}{Discovery}
\newtheorem{discovery}{Neue Entdeckung}
\newtheorem{discovery}{New Discovery}
\newtheorem{discovery}{Revolutionary Discovery}
\newtheorem{entdeckung}{Entdeckung}
\newtheorem{entdeckung}{Revolutionäre Entdeckung}
\newtheorem{erkenntnis}{Erkenntnis}
\newtheorem{erkenntnis}{Schlüsselerkenntnis}
\newtheorem{example}[theorem]{Beispiel}
\newtheorem{example}[theorem]{Example}
\newtheorem{example}{Beispiel}
\newtheorem{example}{Example}
\newtheorem{insight}{Central Insight}
\newtheorem{insight}{Insight}
\newtheorem{insight}{Key Insight}
\newtheorem{insight}{Wichtige Einsicht}
\newtheorem{insight}{Zentrale Einsicht}
\newtheorem{lemma}[theorem]{Lemma}
\newtheorem{lemma}{Lemma}
\newtheorem{principle}{Fundamental Principle}
\newtheorem{principle}{Fundamentales Prinzip}
\newtheorem{principle}{Grundlegendes Prinzip}
\newtheorem{principle}{Principle}
\newtheorem{principle}{Prinzip}
\newtheorem{prinzip}{Grundprinzip}
\newtheorem{proof_step}{Beweisschritt}
\newtheorem{proof_step}{Proof Step}
\newtheorem{proposition}[theorem]{Proposition}
\newtheorem{proposition}{Proposition}
\newtheorem{remark}[theorem]{Bemerkung}
\newtheorem{remark}[theorem]{Remark}
\newtheorem{theorem}{Theorem}
\newtheorem{warning}[theorem]{Warning}
\newtheorem{warning}[theorem]{Warnung}
\newunicodechar{±}{\ensuremath{\pm}
\newunicodechar{×}{\ensuremath{\times}
\newunicodechar{÷}{\ensuremath{\div}
\newunicodechar{ħ}{\ensuremath{\hbar}
\newunicodechar{Α}{\ensuremath{A}
\newunicodechar{Β}{\ensuremath{B}
\newunicodechar{Γ}{\ensuremath{\Gamma}
\newunicodechar{Δ}{\ensuremath{\Delta}
\newunicodechar{Ε}{\ensuremath{E}
\newunicodechar{Ζ}{\ensuremath{Z}
\newunicodechar{Η}{\ensuremath{H}
\newunicodechar{Θ}{\ensuremath{\Theta}
\newunicodechar{Ι}{\ensuremath{I}
\newunicodechar{Κ}{\ensuremath{K}
\newunicodechar{Λ}{\ensuremath{\Lambda}
\newunicodechar{Μ}{\ensuremath{M}
\newunicodechar{Ν}{\ensuremath{N}
\newunicodechar{Ξ}{\ensuremath{\Xi}
\newunicodechar{Ο}{\ensuremath{O}
\newunicodechar{Π}{\ensuremath{\Pi}
\newunicodechar{Ρ}{\ensuremath{P}
\newunicodechar{Σ}{\ensuremath{\Sigma}
\newunicodechar{Τ}{\ensuremath{T}
\newunicodechar{Υ}{\ensuremath{\Upsilon}
\newunicodechar{Φ}{\ensuremath{\Phi}
\newunicodechar{Χ}{\ensuremath{X}
\newunicodechar{Ψ}{\ensuremath{\Psi}
\newunicodechar{Ω}{\ensuremath{\Omega}
\newunicodechar{α}{\ensuremath{\alpha}
\newunicodechar{β}{\ensuremath{\beta}
\newunicodechar{γ}{\ensuremath{\gamma}
\newunicodechar{δ}{\ensuremath{\delta}
\newunicodechar{ε}{\ensuremath{\varepsilon}
\newunicodechar{ζ}{\ensuremath{\zeta}
\newunicodechar{η}{\ensuremath{\eta}
\newunicodechar{θ}{\ensuremath{\theta}
\newunicodechar{ι}{\ensuremath{\iota}
\newunicodechar{κ}{\ensuremath{\kappa}
\newunicodechar{λ}{\ensuremath{\lambda}
\newunicodechar{μ}{\ensuremath{\mu}
\newunicodechar{ν}{\ensuremath{\nu}
\newunicodechar{ξ}{\ensuremath{\xi}
\newunicodechar{ο}{\ensuremath{o}
\newunicodechar{π}{\ensuremath{\pi}
\newunicodechar{ρ}{\ensuremath{\rho}
\newunicodechar{σ}{\ensuremath{\sigma}
\newunicodechar{τ}{\ensuremath{\tau}
\newunicodechar{υ}{\ensuremath{\upsilon}
\newunicodechar{φ}{\ensuremath{\phi}
\newunicodechar{φ}{\ensuremath{\varphi}
\newunicodechar{χ}{\ensuremath{\chi}
\newunicodechar{ψ}{\ensuremath{\psi}
\newunicodechar{ω}{\ensuremath{\omega}
\newunicodechar{←}{\ensuremath{\leftarrow}
\newunicodechar{→}{\ensuremath{\rightarrow}
\newunicodechar{↔}{\ensuremath{\leftrightarrow}
\newunicodechar{⇐}{\ensuremath{\Leftarrow}
\newunicodechar{⇒}{\ensuremath{\Rightarrow}
\newunicodechar{⇔}{\ensuremath{\Leftrightarrow}
\newunicodechar{∂}{\ensuremath{\partial}
\newunicodechar{∅}{\ensuremath{\emptyset}
\newunicodechar{∇}{\ensuremath{\nabla}
\newunicodechar{∈}{\ensuremath{\in}
\newunicodechar{∉}{\ensuremath{\notin}
\newunicodechar{∏}{\ensuremath{\prod}
\newunicodechar{∑}{\ensuremath{\sum}
\newunicodechar{√}{\ensuremath{\sqrt}
\newunicodechar{∝}{\ensuremath{\propto}
\newunicodechar{∞}{\ensuremath{\infty}
\newunicodechar{∩}{\ensuremath{\cap}
\newunicodechar{∪}{\ensuremath{\cup}
\newunicodechar{∫}{\ensuremath{\int}
\newunicodechar{≈}{\ensuremath{\approx}
\newunicodechar{≠}{\ensuremath{\neq}
\newunicodechar{≤}{\ensuremath{\leq}
\newunicodechar{≥}{\ensuremath{\geq}
\newunicodechar{★}{\ensuremath{\star}
\newunicodechar{✓}{\checkmark}
\pgfplotsset{compat=1.17}
\pgfplotsset{compat=1.18}
\renewcommand{\cftchapfont}{\large\bfseries\color{blue}
\renewcommand{\cftchappagefont}{\large\bfseries\color{blue}
\renewcommand{\cftsecfont}{\bfseries}
\renewcommand{\cftsecfont}{\color{blue}
\renewcommand{\cftsecfont}{\large\bfseries\color{blue}
\renewcommand{\cftsecpagefont}{\bfseries}
\renewcommand{\cftsecpagefont}{\color{blue}
\renewcommand{\cftsecpagefont}{\large\bfseries\color{blue}
\renewcommand{\cftsubsecfont}{\color{blue!80!black}
\renewcommand{\cftsubsecfont}{\color{blue}
\renewcommand{\cftsubsecpagefont}{\color{blue!80!black}
\renewcommand{\cftsubsecpagefont}{\color{blue}
\renewcommand{\cftsubsubsecfont}{\color{blue!60!black}
\renewcommand{\cftsubsubsecfont}{\color{blue}
\renewcommand{\cftsubsubsecpagefont}{\color{blue!60!black}
\renewcommand{\cftsubsubsecpagefont}{\color{blue}
\renewcommand{\cfttoctitlefont}{\huge\bfseries\color{blue}
\renewcommand{\cfttoctitlefont}{\huge\bfseries}
\renewcommand{\familydefault}{\sfdefault}
\renewcommand{\footrulewidth}{0.4pt}
\renewcommand{\headrulewidth}{0.4pt}
\sisetup{locale = DE, group-separator = {.}
\sisetup{locale = DE}
\usetikzlibrary{arrows.meta,positioning,shapes.geometric}
\usetikzlibrary{decorations.pathmorphing, patterns, shapes.arrows}
\usetikzlibrary{intersections}
\usetikzlibrary{positioning, arrows.meta}
\usetikzlibrary{positioning, arrows}
\usetikzlibrary{positioning, shapes.geometric, arrows.meta}
\usetikzlibrary{positioning,shapes,arrows}

% Common settings
\setlength{\headheight}{15pt}
\pgfplotsset{compat=1.18}
\usetikzlibrary{positioning,shapes,arrows,arrows.meta}

% Hyperref setup
\hypersetup{
    colorlinks=true,
    linkcolor=blue,
    citecolor=blue,
    urlcolor=blue
}


\title{E-mc2 En}
\author{Johann Pascher}
\date{\today}

\begin{document}

\maketitle
\tableofcontents

\title{E=mc² = E=m: The Constants Illusion Exposed \\
		Why Einstein's c-constant conceals the fundamental error \\
		\large From Dynamic Ratios to the Constants Illusion}
	\author{Johann Pascher\\
		Department of Communications Engineering, \\Higher Technical Federal Institute (HTL), Leonding, Austria\\
		\texttt{johann.pascher@gmail.com}}
	\date{\today}
	
	\maketitle
	
	\begin{abstract}
		This work reveals the central point of Einstein's relativity theory: E=mc² is mathematically identical to E=m. The only difference lies in Einstein's treatment of c as a "constant" instead of a dynamic ratio. By fixing c = 299,792,458 m/s, the natural time-mass duality T·m = 1 is artificially "frozen," leading to apparent complexity. The T0 theory shows: c is not a fundamental law of nature, but only a ratio that must be variable if time is variable. Einstein's error was not E=mc² itself, but the constant-setting of c.
	\end{abstract}
	
	\tableofcontents
	\newpage
	
	# The Central Thesis: E=mc² = E=m
	
	\begin{tcolorbox}[colback=red!5!white,colframe=red!75!black,title=The Fundamental Recognition]
		\textbf{E=mc² and E=m are mathematically identical!}
		
		The only difference: Einstein treats c as a "constant," although c is a dynamic ratio.
		
		\textbf{Einstein's error}: c = 299,792,458 m/s = constant
		
		\textbf{T0 truth}: c = L/T = variable ratio
	\end{tcolorbox}
	
	## The Mathematical Identity
	
	\textbf{In natural units}:
	
```math-equation

		E = mc^2 = m \times c^2 = m \times 1^2 = m
	
```

	
	\textbf{This is not an approximation - this is exactly the same equation!}
	
	## What is c really?
	
	
```math-equation

		c = \frac{\text{Length}}{\text{Time}} = \frac{L}{T}
	
```

	
	\textbf{c is a ratio, not a natural constant!}
	
	# Einstein's Fundamental Error: The Constant-Setting
	
	## The Act of Constant-Setting
	
	Einstein set: $c = 299,792,458$ m/s = \textbf{constant}
	
	\textbf{What does this mean?}
	
```math-equation

		c = \frac{L}{T} = \text{constant} \quad \Rightarrow \quad \frac{L}{T} = \text{fixed}
	
```

	
	\textbf{Implication}: If L and T can vary, their \textbf{ratio} must remain constant.
	
	## The Problem of Time Variability
	
	\textbf{Einstein recognized himself}: Time dilates!
	
```math-equation

		t' = \gamma t \quad \text{(time is variable)}
	
```

	
	\textbf{But simultaneously he claimed}: 
	
```math-equation

		c = \frac{L}{T} = \text{constant}
	
```

	
	\textbf{This is a logical contradiction!}
	
	## The T0 Resolution
	
	\textbf{T0 insight}: $\Tfield \cdot m = 1$
	
	This means:
	
		- Time $\Tfield$ \textbf{must} be variable (coupled to mass)
		- Therefore $c = L/T$ \textbf{cannot} be constant
		- $c$ is a \textbf{dynamic ratio}, not a constant
	
	
	# The Constants Illusion: How it Works
	
	## The Mechanism of the Illusion
	
	\textbf{Step 1}: Einstein sets c = constant
	
```math-equation

		c = 299,792,458 \text{ m/s} = \text{fixed}
	
```

	
	\textbf{Step 2}: Time becomes "frozen" by this
	
```math-equation

		T = \frac{L}{c} = \frac{L}{\text{constant}} = \text{apparently determined}
	
```

	
	\textbf{Step 3}: Time dilation becomes "mysterious effect"
	
```math-equation

		t' = \gamma t \quad \text{(why? $\rightarrow$ complicated relativity theory)}
	
```

	
	## What Really Happens (T0 View)
	
	\textbf{Reality}: Time is naturally variable through $\Tfield \cdot m = 1$
	
	\textbf{Einstein's constant-setting} "freezes" this natural variability artificially
	
	\textbf{Result}: One needs complicated theory to repair the "frozen" dynamics
	
	# c as Ratio vs. c as Constant
	
	## c as Natural Ratio (T0)
	
	
```math-equation

		c(x,t) = \frac{L(x,t)}{T(x,t)}
	
```

	
	\textbf{Properties}:
	
		- $c$ varies with location and time
		- $c$ follows the time-mass duality
		- No artificial constants
		- Natural simplicity: $E = m$
	
	
	## c as Artificial Constant (Einstein)
	
	
```math-equation

		c = 299,792,458 \text{ m/s} = \text{constant everywhere}
	
```

	
	\textbf{Problems}:
	
		- Contradiction to time dilation
		- Artificial "freezing" of time dynamics
		- Complicated repair mathematics needed
		- Inflated formula: $E = mc^2$
	
	
	# The Time Dilation Paradox
	
	## Einstein's Contradiction Exposed
	
	\textbf{Einstein claims simultaneously}:
	
```math-align

		c &= \text{constant} \\
		t' &= \gamma t \quad \text{(time varies)}
	
```

	
	\textbf{But}:
	
```math-equation

		c = \frac{L}{T} \quad \text{and} \quad T \text{ varies} \quad \Rightarrow \quad c \text{ cannot be constant!}
	
```

	
	## Einstein's Hidden Solution
	
	Einstein "solves" the contradiction through:
	
		- Complicated Lorentz transformations
		- Mathematical formalisms
		- Space-time constructions
		- \textbf{But the logical contradiction remains!}
	
	
	## T0's Natural Solution
	
	\textbf{No contradiction in T0}:
	
```math-equation

		\Tfield \cdot m = 1 \quad \Rightarrow \quad \text{time is naturally variable}
	
```

	
	
```math-equation

		c = \frac{L}{T} \quad \Rightarrow \quad \text{c is naturally variable}
	
```

	
	\textbf{No constant-setting $\rightarrow$ No contradictions $\rightarrow$ No complicated repair mathematics}
	
	# The Mathematical Demonstration
	
	## From E=mc² to E=m
	
	\textbf{Starting equation}: $E = mc^2$
	
	\textbf{c in natural units}: $c = 1$
	
	\textbf{Substitution}:
	
```math-equation

		E = mc^2 = m \times 1^2 = m
	
```

	
	\textbf{Result}: $E = m$
	
	## The Reverse Direction: From E=m to E=mc²
	
	\textbf{Starting equation}: $E = m$
	
	\textbf{Artificial constant introduction}: $c = 299,792,458$ m/s
	
	\textbf{Inflating the equation}:
	
```math-equation

		E = m = m \times 1 = m \times \frac{c^2}{c^2} = m \times c^2 \times \frac{1}{c^2}
	
```

	
	\textbf{If one defines $c^2$ as "conversion factor"}:
	
```math-equation

		E = mc^2
	
```

	
	\textbf{This shows}: $E = mc^2$ is only $E = m$ with \textbf{artificial inflation factor} $c^2$!
	
	# The Arbitrariness of Constant Choice: c or Time?
	
	## Einstein's Arbitrary Decision
	
	\begin{tcolorbox}[colback=orange!5!white,colframe=orange!75!black,title=The Fundamental Choice Option]
		\textbf{One can choose what should be "constant"!}
		
		\textbf{Option 1 (Einstein's choice)}: c = constant $\rightarrow$ time becomes variable
		
		\textbf{Option 2 (alternative)}: time = constant $\rightarrow$ c becomes variable
		
		\textbf{Both describe the same physics!}
	\end{tcolorbox}
	
	## Option 1: Einstein's c-constant
	
	\textbf{Einstein chose}:
	
```math-align

		c &= 299,792,458 \text{ m/s} = \text{constant (defined)} \\
		t' &= \gamma t \quad \text{(time becomes automatically variable)}
	
```

	
	\textbf{Language convention}:
	
		- "Speed of light is universally constant"
		- "Time dilates in strong gravitational fields"
		- "Clocks run slower at high velocities"
	
	
	## Option 2: Time-constant (Einstein could have chosen)
	
	\textbf{Alternative choice}:
	
```math-align

		t &= \text{constant (defined)} \\
		c(x,t) &= \frac{L(x,t)}{t} = \text{variable}
	
```

	
	\textbf{Alternative language convention}:
	
		- "Time flows equally everywhere"
		- "Speed of light varies with location"
		- "Light becomes slower in strong gravitational fields"
	
	
	## Mathematical Equivalence of Both Options
	
	\textbf{Both descriptions are mathematically identical}:
	
	\begin{table}[htbp]
		\centering
		\begin{tabular}{|l|c|c|}
			\hline
			\textbf{Phenomenon} & \textbf{Einstein view} & \textbf{Time-constant view} \\
			\hline
			Gravitation & Time slows down & Light slows down \\
			Velocity & Time dilation & c-variation \\
			GPS correction & "Clocks run differently" & "c is different" \\
			Measurements & Same numbers & Same numbers \\
			\hline
		\end{tabular}
		\caption{Two views, identical physics}
	\end{table}
	
	## Why Einstein Chose Option 1
	
	\textbf{Historical reasons for Einstein's decision}:
	
		- \textbf{Michelson-Morley}: c seemed locally constant
		- \textbf{Aesthetics}: "Universal constant" sounded elegant
		- \textbf{Tradition}: Newtonian constant physics
		- \textbf{Conceivability}: c-constancy easier to imagine than time constancy
		- \textbf{Authority effect}: Einstein's prestige fixed this choice
	
	
	\textbf{But it was only a convention, not a natural law!}
	
	## T0's Overcoming of Both Options
	
	\textbf{T0 shows}: Both choices are arbitrary!
	
	
```math-equation

		\Tfield \cdot m = 1 \quad \text{(natural duality without constant constraint)}
	
```

	
	\textbf{T0 insight}:
	
		- \textbf{Neither} c nor time are "really" constant
		- \textbf{Both} are aspects of the same T·m dynamics
		- \textbf{Constancy} is only definition convention
		- \textbf{E = m} is the constant-free truth
	
	
	## Liberation from Constant Constraint
	
	\textbf{Instead of choosing between}:
	
		- c constant, time variable (Einstein)
		- Time constant, c variable (alternative)
	
	
	\textbf{T0 chooses}:
	
		- \textbf{Both dynamically coupled} via T·m = 1
		- \textbf{No arbitrary fixations}
		- \textbf{Natural ratios} instead of artificial constants
	
	
	# The Reference Point Revolution: Earth $\rightarrow$ Sun $\rightarrow$ Nature
	
	## The Reference Point Analogy: Geocentric $\rightarrow$ Heliocentric $\rightarrow$ T0
	
	\begin{tcolorbox}[colback=blue!5!white,colframe=blue!75!black,title=The Reference Point Revolution: From Earth $\rightarrow$ Sun $\rightarrow$ Nature]
		\textbf{Geocentric (Ptolemy)}: Earth at center \\
		- Complicated epicycles needed \\
		- Works, but artificially complicated \\
		
		\textbf{Heliocentric (Copernicus)}: Sun at center \\
		- Simple ellipses \\
		- Much more elegant and simple \\
		
		\textbf{T0-centric}: Natural ratios at center \\
		- $\Tfield \cdot m = 1$ (natural reference point) \\
		- Even more elegant: $E = m$
	\end{tcolorbox}
	
	\textbf{Einstein's c-constant corresponds to the geocentric system}:
	
		- \textbf{Human} reference point at center (like Earth at center)
		- \textbf{Complicated} mathematics needed (like epicycles)
		- \textbf{Works} locally, but artificially inflated
	
	
	\textbf{T0's natural ratios correspond to the heliocentric system}:
	
		- \textbf{Natural} reference point at center (like Sun at center)
		- \textbf{Simple} mathematics (like ellipses)
		- \textbf{Universally} valid and elegant
	
	
	## Why We Need Reference Points
	
	\textbf{Reference points are necessary and natural}:
	
		- \textbf{For measurements}: We need standards for comparison
		- \textbf{For communication}: Common basis for exchange
		- \textbf{For technology}: Practical applications require units
		- \textbf{For science}: Reproducible experiments need standards
	
	
	\textbf{The question is not WHETHER, but WHICH reference point}:
	
	\begin{table}[htbp]
		\centering
		\begin{tabular}{|l|c|c|c|}
			\hline
			\textbf{System} & \textbf{Reference Point} & \textbf{Complexity} & \textbf{Elegance} \\
			\hline
			Geocentric & Earth & Epicycles & Low \\
			Heliocentric & Sun & Ellipses & High \\
			Einstein & c-constant & Relativity theory & Medium \\
			T0 & $\Tfield \cdot m = 1$ & $E = m$ & Maximum \\
			\hline
		\end{tabular}
		\caption{Reference point systems comparison}
	\end{table}
	
	## The Right vs. Wrong Reference Point
	
	\textbf{Einstein's error was not to choose a reference point}:
	
		- \textbf{But to choose the wrong reference point!}
	
	
	\textbf{Wrong reference point (Einstein)}: c = 299,792,458 m/s = constant
	
		- Based on human definition
		- Leads to complicated mathematics
		- Creates logical contradictions
	
	
	\textbf{Right reference point (T0)}: $\Tfield \cdot m = 1$
	
		- Based on natural ratio
		- Leads to simple mathematics: $E = m$
		- No contradictions, pure elegance
	
	
	# When Something Becomes "Constant"
	
	## The Fundamental Reference Point Problem
	
	\begin{tcolorbox}[colback=red!5!white,colframe=red!75!black,title=The Reference Point Illusion]
		\textbf{Something only becomes "constant" when we define a reference point!}
		
		\textbf{Without reference point}: All ratios are relative and dynamic
		
		\textbf{With reference point}: One ratio becomes artificially "fixed"
		
		\textbf{Einstein's error}: He defined an absolute reference point for c
	\end{tcolorbox}
	
	## The Natural Stage: Everything is Relative
	
	\textbf{Before any reference point definition}:
	
```math-align

		c_1 &= \frac{L_1}{T_1} \\
		c_2 &= \frac{L_2}{T_2} \\
		c_3 &= \frac{L_3}{T_3} \\
		&\vdots
	
```

	
	\textbf{All c-values are relative to each other}. None is "constant".
	
	## The Moment of Reference Point Setting
	
	\textbf{Einstein's fatal step}:
	
```math-equation

		\text{"I define: } c = 299,792,458 \text{ m/s = reference point"}
	
```

	
	\textbf{What happens at this moment}:
	
		- An \textbf{arbitrary reference point} is set
		- All other c-values are measured relative to this
		- The \textbf{dynamic ratio} becomes a "constant"
		- The \textbf{natural relativity} is artificially "frozen"
	
	
	## The Reference Point Problematic
	
	\textbf{Every reference point is arbitrary}:
	
		- Why 299,792,458 m/s and not 300,000,000 m/s?
		- Why in m/s and not in other units?
		- Why measured on Earth and not in space?
		- Why at this time and not at another?
	
	
	## T0's Reference Point-Free Physics
	
	\textbf{T0 eliminates all reference points}:
	
```math-equation

		\Tfield \cdot m = 1 \quad \text{(universal relation without reference point)}
	
```

	
	
		- No arbitrary fixations
		- All ratios remain dynamic
		- Natural relativity is preserved
		- Fundamental simplicity: $E = m$
	
	
	## Example: The Meter Definition
	
	\textbf{Historical development of meter definition}:
	
		- \textbf{1793}: 1 meter = 1/10,000,000 of Earth meridian (Earth reference point)
		- \textbf{1889}: 1 meter = prototype meter in Paris (object reference point)  
		- \textbf{1960}: 1 meter = 1,650,763.73 wavelengths of krypton-86 (atom reference point)
		- \textbf{1983}: 1 meter = distance light travels in 1/299,792,458 s (c reference point)
	
	
	\textbf{What does this show?}
	
		- Each definition is \textbf{human arbitrariness}
		- The \textbf{reference point} changes with human technology
		- There is \textbf{no "natural" length unit} - only human agreements
		- \textbf{Humans make c "constant" by definition} - not nature!
	
	
	## The Circular Error: Humans Define Their Own "Constants"
	
	\textbf{In 1983 humans defined}:
	
```math-equation

		1 \text{ meter} = \frac{1}{299,792,458} \times c \times 1 \text{ second}
	
```

	
	\textbf{This makes c automatically "constant"} - through human definition, not through natural law:
	
```math-equation

		c = \frac{299,792,458 \text{ meters}}{1 \text{ second}} = 299,792,458 \text{ m/s}
	
```

	
	\textbf{Circular reasoning}: Humans define c as constant and then "measure" a constant!
	
	\textbf{Nature is not asked in this process!}
	
	## T0's Resolution of the Reference Point Illusion
	
	\textbf{T0 recognizes}:
	
		- \textbf{Definition $\neq$ natural law}
		- \textbf{Measurement reference point $\neq$ physical constant}
		- \textbf{Practical agreement $\neq$ fundamental truth}
	
	
	\textbf{T0 solution}:
	
```math-align

		\text{For measurements:} \quad &\text{Use practical reference points} \\
		\text{For natural laws:} \quad &\text{Use reference point-free relations}
	
```

	
	# Why c-Constancy is Not Provable
	
	## The Fundamental Measurement Problem
	
	\textbf{To measure c, we need}:
	
```math-equation

		c = \frac{L}{T}
	
```

	
	\textbf{But}: We measure L and T with \textbf{the same physical processes} that depend on c!
	
	\textbf{Circular problem}:
	
		- Light measures distances $\rightarrow$ c determines L
		- Atomic clocks use EM transitions $\rightarrow$ c influences T
		- Then we measure c = L/T $\rightarrow$ \textbf{We measure c with c!}
	
	
	## The Gauge Definition Problem
	
	\textbf{Since 1983}: 1 meter = distance light travels in 1/299,792,458 s
	
	
```math-equation

		c = 299,792,458 \text{ m/s} \quad \text{(not measured, but defined!)}
	
```

	
	\textbf{One cannot "prove" what one has defined!}
	
	## The Systematic Compensation Problem
	
	\textbf{If c varies, ALL measuring devices vary equally}:
	
		- \textbf{Laser interferometers}: use light (c-dependent)
		- \textbf{Atomic clocks}: use EM transitions (c-dependent)
		- \textbf{Electronics}: uses EM signals (c-dependent)
	
	
	\textbf{Result}: All devices \textbf{automatically compensate} the c-variation!
	
	## The Burden of Proof Problem
	
	\textbf{Scientifically correct}:
	
		- One \textbf{cannot prove} that something is constant
		- One can only show that it \textbf{appears constant within measurement precision}
		- \textbf{Each new precision level} could show variation
	
	
	\textbf{Einstein's "c-constancy" was belief, not proof!}
	
	## T0 Prediction for Precise Measurements
	
	\textbf{T0 predicts}: At highest precision one will find:
	
```math-equation

		c(x,t) = c_0 \left(1 + \xipar \times \frac{\Tfield(x,t) - \Tfield_0}{\Tfield_0}\right)
	
```

	
	with $\xipar = 1.33 \times 10^{-4}$ (T0 parameter)
	
	\textbf{c varies tiny ($\sim$10$^{-15}$), but measurable in principle!}
	
	# Ontological Consideration: Calculations as Constructs
	
	## The Fundamental Epistemological Limit
	
	\begin{tcolorbox}[colback=purple!5!white,colframe=purple!75!black,title=Ontological Truth]
		\textbf{All calculations are human constructs!}
		
		They can \textbf{at best} give a certain idea of reality.
		
		\textbf{That calculations are internally consistent proves little} about actual reality.
		
		\textbf{Mathematical consistency $\neq$ ontological truth}
	\end{tcolorbox}
	
	## Einstein's Construct vs. T0's Construct
	
	\textbf{Both are human thought structures}:
	
	\textbf{Einstein's construct}:
	
		- E = mc² (mathematically consistent)
		- Relativity theory (internally coherent)
		- 10 field equations (work computationally)
		- \textbf{But}: Based on arbitrary c-constant setting
	
	
	\textbf{T0's construct}:
	
		- E = m (mathematically simpler)
		- T·m = 1 (internally coherent)
		- $\partial^2 E = 0$ (works computationally)
		- \textbf{But}: Also only a human thought model
	
	
	## The Ontological Relativity
	
	\textbf{What is "really" real?}
	
		- \textbf{Einstein's space-time}? (construct)
		- \textbf{T0's energy field}? (construct)
		- \textbf{Newton's absolute time}? (construct)
		- \textbf{Quantum mechanics' probabilities}? (construct)
	
	
	\textbf{All are human interpretive frameworks of the inaccessible reality!}
	
	## Why T0 is Still "Better"
	
	\textbf{Not because of "absolute truth," but because of}:
	
	\textbf{1. Simplicity (Occam's Razor)}:
	
		- E = m is simpler than E = mc²
		- One equation is simpler than 10 equations
		- Fewer arbitrary assumptions
	
	
	\textbf{2. Consistency}:
	
		- No logical contradictions (like Einstein's)
		- No constant arbitrariness
		- Unified thought structure
	
	
	\textbf{3. Predictive power}:
	
		- Testable predictions
		- Fewer free parameters
		- Clearer experimental distinction
	
	
	\textbf{4. Aesthetics}:
	
		- Mathematical elegance
		- Conceptual clarity
		- Unity
	
	
	## The Epistemological Humility
	
	\textbf{T0 does NOT claim to be "absolute truth."}
	
	\textbf{T0 only says}:
	
		- "Here is a \textbf{simpler} construct"
		- "With \textbf{fewer} arbitrary assumptions"
		- "That is \textbf{more consistent} than Einstein's construct"
		- "And makes \textbf{more testable} predictions"
	
	
	\textbf{But ultimately T0 also remains a human thought structure!}
	
	## The Pragmatic Consequence
	
	\textbf{Since all theories are constructs}:
	
	\textbf{Evaluation criteria are}:
	
		- \textbf{Simplicity} (fewer assumptions)
		- \textbf{Consistency} (no contradictions)
		- \textbf{Predictive power} (testable consequences)
		- \textbf{Elegance} (aesthetic criteria)
		- \textbf{Unity} (fewer separate domains)
	
	
	\textbf{By all these criteria T0 is "better" than Einstein - but not "absolutely true".}
	
	## The Ontological Humility
	
	\textbf{The deepest insight}:
	
		- \textbf{Reality itself} is inaccessible
		- \textbf{All theories} are human constructs
		- \textbf{Mathematical consistency} proves no ontological truth
		- \textbf{The best} we have: \textbf{Simpler, more consistent constructs}
	
	
	\textbf{Einstein's error was not only the c-constant setting, but also the claim to absolute truth of his mathematical constructs.}
	
	\textbf{T0's advantage is not absolute truth, but relative superiority as a thought model.}
	
	# The Practical Consequences
	
	## Why E=mc² "Works"
	
	\textbf{E=mc² works because}:
	
		- It is mathematically identical to $E = m$
		- $c^2$ compensates the "frozen" time dynamics
		- The T0 truth is unconsciously contained
		- Local approximations usually suffice
	
	
	## When E=mc² Fails
	
	\textbf{The constants illusion breaks down at}:
	
		- Very precise measurements
		- Extreme conditions (high energies/masses)
		- Cosmological scales
		- Quantum gravity
	
	
	## T0's Universal Validity
	
	\textbf{E = m is valid everywhere and always}:
	
		- No approximations needed
		- No constant assumptions
		- Universal applicability
		- Fundamental simplicity
	
	
	# The Correction of Physics History
	
	## Einstein's True Achievement
	
	\textbf{Einstein's actual discovery was}:
	
```math-equation

		E = m \quad \text{(in natural form)}
	
```

	
	\textbf{His error was}:
	
```math-equation

		E = mc^2 \quad \text{(with artificial constant inflation)}
	
```

	
	## The Historical Irony
	
	\begin{tcolorbox}[colback=blue!5!white,colframe=blue!75!black,title=The Great Irony]
		Einstein discovered the fundamental simplicity $E = m$, 
		
		but \textbf{hid it behind the constants illusion} $E = mc^2$!
		
		The physics world celebrated the complicated form and overlooked the simple truth.
	\end{tcolorbox}
	
	# The T0 Perspective: c as Living Ratio
	
	## c as Expression of Time-Mass Duality
	
	\textbf{In T0 theory}:
	
```math-equation

		c(x,t) = f\left(\frac{L(x,t)}{\Tfield(x,t)}\right) = f\left(\frac{L(x,t) \cdot m(x,t)}{1}\right)
	
```

	
	since $\Tfield \cdot m = 1$.
	
	\textbf{c becomes an expression of the fundamental time-mass duality!}
	
	## The Dynamic Speed of Light
	
	\textbf{T0 prediction}: 
	
```math-equation

		c(x,t) = c_0 \sqrt{1 + \xipar \frac{m(x,t) - m_0}{m_0}}
	
```

	
	\textbf{Light moves faster in more massive regions!}
	
	(Tiny effect, but measurable in principle)
	
	# Experimental Tests of c-Variability
	
	## Proposed Experiments
	
	\textbf{Test 1 - Gravitational dependence}:
	
		- Measure c in different gravitational fields
		- T0 prediction: $c$ varies with $\sim \xipar \times \Delta\Phi_{\text{grav}}$
	
	
	\textbf{Test 2 - Cosmological variation}:
	
		- Measure c over cosmological time periods
		- T0 prediction: $c$ changes with universe expansion
	
	
	\textbf{Test 3 - High-energy physics}:
	
		- Measure c in particle accelerators at highest energies
		- T0 prediction: Tiny deviations at $E \sim$ TeV
	
	
	## Expected Results
	
	\begin{table}[htbp]
		\centering
		\begin{tabular}{|l|c|c|}
			\hline
			\textbf{Experiment} & \textbf{Einstein (c constant)} & \textbf{T0 (c variable)} \\
			\hline
			Gravitational field & $c = 299792458$ m/s & $c(1 \pm 10^{-15})$ \\
			Cosmological time & $c = $ constant & $c(1 + 10^{-12} \times t)$ \\
			High energy & $c = $ constant & $c(1 + 10^{-16})$ \\
			\hline
		\end{tabular}
		\caption{Predicted c-variations}
	\end{table}
	
	# Conclusions
	
	## The Central Recognition
	
	\begin{tcolorbox}[colback=green!5!white,colframe=green!75!black,title=The Fundamental Truth]
		\textbf{E=mc² = E=m}
		
		Einstein's "constant" c is in truth a variable ratio.
		
		The constant-setting was Einstein's fundamental error.
		
		T0 corrects this error by returning to natural variability.
	\end{tcolorbox}
	
	## Physics After the Constants Illusion
	
	\textbf{The future of physics}:
	
		- No artificial constants
		- Dynamic ratios everywhere
		- Living, variable natural laws
		- Fundamental simplicity: $E = m$
	
	
	## Einstein's Corrected Legacy
	
	\textbf{Einstein's true discovery}: $E = m$ (energy-mass identity)
	
	\textbf{Einstein's error}: Constant-setting of c
	
	\textbf{T0's correction}: Return to natural form $E = m$
	
	\textbf{Einstein was brilliant - he just stopped one step too early!}

\end{document}
