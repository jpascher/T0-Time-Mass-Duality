\documentclass[11pt,a4paper,openany]{book}

% Essential packages
\usepackage[utf8]{inputenc}
\usepackage[T1]{fontenc}
\usepackage[english]{babel}
\usepackage[a4paper,margin=2.5cm]{geometry}
\usepackage{lmodern}

% Math and physics packages
\usepackage{amsmath}
\usepackage{amssymb}
\usepackage{amsthm}
\usepackage{mathtools}
\usepackage{physics}
\usepackage{siunitx}

% Graphics and tables
\usepackage{graphicx}
\usepackage[table,xcdraw]{xcolor}
\usepackage{tikz}
\usepackage{pgfplots}
\usepackage{tcolorbox}
\usepackage{booktabs}
\usepackage{array}
\usepackage{longtable}
\usepackage{float}

% Document formatting
\usepackage{fancyhdr}
\usepackage{tocloft}
\usepackage{hyperref}
\usepackage{cleveref}
\usepackage{microtype}
\usepackage{enumitem}
\usepackage{newunicodechar}

% Additional packages
\usepackage{adjustbox}
\usepackage{algorithm}
\usepackage{algorithmic}
\usepackage{amsfonts}
\usepackage{amsmath,amsfonts,amssymb}
\usepackage{amsmath,amsfonts,amssymb,physics}
\usepackage{amsmath,amssymb}
\usepackage{amsmath,amssymb,amsfonts,amsthm}
\usepackage{amsmath,amssymb,amsthm}
\usepackage{amsmath,amssymb,physics,graphicx,xcolor,amsthm}
\usepackage{bm}
\usepackage{booktabs,array,longtable,multirow}
\usepackage{braket}
\usepackage{breakurl}
\usepackage{cancel}
\usepackage{caption}
\usepackage{cite}
\usepackage{color}
\usepackage{colortbl}
\usepackage{csquotes}
\usepackage{doi}
\usepackage{forest}
\usepackage{gensymb}
\usepackage{geometry,fancyhdr}
\usepackage{graphicx,tikz,pgfplots}
\usepackage{hyperref,url}
\usepackage{hyphenat}
\usepackage{listings}
\usepackage{listings,enumerate}
\usepackage{mdframed}
\usepackage{multicol}
\usepackage{multirow}
\usepackage{natbib}
\usepackage{pdflscape}
\usepackage{ragged2e}
\usepackage{setspace}
\usepackage{siunitx,xcolor,graphicx}
\usepackage{slashed}
\usepackage{tabularx}
\usepackage{textcomp}
\usepackage{textgreek}
\usepackage{tikz,pgfplots}
\usepackage{upgreek}
\usepackage{url}

% Custom commands and definitions
\definecolor{blue}
\definecolor{blue}{rgb}{0,0,1}
\definecolor{boxgray}
\definecolor{boxgray}{RGB}{240,240,240}
\definecolor{deepblue}
\definecolor{deepblue}{RGB}{0,0,127}
\definecolor{deepgreen}
\definecolor{deepgreen}{RGB}{0,127,0}
\definecolor{deepred}
\definecolor{deepred}{RGB}{191,0,0}
\definecolor{t0blue}
\definecolor{t0blue}{RGB}{0,102,204}
\definecolor{t0blue}{RGB}{33,150,243}
\definecolor{t0green}
\definecolor{t0green}{RGB}{0,153,0}
\definecolor{t0green}{RGB}{0,153,76}
\definecolor{t0green}{RGB}{76,175,80}
\definecolor{t0orange}
\definecolor{t0orange}{RGB}{255,152,0}
\definecolor{t0purple}
\definecolor{t0purple}{RGB}{102,0,204}
\definecolor{t0purple}{RGB}{156,39,176}
\definecolor{t0red}
\definecolor{t0red}{RGB}{204,0,0}
\definecolor{t0red}{RGB}{204,0,51}
\definecolor{t0red}{RGB}{244,67,54}
\definecolor{t0yellow}
\definecolor{t0yellow}{RGB}{255,204,0}
\geometry{a4paper, left=25mm, right=25mm, top=25mm, bottom=25mm}
\geometry{a4paper, margin=1in}
\geometry{a4paper, margin=2.5cm}
\geometry{a4paper, margin=2cm}
\geometry{left=2.5cm,right=2.5cm,top=2.5cm,bottom=2.5cm}
\geometry{left=2cm,right=2cm,top=2cm,bottom=2cm}
\geometry{margin=1in}
\geometry{margin=2.5cm}
\geometry{margin=2cm}
\hypersetup{
	colorlinks=true,
	linkcolor=blue,
	citecolor=blue,
	urlcolor=blue,
	pdftitle={Analysis and Implications of MNRAS Paper 544 for the T0-Theory}
\hypersetup{
	colorlinks=true,
	linkcolor=blue,
	citecolor=blue,
	urlcolor=blue,
	pdftitle={Beweis: Die Feinstrukturkonstante α = 1 in natürlichen Einheiten}
\hypersetup{
	colorlinks=true,
	linkcolor=blue,
	citecolor=blue,
	urlcolor=blue,
	pdftitle={Beweis: Die Koide-Formel enthält implizit $\xi$}
\hypersetup{
	colorlinks=true,
	linkcolor=blue,
	citecolor=blue,
	urlcolor=blue,
	pdftitle={Chinas Photonischer Quantenchip: 1000x-Speedup und T0-Integration}
\hypersetup{
	colorlinks=true,
	linkcolor=blue,
	citecolor=blue,
	urlcolor=blue,
	pdftitle={Complete Derivation of Higgs Mass and Wilson Coefficients}
\hypersetup{
	colorlinks=true,
	linkcolor=blue,
	citecolor=blue,
	urlcolor=blue,
	pdftitle={Complete Particle Spectrum: Standard Model vs T0 Theory}
\hypersetup{
	colorlinks=true,
	linkcolor=blue,
	citecolor=blue,
	urlcolor=blue,
	pdftitle={Conceptual Comparison of Unified Natural Units and Extended Standard Model}
\hypersetup{
	colorlinks=true,
	linkcolor=blue,
	citecolor=blue,
	urlcolor=blue,
	pdftitle={Connections between the Mizohata-Takeuchi Counterexample and the T0 Time-Mass Duality Theory}
\hypersetup{
	colorlinks=true,
	linkcolor=blue,
	citecolor=blue,
	urlcolor=blue,
	pdftitle={Das Relationale Zahlensystem: Primzahlen als fundamentale Verhältnisse}
\hypersetup{
	colorlinks=true,
	linkcolor=blue,
	citecolor=blue,
	urlcolor=blue,
	pdftitle={Das T0-Modell (Planck-Referenziert): Eine Neuformulierung der Physik}
\hypersetup{
	colorlinks=true,
	linkcolor=blue,
	citecolor=blue,
	urlcolor=blue,
	pdftitle={Das T0-Modell: Zeit-Energie-Dualität und geometrische Ruhemasse}
\hypersetup{
	colorlinks=true,
	linkcolor=blue,
	citecolor=blue,
	urlcolor=blue,
	pdftitle={Der Massenskalierungsexponent κ in der T0-Theorie}
\hypersetup{
	colorlinks=true,
	linkcolor=blue,
	citecolor=blue,
	urlcolor=blue,
	pdftitle={Der geometrische Formalismus der T0-Quantenmechanik und seine Anwendung auf Quantencomputer}
\hypersetup{
	colorlinks=true,
	linkcolor=blue,
	citecolor=blue,
	urlcolor=blue,
	pdftitle={Der xi Parameter und Teilchendifferenzierung in der T0-Theorie}
\hypersetup{
	colorlinks=true,
	linkcolor=blue,
	citecolor=blue,
	urlcolor=blue,
	pdftitle={Deterministic Quantum Mechanics via T0-Energy Field Formulation}
\hypersetup{
	colorlinks=true,
	linkcolor=blue,
	citecolor=blue,
	urlcolor=blue,
	pdftitle={Deterministische Quantenmechanik via T0-Energiefeld-Formulierung}
\hypersetup{
	colorlinks=true,
	linkcolor=blue,
	citecolor=blue,
	urlcolor=blue,
	pdftitle={Die Elektroneneinheitsladung in der T0-Theorie: Jenseits von Punkt-Singularitäten}
\hypersetup{
	colorlinks=true,
	linkcolor=blue,
	citecolor=blue,
	urlcolor=blue,
	pdftitle={Die Feinstrukturkonstante: Verschiedene Darstellungen und Beziehungen}
\hypersetup{
	colorlinks=true,
	linkcolor=blue,
	citecolor=blue,
	urlcolor=blue,
	pdftitle={Die Musikalische Spirale und die 137: Die mathematische Entdeckung der kosmischen Verstimmung}
\hypersetup{
	colorlinks=true,
	linkcolor=blue,
	citecolor=blue,
	urlcolor=blue,
	pdftitle={E=mc² = E=m: Die Konstanten-Illusion entlarvt}
\hypersetup{
	colorlinks=true,
	linkcolor=blue,
	citecolor=blue,
	urlcolor=blue,
	pdftitle={E=mc² = E=m: The Constants Illusion Exposed}
\hypersetup{
	colorlinks=true,
	linkcolor=blue,
	citecolor=blue,
	urlcolor=blue,
	pdftitle={Einfache Lagrange-Revolution: Von der Standardmodell-Komplexität zur T0-Eleganz}
\hypersetup{
	colorlinks=true,
	linkcolor=blue,
	citecolor=blue,
	urlcolor=blue,
	pdftitle={Einführung in die Umsetzung photonischer Bauteile auf Wafern für Nachrichtentechniker}
\hypersetup{
	colorlinks=true,
	linkcolor=blue,
	citecolor=blue,
	urlcolor=blue,
	pdftitle={Einführung in photonische Quantenchips für Nachrichtentechniker}
\hypersetup{
	colorlinks=true,
	linkcolor=blue,
	citecolor=blue,
	urlcolor=blue,
	pdftitle={Elimination der Masse als dimensionaler Platzhalter im T0-Modell}
\hypersetup{
	colorlinks=true,
	linkcolor=blue,
	citecolor=blue,
	urlcolor=blue,
	pdftitle={Elimination of Mass as Dimensional Placeholder in the T0 Model}
\hypersetup{
	colorlinks=true,
	linkcolor=blue,
	citecolor=blue,
	urlcolor=blue,
	pdftitle={Empirical Analysis of Deterministic Factorization Methods}
\hypersetup{
	colorlinks=true,
	linkcolor=blue,
	citecolor=blue,
	urlcolor=blue,
	pdftitle={Empirische Analyse deterministischer Faktorisierungsmethoden}
\hypersetup{
	colorlinks=true,
	linkcolor=blue,
	citecolor=blue,
	urlcolor=blue,
	pdftitle={Integration der Dirac-Gleichung im T0-Modell: Natürliche-Einheiten-Rahmenwerk}
\hypersetup{
	colorlinks=true,
	linkcolor=blue,
	citecolor=blue,
	urlcolor=blue,
	pdftitle={Integration of the Dirac Equation in the T0 Model: Natural Units Framework}
\hypersetup{
	colorlinks=true,
	linkcolor=blue,
	citecolor=blue,
	urlcolor=blue,
	pdftitle={Introduction to Photonic Quantum Chips for Communication Engineers}
\hypersetup{
	colorlinks=true,
	linkcolor=blue,
	citecolor=blue,
	urlcolor=blue,
	pdftitle={Introduction to the Implementation of Photonic Components on Wafers for Communication Engineers}
\hypersetup{
	colorlinks=true,
	linkcolor=blue,
	citecolor=blue,
	urlcolor=blue,
	pdftitle={Konzeptioneller Vergleich von Einheitlichen Natürlichen Einheiten und Erweitertem Standardmodell}
\hypersetup{
	colorlinks=true,
	linkcolor=blue,
	citecolor=blue,
	urlcolor=blue,
	pdftitle={Markov Chains in the Context of T0 Theory: Deterministic or Stochastic? A Treatise on Patterns, Preconditions, and Uncertainty}
\hypersetup{
	colorlinks=true,
	linkcolor=blue,
	citecolor=blue,
	urlcolor=blue,
	pdftitle={Markov-Ketten im Kontext der T0-Theorie: Deterministisch oder stochastisch? Ein Traktat zu Mustern, Voraussetzungen und Unsicherheit}
\hypersetup{
	colorlinks=true,
	linkcolor=blue,
	citecolor=blue,
	urlcolor=blue,
	pdftitle={Mathematical Analysis of T0-Shor Algorithm: Theoretical Framework and Computational Complexity}
\hypersetup{
	colorlinks=true,
	linkcolor=blue,
	citecolor=blue,
	urlcolor=blue,
	pdftitle={Mathematical Constructs of Alternative CMB Models: Unnikrishnan and Peratt in Harmony with the T0 Theory}
\hypersetup{
	colorlinks=true,
	linkcolor=blue,
	citecolor=blue,
	urlcolor=blue,
	pdftitle={Mathematische Analyse des T0-Shor Algorithmus: Theoretischer Rahmen und Berechnungskomplexität}
\hypersetup{
	colorlinks=true,
	linkcolor=blue,
	citecolor=blue,
	urlcolor=blue,
	pdftitle={Mathematische Konstrukte alternativer CMB-Modelle: Unnikrishnan und Peratt im Einklang mit der T0-Theorie}
\hypersetup{
	colorlinks=true,
	linkcolor=blue,
	citecolor=blue,
	urlcolor=blue,
	pdftitle={Natural Unit Systems: Universal Energy Conversion and Fundamental Length Scale Hierarchy}
\hypersetup{
	colorlinks=true,
	linkcolor=blue,
	citecolor=blue,
	urlcolor=blue,
	pdftitle={Natural Units in Theoretical Physics: A Treatise in the Context of T0 Theory}
\hypersetup{
	colorlinks=true,
	linkcolor=blue,
	citecolor=blue,
	urlcolor=blue,
	pdftitle={Natürliche Einheiten in der theoretischen Physik: Eine Abhandlung im Kontext der T0-Theorie}
\hypersetup{
	colorlinks=true,
	linkcolor=blue,
	citecolor=blue,
	urlcolor=blue,
	pdftitle={Natürliche Einheitensysteme: Universelle Energieumwandlung und fundamentale Längenskala-Hierarchie}
\hypersetup{
	colorlinks=true,
	linkcolor=blue,
	citecolor=blue,
	urlcolor=blue,
	pdftitle={Parameter System-Dependency in T0-Model: SI vs. Natural Units}
\hypersetup{
	colorlinks=true,
	linkcolor=blue,
	citecolor=blue,
	urlcolor=blue,
	pdftitle={Parameter-Systemabhängigkeit im T0-Modell: SI- vs. natürliche Einheiten}
\hypersetup{
	colorlinks=true,
	linkcolor=blue,
	citecolor=blue,
	urlcolor=blue,
	pdftitle={Proof: The Fine Structure Constant α = 1 in Natural Units}
\hypersetup{
	colorlinks=true,
	linkcolor=blue,
	citecolor=blue,
	urlcolor=blue,
	pdftitle={Proof: The Koide Formula Implicitly Contains $\xi$}
\hypersetup{
	colorlinks=true,
	linkcolor=blue,
	citecolor=blue,
	urlcolor=blue,
	pdftitle={Pure Energy T0 Theory: Ratio-Based Physics with SI Reference}
\hypersetup{
	colorlinks=true,
	linkcolor=blue,
	citecolor=blue,
	urlcolor=blue,
	pdftitle={Quantum Mechanics in the T0 Model: Field-Theoretic Foundations}
\hypersetup{
	colorlinks=true,
	linkcolor=blue,
	citecolor=blue,
	urlcolor=blue,
	pdftitle={Ratio-Based vs. Absolute: The Role of Fractal Correction in T0 Theory}
\hypersetup{
	colorlinks=true,
	linkcolor=blue,
	citecolor=blue,
	urlcolor=blue,
	pdftitle={Reine Energie T0-Theorie: Verhältnis-basierte Physik mit SI-Referenz}
\hypersetup{
	colorlinks=true,
	linkcolor=blue,
	citecolor=blue,
	urlcolor=blue,
	pdftitle={Simple Lagrangian Revolution: From Standard Model Complexity to T0 Elegance}
\hypersetup{
	colorlinks=true,
	linkcolor=blue,
	citecolor=blue,
	urlcolor=blue,
	pdftitle={Simplified Dirac Equation in T0 Theory: Field Node Approach}
\hypersetup{
	colorlinks=true,
	linkcolor=blue,
	citecolor=blue,
	urlcolor=blue,
	pdftitle={Simplified T0 Theory: Elegant Lagrangian Density for Time-Mass Duality}
\hypersetup{
	colorlinks=true,
	linkcolor=blue,
	citecolor=blue,
	urlcolor=blue,
	pdftitle={T0 Cosmology: Redshift as a Geometric Path Effect in a Static Universe}
\hypersetup{
	colorlinks=true,
	linkcolor=blue,
	citecolor=blue,
	urlcolor=blue,
	pdftitle={T0 Deterministic Quantum Computing: Complete Analysis of Important Algorithms}
\hypersetup{
	colorlinks=true,
	linkcolor=blue,
	citecolor=blue,
	urlcolor=blue,
	pdftitle={T0 Deterministisches Quantencomputing: Vollständige Analyse wichtiger Algorithmen}
\hypersetup{
	colorlinks=true,
	linkcolor=blue,
	citecolor=blue,
	urlcolor=blue,
	pdftitle={T0 Model: Complete Framework - From Time-Energy Duality to Universal Constants}
\hypersetup{
	colorlinks=true,
	linkcolor=blue,
	citecolor=blue,
	urlcolor=blue,
	pdftitle={T0 Model: Complete Parameter-Free Particle Mass Calculation}
\hypersetup{
	colorlinks=true,
	linkcolor=blue,
	citecolor=blue,
	urlcolor=blue,
	pdftitle={T0 Model: Unified Neutrino Formula Structure}
\hypersetup{
	colorlinks=true,
	linkcolor=blue,
	citecolor=blue,
	urlcolor=blue,
	pdftitle={T0 Model: Universal Energy Relations for Mol and Candela Units}
\hypersetup{
	colorlinks=true,
	linkcolor=blue,
	citecolor=blue,
	urlcolor=blue,
	pdftitle={T0 Modell: Vollständiges Framework - Von Zeit-Energie-Dualität zu universellen Konstanten}
\hypersetup{
	colorlinks=true,
	linkcolor=blue,
	citecolor=blue,
	urlcolor=blue,
	pdftitle={T0 Quantenfeldtheorie: QFT, QM und Quantencomputer}
\hypersetup{
	colorlinks=true,
	linkcolor=blue,
	citecolor=blue,
	urlcolor=blue,
	pdftitle={T0 Quantum Field Theory: QFT, QM and Quantum Computers}
\hypersetup{
	colorlinks=true,
	linkcolor=blue,
	citecolor=blue,
	urlcolor=blue,
	pdftitle={T0 Theory vs Bell's Theorem: How Deterministic Energy Fields Circumvent No-Go Theorems}
\hypersetup{
	colorlinks=true,
	linkcolor=blue,
	citecolor=blue,
	urlcolor=blue,
	pdftitle={T0 Theory: Final Extension to Hadrons - Physically Derived Corrections}
\hypersetup{
	colorlinks=true,
	linkcolor=blue,
	citecolor=blue,
	urlcolor=blue,
	pdftitle={T0 Theory: The Fine-Structure Constant}
\hypersetup{
	colorlinks=true,
	linkcolor=blue,
	citecolor=blue,
	urlcolor=blue,
	pdftitle={T0 Theory: The Gravitational Constant}
\hypersetup{
	colorlinks=true,
	linkcolor=blue,
	citecolor=blue,
	urlcolor=blue,
	pdftitle={T0-Kosmologie: Rotverschiebung als geometrischer Pfad-Effekt im statischen Universum}
\hypersetup{
	colorlinks=true,
	linkcolor=blue,
	citecolor=blue,
	urlcolor=blue,
	pdftitle={T0-Model: Complete Document Analysis and Structured Summary}
\hypersetup{
	colorlinks=true,
	linkcolor=blue,
	citecolor=blue,
	urlcolor=blue,
	pdftitle={T0-Model: Kinetic Energy of Electrons and Photons}
\hypersetup{
	colorlinks=true,
	linkcolor=blue,
	citecolor=blue,
	urlcolor=blue,
	pdftitle={T0-Model: The Hubble Parameter in Static Universe}
\hypersetup{
	colorlinks=true,
	linkcolor=blue,
	citecolor=blue,
	urlcolor=blue,
	pdftitle={T0-Modell-Verifikation: Skalen-Verhältnis-basierte Berechnungen}
\hypersetup{
	colorlinks=true,
	linkcolor=blue,
	citecolor=blue,
	urlcolor=blue,
	pdftitle={T0-Modell: Bewegungsenergie von Elektronen und Photonen}
\hypersetup{
	colorlinks=true,
	linkcolor=blue,
	citecolor=blue,
	urlcolor=blue,
	pdftitle={T0-Modell: Die Hubble-Konstante im statischen Universum}
\hypersetup{
	colorlinks=true,
	linkcolor=blue,
	citecolor=blue,
	urlcolor=blue,
	pdftitle={T0-Modell: Einheitliche Neutrino-Formel-Struktur}
\hypersetup{
	colorlinks=true,
	linkcolor=blue,
	citecolor=blue,
	urlcolor=blue,
	pdftitle={T0-Modell: Universelle Energiebeziehungen für Mol- und Candela-Einheiten}
\hypersetup{
	colorlinks=true,
	linkcolor=blue,
	citecolor=blue,
	urlcolor=blue,
	pdftitle={T0-Modell: Vollständige Dokumentenanalyse und strukturierte Zusammenfassung}
\hypersetup{
	colorlinks=true,
	linkcolor=blue,
	citecolor=blue,
	urlcolor=blue,
	pdftitle={T0-Modell: Vollständige parameterfreie Teilchenmassen-Berechnung}
\hypersetup{
	colorlinks=true,
	linkcolor=blue,
	citecolor=blue,
	urlcolor=blue,
	pdftitle={T0-QAT: $\xi$-Aware Quantization-Aware Training}
\hypersetup{
	colorlinks=true,
	linkcolor=blue,
	citecolor=blue,
	urlcolor=blue,
	pdftitle={T0-QFT ML Addendum: Machine Learning Derived Extensions}
\hypersetup{
	colorlinks=true,
	linkcolor=blue,
	citecolor=blue,
	urlcolor=blue,
	pdftitle={T0-QFT ML-Addendum: Maschinelle Lern-abgeleitete Erweiterungen}
\hypersetup{
	colorlinks=true,
	linkcolor=blue,
	citecolor=blue,
	urlcolor=blue,
	pdftitle={T0-Theorie vs Bells Theorem: Wie deterministische Energiefelder No-Go-Theoreme umgehen}
\hypersetup{
	colorlinks=true,
	linkcolor=blue,
	citecolor=blue,
	urlcolor=blue,
	pdftitle={T0-Theorie: Der Terrell-Penrose-Effekt und Massenvariation}
\hypersetup{
	colorlinks=true,
	linkcolor=blue,
	citecolor=blue,
	urlcolor=blue,
	pdftitle={T0-Theorie: Die Feinstrukturkonstante}
\hypersetup{
	colorlinks=true,
	linkcolor=blue,
	citecolor=blue,
	urlcolor=blue,
	pdftitle={T0-Theorie: Die Gravitationskonstante}
\hypersetup{
	colorlinks=true,
	linkcolor=blue,
	citecolor=blue,
	urlcolor=blue,
	pdftitle={T0-Theorie: Die T0-Zeit-Masse-Dualität}
\hypersetup{
	colorlinks=true,
	linkcolor=blue,
	citecolor=blue,
	urlcolor=blue,
	pdftitle={T0-Theorie: Die sieben Rätsel}
\hypersetup{
	colorlinks=true,
	linkcolor=blue,
	citecolor=blue,
	urlcolor=blue,
	pdftitle={T0-Theorie: Erweiterung auf Bell-Tests – ML-Simulationen (November 2025)}
\hypersetup{
	colorlinks=true,
	linkcolor=blue,
	citecolor=blue,
	urlcolor=blue,
	pdftitle={T0-Theorie: Finale Erweiterung auf Hadronen - Physikalisch abgeleitete Korrekturen}
\hypersetup{
	colorlinks=true,
	linkcolor=blue,
	citecolor=blue,
	urlcolor=blue,
	pdftitle={T0-Theorie: Finale Fraktale Massenformeln (November 2025)}
\hypersetup{
	colorlinks=true,
	linkcolor=blue,
	citecolor=blue,
	urlcolor=blue,
	pdftitle={T0-Theorie: Fraktaldimension aus Lepton-Massenverhältnis}
\hypersetup{
	colorlinks=true,
	linkcolor=blue,
	citecolor=blue,
	urlcolor=blue,
	pdftitle={T0-Theorie: Fundamentale Prinzipien}
\hypersetup{
	colorlinks=true,
	linkcolor=blue,
	citecolor=blue,
	urlcolor=blue,
	pdftitle={T0-Theorie: Herleitung der Gravitationskonstanten}
\hypersetup{
	colorlinks=true,
	linkcolor=blue,
	citecolor=blue,
	urlcolor=blue,
	pdftitle={T0-Theorie: Kosmische Beziehungen und universelle $\xi$-Konstante}
\hypersetup{
	colorlinks=true,
	linkcolor=blue,
	citecolor=blue,
	urlcolor=blue,
	pdftitle={T0-Theorie: Kosmologie}
\hypersetup{
	colorlinks=true,
	linkcolor=blue,
	citecolor=blue,
	urlcolor=blue,
	pdftitle={T0-Theorie: Netzwerkdarstellung und Dimensionsanalyse in der T0-Theorie}
\hypersetup{
	colorlinks=true,
	linkcolor=blue,
	citecolor=blue,
	urlcolor=blue,
	pdftitle={T0-Theorie: Teilchenmassen}
\hypersetup{
	colorlinks=true,
	linkcolor=blue,
	citecolor=blue,
	urlcolor=blue,
	pdftitle={T0-Theorie: Vollstaendiger Abschluss}
\hypersetup{
	colorlinks=true,
	linkcolor=blue,
	citecolor=blue,
	urlcolor=blue,
	pdftitle={T0-Theory: Complete Closure}
\hypersetup{
	colorlinks=true,
	linkcolor=blue,
	citecolor=blue,
	urlcolor=blue,
	pdftitle={T0-Theory: Complete Derivation of All Parameters Without Circularity}
\hypersetup{
	colorlinks=true,
	linkcolor=blue,
	citecolor=blue,
	urlcolor=blue,
	pdftitle={T0-Theory: Cosmic Relations and universal $\xi$-constant}
\hypersetup{
	colorlinks=true,
	linkcolor=blue,
	citecolor=blue,
	urlcolor=blue,
	pdftitle={T0-Theory: Cosmology}
\hypersetup{
	colorlinks=true,
	linkcolor=blue,
	citecolor=blue,
	urlcolor=blue,
	pdftitle={T0-Theory: Derivation of the Gravitational Constant}
\hypersetup{
	colorlinks=true,
	linkcolor=blue,
	citecolor=blue,
	urlcolor=blue,
	pdftitle={T0-Theory: Extension to Bell Tests – ML Simulations (November 2025)}
\hypersetup{
	colorlinks=true,
	linkcolor=blue,
	citecolor=blue,
	urlcolor=blue,
	pdftitle={T0-Theory: Final Fractal Mass Formulas (November 2025)}
\hypersetup{
	colorlinks=true,
	linkcolor=blue,
	citecolor=blue,
	urlcolor=blue,
	pdftitle={T0-Theory: Fractal Dimension from Lepton Mass Ratio}
\hypersetup{
	colorlinks=true,
	linkcolor=blue,
	citecolor=blue,
	urlcolor=blue,
	pdftitle={T0-Theory: Fundamental Principles}
\hypersetup{
	colorlinks=true,
	linkcolor=blue,
	citecolor=blue,
	urlcolor=blue,
	pdftitle={T0-Theory: Mass Variation as an Equivalent to Time Dilation}
\hypersetup{
	colorlinks=true,
	linkcolor=blue,
	citecolor=blue,
	urlcolor=blue,
	pdftitle={T0-Theory: Network Representation and Dimensional Analysis in the T0-Theory}
\hypersetup{
	colorlinks=true,
	linkcolor=blue,
	citecolor=blue,
	urlcolor=blue,
	pdftitle={T0-Theory: Neutrinos}
\hypersetup{
	colorlinks=true,
	linkcolor=blue,
	citecolor=blue,
	urlcolor=blue,
	pdftitle={T0-Theory: Particle Masses}
\hypersetup{
	colorlinks=true,
	linkcolor=blue,
	citecolor=blue,
	urlcolor=blue,
	pdftitle={T0-Theory: The Seven Riddles}
\hypersetup{
	colorlinks=true,
	linkcolor=blue,
	citecolor=blue,
	urlcolor=blue,
	pdftitle={T0-Theory: The T0-Time-Mass Duality}
\hypersetup{
	colorlinks=true,
	linkcolor=blue,
	citecolor=blue,
	urlcolor=blue,
	pdftitle={Temperature Units in Natural Units: T0-Theory}
\hypersetup{
	colorlinks=true,
	linkcolor=blue,
	citecolor=blue,
	urlcolor=blue,
	pdftitle={Temperatureinheiten in nat\"urlichen Einheiten: T0-Theorie}
\hypersetup{
	colorlinks=true,
	linkcolor=blue,
	citecolor=blue,
	urlcolor=blue,
	pdftitle={The Electron Unit Charge in T0 Theory: Beyond Point Singularities}
\hypersetup{
	colorlinks=true,
	linkcolor=blue,
	citecolor=blue,
	urlcolor=blue,
	pdftitle={The Fine Structure Constant: Various Representations and Relationships}
\hypersetup{
	colorlinks=true,
	linkcolor=blue,
	citecolor=blue,
	urlcolor=blue,
	pdftitle={The Geometric Formalism of T0 Quantum Mechanics and its Application to Quantum Computing}
\hypersetup{
	colorlinks=true,
	linkcolor=blue,
	citecolor=blue,
	urlcolor=blue,
	pdftitle={The Mass Scaling Exponent κ in T0 Theory}
\hypersetup{
	colorlinks=true,
	linkcolor=blue,
	citecolor=blue,
	urlcolor=blue,
	pdftitle={The Musical Spiral and 137: The Mathematical Discovery of Cosmic Detuning}
\hypersetup{
	colorlinks=true,
	linkcolor=blue,
	citecolor=blue,
	urlcolor=blue,
	pdftitle={The Relational Number System: Prime Numbers as Fundamental Ratios}
\hypersetup{
	colorlinks=true,
	linkcolor=blue,
	citecolor=blue,
	urlcolor=blue,
	pdftitle={The T0 Model (Planck-Referenced): A Reformulation of Physics}
\hypersetup{
	colorlinks=true,
	linkcolor=blue,
	citecolor=blue,
	urlcolor=blue,
	pdftitle={The T0 Model: Time-Energy Duality and Geometric Rest Mass}
\hypersetup{
	colorlinks=true,
	linkcolor=blue,
	citecolor=blue,
	urlcolor=blue,
	pdftitle={The T0-Model (Planck-Referenced): A Reformulation of Physics}
\hypersetup{
	colorlinks=true,
	linkcolor=blue,
	citecolor=blue,
	urlcolor=blue,
	pdftitle={Verbindungen zwischen dem Mizohata-Takeuchi-Gegenbeispiel und der T0-Zeit-Masse-Dualitätstheorie}
\hypersetup{
	colorlinks=true,
	linkcolor=blue,
	citecolor=blue,
	urlcolor=blue,
	pdftitle={Vereinfachte Dirac-Gleichung in der T0-Theorie: Feldknoten-Ansatz}
\hypersetup{
	colorlinks=true,
	linkcolor=blue,
	citecolor=blue,
	urlcolor=blue,
	pdftitle={Vereinfachte T0-Theorie: Elegante Lagrange-Dichte für Zeit-Masse-Dualität}
\hypersetup{
	colorlinks=true,
	linkcolor=blue,
	citecolor=blue,
	urlcolor=blue,
	pdftitle={Verhältnisbasiert vs. Absolut: Die Rolle der fraktalen Korrektur in der T0-Theorie}
\hypersetup{
	colorlinks=true,
	linkcolor=blue,
	citecolor=blue,
	urlcolor=blue,
	pdftitle={Vollständige Herleitung der Higgs-Masse und Wilson-Koeffizienten}
\hypersetup{
	colorlinks=true,
	linkcolor=blue,
	citecolor=blue,
	urlcolor=blue,
	pdftitle={Vollständiges Teilchenspektrum: Standard-Modell vs T0-Theorie}
\hypersetup{
	colorlinks=true,
	linkcolor=blue,
	citecolor=blue,
	urlcolor=blue,
	pdftitle={Warum Zahlenverhältnisse nicht direkt gekürzt werden dürfen}
\hypersetup{
	colorlinks=true,
	linkcolor=blue,
	citecolor=blue,
	urlcolor=blue,
	pdftitle={Why Numerical Ratios Must Not Be Directly Simplified}
\hypersetup{
	colorlinks=true,
	linkcolor=blue,
	citecolor=blue,
	urlcolor=blue,
}
\hypersetup{
	colorlinks=true,
	linkcolor=blue,
	citecolor=red,
	urlcolor=blue,
	bookmarks=true,
	bookmarksnumbered=true,
	pdfstartview=FitH,
	pdftitle={T0 Model - Field-Theoretic Derivation of the Beta Parameter}
\hypersetup{
	colorlinks=true,
	linkcolor=blue,
	citecolor=red,
	urlcolor=blue,
	bookmarks=true,
	bookmarksnumbered=true,
	pdfstartview=FitH,
	pdftitle={T0-Modell - Feldtheoretische Herleitung des Beta-Parameters}
\hypersetup{
	colorlinks=true,
	linkcolor=blue,
	filecolor=magenta,
	urlcolor=cyan,
}
\hypersetup{
	colorlinks=true,
	linkcolor=blue,
	urlcolor=blue,
	citecolor=blue,
	pdftitle={From Time Dilation to Mass Variation: Mathematical Core Formulations of Time-Mass Duality Theory - Updated Framework}
\hypersetup{
	colorlinks=true,
	linkcolor=blue,
	urlcolor=blue,
	citecolor=blue,
	pdftitle={T0 Model: Detailed Formula for Leptonic Anomalies}
\hypersetup{
	colorlinks=true,
	linkcolor=blue,
	urlcolor=blue,
	citecolor=blue,
	pdftitle={T0 Model: Detaillierte Formel für leptonische Anomalien}
\hypersetup{
	colorlinks=true,
	linkcolor=blue,
	urlcolor=blue,
	citecolor=blue,
	pdftitle={T0 Model: Energy-based Formulas with Quadratic Scaling}
\hypersetup{
	colorlinks=true,
	linkcolor=blue,
	urlcolor=blue,
	citecolor=blue,
	pdftitle={T0 Model: Granulation, Limits and Fundamental Asymmetry}
\hypersetup{
	colorlinks=true,
	linkcolor=blue,
	urlcolor=blue,
	citecolor=blue,
	pdftitle={T0-Modell: Energiebasierte Formeln mit quadratischer Skalierung}
\hypersetup{
	colorlinks=true,
	linkcolor=blue,
	urlcolor=blue,
	citecolor=blue,
	pdftitle={T0-Modell: Granulation, Limits und fundamentale Asymmetrie}
\hypersetup{
	colorlinks=true,
	linkcolor=blue,
	urlcolor=blue,
	citecolor=blue,
	pdftitle={Von Zeitdilatation zu Massenvariation: Mathematische Kernformulierungen der Zeit-Masse-Dualitätstheorie - Aktualisiertes Framework}
\hypersetup{
	colorlinks=true,
	linkcolor=t0blue,
	citecolor=t0blue,
	urlcolor=t0blue,
	pdftitle={T0 Model: Complete Theoretical Summary}
\hypersetup{
	colorlinks=true,
	linkcolor=t0blue,
	citecolor=t0blue,
	urlcolor=t0blue,
	pdftitle={T0 Theory: Resolution of Apparent Instantaneity}
\hypersetup{
	colorlinks=true,
	linkcolor=t0blue,
	citecolor=t0blue,
	urlcolor=t0blue,
	pdftitle={T0 vs Synergetics: Vereinfachung durch natürliche Einheiten}
\hypersetup{
	colorlinks=true,
	linkcolor=t0blue,
	citecolor=t0blue,
	urlcolor=t0blue,
	pdftitle={T0-Modell: Vollständige theoretische Zusammenfassung}
\hypersetup{
	colorlinks=true,
	linkcolor=t0blue,
	citecolor=t0blue,
	urlcolor=t0blue,
	pdftitle={T0-Theorie: Auflösung der scheinbaren Instantanität}
\hypersetup{
	colorlinks=true,
	linkcolor=t0blue,
	citecolor=t0blue,
	urlcolor=t0blue,
	pdftitle={T0-Theorie: Vollständige Dokumentenübersicht}
\hypersetup{
	colorlinks=true,
	linkcolor=t0blue,
	citecolor=t0blue,
	urlcolor=t0blue,
	pdftitle={T0-Theory: Complete Document Overview}
\hypersetup{
	colorlinks=true,
	linkcolor=t0blue,
	citecolor=t0blue,
	urlcolor=t0blue,
}
\hypersetup{
	colorlinks=true,
	linkcolor=t0blue,
	citecolor=t0green,
	urlcolor=t0blue,
	pdftitle={Das verborgene Geheimnis von 1/137}
\hypersetup{
	colorlinks=true,
	linkcolor=t0blue,
	citecolor=t0green,
	urlcolor=t0blue,
	pdftitle={The Hidden Secret of 1/137}
\hypersetup{
    colorlinks=true,
    linkcolor=blue,
    citecolor=blue,
    urlcolor=blue,
    pdftitle={Analyse und Implikationen des MNRAS-Papiers 544 für die T0-Theorie}
\hypersetup{
  colorlinks=true,
  linkcolor=blue,
  citecolor=blue,
  urlcolor=blue
}
\hypersetup{
  colorlinks=true,
  linkcolor=blue,
  citecolor=blue,
  urlcolor=blue,
  pdftitle={T0-Theorie: Ein-Uhr-Metrologie und Drei-Uhren-Experiment}
\hypersetup{
  colorlinks=true,
  linkcolor=blue,
  citecolor=blue,
  urlcolor=blue,
  pdftitle={T0-Theory: Single-Clock Metrology and Three-Clock Experiment}
\hypersetup{
colorlinks=true,
linkcolor=blue,
citecolor=blue,
urlcolor=blue,
pdftitle={Quantenmechanik im T0-Modell: Feldtheoretische Grundlagen}
\hypersetup{
colorlinks=true,
linkcolor=blue,
citecolor=blue,
urlcolor=blue,
pdftitle={T0-Theory: Neutrinos}
\newcommand{\Bzero}{B_0}
\newcommand{\CQCD}{C_{\text{QCD}
\newcommand{\Cconv}{C_{\text{conv}
\newcommand{\Cto}{C_{\text{T0}
\newcommand{\Czero}{C_0}
\newcommand{\DTmu}{D_{T,\mu}
\newcommand{\DcovT}[1]{\partial_\mu #1 + #1 \partial_\mu \Tfield}
\newcommand{\Dfrak}{D_f}
\newcommand{\Df}{D_f}
\newcommand{\DhiggsT}{\Tfield (\partial_\mu + ig A_\mu) \Phi + \Phi \partial_\mu \Tfield}
\newcommand{\EPlanck}{E_P}
\newcommand{\EPlanck}{E_{\text{Pl}
\newcommand{\EPratio}[1]{\frac{#1}
\newcommand{\EP}{E_P}
\newcommand{\EP}{E_{\text{P}
\newcommand{\EW}{E_W}
\newcommand{\EZ}{E_Z}
\newcommand{\Echar}{E_{\text{char}
\newcommand{\Ee}{E_e}
\newcommand{\Efield}{E(x,t)}
\newcommand{\Efield}{E_\text{field}
\newcommand{\Efield}{E_{\text{Feld}
\newcommand{\Efield}{E_{\text{Field}
\newcommand{\Efield}{E_{\text{field}
\newcommand{\Efield}{E}
\newcommand{\Egamma}{E_\gamma}
\newcommand{\Eh}{E_h}
\newcommand{\Emu}{E_\mu}
\newcommand{\Enorm}[1]{E_{\text{norm}
\newcommand{\En}{E_n}
\newcommand{\Ep}{E_p}
\newcommand{\Eratio}[2]{\frac{E_{#1}
\newcommand{\Etau}{E_\tau}
\newcommand{\Evis}{E_{\text{vis}
\newcommand{\Exi}{E_\xi}
\newcommand{\Ezero}{E_0}
\newcommand{\GeV}{\,\text{GeV}
\newcommand{\Gnat}{G_{\text{nat}
\newcommand{\Gsi}{G_{\text{SI}
\newcommand{\Hubble}{H_0}
\newcommand{\Kfrak}{K_{\text{frac}
\newcommand{\Kfrak}{K_{\text{frak}
\newcommand{\Kspec}{K_{\text{spec}
\newcommand{\LCDM}{\Lambda\text{CDM}
\newcommand{\LPlanck}{\ell_{\text{Pl}
\newcommand{\Lag}{\mathcal{L}
\newcommand{\Lambdat}{\Lambda_T}
\newcommand{\Leff}{L_{\text{eff}
\newcommand{\Lorentz}[2]{{\Lambda^\mu{}
\newcommand{\Lp}{L_{\text{P}
\newcommand{\Lxi}{L_\xi}
\newcommand{\Lzero}{L_0}
\newcommand{\MPl}{M_{\text{Pl}
\newcommand{\MSbar}{\overline{\text{MS}
\newcommand{\MeV}{\,\text{MeV}
\newcommand{\Mpl}{M_{\text{Pl}
\newcommand{\OmegaDM}{\Omega_{\text{DM}
\newcommand{\OmegaLambda}{\Omega_{\Lambda}
\newcommand{\Omegab}{\Omega_b}
\newcommand{\Phiphoton}{\Phi_{\text{photon}
\newcommand{\Ricci}{R_{\mu\nu}
\newcommand{\Riem}{R^\rho{}
\newcommand{\Rzero}{R_\infty}
\newcommand{\Scal}{R}
\newcommand{\SynchPower}{P_{\text{synch}
\newcommand{\TPlanck}{t_{\text{Pl}
\newcommand{\Tfieldt}{T(\vec{x}
\newcommand{\Tfieldt}{T(x,t)}
\newcommand{\Tfield}{T(x)}
\newcommand{\Tfield}{T(x,t)}
\newcommand{\Tfield}{T_{\text{field}
\newcommand{\Tfield}{T}
\newcommand{\Tfield}{\mathcal{T}
\newcommand{\Tzerot}{T_0(\Tfield)}
\newcommand{\Tzero}{T_0}
\newcommand{\Weyl}{C^\rho{}
\newcommand{\ZPinch}{J \times B = \nabla p}
\newcommand{\aleph}{\aleph}
\newcommand{\alphaEMSI}{\alpha_{\text{EM,SI}
\newcommand{\alphaEMnat}{\alpha_{\text{EM,nat}
\newcommand{\alphaEM}{\alpha_{\text{EM}
\newcommand{\alphaEM}{\ensuremath{\alpha_{\text{EM}
\newcommand{\alphaQCD}{\alpha_s}
\newcommand{\alphaQED}{\alpha_{\text{QED}
\newcommand{\alphaSI}{\alpha_{\text{SI}
\newcommand{\alphaT}{\alpha_{\text{T}
\newcommand{\alphaWSI}{\alpha_{\text{W,SI}
\newcommand{\alphaWnat}{\alpha_{\text{W,nat}
\newcommand{\alphaW}{\alpha_{\text{W}
\newcommand{\alphaem}{\alpha_{EM}
\newcommand{\alphaem}{\alpha}
\newcommand{\alphafine}{\alpha}
\newcommand{\alphagem}{\alpha}
\newcommand{\alphanat}{\alpha_{\text{nat}
\newcommand{\alphapar}{\alpha}
\newcommand{\betaTSI}{\beta_{\text{T,SI}
\newcommand{\betaTnat}{\beta_{\text{T,nat}
\newcommand{\betaT}{\beta_T}
\newcommand{\betaT}{\beta_{T}
\newcommand{\betaT}{\beta_{\text{T}
\newcommand{\betaT}{\ensuremath{\beta_T}
\newcommand{\betapar}{\beta}
\newcommand{\calL}{\mathcal{L}
\newcommand{\checked}{\checkmark}
\newcommand{\checkmarkx}{\checkmark}
\newcommand{\dTdt}{\frac{d\Tfieldt}
\newcommand{\deltaE}{\delta E}
\newcommand{\deltafield}{\ensuremath{\delta m}
\newcommand{\deltam}{\delta m}
\newcommand{\deq}{\displaystyle}
\newcommand{\docref}[1]{\texttt{#1}
\newcommand{\eV}{\,\text{eV}
\newcommand{\epsilonT}{\varepsilon_T}
\newcommand{\epsilonzero}{\varepsilon_0}
\newcommand{\etavis}{\eta_{\text{visual}
\newcommand{\e}{\mathrm{e}
\newcommand{\gW}{g_W}
\newcommand{\gammaf}{\gamma_{\text{Lorentz}
\newcommand{\gammamu}{\gamma^\mu}
\newcommand{\gs}{g_s}
\newcommand{\inftytext}{$\infty$}
\newcommand{\interval}[2]{#1:#2}
\newcommand{\kfrac}{K_{\text{frak}
\newcommand{\lP}{\ell_{\text{P}
\newcommand{\lP}{l_P}
\newcommand{\lambdah}{\ensuremath{\lambda_h}
\newcommand{\lambdah}{\lambda_h}
\newcommand{\lambdazero}{\lambda_0}
\newcommand{\mP}{m_{\text{P}
\newcommand{\mfield}{m(x,t)}
\newcommand{\mfield}{m}
\newcommand{\mh}{m_h}
\newcommand{\micrometer}{\ensuremath{\mu}
\newcommand{\mikrometer}{\ensuremath{\mu}
\newcommand{\myRightarrow}{\ensuremath{\Rightarrow}
\newcommand{\myapprox}{\ensuremath{\approx}
\newcommand{\myomega}{\ensuremath{\omega}
\newcommand{\myphi}{\ensuremath{\phi}
\newcommand{\mypi}{\ensuremath{\pi}
\newcommand{\mypropto}{\ensuremath{\propto}
\newcommand{\myrightarrow}{\ensuremath{\rightarrow}
\newcommand{\mysim}{\ensuremath{\sim}
\newcommand{\mysqrt}{\ensuremath{\sqrt}
\newcommand{\mytimes}{\ensuremath{\times}
\newcommand{\natunits}{\hbar = c = G = k_B = 1}
\newcommand{\natunits}{\text{(nat. Einh.)}
\newcommand{\natunits}{\text{(nat. units)}
\newcommand{\nulep}{\nu}
\newcommand{\nuzero}{\nu_0}
\newcommand{\partialop}{\ensuremath{\partial}
\newcommand{\pdTdt}{\frac{\partial\Tfieldt}
\newcommand{\pdTdx}{\nabla\Tfieldt}
\newcommand{\phiT}{\phi}
\newcommand{\pichar}{\pi}
\newcommand{\primrel}[1]{\mathbf{#1}
\newcommand{\rhoCMB}{\rho_{\text{CMB}
\newcommand{\rhoCasimir}{\rho_{\text{Casimir}
\newcommand{\rhoE}{\rho_E}
\newcommand{\rhofield}{\ensuremath{\rho}
\newcommand{\rzero}{r_0}
\newcommand{\slashk}{\cancel{k}
\newcommand{\slashp}{\cancel{p}
\newcommand{\slashq}{\cancel{q}
\newcommand{\tP}{t_P}
\newcommand{\tP}{t_{\text{P}
\newcommand{\tablescale}{0.9}
\newcommand{\tzero}{t_0}
\newcommand{\vect}[1]{\boldsymbol{#1}
\newcommand{\vecx}{\vec{x}
\newcommand{\vh}{v}
\newcommand{\vr}{\vec{r}
\newcommand{\warningx}{\color{red}
\newcommand{\warningx}{\textbf{!}
\newcommand{\warningx}{{\color{red}
\newcommand{\xiT}{\xi}
\newcommand{\xiconst}{\xi = \frac{4}
\newcommand{\xicoupling}{f(E/\Exi)}
\newcommand{\xigeom}{\xi_{\text{geom}
\newcommand{\xigeom}{\xi}
\newcommand{\xikonst}{\xi = \frac{4}
\newcommand{\xiparticle}{\xi_{\text{particle}
\newcommand{\xipar}{\ensuremath{\xi}
\newcommand{\xipar}{\xi_0}
\newcommand{\xipar}{\xi}
\newcommand{\xirat}{\xi_{\text{ratio}
\newtheorem{axiom}{Axiom}
\newtheorem{category}{Category-Theoretic Basis}
\newtheorem{category}{Kategorientheoretische Basis}
\newtheorem{corollary}[theorem]{Corollary}
\newtheorem{corollary}[theorem]{Korollar}
\newtheorem{corollary}{Corollary}
\newtheorem{corollary}{Korollar}
\newtheorem{definition}[theorem]{Definition}
\newtheorem{definition}{Definition}
\newtheorem{discovery}{Discovery}
\newtheorem{discovery}{Neue Entdeckung}
\newtheorem{discovery}{New Discovery}
\newtheorem{discovery}{Revolutionary Discovery}
\newtheorem{entdeckung}{Entdeckung}
\newtheorem{entdeckung}{Revolutionäre Entdeckung}
\newtheorem{erkenntnis}{Erkenntnis}
\newtheorem{erkenntnis}{Schlüsselerkenntnis}
\newtheorem{example}[theorem]{Beispiel}
\newtheorem{example}[theorem]{Example}
\newtheorem{example}{Beispiel}
\newtheorem{example}{Example}
\newtheorem{insight}{Central Insight}
\newtheorem{insight}{Insight}
\newtheorem{insight}{Key Insight}
\newtheorem{insight}{Wichtige Einsicht}
\newtheorem{insight}{Zentrale Einsicht}
\newtheorem{lemma}[theorem]{Lemma}
\newtheorem{lemma}{Lemma}
\newtheorem{principle}{Fundamental Principle}
\newtheorem{principle}{Fundamentales Prinzip}
\newtheorem{principle}{Grundlegendes Prinzip}
\newtheorem{principle}{Principle}
\newtheorem{principle}{Prinzip}
\newtheorem{prinzip}{Grundprinzip}
\newtheorem{proof_step}{Beweisschritt}
\newtheorem{proof_step}{Proof Step}
\newtheorem{proposition}[theorem]{Proposition}
\newtheorem{proposition}{Proposition}
\newtheorem{remark}[theorem]{Bemerkung}
\newtheorem{remark}[theorem]{Remark}
\newtheorem{theorem}{Theorem}
\newtheorem{warning}[theorem]{Warning}
\newtheorem{warning}[theorem]{Warnung}
\newunicodechar{±}{\ensuremath{\pm}
\newunicodechar{×}{\ensuremath{\times}
\newunicodechar{÷}{\ensuremath{\div}
\newunicodechar{ħ}{\ensuremath{\hbar}
\newunicodechar{Α}{\ensuremath{A}
\newunicodechar{Β}{\ensuremath{B}
\newunicodechar{Γ}{\ensuremath{\Gamma}
\newunicodechar{Δ}{\ensuremath{\Delta}
\newunicodechar{Ε}{\ensuremath{E}
\newunicodechar{Ζ}{\ensuremath{Z}
\newunicodechar{Η}{\ensuremath{H}
\newunicodechar{Θ}{\ensuremath{\Theta}
\newunicodechar{Ι}{\ensuremath{I}
\newunicodechar{Κ}{\ensuremath{K}
\newunicodechar{Λ}{\ensuremath{\Lambda}
\newunicodechar{Μ}{\ensuremath{M}
\newunicodechar{Ν}{\ensuremath{N}
\newunicodechar{Ξ}{\ensuremath{\Xi}
\newunicodechar{Ο}{\ensuremath{O}
\newunicodechar{Π}{\ensuremath{\Pi}
\newunicodechar{Ρ}{\ensuremath{P}
\newunicodechar{Σ}{\ensuremath{\Sigma}
\newunicodechar{Τ}{\ensuremath{T}
\newunicodechar{Υ}{\ensuremath{\Upsilon}
\newunicodechar{Φ}{\ensuremath{\Phi}
\newunicodechar{Χ}{\ensuremath{X}
\newunicodechar{Ψ}{\ensuremath{\Psi}
\newunicodechar{Ω}{\ensuremath{\Omega}
\newunicodechar{α}{\ensuremath{\alpha}
\newunicodechar{β}{\ensuremath{\beta}
\newunicodechar{γ}{\ensuremath{\gamma}
\newunicodechar{δ}{\ensuremath{\delta}
\newunicodechar{ε}{\ensuremath{\varepsilon}
\newunicodechar{ζ}{\ensuremath{\zeta}
\newunicodechar{η}{\ensuremath{\eta}
\newunicodechar{θ}{\ensuremath{\theta}
\newunicodechar{ι}{\ensuremath{\iota}
\newunicodechar{κ}{\ensuremath{\kappa}
\newunicodechar{λ}{\ensuremath{\lambda}
\newunicodechar{μ}{\ensuremath{\mu}
\newunicodechar{ν}{\ensuremath{\nu}
\newunicodechar{ξ}{\ensuremath{\xi}
\newunicodechar{ο}{\ensuremath{o}
\newunicodechar{π}{\ensuremath{\pi}
\newunicodechar{ρ}{\ensuremath{\rho}
\newunicodechar{σ}{\ensuremath{\sigma}
\newunicodechar{τ}{\ensuremath{\tau}
\newunicodechar{υ}{\ensuremath{\upsilon}
\newunicodechar{φ}{\ensuremath{\phi}
\newunicodechar{φ}{\ensuremath{\varphi}
\newunicodechar{χ}{\ensuremath{\chi}
\newunicodechar{ψ}{\ensuremath{\psi}
\newunicodechar{ω}{\ensuremath{\omega}
\newunicodechar{←}{\ensuremath{\leftarrow}
\newunicodechar{→}{\ensuremath{\rightarrow}
\newunicodechar{↔}{\ensuremath{\leftrightarrow}
\newunicodechar{⇐}{\ensuremath{\Leftarrow}
\newunicodechar{⇒}{\ensuremath{\Rightarrow}
\newunicodechar{⇔}{\ensuremath{\Leftrightarrow}
\newunicodechar{∂}{\ensuremath{\partial}
\newunicodechar{∅}{\ensuremath{\emptyset}
\newunicodechar{∇}{\ensuremath{\nabla}
\newunicodechar{∈}{\ensuremath{\in}
\newunicodechar{∉}{\ensuremath{\notin}
\newunicodechar{∏}{\ensuremath{\prod}
\newunicodechar{∑}{\ensuremath{\sum}
\newunicodechar{√}{\ensuremath{\sqrt}
\newunicodechar{∝}{\ensuremath{\propto}
\newunicodechar{∞}{\ensuremath{\infty}
\newunicodechar{∩}{\ensuremath{\cap}
\newunicodechar{∪}{\ensuremath{\cup}
\newunicodechar{∫}{\ensuremath{\int}
\newunicodechar{≈}{\ensuremath{\approx}
\newunicodechar{≠}{\ensuremath{\neq}
\newunicodechar{≤}{\ensuremath{\leq}
\newunicodechar{≥}{\ensuremath{\geq}
\newunicodechar{★}{\ensuremath{\star}
\newunicodechar{✓}{\checkmark}
\pgfplotsset{compat=1.17}
\pgfplotsset{compat=1.18}
\renewcommand{\cftchapfont}{\large\bfseries\color{blue}
\renewcommand{\cftchappagefont}{\large\bfseries\color{blue}
\renewcommand{\cftsecfont}{\bfseries}
\renewcommand{\cftsecfont}{\color{blue}
\renewcommand{\cftsecfont}{\large\bfseries\color{blue}
\renewcommand{\cftsecpagefont}{\bfseries}
\renewcommand{\cftsecpagefont}{\color{blue}
\renewcommand{\cftsecpagefont}{\large\bfseries\color{blue}
\renewcommand{\cftsubsecfont}{\color{blue!80!black}
\renewcommand{\cftsubsecfont}{\color{blue}
\renewcommand{\cftsubsecpagefont}{\color{blue!80!black}
\renewcommand{\cftsubsecpagefont}{\color{blue}
\renewcommand{\cftsubsubsecfont}{\color{blue!60!black}
\renewcommand{\cftsubsubsecfont}{\color{blue}
\renewcommand{\cftsubsubsecpagefont}{\color{blue!60!black}
\renewcommand{\cftsubsubsecpagefont}{\color{blue}
\renewcommand{\cfttoctitlefont}{\huge\bfseries\color{blue}
\renewcommand{\cfttoctitlefont}{\huge\bfseries}
\renewcommand{\familydefault}{\sfdefault}
\renewcommand{\footrulewidth}{0.4pt}
\renewcommand{\headrulewidth}{0.4pt}
\sisetup{locale = DE, group-separator = {.}
\sisetup{locale = DE}
\usetikzlibrary{arrows.meta,positioning,shapes.geometric}
\usetikzlibrary{decorations.pathmorphing, patterns, shapes.arrows}
\usetikzlibrary{intersections}
\usetikzlibrary{positioning, arrows.meta}
\usetikzlibrary{positioning, arrows}
\usetikzlibrary{positioning, shapes.geometric, arrows.meta}
\usetikzlibrary{positioning,shapes,arrows}

% Common settings
\setlength{\headheight}{15pt}
\pgfplotsset{compat=1.18}
\usetikzlibrary{positioning,shapes,arrows,arrows.meta}

% Hyperref setup
\hypersetup{
    colorlinks=true,
    linkcolor=blue,
    citecolor=blue,
    urlcolor=blue
}


\title{RSAtest De}
\author{Johann Pascher}
\date{\today}

\begin{document}

\maketitle
\tableofcontents

\begin{abstract}
		Diese Arbeit dokumentiert empirische Ergebnisse aus systematischen Tests verschiedener Faktorisierungsalgorithmen. 37 Testfälle wurden mit Trial Division, Fermats Methode, Pollard Rho, Pollard $p-1$ und dem T0-Framework durchgeführt. Das primäre Ziel ist die Demonstration, dass deterministische Periodenfindung machbar ist. Alle Ergebnisse basieren auf direkten Messungen ohne theoretische Bewertungen oder Vergleiche.
	\end{abstract}
	
	\tableofcontents
	\newpage
	
	# Methodik
	
	## Getestete Algorithmen
	
	Die folgenden Faktorisierungsalgorithmen wurden implementiert und getestet:
	
	
		- \textbf{Trial Division}: Systematische Divisionsversuche bis $\sqrt{n}$
		- \textbf{Fermats Methode}: Suche nach Darstellung als Differenz von Quadraten
		- \textbf{Pollard Rho}: Probabilistische Periodenfindung in pseudozufälligen Sequenzen
		- \textbf{Pollard $p-1$}: Methode für Zahlen mit glatten Faktoren
		- \textbf{T0-Framework}: Deterministische Periodenfindung in modularer Exponentiation (klassisch Shor-inspiriert)
	
	
	## Testkonfiguration
	
	\begin{table}[H]
		\centering
		\caption{Experimentelle Parameter}
		\begin{tabular}{ll}
			\toprule
			\textbf{Parameter} & \textbf{Wert} \\
			\midrule
			Anzahl Testfälle & 37 \\
			Timeout pro Test & 2,0 Sekunden \\
			Zahlenbereich & 15 bis 16777213 \\
			Bitgröße & 4 bis 24 Bits \\
			Hardware & Standard Desktop-CPU \\
			Wiederholungen & 1 pro Kombination \\
			\bottomrule
		\end{tabular}
		\label{tab:test_config}
	\end{table}
	
	## Metriken
	
	Für jeden Test wurden folgende Werte aufgezeichnet:
	
		- \textbf{Erfolg/Misserfolg}: Binäres Ergebnis
		- \textbf{Ausführungszeit}: Millisekundengenauigkeit
		- \textbf{Gefundene Faktoren}: Für erfolgreiche Tests
		- \textbf{Algorithmusspezifische Parameter}: Je nach Methode
	
	
	# T0-Framework Machbarkeitsdemonstation
	
	## Zweck der Implementierung
	
	Die T0-Framework-Implementierung dient als Machbarkeitsnachweis, um zu demonstrieren, dass deterministische Periodenfindung technisch auf klassischer Hardware möglich ist.
	
	## Implementierungskomponenten
	
	Das T0-Framework implementiert folgende Komponenten zur Demonstration deterministischer Periodenfindung:
	
	\begin{verbatim}
		class UniversalT0Algorithm:
		def __init__(self):
		self.xi_profiles = {
			'universal': Fraction(1, 100),
			'twin_prime_optimized': Fraction(1, 50),
			'medium_size': Fraction(1, 1000),
			'special_cases': Fraction(1, 42)
		}
		self.pi_fraction = Fraction(355, 113)
		self.threshold = Fraction(1, 1000)
	\end{verbatim}
	
	## Adaptive $\xi$-Strategien
	
	Das System verwendet verschiedene $\xi$-Parameter basierend auf Zahleneigenschaften:
	
	\begin{table}[H]
		\centering
		\caption{$\xi$-Strategien im T0-Framework}
		\begin{tabular}{lll}
			\toprule
			\textbf{Strategie} & \textbf{$\xi$-Wert} & \textbf{Anwendung} \\
			\midrule
			twin\_prime\_optimized & $1/50$ & Zwillingsprim-Semiprims \\
			universal & $1/100$ & Allgemeine Semiprims \\
			medium\_size & $1/1000$ & Mittelgroße Zahlen \\
			special\_cases & $1/42$ & Mathematische Konstanten \\
			\bottomrule
		\end{tabular}
		\label{tab:xi_strategies}
	\end{table}
	
	## Resonanzberechnung
	
	Die Resonanzbewertung wird mit exakter rationaler Arithmetik durchgeführt:
	
	
```math-equation

		\omega = \frac{2 \cdot \pi_{\text{ratio}}}{r}
	
```

	
	
```math-equation

		R(r) = \frac{1}{1 + \left|\frac{-(\omega-\pi)^2}{4\xi}\right|}
	
```

	
	# Experimentelle Ergebnisse: Machbarkeitsnachweis
	
	Die experimentellen Ergebnisse dienen der Demonstration der Machbarkeit deterministischer Periodenfindung anstatt dem Vergleich algorithmischer Leistung.
	
	## Erfolgsraten nach Algorithmus
	
	\begin{table}[H]
		\centering
		\caption{Gesamte Erfolgsraten aller Algorithmen}
		\begin{tabular}{lrr}
			\toprule
			\textbf{Algorithmus} & \textbf{Erfolgreiche Tests} & \textbf{Erfolgsrate (\%)} \\
			\midrule
			Trial Division & 37/37 & 100,0 \\
			Fermat & 37/37 & 100,0 \\
			Pollard Rho & 36/37 & 97,3 \\
			Pollard $p-1$ & 12/37 & 32,4 \\
			T0-Adaptive & 31/37 & 83,8 \\
			\bottomrule
		\end{tabular}
		\label{tab:success_rates}
	\end{table}
	
	# Periodenbasierte Faktorisierung: T0, Pollard Rho und Shors Algorithmus
	
	## Vergleich der Periodenfindungsansätze
	
	T0-Framework, Pollard Rho und Shors Quantenalgorithmus sind alle periodenfindende Algorithmen mit verschiedenen Rechenbarkeitssystemen:
	
	\begin{table}[H]
		\centering
		\caption{Vergleich periodenfindender Algorithmen}
		\begin{tabular}{llll}
			\toprule
			\textbf{Aspekt} & \textbf{Pollard Rho} & \textbf{T0-Framework} & \textbf{Shors Algorithmus} \\
			\midrule
			Berechnung & Klassisch prob. & Klassisch det. & Quanten \\
			Periodenerkennung & Floyd-Zyklus & Resonanzanalyse & Quanten-FT \\
			Arithmetik & Modular & Exakt rational & Quantensuperpos. \\
			Reproduzierbarkeit & Variabel & 100\% reprod. & Prob. Messung \\
			Sequenzerzeugung & $f(x) = x^2 + c \bmod n$ & $a^r \equiv 1 \pmod{n}$ & $a^x \bmod n$ \\
			Erfolgskriterium & $\gcd(|x_i - x_j|, n) > 1$ & Resonanzschwelle & Periode aus QFT \\
			Komplexität & $O(n^{1/4})$ erwartet & $O((\log n)^3)$ theor. & $O((\log n)^3)$ theor. \\
			Hardware & Klassischer Rechner & Klassischer Rechner & Quantenrechner \\
			Praktisches Limit & Geburtstags-Paradoxon & Resonanztuning & Quantendekohärenz \\
			\bottomrule
		\end{tabular}
		\label{tab:period_comparison}
	\end{table}
	
	## Gemeinsames Periodenfindungsprinzip
	
	Alle drei Algorithmen nutzen dieselbe mathematische Grundlage:
	
	
		- \textbf{Kernidee}: Finde Periode $r$ wobei $a^r \equiv 1 \pmod{n}$
		- \textbf{Faktorextraktion}: Nutze Periode um $\gcd(a^{r/2} \pm 1, n)$ zu berechnen
		- \textbf{Mathematische Basis}: Eulers Theorem und Ordnung von Elementen in $\mathbb{Z}_n^*$
	
	
	## Theoretische Komplexitätsanalyse
	
	Sowohl T0-Framework als auch Shors Algorithmus teilen denselben theoretischen Komplexitätsvorteil:
	
	
		- \textbf{Periodensuchraum}: Beide suchen nach Perioden $r$ wobei $a^r \equiv 1 \pmod{n}$
		- \textbf{Maximale Periode}: Die Ordnung jedes Elements ist höchstens $n-1$, aber typischerweise viel kleiner
		- \textbf{Erwartete Periodenlänge}: $O(\log n)$ für die meisten Elemente aufgrund Eulers Theorem
		- \textbf{Periodentest}: Jeder Periodentest benötigt $O((\log n)^2)$ Operationen für modulare Exponentiation
		- \textbf{Gesamtkomplexität}: $O(\log n) \times O((\log n)^2) = O((\log n)^3)$
	
	
	## Der gemeinsame polynomiale Vorteil
	
	Sowohl T0 als auch Shors Algorithmus erreichen denselben theoretischen Durchbruch:
	
	
```math-equation

		\text{Klassisch exponentiell: } O(2^{\sqrt{\log n \log \log n}}) \rightarrow \text{Polynomial: } O((\log n)^3)
	
```

	
	Die Schlüsselerkenntnis ist, dass \textbf{beide Algorithmen dieselbe mathematische Struktur ausnutzen}:
	
		- Periodenfindung in der Gruppe $\mathbb{Z}_n^*$
		- Erwartete Periodenlänge $O(\log n)$ aufgrund glatter Zahlen
		- Polynomialzeit-Periodenverifikation
		- Identische Faktorextraktionsmethode
	
	
	\textbf{Der einzige Unterschied}: Shor nutzt Quantensuperposition um Perioden parallel zu suchen, während T0 sie deterministisch sequenziell sucht - aber beide haben dieselbe $O((\log n)^3)$ Komplexitätsgrenze.
	
	## Das Implementierungsparadoxon
	
	Sowohl T0 als auch Shors Algorithmus demonstrieren ein fundamentales Paradoxon in fortgeschrittener Algorithmusentwicklung:
	
	\begin{tcolorbox}[colback=yellow!10,colframe=orange!50,title=Kernproblem]
		\textbf{Perfekte Theorie, unvollkommene Implementierung:} \\
		Beide Algorithmen erreichen denselben theoretischen Durchbruch von exponentieller zu polynomialer Komplexität, aber praktischer Implementierungsaufwand negiert diese theoretischen Vorteile vollständig.
	\end{tcolorbox}
	
	### Gemeinsame Implementierungsmängel
	
		- \textbf{Shors Quantenaufwand}: 
		
			- Quantenfehlerkorrektur benötigt $\sim 10^6$ physische Qubits pro logischem Qubit
			- Dekohärenzzeiten begrenzen Algorithmusausführung
			- Aktuelle Systeme: 1000 Qubits $\rightarrow$ Benötigt: $10^9$ Qubits für RSA-2048
		
		
		- \textbf{T0s klassischer Aufwand}:
		
			- Exakte rationale Arithmetik: Bruchobjekte wachsen exponentiell in der Größe
			- Resonanzbewertung: Komplexe mathematische Operationen pro Periode
			- Adaptive Parameteranpassung: Multiple $\xi$-Strategien erhöhen Berechnungskosten
		
	
	
	# Philosophische Implikationen: Information und Determinismus
	
	## Intrinsische mathematische Information
	
	Eine entscheidende Erkenntnis ergibt sich aus dieser Analyse, die über Berechnungskomplexität hinausgeht:
	
	\begin{tcolorbox}[colback=blue!10,colframe=blue!50,title=Fundamentales Prinzip]
		\textbf{Kein Superdeterminismus erforderlich:} \\
		Alle Information, die aus einer Zahl durch Faktorisierungsalgorithmen extrahiert werden kann, ist intrinsisch in der Zahl selbst enthalten. Die Algorithmen enthüllen lediglich bereits existierende mathematische Beziehungen - sie erzeugen keine Information.
	\end{tcolorbox}
	
	## Vibrationsmodi und prädiktive Muster
	
	Eine tiefere Analyse zeigt, dass die Zahlengröße die möglichen „Vibrationsmodi" in der Faktorisierung beschränkt:
	
	\begin{tcolorbox}[colback=purple!10,colframe=purple!50,title=Vibrationseinschränkungsprinzip]
		\textbf{Größenbestimmter Modusraum:} \\
		Die Größe einer Zahl $n$ bestimmt vorab die Grenzen möglicher Schwingungsmodi. Innerhalb dieser Grenzen sind nur spezifische Resonanzmuster mathematisch möglich, und diese folgen vorhersagbaren Mustern, die es ermöglichen, in die Zukunft des Faktorisierungsprozesses zu blicken.
	\end{tcolorbox}
	
	### Eingeschränkter Schwingungsraum
	
	Für eine Zahl $n$ mit $k = \log_2(n)$ Bits:
	
	
		- \textbf{Maximale Periode}: $r_{\max} = \lambda(n) \leq n-1$ (Carmichael-Funktion)
		- \textbf{Typischer Periodenbereich}: $r_{typical} \in [1, O(\sqrt{n})]$ für die meisten Basen
		- \textbf{Resonanzfrequenzen}: $\omega = 2\pi/r$ beschränkt auf diskrete Werte
		- \textbf{Vibrationsmodi}: Nur $O(\sqrt{n})$ unterschiedliche Schwingungsmuster möglich
	
	
	## Das begrenzte Universum der Schwingungen
	
	
```math-equation

		\Omega_n = \left\{\omega_r = \frac{2\pi}{r} : r \in \mathbb{Z}, 2 \leq r \leq \lambda(n)\right\}
	
```

	
	Dieser Frequenzraum $\Omega_n$ ist:
	
		- \textbf{Endlich}: Durch Zahlengröße beschränkt
		- \textbf{Diskret}: Nur ganzzahlige Perioden erlaubt
		- \textbf{Strukturiert}: Folgt mathematischen Mustern basierend auf $n$s Primstruktur
		- \textbf{Vorhersagbar}: Resonanzspitzen clustern in mathematisch bestimmten Bereichen
	
	
	\begin{tcolorbox}[colback=cyan!10,colframe=cyan!50,title=Vorhersageprinzip]
		\textbf{Mathematische Voraussicht:} \\
		Durch Analyse des eingeschränkten Schwingungsraums und Erkennung struktureller Muster wird es möglich vorherzusagen, welche Perioden starke Resonanzen erzeugen werden, ohne alle Möglichkeiten erschöpfend zu testen. Dies stellt eine Form mathematischer „Zukunftssicht" dar - nicht mystisch, sondern basierend auf tiefer Mustererkennung in zahlentheoretischen Strukturen.
	\end{tcolorbox}
	
	# Neuronale Netzwerk-Implikationen: Lernen mathematischer Muster
	
	## Maschinelles Lernpotenzial
	
	Wenn mathematische Muster in Schwingungsmodi durch Mustererkennung vorhersagbar sind, dann sollten neuronale Netzwerke inhärent fähig sein, diese Muster zu lernen:
	
	\begin{tcolorbox}[colback=green!10,colframe=green!50,title=Neuronales Netzwerk-Hypothese]
		\textbf{Lernbare mathematische Muster:} \\
		Da die Vibrationsmodi und Resonanzmuster mathematisch deterministischen Regeln innerhalb eingeschränkter Räume folgen, sollten neuronale Netzwerke imstande sein zu lernen, optimale Faktorisierungsstrategien ohne erschöpfende Suche vorherzusagen.
	\end{tcolorbox}
	
	## Trainingsdatenstruktur
	
	Die experimentellen Daten liefern perfektes Trainingsmaterial:
	
	
		- \textbf{Eingabemerkmale}: Zahlengröße, Bitlänge, mathematischer Typ (Zwillingsprim, glatt, etc.)
		- \textbf{Zielvorhersagen}: Optimale $\xi$-Strategie, erwartete Resonanzperioden, Erfolgswahrscheinlichkeit
		- \textbf{Musterbeispiele}: 37 Testfälle mit dokumentierten Erfolgs-/Misserfolgsmuster
		- \textbf{Merkmalstechnik}: Extraktion mathematischer Invarianten (Primlücken, Glätte, etc.)
	
	
	## Lernen mathematischer Invarianten
	
	Neuronale Netzwerke könnten lernen zu erkennen:
	
	\begin{table}[H]
		\centering
		\caption{Lernbare mathematische Muster}
		\begin{tabular}{ll}
			\toprule
			\textbf{Math. Muster} & \textbf{NN-Lernziel} \\
			\midrule
			Zwillingsprimstruktur & Vorhersage $\xi = 1/50$ Strategie \\
			Primlückenverteilung & Schätzung Resonanzclustering \\
			Glätteindikatoren & Vorhersage Periodenverteilung \\
			Math. Konstanten & ID Multi-Resonanzmuster \\
			Carmichael-Muster & Schätzung max. Periodengrenzen \\
			Faktorgrößenverhältnisse & Vorhersage opt. Basisauswahl \\
			\bottomrule
		\end{tabular}
		\label{tab:learnable_patterns}
	\end{table}
	
	# Kernimplementierung: factorization\_benchmark\_library.py
	
	\textbf{Quelle}: \url{https://github.com/jpascher/T0-Time-Mass-Duality/blob/main/rsa/factorization_benchmark_library.py}
	
	## Bibliotheksarchitektur
	
	Die Hauptbibliothek (50KB) implementiert das vollständige Universal T0-Framework mit folgenden Kernkomponenten:
	
	
		- \textbf{UniversalT0Algorithm}: Kernimplementierung mit optimierten $\xi$-Profilen
		- \textbf{FactorizationLibrary}: Zentrale API für alle Algorithmen
		- \textbf{FactorizationResult}: Erweiterte Datenstruktur mit T0-Metriken
		- \textbf{TestCase}: Strukturierte Testfalldefinition
	
	
	## Verwendungsbeispiele
	
	\begin{verbatim}
		from factorization_benchmark_library import create_factorization_library
		
		# Grundverwendung
		lib = create_factorization_library()
		result = lib.factorize(143, "t0_adaptive")
		
		# Benchmark mehrerer Methoden
		test_cases = [TestCase(143, [11, 13], "Zwillingsprim", "twin_prime", "easy")]
		results = lib.benchmark(test_cases)
		
		# Schnelle Einzelfaktorisierung
		from factorization_benchmark_library import quick_factorize
		result = quick_factorize(1643, "t0_universal")
	\end{verbatim}
	
	## Verfügbare Methoden
	
	\begin{table}[H]
		\centering
		\caption{Verfügbare Faktorisierungsmethoden}
		\begin{tabular}{ll}
			\toprule
			\textbf{Methode} & \textbf{Beschreibung} \\
			\midrule
			trial\_division & Klassische systematische Division \\
			fermat & Differenz-der-Quadrate-Methode \\
			pollard\_rho & Probabilistische Zykluserkennung \\
			pollard\_p\_minus\_1 & Glatte-Faktoren-Methode \\
			t0\_classic & Original T0 ($\xi = 1/100000$) \\
			t0\_universal & Revolutionäres universelles T0 ($\xi = 1/100$) \\
			t0\_adaptive & Intelligente $\xi$-Strategieauswahl \\
			t0\_medium\_size & Optimiert für N > 1000 ($\xi = 1/1000$) \\
			t0\_special\_cases & Für spezielle Zahlen ($\xi = 1/42$) \\
			\bottomrule
		\end{tabular}
	\end{table}
	
	# Testprogramm-Suite
	
	## easy\_test\_cases.py
	\textbf{Quelle}: \url{https://github.com/jpascher/T0-Time-Mass-Duality/blob/main/rsa/easy_test_cases.py}\\
	\textbf{Zweck}: Demonstration von T0s Überlegenheit bei einfachen Fällen
	
		- Testet 20 einfache Semiprims über verschiedene Kategorien
		- Vergleicht klassische Methoden vs. T0-Framework-Varianten
		- Validiert $\xi$-Revolution bei Zwillingsprims, Cousin-Prims und entfernten Prims
		- Erwartetes Ergebnis: T0-universal erreicht 100\% Erfolgsrate
	
	
	## borderline\_test\_cases.py
	\textbf{Quelle}: \url{https://github.com/jpascher/T0-Time-Mass-Duality/blob/main/rsa/borderline_test_cases.py}\\
	\textbf{Zweck}: Systematische Erforschung algorithmischer Grenzen
	
		- 16-24 Bit Semiprims in der kritischen Übergangszone
		- Fermat-freundliche Fälle mit nahen Faktoren
		- Pollard Rho Grenzfälle mit mittelgroßen Prims
		- Trial Division Grenzen bis $\sqrt{N} \approx 31617$
		- Erwartetes Ergebnis: T0 erweitert Erfolg über klassische Grenzen hinaus
	
	
	## impossible\_test\_cases.py
	\textbf{Quelle}: \url{https://github.com/jpascher/T0-Time-Mass-Duality/blob/main/rsa/impossible_test_cases.py}\\
	\textbf{Zweck}: Bestätigung fundamentaler Faktorisierungsgrenzen
	
		- 60-Bit Zwillingsprims jenseits aller algorithmischen Fähigkeiten
		- RSA-100 (330-Bit) demonstriert kryptographische Sicherheit
		- Carmichael-Zahlen fordern probabilistische Methoden heraus
		- Hardware-Grenzen-Tests (>30-Bit Bereich)
		- Erwartetes Ergebnis: 100\% Versagen über alle Methoden einschließlich T0
	
	
	## automatic\_xi\_optimizer.py
	\textbf{Quelle}: \url{https://github.com/jpascher/T0-Time-Mass-Duality/blob/main/rsa/automatic_xi_optimizer.py}\\
	\textbf{Zweck}: Maschineller Lernansatz zur $\xi$-Parameteroptimierung
	
		- Systematisches Testen von $\xi$-Kandidaten über Zahlenkategorien
		- Mustererkennung für optimale $\xi$-Strategieauswahl
		- Fibonacci-, Prim- und mathematische konstantenbasierte $\xi$-Werte
		- Leistungsanalyse und Empfehlungserzeugung
		- Erwartetes Ergebnis: Validierung von $\xi = 1/100$ als universelles Optimum
	
	
	## focused\_xi\_tester.py
	\textbf{Quelle}: \url{https://github.com/jpascher/T0-Time-Mass-Duality/blob/main/rsa/focused_xi_tester.py}\\
	\textbf{Zweck}: Gezielte Tests problematischer Zahlenkategorien
	
		- Cousin-Prims, Nahe-Zwillinge und entfernte Prims Analyse
		- Kategoriespezifische $\xi$-Kandidatenerzeugung
		- Verbesserungsquantifizierung über Standard $\xi = 1/100000$
		- Erwartetes Ergebnis: Entdeckung kategorieoptimierter $\xi$-Strategien
	
	
	## t0\_uniqueness\_test.py
	\textbf{Quelle}: \url{https://github.com/jpascher/T0-Time-Mass-Duality/blob/main/rsa/t0_uniqueness_test.py}\\
	\textbf{Zweck}: Identifikation von T0s exklusiven Fähigkeiten
	
		- Systematische Suche nach Fällen wo nur T0 erfolgreich ist
		- Geschwindigkeitsvergleichsanalyse zwischen T0 und klassischen Methoden
		- Dokumentation von T0s mathematischer Nische
		- Erwartetes Ergebnis: Beweis von T0s einzigartigen algorithmischen Vorteilen
	
	
	## xi\_strategy\_debug.py
	\textbf{Quelle}: \url{https://github.com/jpascher/T0-Time-Mass-Duality/blob/main/rsa/xi_strategy_debug.py}\\
	\textbf{Zweck}: Debugging der $\xi$-Strategieauswahllogik
	
		- Analyse des Kategorisierungsalgorithmusverhaltens
		- Manuelle $\xi$-Strategieerzwingung für Problemfälle
		- Optimale $\xi$-Wertsuche für spezifische Zahlen
		- Strategieauswahllogikverifikation und -korrektur
	
	
	## updated\_impossible\_tests.py
	\textbf{Quelle}: \url{https://github.com/jpascher/T0-Time-Mass-Duality/blob/main/rsa/updated_impossible_tests.py}\\
	\textbf{Zweck}: Aktualisierte Version unmöglicher Testfälle mit verbesserter T0-Analyse
	
		- Erweiterte 60-Bit Zwillingsprims jenseits aller Fähigkeiten
		- Verbesserte theoretische Grenzdokumentation
		- T0-spezifische Grenzentests für progressive Bitgrößen
		- Umfassende Versagensanalyse über alle Methodenkategorien
		- Erwartetes Ergebnis: Bestätigung dass sogar revolutionäres T0 harte Skalierungsgrenzen hat
	
	
	# Interaktive Werkzeuge
	
	## xi\_explorer\_tool.html
	\textbf{Quelle}: \url{https://github.com/jpascher/T0-Time-Mass-Duality/blob/main/rsa/xi_explorer_tool.html}\\
	Interaktives webbasiertes Werkzeug für Echtzeit-$\xi$-Parametererforschung:
	
		- Visuelle Resonanzmusteranalyse
		- Dynamische $\xi$-Parameteranpassungsschnittstelle
		- Algorithmusleistungsvergleichsdashboard
		- Echtzeit-Faktorisierungstestfähigkeit
	
	
	# Experimentelles Protokoll
	
	## Standard-Testkonfiguration
	
	Alle Tests folgen standardisierten Parametern:
	\begin{table}[H]
		\centering
		\caption{Standardisierte Testparameter}
		\begin{tabular}{ll}
			\toprule
			\textbf{Parameter} & \textbf{Wert} \\
			\midrule
			Timeout pro Algorithmus & 2,0-10,0 Sekunden (methodenabhängig) \\
			T0-Timeout-Erweiterung & 15,0 Sekunden (Komplexitätsbetrachtung) \\
			Messgenauigkeit & Millisekundenzeitnahme \\
			Erfolgsverifikation & Faktorproduktvalidierung \\
			Resonanzschwelle & $\xi$-abhängig (typisch $1/1000$) \\
			Maximal getestete Perioden & 500-2000 (größenabhängig) \\
			\bottomrule
		\end{tabular}
	\end{table}
	
	## Leistungsmetriken
	
	Jeder Test zeichnet umfassende Metriken auf:
	
		- \textbf{Erfolg/Misserfolg}: Binäres algorithmisches Ergebnis
		- \textbf{Ausführungszeit}: Hochpräzise Zeitmessungen
		- \textbf{Faktorkorrektheit}: Produktverifikation gegen Eingabe
		- \textbf{T0-spezifische Daten}: $\xi$-Strategie, Resonanzbewertung, getestete Perioden
		- \textbf{Speichernutzung}: Ressourcenverbrauchsüberwachung
		- \textbf{Methodenspezifische Parameter}: Algorithmusabhängige Metadaten
	
	
	# Kernforschungsergebnisse
	
	## Revolutionäre $\xi$-Optimierungsergebnisse
	
	Experimentelle Validierung der $\xi$-Revolutionshypothese:
	
	\begin{table}[H]
		\centering
		\caption{$\xi$-Strategieeffektivität}
		\begin{tabular}{lll}
			\toprule
			\textbf{Zahlenkategorie} & \textbf{Optimales $\xi$} & \textbf{Erfolgsrate} \\
			\midrule
			Zwillingsprims & $1/50$ & 95\% \\
			Universal (Alle Typen) & $1/100$ & 83,8\% \\
			Mittelgroß ($N > 1000$) & $1/1000$ & 78\% \\
			Spezialfälle & $1/42$ & 67\% \\
			Klassisch nur Zwillinge & $1/100000$ & 45\% \\
			\bottomrule
		\end{tabular}
	\end{table}
	
	## Algorithmische Grenzen
	
	Klare Identifikation fundamentaler Limits:
	
		- \textbf{Klassische Methoden}: Versagen jenseits 20-25 Bits
		- \textbf{T0-Framework}: Erweitert Erfolg auf 25-30 Bits
		- \textbf{Hardware-Grenzen}: Betreffen alle Methoden jenseits 30 Bits
		- \textbf{RSA-Sicherheit}: Beruht auf diesen mathematischen Grenzen
	
	
	# Praktische Anwendungen
	
	## Akademische Forschung
	
		- Periodenfindungsalgorithmusentwicklung
		- Resonanzbasierte mathematische Analyse
		- Quantenalgorithmus-klassische Simulation
		- Zahlentheorie-Mustererkennung
	
	
	## Kryptographische Analyse
	
		- Semiprim-Sicherheitsbewertung
		- RSA-Schlüsselstärkebewertung
		- Post-Quanten-Kryptographievorbereitung
		- Faktorisierungsresistenzmessung
	
	
	## Bildungsdemonstration
	
		- Algorithmuskomplexitätsvisualisierung
		- Klassisch vs. Quanten-Methodenvergleich
		- Mathematische Optimierungsprinzipien
		- Berechnungsgrenzenerforschung
	
	
	# Zukünftige Arbeit
	
	## Neuronale Netzwerkintegration
	Basierend auf demonstrierten Mustererkennungsfähigkeiten:
	
		- Training auf $\xi$-Optimierungsergebnissen
		- Automatisches Strategieauswahllernen
		- Resonanzmustervorhersage
		- Skalierbarkeitsgrenzenerweiterung
	
	
	## Quantenalgorithmussimulation
	T0s polynomiale Komplexität ermöglicht:
	
		- Shors Algorithmus klassische Approximation
		- Quanten-Fourier-Transformationssimulation
		- Quantenperiodenfindungsmodellierung
		- Quantenvorteilsquantifizierung

\end{document}
